\chapter{Conclusiones y Recomendaciones}

Habiendo sido desplegado el sistema en un servidor web provisto por el centro
MEMI, como un medio para evaluación de la estrategia planteada se han hecho uso
de la funcionalidad construida para extraer varios indicadores de medición;
estos habiendo arrojando resultados para el control de los factores críticos son
analizados en este capitulo.

Este capitulo compendia el tiempo que se ha establecido para evaluación, además
de mostrar el conjunto de recomendaciones finales que se han planteado, a partir
de los resultados obtenidos.

\section{Resultados}
Como se menciono en la parte final del capitulo anterior, se ha desarrollado una
funcionalidad capaz de generar graficos estadisticos, de forma que puedan
apreciarse diversos indicadores sobre variables del sistema.

Una vez concluido el sistema, este fue implantado en un servidor del Centro
MEMI, donde con la colaboración de algunos docentes pudo ser puesta en practica
en algunos grupos del departamento de Informatica y Sistemas, los resultados
aqui presentados, se extrayeron de ese tiempo de evaluación.

\subsection{Contexto}
Para ser mas precisos respecto a ese tiempo de evaluación, puede apreciarse en
el cuadro \ref{contexto} algunos datos de contexto referentes al contexto de
aquel periodo.

\begin{table}
\centering
\begin{tabular}{l|l}
Sitio web                           & http://yachay.memi.umss.edu.bo\\
Periodo académico                   & I/2011\\
Tiempo de evaluación:               & 325 días.\\
Fecha de inicio:                    & 23 de Septiembre del 2010.\\
Fecha de fin:                       & 14 de Agosto del 2011.\\
Lugar de evaluación:                & Carrera de Informática y Sistemas.\\
Caídas del servidor:                & 4.\\
Tiempo del servidor fuera de linea: & 2 semanas acumuladas.\\
Docentes participantes:             & 4.\\
Materias participantes:             & 4.\\
Grupos participantes:               & 8.\\
Usuarios participantes:             & 542 (estudiantes de primeros semestres).\\
Espacios virtuales creados:         & 33.\\
Recursos publicados:                & 68.\\
\end{tabular}
\caption{Variables de Contexto para el periodo de evaluación}
\label{contexto}
\end{table}

\subsection{Usuarios}
El primer objeto a analizar fue el grado de uso de los usuarios con respecto del
sistema, puede apreciarse en la cuadro \ref{usuarios_tabla_1} un conteo de
usuarios clasificados por rol, que hicieron cierto conjunto de actividades. Las
actividades en cuestion son:

\begin{itemize}
\item \emph{Registrados}, conteo del numero real de usuarios que poseen ese rol.
\item \emph{Logeados}, conteo del numero de usuarios que al menos uso su cuenta
una vez.
\item \emph{Correo}, conteo del numero de usuarios que establecio su correo
electronico en el sistema.
\item \emph{Usuario}, conteo del numero de usuarios que establecieron un nombre
de usuario propio.
\item \emph{Fotografia}, conteo del numero de personas que utilizan una imagen
de usuario propia.
\end{itemize}

\begin{table}
\centering
\begin{tabular}{l|c c c c c}
$Rol$ & $Registrados$ & $Logeados$ & $Correo$ & $Usuario$ & $Fotografia$ \\
\hline
$Invitado$      & $ 10$ & $  9$ & $10$ & $10$ & $ 1$ \\
$Estudiante$    & $517$ & $103$ & $47$ & $15$ & $11$ \\
$Auxiliar$      & $  7$ & $  6$ & $ 7$ & $ 4$ & $ 3$ \\
$Docente$       & $  3$ & $  3$ & $ 3$ & $ 3$ & $ 1$ \\
$Moderador$     & $  1$ & $  1$ & $ 1$ & $ 1$ & $ 0$ \\
$Desarrollador$ & $  3$ & $  3$ & $ 3$ & $ 3$ & $ 2$ \\
$Administrador$ & $  1$ & $  1$ & $ 1$ & $ 1$ & $ 1$ \\
\end{tabular}
\caption{Intención de los usuarios clasificados por rol}
\label{usuarios_tabla_1}
\end{table}

Destaca en este cuadro la disparidad entre el rol de estudiante y los demás
roles, como puede verse en la figura \ref{usuarios_bars_1}, del conjunto de
usuarios registrados, únicamente el 20\% ingreso alguna vez al sistema, cabe
decir además, que los usuarios fueron registrados automáticamente por sus
respectivos docentes.

\begin{figure}
\centering
%LaTeX with PSTricks extensions
%%Creator: inkscape 0.48.5
%%Please note this file requires PSTricks extensions
\psset{xunit=.5pt,yunit=.5pt,runit=.5pt}
\begin{pspicture}(960,480)
{
\newrgbcolor{curcolor}{0 0 0}
\pscustom[linestyle=none,fillstyle=solid,fillcolor=curcolor]
{
\newpath
\moveto(438.45475342,18.82530273)
\curveto(438.48474374,18.69530247)(438.44474378,18.59530257)(438.33475342,18.52530273)
\curveto(438.28474394,18.49530267)(438.21974401,18.47530269)(438.13975342,18.46530273)
\lineto(437.89975342,18.46530273)
\lineto(437.41975342,18.46530273)
\curveto(437.25974497,18.4653027)(437.14474508,18.50030267)(437.07475342,18.57030273)
\curveto(437.00474522,18.62030255)(436.96474526,18.69530247)(436.95475342,18.79530273)
\lineto(436.95475342,19.12530273)
\lineto(436.95475342,19.23030273)
\curveto(436.96474526,19.2703019)(436.97474525,19.30530186)(436.98475342,19.33530273)
\curveto(436.97474525,19.38530178)(436.97974525,19.43030174)(436.99975342,19.47030273)
\curveto(437.01974521,19.51030166)(437.0247452,19.55030162)(437.01475342,19.59030273)
\lineto(437.04475342,19.77030273)
\lineto(437.07475342,19.95030273)
\lineto(437.16475342,20.62530273)
\curveto(437.16474506,20.69530047)(437.16974506,20.7653004)(437.17975342,20.83530273)
\curveto(437.18974504,20.90530026)(437.19474503,20.98030019)(437.19475342,21.06030273)
\curveto(437.18474504,21.24029993)(437.18474504,21.42029975)(437.19475342,21.60030273)
\curveto(437.20474502,21.78029939)(437.18974504,21.95029922)(437.14975342,22.11030273)
\curveto(437.03974519,22.53029864)(436.77974545,22.81029836)(436.36975342,22.95030273)
\curveto(436.24974598,23.00029817)(436.10974612,23.02529814)(435.94975342,23.02530273)
\curveto(435.79974643,23.03529813)(435.63974659,23.04029813)(435.46975342,23.04030273)
\lineto(432.70975342,23.04030273)
\curveto(432.63974959,23.02029815)(432.57474965,23.00029817)(432.51475342,22.98030273)
\curveto(432.45474977,22.9702982)(432.39974983,22.94029823)(432.34975342,22.89030273)
\curveto(432.25974997,22.79029838)(432.19475003,22.62529854)(432.15475342,22.39530273)
\curveto(432.11475011,22.17529899)(432.07975015,21.98029919)(432.04975342,21.81030273)
\lineto(431.61475342,19.63530273)
\curveto(431.58475064,19.49530167)(431.55475067,19.32030185)(431.52475342,19.11030273)
\curveto(431.49475073,18.91030226)(431.44475078,18.76030241)(431.37475342,18.66030273)
\curveto(431.34475088,18.59030258)(431.29475093,18.54530262)(431.22475342,18.52530273)
\curveto(431.18475104,18.50530266)(431.14475108,18.49530267)(431.10475342,18.49530273)
\curveto(431.07475115,18.49530267)(431.0297512,18.48530268)(430.96975342,18.46530273)
\curveto(430.9297513,18.45530271)(430.88475134,18.45030272)(430.83475342,18.45030273)
\curveto(430.78475144,18.46030271)(430.73475149,18.4653027)(430.68475342,18.46530273)
\lineto(430.36975342,18.46530273)
\curveto(430.26975196,18.47530269)(430.18975204,18.50530266)(430.12975342,18.55530273)
\curveto(430.05975217,18.60530256)(430.03475219,18.69530247)(430.05475342,18.82530273)
\curveto(430.08475214,18.9653022)(430.11475211,19.10030207)(430.14475342,19.23030273)
\lineto(431.95975342,28.35030273)
\curveto(431.97975025,28.46029271)(431.99975023,28.57529259)(432.01975342,28.69530273)
\curveto(432.03975019,28.81529235)(432.08475014,28.91029226)(432.15475342,28.98030273)
\curveto(432.20475002,29.04029213)(432.28974994,29.09029208)(432.40975342,29.13030273)
\curveto(432.4297498,29.14029203)(432.44974978,29.14029203)(432.46975342,29.13030273)
\curveto(432.48974974,29.13029204)(432.50974972,29.13529203)(432.52975342,29.14530273)
\lineto(436.87975342,29.14530273)
\curveto(436.94974528,29.14529202)(437.0247452,29.14529202)(437.10475342,29.14530273)
\curveto(437.18474504,29.15529201)(437.25474497,29.15529201)(437.31475342,29.14530273)
\lineto(437.47975342,29.14530273)
\curveto(437.53974469,29.13529203)(437.59474463,29.12529204)(437.64475342,29.11530273)
\curveto(437.70474452,29.11529205)(437.76974446,29.11029206)(437.83975342,29.10030273)
\curveto(437.91974431,29.08029209)(437.99974423,29.0652921)(438.07975342,29.05530273)
\curveto(438.15974407,29.04529212)(438.23974399,29.03029214)(438.31975342,29.01030273)
\curveto(438.49974373,28.95029222)(438.65974357,28.88529228)(438.79975342,28.81530273)
\curveto(438.94974328,28.74529242)(439.08474314,28.66029251)(439.20475342,28.56030273)
\curveto(439.4247428,28.39029278)(439.58474264,28.18029299)(439.68475342,27.93030273)
\curveto(439.79474243,27.69029348)(439.86474236,27.40529376)(439.89475342,27.07530273)
\curveto(439.90474232,26.99529417)(439.89974233,26.91029426)(439.87975342,26.82030273)
\curveto(439.86974236,26.74029443)(439.86974236,26.66029451)(439.87975342,26.58030273)
\lineto(439.84975342,26.43030273)
\curveto(439.84974238,26.38029479)(439.83974239,26.32029485)(439.81975342,26.25030273)
\curveto(439.79974243,26.19029498)(439.77974245,26.13529503)(439.75975342,26.08530273)
\lineto(439.72975342,25.92030273)
\curveto(439.68974254,25.84029533)(439.65974257,25.7652954)(439.63975342,25.69530273)
\curveto(439.61974261,25.62529554)(439.59474263,25.55529561)(439.56475342,25.48530273)
\curveto(439.48474274,25.33529583)(439.40974282,25.19029598)(439.33975342,25.05030273)
\curveto(439.26974296,24.92029625)(439.17974305,24.79529637)(439.06975342,24.67530273)
\curveto(439.0297432,24.62529654)(438.98974324,24.58029659)(438.94975342,24.54030273)
\curveto(438.90974332,24.50029667)(438.86974336,24.45529671)(438.82975342,24.40530273)
\curveto(438.80974342,24.39529677)(438.79474343,24.38529678)(438.78475342,24.37530273)
\curveto(438.77474345,24.37529679)(438.76474346,24.3702968)(438.75475342,24.36030273)
\curveto(438.73474349,24.34029683)(438.70474352,24.31529685)(438.66475342,24.28530273)
\lineto(438.58975342,24.21030273)
\curveto(438.49974373,24.15029702)(438.41474381,24.09029708)(438.33475342,24.03030273)
\curveto(438.25474397,23.98029719)(438.16974406,23.93029724)(438.07975342,23.88030273)
\curveto(438.01974421,23.85029732)(437.96474426,23.81529735)(437.91475342,23.77530273)
\curveto(437.87474435,23.74529742)(437.83974439,23.70029747)(437.80975342,23.64030273)
\curveto(437.78974444,23.58029759)(437.80474442,23.53029764)(437.85475342,23.49030273)
\curveto(437.90474432,23.45029772)(437.94474428,23.42029775)(437.97475342,23.40030273)
\curveto(438.07474415,23.33029784)(438.16474406,23.25529791)(438.24475342,23.17530273)
\curveto(438.3247439,23.09529807)(438.38474384,23.00029817)(438.42475342,22.89030273)
\curveto(438.51474371,22.75029842)(438.56474366,22.59029858)(438.57475342,22.41030273)
\curveto(438.59474363,22.24029893)(438.60974362,22.05529911)(438.61975342,21.85530273)
\lineto(438.58975342,21.61530273)
\curveto(438.58974364,21.54529962)(438.58474364,21.4702997)(438.57475342,21.39030273)
\curveto(438.58474364,21.32029985)(438.57474365,21.25029992)(438.54475342,21.18030273)
\curveto(438.5247437,21.11030006)(438.51974371,21.04030013)(438.52975342,20.97030273)
\lineto(438.49975342,20.83530273)
\curveto(438.50974372,20.7653004)(438.49974373,20.69030048)(438.46975342,20.61030273)
\curveto(438.43974379,20.53030064)(438.4297438,20.45030072)(438.43975342,20.37030273)
\curveto(438.43974379,20.33030084)(438.43474379,20.29030088)(438.42475342,20.25030273)
\curveto(438.41474381,20.22030095)(438.40974382,20.18030099)(438.40975342,20.13030273)
\curveto(438.40974382,20.03030114)(438.39974383,19.92530124)(438.37975342,19.81530273)
\curveto(438.36974386,19.71530145)(438.37474385,19.62030155)(438.39475342,19.53030273)
\curveto(438.39474383,19.4703017)(438.38974384,19.41030176)(438.37975342,19.35030273)
\curveto(438.37974385,19.30030187)(438.38474384,19.24530192)(438.39475342,19.18530273)
\lineto(438.45475342,18.82530273)
\moveto(437.88475342,25.05030273)
\curveto(437.97474425,25.16029601)(438.04974418,25.27529589)(438.10975342,25.39530273)
\curveto(438.16974406,25.51529565)(438.229744,25.64529552)(438.28975342,25.78530273)
\lineto(438.31975342,25.92030273)
\curveto(438.37974385,26.06029511)(438.41474381,26.21029496)(438.42475342,26.37030273)
\curveto(438.43474379,26.54029463)(438.4297438,26.68029449)(438.40975342,26.79030273)
\curveto(438.34974388,27.29029388)(438.10474412,27.63529353)(437.67475342,27.82530273)
\curveto(437.49474473,27.90529326)(437.26974496,27.95029322)(436.99975342,27.96030273)
\curveto(436.73974549,27.9702932)(436.46974576,27.97529319)(436.18975342,27.97530273)
\lineto(433.71475342,27.97530273)
\curveto(433.69474853,27.9652932)(433.66974856,27.96029321)(433.63975342,27.96030273)
\curveto(433.61974861,27.96029321)(433.59474863,27.95529321)(433.56475342,27.94530273)
\curveto(433.43474879,27.91529325)(433.33974889,27.85029332)(433.27975342,27.75030273)
\curveto(433.21974901,27.66029351)(433.17474905,27.53529363)(433.14475342,27.37530273)
\curveto(433.1247491,27.21529395)(433.09974913,27.0702941)(433.06975342,26.94030273)
\lineto(432.72475342,25.21530273)
\curveto(432.69474953,25.0652961)(432.65974957,24.90529626)(432.61975342,24.73530273)
\curveto(432.58974964,24.57529659)(432.59974963,24.45029672)(432.64975342,24.36030273)
\curveto(432.68974954,24.29029688)(432.75474947,24.24529692)(432.84475342,24.22530273)
\curveto(432.94474928,24.21529695)(433.05474917,24.21029696)(433.17475342,24.21030273)
\lineto(434.10475342,24.21030273)
\curveto(434.49474773,24.21029696)(434.87474735,24.20529696)(435.24475342,24.19530273)
\curveto(435.61474661,24.19529697)(435.95974627,24.21529695)(436.27975342,24.25530273)
\curveto(436.60974562,24.30529686)(436.90974532,24.39029678)(437.17975342,24.51030273)
\curveto(437.44974478,24.63029654)(437.68474454,24.81029636)(437.88475342,25.05030273)
}
}
{
\newrgbcolor{curcolor}{0 0 0}
\pscustom[linestyle=none,fillstyle=solid,fillcolor=curcolor]
{
\newpath
\moveto(447.97295654,22.65030273)
\curveto(447.98294765,22.59029858)(447.97294766,22.49529867)(447.94295654,22.36530273)
\curveto(447.92294771,22.24529892)(447.90294773,22.16029901)(447.88295654,22.11030273)
\lineto(447.85295654,21.96030273)
\curveto(447.82294781,21.88029929)(447.79794784,21.80529936)(447.77795654,21.73530273)
\curveto(447.76794787,21.67529949)(447.74794789,21.60529956)(447.71795654,21.52530273)
\curveto(447.68794795,21.4652997)(447.66294797,21.40529976)(447.64295654,21.34530273)
\curveto(447.632948,21.28529988)(447.60794803,21.22529994)(447.56795654,21.16530273)
\lineto(447.38795654,20.77530273)
\curveto(447.3379483,20.64530052)(447.27294836,20.52530064)(447.19295654,20.41530273)
\curveto(446.89294874,19.93530123)(446.5329491,19.53030164)(446.11295654,19.20030273)
\curveto(445.70294993,18.88030229)(445.22295041,18.63530253)(444.67295654,18.46530273)
\curveto(444.56295107,18.42530274)(444.44295119,18.39530277)(444.31295654,18.37530273)
\curveto(444.18295145,18.35530281)(444.04795159,18.33530283)(443.90795654,18.31530273)
\curveto(443.84795179,18.30530286)(443.78295185,18.30030287)(443.71295654,18.30030273)
\curveto(443.65295198,18.29030288)(443.59295204,18.28530288)(443.53295654,18.28530273)
\curveto(443.49295214,18.27530289)(443.4329522,18.2703029)(443.35295654,18.27030273)
\curveto(443.28295235,18.2703029)(443.2329524,18.27530289)(443.20295654,18.28530273)
\curveto(443.16295247,18.29530287)(443.12295251,18.30030287)(443.08295654,18.30030273)
\curveto(443.04295259,18.29030288)(443.00795263,18.29030288)(442.97795654,18.30030273)
\lineto(442.88795654,18.30030273)
\lineto(442.54295654,18.34530273)
\lineto(442.15295654,18.46530273)
\curveto(442.0329536,18.50530266)(441.91795372,18.55030262)(441.80795654,18.60030273)
\curveto(441.39795424,18.80030237)(441.07795456,19.06030211)(440.84795654,19.38030273)
\curveto(440.62795501,19.70030147)(440.46795517,20.09030108)(440.36795654,20.55030273)
\curveto(440.3379553,20.65030052)(440.31795532,20.75030042)(440.30795654,20.85030273)
\lineto(440.30795654,21.16530273)
\curveto(440.29795534,21.20529996)(440.29795534,21.23529993)(440.30795654,21.25530273)
\curveto(440.31795532,21.28529988)(440.32295531,21.32029985)(440.32295654,21.36030273)
\curveto(440.32295531,21.44029973)(440.32795531,21.52029965)(440.33795654,21.60030273)
\curveto(440.34795529,21.69029948)(440.35295528,21.77529939)(440.35295654,21.85530273)
\curveto(440.36295527,21.90529926)(440.36795527,21.94529922)(440.36795654,21.97530273)
\curveto(440.37795526,22.01529915)(440.38295525,22.06029911)(440.38295654,22.11030273)
\curveto(440.38295525,22.16029901)(440.39295524,22.24529892)(440.41295654,22.36530273)
\curveto(440.44295519,22.49529867)(440.47295516,22.59029858)(440.50295654,22.65030273)
\curveto(440.54295509,22.72029845)(440.56295507,22.79029838)(440.56295654,22.86030273)
\curveto(440.56295507,22.93029824)(440.58295505,23.00029817)(440.62295654,23.07030273)
\curveto(440.64295499,23.12029805)(440.65795498,23.16029801)(440.66795654,23.19030273)
\curveto(440.67795496,23.23029794)(440.69295494,23.27529789)(440.71295654,23.32530273)
\curveto(440.77295486,23.44529772)(440.82295481,23.5652976)(440.86295654,23.68530273)
\curveto(440.91295472,23.80529736)(440.97795466,23.92029725)(441.05795654,24.03030273)
\curveto(441.27795436,24.40029677)(441.52295411,24.73029644)(441.79295654,25.02030273)
\curveto(442.07295356,25.32029585)(442.38795325,25.5702956)(442.73795654,25.77030273)
\curveto(442.86795277,25.85029532)(443.00295263,25.91529525)(443.14295654,25.96530273)
\lineto(443.59295654,26.14530273)
\curveto(443.72295191,26.19529497)(443.85795178,26.22529494)(443.99795654,26.23530273)
\curveto(444.1379515,26.25529491)(444.28295135,26.28529488)(444.43295654,26.32530273)
\lineto(444.62795654,26.32530273)
\lineto(444.83795654,26.35530273)
\curveto(445.72794991,26.3652948)(446.42794921,26.18029499)(446.93795654,25.80030273)
\curveto(447.45794818,25.42029575)(447.78294785,24.92529624)(447.91295654,24.31530273)
\curveto(447.94294769,24.21529695)(447.96294767,24.11529705)(447.97295654,24.01530273)
\curveto(447.98294765,23.91529725)(447.99794764,23.81029736)(448.01795654,23.70030273)
\curveto(448.02794761,23.59029758)(448.02794761,23.4702977)(448.01795654,23.34030273)
\lineto(448.01795654,22.96530273)
\curveto(448.01794762,22.91529825)(448.00794763,22.86029831)(447.98795654,22.80030273)
\curveto(447.97794766,22.75029842)(447.97294766,22.70029847)(447.97295654,22.65030273)
\moveto(446.47295654,21.79530273)
\curveto(446.50294913,21.8652993)(446.52294911,21.94529922)(446.53295654,22.03530273)
\curveto(446.55294908,22.12529904)(446.56794907,22.21029896)(446.57795654,22.29030273)
\curveto(446.65794898,22.68029849)(446.69294894,23.01029816)(446.68295654,23.28030273)
\curveto(446.66294897,23.36029781)(446.64794899,23.44029773)(446.63795654,23.52030273)
\curveto(446.637949,23.60029757)(446.632949,23.67529749)(446.62295654,23.74530273)
\curveto(446.47294916,24.39529677)(446.11794952,24.84529632)(445.55795654,25.09530273)
\curveto(445.48795015,25.12529604)(445.41295022,25.14529602)(445.33295654,25.15530273)
\curveto(445.26295037,25.17529599)(445.18795045,25.19529597)(445.10795654,25.21530273)
\curveto(445.0379506,25.23529593)(444.95795068,25.24529592)(444.86795654,25.24530273)
\lineto(444.59795654,25.24530273)
\lineto(444.31295654,25.20030273)
\curveto(444.21295142,25.18029599)(444.11795152,25.15529601)(444.02795654,25.12530273)
\curveto(443.9379517,25.10529606)(443.84795179,25.07529609)(443.75795654,25.03530273)
\curveto(443.68795195,25.01529615)(443.61795202,24.98529618)(443.54795654,24.94530273)
\curveto(443.47795216,24.90529626)(443.41295222,24.8652963)(443.35295654,24.82530273)
\curveto(443.08295255,24.65529651)(442.84795279,24.45029672)(442.64795654,24.21030273)
\curveto(442.44795319,23.9702972)(442.26295337,23.69029748)(442.09295654,23.37030273)
\curveto(442.04295359,23.2702979)(442.00295363,23.165298)(441.97295654,23.05530273)
\curveto(441.94295369,22.95529821)(441.90295373,22.85029832)(441.85295654,22.74030273)
\curveto(441.84295379,22.70029847)(441.82795381,22.63529853)(441.80795654,22.54530273)
\curveto(441.78795385,22.51529865)(441.77795386,22.48029869)(441.77795654,22.44030273)
\curveto(441.77795386,22.40029877)(441.77295386,22.35529881)(441.76295654,22.30530273)
\lineto(441.70295654,22.00530273)
\curveto(441.68295395,21.90529926)(441.67295396,21.81529935)(441.67295654,21.73530273)
\lineto(441.67295654,21.55530273)
\curveto(441.67295396,21.45529971)(441.66795397,21.35529981)(441.65795654,21.25530273)
\curveto(441.65795398,21.1653)(441.66795397,21.08030009)(441.68795654,21.00030273)
\curveto(441.7379539,20.76030041)(441.80795383,20.53530063)(441.89795654,20.32530273)
\curveto(441.99795364,20.11530105)(442.1329535,19.94030123)(442.30295654,19.80030273)
\curveto(442.35295328,19.7703014)(442.39295324,19.74530142)(442.42295654,19.72530273)
\curveto(442.46295317,19.70530146)(442.50295313,19.68030149)(442.54295654,19.65030273)
\curveto(442.61295302,19.60030157)(442.69295294,19.55530161)(442.78295654,19.51530273)
\curveto(442.87295276,19.48530168)(442.96795267,19.45530171)(443.06795654,19.42530273)
\curveto(443.11795252,19.40530176)(443.16295247,19.39530177)(443.20295654,19.39530273)
\curveto(443.25295238,19.40530176)(443.30295233,19.40530176)(443.35295654,19.39530273)
\curveto(443.38295225,19.38530178)(443.44295219,19.37530179)(443.53295654,19.36530273)
\curveto(443.62295201,19.35530181)(443.69795194,19.36030181)(443.75795654,19.38030273)
\curveto(443.79795184,19.39030178)(443.8379518,19.39030178)(443.87795654,19.38030273)
\curveto(443.91795172,19.38030179)(443.95795168,19.39030178)(443.99795654,19.41030273)
\curveto(444.07795156,19.43030174)(444.15795148,19.44530172)(444.23795654,19.45530273)
\curveto(444.32795131,19.47530169)(444.41295122,19.50030167)(444.49295654,19.53030273)
\curveto(444.85295078,19.6703015)(445.16295047,19.8653013)(445.42295654,20.11530273)
\curveto(445.68294995,20.3653008)(445.91794972,20.66030051)(446.12795654,21.00030273)
\curveto(446.20794943,21.12030005)(446.26794937,21.24529992)(446.30795654,21.37530273)
\curveto(446.34794929,21.51529965)(446.40294923,21.65529951)(446.47295654,21.79530273)
}
}
{
\newrgbcolor{curcolor}{0 0 0}
\pscustom[linestyle=none,fillstyle=solid,fillcolor=curcolor]
{
\newpath
\moveto(451.33623779,29.13030273)
\curveto(451.46623403,29.13029204)(451.6012339,29.13029204)(451.74123779,29.13030273)
\curveto(451.89123361,29.13029204)(451.99123351,29.09529207)(452.04123779,29.02530273)
\curveto(452.08123342,28.95529221)(452.09123341,28.86029231)(452.07123779,28.74030273)
\curveto(452.05123345,28.63029254)(452.03123347,28.51529265)(452.01123779,28.39530273)
\lineto(451.74123779,27.06030273)
\lineto(450.52623779,20.98530273)
\lineto(450.19623779,19.30530273)
\curveto(450.16623533,19.18530198)(450.13623536,19.05530211)(450.10623779,18.91530273)
\curveto(450.08623541,18.77530239)(450.04123546,18.6653025)(449.97123779,18.58530273)
\curveto(449.93123557,18.53530263)(449.88123562,18.50530266)(449.82123779,18.49530273)
\curveto(449.77123573,18.48530268)(449.7012358,18.4703027)(449.61123779,18.45030273)
\lineto(449.40123779,18.45030273)
\lineto(449.08623779,18.45030273)
\curveto(448.98623651,18.46030271)(448.92123658,18.49530267)(448.89123779,18.55530273)
\curveto(448.85123665,18.63530253)(448.84123666,18.73530243)(448.86123779,18.85530273)
\curveto(448.88123662,18.97530219)(448.90623659,19.10030207)(448.93623779,19.23030273)
\lineto(449.20623779,20.61030273)
\lineto(450.45123779,26.85030273)
\lineto(450.75123779,28.32030273)
\curveto(450.77123473,28.43029274)(450.79123471,28.54529262)(450.81123779,28.66530273)
\curveto(450.83123467,28.79529237)(450.87123463,28.89529227)(450.93123779,28.96530273)
\curveto(450.99123451,29.02529214)(451.07623442,29.07529209)(451.18623779,29.11530273)
\curveto(451.21623428,29.12529204)(451.24123426,29.12529204)(451.26123779,29.11530273)
\curveto(451.28123422,29.11529205)(451.30623419,29.12029205)(451.33623779,29.13030273)
}
}
{
\newrgbcolor{curcolor}{0 0 0}
\pscustom[linestyle=none,fillstyle=solid,fillcolor=curcolor]
{
\newpath
\moveto(459.55108154,22.62030273)
\curveto(459.55107304,22.52029865)(459.53107306,22.40529876)(459.49108154,22.27530273)
\curveto(459.45107314,22.15529901)(459.40107319,22.0702991)(459.34108154,22.02030273)
\curveto(459.28107331,21.98029919)(459.20107339,21.95029922)(459.10108154,21.93030273)
\curveto(459.00107359,21.92029925)(458.8910737,21.91529925)(458.77108154,21.91530273)
\lineto(458.41108154,21.91530273)
\curveto(458.30107429,21.92529924)(458.20107439,21.93029924)(458.11108154,21.93030273)
\lineto(454.27108154,21.93030273)
\curveto(454.1910784,21.93029924)(454.10607848,21.92529924)(454.01608154,21.91530273)
\curveto(453.93607865,21.91529925)(453.87107872,21.90029927)(453.82108154,21.87030273)
\curveto(453.77107882,21.85029932)(453.72107887,21.81029936)(453.67108154,21.75030273)
\lineto(453.58108154,21.61530273)
\curveto(453.55107904,21.5652996)(453.54107905,21.51529965)(453.55108154,21.46530273)
\curveto(453.55107904,21.41529975)(453.54607904,21.3702998)(453.53608154,21.33030273)
\lineto(453.53608154,21.21030273)
\lineto(453.53608154,20.95530273)
\curveto(453.54607904,20.87530029)(453.56107903,20.79530037)(453.58108154,20.71530273)
\curveto(453.71107888,20.17530099)(454.01607857,19.79030138)(454.49608154,19.56030273)
\curveto(454.54607804,19.53030164)(454.60607798,19.50530166)(454.67608154,19.48530273)
\curveto(454.74607784,19.4653017)(454.81107778,19.44530172)(454.87108154,19.42530273)
\curveto(454.90107769,19.41530175)(454.95107764,19.41030176)(455.02108154,19.41030273)
\curveto(455.15107744,19.3703018)(455.33107726,19.35030182)(455.56108154,19.35030273)
\curveto(455.7910768,19.35030182)(455.98107661,19.3703018)(456.13108154,19.41030273)
\curveto(456.28107631,19.45030172)(456.41607617,19.49030168)(456.53608154,19.53030273)
\curveto(456.66607592,19.58030159)(456.7860758,19.64030153)(456.89608154,19.71030273)
\curveto(457.01607557,19.78030139)(457.12607546,19.86030131)(457.22608154,19.95030273)
\curveto(457.32607526,20.05030112)(457.41607517,20.15530101)(457.49608154,20.26530273)
\curveto(457.57607501,20.3653008)(457.65107494,20.4703007)(457.72108154,20.58030273)
\curveto(457.7910748,20.69030048)(457.8860747,20.7703004)(458.00608154,20.82030273)
\curveto(458.04607454,20.84030033)(458.11107448,20.85530031)(458.20108154,20.86530273)
\curveto(458.30107429,20.87530029)(458.3910742,20.87530029)(458.47108154,20.86530273)
\curveto(458.56107403,20.8653003)(458.64607394,20.86030031)(458.72608154,20.85030273)
\curveto(458.80607378,20.84030033)(458.85607373,20.82030035)(458.87608154,20.79030273)
\curveto(458.96607362,20.72030045)(458.97107362,20.60530056)(458.89108154,20.44530273)
\curveto(458.75107384,20.17530099)(458.59607399,19.93530123)(458.42608154,19.72530273)
\curveto(458.16607442,19.40530176)(457.8860747,19.14030203)(457.58608154,18.93030273)
\curveto(457.29607529,18.73030244)(456.94107565,18.5653026)(456.52108154,18.43530273)
\curveto(456.41107618,18.39530277)(456.30607628,18.3703028)(456.20608154,18.36030273)
\curveto(456.10607648,18.34030283)(455.99607659,18.32030285)(455.87608154,18.30030273)
\curveto(455.82607676,18.29030288)(455.77607681,18.28530288)(455.72608154,18.28530273)
\curveto(455.6860769,18.28530288)(455.64107695,18.28030289)(455.59108154,18.27030273)
\lineto(455.44108154,18.27030273)
\curveto(455.3910772,18.26030291)(455.33107726,18.25530291)(455.26108154,18.25530273)
\curveto(455.20107739,18.25530291)(455.15107744,18.26030291)(455.11108154,18.27030273)
\lineto(454.97608154,18.27030273)
\curveto(454.92607766,18.28030289)(454.88107771,18.28530288)(454.84108154,18.28530273)
\curveto(454.80107779,18.28530288)(454.76107783,18.29030288)(454.72108154,18.30030273)
\curveto(454.67107792,18.31030286)(454.61607797,18.32030285)(454.55608154,18.33030273)
\curveto(454.50607808,18.33030284)(454.45607813,18.33530283)(454.40608154,18.34530273)
\curveto(454.31607827,18.3653028)(454.22607836,18.39030278)(454.13608154,18.42030273)
\curveto(454.05607853,18.44030273)(453.98107861,18.4653027)(453.91108154,18.49530273)
\curveto(453.87107872,18.51530265)(453.83607875,18.52530264)(453.80608154,18.52530273)
\curveto(453.77607881,18.53530263)(453.74607884,18.55030262)(453.71608154,18.57030273)
\curveto(453.57607901,18.64030253)(453.43107916,18.72530244)(453.28108154,18.82530273)
\curveto(453.03107956,19.01530215)(452.83107976,19.24530192)(452.68108154,19.51530273)
\curveto(452.53108006,19.79530137)(452.42108017,20.10530106)(452.35108154,20.44530273)
\curveto(452.32108027,20.55530061)(452.30608028,20.6703005)(452.30608154,20.79030273)
\curveto(452.30608028,20.91030026)(452.29608029,21.03030014)(452.27608154,21.15030273)
\lineto(452.27608154,21.25530273)
\curveto(452.2860803,21.28529988)(452.2910803,21.32529984)(452.29108154,21.37530273)
\lineto(452.29108154,21.63030273)
\curveto(452.30108029,21.72029945)(452.30608028,21.81029936)(452.30608154,21.90030273)
\lineto(452.35108154,22.11030273)
\curveto(452.35108024,22.15029902)(452.35608023,22.20529896)(452.36608154,22.27530273)
\curveto(452.37608021,22.35529881)(452.3910802,22.42029875)(452.41108154,22.47030273)
\lineto(452.44108154,22.63530273)
\curveto(452.47108012,22.68529848)(452.4860801,22.73529843)(452.48608154,22.78530273)
\curveto(452.49608009,22.84529832)(452.51108008,22.90029827)(452.53108154,22.95030273)
\curveto(452.60107999,23.11029806)(452.66607992,23.2702979)(452.72608154,23.43030273)
\curveto(452.7860798,23.59029758)(452.86107973,23.74029743)(452.95108154,23.88030273)
\curveto(453.02107957,23.99029718)(453.0860795,24.10029707)(453.14608154,24.21030273)
\curveto(453.21607937,24.33029684)(453.29607929,24.44529672)(453.38608154,24.55530273)
\curveto(453.67607891,24.90529626)(453.9860786,25.20529596)(454.31608154,25.45530273)
\curveto(454.64607794,25.71529545)(455.03107756,25.93029524)(455.47108154,26.10030273)
\curveto(455.60107699,26.15029502)(455.73107686,26.18529498)(455.86108154,26.20530273)
\curveto(455.9910766,26.23529493)(456.13107646,26.2652949)(456.28108154,26.29530273)
\curveto(456.33107626,26.30529486)(456.37607621,26.31029486)(456.41608154,26.31030273)
\curveto(456.45607613,26.32029485)(456.50107609,26.32529484)(456.55108154,26.32530273)
\curveto(456.57107602,26.33529483)(456.59607599,26.33529483)(456.62608154,26.32530273)
\curveto(456.65607593,26.31529485)(456.68107591,26.32029485)(456.70108154,26.34030273)
\curveto(457.13107546,26.35029482)(457.4910751,26.30529486)(457.78108154,26.20530273)
\curveto(458.07107452,26.11529505)(458.32607426,25.99029518)(458.54608154,25.83030273)
\curveto(458.586074,25.81029536)(458.61607397,25.78029539)(458.63608154,25.74030273)
\curveto(458.66607392,25.71029546)(458.69607389,25.68529548)(458.72608154,25.66530273)
\curveto(458.79607379,25.60529556)(458.86607372,25.53529563)(458.93608154,25.45530273)
\curveto(459.00607358,25.37529579)(459.06107353,25.29529587)(459.10108154,25.21530273)
\curveto(459.22107337,25.00529616)(459.31607327,24.80529636)(459.38608154,24.61530273)
\curveto(459.43607315,24.50529666)(459.46607312,24.38529678)(459.47608154,24.25530273)
\lineto(459.53608154,23.86530273)
\curveto(459.56607302,23.73529743)(459.57607301,23.60029757)(459.56608154,23.46030273)
\curveto(459.56607302,23.32029785)(459.57107302,23.18029799)(459.58108154,23.04030273)
\curveto(459.58107301,22.9702982)(459.57607301,22.90029827)(459.56608154,22.83030273)
\curveto(459.55607303,22.76029841)(459.55107304,22.69029848)(459.55108154,22.62030273)
\moveto(458.20108154,23.13030273)
\curveto(458.23107436,23.170298)(458.26107433,23.22029795)(458.29108154,23.28030273)
\curveto(458.33107426,23.35029782)(458.34607424,23.42029775)(458.33608154,23.49030273)
\curveto(458.32607426,23.71029746)(458.2860743,23.91529725)(458.21608154,24.10530273)
\curveto(458.11607447,24.33529683)(457.99607459,24.53029664)(457.85608154,24.69030273)
\curveto(457.72607486,24.85029632)(457.53607505,24.98529618)(457.28608154,25.09530273)
\curveto(457.21607537,25.11529605)(457.14607544,25.13029604)(457.07608154,25.14030273)
\curveto(457.01607557,25.16029601)(456.94607564,25.18029599)(456.86608154,25.20030273)
\curveto(456.79607579,25.22029595)(456.71607587,25.23029594)(456.62608154,25.23030273)
\lineto(456.37108154,25.23030273)
\curveto(456.33107626,25.21029596)(456.2910763,25.20029597)(456.25108154,25.20030273)
\curveto(456.21107638,25.21029596)(456.17607641,25.21029596)(456.14608154,25.20030273)
\lineto(455.90608154,25.14030273)
\curveto(455.82607676,25.13029604)(455.75107684,25.11529605)(455.68108154,25.09530273)
\curveto(455.36107723,24.97529619)(455.09607749,24.82529634)(454.88608154,24.64530273)
\curveto(454.67607791,24.4652967)(454.47607811,24.24029693)(454.28608154,23.97030273)
\curveto(454.24607834,23.92029725)(454.20107839,23.85529731)(454.15108154,23.77530273)
\curveto(454.11107848,23.70529746)(454.07107852,23.62529754)(454.03108154,23.53530273)
\curveto(453.9910786,23.44529772)(453.96607862,23.36029781)(453.95608154,23.28030273)
\curveto(453.95607863,23.20029797)(453.98107861,23.14029803)(454.03108154,23.10030273)
\curveto(454.10107849,23.04029813)(454.23107836,23.01029816)(454.42108154,23.01030273)
\curveto(454.62107797,23.02029815)(454.7910778,23.02529814)(454.93108154,23.02530273)
\lineto(457.21108154,23.02530273)
\curveto(457.36107523,23.02529814)(457.54107505,23.02029815)(457.75108154,23.01030273)
\curveto(457.96107463,23.01029816)(458.11107448,23.05029812)(458.20108154,23.13030273)
}
}
{
\newrgbcolor{curcolor}{0 0 0}
\pscustom[linestyle=none,fillstyle=solid,fillcolor=curcolor]
{
\newpath
\moveto(464.03772217,26.35530273)
\curveto(464.75771651,26.3652948)(465.34271593,26.28029489)(465.79272217,26.10030273)
\curveto(466.25271502,25.93029524)(466.5727147,25.62529554)(466.75272217,25.18530273)
\curveto(466.80271447,25.07529609)(466.83271444,24.96029621)(466.84272217,24.84030273)
\curveto(466.86271441,24.73029644)(466.87771439,24.60529656)(466.88772217,24.46530273)
\curveto(466.89771437,24.39529677)(466.88771438,24.32029685)(466.85772217,24.24030273)
\curveto(466.83771443,24.170297)(466.81271446,24.11529705)(466.78272217,24.07530273)
\curveto(466.76271451,24.05529711)(466.73271454,24.03529713)(466.69272217,24.01530273)
\curveto(466.66271461,24.00529716)(466.63771463,23.99029718)(466.61772217,23.97030273)
\curveto(466.55771471,23.95029722)(466.50271477,23.94529722)(466.45272217,23.95530273)
\curveto(466.41271486,23.9652972)(466.3677149,23.9652972)(466.31772217,23.95530273)
\curveto(466.22771504,23.93529723)(466.11771515,23.93029724)(465.98772217,23.94030273)
\curveto(465.8677154,23.96029721)(465.78271549,23.98529718)(465.73272217,24.01530273)
\curveto(465.66271561,24.0652971)(465.62271565,24.13029704)(465.61272217,24.21030273)
\curveto(465.61271566,24.30029687)(465.59271568,24.38529678)(465.55272217,24.46530273)
\curveto(465.50271577,24.62529654)(465.40771586,24.7702964)(465.26772217,24.90030273)
\curveto(465.17771609,24.98029619)(465.0677162,25.04029613)(464.93772217,25.08030273)
\curveto(464.81771645,25.12029605)(464.68771658,25.16029601)(464.54772217,25.20030273)
\curveto(464.50771676,25.22029595)(464.45771681,25.22529594)(464.39772217,25.21530273)
\curveto(464.34771692,25.21529595)(464.30271697,25.22029595)(464.26272217,25.23030273)
\curveto(464.20271707,25.25029592)(464.12771714,25.26029591)(464.03772217,25.26030273)
\curveto(463.94771732,25.26029591)(463.8727174,25.25029592)(463.81272217,25.23030273)
\lineto(463.72272217,25.23030273)
\curveto(463.66271761,25.22029595)(463.60771766,25.21029596)(463.55772217,25.20030273)
\curveto(463.50771776,25.20029597)(463.45771781,25.19529597)(463.40772217,25.18530273)
\curveto(463.13771813,25.12529604)(462.90271837,25.04029613)(462.70272217,24.93030273)
\curveto(462.51271876,24.82029635)(462.36271891,24.63529653)(462.25272217,24.37530273)
\curveto(462.22271905,24.30529686)(462.20771906,24.23529693)(462.20772217,24.16530273)
\curveto(462.20771906,24.09529707)(462.21271906,24.03529713)(462.22272217,23.98530273)
\curveto(462.25271902,23.83529733)(462.30271897,23.72529744)(462.37272217,23.65530273)
\curveto(462.44271883,23.59529757)(462.53771873,23.52529764)(462.65772217,23.44530273)
\curveto(462.79771847,23.34529782)(462.96271831,23.2702979)(463.15272217,23.22030273)
\curveto(463.34271793,23.18029799)(463.53271774,23.13029804)(463.72272217,23.07030273)
\curveto(463.84271743,23.03029814)(463.96271731,23.00029817)(464.08272217,22.98030273)
\curveto(464.21271706,22.96029821)(464.33771693,22.93029824)(464.45772217,22.89030273)
\curveto(464.65771661,22.83029834)(464.85271642,22.7702984)(465.04272217,22.71030273)
\curveto(465.23271604,22.66029851)(465.41771585,22.59529857)(465.59772217,22.51530273)
\curveto(465.64771562,22.49529867)(465.69271558,22.47529869)(465.73272217,22.45530273)
\curveto(465.78271549,22.43529873)(465.83271544,22.41029876)(465.88272217,22.38030273)
\curveto(466.05271522,22.26029891)(466.19771507,22.12529904)(466.31772217,21.97530273)
\curveto(466.43771483,21.82529934)(466.52771474,21.63529953)(466.58772217,21.40530273)
\lineto(466.58772217,21.12030273)
\curveto(466.58771468,21.05030012)(466.58271469,20.97530019)(466.57272217,20.89530273)
\curveto(466.56271471,20.82530034)(466.55271472,20.74530042)(466.54272217,20.65530273)
\lineto(466.51272217,20.50530273)
\curveto(466.4727148,20.43530073)(466.44271483,20.3653008)(466.42272217,20.29530273)
\curveto(466.41271486,20.22530094)(466.39271488,20.15530101)(466.36272217,20.08530273)
\curveto(466.31271496,19.97530119)(466.25771501,19.8703013)(466.19772217,19.77030273)
\curveto(466.13771513,19.6703015)(466.0727152,19.58030159)(466.00272217,19.50030273)
\curveto(465.79271548,19.24030193)(465.54771572,19.03030214)(465.26772217,18.87030273)
\curveto(464.98771628,18.72030245)(464.68271659,18.59030258)(464.35272217,18.48030273)
\curveto(464.25271702,18.45030272)(464.15271712,18.43030274)(464.05272217,18.42030273)
\curveto(463.95271732,18.40030277)(463.85771741,18.37530279)(463.76772217,18.34530273)
\curveto(463.65771761,18.32530284)(463.55271772,18.31530285)(463.45272217,18.31530273)
\curveto(463.35271792,18.31530285)(463.25271802,18.30530286)(463.15272217,18.28530273)
\lineto(463.00272217,18.28530273)
\curveto(462.95271832,18.27530289)(462.88271839,18.2703029)(462.79272217,18.27030273)
\curveto(462.70271857,18.2703029)(462.63271864,18.27530289)(462.58272217,18.28530273)
\lineto(462.41772217,18.28530273)
\curveto(462.35771891,18.30530286)(462.29271898,18.31530285)(462.22272217,18.31530273)
\curveto(462.15271912,18.30530286)(462.09771917,18.31030286)(462.05772217,18.33030273)
\curveto(462.00771926,18.34030283)(461.94271933,18.34530282)(461.86272217,18.34530273)
\curveto(461.78271949,18.3653028)(461.70771956,18.38530278)(461.63772217,18.40530273)
\curveto(461.5677197,18.41530275)(461.49271978,18.43530273)(461.41272217,18.46530273)
\curveto(461.12272015,18.5653026)(460.87772039,18.69030248)(460.67772217,18.84030273)
\curveto(460.47772079,18.99030218)(460.31772095,19.18530198)(460.19772217,19.42530273)
\curveto(460.13772113,19.55530161)(460.08772118,19.69030148)(460.04772217,19.83030273)
\curveto(460.01772125,19.9703012)(459.99772127,20.12530104)(459.98772217,20.29530273)
\curveto(459.97772129,20.35530081)(459.98272129,20.42530074)(460.00272217,20.50530273)
\curveto(460.02272125,20.59530057)(460.04772122,20.6653005)(460.07772217,20.71530273)
\curveto(460.11772115,20.75530041)(460.17772109,20.79530037)(460.25772217,20.83530273)
\curveto(460.30772096,20.85530031)(460.37772089,20.8653003)(460.46772217,20.86530273)
\curveto(460.5677207,20.87530029)(460.65772061,20.87530029)(460.73772217,20.86530273)
\curveto(460.82772044,20.85530031)(460.91272036,20.84030033)(460.99272217,20.82030273)
\curveto(461.08272019,20.81030036)(461.13772013,20.79530037)(461.15772217,20.77530273)
\curveto(461.21772005,20.72530044)(461.24772002,20.65030052)(461.24772217,20.55030273)
\curveto(461.25772001,20.46030071)(461.27771999,20.37530079)(461.30772217,20.29530273)
\curveto(461.35771991,20.07530109)(461.45771981,19.90530126)(461.60772217,19.78530273)
\curveto(461.70771956,19.69530147)(461.82771944,19.62530154)(461.96772217,19.57530273)
\curveto(462.10771916,19.52530164)(462.25771901,19.47530169)(462.41772217,19.42530273)
\lineto(462.73272217,19.38030273)
\lineto(462.82272217,19.38030273)
\curveto(462.88271839,19.36030181)(462.9677183,19.35030182)(463.07772217,19.35030273)
\curveto(463.19771807,19.35030182)(463.30271797,19.36030181)(463.39272217,19.38030273)
\curveto(463.46271781,19.38030179)(463.51771775,19.38530178)(463.55772217,19.39530273)
\curveto(463.61771765,19.40530176)(463.67771759,19.41030176)(463.73772217,19.41030273)
\curveto(463.79771747,19.42030175)(463.85271742,19.43030174)(463.90272217,19.44030273)
\curveto(464.21271706,19.52030165)(464.46271681,19.62530154)(464.65272217,19.75530273)
\curveto(464.85271642,19.88530128)(465.01771625,20.10530106)(465.14772217,20.41530273)
\curveto(465.17771609,20.4653007)(465.19271608,20.52030065)(465.19272217,20.58030273)
\curveto(465.20271607,20.64030053)(465.20271607,20.68530048)(465.19272217,20.71530273)
\curveto(465.18271609,20.90530026)(465.14271613,21.04530012)(465.07272217,21.13530273)
\curveto(465.00271627,21.23529993)(464.90771636,21.32529984)(464.78772217,21.40530273)
\curveto(464.70771656,21.4652997)(464.61271666,21.51529965)(464.50272217,21.55530273)
\lineto(464.20272217,21.67530273)
\curveto(464.1727171,21.68529948)(464.14271713,21.69029948)(464.11272217,21.69030273)
\curveto(464.09271718,21.69029948)(464.0727172,21.70029947)(464.05272217,21.72030273)
\curveto(463.73271754,21.83029934)(463.39271788,21.91029926)(463.03272217,21.96030273)
\curveto(462.68271859,22.02029915)(462.36271891,22.11529905)(462.07272217,22.24530273)
\curveto(461.98271929,22.28529888)(461.89271938,22.32029885)(461.80272217,22.35030273)
\curveto(461.72271955,22.38029879)(461.64771962,22.42029875)(461.57772217,22.47030273)
\curveto(461.40771986,22.58029859)(461.25772001,22.70529846)(461.12772217,22.84530273)
\curveto(460.99772027,22.98529818)(460.90772036,23.16029801)(460.85772217,23.37030273)
\curveto(460.83772043,23.44029773)(460.82772044,23.51029766)(460.82772217,23.58030273)
\lineto(460.82772217,23.80530273)
\curveto(460.81772045,23.92529724)(460.83272044,24.06029711)(460.87272217,24.21030273)
\curveto(460.91272036,24.3702968)(460.95272032,24.50529666)(460.99272217,24.61530273)
\curveto(461.02272025,24.6652965)(461.04272023,24.70529646)(461.05272217,24.73530273)
\curveto(461.0727202,24.77529639)(461.09772017,24.81529635)(461.12772217,24.85530273)
\curveto(461.25772001,25.08529608)(461.41771985,25.28529588)(461.60772217,25.45530273)
\curveto(461.79771947,25.62529554)(462.00771926,25.77529539)(462.23772217,25.90530273)
\curveto(462.39771887,25.99529517)(462.5727187,26.0652951)(462.76272217,26.11530273)
\curveto(462.96271831,26.17529499)(463.1677181,26.23029494)(463.37772217,26.28030273)
\curveto(463.44771782,26.29029488)(463.51271776,26.30029487)(463.57272217,26.31030273)
\curveto(463.64271763,26.32029485)(463.71771755,26.33029484)(463.79772217,26.34030273)
\curveto(463.83771743,26.35029482)(463.87771739,26.35029482)(463.91772217,26.34030273)
\curveto(463.9677173,26.33029484)(464.00771726,26.33529483)(464.03772217,26.35530273)
}
}
{
\newrgbcolor{curcolor}{0 0 0}
\pscustom[linestyle=none,fillstyle=solid,fillcolor=curcolor]
{
}
}
{
\newrgbcolor{curcolor}{0 0 0}
\pscustom[linestyle=none,fillstyle=solid,fillcolor=curcolor]
{
\newpath
\moveto(478.80287842,19.26030273)
\lineto(478.71287842,18.87030273)
\curveto(478.69287049,18.75030242)(478.65287053,18.65030252)(478.59287842,18.57030273)
\curveto(478.52287066,18.50030267)(478.42787075,18.46030271)(478.30787842,18.45030273)
\lineto(477.96287842,18.45030273)
\curveto(477.90287128,18.45030272)(477.84287134,18.44530272)(477.78287842,18.43530273)
\curveto(477.73287145,18.43530273)(477.68787149,18.44530272)(477.64787842,18.46530273)
\curveto(477.56787161,18.48530268)(477.51787166,18.52530264)(477.49787842,18.58530273)
\curveto(477.46787171,18.63530253)(477.45787172,18.69530247)(477.46787842,18.76530273)
\curveto(477.4778717,18.83530233)(477.47287171,18.90530226)(477.45287842,18.97530273)
\curveto(477.45287173,18.99530217)(477.44287174,19.01030216)(477.42287842,19.02030273)
\lineto(477.39287842,19.08030273)
\curveto(477.29287189,19.09030208)(477.20787197,19.0703021)(477.13787842,19.02030273)
\curveto(477.0778721,18.9703022)(477.01287217,18.92030225)(476.94287842,18.87030273)
\curveto(476.71287247,18.72030245)(476.48787269,18.60530256)(476.26787842,18.52530273)
\curveto(476.0778731,18.44530272)(475.85787332,18.38530278)(475.60787842,18.34530273)
\curveto(475.36787381,18.30530286)(475.12287406,18.28530288)(474.87287842,18.28530273)
\curveto(474.63287455,18.27530289)(474.39287479,18.29030288)(474.15287842,18.33030273)
\curveto(473.92287526,18.36030281)(473.72787545,18.41530275)(473.56787842,18.49530273)
\curveto(473.08787609,18.71530245)(472.72287646,19.01030216)(472.47287842,19.38030273)
\curveto(472.23287695,19.76030141)(472.0778771,20.23030094)(472.00787842,20.79030273)
\curveto(471.98787719,20.88030029)(471.9778772,20.9703002)(471.97787842,21.06030273)
\curveto(471.98787719,21.16030001)(471.98787719,21.26029991)(471.97787842,21.36030273)
\curveto(471.9778772,21.41029976)(471.9828772,21.46029971)(471.99287842,21.51030273)
\curveto(472.00287718,21.56029961)(472.00787717,21.61029956)(472.00787842,21.66030273)
\curveto(471.99787718,21.71029946)(471.99787718,21.76029941)(472.00787842,21.81030273)
\curveto(472.02787715,21.8702993)(472.03787714,21.92529924)(472.03787842,21.97530273)
\lineto(472.06787842,22.12530273)
\curveto(472.05787712,22.17529899)(472.05787712,22.24029893)(472.06787842,22.32030273)
\curveto(472.08787709,22.40029877)(472.11287707,22.4652987)(472.14287842,22.51530273)
\lineto(472.18787842,22.68030273)
\curveto(472.21787696,22.75029842)(472.23787694,22.82029835)(472.24787842,22.89030273)
\curveto(472.25787692,22.9702982)(472.2778769,23.04529812)(472.30787842,23.11530273)
\curveto(472.32787685,23.165298)(472.34287684,23.21029796)(472.35287842,23.25030273)
\curveto(472.36287682,23.29029788)(472.3778768,23.33529783)(472.39787842,23.38530273)
\curveto(472.44787673,23.48529768)(472.49287669,23.58029759)(472.53287842,23.67030273)
\curveto(472.57287661,23.7702974)(472.61787656,23.8652973)(472.66787842,23.95530273)
\curveto(472.86787631,24.33529683)(473.09787608,24.67529649)(473.35787842,24.97530273)
\curveto(473.62787555,25.28529588)(473.92787525,25.54029563)(474.25787842,25.74030273)
\curveto(474.45787472,25.86029531)(474.65787452,25.96029521)(474.85787842,26.04030273)
\curveto(475.05787412,26.12029505)(475.27287391,26.19029498)(475.50287842,26.25030273)
\lineto(475.71287842,26.28030273)
\curveto(475.7828734,26.29029488)(475.85287333,26.30529486)(475.92287842,26.32530273)
\lineto(476.07287842,26.32530273)
\curveto(476.16287302,26.34529482)(476.2828729,26.35529481)(476.43287842,26.35530273)
\curveto(476.59287259,26.35529481)(476.70787247,26.34529482)(476.77787842,26.32530273)
\curveto(476.81787236,26.31529485)(476.87287231,26.31029486)(476.94287842,26.31030273)
\curveto(477.04287214,26.28029489)(477.14787203,26.25529491)(477.25787842,26.23530273)
\curveto(477.36787181,26.22529494)(477.46787171,26.19529497)(477.55787842,26.14530273)
\curveto(477.69787148,26.08529508)(477.82787135,26.02029515)(477.94787842,25.95030273)
\curveto(478.06787111,25.88029529)(478.177871,25.80029537)(478.27787842,25.71030273)
\curveto(478.32787085,25.66029551)(478.3778708,25.60529556)(478.42787842,25.54530273)
\curveto(478.48787069,25.49529567)(478.57287061,25.48029569)(478.68287842,25.50030273)
\lineto(478.75787842,25.57530273)
\curveto(478.7778704,25.59529557)(478.79287039,25.62529554)(478.80287842,25.66530273)
\curveto(478.85287033,25.75529541)(478.88787029,25.8702953)(478.90787842,26.01030273)
\curveto(478.93787024,26.15029502)(478.96287022,26.27529489)(478.98287842,26.38530273)
\lineto(479.32787842,28.11030273)
\curveto(479.35786982,28.25029292)(479.38786979,28.40529276)(479.41787842,28.57530273)
\curveto(479.45786972,28.75529241)(479.50786967,28.88529228)(479.56787842,28.96530273)
\curveto(479.62786955,29.03529213)(479.69786948,29.08029209)(479.77787842,29.10030273)
\curveto(479.79786938,29.10029207)(479.82286936,29.10029207)(479.85287842,29.10030273)
\curveto(479.8828693,29.11029206)(479.90786927,29.11529205)(479.92787842,29.11530273)
\curveto(480.0778691,29.12529204)(480.22786895,29.12529204)(480.37787842,29.11530273)
\curveto(480.52786865,29.11529205)(480.62786855,29.07529209)(480.67787842,28.99530273)
\curveto(480.70786847,28.91529225)(480.70786847,28.81529235)(480.67787842,28.69530273)
\curveto(480.65786852,28.57529259)(480.63786854,28.45029272)(480.61787842,28.32030273)
\lineto(478.80287842,19.26030273)
\moveto(478.15787842,22.09530273)
\curveto(478.18787099,22.14529902)(478.20787097,22.21029896)(478.21787842,22.29030273)
\curveto(478.23787094,22.38029879)(478.24287094,22.45029872)(478.23287842,22.50030273)
\lineto(478.27787842,22.72530273)
\curveto(478.2778709,22.81529835)(478.2828709,22.90529826)(478.29287842,22.99530273)
\curveto(478.30287088,23.09529807)(478.29787088,23.18529798)(478.27787842,23.26530273)
\lineto(478.27787842,23.49030273)
\curveto(478.2778709,23.56029761)(478.26787091,23.63029754)(478.24787842,23.70030273)
\curveto(478.18787099,24.00029717)(478.0828711,24.2652969)(477.93287842,24.49530273)
\curveto(477.79287139,24.72529644)(477.59287159,24.90529626)(477.33287842,25.03530273)
\curveto(477.24287194,25.08529608)(477.14787203,25.12029605)(477.04787842,25.14030273)
\curveto(476.94787223,25.170296)(476.83787234,25.19529597)(476.71787842,25.21530273)
\curveto(476.64787253,25.23529593)(476.56287262,25.24529592)(476.46287842,25.24530273)
\lineto(476.19287842,25.24530273)
\lineto(476.04287842,25.21530273)
\lineto(475.90787842,25.21530273)
\curveto(475.82787335,25.19529597)(475.74287344,25.17529599)(475.65287842,25.15530273)
\curveto(475.56287362,25.13529603)(475.4778737,25.11029606)(475.39787842,25.08030273)
\curveto(475.04787413,24.94029623)(474.74787443,24.73529643)(474.49787842,24.46530273)
\curveto(474.24787493,24.20529696)(474.02787515,23.90029727)(473.83787842,23.55030273)
\curveto(473.7778754,23.44029773)(473.72787545,23.32529784)(473.68787842,23.20530273)
\lineto(473.56787842,22.87530273)
\lineto(473.53787842,22.75530273)
\curveto(473.52787565,22.72529844)(473.51787566,22.69029848)(473.50787842,22.65030273)
\curveto(473.4778757,22.60029857)(473.45787572,22.54529862)(473.44787842,22.48530273)
\curveto(473.44787573,22.42529874)(473.44287574,22.3702988)(473.43287842,22.32030273)
\curveto(473.41287577,22.21029896)(473.38787579,22.10029907)(473.35787842,21.99030273)
\curveto(473.33787584,21.89029928)(473.33287585,21.79529937)(473.34287842,21.70530273)
\curveto(473.34287584,21.67529949)(473.33787584,21.62529954)(473.32787842,21.55530273)
\lineto(473.32787842,21.34530273)
\curveto(473.32787585,21.27529989)(473.33287585,21.20529996)(473.34287842,21.13530273)
\curveto(473.3828758,20.78530038)(473.47287571,20.48530068)(473.61287842,20.23530273)
\curveto(473.75287543,19.98530118)(473.95287523,19.78030139)(474.21287842,19.62030273)
\curveto(474.29287489,19.5703016)(474.37287481,19.53030164)(474.45287842,19.50030273)
\curveto(474.54287464,19.4703017)(474.63787454,19.44030173)(474.73787842,19.41030273)
\curveto(474.78787439,19.39030178)(474.83787434,19.38530178)(474.88787842,19.39530273)
\curveto(474.94787423,19.40530176)(475.00287418,19.40030177)(475.05287842,19.38030273)
\curveto(475.0828741,19.3703018)(475.11787406,19.3653018)(475.15787842,19.36530273)
\lineto(475.29287842,19.36530273)
\lineto(475.42787842,19.36530273)
\curveto(475.46787371,19.37530179)(475.52287366,19.38030179)(475.59287842,19.38030273)
\curveto(475.67287351,19.40030177)(475.75287343,19.41530175)(475.83287842,19.42530273)
\curveto(475.92287326,19.44530172)(476.00287318,19.4703017)(476.07287842,19.50030273)
\curveto(476.43287275,19.64030153)(476.73787244,19.81530135)(476.98787842,20.02530273)
\curveto(477.23787194,20.24530092)(477.46287172,20.52030065)(477.66287842,20.85030273)
\curveto(477.73287145,20.96030021)(477.78787139,21.0703001)(477.82787842,21.18030273)
\lineto(477.97787842,21.51030273)
\curveto(478.00787117,21.55029962)(478.02287116,21.58529958)(478.02287842,21.61530273)
\curveto(478.03287115,21.65529951)(478.04787113,21.69529947)(478.06787842,21.73530273)
\curveto(478.08787109,21.79529937)(478.10287108,21.85529931)(478.11287842,21.91530273)
\curveto(478.12287106,21.97529919)(478.13787104,22.03529913)(478.15787842,22.09530273)
}
}
{
\newrgbcolor{curcolor}{0 0 0}
\pscustom[linestyle=none,fillstyle=solid,fillcolor=curcolor]
{
\newpath
\moveto(488.17412842,22.62030273)
\curveto(488.17411991,22.52029865)(488.15411993,22.40529876)(488.11412842,22.27530273)
\curveto(488.07412001,22.15529901)(488.02412006,22.0702991)(487.96412842,22.02030273)
\curveto(487.90412018,21.98029919)(487.82412026,21.95029922)(487.72412842,21.93030273)
\curveto(487.62412046,21.92029925)(487.51412057,21.91529925)(487.39412842,21.91530273)
\lineto(487.03412842,21.91530273)
\curveto(486.92412116,21.92529924)(486.82412126,21.93029924)(486.73412842,21.93030273)
\lineto(482.89412842,21.93030273)
\curveto(482.81412527,21.93029924)(482.72912536,21.92529924)(482.63912842,21.91530273)
\curveto(482.55912553,21.91529925)(482.49412559,21.90029927)(482.44412842,21.87030273)
\curveto(482.39412569,21.85029932)(482.34412574,21.81029936)(482.29412842,21.75030273)
\lineto(482.20412842,21.61530273)
\curveto(482.17412591,21.5652996)(482.16412592,21.51529965)(482.17412842,21.46530273)
\curveto(482.17412591,21.41529975)(482.16912592,21.3702998)(482.15912842,21.33030273)
\lineto(482.15912842,21.21030273)
\lineto(482.15912842,20.95530273)
\curveto(482.16912592,20.87530029)(482.1841259,20.79530037)(482.20412842,20.71530273)
\curveto(482.33412575,20.17530099)(482.63912545,19.79030138)(483.11912842,19.56030273)
\curveto(483.16912492,19.53030164)(483.22912486,19.50530166)(483.29912842,19.48530273)
\curveto(483.36912472,19.4653017)(483.43412465,19.44530172)(483.49412842,19.42530273)
\curveto(483.52412456,19.41530175)(483.57412451,19.41030176)(483.64412842,19.41030273)
\curveto(483.77412431,19.3703018)(483.95412413,19.35030182)(484.18412842,19.35030273)
\curveto(484.41412367,19.35030182)(484.60412348,19.3703018)(484.75412842,19.41030273)
\curveto(484.90412318,19.45030172)(485.03912305,19.49030168)(485.15912842,19.53030273)
\curveto(485.2891228,19.58030159)(485.40912268,19.64030153)(485.51912842,19.71030273)
\curveto(485.63912245,19.78030139)(485.74912234,19.86030131)(485.84912842,19.95030273)
\curveto(485.94912214,20.05030112)(486.03912205,20.15530101)(486.11912842,20.26530273)
\curveto(486.19912189,20.3653008)(486.27412181,20.4703007)(486.34412842,20.58030273)
\curveto(486.41412167,20.69030048)(486.50912158,20.7703004)(486.62912842,20.82030273)
\curveto(486.66912142,20.84030033)(486.73412135,20.85530031)(486.82412842,20.86530273)
\curveto(486.92412116,20.87530029)(487.01412107,20.87530029)(487.09412842,20.86530273)
\curveto(487.1841209,20.8653003)(487.26912082,20.86030031)(487.34912842,20.85030273)
\curveto(487.42912066,20.84030033)(487.47912061,20.82030035)(487.49912842,20.79030273)
\curveto(487.5891205,20.72030045)(487.59412049,20.60530056)(487.51412842,20.44530273)
\curveto(487.37412071,20.17530099)(487.21912087,19.93530123)(487.04912842,19.72530273)
\curveto(486.7891213,19.40530176)(486.50912158,19.14030203)(486.20912842,18.93030273)
\curveto(485.91912217,18.73030244)(485.56412252,18.5653026)(485.14412842,18.43530273)
\curveto(485.03412305,18.39530277)(484.92912316,18.3703028)(484.82912842,18.36030273)
\curveto(484.72912336,18.34030283)(484.61912347,18.32030285)(484.49912842,18.30030273)
\curveto(484.44912364,18.29030288)(484.39912369,18.28530288)(484.34912842,18.28530273)
\curveto(484.30912378,18.28530288)(484.26412382,18.28030289)(484.21412842,18.27030273)
\lineto(484.06412842,18.27030273)
\curveto(484.01412407,18.26030291)(483.95412413,18.25530291)(483.88412842,18.25530273)
\curveto(483.82412426,18.25530291)(483.77412431,18.26030291)(483.73412842,18.27030273)
\lineto(483.59912842,18.27030273)
\curveto(483.54912454,18.28030289)(483.50412458,18.28530288)(483.46412842,18.28530273)
\curveto(483.42412466,18.28530288)(483.3841247,18.29030288)(483.34412842,18.30030273)
\curveto(483.29412479,18.31030286)(483.23912485,18.32030285)(483.17912842,18.33030273)
\curveto(483.12912496,18.33030284)(483.07912501,18.33530283)(483.02912842,18.34530273)
\curveto(482.93912515,18.3653028)(482.84912524,18.39030278)(482.75912842,18.42030273)
\curveto(482.67912541,18.44030273)(482.60412548,18.4653027)(482.53412842,18.49530273)
\curveto(482.49412559,18.51530265)(482.45912563,18.52530264)(482.42912842,18.52530273)
\curveto(482.39912569,18.53530263)(482.36912572,18.55030262)(482.33912842,18.57030273)
\curveto(482.19912589,18.64030253)(482.05412603,18.72530244)(481.90412842,18.82530273)
\curveto(481.65412643,19.01530215)(481.45412663,19.24530192)(481.30412842,19.51530273)
\curveto(481.15412693,19.79530137)(481.04412704,20.10530106)(480.97412842,20.44530273)
\curveto(480.94412714,20.55530061)(480.92912716,20.6703005)(480.92912842,20.79030273)
\curveto(480.92912716,20.91030026)(480.91912717,21.03030014)(480.89912842,21.15030273)
\lineto(480.89912842,21.25530273)
\curveto(480.90912718,21.28529988)(480.91412717,21.32529984)(480.91412842,21.37530273)
\lineto(480.91412842,21.63030273)
\curveto(480.92412716,21.72029945)(480.92912716,21.81029936)(480.92912842,21.90030273)
\lineto(480.97412842,22.11030273)
\curveto(480.97412711,22.15029902)(480.97912711,22.20529896)(480.98912842,22.27530273)
\curveto(480.99912709,22.35529881)(481.01412707,22.42029875)(481.03412842,22.47030273)
\lineto(481.06412842,22.63530273)
\curveto(481.09412699,22.68529848)(481.10912698,22.73529843)(481.10912842,22.78530273)
\curveto(481.11912697,22.84529832)(481.13412695,22.90029827)(481.15412842,22.95030273)
\curveto(481.22412686,23.11029806)(481.2891268,23.2702979)(481.34912842,23.43030273)
\curveto(481.40912668,23.59029758)(481.4841266,23.74029743)(481.57412842,23.88030273)
\curveto(481.64412644,23.99029718)(481.70912638,24.10029707)(481.76912842,24.21030273)
\curveto(481.83912625,24.33029684)(481.91912617,24.44529672)(482.00912842,24.55530273)
\curveto(482.29912579,24.90529626)(482.60912548,25.20529596)(482.93912842,25.45530273)
\curveto(483.26912482,25.71529545)(483.65412443,25.93029524)(484.09412842,26.10030273)
\curveto(484.22412386,26.15029502)(484.35412373,26.18529498)(484.48412842,26.20530273)
\curveto(484.61412347,26.23529493)(484.75412333,26.2652949)(484.90412842,26.29530273)
\curveto(484.95412313,26.30529486)(484.99912309,26.31029486)(485.03912842,26.31030273)
\curveto(485.07912301,26.32029485)(485.12412296,26.32529484)(485.17412842,26.32530273)
\curveto(485.19412289,26.33529483)(485.21912287,26.33529483)(485.24912842,26.32530273)
\curveto(485.27912281,26.31529485)(485.30412278,26.32029485)(485.32412842,26.34030273)
\curveto(485.75412233,26.35029482)(486.11412197,26.30529486)(486.40412842,26.20530273)
\curveto(486.69412139,26.11529505)(486.94912114,25.99029518)(487.16912842,25.83030273)
\curveto(487.20912088,25.81029536)(487.23912085,25.78029539)(487.25912842,25.74030273)
\curveto(487.2891208,25.71029546)(487.31912077,25.68529548)(487.34912842,25.66530273)
\curveto(487.41912067,25.60529556)(487.4891206,25.53529563)(487.55912842,25.45530273)
\curveto(487.62912046,25.37529579)(487.6841204,25.29529587)(487.72412842,25.21530273)
\curveto(487.84412024,25.00529616)(487.93912015,24.80529636)(488.00912842,24.61530273)
\curveto(488.05912003,24.50529666)(488.08912,24.38529678)(488.09912842,24.25530273)
\lineto(488.15912842,23.86530273)
\curveto(488.1891199,23.73529743)(488.19911989,23.60029757)(488.18912842,23.46030273)
\curveto(488.1891199,23.32029785)(488.19411989,23.18029799)(488.20412842,23.04030273)
\curveto(488.20411988,22.9702982)(488.19911989,22.90029827)(488.18912842,22.83030273)
\curveto(488.17911991,22.76029841)(488.17411991,22.69029848)(488.17412842,22.62030273)
\moveto(486.82412842,23.13030273)
\curveto(486.85412123,23.170298)(486.8841212,23.22029795)(486.91412842,23.28030273)
\curveto(486.95412113,23.35029782)(486.96912112,23.42029775)(486.95912842,23.49030273)
\curveto(486.94912114,23.71029746)(486.90912118,23.91529725)(486.83912842,24.10530273)
\curveto(486.73912135,24.33529683)(486.61912147,24.53029664)(486.47912842,24.69030273)
\curveto(486.34912174,24.85029632)(486.15912193,24.98529618)(485.90912842,25.09530273)
\curveto(485.83912225,25.11529605)(485.76912232,25.13029604)(485.69912842,25.14030273)
\curveto(485.63912245,25.16029601)(485.56912252,25.18029599)(485.48912842,25.20030273)
\curveto(485.41912267,25.22029595)(485.33912275,25.23029594)(485.24912842,25.23030273)
\lineto(484.99412842,25.23030273)
\curveto(484.95412313,25.21029596)(484.91412317,25.20029597)(484.87412842,25.20030273)
\curveto(484.83412325,25.21029596)(484.79912329,25.21029596)(484.76912842,25.20030273)
\lineto(484.52912842,25.14030273)
\curveto(484.44912364,25.13029604)(484.37412371,25.11529605)(484.30412842,25.09530273)
\curveto(483.9841241,24.97529619)(483.71912437,24.82529634)(483.50912842,24.64530273)
\curveto(483.29912479,24.4652967)(483.09912499,24.24029693)(482.90912842,23.97030273)
\curveto(482.86912522,23.92029725)(482.82412526,23.85529731)(482.77412842,23.77530273)
\curveto(482.73412535,23.70529746)(482.69412539,23.62529754)(482.65412842,23.53530273)
\curveto(482.61412547,23.44529772)(482.5891255,23.36029781)(482.57912842,23.28030273)
\curveto(482.57912551,23.20029797)(482.60412548,23.14029803)(482.65412842,23.10030273)
\curveto(482.72412536,23.04029813)(482.85412523,23.01029816)(483.04412842,23.01030273)
\curveto(483.24412484,23.02029815)(483.41412467,23.02529814)(483.55412842,23.02530273)
\lineto(485.83412842,23.02530273)
\curveto(485.9841221,23.02529814)(486.16412192,23.02029815)(486.37412842,23.01030273)
\curveto(486.5841215,23.01029816)(486.73412135,23.05029812)(486.82412842,23.13030273)
}
}
{
\newrgbcolor{curcolor}{0 0 0}
\pscustom[linestyle=none,fillstyle=solid,fillcolor=curcolor]
{
\newpath
\moveto(493.24576904,29.25030273)
\curveto(493.42576334,29.26029191)(493.61576315,29.26029191)(493.81576904,29.25030273)
\curveto(494.01576275,29.24029193)(494.14576262,29.18029199)(494.20576904,29.07030273)
\curveto(494.23576253,29.01029216)(494.24576252,28.93529223)(494.23576904,28.84530273)
\curveto(494.22576254,28.7652924)(494.21076256,28.67529249)(494.19076904,28.57530273)
\curveto(494.1707626,28.44529272)(494.12576264,28.34029283)(494.05576904,28.26030273)
\curveto(494.00576276,28.21029296)(493.94076283,28.17529299)(493.86076904,28.15530273)
\curveto(493.78076299,28.14529302)(493.69576307,28.14029303)(493.60576904,28.14030273)
\lineto(493.33576904,28.14030273)
\curveto(493.24576352,28.15029302)(493.16076361,28.15029302)(493.08076904,28.14030273)
\curveto(492.79076398,28.06029311)(492.58576418,27.93029324)(492.46576904,27.75030273)
\curveto(492.34576442,27.58029359)(492.25076452,27.32029385)(492.18076904,26.97030273)
\curveto(492.16076461,26.90029427)(492.13576463,26.82529434)(492.10576904,26.74530273)
\curveto(492.08576468,26.67529449)(492.08076469,26.61029456)(492.09076904,26.55030273)
\curveto(492.09076468,26.40029477)(492.13576463,26.29529487)(492.22576904,26.23530273)
\curveto(492.29576447,26.20529496)(492.39076438,26.19029498)(492.51076904,26.19030273)
\lineto(492.87076904,26.19030273)
\lineto(493.09576904,26.19030273)
\curveto(493.12576364,26.170295)(493.15576361,26.165295)(493.18576904,26.17530273)
\curveto(493.21576355,26.18529498)(493.24576352,26.18029499)(493.27576904,26.16030273)
\curveto(493.3657634,26.13029504)(493.41576335,26.0702951)(493.42576904,25.98030273)
\curveto(493.44576332,25.90029527)(493.44076333,25.79529537)(493.41076904,25.66530273)
\lineto(493.38076904,25.54530273)
\lineto(493.35076904,25.42530273)
\curveto(493.29076348,25.27529589)(493.20576356,25.17529599)(493.09576904,25.12530273)
\curveto(492.95576381,25.07529609)(492.78576398,25.06029611)(492.58576904,25.08030273)
\curveto(492.38576438,25.11029606)(492.21076456,25.10529606)(492.06076904,25.06530273)
\curveto(491.98076479,25.04529612)(491.91576485,25.00529616)(491.86576904,24.94530273)
\curveto(491.81576495,24.89529627)(491.770765,24.82529634)(491.73076904,24.73530273)
\curveto(491.70076507,24.6652965)(491.68076509,24.58529658)(491.67076904,24.49530273)
\curveto(491.66076511,24.40529676)(491.64576512,24.32029685)(491.62576904,24.24030273)
\lineto(491.43076904,23.25030273)
\lineto(490.80076904,20.07030273)
\lineto(490.65076904,19.32030273)
\curveto(490.64076613,19.26030191)(490.63076614,19.19530197)(490.62076904,19.12530273)
\curveto(490.61076616,19.05530211)(490.59076618,18.99530217)(490.56076904,18.94530273)
\lineto(490.53076904,18.82530273)
\lineto(490.47076904,18.70530273)
\curveto(490.46076631,18.6653025)(490.44076633,18.63030254)(490.41076904,18.60030273)
\curveto(490.35076642,18.53030264)(490.2657665,18.49030268)(490.15576904,18.48030273)
\curveto(490.05576671,18.4703027)(489.94576682,18.4653027)(489.82576904,18.46530273)
\lineto(489.54076904,18.46530273)
\curveto(489.50076727,18.48530268)(489.45576731,18.50030267)(489.40576904,18.51030273)
\curveto(489.3657674,18.53030264)(489.33576743,18.5653026)(489.31576904,18.61530273)
\curveto(489.30576746,18.64530252)(489.30076747,18.71030246)(489.30076904,18.81030273)
\lineto(489.31576904,18.91530273)
\curveto(489.30576746,18.9653022)(489.31076746,19.01530215)(489.33076904,19.06530273)
\curveto(489.35076742,19.12530204)(489.3657674,19.18030199)(489.37576904,19.23030273)
\lineto(489.49576904,19.83030273)
\lineto(490.30576904,23.92530273)
\curveto(490.32576644,24.03529713)(490.35076642,24.15029702)(490.38076904,24.27030273)
\curveto(490.41076636,24.39029678)(490.43076634,24.50029667)(490.44076904,24.60030273)
\curveto(490.46076631,24.71029646)(490.46076631,24.80529636)(490.44076904,24.88530273)
\curveto(490.43076634,24.9652962)(490.38576638,25.02029615)(490.30576904,25.05030273)
\curveto(490.25576651,25.08029609)(490.19076658,25.09529607)(490.11076904,25.09530273)
\lineto(489.88576904,25.09530273)
\lineto(489.64576904,25.09530273)
\curveto(489.57576719,25.09529607)(489.51076726,25.10529606)(489.45076904,25.12530273)
\curveto(489.3707674,25.165296)(489.32576744,25.25029592)(489.31576904,25.38030273)
\lineto(489.31576904,25.51530273)
\curveto(489.32576744,25.55529561)(489.33576743,25.60029557)(489.34576904,25.65030273)
\curveto(489.37576739,25.79029538)(489.41076736,25.90029527)(489.45076904,25.98030273)
\curveto(489.50076727,26.0702951)(489.58076719,26.13029504)(489.69076904,26.16030273)
\curveto(489.770767,26.19029498)(489.85576691,26.20029497)(489.94576904,26.19030273)
\lineto(490.21576904,26.19030273)
\curveto(490.31576645,26.19029498)(490.40576636,26.20029497)(490.48576904,26.22030273)
\curveto(490.5657662,26.24029493)(490.63576613,26.28029489)(490.69576904,26.34030273)
\curveto(490.78576598,26.42029475)(490.84576592,26.54529462)(490.87576904,26.71530273)
\curveto(490.90576586,26.88529428)(490.93576583,27.04529412)(490.96576904,27.19530273)
\curveto(491.00576576,27.39529377)(491.05576571,27.58029359)(491.11576904,27.75030273)
\curveto(491.17576559,27.93029324)(491.25076552,28.09029308)(491.34076904,28.23030273)
\curveto(491.49076528,28.4702927)(491.6707651,28.6652925)(491.88076904,28.81530273)
\curveto(492.10076467,28.9652922)(492.35076442,29.08029209)(492.63076904,29.16030273)
\curveto(492.69076408,29.18029199)(492.75576401,29.19029198)(492.82576904,29.19030273)
\curveto(492.89576387,29.20029197)(492.9657638,29.21529195)(493.03576904,29.23530273)
\curveto(493.05576371,29.24529192)(493.09076368,29.24529192)(493.14076904,29.23530273)
\curveto(493.19076358,29.23529193)(493.22576354,29.24029193)(493.24576904,29.25030273)
\moveto(495.19576904,27.67530273)
\curveto(495.25576151,27.62529354)(495.33576143,27.60029357)(495.43576904,27.60030273)
\lineto(495.75076904,27.60030273)
\lineto(495.91576904,27.60030273)
\curveto(495.97576079,27.60029357)(496.03576073,27.61029356)(496.09576904,27.63030273)
\curveto(496.23576053,27.68029349)(496.32076045,27.78529338)(496.35076904,27.94530273)
\curveto(496.39076038,28.10529306)(496.43076034,28.27529289)(496.47076904,28.45530273)
\curveto(496.48076029,28.54529262)(496.49576027,28.63029254)(496.51576904,28.71030273)
\curveto(496.53576023,28.80029237)(496.53576023,28.87529229)(496.51576904,28.93530273)
\curveto(496.48576028,29.04529212)(496.39576037,29.10529206)(496.24576904,29.11530273)
\curveto(496.10576066,29.12529204)(495.95076082,29.13029204)(495.78076904,29.13030273)
\curveto(495.75076102,29.12029205)(495.72576104,29.11529205)(495.70576904,29.11530273)
\curveto(495.68576108,29.12529204)(495.66076111,29.12529204)(495.63076904,29.11530273)
\curveto(495.51076126,29.07529209)(495.42076135,29.01529215)(495.36076904,28.93530273)
\curveto(495.32076145,28.87529229)(495.29076148,28.80029237)(495.27076904,28.71030273)
\curveto(495.25076152,28.62029255)(495.23576153,28.53529263)(495.22576904,28.45530273)
\curveto(495.19576157,28.30529286)(495.1657616,28.15029302)(495.13576904,27.99030273)
\curveto(495.10576166,27.84029333)(495.12576164,27.73529343)(495.19576904,27.67530273)
\moveto(495.87076904,25.51530273)
\curveto(495.89076088,25.61529555)(495.91076086,25.71029546)(495.93076904,25.80030273)
\curveto(495.95076082,25.90029527)(495.94076083,25.98029519)(495.90076904,26.04030273)
\curveto(495.8707609,26.12029505)(495.78576098,26.16029501)(495.64576904,26.16030273)
\curveto(495.51576125,26.170295)(495.38576138,26.17529499)(495.25576904,26.17530273)
\curveto(495.23576153,26.165295)(495.21076156,26.16029501)(495.18076904,26.16030273)
\curveto(495.16076161,26.170295)(495.14076163,26.17529499)(495.12076904,26.17530273)
\curveto(495.06076171,26.15529501)(495.00076177,26.14029503)(494.94076904,26.13030273)
\curveto(494.89076188,26.12029505)(494.84576192,26.09029508)(494.80576904,26.04030273)
\curveto(494.74576202,25.98029519)(494.70576206,25.89529527)(494.68576904,25.78530273)
\curveto(494.6657621,25.68529548)(494.64576212,25.58029559)(494.62576904,25.47030273)
\lineto(493.35076904,19.12530273)
\curveto(493.33076344,19.03530213)(493.31076346,18.94030223)(493.29076904,18.84030273)
\curveto(493.28076349,18.75030242)(493.28576348,18.67530249)(493.30576904,18.61530273)
\curveto(493.34576342,18.53530263)(493.41076336,18.48530268)(493.50076904,18.46530273)
\curveto(493.59076318,18.45530271)(493.70076307,18.45030272)(493.83076904,18.45030273)
\lineto(494.05576904,18.45030273)
\curveto(494.14576262,18.4703027)(494.22076255,18.48530268)(494.28076904,18.49530273)
\curveto(494.34076243,18.51530265)(494.39076238,18.55530261)(494.43076904,18.61530273)
\curveto(494.50076227,18.67530249)(494.54076223,18.75530241)(494.55076904,18.85530273)
\curveto(494.5707622,18.9653022)(494.59076218,19.0703021)(494.61076904,19.17030273)
\lineto(495.87076904,25.51530273)
}
}
{
\newrgbcolor{curcolor}{0 0 0}
\pscustom[linestyle=none,fillstyle=solid,fillcolor=curcolor]
{
\newpath
\moveto(501.66944092,26.32530273)
\curveto(502.3094341,26.34529482)(502.79943361,26.26029491)(503.13944092,26.07030273)
\curveto(503.47943293,25.88029529)(503.72443268,25.59529557)(503.87444092,25.21530273)
\curveto(503.91443249,25.11529605)(503.93943247,25.00529616)(503.94944092,24.88530273)
\curveto(503.96943244,24.77529639)(503.97943243,24.66029651)(503.97944092,24.54030273)
\curveto(503.99943241,24.35029682)(503.98943242,24.14529702)(503.94944092,23.92530273)
\curveto(503.91943249,23.70529746)(503.87943253,23.48029769)(503.82944092,23.25030273)
\lineto(503.51444092,21.64530273)
\lineto(503.04944092,19.30530273)
\lineto(502.92944092,18.79530273)
\curveto(502.88943352,18.62530254)(502.79943361,18.51530265)(502.65944092,18.46530273)
\curveto(502.6094338,18.44530272)(502.55443385,18.43530273)(502.49444092,18.43530273)
\curveto(502.44443396,18.42530274)(502.38943402,18.42030275)(502.32944092,18.42030273)
\curveto(502.19943421,18.42030275)(502.07443433,18.42530274)(501.95444092,18.43530273)
\curveto(501.83443457,18.43530273)(501.75943465,18.47530269)(501.72944092,18.55530273)
\curveto(501.68943472,18.62530254)(501.67943473,18.71530245)(501.69944092,18.82530273)
\curveto(501.71943469,18.93530223)(501.74443466,19.04530212)(501.77444092,19.15530273)
\lineto(502.02944092,20.44530273)
\lineto(502.50944092,22.89030273)
\curveto(502.56943384,23.16029801)(502.61943379,23.42529774)(502.65944092,23.68530273)
\curveto(502.69943371,23.95529721)(502.69943371,24.18529698)(502.65944092,24.37530273)
\curveto(502.61943379,24.57529659)(502.52943388,24.73529643)(502.38944092,24.85530273)
\curveto(502.25943415,24.98529618)(502.09943431,25.08529608)(501.90944092,25.15530273)
\curveto(501.84943456,25.17529599)(501.78443462,25.18529598)(501.71444092,25.18530273)
\curveto(501.65443475,25.19529597)(501.59943481,25.21029596)(501.54944092,25.23030273)
\curveto(501.49943491,25.24029593)(501.41943499,25.24029593)(501.30944092,25.23030273)
\curveto(501.2094352,25.23029594)(501.13443527,25.22529594)(501.08444092,25.21530273)
\curveto(501.04443536,25.19529597)(501.0094354,25.18529598)(500.97944092,25.18530273)
\curveto(500.94943546,25.19529597)(500.91443549,25.19529597)(500.87444092,25.18530273)
\curveto(500.73443567,25.15529601)(500.6044358,25.12029605)(500.48444092,25.08030273)
\curveto(500.36443604,25.05029612)(500.24943616,25.00529616)(500.13944092,24.94530273)
\curveto(500.08943632,24.92529624)(500.04943636,24.90529626)(500.01944092,24.88530273)
\curveto(499.98943642,24.8652963)(499.94943646,24.84529632)(499.89944092,24.82530273)
\curveto(499.49943691,24.57529659)(499.16943724,24.20029697)(498.90944092,23.70030273)
\curveto(498.86943754,23.62029755)(498.83443757,23.53529763)(498.80444092,23.44530273)
\lineto(498.71444092,23.20530273)
\curveto(498.68443772,23.15529801)(498.66943774,23.10529806)(498.66944092,23.05530273)
\curveto(498.66943774,23.01529815)(498.65443775,22.97529819)(498.62444092,22.93530273)
\lineto(498.56444092,22.62030273)
\curveto(498.54443786,22.59029858)(498.53443787,22.55529861)(498.53444092,22.51530273)
\curveto(498.53443787,22.47529869)(498.52943788,22.43029874)(498.51944092,22.38030273)
\lineto(498.42944092,21.93030273)
\lineto(498.12944092,20.49030273)
\lineto(497.87444092,19.17030273)
\curveto(497.85443855,19.06030211)(497.82943858,18.94530222)(497.79944092,18.82530273)
\curveto(497.77943863,18.71530245)(497.73943867,18.62530254)(497.67944092,18.55530273)
\curveto(497.6094388,18.47530269)(497.5094389,18.43530273)(497.37944092,18.43530273)
\curveto(497.25943915,18.42530274)(497.13443927,18.42030275)(497.00444092,18.42030273)
\curveto(496.92443948,18.42030275)(496.84943956,18.42530274)(496.77944092,18.43530273)
\curveto(496.7094397,18.44530272)(496.65443975,18.4703027)(496.61444092,18.51030273)
\curveto(496.54443986,18.56030261)(496.52443988,18.65530251)(496.55444092,18.79530273)
\curveto(496.58443982,18.93530223)(496.6094398,19.0703021)(496.62944092,19.20030273)
\lineto(496.98944092,20.97030273)
\lineto(497.70944092,24.60030273)
\lineto(497.88944092,25.51530273)
\lineto(497.94944092,25.78530273)
\curveto(497.96943844,25.87529529)(498.0044384,25.94529522)(498.05444092,25.99530273)
\curveto(498.09443831,26.05529511)(498.14943826,26.09529507)(498.21944092,26.11530273)
\curveto(498.26943814,26.12529504)(498.32943808,26.13529503)(498.39944092,26.14530273)
\curveto(498.47943793,26.15529501)(498.55943785,26.16029501)(498.63944092,26.16030273)
\curveto(498.71943769,26.16029501)(498.79443761,26.15529501)(498.86444092,26.14530273)
\curveto(498.94443746,26.13529503)(498.99443741,26.12029505)(499.01444092,26.10030273)
\curveto(499.11443729,26.03029514)(499.14943726,25.94029523)(499.11944092,25.83030273)
\curveto(499.08943732,25.73029544)(499.07943733,25.61529555)(499.08944092,25.48530273)
\curveto(499.09943731,25.42529574)(499.12943728,25.37529579)(499.17944092,25.33530273)
\curveto(499.29943711,25.32529584)(499.404437,25.3702958)(499.49444092,25.47030273)
\curveto(499.59443681,25.5702956)(499.68943672,25.65029552)(499.77944092,25.71030273)
\curveto(499.93943647,25.81029536)(500.09943631,25.90029527)(500.25944092,25.98030273)
\curveto(500.41943599,26.0702951)(500.6044358,26.14529502)(500.81444092,26.20530273)
\curveto(500.89443551,26.23529493)(500.98443542,26.25529491)(501.08444092,26.26530273)
\curveto(501.18443522,26.27529489)(501.27943513,26.29029488)(501.36944092,26.31030273)
\curveto(501.41943499,26.32029485)(501.46943494,26.32529484)(501.51944092,26.32530273)
\lineto(501.66944092,26.32530273)
}
}
{
\newrgbcolor{curcolor}{0 0 0}
\pscustom[linestyle=none,fillstyle=solid,fillcolor=curcolor]
{
\newpath
\moveto(506.88405029,27.67530273)
\curveto(506.81404732,27.73529343)(506.79404734,27.84029333)(506.82405029,27.99030273)
\curveto(506.85404728,28.15029302)(506.88404725,28.30529286)(506.91405029,28.45530273)
\curveto(506.92404721,28.53529263)(506.93904719,28.62029255)(506.95905029,28.71030273)
\curveto(506.97904715,28.80029237)(507.00904712,28.87529229)(507.04905029,28.93530273)
\curveto(507.10904702,29.01529215)(507.19904693,29.07529209)(507.31905029,29.11530273)
\curveto(507.34904678,29.12529204)(507.37404676,29.12529204)(507.39405029,29.11530273)
\curveto(507.41404672,29.11529205)(507.43904669,29.12029205)(507.46905029,29.13030273)
\curveto(507.63904649,29.13029204)(507.79404634,29.12529204)(507.93405029,29.11530273)
\curveto(508.08404605,29.10529206)(508.17404596,29.04529212)(508.20405029,28.93530273)
\curveto(508.22404591,28.87529229)(508.22404591,28.80029237)(508.20405029,28.71030273)
\curveto(508.18404595,28.63029254)(508.16904596,28.54529262)(508.15905029,28.45530273)
\curveto(508.11904601,28.27529289)(508.07904605,28.10529306)(508.03905029,27.94530273)
\curveto(508.00904612,27.78529338)(507.92404621,27.68029349)(507.78405029,27.63030273)
\curveto(507.72404641,27.61029356)(507.66404647,27.60029357)(507.60405029,27.60030273)
\lineto(507.43905029,27.60030273)
\lineto(507.12405029,27.60030273)
\curveto(507.02404711,27.60029357)(506.94404719,27.62529354)(506.88405029,27.67530273)
\moveto(506.29905029,19.17030273)
\curveto(506.27904785,19.0703021)(506.25904787,18.9653022)(506.23905029,18.85530273)
\curveto(506.2290479,18.75530241)(506.18904794,18.67530249)(506.11905029,18.61530273)
\curveto(506.07904805,18.55530261)(506.0290481,18.51530265)(505.96905029,18.49530273)
\curveto(505.90904822,18.48530268)(505.8340483,18.4703027)(505.74405029,18.45030273)
\lineto(505.51905029,18.45030273)
\curveto(505.38904874,18.45030272)(505.27904885,18.45530271)(505.18905029,18.46530273)
\curveto(505.09904903,18.48530268)(505.0340491,18.53530263)(504.99405029,18.61530273)
\curveto(504.97404916,18.67530249)(504.96904916,18.75030242)(504.97905029,18.84030273)
\curveto(504.99904913,18.94030223)(505.01904911,19.03530213)(505.03905029,19.12530273)
\lineto(506.31405029,25.47030273)
\curveto(506.3340478,25.58029559)(506.35404778,25.68529548)(506.37405029,25.78530273)
\curveto(506.39404774,25.89529527)(506.4340477,25.98029519)(506.49405029,26.04030273)
\curveto(506.5340476,26.09029508)(506.57904755,26.12029505)(506.62905029,26.13030273)
\curveto(506.68904744,26.14029503)(506.74904738,26.15529501)(506.80905029,26.17530273)
\curveto(506.8290473,26.17529499)(506.84904728,26.170295)(506.86905029,26.16030273)
\curveto(506.89904723,26.16029501)(506.92404721,26.165295)(506.94405029,26.17530273)
\curveto(507.07404706,26.17529499)(507.20404693,26.170295)(507.33405029,26.16030273)
\curveto(507.47404666,26.16029501)(507.55904657,26.12029505)(507.58905029,26.04030273)
\curveto(507.6290465,25.98029519)(507.63904649,25.90029527)(507.61905029,25.80030273)
\curveto(507.59904653,25.71029546)(507.57904655,25.61529555)(507.55905029,25.51530273)
\lineto(506.29905029,19.17030273)
}
}
{
\newrgbcolor{curcolor}{0 0 0}
\pscustom[linestyle=none,fillstyle=solid,fillcolor=curcolor]
{
\newpath
\moveto(515.21889404,19.26030273)
\lineto(515.12889404,18.87030273)
\curveto(515.10888611,18.75030242)(515.06888615,18.65030252)(515.00889404,18.57030273)
\curveto(514.93888628,18.50030267)(514.84388638,18.46030271)(514.72389404,18.45030273)
\lineto(514.37889404,18.45030273)
\curveto(514.3188869,18.45030272)(514.25888696,18.44530272)(514.19889404,18.43530273)
\curveto(514.14888707,18.43530273)(514.10388712,18.44530272)(514.06389404,18.46530273)
\curveto(513.98388724,18.48530268)(513.93388729,18.52530264)(513.91389404,18.58530273)
\curveto(513.88388734,18.63530253)(513.87388735,18.69530247)(513.88389404,18.76530273)
\curveto(513.89388733,18.83530233)(513.88888733,18.90530226)(513.86889404,18.97530273)
\curveto(513.86888735,18.99530217)(513.85888736,19.01030216)(513.83889404,19.02030273)
\lineto(513.80889404,19.08030273)
\curveto(513.70888751,19.09030208)(513.6238876,19.0703021)(513.55389404,19.02030273)
\curveto(513.49388773,18.9703022)(513.42888779,18.92030225)(513.35889404,18.87030273)
\curveto(513.12888809,18.72030245)(512.90388832,18.60530256)(512.68389404,18.52530273)
\curveto(512.49388873,18.44530272)(512.27388895,18.38530278)(512.02389404,18.34530273)
\curveto(511.78388944,18.30530286)(511.53888968,18.28530288)(511.28889404,18.28530273)
\curveto(511.04889017,18.27530289)(510.80889041,18.29030288)(510.56889404,18.33030273)
\curveto(510.33889088,18.36030281)(510.14389108,18.41530275)(509.98389404,18.49530273)
\curveto(509.50389172,18.71530245)(509.13889208,19.01030216)(508.88889404,19.38030273)
\curveto(508.64889257,19.76030141)(508.49389273,20.23030094)(508.42389404,20.79030273)
\curveto(508.40389282,20.88030029)(508.39389283,20.9703002)(508.39389404,21.06030273)
\curveto(508.40389282,21.16030001)(508.40389282,21.26029991)(508.39389404,21.36030273)
\curveto(508.39389283,21.41029976)(508.39889282,21.46029971)(508.40889404,21.51030273)
\curveto(508.4188928,21.56029961)(508.4238928,21.61029956)(508.42389404,21.66030273)
\curveto(508.41389281,21.71029946)(508.41389281,21.76029941)(508.42389404,21.81030273)
\curveto(508.44389278,21.8702993)(508.45389277,21.92529924)(508.45389404,21.97530273)
\lineto(508.48389404,22.12530273)
\curveto(508.47389275,22.17529899)(508.47389275,22.24029893)(508.48389404,22.32030273)
\curveto(508.50389272,22.40029877)(508.52889269,22.4652987)(508.55889404,22.51530273)
\lineto(508.60389404,22.68030273)
\curveto(508.63389259,22.75029842)(508.65389257,22.82029835)(508.66389404,22.89030273)
\curveto(508.67389255,22.9702982)(508.69389253,23.04529812)(508.72389404,23.11530273)
\curveto(508.74389248,23.165298)(508.75889246,23.21029796)(508.76889404,23.25030273)
\curveto(508.77889244,23.29029788)(508.79389243,23.33529783)(508.81389404,23.38530273)
\curveto(508.86389236,23.48529768)(508.90889231,23.58029759)(508.94889404,23.67030273)
\curveto(508.98889223,23.7702974)(509.03389219,23.8652973)(509.08389404,23.95530273)
\curveto(509.28389194,24.33529683)(509.51389171,24.67529649)(509.77389404,24.97530273)
\curveto(510.04389118,25.28529588)(510.34389088,25.54029563)(510.67389404,25.74030273)
\curveto(510.87389035,25.86029531)(511.07389015,25.96029521)(511.27389404,26.04030273)
\curveto(511.47388975,26.12029505)(511.68888953,26.19029498)(511.91889404,26.25030273)
\lineto(512.12889404,26.28030273)
\curveto(512.19888902,26.29029488)(512.26888895,26.30529486)(512.33889404,26.32530273)
\lineto(512.48889404,26.32530273)
\curveto(512.57888864,26.34529482)(512.69888852,26.35529481)(512.84889404,26.35530273)
\curveto(513.00888821,26.35529481)(513.1238881,26.34529482)(513.19389404,26.32530273)
\curveto(513.23388799,26.31529485)(513.28888793,26.31029486)(513.35889404,26.31030273)
\curveto(513.45888776,26.28029489)(513.56388766,26.25529491)(513.67389404,26.23530273)
\curveto(513.78388744,26.22529494)(513.88388734,26.19529497)(513.97389404,26.14530273)
\curveto(514.11388711,26.08529508)(514.24388698,26.02029515)(514.36389404,25.95030273)
\curveto(514.48388674,25.88029529)(514.59388663,25.80029537)(514.69389404,25.71030273)
\curveto(514.74388648,25.66029551)(514.79388643,25.60529556)(514.84389404,25.54530273)
\curveto(514.90388632,25.49529567)(514.98888623,25.48029569)(515.09889404,25.50030273)
\lineto(515.17389404,25.57530273)
\curveto(515.19388603,25.59529557)(515.20888601,25.62529554)(515.21889404,25.66530273)
\curveto(515.26888595,25.75529541)(515.30388592,25.8702953)(515.32389404,26.01030273)
\curveto(515.35388587,26.15029502)(515.37888584,26.27529489)(515.39889404,26.38530273)
\lineto(515.74389404,28.11030273)
\curveto(515.77388545,28.25029292)(515.80388542,28.40529276)(515.83389404,28.57530273)
\curveto(515.87388535,28.75529241)(515.9238853,28.88529228)(515.98389404,28.96530273)
\curveto(516.04388518,29.03529213)(516.11388511,29.08029209)(516.19389404,29.10030273)
\curveto(516.21388501,29.10029207)(516.23888498,29.10029207)(516.26889404,29.10030273)
\curveto(516.29888492,29.11029206)(516.3238849,29.11529205)(516.34389404,29.11530273)
\curveto(516.49388473,29.12529204)(516.64388458,29.12529204)(516.79389404,29.11530273)
\curveto(516.94388428,29.11529205)(517.04388418,29.07529209)(517.09389404,28.99530273)
\curveto(517.1238841,28.91529225)(517.1238841,28.81529235)(517.09389404,28.69530273)
\curveto(517.07388415,28.57529259)(517.05388417,28.45029272)(517.03389404,28.32030273)
\lineto(515.21889404,19.26030273)
\moveto(514.57389404,22.09530273)
\curveto(514.60388662,22.14529902)(514.6238866,22.21029896)(514.63389404,22.29030273)
\curveto(514.65388657,22.38029879)(514.65888656,22.45029872)(514.64889404,22.50030273)
\lineto(514.69389404,22.72530273)
\curveto(514.69388653,22.81529835)(514.69888652,22.90529826)(514.70889404,22.99530273)
\curveto(514.7188865,23.09529807)(514.71388651,23.18529798)(514.69389404,23.26530273)
\lineto(514.69389404,23.49030273)
\curveto(514.69388653,23.56029761)(514.68388654,23.63029754)(514.66389404,23.70030273)
\curveto(514.60388662,24.00029717)(514.49888672,24.2652969)(514.34889404,24.49530273)
\curveto(514.20888701,24.72529644)(514.00888721,24.90529626)(513.74889404,25.03530273)
\curveto(513.65888756,25.08529608)(513.56388766,25.12029605)(513.46389404,25.14030273)
\curveto(513.36388786,25.170296)(513.25388797,25.19529597)(513.13389404,25.21530273)
\curveto(513.06388816,25.23529593)(512.97888824,25.24529592)(512.87889404,25.24530273)
\lineto(512.60889404,25.24530273)
\lineto(512.45889404,25.21530273)
\lineto(512.32389404,25.21530273)
\curveto(512.24388898,25.19529597)(512.15888906,25.17529599)(512.06889404,25.15530273)
\curveto(511.97888924,25.13529603)(511.89388933,25.11029606)(511.81389404,25.08030273)
\curveto(511.46388976,24.94029623)(511.16389006,24.73529643)(510.91389404,24.46530273)
\curveto(510.66389056,24.20529696)(510.44389078,23.90029727)(510.25389404,23.55030273)
\curveto(510.19389103,23.44029773)(510.14389108,23.32529784)(510.10389404,23.20530273)
\lineto(509.98389404,22.87530273)
\lineto(509.95389404,22.75530273)
\curveto(509.94389128,22.72529844)(509.93389129,22.69029848)(509.92389404,22.65030273)
\curveto(509.89389133,22.60029857)(509.87389135,22.54529862)(509.86389404,22.48530273)
\curveto(509.86389136,22.42529874)(509.85889136,22.3702988)(509.84889404,22.32030273)
\curveto(509.82889139,22.21029896)(509.80389142,22.10029907)(509.77389404,21.99030273)
\curveto(509.75389147,21.89029928)(509.74889147,21.79529937)(509.75889404,21.70530273)
\curveto(509.75889146,21.67529949)(509.75389147,21.62529954)(509.74389404,21.55530273)
\lineto(509.74389404,21.34530273)
\curveto(509.74389148,21.27529989)(509.74889147,21.20529996)(509.75889404,21.13530273)
\curveto(509.79889142,20.78530038)(509.88889133,20.48530068)(510.02889404,20.23530273)
\curveto(510.16889105,19.98530118)(510.36889085,19.78030139)(510.62889404,19.62030273)
\curveto(510.70889051,19.5703016)(510.78889043,19.53030164)(510.86889404,19.50030273)
\curveto(510.95889026,19.4703017)(511.05389017,19.44030173)(511.15389404,19.41030273)
\curveto(511.20389002,19.39030178)(511.25388997,19.38530178)(511.30389404,19.39530273)
\curveto(511.36388986,19.40530176)(511.4188898,19.40030177)(511.46889404,19.38030273)
\curveto(511.49888972,19.3703018)(511.53388969,19.3653018)(511.57389404,19.36530273)
\lineto(511.70889404,19.36530273)
\lineto(511.84389404,19.36530273)
\curveto(511.88388934,19.37530179)(511.93888928,19.38030179)(512.00889404,19.38030273)
\curveto(512.08888913,19.40030177)(512.16888905,19.41530175)(512.24889404,19.42530273)
\curveto(512.33888888,19.44530172)(512.4188888,19.4703017)(512.48889404,19.50030273)
\curveto(512.84888837,19.64030153)(513.15388807,19.81530135)(513.40389404,20.02530273)
\curveto(513.65388757,20.24530092)(513.87888734,20.52030065)(514.07889404,20.85030273)
\curveto(514.14888707,20.96030021)(514.20388702,21.0703001)(514.24389404,21.18030273)
\lineto(514.39389404,21.51030273)
\curveto(514.4238868,21.55029962)(514.43888678,21.58529958)(514.43889404,21.61530273)
\curveto(514.44888677,21.65529951)(514.46388676,21.69529947)(514.48389404,21.73530273)
\curveto(514.50388672,21.79529937)(514.5188867,21.85529931)(514.52889404,21.91530273)
\curveto(514.53888668,21.97529919)(514.55388667,22.03529913)(514.57389404,22.09530273)
}
}
{
\newrgbcolor{curcolor}{0 0 0}
\pscustom[linestyle=none,fillstyle=solid,fillcolor=curcolor]
{
\newpath
\moveto(524.96514404,22.65030273)
\curveto(524.97513515,22.59029858)(524.96513516,22.49529867)(524.93514404,22.36530273)
\curveto(524.91513521,22.24529892)(524.89513523,22.16029901)(524.87514404,22.11030273)
\lineto(524.84514404,21.96030273)
\curveto(524.81513531,21.88029929)(524.79013534,21.80529936)(524.77014404,21.73530273)
\curveto(524.76013537,21.67529949)(524.74013539,21.60529956)(524.71014404,21.52530273)
\curveto(524.68013545,21.4652997)(524.65513547,21.40529976)(524.63514404,21.34530273)
\curveto(524.6251355,21.28529988)(524.60013553,21.22529994)(524.56014404,21.16530273)
\lineto(524.38014404,20.77530273)
\curveto(524.3301358,20.64530052)(524.26513586,20.52530064)(524.18514404,20.41530273)
\curveto(523.88513624,19.93530123)(523.5251366,19.53030164)(523.10514404,19.20030273)
\curveto(522.69513743,18.88030229)(522.21513791,18.63530253)(521.66514404,18.46530273)
\curveto(521.55513857,18.42530274)(521.43513869,18.39530277)(521.30514404,18.37530273)
\curveto(521.17513895,18.35530281)(521.04013909,18.33530283)(520.90014404,18.31530273)
\curveto(520.84013929,18.30530286)(520.77513935,18.30030287)(520.70514404,18.30030273)
\curveto(520.64513948,18.29030288)(520.58513954,18.28530288)(520.52514404,18.28530273)
\curveto(520.48513964,18.27530289)(520.4251397,18.2703029)(520.34514404,18.27030273)
\curveto(520.27513985,18.2703029)(520.2251399,18.27530289)(520.19514404,18.28530273)
\curveto(520.15513997,18.29530287)(520.11514001,18.30030287)(520.07514404,18.30030273)
\curveto(520.03514009,18.29030288)(520.00014013,18.29030288)(519.97014404,18.30030273)
\lineto(519.88014404,18.30030273)
\lineto(519.53514404,18.34530273)
\lineto(519.14514404,18.46530273)
\curveto(519.0251411,18.50530266)(518.91014122,18.55030262)(518.80014404,18.60030273)
\curveto(518.39014174,18.80030237)(518.07014206,19.06030211)(517.84014404,19.38030273)
\curveto(517.62014251,19.70030147)(517.46014267,20.09030108)(517.36014404,20.55030273)
\curveto(517.3301428,20.65030052)(517.31014282,20.75030042)(517.30014404,20.85030273)
\lineto(517.30014404,21.16530273)
\curveto(517.29014284,21.20529996)(517.29014284,21.23529993)(517.30014404,21.25530273)
\curveto(517.31014282,21.28529988)(517.31514281,21.32029985)(517.31514404,21.36030273)
\curveto(517.31514281,21.44029973)(517.32014281,21.52029965)(517.33014404,21.60030273)
\curveto(517.34014279,21.69029948)(517.34514278,21.77529939)(517.34514404,21.85530273)
\curveto(517.35514277,21.90529926)(517.36014277,21.94529922)(517.36014404,21.97530273)
\curveto(517.37014276,22.01529915)(517.37514275,22.06029911)(517.37514404,22.11030273)
\curveto(517.37514275,22.16029901)(517.38514274,22.24529892)(517.40514404,22.36530273)
\curveto(517.43514269,22.49529867)(517.46514266,22.59029858)(517.49514404,22.65030273)
\curveto(517.53514259,22.72029845)(517.55514257,22.79029838)(517.55514404,22.86030273)
\curveto(517.55514257,22.93029824)(517.57514255,23.00029817)(517.61514404,23.07030273)
\curveto(517.63514249,23.12029805)(517.65014248,23.16029801)(517.66014404,23.19030273)
\curveto(517.67014246,23.23029794)(517.68514244,23.27529789)(517.70514404,23.32530273)
\curveto(517.76514236,23.44529772)(517.81514231,23.5652976)(517.85514404,23.68530273)
\curveto(517.90514222,23.80529736)(517.97014216,23.92029725)(518.05014404,24.03030273)
\curveto(518.27014186,24.40029677)(518.51514161,24.73029644)(518.78514404,25.02030273)
\curveto(519.06514106,25.32029585)(519.38014075,25.5702956)(519.73014404,25.77030273)
\curveto(519.86014027,25.85029532)(519.99514013,25.91529525)(520.13514404,25.96530273)
\lineto(520.58514404,26.14530273)
\curveto(520.71513941,26.19529497)(520.85013928,26.22529494)(520.99014404,26.23530273)
\curveto(521.130139,26.25529491)(521.27513885,26.28529488)(521.42514404,26.32530273)
\lineto(521.62014404,26.32530273)
\lineto(521.83014404,26.35530273)
\curveto(522.72013741,26.3652948)(523.42013671,26.18029499)(523.93014404,25.80030273)
\curveto(524.45013568,25.42029575)(524.77513535,24.92529624)(524.90514404,24.31530273)
\curveto(524.93513519,24.21529695)(524.95513517,24.11529705)(524.96514404,24.01530273)
\curveto(524.97513515,23.91529725)(524.99013514,23.81029736)(525.01014404,23.70030273)
\curveto(525.02013511,23.59029758)(525.02013511,23.4702977)(525.01014404,23.34030273)
\lineto(525.01014404,22.96530273)
\curveto(525.01013512,22.91529825)(525.00013513,22.86029831)(524.98014404,22.80030273)
\curveto(524.97013516,22.75029842)(524.96513516,22.70029847)(524.96514404,22.65030273)
\moveto(523.46514404,21.79530273)
\curveto(523.49513663,21.8652993)(523.51513661,21.94529922)(523.52514404,22.03530273)
\curveto(523.54513658,22.12529904)(523.56013657,22.21029896)(523.57014404,22.29030273)
\curveto(523.65013648,22.68029849)(523.68513644,23.01029816)(523.67514404,23.28030273)
\curveto(523.65513647,23.36029781)(523.64013649,23.44029773)(523.63014404,23.52030273)
\curveto(523.6301365,23.60029757)(523.6251365,23.67529749)(523.61514404,23.74530273)
\curveto(523.46513666,24.39529677)(523.11013702,24.84529632)(522.55014404,25.09530273)
\curveto(522.48013765,25.12529604)(522.40513772,25.14529602)(522.32514404,25.15530273)
\curveto(522.25513787,25.17529599)(522.18013795,25.19529597)(522.10014404,25.21530273)
\curveto(522.0301381,25.23529593)(521.95013818,25.24529592)(521.86014404,25.24530273)
\lineto(521.59014404,25.24530273)
\lineto(521.30514404,25.20030273)
\curveto(521.20513892,25.18029599)(521.11013902,25.15529601)(521.02014404,25.12530273)
\curveto(520.9301392,25.10529606)(520.84013929,25.07529609)(520.75014404,25.03530273)
\curveto(520.68013945,25.01529615)(520.61013952,24.98529618)(520.54014404,24.94530273)
\curveto(520.47013966,24.90529626)(520.40513972,24.8652963)(520.34514404,24.82530273)
\curveto(520.07514005,24.65529651)(519.84014029,24.45029672)(519.64014404,24.21030273)
\curveto(519.44014069,23.9702972)(519.25514087,23.69029748)(519.08514404,23.37030273)
\curveto(519.03514109,23.2702979)(518.99514113,23.165298)(518.96514404,23.05530273)
\curveto(518.93514119,22.95529821)(518.89514123,22.85029832)(518.84514404,22.74030273)
\curveto(518.83514129,22.70029847)(518.82014131,22.63529853)(518.80014404,22.54530273)
\curveto(518.78014135,22.51529865)(518.77014136,22.48029869)(518.77014404,22.44030273)
\curveto(518.77014136,22.40029877)(518.76514136,22.35529881)(518.75514404,22.30530273)
\lineto(518.69514404,22.00530273)
\curveto(518.67514145,21.90529926)(518.66514146,21.81529935)(518.66514404,21.73530273)
\lineto(518.66514404,21.55530273)
\curveto(518.66514146,21.45529971)(518.66014147,21.35529981)(518.65014404,21.25530273)
\curveto(518.65014148,21.1653)(518.66014147,21.08030009)(518.68014404,21.00030273)
\curveto(518.7301414,20.76030041)(518.80014133,20.53530063)(518.89014404,20.32530273)
\curveto(518.99014114,20.11530105)(519.125141,19.94030123)(519.29514404,19.80030273)
\curveto(519.34514078,19.7703014)(519.38514074,19.74530142)(519.41514404,19.72530273)
\curveto(519.45514067,19.70530146)(519.49514063,19.68030149)(519.53514404,19.65030273)
\curveto(519.60514052,19.60030157)(519.68514044,19.55530161)(519.77514404,19.51530273)
\curveto(519.86514026,19.48530168)(519.96014017,19.45530171)(520.06014404,19.42530273)
\curveto(520.11014002,19.40530176)(520.15513997,19.39530177)(520.19514404,19.39530273)
\curveto(520.24513988,19.40530176)(520.29513983,19.40530176)(520.34514404,19.39530273)
\curveto(520.37513975,19.38530178)(520.43513969,19.37530179)(520.52514404,19.36530273)
\curveto(520.61513951,19.35530181)(520.69013944,19.36030181)(520.75014404,19.38030273)
\curveto(520.79013934,19.39030178)(520.8301393,19.39030178)(520.87014404,19.38030273)
\curveto(520.91013922,19.38030179)(520.95013918,19.39030178)(520.99014404,19.41030273)
\curveto(521.07013906,19.43030174)(521.15013898,19.44530172)(521.23014404,19.45530273)
\curveto(521.32013881,19.47530169)(521.40513872,19.50030167)(521.48514404,19.53030273)
\curveto(521.84513828,19.6703015)(522.15513797,19.8653013)(522.41514404,20.11530273)
\curveto(522.67513745,20.3653008)(522.91013722,20.66030051)(523.12014404,21.00030273)
\curveto(523.20013693,21.12030005)(523.26013687,21.24529992)(523.30014404,21.37530273)
\curveto(523.34013679,21.51529965)(523.39513673,21.65529951)(523.46514404,21.79530273)
}
}
{
\newrgbcolor{curcolor}{0 0 0}
\pscustom[linestyle=none,fillstyle=solid,fillcolor=curcolor]
{
\newpath
\moveto(529.63342529,26.35530273)
\curveto(530.35341964,26.3652948)(530.93841905,26.28029489)(531.38842529,26.10030273)
\curveto(531.84841814,25.93029524)(532.16841782,25.62529554)(532.34842529,25.18530273)
\curveto(532.39841759,25.07529609)(532.42841756,24.96029621)(532.43842529,24.84030273)
\curveto(532.45841753,24.73029644)(532.47341752,24.60529656)(532.48342529,24.46530273)
\curveto(532.4934175,24.39529677)(532.48341751,24.32029685)(532.45342529,24.24030273)
\curveto(532.43341756,24.170297)(532.40841758,24.11529705)(532.37842529,24.07530273)
\curveto(532.35841763,24.05529711)(532.32841766,24.03529713)(532.28842529,24.01530273)
\curveto(532.25841773,24.00529716)(532.23341776,23.99029718)(532.21342529,23.97030273)
\curveto(532.15341784,23.95029722)(532.09841789,23.94529722)(532.04842529,23.95530273)
\curveto(532.00841798,23.9652972)(531.96341803,23.9652972)(531.91342529,23.95530273)
\curveto(531.82341817,23.93529723)(531.71341828,23.93029724)(531.58342529,23.94030273)
\curveto(531.46341853,23.96029721)(531.37841861,23.98529718)(531.32842529,24.01530273)
\curveto(531.25841873,24.0652971)(531.21841877,24.13029704)(531.20842529,24.21030273)
\curveto(531.20841878,24.30029687)(531.1884188,24.38529678)(531.14842529,24.46530273)
\curveto(531.09841889,24.62529654)(531.00341899,24.7702964)(530.86342529,24.90030273)
\curveto(530.77341922,24.98029619)(530.66341933,25.04029613)(530.53342529,25.08030273)
\curveto(530.41341958,25.12029605)(530.28341971,25.16029601)(530.14342529,25.20030273)
\curveto(530.10341989,25.22029595)(530.05341994,25.22529594)(529.99342529,25.21530273)
\curveto(529.94342005,25.21529595)(529.89842009,25.22029595)(529.85842529,25.23030273)
\curveto(529.79842019,25.25029592)(529.72342027,25.26029591)(529.63342529,25.26030273)
\curveto(529.54342045,25.26029591)(529.46842052,25.25029592)(529.40842529,25.23030273)
\lineto(529.31842529,25.23030273)
\curveto(529.25842073,25.22029595)(529.20342079,25.21029596)(529.15342529,25.20030273)
\curveto(529.10342089,25.20029597)(529.05342094,25.19529597)(529.00342529,25.18530273)
\curveto(528.73342126,25.12529604)(528.49842149,25.04029613)(528.29842529,24.93030273)
\curveto(528.10842188,24.82029635)(527.95842203,24.63529653)(527.84842529,24.37530273)
\curveto(527.81842217,24.30529686)(527.80342219,24.23529693)(527.80342529,24.16530273)
\curveto(527.80342219,24.09529707)(527.80842218,24.03529713)(527.81842529,23.98530273)
\curveto(527.84842214,23.83529733)(527.89842209,23.72529744)(527.96842529,23.65530273)
\curveto(528.03842195,23.59529757)(528.13342186,23.52529764)(528.25342529,23.44530273)
\curveto(528.3934216,23.34529782)(528.55842143,23.2702979)(528.74842529,23.22030273)
\curveto(528.93842105,23.18029799)(529.12842086,23.13029804)(529.31842529,23.07030273)
\curveto(529.43842055,23.03029814)(529.55842043,23.00029817)(529.67842529,22.98030273)
\curveto(529.80842018,22.96029821)(529.93342006,22.93029824)(530.05342529,22.89030273)
\curveto(530.25341974,22.83029834)(530.44841954,22.7702984)(530.63842529,22.71030273)
\curveto(530.82841916,22.66029851)(531.01341898,22.59529857)(531.19342529,22.51530273)
\curveto(531.24341875,22.49529867)(531.2884187,22.47529869)(531.32842529,22.45530273)
\curveto(531.37841861,22.43529873)(531.42841856,22.41029876)(531.47842529,22.38030273)
\curveto(531.64841834,22.26029891)(531.7934182,22.12529904)(531.91342529,21.97530273)
\curveto(532.03341796,21.82529934)(532.12341787,21.63529953)(532.18342529,21.40530273)
\lineto(532.18342529,21.12030273)
\curveto(532.18341781,21.05030012)(532.17841781,20.97530019)(532.16842529,20.89530273)
\curveto(532.15841783,20.82530034)(532.14841784,20.74530042)(532.13842529,20.65530273)
\lineto(532.10842529,20.50530273)
\curveto(532.06841792,20.43530073)(532.03841795,20.3653008)(532.01842529,20.29530273)
\curveto(532.00841798,20.22530094)(531.988418,20.15530101)(531.95842529,20.08530273)
\curveto(531.90841808,19.97530119)(531.85341814,19.8703013)(531.79342529,19.77030273)
\curveto(531.73341826,19.6703015)(531.66841832,19.58030159)(531.59842529,19.50030273)
\curveto(531.3884186,19.24030193)(531.14341885,19.03030214)(530.86342529,18.87030273)
\curveto(530.58341941,18.72030245)(530.27841971,18.59030258)(529.94842529,18.48030273)
\curveto(529.84842014,18.45030272)(529.74842024,18.43030274)(529.64842529,18.42030273)
\curveto(529.54842044,18.40030277)(529.45342054,18.37530279)(529.36342529,18.34530273)
\curveto(529.25342074,18.32530284)(529.14842084,18.31530285)(529.04842529,18.31530273)
\curveto(528.94842104,18.31530285)(528.84842114,18.30530286)(528.74842529,18.28530273)
\lineto(528.59842529,18.28530273)
\curveto(528.54842144,18.27530289)(528.47842151,18.2703029)(528.38842529,18.27030273)
\curveto(528.29842169,18.2703029)(528.22842176,18.27530289)(528.17842529,18.28530273)
\lineto(528.01342529,18.28530273)
\curveto(527.95342204,18.30530286)(527.8884221,18.31530285)(527.81842529,18.31530273)
\curveto(527.74842224,18.30530286)(527.6934223,18.31030286)(527.65342529,18.33030273)
\curveto(527.60342239,18.34030283)(527.53842245,18.34530282)(527.45842529,18.34530273)
\curveto(527.37842261,18.3653028)(527.30342269,18.38530278)(527.23342529,18.40530273)
\curveto(527.16342283,18.41530275)(527.0884229,18.43530273)(527.00842529,18.46530273)
\curveto(526.71842327,18.5653026)(526.47342352,18.69030248)(526.27342529,18.84030273)
\curveto(526.07342392,18.99030218)(525.91342408,19.18530198)(525.79342529,19.42530273)
\curveto(525.73342426,19.55530161)(525.68342431,19.69030148)(525.64342529,19.83030273)
\curveto(525.61342438,19.9703012)(525.5934244,20.12530104)(525.58342529,20.29530273)
\curveto(525.57342442,20.35530081)(525.57842441,20.42530074)(525.59842529,20.50530273)
\curveto(525.61842437,20.59530057)(525.64342435,20.6653005)(525.67342529,20.71530273)
\curveto(525.71342428,20.75530041)(525.77342422,20.79530037)(525.85342529,20.83530273)
\curveto(525.90342409,20.85530031)(525.97342402,20.8653003)(526.06342529,20.86530273)
\curveto(526.16342383,20.87530029)(526.25342374,20.87530029)(526.33342529,20.86530273)
\curveto(526.42342357,20.85530031)(526.50842348,20.84030033)(526.58842529,20.82030273)
\curveto(526.67842331,20.81030036)(526.73342326,20.79530037)(526.75342529,20.77530273)
\curveto(526.81342318,20.72530044)(526.84342315,20.65030052)(526.84342529,20.55030273)
\curveto(526.85342314,20.46030071)(526.87342312,20.37530079)(526.90342529,20.29530273)
\curveto(526.95342304,20.07530109)(527.05342294,19.90530126)(527.20342529,19.78530273)
\curveto(527.30342269,19.69530147)(527.42342257,19.62530154)(527.56342529,19.57530273)
\curveto(527.70342229,19.52530164)(527.85342214,19.47530169)(528.01342529,19.42530273)
\lineto(528.32842529,19.38030273)
\lineto(528.41842529,19.38030273)
\curveto(528.47842151,19.36030181)(528.56342143,19.35030182)(528.67342529,19.35030273)
\curveto(528.7934212,19.35030182)(528.89842109,19.36030181)(528.98842529,19.38030273)
\curveto(529.05842093,19.38030179)(529.11342088,19.38530178)(529.15342529,19.39530273)
\curveto(529.21342078,19.40530176)(529.27342072,19.41030176)(529.33342529,19.41030273)
\curveto(529.3934206,19.42030175)(529.44842054,19.43030174)(529.49842529,19.44030273)
\curveto(529.80842018,19.52030165)(530.05841993,19.62530154)(530.24842529,19.75530273)
\curveto(530.44841954,19.88530128)(530.61341938,20.10530106)(530.74342529,20.41530273)
\curveto(530.77341922,20.4653007)(530.7884192,20.52030065)(530.78842529,20.58030273)
\curveto(530.79841919,20.64030053)(530.79841919,20.68530048)(530.78842529,20.71530273)
\curveto(530.77841921,20.90530026)(530.73841925,21.04530012)(530.66842529,21.13530273)
\curveto(530.59841939,21.23529993)(530.50341949,21.32529984)(530.38342529,21.40530273)
\curveto(530.30341969,21.4652997)(530.20841978,21.51529965)(530.09842529,21.55530273)
\lineto(529.79842529,21.67530273)
\curveto(529.76842022,21.68529948)(529.73842025,21.69029948)(529.70842529,21.69030273)
\curveto(529.6884203,21.69029948)(529.66842032,21.70029947)(529.64842529,21.72030273)
\curveto(529.32842066,21.83029934)(528.988421,21.91029926)(528.62842529,21.96030273)
\curveto(528.27842171,22.02029915)(527.95842203,22.11529905)(527.66842529,22.24530273)
\curveto(527.57842241,22.28529888)(527.4884225,22.32029885)(527.39842529,22.35030273)
\curveto(527.31842267,22.38029879)(527.24342275,22.42029875)(527.17342529,22.47030273)
\curveto(527.00342299,22.58029859)(526.85342314,22.70529846)(526.72342529,22.84530273)
\curveto(526.5934234,22.98529818)(526.50342349,23.16029801)(526.45342529,23.37030273)
\curveto(526.43342356,23.44029773)(526.42342357,23.51029766)(526.42342529,23.58030273)
\lineto(526.42342529,23.80530273)
\curveto(526.41342358,23.92529724)(526.42842356,24.06029711)(526.46842529,24.21030273)
\curveto(526.50842348,24.3702968)(526.54842344,24.50529666)(526.58842529,24.61530273)
\curveto(526.61842337,24.6652965)(526.63842335,24.70529646)(526.64842529,24.73530273)
\curveto(526.66842332,24.77529639)(526.6934233,24.81529635)(526.72342529,24.85530273)
\curveto(526.85342314,25.08529608)(527.01342298,25.28529588)(527.20342529,25.45530273)
\curveto(527.3934226,25.62529554)(527.60342239,25.77529539)(527.83342529,25.90530273)
\curveto(527.993422,25.99529517)(528.16842182,26.0652951)(528.35842529,26.11530273)
\curveto(528.55842143,26.17529499)(528.76342123,26.23029494)(528.97342529,26.28030273)
\curveto(529.04342095,26.29029488)(529.10842088,26.30029487)(529.16842529,26.31030273)
\curveto(529.23842075,26.32029485)(529.31342068,26.33029484)(529.39342529,26.34030273)
\curveto(529.43342056,26.35029482)(529.47342052,26.35029482)(529.51342529,26.34030273)
\curveto(529.56342043,26.33029484)(529.60342039,26.33529483)(529.63342529,26.35530273)
}
}
{
\newrgbcolor{curcolor}{0 0 0}
\pscustom[linewidth=1,linecolor=curcolor]
{
\newpath
\moveto(159.01786,88.52252)
\lineto(796.99776,88.52252)
}
}
{
\newrgbcolor{curcolor}{0 0 0}
\pscustom[linewidth=1,linecolor=curcolor]
{
\newpath
\moveto(159.01786,162.51623)
\lineto(796.99776,162.51623)
}
}
{
\newrgbcolor{curcolor}{0 0 0}
\pscustom[linewidth=1,linecolor=curcolor]
{
\newpath
\moveto(159.01786,236.55822)
\lineto(796.99776,236.55822)
}
}
{
\newrgbcolor{curcolor}{0 0 0}
\pscustom[linewidth=1,linecolor=curcolor]
{
\newpath
\moveto(159.01786,311.62335)
\lineto(796.99776,311.62335)
}
}
{
\newrgbcolor{curcolor}{0 0 0}
\pscustom[linewidth=1,linecolor=curcolor]
{
\newpath
\moveto(159.01786,385.589017)
\lineto(796.99776,385.589017)
}
}
{
\newrgbcolor{curcolor}{0 0 0}
\pscustom[linestyle=none,fillstyle=solid,fillcolor=curcolor]
{
\newpath
\moveto(160.44285278,421.81210297)
\lineto(161.73285278,421.81210297)
\curveto(161.84284996,421.81209229)(161.94784985,421.80709229)(162.04785278,421.79710297)
\curveto(162.14784965,421.7970923)(162.22284958,421.76209234)(162.27285278,421.69210297)
\curveto(162.32284948,421.62209248)(162.34784945,421.53209257)(162.34785278,421.42210297)
\curveto(162.35784944,421.31209279)(162.36284944,421.19209291)(162.36285278,421.06210297)
\lineto(162.36285278,419.75710297)
\lineto(162.36285278,414.55210297)
\lineto(162.36285278,412.09210297)
\lineto(162.36285278,411.65710297)
\curveto(162.37284943,411.4971026)(162.35284945,411.37710272)(162.30285278,411.29710297)
\curveto(162.26284954,411.22710287)(162.17284963,411.17210293)(162.03285278,411.13210297)
\curveto(161.96284984,411.11210299)(161.88784991,411.10710299)(161.80785278,411.11710297)
\curveto(161.72785007,411.12710297)(161.64785015,411.13210297)(161.56785278,411.13210297)
\lineto(160.68285278,411.13210297)
\curveto(160.57285123,411.13210297)(160.46785133,411.13710296)(160.36785278,411.14710297)
\curveto(160.27785152,411.15710294)(160.2028516,411.18710291)(160.14285278,411.23710297)
\curveto(160.09285171,411.28710281)(160.06285174,411.36210274)(160.05285278,411.46210297)
\curveto(160.04285176,411.56210254)(160.03785176,411.66710243)(160.03785278,411.77710297)
\lineto(160.03785278,413.08210297)
\lineto(160.03785278,418.55710297)
\lineto(160.03785278,420.74710297)
\curveto(160.03785176,420.88709321)(160.03285177,421.05209305)(160.02285278,421.24210297)
\curveto(160.02285178,421.43209267)(160.04785175,421.56709253)(160.09785278,421.64710297)
\curveto(160.13785166,421.70709239)(160.2028516,421.75709234)(160.29285278,421.79710297)
\curveto(160.32285148,421.7970923)(160.34785145,421.7970923)(160.36785278,421.79710297)
\curveto(160.3978514,421.80709229)(160.42285138,421.81209229)(160.44285278,421.81210297)
}
}
{
\newrgbcolor{curcolor}{0 0 0}
\pscustom[linestyle=none,fillstyle=solid,fillcolor=curcolor]
{
\newpath
\moveto(168.64668091,419.06710297)
\curveto(169.2466751,419.08709501)(169.7466746,419.0020951)(170.14668091,418.81210297)
\curveto(170.5466738,418.62209548)(170.86167349,418.34209576)(171.09168091,417.97210297)
\curveto(171.16167319,417.86209624)(171.21667313,417.74209636)(171.25668091,417.61210297)
\curveto(171.29667305,417.49209661)(171.33667301,417.36709673)(171.37668091,417.23710297)
\curveto(171.39667295,417.15709694)(171.40667294,417.08209702)(171.40668091,417.01210297)
\curveto(171.41667293,416.94209716)(171.43167292,416.87209723)(171.45168091,416.80210297)
\curveto(171.4516729,416.74209736)(171.45667289,416.7020974)(171.46668091,416.68210297)
\curveto(171.48667286,416.54209756)(171.49667285,416.3970977)(171.49668091,416.24710297)
\lineto(171.49668091,415.81210297)
\lineto(171.49668091,414.47710297)
\lineto(171.49668091,412.04710297)
\curveto(171.49667285,411.85710224)(171.49167286,411.67210243)(171.48168091,411.49210297)
\curveto(171.48167287,411.32210278)(171.41167294,411.21210289)(171.27168091,411.16210297)
\curveto(171.21167314,411.14210296)(171.14167321,411.13210297)(171.06168091,411.13210297)
\lineto(170.82168091,411.13210297)
\lineto(170.01168091,411.13210297)
\curveto(169.89167446,411.13210297)(169.78167457,411.13710296)(169.68168091,411.14710297)
\curveto(169.59167476,411.16710293)(169.52167483,411.21210289)(169.47168091,411.28210297)
\curveto(169.43167492,411.34210276)(169.40667494,411.41710268)(169.39668091,411.50710297)
\lineto(169.39668091,411.82210297)
\lineto(169.39668091,412.87210297)
\lineto(169.39668091,415.10710297)
\curveto(169.39667495,415.47709862)(169.38167497,415.81709828)(169.35168091,416.12710297)
\curveto(169.32167503,416.44709765)(169.23167512,416.71709738)(169.08168091,416.93710297)
\curveto(168.94167541,417.13709696)(168.73667561,417.27709682)(168.46668091,417.35710297)
\curveto(168.41667593,417.37709672)(168.36167599,417.38709671)(168.30168091,417.38710297)
\curveto(168.2516761,417.38709671)(168.19667615,417.3970967)(168.13668091,417.41710297)
\curveto(168.08667626,417.42709667)(168.02167633,417.42709667)(167.94168091,417.41710297)
\curveto(167.87167648,417.41709668)(167.81667653,417.41209669)(167.77668091,417.40210297)
\curveto(167.73667661,417.39209671)(167.70167665,417.38709671)(167.67168091,417.38710297)
\curveto(167.64167671,417.38709671)(167.61167674,417.38209672)(167.58168091,417.37210297)
\curveto(167.351677,417.31209679)(167.16667718,417.23209687)(167.02668091,417.13210297)
\curveto(166.70667764,416.9020972)(166.51667783,416.56709753)(166.45668091,416.12710297)
\curveto(166.39667795,415.68709841)(166.36667798,415.19209891)(166.36668091,414.64210297)
\lineto(166.36668091,412.76710297)
\lineto(166.36668091,411.85210297)
\lineto(166.36668091,411.58210297)
\curveto(166.36667798,411.49210261)(166.351678,411.41710268)(166.32168091,411.35710297)
\curveto(166.27167808,411.24710285)(166.19167816,411.18210292)(166.08168091,411.16210297)
\curveto(165.97167838,411.14210296)(165.83667851,411.13210297)(165.67668091,411.13210297)
\lineto(164.92668091,411.13210297)
\curveto(164.81667953,411.13210297)(164.70667964,411.13710296)(164.59668091,411.14710297)
\curveto(164.48667986,411.15710294)(164.40667994,411.19210291)(164.35668091,411.25210297)
\curveto(164.28668006,411.34210276)(164.2516801,411.47210263)(164.25168091,411.64210297)
\curveto(164.26168009,411.81210229)(164.26668008,411.97210213)(164.26668091,412.12210297)
\lineto(164.26668091,414.16210297)
\lineto(164.26668091,417.46210297)
\lineto(164.26668091,418.22710297)
\lineto(164.26668091,418.52710297)
\curveto(164.27668007,418.61709548)(164.30668004,418.69209541)(164.35668091,418.75210297)
\curveto(164.37667997,418.78209532)(164.40667994,418.8020953)(164.44668091,418.81210297)
\curveto(164.49667985,418.83209527)(164.5466798,418.84709525)(164.59668091,418.85710297)
\lineto(164.67168091,418.85710297)
\curveto(164.72167963,418.86709523)(164.77167958,418.87209523)(164.82168091,418.87210297)
\lineto(164.98668091,418.87210297)
\lineto(165.61668091,418.87210297)
\curveto(165.69667865,418.87209523)(165.77167858,418.86709523)(165.84168091,418.85710297)
\curveto(165.92167843,418.85709524)(165.99167836,418.84709525)(166.05168091,418.82710297)
\curveto(166.12167823,418.7970953)(166.16667818,418.75209535)(166.18668091,418.69210297)
\curveto(166.21667813,418.63209547)(166.24167811,418.56209554)(166.26168091,418.48210297)
\curveto(166.27167808,418.44209566)(166.27167808,418.40709569)(166.26168091,418.37710297)
\curveto(166.26167809,418.34709575)(166.27167808,418.31709578)(166.29168091,418.28710297)
\curveto(166.31167804,418.23709586)(166.32667802,418.20709589)(166.33668091,418.19710297)
\curveto(166.35667799,418.18709591)(166.38167797,418.17209593)(166.41168091,418.15210297)
\curveto(166.52167783,418.14209596)(166.61167774,418.17709592)(166.68168091,418.25710297)
\curveto(166.7516776,418.34709575)(166.82667752,418.41709568)(166.90668091,418.46710297)
\curveto(167.17667717,418.66709543)(167.47667687,418.82709527)(167.80668091,418.94710297)
\curveto(167.89667645,418.97709512)(167.98667636,418.9970951)(168.07668091,419.00710297)
\curveto(168.17667617,419.01709508)(168.28167607,419.03209507)(168.39168091,419.05210297)
\curveto(168.42167593,419.06209504)(168.46667588,419.06209504)(168.52668091,419.05210297)
\curveto(168.58667576,419.05209505)(168.62667572,419.05709504)(168.64668091,419.06710297)
}
}
{
\newrgbcolor{curcolor}{0 0 0}
\pscustom[linestyle=none,fillstyle=solid,fillcolor=curcolor]
{
\newpath
\moveto(174.17793091,421.18210297)
\lineto(175.18293091,421.18210297)
\curveto(175.33292792,421.18209292)(175.46292779,421.17209293)(175.57293091,421.15210297)
\curveto(175.69292756,421.14209296)(175.77792748,421.08209302)(175.82793091,420.97210297)
\curveto(175.84792741,420.92209318)(175.8579274,420.86209324)(175.85793091,420.79210297)
\lineto(175.85793091,420.58210297)
\lineto(175.85793091,419.90710297)
\curveto(175.8579274,419.85709424)(175.8529274,419.7970943)(175.84293091,419.72710297)
\curveto(175.84292741,419.66709443)(175.84792741,419.61209449)(175.85793091,419.56210297)
\lineto(175.85793091,419.39710297)
\curveto(175.8579274,419.31709478)(175.86292739,419.24209486)(175.87293091,419.17210297)
\curveto(175.88292737,419.11209499)(175.90792735,419.05709504)(175.94793091,419.00710297)
\curveto(176.01792724,418.91709518)(176.14292711,418.86709523)(176.32293091,418.85710297)
\lineto(176.86293091,418.85710297)
\lineto(177.04293091,418.85710297)
\curveto(177.10292615,418.85709524)(177.1579261,418.84709525)(177.20793091,418.82710297)
\curveto(177.31792594,418.77709532)(177.37792588,418.68709541)(177.38793091,418.55710297)
\curveto(177.40792585,418.42709567)(177.41792584,418.28209582)(177.41793091,418.12210297)
\lineto(177.41793091,417.91210297)
\curveto(177.42792583,417.84209626)(177.42292583,417.78209632)(177.40293091,417.73210297)
\curveto(177.3529259,417.57209653)(177.24792601,417.48709661)(177.08793091,417.47710297)
\curveto(176.92792633,417.46709663)(176.74792651,417.46209664)(176.54793091,417.46210297)
\lineto(176.41293091,417.46210297)
\curveto(176.37292688,417.47209663)(176.33792692,417.47209663)(176.30793091,417.46210297)
\curveto(176.26792699,417.45209665)(176.23292702,417.44709665)(176.20293091,417.44710297)
\curveto(176.17292708,417.45709664)(176.14292711,417.45209665)(176.11293091,417.43210297)
\curveto(176.03292722,417.41209669)(175.97292728,417.36709673)(175.93293091,417.29710297)
\curveto(175.90292735,417.23709686)(175.87792738,417.16209694)(175.85793091,417.07210297)
\curveto(175.84792741,417.02209708)(175.84792741,416.96709713)(175.85793091,416.90710297)
\curveto(175.86792739,416.84709725)(175.86792739,416.79209731)(175.85793091,416.74210297)
\lineto(175.85793091,415.81210297)
\lineto(175.85793091,414.05710297)
\curveto(175.8579274,413.80710029)(175.86292739,413.58710051)(175.87293091,413.39710297)
\curveto(175.89292736,413.21710088)(175.9579273,413.05710104)(176.06793091,412.91710297)
\curveto(176.11792714,412.85710124)(176.18292707,412.81210129)(176.26293091,412.78210297)
\lineto(176.53293091,412.72210297)
\curveto(176.56292669,412.71210139)(176.59292666,412.70710139)(176.62293091,412.70710297)
\curveto(176.66292659,412.71710138)(176.69292656,412.71710138)(176.71293091,412.70710297)
\lineto(176.87793091,412.70710297)
\curveto(176.98792627,412.70710139)(177.08292617,412.7021014)(177.16293091,412.69210297)
\curveto(177.24292601,412.68210142)(177.30792595,412.64210146)(177.35793091,412.57210297)
\curveto(177.39792586,412.51210159)(177.41792584,412.43210167)(177.41793091,412.33210297)
\lineto(177.41793091,412.04710297)
\curveto(177.41792584,411.83710226)(177.41292584,411.64210246)(177.40293091,411.46210297)
\curveto(177.40292585,411.29210281)(177.32292593,411.17710292)(177.16293091,411.11710297)
\curveto(177.11292614,411.097103)(177.06792619,411.09210301)(177.02793091,411.10210297)
\curveto(176.98792627,411.102103)(176.94292631,411.09210301)(176.89293091,411.07210297)
\lineto(176.74293091,411.07210297)
\curveto(176.72292653,411.07210303)(176.69292656,411.07710302)(176.65293091,411.08710297)
\curveto(176.61292664,411.08710301)(176.57792668,411.08210302)(176.54793091,411.07210297)
\curveto(176.49792676,411.06210304)(176.44292681,411.06210304)(176.38293091,411.07210297)
\lineto(176.23293091,411.07210297)
\lineto(176.08293091,411.07210297)
\curveto(176.03292722,411.06210304)(175.98792727,411.06210304)(175.94793091,411.07210297)
\lineto(175.78293091,411.07210297)
\curveto(175.73292752,411.08210302)(175.67792758,411.08710301)(175.61793091,411.08710297)
\curveto(175.5579277,411.08710301)(175.50292775,411.09210301)(175.45293091,411.10210297)
\curveto(175.38292787,411.11210299)(175.31792794,411.12210298)(175.25793091,411.13210297)
\lineto(175.07793091,411.16210297)
\curveto(174.96792829,411.19210291)(174.86292839,411.22710287)(174.76293091,411.26710297)
\curveto(174.66292859,411.30710279)(174.56792869,411.35210275)(174.47793091,411.40210297)
\lineto(174.38793091,411.46210297)
\curveto(174.3579289,411.49210261)(174.32292893,411.52210258)(174.28293091,411.55210297)
\curveto(174.26292899,411.57210253)(174.23792902,411.59210251)(174.20793091,411.61210297)
\lineto(174.13293091,411.68710297)
\curveto(173.99292926,411.87710222)(173.88792937,412.08710201)(173.81793091,412.31710297)
\curveto(173.79792946,412.35710174)(173.78792947,412.39210171)(173.78793091,412.42210297)
\curveto(173.79792946,412.46210164)(173.79792946,412.50710159)(173.78793091,412.55710297)
\curveto(173.77792948,412.57710152)(173.77292948,412.6021015)(173.77293091,412.63210297)
\curveto(173.77292948,412.66210144)(173.76792949,412.68710141)(173.75793091,412.70710297)
\lineto(173.75793091,412.85710297)
\curveto(173.74792951,412.8971012)(173.74292951,412.94210116)(173.74293091,412.99210297)
\curveto(173.7529295,413.04210106)(173.7579295,413.09210101)(173.75793091,413.14210297)
\lineto(173.75793091,413.71210297)
\lineto(173.75793091,415.94710297)
\lineto(173.75793091,416.74210297)
\lineto(173.75793091,416.95210297)
\curveto(173.76792949,417.02209708)(173.76292949,417.08709701)(173.74293091,417.14710297)
\curveto(173.70292955,417.28709681)(173.63292962,417.37709672)(173.53293091,417.41710297)
\curveto(173.42292983,417.46709663)(173.28292997,417.48209662)(173.11293091,417.46210297)
\curveto(172.94293031,417.44209666)(172.79793046,417.45709664)(172.67793091,417.50710297)
\curveto(172.59793066,417.53709656)(172.54793071,417.58209652)(172.52793091,417.64210297)
\curveto(172.50793075,417.7020964)(172.48793077,417.77709632)(172.46793091,417.86710297)
\lineto(172.46793091,418.18210297)
\curveto(172.46793079,418.36209574)(172.47793078,418.50709559)(172.49793091,418.61710297)
\curveto(172.51793074,418.72709537)(172.60293065,418.8020953)(172.75293091,418.84210297)
\curveto(172.79293046,418.86209524)(172.83293042,418.86709523)(172.87293091,418.85710297)
\lineto(173.00793091,418.85710297)
\curveto(173.1579301,418.85709524)(173.29792996,418.86209524)(173.42793091,418.87210297)
\curveto(173.5579297,418.89209521)(173.64792961,418.95209515)(173.69793091,419.05210297)
\curveto(173.72792953,419.12209498)(173.74292951,419.2020949)(173.74293091,419.29210297)
\curveto(173.7529295,419.38209472)(173.7579295,419.47209463)(173.75793091,419.56210297)
\lineto(173.75793091,420.49210297)
\lineto(173.75793091,420.74710297)
\curveto(173.7579295,420.83709326)(173.76792949,420.91209319)(173.78793091,420.97210297)
\curveto(173.83792942,421.07209303)(173.91292934,421.13709296)(174.01293091,421.16710297)
\curveto(174.03292922,421.17709292)(174.0579292,421.17709292)(174.08793091,421.16710297)
\curveto(174.12792913,421.16709293)(174.1579291,421.17209293)(174.17793091,421.18210297)
}
}
{
\newrgbcolor{curcolor}{0 0 0}
\pscustom[linestyle=none,fillstyle=solid,fillcolor=curcolor]
{
\newpath
\moveto(185.76636841,415.07710297)
\curveto(185.78636024,414.9970991)(185.78636024,414.90709919)(185.76636841,414.80710297)
\curveto(185.74636028,414.70709939)(185.71136032,414.64209946)(185.66136841,414.61210297)
\curveto(185.61136042,414.57209953)(185.53636049,414.54209956)(185.43636841,414.52210297)
\curveto(185.34636068,414.51209959)(185.24136079,414.5020996)(185.12136841,414.49210297)
\lineto(184.77636841,414.49210297)
\curveto(184.66636136,414.5020996)(184.56636146,414.50709959)(184.47636841,414.50710297)
\lineto(180.81636841,414.50710297)
\lineto(180.60636841,414.50710297)
\curveto(180.54636548,414.50709959)(180.49136554,414.4970996)(180.44136841,414.47710297)
\curveto(180.36136567,414.43709966)(180.31136572,414.3970997)(180.29136841,414.35710297)
\curveto(180.27136576,414.33709976)(180.25136578,414.2970998)(180.23136841,414.23710297)
\curveto(180.21136582,414.18709991)(180.20636582,414.13709996)(180.21636841,414.08710297)
\curveto(180.23636579,414.02710007)(180.24636578,413.96710013)(180.24636841,413.90710297)
\curveto(180.25636577,413.85710024)(180.27136576,413.8021003)(180.29136841,413.74210297)
\curveto(180.37136566,413.5021006)(180.46636556,413.3021008)(180.57636841,413.14210297)
\curveto(180.69636533,412.99210111)(180.85636517,412.85710124)(181.05636841,412.73710297)
\curveto(181.13636489,412.68710141)(181.21636481,412.65210145)(181.29636841,412.63210297)
\curveto(181.38636464,412.62210148)(181.47636455,412.6021015)(181.56636841,412.57210297)
\curveto(181.64636438,412.55210155)(181.75636427,412.53710156)(181.89636841,412.52710297)
\curveto(182.03636399,412.51710158)(182.15636387,412.52210158)(182.25636841,412.54210297)
\lineto(182.39136841,412.54210297)
\curveto(182.49136354,412.56210154)(182.58136345,412.58210152)(182.66136841,412.60210297)
\curveto(182.75136328,412.63210147)(182.83636319,412.66210144)(182.91636841,412.69210297)
\curveto(183.01636301,412.74210136)(183.1263629,412.80710129)(183.24636841,412.88710297)
\curveto(183.37636265,412.96710113)(183.47136256,413.04710105)(183.53136841,413.12710297)
\curveto(183.58136245,413.1971009)(183.6313624,413.26210084)(183.68136841,413.32210297)
\curveto(183.74136229,413.39210071)(183.81136222,413.44210066)(183.89136841,413.47210297)
\curveto(183.99136204,413.52210058)(184.11636191,413.54210056)(184.26636841,413.53210297)
\lineto(184.70136841,413.53210297)
\lineto(184.88136841,413.53210297)
\curveto(184.95136108,413.54210056)(185.01136102,413.53710056)(185.06136841,413.51710297)
\lineto(185.21136841,413.51710297)
\curveto(185.31136072,413.4971006)(185.38136065,413.47210063)(185.42136841,413.44210297)
\curveto(185.46136057,413.42210068)(185.48136055,413.37710072)(185.48136841,413.30710297)
\curveto(185.49136054,413.23710086)(185.48636054,413.17710092)(185.46636841,413.12710297)
\curveto(185.41636061,412.98710111)(185.36136067,412.86210124)(185.30136841,412.75210297)
\curveto(185.24136079,412.64210146)(185.17136086,412.53210157)(185.09136841,412.42210297)
\curveto(184.87136116,412.09210201)(184.62136141,411.82710227)(184.34136841,411.62710297)
\curveto(184.06136197,411.42710267)(183.71136232,411.25710284)(183.29136841,411.11710297)
\curveto(183.18136285,411.07710302)(183.07136296,411.05210305)(182.96136841,411.04210297)
\curveto(182.85136318,411.03210307)(182.73636329,411.01210309)(182.61636841,410.98210297)
\curveto(182.57636345,410.97210313)(182.5313635,410.97210313)(182.48136841,410.98210297)
\curveto(182.44136359,410.98210312)(182.40136363,410.97710312)(182.36136841,410.96710297)
\lineto(182.19636841,410.96710297)
\curveto(182.14636388,410.94710315)(182.08636394,410.94210316)(182.01636841,410.95210297)
\curveto(181.95636407,410.95210315)(181.90136413,410.95710314)(181.85136841,410.96710297)
\curveto(181.77136426,410.97710312)(181.70136433,410.97710312)(181.64136841,410.96710297)
\curveto(181.58136445,410.95710314)(181.51636451,410.96210314)(181.44636841,410.98210297)
\curveto(181.39636463,411.0021031)(181.34136469,411.01210309)(181.28136841,411.01210297)
\curveto(181.22136481,411.01210309)(181.16636486,411.02210308)(181.11636841,411.04210297)
\curveto(181.00636502,411.06210304)(180.89636513,411.08710301)(180.78636841,411.11710297)
\curveto(180.67636535,411.13710296)(180.57636545,411.17210293)(180.48636841,411.22210297)
\curveto(180.37636565,411.26210284)(180.27136576,411.2971028)(180.17136841,411.32710297)
\curveto(180.08136595,411.36710273)(179.99636603,411.41210269)(179.91636841,411.46210297)
\curveto(179.59636643,411.66210244)(179.31136672,411.89210221)(179.06136841,412.15210297)
\curveto(178.81136722,412.42210168)(178.60636742,412.73210137)(178.44636841,413.08210297)
\curveto(178.39636763,413.19210091)(178.35636767,413.3021008)(178.32636841,413.41210297)
\curveto(178.29636773,413.53210057)(178.25636777,413.65210045)(178.20636841,413.77210297)
\curveto(178.19636783,413.81210029)(178.19136784,413.84710025)(178.19136841,413.87710297)
\curveto(178.19136784,413.91710018)(178.18636784,413.95710014)(178.17636841,413.99710297)
\curveto(178.13636789,414.11709998)(178.11136792,414.24709985)(178.10136841,414.38710297)
\lineto(178.07136841,414.80710297)
\curveto(178.07136796,414.85709924)(178.06636796,414.91209919)(178.05636841,414.97210297)
\curveto(178.05636797,415.03209907)(178.06136797,415.08709901)(178.07136841,415.13710297)
\lineto(178.07136841,415.31710297)
\lineto(178.11636841,415.67710297)
\curveto(178.15636787,415.84709825)(178.19136784,416.01209809)(178.22136841,416.17210297)
\curveto(178.25136778,416.33209777)(178.29636773,416.48209762)(178.35636841,416.62210297)
\curveto(178.78636724,417.66209644)(179.51636651,418.3970957)(180.54636841,418.82710297)
\curveto(180.68636534,418.88709521)(180.8263652,418.92709517)(180.96636841,418.94710297)
\curveto(181.11636491,418.97709512)(181.27136476,419.01209509)(181.43136841,419.05210297)
\curveto(181.51136452,419.06209504)(181.58636444,419.06709503)(181.65636841,419.06710297)
\curveto(181.7263643,419.06709503)(181.80136423,419.07209503)(181.88136841,419.08210297)
\curveto(182.39136364,419.09209501)(182.8263632,419.03209507)(183.18636841,418.90210297)
\curveto(183.55636247,418.78209532)(183.88636214,418.62209548)(184.17636841,418.42210297)
\curveto(184.26636176,418.36209574)(184.35636167,418.29209581)(184.44636841,418.21210297)
\curveto(184.53636149,418.14209596)(184.61636141,418.06709603)(184.68636841,417.98710297)
\curveto(184.71636131,417.93709616)(184.75636127,417.8970962)(184.80636841,417.86710297)
\curveto(184.88636114,417.75709634)(184.96136107,417.64209646)(185.03136841,417.52210297)
\curveto(185.10136093,417.41209669)(185.17636085,417.2970968)(185.25636841,417.17710297)
\curveto(185.30636072,417.08709701)(185.34636068,416.99209711)(185.37636841,416.89210297)
\curveto(185.41636061,416.8020973)(185.45636057,416.7020974)(185.49636841,416.59210297)
\curveto(185.54636048,416.46209764)(185.58636044,416.32709777)(185.61636841,416.18710297)
\curveto(185.64636038,416.04709805)(185.68136035,415.90709819)(185.72136841,415.76710297)
\curveto(185.74136029,415.68709841)(185.74636028,415.5970985)(185.73636841,415.49710297)
\curveto(185.73636029,415.40709869)(185.74636028,415.32209878)(185.76636841,415.24210297)
\lineto(185.76636841,415.07710297)
\moveto(183.51636841,415.96210297)
\curveto(183.58636244,416.06209804)(183.59136244,416.18209792)(183.53136841,416.32210297)
\curveto(183.48136255,416.47209763)(183.44136259,416.58209752)(183.41136841,416.65210297)
\curveto(183.27136276,416.92209718)(183.08636294,417.12709697)(182.85636841,417.26710297)
\curveto(182.6263634,417.41709668)(182.30636372,417.4970966)(181.89636841,417.50710297)
\curveto(181.86636416,417.48709661)(181.8313642,417.48209662)(181.79136841,417.49210297)
\curveto(181.75136428,417.5020966)(181.71636431,417.5020966)(181.68636841,417.49210297)
\curveto(181.63636439,417.47209663)(181.58136445,417.45709664)(181.52136841,417.44710297)
\curveto(181.46136457,417.44709665)(181.40636462,417.43709666)(181.35636841,417.41710297)
\curveto(180.91636511,417.27709682)(180.59136544,417.0020971)(180.38136841,416.59210297)
\curveto(180.36136567,416.55209755)(180.33636569,416.4970976)(180.30636841,416.42710297)
\curveto(180.28636574,416.36709773)(180.27136576,416.3020978)(180.26136841,416.23210297)
\curveto(180.25136578,416.17209793)(180.25136578,416.11209799)(180.26136841,416.05210297)
\curveto(180.28136575,415.99209811)(180.31636571,415.94209816)(180.36636841,415.90210297)
\curveto(180.44636558,415.85209825)(180.55636547,415.82709827)(180.69636841,415.82710297)
\lineto(181.10136841,415.82710297)
\lineto(182.76636841,415.82710297)
\lineto(183.20136841,415.82710297)
\curveto(183.36136267,415.83709826)(183.46636256,415.88209822)(183.51636841,415.96210297)
}
}
{
\newrgbcolor{curcolor}{0 0 0}
\pscustom[linestyle=none,fillstyle=solid,fillcolor=curcolor]
{
\newpath
\moveto(191.43964966,419.06710297)
\curveto(192.03964385,419.08709501)(192.53964335,419.0020951)(192.93964966,418.81210297)
\curveto(193.33964255,418.62209548)(193.65464224,418.34209576)(193.88464966,417.97210297)
\curveto(193.95464194,417.86209624)(194.00964188,417.74209636)(194.04964966,417.61210297)
\curveto(194.0896418,417.49209661)(194.12964176,417.36709673)(194.16964966,417.23710297)
\curveto(194.1896417,417.15709694)(194.19964169,417.08209702)(194.19964966,417.01210297)
\curveto(194.20964168,416.94209716)(194.22464167,416.87209723)(194.24464966,416.80210297)
\curveto(194.24464165,416.74209736)(194.24964164,416.7020974)(194.25964966,416.68210297)
\curveto(194.27964161,416.54209756)(194.2896416,416.3970977)(194.28964966,416.24710297)
\lineto(194.28964966,415.81210297)
\lineto(194.28964966,414.47710297)
\lineto(194.28964966,412.04710297)
\curveto(194.2896416,411.85710224)(194.28464161,411.67210243)(194.27464966,411.49210297)
\curveto(194.27464162,411.32210278)(194.20464169,411.21210289)(194.06464966,411.16210297)
\curveto(194.00464189,411.14210296)(193.93464196,411.13210297)(193.85464966,411.13210297)
\lineto(193.61464966,411.13210297)
\lineto(192.80464966,411.13210297)
\curveto(192.68464321,411.13210297)(192.57464332,411.13710296)(192.47464966,411.14710297)
\curveto(192.38464351,411.16710293)(192.31464358,411.21210289)(192.26464966,411.28210297)
\curveto(192.22464367,411.34210276)(192.19964369,411.41710268)(192.18964966,411.50710297)
\lineto(192.18964966,411.82210297)
\lineto(192.18964966,412.87210297)
\lineto(192.18964966,415.10710297)
\curveto(192.1896437,415.47709862)(192.17464372,415.81709828)(192.14464966,416.12710297)
\curveto(192.11464378,416.44709765)(192.02464387,416.71709738)(191.87464966,416.93710297)
\curveto(191.73464416,417.13709696)(191.52964436,417.27709682)(191.25964966,417.35710297)
\curveto(191.20964468,417.37709672)(191.15464474,417.38709671)(191.09464966,417.38710297)
\curveto(191.04464485,417.38709671)(190.9896449,417.3970967)(190.92964966,417.41710297)
\curveto(190.87964501,417.42709667)(190.81464508,417.42709667)(190.73464966,417.41710297)
\curveto(190.66464523,417.41709668)(190.60964528,417.41209669)(190.56964966,417.40210297)
\curveto(190.52964536,417.39209671)(190.4946454,417.38709671)(190.46464966,417.38710297)
\curveto(190.43464546,417.38709671)(190.40464549,417.38209672)(190.37464966,417.37210297)
\curveto(190.14464575,417.31209679)(189.95964593,417.23209687)(189.81964966,417.13210297)
\curveto(189.49964639,416.9020972)(189.30964658,416.56709753)(189.24964966,416.12710297)
\curveto(189.1896467,415.68709841)(189.15964673,415.19209891)(189.15964966,414.64210297)
\lineto(189.15964966,412.76710297)
\lineto(189.15964966,411.85210297)
\lineto(189.15964966,411.58210297)
\curveto(189.15964673,411.49210261)(189.14464675,411.41710268)(189.11464966,411.35710297)
\curveto(189.06464683,411.24710285)(188.98464691,411.18210292)(188.87464966,411.16210297)
\curveto(188.76464713,411.14210296)(188.62964726,411.13210297)(188.46964966,411.13210297)
\lineto(187.71964966,411.13210297)
\curveto(187.60964828,411.13210297)(187.49964839,411.13710296)(187.38964966,411.14710297)
\curveto(187.27964861,411.15710294)(187.19964869,411.19210291)(187.14964966,411.25210297)
\curveto(187.07964881,411.34210276)(187.04464885,411.47210263)(187.04464966,411.64210297)
\curveto(187.05464884,411.81210229)(187.05964883,411.97210213)(187.05964966,412.12210297)
\lineto(187.05964966,414.16210297)
\lineto(187.05964966,417.46210297)
\lineto(187.05964966,418.22710297)
\lineto(187.05964966,418.52710297)
\curveto(187.06964882,418.61709548)(187.09964879,418.69209541)(187.14964966,418.75210297)
\curveto(187.16964872,418.78209532)(187.19964869,418.8020953)(187.23964966,418.81210297)
\curveto(187.2896486,418.83209527)(187.33964855,418.84709525)(187.38964966,418.85710297)
\lineto(187.46464966,418.85710297)
\curveto(187.51464838,418.86709523)(187.56464833,418.87209523)(187.61464966,418.87210297)
\lineto(187.77964966,418.87210297)
\lineto(188.40964966,418.87210297)
\curveto(188.4896474,418.87209523)(188.56464733,418.86709523)(188.63464966,418.85710297)
\curveto(188.71464718,418.85709524)(188.78464711,418.84709525)(188.84464966,418.82710297)
\curveto(188.91464698,418.7970953)(188.95964693,418.75209535)(188.97964966,418.69210297)
\curveto(189.00964688,418.63209547)(189.03464686,418.56209554)(189.05464966,418.48210297)
\curveto(189.06464683,418.44209566)(189.06464683,418.40709569)(189.05464966,418.37710297)
\curveto(189.05464684,418.34709575)(189.06464683,418.31709578)(189.08464966,418.28710297)
\curveto(189.10464679,418.23709586)(189.11964677,418.20709589)(189.12964966,418.19710297)
\curveto(189.14964674,418.18709591)(189.17464672,418.17209593)(189.20464966,418.15210297)
\curveto(189.31464658,418.14209596)(189.40464649,418.17709592)(189.47464966,418.25710297)
\curveto(189.54464635,418.34709575)(189.61964627,418.41709568)(189.69964966,418.46710297)
\curveto(189.96964592,418.66709543)(190.26964562,418.82709527)(190.59964966,418.94710297)
\curveto(190.6896452,418.97709512)(190.77964511,418.9970951)(190.86964966,419.00710297)
\curveto(190.96964492,419.01709508)(191.07464482,419.03209507)(191.18464966,419.05210297)
\curveto(191.21464468,419.06209504)(191.25964463,419.06209504)(191.31964966,419.05210297)
\curveto(191.37964451,419.05209505)(191.41964447,419.05709504)(191.43964966,419.06710297)
}
}
{
\newrgbcolor{curcolor}{0 0 0}
\pscustom[linestyle=none,fillstyle=solid,fillcolor=curcolor]
{
\newpath
\moveto(199.49089966,419.08210297)
\curveto(200.3008945,419.102095)(200.97589382,418.98209512)(201.51589966,418.72210297)
\curveto(202.06589273,418.46209564)(202.5008923,418.09209601)(202.82089966,417.61210297)
\curveto(202.98089182,417.37209673)(203.1008917,417.097097)(203.18089966,416.78710297)
\curveto(203.2008916,416.73709736)(203.21589158,416.67209743)(203.22589966,416.59210297)
\curveto(203.24589155,416.51209759)(203.24589155,416.44209766)(203.22589966,416.38210297)
\curveto(203.18589161,416.27209783)(203.11589168,416.20709789)(203.01589966,416.18710297)
\curveto(202.91589188,416.17709792)(202.795892,416.17209793)(202.65589966,416.17210297)
\lineto(201.87589966,416.17210297)
\lineto(201.59089966,416.17210297)
\curveto(201.5008933,416.17209793)(201.42589337,416.19209791)(201.36589966,416.23210297)
\curveto(201.28589351,416.27209783)(201.23089357,416.33209777)(201.20089966,416.41210297)
\curveto(201.17089363,416.5020976)(201.13089367,416.59209751)(201.08089966,416.68210297)
\curveto(201.02089378,416.79209731)(200.95589384,416.89209721)(200.88589966,416.98210297)
\curveto(200.81589398,417.07209703)(200.73589406,417.15209695)(200.64589966,417.22210297)
\curveto(200.50589429,417.31209679)(200.35089445,417.38209672)(200.18089966,417.43210297)
\curveto(200.12089468,417.45209665)(200.06089474,417.46209664)(200.00089966,417.46210297)
\curveto(199.94089486,417.46209664)(199.88589491,417.47209663)(199.83589966,417.49210297)
\lineto(199.68589966,417.49210297)
\curveto(199.48589531,417.49209661)(199.32589547,417.47209663)(199.20589966,417.43210297)
\curveto(198.91589588,417.34209676)(198.68089612,417.2020969)(198.50089966,417.01210297)
\curveto(198.32089648,416.83209727)(198.17589662,416.61209749)(198.06589966,416.35210297)
\curveto(198.01589678,416.24209786)(197.97589682,416.12209798)(197.94589966,415.99210297)
\curveto(197.92589687,415.87209823)(197.9008969,415.74209836)(197.87089966,415.60210297)
\curveto(197.86089694,415.56209854)(197.85589694,415.52209858)(197.85589966,415.48210297)
\curveto(197.85589694,415.44209866)(197.85089695,415.4020987)(197.84089966,415.36210297)
\curveto(197.82089698,415.26209884)(197.81089699,415.12209898)(197.81089966,414.94210297)
\curveto(197.82089698,414.76209934)(197.83589696,414.62209948)(197.85589966,414.52210297)
\curveto(197.85589694,414.44209966)(197.86089694,414.38709971)(197.87089966,414.35710297)
\curveto(197.89089691,414.28709981)(197.9008969,414.21709988)(197.90089966,414.14710297)
\curveto(197.91089689,414.07710002)(197.92589687,414.00710009)(197.94589966,413.93710297)
\curveto(198.02589677,413.70710039)(198.12089668,413.4971006)(198.23089966,413.30710297)
\curveto(198.34089646,413.11710098)(198.48089632,412.95710114)(198.65089966,412.82710297)
\curveto(198.69089611,412.7971013)(198.75089605,412.76210134)(198.83089966,412.72210297)
\curveto(198.94089586,412.65210145)(199.05089575,412.60710149)(199.16089966,412.58710297)
\curveto(199.28089552,412.56710153)(199.42589537,412.54710155)(199.59589966,412.52710297)
\lineto(199.68589966,412.52710297)
\curveto(199.72589507,412.52710157)(199.75589504,412.53210157)(199.77589966,412.54210297)
\lineto(199.91089966,412.54210297)
\curveto(199.98089482,412.56210154)(200.04589475,412.57710152)(200.10589966,412.58710297)
\curveto(200.17589462,412.60710149)(200.24089456,412.62710147)(200.30089966,412.64710297)
\curveto(200.6008942,412.77710132)(200.83089397,412.96710113)(200.99089966,413.21710297)
\curveto(201.03089377,413.26710083)(201.06589373,413.32210078)(201.09589966,413.38210297)
\curveto(201.12589367,413.45210065)(201.15089365,413.51210059)(201.17089966,413.56210297)
\curveto(201.21089359,413.67210043)(201.24589355,413.76710033)(201.27589966,413.84710297)
\curveto(201.30589349,413.93710016)(201.37589342,414.00710009)(201.48589966,414.05710297)
\curveto(201.57589322,414.0971)(201.72089308,414.11209999)(201.92089966,414.10210297)
\lineto(202.41589966,414.10210297)
\lineto(202.62589966,414.10210297)
\curveto(202.70589209,414.11209999)(202.77089203,414.10709999)(202.82089966,414.08710297)
\lineto(202.94089966,414.08710297)
\lineto(203.06089966,414.05710297)
\curveto(203.1008917,414.05710004)(203.13089167,414.04710005)(203.15089966,414.02710297)
\curveto(203.2008916,413.98710011)(203.23089157,413.92710017)(203.24089966,413.84710297)
\curveto(203.26089154,413.77710032)(203.26089154,413.7021004)(203.24089966,413.62210297)
\curveto(203.15089165,413.29210081)(203.04089176,412.9971011)(202.91089966,412.73710297)
\curveto(202.5008923,411.96710213)(201.84589295,411.43210267)(200.94589966,411.13210297)
\curveto(200.84589395,411.102103)(200.74089406,411.08210302)(200.63089966,411.07210297)
\curveto(200.52089428,411.05210305)(200.41089439,411.02710307)(200.30089966,410.99710297)
\curveto(200.24089456,410.98710311)(200.18089462,410.98210312)(200.12089966,410.98210297)
\curveto(200.06089474,410.98210312)(200.0008948,410.97710312)(199.94089966,410.96710297)
\lineto(199.77589966,410.96710297)
\curveto(199.72589507,410.94710315)(199.65089515,410.94210316)(199.55089966,410.95210297)
\curveto(199.45089535,410.95210315)(199.37589542,410.95710314)(199.32589966,410.96710297)
\curveto(199.24589555,410.98710311)(199.17089563,410.9971031)(199.10089966,410.99710297)
\curveto(199.04089576,410.98710311)(198.97589582,410.99210311)(198.90589966,411.01210297)
\lineto(198.75589966,411.04210297)
\curveto(198.70589609,411.04210306)(198.65589614,411.04710305)(198.60589966,411.05710297)
\curveto(198.4958963,411.08710301)(198.39089641,411.11710298)(198.29089966,411.14710297)
\curveto(198.19089661,411.17710292)(198.0958967,411.21210289)(198.00589966,411.25210297)
\curveto(197.53589726,411.45210265)(197.14089766,411.70710239)(196.82089966,412.01710297)
\curveto(196.5008983,412.33710176)(196.24089856,412.73210137)(196.04089966,413.20210297)
\curveto(195.99089881,413.29210081)(195.95089885,413.38710071)(195.92089966,413.48710297)
\lineto(195.83089966,413.81710297)
\curveto(195.82089898,413.85710024)(195.81589898,413.89210021)(195.81589966,413.92210297)
\curveto(195.81589898,413.96210014)(195.80589899,414.00710009)(195.78589966,414.05710297)
\curveto(195.76589903,414.12709997)(195.75589904,414.1970999)(195.75589966,414.26710297)
\curveto(195.75589904,414.34709975)(195.74589905,414.42209968)(195.72589966,414.49210297)
\lineto(195.72589966,414.74710297)
\curveto(195.70589909,414.7970993)(195.6958991,414.85209925)(195.69589966,414.91210297)
\curveto(195.6958991,414.98209912)(195.70589909,415.04209906)(195.72589966,415.09210297)
\curveto(195.73589906,415.14209896)(195.73589906,415.18709891)(195.72589966,415.22710297)
\curveto(195.71589908,415.26709883)(195.71589908,415.30709879)(195.72589966,415.34710297)
\curveto(195.74589905,415.41709868)(195.75089905,415.48209862)(195.74089966,415.54210297)
\curveto(195.74089906,415.6020985)(195.75089905,415.66209844)(195.77089966,415.72210297)
\curveto(195.82089898,415.9020982)(195.86089894,416.07209803)(195.89089966,416.23210297)
\curveto(195.92089888,416.4020977)(195.96589883,416.56709753)(196.02589966,416.72710297)
\curveto(196.24589855,417.23709686)(196.52089828,417.66209644)(196.85089966,418.00210297)
\curveto(197.19089761,418.34209576)(197.62089718,418.61709548)(198.14089966,418.82710297)
\curveto(198.28089652,418.88709521)(198.42589637,418.92709517)(198.57589966,418.94710297)
\curveto(198.72589607,418.97709512)(198.88089592,419.01209509)(199.04089966,419.05210297)
\curveto(199.12089568,419.06209504)(199.1958956,419.06709503)(199.26589966,419.06710297)
\curveto(199.33589546,419.06709503)(199.41089539,419.07209503)(199.49089966,419.08210297)
}
}
{
\newrgbcolor{curcolor}{0 0 0}
\pscustom[linestyle=none,fillstyle=solid,fillcolor=curcolor]
{
\newpath
\moveto(206.63418091,421.72210297)
\curveto(206.70417796,421.64209246)(206.73917792,421.52209258)(206.73918091,421.36210297)
\lineto(206.73918091,420.89710297)
\lineto(206.73918091,420.49210297)
\curveto(206.73917792,420.35209375)(206.70417796,420.25709384)(206.63418091,420.20710297)
\curveto(206.57417809,420.15709394)(206.49417817,420.12709397)(206.39418091,420.11710297)
\curveto(206.30417836,420.10709399)(206.20417846,420.102094)(206.09418091,420.10210297)
\lineto(205.25418091,420.10210297)
\curveto(205.14417952,420.102094)(205.04417962,420.10709399)(204.95418091,420.11710297)
\curveto(204.87417979,420.12709397)(204.80417986,420.15709394)(204.74418091,420.20710297)
\curveto(204.70417996,420.23709386)(204.67417999,420.29209381)(204.65418091,420.37210297)
\curveto(204.64418002,420.46209364)(204.63418003,420.55709354)(204.62418091,420.65710297)
\lineto(204.62418091,420.98710297)
\curveto(204.63418003,421.097093)(204.63918002,421.19209291)(204.63918091,421.27210297)
\lineto(204.63918091,421.48210297)
\curveto(204.64918001,421.55209255)(204.66917999,421.61209249)(204.69918091,421.66210297)
\curveto(204.71917994,421.7020924)(204.74417992,421.73209237)(204.77418091,421.75210297)
\lineto(204.89418091,421.81210297)
\curveto(204.91417975,421.81209229)(204.93917972,421.81209229)(204.96918091,421.81210297)
\curveto(204.99917966,421.82209228)(205.02417964,421.82709227)(205.04418091,421.82710297)
\lineto(206.13918091,421.82710297)
\curveto(206.23917842,421.82709227)(206.33417833,421.82209228)(206.42418091,421.81210297)
\curveto(206.51417815,421.8020923)(206.58417808,421.77209233)(206.63418091,421.72210297)
\moveto(206.73918091,411.95710297)
\curveto(206.73917792,411.75710234)(206.73417793,411.58710251)(206.72418091,411.44710297)
\curveto(206.71417795,411.30710279)(206.62417804,411.21210289)(206.45418091,411.16210297)
\curveto(206.39417827,411.14210296)(206.32917833,411.13210297)(206.25918091,411.13210297)
\curveto(206.18917847,411.14210296)(206.11417855,411.14710295)(206.03418091,411.14710297)
\lineto(205.19418091,411.14710297)
\curveto(205.10417956,411.14710295)(205.01417965,411.15210295)(204.92418091,411.16210297)
\curveto(204.84417982,411.17210293)(204.78417988,411.2021029)(204.74418091,411.25210297)
\curveto(204.68417998,411.32210278)(204.64918001,411.40710269)(204.63918091,411.50710297)
\lineto(204.63918091,411.85210297)
\lineto(204.63918091,418.18210297)
\lineto(204.63918091,418.48210297)
\curveto(204.63918002,418.58209552)(204.65918,418.66209544)(204.69918091,418.72210297)
\curveto(204.7591799,418.79209531)(204.84417982,418.83709526)(204.95418091,418.85710297)
\curveto(204.97417969,418.86709523)(204.99917966,418.86709523)(205.02918091,418.85710297)
\curveto(205.06917959,418.85709524)(205.09917956,418.86209524)(205.11918091,418.87210297)
\lineto(205.86918091,418.87210297)
\lineto(206.06418091,418.87210297)
\curveto(206.14417852,418.88209522)(206.20917845,418.88209522)(206.25918091,418.87210297)
\lineto(206.37918091,418.87210297)
\curveto(206.43917822,418.85209525)(206.49417817,418.83709526)(206.54418091,418.82710297)
\curveto(206.59417807,418.81709528)(206.63417803,418.78709531)(206.66418091,418.73710297)
\curveto(206.70417796,418.68709541)(206.72417794,418.61709548)(206.72418091,418.52710297)
\curveto(206.73417793,418.43709566)(206.73917792,418.34209576)(206.73918091,418.24210297)
\lineto(206.73918091,411.95710297)
}
}
{
\newrgbcolor{curcolor}{0 0 0}
\pscustom[linestyle=none,fillstyle=solid,fillcolor=curcolor]
{
\newpath
\moveto(216.17136841,415.31710297)
\curveto(216.15135988,415.36709873)(216.14635988,415.42209868)(216.15636841,415.48210297)
\curveto(216.16635986,415.54209856)(216.16135987,415.5970985)(216.14136841,415.64710297)
\curveto(216.1313599,415.68709841)(216.1263599,415.72709837)(216.12636841,415.76710297)
\curveto(216.1263599,415.80709829)(216.12135991,415.84709825)(216.11136841,415.88710297)
\lineto(216.05136841,416.15710297)
\curveto(216.03136,416.24709785)(216.00636002,416.33209777)(215.97636841,416.41210297)
\curveto(215.9263601,416.55209755)(215.88136015,416.68209742)(215.84136841,416.80210297)
\curveto(215.80136023,416.93209717)(215.74636028,417.05209705)(215.67636841,417.16210297)
\curveto(215.60636042,417.27209683)(215.53636049,417.37709672)(215.46636841,417.47710297)
\curveto(215.40636062,417.57709652)(215.33636069,417.67709642)(215.25636841,417.77710297)
\curveto(215.17636085,417.88709621)(215.07636095,417.98709611)(214.95636841,418.07710297)
\curveto(214.84636118,418.17709592)(214.73636129,418.26709583)(214.62636841,418.34710297)
\curveto(214.29636173,418.57709552)(213.91636211,418.75709534)(213.48636841,418.88710297)
\curveto(213.06636296,419.01709508)(212.56636346,419.07709502)(211.98636841,419.06710297)
\curveto(211.91636411,419.05709504)(211.84636418,419.05209505)(211.77636841,419.05210297)
\curveto(211.70636432,419.05209505)(211.6313644,419.04709505)(211.55136841,419.03710297)
\curveto(211.40136463,418.9970951)(211.25636477,418.96709513)(211.11636841,418.94710297)
\curveto(210.97636505,418.92709517)(210.84136519,418.89209521)(210.71136841,418.84210297)
\curveto(210.60136543,418.79209531)(210.49136554,418.74709535)(210.38136841,418.70710297)
\curveto(210.27136576,418.66709543)(210.16636586,418.62209548)(210.06636841,418.57210297)
\curveto(209.70636632,418.34209576)(209.40136663,418.08709601)(209.15136841,417.80710297)
\curveto(208.90136713,417.53709656)(208.68636734,417.1970969)(208.50636841,416.78710297)
\curveto(208.45636757,416.66709743)(208.41636761,416.54209756)(208.38636841,416.41210297)
\curveto(208.35636767,416.29209781)(208.32136771,416.16709793)(208.28136841,416.03710297)
\curveto(208.26136777,415.98709811)(208.25136778,415.93709816)(208.25136841,415.88710297)
\curveto(208.25136778,415.84709825)(208.24636778,415.8020983)(208.23636841,415.75210297)
\curveto(208.21636781,415.7020984)(208.20636782,415.64709845)(208.20636841,415.58710297)
\curveto(208.21636781,415.53709856)(208.21636781,415.48709861)(208.20636841,415.43710297)
\lineto(208.20636841,415.33210297)
\curveto(208.18636784,415.27209883)(208.17136786,415.18709891)(208.16136841,415.07710297)
\curveto(208.16136787,414.96709913)(208.17136786,414.88209922)(208.19136841,414.82210297)
\lineto(208.19136841,414.68710297)
\curveto(208.19136784,414.64709945)(208.19636783,414.6020995)(208.20636841,414.55210297)
\curveto(208.2263678,414.47209963)(208.23636779,414.38709971)(208.23636841,414.29710297)
\curveto(208.23636779,414.21709988)(208.24636778,414.13709996)(208.26636841,414.05710297)
\curveto(208.28636774,414.00710009)(208.29636773,413.96210014)(208.29636841,413.92210297)
\curveto(208.29636773,413.88210022)(208.30636772,413.83710026)(208.32636841,413.78710297)
\curveto(208.35636767,413.67710042)(208.38136765,413.57210053)(208.40136841,413.47210297)
\curveto(208.4313676,413.37210073)(208.47136756,413.27710082)(208.52136841,413.18710297)
\curveto(208.69136734,412.7971013)(208.90136713,412.46210164)(209.15136841,412.18210297)
\curveto(209.40136663,411.9021022)(209.70136633,411.65710244)(210.05136841,411.44710297)
\curveto(210.16136587,411.38710271)(210.26636576,411.33710276)(210.36636841,411.29710297)
\curveto(210.47636555,411.25710284)(210.59136544,411.21710288)(210.71136841,411.17710297)
\curveto(210.80136523,411.13710296)(210.89636513,411.10710299)(210.99636841,411.08710297)
\curveto(211.09636493,411.06710303)(211.19636483,411.04210306)(211.29636841,411.01210297)
\curveto(211.34636468,411.0021031)(211.38636464,410.9971031)(211.41636841,410.99710297)
\curveto(211.45636457,410.9971031)(211.49636453,410.99210311)(211.53636841,410.98210297)
\curveto(211.58636444,410.96210314)(211.63636439,410.95710314)(211.68636841,410.96710297)
\curveto(211.74636428,410.96710313)(211.80136423,410.96210314)(211.85136841,410.95210297)
\lineto(212.00136841,410.95210297)
\curveto(212.06136397,410.93210317)(212.14636388,410.92710317)(212.25636841,410.93710297)
\curveto(212.36636366,410.93710316)(212.44636358,410.94210316)(212.49636841,410.95210297)
\curveto(212.5263635,410.95210315)(212.55636347,410.95710314)(212.58636841,410.96710297)
\lineto(212.69136841,410.96710297)
\curveto(212.74136329,410.97710312)(212.79636323,410.98210312)(212.85636841,410.98210297)
\curveto(212.91636311,410.98210312)(212.97136306,410.99210311)(213.02136841,411.01210297)
\curveto(213.15136288,411.04210306)(213.27636275,411.07210303)(213.39636841,411.10210297)
\curveto(213.5263625,411.12210298)(213.65136238,411.15710294)(213.77136841,411.20710297)
\curveto(214.25136178,411.40710269)(214.66136137,411.65710244)(215.00136841,411.95710297)
\curveto(215.34136069,412.25710184)(215.61636041,412.64710145)(215.82636841,413.12710297)
\curveto(215.87636015,413.22710087)(215.91636011,413.33210077)(215.94636841,413.44210297)
\curveto(215.97636005,413.56210054)(216.01136002,413.67710042)(216.05136841,413.78710297)
\curveto(216.06135997,413.85710024)(216.07135996,413.92210018)(216.08136841,413.98210297)
\curveto(216.09135994,414.04210006)(216.10635992,414.10709999)(216.12636841,414.17710297)
\curveto(216.14635988,414.25709984)(216.15135988,414.33709976)(216.14136841,414.41710297)
\curveto(216.14135989,414.4970996)(216.15135988,414.57709952)(216.17136841,414.65710297)
\lineto(216.17136841,414.80710297)
\curveto(216.19135984,414.86709923)(216.20135983,414.95209915)(216.20136841,415.06210297)
\curveto(216.20135983,415.17209893)(216.19135984,415.25709884)(216.17136841,415.31710297)
\moveto(214.07136841,414.77710297)
\curveto(214.06136197,414.72709937)(214.05636197,414.67709942)(214.05636841,414.62710297)
\lineto(214.05636841,414.49210297)
\curveto(214.04636198,414.45209965)(214.04136199,414.41209969)(214.04136841,414.37210297)
\curveto(214.04136199,414.34209976)(214.03636199,414.30709979)(214.02636841,414.26710297)
\curveto(213.99636203,414.15709994)(213.97136206,414.05210005)(213.95136841,413.95210297)
\curveto(213.9313621,413.85210025)(213.90136213,413.75210035)(213.86136841,413.65210297)
\curveto(213.75136228,413.4021007)(213.61636241,413.19210091)(213.45636841,413.02210297)
\curveto(213.29636273,412.85210125)(213.08636294,412.71710138)(212.82636841,412.61710297)
\curveto(212.75636327,412.58710151)(212.68136335,412.56710153)(212.60136841,412.55710297)
\curveto(212.52136351,412.54710155)(212.44136359,412.53210157)(212.36136841,412.51210297)
\lineto(212.24136841,412.51210297)
\curveto(212.20136383,412.5021016)(212.15636387,412.4971016)(212.10636841,412.49710297)
\lineto(211.98636841,412.52710297)
\curveto(211.94636408,412.53710156)(211.91136412,412.53710156)(211.88136841,412.52710297)
\curveto(211.85136418,412.52710157)(211.81636421,412.53210157)(211.77636841,412.54210297)
\curveto(211.68636434,412.56210154)(211.59636443,412.58710151)(211.50636841,412.61710297)
\curveto(211.4263646,412.64710145)(211.35136468,412.68710141)(211.28136841,412.73710297)
\curveto(211.031365,412.88710121)(210.84636518,413.05210105)(210.72636841,413.23210297)
\curveto(210.61636541,413.42210068)(210.51136552,413.66710043)(210.41136841,413.96710297)
\curveto(210.39136564,414.04710005)(210.37636565,414.12209998)(210.36636841,414.19210297)
\curveto(210.35636567,414.27209983)(210.34136569,414.35209975)(210.32136841,414.43210297)
\lineto(210.32136841,414.56710297)
\curveto(210.30136573,414.63709946)(210.28636574,414.74209936)(210.27636841,414.88210297)
\curveto(210.27636575,415.02209908)(210.28636574,415.12709897)(210.30636841,415.19710297)
\lineto(210.30636841,415.34710297)
\curveto(210.30636572,415.3970987)(210.31136572,415.44709865)(210.32136841,415.49710297)
\curveto(210.34136569,415.60709849)(210.35636567,415.71709838)(210.36636841,415.82710297)
\curveto(210.38636564,415.93709816)(210.41136562,416.04209806)(210.44136841,416.14210297)
\curveto(210.5313655,416.41209769)(210.65136538,416.64709745)(210.80136841,416.84710297)
\curveto(210.96136507,417.05709704)(211.16636486,417.21709688)(211.41636841,417.32710297)
\curveto(211.46636456,417.35709674)(211.52136451,417.37709672)(211.58136841,417.38710297)
\lineto(211.79136841,417.44710297)
\curveto(211.82136421,417.45709664)(211.85636417,417.45709664)(211.89636841,417.44710297)
\curveto(211.93636409,417.44709665)(211.97136406,417.45709664)(212.00136841,417.47710297)
\lineto(212.27136841,417.47710297)
\curveto(212.36136367,417.48709661)(212.44636358,417.48209662)(212.52636841,417.46210297)
\curveto(212.59636343,417.44209666)(212.66136337,417.42209668)(212.72136841,417.40210297)
\curveto(212.78136325,417.39209671)(212.84136319,417.37709672)(212.90136841,417.35710297)
\curveto(213.15136288,417.24709685)(213.35136268,417.097097)(213.50136841,416.90710297)
\curveto(213.65136238,416.72709737)(213.78136225,416.50709759)(213.89136841,416.24710297)
\curveto(213.92136211,416.16709793)(213.94136209,416.08209802)(213.95136841,415.99210297)
\lineto(214.01136841,415.75210297)
\curveto(214.02136201,415.73209837)(214.026362,415.7020984)(214.02636841,415.66210297)
\curveto(214.03636199,415.61209849)(214.04136199,415.55709854)(214.04136841,415.49710297)
\curveto(214.04136199,415.43709866)(214.05136198,415.38209872)(214.07136841,415.33210297)
\lineto(214.07136841,415.21210297)
\curveto(214.08136195,415.16209894)(214.08636194,415.08709901)(214.08636841,414.98710297)
\curveto(214.08636194,414.8970992)(214.08136195,414.82709927)(214.07136841,414.77710297)
\moveto(212.84136841,421.94710297)
\lineto(213.90636841,421.94710297)
\curveto(213.98636204,421.94709215)(214.08136195,421.94709215)(214.19136841,421.94710297)
\curveto(214.30136173,421.94709215)(214.38136165,421.93209217)(214.43136841,421.90210297)
\curveto(214.45136158,421.89209221)(214.46136157,421.87709222)(214.46136841,421.85710297)
\curveto(214.47136156,421.84709225)(214.48636154,421.83709226)(214.50636841,421.82710297)
\curveto(214.51636151,421.70709239)(214.46636156,421.6020925)(214.35636841,421.51210297)
\curveto(214.25636177,421.42209268)(214.17136186,421.34209276)(214.10136841,421.27210297)
\curveto(214.02136201,421.2020929)(213.94136209,421.12709297)(213.86136841,421.04710297)
\curveto(213.79136224,420.97709312)(213.71636231,420.91209319)(213.63636841,420.85210297)
\curveto(213.59636243,420.82209328)(213.56136247,420.78709331)(213.53136841,420.74710297)
\curveto(213.51136252,420.71709338)(213.48136255,420.69209341)(213.44136841,420.67210297)
\curveto(213.42136261,420.64209346)(213.39636263,420.61709348)(213.36636841,420.59710297)
\lineto(213.21636841,420.44710297)
\lineto(213.06636841,420.32710297)
\lineto(213.02136841,420.28210297)
\curveto(213.02136301,420.27209383)(213.01136302,420.25709384)(212.99136841,420.23710297)
\curveto(212.91136312,420.17709392)(212.8313632,420.11209399)(212.75136841,420.04210297)
\curveto(212.68136335,419.97209413)(212.59136344,419.91709418)(212.48136841,419.87710297)
\curveto(212.44136359,419.86709423)(212.40136363,419.86209424)(212.36136841,419.86210297)
\curveto(212.3313637,419.86209424)(212.29136374,419.85709424)(212.24136841,419.84710297)
\curveto(212.21136382,419.83709426)(212.17136386,419.83209427)(212.12136841,419.83210297)
\curveto(212.07136396,419.84209426)(212.026364,419.84709425)(211.98636841,419.84710297)
\lineto(211.64136841,419.84710297)
\curveto(211.52136451,419.84709425)(211.4313646,419.87209423)(211.37136841,419.92210297)
\curveto(211.31136472,419.96209414)(211.29636473,420.03209407)(211.32636841,420.13210297)
\curveto(211.34636468,420.21209389)(211.38136465,420.28209382)(211.43136841,420.34210297)
\curveto(211.48136455,420.41209369)(211.5263645,420.48209362)(211.56636841,420.55210297)
\curveto(211.66636436,420.69209341)(211.76136427,420.82709327)(211.85136841,420.95710297)
\curveto(211.94136409,421.08709301)(212.031364,421.22209288)(212.12136841,421.36210297)
\curveto(212.17136386,421.44209266)(212.22136381,421.52709257)(212.27136841,421.61710297)
\curveto(212.3313637,421.70709239)(212.39636363,421.77709232)(212.46636841,421.82710297)
\curveto(212.50636352,421.85709224)(212.57636345,421.89209221)(212.67636841,421.93210297)
\curveto(212.69636333,421.94209216)(212.72136331,421.94209216)(212.75136841,421.93210297)
\curveto(212.79136324,421.93209217)(212.82136321,421.93709216)(212.84136841,421.94710297)
}
}
{
\newrgbcolor{curcolor}{0 0 0}
\pscustom[linestyle=none,fillstyle=solid,fillcolor=curcolor]
{
\newpath
\moveto(221.99629028,419.06710297)
\curveto(222.59628448,419.08709501)(223.09628398,419.0020951)(223.49629028,418.81210297)
\curveto(223.89628318,418.62209548)(224.21128286,418.34209576)(224.44129028,417.97210297)
\curveto(224.51128256,417.86209624)(224.56628251,417.74209636)(224.60629028,417.61210297)
\curveto(224.64628243,417.49209661)(224.68628239,417.36709673)(224.72629028,417.23710297)
\curveto(224.74628233,417.15709694)(224.75628232,417.08209702)(224.75629028,417.01210297)
\curveto(224.76628231,416.94209716)(224.78128229,416.87209723)(224.80129028,416.80210297)
\curveto(224.80128227,416.74209736)(224.80628227,416.7020974)(224.81629028,416.68210297)
\curveto(224.83628224,416.54209756)(224.84628223,416.3970977)(224.84629028,416.24710297)
\lineto(224.84629028,415.81210297)
\lineto(224.84629028,414.47710297)
\lineto(224.84629028,412.04710297)
\curveto(224.84628223,411.85710224)(224.84128223,411.67210243)(224.83129028,411.49210297)
\curveto(224.83128224,411.32210278)(224.76128231,411.21210289)(224.62129028,411.16210297)
\curveto(224.56128251,411.14210296)(224.49128258,411.13210297)(224.41129028,411.13210297)
\lineto(224.17129028,411.13210297)
\lineto(223.36129028,411.13210297)
\curveto(223.24128383,411.13210297)(223.13128394,411.13710296)(223.03129028,411.14710297)
\curveto(222.94128413,411.16710293)(222.8712842,411.21210289)(222.82129028,411.28210297)
\curveto(222.78128429,411.34210276)(222.75628432,411.41710268)(222.74629028,411.50710297)
\lineto(222.74629028,411.82210297)
\lineto(222.74629028,412.87210297)
\lineto(222.74629028,415.10710297)
\curveto(222.74628433,415.47709862)(222.73128434,415.81709828)(222.70129028,416.12710297)
\curveto(222.6712844,416.44709765)(222.58128449,416.71709738)(222.43129028,416.93710297)
\curveto(222.29128478,417.13709696)(222.08628499,417.27709682)(221.81629028,417.35710297)
\curveto(221.76628531,417.37709672)(221.71128536,417.38709671)(221.65129028,417.38710297)
\curveto(221.60128547,417.38709671)(221.54628553,417.3970967)(221.48629028,417.41710297)
\curveto(221.43628564,417.42709667)(221.3712857,417.42709667)(221.29129028,417.41710297)
\curveto(221.22128585,417.41709668)(221.16628591,417.41209669)(221.12629028,417.40210297)
\curveto(221.08628599,417.39209671)(221.05128602,417.38709671)(221.02129028,417.38710297)
\curveto(220.99128608,417.38709671)(220.96128611,417.38209672)(220.93129028,417.37210297)
\curveto(220.70128637,417.31209679)(220.51628656,417.23209687)(220.37629028,417.13210297)
\curveto(220.05628702,416.9020972)(219.86628721,416.56709753)(219.80629028,416.12710297)
\curveto(219.74628733,415.68709841)(219.71628736,415.19209891)(219.71629028,414.64210297)
\lineto(219.71629028,412.76710297)
\lineto(219.71629028,411.85210297)
\lineto(219.71629028,411.58210297)
\curveto(219.71628736,411.49210261)(219.70128737,411.41710268)(219.67129028,411.35710297)
\curveto(219.62128745,411.24710285)(219.54128753,411.18210292)(219.43129028,411.16210297)
\curveto(219.32128775,411.14210296)(219.18628789,411.13210297)(219.02629028,411.13210297)
\lineto(218.27629028,411.13210297)
\curveto(218.16628891,411.13210297)(218.05628902,411.13710296)(217.94629028,411.14710297)
\curveto(217.83628924,411.15710294)(217.75628932,411.19210291)(217.70629028,411.25210297)
\curveto(217.63628944,411.34210276)(217.60128947,411.47210263)(217.60129028,411.64210297)
\curveto(217.61128946,411.81210229)(217.61628946,411.97210213)(217.61629028,412.12210297)
\lineto(217.61629028,414.16210297)
\lineto(217.61629028,417.46210297)
\lineto(217.61629028,418.22710297)
\lineto(217.61629028,418.52710297)
\curveto(217.62628945,418.61709548)(217.65628942,418.69209541)(217.70629028,418.75210297)
\curveto(217.72628935,418.78209532)(217.75628932,418.8020953)(217.79629028,418.81210297)
\curveto(217.84628923,418.83209527)(217.89628918,418.84709525)(217.94629028,418.85710297)
\lineto(218.02129028,418.85710297)
\curveto(218.071289,418.86709523)(218.12128895,418.87209523)(218.17129028,418.87210297)
\lineto(218.33629028,418.87210297)
\lineto(218.96629028,418.87210297)
\curveto(219.04628803,418.87209523)(219.12128795,418.86709523)(219.19129028,418.85710297)
\curveto(219.2712878,418.85709524)(219.34128773,418.84709525)(219.40129028,418.82710297)
\curveto(219.4712876,418.7970953)(219.51628756,418.75209535)(219.53629028,418.69210297)
\curveto(219.56628751,418.63209547)(219.59128748,418.56209554)(219.61129028,418.48210297)
\curveto(219.62128745,418.44209566)(219.62128745,418.40709569)(219.61129028,418.37710297)
\curveto(219.61128746,418.34709575)(219.62128745,418.31709578)(219.64129028,418.28710297)
\curveto(219.66128741,418.23709586)(219.6762874,418.20709589)(219.68629028,418.19710297)
\curveto(219.70628737,418.18709591)(219.73128734,418.17209593)(219.76129028,418.15210297)
\curveto(219.8712872,418.14209596)(219.96128711,418.17709592)(220.03129028,418.25710297)
\curveto(220.10128697,418.34709575)(220.1762869,418.41709568)(220.25629028,418.46710297)
\curveto(220.52628655,418.66709543)(220.82628625,418.82709527)(221.15629028,418.94710297)
\curveto(221.24628583,418.97709512)(221.33628574,418.9970951)(221.42629028,419.00710297)
\curveto(221.52628555,419.01709508)(221.63128544,419.03209507)(221.74129028,419.05210297)
\curveto(221.7712853,419.06209504)(221.81628526,419.06209504)(221.87629028,419.05210297)
\curveto(221.93628514,419.05209505)(221.9762851,419.05709504)(221.99629028,419.06710297)
}
}
{
\newrgbcolor{curcolor}{0 0 0}
\pscustom[linestyle=none,fillstyle=solid,fillcolor=curcolor]
{
}
}
{
\newrgbcolor{curcolor}{0 0 0}
\pscustom[linestyle=none,fillstyle=solid,fillcolor=curcolor]
{
\newpath
\moveto(238.22769653,411.98710297)
\lineto(238.22769653,411.56710297)
\curveto(238.22768816,411.43710266)(238.19768819,411.33210277)(238.13769653,411.25210297)
\curveto(238.0876883,411.2021029)(238.02268837,411.16710293)(237.94269653,411.14710297)
\curveto(237.86268853,411.13710296)(237.77268862,411.13210297)(237.67269653,411.13210297)
\lineto(236.84769653,411.13210297)
\lineto(236.56269653,411.13210297)
\curveto(236.48268991,411.14210296)(236.41768997,411.16710293)(236.36769653,411.20710297)
\curveto(236.29769009,411.25710284)(236.25769013,411.32210278)(236.24769653,411.40210297)
\curveto(236.23769015,411.48210262)(236.21769017,411.56210254)(236.18769653,411.64210297)
\curveto(236.16769022,411.66210244)(236.14769024,411.67710242)(236.12769653,411.68710297)
\curveto(236.11769027,411.70710239)(236.10269029,411.72710237)(236.08269653,411.74710297)
\curveto(235.97269042,411.74710235)(235.8926905,411.72210238)(235.84269653,411.67210297)
\lineto(235.69269653,411.52210297)
\curveto(235.62269077,411.47210263)(235.55769083,411.42710267)(235.49769653,411.38710297)
\curveto(235.43769095,411.35710274)(235.37269102,411.31710278)(235.30269653,411.26710297)
\curveto(235.26269113,411.24710285)(235.21769117,411.22710287)(235.16769653,411.20710297)
\curveto(235.12769126,411.18710291)(235.08269131,411.16710293)(235.03269653,411.14710297)
\curveto(234.8926915,411.097103)(234.74269165,411.05210305)(234.58269653,411.01210297)
\curveto(234.53269186,410.99210311)(234.4876919,410.98210312)(234.44769653,410.98210297)
\curveto(234.40769198,410.98210312)(234.36769202,410.97710312)(234.32769653,410.96710297)
\lineto(234.19269653,410.96710297)
\curveto(234.16269223,410.95710314)(234.12269227,410.95210315)(234.07269653,410.95210297)
\lineto(233.93769653,410.95210297)
\curveto(233.87769251,410.93210317)(233.7876926,410.92710317)(233.66769653,410.93710297)
\curveto(233.54769284,410.93710316)(233.46269293,410.94710315)(233.41269653,410.96710297)
\curveto(233.34269305,410.98710311)(233.27769311,410.9971031)(233.21769653,410.99710297)
\curveto(233.16769322,410.98710311)(233.11269328,410.99210311)(233.05269653,411.01210297)
\lineto(232.69269653,411.13210297)
\curveto(232.58269381,411.16210294)(232.47269392,411.2021029)(232.36269653,411.25210297)
\curveto(232.01269438,411.4021027)(231.69769469,411.63210247)(231.41769653,411.94210297)
\curveto(231.14769524,412.26210184)(230.93269546,412.5971015)(230.77269653,412.94710297)
\curveto(230.72269567,413.05710104)(230.68269571,413.16210094)(230.65269653,413.26210297)
\curveto(230.62269577,413.37210073)(230.5876958,413.48210062)(230.54769653,413.59210297)
\curveto(230.53769585,413.63210047)(230.53269586,413.66710043)(230.53269653,413.69710297)
\curveto(230.53269586,413.73710036)(230.52269587,413.78210032)(230.50269653,413.83210297)
\curveto(230.48269591,413.91210019)(230.46269593,413.9971001)(230.44269653,414.08710297)
\curveto(230.43269596,414.18709991)(230.41769597,414.28709981)(230.39769653,414.38710297)
\curveto(230.387696,414.41709968)(230.38269601,414.45209965)(230.38269653,414.49210297)
\curveto(230.392696,414.53209957)(230.392696,414.56709953)(230.38269653,414.59710297)
\lineto(230.38269653,414.73210297)
\curveto(230.38269601,414.78209932)(230.37769601,414.83209927)(230.36769653,414.88210297)
\curveto(230.35769603,414.93209917)(230.35269604,414.98709911)(230.35269653,415.04710297)
\curveto(230.35269604,415.11709898)(230.35769603,415.17209893)(230.36769653,415.21210297)
\curveto(230.37769601,415.26209884)(230.38269601,415.30709879)(230.38269653,415.34710297)
\lineto(230.38269653,415.49710297)
\curveto(230.392696,415.54709855)(230.392696,415.59209851)(230.38269653,415.63210297)
\curveto(230.38269601,415.68209842)(230.392696,415.73209837)(230.41269653,415.78210297)
\curveto(230.43269596,415.89209821)(230.44769594,415.9970981)(230.45769653,416.09710297)
\curveto(230.47769591,416.1970979)(230.50269589,416.2970978)(230.53269653,416.39710297)
\curveto(230.57269582,416.51709758)(230.60769578,416.63209747)(230.63769653,416.74210297)
\curveto(230.66769572,416.85209725)(230.70769568,416.96209714)(230.75769653,417.07210297)
\curveto(230.89769549,417.37209673)(231.07269532,417.65709644)(231.28269653,417.92710297)
\curveto(231.30269509,417.95709614)(231.32769506,417.98209612)(231.35769653,418.00210297)
\curveto(231.39769499,418.03209607)(231.42769496,418.06209604)(231.44769653,418.09210297)
\curveto(231.4876949,418.14209596)(231.52769486,418.18709591)(231.56769653,418.22710297)
\curveto(231.60769478,418.26709583)(231.65269474,418.30709579)(231.70269653,418.34710297)
\curveto(231.74269465,418.36709573)(231.77769461,418.39209571)(231.80769653,418.42210297)
\curveto(231.83769455,418.46209564)(231.87269452,418.49209561)(231.91269653,418.51210297)
\curveto(232.16269423,418.68209542)(232.45269394,418.82209528)(232.78269653,418.93210297)
\curveto(232.85269354,418.95209515)(232.92269347,418.96709513)(232.99269653,418.97710297)
\curveto(233.07269332,418.98709511)(233.15269324,419.0020951)(233.23269653,419.02210297)
\curveto(233.30269309,419.04209506)(233.392693,419.05209505)(233.50269653,419.05210297)
\curveto(233.61269278,419.06209504)(233.72269267,419.06709503)(233.83269653,419.06710297)
\curveto(233.94269245,419.06709503)(234.04769234,419.06209504)(234.14769653,419.05210297)
\curveto(234.25769213,419.04209506)(234.34769204,419.02709507)(234.41769653,419.00710297)
\curveto(234.56769182,418.95709514)(234.71269168,418.91209519)(234.85269653,418.87210297)
\curveto(234.9926914,418.83209527)(235.12269127,418.77709532)(235.24269653,418.70710297)
\curveto(235.31269108,418.65709544)(235.37769101,418.60709549)(235.43769653,418.55710297)
\curveto(235.49769089,418.51709558)(235.56269083,418.47209563)(235.63269653,418.42210297)
\curveto(235.67269072,418.39209571)(235.72769066,418.35209575)(235.79769653,418.30210297)
\curveto(235.87769051,418.25209585)(235.95269044,418.25209585)(236.02269653,418.30210297)
\curveto(236.06269033,418.32209578)(236.08269031,418.35709574)(236.08269653,418.40710297)
\curveto(236.08269031,418.45709564)(236.0926903,418.50709559)(236.11269653,418.55710297)
\lineto(236.11269653,418.70710297)
\curveto(236.12269027,418.73709536)(236.12769026,418.77209533)(236.12769653,418.81210297)
\lineto(236.12769653,418.93210297)
\lineto(236.12769653,420.97210297)
\curveto(236.12769026,421.08209302)(236.12269027,421.2020929)(236.11269653,421.33210297)
\curveto(236.11269028,421.47209263)(236.13769025,421.57709252)(236.18769653,421.64710297)
\curveto(236.22769016,421.72709237)(236.30269009,421.77709232)(236.41269653,421.79710297)
\curveto(236.43268996,421.80709229)(236.45268994,421.80709229)(236.47269653,421.79710297)
\curveto(236.4926899,421.7970923)(236.51268988,421.8020923)(236.53269653,421.81210297)
\lineto(237.59769653,421.81210297)
\curveto(237.71768867,421.81209229)(237.82768856,421.80709229)(237.92769653,421.79710297)
\curveto(238.02768836,421.78709231)(238.10268829,421.74709235)(238.15269653,421.67710297)
\curveto(238.20268819,421.5970925)(238.22768816,421.49209261)(238.22769653,421.36210297)
\lineto(238.22769653,421.00210297)
\lineto(238.22769653,411.98710297)
\moveto(236.18769653,414.92710297)
\curveto(236.19769019,414.96709913)(236.19769019,415.00709909)(236.18769653,415.04710297)
\lineto(236.18769653,415.18210297)
\curveto(236.1876902,415.28209882)(236.18269021,415.38209872)(236.17269653,415.48210297)
\curveto(236.16269023,415.58209852)(236.14769024,415.67209843)(236.12769653,415.75210297)
\curveto(236.10769028,415.86209824)(236.0876903,415.96209814)(236.06769653,416.05210297)
\curveto(236.05769033,416.14209796)(236.03269036,416.22709787)(235.99269653,416.30710297)
\curveto(235.85269054,416.66709743)(235.64769074,416.95209715)(235.37769653,417.16210297)
\curveto(235.11769127,417.37209673)(234.73769165,417.47709662)(234.23769653,417.47710297)
\curveto(234.17769221,417.47709662)(234.09769229,417.46709663)(233.99769653,417.44710297)
\curveto(233.91769247,417.42709667)(233.84269255,417.40709669)(233.77269653,417.38710297)
\curveto(233.71269268,417.37709672)(233.65269274,417.35709674)(233.59269653,417.32710297)
\curveto(233.32269307,417.21709688)(233.11269328,417.04709705)(232.96269653,416.81710297)
\curveto(232.81269358,416.58709751)(232.6926937,416.32709777)(232.60269653,416.03710297)
\curveto(232.57269382,415.93709816)(232.55269384,415.83709826)(232.54269653,415.73710297)
\curveto(232.53269386,415.63709846)(232.51269388,415.53209857)(232.48269653,415.42210297)
\lineto(232.48269653,415.21210297)
\curveto(232.46269393,415.12209898)(232.45769393,414.9970991)(232.46769653,414.83710297)
\curveto(232.47769391,414.68709941)(232.4926939,414.57709952)(232.51269653,414.50710297)
\lineto(232.51269653,414.41710297)
\curveto(232.52269387,414.3970997)(232.52769386,414.37709972)(232.52769653,414.35710297)
\curveto(232.54769384,414.27709982)(232.56269383,414.2020999)(232.57269653,414.13210297)
\curveto(232.5926938,414.06210004)(232.61269378,413.98710011)(232.63269653,413.90710297)
\curveto(232.80269359,413.38710071)(233.0926933,413.0021011)(233.50269653,412.75210297)
\curveto(233.63269276,412.66210144)(233.81269258,412.59210151)(234.04269653,412.54210297)
\curveto(234.08269231,412.53210157)(234.14269225,412.52710157)(234.22269653,412.52710297)
\curveto(234.25269214,412.51710158)(234.29769209,412.50710159)(234.35769653,412.49710297)
\curveto(234.42769196,412.4971016)(234.48269191,412.5021016)(234.52269653,412.51210297)
\curveto(234.60269179,412.53210157)(234.68269171,412.54710155)(234.76269653,412.55710297)
\curveto(234.84269155,412.56710153)(234.92269147,412.58710151)(235.00269653,412.61710297)
\curveto(235.25269114,412.72710137)(235.45269094,412.86710123)(235.60269653,413.03710297)
\curveto(235.75269064,413.20710089)(235.88269051,413.42210068)(235.99269653,413.68210297)
\curveto(236.03269036,413.77210033)(236.06269033,413.86210024)(236.08269653,413.95210297)
\curveto(236.10269029,414.05210005)(236.12269027,414.15709994)(236.14269653,414.26710297)
\curveto(236.15269024,414.31709978)(236.15269024,414.36209974)(236.14269653,414.40210297)
\curveto(236.14269025,414.45209965)(236.15269024,414.5020996)(236.17269653,414.55210297)
\curveto(236.18269021,414.58209952)(236.1876902,414.61709948)(236.18769653,414.65710297)
\lineto(236.18769653,414.79210297)
\lineto(236.18769653,414.92710297)
}
}
{
\newrgbcolor{curcolor}{0 0 0}
\pscustom[linestyle=none,fillstyle=solid,fillcolor=curcolor]
{
\newpath
\moveto(247.17261841,415.07710297)
\curveto(247.19261024,414.9970991)(247.19261024,414.90709919)(247.17261841,414.80710297)
\curveto(247.15261028,414.70709939)(247.11761032,414.64209946)(247.06761841,414.61210297)
\curveto(247.01761042,414.57209953)(246.94261049,414.54209956)(246.84261841,414.52210297)
\curveto(246.75261068,414.51209959)(246.64761079,414.5020996)(246.52761841,414.49210297)
\lineto(246.18261841,414.49210297)
\curveto(246.07261136,414.5020996)(245.97261146,414.50709959)(245.88261841,414.50710297)
\lineto(242.22261841,414.50710297)
\lineto(242.01261841,414.50710297)
\curveto(241.95261548,414.50709959)(241.89761554,414.4970996)(241.84761841,414.47710297)
\curveto(241.76761567,414.43709966)(241.71761572,414.3970997)(241.69761841,414.35710297)
\curveto(241.67761576,414.33709976)(241.65761578,414.2970998)(241.63761841,414.23710297)
\curveto(241.61761582,414.18709991)(241.61261582,414.13709996)(241.62261841,414.08710297)
\curveto(241.64261579,414.02710007)(241.65261578,413.96710013)(241.65261841,413.90710297)
\curveto(241.66261577,413.85710024)(241.67761576,413.8021003)(241.69761841,413.74210297)
\curveto(241.77761566,413.5021006)(241.87261556,413.3021008)(241.98261841,413.14210297)
\curveto(242.10261533,412.99210111)(242.26261517,412.85710124)(242.46261841,412.73710297)
\curveto(242.54261489,412.68710141)(242.62261481,412.65210145)(242.70261841,412.63210297)
\curveto(242.79261464,412.62210148)(242.88261455,412.6021015)(242.97261841,412.57210297)
\curveto(243.05261438,412.55210155)(243.16261427,412.53710156)(243.30261841,412.52710297)
\curveto(243.44261399,412.51710158)(243.56261387,412.52210158)(243.66261841,412.54210297)
\lineto(243.79761841,412.54210297)
\curveto(243.89761354,412.56210154)(243.98761345,412.58210152)(244.06761841,412.60210297)
\curveto(244.15761328,412.63210147)(244.24261319,412.66210144)(244.32261841,412.69210297)
\curveto(244.42261301,412.74210136)(244.5326129,412.80710129)(244.65261841,412.88710297)
\curveto(244.78261265,412.96710113)(244.87761256,413.04710105)(244.93761841,413.12710297)
\curveto(244.98761245,413.1971009)(245.0376124,413.26210084)(245.08761841,413.32210297)
\curveto(245.14761229,413.39210071)(245.21761222,413.44210066)(245.29761841,413.47210297)
\curveto(245.39761204,413.52210058)(245.52261191,413.54210056)(245.67261841,413.53210297)
\lineto(246.10761841,413.53210297)
\lineto(246.28761841,413.53210297)
\curveto(246.35761108,413.54210056)(246.41761102,413.53710056)(246.46761841,413.51710297)
\lineto(246.61761841,413.51710297)
\curveto(246.71761072,413.4971006)(246.78761065,413.47210063)(246.82761841,413.44210297)
\curveto(246.86761057,413.42210068)(246.88761055,413.37710072)(246.88761841,413.30710297)
\curveto(246.89761054,413.23710086)(246.89261054,413.17710092)(246.87261841,413.12710297)
\curveto(246.82261061,412.98710111)(246.76761067,412.86210124)(246.70761841,412.75210297)
\curveto(246.64761079,412.64210146)(246.57761086,412.53210157)(246.49761841,412.42210297)
\curveto(246.27761116,412.09210201)(246.02761141,411.82710227)(245.74761841,411.62710297)
\curveto(245.46761197,411.42710267)(245.11761232,411.25710284)(244.69761841,411.11710297)
\curveto(244.58761285,411.07710302)(244.47761296,411.05210305)(244.36761841,411.04210297)
\curveto(244.25761318,411.03210307)(244.14261329,411.01210309)(244.02261841,410.98210297)
\curveto(243.98261345,410.97210313)(243.9376135,410.97210313)(243.88761841,410.98210297)
\curveto(243.84761359,410.98210312)(243.80761363,410.97710312)(243.76761841,410.96710297)
\lineto(243.60261841,410.96710297)
\curveto(243.55261388,410.94710315)(243.49261394,410.94210316)(243.42261841,410.95210297)
\curveto(243.36261407,410.95210315)(243.30761413,410.95710314)(243.25761841,410.96710297)
\curveto(243.17761426,410.97710312)(243.10761433,410.97710312)(243.04761841,410.96710297)
\curveto(242.98761445,410.95710314)(242.92261451,410.96210314)(242.85261841,410.98210297)
\curveto(242.80261463,411.0021031)(242.74761469,411.01210309)(242.68761841,411.01210297)
\curveto(242.62761481,411.01210309)(242.57261486,411.02210308)(242.52261841,411.04210297)
\curveto(242.41261502,411.06210304)(242.30261513,411.08710301)(242.19261841,411.11710297)
\curveto(242.08261535,411.13710296)(241.98261545,411.17210293)(241.89261841,411.22210297)
\curveto(241.78261565,411.26210284)(241.67761576,411.2971028)(241.57761841,411.32710297)
\curveto(241.48761595,411.36710273)(241.40261603,411.41210269)(241.32261841,411.46210297)
\curveto(241.00261643,411.66210244)(240.71761672,411.89210221)(240.46761841,412.15210297)
\curveto(240.21761722,412.42210168)(240.01261742,412.73210137)(239.85261841,413.08210297)
\curveto(239.80261763,413.19210091)(239.76261767,413.3021008)(239.73261841,413.41210297)
\curveto(239.70261773,413.53210057)(239.66261777,413.65210045)(239.61261841,413.77210297)
\curveto(239.60261783,413.81210029)(239.59761784,413.84710025)(239.59761841,413.87710297)
\curveto(239.59761784,413.91710018)(239.59261784,413.95710014)(239.58261841,413.99710297)
\curveto(239.54261789,414.11709998)(239.51761792,414.24709985)(239.50761841,414.38710297)
\lineto(239.47761841,414.80710297)
\curveto(239.47761796,414.85709924)(239.47261796,414.91209919)(239.46261841,414.97210297)
\curveto(239.46261797,415.03209907)(239.46761797,415.08709901)(239.47761841,415.13710297)
\lineto(239.47761841,415.31710297)
\lineto(239.52261841,415.67710297)
\curveto(239.56261787,415.84709825)(239.59761784,416.01209809)(239.62761841,416.17210297)
\curveto(239.65761778,416.33209777)(239.70261773,416.48209762)(239.76261841,416.62210297)
\curveto(240.19261724,417.66209644)(240.92261651,418.3970957)(241.95261841,418.82710297)
\curveto(242.09261534,418.88709521)(242.2326152,418.92709517)(242.37261841,418.94710297)
\curveto(242.52261491,418.97709512)(242.67761476,419.01209509)(242.83761841,419.05210297)
\curveto(242.91761452,419.06209504)(242.99261444,419.06709503)(243.06261841,419.06710297)
\curveto(243.1326143,419.06709503)(243.20761423,419.07209503)(243.28761841,419.08210297)
\curveto(243.79761364,419.09209501)(244.2326132,419.03209507)(244.59261841,418.90210297)
\curveto(244.96261247,418.78209532)(245.29261214,418.62209548)(245.58261841,418.42210297)
\curveto(245.67261176,418.36209574)(245.76261167,418.29209581)(245.85261841,418.21210297)
\curveto(245.94261149,418.14209596)(246.02261141,418.06709603)(246.09261841,417.98710297)
\curveto(246.12261131,417.93709616)(246.16261127,417.8970962)(246.21261841,417.86710297)
\curveto(246.29261114,417.75709634)(246.36761107,417.64209646)(246.43761841,417.52210297)
\curveto(246.50761093,417.41209669)(246.58261085,417.2970968)(246.66261841,417.17710297)
\curveto(246.71261072,417.08709701)(246.75261068,416.99209711)(246.78261841,416.89210297)
\curveto(246.82261061,416.8020973)(246.86261057,416.7020974)(246.90261841,416.59210297)
\curveto(246.95261048,416.46209764)(246.99261044,416.32709777)(247.02261841,416.18710297)
\curveto(247.05261038,416.04709805)(247.08761035,415.90709819)(247.12761841,415.76710297)
\curveto(247.14761029,415.68709841)(247.15261028,415.5970985)(247.14261841,415.49710297)
\curveto(247.14261029,415.40709869)(247.15261028,415.32209878)(247.17261841,415.24210297)
\lineto(247.17261841,415.07710297)
\moveto(244.92261841,415.96210297)
\curveto(244.99261244,416.06209804)(244.99761244,416.18209792)(244.93761841,416.32210297)
\curveto(244.88761255,416.47209763)(244.84761259,416.58209752)(244.81761841,416.65210297)
\curveto(244.67761276,416.92209718)(244.49261294,417.12709697)(244.26261841,417.26710297)
\curveto(244.0326134,417.41709668)(243.71261372,417.4970966)(243.30261841,417.50710297)
\curveto(243.27261416,417.48709661)(243.2376142,417.48209662)(243.19761841,417.49210297)
\curveto(243.15761428,417.5020966)(243.12261431,417.5020966)(243.09261841,417.49210297)
\curveto(243.04261439,417.47209663)(242.98761445,417.45709664)(242.92761841,417.44710297)
\curveto(242.86761457,417.44709665)(242.81261462,417.43709666)(242.76261841,417.41710297)
\curveto(242.32261511,417.27709682)(241.99761544,417.0020971)(241.78761841,416.59210297)
\curveto(241.76761567,416.55209755)(241.74261569,416.4970976)(241.71261841,416.42710297)
\curveto(241.69261574,416.36709773)(241.67761576,416.3020978)(241.66761841,416.23210297)
\curveto(241.65761578,416.17209793)(241.65761578,416.11209799)(241.66761841,416.05210297)
\curveto(241.68761575,415.99209811)(241.72261571,415.94209816)(241.77261841,415.90210297)
\curveto(241.85261558,415.85209825)(241.96261547,415.82709827)(242.10261841,415.82710297)
\lineto(242.50761841,415.82710297)
\lineto(244.17261841,415.82710297)
\lineto(244.60761841,415.82710297)
\curveto(244.76761267,415.83709826)(244.87261256,415.88209822)(244.92261841,415.96210297)
}
}
{
\newrgbcolor{curcolor}{0 0 0}
\pscustom[linestyle=none,fillstyle=solid,fillcolor=curcolor]
{
}
}
{
\newrgbcolor{curcolor}{0 0 0}
\pscustom[linestyle=none,fillstyle=solid,fillcolor=curcolor]
{
\newpath
\moveto(253.09105591,421.82710297)
\lineto(254.18605591,421.82710297)
\curveto(254.28605342,421.82709227)(254.38105333,421.82209228)(254.47105591,421.81210297)
\curveto(254.56105315,421.8020923)(254.63105308,421.77209233)(254.68105591,421.72210297)
\curveto(254.74105297,421.65209245)(254.77105294,421.55709254)(254.77105591,421.43710297)
\curveto(254.78105293,421.32709277)(254.78605292,421.21209289)(254.78605591,421.09210297)
\lineto(254.78605591,419.75710297)
\lineto(254.78605591,414.37210297)
\lineto(254.78605591,412.07710297)
\lineto(254.78605591,411.65710297)
\curveto(254.79605291,411.50710259)(254.77605293,411.39210271)(254.72605591,411.31210297)
\curveto(254.67605303,411.23210287)(254.58605312,411.17710292)(254.45605591,411.14710297)
\curveto(254.39605331,411.12710297)(254.32605338,411.12210298)(254.24605591,411.13210297)
\curveto(254.17605353,411.14210296)(254.1060536,411.14710295)(254.03605591,411.14710297)
\lineto(253.31605591,411.14710297)
\curveto(253.2060545,411.14710295)(253.1060546,411.15210295)(253.01605591,411.16210297)
\curveto(252.92605478,411.17210293)(252.85105486,411.2021029)(252.79105591,411.25210297)
\curveto(252.73105498,411.3021028)(252.69605501,411.37710272)(252.68605591,411.47710297)
\lineto(252.68605591,411.80710297)
\lineto(252.68605591,413.14210297)
\lineto(252.68605591,418.76710297)
\lineto(252.68605591,420.80710297)
\curveto(252.68605502,420.93709316)(252.68105503,421.09209301)(252.67105591,421.27210297)
\curveto(252.67105504,421.45209265)(252.69605501,421.58209252)(252.74605591,421.66210297)
\curveto(252.76605494,421.7020924)(252.79105492,421.73209237)(252.82105591,421.75210297)
\lineto(252.94105591,421.81210297)
\curveto(252.96105475,421.81209229)(252.98605472,421.81209229)(253.01605591,421.81210297)
\curveto(253.04605466,421.82209228)(253.07105464,421.82709227)(253.09105591,421.82710297)
}
}
{
\newrgbcolor{curcolor}{0 0 0}
\pscustom[linestyle=none,fillstyle=solid,fillcolor=curcolor]
{
\newpath
\moveto(264.21824341,415.31710297)
\curveto(264.23823484,415.25709884)(264.24823483,415.17209893)(264.24824341,415.06210297)
\curveto(264.24823483,414.95209915)(264.23823484,414.86709923)(264.21824341,414.80710297)
\lineto(264.21824341,414.65710297)
\curveto(264.19823488,414.57709952)(264.18823489,414.4970996)(264.18824341,414.41710297)
\curveto(264.19823488,414.33709976)(264.19323488,414.25709984)(264.17324341,414.17710297)
\curveto(264.15323492,414.10709999)(264.13823494,414.04210006)(264.12824341,413.98210297)
\curveto(264.11823496,413.92210018)(264.10823497,413.85710024)(264.09824341,413.78710297)
\curveto(264.05823502,413.67710042)(264.02323505,413.56210054)(263.99324341,413.44210297)
\curveto(263.96323511,413.33210077)(263.92323515,413.22710087)(263.87324341,413.12710297)
\curveto(263.66323541,412.64710145)(263.38823569,412.25710184)(263.04824341,411.95710297)
\curveto(262.70823637,411.65710244)(262.29823678,411.40710269)(261.81824341,411.20710297)
\curveto(261.69823738,411.15710294)(261.5732375,411.12210298)(261.44324341,411.10210297)
\curveto(261.32323775,411.07210303)(261.19823788,411.04210306)(261.06824341,411.01210297)
\curveto(261.01823806,410.99210311)(260.96323811,410.98210312)(260.90324341,410.98210297)
\curveto(260.84323823,410.98210312)(260.78823829,410.97710312)(260.73824341,410.96710297)
\lineto(260.63324341,410.96710297)
\curveto(260.60323847,410.95710314)(260.5732385,410.95210315)(260.54324341,410.95210297)
\curveto(260.49323858,410.94210316)(260.41323866,410.93710316)(260.30324341,410.93710297)
\curveto(260.19323888,410.92710317)(260.10823897,410.93210317)(260.04824341,410.95210297)
\lineto(259.89824341,410.95210297)
\curveto(259.84823923,410.96210314)(259.79323928,410.96710313)(259.73324341,410.96710297)
\curveto(259.68323939,410.95710314)(259.63323944,410.96210314)(259.58324341,410.98210297)
\curveto(259.54323953,410.99210311)(259.50323957,410.9971031)(259.46324341,410.99710297)
\curveto(259.43323964,410.9971031)(259.39323968,411.0021031)(259.34324341,411.01210297)
\curveto(259.24323983,411.04210306)(259.14323993,411.06710303)(259.04324341,411.08710297)
\curveto(258.94324013,411.10710299)(258.84824023,411.13710296)(258.75824341,411.17710297)
\curveto(258.63824044,411.21710288)(258.52324055,411.25710284)(258.41324341,411.29710297)
\curveto(258.31324076,411.33710276)(258.20824087,411.38710271)(258.09824341,411.44710297)
\curveto(257.74824133,411.65710244)(257.44824163,411.9021022)(257.19824341,412.18210297)
\curveto(256.94824213,412.46210164)(256.73824234,412.7971013)(256.56824341,413.18710297)
\curveto(256.51824256,413.27710082)(256.4782426,413.37210073)(256.44824341,413.47210297)
\curveto(256.42824265,413.57210053)(256.40324267,413.67710042)(256.37324341,413.78710297)
\curveto(256.35324272,413.83710026)(256.34324273,413.88210022)(256.34324341,413.92210297)
\curveto(256.34324273,413.96210014)(256.33324274,414.00710009)(256.31324341,414.05710297)
\curveto(256.29324278,414.13709996)(256.28324279,414.21709988)(256.28324341,414.29710297)
\curveto(256.28324279,414.38709971)(256.2732428,414.47209963)(256.25324341,414.55210297)
\curveto(256.24324283,414.6020995)(256.23824284,414.64709945)(256.23824341,414.68710297)
\lineto(256.23824341,414.82210297)
\curveto(256.21824286,414.88209922)(256.20824287,414.96709913)(256.20824341,415.07710297)
\curveto(256.21824286,415.18709891)(256.23324284,415.27209883)(256.25324341,415.33210297)
\lineto(256.25324341,415.43710297)
\curveto(256.26324281,415.48709861)(256.26324281,415.53709856)(256.25324341,415.58710297)
\curveto(256.25324282,415.64709845)(256.26324281,415.7020984)(256.28324341,415.75210297)
\curveto(256.29324278,415.8020983)(256.29824278,415.84709825)(256.29824341,415.88710297)
\curveto(256.29824278,415.93709816)(256.30824277,415.98709811)(256.32824341,416.03710297)
\curveto(256.36824271,416.16709793)(256.40324267,416.29209781)(256.43324341,416.41210297)
\curveto(256.46324261,416.54209756)(256.50324257,416.66709743)(256.55324341,416.78710297)
\curveto(256.73324234,417.1970969)(256.94824213,417.53709656)(257.19824341,417.80710297)
\curveto(257.44824163,418.08709601)(257.75324132,418.34209576)(258.11324341,418.57210297)
\curveto(258.21324086,418.62209548)(258.31824076,418.66709543)(258.42824341,418.70710297)
\curveto(258.53824054,418.74709535)(258.64824043,418.79209531)(258.75824341,418.84210297)
\curveto(258.88824019,418.89209521)(259.02324005,418.92709517)(259.16324341,418.94710297)
\curveto(259.30323977,418.96709513)(259.44823963,418.9970951)(259.59824341,419.03710297)
\curveto(259.6782394,419.04709505)(259.75323932,419.05209505)(259.82324341,419.05210297)
\curveto(259.89323918,419.05209505)(259.96323911,419.05709504)(260.03324341,419.06710297)
\curveto(260.61323846,419.07709502)(261.11323796,419.01709508)(261.53324341,418.88710297)
\curveto(261.96323711,418.75709534)(262.34323673,418.57709552)(262.67324341,418.34710297)
\curveto(262.78323629,418.26709583)(262.89323618,418.17709592)(263.00324341,418.07710297)
\curveto(263.12323595,417.98709611)(263.22323585,417.88709621)(263.30324341,417.77710297)
\curveto(263.38323569,417.67709642)(263.45323562,417.57709652)(263.51324341,417.47710297)
\curveto(263.58323549,417.37709672)(263.65323542,417.27209683)(263.72324341,417.16210297)
\curveto(263.79323528,417.05209705)(263.84823523,416.93209717)(263.88824341,416.80210297)
\curveto(263.92823515,416.68209742)(263.9732351,416.55209755)(264.02324341,416.41210297)
\curveto(264.05323502,416.33209777)(264.078235,416.24709785)(264.09824341,416.15710297)
\lineto(264.15824341,415.88710297)
\curveto(264.16823491,415.84709825)(264.1732349,415.80709829)(264.17324341,415.76710297)
\curveto(264.1732349,415.72709837)(264.1782349,415.68709841)(264.18824341,415.64710297)
\curveto(264.20823487,415.5970985)(264.21323486,415.54209856)(264.20324341,415.48210297)
\curveto(264.19323488,415.42209868)(264.19823488,415.36709873)(264.21824341,415.31710297)
\moveto(262.11824341,414.77710297)
\curveto(262.12823695,414.82709927)(262.13323694,414.8970992)(262.13324341,414.98710297)
\curveto(262.13323694,415.08709901)(262.12823695,415.16209894)(262.11824341,415.21210297)
\lineto(262.11824341,415.33210297)
\curveto(262.09823698,415.38209872)(262.08823699,415.43709866)(262.08824341,415.49710297)
\curveto(262.08823699,415.55709854)(262.08323699,415.61209849)(262.07324341,415.66210297)
\curveto(262.073237,415.7020984)(262.06823701,415.73209837)(262.05824341,415.75210297)
\lineto(261.99824341,415.99210297)
\curveto(261.98823709,416.08209802)(261.96823711,416.16709793)(261.93824341,416.24710297)
\curveto(261.82823725,416.50709759)(261.69823738,416.72709737)(261.54824341,416.90710297)
\curveto(261.39823768,417.097097)(261.19823788,417.24709685)(260.94824341,417.35710297)
\curveto(260.88823819,417.37709672)(260.82823825,417.39209671)(260.76824341,417.40210297)
\curveto(260.70823837,417.42209668)(260.64323843,417.44209666)(260.57324341,417.46210297)
\curveto(260.49323858,417.48209662)(260.40823867,417.48709661)(260.31824341,417.47710297)
\lineto(260.04824341,417.47710297)
\curveto(260.01823906,417.45709664)(259.98323909,417.44709665)(259.94324341,417.44710297)
\curveto(259.90323917,417.45709664)(259.86823921,417.45709664)(259.83824341,417.44710297)
\lineto(259.62824341,417.38710297)
\curveto(259.56823951,417.37709672)(259.51323956,417.35709674)(259.46324341,417.32710297)
\curveto(259.21323986,417.21709688)(259.00824007,417.05709704)(258.84824341,416.84710297)
\curveto(258.69824038,416.64709745)(258.5782405,416.41209769)(258.48824341,416.14210297)
\curveto(258.45824062,416.04209806)(258.43324064,415.93709816)(258.41324341,415.82710297)
\curveto(258.40324067,415.71709838)(258.38824069,415.60709849)(258.36824341,415.49710297)
\curveto(258.35824072,415.44709865)(258.35324072,415.3970987)(258.35324341,415.34710297)
\lineto(258.35324341,415.19710297)
\curveto(258.33324074,415.12709897)(258.32324075,415.02209908)(258.32324341,414.88210297)
\curveto(258.33324074,414.74209936)(258.34824073,414.63709946)(258.36824341,414.56710297)
\lineto(258.36824341,414.43210297)
\curveto(258.38824069,414.35209975)(258.40324067,414.27209983)(258.41324341,414.19210297)
\curveto(258.42324065,414.12209998)(258.43824064,414.04710005)(258.45824341,413.96710297)
\curveto(258.55824052,413.66710043)(258.66324041,413.42210068)(258.77324341,413.23210297)
\curveto(258.89324018,413.05210105)(259.07824,412.88710121)(259.32824341,412.73710297)
\curveto(259.39823968,412.68710141)(259.4732396,412.64710145)(259.55324341,412.61710297)
\curveto(259.64323943,412.58710151)(259.73323934,412.56210154)(259.82324341,412.54210297)
\curveto(259.86323921,412.53210157)(259.89823918,412.52710157)(259.92824341,412.52710297)
\curveto(259.95823912,412.53710156)(259.99323908,412.53710156)(260.03324341,412.52710297)
\lineto(260.15324341,412.49710297)
\curveto(260.20323887,412.4971016)(260.24823883,412.5021016)(260.28824341,412.51210297)
\lineto(260.40824341,412.51210297)
\curveto(260.48823859,412.53210157)(260.56823851,412.54710155)(260.64824341,412.55710297)
\curveto(260.72823835,412.56710153)(260.80323827,412.58710151)(260.87324341,412.61710297)
\curveto(261.13323794,412.71710138)(261.34323773,412.85210125)(261.50324341,413.02210297)
\curveto(261.66323741,413.19210091)(261.79823728,413.4021007)(261.90824341,413.65210297)
\curveto(261.94823713,413.75210035)(261.9782371,413.85210025)(261.99824341,413.95210297)
\curveto(262.01823706,414.05210005)(262.04323703,414.15709994)(262.07324341,414.26710297)
\curveto(262.08323699,414.30709979)(262.08823699,414.34209976)(262.08824341,414.37210297)
\curveto(262.08823699,414.41209969)(262.09323698,414.45209965)(262.10324341,414.49210297)
\lineto(262.10324341,414.62710297)
\curveto(262.10323697,414.67709942)(262.10823697,414.72709937)(262.11824341,414.77710297)
}
}
{
\newrgbcolor{curcolor}{0 0 0}
\pscustom[linestyle=none,fillstyle=solid,fillcolor=curcolor]
{
\newpath
\moveto(268.58816528,419.08210297)
\curveto(269.33816078,419.102095)(269.98816013,419.01709508)(270.53816528,418.82710297)
\curveto(271.09815902,418.64709545)(271.5231586,418.33209577)(271.81316528,417.88210297)
\curveto(271.88315824,417.77209633)(271.94315818,417.65709644)(271.99316528,417.53710297)
\curveto(272.05315807,417.42709667)(272.10315802,417.3020968)(272.14316528,417.16210297)
\curveto(272.16315796,417.102097)(272.17315795,417.03709706)(272.17316528,416.96710297)
\curveto(272.17315795,416.8970972)(272.16315796,416.83709726)(272.14316528,416.78710297)
\curveto(272.10315802,416.72709737)(272.04815807,416.68709741)(271.97816528,416.66710297)
\curveto(271.92815819,416.64709745)(271.86815825,416.63709746)(271.79816528,416.63710297)
\lineto(271.58816528,416.63710297)
\lineto(270.92816528,416.63710297)
\curveto(270.85815926,416.63709746)(270.78815933,416.63209747)(270.71816528,416.62210297)
\curveto(270.64815947,416.62209748)(270.58315954,416.63209747)(270.52316528,416.65210297)
\curveto(270.4231597,416.67209743)(270.34815977,416.71209739)(270.29816528,416.77210297)
\curveto(270.24815987,416.83209727)(270.20315992,416.89209721)(270.16316528,416.95210297)
\lineto(270.04316528,417.16210297)
\curveto(270.01316011,417.24209686)(269.96316016,417.30709679)(269.89316528,417.35710297)
\curveto(269.79316033,417.43709666)(269.69316043,417.4970966)(269.59316528,417.53710297)
\curveto(269.50316062,417.57709652)(269.38816073,417.61209649)(269.24816528,417.64210297)
\curveto(269.17816094,417.66209644)(269.07316105,417.67709642)(268.93316528,417.68710297)
\curveto(268.80316132,417.6970964)(268.70316142,417.69209641)(268.63316528,417.67210297)
\lineto(268.52816528,417.67210297)
\lineto(268.37816528,417.64210297)
\curveto(268.33816178,417.64209646)(268.29316183,417.63709646)(268.24316528,417.62710297)
\curveto(268.07316205,417.57709652)(267.93316219,417.50709659)(267.82316528,417.41710297)
\curveto(267.7231624,417.33709676)(267.65316247,417.21209689)(267.61316528,417.04210297)
\curveto(267.59316253,416.97209713)(267.59316253,416.90709719)(267.61316528,416.84710297)
\curveto(267.63316249,416.78709731)(267.65316247,416.73709736)(267.67316528,416.69710297)
\curveto(267.74316238,416.57709752)(267.8231623,416.48209762)(267.91316528,416.41210297)
\curveto(268.01316211,416.34209776)(268.12816199,416.28209782)(268.25816528,416.23210297)
\curveto(268.44816167,416.15209795)(268.65316147,416.08209802)(268.87316528,416.02210297)
\lineto(269.56316528,415.87210297)
\curveto(269.80316032,415.83209827)(270.03316009,415.78209832)(270.25316528,415.72210297)
\curveto(270.48315964,415.67209843)(270.69815942,415.60709849)(270.89816528,415.52710297)
\curveto(270.98815913,415.48709861)(271.07315905,415.45209865)(271.15316528,415.42210297)
\curveto(271.24315888,415.4020987)(271.32815879,415.36709873)(271.40816528,415.31710297)
\curveto(271.59815852,415.1970989)(271.76815835,415.06709903)(271.91816528,414.92710297)
\curveto(272.07815804,414.78709931)(272.20315792,414.61209949)(272.29316528,414.40210297)
\curveto(272.3231578,414.33209977)(272.34815777,414.26209984)(272.36816528,414.19210297)
\curveto(272.38815773,414.12209998)(272.40815771,414.04710005)(272.42816528,413.96710297)
\curveto(272.43815768,413.90710019)(272.44315768,413.81210029)(272.44316528,413.68210297)
\curveto(272.45315767,413.56210054)(272.45315767,413.46710063)(272.44316528,413.39710297)
\lineto(272.44316528,413.32210297)
\curveto(272.4231577,413.26210084)(272.40815771,413.2021009)(272.39816528,413.14210297)
\curveto(272.39815772,413.09210101)(272.39315773,413.04210106)(272.38316528,412.99210297)
\curveto(272.31315781,412.69210141)(272.20315792,412.42710167)(272.05316528,412.19710297)
\curveto(271.89315823,411.95710214)(271.69815842,411.76210234)(271.46816528,411.61210297)
\curveto(271.23815888,411.46210264)(270.97815914,411.33210277)(270.68816528,411.22210297)
\curveto(270.57815954,411.17210293)(270.45815966,411.13710296)(270.32816528,411.11710297)
\curveto(270.20815991,411.097103)(270.08816003,411.07210303)(269.96816528,411.04210297)
\curveto(269.87816024,411.02210308)(269.78316034,411.01210309)(269.68316528,411.01210297)
\curveto(269.59316053,411.0021031)(269.50316062,410.98710311)(269.41316528,410.96710297)
\lineto(269.14316528,410.96710297)
\curveto(269.08316104,410.94710315)(268.97816114,410.93710316)(268.82816528,410.93710297)
\curveto(268.68816143,410.93710316)(268.58816153,410.94710315)(268.52816528,410.96710297)
\curveto(268.49816162,410.96710313)(268.46316166,410.97210313)(268.42316528,410.98210297)
\lineto(268.31816528,410.98210297)
\curveto(268.19816192,411.0021031)(268.07816204,411.01710308)(267.95816528,411.02710297)
\curveto(267.83816228,411.03710306)(267.7231624,411.05710304)(267.61316528,411.08710297)
\curveto(267.2231629,411.1971029)(266.87816324,411.32210278)(266.57816528,411.46210297)
\curveto(266.27816384,411.61210249)(266.0231641,411.83210227)(265.81316528,412.12210297)
\curveto(265.67316445,412.31210179)(265.55316457,412.53210157)(265.45316528,412.78210297)
\curveto(265.43316469,412.84210126)(265.41316471,412.92210118)(265.39316528,413.02210297)
\curveto(265.37316475,413.07210103)(265.35816476,413.14210096)(265.34816528,413.23210297)
\curveto(265.33816478,413.32210078)(265.34316478,413.3971007)(265.36316528,413.45710297)
\curveto(265.39316473,413.52710057)(265.44316468,413.57710052)(265.51316528,413.60710297)
\curveto(265.56316456,413.62710047)(265.6231645,413.63710046)(265.69316528,413.63710297)
\lineto(265.91816528,413.63710297)
\lineto(266.62316528,413.63710297)
\lineto(266.86316528,413.63710297)
\curveto(266.94316318,413.63710046)(267.01316311,413.62710047)(267.07316528,413.60710297)
\curveto(267.18316294,413.56710053)(267.25316287,413.5021006)(267.28316528,413.41210297)
\curveto(267.3231628,413.32210078)(267.36816275,413.22710087)(267.41816528,413.12710297)
\curveto(267.43816268,413.07710102)(267.47316265,413.01210109)(267.52316528,412.93210297)
\curveto(267.58316254,412.85210125)(267.63316249,412.8021013)(267.67316528,412.78210297)
\curveto(267.79316233,412.68210142)(267.90816221,412.6021015)(268.01816528,412.54210297)
\curveto(268.12816199,412.49210161)(268.26816185,412.44210166)(268.43816528,412.39210297)
\curveto(268.48816163,412.37210173)(268.53816158,412.36210174)(268.58816528,412.36210297)
\curveto(268.63816148,412.37210173)(268.68816143,412.37210173)(268.73816528,412.36210297)
\curveto(268.8181613,412.34210176)(268.90316122,412.33210177)(268.99316528,412.33210297)
\curveto(269.09316103,412.34210176)(269.17816094,412.35710174)(269.24816528,412.37710297)
\curveto(269.29816082,412.38710171)(269.34316078,412.39210171)(269.38316528,412.39210297)
\curveto(269.43316069,412.39210171)(269.48316064,412.4021017)(269.53316528,412.42210297)
\curveto(269.67316045,412.47210163)(269.79816032,412.53210157)(269.90816528,412.60210297)
\curveto(270.02816009,412.67210143)(270.12316,412.76210134)(270.19316528,412.87210297)
\curveto(270.24315988,412.95210115)(270.28315984,413.07710102)(270.31316528,413.24710297)
\curveto(270.33315979,413.31710078)(270.33315979,413.38210072)(270.31316528,413.44210297)
\curveto(270.29315983,413.5021006)(270.27315985,413.55210055)(270.25316528,413.59210297)
\curveto(270.18315994,413.73210037)(270.09316003,413.83710026)(269.98316528,413.90710297)
\curveto(269.88316024,413.97710012)(269.76316036,414.04210006)(269.62316528,414.10210297)
\curveto(269.43316069,414.18209992)(269.23316089,414.24709985)(269.02316528,414.29710297)
\curveto(268.81316131,414.34709975)(268.60316152,414.4020997)(268.39316528,414.46210297)
\curveto(268.31316181,414.48209962)(268.22816189,414.4970996)(268.13816528,414.50710297)
\curveto(268.05816206,414.51709958)(267.97816214,414.53209957)(267.89816528,414.55210297)
\curveto(267.57816254,414.64209946)(267.27316285,414.72709937)(266.98316528,414.80710297)
\curveto(266.69316343,414.8970992)(266.42816369,415.02709907)(266.18816528,415.19710297)
\curveto(265.90816421,415.3970987)(265.70316442,415.66709843)(265.57316528,416.00710297)
\curveto(265.55316457,416.07709802)(265.53316459,416.17209793)(265.51316528,416.29210297)
\curveto(265.49316463,416.36209774)(265.47816464,416.44709765)(265.46816528,416.54710297)
\curveto(265.45816466,416.64709745)(265.46316466,416.73709736)(265.48316528,416.81710297)
\curveto(265.50316462,416.86709723)(265.50816461,416.90709719)(265.49816528,416.93710297)
\curveto(265.48816463,416.97709712)(265.49316463,417.02209708)(265.51316528,417.07210297)
\curveto(265.53316459,417.18209692)(265.55316457,417.28209682)(265.57316528,417.37210297)
\curveto(265.60316452,417.47209663)(265.63816448,417.56709653)(265.67816528,417.65710297)
\curveto(265.80816431,417.94709615)(265.98816413,418.18209592)(266.21816528,418.36210297)
\curveto(266.44816367,418.54209556)(266.70816341,418.68709541)(266.99816528,418.79710297)
\curveto(267.10816301,418.84709525)(267.2231629,418.88209522)(267.34316528,418.90210297)
\curveto(267.46316266,418.93209517)(267.58816253,418.96209514)(267.71816528,418.99210297)
\curveto(267.77816234,419.01209509)(267.83816228,419.02209508)(267.89816528,419.02210297)
\lineto(268.07816528,419.05210297)
\curveto(268.15816196,419.06209504)(268.24316188,419.06709503)(268.33316528,419.06710297)
\curveto(268.4231617,419.06709503)(268.50816161,419.07209503)(268.58816528,419.08210297)
}
}
{
\newrgbcolor{curcolor}{0 0 0}
\pscustom[linestyle=none,fillstyle=solid,fillcolor=curcolor]
{
}
}
{
\newrgbcolor{curcolor}{0 0 0}
\pscustom[linestyle=none,fillstyle=solid,fillcolor=curcolor]
{
\newpath
\moveto(278.25496216,418.85710297)
\lineto(279.37996216,418.85710297)
\curveto(279.48995972,418.85709524)(279.58995962,418.85209525)(279.67996216,418.84210297)
\curveto(279.76995944,418.83209527)(279.83495938,418.7970953)(279.87496216,418.73710297)
\curveto(279.92495929,418.67709542)(279.95495926,418.59209551)(279.96496216,418.48210297)
\curveto(279.97495924,418.38209572)(279.97995923,418.27709582)(279.97996216,418.16710297)
\lineto(279.97996216,417.11710297)
\lineto(279.97996216,414.88210297)
\curveto(279.97995923,414.52209958)(279.99495922,414.18209992)(280.02496216,413.86210297)
\curveto(280.05495916,413.54210056)(280.14495907,413.27710082)(280.29496216,413.06710297)
\curveto(280.43495878,412.85710124)(280.65995855,412.70710139)(280.96996216,412.61710297)
\curveto(281.01995819,412.60710149)(281.05995815,412.6021015)(281.08996216,412.60210297)
\curveto(281.12995808,412.6021015)(281.17495804,412.5971015)(281.22496216,412.58710297)
\curveto(281.27495794,412.57710152)(281.32995788,412.57210153)(281.38996216,412.57210297)
\curveto(281.44995776,412.57210153)(281.49495772,412.57710152)(281.52496216,412.58710297)
\curveto(281.57495764,412.60710149)(281.6149576,412.61210149)(281.64496216,412.60210297)
\curveto(281.68495753,412.59210151)(281.72495749,412.5971015)(281.76496216,412.61710297)
\curveto(281.97495724,412.66710143)(282.13995707,412.73210137)(282.25996216,412.81210297)
\curveto(282.43995677,412.92210118)(282.57995663,413.06210104)(282.67996216,413.23210297)
\curveto(282.78995642,413.41210069)(282.86495635,413.60710049)(282.90496216,413.81710297)
\curveto(282.95495626,414.03710006)(282.98495623,414.27709982)(282.99496216,414.53710297)
\curveto(283.00495621,414.80709929)(283.0099562,415.08709901)(283.00996216,415.37710297)
\lineto(283.00996216,417.19210297)
\lineto(283.00996216,418.16710297)
\lineto(283.00996216,418.43710297)
\curveto(283.0099562,418.53709556)(283.02995618,418.61709548)(283.06996216,418.67710297)
\curveto(283.11995609,418.76709533)(283.19495602,418.81709528)(283.29496216,418.82710297)
\curveto(283.39495582,418.84709525)(283.5149557,418.85709524)(283.65496216,418.85710297)
\lineto(284.44996216,418.85710297)
\lineto(284.73496216,418.85710297)
\curveto(284.82495439,418.85709524)(284.89995431,418.83709526)(284.95996216,418.79710297)
\curveto(285.03995417,418.74709535)(285.08495413,418.67209543)(285.09496216,418.57210297)
\curveto(285.10495411,418.47209563)(285.1099541,418.35709574)(285.10996216,418.22710297)
\lineto(285.10996216,417.08710297)
\lineto(285.10996216,412.87210297)
\lineto(285.10996216,411.80710297)
\lineto(285.10996216,411.50710297)
\curveto(285.1099541,411.40710269)(285.08995412,411.33210277)(285.04996216,411.28210297)
\curveto(284.99995421,411.2021029)(284.92495429,411.15710294)(284.82496216,411.14710297)
\curveto(284.72495449,411.13710296)(284.61995459,411.13210297)(284.50996216,411.13210297)
\lineto(283.69996216,411.13210297)
\curveto(283.58995562,411.13210297)(283.48995572,411.13710296)(283.39996216,411.14710297)
\curveto(283.31995589,411.15710294)(283.25495596,411.1971029)(283.20496216,411.26710297)
\curveto(283.18495603,411.2971028)(283.16495605,411.34210276)(283.14496216,411.40210297)
\curveto(283.13495608,411.46210264)(283.11995609,411.52210258)(283.09996216,411.58210297)
\curveto(283.08995612,411.64210246)(283.07495614,411.6971024)(283.05496216,411.74710297)
\curveto(283.03495618,411.7971023)(283.00495621,411.82710227)(282.96496216,411.83710297)
\curveto(282.94495627,411.85710224)(282.91995629,411.86210224)(282.88996216,411.85210297)
\curveto(282.85995635,411.84210226)(282.83495638,411.83210227)(282.81496216,411.82210297)
\curveto(282.74495647,411.78210232)(282.68495653,411.73710236)(282.63496216,411.68710297)
\curveto(282.58495663,411.63710246)(282.52995668,411.59210251)(282.46996216,411.55210297)
\curveto(282.42995678,411.52210258)(282.38995682,411.48710261)(282.34996216,411.44710297)
\curveto(282.31995689,411.41710268)(282.27995693,411.38710271)(282.22996216,411.35710297)
\curveto(281.99995721,411.21710288)(281.72995748,411.10710299)(281.41996216,411.02710297)
\curveto(281.34995786,411.00710309)(281.27995793,410.9971031)(281.20996216,410.99710297)
\curveto(281.13995807,410.98710311)(281.06495815,410.97210313)(280.98496216,410.95210297)
\curveto(280.94495827,410.94210316)(280.89995831,410.94210316)(280.84996216,410.95210297)
\curveto(280.8099584,410.95210315)(280.76995844,410.94710315)(280.72996216,410.93710297)
\curveto(280.69995851,410.92710317)(280.63495858,410.92710317)(280.53496216,410.93710297)
\curveto(280.44495877,410.93710316)(280.38495883,410.94210316)(280.35496216,410.95210297)
\curveto(280.30495891,410.95210315)(280.25495896,410.95710314)(280.20496216,410.96710297)
\lineto(280.05496216,410.96710297)
\curveto(279.93495928,410.9971031)(279.81995939,411.02210308)(279.70996216,411.04210297)
\curveto(279.59995961,411.06210304)(279.48995972,411.09210301)(279.37996216,411.13210297)
\curveto(279.32995988,411.15210295)(279.28495993,411.16710293)(279.24496216,411.17710297)
\curveto(279.21496,411.1971029)(279.17496004,411.21710288)(279.12496216,411.23710297)
\curveto(278.77496044,411.42710267)(278.49496072,411.69210241)(278.28496216,412.03210297)
\curveto(278.15496106,412.24210186)(278.05996115,412.49210161)(277.99996216,412.78210297)
\curveto(277.93996127,413.08210102)(277.89996131,413.3971007)(277.87996216,413.72710297)
\curveto(277.86996134,414.06710003)(277.86496135,414.41209969)(277.86496216,414.76210297)
\curveto(277.87496134,415.12209898)(277.87996133,415.47709862)(277.87996216,415.82710297)
\lineto(277.87996216,417.86710297)
\curveto(277.87996133,417.9970961)(277.87496134,418.14709595)(277.86496216,418.31710297)
\curveto(277.86496135,418.4970956)(277.88996132,418.62709547)(277.93996216,418.70710297)
\curveto(277.96996124,418.75709534)(278.02996118,418.8020953)(278.11996216,418.84210297)
\curveto(278.17996103,418.84209526)(278.22496099,418.84709525)(278.25496216,418.85710297)
}
}
{
\newrgbcolor{curcolor}{0 0 0}
\pscustom[linestyle=none,fillstyle=solid,fillcolor=curcolor]
{
\newpath
\moveto(289.71121216,419.08210297)
\curveto(290.46120766,419.102095)(291.11120701,419.01709508)(291.66121216,418.82710297)
\curveto(292.2212059,418.64709545)(292.64620547,418.33209577)(292.93621216,417.88210297)
\curveto(293.00620511,417.77209633)(293.06620505,417.65709644)(293.11621216,417.53710297)
\curveto(293.17620494,417.42709667)(293.22620489,417.3020968)(293.26621216,417.16210297)
\curveto(293.28620483,417.102097)(293.29620482,417.03709706)(293.29621216,416.96710297)
\curveto(293.29620482,416.8970972)(293.28620483,416.83709726)(293.26621216,416.78710297)
\curveto(293.22620489,416.72709737)(293.17120495,416.68709741)(293.10121216,416.66710297)
\curveto(293.05120507,416.64709745)(292.99120513,416.63709746)(292.92121216,416.63710297)
\lineto(292.71121216,416.63710297)
\lineto(292.05121216,416.63710297)
\curveto(291.98120614,416.63709746)(291.91120621,416.63209747)(291.84121216,416.62210297)
\curveto(291.77120635,416.62209748)(291.70620641,416.63209747)(291.64621216,416.65210297)
\curveto(291.54620657,416.67209743)(291.47120665,416.71209739)(291.42121216,416.77210297)
\curveto(291.37120675,416.83209727)(291.32620679,416.89209721)(291.28621216,416.95210297)
\lineto(291.16621216,417.16210297)
\curveto(291.13620698,417.24209686)(291.08620703,417.30709679)(291.01621216,417.35710297)
\curveto(290.9162072,417.43709666)(290.8162073,417.4970966)(290.71621216,417.53710297)
\curveto(290.62620749,417.57709652)(290.51120761,417.61209649)(290.37121216,417.64210297)
\curveto(290.30120782,417.66209644)(290.19620792,417.67709642)(290.05621216,417.68710297)
\curveto(289.92620819,417.6970964)(289.82620829,417.69209641)(289.75621216,417.67210297)
\lineto(289.65121216,417.67210297)
\lineto(289.50121216,417.64210297)
\curveto(289.46120866,417.64209646)(289.4162087,417.63709646)(289.36621216,417.62710297)
\curveto(289.19620892,417.57709652)(289.05620906,417.50709659)(288.94621216,417.41710297)
\curveto(288.84620927,417.33709676)(288.77620934,417.21209689)(288.73621216,417.04210297)
\curveto(288.7162094,416.97209713)(288.7162094,416.90709719)(288.73621216,416.84710297)
\curveto(288.75620936,416.78709731)(288.77620934,416.73709736)(288.79621216,416.69710297)
\curveto(288.86620925,416.57709752)(288.94620917,416.48209762)(289.03621216,416.41210297)
\curveto(289.13620898,416.34209776)(289.25120887,416.28209782)(289.38121216,416.23210297)
\curveto(289.57120855,416.15209795)(289.77620834,416.08209802)(289.99621216,416.02210297)
\lineto(290.68621216,415.87210297)
\curveto(290.92620719,415.83209827)(291.15620696,415.78209832)(291.37621216,415.72210297)
\curveto(291.60620651,415.67209843)(291.8212063,415.60709849)(292.02121216,415.52710297)
\curveto(292.11120601,415.48709861)(292.19620592,415.45209865)(292.27621216,415.42210297)
\curveto(292.36620575,415.4020987)(292.45120567,415.36709873)(292.53121216,415.31710297)
\curveto(292.7212054,415.1970989)(292.89120523,415.06709903)(293.04121216,414.92710297)
\curveto(293.20120492,414.78709931)(293.32620479,414.61209949)(293.41621216,414.40210297)
\curveto(293.44620467,414.33209977)(293.47120465,414.26209984)(293.49121216,414.19210297)
\curveto(293.51120461,414.12209998)(293.53120459,414.04710005)(293.55121216,413.96710297)
\curveto(293.56120456,413.90710019)(293.56620455,413.81210029)(293.56621216,413.68210297)
\curveto(293.57620454,413.56210054)(293.57620454,413.46710063)(293.56621216,413.39710297)
\lineto(293.56621216,413.32210297)
\curveto(293.54620457,413.26210084)(293.53120459,413.2021009)(293.52121216,413.14210297)
\curveto(293.5212046,413.09210101)(293.5162046,413.04210106)(293.50621216,412.99210297)
\curveto(293.43620468,412.69210141)(293.32620479,412.42710167)(293.17621216,412.19710297)
\curveto(293.0162051,411.95710214)(292.8212053,411.76210234)(292.59121216,411.61210297)
\curveto(292.36120576,411.46210264)(292.10120602,411.33210277)(291.81121216,411.22210297)
\curveto(291.70120642,411.17210293)(291.58120654,411.13710296)(291.45121216,411.11710297)
\curveto(291.33120679,411.097103)(291.21120691,411.07210303)(291.09121216,411.04210297)
\curveto(291.00120712,411.02210308)(290.90620721,411.01210309)(290.80621216,411.01210297)
\curveto(290.7162074,411.0021031)(290.62620749,410.98710311)(290.53621216,410.96710297)
\lineto(290.26621216,410.96710297)
\curveto(290.20620791,410.94710315)(290.10120802,410.93710316)(289.95121216,410.93710297)
\curveto(289.81120831,410.93710316)(289.71120841,410.94710315)(289.65121216,410.96710297)
\curveto(289.6212085,410.96710313)(289.58620853,410.97210313)(289.54621216,410.98210297)
\lineto(289.44121216,410.98210297)
\curveto(289.3212088,411.0021031)(289.20120892,411.01710308)(289.08121216,411.02710297)
\curveto(288.96120916,411.03710306)(288.84620927,411.05710304)(288.73621216,411.08710297)
\curveto(288.34620977,411.1971029)(288.00121012,411.32210278)(287.70121216,411.46210297)
\curveto(287.40121072,411.61210249)(287.14621097,411.83210227)(286.93621216,412.12210297)
\curveto(286.79621132,412.31210179)(286.67621144,412.53210157)(286.57621216,412.78210297)
\curveto(286.55621156,412.84210126)(286.53621158,412.92210118)(286.51621216,413.02210297)
\curveto(286.49621162,413.07210103)(286.48121164,413.14210096)(286.47121216,413.23210297)
\curveto(286.46121166,413.32210078)(286.46621165,413.3971007)(286.48621216,413.45710297)
\curveto(286.5162116,413.52710057)(286.56621155,413.57710052)(286.63621216,413.60710297)
\curveto(286.68621143,413.62710047)(286.74621137,413.63710046)(286.81621216,413.63710297)
\lineto(287.04121216,413.63710297)
\lineto(287.74621216,413.63710297)
\lineto(287.98621216,413.63710297)
\curveto(288.06621005,413.63710046)(288.13620998,413.62710047)(288.19621216,413.60710297)
\curveto(288.30620981,413.56710053)(288.37620974,413.5021006)(288.40621216,413.41210297)
\curveto(288.44620967,413.32210078)(288.49120963,413.22710087)(288.54121216,413.12710297)
\curveto(288.56120956,413.07710102)(288.59620952,413.01210109)(288.64621216,412.93210297)
\curveto(288.70620941,412.85210125)(288.75620936,412.8021013)(288.79621216,412.78210297)
\curveto(288.9162092,412.68210142)(289.03120909,412.6021015)(289.14121216,412.54210297)
\curveto(289.25120887,412.49210161)(289.39120873,412.44210166)(289.56121216,412.39210297)
\curveto(289.61120851,412.37210173)(289.66120846,412.36210174)(289.71121216,412.36210297)
\curveto(289.76120836,412.37210173)(289.81120831,412.37210173)(289.86121216,412.36210297)
\curveto(289.94120818,412.34210176)(290.02620809,412.33210177)(290.11621216,412.33210297)
\curveto(290.2162079,412.34210176)(290.30120782,412.35710174)(290.37121216,412.37710297)
\curveto(290.4212077,412.38710171)(290.46620765,412.39210171)(290.50621216,412.39210297)
\curveto(290.55620756,412.39210171)(290.60620751,412.4021017)(290.65621216,412.42210297)
\curveto(290.79620732,412.47210163)(290.9212072,412.53210157)(291.03121216,412.60210297)
\curveto(291.15120697,412.67210143)(291.24620687,412.76210134)(291.31621216,412.87210297)
\curveto(291.36620675,412.95210115)(291.40620671,413.07710102)(291.43621216,413.24710297)
\curveto(291.45620666,413.31710078)(291.45620666,413.38210072)(291.43621216,413.44210297)
\curveto(291.4162067,413.5021006)(291.39620672,413.55210055)(291.37621216,413.59210297)
\curveto(291.30620681,413.73210037)(291.2162069,413.83710026)(291.10621216,413.90710297)
\curveto(291.00620711,413.97710012)(290.88620723,414.04210006)(290.74621216,414.10210297)
\curveto(290.55620756,414.18209992)(290.35620776,414.24709985)(290.14621216,414.29710297)
\curveto(289.93620818,414.34709975)(289.72620839,414.4020997)(289.51621216,414.46210297)
\curveto(289.43620868,414.48209962)(289.35120877,414.4970996)(289.26121216,414.50710297)
\curveto(289.18120894,414.51709958)(289.10120902,414.53209957)(289.02121216,414.55210297)
\curveto(288.70120942,414.64209946)(288.39620972,414.72709937)(288.10621216,414.80710297)
\curveto(287.8162103,414.8970992)(287.55121057,415.02709907)(287.31121216,415.19710297)
\curveto(287.03121109,415.3970987)(286.82621129,415.66709843)(286.69621216,416.00710297)
\curveto(286.67621144,416.07709802)(286.65621146,416.17209793)(286.63621216,416.29210297)
\curveto(286.6162115,416.36209774)(286.60121152,416.44709765)(286.59121216,416.54710297)
\curveto(286.58121154,416.64709745)(286.58621153,416.73709736)(286.60621216,416.81710297)
\curveto(286.62621149,416.86709723)(286.63121149,416.90709719)(286.62121216,416.93710297)
\curveto(286.61121151,416.97709712)(286.6162115,417.02209708)(286.63621216,417.07210297)
\curveto(286.65621146,417.18209692)(286.67621144,417.28209682)(286.69621216,417.37210297)
\curveto(286.72621139,417.47209663)(286.76121136,417.56709653)(286.80121216,417.65710297)
\curveto(286.93121119,417.94709615)(287.11121101,418.18209592)(287.34121216,418.36210297)
\curveto(287.57121055,418.54209556)(287.83121029,418.68709541)(288.12121216,418.79710297)
\curveto(288.23120989,418.84709525)(288.34620977,418.88209522)(288.46621216,418.90210297)
\curveto(288.58620953,418.93209517)(288.71120941,418.96209514)(288.84121216,418.99210297)
\curveto(288.90120922,419.01209509)(288.96120916,419.02209508)(289.02121216,419.02210297)
\lineto(289.20121216,419.05210297)
\curveto(289.28120884,419.06209504)(289.36620875,419.06709503)(289.45621216,419.06710297)
\curveto(289.54620857,419.06709503)(289.63120849,419.07209503)(289.71121216,419.08210297)
}
}
{
\newrgbcolor{curcolor}{0 0 0}
\pscustom[linestyle=none,fillstyle=solid,fillcolor=curcolor]
{
\newpath
\moveto(295.21785278,418.85710297)
\lineto(296.34285278,418.85710297)
\curveto(296.45285035,418.85709524)(296.55285025,418.85209525)(296.64285278,418.84210297)
\curveto(296.73285007,418.83209527)(296.79785,418.7970953)(296.83785278,418.73710297)
\curveto(296.88784991,418.67709542)(296.91784988,418.59209551)(296.92785278,418.48210297)
\curveto(296.93784986,418.38209572)(296.94284986,418.27709582)(296.94285278,418.16710297)
\lineto(296.94285278,417.11710297)
\lineto(296.94285278,414.88210297)
\curveto(296.94284986,414.52209958)(296.95784984,414.18209992)(296.98785278,413.86210297)
\curveto(297.01784978,413.54210056)(297.10784969,413.27710082)(297.25785278,413.06710297)
\curveto(297.3978494,412.85710124)(297.62284918,412.70710139)(297.93285278,412.61710297)
\curveto(297.98284882,412.60710149)(298.02284878,412.6021015)(298.05285278,412.60210297)
\curveto(298.09284871,412.6021015)(298.13784866,412.5971015)(298.18785278,412.58710297)
\curveto(298.23784856,412.57710152)(298.29284851,412.57210153)(298.35285278,412.57210297)
\curveto(298.41284839,412.57210153)(298.45784834,412.57710152)(298.48785278,412.58710297)
\curveto(298.53784826,412.60710149)(298.57784822,412.61210149)(298.60785278,412.60210297)
\curveto(298.64784815,412.59210151)(298.68784811,412.5971015)(298.72785278,412.61710297)
\curveto(298.93784786,412.66710143)(299.1028477,412.73210137)(299.22285278,412.81210297)
\curveto(299.4028474,412.92210118)(299.54284726,413.06210104)(299.64285278,413.23210297)
\curveto(299.75284705,413.41210069)(299.82784697,413.60710049)(299.86785278,413.81710297)
\curveto(299.91784688,414.03710006)(299.94784685,414.27709982)(299.95785278,414.53710297)
\curveto(299.96784683,414.80709929)(299.97284683,415.08709901)(299.97285278,415.37710297)
\lineto(299.97285278,417.19210297)
\lineto(299.97285278,418.16710297)
\lineto(299.97285278,418.43710297)
\curveto(299.97284683,418.53709556)(299.99284681,418.61709548)(300.03285278,418.67710297)
\curveto(300.08284672,418.76709533)(300.15784664,418.81709528)(300.25785278,418.82710297)
\curveto(300.35784644,418.84709525)(300.47784632,418.85709524)(300.61785278,418.85710297)
\lineto(301.41285278,418.85710297)
\lineto(301.69785278,418.85710297)
\curveto(301.78784501,418.85709524)(301.86284494,418.83709526)(301.92285278,418.79710297)
\curveto(302.0028448,418.74709535)(302.04784475,418.67209543)(302.05785278,418.57210297)
\curveto(302.06784473,418.47209563)(302.07284473,418.35709574)(302.07285278,418.22710297)
\lineto(302.07285278,417.08710297)
\lineto(302.07285278,412.87210297)
\lineto(302.07285278,411.80710297)
\lineto(302.07285278,411.50710297)
\curveto(302.07284473,411.40710269)(302.05284475,411.33210277)(302.01285278,411.28210297)
\curveto(301.96284484,411.2021029)(301.88784491,411.15710294)(301.78785278,411.14710297)
\curveto(301.68784511,411.13710296)(301.58284522,411.13210297)(301.47285278,411.13210297)
\lineto(300.66285278,411.13210297)
\curveto(300.55284625,411.13210297)(300.45284635,411.13710296)(300.36285278,411.14710297)
\curveto(300.28284652,411.15710294)(300.21784658,411.1971029)(300.16785278,411.26710297)
\curveto(300.14784665,411.2971028)(300.12784667,411.34210276)(300.10785278,411.40210297)
\curveto(300.0978467,411.46210264)(300.08284672,411.52210258)(300.06285278,411.58210297)
\curveto(300.05284675,411.64210246)(300.03784676,411.6971024)(300.01785278,411.74710297)
\curveto(299.9978468,411.7971023)(299.96784683,411.82710227)(299.92785278,411.83710297)
\curveto(299.90784689,411.85710224)(299.88284692,411.86210224)(299.85285278,411.85210297)
\curveto(299.82284698,411.84210226)(299.797847,411.83210227)(299.77785278,411.82210297)
\curveto(299.70784709,411.78210232)(299.64784715,411.73710236)(299.59785278,411.68710297)
\curveto(299.54784725,411.63710246)(299.49284731,411.59210251)(299.43285278,411.55210297)
\curveto(299.39284741,411.52210258)(299.35284745,411.48710261)(299.31285278,411.44710297)
\curveto(299.28284752,411.41710268)(299.24284756,411.38710271)(299.19285278,411.35710297)
\curveto(298.96284784,411.21710288)(298.69284811,411.10710299)(298.38285278,411.02710297)
\curveto(298.31284849,411.00710309)(298.24284856,410.9971031)(298.17285278,410.99710297)
\curveto(298.1028487,410.98710311)(298.02784877,410.97210313)(297.94785278,410.95210297)
\curveto(297.90784889,410.94210316)(297.86284894,410.94210316)(297.81285278,410.95210297)
\curveto(297.77284903,410.95210315)(297.73284907,410.94710315)(297.69285278,410.93710297)
\curveto(297.66284914,410.92710317)(297.5978492,410.92710317)(297.49785278,410.93710297)
\curveto(297.40784939,410.93710316)(297.34784945,410.94210316)(297.31785278,410.95210297)
\curveto(297.26784953,410.95210315)(297.21784958,410.95710314)(297.16785278,410.96710297)
\lineto(297.01785278,410.96710297)
\curveto(296.8978499,410.9971031)(296.78285002,411.02210308)(296.67285278,411.04210297)
\curveto(296.56285024,411.06210304)(296.45285035,411.09210301)(296.34285278,411.13210297)
\curveto(296.29285051,411.15210295)(296.24785055,411.16710293)(296.20785278,411.17710297)
\curveto(296.17785062,411.1971029)(296.13785066,411.21710288)(296.08785278,411.23710297)
\curveto(295.73785106,411.42710267)(295.45785134,411.69210241)(295.24785278,412.03210297)
\curveto(295.11785168,412.24210186)(295.02285178,412.49210161)(294.96285278,412.78210297)
\curveto(294.9028519,413.08210102)(294.86285194,413.3971007)(294.84285278,413.72710297)
\curveto(294.83285197,414.06710003)(294.82785197,414.41209969)(294.82785278,414.76210297)
\curveto(294.83785196,415.12209898)(294.84285196,415.47709862)(294.84285278,415.82710297)
\lineto(294.84285278,417.86710297)
\curveto(294.84285196,417.9970961)(294.83785196,418.14709595)(294.82785278,418.31710297)
\curveto(294.82785197,418.4970956)(294.85285195,418.62709547)(294.90285278,418.70710297)
\curveto(294.93285187,418.75709534)(294.99285181,418.8020953)(295.08285278,418.84210297)
\curveto(295.14285166,418.84209526)(295.18785161,418.84709525)(295.21785278,418.85710297)
}
}
{
\newrgbcolor{curcolor}{0 0 0}
\pscustom[linestyle=none,fillstyle=solid,fillcolor=curcolor]
{
\newpath
\moveto(310.75410278,411.73210297)
\curveto(310.77409493,411.62210248)(310.78409492,411.51210259)(310.78410278,411.40210297)
\curveto(310.79409491,411.29210281)(310.74409496,411.21710288)(310.63410278,411.17710297)
\curveto(310.57409513,411.14710295)(310.5040952,411.13210297)(310.42410278,411.13210297)
\lineto(310.18410278,411.13210297)
\lineto(309.37410278,411.13210297)
\lineto(309.10410278,411.13210297)
\curveto(309.02409668,411.14210296)(308.95909675,411.16710293)(308.90910278,411.20710297)
\curveto(308.83909687,411.24710285)(308.78409692,411.3021028)(308.74410278,411.37210297)
\curveto(308.71409699,411.45210265)(308.66909704,411.51710258)(308.60910278,411.56710297)
\curveto(308.58909712,411.58710251)(308.56409714,411.6021025)(308.53410278,411.61210297)
\curveto(308.5040972,411.63210247)(308.46409724,411.63710246)(308.41410278,411.62710297)
\curveto(308.36409734,411.60710249)(308.31409739,411.58210252)(308.26410278,411.55210297)
\curveto(308.22409748,411.52210258)(308.17909753,411.4971026)(308.12910278,411.47710297)
\curveto(308.07909763,411.43710266)(308.02409768,411.4021027)(307.96410278,411.37210297)
\lineto(307.78410278,411.28210297)
\curveto(307.65409805,411.22210288)(307.51909819,411.17210293)(307.37910278,411.13210297)
\curveto(307.23909847,411.102103)(307.09409861,411.06710303)(306.94410278,411.02710297)
\curveto(306.87409883,411.00710309)(306.8040989,410.9971031)(306.73410278,410.99710297)
\curveto(306.67409903,410.98710311)(306.6090991,410.97710312)(306.53910278,410.96710297)
\lineto(306.44910278,410.96710297)
\curveto(306.41909929,410.95710314)(306.38909932,410.95210315)(306.35910278,410.95210297)
\lineto(306.19410278,410.95210297)
\curveto(306.09409961,410.93210317)(305.99409971,410.93210317)(305.89410278,410.95210297)
\lineto(305.75910278,410.95210297)
\curveto(305.68910002,410.97210313)(305.61910009,410.98210312)(305.54910278,410.98210297)
\curveto(305.48910022,410.97210313)(305.42910028,410.97710312)(305.36910278,410.99710297)
\curveto(305.26910044,411.01710308)(305.17410053,411.03710306)(305.08410278,411.05710297)
\curveto(304.99410071,411.06710303)(304.9091008,411.09210301)(304.82910278,411.13210297)
\curveto(304.53910117,411.24210286)(304.28910142,411.38210272)(304.07910278,411.55210297)
\curveto(303.87910183,411.73210237)(303.71910199,411.96710213)(303.59910278,412.25710297)
\curveto(303.56910214,412.32710177)(303.53910217,412.4021017)(303.50910278,412.48210297)
\curveto(303.48910222,412.56210154)(303.46910224,412.64710145)(303.44910278,412.73710297)
\curveto(303.42910228,412.78710131)(303.41910229,412.83710126)(303.41910278,412.88710297)
\curveto(303.42910228,412.93710116)(303.42910228,412.98710111)(303.41910278,413.03710297)
\curveto(303.4091023,413.06710103)(303.39910231,413.12710097)(303.38910278,413.21710297)
\curveto(303.38910232,413.31710078)(303.39410231,413.38710071)(303.40410278,413.42710297)
\curveto(303.42410228,413.52710057)(303.43410227,413.61210049)(303.43410278,413.68210297)
\lineto(303.52410278,414.01210297)
\curveto(303.55410215,414.13209997)(303.59410211,414.23709986)(303.64410278,414.32710297)
\curveto(303.81410189,414.61709948)(304.0091017,414.83709926)(304.22910278,414.98710297)
\curveto(304.44910126,415.13709896)(304.72910098,415.26709883)(305.06910278,415.37710297)
\curveto(305.19910051,415.42709867)(305.33410037,415.46209864)(305.47410278,415.48210297)
\curveto(305.61410009,415.5020986)(305.75409995,415.52709857)(305.89410278,415.55710297)
\curveto(305.97409973,415.57709852)(306.05909965,415.58709851)(306.14910278,415.58710297)
\curveto(306.23909947,415.5970985)(306.32909938,415.61209849)(306.41910278,415.63210297)
\curveto(306.48909922,415.65209845)(306.55909915,415.65709844)(306.62910278,415.64710297)
\curveto(306.69909901,415.64709845)(306.77409893,415.65709844)(306.85410278,415.67710297)
\curveto(306.92409878,415.6970984)(306.99409871,415.70709839)(307.06410278,415.70710297)
\curveto(307.13409857,415.70709839)(307.2090985,415.71709838)(307.28910278,415.73710297)
\curveto(307.49909821,415.78709831)(307.68909802,415.82709827)(307.85910278,415.85710297)
\curveto(308.03909767,415.8970982)(308.19909751,415.98709811)(308.33910278,416.12710297)
\curveto(308.42909728,416.21709788)(308.48909722,416.31709778)(308.51910278,416.42710297)
\curveto(308.52909718,416.45709764)(308.52909718,416.48209762)(308.51910278,416.50210297)
\curveto(308.51909719,416.52209758)(308.52409718,416.54209756)(308.53410278,416.56210297)
\curveto(308.54409716,416.58209752)(308.54909716,416.61209749)(308.54910278,416.65210297)
\lineto(308.54910278,416.74210297)
\lineto(308.51910278,416.86210297)
\curveto(308.51909719,416.9020972)(308.51409719,416.93709716)(308.50410278,416.96710297)
\curveto(308.4040973,417.26709683)(308.19409751,417.47209663)(307.87410278,417.58210297)
\curveto(307.78409792,417.61209649)(307.67409803,417.63209647)(307.54410278,417.64210297)
\curveto(307.42409828,417.66209644)(307.29909841,417.66709643)(307.16910278,417.65710297)
\curveto(307.03909867,417.65709644)(306.91409879,417.64709645)(306.79410278,417.62710297)
\curveto(306.67409903,417.60709649)(306.56909914,417.58209652)(306.47910278,417.55210297)
\curveto(306.41909929,417.53209657)(306.35909935,417.5020966)(306.29910278,417.46210297)
\curveto(306.24909946,417.43209667)(306.19909951,417.3970967)(306.14910278,417.35710297)
\curveto(306.09909961,417.31709678)(306.04409966,417.26209684)(305.98410278,417.19210297)
\curveto(305.93409977,417.12209698)(305.89909981,417.05709704)(305.87910278,416.99710297)
\curveto(305.82909988,416.8970972)(305.78409992,416.8020973)(305.74410278,416.71210297)
\curveto(305.71409999,416.62209748)(305.64410006,416.56209754)(305.53410278,416.53210297)
\curveto(305.45410025,416.51209759)(305.36910034,416.5020976)(305.27910278,416.50210297)
\lineto(305.00910278,416.50210297)
\lineto(304.43910278,416.50210297)
\curveto(304.38910132,416.5020976)(304.33910137,416.4970976)(304.28910278,416.48710297)
\curveto(304.23910147,416.48709761)(304.19410151,416.49209761)(304.15410278,416.50210297)
\lineto(304.01910278,416.50210297)
\curveto(303.99910171,416.51209759)(303.97410173,416.51709758)(303.94410278,416.51710297)
\curveto(303.91410179,416.51709758)(303.88910182,416.52709757)(303.86910278,416.54710297)
\curveto(303.78910192,416.56709753)(303.73410197,416.63209747)(303.70410278,416.74210297)
\curveto(303.69410201,416.79209731)(303.69410201,416.84209726)(303.70410278,416.89210297)
\curveto(303.71410199,416.94209716)(303.72410198,416.98709711)(303.73410278,417.02710297)
\curveto(303.76410194,417.13709696)(303.79410191,417.23709686)(303.82410278,417.32710297)
\curveto(303.86410184,417.42709667)(303.9091018,417.51709658)(303.95910278,417.59710297)
\lineto(304.04910278,417.74710297)
\lineto(304.13910278,417.89710297)
\curveto(304.21910149,418.00709609)(304.31910139,418.11209599)(304.43910278,418.21210297)
\curveto(304.45910125,418.22209588)(304.48910122,418.24709585)(304.52910278,418.28710297)
\curveto(304.57910113,418.32709577)(304.62410108,418.36209574)(304.66410278,418.39210297)
\curveto(304.704101,418.42209568)(304.74910096,418.45209565)(304.79910278,418.48210297)
\curveto(304.96910074,418.59209551)(305.14910056,418.67709542)(305.33910278,418.73710297)
\curveto(305.52910018,418.80709529)(305.72409998,418.87209523)(305.92410278,418.93210297)
\curveto(306.04409966,418.96209514)(306.16909954,418.98209512)(306.29910278,418.99210297)
\curveto(306.42909928,419.0020951)(306.55909915,419.02209508)(306.68910278,419.05210297)
\curveto(306.72909898,419.06209504)(306.78909892,419.06209504)(306.86910278,419.05210297)
\curveto(306.95909875,419.04209506)(307.01409869,419.04709505)(307.03410278,419.06710297)
\curveto(307.44409826,419.07709502)(307.83409787,419.06209504)(308.20410278,419.02210297)
\curveto(308.58409712,418.98209512)(308.92409678,418.90709519)(309.22410278,418.79710297)
\curveto(309.53409617,418.68709541)(309.79909591,418.53709556)(310.01910278,418.34710297)
\curveto(310.23909547,418.16709593)(310.4090953,417.93209617)(310.52910278,417.64210297)
\curveto(310.59909511,417.47209663)(310.63909507,417.27709682)(310.64910278,417.05710297)
\curveto(310.65909505,416.83709726)(310.66409504,416.61209749)(310.66410278,416.38210297)
\lineto(310.66410278,413.03710297)
\lineto(310.66410278,412.45210297)
\curveto(310.66409504,412.26210184)(310.68409502,412.08710201)(310.72410278,411.92710297)
\curveto(310.73409497,411.8971022)(310.73909497,411.86210224)(310.73910278,411.82210297)
\curveto(310.73909497,411.79210231)(310.74409496,411.76210234)(310.75410278,411.73210297)
\moveto(308.54910278,414.04210297)
\curveto(308.55909715,414.09210001)(308.56409714,414.14709995)(308.56410278,414.20710297)
\curveto(308.56409714,414.27709982)(308.55909715,414.33709976)(308.54910278,414.38710297)
\curveto(308.52909718,414.44709965)(308.51909719,414.5020996)(308.51910278,414.55210297)
\curveto(308.51909719,414.6020995)(308.49909721,414.64209946)(308.45910278,414.67210297)
\curveto(308.4090973,414.71209939)(308.33409737,414.73209937)(308.23410278,414.73210297)
\curveto(308.19409751,414.72209938)(308.15909755,414.71209939)(308.12910278,414.70210297)
\curveto(308.09909761,414.7020994)(308.06409764,414.6970994)(308.02410278,414.68710297)
\curveto(307.95409775,414.66709943)(307.87909783,414.65209945)(307.79910278,414.64210297)
\curveto(307.71909799,414.63209947)(307.63909807,414.61709948)(307.55910278,414.59710297)
\curveto(307.52909818,414.58709951)(307.48409822,414.58209952)(307.42410278,414.58210297)
\curveto(307.29409841,414.55209955)(307.16409854,414.53209957)(307.03410278,414.52210297)
\curveto(306.9040988,414.51209959)(306.77909893,414.48709961)(306.65910278,414.44710297)
\curveto(306.57909913,414.42709967)(306.5040992,414.40709969)(306.43410278,414.38710297)
\curveto(306.36409934,414.37709972)(306.29409941,414.35709974)(306.22410278,414.32710297)
\curveto(306.01409969,414.23709986)(305.83409987,414.1021)(305.68410278,413.92210297)
\curveto(305.54410016,413.74210036)(305.49410021,413.49210061)(305.53410278,413.17210297)
\curveto(305.55410015,413.0021011)(305.6091001,412.86210124)(305.69910278,412.75210297)
\curveto(305.76909994,412.64210146)(305.87409983,412.55210155)(306.01410278,412.48210297)
\curveto(306.15409955,412.42210168)(306.3040994,412.37710172)(306.46410278,412.34710297)
\curveto(306.63409907,412.31710178)(306.8090989,412.30710179)(306.98910278,412.31710297)
\curveto(307.17909853,412.33710176)(307.35409835,412.37210173)(307.51410278,412.42210297)
\curveto(307.77409793,412.5021016)(307.97909773,412.62710147)(308.12910278,412.79710297)
\curveto(308.27909743,412.97710112)(308.39409731,413.1971009)(308.47410278,413.45710297)
\curveto(308.49409721,413.52710057)(308.5040972,413.5971005)(308.50410278,413.66710297)
\curveto(308.51409719,413.74710035)(308.52909718,413.82710027)(308.54910278,413.90710297)
\lineto(308.54910278,414.04210297)
}
}
{
\newrgbcolor{curcolor}{0 0 0}
\pscustom[linestyle=none,fillstyle=solid,fillcolor=curcolor]
{
\newpath
\moveto(316.74238403,419.06710297)
\curveto(316.85237872,419.06709503)(316.94737862,419.05709504)(317.02738403,419.03710297)
\curveto(317.11737845,419.01709508)(317.18737838,418.97209513)(317.23738403,418.90210297)
\curveto(317.29737827,418.82209528)(317.32737824,418.68209542)(317.32738403,418.48210297)
\lineto(317.32738403,417.97210297)
\lineto(317.32738403,417.59710297)
\curveto(317.33737823,417.45709664)(317.32237825,417.34709675)(317.28238403,417.26710297)
\curveto(317.24237833,417.1970969)(317.18237839,417.15209695)(317.10238403,417.13210297)
\curveto(317.03237854,417.11209699)(316.94737862,417.102097)(316.84738403,417.10210297)
\curveto(316.75737881,417.102097)(316.65737891,417.10709699)(316.54738403,417.11710297)
\curveto(316.44737912,417.12709697)(316.35237922,417.12209698)(316.26238403,417.10210297)
\curveto(316.19237938,417.08209702)(316.12237945,417.06709703)(316.05238403,417.05710297)
\curveto(315.98237959,417.05709704)(315.91737965,417.04709705)(315.85738403,417.02710297)
\curveto(315.69737987,416.97709712)(315.53738003,416.9020972)(315.37738403,416.80210297)
\curveto(315.21738035,416.71209739)(315.09238048,416.60709749)(315.00238403,416.48710297)
\curveto(314.95238062,416.40709769)(314.89738067,416.32209778)(314.83738403,416.23210297)
\curveto(314.78738078,416.15209795)(314.73738083,416.06709803)(314.68738403,415.97710297)
\curveto(314.65738091,415.8970982)(314.62738094,415.81209829)(314.59738403,415.72210297)
\lineto(314.53738403,415.48210297)
\curveto(314.51738105,415.41209869)(314.50738106,415.33709876)(314.50738403,415.25710297)
\curveto(314.50738106,415.18709891)(314.49738107,415.11709898)(314.47738403,415.04710297)
\curveto(314.4673811,415.00709909)(314.46238111,414.96709913)(314.46238403,414.92710297)
\curveto(314.4723811,414.8970992)(314.4723811,414.86709923)(314.46238403,414.83710297)
\lineto(314.46238403,414.59710297)
\curveto(314.44238113,414.52709957)(314.43738113,414.44709965)(314.44738403,414.35710297)
\curveto(314.45738111,414.27709982)(314.46238111,414.1970999)(314.46238403,414.11710297)
\lineto(314.46238403,413.15710297)
\lineto(314.46238403,411.88210297)
\curveto(314.46238111,411.75210235)(314.45738111,411.63210247)(314.44738403,411.52210297)
\curveto(314.43738113,411.41210269)(314.40738116,411.32210278)(314.35738403,411.25210297)
\curveto(314.33738123,411.22210288)(314.30238127,411.1971029)(314.25238403,411.17710297)
\curveto(314.21238136,411.16710293)(314.1673814,411.15710294)(314.11738403,411.14710297)
\lineto(314.04238403,411.14710297)
\curveto(313.99238158,411.13710296)(313.93738163,411.13210297)(313.87738403,411.13210297)
\lineto(313.71238403,411.13210297)
\lineto(313.06738403,411.13210297)
\curveto(313.00738256,411.14210296)(312.94238263,411.14710295)(312.87238403,411.14710297)
\lineto(312.67738403,411.14710297)
\curveto(312.62738294,411.16710293)(312.57738299,411.18210292)(312.52738403,411.19210297)
\curveto(312.47738309,411.21210289)(312.44238313,411.24710285)(312.42238403,411.29710297)
\curveto(312.38238319,411.34710275)(312.35738321,411.41710268)(312.34738403,411.50710297)
\lineto(312.34738403,411.80710297)
\lineto(312.34738403,412.82710297)
\lineto(312.34738403,417.05710297)
\lineto(312.34738403,418.16710297)
\lineto(312.34738403,418.45210297)
\curveto(312.34738322,418.55209555)(312.3673832,418.63209547)(312.40738403,418.69210297)
\curveto(312.45738311,418.77209533)(312.53238304,418.82209528)(312.63238403,418.84210297)
\curveto(312.73238284,418.86209524)(312.85238272,418.87209523)(312.99238403,418.87210297)
\lineto(313.75738403,418.87210297)
\curveto(313.87738169,418.87209523)(313.98238159,418.86209524)(314.07238403,418.84210297)
\curveto(314.16238141,418.83209527)(314.23238134,418.78709531)(314.28238403,418.70710297)
\curveto(314.31238126,418.65709544)(314.32738124,418.58709551)(314.32738403,418.49710297)
\lineto(314.35738403,418.22710297)
\curveto(314.3673812,418.14709595)(314.38238119,418.07209603)(314.40238403,418.00210297)
\curveto(314.43238114,417.93209617)(314.48238109,417.8970962)(314.55238403,417.89710297)
\curveto(314.572381,417.91709618)(314.59238098,417.92709617)(314.61238403,417.92710297)
\curveto(314.63238094,417.92709617)(314.65238092,417.93709616)(314.67238403,417.95710297)
\curveto(314.73238084,418.00709609)(314.78238079,418.06209604)(314.82238403,418.12210297)
\curveto(314.8723807,418.19209591)(314.93238064,418.25209585)(315.00238403,418.30210297)
\curveto(315.04238053,418.33209577)(315.07738049,418.36209574)(315.10738403,418.39210297)
\curveto(315.13738043,418.43209567)(315.1723804,418.46709563)(315.21238403,418.49710297)
\lineto(315.48238403,418.67710297)
\curveto(315.58237999,418.73709536)(315.68237989,418.79209531)(315.78238403,418.84210297)
\curveto(315.88237969,418.88209522)(315.98237959,418.91709518)(316.08238403,418.94710297)
\lineto(316.41238403,419.03710297)
\curveto(316.44237913,419.04709505)(316.49737907,419.04709505)(316.57738403,419.03710297)
\curveto(316.6673789,419.03709506)(316.72237885,419.04709505)(316.74238403,419.06710297)
}
}
{
\newrgbcolor{curcolor}{0 0 0}
\pscustom[linestyle=none,fillstyle=solid,fillcolor=curcolor]
{
\newpath
\moveto(320.24746216,421.72210297)
\curveto(320.31745921,421.64209246)(320.35245917,421.52209258)(320.35246216,421.36210297)
\lineto(320.35246216,420.89710297)
\lineto(320.35246216,420.49210297)
\curveto(320.35245917,420.35209375)(320.31745921,420.25709384)(320.24746216,420.20710297)
\curveto(320.18745934,420.15709394)(320.10745942,420.12709397)(320.00746216,420.11710297)
\curveto(319.91745961,420.10709399)(319.81745971,420.102094)(319.70746216,420.10210297)
\lineto(318.86746216,420.10210297)
\curveto(318.75746077,420.102094)(318.65746087,420.10709399)(318.56746216,420.11710297)
\curveto(318.48746104,420.12709397)(318.41746111,420.15709394)(318.35746216,420.20710297)
\curveto(318.31746121,420.23709386)(318.28746124,420.29209381)(318.26746216,420.37210297)
\curveto(318.25746127,420.46209364)(318.24746128,420.55709354)(318.23746216,420.65710297)
\lineto(318.23746216,420.98710297)
\curveto(318.24746128,421.097093)(318.25246127,421.19209291)(318.25246216,421.27210297)
\lineto(318.25246216,421.48210297)
\curveto(318.26246126,421.55209255)(318.28246124,421.61209249)(318.31246216,421.66210297)
\curveto(318.33246119,421.7020924)(318.35746117,421.73209237)(318.38746216,421.75210297)
\lineto(318.50746216,421.81210297)
\curveto(318.527461,421.81209229)(318.55246097,421.81209229)(318.58246216,421.81210297)
\curveto(318.61246091,421.82209228)(318.63746089,421.82709227)(318.65746216,421.82710297)
\lineto(319.75246216,421.82710297)
\curveto(319.85245967,421.82709227)(319.94745958,421.82209228)(320.03746216,421.81210297)
\curveto(320.1274594,421.8020923)(320.19745933,421.77209233)(320.24746216,421.72210297)
\moveto(320.35246216,411.95710297)
\curveto(320.35245917,411.75710234)(320.34745918,411.58710251)(320.33746216,411.44710297)
\curveto(320.3274592,411.30710279)(320.23745929,411.21210289)(320.06746216,411.16210297)
\curveto(320.00745952,411.14210296)(319.94245958,411.13210297)(319.87246216,411.13210297)
\curveto(319.80245972,411.14210296)(319.7274598,411.14710295)(319.64746216,411.14710297)
\lineto(318.80746216,411.14710297)
\curveto(318.71746081,411.14710295)(318.6274609,411.15210295)(318.53746216,411.16210297)
\curveto(318.45746107,411.17210293)(318.39746113,411.2021029)(318.35746216,411.25210297)
\curveto(318.29746123,411.32210278)(318.26246126,411.40710269)(318.25246216,411.50710297)
\lineto(318.25246216,411.85210297)
\lineto(318.25246216,418.18210297)
\lineto(318.25246216,418.48210297)
\curveto(318.25246127,418.58209552)(318.27246125,418.66209544)(318.31246216,418.72210297)
\curveto(318.37246115,418.79209531)(318.45746107,418.83709526)(318.56746216,418.85710297)
\curveto(318.58746094,418.86709523)(318.61246091,418.86709523)(318.64246216,418.85710297)
\curveto(318.68246084,418.85709524)(318.71246081,418.86209524)(318.73246216,418.87210297)
\lineto(319.48246216,418.87210297)
\lineto(319.67746216,418.87210297)
\curveto(319.75745977,418.88209522)(319.8224597,418.88209522)(319.87246216,418.87210297)
\lineto(319.99246216,418.87210297)
\curveto(320.05245947,418.85209525)(320.10745942,418.83709526)(320.15746216,418.82710297)
\curveto(320.20745932,418.81709528)(320.24745928,418.78709531)(320.27746216,418.73710297)
\curveto(320.31745921,418.68709541)(320.33745919,418.61709548)(320.33746216,418.52710297)
\curveto(320.34745918,418.43709566)(320.35245917,418.34209576)(320.35246216,418.24210297)
\lineto(320.35246216,411.95710297)
}
}
{
\newrgbcolor{curcolor}{0 0 0}
\pscustom[linestyle=none,fillstyle=solid,fillcolor=curcolor]
{
\newpath
\moveto(329.78464966,415.31710297)
\curveto(329.80464109,415.25709884)(329.81464108,415.17209893)(329.81464966,415.06210297)
\curveto(329.81464108,414.95209915)(329.80464109,414.86709923)(329.78464966,414.80710297)
\lineto(329.78464966,414.65710297)
\curveto(329.76464113,414.57709952)(329.75464114,414.4970996)(329.75464966,414.41710297)
\curveto(329.76464113,414.33709976)(329.75964113,414.25709984)(329.73964966,414.17710297)
\curveto(329.71964117,414.10709999)(329.70464119,414.04210006)(329.69464966,413.98210297)
\curveto(329.68464121,413.92210018)(329.67464122,413.85710024)(329.66464966,413.78710297)
\curveto(329.62464127,413.67710042)(329.5896413,413.56210054)(329.55964966,413.44210297)
\curveto(329.52964136,413.33210077)(329.4896414,413.22710087)(329.43964966,413.12710297)
\curveto(329.22964166,412.64710145)(328.95464194,412.25710184)(328.61464966,411.95710297)
\curveto(328.27464262,411.65710244)(327.86464303,411.40710269)(327.38464966,411.20710297)
\curveto(327.26464363,411.15710294)(327.13964375,411.12210298)(327.00964966,411.10210297)
\curveto(326.889644,411.07210303)(326.76464413,411.04210306)(326.63464966,411.01210297)
\curveto(326.58464431,410.99210311)(326.52964436,410.98210312)(326.46964966,410.98210297)
\curveto(326.40964448,410.98210312)(326.35464454,410.97710312)(326.30464966,410.96710297)
\lineto(326.19964966,410.96710297)
\curveto(326.16964472,410.95710314)(326.13964475,410.95210315)(326.10964966,410.95210297)
\curveto(326.05964483,410.94210316)(325.97964491,410.93710316)(325.86964966,410.93710297)
\curveto(325.75964513,410.92710317)(325.67464522,410.93210317)(325.61464966,410.95210297)
\lineto(325.46464966,410.95210297)
\curveto(325.41464548,410.96210314)(325.35964553,410.96710313)(325.29964966,410.96710297)
\curveto(325.24964564,410.95710314)(325.19964569,410.96210314)(325.14964966,410.98210297)
\curveto(325.10964578,410.99210311)(325.06964582,410.9971031)(325.02964966,410.99710297)
\curveto(324.99964589,410.9971031)(324.95964593,411.0021031)(324.90964966,411.01210297)
\curveto(324.80964608,411.04210306)(324.70964618,411.06710303)(324.60964966,411.08710297)
\curveto(324.50964638,411.10710299)(324.41464648,411.13710296)(324.32464966,411.17710297)
\curveto(324.20464669,411.21710288)(324.0896468,411.25710284)(323.97964966,411.29710297)
\curveto(323.87964701,411.33710276)(323.77464712,411.38710271)(323.66464966,411.44710297)
\curveto(323.31464758,411.65710244)(323.01464788,411.9021022)(322.76464966,412.18210297)
\curveto(322.51464838,412.46210164)(322.30464859,412.7971013)(322.13464966,413.18710297)
\curveto(322.08464881,413.27710082)(322.04464885,413.37210073)(322.01464966,413.47210297)
\curveto(321.9946489,413.57210053)(321.96964892,413.67710042)(321.93964966,413.78710297)
\curveto(321.91964897,413.83710026)(321.90964898,413.88210022)(321.90964966,413.92210297)
\curveto(321.90964898,413.96210014)(321.89964899,414.00710009)(321.87964966,414.05710297)
\curveto(321.85964903,414.13709996)(321.84964904,414.21709988)(321.84964966,414.29710297)
\curveto(321.84964904,414.38709971)(321.83964905,414.47209963)(321.81964966,414.55210297)
\curveto(321.80964908,414.6020995)(321.80464909,414.64709945)(321.80464966,414.68710297)
\lineto(321.80464966,414.82210297)
\curveto(321.78464911,414.88209922)(321.77464912,414.96709913)(321.77464966,415.07710297)
\curveto(321.78464911,415.18709891)(321.79964909,415.27209883)(321.81964966,415.33210297)
\lineto(321.81964966,415.43710297)
\curveto(321.82964906,415.48709861)(321.82964906,415.53709856)(321.81964966,415.58710297)
\curveto(321.81964907,415.64709845)(321.82964906,415.7020984)(321.84964966,415.75210297)
\curveto(321.85964903,415.8020983)(321.86464903,415.84709825)(321.86464966,415.88710297)
\curveto(321.86464903,415.93709816)(321.87464902,415.98709811)(321.89464966,416.03710297)
\curveto(321.93464896,416.16709793)(321.96964892,416.29209781)(321.99964966,416.41210297)
\curveto(322.02964886,416.54209756)(322.06964882,416.66709743)(322.11964966,416.78710297)
\curveto(322.29964859,417.1970969)(322.51464838,417.53709656)(322.76464966,417.80710297)
\curveto(323.01464788,418.08709601)(323.31964757,418.34209576)(323.67964966,418.57210297)
\curveto(323.77964711,418.62209548)(323.88464701,418.66709543)(323.99464966,418.70710297)
\curveto(324.10464679,418.74709535)(324.21464668,418.79209531)(324.32464966,418.84210297)
\curveto(324.45464644,418.89209521)(324.5896463,418.92709517)(324.72964966,418.94710297)
\curveto(324.86964602,418.96709513)(325.01464588,418.9970951)(325.16464966,419.03710297)
\curveto(325.24464565,419.04709505)(325.31964557,419.05209505)(325.38964966,419.05210297)
\curveto(325.45964543,419.05209505)(325.52964536,419.05709504)(325.59964966,419.06710297)
\curveto(326.17964471,419.07709502)(326.67964421,419.01709508)(327.09964966,418.88710297)
\curveto(327.52964336,418.75709534)(327.90964298,418.57709552)(328.23964966,418.34710297)
\curveto(328.34964254,418.26709583)(328.45964243,418.17709592)(328.56964966,418.07710297)
\curveto(328.6896422,417.98709611)(328.7896421,417.88709621)(328.86964966,417.77710297)
\curveto(328.94964194,417.67709642)(329.01964187,417.57709652)(329.07964966,417.47710297)
\curveto(329.14964174,417.37709672)(329.21964167,417.27209683)(329.28964966,417.16210297)
\curveto(329.35964153,417.05209705)(329.41464148,416.93209717)(329.45464966,416.80210297)
\curveto(329.4946414,416.68209742)(329.53964135,416.55209755)(329.58964966,416.41210297)
\curveto(329.61964127,416.33209777)(329.64464125,416.24709785)(329.66464966,416.15710297)
\lineto(329.72464966,415.88710297)
\curveto(329.73464116,415.84709825)(329.73964115,415.80709829)(329.73964966,415.76710297)
\curveto(329.73964115,415.72709837)(329.74464115,415.68709841)(329.75464966,415.64710297)
\curveto(329.77464112,415.5970985)(329.77964111,415.54209856)(329.76964966,415.48210297)
\curveto(329.75964113,415.42209868)(329.76464113,415.36709873)(329.78464966,415.31710297)
\moveto(327.68464966,414.77710297)
\curveto(327.6946432,414.82709927)(327.69964319,414.8970992)(327.69964966,414.98710297)
\curveto(327.69964319,415.08709901)(327.6946432,415.16209894)(327.68464966,415.21210297)
\lineto(327.68464966,415.33210297)
\curveto(327.66464323,415.38209872)(327.65464324,415.43709866)(327.65464966,415.49710297)
\curveto(327.65464324,415.55709854)(327.64964324,415.61209849)(327.63964966,415.66210297)
\curveto(327.63964325,415.7020984)(327.63464326,415.73209837)(327.62464966,415.75210297)
\lineto(327.56464966,415.99210297)
\curveto(327.55464334,416.08209802)(327.53464336,416.16709793)(327.50464966,416.24710297)
\curveto(327.3946435,416.50709759)(327.26464363,416.72709737)(327.11464966,416.90710297)
\curveto(326.96464393,417.097097)(326.76464413,417.24709685)(326.51464966,417.35710297)
\curveto(326.45464444,417.37709672)(326.3946445,417.39209671)(326.33464966,417.40210297)
\curveto(326.27464462,417.42209668)(326.20964468,417.44209666)(326.13964966,417.46210297)
\curveto(326.05964483,417.48209662)(325.97464492,417.48709661)(325.88464966,417.47710297)
\lineto(325.61464966,417.47710297)
\curveto(325.58464531,417.45709664)(325.54964534,417.44709665)(325.50964966,417.44710297)
\curveto(325.46964542,417.45709664)(325.43464546,417.45709664)(325.40464966,417.44710297)
\lineto(325.19464966,417.38710297)
\curveto(325.13464576,417.37709672)(325.07964581,417.35709674)(325.02964966,417.32710297)
\curveto(324.77964611,417.21709688)(324.57464632,417.05709704)(324.41464966,416.84710297)
\curveto(324.26464663,416.64709745)(324.14464675,416.41209769)(324.05464966,416.14210297)
\curveto(324.02464687,416.04209806)(323.99964689,415.93709816)(323.97964966,415.82710297)
\curveto(323.96964692,415.71709838)(323.95464694,415.60709849)(323.93464966,415.49710297)
\curveto(323.92464697,415.44709865)(323.91964697,415.3970987)(323.91964966,415.34710297)
\lineto(323.91964966,415.19710297)
\curveto(323.89964699,415.12709897)(323.889647,415.02209908)(323.88964966,414.88210297)
\curveto(323.89964699,414.74209936)(323.91464698,414.63709946)(323.93464966,414.56710297)
\lineto(323.93464966,414.43210297)
\curveto(323.95464694,414.35209975)(323.96964692,414.27209983)(323.97964966,414.19210297)
\curveto(323.9896469,414.12209998)(324.00464689,414.04710005)(324.02464966,413.96710297)
\curveto(324.12464677,413.66710043)(324.22964666,413.42210068)(324.33964966,413.23210297)
\curveto(324.45964643,413.05210105)(324.64464625,412.88710121)(324.89464966,412.73710297)
\curveto(324.96464593,412.68710141)(325.03964585,412.64710145)(325.11964966,412.61710297)
\curveto(325.20964568,412.58710151)(325.29964559,412.56210154)(325.38964966,412.54210297)
\curveto(325.42964546,412.53210157)(325.46464543,412.52710157)(325.49464966,412.52710297)
\curveto(325.52464537,412.53710156)(325.55964533,412.53710156)(325.59964966,412.52710297)
\lineto(325.71964966,412.49710297)
\curveto(325.76964512,412.4971016)(325.81464508,412.5021016)(325.85464966,412.51210297)
\lineto(325.97464966,412.51210297)
\curveto(326.05464484,412.53210157)(326.13464476,412.54710155)(326.21464966,412.55710297)
\curveto(326.2946446,412.56710153)(326.36964452,412.58710151)(326.43964966,412.61710297)
\curveto(326.69964419,412.71710138)(326.90964398,412.85210125)(327.06964966,413.02210297)
\curveto(327.22964366,413.19210091)(327.36464353,413.4021007)(327.47464966,413.65210297)
\curveto(327.51464338,413.75210035)(327.54464335,413.85210025)(327.56464966,413.95210297)
\curveto(327.58464331,414.05210005)(327.60964328,414.15709994)(327.63964966,414.26710297)
\curveto(327.64964324,414.30709979)(327.65464324,414.34209976)(327.65464966,414.37210297)
\curveto(327.65464324,414.41209969)(327.65964323,414.45209965)(327.66964966,414.49210297)
\lineto(327.66964966,414.62710297)
\curveto(327.66964322,414.67709942)(327.67464322,414.72709937)(327.68464966,414.77710297)
}
}
{
\newrgbcolor{curcolor}{0 0 0}
\pscustom[linestyle=none,fillstyle=solid,fillcolor=curcolor]
{
\newpath
\moveto(334.15457153,419.08210297)
\curveto(334.90456703,419.102095)(335.55456638,419.01709508)(336.10457153,418.82710297)
\curveto(336.66456527,418.64709545)(337.08956485,418.33209577)(337.37957153,417.88210297)
\curveto(337.44956449,417.77209633)(337.50956443,417.65709644)(337.55957153,417.53710297)
\curveto(337.61956432,417.42709667)(337.66956427,417.3020968)(337.70957153,417.16210297)
\curveto(337.72956421,417.102097)(337.7395642,417.03709706)(337.73957153,416.96710297)
\curveto(337.7395642,416.8970972)(337.72956421,416.83709726)(337.70957153,416.78710297)
\curveto(337.66956427,416.72709737)(337.61456432,416.68709741)(337.54457153,416.66710297)
\curveto(337.49456444,416.64709745)(337.4345645,416.63709746)(337.36457153,416.63710297)
\lineto(337.15457153,416.63710297)
\lineto(336.49457153,416.63710297)
\curveto(336.42456551,416.63709746)(336.35456558,416.63209747)(336.28457153,416.62210297)
\curveto(336.21456572,416.62209748)(336.14956579,416.63209747)(336.08957153,416.65210297)
\curveto(335.98956595,416.67209743)(335.91456602,416.71209739)(335.86457153,416.77210297)
\curveto(335.81456612,416.83209727)(335.76956617,416.89209721)(335.72957153,416.95210297)
\lineto(335.60957153,417.16210297)
\curveto(335.57956636,417.24209686)(335.52956641,417.30709679)(335.45957153,417.35710297)
\curveto(335.35956658,417.43709666)(335.25956668,417.4970966)(335.15957153,417.53710297)
\curveto(335.06956687,417.57709652)(334.95456698,417.61209649)(334.81457153,417.64210297)
\curveto(334.74456719,417.66209644)(334.6395673,417.67709642)(334.49957153,417.68710297)
\curveto(334.36956757,417.6970964)(334.26956767,417.69209641)(334.19957153,417.67210297)
\lineto(334.09457153,417.67210297)
\lineto(333.94457153,417.64210297)
\curveto(333.90456803,417.64209646)(333.85956808,417.63709646)(333.80957153,417.62710297)
\curveto(333.6395683,417.57709652)(333.49956844,417.50709659)(333.38957153,417.41710297)
\curveto(333.28956865,417.33709676)(333.21956872,417.21209689)(333.17957153,417.04210297)
\curveto(333.15956878,416.97209713)(333.15956878,416.90709719)(333.17957153,416.84710297)
\curveto(333.19956874,416.78709731)(333.21956872,416.73709736)(333.23957153,416.69710297)
\curveto(333.30956863,416.57709752)(333.38956855,416.48209762)(333.47957153,416.41210297)
\curveto(333.57956836,416.34209776)(333.69456824,416.28209782)(333.82457153,416.23210297)
\curveto(334.01456792,416.15209795)(334.21956772,416.08209802)(334.43957153,416.02210297)
\lineto(335.12957153,415.87210297)
\curveto(335.36956657,415.83209827)(335.59956634,415.78209832)(335.81957153,415.72210297)
\curveto(336.04956589,415.67209843)(336.26456567,415.60709849)(336.46457153,415.52710297)
\curveto(336.55456538,415.48709861)(336.6395653,415.45209865)(336.71957153,415.42210297)
\curveto(336.80956513,415.4020987)(336.89456504,415.36709873)(336.97457153,415.31710297)
\curveto(337.16456477,415.1970989)(337.3345646,415.06709903)(337.48457153,414.92710297)
\curveto(337.64456429,414.78709931)(337.76956417,414.61209949)(337.85957153,414.40210297)
\curveto(337.88956405,414.33209977)(337.91456402,414.26209984)(337.93457153,414.19210297)
\curveto(337.95456398,414.12209998)(337.97456396,414.04710005)(337.99457153,413.96710297)
\curveto(338.00456393,413.90710019)(338.00956393,413.81210029)(338.00957153,413.68210297)
\curveto(338.01956392,413.56210054)(338.01956392,413.46710063)(338.00957153,413.39710297)
\lineto(338.00957153,413.32210297)
\curveto(337.98956395,413.26210084)(337.97456396,413.2021009)(337.96457153,413.14210297)
\curveto(337.96456397,413.09210101)(337.95956398,413.04210106)(337.94957153,412.99210297)
\curveto(337.87956406,412.69210141)(337.76956417,412.42710167)(337.61957153,412.19710297)
\curveto(337.45956448,411.95710214)(337.26456467,411.76210234)(337.03457153,411.61210297)
\curveto(336.80456513,411.46210264)(336.54456539,411.33210277)(336.25457153,411.22210297)
\curveto(336.14456579,411.17210293)(336.02456591,411.13710296)(335.89457153,411.11710297)
\curveto(335.77456616,411.097103)(335.65456628,411.07210303)(335.53457153,411.04210297)
\curveto(335.44456649,411.02210308)(335.34956659,411.01210309)(335.24957153,411.01210297)
\curveto(335.15956678,411.0021031)(335.06956687,410.98710311)(334.97957153,410.96710297)
\lineto(334.70957153,410.96710297)
\curveto(334.64956729,410.94710315)(334.54456739,410.93710316)(334.39457153,410.93710297)
\curveto(334.25456768,410.93710316)(334.15456778,410.94710315)(334.09457153,410.96710297)
\curveto(334.06456787,410.96710313)(334.02956791,410.97210313)(333.98957153,410.98210297)
\lineto(333.88457153,410.98210297)
\curveto(333.76456817,411.0021031)(333.64456829,411.01710308)(333.52457153,411.02710297)
\curveto(333.40456853,411.03710306)(333.28956865,411.05710304)(333.17957153,411.08710297)
\curveto(332.78956915,411.1971029)(332.44456949,411.32210278)(332.14457153,411.46210297)
\curveto(331.84457009,411.61210249)(331.58957035,411.83210227)(331.37957153,412.12210297)
\curveto(331.2395707,412.31210179)(331.11957082,412.53210157)(331.01957153,412.78210297)
\curveto(330.99957094,412.84210126)(330.97957096,412.92210118)(330.95957153,413.02210297)
\curveto(330.939571,413.07210103)(330.92457101,413.14210096)(330.91457153,413.23210297)
\curveto(330.90457103,413.32210078)(330.90957103,413.3971007)(330.92957153,413.45710297)
\curveto(330.95957098,413.52710057)(331.00957093,413.57710052)(331.07957153,413.60710297)
\curveto(331.12957081,413.62710047)(331.18957075,413.63710046)(331.25957153,413.63710297)
\lineto(331.48457153,413.63710297)
\lineto(332.18957153,413.63710297)
\lineto(332.42957153,413.63710297)
\curveto(332.50956943,413.63710046)(332.57956936,413.62710047)(332.63957153,413.60710297)
\curveto(332.74956919,413.56710053)(332.81956912,413.5021006)(332.84957153,413.41210297)
\curveto(332.88956905,413.32210078)(332.934569,413.22710087)(332.98457153,413.12710297)
\curveto(333.00456893,413.07710102)(333.0395689,413.01210109)(333.08957153,412.93210297)
\curveto(333.14956879,412.85210125)(333.19956874,412.8021013)(333.23957153,412.78210297)
\curveto(333.35956858,412.68210142)(333.47456846,412.6021015)(333.58457153,412.54210297)
\curveto(333.69456824,412.49210161)(333.8345681,412.44210166)(334.00457153,412.39210297)
\curveto(334.05456788,412.37210173)(334.10456783,412.36210174)(334.15457153,412.36210297)
\curveto(334.20456773,412.37210173)(334.25456768,412.37210173)(334.30457153,412.36210297)
\curveto(334.38456755,412.34210176)(334.46956747,412.33210177)(334.55957153,412.33210297)
\curveto(334.65956728,412.34210176)(334.74456719,412.35710174)(334.81457153,412.37710297)
\curveto(334.86456707,412.38710171)(334.90956703,412.39210171)(334.94957153,412.39210297)
\curveto(334.99956694,412.39210171)(335.04956689,412.4021017)(335.09957153,412.42210297)
\curveto(335.2395667,412.47210163)(335.36456657,412.53210157)(335.47457153,412.60210297)
\curveto(335.59456634,412.67210143)(335.68956625,412.76210134)(335.75957153,412.87210297)
\curveto(335.80956613,412.95210115)(335.84956609,413.07710102)(335.87957153,413.24710297)
\curveto(335.89956604,413.31710078)(335.89956604,413.38210072)(335.87957153,413.44210297)
\curveto(335.85956608,413.5021006)(335.8395661,413.55210055)(335.81957153,413.59210297)
\curveto(335.74956619,413.73210037)(335.65956628,413.83710026)(335.54957153,413.90710297)
\curveto(335.44956649,413.97710012)(335.32956661,414.04210006)(335.18957153,414.10210297)
\curveto(334.99956694,414.18209992)(334.79956714,414.24709985)(334.58957153,414.29710297)
\curveto(334.37956756,414.34709975)(334.16956777,414.4020997)(333.95957153,414.46210297)
\curveto(333.87956806,414.48209962)(333.79456814,414.4970996)(333.70457153,414.50710297)
\curveto(333.62456831,414.51709958)(333.54456839,414.53209957)(333.46457153,414.55210297)
\curveto(333.14456879,414.64209946)(332.8395691,414.72709937)(332.54957153,414.80710297)
\curveto(332.25956968,414.8970992)(331.99456994,415.02709907)(331.75457153,415.19710297)
\curveto(331.47457046,415.3970987)(331.26957067,415.66709843)(331.13957153,416.00710297)
\curveto(331.11957082,416.07709802)(331.09957084,416.17209793)(331.07957153,416.29210297)
\curveto(331.05957088,416.36209774)(331.04457089,416.44709765)(331.03457153,416.54710297)
\curveto(331.02457091,416.64709745)(331.02957091,416.73709736)(331.04957153,416.81710297)
\curveto(331.06957087,416.86709723)(331.07457086,416.90709719)(331.06457153,416.93710297)
\curveto(331.05457088,416.97709712)(331.05957088,417.02209708)(331.07957153,417.07210297)
\curveto(331.09957084,417.18209692)(331.11957082,417.28209682)(331.13957153,417.37210297)
\curveto(331.16957077,417.47209663)(331.20457073,417.56709653)(331.24457153,417.65710297)
\curveto(331.37457056,417.94709615)(331.55457038,418.18209592)(331.78457153,418.36210297)
\curveto(332.01456992,418.54209556)(332.27456966,418.68709541)(332.56457153,418.79710297)
\curveto(332.67456926,418.84709525)(332.78956915,418.88209522)(332.90957153,418.90210297)
\curveto(333.02956891,418.93209517)(333.15456878,418.96209514)(333.28457153,418.99210297)
\curveto(333.34456859,419.01209509)(333.40456853,419.02209508)(333.46457153,419.02210297)
\lineto(333.64457153,419.05210297)
\curveto(333.72456821,419.06209504)(333.80956813,419.06709503)(333.89957153,419.06710297)
\curveto(333.98956795,419.06709503)(334.07456786,419.07209503)(334.15457153,419.08210297)
}
}
{
\newrgbcolor{curcolor}{0 0 0}
\pscustom[linestyle=none,fillstyle=solid,fillcolor=curcolor]
{
}
}
{
\newrgbcolor{curcolor}{0 0 0}
\pscustom[linestyle=none,fillstyle=solid,fillcolor=curcolor]
{
\newpath
\moveto(351.29136841,415.09210297)
\curveto(351.30135973,415.03209907)(351.30635972,414.94209916)(351.30636841,414.82210297)
\curveto(351.30635972,414.7020994)(351.29635973,414.61709948)(351.27636841,414.56710297)
\lineto(351.27636841,414.37210297)
\curveto(351.24635978,414.26209984)(351.2263598,414.15709994)(351.21636841,414.05710297)
\curveto(351.21635981,413.95710014)(351.20135983,413.85710024)(351.17136841,413.75710297)
\curveto(351.15135988,413.66710043)(351.1313599,413.57210053)(351.11136841,413.47210297)
\curveto(351.09135994,413.38210072)(351.06135997,413.29210081)(351.02136841,413.20210297)
\curveto(350.95136008,413.03210107)(350.88136015,412.87210123)(350.81136841,412.72210297)
\curveto(350.74136029,412.58210152)(350.66136037,412.44210166)(350.57136841,412.30210297)
\curveto(350.51136052,412.21210189)(350.44636058,412.12710197)(350.37636841,412.04710297)
\curveto(350.31636071,411.97710212)(350.24636078,411.9021022)(350.16636841,411.82210297)
\lineto(350.06136841,411.71710297)
\curveto(350.01136102,411.66710243)(349.95636107,411.62210248)(349.89636841,411.58210297)
\lineto(349.74636841,411.46210297)
\curveto(349.66636136,411.4021027)(349.57636145,411.34710275)(349.47636841,411.29710297)
\curveto(349.38636164,411.25710284)(349.29136174,411.21210289)(349.19136841,411.16210297)
\curveto(349.09136194,411.11210299)(348.98636204,411.07710302)(348.87636841,411.05710297)
\curveto(348.77636225,411.03710306)(348.67136236,411.01710308)(348.56136841,410.99710297)
\curveto(348.50136253,410.97710312)(348.43636259,410.96710313)(348.36636841,410.96710297)
\curveto(348.30636272,410.96710313)(348.24136279,410.95710314)(348.17136841,410.93710297)
\lineto(348.03636841,410.93710297)
\curveto(347.95636307,410.91710318)(347.88136315,410.91710318)(347.81136841,410.93710297)
\lineto(347.66136841,410.93710297)
\curveto(347.60136343,410.95710314)(347.53636349,410.96710313)(347.46636841,410.96710297)
\curveto(347.40636362,410.95710314)(347.34636368,410.96210314)(347.28636841,410.98210297)
\curveto(347.1263639,411.03210307)(346.97136406,411.07710302)(346.82136841,411.11710297)
\curveto(346.68136435,411.15710294)(346.55136448,411.21710288)(346.43136841,411.29710297)
\curveto(346.36136467,411.33710276)(346.29636473,411.37710272)(346.23636841,411.41710297)
\curveto(346.17636485,411.46710263)(346.11136492,411.51710258)(346.04136841,411.56710297)
\lineto(345.86136841,411.70210297)
\curveto(345.78136525,411.76210234)(345.71136532,411.76710233)(345.65136841,411.71710297)
\curveto(345.60136543,411.68710241)(345.57636545,411.64710245)(345.57636841,411.59710297)
\curveto(345.57636545,411.55710254)(345.56636546,411.50710259)(345.54636841,411.44710297)
\curveto(345.5263655,411.34710275)(345.51636551,411.23210287)(345.51636841,411.10210297)
\curveto(345.5263655,410.97210313)(345.5313655,410.85210325)(345.53136841,410.74210297)
\lineto(345.53136841,409.21210297)
\curveto(345.5313655,409.08210502)(345.5263655,408.95710514)(345.51636841,408.83710297)
\curveto(345.51636551,408.70710539)(345.49136554,408.6021055)(345.44136841,408.52210297)
\curveto(345.41136562,408.48210562)(345.35636567,408.45210565)(345.27636841,408.43210297)
\curveto(345.19636583,408.41210569)(345.10636592,408.4021057)(345.00636841,408.40210297)
\curveto(344.90636612,408.39210571)(344.80636622,408.39210571)(344.70636841,408.40210297)
\lineto(344.45136841,408.40210297)
\lineto(344.04636841,408.40210297)
\lineto(343.94136841,408.40210297)
\curveto(343.90136713,408.4021057)(343.86636716,408.40710569)(343.83636841,408.41710297)
\lineto(343.71636841,408.41710297)
\curveto(343.54636748,408.46710563)(343.45636757,408.56710553)(343.44636841,408.71710297)
\curveto(343.43636759,408.85710524)(343.4313676,409.02710507)(343.43136841,409.22710297)
\lineto(343.43136841,418.03210297)
\curveto(343.4313676,418.14209596)(343.4263676,418.25709584)(343.41636841,418.37710297)
\curveto(343.41636761,418.50709559)(343.44136759,418.60709549)(343.49136841,418.67710297)
\curveto(343.5313675,418.74709535)(343.58636744,418.79209531)(343.65636841,418.81210297)
\curveto(343.70636732,418.83209527)(343.76636726,418.84209526)(343.83636841,418.84210297)
\lineto(344.06136841,418.84210297)
\lineto(344.78136841,418.84210297)
\lineto(345.06636841,418.84210297)
\curveto(345.15636587,418.84209526)(345.2313658,418.81709528)(345.29136841,418.76710297)
\curveto(345.36136567,418.71709538)(345.39636563,418.65209545)(345.39636841,418.57210297)
\curveto(345.40636562,418.5020956)(345.4313656,418.42709567)(345.47136841,418.34710297)
\curveto(345.48136555,418.31709578)(345.49136554,418.29209581)(345.50136841,418.27210297)
\curveto(345.52136551,418.26209584)(345.54136549,418.24709585)(345.56136841,418.22710297)
\curveto(345.67136536,418.21709588)(345.76136527,418.24709585)(345.83136841,418.31710297)
\curveto(345.90136513,418.38709571)(345.97136506,418.44709565)(346.04136841,418.49710297)
\curveto(346.17136486,418.58709551)(346.30636472,418.66709543)(346.44636841,418.73710297)
\curveto(346.58636444,418.81709528)(346.74136429,418.88209522)(346.91136841,418.93210297)
\curveto(346.99136404,418.96209514)(347.07636395,418.98209512)(347.16636841,418.99210297)
\curveto(347.26636376,419.0020951)(347.36136367,419.01709508)(347.45136841,419.03710297)
\curveto(347.49136354,419.04709505)(347.5313635,419.04709505)(347.57136841,419.03710297)
\curveto(347.62136341,419.02709507)(347.66136337,419.03209507)(347.69136841,419.05210297)
\curveto(348.26136277,419.07209503)(348.74136229,418.99209511)(349.13136841,418.81210297)
\curveto(349.5313615,418.64209546)(349.87136116,418.41709568)(350.15136841,418.13710297)
\curveto(350.20136083,418.08709601)(350.24636078,418.03709606)(350.28636841,417.98710297)
\curveto(350.3263607,417.94709615)(350.36636066,417.9020962)(350.40636841,417.85210297)
\curveto(350.47636055,417.76209634)(350.53636049,417.67209643)(350.58636841,417.58210297)
\curveto(350.64636038,417.49209661)(350.70136033,417.4020967)(350.75136841,417.31210297)
\curveto(350.77136026,417.29209681)(350.78136025,417.26709683)(350.78136841,417.23710297)
\curveto(350.79136024,417.20709689)(350.80636022,417.17209693)(350.82636841,417.13210297)
\curveto(350.88636014,417.03209707)(350.94136009,416.91209719)(350.99136841,416.77210297)
\curveto(351.01136002,416.71209739)(351.03136,416.64709745)(351.05136841,416.57710297)
\curveto(351.07135996,416.51709758)(351.09135994,416.45209765)(351.11136841,416.38210297)
\curveto(351.15135988,416.26209784)(351.17635985,416.13709796)(351.18636841,416.00710297)
\curveto(351.20635982,415.87709822)(351.2313598,415.74209836)(351.26136841,415.60210297)
\lineto(351.26136841,415.43710297)
\lineto(351.29136841,415.25710297)
\lineto(351.29136841,415.09210297)
\moveto(349.17636841,414.74710297)
\curveto(349.18636184,414.7970993)(349.19136184,414.86209924)(349.19136841,414.94210297)
\curveto(349.19136184,415.03209907)(349.18636184,415.102099)(349.17636841,415.15210297)
\lineto(349.17636841,415.28710297)
\curveto(349.15636187,415.34709875)(349.14636188,415.41209869)(349.14636841,415.48210297)
\curveto(349.14636188,415.55209855)(349.13636189,415.62209848)(349.11636841,415.69210297)
\curveto(349.09636193,415.79209831)(349.07636195,415.88709821)(349.05636841,415.97710297)
\curveto(349.03636199,416.07709802)(349.00636202,416.16709793)(348.96636841,416.24710297)
\curveto(348.84636218,416.56709753)(348.69136234,416.82209728)(348.50136841,417.01210297)
\curveto(348.31136272,417.2020969)(348.04136299,417.34209676)(347.69136841,417.43210297)
\curveto(347.61136342,417.45209665)(347.52136351,417.46209664)(347.42136841,417.46210297)
\lineto(347.15136841,417.46210297)
\curveto(347.11136392,417.45209665)(347.07636395,417.44709665)(347.04636841,417.44710297)
\curveto(347.01636401,417.44709665)(346.98136405,417.44209666)(346.94136841,417.43210297)
\lineto(346.73136841,417.37210297)
\curveto(346.67136436,417.36209674)(346.61136442,417.34209676)(346.55136841,417.31210297)
\curveto(346.29136474,417.2020969)(346.08636494,417.03209707)(345.93636841,416.80210297)
\curveto(345.79636523,416.57209753)(345.68136535,416.31709778)(345.59136841,416.03710297)
\curveto(345.57136546,415.95709814)(345.55636547,415.87209823)(345.54636841,415.78210297)
\curveto(345.53636549,415.7020984)(345.52136551,415.62209848)(345.50136841,415.54210297)
\curveto(345.49136554,415.5020986)(345.48636554,415.43709866)(345.48636841,415.34710297)
\curveto(345.46636556,415.30709879)(345.46136557,415.25709884)(345.47136841,415.19710297)
\curveto(345.48136555,415.14709895)(345.48136555,415.097099)(345.47136841,415.04710297)
\curveto(345.45136558,414.98709911)(345.45136558,414.93209917)(345.47136841,414.88210297)
\lineto(345.47136841,414.70210297)
\lineto(345.47136841,414.56710297)
\curveto(345.47136556,414.52709957)(345.48136555,414.48709961)(345.50136841,414.44710297)
\curveto(345.50136553,414.37709972)(345.50636552,414.32209978)(345.51636841,414.28210297)
\lineto(345.54636841,414.10210297)
\curveto(345.55636547,414.04210006)(345.57136546,413.98210012)(345.59136841,413.92210297)
\curveto(345.68136535,413.63210047)(345.78636524,413.39210071)(345.90636841,413.20210297)
\curveto(346.03636499,413.02210108)(346.21636481,412.86210124)(346.44636841,412.72210297)
\curveto(346.58636444,412.64210146)(346.75136428,412.57710152)(346.94136841,412.52710297)
\curveto(346.98136405,412.51710158)(347.01636401,412.51210159)(347.04636841,412.51210297)
\curveto(347.07636395,412.52210158)(347.11136392,412.52210158)(347.15136841,412.51210297)
\curveto(347.19136384,412.5021016)(347.25136378,412.49210161)(347.33136841,412.48210297)
\curveto(347.41136362,412.48210162)(347.47636355,412.48710161)(347.52636841,412.49710297)
\curveto(347.60636342,412.51710158)(347.68636334,412.53210157)(347.76636841,412.54210297)
\curveto(347.85636317,412.56210154)(347.94136309,412.58710151)(348.02136841,412.61710297)
\curveto(348.26136277,412.71710138)(348.45636257,412.85710124)(348.60636841,413.03710297)
\curveto(348.75636227,413.21710088)(348.88136215,413.42710067)(348.98136841,413.66710297)
\curveto(349.031362,413.78710031)(349.06636196,413.91210019)(349.08636841,414.04210297)
\curveto(349.10636192,414.17209993)(349.1313619,414.30709979)(349.16136841,414.44710297)
\lineto(349.16136841,414.59710297)
\curveto(349.17136186,414.64709945)(349.17636185,414.6970994)(349.17636841,414.74710297)
}
}
{
\newrgbcolor{curcolor}{0 0 0}
\pscustom[linestyle=none,fillstyle=solid,fillcolor=curcolor]
{
\newpath
\moveto(360.34129028,415.31710297)
\curveto(360.36128171,415.25709884)(360.3712817,415.17209893)(360.37129028,415.06210297)
\curveto(360.3712817,414.95209915)(360.36128171,414.86709923)(360.34129028,414.80710297)
\lineto(360.34129028,414.65710297)
\curveto(360.32128175,414.57709952)(360.31128176,414.4970996)(360.31129028,414.41710297)
\curveto(360.32128175,414.33709976)(360.31628176,414.25709984)(360.29629028,414.17710297)
\curveto(360.2762818,414.10709999)(360.26128181,414.04210006)(360.25129028,413.98210297)
\curveto(360.24128183,413.92210018)(360.23128184,413.85710024)(360.22129028,413.78710297)
\curveto(360.18128189,413.67710042)(360.14628193,413.56210054)(360.11629028,413.44210297)
\curveto(360.08628199,413.33210077)(360.04628203,413.22710087)(359.99629028,413.12710297)
\curveto(359.78628229,412.64710145)(359.51128256,412.25710184)(359.17129028,411.95710297)
\curveto(358.83128324,411.65710244)(358.42128365,411.40710269)(357.94129028,411.20710297)
\curveto(357.82128425,411.15710294)(357.69628438,411.12210298)(357.56629028,411.10210297)
\curveto(357.44628463,411.07210303)(357.32128475,411.04210306)(357.19129028,411.01210297)
\curveto(357.14128493,410.99210311)(357.08628499,410.98210312)(357.02629028,410.98210297)
\curveto(356.96628511,410.98210312)(356.91128516,410.97710312)(356.86129028,410.96710297)
\lineto(356.75629028,410.96710297)
\curveto(356.72628535,410.95710314)(356.69628538,410.95210315)(356.66629028,410.95210297)
\curveto(356.61628546,410.94210316)(356.53628554,410.93710316)(356.42629028,410.93710297)
\curveto(356.31628576,410.92710317)(356.23128584,410.93210317)(356.17129028,410.95210297)
\lineto(356.02129028,410.95210297)
\curveto(355.9712861,410.96210314)(355.91628616,410.96710313)(355.85629028,410.96710297)
\curveto(355.80628627,410.95710314)(355.75628632,410.96210314)(355.70629028,410.98210297)
\curveto(355.66628641,410.99210311)(355.62628645,410.9971031)(355.58629028,410.99710297)
\curveto(355.55628652,410.9971031)(355.51628656,411.0021031)(355.46629028,411.01210297)
\curveto(355.36628671,411.04210306)(355.26628681,411.06710303)(355.16629028,411.08710297)
\curveto(355.06628701,411.10710299)(354.9712871,411.13710296)(354.88129028,411.17710297)
\curveto(354.76128731,411.21710288)(354.64628743,411.25710284)(354.53629028,411.29710297)
\curveto(354.43628764,411.33710276)(354.33128774,411.38710271)(354.22129028,411.44710297)
\curveto(353.8712882,411.65710244)(353.5712885,411.9021022)(353.32129028,412.18210297)
\curveto(353.071289,412.46210164)(352.86128921,412.7971013)(352.69129028,413.18710297)
\curveto(352.64128943,413.27710082)(352.60128947,413.37210073)(352.57129028,413.47210297)
\curveto(352.55128952,413.57210053)(352.52628955,413.67710042)(352.49629028,413.78710297)
\curveto(352.4762896,413.83710026)(352.46628961,413.88210022)(352.46629028,413.92210297)
\curveto(352.46628961,413.96210014)(352.45628962,414.00710009)(352.43629028,414.05710297)
\curveto(352.41628966,414.13709996)(352.40628967,414.21709988)(352.40629028,414.29710297)
\curveto(352.40628967,414.38709971)(352.39628968,414.47209963)(352.37629028,414.55210297)
\curveto(352.36628971,414.6020995)(352.36128971,414.64709945)(352.36129028,414.68710297)
\lineto(352.36129028,414.82210297)
\curveto(352.34128973,414.88209922)(352.33128974,414.96709913)(352.33129028,415.07710297)
\curveto(352.34128973,415.18709891)(352.35628972,415.27209883)(352.37629028,415.33210297)
\lineto(352.37629028,415.43710297)
\curveto(352.38628969,415.48709861)(352.38628969,415.53709856)(352.37629028,415.58710297)
\curveto(352.3762897,415.64709845)(352.38628969,415.7020984)(352.40629028,415.75210297)
\curveto(352.41628966,415.8020983)(352.42128965,415.84709825)(352.42129028,415.88710297)
\curveto(352.42128965,415.93709816)(352.43128964,415.98709811)(352.45129028,416.03710297)
\curveto(352.49128958,416.16709793)(352.52628955,416.29209781)(352.55629028,416.41210297)
\curveto(352.58628949,416.54209756)(352.62628945,416.66709743)(352.67629028,416.78710297)
\curveto(352.85628922,417.1970969)(353.071289,417.53709656)(353.32129028,417.80710297)
\curveto(353.5712885,418.08709601)(353.8762882,418.34209576)(354.23629028,418.57210297)
\curveto(354.33628774,418.62209548)(354.44128763,418.66709543)(354.55129028,418.70710297)
\curveto(354.66128741,418.74709535)(354.7712873,418.79209531)(354.88129028,418.84210297)
\curveto(355.01128706,418.89209521)(355.14628693,418.92709517)(355.28629028,418.94710297)
\curveto(355.42628665,418.96709513)(355.5712865,418.9970951)(355.72129028,419.03710297)
\curveto(355.80128627,419.04709505)(355.8762862,419.05209505)(355.94629028,419.05210297)
\curveto(356.01628606,419.05209505)(356.08628599,419.05709504)(356.15629028,419.06710297)
\curveto(356.73628534,419.07709502)(357.23628484,419.01709508)(357.65629028,418.88710297)
\curveto(358.08628399,418.75709534)(358.46628361,418.57709552)(358.79629028,418.34710297)
\curveto(358.90628317,418.26709583)(359.01628306,418.17709592)(359.12629028,418.07710297)
\curveto(359.24628283,417.98709611)(359.34628273,417.88709621)(359.42629028,417.77710297)
\curveto(359.50628257,417.67709642)(359.5762825,417.57709652)(359.63629028,417.47710297)
\curveto(359.70628237,417.37709672)(359.7762823,417.27209683)(359.84629028,417.16210297)
\curveto(359.91628216,417.05209705)(359.9712821,416.93209717)(360.01129028,416.80210297)
\curveto(360.05128202,416.68209742)(360.09628198,416.55209755)(360.14629028,416.41210297)
\curveto(360.1762819,416.33209777)(360.20128187,416.24709785)(360.22129028,416.15710297)
\lineto(360.28129028,415.88710297)
\curveto(360.29128178,415.84709825)(360.29628178,415.80709829)(360.29629028,415.76710297)
\curveto(360.29628178,415.72709837)(360.30128177,415.68709841)(360.31129028,415.64710297)
\curveto(360.33128174,415.5970985)(360.33628174,415.54209856)(360.32629028,415.48210297)
\curveto(360.31628176,415.42209868)(360.32128175,415.36709873)(360.34129028,415.31710297)
\moveto(358.24129028,414.77710297)
\curveto(358.25128382,414.82709927)(358.25628382,414.8970992)(358.25629028,414.98710297)
\curveto(358.25628382,415.08709901)(358.25128382,415.16209894)(358.24129028,415.21210297)
\lineto(358.24129028,415.33210297)
\curveto(358.22128385,415.38209872)(358.21128386,415.43709866)(358.21129028,415.49710297)
\curveto(358.21128386,415.55709854)(358.20628387,415.61209849)(358.19629028,415.66210297)
\curveto(358.19628388,415.7020984)(358.19128388,415.73209837)(358.18129028,415.75210297)
\lineto(358.12129028,415.99210297)
\curveto(358.11128396,416.08209802)(358.09128398,416.16709793)(358.06129028,416.24710297)
\curveto(357.95128412,416.50709759)(357.82128425,416.72709737)(357.67129028,416.90710297)
\curveto(357.52128455,417.097097)(357.32128475,417.24709685)(357.07129028,417.35710297)
\curveto(357.01128506,417.37709672)(356.95128512,417.39209671)(356.89129028,417.40210297)
\curveto(356.83128524,417.42209668)(356.76628531,417.44209666)(356.69629028,417.46210297)
\curveto(356.61628546,417.48209662)(356.53128554,417.48709661)(356.44129028,417.47710297)
\lineto(356.17129028,417.47710297)
\curveto(356.14128593,417.45709664)(356.10628597,417.44709665)(356.06629028,417.44710297)
\curveto(356.02628605,417.45709664)(355.99128608,417.45709664)(355.96129028,417.44710297)
\lineto(355.75129028,417.38710297)
\curveto(355.69128638,417.37709672)(355.63628644,417.35709674)(355.58629028,417.32710297)
\curveto(355.33628674,417.21709688)(355.13128694,417.05709704)(354.97129028,416.84710297)
\curveto(354.82128725,416.64709745)(354.70128737,416.41209769)(354.61129028,416.14210297)
\curveto(354.58128749,416.04209806)(354.55628752,415.93709816)(354.53629028,415.82710297)
\curveto(354.52628755,415.71709838)(354.51128756,415.60709849)(354.49129028,415.49710297)
\curveto(354.48128759,415.44709865)(354.4762876,415.3970987)(354.47629028,415.34710297)
\lineto(354.47629028,415.19710297)
\curveto(354.45628762,415.12709897)(354.44628763,415.02209908)(354.44629028,414.88210297)
\curveto(354.45628762,414.74209936)(354.4712876,414.63709946)(354.49129028,414.56710297)
\lineto(354.49129028,414.43210297)
\curveto(354.51128756,414.35209975)(354.52628755,414.27209983)(354.53629028,414.19210297)
\curveto(354.54628753,414.12209998)(354.56128751,414.04710005)(354.58129028,413.96710297)
\curveto(354.68128739,413.66710043)(354.78628729,413.42210068)(354.89629028,413.23210297)
\curveto(355.01628706,413.05210105)(355.20128687,412.88710121)(355.45129028,412.73710297)
\curveto(355.52128655,412.68710141)(355.59628648,412.64710145)(355.67629028,412.61710297)
\curveto(355.76628631,412.58710151)(355.85628622,412.56210154)(355.94629028,412.54210297)
\curveto(355.98628609,412.53210157)(356.02128605,412.52710157)(356.05129028,412.52710297)
\curveto(356.08128599,412.53710156)(356.11628596,412.53710156)(356.15629028,412.52710297)
\lineto(356.27629028,412.49710297)
\curveto(356.32628575,412.4971016)(356.3712857,412.5021016)(356.41129028,412.51210297)
\lineto(356.53129028,412.51210297)
\curveto(356.61128546,412.53210157)(356.69128538,412.54710155)(356.77129028,412.55710297)
\curveto(356.85128522,412.56710153)(356.92628515,412.58710151)(356.99629028,412.61710297)
\curveto(357.25628482,412.71710138)(357.46628461,412.85210125)(357.62629028,413.02210297)
\curveto(357.78628429,413.19210091)(357.92128415,413.4021007)(358.03129028,413.65210297)
\curveto(358.071284,413.75210035)(358.10128397,413.85210025)(358.12129028,413.95210297)
\curveto(358.14128393,414.05210005)(358.16628391,414.15709994)(358.19629028,414.26710297)
\curveto(358.20628387,414.30709979)(358.21128386,414.34209976)(358.21129028,414.37210297)
\curveto(358.21128386,414.41209969)(358.21628386,414.45209965)(358.22629028,414.49210297)
\lineto(358.22629028,414.62710297)
\curveto(358.22628385,414.67709942)(358.23128384,414.72709937)(358.24129028,414.77710297)
}
}
{
\newrgbcolor{curcolor}{0 0 0}
\pscustom[linestyle=none,fillstyle=solid,fillcolor=curcolor]
{
\newpath
\moveto(366.16621216,419.06710297)
\curveto(366.27620684,419.06709503)(366.37120675,419.05709504)(366.45121216,419.03710297)
\curveto(366.54120658,419.01709508)(366.61120651,418.97209513)(366.66121216,418.90210297)
\curveto(366.7212064,418.82209528)(366.75120637,418.68209542)(366.75121216,418.48210297)
\lineto(366.75121216,417.97210297)
\lineto(366.75121216,417.59710297)
\curveto(366.76120636,417.45709664)(366.74620637,417.34709675)(366.70621216,417.26710297)
\curveto(366.66620645,417.1970969)(366.60620651,417.15209695)(366.52621216,417.13210297)
\curveto(366.45620666,417.11209699)(366.37120675,417.102097)(366.27121216,417.10210297)
\curveto(366.18120694,417.102097)(366.08120704,417.10709699)(365.97121216,417.11710297)
\curveto(365.87120725,417.12709697)(365.77620734,417.12209698)(365.68621216,417.10210297)
\curveto(365.6162075,417.08209702)(365.54620757,417.06709703)(365.47621216,417.05710297)
\curveto(365.40620771,417.05709704)(365.34120778,417.04709705)(365.28121216,417.02710297)
\curveto(365.121208,416.97709712)(364.96120816,416.9020972)(364.80121216,416.80210297)
\curveto(364.64120848,416.71209739)(364.5162086,416.60709749)(364.42621216,416.48710297)
\curveto(364.37620874,416.40709769)(364.3212088,416.32209778)(364.26121216,416.23210297)
\curveto(364.21120891,416.15209795)(364.16120896,416.06709803)(364.11121216,415.97710297)
\curveto(364.08120904,415.8970982)(364.05120907,415.81209829)(364.02121216,415.72210297)
\lineto(363.96121216,415.48210297)
\curveto(363.94120918,415.41209869)(363.93120919,415.33709876)(363.93121216,415.25710297)
\curveto(363.93120919,415.18709891)(363.9212092,415.11709898)(363.90121216,415.04710297)
\curveto(363.89120923,415.00709909)(363.88620923,414.96709913)(363.88621216,414.92710297)
\curveto(363.89620922,414.8970992)(363.89620922,414.86709923)(363.88621216,414.83710297)
\lineto(363.88621216,414.59710297)
\curveto(363.86620925,414.52709957)(363.86120926,414.44709965)(363.87121216,414.35710297)
\curveto(363.88120924,414.27709982)(363.88620923,414.1970999)(363.88621216,414.11710297)
\lineto(363.88621216,413.15710297)
\lineto(363.88621216,411.88210297)
\curveto(363.88620923,411.75210235)(363.88120924,411.63210247)(363.87121216,411.52210297)
\curveto(363.86120926,411.41210269)(363.83120929,411.32210278)(363.78121216,411.25210297)
\curveto(363.76120936,411.22210288)(363.72620939,411.1971029)(363.67621216,411.17710297)
\curveto(363.63620948,411.16710293)(363.59120953,411.15710294)(363.54121216,411.14710297)
\lineto(363.46621216,411.14710297)
\curveto(363.4162097,411.13710296)(363.36120976,411.13210297)(363.30121216,411.13210297)
\lineto(363.13621216,411.13210297)
\lineto(362.49121216,411.13210297)
\curveto(362.43121069,411.14210296)(362.36621075,411.14710295)(362.29621216,411.14710297)
\lineto(362.10121216,411.14710297)
\curveto(362.05121107,411.16710293)(362.00121112,411.18210292)(361.95121216,411.19210297)
\curveto(361.90121122,411.21210289)(361.86621125,411.24710285)(361.84621216,411.29710297)
\curveto(361.80621131,411.34710275)(361.78121134,411.41710268)(361.77121216,411.50710297)
\lineto(361.77121216,411.80710297)
\lineto(361.77121216,412.82710297)
\lineto(361.77121216,417.05710297)
\lineto(361.77121216,418.16710297)
\lineto(361.77121216,418.45210297)
\curveto(361.77121135,418.55209555)(361.79121133,418.63209547)(361.83121216,418.69210297)
\curveto(361.88121124,418.77209533)(361.95621116,418.82209528)(362.05621216,418.84210297)
\curveto(362.15621096,418.86209524)(362.27621084,418.87209523)(362.41621216,418.87210297)
\lineto(363.18121216,418.87210297)
\curveto(363.30120982,418.87209523)(363.40620971,418.86209524)(363.49621216,418.84210297)
\curveto(363.58620953,418.83209527)(363.65620946,418.78709531)(363.70621216,418.70710297)
\curveto(363.73620938,418.65709544)(363.75120937,418.58709551)(363.75121216,418.49710297)
\lineto(363.78121216,418.22710297)
\curveto(363.79120933,418.14709595)(363.80620931,418.07209603)(363.82621216,418.00210297)
\curveto(363.85620926,417.93209617)(363.90620921,417.8970962)(363.97621216,417.89710297)
\curveto(363.99620912,417.91709618)(364.0162091,417.92709617)(364.03621216,417.92710297)
\curveto(364.05620906,417.92709617)(364.07620904,417.93709616)(364.09621216,417.95710297)
\curveto(364.15620896,418.00709609)(364.20620891,418.06209604)(364.24621216,418.12210297)
\curveto(364.29620882,418.19209591)(364.35620876,418.25209585)(364.42621216,418.30210297)
\curveto(364.46620865,418.33209577)(364.50120862,418.36209574)(364.53121216,418.39210297)
\curveto(364.56120856,418.43209567)(364.59620852,418.46709563)(364.63621216,418.49710297)
\lineto(364.90621216,418.67710297)
\curveto(365.00620811,418.73709536)(365.10620801,418.79209531)(365.20621216,418.84210297)
\curveto(365.30620781,418.88209522)(365.40620771,418.91709518)(365.50621216,418.94710297)
\lineto(365.83621216,419.03710297)
\curveto(365.86620725,419.04709505)(365.9212072,419.04709505)(366.00121216,419.03710297)
\curveto(366.09120703,419.03709506)(366.14620697,419.04709505)(366.16621216,419.06710297)
}
}
{
\newrgbcolor{curcolor}{0 0 0}
\pscustom[linestyle=none,fillstyle=solid,fillcolor=curcolor]
{
}
}
{
\newrgbcolor{curcolor}{0 0 0}
\pscustom[linestyle=none,fillstyle=solid,fillcolor=curcolor]
{
\newpath
\moveto(376.15644653,419.06710297)
\curveto(376.26644122,419.06709503)(376.36144112,419.05709504)(376.44144653,419.03710297)
\curveto(376.53144095,419.01709508)(376.60144088,418.97209513)(376.65144653,418.90210297)
\curveto(376.71144077,418.82209528)(376.74144074,418.68209542)(376.74144653,418.48210297)
\lineto(376.74144653,417.97210297)
\lineto(376.74144653,417.59710297)
\curveto(376.75144073,417.45709664)(376.73644075,417.34709675)(376.69644653,417.26710297)
\curveto(376.65644083,417.1970969)(376.59644089,417.15209695)(376.51644653,417.13210297)
\curveto(376.44644104,417.11209699)(376.36144112,417.102097)(376.26144653,417.10210297)
\curveto(376.17144131,417.102097)(376.07144141,417.10709699)(375.96144653,417.11710297)
\curveto(375.86144162,417.12709697)(375.76644172,417.12209698)(375.67644653,417.10210297)
\curveto(375.60644188,417.08209702)(375.53644195,417.06709703)(375.46644653,417.05710297)
\curveto(375.39644209,417.05709704)(375.33144215,417.04709705)(375.27144653,417.02710297)
\curveto(375.11144237,416.97709712)(374.95144253,416.9020972)(374.79144653,416.80210297)
\curveto(374.63144285,416.71209739)(374.50644298,416.60709749)(374.41644653,416.48710297)
\curveto(374.36644312,416.40709769)(374.31144317,416.32209778)(374.25144653,416.23210297)
\curveto(374.20144328,416.15209795)(374.15144333,416.06709803)(374.10144653,415.97710297)
\curveto(374.07144341,415.8970982)(374.04144344,415.81209829)(374.01144653,415.72210297)
\lineto(373.95144653,415.48210297)
\curveto(373.93144355,415.41209869)(373.92144356,415.33709876)(373.92144653,415.25710297)
\curveto(373.92144356,415.18709891)(373.91144357,415.11709898)(373.89144653,415.04710297)
\curveto(373.8814436,415.00709909)(373.87644361,414.96709913)(373.87644653,414.92710297)
\curveto(373.8864436,414.8970992)(373.8864436,414.86709923)(373.87644653,414.83710297)
\lineto(373.87644653,414.59710297)
\curveto(373.85644363,414.52709957)(373.85144363,414.44709965)(373.86144653,414.35710297)
\curveto(373.87144361,414.27709982)(373.87644361,414.1970999)(373.87644653,414.11710297)
\lineto(373.87644653,413.15710297)
\lineto(373.87644653,411.88210297)
\curveto(373.87644361,411.75210235)(373.87144361,411.63210247)(373.86144653,411.52210297)
\curveto(373.85144363,411.41210269)(373.82144366,411.32210278)(373.77144653,411.25210297)
\curveto(373.75144373,411.22210288)(373.71644377,411.1971029)(373.66644653,411.17710297)
\curveto(373.62644386,411.16710293)(373.5814439,411.15710294)(373.53144653,411.14710297)
\lineto(373.45644653,411.14710297)
\curveto(373.40644408,411.13710296)(373.35144413,411.13210297)(373.29144653,411.13210297)
\lineto(373.12644653,411.13210297)
\lineto(372.48144653,411.13210297)
\curveto(372.42144506,411.14210296)(372.35644513,411.14710295)(372.28644653,411.14710297)
\lineto(372.09144653,411.14710297)
\curveto(372.04144544,411.16710293)(371.99144549,411.18210292)(371.94144653,411.19210297)
\curveto(371.89144559,411.21210289)(371.85644563,411.24710285)(371.83644653,411.29710297)
\curveto(371.79644569,411.34710275)(371.77144571,411.41710268)(371.76144653,411.50710297)
\lineto(371.76144653,411.80710297)
\lineto(371.76144653,412.82710297)
\lineto(371.76144653,417.05710297)
\lineto(371.76144653,418.16710297)
\lineto(371.76144653,418.45210297)
\curveto(371.76144572,418.55209555)(371.7814457,418.63209547)(371.82144653,418.69210297)
\curveto(371.87144561,418.77209533)(371.94644554,418.82209528)(372.04644653,418.84210297)
\curveto(372.14644534,418.86209524)(372.26644522,418.87209523)(372.40644653,418.87210297)
\lineto(373.17144653,418.87210297)
\curveto(373.29144419,418.87209523)(373.39644409,418.86209524)(373.48644653,418.84210297)
\curveto(373.57644391,418.83209527)(373.64644384,418.78709531)(373.69644653,418.70710297)
\curveto(373.72644376,418.65709544)(373.74144374,418.58709551)(373.74144653,418.49710297)
\lineto(373.77144653,418.22710297)
\curveto(373.7814437,418.14709595)(373.79644369,418.07209603)(373.81644653,418.00210297)
\curveto(373.84644364,417.93209617)(373.89644359,417.8970962)(373.96644653,417.89710297)
\curveto(373.9864435,417.91709618)(374.00644348,417.92709617)(374.02644653,417.92710297)
\curveto(374.04644344,417.92709617)(374.06644342,417.93709616)(374.08644653,417.95710297)
\curveto(374.14644334,418.00709609)(374.19644329,418.06209604)(374.23644653,418.12210297)
\curveto(374.2864432,418.19209591)(374.34644314,418.25209585)(374.41644653,418.30210297)
\curveto(374.45644303,418.33209577)(374.49144299,418.36209574)(374.52144653,418.39210297)
\curveto(374.55144293,418.43209567)(374.5864429,418.46709563)(374.62644653,418.49710297)
\lineto(374.89644653,418.67710297)
\curveto(374.99644249,418.73709536)(375.09644239,418.79209531)(375.19644653,418.84210297)
\curveto(375.29644219,418.88209522)(375.39644209,418.91709518)(375.49644653,418.94710297)
\lineto(375.82644653,419.03710297)
\curveto(375.85644163,419.04709505)(375.91144157,419.04709505)(375.99144653,419.03710297)
\curveto(376.0814414,419.03709506)(376.13644135,419.04709505)(376.15644653,419.06710297)
}
}
{
\newrgbcolor{curcolor}{0 0 0}
\pscustom[linestyle=none,fillstyle=solid,fillcolor=curcolor]
{
\newpath
\moveto(385.06785278,415.31710297)
\curveto(385.08784421,415.25709884)(385.0978442,415.17209893)(385.09785278,415.06210297)
\curveto(385.0978442,414.95209915)(385.08784421,414.86709923)(385.06785278,414.80710297)
\lineto(385.06785278,414.65710297)
\curveto(385.04784425,414.57709952)(385.03784426,414.4970996)(385.03785278,414.41710297)
\curveto(385.04784425,414.33709976)(385.04284426,414.25709984)(385.02285278,414.17710297)
\curveto(385.0028443,414.10709999)(384.98784431,414.04210006)(384.97785278,413.98210297)
\curveto(384.96784433,413.92210018)(384.95784434,413.85710024)(384.94785278,413.78710297)
\curveto(384.90784439,413.67710042)(384.87284443,413.56210054)(384.84285278,413.44210297)
\curveto(384.81284449,413.33210077)(384.77284453,413.22710087)(384.72285278,413.12710297)
\curveto(384.51284479,412.64710145)(384.23784506,412.25710184)(383.89785278,411.95710297)
\curveto(383.55784574,411.65710244)(383.14784615,411.40710269)(382.66785278,411.20710297)
\curveto(382.54784675,411.15710294)(382.42284688,411.12210298)(382.29285278,411.10210297)
\curveto(382.17284713,411.07210303)(382.04784725,411.04210306)(381.91785278,411.01210297)
\curveto(381.86784743,410.99210311)(381.81284749,410.98210312)(381.75285278,410.98210297)
\curveto(381.69284761,410.98210312)(381.63784766,410.97710312)(381.58785278,410.96710297)
\lineto(381.48285278,410.96710297)
\curveto(381.45284785,410.95710314)(381.42284788,410.95210315)(381.39285278,410.95210297)
\curveto(381.34284796,410.94210316)(381.26284804,410.93710316)(381.15285278,410.93710297)
\curveto(381.04284826,410.92710317)(380.95784834,410.93210317)(380.89785278,410.95210297)
\lineto(380.74785278,410.95210297)
\curveto(380.6978486,410.96210314)(380.64284866,410.96710313)(380.58285278,410.96710297)
\curveto(380.53284877,410.95710314)(380.48284882,410.96210314)(380.43285278,410.98210297)
\curveto(380.39284891,410.99210311)(380.35284895,410.9971031)(380.31285278,410.99710297)
\curveto(380.28284902,410.9971031)(380.24284906,411.0021031)(380.19285278,411.01210297)
\curveto(380.09284921,411.04210306)(379.99284931,411.06710303)(379.89285278,411.08710297)
\curveto(379.79284951,411.10710299)(379.6978496,411.13710296)(379.60785278,411.17710297)
\curveto(379.48784981,411.21710288)(379.37284993,411.25710284)(379.26285278,411.29710297)
\curveto(379.16285014,411.33710276)(379.05785024,411.38710271)(378.94785278,411.44710297)
\curveto(378.5978507,411.65710244)(378.297851,411.9021022)(378.04785278,412.18210297)
\curveto(377.7978515,412.46210164)(377.58785171,412.7971013)(377.41785278,413.18710297)
\curveto(377.36785193,413.27710082)(377.32785197,413.37210073)(377.29785278,413.47210297)
\curveto(377.27785202,413.57210053)(377.25285205,413.67710042)(377.22285278,413.78710297)
\curveto(377.2028521,413.83710026)(377.19285211,413.88210022)(377.19285278,413.92210297)
\curveto(377.19285211,413.96210014)(377.18285212,414.00710009)(377.16285278,414.05710297)
\curveto(377.14285216,414.13709996)(377.13285217,414.21709988)(377.13285278,414.29710297)
\curveto(377.13285217,414.38709971)(377.12285218,414.47209963)(377.10285278,414.55210297)
\curveto(377.09285221,414.6020995)(377.08785221,414.64709945)(377.08785278,414.68710297)
\lineto(377.08785278,414.82210297)
\curveto(377.06785223,414.88209922)(377.05785224,414.96709913)(377.05785278,415.07710297)
\curveto(377.06785223,415.18709891)(377.08285222,415.27209883)(377.10285278,415.33210297)
\lineto(377.10285278,415.43710297)
\curveto(377.11285219,415.48709861)(377.11285219,415.53709856)(377.10285278,415.58710297)
\curveto(377.1028522,415.64709845)(377.11285219,415.7020984)(377.13285278,415.75210297)
\curveto(377.14285216,415.8020983)(377.14785215,415.84709825)(377.14785278,415.88710297)
\curveto(377.14785215,415.93709816)(377.15785214,415.98709811)(377.17785278,416.03710297)
\curveto(377.21785208,416.16709793)(377.25285205,416.29209781)(377.28285278,416.41210297)
\curveto(377.31285199,416.54209756)(377.35285195,416.66709743)(377.40285278,416.78710297)
\curveto(377.58285172,417.1970969)(377.7978515,417.53709656)(378.04785278,417.80710297)
\curveto(378.297851,418.08709601)(378.6028507,418.34209576)(378.96285278,418.57210297)
\curveto(379.06285024,418.62209548)(379.16785013,418.66709543)(379.27785278,418.70710297)
\curveto(379.38784991,418.74709535)(379.4978498,418.79209531)(379.60785278,418.84210297)
\curveto(379.73784956,418.89209521)(379.87284943,418.92709517)(380.01285278,418.94710297)
\curveto(380.15284915,418.96709513)(380.297849,418.9970951)(380.44785278,419.03710297)
\curveto(380.52784877,419.04709505)(380.6028487,419.05209505)(380.67285278,419.05210297)
\curveto(380.74284856,419.05209505)(380.81284849,419.05709504)(380.88285278,419.06710297)
\curveto(381.46284784,419.07709502)(381.96284734,419.01709508)(382.38285278,418.88710297)
\curveto(382.81284649,418.75709534)(383.19284611,418.57709552)(383.52285278,418.34710297)
\curveto(383.63284567,418.26709583)(383.74284556,418.17709592)(383.85285278,418.07710297)
\curveto(383.97284533,417.98709611)(384.07284523,417.88709621)(384.15285278,417.77710297)
\curveto(384.23284507,417.67709642)(384.302845,417.57709652)(384.36285278,417.47710297)
\curveto(384.43284487,417.37709672)(384.5028448,417.27209683)(384.57285278,417.16210297)
\curveto(384.64284466,417.05209705)(384.6978446,416.93209717)(384.73785278,416.80210297)
\curveto(384.77784452,416.68209742)(384.82284448,416.55209755)(384.87285278,416.41210297)
\curveto(384.9028444,416.33209777)(384.92784437,416.24709785)(384.94785278,416.15710297)
\lineto(385.00785278,415.88710297)
\curveto(385.01784428,415.84709825)(385.02284428,415.80709829)(385.02285278,415.76710297)
\curveto(385.02284428,415.72709837)(385.02784427,415.68709841)(385.03785278,415.64710297)
\curveto(385.05784424,415.5970985)(385.06284424,415.54209856)(385.05285278,415.48210297)
\curveto(385.04284426,415.42209868)(385.04784425,415.36709873)(385.06785278,415.31710297)
\moveto(382.96785278,414.77710297)
\curveto(382.97784632,414.82709927)(382.98284632,414.8970992)(382.98285278,414.98710297)
\curveto(382.98284632,415.08709901)(382.97784632,415.16209894)(382.96785278,415.21210297)
\lineto(382.96785278,415.33210297)
\curveto(382.94784635,415.38209872)(382.93784636,415.43709866)(382.93785278,415.49710297)
\curveto(382.93784636,415.55709854)(382.93284637,415.61209849)(382.92285278,415.66210297)
\curveto(382.92284638,415.7020984)(382.91784638,415.73209837)(382.90785278,415.75210297)
\lineto(382.84785278,415.99210297)
\curveto(382.83784646,416.08209802)(382.81784648,416.16709793)(382.78785278,416.24710297)
\curveto(382.67784662,416.50709759)(382.54784675,416.72709737)(382.39785278,416.90710297)
\curveto(382.24784705,417.097097)(382.04784725,417.24709685)(381.79785278,417.35710297)
\curveto(381.73784756,417.37709672)(381.67784762,417.39209671)(381.61785278,417.40210297)
\curveto(381.55784774,417.42209668)(381.49284781,417.44209666)(381.42285278,417.46210297)
\curveto(381.34284796,417.48209662)(381.25784804,417.48709661)(381.16785278,417.47710297)
\lineto(380.89785278,417.47710297)
\curveto(380.86784843,417.45709664)(380.83284847,417.44709665)(380.79285278,417.44710297)
\curveto(380.75284855,417.45709664)(380.71784858,417.45709664)(380.68785278,417.44710297)
\lineto(380.47785278,417.38710297)
\curveto(380.41784888,417.37709672)(380.36284894,417.35709674)(380.31285278,417.32710297)
\curveto(380.06284924,417.21709688)(379.85784944,417.05709704)(379.69785278,416.84710297)
\curveto(379.54784975,416.64709745)(379.42784987,416.41209769)(379.33785278,416.14210297)
\curveto(379.30784999,416.04209806)(379.28285002,415.93709816)(379.26285278,415.82710297)
\curveto(379.25285005,415.71709838)(379.23785006,415.60709849)(379.21785278,415.49710297)
\curveto(379.20785009,415.44709865)(379.2028501,415.3970987)(379.20285278,415.34710297)
\lineto(379.20285278,415.19710297)
\curveto(379.18285012,415.12709897)(379.17285013,415.02209908)(379.17285278,414.88210297)
\curveto(379.18285012,414.74209936)(379.1978501,414.63709946)(379.21785278,414.56710297)
\lineto(379.21785278,414.43210297)
\curveto(379.23785006,414.35209975)(379.25285005,414.27209983)(379.26285278,414.19210297)
\curveto(379.27285003,414.12209998)(379.28785001,414.04710005)(379.30785278,413.96710297)
\curveto(379.40784989,413.66710043)(379.51284979,413.42210068)(379.62285278,413.23210297)
\curveto(379.74284956,413.05210105)(379.92784937,412.88710121)(380.17785278,412.73710297)
\curveto(380.24784905,412.68710141)(380.32284898,412.64710145)(380.40285278,412.61710297)
\curveto(380.49284881,412.58710151)(380.58284872,412.56210154)(380.67285278,412.54210297)
\curveto(380.71284859,412.53210157)(380.74784855,412.52710157)(380.77785278,412.52710297)
\curveto(380.80784849,412.53710156)(380.84284846,412.53710156)(380.88285278,412.52710297)
\lineto(381.00285278,412.49710297)
\curveto(381.05284825,412.4971016)(381.0978482,412.5021016)(381.13785278,412.51210297)
\lineto(381.25785278,412.51210297)
\curveto(381.33784796,412.53210157)(381.41784788,412.54710155)(381.49785278,412.55710297)
\curveto(381.57784772,412.56710153)(381.65284765,412.58710151)(381.72285278,412.61710297)
\curveto(381.98284732,412.71710138)(382.19284711,412.85210125)(382.35285278,413.02210297)
\curveto(382.51284679,413.19210091)(382.64784665,413.4021007)(382.75785278,413.65210297)
\curveto(382.7978465,413.75210035)(382.82784647,413.85210025)(382.84785278,413.95210297)
\curveto(382.86784643,414.05210005)(382.89284641,414.15709994)(382.92285278,414.26710297)
\curveto(382.93284637,414.30709979)(382.93784636,414.34209976)(382.93785278,414.37210297)
\curveto(382.93784636,414.41209969)(382.94284636,414.45209965)(382.95285278,414.49210297)
\lineto(382.95285278,414.62710297)
\curveto(382.95284635,414.67709942)(382.95784634,414.72709937)(382.96785278,414.77710297)
}
}
{
\newrgbcolor{curcolor}{0 0 0}
\pscustom[linestyle=none,fillstyle=solid,fillcolor=curcolor]
{
\newpath
\moveto(386.97777466,421.82710297)
\lineto(388.07277466,421.82710297)
\curveto(388.17277217,421.82709227)(388.26777208,421.82209228)(388.35777466,421.81210297)
\curveto(388.4477719,421.8020923)(388.51777183,421.77209233)(388.56777466,421.72210297)
\curveto(388.62777172,421.65209245)(388.65777169,421.55709254)(388.65777466,421.43710297)
\curveto(388.66777168,421.32709277)(388.67277167,421.21209289)(388.67277466,421.09210297)
\lineto(388.67277466,419.75710297)
\lineto(388.67277466,414.37210297)
\lineto(388.67277466,412.07710297)
\lineto(388.67277466,411.65710297)
\curveto(388.68277166,411.50710259)(388.66277168,411.39210271)(388.61277466,411.31210297)
\curveto(388.56277178,411.23210287)(388.47277187,411.17710292)(388.34277466,411.14710297)
\curveto(388.28277206,411.12710297)(388.21277213,411.12210298)(388.13277466,411.13210297)
\curveto(388.06277228,411.14210296)(387.99277235,411.14710295)(387.92277466,411.14710297)
\lineto(387.20277466,411.14710297)
\curveto(387.09277325,411.14710295)(386.99277335,411.15210295)(386.90277466,411.16210297)
\curveto(386.81277353,411.17210293)(386.73777361,411.2021029)(386.67777466,411.25210297)
\curveto(386.61777373,411.3021028)(386.58277376,411.37710272)(386.57277466,411.47710297)
\lineto(386.57277466,411.80710297)
\lineto(386.57277466,413.14210297)
\lineto(386.57277466,418.76710297)
\lineto(386.57277466,420.80710297)
\curveto(386.57277377,420.93709316)(386.56777378,421.09209301)(386.55777466,421.27210297)
\curveto(386.55777379,421.45209265)(386.58277376,421.58209252)(386.63277466,421.66210297)
\curveto(386.65277369,421.7020924)(386.67777367,421.73209237)(386.70777466,421.75210297)
\lineto(386.82777466,421.81210297)
\curveto(386.8477735,421.81209229)(386.87277347,421.81209229)(386.90277466,421.81210297)
\curveto(386.93277341,421.82209228)(386.95777339,421.82709227)(386.97777466,421.82710297)
}
}
{
\newrgbcolor{curcolor}{0 0 0}
\pscustom[linestyle=none,fillstyle=solid,fillcolor=curcolor]
{
}
}
{
\newrgbcolor{curcolor}{0 0 0}
\pscustom[linestyle=none,fillstyle=solid,fillcolor=curcolor]
{
\newpath
\moveto(402.08511841,411.98710297)
\lineto(402.08511841,411.56710297)
\curveto(402.08511004,411.43710266)(402.05511007,411.33210277)(401.99511841,411.25210297)
\curveto(401.94511018,411.2021029)(401.88011024,411.16710293)(401.80011841,411.14710297)
\curveto(401.7201104,411.13710296)(401.63011049,411.13210297)(401.53011841,411.13210297)
\lineto(400.70511841,411.13210297)
\lineto(400.42011841,411.13210297)
\curveto(400.34011178,411.14210296)(400.27511185,411.16710293)(400.22511841,411.20710297)
\curveto(400.15511197,411.25710284)(400.11511201,411.32210278)(400.10511841,411.40210297)
\curveto(400.09511203,411.48210262)(400.07511205,411.56210254)(400.04511841,411.64210297)
\curveto(400.0251121,411.66210244)(400.00511212,411.67710242)(399.98511841,411.68710297)
\curveto(399.97511215,411.70710239)(399.96011216,411.72710237)(399.94011841,411.74710297)
\curveto(399.83011229,411.74710235)(399.75011237,411.72210238)(399.70011841,411.67210297)
\lineto(399.55011841,411.52210297)
\curveto(399.48011264,411.47210263)(399.41511271,411.42710267)(399.35511841,411.38710297)
\curveto(399.29511283,411.35710274)(399.23011289,411.31710278)(399.16011841,411.26710297)
\curveto(399.120113,411.24710285)(399.07511305,411.22710287)(399.02511841,411.20710297)
\curveto(398.98511314,411.18710291)(398.94011318,411.16710293)(398.89011841,411.14710297)
\curveto(398.75011337,411.097103)(398.60011352,411.05210305)(398.44011841,411.01210297)
\curveto(398.39011373,410.99210311)(398.34511378,410.98210312)(398.30511841,410.98210297)
\curveto(398.26511386,410.98210312)(398.2251139,410.97710312)(398.18511841,410.96710297)
\lineto(398.05011841,410.96710297)
\curveto(398.0201141,410.95710314)(397.98011414,410.95210315)(397.93011841,410.95210297)
\lineto(397.79511841,410.95210297)
\curveto(397.73511439,410.93210317)(397.64511448,410.92710317)(397.52511841,410.93710297)
\curveto(397.40511472,410.93710316)(397.3201148,410.94710315)(397.27011841,410.96710297)
\curveto(397.20011492,410.98710311)(397.13511499,410.9971031)(397.07511841,410.99710297)
\curveto(397.0251151,410.98710311)(396.97011515,410.99210311)(396.91011841,411.01210297)
\lineto(396.55011841,411.13210297)
\curveto(396.44011568,411.16210294)(396.33011579,411.2021029)(396.22011841,411.25210297)
\curveto(395.87011625,411.4021027)(395.55511657,411.63210247)(395.27511841,411.94210297)
\curveto(395.00511712,412.26210184)(394.79011733,412.5971015)(394.63011841,412.94710297)
\curveto(394.58011754,413.05710104)(394.54011758,413.16210094)(394.51011841,413.26210297)
\curveto(394.48011764,413.37210073)(394.44511768,413.48210062)(394.40511841,413.59210297)
\curveto(394.39511773,413.63210047)(394.39011773,413.66710043)(394.39011841,413.69710297)
\curveto(394.39011773,413.73710036)(394.38011774,413.78210032)(394.36011841,413.83210297)
\curveto(394.34011778,413.91210019)(394.3201178,413.9971001)(394.30011841,414.08710297)
\curveto(394.29011783,414.18709991)(394.27511785,414.28709981)(394.25511841,414.38710297)
\curveto(394.24511788,414.41709968)(394.24011788,414.45209965)(394.24011841,414.49210297)
\curveto(394.25011787,414.53209957)(394.25011787,414.56709953)(394.24011841,414.59710297)
\lineto(394.24011841,414.73210297)
\curveto(394.24011788,414.78209932)(394.23511789,414.83209927)(394.22511841,414.88210297)
\curveto(394.21511791,414.93209917)(394.21011791,414.98709911)(394.21011841,415.04710297)
\curveto(394.21011791,415.11709898)(394.21511791,415.17209893)(394.22511841,415.21210297)
\curveto(394.23511789,415.26209884)(394.24011788,415.30709879)(394.24011841,415.34710297)
\lineto(394.24011841,415.49710297)
\curveto(394.25011787,415.54709855)(394.25011787,415.59209851)(394.24011841,415.63210297)
\curveto(394.24011788,415.68209842)(394.25011787,415.73209837)(394.27011841,415.78210297)
\curveto(394.29011783,415.89209821)(394.30511782,415.9970981)(394.31511841,416.09710297)
\curveto(394.33511779,416.1970979)(394.36011776,416.2970978)(394.39011841,416.39710297)
\curveto(394.43011769,416.51709758)(394.46511766,416.63209747)(394.49511841,416.74210297)
\curveto(394.5251176,416.85209725)(394.56511756,416.96209714)(394.61511841,417.07210297)
\curveto(394.75511737,417.37209673)(394.93011719,417.65709644)(395.14011841,417.92710297)
\curveto(395.16011696,417.95709614)(395.18511694,417.98209612)(395.21511841,418.00210297)
\curveto(395.25511687,418.03209607)(395.28511684,418.06209604)(395.30511841,418.09210297)
\curveto(395.34511678,418.14209596)(395.38511674,418.18709591)(395.42511841,418.22710297)
\curveto(395.46511666,418.26709583)(395.51011661,418.30709579)(395.56011841,418.34710297)
\curveto(395.60011652,418.36709573)(395.63511649,418.39209571)(395.66511841,418.42210297)
\curveto(395.69511643,418.46209564)(395.73011639,418.49209561)(395.77011841,418.51210297)
\curveto(396.0201161,418.68209542)(396.31011581,418.82209528)(396.64011841,418.93210297)
\curveto(396.71011541,418.95209515)(396.78011534,418.96709513)(396.85011841,418.97710297)
\curveto(396.93011519,418.98709511)(397.01011511,419.0020951)(397.09011841,419.02210297)
\curveto(397.16011496,419.04209506)(397.25011487,419.05209505)(397.36011841,419.05210297)
\curveto(397.47011465,419.06209504)(397.58011454,419.06709503)(397.69011841,419.06710297)
\curveto(397.80011432,419.06709503)(397.90511422,419.06209504)(398.00511841,419.05210297)
\curveto(398.11511401,419.04209506)(398.20511392,419.02709507)(398.27511841,419.00710297)
\curveto(398.4251137,418.95709514)(398.57011355,418.91209519)(398.71011841,418.87210297)
\curveto(398.85011327,418.83209527)(398.98011314,418.77709532)(399.10011841,418.70710297)
\curveto(399.17011295,418.65709544)(399.23511289,418.60709549)(399.29511841,418.55710297)
\curveto(399.35511277,418.51709558)(399.4201127,418.47209563)(399.49011841,418.42210297)
\curveto(399.53011259,418.39209571)(399.58511254,418.35209575)(399.65511841,418.30210297)
\curveto(399.73511239,418.25209585)(399.81011231,418.25209585)(399.88011841,418.30210297)
\curveto(399.9201122,418.32209578)(399.94011218,418.35709574)(399.94011841,418.40710297)
\curveto(399.94011218,418.45709564)(399.95011217,418.50709559)(399.97011841,418.55710297)
\lineto(399.97011841,418.70710297)
\curveto(399.98011214,418.73709536)(399.98511214,418.77209533)(399.98511841,418.81210297)
\lineto(399.98511841,418.93210297)
\lineto(399.98511841,420.97210297)
\curveto(399.98511214,421.08209302)(399.98011214,421.2020929)(399.97011841,421.33210297)
\curveto(399.97011215,421.47209263)(399.99511213,421.57709252)(400.04511841,421.64710297)
\curveto(400.08511204,421.72709237)(400.16011196,421.77709232)(400.27011841,421.79710297)
\curveto(400.29011183,421.80709229)(400.31011181,421.80709229)(400.33011841,421.79710297)
\curveto(400.35011177,421.7970923)(400.37011175,421.8020923)(400.39011841,421.81210297)
\lineto(401.45511841,421.81210297)
\curveto(401.57511055,421.81209229)(401.68511044,421.80709229)(401.78511841,421.79710297)
\curveto(401.88511024,421.78709231)(401.96011016,421.74709235)(402.01011841,421.67710297)
\curveto(402.06011006,421.5970925)(402.08511004,421.49209261)(402.08511841,421.36210297)
\lineto(402.08511841,421.00210297)
\lineto(402.08511841,411.98710297)
\moveto(400.04511841,414.92710297)
\curveto(400.05511207,414.96709913)(400.05511207,415.00709909)(400.04511841,415.04710297)
\lineto(400.04511841,415.18210297)
\curveto(400.04511208,415.28209882)(400.04011208,415.38209872)(400.03011841,415.48210297)
\curveto(400.0201121,415.58209852)(400.00511212,415.67209843)(399.98511841,415.75210297)
\curveto(399.96511216,415.86209824)(399.94511218,415.96209814)(399.92511841,416.05210297)
\curveto(399.91511221,416.14209796)(399.89011223,416.22709787)(399.85011841,416.30710297)
\curveto(399.71011241,416.66709743)(399.50511262,416.95209715)(399.23511841,417.16210297)
\curveto(398.97511315,417.37209673)(398.59511353,417.47709662)(398.09511841,417.47710297)
\curveto(398.03511409,417.47709662)(397.95511417,417.46709663)(397.85511841,417.44710297)
\curveto(397.77511435,417.42709667)(397.70011442,417.40709669)(397.63011841,417.38710297)
\curveto(397.57011455,417.37709672)(397.51011461,417.35709674)(397.45011841,417.32710297)
\curveto(397.18011494,417.21709688)(396.97011515,417.04709705)(396.82011841,416.81710297)
\curveto(396.67011545,416.58709751)(396.55011557,416.32709777)(396.46011841,416.03710297)
\curveto(396.43011569,415.93709816)(396.41011571,415.83709826)(396.40011841,415.73710297)
\curveto(396.39011573,415.63709846)(396.37011575,415.53209857)(396.34011841,415.42210297)
\lineto(396.34011841,415.21210297)
\curveto(396.3201158,415.12209898)(396.31511581,414.9970991)(396.32511841,414.83710297)
\curveto(396.33511579,414.68709941)(396.35011577,414.57709952)(396.37011841,414.50710297)
\lineto(396.37011841,414.41710297)
\curveto(396.38011574,414.3970997)(396.38511574,414.37709972)(396.38511841,414.35710297)
\curveto(396.40511572,414.27709982)(396.4201157,414.2020999)(396.43011841,414.13210297)
\curveto(396.45011567,414.06210004)(396.47011565,413.98710011)(396.49011841,413.90710297)
\curveto(396.66011546,413.38710071)(396.95011517,413.0021011)(397.36011841,412.75210297)
\curveto(397.49011463,412.66210144)(397.67011445,412.59210151)(397.90011841,412.54210297)
\curveto(397.94011418,412.53210157)(398.00011412,412.52710157)(398.08011841,412.52710297)
\curveto(398.11011401,412.51710158)(398.15511397,412.50710159)(398.21511841,412.49710297)
\curveto(398.28511384,412.4971016)(398.34011378,412.5021016)(398.38011841,412.51210297)
\curveto(398.46011366,412.53210157)(398.54011358,412.54710155)(398.62011841,412.55710297)
\curveto(398.70011342,412.56710153)(398.78011334,412.58710151)(398.86011841,412.61710297)
\curveto(399.11011301,412.72710137)(399.31011281,412.86710123)(399.46011841,413.03710297)
\curveto(399.61011251,413.20710089)(399.74011238,413.42210068)(399.85011841,413.68210297)
\curveto(399.89011223,413.77210033)(399.9201122,413.86210024)(399.94011841,413.95210297)
\curveto(399.96011216,414.05210005)(399.98011214,414.15709994)(400.00011841,414.26710297)
\curveto(400.01011211,414.31709978)(400.01011211,414.36209974)(400.00011841,414.40210297)
\curveto(400.00011212,414.45209965)(400.01011211,414.5020996)(400.03011841,414.55210297)
\curveto(400.04011208,414.58209952)(400.04511208,414.61709948)(400.04511841,414.65710297)
\lineto(400.04511841,414.79210297)
\lineto(400.04511841,414.92710297)
}
}
{
\newrgbcolor{curcolor}{0 0 0}
\pscustom[linestyle=none,fillstyle=solid,fillcolor=curcolor]
{
\newpath
\moveto(411.03004028,415.07710297)
\curveto(411.05003212,414.9970991)(411.05003212,414.90709919)(411.03004028,414.80710297)
\curveto(411.01003216,414.70709939)(410.97503219,414.64209946)(410.92504028,414.61210297)
\curveto(410.87503229,414.57209953)(410.80003237,414.54209956)(410.70004028,414.52210297)
\curveto(410.61003256,414.51209959)(410.50503266,414.5020996)(410.38504028,414.49210297)
\lineto(410.04004028,414.49210297)
\curveto(409.93003324,414.5020996)(409.83003334,414.50709959)(409.74004028,414.50710297)
\lineto(406.08004028,414.50710297)
\lineto(405.87004028,414.50710297)
\curveto(405.81003736,414.50709959)(405.75503741,414.4970996)(405.70504028,414.47710297)
\curveto(405.62503754,414.43709966)(405.57503759,414.3970997)(405.55504028,414.35710297)
\curveto(405.53503763,414.33709976)(405.51503765,414.2970998)(405.49504028,414.23710297)
\curveto(405.47503769,414.18709991)(405.4700377,414.13709996)(405.48004028,414.08710297)
\curveto(405.50003767,414.02710007)(405.51003766,413.96710013)(405.51004028,413.90710297)
\curveto(405.52003765,413.85710024)(405.53503763,413.8021003)(405.55504028,413.74210297)
\curveto(405.63503753,413.5021006)(405.73003744,413.3021008)(405.84004028,413.14210297)
\curveto(405.96003721,412.99210111)(406.12003705,412.85710124)(406.32004028,412.73710297)
\curveto(406.40003677,412.68710141)(406.48003669,412.65210145)(406.56004028,412.63210297)
\curveto(406.65003652,412.62210148)(406.74003643,412.6021015)(406.83004028,412.57210297)
\curveto(406.91003626,412.55210155)(407.02003615,412.53710156)(407.16004028,412.52710297)
\curveto(407.30003587,412.51710158)(407.42003575,412.52210158)(407.52004028,412.54210297)
\lineto(407.65504028,412.54210297)
\curveto(407.75503541,412.56210154)(407.84503532,412.58210152)(407.92504028,412.60210297)
\curveto(408.01503515,412.63210147)(408.10003507,412.66210144)(408.18004028,412.69210297)
\curveto(408.28003489,412.74210136)(408.39003478,412.80710129)(408.51004028,412.88710297)
\curveto(408.64003453,412.96710113)(408.73503443,413.04710105)(408.79504028,413.12710297)
\curveto(408.84503432,413.1971009)(408.89503427,413.26210084)(408.94504028,413.32210297)
\curveto(409.00503416,413.39210071)(409.07503409,413.44210066)(409.15504028,413.47210297)
\curveto(409.25503391,413.52210058)(409.38003379,413.54210056)(409.53004028,413.53210297)
\lineto(409.96504028,413.53210297)
\lineto(410.14504028,413.53210297)
\curveto(410.21503295,413.54210056)(410.27503289,413.53710056)(410.32504028,413.51710297)
\lineto(410.47504028,413.51710297)
\curveto(410.57503259,413.4971006)(410.64503252,413.47210063)(410.68504028,413.44210297)
\curveto(410.72503244,413.42210068)(410.74503242,413.37710072)(410.74504028,413.30710297)
\curveto(410.75503241,413.23710086)(410.75003242,413.17710092)(410.73004028,413.12710297)
\curveto(410.68003249,412.98710111)(410.62503254,412.86210124)(410.56504028,412.75210297)
\curveto(410.50503266,412.64210146)(410.43503273,412.53210157)(410.35504028,412.42210297)
\curveto(410.13503303,412.09210201)(409.88503328,411.82710227)(409.60504028,411.62710297)
\curveto(409.32503384,411.42710267)(408.97503419,411.25710284)(408.55504028,411.11710297)
\curveto(408.44503472,411.07710302)(408.33503483,411.05210305)(408.22504028,411.04210297)
\curveto(408.11503505,411.03210307)(408.00003517,411.01210309)(407.88004028,410.98210297)
\curveto(407.84003533,410.97210313)(407.79503537,410.97210313)(407.74504028,410.98210297)
\curveto(407.70503546,410.98210312)(407.6650355,410.97710312)(407.62504028,410.96710297)
\lineto(407.46004028,410.96710297)
\curveto(407.41003576,410.94710315)(407.35003582,410.94210316)(407.28004028,410.95210297)
\curveto(407.22003595,410.95210315)(407.165036,410.95710314)(407.11504028,410.96710297)
\curveto(407.03503613,410.97710312)(406.9650362,410.97710312)(406.90504028,410.96710297)
\curveto(406.84503632,410.95710314)(406.78003639,410.96210314)(406.71004028,410.98210297)
\curveto(406.66003651,411.0021031)(406.60503656,411.01210309)(406.54504028,411.01210297)
\curveto(406.48503668,411.01210309)(406.43003674,411.02210308)(406.38004028,411.04210297)
\curveto(406.2700369,411.06210304)(406.16003701,411.08710301)(406.05004028,411.11710297)
\curveto(405.94003723,411.13710296)(405.84003733,411.17210293)(405.75004028,411.22210297)
\curveto(405.64003753,411.26210284)(405.53503763,411.2971028)(405.43504028,411.32710297)
\curveto(405.34503782,411.36710273)(405.26003791,411.41210269)(405.18004028,411.46210297)
\curveto(404.86003831,411.66210244)(404.57503859,411.89210221)(404.32504028,412.15210297)
\curveto(404.07503909,412.42210168)(403.8700393,412.73210137)(403.71004028,413.08210297)
\curveto(403.66003951,413.19210091)(403.62003955,413.3021008)(403.59004028,413.41210297)
\curveto(403.56003961,413.53210057)(403.52003965,413.65210045)(403.47004028,413.77210297)
\curveto(403.46003971,413.81210029)(403.45503971,413.84710025)(403.45504028,413.87710297)
\curveto(403.45503971,413.91710018)(403.45003972,413.95710014)(403.44004028,413.99710297)
\curveto(403.40003977,414.11709998)(403.37503979,414.24709985)(403.36504028,414.38710297)
\lineto(403.33504028,414.80710297)
\curveto(403.33503983,414.85709924)(403.33003984,414.91209919)(403.32004028,414.97210297)
\curveto(403.32003985,415.03209907)(403.32503984,415.08709901)(403.33504028,415.13710297)
\lineto(403.33504028,415.31710297)
\lineto(403.38004028,415.67710297)
\curveto(403.42003975,415.84709825)(403.45503971,416.01209809)(403.48504028,416.17210297)
\curveto(403.51503965,416.33209777)(403.56003961,416.48209762)(403.62004028,416.62210297)
\curveto(404.05003912,417.66209644)(404.78003839,418.3970957)(405.81004028,418.82710297)
\curveto(405.95003722,418.88709521)(406.09003708,418.92709517)(406.23004028,418.94710297)
\curveto(406.38003679,418.97709512)(406.53503663,419.01209509)(406.69504028,419.05210297)
\curveto(406.77503639,419.06209504)(406.85003632,419.06709503)(406.92004028,419.06710297)
\curveto(406.99003618,419.06709503)(407.0650361,419.07209503)(407.14504028,419.08210297)
\curveto(407.65503551,419.09209501)(408.09003508,419.03209507)(408.45004028,418.90210297)
\curveto(408.82003435,418.78209532)(409.15003402,418.62209548)(409.44004028,418.42210297)
\curveto(409.53003364,418.36209574)(409.62003355,418.29209581)(409.71004028,418.21210297)
\curveto(409.80003337,418.14209596)(409.88003329,418.06709603)(409.95004028,417.98710297)
\curveto(409.98003319,417.93709616)(410.02003315,417.8970962)(410.07004028,417.86710297)
\curveto(410.15003302,417.75709634)(410.22503294,417.64209646)(410.29504028,417.52210297)
\curveto(410.3650328,417.41209669)(410.44003273,417.2970968)(410.52004028,417.17710297)
\curveto(410.5700326,417.08709701)(410.61003256,416.99209711)(410.64004028,416.89210297)
\curveto(410.68003249,416.8020973)(410.72003245,416.7020974)(410.76004028,416.59210297)
\curveto(410.81003236,416.46209764)(410.85003232,416.32709777)(410.88004028,416.18710297)
\curveto(410.91003226,416.04709805)(410.94503222,415.90709819)(410.98504028,415.76710297)
\curveto(411.00503216,415.68709841)(411.01003216,415.5970985)(411.00004028,415.49710297)
\curveto(411.00003217,415.40709869)(411.01003216,415.32209878)(411.03004028,415.24210297)
\lineto(411.03004028,415.07710297)
\moveto(408.78004028,415.96210297)
\curveto(408.85003432,416.06209804)(408.85503431,416.18209792)(408.79504028,416.32210297)
\curveto(408.74503442,416.47209763)(408.70503446,416.58209752)(408.67504028,416.65210297)
\curveto(408.53503463,416.92209718)(408.35003482,417.12709697)(408.12004028,417.26710297)
\curveto(407.89003528,417.41709668)(407.5700356,417.4970966)(407.16004028,417.50710297)
\curveto(407.13003604,417.48709661)(407.09503607,417.48209662)(407.05504028,417.49210297)
\curveto(407.01503615,417.5020966)(406.98003619,417.5020966)(406.95004028,417.49210297)
\curveto(406.90003627,417.47209663)(406.84503632,417.45709664)(406.78504028,417.44710297)
\curveto(406.72503644,417.44709665)(406.6700365,417.43709666)(406.62004028,417.41710297)
\curveto(406.18003699,417.27709682)(405.85503731,417.0020971)(405.64504028,416.59210297)
\curveto(405.62503754,416.55209755)(405.60003757,416.4970976)(405.57004028,416.42710297)
\curveto(405.55003762,416.36709773)(405.53503763,416.3020978)(405.52504028,416.23210297)
\curveto(405.51503765,416.17209793)(405.51503765,416.11209799)(405.52504028,416.05210297)
\curveto(405.54503762,415.99209811)(405.58003759,415.94209816)(405.63004028,415.90210297)
\curveto(405.71003746,415.85209825)(405.82003735,415.82709827)(405.96004028,415.82710297)
\lineto(406.36504028,415.82710297)
\lineto(408.03004028,415.82710297)
\lineto(408.46504028,415.82710297)
\curveto(408.62503454,415.83709826)(408.73003444,415.88209822)(408.78004028,415.96210297)
}
}
{
\newrgbcolor{curcolor}{0 0 0}
\pscustom[linestyle=none,fillstyle=solid,fillcolor=curcolor]
{
\newpath
\moveto(415.24832153,421.82710297)
\curveto(415.33831769,421.82709227)(415.43831759,421.82709227)(415.54832153,421.82710297)
\curveto(415.66831736,421.82709227)(415.78331725,421.82209228)(415.89332153,421.81210297)
\curveto(416.01331702,421.8020923)(416.11831691,421.78209232)(416.20832153,421.75210297)
\curveto(416.29831673,421.73209237)(416.35831667,421.6970924)(416.38832153,421.64710297)
\curveto(416.44831658,421.56709253)(416.47831655,421.45209265)(416.47832153,421.30210297)
\lineto(416.47832153,420.89710297)
\curveto(416.47831655,420.7970933)(416.47331656,420.6970934)(416.46332153,420.59710297)
\curveto(416.46331657,420.4970936)(416.44331659,420.42209368)(416.40332153,420.37210297)
\curveto(416.36331667,420.31209379)(416.31331672,420.27209383)(416.25332153,420.25210297)
\curveto(416.19331684,420.24209386)(416.12331691,420.23709386)(416.04332153,420.23710297)
\lineto(415.81832153,420.23710297)
\curveto(415.74831728,420.24709385)(415.67831735,420.24709385)(415.60832153,420.23710297)
\curveto(415.4283176,420.1970939)(415.28831774,420.14709395)(415.18832153,420.08710297)
\curveto(415.08831794,420.03709406)(415.00831802,419.92709417)(414.94832153,419.75710297)
\curveto(414.9283181,419.72709437)(414.91831811,419.6970944)(414.91832153,419.66710297)
\curveto(414.9283181,419.64709445)(414.9283181,419.62209448)(414.91832153,419.59210297)
\curveto(414.90831812,419.55209455)(414.89831813,419.49209461)(414.88832153,419.41210297)
\curveto(414.87831815,419.33209477)(414.87831815,419.26709483)(414.88832153,419.21710297)
\curveto(414.90831812,419.14709495)(414.9333181,419.08709501)(414.96332153,419.03710297)
\curveto(414.99331804,418.98709511)(415.03831799,418.94709515)(415.09832153,418.91710297)
\curveto(415.19831783,418.86709523)(415.31831771,418.85209525)(415.45832153,418.87210297)
\curveto(415.59831743,418.89209521)(415.7283173,418.89209521)(415.84832153,418.87210297)
\curveto(415.89831713,418.86209524)(415.93831709,418.85709524)(415.96832153,418.85710297)
\curveto(416.00831702,418.86709523)(416.04831698,418.86709523)(416.08832153,418.85710297)
\curveto(416.17831685,418.81709528)(416.24331679,418.77209533)(416.28332153,418.72210297)
\curveto(416.30331673,418.69209541)(416.31831671,418.64209546)(416.32832153,418.57210297)
\curveto(416.33831669,418.51209559)(416.34831668,418.44209566)(416.35832153,418.36210297)
\curveto(416.36831666,418.29209581)(416.36831666,418.21709588)(416.35832153,418.13710297)
\curveto(416.35831667,418.06709603)(416.35331668,418.01209609)(416.34332153,417.97210297)
\curveto(416.3333167,417.93209617)(416.3333167,417.89209621)(416.34332153,417.85210297)
\curveto(416.35331668,417.82209628)(416.34831668,417.78709631)(416.32832153,417.74710297)
\curveto(416.30831672,417.62709647)(416.24831678,417.55209655)(416.14832153,417.52210297)
\curveto(416.06831696,417.48209662)(415.97331706,417.46209664)(415.86332153,417.46210297)
\curveto(415.75331728,417.47209663)(415.64331739,417.47709662)(415.53332153,417.47710297)
\lineto(415.42832153,417.47710297)
\curveto(415.38831764,417.47709662)(415.35331768,417.47209663)(415.32332153,417.46210297)
\lineto(415.20332153,417.46210297)
\curveto(415.033318,417.42209668)(414.9283181,417.31209679)(414.88832153,417.13210297)
\curveto(414.86831816,417.07209703)(414.86331817,417.0020971)(414.87332153,416.92210297)
\curveto(414.88331815,416.84209726)(414.88831814,416.76209734)(414.88832153,416.68210297)
\lineto(414.88832153,415.76710297)
\lineto(414.88832153,412.84210297)
\lineto(414.88832153,412.13710297)
\lineto(414.88832153,411.94210297)
\curveto(414.89831813,411.88210222)(414.89331814,411.82710227)(414.87332153,411.77710297)
\lineto(414.87332153,411.61210297)
\curveto(414.87331816,411.45210265)(414.84831818,411.33710276)(414.79832153,411.26710297)
\curveto(414.77831825,411.23710286)(414.74331829,411.21210289)(414.69332153,411.19210297)
\curveto(414.64331839,411.18210292)(414.59331844,411.16710293)(414.54332153,411.14710297)
\lineto(414.46832153,411.14710297)
\curveto(414.41831861,411.13710296)(414.36331867,411.13210297)(414.30332153,411.13210297)
\curveto(414.24331879,411.14210296)(414.18831884,411.14710295)(414.13832153,411.14710297)
\lineto(413.47832153,411.14710297)
\curveto(413.40831962,411.14710295)(413.3333197,411.14210296)(413.25332153,411.13210297)
\curveto(413.18331985,411.13210297)(413.12331991,411.14210296)(413.07332153,411.16210297)
\curveto(412.95332008,411.19210291)(412.87332016,411.24210286)(412.83332153,411.31210297)
\curveto(412.80332023,411.36210274)(412.78332025,411.42710267)(412.77332153,411.50710297)
\lineto(412.77332153,411.74710297)
\lineto(412.77332153,412.52710297)
\lineto(412.77332153,416.72710297)
\curveto(412.77332026,416.8970972)(412.76332027,417.04209706)(412.74332153,417.16210297)
\curveto(412.72332031,417.29209681)(412.65332038,417.38209672)(412.53332153,417.43210297)
\curveto(412.42332061,417.48209662)(412.28832074,417.49209661)(412.12832153,417.46210297)
\curveto(411.96832106,417.44209666)(411.8333212,417.45709664)(411.72332153,417.50710297)
\curveto(411.61332142,417.55709654)(411.54332149,417.64209646)(411.51332153,417.76210297)
\curveto(411.49332154,417.81209629)(411.48832154,417.87209623)(411.49832153,417.94210297)
\lineto(411.49832153,418.15210297)
\curveto(411.49832153,418.33209577)(411.50832152,418.48209562)(411.52832153,418.60210297)
\curveto(411.54832148,418.72209538)(411.6333214,418.80709529)(411.78332153,418.85710297)
\curveto(411.86332117,418.87709522)(411.94832108,418.88709521)(412.03832153,418.88710297)
\lineto(412.29332153,418.88710297)
\curveto(412.38332065,418.88709521)(412.46332057,418.89209521)(412.53332153,418.90210297)
\curveto(412.60332043,418.92209518)(412.65832037,418.96209514)(412.69832153,419.02210297)
\curveto(412.76832026,419.12209498)(412.79332024,419.24709485)(412.77332153,419.39710297)
\curveto(412.76332027,419.55709454)(412.77332026,419.70709439)(412.80332153,419.84710297)
\curveto(412.81332022,419.88709421)(412.81832021,419.92709417)(412.81832153,419.96710297)
\curveto(412.8283202,420.00709409)(412.83832019,420.05209405)(412.84832153,420.10210297)
\curveto(412.88832014,420.24209386)(412.9283201,420.36709373)(412.96832153,420.47710297)
\curveto(413.00832002,420.5970935)(413.06331997,420.70709339)(413.13332153,420.80710297)
\curveto(413.27331976,421.04709305)(413.45831957,421.23709286)(413.68832153,421.37710297)
\curveto(413.91831911,421.52709257)(414.17831885,421.64209246)(414.46832153,421.72210297)
\curveto(414.54831848,421.75209235)(414.6333184,421.76709233)(414.72332153,421.76710297)
\curveto(414.81331822,421.77709232)(414.90331813,421.79209231)(414.99332153,421.81210297)
\curveto(415.02331801,421.82209228)(415.06831796,421.82209228)(415.12832153,421.81210297)
\curveto(415.18831784,421.8020923)(415.2283178,421.80709229)(415.24832153,421.82710297)
\moveto(419.67332153,421.72210297)
\curveto(419.62331341,421.77209233)(419.55331348,421.8020923)(419.46332153,421.81210297)
\curveto(419.37331366,421.82209228)(419.27831375,421.82709227)(419.17832153,421.82710297)
\lineto(418.08332153,421.82710297)
\curveto(418.06331497,421.82709227)(418.03831499,421.82209228)(418.00832153,421.81210297)
\curveto(417.97831505,421.81209229)(417.95331508,421.81209229)(417.93332153,421.81210297)
\lineto(417.81332153,421.75210297)
\curveto(417.78331525,421.73209237)(417.75831527,421.7020924)(417.73832153,421.66210297)
\curveto(417.70831532,421.61209249)(417.68831534,421.55209255)(417.67832153,421.48210297)
\lineto(417.67832153,421.27210297)
\curveto(417.67831535,421.19209291)(417.67331536,421.097093)(417.66332153,420.98710297)
\lineto(417.66332153,420.65710297)
\curveto(417.67331536,420.55709354)(417.68331535,420.46209364)(417.69332153,420.37210297)
\curveto(417.71331532,420.29209381)(417.74331529,420.23709386)(417.78332153,420.20710297)
\curveto(417.84331519,420.15709394)(417.91331512,420.12709397)(417.99332153,420.11710297)
\curveto(418.08331495,420.10709399)(418.18331485,420.102094)(418.29332153,420.10210297)
\lineto(419.13332153,420.10210297)
\curveto(419.24331379,420.102094)(419.34331369,420.10709399)(419.43332153,420.11710297)
\curveto(419.5333135,420.12709397)(419.61331342,420.15709394)(419.67332153,420.20710297)
\curveto(419.74331329,420.25709384)(419.77831325,420.35209375)(419.77832153,420.49210297)
\lineto(419.77832153,420.89710297)
\lineto(419.77832153,421.36210297)
\curveto(419.77831325,421.52209258)(419.74331329,421.64209246)(419.67332153,421.72210297)
\moveto(419.77832153,418.24210297)
\curveto(419.77831325,418.34209576)(419.77331326,418.43709566)(419.76332153,418.52710297)
\curveto(419.76331327,418.61709548)(419.74331329,418.68709541)(419.70332153,418.73710297)
\curveto(419.67331336,418.78709531)(419.6333134,418.81709528)(419.58332153,418.82710297)
\curveto(419.5333135,418.83709526)(419.47831355,418.85209525)(419.41832153,418.87210297)
\lineto(419.29832153,418.87210297)
\curveto(419.24831378,418.88209522)(419.18331385,418.88209522)(419.10332153,418.87210297)
\lineto(418.90832153,418.87210297)
\lineto(418.15832153,418.87210297)
\curveto(418.13831489,418.86209524)(418.10831492,418.85709524)(418.06832153,418.85710297)
\curveto(418.03831499,418.86709523)(418.01331502,418.86709523)(417.99332153,418.85710297)
\curveto(417.88331515,418.83709526)(417.79831523,418.79209531)(417.73832153,418.72210297)
\curveto(417.69831533,418.66209544)(417.67831535,418.58209552)(417.67832153,418.48210297)
\lineto(417.67832153,418.18210297)
\lineto(417.67832153,411.85210297)
\lineto(417.67832153,411.50710297)
\curveto(417.68831534,411.40710269)(417.72331531,411.32210278)(417.78332153,411.25210297)
\curveto(417.82331521,411.2021029)(417.88331515,411.17210293)(417.96332153,411.16210297)
\curveto(418.05331498,411.15210295)(418.14331489,411.14710295)(418.23332153,411.14710297)
\lineto(419.07332153,411.14710297)
\curveto(419.15331388,411.14710295)(419.2283138,411.14210296)(419.29832153,411.13210297)
\curveto(419.36831366,411.13210297)(419.4333136,411.14210296)(419.49332153,411.16210297)
\curveto(419.66331337,411.21210289)(419.75331328,411.30710279)(419.76332153,411.44710297)
\curveto(419.77331326,411.58710251)(419.77831325,411.75710234)(419.77832153,411.95710297)
\lineto(419.77832153,418.24210297)
}
}
{
\newrgbcolor{curcolor}{0 0 0}
\pscustom[linestyle=none,fillstyle=solid,fillcolor=curcolor]
{
\newpath
\moveto(425.87324341,419.06710297)
\curveto(426.4732376,419.08709501)(426.9732371,419.0020951)(427.37324341,418.81210297)
\curveto(427.7732363,418.62209548)(428.08823599,418.34209576)(428.31824341,417.97210297)
\curveto(428.38823569,417.86209624)(428.44323563,417.74209636)(428.48324341,417.61210297)
\curveto(428.52323555,417.49209661)(428.56323551,417.36709673)(428.60324341,417.23710297)
\curveto(428.62323545,417.15709694)(428.63323544,417.08209702)(428.63324341,417.01210297)
\curveto(428.64323543,416.94209716)(428.65823542,416.87209723)(428.67824341,416.80210297)
\curveto(428.6782354,416.74209736)(428.68323539,416.7020974)(428.69324341,416.68210297)
\curveto(428.71323536,416.54209756)(428.72323535,416.3970977)(428.72324341,416.24710297)
\lineto(428.72324341,415.81210297)
\lineto(428.72324341,414.47710297)
\lineto(428.72324341,412.04710297)
\curveto(428.72323535,411.85710224)(428.71823536,411.67210243)(428.70824341,411.49210297)
\curveto(428.70823537,411.32210278)(428.63823544,411.21210289)(428.49824341,411.16210297)
\curveto(428.43823564,411.14210296)(428.36823571,411.13210297)(428.28824341,411.13210297)
\lineto(428.04824341,411.13210297)
\lineto(427.23824341,411.13210297)
\curveto(427.11823696,411.13210297)(427.00823707,411.13710296)(426.90824341,411.14710297)
\curveto(426.81823726,411.16710293)(426.74823733,411.21210289)(426.69824341,411.28210297)
\curveto(426.65823742,411.34210276)(426.63323744,411.41710268)(426.62324341,411.50710297)
\lineto(426.62324341,411.82210297)
\lineto(426.62324341,412.87210297)
\lineto(426.62324341,415.10710297)
\curveto(426.62323745,415.47709862)(426.60823747,415.81709828)(426.57824341,416.12710297)
\curveto(426.54823753,416.44709765)(426.45823762,416.71709738)(426.30824341,416.93710297)
\curveto(426.16823791,417.13709696)(425.96323811,417.27709682)(425.69324341,417.35710297)
\curveto(425.64323843,417.37709672)(425.58823849,417.38709671)(425.52824341,417.38710297)
\curveto(425.4782386,417.38709671)(425.42323865,417.3970967)(425.36324341,417.41710297)
\curveto(425.31323876,417.42709667)(425.24823883,417.42709667)(425.16824341,417.41710297)
\curveto(425.09823898,417.41709668)(425.04323903,417.41209669)(425.00324341,417.40210297)
\curveto(424.96323911,417.39209671)(424.92823915,417.38709671)(424.89824341,417.38710297)
\curveto(424.86823921,417.38709671)(424.83823924,417.38209672)(424.80824341,417.37210297)
\curveto(424.5782395,417.31209679)(424.39323968,417.23209687)(424.25324341,417.13210297)
\curveto(423.93324014,416.9020972)(423.74324033,416.56709753)(423.68324341,416.12710297)
\curveto(423.62324045,415.68709841)(423.59324048,415.19209891)(423.59324341,414.64210297)
\lineto(423.59324341,412.76710297)
\lineto(423.59324341,411.85210297)
\lineto(423.59324341,411.58210297)
\curveto(423.59324048,411.49210261)(423.5782405,411.41710268)(423.54824341,411.35710297)
\curveto(423.49824058,411.24710285)(423.41824066,411.18210292)(423.30824341,411.16210297)
\curveto(423.19824088,411.14210296)(423.06324101,411.13210297)(422.90324341,411.13210297)
\lineto(422.15324341,411.13210297)
\curveto(422.04324203,411.13210297)(421.93324214,411.13710296)(421.82324341,411.14710297)
\curveto(421.71324236,411.15710294)(421.63324244,411.19210291)(421.58324341,411.25210297)
\curveto(421.51324256,411.34210276)(421.4782426,411.47210263)(421.47824341,411.64210297)
\curveto(421.48824259,411.81210229)(421.49324258,411.97210213)(421.49324341,412.12210297)
\lineto(421.49324341,414.16210297)
\lineto(421.49324341,417.46210297)
\lineto(421.49324341,418.22710297)
\lineto(421.49324341,418.52710297)
\curveto(421.50324257,418.61709548)(421.53324254,418.69209541)(421.58324341,418.75210297)
\curveto(421.60324247,418.78209532)(421.63324244,418.8020953)(421.67324341,418.81210297)
\curveto(421.72324235,418.83209527)(421.7732423,418.84709525)(421.82324341,418.85710297)
\lineto(421.89824341,418.85710297)
\curveto(421.94824213,418.86709523)(421.99824208,418.87209523)(422.04824341,418.87210297)
\lineto(422.21324341,418.87210297)
\lineto(422.84324341,418.87210297)
\curveto(422.92324115,418.87209523)(422.99824108,418.86709523)(423.06824341,418.85710297)
\curveto(423.14824093,418.85709524)(423.21824086,418.84709525)(423.27824341,418.82710297)
\curveto(423.34824073,418.7970953)(423.39324068,418.75209535)(423.41324341,418.69210297)
\curveto(423.44324063,418.63209547)(423.46824061,418.56209554)(423.48824341,418.48210297)
\curveto(423.49824058,418.44209566)(423.49824058,418.40709569)(423.48824341,418.37710297)
\curveto(423.48824059,418.34709575)(423.49824058,418.31709578)(423.51824341,418.28710297)
\curveto(423.53824054,418.23709586)(423.55324052,418.20709589)(423.56324341,418.19710297)
\curveto(423.58324049,418.18709591)(423.60824047,418.17209593)(423.63824341,418.15210297)
\curveto(423.74824033,418.14209596)(423.83824024,418.17709592)(423.90824341,418.25710297)
\curveto(423.9782401,418.34709575)(424.05324002,418.41709568)(424.13324341,418.46710297)
\curveto(424.40323967,418.66709543)(424.70323937,418.82709527)(425.03324341,418.94710297)
\curveto(425.12323895,418.97709512)(425.21323886,418.9970951)(425.30324341,419.00710297)
\curveto(425.40323867,419.01709508)(425.50823857,419.03209507)(425.61824341,419.05210297)
\curveto(425.64823843,419.06209504)(425.69323838,419.06209504)(425.75324341,419.05210297)
\curveto(425.81323826,419.05209505)(425.85323822,419.05709504)(425.87324341,419.06710297)
}
}
{
\newrgbcolor{curcolor}{0 0 0}
\pscustom[linestyle=none,fillstyle=solid,fillcolor=curcolor]
{
\newpath
\moveto(432.45449341,421.72210297)
\curveto(432.52449046,421.64209246)(432.55949042,421.52209258)(432.55949341,421.36210297)
\lineto(432.55949341,420.89710297)
\lineto(432.55949341,420.49210297)
\curveto(432.55949042,420.35209375)(432.52449046,420.25709384)(432.45449341,420.20710297)
\curveto(432.39449059,420.15709394)(432.31449067,420.12709397)(432.21449341,420.11710297)
\curveto(432.12449086,420.10709399)(432.02449096,420.102094)(431.91449341,420.10210297)
\lineto(431.07449341,420.10210297)
\curveto(430.96449202,420.102094)(430.86449212,420.10709399)(430.77449341,420.11710297)
\curveto(430.69449229,420.12709397)(430.62449236,420.15709394)(430.56449341,420.20710297)
\curveto(430.52449246,420.23709386)(430.49449249,420.29209381)(430.47449341,420.37210297)
\curveto(430.46449252,420.46209364)(430.45449253,420.55709354)(430.44449341,420.65710297)
\lineto(430.44449341,420.98710297)
\curveto(430.45449253,421.097093)(430.45949252,421.19209291)(430.45949341,421.27210297)
\lineto(430.45949341,421.48210297)
\curveto(430.46949251,421.55209255)(430.48949249,421.61209249)(430.51949341,421.66210297)
\curveto(430.53949244,421.7020924)(430.56449242,421.73209237)(430.59449341,421.75210297)
\lineto(430.71449341,421.81210297)
\curveto(430.73449225,421.81209229)(430.75949222,421.81209229)(430.78949341,421.81210297)
\curveto(430.81949216,421.82209228)(430.84449214,421.82709227)(430.86449341,421.82710297)
\lineto(431.95949341,421.82710297)
\curveto(432.05949092,421.82709227)(432.15449083,421.82209228)(432.24449341,421.81210297)
\curveto(432.33449065,421.8020923)(432.40449058,421.77209233)(432.45449341,421.72210297)
\moveto(432.55949341,411.95710297)
\curveto(432.55949042,411.75710234)(432.55449043,411.58710251)(432.54449341,411.44710297)
\curveto(432.53449045,411.30710279)(432.44449054,411.21210289)(432.27449341,411.16210297)
\curveto(432.21449077,411.14210296)(432.14949083,411.13210297)(432.07949341,411.13210297)
\curveto(432.00949097,411.14210296)(431.93449105,411.14710295)(431.85449341,411.14710297)
\lineto(431.01449341,411.14710297)
\curveto(430.92449206,411.14710295)(430.83449215,411.15210295)(430.74449341,411.16210297)
\curveto(430.66449232,411.17210293)(430.60449238,411.2021029)(430.56449341,411.25210297)
\curveto(430.50449248,411.32210278)(430.46949251,411.40710269)(430.45949341,411.50710297)
\lineto(430.45949341,411.85210297)
\lineto(430.45949341,418.18210297)
\lineto(430.45949341,418.48210297)
\curveto(430.45949252,418.58209552)(430.4794925,418.66209544)(430.51949341,418.72210297)
\curveto(430.5794924,418.79209531)(430.66449232,418.83709526)(430.77449341,418.85710297)
\curveto(430.79449219,418.86709523)(430.81949216,418.86709523)(430.84949341,418.85710297)
\curveto(430.88949209,418.85709524)(430.91949206,418.86209524)(430.93949341,418.87210297)
\lineto(431.68949341,418.87210297)
\lineto(431.88449341,418.87210297)
\curveto(431.96449102,418.88209522)(432.02949095,418.88209522)(432.07949341,418.87210297)
\lineto(432.19949341,418.87210297)
\curveto(432.25949072,418.85209525)(432.31449067,418.83709526)(432.36449341,418.82710297)
\curveto(432.41449057,418.81709528)(432.45449053,418.78709531)(432.48449341,418.73710297)
\curveto(432.52449046,418.68709541)(432.54449044,418.61709548)(432.54449341,418.52710297)
\curveto(432.55449043,418.43709566)(432.55949042,418.34209576)(432.55949341,418.24210297)
\lineto(432.55949341,411.95710297)
}
}
{
\newrgbcolor{curcolor}{0 0 0}
\pscustom[linestyle=none,fillstyle=solid,fillcolor=curcolor]
{
\newpath
\moveto(441.81168091,411.98710297)
\lineto(441.81168091,411.56710297)
\curveto(441.81167254,411.43710266)(441.78167257,411.33210277)(441.72168091,411.25210297)
\curveto(441.67167268,411.2021029)(441.60667274,411.16710293)(441.52668091,411.14710297)
\curveto(441.4466729,411.13710296)(441.35667299,411.13210297)(441.25668091,411.13210297)
\lineto(440.43168091,411.13210297)
\lineto(440.14668091,411.13210297)
\curveto(440.06667428,411.14210296)(440.00167435,411.16710293)(439.95168091,411.20710297)
\curveto(439.88167447,411.25710284)(439.84167451,411.32210278)(439.83168091,411.40210297)
\curveto(439.82167453,411.48210262)(439.80167455,411.56210254)(439.77168091,411.64210297)
\curveto(439.7516746,411.66210244)(439.73167462,411.67710242)(439.71168091,411.68710297)
\curveto(439.70167465,411.70710239)(439.68667466,411.72710237)(439.66668091,411.74710297)
\curveto(439.55667479,411.74710235)(439.47667487,411.72210238)(439.42668091,411.67210297)
\lineto(439.27668091,411.52210297)
\curveto(439.20667514,411.47210263)(439.14167521,411.42710267)(439.08168091,411.38710297)
\curveto(439.02167533,411.35710274)(438.95667539,411.31710278)(438.88668091,411.26710297)
\curveto(438.8466755,411.24710285)(438.80167555,411.22710287)(438.75168091,411.20710297)
\curveto(438.71167564,411.18710291)(438.66667568,411.16710293)(438.61668091,411.14710297)
\curveto(438.47667587,411.097103)(438.32667602,411.05210305)(438.16668091,411.01210297)
\curveto(438.11667623,410.99210311)(438.07167628,410.98210312)(438.03168091,410.98210297)
\curveto(437.99167636,410.98210312)(437.9516764,410.97710312)(437.91168091,410.96710297)
\lineto(437.77668091,410.96710297)
\curveto(437.7466766,410.95710314)(437.70667664,410.95210315)(437.65668091,410.95210297)
\lineto(437.52168091,410.95210297)
\curveto(437.46167689,410.93210317)(437.37167698,410.92710317)(437.25168091,410.93710297)
\curveto(437.13167722,410.93710316)(437.0466773,410.94710315)(436.99668091,410.96710297)
\curveto(436.92667742,410.98710311)(436.86167749,410.9971031)(436.80168091,410.99710297)
\curveto(436.7516776,410.98710311)(436.69667765,410.99210311)(436.63668091,411.01210297)
\lineto(436.27668091,411.13210297)
\curveto(436.16667818,411.16210294)(436.05667829,411.2021029)(435.94668091,411.25210297)
\curveto(435.59667875,411.4021027)(435.28167907,411.63210247)(435.00168091,411.94210297)
\curveto(434.73167962,412.26210184)(434.51667983,412.5971015)(434.35668091,412.94710297)
\curveto(434.30668004,413.05710104)(434.26668008,413.16210094)(434.23668091,413.26210297)
\curveto(434.20668014,413.37210073)(434.17168018,413.48210062)(434.13168091,413.59210297)
\curveto(434.12168023,413.63210047)(434.11668023,413.66710043)(434.11668091,413.69710297)
\curveto(434.11668023,413.73710036)(434.10668024,413.78210032)(434.08668091,413.83210297)
\curveto(434.06668028,413.91210019)(434.0466803,413.9971001)(434.02668091,414.08710297)
\curveto(434.01668033,414.18709991)(434.00168035,414.28709981)(433.98168091,414.38710297)
\curveto(433.97168038,414.41709968)(433.96668038,414.45209965)(433.96668091,414.49210297)
\curveto(433.97668037,414.53209957)(433.97668037,414.56709953)(433.96668091,414.59710297)
\lineto(433.96668091,414.73210297)
\curveto(433.96668038,414.78209932)(433.96168039,414.83209927)(433.95168091,414.88210297)
\curveto(433.94168041,414.93209917)(433.93668041,414.98709911)(433.93668091,415.04710297)
\curveto(433.93668041,415.11709898)(433.94168041,415.17209893)(433.95168091,415.21210297)
\curveto(433.96168039,415.26209884)(433.96668038,415.30709879)(433.96668091,415.34710297)
\lineto(433.96668091,415.49710297)
\curveto(433.97668037,415.54709855)(433.97668037,415.59209851)(433.96668091,415.63210297)
\curveto(433.96668038,415.68209842)(433.97668037,415.73209837)(433.99668091,415.78210297)
\curveto(434.01668033,415.89209821)(434.03168032,415.9970981)(434.04168091,416.09710297)
\curveto(434.06168029,416.1970979)(434.08668026,416.2970978)(434.11668091,416.39710297)
\curveto(434.15668019,416.51709758)(434.19168016,416.63209747)(434.22168091,416.74210297)
\curveto(434.2516801,416.85209725)(434.29168006,416.96209714)(434.34168091,417.07210297)
\curveto(434.48167987,417.37209673)(434.65667969,417.65709644)(434.86668091,417.92710297)
\curveto(434.88667946,417.95709614)(434.91167944,417.98209612)(434.94168091,418.00210297)
\curveto(434.98167937,418.03209607)(435.01167934,418.06209604)(435.03168091,418.09210297)
\curveto(435.07167928,418.14209596)(435.11167924,418.18709591)(435.15168091,418.22710297)
\curveto(435.19167916,418.26709583)(435.23667911,418.30709579)(435.28668091,418.34710297)
\curveto(435.32667902,418.36709573)(435.36167899,418.39209571)(435.39168091,418.42210297)
\curveto(435.42167893,418.46209564)(435.45667889,418.49209561)(435.49668091,418.51210297)
\curveto(435.7466786,418.68209542)(436.03667831,418.82209528)(436.36668091,418.93210297)
\curveto(436.43667791,418.95209515)(436.50667784,418.96709513)(436.57668091,418.97710297)
\curveto(436.65667769,418.98709511)(436.73667761,419.0020951)(436.81668091,419.02210297)
\curveto(436.88667746,419.04209506)(436.97667737,419.05209505)(437.08668091,419.05210297)
\curveto(437.19667715,419.06209504)(437.30667704,419.06709503)(437.41668091,419.06710297)
\curveto(437.52667682,419.06709503)(437.63167672,419.06209504)(437.73168091,419.05210297)
\curveto(437.84167651,419.04209506)(437.93167642,419.02709507)(438.00168091,419.00710297)
\curveto(438.1516762,418.95709514)(438.29667605,418.91209519)(438.43668091,418.87210297)
\curveto(438.57667577,418.83209527)(438.70667564,418.77709532)(438.82668091,418.70710297)
\curveto(438.89667545,418.65709544)(438.96167539,418.60709549)(439.02168091,418.55710297)
\curveto(439.08167527,418.51709558)(439.1466752,418.47209563)(439.21668091,418.42210297)
\curveto(439.25667509,418.39209571)(439.31167504,418.35209575)(439.38168091,418.30210297)
\curveto(439.46167489,418.25209585)(439.53667481,418.25209585)(439.60668091,418.30210297)
\curveto(439.6466747,418.32209578)(439.66667468,418.35709574)(439.66668091,418.40710297)
\curveto(439.66667468,418.45709564)(439.67667467,418.50709559)(439.69668091,418.55710297)
\lineto(439.69668091,418.70710297)
\curveto(439.70667464,418.73709536)(439.71167464,418.77209533)(439.71168091,418.81210297)
\lineto(439.71168091,418.93210297)
\lineto(439.71168091,420.97210297)
\curveto(439.71167464,421.08209302)(439.70667464,421.2020929)(439.69668091,421.33210297)
\curveto(439.69667465,421.47209263)(439.72167463,421.57709252)(439.77168091,421.64710297)
\curveto(439.81167454,421.72709237)(439.88667446,421.77709232)(439.99668091,421.79710297)
\curveto(440.01667433,421.80709229)(440.03667431,421.80709229)(440.05668091,421.79710297)
\curveto(440.07667427,421.7970923)(440.09667425,421.8020923)(440.11668091,421.81210297)
\lineto(441.18168091,421.81210297)
\curveto(441.30167305,421.81209229)(441.41167294,421.80709229)(441.51168091,421.79710297)
\curveto(441.61167274,421.78709231)(441.68667266,421.74709235)(441.73668091,421.67710297)
\curveto(441.78667256,421.5970925)(441.81167254,421.49209261)(441.81168091,421.36210297)
\lineto(441.81168091,421.00210297)
\lineto(441.81168091,411.98710297)
\moveto(439.77168091,414.92710297)
\curveto(439.78167457,414.96709913)(439.78167457,415.00709909)(439.77168091,415.04710297)
\lineto(439.77168091,415.18210297)
\curveto(439.77167458,415.28209882)(439.76667458,415.38209872)(439.75668091,415.48210297)
\curveto(439.7466746,415.58209852)(439.73167462,415.67209843)(439.71168091,415.75210297)
\curveto(439.69167466,415.86209824)(439.67167468,415.96209814)(439.65168091,416.05210297)
\curveto(439.64167471,416.14209796)(439.61667473,416.22709787)(439.57668091,416.30710297)
\curveto(439.43667491,416.66709743)(439.23167512,416.95209715)(438.96168091,417.16210297)
\curveto(438.70167565,417.37209673)(438.32167603,417.47709662)(437.82168091,417.47710297)
\curveto(437.76167659,417.47709662)(437.68167667,417.46709663)(437.58168091,417.44710297)
\curveto(437.50167685,417.42709667)(437.42667692,417.40709669)(437.35668091,417.38710297)
\curveto(437.29667705,417.37709672)(437.23667711,417.35709674)(437.17668091,417.32710297)
\curveto(436.90667744,417.21709688)(436.69667765,417.04709705)(436.54668091,416.81710297)
\curveto(436.39667795,416.58709751)(436.27667807,416.32709777)(436.18668091,416.03710297)
\curveto(436.15667819,415.93709816)(436.13667821,415.83709826)(436.12668091,415.73710297)
\curveto(436.11667823,415.63709846)(436.09667825,415.53209857)(436.06668091,415.42210297)
\lineto(436.06668091,415.21210297)
\curveto(436.0466783,415.12209898)(436.04167831,414.9970991)(436.05168091,414.83710297)
\curveto(436.06167829,414.68709941)(436.07667827,414.57709952)(436.09668091,414.50710297)
\lineto(436.09668091,414.41710297)
\curveto(436.10667824,414.3970997)(436.11167824,414.37709972)(436.11168091,414.35710297)
\curveto(436.13167822,414.27709982)(436.1466782,414.2020999)(436.15668091,414.13210297)
\curveto(436.17667817,414.06210004)(436.19667815,413.98710011)(436.21668091,413.90710297)
\curveto(436.38667796,413.38710071)(436.67667767,413.0021011)(437.08668091,412.75210297)
\curveto(437.21667713,412.66210144)(437.39667695,412.59210151)(437.62668091,412.54210297)
\curveto(437.66667668,412.53210157)(437.72667662,412.52710157)(437.80668091,412.52710297)
\curveto(437.83667651,412.51710158)(437.88167647,412.50710159)(437.94168091,412.49710297)
\curveto(438.01167634,412.4971016)(438.06667628,412.5021016)(438.10668091,412.51210297)
\curveto(438.18667616,412.53210157)(438.26667608,412.54710155)(438.34668091,412.55710297)
\curveto(438.42667592,412.56710153)(438.50667584,412.58710151)(438.58668091,412.61710297)
\curveto(438.83667551,412.72710137)(439.03667531,412.86710123)(439.18668091,413.03710297)
\curveto(439.33667501,413.20710089)(439.46667488,413.42210068)(439.57668091,413.68210297)
\curveto(439.61667473,413.77210033)(439.6466747,413.86210024)(439.66668091,413.95210297)
\curveto(439.68667466,414.05210005)(439.70667464,414.15709994)(439.72668091,414.26710297)
\curveto(439.73667461,414.31709978)(439.73667461,414.36209974)(439.72668091,414.40210297)
\curveto(439.72667462,414.45209965)(439.73667461,414.5020996)(439.75668091,414.55210297)
\curveto(439.76667458,414.58209952)(439.77167458,414.61709948)(439.77168091,414.65710297)
\lineto(439.77168091,414.79210297)
\lineto(439.77168091,414.92710297)
}
}
{
\newrgbcolor{curcolor}{0 0 0}
\pscustom[linestyle=none,fillstyle=solid,fillcolor=curcolor]
{
\newpath
\moveto(451.16160278,415.31710297)
\curveto(451.18159421,415.25709884)(451.1915942,415.17209893)(451.19160278,415.06210297)
\curveto(451.1915942,414.95209915)(451.18159421,414.86709923)(451.16160278,414.80710297)
\lineto(451.16160278,414.65710297)
\curveto(451.14159425,414.57709952)(451.13159426,414.4970996)(451.13160278,414.41710297)
\curveto(451.14159425,414.33709976)(451.13659426,414.25709984)(451.11660278,414.17710297)
\curveto(451.0965943,414.10709999)(451.08159431,414.04210006)(451.07160278,413.98210297)
\curveto(451.06159433,413.92210018)(451.05159434,413.85710024)(451.04160278,413.78710297)
\curveto(451.00159439,413.67710042)(450.96659443,413.56210054)(450.93660278,413.44210297)
\curveto(450.90659449,413.33210077)(450.86659453,413.22710087)(450.81660278,413.12710297)
\curveto(450.60659479,412.64710145)(450.33159506,412.25710184)(449.99160278,411.95710297)
\curveto(449.65159574,411.65710244)(449.24159615,411.40710269)(448.76160278,411.20710297)
\curveto(448.64159675,411.15710294)(448.51659688,411.12210298)(448.38660278,411.10210297)
\curveto(448.26659713,411.07210303)(448.14159725,411.04210306)(448.01160278,411.01210297)
\curveto(447.96159743,410.99210311)(447.90659749,410.98210312)(447.84660278,410.98210297)
\curveto(447.78659761,410.98210312)(447.73159766,410.97710312)(447.68160278,410.96710297)
\lineto(447.57660278,410.96710297)
\curveto(447.54659785,410.95710314)(447.51659788,410.95210315)(447.48660278,410.95210297)
\curveto(447.43659796,410.94210316)(447.35659804,410.93710316)(447.24660278,410.93710297)
\curveto(447.13659826,410.92710317)(447.05159834,410.93210317)(446.99160278,410.95210297)
\lineto(446.84160278,410.95210297)
\curveto(446.7915986,410.96210314)(446.73659866,410.96710313)(446.67660278,410.96710297)
\curveto(446.62659877,410.95710314)(446.57659882,410.96210314)(446.52660278,410.98210297)
\curveto(446.48659891,410.99210311)(446.44659895,410.9971031)(446.40660278,410.99710297)
\curveto(446.37659902,410.9971031)(446.33659906,411.0021031)(446.28660278,411.01210297)
\curveto(446.18659921,411.04210306)(446.08659931,411.06710303)(445.98660278,411.08710297)
\curveto(445.88659951,411.10710299)(445.7915996,411.13710296)(445.70160278,411.17710297)
\curveto(445.58159981,411.21710288)(445.46659993,411.25710284)(445.35660278,411.29710297)
\curveto(445.25660014,411.33710276)(445.15160024,411.38710271)(445.04160278,411.44710297)
\curveto(444.6916007,411.65710244)(444.391601,411.9021022)(444.14160278,412.18210297)
\curveto(443.8916015,412.46210164)(443.68160171,412.7971013)(443.51160278,413.18710297)
\curveto(443.46160193,413.27710082)(443.42160197,413.37210073)(443.39160278,413.47210297)
\curveto(443.37160202,413.57210053)(443.34660205,413.67710042)(443.31660278,413.78710297)
\curveto(443.2966021,413.83710026)(443.28660211,413.88210022)(443.28660278,413.92210297)
\curveto(443.28660211,413.96210014)(443.27660212,414.00710009)(443.25660278,414.05710297)
\curveto(443.23660216,414.13709996)(443.22660217,414.21709988)(443.22660278,414.29710297)
\curveto(443.22660217,414.38709971)(443.21660218,414.47209963)(443.19660278,414.55210297)
\curveto(443.18660221,414.6020995)(443.18160221,414.64709945)(443.18160278,414.68710297)
\lineto(443.18160278,414.82210297)
\curveto(443.16160223,414.88209922)(443.15160224,414.96709913)(443.15160278,415.07710297)
\curveto(443.16160223,415.18709891)(443.17660222,415.27209883)(443.19660278,415.33210297)
\lineto(443.19660278,415.43710297)
\curveto(443.20660219,415.48709861)(443.20660219,415.53709856)(443.19660278,415.58710297)
\curveto(443.1966022,415.64709845)(443.20660219,415.7020984)(443.22660278,415.75210297)
\curveto(443.23660216,415.8020983)(443.24160215,415.84709825)(443.24160278,415.88710297)
\curveto(443.24160215,415.93709816)(443.25160214,415.98709811)(443.27160278,416.03710297)
\curveto(443.31160208,416.16709793)(443.34660205,416.29209781)(443.37660278,416.41210297)
\curveto(443.40660199,416.54209756)(443.44660195,416.66709743)(443.49660278,416.78710297)
\curveto(443.67660172,417.1970969)(443.8916015,417.53709656)(444.14160278,417.80710297)
\curveto(444.391601,418.08709601)(444.6966007,418.34209576)(445.05660278,418.57210297)
\curveto(445.15660024,418.62209548)(445.26160013,418.66709543)(445.37160278,418.70710297)
\curveto(445.48159991,418.74709535)(445.5915998,418.79209531)(445.70160278,418.84210297)
\curveto(445.83159956,418.89209521)(445.96659943,418.92709517)(446.10660278,418.94710297)
\curveto(446.24659915,418.96709513)(446.391599,418.9970951)(446.54160278,419.03710297)
\curveto(446.62159877,419.04709505)(446.6965987,419.05209505)(446.76660278,419.05210297)
\curveto(446.83659856,419.05209505)(446.90659849,419.05709504)(446.97660278,419.06710297)
\curveto(447.55659784,419.07709502)(448.05659734,419.01709508)(448.47660278,418.88710297)
\curveto(448.90659649,418.75709534)(449.28659611,418.57709552)(449.61660278,418.34710297)
\curveto(449.72659567,418.26709583)(449.83659556,418.17709592)(449.94660278,418.07710297)
\curveto(450.06659533,417.98709611)(450.16659523,417.88709621)(450.24660278,417.77710297)
\curveto(450.32659507,417.67709642)(450.396595,417.57709652)(450.45660278,417.47710297)
\curveto(450.52659487,417.37709672)(450.5965948,417.27209683)(450.66660278,417.16210297)
\curveto(450.73659466,417.05209705)(450.7915946,416.93209717)(450.83160278,416.80210297)
\curveto(450.87159452,416.68209742)(450.91659448,416.55209755)(450.96660278,416.41210297)
\curveto(450.9965944,416.33209777)(451.02159437,416.24709785)(451.04160278,416.15710297)
\lineto(451.10160278,415.88710297)
\curveto(451.11159428,415.84709825)(451.11659428,415.80709829)(451.11660278,415.76710297)
\curveto(451.11659428,415.72709837)(451.12159427,415.68709841)(451.13160278,415.64710297)
\curveto(451.15159424,415.5970985)(451.15659424,415.54209856)(451.14660278,415.48210297)
\curveto(451.13659426,415.42209868)(451.14159425,415.36709873)(451.16160278,415.31710297)
\moveto(449.06160278,414.77710297)
\curveto(449.07159632,414.82709927)(449.07659632,414.8970992)(449.07660278,414.98710297)
\curveto(449.07659632,415.08709901)(449.07159632,415.16209894)(449.06160278,415.21210297)
\lineto(449.06160278,415.33210297)
\curveto(449.04159635,415.38209872)(449.03159636,415.43709866)(449.03160278,415.49710297)
\curveto(449.03159636,415.55709854)(449.02659637,415.61209849)(449.01660278,415.66210297)
\curveto(449.01659638,415.7020984)(449.01159638,415.73209837)(449.00160278,415.75210297)
\lineto(448.94160278,415.99210297)
\curveto(448.93159646,416.08209802)(448.91159648,416.16709793)(448.88160278,416.24710297)
\curveto(448.77159662,416.50709759)(448.64159675,416.72709737)(448.49160278,416.90710297)
\curveto(448.34159705,417.097097)(448.14159725,417.24709685)(447.89160278,417.35710297)
\curveto(447.83159756,417.37709672)(447.77159762,417.39209671)(447.71160278,417.40210297)
\curveto(447.65159774,417.42209668)(447.58659781,417.44209666)(447.51660278,417.46210297)
\curveto(447.43659796,417.48209662)(447.35159804,417.48709661)(447.26160278,417.47710297)
\lineto(446.99160278,417.47710297)
\curveto(446.96159843,417.45709664)(446.92659847,417.44709665)(446.88660278,417.44710297)
\curveto(446.84659855,417.45709664)(446.81159858,417.45709664)(446.78160278,417.44710297)
\lineto(446.57160278,417.38710297)
\curveto(446.51159888,417.37709672)(446.45659894,417.35709674)(446.40660278,417.32710297)
\curveto(446.15659924,417.21709688)(445.95159944,417.05709704)(445.79160278,416.84710297)
\curveto(445.64159975,416.64709745)(445.52159987,416.41209769)(445.43160278,416.14210297)
\curveto(445.40159999,416.04209806)(445.37660002,415.93709816)(445.35660278,415.82710297)
\curveto(445.34660005,415.71709838)(445.33160006,415.60709849)(445.31160278,415.49710297)
\curveto(445.30160009,415.44709865)(445.2966001,415.3970987)(445.29660278,415.34710297)
\lineto(445.29660278,415.19710297)
\curveto(445.27660012,415.12709897)(445.26660013,415.02209908)(445.26660278,414.88210297)
\curveto(445.27660012,414.74209936)(445.2916001,414.63709946)(445.31160278,414.56710297)
\lineto(445.31160278,414.43210297)
\curveto(445.33160006,414.35209975)(445.34660005,414.27209983)(445.35660278,414.19210297)
\curveto(445.36660003,414.12209998)(445.38160001,414.04710005)(445.40160278,413.96710297)
\curveto(445.50159989,413.66710043)(445.60659979,413.42210068)(445.71660278,413.23210297)
\curveto(445.83659956,413.05210105)(446.02159937,412.88710121)(446.27160278,412.73710297)
\curveto(446.34159905,412.68710141)(446.41659898,412.64710145)(446.49660278,412.61710297)
\curveto(446.58659881,412.58710151)(446.67659872,412.56210154)(446.76660278,412.54210297)
\curveto(446.80659859,412.53210157)(446.84159855,412.52710157)(446.87160278,412.52710297)
\curveto(446.90159849,412.53710156)(446.93659846,412.53710156)(446.97660278,412.52710297)
\lineto(447.09660278,412.49710297)
\curveto(447.14659825,412.4971016)(447.1915982,412.5021016)(447.23160278,412.51210297)
\lineto(447.35160278,412.51210297)
\curveto(447.43159796,412.53210157)(447.51159788,412.54710155)(447.59160278,412.55710297)
\curveto(447.67159772,412.56710153)(447.74659765,412.58710151)(447.81660278,412.61710297)
\curveto(448.07659732,412.71710138)(448.28659711,412.85210125)(448.44660278,413.02210297)
\curveto(448.60659679,413.19210091)(448.74159665,413.4021007)(448.85160278,413.65210297)
\curveto(448.8915965,413.75210035)(448.92159647,413.85210025)(448.94160278,413.95210297)
\curveto(448.96159643,414.05210005)(448.98659641,414.15709994)(449.01660278,414.26710297)
\curveto(449.02659637,414.30709979)(449.03159636,414.34209976)(449.03160278,414.37210297)
\curveto(449.03159636,414.41209969)(449.03659636,414.45209965)(449.04660278,414.49210297)
\lineto(449.04660278,414.62710297)
\curveto(449.04659635,414.67709942)(449.05159634,414.72709937)(449.06160278,414.77710297)
}
}
{
\newrgbcolor{curcolor}{0 0 0}
\pscustom[linestyle=none,fillstyle=solid,fillcolor=curcolor]
{
\newpath
\moveto(52.84082169,176.88068481)
\curveto(52.82083264,177.86067612)(52.98083248,178.65067533)(53.32082169,179.25068481)
\curveto(53.65083181,179.85067413)(54.11083135,180.27567371)(54.70082169,180.52568481)
\curveto(54.8608306,180.59567339)(55.01583045,180.64567334)(55.16582169,180.67568481)
\curveto(55.30583016,180.71567327)(55.47582999,180.74567324)(55.67582169,180.76568481)
\curveto(55.72582974,180.77567321)(55.79582967,180.7806732)(55.88582169,180.78068481)
\curveto(55.9658295,180.79067319)(56.04082942,180.77067321)(56.11082169,180.72068481)
\curveto(56.17082929,180.69067329)(56.21082925,180.64567334)(56.23082169,180.58568481)
\curveto(56.24082922,180.52567346)(56.25582921,180.46567352)(56.27582169,180.40568481)
\lineto(56.27582169,180.25568481)
\curveto(56.28582918,180.22567376)(56.29082917,180.1856738)(56.29082169,180.13568481)
\lineto(56.29082169,180.01568481)
\curveto(56.29082917,179.87567411)(56.28582918,179.74567424)(56.27582169,179.62568481)
\curveto(56.25582921,179.51567447)(56.20582926,179.44567454)(56.12582169,179.41568481)
\curveto(56.02582944,179.36567462)(55.91082955,179.33067465)(55.78082169,179.31068481)
\curveto(55.65082981,179.30067468)(55.53082993,179.27067471)(55.42082169,179.22068481)
\curveto(55.20083026,179.14067484)(55.01083045,179.03067495)(54.85082169,178.89068481)
\curveto(54.69083077,178.76067522)(54.54083092,178.60067538)(54.40082169,178.41068481)
\curveto(54.32083114,178.31067567)(54.2608312,178.19067579)(54.22082169,178.05068481)
\curveto(54.18083128,177.91067607)(54.14083132,177.77067621)(54.10082169,177.63068481)
\curveto(54.07083139,177.53067645)(54.05083141,177.41067657)(54.04082169,177.27068481)
\curveto(54.02083144,177.13067685)(54.01083145,176.980677)(54.01082169,176.82068481)
\curveto(54.01083145,176.67067731)(54.02083144,176.52067746)(54.04082169,176.37068481)
\curveto(54.05083141,176.23067775)(54.07083139,176.10567788)(54.10082169,175.99568481)
\lineto(54.16082169,175.69568481)
\curveto(54.18083128,175.60567838)(54.21083125,175.51067847)(54.25082169,175.41068481)
\curveto(54.60083086,174.50067948)(55.20083026,173.77568021)(56.05082169,173.23568481)
\curveto(56.19082927,173.13568085)(56.34082912,173.04568094)(56.50082169,172.96568481)
\curveto(56.65082881,172.89568109)(56.80582866,172.82068116)(56.96582169,172.74068481)
\curveto(57.02582844,172.71068127)(57.09082837,172.6856813)(57.16082169,172.66568481)
\curveto(57.22082824,172.65568133)(57.28082818,172.63568135)(57.34082169,172.60568481)
\curveto(57.38082808,172.5856814)(57.41582805,172.57068141)(57.44582169,172.56068481)
\curveto(57.47582799,172.56068142)(57.51082795,172.55068143)(57.55082169,172.53068481)
\curveto(57.6608278,172.49068149)(57.77582769,172.45568153)(57.89582169,172.42568481)
\curveto(58.00582746,172.39568159)(58.12082734,172.37068161)(58.24082169,172.35068481)
\curveto(58.31082715,172.33068165)(58.38582708,172.31068167)(58.46582169,172.29068481)
\curveto(58.53582693,172.27068171)(58.60082686,172.26068172)(58.66082169,172.26068481)
\lineto(58.81082169,172.23068481)
\curveto(58.88082658,172.24068174)(58.95082651,172.23568175)(59.02082169,172.21568481)
\curveto(59.09082637,172.19568179)(59.1608263,172.19068179)(59.23082169,172.20068481)
\curveto(59.29082617,172.20068178)(59.35082611,172.19068179)(59.41082169,172.17068481)
\curveto(59.47082599,172.16068182)(59.52582594,172.16068182)(59.57582169,172.17068481)
\lineto(59.96582169,172.17068481)
\curveto(60.08582538,172.17068181)(60.20582526,172.1806818)(60.32582169,172.20068481)
\curveto(60.82582464,172.27068171)(61.25582421,172.40568158)(61.61582169,172.60568481)
\curveto(61.9658235,172.80568118)(62.25582321,173.10068088)(62.48582169,173.49068481)
\curveto(62.55582291,173.62068036)(62.61082285,173.75068023)(62.65082169,173.88068481)
\curveto(62.69082277,174.02067996)(62.73582273,174.17067981)(62.78582169,174.33068481)
\curveto(62.80582266,174.40067958)(62.81582265,174.46567952)(62.81582169,174.52568481)
\curveto(62.80582266,174.5856794)(62.81082265,174.65567933)(62.83082169,174.73568481)
\curveto(62.84082262,174.77567921)(62.85082261,174.86067912)(62.86082169,174.99068481)
\curveto(62.8608226,175.12067886)(62.85082261,175.22067876)(62.83082169,175.29068481)
\lineto(62.83082169,175.39568481)
\curveto(62.82082264,175.43567855)(62.82082264,175.47567851)(62.83082169,175.51568481)
\curveto(62.83082263,175.55567843)(62.82082264,175.59567839)(62.80082169,175.63568481)
\curveto(62.78082268,175.73567825)(62.7658227,175.83567815)(62.75582169,175.93568481)
\curveto(62.73582273,176.03567795)(62.70582276,176.13567785)(62.66582169,176.23568481)
\curveto(62.34582312,177.07567691)(61.82082364,177.73067625)(61.09082169,178.20068481)
\curveto(60.91082455,178.31067567)(60.69582477,178.42567556)(60.44582169,178.54568481)
\curveto(60.35582511,178.5856754)(60.2658252,178.62067536)(60.17582169,178.65068481)
\curveto(60.07582539,178.69067529)(59.98582548,178.74567524)(59.90582169,178.81568481)
\curveto(59.82582564,178.87567511)(59.78082568,178.96067502)(59.77082169,179.07068481)
\curveto(59.7608257,179.19067479)(59.75582571,179.31567467)(59.75582169,179.44568481)
\lineto(59.75582169,179.59568481)
\curveto(59.75582571,179.64567434)(59.7608257,179.6856743)(59.77082169,179.71568481)
\lineto(59.77082169,179.82068481)
\lineto(59.80082169,179.91068481)
\curveto(59.80082566,179.95067403)(59.81082565,179.980674)(59.83082169,180.00068481)
\curveto(59.87082559,180.06067392)(59.94582552,180.0856739)(60.05582169,180.07568481)
\curveto(60.15582531,180.06567392)(60.25582521,180.03567395)(60.35582169,179.98568481)
\curveto(60.58582488,179.87567411)(60.80582466,179.77567421)(61.01582169,179.68568481)
\curveto(61.22582424,179.59567439)(61.42582404,179.4806745)(61.61582169,179.34068481)
\curveto(61.74582372,179.23067475)(61.87082359,179.13067485)(61.99082169,179.04068481)
\curveto(62.11082335,178.95067503)(62.23082323,178.85567513)(62.35082169,178.75568481)
\curveto(62.61082285,178.52567546)(62.85082261,178.25567573)(63.07082169,177.94568481)
\curveto(63.28082218,177.63567635)(63.45582201,177.31067667)(63.59582169,176.97068481)
\curveto(63.64582182,176.85067713)(63.68582178,176.73567725)(63.71582169,176.62568481)
\curveto(63.74582172,176.51567747)(63.78082168,176.40067758)(63.82082169,176.28068481)
\curveto(63.8608216,176.17067781)(63.88582158,176.05567793)(63.89582169,175.93568481)
\lineto(63.95582169,175.57568481)
\curveto(63.97582149,175.51567847)(63.98082148,175.46567852)(63.97082169,175.42568481)
\curveto(63.97082149,175.3856786)(63.97582149,175.34567864)(63.98582169,175.30568481)
\curveto(63.99582147,175.24567874)(64.00082146,175.1856788)(64.00082169,175.12568481)
\curveto(64.00082146,175.06567892)(64.00582146,175.00067898)(64.01582169,174.93068481)
\curveto(64.02582144,174.90067908)(64.02582144,174.83067915)(64.01582169,174.72068481)
\curveto(64.01582145,174.62067936)(64.01082145,174.55567943)(64.00082169,174.52568481)
\curveto(63.99082147,174.47567951)(63.98582148,174.42567956)(63.98582169,174.37568481)
\curveto(63.99582147,174.33567965)(63.99582147,174.29067969)(63.98582169,174.24068481)
\lineto(63.98582169,174.09068481)
\curveto(63.9658215,174.02067996)(63.95082151,173.95068003)(63.94082169,173.88068481)
\curveto(63.94082152,173.81068017)(63.93082153,173.73568025)(63.91082169,173.65568481)
\curveto(63.89082157,173.57568041)(63.87082159,173.49068049)(63.85082169,173.40068481)
\curveto(63.84082162,173.31068067)(63.82082164,173.23068075)(63.79082169,173.16068481)
\curveto(63.73082173,172.96068102)(63.65582181,172.7856812)(63.56582169,172.63568481)
\curveto(63.29582217,172.05568193)(62.91082255,171.62068236)(62.41082169,171.33068481)
\curveto(61.91082355,171.04068294)(61.32082414,170.85068313)(60.64082169,170.76068481)
\curveto(60.52082494,170.74068324)(60.39582507,170.73068325)(60.26582169,170.73068481)
\lineto(59.86082169,170.73068481)
\curveto(59.81082565,170.72068326)(59.7658257,170.72068326)(59.72582169,170.73068481)
\curveto(59.68582578,170.75068323)(59.64082582,170.76068322)(59.59082169,170.76068481)
\curveto(59.50082596,170.76068322)(59.40582606,170.76568322)(59.30582169,170.77568481)
\curveto(59.20582626,170.79568319)(59.11082635,170.80068318)(59.02082169,170.79068481)
\lineto(58.73582169,170.85068481)
\curveto(58.68582678,170.84068314)(58.60082686,170.84568314)(58.48082169,170.86568481)
\curveto(58.3608271,170.89568309)(58.27582719,170.92568306)(58.22582169,170.95568481)
\curveto(58.19582727,170.97568301)(58.1658273,170.980683)(58.13582169,170.97068481)
\curveto(58.09582737,170.97068301)(58.0658274,170.97568301)(58.04582169,170.98568481)
\lineto(57.91082169,171.01568481)
\curveto(57.83082763,171.04568294)(57.75082771,171.07068291)(57.67082169,171.09068481)
\curveto(57.58082788,171.11068287)(57.49582797,171.14068284)(57.41582169,171.18068481)
\curveto(57.35582811,171.21068277)(57.29582817,171.23068275)(57.23582169,171.24068481)
\curveto(57.1658283,171.26068272)(57.09582837,171.2856827)(57.02582169,171.31568481)
\curveto(56.85582861,171.39568259)(56.69082877,171.46568252)(56.53082169,171.52568481)
\curveto(56.37082909,171.59568239)(56.22082924,171.67568231)(56.08082169,171.76568481)
\curveto(55.95082951,171.83568215)(55.82082964,171.91068207)(55.69082169,171.99068481)
\curveto(55.55082991,172.0806819)(55.42583004,172.17068181)(55.31582169,172.26068481)
\curveto(55.05583041,172.46068152)(54.82083064,172.65568133)(54.61082169,172.84568481)
\curveto(54.5608309,172.8856811)(54.52083094,172.93068105)(54.49082169,172.98068481)
\curveto(54.45083101,173.04068094)(54.40583106,173.09068089)(54.35582169,173.13068481)
\curveto(54.27583119,173.20068078)(54.20583126,173.27568071)(54.14582169,173.35568481)
\lineto(53.96582169,173.59568481)
\curveto(53.91583155,173.66568032)(53.87083159,173.73068025)(53.83082169,173.79068481)
\curveto(53.78083168,173.86068012)(53.73083173,173.93568005)(53.68082169,174.01568481)
\curveto(53.57083189,174.1856798)(53.47583199,174.36067962)(53.39582169,174.54068481)
\curveto(53.31583215,174.72067926)(53.23583223,174.91067907)(53.15582169,175.11068481)
\curveto(53.10583236,175.23067875)(53.07083239,175.35567863)(53.05082169,175.48568481)
\curveto(53.02083244,175.61567837)(52.98583248,175.74067824)(52.94582169,175.86068481)
\curveto(52.93583253,175.94067804)(52.92583254,176.01067797)(52.91582169,176.07068481)
\lineto(52.88582169,176.25068481)
\curveto(52.87583259,176.33067765)(52.87083259,176.41067757)(52.87082169,176.49068481)
\lineto(52.84082169,176.73068481)
\curveto(52.83083263,176.75067723)(52.83083263,176.77567721)(52.84082169,176.80568481)
\curveto(52.85083261,176.83567715)(52.85083261,176.86067712)(52.84082169,176.88068481)
}
}
{
\newrgbcolor{curcolor}{0 0 0}
\pscustom[linestyle=none,fillstyle=solid,fillcolor=curcolor]
{
\newpath
\moveto(63.20582169,187.93052856)
\curveto(63.3658221,187.92052065)(63.50082196,187.8755207)(63.61082169,187.79552856)
\curveto(63.71082175,187.71552086)(63.78582168,187.62052095)(63.83582169,187.51052856)
\curveto(63.85582161,187.46052111)(63.8658216,187.40552117)(63.86582169,187.34552856)
\curveto(63.8658216,187.29552128)(63.87582159,187.23552134)(63.89582169,187.16552856)
\curveto(63.94582152,186.93552164)(63.93082153,186.72052185)(63.85082169,186.52052856)
\curveto(63.78082168,186.32052225)(63.69082177,186.19552238)(63.58082169,186.14552856)
\curveto(63.51082195,186.10552247)(63.43082203,186.0755225)(63.34082169,186.05552856)
\curveto(63.24082222,186.03552254)(63.1608223,186.00052257)(63.10082169,185.95052856)
\lineto(63.04082169,185.89052856)
\curveto(63.02082244,185.8705227)(63.01582245,185.84052273)(63.02582169,185.80052856)
\curveto(63.05582241,185.68052289)(63.11082235,185.56552301)(63.19082169,185.45552856)
\curveto(63.27082219,185.34552323)(63.34082212,185.24052333)(63.40082169,185.14052856)
\curveto(63.48082198,184.99052358)(63.55582191,184.83552374)(63.62582169,184.67552856)
\curveto(63.68582178,184.51552406)(63.74082172,184.34552423)(63.79082169,184.16552856)
\curveto(63.82082164,184.05552452)(63.84082162,183.94052463)(63.85082169,183.82052856)
\curveto(63.8608216,183.71052486)(63.87582159,183.59552498)(63.89582169,183.47552856)
\curveto(63.90582156,183.42552515)(63.91082155,183.38052519)(63.91082169,183.34052856)
\lineto(63.91082169,183.23552856)
\curveto(63.93082153,183.12552545)(63.93082153,183.02052555)(63.91082169,182.92052856)
\lineto(63.91082169,182.78552856)
\curveto(63.90082156,182.73552584)(63.89582157,182.68552589)(63.89582169,182.63552856)
\curveto(63.89582157,182.58552599)(63.88582158,182.54552603)(63.86582169,182.51552856)
\curveto(63.85582161,182.4755261)(63.85082161,182.44052613)(63.85082169,182.41052856)
\curveto(63.8608216,182.39052618)(63.8608216,182.36552621)(63.85082169,182.33552856)
\lineto(63.79082169,182.09552856)
\curveto(63.78082168,182.02552655)(63.7608217,181.96052661)(63.73082169,181.90052856)
\curveto(63.60082186,181.62052695)(63.45582201,181.40552717)(63.29582169,181.25552856)
\curveto(63.12582234,181.10552747)(62.89082257,181.00052757)(62.59082169,180.94052856)
\curveto(62.37082309,180.89052768)(62.10582336,180.89552768)(61.79582169,180.95552856)
\lineto(61.48082169,181.03052856)
\curveto(61.43082403,181.05052752)(61.38082408,181.06552751)(61.33082169,181.07552856)
\lineto(61.15082169,181.13552856)
\lineto(60.82082169,181.31552856)
\curveto(60.71082475,181.38552719)(60.61082485,181.45552712)(60.52082169,181.52552856)
\curveto(60.23082523,181.76552681)(60.01582545,182.05552652)(59.87582169,182.39552856)
\curveto(59.73582573,182.73552584)(59.61082585,183.10052547)(59.50082169,183.49052856)
\curveto(59.460826,183.64052493)(59.43082603,183.79052478)(59.41082169,183.94052856)
\curveto(59.39082607,184.10052447)(59.3658261,184.25552432)(59.33582169,184.40552856)
\curveto(59.31582615,184.48552409)(59.30582616,184.55552402)(59.30582169,184.61552856)
\curveto(59.30582616,184.68552389)(59.29582617,184.76052381)(59.27582169,184.84052856)
\curveto(59.25582621,184.91052366)(59.24582622,184.98052359)(59.24582169,185.05052856)
\curveto(59.23582623,185.13052344)(59.22082624,185.21052336)(59.20082169,185.29052856)
\curveto(59.14082632,185.55052302)(59.09082637,185.79552278)(59.05082169,186.02552856)
\curveto(59.00082646,186.25552232)(58.88582658,186.45552212)(58.70582169,186.62552856)
\curveto(58.62582684,186.69552188)(58.52582694,186.76052181)(58.40582169,186.82052856)
\curveto(58.27582719,186.89052168)(58.13582733,186.92052165)(57.98582169,186.91052856)
\curveto(57.74582772,186.90052167)(57.55582791,186.85052172)(57.41582169,186.76052856)
\curveto(57.27582819,186.68052189)(57.1658283,186.54052203)(57.08582169,186.34052856)
\curveto(57.03582843,186.23052234)(57.00082846,186.09552248)(56.98082169,185.93552856)
\curveto(56.9608285,185.7755228)(56.95082851,185.60552297)(56.95082169,185.42552856)
\curveto(56.95082851,185.24552333)(56.9608285,185.06552351)(56.98082169,184.88552856)
\curveto(57.00082846,184.71552386)(57.03082843,184.56552401)(57.07082169,184.43552856)
\curveto(57.13082833,184.25552432)(57.21582825,184.0755245)(57.32582169,183.89552856)
\curveto(57.38582808,183.80552477)(57.465828,183.71552486)(57.56582169,183.62552856)
\curveto(57.65582781,183.54552503)(57.75582771,183.4705251)(57.86582169,183.40052856)
\curveto(57.94582752,183.35052522)(58.03082743,183.30552527)(58.12082169,183.26552856)
\curveto(58.21082725,183.22552535)(58.28082718,183.16552541)(58.33082169,183.08552856)
\curveto(58.3608271,183.03552554)(58.38582708,182.96052561)(58.40582169,182.86052856)
\curveto(58.41582705,182.76052581)(58.42082704,182.66052591)(58.42082169,182.56052856)
\curveto(58.42082704,182.46052611)(58.41582705,182.36552621)(58.40582169,182.27552856)
\curveto(58.38582708,182.18552639)(58.3608271,182.12552645)(58.33082169,182.09552856)
\curveto(58.30082716,182.05552652)(58.25082721,182.03052654)(58.18082169,182.02052856)
\curveto(58.11082735,182.02052655)(58.03582743,182.04052653)(57.95582169,182.08052856)
\curveto(57.82582764,182.13052644)(57.70582776,182.18552639)(57.59582169,182.24552856)
\curveto(57.47582799,182.30552627)(57.3608281,182.3705262)(57.25082169,182.44052856)
\curveto(56.90082856,182.70052587)(56.63082883,182.99552558)(56.44082169,183.32552856)
\curveto(56.24082922,183.65552492)(56.08082938,184.04552453)(55.96082169,184.49552856)
\curveto(55.94082952,184.60552397)(55.92582954,184.71052386)(55.91582169,184.81052856)
\curveto(55.90582956,184.92052365)(55.89082957,185.03052354)(55.87082169,185.14052856)
\curveto(55.8608296,185.19052338)(55.8608296,185.25552332)(55.87082169,185.33552856)
\curveto(55.87082959,185.42552315)(55.8608296,185.48552309)(55.84082169,185.51552856)
\curveto(55.83082963,186.21552236)(55.91082955,186.80552177)(56.08082169,187.28552856)
\curveto(56.25082921,187.7755208)(56.57582889,188.08052049)(57.05582169,188.20052856)
\curveto(57.25582821,188.25052032)(57.49082797,188.25552032)(57.76082169,188.21552856)
\curveto(58.02082744,188.1755204)(58.29582717,188.12552045)(58.58582169,188.06552856)
\lineto(61.90082169,187.40552856)
\curveto(62.04082342,187.3755212)(62.17582329,187.35052122)(62.30582169,187.33052856)
\curveto(62.43582303,187.32052125)(62.54082292,187.33052124)(62.62082169,187.36052856)
\curveto(62.69082277,187.40052117)(62.74082272,187.45552112)(62.77082169,187.52552856)
\curveto(62.81082265,187.61552096)(62.84082262,187.69552088)(62.86082169,187.76552856)
\curveto(62.87082259,187.84552073)(62.91582255,187.89552068)(62.99582169,187.91552856)
\curveto(63.02582244,187.93552064)(63.05582241,187.94052063)(63.08582169,187.93052856)
\lineto(63.20582169,187.93052856)
\moveto(61.54082169,186.11552856)
\curveto(61.40082406,186.20552237)(61.24082422,186.2705223)(61.06082169,186.31052856)
\curveto(60.87082459,186.35052222)(60.67582479,186.39052218)(60.47582169,186.43052856)
\curveto(60.3658251,186.45052212)(60.2658252,186.46552211)(60.17582169,186.47552856)
\curveto(60.08582538,186.48552209)(60.01582545,186.46052211)(59.96582169,186.40052856)
\curveto(59.94582552,186.3705222)(59.93582553,186.30052227)(59.93582169,186.19052856)
\curveto(59.95582551,186.1705224)(59.9658255,186.13552244)(59.96582169,186.08552856)
\curveto(59.9658255,186.03552254)(59.97582549,185.98552259)(59.99582169,185.93552856)
\curveto(60.01582545,185.85552272)(60.03582543,185.76052281)(60.05582169,185.65052856)
\lineto(60.11582169,185.35052856)
\curveto(60.11582535,185.32052325)(60.12082534,185.28552329)(60.13082169,185.24552856)
\lineto(60.13082169,185.14052856)
\curveto(60.17082529,184.98052359)(60.19582527,184.81052376)(60.20582169,184.63052856)
\curveto(60.20582526,184.46052411)(60.22582524,184.29552428)(60.26582169,184.13552856)
\curveto(60.28582518,184.04552453)(60.30582516,183.96552461)(60.32582169,183.89552856)
\curveto(60.33582513,183.83552474)(60.35082511,183.76052481)(60.37082169,183.67052856)
\curveto(60.42082504,183.50052507)(60.48582498,183.33552524)(60.56582169,183.17552856)
\curveto(60.63582483,183.02552555)(60.72582474,182.89052568)(60.83582169,182.77052856)
\curveto(60.94582452,182.65052592)(61.08082438,182.55052602)(61.24082169,182.47052856)
\curveto(61.39082407,182.39052618)(61.57582389,182.33052624)(61.79582169,182.29052856)
\curveto(61.89582357,182.2705263)(61.99082347,182.2705263)(62.08082169,182.29052856)
\curveto(62.1608233,182.31052626)(62.23582323,182.34052623)(62.30582169,182.38052856)
\curveto(62.41582305,182.43052614)(62.51082295,182.51052606)(62.59082169,182.62052856)
\curveto(62.6608228,182.74052583)(62.72082274,182.8705257)(62.77082169,183.01052856)
\curveto(62.78082268,183.06052551)(62.78582268,183.11052546)(62.78582169,183.16052856)
\curveto(62.78582268,183.21052536)(62.79082267,183.26052531)(62.80082169,183.31052856)
\curveto(62.82082264,183.38052519)(62.83582263,183.46552511)(62.84582169,183.56552856)
\curveto(62.84582262,183.66552491)(62.83582263,183.75552482)(62.81582169,183.83552856)
\curveto(62.79582267,183.89552468)(62.79082267,183.95552462)(62.80082169,184.01552856)
\curveto(62.80082266,184.0755245)(62.79082267,184.13552444)(62.77082169,184.19552856)
\curveto(62.75082271,184.28552429)(62.73582273,184.36552421)(62.72582169,184.43552856)
\curveto(62.71582275,184.51552406)(62.69582277,184.59552398)(62.66582169,184.67552856)
\curveto(62.54582292,184.98552359)(62.40082306,185.26052331)(62.23082169,185.50052856)
\curveto(62.0608234,185.74052283)(61.83082363,185.94552263)(61.54082169,186.11552856)
}
}
{
\newrgbcolor{curcolor}{0 0 0}
\pscustom[linestyle=none,fillstyle=solid,fillcolor=curcolor]
{
\newpath
\moveto(55.88582169,194.24716919)
\curveto(55.8658296,194.88716237)(55.95082951,195.37716188)(56.14082169,195.71716919)
\curveto(56.33082913,196.0571612)(56.61582885,196.30216095)(56.99582169,196.45216919)
\curveto(57.09582837,196.49216076)(57.20582826,196.51716074)(57.32582169,196.52716919)
\curveto(57.43582803,196.54716071)(57.55082791,196.5571607)(57.67082169,196.55716919)
\curveto(57.8608276,196.57716068)(58.0658274,196.56716069)(58.28582169,196.52716919)
\curveto(58.50582696,196.49716076)(58.73082673,196.4571608)(58.96082169,196.40716919)
\lineto(60.56582169,196.09216919)
\lineto(62.90582169,195.62716919)
\lineto(63.41582169,195.50716919)
\curveto(63.58582188,195.46716179)(63.69582177,195.37716188)(63.74582169,195.23716919)
\curveto(63.7658217,195.18716207)(63.77582169,195.13216212)(63.77582169,195.07216919)
\curveto(63.78582168,195.02216223)(63.79082167,194.96716229)(63.79082169,194.90716919)
\curveto(63.79082167,194.77716248)(63.78582168,194.6521626)(63.77582169,194.53216919)
\curveto(63.77582169,194.41216284)(63.73582173,194.33716292)(63.65582169,194.30716919)
\curveto(63.58582188,194.26716299)(63.49582197,194.257163)(63.38582169,194.27716919)
\curveto(63.27582219,194.29716296)(63.1658223,194.32216293)(63.05582169,194.35216919)
\lineto(61.76582169,194.60716919)
\lineto(59.32082169,195.08716919)
\curveto(59.05082641,195.14716211)(58.78582668,195.19716206)(58.52582169,195.23716919)
\curveto(58.25582721,195.27716198)(58.02582744,195.27716198)(57.83582169,195.23716919)
\curveto(57.63582783,195.19716206)(57.47582799,195.10716215)(57.35582169,194.96716919)
\curveto(57.22582824,194.83716242)(57.12582834,194.67716258)(57.05582169,194.48716919)
\curveto(57.03582843,194.42716283)(57.02582844,194.36216289)(57.02582169,194.29216919)
\curveto(57.01582845,194.23216302)(57.00082846,194.17716308)(56.98082169,194.12716919)
\curveto(56.97082849,194.07716318)(56.97082849,193.99716326)(56.98082169,193.88716919)
\curveto(56.98082848,193.78716347)(56.98582848,193.71216354)(56.99582169,193.66216919)
\curveto(57.01582845,193.62216363)(57.02582844,193.58716367)(57.02582169,193.55716919)
\curveto(57.01582845,193.52716373)(57.01582845,193.49216376)(57.02582169,193.45216919)
\curveto(57.05582841,193.31216394)(57.09082837,193.18216407)(57.13082169,193.06216919)
\curveto(57.1608283,192.94216431)(57.20582826,192.82716443)(57.26582169,192.71716919)
\curveto(57.28582818,192.66716459)(57.30582816,192.62716463)(57.32582169,192.59716919)
\curveto(57.34582812,192.56716469)(57.3658281,192.52716473)(57.38582169,192.47716919)
\curveto(57.63582783,192.07716518)(58.01082745,191.74716551)(58.51082169,191.48716919)
\curveto(58.59082687,191.44716581)(58.67582679,191.41216584)(58.76582169,191.38216919)
\lineto(59.00582169,191.29216919)
\curveto(59.05582641,191.26216599)(59.10582636,191.24716601)(59.15582169,191.24716919)
\curveto(59.19582627,191.24716601)(59.23582623,191.23216602)(59.27582169,191.20216919)
\lineto(59.59082169,191.14216919)
\curveto(59.62082584,191.12216613)(59.65582581,191.11216614)(59.69582169,191.11216919)
\curveto(59.73582573,191.11216614)(59.78082568,191.10716615)(59.83082169,191.09716919)
\lineto(60.28082169,191.00716919)
\lineto(61.72082169,190.70716919)
\lineto(63.04082169,190.45216919)
\curveto(63.15082231,190.43216682)(63.2658222,190.40716685)(63.38582169,190.37716919)
\curveto(63.49582197,190.3571669)(63.58582188,190.31716694)(63.65582169,190.25716919)
\curveto(63.73582173,190.18716707)(63.77582169,190.08716717)(63.77582169,189.95716919)
\curveto(63.78582168,189.83716742)(63.79082167,189.71216754)(63.79082169,189.58216919)
\curveto(63.79082167,189.50216775)(63.78582168,189.42716783)(63.77582169,189.35716919)
\curveto(63.7658217,189.28716797)(63.74082172,189.23216802)(63.70082169,189.19216919)
\curveto(63.65082181,189.12216813)(63.55582191,189.10216815)(63.41582169,189.13216919)
\curveto(63.27582219,189.16216809)(63.14082232,189.18716807)(63.01082169,189.20716919)
\lineto(61.24082169,189.56716919)
\lineto(57.61082169,190.28716919)
\lineto(56.69582169,190.46716919)
\lineto(56.42582169,190.52716919)
\curveto(56.33582913,190.54716671)(56.2658292,190.58216667)(56.21582169,190.63216919)
\curveto(56.15582931,190.67216658)(56.11582935,190.72716653)(56.09582169,190.79716919)
\curveto(56.08582938,190.84716641)(56.07582939,190.90716635)(56.06582169,190.97716919)
\curveto(56.05582941,191.0571662)(56.05082941,191.13716612)(56.05082169,191.21716919)
\curveto(56.05082941,191.29716596)(56.05582941,191.37216588)(56.06582169,191.44216919)
\curveto(56.07582939,191.52216573)(56.09082937,191.57216568)(56.11082169,191.59216919)
\curveto(56.18082928,191.69216556)(56.27082919,191.72716553)(56.38082169,191.69716919)
\curveto(56.48082898,191.66716559)(56.59582887,191.6571656)(56.72582169,191.66716919)
\curveto(56.78582868,191.67716558)(56.83582863,191.70716555)(56.87582169,191.75716919)
\curveto(56.88582858,191.87716538)(56.84082862,191.98216527)(56.74082169,192.07216919)
\curveto(56.64082882,192.17216508)(56.5608289,192.26716499)(56.50082169,192.35716919)
\curveto(56.40082906,192.51716474)(56.31082915,192.67716458)(56.23082169,192.83716919)
\curveto(56.14082932,192.99716426)(56.0658294,193.18216407)(56.00582169,193.39216919)
\curveto(55.97582949,193.47216378)(55.95582951,193.56216369)(55.94582169,193.66216919)
\curveto(55.93582953,193.76216349)(55.92082954,193.8571634)(55.90082169,193.94716919)
\curveto(55.89082957,193.99716326)(55.88582958,194.04716321)(55.88582169,194.09716919)
\lineto(55.88582169,194.24716919)
}
}
{
\newrgbcolor{curcolor}{0 0 0}
\pscustom[linestyle=none,fillstyle=solid,fillcolor=curcolor]
{
\newpath
\moveto(53.69582169,200.28677856)
\curveto(53.69583177,200.43677454)(53.70083176,200.58677439)(53.71082169,200.73677856)
\curveto(53.71083175,200.88677409)(53.75083171,200.98677399)(53.83082169,201.03677856)
\curveto(53.89083157,201.06677391)(53.97583149,201.07177391)(54.08582169,201.05177856)
\curveto(54.18583128,201.04177394)(54.29083117,201.02677395)(54.40082169,201.00677856)
\lineto(55.27082169,200.82677856)
\curveto(55.35083011,200.81677416)(55.43583003,200.79677418)(55.52582169,200.76677856)
\curveto(55.60582986,200.74677423)(55.67582979,200.74177424)(55.73582169,200.75177856)
\curveto(55.87582959,200.76177422)(55.9658295,200.83677414)(56.00582169,200.97677856)
\curveto(56.01582945,201.01677396)(56.02082944,201.05677392)(56.02082169,201.09677856)
\lineto(56.02082169,201.24677856)
\lineto(56.02082169,201.65177856)
\curveto(56.01082945,201.82177316)(56.02082944,201.93677304)(56.05082169,201.99677856)
\curveto(56.11082935,202.0767729)(56.17082929,202.12677285)(56.23082169,202.14677856)
\curveto(56.27082919,202.15677282)(56.31582915,202.15677282)(56.36582169,202.14677856)
\lineto(56.51582169,202.11677856)
\curveto(56.62582884,202.09677288)(56.73082873,202.07177291)(56.83082169,202.04177856)
\curveto(56.92082854,202.01177297)(56.99082847,201.96177302)(57.04082169,201.89177856)
\curveto(57.09082837,201.82177316)(57.12082834,201.73177325)(57.13082169,201.62177856)
\lineto(57.13082169,201.29177856)
\curveto(57.12082834,201.1817738)(57.11582835,201.07177391)(57.11582169,200.96177856)
\curveto(57.11582835,200.85177413)(57.13082833,200.75177423)(57.16082169,200.66177856)
\curveto(57.19082827,200.59177439)(57.24082822,200.53177445)(57.31082169,200.48177856)
\curveto(57.38082808,200.44177454)(57.465828,200.40677457)(57.56582169,200.37677856)
\curveto(57.65582781,200.34677463)(57.75582771,200.32177466)(57.86582169,200.30177856)
\curveto(57.9658275,200.29177469)(58.0658274,200.2767747)(58.16582169,200.25677856)
\lineto(61.13582169,199.65677856)
\curveto(61.35582411,199.61677536)(61.59082387,199.56677541)(61.84082169,199.50677856)
\curveto(62.08082338,199.45677552)(62.2658232,199.46177552)(62.39582169,199.52177856)
\curveto(62.47582299,199.56177542)(62.53082293,199.61677536)(62.56082169,199.68677856)
\curveto(62.59082287,199.76677521)(62.61582285,199.85677512)(62.63582169,199.95677856)
\curveto(62.64582282,199.98677499)(62.65082281,200.01677496)(62.65082169,200.04677856)
\curveto(62.64082282,200.08677489)(62.64082282,200.12177486)(62.65082169,200.15177856)
\lineto(62.65082169,200.34677856)
\curveto(62.65082281,200.44677453)(62.6608228,200.53677444)(62.68082169,200.61677856)
\curveto(62.69082277,200.69677428)(62.72582274,200.75177423)(62.78582169,200.78177856)
\curveto(62.81582265,200.80177418)(62.87082259,200.81177417)(62.95082169,200.81177856)
\curveto(63.02082244,200.81177417)(63.09582237,200.80177418)(63.17582169,200.78177856)
\curveto(63.25582221,200.77177421)(63.33582213,200.75177423)(63.41582169,200.72177856)
\curveto(63.48582198,200.70177428)(63.54082192,200.6767743)(63.58082169,200.64677856)
\curveto(63.65082181,200.58677439)(63.70082176,200.50177448)(63.73082169,200.39177856)
\curveto(63.75082171,200.30177468)(63.7608217,200.20677477)(63.76082169,200.10677856)
\curveto(63.75082171,200.00677497)(63.74582172,199.91677506)(63.74582169,199.83677856)
\curveto(63.74582172,199.7767752)(63.75082171,199.71677526)(63.76082169,199.65677856)
\curveto(63.7608217,199.59677538)(63.75582171,199.54177544)(63.74582169,199.49177856)
\lineto(63.74582169,199.31177856)
\curveto(63.73582173,199.27177571)(63.73082173,199.22677575)(63.73082169,199.17677856)
\curveto(63.72082174,199.13677584)(63.71582175,199.09177589)(63.71582169,199.04177856)
\curveto(63.6658218,198.85177613)(63.61082185,198.68677629)(63.55082169,198.54677856)
\curveto(63.49082197,198.41677656)(63.38582208,198.31677666)(63.23582169,198.24677856)
\curveto(63.03582243,198.14677683)(62.78582268,198.11677686)(62.48582169,198.15677856)
\curveto(62.17582329,198.19677678)(61.84582362,198.25177673)(61.49582169,198.32177856)
\lineto(57.56582169,199.11677856)
\curveto(57.43582803,199.10677587)(57.34082812,199.09677588)(57.28082169,199.08677856)
\curveto(57.22082824,199.0767759)(57.17082829,199.01677596)(57.13082169,198.90677856)
\curveto(57.12082834,198.86677611)(57.12082834,198.82177616)(57.13082169,198.77177856)
\curveto(57.14082832,198.73177625)(57.13582833,198.69677628)(57.11582169,198.66677856)
\lineto(57.11582169,198.42677856)
\curveto(57.11582835,198.29677668)(57.10582836,198.19177679)(57.08582169,198.11177856)
\curveto(57.05582841,198.03177695)(56.99582847,197.98677699)(56.90582169,197.97677856)
\curveto(56.8658286,197.95677702)(56.82082864,197.94677703)(56.77082169,197.94677856)
\lineto(56.62082169,197.97677856)
\curveto(56.48082898,198.00677697)(56.3658291,198.04177694)(56.27582169,198.08177856)
\curveto(56.17582929,198.12177686)(56.10082936,198.19677678)(56.05082169,198.30677856)
\curveto(56.01082945,198.42677655)(56.00082946,198.57177641)(56.02082169,198.74177856)
\curveto(56.04082942,198.91177607)(56.03082943,199.06177592)(55.99082169,199.19177856)
\curveto(55.94082952,199.29177569)(55.87082959,199.3767756)(55.78082169,199.44677856)
\curveto(55.72082974,199.4767755)(55.64582982,199.49677548)(55.55582169,199.50677856)
\curveto(55.46583,199.52677545)(55.38083008,199.54677543)(55.30082169,199.56677856)
\lineto(54.37082169,199.74677856)
\curveto(54.29083117,199.76677521)(54.21083125,199.7817752)(54.13082169,199.79177856)
\curveto(54.04083142,199.81177517)(53.9658315,199.84177514)(53.90582169,199.88177856)
\curveto(53.82583164,199.93177505)(53.7608317,200.01677496)(53.71082169,200.13677856)
\curveto(53.71083175,200.16677481)(53.71083175,200.19177479)(53.71082169,200.21177856)
\curveto(53.70083176,200.24177474)(53.69583177,200.26677471)(53.69582169,200.28677856)
}
}
{
\newrgbcolor{curcolor}{0 0 0}
\pscustom[linestyle=none,fillstyle=solid,fillcolor=curcolor]
{
\newpath
\moveto(54.53582169,204.17857544)
\curveto(54.47583099,204.10857246)(54.37083109,204.08857248)(54.22082169,204.11857544)
\curveto(54.0608314,204.14857242)(53.90583156,204.17857239)(53.75582169,204.20857544)
\curveto(53.67583179,204.21857235)(53.59083187,204.23357234)(53.50082169,204.25357544)
\curveto(53.41083205,204.2735723)(53.33583213,204.30357227)(53.27582169,204.34357544)
\curveto(53.19583227,204.40357217)(53.13583233,204.49357208)(53.09582169,204.61357544)
\curveto(53.08583238,204.64357193)(53.08583238,204.6685719)(53.09582169,204.68857544)
\curveto(53.09583237,204.70857186)(53.09083237,204.73357184)(53.08082169,204.76357544)
\curveto(53.08083238,204.93357164)(53.08583238,205.08857148)(53.09582169,205.22857544)
\curveto(53.10583236,205.37857119)(53.1658323,205.4685711)(53.27582169,205.49857544)
\curveto(53.33583213,205.51857105)(53.41083205,205.51857105)(53.50082169,205.49857544)
\curveto(53.58083188,205.47857109)(53.6658318,205.46357111)(53.75582169,205.45357544)
\curveto(53.93583153,205.41357116)(54.10583136,205.3735712)(54.26582169,205.33357544)
\curveto(54.42583104,205.30357127)(54.53083093,205.21857135)(54.58082169,205.07857544)
\curveto(54.60083086,205.01857155)(54.61083085,204.95857161)(54.61082169,204.89857544)
\lineto(54.61082169,204.73357544)
\lineto(54.61082169,204.41857544)
\curveto(54.61083085,204.31857225)(54.58583088,204.23857233)(54.53582169,204.17857544)
\moveto(63.04082169,203.59357544)
\curveto(63.14082232,203.573573)(63.24582222,203.55357302)(63.35582169,203.53357544)
\curveto(63.45582201,203.52357305)(63.53582193,203.48357309)(63.59582169,203.41357544)
\curveto(63.65582181,203.3735732)(63.69582177,203.32357325)(63.71582169,203.26357544)
\curveto(63.72582174,203.20357337)(63.74082172,203.12857344)(63.76082169,203.03857544)
\lineto(63.76082169,202.81357544)
\curveto(63.7608217,202.68357389)(63.75582171,202.573574)(63.74582169,202.48357544)
\curveto(63.72582174,202.39357418)(63.67582179,202.32857424)(63.59582169,202.28857544)
\curveto(63.53582193,202.2685743)(63.460822,202.26357431)(63.37082169,202.27357544)
\curveto(63.27082219,202.29357428)(63.17582229,202.31357426)(63.08582169,202.33357544)
\lineto(56.74082169,203.60857544)
\curveto(56.63082883,203.62857294)(56.52582894,203.64857292)(56.42582169,203.66857544)
\curveto(56.31582915,203.68857288)(56.23082923,203.72857284)(56.17082169,203.78857544)
\curveto(56.12082934,203.82857274)(56.09082937,203.8735727)(56.08082169,203.92357544)
\curveto(56.07082939,203.98357259)(56.05582941,204.04357253)(56.03582169,204.10357544)
\curveto(56.03582943,204.12357245)(56.04082942,204.14357243)(56.05082169,204.16357544)
\curveto(56.05082941,204.19357238)(56.04582942,204.21857235)(56.03582169,204.23857544)
\curveto(56.03582943,204.3685722)(56.04082942,204.49857207)(56.05082169,204.62857544)
\curveto(56.05082941,204.7685718)(56.09082937,204.85357172)(56.17082169,204.88357544)
\curveto(56.23082923,204.92357165)(56.31082915,204.93357164)(56.41082169,204.91357544)
\curveto(56.50082896,204.89357168)(56.59582887,204.8735717)(56.69582169,204.85357544)
\lineto(63.04082169,203.59357544)
}
}
{
\newrgbcolor{curcolor}{0 0 0}
\pscustom[linestyle=none,fillstyle=solid,fillcolor=curcolor]
{
\newpath
\moveto(62.95082169,212.51341919)
\lineto(63.34082169,212.42341919)
\curveto(63.460822,212.40341126)(63.5608219,212.3634113)(63.64082169,212.30341919)
\curveto(63.71082175,212.23341143)(63.75082171,212.13841152)(63.76082169,212.01841919)
\lineto(63.76082169,211.67341919)
\curveto(63.7608217,211.61341205)(63.7658217,211.55341211)(63.77582169,211.49341919)
\curveto(63.77582169,211.44341222)(63.7658217,211.39841226)(63.74582169,211.35841919)
\curveto(63.72582174,211.27841238)(63.68582178,211.22841243)(63.62582169,211.20841919)
\curveto(63.57582189,211.17841248)(63.51582195,211.16841249)(63.44582169,211.17841919)
\curveto(63.37582209,211.18841247)(63.30582216,211.18341248)(63.23582169,211.16341919)
\curveto(63.21582225,211.1634125)(63.20082226,211.15341251)(63.19082169,211.13341919)
\lineto(63.13082169,211.10341919)
\curveto(63.12082234,211.00341266)(63.14082232,210.91841274)(63.19082169,210.84841919)
\curveto(63.24082222,210.78841287)(63.29082217,210.72341294)(63.34082169,210.65341919)
\curveto(63.49082197,210.42341324)(63.60582186,210.19841346)(63.68582169,209.97841919)
\curveto(63.7658217,209.78841387)(63.82582164,209.56841409)(63.86582169,209.31841919)
\curveto(63.90582156,209.07841458)(63.92582154,208.83341483)(63.92582169,208.58341919)
\curveto(63.93582153,208.34341532)(63.92082154,208.10341556)(63.88082169,207.86341919)
\curveto(63.85082161,207.63341603)(63.79582167,207.43841622)(63.71582169,207.27841919)
\curveto(63.49582197,206.79841686)(63.20082226,206.43341723)(62.83082169,206.18341919)
\curveto(62.45082301,205.94341772)(61.98082348,205.78841787)(61.42082169,205.71841919)
\curveto(61.33082413,205.69841796)(61.24082422,205.68841797)(61.15082169,205.68841919)
\curveto(61.05082441,205.69841796)(60.95082451,205.69841796)(60.85082169,205.68841919)
\curveto(60.80082466,205.68841797)(60.75082471,205.69341797)(60.70082169,205.70341919)
\curveto(60.65082481,205.71341795)(60.60082486,205.71841794)(60.55082169,205.71841919)
\curveto(60.50082496,205.70841795)(60.45082501,205.70841795)(60.40082169,205.71841919)
\curveto(60.34082512,205.73841792)(60.28582518,205.74841791)(60.23582169,205.74841919)
\lineto(60.08582169,205.77841919)
\curveto(60.03582543,205.76841789)(59.97082549,205.76841789)(59.89082169,205.77841919)
\curveto(59.81082565,205.79841786)(59.74582572,205.82341784)(59.69582169,205.85341919)
\lineto(59.53082169,205.89841919)
\curveto(59.460826,205.92841773)(59.39082607,205.94841771)(59.32082169,205.95841919)
\curveto(59.24082622,205.96841769)(59.1658263,205.98841767)(59.09582169,206.01841919)
\curveto(59.04582642,206.03841762)(59.00082646,206.05341761)(58.96082169,206.06341919)
\curveto(58.92082654,206.07341759)(58.87582659,206.08841757)(58.82582169,206.10841919)
\curveto(58.72582674,206.1584175)(58.63082683,206.20341746)(58.54082169,206.24341919)
\curveto(58.44082702,206.28341738)(58.34582712,206.32841733)(58.25582169,206.37841919)
\curveto(57.87582759,206.57841708)(57.53582793,206.80841685)(57.23582169,207.06841919)
\curveto(56.92582854,207.33841632)(56.67082879,207.63841602)(56.47082169,207.96841919)
\curveto(56.35082911,208.16841549)(56.25082921,208.36841529)(56.17082169,208.56841919)
\curveto(56.09082937,208.76841489)(56.02082944,208.98341468)(55.96082169,209.21341919)
\lineto(55.93082169,209.42341919)
\curveto(55.92082954,209.49341417)(55.90582956,209.5634141)(55.88582169,209.63341919)
\lineto(55.88582169,209.78341919)
\curveto(55.8658296,209.87341379)(55.85582961,209.99341367)(55.85582169,210.14341919)
\curveto(55.85582961,210.30341336)(55.8658296,210.41841324)(55.88582169,210.48841919)
\curveto(55.89582957,210.52841313)(55.90082956,210.58341308)(55.90082169,210.65341919)
\curveto(55.93082953,210.75341291)(55.95582951,210.8584128)(55.97582169,210.96841919)
\curveto(55.98582948,211.07841258)(56.01582945,211.17841248)(56.06582169,211.26841919)
\curveto(56.12582934,211.40841225)(56.19082927,211.53841212)(56.26082169,211.65841919)
\curveto(56.33082913,211.77841188)(56.41082905,211.88841177)(56.50082169,211.98841919)
\curveto(56.55082891,212.03841162)(56.60582886,212.08841157)(56.66582169,212.13841919)
\curveto(56.71582875,212.19841146)(56.73082873,212.28341138)(56.71082169,212.39341919)
\lineto(56.63582169,212.46841919)
\curveto(56.61582885,212.48841117)(56.58582888,212.50341116)(56.54582169,212.51341919)
\curveto(56.45582901,212.5634111)(56.34082912,212.59841106)(56.20082169,212.61841919)
\curveto(56.0608294,212.64841101)(55.93582953,212.67341099)(55.82582169,212.69341919)
\lineto(54.10082169,213.03841919)
\curveto(53.9608315,213.06841059)(53.80583166,213.09841056)(53.63582169,213.12841919)
\curveto(53.45583201,213.16841049)(53.32583214,213.21841044)(53.24582169,213.27841919)
\curveto(53.17583229,213.33841032)(53.13083233,213.40841025)(53.11082169,213.48841919)
\curveto(53.11083235,213.50841015)(53.11083235,213.53341013)(53.11082169,213.56341919)
\curveto(53.10083236,213.59341007)(53.09583237,213.61841004)(53.09582169,213.63841919)
\curveto(53.08583238,213.78840987)(53.08583238,213.93840972)(53.09582169,214.08841919)
\curveto(53.09583237,214.23840942)(53.13583233,214.33840932)(53.21582169,214.38841919)
\curveto(53.29583217,214.41840924)(53.39583207,214.41840924)(53.51582169,214.38841919)
\curveto(53.63583183,214.36840929)(53.7608317,214.34840931)(53.89082169,214.32841919)
\lineto(62.95082169,212.51341919)
\moveto(60.11582169,211.86841919)
\curveto(60.0658254,211.89841176)(60.00082546,211.91841174)(59.92082169,211.92841919)
\curveto(59.83082563,211.94841171)(59.7608257,211.95341171)(59.71082169,211.94341919)
\lineto(59.48582169,211.98841919)
\curveto(59.39582607,211.98841167)(59.30582616,211.99341167)(59.21582169,212.00341919)
\curveto(59.11582635,212.01341165)(59.02582644,212.00841165)(58.94582169,211.98841919)
\lineto(58.72082169,211.98841919)
\curveto(58.65082681,211.98841167)(58.58082688,211.97841168)(58.51082169,211.95841919)
\curveto(58.21082725,211.89841176)(57.94582752,211.79341187)(57.71582169,211.64341919)
\curveto(57.48582798,211.50341216)(57.30582816,211.30341236)(57.17582169,211.04341919)
\curveto(57.12582834,210.95341271)(57.09082837,210.8584128)(57.07082169,210.75841919)
\curveto(57.04082842,210.658413)(57.01582845,210.54841311)(56.99582169,210.42841919)
\curveto(56.97582849,210.3584133)(56.9658285,210.27341339)(56.96582169,210.17341919)
\lineto(56.96582169,209.90341919)
\lineto(56.99582169,209.75341919)
\lineto(56.99582169,209.61841919)
\curveto(57.01582845,209.53841412)(57.03582843,209.45341421)(57.05582169,209.36341919)
\curveto(57.07582839,209.27341439)(57.10082836,209.18841447)(57.13082169,209.10841919)
\curveto(57.27082819,208.7584149)(57.47582799,208.4584152)(57.74582169,208.20841919)
\curveto(58.00582746,207.9584157)(58.31082715,207.73841592)(58.66082169,207.54841919)
\curveto(58.77082669,207.48841617)(58.88582658,207.43841622)(59.00582169,207.39841919)
\lineto(59.33582169,207.27841919)
\lineto(59.45582169,207.24841919)
\curveto(59.48582598,207.23841642)(59.52082594,207.22841643)(59.56082169,207.21841919)
\curveto(59.61082585,207.18841647)(59.6658258,207.16841649)(59.72582169,207.15841919)
\curveto(59.78582568,207.1584165)(59.84082562,207.15341651)(59.89082169,207.14341919)
\curveto(60.00082546,207.12341654)(60.11082535,207.09841656)(60.22082169,207.06841919)
\curveto(60.32082514,207.04841661)(60.41582505,207.04341662)(60.50582169,207.05341919)
\curveto(60.53582493,207.05341661)(60.58582488,207.04841661)(60.65582169,207.03841919)
\lineto(60.86582169,207.03841919)
\curveto(60.93582453,207.03841662)(61.00582446,207.04341662)(61.07582169,207.05341919)
\curveto(61.42582404,207.09341657)(61.72582374,207.18341648)(61.97582169,207.32341919)
\curveto(62.22582324,207.4634162)(62.43082303,207.663416)(62.59082169,207.92341919)
\curveto(62.64082282,208.00341566)(62.68082278,208.08341558)(62.71082169,208.16341919)
\curveto(62.74082272,208.25341541)(62.77082269,208.34841531)(62.80082169,208.44841919)
\curveto(62.82082264,208.49841516)(62.82582264,208.54841511)(62.81582169,208.59841919)
\curveto(62.80582266,208.658415)(62.81082265,208.71341495)(62.83082169,208.76341919)
\curveto(62.84082262,208.79341487)(62.84582262,208.82841483)(62.84582169,208.86841919)
\lineto(62.84582169,209.00341919)
\lineto(62.84582169,209.13841919)
\curveto(62.83582263,209.17841448)(62.83082263,209.23341443)(62.83082169,209.30341919)
\curveto(62.81082265,209.38341428)(62.79582267,209.4634142)(62.78582169,209.54341919)
\curveto(62.7658227,209.63341403)(62.74082272,209.71341395)(62.71082169,209.78341919)
\curveto(62.57082289,210.14341352)(62.39582307,210.44841321)(62.18582169,210.69841919)
\curveto(61.9658235,210.94841271)(61.69082377,211.17341249)(61.36082169,211.37341919)
\curveto(61.25082421,211.44341222)(61.14082432,211.49841216)(61.03082169,211.53841919)
\lineto(60.70082169,211.68841919)
\curveto(60.6608248,211.71841194)(60.62582484,211.73341193)(60.59582169,211.73341919)
\curveto(60.55582491,211.74341192)(60.51582495,211.7584119)(60.47582169,211.77841919)
\curveto(60.41582505,211.79841186)(60.35582511,211.81341185)(60.29582169,211.82341919)
\curveto(60.23582523,211.83341183)(60.17582529,211.84841181)(60.11582169,211.86841919)
}
}
{
\newrgbcolor{curcolor}{0 0 0}
\pscustom[linestyle=none,fillstyle=solid,fillcolor=curcolor]
{
\newpath
\moveto(63.20582169,221.29966919)
\curveto(63.3658221,221.28966128)(63.50082196,221.24466132)(63.61082169,221.16466919)
\curveto(63.71082175,221.08466148)(63.78582168,220.98966158)(63.83582169,220.87966919)
\curveto(63.85582161,220.82966174)(63.8658216,220.77466179)(63.86582169,220.71466919)
\curveto(63.8658216,220.6646619)(63.87582159,220.60466196)(63.89582169,220.53466919)
\curveto(63.94582152,220.30466226)(63.93082153,220.08966248)(63.85082169,219.88966919)
\curveto(63.78082168,219.68966288)(63.69082177,219.564663)(63.58082169,219.51466919)
\curveto(63.51082195,219.47466309)(63.43082203,219.44466312)(63.34082169,219.42466919)
\curveto(63.24082222,219.40466316)(63.1608223,219.3696632)(63.10082169,219.31966919)
\lineto(63.04082169,219.25966919)
\curveto(63.02082244,219.23966333)(63.01582245,219.20966336)(63.02582169,219.16966919)
\curveto(63.05582241,219.04966352)(63.11082235,218.93466363)(63.19082169,218.82466919)
\curveto(63.27082219,218.71466385)(63.34082212,218.60966396)(63.40082169,218.50966919)
\curveto(63.48082198,218.35966421)(63.55582191,218.20466436)(63.62582169,218.04466919)
\curveto(63.68582178,217.88466468)(63.74082172,217.71466485)(63.79082169,217.53466919)
\curveto(63.82082164,217.42466514)(63.84082162,217.30966526)(63.85082169,217.18966919)
\curveto(63.8608216,217.07966549)(63.87582159,216.9646656)(63.89582169,216.84466919)
\curveto(63.90582156,216.79466577)(63.91082155,216.74966582)(63.91082169,216.70966919)
\lineto(63.91082169,216.60466919)
\curveto(63.93082153,216.49466607)(63.93082153,216.38966618)(63.91082169,216.28966919)
\lineto(63.91082169,216.15466919)
\curveto(63.90082156,216.10466646)(63.89582157,216.05466651)(63.89582169,216.00466919)
\curveto(63.89582157,215.95466661)(63.88582158,215.91466665)(63.86582169,215.88466919)
\curveto(63.85582161,215.84466672)(63.85082161,215.80966676)(63.85082169,215.77966919)
\curveto(63.8608216,215.75966681)(63.8608216,215.73466683)(63.85082169,215.70466919)
\lineto(63.79082169,215.46466919)
\curveto(63.78082168,215.39466717)(63.7608217,215.32966724)(63.73082169,215.26966919)
\curveto(63.60082186,214.98966758)(63.45582201,214.77466779)(63.29582169,214.62466919)
\curveto(63.12582234,214.47466809)(62.89082257,214.3696682)(62.59082169,214.30966919)
\curveto(62.37082309,214.25966831)(62.10582336,214.2646683)(61.79582169,214.32466919)
\lineto(61.48082169,214.39966919)
\curveto(61.43082403,214.41966815)(61.38082408,214.43466813)(61.33082169,214.44466919)
\lineto(61.15082169,214.50466919)
\lineto(60.82082169,214.68466919)
\curveto(60.71082475,214.75466781)(60.61082485,214.82466774)(60.52082169,214.89466919)
\curveto(60.23082523,215.13466743)(60.01582545,215.42466714)(59.87582169,215.76466919)
\curveto(59.73582573,216.10466646)(59.61082585,216.4696661)(59.50082169,216.85966919)
\curveto(59.460826,217.00966556)(59.43082603,217.15966541)(59.41082169,217.30966919)
\curveto(59.39082607,217.4696651)(59.3658261,217.62466494)(59.33582169,217.77466919)
\curveto(59.31582615,217.85466471)(59.30582616,217.92466464)(59.30582169,217.98466919)
\curveto(59.30582616,218.05466451)(59.29582617,218.12966444)(59.27582169,218.20966919)
\curveto(59.25582621,218.27966429)(59.24582622,218.34966422)(59.24582169,218.41966919)
\curveto(59.23582623,218.49966407)(59.22082624,218.57966399)(59.20082169,218.65966919)
\curveto(59.14082632,218.91966365)(59.09082637,219.1646634)(59.05082169,219.39466919)
\curveto(59.00082646,219.62466294)(58.88582658,219.82466274)(58.70582169,219.99466919)
\curveto(58.62582684,220.0646625)(58.52582694,220.12966244)(58.40582169,220.18966919)
\curveto(58.27582719,220.25966231)(58.13582733,220.28966228)(57.98582169,220.27966919)
\curveto(57.74582772,220.2696623)(57.55582791,220.21966235)(57.41582169,220.12966919)
\curveto(57.27582819,220.04966252)(57.1658283,219.90966266)(57.08582169,219.70966919)
\curveto(57.03582843,219.59966297)(57.00082846,219.4646631)(56.98082169,219.30466919)
\curveto(56.9608285,219.14466342)(56.95082851,218.97466359)(56.95082169,218.79466919)
\curveto(56.95082851,218.61466395)(56.9608285,218.43466413)(56.98082169,218.25466919)
\curveto(57.00082846,218.08466448)(57.03082843,217.93466463)(57.07082169,217.80466919)
\curveto(57.13082833,217.62466494)(57.21582825,217.44466512)(57.32582169,217.26466919)
\curveto(57.38582808,217.17466539)(57.465828,217.08466548)(57.56582169,216.99466919)
\curveto(57.65582781,216.91466565)(57.75582771,216.83966573)(57.86582169,216.76966919)
\curveto(57.94582752,216.71966585)(58.03082743,216.67466589)(58.12082169,216.63466919)
\curveto(58.21082725,216.59466597)(58.28082718,216.53466603)(58.33082169,216.45466919)
\curveto(58.3608271,216.40466616)(58.38582708,216.32966624)(58.40582169,216.22966919)
\curveto(58.41582705,216.12966644)(58.42082704,216.02966654)(58.42082169,215.92966919)
\curveto(58.42082704,215.82966674)(58.41582705,215.73466683)(58.40582169,215.64466919)
\curveto(58.38582708,215.55466701)(58.3608271,215.49466707)(58.33082169,215.46466919)
\curveto(58.30082716,215.42466714)(58.25082721,215.39966717)(58.18082169,215.38966919)
\curveto(58.11082735,215.38966718)(58.03582743,215.40966716)(57.95582169,215.44966919)
\curveto(57.82582764,215.49966707)(57.70582776,215.55466701)(57.59582169,215.61466919)
\curveto(57.47582799,215.67466689)(57.3608281,215.73966683)(57.25082169,215.80966919)
\curveto(56.90082856,216.0696665)(56.63082883,216.3646662)(56.44082169,216.69466919)
\curveto(56.24082922,217.02466554)(56.08082938,217.41466515)(55.96082169,217.86466919)
\curveto(55.94082952,217.97466459)(55.92582954,218.07966449)(55.91582169,218.17966919)
\curveto(55.90582956,218.28966428)(55.89082957,218.39966417)(55.87082169,218.50966919)
\curveto(55.8608296,218.55966401)(55.8608296,218.62466394)(55.87082169,218.70466919)
\curveto(55.87082959,218.79466377)(55.8608296,218.85466371)(55.84082169,218.88466919)
\curveto(55.83082963,219.58466298)(55.91082955,220.17466239)(56.08082169,220.65466919)
\curveto(56.25082921,221.14466142)(56.57582889,221.44966112)(57.05582169,221.56966919)
\curveto(57.25582821,221.61966095)(57.49082797,221.62466094)(57.76082169,221.58466919)
\curveto(58.02082744,221.54466102)(58.29582717,221.49466107)(58.58582169,221.43466919)
\lineto(61.90082169,220.77466919)
\curveto(62.04082342,220.74466182)(62.17582329,220.71966185)(62.30582169,220.69966919)
\curveto(62.43582303,220.68966188)(62.54082292,220.69966187)(62.62082169,220.72966919)
\curveto(62.69082277,220.7696618)(62.74082272,220.82466174)(62.77082169,220.89466919)
\curveto(62.81082265,220.98466158)(62.84082262,221.0646615)(62.86082169,221.13466919)
\curveto(62.87082259,221.21466135)(62.91582255,221.2646613)(62.99582169,221.28466919)
\curveto(63.02582244,221.30466126)(63.05582241,221.30966126)(63.08582169,221.29966919)
\lineto(63.20582169,221.29966919)
\moveto(61.54082169,219.48466919)
\curveto(61.40082406,219.57466299)(61.24082422,219.63966293)(61.06082169,219.67966919)
\curveto(60.87082459,219.71966285)(60.67582479,219.75966281)(60.47582169,219.79966919)
\curveto(60.3658251,219.81966275)(60.2658252,219.83466273)(60.17582169,219.84466919)
\curveto(60.08582538,219.85466271)(60.01582545,219.82966274)(59.96582169,219.76966919)
\curveto(59.94582552,219.73966283)(59.93582553,219.6696629)(59.93582169,219.55966919)
\curveto(59.95582551,219.53966303)(59.9658255,219.50466306)(59.96582169,219.45466919)
\curveto(59.9658255,219.40466316)(59.97582549,219.35466321)(59.99582169,219.30466919)
\curveto(60.01582545,219.22466334)(60.03582543,219.12966344)(60.05582169,219.01966919)
\lineto(60.11582169,218.71966919)
\curveto(60.11582535,218.68966388)(60.12082534,218.65466391)(60.13082169,218.61466919)
\lineto(60.13082169,218.50966919)
\curveto(60.17082529,218.34966422)(60.19582527,218.17966439)(60.20582169,217.99966919)
\curveto(60.20582526,217.82966474)(60.22582524,217.6646649)(60.26582169,217.50466919)
\curveto(60.28582518,217.41466515)(60.30582516,217.33466523)(60.32582169,217.26466919)
\curveto(60.33582513,217.20466536)(60.35082511,217.12966544)(60.37082169,217.03966919)
\curveto(60.42082504,216.8696657)(60.48582498,216.70466586)(60.56582169,216.54466919)
\curveto(60.63582483,216.39466617)(60.72582474,216.25966631)(60.83582169,216.13966919)
\curveto(60.94582452,216.01966655)(61.08082438,215.91966665)(61.24082169,215.83966919)
\curveto(61.39082407,215.75966681)(61.57582389,215.69966687)(61.79582169,215.65966919)
\curveto(61.89582357,215.63966693)(61.99082347,215.63966693)(62.08082169,215.65966919)
\curveto(62.1608233,215.67966689)(62.23582323,215.70966686)(62.30582169,215.74966919)
\curveto(62.41582305,215.79966677)(62.51082295,215.87966669)(62.59082169,215.98966919)
\curveto(62.6608228,216.10966646)(62.72082274,216.23966633)(62.77082169,216.37966919)
\curveto(62.78082268,216.42966614)(62.78582268,216.47966609)(62.78582169,216.52966919)
\curveto(62.78582268,216.57966599)(62.79082267,216.62966594)(62.80082169,216.67966919)
\curveto(62.82082264,216.74966582)(62.83582263,216.83466573)(62.84582169,216.93466919)
\curveto(62.84582262,217.03466553)(62.83582263,217.12466544)(62.81582169,217.20466919)
\curveto(62.79582267,217.2646653)(62.79082267,217.32466524)(62.80082169,217.38466919)
\curveto(62.80082266,217.44466512)(62.79082267,217.50466506)(62.77082169,217.56466919)
\curveto(62.75082271,217.65466491)(62.73582273,217.73466483)(62.72582169,217.80466919)
\curveto(62.71582275,217.88466468)(62.69582277,217.9646646)(62.66582169,218.04466919)
\curveto(62.54582292,218.35466421)(62.40082306,218.62966394)(62.23082169,218.86966919)
\curveto(62.0608234,219.10966346)(61.83082363,219.31466325)(61.54082169,219.48466919)
}
}
{
\newrgbcolor{curcolor}{0 0 0}
\pscustom[linestyle=none,fillstyle=solid,fillcolor=curcolor]
{
\newpath
\moveto(62.95082169,229.47630981)
\lineto(63.34082169,229.38630981)
\curveto(63.460822,229.36630188)(63.5608219,229.32630192)(63.64082169,229.26630981)
\curveto(63.71082175,229.19630205)(63.75082171,229.10130215)(63.76082169,228.98130981)
\lineto(63.76082169,228.63630981)
\curveto(63.7608217,228.57630267)(63.7658217,228.51630273)(63.77582169,228.45630981)
\curveto(63.77582169,228.40630284)(63.7658217,228.36130289)(63.74582169,228.32130981)
\curveto(63.72582174,228.24130301)(63.68582178,228.19130306)(63.62582169,228.17130981)
\curveto(63.57582189,228.14130311)(63.51582195,228.13130312)(63.44582169,228.14130981)
\curveto(63.37582209,228.1513031)(63.30582216,228.1463031)(63.23582169,228.12630981)
\curveto(63.21582225,228.12630312)(63.20082226,228.11630313)(63.19082169,228.09630981)
\lineto(63.13082169,228.06630981)
\curveto(63.12082234,227.96630328)(63.14082232,227.88130337)(63.19082169,227.81130981)
\curveto(63.24082222,227.7513035)(63.29082217,227.68630356)(63.34082169,227.61630981)
\curveto(63.49082197,227.38630386)(63.60582186,227.16130409)(63.68582169,226.94130981)
\curveto(63.7658217,226.7513045)(63.82582164,226.53130472)(63.86582169,226.28130981)
\curveto(63.90582156,226.04130521)(63.92582154,225.79630545)(63.92582169,225.54630981)
\curveto(63.93582153,225.30630594)(63.92082154,225.06630618)(63.88082169,224.82630981)
\curveto(63.85082161,224.59630665)(63.79582167,224.40130685)(63.71582169,224.24130981)
\curveto(63.49582197,223.76130749)(63.20082226,223.39630785)(62.83082169,223.14630981)
\curveto(62.45082301,222.90630834)(61.98082348,222.7513085)(61.42082169,222.68130981)
\curveto(61.33082413,222.66130859)(61.24082422,222.6513086)(61.15082169,222.65130981)
\curveto(61.05082441,222.66130859)(60.95082451,222.66130859)(60.85082169,222.65130981)
\curveto(60.80082466,222.6513086)(60.75082471,222.65630859)(60.70082169,222.66630981)
\curveto(60.65082481,222.67630857)(60.60082486,222.68130857)(60.55082169,222.68130981)
\curveto(60.50082496,222.67130858)(60.45082501,222.67130858)(60.40082169,222.68130981)
\curveto(60.34082512,222.70130855)(60.28582518,222.71130854)(60.23582169,222.71130981)
\lineto(60.08582169,222.74130981)
\curveto(60.03582543,222.73130852)(59.97082549,222.73130852)(59.89082169,222.74130981)
\curveto(59.81082565,222.76130849)(59.74582572,222.78630846)(59.69582169,222.81630981)
\lineto(59.53082169,222.86130981)
\curveto(59.460826,222.89130836)(59.39082607,222.91130834)(59.32082169,222.92130981)
\curveto(59.24082622,222.93130832)(59.1658263,222.9513083)(59.09582169,222.98130981)
\curveto(59.04582642,223.00130825)(59.00082646,223.01630823)(58.96082169,223.02630981)
\curveto(58.92082654,223.03630821)(58.87582659,223.0513082)(58.82582169,223.07130981)
\curveto(58.72582674,223.12130813)(58.63082683,223.16630808)(58.54082169,223.20630981)
\curveto(58.44082702,223.246308)(58.34582712,223.29130796)(58.25582169,223.34130981)
\curveto(57.87582759,223.54130771)(57.53582793,223.77130748)(57.23582169,224.03130981)
\curveto(56.92582854,224.30130695)(56.67082879,224.60130665)(56.47082169,224.93130981)
\curveto(56.35082911,225.13130612)(56.25082921,225.33130592)(56.17082169,225.53130981)
\curveto(56.09082937,225.73130552)(56.02082944,225.9463053)(55.96082169,226.17630981)
\lineto(55.93082169,226.38630981)
\curveto(55.92082954,226.45630479)(55.90582956,226.52630472)(55.88582169,226.59630981)
\lineto(55.88582169,226.74630981)
\curveto(55.8658296,226.83630441)(55.85582961,226.95630429)(55.85582169,227.10630981)
\curveto(55.85582961,227.26630398)(55.8658296,227.38130387)(55.88582169,227.45130981)
\curveto(55.89582957,227.49130376)(55.90082956,227.5463037)(55.90082169,227.61630981)
\curveto(55.93082953,227.71630353)(55.95582951,227.82130343)(55.97582169,227.93130981)
\curveto(55.98582948,228.04130321)(56.01582945,228.14130311)(56.06582169,228.23130981)
\curveto(56.12582934,228.37130288)(56.19082927,228.50130275)(56.26082169,228.62130981)
\curveto(56.33082913,228.74130251)(56.41082905,228.8513024)(56.50082169,228.95130981)
\curveto(56.55082891,229.00130225)(56.60582886,229.0513022)(56.66582169,229.10130981)
\curveto(56.71582875,229.16130209)(56.73082873,229.246302)(56.71082169,229.35630981)
\lineto(56.63582169,229.43130981)
\curveto(56.61582885,229.4513018)(56.58582888,229.46630178)(56.54582169,229.47630981)
\curveto(56.45582901,229.52630172)(56.34082912,229.56130169)(56.20082169,229.58130981)
\curveto(56.0608294,229.61130164)(55.93582953,229.63630161)(55.82582169,229.65630981)
\lineto(54.10082169,230.00130981)
\curveto(53.9608315,230.03130122)(53.80583166,230.06130119)(53.63582169,230.09130981)
\curveto(53.45583201,230.13130112)(53.32583214,230.18130107)(53.24582169,230.24130981)
\curveto(53.17583229,230.30130095)(53.13083233,230.37130088)(53.11082169,230.45130981)
\curveto(53.11083235,230.47130078)(53.11083235,230.49630075)(53.11082169,230.52630981)
\curveto(53.10083236,230.55630069)(53.09583237,230.58130067)(53.09582169,230.60130981)
\curveto(53.08583238,230.7513005)(53.08583238,230.90130035)(53.09582169,231.05130981)
\curveto(53.09583237,231.20130005)(53.13583233,231.30129995)(53.21582169,231.35130981)
\curveto(53.29583217,231.38129987)(53.39583207,231.38129987)(53.51582169,231.35130981)
\curveto(53.63583183,231.33129992)(53.7608317,231.31129994)(53.89082169,231.29130981)
\lineto(62.95082169,229.47630981)
\moveto(60.11582169,228.83130981)
\curveto(60.0658254,228.86130239)(60.00082546,228.88130237)(59.92082169,228.89130981)
\curveto(59.83082563,228.91130234)(59.7608257,228.91630233)(59.71082169,228.90630981)
\lineto(59.48582169,228.95130981)
\curveto(59.39582607,228.9513023)(59.30582616,228.95630229)(59.21582169,228.96630981)
\curveto(59.11582635,228.97630227)(59.02582644,228.97130228)(58.94582169,228.95130981)
\lineto(58.72082169,228.95130981)
\curveto(58.65082681,228.9513023)(58.58082688,228.94130231)(58.51082169,228.92130981)
\curveto(58.21082725,228.86130239)(57.94582752,228.75630249)(57.71582169,228.60630981)
\curveto(57.48582798,228.46630278)(57.30582816,228.26630298)(57.17582169,228.00630981)
\curveto(57.12582834,227.91630333)(57.09082837,227.82130343)(57.07082169,227.72130981)
\curveto(57.04082842,227.62130363)(57.01582845,227.51130374)(56.99582169,227.39130981)
\curveto(56.97582849,227.32130393)(56.9658285,227.23630401)(56.96582169,227.13630981)
\lineto(56.96582169,226.86630981)
\lineto(56.99582169,226.71630981)
\lineto(56.99582169,226.58130981)
\curveto(57.01582845,226.50130475)(57.03582843,226.41630483)(57.05582169,226.32630981)
\curveto(57.07582839,226.23630501)(57.10082836,226.1513051)(57.13082169,226.07130981)
\curveto(57.27082819,225.72130553)(57.47582799,225.42130583)(57.74582169,225.17130981)
\curveto(58.00582746,224.92130633)(58.31082715,224.70130655)(58.66082169,224.51130981)
\curveto(58.77082669,224.4513068)(58.88582658,224.40130685)(59.00582169,224.36130981)
\lineto(59.33582169,224.24130981)
\lineto(59.45582169,224.21130981)
\curveto(59.48582598,224.20130705)(59.52082594,224.19130706)(59.56082169,224.18130981)
\curveto(59.61082585,224.1513071)(59.6658258,224.13130712)(59.72582169,224.12130981)
\curveto(59.78582568,224.12130713)(59.84082562,224.11630713)(59.89082169,224.10630981)
\curveto(60.00082546,224.08630716)(60.11082535,224.06130719)(60.22082169,224.03130981)
\curveto(60.32082514,224.01130724)(60.41582505,224.00630724)(60.50582169,224.01630981)
\curveto(60.53582493,224.01630723)(60.58582488,224.01130724)(60.65582169,224.00130981)
\lineto(60.86582169,224.00130981)
\curveto(60.93582453,224.00130725)(61.00582446,224.00630724)(61.07582169,224.01630981)
\curveto(61.42582404,224.05630719)(61.72582374,224.1463071)(61.97582169,224.28630981)
\curveto(62.22582324,224.42630682)(62.43082303,224.62630662)(62.59082169,224.88630981)
\curveto(62.64082282,224.96630628)(62.68082278,225.0463062)(62.71082169,225.12630981)
\curveto(62.74082272,225.21630603)(62.77082269,225.31130594)(62.80082169,225.41130981)
\curveto(62.82082264,225.46130579)(62.82582264,225.51130574)(62.81582169,225.56130981)
\curveto(62.80582266,225.62130563)(62.81082265,225.67630557)(62.83082169,225.72630981)
\curveto(62.84082262,225.75630549)(62.84582262,225.79130546)(62.84582169,225.83130981)
\lineto(62.84582169,225.96630981)
\lineto(62.84582169,226.10130981)
\curveto(62.83582263,226.14130511)(62.83082263,226.19630505)(62.83082169,226.26630981)
\curveto(62.81082265,226.3463049)(62.79582267,226.42630482)(62.78582169,226.50630981)
\curveto(62.7658227,226.59630465)(62.74082272,226.67630457)(62.71082169,226.74630981)
\curveto(62.57082289,227.10630414)(62.39582307,227.41130384)(62.18582169,227.66130981)
\curveto(61.9658235,227.91130334)(61.69082377,228.13630311)(61.36082169,228.33630981)
\curveto(61.25082421,228.40630284)(61.14082432,228.46130279)(61.03082169,228.50130981)
\lineto(60.70082169,228.65130981)
\curveto(60.6608248,228.68130257)(60.62582484,228.69630255)(60.59582169,228.69630981)
\curveto(60.55582491,228.70630254)(60.51582495,228.72130253)(60.47582169,228.74130981)
\curveto(60.41582505,228.76130249)(60.35582511,228.77630247)(60.29582169,228.78630981)
\curveto(60.23582523,228.79630245)(60.17582529,228.81130244)(60.11582169,228.83130981)
}
}
{
\newrgbcolor{curcolor}{0 0 0}
\pscustom[linestyle=none,fillstyle=solid,fillcolor=curcolor]
{
}
}
{
\newrgbcolor{curcolor}{0 0 0}
\pscustom[linestyle=none,fillstyle=solid,fillcolor=curcolor]
{
\newpath
\moveto(62.95082169,242.54271606)
\lineto(63.34082169,242.45271606)
\curveto(63.460822,242.43270813)(63.5608219,242.39270817)(63.64082169,242.33271606)
\curveto(63.71082175,242.2627083)(63.75082171,242.1677084)(63.76082169,242.04771606)
\lineto(63.76082169,241.70271606)
\curveto(63.7608217,241.64270892)(63.7658217,241.58270898)(63.77582169,241.52271606)
\curveto(63.77582169,241.47270909)(63.7658217,241.42770914)(63.74582169,241.38771606)
\curveto(63.72582174,241.30770926)(63.68582178,241.25770931)(63.62582169,241.23771606)
\curveto(63.57582189,241.20770936)(63.51582195,241.19770937)(63.44582169,241.20771606)
\curveto(63.37582209,241.21770935)(63.30582216,241.21270935)(63.23582169,241.19271606)
\curveto(63.21582225,241.19270937)(63.20082226,241.18270938)(63.19082169,241.16271606)
\lineto(63.13082169,241.13271606)
\curveto(63.12082234,241.03270953)(63.14082232,240.94770962)(63.19082169,240.87771606)
\curveto(63.24082222,240.81770975)(63.29082217,240.75270981)(63.34082169,240.68271606)
\curveto(63.49082197,240.45271011)(63.60582186,240.22771034)(63.68582169,240.00771606)
\curveto(63.7658217,239.81771075)(63.82582164,239.59771097)(63.86582169,239.34771606)
\curveto(63.90582156,239.10771146)(63.92582154,238.8627117)(63.92582169,238.61271606)
\curveto(63.93582153,238.37271219)(63.92082154,238.13271243)(63.88082169,237.89271606)
\curveto(63.85082161,237.6627129)(63.79582167,237.4677131)(63.71582169,237.30771606)
\curveto(63.49582197,236.82771374)(63.20082226,236.4627141)(62.83082169,236.21271606)
\curveto(62.45082301,235.97271459)(61.98082348,235.81771475)(61.42082169,235.74771606)
\curveto(61.33082413,235.72771484)(61.24082422,235.71771485)(61.15082169,235.71771606)
\curveto(61.05082441,235.72771484)(60.95082451,235.72771484)(60.85082169,235.71771606)
\curveto(60.80082466,235.71771485)(60.75082471,235.72271484)(60.70082169,235.73271606)
\curveto(60.65082481,235.74271482)(60.60082486,235.74771482)(60.55082169,235.74771606)
\curveto(60.50082496,235.73771483)(60.45082501,235.73771483)(60.40082169,235.74771606)
\curveto(60.34082512,235.7677148)(60.28582518,235.77771479)(60.23582169,235.77771606)
\lineto(60.08582169,235.80771606)
\curveto(60.03582543,235.79771477)(59.97082549,235.79771477)(59.89082169,235.80771606)
\curveto(59.81082565,235.82771474)(59.74582572,235.85271471)(59.69582169,235.88271606)
\lineto(59.53082169,235.92771606)
\curveto(59.460826,235.95771461)(59.39082607,235.97771459)(59.32082169,235.98771606)
\curveto(59.24082622,235.99771457)(59.1658263,236.01771455)(59.09582169,236.04771606)
\curveto(59.04582642,236.0677145)(59.00082646,236.08271448)(58.96082169,236.09271606)
\curveto(58.92082654,236.10271446)(58.87582659,236.11771445)(58.82582169,236.13771606)
\curveto(58.72582674,236.18771438)(58.63082683,236.23271433)(58.54082169,236.27271606)
\curveto(58.44082702,236.31271425)(58.34582712,236.35771421)(58.25582169,236.40771606)
\curveto(57.87582759,236.60771396)(57.53582793,236.83771373)(57.23582169,237.09771606)
\curveto(56.92582854,237.3677132)(56.67082879,237.6677129)(56.47082169,237.99771606)
\curveto(56.35082911,238.19771237)(56.25082921,238.39771217)(56.17082169,238.59771606)
\curveto(56.09082937,238.79771177)(56.02082944,239.01271155)(55.96082169,239.24271606)
\lineto(55.93082169,239.45271606)
\curveto(55.92082954,239.52271104)(55.90582956,239.59271097)(55.88582169,239.66271606)
\lineto(55.88582169,239.81271606)
\curveto(55.8658296,239.90271066)(55.85582961,240.02271054)(55.85582169,240.17271606)
\curveto(55.85582961,240.33271023)(55.8658296,240.44771012)(55.88582169,240.51771606)
\curveto(55.89582957,240.55771001)(55.90082956,240.61270995)(55.90082169,240.68271606)
\curveto(55.93082953,240.78270978)(55.95582951,240.88770968)(55.97582169,240.99771606)
\curveto(55.98582948,241.10770946)(56.01582945,241.20770936)(56.06582169,241.29771606)
\curveto(56.12582934,241.43770913)(56.19082927,241.567709)(56.26082169,241.68771606)
\curveto(56.33082913,241.80770876)(56.41082905,241.91770865)(56.50082169,242.01771606)
\curveto(56.55082891,242.0677085)(56.60582886,242.11770845)(56.66582169,242.16771606)
\curveto(56.71582875,242.22770834)(56.73082873,242.31270825)(56.71082169,242.42271606)
\lineto(56.63582169,242.49771606)
\curveto(56.61582885,242.51770805)(56.58582888,242.53270803)(56.54582169,242.54271606)
\curveto(56.45582901,242.59270797)(56.34082912,242.62770794)(56.20082169,242.64771606)
\curveto(56.0608294,242.67770789)(55.93582953,242.70270786)(55.82582169,242.72271606)
\lineto(54.10082169,243.06771606)
\curveto(53.9608315,243.09770747)(53.80583166,243.12770744)(53.63582169,243.15771606)
\curveto(53.45583201,243.19770737)(53.32583214,243.24770732)(53.24582169,243.30771606)
\curveto(53.17583229,243.3677072)(53.13083233,243.43770713)(53.11082169,243.51771606)
\curveto(53.11083235,243.53770703)(53.11083235,243.562707)(53.11082169,243.59271606)
\curveto(53.10083236,243.62270694)(53.09583237,243.64770692)(53.09582169,243.66771606)
\curveto(53.08583238,243.81770675)(53.08583238,243.9677066)(53.09582169,244.11771606)
\curveto(53.09583237,244.2677063)(53.13583233,244.3677062)(53.21582169,244.41771606)
\curveto(53.29583217,244.44770612)(53.39583207,244.44770612)(53.51582169,244.41771606)
\curveto(53.63583183,244.39770617)(53.7608317,244.37770619)(53.89082169,244.35771606)
\lineto(62.95082169,242.54271606)
\moveto(60.11582169,241.89771606)
\curveto(60.0658254,241.92770864)(60.00082546,241.94770862)(59.92082169,241.95771606)
\curveto(59.83082563,241.97770859)(59.7608257,241.98270858)(59.71082169,241.97271606)
\lineto(59.48582169,242.01771606)
\curveto(59.39582607,242.01770855)(59.30582616,242.02270854)(59.21582169,242.03271606)
\curveto(59.11582635,242.04270852)(59.02582644,242.03770853)(58.94582169,242.01771606)
\lineto(58.72082169,242.01771606)
\curveto(58.65082681,242.01770855)(58.58082688,242.00770856)(58.51082169,241.98771606)
\curveto(58.21082725,241.92770864)(57.94582752,241.82270874)(57.71582169,241.67271606)
\curveto(57.48582798,241.53270903)(57.30582816,241.33270923)(57.17582169,241.07271606)
\curveto(57.12582834,240.98270958)(57.09082837,240.88770968)(57.07082169,240.78771606)
\curveto(57.04082842,240.68770988)(57.01582845,240.57770999)(56.99582169,240.45771606)
\curveto(56.97582849,240.38771018)(56.9658285,240.30271026)(56.96582169,240.20271606)
\lineto(56.96582169,239.93271606)
\lineto(56.99582169,239.78271606)
\lineto(56.99582169,239.64771606)
\curveto(57.01582845,239.567711)(57.03582843,239.48271108)(57.05582169,239.39271606)
\curveto(57.07582839,239.30271126)(57.10082836,239.21771135)(57.13082169,239.13771606)
\curveto(57.27082819,238.78771178)(57.47582799,238.48771208)(57.74582169,238.23771606)
\curveto(58.00582746,237.98771258)(58.31082715,237.7677128)(58.66082169,237.57771606)
\curveto(58.77082669,237.51771305)(58.88582658,237.4677131)(59.00582169,237.42771606)
\lineto(59.33582169,237.30771606)
\lineto(59.45582169,237.27771606)
\curveto(59.48582598,237.2677133)(59.52082594,237.25771331)(59.56082169,237.24771606)
\curveto(59.61082585,237.21771335)(59.6658258,237.19771337)(59.72582169,237.18771606)
\curveto(59.78582568,237.18771338)(59.84082562,237.18271338)(59.89082169,237.17271606)
\curveto(60.00082546,237.15271341)(60.11082535,237.12771344)(60.22082169,237.09771606)
\curveto(60.32082514,237.07771349)(60.41582505,237.07271349)(60.50582169,237.08271606)
\curveto(60.53582493,237.08271348)(60.58582488,237.07771349)(60.65582169,237.06771606)
\lineto(60.86582169,237.06771606)
\curveto(60.93582453,237.0677135)(61.00582446,237.07271349)(61.07582169,237.08271606)
\curveto(61.42582404,237.12271344)(61.72582374,237.21271335)(61.97582169,237.35271606)
\curveto(62.22582324,237.49271307)(62.43082303,237.69271287)(62.59082169,237.95271606)
\curveto(62.64082282,238.03271253)(62.68082278,238.11271245)(62.71082169,238.19271606)
\curveto(62.74082272,238.28271228)(62.77082269,238.37771219)(62.80082169,238.47771606)
\curveto(62.82082264,238.52771204)(62.82582264,238.57771199)(62.81582169,238.62771606)
\curveto(62.80582266,238.68771188)(62.81082265,238.74271182)(62.83082169,238.79271606)
\curveto(62.84082262,238.82271174)(62.84582262,238.85771171)(62.84582169,238.89771606)
\lineto(62.84582169,239.03271606)
\lineto(62.84582169,239.16771606)
\curveto(62.83582263,239.20771136)(62.83082263,239.2627113)(62.83082169,239.33271606)
\curveto(62.81082265,239.41271115)(62.79582267,239.49271107)(62.78582169,239.57271606)
\curveto(62.7658227,239.6627109)(62.74082272,239.74271082)(62.71082169,239.81271606)
\curveto(62.57082289,240.17271039)(62.39582307,240.47771009)(62.18582169,240.72771606)
\curveto(61.9658235,240.97770959)(61.69082377,241.20270936)(61.36082169,241.40271606)
\curveto(61.25082421,241.47270909)(61.14082432,241.52770904)(61.03082169,241.56771606)
\lineto(60.70082169,241.71771606)
\curveto(60.6608248,241.74770882)(60.62582484,241.7627088)(60.59582169,241.76271606)
\curveto(60.55582491,241.77270879)(60.51582495,241.78770878)(60.47582169,241.80771606)
\curveto(60.41582505,241.82770874)(60.35582511,241.84270872)(60.29582169,241.85271606)
\curveto(60.23582523,241.8627087)(60.17582529,241.87770869)(60.11582169,241.89771606)
}
}
{
\newrgbcolor{curcolor}{0 0 0}
\pscustom[linestyle=none,fillstyle=solid,fillcolor=curcolor]
{
\newpath
\moveto(59.59082169,251.91396606)
\curveto(59.69082577,251.91395756)(59.80582566,251.89395758)(59.93582169,251.85396606)
\curveto(60.05582541,251.81395766)(60.14082532,251.76395771)(60.19082169,251.70396606)
\curveto(60.23082523,251.64395783)(60.2608252,251.56395791)(60.28082169,251.46396606)
\curveto(60.29082517,251.36395811)(60.29582517,251.25395822)(60.29582169,251.13396606)
\lineto(60.29582169,250.77396606)
\curveto(60.28582518,250.66395881)(60.28082518,250.56395891)(60.28082169,250.47396606)
\lineto(60.28082169,246.63396606)
\curveto(60.28082518,246.55396292)(60.28582518,246.468963)(60.29582169,246.37896606)
\curveto(60.29582517,246.29896317)(60.31082515,246.23396324)(60.34082169,246.18396606)
\curveto(60.3608251,246.13396334)(60.40082506,246.08396339)(60.46082169,246.03396606)
\lineto(60.59582169,245.94396606)
\curveto(60.64582482,245.91396356)(60.69582477,245.90396357)(60.74582169,245.91396606)
\curveto(60.79582467,245.91396356)(60.84082462,245.90896356)(60.88082169,245.89896606)
\lineto(61.00082169,245.89896606)
\lineto(61.25582169,245.89896606)
\curveto(61.33582413,245.90896356)(61.41582405,245.92396355)(61.49582169,245.94396606)
\curveto(62.03582343,246.0739634)(62.42082304,246.37896309)(62.65082169,246.85896606)
\curveto(62.68082278,246.90896256)(62.70582276,246.9689625)(62.72582169,247.03896606)
\curveto(62.74582272,247.10896236)(62.7658227,247.1739623)(62.78582169,247.23396606)
\curveto(62.79582267,247.26396221)(62.80082266,247.31396216)(62.80082169,247.38396606)
\curveto(62.84082262,247.51396196)(62.8608226,247.69396178)(62.86082169,247.92396606)
\curveto(62.8608226,248.15396132)(62.84082262,248.34396113)(62.80082169,248.49396606)
\curveto(62.7608227,248.64396083)(62.72082274,248.77896069)(62.68082169,248.89896606)
\curveto(62.63082283,249.02896044)(62.57082289,249.14896032)(62.50082169,249.25896606)
\curveto(62.43082303,249.37896009)(62.35082311,249.48895998)(62.26082169,249.58896606)
\curveto(62.1608233,249.68895978)(62.05582341,249.77895969)(61.94582169,249.85896606)
\curveto(61.84582362,249.93895953)(61.74082372,250.01395946)(61.63082169,250.08396606)
\curveto(61.52082394,250.15395932)(61.44082402,250.24895922)(61.39082169,250.36896606)
\curveto(61.37082409,250.40895906)(61.35582411,250.473959)(61.34582169,250.56396606)
\curveto(61.33582413,250.66395881)(61.33582413,250.75395872)(61.34582169,250.83396606)
\curveto(61.34582412,250.92395855)(61.35082411,251.00895846)(61.36082169,251.08896606)
\curveto(61.37082409,251.1689583)(61.39082407,251.21895825)(61.42082169,251.23896606)
\curveto(61.49082397,251.32895814)(61.60582386,251.33395814)(61.76582169,251.25396606)
\curveto(62.03582343,251.11395836)(62.27582319,250.95895851)(62.48582169,250.78896606)
\curveto(62.80582266,250.52895894)(63.07082239,250.24895922)(63.28082169,249.94896606)
\curveto(63.48082198,249.65895981)(63.64582182,249.30396017)(63.77582169,248.88396606)
\curveto(63.81582165,248.7739607)(63.84082162,248.6689608)(63.85082169,248.56896606)
\curveto(63.87082159,248.468961)(63.89082157,248.35896111)(63.91082169,248.23896606)
\curveto(63.92082154,248.18896128)(63.92582154,248.13896133)(63.92582169,248.08896606)
\curveto(63.92582154,248.04896142)(63.93082153,248.00396147)(63.94082169,247.95396606)
\lineto(63.94082169,247.80396606)
\curveto(63.95082151,247.75396172)(63.95582151,247.69396178)(63.95582169,247.62396606)
\curveto(63.95582151,247.56396191)(63.95082151,247.51396196)(63.94082169,247.47396606)
\lineto(63.94082169,247.33896606)
\curveto(63.93082153,247.28896218)(63.92582154,247.24396223)(63.92582169,247.20396606)
\curveto(63.92582154,247.16396231)(63.92082154,247.12396235)(63.91082169,247.08396606)
\curveto(63.90082156,247.03396244)(63.89082157,246.97896249)(63.88082169,246.91896606)
\curveto(63.88082158,246.8689626)(63.87582159,246.81896265)(63.86582169,246.76896606)
\curveto(63.84582162,246.67896279)(63.82082164,246.58896288)(63.79082169,246.49896606)
\curveto(63.77082169,246.41896305)(63.74582172,246.34396313)(63.71582169,246.27396606)
\curveto(63.69582177,246.23396324)(63.68582178,246.19896327)(63.68582169,246.16896606)
\curveto(63.67582179,246.13896333)(63.6608218,246.10896336)(63.64082169,246.07896606)
\curveto(63.57082189,245.93896353)(63.48582198,245.79396368)(63.38582169,245.64396606)
\curveto(63.19582227,245.39396408)(62.9658225,245.19396428)(62.69582169,245.04396606)
\curveto(62.41582305,244.89396458)(62.10582336,244.78396469)(61.76582169,244.71396606)
\curveto(61.65582381,244.68396479)(61.54082392,244.6689648)(61.42082169,244.66896606)
\curveto(61.30082416,244.6689648)(61.18082428,244.65896481)(61.06082169,244.63896606)
\lineto(60.95582169,244.63896606)
\curveto(60.92582454,244.64896482)(60.88582458,244.65396482)(60.83582169,244.65396606)
\lineto(60.58082169,244.65396606)
\curveto(60.49082497,244.66396481)(60.40082506,244.6689648)(60.31082169,244.66896606)
\lineto(60.10082169,244.71396606)
\curveto(60.0608254,244.71396476)(60.00582546,244.71896475)(59.93582169,244.72896606)
\curveto(59.85582561,244.73896473)(59.79082567,244.75396472)(59.74082169,244.77396606)
\lineto(59.57582169,244.80396606)
\curveto(59.52582594,244.83396464)(59.47582599,244.84896462)(59.42582169,244.84896606)
\curveto(59.3658261,244.85896461)(59.31082615,244.8739646)(59.26082169,244.89396606)
\curveto(59.10082636,244.96396451)(58.94082652,245.02896444)(58.78082169,245.08896606)
\curveto(58.62082684,245.14896432)(58.47082699,245.22396425)(58.33082169,245.31396606)
\curveto(58.22082724,245.38396409)(58.11082735,245.44896402)(58.00082169,245.50896606)
\curveto(57.88082758,245.57896389)(57.7658277,245.65896381)(57.65582169,245.74896606)
\curveto(57.30582816,246.03896343)(57.00582846,246.34896312)(56.75582169,246.67896606)
\curveto(56.49582897,247.00896246)(56.28082918,247.39396208)(56.11082169,247.83396606)
\curveto(56.0608294,247.96396151)(56.02582944,248.09396138)(56.00582169,248.22396606)
\curveto(55.97582949,248.35396112)(55.94582952,248.49396098)(55.91582169,248.64396606)
\curveto(55.90582956,248.69396078)(55.90082956,248.73896073)(55.90082169,248.77896606)
\curveto(55.89082957,248.81896065)(55.88582958,248.86396061)(55.88582169,248.91396606)
\curveto(55.87582959,248.93396054)(55.87582959,248.95896051)(55.88582169,248.98896606)
\curveto(55.89582957,249.01896045)(55.89082957,249.04396043)(55.87082169,249.06396606)
\curveto(55.8608296,249.49395998)(55.90582956,249.85395962)(56.00582169,250.14396606)
\curveto(56.09582937,250.43395904)(56.22082924,250.68895878)(56.38082169,250.90896606)
\curveto(56.40082906,250.94895852)(56.43082903,250.97895849)(56.47082169,250.99896606)
\curveto(56.50082896,251.02895844)(56.52582894,251.05895841)(56.54582169,251.08896606)
\curveto(56.60582886,251.15895831)(56.67582879,251.22895824)(56.75582169,251.29896606)
\curveto(56.83582863,251.3689581)(56.91582855,251.42395805)(56.99582169,251.46396606)
\curveto(57.20582826,251.58395789)(57.40582806,251.67895779)(57.59582169,251.74896606)
\curveto(57.70582776,251.79895767)(57.82582764,251.82895764)(57.95582169,251.83896606)
\lineto(58.34582169,251.89896606)
\curveto(58.47582699,251.92895754)(58.61082685,251.93895753)(58.75082169,251.92896606)
\curveto(58.89082657,251.92895754)(59.03082643,251.93395754)(59.17082169,251.94396606)
\curveto(59.24082622,251.94395753)(59.31082615,251.93895753)(59.38082169,251.92896606)
\curveto(59.45082601,251.91895755)(59.52082594,251.91395756)(59.59082169,251.91396606)
\moveto(59.08082169,250.56396606)
\curveto(59.04082642,250.59395888)(58.99082647,250.62395885)(58.93082169,250.65396606)
\curveto(58.8608266,250.69395878)(58.79082667,250.70895876)(58.72082169,250.69896606)
\curveto(58.50082696,250.68895878)(58.29582717,250.64895882)(58.10582169,250.57896606)
\curveto(57.87582759,250.47895899)(57.68082778,250.35895911)(57.52082169,250.21896606)
\curveto(57.3608281,250.08895938)(57.22582824,249.89895957)(57.11582169,249.64896606)
\curveto(57.09582837,249.57895989)(57.08082838,249.50895996)(57.07082169,249.43896606)
\curveto(57.05082841,249.37896009)(57.03082843,249.30896016)(57.01082169,249.22896606)
\curveto(56.99082847,249.15896031)(56.98082848,249.07896039)(56.98082169,248.98896606)
\lineto(56.98082169,248.73396606)
\curveto(57.00082846,248.69396078)(57.01082845,248.65396082)(57.01082169,248.61396606)
\curveto(57.00082846,248.5739609)(57.00082846,248.53896093)(57.01082169,248.50896606)
\lineto(57.07082169,248.26896606)
\curveto(57.08082838,248.18896128)(57.09582837,248.11396136)(57.11582169,248.04396606)
\curveto(57.23582823,247.72396175)(57.38582808,247.45896201)(57.56582169,247.24896606)
\curveto(57.74582772,247.03896243)(57.97082749,246.83896263)(58.24082169,246.64896606)
\curveto(58.29082717,246.60896286)(58.35582711,246.56396291)(58.43582169,246.51396606)
\curveto(58.50582696,246.473963)(58.58582688,246.43396304)(58.67582169,246.39396606)
\curveto(58.7658267,246.35396312)(58.85082661,246.32896314)(58.93082169,246.31896606)
\curveto(59.01082645,246.31896315)(59.07082639,246.34396313)(59.11082169,246.39396606)
\curveto(59.17082629,246.46396301)(59.20082626,246.59396288)(59.20082169,246.78396606)
\curveto(59.19082627,246.98396249)(59.18582628,247.15396232)(59.18582169,247.29396606)
\lineto(59.18582169,249.57396606)
\curveto(59.18582628,249.72395975)(59.19082627,249.90395957)(59.20082169,250.11396606)
\curveto(59.20082626,250.32395915)(59.1608263,250.473959)(59.08082169,250.56396606)
}
}
{
\newrgbcolor{curcolor}{0 0 0}
\pscustom[linestyle=none,fillstyle=solid,fillcolor=curcolor]
{
}
}
{
\newrgbcolor{curcolor}{0 0 0}
\pscustom[linestyle=none,fillstyle=solid,fillcolor=curcolor]
{
\newpath
\moveto(56.03582169,258.55076294)
\lineto(56.03582169,258.98576294)
\curveto(56.03582943,259.13575943)(56.07582939,259.23575933)(56.15582169,259.28576294)
\curveto(56.23582923,259.31575925)(56.33582913,259.32075924)(56.45582169,259.30076294)
\lineto(56.81582169,259.24076294)
\lineto(58.24082169,258.95576294)
\lineto(60.50582169,258.50576294)
\curveto(60.72582474,258.45576011)(60.95582451,258.40576016)(61.19582169,258.35576294)
\curveto(61.42582404,258.31576025)(61.62582384,258.30076026)(61.79582169,258.31076294)
\curveto(62.24582322,258.38076018)(62.5608229,258.62075994)(62.74082169,259.03076294)
\curveto(62.83082263,259.23075933)(62.8658226,259.48575908)(62.84582169,259.79576294)
\curveto(62.81582265,260.11575845)(62.7608227,260.38075818)(62.68082169,260.59076294)
\curveto(62.54082292,260.94075762)(62.3658231,261.23575733)(62.15582169,261.47576294)
\curveto(61.93582353,261.71575685)(61.65082381,261.92575664)(61.30082169,262.10576294)
\curveto(61.22082424,262.15575641)(61.14082432,262.19075637)(61.06082169,262.21076294)
\curveto(60.98082448,262.24075632)(60.89582457,262.27575629)(60.80582169,262.31576294)
\curveto(60.75582471,262.33575623)(60.71082475,262.34575622)(60.67082169,262.34576294)
\curveto(60.63082483,262.34575622)(60.58582488,262.3607562)(60.53582169,262.39076294)
\lineto(60.22082169,262.45076294)
\curveto(60.14082532,262.49075607)(60.05082541,262.51575605)(59.95082169,262.52576294)
\curveto(59.84082562,262.53575603)(59.74082572,262.55075601)(59.65082169,262.57076294)
\lineto(58.48082169,262.81076294)
\lineto(56.89082169,263.12576294)
\curveto(56.77082869,263.14575542)(56.64582882,263.1657554)(56.51582169,263.18576294)
\curveto(56.37582909,263.21575535)(56.2658292,263.2607553)(56.18582169,263.32076294)
\curveto(56.13582933,263.37075519)(56.10582936,263.42575514)(56.09582169,263.48576294)
\curveto(56.07582939,263.54575502)(56.05582941,263.61575495)(56.03582169,263.69576294)
\lineto(56.03582169,263.92076294)
\curveto(56.03582943,264.04075452)(56.04082942,264.14575442)(56.05082169,264.23576294)
\curveto(56.0608294,264.33575423)(56.10582936,264.40075416)(56.18582169,264.43076294)
\curveto(56.23582923,264.47075409)(56.31082915,264.48075408)(56.41082169,264.46076294)
\curveto(56.50082896,264.44075412)(56.59582887,264.42075414)(56.69582169,264.40076294)
\lineto(57.71582169,264.20576294)
\lineto(61.75082169,263.39576294)
\lineto(63.10082169,263.12576294)
\curveto(63.22082224,263.10575546)(63.33582213,263.08075548)(63.44582169,263.05076294)
\curveto(63.54582192,263.02075554)(63.62082184,262.9657556)(63.67082169,262.88576294)
\curveto(63.70082176,262.84575572)(63.72582174,262.78075578)(63.74582169,262.69076294)
\curveto(63.75582171,262.61075595)(63.7658217,262.52075604)(63.77582169,262.42076294)
\curveto(63.77582169,262.33075623)(63.77082169,262.24075632)(63.76082169,262.15076294)
\curveto(63.75082171,262.07075649)(63.73082173,262.01575655)(63.70082169,261.98576294)
\curveto(63.6608218,261.94575662)(63.59582187,261.91575665)(63.50582169,261.89576294)
\curveto(63.465822,261.88575668)(63.41082205,261.88575668)(63.34082169,261.89576294)
\curveto(63.27082219,261.90575666)(63.20582226,261.91075665)(63.14582169,261.91076294)
\curveto(63.07582239,261.92075664)(63.02082244,261.91075665)(62.98082169,261.88076294)
\curveto(62.94082252,261.8607567)(62.92582254,261.82075674)(62.93582169,261.76076294)
\curveto(62.95582251,261.68075688)(63.01582245,261.59075697)(63.11582169,261.49076294)
\curveto(63.20582226,261.39075717)(63.27582219,261.30075726)(63.32582169,261.22076294)
\curveto(63.48582198,260.97075759)(63.62582184,260.69075787)(63.74582169,260.38076294)
\curveto(63.79582167,260.2607583)(63.82582164,260.14075842)(63.83582169,260.02076294)
\curveto(63.85582161,259.91075865)(63.88082158,259.79075877)(63.91082169,259.66076294)
\curveto(63.92082154,259.61075895)(63.92082154,259.55575901)(63.91082169,259.49576294)
\curveto(63.90082156,259.44575912)(63.90582156,259.39575917)(63.92582169,259.34576294)
\curveto(63.94582152,259.24575932)(63.94582152,259.15575941)(63.92582169,259.07576294)
\lineto(63.92582169,258.92576294)
\curveto(63.90582156,258.87575969)(63.89582157,258.81575975)(63.89582169,258.74576294)
\curveto(63.89582157,258.68575988)(63.89082157,258.63575993)(63.88082169,258.59576294)
\curveto(63.8608216,258.55576001)(63.85082161,258.51576005)(63.85082169,258.47576294)
\curveto(63.8608216,258.44576012)(63.85582161,258.40576016)(63.83582169,258.35576294)
\curveto(63.81582165,258.28576028)(63.79582167,258.21076035)(63.77582169,258.13076294)
\curveto(63.75582171,258.0607605)(63.72582174,257.99076057)(63.68582169,257.92076294)
\curveto(63.57582189,257.68076088)(63.43082203,257.49076107)(63.25082169,257.35076294)
\curveto(63.0608224,257.22076134)(62.83582263,257.12576144)(62.57582169,257.06576294)
\curveto(62.48582298,257.04576152)(62.39582307,257.03576153)(62.30582169,257.03576294)
\lineto(62.00582169,257.03576294)
\curveto(61.94582352,257.02576154)(61.89082357,257.02576154)(61.84082169,257.03576294)
\curveto(61.78082368,257.05576151)(61.71582375,257.0607615)(61.64582169,257.05076294)
\lineto(61.57082169,257.05076294)
\curveto(61.53082393,257.0607615)(61.49582397,257.0657615)(61.46582169,257.06576294)
\lineto(61.31582169,257.09576294)
\curveto(61.27582419,257.09576147)(61.23082423,257.10076146)(61.18082169,257.11076294)
\curveto(61.12082434,257.13076143)(61.0658244,257.14576142)(61.01582169,257.15576294)
\lineto(60.41582169,257.27576294)
\lineto(57.65582169,257.83076294)
\lineto(56.69582169,258.01076294)
\lineto(56.42582169,258.07076294)
\curveto(56.33582913,258.09076047)(56.2608292,258.12576044)(56.20082169,258.17576294)
\curveto(56.13082933,258.22576034)(56.08082938,258.31076025)(56.05082169,258.43076294)
\curveto(56.04082942,258.45076011)(56.04082942,258.47076009)(56.05082169,258.49076294)
\curveto(56.05082941,258.51076005)(56.04582942,258.53076003)(56.03582169,258.55076294)
}
}
{
\newrgbcolor{curcolor}{0 0 0}
\pscustom[linestyle=none,fillstyle=solid,fillcolor=curcolor]
{
\newpath
\moveto(55.85582169,268.91037231)
\curveto(55.84582962,269.63036666)(55.93082953,270.21536607)(56.11082169,270.66537231)
\curveto(56.28082918,271.12536516)(56.58582888,271.44536484)(57.02582169,271.62537231)
\curveto(57.13582833,271.67536461)(57.25082821,271.70536458)(57.37082169,271.71537231)
\curveto(57.48082798,271.73536455)(57.60582786,271.75036454)(57.74582169,271.76037231)
\curveto(57.81582765,271.77036452)(57.89082757,271.76036453)(57.97082169,271.73037231)
\curveto(58.04082742,271.71036458)(58.09582737,271.6853646)(58.13582169,271.65537231)
\curveto(58.15582731,271.63536465)(58.17582729,271.60536468)(58.19582169,271.56537231)
\curveto(58.20582726,271.53536475)(58.22082724,271.51036478)(58.24082169,271.49037231)
\curveto(58.2608272,271.43036486)(58.2658272,271.37536491)(58.25582169,271.32537231)
\curveto(58.24582722,271.285365)(58.24582722,271.24036505)(58.25582169,271.19037231)
\curveto(58.27582719,271.10036519)(58.28082718,270.9903653)(58.27082169,270.86037231)
\curveto(58.25082721,270.74036555)(58.22582724,270.65536563)(58.19582169,270.60537231)
\curveto(58.14582732,270.53536575)(58.08082738,270.49536579)(58.00082169,270.48537231)
\curveto(57.91082755,270.4853658)(57.82582764,270.46536582)(57.74582169,270.42537231)
\curveto(57.58582788,270.37536591)(57.44082802,270.28036601)(57.31082169,270.14037231)
\curveto(57.23082823,270.05036624)(57.17082829,269.94036635)(57.13082169,269.81037231)
\curveto(57.09082837,269.6903666)(57.05082841,269.56036673)(57.01082169,269.42037231)
\curveto(56.99082847,269.38036691)(56.98582848,269.33036696)(56.99582169,269.27037231)
\curveto(56.99582847,269.22036707)(56.99082847,269.17536711)(56.98082169,269.13537231)
\curveto(56.9608285,269.07536721)(56.95082851,269.00036729)(56.95082169,268.91037231)
\curveto(56.95082851,268.82036747)(56.9608285,268.74536754)(56.98082169,268.68537231)
\lineto(56.98082169,268.59537231)
\curveto(56.99082847,268.53536775)(57.00082846,268.48036781)(57.01082169,268.43037231)
\curveto(57.01082845,268.38036791)(57.01582845,268.33036796)(57.02582169,268.28037231)
\curveto(57.08582838,268.01036828)(57.17082829,267.77536851)(57.28082169,267.57537231)
\curveto(57.39082807,267.3853689)(57.57582789,267.23536905)(57.83582169,267.12537231)
\curveto(57.90582756,267.09536919)(57.97582749,267.08036921)(58.04582169,267.08037231)
\curveto(58.11582735,267.08036921)(58.17582729,267.0853692)(58.22582169,267.09537231)
\curveto(58.37582709,267.12536916)(58.48582698,267.17536911)(58.55582169,267.24537231)
\curveto(58.61582685,267.31536897)(58.68582678,267.41036888)(58.76582169,267.53037231)
\curveto(58.8658266,267.67036862)(58.94082652,267.83536845)(58.99082169,268.02537231)
\curveto(59.03082643,268.21536807)(59.08082638,268.40536788)(59.14082169,268.59537231)
\curveto(59.18082628,268.71536757)(59.21082625,268.83536745)(59.23082169,268.95537231)
\curveto(59.25082621,269.0853672)(59.28082618,269.21036708)(59.32082169,269.33037231)
\curveto(59.38082608,269.53036676)(59.44082602,269.72536656)(59.50082169,269.91537231)
\curveto(59.55082591,270.10536618)(59.61582585,270.290366)(59.69582169,270.47037231)
\curveto(59.71582575,270.52036577)(59.73582573,270.56536572)(59.75582169,270.60537231)
\curveto(59.77582569,270.65536563)(59.80082566,270.70536558)(59.83082169,270.75537231)
\curveto(59.95082551,270.92536536)(60.08582538,271.07036522)(60.23582169,271.19037231)
\curveto(60.38582508,271.31036498)(60.57582489,271.40036489)(60.80582169,271.46037231)
\lineto(61.09082169,271.46037231)
\curveto(61.1608243,271.46036483)(61.23582423,271.45536483)(61.31582169,271.44537231)
\curveto(61.38582408,271.43536485)(61.465824,271.42536486)(61.55582169,271.41537231)
\lineto(61.70582169,271.38537231)
\curveto(61.77582369,271.34536494)(61.84582362,271.31536497)(61.91582169,271.29537231)
\curveto(61.98582348,271.285365)(62.05582341,271.26536502)(62.12582169,271.23537231)
\curveto(62.23582323,271.1853651)(62.34082312,271.13036516)(62.44082169,271.07037231)
\curveto(62.54082292,271.01036528)(62.63082283,270.94536534)(62.71082169,270.87537231)
\curveto(62.97082249,270.66536562)(63.18082228,270.42036587)(63.34082169,270.14037231)
\curveto(63.49082197,269.86036643)(63.62082184,269.55536673)(63.73082169,269.22537231)
\curveto(63.7608217,269.12536716)(63.78082168,269.02536726)(63.79082169,268.92537231)
\curveto(63.81082165,268.82536746)(63.83582163,268.73036756)(63.86582169,268.64037231)
\curveto(63.88582158,268.53036776)(63.89582157,268.42536786)(63.89582169,268.32537231)
\curveto(63.89582157,268.22536806)(63.90582156,268.12536816)(63.92582169,268.02537231)
\lineto(63.92582169,267.87537231)
\curveto(63.93582153,267.82536846)(63.94082152,267.75536853)(63.94082169,267.66537231)
\curveto(63.94082152,267.57536871)(63.93582153,267.50536878)(63.92582169,267.45537231)
\lineto(63.92582169,267.29037231)
\curveto(63.90582156,267.23036906)(63.89582157,267.16536912)(63.89582169,267.09537231)
\curveto(63.90582156,267.02536926)(63.90082156,266.97036932)(63.88082169,266.93037231)
\curveto(63.87082159,266.88036941)(63.8658216,266.81536947)(63.86582169,266.73537231)
\curveto(63.84582162,266.65536963)(63.82582164,266.58036971)(63.80582169,266.51037231)
\curveto(63.79582167,266.44036985)(63.77582169,266.36536992)(63.74582169,266.28537231)
\curveto(63.64582182,265.99537029)(63.52082194,265.75037054)(63.37082169,265.55037231)
\curveto(63.22082224,265.35037094)(63.02582244,265.1903711)(62.78582169,265.07037231)
\curveto(62.65582281,265.01037128)(62.52082294,264.96037133)(62.38082169,264.92037231)
\curveto(62.24082322,264.8903714)(62.08582338,264.87037142)(61.91582169,264.86037231)
\curveto(61.85582361,264.85037144)(61.78582368,264.85537143)(61.70582169,264.87537231)
\curveto(61.61582385,264.89537139)(61.54582392,264.92037137)(61.49582169,264.95037231)
\curveto(61.45582401,264.9903713)(61.41582405,265.05037124)(61.37582169,265.13037231)
\curveto(61.35582411,265.18037111)(61.34582412,265.25037104)(61.34582169,265.34037231)
\curveto(61.33582413,265.44037085)(61.33582413,265.53037076)(61.34582169,265.61037231)
\curveto(61.35582411,265.70037059)(61.37082409,265.7853705)(61.39082169,265.86537231)
\curveto(61.40082406,265.95537033)(61.41582405,266.01037028)(61.43582169,266.03037231)
\curveto(61.48582398,266.0903702)(61.5608239,266.12037017)(61.66082169,266.12037231)
\curveto(61.75082371,266.13037016)(61.83582363,266.15037014)(61.91582169,266.18037231)
\curveto(62.13582333,266.23037006)(62.30582316,266.33036996)(62.42582169,266.48037231)
\curveto(62.51582295,266.58036971)(62.58582288,266.70036959)(62.63582169,266.84037231)
\curveto(62.68582278,266.98036931)(62.73582273,267.13036916)(62.78582169,267.29037231)
\lineto(62.83082169,267.60537231)
\lineto(62.83082169,267.69537231)
\curveto(62.85082261,267.75536853)(62.8608226,267.84036845)(62.86082169,267.95037231)
\curveto(62.8608226,268.07036822)(62.85082261,268.17536811)(62.83082169,268.26537231)
\curveto(62.83082263,268.33536795)(62.82582264,268.3903679)(62.81582169,268.43037231)
\curveto(62.80582266,268.4903678)(62.80082266,268.55036774)(62.80082169,268.61037231)
\curveto(62.79082267,268.67036762)(62.78082268,268.72536756)(62.77082169,268.77537231)
\curveto(62.69082277,269.0853672)(62.58582288,269.33536695)(62.45582169,269.52537231)
\curveto(62.32582314,269.72536656)(62.10582336,269.8903664)(61.79582169,270.02037231)
\curveto(61.74582372,270.05036624)(61.69082377,270.06536622)(61.63082169,270.06537231)
\curveto(61.57082389,270.07536621)(61.52582394,270.07536621)(61.49582169,270.06537231)
\curveto(61.30582416,270.05536623)(61.1658243,270.01536627)(61.07582169,269.94537231)
\curveto(60.97582449,269.87536641)(60.88582458,269.78036651)(60.80582169,269.66037231)
\curveto(60.74582472,269.58036671)(60.69582477,269.4853668)(60.65582169,269.37537231)
\lineto(60.53582169,269.07537231)
\curveto(60.52582494,269.04536724)(60.52082494,269.01536727)(60.52082169,268.98537231)
\curveto(60.52082494,268.96536732)(60.51082495,268.94536734)(60.49082169,268.92537231)
\curveto(60.38082508,268.60536768)(60.30082516,268.26536802)(60.25082169,267.90537231)
\curveto(60.19082527,267.55536873)(60.09582537,267.23536905)(59.96582169,266.94537231)
\curveto(59.92582554,266.85536943)(59.89082557,266.76536952)(59.86082169,266.67537231)
\curveto(59.83082563,266.59536969)(59.79082567,266.52036977)(59.74082169,266.45037231)
\curveto(59.63082583,266.28037001)(59.50582596,266.13037016)(59.36582169,266.00037231)
\curveto(59.22582624,265.87037042)(59.05082641,265.78037051)(58.84082169,265.73037231)
\curveto(58.77082669,265.71037058)(58.70082676,265.70037059)(58.63082169,265.70037231)
\lineto(58.40582169,265.70037231)
\curveto(58.28582718,265.6903706)(58.15082731,265.70537058)(58.00082169,265.74537231)
\curveto(57.84082762,265.7853705)(57.70582776,265.82537046)(57.59582169,265.86537231)
\curveto(57.54582792,265.89537039)(57.50582796,265.91537037)(57.47582169,265.92537231)
\curveto(57.43582803,265.94537034)(57.39582807,265.97037032)(57.35582169,266.00037231)
\curveto(57.12582834,266.13037016)(56.92582854,266.29037)(56.75582169,266.48037231)
\curveto(56.58582888,266.67036962)(56.43582903,266.88036941)(56.30582169,267.11037231)
\curveto(56.21582925,267.27036902)(56.14582932,267.44536884)(56.09582169,267.63537231)
\curveto(56.03582943,267.83536845)(55.98082948,268.04036825)(55.93082169,268.25037231)
\curveto(55.92082954,268.32036797)(55.91082955,268.3853679)(55.90082169,268.44537231)
\curveto(55.89082957,268.51536777)(55.88082958,268.5903677)(55.87082169,268.67037231)
\curveto(55.8608296,268.71036758)(55.8608296,268.75036754)(55.87082169,268.79037231)
\curveto(55.88082958,268.84036745)(55.87582959,268.88036741)(55.85582169,268.91037231)
}
}
{
\newrgbcolor{curcolor}{0 0 0}
\pscustom[linestyle=none,fillstyle=solid,fillcolor=curcolor]
{
\newpath
\moveto(56.03582169,274.40037231)
\lineto(56.03582169,274.83537231)
\curveto(56.03582943,274.9853688)(56.07582939,275.0853687)(56.15582169,275.13537231)
\curveto(56.23582923,275.16536862)(56.33582913,275.17036862)(56.45582169,275.15037231)
\lineto(56.81582169,275.09037231)
\lineto(58.24082169,274.80537231)
\lineto(60.50582169,274.35537231)
\curveto(60.72582474,274.30536948)(60.95582451,274.25536953)(61.19582169,274.20537231)
\curveto(61.42582404,274.16536962)(61.62582384,274.15036964)(61.79582169,274.16037231)
\curveto(62.24582322,274.23036956)(62.5608229,274.47036932)(62.74082169,274.88037231)
\curveto(62.83082263,275.08036871)(62.8658226,275.33536845)(62.84582169,275.64537231)
\curveto(62.81582265,275.96536782)(62.7608227,276.23036756)(62.68082169,276.44037231)
\curveto(62.54082292,276.790367)(62.3658231,277.0853667)(62.15582169,277.32537231)
\curveto(61.93582353,277.56536622)(61.65082381,277.77536601)(61.30082169,277.95537231)
\curveto(61.22082424,278.00536578)(61.14082432,278.04036575)(61.06082169,278.06037231)
\curveto(60.98082448,278.0903657)(60.89582457,278.12536566)(60.80582169,278.16537231)
\curveto(60.75582471,278.1853656)(60.71082475,278.19536559)(60.67082169,278.19537231)
\curveto(60.63082483,278.19536559)(60.58582488,278.21036558)(60.53582169,278.24037231)
\lineto(60.22082169,278.30037231)
\curveto(60.14082532,278.34036545)(60.05082541,278.36536542)(59.95082169,278.37537231)
\curveto(59.84082562,278.3853654)(59.74082572,278.40036539)(59.65082169,278.42037231)
\lineto(58.48082169,278.66037231)
\lineto(56.89082169,278.97537231)
\curveto(56.77082869,278.99536479)(56.64582882,279.01536477)(56.51582169,279.03537231)
\curveto(56.37582909,279.06536472)(56.2658292,279.11036468)(56.18582169,279.17037231)
\curveto(56.13582933,279.22036457)(56.10582936,279.27536451)(56.09582169,279.33537231)
\curveto(56.07582939,279.39536439)(56.05582941,279.46536432)(56.03582169,279.54537231)
\lineto(56.03582169,279.77037231)
\curveto(56.03582943,279.8903639)(56.04082942,279.99536379)(56.05082169,280.08537231)
\curveto(56.0608294,280.1853636)(56.10582936,280.25036354)(56.18582169,280.28037231)
\curveto(56.23582923,280.32036347)(56.31082915,280.33036346)(56.41082169,280.31037231)
\curveto(56.50082896,280.2903635)(56.59582887,280.27036352)(56.69582169,280.25037231)
\lineto(57.71582169,280.05537231)
\lineto(61.75082169,279.24537231)
\lineto(63.10082169,278.97537231)
\curveto(63.22082224,278.95536483)(63.33582213,278.93036486)(63.44582169,278.90037231)
\curveto(63.54582192,278.87036492)(63.62082184,278.81536497)(63.67082169,278.73537231)
\curveto(63.70082176,278.69536509)(63.72582174,278.63036516)(63.74582169,278.54037231)
\curveto(63.75582171,278.46036533)(63.7658217,278.37036542)(63.77582169,278.27037231)
\curveto(63.77582169,278.18036561)(63.77082169,278.0903657)(63.76082169,278.00037231)
\curveto(63.75082171,277.92036587)(63.73082173,277.86536592)(63.70082169,277.83537231)
\curveto(63.6608218,277.79536599)(63.59582187,277.76536602)(63.50582169,277.74537231)
\curveto(63.465822,277.73536605)(63.41082205,277.73536605)(63.34082169,277.74537231)
\curveto(63.27082219,277.75536603)(63.20582226,277.76036603)(63.14582169,277.76037231)
\curveto(63.07582239,277.77036602)(63.02082244,277.76036603)(62.98082169,277.73037231)
\curveto(62.94082252,277.71036608)(62.92582254,277.67036612)(62.93582169,277.61037231)
\curveto(62.95582251,277.53036626)(63.01582245,277.44036635)(63.11582169,277.34037231)
\curveto(63.20582226,277.24036655)(63.27582219,277.15036664)(63.32582169,277.07037231)
\curveto(63.48582198,276.82036697)(63.62582184,276.54036725)(63.74582169,276.23037231)
\curveto(63.79582167,276.11036768)(63.82582164,275.9903678)(63.83582169,275.87037231)
\curveto(63.85582161,275.76036803)(63.88082158,275.64036815)(63.91082169,275.51037231)
\curveto(63.92082154,275.46036833)(63.92082154,275.40536838)(63.91082169,275.34537231)
\curveto(63.90082156,275.29536849)(63.90582156,275.24536854)(63.92582169,275.19537231)
\curveto(63.94582152,275.09536869)(63.94582152,275.00536878)(63.92582169,274.92537231)
\lineto(63.92582169,274.77537231)
\curveto(63.90582156,274.72536906)(63.89582157,274.66536912)(63.89582169,274.59537231)
\curveto(63.89582157,274.53536925)(63.89082157,274.4853693)(63.88082169,274.44537231)
\curveto(63.8608216,274.40536938)(63.85082161,274.36536942)(63.85082169,274.32537231)
\curveto(63.8608216,274.29536949)(63.85582161,274.25536953)(63.83582169,274.20537231)
\curveto(63.81582165,274.13536965)(63.79582167,274.06036973)(63.77582169,273.98037231)
\curveto(63.75582171,273.91036988)(63.72582174,273.84036995)(63.68582169,273.77037231)
\curveto(63.57582189,273.53037026)(63.43082203,273.34037045)(63.25082169,273.20037231)
\curveto(63.0608224,273.07037072)(62.83582263,272.97537081)(62.57582169,272.91537231)
\curveto(62.48582298,272.89537089)(62.39582307,272.8853709)(62.30582169,272.88537231)
\lineto(62.00582169,272.88537231)
\curveto(61.94582352,272.87537091)(61.89082357,272.87537091)(61.84082169,272.88537231)
\curveto(61.78082368,272.90537088)(61.71582375,272.91037088)(61.64582169,272.90037231)
\lineto(61.57082169,272.90037231)
\curveto(61.53082393,272.91037088)(61.49582397,272.91537087)(61.46582169,272.91537231)
\lineto(61.31582169,272.94537231)
\curveto(61.27582419,272.94537084)(61.23082423,272.95037084)(61.18082169,272.96037231)
\curveto(61.12082434,272.98037081)(61.0658244,272.99537079)(61.01582169,273.00537231)
\lineto(60.41582169,273.12537231)
\lineto(57.65582169,273.68037231)
\lineto(56.69582169,273.86037231)
\lineto(56.42582169,273.92037231)
\curveto(56.33582913,273.94036985)(56.2608292,273.97536981)(56.20082169,274.02537231)
\curveto(56.13082933,274.07536971)(56.08082938,274.16036963)(56.05082169,274.28037231)
\curveto(56.04082942,274.30036949)(56.04082942,274.32036947)(56.05082169,274.34037231)
\curveto(56.05082941,274.36036943)(56.04582942,274.38036941)(56.03582169,274.40037231)
}
}
{
\newrgbcolor{curcolor}{0 0 0}
\pscustom[linestyle=none,fillstyle=solid,fillcolor=curcolor]
{
\newpath
\moveto(63.20582169,287.74498169)
\curveto(63.3658221,287.73497378)(63.50082196,287.68997382)(63.61082169,287.60998169)
\curveto(63.71082175,287.52997398)(63.78582168,287.43497408)(63.83582169,287.32498169)
\curveto(63.85582161,287.27497424)(63.8658216,287.21997429)(63.86582169,287.15998169)
\curveto(63.8658216,287.1099744)(63.87582159,287.04997446)(63.89582169,286.97998169)
\curveto(63.94582152,286.74997476)(63.93082153,286.53497498)(63.85082169,286.33498169)
\curveto(63.78082168,286.13497538)(63.69082177,286.0099755)(63.58082169,285.95998169)
\curveto(63.51082195,285.91997559)(63.43082203,285.88997562)(63.34082169,285.86998169)
\curveto(63.24082222,285.84997566)(63.1608223,285.8149757)(63.10082169,285.76498169)
\lineto(63.04082169,285.70498169)
\curveto(63.02082244,285.68497583)(63.01582245,285.65497586)(63.02582169,285.61498169)
\curveto(63.05582241,285.49497602)(63.11082235,285.37997613)(63.19082169,285.26998169)
\curveto(63.27082219,285.15997635)(63.34082212,285.05497646)(63.40082169,284.95498169)
\curveto(63.48082198,284.80497671)(63.55582191,284.64997686)(63.62582169,284.48998169)
\curveto(63.68582178,284.32997718)(63.74082172,284.15997735)(63.79082169,283.97998169)
\curveto(63.82082164,283.86997764)(63.84082162,283.75497776)(63.85082169,283.63498169)
\curveto(63.8608216,283.52497799)(63.87582159,283.4099781)(63.89582169,283.28998169)
\curveto(63.90582156,283.23997827)(63.91082155,283.19497832)(63.91082169,283.15498169)
\lineto(63.91082169,283.04998169)
\curveto(63.93082153,282.93997857)(63.93082153,282.83497868)(63.91082169,282.73498169)
\lineto(63.91082169,282.59998169)
\curveto(63.90082156,282.54997896)(63.89582157,282.49997901)(63.89582169,282.44998169)
\curveto(63.89582157,282.39997911)(63.88582158,282.35997915)(63.86582169,282.32998169)
\curveto(63.85582161,282.28997922)(63.85082161,282.25497926)(63.85082169,282.22498169)
\curveto(63.8608216,282.20497931)(63.8608216,282.17997933)(63.85082169,282.14998169)
\lineto(63.79082169,281.90998169)
\curveto(63.78082168,281.83997967)(63.7608217,281.77497974)(63.73082169,281.71498169)
\curveto(63.60082186,281.43498008)(63.45582201,281.21998029)(63.29582169,281.06998169)
\curveto(63.12582234,280.91998059)(62.89082257,280.8149807)(62.59082169,280.75498169)
\curveto(62.37082309,280.70498081)(62.10582336,280.7099808)(61.79582169,280.76998169)
\lineto(61.48082169,280.84498169)
\curveto(61.43082403,280.86498065)(61.38082408,280.87998063)(61.33082169,280.88998169)
\lineto(61.15082169,280.94998169)
\lineto(60.82082169,281.12998169)
\curveto(60.71082475,281.19998031)(60.61082485,281.26998024)(60.52082169,281.33998169)
\curveto(60.23082523,281.57997993)(60.01582545,281.86997964)(59.87582169,282.20998169)
\curveto(59.73582573,282.54997896)(59.61082585,282.9149786)(59.50082169,283.30498169)
\curveto(59.460826,283.45497806)(59.43082603,283.60497791)(59.41082169,283.75498169)
\curveto(59.39082607,283.9149776)(59.3658261,284.06997744)(59.33582169,284.21998169)
\curveto(59.31582615,284.29997721)(59.30582616,284.36997714)(59.30582169,284.42998169)
\curveto(59.30582616,284.49997701)(59.29582617,284.57497694)(59.27582169,284.65498169)
\curveto(59.25582621,284.72497679)(59.24582622,284.79497672)(59.24582169,284.86498169)
\curveto(59.23582623,284.94497657)(59.22082624,285.02497649)(59.20082169,285.10498169)
\curveto(59.14082632,285.36497615)(59.09082637,285.6099759)(59.05082169,285.83998169)
\curveto(59.00082646,286.06997544)(58.88582658,286.26997524)(58.70582169,286.43998169)
\curveto(58.62582684,286.509975)(58.52582694,286.57497494)(58.40582169,286.63498169)
\curveto(58.27582719,286.70497481)(58.13582733,286.73497478)(57.98582169,286.72498169)
\curveto(57.74582772,286.7149748)(57.55582791,286.66497485)(57.41582169,286.57498169)
\curveto(57.27582819,286.49497502)(57.1658283,286.35497516)(57.08582169,286.15498169)
\curveto(57.03582843,286.04497547)(57.00082846,285.9099756)(56.98082169,285.74998169)
\curveto(56.9608285,285.58997592)(56.95082851,285.41997609)(56.95082169,285.23998169)
\curveto(56.95082851,285.05997645)(56.9608285,284.87997663)(56.98082169,284.69998169)
\curveto(57.00082846,284.52997698)(57.03082843,284.37997713)(57.07082169,284.24998169)
\curveto(57.13082833,284.06997744)(57.21582825,283.88997762)(57.32582169,283.70998169)
\curveto(57.38582808,283.61997789)(57.465828,283.52997798)(57.56582169,283.43998169)
\curveto(57.65582781,283.35997815)(57.75582771,283.28497823)(57.86582169,283.21498169)
\curveto(57.94582752,283.16497835)(58.03082743,283.11997839)(58.12082169,283.07998169)
\curveto(58.21082725,283.03997847)(58.28082718,282.97997853)(58.33082169,282.89998169)
\curveto(58.3608271,282.84997866)(58.38582708,282.77497874)(58.40582169,282.67498169)
\curveto(58.41582705,282.57497894)(58.42082704,282.47497904)(58.42082169,282.37498169)
\curveto(58.42082704,282.27497924)(58.41582705,282.17997933)(58.40582169,282.08998169)
\curveto(58.38582708,281.99997951)(58.3608271,281.93997957)(58.33082169,281.90998169)
\curveto(58.30082716,281.86997964)(58.25082721,281.84497967)(58.18082169,281.83498169)
\curveto(58.11082735,281.83497968)(58.03582743,281.85497966)(57.95582169,281.89498169)
\curveto(57.82582764,281.94497957)(57.70582776,281.99997951)(57.59582169,282.05998169)
\curveto(57.47582799,282.11997939)(57.3608281,282.18497933)(57.25082169,282.25498169)
\curveto(56.90082856,282.514979)(56.63082883,282.8099787)(56.44082169,283.13998169)
\curveto(56.24082922,283.46997804)(56.08082938,283.85997765)(55.96082169,284.30998169)
\curveto(55.94082952,284.41997709)(55.92582954,284.52497699)(55.91582169,284.62498169)
\curveto(55.90582956,284.73497678)(55.89082957,284.84497667)(55.87082169,284.95498169)
\curveto(55.8608296,285.00497651)(55.8608296,285.06997644)(55.87082169,285.14998169)
\curveto(55.87082959,285.23997627)(55.8608296,285.29997621)(55.84082169,285.32998169)
\curveto(55.83082963,286.02997548)(55.91082955,286.61997489)(56.08082169,287.09998169)
\curveto(56.25082921,287.58997392)(56.57582889,287.89497362)(57.05582169,288.01498169)
\curveto(57.25582821,288.06497345)(57.49082797,288.06997344)(57.76082169,288.02998169)
\curveto(58.02082744,287.98997352)(58.29582717,287.93997357)(58.58582169,287.87998169)
\lineto(61.90082169,287.21998169)
\curveto(62.04082342,287.18997432)(62.17582329,287.16497435)(62.30582169,287.14498169)
\curveto(62.43582303,287.13497438)(62.54082292,287.14497437)(62.62082169,287.17498169)
\curveto(62.69082277,287.2149743)(62.74082272,287.26997424)(62.77082169,287.33998169)
\curveto(62.81082265,287.42997408)(62.84082262,287.509974)(62.86082169,287.57998169)
\curveto(62.87082259,287.65997385)(62.91582255,287.7099738)(62.99582169,287.72998169)
\curveto(63.02582244,287.74997376)(63.05582241,287.75497376)(63.08582169,287.74498169)
\lineto(63.20582169,287.74498169)
\moveto(61.54082169,285.92998169)
\curveto(61.40082406,286.01997549)(61.24082422,286.08497543)(61.06082169,286.12498169)
\curveto(60.87082459,286.16497535)(60.67582479,286.20497531)(60.47582169,286.24498169)
\curveto(60.3658251,286.26497525)(60.2658252,286.27997523)(60.17582169,286.28998169)
\curveto(60.08582538,286.29997521)(60.01582545,286.27497524)(59.96582169,286.21498169)
\curveto(59.94582552,286.18497533)(59.93582553,286.1149754)(59.93582169,286.00498169)
\curveto(59.95582551,285.98497553)(59.9658255,285.94997556)(59.96582169,285.89998169)
\curveto(59.9658255,285.84997566)(59.97582549,285.79997571)(59.99582169,285.74998169)
\curveto(60.01582545,285.66997584)(60.03582543,285.57497594)(60.05582169,285.46498169)
\lineto(60.11582169,285.16498169)
\curveto(60.11582535,285.13497638)(60.12082534,285.09997641)(60.13082169,285.05998169)
\lineto(60.13082169,284.95498169)
\curveto(60.17082529,284.79497672)(60.19582527,284.62497689)(60.20582169,284.44498169)
\curveto(60.20582526,284.27497724)(60.22582524,284.1099774)(60.26582169,283.94998169)
\curveto(60.28582518,283.85997765)(60.30582516,283.77997773)(60.32582169,283.70998169)
\curveto(60.33582513,283.64997786)(60.35082511,283.57497794)(60.37082169,283.48498169)
\curveto(60.42082504,283.3149782)(60.48582498,283.14997836)(60.56582169,282.98998169)
\curveto(60.63582483,282.83997867)(60.72582474,282.70497881)(60.83582169,282.58498169)
\curveto(60.94582452,282.46497905)(61.08082438,282.36497915)(61.24082169,282.28498169)
\curveto(61.39082407,282.20497931)(61.57582389,282.14497937)(61.79582169,282.10498169)
\curveto(61.89582357,282.08497943)(61.99082347,282.08497943)(62.08082169,282.10498169)
\curveto(62.1608233,282.12497939)(62.23582323,282.15497936)(62.30582169,282.19498169)
\curveto(62.41582305,282.24497927)(62.51082295,282.32497919)(62.59082169,282.43498169)
\curveto(62.6608228,282.55497896)(62.72082274,282.68497883)(62.77082169,282.82498169)
\curveto(62.78082268,282.87497864)(62.78582268,282.92497859)(62.78582169,282.97498169)
\curveto(62.78582268,283.02497849)(62.79082267,283.07497844)(62.80082169,283.12498169)
\curveto(62.82082264,283.19497832)(62.83582263,283.27997823)(62.84582169,283.37998169)
\curveto(62.84582262,283.47997803)(62.83582263,283.56997794)(62.81582169,283.64998169)
\curveto(62.79582267,283.7099778)(62.79082267,283.76997774)(62.80082169,283.82998169)
\curveto(62.80082266,283.88997762)(62.79082267,283.94997756)(62.77082169,284.00998169)
\curveto(62.75082271,284.09997741)(62.73582273,284.17997733)(62.72582169,284.24998169)
\curveto(62.71582275,284.32997718)(62.69582277,284.4099771)(62.66582169,284.48998169)
\curveto(62.54582292,284.79997671)(62.40082306,285.07497644)(62.23082169,285.31498169)
\curveto(62.0608234,285.55497596)(61.83082363,285.75997575)(61.54082169,285.92998169)
}
}
{
\newrgbcolor{curcolor}{0 0 0}
\pscustom[linestyle=none,fillstyle=solid,fillcolor=curcolor]
{
\newpath
\moveto(55.85582169,294.03162231)
\curveto(55.85582961,294.26161593)(55.91582955,294.37661582)(56.03582169,294.37662231)
\curveto(56.14582932,294.38661581)(56.31082915,294.37161582)(56.53082169,294.33162231)
\curveto(56.63082883,294.31161588)(56.72582874,294.2916159)(56.81582169,294.27162231)
\curveto(56.90582856,294.25161594)(56.98082848,294.21161598)(57.04082169,294.15162231)
\curveto(57.12082834,294.07161612)(57.1658283,293.97661622)(57.17582169,293.86662231)
\curveto(57.17582829,293.75661644)(57.19082827,293.64161655)(57.22082169,293.52162231)
\curveto(57.25082821,293.38161681)(57.28082818,293.24661695)(57.31082169,293.11662231)
\curveto(57.34082812,292.98661721)(57.38082808,292.86161733)(57.43082169,292.74162231)
\curveto(57.5608279,292.42161777)(57.74082772,292.14661805)(57.97082169,291.91662231)
\curveto(58.19082727,291.6966185)(58.44582702,291.4916187)(58.73582169,291.30162231)
\curveto(58.84582662,291.24161895)(58.9608265,291.18661901)(59.08082169,291.13662231)
\curveto(59.19082627,291.0966191)(59.30582616,291.05161914)(59.42582169,291.00162231)
\curveto(59.47582599,290.98161921)(59.52582594,290.96661923)(59.57582169,290.95662231)
\curveto(59.62582584,290.95661924)(59.67582579,290.94661925)(59.72582169,290.92662231)
\curveto(59.84582562,290.87661932)(59.98582548,290.83661936)(60.14582169,290.80662231)
\curveto(60.29582517,290.77661942)(60.44082502,290.75161944)(60.58082169,290.73162231)
\lineto(62.42582169,290.35662231)
\curveto(62.53582293,290.33661986)(62.65082281,290.31161988)(62.77082169,290.28162231)
\curveto(62.89082257,290.26161993)(63.00582246,290.23661996)(63.11582169,290.20662231)
\curveto(63.22582224,290.17662002)(63.32082214,290.15162004)(63.40082169,290.13162231)
\curveto(63.48082198,290.12162007)(63.55082191,290.08662011)(63.61082169,290.02662231)
\curveto(63.68082178,289.96662023)(63.72082174,289.88162031)(63.73082169,289.77162231)
\curveto(63.74082172,289.66162053)(63.74582172,289.54662065)(63.74582169,289.42662231)
\lineto(63.74582169,289.15662231)
\curveto(63.72582174,289.11662108)(63.71082175,289.07162112)(63.70082169,289.02162231)
\curveto(63.68082178,288.98162121)(63.65582181,288.95662124)(63.62582169,288.94662231)
\curveto(63.55582191,288.90662129)(63.47082199,288.8966213)(63.37082169,288.91662231)
\lineto(63.04082169,288.97662231)
\lineto(61.88582169,289.20162231)
\lineto(57.73082169,290.04162231)
\lineto(56.69582169,290.23662231)
\curveto(56.58582888,290.26661993)(56.48582898,290.2916199)(56.39582169,290.31162231)
\curveto(56.29582917,290.33161986)(56.21082925,290.37661982)(56.14082169,290.44662231)
\curveto(56.10082936,290.48661971)(56.07082939,290.54161965)(56.05082169,290.61162231)
\curveto(56.03082943,290.6916195)(56.02082944,290.77661942)(56.02082169,290.86662231)
\curveto(56.01082945,290.96661923)(56.01082945,291.05661914)(56.02082169,291.13662231)
\curveto(56.03082943,291.22661897)(56.04582942,291.2966189)(56.06582169,291.34662231)
\curveto(56.09582937,291.41661878)(56.15582931,291.45661874)(56.24582169,291.46662231)
\curveto(56.32582914,291.47661872)(56.41582905,291.47161872)(56.51582169,291.45162231)
\curveto(56.60582886,291.43161876)(56.70582876,291.41161878)(56.81582169,291.39162231)
\curveto(56.91582855,291.37161882)(57.00582846,291.37161882)(57.08582169,291.39162231)
\curveto(57.10582836,291.40161879)(57.12082834,291.40661879)(57.13082169,291.40662231)
\lineto(57.17582169,291.45162231)
\curveto(57.17582829,291.56161863)(57.13082833,291.66161853)(57.04082169,291.75162231)
\curveto(56.94082852,291.84161835)(56.8608286,291.91661828)(56.80082169,291.97662231)
\lineto(56.71082169,292.08162231)
\curveto(56.60082886,292.191618)(56.48582898,292.33661786)(56.36582169,292.51662231)
\curveto(56.24582922,292.70661749)(56.15582931,292.87661732)(56.09582169,293.02662231)
\curveto(56.04582942,293.12661707)(56.01082945,293.22661697)(55.99082169,293.32662231)
\curveto(55.9608295,293.43661676)(55.93082953,293.55161664)(55.90082169,293.67162231)
\curveto(55.89082957,293.73161646)(55.88582958,293.7916164)(55.88582169,293.85162231)
\lineto(55.85582169,294.03162231)
}
}
{
\newrgbcolor{curcolor}{0 0 0}
\pscustom[linestyle=none,fillstyle=solid,fillcolor=curcolor]
{
\newpath
\moveto(54.53582169,295.93638794)
\curveto(54.47583099,295.86638496)(54.37083109,295.84638498)(54.22082169,295.87638794)
\curveto(54.0608314,295.90638492)(53.90583156,295.93638489)(53.75582169,295.96638794)
\curveto(53.67583179,295.97638485)(53.59083187,295.99138484)(53.50082169,296.01138794)
\curveto(53.41083205,296.0313848)(53.33583213,296.06138477)(53.27582169,296.10138794)
\curveto(53.19583227,296.16138467)(53.13583233,296.25138458)(53.09582169,296.37138794)
\curveto(53.08583238,296.40138443)(53.08583238,296.4263844)(53.09582169,296.44638794)
\curveto(53.09583237,296.46638436)(53.09083237,296.49138434)(53.08082169,296.52138794)
\curveto(53.08083238,296.69138414)(53.08583238,296.84638398)(53.09582169,296.98638794)
\curveto(53.10583236,297.13638369)(53.1658323,297.2263836)(53.27582169,297.25638794)
\curveto(53.33583213,297.27638355)(53.41083205,297.27638355)(53.50082169,297.25638794)
\curveto(53.58083188,297.23638359)(53.6658318,297.22138361)(53.75582169,297.21138794)
\curveto(53.93583153,297.17138366)(54.10583136,297.1313837)(54.26582169,297.09138794)
\curveto(54.42583104,297.06138377)(54.53083093,296.97638385)(54.58082169,296.83638794)
\curveto(54.60083086,296.77638405)(54.61083085,296.71638411)(54.61082169,296.65638794)
\lineto(54.61082169,296.49138794)
\lineto(54.61082169,296.17638794)
\curveto(54.61083085,296.07638475)(54.58583088,295.99638483)(54.53582169,295.93638794)
\moveto(63.04082169,295.35138794)
\curveto(63.14082232,295.3313855)(63.24582222,295.31138552)(63.35582169,295.29138794)
\curveto(63.45582201,295.28138555)(63.53582193,295.24138559)(63.59582169,295.17138794)
\curveto(63.65582181,295.1313857)(63.69582177,295.08138575)(63.71582169,295.02138794)
\curveto(63.72582174,294.96138587)(63.74082172,294.88638594)(63.76082169,294.79638794)
\lineto(63.76082169,294.57138794)
\curveto(63.7608217,294.44138639)(63.75582171,294.3313865)(63.74582169,294.24138794)
\curveto(63.72582174,294.15138668)(63.67582179,294.08638674)(63.59582169,294.04638794)
\curveto(63.53582193,294.0263868)(63.460822,294.02138681)(63.37082169,294.03138794)
\curveto(63.27082219,294.05138678)(63.17582229,294.07138676)(63.08582169,294.09138794)
\lineto(56.74082169,295.36638794)
\curveto(56.63082883,295.38638544)(56.52582894,295.40638542)(56.42582169,295.42638794)
\curveto(56.31582915,295.44638538)(56.23082923,295.48638534)(56.17082169,295.54638794)
\curveto(56.12082934,295.58638524)(56.09082937,295.6313852)(56.08082169,295.68138794)
\curveto(56.07082939,295.74138509)(56.05582941,295.80138503)(56.03582169,295.86138794)
\curveto(56.03582943,295.88138495)(56.04082942,295.90138493)(56.05082169,295.92138794)
\curveto(56.05082941,295.95138488)(56.04582942,295.97638485)(56.03582169,295.99638794)
\curveto(56.03582943,296.1263847)(56.04082942,296.25638457)(56.05082169,296.38638794)
\curveto(56.05082941,296.5263843)(56.09082937,296.61138422)(56.17082169,296.64138794)
\curveto(56.23082923,296.68138415)(56.31082915,296.69138414)(56.41082169,296.67138794)
\curveto(56.50082896,296.65138418)(56.59582887,296.6313842)(56.69582169,296.61138794)
\lineto(63.04082169,295.35138794)
}
}
{
\newrgbcolor{curcolor}{0 0 0}
\pscustom[linestyle=none,fillstyle=solid,fillcolor=curcolor]
{
\newpath
\moveto(59.56082169,305.11123169)
\curveto(59.62082584,305.1212228)(59.71582575,305.11122281)(59.84582169,305.08123169)
\curveto(59.9658255,305.06122286)(60.05082541,305.04122288)(60.10082169,305.02123169)
\lineto(60.25082169,304.99123169)
\curveto(60.33082513,304.96122296)(60.40582506,304.93622298)(60.47582169,304.91623169)
\curveto(60.53582493,304.90622301)(60.60582486,304.88622303)(60.68582169,304.85623169)
\curveto(60.74582472,304.82622309)(60.80582466,304.80122312)(60.86582169,304.78123169)
\curveto(60.92582454,304.77122315)(60.98582448,304.74622317)(61.04582169,304.70623169)
\lineto(61.43582169,304.52623169)
\curveto(61.5658239,304.47622344)(61.68582378,304.41122351)(61.79582169,304.33123169)
\curveto(62.27582319,304.03122389)(62.68082278,303.67122425)(63.01082169,303.25123169)
\curveto(63.33082213,302.84122508)(63.57582189,302.36122556)(63.74582169,301.81123169)
\curveto(63.78582168,301.70122622)(63.81582165,301.58122634)(63.83582169,301.45123169)
\curveto(63.85582161,301.3212266)(63.87582159,301.18622673)(63.89582169,301.04623169)
\curveto(63.90582156,300.98622693)(63.91082155,300.921227)(63.91082169,300.85123169)
\curveto(63.92082154,300.79122713)(63.92582154,300.73122719)(63.92582169,300.67123169)
\curveto(63.93582153,300.63122729)(63.94082152,300.57122735)(63.94082169,300.49123169)
\curveto(63.94082152,300.4212275)(63.93582153,300.37122755)(63.92582169,300.34123169)
\curveto(63.91582155,300.30122762)(63.91082155,300.26122766)(63.91082169,300.22123169)
\curveto(63.92082154,300.18122774)(63.92082154,300.14622777)(63.91082169,300.11623169)
\lineto(63.91082169,300.02623169)
\lineto(63.86582169,299.68123169)
\lineto(63.74582169,299.29123169)
\curveto(63.70582176,299.17122875)(63.6608218,299.05622886)(63.61082169,298.94623169)
\curveto(63.41082205,298.53622938)(63.15082231,298.2162297)(62.83082169,297.98623169)
\curveto(62.51082295,297.76623015)(62.12082334,297.60623031)(61.66082169,297.50623169)
\curveto(61.5608239,297.47623044)(61.460824,297.45623046)(61.36082169,297.44623169)
\lineto(61.04582169,297.44623169)
\curveto(61.00582446,297.43623048)(60.97582449,297.43623048)(60.95582169,297.44623169)
\curveto(60.92582454,297.45623046)(60.89082457,297.46123046)(60.85082169,297.46123169)
\curveto(60.77082469,297.46123046)(60.69082477,297.46623045)(60.61082169,297.47623169)
\curveto(60.52082494,297.48623043)(60.43582503,297.49123043)(60.35582169,297.49123169)
\curveto(60.30582516,297.50123042)(60.2658252,297.50623041)(60.23582169,297.50623169)
\curveto(60.19582527,297.5162304)(60.15082531,297.5212304)(60.10082169,297.52123169)
\curveto(60.05082541,297.5212304)(59.9658255,297.53123039)(59.84582169,297.55123169)
\curveto(59.71582575,297.58123034)(59.62082584,297.61123031)(59.56082169,297.64123169)
\curveto(59.49082597,297.68123024)(59.42082604,297.70123022)(59.35082169,297.70123169)
\curveto(59.28082618,297.70123022)(59.21082625,297.7212302)(59.14082169,297.76123169)
\curveto(59.09082637,297.78123014)(59.05082641,297.79623012)(59.02082169,297.80623169)
\curveto(58.98082648,297.8162301)(58.93582653,297.83123009)(58.88582169,297.85123169)
\curveto(58.7658267,297.91123001)(58.64582682,297.96122996)(58.52582169,298.00123169)
\curveto(58.40582706,298.05122987)(58.29082717,298.1162298)(58.18082169,298.19623169)
\curveto(57.81082765,298.4162295)(57.48082798,298.66122926)(57.19082169,298.93123169)
\curveto(56.89082857,299.21122871)(56.64082882,299.52622839)(56.44082169,299.87623169)
\curveto(56.3608291,300.00622791)(56.29582917,300.14122778)(56.24582169,300.28123169)
\lineto(56.06582169,300.73123169)
\curveto(56.01582945,300.86122706)(55.98582948,300.99622692)(55.97582169,301.13623169)
\curveto(55.95582951,301.27622664)(55.92582954,301.4212265)(55.88582169,301.57123169)
\lineto(55.88582169,301.76623169)
\lineto(55.85582169,301.97623169)
\curveto(55.84582962,302.86622505)(56.03082943,303.56622435)(56.41082169,304.07623169)
\curveto(56.79082867,304.59622332)(57.28582818,304.921223)(57.89582169,305.05123169)
\curveto(57.99582747,305.08122284)(58.09582737,305.10122282)(58.19582169,305.11123169)
\curveto(58.29582717,305.1212228)(58.40082706,305.13622278)(58.51082169,305.15623169)
\curveto(58.62082684,305.16622275)(58.74082672,305.16622275)(58.87082169,305.15623169)
\lineto(59.24582169,305.15623169)
\curveto(59.29582617,305.15622276)(59.35082611,305.14622277)(59.41082169,305.12623169)
\curveto(59.460826,305.1162228)(59.51082595,305.11122281)(59.56082169,305.11123169)
\moveto(60.41582169,303.61123169)
\curveto(60.34582512,303.64122428)(60.2658252,303.66122426)(60.17582169,303.67123169)
\curveto(60.08582538,303.69122423)(60.00082546,303.70622421)(59.92082169,303.71623169)
\curveto(59.53082593,303.79622412)(59.20082626,303.83122409)(58.93082169,303.82123169)
\curveto(58.85082661,303.80122412)(58.77082669,303.78622413)(58.69082169,303.77623169)
\curveto(58.61082685,303.77622414)(58.53582693,303.77122415)(58.46582169,303.76123169)
\curveto(57.81582765,303.61122431)(57.3658281,303.25622466)(57.11582169,302.69623169)
\curveto(57.08582838,302.62622529)(57.0658284,302.55122537)(57.05582169,302.47123169)
\curveto(57.03582843,302.40122552)(57.01582845,302.32622559)(56.99582169,302.24623169)
\curveto(56.97582849,302.17622574)(56.9658285,302.09622582)(56.96582169,302.00623169)
\lineto(56.96582169,301.73623169)
\lineto(57.01082169,301.45123169)
\curveto(57.03082843,301.35122657)(57.05582841,301.25622666)(57.08582169,301.16623169)
\curveto(57.10582836,301.07622684)(57.13582833,300.98622693)(57.17582169,300.89623169)
\curveto(57.19582827,300.82622709)(57.22582824,300.75622716)(57.26582169,300.68623169)
\curveto(57.30582816,300.6162273)(57.34582812,300.55122737)(57.38582169,300.49123169)
\curveto(57.55582791,300.2212277)(57.7608277,299.98622793)(58.00082169,299.78623169)
\curveto(58.24082722,299.58622833)(58.52082694,299.40122852)(58.84082169,299.23123169)
\curveto(58.94082652,299.18122874)(59.04582642,299.14122878)(59.15582169,299.11123169)
\curveto(59.25582621,299.08122884)(59.3608261,299.04122888)(59.47082169,298.99123169)
\curveto(59.51082595,298.98122894)(59.57582589,298.96622895)(59.66582169,298.94623169)
\curveto(59.69582577,298.92622899)(59.73082573,298.916229)(59.77082169,298.91623169)
\curveto(59.81082565,298.916229)(59.85582561,298.91122901)(59.90582169,298.90123169)
\lineto(60.20582169,298.84123169)
\curveto(60.30582516,298.8212291)(60.39582507,298.81122911)(60.47582169,298.81123169)
\lineto(60.65582169,298.81123169)
\curveto(60.75582471,298.81122911)(60.85582461,298.80622911)(60.95582169,298.79623169)
\curveto(61.04582442,298.79622912)(61.13082433,298.80622911)(61.21082169,298.82623169)
\curveto(61.45082401,298.87622904)(61.67582379,298.94622897)(61.88582169,299.03623169)
\curveto(62.09582337,299.13622878)(62.27082319,299.27122865)(62.41082169,299.44123169)
\curveto(62.44082302,299.49122843)(62.465823,299.53122839)(62.48582169,299.56123169)
\curveto(62.50582296,299.60122832)(62.53082293,299.64122828)(62.56082169,299.68123169)
\curveto(62.61082285,299.75122817)(62.65582281,299.83122809)(62.69582169,299.92123169)
\curveto(62.72582274,300.01122791)(62.75582271,300.10622781)(62.78582169,300.20623169)
\curveto(62.80582266,300.25622766)(62.81582265,300.30122762)(62.81582169,300.34123169)
\curveto(62.80582266,300.39122753)(62.80582266,300.44122748)(62.81582169,300.49123169)
\curveto(62.82582264,300.5212274)(62.83582263,300.58122734)(62.84582169,300.67123169)
\curveto(62.85582261,300.76122716)(62.85082261,300.83622708)(62.83082169,300.89623169)
\curveto(62.82082264,300.93622698)(62.82082264,300.97622694)(62.83082169,301.01623169)
\curveto(62.83082263,301.05622686)(62.82082264,301.09622682)(62.80082169,301.13623169)
\curveto(62.78082268,301.2162267)(62.7658227,301.29622662)(62.75582169,301.37623169)
\curveto(62.73582273,301.46622645)(62.71082275,301.55122637)(62.68082169,301.63123169)
\curveto(62.54082292,301.99122593)(62.34582312,302.30122562)(62.09582169,302.56123169)
\curveto(61.84582362,302.8212251)(61.55082391,303.05622486)(61.21082169,303.26623169)
\curveto(61.09082437,303.34622457)(60.9658245,303.40622451)(60.83582169,303.44623169)
\curveto(60.69582477,303.48622443)(60.55582491,303.54122438)(60.41582169,303.61123169)
}
}
{
\newrgbcolor{curcolor}{0 0 0}
\pscustom[linestyle=none,fillstyle=solid,fillcolor=curcolor]
{
\newpath
\moveto(55.85582169,309.77951294)
\curveto(55.84582962,310.49950728)(55.93082953,311.0845067)(56.11082169,311.53451294)
\curveto(56.28082918,311.99450579)(56.58582888,312.31450547)(57.02582169,312.49451294)
\curveto(57.13582833,312.54450524)(57.25082821,312.57450521)(57.37082169,312.58451294)
\curveto(57.48082798,312.60450518)(57.60582786,312.61950516)(57.74582169,312.62951294)
\curveto(57.81582765,312.63950514)(57.89082757,312.62950515)(57.97082169,312.59951294)
\curveto(58.04082742,312.5795052)(58.09582737,312.55450523)(58.13582169,312.52451294)
\curveto(58.15582731,312.50450528)(58.17582729,312.47450531)(58.19582169,312.43451294)
\curveto(58.20582726,312.40450538)(58.22082724,312.3795054)(58.24082169,312.35951294)
\curveto(58.2608272,312.29950548)(58.2658272,312.24450554)(58.25582169,312.19451294)
\curveto(58.24582722,312.15450563)(58.24582722,312.10950567)(58.25582169,312.05951294)
\curveto(58.27582719,311.96950581)(58.28082718,311.85950592)(58.27082169,311.72951294)
\curveto(58.25082721,311.60950617)(58.22582724,311.52450626)(58.19582169,311.47451294)
\curveto(58.14582732,311.40450638)(58.08082738,311.36450642)(58.00082169,311.35451294)
\curveto(57.91082755,311.35450643)(57.82582764,311.33450645)(57.74582169,311.29451294)
\curveto(57.58582788,311.24450654)(57.44082802,311.14950663)(57.31082169,311.00951294)
\curveto(57.23082823,310.91950686)(57.17082829,310.80950697)(57.13082169,310.67951294)
\curveto(57.09082837,310.55950722)(57.05082841,310.42950735)(57.01082169,310.28951294)
\curveto(56.99082847,310.24950753)(56.98582848,310.19950758)(56.99582169,310.13951294)
\curveto(56.99582847,310.08950769)(56.99082847,310.04450774)(56.98082169,310.00451294)
\curveto(56.9608285,309.94450784)(56.95082851,309.86950791)(56.95082169,309.77951294)
\curveto(56.95082851,309.68950809)(56.9608285,309.61450817)(56.98082169,309.55451294)
\lineto(56.98082169,309.46451294)
\curveto(56.99082847,309.40450838)(57.00082846,309.34950843)(57.01082169,309.29951294)
\curveto(57.01082845,309.24950853)(57.01582845,309.19950858)(57.02582169,309.14951294)
\curveto(57.08582838,308.8795089)(57.17082829,308.64450914)(57.28082169,308.44451294)
\curveto(57.39082807,308.25450953)(57.57582789,308.10450968)(57.83582169,307.99451294)
\curveto(57.90582756,307.96450982)(57.97582749,307.94950983)(58.04582169,307.94951294)
\curveto(58.11582735,307.94950983)(58.17582729,307.95450983)(58.22582169,307.96451294)
\curveto(58.37582709,307.99450979)(58.48582698,308.04450974)(58.55582169,308.11451294)
\curveto(58.61582685,308.1845096)(58.68582678,308.2795095)(58.76582169,308.39951294)
\curveto(58.8658266,308.53950924)(58.94082652,308.70450908)(58.99082169,308.89451294)
\curveto(59.03082643,309.0845087)(59.08082638,309.27450851)(59.14082169,309.46451294)
\curveto(59.18082628,309.5845082)(59.21082625,309.70450808)(59.23082169,309.82451294)
\curveto(59.25082621,309.95450783)(59.28082618,310.0795077)(59.32082169,310.19951294)
\curveto(59.38082608,310.39950738)(59.44082602,310.59450719)(59.50082169,310.78451294)
\curveto(59.55082591,310.97450681)(59.61582585,311.15950662)(59.69582169,311.33951294)
\curveto(59.71582575,311.38950639)(59.73582573,311.43450635)(59.75582169,311.47451294)
\curveto(59.77582569,311.52450626)(59.80082566,311.57450621)(59.83082169,311.62451294)
\curveto(59.95082551,311.79450599)(60.08582538,311.93950584)(60.23582169,312.05951294)
\curveto(60.38582508,312.1795056)(60.57582489,312.26950551)(60.80582169,312.32951294)
\lineto(61.09082169,312.32951294)
\curveto(61.1608243,312.32950545)(61.23582423,312.32450546)(61.31582169,312.31451294)
\curveto(61.38582408,312.30450548)(61.465824,312.29450549)(61.55582169,312.28451294)
\lineto(61.70582169,312.25451294)
\curveto(61.77582369,312.21450557)(61.84582362,312.1845056)(61.91582169,312.16451294)
\curveto(61.98582348,312.15450563)(62.05582341,312.13450565)(62.12582169,312.10451294)
\curveto(62.23582323,312.05450573)(62.34082312,311.99950578)(62.44082169,311.93951294)
\curveto(62.54082292,311.8795059)(62.63082283,311.81450597)(62.71082169,311.74451294)
\curveto(62.97082249,311.53450625)(63.18082228,311.28950649)(63.34082169,311.00951294)
\curveto(63.49082197,310.72950705)(63.62082184,310.42450736)(63.73082169,310.09451294)
\curveto(63.7608217,309.99450779)(63.78082168,309.89450789)(63.79082169,309.79451294)
\curveto(63.81082165,309.69450809)(63.83582163,309.59950818)(63.86582169,309.50951294)
\curveto(63.88582158,309.39950838)(63.89582157,309.29450849)(63.89582169,309.19451294)
\curveto(63.89582157,309.09450869)(63.90582156,308.99450879)(63.92582169,308.89451294)
\lineto(63.92582169,308.74451294)
\curveto(63.93582153,308.69450909)(63.94082152,308.62450916)(63.94082169,308.53451294)
\curveto(63.94082152,308.44450934)(63.93582153,308.37450941)(63.92582169,308.32451294)
\lineto(63.92582169,308.15951294)
\curveto(63.90582156,308.09950968)(63.89582157,308.03450975)(63.89582169,307.96451294)
\curveto(63.90582156,307.89450989)(63.90082156,307.83950994)(63.88082169,307.79951294)
\curveto(63.87082159,307.74951003)(63.8658216,307.6845101)(63.86582169,307.60451294)
\curveto(63.84582162,307.52451026)(63.82582164,307.44951033)(63.80582169,307.37951294)
\curveto(63.79582167,307.30951047)(63.77582169,307.23451055)(63.74582169,307.15451294)
\curveto(63.64582182,306.86451092)(63.52082194,306.61951116)(63.37082169,306.41951294)
\curveto(63.22082224,306.21951156)(63.02582244,306.05951172)(62.78582169,305.93951294)
\curveto(62.65582281,305.8795119)(62.52082294,305.82951195)(62.38082169,305.78951294)
\curveto(62.24082322,305.75951202)(62.08582338,305.73951204)(61.91582169,305.72951294)
\curveto(61.85582361,305.71951206)(61.78582368,305.72451206)(61.70582169,305.74451294)
\curveto(61.61582385,305.76451202)(61.54582392,305.78951199)(61.49582169,305.81951294)
\curveto(61.45582401,305.85951192)(61.41582405,305.91951186)(61.37582169,305.99951294)
\curveto(61.35582411,306.04951173)(61.34582412,306.11951166)(61.34582169,306.20951294)
\curveto(61.33582413,306.30951147)(61.33582413,306.39951138)(61.34582169,306.47951294)
\curveto(61.35582411,306.56951121)(61.37082409,306.65451113)(61.39082169,306.73451294)
\curveto(61.40082406,306.82451096)(61.41582405,306.8795109)(61.43582169,306.89951294)
\curveto(61.48582398,306.95951082)(61.5608239,306.98951079)(61.66082169,306.98951294)
\curveto(61.75082371,306.99951078)(61.83582363,307.01951076)(61.91582169,307.04951294)
\curveto(62.13582333,307.09951068)(62.30582316,307.19951058)(62.42582169,307.34951294)
\curveto(62.51582295,307.44951033)(62.58582288,307.56951021)(62.63582169,307.70951294)
\curveto(62.68582278,307.84950993)(62.73582273,307.99950978)(62.78582169,308.15951294)
\lineto(62.83082169,308.47451294)
\lineto(62.83082169,308.56451294)
\curveto(62.85082261,308.62450916)(62.8608226,308.70950907)(62.86082169,308.81951294)
\curveto(62.8608226,308.93950884)(62.85082261,309.04450874)(62.83082169,309.13451294)
\curveto(62.83082263,309.20450858)(62.82582264,309.25950852)(62.81582169,309.29951294)
\curveto(62.80582266,309.35950842)(62.80082266,309.41950836)(62.80082169,309.47951294)
\curveto(62.79082267,309.53950824)(62.78082268,309.59450819)(62.77082169,309.64451294)
\curveto(62.69082277,309.95450783)(62.58582288,310.20450758)(62.45582169,310.39451294)
\curveto(62.32582314,310.59450719)(62.10582336,310.75950702)(61.79582169,310.88951294)
\curveto(61.74582372,310.91950686)(61.69082377,310.93450685)(61.63082169,310.93451294)
\curveto(61.57082389,310.94450684)(61.52582394,310.94450684)(61.49582169,310.93451294)
\curveto(61.30582416,310.92450686)(61.1658243,310.8845069)(61.07582169,310.81451294)
\curveto(60.97582449,310.74450704)(60.88582458,310.64950713)(60.80582169,310.52951294)
\curveto(60.74582472,310.44950733)(60.69582477,310.35450743)(60.65582169,310.24451294)
\lineto(60.53582169,309.94451294)
\curveto(60.52582494,309.91450787)(60.52082494,309.8845079)(60.52082169,309.85451294)
\curveto(60.52082494,309.83450795)(60.51082495,309.81450797)(60.49082169,309.79451294)
\curveto(60.38082508,309.47450831)(60.30082516,309.13450865)(60.25082169,308.77451294)
\curveto(60.19082527,308.42450936)(60.09582537,308.10450968)(59.96582169,307.81451294)
\curveto(59.92582554,307.72451006)(59.89082557,307.63451015)(59.86082169,307.54451294)
\curveto(59.83082563,307.46451032)(59.79082567,307.38951039)(59.74082169,307.31951294)
\curveto(59.63082583,307.14951063)(59.50582596,306.99951078)(59.36582169,306.86951294)
\curveto(59.22582624,306.73951104)(59.05082641,306.64951113)(58.84082169,306.59951294)
\curveto(58.77082669,306.5795112)(58.70082676,306.56951121)(58.63082169,306.56951294)
\lineto(58.40582169,306.56951294)
\curveto(58.28582718,306.55951122)(58.15082731,306.57451121)(58.00082169,306.61451294)
\curveto(57.84082762,306.65451113)(57.70582776,306.69451109)(57.59582169,306.73451294)
\curveto(57.54582792,306.76451102)(57.50582796,306.784511)(57.47582169,306.79451294)
\curveto(57.43582803,306.81451097)(57.39582807,306.83951094)(57.35582169,306.86951294)
\curveto(57.12582834,306.99951078)(56.92582854,307.15951062)(56.75582169,307.34951294)
\curveto(56.58582888,307.53951024)(56.43582903,307.74951003)(56.30582169,307.97951294)
\curveto(56.21582925,308.13950964)(56.14582932,308.31450947)(56.09582169,308.50451294)
\curveto(56.03582943,308.70450908)(55.98082948,308.90950887)(55.93082169,309.11951294)
\curveto(55.92082954,309.18950859)(55.91082955,309.25450853)(55.90082169,309.31451294)
\curveto(55.89082957,309.3845084)(55.88082958,309.45950832)(55.87082169,309.53951294)
\curveto(55.8608296,309.5795082)(55.8608296,309.61950816)(55.87082169,309.65951294)
\curveto(55.88082958,309.70950807)(55.87582959,309.74950803)(55.85582169,309.77951294)
}
}
{
\newrgbcolor{curcolor}{0 0 0}
\pscustom[linestyle=none,fillstyle=solid,fillcolor=curcolor]
{
\newpath
\moveto(181.45952515,78.65651611)
\curveto(181.52952341,78.65650545)(181.60952333,78.65650545)(181.69952515,78.65651611)
\curveto(181.78952315,78.66650544)(181.87452307,78.66650544)(181.95452515,78.65651611)
\curveto(182.0445229,78.65650545)(182.12452282,78.64650546)(182.19452515,78.62651611)
\curveto(182.26452268,78.6065055)(182.31452263,78.57650553)(182.34452515,78.53651611)
\curveto(182.40452254,78.46650564)(182.43452251,78.36650574)(182.43452515,78.23651611)
\curveto(182.4445225,78.11650599)(182.44952249,77.99150611)(182.44952515,77.86151611)
\lineto(182.44952515,76.40651611)
\lineto(182.44952515,70.61651611)
\lineto(182.44952515,68.86151611)
\lineto(182.44952515,68.44151611)
\curveto(182.44952249,68.3015158)(182.42452252,68.19151591)(182.37452515,68.11151611)
\curveto(182.33452261,68.06151604)(182.28452266,68.03151607)(182.22452515,68.02151611)
\curveto(182.17452277,68.01151609)(182.10952283,67.99651611)(182.02952515,67.97651611)
\lineto(181.74452515,67.97651611)
\curveto(181.60452334,67.97651613)(181.47452347,67.98151612)(181.35452515,67.99151611)
\curveto(181.23452371,68.0015161)(181.14952379,68.05151605)(181.09952515,68.14151611)
\curveto(181.05952388,68.2015159)(181.0395239,68.28151582)(181.03952515,68.38151611)
\lineto(181.03952515,68.71151611)
\lineto(181.03952515,69.91151611)
\lineto(181.03952515,76.18151611)
\lineto(181.03952515,77.80151611)
\curveto(181.0395239,77.91150619)(181.03452391,78.03150607)(181.02452515,78.16151611)
\curveto(181.02452392,78.3015058)(181.04952389,78.41150569)(181.09952515,78.49151611)
\curveto(181.1395238,78.56150554)(181.21952372,78.61150549)(181.33952515,78.64151611)
\curveto(181.35952358,78.65150545)(181.37952356,78.65150545)(181.39952515,78.64151611)
\curveto(181.41952352,78.64150546)(181.4395235,78.64650546)(181.45952515,78.65651611)
}
}
{
\newrgbcolor{curcolor}{0 0 0}
\pscustom[linestyle=none,fillstyle=solid,fillcolor=curcolor]
{
\newpath
\moveto(188.29600952,75.88151611)
\curveto(188.92600429,75.9015082)(189.43100378,75.81650829)(189.81100952,75.62651611)
\curveto(190.19100302,75.43650867)(190.49600272,75.15150895)(190.72600952,74.77151611)
\curveto(190.78600243,74.67150943)(190.83100238,74.56150954)(190.86100952,74.44151611)
\curveto(190.90100231,74.33150977)(190.93600228,74.21650989)(190.96600952,74.09651611)
\curveto(191.0160022,73.9065102)(191.04600217,73.7015104)(191.05600952,73.48151611)
\curveto(191.06600215,73.26151084)(191.07100214,73.03651107)(191.07100952,72.80651611)
\lineto(191.07100952,71.20151611)
\lineto(191.07100952,68.86151611)
\curveto(191.07100214,68.69151541)(191.06600215,68.52151558)(191.05600952,68.35151611)
\curveto(191.05600216,68.18151592)(190.99100222,68.07151603)(190.86100952,68.02151611)
\curveto(190.8110024,68.0015161)(190.75600246,67.99151611)(190.69600952,67.99151611)
\curveto(190.64600257,67.98151612)(190.59100262,67.97651613)(190.53100952,67.97651611)
\curveto(190.40100281,67.97651613)(190.27600294,67.98151612)(190.15600952,67.99151611)
\curveto(190.03600318,67.99151611)(189.95100326,68.03151607)(189.90100952,68.11151611)
\curveto(189.85100336,68.18151592)(189.82600339,68.27151583)(189.82600952,68.38151611)
\lineto(189.82600952,68.71151611)
\lineto(189.82600952,70.00151611)
\lineto(189.82600952,72.44651611)
\curveto(189.82600339,72.71651139)(189.82100339,72.98151112)(189.81100952,73.24151611)
\curveto(189.80100341,73.51151059)(189.75600346,73.74151036)(189.67600952,73.93151611)
\curveto(189.59600362,74.13150997)(189.47600374,74.29150981)(189.31600952,74.41151611)
\curveto(189.15600406,74.54150956)(188.97100424,74.64150946)(188.76100952,74.71151611)
\curveto(188.70100451,74.73150937)(188.63600458,74.74150936)(188.56600952,74.74151611)
\curveto(188.50600471,74.75150935)(188.44600477,74.76650934)(188.38600952,74.78651611)
\curveto(188.33600488,74.79650931)(188.25600496,74.79650931)(188.14600952,74.78651611)
\curveto(188.04600517,74.78650932)(187.97600524,74.78150932)(187.93600952,74.77151611)
\curveto(187.89600532,74.75150935)(187.86100535,74.74150936)(187.83100952,74.74151611)
\curveto(187.80100541,74.75150935)(187.76600545,74.75150935)(187.72600952,74.74151611)
\curveto(187.59600562,74.71150939)(187.47100574,74.67650943)(187.35100952,74.63651611)
\curveto(187.24100597,74.6065095)(187.13600608,74.56150954)(187.03600952,74.50151611)
\curveto(186.99600622,74.48150962)(186.96100625,74.46150964)(186.93100952,74.44151611)
\curveto(186.90100631,74.42150968)(186.86600635,74.4015097)(186.82600952,74.38151611)
\curveto(186.47600674,74.13150997)(186.22100699,73.75651035)(186.06100952,73.25651611)
\curveto(186.03100718,73.17651093)(186.0110072,73.09151101)(186.00100952,73.00151611)
\curveto(185.99100722,72.92151118)(185.97600724,72.84151126)(185.95600952,72.76151611)
\curveto(185.93600728,72.71151139)(185.93100728,72.66151144)(185.94100952,72.61151611)
\curveto(185.95100726,72.57151153)(185.94600727,72.53151157)(185.92600952,72.49151611)
\lineto(185.92600952,72.17651611)
\curveto(185.9160073,72.14651196)(185.9110073,72.11151199)(185.91100952,72.07151611)
\curveto(185.92100729,72.03151207)(185.92600729,71.98651212)(185.92600952,71.93651611)
\lineto(185.92600952,71.48651611)
\lineto(185.92600952,70.04651611)
\lineto(185.92600952,68.72651611)
\lineto(185.92600952,68.38151611)
\curveto(185.92600729,68.27151583)(185.90100731,68.18151592)(185.85100952,68.11151611)
\curveto(185.80100741,68.03151607)(185.7110075,67.99151611)(185.58100952,67.99151611)
\curveto(185.46100775,67.98151612)(185.33600788,67.97651613)(185.20600952,67.97651611)
\curveto(185.12600809,67.97651613)(185.05100816,67.98151612)(184.98100952,67.99151611)
\curveto(184.9110083,68.0015161)(184.85100836,68.02651608)(184.80100952,68.06651611)
\curveto(184.72100849,68.11651599)(184.68100853,68.21151589)(184.68100952,68.35151611)
\lineto(184.68100952,68.75651611)
\lineto(184.68100952,70.52651611)
\lineto(184.68100952,74.15651611)
\lineto(184.68100952,75.07151611)
\lineto(184.68100952,75.34151611)
\curveto(184.68100853,75.43150867)(184.70100851,75.5015086)(184.74100952,75.55151611)
\curveto(184.77100844,75.61150849)(184.82100839,75.65150845)(184.89100952,75.67151611)
\curveto(184.93100828,75.68150842)(184.98600823,75.69150841)(185.05600952,75.70151611)
\curveto(185.13600808,75.71150839)(185.216008,75.71650839)(185.29600952,75.71651611)
\curveto(185.37600784,75.71650839)(185.45100776,75.71150839)(185.52100952,75.70151611)
\curveto(185.60100761,75.69150841)(185.65600756,75.67650843)(185.68600952,75.65651611)
\curveto(185.79600742,75.58650852)(185.84600737,75.49650861)(185.83600952,75.38651611)
\curveto(185.82600739,75.28650882)(185.84100737,75.17150893)(185.88100952,75.04151611)
\curveto(185.90100731,74.98150912)(185.94100727,74.93150917)(186.00100952,74.89151611)
\curveto(186.12100709,74.88150922)(186.216007,74.92650918)(186.28600952,75.02651611)
\curveto(186.36600685,75.12650898)(186.44600677,75.2065089)(186.52600952,75.26651611)
\curveto(186.66600655,75.36650874)(186.80600641,75.45650865)(186.94600952,75.53651611)
\curveto(187.09600612,75.62650848)(187.26600595,75.7015084)(187.45600952,75.76151611)
\curveto(187.53600568,75.79150831)(187.62100559,75.81150829)(187.71100952,75.82151611)
\curveto(187.8110054,75.83150827)(187.90600531,75.84650826)(187.99600952,75.86651611)
\curveto(188.04600517,75.87650823)(188.09600512,75.88150822)(188.14600952,75.88151611)
\lineto(188.29600952,75.88151611)
}
}
{
\newrgbcolor{curcolor}{0 0 0}
\pscustom[linestyle=none,fillstyle=solid,fillcolor=curcolor]
{
\newpath
\moveto(192.7306189,75.73151611)
\lineto(193.2106189,75.73151611)
\curveto(193.38061756,75.73150837)(193.51061743,75.7015084)(193.6006189,75.64151611)
\curveto(193.67061727,75.59150851)(193.71561722,75.52650858)(193.7356189,75.44651611)
\curveto(193.76561717,75.37650873)(193.79561714,75.3015088)(193.8256189,75.22151611)
\curveto(193.88561705,75.08150902)(193.935617,74.94150916)(193.9756189,74.80151611)
\curveto(194.01561692,74.66150944)(194.06061688,74.52150958)(194.1106189,74.38151611)
\curveto(194.31061663,73.84151026)(194.49561644,73.29651081)(194.6656189,72.74651611)
\curveto(194.8356161,72.2065119)(195.02061592,71.66651244)(195.2206189,71.12651611)
\curveto(195.29061565,70.94651316)(195.35061559,70.76151334)(195.4006189,70.57151611)
\curveto(195.45061549,70.39151371)(195.51561542,70.21151389)(195.5956189,70.03151611)
\curveto(195.61561532,69.96151414)(195.6406153,69.88651422)(195.6706189,69.80651611)
\curveto(195.70061524,69.72651438)(195.75061519,69.67651443)(195.8206189,69.65651611)
\curveto(195.90061504,69.63651447)(195.96061498,69.67151443)(196.0006189,69.76151611)
\curveto(196.05061489,69.85151425)(196.08561485,69.92151418)(196.1056189,69.97151611)
\curveto(196.18561475,70.16151394)(196.25061469,70.35151375)(196.3006189,70.54151611)
\curveto(196.36061458,70.74151336)(196.42561451,70.94151316)(196.4956189,71.14151611)
\curveto(196.62561431,71.52151258)(196.75061419,71.89651221)(196.8706189,72.26651611)
\curveto(196.99061395,72.64651146)(197.11561382,73.02651108)(197.2456189,73.40651611)
\curveto(197.29561364,73.57651053)(197.34561359,73.74151036)(197.3956189,73.90151611)
\curveto(197.44561349,74.07151003)(197.50561343,74.23650987)(197.5756189,74.39651611)
\curveto(197.62561331,74.53650957)(197.67061327,74.67650943)(197.7106189,74.81651611)
\curveto(197.75061319,74.95650915)(197.79561314,75.09650901)(197.8456189,75.23651611)
\curveto(197.86561307,75.3065088)(197.89061305,75.37650873)(197.9206189,75.44651611)
\curveto(197.95061299,75.51650859)(197.99061295,75.57650853)(198.0406189,75.62651611)
\curveto(198.12061282,75.67650843)(198.21061273,75.7065084)(198.3106189,75.71651611)
\curveto(198.41061253,75.72650838)(198.53061241,75.73150837)(198.6706189,75.73151611)
\curveto(198.7406122,75.73150837)(198.80561213,75.72650838)(198.8656189,75.71651611)
\curveto(198.92561201,75.71650839)(198.98061196,75.7065084)(199.0306189,75.68651611)
\curveto(199.12061182,75.64650846)(199.16561177,75.58150852)(199.1656189,75.49151611)
\curveto(199.17561176,75.4015087)(199.16061178,75.31150879)(199.1206189,75.22151611)
\curveto(199.06061188,75.05150905)(199.00061194,74.87650923)(198.9406189,74.69651611)
\curveto(198.88061206,74.51650959)(198.81061213,74.34150976)(198.7306189,74.17151611)
\curveto(198.71061223,74.12150998)(198.69561224,74.07151003)(198.6856189,74.02151611)
\curveto(198.67561226,73.98151012)(198.66061228,73.93651017)(198.6406189,73.88651611)
\curveto(198.56061238,73.71651039)(198.49561244,73.54151056)(198.4456189,73.36151611)
\curveto(198.39561254,73.18151092)(198.33061261,73.0015111)(198.2506189,72.82151611)
\curveto(198.20061274,72.69151141)(198.15061279,72.55651155)(198.1006189,72.41651611)
\curveto(198.06061288,72.28651182)(198.01061293,72.15651195)(197.9506189,72.02651611)
\curveto(197.78061316,71.61651249)(197.62561331,71.2015129)(197.4856189,70.78151611)
\curveto(197.35561358,70.36151374)(197.20561373,69.94651416)(197.0356189,69.53651611)
\curveto(196.97561396,69.37651473)(196.92061402,69.21651489)(196.8706189,69.05651611)
\curveto(196.82061412,68.89651521)(196.76061418,68.73651537)(196.6906189,68.57651611)
\curveto(196.6406143,68.46651564)(196.59561434,68.36151574)(196.5556189,68.26151611)
\curveto(196.52561441,68.17151593)(196.45561448,68.101516)(196.3456189,68.05151611)
\curveto(196.28561465,68.02151608)(196.21561472,68.0065161)(196.1356189,68.00651611)
\lineto(195.9106189,68.00651611)
\lineto(195.4456189,68.00651611)
\curveto(195.29561564,68.01651609)(195.18561575,68.06651604)(195.1156189,68.15651611)
\curveto(195.04561589,68.23651587)(194.99561594,68.33151577)(194.9656189,68.44151611)
\curveto(194.935616,68.56151554)(194.89561604,68.67651543)(194.8456189,68.78651611)
\curveto(194.78561615,68.92651518)(194.72561621,69.07151503)(194.6656189,69.22151611)
\curveto(194.61561632,69.38151472)(194.56561637,69.53151457)(194.5156189,69.67151611)
\curveto(194.49561644,69.72151438)(194.48061646,69.76151434)(194.4706189,69.79151611)
\curveto(194.46061648,69.83151427)(194.44561649,69.87651423)(194.4256189,69.92651611)
\curveto(194.22561671,70.4065137)(194.0406169,70.89151321)(193.8706189,71.38151611)
\curveto(193.71061723,71.87151223)(193.53061741,72.35651175)(193.3306189,72.83651611)
\curveto(193.27061767,72.99651111)(193.21061773,73.15151095)(193.1506189,73.30151611)
\curveto(193.10061784,73.46151064)(193.04561789,73.62151048)(192.9856189,73.78151611)
\lineto(192.9256189,73.93151611)
\curveto(192.91561802,73.99151011)(192.90061804,74.04651006)(192.8806189,74.09651611)
\curveto(192.80061814,74.26650984)(192.73061821,74.43650967)(192.6706189,74.60651611)
\curveto(192.62061832,74.77650933)(192.56061838,74.94650916)(192.4906189,75.11651611)
\curveto(192.47061847,75.17650893)(192.44561849,75.25650885)(192.4156189,75.35651611)
\curveto(192.38561855,75.45650865)(192.39061855,75.54150856)(192.4306189,75.61151611)
\curveto(192.48061846,75.66150844)(192.5406184,75.69650841)(192.6106189,75.71651611)
\curveto(192.68061826,75.71650839)(192.72061822,75.72150838)(192.7306189,75.73151611)
}
}
{
\newrgbcolor{curcolor}{0 0 0}
\pscustom[linestyle=none,fillstyle=solid,fillcolor=curcolor]
{
\newpath
\moveto(200.7406189,77.23151611)
\curveto(200.66061778,77.29150681)(200.61561782,77.39650671)(200.6056189,77.54651611)
\lineto(200.6056189,78.01151611)
\lineto(200.6056189,78.26651611)
\curveto(200.60561783,78.35650575)(200.62061782,78.43150567)(200.6506189,78.49151611)
\curveto(200.69061775,78.57150553)(200.77061767,78.63150547)(200.8906189,78.67151611)
\curveto(200.91061753,78.68150542)(200.93061751,78.68150542)(200.9506189,78.67151611)
\curveto(200.98061746,78.67150543)(201.00561743,78.67650543)(201.0256189,78.68651611)
\curveto(201.19561724,78.68650542)(201.35561708,78.68150542)(201.5056189,78.67151611)
\curveto(201.65561678,78.66150544)(201.75561668,78.6015055)(201.8056189,78.49151611)
\curveto(201.8356166,78.43150567)(201.85061659,78.35650575)(201.8506189,78.26651611)
\lineto(201.8506189,78.01151611)
\curveto(201.85061659,77.83150627)(201.84561659,77.66150644)(201.8356189,77.50151611)
\curveto(201.8356166,77.34150676)(201.77061667,77.23650687)(201.6406189,77.18651611)
\curveto(201.59061685,77.16650694)(201.5356169,77.15650695)(201.4756189,77.15651611)
\lineto(201.3106189,77.15651611)
\lineto(200.9956189,77.15651611)
\curveto(200.89561754,77.15650695)(200.81061763,77.18150692)(200.7406189,77.23151611)
\moveto(201.8506189,68.72651611)
\lineto(201.8506189,68.41151611)
\curveto(201.86061658,68.31151579)(201.8406166,68.23151587)(201.7906189,68.17151611)
\curveto(201.76061668,68.11151599)(201.71561672,68.07151603)(201.6556189,68.05151611)
\curveto(201.59561684,68.04151606)(201.52561691,68.02651608)(201.4456189,68.00651611)
\lineto(201.2206189,68.00651611)
\curveto(201.09061735,68.0065161)(200.97561746,68.01151609)(200.8756189,68.02151611)
\curveto(200.78561765,68.04151606)(200.71561772,68.09151601)(200.6656189,68.17151611)
\curveto(200.62561781,68.23151587)(200.60561783,68.3065158)(200.6056189,68.39651611)
\lineto(200.6056189,68.68151611)
\lineto(200.6056189,75.02651611)
\lineto(200.6056189,75.34151611)
\curveto(200.60561783,75.45150865)(200.63061781,75.53650857)(200.6806189,75.59651611)
\curveto(200.71061773,75.64650846)(200.75061769,75.67650843)(200.8006189,75.68651611)
\curveto(200.85061759,75.69650841)(200.90561753,75.71150839)(200.9656189,75.73151611)
\curveto(200.98561745,75.73150837)(201.00561743,75.72650838)(201.0256189,75.71651611)
\curveto(201.05561738,75.71650839)(201.08061736,75.72150838)(201.1006189,75.73151611)
\curveto(201.23061721,75.73150837)(201.36061708,75.72650838)(201.4906189,75.71651611)
\curveto(201.63061681,75.71650839)(201.72561671,75.67650843)(201.7756189,75.59651611)
\curveto(201.82561661,75.53650857)(201.85061659,75.45650865)(201.8506189,75.35651611)
\lineto(201.8506189,75.07151611)
\lineto(201.8506189,68.72651611)
}
}
{
\newrgbcolor{curcolor}{0 0 0}
\pscustom[linestyle=none,fillstyle=solid,fillcolor=curcolor]
{
\newpath
\moveto(204.74046265,78.07151611)
\curveto(204.89046064,78.07150603)(205.04046049,78.06650604)(205.19046265,78.05651611)
\curveto(205.34046019,78.05650605)(205.44546008,78.01650609)(205.50546265,77.93651611)
\curveto(205.55545997,77.87650623)(205.58045995,77.79150631)(205.58046265,77.68151611)
\curveto(205.59045994,77.58150652)(205.59545993,77.47650663)(205.59546265,77.36651611)
\lineto(205.59546265,76.49651611)
\curveto(205.59545993,76.41650769)(205.59045994,76.33150777)(205.58046265,76.24151611)
\curveto(205.58045995,76.16150794)(205.59045994,76.09150801)(205.61046265,76.03151611)
\curveto(205.65045988,75.89150821)(205.74045979,75.8015083)(205.88046265,75.76151611)
\curveto(205.9304596,75.75150835)(205.97545955,75.74650836)(206.01546265,75.74651611)
\lineto(206.16546265,75.74651611)
\lineto(206.57046265,75.74651611)
\curveto(206.7304588,75.75650835)(206.84545868,75.74650836)(206.91546265,75.71651611)
\curveto(207.00545852,75.65650845)(207.06545846,75.59650851)(207.09546265,75.53651611)
\curveto(207.11545841,75.49650861)(207.1254584,75.45150865)(207.12546265,75.40151611)
\lineto(207.12546265,75.25151611)
\curveto(207.1254584,75.14150896)(207.12045841,75.03650907)(207.11046265,74.93651611)
\curveto(207.10045843,74.84650926)(207.06545846,74.77650933)(207.00546265,74.72651611)
\curveto(206.94545858,74.67650943)(206.86045867,74.64650946)(206.75046265,74.63651611)
\lineto(206.42046265,74.63651611)
\curveto(206.31045922,74.64650946)(206.20045933,74.65150945)(206.09046265,74.65151611)
\curveto(205.98045955,74.65150945)(205.88545964,74.63650947)(205.80546265,74.60651611)
\curveto(205.73545979,74.57650953)(205.68545984,74.52650958)(205.65546265,74.45651611)
\curveto(205.6254599,74.38650972)(205.60545992,74.3015098)(205.59546265,74.20151611)
\curveto(205.58545994,74.11150999)(205.58045995,74.01151009)(205.58046265,73.90151611)
\curveto(205.59045994,73.8015103)(205.59545993,73.7015104)(205.59546265,73.60151611)
\lineto(205.59546265,70.63151611)
\curveto(205.59545993,70.41151369)(205.59045994,70.17651393)(205.58046265,69.92651611)
\curveto(205.58045995,69.68651442)(205.6254599,69.5015146)(205.71546265,69.37151611)
\curveto(205.76545976,69.29151481)(205.8304597,69.23651487)(205.91046265,69.20651611)
\curveto(205.99045954,69.17651493)(206.08545944,69.15151495)(206.19546265,69.13151611)
\curveto(206.2254593,69.12151498)(206.25545927,69.11651499)(206.28546265,69.11651611)
\curveto(206.3254592,69.12651498)(206.36045917,69.12651498)(206.39046265,69.11651611)
\lineto(206.58546265,69.11651611)
\curveto(206.68545884,69.11651499)(206.77545875,69.106515)(206.85546265,69.08651611)
\curveto(206.94545858,69.07651503)(207.01045852,69.04151506)(207.05046265,68.98151611)
\curveto(207.07045846,68.95151515)(207.08545844,68.89651521)(207.09546265,68.81651611)
\curveto(207.11545841,68.74651536)(207.1254584,68.67151543)(207.12546265,68.59151611)
\curveto(207.13545839,68.51151559)(207.13545839,68.43151567)(207.12546265,68.35151611)
\curveto(207.11545841,68.28151582)(207.09545843,68.22651588)(207.06546265,68.18651611)
\curveto(207.0254585,68.11651599)(206.95045858,68.06651604)(206.84046265,68.03651611)
\curveto(206.76045877,68.01651609)(206.67045886,68.0065161)(206.57046265,68.00651611)
\curveto(206.47045906,68.01651609)(206.38045915,68.02151608)(206.30046265,68.02151611)
\curveto(206.24045929,68.02151608)(206.18045935,68.01651609)(206.12046265,68.00651611)
\curveto(206.06045947,68.0065161)(206.00545952,68.01151609)(205.95546265,68.02151611)
\lineto(205.77546265,68.02151611)
\curveto(205.7254598,68.03151607)(205.67545985,68.03651607)(205.62546265,68.03651611)
\curveto(205.58545994,68.04651606)(205.54045999,68.05151605)(205.49046265,68.05151611)
\curveto(205.29046024,68.101516)(205.11546041,68.15651595)(204.96546265,68.21651611)
\curveto(204.8254607,68.27651583)(204.70546082,68.38151572)(204.60546265,68.53151611)
\curveto(204.46546106,68.73151537)(204.38546114,68.98151512)(204.36546265,69.28151611)
\curveto(204.34546118,69.59151451)(204.33546119,69.92151418)(204.33546265,70.27151611)
\lineto(204.33546265,74.20151611)
\curveto(204.30546122,74.33150977)(204.27546125,74.42650968)(204.24546265,74.48651611)
\curveto(204.2254613,74.54650956)(204.15546137,74.59650951)(204.03546265,74.63651611)
\curveto(203.99546153,74.64650946)(203.95546157,74.64650946)(203.91546265,74.63651611)
\curveto(203.87546165,74.62650948)(203.83546169,74.63150947)(203.79546265,74.65151611)
\lineto(203.55546265,74.65151611)
\curveto(203.4254621,74.65150945)(203.31546221,74.66150944)(203.22546265,74.68151611)
\curveto(203.14546238,74.71150939)(203.09046244,74.77150933)(203.06046265,74.86151611)
\curveto(203.04046249,74.9015092)(203.0254625,74.94650916)(203.01546265,74.99651611)
\lineto(203.01546265,75.14651611)
\curveto(203.01546251,75.28650882)(203.0254625,75.4015087)(203.04546265,75.49151611)
\curveto(203.06546246,75.59150851)(203.1254624,75.66650844)(203.22546265,75.71651611)
\curveto(203.33546219,75.75650835)(203.47546205,75.76650834)(203.64546265,75.74651611)
\curveto(203.8254617,75.72650838)(203.97546155,75.73650837)(204.09546265,75.77651611)
\curveto(204.18546134,75.82650828)(204.25546127,75.89650821)(204.30546265,75.98651611)
\curveto(204.3254612,76.04650806)(204.33546119,76.12150798)(204.33546265,76.21151611)
\lineto(204.33546265,76.46651611)
\lineto(204.33546265,77.39651611)
\lineto(204.33546265,77.63651611)
\curveto(204.33546119,77.72650638)(204.34546118,77.8015063)(204.36546265,77.86151611)
\curveto(204.40546112,77.94150616)(204.48046105,78.0065061)(204.59046265,78.05651611)
\curveto(204.62046091,78.05650605)(204.64546088,78.05650605)(204.66546265,78.05651611)
\curveto(204.69546083,78.06650604)(204.72046081,78.07150603)(204.74046265,78.07151611)
}
}
{
\newrgbcolor{curcolor}{0 0 0}
\pscustom[linestyle=none,fillstyle=solid,fillcolor=curcolor]
{
\newpath
\moveto(215.39725952,68.56151611)
\curveto(215.42725169,68.4015157)(215.41225171,68.26651584)(215.35225952,68.15651611)
\curveto(215.29225183,68.05651605)(215.21225191,67.98151612)(215.11225952,67.93151611)
\curveto(215.06225206,67.91151619)(215.00725211,67.9015162)(214.94725952,67.90151611)
\curveto(214.89725222,67.9015162)(214.84225228,67.89151621)(214.78225952,67.87151611)
\curveto(214.56225256,67.82151628)(214.34225278,67.83651627)(214.12225952,67.91651611)
\curveto(213.91225321,67.98651612)(213.76725335,68.07651603)(213.68725952,68.18651611)
\curveto(213.63725348,68.25651585)(213.59225353,68.33651577)(213.55225952,68.42651611)
\curveto(213.51225361,68.52651558)(213.46225366,68.6065155)(213.40225952,68.66651611)
\curveto(213.38225374,68.68651542)(213.35725376,68.7065154)(213.32725952,68.72651611)
\curveto(213.30725381,68.74651536)(213.27725384,68.75151535)(213.23725952,68.74151611)
\curveto(213.12725399,68.71151539)(213.0222541,68.65651545)(212.92225952,68.57651611)
\curveto(212.83225429,68.49651561)(212.74225438,68.42651568)(212.65225952,68.36651611)
\curveto(212.5222546,68.28651582)(212.38225474,68.21151589)(212.23225952,68.14151611)
\curveto(212.08225504,68.08151602)(211.9222552,68.02651608)(211.75225952,67.97651611)
\curveto(211.65225547,67.94651616)(211.54225558,67.92651618)(211.42225952,67.91651611)
\curveto(211.31225581,67.9065162)(211.20225592,67.89151621)(211.09225952,67.87151611)
\curveto(211.04225608,67.86151624)(210.99725612,67.85651625)(210.95725952,67.85651611)
\lineto(210.85225952,67.85651611)
\curveto(210.74225638,67.83651627)(210.63725648,67.83651627)(210.53725952,67.85651611)
\lineto(210.40225952,67.85651611)
\curveto(210.35225677,67.86651624)(210.30225682,67.87151623)(210.25225952,67.87151611)
\curveto(210.20225692,67.87151623)(210.15725696,67.88151622)(210.11725952,67.90151611)
\curveto(210.07725704,67.91151619)(210.04225708,67.91651619)(210.01225952,67.91651611)
\curveto(209.99225713,67.9065162)(209.96725715,67.9065162)(209.93725952,67.91651611)
\lineto(209.69725952,67.97651611)
\curveto(209.6172575,67.98651612)(209.54225758,68.0065161)(209.47225952,68.03651611)
\curveto(209.17225795,68.16651594)(208.92725819,68.31151579)(208.73725952,68.47151611)
\curveto(208.55725856,68.64151546)(208.40725871,68.87651523)(208.28725952,69.17651611)
\curveto(208.19725892,69.39651471)(208.15225897,69.66151444)(208.15225952,69.97151611)
\lineto(208.15225952,70.28651611)
\curveto(208.16225896,70.33651377)(208.16725895,70.38651372)(208.16725952,70.43651611)
\lineto(208.19725952,70.61651611)
\lineto(208.31725952,70.94651611)
\curveto(208.35725876,71.05651305)(208.40725871,71.15651295)(208.46725952,71.24651611)
\curveto(208.64725847,71.53651257)(208.89225823,71.75151235)(209.20225952,71.89151611)
\curveto(209.51225761,72.03151207)(209.85225727,72.15651195)(210.22225952,72.26651611)
\curveto(210.36225676,72.3065118)(210.50725661,72.33651177)(210.65725952,72.35651611)
\curveto(210.80725631,72.37651173)(210.95725616,72.4015117)(211.10725952,72.43151611)
\curveto(211.17725594,72.45151165)(211.24225588,72.46151164)(211.30225952,72.46151611)
\curveto(211.37225575,72.46151164)(211.44725567,72.47151163)(211.52725952,72.49151611)
\curveto(211.59725552,72.51151159)(211.66725545,72.52151158)(211.73725952,72.52151611)
\curveto(211.80725531,72.53151157)(211.88225524,72.54651156)(211.96225952,72.56651611)
\curveto(212.21225491,72.62651148)(212.44725467,72.67651143)(212.66725952,72.71651611)
\curveto(212.88725423,72.76651134)(213.06225406,72.88151122)(213.19225952,73.06151611)
\curveto(213.25225387,73.14151096)(213.30225382,73.24151086)(213.34225952,73.36151611)
\curveto(213.38225374,73.49151061)(213.38225374,73.63151047)(213.34225952,73.78151611)
\curveto(213.28225384,74.02151008)(213.19225393,74.21150989)(213.07225952,74.35151611)
\curveto(212.96225416,74.49150961)(212.80225432,74.6015095)(212.59225952,74.68151611)
\curveto(212.47225465,74.73150937)(212.32725479,74.76650934)(212.15725952,74.78651611)
\curveto(211.99725512,74.8065093)(211.82725529,74.81650929)(211.64725952,74.81651611)
\curveto(211.46725565,74.81650929)(211.29225583,74.8065093)(211.12225952,74.78651611)
\curveto(210.95225617,74.76650934)(210.80725631,74.73650937)(210.68725952,74.69651611)
\curveto(210.5172566,74.63650947)(210.35225677,74.55150955)(210.19225952,74.44151611)
\curveto(210.11225701,74.38150972)(210.03725708,74.3015098)(209.96725952,74.20151611)
\curveto(209.90725721,74.11150999)(209.85225727,74.01151009)(209.80225952,73.90151611)
\curveto(209.77225735,73.82151028)(209.74225738,73.73651037)(209.71225952,73.64651611)
\curveto(209.69225743,73.55651055)(209.64725747,73.48651062)(209.57725952,73.43651611)
\curveto(209.53725758,73.4065107)(209.46725765,73.38151072)(209.36725952,73.36151611)
\curveto(209.27725784,73.35151075)(209.18225794,73.34651076)(209.08225952,73.34651611)
\curveto(208.98225814,73.34651076)(208.88225824,73.35151075)(208.78225952,73.36151611)
\curveto(208.69225843,73.38151072)(208.62725849,73.4065107)(208.58725952,73.43651611)
\curveto(208.54725857,73.46651064)(208.5172586,73.51651059)(208.49725952,73.58651611)
\curveto(208.47725864,73.65651045)(208.47725864,73.73151037)(208.49725952,73.81151611)
\curveto(208.52725859,73.94151016)(208.55725856,74.06151004)(208.58725952,74.17151611)
\curveto(208.62725849,74.29150981)(208.67225845,74.4065097)(208.72225952,74.51651611)
\curveto(208.91225821,74.86650924)(209.15225797,75.13650897)(209.44225952,75.32651611)
\curveto(209.73225739,75.52650858)(210.09225703,75.68650842)(210.52225952,75.80651611)
\curveto(210.6222565,75.82650828)(210.7222564,75.84150826)(210.82225952,75.85151611)
\curveto(210.93225619,75.86150824)(211.04225608,75.87650823)(211.15225952,75.89651611)
\curveto(211.19225593,75.9065082)(211.25725586,75.9065082)(211.34725952,75.89651611)
\curveto(211.43725568,75.89650821)(211.49225563,75.9065082)(211.51225952,75.92651611)
\curveto(212.21225491,75.93650817)(212.8222543,75.85650825)(213.34225952,75.68651611)
\curveto(213.86225326,75.51650859)(214.22725289,75.19150891)(214.43725952,74.71151611)
\curveto(214.52725259,74.51150959)(214.57725254,74.27650983)(214.58725952,74.00651611)
\curveto(214.60725251,73.74651036)(214.6172525,73.47151063)(214.61725952,73.18151611)
\lineto(214.61725952,69.86651611)
\curveto(214.6172525,69.72651438)(214.6222525,69.59151451)(214.63225952,69.46151611)
\curveto(214.64225248,69.33151477)(214.67225245,69.22651488)(214.72225952,69.14651611)
\curveto(214.77225235,69.07651503)(214.83725228,69.02651508)(214.91725952,68.99651611)
\curveto(215.00725211,68.95651515)(215.09225203,68.92651518)(215.17225952,68.90651611)
\curveto(215.25225187,68.89651521)(215.31225181,68.85151525)(215.35225952,68.77151611)
\curveto(215.37225175,68.74151536)(215.38225174,68.71151539)(215.38225952,68.68151611)
\curveto(215.38225174,68.65151545)(215.38725173,68.61151549)(215.39725952,68.56151611)
\moveto(213.25225952,70.22651611)
\curveto(213.31225381,70.36651374)(213.34225378,70.52651358)(213.34225952,70.70651611)
\curveto(213.35225377,70.89651321)(213.35725376,71.09151301)(213.35725952,71.29151611)
\curveto(213.35725376,71.4015127)(213.35225377,71.5015126)(213.34225952,71.59151611)
\curveto(213.33225379,71.68151242)(213.29225383,71.75151235)(213.22225952,71.80151611)
\curveto(213.19225393,71.82151228)(213.122254,71.83151227)(213.01225952,71.83151611)
\curveto(212.99225413,71.81151229)(212.95725416,71.8015123)(212.90725952,71.80151611)
\curveto(212.85725426,71.8015123)(212.81225431,71.79151231)(212.77225952,71.77151611)
\curveto(212.69225443,71.75151235)(212.60225452,71.73151237)(212.50225952,71.71151611)
\lineto(212.20225952,71.65151611)
\curveto(212.17225495,71.65151245)(212.13725498,71.64651246)(212.09725952,71.63651611)
\lineto(211.99225952,71.63651611)
\curveto(211.84225528,71.59651251)(211.67725544,71.57151253)(211.49725952,71.56151611)
\curveto(211.32725579,71.56151254)(211.16725595,71.54151256)(211.01725952,71.50151611)
\curveto(210.93725618,71.48151262)(210.86225626,71.46151264)(210.79225952,71.44151611)
\curveto(210.73225639,71.43151267)(210.66225646,71.41651269)(210.58225952,71.39651611)
\curveto(210.4222567,71.34651276)(210.27225685,71.28151282)(210.13225952,71.20151611)
\curveto(209.99225713,71.13151297)(209.87225725,71.04151306)(209.77225952,70.93151611)
\curveto(209.67225745,70.82151328)(209.59725752,70.68651342)(209.54725952,70.52651611)
\curveto(209.49725762,70.37651373)(209.47725764,70.19151391)(209.48725952,69.97151611)
\curveto(209.48725763,69.87151423)(209.50225762,69.77651433)(209.53225952,69.68651611)
\curveto(209.57225755,69.6065145)(209.6172575,69.53151457)(209.66725952,69.46151611)
\curveto(209.74725737,69.35151475)(209.85225727,69.25651485)(209.98225952,69.17651611)
\curveto(210.11225701,69.106515)(210.25225687,69.04651506)(210.40225952,68.99651611)
\curveto(210.45225667,68.98651512)(210.50225662,68.98151512)(210.55225952,68.98151611)
\curveto(210.60225652,68.98151512)(210.65225647,68.97651513)(210.70225952,68.96651611)
\curveto(210.77225635,68.94651516)(210.85725626,68.93151517)(210.95725952,68.92151611)
\curveto(211.06725605,68.92151518)(211.15725596,68.93151517)(211.22725952,68.95151611)
\curveto(211.28725583,68.97151513)(211.34725577,68.97651513)(211.40725952,68.96651611)
\curveto(211.46725565,68.96651514)(211.52725559,68.97651513)(211.58725952,68.99651611)
\curveto(211.66725545,69.01651509)(211.74225538,69.03151507)(211.81225952,69.04151611)
\curveto(211.89225523,69.05151505)(211.96725515,69.07151503)(212.03725952,69.10151611)
\curveto(212.32725479,69.22151488)(212.57225455,69.36651474)(212.77225952,69.53651611)
\curveto(212.98225414,69.7065144)(213.14225398,69.93651417)(213.25225952,70.22651611)
}
}
{
\newrgbcolor{curcolor}{0 0 0}
\pscustom[linestyle=none,fillstyle=solid,fillcolor=curcolor]
{
\newpath
\moveto(223.52890015,68.81651611)
\lineto(223.52890015,68.42651611)
\curveto(223.52889227,68.3065158)(223.5038923,68.2065159)(223.45390015,68.12651611)
\curveto(223.4038924,68.05651605)(223.31889248,68.01651609)(223.19890015,68.00651611)
\lineto(222.85390015,68.00651611)
\curveto(222.79389301,68.0065161)(222.73389307,68.0015161)(222.67390015,67.99151611)
\curveto(222.62389318,67.99151611)(222.57889322,68.0015161)(222.53890015,68.02151611)
\curveto(222.44889335,68.04151606)(222.38889341,68.08151602)(222.35890015,68.14151611)
\curveto(222.31889348,68.19151591)(222.29389351,68.25151585)(222.28390015,68.32151611)
\curveto(222.28389352,68.39151571)(222.26889353,68.46151564)(222.23890015,68.53151611)
\curveto(222.22889357,68.55151555)(222.21389359,68.56651554)(222.19390015,68.57651611)
\curveto(222.18389362,68.59651551)(222.16889363,68.61651549)(222.14890015,68.63651611)
\curveto(222.04889375,68.64651546)(221.96889383,68.62651548)(221.90890015,68.57651611)
\curveto(221.85889394,68.52651558)(221.803894,68.47651563)(221.74390015,68.42651611)
\curveto(221.54389426,68.27651583)(221.34389446,68.16151594)(221.14390015,68.08151611)
\curveto(220.96389484,68.0015161)(220.75389505,67.94151616)(220.51390015,67.90151611)
\curveto(220.28389552,67.86151624)(220.04389576,67.84151626)(219.79390015,67.84151611)
\curveto(219.55389625,67.83151627)(219.31389649,67.84651626)(219.07390015,67.88651611)
\curveto(218.83389697,67.91651619)(218.62389718,67.97151613)(218.44390015,68.05151611)
\curveto(217.92389788,68.27151583)(217.5038983,68.56651554)(217.18390015,68.93651611)
\curveto(216.86389894,69.31651479)(216.61389919,69.78651432)(216.43390015,70.34651611)
\curveto(216.39389941,70.43651367)(216.36389944,70.52651358)(216.34390015,70.61651611)
\curveto(216.33389947,70.71651339)(216.31389949,70.81651329)(216.28390015,70.91651611)
\curveto(216.27389953,70.96651314)(216.26889953,71.01651309)(216.26890015,71.06651611)
\curveto(216.26889953,71.11651299)(216.26389954,71.16651294)(216.25390015,71.21651611)
\curveto(216.23389957,71.26651284)(216.22389958,71.31651279)(216.22390015,71.36651611)
\curveto(216.23389957,71.42651268)(216.23389957,71.48151262)(216.22390015,71.53151611)
\lineto(216.22390015,71.68151611)
\curveto(216.2038996,71.73151237)(216.19389961,71.79651231)(216.19390015,71.87651611)
\curveto(216.19389961,71.95651215)(216.2038996,72.02151208)(216.22390015,72.07151611)
\lineto(216.22390015,72.23651611)
\curveto(216.24389956,72.3065118)(216.24889955,72.37651173)(216.23890015,72.44651611)
\curveto(216.23889956,72.52651158)(216.24889955,72.6015115)(216.26890015,72.67151611)
\curveto(216.27889952,72.72151138)(216.28389952,72.76651134)(216.28390015,72.80651611)
\curveto(216.28389952,72.84651126)(216.28889951,72.89151121)(216.29890015,72.94151611)
\curveto(216.32889947,73.04151106)(216.35389945,73.13651097)(216.37390015,73.22651611)
\curveto(216.39389941,73.32651078)(216.41889938,73.42151068)(216.44890015,73.51151611)
\curveto(216.57889922,73.89151021)(216.74389906,74.23150987)(216.94390015,74.53151611)
\curveto(217.15389865,74.84150926)(217.4038984,75.09650901)(217.69390015,75.29651611)
\curveto(217.86389794,75.41650869)(218.03889776,75.51650859)(218.21890015,75.59651611)
\curveto(218.40889739,75.67650843)(218.61389719,75.74650836)(218.83390015,75.80651611)
\curveto(218.9038969,75.81650829)(218.96889683,75.82650828)(219.02890015,75.83651611)
\curveto(219.0988967,75.84650826)(219.16889663,75.86150824)(219.23890015,75.88151611)
\lineto(219.38890015,75.88151611)
\curveto(219.46889633,75.9015082)(219.58389622,75.91150819)(219.73390015,75.91151611)
\curveto(219.89389591,75.91150819)(220.01389579,75.9015082)(220.09390015,75.88151611)
\curveto(220.13389567,75.87150823)(220.18889561,75.86650824)(220.25890015,75.86651611)
\curveto(220.36889543,75.83650827)(220.47889532,75.81150829)(220.58890015,75.79151611)
\curveto(220.6988951,75.78150832)(220.803895,75.75150835)(220.90390015,75.70151611)
\curveto(221.05389475,75.64150846)(221.19389461,75.57650853)(221.32390015,75.50651611)
\curveto(221.46389434,75.43650867)(221.59389421,75.35650875)(221.71390015,75.26651611)
\curveto(221.77389403,75.21650889)(221.83389397,75.16150894)(221.89390015,75.10151611)
\curveto(221.96389384,75.05150905)(222.05389375,75.03650907)(222.16390015,75.05651611)
\curveto(222.18389362,75.08650902)(222.1988936,75.11150899)(222.20890015,75.13151611)
\curveto(222.22889357,75.15150895)(222.24389356,75.18150892)(222.25390015,75.22151611)
\curveto(222.28389352,75.31150879)(222.29389351,75.42650868)(222.28390015,75.56651611)
\lineto(222.28390015,75.94151611)
\lineto(222.28390015,77.66651611)
\lineto(222.28390015,78.13151611)
\curveto(222.28389352,78.31150579)(222.30889349,78.44150566)(222.35890015,78.52151611)
\curveto(222.3988934,78.59150551)(222.45889334,78.63650547)(222.53890015,78.65651611)
\curveto(222.55889324,78.65650545)(222.58389322,78.65650545)(222.61390015,78.65651611)
\curveto(222.64389316,78.66650544)(222.66889313,78.67150543)(222.68890015,78.67151611)
\curveto(222.82889297,78.68150542)(222.97389283,78.68150542)(223.12390015,78.67151611)
\curveto(223.28389252,78.67150543)(223.39389241,78.63150547)(223.45390015,78.55151611)
\curveto(223.5038923,78.47150563)(223.52889227,78.37150573)(223.52890015,78.25151611)
\lineto(223.52890015,77.87651611)
\lineto(223.52890015,68.81651611)
\moveto(222.31390015,71.65151611)
\curveto(222.33389347,71.7015124)(222.34389346,71.76651234)(222.34390015,71.84651611)
\curveto(222.34389346,71.93651217)(222.33389347,72.0065121)(222.31390015,72.05651611)
\lineto(222.31390015,72.28151611)
\curveto(222.29389351,72.37151173)(222.27889352,72.46151164)(222.26890015,72.55151611)
\curveto(222.25889354,72.65151145)(222.23889356,72.74151136)(222.20890015,72.82151611)
\curveto(222.18889361,72.9015112)(222.16889363,72.97651113)(222.14890015,73.04651611)
\curveto(222.13889366,73.11651099)(222.11889368,73.18651092)(222.08890015,73.25651611)
\curveto(221.96889383,73.55651055)(221.81389399,73.82151028)(221.62390015,74.05151611)
\curveto(221.43389437,74.28150982)(221.19389461,74.46150964)(220.90390015,74.59151611)
\curveto(220.803895,74.64150946)(220.6988951,74.67650943)(220.58890015,74.69651611)
\curveto(220.48889531,74.72650938)(220.37889542,74.75150935)(220.25890015,74.77151611)
\curveto(220.17889562,74.79150931)(220.08889571,74.8015093)(219.98890015,74.80151611)
\lineto(219.71890015,74.80151611)
\curveto(219.66889613,74.79150931)(219.62389618,74.78150932)(219.58390015,74.77151611)
\lineto(219.44890015,74.77151611)
\curveto(219.36889643,74.75150935)(219.28389652,74.73150937)(219.19390015,74.71151611)
\curveto(219.11389669,74.69150941)(219.03389677,74.66650944)(218.95390015,74.63651611)
\curveto(218.63389717,74.49650961)(218.37389743,74.29150981)(218.17390015,74.02151611)
\curveto(217.98389782,73.76151034)(217.82889797,73.45651065)(217.70890015,73.10651611)
\curveto(217.66889813,72.99651111)(217.63889816,72.88151122)(217.61890015,72.76151611)
\curveto(217.60889819,72.65151145)(217.59389821,72.54151156)(217.57390015,72.43151611)
\curveto(217.57389823,72.39151171)(217.56889823,72.35151175)(217.55890015,72.31151611)
\lineto(217.55890015,72.20651611)
\curveto(217.53889826,72.15651195)(217.52889827,72.101512)(217.52890015,72.04151611)
\curveto(217.53889826,71.98151212)(217.54389826,71.92651218)(217.54390015,71.87651611)
\lineto(217.54390015,71.54651611)
\curveto(217.54389826,71.44651266)(217.55389825,71.35151275)(217.57390015,71.26151611)
\curveto(217.58389822,71.23151287)(217.58889821,71.18151292)(217.58890015,71.11151611)
\curveto(217.60889819,71.04151306)(217.62389818,70.97151313)(217.63390015,70.90151611)
\lineto(217.69390015,70.69151611)
\curveto(217.803898,70.34151376)(217.95389785,70.04151406)(218.14390015,69.79151611)
\curveto(218.33389747,69.54151456)(218.57389723,69.33651477)(218.86390015,69.17651611)
\curveto(218.95389685,69.12651498)(219.04389676,69.08651502)(219.13390015,69.05651611)
\curveto(219.22389658,69.02651508)(219.32389648,68.99651511)(219.43390015,68.96651611)
\curveto(219.48389632,68.94651516)(219.53389627,68.94151516)(219.58390015,68.95151611)
\curveto(219.64389616,68.96151514)(219.6988961,68.95651515)(219.74890015,68.93651611)
\curveto(219.78889601,68.92651518)(219.82889597,68.92151518)(219.86890015,68.92151611)
\lineto(220.00390015,68.92151611)
\lineto(220.13890015,68.92151611)
\curveto(220.16889563,68.93151517)(220.21889558,68.93651517)(220.28890015,68.93651611)
\curveto(220.36889543,68.95651515)(220.44889535,68.97151513)(220.52890015,68.98151611)
\curveto(220.60889519,69.0015151)(220.68389512,69.02651508)(220.75390015,69.05651611)
\curveto(221.08389472,69.19651491)(221.34889445,69.37151473)(221.54890015,69.58151611)
\curveto(221.75889404,69.8015143)(221.93389387,70.07651403)(222.07390015,70.40651611)
\curveto(222.12389368,70.51651359)(222.15889364,70.62651348)(222.17890015,70.73651611)
\curveto(222.1988936,70.84651326)(222.22389358,70.95651315)(222.25390015,71.06651611)
\curveto(222.27389353,71.106513)(222.28389352,71.14151296)(222.28390015,71.17151611)
\curveto(222.28389352,71.21151289)(222.28889351,71.25151285)(222.29890015,71.29151611)
\curveto(222.30889349,71.35151275)(222.30889349,71.41151269)(222.29890015,71.47151611)
\curveto(222.2988935,71.53151257)(222.3038935,71.59151251)(222.31390015,71.65151611)
}
}
{
\newrgbcolor{curcolor}{0 0 0}
\pscustom[linestyle=none,fillstyle=solid,fillcolor=curcolor]
{
\newpath
\moveto(232.60015015,72.20651611)
\curveto(232.62014209,72.14651196)(232.63014208,72.05151205)(232.63015015,71.92151611)
\curveto(232.63014208,71.8015123)(232.62514208,71.71651239)(232.61515015,71.66651611)
\lineto(232.61515015,71.51651611)
\curveto(232.6051421,71.43651267)(232.59514211,71.36151274)(232.58515015,71.29151611)
\curveto(232.58514212,71.23151287)(232.58014213,71.16151294)(232.57015015,71.08151611)
\curveto(232.55014216,71.02151308)(232.53514217,70.96151314)(232.52515015,70.90151611)
\curveto(232.52514218,70.84151326)(232.51514219,70.78151332)(232.49515015,70.72151611)
\curveto(232.45514225,70.59151351)(232.42014229,70.46151364)(232.39015015,70.33151611)
\curveto(232.36014235,70.2015139)(232.32014239,70.08151402)(232.27015015,69.97151611)
\curveto(232.06014265,69.49151461)(231.78014293,69.08651502)(231.43015015,68.75651611)
\curveto(231.08014363,68.43651567)(230.65014406,68.19151591)(230.14015015,68.02151611)
\curveto(230.03014468,67.98151612)(229.9101448,67.95151615)(229.78015015,67.93151611)
\curveto(229.66014505,67.91151619)(229.53514517,67.89151621)(229.40515015,67.87151611)
\curveto(229.34514536,67.86151624)(229.28014543,67.85651625)(229.21015015,67.85651611)
\curveto(229.15014556,67.84651626)(229.09014562,67.84151626)(229.03015015,67.84151611)
\curveto(228.99014572,67.83151627)(228.93014578,67.82651628)(228.85015015,67.82651611)
\curveto(228.78014593,67.82651628)(228.73014598,67.83151627)(228.70015015,67.84151611)
\curveto(228.66014605,67.85151625)(228.62014609,67.85651625)(228.58015015,67.85651611)
\curveto(228.54014617,67.84651626)(228.5051462,67.84651626)(228.47515015,67.85651611)
\lineto(228.38515015,67.85651611)
\lineto(228.02515015,67.90151611)
\curveto(227.88514682,67.94151616)(227.75014696,67.98151612)(227.62015015,68.02151611)
\curveto(227.49014722,68.06151604)(227.36514734,68.106516)(227.24515015,68.15651611)
\curveto(226.79514791,68.35651575)(226.42514828,68.61651549)(226.13515015,68.93651611)
\curveto(225.84514886,69.25651485)(225.6051491,69.64651446)(225.41515015,70.10651611)
\curveto(225.36514934,70.2065139)(225.32514938,70.3065138)(225.29515015,70.40651611)
\curveto(225.27514943,70.5065136)(225.25514945,70.61151349)(225.23515015,70.72151611)
\curveto(225.21514949,70.76151334)(225.2051495,70.79151331)(225.20515015,70.81151611)
\curveto(225.21514949,70.84151326)(225.21514949,70.87651323)(225.20515015,70.91651611)
\curveto(225.18514952,70.99651311)(225.17014954,71.07651303)(225.16015015,71.15651611)
\curveto(225.16014955,71.24651286)(225.15014956,71.33151277)(225.13015015,71.41151611)
\lineto(225.13015015,71.53151611)
\curveto(225.13014958,71.57151253)(225.12514958,71.61651249)(225.11515015,71.66651611)
\curveto(225.1051496,71.71651239)(225.10014961,71.8015123)(225.10015015,71.92151611)
\curveto(225.10014961,72.05151205)(225.1101496,72.14651196)(225.13015015,72.20651611)
\curveto(225.15014956,72.27651183)(225.15514955,72.34651176)(225.14515015,72.41651611)
\curveto(225.13514957,72.48651162)(225.14014957,72.55651155)(225.16015015,72.62651611)
\curveto(225.17014954,72.67651143)(225.17514953,72.71651139)(225.17515015,72.74651611)
\curveto(225.18514952,72.78651132)(225.19514951,72.83151127)(225.20515015,72.88151611)
\curveto(225.23514947,73.0015111)(225.26014945,73.12151098)(225.28015015,73.24151611)
\curveto(225.3101494,73.36151074)(225.35014936,73.47651063)(225.40015015,73.58651611)
\curveto(225.55014916,73.95651015)(225.73014898,74.28650982)(225.94015015,74.57651611)
\curveto(226.16014855,74.87650923)(226.42514828,75.12650898)(226.73515015,75.32651611)
\curveto(226.85514785,75.4065087)(226.98014773,75.47150863)(227.11015015,75.52151611)
\curveto(227.24014747,75.58150852)(227.37514733,75.64150846)(227.51515015,75.70151611)
\curveto(227.63514707,75.75150835)(227.76514694,75.78150832)(227.90515015,75.79151611)
\curveto(228.04514666,75.81150829)(228.18514652,75.84150826)(228.32515015,75.88151611)
\lineto(228.52015015,75.88151611)
\curveto(228.59014612,75.89150821)(228.65514605,75.9015082)(228.71515015,75.91151611)
\curveto(229.6051451,75.92150818)(230.34514436,75.73650837)(230.93515015,75.35651611)
\curveto(231.52514318,74.97650913)(231.95014276,74.48150962)(232.21015015,73.87151611)
\curveto(232.26014245,73.77151033)(232.30014241,73.67151043)(232.33015015,73.57151611)
\curveto(232.36014235,73.47151063)(232.39514231,73.36651074)(232.43515015,73.25651611)
\curveto(232.46514224,73.14651096)(232.49014222,73.02651108)(232.51015015,72.89651611)
\curveto(232.53014218,72.77651133)(232.55514215,72.65151145)(232.58515015,72.52151611)
\curveto(232.59514211,72.47151163)(232.59514211,72.41651169)(232.58515015,72.35651611)
\curveto(232.58514212,72.3065118)(232.59014212,72.25651185)(232.60015015,72.20651611)
\moveto(231.26515015,71.35151611)
\curveto(231.28514342,71.42151268)(231.29014342,71.5015126)(231.28015015,71.59151611)
\lineto(231.28015015,71.84651611)
\curveto(231.28014343,72.23651187)(231.24514346,72.56651154)(231.17515015,72.83651611)
\curveto(231.14514356,72.91651119)(231.12014359,72.99651111)(231.10015015,73.07651611)
\curveto(231.08014363,73.15651095)(231.05514365,73.23151087)(231.02515015,73.30151611)
\curveto(230.74514396,73.95151015)(230.30014441,74.4015097)(229.69015015,74.65151611)
\curveto(229.62014509,74.68150942)(229.54514516,74.7015094)(229.46515015,74.71151611)
\lineto(229.22515015,74.77151611)
\curveto(229.14514556,74.79150931)(229.06014565,74.8015093)(228.97015015,74.80151611)
\lineto(228.70015015,74.80151611)
\lineto(228.43015015,74.75651611)
\curveto(228.33014638,74.73650937)(228.23514647,74.71150939)(228.14515015,74.68151611)
\curveto(228.06514664,74.66150944)(227.98514672,74.63150947)(227.90515015,74.59151611)
\curveto(227.83514687,74.57150953)(227.77014694,74.54150956)(227.71015015,74.50151611)
\curveto(227.65014706,74.46150964)(227.59514711,74.42150968)(227.54515015,74.38151611)
\curveto(227.3051474,74.21150989)(227.1101476,74.0065101)(226.96015015,73.76651611)
\curveto(226.8101479,73.52651058)(226.68014803,73.24651086)(226.57015015,72.92651611)
\curveto(226.54014817,72.82651128)(226.52014819,72.72151138)(226.51015015,72.61151611)
\curveto(226.50014821,72.51151159)(226.48514822,72.4065117)(226.46515015,72.29651611)
\curveto(226.45514825,72.25651185)(226.45014826,72.19151191)(226.45015015,72.10151611)
\curveto(226.44014827,72.07151203)(226.43514827,72.03651207)(226.43515015,71.99651611)
\curveto(226.44514826,71.95651215)(226.45014826,71.91151219)(226.45015015,71.86151611)
\lineto(226.45015015,71.56151611)
\curveto(226.45014826,71.46151264)(226.46014825,71.37151273)(226.48015015,71.29151611)
\lineto(226.51015015,71.11151611)
\curveto(226.53014818,71.01151309)(226.54514816,70.91151319)(226.55515015,70.81151611)
\curveto(226.57514813,70.72151338)(226.6051481,70.63651347)(226.64515015,70.55651611)
\curveto(226.74514796,70.31651379)(226.86014785,70.09151401)(226.99015015,69.88151611)
\curveto(227.13014758,69.67151443)(227.30014741,69.49651461)(227.50015015,69.35651611)
\curveto(227.55014716,69.32651478)(227.59514711,69.3015148)(227.63515015,69.28151611)
\curveto(227.67514703,69.26151484)(227.72014699,69.23651487)(227.77015015,69.20651611)
\curveto(227.85014686,69.15651495)(227.93514677,69.11151499)(228.02515015,69.07151611)
\curveto(228.12514658,69.04151506)(228.23014648,69.01151509)(228.34015015,68.98151611)
\curveto(228.39014632,68.96151514)(228.43514627,68.95151515)(228.47515015,68.95151611)
\curveto(228.52514618,68.96151514)(228.57514613,68.96151514)(228.62515015,68.95151611)
\curveto(228.65514605,68.94151516)(228.71514599,68.93151517)(228.80515015,68.92151611)
\curveto(228.9051458,68.91151519)(228.98014573,68.91651519)(229.03015015,68.93651611)
\curveto(229.07014564,68.94651516)(229.1101456,68.94651516)(229.15015015,68.93651611)
\curveto(229.19014552,68.93651517)(229.23014548,68.94651516)(229.27015015,68.96651611)
\curveto(229.35014536,68.98651512)(229.43014528,69.0015151)(229.51015015,69.01151611)
\curveto(229.59014512,69.03151507)(229.66514504,69.05651505)(229.73515015,69.08651611)
\curveto(230.07514463,69.22651488)(230.35014436,69.42151468)(230.56015015,69.67151611)
\curveto(230.77014394,69.92151418)(230.94514376,70.21651389)(231.08515015,70.55651611)
\curveto(231.13514357,70.67651343)(231.16514354,70.8015133)(231.17515015,70.93151611)
\curveto(231.19514351,71.07151303)(231.22514348,71.21151289)(231.26515015,71.35151611)
}
}
{
\newrgbcolor{curcolor}{0 0 0}
\pscustom[linestyle=none,fillstyle=solid,fillcolor=curcolor]
{
\newpath
\moveto(373.29141846,68.78651611)
\curveto(373.31140891,68.73651537)(373.33640889,68.67651543)(373.36641846,68.60651611)
\curveto(373.39640883,68.53651557)(373.41640881,68.46151564)(373.42641846,68.38151611)
\curveto(373.44640878,68.31151579)(373.44640878,68.24151586)(373.42641846,68.17151611)
\curveto(373.41640881,68.11151599)(373.37640885,68.06651604)(373.30641846,68.03651611)
\curveto(373.25640897,68.01651609)(373.19640903,68.0065161)(373.12641846,68.00651611)
\lineto(372.91641846,68.00651611)
\lineto(372.46641846,68.00651611)
\curveto(372.31640991,68.0065161)(372.19641003,68.03151607)(372.10641846,68.08151611)
\curveto(372.00641022,68.14151596)(371.93141029,68.24651586)(371.88141846,68.39651611)
\curveto(371.84141038,68.54651556)(371.79641043,68.68151542)(371.74641846,68.80151611)
\curveto(371.63641059,69.06151504)(371.53641069,69.33151477)(371.44641846,69.61151611)
\curveto(371.35641087,69.89151421)(371.25641097,70.16651394)(371.14641846,70.43651611)
\curveto(371.11641111,70.52651358)(371.08641114,70.61151349)(371.05641846,70.69151611)
\curveto(371.03641119,70.77151333)(371.00641122,70.84651326)(370.96641846,70.91651611)
\curveto(370.93641129,70.98651312)(370.89141133,71.04651306)(370.83141846,71.09651611)
\curveto(370.77141145,71.14651296)(370.69141153,71.18651292)(370.59141846,71.21651611)
\curveto(370.54141168,71.23651287)(370.48141174,71.24151286)(370.41141846,71.23151611)
\lineto(370.21641846,71.23151611)
\lineto(367.38141846,71.23151611)
\lineto(367.08141846,71.23151611)
\curveto(366.97141525,71.24151286)(366.86641536,71.24151286)(366.76641846,71.23151611)
\curveto(366.66641556,71.22151288)(366.57141565,71.2065129)(366.48141846,71.18651611)
\curveto(366.40141582,71.16651294)(366.34141588,71.12651298)(366.30141846,71.06651611)
\curveto(366.221416,70.96651314)(366.16141606,70.85151325)(366.12141846,70.72151611)
\curveto(366.09141613,70.6015135)(366.05141617,70.47651363)(366.00141846,70.34651611)
\curveto(365.90141632,70.11651399)(365.80641642,69.87651423)(365.71641846,69.62651611)
\curveto(365.63641659,69.37651473)(365.54641668,69.13651497)(365.44641846,68.90651611)
\curveto(365.4264168,68.84651526)(365.40141682,68.77651533)(365.37141846,68.69651611)
\curveto(365.35141687,68.62651548)(365.3264169,68.55151555)(365.29641846,68.47151611)
\curveto(365.26641696,68.39151571)(365.23141699,68.31651579)(365.19141846,68.24651611)
\curveto(365.16141706,68.18651592)(365.1264171,68.14151596)(365.08641846,68.11151611)
\curveto(365.00641722,68.05151605)(364.89641733,68.01651609)(364.75641846,68.00651611)
\lineto(364.33641846,68.00651611)
\lineto(364.09641846,68.00651611)
\curveto(364.0264182,68.01651609)(363.96641826,68.04151606)(363.91641846,68.08151611)
\curveto(363.86641836,68.11151599)(363.83641839,68.15651595)(363.82641846,68.21651611)
\curveto(363.8264184,68.27651583)(363.83141839,68.33651577)(363.84141846,68.39651611)
\curveto(363.86141836,68.46651564)(363.88141834,68.53151557)(363.90141846,68.59151611)
\curveto(363.93141829,68.66151544)(363.95641827,68.71151539)(363.97641846,68.74151611)
\curveto(364.11641811,69.06151504)(364.24141798,69.37651473)(364.35141846,69.68651611)
\curveto(364.46141776,70.0065141)(364.58141764,70.32651378)(364.71141846,70.64651611)
\curveto(364.80141742,70.86651324)(364.88641734,71.08151302)(364.96641846,71.29151611)
\curveto(365.04641718,71.51151259)(365.13141709,71.73151237)(365.22141846,71.95151611)
\curveto(365.5214167,72.67151143)(365.80641642,73.39651071)(366.07641846,74.12651611)
\curveto(366.34641588,74.86650924)(366.63141559,75.6015085)(366.93141846,76.33151611)
\curveto(367.04141518,76.59150751)(367.14141508,76.85650725)(367.23141846,77.12651611)
\curveto(367.33141489,77.39650671)(367.43641479,77.66150644)(367.54641846,77.92151611)
\curveto(367.59641463,78.03150607)(367.64141458,78.15150595)(367.68141846,78.28151611)
\curveto(367.73141449,78.42150568)(367.80141442,78.52150558)(367.89141846,78.58151611)
\curveto(367.93141429,78.62150548)(367.99641423,78.65150545)(368.08641846,78.67151611)
\curveto(368.10641412,78.68150542)(368.1264141,78.68150542)(368.14641846,78.67151611)
\curveto(368.17641405,78.67150543)(368.20141402,78.67650543)(368.22141846,78.68651611)
\curveto(368.40141382,78.68650542)(368.61141361,78.68650542)(368.85141846,78.68651611)
\curveto(369.09141313,78.69650541)(369.26641296,78.66150544)(369.37641846,78.58151611)
\curveto(369.45641277,78.52150558)(369.51641271,78.42150568)(369.55641846,78.28151611)
\curveto(369.60641262,78.15150595)(369.65641257,78.03150607)(369.70641846,77.92151611)
\curveto(369.80641242,77.69150641)(369.89641233,77.46150664)(369.97641846,77.23151611)
\curveto(370.05641217,77.0015071)(370.14641208,76.77150733)(370.24641846,76.54151611)
\curveto(370.3264119,76.34150776)(370.40141182,76.13650797)(370.47141846,75.92651611)
\curveto(370.55141167,75.71650839)(370.63641159,75.51150859)(370.72641846,75.31151611)
\curveto(371.0264112,74.58150952)(371.31141091,73.84151026)(371.58141846,73.09151611)
\curveto(371.86141036,72.35151175)(372.15641007,71.61651249)(372.46641846,70.88651611)
\curveto(372.50640972,70.79651331)(372.53640969,70.71151339)(372.55641846,70.63151611)
\curveto(372.58640964,70.55151355)(372.61640961,70.46651364)(372.64641846,70.37651611)
\curveto(372.75640947,70.11651399)(372.86140936,69.85151425)(372.96141846,69.58151611)
\curveto(373.07140915,69.31151479)(373.18140904,69.04651506)(373.29141846,68.78651611)
\moveto(370.08141846,72.43151611)
\curveto(370.17141205,72.46151164)(370.226412,72.51151159)(370.24641846,72.58151611)
\curveto(370.27641195,72.65151145)(370.28141194,72.72651138)(370.26141846,72.80651611)
\curveto(370.25141197,72.89651121)(370.226412,72.98151112)(370.18641846,73.06151611)
\curveto(370.15641207,73.15151095)(370.1264121,73.22651088)(370.09641846,73.28651611)
\curveto(370.07641215,73.32651078)(370.06641216,73.36151074)(370.06641846,73.39151611)
\curveto(370.06641216,73.42151068)(370.05641217,73.45651065)(370.03641846,73.49651611)
\lineto(369.94641846,73.73651611)
\curveto(369.9264123,73.82651028)(369.89641233,73.91651019)(369.85641846,74.00651611)
\curveto(369.70641252,74.36650974)(369.57141265,74.73150937)(369.45141846,75.10151611)
\curveto(369.34141288,75.48150862)(369.21141301,75.85150825)(369.06141846,76.21151611)
\curveto(369.01141321,76.32150778)(368.96641326,76.43150767)(368.92641846,76.54151611)
\curveto(368.89641333,76.65150745)(368.85641337,76.75650735)(368.80641846,76.85651611)
\curveto(368.78641344,76.9065072)(368.76141346,76.95150715)(368.73141846,76.99151611)
\curveto(368.71141351,77.04150706)(368.66141356,77.06650704)(368.58141846,77.06651611)
\curveto(368.56141366,77.04650706)(368.54141368,77.03150707)(368.52141846,77.02151611)
\curveto(368.50141372,77.01150709)(368.48141374,76.99650711)(368.46141846,76.97651611)
\curveto(368.4214138,76.92650718)(368.39141383,76.87150723)(368.37141846,76.81151611)
\curveto(368.35141387,76.76150734)(368.33141389,76.7065074)(368.31141846,76.64651611)
\curveto(368.26141396,76.53650757)(368.221414,76.42650768)(368.19141846,76.31651611)
\curveto(368.16141406,76.2065079)(368.1214141,76.09650801)(368.07141846,75.98651611)
\curveto(367.90141432,75.59650851)(367.75141447,75.2015089)(367.62141846,74.80151611)
\curveto(367.50141472,74.4015097)(367.36141486,74.01151009)(367.20141846,73.63151611)
\lineto(367.14141846,73.48151611)
\curveto(367.13141509,73.43151067)(367.11641511,73.38151072)(367.09641846,73.33151611)
\lineto(367.00641846,73.09151611)
\curveto(366.97641525,73.01151109)(366.95141527,72.93151117)(366.93141846,72.85151611)
\curveto(366.91141531,72.8015113)(366.90141532,72.74651136)(366.90141846,72.68651611)
\curveto(366.91141531,72.62651148)(366.9264153,72.57651153)(366.94641846,72.53651611)
\curveto(366.99641523,72.45651165)(367.10141512,72.41151169)(367.26141846,72.40151611)
\lineto(367.71141846,72.40151611)
\lineto(369.31641846,72.40151611)
\curveto(369.4264128,72.4015117)(369.56141266,72.39651171)(369.72141846,72.38651611)
\curveto(369.88141234,72.38651172)(370.00141222,72.4015117)(370.08141846,72.43151611)
}
}
{
\newrgbcolor{curcolor}{0 0 0}
\pscustom[linestyle=none,fillstyle=solid,fillcolor=curcolor]
{
\newpath
\moveto(374.87298096,75.73151611)
\lineto(375.30798096,75.73151611)
\curveto(375.45797899,75.73150837)(375.56297889,75.69150841)(375.62298096,75.61151611)
\curveto(375.67297878,75.53150857)(375.69797875,75.43150867)(375.69798096,75.31151611)
\curveto(375.70797874,75.19150891)(375.71297874,75.07150903)(375.71298096,74.95151611)
\lineto(375.71298096,73.52651611)
\lineto(375.71298096,71.26151611)
\lineto(375.71298096,70.57151611)
\curveto(375.71297874,70.34151376)(375.73797871,70.14151396)(375.78798096,69.97151611)
\curveto(375.9479785,69.52151458)(376.2479782,69.2065149)(376.68798096,69.02651611)
\curveto(376.90797754,68.93651517)(377.17297728,68.9015152)(377.48298096,68.92151611)
\curveto(377.79297666,68.95151515)(378.04297641,69.0065151)(378.23298096,69.08651611)
\curveto(378.56297589,69.22651488)(378.82297563,69.4015147)(379.01298096,69.61151611)
\curveto(379.21297524,69.83151427)(379.36797508,70.11651399)(379.47798096,70.46651611)
\curveto(379.50797494,70.54651356)(379.52797492,70.62651348)(379.53798096,70.70651611)
\curveto(379.5479749,70.78651332)(379.56297489,70.87151323)(379.58298096,70.96151611)
\curveto(379.59297486,71.01151309)(379.59297486,71.05651305)(379.58298096,71.09651611)
\curveto(379.58297487,71.13651297)(379.59297486,71.18151292)(379.61298096,71.23151611)
\lineto(379.61298096,71.54651611)
\curveto(379.63297482,71.62651248)(379.63797481,71.71651239)(379.62798096,71.81651611)
\curveto(379.61797483,71.92651218)(379.61297484,72.02651208)(379.61298096,72.11651611)
\lineto(379.61298096,73.28651611)
\lineto(379.61298096,74.87651611)
\curveto(379.61297484,74.99650911)(379.60797484,75.12150898)(379.59798096,75.25151611)
\curveto(379.59797485,75.39150871)(379.62297483,75.5015086)(379.67298096,75.58151611)
\curveto(379.71297474,75.63150847)(379.75797469,75.66150844)(379.80798096,75.67151611)
\curveto(379.86797458,75.69150841)(379.93797451,75.71150839)(380.01798096,75.73151611)
\lineto(380.24298096,75.73151611)
\curveto(380.36297409,75.73150837)(380.46797398,75.72650838)(380.55798096,75.71651611)
\curveto(380.65797379,75.7065084)(380.73297372,75.66150844)(380.78298096,75.58151611)
\curveto(380.83297362,75.53150857)(380.85797359,75.45650865)(380.85798096,75.35651611)
\lineto(380.85798096,75.07151611)
\lineto(380.85798096,74.05151611)
\lineto(380.85798096,70.01651611)
\lineto(380.85798096,68.66651611)
\curveto(380.85797359,68.54651556)(380.8529736,68.43151567)(380.84298096,68.32151611)
\curveto(380.84297361,68.22151588)(380.80797364,68.14651596)(380.73798096,68.09651611)
\curveto(380.69797375,68.06651604)(380.63797381,68.04151606)(380.55798096,68.02151611)
\curveto(380.47797397,68.01151609)(380.38797406,68.0015161)(380.28798096,67.99151611)
\curveto(380.19797425,67.99151611)(380.10797434,67.99651611)(380.01798096,68.00651611)
\curveto(379.93797451,68.01651609)(379.87797457,68.03651607)(379.83798096,68.06651611)
\curveto(379.78797466,68.106516)(379.74297471,68.17151593)(379.70298096,68.26151611)
\curveto(379.69297476,68.3015158)(379.68297477,68.35651575)(379.67298096,68.42651611)
\curveto(379.67297478,68.49651561)(379.66797478,68.56151554)(379.65798096,68.62151611)
\curveto(379.6479748,68.69151541)(379.62797482,68.74651536)(379.59798096,68.78651611)
\curveto(379.56797488,68.82651528)(379.52297493,68.84151526)(379.46298096,68.83151611)
\curveto(379.38297507,68.81151529)(379.30297515,68.75151535)(379.22298096,68.65151611)
\curveto(379.14297531,68.56151554)(379.06797538,68.49151561)(378.99798096,68.44151611)
\curveto(378.77797567,68.28151582)(378.52797592,68.14151596)(378.24798096,68.02151611)
\curveto(378.13797631,67.97151613)(378.02297643,67.94151616)(377.90298096,67.93151611)
\curveto(377.79297666,67.91151619)(377.67797677,67.88651622)(377.55798096,67.85651611)
\curveto(377.50797694,67.84651626)(377.452977,67.84651626)(377.39298096,67.85651611)
\curveto(377.34297711,67.86651624)(377.29297716,67.86151624)(377.24298096,67.84151611)
\curveto(377.14297731,67.82151628)(377.0529774,67.82151628)(376.97298096,67.84151611)
\lineto(376.82298096,67.84151611)
\curveto(376.77297768,67.86151624)(376.71297774,67.87151623)(376.64298096,67.87151611)
\curveto(376.58297787,67.87151623)(376.52797792,67.87651623)(376.47798096,67.88651611)
\curveto(376.43797801,67.9065162)(376.39797805,67.91651619)(376.35798096,67.91651611)
\curveto(376.32797812,67.9065162)(376.28797816,67.91151619)(376.23798096,67.93151611)
\lineto(375.99798096,67.99151611)
\curveto(375.92797852,68.01151609)(375.8529786,68.04151606)(375.77298096,68.08151611)
\curveto(375.51297894,68.19151591)(375.29297916,68.33651577)(375.11298096,68.51651611)
\curveto(374.94297951,68.7065154)(374.80297965,68.93151517)(374.69298096,69.19151611)
\curveto(374.6529798,69.28151482)(374.62297983,69.37151473)(374.60298096,69.46151611)
\lineto(374.54298096,69.76151611)
\curveto(374.52297993,69.82151428)(374.51297994,69.87651423)(374.51298096,69.92651611)
\curveto(374.52297993,69.98651412)(374.51797993,70.05151405)(374.49798096,70.12151611)
\curveto(374.48797996,70.14151396)(374.48297997,70.16651394)(374.48298096,70.19651611)
\curveto(374.48297997,70.23651387)(374.47797997,70.27151383)(374.46798096,70.30151611)
\lineto(374.46798096,70.45151611)
\curveto(374.45797999,70.49151361)(374.45298,70.53651357)(374.45298096,70.58651611)
\curveto(374.46297999,70.64651346)(374.46797998,70.7015134)(374.46798096,70.75151611)
\lineto(374.46798096,71.35151611)
\lineto(374.46798096,74.11151611)
\lineto(374.46798096,75.07151611)
\lineto(374.46798096,75.34151611)
\curveto(374.46797998,75.43150867)(374.48797996,75.5065086)(374.52798096,75.56651611)
\curveto(374.56797988,75.63650847)(374.64297981,75.68650842)(374.75298096,75.71651611)
\curveto(374.77297968,75.72650838)(374.79297966,75.72650838)(374.81298096,75.71651611)
\curveto(374.83297962,75.71650839)(374.8529796,75.72150838)(374.87298096,75.73151611)
}
}
{
\newrgbcolor{curcolor}{0 0 0}
\pscustom[linestyle=none,fillstyle=solid,fillcolor=curcolor]
{
\newpath
\moveto(382.77259033,75.73151611)
\lineto(383.29759033,75.73151611)
\curveto(383.49758868,75.74150836)(383.64758853,75.72150838)(383.74759033,75.67151611)
\curveto(383.86758831,75.62150848)(383.96258821,75.54150856)(384.03259033,75.43151611)
\curveto(384.11258806,75.32150878)(384.18758799,75.21150889)(384.25759033,75.10151611)
\curveto(384.38758779,74.9015092)(384.51758766,74.7065094)(384.64759033,74.51651611)
\curveto(384.7775874,74.33650977)(384.91258726,74.14650996)(385.05259033,73.94651611)
\curveto(385.10258707,73.86651024)(385.15258702,73.79151031)(385.20259033,73.72151611)
\curveto(385.26258691,73.65151045)(385.31758686,73.58151052)(385.36759033,73.51151611)
\curveto(385.40758677,73.45151065)(385.44758673,73.39651071)(385.48759033,73.34651611)
\curveto(385.52758665,73.29651081)(385.58758659,73.26151084)(385.66759033,73.24151611)
\curveto(385.71758646,73.22151088)(385.75758642,73.22151088)(385.78759033,73.24151611)
\curveto(385.82758635,73.27151083)(385.85758632,73.29651081)(385.87759033,73.31651611)
\curveto(385.95758622,73.36651074)(386.02258615,73.43651067)(386.07259033,73.52651611)
\curveto(386.13258604,73.61651049)(386.18758599,73.7015104)(386.23759033,73.78151611)
\curveto(386.38758579,73.98151012)(386.53758564,74.18650992)(386.68759033,74.39651611)
\lineto(387.13759033,75.02651611)
\curveto(387.21758496,75.13650897)(387.29758488,75.25150885)(387.37759033,75.37151611)
\curveto(387.45758472,75.49150861)(387.55258462,75.58650852)(387.66259033,75.65651611)
\curveto(387.74258443,75.7065084)(387.83758434,75.73150837)(387.94759033,75.73151611)
\lineto(388.29259033,75.73151611)
\lineto(388.42759033,75.73151611)
\curveto(388.4775837,75.73150837)(388.52758365,75.72650838)(388.57759033,75.71651611)
\lineto(388.65259033,75.71651611)
\curveto(388.7725834,75.69650841)(388.85258332,75.65650845)(388.89259033,75.59651611)
\curveto(388.91258326,75.54650856)(388.90758327,75.49150861)(388.87759033,75.43151611)
\curveto(388.85758332,75.38150872)(388.83758334,75.34150876)(388.81759033,75.31151611)
\lineto(388.60759033,75.01151611)
\curveto(388.53758364,74.92150918)(388.46258371,74.82650928)(388.38259033,74.72651611)
\curveto(388.15258402,74.4065097)(387.91758426,74.09151001)(387.67759033,73.78151611)
\curveto(387.44758473,73.48151062)(387.21758496,73.17151093)(386.98759033,72.85151611)
\curveto(386.93758524,72.77151133)(386.88258529,72.69151141)(386.82259033,72.61151611)
\curveto(386.76258541,72.54151156)(386.70758547,72.46151164)(386.65759033,72.37151611)
\curveto(386.63758554,72.34151176)(386.61758556,72.3015118)(386.59759033,72.25151611)
\curveto(386.5775856,72.21151189)(386.5775856,72.16151194)(386.59759033,72.10151611)
\curveto(386.61758556,72.01151209)(386.64758553,71.93651217)(386.68759033,71.87651611)
\curveto(386.73758544,71.81651229)(386.78758539,71.75151235)(386.83759033,71.68151611)
\lineto(387.01759033,71.41151611)
\curveto(387.08758509,71.32151278)(387.15258502,71.23151287)(387.21259033,71.14151611)
\lineto(387.90259033,70.18151611)
\lineto(388.59259033,69.22151611)
\curveto(388.6725835,69.11151499)(388.75258342,68.99651511)(388.83259033,68.87651611)
\lineto(389.07259033,68.54651611)
\curveto(389.12258305,68.47651563)(389.16258301,68.41151569)(389.19259033,68.35151611)
\curveto(389.23258294,68.3015158)(389.24258293,68.22151588)(389.22259033,68.11151611)
\curveto(389.20258297,68.101516)(389.18258299,68.08651602)(389.16259033,68.06651611)
\curveto(389.15258302,68.05651605)(389.13758304,68.04651606)(389.11759033,68.03651611)
\curveto(389.06758311,68.01651609)(389.00258317,68.0065161)(388.92259033,68.00651611)
\lineto(388.68259033,68.00651611)
\lineto(388.17259033,68.00651611)
\curveto(388.03258414,68.01651609)(387.90758427,68.06151604)(387.79759033,68.14151611)
\curveto(387.74758443,68.17151593)(387.70758447,68.2065159)(387.67759033,68.24651611)
\curveto(387.65758452,68.29651581)(387.63258454,68.34651576)(387.60259033,68.39651611)
\lineto(387.45259033,68.60651611)
\curveto(387.40258477,68.67651543)(387.35258482,68.75151535)(387.30259033,68.83151611)
\lineto(386.35759033,70.22651611)
\curveto(386.30758587,70.3065138)(386.25758592,70.38151372)(386.20759033,70.45151611)
\curveto(386.15758602,70.52151358)(386.10758607,70.59651351)(386.05759033,70.67651611)
\curveto(386.00758617,70.74651336)(385.95758622,70.8065133)(385.90759033,70.85651611)
\curveto(385.86758631,70.91651319)(385.80758637,70.95651315)(385.72759033,70.97651611)
\curveto(385.6775865,70.99651311)(385.62758655,70.98651312)(385.57759033,70.94651611)
\curveto(385.53758664,70.91651319)(385.50758667,70.89151321)(385.48759033,70.87151611)
\curveto(385.40758677,70.79151331)(385.33758684,70.7015134)(385.27759033,70.60151611)
\curveto(385.21758696,70.5015136)(385.15758702,70.4065137)(385.09759033,70.31651611)
\curveto(384.92758725,70.05651405)(384.75258742,69.79651431)(384.57259033,69.53651611)
\curveto(384.40258777,69.28651482)(384.22758795,69.03651507)(384.04759033,68.78651611)
\curveto(383.99758818,68.7065154)(383.94258823,68.62651548)(383.88259033,68.54651611)
\lineto(383.73259033,68.30651611)
\curveto(383.71258846,68.27651583)(383.68758849,68.24151586)(383.65759033,68.20151611)
\curveto(383.63758854,68.17151593)(383.61258856,68.14651596)(383.58259033,68.12651611)
\curveto(383.48258869,68.05651605)(383.36258881,68.01651609)(383.22259033,68.00651611)
\lineto(382.77259033,68.00651611)
\lineto(382.54759033,68.00651611)
\curveto(382.4775897,68.0065161)(382.41758976,68.01651609)(382.36759033,68.03651611)
\curveto(382.33758984,68.05651605)(382.31258986,68.07151603)(382.29259033,68.08151611)
\curveto(382.28258989,68.101516)(382.26758991,68.12151598)(382.24759033,68.14151611)
\curveto(382.23758994,68.25151585)(382.25258992,68.33651577)(382.29259033,68.39651611)
\curveto(382.34258983,68.45651565)(382.39258978,68.52151558)(382.44259033,68.59151611)
\curveto(382.52258965,68.7015154)(382.59758958,68.8015153)(382.66759033,68.89151611)
\curveto(382.73758944,68.99151511)(382.80758937,69.09651501)(382.87759033,69.20651611)
\curveto(383.09758908,69.5065146)(383.31258886,69.8065143)(383.52259033,70.10651611)
\lineto(384.15259033,71.00651611)
\curveto(384.22258795,71.09651301)(384.28758789,71.18651292)(384.34759033,71.27651611)
\curveto(384.41758776,71.36651274)(384.48258769,71.46151264)(384.54259033,71.56151611)
\curveto(384.59258758,71.63151247)(384.64258753,71.69651241)(384.69259033,71.75651611)
\curveto(384.74258743,71.82651228)(384.7775874,71.91651219)(384.79759033,72.02651611)
\curveto(384.81758736,72.07651203)(384.81258736,72.12651198)(384.78259033,72.17651611)
\curveto(384.76258741,72.22651188)(384.74258743,72.26651184)(384.72259033,72.29651611)
\curveto(384.6725875,72.38651172)(384.61758756,72.47151163)(384.55759033,72.55151611)
\lineto(384.37759033,72.79151611)
\curveto(384.14758803,73.11151099)(383.91258826,73.43151067)(383.67259033,73.75151611)
\lineto(382.98259033,74.71151611)
\curveto(382.90258927,74.82150928)(382.82258935,74.92150918)(382.74259033,75.01151611)
\curveto(382.6725895,75.101509)(382.60258957,75.2015089)(382.53259033,75.31151611)
\curveto(382.51258966,75.34150876)(382.49258968,75.38150872)(382.47259033,75.43151611)
\curveto(382.45258972,75.49150861)(382.45258972,75.54150856)(382.47259033,75.58151611)
\curveto(382.49258968,75.63150847)(382.52258965,75.66150844)(382.56259033,75.67151611)
\curveto(382.60258957,75.69150841)(382.64758953,75.7065084)(382.69759033,75.71651611)
\curveto(382.71758946,75.72650838)(382.73258944,75.72650838)(382.74259033,75.71651611)
\curveto(382.75258942,75.71650839)(382.76258941,75.72150838)(382.77259033,75.73151611)
}
}
{
\newrgbcolor{curcolor}{0 0 0}
\pscustom[linestyle=none,fillstyle=solid,fillcolor=curcolor]
{
\newpath
\moveto(390.80626221,77.23151611)
\curveto(390.72626109,77.29150681)(390.68126113,77.39650671)(390.67126221,77.54651611)
\lineto(390.67126221,78.01151611)
\lineto(390.67126221,78.26651611)
\curveto(390.67126114,78.35650575)(390.68626113,78.43150567)(390.71626221,78.49151611)
\curveto(390.75626106,78.57150553)(390.83626098,78.63150547)(390.95626221,78.67151611)
\curveto(390.97626084,78.68150542)(390.99626082,78.68150542)(391.01626221,78.67151611)
\curveto(391.04626077,78.67150543)(391.07126074,78.67650543)(391.09126221,78.68651611)
\curveto(391.26126055,78.68650542)(391.42126039,78.68150542)(391.57126221,78.67151611)
\curveto(391.72126009,78.66150544)(391.82125999,78.6015055)(391.87126221,78.49151611)
\curveto(391.90125991,78.43150567)(391.9162599,78.35650575)(391.91626221,78.26651611)
\lineto(391.91626221,78.01151611)
\curveto(391.9162599,77.83150627)(391.9112599,77.66150644)(391.90126221,77.50151611)
\curveto(391.90125991,77.34150676)(391.83625998,77.23650687)(391.70626221,77.18651611)
\curveto(391.65626016,77.16650694)(391.60126021,77.15650695)(391.54126221,77.15651611)
\lineto(391.37626221,77.15651611)
\lineto(391.06126221,77.15651611)
\curveto(390.96126085,77.15650695)(390.87626094,77.18150692)(390.80626221,77.23151611)
\moveto(391.91626221,68.72651611)
\lineto(391.91626221,68.41151611)
\curveto(391.92625989,68.31151579)(391.90625991,68.23151587)(391.85626221,68.17151611)
\curveto(391.82625999,68.11151599)(391.78126003,68.07151603)(391.72126221,68.05151611)
\curveto(391.66126015,68.04151606)(391.59126022,68.02651608)(391.51126221,68.00651611)
\lineto(391.28626221,68.00651611)
\curveto(391.15626066,68.0065161)(391.04126077,68.01151609)(390.94126221,68.02151611)
\curveto(390.85126096,68.04151606)(390.78126103,68.09151601)(390.73126221,68.17151611)
\curveto(390.69126112,68.23151587)(390.67126114,68.3065158)(390.67126221,68.39651611)
\lineto(390.67126221,68.68151611)
\lineto(390.67126221,75.02651611)
\lineto(390.67126221,75.34151611)
\curveto(390.67126114,75.45150865)(390.69626112,75.53650857)(390.74626221,75.59651611)
\curveto(390.77626104,75.64650846)(390.816261,75.67650843)(390.86626221,75.68651611)
\curveto(390.9162609,75.69650841)(390.97126084,75.71150839)(391.03126221,75.73151611)
\curveto(391.05126076,75.73150837)(391.07126074,75.72650838)(391.09126221,75.71651611)
\curveto(391.12126069,75.71650839)(391.14626067,75.72150838)(391.16626221,75.73151611)
\curveto(391.29626052,75.73150837)(391.42626039,75.72650838)(391.55626221,75.71651611)
\curveto(391.69626012,75.71650839)(391.79126002,75.67650843)(391.84126221,75.59651611)
\curveto(391.89125992,75.53650857)(391.9162599,75.45650865)(391.91626221,75.35651611)
\lineto(391.91626221,75.07151611)
\lineto(391.91626221,68.72651611)
}
}
{
\newrgbcolor{curcolor}{0 0 0}
\pscustom[linestyle=none,fillstyle=solid,fillcolor=curcolor]
{
\newpath
\moveto(394.43110596,78.68651611)
\curveto(394.56110434,78.68650542)(394.69610421,78.68650542)(394.83610596,78.68651611)
\curveto(394.98610392,78.68650542)(395.09610381,78.65150545)(395.16610596,78.58151611)
\curveto(395.21610369,78.51150559)(395.24110366,78.41650569)(395.24110596,78.29651611)
\curveto(395.25110365,78.18650592)(395.25610365,78.07150603)(395.25610596,77.95151611)
\lineto(395.25610596,76.61651611)
\lineto(395.25610596,70.54151611)
\lineto(395.25610596,68.86151611)
\lineto(395.25610596,68.47151611)
\curveto(395.25610365,68.33151577)(395.23110367,68.22151588)(395.18110596,68.14151611)
\curveto(395.15110375,68.09151601)(395.1061038,68.06151604)(395.04610596,68.05151611)
\curveto(394.99610391,68.04151606)(394.93110397,68.02651608)(394.85110596,68.00651611)
\lineto(394.64110596,68.00651611)
\lineto(394.32610596,68.00651611)
\curveto(394.22610468,68.01651609)(394.15110475,68.05151605)(394.10110596,68.11151611)
\curveto(394.05110485,68.19151591)(394.02110488,68.29151581)(394.01110596,68.41151611)
\lineto(394.01110596,68.78651611)
\lineto(394.01110596,70.16651611)
\lineto(394.01110596,76.40651611)
\lineto(394.01110596,77.87651611)
\curveto(394.01110489,77.98650612)(394.0061049,78.101506)(393.99610596,78.22151611)
\curveto(393.99610491,78.35150575)(394.02110488,78.45150565)(394.07110596,78.52151611)
\curveto(394.11110479,78.58150552)(394.18610472,78.63150547)(394.29610596,78.67151611)
\curveto(394.31610459,78.68150542)(394.33610457,78.68150542)(394.35610596,78.67151611)
\curveto(394.38610452,78.67150543)(394.41110449,78.67650543)(394.43110596,78.68651611)
}
}
{
\newrgbcolor{curcolor}{0 0 0}
\pscustom[linestyle=none,fillstyle=solid,fillcolor=curcolor]
{
\newpath
\moveto(397.48594971,77.23151611)
\curveto(397.40594859,77.29150681)(397.36094863,77.39650671)(397.35094971,77.54651611)
\lineto(397.35094971,78.01151611)
\lineto(397.35094971,78.26651611)
\curveto(397.35094864,78.35650575)(397.36594863,78.43150567)(397.39594971,78.49151611)
\curveto(397.43594856,78.57150553)(397.51594848,78.63150547)(397.63594971,78.67151611)
\curveto(397.65594834,78.68150542)(397.67594832,78.68150542)(397.69594971,78.67151611)
\curveto(397.72594827,78.67150543)(397.75094824,78.67650543)(397.77094971,78.68651611)
\curveto(397.94094805,78.68650542)(398.10094789,78.68150542)(398.25094971,78.67151611)
\curveto(398.40094759,78.66150544)(398.50094749,78.6015055)(398.55094971,78.49151611)
\curveto(398.58094741,78.43150567)(398.5959474,78.35650575)(398.59594971,78.26651611)
\lineto(398.59594971,78.01151611)
\curveto(398.5959474,77.83150627)(398.5909474,77.66150644)(398.58094971,77.50151611)
\curveto(398.58094741,77.34150676)(398.51594748,77.23650687)(398.38594971,77.18651611)
\curveto(398.33594766,77.16650694)(398.28094771,77.15650695)(398.22094971,77.15651611)
\lineto(398.05594971,77.15651611)
\lineto(397.74094971,77.15651611)
\curveto(397.64094835,77.15650695)(397.55594844,77.18150692)(397.48594971,77.23151611)
\moveto(398.59594971,68.72651611)
\lineto(398.59594971,68.41151611)
\curveto(398.60594739,68.31151579)(398.58594741,68.23151587)(398.53594971,68.17151611)
\curveto(398.50594749,68.11151599)(398.46094753,68.07151603)(398.40094971,68.05151611)
\curveto(398.34094765,68.04151606)(398.27094772,68.02651608)(398.19094971,68.00651611)
\lineto(397.96594971,68.00651611)
\curveto(397.83594816,68.0065161)(397.72094827,68.01151609)(397.62094971,68.02151611)
\curveto(397.53094846,68.04151606)(397.46094853,68.09151601)(397.41094971,68.17151611)
\curveto(397.37094862,68.23151587)(397.35094864,68.3065158)(397.35094971,68.39651611)
\lineto(397.35094971,68.68151611)
\lineto(397.35094971,75.02651611)
\lineto(397.35094971,75.34151611)
\curveto(397.35094864,75.45150865)(397.37594862,75.53650857)(397.42594971,75.59651611)
\curveto(397.45594854,75.64650846)(397.4959485,75.67650843)(397.54594971,75.68651611)
\curveto(397.5959484,75.69650841)(397.65094834,75.71150839)(397.71094971,75.73151611)
\curveto(397.73094826,75.73150837)(397.75094824,75.72650838)(397.77094971,75.71651611)
\curveto(397.80094819,75.71650839)(397.82594817,75.72150838)(397.84594971,75.73151611)
\curveto(397.97594802,75.73150837)(398.10594789,75.72650838)(398.23594971,75.71651611)
\curveto(398.37594762,75.71650839)(398.47094752,75.67650843)(398.52094971,75.59651611)
\curveto(398.57094742,75.53650857)(398.5959474,75.45650865)(398.59594971,75.35651611)
\lineto(398.59594971,75.07151611)
\lineto(398.59594971,68.72651611)
}
}
{
\newrgbcolor{curcolor}{0 0 0}
\pscustom[linestyle=none,fillstyle=solid,fillcolor=curcolor]
{
\newpath
\moveto(407.42579346,68.56151611)
\curveto(407.45578563,68.4015157)(407.44078564,68.26651584)(407.38079346,68.15651611)
\curveto(407.32078576,68.05651605)(407.24078584,67.98151612)(407.14079346,67.93151611)
\curveto(407.09078599,67.91151619)(407.03578605,67.9015162)(406.97579346,67.90151611)
\curveto(406.92578616,67.9015162)(406.87078621,67.89151621)(406.81079346,67.87151611)
\curveto(406.59078649,67.82151628)(406.37078671,67.83651627)(406.15079346,67.91651611)
\curveto(405.94078714,67.98651612)(405.79578729,68.07651603)(405.71579346,68.18651611)
\curveto(405.66578742,68.25651585)(405.62078746,68.33651577)(405.58079346,68.42651611)
\curveto(405.54078754,68.52651558)(405.49078759,68.6065155)(405.43079346,68.66651611)
\curveto(405.41078767,68.68651542)(405.3857877,68.7065154)(405.35579346,68.72651611)
\curveto(405.33578775,68.74651536)(405.30578778,68.75151535)(405.26579346,68.74151611)
\curveto(405.15578793,68.71151539)(405.05078803,68.65651545)(404.95079346,68.57651611)
\curveto(404.86078822,68.49651561)(404.77078831,68.42651568)(404.68079346,68.36651611)
\curveto(404.55078853,68.28651582)(404.41078867,68.21151589)(404.26079346,68.14151611)
\curveto(404.11078897,68.08151602)(403.95078913,68.02651608)(403.78079346,67.97651611)
\curveto(403.6807894,67.94651616)(403.57078951,67.92651618)(403.45079346,67.91651611)
\curveto(403.34078974,67.9065162)(403.23078985,67.89151621)(403.12079346,67.87151611)
\curveto(403.07079001,67.86151624)(403.02579006,67.85651625)(402.98579346,67.85651611)
\lineto(402.88079346,67.85651611)
\curveto(402.77079031,67.83651627)(402.66579042,67.83651627)(402.56579346,67.85651611)
\lineto(402.43079346,67.85651611)
\curveto(402.3807907,67.86651624)(402.33079075,67.87151623)(402.28079346,67.87151611)
\curveto(402.23079085,67.87151623)(402.1857909,67.88151622)(402.14579346,67.90151611)
\curveto(402.10579098,67.91151619)(402.07079101,67.91651619)(402.04079346,67.91651611)
\curveto(402.02079106,67.9065162)(401.99579109,67.9065162)(401.96579346,67.91651611)
\lineto(401.72579346,67.97651611)
\curveto(401.64579144,67.98651612)(401.57079151,68.0065161)(401.50079346,68.03651611)
\curveto(401.20079188,68.16651594)(400.95579213,68.31151579)(400.76579346,68.47151611)
\curveto(400.5857925,68.64151546)(400.43579265,68.87651523)(400.31579346,69.17651611)
\curveto(400.22579286,69.39651471)(400.1807929,69.66151444)(400.18079346,69.97151611)
\lineto(400.18079346,70.28651611)
\curveto(400.19079289,70.33651377)(400.19579289,70.38651372)(400.19579346,70.43651611)
\lineto(400.22579346,70.61651611)
\lineto(400.34579346,70.94651611)
\curveto(400.3857927,71.05651305)(400.43579265,71.15651295)(400.49579346,71.24651611)
\curveto(400.67579241,71.53651257)(400.92079216,71.75151235)(401.23079346,71.89151611)
\curveto(401.54079154,72.03151207)(401.8807912,72.15651195)(402.25079346,72.26651611)
\curveto(402.39079069,72.3065118)(402.53579055,72.33651177)(402.68579346,72.35651611)
\curveto(402.83579025,72.37651173)(402.9857901,72.4015117)(403.13579346,72.43151611)
\curveto(403.20578988,72.45151165)(403.27078981,72.46151164)(403.33079346,72.46151611)
\curveto(403.40078968,72.46151164)(403.47578961,72.47151163)(403.55579346,72.49151611)
\curveto(403.62578946,72.51151159)(403.69578939,72.52151158)(403.76579346,72.52151611)
\curveto(403.83578925,72.53151157)(403.91078917,72.54651156)(403.99079346,72.56651611)
\curveto(404.24078884,72.62651148)(404.47578861,72.67651143)(404.69579346,72.71651611)
\curveto(404.91578817,72.76651134)(405.09078799,72.88151122)(405.22079346,73.06151611)
\curveto(405.2807878,73.14151096)(405.33078775,73.24151086)(405.37079346,73.36151611)
\curveto(405.41078767,73.49151061)(405.41078767,73.63151047)(405.37079346,73.78151611)
\curveto(405.31078777,74.02151008)(405.22078786,74.21150989)(405.10079346,74.35151611)
\curveto(404.99078809,74.49150961)(404.83078825,74.6015095)(404.62079346,74.68151611)
\curveto(404.50078858,74.73150937)(404.35578873,74.76650934)(404.18579346,74.78651611)
\curveto(404.02578906,74.8065093)(403.85578923,74.81650929)(403.67579346,74.81651611)
\curveto(403.49578959,74.81650929)(403.32078976,74.8065093)(403.15079346,74.78651611)
\curveto(402.9807901,74.76650934)(402.83579025,74.73650937)(402.71579346,74.69651611)
\curveto(402.54579054,74.63650947)(402.3807907,74.55150955)(402.22079346,74.44151611)
\curveto(402.14079094,74.38150972)(402.06579102,74.3015098)(401.99579346,74.20151611)
\curveto(401.93579115,74.11150999)(401.8807912,74.01151009)(401.83079346,73.90151611)
\curveto(401.80079128,73.82151028)(401.77079131,73.73651037)(401.74079346,73.64651611)
\curveto(401.72079136,73.55651055)(401.67579141,73.48651062)(401.60579346,73.43651611)
\curveto(401.56579152,73.4065107)(401.49579159,73.38151072)(401.39579346,73.36151611)
\curveto(401.30579178,73.35151075)(401.21079187,73.34651076)(401.11079346,73.34651611)
\curveto(401.01079207,73.34651076)(400.91079217,73.35151075)(400.81079346,73.36151611)
\curveto(400.72079236,73.38151072)(400.65579243,73.4065107)(400.61579346,73.43651611)
\curveto(400.57579251,73.46651064)(400.54579254,73.51651059)(400.52579346,73.58651611)
\curveto(400.50579258,73.65651045)(400.50579258,73.73151037)(400.52579346,73.81151611)
\curveto(400.55579253,73.94151016)(400.5857925,74.06151004)(400.61579346,74.17151611)
\curveto(400.65579243,74.29150981)(400.70079238,74.4065097)(400.75079346,74.51651611)
\curveto(400.94079214,74.86650924)(401.1807919,75.13650897)(401.47079346,75.32651611)
\curveto(401.76079132,75.52650858)(402.12079096,75.68650842)(402.55079346,75.80651611)
\curveto(402.65079043,75.82650828)(402.75079033,75.84150826)(402.85079346,75.85151611)
\curveto(402.96079012,75.86150824)(403.07079001,75.87650823)(403.18079346,75.89651611)
\curveto(403.22078986,75.9065082)(403.2857898,75.9065082)(403.37579346,75.89651611)
\curveto(403.46578962,75.89650821)(403.52078956,75.9065082)(403.54079346,75.92651611)
\curveto(404.24078884,75.93650817)(404.85078823,75.85650825)(405.37079346,75.68651611)
\curveto(405.89078719,75.51650859)(406.25578683,75.19150891)(406.46579346,74.71151611)
\curveto(406.55578653,74.51150959)(406.60578648,74.27650983)(406.61579346,74.00651611)
\curveto(406.63578645,73.74651036)(406.64578644,73.47151063)(406.64579346,73.18151611)
\lineto(406.64579346,69.86651611)
\curveto(406.64578644,69.72651438)(406.65078643,69.59151451)(406.66079346,69.46151611)
\curveto(406.67078641,69.33151477)(406.70078638,69.22651488)(406.75079346,69.14651611)
\curveto(406.80078628,69.07651503)(406.86578622,69.02651508)(406.94579346,68.99651611)
\curveto(407.03578605,68.95651515)(407.12078596,68.92651518)(407.20079346,68.90651611)
\curveto(407.2807858,68.89651521)(407.34078574,68.85151525)(407.38079346,68.77151611)
\curveto(407.40078568,68.74151536)(407.41078567,68.71151539)(407.41079346,68.68151611)
\curveto(407.41078567,68.65151545)(407.41578567,68.61151549)(407.42579346,68.56151611)
\moveto(405.28079346,70.22651611)
\curveto(405.34078774,70.36651374)(405.37078771,70.52651358)(405.37079346,70.70651611)
\curveto(405.3807877,70.89651321)(405.3857877,71.09151301)(405.38579346,71.29151611)
\curveto(405.3857877,71.4015127)(405.3807877,71.5015126)(405.37079346,71.59151611)
\curveto(405.36078772,71.68151242)(405.32078776,71.75151235)(405.25079346,71.80151611)
\curveto(405.22078786,71.82151228)(405.15078793,71.83151227)(405.04079346,71.83151611)
\curveto(405.02078806,71.81151229)(404.9857881,71.8015123)(404.93579346,71.80151611)
\curveto(404.8857882,71.8015123)(404.84078824,71.79151231)(404.80079346,71.77151611)
\curveto(404.72078836,71.75151235)(404.63078845,71.73151237)(404.53079346,71.71151611)
\lineto(404.23079346,71.65151611)
\curveto(404.20078888,71.65151245)(404.16578892,71.64651246)(404.12579346,71.63651611)
\lineto(404.02079346,71.63651611)
\curveto(403.87078921,71.59651251)(403.70578938,71.57151253)(403.52579346,71.56151611)
\curveto(403.35578973,71.56151254)(403.19578989,71.54151256)(403.04579346,71.50151611)
\curveto(402.96579012,71.48151262)(402.89079019,71.46151264)(402.82079346,71.44151611)
\curveto(402.76079032,71.43151267)(402.69079039,71.41651269)(402.61079346,71.39651611)
\curveto(402.45079063,71.34651276)(402.30079078,71.28151282)(402.16079346,71.20151611)
\curveto(402.02079106,71.13151297)(401.90079118,71.04151306)(401.80079346,70.93151611)
\curveto(401.70079138,70.82151328)(401.62579146,70.68651342)(401.57579346,70.52651611)
\curveto(401.52579156,70.37651373)(401.50579158,70.19151391)(401.51579346,69.97151611)
\curveto(401.51579157,69.87151423)(401.53079155,69.77651433)(401.56079346,69.68651611)
\curveto(401.60079148,69.6065145)(401.64579144,69.53151457)(401.69579346,69.46151611)
\curveto(401.77579131,69.35151475)(401.8807912,69.25651485)(402.01079346,69.17651611)
\curveto(402.14079094,69.106515)(402.2807908,69.04651506)(402.43079346,68.99651611)
\curveto(402.4807906,68.98651512)(402.53079055,68.98151512)(402.58079346,68.98151611)
\curveto(402.63079045,68.98151512)(402.6807904,68.97651513)(402.73079346,68.96651611)
\curveto(402.80079028,68.94651516)(402.8857902,68.93151517)(402.98579346,68.92151611)
\curveto(403.09578999,68.92151518)(403.1857899,68.93151517)(403.25579346,68.95151611)
\curveto(403.31578977,68.97151513)(403.37578971,68.97651513)(403.43579346,68.96651611)
\curveto(403.49578959,68.96651514)(403.55578953,68.97651513)(403.61579346,68.99651611)
\curveto(403.69578939,69.01651509)(403.77078931,69.03151507)(403.84079346,69.04151611)
\curveto(403.92078916,69.05151505)(403.99578909,69.07151503)(404.06579346,69.10151611)
\curveto(404.35578873,69.22151488)(404.60078848,69.36651474)(404.80079346,69.53651611)
\curveto(405.01078807,69.7065144)(405.17078791,69.93651417)(405.28079346,70.22651611)
}
}
{
\newrgbcolor{curcolor}{0 0 0}
\pscustom[linestyle=none,fillstyle=solid,fillcolor=curcolor]
{
\newpath
\moveto(412.24243408,75.91151611)
\curveto(412.47242929,75.91150819)(412.60242916,75.85150825)(412.63243408,75.73151611)
\curveto(412.6624291,75.62150848)(412.67742909,75.45650865)(412.67743408,75.23651611)
\lineto(412.67743408,74.95151611)
\curveto(412.67742909,74.86150924)(412.65242911,74.78650932)(412.60243408,74.72651611)
\curveto(412.54242922,74.64650946)(412.45742931,74.6015095)(412.34743408,74.59151611)
\curveto(412.23742953,74.59150951)(412.12742964,74.57650953)(412.01743408,74.54651611)
\curveto(411.87742989,74.51650959)(411.74243002,74.48650962)(411.61243408,74.45651611)
\curveto(411.49243027,74.42650968)(411.37743039,74.38650972)(411.26743408,74.33651611)
\curveto(410.97743079,74.2065099)(410.74243102,74.02651008)(410.56243408,73.79651611)
\curveto(410.38243138,73.57651053)(410.22743154,73.32151078)(410.09743408,73.03151611)
\curveto(410.05743171,72.92151118)(410.02743174,72.8065113)(410.00743408,72.68651611)
\curveto(409.98743178,72.57651153)(409.9624318,72.46151164)(409.93243408,72.34151611)
\curveto(409.92243184,72.29151181)(409.91743185,72.24151186)(409.91743408,72.19151611)
\curveto(409.92743184,72.14151196)(409.92743184,72.09151201)(409.91743408,72.04151611)
\curveto(409.88743188,71.92151218)(409.87243189,71.78151232)(409.87243408,71.62151611)
\curveto(409.88243188,71.47151263)(409.88743188,71.32651278)(409.88743408,71.18651611)
\lineto(409.88743408,69.34151611)
\lineto(409.88743408,68.99651611)
\curveto(409.88743188,68.87651523)(409.88243188,68.76151534)(409.87243408,68.65151611)
\curveto(409.8624319,68.54151556)(409.85743191,68.44651566)(409.85743408,68.36651611)
\curveto(409.8674319,68.28651582)(409.84743192,68.21651589)(409.79743408,68.15651611)
\curveto(409.74743202,68.08651602)(409.6674321,68.04651606)(409.55743408,68.03651611)
\curveto(409.45743231,68.02651608)(409.34743242,68.02151608)(409.22743408,68.02151611)
\lineto(408.95743408,68.02151611)
\curveto(408.90743286,68.04151606)(408.85743291,68.05651605)(408.80743408,68.06651611)
\curveto(408.767433,68.08651602)(408.73743303,68.11151599)(408.71743408,68.14151611)
\curveto(408.6674331,68.21151589)(408.63743313,68.29651581)(408.62743408,68.39651611)
\lineto(408.62743408,68.72651611)
\lineto(408.62743408,69.88151611)
\lineto(408.62743408,74.03651611)
\lineto(408.62743408,75.07151611)
\lineto(408.62743408,75.37151611)
\curveto(408.63743313,75.47150863)(408.6674331,75.55650855)(408.71743408,75.62651611)
\curveto(408.74743302,75.66650844)(408.79743297,75.69650841)(408.86743408,75.71651611)
\curveto(408.94743282,75.73650837)(409.03243273,75.74650836)(409.12243408,75.74651611)
\curveto(409.21243255,75.75650835)(409.30243246,75.75650835)(409.39243408,75.74651611)
\curveto(409.48243228,75.73650837)(409.55243221,75.72150838)(409.60243408,75.70151611)
\curveto(409.68243208,75.67150843)(409.73243203,75.61150849)(409.75243408,75.52151611)
\curveto(409.78243198,75.44150866)(409.79743197,75.35150875)(409.79743408,75.25151611)
\lineto(409.79743408,74.95151611)
\curveto(409.79743197,74.85150925)(409.81743195,74.76150934)(409.85743408,74.68151611)
\curveto(409.8674319,74.66150944)(409.87743189,74.64650946)(409.88743408,74.63651611)
\lineto(409.93243408,74.59151611)
\curveto(410.04243172,74.59150951)(410.13243163,74.63650947)(410.20243408,74.72651611)
\curveto(410.27243149,74.82650928)(410.33243143,74.9065092)(410.38243408,74.96651611)
\lineto(410.47243408,75.05651611)
\curveto(410.5624312,75.16650894)(410.68743108,75.28150882)(410.84743408,75.40151611)
\curveto(411.00743076,75.52150858)(411.15743061,75.61150849)(411.29743408,75.67151611)
\curveto(411.38743038,75.72150838)(411.48243028,75.75650835)(411.58243408,75.77651611)
\curveto(411.68243008,75.8065083)(411.78742998,75.83650827)(411.89743408,75.86651611)
\curveto(411.95742981,75.87650823)(412.01742975,75.88150822)(412.07743408,75.88151611)
\curveto(412.13742963,75.89150821)(412.19242957,75.9015082)(412.24243408,75.91151611)
}
}
{
\newrgbcolor{curcolor}{0 0 0}
\pscustom[linestyle=none,fillstyle=solid,fillcolor=curcolor]
{
\newpath
\moveto(536.28160645,78.68651611)
\lineto(537.19660645,78.68651611)
\curveto(537.2966038,78.68650542)(537.3916037,78.68650542)(537.48160645,78.68651611)
\curveto(537.57160352,78.68650542)(537.64660345,78.66650544)(537.70660645,78.62651611)
\curveto(537.7966033,78.56650554)(537.85660324,78.48650562)(537.88660645,78.38651611)
\curveto(537.92660317,78.28650582)(537.97160312,78.18150592)(538.02160645,78.07151611)
\curveto(538.10160299,77.88150622)(538.17160292,77.69150641)(538.23160645,77.50151611)
\curveto(538.30160279,77.31150679)(538.37660272,77.12150698)(538.45660645,76.93151611)
\curveto(538.52660257,76.75150735)(538.5916025,76.56650754)(538.65160645,76.37651611)
\curveto(538.71160238,76.19650791)(538.78160231,76.01650809)(538.86160645,75.83651611)
\curveto(538.92160217,75.69650841)(538.97660212,75.55150855)(539.02660645,75.40151611)
\curveto(539.07660202,75.25150885)(539.13160196,75.106509)(539.19160645,74.96651611)
\curveto(539.37160172,74.51650959)(539.54160155,74.06151004)(539.70160645,73.60151611)
\curveto(539.86160123,73.15151095)(540.03160106,72.7015114)(540.21160645,72.25151611)
\curveto(540.23160086,72.2015119)(540.24660085,72.15151195)(540.25660645,72.10151611)
\lineto(540.31660645,71.95151611)
\curveto(540.40660069,71.73151237)(540.4916006,71.5065126)(540.57160645,71.27651611)
\curveto(540.65160044,71.05651305)(540.73660036,70.83651327)(540.82660645,70.61651611)
\curveto(540.86660023,70.52651358)(540.90660019,70.41651369)(540.94660645,70.28651611)
\curveto(540.98660011,70.16651394)(541.05160004,70.09651401)(541.14160645,70.07651611)
\curveto(541.18159991,70.06651404)(541.21159988,70.06651404)(541.23160645,70.07651611)
\lineto(541.29160645,70.13651611)
\curveto(541.34159975,70.18651392)(541.37659972,70.24151386)(541.39660645,70.30151611)
\curveto(541.42659967,70.36151374)(541.45659964,70.42651368)(541.48660645,70.49651611)
\lineto(541.72660645,71.12651611)
\curveto(541.80659929,71.34651276)(541.88659921,71.56151254)(541.96660645,71.77151611)
\lineto(542.02660645,71.92151611)
\lineto(542.08660645,72.10151611)
\curveto(542.16659893,72.29151181)(542.23659886,72.48151162)(542.29660645,72.67151611)
\curveto(542.36659873,72.87151123)(542.44159865,73.07151103)(542.52160645,73.27151611)
\curveto(542.76159833,73.85151025)(542.98159811,74.43650967)(543.18160645,75.02651611)
\curveto(543.3915977,75.61650849)(543.61659748,76.2015079)(543.85660645,76.78151611)
\curveto(543.93659716,76.98150712)(544.01159708,77.18650692)(544.08160645,77.39651611)
\curveto(544.16159693,77.6065065)(544.24159685,77.81150629)(544.32160645,78.01151611)
\curveto(544.36159673,78.09150601)(544.3965967,78.19150591)(544.42660645,78.31151611)
\curveto(544.46659663,78.43150567)(544.52159657,78.51650559)(544.59160645,78.56651611)
\curveto(544.65159644,78.6065055)(544.72659637,78.63650547)(544.81660645,78.65651611)
\curveto(544.91659618,78.67650543)(545.02659607,78.68650542)(545.14660645,78.68651611)
\curveto(545.26659583,78.69650541)(545.38659571,78.69650541)(545.50660645,78.68651611)
\curveto(545.62659547,78.68650542)(545.73659536,78.68650542)(545.83660645,78.68651611)
\curveto(545.92659517,78.68650542)(546.01659508,78.68650542)(546.10660645,78.68651611)
\curveto(546.20659489,78.68650542)(546.28159481,78.66650544)(546.33160645,78.62651611)
\curveto(546.42159467,78.57650553)(546.47159462,78.48650562)(546.48160645,78.35651611)
\curveto(546.4915946,78.22650588)(546.4965946,78.08650602)(546.49660645,77.93651611)
\lineto(546.49660645,76.28651611)
\lineto(546.49660645,70.01651611)
\lineto(546.49660645,68.75651611)
\curveto(546.4965946,68.64651546)(546.4965946,68.53651557)(546.49660645,68.42651611)
\curveto(546.50659459,68.31651579)(546.48659461,68.23151587)(546.43660645,68.17151611)
\curveto(546.40659469,68.11151599)(546.36159473,68.07151603)(546.30160645,68.05151611)
\curveto(546.24159485,68.04151606)(546.17159492,68.02651608)(546.09160645,68.00651611)
\lineto(545.85160645,68.00651611)
\lineto(545.49160645,68.00651611)
\curveto(545.38159571,68.01651609)(545.30159579,68.06151604)(545.25160645,68.14151611)
\curveto(545.23159586,68.17151593)(545.21659588,68.2015159)(545.20660645,68.23151611)
\curveto(545.20659589,68.27151583)(545.1965959,68.31651579)(545.17660645,68.36651611)
\lineto(545.17660645,68.53151611)
\curveto(545.16659593,68.59151551)(545.16159593,68.66151544)(545.16160645,68.74151611)
\curveto(545.17159592,68.82151528)(545.17659592,68.89651521)(545.17660645,68.96651611)
\lineto(545.17660645,69.80651611)
\lineto(545.17660645,74.23151611)
\curveto(545.17659592,74.48150962)(545.17659592,74.73150937)(545.17660645,74.98151611)
\curveto(545.17659592,75.24150886)(545.17159592,75.49150861)(545.16160645,75.73151611)
\curveto(545.16159593,75.83150827)(545.15659594,75.94150816)(545.14660645,76.06151611)
\curveto(545.13659596,76.18150792)(545.08159601,76.24150786)(544.98160645,76.24151611)
\lineto(544.98160645,76.22651611)
\curveto(544.91159618,76.2065079)(544.85159624,76.14150796)(544.80160645,76.03151611)
\curveto(544.76159633,75.92150818)(544.72659637,75.82650828)(544.69660645,75.74651611)
\curveto(544.62659647,75.57650853)(544.56159653,75.4015087)(544.50160645,75.22151611)
\curveto(544.44159665,75.05150905)(544.37159672,74.88150922)(544.29160645,74.71151611)
\curveto(544.27159682,74.66150944)(544.25659684,74.61650949)(544.24660645,74.57651611)
\curveto(544.23659686,74.53650957)(544.22159687,74.49150961)(544.20160645,74.44151611)
\curveto(544.12159697,74.26150984)(544.05159704,74.07651003)(543.99160645,73.88651611)
\curveto(543.94159715,73.7065104)(543.87659722,73.52651058)(543.79660645,73.34651611)
\curveto(543.72659737,73.19651091)(543.66659743,73.04151106)(543.61660645,72.88151611)
\curveto(543.56659753,72.73151137)(543.51159758,72.58151152)(543.45160645,72.43151611)
\curveto(543.25159784,71.96151214)(543.07159802,71.48651262)(542.91160645,71.00651611)
\curveto(542.75159834,70.53651357)(542.57659852,70.07151403)(542.38660645,69.61151611)
\curveto(542.30659879,69.43151467)(542.23659886,69.25151485)(542.17660645,69.07151611)
\curveto(542.11659898,68.89151521)(542.05159904,68.71151539)(541.98160645,68.53151611)
\curveto(541.93159916,68.42151568)(541.88159921,68.31651579)(541.83160645,68.21651611)
\curveto(541.7915993,68.12651598)(541.70659939,68.06151604)(541.57660645,68.02151611)
\curveto(541.55659954,68.01151609)(541.53159956,68.0065161)(541.50160645,68.00651611)
\curveto(541.48159961,68.01651609)(541.45659964,68.01651609)(541.42660645,68.00651611)
\curveto(541.3965997,67.99651611)(541.36159973,67.99151611)(541.32160645,67.99151611)
\curveto(541.28159981,68.0015161)(541.24159985,68.0065161)(541.20160645,68.00651611)
\lineto(540.90160645,68.00651611)
\curveto(540.80160029,68.0065161)(540.72160037,68.03151607)(540.66160645,68.08151611)
\curveto(540.58160051,68.13151597)(540.52160057,68.2015159)(540.48160645,68.29151611)
\curveto(540.45160064,68.39151571)(540.41160068,68.49151561)(540.36160645,68.59151611)
\curveto(540.28160081,68.79151531)(540.20160089,68.99651511)(540.12160645,69.20651611)
\curveto(540.05160104,69.42651468)(539.97660112,69.63651447)(539.89660645,69.83651611)
\curveto(539.81660128,70.01651409)(539.74660135,70.19651391)(539.68660645,70.37651611)
\curveto(539.63660146,70.56651354)(539.57160152,70.75151335)(539.49160645,70.93151611)
\curveto(539.26160183,71.49151261)(539.04660205,72.05651205)(538.84660645,72.62651611)
\curveto(538.64660245,73.19651091)(538.43160266,73.76151034)(538.20160645,74.32151611)
\lineto(537.96160645,74.95151611)
\curveto(537.8916032,75.17150893)(537.81660328,75.38150872)(537.73660645,75.58151611)
\curveto(537.68660341,75.69150841)(537.64160345,75.79650831)(537.60160645,75.89651611)
\curveto(537.57160352,76.0065081)(537.52160357,76.101508)(537.45160645,76.18151611)
\curveto(537.44160365,76.2015079)(537.43160366,76.21150789)(537.42160645,76.21151611)
\lineto(537.39160645,76.24151611)
\lineto(537.31660645,76.24151611)
\lineto(537.28660645,76.21151611)
\curveto(537.27660382,76.21150789)(537.26660383,76.2065079)(537.25660645,76.19651611)
\curveto(537.23660386,76.14650796)(537.22660387,76.09150801)(537.22660645,76.03151611)
\curveto(537.22660387,75.97150813)(537.21660388,75.91150819)(537.19660645,75.85151611)
\lineto(537.19660645,75.68651611)
\curveto(537.17660392,75.62650848)(537.17160392,75.56150854)(537.18160645,75.49151611)
\curveto(537.1916039,75.42150868)(537.1966039,75.35150875)(537.19660645,75.28151611)
\lineto(537.19660645,74.47151611)
\lineto(537.19660645,69.91151611)
\lineto(537.19660645,68.72651611)
\curveto(537.1966039,68.61651549)(537.1916039,68.5065156)(537.18160645,68.39651611)
\curveto(537.18160391,68.28651582)(537.15660394,68.2015159)(537.10660645,68.14151611)
\curveto(537.05660404,68.06151604)(536.96660413,68.01651609)(536.83660645,68.00651611)
\lineto(536.44660645,68.00651611)
\lineto(536.25160645,68.00651611)
\curveto(536.20160489,68.0065161)(536.15160494,68.01651609)(536.10160645,68.03651611)
\curveto(535.97160512,68.07651603)(535.8966052,68.16151594)(535.87660645,68.29151611)
\curveto(535.86660523,68.42151568)(535.86160523,68.57151553)(535.86160645,68.74151611)
\lineto(535.86160645,70.48151611)
\lineto(535.86160645,76.48151611)
\lineto(535.86160645,77.89151611)
\curveto(535.86160523,78.0015061)(535.85660524,78.11650599)(535.84660645,78.23651611)
\curveto(535.84660525,78.35650575)(535.87160522,78.45150565)(535.92160645,78.52151611)
\curveto(535.96160513,78.58150552)(536.03660506,78.63150547)(536.14660645,78.67151611)
\curveto(536.16660493,78.68150542)(536.18660491,78.68150542)(536.20660645,78.67151611)
\curveto(536.23660486,78.67150543)(536.26160483,78.67650543)(536.28160645,78.68651611)
}
}
{
\newrgbcolor{curcolor}{0 0 0}
\pscustom[linestyle=none,fillstyle=solid,fillcolor=curcolor]
{
\newpath
\moveto(555.72371582,72.20651611)
\curveto(555.74370776,72.14651196)(555.75370775,72.05151205)(555.75371582,71.92151611)
\curveto(555.75370775,71.8015123)(555.74870776,71.71651239)(555.73871582,71.66651611)
\lineto(555.73871582,71.51651611)
\curveto(555.72870778,71.43651267)(555.71870779,71.36151274)(555.70871582,71.29151611)
\curveto(555.7087078,71.23151287)(555.7037078,71.16151294)(555.69371582,71.08151611)
\curveto(555.67370783,71.02151308)(555.65870785,70.96151314)(555.64871582,70.90151611)
\curveto(555.64870786,70.84151326)(555.63870787,70.78151332)(555.61871582,70.72151611)
\curveto(555.57870793,70.59151351)(555.54370796,70.46151364)(555.51371582,70.33151611)
\curveto(555.48370802,70.2015139)(555.44370806,70.08151402)(555.39371582,69.97151611)
\curveto(555.18370832,69.49151461)(554.9037086,69.08651502)(554.55371582,68.75651611)
\curveto(554.2037093,68.43651567)(553.77370973,68.19151591)(553.26371582,68.02151611)
\curveto(553.15371035,67.98151612)(553.03371047,67.95151615)(552.90371582,67.93151611)
\curveto(552.78371072,67.91151619)(552.65871085,67.89151621)(552.52871582,67.87151611)
\curveto(552.46871104,67.86151624)(552.4037111,67.85651625)(552.33371582,67.85651611)
\curveto(552.27371123,67.84651626)(552.21371129,67.84151626)(552.15371582,67.84151611)
\curveto(552.11371139,67.83151627)(552.05371145,67.82651628)(551.97371582,67.82651611)
\curveto(551.9037116,67.82651628)(551.85371165,67.83151627)(551.82371582,67.84151611)
\curveto(551.78371172,67.85151625)(551.74371176,67.85651625)(551.70371582,67.85651611)
\curveto(551.66371184,67.84651626)(551.62871188,67.84651626)(551.59871582,67.85651611)
\lineto(551.50871582,67.85651611)
\lineto(551.14871582,67.90151611)
\curveto(551.0087125,67.94151616)(550.87371263,67.98151612)(550.74371582,68.02151611)
\curveto(550.61371289,68.06151604)(550.48871302,68.106516)(550.36871582,68.15651611)
\curveto(549.91871359,68.35651575)(549.54871396,68.61651549)(549.25871582,68.93651611)
\curveto(548.96871454,69.25651485)(548.72871478,69.64651446)(548.53871582,70.10651611)
\curveto(548.48871502,70.2065139)(548.44871506,70.3065138)(548.41871582,70.40651611)
\curveto(548.39871511,70.5065136)(548.37871513,70.61151349)(548.35871582,70.72151611)
\curveto(548.33871517,70.76151334)(548.32871518,70.79151331)(548.32871582,70.81151611)
\curveto(548.33871517,70.84151326)(548.33871517,70.87651323)(548.32871582,70.91651611)
\curveto(548.3087152,70.99651311)(548.29371521,71.07651303)(548.28371582,71.15651611)
\curveto(548.28371522,71.24651286)(548.27371523,71.33151277)(548.25371582,71.41151611)
\lineto(548.25371582,71.53151611)
\curveto(548.25371525,71.57151253)(548.24871526,71.61651249)(548.23871582,71.66651611)
\curveto(548.22871528,71.71651239)(548.22371528,71.8015123)(548.22371582,71.92151611)
\curveto(548.22371528,72.05151205)(548.23371527,72.14651196)(548.25371582,72.20651611)
\curveto(548.27371523,72.27651183)(548.27871523,72.34651176)(548.26871582,72.41651611)
\curveto(548.25871525,72.48651162)(548.26371524,72.55651155)(548.28371582,72.62651611)
\curveto(548.29371521,72.67651143)(548.29871521,72.71651139)(548.29871582,72.74651611)
\curveto(548.3087152,72.78651132)(548.31871519,72.83151127)(548.32871582,72.88151611)
\curveto(548.35871515,73.0015111)(548.38371512,73.12151098)(548.40371582,73.24151611)
\curveto(548.43371507,73.36151074)(548.47371503,73.47651063)(548.52371582,73.58651611)
\curveto(548.67371483,73.95651015)(548.85371465,74.28650982)(549.06371582,74.57651611)
\curveto(549.28371422,74.87650923)(549.54871396,75.12650898)(549.85871582,75.32651611)
\curveto(549.97871353,75.4065087)(550.1037134,75.47150863)(550.23371582,75.52151611)
\curveto(550.36371314,75.58150852)(550.49871301,75.64150846)(550.63871582,75.70151611)
\curveto(550.75871275,75.75150835)(550.88871262,75.78150832)(551.02871582,75.79151611)
\curveto(551.16871234,75.81150829)(551.3087122,75.84150826)(551.44871582,75.88151611)
\lineto(551.64371582,75.88151611)
\curveto(551.71371179,75.89150821)(551.77871173,75.9015082)(551.83871582,75.91151611)
\curveto(552.72871078,75.92150818)(553.46871004,75.73650837)(554.05871582,75.35651611)
\curveto(554.64870886,74.97650913)(555.07370843,74.48150962)(555.33371582,73.87151611)
\curveto(555.38370812,73.77151033)(555.42370808,73.67151043)(555.45371582,73.57151611)
\curveto(555.48370802,73.47151063)(555.51870799,73.36651074)(555.55871582,73.25651611)
\curveto(555.58870792,73.14651096)(555.61370789,73.02651108)(555.63371582,72.89651611)
\curveto(555.65370785,72.77651133)(555.67870783,72.65151145)(555.70871582,72.52151611)
\curveto(555.71870779,72.47151163)(555.71870779,72.41651169)(555.70871582,72.35651611)
\curveto(555.7087078,72.3065118)(555.71370779,72.25651185)(555.72371582,72.20651611)
\moveto(554.38871582,71.35151611)
\curveto(554.4087091,71.42151268)(554.41370909,71.5015126)(554.40371582,71.59151611)
\lineto(554.40371582,71.84651611)
\curveto(554.4037091,72.23651187)(554.36870914,72.56651154)(554.29871582,72.83651611)
\curveto(554.26870924,72.91651119)(554.24370926,72.99651111)(554.22371582,73.07651611)
\curveto(554.2037093,73.15651095)(554.17870933,73.23151087)(554.14871582,73.30151611)
\curveto(553.86870964,73.95151015)(553.42371008,74.4015097)(552.81371582,74.65151611)
\curveto(552.74371076,74.68150942)(552.66871084,74.7015094)(552.58871582,74.71151611)
\lineto(552.34871582,74.77151611)
\curveto(552.26871124,74.79150931)(552.18371132,74.8015093)(552.09371582,74.80151611)
\lineto(551.82371582,74.80151611)
\lineto(551.55371582,74.75651611)
\curveto(551.45371205,74.73650937)(551.35871215,74.71150939)(551.26871582,74.68151611)
\curveto(551.18871232,74.66150944)(551.1087124,74.63150947)(551.02871582,74.59151611)
\curveto(550.95871255,74.57150953)(550.89371261,74.54150956)(550.83371582,74.50151611)
\curveto(550.77371273,74.46150964)(550.71871279,74.42150968)(550.66871582,74.38151611)
\curveto(550.42871308,74.21150989)(550.23371327,74.0065101)(550.08371582,73.76651611)
\curveto(549.93371357,73.52651058)(549.8037137,73.24651086)(549.69371582,72.92651611)
\curveto(549.66371384,72.82651128)(549.64371386,72.72151138)(549.63371582,72.61151611)
\curveto(549.62371388,72.51151159)(549.6087139,72.4065117)(549.58871582,72.29651611)
\curveto(549.57871393,72.25651185)(549.57371393,72.19151191)(549.57371582,72.10151611)
\curveto(549.56371394,72.07151203)(549.55871395,72.03651207)(549.55871582,71.99651611)
\curveto(549.56871394,71.95651215)(549.57371393,71.91151219)(549.57371582,71.86151611)
\lineto(549.57371582,71.56151611)
\curveto(549.57371393,71.46151264)(549.58371392,71.37151273)(549.60371582,71.29151611)
\lineto(549.63371582,71.11151611)
\curveto(549.65371385,71.01151309)(549.66871384,70.91151319)(549.67871582,70.81151611)
\curveto(549.69871381,70.72151338)(549.72871378,70.63651347)(549.76871582,70.55651611)
\curveto(549.86871364,70.31651379)(549.98371352,70.09151401)(550.11371582,69.88151611)
\curveto(550.25371325,69.67151443)(550.42371308,69.49651461)(550.62371582,69.35651611)
\curveto(550.67371283,69.32651478)(550.71871279,69.3015148)(550.75871582,69.28151611)
\curveto(550.79871271,69.26151484)(550.84371266,69.23651487)(550.89371582,69.20651611)
\curveto(550.97371253,69.15651495)(551.05871245,69.11151499)(551.14871582,69.07151611)
\curveto(551.24871226,69.04151506)(551.35371215,69.01151509)(551.46371582,68.98151611)
\curveto(551.51371199,68.96151514)(551.55871195,68.95151515)(551.59871582,68.95151611)
\curveto(551.64871186,68.96151514)(551.69871181,68.96151514)(551.74871582,68.95151611)
\curveto(551.77871173,68.94151516)(551.83871167,68.93151517)(551.92871582,68.92151611)
\curveto(552.02871148,68.91151519)(552.1037114,68.91651519)(552.15371582,68.93651611)
\curveto(552.19371131,68.94651516)(552.23371127,68.94651516)(552.27371582,68.93651611)
\curveto(552.31371119,68.93651517)(552.35371115,68.94651516)(552.39371582,68.96651611)
\curveto(552.47371103,68.98651512)(552.55371095,69.0015151)(552.63371582,69.01151611)
\curveto(552.71371079,69.03151507)(552.78871072,69.05651505)(552.85871582,69.08651611)
\curveto(553.19871031,69.22651488)(553.47371003,69.42151468)(553.68371582,69.67151611)
\curveto(553.89370961,69.92151418)(554.06870944,70.21651389)(554.20871582,70.55651611)
\curveto(554.25870925,70.67651343)(554.28870922,70.8015133)(554.29871582,70.93151611)
\curveto(554.31870919,71.07151303)(554.34870916,71.21151289)(554.38871582,71.35151611)
}
}
{
\newrgbcolor{curcolor}{0 0 0}
\pscustom[linestyle=none,fillstyle=solid,fillcolor=curcolor]
{
\newpath
\moveto(564.17199707,68.81651611)
\lineto(564.17199707,68.42651611)
\curveto(564.1719892,68.3065158)(564.14698922,68.2065159)(564.09699707,68.12651611)
\curveto(564.04698932,68.05651605)(563.96198941,68.01651609)(563.84199707,68.00651611)
\lineto(563.49699707,68.00651611)
\curveto(563.43698993,68.0065161)(563.37698999,68.0015161)(563.31699707,67.99151611)
\curveto(563.2669901,67.99151611)(563.22199015,68.0015161)(563.18199707,68.02151611)
\curveto(563.09199028,68.04151606)(563.03199034,68.08151602)(563.00199707,68.14151611)
\curveto(562.96199041,68.19151591)(562.93699043,68.25151585)(562.92699707,68.32151611)
\curveto(562.92699044,68.39151571)(562.91199046,68.46151564)(562.88199707,68.53151611)
\curveto(562.8719905,68.55151555)(562.85699051,68.56651554)(562.83699707,68.57651611)
\curveto(562.82699054,68.59651551)(562.81199056,68.61651549)(562.79199707,68.63651611)
\curveto(562.69199068,68.64651546)(562.61199076,68.62651548)(562.55199707,68.57651611)
\curveto(562.50199087,68.52651558)(562.44699092,68.47651563)(562.38699707,68.42651611)
\curveto(562.18699118,68.27651583)(561.98699138,68.16151594)(561.78699707,68.08151611)
\curveto(561.60699176,68.0015161)(561.39699197,67.94151616)(561.15699707,67.90151611)
\curveto(560.92699244,67.86151624)(560.68699268,67.84151626)(560.43699707,67.84151611)
\curveto(560.19699317,67.83151627)(559.95699341,67.84651626)(559.71699707,67.88651611)
\curveto(559.47699389,67.91651619)(559.2669941,67.97151613)(559.08699707,68.05151611)
\curveto(558.5669948,68.27151583)(558.14699522,68.56651554)(557.82699707,68.93651611)
\curveto(557.50699586,69.31651479)(557.25699611,69.78651432)(557.07699707,70.34651611)
\curveto(557.03699633,70.43651367)(557.00699636,70.52651358)(556.98699707,70.61651611)
\curveto(556.97699639,70.71651339)(556.95699641,70.81651329)(556.92699707,70.91651611)
\curveto(556.91699645,70.96651314)(556.91199646,71.01651309)(556.91199707,71.06651611)
\curveto(556.91199646,71.11651299)(556.90699646,71.16651294)(556.89699707,71.21651611)
\curveto(556.87699649,71.26651284)(556.8669965,71.31651279)(556.86699707,71.36651611)
\curveto(556.87699649,71.42651268)(556.87699649,71.48151262)(556.86699707,71.53151611)
\lineto(556.86699707,71.68151611)
\curveto(556.84699652,71.73151237)(556.83699653,71.79651231)(556.83699707,71.87651611)
\curveto(556.83699653,71.95651215)(556.84699652,72.02151208)(556.86699707,72.07151611)
\lineto(556.86699707,72.23651611)
\curveto(556.88699648,72.3065118)(556.89199648,72.37651173)(556.88199707,72.44651611)
\curveto(556.88199649,72.52651158)(556.89199648,72.6015115)(556.91199707,72.67151611)
\curveto(556.92199645,72.72151138)(556.92699644,72.76651134)(556.92699707,72.80651611)
\curveto(556.92699644,72.84651126)(556.93199644,72.89151121)(556.94199707,72.94151611)
\curveto(556.9719964,73.04151106)(556.99699637,73.13651097)(557.01699707,73.22651611)
\curveto(557.03699633,73.32651078)(557.06199631,73.42151068)(557.09199707,73.51151611)
\curveto(557.22199615,73.89151021)(557.38699598,74.23150987)(557.58699707,74.53151611)
\curveto(557.79699557,74.84150926)(558.04699532,75.09650901)(558.33699707,75.29651611)
\curveto(558.50699486,75.41650869)(558.68199469,75.51650859)(558.86199707,75.59651611)
\curveto(559.05199432,75.67650843)(559.25699411,75.74650836)(559.47699707,75.80651611)
\curveto(559.54699382,75.81650829)(559.61199376,75.82650828)(559.67199707,75.83651611)
\curveto(559.74199363,75.84650826)(559.81199356,75.86150824)(559.88199707,75.88151611)
\lineto(560.03199707,75.88151611)
\curveto(560.11199326,75.9015082)(560.22699314,75.91150819)(560.37699707,75.91151611)
\curveto(560.53699283,75.91150819)(560.65699271,75.9015082)(560.73699707,75.88151611)
\curveto(560.77699259,75.87150823)(560.83199254,75.86650824)(560.90199707,75.86651611)
\curveto(561.01199236,75.83650827)(561.12199225,75.81150829)(561.23199707,75.79151611)
\curveto(561.34199203,75.78150832)(561.44699192,75.75150835)(561.54699707,75.70151611)
\curveto(561.69699167,75.64150846)(561.83699153,75.57650853)(561.96699707,75.50651611)
\curveto(562.10699126,75.43650867)(562.23699113,75.35650875)(562.35699707,75.26651611)
\curveto(562.41699095,75.21650889)(562.47699089,75.16150894)(562.53699707,75.10151611)
\curveto(562.60699076,75.05150905)(562.69699067,75.03650907)(562.80699707,75.05651611)
\curveto(562.82699054,75.08650902)(562.84199053,75.11150899)(562.85199707,75.13151611)
\curveto(562.8719905,75.15150895)(562.88699048,75.18150892)(562.89699707,75.22151611)
\curveto(562.92699044,75.31150879)(562.93699043,75.42650868)(562.92699707,75.56651611)
\lineto(562.92699707,75.94151611)
\lineto(562.92699707,77.66651611)
\lineto(562.92699707,78.13151611)
\curveto(562.92699044,78.31150579)(562.95199042,78.44150566)(563.00199707,78.52151611)
\curveto(563.04199033,78.59150551)(563.10199027,78.63650547)(563.18199707,78.65651611)
\curveto(563.20199017,78.65650545)(563.22699014,78.65650545)(563.25699707,78.65651611)
\curveto(563.28699008,78.66650544)(563.31199006,78.67150543)(563.33199707,78.67151611)
\curveto(563.4719899,78.68150542)(563.61698975,78.68150542)(563.76699707,78.67151611)
\curveto(563.92698944,78.67150543)(564.03698933,78.63150547)(564.09699707,78.55151611)
\curveto(564.14698922,78.47150563)(564.1719892,78.37150573)(564.17199707,78.25151611)
\lineto(564.17199707,77.87651611)
\lineto(564.17199707,68.81651611)
\moveto(562.95699707,71.65151611)
\curveto(562.97699039,71.7015124)(562.98699038,71.76651234)(562.98699707,71.84651611)
\curveto(562.98699038,71.93651217)(562.97699039,72.0065121)(562.95699707,72.05651611)
\lineto(562.95699707,72.28151611)
\curveto(562.93699043,72.37151173)(562.92199045,72.46151164)(562.91199707,72.55151611)
\curveto(562.90199047,72.65151145)(562.88199049,72.74151136)(562.85199707,72.82151611)
\curveto(562.83199054,72.9015112)(562.81199056,72.97651113)(562.79199707,73.04651611)
\curveto(562.78199059,73.11651099)(562.76199061,73.18651092)(562.73199707,73.25651611)
\curveto(562.61199076,73.55651055)(562.45699091,73.82151028)(562.26699707,74.05151611)
\curveto(562.07699129,74.28150982)(561.83699153,74.46150964)(561.54699707,74.59151611)
\curveto(561.44699192,74.64150946)(561.34199203,74.67650943)(561.23199707,74.69651611)
\curveto(561.13199224,74.72650938)(561.02199235,74.75150935)(560.90199707,74.77151611)
\curveto(560.82199255,74.79150931)(560.73199264,74.8015093)(560.63199707,74.80151611)
\lineto(560.36199707,74.80151611)
\curveto(560.31199306,74.79150931)(560.2669931,74.78150932)(560.22699707,74.77151611)
\lineto(560.09199707,74.77151611)
\curveto(560.01199336,74.75150935)(559.92699344,74.73150937)(559.83699707,74.71151611)
\curveto(559.75699361,74.69150941)(559.67699369,74.66650944)(559.59699707,74.63651611)
\curveto(559.27699409,74.49650961)(559.01699435,74.29150981)(558.81699707,74.02151611)
\curveto(558.62699474,73.76151034)(558.4719949,73.45651065)(558.35199707,73.10651611)
\curveto(558.31199506,72.99651111)(558.28199509,72.88151122)(558.26199707,72.76151611)
\curveto(558.25199512,72.65151145)(558.23699513,72.54151156)(558.21699707,72.43151611)
\curveto(558.21699515,72.39151171)(558.21199516,72.35151175)(558.20199707,72.31151611)
\lineto(558.20199707,72.20651611)
\curveto(558.18199519,72.15651195)(558.1719952,72.101512)(558.17199707,72.04151611)
\curveto(558.18199519,71.98151212)(558.18699518,71.92651218)(558.18699707,71.87651611)
\lineto(558.18699707,71.54651611)
\curveto(558.18699518,71.44651266)(558.19699517,71.35151275)(558.21699707,71.26151611)
\curveto(558.22699514,71.23151287)(558.23199514,71.18151292)(558.23199707,71.11151611)
\curveto(558.25199512,71.04151306)(558.2669951,70.97151313)(558.27699707,70.90151611)
\lineto(558.33699707,70.69151611)
\curveto(558.44699492,70.34151376)(558.59699477,70.04151406)(558.78699707,69.79151611)
\curveto(558.97699439,69.54151456)(559.21699415,69.33651477)(559.50699707,69.17651611)
\curveto(559.59699377,69.12651498)(559.68699368,69.08651502)(559.77699707,69.05651611)
\curveto(559.8669935,69.02651508)(559.9669934,68.99651511)(560.07699707,68.96651611)
\curveto(560.12699324,68.94651516)(560.17699319,68.94151516)(560.22699707,68.95151611)
\curveto(560.28699308,68.96151514)(560.34199303,68.95651515)(560.39199707,68.93651611)
\curveto(560.43199294,68.92651518)(560.4719929,68.92151518)(560.51199707,68.92151611)
\lineto(560.64699707,68.92151611)
\lineto(560.78199707,68.92151611)
\curveto(560.81199256,68.93151517)(560.86199251,68.93651517)(560.93199707,68.93651611)
\curveto(561.01199236,68.95651515)(561.09199228,68.97151513)(561.17199707,68.98151611)
\curveto(561.25199212,69.0015151)(561.32699204,69.02651508)(561.39699707,69.05651611)
\curveto(561.72699164,69.19651491)(561.99199138,69.37151473)(562.19199707,69.58151611)
\curveto(562.40199097,69.8015143)(562.57699079,70.07651403)(562.71699707,70.40651611)
\curveto(562.7669906,70.51651359)(562.80199057,70.62651348)(562.82199707,70.73651611)
\curveto(562.84199053,70.84651326)(562.8669905,70.95651315)(562.89699707,71.06651611)
\curveto(562.91699045,71.106513)(562.92699044,71.14151296)(562.92699707,71.17151611)
\curveto(562.92699044,71.21151289)(562.93199044,71.25151285)(562.94199707,71.29151611)
\curveto(562.95199042,71.35151275)(562.95199042,71.41151269)(562.94199707,71.47151611)
\curveto(562.94199043,71.53151257)(562.94699042,71.59151251)(562.95699707,71.65151611)
}
}
{
\newrgbcolor{curcolor}{0 0 0}
\pscustom[linestyle=none,fillstyle=solid,fillcolor=curcolor]
{
\newpath
\moveto(572.86824707,72.17651611)
\curveto(572.88823939,72.07651203)(572.88823939,71.96151214)(572.86824707,71.83151611)
\curveto(572.85823942,71.71151239)(572.82823945,71.62651248)(572.77824707,71.57651611)
\curveto(572.72823955,71.53651257)(572.65323962,71.5065126)(572.55324707,71.48651611)
\curveto(572.46323981,71.47651263)(572.35823992,71.47151263)(572.23824707,71.47151611)
\lineto(571.87824707,71.47151611)
\curveto(571.75824052,71.48151262)(571.65324062,71.48651262)(571.56324707,71.48651611)
\lineto(567.72324707,71.48651611)
\curveto(567.64324463,71.48651262)(567.56324471,71.48151262)(567.48324707,71.47151611)
\curveto(567.40324487,71.47151263)(567.33824494,71.45651265)(567.28824707,71.42651611)
\curveto(567.24824503,71.4065127)(567.20824507,71.36651274)(567.16824707,71.30651611)
\curveto(567.14824513,71.27651283)(567.12824515,71.23151287)(567.10824707,71.17151611)
\curveto(567.08824519,71.12151298)(567.08824519,71.07151303)(567.10824707,71.02151611)
\curveto(567.11824516,70.97151313)(567.12324515,70.92651318)(567.12324707,70.88651611)
\curveto(567.12324515,70.84651326)(567.12824515,70.8065133)(567.13824707,70.76651611)
\curveto(567.15824512,70.68651342)(567.1782451,70.6015135)(567.19824707,70.51151611)
\curveto(567.21824506,70.43151367)(567.24824503,70.35151375)(567.28824707,70.27151611)
\curveto(567.51824476,69.73151437)(567.89824438,69.34651476)(568.42824707,69.11651611)
\curveto(568.48824379,69.08651502)(568.55324372,69.06151504)(568.62324707,69.04151611)
\lineto(568.83324707,68.98151611)
\curveto(568.86324341,68.97151513)(568.91324336,68.96651514)(568.98324707,68.96651611)
\curveto(569.12324315,68.92651518)(569.30824297,68.9065152)(569.53824707,68.90651611)
\curveto(569.76824251,68.9065152)(569.95324232,68.92651518)(570.09324707,68.96651611)
\curveto(570.23324204,69.0065151)(570.35824192,69.04651506)(570.46824707,69.08651611)
\curveto(570.58824169,69.13651497)(570.69824158,69.19651491)(570.79824707,69.26651611)
\curveto(570.90824137,69.33651477)(571.00324127,69.41651469)(571.08324707,69.50651611)
\curveto(571.16324111,69.6065145)(571.23324104,69.71151439)(571.29324707,69.82151611)
\curveto(571.35324092,69.92151418)(571.40324087,70.02651408)(571.44324707,70.13651611)
\curveto(571.49324078,70.24651386)(571.5732407,70.32651378)(571.68324707,70.37651611)
\curveto(571.72324055,70.39651371)(571.78824049,70.41151369)(571.87824707,70.42151611)
\curveto(571.96824031,70.43151367)(572.05824022,70.43151367)(572.14824707,70.42151611)
\curveto(572.23824004,70.42151368)(572.32323995,70.41651369)(572.40324707,70.40651611)
\curveto(572.48323979,70.39651371)(572.53823974,70.37651373)(572.56824707,70.34651611)
\curveto(572.66823961,70.27651383)(572.69323958,70.16151394)(572.64324707,70.00151611)
\curveto(572.56323971,69.73151437)(572.45823982,69.49151461)(572.32824707,69.28151611)
\curveto(572.12824015,68.96151514)(571.89824038,68.69651541)(571.63824707,68.48651611)
\curveto(571.38824089,68.28651582)(571.06824121,68.12151598)(570.67824707,67.99151611)
\curveto(570.5782417,67.95151615)(570.4782418,67.92651618)(570.37824707,67.91651611)
\curveto(570.278242,67.89651621)(570.1732421,67.87651623)(570.06324707,67.85651611)
\curveto(570.01324226,67.84651626)(569.96324231,67.84151626)(569.91324707,67.84151611)
\curveto(569.8732424,67.84151626)(569.82824245,67.83651627)(569.77824707,67.82651611)
\lineto(569.62824707,67.82651611)
\curveto(569.5782427,67.81651629)(569.51824276,67.81151629)(569.44824707,67.81151611)
\curveto(569.38824289,67.81151629)(569.33824294,67.81651629)(569.29824707,67.82651611)
\lineto(569.16324707,67.82651611)
\curveto(569.11324316,67.83651627)(569.06824321,67.84151626)(569.02824707,67.84151611)
\curveto(568.98824329,67.84151626)(568.94824333,67.84651626)(568.90824707,67.85651611)
\curveto(568.85824342,67.86651624)(568.80324347,67.87651623)(568.74324707,67.88651611)
\curveto(568.68324359,67.88651622)(568.62824365,67.89151621)(568.57824707,67.90151611)
\curveto(568.48824379,67.92151618)(568.39824388,67.94651616)(568.30824707,67.97651611)
\curveto(568.21824406,67.99651611)(568.13324414,68.02151608)(568.05324707,68.05151611)
\curveto(568.01324426,68.07151603)(567.9782443,68.08151602)(567.94824707,68.08151611)
\curveto(567.91824436,68.09151601)(567.88324439,68.106516)(567.84324707,68.12651611)
\curveto(567.69324458,68.19651591)(567.53324474,68.28151582)(567.36324707,68.38151611)
\curveto(567.0732452,68.57151553)(566.82324545,68.8015153)(566.61324707,69.07151611)
\curveto(566.41324586,69.35151475)(566.24324603,69.66151444)(566.10324707,70.00151611)
\curveto(566.05324622,70.11151399)(566.01324626,70.22651388)(565.98324707,70.34651611)
\curveto(565.96324631,70.46651364)(565.93324634,70.58651352)(565.89324707,70.70651611)
\curveto(565.88324639,70.74651336)(565.8782464,70.78151332)(565.87824707,70.81151611)
\curveto(565.8782464,70.84151326)(565.8732464,70.88151322)(565.86324707,70.93151611)
\curveto(565.84324643,71.01151309)(565.82824645,71.09651301)(565.81824707,71.18651611)
\curveto(565.80824647,71.27651283)(565.79324648,71.36651274)(565.77324707,71.45651611)
\lineto(565.77324707,71.66651611)
\curveto(565.76324651,71.7065124)(565.75324652,71.76151234)(565.74324707,71.83151611)
\curveto(565.74324653,71.91151219)(565.74824653,71.97651213)(565.75824707,72.02651611)
\lineto(565.75824707,72.19151611)
\curveto(565.7782465,72.24151186)(565.78324649,72.29151181)(565.77324707,72.34151611)
\curveto(565.7732465,72.4015117)(565.7782465,72.45651165)(565.78824707,72.50651611)
\curveto(565.82824645,72.66651144)(565.85824642,72.82651128)(565.87824707,72.98651611)
\curveto(565.90824637,73.14651096)(565.95324632,73.29651081)(566.01324707,73.43651611)
\curveto(566.06324621,73.54651056)(566.10824617,73.65651045)(566.14824707,73.76651611)
\curveto(566.19824608,73.88651022)(566.25324602,74.0015101)(566.31324707,74.11151611)
\curveto(566.53324574,74.46150964)(566.78324549,74.76150934)(567.06324707,75.01151611)
\curveto(567.34324493,75.27150883)(567.68824459,75.48650862)(568.09824707,75.65651611)
\curveto(568.21824406,75.7065084)(568.33824394,75.74150836)(568.45824707,75.76151611)
\curveto(568.58824369,75.79150831)(568.72324355,75.82150828)(568.86324707,75.85151611)
\curveto(568.91324336,75.86150824)(568.95824332,75.86650824)(568.99824707,75.86651611)
\curveto(569.03824324,75.87650823)(569.08324319,75.88150822)(569.13324707,75.88151611)
\curveto(569.15324312,75.89150821)(569.1782431,75.89150821)(569.20824707,75.88151611)
\curveto(569.23824304,75.87150823)(569.26324301,75.87650823)(569.28324707,75.89651611)
\curveto(569.70324257,75.9065082)(570.06824221,75.86150824)(570.37824707,75.76151611)
\curveto(570.68824159,75.67150843)(570.96824131,75.54650856)(571.21824707,75.38651611)
\curveto(571.26824101,75.36650874)(571.30824097,75.33650877)(571.33824707,75.29651611)
\curveto(571.36824091,75.26650884)(571.40324087,75.24150886)(571.44324707,75.22151611)
\curveto(571.52324075,75.16150894)(571.60324067,75.09150901)(571.68324707,75.01151611)
\curveto(571.7732405,74.93150917)(571.84824043,74.85150925)(571.90824707,74.77151611)
\curveto(572.06824021,74.56150954)(572.20324007,74.36150974)(572.31324707,74.17151611)
\curveto(572.38323989,74.06151004)(572.43823984,73.94151016)(572.47824707,73.81151611)
\curveto(572.51823976,73.68151042)(572.56323971,73.55151055)(572.61324707,73.42151611)
\curveto(572.66323961,73.29151081)(572.69823958,73.15651095)(572.71824707,73.01651611)
\curveto(572.74823953,72.87651123)(572.78323949,72.73651137)(572.82324707,72.59651611)
\curveto(572.83323944,72.52651158)(572.83823944,72.45651165)(572.83824707,72.38651611)
\lineto(572.86824707,72.17651611)
\moveto(571.41324707,72.68651611)
\curveto(571.44324083,72.72651138)(571.46824081,72.77651133)(571.48824707,72.83651611)
\curveto(571.50824077,72.9065112)(571.50824077,72.97651113)(571.48824707,73.04651611)
\curveto(571.42824085,73.26651084)(571.34324093,73.47151063)(571.23324707,73.66151611)
\curveto(571.09324118,73.89151021)(570.93824134,74.08651002)(570.76824707,74.24651611)
\curveto(570.59824168,74.4065097)(570.3782419,74.54150956)(570.10824707,74.65151611)
\curveto(570.03824224,74.67150943)(569.96824231,74.68650942)(569.89824707,74.69651611)
\curveto(569.82824245,74.71650939)(569.75324252,74.73650937)(569.67324707,74.75651611)
\curveto(569.59324268,74.77650933)(569.50824277,74.78650932)(569.41824707,74.78651611)
\lineto(569.16324707,74.78651611)
\curveto(569.13324314,74.76650934)(569.09824318,74.75650935)(569.05824707,74.75651611)
\curveto(569.01824326,74.76650934)(568.98324329,74.76650934)(568.95324707,74.75651611)
\lineto(568.71324707,74.69651611)
\curveto(568.64324363,74.68650942)(568.5732437,74.67150943)(568.50324707,74.65151611)
\curveto(568.21324406,74.53150957)(567.9782443,74.38150972)(567.79824707,74.20151611)
\curveto(567.62824465,74.02151008)(567.4732448,73.79651031)(567.33324707,73.52651611)
\curveto(567.30324497,73.47651063)(567.273245,73.41151069)(567.24324707,73.33151611)
\curveto(567.21324506,73.26151084)(567.18824509,73.18151092)(567.16824707,73.09151611)
\curveto(567.14824513,73.0015111)(567.14324513,72.91651119)(567.15324707,72.83651611)
\curveto(567.16324511,72.75651135)(567.19824508,72.69651141)(567.25824707,72.65651611)
\curveto(567.33824494,72.59651151)(567.4732448,72.56651154)(567.66324707,72.56651611)
\curveto(567.86324441,72.57651153)(568.03324424,72.58151152)(568.17324707,72.58151611)
\lineto(570.45324707,72.58151611)
\curveto(570.60324167,72.58151152)(570.78324149,72.57651153)(570.99324707,72.56651611)
\curveto(571.20324107,72.56651154)(571.34324093,72.6065115)(571.41324707,72.68651611)
}
}
{
\newrgbcolor{curcolor}{0 0 0}
\pscustom[linestyle=none,fillstyle=solid,fillcolor=curcolor]
{
\newpath
\moveto(577.8198877,75.91151611)
\curveto(578.04988291,75.91150819)(578.17988278,75.85150825)(578.2098877,75.73151611)
\curveto(578.23988272,75.62150848)(578.2548827,75.45650865)(578.2548877,75.23651611)
\lineto(578.2548877,74.95151611)
\curveto(578.2548827,74.86150924)(578.22988273,74.78650932)(578.1798877,74.72651611)
\curveto(578.11988284,74.64650946)(578.03488292,74.6015095)(577.9248877,74.59151611)
\curveto(577.81488314,74.59150951)(577.70488325,74.57650953)(577.5948877,74.54651611)
\curveto(577.4548835,74.51650959)(577.31988364,74.48650962)(577.1898877,74.45651611)
\curveto(577.06988389,74.42650968)(576.954884,74.38650972)(576.8448877,74.33651611)
\curveto(576.5548844,74.2065099)(576.31988464,74.02651008)(576.1398877,73.79651611)
\curveto(575.959885,73.57651053)(575.80488515,73.32151078)(575.6748877,73.03151611)
\curveto(575.63488532,72.92151118)(575.60488535,72.8065113)(575.5848877,72.68651611)
\curveto(575.56488539,72.57651153)(575.53988542,72.46151164)(575.5098877,72.34151611)
\curveto(575.49988546,72.29151181)(575.49488546,72.24151186)(575.4948877,72.19151611)
\curveto(575.50488545,72.14151196)(575.50488545,72.09151201)(575.4948877,72.04151611)
\curveto(575.46488549,71.92151218)(575.44988551,71.78151232)(575.4498877,71.62151611)
\curveto(575.4598855,71.47151263)(575.46488549,71.32651278)(575.4648877,71.18651611)
\lineto(575.4648877,69.34151611)
\lineto(575.4648877,68.99651611)
\curveto(575.46488549,68.87651523)(575.4598855,68.76151534)(575.4498877,68.65151611)
\curveto(575.43988552,68.54151556)(575.43488552,68.44651566)(575.4348877,68.36651611)
\curveto(575.44488551,68.28651582)(575.42488553,68.21651589)(575.3748877,68.15651611)
\curveto(575.32488563,68.08651602)(575.24488571,68.04651606)(575.1348877,68.03651611)
\curveto(575.03488592,68.02651608)(574.92488603,68.02151608)(574.8048877,68.02151611)
\lineto(574.5348877,68.02151611)
\curveto(574.48488647,68.04151606)(574.43488652,68.05651605)(574.3848877,68.06651611)
\curveto(574.34488661,68.08651602)(574.31488664,68.11151599)(574.2948877,68.14151611)
\curveto(574.24488671,68.21151589)(574.21488674,68.29651581)(574.2048877,68.39651611)
\lineto(574.2048877,68.72651611)
\lineto(574.2048877,69.88151611)
\lineto(574.2048877,74.03651611)
\lineto(574.2048877,75.07151611)
\lineto(574.2048877,75.37151611)
\curveto(574.21488674,75.47150863)(574.24488671,75.55650855)(574.2948877,75.62651611)
\curveto(574.32488663,75.66650844)(574.37488658,75.69650841)(574.4448877,75.71651611)
\curveto(574.52488643,75.73650837)(574.60988635,75.74650836)(574.6998877,75.74651611)
\curveto(574.78988617,75.75650835)(574.87988608,75.75650835)(574.9698877,75.74651611)
\curveto(575.0598859,75.73650837)(575.12988583,75.72150838)(575.1798877,75.70151611)
\curveto(575.2598857,75.67150843)(575.30988565,75.61150849)(575.3298877,75.52151611)
\curveto(575.3598856,75.44150866)(575.37488558,75.35150875)(575.3748877,75.25151611)
\lineto(575.3748877,74.95151611)
\curveto(575.37488558,74.85150925)(575.39488556,74.76150934)(575.4348877,74.68151611)
\curveto(575.44488551,74.66150944)(575.4548855,74.64650946)(575.4648877,74.63651611)
\lineto(575.5098877,74.59151611)
\curveto(575.61988534,74.59150951)(575.70988525,74.63650947)(575.7798877,74.72651611)
\curveto(575.84988511,74.82650928)(575.90988505,74.9065092)(575.9598877,74.96651611)
\lineto(576.0498877,75.05651611)
\curveto(576.13988482,75.16650894)(576.26488469,75.28150882)(576.4248877,75.40151611)
\curveto(576.58488437,75.52150858)(576.73488422,75.61150849)(576.8748877,75.67151611)
\curveto(576.96488399,75.72150838)(577.0598839,75.75650835)(577.1598877,75.77651611)
\curveto(577.2598837,75.8065083)(577.36488359,75.83650827)(577.4748877,75.86651611)
\curveto(577.53488342,75.87650823)(577.59488336,75.88150822)(577.6548877,75.88151611)
\curveto(577.71488324,75.89150821)(577.76988319,75.9015082)(577.8198877,75.91151611)
}
}
{
\newrgbcolor{curcolor}{0 0 0}
\pscustom[linestyle=none,fillstyle=solid,fillcolor=curcolor]
{
\newpath
\moveto(586.06965332,68.56151611)
\curveto(586.09964549,68.4015157)(586.08464551,68.26651584)(586.02465332,68.15651611)
\curveto(585.96464563,68.05651605)(585.88464571,67.98151612)(585.78465332,67.93151611)
\curveto(585.73464586,67.91151619)(585.67964591,67.9015162)(585.61965332,67.90151611)
\curveto(585.56964602,67.9015162)(585.51464608,67.89151621)(585.45465332,67.87151611)
\curveto(585.23464636,67.82151628)(585.01464658,67.83651627)(584.79465332,67.91651611)
\curveto(584.58464701,67.98651612)(584.43964715,68.07651603)(584.35965332,68.18651611)
\curveto(584.30964728,68.25651585)(584.26464733,68.33651577)(584.22465332,68.42651611)
\curveto(584.18464741,68.52651558)(584.13464746,68.6065155)(584.07465332,68.66651611)
\curveto(584.05464754,68.68651542)(584.02964756,68.7065154)(583.99965332,68.72651611)
\curveto(583.97964761,68.74651536)(583.94964764,68.75151535)(583.90965332,68.74151611)
\curveto(583.79964779,68.71151539)(583.6946479,68.65651545)(583.59465332,68.57651611)
\curveto(583.50464809,68.49651561)(583.41464818,68.42651568)(583.32465332,68.36651611)
\curveto(583.1946484,68.28651582)(583.05464854,68.21151589)(582.90465332,68.14151611)
\curveto(582.75464884,68.08151602)(582.594649,68.02651608)(582.42465332,67.97651611)
\curveto(582.32464927,67.94651616)(582.21464938,67.92651618)(582.09465332,67.91651611)
\curveto(581.98464961,67.9065162)(581.87464972,67.89151621)(581.76465332,67.87151611)
\curveto(581.71464988,67.86151624)(581.66964992,67.85651625)(581.62965332,67.85651611)
\lineto(581.52465332,67.85651611)
\curveto(581.41465018,67.83651627)(581.30965028,67.83651627)(581.20965332,67.85651611)
\lineto(581.07465332,67.85651611)
\curveto(581.02465057,67.86651624)(580.97465062,67.87151623)(580.92465332,67.87151611)
\curveto(580.87465072,67.87151623)(580.82965076,67.88151622)(580.78965332,67.90151611)
\curveto(580.74965084,67.91151619)(580.71465088,67.91651619)(580.68465332,67.91651611)
\curveto(580.66465093,67.9065162)(580.63965095,67.9065162)(580.60965332,67.91651611)
\lineto(580.36965332,67.97651611)
\curveto(580.2896513,67.98651612)(580.21465138,68.0065161)(580.14465332,68.03651611)
\curveto(579.84465175,68.16651594)(579.59965199,68.31151579)(579.40965332,68.47151611)
\curveto(579.22965236,68.64151546)(579.07965251,68.87651523)(578.95965332,69.17651611)
\curveto(578.86965272,69.39651471)(578.82465277,69.66151444)(578.82465332,69.97151611)
\lineto(578.82465332,70.28651611)
\curveto(578.83465276,70.33651377)(578.83965275,70.38651372)(578.83965332,70.43651611)
\lineto(578.86965332,70.61651611)
\lineto(578.98965332,70.94651611)
\curveto(579.02965256,71.05651305)(579.07965251,71.15651295)(579.13965332,71.24651611)
\curveto(579.31965227,71.53651257)(579.56465203,71.75151235)(579.87465332,71.89151611)
\curveto(580.18465141,72.03151207)(580.52465107,72.15651195)(580.89465332,72.26651611)
\curveto(581.03465056,72.3065118)(581.17965041,72.33651177)(581.32965332,72.35651611)
\curveto(581.47965011,72.37651173)(581.62964996,72.4015117)(581.77965332,72.43151611)
\curveto(581.84964974,72.45151165)(581.91464968,72.46151164)(581.97465332,72.46151611)
\curveto(582.04464955,72.46151164)(582.11964947,72.47151163)(582.19965332,72.49151611)
\curveto(582.26964932,72.51151159)(582.33964925,72.52151158)(582.40965332,72.52151611)
\curveto(582.47964911,72.53151157)(582.55464904,72.54651156)(582.63465332,72.56651611)
\curveto(582.88464871,72.62651148)(583.11964847,72.67651143)(583.33965332,72.71651611)
\curveto(583.55964803,72.76651134)(583.73464786,72.88151122)(583.86465332,73.06151611)
\curveto(583.92464767,73.14151096)(583.97464762,73.24151086)(584.01465332,73.36151611)
\curveto(584.05464754,73.49151061)(584.05464754,73.63151047)(584.01465332,73.78151611)
\curveto(583.95464764,74.02151008)(583.86464773,74.21150989)(583.74465332,74.35151611)
\curveto(583.63464796,74.49150961)(583.47464812,74.6015095)(583.26465332,74.68151611)
\curveto(583.14464845,74.73150937)(582.99964859,74.76650934)(582.82965332,74.78651611)
\curveto(582.66964892,74.8065093)(582.49964909,74.81650929)(582.31965332,74.81651611)
\curveto(582.13964945,74.81650929)(581.96464963,74.8065093)(581.79465332,74.78651611)
\curveto(581.62464997,74.76650934)(581.47965011,74.73650937)(581.35965332,74.69651611)
\curveto(581.1896504,74.63650947)(581.02465057,74.55150955)(580.86465332,74.44151611)
\curveto(580.78465081,74.38150972)(580.70965088,74.3015098)(580.63965332,74.20151611)
\curveto(580.57965101,74.11150999)(580.52465107,74.01151009)(580.47465332,73.90151611)
\curveto(580.44465115,73.82151028)(580.41465118,73.73651037)(580.38465332,73.64651611)
\curveto(580.36465123,73.55651055)(580.31965127,73.48651062)(580.24965332,73.43651611)
\curveto(580.20965138,73.4065107)(580.13965145,73.38151072)(580.03965332,73.36151611)
\curveto(579.94965164,73.35151075)(579.85465174,73.34651076)(579.75465332,73.34651611)
\curveto(579.65465194,73.34651076)(579.55465204,73.35151075)(579.45465332,73.36151611)
\curveto(579.36465223,73.38151072)(579.29965229,73.4065107)(579.25965332,73.43651611)
\curveto(579.21965237,73.46651064)(579.1896524,73.51651059)(579.16965332,73.58651611)
\curveto(579.14965244,73.65651045)(579.14965244,73.73151037)(579.16965332,73.81151611)
\curveto(579.19965239,73.94151016)(579.22965236,74.06151004)(579.25965332,74.17151611)
\curveto(579.29965229,74.29150981)(579.34465225,74.4065097)(579.39465332,74.51651611)
\curveto(579.58465201,74.86650924)(579.82465177,75.13650897)(580.11465332,75.32651611)
\curveto(580.40465119,75.52650858)(580.76465083,75.68650842)(581.19465332,75.80651611)
\curveto(581.2946503,75.82650828)(581.3946502,75.84150826)(581.49465332,75.85151611)
\curveto(581.60464999,75.86150824)(581.71464988,75.87650823)(581.82465332,75.89651611)
\curveto(581.86464973,75.9065082)(581.92964966,75.9065082)(582.01965332,75.89651611)
\curveto(582.10964948,75.89650821)(582.16464943,75.9065082)(582.18465332,75.92651611)
\curveto(582.88464871,75.93650817)(583.4946481,75.85650825)(584.01465332,75.68651611)
\curveto(584.53464706,75.51650859)(584.89964669,75.19150891)(585.10965332,74.71151611)
\curveto(585.19964639,74.51150959)(585.24964634,74.27650983)(585.25965332,74.00651611)
\curveto(585.27964631,73.74651036)(585.2896463,73.47151063)(585.28965332,73.18151611)
\lineto(585.28965332,69.86651611)
\curveto(585.2896463,69.72651438)(585.2946463,69.59151451)(585.30465332,69.46151611)
\curveto(585.31464628,69.33151477)(585.34464625,69.22651488)(585.39465332,69.14651611)
\curveto(585.44464615,69.07651503)(585.50964608,69.02651508)(585.58965332,68.99651611)
\curveto(585.67964591,68.95651515)(585.76464583,68.92651518)(585.84465332,68.90651611)
\curveto(585.92464567,68.89651521)(585.98464561,68.85151525)(586.02465332,68.77151611)
\curveto(586.04464555,68.74151536)(586.05464554,68.71151539)(586.05465332,68.68151611)
\curveto(586.05464554,68.65151545)(586.05964553,68.61151549)(586.06965332,68.56151611)
\moveto(583.92465332,70.22651611)
\curveto(583.98464761,70.36651374)(584.01464758,70.52651358)(584.01465332,70.70651611)
\curveto(584.02464757,70.89651321)(584.02964756,71.09151301)(584.02965332,71.29151611)
\curveto(584.02964756,71.4015127)(584.02464757,71.5015126)(584.01465332,71.59151611)
\curveto(584.00464759,71.68151242)(583.96464763,71.75151235)(583.89465332,71.80151611)
\curveto(583.86464773,71.82151228)(583.7946478,71.83151227)(583.68465332,71.83151611)
\curveto(583.66464793,71.81151229)(583.62964796,71.8015123)(583.57965332,71.80151611)
\curveto(583.52964806,71.8015123)(583.48464811,71.79151231)(583.44465332,71.77151611)
\curveto(583.36464823,71.75151235)(583.27464832,71.73151237)(583.17465332,71.71151611)
\lineto(582.87465332,71.65151611)
\curveto(582.84464875,71.65151245)(582.80964878,71.64651246)(582.76965332,71.63651611)
\lineto(582.66465332,71.63651611)
\curveto(582.51464908,71.59651251)(582.34964924,71.57151253)(582.16965332,71.56151611)
\curveto(581.99964959,71.56151254)(581.83964975,71.54151256)(581.68965332,71.50151611)
\curveto(581.60964998,71.48151262)(581.53465006,71.46151264)(581.46465332,71.44151611)
\curveto(581.40465019,71.43151267)(581.33465026,71.41651269)(581.25465332,71.39651611)
\curveto(581.0946505,71.34651276)(580.94465065,71.28151282)(580.80465332,71.20151611)
\curveto(580.66465093,71.13151297)(580.54465105,71.04151306)(580.44465332,70.93151611)
\curveto(580.34465125,70.82151328)(580.26965132,70.68651342)(580.21965332,70.52651611)
\curveto(580.16965142,70.37651373)(580.14965144,70.19151391)(580.15965332,69.97151611)
\curveto(580.15965143,69.87151423)(580.17465142,69.77651433)(580.20465332,69.68651611)
\curveto(580.24465135,69.6065145)(580.2896513,69.53151457)(580.33965332,69.46151611)
\curveto(580.41965117,69.35151475)(580.52465107,69.25651485)(580.65465332,69.17651611)
\curveto(580.78465081,69.106515)(580.92465067,69.04651506)(581.07465332,68.99651611)
\curveto(581.12465047,68.98651512)(581.17465042,68.98151512)(581.22465332,68.98151611)
\curveto(581.27465032,68.98151512)(581.32465027,68.97651513)(581.37465332,68.96651611)
\curveto(581.44465015,68.94651516)(581.52965006,68.93151517)(581.62965332,68.92151611)
\curveto(581.73964985,68.92151518)(581.82964976,68.93151517)(581.89965332,68.95151611)
\curveto(581.95964963,68.97151513)(582.01964957,68.97651513)(582.07965332,68.96651611)
\curveto(582.13964945,68.96651514)(582.19964939,68.97651513)(582.25965332,68.99651611)
\curveto(582.33964925,69.01651509)(582.41464918,69.03151507)(582.48465332,69.04151611)
\curveto(582.56464903,69.05151505)(582.63964895,69.07151503)(582.70965332,69.10151611)
\curveto(582.99964859,69.22151488)(583.24464835,69.36651474)(583.44465332,69.53651611)
\curveto(583.65464794,69.7065144)(583.81464778,69.93651417)(583.92465332,70.22651611)
}
}
{
\newrgbcolor{curcolor}{0 0 0}
\pscustom[linestyle=none,fillstyle=solid,fillcolor=curcolor]
{
\newpath
\moveto(594.20129395,68.81651611)
\lineto(594.20129395,68.42651611)
\curveto(594.20128607,68.3065158)(594.1762861,68.2065159)(594.12629395,68.12651611)
\curveto(594.0762862,68.05651605)(593.99128628,68.01651609)(593.87129395,68.00651611)
\lineto(593.52629395,68.00651611)
\curveto(593.46628681,68.0065161)(593.40628687,68.0015161)(593.34629395,67.99151611)
\curveto(593.29628698,67.99151611)(593.25128702,68.0015161)(593.21129395,68.02151611)
\curveto(593.12128715,68.04151606)(593.06128721,68.08151602)(593.03129395,68.14151611)
\curveto(592.99128728,68.19151591)(592.96628731,68.25151585)(592.95629395,68.32151611)
\curveto(592.95628732,68.39151571)(592.94128733,68.46151564)(592.91129395,68.53151611)
\curveto(592.90128737,68.55151555)(592.88628739,68.56651554)(592.86629395,68.57651611)
\curveto(592.85628742,68.59651551)(592.84128743,68.61651549)(592.82129395,68.63651611)
\curveto(592.72128755,68.64651546)(592.64128763,68.62651548)(592.58129395,68.57651611)
\curveto(592.53128774,68.52651558)(592.4762878,68.47651563)(592.41629395,68.42651611)
\curveto(592.21628806,68.27651583)(592.01628826,68.16151594)(591.81629395,68.08151611)
\curveto(591.63628864,68.0015161)(591.42628885,67.94151616)(591.18629395,67.90151611)
\curveto(590.95628932,67.86151624)(590.71628956,67.84151626)(590.46629395,67.84151611)
\curveto(590.22629005,67.83151627)(589.98629029,67.84651626)(589.74629395,67.88651611)
\curveto(589.50629077,67.91651619)(589.29629098,67.97151613)(589.11629395,68.05151611)
\curveto(588.59629168,68.27151583)(588.1762921,68.56651554)(587.85629395,68.93651611)
\curveto(587.53629274,69.31651479)(587.28629299,69.78651432)(587.10629395,70.34651611)
\curveto(587.06629321,70.43651367)(587.03629324,70.52651358)(587.01629395,70.61651611)
\curveto(587.00629327,70.71651339)(586.98629329,70.81651329)(586.95629395,70.91651611)
\curveto(586.94629333,70.96651314)(586.94129333,71.01651309)(586.94129395,71.06651611)
\curveto(586.94129333,71.11651299)(586.93629334,71.16651294)(586.92629395,71.21651611)
\curveto(586.90629337,71.26651284)(586.89629338,71.31651279)(586.89629395,71.36651611)
\curveto(586.90629337,71.42651268)(586.90629337,71.48151262)(586.89629395,71.53151611)
\lineto(586.89629395,71.68151611)
\curveto(586.8762934,71.73151237)(586.86629341,71.79651231)(586.86629395,71.87651611)
\curveto(586.86629341,71.95651215)(586.8762934,72.02151208)(586.89629395,72.07151611)
\lineto(586.89629395,72.23651611)
\curveto(586.91629336,72.3065118)(586.92129335,72.37651173)(586.91129395,72.44651611)
\curveto(586.91129336,72.52651158)(586.92129335,72.6015115)(586.94129395,72.67151611)
\curveto(586.95129332,72.72151138)(586.95629332,72.76651134)(586.95629395,72.80651611)
\curveto(586.95629332,72.84651126)(586.96129331,72.89151121)(586.97129395,72.94151611)
\curveto(587.00129327,73.04151106)(587.02629325,73.13651097)(587.04629395,73.22651611)
\curveto(587.06629321,73.32651078)(587.09129318,73.42151068)(587.12129395,73.51151611)
\curveto(587.25129302,73.89151021)(587.41629286,74.23150987)(587.61629395,74.53151611)
\curveto(587.82629245,74.84150926)(588.0762922,75.09650901)(588.36629395,75.29651611)
\curveto(588.53629174,75.41650869)(588.71129156,75.51650859)(588.89129395,75.59651611)
\curveto(589.08129119,75.67650843)(589.28629099,75.74650836)(589.50629395,75.80651611)
\curveto(589.5762907,75.81650829)(589.64129063,75.82650828)(589.70129395,75.83651611)
\curveto(589.7712905,75.84650826)(589.84129043,75.86150824)(589.91129395,75.88151611)
\lineto(590.06129395,75.88151611)
\curveto(590.14129013,75.9015082)(590.25629002,75.91150819)(590.40629395,75.91151611)
\curveto(590.56628971,75.91150819)(590.68628959,75.9015082)(590.76629395,75.88151611)
\curveto(590.80628947,75.87150823)(590.86128941,75.86650824)(590.93129395,75.86651611)
\curveto(591.04128923,75.83650827)(591.15128912,75.81150829)(591.26129395,75.79151611)
\curveto(591.3712889,75.78150832)(591.4762888,75.75150835)(591.57629395,75.70151611)
\curveto(591.72628855,75.64150846)(591.86628841,75.57650853)(591.99629395,75.50651611)
\curveto(592.13628814,75.43650867)(592.26628801,75.35650875)(592.38629395,75.26651611)
\curveto(592.44628783,75.21650889)(592.50628777,75.16150894)(592.56629395,75.10151611)
\curveto(592.63628764,75.05150905)(592.72628755,75.03650907)(592.83629395,75.05651611)
\curveto(592.85628742,75.08650902)(592.8712874,75.11150899)(592.88129395,75.13151611)
\curveto(592.90128737,75.15150895)(592.91628736,75.18150892)(592.92629395,75.22151611)
\curveto(592.95628732,75.31150879)(592.96628731,75.42650868)(592.95629395,75.56651611)
\lineto(592.95629395,75.94151611)
\lineto(592.95629395,77.66651611)
\lineto(592.95629395,78.13151611)
\curveto(592.95628732,78.31150579)(592.98128729,78.44150566)(593.03129395,78.52151611)
\curveto(593.0712872,78.59150551)(593.13128714,78.63650547)(593.21129395,78.65651611)
\curveto(593.23128704,78.65650545)(593.25628702,78.65650545)(593.28629395,78.65651611)
\curveto(593.31628696,78.66650544)(593.34128693,78.67150543)(593.36129395,78.67151611)
\curveto(593.50128677,78.68150542)(593.64628663,78.68150542)(593.79629395,78.67151611)
\curveto(593.95628632,78.67150543)(594.06628621,78.63150547)(594.12629395,78.55151611)
\curveto(594.1762861,78.47150563)(594.20128607,78.37150573)(594.20129395,78.25151611)
\lineto(594.20129395,77.87651611)
\lineto(594.20129395,68.81651611)
\moveto(592.98629395,71.65151611)
\curveto(593.00628727,71.7015124)(593.01628726,71.76651234)(593.01629395,71.84651611)
\curveto(593.01628726,71.93651217)(593.00628727,72.0065121)(592.98629395,72.05651611)
\lineto(592.98629395,72.28151611)
\curveto(592.96628731,72.37151173)(592.95128732,72.46151164)(592.94129395,72.55151611)
\curveto(592.93128734,72.65151145)(592.91128736,72.74151136)(592.88129395,72.82151611)
\curveto(592.86128741,72.9015112)(592.84128743,72.97651113)(592.82129395,73.04651611)
\curveto(592.81128746,73.11651099)(592.79128748,73.18651092)(592.76129395,73.25651611)
\curveto(592.64128763,73.55651055)(592.48628779,73.82151028)(592.29629395,74.05151611)
\curveto(592.10628817,74.28150982)(591.86628841,74.46150964)(591.57629395,74.59151611)
\curveto(591.4762888,74.64150946)(591.3712889,74.67650943)(591.26129395,74.69651611)
\curveto(591.16128911,74.72650938)(591.05128922,74.75150935)(590.93129395,74.77151611)
\curveto(590.85128942,74.79150931)(590.76128951,74.8015093)(590.66129395,74.80151611)
\lineto(590.39129395,74.80151611)
\curveto(590.34128993,74.79150931)(590.29628998,74.78150932)(590.25629395,74.77151611)
\lineto(590.12129395,74.77151611)
\curveto(590.04129023,74.75150935)(589.95629032,74.73150937)(589.86629395,74.71151611)
\curveto(589.78629049,74.69150941)(589.70629057,74.66650944)(589.62629395,74.63651611)
\curveto(589.30629097,74.49650961)(589.04629123,74.29150981)(588.84629395,74.02151611)
\curveto(588.65629162,73.76151034)(588.50129177,73.45651065)(588.38129395,73.10651611)
\curveto(588.34129193,72.99651111)(588.31129196,72.88151122)(588.29129395,72.76151611)
\curveto(588.28129199,72.65151145)(588.26629201,72.54151156)(588.24629395,72.43151611)
\curveto(588.24629203,72.39151171)(588.24129203,72.35151175)(588.23129395,72.31151611)
\lineto(588.23129395,72.20651611)
\curveto(588.21129206,72.15651195)(588.20129207,72.101512)(588.20129395,72.04151611)
\curveto(588.21129206,71.98151212)(588.21629206,71.92651218)(588.21629395,71.87651611)
\lineto(588.21629395,71.54651611)
\curveto(588.21629206,71.44651266)(588.22629205,71.35151275)(588.24629395,71.26151611)
\curveto(588.25629202,71.23151287)(588.26129201,71.18151292)(588.26129395,71.11151611)
\curveto(588.28129199,71.04151306)(588.29629198,70.97151313)(588.30629395,70.90151611)
\lineto(588.36629395,70.69151611)
\curveto(588.4762918,70.34151376)(588.62629165,70.04151406)(588.81629395,69.79151611)
\curveto(589.00629127,69.54151456)(589.24629103,69.33651477)(589.53629395,69.17651611)
\curveto(589.62629065,69.12651498)(589.71629056,69.08651502)(589.80629395,69.05651611)
\curveto(589.89629038,69.02651508)(589.99629028,68.99651511)(590.10629395,68.96651611)
\curveto(590.15629012,68.94651516)(590.20629007,68.94151516)(590.25629395,68.95151611)
\curveto(590.31628996,68.96151514)(590.3712899,68.95651515)(590.42129395,68.93651611)
\curveto(590.46128981,68.92651518)(590.50128977,68.92151518)(590.54129395,68.92151611)
\lineto(590.67629395,68.92151611)
\lineto(590.81129395,68.92151611)
\curveto(590.84128943,68.93151517)(590.89128938,68.93651517)(590.96129395,68.93651611)
\curveto(591.04128923,68.95651515)(591.12128915,68.97151513)(591.20129395,68.98151611)
\curveto(591.28128899,69.0015151)(591.35628892,69.02651508)(591.42629395,69.05651611)
\curveto(591.75628852,69.19651491)(592.02128825,69.37151473)(592.22129395,69.58151611)
\curveto(592.43128784,69.8015143)(592.60628767,70.07651403)(592.74629395,70.40651611)
\curveto(592.79628748,70.51651359)(592.83128744,70.62651348)(592.85129395,70.73651611)
\curveto(592.8712874,70.84651326)(592.89628738,70.95651315)(592.92629395,71.06651611)
\curveto(592.94628733,71.106513)(592.95628732,71.14151296)(592.95629395,71.17151611)
\curveto(592.95628732,71.21151289)(592.96128731,71.25151285)(592.97129395,71.29151611)
\curveto(592.98128729,71.35151275)(592.98128729,71.41151269)(592.97129395,71.47151611)
\curveto(592.9712873,71.53151257)(592.9762873,71.59151251)(592.98629395,71.65151611)
}
}
{
\newrgbcolor{curcolor}{0 0 0}
\pscustom[linestyle=none,fillstyle=solid,fillcolor=curcolor]
{
\newpath
\moveto(603.27254395,72.20651611)
\curveto(603.29253589,72.14651196)(603.30253588,72.05151205)(603.30254395,71.92151611)
\curveto(603.30253588,71.8015123)(603.29753588,71.71651239)(603.28754395,71.66651611)
\lineto(603.28754395,71.51651611)
\curveto(603.2775359,71.43651267)(603.26753591,71.36151274)(603.25754395,71.29151611)
\curveto(603.25753592,71.23151287)(603.25253593,71.16151294)(603.24254395,71.08151611)
\curveto(603.22253596,71.02151308)(603.20753597,70.96151314)(603.19754395,70.90151611)
\curveto(603.19753598,70.84151326)(603.18753599,70.78151332)(603.16754395,70.72151611)
\curveto(603.12753605,70.59151351)(603.09253609,70.46151364)(603.06254395,70.33151611)
\curveto(603.03253615,70.2015139)(602.99253619,70.08151402)(602.94254395,69.97151611)
\curveto(602.73253645,69.49151461)(602.45253673,69.08651502)(602.10254395,68.75651611)
\curveto(601.75253743,68.43651567)(601.32253786,68.19151591)(600.81254395,68.02151611)
\curveto(600.70253848,67.98151612)(600.5825386,67.95151615)(600.45254395,67.93151611)
\curveto(600.33253885,67.91151619)(600.20753897,67.89151621)(600.07754395,67.87151611)
\curveto(600.01753916,67.86151624)(599.95253923,67.85651625)(599.88254395,67.85651611)
\curveto(599.82253936,67.84651626)(599.76253942,67.84151626)(599.70254395,67.84151611)
\curveto(599.66253952,67.83151627)(599.60253958,67.82651628)(599.52254395,67.82651611)
\curveto(599.45253973,67.82651628)(599.40253978,67.83151627)(599.37254395,67.84151611)
\curveto(599.33253985,67.85151625)(599.29253989,67.85651625)(599.25254395,67.85651611)
\curveto(599.21253997,67.84651626)(599.17754,67.84651626)(599.14754395,67.85651611)
\lineto(599.05754395,67.85651611)
\lineto(598.69754395,67.90151611)
\curveto(598.55754062,67.94151616)(598.42254076,67.98151612)(598.29254395,68.02151611)
\curveto(598.16254102,68.06151604)(598.03754114,68.106516)(597.91754395,68.15651611)
\curveto(597.46754171,68.35651575)(597.09754208,68.61651549)(596.80754395,68.93651611)
\curveto(596.51754266,69.25651485)(596.2775429,69.64651446)(596.08754395,70.10651611)
\curveto(596.03754314,70.2065139)(595.99754318,70.3065138)(595.96754395,70.40651611)
\curveto(595.94754323,70.5065136)(595.92754325,70.61151349)(595.90754395,70.72151611)
\curveto(595.88754329,70.76151334)(595.8775433,70.79151331)(595.87754395,70.81151611)
\curveto(595.88754329,70.84151326)(595.88754329,70.87651323)(595.87754395,70.91651611)
\curveto(595.85754332,70.99651311)(595.84254334,71.07651303)(595.83254395,71.15651611)
\curveto(595.83254335,71.24651286)(595.82254336,71.33151277)(595.80254395,71.41151611)
\lineto(595.80254395,71.53151611)
\curveto(595.80254338,71.57151253)(595.79754338,71.61651249)(595.78754395,71.66651611)
\curveto(595.7775434,71.71651239)(595.77254341,71.8015123)(595.77254395,71.92151611)
\curveto(595.77254341,72.05151205)(595.7825434,72.14651196)(595.80254395,72.20651611)
\curveto(595.82254336,72.27651183)(595.82754335,72.34651176)(595.81754395,72.41651611)
\curveto(595.80754337,72.48651162)(595.81254337,72.55651155)(595.83254395,72.62651611)
\curveto(595.84254334,72.67651143)(595.84754333,72.71651139)(595.84754395,72.74651611)
\curveto(595.85754332,72.78651132)(595.86754331,72.83151127)(595.87754395,72.88151611)
\curveto(595.90754327,73.0015111)(595.93254325,73.12151098)(595.95254395,73.24151611)
\curveto(595.9825432,73.36151074)(596.02254316,73.47651063)(596.07254395,73.58651611)
\curveto(596.22254296,73.95651015)(596.40254278,74.28650982)(596.61254395,74.57651611)
\curveto(596.83254235,74.87650923)(597.09754208,75.12650898)(597.40754395,75.32651611)
\curveto(597.52754165,75.4065087)(597.65254153,75.47150863)(597.78254395,75.52151611)
\curveto(597.91254127,75.58150852)(598.04754113,75.64150846)(598.18754395,75.70151611)
\curveto(598.30754087,75.75150835)(598.43754074,75.78150832)(598.57754395,75.79151611)
\curveto(598.71754046,75.81150829)(598.85754032,75.84150826)(598.99754395,75.88151611)
\lineto(599.19254395,75.88151611)
\curveto(599.26253992,75.89150821)(599.32753985,75.9015082)(599.38754395,75.91151611)
\curveto(600.2775389,75.92150818)(601.01753816,75.73650837)(601.60754395,75.35651611)
\curveto(602.19753698,74.97650913)(602.62253656,74.48150962)(602.88254395,73.87151611)
\curveto(602.93253625,73.77151033)(602.97253621,73.67151043)(603.00254395,73.57151611)
\curveto(603.03253615,73.47151063)(603.06753611,73.36651074)(603.10754395,73.25651611)
\curveto(603.13753604,73.14651096)(603.16253602,73.02651108)(603.18254395,72.89651611)
\curveto(603.20253598,72.77651133)(603.22753595,72.65151145)(603.25754395,72.52151611)
\curveto(603.26753591,72.47151163)(603.26753591,72.41651169)(603.25754395,72.35651611)
\curveto(603.25753592,72.3065118)(603.26253592,72.25651185)(603.27254395,72.20651611)
\moveto(601.93754395,71.35151611)
\curveto(601.95753722,71.42151268)(601.96253722,71.5015126)(601.95254395,71.59151611)
\lineto(601.95254395,71.84651611)
\curveto(601.95253723,72.23651187)(601.91753726,72.56651154)(601.84754395,72.83651611)
\curveto(601.81753736,72.91651119)(601.79253739,72.99651111)(601.77254395,73.07651611)
\curveto(601.75253743,73.15651095)(601.72753745,73.23151087)(601.69754395,73.30151611)
\curveto(601.41753776,73.95151015)(600.97253821,74.4015097)(600.36254395,74.65151611)
\curveto(600.29253889,74.68150942)(600.21753896,74.7015094)(600.13754395,74.71151611)
\lineto(599.89754395,74.77151611)
\curveto(599.81753936,74.79150931)(599.73253945,74.8015093)(599.64254395,74.80151611)
\lineto(599.37254395,74.80151611)
\lineto(599.10254395,74.75651611)
\curveto(599.00254018,74.73650937)(598.90754027,74.71150939)(598.81754395,74.68151611)
\curveto(598.73754044,74.66150944)(598.65754052,74.63150947)(598.57754395,74.59151611)
\curveto(598.50754067,74.57150953)(598.44254074,74.54150956)(598.38254395,74.50151611)
\curveto(598.32254086,74.46150964)(598.26754091,74.42150968)(598.21754395,74.38151611)
\curveto(597.9775412,74.21150989)(597.7825414,74.0065101)(597.63254395,73.76651611)
\curveto(597.4825417,73.52651058)(597.35254183,73.24651086)(597.24254395,72.92651611)
\curveto(597.21254197,72.82651128)(597.19254199,72.72151138)(597.18254395,72.61151611)
\curveto(597.17254201,72.51151159)(597.15754202,72.4065117)(597.13754395,72.29651611)
\curveto(597.12754205,72.25651185)(597.12254206,72.19151191)(597.12254395,72.10151611)
\curveto(597.11254207,72.07151203)(597.10754207,72.03651207)(597.10754395,71.99651611)
\curveto(597.11754206,71.95651215)(597.12254206,71.91151219)(597.12254395,71.86151611)
\lineto(597.12254395,71.56151611)
\curveto(597.12254206,71.46151264)(597.13254205,71.37151273)(597.15254395,71.29151611)
\lineto(597.18254395,71.11151611)
\curveto(597.20254198,71.01151309)(597.21754196,70.91151319)(597.22754395,70.81151611)
\curveto(597.24754193,70.72151338)(597.2775419,70.63651347)(597.31754395,70.55651611)
\curveto(597.41754176,70.31651379)(597.53254165,70.09151401)(597.66254395,69.88151611)
\curveto(597.80254138,69.67151443)(597.97254121,69.49651461)(598.17254395,69.35651611)
\curveto(598.22254096,69.32651478)(598.26754091,69.3015148)(598.30754395,69.28151611)
\curveto(598.34754083,69.26151484)(598.39254079,69.23651487)(598.44254395,69.20651611)
\curveto(598.52254066,69.15651495)(598.60754057,69.11151499)(598.69754395,69.07151611)
\curveto(598.79754038,69.04151506)(598.90254028,69.01151509)(599.01254395,68.98151611)
\curveto(599.06254012,68.96151514)(599.10754007,68.95151515)(599.14754395,68.95151611)
\curveto(599.19753998,68.96151514)(599.24753993,68.96151514)(599.29754395,68.95151611)
\curveto(599.32753985,68.94151516)(599.38753979,68.93151517)(599.47754395,68.92151611)
\curveto(599.5775396,68.91151519)(599.65253953,68.91651519)(599.70254395,68.93651611)
\curveto(599.74253944,68.94651516)(599.7825394,68.94651516)(599.82254395,68.93651611)
\curveto(599.86253932,68.93651517)(599.90253928,68.94651516)(599.94254395,68.96651611)
\curveto(600.02253916,68.98651512)(600.10253908,69.0015151)(600.18254395,69.01151611)
\curveto(600.26253892,69.03151507)(600.33753884,69.05651505)(600.40754395,69.08651611)
\curveto(600.74753843,69.22651488)(601.02253816,69.42151468)(601.23254395,69.67151611)
\curveto(601.44253774,69.92151418)(601.61753756,70.21651389)(601.75754395,70.55651611)
\curveto(601.80753737,70.67651343)(601.83753734,70.8015133)(601.84754395,70.93151611)
\curveto(601.86753731,71.07151303)(601.89753728,71.21151289)(601.93754395,71.35151611)
}
}
{
\newrgbcolor{curcolor}{0 0 0}
\pscustom[linestyle=none,fillstyle=solid,fillcolor=curcolor]
{
\newpath
\moveto(608.4058252,75.91151611)
\curveto(608.63582041,75.91150819)(608.76582028,75.85150825)(608.7958252,75.73151611)
\curveto(608.82582022,75.62150848)(608.8408202,75.45650865)(608.8408252,75.23651611)
\lineto(608.8408252,74.95151611)
\curveto(608.8408202,74.86150924)(608.81582023,74.78650932)(608.7658252,74.72651611)
\curveto(608.70582034,74.64650946)(608.62082042,74.6015095)(608.5108252,74.59151611)
\curveto(608.40082064,74.59150951)(608.29082075,74.57650953)(608.1808252,74.54651611)
\curveto(608.040821,74.51650959)(607.90582114,74.48650962)(607.7758252,74.45651611)
\curveto(607.65582139,74.42650968)(607.5408215,74.38650972)(607.4308252,74.33651611)
\curveto(607.1408219,74.2065099)(606.90582214,74.02651008)(606.7258252,73.79651611)
\curveto(606.5458225,73.57651053)(606.39082265,73.32151078)(606.2608252,73.03151611)
\curveto(606.22082282,72.92151118)(606.19082285,72.8065113)(606.1708252,72.68651611)
\curveto(606.15082289,72.57651153)(606.12582292,72.46151164)(606.0958252,72.34151611)
\curveto(606.08582296,72.29151181)(606.08082296,72.24151186)(606.0808252,72.19151611)
\curveto(606.09082295,72.14151196)(606.09082295,72.09151201)(606.0808252,72.04151611)
\curveto(606.05082299,71.92151218)(606.03582301,71.78151232)(606.0358252,71.62151611)
\curveto(606.045823,71.47151263)(606.05082299,71.32651278)(606.0508252,71.18651611)
\lineto(606.0508252,69.34151611)
\lineto(606.0508252,68.99651611)
\curveto(606.05082299,68.87651523)(606.045823,68.76151534)(606.0358252,68.65151611)
\curveto(606.02582302,68.54151556)(606.02082302,68.44651566)(606.0208252,68.36651611)
\curveto(606.03082301,68.28651582)(606.01082303,68.21651589)(605.9608252,68.15651611)
\curveto(605.91082313,68.08651602)(605.83082321,68.04651606)(605.7208252,68.03651611)
\curveto(605.62082342,68.02651608)(605.51082353,68.02151608)(605.3908252,68.02151611)
\lineto(605.1208252,68.02151611)
\curveto(605.07082397,68.04151606)(605.02082402,68.05651605)(604.9708252,68.06651611)
\curveto(604.93082411,68.08651602)(604.90082414,68.11151599)(604.8808252,68.14151611)
\curveto(604.83082421,68.21151589)(604.80082424,68.29651581)(604.7908252,68.39651611)
\lineto(604.7908252,68.72651611)
\lineto(604.7908252,69.88151611)
\lineto(604.7908252,74.03651611)
\lineto(604.7908252,75.07151611)
\lineto(604.7908252,75.37151611)
\curveto(604.80082424,75.47150863)(604.83082421,75.55650855)(604.8808252,75.62651611)
\curveto(604.91082413,75.66650844)(604.96082408,75.69650841)(605.0308252,75.71651611)
\curveto(605.11082393,75.73650837)(605.19582385,75.74650836)(605.2858252,75.74651611)
\curveto(605.37582367,75.75650835)(605.46582358,75.75650835)(605.5558252,75.74651611)
\curveto(605.6458234,75.73650837)(605.71582333,75.72150838)(605.7658252,75.70151611)
\curveto(605.8458232,75.67150843)(605.89582315,75.61150849)(605.9158252,75.52151611)
\curveto(605.9458231,75.44150866)(605.96082308,75.35150875)(605.9608252,75.25151611)
\lineto(605.9608252,74.95151611)
\curveto(605.96082308,74.85150925)(605.98082306,74.76150934)(606.0208252,74.68151611)
\curveto(606.03082301,74.66150944)(606.040823,74.64650946)(606.0508252,74.63651611)
\lineto(606.0958252,74.59151611)
\curveto(606.20582284,74.59150951)(606.29582275,74.63650947)(606.3658252,74.72651611)
\curveto(606.43582261,74.82650928)(606.49582255,74.9065092)(606.5458252,74.96651611)
\lineto(606.6358252,75.05651611)
\curveto(606.72582232,75.16650894)(606.85082219,75.28150882)(607.0108252,75.40151611)
\curveto(607.17082187,75.52150858)(607.32082172,75.61150849)(607.4608252,75.67151611)
\curveto(607.55082149,75.72150838)(607.6458214,75.75650835)(607.7458252,75.77651611)
\curveto(607.8458212,75.8065083)(607.95082109,75.83650827)(608.0608252,75.86651611)
\curveto(608.12082092,75.87650823)(608.18082086,75.88150822)(608.2408252,75.88151611)
\curveto(608.30082074,75.89150821)(608.35582069,75.9015082)(608.4058252,75.91151611)
}
}
{
\newrgbcolor{curcolor}{0 0 0}
\pscustom[linestyle=none,fillstyle=solid,fillcolor=curcolor]
{
\newpath
\moveto(718.2480835,68.78651611)
\curveto(718.26807395,68.73651537)(718.29307393,68.67651543)(718.3230835,68.60651611)
\curveto(718.35307387,68.53651557)(718.37307385,68.46151564)(718.3830835,68.38151611)
\curveto(718.40307382,68.31151579)(718.40307382,68.24151586)(718.3830835,68.17151611)
\curveto(718.37307385,68.11151599)(718.33307389,68.06651604)(718.2630835,68.03651611)
\curveto(718.21307401,68.01651609)(718.15307407,68.0065161)(718.0830835,68.00651611)
\lineto(717.8730835,68.00651611)
\lineto(717.4230835,68.00651611)
\curveto(717.27307495,68.0065161)(717.15307507,68.03151607)(717.0630835,68.08151611)
\curveto(716.96307526,68.14151596)(716.88807533,68.24651586)(716.8380835,68.39651611)
\curveto(716.79807542,68.54651556)(716.75307547,68.68151542)(716.7030835,68.80151611)
\curveto(716.59307563,69.06151504)(716.49307573,69.33151477)(716.4030835,69.61151611)
\curveto(716.31307591,69.89151421)(716.21307601,70.16651394)(716.1030835,70.43651611)
\curveto(716.07307615,70.52651358)(716.04307618,70.61151349)(716.0130835,70.69151611)
\curveto(715.99307623,70.77151333)(715.96307626,70.84651326)(715.9230835,70.91651611)
\curveto(715.89307633,70.98651312)(715.84807637,71.04651306)(715.7880835,71.09651611)
\curveto(715.72807649,71.14651296)(715.64807657,71.18651292)(715.5480835,71.21651611)
\curveto(715.49807672,71.23651287)(715.43807678,71.24151286)(715.3680835,71.23151611)
\lineto(715.1730835,71.23151611)
\lineto(712.3380835,71.23151611)
\lineto(712.0380835,71.23151611)
\curveto(711.92808029,71.24151286)(711.8230804,71.24151286)(711.7230835,71.23151611)
\curveto(711.6230806,71.22151288)(711.52808069,71.2065129)(711.4380835,71.18651611)
\curveto(711.35808086,71.16651294)(711.29808092,71.12651298)(711.2580835,71.06651611)
\curveto(711.17808104,70.96651314)(711.1180811,70.85151325)(711.0780835,70.72151611)
\curveto(711.04808117,70.6015135)(711.00808121,70.47651363)(710.9580835,70.34651611)
\curveto(710.85808136,70.11651399)(710.76308146,69.87651423)(710.6730835,69.62651611)
\curveto(710.59308163,69.37651473)(710.50308172,69.13651497)(710.4030835,68.90651611)
\curveto(710.38308184,68.84651526)(710.35808186,68.77651533)(710.3280835,68.69651611)
\curveto(710.30808191,68.62651548)(710.28308194,68.55151555)(710.2530835,68.47151611)
\curveto(710.223082,68.39151571)(710.18808203,68.31651579)(710.1480835,68.24651611)
\curveto(710.1180821,68.18651592)(710.08308214,68.14151596)(710.0430835,68.11151611)
\curveto(709.96308226,68.05151605)(709.85308237,68.01651609)(709.7130835,68.00651611)
\lineto(709.2930835,68.00651611)
\lineto(709.0530835,68.00651611)
\curveto(708.98308324,68.01651609)(708.9230833,68.04151606)(708.8730835,68.08151611)
\curveto(708.8230834,68.11151599)(708.79308343,68.15651595)(708.7830835,68.21651611)
\curveto(708.78308344,68.27651583)(708.78808343,68.33651577)(708.7980835,68.39651611)
\curveto(708.8180834,68.46651564)(708.83808338,68.53151557)(708.8580835,68.59151611)
\curveto(708.88808333,68.66151544)(708.91308331,68.71151539)(708.9330835,68.74151611)
\curveto(709.07308315,69.06151504)(709.19808302,69.37651473)(709.3080835,69.68651611)
\curveto(709.4180828,70.0065141)(709.53808268,70.32651378)(709.6680835,70.64651611)
\curveto(709.75808246,70.86651324)(709.84308238,71.08151302)(709.9230835,71.29151611)
\curveto(710.00308222,71.51151259)(710.08808213,71.73151237)(710.1780835,71.95151611)
\curveto(710.47808174,72.67151143)(710.76308146,73.39651071)(711.0330835,74.12651611)
\curveto(711.30308092,74.86650924)(711.58808063,75.6015085)(711.8880835,76.33151611)
\curveto(711.99808022,76.59150751)(712.09808012,76.85650725)(712.1880835,77.12651611)
\curveto(712.28807993,77.39650671)(712.39307983,77.66150644)(712.5030835,77.92151611)
\curveto(712.55307967,78.03150607)(712.59807962,78.15150595)(712.6380835,78.28151611)
\curveto(712.68807953,78.42150568)(712.75807946,78.52150558)(712.8480835,78.58151611)
\curveto(712.88807933,78.62150548)(712.95307927,78.65150545)(713.0430835,78.67151611)
\curveto(713.06307916,78.68150542)(713.08307914,78.68150542)(713.1030835,78.67151611)
\curveto(713.13307909,78.67150543)(713.15807906,78.67650543)(713.1780835,78.68651611)
\curveto(713.35807886,78.68650542)(713.56807865,78.68650542)(713.8080835,78.68651611)
\curveto(714.04807817,78.69650541)(714.223078,78.66150544)(714.3330835,78.58151611)
\curveto(714.41307781,78.52150558)(714.47307775,78.42150568)(714.5130835,78.28151611)
\curveto(714.56307766,78.15150595)(714.61307761,78.03150607)(714.6630835,77.92151611)
\curveto(714.76307746,77.69150641)(714.85307737,77.46150664)(714.9330835,77.23151611)
\curveto(715.01307721,77.0015071)(715.10307712,76.77150733)(715.2030835,76.54151611)
\curveto(715.28307694,76.34150776)(715.35807686,76.13650797)(715.4280835,75.92651611)
\curveto(715.50807671,75.71650839)(715.59307663,75.51150859)(715.6830835,75.31151611)
\curveto(715.98307624,74.58150952)(716.26807595,73.84151026)(716.5380835,73.09151611)
\curveto(716.8180754,72.35151175)(717.11307511,71.61651249)(717.4230835,70.88651611)
\curveto(717.46307476,70.79651331)(717.49307473,70.71151339)(717.5130835,70.63151611)
\curveto(717.54307468,70.55151355)(717.57307465,70.46651364)(717.6030835,70.37651611)
\curveto(717.71307451,70.11651399)(717.8180744,69.85151425)(717.9180835,69.58151611)
\curveto(718.02807419,69.31151479)(718.13807408,69.04651506)(718.2480835,68.78651611)
\moveto(715.0380835,72.43151611)
\curveto(715.12807709,72.46151164)(715.18307704,72.51151159)(715.2030835,72.58151611)
\curveto(715.23307699,72.65151145)(715.23807698,72.72651138)(715.2180835,72.80651611)
\curveto(715.20807701,72.89651121)(715.18307704,72.98151112)(715.1430835,73.06151611)
\curveto(715.11307711,73.15151095)(715.08307714,73.22651088)(715.0530835,73.28651611)
\curveto(715.03307719,73.32651078)(715.0230772,73.36151074)(715.0230835,73.39151611)
\curveto(715.0230772,73.42151068)(715.01307721,73.45651065)(714.9930835,73.49651611)
\lineto(714.9030835,73.73651611)
\curveto(714.88307734,73.82651028)(714.85307737,73.91651019)(714.8130835,74.00651611)
\curveto(714.66307756,74.36650974)(714.52807769,74.73150937)(714.4080835,75.10151611)
\curveto(714.29807792,75.48150862)(714.16807805,75.85150825)(714.0180835,76.21151611)
\curveto(713.96807825,76.32150778)(713.9230783,76.43150767)(713.8830835,76.54151611)
\curveto(713.85307837,76.65150745)(713.81307841,76.75650735)(713.7630835,76.85651611)
\curveto(713.74307848,76.9065072)(713.7180785,76.95150715)(713.6880835,76.99151611)
\curveto(713.66807855,77.04150706)(713.6180786,77.06650704)(713.5380835,77.06651611)
\curveto(713.5180787,77.04650706)(713.49807872,77.03150707)(713.4780835,77.02151611)
\curveto(713.45807876,77.01150709)(713.43807878,76.99650711)(713.4180835,76.97651611)
\curveto(713.37807884,76.92650718)(713.34807887,76.87150723)(713.3280835,76.81151611)
\curveto(713.30807891,76.76150734)(713.28807893,76.7065074)(713.2680835,76.64651611)
\curveto(713.218079,76.53650757)(713.17807904,76.42650768)(713.1480835,76.31651611)
\curveto(713.1180791,76.2065079)(713.07807914,76.09650801)(713.0280835,75.98651611)
\curveto(712.85807936,75.59650851)(712.70807951,75.2015089)(712.5780835,74.80151611)
\curveto(712.45807976,74.4015097)(712.3180799,74.01151009)(712.1580835,73.63151611)
\lineto(712.0980835,73.48151611)
\curveto(712.08808013,73.43151067)(712.07308015,73.38151072)(712.0530835,73.33151611)
\lineto(711.9630835,73.09151611)
\curveto(711.93308029,73.01151109)(711.90808031,72.93151117)(711.8880835,72.85151611)
\curveto(711.86808035,72.8015113)(711.85808036,72.74651136)(711.8580835,72.68651611)
\curveto(711.86808035,72.62651148)(711.88308034,72.57651153)(711.9030835,72.53651611)
\curveto(711.95308027,72.45651165)(712.05808016,72.41151169)(712.2180835,72.40151611)
\lineto(712.6680835,72.40151611)
\lineto(714.2730835,72.40151611)
\curveto(714.38307784,72.4015117)(714.5180777,72.39651171)(714.6780835,72.38651611)
\curveto(714.83807738,72.38651172)(714.95807726,72.4015117)(715.0380835,72.43151611)
}
}
{
\newrgbcolor{curcolor}{0 0 0}
\pscustom[linestyle=none,fillstyle=solid,fillcolor=curcolor]
{
\newpath
\moveto(726.324646,68.81651611)
\lineto(726.324646,68.42651611)
\curveto(726.32463812,68.3065158)(726.29963815,68.2065159)(726.249646,68.12651611)
\curveto(726.19963825,68.05651605)(726.11463833,68.01651609)(725.994646,68.00651611)
\lineto(725.649646,68.00651611)
\curveto(725.58963886,68.0065161)(725.52963892,68.0015161)(725.469646,67.99151611)
\curveto(725.41963903,67.99151611)(725.37463907,68.0015161)(725.334646,68.02151611)
\curveto(725.2446392,68.04151606)(725.18463926,68.08151602)(725.154646,68.14151611)
\curveto(725.11463933,68.19151591)(725.08963936,68.25151585)(725.079646,68.32151611)
\curveto(725.07963937,68.39151571)(725.06463938,68.46151564)(725.034646,68.53151611)
\curveto(725.02463942,68.55151555)(725.00963944,68.56651554)(724.989646,68.57651611)
\curveto(724.97963947,68.59651551)(724.96463948,68.61651549)(724.944646,68.63651611)
\curveto(724.8446396,68.64651546)(724.76463968,68.62651548)(724.704646,68.57651611)
\curveto(724.65463979,68.52651558)(724.59963985,68.47651563)(724.539646,68.42651611)
\curveto(724.33964011,68.27651583)(724.13964031,68.16151594)(723.939646,68.08151611)
\curveto(723.75964069,68.0015161)(723.5496409,67.94151616)(723.309646,67.90151611)
\curveto(723.07964137,67.86151624)(722.83964161,67.84151626)(722.589646,67.84151611)
\curveto(722.3496421,67.83151627)(722.10964234,67.84651626)(721.869646,67.88651611)
\curveto(721.62964282,67.91651619)(721.41964303,67.97151613)(721.239646,68.05151611)
\curveto(720.71964373,68.27151583)(720.29964415,68.56651554)(719.979646,68.93651611)
\curveto(719.65964479,69.31651479)(719.40964504,69.78651432)(719.229646,70.34651611)
\curveto(719.18964526,70.43651367)(719.15964529,70.52651358)(719.139646,70.61651611)
\curveto(719.12964532,70.71651339)(719.10964534,70.81651329)(719.079646,70.91651611)
\curveto(719.06964538,70.96651314)(719.06464538,71.01651309)(719.064646,71.06651611)
\curveto(719.06464538,71.11651299)(719.05964539,71.16651294)(719.049646,71.21651611)
\curveto(719.02964542,71.26651284)(719.01964543,71.31651279)(719.019646,71.36651611)
\curveto(719.02964542,71.42651268)(719.02964542,71.48151262)(719.019646,71.53151611)
\lineto(719.019646,71.68151611)
\curveto(718.99964545,71.73151237)(718.98964546,71.79651231)(718.989646,71.87651611)
\curveto(718.98964546,71.95651215)(718.99964545,72.02151208)(719.019646,72.07151611)
\lineto(719.019646,72.23651611)
\curveto(719.03964541,72.3065118)(719.0446454,72.37651173)(719.034646,72.44651611)
\curveto(719.03464541,72.52651158)(719.0446454,72.6015115)(719.064646,72.67151611)
\curveto(719.07464537,72.72151138)(719.07964537,72.76651134)(719.079646,72.80651611)
\curveto(719.07964537,72.84651126)(719.08464536,72.89151121)(719.094646,72.94151611)
\curveto(719.12464532,73.04151106)(719.1496453,73.13651097)(719.169646,73.22651611)
\curveto(719.18964526,73.32651078)(719.21464523,73.42151068)(719.244646,73.51151611)
\curveto(719.37464507,73.89151021)(719.53964491,74.23150987)(719.739646,74.53151611)
\curveto(719.9496445,74.84150926)(720.19964425,75.09650901)(720.489646,75.29651611)
\curveto(720.65964379,75.41650869)(720.83464361,75.51650859)(721.014646,75.59651611)
\curveto(721.20464324,75.67650843)(721.40964304,75.74650836)(721.629646,75.80651611)
\curveto(721.69964275,75.81650829)(721.76464268,75.82650828)(721.824646,75.83651611)
\curveto(721.89464255,75.84650826)(721.96464248,75.86150824)(722.034646,75.88151611)
\lineto(722.184646,75.88151611)
\curveto(722.26464218,75.9015082)(722.37964207,75.91150819)(722.529646,75.91151611)
\curveto(722.68964176,75.91150819)(722.80964164,75.9015082)(722.889646,75.88151611)
\curveto(722.92964152,75.87150823)(722.98464146,75.86650824)(723.054646,75.86651611)
\curveto(723.16464128,75.83650827)(723.27464117,75.81150829)(723.384646,75.79151611)
\curveto(723.49464095,75.78150832)(723.59964085,75.75150835)(723.699646,75.70151611)
\curveto(723.8496406,75.64150846)(723.98964046,75.57650853)(724.119646,75.50651611)
\curveto(724.25964019,75.43650867)(724.38964006,75.35650875)(724.509646,75.26651611)
\curveto(724.56963988,75.21650889)(724.62963982,75.16150894)(724.689646,75.10151611)
\curveto(724.75963969,75.05150905)(724.8496396,75.03650907)(724.959646,75.05651611)
\curveto(724.97963947,75.08650902)(724.99463945,75.11150899)(725.004646,75.13151611)
\curveto(725.02463942,75.15150895)(725.03963941,75.18150892)(725.049646,75.22151611)
\curveto(725.07963937,75.31150879)(725.08963936,75.42650868)(725.079646,75.56651611)
\lineto(725.079646,75.94151611)
\lineto(725.079646,77.66651611)
\lineto(725.079646,78.13151611)
\curveto(725.07963937,78.31150579)(725.10463934,78.44150566)(725.154646,78.52151611)
\curveto(725.19463925,78.59150551)(725.25463919,78.63650547)(725.334646,78.65651611)
\curveto(725.35463909,78.65650545)(725.37963907,78.65650545)(725.409646,78.65651611)
\curveto(725.43963901,78.66650544)(725.46463898,78.67150543)(725.484646,78.67151611)
\curveto(725.62463882,78.68150542)(725.76963868,78.68150542)(725.919646,78.67151611)
\curveto(726.07963837,78.67150543)(726.18963826,78.63150547)(726.249646,78.55151611)
\curveto(726.29963815,78.47150563)(726.32463812,78.37150573)(726.324646,78.25151611)
\lineto(726.324646,77.87651611)
\lineto(726.324646,68.81651611)
\moveto(725.109646,71.65151611)
\curveto(725.12963932,71.7015124)(725.13963931,71.76651234)(725.139646,71.84651611)
\curveto(725.13963931,71.93651217)(725.12963932,72.0065121)(725.109646,72.05651611)
\lineto(725.109646,72.28151611)
\curveto(725.08963936,72.37151173)(725.07463937,72.46151164)(725.064646,72.55151611)
\curveto(725.05463939,72.65151145)(725.03463941,72.74151136)(725.004646,72.82151611)
\curveto(724.98463946,72.9015112)(724.96463948,72.97651113)(724.944646,73.04651611)
\curveto(724.93463951,73.11651099)(724.91463953,73.18651092)(724.884646,73.25651611)
\curveto(724.76463968,73.55651055)(724.60963984,73.82151028)(724.419646,74.05151611)
\curveto(724.22964022,74.28150982)(723.98964046,74.46150964)(723.699646,74.59151611)
\curveto(723.59964085,74.64150946)(723.49464095,74.67650943)(723.384646,74.69651611)
\curveto(723.28464116,74.72650938)(723.17464127,74.75150935)(723.054646,74.77151611)
\curveto(722.97464147,74.79150931)(722.88464156,74.8015093)(722.784646,74.80151611)
\lineto(722.514646,74.80151611)
\curveto(722.46464198,74.79150931)(722.41964203,74.78150932)(722.379646,74.77151611)
\lineto(722.244646,74.77151611)
\curveto(722.16464228,74.75150935)(722.07964237,74.73150937)(721.989646,74.71151611)
\curveto(721.90964254,74.69150941)(721.82964262,74.66650944)(721.749646,74.63651611)
\curveto(721.42964302,74.49650961)(721.16964328,74.29150981)(720.969646,74.02151611)
\curveto(720.77964367,73.76151034)(720.62464382,73.45651065)(720.504646,73.10651611)
\curveto(720.46464398,72.99651111)(720.43464401,72.88151122)(720.414646,72.76151611)
\curveto(720.40464404,72.65151145)(720.38964406,72.54151156)(720.369646,72.43151611)
\curveto(720.36964408,72.39151171)(720.36464408,72.35151175)(720.354646,72.31151611)
\lineto(720.354646,72.20651611)
\curveto(720.33464411,72.15651195)(720.32464412,72.101512)(720.324646,72.04151611)
\curveto(720.33464411,71.98151212)(720.33964411,71.92651218)(720.339646,71.87651611)
\lineto(720.339646,71.54651611)
\curveto(720.33964411,71.44651266)(720.3496441,71.35151275)(720.369646,71.26151611)
\curveto(720.37964407,71.23151287)(720.38464406,71.18151292)(720.384646,71.11151611)
\curveto(720.40464404,71.04151306)(720.41964403,70.97151313)(720.429646,70.90151611)
\lineto(720.489646,70.69151611)
\curveto(720.59964385,70.34151376)(720.7496437,70.04151406)(720.939646,69.79151611)
\curveto(721.12964332,69.54151456)(721.36964308,69.33651477)(721.659646,69.17651611)
\curveto(721.7496427,69.12651498)(721.83964261,69.08651502)(721.929646,69.05651611)
\curveto(722.01964243,69.02651508)(722.11964233,68.99651511)(722.229646,68.96651611)
\curveto(722.27964217,68.94651516)(722.32964212,68.94151516)(722.379646,68.95151611)
\curveto(722.43964201,68.96151514)(722.49464195,68.95651515)(722.544646,68.93651611)
\curveto(722.58464186,68.92651518)(722.62464182,68.92151518)(722.664646,68.92151611)
\lineto(722.799646,68.92151611)
\lineto(722.934646,68.92151611)
\curveto(722.96464148,68.93151517)(723.01464143,68.93651517)(723.084646,68.93651611)
\curveto(723.16464128,68.95651515)(723.2446412,68.97151513)(723.324646,68.98151611)
\curveto(723.40464104,69.0015151)(723.47964097,69.02651508)(723.549646,69.05651611)
\curveto(723.87964057,69.19651491)(724.1446403,69.37151473)(724.344646,69.58151611)
\curveto(724.55463989,69.8015143)(724.72963972,70.07651403)(724.869646,70.40651611)
\curveto(724.91963953,70.51651359)(724.95463949,70.62651348)(724.974646,70.73651611)
\curveto(724.99463945,70.84651326)(725.01963943,70.95651315)(725.049646,71.06651611)
\curveto(725.06963938,71.106513)(725.07963937,71.14151296)(725.079646,71.17151611)
\curveto(725.07963937,71.21151289)(725.08463936,71.25151285)(725.094646,71.29151611)
\curveto(725.10463934,71.35151275)(725.10463934,71.41151269)(725.094646,71.47151611)
\curveto(725.09463935,71.53151257)(725.09963935,71.59151251)(725.109646,71.65151611)
}
}
{
\newrgbcolor{curcolor}{0 0 0}
\pscustom[linestyle=none,fillstyle=solid,fillcolor=curcolor]
{
\newpath
\moveto(731.960896,75.91151611)
\curveto(732.34089101,75.92150818)(732.66089069,75.88150822)(732.920896,75.79151611)
\curveto(733.19089016,75.7015084)(733.43588992,75.57150853)(733.655896,75.40151611)
\curveto(733.73588962,75.35150875)(733.80088955,75.28150882)(733.850896,75.19151611)
\curveto(733.91088944,75.11150899)(733.97588938,75.03650907)(734.045896,74.96651611)
\curveto(734.06588929,74.94650916)(734.09588926,74.92150918)(734.135896,74.89151611)
\curveto(734.17588918,74.86150924)(734.22588913,74.85150925)(734.285896,74.86151611)
\curveto(734.38588897,74.89150921)(734.47088888,74.95150915)(734.540896,75.04151611)
\curveto(734.62088873,75.14150896)(734.70088865,75.21650889)(734.780896,75.26651611)
\curveto(734.92088843,75.37650873)(735.06588829,75.47150863)(735.215896,75.55151611)
\curveto(735.36588799,75.64150846)(735.53088782,75.71650839)(735.710896,75.77651611)
\curveto(735.79088756,75.8065083)(735.87588748,75.82650828)(735.965896,75.83651611)
\curveto(736.06588729,75.85650825)(736.16088719,75.87650823)(736.250896,75.89651611)
\curveto(736.30088705,75.9065082)(736.34588701,75.91150819)(736.385896,75.91151611)
\lineto(736.535896,75.91151611)
\curveto(736.58588677,75.93150817)(736.6558867,75.93650817)(736.745896,75.92651611)
\curveto(736.83588652,75.92650818)(736.90088645,75.92150818)(736.940896,75.91151611)
\curveto(736.99088636,75.9015082)(737.06588629,75.89650821)(737.165896,75.89651611)
\curveto(737.2558861,75.87650823)(737.34088601,75.85650825)(737.420896,75.83651611)
\curveto(737.51088584,75.82650828)(737.59588576,75.8065083)(737.675896,75.77651611)
\curveto(737.72588563,75.75650835)(737.77088558,75.74150836)(737.810896,75.73151611)
\curveto(737.86088549,75.73150837)(737.91088544,75.72150838)(737.960896,75.70151611)
\curveto(738.46088489,75.48150862)(738.80588455,75.14150896)(738.995896,74.68151611)
\curveto(739.03588432,74.6015095)(739.06588429,74.51150959)(739.085896,74.41151611)
\curveto(739.10588425,74.32150978)(739.12588423,74.22150988)(739.145896,74.11151611)
\curveto(739.16588419,74.08151002)(739.17088418,74.04651006)(739.160896,74.00651611)
\curveto(739.16088419,73.97651013)(739.16588419,73.94651016)(739.175896,73.91651611)
\lineto(739.175896,73.78151611)
\curveto(739.18588417,73.74151036)(739.18588417,73.69651041)(739.175896,73.64651611)
\curveto(739.17588418,73.59651051)(739.17588418,73.54651056)(739.175896,73.49651611)
\lineto(739.175896,72.91151611)
\lineto(739.175896,71.95151611)
\lineto(739.175896,69.10151611)
\curveto(739.17588418,68.94151516)(739.17588418,68.75151535)(739.175896,68.53151611)
\curveto(739.18588417,68.31151579)(739.14588421,68.16651594)(739.055896,68.09651611)
\curveto(739.01588434,68.06651604)(738.9508844,68.04151606)(738.860896,68.02151611)
\curveto(738.77088458,68.01151609)(738.67588468,68.0065161)(738.575896,68.00651611)
\curveto(738.47588488,68.0065161)(738.37588498,68.01151609)(738.275896,68.02151611)
\curveto(738.18588517,68.03151607)(738.12088523,68.05151605)(738.080896,68.08151611)
\curveto(738.02088533,68.11151599)(737.98088537,68.17151593)(737.960896,68.26151611)
\curveto(737.94088541,68.32151578)(737.93588542,68.38151572)(737.945896,68.44151611)
\curveto(737.9558854,68.51151559)(737.9508854,68.57651553)(737.930896,68.63651611)
\curveto(737.92088543,68.68651542)(737.91588544,68.74151536)(737.915896,68.80151611)
\curveto(737.92588543,68.87151523)(737.93088542,68.93651517)(737.930896,68.99651611)
\lineto(737.930896,69.67151611)
\lineto(737.930896,72.53651611)
\curveto(737.93088542,72.86651124)(737.92088543,73.17651093)(737.900896,73.46651611)
\curveto(737.89088546,73.76651034)(737.82088553,74.01651009)(737.690896,74.21651611)
\curveto(737.54088581,74.45650965)(737.31088604,74.63150947)(737.000896,74.74151611)
\curveto(736.94088641,74.76150934)(736.87588648,74.77150933)(736.805896,74.77151611)
\curveto(736.74588661,74.78150932)(736.68088667,74.79650931)(736.610896,74.81651611)
\curveto(736.57088678,74.82650928)(736.50588685,74.82650928)(736.415896,74.81651611)
\curveto(736.32588703,74.81650929)(736.26588709,74.81150929)(736.235896,74.80151611)
\curveto(736.18588717,74.79150931)(736.13588722,74.78650932)(736.085896,74.78651611)
\curveto(736.03588732,74.79650931)(735.98588737,74.79150931)(735.935896,74.77151611)
\curveto(735.79588756,74.74150936)(735.66088769,74.7015094)(735.530896,74.65151611)
\curveto(735.01088834,74.43150967)(734.66088869,74.04651006)(734.480896,73.49651611)
\curveto(734.43088892,73.32651078)(734.40088895,73.13151097)(734.390896,72.91151611)
\lineto(734.390896,72.23651611)
\lineto(734.390896,70.27151611)
\lineto(734.390896,68.81651611)
\lineto(734.390896,68.44151611)
\curveto(734.39088896,68.32151578)(734.36588899,68.22651588)(734.315896,68.15651611)
\curveto(734.26588909,68.07651603)(734.18088917,68.03151607)(734.060896,68.02151611)
\curveto(733.94088941,68.01151609)(733.81588954,68.0065161)(733.685896,68.00651611)
\curveto(733.51588984,68.0065161)(733.39088996,68.02651608)(733.310896,68.06651611)
\curveto(733.22089013,68.11651599)(733.16589019,68.19651591)(733.145896,68.30651611)
\curveto(733.13589022,68.42651568)(733.13089022,68.55651555)(733.130896,68.69651611)
\lineto(733.130896,70.12151611)
\lineto(733.130896,72.59651611)
\curveto(733.13089022,72.91651119)(733.12089023,73.21151089)(733.100896,73.48151611)
\curveto(733.08089027,73.76151034)(733.01089034,74.0015101)(732.890896,74.20151611)
\curveto(732.78089057,74.38150972)(732.6558907,74.51150959)(732.515896,74.59151611)
\curveto(732.37589098,74.68150942)(732.18589117,74.75150935)(731.945896,74.80151611)
\curveto(731.90589145,74.81150929)(731.86089149,74.81650929)(731.810896,74.81651611)
\lineto(731.675896,74.81651611)
\curveto(731.4558919,74.81650929)(731.26089209,74.79150931)(731.090896,74.74151611)
\curveto(730.93089242,74.69150941)(730.78589257,74.62650948)(730.655896,74.54651611)
\curveto(730.14589321,74.23650987)(729.80589355,73.77151033)(729.635896,73.15151611)
\curveto(729.59589376,73.02151108)(729.57589378,72.87151123)(729.575896,72.70151611)
\curveto(729.58589377,72.54151156)(729.59089376,72.38151172)(729.590896,72.22151611)
\lineto(729.590896,70.52651611)
\lineto(729.590896,68.87651611)
\lineto(729.590896,68.47151611)
\curveto(729.59089376,68.33151577)(729.56089379,68.22151588)(729.500896,68.14151611)
\curveto(729.4508939,68.07151603)(729.37589398,68.03151607)(729.275896,68.02151611)
\curveto(729.17589418,68.01151609)(729.07089428,68.0065161)(728.960896,68.00651611)
\lineto(728.735896,68.00651611)
\curveto(728.67589468,68.02651608)(728.61589474,68.04151606)(728.555896,68.05151611)
\curveto(728.50589485,68.06151604)(728.46089489,68.09151601)(728.420896,68.14151611)
\curveto(728.37089498,68.2015159)(728.34589501,68.27651583)(728.345896,68.36651611)
\lineto(728.345896,68.68151611)
\lineto(728.345896,69.65651611)
\lineto(728.345896,73.94651611)
\lineto(728.345896,75.05651611)
\lineto(728.345896,75.34151611)
\curveto(728.34589501,75.44150866)(728.36589499,75.52150858)(728.405896,75.58151611)
\curveto(728.43589492,75.64150846)(728.48089487,75.68150842)(728.540896,75.70151611)
\curveto(728.62089473,75.73150837)(728.74589461,75.74650836)(728.915896,75.74651611)
\curveto(729.09589426,75.74650836)(729.22589413,75.73150837)(729.305896,75.70151611)
\curveto(729.38589397,75.66150844)(729.44089391,75.61150849)(729.470896,75.55151611)
\curveto(729.49089386,75.5015086)(729.50089385,75.44150866)(729.500896,75.37151611)
\curveto(729.51089384,75.3015088)(729.52089383,75.23650887)(729.530896,75.17651611)
\curveto(729.54089381,75.11650899)(729.56089379,75.06650904)(729.590896,75.02651611)
\curveto(729.62089373,74.98650912)(729.67089368,74.96650914)(729.740896,74.96651611)
\curveto(729.76089359,74.98650912)(729.78089357,74.99650911)(729.800896,74.99651611)
\curveto(729.83089352,74.99650911)(729.8558935,75.0065091)(729.875896,75.02651611)
\curveto(729.93589342,75.07650903)(729.99089336,75.12650898)(730.040896,75.17651611)
\lineto(730.220896,75.32651611)
\curveto(730.44089291,75.48650862)(730.69089266,75.62650848)(730.970896,75.74651611)
\curveto(731.07089228,75.78650832)(731.17089218,75.81150829)(731.270896,75.82151611)
\curveto(731.37089198,75.84150826)(731.47589188,75.86650824)(731.585896,75.89651611)
\lineto(731.765896,75.89651611)
\curveto(731.83589152,75.9065082)(731.90089145,75.91150819)(731.960896,75.91151611)
}
}
{
\newrgbcolor{curcolor}{0 0 0}
\pscustom[linestyle=none,fillstyle=solid,fillcolor=curcolor]
{
\newpath
\moveto(741.35863037,77.23151611)
\curveto(741.27862925,77.29150681)(741.2336293,77.39650671)(741.22363037,77.54651611)
\lineto(741.22363037,78.01151611)
\lineto(741.22363037,78.26651611)
\curveto(741.22362931,78.35650575)(741.23862929,78.43150567)(741.26863037,78.49151611)
\curveto(741.30862922,78.57150553)(741.38862914,78.63150547)(741.50863037,78.67151611)
\curveto(741.528629,78.68150542)(741.54862898,78.68150542)(741.56863037,78.67151611)
\curveto(741.59862893,78.67150543)(741.62362891,78.67650543)(741.64363037,78.68651611)
\curveto(741.81362872,78.68650542)(741.97362856,78.68150542)(742.12363037,78.67151611)
\curveto(742.27362826,78.66150544)(742.37362816,78.6015055)(742.42363037,78.49151611)
\curveto(742.45362808,78.43150567)(742.46862806,78.35650575)(742.46863037,78.26651611)
\lineto(742.46863037,78.01151611)
\curveto(742.46862806,77.83150627)(742.46362807,77.66150644)(742.45363037,77.50151611)
\curveto(742.45362808,77.34150676)(742.38862814,77.23650687)(742.25863037,77.18651611)
\curveto(742.20862832,77.16650694)(742.15362838,77.15650695)(742.09363037,77.15651611)
\lineto(741.92863037,77.15651611)
\lineto(741.61363037,77.15651611)
\curveto(741.51362902,77.15650695)(741.4286291,77.18150692)(741.35863037,77.23151611)
\moveto(742.46863037,68.72651611)
\lineto(742.46863037,68.41151611)
\curveto(742.47862805,68.31151579)(742.45862807,68.23151587)(742.40863037,68.17151611)
\curveto(742.37862815,68.11151599)(742.3336282,68.07151603)(742.27363037,68.05151611)
\curveto(742.21362832,68.04151606)(742.14362839,68.02651608)(742.06363037,68.00651611)
\lineto(741.83863037,68.00651611)
\curveto(741.70862882,68.0065161)(741.59362894,68.01151609)(741.49363037,68.02151611)
\curveto(741.40362913,68.04151606)(741.3336292,68.09151601)(741.28363037,68.17151611)
\curveto(741.24362929,68.23151587)(741.22362931,68.3065158)(741.22363037,68.39651611)
\lineto(741.22363037,68.68151611)
\lineto(741.22363037,75.02651611)
\lineto(741.22363037,75.34151611)
\curveto(741.22362931,75.45150865)(741.24862928,75.53650857)(741.29863037,75.59651611)
\curveto(741.3286292,75.64650846)(741.36862916,75.67650843)(741.41863037,75.68651611)
\curveto(741.46862906,75.69650841)(741.52362901,75.71150839)(741.58363037,75.73151611)
\curveto(741.60362893,75.73150837)(741.62362891,75.72650838)(741.64363037,75.71651611)
\curveto(741.67362886,75.71650839)(741.69862883,75.72150838)(741.71863037,75.73151611)
\curveto(741.84862868,75.73150837)(741.97862855,75.72650838)(742.10863037,75.71651611)
\curveto(742.24862828,75.71650839)(742.34362819,75.67650843)(742.39363037,75.59651611)
\curveto(742.44362809,75.53650857)(742.46862806,75.45650865)(742.46863037,75.35651611)
\lineto(742.46863037,75.07151611)
\lineto(742.46863037,68.72651611)
}
}
{
\newrgbcolor{curcolor}{0 0 0}
\pscustom[linestyle=none,fillstyle=solid,fillcolor=curcolor]
{
\newpath
\moveto(748.10347412,75.88151611)
\curveto(748.73346889,75.9015082)(749.23846838,75.81650829)(749.61847412,75.62651611)
\curveto(749.99846762,75.43650867)(750.30346732,75.15150895)(750.53347412,74.77151611)
\curveto(750.59346703,74.67150943)(750.63846698,74.56150954)(750.66847412,74.44151611)
\curveto(750.70846691,74.33150977)(750.74346688,74.21650989)(750.77347412,74.09651611)
\curveto(750.8234668,73.9065102)(750.85346677,73.7015104)(750.86347412,73.48151611)
\curveto(750.87346675,73.26151084)(750.87846674,73.03651107)(750.87847412,72.80651611)
\lineto(750.87847412,71.20151611)
\lineto(750.87847412,68.86151611)
\curveto(750.87846674,68.69151541)(750.87346675,68.52151558)(750.86347412,68.35151611)
\curveto(750.86346676,68.18151592)(750.79846682,68.07151603)(750.66847412,68.02151611)
\curveto(750.618467,68.0015161)(750.56346706,67.99151611)(750.50347412,67.99151611)
\curveto(750.45346717,67.98151612)(750.39846722,67.97651613)(750.33847412,67.97651611)
\curveto(750.20846741,67.97651613)(750.08346754,67.98151612)(749.96347412,67.99151611)
\curveto(749.84346778,67.99151611)(749.75846786,68.03151607)(749.70847412,68.11151611)
\curveto(749.65846796,68.18151592)(749.63346799,68.27151583)(749.63347412,68.38151611)
\lineto(749.63347412,68.71151611)
\lineto(749.63347412,70.00151611)
\lineto(749.63347412,72.44651611)
\curveto(749.63346799,72.71651139)(749.62846799,72.98151112)(749.61847412,73.24151611)
\curveto(749.60846801,73.51151059)(749.56346806,73.74151036)(749.48347412,73.93151611)
\curveto(749.40346822,74.13150997)(749.28346834,74.29150981)(749.12347412,74.41151611)
\curveto(748.96346866,74.54150956)(748.77846884,74.64150946)(748.56847412,74.71151611)
\curveto(748.50846911,74.73150937)(748.44346918,74.74150936)(748.37347412,74.74151611)
\curveto(748.31346931,74.75150935)(748.25346937,74.76650934)(748.19347412,74.78651611)
\curveto(748.14346948,74.79650931)(748.06346956,74.79650931)(747.95347412,74.78651611)
\curveto(747.85346977,74.78650932)(747.78346984,74.78150932)(747.74347412,74.77151611)
\curveto(747.70346992,74.75150935)(747.66846995,74.74150936)(747.63847412,74.74151611)
\curveto(747.60847001,74.75150935)(747.57347005,74.75150935)(747.53347412,74.74151611)
\curveto(747.40347022,74.71150939)(747.27847034,74.67650943)(747.15847412,74.63651611)
\curveto(747.04847057,74.6065095)(746.94347068,74.56150954)(746.84347412,74.50151611)
\curveto(746.80347082,74.48150962)(746.76847085,74.46150964)(746.73847412,74.44151611)
\curveto(746.70847091,74.42150968)(746.67347095,74.4015097)(746.63347412,74.38151611)
\curveto(746.28347134,74.13150997)(746.02847159,73.75651035)(745.86847412,73.25651611)
\curveto(745.83847178,73.17651093)(745.8184718,73.09151101)(745.80847412,73.00151611)
\curveto(745.79847182,72.92151118)(745.78347184,72.84151126)(745.76347412,72.76151611)
\curveto(745.74347188,72.71151139)(745.73847188,72.66151144)(745.74847412,72.61151611)
\curveto(745.75847186,72.57151153)(745.75347187,72.53151157)(745.73347412,72.49151611)
\lineto(745.73347412,72.17651611)
\curveto(745.7234719,72.14651196)(745.7184719,72.11151199)(745.71847412,72.07151611)
\curveto(745.72847189,72.03151207)(745.73347189,71.98651212)(745.73347412,71.93651611)
\lineto(745.73347412,71.48651611)
\lineto(745.73347412,70.04651611)
\lineto(745.73347412,68.72651611)
\lineto(745.73347412,68.38151611)
\curveto(745.73347189,68.27151583)(745.70847191,68.18151592)(745.65847412,68.11151611)
\curveto(745.60847201,68.03151607)(745.5184721,67.99151611)(745.38847412,67.99151611)
\curveto(745.26847235,67.98151612)(745.14347248,67.97651613)(745.01347412,67.97651611)
\curveto(744.93347269,67.97651613)(744.85847276,67.98151612)(744.78847412,67.99151611)
\curveto(744.7184729,68.0015161)(744.65847296,68.02651608)(744.60847412,68.06651611)
\curveto(744.52847309,68.11651599)(744.48847313,68.21151589)(744.48847412,68.35151611)
\lineto(744.48847412,68.75651611)
\lineto(744.48847412,70.52651611)
\lineto(744.48847412,74.15651611)
\lineto(744.48847412,75.07151611)
\lineto(744.48847412,75.34151611)
\curveto(744.48847313,75.43150867)(744.50847311,75.5015086)(744.54847412,75.55151611)
\curveto(744.57847304,75.61150849)(744.62847299,75.65150845)(744.69847412,75.67151611)
\curveto(744.73847288,75.68150842)(744.79347283,75.69150841)(744.86347412,75.70151611)
\curveto(744.94347268,75.71150839)(745.0234726,75.71650839)(745.10347412,75.71651611)
\curveto(745.18347244,75.71650839)(745.25847236,75.71150839)(745.32847412,75.70151611)
\curveto(745.40847221,75.69150841)(745.46347216,75.67650843)(745.49347412,75.65651611)
\curveto(745.60347202,75.58650852)(745.65347197,75.49650861)(745.64347412,75.38651611)
\curveto(745.63347199,75.28650882)(745.64847197,75.17150893)(745.68847412,75.04151611)
\curveto(745.70847191,74.98150912)(745.74847187,74.93150917)(745.80847412,74.89151611)
\curveto(745.92847169,74.88150922)(746.0234716,74.92650918)(746.09347412,75.02651611)
\curveto(746.17347145,75.12650898)(746.25347137,75.2065089)(746.33347412,75.26651611)
\curveto(746.47347115,75.36650874)(746.61347101,75.45650865)(746.75347412,75.53651611)
\curveto(746.90347072,75.62650848)(747.07347055,75.7015084)(747.26347412,75.76151611)
\curveto(747.34347028,75.79150831)(747.42847019,75.81150829)(747.51847412,75.82151611)
\curveto(747.61847,75.83150827)(747.71346991,75.84650826)(747.80347412,75.86651611)
\curveto(747.85346977,75.87650823)(747.90346972,75.88150822)(747.95347412,75.88151611)
\lineto(748.10347412,75.88151611)
}
}
{
\newrgbcolor{curcolor}{0 0 0}
\pscustom[linestyle=none,fillstyle=solid,fillcolor=curcolor]
{
\newpath
\moveto(753.0480835,77.23151611)
\curveto(752.96808238,77.29150681)(752.92308242,77.39650671)(752.9130835,77.54651611)
\lineto(752.9130835,78.01151611)
\lineto(752.9130835,78.26651611)
\curveto(752.91308243,78.35650575)(752.92808242,78.43150567)(752.9580835,78.49151611)
\curveto(752.99808235,78.57150553)(753.07808227,78.63150547)(753.1980835,78.67151611)
\curveto(753.21808213,78.68150542)(753.23808211,78.68150542)(753.2580835,78.67151611)
\curveto(753.28808206,78.67150543)(753.31308203,78.67650543)(753.3330835,78.68651611)
\curveto(753.50308184,78.68650542)(753.66308168,78.68150542)(753.8130835,78.67151611)
\curveto(753.96308138,78.66150544)(754.06308128,78.6015055)(754.1130835,78.49151611)
\curveto(754.1430812,78.43150567)(754.15808119,78.35650575)(754.1580835,78.26651611)
\lineto(754.1580835,78.01151611)
\curveto(754.15808119,77.83150627)(754.15308119,77.66150644)(754.1430835,77.50151611)
\curveto(754.1430812,77.34150676)(754.07808127,77.23650687)(753.9480835,77.18651611)
\curveto(753.89808145,77.16650694)(753.8430815,77.15650695)(753.7830835,77.15651611)
\lineto(753.6180835,77.15651611)
\lineto(753.3030835,77.15651611)
\curveto(753.20308214,77.15650695)(753.11808223,77.18150692)(753.0480835,77.23151611)
\moveto(754.1580835,68.72651611)
\lineto(754.1580835,68.41151611)
\curveto(754.16808118,68.31151579)(754.1480812,68.23151587)(754.0980835,68.17151611)
\curveto(754.06808128,68.11151599)(754.02308132,68.07151603)(753.9630835,68.05151611)
\curveto(753.90308144,68.04151606)(753.83308151,68.02651608)(753.7530835,68.00651611)
\lineto(753.5280835,68.00651611)
\curveto(753.39808195,68.0065161)(753.28308206,68.01151609)(753.1830835,68.02151611)
\curveto(753.09308225,68.04151606)(753.02308232,68.09151601)(752.9730835,68.17151611)
\curveto(752.93308241,68.23151587)(752.91308243,68.3065158)(752.9130835,68.39651611)
\lineto(752.9130835,68.68151611)
\lineto(752.9130835,75.02651611)
\lineto(752.9130835,75.34151611)
\curveto(752.91308243,75.45150865)(752.93808241,75.53650857)(752.9880835,75.59651611)
\curveto(753.01808233,75.64650846)(753.05808229,75.67650843)(753.1080835,75.68651611)
\curveto(753.15808219,75.69650841)(753.21308213,75.71150839)(753.2730835,75.73151611)
\curveto(753.29308205,75.73150837)(753.31308203,75.72650838)(753.3330835,75.71651611)
\curveto(753.36308198,75.71650839)(753.38808196,75.72150838)(753.4080835,75.73151611)
\curveto(753.53808181,75.73150837)(753.66808168,75.72650838)(753.7980835,75.71651611)
\curveto(753.93808141,75.71650839)(754.03308131,75.67650843)(754.0830835,75.59651611)
\curveto(754.13308121,75.53650857)(754.15808119,75.45650865)(754.1580835,75.35651611)
\lineto(754.1580835,75.07151611)
\lineto(754.1580835,68.72651611)
}
}
{
\newrgbcolor{curcolor}{0 0 0}
\pscustom[linestyle=none,fillstyle=solid,fillcolor=curcolor]
{
\newpath
\moveto(758.53292725,75.91151611)
\curveto(759.25292318,75.92150818)(759.85792258,75.83650827)(760.34792725,75.65651611)
\curveto(760.8379216,75.48650862)(761.21792122,75.18150892)(761.48792725,74.74151611)
\curveto(761.55792088,74.63150947)(761.61292082,74.51650959)(761.65292725,74.39651611)
\curveto(761.69292074,74.28650982)(761.7329207,74.16150994)(761.77292725,74.02151611)
\curveto(761.79292064,73.95151015)(761.79792064,73.87651023)(761.78792725,73.79651611)
\curveto(761.77792066,73.72651038)(761.76292067,73.67151043)(761.74292725,73.63151611)
\curveto(761.72292071,73.61151049)(761.69792074,73.59151051)(761.66792725,73.57151611)
\curveto(761.6379208,73.56151054)(761.61292082,73.54651056)(761.59292725,73.52651611)
\curveto(761.54292089,73.5065106)(761.49292094,73.5015106)(761.44292725,73.51151611)
\curveto(761.39292104,73.52151058)(761.34292109,73.52151058)(761.29292725,73.51151611)
\curveto(761.21292122,73.49151061)(761.10792133,73.48651062)(760.97792725,73.49651611)
\curveto(760.84792159,73.51651059)(760.75792168,73.54151056)(760.70792725,73.57151611)
\curveto(760.62792181,73.62151048)(760.57292186,73.68651042)(760.54292725,73.76651611)
\curveto(760.52292191,73.85651025)(760.48792195,73.94151016)(760.43792725,74.02151611)
\curveto(760.34792209,74.18150992)(760.22292221,74.32650978)(760.06292725,74.45651611)
\curveto(759.95292248,74.53650957)(759.8329226,74.59650951)(759.70292725,74.63651611)
\curveto(759.57292286,74.67650943)(759.432923,74.71650939)(759.28292725,74.75651611)
\curveto(759.2329232,74.77650933)(759.18292325,74.78150932)(759.13292725,74.77151611)
\curveto(759.08292335,74.77150933)(759.0329234,74.77650933)(758.98292725,74.78651611)
\curveto(758.92292351,74.8065093)(758.84792359,74.81650929)(758.75792725,74.81651611)
\curveto(758.66792377,74.81650929)(758.59292384,74.8065093)(758.53292725,74.78651611)
\lineto(758.44292725,74.78651611)
\lineto(758.29292725,74.75651611)
\curveto(758.24292419,74.75650935)(758.19292424,74.75150935)(758.14292725,74.74151611)
\curveto(757.88292455,74.68150942)(757.66792477,74.59650951)(757.49792725,74.48651611)
\curveto(757.32792511,74.37650973)(757.21292522,74.19150991)(757.15292725,73.93151611)
\curveto(757.1329253,73.86151024)(757.12792531,73.79151031)(757.13792725,73.72151611)
\curveto(757.15792528,73.65151045)(757.17792526,73.59151051)(757.19792725,73.54151611)
\curveto(757.25792518,73.39151071)(757.32792511,73.28151082)(757.40792725,73.21151611)
\curveto(757.49792494,73.15151095)(757.60792483,73.08151102)(757.73792725,73.00151611)
\curveto(757.89792454,72.9015112)(758.07792436,72.82651128)(758.27792725,72.77651611)
\curveto(758.47792396,72.73651137)(758.67792376,72.68651142)(758.87792725,72.62651611)
\curveto(759.00792343,72.58651152)(759.1379233,72.55651155)(759.26792725,72.53651611)
\curveto(759.39792304,72.51651159)(759.52792291,72.48651162)(759.65792725,72.44651611)
\curveto(759.86792257,72.38651172)(760.07292236,72.32651178)(760.27292725,72.26651611)
\curveto(760.47292196,72.21651189)(760.67292176,72.15151195)(760.87292725,72.07151611)
\lineto(761.02292725,72.01151611)
\curveto(761.07292136,71.99151211)(761.12292131,71.96651214)(761.17292725,71.93651611)
\curveto(761.37292106,71.81651229)(761.54792089,71.68151242)(761.69792725,71.53151611)
\curveto(761.84792059,71.38151272)(761.97292046,71.19151291)(762.07292725,70.96151611)
\curveto(762.09292034,70.89151321)(762.11292032,70.79651331)(762.13292725,70.67651611)
\curveto(762.15292028,70.6065135)(762.16292027,70.53151357)(762.16292725,70.45151611)
\curveto(762.17292026,70.38151372)(762.17792026,70.3015138)(762.17792725,70.21151611)
\lineto(762.17792725,70.06151611)
\curveto(762.15792028,69.99151411)(762.14792029,69.92151418)(762.14792725,69.85151611)
\curveto(762.14792029,69.78151432)(762.1379203,69.71151439)(762.11792725,69.64151611)
\curveto(762.08792035,69.53151457)(762.05292038,69.42651468)(762.01292725,69.32651611)
\curveto(761.97292046,69.22651488)(761.92792051,69.13651497)(761.87792725,69.05651611)
\curveto(761.71792072,68.79651531)(761.51292092,68.58651552)(761.26292725,68.42651611)
\curveto(761.01292142,68.27651583)(760.7329217,68.14651596)(760.42292725,68.03651611)
\curveto(760.3329221,68.0065161)(760.2379222,67.98651612)(760.13792725,67.97651611)
\curveto(760.04792239,67.95651615)(759.95792248,67.93151617)(759.86792725,67.90151611)
\curveto(759.76792267,67.88151622)(759.66792277,67.87151623)(759.56792725,67.87151611)
\curveto(759.46792297,67.87151623)(759.36792307,67.86151624)(759.26792725,67.84151611)
\lineto(759.11792725,67.84151611)
\curveto(759.06792337,67.83151627)(758.99792344,67.82651628)(758.90792725,67.82651611)
\curveto(758.81792362,67.82651628)(758.74792369,67.83151627)(758.69792725,67.84151611)
\lineto(758.53292725,67.84151611)
\curveto(758.47292396,67.86151624)(758.40792403,67.87151623)(758.33792725,67.87151611)
\curveto(758.26792417,67.86151624)(758.20792423,67.86651624)(758.15792725,67.88651611)
\curveto(758.10792433,67.89651621)(758.04292439,67.9015162)(757.96292725,67.90151611)
\lineto(757.72292725,67.96151611)
\curveto(757.65292478,67.97151613)(757.57792486,67.99151611)(757.49792725,68.02151611)
\curveto(757.18792525,68.12151598)(756.91792552,68.24651586)(756.68792725,68.39651611)
\curveto(756.45792598,68.54651556)(756.25792618,68.74151536)(756.08792725,68.98151611)
\curveto(755.99792644,69.11151499)(755.92292651,69.24651486)(755.86292725,69.38651611)
\curveto(755.80292663,69.52651458)(755.74792669,69.68151442)(755.69792725,69.85151611)
\curveto(755.67792676,69.91151419)(755.66792677,69.98151412)(755.66792725,70.06151611)
\curveto(755.67792676,70.15151395)(755.69292674,70.22151388)(755.71292725,70.27151611)
\curveto(755.74292669,70.31151379)(755.79292664,70.35151375)(755.86292725,70.39151611)
\curveto(755.91292652,70.41151369)(755.98292645,70.42151368)(756.07292725,70.42151611)
\curveto(756.16292627,70.43151367)(756.25292618,70.43151367)(756.34292725,70.42151611)
\curveto(756.432926,70.41151369)(756.51792592,70.39651371)(756.59792725,70.37651611)
\curveto(756.68792575,70.36651374)(756.74792569,70.35151375)(756.77792725,70.33151611)
\curveto(756.84792559,70.28151382)(756.89292554,70.2065139)(756.91292725,70.10651611)
\curveto(756.94292549,70.01651409)(756.97792546,69.93151417)(757.01792725,69.85151611)
\curveto(757.11792532,69.63151447)(757.25292518,69.46151464)(757.42292725,69.34151611)
\curveto(757.54292489,69.25151485)(757.67792476,69.18151492)(757.82792725,69.13151611)
\curveto(757.97792446,69.08151502)(758.1379243,69.03151507)(758.30792725,68.98151611)
\lineto(758.62292725,68.93651611)
\lineto(758.71292725,68.93651611)
\curveto(758.78292365,68.91651519)(758.87292356,68.9065152)(758.98292725,68.90651611)
\curveto(759.10292333,68.9065152)(759.20292323,68.91651519)(759.28292725,68.93651611)
\curveto(759.35292308,68.93651517)(759.40792303,68.94151516)(759.44792725,68.95151611)
\curveto(759.50792293,68.96151514)(759.56792287,68.96651514)(759.62792725,68.96651611)
\curveto(759.68792275,68.97651513)(759.74292269,68.98651512)(759.79292725,68.99651611)
\curveto(760.08292235,69.07651503)(760.31292212,69.18151492)(760.48292725,69.31151611)
\curveto(760.65292178,69.44151466)(760.77292166,69.66151444)(760.84292725,69.97151611)
\curveto(760.86292157,70.02151408)(760.86792157,70.07651403)(760.85792725,70.13651611)
\curveto(760.84792159,70.19651391)(760.8379216,70.24151386)(760.82792725,70.27151611)
\curveto(760.77792166,70.46151364)(760.70792173,70.6015135)(760.61792725,70.69151611)
\curveto(760.52792191,70.79151331)(760.41292202,70.88151322)(760.27292725,70.96151611)
\curveto(760.18292225,71.02151308)(760.08292235,71.07151303)(759.97292725,71.11151611)
\lineto(759.64292725,71.23151611)
\curveto(759.61292282,71.24151286)(759.58292285,71.24651286)(759.55292725,71.24651611)
\curveto(759.5329229,71.24651286)(759.50792293,71.25651285)(759.47792725,71.27651611)
\curveto(759.1379233,71.38651272)(758.78292365,71.46651264)(758.41292725,71.51651611)
\curveto(758.05292438,71.57651253)(757.71292472,71.67151243)(757.39292725,71.80151611)
\curveto(757.29292514,71.84151226)(757.19792524,71.87651223)(757.10792725,71.90651611)
\curveto(757.01792542,71.93651217)(756.9329255,71.97651213)(756.85292725,72.02651611)
\curveto(756.66292577,72.13651197)(756.48792595,72.26151184)(756.32792725,72.40151611)
\curveto(756.16792627,72.54151156)(756.04292639,72.71651139)(755.95292725,72.92651611)
\curveto(755.92292651,72.99651111)(755.89792654,73.06651104)(755.87792725,73.13651611)
\curveto(755.86792657,73.2065109)(755.85292658,73.28151082)(755.83292725,73.36151611)
\curveto(755.80292663,73.48151062)(755.79292664,73.61651049)(755.80292725,73.76651611)
\curveto(755.81292662,73.92651018)(755.82792661,74.06151004)(755.84792725,74.17151611)
\curveto(755.86792657,74.22150988)(755.87792656,74.26150984)(755.87792725,74.29151611)
\curveto(755.88792655,74.33150977)(755.90292653,74.37150973)(755.92292725,74.41151611)
\curveto(756.01292642,74.64150946)(756.1329263,74.84150926)(756.28292725,75.01151611)
\curveto(756.44292599,75.18150892)(756.62292581,75.33150877)(756.82292725,75.46151611)
\curveto(756.97292546,75.55150855)(757.1379253,75.62150848)(757.31792725,75.67151611)
\curveto(757.49792494,75.73150837)(757.68792475,75.78650832)(757.88792725,75.83651611)
\curveto(757.95792448,75.84650826)(758.02292441,75.85650825)(758.08292725,75.86651611)
\curveto(758.15292428,75.87650823)(758.22792421,75.88650822)(758.30792725,75.89651611)
\curveto(758.3379241,75.9065082)(758.37792406,75.9065082)(758.42792725,75.89651611)
\curveto(758.47792396,75.88650822)(758.51292392,75.89150821)(758.53292725,75.91151611)
}
}
{
\newrgbcolor{curcolor}{0 0 0}
\pscustom[linestyle=none,fillstyle=solid,fillcolor=curcolor]
{
\newpath
\moveto(764.54792725,78.07151611)
\curveto(764.69792524,78.07150603)(764.84792509,78.06650604)(764.99792725,78.05651611)
\curveto(765.14792479,78.05650605)(765.25292468,78.01650609)(765.31292725,77.93651611)
\curveto(765.36292457,77.87650623)(765.38792455,77.79150631)(765.38792725,77.68151611)
\curveto(765.39792454,77.58150652)(765.40292453,77.47650663)(765.40292725,77.36651611)
\lineto(765.40292725,76.49651611)
\curveto(765.40292453,76.41650769)(765.39792454,76.33150777)(765.38792725,76.24151611)
\curveto(765.38792455,76.16150794)(765.39792454,76.09150801)(765.41792725,76.03151611)
\curveto(765.45792448,75.89150821)(765.54792439,75.8015083)(765.68792725,75.76151611)
\curveto(765.7379242,75.75150835)(765.78292415,75.74650836)(765.82292725,75.74651611)
\lineto(765.97292725,75.74651611)
\lineto(766.37792725,75.74651611)
\curveto(766.5379234,75.75650835)(766.65292328,75.74650836)(766.72292725,75.71651611)
\curveto(766.81292312,75.65650845)(766.87292306,75.59650851)(766.90292725,75.53651611)
\curveto(766.92292301,75.49650861)(766.932923,75.45150865)(766.93292725,75.40151611)
\lineto(766.93292725,75.25151611)
\curveto(766.932923,75.14150896)(766.92792301,75.03650907)(766.91792725,74.93651611)
\curveto(766.90792303,74.84650926)(766.87292306,74.77650933)(766.81292725,74.72651611)
\curveto(766.75292318,74.67650943)(766.66792327,74.64650946)(766.55792725,74.63651611)
\lineto(766.22792725,74.63651611)
\curveto(766.11792382,74.64650946)(766.00792393,74.65150945)(765.89792725,74.65151611)
\curveto(765.78792415,74.65150945)(765.69292424,74.63650947)(765.61292725,74.60651611)
\curveto(765.54292439,74.57650953)(765.49292444,74.52650958)(765.46292725,74.45651611)
\curveto(765.4329245,74.38650972)(765.41292452,74.3015098)(765.40292725,74.20151611)
\curveto(765.39292454,74.11150999)(765.38792455,74.01151009)(765.38792725,73.90151611)
\curveto(765.39792454,73.8015103)(765.40292453,73.7015104)(765.40292725,73.60151611)
\lineto(765.40292725,70.63151611)
\curveto(765.40292453,70.41151369)(765.39792454,70.17651393)(765.38792725,69.92651611)
\curveto(765.38792455,69.68651442)(765.4329245,69.5015146)(765.52292725,69.37151611)
\curveto(765.57292436,69.29151481)(765.6379243,69.23651487)(765.71792725,69.20651611)
\curveto(765.79792414,69.17651493)(765.89292404,69.15151495)(766.00292725,69.13151611)
\curveto(766.0329239,69.12151498)(766.06292387,69.11651499)(766.09292725,69.11651611)
\curveto(766.1329238,69.12651498)(766.16792377,69.12651498)(766.19792725,69.11651611)
\lineto(766.39292725,69.11651611)
\curveto(766.49292344,69.11651499)(766.58292335,69.106515)(766.66292725,69.08651611)
\curveto(766.75292318,69.07651503)(766.81792312,69.04151506)(766.85792725,68.98151611)
\curveto(766.87792306,68.95151515)(766.89292304,68.89651521)(766.90292725,68.81651611)
\curveto(766.92292301,68.74651536)(766.932923,68.67151543)(766.93292725,68.59151611)
\curveto(766.94292299,68.51151559)(766.94292299,68.43151567)(766.93292725,68.35151611)
\curveto(766.92292301,68.28151582)(766.90292303,68.22651588)(766.87292725,68.18651611)
\curveto(766.8329231,68.11651599)(766.75792318,68.06651604)(766.64792725,68.03651611)
\curveto(766.56792337,68.01651609)(766.47792346,68.0065161)(766.37792725,68.00651611)
\curveto(766.27792366,68.01651609)(766.18792375,68.02151608)(766.10792725,68.02151611)
\curveto(766.04792389,68.02151608)(765.98792395,68.01651609)(765.92792725,68.00651611)
\curveto(765.86792407,68.0065161)(765.81292412,68.01151609)(765.76292725,68.02151611)
\lineto(765.58292725,68.02151611)
\curveto(765.5329244,68.03151607)(765.48292445,68.03651607)(765.43292725,68.03651611)
\curveto(765.39292454,68.04651606)(765.34792459,68.05151605)(765.29792725,68.05151611)
\curveto(765.09792484,68.101516)(764.92292501,68.15651595)(764.77292725,68.21651611)
\curveto(764.6329253,68.27651583)(764.51292542,68.38151572)(764.41292725,68.53151611)
\curveto(764.27292566,68.73151537)(764.19292574,68.98151512)(764.17292725,69.28151611)
\curveto(764.15292578,69.59151451)(764.14292579,69.92151418)(764.14292725,70.27151611)
\lineto(764.14292725,74.20151611)
\curveto(764.11292582,74.33150977)(764.08292585,74.42650968)(764.05292725,74.48651611)
\curveto(764.0329259,74.54650956)(763.96292597,74.59650951)(763.84292725,74.63651611)
\curveto(763.80292613,74.64650946)(763.76292617,74.64650946)(763.72292725,74.63651611)
\curveto(763.68292625,74.62650948)(763.64292629,74.63150947)(763.60292725,74.65151611)
\lineto(763.36292725,74.65151611)
\curveto(763.2329267,74.65150945)(763.12292681,74.66150944)(763.03292725,74.68151611)
\curveto(762.95292698,74.71150939)(762.89792704,74.77150933)(762.86792725,74.86151611)
\curveto(762.84792709,74.9015092)(762.8329271,74.94650916)(762.82292725,74.99651611)
\lineto(762.82292725,75.14651611)
\curveto(762.82292711,75.28650882)(762.8329271,75.4015087)(762.85292725,75.49151611)
\curveto(762.87292706,75.59150851)(762.932927,75.66650844)(763.03292725,75.71651611)
\curveto(763.14292679,75.75650835)(763.28292665,75.76650834)(763.45292725,75.74651611)
\curveto(763.6329263,75.72650838)(763.78292615,75.73650837)(763.90292725,75.77651611)
\curveto(763.99292594,75.82650828)(764.06292587,75.89650821)(764.11292725,75.98651611)
\curveto(764.1329258,76.04650806)(764.14292579,76.12150798)(764.14292725,76.21151611)
\lineto(764.14292725,76.46651611)
\lineto(764.14292725,77.39651611)
\lineto(764.14292725,77.63651611)
\curveto(764.14292579,77.72650638)(764.15292578,77.8015063)(764.17292725,77.86151611)
\curveto(764.21292572,77.94150616)(764.28792565,78.0065061)(764.39792725,78.05651611)
\curveto(764.42792551,78.05650605)(764.45292548,78.05650605)(764.47292725,78.05651611)
\curveto(764.50292543,78.06650604)(764.52792541,78.07150603)(764.54792725,78.07151611)
}
}
{
\newrgbcolor{curcolor}{0 0 0}
\pscustom[linestyle=none,fillstyle=solid,fillcolor=curcolor]
{
\newpath
\moveto(771.96472412,75.91151611)
\curveto(772.19471933,75.91150819)(772.3247192,75.85150825)(772.35472412,75.73151611)
\curveto(772.38471914,75.62150848)(772.39971913,75.45650865)(772.39972412,75.23651611)
\lineto(772.39972412,74.95151611)
\curveto(772.39971913,74.86150924)(772.37471915,74.78650932)(772.32472412,74.72651611)
\curveto(772.26471926,74.64650946)(772.17971935,74.6015095)(772.06972412,74.59151611)
\curveto(771.95971957,74.59150951)(771.84971968,74.57650953)(771.73972412,74.54651611)
\curveto(771.59971993,74.51650959)(771.46472006,74.48650962)(771.33472412,74.45651611)
\curveto(771.21472031,74.42650968)(771.09972043,74.38650972)(770.98972412,74.33651611)
\curveto(770.69972083,74.2065099)(770.46472106,74.02651008)(770.28472412,73.79651611)
\curveto(770.10472142,73.57651053)(769.94972158,73.32151078)(769.81972412,73.03151611)
\curveto(769.77972175,72.92151118)(769.74972178,72.8065113)(769.72972412,72.68651611)
\curveto(769.70972182,72.57651153)(769.68472184,72.46151164)(769.65472412,72.34151611)
\curveto(769.64472188,72.29151181)(769.63972189,72.24151186)(769.63972412,72.19151611)
\curveto(769.64972188,72.14151196)(769.64972188,72.09151201)(769.63972412,72.04151611)
\curveto(769.60972192,71.92151218)(769.59472193,71.78151232)(769.59472412,71.62151611)
\curveto(769.60472192,71.47151263)(769.60972192,71.32651278)(769.60972412,71.18651611)
\lineto(769.60972412,69.34151611)
\lineto(769.60972412,68.99651611)
\curveto(769.60972192,68.87651523)(769.60472192,68.76151534)(769.59472412,68.65151611)
\curveto(769.58472194,68.54151556)(769.57972195,68.44651566)(769.57972412,68.36651611)
\curveto(769.58972194,68.28651582)(769.56972196,68.21651589)(769.51972412,68.15651611)
\curveto(769.46972206,68.08651602)(769.38972214,68.04651606)(769.27972412,68.03651611)
\curveto(769.17972235,68.02651608)(769.06972246,68.02151608)(768.94972412,68.02151611)
\lineto(768.67972412,68.02151611)
\curveto(768.6297229,68.04151606)(768.57972295,68.05651605)(768.52972412,68.06651611)
\curveto(768.48972304,68.08651602)(768.45972307,68.11151599)(768.43972412,68.14151611)
\curveto(768.38972314,68.21151589)(768.35972317,68.29651581)(768.34972412,68.39651611)
\lineto(768.34972412,68.72651611)
\lineto(768.34972412,69.88151611)
\lineto(768.34972412,74.03651611)
\lineto(768.34972412,75.07151611)
\lineto(768.34972412,75.37151611)
\curveto(768.35972317,75.47150863)(768.38972314,75.55650855)(768.43972412,75.62651611)
\curveto(768.46972306,75.66650844)(768.51972301,75.69650841)(768.58972412,75.71651611)
\curveto(768.66972286,75.73650837)(768.75472277,75.74650836)(768.84472412,75.74651611)
\curveto(768.93472259,75.75650835)(769.0247225,75.75650835)(769.11472412,75.74651611)
\curveto(769.20472232,75.73650837)(769.27472225,75.72150838)(769.32472412,75.70151611)
\curveto(769.40472212,75.67150843)(769.45472207,75.61150849)(769.47472412,75.52151611)
\curveto(769.50472202,75.44150866)(769.51972201,75.35150875)(769.51972412,75.25151611)
\lineto(769.51972412,74.95151611)
\curveto(769.51972201,74.85150925)(769.53972199,74.76150934)(769.57972412,74.68151611)
\curveto(769.58972194,74.66150944)(769.59972193,74.64650946)(769.60972412,74.63651611)
\lineto(769.65472412,74.59151611)
\curveto(769.76472176,74.59150951)(769.85472167,74.63650947)(769.92472412,74.72651611)
\curveto(769.99472153,74.82650928)(770.05472147,74.9065092)(770.10472412,74.96651611)
\lineto(770.19472412,75.05651611)
\curveto(770.28472124,75.16650894)(770.40972112,75.28150882)(770.56972412,75.40151611)
\curveto(770.7297208,75.52150858)(770.87972065,75.61150849)(771.01972412,75.67151611)
\curveto(771.10972042,75.72150838)(771.20472032,75.75650835)(771.30472412,75.77651611)
\curveto(771.40472012,75.8065083)(771.50972002,75.83650827)(771.61972412,75.86651611)
\curveto(771.67971985,75.87650823)(771.73971979,75.88150822)(771.79972412,75.88151611)
\curveto(771.85971967,75.89150821)(771.91471961,75.9015082)(771.96472412,75.91151611)
}
}
{
\newrgbcolor{curcolor}{0 0 0}
\pscustom[linestyle=none,fillstyle=solid,fillcolor=curcolor]
{
\newpath
\moveto(780.21448975,68.56151611)
\curveto(780.24448192,68.4015157)(780.22948193,68.26651584)(780.16948975,68.15651611)
\curveto(780.10948205,68.05651605)(780.02948213,67.98151612)(779.92948975,67.93151611)
\curveto(779.87948228,67.91151619)(779.82448234,67.9015162)(779.76448975,67.90151611)
\curveto(779.71448245,67.9015162)(779.6594825,67.89151621)(779.59948975,67.87151611)
\curveto(779.37948278,67.82151628)(779.159483,67.83651627)(778.93948975,67.91651611)
\curveto(778.72948343,67.98651612)(778.58448358,68.07651603)(778.50448975,68.18651611)
\curveto(778.45448371,68.25651585)(778.40948375,68.33651577)(778.36948975,68.42651611)
\curveto(778.32948383,68.52651558)(778.27948388,68.6065155)(778.21948975,68.66651611)
\curveto(778.19948396,68.68651542)(778.17448399,68.7065154)(778.14448975,68.72651611)
\curveto(778.12448404,68.74651536)(778.09448407,68.75151535)(778.05448975,68.74151611)
\curveto(777.94448422,68.71151539)(777.83948432,68.65651545)(777.73948975,68.57651611)
\curveto(777.64948451,68.49651561)(777.5594846,68.42651568)(777.46948975,68.36651611)
\curveto(777.33948482,68.28651582)(777.19948496,68.21151589)(777.04948975,68.14151611)
\curveto(776.89948526,68.08151602)(776.73948542,68.02651608)(776.56948975,67.97651611)
\curveto(776.46948569,67.94651616)(776.3594858,67.92651618)(776.23948975,67.91651611)
\curveto(776.12948603,67.9065162)(776.01948614,67.89151621)(775.90948975,67.87151611)
\curveto(775.8594863,67.86151624)(775.81448635,67.85651625)(775.77448975,67.85651611)
\lineto(775.66948975,67.85651611)
\curveto(775.5594866,67.83651627)(775.45448671,67.83651627)(775.35448975,67.85651611)
\lineto(775.21948975,67.85651611)
\curveto(775.16948699,67.86651624)(775.11948704,67.87151623)(775.06948975,67.87151611)
\curveto(775.01948714,67.87151623)(774.97448719,67.88151622)(774.93448975,67.90151611)
\curveto(774.89448727,67.91151619)(774.8594873,67.91651619)(774.82948975,67.91651611)
\curveto(774.80948735,67.9065162)(774.78448738,67.9065162)(774.75448975,67.91651611)
\lineto(774.51448975,67.97651611)
\curveto(774.43448773,67.98651612)(774.3594878,68.0065161)(774.28948975,68.03651611)
\curveto(773.98948817,68.16651594)(773.74448842,68.31151579)(773.55448975,68.47151611)
\curveto(773.37448879,68.64151546)(773.22448894,68.87651523)(773.10448975,69.17651611)
\curveto(773.01448915,69.39651471)(772.96948919,69.66151444)(772.96948975,69.97151611)
\lineto(772.96948975,70.28651611)
\curveto(772.97948918,70.33651377)(772.98448918,70.38651372)(772.98448975,70.43651611)
\lineto(773.01448975,70.61651611)
\lineto(773.13448975,70.94651611)
\curveto(773.17448899,71.05651305)(773.22448894,71.15651295)(773.28448975,71.24651611)
\curveto(773.4644887,71.53651257)(773.70948845,71.75151235)(774.01948975,71.89151611)
\curveto(774.32948783,72.03151207)(774.66948749,72.15651195)(775.03948975,72.26651611)
\curveto(775.17948698,72.3065118)(775.32448684,72.33651177)(775.47448975,72.35651611)
\curveto(775.62448654,72.37651173)(775.77448639,72.4015117)(775.92448975,72.43151611)
\curveto(775.99448617,72.45151165)(776.0594861,72.46151164)(776.11948975,72.46151611)
\curveto(776.18948597,72.46151164)(776.2644859,72.47151163)(776.34448975,72.49151611)
\curveto(776.41448575,72.51151159)(776.48448568,72.52151158)(776.55448975,72.52151611)
\curveto(776.62448554,72.53151157)(776.69948546,72.54651156)(776.77948975,72.56651611)
\curveto(777.02948513,72.62651148)(777.2644849,72.67651143)(777.48448975,72.71651611)
\curveto(777.70448446,72.76651134)(777.87948428,72.88151122)(778.00948975,73.06151611)
\curveto(778.06948409,73.14151096)(778.11948404,73.24151086)(778.15948975,73.36151611)
\curveto(778.19948396,73.49151061)(778.19948396,73.63151047)(778.15948975,73.78151611)
\curveto(778.09948406,74.02151008)(778.00948415,74.21150989)(777.88948975,74.35151611)
\curveto(777.77948438,74.49150961)(777.61948454,74.6015095)(777.40948975,74.68151611)
\curveto(777.28948487,74.73150937)(777.14448502,74.76650934)(776.97448975,74.78651611)
\curveto(776.81448535,74.8065093)(776.64448552,74.81650929)(776.46448975,74.81651611)
\curveto(776.28448588,74.81650929)(776.10948605,74.8065093)(775.93948975,74.78651611)
\curveto(775.76948639,74.76650934)(775.62448654,74.73650937)(775.50448975,74.69651611)
\curveto(775.33448683,74.63650947)(775.16948699,74.55150955)(775.00948975,74.44151611)
\curveto(774.92948723,74.38150972)(774.85448731,74.3015098)(774.78448975,74.20151611)
\curveto(774.72448744,74.11150999)(774.66948749,74.01151009)(774.61948975,73.90151611)
\curveto(774.58948757,73.82151028)(774.5594876,73.73651037)(774.52948975,73.64651611)
\curveto(774.50948765,73.55651055)(774.4644877,73.48651062)(774.39448975,73.43651611)
\curveto(774.35448781,73.4065107)(774.28448788,73.38151072)(774.18448975,73.36151611)
\curveto(774.09448807,73.35151075)(773.99948816,73.34651076)(773.89948975,73.34651611)
\curveto(773.79948836,73.34651076)(773.69948846,73.35151075)(773.59948975,73.36151611)
\curveto(773.50948865,73.38151072)(773.44448872,73.4065107)(773.40448975,73.43651611)
\curveto(773.3644888,73.46651064)(773.33448883,73.51651059)(773.31448975,73.58651611)
\curveto(773.29448887,73.65651045)(773.29448887,73.73151037)(773.31448975,73.81151611)
\curveto(773.34448882,73.94151016)(773.37448879,74.06151004)(773.40448975,74.17151611)
\curveto(773.44448872,74.29150981)(773.48948867,74.4065097)(773.53948975,74.51651611)
\curveto(773.72948843,74.86650924)(773.96948819,75.13650897)(774.25948975,75.32651611)
\curveto(774.54948761,75.52650858)(774.90948725,75.68650842)(775.33948975,75.80651611)
\curveto(775.43948672,75.82650828)(775.53948662,75.84150826)(775.63948975,75.85151611)
\curveto(775.74948641,75.86150824)(775.8594863,75.87650823)(775.96948975,75.89651611)
\curveto(776.00948615,75.9065082)(776.07448609,75.9065082)(776.16448975,75.89651611)
\curveto(776.25448591,75.89650821)(776.30948585,75.9065082)(776.32948975,75.92651611)
\curveto(777.02948513,75.93650817)(777.63948452,75.85650825)(778.15948975,75.68651611)
\curveto(778.67948348,75.51650859)(779.04448312,75.19150891)(779.25448975,74.71151611)
\curveto(779.34448282,74.51150959)(779.39448277,74.27650983)(779.40448975,74.00651611)
\curveto(779.42448274,73.74651036)(779.43448273,73.47151063)(779.43448975,73.18151611)
\lineto(779.43448975,69.86651611)
\curveto(779.43448273,69.72651438)(779.43948272,69.59151451)(779.44948975,69.46151611)
\curveto(779.4594827,69.33151477)(779.48948267,69.22651488)(779.53948975,69.14651611)
\curveto(779.58948257,69.07651503)(779.65448251,69.02651508)(779.73448975,68.99651611)
\curveto(779.82448234,68.95651515)(779.90948225,68.92651518)(779.98948975,68.90651611)
\curveto(780.06948209,68.89651521)(780.12948203,68.85151525)(780.16948975,68.77151611)
\curveto(780.18948197,68.74151536)(780.19948196,68.71151539)(780.19948975,68.68151611)
\curveto(780.19948196,68.65151545)(780.20448196,68.61151549)(780.21448975,68.56151611)
\moveto(778.06948975,70.22651611)
\curveto(778.12948403,70.36651374)(778.159484,70.52651358)(778.15948975,70.70651611)
\curveto(778.16948399,70.89651321)(778.17448399,71.09151301)(778.17448975,71.29151611)
\curveto(778.17448399,71.4015127)(778.16948399,71.5015126)(778.15948975,71.59151611)
\curveto(778.14948401,71.68151242)(778.10948405,71.75151235)(778.03948975,71.80151611)
\curveto(778.00948415,71.82151228)(777.93948422,71.83151227)(777.82948975,71.83151611)
\curveto(777.80948435,71.81151229)(777.77448439,71.8015123)(777.72448975,71.80151611)
\curveto(777.67448449,71.8015123)(777.62948453,71.79151231)(777.58948975,71.77151611)
\curveto(777.50948465,71.75151235)(777.41948474,71.73151237)(777.31948975,71.71151611)
\lineto(777.01948975,71.65151611)
\curveto(776.98948517,71.65151245)(776.95448521,71.64651246)(776.91448975,71.63651611)
\lineto(776.80948975,71.63651611)
\curveto(776.6594855,71.59651251)(776.49448567,71.57151253)(776.31448975,71.56151611)
\curveto(776.14448602,71.56151254)(775.98448618,71.54151256)(775.83448975,71.50151611)
\curveto(775.75448641,71.48151262)(775.67948648,71.46151264)(775.60948975,71.44151611)
\curveto(775.54948661,71.43151267)(775.47948668,71.41651269)(775.39948975,71.39651611)
\curveto(775.23948692,71.34651276)(775.08948707,71.28151282)(774.94948975,71.20151611)
\curveto(774.80948735,71.13151297)(774.68948747,71.04151306)(774.58948975,70.93151611)
\curveto(774.48948767,70.82151328)(774.41448775,70.68651342)(774.36448975,70.52651611)
\curveto(774.31448785,70.37651373)(774.29448787,70.19151391)(774.30448975,69.97151611)
\curveto(774.30448786,69.87151423)(774.31948784,69.77651433)(774.34948975,69.68651611)
\curveto(774.38948777,69.6065145)(774.43448773,69.53151457)(774.48448975,69.46151611)
\curveto(774.5644876,69.35151475)(774.66948749,69.25651485)(774.79948975,69.17651611)
\curveto(774.92948723,69.106515)(775.06948709,69.04651506)(775.21948975,68.99651611)
\curveto(775.26948689,68.98651512)(775.31948684,68.98151512)(775.36948975,68.98151611)
\curveto(775.41948674,68.98151512)(775.46948669,68.97651513)(775.51948975,68.96651611)
\curveto(775.58948657,68.94651516)(775.67448649,68.93151517)(775.77448975,68.92151611)
\curveto(775.88448628,68.92151518)(775.97448619,68.93151517)(776.04448975,68.95151611)
\curveto(776.10448606,68.97151513)(776.164486,68.97651513)(776.22448975,68.96651611)
\curveto(776.28448588,68.96651514)(776.34448582,68.97651513)(776.40448975,68.99651611)
\curveto(776.48448568,69.01651509)(776.5594856,69.03151507)(776.62948975,69.04151611)
\curveto(776.70948545,69.05151505)(776.78448538,69.07151503)(776.85448975,69.10151611)
\curveto(777.14448502,69.22151488)(777.38948477,69.36651474)(777.58948975,69.53651611)
\curveto(777.79948436,69.7065144)(777.9594842,69.93651417)(778.06948975,70.22651611)
}
}
{
\newrgbcolor{curcolor}{0 0 0}
\pscustom[linestyle=none,fillstyle=solid,fillcolor=curcolor]
{
\newpath
\moveto(788.34613037,68.81651611)
\lineto(788.34613037,68.42651611)
\curveto(788.3461225,68.3065158)(788.32112252,68.2065159)(788.27113037,68.12651611)
\curveto(788.22112262,68.05651605)(788.13612271,68.01651609)(788.01613037,68.00651611)
\lineto(787.67113037,68.00651611)
\curveto(787.61112323,68.0065161)(787.55112329,68.0015161)(787.49113037,67.99151611)
\curveto(787.4411234,67.99151611)(787.39612345,68.0015161)(787.35613037,68.02151611)
\curveto(787.26612358,68.04151606)(787.20612364,68.08151602)(787.17613037,68.14151611)
\curveto(787.13612371,68.19151591)(787.11112373,68.25151585)(787.10113037,68.32151611)
\curveto(787.10112374,68.39151571)(787.08612376,68.46151564)(787.05613037,68.53151611)
\curveto(787.0461238,68.55151555)(787.03112381,68.56651554)(787.01113037,68.57651611)
\curveto(787.00112384,68.59651551)(786.98612386,68.61651549)(786.96613037,68.63651611)
\curveto(786.86612398,68.64651546)(786.78612406,68.62651548)(786.72613037,68.57651611)
\curveto(786.67612417,68.52651558)(786.62112422,68.47651563)(786.56113037,68.42651611)
\curveto(786.36112448,68.27651583)(786.16112468,68.16151594)(785.96113037,68.08151611)
\curveto(785.78112506,68.0015161)(785.57112527,67.94151616)(785.33113037,67.90151611)
\curveto(785.10112574,67.86151624)(784.86112598,67.84151626)(784.61113037,67.84151611)
\curveto(784.37112647,67.83151627)(784.13112671,67.84651626)(783.89113037,67.88651611)
\curveto(783.65112719,67.91651619)(783.4411274,67.97151613)(783.26113037,68.05151611)
\curveto(782.7411281,68.27151583)(782.32112852,68.56651554)(782.00113037,68.93651611)
\curveto(781.68112916,69.31651479)(781.43112941,69.78651432)(781.25113037,70.34651611)
\curveto(781.21112963,70.43651367)(781.18112966,70.52651358)(781.16113037,70.61651611)
\curveto(781.15112969,70.71651339)(781.13112971,70.81651329)(781.10113037,70.91651611)
\curveto(781.09112975,70.96651314)(781.08612976,71.01651309)(781.08613037,71.06651611)
\curveto(781.08612976,71.11651299)(781.08112976,71.16651294)(781.07113037,71.21651611)
\curveto(781.05112979,71.26651284)(781.0411298,71.31651279)(781.04113037,71.36651611)
\curveto(781.05112979,71.42651268)(781.05112979,71.48151262)(781.04113037,71.53151611)
\lineto(781.04113037,71.68151611)
\curveto(781.02112982,71.73151237)(781.01112983,71.79651231)(781.01113037,71.87651611)
\curveto(781.01112983,71.95651215)(781.02112982,72.02151208)(781.04113037,72.07151611)
\lineto(781.04113037,72.23651611)
\curveto(781.06112978,72.3065118)(781.06612978,72.37651173)(781.05613037,72.44651611)
\curveto(781.05612979,72.52651158)(781.06612978,72.6015115)(781.08613037,72.67151611)
\curveto(781.09612975,72.72151138)(781.10112974,72.76651134)(781.10113037,72.80651611)
\curveto(781.10112974,72.84651126)(781.10612974,72.89151121)(781.11613037,72.94151611)
\curveto(781.1461297,73.04151106)(781.17112967,73.13651097)(781.19113037,73.22651611)
\curveto(781.21112963,73.32651078)(781.23612961,73.42151068)(781.26613037,73.51151611)
\curveto(781.39612945,73.89151021)(781.56112928,74.23150987)(781.76113037,74.53151611)
\curveto(781.97112887,74.84150926)(782.22112862,75.09650901)(782.51113037,75.29651611)
\curveto(782.68112816,75.41650869)(782.85612799,75.51650859)(783.03613037,75.59651611)
\curveto(783.22612762,75.67650843)(783.43112741,75.74650836)(783.65113037,75.80651611)
\curveto(783.72112712,75.81650829)(783.78612706,75.82650828)(783.84613037,75.83651611)
\curveto(783.91612693,75.84650826)(783.98612686,75.86150824)(784.05613037,75.88151611)
\lineto(784.20613037,75.88151611)
\curveto(784.28612656,75.9015082)(784.40112644,75.91150819)(784.55113037,75.91151611)
\curveto(784.71112613,75.91150819)(784.83112601,75.9015082)(784.91113037,75.88151611)
\curveto(784.95112589,75.87150823)(785.00612584,75.86650824)(785.07613037,75.86651611)
\curveto(785.18612566,75.83650827)(785.29612555,75.81150829)(785.40613037,75.79151611)
\curveto(785.51612533,75.78150832)(785.62112522,75.75150835)(785.72113037,75.70151611)
\curveto(785.87112497,75.64150846)(786.01112483,75.57650853)(786.14113037,75.50651611)
\curveto(786.28112456,75.43650867)(786.41112443,75.35650875)(786.53113037,75.26651611)
\curveto(786.59112425,75.21650889)(786.65112419,75.16150894)(786.71113037,75.10151611)
\curveto(786.78112406,75.05150905)(786.87112397,75.03650907)(786.98113037,75.05651611)
\curveto(787.00112384,75.08650902)(787.01612383,75.11150899)(787.02613037,75.13151611)
\curveto(787.0461238,75.15150895)(787.06112378,75.18150892)(787.07113037,75.22151611)
\curveto(787.10112374,75.31150879)(787.11112373,75.42650868)(787.10113037,75.56651611)
\lineto(787.10113037,75.94151611)
\lineto(787.10113037,77.66651611)
\lineto(787.10113037,78.13151611)
\curveto(787.10112374,78.31150579)(787.12612372,78.44150566)(787.17613037,78.52151611)
\curveto(787.21612363,78.59150551)(787.27612357,78.63650547)(787.35613037,78.65651611)
\curveto(787.37612347,78.65650545)(787.40112344,78.65650545)(787.43113037,78.65651611)
\curveto(787.46112338,78.66650544)(787.48612336,78.67150543)(787.50613037,78.67151611)
\curveto(787.6461232,78.68150542)(787.79112305,78.68150542)(787.94113037,78.67151611)
\curveto(788.10112274,78.67150543)(788.21112263,78.63150547)(788.27113037,78.55151611)
\curveto(788.32112252,78.47150563)(788.3461225,78.37150573)(788.34613037,78.25151611)
\lineto(788.34613037,77.87651611)
\lineto(788.34613037,68.81651611)
\moveto(787.13113037,71.65151611)
\curveto(787.15112369,71.7015124)(787.16112368,71.76651234)(787.16113037,71.84651611)
\curveto(787.16112368,71.93651217)(787.15112369,72.0065121)(787.13113037,72.05651611)
\lineto(787.13113037,72.28151611)
\curveto(787.11112373,72.37151173)(787.09612375,72.46151164)(787.08613037,72.55151611)
\curveto(787.07612377,72.65151145)(787.05612379,72.74151136)(787.02613037,72.82151611)
\curveto(787.00612384,72.9015112)(786.98612386,72.97651113)(786.96613037,73.04651611)
\curveto(786.95612389,73.11651099)(786.93612391,73.18651092)(786.90613037,73.25651611)
\curveto(786.78612406,73.55651055)(786.63112421,73.82151028)(786.44113037,74.05151611)
\curveto(786.25112459,74.28150982)(786.01112483,74.46150964)(785.72113037,74.59151611)
\curveto(785.62112522,74.64150946)(785.51612533,74.67650943)(785.40613037,74.69651611)
\curveto(785.30612554,74.72650938)(785.19612565,74.75150935)(785.07613037,74.77151611)
\curveto(784.99612585,74.79150931)(784.90612594,74.8015093)(784.80613037,74.80151611)
\lineto(784.53613037,74.80151611)
\curveto(784.48612636,74.79150931)(784.4411264,74.78150932)(784.40113037,74.77151611)
\lineto(784.26613037,74.77151611)
\curveto(784.18612666,74.75150935)(784.10112674,74.73150937)(784.01113037,74.71151611)
\curveto(783.93112691,74.69150941)(783.85112699,74.66650944)(783.77113037,74.63651611)
\curveto(783.45112739,74.49650961)(783.19112765,74.29150981)(782.99113037,74.02151611)
\curveto(782.80112804,73.76151034)(782.6461282,73.45651065)(782.52613037,73.10651611)
\curveto(782.48612836,72.99651111)(782.45612839,72.88151122)(782.43613037,72.76151611)
\curveto(782.42612842,72.65151145)(782.41112843,72.54151156)(782.39113037,72.43151611)
\curveto(782.39112845,72.39151171)(782.38612846,72.35151175)(782.37613037,72.31151611)
\lineto(782.37613037,72.20651611)
\curveto(782.35612849,72.15651195)(782.3461285,72.101512)(782.34613037,72.04151611)
\curveto(782.35612849,71.98151212)(782.36112848,71.92651218)(782.36113037,71.87651611)
\lineto(782.36113037,71.54651611)
\curveto(782.36112848,71.44651266)(782.37112847,71.35151275)(782.39113037,71.26151611)
\curveto(782.40112844,71.23151287)(782.40612844,71.18151292)(782.40613037,71.11151611)
\curveto(782.42612842,71.04151306)(782.4411284,70.97151313)(782.45113037,70.90151611)
\lineto(782.51113037,70.69151611)
\curveto(782.62112822,70.34151376)(782.77112807,70.04151406)(782.96113037,69.79151611)
\curveto(783.15112769,69.54151456)(783.39112745,69.33651477)(783.68113037,69.17651611)
\curveto(783.77112707,69.12651498)(783.86112698,69.08651502)(783.95113037,69.05651611)
\curveto(784.0411268,69.02651508)(784.1411267,68.99651511)(784.25113037,68.96651611)
\curveto(784.30112654,68.94651516)(784.35112649,68.94151516)(784.40113037,68.95151611)
\curveto(784.46112638,68.96151514)(784.51612633,68.95651515)(784.56613037,68.93651611)
\curveto(784.60612624,68.92651518)(784.6461262,68.92151518)(784.68613037,68.92151611)
\lineto(784.82113037,68.92151611)
\lineto(784.95613037,68.92151611)
\curveto(784.98612586,68.93151517)(785.03612581,68.93651517)(785.10613037,68.93651611)
\curveto(785.18612566,68.95651515)(785.26612558,68.97151513)(785.34613037,68.98151611)
\curveto(785.42612542,69.0015151)(785.50112534,69.02651508)(785.57113037,69.05651611)
\curveto(785.90112494,69.19651491)(786.16612468,69.37151473)(786.36613037,69.58151611)
\curveto(786.57612427,69.8015143)(786.75112409,70.07651403)(786.89113037,70.40651611)
\curveto(786.9411239,70.51651359)(786.97612387,70.62651348)(786.99613037,70.73651611)
\curveto(787.01612383,70.84651326)(787.0411238,70.95651315)(787.07113037,71.06651611)
\curveto(787.09112375,71.106513)(787.10112374,71.14151296)(787.10113037,71.17151611)
\curveto(787.10112374,71.21151289)(787.10612374,71.25151285)(787.11613037,71.29151611)
\curveto(787.12612372,71.35151275)(787.12612372,71.41151269)(787.11613037,71.47151611)
\curveto(787.11612373,71.53151257)(787.12112372,71.59151251)(787.13113037,71.65151611)
}
}
{
\newrgbcolor{curcolor}{0 0 0}
\pscustom[linestyle=none,fillstyle=solid,fillcolor=curcolor]
{
\newpath
\moveto(797.41738037,72.20651611)
\curveto(797.43737231,72.14651196)(797.4473723,72.05151205)(797.44738037,71.92151611)
\curveto(797.4473723,71.8015123)(797.44237231,71.71651239)(797.43238037,71.66651611)
\lineto(797.43238037,71.51651611)
\curveto(797.42237233,71.43651267)(797.41237234,71.36151274)(797.40238037,71.29151611)
\curveto(797.40237235,71.23151287)(797.39737235,71.16151294)(797.38738037,71.08151611)
\curveto(797.36737238,71.02151308)(797.3523724,70.96151314)(797.34238037,70.90151611)
\curveto(797.34237241,70.84151326)(797.33237242,70.78151332)(797.31238037,70.72151611)
\curveto(797.27237248,70.59151351)(797.23737251,70.46151364)(797.20738037,70.33151611)
\curveto(797.17737257,70.2015139)(797.13737261,70.08151402)(797.08738037,69.97151611)
\curveto(796.87737287,69.49151461)(796.59737315,69.08651502)(796.24738037,68.75651611)
\curveto(795.89737385,68.43651567)(795.46737428,68.19151591)(794.95738037,68.02151611)
\curveto(794.8473749,67.98151612)(794.72737502,67.95151615)(794.59738037,67.93151611)
\curveto(794.47737527,67.91151619)(794.3523754,67.89151621)(794.22238037,67.87151611)
\curveto(794.16237559,67.86151624)(794.09737565,67.85651625)(794.02738037,67.85651611)
\curveto(793.96737578,67.84651626)(793.90737584,67.84151626)(793.84738037,67.84151611)
\curveto(793.80737594,67.83151627)(793.747376,67.82651628)(793.66738037,67.82651611)
\curveto(793.59737615,67.82651628)(793.5473762,67.83151627)(793.51738037,67.84151611)
\curveto(793.47737627,67.85151625)(793.43737631,67.85651625)(793.39738037,67.85651611)
\curveto(793.35737639,67.84651626)(793.32237643,67.84651626)(793.29238037,67.85651611)
\lineto(793.20238037,67.85651611)
\lineto(792.84238037,67.90151611)
\curveto(792.70237705,67.94151616)(792.56737718,67.98151612)(792.43738037,68.02151611)
\curveto(792.30737744,68.06151604)(792.18237757,68.106516)(792.06238037,68.15651611)
\curveto(791.61237814,68.35651575)(791.24237851,68.61651549)(790.95238037,68.93651611)
\curveto(790.66237909,69.25651485)(790.42237933,69.64651446)(790.23238037,70.10651611)
\curveto(790.18237957,70.2065139)(790.14237961,70.3065138)(790.11238037,70.40651611)
\curveto(790.09237966,70.5065136)(790.07237968,70.61151349)(790.05238037,70.72151611)
\curveto(790.03237972,70.76151334)(790.02237973,70.79151331)(790.02238037,70.81151611)
\curveto(790.03237972,70.84151326)(790.03237972,70.87651323)(790.02238037,70.91651611)
\curveto(790.00237975,70.99651311)(789.98737976,71.07651303)(789.97738037,71.15651611)
\curveto(789.97737977,71.24651286)(789.96737978,71.33151277)(789.94738037,71.41151611)
\lineto(789.94738037,71.53151611)
\curveto(789.9473798,71.57151253)(789.94237981,71.61651249)(789.93238037,71.66651611)
\curveto(789.92237983,71.71651239)(789.91737983,71.8015123)(789.91738037,71.92151611)
\curveto(789.91737983,72.05151205)(789.92737982,72.14651196)(789.94738037,72.20651611)
\curveto(789.96737978,72.27651183)(789.97237978,72.34651176)(789.96238037,72.41651611)
\curveto(789.9523798,72.48651162)(789.95737979,72.55651155)(789.97738037,72.62651611)
\curveto(789.98737976,72.67651143)(789.99237976,72.71651139)(789.99238037,72.74651611)
\curveto(790.00237975,72.78651132)(790.01237974,72.83151127)(790.02238037,72.88151611)
\curveto(790.0523797,73.0015111)(790.07737967,73.12151098)(790.09738037,73.24151611)
\curveto(790.12737962,73.36151074)(790.16737958,73.47651063)(790.21738037,73.58651611)
\curveto(790.36737938,73.95651015)(790.5473792,74.28650982)(790.75738037,74.57651611)
\curveto(790.97737877,74.87650923)(791.24237851,75.12650898)(791.55238037,75.32651611)
\curveto(791.67237808,75.4065087)(791.79737795,75.47150863)(791.92738037,75.52151611)
\curveto(792.05737769,75.58150852)(792.19237756,75.64150846)(792.33238037,75.70151611)
\curveto(792.4523773,75.75150835)(792.58237717,75.78150832)(792.72238037,75.79151611)
\curveto(792.86237689,75.81150829)(793.00237675,75.84150826)(793.14238037,75.88151611)
\lineto(793.33738037,75.88151611)
\curveto(793.40737634,75.89150821)(793.47237628,75.9015082)(793.53238037,75.91151611)
\curveto(794.42237533,75.92150818)(795.16237459,75.73650837)(795.75238037,75.35651611)
\curveto(796.34237341,74.97650913)(796.76737298,74.48150962)(797.02738037,73.87151611)
\curveto(797.07737267,73.77151033)(797.11737263,73.67151043)(797.14738037,73.57151611)
\curveto(797.17737257,73.47151063)(797.21237254,73.36651074)(797.25238037,73.25651611)
\curveto(797.28237247,73.14651096)(797.30737244,73.02651108)(797.32738037,72.89651611)
\curveto(797.3473724,72.77651133)(797.37237238,72.65151145)(797.40238037,72.52151611)
\curveto(797.41237234,72.47151163)(797.41237234,72.41651169)(797.40238037,72.35651611)
\curveto(797.40237235,72.3065118)(797.40737234,72.25651185)(797.41738037,72.20651611)
\moveto(796.08238037,71.35151611)
\curveto(796.10237365,71.42151268)(796.10737364,71.5015126)(796.09738037,71.59151611)
\lineto(796.09738037,71.84651611)
\curveto(796.09737365,72.23651187)(796.06237369,72.56651154)(795.99238037,72.83651611)
\curveto(795.96237379,72.91651119)(795.93737381,72.99651111)(795.91738037,73.07651611)
\curveto(795.89737385,73.15651095)(795.87237388,73.23151087)(795.84238037,73.30151611)
\curveto(795.56237419,73.95151015)(795.11737463,74.4015097)(794.50738037,74.65151611)
\curveto(794.43737531,74.68150942)(794.36237539,74.7015094)(794.28238037,74.71151611)
\lineto(794.04238037,74.77151611)
\curveto(793.96237579,74.79150931)(793.87737587,74.8015093)(793.78738037,74.80151611)
\lineto(793.51738037,74.80151611)
\lineto(793.24738037,74.75651611)
\curveto(793.1473766,74.73650937)(793.0523767,74.71150939)(792.96238037,74.68151611)
\curveto(792.88237687,74.66150944)(792.80237695,74.63150947)(792.72238037,74.59151611)
\curveto(792.6523771,74.57150953)(792.58737716,74.54150956)(792.52738037,74.50151611)
\curveto(792.46737728,74.46150964)(792.41237734,74.42150968)(792.36238037,74.38151611)
\curveto(792.12237763,74.21150989)(791.92737782,74.0065101)(791.77738037,73.76651611)
\curveto(791.62737812,73.52651058)(791.49737825,73.24651086)(791.38738037,72.92651611)
\curveto(791.35737839,72.82651128)(791.33737841,72.72151138)(791.32738037,72.61151611)
\curveto(791.31737843,72.51151159)(791.30237845,72.4065117)(791.28238037,72.29651611)
\curveto(791.27237848,72.25651185)(791.26737848,72.19151191)(791.26738037,72.10151611)
\curveto(791.25737849,72.07151203)(791.2523785,72.03651207)(791.25238037,71.99651611)
\curveto(791.26237849,71.95651215)(791.26737848,71.91151219)(791.26738037,71.86151611)
\lineto(791.26738037,71.56151611)
\curveto(791.26737848,71.46151264)(791.27737847,71.37151273)(791.29738037,71.29151611)
\lineto(791.32738037,71.11151611)
\curveto(791.3473784,71.01151309)(791.36237839,70.91151319)(791.37238037,70.81151611)
\curveto(791.39237836,70.72151338)(791.42237833,70.63651347)(791.46238037,70.55651611)
\curveto(791.56237819,70.31651379)(791.67737807,70.09151401)(791.80738037,69.88151611)
\curveto(791.9473778,69.67151443)(792.11737763,69.49651461)(792.31738037,69.35651611)
\curveto(792.36737738,69.32651478)(792.41237734,69.3015148)(792.45238037,69.28151611)
\curveto(792.49237726,69.26151484)(792.53737721,69.23651487)(792.58738037,69.20651611)
\curveto(792.66737708,69.15651495)(792.752377,69.11151499)(792.84238037,69.07151611)
\curveto(792.94237681,69.04151506)(793.0473767,69.01151509)(793.15738037,68.98151611)
\curveto(793.20737654,68.96151514)(793.2523765,68.95151515)(793.29238037,68.95151611)
\curveto(793.34237641,68.96151514)(793.39237636,68.96151514)(793.44238037,68.95151611)
\curveto(793.47237628,68.94151516)(793.53237622,68.93151517)(793.62238037,68.92151611)
\curveto(793.72237603,68.91151519)(793.79737595,68.91651519)(793.84738037,68.93651611)
\curveto(793.88737586,68.94651516)(793.92737582,68.94651516)(793.96738037,68.93651611)
\curveto(794.00737574,68.93651517)(794.0473757,68.94651516)(794.08738037,68.96651611)
\curveto(794.16737558,68.98651512)(794.2473755,69.0015151)(794.32738037,69.01151611)
\curveto(794.40737534,69.03151507)(794.48237527,69.05651505)(794.55238037,69.08651611)
\curveto(794.89237486,69.22651488)(795.16737458,69.42151468)(795.37738037,69.67151611)
\curveto(795.58737416,69.92151418)(795.76237399,70.21651389)(795.90238037,70.55651611)
\curveto(795.9523738,70.67651343)(795.98237377,70.8015133)(795.99238037,70.93151611)
\curveto(796.01237374,71.07151303)(796.04237371,71.21151289)(796.08238037,71.35151611)
}
}
{
\newrgbcolor{curcolor}{0 0 0}
\pscustom[linestyle=none,fillstyle=solid,fillcolor=curcolor]
{
\newpath
\moveto(802.55066162,75.91151611)
\curveto(802.78065683,75.91150819)(802.9106567,75.85150825)(802.94066162,75.73151611)
\curveto(802.97065664,75.62150848)(802.98565663,75.45650865)(802.98566162,75.23651611)
\lineto(802.98566162,74.95151611)
\curveto(802.98565663,74.86150924)(802.96065665,74.78650932)(802.91066162,74.72651611)
\curveto(802.85065676,74.64650946)(802.76565685,74.6015095)(802.65566162,74.59151611)
\curveto(802.54565707,74.59150951)(802.43565718,74.57650953)(802.32566162,74.54651611)
\curveto(802.18565743,74.51650959)(802.05065756,74.48650962)(801.92066162,74.45651611)
\curveto(801.80065781,74.42650968)(801.68565793,74.38650972)(801.57566162,74.33651611)
\curveto(801.28565833,74.2065099)(801.05065856,74.02651008)(800.87066162,73.79651611)
\curveto(800.69065892,73.57651053)(800.53565908,73.32151078)(800.40566162,73.03151611)
\curveto(800.36565925,72.92151118)(800.33565928,72.8065113)(800.31566162,72.68651611)
\curveto(800.29565932,72.57651153)(800.27065934,72.46151164)(800.24066162,72.34151611)
\curveto(800.23065938,72.29151181)(800.22565939,72.24151186)(800.22566162,72.19151611)
\curveto(800.23565938,72.14151196)(800.23565938,72.09151201)(800.22566162,72.04151611)
\curveto(800.19565942,71.92151218)(800.18065943,71.78151232)(800.18066162,71.62151611)
\curveto(800.19065942,71.47151263)(800.19565942,71.32651278)(800.19566162,71.18651611)
\lineto(800.19566162,69.34151611)
\lineto(800.19566162,68.99651611)
\curveto(800.19565942,68.87651523)(800.19065942,68.76151534)(800.18066162,68.65151611)
\curveto(800.17065944,68.54151556)(800.16565945,68.44651566)(800.16566162,68.36651611)
\curveto(800.17565944,68.28651582)(800.15565946,68.21651589)(800.10566162,68.15651611)
\curveto(800.05565956,68.08651602)(799.97565964,68.04651606)(799.86566162,68.03651611)
\curveto(799.76565985,68.02651608)(799.65565996,68.02151608)(799.53566162,68.02151611)
\lineto(799.26566162,68.02151611)
\curveto(799.2156604,68.04151606)(799.16566045,68.05651605)(799.11566162,68.06651611)
\curveto(799.07566054,68.08651602)(799.04566057,68.11151599)(799.02566162,68.14151611)
\curveto(798.97566064,68.21151589)(798.94566067,68.29651581)(798.93566162,68.39651611)
\lineto(798.93566162,68.72651611)
\lineto(798.93566162,69.88151611)
\lineto(798.93566162,74.03651611)
\lineto(798.93566162,75.07151611)
\lineto(798.93566162,75.37151611)
\curveto(798.94566067,75.47150863)(798.97566064,75.55650855)(799.02566162,75.62651611)
\curveto(799.05566056,75.66650844)(799.10566051,75.69650841)(799.17566162,75.71651611)
\curveto(799.25566036,75.73650837)(799.34066027,75.74650836)(799.43066162,75.74651611)
\curveto(799.52066009,75.75650835)(799.61066,75.75650835)(799.70066162,75.74651611)
\curveto(799.79065982,75.73650837)(799.86065975,75.72150838)(799.91066162,75.70151611)
\curveto(799.99065962,75.67150843)(800.04065957,75.61150849)(800.06066162,75.52151611)
\curveto(800.09065952,75.44150866)(800.10565951,75.35150875)(800.10566162,75.25151611)
\lineto(800.10566162,74.95151611)
\curveto(800.10565951,74.85150925)(800.12565949,74.76150934)(800.16566162,74.68151611)
\curveto(800.17565944,74.66150944)(800.18565943,74.64650946)(800.19566162,74.63651611)
\lineto(800.24066162,74.59151611)
\curveto(800.35065926,74.59150951)(800.44065917,74.63650947)(800.51066162,74.72651611)
\curveto(800.58065903,74.82650928)(800.64065897,74.9065092)(800.69066162,74.96651611)
\lineto(800.78066162,75.05651611)
\curveto(800.87065874,75.16650894)(800.99565862,75.28150882)(801.15566162,75.40151611)
\curveto(801.3156583,75.52150858)(801.46565815,75.61150849)(801.60566162,75.67151611)
\curveto(801.69565792,75.72150838)(801.79065782,75.75650835)(801.89066162,75.77651611)
\curveto(801.99065762,75.8065083)(802.09565752,75.83650827)(802.20566162,75.86651611)
\curveto(802.26565735,75.87650823)(802.32565729,75.88150822)(802.38566162,75.88151611)
\curveto(802.44565717,75.89150821)(802.50065711,75.9015082)(802.55066162,75.91151611)
}
}
{
\newrgbcolor{curcolor}{0 0 0}
\pscustom[linestyle=none,fillstyle=solid,fillcolor=curcolor]
{
\newpath
\moveto(264.47920532,60.82939941)
\lineto(269.38420532,60.82939941)
\lineto(270.67420532,60.82939941)
\curveto(270.78419744,60.82938872)(270.89419733,60.82938872)(271.00420532,60.82939941)
\curveto(271.11419711,60.83938871)(271.20419702,60.81938873)(271.27420532,60.76939941)
\curveto(271.30419692,60.7493888)(271.3291969,60.72438882)(271.34920532,60.69439941)
\curveto(271.36919686,60.66438888)(271.38919684,60.63438891)(271.40920532,60.60439941)
\curveto(271.4291968,60.53438901)(271.43919679,60.41938913)(271.43920532,60.25939941)
\curveto(271.43919679,60.10938944)(271.4291968,59.99438955)(271.40920532,59.91439941)
\curveto(271.36919686,59.77438977)(271.28419694,59.69438985)(271.15420532,59.67439941)
\curveto(271.0241972,59.66438988)(270.86919736,59.65938989)(270.68920532,59.65939941)
\lineto(269.18920532,59.65939941)
\lineto(266.66920532,59.65939941)
\lineto(266.09920532,59.65939941)
\curveto(265.88920234,59.66938988)(265.73420249,59.6443899)(265.63420532,59.58439941)
\curveto(265.53420269,59.52439002)(265.47920275,59.41939013)(265.46920532,59.26939941)
\lineto(265.46920532,58.80439941)
\lineto(265.46920532,57.27439941)
\curveto(265.46920276,57.16439238)(265.46420276,57.03439251)(265.45420532,56.88439941)
\curveto(265.45420277,56.73439281)(265.46420276,56.61439293)(265.48420532,56.52439941)
\curveto(265.51420271,56.40439314)(265.57420265,56.32439322)(265.66420532,56.28439941)
\curveto(265.70420252,56.26439328)(265.77420245,56.2443933)(265.87420532,56.22439941)
\lineto(266.02420532,56.22439941)
\curveto(266.06420216,56.21439333)(266.10420212,56.20939334)(266.14420532,56.20939941)
\curveto(266.19420203,56.21939333)(266.24420198,56.22439332)(266.29420532,56.22439941)
\lineto(266.80420532,56.22439941)
\lineto(269.74420532,56.22439941)
\lineto(270.04420532,56.22439941)
\curveto(270.15419807,56.23439331)(270.26419796,56.23439331)(270.37420532,56.22439941)
\curveto(270.49419773,56.22439332)(270.59919763,56.21439333)(270.68920532,56.19439941)
\curveto(270.78919744,56.18439336)(270.86419736,56.16439338)(270.91420532,56.13439941)
\curveto(270.94419728,56.11439343)(270.96919726,56.06939348)(270.98920532,55.99939941)
\curveto(271.00919722,55.92939362)(271.0241972,55.85439369)(271.03420532,55.77439941)
\curveto(271.04419718,55.69439385)(271.04419718,55.60939394)(271.03420532,55.51939941)
\curveto(271.03419719,55.43939411)(271.0241972,55.36939418)(271.00420532,55.30939941)
\curveto(270.98419724,55.21939433)(270.93919729,55.15439439)(270.86920532,55.11439941)
\curveto(270.84919738,55.09439445)(270.81919741,55.07939447)(270.77920532,55.06939941)
\curveto(270.74919748,55.06939448)(270.71919751,55.06439448)(270.68920532,55.05439941)
\lineto(270.59920532,55.05439941)
\curveto(270.54919768,55.0443945)(270.49919773,55.03939451)(270.44920532,55.03939941)
\curveto(270.39919783,55.0493945)(270.34919788,55.05439449)(270.29920532,55.05439941)
\lineto(269.74420532,55.05439941)
\lineto(266.57920532,55.05439941)
\lineto(266.21920532,55.05439941)
\curveto(266.10920212,55.06439448)(266.00420222,55.05939449)(265.90420532,55.03939941)
\curveto(265.80420242,55.02939452)(265.71420251,55.00439454)(265.63420532,54.96439941)
\curveto(265.56420266,54.92439462)(265.51420271,54.85439469)(265.48420532,54.75439941)
\curveto(265.46420276,54.69439485)(265.45420277,54.62439492)(265.45420532,54.54439941)
\curveto(265.46420276,54.46439508)(265.46920276,54.38439516)(265.46920532,54.30439941)
\lineto(265.46920532,53.46439941)
\lineto(265.46920532,52.03939941)
\curveto(265.46920276,51.89939765)(265.47420275,51.76939778)(265.48420532,51.64939941)
\curveto(265.49420273,51.53939801)(265.53420269,51.45939809)(265.60420532,51.40939941)
\curveto(265.67420255,51.35939819)(265.75420247,51.32939822)(265.84420532,51.31939941)
\lineto(266.14420532,51.31939941)
\lineto(267.10420532,51.31939941)
\lineto(269.87920532,51.31939941)
\lineto(270.73420532,51.31939941)
\lineto(270.97420532,51.31939941)
\curveto(271.05419717,51.32939822)(271.1241971,51.32439822)(271.18420532,51.30439941)
\curveto(271.30419692,51.26439828)(271.38419684,51.20939834)(271.42420532,51.13939941)
\curveto(271.44419678,51.10939844)(271.45919677,51.05939849)(271.46920532,50.98939941)
\curveto(271.47919675,50.91939863)(271.48419674,50.8443987)(271.48420532,50.76439941)
\curveto(271.49419673,50.69439885)(271.49419673,50.61939893)(271.48420532,50.53939941)
\curveto(271.47419675,50.46939908)(271.46419676,50.41439913)(271.45420532,50.37439941)
\curveto(271.41419681,50.29439925)(271.36919686,50.23939931)(271.31920532,50.20939941)
\curveto(271.25919697,50.16939938)(271.17919705,50.1493994)(271.07920532,50.14939941)
\lineto(270.80920532,50.14939941)
\lineto(269.75920532,50.14939941)
\lineto(265.76920532,50.14939941)
\lineto(264.71920532,50.14939941)
\curveto(264.57920365,50.1493994)(264.45920377,50.15439939)(264.35920532,50.16439941)
\curveto(264.25920397,50.18439936)(264.18420404,50.23439931)(264.13420532,50.31439941)
\curveto(264.09420413,50.37439917)(264.07420415,50.4493991)(264.07420532,50.53939941)
\lineto(264.07420532,50.82439941)
\lineto(264.07420532,51.87439941)
\lineto(264.07420532,55.89439941)
\lineto(264.07420532,59.25439941)
\lineto(264.07420532,60.18439941)
\lineto(264.07420532,60.45439941)
\curveto(264.07420415,60.544389)(264.09420413,60.61438893)(264.13420532,60.66439941)
\curveto(264.17420405,60.73438881)(264.24920398,60.78438876)(264.35920532,60.81439941)
\curveto(264.37920385,60.82438872)(264.39920383,60.82438872)(264.41920532,60.81439941)
\curveto(264.43920379,60.81438873)(264.45920377,60.81938873)(264.47920532,60.82939941)
}
}
{
\newrgbcolor{curcolor}{0 0 0}
\pscustom[linestyle=none,fillstyle=solid,fillcolor=curcolor]
{
\newpath
\moveto(275.4191272,58.05439941)
\curveto(276.13912313,58.06439148)(276.74412253,57.97939157)(277.2341272,57.79939941)
\curveto(277.72412155,57.62939192)(278.10412117,57.32439222)(278.3741272,56.88439941)
\curveto(278.44412083,56.77439277)(278.49912077,56.65939289)(278.5391272,56.53939941)
\curveto(278.57912069,56.42939312)(278.61912065,56.30439324)(278.6591272,56.16439941)
\curveto(278.67912059,56.09439345)(278.68412059,56.01939353)(278.6741272,55.93939941)
\curveto(278.66412061,55.86939368)(278.64912062,55.81439373)(278.6291272,55.77439941)
\curveto(278.60912066,55.75439379)(278.58412069,55.73439381)(278.5541272,55.71439941)
\curveto(278.52412075,55.70439384)(278.49912077,55.68939386)(278.4791272,55.66939941)
\curveto(278.42912084,55.6493939)(278.37912089,55.6443939)(278.3291272,55.65439941)
\curveto(278.27912099,55.66439388)(278.22912104,55.66439388)(278.1791272,55.65439941)
\curveto(278.09912117,55.63439391)(277.99412128,55.62939392)(277.8641272,55.63939941)
\curveto(277.73412154,55.65939389)(277.64412163,55.68439386)(277.5941272,55.71439941)
\curveto(277.51412176,55.76439378)(277.45912181,55.82939372)(277.4291272,55.90939941)
\curveto(277.40912186,55.99939355)(277.3741219,56.08439346)(277.3241272,56.16439941)
\curveto(277.23412204,56.32439322)(277.10912216,56.46939308)(276.9491272,56.59939941)
\curveto(276.83912243,56.67939287)(276.71912255,56.73939281)(276.5891272,56.77939941)
\curveto(276.45912281,56.81939273)(276.31912295,56.85939269)(276.1691272,56.89939941)
\curveto(276.11912315,56.91939263)(276.0691232,56.92439262)(276.0191272,56.91439941)
\curveto(275.9691233,56.91439263)(275.91912335,56.91939263)(275.8691272,56.92939941)
\curveto(275.80912346,56.9493926)(275.73412354,56.95939259)(275.6441272,56.95939941)
\curveto(275.55412372,56.95939259)(275.47912379,56.9493926)(275.4191272,56.92939941)
\lineto(275.3291272,56.92939941)
\lineto(275.1791272,56.89939941)
\curveto(275.12912414,56.89939265)(275.07912419,56.89439265)(275.0291272,56.88439941)
\curveto(274.7691245,56.82439272)(274.55412472,56.73939281)(274.3841272,56.62939941)
\curveto(274.21412506,56.51939303)(274.09912517,56.33439321)(274.0391272,56.07439941)
\curveto(274.01912525,56.00439354)(274.01412526,55.93439361)(274.0241272,55.86439941)
\curveto(274.04412523,55.79439375)(274.06412521,55.73439381)(274.0841272,55.68439941)
\curveto(274.14412513,55.53439401)(274.21412506,55.42439412)(274.2941272,55.35439941)
\curveto(274.38412489,55.29439425)(274.49412478,55.22439432)(274.6241272,55.14439941)
\curveto(274.78412449,55.0443945)(274.96412431,54.96939458)(275.1641272,54.91939941)
\curveto(275.36412391,54.87939467)(275.56412371,54.82939472)(275.7641272,54.76939941)
\curveto(275.89412338,54.72939482)(276.02412325,54.69939485)(276.1541272,54.67939941)
\curveto(276.28412299,54.65939489)(276.41412286,54.62939492)(276.5441272,54.58939941)
\curveto(276.75412252,54.52939502)(276.95912231,54.46939508)(277.1591272,54.40939941)
\curveto(277.35912191,54.35939519)(277.55912171,54.29439525)(277.7591272,54.21439941)
\lineto(277.9091272,54.15439941)
\curveto(277.95912131,54.13439541)(278.00912126,54.10939544)(278.0591272,54.07939941)
\curveto(278.25912101,53.95939559)(278.43412084,53.82439572)(278.5841272,53.67439941)
\curveto(278.73412054,53.52439602)(278.85912041,53.33439621)(278.9591272,53.10439941)
\curveto(278.97912029,53.03439651)(278.99912027,52.93939661)(279.0191272,52.81939941)
\curveto(279.03912023,52.7493968)(279.04912022,52.67439687)(279.0491272,52.59439941)
\curveto(279.05912021,52.52439702)(279.06412021,52.4443971)(279.0641272,52.35439941)
\lineto(279.0641272,52.20439941)
\curveto(279.04412023,52.13439741)(279.03412024,52.06439748)(279.0341272,51.99439941)
\curveto(279.03412024,51.92439762)(279.02412025,51.85439769)(279.0041272,51.78439941)
\curveto(278.9741203,51.67439787)(278.93912033,51.56939798)(278.8991272,51.46939941)
\curveto(278.85912041,51.36939818)(278.81412046,51.27939827)(278.7641272,51.19939941)
\curveto(278.60412067,50.93939861)(278.39912087,50.72939882)(278.1491272,50.56939941)
\curveto(277.89912137,50.41939913)(277.61912165,50.28939926)(277.3091272,50.17939941)
\curveto(277.21912205,50.1493994)(277.12412215,50.12939942)(277.0241272,50.11939941)
\curveto(276.93412234,50.09939945)(276.84412243,50.07439947)(276.7541272,50.04439941)
\curveto(276.65412262,50.02439952)(276.55412272,50.01439953)(276.4541272,50.01439941)
\curveto(276.35412292,50.01439953)(276.25412302,50.00439954)(276.1541272,49.98439941)
\lineto(276.0041272,49.98439941)
\curveto(275.95412332,49.97439957)(275.88412339,49.96939958)(275.7941272,49.96939941)
\curveto(275.70412357,49.96939958)(275.63412364,49.97439957)(275.5841272,49.98439941)
\lineto(275.4191272,49.98439941)
\curveto(275.35912391,50.00439954)(275.29412398,50.01439953)(275.2241272,50.01439941)
\curveto(275.15412412,50.00439954)(275.09412418,50.00939954)(275.0441272,50.02939941)
\curveto(274.99412428,50.03939951)(274.92912434,50.0443995)(274.8491272,50.04439941)
\lineto(274.6091272,50.10439941)
\curveto(274.53912473,50.11439943)(274.46412481,50.13439941)(274.3841272,50.16439941)
\curveto(274.0741252,50.26439928)(273.80412547,50.38939916)(273.5741272,50.53939941)
\curveto(273.34412593,50.68939886)(273.14412613,50.88439866)(272.9741272,51.12439941)
\curveto(272.88412639,51.25439829)(272.80912646,51.38939816)(272.7491272,51.52939941)
\curveto(272.68912658,51.66939788)(272.63412664,51.82439772)(272.5841272,51.99439941)
\curveto(272.56412671,52.05439749)(272.55412672,52.12439742)(272.5541272,52.20439941)
\curveto(272.56412671,52.29439725)(272.57912669,52.36439718)(272.5991272,52.41439941)
\curveto(272.62912664,52.45439709)(272.67912659,52.49439705)(272.7491272,52.53439941)
\curveto(272.79912647,52.55439699)(272.8691264,52.56439698)(272.9591272,52.56439941)
\curveto(273.04912622,52.57439697)(273.13912613,52.57439697)(273.2291272,52.56439941)
\curveto(273.31912595,52.55439699)(273.40412587,52.53939701)(273.4841272,52.51939941)
\curveto(273.5741257,52.50939704)(273.63412564,52.49439705)(273.6641272,52.47439941)
\curveto(273.73412554,52.42439712)(273.77912549,52.3493972)(273.7991272,52.24939941)
\curveto(273.82912544,52.15939739)(273.86412541,52.07439747)(273.9041272,51.99439941)
\curveto(274.00412527,51.77439777)(274.13912513,51.60439794)(274.3091272,51.48439941)
\curveto(274.42912484,51.39439815)(274.56412471,51.32439822)(274.7141272,51.27439941)
\curveto(274.86412441,51.22439832)(275.02412425,51.17439837)(275.1941272,51.12439941)
\lineto(275.5091272,51.07939941)
\lineto(275.5991272,51.07939941)
\curveto(275.6691236,51.05939849)(275.75912351,51.0493985)(275.8691272,51.04939941)
\curveto(275.98912328,51.0493985)(276.08912318,51.05939849)(276.1691272,51.07939941)
\curveto(276.23912303,51.07939847)(276.29412298,51.08439846)(276.3341272,51.09439941)
\curveto(276.39412288,51.10439844)(276.45412282,51.10939844)(276.5141272,51.10939941)
\curveto(276.5741227,51.11939843)(276.62912264,51.12939842)(276.6791272,51.13939941)
\curveto(276.9691223,51.21939833)(277.19912207,51.32439822)(277.3691272,51.45439941)
\curveto(277.53912173,51.58439796)(277.65912161,51.80439774)(277.7291272,52.11439941)
\curveto(277.74912152,52.16439738)(277.75412152,52.21939733)(277.7441272,52.27939941)
\curveto(277.73412154,52.33939721)(277.72412155,52.38439716)(277.7141272,52.41439941)
\curveto(277.66412161,52.60439694)(277.59412168,52.7443968)(277.5041272,52.83439941)
\curveto(277.41412186,52.93439661)(277.29912197,53.02439652)(277.1591272,53.10439941)
\curveto(277.0691222,53.16439638)(276.9691223,53.21439633)(276.8591272,53.25439941)
\lineto(276.5291272,53.37439941)
\curveto(276.49912277,53.38439616)(276.4691228,53.38939616)(276.4391272,53.38939941)
\curveto(276.41912285,53.38939616)(276.39412288,53.39939615)(276.3641272,53.41939941)
\curveto(276.02412325,53.52939602)(275.6691236,53.60939594)(275.2991272,53.65939941)
\curveto(274.93912433,53.71939583)(274.59912467,53.81439573)(274.2791272,53.94439941)
\curveto(274.17912509,53.98439556)(274.08412519,54.01939553)(273.9941272,54.04939941)
\curveto(273.90412537,54.07939547)(273.81912545,54.11939543)(273.7391272,54.16939941)
\curveto(273.54912572,54.27939527)(273.3741259,54.40439514)(273.2141272,54.54439941)
\curveto(273.05412622,54.68439486)(272.92912634,54.85939469)(272.8391272,55.06939941)
\curveto(272.80912646,55.13939441)(272.78412649,55.20939434)(272.7641272,55.27939941)
\curveto(272.75412652,55.3493942)(272.73912653,55.42439412)(272.7191272,55.50439941)
\curveto(272.68912658,55.62439392)(272.67912659,55.75939379)(272.6891272,55.90939941)
\curveto(272.69912657,56.06939348)(272.71412656,56.20439334)(272.7341272,56.31439941)
\curveto(272.75412652,56.36439318)(272.76412651,56.40439314)(272.7641272,56.43439941)
\curveto(272.7741265,56.47439307)(272.78912648,56.51439303)(272.8091272,56.55439941)
\curveto(272.89912637,56.78439276)(273.01912625,56.98439256)(273.1691272,57.15439941)
\curveto(273.32912594,57.32439222)(273.50912576,57.47439207)(273.7091272,57.60439941)
\curveto(273.85912541,57.69439185)(274.02412525,57.76439178)(274.2041272,57.81439941)
\curveto(274.38412489,57.87439167)(274.5741247,57.92939162)(274.7741272,57.97939941)
\curveto(274.84412443,57.98939156)(274.90912436,57.99939155)(274.9691272,58.00939941)
\curveto(275.03912423,58.01939153)(275.11412416,58.02939152)(275.1941272,58.03939941)
\curveto(275.22412405,58.0493915)(275.26412401,58.0493915)(275.3141272,58.03939941)
\curveto(275.36412391,58.02939152)(275.39912387,58.03439151)(275.4191272,58.05439941)
}
}
{
\newrgbcolor{curcolor}{0 0 0}
\pscustom[linestyle=none,fillstyle=solid,fillcolor=curcolor]
{
\newpath
\moveto(281.4341272,60.21439941)
\curveto(281.58412519,60.21438933)(281.73412504,60.20938934)(281.8841272,60.19939941)
\curveto(282.03412474,60.19938935)(282.13912463,60.15938939)(282.1991272,60.07939941)
\curveto(282.24912452,60.01938953)(282.2741245,59.93438961)(282.2741272,59.82439941)
\curveto(282.28412449,59.72438982)(282.28912448,59.61938993)(282.2891272,59.50939941)
\lineto(282.2891272,58.63939941)
\curveto(282.28912448,58.55939099)(282.28412449,58.47439107)(282.2741272,58.38439941)
\curveto(282.2741245,58.30439124)(282.28412449,58.23439131)(282.3041272,58.17439941)
\curveto(282.34412443,58.03439151)(282.43412434,57.9443916)(282.5741272,57.90439941)
\curveto(282.62412415,57.89439165)(282.6691241,57.88939166)(282.7091272,57.88939941)
\lineto(282.8591272,57.88939941)
\lineto(283.2641272,57.88939941)
\curveto(283.42412335,57.89939165)(283.53912323,57.88939166)(283.6091272,57.85939941)
\curveto(283.69912307,57.79939175)(283.75912301,57.73939181)(283.7891272,57.67939941)
\curveto(283.80912296,57.63939191)(283.81912295,57.59439195)(283.8191272,57.54439941)
\lineto(283.8191272,57.39439941)
\curveto(283.81912295,57.28439226)(283.81412296,57.17939237)(283.8041272,57.07939941)
\curveto(283.79412298,56.98939256)(283.75912301,56.91939263)(283.6991272,56.86939941)
\curveto(283.63912313,56.81939273)(283.55412322,56.78939276)(283.4441272,56.77939941)
\lineto(283.1141272,56.77939941)
\curveto(283.00412377,56.78939276)(282.89412388,56.79439275)(282.7841272,56.79439941)
\curveto(282.6741241,56.79439275)(282.57912419,56.77939277)(282.4991272,56.74939941)
\curveto(282.42912434,56.71939283)(282.37912439,56.66939288)(282.3491272,56.59939941)
\curveto(282.31912445,56.52939302)(282.29912447,56.4443931)(282.2891272,56.34439941)
\curveto(282.27912449,56.25439329)(282.2741245,56.15439339)(282.2741272,56.04439941)
\curveto(282.28412449,55.9443936)(282.28912448,55.8443937)(282.2891272,55.74439941)
\lineto(282.2891272,52.77439941)
\curveto(282.28912448,52.55439699)(282.28412449,52.31939723)(282.2741272,52.06939941)
\curveto(282.2741245,51.82939772)(282.31912445,51.6443979)(282.4091272,51.51439941)
\curveto(282.45912431,51.43439811)(282.52412425,51.37939817)(282.6041272,51.34939941)
\curveto(282.68412409,51.31939823)(282.77912399,51.29439825)(282.8891272,51.27439941)
\curveto(282.91912385,51.26439828)(282.94912382,51.25939829)(282.9791272,51.25939941)
\curveto(283.01912375,51.26939828)(283.05412372,51.26939828)(283.0841272,51.25939941)
\lineto(283.2791272,51.25939941)
\curveto(283.37912339,51.25939829)(283.4691233,51.2493983)(283.5491272,51.22939941)
\curveto(283.63912313,51.21939833)(283.70412307,51.18439836)(283.7441272,51.12439941)
\curveto(283.76412301,51.09439845)(283.77912299,51.03939851)(283.7891272,50.95939941)
\curveto(283.80912296,50.88939866)(283.81912295,50.81439873)(283.8191272,50.73439941)
\curveto(283.82912294,50.65439889)(283.82912294,50.57439897)(283.8191272,50.49439941)
\curveto(283.80912296,50.42439912)(283.78912298,50.36939918)(283.7591272,50.32939941)
\curveto(283.71912305,50.25939929)(283.64412313,50.20939934)(283.5341272,50.17939941)
\curveto(283.45412332,50.15939939)(283.36412341,50.1493994)(283.2641272,50.14939941)
\curveto(283.16412361,50.15939939)(283.0741237,50.16439938)(282.9941272,50.16439941)
\curveto(282.93412384,50.16439938)(282.8741239,50.15939939)(282.8141272,50.14939941)
\curveto(282.75412402,50.1493994)(282.69912407,50.15439939)(282.6491272,50.16439941)
\lineto(282.4691272,50.16439941)
\curveto(282.41912435,50.17439937)(282.3691244,50.17939937)(282.3191272,50.17939941)
\curveto(282.27912449,50.18939936)(282.23412454,50.19439935)(282.1841272,50.19439941)
\curveto(281.98412479,50.2443993)(281.80912496,50.29939925)(281.6591272,50.35939941)
\curveto(281.51912525,50.41939913)(281.39912537,50.52439902)(281.2991272,50.67439941)
\curveto(281.15912561,50.87439867)(281.07912569,51.12439842)(281.0591272,51.42439941)
\curveto(281.03912573,51.73439781)(281.02912574,52.06439748)(281.0291272,52.41439941)
\lineto(281.0291272,56.34439941)
\curveto(280.99912577,56.47439307)(280.9691258,56.56939298)(280.9391272,56.62939941)
\curveto(280.91912585,56.68939286)(280.84912592,56.73939281)(280.7291272,56.77939941)
\curveto(280.68912608,56.78939276)(280.64912612,56.78939276)(280.6091272,56.77939941)
\curveto(280.5691262,56.76939278)(280.52912624,56.77439277)(280.4891272,56.79439941)
\lineto(280.2491272,56.79439941)
\curveto(280.11912665,56.79439275)(280.00912676,56.80439274)(279.9191272,56.82439941)
\curveto(279.83912693,56.85439269)(279.78412699,56.91439263)(279.7541272,57.00439941)
\curveto(279.73412704,57.0443925)(279.71912705,57.08939246)(279.7091272,57.13939941)
\lineto(279.7091272,57.28939941)
\curveto(279.70912706,57.42939212)(279.71912705,57.544392)(279.7391272,57.63439941)
\curveto(279.75912701,57.73439181)(279.81912695,57.80939174)(279.9191272,57.85939941)
\curveto(280.02912674,57.89939165)(280.1691266,57.90939164)(280.3391272,57.88939941)
\curveto(280.51912625,57.86939168)(280.6691261,57.87939167)(280.7891272,57.91939941)
\curveto(280.87912589,57.96939158)(280.94912582,58.03939151)(280.9991272,58.12939941)
\curveto(281.01912575,58.18939136)(281.02912574,58.26439128)(281.0291272,58.35439941)
\lineto(281.0291272,58.60939941)
\lineto(281.0291272,59.53939941)
\lineto(281.0291272,59.77939941)
\curveto(281.02912574,59.86938968)(281.03912573,59.9443896)(281.0591272,60.00439941)
\curveto(281.09912567,60.08438946)(281.1741256,60.1493894)(281.2841272,60.19939941)
\curveto(281.31412546,60.19938935)(281.33912543,60.19938935)(281.3591272,60.19939941)
\curveto(281.38912538,60.20938934)(281.41412536,60.21438933)(281.4341272,60.21439941)
}
}
{
\newrgbcolor{curcolor}{0 0 0}
\pscustom[linestyle=none,fillstyle=solid,fillcolor=curcolor]
{
\newpath
\moveto(285.67092407,57.87439941)
\lineto(286.10592407,57.87439941)
\curveto(286.25592211,57.87439167)(286.360922,57.83439171)(286.42092407,57.75439941)
\curveto(286.47092189,57.67439187)(286.49592187,57.57439197)(286.49592407,57.45439941)
\curveto(286.50592186,57.33439221)(286.51092185,57.21439233)(286.51092407,57.09439941)
\lineto(286.51092407,55.66939941)
\lineto(286.51092407,53.40439941)
\lineto(286.51092407,52.71439941)
\curveto(286.51092185,52.48439706)(286.53592183,52.28439726)(286.58592407,52.11439941)
\curveto(286.74592162,51.66439788)(287.04592132,51.3493982)(287.48592407,51.16939941)
\curveto(287.70592066,51.07939847)(287.97092039,51.0443985)(288.28092407,51.06439941)
\curveto(288.59091977,51.09439845)(288.84091952,51.1493984)(289.03092407,51.22939941)
\curveto(289.360919,51.36939818)(289.62091874,51.544398)(289.81092407,51.75439941)
\curveto(290.01091835,51.97439757)(290.1659182,52.25939729)(290.27592407,52.60939941)
\curveto(290.30591806,52.68939686)(290.32591804,52.76939678)(290.33592407,52.84939941)
\curveto(290.34591802,52.92939662)(290.360918,53.01439653)(290.38092407,53.10439941)
\curveto(290.39091797,53.15439639)(290.39091797,53.19939635)(290.38092407,53.23939941)
\curveto(290.38091798,53.27939627)(290.39091797,53.32439622)(290.41092407,53.37439941)
\lineto(290.41092407,53.68939941)
\curveto(290.43091793,53.76939578)(290.43591793,53.85939569)(290.42592407,53.95939941)
\curveto(290.41591795,54.06939548)(290.41091795,54.16939538)(290.41092407,54.25939941)
\lineto(290.41092407,55.42939941)
\lineto(290.41092407,57.01939941)
\curveto(290.41091795,57.13939241)(290.40591796,57.26439228)(290.39592407,57.39439941)
\curveto(290.39591797,57.53439201)(290.42091794,57.6443919)(290.47092407,57.72439941)
\curveto(290.51091785,57.77439177)(290.55591781,57.80439174)(290.60592407,57.81439941)
\curveto(290.6659177,57.83439171)(290.73591763,57.85439169)(290.81592407,57.87439941)
\lineto(291.04092407,57.87439941)
\curveto(291.1609172,57.87439167)(291.2659171,57.86939168)(291.35592407,57.85939941)
\curveto(291.45591691,57.8493917)(291.53091683,57.80439174)(291.58092407,57.72439941)
\curveto(291.63091673,57.67439187)(291.65591671,57.59939195)(291.65592407,57.49939941)
\lineto(291.65592407,57.21439941)
\lineto(291.65592407,56.19439941)
\lineto(291.65592407,52.15939941)
\lineto(291.65592407,50.80939941)
\curveto(291.65591671,50.68939886)(291.65091671,50.57439897)(291.64092407,50.46439941)
\curveto(291.64091672,50.36439918)(291.60591676,50.28939926)(291.53592407,50.23939941)
\curveto(291.49591687,50.20939934)(291.43591693,50.18439936)(291.35592407,50.16439941)
\curveto(291.27591709,50.15439939)(291.18591718,50.1443994)(291.08592407,50.13439941)
\curveto(290.99591737,50.13439941)(290.90591746,50.13939941)(290.81592407,50.14939941)
\curveto(290.73591763,50.15939939)(290.67591769,50.17939937)(290.63592407,50.20939941)
\curveto(290.58591778,50.2493993)(290.54091782,50.31439923)(290.50092407,50.40439941)
\curveto(290.49091787,50.4443991)(290.48091788,50.49939905)(290.47092407,50.56939941)
\curveto(290.47091789,50.63939891)(290.4659179,50.70439884)(290.45592407,50.76439941)
\curveto(290.44591792,50.83439871)(290.42591794,50.88939866)(290.39592407,50.92939941)
\curveto(290.365918,50.96939858)(290.32091804,50.98439856)(290.26092407,50.97439941)
\curveto(290.18091818,50.95439859)(290.10091826,50.89439865)(290.02092407,50.79439941)
\curveto(289.94091842,50.70439884)(289.8659185,50.63439891)(289.79592407,50.58439941)
\curveto(289.57591879,50.42439912)(289.32591904,50.28439926)(289.04592407,50.16439941)
\curveto(288.93591943,50.11439943)(288.82091954,50.08439946)(288.70092407,50.07439941)
\curveto(288.59091977,50.05439949)(288.47591989,50.02939952)(288.35592407,49.99939941)
\curveto(288.30592006,49.98939956)(288.25092011,49.98939956)(288.19092407,49.99939941)
\curveto(288.14092022,50.00939954)(288.09092027,50.00439954)(288.04092407,49.98439941)
\curveto(287.94092042,49.96439958)(287.85092051,49.96439958)(287.77092407,49.98439941)
\lineto(287.62092407,49.98439941)
\curveto(287.57092079,50.00439954)(287.51092085,50.01439953)(287.44092407,50.01439941)
\curveto(287.38092098,50.01439953)(287.32592104,50.01939953)(287.27592407,50.02939941)
\curveto(287.23592113,50.0493995)(287.19592117,50.05939949)(287.15592407,50.05939941)
\curveto(287.12592124,50.0493995)(287.08592128,50.05439949)(287.03592407,50.07439941)
\lineto(286.79592407,50.13439941)
\curveto(286.72592164,50.15439939)(286.65092171,50.18439936)(286.57092407,50.22439941)
\curveto(286.31092205,50.33439921)(286.09092227,50.47939907)(285.91092407,50.65939941)
\curveto(285.74092262,50.8493987)(285.60092276,51.07439847)(285.49092407,51.33439941)
\curveto(285.45092291,51.42439812)(285.42092294,51.51439803)(285.40092407,51.60439941)
\lineto(285.34092407,51.90439941)
\curveto(285.32092304,51.96439758)(285.31092305,52.01939753)(285.31092407,52.06939941)
\curveto(285.32092304,52.12939742)(285.31592305,52.19439735)(285.29592407,52.26439941)
\curveto(285.28592308,52.28439726)(285.28092308,52.30939724)(285.28092407,52.33939941)
\curveto(285.28092308,52.37939717)(285.27592309,52.41439713)(285.26592407,52.44439941)
\lineto(285.26592407,52.59439941)
\curveto(285.25592311,52.63439691)(285.25092311,52.67939687)(285.25092407,52.72939941)
\curveto(285.2609231,52.78939676)(285.2659231,52.8443967)(285.26592407,52.89439941)
\lineto(285.26592407,53.49439941)
\lineto(285.26592407,56.25439941)
\lineto(285.26592407,57.21439941)
\lineto(285.26592407,57.48439941)
\curveto(285.2659231,57.57439197)(285.28592308,57.6493919)(285.32592407,57.70939941)
\curveto(285.365923,57.77939177)(285.44092292,57.82939172)(285.55092407,57.85939941)
\curveto(285.57092279,57.86939168)(285.59092277,57.86939168)(285.61092407,57.85939941)
\curveto(285.63092273,57.85939169)(285.65092271,57.86439168)(285.67092407,57.87439941)
}
}
{
\newrgbcolor{curcolor}{0 0 0}
\pscustom[linestyle=none,fillstyle=solid,fillcolor=curcolor]
{
\newpath
\moveto(300.51553345,50.95939941)
\lineto(300.51553345,50.56939941)
\curveto(300.51552557,50.4493991)(300.4905256,50.3493992)(300.44053345,50.26939941)
\curveto(300.3905257,50.19939935)(300.30552578,50.15939939)(300.18553345,50.14939941)
\lineto(299.84053345,50.14939941)
\curveto(299.78052631,50.1493994)(299.72052637,50.1443994)(299.66053345,50.13439941)
\curveto(299.61052648,50.13439941)(299.56552652,50.1443994)(299.52553345,50.16439941)
\curveto(299.43552665,50.18439936)(299.37552671,50.22439932)(299.34553345,50.28439941)
\curveto(299.30552678,50.33439921)(299.28052681,50.39439915)(299.27053345,50.46439941)
\curveto(299.27052682,50.53439901)(299.25552683,50.60439894)(299.22553345,50.67439941)
\curveto(299.21552687,50.69439885)(299.20052689,50.70939884)(299.18053345,50.71939941)
\curveto(299.17052692,50.73939881)(299.15552693,50.75939879)(299.13553345,50.77939941)
\curveto(299.03552705,50.78939876)(298.95552713,50.76939878)(298.89553345,50.71939941)
\curveto(298.84552724,50.66939888)(298.7905273,50.61939893)(298.73053345,50.56939941)
\curveto(298.53052756,50.41939913)(298.33052776,50.30439924)(298.13053345,50.22439941)
\curveto(297.95052814,50.1443994)(297.74052835,50.08439946)(297.50053345,50.04439941)
\curveto(297.27052882,50.00439954)(297.03052906,49.98439956)(296.78053345,49.98439941)
\curveto(296.54052955,49.97439957)(296.30052979,49.98939956)(296.06053345,50.02939941)
\curveto(295.82053027,50.05939949)(295.61053048,50.11439943)(295.43053345,50.19439941)
\curveto(294.91053118,50.41439913)(294.4905316,50.70939884)(294.17053345,51.07939941)
\curveto(293.85053224,51.45939809)(293.60053249,51.92939762)(293.42053345,52.48939941)
\curveto(293.38053271,52.57939697)(293.35053274,52.66939688)(293.33053345,52.75939941)
\curveto(293.32053277,52.85939669)(293.30053279,52.95939659)(293.27053345,53.05939941)
\curveto(293.26053283,53.10939644)(293.25553283,53.15939639)(293.25553345,53.20939941)
\curveto(293.25553283,53.25939629)(293.25053284,53.30939624)(293.24053345,53.35939941)
\curveto(293.22053287,53.40939614)(293.21053288,53.45939609)(293.21053345,53.50939941)
\curveto(293.22053287,53.56939598)(293.22053287,53.62439592)(293.21053345,53.67439941)
\lineto(293.21053345,53.82439941)
\curveto(293.1905329,53.87439567)(293.18053291,53.93939561)(293.18053345,54.01939941)
\curveto(293.18053291,54.09939545)(293.1905329,54.16439538)(293.21053345,54.21439941)
\lineto(293.21053345,54.37939941)
\curveto(293.23053286,54.4493951)(293.23553285,54.51939503)(293.22553345,54.58939941)
\curveto(293.22553286,54.66939488)(293.23553285,54.7443948)(293.25553345,54.81439941)
\curveto(293.26553282,54.86439468)(293.27053282,54.90939464)(293.27053345,54.94939941)
\curveto(293.27053282,54.98939456)(293.27553281,55.03439451)(293.28553345,55.08439941)
\curveto(293.31553277,55.18439436)(293.34053275,55.27939427)(293.36053345,55.36939941)
\curveto(293.38053271,55.46939408)(293.40553268,55.56439398)(293.43553345,55.65439941)
\curveto(293.56553252,56.03439351)(293.73053236,56.37439317)(293.93053345,56.67439941)
\curveto(294.14053195,56.98439256)(294.3905317,57.23939231)(294.68053345,57.43939941)
\curveto(294.85053124,57.55939199)(295.02553106,57.65939189)(295.20553345,57.73939941)
\curveto(295.39553069,57.81939173)(295.60053049,57.88939166)(295.82053345,57.94939941)
\curveto(295.8905302,57.95939159)(295.95553013,57.96939158)(296.01553345,57.97939941)
\curveto(296.08553,57.98939156)(296.15552993,58.00439154)(296.22553345,58.02439941)
\lineto(296.37553345,58.02439941)
\curveto(296.45552963,58.0443915)(296.57052952,58.05439149)(296.72053345,58.05439941)
\curveto(296.88052921,58.05439149)(297.00052909,58.0443915)(297.08053345,58.02439941)
\curveto(297.12052897,58.01439153)(297.17552891,58.00939154)(297.24553345,58.00939941)
\curveto(297.35552873,57.97939157)(297.46552862,57.95439159)(297.57553345,57.93439941)
\curveto(297.6855284,57.92439162)(297.7905283,57.89439165)(297.89053345,57.84439941)
\curveto(298.04052805,57.78439176)(298.18052791,57.71939183)(298.31053345,57.64939941)
\curveto(298.45052764,57.57939197)(298.58052751,57.49939205)(298.70053345,57.40939941)
\curveto(298.76052733,57.35939219)(298.82052727,57.30439224)(298.88053345,57.24439941)
\curveto(298.95052714,57.19439235)(299.04052705,57.17939237)(299.15053345,57.19939941)
\curveto(299.17052692,57.22939232)(299.1855269,57.25439229)(299.19553345,57.27439941)
\curveto(299.21552687,57.29439225)(299.23052686,57.32439222)(299.24053345,57.36439941)
\curveto(299.27052682,57.45439209)(299.28052681,57.56939198)(299.27053345,57.70939941)
\lineto(299.27053345,58.08439941)
\lineto(299.27053345,59.80939941)
\lineto(299.27053345,60.27439941)
\curveto(299.27052682,60.45438909)(299.29552679,60.58438896)(299.34553345,60.66439941)
\curveto(299.3855267,60.73438881)(299.44552664,60.77938877)(299.52553345,60.79939941)
\curveto(299.54552654,60.79938875)(299.57052652,60.79938875)(299.60053345,60.79939941)
\curveto(299.63052646,60.80938874)(299.65552643,60.81438873)(299.67553345,60.81439941)
\curveto(299.81552627,60.82438872)(299.96052613,60.82438872)(300.11053345,60.81439941)
\curveto(300.27052582,60.81438873)(300.38052571,60.77438877)(300.44053345,60.69439941)
\curveto(300.4905256,60.61438893)(300.51552557,60.51438903)(300.51553345,60.39439941)
\lineto(300.51553345,60.01939941)
\lineto(300.51553345,50.95939941)
\moveto(299.30053345,53.79439941)
\curveto(299.32052677,53.8443957)(299.33052676,53.90939564)(299.33053345,53.98939941)
\curveto(299.33052676,54.07939547)(299.32052677,54.1493954)(299.30053345,54.19939941)
\lineto(299.30053345,54.42439941)
\curveto(299.28052681,54.51439503)(299.26552682,54.60439494)(299.25553345,54.69439941)
\curveto(299.24552684,54.79439475)(299.22552686,54.88439466)(299.19553345,54.96439941)
\curveto(299.17552691,55.0443945)(299.15552693,55.11939443)(299.13553345,55.18939941)
\curveto(299.12552696,55.25939429)(299.10552698,55.32939422)(299.07553345,55.39939941)
\curveto(298.95552713,55.69939385)(298.80052729,55.96439358)(298.61053345,56.19439941)
\curveto(298.42052767,56.42439312)(298.18052791,56.60439294)(297.89053345,56.73439941)
\curveto(297.7905283,56.78439276)(297.6855284,56.81939273)(297.57553345,56.83939941)
\curveto(297.47552861,56.86939268)(297.36552872,56.89439265)(297.24553345,56.91439941)
\curveto(297.16552892,56.93439261)(297.07552901,56.9443926)(296.97553345,56.94439941)
\lineto(296.70553345,56.94439941)
\curveto(296.65552943,56.93439261)(296.61052948,56.92439262)(296.57053345,56.91439941)
\lineto(296.43553345,56.91439941)
\curveto(296.35552973,56.89439265)(296.27052982,56.87439267)(296.18053345,56.85439941)
\curveto(296.10052999,56.83439271)(296.02053007,56.80939274)(295.94053345,56.77939941)
\curveto(295.62053047,56.63939291)(295.36053073,56.43439311)(295.16053345,56.16439941)
\curveto(294.97053112,55.90439364)(294.81553127,55.59939395)(294.69553345,55.24939941)
\curveto(294.65553143,55.13939441)(294.62553146,55.02439452)(294.60553345,54.90439941)
\curveto(294.59553149,54.79439475)(294.58053151,54.68439486)(294.56053345,54.57439941)
\curveto(294.56053153,54.53439501)(294.55553153,54.49439505)(294.54553345,54.45439941)
\lineto(294.54553345,54.34939941)
\curveto(294.52553156,54.29939525)(294.51553157,54.2443953)(294.51553345,54.18439941)
\curveto(294.52553156,54.12439542)(294.53053156,54.06939548)(294.53053345,54.01939941)
\lineto(294.53053345,53.68939941)
\curveto(294.53053156,53.58939596)(294.54053155,53.49439605)(294.56053345,53.40439941)
\curveto(294.57053152,53.37439617)(294.57553151,53.32439622)(294.57553345,53.25439941)
\curveto(294.59553149,53.18439636)(294.61053148,53.11439643)(294.62053345,53.04439941)
\lineto(294.68053345,52.83439941)
\curveto(294.7905313,52.48439706)(294.94053115,52.18439736)(295.13053345,51.93439941)
\curveto(295.32053077,51.68439786)(295.56053053,51.47939807)(295.85053345,51.31939941)
\curveto(295.94053015,51.26939828)(296.03053006,51.22939832)(296.12053345,51.19939941)
\curveto(296.21052988,51.16939838)(296.31052978,51.13939841)(296.42053345,51.10939941)
\curveto(296.47052962,51.08939846)(296.52052957,51.08439846)(296.57053345,51.09439941)
\curveto(296.63052946,51.10439844)(296.6855294,51.09939845)(296.73553345,51.07939941)
\curveto(296.77552931,51.06939848)(296.81552927,51.06439848)(296.85553345,51.06439941)
\lineto(296.99053345,51.06439941)
\lineto(297.12553345,51.06439941)
\curveto(297.15552893,51.07439847)(297.20552888,51.07939847)(297.27553345,51.07939941)
\curveto(297.35552873,51.09939845)(297.43552865,51.11439843)(297.51553345,51.12439941)
\curveto(297.59552849,51.1443984)(297.67052842,51.16939838)(297.74053345,51.19939941)
\curveto(298.07052802,51.33939821)(298.33552775,51.51439803)(298.53553345,51.72439941)
\curveto(298.74552734,51.9443976)(298.92052717,52.21939733)(299.06053345,52.54939941)
\curveto(299.11052698,52.65939689)(299.14552694,52.76939678)(299.16553345,52.87939941)
\curveto(299.1855269,52.98939656)(299.21052688,53.09939645)(299.24053345,53.20939941)
\curveto(299.26052683,53.2493963)(299.27052682,53.28439626)(299.27053345,53.31439941)
\curveto(299.27052682,53.35439619)(299.27552681,53.39439615)(299.28553345,53.43439941)
\curveto(299.29552679,53.49439605)(299.29552679,53.55439599)(299.28553345,53.61439941)
\curveto(299.2855268,53.67439587)(299.2905268,53.73439581)(299.30053345,53.79439941)
}
}
{
\newrgbcolor{curcolor}{0 0 0}
\pscustom[linestyle=none,fillstyle=solid,fillcolor=curcolor]
{
\newpath
\moveto(302.74678345,59.37439941)
\curveto(302.66678233,59.43439011)(302.62178237,59.53939001)(302.61178345,59.68939941)
\lineto(302.61178345,60.15439941)
\lineto(302.61178345,60.40939941)
\curveto(302.61178238,60.49938905)(302.62678237,60.57438897)(302.65678345,60.63439941)
\curveto(302.6967823,60.71438883)(302.77678222,60.77438877)(302.89678345,60.81439941)
\curveto(302.91678208,60.82438872)(302.93678206,60.82438872)(302.95678345,60.81439941)
\curveto(302.98678201,60.81438873)(303.01178198,60.81938873)(303.03178345,60.82939941)
\curveto(303.20178179,60.82938872)(303.36178163,60.82438872)(303.51178345,60.81439941)
\curveto(303.66178133,60.80438874)(303.76178123,60.7443888)(303.81178345,60.63439941)
\curveto(303.84178115,60.57438897)(303.85678114,60.49938905)(303.85678345,60.40939941)
\lineto(303.85678345,60.15439941)
\curveto(303.85678114,59.97438957)(303.85178114,59.80438974)(303.84178345,59.64439941)
\curveto(303.84178115,59.48439006)(303.77678122,59.37939017)(303.64678345,59.32939941)
\curveto(303.5967814,59.30939024)(303.54178145,59.29939025)(303.48178345,59.29939941)
\lineto(303.31678345,59.29939941)
\lineto(303.00178345,59.29939941)
\curveto(302.90178209,59.29939025)(302.81678218,59.32439022)(302.74678345,59.37439941)
\moveto(303.85678345,50.86939941)
\lineto(303.85678345,50.55439941)
\curveto(303.86678113,50.45439909)(303.84678115,50.37439917)(303.79678345,50.31439941)
\curveto(303.76678123,50.25439929)(303.72178127,50.21439933)(303.66178345,50.19439941)
\curveto(303.60178139,50.18439936)(303.53178146,50.16939938)(303.45178345,50.14939941)
\lineto(303.22678345,50.14939941)
\curveto(303.0967819,50.1493994)(302.98178201,50.15439939)(302.88178345,50.16439941)
\curveto(302.7917822,50.18439936)(302.72178227,50.23439931)(302.67178345,50.31439941)
\curveto(302.63178236,50.37439917)(302.61178238,50.4493991)(302.61178345,50.53939941)
\lineto(302.61178345,50.82439941)
\lineto(302.61178345,57.16939941)
\lineto(302.61178345,57.48439941)
\curveto(302.61178238,57.59439195)(302.63678236,57.67939187)(302.68678345,57.73939941)
\curveto(302.71678228,57.78939176)(302.75678224,57.81939173)(302.80678345,57.82939941)
\curveto(302.85678214,57.83939171)(302.91178208,57.85439169)(302.97178345,57.87439941)
\curveto(302.991782,57.87439167)(303.01178198,57.86939168)(303.03178345,57.85939941)
\curveto(303.06178193,57.85939169)(303.08678191,57.86439168)(303.10678345,57.87439941)
\curveto(303.23678176,57.87439167)(303.36678163,57.86939168)(303.49678345,57.85939941)
\curveto(303.63678136,57.85939169)(303.73178126,57.81939173)(303.78178345,57.73939941)
\curveto(303.83178116,57.67939187)(303.85678114,57.59939195)(303.85678345,57.49939941)
\lineto(303.85678345,57.21439941)
\lineto(303.85678345,50.86939941)
}
}
{
\newrgbcolor{curcolor}{0 0 0}
\pscustom[linestyle=none,fillstyle=solid,fillcolor=curcolor]
{
\newpath
\moveto(312.6866272,50.70439941)
\curveto(312.71661937,50.544399)(312.70161938,50.40939914)(312.6416272,50.29939941)
\curveto(312.5816195,50.19939935)(312.50161958,50.12439942)(312.4016272,50.07439941)
\curveto(312.35161973,50.05439949)(312.29661979,50.0443995)(312.2366272,50.04439941)
\curveto(312.1866199,50.0443995)(312.13161995,50.03439951)(312.0716272,50.01439941)
\curveto(311.85162023,49.96439958)(311.63162045,49.97939957)(311.4116272,50.05939941)
\curveto(311.20162088,50.12939942)(311.05662103,50.21939933)(310.9766272,50.32939941)
\curveto(310.92662116,50.39939915)(310.8816212,50.47939907)(310.8416272,50.56939941)
\curveto(310.80162128,50.66939888)(310.75162133,50.7493988)(310.6916272,50.80939941)
\curveto(310.67162141,50.82939872)(310.64662144,50.8493987)(310.6166272,50.86939941)
\curveto(310.59662149,50.88939866)(310.56662152,50.89439865)(310.5266272,50.88439941)
\curveto(310.41662167,50.85439869)(310.31162177,50.79939875)(310.2116272,50.71939941)
\curveto(310.12162196,50.63939891)(310.03162205,50.56939898)(309.9416272,50.50939941)
\curveto(309.81162227,50.42939912)(309.67162241,50.35439919)(309.5216272,50.28439941)
\curveto(309.37162271,50.22439932)(309.21162287,50.16939938)(309.0416272,50.11939941)
\curveto(308.94162314,50.08939946)(308.83162325,50.06939948)(308.7116272,50.05939941)
\curveto(308.60162348,50.0493995)(308.49162359,50.03439951)(308.3816272,50.01439941)
\curveto(308.33162375,50.00439954)(308.2866238,49.99939955)(308.2466272,49.99939941)
\lineto(308.1416272,49.99939941)
\curveto(308.03162405,49.97939957)(307.92662416,49.97939957)(307.8266272,49.99939941)
\lineto(307.6916272,49.99939941)
\curveto(307.64162444,50.00939954)(307.59162449,50.01439953)(307.5416272,50.01439941)
\curveto(307.49162459,50.01439953)(307.44662464,50.02439952)(307.4066272,50.04439941)
\curveto(307.36662472,50.05439949)(307.33162475,50.05939949)(307.3016272,50.05939941)
\curveto(307.2816248,50.0493995)(307.25662483,50.0493995)(307.2266272,50.05939941)
\lineto(306.9866272,50.11939941)
\curveto(306.90662518,50.12939942)(306.83162525,50.1493994)(306.7616272,50.17939941)
\curveto(306.46162562,50.30939924)(306.21662587,50.45439909)(306.0266272,50.61439941)
\curveto(305.84662624,50.78439876)(305.69662639,51.01939853)(305.5766272,51.31939941)
\curveto(305.4866266,51.53939801)(305.44162664,51.80439774)(305.4416272,52.11439941)
\lineto(305.4416272,52.42939941)
\curveto(305.45162663,52.47939707)(305.45662663,52.52939702)(305.4566272,52.57939941)
\lineto(305.4866272,52.75939941)
\lineto(305.6066272,53.08939941)
\curveto(305.64662644,53.19939635)(305.69662639,53.29939625)(305.7566272,53.38939941)
\curveto(305.93662615,53.67939587)(306.1816259,53.89439565)(306.4916272,54.03439941)
\curveto(306.80162528,54.17439537)(307.14162494,54.29939525)(307.5116272,54.40939941)
\curveto(307.65162443,54.4493951)(307.79662429,54.47939507)(307.9466272,54.49939941)
\curveto(308.09662399,54.51939503)(308.24662384,54.544395)(308.3966272,54.57439941)
\curveto(308.46662362,54.59439495)(308.53162355,54.60439494)(308.5916272,54.60439941)
\curveto(308.66162342,54.60439494)(308.73662335,54.61439493)(308.8166272,54.63439941)
\curveto(308.8866232,54.65439489)(308.95662313,54.66439488)(309.0266272,54.66439941)
\curveto(309.09662299,54.67439487)(309.17162291,54.68939486)(309.2516272,54.70939941)
\curveto(309.50162258,54.76939478)(309.73662235,54.81939473)(309.9566272,54.85939941)
\curveto(310.17662191,54.90939464)(310.35162173,55.02439452)(310.4816272,55.20439941)
\curveto(310.54162154,55.28439426)(310.59162149,55.38439416)(310.6316272,55.50439941)
\curveto(310.67162141,55.63439391)(310.67162141,55.77439377)(310.6316272,55.92439941)
\curveto(310.57162151,56.16439338)(310.4816216,56.35439319)(310.3616272,56.49439941)
\curveto(310.25162183,56.63439291)(310.09162199,56.7443928)(309.8816272,56.82439941)
\curveto(309.76162232,56.87439267)(309.61662247,56.90939264)(309.4466272,56.92939941)
\curveto(309.2866228,56.9493926)(309.11662297,56.95939259)(308.9366272,56.95939941)
\curveto(308.75662333,56.95939259)(308.5816235,56.9493926)(308.4116272,56.92939941)
\curveto(308.24162384,56.90939264)(308.09662399,56.87939267)(307.9766272,56.83939941)
\curveto(307.80662428,56.77939277)(307.64162444,56.69439285)(307.4816272,56.58439941)
\curveto(307.40162468,56.52439302)(307.32662476,56.4443931)(307.2566272,56.34439941)
\curveto(307.19662489,56.25439329)(307.14162494,56.15439339)(307.0916272,56.04439941)
\curveto(307.06162502,55.96439358)(307.03162505,55.87939367)(307.0016272,55.78939941)
\curveto(306.9816251,55.69939385)(306.93662515,55.62939392)(306.8666272,55.57939941)
\curveto(306.82662526,55.549394)(306.75662533,55.52439402)(306.6566272,55.50439941)
\curveto(306.56662552,55.49439405)(306.47162561,55.48939406)(306.3716272,55.48939941)
\curveto(306.27162581,55.48939406)(306.17162591,55.49439405)(306.0716272,55.50439941)
\curveto(305.9816261,55.52439402)(305.91662617,55.549394)(305.8766272,55.57939941)
\curveto(305.83662625,55.60939394)(305.80662628,55.65939389)(305.7866272,55.72939941)
\curveto(305.76662632,55.79939375)(305.76662632,55.87439367)(305.7866272,55.95439941)
\curveto(305.81662627,56.08439346)(305.84662624,56.20439334)(305.8766272,56.31439941)
\curveto(305.91662617,56.43439311)(305.96162612,56.549393)(306.0116272,56.65939941)
\curveto(306.20162588,57.00939254)(306.44162564,57.27939227)(306.7316272,57.46939941)
\curveto(307.02162506,57.66939188)(307.3816247,57.82939172)(307.8116272,57.94939941)
\curveto(307.91162417,57.96939158)(308.01162407,57.98439156)(308.1116272,57.99439941)
\curveto(308.22162386,58.00439154)(308.33162375,58.01939153)(308.4416272,58.03939941)
\curveto(308.4816236,58.0493915)(308.54662354,58.0493915)(308.6366272,58.03939941)
\curveto(308.72662336,58.03939151)(308.7816233,58.0493915)(308.8016272,58.06939941)
\curveto(309.50162258,58.07939147)(310.11162197,57.99939155)(310.6316272,57.82939941)
\curveto(311.15162093,57.65939189)(311.51662057,57.33439221)(311.7266272,56.85439941)
\curveto(311.81662027,56.65439289)(311.86662022,56.41939313)(311.8766272,56.14939941)
\curveto(311.89662019,55.88939366)(311.90662018,55.61439393)(311.9066272,55.32439941)
\lineto(311.9066272,52.00939941)
\curveto(311.90662018,51.86939768)(311.91162017,51.73439781)(311.9216272,51.60439941)
\curveto(311.93162015,51.47439807)(311.96162012,51.36939818)(312.0116272,51.28939941)
\curveto(312.06162002,51.21939833)(312.12661996,51.16939838)(312.2066272,51.13939941)
\curveto(312.29661979,51.09939845)(312.3816197,51.06939848)(312.4616272,51.04939941)
\curveto(312.54161954,51.03939851)(312.60161948,50.99439855)(312.6416272,50.91439941)
\curveto(312.66161942,50.88439866)(312.67161941,50.85439869)(312.6716272,50.82439941)
\curveto(312.67161941,50.79439875)(312.67661941,50.75439879)(312.6866272,50.70439941)
\moveto(310.5416272,52.36939941)
\curveto(310.60162148,52.50939704)(310.63162145,52.66939688)(310.6316272,52.84939941)
\curveto(310.64162144,53.03939651)(310.64662144,53.23439631)(310.6466272,53.43439941)
\curveto(310.64662144,53.544396)(310.64162144,53.6443959)(310.6316272,53.73439941)
\curveto(310.62162146,53.82439572)(310.5816215,53.89439565)(310.5116272,53.94439941)
\curveto(310.4816216,53.96439558)(310.41162167,53.97439557)(310.3016272,53.97439941)
\curveto(310.2816218,53.95439559)(310.24662184,53.9443956)(310.1966272,53.94439941)
\curveto(310.14662194,53.9443956)(310.10162198,53.93439561)(310.0616272,53.91439941)
\curveto(309.9816221,53.89439565)(309.89162219,53.87439567)(309.7916272,53.85439941)
\lineto(309.4916272,53.79439941)
\curveto(309.46162262,53.79439575)(309.42662266,53.78939576)(309.3866272,53.77939941)
\lineto(309.2816272,53.77939941)
\curveto(309.13162295,53.73939581)(308.96662312,53.71439583)(308.7866272,53.70439941)
\curveto(308.61662347,53.70439584)(308.45662363,53.68439586)(308.3066272,53.64439941)
\curveto(308.22662386,53.62439592)(308.15162393,53.60439594)(308.0816272,53.58439941)
\curveto(308.02162406,53.57439597)(307.95162413,53.55939599)(307.8716272,53.53939941)
\curveto(307.71162437,53.48939606)(307.56162452,53.42439612)(307.4216272,53.34439941)
\curveto(307.2816248,53.27439627)(307.16162492,53.18439636)(307.0616272,53.07439941)
\curveto(306.96162512,52.96439658)(306.8866252,52.82939672)(306.8366272,52.66939941)
\curveto(306.7866253,52.51939703)(306.76662532,52.33439721)(306.7766272,52.11439941)
\curveto(306.77662531,52.01439753)(306.79162529,51.91939763)(306.8216272,51.82939941)
\curveto(306.86162522,51.7493978)(306.90662518,51.67439787)(306.9566272,51.60439941)
\curveto(307.03662505,51.49439805)(307.14162494,51.39939815)(307.2716272,51.31939941)
\curveto(307.40162468,51.2493983)(307.54162454,51.18939836)(307.6916272,51.13939941)
\curveto(307.74162434,51.12939842)(307.79162429,51.12439842)(307.8416272,51.12439941)
\curveto(307.89162419,51.12439842)(307.94162414,51.11939843)(307.9916272,51.10939941)
\curveto(308.06162402,51.08939846)(308.14662394,51.07439847)(308.2466272,51.06439941)
\curveto(308.35662373,51.06439848)(308.44662364,51.07439847)(308.5166272,51.09439941)
\curveto(308.57662351,51.11439843)(308.63662345,51.11939843)(308.6966272,51.10939941)
\curveto(308.75662333,51.10939844)(308.81662327,51.11939843)(308.8766272,51.13939941)
\curveto(308.95662313,51.15939839)(309.03162305,51.17439837)(309.1016272,51.18439941)
\curveto(309.1816229,51.19439835)(309.25662283,51.21439833)(309.3266272,51.24439941)
\curveto(309.61662247,51.36439818)(309.86162222,51.50939804)(310.0616272,51.67939941)
\curveto(310.27162181,51.8493977)(310.43162165,52.07939747)(310.5416272,52.36939941)
}
}
{
\newrgbcolor{curcolor}{0 0 0}
\pscustom[linestyle=none,fillstyle=solid,fillcolor=curcolor]
{
\newpath
\moveto(317.54826782,58.02439941)
\curveto(318.17826259,58.0443915)(318.68326208,57.95939159)(319.06326782,57.76939941)
\curveto(319.44326132,57.57939197)(319.74826102,57.29439225)(319.97826782,56.91439941)
\curveto(320.03826073,56.81439273)(320.08326068,56.70439284)(320.11326782,56.58439941)
\curveto(320.15326061,56.47439307)(320.18826058,56.35939319)(320.21826782,56.23939941)
\curveto(320.2682605,56.0493935)(320.29826047,55.8443937)(320.30826782,55.62439941)
\curveto(320.31826045,55.40439414)(320.32326044,55.17939437)(320.32326782,54.94939941)
\lineto(320.32326782,53.34439941)
\lineto(320.32326782,51.00439941)
\curveto(320.32326044,50.83439871)(320.31826045,50.66439888)(320.30826782,50.49439941)
\curveto(320.30826046,50.32439922)(320.24326052,50.21439933)(320.11326782,50.16439941)
\curveto(320.0632607,50.1443994)(320.00826076,50.13439941)(319.94826782,50.13439941)
\curveto(319.89826087,50.12439942)(319.84326092,50.11939943)(319.78326782,50.11939941)
\curveto(319.65326111,50.11939943)(319.52826124,50.12439942)(319.40826782,50.13439941)
\curveto(319.28826148,50.13439941)(319.20326156,50.17439937)(319.15326782,50.25439941)
\curveto(319.10326166,50.32439922)(319.07826169,50.41439913)(319.07826782,50.52439941)
\lineto(319.07826782,50.85439941)
\lineto(319.07826782,52.14439941)
\lineto(319.07826782,54.58939941)
\curveto(319.07826169,54.85939469)(319.07326169,55.12439442)(319.06326782,55.38439941)
\curveto(319.05326171,55.65439389)(319.00826176,55.88439366)(318.92826782,56.07439941)
\curveto(318.84826192,56.27439327)(318.72826204,56.43439311)(318.56826782,56.55439941)
\curveto(318.40826236,56.68439286)(318.22326254,56.78439276)(318.01326782,56.85439941)
\curveto(317.95326281,56.87439267)(317.88826288,56.88439266)(317.81826782,56.88439941)
\curveto(317.75826301,56.89439265)(317.69826307,56.90939264)(317.63826782,56.92939941)
\curveto(317.58826318,56.93939261)(317.50826326,56.93939261)(317.39826782,56.92939941)
\curveto(317.29826347,56.92939262)(317.22826354,56.92439262)(317.18826782,56.91439941)
\curveto(317.14826362,56.89439265)(317.11326365,56.88439266)(317.08326782,56.88439941)
\curveto(317.05326371,56.89439265)(317.01826375,56.89439265)(316.97826782,56.88439941)
\curveto(316.84826392,56.85439269)(316.72326404,56.81939273)(316.60326782,56.77939941)
\curveto(316.49326427,56.7493928)(316.38826438,56.70439284)(316.28826782,56.64439941)
\curveto(316.24826452,56.62439292)(316.21326455,56.60439294)(316.18326782,56.58439941)
\curveto(316.15326461,56.56439298)(316.11826465,56.544393)(316.07826782,56.52439941)
\curveto(315.72826504,56.27439327)(315.47326529,55.89939365)(315.31326782,55.39939941)
\curveto(315.28326548,55.31939423)(315.2632655,55.23439431)(315.25326782,55.14439941)
\curveto(315.24326552,55.06439448)(315.22826554,54.98439456)(315.20826782,54.90439941)
\curveto(315.18826558,54.85439469)(315.18326558,54.80439474)(315.19326782,54.75439941)
\curveto(315.20326556,54.71439483)(315.19826557,54.67439487)(315.17826782,54.63439941)
\lineto(315.17826782,54.31939941)
\curveto(315.1682656,54.28939526)(315.1632656,54.25439529)(315.16326782,54.21439941)
\curveto(315.17326559,54.17439537)(315.17826559,54.12939542)(315.17826782,54.07939941)
\lineto(315.17826782,53.62939941)
\lineto(315.17826782,52.18939941)
\lineto(315.17826782,50.86939941)
\lineto(315.17826782,50.52439941)
\curveto(315.17826559,50.41439913)(315.15326561,50.32439922)(315.10326782,50.25439941)
\curveto(315.05326571,50.17439937)(314.9632658,50.13439941)(314.83326782,50.13439941)
\curveto(314.71326605,50.12439942)(314.58826618,50.11939943)(314.45826782,50.11939941)
\curveto(314.37826639,50.11939943)(314.30326646,50.12439942)(314.23326782,50.13439941)
\curveto(314.1632666,50.1443994)(314.10326666,50.16939938)(314.05326782,50.20939941)
\curveto(313.97326679,50.25939929)(313.93326683,50.35439919)(313.93326782,50.49439941)
\lineto(313.93326782,50.89939941)
\lineto(313.93326782,52.66939941)
\lineto(313.93326782,56.29939941)
\lineto(313.93326782,57.21439941)
\lineto(313.93326782,57.48439941)
\curveto(313.93326683,57.57439197)(313.95326681,57.6443919)(313.99326782,57.69439941)
\curveto(314.02326674,57.75439179)(314.07326669,57.79439175)(314.14326782,57.81439941)
\curveto(314.18326658,57.82439172)(314.23826653,57.83439171)(314.30826782,57.84439941)
\curveto(314.38826638,57.85439169)(314.4682663,57.85939169)(314.54826782,57.85939941)
\curveto(314.62826614,57.85939169)(314.70326606,57.85439169)(314.77326782,57.84439941)
\curveto(314.85326591,57.83439171)(314.90826586,57.81939173)(314.93826782,57.79939941)
\curveto(315.04826572,57.72939182)(315.09826567,57.63939191)(315.08826782,57.52939941)
\curveto(315.07826569,57.42939212)(315.09326567,57.31439223)(315.13326782,57.18439941)
\curveto(315.15326561,57.12439242)(315.19326557,57.07439247)(315.25326782,57.03439941)
\curveto(315.37326539,57.02439252)(315.4682653,57.06939248)(315.53826782,57.16939941)
\curveto(315.61826515,57.26939228)(315.69826507,57.3493922)(315.77826782,57.40939941)
\curveto(315.91826485,57.50939204)(316.05826471,57.59939195)(316.19826782,57.67939941)
\curveto(316.34826442,57.76939178)(316.51826425,57.8443917)(316.70826782,57.90439941)
\curveto(316.78826398,57.93439161)(316.87326389,57.95439159)(316.96326782,57.96439941)
\curveto(317.0632637,57.97439157)(317.15826361,57.98939156)(317.24826782,58.00939941)
\curveto(317.29826347,58.01939153)(317.34826342,58.02439152)(317.39826782,58.02439941)
\lineto(317.54826782,58.02439941)
}
}
{
\newrgbcolor{curcolor}{0 0 0}
\pscustom[linestyle=none,fillstyle=solid,fillcolor=curcolor]
{
\newpath
\moveto(323.1528772,60.21439941)
\curveto(323.30287519,60.21438933)(323.45287504,60.20938934)(323.6028772,60.19939941)
\curveto(323.75287474,60.19938935)(323.85787463,60.15938939)(323.9178772,60.07939941)
\curveto(323.96787452,60.01938953)(323.9928745,59.93438961)(323.9928772,59.82439941)
\curveto(324.00287449,59.72438982)(324.00787448,59.61938993)(324.0078772,59.50939941)
\lineto(324.0078772,58.63939941)
\curveto(324.00787448,58.55939099)(324.00287449,58.47439107)(323.9928772,58.38439941)
\curveto(323.9928745,58.30439124)(324.00287449,58.23439131)(324.0228772,58.17439941)
\curveto(324.06287443,58.03439151)(324.15287434,57.9443916)(324.2928772,57.90439941)
\curveto(324.34287415,57.89439165)(324.3878741,57.88939166)(324.4278772,57.88939941)
\lineto(324.5778772,57.88939941)
\lineto(324.9828772,57.88939941)
\curveto(325.14287335,57.89939165)(325.25787323,57.88939166)(325.3278772,57.85939941)
\curveto(325.41787307,57.79939175)(325.47787301,57.73939181)(325.5078772,57.67939941)
\curveto(325.52787296,57.63939191)(325.53787295,57.59439195)(325.5378772,57.54439941)
\lineto(325.5378772,57.39439941)
\curveto(325.53787295,57.28439226)(325.53287296,57.17939237)(325.5228772,57.07939941)
\curveto(325.51287298,56.98939256)(325.47787301,56.91939263)(325.4178772,56.86939941)
\curveto(325.35787313,56.81939273)(325.27287322,56.78939276)(325.1628772,56.77939941)
\lineto(324.8328772,56.77939941)
\curveto(324.72287377,56.78939276)(324.61287388,56.79439275)(324.5028772,56.79439941)
\curveto(324.3928741,56.79439275)(324.29787419,56.77939277)(324.2178772,56.74939941)
\curveto(324.14787434,56.71939283)(324.09787439,56.66939288)(324.0678772,56.59939941)
\curveto(324.03787445,56.52939302)(324.01787447,56.4443931)(324.0078772,56.34439941)
\curveto(323.99787449,56.25439329)(323.9928745,56.15439339)(323.9928772,56.04439941)
\curveto(324.00287449,55.9443936)(324.00787448,55.8443937)(324.0078772,55.74439941)
\lineto(324.0078772,52.77439941)
\curveto(324.00787448,52.55439699)(324.00287449,52.31939723)(323.9928772,52.06939941)
\curveto(323.9928745,51.82939772)(324.03787445,51.6443979)(324.1278772,51.51439941)
\curveto(324.17787431,51.43439811)(324.24287425,51.37939817)(324.3228772,51.34939941)
\curveto(324.40287409,51.31939823)(324.49787399,51.29439825)(324.6078772,51.27439941)
\curveto(324.63787385,51.26439828)(324.66787382,51.25939829)(324.6978772,51.25939941)
\curveto(324.73787375,51.26939828)(324.77287372,51.26939828)(324.8028772,51.25939941)
\lineto(324.9978772,51.25939941)
\curveto(325.09787339,51.25939829)(325.1878733,51.2493983)(325.2678772,51.22939941)
\curveto(325.35787313,51.21939833)(325.42287307,51.18439836)(325.4628772,51.12439941)
\curveto(325.48287301,51.09439845)(325.49787299,51.03939851)(325.5078772,50.95939941)
\curveto(325.52787296,50.88939866)(325.53787295,50.81439873)(325.5378772,50.73439941)
\curveto(325.54787294,50.65439889)(325.54787294,50.57439897)(325.5378772,50.49439941)
\curveto(325.52787296,50.42439912)(325.50787298,50.36939918)(325.4778772,50.32939941)
\curveto(325.43787305,50.25939929)(325.36287313,50.20939934)(325.2528772,50.17939941)
\curveto(325.17287332,50.15939939)(325.08287341,50.1493994)(324.9828772,50.14939941)
\curveto(324.88287361,50.15939939)(324.7928737,50.16439938)(324.7128772,50.16439941)
\curveto(324.65287384,50.16439938)(324.5928739,50.15939939)(324.5328772,50.14939941)
\curveto(324.47287402,50.1493994)(324.41787407,50.15439939)(324.3678772,50.16439941)
\lineto(324.1878772,50.16439941)
\curveto(324.13787435,50.17439937)(324.0878744,50.17939937)(324.0378772,50.17939941)
\curveto(323.99787449,50.18939936)(323.95287454,50.19439935)(323.9028772,50.19439941)
\curveto(323.70287479,50.2443993)(323.52787496,50.29939925)(323.3778772,50.35939941)
\curveto(323.23787525,50.41939913)(323.11787537,50.52439902)(323.0178772,50.67439941)
\curveto(322.87787561,50.87439867)(322.79787569,51.12439842)(322.7778772,51.42439941)
\curveto(322.75787573,51.73439781)(322.74787574,52.06439748)(322.7478772,52.41439941)
\lineto(322.7478772,56.34439941)
\curveto(322.71787577,56.47439307)(322.6878758,56.56939298)(322.6578772,56.62939941)
\curveto(322.63787585,56.68939286)(322.56787592,56.73939281)(322.4478772,56.77939941)
\curveto(322.40787608,56.78939276)(322.36787612,56.78939276)(322.3278772,56.77939941)
\curveto(322.2878762,56.76939278)(322.24787624,56.77439277)(322.2078772,56.79439941)
\lineto(321.9678772,56.79439941)
\curveto(321.83787665,56.79439275)(321.72787676,56.80439274)(321.6378772,56.82439941)
\curveto(321.55787693,56.85439269)(321.50287699,56.91439263)(321.4728772,57.00439941)
\curveto(321.45287704,57.0443925)(321.43787705,57.08939246)(321.4278772,57.13939941)
\lineto(321.4278772,57.28939941)
\curveto(321.42787706,57.42939212)(321.43787705,57.544392)(321.4578772,57.63439941)
\curveto(321.47787701,57.73439181)(321.53787695,57.80939174)(321.6378772,57.85939941)
\curveto(321.74787674,57.89939165)(321.8878766,57.90939164)(322.0578772,57.88939941)
\curveto(322.23787625,57.86939168)(322.3878761,57.87939167)(322.5078772,57.91939941)
\curveto(322.59787589,57.96939158)(322.66787582,58.03939151)(322.7178772,58.12939941)
\curveto(322.73787575,58.18939136)(322.74787574,58.26439128)(322.7478772,58.35439941)
\lineto(322.7478772,58.60939941)
\lineto(322.7478772,59.53939941)
\lineto(322.7478772,59.77939941)
\curveto(322.74787574,59.86938968)(322.75787573,59.9443896)(322.7778772,60.00439941)
\curveto(322.81787567,60.08438946)(322.8928756,60.1493894)(323.0028772,60.19939941)
\curveto(323.03287546,60.19938935)(323.05787543,60.19938935)(323.0778772,60.19939941)
\curveto(323.10787538,60.20938934)(323.13287536,60.21438933)(323.1528772,60.21439941)
}
}
{
\newrgbcolor{curcolor}{0 0 0}
\pscustom[linestyle=none,fillstyle=solid,fillcolor=curcolor]
{
\newpath
\moveto(333.67467407,54.31939941)
\curveto(333.69466639,54.21939533)(333.69466639,54.10439544)(333.67467407,53.97439941)
\curveto(333.66466642,53.85439569)(333.63466645,53.76939578)(333.58467407,53.71939941)
\curveto(333.53466655,53.67939587)(333.45966662,53.6493959)(333.35967407,53.62939941)
\curveto(333.26966681,53.61939593)(333.16466692,53.61439593)(333.04467407,53.61439941)
\lineto(332.68467407,53.61439941)
\curveto(332.56466752,53.62439592)(332.45966762,53.62939592)(332.36967407,53.62939941)
\lineto(328.52967407,53.62939941)
\curveto(328.44967163,53.62939592)(328.36967171,53.62439592)(328.28967407,53.61439941)
\curveto(328.20967187,53.61439593)(328.14467194,53.59939595)(328.09467407,53.56939941)
\curveto(328.05467203,53.549396)(328.01467207,53.50939604)(327.97467407,53.44939941)
\curveto(327.95467213,53.41939613)(327.93467215,53.37439617)(327.91467407,53.31439941)
\curveto(327.89467219,53.26439628)(327.89467219,53.21439633)(327.91467407,53.16439941)
\curveto(327.92467216,53.11439643)(327.92967215,53.06939648)(327.92967407,53.02939941)
\curveto(327.92967215,52.98939656)(327.93467215,52.9493966)(327.94467407,52.90939941)
\curveto(327.96467212,52.82939672)(327.9846721,52.7443968)(328.00467407,52.65439941)
\curveto(328.02467206,52.57439697)(328.05467203,52.49439705)(328.09467407,52.41439941)
\curveto(328.32467176,51.87439767)(328.70467138,51.48939806)(329.23467407,51.25939941)
\curveto(329.29467079,51.22939832)(329.35967072,51.20439834)(329.42967407,51.18439941)
\lineto(329.63967407,51.12439941)
\curveto(329.66967041,51.11439843)(329.71967036,51.10939844)(329.78967407,51.10939941)
\curveto(329.92967015,51.06939848)(330.11466997,51.0493985)(330.34467407,51.04939941)
\curveto(330.57466951,51.0493985)(330.75966932,51.06939848)(330.89967407,51.10939941)
\curveto(331.03966904,51.1493984)(331.16466892,51.18939836)(331.27467407,51.22939941)
\curveto(331.39466869,51.27939827)(331.50466858,51.33939821)(331.60467407,51.40939941)
\curveto(331.71466837,51.47939807)(331.80966827,51.55939799)(331.88967407,51.64939941)
\curveto(331.96966811,51.7493978)(332.03966804,51.85439769)(332.09967407,51.96439941)
\curveto(332.15966792,52.06439748)(332.20966787,52.16939738)(332.24967407,52.27939941)
\curveto(332.29966778,52.38939716)(332.3796677,52.46939708)(332.48967407,52.51939941)
\curveto(332.52966755,52.53939701)(332.59466749,52.55439699)(332.68467407,52.56439941)
\curveto(332.77466731,52.57439697)(332.86466722,52.57439697)(332.95467407,52.56439941)
\curveto(333.04466704,52.56439698)(333.12966695,52.55939699)(333.20967407,52.54939941)
\curveto(333.28966679,52.53939701)(333.34466674,52.51939703)(333.37467407,52.48939941)
\curveto(333.47466661,52.41939713)(333.49966658,52.30439724)(333.44967407,52.14439941)
\curveto(333.36966671,51.87439767)(333.26466682,51.63439791)(333.13467407,51.42439941)
\curveto(332.93466715,51.10439844)(332.70466738,50.83939871)(332.44467407,50.62939941)
\curveto(332.19466789,50.42939912)(331.87466821,50.26439928)(331.48467407,50.13439941)
\curveto(331.3846687,50.09439945)(331.2846688,50.06939948)(331.18467407,50.05939941)
\curveto(331.084669,50.03939951)(330.9796691,50.01939953)(330.86967407,49.99939941)
\curveto(330.81966926,49.98939956)(330.76966931,49.98439956)(330.71967407,49.98439941)
\curveto(330.6796694,49.98439956)(330.63466945,49.97939957)(330.58467407,49.96939941)
\lineto(330.43467407,49.96939941)
\curveto(330.3846697,49.95939959)(330.32466976,49.95439959)(330.25467407,49.95439941)
\curveto(330.19466989,49.95439959)(330.14466994,49.95939959)(330.10467407,49.96939941)
\lineto(329.96967407,49.96939941)
\curveto(329.91967016,49.97939957)(329.87467021,49.98439956)(329.83467407,49.98439941)
\curveto(329.79467029,49.98439956)(329.75467033,49.98939956)(329.71467407,49.99939941)
\curveto(329.66467042,50.00939954)(329.60967047,50.01939953)(329.54967407,50.02939941)
\curveto(329.48967059,50.02939952)(329.43467065,50.03439951)(329.38467407,50.04439941)
\curveto(329.29467079,50.06439948)(329.20467088,50.08939946)(329.11467407,50.11939941)
\curveto(329.02467106,50.13939941)(328.93967114,50.16439938)(328.85967407,50.19439941)
\curveto(328.81967126,50.21439933)(328.7846713,50.22439932)(328.75467407,50.22439941)
\curveto(328.72467136,50.23439931)(328.68967139,50.2493993)(328.64967407,50.26939941)
\curveto(328.49967158,50.33939921)(328.33967174,50.42439912)(328.16967407,50.52439941)
\curveto(327.8796722,50.71439883)(327.62967245,50.9443986)(327.41967407,51.21439941)
\curveto(327.21967286,51.49439805)(327.04967303,51.80439774)(326.90967407,52.14439941)
\curveto(326.85967322,52.25439729)(326.81967326,52.36939718)(326.78967407,52.48939941)
\curveto(326.76967331,52.60939694)(326.73967334,52.72939682)(326.69967407,52.84939941)
\curveto(326.68967339,52.88939666)(326.6846734,52.92439662)(326.68467407,52.95439941)
\curveto(326.6846734,52.98439656)(326.6796734,53.02439652)(326.66967407,53.07439941)
\curveto(326.64967343,53.15439639)(326.63467345,53.23939631)(326.62467407,53.32939941)
\curveto(326.61467347,53.41939613)(326.59967348,53.50939604)(326.57967407,53.59939941)
\lineto(326.57967407,53.80939941)
\curveto(326.56967351,53.8493957)(326.55967352,53.90439564)(326.54967407,53.97439941)
\curveto(326.54967353,54.05439549)(326.55467353,54.11939543)(326.56467407,54.16939941)
\lineto(326.56467407,54.33439941)
\curveto(326.5846735,54.38439516)(326.58967349,54.43439511)(326.57967407,54.48439941)
\curveto(326.5796735,54.544395)(326.5846735,54.59939495)(326.59467407,54.64939941)
\curveto(326.63467345,54.80939474)(326.66467342,54.96939458)(326.68467407,55.12939941)
\curveto(326.71467337,55.28939426)(326.75967332,55.43939411)(326.81967407,55.57939941)
\curveto(326.86967321,55.68939386)(326.91467317,55.79939375)(326.95467407,55.90939941)
\curveto(327.00467308,56.02939352)(327.05967302,56.1443934)(327.11967407,56.25439941)
\curveto(327.33967274,56.60439294)(327.58967249,56.90439264)(327.86967407,57.15439941)
\curveto(328.14967193,57.41439213)(328.49467159,57.62939192)(328.90467407,57.79939941)
\curveto(329.02467106,57.8493917)(329.14467094,57.88439166)(329.26467407,57.90439941)
\curveto(329.39467069,57.93439161)(329.52967055,57.96439158)(329.66967407,57.99439941)
\curveto(329.71967036,58.00439154)(329.76467032,58.00939154)(329.80467407,58.00939941)
\curveto(329.84467024,58.01939153)(329.88967019,58.02439152)(329.93967407,58.02439941)
\curveto(329.95967012,58.03439151)(329.9846701,58.03439151)(330.01467407,58.02439941)
\curveto(330.04467004,58.01439153)(330.06967001,58.01939153)(330.08967407,58.03939941)
\curveto(330.50966957,58.0493915)(330.87466921,58.00439154)(331.18467407,57.90439941)
\curveto(331.49466859,57.81439173)(331.77466831,57.68939186)(332.02467407,57.52939941)
\curveto(332.07466801,57.50939204)(332.11466797,57.47939207)(332.14467407,57.43939941)
\curveto(332.17466791,57.40939214)(332.20966787,57.38439216)(332.24967407,57.36439941)
\curveto(332.32966775,57.30439224)(332.40966767,57.23439231)(332.48967407,57.15439941)
\curveto(332.5796675,57.07439247)(332.65466743,56.99439255)(332.71467407,56.91439941)
\curveto(332.87466721,56.70439284)(333.00966707,56.50439304)(333.11967407,56.31439941)
\curveto(333.18966689,56.20439334)(333.24466684,56.08439346)(333.28467407,55.95439941)
\curveto(333.32466676,55.82439372)(333.36966671,55.69439385)(333.41967407,55.56439941)
\curveto(333.46966661,55.43439411)(333.50466658,55.29939425)(333.52467407,55.15939941)
\curveto(333.55466653,55.01939453)(333.58966649,54.87939467)(333.62967407,54.73939941)
\curveto(333.63966644,54.66939488)(333.64466644,54.59939495)(333.64467407,54.52939941)
\lineto(333.67467407,54.31939941)
\moveto(332.21967407,54.82939941)
\curveto(332.24966783,54.86939468)(332.27466781,54.91939463)(332.29467407,54.97939941)
\curveto(332.31466777,55.0493945)(332.31466777,55.11939443)(332.29467407,55.18939941)
\curveto(332.23466785,55.40939414)(332.14966793,55.61439393)(332.03967407,55.80439941)
\curveto(331.89966818,56.03439351)(331.74466834,56.22939332)(331.57467407,56.38939941)
\curveto(331.40466868,56.549393)(331.1846689,56.68439286)(330.91467407,56.79439941)
\curveto(330.84466924,56.81439273)(330.77466931,56.82939272)(330.70467407,56.83939941)
\curveto(330.63466945,56.85939269)(330.55966952,56.87939267)(330.47967407,56.89939941)
\curveto(330.39966968,56.91939263)(330.31466977,56.92939262)(330.22467407,56.92939941)
\lineto(329.96967407,56.92939941)
\curveto(329.93967014,56.90939264)(329.90467018,56.89939265)(329.86467407,56.89939941)
\curveto(329.82467026,56.90939264)(329.78967029,56.90939264)(329.75967407,56.89939941)
\lineto(329.51967407,56.83939941)
\curveto(329.44967063,56.82939272)(329.3796707,56.81439273)(329.30967407,56.79439941)
\curveto(329.01967106,56.67439287)(328.7846713,56.52439302)(328.60467407,56.34439941)
\curveto(328.43467165,56.16439338)(328.2796718,55.93939361)(328.13967407,55.66939941)
\curveto(328.10967197,55.61939393)(328.079672,55.55439399)(328.04967407,55.47439941)
\curveto(328.01967206,55.40439414)(327.99467209,55.32439422)(327.97467407,55.23439941)
\curveto(327.95467213,55.1443944)(327.94967213,55.05939449)(327.95967407,54.97939941)
\curveto(327.96967211,54.89939465)(328.00467208,54.83939471)(328.06467407,54.79939941)
\curveto(328.14467194,54.73939481)(328.2796718,54.70939484)(328.46967407,54.70939941)
\curveto(328.66967141,54.71939483)(328.83967124,54.72439482)(328.97967407,54.72439941)
\lineto(331.25967407,54.72439941)
\curveto(331.40966867,54.72439482)(331.58966849,54.71939483)(331.79967407,54.70939941)
\curveto(332.00966807,54.70939484)(332.14966793,54.7493948)(332.21967407,54.82939941)
}
}
{
\newrgbcolor{curcolor}{0 0 0}
\pscustom[linestyle=none,fillstyle=solid,fillcolor=curcolor]
{
\newpath
\moveto(461.67456909,55.72938477)
\lineto(461.67456909,55.45938477)
\curveto(461.68455912,55.36937952)(461.67955913,55.2893796)(461.65956909,55.21938477)
\lineto(461.65956909,55.06938477)
\curveto(461.64955916,55.03937985)(461.64455916,55.00437988)(461.64456909,54.96438477)
\curveto(461.65455915,54.92437996)(461.65455915,54.89437999)(461.64456909,54.87438477)
\curveto(461.63455917,54.82438006)(461.62955918,54.76938012)(461.62956909,54.70938477)
\curveto(461.62955918,54.65938023)(461.62455918,54.60938028)(461.61456909,54.55938477)
\curveto(461.58455922,54.41938047)(461.56455924,54.26938062)(461.55456909,54.10938477)
\curveto(461.54455926,53.95938093)(461.51455929,53.81438107)(461.46456909,53.67438477)
\curveto(461.43455937,53.55438133)(461.39955941,53.42938146)(461.35956909,53.29938477)
\curveto(461.32955948,53.17938171)(461.28955952,53.05938183)(461.23956909,52.93938477)
\curveto(461.06955974,52.50938238)(460.85455995,52.11938277)(460.59456909,51.76938477)
\curveto(460.34456046,51.42938346)(460.02956078,51.13938375)(459.64956909,50.89938477)
\curveto(459.45956135,50.77938411)(459.25456155,50.67438421)(459.03456909,50.58438477)
\curveto(458.82456198,50.50438438)(458.59456221,50.42438446)(458.34456909,50.34438477)
\curveto(458.23456257,50.30438458)(458.11456269,50.27438461)(457.98456909,50.25438477)
\curveto(457.86456294,50.24438464)(457.74456306,50.22438466)(457.62456909,50.19438477)
\curveto(457.51456329,50.17438471)(457.4045634,50.16438472)(457.29456909,50.16438477)
\curveto(457.19456361,50.16438472)(457.09456371,50.15438473)(456.99456909,50.13438477)
\lineto(456.78456909,50.13438477)
\curveto(456.75456405,50.12438476)(456.71956409,50.11938477)(456.67956909,50.11938477)
\curveto(456.63956417,50.12938476)(456.59956421,50.13438475)(456.55956909,50.13438477)
\lineto(453.55956909,50.13438477)
\curveto(453.4095674,50.13438475)(453.27456753,50.13938475)(453.15456909,50.14938477)
\curveto(453.04456776,50.16938472)(452.96956784,50.23438465)(452.92956909,50.34438477)
\curveto(452.88956792,50.42438446)(452.86956794,50.53938435)(452.86956909,50.68938477)
\curveto(452.87956793,50.83938405)(452.88456792,50.97438391)(452.88456909,51.09438477)
\lineto(452.88456909,59.95938477)
\curveto(452.88456792,60.07937481)(452.87956793,60.20437468)(452.86956909,60.33438477)
\curveto(452.86956794,60.47437441)(452.89456791,60.5843743)(452.94456909,60.66438477)
\curveto(452.98456782,60.73437415)(453.05956775,60.77937411)(453.16956909,60.79938477)
\curveto(453.18956762,60.80937408)(453.2095676,60.80937408)(453.22956909,60.79938477)
\curveto(453.24956756,60.79937409)(453.26956754,60.80437408)(453.28956909,60.81438477)
\lineto(456.54456909,60.81438477)
\curveto(456.59456421,60.81437407)(456.63956417,60.81437407)(456.67956909,60.81438477)
\curveto(456.72956408,60.82437406)(456.77456403,60.82437406)(456.81456909,60.81438477)
\curveto(456.86456394,60.79437409)(456.91456389,60.7893741)(456.96456909,60.79938477)
\curveto(457.02456378,60.80937408)(457.07956373,60.80937408)(457.12956909,60.79938477)
\curveto(457.17956363,60.7893741)(457.23456357,60.7843741)(457.29456909,60.78438477)
\curveto(457.35456345,60.7843741)(457.4095634,60.77937411)(457.45956909,60.76938477)
\curveto(457.5095633,60.75937413)(457.55456325,60.75437413)(457.59456909,60.75438477)
\curveto(457.64456316,60.75437413)(457.69456311,60.74937414)(457.74456909,60.73938477)
\curveto(457.85456295,60.71937417)(457.95956285,60.69937419)(458.05956909,60.67938477)
\curveto(458.15956265,60.66937422)(458.25956255,60.64937424)(458.35956909,60.61938477)
\curveto(458.57956223,60.54937434)(458.78956202,60.47937441)(458.98956909,60.40938477)
\curveto(459.18956162,60.34937454)(459.37456143,60.26437462)(459.54456909,60.15438477)
\curveto(459.68456112,60.07437481)(459.809561,59.99437489)(459.91956909,59.91438477)
\curveto(459.94956086,59.89437499)(459.97956083,59.86937502)(460.00956909,59.83938477)
\curveto(460.03956077,59.81937507)(460.06956074,59.79937509)(460.09956909,59.77938477)
\curveto(460.15956065,59.72937516)(460.21456059,59.67937521)(460.26456909,59.62938477)
\curveto(460.31456049,59.57937531)(460.36456044,59.52937536)(460.41456909,59.47938477)
\curveto(460.46456034,59.42937546)(460.5045603,59.39437549)(460.53456909,59.37438477)
\curveto(460.57456023,59.31437557)(460.61456019,59.25937563)(460.65456909,59.20938477)
\curveto(460.7045601,59.15937573)(460.74956006,59.10437578)(460.78956909,59.04438477)
\curveto(460.83955997,58.9843759)(460.87955993,58.91937597)(460.90956909,58.84938477)
\curveto(460.94955986,58.7893761)(460.99455981,58.72437616)(461.04456909,58.65438477)
\curveto(461.06455974,58.61437627)(461.07955973,58.57937631)(461.08956909,58.54938477)
\curveto(461.09955971,58.51937637)(461.11455969,58.4843764)(461.13456909,58.44438477)
\curveto(461.17455963,58.36437652)(461.2095596,58.2843766)(461.23956909,58.20438477)
\curveto(461.26955954,58.13437675)(461.3045595,58.05937683)(461.34456909,57.97938477)
\curveto(461.38455942,57.86937702)(461.41455939,57.75437713)(461.43456909,57.63438477)
\curveto(461.46455934,57.52437736)(461.49455931,57.41437747)(461.52456909,57.30438477)
\curveto(461.54455926,57.24437764)(461.55455925,57.1843777)(461.55456909,57.12438477)
\curveto(461.55455925,57.07437781)(461.56455924,57.01937787)(461.58456909,56.95938477)
\curveto(461.63455917,56.77937811)(461.65955915,56.57937831)(461.65956909,56.35938477)
\curveto(461.66955914,56.14937874)(461.67455913,55.93937895)(461.67456909,55.72938477)
\moveto(460.24956909,54.94938477)
\curveto(460.26956054,55.04937984)(460.27956053,55.15437973)(460.27956909,55.26438477)
\lineto(460.27956909,55.60938477)
\lineto(460.27956909,55.83438477)
\curveto(460.28956052,55.91437897)(460.28456052,55.9893789)(460.26456909,56.05938477)
\curveto(460.26456054,56.0893788)(460.25956055,56.11937877)(460.24956909,56.14938477)
\lineto(460.24956909,56.25438477)
\curveto(460.22956058,56.36437852)(460.21456059,56.47437841)(460.20456909,56.58438477)
\curveto(460.2045606,56.69437819)(460.18956062,56.80437808)(460.15956909,56.91438477)
\curveto(460.13956067,56.99437789)(460.11956069,57.06937782)(460.09956909,57.13938477)
\curveto(460.08956072,57.21937767)(460.07456073,57.29937759)(460.05456909,57.37938477)
\curveto(459.94456086,57.73937715)(459.804561,58.05437683)(459.63456909,58.32438477)
\curveto(459.35456145,58.77437611)(458.93956187,59.11437577)(458.38956909,59.34438477)
\curveto(458.29956251,59.39437549)(458.2045626,59.42937546)(458.10456909,59.44938477)
\curveto(458.0045628,59.47937541)(457.89956291,59.50937538)(457.78956909,59.53938477)
\curveto(457.67956313,59.56937532)(457.56456324,59.5843753)(457.44456909,59.58438477)
\curveto(457.33456347,59.59437529)(457.22456358,59.60937528)(457.11456909,59.62938477)
\lineto(456.79956909,59.62938477)
\curveto(456.76956404,59.63937525)(456.73456407,59.64437524)(456.69456909,59.64438477)
\lineto(456.57456909,59.64438477)
\lineto(454.74456909,59.64438477)
\curveto(454.72456608,59.63437525)(454.69956611,59.62937526)(454.66956909,59.62938477)
\curveto(454.63956617,59.63937525)(454.61456619,59.63937525)(454.59456909,59.62938477)
\lineto(454.44456909,59.56938477)
\curveto(454.4045664,59.54937534)(454.37456643,59.51937537)(454.35456909,59.47938477)
\curveto(454.33456647,59.43937545)(454.31456649,59.36937552)(454.29456909,59.26938477)
\lineto(454.29456909,59.14938477)
\curveto(454.28456652,59.10937578)(454.27956653,59.06437582)(454.27956909,59.01438477)
\lineto(454.27956909,58.87938477)
\lineto(454.27956909,52.06938477)
\lineto(454.27956909,51.91938477)
\curveto(454.27956653,51.87938301)(454.28456652,51.83938305)(454.29456909,51.79938477)
\lineto(454.29456909,51.67938477)
\curveto(454.31456649,51.57938331)(454.33456647,51.50938338)(454.35456909,51.46938477)
\curveto(454.43456637,51.34938354)(454.58456622,51.2893836)(454.80456909,51.28938477)
\curveto(455.02456578,51.29938359)(455.23456557,51.30438358)(455.43456909,51.30438477)
\lineto(456.30456909,51.30438477)
\curveto(456.37456443,51.30438358)(456.44956436,51.29938359)(456.52956909,51.28938477)
\curveto(456.6095642,51.2893836)(456.67956413,51.29938359)(456.73956909,51.31938477)
\lineto(456.90456909,51.31938477)
\curveto(456.95456385,51.32938356)(457.0095638,51.32938356)(457.06956909,51.31938477)
\curveto(457.12956368,51.31938357)(457.18956362,51.32438356)(457.24956909,51.33438477)
\curveto(457.3095635,51.35438353)(457.36956344,51.36438352)(457.42956909,51.36438477)
\curveto(457.48956332,51.37438351)(457.55456325,51.3893835)(457.62456909,51.40938477)
\curveto(457.73456307,51.43938345)(457.83956297,51.46938342)(457.93956909,51.49938477)
\curveto(458.04956276,51.52938336)(458.15956265,51.56938332)(458.26956909,51.61938477)
\curveto(458.63956217,51.77938311)(458.95456185,51.99438289)(459.21456909,52.26438477)
\curveto(459.48456132,52.54438234)(459.7045611,52.87438201)(459.87456909,53.25438477)
\curveto(459.92456088,53.36438152)(459.96456084,53.47938141)(459.99456909,53.59938477)
\lineto(460.11456909,53.98938477)
\curveto(460.14456066,54.09938079)(460.16456064,54.21438067)(460.17456909,54.33438477)
\curveto(460.19456061,54.46438042)(460.21456059,54.5893803)(460.23456909,54.70938477)
\curveto(460.24456056,54.75938013)(460.24956056,54.79938009)(460.24956909,54.82938477)
\lineto(460.24956909,54.94938477)
}
}
{
\newrgbcolor{curcolor}{0 0 0}
\pscustom[linestyle=none,fillstyle=solid,fillcolor=curcolor]
{
\newpath
\moveto(470.30144409,54.33438477)
\curveto(470.32143603,54.27438061)(470.33143602,54.17938071)(470.33144409,54.04938477)
\curveto(470.33143602,53.92938096)(470.32643603,53.84438104)(470.31644409,53.79438477)
\lineto(470.31644409,53.64438477)
\curveto(470.30643605,53.56438132)(470.29643606,53.4893814)(470.28644409,53.41938477)
\curveto(470.28643607,53.35938153)(470.28143607,53.2893816)(470.27144409,53.20938477)
\curveto(470.2514361,53.14938174)(470.23643612,53.0893818)(470.22644409,53.02938477)
\curveto(470.22643613,52.96938192)(470.21643614,52.90938198)(470.19644409,52.84938477)
\curveto(470.1564362,52.71938217)(470.12143623,52.5893823)(470.09144409,52.45938477)
\curveto(470.06143629,52.32938256)(470.02143633,52.20938268)(469.97144409,52.09938477)
\curveto(469.76143659,51.61938327)(469.48143687,51.21438367)(469.13144409,50.88438477)
\curveto(468.78143757,50.56438432)(468.351438,50.31938457)(467.84144409,50.14938477)
\curveto(467.73143862,50.10938478)(467.61143874,50.07938481)(467.48144409,50.05938477)
\curveto(467.36143899,50.03938485)(467.23643912,50.01938487)(467.10644409,49.99938477)
\curveto(467.04643931,49.9893849)(466.98143937,49.9843849)(466.91144409,49.98438477)
\curveto(466.8514395,49.97438491)(466.79143956,49.96938492)(466.73144409,49.96938477)
\curveto(466.69143966,49.95938493)(466.63143972,49.95438493)(466.55144409,49.95438477)
\curveto(466.48143987,49.95438493)(466.43143992,49.95938493)(466.40144409,49.96938477)
\curveto(466.36143999,49.97938491)(466.32144003,49.9843849)(466.28144409,49.98438477)
\curveto(466.24144011,49.97438491)(466.20644015,49.97438491)(466.17644409,49.98438477)
\lineto(466.08644409,49.98438477)
\lineto(465.72644409,50.02938477)
\curveto(465.58644077,50.06938482)(465.4514409,50.10938478)(465.32144409,50.14938477)
\curveto(465.19144116,50.1893847)(465.06644129,50.23438465)(464.94644409,50.28438477)
\curveto(464.49644186,50.4843844)(464.12644223,50.74438414)(463.83644409,51.06438477)
\curveto(463.54644281,51.3843835)(463.30644305,51.77438311)(463.11644409,52.23438477)
\curveto(463.06644329,52.33438255)(463.02644333,52.43438245)(462.99644409,52.53438477)
\curveto(462.97644338,52.63438225)(462.9564434,52.73938215)(462.93644409,52.84938477)
\curveto(462.91644344,52.889382)(462.90644345,52.91938197)(462.90644409,52.93938477)
\curveto(462.91644344,52.96938192)(462.91644344,53.00438188)(462.90644409,53.04438477)
\curveto(462.88644347,53.12438176)(462.87144348,53.20438168)(462.86144409,53.28438477)
\curveto(462.86144349,53.37438151)(462.8514435,53.45938143)(462.83144409,53.53938477)
\lineto(462.83144409,53.65938477)
\curveto(462.83144352,53.69938119)(462.82644353,53.74438114)(462.81644409,53.79438477)
\curveto(462.80644355,53.84438104)(462.80144355,53.92938096)(462.80144409,54.04938477)
\curveto(462.80144355,54.17938071)(462.81144354,54.27438061)(462.83144409,54.33438477)
\curveto(462.8514435,54.40438048)(462.8564435,54.47438041)(462.84644409,54.54438477)
\curveto(462.83644352,54.61438027)(462.84144351,54.6843802)(462.86144409,54.75438477)
\curveto(462.87144348,54.80438008)(462.87644348,54.84438004)(462.87644409,54.87438477)
\curveto(462.88644347,54.91437997)(462.89644346,54.95937993)(462.90644409,55.00938477)
\curveto(462.93644342,55.12937976)(462.96144339,55.24937964)(462.98144409,55.36938477)
\curveto(463.01144334,55.4893794)(463.0514433,55.60437928)(463.10144409,55.71438477)
\curveto(463.2514431,56.0843788)(463.43144292,56.41437847)(463.64144409,56.70438477)
\curveto(463.86144249,57.00437788)(464.12644223,57.25437763)(464.43644409,57.45438477)
\curveto(464.5564418,57.53437735)(464.68144167,57.59937729)(464.81144409,57.64938477)
\curveto(464.94144141,57.70937718)(465.07644128,57.76937712)(465.21644409,57.82938477)
\curveto(465.33644102,57.87937701)(465.46644089,57.90937698)(465.60644409,57.91938477)
\curveto(465.74644061,57.93937695)(465.88644047,57.96937692)(466.02644409,58.00938477)
\lineto(466.22144409,58.00938477)
\curveto(466.29144006,58.01937687)(466.35644,58.02937686)(466.41644409,58.03938477)
\curveto(467.30643905,58.04937684)(468.04643831,57.86437702)(468.63644409,57.48438477)
\curveto(469.22643713,57.10437778)(469.6514367,56.60937828)(469.91144409,55.99938477)
\curveto(469.96143639,55.89937899)(470.00143635,55.79937909)(470.03144409,55.69938477)
\curveto(470.06143629,55.59937929)(470.09643626,55.49437939)(470.13644409,55.38438477)
\curveto(470.16643619,55.27437961)(470.19143616,55.15437973)(470.21144409,55.02438477)
\curveto(470.23143612,54.90437998)(470.2564361,54.77938011)(470.28644409,54.64938477)
\curveto(470.29643606,54.59938029)(470.29643606,54.54438034)(470.28644409,54.48438477)
\curveto(470.28643607,54.43438045)(470.29143606,54.3843805)(470.30144409,54.33438477)
\moveto(468.96644409,53.47938477)
\curveto(468.98643737,53.54938134)(468.99143736,53.62938126)(468.98144409,53.71938477)
\lineto(468.98144409,53.97438477)
\curveto(468.98143737,54.36438052)(468.94643741,54.69438019)(468.87644409,54.96438477)
\curveto(468.84643751,55.04437984)(468.82143753,55.12437976)(468.80144409,55.20438477)
\curveto(468.78143757,55.2843796)(468.7564376,55.35937953)(468.72644409,55.42938477)
\curveto(468.44643791,56.07937881)(468.00143835,56.52937836)(467.39144409,56.77938477)
\curveto(467.32143903,56.80937808)(467.24643911,56.82937806)(467.16644409,56.83938477)
\lineto(466.92644409,56.89938477)
\curveto(466.84643951,56.91937797)(466.76143959,56.92937796)(466.67144409,56.92938477)
\lineto(466.40144409,56.92938477)
\lineto(466.13144409,56.88438477)
\curveto(466.03144032,56.86437802)(465.93644042,56.83937805)(465.84644409,56.80938477)
\curveto(465.76644059,56.7893781)(465.68644067,56.75937813)(465.60644409,56.71938477)
\curveto(465.53644082,56.69937819)(465.47144088,56.66937822)(465.41144409,56.62938477)
\curveto(465.351441,56.5893783)(465.29644106,56.54937834)(465.24644409,56.50938477)
\curveto(465.00644135,56.33937855)(464.81144154,56.13437875)(464.66144409,55.89438477)
\curveto(464.51144184,55.65437923)(464.38144197,55.37437951)(464.27144409,55.05438477)
\curveto(464.24144211,54.95437993)(464.22144213,54.84938004)(464.21144409,54.73938477)
\curveto(464.20144215,54.63938025)(464.18644217,54.53438035)(464.16644409,54.42438477)
\curveto(464.1564422,54.3843805)(464.1514422,54.31938057)(464.15144409,54.22938477)
\curveto(464.14144221,54.19938069)(464.13644222,54.16438072)(464.13644409,54.12438477)
\curveto(464.14644221,54.0843808)(464.1514422,54.03938085)(464.15144409,53.98938477)
\lineto(464.15144409,53.68938477)
\curveto(464.1514422,53.5893813)(464.16144219,53.49938139)(464.18144409,53.41938477)
\lineto(464.21144409,53.23938477)
\curveto(464.23144212,53.13938175)(464.24644211,53.03938185)(464.25644409,52.93938477)
\curveto(464.27644208,52.84938204)(464.30644205,52.76438212)(464.34644409,52.68438477)
\curveto(464.44644191,52.44438244)(464.56144179,52.21938267)(464.69144409,52.00938477)
\curveto(464.83144152,51.79938309)(465.00144135,51.62438326)(465.20144409,51.48438477)
\curveto(465.2514411,51.45438343)(465.29644106,51.42938346)(465.33644409,51.40938477)
\curveto(465.37644098,51.3893835)(465.42144093,51.36438352)(465.47144409,51.33438477)
\curveto(465.5514408,51.2843836)(465.63644072,51.23938365)(465.72644409,51.19938477)
\curveto(465.82644053,51.16938372)(465.93144042,51.13938375)(466.04144409,51.10938477)
\curveto(466.09144026,51.0893838)(466.13644022,51.07938381)(466.17644409,51.07938477)
\curveto(466.22644013,51.0893838)(466.27644008,51.0893838)(466.32644409,51.07938477)
\curveto(466.35644,51.06938382)(466.41643994,51.05938383)(466.50644409,51.04938477)
\curveto(466.60643975,51.03938385)(466.68143967,51.04438384)(466.73144409,51.06438477)
\curveto(466.77143958,51.07438381)(466.81143954,51.07438381)(466.85144409,51.06438477)
\curveto(466.89143946,51.06438382)(466.93143942,51.07438381)(466.97144409,51.09438477)
\curveto(467.0514393,51.11438377)(467.13143922,51.12938376)(467.21144409,51.13938477)
\curveto(467.29143906,51.15938373)(467.36643899,51.1843837)(467.43644409,51.21438477)
\curveto(467.77643858,51.35438353)(468.0514383,51.54938334)(468.26144409,51.79938477)
\curveto(468.47143788,52.04938284)(468.64643771,52.34438254)(468.78644409,52.68438477)
\curveto(468.83643752,52.80438208)(468.86643749,52.92938196)(468.87644409,53.05938477)
\curveto(468.89643746,53.19938169)(468.92643743,53.33938155)(468.96644409,53.47938477)
}
}
{
\newrgbcolor{curcolor}{0 0 0}
\pscustom[linestyle=none,fillstyle=solid,fillcolor=curcolor]
{
\newpath
\moveto(474.92472534,58.03938477)
\curveto(475.66472055,58.04937684)(476.27971994,57.93937695)(476.76972534,57.70938477)
\curveto(477.26971895,57.4893774)(477.66471855,57.15437773)(477.95472534,56.70438477)
\curveto(478.08471813,56.50437838)(478.19471802,56.25937863)(478.28472534,55.96938477)
\curveto(478.30471791,55.91937897)(478.3197179,55.85437903)(478.32972534,55.77438477)
\curveto(478.33971788,55.69437919)(478.33471788,55.62437926)(478.31472534,55.56438477)
\curveto(478.28471793,55.51437937)(478.23471798,55.46937942)(478.16472534,55.42938477)
\curveto(478.13471808,55.40937948)(478.10471811,55.39937949)(478.07472534,55.39938477)
\curveto(478.04471817,55.40937948)(478.00971821,55.40937948)(477.96972534,55.39938477)
\curveto(477.92971829,55.3893795)(477.88971833,55.3843795)(477.84972534,55.38438477)
\curveto(477.80971841,55.39437949)(477.76971845,55.39937949)(477.72972534,55.39938477)
\lineto(477.41472534,55.39938477)
\curveto(477.3147189,55.40937948)(477.22971899,55.43937945)(477.15972534,55.48938477)
\curveto(477.07971914,55.54937934)(477.02471919,55.63437925)(476.99472534,55.74438477)
\curveto(476.96471925,55.85437903)(476.92471929,55.94937894)(476.87472534,56.02938477)
\curveto(476.72471949,56.2893786)(476.52971969,56.49437839)(476.28972534,56.64438477)
\curveto(476.20972001,56.69437819)(476.12472009,56.73437815)(476.03472534,56.76438477)
\curveto(475.94472027,56.80437808)(475.84972037,56.83937805)(475.74972534,56.86938477)
\curveto(475.60972061,56.90937798)(475.42472079,56.92937796)(475.19472534,56.92938477)
\curveto(474.96472125,56.93937795)(474.77472144,56.91937797)(474.62472534,56.86938477)
\curveto(474.55472166,56.84937804)(474.48972173,56.83437805)(474.42972534,56.82438477)
\curveto(474.36972185,56.81437807)(474.30472191,56.79937809)(474.23472534,56.77938477)
\curveto(473.97472224,56.66937822)(473.74472247,56.51937837)(473.54472534,56.32938477)
\curveto(473.34472287,56.13937875)(473.18972303,55.91437897)(473.07972534,55.65438477)
\curveto(473.03972318,55.56437932)(473.00472321,55.46937942)(472.97472534,55.36938477)
\curveto(472.94472327,55.27937961)(472.9147233,55.17937971)(472.88472534,55.06938477)
\lineto(472.79472534,54.66438477)
\curveto(472.78472343,54.61438027)(472.77972344,54.55938033)(472.77972534,54.49938477)
\curveto(472.78972343,54.43938045)(472.78472343,54.3843805)(472.76472534,54.33438477)
\lineto(472.76472534,54.21438477)
\curveto(472.75472346,54.17438071)(472.74472347,54.10938078)(472.73472534,54.01938477)
\curveto(472.73472348,53.92938096)(472.74472347,53.86438102)(472.76472534,53.82438477)
\curveto(472.77472344,53.77438111)(472.77472344,53.72438116)(472.76472534,53.67438477)
\curveto(472.75472346,53.62438126)(472.75472346,53.57438131)(472.76472534,53.52438477)
\curveto(472.77472344,53.4843814)(472.77972344,53.41438147)(472.77972534,53.31438477)
\curveto(472.79972342,53.23438165)(472.8147234,53.14938174)(472.82472534,53.05938477)
\curveto(472.84472337,52.96938192)(472.86472335,52.884382)(472.88472534,52.80438477)
\curveto(472.99472322,52.4843824)(473.1197231,52.20438268)(473.25972534,51.96438477)
\curveto(473.40972281,51.73438315)(473.6147226,51.53438335)(473.87472534,51.36438477)
\curveto(473.96472225,51.31438357)(474.05472216,51.26938362)(474.14472534,51.22938477)
\curveto(474.24472197,51.1893837)(474.34972187,51.14938374)(474.45972534,51.10938477)
\curveto(474.50972171,51.09938379)(474.54972167,51.09438379)(474.57972534,51.09438477)
\curveto(474.60972161,51.09438379)(474.64972157,51.0893838)(474.69972534,51.07938477)
\curveto(474.72972149,51.06938382)(474.77972144,51.06438382)(474.84972534,51.06438477)
\lineto(475.01472534,51.06438477)
\curveto(475.0147212,51.05438383)(475.03472118,51.04938384)(475.07472534,51.04938477)
\curveto(475.09472112,51.05938383)(475.1197211,51.05938383)(475.14972534,51.04938477)
\curveto(475.17972104,51.04938384)(475.20972101,51.05438383)(475.23972534,51.06438477)
\curveto(475.30972091,51.0843838)(475.37472084,51.0893838)(475.43472534,51.07938477)
\curveto(475.50472071,51.07938381)(475.57472064,51.0893838)(475.64472534,51.10938477)
\curveto(475.90472031,51.1893837)(476.12972009,51.2893836)(476.31972534,51.40938477)
\curveto(476.50971971,51.53938335)(476.66971955,51.70438318)(476.79972534,51.90438477)
\curveto(476.84971937,51.9843829)(476.89471932,52.06938282)(476.93472534,52.15938477)
\lineto(477.05472534,52.42938477)
\curveto(477.07471914,52.50938238)(477.09471912,52.5843823)(477.11472534,52.65438477)
\curveto(477.14471907,52.73438215)(477.19471902,52.79938209)(477.26472534,52.84938477)
\curveto(477.29471892,52.87938201)(477.35471886,52.89938199)(477.44472534,52.90938477)
\curveto(477.53471868,52.92938196)(477.62971859,52.93938195)(477.72972534,52.93938477)
\curveto(477.83971838,52.94938194)(477.93971828,52.94938194)(478.02972534,52.93938477)
\curveto(478.12971809,52.92938196)(478.19971802,52.90938198)(478.23972534,52.87938477)
\curveto(478.29971792,52.83938205)(478.33471788,52.77938211)(478.34472534,52.69938477)
\curveto(478.36471785,52.61938227)(478.36471785,52.53438235)(478.34472534,52.44438477)
\curveto(478.29471792,52.29438259)(478.24471797,52.14938274)(478.19472534,52.00938477)
\curveto(478.15471806,51.87938301)(478.09971812,51.74938314)(478.02972534,51.61938477)
\curveto(477.87971834,51.31938357)(477.68971853,51.05438383)(477.45972534,50.82438477)
\curveto(477.23971898,50.59438429)(476.96971925,50.40938448)(476.64972534,50.26938477)
\curveto(476.56971965,50.22938466)(476.48471973,50.19438469)(476.39472534,50.16438477)
\curveto(476.30471991,50.14438474)(476.20972001,50.11938477)(476.10972534,50.08938477)
\curveto(475.99972022,50.04938484)(475.88972033,50.02938486)(475.77972534,50.02938477)
\curveto(475.66972055,50.01938487)(475.55972066,50.00438488)(475.44972534,49.98438477)
\curveto(475.40972081,49.96438492)(475.36972085,49.95938493)(475.32972534,49.96938477)
\curveto(475.28972093,49.97938491)(475.24972097,49.97938491)(475.20972534,49.96938477)
\lineto(475.07472534,49.96938477)
\lineto(474.83472534,49.96938477)
\curveto(474.76472145,49.95938493)(474.69972152,49.96438492)(474.63972534,49.98438477)
\lineto(474.56472534,49.98438477)
\lineto(474.20472534,50.02938477)
\curveto(474.07472214,50.06938482)(473.94972227,50.10438478)(473.82972534,50.13438477)
\curveto(473.70972251,50.16438472)(473.59472262,50.20438468)(473.48472534,50.25438477)
\curveto(473.12472309,50.41438447)(472.82472339,50.60438428)(472.58472534,50.82438477)
\curveto(472.35472386,51.04438384)(472.13972408,51.31438357)(471.93972534,51.63438477)
\curveto(471.88972433,51.71438317)(471.84472437,51.80438308)(471.80472534,51.90438477)
\lineto(471.68472534,52.20438477)
\curveto(471.63472458,52.31438257)(471.59972462,52.42938246)(471.57972534,52.54938477)
\curveto(471.55972466,52.66938222)(471.53472468,52.7893821)(471.50472534,52.90938477)
\curveto(471.49472472,52.94938194)(471.48972473,52.9893819)(471.48972534,53.02938477)
\curveto(471.48972473,53.06938182)(471.48472473,53.10938178)(471.47472534,53.14938477)
\curveto(471.45472476,53.20938168)(471.44472477,53.27438161)(471.44472534,53.34438477)
\curveto(471.45472476,53.41438147)(471.44972477,53.47938141)(471.42972534,53.53938477)
\lineto(471.42972534,53.68938477)
\curveto(471.4197248,53.73938115)(471.4147248,53.80938108)(471.41472534,53.89938477)
\curveto(471.4147248,53.9893809)(471.4197248,54.05938083)(471.42972534,54.10938477)
\curveto(471.43972478,54.15938073)(471.43972478,54.20438068)(471.42972534,54.24438477)
\curveto(471.42972479,54.2843806)(471.43472478,54.32438056)(471.44472534,54.36438477)
\curveto(471.46472475,54.43438045)(471.46972475,54.50438038)(471.45972534,54.57438477)
\curveto(471.45972476,54.64438024)(471.46972475,54.70938018)(471.48972534,54.76938477)
\curveto(471.52972469,54.93937995)(471.56472465,55.10937978)(471.59472534,55.27938477)
\curveto(471.62472459,55.44937944)(471.66972455,55.60937928)(471.72972534,55.75938477)
\curveto(471.93972428,56.27937861)(472.19472402,56.69937819)(472.49472534,57.01938477)
\curveto(472.79472342,57.33937755)(473.20472301,57.60437728)(473.72472534,57.81438477)
\curveto(473.83472238,57.86437702)(473.95472226,57.89937699)(474.08472534,57.91938477)
\curveto(474.214722,57.93937695)(474.34972187,57.96437692)(474.48972534,57.99438477)
\curveto(474.55972166,58.00437688)(474.62972159,58.00937688)(474.69972534,58.00938477)
\curveto(474.76972145,58.01937687)(474.84472137,58.02937686)(474.92472534,58.03938477)
}
}
{
\newrgbcolor{curcolor}{0 0 0}
\pscustom[linestyle=none,fillstyle=solid,fillcolor=curcolor]
{
\newpath
\moveto(486.59636597,54.30438477)
\curveto(486.61635828,54.20438068)(486.61635828,54.0893808)(486.59636597,53.95938477)
\curveto(486.58635831,53.83938105)(486.55635834,53.75438113)(486.50636597,53.70438477)
\curveto(486.45635844,53.66438122)(486.38135852,53.63438125)(486.28136597,53.61438477)
\curveto(486.19135871,53.60438128)(486.08635881,53.59938129)(485.96636597,53.59938477)
\lineto(485.60636597,53.59938477)
\curveto(485.48635941,53.60938128)(485.38135952,53.61438127)(485.29136597,53.61438477)
\lineto(481.45136597,53.61438477)
\curveto(481.37136353,53.61438127)(481.29136361,53.60938128)(481.21136597,53.59938477)
\curveto(481.13136377,53.59938129)(481.06636383,53.5843813)(481.01636597,53.55438477)
\curveto(480.97636392,53.53438135)(480.93636396,53.49438139)(480.89636597,53.43438477)
\curveto(480.87636402,53.40438148)(480.85636404,53.35938153)(480.83636597,53.29938477)
\curveto(480.81636408,53.24938164)(480.81636408,53.19938169)(480.83636597,53.14938477)
\curveto(480.84636405,53.09938179)(480.85136405,53.05438183)(480.85136597,53.01438477)
\curveto(480.85136405,52.97438191)(480.85636404,52.93438195)(480.86636597,52.89438477)
\curveto(480.88636401,52.81438207)(480.90636399,52.72938216)(480.92636597,52.63938477)
\curveto(480.94636395,52.55938233)(480.97636392,52.47938241)(481.01636597,52.39938477)
\curveto(481.24636365,51.85938303)(481.62636327,51.47438341)(482.15636597,51.24438477)
\curveto(482.21636268,51.21438367)(482.28136262,51.1893837)(482.35136597,51.16938477)
\lineto(482.56136597,51.10938477)
\curveto(482.59136231,51.09938379)(482.64136226,51.09438379)(482.71136597,51.09438477)
\curveto(482.85136205,51.05438383)(483.03636186,51.03438385)(483.26636597,51.03438477)
\curveto(483.4963614,51.03438385)(483.68136122,51.05438383)(483.82136597,51.09438477)
\curveto(483.96136094,51.13438375)(484.08636081,51.17438371)(484.19636597,51.21438477)
\curveto(484.31636058,51.26438362)(484.42636047,51.32438356)(484.52636597,51.39438477)
\curveto(484.63636026,51.46438342)(484.73136017,51.54438334)(484.81136597,51.63438477)
\curveto(484.89136001,51.73438315)(484.96135994,51.83938305)(485.02136597,51.94938477)
\curveto(485.08135982,52.04938284)(485.13135977,52.15438273)(485.17136597,52.26438477)
\curveto(485.22135968,52.37438251)(485.3013596,52.45438243)(485.41136597,52.50438477)
\curveto(485.45135945,52.52438236)(485.51635938,52.53938235)(485.60636597,52.54938477)
\curveto(485.6963592,52.55938233)(485.78635911,52.55938233)(485.87636597,52.54938477)
\curveto(485.96635893,52.54938234)(486.05135885,52.54438234)(486.13136597,52.53438477)
\curveto(486.21135869,52.52438236)(486.26635863,52.50438238)(486.29636597,52.47438477)
\curveto(486.3963585,52.40438248)(486.42135848,52.2893826)(486.37136597,52.12938477)
\curveto(486.29135861,51.85938303)(486.18635871,51.61938327)(486.05636597,51.40938477)
\curveto(485.85635904,51.0893838)(485.62635927,50.82438406)(485.36636597,50.61438477)
\curveto(485.11635978,50.41438447)(484.7963601,50.24938464)(484.40636597,50.11938477)
\curveto(484.30636059,50.07938481)(484.20636069,50.05438483)(484.10636597,50.04438477)
\curveto(484.00636089,50.02438486)(483.901361,50.00438488)(483.79136597,49.98438477)
\curveto(483.74136116,49.97438491)(483.69136121,49.96938492)(483.64136597,49.96938477)
\curveto(483.6013613,49.96938492)(483.55636134,49.96438492)(483.50636597,49.95438477)
\lineto(483.35636597,49.95438477)
\curveto(483.30636159,49.94438494)(483.24636165,49.93938495)(483.17636597,49.93938477)
\curveto(483.11636178,49.93938495)(483.06636183,49.94438494)(483.02636597,49.95438477)
\lineto(482.89136597,49.95438477)
\curveto(482.84136206,49.96438492)(482.7963621,49.96938492)(482.75636597,49.96938477)
\curveto(482.71636218,49.96938492)(482.67636222,49.97438491)(482.63636597,49.98438477)
\curveto(482.58636231,49.99438489)(482.53136237,50.00438488)(482.47136597,50.01438477)
\curveto(482.41136249,50.01438487)(482.35636254,50.01938487)(482.30636597,50.02938477)
\curveto(482.21636268,50.04938484)(482.12636277,50.07438481)(482.03636597,50.10438477)
\curveto(481.94636295,50.12438476)(481.86136304,50.14938474)(481.78136597,50.17938477)
\curveto(481.74136316,50.19938469)(481.70636319,50.20938468)(481.67636597,50.20938477)
\curveto(481.64636325,50.21938467)(481.61136329,50.23438465)(481.57136597,50.25438477)
\curveto(481.42136348,50.32438456)(481.26136364,50.40938448)(481.09136597,50.50938477)
\curveto(480.8013641,50.69938419)(480.55136435,50.92938396)(480.34136597,51.19938477)
\curveto(480.14136476,51.47938341)(479.97136493,51.7893831)(479.83136597,52.12938477)
\curveto(479.78136512,52.23938265)(479.74136516,52.35438253)(479.71136597,52.47438477)
\curveto(479.69136521,52.59438229)(479.66136524,52.71438217)(479.62136597,52.83438477)
\curveto(479.61136529,52.87438201)(479.60636529,52.90938198)(479.60636597,52.93938477)
\curveto(479.60636529,52.96938192)(479.6013653,53.00938188)(479.59136597,53.05938477)
\curveto(479.57136533,53.13938175)(479.55636534,53.22438166)(479.54636597,53.31438477)
\curveto(479.53636536,53.40438148)(479.52136538,53.49438139)(479.50136597,53.58438477)
\lineto(479.50136597,53.79438477)
\curveto(479.49136541,53.83438105)(479.48136542,53.889381)(479.47136597,53.95938477)
\curveto(479.47136543,54.03938085)(479.47636542,54.10438078)(479.48636597,54.15438477)
\lineto(479.48636597,54.31938477)
\curveto(479.50636539,54.36938052)(479.51136539,54.41938047)(479.50136597,54.46938477)
\curveto(479.5013654,54.52938036)(479.50636539,54.5843803)(479.51636597,54.63438477)
\curveto(479.55636534,54.79438009)(479.58636531,54.95437993)(479.60636597,55.11438477)
\curveto(479.63636526,55.27437961)(479.68136522,55.42437946)(479.74136597,55.56438477)
\curveto(479.79136511,55.67437921)(479.83636506,55.7843791)(479.87636597,55.89438477)
\curveto(479.92636497,56.01437887)(479.98136492,56.12937876)(480.04136597,56.23938477)
\curveto(480.26136464,56.5893783)(480.51136439,56.889378)(480.79136597,57.13938477)
\curveto(481.07136383,57.39937749)(481.41636348,57.61437727)(481.82636597,57.78438477)
\curveto(481.94636295,57.83437705)(482.06636283,57.86937702)(482.18636597,57.88938477)
\curveto(482.31636258,57.91937697)(482.45136245,57.94937694)(482.59136597,57.97938477)
\curveto(482.64136226,57.9893769)(482.68636221,57.99437689)(482.72636597,57.99438477)
\curveto(482.76636213,58.00437688)(482.81136209,58.00937688)(482.86136597,58.00938477)
\curveto(482.88136202,58.01937687)(482.90636199,58.01937687)(482.93636597,58.00938477)
\curveto(482.96636193,57.99937689)(482.99136191,58.00437688)(483.01136597,58.02438477)
\curveto(483.43136147,58.03437685)(483.7963611,57.9893769)(484.10636597,57.88938477)
\curveto(484.41636048,57.79937709)(484.6963602,57.67437721)(484.94636597,57.51438477)
\curveto(484.9963599,57.49437739)(485.03635986,57.46437742)(485.06636597,57.42438477)
\curveto(485.0963598,57.39437749)(485.13135977,57.36937752)(485.17136597,57.34938477)
\curveto(485.25135965,57.2893776)(485.33135957,57.21937767)(485.41136597,57.13938477)
\curveto(485.5013594,57.05937783)(485.57635932,56.97937791)(485.63636597,56.89938477)
\curveto(485.7963591,56.6893782)(485.93135897,56.4893784)(486.04136597,56.29938477)
\curveto(486.11135879,56.1893787)(486.16635873,56.06937882)(486.20636597,55.93938477)
\curveto(486.24635865,55.80937908)(486.29135861,55.67937921)(486.34136597,55.54938477)
\curveto(486.39135851,55.41937947)(486.42635847,55.2843796)(486.44636597,55.14438477)
\curveto(486.47635842,55.00437988)(486.51135839,54.86438002)(486.55136597,54.72438477)
\curveto(486.56135834,54.65438023)(486.56635833,54.5843803)(486.56636597,54.51438477)
\lineto(486.59636597,54.30438477)
\moveto(485.14136597,54.81438477)
\curveto(485.17135973,54.85438003)(485.1963597,54.90437998)(485.21636597,54.96438477)
\curveto(485.23635966,55.03437985)(485.23635966,55.10437978)(485.21636597,55.17438477)
\curveto(485.15635974,55.39437949)(485.07135983,55.59937929)(484.96136597,55.78938477)
\curveto(484.82136008,56.01937887)(484.66636023,56.21437867)(484.49636597,56.37438477)
\curveto(484.32636057,56.53437835)(484.10636079,56.66937822)(483.83636597,56.77938477)
\curveto(483.76636113,56.79937809)(483.6963612,56.81437807)(483.62636597,56.82438477)
\curveto(483.55636134,56.84437804)(483.48136142,56.86437802)(483.40136597,56.88438477)
\curveto(483.32136158,56.90437798)(483.23636166,56.91437797)(483.14636597,56.91438477)
\lineto(482.89136597,56.91438477)
\curveto(482.86136204,56.89437799)(482.82636207,56.884378)(482.78636597,56.88438477)
\curveto(482.74636215,56.89437799)(482.71136219,56.89437799)(482.68136597,56.88438477)
\lineto(482.44136597,56.82438477)
\curveto(482.37136253,56.81437807)(482.3013626,56.79937809)(482.23136597,56.77938477)
\curveto(481.94136296,56.65937823)(481.70636319,56.50937838)(481.52636597,56.32938477)
\curveto(481.35636354,56.14937874)(481.2013637,55.92437896)(481.06136597,55.65438477)
\curveto(481.03136387,55.60437928)(481.0013639,55.53937935)(480.97136597,55.45938477)
\curveto(480.94136396,55.3893795)(480.91636398,55.30937958)(480.89636597,55.21938477)
\curveto(480.87636402,55.12937976)(480.87136403,55.04437984)(480.88136597,54.96438477)
\curveto(480.89136401,54.88438)(480.92636397,54.82438006)(480.98636597,54.78438477)
\curveto(481.06636383,54.72438016)(481.2013637,54.69438019)(481.39136597,54.69438477)
\curveto(481.59136331,54.70438018)(481.76136314,54.70938018)(481.90136597,54.70938477)
\lineto(484.18136597,54.70938477)
\curveto(484.33136057,54.70938018)(484.51136039,54.70438018)(484.72136597,54.69438477)
\curveto(484.93135997,54.69438019)(485.07135983,54.73438015)(485.14136597,54.81438477)
}
}
{
\newrgbcolor{curcolor}{0 0 0}
\pscustom[linestyle=none,fillstyle=solid,fillcolor=curcolor]
{
\newpath
\moveto(491.59300659,58.00938477)
\curveto(492.22300136,58.02937686)(492.72800085,57.94437694)(493.10800659,57.75438477)
\curveto(493.48800009,57.56437732)(493.79299979,57.27937761)(494.02300659,56.89938477)
\curveto(494.0829995,56.79937809)(494.12799945,56.6893782)(494.15800659,56.56938477)
\curveto(494.19799938,56.45937843)(494.23299935,56.34437854)(494.26300659,56.22438477)
\curveto(494.31299927,56.03437885)(494.34299924,55.82937906)(494.35300659,55.60938477)
\curveto(494.36299922,55.3893795)(494.36799921,55.16437972)(494.36800659,54.93438477)
\lineto(494.36800659,53.32938477)
\lineto(494.36800659,50.98938477)
\curveto(494.36799921,50.81938407)(494.36299922,50.64938424)(494.35300659,50.47938477)
\curveto(494.35299923,50.30938458)(494.28799929,50.19938469)(494.15800659,50.14938477)
\curveto(494.10799947,50.12938476)(494.05299953,50.11938477)(493.99300659,50.11938477)
\curveto(493.94299964,50.10938478)(493.88799969,50.10438478)(493.82800659,50.10438477)
\curveto(493.69799988,50.10438478)(493.57300001,50.10938478)(493.45300659,50.11938477)
\curveto(493.33300025,50.11938477)(493.24800033,50.15938473)(493.19800659,50.23938477)
\curveto(493.14800043,50.30938458)(493.12300046,50.39938449)(493.12300659,50.50938477)
\lineto(493.12300659,50.83938477)
\lineto(493.12300659,52.12938477)
\lineto(493.12300659,54.57438477)
\curveto(493.12300046,54.84438004)(493.11800046,55.10937978)(493.10800659,55.36938477)
\curveto(493.09800048,55.63937925)(493.05300053,55.86937902)(492.97300659,56.05938477)
\curveto(492.89300069,56.25937863)(492.77300081,56.41937847)(492.61300659,56.53938477)
\curveto(492.45300113,56.66937822)(492.26800131,56.76937812)(492.05800659,56.83938477)
\curveto(491.99800158,56.85937803)(491.93300165,56.86937802)(491.86300659,56.86938477)
\curveto(491.80300178,56.87937801)(491.74300184,56.89437799)(491.68300659,56.91438477)
\curveto(491.63300195,56.92437796)(491.55300203,56.92437796)(491.44300659,56.91438477)
\curveto(491.34300224,56.91437797)(491.27300231,56.90937798)(491.23300659,56.89938477)
\curveto(491.19300239,56.87937801)(491.15800242,56.86937802)(491.12800659,56.86938477)
\curveto(491.09800248,56.87937801)(491.06300252,56.87937801)(491.02300659,56.86938477)
\curveto(490.89300269,56.83937805)(490.76800281,56.80437808)(490.64800659,56.76438477)
\curveto(490.53800304,56.73437815)(490.43300315,56.6893782)(490.33300659,56.62938477)
\curveto(490.29300329,56.60937828)(490.25800332,56.5893783)(490.22800659,56.56938477)
\curveto(490.19800338,56.54937834)(490.16300342,56.52937836)(490.12300659,56.50938477)
\curveto(489.77300381,56.25937863)(489.51800406,55.884379)(489.35800659,55.38438477)
\curveto(489.32800425,55.30437958)(489.30800427,55.21937967)(489.29800659,55.12938477)
\curveto(489.28800429,55.04937984)(489.27300431,54.96937992)(489.25300659,54.88938477)
\curveto(489.23300435,54.83938005)(489.22800435,54.7893801)(489.23800659,54.73938477)
\curveto(489.24800433,54.69938019)(489.24300434,54.65938023)(489.22300659,54.61938477)
\lineto(489.22300659,54.30438477)
\curveto(489.21300437,54.27438061)(489.20800437,54.23938065)(489.20800659,54.19938477)
\curveto(489.21800436,54.15938073)(489.22300436,54.11438077)(489.22300659,54.06438477)
\lineto(489.22300659,53.61438477)
\lineto(489.22300659,52.17438477)
\lineto(489.22300659,50.85438477)
\lineto(489.22300659,50.50938477)
\curveto(489.22300436,50.39938449)(489.19800438,50.30938458)(489.14800659,50.23938477)
\curveto(489.09800448,50.15938473)(489.00800457,50.11938477)(488.87800659,50.11938477)
\curveto(488.75800482,50.10938478)(488.63300495,50.10438478)(488.50300659,50.10438477)
\curveto(488.42300516,50.10438478)(488.34800523,50.10938478)(488.27800659,50.11938477)
\curveto(488.20800537,50.12938476)(488.14800543,50.15438473)(488.09800659,50.19438477)
\curveto(488.01800556,50.24438464)(487.9780056,50.33938455)(487.97800659,50.47938477)
\lineto(487.97800659,50.88438477)
\lineto(487.97800659,52.65438477)
\lineto(487.97800659,56.28438477)
\lineto(487.97800659,57.19938477)
\lineto(487.97800659,57.46938477)
\curveto(487.9780056,57.55937733)(487.99800558,57.62937726)(488.03800659,57.67938477)
\curveto(488.06800551,57.73937715)(488.11800546,57.77937711)(488.18800659,57.79938477)
\curveto(488.22800535,57.80937708)(488.2830053,57.81937707)(488.35300659,57.82938477)
\curveto(488.43300515,57.83937705)(488.51300507,57.84437704)(488.59300659,57.84438477)
\curveto(488.67300491,57.84437704)(488.74800483,57.83937705)(488.81800659,57.82938477)
\curveto(488.89800468,57.81937707)(488.95300463,57.80437708)(488.98300659,57.78438477)
\curveto(489.09300449,57.71437717)(489.14300444,57.62437726)(489.13300659,57.51438477)
\curveto(489.12300446,57.41437747)(489.13800444,57.29937759)(489.17800659,57.16938477)
\curveto(489.19800438,57.10937778)(489.23800434,57.05937783)(489.29800659,57.01938477)
\curveto(489.41800416,57.00937788)(489.51300407,57.05437783)(489.58300659,57.15438477)
\curveto(489.66300392,57.25437763)(489.74300384,57.33437755)(489.82300659,57.39438477)
\curveto(489.96300362,57.49437739)(490.10300348,57.5843773)(490.24300659,57.66438477)
\curveto(490.39300319,57.75437713)(490.56300302,57.82937706)(490.75300659,57.88938477)
\curveto(490.83300275,57.91937697)(490.91800266,57.93937695)(491.00800659,57.94938477)
\curveto(491.10800247,57.95937693)(491.20300238,57.97437691)(491.29300659,57.99438477)
\curveto(491.34300224,58.00437688)(491.39300219,58.00937688)(491.44300659,58.00938477)
\lineto(491.59300659,58.00938477)
}
}
{
\newrgbcolor{curcolor}{0 0 0}
\pscustom[linestyle=none,fillstyle=solid,fillcolor=curcolor]
{
\newpath
\moveto(497.19761597,60.19938477)
\curveto(497.34761396,60.19937469)(497.49761381,60.19437469)(497.64761597,60.18438477)
\curveto(497.79761351,60.1843747)(497.9026134,60.14437474)(497.96261597,60.06438477)
\curveto(498.01261329,60.00437488)(498.03761327,59.91937497)(498.03761597,59.80938477)
\curveto(498.04761326,59.70937518)(498.05261325,59.60437528)(498.05261597,59.49438477)
\lineto(498.05261597,58.62438477)
\curveto(498.05261325,58.54437634)(498.04761326,58.45937643)(498.03761597,58.36938477)
\curveto(498.03761327,58.2893766)(498.04761326,58.21937667)(498.06761597,58.15938477)
\curveto(498.1076132,58.01937687)(498.19761311,57.92937696)(498.33761597,57.88938477)
\curveto(498.38761292,57.87937701)(498.43261287,57.87437701)(498.47261597,57.87438477)
\lineto(498.62261597,57.87438477)
\lineto(499.02761597,57.87438477)
\curveto(499.18761212,57.884377)(499.302612,57.87437701)(499.37261597,57.84438477)
\curveto(499.46261184,57.7843771)(499.52261178,57.72437716)(499.55261597,57.66438477)
\curveto(499.57261173,57.62437726)(499.58261172,57.57937731)(499.58261597,57.52938477)
\lineto(499.58261597,57.37938477)
\curveto(499.58261172,57.26937762)(499.57761173,57.16437772)(499.56761597,57.06438477)
\curveto(499.55761175,56.97437791)(499.52261178,56.90437798)(499.46261597,56.85438477)
\curveto(499.4026119,56.80437808)(499.31761199,56.77437811)(499.20761597,56.76438477)
\lineto(498.87761597,56.76438477)
\curveto(498.76761254,56.77437811)(498.65761265,56.77937811)(498.54761597,56.77938477)
\curveto(498.43761287,56.77937811)(498.34261296,56.76437812)(498.26261597,56.73438477)
\curveto(498.19261311,56.70437818)(498.14261316,56.65437823)(498.11261597,56.58438477)
\curveto(498.08261322,56.51437837)(498.06261324,56.42937846)(498.05261597,56.32938477)
\curveto(498.04261326,56.23937865)(498.03761327,56.13937875)(498.03761597,56.02938477)
\curveto(498.04761326,55.92937896)(498.05261325,55.82937906)(498.05261597,55.72938477)
\lineto(498.05261597,52.75938477)
\curveto(498.05261325,52.53938235)(498.04761326,52.30438258)(498.03761597,52.05438477)
\curveto(498.03761327,51.81438307)(498.08261322,51.62938326)(498.17261597,51.49938477)
\curveto(498.22261308,51.41938347)(498.28761302,51.36438352)(498.36761597,51.33438477)
\curveto(498.44761286,51.30438358)(498.54261276,51.27938361)(498.65261597,51.25938477)
\curveto(498.68261262,51.24938364)(498.71261259,51.24438364)(498.74261597,51.24438477)
\curveto(498.78261252,51.25438363)(498.81761249,51.25438363)(498.84761597,51.24438477)
\lineto(499.04261597,51.24438477)
\curveto(499.14261216,51.24438364)(499.23261207,51.23438365)(499.31261597,51.21438477)
\curveto(499.4026119,51.20438368)(499.46761184,51.16938372)(499.50761597,51.10938477)
\curveto(499.52761178,51.07938381)(499.54261176,51.02438386)(499.55261597,50.94438477)
\curveto(499.57261173,50.87438401)(499.58261172,50.79938409)(499.58261597,50.71938477)
\curveto(499.59261171,50.63938425)(499.59261171,50.55938433)(499.58261597,50.47938477)
\curveto(499.57261173,50.40938448)(499.55261175,50.35438453)(499.52261597,50.31438477)
\curveto(499.48261182,50.24438464)(499.4076119,50.19438469)(499.29761597,50.16438477)
\curveto(499.21761209,50.14438474)(499.12761218,50.13438475)(499.02761597,50.13438477)
\curveto(498.92761238,50.14438474)(498.83761247,50.14938474)(498.75761597,50.14938477)
\curveto(498.69761261,50.14938474)(498.63761267,50.14438474)(498.57761597,50.13438477)
\curveto(498.51761279,50.13438475)(498.46261284,50.13938475)(498.41261597,50.14938477)
\lineto(498.23261597,50.14938477)
\curveto(498.18261312,50.15938473)(498.13261317,50.16438472)(498.08261597,50.16438477)
\curveto(498.04261326,50.17438471)(497.99761331,50.17938471)(497.94761597,50.17938477)
\curveto(497.74761356,50.22938466)(497.57261373,50.2843846)(497.42261597,50.34438477)
\curveto(497.28261402,50.40438448)(497.16261414,50.50938438)(497.06261597,50.65938477)
\curveto(496.92261438,50.85938403)(496.84261446,51.10938378)(496.82261597,51.40938477)
\curveto(496.8026145,51.71938317)(496.79261451,52.04938284)(496.79261597,52.39938477)
\lineto(496.79261597,56.32938477)
\curveto(496.76261454,56.45937843)(496.73261457,56.55437833)(496.70261597,56.61438477)
\curveto(496.68261462,56.67437821)(496.61261469,56.72437816)(496.49261597,56.76438477)
\curveto(496.45261485,56.77437811)(496.41261489,56.77437811)(496.37261597,56.76438477)
\curveto(496.33261497,56.75437813)(496.29261501,56.75937813)(496.25261597,56.77938477)
\lineto(496.01261597,56.77938477)
\curveto(495.88261542,56.77937811)(495.77261553,56.7893781)(495.68261597,56.80938477)
\curveto(495.6026157,56.83937805)(495.54761576,56.89937799)(495.51761597,56.98938477)
\curveto(495.49761581,57.02937786)(495.48261582,57.07437781)(495.47261597,57.12438477)
\lineto(495.47261597,57.27438477)
\curveto(495.47261583,57.41437747)(495.48261582,57.52937736)(495.50261597,57.61938477)
\curveto(495.52261578,57.71937717)(495.58261572,57.79437709)(495.68261597,57.84438477)
\curveto(495.79261551,57.884377)(495.93261537,57.89437699)(496.10261597,57.87438477)
\curveto(496.28261502,57.85437703)(496.43261487,57.86437702)(496.55261597,57.90438477)
\curveto(496.64261466,57.95437693)(496.71261459,58.02437686)(496.76261597,58.11438477)
\curveto(496.78261452,58.17437671)(496.79261451,58.24937664)(496.79261597,58.33938477)
\lineto(496.79261597,58.59438477)
\lineto(496.79261597,59.52438477)
\lineto(496.79261597,59.76438477)
\curveto(496.79261451,59.85437503)(496.8026145,59.92937496)(496.82261597,59.98938477)
\curveto(496.86261444,60.06937482)(496.93761437,60.13437475)(497.04761597,60.18438477)
\curveto(497.07761423,60.1843747)(497.1026142,60.1843747)(497.12261597,60.18438477)
\curveto(497.15261415,60.19437469)(497.17761413,60.19937469)(497.19761597,60.19938477)
}
}
{
\newrgbcolor{curcolor}{0 0 0}
\pscustom[linestyle=none,fillstyle=solid,fillcolor=curcolor]
{
\newpath
\moveto(507.71941284,54.30438477)
\curveto(507.73940516,54.20438068)(507.73940516,54.0893808)(507.71941284,53.95938477)
\curveto(507.70940519,53.83938105)(507.67940522,53.75438113)(507.62941284,53.70438477)
\curveto(507.57940532,53.66438122)(507.50440539,53.63438125)(507.40441284,53.61438477)
\curveto(507.31440558,53.60438128)(507.20940569,53.59938129)(507.08941284,53.59938477)
\lineto(506.72941284,53.59938477)
\curveto(506.60940629,53.60938128)(506.50440639,53.61438127)(506.41441284,53.61438477)
\lineto(502.57441284,53.61438477)
\curveto(502.4944104,53.61438127)(502.41441048,53.60938128)(502.33441284,53.59938477)
\curveto(502.25441064,53.59938129)(502.18941071,53.5843813)(502.13941284,53.55438477)
\curveto(502.0994108,53.53438135)(502.05941084,53.49438139)(502.01941284,53.43438477)
\curveto(501.9994109,53.40438148)(501.97941092,53.35938153)(501.95941284,53.29938477)
\curveto(501.93941096,53.24938164)(501.93941096,53.19938169)(501.95941284,53.14938477)
\curveto(501.96941093,53.09938179)(501.97441092,53.05438183)(501.97441284,53.01438477)
\curveto(501.97441092,52.97438191)(501.97941092,52.93438195)(501.98941284,52.89438477)
\curveto(502.00941089,52.81438207)(502.02941087,52.72938216)(502.04941284,52.63938477)
\curveto(502.06941083,52.55938233)(502.0994108,52.47938241)(502.13941284,52.39938477)
\curveto(502.36941053,51.85938303)(502.74941015,51.47438341)(503.27941284,51.24438477)
\curveto(503.33940956,51.21438367)(503.40440949,51.1893837)(503.47441284,51.16938477)
\lineto(503.68441284,51.10938477)
\curveto(503.71440918,51.09938379)(503.76440913,51.09438379)(503.83441284,51.09438477)
\curveto(503.97440892,51.05438383)(504.15940874,51.03438385)(504.38941284,51.03438477)
\curveto(504.61940828,51.03438385)(504.80440809,51.05438383)(504.94441284,51.09438477)
\curveto(505.08440781,51.13438375)(505.20940769,51.17438371)(505.31941284,51.21438477)
\curveto(505.43940746,51.26438362)(505.54940735,51.32438356)(505.64941284,51.39438477)
\curveto(505.75940714,51.46438342)(505.85440704,51.54438334)(505.93441284,51.63438477)
\curveto(506.01440688,51.73438315)(506.08440681,51.83938305)(506.14441284,51.94938477)
\curveto(506.20440669,52.04938284)(506.25440664,52.15438273)(506.29441284,52.26438477)
\curveto(506.34440655,52.37438251)(506.42440647,52.45438243)(506.53441284,52.50438477)
\curveto(506.57440632,52.52438236)(506.63940626,52.53938235)(506.72941284,52.54938477)
\curveto(506.81940608,52.55938233)(506.90940599,52.55938233)(506.99941284,52.54938477)
\curveto(507.08940581,52.54938234)(507.17440572,52.54438234)(507.25441284,52.53438477)
\curveto(507.33440556,52.52438236)(507.38940551,52.50438238)(507.41941284,52.47438477)
\curveto(507.51940538,52.40438248)(507.54440535,52.2893826)(507.49441284,52.12938477)
\curveto(507.41440548,51.85938303)(507.30940559,51.61938327)(507.17941284,51.40938477)
\curveto(506.97940592,51.0893838)(506.74940615,50.82438406)(506.48941284,50.61438477)
\curveto(506.23940666,50.41438447)(505.91940698,50.24938464)(505.52941284,50.11938477)
\curveto(505.42940747,50.07938481)(505.32940757,50.05438483)(505.22941284,50.04438477)
\curveto(505.12940777,50.02438486)(505.02440787,50.00438488)(504.91441284,49.98438477)
\curveto(504.86440803,49.97438491)(504.81440808,49.96938492)(504.76441284,49.96938477)
\curveto(504.72440817,49.96938492)(504.67940822,49.96438492)(504.62941284,49.95438477)
\lineto(504.47941284,49.95438477)
\curveto(504.42940847,49.94438494)(504.36940853,49.93938495)(504.29941284,49.93938477)
\curveto(504.23940866,49.93938495)(504.18940871,49.94438494)(504.14941284,49.95438477)
\lineto(504.01441284,49.95438477)
\curveto(503.96440893,49.96438492)(503.91940898,49.96938492)(503.87941284,49.96938477)
\curveto(503.83940906,49.96938492)(503.7994091,49.97438491)(503.75941284,49.98438477)
\curveto(503.70940919,49.99438489)(503.65440924,50.00438488)(503.59441284,50.01438477)
\curveto(503.53440936,50.01438487)(503.47940942,50.01938487)(503.42941284,50.02938477)
\curveto(503.33940956,50.04938484)(503.24940965,50.07438481)(503.15941284,50.10438477)
\curveto(503.06940983,50.12438476)(502.98440991,50.14938474)(502.90441284,50.17938477)
\curveto(502.86441003,50.19938469)(502.82941007,50.20938468)(502.79941284,50.20938477)
\curveto(502.76941013,50.21938467)(502.73441016,50.23438465)(502.69441284,50.25438477)
\curveto(502.54441035,50.32438456)(502.38441051,50.40938448)(502.21441284,50.50938477)
\curveto(501.92441097,50.69938419)(501.67441122,50.92938396)(501.46441284,51.19938477)
\curveto(501.26441163,51.47938341)(501.0944118,51.7893831)(500.95441284,52.12938477)
\curveto(500.90441199,52.23938265)(500.86441203,52.35438253)(500.83441284,52.47438477)
\curveto(500.81441208,52.59438229)(500.78441211,52.71438217)(500.74441284,52.83438477)
\curveto(500.73441216,52.87438201)(500.72941217,52.90938198)(500.72941284,52.93938477)
\curveto(500.72941217,52.96938192)(500.72441217,53.00938188)(500.71441284,53.05938477)
\curveto(500.6944122,53.13938175)(500.67941222,53.22438166)(500.66941284,53.31438477)
\curveto(500.65941224,53.40438148)(500.64441225,53.49438139)(500.62441284,53.58438477)
\lineto(500.62441284,53.79438477)
\curveto(500.61441228,53.83438105)(500.60441229,53.889381)(500.59441284,53.95938477)
\curveto(500.5944123,54.03938085)(500.5994123,54.10438078)(500.60941284,54.15438477)
\lineto(500.60941284,54.31938477)
\curveto(500.62941227,54.36938052)(500.63441226,54.41938047)(500.62441284,54.46938477)
\curveto(500.62441227,54.52938036)(500.62941227,54.5843803)(500.63941284,54.63438477)
\curveto(500.67941222,54.79438009)(500.70941219,54.95437993)(500.72941284,55.11438477)
\curveto(500.75941214,55.27437961)(500.80441209,55.42437946)(500.86441284,55.56438477)
\curveto(500.91441198,55.67437921)(500.95941194,55.7843791)(500.99941284,55.89438477)
\curveto(501.04941185,56.01437887)(501.10441179,56.12937876)(501.16441284,56.23938477)
\curveto(501.38441151,56.5893783)(501.63441126,56.889378)(501.91441284,57.13938477)
\curveto(502.1944107,57.39937749)(502.53941036,57.61437727)(502.94941284,57.78438477)
\curveto(503.06940983,57.83437705)(503.18940971,57.86937702)(503.30941284,57.88938477)
\curveto(503.43940946,57.91937697)(503.57440932,57.94937694)(503.71441284,57.97938477)
\curveto(503.76440913,57.9893769)(503.80940909,57.99437689)(503.84941284,57.99438477)
\curveto(503.88940901,58.00437688)(503.93440896,58.00937688)(503.98441284,58.00938477)
\curveto(504.00440889,58.01937687)(504.02940887,58.01937687)(504.05941284,58.00938477)
\curveto(504.08940881,57.99937689)(504.11440878,58.00437688)(504.13441284,58.02438477)
\curveto(504.55440834,58.03437685)(504.91940798,57.9893769)(505.22941284,57.88938477)
\curveto(505.53940736,57.79937709)(505.81940708,57.67437721)(506.06941284,57.51438477)
\curveto(506.11940678,57.49437739)(506.15940674,57.46437742)(506.18941284,57.42438477)
\curveto(506.21940668,57.39437749)(506.25440664,57.36937752)(506.29441284,57.34938477)
\curveto(506.37440652,57.2893776)(506.45440644,57.21937767)(506.53441284,57.13938477)
\curveto(506.62440627,57.05937783)(506.6994062,56.97937791)(506.75941284,56.89938477)
\curveto(506.91940598,56.6893782)(507.05440584,56.4893784)(507.16441284,56.29938477)
\curveto(507.23440566,56.1893787)(507.28940561,56.06937882)(507.32941284,55.93938477)
\curveto(507.36940553,55.80937908)(507.41440548,55.67937921)(507.46441284,55.54938477)
\curveto(507.51440538,55.41937947)(507.54940535,55.2843796)(507.56941284,55.14438477)
\curveto(507.5994053,55.00437988)(507.63440526,54.86438002)(507.67441284,54.72438477)
\curveto(507.68440521,54.65438023)(507.68940521,54.5843803)(507.68941284,54.51438477)
\lineto(507.71941284,54.30438477)
\moveto(506.26441284,54.81438477)
\curveto(506.2944066,54.85438003)(506.31940658,54.90437998)(506.33941284,54.96438477)
\curveto(506.35940654,55.03437985)(506.35940654,55.10437978)(506.33941284,55.17438477)
\curveto(506.27940662,55.39437949)(506.1944067,55.59937929)(506.08441284,55.78938477)
\curveto(505.94440695,56.01937887)(505.78940711,56.21437867)(505.61941284,56.37438477)
\curveto(505.44940745,56.53437835)(505.22940767,56.66937822)(504.95941284,56.77938477)
\curveto(504.88940801,56.79937809)(504.81940808,56.81437807)(504.74941284,56.82438477)
\curveto(504.67940822,56.84437804)(504.60440829,56.86437802)(504.52441284,56.88438477)
\curveto(504.44440845,56.90437798)(504.35940854,56.91437797)(504.26941284,56.91438477)
\lineto(504.01441284,56.91438477)
\curveto(503.98440891,56.89437799)(503.94940895,56.884378)(503.90941284,56.88438477)
\curveto(503.86940903,56.89437799)(503.83440906,56.89437799)(503.80441284,56.88438477)
\lineto(503.56441284,56.82438477)
\curveto(503.4944094,56.81437807)(503.42440947,56.79937809)(503.35441284,56.77938477)
\curveto(503.06440983,56.65937823)(502.82941007,56.50937838)(502.64941284,56.32938477)
\curveto(502.47941042,56.14937874)(502.32441057,55.92437896)(502.18441284,55.65438477)
\curveto(502.15441074,55.60437928)(502.12441077,55.53937935)(502.09441284,55.45938477)
\curveto(502.06441083,55.3893795)(502.03941086,55.30937958)(502.01941284,55.21938477)
\curveto(501.9994109,55.12937976)(501.9944109,55.04437984)(502.00441284,54.96438477)
\curveto(502.01441088,54.88438)(502.04941085,54.82438006)(502.10941284,54.78438477)
\curveto(502.18941071,54.72438016)(502.32441057,54.69438019)(502.51441284,54.69438477)
\curveto(502.71441018,54.70438018)(502.88441001,54.70938018)(503.02441284,54.70938477)
\lineto(505.30441284,54.70938477)
\curveto(505.45440744,54.70938018)(505.63440726,54.70438018)(505.84441284,54.69438477)
\curveto(506.05440684,54.69438019)(506.1944067,54.73438015)(506.26441284,54.81438477)
}
}
{
\newrgbcolor{curcolor}{0 0 0}
\pscustom[linestyle=none,fillstyle=solid,fillcolor=curcolor]
{
\newpath
\moveto(628.93253418,55.74439941)
\lineto(628.93253418,55.47439941)
\curveto(628.94252421,55.38439416)(628.93752421,55.30439424)(628.91753418,55.23439941)
\lineto(628.91753418,55.08439941)
\curveto(628.90752424,55.05439449)(628.90252425,55.01939453)(628.90253418,54.97939941)
\curveto(628.91252424,54.93939461)(628.91252424,54.90939464)(628.90253418,54.88939941)
\curveto(628.89252426,54.83939471)(628.88752426,54.78439476)(628.88753418,54.72439941)
\curveto(628.88752426,54.67439487)(628.88252427,54.62439492)(628.87253418,54.57439941)
\curveto(628.84252431,54.43439511)(628.82252433,54.28439526)(628.81253418,54.12439941)
\curveto(628.80252435,53.97439557)(628.77252438,53.82939572)(628.72253418,53.68939941)
\curveto(628.69252446,53.56939598)(628.65752449,53.4443961)(628.61753418,53.31439941)
\curveto(628.58752456,53.19439635)(628.5475246,53.07439647)(628.49753418,52.95439941)
\curveto(628.32752482,52.52439702)(628.11252504,52.13439741)(627.85253418,51.78439941)
\curveto(627.60252555,51.4443981)(627.28752586,51.15439839)(626.90753418,50.91439941)
\curveto(626.71752643,50.79439875)(626.51252664,50.68939886)(626.29253418,50.59939941)
\curveto(626.08252707,50.51939903)(625.8525273,50.43939911)(625.60253418,50.35939941)
\curveto(625.49252766,50.31939923)(625.37252778,50.28939926)(625.24253418,50.26939941)
\curveto(625.12252803,50.25939929)(625.00252815,50.23939931)(624.88253418,50.20939941)
\curveto(624.77252838,50.18939936)(624.66252849,50.17939937)(624.55253418,50.17939941)
\curveto(624.4525287,50.17939937)(624.3525288,50.16939938)(624.25253418,50.14939941)
\lineto(624.04253418,50.14939941)
\curveto(624.01252914,50.13939941)(623.97752917,50.13439941)(623.93753418,50.13439941)
\curveto(623.89752925,50.1443994)(623.85752929,50.1493994)(623.81753418,50.14939941)
\lineto(620.81753418,50.14939941)
\curveto(620.66753248,50.1493994)(620.53253262,50.15439939)(620.41253418,50.16439941)
\curveto(620.30253285,50.18439936)(620.22753292,50.2493993)(620.18753418,50.35939941)
\curveto(620.147533,50.43939911)(620.12753302,50.55439899)(620.12753418,50.70439941)
\curveto(620.13753301,50.85439869)(620.14253301,50.98939856)(620.14253418,51.10939941)
\lineto(620.14253418,59.97439941)
\curveto(620.14253301,60.09438945)(620.13753301,60.21938933)(620.12753418,60.34939941)
\curveto(620.12753302,60.48938906)(620.152533,60.59938895)(620.20253418,60.67939941)
\curveto(620.24253291,60.7493888)(620.31753283,60.79438875)(620.42753418,60.81439941)
\curveto(620.4475327,60.82438872)(620.46753268,60.82438872)(620.48753418,60.81439941)
\curveto(620.50753264,60.81438873)(620.52753262,60.81938873)(620.54753418,60.82939941)
\lineto(623.80253418,60.82939941)
\curveto(623.8525293,60.82938872)(623.89752925,60.82938872)(623.93753418,60.82939941)
\curveto(623.98752916,60.83938871)(624.03252912,60.83938871)(624.07253418,60.82939941)
\curveto(624.12252903,60.80938874)(624.17252898,60.80438874)(624.22253418,60.81439941)
\curveto(624.28252887,60.82438872)(624.33752881,60.82438872)(624.38753418,60.81439941)
\curveto(624.43752871,60.80438874)(624.49252866,60.79938875)(624.55253418,60.79939941)
\curveto(624.61252854,60.79938875)(624.66752848,60.79438875)(624.71753418,60.78439941)
\curveto(624.76752838,60.77438877)(624.81252834,60.76938878)(624.85253418,60.76939941)
\curveto(624.90252825,60.76938878)(624.9525282,60.76438878)(625.00253418,60.75439941)
\curveto(625.11252804,60.73438881)(625.21752793,60.71438883)(625.31753418,60.69439941)
\curveto(625.41752773,60.68438886)(625.51752763,60.66438888)(625.61753418,60.63439941)
\curveto(625.83752731,60.56438898)(626.0475271,60.49438905)(626.24753418,60.42439941)
\curveto(626.4475267,60.36438918)(626.63252652,60.27938927)(626.80253418,60.16939941)
\curveto(626.94252621,60.08938946)(627.06752608,60.00938954)(627.17753418,59.92939941)
\curveto(627.20752594,59.90938964)(627.23752591,59.88438966)(627.26753418,59.85439941)
\curveto(627.29752585,59.83438971)(627.32752582,59.81438973)(627.35753418,59.79439941)
\curveto(627.41752573,59.7443898)(627.47252568,59.69438985)(627.52253418,59.64439941)
\curveto(627.57252558,59.59438995)(627.62252553,59.54439)(627.67253418,59.49439941)
\curveto(627.72252543,59.4443901)(627.76252539,59.40939014)(627.79253418,59.38939941)
\curveto(627.83252532,59.32939022)(627.87252528,59.27439027)(627.91253418,59.22439941)
\curveto(627.96252519,59.17439037)(628.00752514,59.11939043)(628.04753418,59.05939941)
\curveto(628.09752505,58.99939055)(628.13752501,58.93439061)(628.16753418,58.86439941)
\curveto(628.20752494,58.80439074)(628.2525249,58.73939081)(628.30253418,58.66939941)
\curveto(628.32252483,58.62939092)(628.33752481,58.59439095)(628.34753418,58.56439941)
\curveto(628.35752479,58.53439101)(628.37252478,58.49939105)(628.39253418,58.45939941)
\curveto(628.43252472,58.37939117)(628.46752468,58.29939125)(628.49753418,58.21939941)
\curveto(628.52752462,58.1493914)(628.56252459,58.07439147)(628.60253418,57.99439941)
\curveto(628.64252451,57.88439166)(628.67252448,57.76939178)(628.69253418,57.64939941)
\curveto(628.72252443,57.53939201)(628.7525244,57.42939212)(628.78253418,57.31939941)
\curveto(628.80252435,57.25939229)(628.81252434,57.19939235)(628.81253418,57.13939941)
\curveto(628.81252434,57.08939246)(628.82252433,57.03439251)(628.84253418,56.97439941)
\curveto(628.89252426,56.79439275)(628.91752423,56.59439295)(628.91753418,56.37439941)
\curveto(628.92752422,56.16439338)(628.93252422,55.95439359)(628.93253418,55.74439941)
\moveto(627.50753418,54.96439941)
\curveto(627.52752562,55.06439448)(627.53752561,55.16939438)(627.53753418,55.27939941)
\lineto(627.53753418,55.62439941)
\lineto(627.53753418,55.84939941)
\curveto(627.5475256,55.92939362)(627.54252561,56.00439354)(627.52253418,56.07439941)
\curveto(627.52252563,56.10439344)(627.51752563,56.13439341)(627.50753418,56.16439941)
\lineto(627.50753418,56.26939941)
\curveto(627.48752566,56.37939317)(627.47252568,56.48939306)(627.46253418,56.59939941)
\curveto(627.46252569,56.70939284)(627.4475257,56.81939273)(627.41753418,56.92939941)
\curveto(627.39752575,57.00939254)(627.37752577,57.08439246)(627.35753418,57.15439941)
\curveto(627.3475258,57.23439231)(627.33252582,57.31439223)(627.31253418,57.39439941)
\curveto(627.20252595,57.75439179)(627.06252609,58.06939148)(626.89253418,58.33939941)
\curveto(626.61252654,58.78939076)(626.19752695,59.12939042)(625.64753418,59.35939941)
\curveto(625.55752759,59.40939014)(625.46252769,59.4443901)(625.36253418,59.46439941)
\curveto(625.26252789,59.49439005)(625.15752799,59.52439002)(625.04753418,59.55439941)
\curveto(624.93752821,59.58438996)(624.82252833,59.59938995)(624.70253418,59.59939941)
\curveto(624.59252856,59.60938994)(624.48252867,59.62438992)(624.37253418,59.64439941)
\lineto(624.05753418,59.64439941)
\curveto(624.02752912,59.65438989)(623.99252916,59.65938989)(623.95253418,59.65939941)
\lineto(623.83253418,59.65939941)
\lineto(622.00253418,59.65939941)
\curveto(621.98253117,59.6493899)(621.95753119,59.6443899)(621.92753418,59.64439941)
\curveto(621.89753125,59.65438989)(621.87253128,59.65438989)(621.85253418,59.64439941)
\lineto(621.70253418,59.58439941)
\curveto(621.66253149,59.56438998)(621.63253152,59.53439001)(621.61253418,59.49439941)
\curveto(621.59253156,59.45439009)(621.57253158,59.38439016)(621.55253418,59.28439941)
\lineto(621.55253418,59.16439941)
\curveto(621.54253161,59.12439042)(621.53753161,59.07939047)(621.53753418,59.02939941)
\lineto(621.53753418,58.89439941)
\lineto(621.53753418,52.08439941)
\lineto(621.53753418,51.93439941)
\curveto(621.53753161,51.89439765)(621.54253161,51.85439769)(621.55253418,51.81439941)
\lineto(621.55253418,51.69439941)
\curveto(621.57253158,51.59439795)(621.59253156,51.52439802)(621.61253418,51.48439941)
\curveto(621.69253146,51.36439818)(621.84253131,51.30439824)(622.06253418,51.30439941)
\curveto(622.28253087,51.31439823)(622.49253066,51.31939823)(622.69253418,51.31939941)
\lineto(623.56253418,51.31939941)
\curveto(623.63252952,51.31939823)(623.70752944,51.31439823)(623.78753418,51.30439941)
\curveto(623.86752928,51.30439824)(623.93752921,51.31439823)(623.99753418,51.33439941)
\lineto(624.16253418,51.33439941)
\curveto(624.21252894,51.3443982)(624.26752888,51.3443982)(624.32753418,51.33439941)
\curveto(624.38752876,51.33439821)(624.4475287,51.33939821)(624.50753418,51.34939941)
\curveto(624.56752858,51.36939818)(624.62752852,51.37939817)(624.68753418,51.37939941)
\curveto(624.7475284,51.38939816)(624.81252834,51.40439814)(624.88253418,51.42439941)
\curveto(624.99252816,51.45439809)(625.09752805,51.48439806)(625.19753418,51.51439941)
\curveto(625.30752784,51.544398)(625.41752773,51.58439796)(625.52753418,51.63439941)
\curveto(625.89752725,51.79439775)(626.21252694,52.00939754)(626.47253418,52.27939941)
\curveto(626.74252641,52.55939699)(626.96252619,52.88939666)(627.13253418,53.26939941)
\curveto(627.18252597,53.37939617)(627.22252593,53.49439605)(627.25253418,53.61439941)
\lineto(627.37253418,54.00439941)
\curveto(627.40252575,54.11439543)(627.42252573,54.22939532)(627.43253418,54.34939941)
\curveto(627.4525257,54.47939507)(627.47252568,54.60439494)(627.49253418,54.72439941)
\curveto(627.50252565,54.77439477)(627.50752564,54.81439473)(627.50753418,54.84439941)
\lineto(627.50753418,54.96439941)
}
}
{
\newrgbcolor{curcolor}{0 0 0}
\pscustom[linestyle=none,fillstyle=solid,fillcolor=curcolor]
{
\newpath
\moveto(637.18440918,54.31939941)
\curveto(637.20440149,54.21939533)(637.20440149,54.10439544)(637.18440918,53.97439941)
\curveto(637.17440152,53.85439569)(637.14440155,53.76939578)(637.09440918,53.71939941)
\curveto(637.04440165,53.67939587)(636.96940173,53.6493959)(636.86940918,53.62939941)
\curveto(636.77940192,53.61939593)(636.67440202,53.61439593)(636.55440918,53.61439941)
\lineto(636.19440918,53.61439941)
\curveto(636.07440262,53.62439592)(635.96940273,53.62939592)(635.87940918,53.62939941)
\lineto(632.03940918,53.62939941)
\curveto(631.95940674,53.62939592)(631.87940682,53.62439592)(631.79940918,53.61439941)
\curveto(631.71940698,53.61439593)(631.65440704,53.59939595)(631.60440918,53.56939941)
\curveto(631.56440713,53.549396)(631.52440717,53.50939604)(631.48440918,53.44939941)
\curveto(631.46440723,53.41939613)(631.44440725,53.37439617)(631.42440918,53.31439941)
\curveto(631.40440729,53.26439628)(631.40440729,53.21439633)(631.42440918,53.16439941)
\curveto(631.43440726,53.11439643)(631.43940726,53.06939648)(631.43940918,53.02939941)
\curveto(631.43940726,52.98939656)(631.44440725,52.9493966)(631.45440918,52.90939941)
\curveto(631.47440722,52.82939672)(631.4944072,52.7443968)(631.51440918,52.65439941)
\curveto(631.53440716,52.57439697)(631.56440713,52.49439705)(631.60440918,52.41439941)
\curveto(631.83440686,51.87439767)(632.21440648,51.48939806)(632.74440918,51.25939941)
\curveto(632.80440589,51.22939832)(632.86940583,51.20439834)(632.93940918,51.18439941)
\lineto(633.14940918,51.12439941)
\curveto(633.17940552,51.11439843)(633.22940547,51.10939844)(633.29940918,51.10939941)
\curveto(633.43940526,51.06939848)(633.62440507,51.0493985)(633.85440918,51.04939941)
\curveto(634.08440461,51.0493985)(634.26940443,51.06939848)(634.40940918,51.10939941)
\curveto(634.54940415,51.1493984)(634.67440402,51.18939836)(634.78440918,51.22939941)
\curveto(634.90440379,51.27939827)(635.01440368,51.33939821)(635.11440918,51.40939941)
\curveto(635.22440347,51.47939807)(635.31940338,51.55939799)(635.39940918,51.64939941)
\curveto(635.47940322,51.7493978)(635.54940315,51.85439769)(635.60940918,51.96439941)
\curveto(635.66940303,52.06439748)(635.71940298,52.16939738)(635.75940918,52.27939941)
\curveto(635.80940289,52.38939716)(635.88940281,52.46939708)(635.99940918,52.51939941)
\curveto(636.03940266,52.53939701)(636.10440259,52.55439699)(636.19440918,52.56439941)
\curveto(636.28440241,52.57439697)(636.37440232,52.57439697)(636.46440918,52.56439941)
\curveto(636.55440214,52.56439698)(636.63940206,52.55939699)(636.71940918,52.54939941)
\curveto(636.7994019,52.53939701)(636.85440184,52.51939703)(636.88440918,52.48939941)
\curveto(636.98440171,52.41939713)(637.00940169,52.30439724)(636.95940918,52.14439941)
\curveto(636.87940182,51.87439767)(636.77440192,51.63439791)(636.64440918,51.42439941)
\curveto(636.44440225,51.10439844)(636.21440248,50.83939871)(635.95440918,50.62939941)
\curveto(635.70440299,50.42939912)(635.38440331,50.26439928)(634.99440918,50.13439941)
\curveto(634.8944038,50.09439945)(634.7944039,50.06939948)(634.69440918,50.05939941)
\curveto(634.5944041,50.03939951)(634.48940421,50.01939953)(634.37940918,49.99939941)
\curveto(634.32940437,49.98939956)(634.27940442,49.98439956)(634.22940918,49.98439941)
\curveto(634.18940451,49.98439956)(634.14440455,49.97939957)(634.09440918,49.96939941)
\lineto(633.94440918,49.96939941)
\curveto(633.8944048,49.95939959)(633.83440486,49.95439959)(633.76440918,49.95439941)
\curveto(633.70440499,49.95439959)(633.65440504,49.95939959)(633.61440918,49.96939941)
\lineto(633.47940918,49.96939941)
\curveto(633.42940527,49.97939957)(633.38440531,49.98439956)(633.34440918,49.98439941)
\curveto(633.30440539,49.98439956)(633.26440543,49.98939956)(633.22440918,49.99939941)
\curveto(633.17440552,50.00939954)(633.11940558,50.01939953)(633.05940918,50.02939941)
\curveto(632.9994057,50.02939952)(632.94440575,50.03439951)(632.89440918,50.04439941)
\curveto(632.80440589,50.06439948)(632.71440598,50.08939946)(632.62440918,50.11939941)
\curveto(632.53440616,50.13939941)(632.44940625,50.16439938)(632.36940918,50.19439941)
\curveto(632.32940637,50.21439933)(632.2944064,50.22439932)(632.26440918,50.22439941)
\curveto(632.23440646,50.23439931)(632.1994065,50.2493993)(632.15940918,50.26939941)
\curveto(632.00940669,50.33939921)(631.84940685,50.42439912)(631.67940918,50.52439941)
\curveto(631.38940731,50.71439883)(631.13940756,50.9443986)(630.92940918,51.21439941)
\curveto(630.72940797,51.49439805)(630.55940814,51.80439774)(630.41940918,52.14439941)
\curveto(630.36940833,52.25439729)(630.32940837,52.36939718)(630.29940918,52.48939941)
\curveto(630.27940842,52.60939694)(630.24940845,52.72939682)(630.20940918,52.84939941)
\curveto(630.1994085,52.88939666)(630.1944085,52.92439662)(630.19440918,52.95439941)
\curveto(630.1944085,52.98439656)(630.18940851,53.02439652)(630.17940918,53.07439941)
\curveto(630.15940854,53.15439639)(630.14440855,53.23939631)(630.13440918,53.32939941)
\curveto(630.12440857,53.41939613)(630.10940859,53.50939604)(630.08940918,53.59939941)
\lineto(630.08940918,53.80939941)
\curveto(630.07940862,53.8493957)(630.06940863,53.90439564)(630.05940918,53.97439941)
\curveto(630.05940864,54.05439549)(630.06440863,54.11939543)(630.07440918,54.16939941)
\lineto(630.07440918,54.33439941)
\curveto(630.0944086,54.38439516)(630.0994086,54.43439511)(630.08940918,54.48439941)
\curveto(630.08940861,54.544395)(630.0944086,54.59939495)(630.10440918,54.64939941)
\curveto(630.14440855,54.80939474)(630.17440852,54.96939458)(630.19440918,55.12939941)
\curveto(630.22440847,55.28939426)(630.26940843,55.43939411)(630.32940918,55.57939941)
\curveto(630.37940832,55.68939386)(630.42440827,55.79939375)(630.46440918,55.90939941)
\curveto(630.51440818,56.02939352)(630.56940813,56.1443934)(630.62940918,56.25439941)
\curveto(630.84940785,56.60439294)(631.0994076,56.90439264)(631.37940918,57.15439941)
\curveto(631.65940704,57.41439213)(632.00440669,57.62939192)(632.41440918,57.79939941)
\curveto(632.53440616,57.8493917)(632.65440604,57.88439166)(632.77440918,57.90439941)
\curveto(632.90440579,57.93439161)(633.03940566,57.96439158)(633.17940918,57.99439941)
\curveto(633.22940547,58.00439154)(633.27440542,58.00939154)(633.31440918,58.00939941)
\curveto(633.35440534,58.01939153)(633.3994053,58.02439152)(633.44940918,58.02439941)
\curveto(633.46940523,58.03439151)(633.4944052,58.03439151)(633.52440918,58.02439941)
\curveto(633.55440514,58.01439153)(633.57940512,58.01939153)(633.59940918,58.03939941)
\curveto(634.01940468,58.0493915)(634.38440431,58.00439154)(634.69440918,57.90439941)
\curveto(635.00440369,57.81439173)(635.28440341,57.68939186)(635.53440918,57.52939941)
\curveto(635.58440311,57.50939204)(635.62440307,57.47939207)(635.65440918,57.43939941)
\curveto(635.68440301,57.40939214)(635.71940298,57.38439216)(635.75940918,57.36439941)
\curveto(635.83940286,57.30439224)(635.91940278,57.23439231)(635.99940918,57.15439941)
\curveto(636.08940261,57.07439247)(636.16440253,56.99439255)(636.22440918,56.91439941)
\curveto(636.38440231,56.70439284)(636.51940218,56.50439304)(636.62940918,56.31439941)
\curveto(636.699402,56.20439334)(636.75440194,56.08439346)(636.79440918,55.95439941)
\curveto(636.83440186,55.82439372)(636.87940182,55.69439385)(636.92940918,55.56439941)
\curveto(636.97940172,55.43439411)(637.01440168,55.29939425)(637.03440918,55.15939941)
\curveto(637.06440163,55.01939453)(637.0994016,54.87939467)(637.13940918,54.73939941)
\curveto(637.14940155,54.66939488)(637.15440154,54.59939495)(637.15440918,54.52939941)
\lineto(637.18440918,54.31939941)
\moveto(635.72940918,54.82939941)
\curveto(635.75940294,54.86939468)(635.78440291,54.91939463)(635.80440918,54.97939941)
\curveto(635.82440287,55.0493945)(635.82440287,55.11939443)(635.80440918,55.18939941)
\curveto(635.74440295,55.40939414)(635.65940304,55.61439393)(635.54940918,55.80439941)
\curveto(635.40940329,56.03439351)(635.25440344,56.22939332)(635.08440918,56.38939941)
\curveto(634.91440378,56.549393)(634.694404,56.68439286)(634.42440918,56.79439941)
\curveto(634.35440434,56.81439273)(634.28440441,56.82939272)(634.21440918,56.83939941)
\curveto(634.14440455,56.85939269)(634.06940463,56.87939267)(633.98940918,56.89939941)
\curveto(633.90940479,56.91939263)(633.82440487,56.92939262)(633.73440918,56.92939941)
\lineto(633.47940918,56.92939941)
\curveto(633.44940525,56.90939264)(633.41440528,56.89939265)(633.37440918,56.89939941)
\curveto(633.33440536,56.90939264)(633.2994054,56.90939264)(633.26940918,56.89939941)
\lineto(633.02940918,56.83939941)
\curveto(632.95940574,56.82939272)(632.88940581,56.81439273)(632.81940918,56.79439941)
\curveto(632.52940617,56.67439287)(632.2944064,56.52439302)(632.11440918,56.34439941)
\curveto(631.94440675,56.16439338)(631.78940691,55.93939361)(631.64940918,55.66939941)
\curveto(631.61940708,55.61939393)(631.58940711,55.55439399)(631.55940918,55.47439941)
\curveto(631.52940717,55.40439414)(631.50440719,55.32439422)(631.48440918,55.23439941)
\curveto(631.46440723,55.1443944)(631.45940724,55.05939449)(631.46940918,54.97939941)
\curveto(631.47940722,54.89939465)(631.51440718,54.83939471)(631.57440918,54.79939941)
\curveto(631.65440704,54.73939481)(631.78940691,54.70939484)(631.97940918,54.70939941)
\curveto(632.17940652,54.71939483)(632.34940635,54.72439482)(632.48940918,54.72439941)
\lineto(634.76940918,54.72439941)
\curveto(634.91940378,54.72439482)(635.0994036,54.71939483)(635.30940918,54.70939941)
\curveto(635.51940318,54.70939484)(635.65940304,54.7493948)(635.72940918,54.82939941)
}
}
{
\newrgbcolor{curcolor}{0 0 0}
\pscustom[linestyle=none,fillstyle=solid,fillcolor=curcolor]
{
\newpath
\moveto(640.9210498,58.05439941)
\curveto(641.64104574,58.06439148)(642.24604513,57.97939157)(642.7360498,57.79939941)
\curveto(643.22604415,57.62939192)(643.60604377,57.32439222)(643.8760498,56.88439941)
\curveto(643.94604343,56.77439277)(644.00104338,56.65939289)(644.0410498,56.53939941)
\curveto(644.0810433,56.42939312)(644.12104326,56.30439324)(644.1610498,56.16439941)
\curveto(644.1810432,56.09439345)(644.18604319,56.01939353)(644.1760498,55.93939941)
\curveto(644.16604321,55.86939368)(644.15104323,55.81439373)(644.1310498,55.77439941)
\curveto(644.11104327,55.75439379)(644.08604329,55.73439381)(644.0560498,55.71439941)
\curveto(644.02604335,55.70439384)(644.00104338,55.68939386)(643.9810498,55.66939941)
\curveto(643.93104345,55.6493939)(643.8810435,55.6443939)(643.8310498,55.65439941)
\curveto(643.7810436,55.66439388)(643.73104365,55.66439388)(643.6810498,55.65439941)
\curveto(643.60104378,55.63439391)(643.49604388,55.62939392)(643.3660498,55.63939941)
\curveto(643.23604414,55.65939389)(643.14604423,55.68439386)(643.0960498,55.71439941)
\curveto(643.01604436,55.76439378)(642.96104442,55.82939372)(642.9310498,55.90939941)
\curveto(642.91104447,55.99939355)(642.8760445,56.08439346)(642.8260498,56.16439941)
\curveto(642.73604464,56.32439322)(642.61104477,56.46939308)(642.4510498,56.59939941)
\curveto(642.34104504,56.67939287)(642.22104516,56.73939281)(642.0910498,56.77939941)
\curveto(641.96104542,56.81939273)(641.82104556,56.85939269)(641.6710498,56.89939941)
\curveto(641.62104576,56.91939263)(641.57104581,56.92439262)(641.5210498,56.91439941)
\curveto(641.47104591,56.91439263)(641.42104596,56.91939263)(641.3710498,56.92939941)
\curveto(641.31104607,56.9493926)(641.23604614,56.95939259)(641.1460498,56.95939941)
\curveto(641.05604632,56.95939259)(640.9810464,56.9493926)(640.9210498,56.92939941)
\lineto(640.8310498,56.92939941)
\lineto(640.6810498,56.89939941)
\curveto(640.63104675,56.89939265)(640.5810468,56.89439265)(640.5310498,56.88439941)
\curveto(640.27104711,56.82439272)(640.05604732,56.73939281)(639.8860498,56.62939941)
\curveto(639.71604766,56.51939303)(639.60104778,56.33439321)(639.5410498,56.07439941)
\curveto(639.52104786,56.00439354)(639.51604786,55.93439361)(639.5260498,55.86439941)
\curveto(639.54604783,55.79439375)(639.56604781,55.73439381)(639.5860498,55.68439941)
\curveto(639.64604773,55.53439401)(639.71604766,55.42439412)(639.7960498,55.35439941)
\curveto(639.88604749,55.29439425)(639.99604738,55.22439432)(640.1260498,55.14439941)
\curveto(640.28604709,55.0443945)(640.46604691,54.96939458)(640.6660498,54.91939941)
\curveto(640.86604651,54.87939467)(641.06604631,54.82939472)(641.2660498,54.76939941)
\curveto(641.39604598,54.72939482)(641.52604585,54.69939485)(641.6560498,54.67939941)
\curveto(641.78604559,54.65939489)(641.91604546,54.62939492)(642.0460498,54.58939941)
\curveto(642.25604512,54.52939502)(642.46104492,54.46939508)(642.6610498,54.40939941)
\curveto(642.86104452,54.35939519)(643.06104432,54.29439525)(643.2610498,54.21439941)
\lineto(643.4110498,54.15439941)
\curveto(643.46104392,54.13439541)(643.51104387,54.10939544)(643.5610498,54.07939941)
\curveto(643.76104362,53.95939559)(643.93604344,53.82439572)(644.0860498,53.67439941)
\curveto(644.23604314,53.52439602)(644.36104302,53.33439621)(644.4610498,53.10439941)
\curveto(644.4810429,53.03439651)(644.50104288,52.93939661)(644.5210498,52.81939941)
\curveto(644.54104284,52.7493968)(644.55104283,52.67439687)(644.5510498,52.59439941)
\curveto(644.56104282,52.52439702)(644.56604281,52.4443971)(644.5660498,52.35439941)
\lineto(644.5660498,52.20439941)
\curveto(644.54604283,52.13439741)(644.53604284,52.06439748)(644.5360498,51.99439941)
\curveto(644.53604284,51.92439762)(644.52604285,51.85439769)(644.5060498,51.78439941)
\curveto(644.4760429,51.67439787)(644.44104294,51.56939798)(644.4010498,51.46939941)
\curveto(644.36104302,51.36939818)(644.31604306,51.27939827)(644.2660498,51.19939941)
\curveto(644.10604327,50.93939861)(643.90104348,50.72939882)(643.6510498,50.56939941)
\curveto(643.40104398,50.41939913)(643.12104426,50.28939926)(642.8110498,50.17939941)
\curveto(642.72104466,50.1493994)(642.62604475,50.12939942)(642.5260498,50.11939941)
\curveto(642.43604494,50.09939945)(642.34604503,50.07439947)(642.2560498,50.04439941)
\curveto(642.15604522,50.02439952)(642.05604532,50.01439953)(641.9560498,50.01439941)
\curveto(641.85604552,50.01439953)(641.75604562,50.00439954)(641.6560498,49.98439941)
\lineto(641.5060498,49.98439941)
\curveto(641.45604592,49.97439957)(641.38604599,49.96939958)(641.2960498,49.96939941)
\curveto(641.20604617,49.96939958)(641.13604624,49.97439957)(641.0860498,49.98439941)
\lineto(640.9210498,49.98439941)
\curveto(640.86104652,50.00439954)(640.79604658,50.01439953)(640.7260498,50.01439941)
\curveto(640.65604672,50.00439954)(640.59604678,50.00939954)(640.5460498,50.02939941)
\curveto(640.49604688,50.03939951)(640.43104695,50.0443995)(640.3510498,50.04439941)
\lineto(640.1110498,50.10439941)
\curveto(640.04104734,50.11439943)(639.96604741,50.13439941)(639.8860498,50.16439941)
\curveto(639.5760478,50.26439928)(639.30604807,50.38939916)(639.0760498,50.53939941)
\curveto(638.84604853,50.68939886)(638.64604873,50.88439866)(638.4760498,51.12439941)
\curveto(638.38604899,51.25439829)(638.31104907,51.38939816)(638.2510498,51.52939941)
\curveto(638.19104919,51.66939788)(638.13604924,51.82439772)(638.0860498,51.99439941)
\curveto(638.06604931,52.05439749)(638.05604932,52.12439742)(638.0560498,52.20439941)
\curveto(638.06604931,52.29439725)(638.0810493,52.36439718)(638.1010498,52.41439941)
\curveto(638.13104925,52.45439709)(638.1810492,52.49439705)(638.2510498,52.53439941)
\curveto(638.30104908,52.55439699)(638.37104901,52.56439698)(638.4610498,52.56439941)
\curveto(638.55104883,52.57439697)(638.64104874,52.57439697)(638.7310498,52.56439941)
\curveto(638.82104856,52.55439699)(638.90604847,52.53939701)(638.9860498,52.51939941)
\curveto(639.0760483,52.50939704)(639.13604824,52.49439705)(639.1660498,52.47439941)
\curveto(639.23604814,52.42439712)(639.2810481,52.3493972)(639.3010498,52.24939941)
\curveto(639.33104805,52.15939739)(639.36604801,52.07439747)(639.4060498,51.99439941)
\curveto(639.50604787,51.77439777)(639.64104774,51.60439794)(639.8110498,51.48439941)
\curveto(639.93104745,51.39439815)(640.06604731,51.32439822)(640.2160498,51.27439941)
\curveto(640.36604701,51.22439832)(640.52604685,51.17439837)(640.6960498,51.12439941)
\lineto(641.0110498,51.07939941)
\lineto(641.1010498,51.07939941)
\curveto(641.17104621,51.05939849)(641.26104612,51.0493985)(641.3710498,51.04939941)
\curveto(641.49104589,51.0493985)(641.59104579,51.05939849)(641.6710498,51.07939941)
\curveto(641.74104564,51.07939847)(641.79604558,51.08439846)(641.8360498,51.09439941)
\curveto(641.89604548,51.10439844)(641.95604542,51.10939844)(642.0160498,51.10939941)
\curveto(642.0760453,51.11939843)(642.13104525,51.12939842)(642.1810498,51.13939941)
\curveto(642.47104491,51.21939833)(642.70104468,51.32439822)(642.8710498,51.45439941)
\curveto(643.04104434,51.58439796)(643.16104422,51.80439774)(643.2310498,52.11439941)
\curveto(643.25104413,52.16439738)(643.25604412,52.21939733)(643.2460498,52.27939941)
\curveto(643.23604414,52.33939721)(643.22604415,52.38439716)(643.2160498,52.41439941)
\curveto(643.16604421,52.60439694)(643.09604428,52.7443968)(643.0060498,52.83439941)
\curveto(642.91604446,52.93439661)(642.80104458,53.02439652)(642.6610498,53.10439941)
\curveto(642.57104481,53.16439638)(642.47104491,53.21439633)(642.3610498,53.25439941)
\lineto(642.0310498,53.37439941)
\curveto(642.00104538,53.38439616)(641.97104541,53.38939616)(641.9410498,53.38939941)
\curveto(641.92104546,53.38939616)(641.89604548,53.39939615)(641.8660498,53.41939941)
\curveto(641.52604585,53.52939602)(641.17104621,53.60939594)(640.8010498,53.65939941)
\curveto(640.44104694,53.71939583)(640.10104728,53.81439573)(639.7810498,53.94439941)
\curveto(639.6810477,53.98439556)(639.58604779,54.01939553)(639.4960498,54.04939941)
\curveto(639.40604797,54.07939547)(639.32104806,54.11939543)(639.2410498,54.16939941)
\curveto(639.05104833,54.27939527)(638.8760485,54.40439514)(638.7160498,54.54439941)
\curveto(638.55604882,54.68439486)(638.43104895,54.85939469)(638.3410498,55.06939941)
\curveto(638.31104907,55.13939441)(638.28604909,55.20939434)(638.2660498,55.27939941)
\curveto(638.25604912,55.3493942)(638.24104914,55.42439412)(638.2210498,55.50439941)
\curveto(638.19104919,55.62439392)(638.1810492,55.75939379)(638.1910498,55.90939941)
\curveto(638.20104918,56.06939348)(638.21604916,56.20439334)(638.2360498,56.31439941)
\curveto(638.25604912,56.36439318)(638.26604911,56.40439314)(638.2660498,56.43439941)
\curveto(638.2760491,56.47439307)(638.29104909,56.51439303)(638.3110498,56.55439941)
\curveto(638.40104898,56.78439276)(638.52104886,56.98439256)(638.6710498,57.15439941)
\curveto(638.83104855,57.32439222)(639.01104837,57.47439207)(639.2110498,57.60439941)
\curveto(639.36104802,57.69439185)(639.52604785,57.76439178)(639.7060498,57.81439941)
\curveto(639.88604749,57.87439167)(640.0760473,57.92939162)(640.2760498,57.97939941)
\curveto(640.34604703,57.98939156)(640.41104697,57.99939155)(640.4710498,58.00939941)
\curveto(640.54104684,58.01939153)(640.61604676,58.02939152)(640.6960498,58.03939941)
\curveto(640.72604665,58.0493915)(640.76604661,58.0493915)(640.8160498,58.03939941)
\curveto(640.86604651,58.02939152)(640.90104648,58.03439151)(640.9210498,58.05439941)
}
}
{
\newrgbcolor{curcolor}{0 0 0}
\pscustom[linestyle=none,fillstyle=solid,fillcolor=curcolor]
{
\newpath
\moveto(652.8760498,50.70439941)
\curveto(652.90604197,50.544399)(652.89104199,50.40939914)(652.8310498,50.29939941)
\curveto(652.77104211,50.19939935)(652.69104219,50.12439942)(652.5910498,50.07439941)
\curveto(652.54104234,50.05439949)(652.48604239,50.0443995)(652.4260498,50.04439941)
\curveto(652.3760425,50.0443995)(652.32104256,50.03439951)(652.2610498,50.01439941)
\curveto(652.04104284,49.96439958)(651.82104306,49.97939957)(651.6010498,50.05939941)
\curveto(651.39104349,50.12939942)(651.24604363,50.21939933)(651.1660498,50.32939941)
\curveto(651.11604376,50.39939915)(651.07104381,50.47939907)(651.0310498,50.56939941)
\curveto(650.99104389,50.66939888)(650.94104394,50.7493988)(650.8810498,50.80939941)
\curveto(650.86104402,50.82939872)(650.83604404,50.8493987)(650.8060498,50.86939941)
\curveto(650.78604409,50.88939866)(650.75604412,50.89439865)(650.7160498,50.88439941)
\curveto(650.60604427,50.85439869)(650.50104438,50.79939875)(650.4010498,50.71939941)
\curveto(650.31104457,50.63939891)(650.22104466,50.56939898)(650.1310498,50.50939941)
\curveto(650.00104488,50.42939912)(649.86104502,50.35439919)(649.7110498,50.28439941)
\curveto(649.56104532,50.22439932)(649.40104548,50.16939938)(649.2310498,50.11939941)
\curveto(649.13104575,50.08939946)(649.02104586,50.06939948)(648.9010498,50.05939941)
\curveto(648.79104609,50.0493995)(648.6810462,50.03439951)(648.5710498,50.01439941)
\curveto(648.52104636,50.00439954)(648.4760464,49.99939955)(648.4360498,49.99939941)
\lineto(648.3310498,49.99939941)
\curveto(648.22104666,49.97939957)(648.11604676,49.97939957)(648.0160498,49.99939941)
\lineto(647.8810498,49.99939941)
\curveto(647.83104705,50.00939954)(647.7810471,50.01439953)(647.7310498,50.01439941)
\curveto(647.6810472,50.01439953)(647.63604724,50.02439952)(647.5960498,50.04439941)
\curveto(647.55604732,50.05439949)(647.52104736,50.05939949)(647.4910498,50.05939941)
\curveto(647.47104741,50.0493995)(647.44604743,50.0493995)(647.4160498,50.05939941)
\lineto(647.1760498,50.11939941)
\curveto(647.09604778,50.12939942)(647.02104786,50.1493994)(646.9510498,50.17939941)
\curveto(646.65104823,50.30939924)(646.40604847,50.45439909)(646.2160498,50.61439941)
\curveto(646.03604884,50.78439876)(645.88604899,51.01939853)(645.7660498,51.31939941)
\curveto(645.6760492,51.53939801)(645.63104925,51.80439774)(645.6310498,52.11439941)
\lineto(645.6310498,52.42939941)
\curveto(645.64104924,52.47939707)(645.64604923,52.52939702)(645.6460498,52.57939941)
\lineto(645.6760498,52.75939941)
\lineto(645.7960498,53.08939941)
\curveto(645.83604904,53.19939635)(645.88604899,53.29939625)(645.9460498,53.38939941)
\curveto(646.12604875,53.67939587)(646.37104851,53.89439565)(646.6810498,54.03439941)
\curveto(646.99104789,54.17439537)(647.33104755,54.29939525)(647.7010498,54.40939941)
\curveto(647.84104704,54.4493951)(647.98604689,54.47939507)(648.1360498,54.49939941)
\curveto(648.28604659,54.51939503)(648.43604644,54.544395)(648.5860498,54.57439941)
\curveto(648.65604622,54.59439495)(648.72104616,54.60439494)(648.7810498,54.60439941)
\curveto(648.85104603,54.60439494)(648.92604595,54.61439493)(649.0060498,54.63439941)
\curveto(649.0760458,54.65439489)(649.14604573,54.66439488)(649.2160498,54.66439941)
\curveto(649.28604559,54.67439487)(649.36104552,54.68939486)(649.4410498,54.70939941)
\curveto(649.69104519,54.76939478)(649.92604495,54.81939473)(650.1460498,54.85939941)
\curveto(650.36604451,54.90939464)(650.54104434,55.02439452)(650.6710498,55.20439941)
\curveto(650.73104415,55.28439426)(650.7810441,55.38439416)(650.8210498,55.50439941)
\curveto(650.86104402,55.63439391)(650.86104402,55.77439377)(650.8210498,55.92439941)
\curveto(650.76104412,56.16439338)(650.67104421,56.35439319)(650.5510498,56.49439941)
\curveto(650.44104444,56.63439291)(650.2810446,56.7443928)(650.0710498,56.82439941)
\curveto(649.95104493,56.87439267)(649.80604507,56.90939264)(649.6360498,56.92939941)
\curveto(649.4760454,56.9493926)(649.30604557,56.95939259)(649.1260498,56.95939941)
\curveto(648.94604593,56.95939259)(648.77104611,56.9493926)(648.6010498,56.92939941)
\curveto(648.43104645,56.90939264)(648.28604659,56.87939267)(648.1660498,56.83939941)
\curveto(647.99604688,56.77939277)(647.83104705,56.69439285)(647.6710498,56.58439941)
\curveto(647.59104729,56.52439302)(647.51604736,56.4443931)(647.4460498,56.34439941)
\curveto(647.38604749,56.25439329)(647.33104755,56.15439339)(647.2810498,56.04439941)
\curveto(647.25104763,55.96439358)(647.22104766,55.87939367)(647.1910498,55.78939941)
\curveto(647.17104771,55.69939385)(647.12604775,55.62939392)(647.0560498,55.57939941)
\curveto(647.01604786,55.549394)(646.94604793,55.52439402)(646.8460498,55.50439941)
\curveto(646.75604812,55.49439405)(646.66104822,55.48939406)(646.5610498,55.48939941)
\curveto(646.46104842,55.48939406)(646.36104852,55.49439405)(646.2610498,55.50439941)
\curveto(646.17104871,55.52439402)(646.10604877,55.549394)(646.0660498,55.57939941)
\curveto(646.02604885,55.60939394)(645.99604888,55.65939389)(645.9760498,55.72939941)
\curveto(645.95604892,55.79939375)(645.95604892,55.87439367)(645.9760498,55.95439941)
\curveto(646.00604887,56.08439346)(646.03604884,56.20439334)(646.0660498,56.31439941)
\curveto(646.10604877,56.43439311)(646.15104873,56.549393)(646.2010498,56.65939941)
\curveto(646.39104849,57.00939254)(646.63104825,57.27939227)(646.9210498,57.46939941)
\curveto(647.21104767,57.66939188)(647.57104731,57.82939172)(648.0010498,57.94939941)
\curveto(648.10104678,57.96939158)(648.20104668,57.98439156)(648.3010498,57.99439941)
\curveto(648.41104647,58.00439154)(648.52104636,58.01939153)(648.6310498,58.03939941)
\curveto(648.67104621,58.0493915)(648.73604614,58.0493915)(648.8260498,58.03939941)
\curveto(648.91604596,58.03939151)(648.97104591,58.0493915)(648.9910498,58.06939941)
\curveto(649.69104519,58.07939147)(650.30104458,57.99939155)(650.8210498,57.82939941)
\curveto(651.34104354,57.65939189)(651.70604317,57.33439221)(651.9160498,56.85439941)
\curveto(652.00604287,56.65439289)(652.05604282,56.41939313)(652.0660498,56.14939941)
\curveto(652.08604279,55.88939366)(652.09604278,55.61439393)(652.0960498,55.32439941)
\lineto(652.0960498,52.00939941)
\curveto(652.09604278,51.86939768)(652.10104278,51.73439781)(652.1110498,51.60439941)
\curveto(652.12104276,51.47439807)(652.15104273,51.36939818)(652.2010498,51.28939941)
\curveto(652.25104263,51.21939833)(652.31604256,51.16939838)(652.3960498,51.13939941)
\curveto(652.48604239,51.09939845)(652.57104231,51.06939848)(652.6510498,51.04939941)
\curveto(652.73104215,51.03939851)(652.79104209,50.99439855)(652.8310498,50.91439941)
\curveto(652.85104203,50.88439866)(652.86104202,50.85439869)(652.8610498,50.82439941)
\curveto(652.86104202,50.79439875)(652.86604201,50.75439879)(652.8760498,50.70439941)
\moveto(650.7310498,52.36939941)
\curveto(650.79104409,52.50939704)(650.82104406,52.66939688)(650.8210498,52.84939941)
\curveto(650.83104405,53.03939651)(650.83604404,53.23439631)(650.8360498,53.43439941)
\curveto(650.83604404,53.544396)(650.83104405,53.6443959)(650.8210498,53.73439941)
\curveto(650.81104407,53.82439572)(650.77104411,53.89439565)(650.7010498,53.94439941)
\curveto(650.67104421,53.96439558)(650.60104428,53.97439557)(650.4910498,53.97439941)
\curveto(650.47104441,53.95439559)(650.43604444,53.9443956)(650.3860498,53.94439941)
\curveto(650.33604454,53.9443956)(650.29104459,53.93439561)(650.2510498,53.91439941)
\curveto(650.17104471,53.89439565)(650.0810448,53.87439567)(649.9810498,53.85439941)
\lineto(649.6810498,53.79439941)
\curveto(649.65104523,53.79439575)(649.61604526,53.78939576)(649.5760498,53.77939941)
\lineto(649.4710498,53.77939941)
\curveto(649.32104556,53.73939581)(649.15604572,53.71439583)(648.9760498,53.70439941)
\curveto(648.80604607,53.70439584)(648.64604623,53.68439586)(648.4960498,53.64439941)
\curveto(648.41604646,53.62439592)(648.34104654,53.60439594)(648.2710498,53.58439941)
\curveto(648.21104667,53.57439597)(648.14104674,53.55939599)(648.0610498,53.53939941)
\curveto(647.90104698,53.48939606)(647.75104713,53.42439612)(647.6110498,53.34439941)
\curveto(647.47104741,53.27439627)(647.35104753,53.18439636)(647.2510498,53.07439941)
\curveto(647.15104773,52.96439658)(647.0760478,52.82939672)(647.0260498,52.66939941)
\curveto(646.9760479,52.51939703)(646.95604792,52.33439721)(646.9660498,52.11439941)
\curveto(646.96604791,52.01439753)(646.9810479,51.91939763)(647.0110498,51.82939941)
\curveto(647.05104783,51.7493978)(647.09604778,51.67439787)(647.1460498,51.60439941)
\curveto(647.22604765,51.49439805)(647.33104755,51.39939815)(647.4610498,51.31939941)
\curveto(647.59104729,51.2493983)(647.73104715,51.18939836)(647.8810498,51.13939941)
\curveto(647.93104695,51.12939842)(647.9810469,51.12439842)(648.0310498,51.12439941)
\curveto(648.0810468,51.12439842)(648.13104675,51.11939843)(648.1810498,51.10939941)
\curveto(648.25104663,51.08939846)(648.33604654,51.07439847)(648.4360498,51.06439941)
\curveto(648.54604633,51.06439848)(648.63604624,51.07439847)(648.7060498,51.09439941)
\curveto(648.76604611,51.11439843)(648.82604605,51.11939843)(648.8860498,51.10939941)
\curveto(648.94604593,51.10939844)(649.00604587,51.11939843)(649.0660498,51.13939941)
\curveto(649.14604573,51.15939839)(649.22104566,51.17439837)(649.2910498,51.18439941)
\curveto(649.37104551,51.19439835)(649.44604543,51.21439833)(649.5160498,51.24439941)
\curveto(649.80604507,51.36439818)(650.05104483,51.50939804)(650.2510498,51.67939941)
\curveto(650.46104442,51.8493977)(650.62104426,52.07939747)(650.7310498,52.36939941)
}
}
{
\newrgbcolor{curcolor}{0 0 0}
\pscustom[linestyle=none,fillstyle=solid,fillcolor=curcolor]
{
\newpath
\moveto(657.69269043,58.05439941)
\curveto(657.92268564,58.05439149)(658.05268551,57.99439155)(658.08269043,57.87439941)
\curveto(658.11268545,57.76439178)(658.12768543,57.59939195)(658.12769043,57.37939941)
\lineto(658.12769043,57.09439941)
\curveto(658.12768543,57.00439254)(658.10268546,56.92939262)(658.05269043,56.86939941)
\curveto(657.99268557,56.78939276)(657.90768565,56.7443928)(657.79769043,56.73439941)
\curveto(657.68768587,56.73439281)(657.57768598,56.71939283)(657.46769043,56.68939941)
\curveto(657.32768623,56.65939289)(657.19268637,56.62939292)(657.06269043,56.59939941)
\curveto(656.94268662,56.56939298)(656.82768673,56.52939302)(656.71769043,56.47939941)
\curveto(656.42768713,56.3493932)(656.19268737,56.16939338)(656.01269043,55.93939941)
\curveto(655.83268773,55.71939383)(655.67768788,55.46439408)(655.54769043,55.17439941)
\curveto(655.50768805,55.06439448)(655.47768808,54.9493946)(655.45769043,54.82939941)
\curveto(655.43768812,54.71939483)(655.41268815,54.60439494)(655.38269043,54.48439941)
\curveto(655.37268819,54.43439511)(655.36768819,54.38439516)(655.36769043,54.33439941)
\curveto(655.37768818,54.28439526)(655.37768818,54.23439531)(655.36769043,54.18439941)
\curveto(655.33768822,54.06439548)(655.32268824,53.92439562)(655.32269043,53.76439941)
\curveto(655.33268823,53.61439593)(655.33768822,53.46939608)(655.33769043,53.32939941)
\lineto(655.33769043,51.48439941)
\lineto(655.33769043,51.13939941)
\curveto(655.33768822,51.01939853)(655.33268823,50.90439864)(655.32269043,50.79439941)
\curveto(655.31268825,50.68439886)(655.30768825,50.58939896)(655.30769043,50.50939941)
\curveto(655.31768824,50.42939912)(655.29768826,50.35939919)(655.24769043,50.29939941)
\curveto(655.19768836,50.22939932)(655.11768844,50.18939936)(655.00769043,50.17939941)
\curveto(654.90768865,50.16939938)(654.79768876,50.16439938)(654.67769043,50.16439941)
\lineto(654.40769043,50.16439941)
\curveto(654.3576892,50.18439936)(654.30768925,50.19939935)(654.25769043,50.20939941)
\curveto(654.21768934,50.22939932)(654.18768937,50.25439929)(654.16769043,50.28439941)
\curveto(654.11768944,50.35439919)(654.08768947,50.43939911)(654.07769043,50.53939941)
\lineto(654.07769043,50.86939941)
\lineto(654.07769043,52.02439941)
\lineto(654.07769043,56.17939941)
\lineto(654.07769043,57.21439941)
\lineto(654.07769043,57.51439941)
\curveto(654.08768947,57.61439193)(654.11768944,57.69939185)(654.16769043,57.76939941)
\curveto(654.19768936,57.80939174)(654.24768931,57.83939171)(654.31769043,57.85939941)
\curveto(654.39768916,57.87939167)(654.48268908,57.88939166)(654.57269043,57.88939941)
\curveto(654.6626889,57.89939165)(654.75268881,57.89939165)(654.84269043,57.88939941)
\curveto(654.93268863,57.87939167)(655.00268856,57.86439168)(655.05269043,57.84439941)
\curveto(655.13268843,57.81439173)(655.18268838,57.75439179)(655.20269043,57.66439941)
\curveto(655.23268833,57.58439196)(655.24768831,57.49439205)(655.24769043,57.39439941)
\lineto(655.24769043,57.09439941)
\curveto(655.24768831,56.99439255)(655.26768829,56.90439264)(655.30769043,56.82439941)
\curveto(655.31768824,56.80439274)(655.32768823,56.78939276)(655.33769043,56.77939941)
\lineto(655.38269043,56.73439941)
\curveto(655.49268807,56.73439281)(655.58268798,56.77939277)(655.65269043,56.86939941)
\curveto(655.72268784,56.96939258)(655.78268778,57.0493925)(655.83269043,57.10939941)
\lineto(655.92269043,57.19939941)
\curveto(656.01268755,57.30939224)(656.13768742,57.42439212)(656.29769043,57.54439941)
\curveto(656.4576871,57.66439188)(656.60768695,57.75439179)(656.74769043,57.81439941)
\curveto(656.83768672,57.86439168)(656.93268663,57.89939165)(657.03269043,57.91939941)
\curveto(657.13268643,57.9493916)(657.23768632,57.97939157)(657.34769043,58.00939941)
\curveto(657.40768615,58.01939153)(657.46768609,58.02439152)(657.52769043,58.02439941)
\curveto(657.58768597,58.03439151)(657.64268592,58.0443915)(657.69269043,58.05439941)
}
}
{
\newrgbcolor{curcolor}{0 0 0}
\pscustom[linestyle=none,fillstyle=solid,fillcolor=curcolor]
{
\newpath
\moveto(662.70245605,58.05439941)
\curveto(662.93245126,58.05439149)(663.06245113,57.99439155)(663.09245605,57.87439941)
\curveto(663.12245107,57.76439178)(663.13745106,57.59939195)(663.13745605,57.37939941)
\lineto(663.13745605,57.09439941)
\curveto(663.13745106,57.00439254)(663.11245108,56.92939262)(663.06245605,56.86939941)
\curveto(663.00245119,56.78939276)(662.91745128,56.7443928)(662.80745605,56.73439941)
\curveto(662.6974515,56.73439281)(662.58745161,56.71939283)(662.47745605,56.68939941)
\curveto(662.33745186,56.65939289)(662.20245199,56.62939292)(662.07245605,56.59939941)
\curveto(661.95245224,56.56939298)(661.83745236,56.52939302)(661.72745605,56.47939941)
\curveto(661.43745276,56.3493932)(661.20245299,56.16939338)(661.02245605,55.93939941)
\curveto(660.84245335,55.71939383)(660.68745351,55.46439408)(660.55745605,55.17439941)
\curveto(660.51745368,55.06439448)(660.48745371,54.9493946)(660.46745605,54.82939941)
\curveto(660.44745375,54.71939483)(660.42245377,54.60439494)(660.39245605,54.48439941)
\curveto(660.38245381,54.43439511)(660.37745382,54.38439516)(660.37745605,54.33439941)
\curveto(660.38745381,54.28439526)(660.38745381,54.23439531)(660.37745605,54.18439941)
\curveto(660.34745385,54.06439548)(660.33245386,53.92439562)(660.33245605,53.76439941)
\curveto(660.34245385,53.61439593)(660.34745385,53.46939608)(660.34745605,53.32939941)
\lineto(660.34745605,51.48439941)
\lineto(660.34745605,51.13939941)
\curveto(660.34745385,51.01939853)(660.34245385,50.90439864)(660.33245605,50.79439941)
\curveto(660.32245387,50.68439886)(660.31745388,50.58939896)(660.31745605,50.50939941)
\curveto(660.32745387,50.42939912)(660.30745389,50.35939919)(660.25745605,50.29939941)
\curveto(660.20745399,50.22939932)(660.12745407,50.18939936)(660.01745605,50.17939941)
\curveto(659.91745428,50.16939938)(659.80745439,50.16439938)(659.68745605,50.16439941)
\lineto(659.41745605,50.16439941)
\curveto(659.36745483,50.18439936)(659.31745488,50.19939935)(659.26745605,50.20939941)
\curveto(659.22745497,50.22939932)(659.197455,50.25439929)(659.17745605,50.28439941)
\curveto(659.12745507,50.35439919)(659.0974551,50.43939911)(659.08745605,50.53939941)
\lineto(659.08745605,50.86939941)
\lineto(659.08745605,52.02439941)
\lineto(659.08745605,56.17939941)
\lineto(659.08745605,57.21439941)
\lineto(659.08745605,57.51439941)
\curveto(659.0974551,57.61439193)(659.12745507,57.69939185)(659.17745605,57.76939941)
\curveto(659.20745499,57.80939174)(659.25745494,57.83939171)(659.32745605,57.85939941)
\curveto(659.40745479,57.87939167)(659.4924547,57.88939166)(659.58245605,57.88939941)
\curveto(659.67245452,57.89939165)(659.76245443,57.89939165)(659.85245605,57.88939941)
\curveto(659.94245425,57.87939167)(660.01245418,57.86439168)(660.06245605,57.84439941)
\curveto(660.14245405,57.81439173)(660.192454,57.75439179)(660.21245605,57.66439941)
\curveto(660.24245395,57.58439196)(660.25745394,57.49439205)(660.25745605,57.39439941)
\lineto(660.25745605,57.09439941)
\curveto(660.25745394,56.99439255)(660.27745392,56.90439264)(660.31745605,56.82439941)
\curveto(660.32745387,56.80439274)(660.33745386,56.78939276)(660.34745605,56.77939941)
\lineto(660.39245605,56.73439941)
\curveto(660.50245369,56.73439281)(660.5924536,56.77939277)(660.66245605,56.86939941)
\curveto(660.73245346,56.96939258)(660.7924534,57.0493925)(660.84245605,57.10939941)
\lineto(660.93245605,57.19939941)
\curveto(661.02245317,57.30939224)(661.14745305,57.42439212)(661.30745605,57.54439941)
\curveto(661.46745273,57.66439188)(661.61745258,57.75439179)(661.75745605,57.81439941)
\curveto(661.84745235,57.86439168)(661.94245225,57.89939165)(662.04245605,57.91939941)
\curveto(662.14245205,57.9493916)(662.24745195,57.97939157)(662.35745605,58.00939941)
\curveto(662.41745178,58.01939153)(662.47745172,58.02439152)(662.53745605,58.02439941)
\curveto(662.5974516,58.03439151)(662.65245154,58.0443915)(662.70245605,58.05439941)
}
}
{
\newrgbcolor{curcolor}{0 0 0}
\pscustom[linestyle=none,fillstyle=solid,fillcolor=curcolor]
{
\newpath
\moveto(671.19222168,54.34939941)
\curveto(671.21221362,54.28939526)(671.22221361,54.19439535)(671.22222168,54.06439941)
\curveto(671.22221361,53.9443956)(671.21721361,53.85939569)(671.20722168,53.80939941)
\lineto(671.20722168,53.65939941)
\curveto(671.19721363,53.57939597)(671.18721364,53.50439604)(671.17722168,53.43439941)
\curveto(671.17721365,53.37439617)(671.17221366,53.30439624)(671.16222168,53.22439941)
\curveto(671.14221369,53.16439638)(671.1272137,53.10439644)(671.11722168,53.04439941)
\curveto(671.11721371,52.98439656)(671.10721372,52.92439662)(671.08722168,52.86439941)
\curveto(671.04721378,52.73439681)(671.01221382,52.60439694)(670.98222168,52.47439941)
\curveto(670.95221388,52.3443972)(670.91221392,52.22439732)(670.86222168,52.11439941)
\curveto(670.65221418,51.63439791)(670.37221446,51.22939832)(670.02222168,50.89939941)
\curveto(669.67221516,50.57939897)(669.24221559,50.33439921)(668.73222168,50.16439941)
\curveto(668.62221621,50.12439942)(668.50221633,50.09439945)(668.37222168,50.07439941)
\curveto(668.25221658,50.05439949)(668.1272167,50.03439951)(667.99722168,50.01439941)
\curveto(667.93721689,50.00439954)(667.87221696,49.99939955)(667.80222168,49.99939941)
\curveto(667.74221709,49.98939956)(667.68221715,49.98439956)(667.62222168,49.98439941)
\curveto(667.58221725,49.97439957)(667.52221731,49.96939958)(667.44222168,49.96939941)
\curveto(667.37221746,49.96939958)(667.32221751,49.97439957)(667.29222168,49.98439941)
\curveto(667.25221758,49.99439955)(667.21221762,49.99939955)(667.17222168,49.99939941)
\curveto(667.1322177,49.98939956)(667.09721773,49.98939956)(667.06722168,49.99939941)
\lineto(666.97722168,49.99939941)
\lineto(666.61722168,50.04439941)
\curveto(666.47721835,50.08439946)(666.34221849,50.12439942)(666.21222168,50.16439941)
\curveto(666.08221875,50.20439934)(665.95721887,50.2493993)(665.83722168,50.29939941)
\curveto(665.38721944,50.49939905)(665.01721981,50.75939879)(664.72722168,51.07939941)
\curveto(664.43722039,51.39939815)(664.19722063,51.78939776)(664.00722168,52.24939941)
\curveto(663.95722087,52.3493972)(663.91722091,52.4493971)(663.88722168,52.54939941)
\curveto(663.86722096,52.6493969)(663.84722098,52.75439679)(663.82722168,52.86439941)
\curveto(663.80722102,52.90439664)(663.79722103,52.93439661)(663.79722168,52.95439941)
\curveto(663.80722102,52.98439656)(663.80722102,53.01939653)(663.79722168,53.05939941)
\curveto(663.77722105,53.13939641)(663.76222107,53.21939633)(663.75222168,53.29939941)
\curveto(663.75222108,53.38939616)(663.74222109,53.47439607)(663.72222168,53.55439941)
\lineto(663.72222168,53.67439941)
\curveto(663.72222111,53.71439583)(663.71722111,53.75939579)(663.70722168,53.80939941)
\curveto(663.69722113,53.85939569)(663.69222114,53.9443956)(663.69222168,54.06439941)
\curveto(663.69222114,54.19439535)(663.70222113,54.28939526)(663.72222168,54.34939941)
\curveto(663.74222109,54.41939513)(663.74722108,54.48939506)(663.73722168,54.55939941)
\curveto(663.7272211,54.62939492)(663.7322211,54.69939485)(663.75222168,54.76939941)
\curveto(663.76222107,54.81939473)(663.76722106,54.85939469)(663.76722168,54.88939941)
\curveto(663.77722105,54.92939462)(663.78722104,54.97439457)(663.79722168,55.02439941)
\curveto(663.827221,55.1443944)(663.85222098,55.26439428)(663.87222168,55.38439941)
\curveto(663.90222093,55.50439404)(663.94222089,55.61939393)(663.99222168,55.72939941)
\curveto(664.14222069,56.09939345)(664.32222051,56.42939312)(664.53222168,56.71939941)
\curveto(664.75222008,57.01939253)(665.01721981,57.26939228)(665.32722168,57.46939941)
\curveto(665.44721938,57.549392)(665.57221926,57.61439193)(665.70222168,57.66439941)
\curveto(665.832219,57.72439182)(665.96721886,57.78439176)(666.10722168,57.84439941)
\curveto(666.2272186,57.89439165)(666.35721847,57.92439162)(666.49722168,57.93439941)
\curveto(666.63721819,57.95439159)(666.77721805,57.98439156)(666.91722168,58.02439941)
\lineto(667.11222168,58.02439941)
\curveto(667.18221765,58.03439151)(667.24721758,58.0443915)(667.30722168,58.05439941)
\curveto(668.19721663,58.06439148)(668.93721589,57.87939167)(669.52722168,57.49939941)
\curveto(670.11721471,57.11939243)(670.54221429,56.62439292)(670.80222168,56.01439941)
\curveto(670.85221398,55.91439363)(670.89221394,55.81439373)(670.92222168,55.71439941)
\curveto(670.95221388,55.61439393)(670.98721384,55.50939404)(671.02722168,55.39939941)
\curveto(671.05721377,55.28939426)(671.08221375,55.16939438)(671.10222168,55.03939941)
\curveto(671.12221371,54.91939463)(671.14721368,54.79439475)(671.17722168,54.66439941)
\curveto(671.18721364,54.61439493)(671.18721364,54.55939499)(671.17722168,54.49939941)
\curveto(671.17721365,54.4493951)(671.18221365,54.39939515)(671.19222168,54.34939941)
\moveto(669.85722168,53.49439941)
\curveto(669.87721495,53.56439598)(669.88221495,53.6443959)(669.87222168,53.73439941)
\lineto(669.87222168,53.98939941)
\curveto(669.87221496,54.37939517)(669.83721499,54.70939484)(669.76722168,54.97939941)
\curveto(669.73721509,55.05939449)(669.71221512,55.13939441)(669.69222168,55.21939941)
\curveto(669.67221516,55.29939425)(669.64721518,55.37439417)(669.61722168,55.44439941)
\curveto(669.33721549,56.09439345)(668.89221594,56.544393)(668.28222168,56.79439941)
\curveto(668.21221662,56.82439272)(668.13721669,56.8443927)(668.05722168,56.85439941)
\lineto(667.81722168,56.91439941)
\curveto(667.73721709,56.93439261)(667.65221718,56.9443926)(667.56222168,56.94439941)
\lineto(667.29222168,56.94439941)
\lineto(667.02222168,56.89939941)
\curveto(666.92221791,56.87939267)(666.827218,56.85439269)(666.73722168,56.82439941)
\curveto(666.65721817,56.80439274)(666.57721825,56.77439277)(666.49722168,56.73439941)
\curveto(666.4272184,56.71439283)(666.36221847,56.68439286)(666.30222168,56.64439941)
\curveto(666.24221859,56.60439294)(666.18721864,56.56439298)(666.13722168,56.52439941)
\curveto(665.89721893,56.35439319)(665.70221913,56.1493934)(665.55222168,55.90939941)
\curveto(665.40221943,55.66939388)(665.27221956,55.38939416)(665.16222168,55.06939941)
\curveto(665.1322197,54.96939458)(665.11221972,54.86439468)(665.10222168,54.75439941)
\curveto(665.09221974,54.65439489)(665.07721975,54.549395)(665.05722168,54.43939941)
\curveto(665.04721978,54.39939515)(665.04221979,54.33439521)(665.04222168,54.24439941)
\curveto(665.0322198,54.21439533)(665.0272198,54.17939537)(665.02722168,54.13939941)
\curveto(665.03721979,54.09939545)(665.04221979,54.05439549)(665.04222168,54.00439941)
\lineto(665.04222168,53.70439941)
\curveto(665.04221979,53.60439594)(665.05221978,53.51439603)(665.07222168,53.43439941)
\lineto(665.10222168,53.25439941)
\curveto(665.12221971,53.15439639)(665.13721969,53.05439649)(665.14722168,52.95439941)
\curveto(665.16721966,52.86439668)(665.19721963,52.77939677)(665.23722168,52.69939941)
\curveto(665.33721949,52.45939709)(665.45221938,52.23439731)(665.58222168,52.02439941)
\curveto(665.72221911,51.81439773)(665.89221894,51.63939791)(666.09222168,51.49939941)
\curveto(666.14221869,51.46939808)(666.18721864,51.4443981)(666.22722168,51.42439941)
\curveto(666.26721856,51.40439814)(666.31221852,51.37939817)(666.36222168,51.34939941)
\curveto(666.44221839,51.29939825)(666.5272183,51.25439829)(666.61722168,51.21439941)
\curveto(666.71721811,51.18439836)(666.82221801,51.15439839)(666.93222168,51.12439941)
\curveto(666.98221785,51.10439844)(667.0272178,51.09439845)(667.06722168,51.09439941)
\curveto(667.11721771,51.10439844)(667.16721766,51.10439844)(667.21722168,51.09439941)
\curveto(667.24721758,51.08439846)(667.30721752,51.07439847)(667.39722168,51.06439941)
\curveto(667.49721733,51.05439849)(667.57221726,51.05939849)(667.62222168,51.07939941)
\curveto(667.66221717,51.08939846)(667.70221713,51.08939846)(667.74222168,51.07939941)
\curveto(667.78221705,51.07939847)(667.82221701,51.08939846)(667.86222168,51.10939941)
\curveto(667.94221689,51.12939842)(668.02221681,51.1443984)(668.10222168,51.15439941)
\curveto(668.18221665,51.17439837)(668.25721657,51.19939835)(668.32722168,51.22939941)
\curveto(668.66721616,51.36939818)(668.94221589,51.56439798)(669.15222168,51.81439941)
\curveto(669.36221547,52.06439748)(669.53721529,52.35939719)(669.67722168,52.69939941)
\curveto(669.7272151,52.81939673)(669.75721507,52.9443966)(669.76722168,53.07439941)
\curveto(669.78721504,53.21439633)(669.81721501,53.35439619)(669.85722168,53.49439941)
}
}
{
\newrgbcolor{curcolor}{0 0 0}
\pscustom[linestyle=none,fillstyle=solid,fillcolor=curcolor]
{
\newpath
\moveto(673.25050293,60.82939941)
\curveto(673.38050131,60.82938872)(673.51550118,60.82938872)(673.65550293,60.82939941)
\curveto(673.80550089,60.82938872)(673.91550078,60.79438875)(673.98550293,60.72439941)
\curveto(674.03550066,60.65438889)(674.06050063,60.55938899)(674.06050293,60.43939941)
\curveto(674.07050062,60.32938922)(674.07550062,60.21438933)(674.07550293,60.09439941)
\lineto(674.07550293,58.75939941)
\lineto(674.07550293,52.68439941)
\lineto(674.07550293,51.00439941)
\lineto(674.07550293,50.61439941)
\curveto(674.07550062,50.47439907)(674.05050064,50.36439918)(674.00050293,50.28439941)
\curveto(673.97050072,50.23439931)(673.92550077,50.20439934)(673.86550293,50.19439941)
\curveto(673.81550088,50.18439936)(673.75050094,50.16939938)(673.67050293,50.14939941)
\lineto(673.46050293,50.14939941)
\lineto(673.14550293,50.14939941)
\curveto(673.04550165,50.15939939)(672.97050172,50.19439935)(672.92050293,50.25439941)
\curveto(672.87050182,50.33439921)(672.84050185,50.43439911)(672.83050293,50.55439941)
\lineto(672.83050293,50.92939941)
\lineto(672.83050293,52.30939941)
\lineto(672.83050293,58.54939941)
\lineto(672.83050293,60.01939941)
\curveto(672.83050186,60.12938942)(672.82550187,60.2443893)(672.81550293,60.36439941)
\curveto(672.81550188,60.49438905)(672.84050185,60.59438895)(672.89050293,60.66439941)
\curveto(672.93050176,60.72438882)(673.00550169,60.77438877)(673.11550293,60.81439941)
\curveto(673.13550156,60.82438872)(673.15550154,60.82438872)(673.17550293,60.81439941)
\curveto(673.20550149,60.81438873)(673.23050146,60.81938873)(673.25050293,60.82939941)
}
}
{
\newrgbcolor{curcolor}{0 0 0}
\pscustom[linestyle=none,fillstyle=solid,fillcolor=curcolor]
{
\newpath
\moveto(676.59034668,60.82939941)
\curveto(676.72034506,60.82938872)(676.85534493,60.82938872)(676.99534668,60.82939941)
\curveto(677.14534464,60.82938872)(677.25534453,60.79438875)(677.32534668,60.72439941)
\curveto(677.37534441,60.65438889)(677.40034438,60.55938899)(677.40034668,60.43939941)
\curveto(677.41034437,60.32938922)(677.41534437,60.21438933)(677.41534668,60.09439941)
\lineto(677.41534668,58.75939941)
\lineto(677.41534668,52.68439941)
\lineto(677.41534668,51.00439941)
\lineto(677.41534668,50.61439941)
\curveto(677.41534437,50.47439907)(677.39034439,50.36439918)(677.34034668,50.28439941)
\curveto(677.31034447,50.23439931)(677.26534452,50.20439934)(677.20534668,50.19439941)
\curveto(677.15534463,50.18439936)(677.09034469,50.16939938)(677.01034668,50.14939941)
\lineto(676.80034668,50.14939941)
\lineto(676.48534668,50.14939941)
\curveto(676.3853454,50.15939939)(676.31034547,50.19439935)(676.26034668,50.25439941)
\curveto(676.21034557,50.33439921)(676.1803456,50.43439911)(676.17034668,50.55439941)
\lineto(676.17034668,50.92939941)
\lineto(676.17034668,52.30939941)
\lineto(676.17034668,58.54939941)
\lineto(676.17034668,60.01939941)
\curveto(676.17034561,60.12938942)(676.16534562,60.2443893)(676.15534668,60.36439941)
\curveto(676.15534563,60.49438905)(676.1803456,60.59438895)(676.23034668,60.66439941)
\curveto(676.27034551,60.72438882)(676.34534544,60.77438877)(676.45534668,60.81439941)
\curveto(676.47534531,60.82438872)(676.49534529,60.82438872)(676.51534668,60.81439941)
\curveto(676.54534524,60.81438873)(676.57034521,60.81938873)(676.59034668,60.82939941)
}
}
{
\newrgbcolor{curcolor}{0 0 0}
\pscustom[linestyle=none,fillstyle=solid,fillcolor=curcolor]
{
\newpath
\moveto(686.24519043,50.70439941)
\curveto(686.2751826,50.544399)(686.26018261,50.40939914)(686.20019043,50.29939941)
\curveto(686.14018273,50.19939935)(686.06018281,50.12439942)(685.96019043,50.07439941)
\curveto(685.91018296,50.05439949)(685.85518302,50.0443995)(685.79519043,50.04439941)
\curveto(685.74518313,50.0443995)(685.69018318,50.03439951)(685.63019043,50.01439941)
\curveto(685.41018346,49.96439958)(685.19018368,49.97939957)(684.97019043,50.05939941)
\curveto(684.76018411,50.12939942)(684.61518426,50.21939933)(684.53519043,50.32939941)
\curveto(684.48518439,50.39939915)(684.44018443,50.47939907)(684.40019043,50.56939941)
\curveto(684.36018451,50.66939888)(684.31018456,50.7493988)(684.25019043,50.80939941)
\curveto(684.23018464,50.82939872)(684.20518467,50.8493987)(684.17519043,50.86939941)
\curveto(684.15518472,50.88939866)(684.12518475,50.89439865)(684.08519043,50.88439941)
\curveto(683.9751849,50.85439869)(683.870185,50.79939875)(683.77019043,50.71939941)
\curveto(683.68018519,50.63939891)(683.59018528,50.56939898)(683.50019043,50.50939941)
\curveto(683.3701855,50.42939912)(683.23018564,50.35439919)(683.08019043,50.28439941)
\curveto(682.93018594,50.22439932)(682.7701861,50.16939938)(682.60019043,50.11939941)
\curveto(682.50018637,50.08939946)(682.39018648,50.06939948)(682.27019043,50.05939941)
\curveto(682.16018671,50.0493995)(682.05018682,50.03439951)(681.94019043,50.01439941)
\curveto(681.89018698,50.00439954)(681.84518703,49.99939955)(681.80519043,49.99939941)
\lineto(681.70019043,49.99939941)
\curveto(681.59018728,49.97939957)(681.48518739,49.97939957)(681.38519043,49.99939941)
\lineto(681.25019043,49.99939941)
\curveto(681.20018767,50.00939954)(681.15018772,50.01439953)(681.10019043,50.01439941)
\curveto(681.05018782,50.01439953)(681.00518787,50.02439952)(680.96519043,50.04439941)
\curveto(680.92518795,50.05439949)(680.89018798,50.05939949)(680.86019043,50.05939941)
\curveto(680.84018803,50.0493995)(680.81518806,50.0493995)(680.78519043,50.05939941)
\lineto(680.54519043,50.11939941)
\curveto(680.46518841,50.12939942)(680.39018848,50.1493994)(680.32019043,50.17939941)
\curveto(680.02018885,50.30939924)(679.7751891,50.45439909)(679.58519043,50.61439941)
\curveto(679.40518947,50.78439876)(679.25518962,51.01939853)(679.13519043,51.31939941)
\curveto(679.04518983,51.53939801)(679.00018987,51.80439774)(679.00019043,52.11439941)
\lineto(679.00019043,52.42939941)
\curveto(679.01018986,52.47939707)(679.01518986,52.52939702)(679.01519043,52.57939941)
\lineto(679.04519043,52.75939941)
\lineto(679.16519043,53.08939941)
\curveto(679.20518967,53.19939635)(679.25518962,53.29939625)(679.31519043,53.38939941)
\curveto(679.49518938,53.67939587)(679.74018913,53.89439565)(680.05019043,54.03439941)
\curveto(680.36018851,54.17439537)(680.70018817,54.29939525)(681.07019043,54.40939941)
\curveto(681.21018766,54.4493951)(681.35518752,54.47939507)(681.50519043,54.49939941)
\curveto(681.65518722,54.51939503)(681.80518707,54.544395)(681.95519043,54.57439941)
\curveto(682.02518685,54.59439495)(682.09018678,54.60439494)(682.15019043,54.60439941)
\curveto(682.22018665,54.60439494)(682.29518658,54.61439493)(682.37519043,54.63439941)
\curveto(682.44518643,54.65439489)(682.51518636,54.66439488)(682.58519043,54.66439941)
\curveto(682.65518622,54.67439487)(682.73018614,54.68939486)(682.81019043,54.70939941)
\curveto(683.06018581,54.76939478)(683.29518558,54.81939473)(683.51519043,54.85939941)
\curveto(683.73518514,54.90939464)(683.91018496,55.02439452)(684.04019043,55.20439941)
\curveto(684.10018477,55.28439426)(684.15018472,55.38439416)(684.19019043,55.50439941)
\curveto(684.23018464,55.63439391)(684.23018464,55.77439377)(684.19019043,55.92439941)
\curveto(684.13018474,56.16439338)(684.04018483,56.35439319)(683.92019043,56.49439941)
\curveto(683.81018506,56.63439291)(683.65018522,56.7443928)(683.44019043,56.82439941)
\curveto(683.32018555,56.87439267)(683.1751857,56.90939264)(683.00519043,56.92939941)
\curveto(682.84518603,56.9493926)(682.6751862,56.95939259)(682.49519043,56.95939941)
\curveto(682.31518656,56.95939259)(682.14018673,56.9493926)(681.97019043,56.92939941)
\curveto(681.80018707,56.90939264)(681.65518722,56.87939267)(681.53519043,56.83939941)
\curveto(681.36518751,56.77939277)(681.20018767,56.69439285)(681.04019043,56.58439941)
\curveto(680.96018791,56.52439302)(680.88518799,56.4443931)(680.81519043,56.34439941)
\curveto(680.75518812,56.25439329)(680.70018817,56.15439339)(680.65019043,56.04439941)
\curveto(680.62018825,55.96439358)(680.59018828,55.87939367)(680.56019043,55.78939941)
\curveto(680.54018833,55.69939385)(680.49518838,55.62939392)(680.42519043,55.57939941)
\curveto(680.38518849,55.549394)(680.31518856,55.52439402)(680.21519043,55.50439941)
\curveto(680.12518875,55.49439405)(680.03018884,55.48939406)(679.93019043,55.48939941)
\curveto(679.83018904,55.48939406)(679.73018914,55.49439405)(679.63019043,55.50439941)
\curveto(679.54018933,55.52439402)(679.4751894,55.549394)(679.43519043,55.57939941)
\curveto(679.39518948,55.60939394)(679.36518951,55.65939389)(679.34519043,55.72939941)
\curveto(679.32518955,55.79939375)(679.32518955,55.87439367)(679.34519043,55.95439941)
\curveto(679.3751895,56.08439346)(679.40518947,56.20439334)(679.43519043,56.31439941)
\curveto(679.4751894,56.43439311)(679.52018935,56.549393)(679.57019043,56.65939941)
\curveto(679.76018911,57.00939254)(680.00018887,57.27939227)(680.29019043,57.46939941)
\curveto(680.58018829,57.66939188)(680.94018793,57.82939172)(681.37019043,57.94939941)
\curveto(681.4701874,57.96939158)(681.5701873,57.98439156)(681.67019043,57.99439941)
\curveto(681.78018709,58.00439154)(681.89018698,58.01939153)(682.00019043,58.03939941)
\curveto(682.04018683,58.0493915)(682.10518677,58.0493915)(682.19519043,58.03939941)
\curveto(682.28518659,58.03939151)(682.34018653,58.0493915)(682.36019043,58.06939941)
\curveto(683.06018581,58.07939147)(683.6701852,57.99939155)(684.19019043,57.82939941)
\curveto(684.71018416,57.65939189)(685.0751838,57.33439221)(685.28519043,56.85439941)
\curveto(685.3751835,56.65439289)(685.42518345,56.41939313)(685.43519043,56.14939941)
\curveto(685.45518342,55.88939366)(685.46518341,55.61439393)(685.46519043,55.32439941)
\lineto(685.46519043,52.00939941)
\curveto(685.46518341,51.86939768)(685.4701834,51.73439781)(685.48019043,51.60439941)
\curveto(685.49018338,51.47439807)(685.52018335,51.36939818)(685.57019043,51.28939941)
\curveto(685.62018325,51.21939833)(685.68518319,51.16939838)(685.76519043,51.13939941)
\curveto(685.85518302,51.09939845)(685.94018293,51.06939848)(686.02019043,51.04939941)
\curveto(686.10018277,51.03939851)(686.16018271,50.99439855)(686.20019043,50.91439941)
\curveto(686.22018265,50.88439866)(686.23018264,50.85439869)(686.23019043,50.82439941)
\curveto(686.23018264,50.79439875)(686.23518264,50.75439879)(686.24519043,50.70439941)
\moveto(684.10019043,52.36939941)
\curveto(684.16018471,52.50939704)(684.19018468,52.66939688)(684.19019043,52.84939941)
\curveto(684.20018467,53.03939651)(684.20518467,53.23439631)(684.20519043,53.43439941)
\curveto(684.20518467,53.544396)(684.20018467,53.6443959)(684.19019043,53.73439941)
\curveto(684.18018469,53.82439572)(684.14018473,53.89439565)(684.07019043,53.94439941)
\curveto(684.04018483,53.96439558)(683.9701849,53.97439557)(683.86019043,53.97439941)
\curveto(683.84018503,53.95439559)(683.80518507,53.9443956)(683.75519043,53.94439941)
\curveto(683.70518517,53.9443956)(683.66018521,53.93439561)(683.62019043,53.91439941)
\curveto(683.54018533,53.89439565)(683.45018542,53.87439567)(683.35019043,53.85439941)
\lineto(683.05019043,53.79439941)
\curveto(683.02018585,53.79439575)(682.98518589,53.78939576)(682.94519043,53.77939941)
\lineto(682.84019043,53.77939941)
\curveto(682.69018618,53.73939581)(682.52518635,53.71439583)(682.34519043,53.70439941)
\curveto(682.1751867,53.70439584)(682.01518686,53.68439586)(681.86519043,53.64439941)
\curveto(681.78518709,53.62439592)(681.71018716,53.60439594)(681.64019043,53.58439941)
\curveto(681.58018729,53.57439597)(681.51018736,53.55939599)(681.43019043,53.53939941)
\curveto(681.2701876,53.48939606)(681.12018775,53.42439612)(680.98019043,53.34439941)
\curveto(680.84018803,53.27439627)(680.72018815,53.18439636)(680.62019043,53.07439941)
\curveto(680.52018835,52.96439658)(680.44518843,52.82939672)(680.39519043,52.66939941)
\curveto(680.34518853,52.51939703)(680.32518855,52.33439721)(680.33519043,52.11439941)
\curveto(680.33518854,52.01439753)(680.35018852,51.91939763)(680.38019043,51.82939941)
\curveto(680.42018845,51.7493978)(680.46518841,51.67439787)(680.51519043,51.60439941)
\curveto(680.59518828,51.49439805)(680.70018817,51.39939815)(680.83019043,51.31939941)
\curveto(680.96018791,51.2493983)(681.10018777,51.18939836)(681.25019043,51.13939941)
\curveto(681.30018757,51.12939842)(681.35018752,51.12439842)(681.40019043,51.12439941)
\curveto(681.45018742,51.12439842)(681.50018737,51.11939843)(681.55019043,51.10939941)
\curveto(681.62018725,51.08939846)(681.70518717,51.07439847)(681.80519043,51.06439941)
\curveto(681.91518696,51.06439848)(682.00518687,51.07439847)(682.07519043,51.09439941)
\curveto(682.13518674,51.11439843)(682.19518668,51.11939843)(682.25519043,51.10939941)
\curveto(682.31518656,51.10939844)(682.3751865,51.11939843)(682.43519043,51.13939941)
\curveto(682.51518636,51.15939839)(682.59018628,51.17439837)(682.66019043,51.18439941)
\curveto(682.74018613,51.19439835)(682.81518606,51.21439833)(682.88519043,51.24439941)
\curveto(683.1751857,51.36439818)(683.42018545,51.50939804)(683.62019043,51.67939941)
\curveto(683.83018504,51.8493977)(683.99018488,52.07939747)(684.10019043,52.36939941)
}
}
{
\newrgbcolor{curcolor}{0 0 0}
\pscustom[linestyle=none,fillstyle=solid,fillcolor=curcolor]
{
\newpath
\moveto(694.37683105,50.95939941)
\lineto(694.37683105,50.56939941)
\curveto(694.37682318,50.4493991)(694.3518232,50.3493992)(694.30183105,50.26939941)
\curveto(694.2518233,50.19939935)(694.16682339,50.15939939)(694.04683105,50.14939941)
\lineto(693.70183105,50.14939941)
\curveto(693.64182391,50.1493994)(693.58182397,50.1443994)(693.52183105,50.13439941)
\curveto(693.47182408,50.13439941)(693.42682413,50.1443994)(693.38683105,50.16439941)
\curveto(693.29682426,50.18439936)(693.23682432,50.22439932)(693.20683105,50.28439941)
\curveto(693.16682439,50.33439921)(693.14182441,50.39439915)(693.13183105,50.46439941)
\curveto(693.13182442,50.53439901)(693.11682444,50.60439894)(693.08683105,50.67439941)
\curveto(693.07682448,50.69439885)(693.06182449,50.70939884)(693.04183105,50.71939941)
\curveto(693.03182452,50.73939881)(693.01682454,50.75939879)(692.99683105,50.77939941)
\curveto(692.89682466,50.78939876)(692.81682474,50.76939878)(692.75683105,50.71939941)
\curveto(692.70682485,50.66939888)(692.6518249,50.61939893)(692.59183105,50.56939941)
\curveto(692.39182516,50.41939913)(692.19182536,50.30439924)(691.99183105,50.22439941)
\curveto(691.81182574,50.1443994)(691.60182595,50.08439946)(691.36183105,50.04439941)
\curveto(691.13182642,50.00439954)(690.89182666,49.98439956)(690.64183105,49.98439941)
\curveto(690.40182715,49.97439957)(690.16182739,49.98939956)(689.92183105,50.02939941)
\curveto(689.68182787,50.05939949)(689.47182808,50.11439943)(689.29183105,50.19439941)
\curveto(688.77182878,50.41439913)(688.3518292,50.70939884)(688.03183105,51.07939941)
\curveto(687.71182984,51.45939809)(687.46183009,51.92939762)(687.28183105,52.48939941)
\curveto(687.24183031,52.57939697)(687.21183034,52.66939688)(687.19183105,52.75939941)
\curveto(687.18183037,52.85939669)(687.16183039,52.95939659)(687.13183105,53.05939941)
\curveto(687.12183043,53.10939644)(687.11683044,53.15939639)(687.11683105,53.20939941)
\curveto(687.11683044,53.25939629)(687.11183044,53.30939624)(687.10183105,53.35939941)
\curveto(687.08183047,53.40939614)(687.07183048,53.45939609)(687.07183105,53.50939941)
\curveto(687.08183047,53.56939598)(687.08183047,53.62439592)(687.07183105,53.67439941)
\lineto(687.07183105,53.82439941)
\curveto(687.0518305,53.87439567)(687.04183051,53.93939561)(687.04183105,54.01939941)
\curveto(687.04183051,54.09939545)(687.0518305,54.16439538)(687.07183105,54.21439941)
\lineto(687.07183105,54.37939941)
\curveto(687.09183046,54.4493951)(687.09683046,54.51939503)(687.08683105,54.58939941)
\curveto(687.08683047,54.66939488)(687.09683046,54.7443948)(687.11683105,54.81439941)
\curveto(687.12683043,54.86439468)(687.13183042,54.90939464)(687.13183105,54.94939941)
\curveto(687.13183042,54.98939456)(687.13683042,55.03439451)(687.14683105,55.08439941)
\curveto(687.17683038,55.18439436)(687.20183035,55.27939427)(687.22183105,55.36939941)
\curveto(687.24183031,55.46939408)(687.26683029,55.56439398)(687.29683105,55.65439941)
\curveto(687.42683013,56.03439351)(687.59182996,56.37439317)(687.79183105,56.67439941)
\curveto(688.00182955,56.98439256)(688.2518293,57.23939231)(688.54183105,57.43939941)
\curveto(688.71182884,57.55939199)(688.88682867,57.65939189)(689.06683105,57.73939941)
\curveto(689.2568283,57.81939173)(689.46182809,57.88939166)(689.68183105,57.94939941)
\curveto(689.7518278,57.95939159)(689.81682774,57.96939158)(689.87683105,57.97939941)
\curveto(689.94682761,57.98939156)(690.01682754,58.00439154)(690.08683105,58.02439941)
\lineto(690.23683105,58.02439941)
\curveto(690.31682724,58.0443915)(690.43182712,58.05439149)(690.58183105,58.05439941)
\curveto(690.74182681,58.05439149)(690.86182669,58.0443915)(690.94183105,58.02439941)
\curveto(690.98182657,58.01439153)(691.03682652,58.00939154)(691.10683105,58.00939941)
\curveto(691.21682634,57.97939157)(691.32682623,57.95439159)(691.43683105,57.93439941)
\curveto(691.54682601,57.92439162)(691.6518259,57.89439165)(691.75183105,57.84439941)
\curveto(691.90182565,57.78439176)(692.04182551,57.71939183)(692.17183105,57.64939941)
\curveto(692.31182524,57.57939197)(692.44182511,57.49939205)(692.56183105,57.40939941)
\curveto(692.62182493,57.35939219)(692.68182487,57.30439224)(692.74183105,57.24439941)
\curveto(692.81182474,57.19439235)(692.90182465,57.17939237)(693.01183105,57.19939941)
\curveto(693.03182452,57.22939232)(693.04682451,57.25439229)(693.05683105,57.27439941)
\curveto(693.07682448,57.29439225)(693.09182446,57.32439222)(693.10183105,57.36439941)
\curveto(693.13182442,57.45439209)(693.14182441,57.56939198)(693.13183105,57.70939941)
\lineto(693.13183105,58.08439941)
\lineto(693.13183105,59.80939941)
\lineto(693.13183105,60.27439941)
\curveto(693.13182442,60.45438909)(693.1568244,60.58438896)(693.20683105,60.66439941)
\curveto(693.24682431,60.73438881)(693.30682425,60.77938877)(693.38683105,60.79939941)
\curveto(693.40682415,60.79938875)(693.43182412,60.79938875)(693.46183105,60.79939941)
\curveto(693.49182406,60.80938874)(693.51682404,60.81438873)(693.53683105,60.81439941)
\curveto(693.67682388,60.82438872)(693.82182373,60.82438872)(693.97183105,60.81439941)
\curveto(694.13182342,60.81438873)(694.24182331,60.77438877)(694.30183105,60.69439941)
\curveto(694.3518232,60.61438893)(694.37682318,60.51438903)(694.37683105,60.39439941)
\lineto(694.37683105,60.01939941)
\lineto(694.37683105,50.95939941)
\moveto(693.16183105,53.79439941)
\curveto(693.18182437,53.8443957)(693.19182436,53.90939564)(693.19183105,53.98939941)
\curveto(693.19182436,54.07939547)(693.18182437,54.1493954)(693.16183105,54.19939941)
\lineto(693.16183105,54.42439941)
\curveto(693.14182441,54.51439503)(693.12682443,54.60439494)(693.11683105,54.69439941)
\curveto(693.10682445,54.79439475)(693.08682447,54.88439466)(693.05683105,54.96439941)
\curveto(693.03682452,55.0443945)(693.01682454,55.11939443)(692.99683105,55.18939941)
\curveto(692.98682457,55.25939429)(692.96682459,55.32939422)(692.93683105,55.39939941)
\curveto(692.81682474,55.69939385)(692.66182489,55.96439358)(692.47183105,56.19439941)
\curveto(692.28182527,56.42439312)(692.04182551,56.60439294)(691.75183105,56.73439941)
\curveto(691.6518259,56.78439276)(691.54682601,56.81939273)(691.43683105,56.83939941)
\curveto(691.33682622,56.86939268)(691.22682633,56.89439265)(691.10683105,56.91439941)
\curveto(691.02682653,56.93439261)(690.93682662,56.9443926)(690.83683105,56.94439941)
\lineto(690.56683105,56.94439941)
\curveto(690.51682704,56.93439261)(690.47182708,56.92439262)(690.43183105,56.91439941)
\lineto(690.29683105,56.91439941)
\curveto(690.21682734,56.89439265)(690.13182742,56.87439267)(690.04183105,56.85439941)
\curveto(689.96182759,56.83439271)(689.88182767,56.80939274)(689.80183105,56.77939941)
\curveto(689.48182807,56.63939291)(689.22182833,56.43439311)(689.02183105,56.16439941)
\curveto(688.83182872,55.90439364)(688.67682888,55.59939395)(688.55683105,55.24939941)
\curveto(688.51682904,55.13939441)(688.48682907,55.02439452)(688.46683105,54.90439941)
\curveto(688.4568291,54.79439475)(688.44182911,54.68439486)(688.42183105,54.57439941)
\curveto(688.42182913,54.53439501)(688.41682914,54.49439505)(688.40683105,54.45439941)
\lineto(688.40683105,54.34939941)
\curveto(688.38682917,54.29939525)(688.37682918,54.2443953)(688.37683105,54.18439941)
\curveto(688.38682917,54.12439542)(688.39182916,54.06939548)(688.39183105,54.01939941)
\lineto(688.39183105,53.68939941)
\curveto(688.39182916,53.58939596)(688.40182915,53.49439605)(688.42183105,53.40439941)
\curveto(688.43182912,53.37439617)(688.43682912,53.32439622)(688.43683105,53.25439941)
\curveto(688.4568291,53.18439636)(688.47182908,53.11439643)(688.48183105,53.04439941)
\lineto(688.54183105,52.83439941)
\curveto(688.6518289,52.48439706)(688.80182875,52.18439736)(688.99183105,51.93439941)
\curveto(689.18182837,51.68439786)(689.42182813,51.47939807)(689.71183105,51.31939941)
\curveto(689.80182775,51.26939828)(689.89182766,51.22939832)(689.98183105,51.19939941)
\curveto(690.07182748,51.16939838)(690.17182738,51.13939841)(690.28183105,51.10939941)
\curveto(690.33182722,51.08939846)(690.38182717,51.08439846)(690.43183105,51.09439941)
\curveto(690.49182706,51.10439844)(690.54682701,51.09939845)(690.59683105,51.07939941)
\curveto(690.63682692,51.06939848)(690.67682688,51.06439848)(690.71683105,51.06439941)
\lineto(690.85183105,51.06439941)
\lineto(690.98683105,51.06439941)
\curveto(691.01682654,51.07439847)(691.06682649,51.07939847)(691.13683105,51.07939941)
\curveto(691.21682634,51.09939845)(691.29682626,51.11439843)(691.37683105,51.12439941)
\curveto(691.4568261,51.1443984)(691.53182602,51.16939838)(691.60183105,51.19939941)
\curveto(691.93182562,51.33939821)(692.19682536,51.51439803)(692.39683105,51.72439941)
\curveto(692.60682495,51.9443976)(692.78182477,52.21939733)(692.92183105,52.54939941)
\curveto(692.97182458,52.65939689)(693.00682455,52.76939678)(693.02683105,52.87939941)
\curveto(693.04682451,52.98939656)(693.07182448,53.09939645)(693.10183105,53.20939941)
\curveto(693.12182443,53.2493963)(693.13182442,53.28439626)(693.13183105,53.31439941)
\curveto(693.13182442,53.35439619)(693.13682442,53.39439615)(693.14683105,53.43439941)
\curveto(693.1568244,53.49439605)(693.1568244,53.55439599)(693.14683105,53.61439941)
\curveto(693.14682441,53.67439587)(693.1518244,53.73439581)(693.16183105,53.79439941)
}
}
{
\newrgbcolor{curcolor}{0 0 0}
\pscustom[linestyle=none,fillstyle=solid,fillcolor=curcolor]
{
\newpath
\moveto(703.44808105,54.34939941)
\curveto(703.46807299,54.28939526)(703.47807298,54.19439535)(703.47808105,54.06439941)
\curveto(703.47807298,53.9443956)(703.47307299,53.85939569)(703.46308105,53.80939941)
\lineto(703.46308105,53.65939941)
\curveto(703.45307301,53.57939597)(703.44307302,53.50439604)(703.43308105,53.43439941)
\curveto(703.43307303,53.37439617)(703.42807303,53.30439624)(703.41808105,53.22439941)
\curveto(703.39807306,53.16439638)(703.38307308,53.10439644)(703.37308105,53.04439941)
\curveto(703.37307309,52.98439656)(703.3630731,52.92439662)(703.34308105,52.86439941)
\curveto(703.30307316,52.73439681)(703.26807319,52.60439694)(703.23808105,52.47439941)
\curveto(703.20807325,52.3443972)(703.16807329,52.22439732)(703.11808105,52.11439941)
\curveto(702.90807355,51.63439791)(702.62807383,51.22939832)(702.27808105,50.89939941)
\curveto(701.92807453,50.57939897)(701.49807496,50.33439921)(700.98808105,50.16439941)
\curveto(700.87807558,50.12439942)(700.7580757,50.09439945)(700.62808105,50.07439941)
\curveto(700.50807595,50.05439949)(700.38307608,50.03439951)(700.25308105,50.01439941)
\curveto(700.19307627,50.00439954)(700.12807633,49.99939955)(700.05808105,49.99939941)
\curveto(699.99807646,49.98939956)(699.93807652,49.98439956)(699.87808105,49.98439941)
\curveto(699.83807662,49.97439957)(699.77807668,49.96939958)(699.69808105,49.96939941)
\curveto(699.62807683,49.96939958)(699.57807688,49.97439957)(699.54808105,49.98439941)
\curveto(699.50807695,49.99439955)(699.46807699,49.99939955)(699.42808105,49.99939941)
\curveto(699.38807707,49.98939956)(699.35307711,49.98939956)(699.32308105,49.99939941)
\lineto(699.23308105,49.99939941)
\lineto(698.87308105,50.04439941)
\curveto(698.73307773,50.08439946)(698.59807786,50.12439942)(698.46808105,50.16439941)
\curveto(698.33807812,50.20439934)(698.21307825,50.2493993)(698.09308105,50.29939941)
\curveto(697.64307882,50.49939905)(697.27307919,50.75939879)(696.98308105,51.07939941)
\curveto(696.69307977,51.39939815)(696.45308001,51.78939776)(696.26308105,52.24939941)
\curveto(696.21308025,52.3493972)(696.17308029,52.4493971)(696.14308105,52.54939941)
\curveto(696.12308034,52.6493969)(696.10308036,52.75439679)(696.08308105,52.86439941)
\curveto(696.0630804,52.90439664)(696.05308041,52.93439661)(696.05308105,52.95439941)
\curveto(696.0630804,52.98439656)(696.0630804,53.01939653)(696.05308105,53.05939941)
\curveto(696.03308043,53.13939641)(696.01808044,53.21939633)(696.00808105,53.29939941)
\curveto(696.00808045,53.38939616)(695.99808046,53.47439607)(695.97808105,53.55439941)
\lineto(695.97808105,53.67439941)
\curveto(695.97808048,53.71439583)(695.97308049,53.75939579)(695.96308105,53.80939941)
\curveto(695.95308051,53.85939569)(695.94808051,53.9443956)(695.94808105,54.06439941)
\curveto(695.94808051,54.19439535)(695.9580805,54.28939526)(695.97808105,54.34939941)
\curveto(695.99808046,54.41939513)(696.00308046,54.48939506)(695.99308105,54.55939941)
\curveto(695.98308048,54.62939492)(695.98808047,54.69939485)(696.00808105,54.76939941)
\curveto(696.01808044,54.81939473)(696.02308044,54.85939469)(696.02308105,54.88939941)
\curveto(696.03308043,54.92939462)(696.04308042,54.97439457)(696.05308105,55.02439941)
\curveto(696.08308038,55.1443944)(696.10808035,55.26439428)(696.12808105,55.38439941)
\curveto(696.1580803,55.50439404)(696.19808026,55.61939393)(696.24808105,55.72939941)
\curveto(696.39808006,56.09939345)(696.57807988,56.42939312)(696.78808105,56.71939941)
\curveto(697.00807945,57.01939253)(697.27307919,57.26939228)(697.58308105,57.46939941)
\curveto(697.70307876,57.549392)(697.82807863,57.61439193)(697.95808105,57.66439941)
\curveto(698.08807837,57.72439182)(698.22307824,57.78439176)(698.36308105,57.84439941)
\curveto(698.48307798,57.89439165)(698.61307785,57.92439162)(698.75308105,57.93439941)
\curveto(698.89307757,57.95439159)(699.03307743,57.98439156)(699.17308105,58.02439941)
\lineto(699.36808105,58.02439941)
\curveto(699.43807702,58.03439151)(699.50307696,58.0443915)(699.56308105,58.05439941)
\curveto(700.45307601,58.06439148)(701.19307527,57.87939167)(701.78308105,57.49939941)
\curveto(702.37307409,57.11939243)(702.79807366,56.62439292)(703.05808105,56.01439941)
\curveto(703.10807335,55.91439363)(703.14807331,55.81439373)(703.17808105,55.71439941)
\curveto(703.20807325,55.61439393)(703.24307322,55.50939404)(703.28308105,55.39939941)
\curveto(703.31307315,55.28939426)(703.33807312,55.16939438)(703.35808105,55.03939941)
\curveto(703.37807308,54.91939463)(703.40307306,54.79439475)(703.43308105,54.66439941)
\curveto(703.44307302,54.61439493)(703.44307302,54.55939499)(703.43308105,54.49939941)
\curveto(703.43307303,54.4493951)(703.43807302,54.39939515)(703.44808105,54.34939941)
\moveto(702.11308105,53.49439941)
\curveto(702.13307433,53.56439598)(702.13807432,53.6443959)(702.12808105,53.73439941)
\lineto(702.12808105,53.98939941)
\curveto(702.12807433,54.37939517)(702.09307437,54.70939484)(702.02308105,54.97939941)
\curveto(701.99307447,55.05939449)(701.96807449,55.13939441)(701.94808105,55.21939941)
\curveto(701.92807453,55.29939425)(701.90307456,55.37439417)(701.87308105,55.44439941)
\curveto(701.59307487,56.09439345)(701.14807531,56.544393)(700.53808105,56.79439941)
\curveto(700.46807599,56.82439272)(700.39307607,56.8443927)(700.31308105,56.85439941)
\lineto(700.07308105,56.91439941)
\curveto(699.99307647,56.93439261)(699.90807655,56.9443926)(699.81808105,56.94439941)
\lineto(699.54808105,56.94439941)
\lineto(699.27808105,56.89939941)
\curveto(699.17807728,56.87939267)(699.08307738,56.85439269)(698.99308105,56.82439941)
\curveto(698.91307755,56.80439274)(698.83307763,56.77439277)(698.75308105,56.73439941)
\curveto(698.68307778,56.71439283)(698.61807784,56.68439286)(698.55808105,56.64439941)
\curveto(698.49807796,56.60439294)(698.44307802,56.56439298)(698.39308105,56.52439941)
\curveto(698.15307831,56.35439319)(697.9580785,56.1493934)(697.80808105,55.90939941)
\curveto(697.6580788,55.66939388)(697.52807893,55.38939416)(697.41808105,55.06939941)
\curveto(697.38807907,54.96939458)(697.36807909,54.86439468)(697.35808105,54.75439941)
\curveto(697.34807911,54.65439489)(697.33307913,54.549395)(697.31308105,54.43939941)
\curveto(697.30307916,54.39939515)(697.29807916,54.33439521)(697.29808105,54.24439941)
\curveto(697.28807917,54.21439533)(697.28307918,54.17939537)(697.28308105,54.13939941)
\curveto(697.29307917,54.09939545)(697.29807916,54.05439549)(697.29808105,54.00439941)
\lineto(697.29808105,53.70439941)
\curveto(697.29807916,53.60439594)(697.30807915,53.51439603)(697.32808105,53.43439941)
\lineto(697.35808105,53.25439941)
\curveto(697.37807908,53.15439639)(697.39307907,53.05439649)(697.40308105,52.95439941)
\curveto(697.42307904,52.86439668)(697.45307901,52.77939677)(697.49308105,52.69939941)
\curveto(697.59307887,52.45939709)(697.70807875,52.23439731)(697.83808105,52.02439941)
\curveto(697.97807848,51.81439773)(698.14807831,51.63939791)(698.34808105,51.49939941)
\curveto(698.39807806,51.46939808)(698.44307802,51.4443981)(698.48308105,51.42439941)
\curveto(698.52307794,51.40439814)(698.56807789,51.37939817)(698.61808105,51.34939941)
\curveto(698.69807776,51.29939825)(698.78307768,51.25439829)(698.87308105,51.21439941)
\curveto(698.97307749,51.18439836)(699.07807738,51.15439839)(699.18808105,51.12439941)
\curveto(699.23807722,51.10439844)(699.28307718,51.09439845)(699.32308105,51.09439941)
\curveto(699.37307709,51.10439844)(699.42307704,51.10439844)(699.47308105,51.09439941)
\curveto(699.50307696,51.08439846)(699.5630769,51.07439847)(699.65308105,51.06439941)
\curveto(699.75307671,51.05439849)(699.82807663,51.05939849)(699.87808105,51.07939941)
\curveto(699.91807654,51.08939846)(699.9580765,51.08939846)(699.99808105,51.07939941)
\curveto(700.03807642,51.07939847)(700.07807638,51.08939846)(700.11808105,51.10939941)
\curveto(700.19807626,51.12939842)(700.27807618,51.1443984)(700.35808105,51.15439941)
\curveto(700.43807602,51.17439837)(700.51307595,51.19939835)(700.58308105,51.22939941)
\curveto(700.92307554,51.36939818)(701.19807526,51.56439798)(701.40808105,51.81439941)
\curveto(701.61807484,52.06439748)(701.79307467,52.35939719)(701.93308105,52.69939941)
\curveto(701.98307448,52.81939673)(702.01307445,52.9443966)(702.02308105,53.07439941)
\curveto(702.04307442,53.21439633)(702.07307439,53.35439619)(702.11308105,53.49439941)
}
}
{
\newrgbcolor{curcolor}{0 0 0}
\pscustom[linestyle=none,fillstyle=solid,fillcolor=curcolor]
{
\newpath
\moveto(708.5813623,58.05439941)
\curveto(708.81135751,58.05439149)(708.94135738,57.99439155)(708.9713623,57.87439941)
\curveto(709.00135732,57.76439178)(709.01635731,57.59939195)(709.0163623,57.37939941)
\lineto(709.0163623,57.09439941)
\curveto(709.01635731,57.00439254)(708.99135733,56.92939262)(708.9413623,56.86939941)
\curveto(708.88135744,56.78939276)(708.79635753,56.7443928)(708.6863623,56.73439941)
\curveto(708.57635775,56.73439281)(708.46635786,56.71939283)(708.3563623,56.68939941)
\curveto(708.21635811,56.65939289)(708.08135824,56.62939292)(707.9513623,56.59939941)
\curveto(707.83135849,56.56939298)(707.71635861,56.52939302)(707.6063623,56.47939941)
\curveto(707.31635901,56.3493932)(707.08135924,56.16939338)(706.9013623,55.93939941)
\curveto(706.7213596,55.71939383)(706.56635976,55.46439408)(706.4363623,55.17439941)
\curveto(706.39635993,55.06439448)(706.36635996,54.9493946)(706.3463623,54.82939941)
\curveto(706.32636,54.71939483)(706.30136002,54.60439494)(706.2713623,54.48439941)
\curveto(706.26136006,54.43439511)(706.25636007,54.38439516)(706.2563623,54.33439941)
\curveto(706.26636006,54.28439526)(706.26636006,54.23439531)(706.2563623,54.18439941)
\curveto(706.2263601,54.06439548)(706.21136011,53.92439562)(706.2113623,53.76439941)
\curveto(706.2213601,53.61439593)(706.2263601,53.46939608)(706.2263623,53.32939941)
\lineto(706.2263623,51.48439941)
\lineto(706.2263623,51.13939941)
\curveto(706.2263601,51.01939853)(706.2213601,50.90439864)(706.2113623,50.79439941)
\curveto(706.20136012,50.68439886)(706.19636013,50.58939896)(706.1963623,50.50939941)
\curveto(706.20636012,50.42939912)(706.18636014,50.35939919)(706.1363623,50.29939941)
\curveto(706.08636024,50.22939932)(706.00636032,50.18939936)(705.8963623,50.17939941)
\curveto(705.79636053,50.16939938)(705.68636064,50.16439938)(705.5663623,50.16439941)
\lineto(705.2963623,50.16439941)
\curveto(705.24636108,50.18439936)(705.19636113,50.19939935)(705.1463623,50.20939941)
\curveto(705.10636122,50.22939932)(705.07636125,50.25439929)(705.0563623,50.28439941)
\curveto(705.00636132,50.35439919)(704.97636135,50.43939911)(704.9663623,50.53939941)
\lineto(704.9663623,50.86939941)
\lineto(704.9663623,52.02439941)
\lineto(704.9663623,56.17939941)
\lineto(704.9663623,57.21439941)
\lineto(704.9663623,57.51439941)
\curveto(704.97636135,57.61439193)(705.00636132,57.69939185)(705.0563623,57.76939941)
\curveto(705.08636124,57.80939174)(705.13636119,57.83939171)(705.2063623,57.85939941)
\curveto(705.28636104,57.87939167)(705.37136095,57.88939166)(705.4613623,57.88939941)
\curveto(705.55136077,57.89939165)(705.64136068,57.89939165)(705.7313623,57.88939941)
\curveto(705.8213605,57.87939167)(705.89136043,57.86439168)(705.9413623,57.84439941)
\curveto(706.0213603,57.81439173)(706.07136025,57.75439179)(706.0913623,57.66439941)
\curveto(706.1213602,57.58439196)(706.13636019,57.49439205)(706.1363623,57.39439941)
\lineto(706.1363623,57.09439941)
\curveto(706.13636019,56.99439255)(706.15636017,56.90439264)(706.1963623,56.82439941)
\curveto(706.20636012,56.80439274)(706.21636011,56.78939276)(706.2263623,56.77939941)
\lineto(706.2713623,56.73439941)
\curveto(706.38135994,56.73439281)(706.47135985,56.77939277)(706.5413623,56.86939941)
\curveto(706.61135971,56.96939258)(706.67135965,57.0493925)(706.7213623,57.10939941)
\lineto(706.8113623,57.19939941)
\curveto(706.90135942,57.30939224)(707.0263593,57.42439212)(707.1863623,57.54439941)
\curveto(707.34635898,57.66439188)(707.49635883,57.75439179)(707.6363623,57.81439941)
\curveto(707.7263586,57.86439168)(707.8213585,57.89939165)(707.9213623,57.91939941)
\curveto(708.0213583,57.9493916)(708.1263582,57.97939157)(708.2363623,58.00939941)
\curveto(708.29635803,58.01939153)(708.35635797,58.02439152)(708.4163623,58.02439941)
\curveto(708.47635785,58.03439151)(708.53135779,58.0443915)(708.5813623,58.05439941)
}
}
{
\newrgbcolor{curcolor}{0 0 0}
\pscustom[linestyle=none,fillstyle=solid,fillcolor=curcolor]
{
\newpath
\moveto(143.59249146,89.16295776)
\lineto(143.59249146,88.90795776)
\curveto(143.60248375,88.827953)(143.59748376,88.75295307)(143.57749146,88.68295776)
\lineto(143.57749146,88.44295776)
\lineto(143.57749146,88.27795776)
\curveto(143.5574838,88.17795365)(143.54748381,88.07295375)(143.54749146,87.96295776)
\curveto(143.54748381,87.86295396)(143.53748382,87.76295406)(143.51749146,87.66295776)
\lineto(143.51749146,87.51295776)
\curveto(143.48748387,87.37295445)(143.46748389,87.23295459)(143.45749146,87.09295776)
\curveto(143.44748391,86.96295486)(143.42248393,86.83295499)(143.38249146,86.70295776)
\curveto(143.36248399,86.6229552)(143.34248401,86.53795529)(143.32249146,86.44795776)
\lineto(143.26249146,86.20795776)
\lineto(143.14249146,85.90795776)
\curveto(143.11248424,85.81795601)(143.07748428,85.7279561)(143.03749146,85.63795776)
\curveto(142.93748442,85.41795641)(142.80248455,85.20295662)(142.63249146,84.99295776)
\curveto(142.47248488,84.78295704)(142.29748506,84.61295721)(142.10749146,84.48295776)
\curveto(142.0574853,84.44295738)(141.99748536,84.40295742)(141.92749146,84.36295776)
\curveto(141.86748549,84.33295749)(141.80748555,84.29795753)(141.74749146,84.25795776)
\curveto(141.66748569,84.20795762)(141.57248578,84.16795766)(141.46249146,84.13795776)
\curveto(141.352486,84.10795772)(141.24748611,84.07795775)(141.14749146,84.04795776)
\curveto(141.03748632,84.00795782)(140.92748643,83.98295784)(140.81749146,83.97295776)
\curveto(140.70748665,83.96295786)(140.59248676,83.94795788)(140.47249146,83.92795776)
\curveto(140.43248692,83.91795791)(140.38748697,83.91795791)(140.33749146,83.92795776)
\curveto(140.29748706,83.9279579)(140.2574871,83.9229579)(140.21749146,83.91295776)
\curveto(140.17748718,83.90295792)(140.12248723,83.89795793)(140.05249146,83.89795776)
\curveto(139.98248737,83.89795793)(139.93248742,83.90295792)(139.90249146,83.91295776)
\curveto(139.8524875,83.93295789)(139.80748755,83.93795789)(139.76749146,83.92795776)
\curveto(139.72748763,83.91795791)(139.69248766,83.91795791)(139.66249146,83.92795776)
\lineto(139.57249146,83.92795776)
\curveto(139.51248784,83.94795788)(139.44748791,83.96295786)(139.37749146,83.97295776)
\curveto(139.31748804,83.97295785)(139.2524881,83.97795785)(139.18249146,83.98795776)
\curveto(139.01248834,84.03795779)(138.8524885,84.08795774)(138.70249146,84.13795776)
\curveto(138.5524888,84.18795764)(138.40748895,84.25295757)(138.26749146,84.33295776)
\curveto(138.21748914,84.37295745)(138.16248919,84.40295742)(138.10249146,84.42295776)
\curveto(138.0524893,84.45295737)(138.00248935,84.48795734)(137.95249146,84.52795776)
\curveto(137.71248964,84.70795712)(137.51248984,84.9279569)(137.35249146,85.18795776)
\curveto(137.19249016,85.44795638)(137.0524903,85.73295609)(136.93249146,86.04295776)
\curveto(136.87249048,86.18295564)(136.82749053,86.3229555)(136.79749146,86.46295776)
\curveto(136.76749059,86.61295521)(136.73249062,86.76795506)(136.69249146,86.92795776)
\curveto(136.67249068,87.03795479)(136.6574907,87.14795468)(136.64749146,87.25795776)
\curveto(136.63749072,87.36795446)(136.62249073,87.47795435)(136.60249146,87.58795776)
\curveto(136.59249076,87.6279542)(136.58749077,87.66795416)(136.58749146,87.70795776)
\curveto(136.59749076,87.74795408)(136.59749076,87.78795404)(136.58749146,87.82795776)
\curveto(136.57749078,87.87795395)(136.57249078,87.9279539)(136.57249146,87.97795776)
\lineto(136.57249146,88.14295776)
\curveto(136.5524908,88.19295363)(136.54749081,88.24295358)(136.55749146,88.29295776)
\curveto(136.56749079,88.35295347)(136.56749079,88.40795342)(136.55749146,88.45795776)
\curveto(136.54749081,88.49795333)(136.54749081,88.54295328)(136.55749146,88.59295776)
\curveto(136.56749079,88.64295318)(136.56249079,88.69295313)(136.54249146,88.74295776)
\curveto(136.52249083,88.81295301)(136.51749084,88.88795294)(136.52749146,88.96795776)
\curveto(136.53749082,89.05795277)(136.54249081,89.14295268)(136.54249146,89.22295776)
\curveto(136.54249081,89.31295251)(136.53749082,89.41295241)(136.52749146,89.52295776)
\curveto(136.51749084,89.64295218)(136.52249083,89.74295208)(136.54249146,89.82295776)
\lineto(136.54249146,90.10795776)
\lineto(136.58749146,90.73795776)
\curveto(136.59749076,90.83795099)(136.60749075,90.93295089)(136.61749146,91.02295776)
\lineto(136.64749146,91.32295776)
\curveto(136.66749069,91.37295045)(136.67249068,91.4229504)(136.66249146,91.47295776)
\curveto(136.66249069,91.53295029)(136.67249068,91.58795024)(136.69249146,91.63795776)
\curveto(136.74249061,91.80795002)(136.78249057,91.97294985)(136.81249146,92.13295776)
\curveto(136.84249051,92.30294952)(136.89249046,92.46294936)(136.96249146,92.61295776)
\curveto(137.1524902,93.07294875)(137.37248998,93.44794838)(137.62249146,93.73795776)
\curveto(137.88248947,94.0279478)(138.24248911,94.27294755)(138.70249146,94.47295776)
\curveto(138.83248852,94.5229473)(138.96248839,94.55794727)(139.09249146,94.57795776)
\curveto(139.23248812,94.59794723)(139.37248798,94.6229472)(139.51249146,94.65295776)
\curveto(139.58248777,94.66294716)(139.64748771,94.66794716)(139.70749146,94.66795776)
\curveto(139.76748759,94.66794716)(139.83248752,94.67294715)(139.90249146,94.68295776)
\curveto(140.73248662,94.70294712)(141.40248595,94.55294727)(141.91249146,94.23295776)
\curveto(142.42248493,93.9229479)(142.80248455,93.48294834)(143.05249146,92.91295776)
\curveto(143.10248425,92.79294903)(143.14748421,92.66794916)(143.18749146,92.53795776)
\curveto(143.22748413,92.40794942)(143.27248408,92.27294955)(143.32249146,92.13295776)
\curveto(143.34248401,92.05294977)(143.357484,91.96794986)(143.36749146,91.87795776)
\lineto(143.42749146,91.63795776)
\curveto(143.4574839,91.5279503)(143.47248388,91.41795041)(143.47249146,91.30795776)
\curveto(143.48248387,91.19795063)(143.49748386,91.08795074)(143.51749146,90.97795776)
\curveto(143.53748382,90.9279509)(143.54248381,90.88295094)(143.53249146,90.84295776)
\curveto(143.53248382,90.80295102)(143.53748382,90.76295106)(143.54749146,90.72295776)
\curveto(143.5574838,90.67295115)(143.5574838,90.61795121)(143.54749146,90.55795776)
\curveto(143.54748381,90.50795132)(143.5524838,90.45795137)(143.56249146,90.40795776)
\lineto(143.56249146,90.27295776)
\curveto(143.58248377,90.21295161)(143.58248377,90.14295168)(143.56249146,90.06295776)
\curveto(143.5524838,89.99295183)(143.5574838,89.9279519)(143.57749146,89.86795776)
\curveto(143.58748377,89.83795199)(143.59248376,89.79795203)(143.59249146,89.74795776)
\lineto(143.59249146,89.62795776)
\lineto(143.59249146,89.16295776)
\moveto(142.04749146,86.83795776)
\curveto(142.14748521,87.15795467)(142.20748515,87.5229543)(142.22749146,87.93295776)
\curveto(142.24748511,88.34295348)(142.2574851,88.75295307)(142.25749146,89.16295776)
\curveto(142.2574851,89.59295223)(142.24748511,90.01295181)(142.22749146,90.42295776)
\curveto(142.20748515,90.83295099)(142.16248519,91.21795061)(142.09249146,91.57795776)
\curveto(142.02248533,91.93794989)(141.91248544,92.25794957)(141.76249146,92.53795776)
\curveto(141.62248573,92.827949)(141.42748593,93.06294876)(141.17749146,93.24295776)
\curveto(141.01748634,93.35294847)(140.83748652,93.43294839)(140.63749146,93.48295776)
\curveto(140.43748692,93.54294828)(140.19248716,93.57294825)(139.90249146,93.57295776)
\curveto(139.88248747,93.55294827)(139.84748751,93.54294828)(139.79749146,93.54295776)
\curveto(139.74748761,93.55294827)(139.70748765,93.55294827)(139.67749146,93.54295776)
\curveto(139.59748776,93.5229483)(139.52248783,93.50294832)(139.45249146,93.48295776)
\curveto(139.39248796,93.47294835)(139.32748803,93.45294837)(139.25749146,93.42295776)
\curveto(138.98748837,93.30294852)(138.76748859,93.13294869)(138.59749146,92.91295776)
\curveto(138.43748892,92.70294912)(138.30248905,92.45794937)(138.19249146,92.17795776)
\curveto(138.14248921,92.06794976)(138.10248925,91.94794988)(138.07249146,91.81795776)
\curveto(138.0524893,91.69795013)(138.02748933,91.57295025)(137.99749146,91.44295776)
\curveto(137.97748938,91.39295043)(137.96748939,91.33795049)(137.96749146,91.27795776)
\curveto(137.96748939,91.2279506)(137.96248939,91.17795065)(137.95249146,91.12795776)
\curveto(137.94248941,91.03795079)(137.93248942,90.94295088)(137.92249146,90.84295776)
\curveto(137.91248944,90.75295107)(137.90248945,90.65795117)(137.89249146,90.55795776)
\curveto(137.89248946,90.47795135)(137.88748947,90.39295143)(137.87749146,90.30295776)
\lineto(137.87749146,90.06295776)
\lineto(137.87749146,89.88295776)
\curveto(137.86748949,89.85295197)(137.86248949,89.81795201)(137.86249146,89.77795776)
\lineto(137.86249146,89.64295776)
\lineto(137.86249146,89.19295776)
\curveto(137.86248949,89.11295271)(137.8574895,89.0279528)(137.84749146,88.93795776)
\curveto(137.84748951,88.85795297)(137.8574895,88.78295304)(137.87749146,88.71295776)
\lineto(137.87749146,88.44295776)
\curveto(137.87748948,88.4229534)(137.87248948,88.39295343)(137.86249146,88.35295776)
\curveto(137.86248949,88.3229535)(137.86748949,88.29795353)(137.87749146,88.27795776)
\curveto(137.88748947,88.17795365)(137.89248946,88.07795375)(137.89249146,87.97795776)
\curveto(137.90248945,87.88795394)(137.91248944,87.78795404)(137.92249146,87.67795776)
\curveto(137.9524894,87.55795427)(137.96748939,87.43295439)(137.96749146,87.30295776)
\curveto(137.97748938,87.18295464)(138.00248935,87.06795476)(138.04249146,86.95795776)
\curveto(138.12248923,86.65795517)(138.20748915,86.39295543)(138.29749146,86.16295776)
\curveto(138.39748896,85.93295589)(138.54248881,85.71795611)(138.73249146,85.51795776)
\curveto(138.94248841,85.31795651)(139.20748815,85.16795666)(139.52749146,85.06795776)
\curveto(139.56748779,85.04795678)(139.60248775,85.03795679)(139.63249146,85.03795776)
\curveto(139.67248768,85.04795678)(139.71748764,85.04295678)(139.76749146,85.02295776)
\curveto(139.80748755,85.01295681)(139.87748748,85.00295682)(139.97749146,84.99295776)
\curveto(140.08748727,84.98295684)(140.17248718,84.98795684)(140.23249146,85.00795776)
\curveto(140.30248705,85.0279568)(140.37248698,85.03795679)(140.44249146,85.03795776)
\curveto(140.51248684,85.04795678)(140.57748678,85.06295676)(140.63749146,85.08295776)
\curveto(140.83748652,85.14295668)(141.01748634,85.2279566)(141.17749146,85.33795776)
\curveto(141.20748615,85.35795647)(141.23248612,85.37795645)(141.25249146,85.39795776)
\lineto(141.31249146,85.45795776)
\curveto(141.352486,85.47795635)(141.40248595,85.51795631)(141.46249146,85.57795776)
\curveto(141.56248579,85.71795611)(141.64748571,85.84795598)(141.71749146,85.96795776)
\curveto(141.78748557,86.08795574)(141.8574855,86.23295559)(141.92749146,86.40295776)
\curveto(141.9574854,86.47295535)(141.97748538,86.54295528)(141.98749146,86.61295776)
\curveto(142.00748535,86.68295514)(142.02748533,86.75795507)(142.04749146,86.83795776)
}
}
{
\newrgbcolor{curcolor}{0 0 0}
\pscustom[linestyle=none,fillstyle=solid,fillcolor=curcolor]
{
\newpath
\moveto(123.95343292,168.96866333)
\curveto(124.05342807,168.96865271)(124.14842797,168.95865272)(124.23843292,168.93866333)
\curveto(124.32842779,168.92865275)(124.39342773,168.89865278)(124.43343292,168.84866333)
\curveto(124.49342763,168.76865291)(124.5234276,168.66365302)(124.52343292,168.53366333)
\lineto(124.52343292,168.14366333)
\lineto(124.52343292,166.64366333)
\lineto(124.52343292,160.25366333)
\lineto(124.52343292,159.08366333)
\lineto(124.52343292,158.76866333)
\curveto(124.53342759,158.66866301)(124.5184276,158.58866309)(124.47843292,158.52866333)
\curveto(124.42842769,158.44866323)(124.35342777,158.39866328)(124.25343292,158.37866333)
\curveto(124.16342796,158.36866331)(124.05342807,158.36366332)(123.92343292,158.36366333)
\lineto(123.69843292,158.36366333)
\curveto(123.6184285,158.3836633)(123.54842857,158.39866328)(123.48843292,158.40866333)
\curveto(123.42842869,158.42866325)(123.37842874,158.46866321)(123.33843292,158.52866333)
\curveto(123.29842882,158.58866309)(123.27842884,158.66366302)(123.27843292,158.75366333)
\lineto(123.27843292,159.05366333)
\lineto(123.27843292,160.14866333)
\lineto(123.27843292,165.48866333)
\curveto(123.25842886,165.5786561)(123.24342888,165.65365603)(123.23343292,165.71366333)
\curveto(123.23342889,165.7836559)(123.20342892,165.84365584)(123.14343292,165.89366333)
\curveto(123.07342905,165.94365574)(122.98342914,165.96865571)(122.87343292,165.96866333)
\curveto(122.77342935,165.9786557)(122.66342946,165.9836557)(122.54343292,165.98366333)
\lineto(121.40343292,165.98366333)
\lineto(120.90843292,165.98366333)
\curveto(120.74843137,165.99365569)(120.63843148,166.05365563)(120.57843292,166.16366333)
\curveto(120.55843156,166.19365549)(120.54843157,166.22365546)(120.54843292,166.25366333)
\curveto(120.54843157,166.29365539)(120.54343158,166.33865534)(120.53343292,166.38866333)
\curveto(120.51343161,166.50865517)(120.5184316,166.61865506)(120.54843292,166.71866333)
\curveto(120.58843153,166.81865486)(120.64343148,166.88865479)(120.71343292,166.92866333)
\curveto(120.79343133,166.9786547)(120.91343121,167.00365468)(121.07343292,167.00366333)
\curveto(121.23343089,167.00365468)(121.36843075,167.01865466)(121.47843292,167.04866333)
\curveto(121.52843059,167.05865462)(121.58343054,167.06365462)(121.64343292,167.06366333)
\curveto(121.70343042,167.07365461)(121.76343036,167.08865459)(121.82343292,167.10866333)
\curveto(121.97343015,167.15865452)(122.11843,167.20865447)(122.25843292,167.25866333)
\curveto(122.39842972,167.31865436)(122.53342959,167.38865429)(122.66343292,167.46866333)
\curveto(122.80342932,167.55865412)(122.9234292,167.66365402)(123.02343292,167.78366333)
\curveto(123.123429,167.90365378)(123.2184289,168.03365365)(123.30843292,168.17366333)
\curveto(123.36842875,168.27365341)(123.41342871,168.3836533)(123.44343292,168.50366333)
\curveto(123.48342864,168.62365306)(123.53342859,168.72865295)(123.59343292,168.81866333)
\curveto(123.64342848,168.8786528)(123.71342841,168.91865276)(123.80343292,168.93866333)
\curveto(123.8234283,168.94865273)(123.84842827,168.95365273)(123.87843292,168.95366333)
\curveto(123.90842821,168.95365273)(123.93342819,168.95865272)(123.95343292,168.96866333)
}
}
{
\newrgbcolor{curcolor}{0 0 0}
\pscustom[linestyle=none,fillstyle=solid,fillcolor=curcolor]
{
\newpath
\moveto(129.7530423,168.77366333)
\lineto(133.3530423,168.77366333)
\lineto(133.9980423,168.77366333)
\curveto(134.07803577,168.77365291)(134.15303569,168.76865291)(134.2230423,168.75866333)
\curveto(134.29303555,168.75865292)(134.35303549,168.74865293)(134.4030423,168.72866333)
\curveto(134.47303537,168.69865298)(134.52803532,168.63865304)(134.5680423,168.54866333)
\curveto(134.58803526,168.51865316)(134.59803525,168.4786532)(134.5980423,168.42866333)
\lineto(134.5980423,168.29366333)
\curveto(134.60803524,168.1836535)(134.60303524,168.0786536)(134.5830423,167.97866333)
\curveto(134.57303527,167.8786538)(134.53803531,167.80865387)(134.4780423,167.76866333)
\curveto(134.38803546,167.69865398)(134.25303559,167.66365402)(134.0730423,167.66366333)
\curveto(133.89303595,167.67365401)(133.72803612,167.678654)(133.5780423,167.67866333)
\lineto(131.5830423,167.67866333)
\lineto(131.0880423,167.67866333)
\lineto(130.9530423,167.67866333)
\curveto(130.91303893,167.678654)(130.87303897,167.67365401)(130.8330423,167.66366333)
\lineto(130.6230423,167.66366333)
\curveto(130.51303933,167.63365405)(130.43303941,167.59365409)(130.3830423,167.54366333)
\curveto(130.33303951,167.50365418)(130.29803955,167.44865423)(130.2780423,167.37866333)
\curveto(130.25803959,167.31865436)(130.2430396,167.24865443)(130.2330423,167.16866333)
\curveto(130.22303962,167.08865459)(130.20303964,166.99865468)(130.1730423,166.89866333)
\curveto(130.12303972,166.69865498)(130.08303976,166.49365519)(130.0530423,166.28366333)
\curveto(130.02303982,166.07365561)(129.98303986,165.86865581)(129.9330423,165.66866333)
\curveto(129.91303993,165.59865608)(129.90303994,165.52865615)(129.9030423,165.45866333)
\curveto(129.90303994,165.39865628)(129.89303995,165.33365635)(129.8730423,165.26366333)
\curveto(129.86303998,165.23365645)(129.85303999,165.19365649)(129.8430423,165.14366333)
\curveto(129.84304,165.10365658)(129.84804,165.06365662)(129.8580423,165.02366333)
\curveto(129.87803997,164.97365671)(129.90303994,164.92865675)(129.9330423,164.88866333)
\curveto(129.97303987,164.85865682)(130.03303981,164.85365683)(130.1130423,164.87366333)
\curveto(130.17303967,164.89365679)(130.23303961,164.91865676)(130.2930423,164.94866333)
\curveto(130.35303949,164.98865669)(130.41303943,165.02365666)(130.4730423,165.05366333)
\curveto(130.53303931,165.07365661)(130.58303926,165.08865659)(130.6230423,165.09866333)
\curveto(130.81303903,165.1786565)(131.01803883,165.23365645)(131.2380423,165.26366333)
\curveto(131.46803838,165.29365639)(131.69803815,165.30365638)(131.9280423,165.29366333)
\curveto(132.16803768,165.29365639)(132.39803745,165.26865641)(132.6180423,165.21866333)
\curveto(132.83803701,165.1786565)(133.03803681,165.11865656)(133.2180423,165.03866333)
\curveto(133.26803658,165.01865666)(133.31303653,164.99865668)(133.3530423,164.97866333)
\curveto(133.40303644,164.95865672)(133.45303639,164.93365675)(133.5030423,164.90366333)
\curveto(133.85303599,164.69365699)(134.13303571,164.46365722)(134.3430423,164.21366333)
\curveto(134.56303528,163.96365772)(134.75803509,163.63865804)(134.9280423,163.23866333)
\curveto(134.97803487,163.12865855)(135.01303483,163.01865866)(135.0330423,162.90866333)
\curveto(135.05303479,162.79865888)(135.07803477,162.683659)(135.1080423,162.56366333)
\curveto(135.11803473,162.53365915)(135.12303472,162.48865919)(135.1230423,162.42866333)
\curveto(135.1430347,162.36865931)(135.15303469,162.29865938)(135.1530423,162.21866333)
\curveto(135.15303469,162.14865953)(135.16303468,162.0836596)(135.1830423,162.02366333)
\lineto(135.1830423,161.85866333)
\curveto(135.19303465,161.80865987)(135.19803465,161.73865994)(135.1980423,161.64866333)
\curveto(135.19803465,161.55866012)(135.18803466,161.48866019)(135.1680423,161.43866333)
\curveto(135.1480347,161.3786603)(135.1430347,161.31866036)(135.1530423,161.25866333)
\curveto(135.16303468,161.20866047)(135.15803469,161.15866052)(135.1380423,161.10866333)
\curveto(135.09803475,160.94866073)(135.06303478,160.79866088)(135.0330423,160.65866333)
\curveto(135.00303484,160.51866116)(134.95803489,160.3836613)(134.8980423,160.25366333)
\curveto(134.73803511,159.8836618)(134.51803533,159.54866213)(134.2380423,159.24866333)
\curveto(133.95803589,158.94866273)(133.63803621,158.71866296)(133.2780423,158.55866333)
\curveto(133.10803674,158.4786632)(132.90803694,158.40366328)(132.6780423,158.33366333)
\curveto(132.56803728,158.29366339)(132.45303739,158.26866341)(132.3330423,158.25866333)
\curveto(132.21303763,158.24866343)(132.09303775,158.22866345)(131.9730423,158.19866333)
\curveto(131.92303792,158.1786635)(131.86803798,158.1786635)(131.8080423,158.19866333)
\curveto(131.7480381,158.20866347)(131.68803816,158.20366348)(131.6280423,158.18366333)
\curveto(131.52803832,158.16366352)(131.42803842,158.16366352)(131.3280423,158.18366333)
\lineto(131.1930423,158.18366333)
\curveto(131.1430387,158.20366348)(131.08303876,158.21366347)(131.0130423,158.21366333)
\curveto(130.95303889,158.20366348)(130.89803895,158.20866347)(130.8480423,158.22866333)
\curveto(130.80803904,158.23866344)(130.77303907,158.24366344)(130.7430423,158.24366333)
\curveto(130.71303913,158.24366344)(130.67803917,158.24866343)(130.6380423,158.25866333)
\lineto(130.3680423,158.31866333)
\curveto(130.27803957,158.33866334)(130.19303965,158.36866331)(130.1130423,158.40866333)
\curveto(129.77304007,158.54866313)(129.48304036,158.70366298)(129.2430423,158.87366333)
\curveto(129.00304084,159.05366263)(128.78304106,159.2836624)(128.5830423,159.56366333)
\curveto(128.43304141,159.79366189)(128.31804153,160.03366165)(128.2380423,160.28366333)
\curveto(128.21804163,160.33366135)(128.20804164,160.3786613)(128.2080423,160.41866333)
\curveto(128.20804164,160.46866121)(128.19804165,160.51866116)(128.1780423,160.56866333)
\curveto(128.15804169,160.62866105)(128.1430417,160.70866097)(128.1330423,160.80866333)
\curveto(128.13304171,160.90866077)(128.15304169,160.9836607)(128.1930423,161.03366333)
\curveto(128.2430416,161.11366057)(128.32304152,161.15866052)(128.4330423,161.16866333)
\curveto(128.5430413,161.1786605)(128.65804119,161.1836605)(128.7780423,161.18366333)
\lineto(128.9430423,161.18366333)
\curveto(129.00304084,161.1836605)(129.05804079,161.17366051)(129.1080423,161.15366333)
\curveto(129.19804065,161.13366055)(129.26804058,161.09366059)(129.3180423,161.03366333)
\curveto(129.38804046,160.94366074)(129.43304041,160.83366085)(129.4530423,160.70366333)
\curveto(129.48304036,160.5836611)(129.52804032,160.4786612)(129.5880423,160.38866333)
\curveto(129.77804007,160.04866163)(130.03803981,159.7786619)(130.3680423,159.57866333)
\curveto(130.46803938,159.51866216)(130.57303927,159.46866221)(130.6830423,159.42866333)
\curveto(130.80303904,159.39866228)(130.92303892,159.36366232)(131.0430423,159.32366333)
\curveto(131.21303863,159.27366241)(131.41803843,159.25366243)(131.6580423,159.26366333)
\curveto(131.90803794,159.2836624)(132.10803774,159.31866236)(132.2580423,159.36866333)
\curveto(132.62803722,159.48866219)(132.91803693,159.64866203)(133.1280423,159.84866333)
\curveto(133.3480365,160.05866162)(133.52803632,160.33866134)(133.6680423,160.68866333)
\curveto(133.71803613,160.78866089)(133.7480361,160.89366079)(133.7580423,161.00366333)
\curveto(133.77803607,161.11366057)(133.80303604,161.22866045)(133.8330423,161.34866333)
\lineto(133.8330423,161.45366333)
\curveto(133.843036,161.49366019)(133.848036,161.53366015)(133.8480423,161.57366333)
\curveto(133.85803599,161.60366008)(133.85803599,161.63866004)(133.8480423,161.67866333)
\lineto(133.8480423,161.79866333)
\curveto(133.848036,162.05865962)(133.81803603,162.30365938)(133.7580423,162.53366333)
\curveto(133.6480362,162.8836588)(133.49303635,163.1786585)(133.2930423,163.41866333)
\curveto(133.09303675,163.66865801)(132.83303701,163.86365782)(132.5130423,164.00366333)
\lineto(132.3330423,164.06366333)
\curveto(132.28303756,164.0836576)(132.22303762,164.10365758)(132.1530423,164.12366333)
\curveto(132.10303774,164.14365754)(132.0430378,164.15365753)(131.9730423,164.15366333)
\curveto(131.91303793,164.16365752)(131.848038,164.1786575)(131.7780423,164.19866333)
\lineto(131.6280423,164.19866333)
\curveto(131.58803826,164.21865746)(131.53303831,164.22865745)(131.4630423,164.22866333)
\curveto(131.40303844,164.22865745)(131.3480385,164.21865746)(131.2980423,164.19866333)
\lineto(131.1930423,164.19866333)
\curveto(131.16303868,164.19865748)(131.12803872,164.19365749)(131.0880423,164.18366333)
\lineto(130.8480423,164.12366333)
\curveto(130.76803908,164.11365757)(130.68803916,164.09365759)(130.6080423,164.06366333)
\curveto(130.36803948,163.96365772)(130.13803971,163.82865785)(129.9180423,163.65866333)
\curveto(129.82804002,163.58865809)(129.7430401,163.51365817)(129.6630423,163.43366333)
\curveto(129.58304026,163.36365832)(129.48304036,163.30865837)(129.3630423,163.26866333)
\curveto(129.27304057,163.23865844)(129.13304071,163.22865845)(128.9430423,163.23866333)
\curveto(128.76304108,163.24865843)(128.6430412,163.27365841)(128.5830423,163.31366333)
\curveto(128.53304131,163.35365833)(128.49304135,163.41365827)(128.4630423,163.49366333)
\curveto(128.4430414,163.57365811)(128.4430414,163.65865802)(128.4630423,163.74866333)
\curveto(128.49304135,163.86865781)(128.51304133,163.98865769)(128.5230423,164.10866333)
\curveto(128.5430413,164.23865744)(128.56804128,164.36365732)(128.5980423,164.48366333)
\curveto(128.61804123,164.52365716)(128.62304122,164.55865712)(128.6130423,164.58866333)
\curveto(128.61304123,164.62865705)(128.62304122,164.67365701)(128.6430423,164.72366333)
\curveto(128.66304118,164.81365687)(128.67804117,164.90365678)(128.6880423,164.99366333)
\curveto(128.69804115,165.09365659)(128.71804113,165.18865649)(128.7480423,165.27866333)
\curveto(128.75804109,165.33865634)(128.76304108,165.39865628)(128.7630423,165.45866333)
\curveto(128.77304107,165.51865616)(128.78804106,165.5786561)(128.8080423,165.63866333)
\curveto(128.85804099,165.83865584)(128.89304095,166.04365564)(128.9130423,166.25366333)
\curveto(128.9430409,166.47365521)(128.98304086,166.683655)(129.0330423,166.88366333)
\curveto(129.06304078,166.9836547)(129.08304076,167.0836546)(129.0930423,167.18366333)
\curveto(129.10304074,167.2836544)(129.11804073,167.3836543)(129.1380423,167.48366333)
\curveto(129.1480407,167.51365417)(129.15304069,167.55365413)(129.1530423,167.60366333)
\curveto(129.18304066,167.71365397)(129.20304064,167.81865386)(129.2130423,167.91866333)
\curveto(129.23304061,168.02865365)(129.25804059,168.13865354)(129.2880423,168.24866333)
\curveto(129.30804054,168.32865335)(129.32304052,168.39865328)(129.3330423,168.45866333)
\curveto(129.3430405,168.52865315)(129.36804048,168.58865309)(129.4080423,168.63866333)
\curveto(129.42804042,168.66865301)(129.45804039,168.68865299)(129.4980423,168.69866333)
\curveto(129.53804031,168.71865296)(129.58304026,168.73865294)(129.6330423,168.75866333)
\curveto(129.69304015,168.75865292)(129.73304011,168.76365292)(129.7530423,168.77366333)
}
}
{
\newrgbcolor{curcolor}{0 0 0}
\pscustom[linestyle=none,fillstyle=solid,fillcolor=curcolor]
{
\newpath
\moveto(143.59265167,163.44866333)
\lineto(143.59265167,163.19366333)
\curveto(143.60264397,163.11365857)(143.59764397,163.03865864)(143.57765167,162.96866333)
\lineto(143.57765167,162.72866333)
\lineto(143.57765167,162.56366333)
\curveto(143.55764401,162.46365922)(143.54764402,162.35865932)(143.54765167,162.24866333)
\curveto(143.54764402,162.14865953)(143.53764403,162.04865963)(143.51765167,161.94866333)
\lineto(143.51765167,161.79866333)
\curveto(143.48764408,161.65866002)(143.4676441,161.51866016)(143.45765167,161.37866333)
\curveto(143.44764412,161.24866043)(143.42264415,161.11866056)(143.38265167,160.98866333)
\curveto(143.36264421,160.90866077)(143.34264423,160.82366086)(143.32265167,160.73366333)
\lineto(143.26265167,160.49366333)
\lineto(143.14265167,160.19366333)
\curveto(143.11264446,160.10366158)(143.07764449,160.01366167)(143.03765167,159.92366333)
\curveto(142.93764463,159.70366198)(142.80264477,159.48866219)(142.63265167,159.27866333)
\curveto(142.4726451,159.06866261)(142.29764527,158.89866278)(142.10765167,158.76866333)
\curveto(142.05764551,158.72866295)(141.99764557,158.68866299)(141.92765167,158.64866333)
\curveto(141.8676457,158.61866306)(141.80764576,158.5836631)(141.74765167,158.54366333)
\curveto(141.6676459,158.49366319)(141.572646,158.45366323)(141.46265167,158.42366333)
\curveto(141.35264622,158.39366329)(141.24764632,158.36366332)(141.14765167,158.33366333)
\curveto(141.03764653,158.29366339)(140.92764664,158.26866341)(140.81765167,158.25866333)
\curveto(140.70764686,158.24866343)(140.59264698,158.23366345)(140.47265167,158.21366333)
\curveto(140.43264714,158.20366348)(140.38764718,158.20366348)(140.33765167,158.21366333)
\curveto(140.29764727,158.21366347)(140.25764731,158.20866347)(140.21765167,158.19866333)
\curveto(140.17764739,158.18866349)(140.12264745,158.1836635)(140.05265167,158.18366333)
\curveto(139.98264759,158.1836635)(139.93264764,158.18866349)(139.90265167,158.19866333)
\curveto(139.85264772,158.21866346)(139.80764776,158.22366346)(139.76765167,158.21366333)
\curveto(139.72764784,158.20366348)(139.69264788,158.20366348)(139.66265167,158.21366333)
\lineto(139.57265167,158.21366333)
\curveto(139.51264806,158.23366345)(139.44764812,158.24866343)(139.37765167,158.25866333)
\curveto(139.31764825,158.25866342)(139.25264832,158.26366342)(139.18265167,158.27366333)
\curveto(139.01264856,158.32366336)(138.85264872,158.37366331)(138.70265167,158.42366333)
\curveto(138.55264902,158.47366321)(138.40764916,158.53866314)(138.26765167,158.61866333)
\curveto(138.21764935,158.65866302)(138.16264941,158.68866299)(138.10265167,158.70866333)
\curveto(138.05264952,158.73866294)(138.00264957,158.77366291)(137.95265167,158.81366333)
\curveto(137.71264986,158.99366269)(137.51265006,159.21366247)(137.35265167,159.47366333)
\curveto(137.19265038,159.73366195)(137.05265052,160.01866166)(136.93265167,160.32866333)
\curveto(136.8726507,160.46866121)(136.82765074,160.60866107)(136.79765167,160.74866333)
\curveto(136.7676508,160.89866078)(136.73265084,161.05366063)(136.69265167,161.21366333)
\curveto(136.6726509,161.32366036)(136.65765091,161.43366025)(136.64765167,161.54366333)
\curveto(136.63765093,161.65366003)(136.62265095,161.76365992)(136.60265167,161.87366333)
\curveto(136.59265098,161.91365977)(136.58765098,161.95365973)(136.58765167,161.99366333)
\curveto(136.59765097,162.03365965)(136.59765097,162.07365961)(136.58765167,162.11366333)
\curveto(136.57765099,162.16365952)(136.572651,162.21365947)(136.57265167,162.26366333)
\lineto(136.57265167,162.42866333)
\curveto(136.55265102,162.4786592)(136.54765102,162.52865915)(136.55765167,162.57866333)
\curveto(136.567651,162.63865904)(136.567651,162.69365899)(136.55765167,162.74366333)
\curveto(136.54765102,162.7836589)(136.54765102,162.82865885)(136.55765167,162.87866333)
\curveto(136.567651,162.92865875)(136.56265101,162.9786587)(136.54265167,163.02866333)
\curveto(136.52265105,163.09865858)(136.51765105,163.17365851)(136.52765167,163.25366333)
\curveto(136.53765103,163.34365834)(136.54265103,163.42865825)(136.54265167,163.50866333)
\curveto(136.54265103,163.59865808)(136.53765103,163.69865798)(136.52765167,163.80866333)
\curveto(136.51765105,163.92865775)(136.52265105,164.02865765)(136.54265167,164.10866333)
\lineto(136.54265167,164.39366333)
\lineto(136.58765167,165.02366333)
\curveto(136.59765097,165.12365656)(136.60765096,165.21865646)(136.61765167,165.30866333)
\lineto(136.64765167,165.60866333)
\curveto(136.6676509,165.65865602)(136.6726509,165.70865597)(136.66265167,165.75866333)
\curveto(136.66265091,165.81865586)(136.6726509,165.87365581)(136.69265167,165.92366333)
\curveto(136.74265083,166.09365559)(136.78265079,166.25865542)(136.81265167,166.41866333)
\curveto(136.84265073,166.58865509)(136.89265068,166.74865493)(136.96265167,166.89866333)
\curveto(137.15265042,167.35865432)(137.3726502,167.73365395)(137.62265167,168.02366333)
\curveto(137.88264969,168.31365337)(138.24264933,168.55865312)(138.70265167,168.75866333)
\curveto(138.83264874,168.80865287)(138.96264861,168.84365284)(139.09265167,168.86366333)
\curveto(139.23264834,168.8836528)(139.3726482,168.90865277)(139.51265167,168.93866333)
\curveto(139.58264799,168.94865273)(139.64764792,168.95365273)(139.70765167,168.95366333)
\curveto(139.7676478,168.95365273)(139.83264774,168.95865272)(139.90265167,168.96866333)
\curveto(140.73264684,168.98865269)(141.40264617,168.83865284)(141.91265167,168.51866333)
\curveto(142.42264515,168.20865347)(142.80264477,167.76865391)(143.05265167,167.19866333)
\curveto(143.10264447,167.0786546)(143.14764442,166.95365473)(143.18765167,166.82366333)
\curveto(143.22764434,166.69365499)(143.2726443,166.55865512)(143.32265167,166.41866333)
\curveto(143.34264423,166.33865534)(143.35764421,166.25365543)(143.36765167,166.16366333)
\lineto(143.42765167,165.92366333)
\curveto(143.45764411,165.81365587)(143.4726441,165.70365598)(143.47265167,165.59366333)
\curveto(143.48264409,165.4836562)(143.49764407,165.37365631)(143.51765167,165.26366333)
\curveto(143.53764403,165.21365647)(143.54264403,165.16865651)(143.53265167,165.12866333)
\curveto(143.53264404,165.08865659)(143.53764403,165.04865663)(143.54765167,165.00866333)
\curveto(143.55764401,164.95865672)(143.55764401,164.90365678)(143.54765167,164.84366333)
\curveto(143.54764402,164.79365689)(143.55264402,164.74365694)(143.56265167,164.69366333)
\lineto(143.56265167,164.55866333)
\curveto(143.58264399,164.49865718)(143.58264399,164.42865725)(143.56265167,164.34866333)
\curveto(143.55264402,164.2786574)(143.55764401,164.21365747)(143.57765167,164.15366333)
\curveto(143.58764398,164.12365756)(143.59264398,164.0836576)(143.59265167,164.03366333)
\lineto(143.59265167,163.91366333)
\lineto(143.59265167,163.44866333)
\moveto(142.04765167,161.12366333)
\curveto(142.14764542,161.44366024)(142.20764536,161.80865987)(142.22765167,162.21866333)
\curveto(142.24764532,162.62865905)(142.25764531,163.03865864)(142.25765167,163.44866333)
\curveto(142.25764531,163.8786578)(142.24764532,164.29865738)(142.22765167,164.70866333)
\curveto(142.20764536,165.11865656)(142.16264541,165.50365618)(142.09265167,165.86366333)
\curveto(142.02264555,166.22365546)(141.91264566,166.54365514)(141.76265167,166.82366333)
\curveto(141.62264595,167.11365457)(141.42764614,167.34865433)(141.17765167,167.52866333)
\curveto(141.01764655,167.63865404)(140.83764673,167.71865396)(140.63765167,167.76866333)
\curveto(140.43764713,167.82865385)(140.19264738,167.85865382)(139.90265167,167.85866333)
\curveto(139.88264769,167.83865384)(139.84764772,167.82865385)(139.79765167,167.82866333)
\curveto(139.74764782,167.83865384)(139.70764786,167.83865384)(139.67765167,167.82866333)
\curveto(139.59764797,167.80865387)(139.52264805,167.78865389)(139.45265167,167.76866333)
\curveto(139.39264818,167.75865392)(139.32764824,167.73865394)(139.25765167,167.70866333)
\curveto(138.98764858,167.58865409)(138.7676488,167.41865426)(138.59765167,167.19866333)
\curveto(138.43764913,166.98865469)(138.30264927,166.74365494)(138.19265167,166.46366333)
\curveto(138.14264943,166.35365533)(138.10264947,166.23365545)(138.07265167,166.10366333)
\curveto(138.05264952,165.9836557)(138.02764954,165.85865582)(137.99765167,165.72866333)
\curveto(137.97764959,165.678656)(137.9676496,165.62365606)(137.96765167,165.56366333)
\curveto(137.9676496,165.51365617)(137.96264961,165.46365622)(137.95265167,165.41366333)
\curveto(137.94264963,165.32365636)(137.93264964,165.22865645)(137.92265167,165.12866333)
\curveto(137.91264966,165.03865664)(137.90264967,164.94365674)(137.89265167,164.84366333)
\curveto(137.89264968,164.76365692)(137.88764968,164.678657)(137.87765167,164.58866333)
\lineto(137.87765167,164.34866333)
\lineto(137.87765167,164.16866333)
\curveto(137.8676497,164.13865754)(137.86264971,164.10365758)(137.86265167,164.06366333)
\lineto(137.86265167,163.92866333)
\lineto(137.86265167,163.47866333)
\curveto(137.86264971,163.39865828)(137.85764971,163.31365837)(137.84765167,163.22366333)
\curveto(137.84764972,163.14365854)(137.85764971,163.06865861)(137.87765167,162.99866333)
\lineto(137.87765167,162.72866333)
\curveto(137.87764969,162.70865897)(137.8726497,162.678659)(137.86265167,162.63866333)
\curveto(137.86264971,162.60865907)(137.8676497,162.5836591)(137.87765167,162.56366333)
\curveto(137.88764968,162.46365922)(137.89264968,162.36365932)(137.89265167,162.26366333)
\curveto(137.90264967,162.17365951)(137.91264966,162.07365961)(137.92265167,161.96366333)
\curveto(137.95264962,161.84365984)(137.9676496,161.71865996)(137.96765167,161.58866333)
\curveto(137.97764959,161.46866021)(138.00264957,161.35366033)(138.04265167,161.24366333)
\curveto(138.12264945,160.94366074)(138.20764936,160.678661)(138.29765167,160.44866333)
\curveto(138.39764917,160.21866146)(138.54264903,160.00366168)(138.73265167,159.80366333)
\curveto(138.94264863,159.60366208)(139.20764836,159.45366223)(139.52765167,159.35366333)
\curveto(139.567648,159.33366235)(139.60264797,159.32366236)(139.63265167,159.32366333)
\curveto(139.6726479,159.33366235)(139.71764785,159.32866235)(139.76765167,159.30866333)
\curveto(139.80764776,159.29866238)(139.87764769,159.28866239)(139.97765167,159.27866333)
\curveto(140.08764748,159.26866241)(140.1726474,159.27366241)(140.23265167,159.29366333)
\curveto(140.30264727,159.31366237)(140.3726472,159.32366236)(140.44265167,159.32366333)
\curveto(140.51264706,159.33366235)(140.57764699,159.34866233)(140.63765167,159.36866333)
\curveto(140.83764673,159.42866225)(141.01764655,159.51366217)(141.17765167,159.62366333)
\curveto(141.20764636,159.64366204)(141.23264634,159.66366202)(141.25265167,159.68366333)
\lineto(141.31265167,159.74366333)
\curveto(141.35264622,159.76366192)(141.40264617,159.80366188)(141.46265167,159.86366333)
\curveto(141.56264601,160.00366168)(141.64764592,160.13366155)(141.71765167,160.25366333)
\curveto(141.78764578,160.37366131)(141.85764571,160.51866116)(141.92765167,160.68866333)
\curveto(141.95764561,160.75866092)(141.97764559,160.82866085)(141.98765167,160.89866333)
\curveto(142.00764556,160.96866071)(142.02764554,161.04366064)(142.04765167,161.12366333)
}
}
{
\newrgbcolor{curcolor}{0 0 0}
\pscustom[linestyle=none,fillstyle=solid,fillcolor=curcolor]
{
\newpath
\moveto(123.00843292,242.89723694)
\curveto(124.63842748,242.92722629)(125.68842643,242.37222684)(126.15843292,241.23223694)
\curveto(126.25842586,241.00222821)(126.3234258,240.7122285)(126.35343292,240.36223694)
\curveto(126.39342573,240.02222919)(126.36842575,239.7122295)(126.27843292,239.43223694)
\curveto(126.18842593,239.17223004)(126.06842605,238.94723027)(125.91843292,238.75723694)
\curveto(125.89842622,238.7172305)(125.87342625,238.68223053)(125.84343292,238.65223694)
\curveto(125.81342631,238.63223058)(125.78842633,238.60723061)(125.76843292,238.57723694)
\lineto(125.67843292,238.45723694)
\curveto(125.64842647,238.42723079)(125.61342651,238.40223081)(125.57343292,238.38223694)
\curveto(125.5234266,238.33223088)(125.46842665,238.28723093)(125.40843292,238.24723694)
\curveto(125.35842676,238.20723101)(125.31342681,238.15723106)(125.27343292,238.09723694)
\curveto(125.23342689,238.05723116)(125.2184269,238.00723121)(125.22843292,237.94723694)
\curveto(125.23842688,237.89723132)(125.26842685,237.85223136)(125.31843292,237.81223694)
\curveto(125.36842675,237.77223144)(125.4234267,237.73223148)(125.48343292,237.69223694)
\curveto(125.55342657,237.66223155)(125.6184265,237.63223158)(125.67843292,237.60223694)
\curveto(125.73842638,237.57223164)(125.78842633,237.54223167)(125.82843292,237.51223694)
\curveto(126.14842597,237.29223192)(126.40342572,236.98223223)(126.59343292,236.58223694)
\curveto(126.63342549,236.49223272)(126.66342546,236.39723282)(126.68343292,236.29723694)
\curveto(126.71342541,236.20723301)(126.73842538,236.1172331)(126.75843292,236.02723694)
\curveto(126.76842535,235.97723324)(126.77342535,235.92723329)(126.77343292,235.87723694)
\curveto(126.78342534,235.83723338)(126.79342533,235.79223342)(126.80343292,235.74223694)
\curveto(126.81342531,235.69223352)(126.81342531,235.64223357)(126.80343292,235.59223694)
\curveto(126.79342533,235.54223367)(126.79842532,235.49223372)(126.81843292,235.44223694)
\curveto(126.82842529,235.39223382)(126.83342529,235.33223388)(126.83343292,235.26223694)
\curveto(126.83342529,235.19223402)(126.8234253,235.13223408)(126.80343292,235.08223694)
\lineto(126.80343292,234.85723694)
\lineto(126.74343292,234.61723694)
\curveto(126.73342539,234.54723467)(126.7184254,234.47723474)(126.69843292,234.40723694)
\curveto(126.66842545,234.3172349)(126.63842548,234.23223498)(126.60843292,234.15223694)
\curveto(126.58842553,234.07223514)(126.55842556,233.99223522)(126.51843292,233.91223694)
\curveto(126.49842562,233.85223536)(126.46842565,233.79223542)(126.42843292,233.73223694)
\curveto(126.39842572,233.68223553)(126.36342576,233.63223558)(126.32343292,233.58223694)
\curveto(126.123426,233.27223594)(125.87342625,233.0122362)(125.57343292,232.80223694)
\curveto(125.27342685,232.60223661)(124.92842719,232.43723678)(124.53843292,232.30723694)
\curveto(124.4184277,232.26723695)(124.28842783,232.24223697)(124.14843292,232.23223694)
\curveto(124.0184281,232.212237)(123.88342824,232.18723703)(123.74343292,232.15723694)
\curveto(123.67342845,232.14723707)(123.60342852,232.14223707)(123.53343292,232.14223694)
\curveto(123.47342865,232.14223707)(123.40842871,232.13723708)(123.33843292,232.12723694)
\curveto(123.29842882,232.1172371)(123.23842888,232.1122371)(123.15843292,232.11223694)
\curveto(123.08842903,232.1122371)(123.03842908,232.1172371)(123.00843292,232.12723694)
\curveto(122.95842916,232.13723708)(122.91342921,232.14223707)(122.87343292,232.14223694)
\lineto(122.75343292,232.14223694)
\curveto(122.65342947,232.16223705)(122.55342957,232.17723704)(122.45343292,232.18723694)
\curveto(122.35342977,232.19723702)(122.25842986,232.212237)(122.16843292,232.23223694)
\curveto(122.05843006,232.26223695)(121.94843017,232.28723693)(121.83843292,232.30723694)
\curveto(121.73843038,232.33723688)(121.63343049,232.37723684)(121.52343292,232.42723694)
\curveto(121.15343097,232.58723663)(120.83843128,232.78723643)(120.57843292,233.02723694)
\curveto(120.3184318,233.27723594)(120.10843201,233.58723563)(119.94843292,233.95723694)
\curveto(119.90843221,234.04723517)(119.87343225,234.14223507)(119.84343292,234.24223694)
\curveto(119.81343231,234.34223487)(119.78343234,234.44723477)(119.75343292,234.55723694)
\curveto(119.73343239,234.60723461)(119.7234324,234.65723456)(119.72343292,234.70723694)
\curveto(119.7234324,234.76723445)(119.71343241,234.82723439)(119.69343292,234.88723694)
\curveto(119.67343245,234.94723427)(119.66343246,235.02723419)(119.66343292,235.12723694)
\curveto(119.66343246,235.22723399)(119.67843244,235.30223391)(119.70843292,235.35223694)
\curveto(119.7184324,235.38223383)(119.73343239,235.40723381)(119.75343292,235.42723694)
\lineto(119.81343292,235.48723694)
\curveto(119.85343227,235.50723371)(119.91343221,235.52223369)(119.99343292,235.53223694)
\curveto(120.08343204,235.54223367)(120.17343195,235.54723367)(120.26343292,235.54723694)
\curveto(120.35343177,235.54723367)(120.43843168,235.54223367)(120.51843292,235.53223694)
\curveto(120.60843151,235.52223369)(120.67343145,235.5122337)(120.71343292,235.50223694)
\curveto(120.73343139,235.48223373)(120.75343137,235.46723375)(120.77343292,235.45723694)
\curveto(120.79343133,235.45723376)(120.81343131,235.44723377)(120.83343292,235.42723694)
\curveto(120.90343122,235.33723388)(120.94343118,235.22223399)(120.95343292,235.08223694)
\curveto(120.97343115,234.94223427)(121.00343112,234.8172344)(121.04343292,234.70723694)
\lineto(121.19343292,234.34723694)
\curveto(121.24343088,234.23723498)(121.30843081,234.13223508)(121.38843292,234.03223694)
\curveto(121.40843071,234.00223521)(121.42843069,233.97723524)(121.44843292,233.95723694)
\curveto(121.47843064,233.93723528)(121.50343062,233.9122353)(121.52343292,233.88223694)
\curveto(121.56343056,233.82223539)(121.59843052,233.77723544)(121.62843292,233.74723694)
\curveto(121.66843045,233.7172355)(121.70343042,233.68723553)(121.73343292,233.65723694)
\curveto(121.77343035,233.62723559)(121.8184303,233.59723562)(121.86843292,233.56723694)
\curveto(121.95843016,233.50723571)(122.05343007,233.45723576)(122.15343292,233.41723694)
\lineto(122.48343292,233.29723694)
\curveto(122.63342949,233.24723597)(122.83342929,233.217236)(123.08343292,233.20723694)
\curveto(123.33342879,233.19723602)(123.54342858,233.217236)(123.71343292,233.26723694)
\curveto(123.79342833,233.28723593)(123.86342826,233.30223591)(123.92343292,233.31223694)
\lineto(124.13343292,233.37223694)
\curveto(124.41342771,233.49223572)(124.65342747,233.64223557)(124.85343292,233.82223694)
\curveto(125.06342706,234.00223521)(125.22842689,234.23223498)(125.34843292,234.51223694)
\curveto(125.37842674,234.58223463)(125.39842672,234.65223456)(125.40843292,234.72223694)
\lineto(125.46843292,234.96223694)
\curveto(125.50842661,235.10223411)(125.5184266,235.26223395)(125.49843292,235.44223694)
\curveto(125.47842664,235.63223358)(125.44842667,235.78223343)(125.40843292,235.89223694)
\curveto(125.27842684,236.27223294)(125.09342703,236.56223265)(124.85343292,236.76223694)
\curveto(124.6234275,236.96223225)(124.31342781,237.12223209)(123.92343292,237.24223694)
\curveto(123.81342831,237.27223194)(123.69342843,237.29223192)(123.56343292,237.30223694)
\curveto(123.44342868,237.3122319)(123.3184288,237.3172319)(123.18843292,237.31723694)
\curveto(123.02842909,237.3172319)(122.88842923,237.32223189)(122.76843292,237.33223694)
\curveto(122.64842947,237.34223187)(122.56342956,237.40223181)(122.51343292,237.51223694)
\curveto(122.49342963,237.54223167)(122.48342964,237.57723164)(122.48343292,237.61723694)
\lineto(122.48343292,237.75223694)
\curveto(122.47342965,237.85223136)(122.47342965,237.94723127)(122.48343292,238.03723694)
\curveto(122.50342962,238.12723109)(122.54342958,238.19223102)(122.60343292,238.23223694)
\curveto(122.64342948,238.26223095)(122.68342944,238.28223093)(122.72343292,238.29223694)
\curveto(122.77342935,238.30223091)(122.82842929,238.3122309)(122.88843292,238.32223694)
\curveto(122.90842921,238.33223088)(122.93342919,238.33223088)(122.96343292,238.32223694)
\curveto(122.99342913,238.32223089)(123.0184291,238.32723089)(123.03843292,238.33723694)
\lineto(123.17343292,238.33723694)
\curveto(123.28342884,238.35723086)(123.38342874,238.36723085)(123.47343292,238.36723694)
\curveto(123.57342855,238.37723084)(123.66842845,238.39723082)(123.75843292,238.42723694)
\curveto(124.07842804,238.53723068)(124.33342779,238.68223053)(124.52343292,238.86223694)
\curveto(124.71342741,239.04223017)(124.86342726,239.29222992)(124.97343292,239.61223694)
\curveto(125.00342712,239.7122295)(125.0234271,239.83722938)(125.03343292,239.98723694)
\curveto(125.05342707,240.14722907)(125.04842707,240.29222892)(125.01843292,240.42223694)
\curveto(124.99842712,240.49222872)(124.97842714,240.55722866)(124.95843292,240.61723694)
\curveto(124.94842717,240.68722853)(124.92842719,240.75222846)(124.89843292,240.81223694)
\curveto(124.79842732,241.05222816)(124.65342747,241.24222797)(124.46343292,241.38223694)
\curveto(124.27342785,241.52222769)(124.04842807,241.63222758)(123.78843292,241.71223694)
\curveto(123.72842839,241.73222748)(123.66842845,241.74222747)(123.60843292,241.74223694)
\curveto(123.54842857,241.74222747)(123.48342864,241.75222746)(123.41343292,241.77223694)
\curveto(123.33342879,241.79222742)(123.23842888,241.80222741)(123.12843292,241.80223694)
\curveto(123.0184291,241.80222741)(122.9234292,241.79222742)(122.84343292,241.77223694)
\curveto(122.79342933,241.75222746)(122.74342938,241.74222747)(122.69343292,241.74223694)
\curveto(122.65342947,241.74222747)(122.60842951,241.73222748)(122.55843292,241.71223694)
\curveto(122.37842974,241.66222755)(122.20842991,241.58722763)(122.04843292,241.48723694)
\curveto(121.89843022,241.39722782)(121.76843035,241.28222793)(121.65843292,241.14223694)
\curveto(121.56843055,241.02222819)(121.48843063,240.89222832)(121.41843292,240.75223694)
\curveto(121.34843077,240.6122286)(121.28343084,240.45722876)(121.22343292,240.28723694)
\curveto(121.19343093,240.17722904)(121.17343095,240.05722916)(121.16343292,239.92723694)
\curveto(121.15343097,239.80722941)(121.118431,239.70722951)(121.05843292,239.62723694)
\curveto(121.03843108,239.58722963)(120.97843114,239.54722967)(120.87843292,239.50723694)
\curveto(120.83843128,239.49722972)(120.77843134,239.48722973)(120.69843292,239.47723694)
\lineto(120.44343292,239.47723694)
\curveto(120.35343177,239.48722973)(120.26843185,239.49722972)(120.18843292,239.50723694)
\curveto(120.118432,239.5172297)(120.06843205,239.53222968)(120.03843292,239.55223694)
\curveto(119.99843212,239.58222963)(119.96343216,239.63722958)(119.93343292,239.71723694)
\curveto(119.90343222,239.79722942)(119.89843222,239.88222933)(119.91843292,239.97223694)
\curveto(119.92843219,240.02222919)(119.93343219,240.07222914)(119.93343292,240.12223694)
\lineto(119.96343292,240.30223694)
\curveto(119.99343213,240.40222881)(120.0184321,240.50222871)(120.03843292,240.60223694)
\curveto(120.06843205,240.70222851)(120.10343202,240.79222842)(120.14343292,240.87223694)
\curveto(120.19343193,240.98222823)(120.23843188,241.08722813)(120.27843292,241.18723694)
\curveto(120.3184318,241.29722792)(120.36843175,241.40222781)(120.42843292,241.50223694)
\curveto(120.75843136,242.04222717)(121.22843089,242.43722678)(121.83843292,242.68723694)
\curveto(121.95843016,242.73722648)(122.08343004,242.77222644)(122.21343292,242.79223694)
\curveto(122.35342977,242.8122264)(122.49342963,242.83722638)(122.63343292,242.86723694)
\curveto(122.69342943,242.87722634)(122.75342937,242.88222633)(122.81343292,242.88223694)
\curveto(122.88342924,242.88222633)(122.94842917,242.88722633)(123.00843292,242.89723694)
}
}
{
\newrgbcolor{curcolor}{0 0 0}
\pscustom[linestyle=none,fillstyle=solid,fillcolor=curcolor]
{
\newpath
\moveto(135.2430423,237.37723694)
\lineto(135.2430423,237.12223694)
\curveto(135.25303459,237.04223217)(135.2480346,236.96723225)(135.2280423,236.89723694)
\lineto(135.2280423,236.65723694)
\lineto(135.2280423,236.49223694)
\curveto(135.20803464,236.39223282)(135.19803465,236.28723293)(135.1980423,236.17723694)
\curveto(135.19803465,236.07723314)(135.18803466,235.97723324)(135.1680423,235.87723694)
\lineto(135.1680423,235.72723694)
\curveto(135.13803471,235.58723363)(135.11803473,235.44723377)(135.1080423,235.30723694)
\curveto(135.09803475,235.17723404)(135.07303477,235.04723417)(135.0330423,234.91723694)
\curveto(135.01303483,234.83723438)(134.99303485,234.75223446)(134.9730423,234.66223694)
\lineto(134.9130423,234.42223694)
\lineto(134.7930423,234.12223694)
\curveto(134.76303508,234.03223518)(134.72803512,233.94223527)(134.6880423,233.85223694)
\curveto(134.58803526,233.63223558)(134.45303539,233.4172358)(134.2830423,233.20723694)
\curveto(134.12303572,232.99723622)(133.9480359,232.82723639)(133.7580423,232.69723694)
\curveto(133.70803614,232.65723656)(133.6480362,232.6172366)(133.5780423,232.57723694)
\curveto(133.51803633,232.54723667)(133.45803639,232.5122367)(133.3980423,232.47223694)
\curveto(133.31803653,232.42223679)(133.22303662,232.38223683)(133.1130423,232.35223694)
\curveto(133.00303684,232.32223689)(132.89803695,232.29223692)(132.7980423,232.26223694)
\curveto(132.68803716,232.22223699)(132.57803727,232.19723702)(132.4680423,232.18723694)
\curveto(132.35803749,232.17723704)(132.2430376,232.16223705)(132.1230423,232.14223694)
\curveto(132.08303776,232.13223708)(132.03803781,232.13223708)(131.9880423,232.14223694)
\curveto(131.9480379,232.14223707)(131.90803794,232.13723708)(131.8680423,232.12723694)
\curveto(131.82803802,232.1172371)(131.77303807,232.1122371)(131.7030423,232.11223694)
\curveto(131.63303821,232.1122371)(131.58303826,232.1172371)(131.5530423,232.12723694)
\curveto(131.50303834,232.14723707)(131.45803839,232.15223706)(131.4180423,232.14223694)
\curveto(131.37803847,232.13223708)(131.3430385,232.13223708)(131.3130423,232.14223694)
\lineto(131.2230423,232.14223694)
\curveto(131.16303868,232.16223705)(131.09803875,232.17723704)(131.0280423,232.18723694)
\curveto(130.96803888,232.18723703)(130.90303894,232.19223702)(130.8330423,232.20223694)
\curveto(130.66303918,232.25223696)(130.50303934,232.30223691)(130.3530423,232.35223694)
\curveto(130.20303964,232.40223681)(130.05803979,232.46723675)(129.9180423,232.54723694)
\curveto(129.86803998,232.58723663)(129.81304003,232.6172366)(129.7530423,232.63723694)
\curveto(129.70304014,232.66723655)(129.65304019,232.70223651)(129.6030423,232.74223694)
\curveto(129.36304048,232.92223629)(129.16304068,233.14223607)(129.0030423,233.40223694)
\curveto(128.843041,233.66223555)(128.70304114,233.94723527)(128.5830423,234.25723694)
\curveto(128.52304132,234.39723482)(128.47804137,234.53723468)(128.4480423,234.67723694)
\curveto(128.41804143,234.82723439)(128.38304146,234.98223423)(128.3430423,235.14223694)
\curveto(128.32304152,235.25223396)(128.30804154,235.36223385)(128.2980423,235.47223694)
\curveto(128.28804156,235.58223363)(128.27304157,235.69223352)(128.2530423,235.80223694)
\curveto(128.2430416,235.84223337)(128.23804161,235.88223333)(128.2380423,235.92223694)
\curveto(128.2480416,235.96223325)(128.2480416,236.00223321)(128.2380423,236.04223694)
\curveto(128.22804162,236.09223312)(128.22304162,236.14223307)(128.2230423,236.19223694)
\lineto(128.2230423,236.35723694)
\curveto(128.20304164,236.40723281)(128.19804165,236.45723276)(128.2080423,236.50723694)
\curveto(128.21804163,236.56723265)(128.21804163,236.62223259)(128.2080423,236.67223694)
\curveto(128.19804165,236.7122325)(128.19804165,236.75723246)(128.2080423,236.80723694)
\curveto(128.21804163,236.85723236)(128.21304163,236.90723231)(128.1930423,236.95723694)
\curveto(128.17304167,237.02723219)(128.16804168,237.10223211)(128.1780423,237.18223694)
\curveto(128.18804166,237.27223194)(128.19304165,237.35723186)(128.1930423,237.43723694)
\curveto(128.19304165,237.52723169)(128.18804166,237.62723159)(128.1780423,237.73723694)
\curveto(128.16804168,237.85723136)(128.17304167,237.95723126)(128.1930423,238.03723694)
\lineto(128.1930423,238.32223694)
\lineto(128.2380423,238.95223694)
\curveto(128.2480416,239.05223016)(128.25804159,239.14723007)(128.2680423,239.23723694)
\lineto(128.2980423,239.53723694)
\curveto(128.31804153,239.58722963)(128.32304152,239.63722958)(128.3130423,239.68723694)
\curveto(128.31304153,239.74722947)(128.32304152,239.80222941)(128.3430423,239.85223694)
\curveto(128.39304145,240.02222919)(128.43304141,240.18722903)(128.4630423,240.34723694)
\curveto(128.49304135,240.5172287)(128.5430413,240.67722854)(128.6130423,240.82723694)
\curveto(128.80304104,241.28722793)(129.02304082,241.66222755)(129.2730423,241.95223694)
\curveto(129.53304031,242.24222697)(129.89303995,242.48722673)(130.3530423,242.68723694)
\curveto(130.48303936,242.73722648)(130.61303923,242.77222644)(130.7430423,242.79223694)
\curveto(130.88303896,242.8122264)(131.02303882,242.83722638)(131.1630423,242.86723694)
\curveto(131.23303861,242.87722634)(131.29803855,242.88222633)(131.3580423,242.88223694)
\curveto(131.41803843,242.88222633)(131.48303836,242.88722633)(131.5530423,242.89723694)
\curveto(132.38303746,242.9172263)(133.05303679,242.76722645)(133.5630423,242.44723694)
\curveto(134.07303577,242.13722708)(134.45303539,241.69722752)(134.7030423,241.12723694)
\curveto(134.75303509,241.00722821)(134.79803505,240.88222833)(134.8380423,240.75223694)
\curveto(134.87803497,240.62222859)(134.92303492,240.48722873)(134.9730423,240.34723694)
\curveto(134.99303485,240.26722895)(135.00803484,240.18222903)(135.0180423,240.09223694)
\lineto(135.0780423,239.85223694)
\curveto(135.10803474,239.74222947)(135.12303472,239.63222958)(135.1230423,239.52223694)
\curveto(135.13303471,239.4122298)(135.1480347,239.30222991)(135.1680423,239.19223694)
\curveto(135.18803466,239.14223007)(135.19303465,239.09723012)(135.1830423,239.05723694)
\curveto(135.18303466,239.0172302)(135.18803466,238.97723024)(135.1980423,238.93723694)
\curveto(135.20803464,238.88723033)(135.20803464,238.83223038)(135.1980423,238.77223694)
\curveto(135.19803465,238.72223049)(135.20303464,238.67223054)(135.2130423,238.62223694)
\lineto(135.2130423,238.48723694)
\curveto(135.23303461,238.42723079)(135.23303461,238.35723086)(135.2130423,238.27723694)
\curveto(135.20303464,238.20723101)(135.20803464,238.14223107)(135.2280423,238.08223694)
\curveto(135.23803461,238.05223116)(135.2430346,238.0122312)(135.2430423,237.96223694)
\lineto(135.2430423,237.84223694)
\lineto(135.2430423,237.37723694)
\moveto(133.6980423,235.05223694)
\curveto(133.79803605,235.37223384)(133.85803599,235.73723348)(133.8780423,236.14723694)
\curveto(133.89803595,236.55723266)(133.90803594,236.96723225)(133.9080423,237.37723694)
\curveto(133.90803594,237.80723141)(133.89803595,238.22723099)(133.8780423,238.63723694)
\curveto(133.85803599,239.04723017)(133.81303603,239.43222978)(133.7430423,239.79223694)
\curveto(133.67303617,240.15222906)(133.56303628,240.47222874)(133.4130423,240.75223694)
\curveto(133.27303657,241.04222817)(133.07803677,241.27722794)(132.8280423,241.45723694)
\curveto(132.66803718,241.56722765)(132.48803736,241.64722757)(132.2880423,241.69723694)
\curveto(132.08803776,241.75722746)(131.843038,241.78722743)(131.5530423,241.78723694)
\curveto(131.53303831,241.76722745)(131.49803835,241.75722746)(131.4480423,241.75723694)
\curveto(131.39803845,241.76722745)(131.35803849,241.76722745)(131.3280423,241.75723694)
\curveto(131.2480386,241.73722748)(131.17303867,241.7172275)(131.1030423,241.69723694)
\curveto(131.0430388,241.68722753)(130.97803887,241.66722755)(130.9080423,241.63723694)
\curveto(130.63803921,241.5172277)(130.41803943,241.34722787)(130.2480423,241.12723694)
\curveto(130.08803976,240.9172283)(129.95303989,240.67222854)(129.8430423,240.39223694)
\curveto(129.79304005,240.28222893)(129.75304009,240.16222905)(129.7230423,240.03223694)
\curveto(129.70304014,239.9122293)(129.67804017,239.78722943)(129.6480423,239.65723694)
\curveto(129.62804022,239.60722961)(129.61804023,239.55222966)(129.6180423,239.49223694)
\curveto(129.61804023,239.44222977)(129.61304023,239.39222982)(129.6030423,239.34223694)
\curveto(129.59304025,239.25222996)(129.58304026,239.15723006)(129.5730423,239.05723694)
\curveto(129.56304028,238.96723025)(129.55304029,238.87223034)(129.5430423,238.77223694)
\curveto(129.5430403,238.69223052)(129.53804031,238.60723061)(129.5280423,238.51723694)
\lineto(129.5280423,238.27723694)
\lineto(129.5280423,238.09723694)
\curveto(129.51804033,238.06723115)(129.51304033,238.03223118)(129.5130423,237.99223694)
\lineto(129.5130423,237.85723694)
\lineto(129.5130423,237.40723694)
\curveto(129.51304033,237.32723189)(129.50804034,237.24223197)(129.4980423,237.15223694)
\curveto(129.49804035,237.07223214)(129.50804034,236.99723222)(129.5280423,236.92723694)
\lineto(129.5280423,236.65723694)
\curveto(129.52804032,236.63723258)(129.52304032,236.60723261)(129.5130423,236.56723694)
\curveto(129.51304033,236.53723268)(129.51804033,236.5122327)(129.5280423,236.49223694)
\curveto(129.53804031,236.39223282)(129.5430403,236.29223292)(129.5430423,236.19223694)
\curveto(129.55304029,236.10223311)(129.56304028,236.00223321)(129.5730423,235.89223694)
\curveto(129.60304024,235.77223344)(129.61804023,235.64723357)(129.6180423,235.51723694)
\curveto(129.62804022,235.39723382)(129.65304019,235.28223393)(129.6930423,235.17223694)
\curveto(129.77304007,234.87223434)(129.85803999,234.60723461)(129.9480423,234.37723694)
\curveto(130.0480398,234.14723507)(130.19303965,233.93223528)(130.3830423,233.73223694)
\curveto(130.59303925,233.53223568)(130.85803899,233.38223583)(131.1780423,233.28223694)
\curveto(131.21803863,233.26223595)(131.25303859,233.25223596)(131.2830423,233.25223694)
\curveto(131.32303852,233.26223595)(131.36803848,233.25723596)(131.4180423,233.23723694)
\curveto(131.45803839,233.22723599)(131.52803832,233.217236)(131.6280423,233.20723694)
\curveto(131.73803811,233.19723602)(131.82303802,233.20223601)(131.8830423,233.22223694)
\curveto(131.95303789,233.24223597)(132.02303782,233.25223596)(132.0930423,233.25223694)
\curveto(132.16303768,233.26223595)(132.22803762,233.27723594)(132.2880423,233.29723694)
\curveto(132.48803736,233.35723586)(132.66803718,233.44223577)(132.8280423,233.55223694)
\curveto(132.85803699,233.57223564)(132.88303696,233.59223562)(132.9030423,233.61223694)
\lineto(132.9630423,233.67223694)
\curveto(133.00303684,233.69223552)(133.05303679,233.73223548)(133.1130423,233.79223694)
\curveto(133.21303663,233.93223528)(133.29803655,234.06223515)(133.3680423,234.18223694)
\curveto(133.43803641,234.30223491)(133.50803634,234.44723477)(133.5780423,234.61723694)
\curveto(133.60803624,234.68723453)(133.62803622,234.75723446)(133.6380423,234.82723694)
\curveto(133.65803619,234.89723432)(133.67803617,234.97223424)(133.6980423,235.05223694)
}
}
{
\newrgbcolor{curcolor}{0 0 0}
\pscustom[linestyle=none,fillstyle=solid,fillcolor=curcolor]
{
\newpath
\moveto(143.59265167,237.37723694)
\lineto(143.59265167,237.12223694)
\curveto(143.60264397,237.04223217)(143.59764397,236.96723225)(143.57765167,236.89723694)
\lineto(143.57765167,236.65723694)
\lineto(143.57765167,236.49223694)
\curveto(143.55764401,236.39223282)(143.54764402,236.28723293)(143.54765167,236.17723694)
\curveto(143.54764402,236.07723314)(143.53764403,235.97723324)(143.51765167,235.87723694)
\lineto(143.51765167,235.72723694)
\curveto(143.48764408,235.58723363)(143.4676441,235.44723377)(143.45765167,235.30723694)
\curveto(143.44764412,235.17723404)(143.42264415,235.04723417)(143.38265167,234.91723694)
\curveto(143.36264421,234.83723438)(143.34264423,234.75223446)(143.32265167,234.66223694)
\lineto(143.26265167,234.42223694)
\lineto(143.14265167,234.12223694)
\curveto(143.11264446,234.03223518)(143.07764449,233.94223527)(143.03765167,233.85223694)
\curveto(142.93764463,233.63223558)(142.80264477,233.4172358)(142.63265167,233.20723694)
\curveto(142.4726451,232.99723622)(142.29764527,232.82723639)(142.10765167,232.69723694)
\curveto(142.05764551,232.65723656)(141.99764557,232.6172366)(141.92765167,232.57723694)
\curveto(141.8676457,232.54723667)(141.80764576,232.5122367)(141.74765167,232.47223694)
\curveto(141.6676459,232.42223679)(141.572646,232.38223683)(141.46265167,232.35223694)
\curveto(141.35264622,232.32223689)(141.24764632,232.29223692)(141.14765167,232.26223694)
\curveto(141.03764653,232.22223699)(140.92764664,232.19723702)(140.81765167,232.18723694)
\curveto(140.70764686,232.17723704)(140.59264698,232.16223705)(140.47265167,232.14223694)
\curveto(140.43264714,232.13223708)(140.38764718,232.13223708)(140.33765167,232.14223694)
\curveto(140.29764727,232.14223707)(140.25764731,232.13723708)(140.21765167,232.12723694)
\curveto(140.17764739,232.1172371)(140.12264745,232.1122371)(140.05265167,232.11223694)
\curveto(139.98264759,232.1122371)(139.93264764,232.1172371)(139.90265167,232.12723694)
\curveto(139.85264772,232.14723707)(139.80764776,232.15223706)(139.76765167,232.14223694)
\curveto(139.72764784,232.13223708)(139.69264788,232.13223708)(139.66265167,232.14223694)
\lineto(139.57265167,232.14223694)
\curveto(139.51264806,232.16223705)(139.44764812,232.17723704)(139.37765167,232.18723694)
\curveto(139.31764825,232.18723703)(139.25264832,232.19223702)(139.18265167,232.20223694)
\curveto(139.01264856,232.25223696)(138.85264872,232.30223691)(138.70265167,232.35223694)
\curveto(138.55264902,232.40223681)(138.40764916,232.46723675)(138.26765167,232.54723694)
\curveto(138.21764935,232.58723663)(138.16264941,232.6172366)(138.10265167,232.63723694)
\curveto(138.05264952,232.66723655)(138.00264957,232.70223651)(137.95265167,232.74223694)
\curveto(137.71264986,232.92223629)(137.51265006,233.14223607)(137.35265167,233.40223694)
\curveto(137.19265038,233.66223555)(137.05265052,233.94723527)(136.93265167,234.25723694)
\curveto(136.8726507,234.39723482)(136.82765074,234.53723468)(136.79765167,234.67723694)
\curveto(136.7676508,234.82723439)(136.73265084,234.98223423)(136.69265167,235.14223694)
\curveto(136.6726509,235.25223396)(136.65765091,235.36223385)(136.64765167,235.47223694)
\curveto(136.63765093,235.58223363)(136.62265095,235.69223352)(136.60265167,235.80223694)
\curveto(136.59265098,235.84223337)(136.58765098,235.88223333)(136.58765167,235.92223694)
\curveto(136.59765097,235.96223325)(136.59765097,236.00223321)(136.58765167,236.04223694)
\curveto(136.57765099,236.09223312)(136.572651,236.14223307)(136.57265167,236.19223694)
\lineto(136.57265167,236.35723694)
\curveto(136.55265102,236.40723281)(136.54765102,236.45723276)(136.55765167,236.50723694)
\curveto(136.567651,236.56723265)(136.567651,236.62223259)(136.55765167,236.67223694)
\curveto(136.54765102,236.7122325)(136.54765102,236.75723246)(136.55765167,236.80723694)
\curveto(136.567651,236.85723236)(136.56265101,236.90723231)(136.54265167,236.95723694)
\curveto(136.52265105,237.02723219)(136.51765105,237.10223211)(136.52765167,237.18223694)
\curveto(136.53765103,237.27223194)(136.54265103,237.35723186)(136.54265167,237.43723694)
\curveto(136.54265103,237.52723169)(136.53765103,237.62723159)(136.52765167,237.73723694)
\curveto(136.51765105,237.85723136)(136.52265105,237.95723126)(136.54265167,238.03723694)
\lineto(136.54265167,238.32223694)
\lineto(136.58765167,238.95223694)
\curveto(136.59765097,239.05223016)(136.60765096,239.14723007)(136.61765167,239.23723694)
\lineto(136.64765167,239.53723694)
\curveto(136.6676509,239.58722963)(136.6726509,239.63722958)(136.66265167,239.68723694)
\curveto(136.66265091,239.74722947)(136.6726509,239.80222941)(136.69265167,239.85223694)
\curveto(136.74265083,240.02222919)(136.78265079,240.18722903)(136.81265167,240.34723694)
\curveto(136.84265073,240.5172287)(136.89265068,240.67722854)(136.96265167,240.82723694)
\curveto(137.15265042,241.28722793)(137.3726502,241.66222755)(137.62265167,241.95223694)
\curveto(137.88264969,242.24222697)(138.24264933,242.48722673)(138.70265167,242.68723694)
\curveto(138.83264874,242.73722648)(138.96264861,242.77222644)(139.09265167,242.79223694)
\curveto(139.23264834,242.8122264)(139.3726482,242.83722638)(139.51265167,242.86723694)
\curveto(139.58264799,242.87722634)(139.64764792,242.88222633)(139.70765167,242.88223694)
\curveto(139.7676478,242.88222633)(139.83264774,242.88722633)(139.90265167,242.89723694)
\curveto(140.73264684,242.9172263)(141.40264617,242.76722645)(141.91265167,242.44723694)
\curveto(142.42264515,242.13722708)(142.80264477,241.69722752)(143.05265167,241.12723694)
\curveto(143.10264447,241.00722821)(143.14764442,240.88222833)(143.18765167,240.75223694)
\curveto(143.22764434,240.62222859)(143.2726443,240.48722873)(143.32265167,240.34723694)
\curveto(143.34264423,240.26722895)(143.35764421,240.18222903)(143.36765167,240.09223694)
\lineto(143.42765167,239.85223694)
\curveto(143.45764411,239.74222947)(143.4726441,239.63222958)(143.47265167,239.52223694)
\curveto(143.48264409,239.4122298)(143.49764407,239.30222991)(143.51765167,239.19223694)
\curveto(143.53764403,239.14223007)(143.54264403,239.09723012)(143.53265167,239.05723694)
\curveto(143.53264404,239.0172302)(143.53764403,238.97723024)(143.54765167,238.93723694)
\curveto(143.55764401,238.88723033)(143.55764401,238.83223038)(143.54765167,238.77223694)
\curveto(143.54764402,238.72223049)(143.55264402,238.67223054)(143.56265167,238.62223694)
\lineto(143.56265167,238.48723694)
\curveto(143.58264399,238.42723079)(143.58264399,238.35723086)(143.56265167,238.27723694)
\curveto(143.55264402,238.20723101)(143.55764401,238.14223107)(143.57765167,238.08223694)
\curveto(143.58764398,238.05223116)(143.59264398,238.0122312)(143.59265167,237.96223694)
\lineto(143.59265167,237.84223694)
\lineto(143.59265167,237.37723694)
\moveto(142.04765167,235.05223694)
\curveto(142.14764542,235.37223384)(142.20764536,235.73723348)(142.22765167,236.14723694)
\curveto(142.24764532,236.55723266)(142.25764531,236.96723225)(142.25765167,237.37723694)
\curveto(142.25764531,237.80723141)(142.24764532,238.22723099)(142.22765167,238.63723694)
\curveto(142.20764536,239.04723017)(142.16264541,239.43222978)(142.09265167,239.79223694)
\curveto(142.02264555,240.15222906)(141.91264566,240.47222874)(141.76265167,240.75223694)
\curveto(141.62264595,241.04222817)(141.42764614,241.27722794)(141.17765167,241.45723694)
\curveto(141.01764655,241.56722765)(140.83764673,241.64722757)(140.63765167,241.69723694)
\curveto(140.43764713,241.75722746)(140.19264738,241.78722743)(139.90265167,241.78723694)
\curveto(139.88264769,241.76722745)(139.84764772,241.75722746)(139.79765167,241.75723694)
\curveto(139.74764782,241.76722745)(139.70764786,241.76722745)(139.67765167,241.75723694)
\curveto(139.59764797,241.73722748)(139.52264805,241.7172275)(139.45265167,241.69723694)
\curveto(139.39264818,241.68722753)(139.32764824,241.66722755)(139.25765167,241.63723694)
\curveto(138.98764858,241.5172277)(138.7676488,241.34722787)(138.59765167,241.12723694)
\curveto(138.43764913,240.9172283)(138.30264927,240.67222854)(138.19265167,240.39223694)
\curveto(138.14264943,240.28222893)(138.10264947,240.16222905)(138.07265167,240.03223694)
\curveto(138.05264952,239.9122293)(138.02764954,239.78722943)(137.99765167,239.65723694)
\curveto(137.97764959,239.60722961)(137.9676496,239.55222966)(137.96765167,239.49223694)
\curveto(137.9676496,239.44222977)(137.96264961,239.39222982)(137.95265167,239.34223694)
\curveto(137.94264963,239.25222996)(137.93264964,239.15723006)(137.92265167,239.05723694)
\curveto(137.91264966,238.96723025)(137.90264967,238.87223034)(137.89265167,238.77223694)
\curveto(137.89264968,238.69223052)(137.88764968,238.60723061)(137.87765167,238.51723694)
\lineto(137.87765167,238.27723694)
\lineto(137.87765167,238.09723694)
\curveto(137.8676497,238.06723115)(137.86264971,238.03223118)(137.86265167,237.99223694)
\lineto(137.86265167,237.85723694)
\lineto(137.86265167,237.40723694)
\curveto(137.86264971,237.32723189)(137.85764971,237.24223197)(137.84765167,237.15223694)
\curveto(137.84764972,237.07223214)(137.85764971,236.99723222)(137.87765167,236.92723694)
\lineto(137.87765167,236.65723694)
\curveto(137.87764969,236.63723258)(137.8726497,236.60723261)(137.86265167,236.56723694)
\curveto(137.86264971,236.53723268)(137.8676497,236.5122327)(137.87765167,236.49223694)
\curveto(137.88764968,236.39223282)(137.89264968,236.29223292)(137.89265167,236.19223694)
\curveto(137.90264967,236.10223311)(137.91264966,236.00223321)(137.92265167,235.89223694)
\curveto(137.95264962,235.77223344)(137.9676496,235.64723357)(137.96765167,235.51723694)
\curveto(137.97764959,235.39723382)(138.00264957,235.28223393)(138.04265167,235.17223694)
\curveto(138.12264945,234.87223434)(138.20764936,234.60723461)(138.29765167,234.37723694)
\curveto(138.39764917,234.14723507)(138.54264903,233.93223528)(138.73265167,233.73223694)
\curveto(138.94264863,233.53223568)(139.20764836,233.38223583)(139.52765167,233.28223694)
\curveto(139.567648,233.26223595)(139.60264797,233.25223596)(139.63265167,233.25223694)
\curveto(139.6726479,233.26223595)(139.71764785,233.25723596)(139.76765167,233.23723694)
\curveto(139.80764776,233.22723599)(139.87764769,233.217236)(139.97765167,233.20723694)
\curveto(140.08764748,233.19723602)(140.1726474,233.20223601)(140.23265167,233.22223694)
\curveto(140.30264727,233.24223597)(140.3726472,233.25223596)(140.44265167,233.25223694)
\curveto(140.51264706,233.26223595)(140.57764699,233.27723594)(140.63765167,233.29723694)
\curveto(140.83764673,233.35723586)(141.01764655,233.44223577)(141.17765167,233.55223694)
\curveto(141.20764636,233.57223564)(141.23264634,233.59223562)(141.25265167,233.61223694)
\lineto(141.31265167,233.67223694)
\curveto(141.35264622,233.69223552)(141.40264617,233.73223548)(141.46265167,233.79223694)
\curveto(141.56264601,233.93223528)(141.64764592,234.06223515)(141.71765167,234.18223694)
\curveto(141.78764578,234.30223491)(141.85764571,234.44723477)(141.92765167,234.61723694)
\curveto(141.95764561,234.68723453)(141.97764559,234.75723446)(141.98765167,234.82723694)
\curveto(142.00764556,234.89723432)(142.02764554,234.97223424)(142.04765167,235.05223694)
}
}
{
\newrgbcolor{curcolor}{0 0 0}
\pscustom[linestyle=none,fillstyle=solid,fillcolor=curcolor]
{
\newpath
\moveto(126.77343292,310.78723694)
\curveto(126.84342528,310.73723348)(126.88342524,310.66723355)(126.89343292,310.57723694)
\curveto(126.91342521,310.48723373)(126.9234252,310.38223383)(126.92343292,310.26223694)
\curveto(126.9234252,310.212234)(126.9184252,310.16223405)(126.90843292,310.11223694)
\curveto(126.90842521,310.06223415)(126.89842522,310.0172342)(126.87843292,309.97723694)
\curveto(126.84842527,309.88723433)(126.78842533,309.82723439)(126.69843292,309.79723694)
\curveto(126.6184255,309.77723444)(126.5234256,309.76723445)(126.41343292,309.76723694)
\lineto(126.09843292,309.76723694)
\curveto(125.98842613,309.77723444)(125.88342624,309.76723445)(125.78343292,309.73723694)
\curveto(125.64342648,309.70723451)(125.55342657,309.62723459)(125.51343292,309.49723694)
\curveto(125.49342663,309.42723479)(125.48342664,309.34223487)(125.48343292,309.24223694)
\lineto(125.48343292,308.97223694)
\lineto(125.48343292,308.02723694)
\lineto(125.48343292,307.69723694)
\curveto(125.48342664,307.58723663)(125.46342666,307.50223671)(125.42343292,307.44223694)
\curveto(125.38342674,307.38223683)(125.33342679,307.34223687)(125.27343292,307.32223694)
\curveto(125.2234269,307.3122369)(125.15842696,307.29723692)(125.07843292,307.27723694)
\lineto(124.88343292,307.27723694)
\curveto(124.76342736,307.27723694)(124.65842746,307.28223693)(124.56843292,307.29223694)
\curveto(124.47842764,307.3122369)(124.40842771,307.36223685)(124.35843292,307.44223694)
\curveto(124.32842779,307.49223672)(124.31342781,307.56223665)(124.31343292,307.65223694)
\lineto(124.31343292,307.95223694)
\lineto(124.31343292,308.98723694)
\curveto(124.31342781,309.14723507)(124.30342782,309.29223492)(124.28343292,309.42223694)
\curveto(124.27342785,309.56223465)(124.2184279,309.65723456)(124.11843292,309.70723694)
\curveto(124.06842805,309.72723449)(123.99842812,309.74223447)(123.90843292,309.75223694)
\curveto(123.82842829,309.76223445)(123.73842838,309.76723445)(123.63843292,309.76723694)
\lineto(123.35343292,309.76723694)
\lineto(123.11343292,309.76723694)
\lineto(120.84843292,309.76723694)
\curveto(120.75843136,309.76723445)(120.65343147,309.76223445)(120.53343292,309.75223694)
\lineto(120.20343292,309.75223694)
\curveto(120.09343203,309.75223446)(119.99343213,309.76223445)(119.90343292,309.78223694)
\curveto(119.81343231,309.80223441)(119.75343237,309.83723438)(119.72343292,309.88723694)
\curveto(119.67343245,309.95723426)(119.64843247,310.05223416)(119.64843292,310.17223694)
\lineto(119.64843292,310.51723694)
\lineto(119.64843292,310.78723694)
\curveto(119.68843243,310.95723326)(119.74343238,311.09723312)(119.81343292,311.20723694)
\curveto(119.88343224,311.3172329)(119.96343216,311.43223278)(120.05343292,311.55223694)
\lineto(120.41343292,312.09223694)
\curveto(120.85343127,312.72223149)(121.28843083,313.34223087)(121.71843292,313.95223694)
\lineto(123.03843292,315.81223694)
\curveto(123.19842892,316.04222817)(123.35342877,316.26222795)(123.50343292,316.47223694)
\curveto(123.65342847,316.69222752)(123.80842831,316.9172273)(123.96843292,317.14723694)
\curveto(124.0184281,317.217227)(124.06842805,317.28222693)(124.11843292,317.34223694)
\curveto(124.16842795,317.4122268)(124.2184279,317.48722673)(124.26843292,317.56723694)
\lineto(124.32843292,317.65723694)
\curveto(124.35842776,317.69722652)(124.38842773,317.72722649)(124.41843292,317.74723694)
\curveto(124.45842766,317.77722644)(124.49842762,317.79722642)(124.53843292,317.80723694)
\curveto(124.57842754,317.82722639)(124.6234275,317.84722637)(124.67343292,317.86723694)
\curveto(124.69342743,317.86722635)(124.71342741,317.86222635)(124.73343292,317.85223694)
\curveto(124.76342736,317.85222636)(124.78842733,317.86222635)(124.80843292,317.88223694)
\curveto(124.93842718,317.88222633)(125.05842706,317.87722634)(125.16843292,317.86723694)
\curveto(125.27842684,317.85722636)(125.35842676,317.8122264)(125.40843292,317.73223694)
\curveto(125.44842667,317.68222653)(125.46842665,317.6122266)(125.46843292,317.52223694)
\curveto(125.47842664,317.43222678)(125.48342664,317.33722688)(125.48343292,317.23723694)
\lineto(125.48343292,311.77723694)
\curveto(125.48342664,311.70723251)(125.47842664,311.63223258)(125.46843292,311.55223694)
\curveto(125.46842665,311.48223273)(125.47342665,311.4122328)(125.48343292,311.34223694)
\lineto(125.48343292,311.23723694)
\curveto(125.50342662,311.18723303)(125.5184266,311.13223308)(125.52843292,311.07223694)
\curveto(125.53842658,311.02223319)(125.56342656,310.98223323)(125.60343292,310.95223694)
\curveto(125.67342645,310.90223331)(125.75842636,310.87223334)(125.85843292,310.86223694)
\lineto(126.18843292,310.86223694)
\curveto(126.29842582,310.86223335)(126.40342572,310.85723336)(126.50343292,310.84723694)
\curveto(126.61342551,310.84723337)(126.70342542,310.82723339)(126.77343292,310.78723694)
\moveto(124.20843292,310.98223694)
\curveto(124.28842783,311.09223312)(124.3234278,311.26223295)(124.31343292,311.49223694)
\lineto(124.31343292,312.10723694)
\lineto(124.31343292,314.58223694)
\lineto(124.31343292,314.89723694)
\curveto(124.3234278,315.0172292)(124.3184278,315.1172291)(124.29843292,315.19723694)
\lineto(124.29843292,315.34723694)
\curveto(124.29842782,315.43722878)(124.28342784,315.52222869)(124.25343292,315.60223694)
\curveto(124.24342788,315.62222859)(124.23342789,315.63222858)(124.22343292,315.63223694)
\lineto(124.17843292,315.67723694)
\curveto(124.15842796,315.68722853)(124.12842799,315.69222852)(124.08843292,315.69223694)
\curveto(124.06842805,315.67222854)(124.04842807,315.65722856)(124.02843292,315.64723694)
\curveto(124.0184281,315.64722857)(124.00342812,315.64222857)(123.98343292,315.63223694)
\curveto(123.9234282,315.58222863)(123.86342826,315.5122287)(123.80343292,315.42223694)
\curveto(123.74342838,315.33222888)(123.68842843,315.25222896)(123.63843292,315.18223694)
\curveto(123.53842858,315.04222917)(123.44342868,314.89722932)(123.35343292,314.74723694)
\curveto(123.26342886,314.60722961)(123.16842895,314.46722975)(123.06843292,314.32723694)
\lineto(122.52843292,313.54723694)
\curveto(122.35842976,313.28723093)(122.18342994,313.02723119)(122.00343292,312.76723694)
\curveto(121.9234302,312.65723156)(121.84843027,312.55223166)(121.77843292,312.45223694)
\lineto(121.56843292,312.15223694)
\curveto(121.5184306,312.07223214)(121.46843065,311.99723222)(121.41843292,311.92723694)
\curveto(121.37843074,311.85723236)(121.33343079,311.78223243)(121.28343292,311.70223694)
\curveto(121.23343089,311.64223257)(121.18343094,311.57723264)(121.13343292,311.50723694)
\curveto(121.09343103,311.44723277)(121.05343107,311.37723284)(121.01343292,311.29723694)
\curveto(120.97343115,311.23723298)(120.94843117,311.16723305)(120.93843292,311.08723694)
\curveto(120.92843119,311.0172332)(120.96343116,310.96223325)(121.04343292,310.92223694)
\curveto(121.11343101,310.87223334)(121.2234309,310.84723337)(121.37343292,310.84723694)
\curveto(121.53343059,310.85723336)(121.66843045,310.86223335)(121.77843292,310.86223694)
\lineto(123.45843292,310.86223694)
\lineto(123.89343292,310.86223694)
\curveto(124.04342808,310.86223335)(124.14842797,310.90223331)(124.20843292,310.98223694)
}
}
{
\newrgbcolor{curcolor}{0 0 0}
\pscustom[linestyle=none,fillstyle=solid,fillcolor=curcolor]
{
\newpath
\moveto(129.7530423,317.70223694)
\lineto(133.3530423,317.70223694)
\lineto(133.9980423,317.70223694)
\curveto(134.07803577,317.70222651)(134.15303569,317.69722652)(134.2230423,317.68723694)
\curveto(134.29303555,317.68722653)(134.35303549,317.67722654)(134.4030423,317.65723694)
\curveto(134.47303537,317.62722659)(134.52803532,317.56722665)(134.5680423,317.47723694)
\curveto(134.58803526,317.44722677)(134.59803525,317.40722681)(134.5980423,317.35723694)
\lineto(134.5980423,317.22223694)
\curveto(134.60803524,317.1122271)(134.60303524,317.00722721)(134.5830423,316.90723694)
\curveto(134.57303527,316.80722741)(134.53803531,316.73722748)(134.4780423,316.69723694)
\curveto(134.38803546,316.62722759)(134.25303559,316.59222762)(134.0730423,316.59223694)
\curveto(133.89303595,316.60222761)(133.72803612,316.60722761)(133.5780423,316.60723694)
\lineto(131.5830423,316.60723694)
\lineto(131.0880423,316.60723694)
\lineto(130.9530423,316.60723694)
\curveto(130.91303893,316.60722761)(130.87303897,316.60222761)(130.8330423,316.59223694)
\lineto(130.6230423,316.59223694)
\curveto(130.51303933,316.56222765)(130.43303941,316.52222769)(130.3830423,316.47223694)
\curveto(130.33303951,316.43222778)(130.29803955,316.37722784)(130.2780423,316.30723694)
\curveto(130.25803959,316.24722797)(130.2430396,316.17722804)(130.2330423,316.09723694)
\curveto(130.22303962,316.0172282)(130.20303964,315.92722829)(130.1730423,315.82723694)
\curveto(130.12303972,315.62722859)(130.08303976,315.42222879)(130.0530423,315.21223694)
\curveto(130.02303982,315.00222921)(129.98303986,314.79722942)(129.9330423,314.59723694)
\curveto(129.91303993,314.52722969)(129.90303994,314.45722976)(129.9030423,314.38723694)
\curveto(129.90303994,314.32722989)(129.89303995,314.26222995)(129.8730423,314.19223694)
\curveto(129.86303998,314.16223005)(129.85303999,314.12223009)(129.8430423,314.07223694)
\curveto(129.84304,314.03223018)(129.84804,313.99223022)(129.8580423,313.95223694)
\curveto(129.87803997,313.90223031)(129.90303994,313.85723036)(129.9330423,313.81723694)
\curveto(129.97303987,313.78723043)(130.03303981,313.78223043)(130.1130423,313.80223694)
\curveto(130.17303967,313.82223039)(130.23303961,313.84723037)(130.2930423,313.87723694)
\curveto(130.35303949,313.9172303)(130.41303943,313.95223026)(130.4730423,313.98223694)
\curveto(130.53303931,314.00223021)(130.58303926,314.0172302)(130.6230423,314.02723694)
\curveto(130.81303903,314.10723011)(131.01803883,314.16223005)(131.2380423,314.19223694)
\curveto(131.46803838,314.22222999)(131.69803815,314.23222998)(131.9280423,314.22223694)
\curveto(132.16803768,314.22222999)(132.39803745,314.19723002)(132.6180423,314.14723694)
\curveto(132.83803701,314.10723011)(133.03803681,314.04723017)(133.2180423,313.96723694)
\curveto(133.26803658,313.94723027)(133.31303653,313.92723029)(133.3530423,313.90723694)
\curveto(133.40303644,313.88723033)(133.45303639,313.86223035)(133.5030423,313.83223694)
\curveto(133.85303599,313.62223059)(134.13303571,313.39223082)(134.3430423,313.14223694)
\curveto(134.56303528,312.89223132)(134.75803509,312.56723165)(134.9280423,312.16723694)
\curveto(134.97803487,312.05723216)(135.01303483,311.94723227)(135.0330423,311.83723694)
\curveto(135.05303479,311.72723249)(135.07803477,311.6122326)(135.1080423,311.49223694)
\curveto(135.11803473,311.46223275)(135.12303472,311.4172328)(135.1230423,311.35723694)
\curveto(135.1430347,311.29723292)(135.15303469,311.22723299)(135.1530423,311.14723694)
\curveto(135.15303469,311.07723314)(135.16303468,311.0122332)(135.1830423,310.95223694)
\lineto(135.1830423,310.78723694)
\curveto(135.19303465,310.73723348)(135.19803465,310.66723355)(135.1980423,310.57723694)
\curveto(135.19803465,310.48723373)(135.18803466,310.4172338)(135.1680423,310.36723694)
\curveto(135.1480347,310.30723391)(135.1430347,310.24723397)(135.1530423,310.18723694)
\curveto(135.16303468,310.13723408)(135.15803469,310.08723413)(135.1380423,310.03723694)
\curveto(135.09803475,309.87723434)(135.06303478,309.72723449)(135.0330423,309.58723694)
\curveto(135.00303484,309.44723477)(134.95803489,309.3122349)(134.8980423,309.18223694)
\curveto(134.73803511,308.8122354)(134.51803533,308.47723574)(134.2380423,308.17723694)
\curveto(133.95803589,307.87723634)(133.63803621,307.64723657)(133.2780423,307.48723694)
\curveto(133.10803674,307.40723681)(132.90803694,307.33223688)(132.6780423,307.26223694)
\curveto(132.56803728,307.22223699)(132.45303739,307.19723702)(132.3330423,307.18723694)
\curveto(132.21303763,307.17723704)(132.09303775,307.15723706)(131.9730423,307.12723694)
\curveto(131.92303792,307.10723711)(131.86803798,307.10723711)(131.8080423,307.12723694)
\curveto(131.7480381,307.13723708)(131.68803816,307.13223708)(131.6280423,307.11223694)
\curveto(131.52803832,307.09223712)(131.42803842,307.09223712)(131.3280423,307.11223694)
\lineto(131.1930423,307.11223694)
\curveto(131.1430387,307.13223708)(131.08303876,307.14223707)(131.0130423,307.14223694)
\curveto(130.95303889,307.13223708)(130.89803895,307.13723708)(130.8480423,307.15723694)
\curveto(130.80803904,307.16723705)(130.77303907,307.17223704)(130.7430423,307.17223694)
\curveto(130.71303913,307.17223704)(130.67803917,307.17723704)(130.6380423,307.18723694)
\lineto(130.3680423,307.24723694)
\curveto(130.27803957,307.26723695)(130.19303965,307.29723692)(130.1130423,307.33723694)
\curveto(129.77304007,307.47723674)(129.48304036,307.63223658)(129.2430423,307.80223694)
\curveto(129.00304084,307.98223623)(128.78304106,308.212236)(128.5830423,308.49223694)
\curveto(128.43304141,308.72223549)(128.31804153,308.96223525)(128.2380423,309.21223694)
\curveto(128.21804163,309.26223495)(128.20804164,309.30723491)(128.2080423,309.34723694)
\curveto(128.20804164,309.39723482)(128.19804165,309.44723477)(128.1780423,309.49723694)
\curveto(128.15804169,309.55723466)(128.1430417,309.63723458)(128.1330423,309.73723694)
\curveto(128.13304171,309.83723438)(128.15304169,309.9122343)(128.1930423,309.96223694)
\curveto(128.2430416,310.04223417)(128.32304152,310.08723413)(128.4330423,310.09723694)
\curveto(128.5430413,310.10723411)(128.65804119,310.1122341)(128.7780423,310.11223694)
\lineto(128.9430423,310.11223694)
\curveto(129.00304084,310.1122341)(129.05804079,310.10223411)(129.1080423,310.08223694)
\curveto(129.19804065,310.06223415)(129.26804058,310.02223419)(129.3180423,309.96223694)
\curveto(129.38804046,309.87223434)(129.43304041,309.76223445)(129.4530423,309.63223694)
\curveto(129.48304036,309.5122347)(129.52804032,309.40723481)(129.5880423,309.31723694)
\curveto(129.77804007,308.97723524)(130.03803981,308.70723551)(130.3680423,308.50723694)
\curveto(130.46803938,308.44723577)(130.57303927,308.39723582)(130.6830423,308.35723694)
\curveto(130.80303904,308.32723589)(130.92303892,308.29223592)(131.0430423,308.25223694)
\curveto(131.21303863,308.20223601)(131.41803843,308.18223603)(131.6580423,308.19223694)
\curveto(131.90803794,308.212236)(132.10803774,308.24723597)(132.2580423,308.29723694)
\curveto(132.62803722,308.4172358)(132.91803693,308.57723564)(133.1280423,308.77723694)
\curveto(133.3480365,308.98723523)(133.52803632,309.26723495)(133.6680423,309.61723694)
\curveto(133.71803613,309.7172345)(133.7480361,309.82223439)(133.7580423,309.93223694)
\curveto(133.77803607,310.04223417)(133.80303604,310.15723406)(133.8330423,310.27723694)
\lineto(133.8330423,310.38223694)
\curveto(133.843036,310.42223379)(133.848036,310.46223375)(133.8480423,310.50223694)
\curveto(133.85803599,310.53223368)(133.85803599,310.56723365)(133.8480423,310.60723694)
\lineto(133.8480423,310.72723694)
\curveto(133.848036,310.98723323)(133.81803603,311.23223298)(133.7580423,311.46223694)
\curveto(133.6480362,311.8122324)(133.49303635,312.10723211)(133.2930423,312.34723694)
\curveto(133.09303675,312.59723162)(132.83303701,312.79223142)(132.5130423,312.93223694)
\lineto(132.3330423,312.99223694)
\curveto(132.28303756,313.0122312)(132.22303762,313.03223118)(132.1530423,313.05223694)
\curveto(132.10303774,313.07223114)(132.0430378,313.08223113)(131.9730423,313.08223694)
\curveto(131.91303793,313.09223112)(131.848038,313.10723111)(131.7780423,313.12723694)
\lineto(131.6280423,313.12723694)
\curveto(131.58803826,313.14723107)(131.53303831,313.15723106)(131.4630423,313.15723694)
\curveto(131.40303844,313.15723106)(131.3480385,313.14723107)(131.2980423,313.12723694)
\lineto(131.1930423,313.12723694)
\curveto(131.16303868,313.12723109)(131.12803872,313.12223109)(131.0880423,313.11223694)
\lineto(130.8480423,313.05223694)
\curveto(130.76803908,313.04223117)(130.68803916,313.02223119)(130.6080423,312.99223694)
\curveto(130.36803948,312.89223132)(130.13803971,312.75723146)(129.9180423,312.58723694)
\curveto(129.82804002,312.5172317)(129.7430401,312.44223177)(129.6630423,312.36223694)
\curveto(129.58304026,312.29223192)(129.48304036,312.23723198)(129.3630423,312.19723694)
\curveto(129.27304057,312.16723205)(129.13304071,312.15723206)(128.9430423,312.16723694)
\curveto(128.76304108,312.17723204)(128.6430412,312.20223201)(128.5830423,312.24223694)
\curveto(128.53304131,312.28223193)(128.49304135,312.34223187)(128.4630423,312.42223694)
\curveto(128.4430414,312.50223171)(128.4430414,312.58723163)(128.4630423,312.67723694)
\curveto(128.49304135,312.79723142)(128.51304133,312.9172313)(128.5230423,313.03723694)
\curveto(128.5430413,313.16723105)(128.56804128,313.29223092)(128.5980423,313.41223694)
\curveto(128.61804123,313.45223076)(128.62304122,313.48723073)(128.6130423,313.51723694)
\curveto(128.61304123,313.55723066)(128.62304122,313.60223061)(128.6430423,313.65223694)
\curveto(128.66304118,313.74223047)(128.67804117,313.83223038)(128.6880423,313.92223694)
\curveto(128.69804115,314.02223019)(128.71804113,314.1172301)(128.7480423,314.20723694)
\curveto(128.75804109,314.26722995)(128.76304108,314.32722989)(128.7630423,314.38723694)
\curveto(128.77304107,314.44722977)(128.78804106,314.50722971)(128.8080423,314.56723694)
\curveto(128.85804099,314.76722945)(128.89304095,314.97222924)(128.9130423,315.18223694)
\curveto(128.9430409,315.40222881)(128.98304086,315.6122286)(129.0330423,315.81223694)
\curveto(129.06304078,315.9122283)(129.08304076,316.0122282)(129.0930423,316.11223694)
\curveto(129.10304074,316.212228)(129.11804073,316.3122279)(129.1380423,316.41223694)
\curveto(129.1480407,316.44222777)(129.15304069,316.48222773)(129.1530423,316.53223694)
\curveto(129.18304066,316.64222757)(129.20304064,316.74722747)(129.2130423,316.84723694)
\curveto(129.23304061,316.95722726)(129.25804059,317.06722715)(129.2880423,317.17723694)
\curveto(129.30804054,317.25722696)(129.32304052,317.32722689)(129.3330423,317.38723694)
\curveto(129.3430405,317.45722676)(129.36804048,317.5172267)(129.4080423,317.56723694)
\curveto(129.42804042,317.59722662)(129.45804039,317.6172266)(129.4980423,317.62723694)
\curveto(129.53804031,317.64722657)(129.58304026,317.66722655)(129.6330423,317.68723694)
\curveto(129.69304015,317.68722653)(129.73304011,317.69222652)(129.7530423,317.70223694)
}
}
{
\newrgbcolor{curcolor}{0 0 0}
\pscustom[linestyle=none,fillstyle=solid,fillcolor=curcolor]
{
\newpath
\moveto(143.59265167,312.37723694)
\lineto(143.59265167,312.12223694)
\curveto(143.60264397,312.04223217)(143.59764397,311.96723225)(143.57765167,311.89723694)
\lineto(143.57765167,311.65723694)
\lineto(143.57765167,311.49223694)
\curveto(143.55764401,311.39223282)(143.54764402,311.28723293)(143.54765167,311.17723694)
\curveto(143.54764402,311.07723314)(143.53764403,310.97723324)(143.51765167,310.87723694)
\lineto(143.51765167,310.72723694)
\curveto(143.48764408,310.58723363)(143.4676441,310.44723377)(143.45765167,310.30723694)
\curveto(143.44764412,310.17723404)(143.42264415,310.04723417)(143.38265167,309.91723694)
\curveto(143.36264421,309.83723438)(143.34264423,309.75223446)(143.32265167,309.66223694)
\lineto(143.26265167,309.42223694)
\lineto(143.14265167,309.12223694)
\curveto(143.11264446,309.03223518)(143.07764449,308.94223527)(143.03765167,308.85223694)
\curveto(142.93764463,308.63223558)(142.80264477,308.4172358)(142.63265167,308.20723694)
\curveto(142.4726451,307.99723622)(142.29764527,307.82723639)(142.10765167,307.69723694)
\curveto(142.05764551,307.65723656)(141.99764557,307.6172366)(141.92765167,307.57723694)
\curveto(141.8676457,307.54723667)(141.80764576,307.5122367)(141.74765167,307.47223694)
\curveto(141.6676459,307.42223679)(141.572646,307.38223683)(141.46265167,307.35223694)
\curveto(141.35264622,307.32223689)(141.24764632,307.29223692)(141.14765167,307.26223694)
\curveto(141.03764653,307.22223699)(140.92764664,307.19723702)(140.81765167,307.18723694)
\curveto(140.70764686,307.17723704)(140.59264698,307.16223705)(140.47265167,307.14223694)
\curveto(140.43264714,307.13223708)(140.38764718,307.13223708)(140.33765167,307.14223694)
\curveto(140.29764727,307.14223707)(140.25764731,307.13723708)(140.21765167,307.12723694)
\curveto(140.17764739,307.1172371)(140.12264745,307.1122371)(140.05265167,307.11223694)
\curveto(139.98264759,307.1122371)(139.93264764,307.1172371)(139.90265167,307.12723694)
\curveto(139.85264772,307.14723707)(139.80764776,307.15223706)(139.76765167,307.14223694)
\curveto(139.72764784,307.13223708)(139.69264788,307.13223708)(139.66265167,307.14223694)
\lineto(139.57265167,307.14223694)
\curveto(139.51264806,307.16223705)(139.44764812,307.17723704)(139.37765167,307.18723694)
\curveto(139.31764825,307.18723703)(139.25264832,307.19223702)(139.18265167,307.20223694)
\curveto(139.01264856,307.25223696)(138.85264872,307.30223691)(138.70265167,307.35223694)
\curveto(138.55264902,307.40223681)(138.40764916,307.46723675)(138.26765167,307.54723694)
\curveto(138.21764935,307.58723663)(138.16264941,307.6172366)(138.10265167,307.63723694)
\curveto(138.05264952,307.66723655)(138.00264957,307.70223651)(137.95265167,307.74223694)
\curveto(137.71264986,307.92223629)(137.51265006,308.14223607)(137.35265167,308.40223694)
\curveto(137.19265038,308.66223555)(137.05265052,308.94723527)(136.93265167,309.25723694)
\curveto(136.8726507,309.39723482)(136.82765074,309.53723468)(136.79765167,309.67723694)
\curveto(136.7676508,309.82723439)(136.73265084,309.98223423)(136.69265167,310.14223694)
\curveto(136.6726509,310.25223396)(136.65765091,310.36223385)(136.64765167,310.47223694)
\curveto(136.63765093,310.58223363)(136.62265095,310.69223352)(136.60265167,310.80223694)
\curveto(136.59265098,310.84223337)(136.58765098,310.88223333)(136.58765167,310.92223694)
\curveto(136.59765097,310.96223325)(136.59765097,311.00223321)(136.58765167,311.04223694)
\curveto(136.57765099,311.09223312)(136.572651,311.14223307)(136.57265167,311.19223694)
\lineto(136.57265167,311.35723694)
\curveto(136.55265102,311.40723281)(136.54765102,311.45723276)(136.55765167,311.50723694)
\curveto(136.567651,311.56723265)(136.567651,311.62223259)(136.55765167,311.67223694)
\curveto(136.54765102,311.7122325)(136.54765102,311.75723246)(136.55765167,311.80723694)
\curveto(136.567651,311.85723236)(136.56265101,311.90723231)(136.54265167,311.95723694)
\curveto(136.52265105,312.02723219)(136.51765105,312.10223211)(136.52765167,312.18223694)
\curveto(136.53765103,312.27223194)(136.54265103,312.35723186)(136.54265167,312.43723694)
\curveto(136.54265103,312.52723169)(136.53765103,312.62723159)(136.52765167,312.73723694)
\curveto(136.51765105,312.85723136)(136.52265105,312.95723126)(136.54265167,313.03723694)
\lineto(136.54265167,313.32223694)
\lineto(136.58765167,313.95223694)
\curveto(136.59765097,314.05223016)(136.60765096,314.14723007)(136.61765167,314.23723694)
\lineto(136.64765167,314.53723694)
\curveto(136.6676509,314.58722963)(136.6726509,314.63722958)(136.66265167,314.68723694)
\curveto(136.66265091,314.74722947)(136.6726509,314.80222941)(136.69265167,314.85223694)
\curveto(136.74265083,315.02222919)(136.78265079,315.18722903)(136.81265167,315.34723694)
\curveto(136.84265073,315.5172287)(136.89265068,315.67722854)(136.96265167,315.82723694)
\curveto(137.15265042,316.28722793)(137.3726502,316.66222755)(137.62265167,316.95223694)
\curveto(137.88264969,317.24222697)(138.24264933,317.48722673)(138.70265167,317.68723694)
\curveto(138.83264874,317.73722648)(138.96264861,317.77222644)(139.09265167,317.79223694)
\curveto(139.23264834,317.8122264)(139.3726482,317.83722638)(139.51265167,317.86723694)
\curveto(139.58264799,317.87722634)(139.64764792,317.88222633)(139.70765167,317.88223694)
\curveto(139.7676478,317.88222633)(139.83264774,317.88722633)(139.90265167,317.89723694)
\curveto(140.73264684,317.9172263)(141.40264617,317.76722645)(141.91265167,317.44723694)
\curveto(142.42264515,317.13722708)(142.80264477,316.69722752)(143.05265167,316.12723694)
\curveto(143.10264447,316.00722821)(143.14764442,315.88222833)(143.18765167,315.75223694)
\curveto(143.22764434,315.62222859)(143.2726443,315.48722873)(143.32265167,315.34723694)
\curveto(143.34264423,315.26722895)(143.35764421,315.18222903)(143.36765167,315.09223694)
\lineto(143.42765167,314.85223694)
\curveto(143.45764411,314.74222947)(143.4726441,314.63222958)(143.47265167,314.52223694)
\curveto(143.48264409,314.4122298)(143.49764407,314.30222991)(143.51765167,314.19223694)
\curveto(143.53764403,314.14223007)(143.54264403,314.09723012)(143.53265167,314.05723694)
\curveto(143.53264404,314.0172302)(143.53764403,313.97723024)(143.54765167,313.93723694)
\curveto(143.55764401,313.88723033)(143.55764401,313.83223038)(143.54765167,313.77223694)
\curveto(143.54764402,313.72223049)(143.55264402,313.67223054)(143.56265167,313.62223694)
\lineto(143.56265167,313.48723694)
\curveto(143.58264399,313.42723079)(143.58264399,313.35723086)(143.56265167,313.27723694)
\curveto(143.55264402,313.20723101)(143.55764401,313.14223107)(143.57765167,313.08223694)
\curveto(143.58764398,313.05223116)(143.59264398,313.0122312)(143.59265167,312.96223694)
\lineto(143.59265167,312.84223694)
\lineto(143.59265167,312.37723694)
\moveto(142.04765167,310.05223694)
\curveto(142.14764542,310.37223384)(142.20764536,310.73723348)(142.22765167,311.14723694)
\curveto(142.24764532,311.55723266)(142.25764531,311.96723225)(142.25765167,312.37723694)
\curveto(142.25764531,312.80723141)(142.24764532,313.22723099)(142.22765167,313.63723694)
\curveto(142.20764536,314.04723017)(142.16264541,314.43222978)(142.09265167,314.79223694)
\curveto(142.02264555,315.15222906)(141.91264566,315.47222874)(141.76265167,315.75223694)
\curveto(141.62264595,316.04222817)(141.42764614,316.27722794)(141.17765167,316.45723694)
\curveto(141.01764655,316.56722765)(140.83764673,316.64722757)(140.63765167,316.69723694)
\curveto(140.43764713,316.75722746)(140.19264738,316.78722743)(139.90265167,316.78723694)
\curveto(139.88264769,316.76722745)(139.84764772,316.75722746)(139.79765167,316.75723694)
\curveto(139.74764782,316.76722745)(139.70764786,316.76722745)(139.67765167,316.75723694)
\curveto(139.59764797,316.73722748)(139.52264805,316.7172275)(139.45265167,316.69723694)
\curveto(139.39264818,316.68722753)(139.32764824,316.66722755)(139.25765167,316.63723694)
\curveto(138.98764858,316.5172277)(138.7676488,316.34722787)(138.59765167,316.12723694)
\curveto(138.43764913,315.9172283)(138.30264927,315.67222854)(138.19265167,315.39223694)
\curveto(138.14264943,315.28222893)(138.10264947,315.16222905)(138.07265167,315.03223694)
\curveto(138.05264952,314.9122293)(138.02764954,314.78722943)(137.99765167,314.65723694)
\curveto(137.97764959,314.60722961)(137.9676496,314.55222966)(137.96765167,314.49223694)
\curveto(137.9676496,314.44222977)(137.96264961,314.39222982)(137.95265167,314.34223694)
\curveto(137.94264963,314.25222996)(137.93264964,314.15723006)(137.92265167,314.05723694)
\curveto(137.91264966,313.96723025)(137.90264967,313.87223034)(137.89265167,313.77223694)
\curveto(137.89264968,313.69223052)(137.88764968,313.60723061)(137.87765167,313.51723694)
\lineto(137.87765167,313.27723694)
\lineto(137.87765167,313.09723694)
\curveto(137.8676497,313.06723115)(137.86264971,313.03223118)(137.86265167,312.99223694)
\lineto(137.86265167,312.85723694)
\lineto(137.86265167,312.40723694)
\curveto(137.86264971,312.32723189)(137.85764971,312.24223197)(137.84765167,312.15223694)
\curveto(137.84764972,312.07223214)(137.85764971,311.99723222)(137.87765167,311.92723694)
\lineto(137.87765167,311.65723694)
\curveto(137.87764969,311.63723258)(137.8726497,311.60723261)(137.86265167,311.56723694)
\curveto(137.86264971,311.53723268)(137.8676497,311.5122327)(137.87765167,311.49223694)
\curveto(137.88764968,311.39223282)(137.89264968,311.29223292)(137.89265167,311.19223694)
\curveto(137.90264967,311.10223311)(137.91264966,311.00223321)(137.92265167,310.89223694)
\curveto(137.95264962,310.77223344)(137.9676496,310.64723357)(137.96765167,310.51723694)
\curveto(137.97764959,310.39723382)(138.00264957,310.28223393)(138.04265167,310.17223694)
\curveto(138.12264945,309.87223434)(138.20764936,309.60723461)(138.29765167,309.37723694)
\curveto(138.39764917,309.14723507)(138.54264903,308.93223528)(138.73265167,308.73223694)
\curveto(138.94264863,308.53223568)(139.20764836,308.38223583)(139.52765167,308.28223694)
\curveto(139.567648,308.26223595)(139.60264797,308.25223596)(139.63265167,308.25223694)
\curveto(139.6726479,308.26223595)(139.71764785,308.25723596)(139.76765167,308.23723694)
\curveto(139.80764776,308.22723599)(139.87764769,308.217236)(139.97765167,308.20723694)
\curveto(140.08764748,308.19723602)(140.1726474,308.20223601)(140.23265167,308.22223694)
\curveto(140.30264727,308.24223597)(140.3726472,308.25223596)(140.44265167,308.25223694)
\curveto(140.51264706,308.26223595)(140.57764699,308.27723594)(140.63765167,308.29723694)
\curveto(140.83764673,308.35723586)(141.01764655,308.44223577)(141.17765167,308.55223694)
\curveto(141.20764636,308.57223564)(141.23264634,308.59223562)(141.25265167,308.61223694)
\lineto(141.31265167,308.67223694)
\curveto(141.35264622,308.69223552)(141.40264617,308.73223548)(141.46265167,308.79223694)
\curveto(141.56264601,308.93223528)(141.64764592,309.06223515)(141.71765167,309.18223694)
\curveto(141.78764578,309.30223491)(141.85764571,309.44723477)(141.92765167,309.61723694)
\curveto(141.95764561,309.68723453)(141.97764559,309.75723446)(141.98765167,309.82723694)
\curveto(142.00764556,309.89723432)(142.02764554,309.97223424)(142.04765167,310.05223694)
}
}
{
\newrgbcolor{curcolor}{0 0 0}
\pscustom[linestyle=none,fillstyle=solid,fillcolor=curcolor]
{
\newpath
\moveto(126.89343292,385.89795013)
\curveto(126.9234252,385.77794592)(126.94842517,385.63794606)(126.96843292,385.47795013)
\curveto(126.98842513,385.31794638)(126.99842512,385.15294654)(126.99843292,384.98295013)
\curveto(126.99842512,384.81294688)(126.98842513,384.64794705)(126.96843292,384.48795013)
\curveto(126.94842517,384.32794737)(126.9234252,384.18794751)(126.89343292,384.06795013)
\curveto(126.85342527,383.92794777)(126.8184253,383.80294789)(126.78843292,383.69295013)
\curveto(126.75842536,383.58294811)(126.7184254,383.47294822)(126.66843292,383.36295013)
\curveto(126.39842572,382.72294897)(125.98342614,382.23794946)(125.42343292,381.90795013)
\curveto(125.34342678,381.84794985)(125.25842686,381.7979499)(125.16843292,381.75795013)
\curveto(125.07842704,381.72794997)(124.97842714,381.69295)(124.86843292,381.65295013)
\curveto(124.75842736,381.60295009)(124.63842748,381.56795013)(124.50843292,381.54795013)
\curveto(124.38842773,381.51795018)(124.25842786,381.48795021)(124.11843292,381.45795013)
\curveto(124.05842806,381.43795026)(123.99842812,381.43295026)(123.93843292,381.44295013)
\curveto(123.88842823,381.45295024)(123.82842829,381.44795025)(123.75843292,381.42795013)
\curveto(123.73842838,381.41795028)(123.71342841,381.41795028)(123.68343292,381.42795013)
\curveto(123.65342847,381.42795027)(123.62842849,381.42295027)(123.60843292,381.41295013)
\lineto(123.45843292,381.41295013)
\curveto(123.38842873,381.40295029)(123.33842878,381.40295029)(123.30843292,381.41295013)
\curveto(123.26842885,381.42295027)(123.2234289,381.42795027)(123.17343292,381.42795013)
\curveto(123.13342899,381.41795028)(123.09342903,381.41795028)(123.05343292,381.42795013)
\curveto(122.96342916,381.44795025)(122.87342925,381.46295023)(122.78343292,381.47295013)
\curveto(122.69342943,381.47295022)(122.60342952,381.48295021)(122.51343292,381.50295013)
\curveto(122.4234297,381.53295016)(122.33342979,381.55795014)(122.24343292,381.57795013)
\curveto(122.15342997,381.5979501)(122.06843005,381.62795007)(121.98843292,381.66795013)
\curveto(121.74843037,381.77794992)(121.5234306,381.90794979)(121.31343292,382.05795013)
\curveto(121.10343102,382.21794948)(120.9234312,382.3979493)(120.77343292,382.59795013)
\curveto(120.65343147,382.76794893)(120.54843157,382.94294875)(120.45843292,383.12295013)
\curveto(120.36843175,383.30294839)(120.27843184,383.4929482)(120.18843292,383.69295013)
\curveto(120.14843197,383.7929479)(120.11343201,383.8929478)(120.08343292,383.99295013)
\curveto(120.06343206,384.10294759)(120.03843208,384.21294748)(120.00843292,384.32295013)
\curveto(119.96843215,384.46294723)(119.94343218,384.60294709)(119.93343292,384.74295013)
\curveto(119.9234322,384.88294681)(119.90343222,385.02294667)(119.87343292,385.16295013)
\curveto(119.86343226,385.27294642)(119.85343227,385.37294632)(119.84343292,385.46295013)
\curveto(119.84343228,385.56294613)(119.83343229,385.66294603)(119.81343292,385.76295013)
\lineto(119.81343292,385.85295013)
\curveto(119.8234323,385.88294581)(119.8234323,385.90794579)(119.81343292,385.92795013)
\lineto(119.81343292,386.13795013)
\curveto(119.79343233,386.1979455)(119.78343234,386.26294543)(119.78343292,386.33295013)
\curveto(119.79343233,386.41294528)(119.79843232,386.48794521)(119.79843292,386.55795013)
\lineto(119.79843292,386.70795013)
\curveto(119.79843232,386.75794494)(119.80343232,386.80794489)(119.81343292,386.85795013)
\lineto(119.81343292,387.23295013)
\curveto(119.8234323,387.26294443)(119.8234323,387.2979444)(119.81343292,387.33795013)
\curveto(119.81343231,387.37794432)(119.8184323,387.41794428)(119.82843292,387.45795013)
\curveto(119.84843227,387.56794413)(119.86343226,387.67794402)(119.87343292,387.78795013)
\curveto(119.88343224,387.90794379)(119.89343223,388.02294367)(119.90343292,388.13295013)
\curveto(119.94343218,388.28294341)(119.96843215,388.42794327)(119.97843292,388.56795013)
\curveto(119.99843212,388.71794298)(120.02843209,388.86294283)(120.06843292,389.00295013)
\curveto(120.15843196,389.30294239)(120.25343187,389.58794211)(120.35343292,389.85795013)
\curveto(120.45343167,390.12794157)(120.57843154,390.37794132)(120.72843292,390.60795013)
\curveto(120.92843119,390.92794077)(121.17343095,391.20794049)(121.46343292,391.44795013)
\curveto(121.75343037,391.68794001)(122.09343003,391.87293982)(122.48343292,392.00295013)
\curveto(122.59342953,392.04293965)(122.70342942,392.06793963)(122.81343292,392.07795013)
\curveto(122.93342919,392.0979396)(123.05342907,392.12293957)(123.17343292,392.15295013)
\curveto(123.24342888,392.16293953)(123.30842881,392.16793953)(123.36843292,392.16795013)
\curveto(123.42842869,392.16793953)(123.49342863,392.17293952)(123.56343292,392.18295013)
\curveto(124.26342786,392.20293949)(124.83842728,392.08793961)(125.28843292,391.83795013)
\curveto(125.73842638,391.58794011)(126.08342604,391.23794046)(126.32343292,390.78795013)
\curveto(126.43342569,390.55794114)(126.53342559,390.28294141)(126.62343292,389.96295013)
\curveto(126.64342548,389.8929418)(126.64342548,389.81794188)(126.62343292,389.73795013)
\curveto(126.61342551,389.66794203)(126.58842553,389.61794208)(126.54843292,389.58795013)
\curveto(126.5184256,389.55794214)(126.45842566,389.53294216)(126.36843292,389.51295013)
\curveto(126.27842584,389.50294219)(126.17842594,389.4929422)(126.06843292,389.48295013)
\curveto(125.96842615,389.48294221)(125.86842625,389.48794221)(125.76843292,389.49795013)
\curveto(125.67842644,389.50794219)(125.61342651,389.52794217)(125.57343292,389.55795013)
\curveto(125.46342666,389.62794207)(125.38342674,389.73794196)(125.33343292,389.88795013)
\curveto(125.29342683,390.03794166)(125.23842688,390.16794153)(125.16843292,390.27795013)
\curveto(124.97842714,390.58794111)(124.69842742,390.81794088)(124.32843292,390.96795013)
\curveto(124.25842786,390.9979407)(124.18342794,391.01794068)(124.10343292,391.02795013)
\curveto(124.03342809,391.03794066)(123.95842816,391.05294064)(123.87843292,391.07295013)
\curveto(123.82842829,391.08294061)(123.75842836,391.08794061)(123.66843292,391.08795013)
\curveto(123.58842853,391.08794061)(123.5234286,391.08294061)(123.47343292,391.07295013)
\curveto(123.43342869,391.05294064)(123.39842872,391.04794065)(123.36843292,391.05795013)
\curveto(123.33842878,391.06794063)(123.30342882,391.06794063)(123.26343292,391.05795013)
\lineto(123.02343292,390.99795013)
\curveto(122.95342917,390.97794072)(122.88342924,390.95294074)(122.81343292,390.92295013)
\curveto(122.43342969,390.76294093)(122.14342998,390.55294114)(121.94343292,390.29295013)
\curveto(121.75343037,390.03294166)(121.57843054,389.71794198)(121.41843292,389.34795013)
\curveto(121.38843073,389.26794243)(121.36343076,389.18794251)(121.34343292,389.10795013)
\curveto(121.33343079,389.02794267)(121.31343081,388.94794275)(121.28343292,388.86795013)
\curveto(121.25343087,388.75794294)(121.22843089,388.64294305)(121.20843292,388.52295013)
\curveto(121.19843092,388.40294329)(121.17843094,388.28294341)(121.14843292,388.16295013)
\curveto(121.12843099,388.11294358)(121.118431,388.06294363)(121.11843292,388.01295013)
\curveto(121.12843099,387.96294373)(121.123431,387.91294378)(121.10343292,387.86295013)
\curveto(121.09343103,387.80294389)(121.09343103,387.72294397)(121.10343292,387.62295013)
\curveto(121.11343101,387.53294416)(121.12843099,387.47794422)(121.14843292,387.45795013)
\curveto(121.16843095,387.41794428)(121.19843092,387.3979443)(121.23843292,387.39795013)
\curveto(121.28843083,387.3979443)(121.33343079,387.40794429)(121.37343292,387.42795013)
\curveto(121.44343068,387.46794423)(121.50343062,387.51294418)(121.55343292,387.56295013)
\curveto(121.60343052,387.61294408)(121.66343046,387.66294403)(121.73343292,387.71295013)
\lineto(121.79343292,387.77295013)
\curveto(121.8234303,387.80294389)(121.85343027,387.82794387)(121.88343292,387.84795013)
\curveto(122.11343001,388.00794369)(122.38842973,388.14294355)(122.70843292,388.25295013)
\curveto(122.77842934,388.27294342)(122.84842927,388.28794341)(122.91843292,388.29795013)
\curveto(122.98842913,388.30794339)(123.06342906,388.32294337)(123.14343292,388.34295013)
\curveto(123.18342894,388.34294335)(123.2184289,388.34794335)(123.24843292,388.35795013)
\curveto(123.27842884,388.36794333)(123.31342881,388.36794333)(123.35343292,388.35795013)
\curveto(123.40342872,388.35794334)(123.44342868,388.36794333)(123.47343292,388.38795013)
\lineto(123.63843292,388.38795013)
\lineto(123.72843292,388.38795013)
\curveto(123.77842834,388.3979433)(123.8184283,388.3979433)(123.84843292,388.38795013)
\curveto(123.89842822,388.37794332)(123.94842817,388.37294332)(123.99843292,388.37295013)
\curveto(124.05842806,388.38294331)(124.11342801,388.38294331)(124.16343292,388.37295013)
\curveto(124.27342785,388.34294335)(124.37842774,388.32294337)(124.47843292,388.31295013)
\curveto(124.58842753,388.30294339)(124.69342743,388.27794342)(124.79343292,388.23795013)
\curveto(125.21342691,388.0979436)(125.55842656,387.91294378)(125.82843292,387.68295013)
\curveto(126.09842602,387.46294423)(126.33842578,387.17794452)(126.54843292,386.82795013)
\curveto(126.62842549,386.68794501)(126.69342543,386.53794516)(126.74343292,386.37795013)
\curveto(126.79342533,386.22794547)(126.84342528,386.06794563)(126.89343292,385.89795013)
\moveto(125.64843292,384.59295013)
\curveto(125.65842646,384.64294705)(125.66342646,384.68794701)(125.66343292,384.72795013)
\lineto(125.66343292,384.87795013)
\curveto(125.66342646,385.18794651)(125.6234265,385.47294622)(125.54343292,385.73295013)
\curveto(125.5234266,385.7929459)(125.50342662,385.84794585)(125.48343292,385.89795013)
\curveto(125.47342665,385.95794574)(125.45842666,386.01294568)(125.43843292,386.06295013)
\curveto(125.2184269,386.55294514)(124.87342725,386.90294479)(124.40343292,387.11295013)
\curveto(124.3234278,387.14294455)(124.24342788,387.16794453)(124.16343292,387.18795013)
\lineto(123.92343292,387.24795013)
\curveto(123.84342828,387.26794443)(123.75342837,387.27794442)(123.65343292,387.27795013)
\lineto(123.33843292,387.27795013)
\curveto(123.3184288,387.25794444)(123.27842884,387.24794445)(123.21843292,387.24795013)
\curveto(123.16842895,387.25794444)(123.123429,387.25794444)(123.08343292,387.24795013)
\lineto(122.84343292,387.18795013)
\curveto(122.77342935,387.17794452)(122.70342942,387.15794454)(122.63343292,387.12795013)
\curveto(122.03343009,386.86794483)(121.62843049,386.40294529)(121.41843292,385.73295013)
\curveto(121.38843073,385.65294604)(121.36843075,385.57294612)(121.35843292,385.49295013)
\curveto(121.34843077,385.41294628)(121.33343079,385.32794637)(121.31343292,385.23795013)
\lineto(121.31343292,385.08795013)
\curveto(121.30343082,385.04794665)(121.29843082,384.97794672)(121.29843292,384.87795013)
\curveto(121.29843082,384.64794705)(121.3184308,384.45294724)(121.35843292,384.29295013)
\curveto(121.37843074,384.22294747)(121.39343073,384.15794754)(121.40343292,384.09795013)
\curveto(121.41343071,384.03794766)(121.43343069,383.97294772)(121.46343292,383.90295013)
\curveto(121.57343055,383.62294807)(121.7184304,383.37794832)(121.89843292,383.16795013)
\curveto(122.07843004,382.96794873)(122.31342981,382.80794889)(122.60343292,382.68795013)
\lineto(122.84343292,382.59795013)
\lineto(123.08343292,382.53795013)
\curveto(123.13342899,382.51794918)(123.17342895,382.51294918)(123.20343292,382.52295013)
\curveto(123.24342888,382.53294916)(123.28842883,382.52794917)(123.33843292,382.50795013)
\curveto(123.36842875,382.4979492)(123.4234287,382.4929492)(123.50343292,382.49295013)
\curveto(123.58342854,382.4929492)(123.64342848,382.4979492)(123.68343292,382.50795013)
\curveto(123.79342833,382.52794917)(123.89842822,382.54294915)(123.99843292,382.55295013)
\curveto(124.09842802,382.56294913)(124.19342793,382.5929491)(124.28343292,382.64295013)
\curveto(124.81342731,382.84294885)(125.20342692,383.21794848)(125.45343292,383.76795013)
\curveto(125.49342663,383.86794783)(125.5234266,383.97294772)(125.54343292,384.08295013)
\lineto(125.63343292,384.41295013)
\curveto(125.63342649,384.4929472)(125.63842648,384.55294714)(125.64843292,384.59295013)
}
}
{
\newrgbcolor{curcolor}{0 0 0}
\pscustom[linestyle=none,fillstyle=solid,fillcolor=curcolor]
{
\newpath
\moveto(135.2430423,386.66295013)
\lineto(135.2430423,386.40795013)
\curveto(135.25303459,386.32794537)(135.2480346,386.25294544)(135.2280423,386.18295013)
\lineto(135.2280423,385.94295013)
\lineto(135.2280423,385.77795013)
\curveto(135.20803464,385.67794602)(135.19803465,385.57294612)(135.1980423,385.46295013)
\curveto(135.19803465,385.36294633)(135.18803466,385.26294643)(135.1680423,385.16295013)
\lineto(135.1680423,385.01295013)
\curveto(135.13803471,384.87294682)(135.11803473,384.73294696)(135.1080423,384.59295013)
\curveto(135.09803475,384.46294723)(135.07303477,384.33294736)(135.0330423,384.20295013)
\curveto(135.01303483,384.12294757)(134.99303485,384.03794766)(134.9730423,383.94795013)
\lineto(134.9130423,383.70795013)
\lineto(134.7930423,383.40795013)
\curveto(134.76303508,383.31794838)(134.72803512,383.22794847)(134.6880423,383.13795013)
\curveto(134.58803526,382.91794878)(134.45303539,382.70294899)(134.2830423,382.49295013)
\curveto(134.12303572,382.28294941)(133.9480359,382.11294958)(133.7580423,381.98295013)
\curveto(133.70803614,381.94294975)(133.6480362,381.90294979)(133.5780423,381.86295013)
\curveto(133.51803633,381.83294986)(133.45803639,381.7979499)(133.3980423,381.75795013)
\curveto(133.31803653,381.70794999)(133.22303662,381.66795003)(133.1130423,381.63795013)
\curveto(133.00303684,381.60795009)(132.89803695,381.57795012)(132.7980423,381.54795013)
\curveto(132.68803716,381.50795019)(132.57803727,381.48295021)(132.4680423,381.47295013)
\curveto(132.35803749,381.46295023)(132.2430376,381.44795025)(132.1230423,381.42795013)
\curveto(132.08303776,381.41795028)(132.03803781,381.41795028)(131.9880423,381.42795013)
\curveto(131.9480379,381.42795027)(131.90803794,381.42295027)(131.8680423,381.41295013)
\curveto(131.82803802,381.40295029)(131.77303807,381.3979503)(131.7030423,381.39795013)
\curveto(131.63303821,381.3979503)(131.58303826,381.40295029)(131.5530423,381.41295013)
\curveto(131.50303834,381.43295026)(131.45803839,381.43795026)(131.4180423,381.42795013)
\curveto(131.37803847,381.41795028)(131.3430385,381.41795028)(131.3130423,381.42795013)
\lineto(131.2230423,381.42795013)
\curveto(131.16303868,381.44795025)(131.09803875,381.46295023)(131.0280423,381.47295013)
\curveto(130.96803888,381.47295022)(130.90303894,381.47795022)(130.8330423,381.48795013)
\curveto(130.66303918,381.53795016)(130.50303934,381.58795011)(130.3530423,381.63795013)
\curveto(130.20303964,381.68795001)(130.05803979,381.75294994)(129.9180423,381.83295013)
\curveto(129.86803998,381.87294982)(129.81304003,381.90294979)(129.7530423,381.92295013)
\curveto(129.70304014,381.95294974)(129.65304019,381.98794971)(129.6030423,382.02795013)
\curveto(129.36304048,382.20794949)(129.16304068,382.42794927)(129.0030423,382.68795013)
\curveto(128.843041,382.94794875)(128.70304114,383.23294846)(128.5830423,383.54295013)
\curveto(128.52304132,383.68294801)(128.47804137,383.82294787)(128.4480423,383.96295013)
\curveto(128.41804143,384.11294758)(128.38304146,384.26794743)(128.3430423,384.42795013)
\curveto(128.32304152,384.53794716)(128.30804154,384.64794705)(128.2980423,384.75795013)
\curveto(128.28804156,384.86794683)(128.27304157,384.97794672)(128.2530423,385.08795013)
\curveto(128.2430416,385.12794657)(128.23804161,385.16794653)(128.2380423,385.20795013)
\curveto(128.2480416,385.24794645)(128.2480416,385.28794641)(128.2380423,385.32795013)
\curveto(128.22804162,385.37794632)(128.22304162,385.42794627)(128.2230423,385.47795013)
\lineto(128.2230423,385.64295013)
\curveto(128.20304164,385.692946)(128.19804165,385.74294595)(128.2080423,385.79295013)
\curveto(128.21804163,385.85294584)(128.21804163,385.90794579)(128.2080423,385.95795013)
\curveto(128.19804165,385.9979457)(128.19804165,386.04294565)(128.2080423,386.09295013)
\curveto(128.21804163,386.14294555)(128.21304163,386.1929455)(128.1930423,386.24295013)
\curveto(128.17304167,386.31294538)(128.16804168,386.38794531)(128.1780423,386.46795013)
\curveto(128.18804166,386.55794514)(128.19304165,386.64294505)(128.1930423,386.72295013)
\curveto(128.19304165,386.81294488)(128.18804166,386.91294478)(128.1780423,387.02295013)
\curveto(128.16804168,387.14294455)(128.17304167,387.24294445)(128.1930423,387.32295013)
\lineto(128.1930423,387.60795013)
\lineto(128.2380423,388.23795013)
\curveto(128.2480416,388.33794336)(128.25804159,388.43294326)(128.2680423,388.52295013)
\lineto(128.2980423,388.82295013)
\curveto(128.31804153,388.87294282)(128.32304152,388.92294277)(128.3130423,388.97295013)
\curveto(128.31304153,389.03294266)(128.32304152,389.08794261)(128.3430423,389.13795013)
\curveto(128.39304145,389.30794239)(128.43304141,389.47294222)(128.4630423,389.63295013)
\curveto(128.49304135,389.80294189)(128.5430413,389.96294173)(128.6130423,390.11295013)
\curveto(128.80304104,390.57294112)(129.02304082,390.94794075)(129.2730423,391.23795013)
\curveto(129.53304031,391.52794017)(129.89303995,391.77293992)(130.3530423,391.97295013)
\curveto(130.48303936,392.02293967)(130.61303923,392.05793964)(130.7430423,392.07795013)
\curveto(130.88303896,392.0979396)(131.02303882,392.12293957)(131.1630423,392.15295013)
\curveto(131.23303861,392.16293953)(131.29803855,392.16793953)(131.3580423,392.16795013)
\curveto(131.41803843,392.16793953)(131.48303836,392.17293952)(131.5530423,392.18295013)
\curveto(132.38303746,392.20293949)(133.05303679,392.05293964)(133.5630423,391.73295013)
\curveto(134.07303577,391.42294027)(134.45303539,390.98294071)(134.7030423,390.41295013)
\curveto(134.75303509,390.2929414)(134.79803505,390.16794153)(134.8380423,390.03795013)
\curveto(134.87803497,389.90794179)(134.92303492,389.77294192)(134.9730423,389.63295013)
\curveto(134.99303485,389.55294214)(135.00803484,389.46794223)(135.0180423,389.37795013)
\lineto(135.0780423,389.13795013)
\curveto(135.10803474,389.02794267)(135.12303472,388.91794278)(135.1230423,388.80795013)
\curveto(135.13303471,388.697943)(135.1480347,388.58794311)(135.1680423,388.47795013)
\curveto(135.18803466,388.42794327)(135.19303465,388.38294331)(135.1830423,388.34295013)
\curveto(135.18303466,388.30294339)(135.18803466,388.26294343)(135.1980423,388.22295013)
\curveto(135.20803464,388.17294352)(135.20803464,388.11794358)(135.1980423,388.05795013)
\curveto(135.19803465,388.00794369)(135.20303464,387.95794374)(135.2130423,387.90795013)
\lineto(135.2130423,387.77295013)
\curveto(135.23303461,387.71294398)(135.23303461,387.64294405)(135.2130423,387.56295013)
\curveto(135.20303464,387.4929442)(135.20803464,387.42794427)(135.2280423,387.36795013)
\curveto(135.23803461,387.33794436)(135.2430346,387.2979444)(135.2430423,387.24795013)
\lineto(135.2430423,387.12795013)
\lineto(135.2430423,386.66295013)
\moveto(133.6980423,384.33795013)
\curveto(133.79803605,384.65794704)(133.85803599,385.02294667)(133.8780423,385.43295013)
\curveto(133.89803595,385.84294585)(133.90803594,386.25294544)(133.9080423,386.66295013)
\curveto(133.90803594,387.0929446)(133.89803595,387.51294418)(133.8780423,387.92295013)
\curveto(133.85803599,388.33294336)(133.81303603,388.71794298)(133.7430423,389.07795013)
\curveto(133.67303617,389.43794226)(133.56303628,389.75794194)(133.4130423,390.03795013)
\curveto(133.27303657,390.32794137)(133.07803677,390.56294113)(132.8280423,390.74295013)
\curveto(132.66803718,390.85294084)(132.48803736,390.93294076)(132.2880423,390.98295013)
\curveto(132.08803776,391.04294065)(131.843038,391.07294062)(131.5530423,391.07295013)
\curveto(131.53303831,391.05294064)(131.49803835,391.04294065)(131.4480423,391.04295013)
\curveto(131.39803845,391.05294064)(131.35803849,391.05294064)(131.3280423,391.04295013)
\curveto(131.2480386,391.02294067)(131.17303867,391.00294069)(131.1030423,390.98295013)
\curveto(131.0430388,390.97294072)(130.97803887,390.95294074)(130.9080423,390.92295013)
\curveto(130.63803921,390.80294089)(130.41803943,390.63294106)(130.2480423,390.41295013)
\curveto(130.08803976,390.20294149)(129.95303989,389.95794174)(129.8430423,389.67795013)
\curveto(129.79304005,389.56794213)(129.75304009,389.44794225)(129.7230423,389.31795013)
\curveto(129.70304014,389.1979425)(129.67804017,389.07294262)(129.6480423,388.94295013)
\curveto(129.62804022,388.8929428)(129.61804023,388.83794286)(129.6180423,388.77795013)
\curveto(129.61804023,388.72794297)(129.61304023,388.67794302)(129.6030423,388.62795013)
\curveto(129.59304025,388.53794316)(129.58304026,388.44294325)(129.5730423,388.34295013)
\curveto(129.56304028,388.25294344)(129.55304029,388.15794354)(129.5430423,388.05795013)
\curveto(129.5430403,387.97794372)(129.53804031,387.8929438)(129.5280423,387.80295013)
\lineto(129.5280423,387.56295013)
\lineto(129.5280423,387.38295013)
\curveto(129.51804033,387.35294434)(129.51304033,387.31794438)(129.5130423,387.27795013)
\lineto(129.5130423,387.14295013)
\lineto(129.5130423,386.69295013)
\curveto(129.51304033,386.61294508)(129.50804034,386.52794517)(129.4980423,386.43795013)
\curveto(129.49804035,386.35794534)(129.50804034,386.28294541)(129.5280423,386.21295013)
\lineto(129.5280423,385.94295013)
\curveto(129.52804032,385.92294577)(129.52304032,385.8929458)(129.5130423,385.85295013)
\curveto(129.51304033,385.82294587)(129.51804033,385.7979459)(129.5280423,385.77795013)
\curveto(129.53804031,385.67794602)(129.5430403,385.57794612)(129.5430423,385.47795013)
\curveto(129.55304029,385.38794631)(129.56304028,385.28794641)(129.5730423,385.17795013)
\curveto(129.60304024,385.05794664)(129.61804023,384.93294676)(129.6180423,384.80295013)
\curveto(129.62804022,384.68294701)(129.65304019,384.56794713)(129.6930423,384.45795013)
\curveto(129.77304007,384.15794754)(129.85803999,383.8929478)(129.9480423,383.66295013)
\curveto(130.0480398,383.43294826)(130.19303965,383.21794848)(130.3830423,383.01795013)
\curveto(130.59303925,382.81794888)(130.85803899,382.66794903)(131.1780423,382.56795013)
\curveto(131.21803863,382.54794915)(131.25303859,382.53794916)(131.2830423,382.53795013)
\curveto(131.32303852,382.54794915)(131.36803848,382.54294915)(131.4180423,382.52295013)
\curveto(131.45803839,382.51294918)(131.52803832,382.50294919)(131.6280423,382.49295013)
\curveto(131.73803811,382.48294921)(131.82303802,382.48794921)(131.8830423,382.50795013)
\curveto(131.95303789,382.52794917)(132.02303782,382.53794916)(132.0930423,382.53795013)
\curveto(132.16303768,382.54794915)(132.22803762,382.56294913)(132.2880423,382.58295013)
\curveto(132.48803736,382.64294905)(132.66803718,382.72794897)(132.8280423,382.83795013)
\curveto(132.85803699,382.85794884)(132.88303696,382.87794882)(132.9030423,382.89795013)
\lineto(132.9630423,382.95795013)
\curveto(133.00303684,382.97794872)(133.05303679,383.01794868)(133.1130423,383.07795013)
\curveto(133.21303663,383.21794848)(133.29803655,383.34794835)(133.3680423,383.46795013)
\curveto(133.43803641,383.58794811)(133.50803634,383.73294796)(133.5780423,383.90295013)
\curveto(133.60803624,383.97294772)(133.62803622,384.04294765)(133.6380423,384.11295013)
\curveto(133.65803619,384.18294751)(133.67803617,384.25794744)(133.6980423,384.33795013)
}
}
{
\newrgbcolor{curcolor}{0 0 0}
\pscustom[linestyle=none,fillstyle=solid,fillcolor=curcolor]
{
\newpath
\moveto(143.59265167,386.66295013)
\lineto(143.59265167,386.40795013)
\curveto(143.60264397,386.32794537)(143.59764397,386.25294544)(143.57765167,386.18295013)
\lineto(143.57765167,385.94295013)
\lineto(143.57765167,385.77795013)
\curveto(143.55764401,385.67794602)(143.54764402,385.57294612)(143.54765167,385.46295013)
\curveto(143.54764402,385.36294633)(143.53764403,385.26294643)(143.51765167,385.16295013)
\lineto(143.51765167,385.01295013)
\curveto(143.48764408,384.87294682)(143.4676441,384.73294696)(143.45765167,384.59295013)
\curveto(143.44764412,384.46294723)(143.42264415,384.33294736)(143.38265167,384.20295013)
\curveto(143.36264421,384.12294757)(143.34264423,384.03794766)(143.32265167,383.94795013)
\lineto(143.26265167,383.70795013)
\lineto(143.14265167,383.40795013)
\curveto(143.11264446,383.31794838)(143.07764449,383.22794847)(143.03765167,383.13795013)
\curveto(142.93764463,382.91794878)(142.80264477,382.70294899)(142.63265167,382.49295013)
\curveto(142.4726451,382.28294941)(142.29764527,382.11294958)(142.10765167,381.98295013)
\curveto(142.05764551,381.94294975)(141.99764557,381.90294979)(141.92765167,381.86295013)
\curveto(141.8676457,381.83294986)(141.80764576,381.7979499)(141.74765167,381.75795013)
\curveto(141.6676459,381.70794999)(141.572646,381.66795003)(141.46265167,381.63795013)
\curveto(141.35264622,381.60795009)(141.24764632,381.57795012)(141.14765167,381.54795013)
\curveto(141.03764653,381.50795019)(140.92764664,381.48295021)(140.81765167,381.47295013)
\curveto(140.70764686,381.46295023)(140.59264698,381.44795025)(140.47265167,381.42795013)
\curveto(140.43264714,381.41795028)(140.38764718,381.41795028)(140.33765167,381.42795013)
\curveto(140.29764727,381.42795027)(140.25764731,381.42295027)(140.21765167,381.41295013)
\curveto(140.17764739,381.40295029)(140.12264745,381.3979503)(140.05265167,381.39795013)
\curveto(139.98264759,381.3979503)(139.93264764,381.40295029)(139.90265167,381.41295013)
\curveto(139.85264772,381.43295026)(139.80764776,381.43795026)(139.76765167,381.42795013)
\curveto(139.72764784,381.41795028)(139.69264788,381.41795028)(139.66265167,381.42795013)
\lineto(139.57265167,381.42795013)
\curveto(139.51264806,381.44795025)(139.44764812,381.46295023)(139.37765167,381.47295013)
\curveto(139.31764825,381.47295022)(139.25264832,381.47795022)(139.18265167,381.48795013)
\curveto(139.01264856,381.53795016)(138.85264872,381.58795011)(138.70265167,381.63795013)
\curveto(138.55264902,381.68795001)(138.40764916,381.75294994)(138.26765167,381.83295013)
\curveto(138.21764935,381.87294982)(138.16264941,381.90294979)(138.10265167,381.92295013)
\curveto(138.05264952,381.95294974)(138.00264957,381.98794971)(137.95265167,382.02795013)
\curveto(137.71264986,382.20794949)(137.51265006,382.42794927)(137.35265167,382.68795013)
\curveto(137.19265038,382.94794875)(137.05265052,383.23294846)(136.93265167,383.54295013)
\curveto(136.8726507,383.68294801)(136.82765074,383.82294787)(136.79765167,383.96295013)
\curveto(136.7676508,384.11294758)(136.73265084,384.26794743)(136.69265167,384.42795013)
\curveto(136.6726509,384.53794716)(136.65765091,384.64794705)(136.64765167,384.75795013)
\curveto(136.63765093,384.86794683)(136.62265095,384.97794672)(136.60265167,385.08795013)
\curveto(136.59265098,385.12794657)(136.58765098,385.16794653)(136.58765167,385.20795013)
\curveto(136.59765097,385.24794645)(136.59765097,385.28794641)(136.58765167,385.32795013)
\curveto(136.57765099,385.37794632)(136.572651,385.42794627)(136.57265167,385.47795013)
\lineto(136.57265167,385.64295013)
\curveto(136.55265102,385.692946)(136.54765102,385.74294595)(136.55765167,385.79295013)
\curveto(136.567651,385.85294584)(136.567651,385.90794579)(136.55765167,385.95795013)
\curveto(136.54765102,385.9979457)(136.54765102,386.04294565)(136.55765167,386.09295013)
\curveto(136.567651,386.14294555)(136.56265101,386.1929455)(136.54265167,386.24295013)
\curveto(136.52265105,386.31294538)(136.51765105,386.38794531)(136.52765167,386.46795013)
\curveto(136.53765103,386.55794514)(136.54265103,386.64294505)(136.54265167,386.72295013)
\curveto(136.54265103,386.81294488)(136.53765103,386.91294478)(136.52765167,387.02295013)
\curveto(136.51765105,387.14294455)(136.52265105,387.24294445)(136.54265167,387.32295013)
\lineto(136.54265167,387.60795013)
\lineto(136.58765167,388.23795013)
\curveto(136.59765097,388.33794336)(136.60765096,388.43294326)(136.61765167,388.52295013)
\lineto(136.64765167,388.82295013)
\curveto(136.6676509,388.87294282)(136.6726509,388.92294277)(136.66265167,388.97295013)
\curveto(136.66265091,389.03294266)(136.6726509,389.08794261)(136.69265167,389.13795013)
\curveto(136.74265083,389.30794239)(136.78265079,389.47294222)(136.81265167,389.63295013)
\curveto(136.84265073,389.80294189)(136.89265068,389.96294173)(136.96265167,390.11295013)
\curveto(137.15265042,390.57294112)(137.3726502,390.94794075)(137.62265167,391.23795013)
\curveto(137.88264969,391.52794017)(138.24264933,391.77293992)(138.70265167,391.97295013)
\curveto(138.83264874,392.02293967)(138.96264861,392.05793964)(139.09265167,392.07795013)
\curveto(139.23264834,392.0979396)(139.3726482,392.12293957)(139.51265167,392.15295013)
\curveto(139.58264799,392.16293953)(139.64764792,392.16793953)(139.70765167,392.16795013)
\curveto(139.7676478,392.16793953)(139.83264774,392.17293952)(139.90265167,392.18295013)
\curveto(140.73264684,392.20293949)(141.40264617,392.05293964)(141.91265167,391.73295013)
\curveto(142.42264515,391.42294027)(142.80264477,390.98294071)(143.05265167,390.41295013)
\curveto(143.10264447,390.2929414)(143.14764442,390.16794153)(143.18765167,390.03795013)
\curveto(143.22764434,389.90794179)(143.2726443,389.77294192)(143.32265167,389.63295013)
\curveto(143.34264423,389.55294214)(143.35764421,389.46794223)(143.36765167,389.37795013)
\lineto(143.42765167,389.13795013)
\curveto(143.45764411,389.02794267)(143.4726441,388.91794278)(143.47265167,388.80795013)
\curveto(143.48264409,388.697943)(143.49764407,388.58794311)(143.51765167,388.47795013)
\curveto(143.53764403,388.42794327)(143.54264403,388.38294331)(143.53265167,388.34295013)
\curveto(143.53264404,388.30294339)(143.53764403,388.26294343)(143.54765167,388.22295013)
\curveto(143.55764401,388.17294352)(143.55764401,388.11794358)(143.54765167,388.05795013)
\curveto(143.54764402,388.00794369)(143.55264402,387.95794374)(143.56265167,387.90795013)
\lineto(143.56265167,387.77295013)
\curveto(143.58264399,387.71294398)(143.58264399,387.64294405)(143.56265167,387.56295013)
\curveto(143.55264402,387.4929442)(143.55764401,387.42794427)(143.57765167,387.36795013)
\curveto(143.58764398,387.33794436)(143.59264398,387.2979444)(143.59265167,387.24795013)
\lineto(143.59265167,387.12795013)
\lineto(143.59265167,386.66295013)
\moveto(142.04765167,384.33795013)
\curveto(142.14764542,384.65794704)(142.20764536,385.02294667)(142.22765167,385.43295013)
\curveto(142.24764532,385.84294585)(142.25764531,386.25294544)(142.25765167,386.66295013)
\curveto(142.25764531,387.0929446)(142.24764532,387.51294418)(142.22765167,387.92295013)
\curveto(142.20764536,388.33294336)(142.16264541,388.71794298)(142.09265167,389.07795013)
\curveto(142.02264555,389.43794226)(141.91264566,389.75794194)(141.76265167,390.03795013)
\curveto(141.62264595,390.32794137)(141.42764614,390.56294113)(141.17765167,390.74295013)
\curveto(141.01764655,390.85294084)(140.83764673,390.93294076)(140.63765167,390.98295013)
\curveto(140.43764713,391.04294065)(140.19264738,391.07294062)(139.90265167,391.07295013)
\curveto(139.88264769,391.05294064)(139.84764772,391.04294065)(139.79765167,391.04295013)
\curveto(139.74764782,391.05294064)(139.70764786,391.05294064)(139.67765167,391.04295013)
\curveto(139.59764797,391.02294067)(139.52264805,391.00294069)(139.45265167,390.98295013)
\curveto(139.39264818,390.97294072)(139.32764824,390.95294074)(139.25765167,390.92295013)
\curveto(138.98764858,390.80294089)(138.7676488,390.63294106)(138.59765167,390.41295013)
\curveto(138.43764913,390.20294149)(138.30264927,389.95794174)(138.19265167,389.67795013)
\curveto(138.14264943,389.56794213)(138.10264947,389.44794225)(138.07265167,389.31795013)
\curveto(138.05264952,389.1979425)(138.02764954,389.07294262)(137.99765167,388.94295013)
\curveto(137.97764959,388.8929428)(137.9676496,388.83794286)(137.96765167,388.77795013)
\curveto(137.9676496,388.72794297)(137.96264961,388.67794302)(137.95265167,388.62795013)
\curveto(137.94264963,388.53794316)(137.93264964,388.44294325)(137.92265167,388.34295013)
\curveto(137.91264966,388.25294344)(137.90264967,388.15794354)(137.89265167,388.05795013)
\curveto(137.89264968,387.97794372)(137.88764968,387.8929438)(137.87765167,387.80295013)
\lineto(137.87765167,387.56295013)
\lineto(137.87765167,387.38295013)
\curveto(137.8676497,387.35294434)(137.86264971,387.31794438)(137.86265167,387.27795013)
\lineto(137.86265167,387.14295013)
\lineto(137.86265167,386.69295013)
\curveto(137.86264971,386.61294508)(137.85764971,386.52794517)(137.84765167,386.43795013)
\curveto(137.84764972,386.35794534)(137.85764971,386.28294541)(137.87765167,386.21295013)
\lineto(137.87765167,385.94295013)
\curveto(137.87764969,385.92294577)(137.8726497,385.8929458)(137.86265167,385.85295013)
\curveto(137.86264971,385.82294587)(137.8676497,385.7979459)(137.87765167,385.77795013)
\curveto(137.88764968,385.67794602)(137.89264968,385.57794612)(137.89265167,385.47795013)
\curveto(137.90264967,385.38794631)(137.91264966,385.28794641)(137.92265167,385.17795013)
\curveto(137.95264962,385.05794664)(137.9676496,384.93294676)(137.96765167,384.80295013)
\curveto(137.97764959,384.68294701)(138.00264957,384.56794713)(138.04265167,384.45795013)
\curveto(138.12264945,384.15794754)(138.20764936,383.8929478)(138.29765167,383.66295013)
\curveto(138.39764917,383.43294826)(138.54264903,383.21794848)(138.73265167,383.01795013)
\curveto(138.94264863,382.81794888)(139.20764836,382.66794903)(139.52765167,382.56795013)
\curveto(139.567648,382.54794915)(139.60264797,382.53794916)(139.63265167,382.53795013)
\curveto(139.6726479,382.54794915)(139.71764785,382.54294915)(139.76765167,382.52295013)
\curveto(139.80764776,382.51294918)(139.87764769,382.50294919)(139.97765167,382.49295013)
\curveto(140.08764748,382.48294921)(140.1726474,382.48794921)(140.23265167,382.50795013)
\curveto(140.30264727,382.52794917)(140.3726472,382.53794916)(140.44265167,382.53795013)
\curveto(140.51264706,382.54794915)(140.57764699,382.56294913)(140.63765167,382.58295013)
\curveto(140.83764673,382.64294905)(141.01764655,382.72794897)(141.17765167,382.83795013)
\curveto(141.20764636,382.85794884)(141.23264634,382.87794882)(141.25265167,382.89795013)
\lineto(141.31265167,382.95795013)
\curveto(141.35264622,382.97794872)(141.40264617,383.01794868)(141.46265167,383.07795013)
\curveto(141.56264601,383.21794848)(141.64764592,383.34794835)(141.71765167,383.46795013)
\curveto(141.78764578,383.58794811)(141.85764571,383.73294796)(141.92765167,383.90295013)
\curveto(141.95764561,383.97294772)(141.97764559,384.04294765)(141.98765167,384.11295013)
\curveto(142.00764556,384.18294751)(142.02764554,384.25794744)(142.04765167,384.33795013)
}
}
{
\newrgbcolor{curcolor}{0 0 0}
\pscustom[linestyle=none,fillstyle=solid,fillcolor=curcolor]
{
\newpath
\moveto(839.95647461,374.68897308)
\curveto(840.00646499,374.55897282)(839.98646501,374.45897292)(839.89647461,374.38897308)
\curveto(839.84646515,374.35897302)(839.78146521,374.33897304)(839.70147461,374.32897308)
\lineto(839.47647461,374.32897308)
\lineto(838.99647461,374.32897308)
\curveto(838.83646616,374.32897305)(838.71146628,374.36397302)(838.62147461,374.43397308)
\curveto(838.54146645,374.4839729)(838.48646651,374.55897282)(838.45647461,374.65897308)
\lineto(838.39647461,374.98897308)
\curveto(838.38646661,375.02897235)(838.38146661,375.06397232)(838.38147461,375.09397308)
\lineto(838.38147461,375.19897308)
\curveto(838.36146663,375.24897213)(838.35646664,375.29397209)(838.36647461,375.33397308)
\curveto(838.37646662,375.37397201)(838.37646662,375.41397197)(838.36647461,375.45397308)
\curveto(838.35646664,375.51397187)(838.35146664,375.57397181)(838.35147461,375.63397308)
\lineto(838.35147461,375.81397308)
\lineto(838.30647461,376.48897308)
\curveto(838.28646671,376.55897082)(838.27646672,376.62897075)(838.27647461,376.69897308)
\curveto(838.27646672,376.76897061)(838.26646673,376.84397054)(838.24647461,376.92397308)
\curveto(838.1964668,377.10397028)(838.15646684,377.2839701)(838.12647461,377.46397308)
\curveto(838.10646689,377.64396974)(838.06146693,377.81396957)(837.99147461,377.97397308)
\curveto(837.80146719,378.39396899)(837.48646751,378.67396871)(837.04647461,378.81397308)
\curveto(836.91646808,378.86396852)(836.77146822,378.88896849)(836.61147461,378.88897308)
\curveto(836.46146853,378.89896848)(836.30146869,378.90396848)(836.13147461,378.90397308)
\lineto(833.37147461,378.90397308)
\curveto(833.30147169,378.8839685)(833.23647176,378.86396852)(833.17647461,378.84397308)
\curveto(833.12647187,378.83396855)(833.08147191,378.80396858)(833.04147461,378.75397308)
\curveto(832.97147202,378.65396873)(832.93647206,378.48896889)(832.93647461,378.25897308)
\curveto(832.94647205,378.03896934)(832.95147204,377.84396954)(832.95147461,377.67397308)
\lineto(832.95147461,375.49897308)
\curveto(832.95147204,375.35897202)(832.95647204,375.1839722)(832.96647461,374.97397308)
\curveto(832.97647202,374.77397261)(832.95647204,374.62397276)(832.90647461,374.52397308)
\curveto(832.88647211,374.45397293)(832.84647215,374.40897297)(832.78647461,374.38897308)
\curveto(832.74647225,374.36897301)(832.70647229,374.35897302)(832.66647461,374.35897308)
\curveto(832.63647236,374.35897302)(832.5964724,374.34897303)(832.54647461,374.32897308)
\curveto(832.50647249,374.31897306)(832.46147253,374.31397307)(832.41147461,374.31397308)
\curveto(832.36147263,374.32397306)(832.31147268,374.32897305)(832.26147461,374.32897308)
\lineto(831.93147461,374.32897308)
\curveto(831.83147316,374.33897304)(831.74647325,374.36897301)(831.67647461,374.41897308)
\curveto(831.5964734,374.46897291)(831.55647344,374.55897282)(831.55647461,374.68897308)
\lineto(831.55647461,375.09397308)
\lineto(831.55647461,384.21397308)
\curveto(831.55647344,384.32396306)(831.55147344,384.43896294)(831.54147461,384.55897308)
\curveto(831.54147345,384.6789627)(831.56647343,384.77396261)(831.61647461,384.84397308)
\curveto(831.65647334,384.90396248)(831.73147326,384.95396243)(831.84147461,384.99397308)
\curveto(831.86147313,385.00396238)(831.88147311,385.00396238)(831.90147461,384.99397308)
\curveto(831.92147307,384.99396239)(831.94147305,384.99896238)(831.96147461,385.00897308)
\lineto(836.31147461,385.00897308)
\curveto(836.38146861,385.00896237)(836.45646854,385.00896237)(836.53647461,385.00897308)
\curveto(836.61646838,385.01896236)(836.68646831,385.01896236)(836.74647461,385.00897308)
\lineto(836.91147461,385.00897308)
\curveto(836.97146802,384.99896238)(837.03146796,384.98896239)(837.09147461,384.97897308)
\curveto(837.15146784,384.9789624)(837.21646778,384.97396241)(837.28647461,384.96397308)
\curveto(837.36646763,384.94396244)(837.44646755,384.92896245)(837.52647461,384.91897308)
\curveto(837.61646738,384.90896247)(837.70146729,384.89396249)(837.78147461,384.87397308)
\curveto(837.97146702,384.81396257)(838.14646685,384.74896263)(838.30647461,384.67897308)
\curveto(838.46646653,384.60896277)(838.61646638,384.52396286)(838.75647461,384.42397308)
\curveto(839.00646599,384.25396313)(839.20646579,384.04396334)(839.35647461,383.79397308)
\curveto(839.51646548,383.55396383)(839.64646535,383.26896411)(839.74647461,382.93897308)
\curveto(839.76646523,382.85896452)(839.77646522,382.77396461)(839.77647461,382.68397308)
\curveto(839.78646521,382.60396478)(839.80146519,382.52396486)(839.82147461,382.44397308)
\lineto(839.82147461,382.29397308)
\curveto(839.83146516,382.24396514)(839.83146516,382.1839652)(839.82147461,382.11397308)
\curveto(839.82146517,382.05396533)(839.81646518,381.99896538)(839.80647461,381.94897308)
\lineto(839.80647461,381.78397308)
\curveto(839.78646521,381.70396568)(839.77146522,381.62896575)(839.76147461,381.55897308)
\curveto(839.76146523,381.48896589)(839.75146524,381.41896596)(839.73147461,381.34897308)
\curveto(839.68146531,381.19896618)(839.63146536,381.05396633)(839.58147461,380.91397308)
\curveto(839.54146545,380.7839666)(839.48146551,380.65896672)(839.40147461,380.53897308)
\curveto(839.37146562,380.48896689)(839.33646566,380.44396694)(839.29647461,380.40397308)
\curveto(839.26646573,380.36396702)(839.23646576,380.31896706)(839.20647461,380.26897308)
\lineto(839.17647461,380.23897308)
\curveto(839.16646583,380.23896714)(839.15646584,380.23396715)(839.14647461,380.22397308)
\lineto(839.07147461,380.14897308)
\curveto(839.05146594,380.11896726)(839.03146596,380.09396729)(839.01147461,380.07397308)
\curveto(838.93146606,380.01396737)(838.85646614,379.95396743)(838.78647461,379.89397308)
\curveto(838.71646628,379.84396754)(838.64146635,379.79396759)(838.56147461,379.74397308)
\curveto(838.51146648,379.71396767)(838.46646653,379.6789677)(838.42647461,379.63897308)
\curveto(838.38646661,379.60896777)(838.36146663,379.56396782)(838.35147461,379.50397308)
\curveto(838.34146665,379.44396794)(838.36146663,379.39396799)(838.41147461,379.35397308)
\curveto(838.47146652,379.31396807)(838.52146647,379.2839681)(838.56147461,379.26397308)
\curveto(838.67146632,379.19396819)(838.77146622,379.11896826)(838.86147461,379.03897308)
\curveto(838.96146603,378.95896842)(839.04646595,378.86396852)(839.11647461,378.75397308)
\curveto(839.22646577,378.61396877)(839.30646569,378.45396893)(839.35647461,378.27397308)
\curveto(839.40646559,378.10396928)(839.45646554,377.91896946)(839.50647461,377.71897308)
\lineto(839.53647461,377.47897308)
\curveto(839.54646545,377.40896997)(839.55646544,377.33397005)(839.56647461,377.25397308)
\curveto(839.58646541,377.1839702)(839.5914654,377.11397027)(839.58147461,377.04397308)
\curveto(839.57146542,376.97397041)(839.57646542,376.90397048)(839.59647461,376.83397308)
\lineto(839.59647461,376.69897308)
\curveto(839.61646538,376.62897075)(839.62146537,376.55397083)(839.61147461,376.47397308)
\curveto(839.60146539,376.39397099)(839.60646539,376.31397107)(839.62647461,376.23397308)
\curveto(839.63646536,376.19397119)(839.63646536,376.15397123)(839.62647461,376.11397308)
\curveto(839.62646537,376.0839713)(839.63146536,376.04397134)(839.64147461,375.99397308)
\curveto(839.66146533,375.89397149)(839.67646532,375.78897159)(839.68647461,375.67897308)
\curveto(839.6964653,375.5789718)(839.71646528,375.4839719)(839.74647461,375.39397308)
\curveto(839.76646523,375.33397205)(839.77646522,375.27397211)(839.77647461,375.21397308)
\curveto(839.78646521,375.16397222)(839.80146519,375.10897227)(839.82147461,375.04897308)
\lineto(839.95647461,374.68897308)
\moveto(838.14147461,380.91397308)
\curveto(838.21146678,381.02396636)(838.26146673,381.13896624)(838.29147461,381.25897308)
\curveto(838.33146666,381.378966)(838.36646663,381.50896587)(838.39647461,381.64897308)
\lineto(838.39647461,381.78397308)
\curveto(838.42646657,381.92396546)(838.43146656,382.07396531)(838.41147461,382.23397308)
\curveto(838.3914666,382.40396498)(838.36146663,382.54396484)(838.32147461,382.65397308)
\curveto(838.16146683,383.15396423)(837.84646715,383.49896388)(837.37647461,383.68897308)
\curveto(837.17646782,383.76896361)(836.94146805,383.81396357)(836.67147461,383.82397308)
\curveto(836.41146858,383.83396355)(836.14146885,383.83896354)(835.86147461,383.83897308)
\lineto(833.38647461,383.83897308)
\curveto(833.36647163,383.82896355)(833.34147165,383.82396356)(833.31147461,383.82397308)
\curveto(833.2914717,383.82396356)(833.26647173,383.81896356)(833.23647461,383.80897308)
\curveto(833.11647188,383.7789636)(833.03647196,383.71396367)(832.99647461,383.61397308)
\curveto(832.95647204,383.52396386)(832.93647206,383.39896398)(832.93647461,383.23897308)
\curveto(832.94647205,383.0789643)(832.95147204,382.93396445)(832.95147461,382.80397308)
\lineto(832.95147461,381.07897308)
\curveto(832.95147204,380.92896645)(832.94647205,380.76896661)(832.93647461,380.59897308)
\curveto(832.93647206,380.43896694)(832.97147202,380.31396707)(833.04147461,380.22397308)
\curveto(833.0914719,380.15396723)(833.16647183,380.10896727)(833.26647461,380.08897308)
\curveto(833.36647163,380.0789673)(833.47647152,380.07396731)(833.59647461,380.07397308)
\lineto(834.52647461,380.07397308)
\curveto(834.91647008,380.07396731)(835.2964697,380.06896731)(835.66647461,380.05897308)
\curveto(836.03646896,380.05896732)(836.37646862,380.0789673)(836.68647461,380.11897308)
\curveto(837.00646799,380.16896721)(837.2914677,380.25396713)(837.54147461,380.37397308)
\curveto(837.7914672,380.49396689)(837.991467,380.67396671)(838.14147461,380.91397308)
}
}
{
\newrgbcolor{curcolor}{0 0 0}
\pscustom[linestyle=none,fillstyle=solid,fillcolor=curcolor]
{
\newpath
\moveto(848.33467773,378.48397308)
\curveto(848.35467005,378.383969)(848.35467005,378.26896911)(848.33467773,378.13897308)
\curveto(848.32467008,378.01896936)(848.29467011,377.93396945)(848.24467773,377.88397308)
\curveto(848.19467021,377.84396954)(848.11967028,377.81396957)(848.01967773,377.79397308)
\curveto(847.92967047,377.7839696)(847.82467058,377.7789696)(847.70467773,377.77897308)
\lineto(847.34467773,377.77897308)
\curveto(847.22467118,377.78896959)(847.11967128,377.79396959)(847.02967773,377.79397308)
\lineto(843.18967773,377.79397308)
\curveto(843.10967529,377.79396959)(843.02967537,377.78896959)(842.94967773,377.77897308)
\curveto(842.86967553,377.7789696)(842.8046756,377.76396962)(842.75467773,377.73397308)
\curveto(842.71467569,377.71396967)(842.67467573,377.67396971)(842.63467773,377.61397308)
\curveto(842.61467579,377.5839698)(842.59467581,377.53896984)(842.57467773,377.47897308)
\curveto(842.55467585,377.42896995)(842.55467585,377.37897)(842.57467773,377.32897308)
\curveto(842.58467582,377.2789701)(842.58967581,377.23397015)(842.58967773,377.19397308)
\curveto(842.58967581,377.15397023)(842.59467581,377.11397027)(842.60467773,377.07397308)
\curveto(842.62467578,376.99397039)(842.64467576,376.90897047)(842.66467773,376.81897308)
\curveto(842.68467572,376.73897064)(842.71467569,376.65897072)(842.75467773,376.57897308)
\curveto(842.98467542,376.03897134)(843.36467504,375.65397173)(843.89467773,375.42397308)
\curveto(843.95467445,375.39397199)(844.01967438,375.36897201)(844.08967773,375.34897308)
\lineto(844.29967773,375.28897308)
\curveto(844.32967407,375.2789721)(844.37967402,375.27397211)(844.44967773,375.27397308)
\curveto(844.58967381,375.23397215)(844.77467363,375.21397217)(845.00467773,375.21397308)
\curveto(845.23467317,375.21397217)(845.41967298,375.23397215)(845.55967773,375.27397308)
\curveto(845.6996727,375.31397207)(845.82467258,375.35397203)(845.93467773,375.39397308)
\curveto(846.05467235,375.44397194)(846.16467224,375.50397188)(846.26467773,375.57397308)
\curveto(846.37467203,375.64397174)(846.46967193,375.72397166)(846.54967773,375.81397308)
\curveto(846.62967177,375.91397147)(846.6996717,376.01897136)(846.75967773,376.12897308)
\curveto(846.81967158,376.22897115)(846.86967153,376.33397105)(846.90967773,376.44397308)
\curveto(846.95967144,376.55397083)(847.03967136,376.63397075)(847.14967773,376.68397308)
\curveto(847.18967121,376.70397068)(847.25467115,376.71897066)(847.34467773,376.72897308)
\curveto(847.43467097,376.73897064)(847.52467088,376.73897064)(847.61467773,376.72897308)
\curveto(847.7046707,376.72897065)(847.78967061,376.72397066)(847.86967773,376.71397308)
\curveto(847.94967045,376.70397068)(848.0046704,376.6839707)(848.03467773,376.65397308)
\curveto(848.13467027,376.5839708)(848.15967024,376.46897091)(848.10967773,376.30897308)
\curveto(848.02967037,376.03897134)(847.92467048,375.79897158)(847.79467773,375.58897308)
\curveto(847.59467081,375.26897211)(847.36467104,375.00397238)(847.10467773,374.79397308)
\curveto(846.85467155,374.59397279)(846.53467187,374.42897295)(846.14467773,374.29897308)
\curveto(846.04467236,374.25897312)(845.94467246,374.23397315)(845.84467773,374.22397308)
\curveto(845.74467266,374.20397318)(845.63967276,374.1839732)(845.52967773,374.16397308)
\curveto(845.47967292,374.15397323)(845.42967297,374.14897323)(845.37967773,374.14897308)
\curveto(845.33967306,374.14897323)(845.29467311,374.14397324)(845.24467773,374.13397308)
\lineto(845.09467773,374.13397308)
\curveto(845.04467336,374.12397326)(844.98467342,374.11897326)(844.91467773,374.11897308)
\curveto(844.85467355,374.11897326)(844.8046736,374.12397326)(844.76467773,374.13397308)
\lineto(844.62967773,374.13397308)
\curveto(844.57967382,374.14397324)(844.53467387,374.14897323)(844.49467773,374.14897308)
\curveto(844.45467395,374.14897323)(844.41467399,374.15397323)(844.37467773,374.16397308)
\curveto(844.32467408,374.17397321)(844.26967413,374.1839732)(844.20967773,374.19397308)
\curveto(844.14967425,374.19397319)(844.09467431,374.19897318)(844.04467773,374.20897308)
\curveto(843.95467445,374.22897315)(843.86467454,374.25397313)(843.77467773,374.28397308)
\curveto(843.68467472,374.30397308)(843.5996748,374.32897305)(843.51967773,374.35897308)
\curveto(843.47967492,374.378973)(843.44467496,374.38897299)(843.41467773,374.38897308)
\curveto(843.38467502,374.39897298)(843.34967505,374.41397297)(843.30967773,374.43397308)
\curveto(843.15967524,374.50397288)(842.9996754,374.58897279)(842.82967773,374.68897308)
\curveto(842.53967586,374.8789725)(842.28967611,375.10897227)(842.07967773,375.37897308)
\curveto(841.87967652,375.65897172)(841.70967669,375.96897141)(841.56967773,376.30897308)
\curveto(841.51967688,376.41897096)(841.47967692,376.53397085)(841.44967773,376.65397308)
\curveto(841.42967697,376.77397061)(841.399677,376.89397049)(841.35967773,377.01397308)
\curveto(841.34967705,377.05397033)(841.34467706,377.08897029)(841.34467773,377.11897308)
\curveto(841.34467706,377.14897023)(841.33967706,377.18897019)(841.32967773,377.23897308)
\curveto(841.30967709,377.31897006)(841.29467711,377.40396998)(841.28467773,377.49397308)
\curveto(841.27467713,377.5839698)(841.25967714,377.67396971)(841.23967773,377.76397308)
\lineto(841.23967773,377.97397308)
\curveto(841.22967717,378.01396937)(841.21967718,378.06896931)(841.20967773,378.13897308)
\curveto(841.20967719,378.21896916)(841.21467719,378.2839691)(841.22467773,378.33397308)
\lineto(841.22467773,378.49897308)
\curveto(841.24467716,378.54896883)(841.24967715,378.59896878)(841.23967773,378.64897308)
\curveto(841.23967716,378.70896867)(841.24467716,378.76396862)(841.25467773,378.81397308)
\curveto(841.29467711,378.97396841)(841.32467708,379.13396825)(841.34467773,379.29397308)
\curveto(841.37467703,379.45396793)(841.41967698,379.60396778)(841.47967773,379.74397308)
\curveto(841.52967687,379.85396753)(841.57467683,379.96396742)(841.61467773,380.07397308)
\curveto(841.66467674,380.19396719)(841.71967668,380.30896707)(841.77967773,380.41897308)
\curveto(841.9996764,380.76896661)(842.24967615,381.06896631)(842.52967773,381.31897308)
\curveto(842.80967559,381.5789658)(843.15467525,381.79396559)(843.56467773,381.96397308)
\curveto(843.68467472,382.01396537)(843.8046746,382.04896533)(843.92467773,382.06897308)
\curveto(844.05467435,382.09896528)(844.18967421,382.12896525)(844.32967773,382.15897308)
\curveto(844.37967402,382.16896521)(844.42467398,382.17396521)(844.46467773,382.17397308)
\curveto(844.5046739,382.1839652)(844.54967385,382.18896519)(844.59967773,382.18897308)
\curveto(844.61967378,382.19896518)(844.64467376,382.19896518)(844.67467773,382.18897308)
\curveto(844.7046737,382.1789652)(844.72967367,382.1839652)(844.74967773,382.20397308)
\curveto(845.16967323,382.21396517)(845.53467287,382.16896521)(845.84467773,382.06897308)
\curveto(846.15467225,381.9789654)(846.43467197,381.85396553)(846.68467773,381.69397308)
\curveto(846.73467167,381.67396571)(846.77467163,381.64396574)(846.80467773,381.60397308)
\curveto(846.83467157,381.57396581)(846.86967153,381.54896583)(846.90967773,381.52897308)
\curveto(846.98967141,381.46896591)(847.06967133,381.39896598)(847.14967773,381.31897308)
\curveto(847.23967116,381.23896614)(847.31467109,381.15896622)(847.37467773,381.07897308)
\curveto(847.53467087,380.86896651)(847.66967073,380.66896671)(847.77967773,380.47897308)
\curveto(847.84967055,380.36896701)(847.9046705,380.24896713)(847.94467773,380.11897308)
\curveto(847.98467042,379.98896739)(848.02967037,379.85896752)(848.07967773,379.72897308)
\curveto(848.12967027,379.59896778)(848.16467024,379.46396792)(848.18467773,379.32397308)
\curveto(848.21467019,379.1839682)(848.24967015,379.04396834)(848.28967773,378.90397308)
\curveto(848.2996701,378.83396855)(848.3046701,378.76396862)(848.30467773,378.69397308)
\lineto(848.33467773,378.48397308)
\moveto(846.87967773,378.99397308)
\curveto(846.90967149,379.03396835)(846.93467147,379.0839683)(846.95467773,379.14397308)
\curveto(846.97467143,379.21396817)(846.97467143,379.2839681)(846.95467773,379.35397308)
\curveto(846.89467151,379.57396781)(846.80967159,379.7789676)(846.69967773,379.96897308)
\curveto(846.55967184,380.19896718)(846.404672,380.39396699)(846.23467773,380.55397308)
\curveto(846.06467234,380.71396667)(845.84467256,380.84896653)(845.57467773,380.95897308)
\curveto(845.5046729,380.9789664)(845.43467297,380.99396639)(845.36467773,381.00397308)
\curveto(845.29467311,381.02396636)(845.21967318,381.04396634)(845.13967773,381.06397308)
\curveto(845.05967334,381.0839663)(844.97467343,381.09396629)(844.88467773,381.09397308)
\lineto(844.62967773,381.09397308)
\curveto(844.5996738,381.07396631)(844.56467384,381.06396632)(844.52467773,381.06397308)
\curveto(844.48467392,381.07396631)(844.44967395,381.07396631)(844.41967773,381.06397308)
\lineto(844.17967773,381.00397308)
\curveto(844.10967429,380.99396639)(844.03967436,380.9789664)(843.96967773,380.95897308)
\curveto(843.67967472,380.83896654)(843.44467496,380.68896669)(843.26467773,380.50897308)
\curveto(843.09467531,380.32896705)(842.93967546,380.10396728)(842.79967773,379.83397308)
\curveto(842.76967563,379.7839676)(842.73967566,379.71896766)(842.70967773,379.63897308)
\curveto(842.67967572,379.56896781)(842.65467575,379.48896789)(842.63467773,379.39897308)
\curveto(842.61467579,379.30896807)(842.60967579,379.22396816)(842.61967773,379.14397308)
\curveto(842.62967577,379.06396832)(842.66467574,379.00396838)(842.72467773,378.96397308)
\curveto(842.8046756,378.90396848)(842.93967546,378.87396851)(843.12967773,378.87397308)
\curveto(843.32967507,378.8839685)(843.4996749,378.88896849)(843.63967773,378.88897308)
\lineto(845.91967773,378.88897308)
\curveto(846.06967233,378.88896849)(846.24967215,378.8839685)(846.45967773,378.87397308)
\curveto(846.66967173,378.87396851)(846.80967159,378.91396847)(846.87967773,378.99397308)
}
}
{
\newrgbcolor{curcolor}{0 0 0}
\pscustom[linestyle=none,fillstyle=solid,fillcolor=curcolor]
{
\newpath
\moveto(856.25631836,381.93397308)
\curveto(856.32631076,381.8839655)(856.36131072,381.80896557)(856.36131836,381.70897308)
\curveto(856.37131071,381.60896577)(856.37631071,381.50396588)(856.37631836,381.39397308)
\lineto(856.37631836,375.12397308)
\lineto(856.37631836,374.52397308)
\curveto(856.35631073,374.47397291)(856.35131073,374.42397296)(856.36131836,374.37397308)
\curveto(856.37131071,374.33397305)(856.36631072,374.28897309)(856.34631836,374.23897308)
\curveto(856.32631076,374.13897324)(856.31131077,374.03897334)(856.30131836,373.93897308)
\curveto(856.30131078,373.82897355)(856.2863108,373.72397366)(856.25631836,373.62397308)
\curveto(856.22631086,373.51397387)(856.19631089,373.40897397)(856.16631836,373.30897308)
\curveto(856.14631094,373.20897417)(856.11131097,373.10897427)(856.06131836,373.00897308)
\curveto(855.96131112,372.74897463)(855.83131125,372.51397487)(855.67131836,372.30397308)
\curveto(855.52131156,372.09397529)(855.34131174,371.91897546)(855.13131836,371.77897308)
\curveto(854.96131212,371.65897572)(854.7813123,371.56397582)(854.59131836,371.49397308)
\curveto(854.40131268,371.41397597)(854.19631289,371.33897604)(853.97631836,371.26897308)
\curveto(853.8863132,371.24897613)(853.79631329,371.23897614)(853.70631836,371.23897308)
\curveto(853.61631347,371.22897615)(853.52631356,371.21397617)(853.43631836,371.19397308)
\lineto(853.34631836,371.19397308)
\curveto(853.32631376,371.1839762)(853.30631378,371.1789762)(853.28631836,371.17897308)
\curveto(853.23631385,371.16897621)(853.1863139,371.16897621)(853.13631836,371.17897308)
\curveto(853.09631399,371.18897619)(853.05131403,371.1839762)(853.00131836,371.16397308)
\curveto(852.93131415,371.14397624)(852.82131426,371.13897624)(852.67131836,371.14897308)
\curveto(852.53131455,371.14897623)(852.43131465,371.15897622)(852.37131836,371.17897308)
\curveto(852.34131474,371.1789762)(852.31131477,371.1839762)(852.28131836,371.19397308)
\lineto(852.22131836,371.19397308)
\curveto(852.13131495,371.21397617)(852.04131504,371.22897615)(851.95131836,371.23897308)
\curveto(851.86131522,371.23897614)(851.77631531,371.24897613)(851.69631836,371.26897308)
\curveto(851.61631547,371.28897609)(851.53631555,371.31397607)(851.45631836,371.34397308)
\curveto(851.37631571,371.36397602)(851.29631579,371.38897599)(851.21631836,371.41897308)
\curveto(850.89631619,371.54897583)(850.62631646,371.69397569)(850.40631836,371.85397308)
\curveto(850.19631689,372.01397537)(850.00631708,372.23897514)(849.83631836,372.52897308)
\curveto(849.81631727,372.54897483)(849.80131728,372.57397481)(849.79131836,372.60397308)
\curveto(849.79131729,372.62397476)(849.7813173,372.64897473)(849.76131836,372.67897308)
\curveto(849.73131735,372.75897462)(849.69631739,372.87397451)(849.65631836,373.02397308)
\curveto(849.62631746,373.16397422)(849.65631743,373.26897411)(849.74631836,373.33897308)
\curveto(849.80631728,373.38897399)(849.8863172,373.41397397)(849.98631836,373.41397308)
\lineto(850.31631836,373.41397308)
\lineto(850.48131836,373.41397308)
\curveto(850.54131654,373.41397397)(850.59631649,373.40397398)(850.64631836,373.38397308)
\curveto(850.73631635,373.35397403)(850.80131628,373.30397408)(850.84131836,373.23397308)
\curveto(850.8813162,373.16397422)(850.92631616,373.08897429)(850.97631836,373.00897308)
\lineto(851.09631836,372.82897308)
\curveto(851.14631594,372.75897462)(851.19631589,372.70397468)(851.24631836,372.66397308)
\curveto(851.49631559,372.47397491)(851.79631529,372.33397505)(852.14631836,372.24397308)
\curveto(852.20631488,372.22397516)(852.26631482,372.21397517)(852.32631836,372.21397308)
\curveto(852.39631469,372.20397518)(852.46131462,372.18897519)(852.52131836,372.16897308)
\lineto(852.61131836,372.16897308)
\curveto(852.6813144,372.14897523)(852.76631432,372.13897524)(852.86631836,372.13897308)
\curveto(852.96631412,372.13897524)(853.05631403,372.14897523)(853.13631836,372.16897308)
\curveto(853.16631392,372.1789752)(853.20631388,372.1839752)(853.25631836,372.18397308)
\curveto(853.35631373,372.20397518)(853.45131363,372.22397516)(853.54131836,372.24397308)
\curveto(853.63131345,372.25397513)(853.71631337,372.2789751)(853.79631836,372.31897308)
\curveto(854.086313,372.43897494)(854.32131276,372.60397478)(854.50131836,372.81397308)
\curveto(854.69131239,373.01397437)(854.84631224,373.25897412)(854.96631836,373.54897308)
\curveto(855.00631208,373.63897374)(855.03131205,373.73397365)(855.04131836,373.83397308)
\curveto(855.06131202,373.93397345)(855.086312,374.03897334)(855.11631836,374.14897308)
\curveto(855.13631195,374.19897318)(855.14631194,374.24897313)(855.14631836,374.29897308)
\curveto(855.14631194,374.34897303)(855.15131193,374.39897298)(855.16131836,374.44897308)
\curveto(855.17131191,374.4789729)(855.17631191,374.52897285)(855.17631836,374.59897308)
\curveto(855.19631189,374.6789727)(855.19631189,374.76397262)(855.17631836,374.85397308)
\curveto(855.16631192,374.90397248)(855.16131192,374.94897243)(855.16131836,374.98897308)
\curveto(855.17131191,375.02897235)(855.16631192,375.06397232)(855.14631836,375.09397308)
\curveto(855.12631196,375.11397227)(855.11131197,375.12397226)(855.10131836,375.12397308)
\lineto(855.05631836,375.16897308)
\curveto(854.95631213,375.16897221)(854.8813122,375.13897224)(854.83131836,375.07897308)
\curveto(854.79131229,375.02897235)(854.74131234,374.9839724)(854.68131836,374.94397308)
\lineto(854.44131836,374.73397308)
\curveto(854.36131272,374.67397271)(854.27131281,374.61897276)(854.17131836,374.56897308)
\curveto(854.03131305,374.4789729)(853.85631323,374.40397298)(853.64631836,374.34397308)
\curveto(853.43631365,374.29397309)(853.21631387,374.25897312)(852.98631836,374.23897308)
\curveto(852.75631433,374.21897316)(852.52631456,374.22397316)(852.29631836,374.25397308)
\curveto(852.06631502,374.27397311)(851.85631523,374.31397307)(851.66631836,374.37397308)
\curveto(850.72631636,374.6839727)(850.06631702,375.2789721)(849.68631836,376.15897308)
\curveto(849.63631745,376.25897112)(849.59631749,376.35397103)(849.56631836,376.44397308)
\curveto(849.53631755,376.54397084)(849.50131758,376.64897073)(849.46131836,376.75897308)
\curveto(849.44131764,376.80897057)(849.43131765,376.85397053)(849.43131836,376.89397308)
\curveto(849.43131765,376.93397045)(849.42131766,376.9789704)(849.40131836,377.02897308)
\curveto(849.3813177,377.09897028)(849.36631772,377.16897021)(849.35631836,377.23897308)
\curveto(849.35631773,377.31897006)(849.34631774,377.39396999)(849.32631836,377.46397308)
\curveto(849.31631777,377.50396988)(849.31131777,377.53896984)(849.31131836,377.56897308)
\curveto(849.32131776,377.60896977)(849.32131776,377.64896973)(849.31131836,377.68897308)
\curveto(849.31131777,377.72896965)(849.30631778,377.76896961)(849.29631836,377.80897308)
\lineto(849.29631836,377.92897308)
\curveto(849.27631781,378.04896933)(849.27631781,378.17396921)(849.29631836,378.30397308)
\curveto(849.30631778,378.36396902)(849.31131777,378.42396896)(849.31131836,378.48397308)
\lineto(849.31131836,378.64897308)
\curveto(849.32131776,378.69896868)(849.32631776,378.73896864)(849.32631836,378.76897308)
\curveto(849.32631776,378.80896857)(849.33131775,378.85396853)(849.34131836,378.90397308)
\curveto(849.37131771,379.01396837)(849.39131769,379.11896826)(849.40131836,379.21897308)
\curveto(849.41131767,379.32896805)(849.43631765,379.43896794)(849.47631836,379.54897308)
\curveto(849.51631757,379.66896771)(849.55131753,379.7839676)(849.58131836,379.89397308)
\curveto(849.62131746,380.01396737)(849.66631742,380.12896725)(849.71631836,380.23897308)
\curveto(849.7863173,380.39896698)(849.86631722,380.54396684)(849.95631836,380.67397308)
\curveto(850.04631704,380.81396657)(850.14131694,380.94896643)(850.24131836,381.07897308)
\curveto(850.31131677,381.18896619)(850.40131668,381.2789661)(850.51131836,381.34897308)
\lineto(850.57131836,381.40897308)
\lineto(850.63131836,381.46897308)
\lineto(850.78131836,381.58897308)
\lineto(850.96131836,381.70897308)
\curveto(851.09131599,381.78896559)(851.22631586,381.85896552)(851.36631836,381.91897308)
\curveto(851.51631557,381.9789654)(851.67631541,382.03396535)(851.84631836,382.08397308)
\curveto(851.94631514,382.11396527)(852.04631504,382.13396525)(852.14631836,382.14397308)
\curveto(852.25631483,382.15396523)(852.36631472,382.16896521)(852.47631836,382.18897308)
\curveto(852.51631457,382.19896518)(852.56631452,382.19896518)(852.62631836,382.18897308)
\curveto(852.69631439,382.1789652)(852.74631434,382.1839652)(852.77631836,382.20397308)
\curveto(853.09631399,382.21396517)(853.3813137,382.1839652)(853.63131836,382.11397308)
\curveto(853.89131319,382.04396534)(854.12131296,381.94396544)(854.32131836,381.81397308)
\curveto(854.39131269,381.77396561)(854.45631263,381.72896565)(854.51631836,381.67897308)
\lineto(854.69631836,381.52897308)
\curveto(854.74631234,381.48896589)(854.79131229,381.44396594)(854.83131836,381.39397308)
\curveto(854.8813122,381.35396603)(854.95631213,381.33396605)(855.05631836,381.33397308)
\lineto(855.10131836,381.37897308)
\curveto(855.12131196,381.39896598)(855.14131194,381.42396596)(855.16131836,381.45397308)
\curveto(855.19131189,381.53396585)(855.20631188,381.61396577)(855.20631836,381.69397308)
\curveto(855.21631187,381.77396561)(855.24631184,381.84396554)(855.29631836,381.90397308)
\curveto(855.32631176,381.94396544)(855.3863117,381.97396541)(855.47631836,381.99397308)
\curveto(855.56631152,382.02396536)(855.66131142,382.03896534)(855.76131836,382.03897308)
\curveto(855.86131122,382.03896534)(855.95631113,382.02896535)(856.04631836,382.00897308)
\curveto(856.14631094,381.98896539)(856.21631087,381.96396542)(856.25631836,381.93397308)
\moveto(855.13131836,378.15397308)
\curveto(855.14131194,378.19396919)(855.14631194,378.24396914)(855.14631836,378.30397308)
\curveto(855.14631194,378.37396901)(855.14131194,378.42896895)(855.13131836,378.46897308)
\lineto(855.13131836,378.70897308)
\curveto(855.11131197,378.79896858)(855.09631199,378.8839685)(855.08631836,378.96397308)
\curveto(855.07631201,379.05396833)(855.06131202,379.13896824)(855.04131836,379.21897308)
\curveto(855.02131206,379.29896808)(855.00131208,379.37396801)(854.98131836,379.44397308)
\curveto(854.97131211,379.52396786)(854.95131213,379.59896778)(854.92131836,379.66897308)
\curveto(854.81131227,379.94896743)(854.66631242,380.19896718)(854.48631836,380.41897308)
\curveto(854.31631277,380.63896674)(854.09631299,380.80396658)(853.82631836,380.91397308)
\curveto(853.74631334,380.95396643)(853.66131342,380.9839664)(853.57131836,381.00397308)
\curveto(853.4813136,381.03396635)(853.3863137,381.05896632)(853.28631836,381.07897308)
\curveto(853.20631388,381.09896628)(853.11631397,381.10396628)(853.01631836,381.09397308)
\lineto(852.74631836,381.09397308)
\curveto(852.69631439,381.0839663)(852.64631444,381.0789663)(852.59631836,381.07897308)
\curveto(852.55631453,381.0789663)(852.51131457,381.07396631)(852.46131836,381.06397308)
\curveto(852.27131481,381.01396637)(852.11131497,380.96396642)(851.98131836,380.91397308)
\curveto(851.64131544,380.77396661)(851.37631571,380.56396682)(851.18631836,380.28397308)
\curveto(850.99631609,380.00396738)(850.84631624,379.6789677)(850.73631836,379.30897308)
\curveto(850.71631637,379.22896815)(850.70131638,379.14896823)(850.69131836,379.06897308)
\curveto(850.69131639,378.99896838)(850.6813164,378.92396846)(850.66131836,378.84397308)
\curveto(850.64131644,378.81396857)(850.63131645,378.7789686)(850.63131836,378.73897308)
\curveto(850.64131644,378.69896868)(850.64131644,378.66396872)(850.63131836,378.63397308)
\lineto(850.63131836,378.30397308)
\lineto(850.63131836,377.95897308)
\curveto(850.63131645,377.84896953)(850.64131644,377.74396964)(850.66131836,377.64397308)
\lineto(850.66131836,377.56897308)
\curveto(850.67131641,377.53896984)(850.67631641,377.51396987)(850.67631836,377.49397308)
\curveto(850.69631639,377.40396998)(850.71131637,377.31397007)(850.72131836,377.22397308)
\curveto(850.74131634,377.13397025)(850.76631632,377.04897033)(850.79631836,376.96897308)
\curveto(850.87631621,376.70897067)(850.97631611,376.46897091)(851.09631836,376.24897308)
\curveto(851.21631587,376.02897135)(851.37631571,375.84897153)(851.57631836,375.70897308)
\lineto(851.69631836,375.61897308)
\curveto(851.73631535,375.59897178)(851.7813153,375.5789718)(851.83131836,375.55897308)
\curveto(851.91131517,375.50897187)(851.99631509,375.46897191)(852.08631836,375.43897308)
\curveto(852.17631491,375.40897197)(852.27631481,375.378972)(852.38631836,375.34897308)
\curveto(852.43631465,375.33897204)(852.4813146,375.33397205)(852.52131836,375.33397308)
\curveto(852.57131451,375.34397204)(852.62131446,375.33897204)(852.67131836,375.31897308)
\curveto(852.70131438,375.30897207)(852.75131433,375.30397208)(852.82131836,375.30397308)
\curveto(852.89131419,375.30397208)(852.94131414,375.30897207)(852.97131836,375.31897308)
\curveto(853.00131408,375.32897205)(853.03131405,375.32897205)(853.06131836,375.31897308)
\curveto(853.10131398,375.31897206)(853.14131394,375.32397206)(853.18131836,375.33397308)
\curveto(853.27131381,375.35397203)(853.35631373,375.37397201)(853.43631836,375.39397308)
\curveto(853.51631357,375.41397197)(853.59631349,375.43897194)(853.67631836,375.46897308)
\curveto(854.01631307,375.61897176)(854.2863128,375.82897155)(854.48631836,376.09897308)
\curveto(854.6863124,376.36897101)(854.84631224,376.6839707)(854.96631836,377.04397308)
\curveto(854.99631209,377.13397025)(855.01631207,377.22397016)(855.02631836,377.31397308)
\curveto(855.04631204,377.41396997)(855.06631202,377.50896987)(855.08631836,377.59897308)
\curveto(855.09631199,377.63896974)(855.10131198,377.67396971)(855.10131836,377.70397308)
\curveto(855.10131198,377.74396964)(855.10631198,377.7839696)(855.11631836,377.82397308)
\curveto(855.13631195,377.87396951)(855.13631195,377.92396946)(855.11631836,377.97397308)
\curveto(855.10631198,378.03396935)(855.11131197,378.09396929)(855.13131836,378.15397308)
}
}
{
\newrgbcolor{curcolor}{0 0 0}
\pscustom[linestyle=none,fillstyle=solid,fillcolor=curcolor]
{
\newpath
\moveto(858.53959961,383.53897308)
\curveto(858.45959849,383.59896378)(858.41459853,383.70396368)(858.40459961,383.85397308)
\lineto(858.40459961,384.31897308)
\lineto(858.40459961,384.57397308)
\curveto(858.40459854,384.66396272)(858.41959853,384.73896264)(858.44959961,384.79897308)
\curveto(858.48959846,384.8789625)(858.56959838,384.93896244)(858.68959961,384.97897308)
\curveto(858.70959824,384.98896239)(858.72959822,384.98896239)(858.74959961,384.97897308)
\curveto(858.77959817,384.9789624)(858.80459814,384.9839624)(858.82459961,384.99397308)
\curveto(858.99459795,384.99396239)(859.15459779,384.98896239)(859.30459961,384.97897308)
\curveto(859.45459749,384.96896241)(859.55459739,384.90896247)(859.60459961,384.79897308)
\curveto(859.63459731,384.73896264)(859.6495973,384.66396272)(859.64959961,384.57397308)
\lineto(859.64959961,384.31897308)
\curveto(859.6495973,384.13896324)(859.6445973,383.96896341)(859.63459961,383.80897308)
\curveto(859.63459731,383.64896373)(859.56959738,383.54396384)(859.43959961,383.49397308)
\curveto(859.38959756,383.47396391)(859.33459761,383.46396392)(859.27459961,383.46397308)
\lineto(859.10959961,383.46397308)
\lineto(858.79459961,383.46397308)
\curveto(858.69459825,383.46396392)(858.60959834,383.48896389)(858.53959961,383.53897308)
\moveto(859.64959961,375.03397308)
\lineto(859.64959961,374.71897308)
\curveto(859.65959729,374.61897276)(859.63959731,374.53897284)(859.58959961,374.47897308)
\curveto(859.55959739,374.41897296)(859.51459743,374.378973)(859.45459961,374.35897308)
\curveto(859.39459755,374.34897303)(859.32459762,374.33397305)(859.24459961,374.31397308)
\lineto(859.01959961,374.31397308)
\curveto(858.88959806,374.31397307)(858.77459817,374.31897306)(858.67459961,374.32897308)
\curveto(858.58459836,374.34897303)(858.51459843,374.39897298)(858.46459961,374.47897308)
\curveto(858.42459852,374.53897284)(858.40459854,374.61397277)(858.40459961,374.70397308)
\lineto(858.40459961,374.98897308)
\lineto(858.40459961,381.33397308)
\lineto(858.40459961,381.64897308)
\curveto(858.40459854,381.75896562)(858.42959852,381.84396554)(858.47959961,381.90397308)
\curveto(858.50959844,381.95396543)(858.5495984,381.9839654)(858.59959961,381.99397308)
\curveto(858.6495983,382.00396538)(858.70459824,382.01896536)(858.76459961,382.03897308)
\curveto(858.78459816,382.03896534)(858.80459814,382.03396535)(858.82459961,382.02397308)
\curveto(858.85459809,382.02396536)(858.87959807,382.02896535)(858.89959961,382.03897308)
\curveto(859.02959792,382.03896534)(859.15959779,382.03396535)(859.28959961,382.02397308)
\curveto(859.42959752,382.02396536)(859.52459742,381.9839654)(859.57459961,381.90397308)
\curveto(859.62459732,381.84396554)(859.6495973,381.76396562)(859.64959961,381.66397308)
\lineto(859.64959961,381.37897308)
\lineto(859.64959961,375.03397308)
}
}
{
\newrgbcolor{curcolor}{0 0 0}
\pscustom[linestyle=none,fillstyle=solid,fillcolor=curcolor]
{
\newpath
\moveto(864.02444336,382.21897308)
\curveto(864.74443929,382.22896515)(865.34943869,382.14396524)(865.83944336,381.96397308)
\curveto(866.32943771,381.79396559)(866.70943733,381.48896589)(866.97944336,381.04897308)
\curveto(867.04943699,380.93896644)(867.10443693,380.82396656)(867.14444336,380.70397308)
\curveto(867.18443685,380.59396679)(867.22443681,380.46896691)(867.26444336,380.32897308)
\curveto(867.28443675,380.25896712)(867.28943675,380.1839672)(867.27944336,380.10397308)
\curveto(867.26943677,380.03396735)(867.25443678,379.9789674)(867.23444336,379.93897308)
\curveto(867.21443682,379.91896746)(867.18943685,379.89896748)(867.15944336,379.87897308)
\curveto(867.12943691,379.86896751)(867.10443693,379.85396753)(867.08444336,379.83397308)
\curveto(867.034437,379.81396757)(866.98443705,379.80896757)(866.93444336,379.81897308)
\curveto(866.88443715,379.82896755)(866.8344372,379.82896755)(866.78444336,379.81897308)
\curveto(866.70443733,379.79896758)(866.59943744,379.79396759)(866.46944336,379.80397308)
\curveto(866.3394377,379.82396756)(866.24943779,379.84896753)(866.19944336,379.87897308)
\curveto(866.11943792,379.92896745)(866.06443797,379.99396739)(866.03444336,380.07397308)
\curveto(866.01443802,380.16396722)(865.97943806,380.24896713)(865.92944336,380.32897308)
\curveto(865.8394382,380.48896689)(865.71443832,380.63396675)(865.55444336,380.76397308)
\curveto(865.44443859,380.84396654)(865.32443871,380.90396648)(865.19444336,380.94397308)
\curveto(865.06443897,380.9839664)(864.92443911,381.02396636)(864.77444336,381.06397308)
\curveto(864.72443931,381.0839663)(864.67443936,381.08896629)(864.62444336,381.07897308)
\curveto(864.57443946,381.0789663)(864.52443951,381.0839663)(864.47444336,381.09397308)
\curveto(864.41443962,381.11396627)(864.3394397,381.12396626)(864.24944336,381.12397308)
\curveto(864.15943988,381.12396626)(864.08443995,381.11396627)(864.02444336,381.09397308)
\lineto(863.93444336,381.09397308)
\lineto(863.78444336,381.06397308)
\curveto(863.7344403,381.06396632)(863.68444035,381.05896632)(863.63444336,381.04897308)
\curveto(863.37444066,380.98896639)(863.15944088,380.90396648)(862.98944336,380.79397308)
\curveto(862.81944122,380.6839667)(862.70444133,380.49896688)(862.64444336,380.23897308)
\curveto(862.62444141,380.16896721)(862.61944142,380.09896728)(862.62944336,380.02897308)
\curveto(862.64944139,379.95896742)(862.66944137,379.89896748)(862.68944336,379.84897308)
\curveto(862.74944129,379.69896768)(862.81944122,379.58896779)(862.89944336,379.51897308)
\curveto(862.98944105,379.45896792)(863.09944094,379.38896799)(863.22944336,379.30897308)
\curveto(863.38944065,379.20896817)(863.56944047,379.13396825)(863.76944336,379.08397308)
\curveto(863.96944007,379.04396834)(864.16943987,378.99396839)(864.36944336,378.93397308)
\curveto(864.49943954,378.89396849)(864.62943941,378.86396852)(864.75944336,378.84397308)
\curveto(864.88943915,378.82396856)(865.01943902,378.79396859)(865.14944336,378.75397308)
\curveto(865.35943868,378.69396869)(865.56443847,378.63396875)(865.76444336,378.57397308)
\curveto(865.96443807,378.52396886)(866.16443787,378.45896892)(866.36444336,378.37897308)
\lineto(866.51444336,378.31897308)
\curveto(866.56443747,378.29896908)(866.61443742,378.27396911)(866.66444336,378.24397308)
\curveto(866.86443717,378.12396926)(867.039437,377.98896939)(867.18944336,377.83897308)
\curveto(867.3394367,377.68896969)(867.46443657,377.49896988)(867.56444336,377.26897308)
\curveto(867.58443645,377.19897018)(867.60443643,377.10397028)(867.62444336,376.98397308)
\curveto(867.64443639,376.91397047)(867.65443638,376.83897054)(867.65444336,376.75897308)
\curveto(867.66443637,376.68897069)(867.66943637,376.60897077)(867.66944336,376.51897308)
\lineto(867.66944336,376.36897308)
\curveto(867.64943639,376.29897108)(867.6394364,376.22897115)(867.63944336,376.15897308)
\curveto(867.6394364,376.08897129)(867.62943641,376.01897136)(867.60944336,375.94897308)
\curveto(867.57943646,375.83897154)(867.54443649,375.73397165)(867.50444336,375.63397308)
\curveto(867.46443657,375.53397185)(867.41943662,375.44397194)(867.36944336,375.36397308)
\curveto(867.20943683,375.10397228)(867.00443703,374.89397249)(866.75444336,374.73397308)
\curveto(866.50443753,374.5839728)(866.22443781,374.45397293)(865.91444336,374.34397308)
\curveto(865.82443821,374.31397307)(865.72943831,374.29397309)(865.62944336,374.28397308)
\curveto(865.5394385,374.26397312)(865.44943859,374.23897314)(865.35944336,374.20897308)
\curveto(865.25943878,374.18897319)(865.15943888,374.1789732)(865.05944336,374.17897308)
\curveto(864.95943908,374.1789732)(864.85943918,374.16897321)(864.75944336,374.14897308)
\lineto(864.60944336,374.14897308)
\curveto(864.55943948,374.13897324)(864.48943955,374.13397325)(864.39944336,374.13397308)
\curveto(864.30943973,374.13397325)(864.2394398,374.13897324)(864.18944336,374.14897308)
\lineto(864.02444336,374.14897308)
\curveto(863.96444007,374.16897321)(863.89944014,374.1789732)(863.82944336,374.17897308)
\curveto(863.75944028,374.16897321)(863.69944034,374.17397321)(863.64944336,374.19397308)
\curveto(863.59944044,374.20397318)(863.5344405,374.20897317)(863.45444336,374.20897308)
\lineto(863.21444336,374.26897308)
\curveto(863.14444089,374.2789731)(863.06944097,374.29897308)(862.98944336,374.32897308)
\curveto(862.67944136,374.42897295)(862.40944163,374.55397283)(862.17944336,374.70397308)
\curveto(861.94944209,374.85397253)(861.74944229,375.04897233)(861.57944336,375.28897308)
\curveto(861.48944255,375.41897196)(861.41444262,375.55397183)(861.35444336,375.69397308)
\curveto(861.29444274,375.83397155)(861.2394428,375.98897139)(861.18944336,376.15897308)
\curveto(861.16944287,376.21897116)(861.15944288,376.28897109)(861.15944336,376.36897308)
\curveto(861.16944287,376.45897092)(861.18444285,376.52897085)(861.20444336,376.57897308)
\curveto(861.2344428,376.61897076)(861.28444275,376.65897072)(861.35444336,376.69897308)
\curveto(861.40444263,376.71897066)(861.47444256,376.72897065)(861.56444336,376.72897308)
\curveto(861.65444238,376.73897064)(861.74444229,376.73897064)(861.83444336,376.72897308)
\curveto(861.92444211,376.71897066)(862.00944203,376.70397068)(862.08944336,376.68397308)
\curveto(862.17944186,376.67397071)(862.2394418,376.65897072)(862.26944336,376.63897308)
\curveto(862.3394417,376.58897079)(862.38444165,376.51397087)(862.40444336,376.41397308)
\curveto(862.4344416,376.32397106)(862.46944157,376.23897114)(862.50944336,376.15897308)
\curveto(862.60944143,375.93897144)(862.74444129,375.76897161)(862.91444336,375.64897308)
\curveto(863.034441,375.55897182)(863.16944087,375.48897189)(863.31944336,375.43897308)
\curveto(863.46944057,375.38897199)(863.62944041,375.33897204)(863.79944336,375.28897308)
\lineto(864.11444336,375.24397308)
\lineto(864.20444336,375.24397308)
\curveto(864.27443976,375.22397216)(864.36443967,375.21397217)(864.47444336,375.21397308)
\curveto(864.59443944,375.21397217)(864.69443934,375.22397216)(864.77444336,375.24397308)
\curveto(864.84443919,375.24397214)(864.89943914,375.24897213)(864.93944336,375.25897308)
\curveto(864.99943904,375.26897211)(865.05943898,375.27397211)(865.11944336,375.27397308)
\curveto(865.17943886,375.2839721)(865.2344388,375.29397209)(865.28444336,375.30397308)
\curveto(865.57443846,375.383972)(865.80443823,375.48897189)(865.97444336,375.61897308)
\curveto(866.14443789,375.74897163)(866.26443777,375.96897141)(866.33444336,376.27897308)
\curveto(866.35443768,376.32897105)(866.35943768,376.383971)(866.34944336,376.44397308)
\curveto(866.3394377,376.50397088)(866.32943771,376.54897083)(866.31944336,376.57897308)
\curveto(866.26943777,376.76897061)(866.19943784,376.90897047)(866.10944336,376.99897308)
\curveto(866.01943802,377.09897028)(865.90443813,377.18897019)(865.76444336,377.26897308)
\curveto(865.67443836,377.32897005)(865.57443846,377.37897)(865.46444336,377.41897308)
\lineto(865.13444336,377.53897308)
\curveto(865.10443893,377.54896983)(865.07443896,377.55396983)(865.04444336,377.55397308)
\curveto(865.02443901,377.55396983)(864.99943904,377.56396982)(864.96944336,377.58397308)
\curveto(864.62943941,377.69396969)(864.27443976,377.77396961)(863.90444336,377.82397308)
\curveto(863.54444049,377.8839695)(863.20444083,377.9789694)(862.88444336,378.10897308)
\curveto(862.78444125,378.14896923)(862.68944135,378.1839692)(862.59944336,378.21397308)
\curveto(862.50944153,378.24396914)(862.42444161,378.2839691)(862.34444336,378.33397308)
\curveto(862.15444188,378.44396894)(861.97944206,378.56896881)(861.81944336,378.70897308)
\curveto(861.65944238,378.84896853)(861.5344425,379.02396836)(861.44444336,379.23397308)
\curveto(861.41444262,379.30396808)(861.38944265,379.37396801)(861.36944336,379.44397308)
\curveto(861.35944268,379.51396787)(861.34444269,379.58896779)(861.32444336,379.66897308)
\curveto(861.29444274,379.78896759)(861.28444275,379.92396746)(861.29444336,380.07397308)
\curveto(861.30444273,380.23396715)(861.31944272,380.36896701)(861.33944336,380.47897308)
\curveto(861.35944268,380.52896685)(861.36944267,380.56896681)(861.36944336,380.59897308)
\curveto(861.37944266,380.63896674)(861.39444264,380.6789667)(861.41444336,380.71897308)
\curveto(861.50444253,380.94896643)(861.62444241,381.14896623)(861.77444336,381.31897308)
\curveto(861.9344421,381.48896589)(862.11444192,381.63896574)(862.31444336,381.76897308)
\curveto(862.46444157,381.85896552)(862.62944141,381.92896545)(862.80944336,381.97897308)
\curveto(862.98944105,382.03896534)(863.17944086,382.09396529)(863.37944336,382.14397308)
\curveto(863.44944059,382.15396523)(863.51444052,382.16396522)(863.57444336,382.17397308)
\curveto(863.64444039,382.1839652)(863.71944032,382.19396519)(863.79944336,382.20397308)
\curveto(863.82944021,382.21396517)(863.86944017,382.21396517)(863.91944336,382.20397308)
\curveto(863.96944007,382.19396519)(864.00444003,382.19896518)(864.02444336,382.21897308)
}
}
{
\newrgbcolor{curcolor}{0 0 0}
\pscustom[linestyle=none,fillstyle=solid,fillcolor=curcolor]
{
\newpath
\moveto(870.03944336,384.37897308)
\curveto(870.18944135,384.378963)(870.3394412,384.37396301)(870.48944336,384.36397308)
\curveto(870.6394409,384.36396302)(870.74444079,384.32396306)(870.80444336,384.24397308)
\curveto(870.85444068,384.1839632)(870.87944066,384.09896328)(870.87944336,383.98897308)
\curveto(870.88944065,383.88896349)(870.89444064,383.7839636)(870.89444336,383.67397308)
\lineto(870.89444336,382.80397308)
\curveto(870.89444064,382.72396466)(870.88944065,382.63896474)(870.87944336,382.54897308)
\curveto(870.87944066,382.46896491)(870.88944065,382.39896498)(870.90944336,382.33897308)
\curveto(870.94944059,382.19896518)(871.0394405,382.10896527)(871.17944336,382.06897308)
\curveto(871.22944031,382.05896532)(871.27444026,382.05396533)(871.31444336,382.05397308)
\lineto(871.46444336,382.05397308)
\lineto(871.86944336,382.05397308)
\curveto(872.02943951,382.06396532)(872.14443939,382.05396533)(872.21444336,382.02397308)
\curveto(872.30443923,381.96396542)(872.36443917,381.90396548)(872.39444336,381.84397308)
\curveto(872.41443912,381.80396558)(872.42443911,381.75896562)(872.42444336,381.70897308)
\lineto(872.42444336,381.55897308)
\curveto(872.42443911,381.44896593)(872.41943912,381.34396604)(872.40944336,381.24397308)
\curveto(872.39943914,381.15396623)(872.36443917,381.0839663)(872.30444336,381.03397308)
\curveto(872.24443929,380.9839664)(872.15943938,380.95396643)(872.04944336,380.94397308)
\lineto(871.71944336,380.94397308)
\curveto(871.60943993,380.95396643)(871.49944004,380.95896642)(871.38944336,380.95897308)
\curveto(871.27944026,380.95896642)(871.18444035,380.94396644)(871.10444336,380.91397308)
\curveto(871.0344405,380.8839665)(870.98444055,380.83396655)(870.95444336,380.76397308)
\curveto(870.92444061,380.69396669)(870.90444063,380.60896677)(870.89444336,380.50897308)
\curveto(870.88444065,380.41896696)(870.87944066,380.31896706)(870.87944336,380.20897308)
\curveto(870.88944065,380.10896727)(870.89444064,380.00896737)(870.89444336,379.90897308)
\lineto(870.89444336,376.93897308)
\curveto(870.89444064,376.71897066)(870.88944065,376.4839709)(870.87944336,376.23397308)
\curveto(870.87944066,375.99397139)(870.92444061,375.80897157)(871.01444336,375.67897308)
\curveto(871.06444047,375.59897178)(871.12944041,375.54397184)(871.20944336,375.51397308)
\curveto(871.28944025,375.4839719)(871.38444015,375.45897192)(871.49444336,375.43897308)
\curveto(871.52444001,375.42897195)(871.55443998,375.42397196)(871.58444336,375.42397308)
\curveto(871.62443991,375.43397195)(871.65943988,375.43397195)(871.68944336,375.42397308)
\lineto(871.88444336,375.42397308)
\curveto(871.98443955,375.42397196)(872.07443946,375.41397197)(872.15444336,375.39397308)
\curveto(872.24443929,375.383972)(872.30943923,375.34897203)(872.34944336,375.28897308)
\curveto(872.36943917,375.25897212)(872.38443915,375.20397218)(872.39444336,375.12397308)
\curveto(872.41443912,375.05397233)(872.42443911,374.9789724)(872.42444336,374.89897308)
\curveto(872.4344391,374.81897256)(872.4344391,374.73897264)(872.42444336,374.65897308)
\curveto(872.41443912,374.58897279)(872.39443914,374.53397285)(872.36444336,374.49397308)
\curveto(872.32443921,374.42397296)(872.24943929,374.37397301)(872.13944336,374.34397308)
\curveto(872.05943948,374.32397306)(871.96943957,374.31397307)(871.86944336,374.31397308)
\curveto(871.76943977,374.32397306)(871.67943986,374.32897305)(871.59944336,374.32897308)
\curveto(871.53944,374.32897305)(871.47944006,374.32397306)(871.41944336,374.31397308)
\curveto(871.35944018,374.31397307)(871.30444023,374.31897306)(871.25444336,374.32897308)
\lineto(871.07444336,374.32897308)
\curveto(871.02444051,374.33897304)(870.97444056,374.34397304)(870.92444336,374.34397308)
\curveto(870.88444065,374.35397303)(870.8394407,374.35897302)(870.78944336,374.35897308)
\curveto(870.58944095,374.40897297)(870.41444112,374.46397292)(870.26444336,374.52397308)
\curveto(870.12444141,374.5839728)(870.00444153,374.68897269)(869.90444336,374.83897308)
\curveto(869.76444177,375.03897234)(869.68444185,375.28897209)(869.66444336,375.58897308)
\curveto(869.64444189,375.89897148)(869.6344419,376.22897115)(869.63444336,376.57897308)
\lineto(869.63444336,380.50897308)
\curveto(869.60444193,380.63896674)(869.57444196,380.73396665)(869.54444336,380.79397308)
\curveto(869.52444201,380.85396653)(869.45444208,380.90396648)(869.33444336,380.94397308)
\curveto(869.29444224,380.95396643)(869.25444228,380.95396643)(869.21444336,380.94397308)
\curveto(869.17444236,380.93396645)(869.1344424,380.93896644)(869.09444336,380.95897308)
\lineto(868.85444336,380.95897308)
\curveto(868.72444281,380.95896642)(868.61444292,380.96896641)(868.52444336,380.98897308)
\curveto(868.44444309,381.01896636)(868.38944315,381.0789663)(868.35944336,381.16897308)
\curveto(868.3394432,381.20896617)(868.32444321,381.25396613)(868.31444336,381.30397308)
\lineto(868.31444336,381.45397308)
\curveto(868.31444322,381.59396579)(868.32444321,381.70896567)(868.34444336,381.79897308)
\curveto(868.36444317,381.89896548)(868.42444311,381.97396541)(868.52444336,382.02397308)
\curveto(868.6344429,382.06396532)(868.77444276,382.07396531)(868.94444336,382.05397308)
\curveto(869.12444241,382.03396535)(869.27444226,382.04396534)(869.39444336,382.08397308)
\curveto(869.48444205,382.13396525)(869.55444198,382.20396518)(869.60444336,382.29397308)
\curveto(869.62444191,382.35396503)(869.6344419,382.42896495)(869.63444336,382.51897308)
\lineto(869.63444336,382.77397308)
\lineto(869.63444336,383.70397308)
\lineto(869.63444336,383.94397308)
\curveto(869.6344419,384.03396335)(869.64444189,384.10896327)(869.66444336,384.16897308)
\curveto(869.70444183,384.24896313)(869.77944176,384.31396307)(869.88944336,384.36397308)
\curveto(869.91944162,384.36396302)(869.94444159,384.36396302)(869.96444336,384.36397308)
\curveto(869.99444154,384.37396301)(870.01944152,384.378963)(870.03944336,384.37897308)
}
}
{
\newrgbcolor{curcolor}{0 0 0}
\pscustom[linestyle=none,fillstyle=solid,fillcolor=curcolor]
{
\newpath
\moveto(877.45624023,382.21897308)
\curveto(877.68623544,382.21896516)(877.81623531,382.15896522)(877.84624023,382.03897308)
\curveto(877.87623525,381.92896545)(877.89123524,381.76396562)(877.89124023,381.54397308)
\lineto(877.89124023,381.25897308)
\curveto(877.89123524,381.16896621)(877.86623526,381.09396629)(877.81624023,381.03397308)
\curveto(877.75623537,380.95396643)(877.67123546,380.90896647)(877.56124023,380.89897308)
\curveto(877.45123568,380.89896648)(877.34123579,380.8839665)(877.23124023,380.85397308)
\curveto(877.09123604,380.82396656)(876.95623617,380.79396659)(876.82624023,380.76397308)
\curveto(876.70623642,380.73396665)(876.59123654,380.69396669)(876.48124023,380.64397308)
\curveto(876.19123694,380.51396687)(875.95623717,380.33396705)(875.77624023,380.10397308)
\curveto(875.59623753,379.8839675)(875.44123769,379.62896775)(875.31124023,379.33897308)
\curveto(875.27123786,379.22896815)(875.24123789,379.11396827)(875.22124023,378.99397308)
\curveto(875.20123793,378.8839685)(875.17623795,378.76896861)(875.14624023,378.64897308)
\curveto(875.13623799,378.59896878)(875.131238,378.54896883)(875.13124023,378.49897308)
\curveto(875.14123799,378.44896893)(875.14123799,378.39896898)(875.13124023,378.34897308)
\curveto(875.10123803,378.22896915)(875.08623804,378.08896929)(875.08624023,377.92897308)
\curveto(875.09623803,377.7789696)(875.10123803,377.63396975)(875.10124023,377.49397308)
\lineto(875.10124023,375.64897308)
\lineto(875.10124023,375.30397308)
\curveto(875.10123803,375.1839722)(875.09623803,375.06897231)(875.08624023,374.95897308)
\curveto(875.07623805,374.84897253)(875.07123806,374.75397263)(875.07124023,374.67397308)
\curveto(875.08123805,374.59397279)(875.06123807,374.52397286)(875.01124023,374.46397308)
\curveto(874.96123817,374.39397299)(874.88123825,374.35397303)(874.77124023,374.34397308)
\curveto(874.67123846,374.33397305)(874.56123857,374.32897305)(874.44124023,374.32897308)
\lineto(874.17124023,374.32897308)
\curveto(874.12123901,374.34897303)(874.07123906,374.36397302)(874.02124023,374.37397308)
\curveto(873.98123915,374.39397299)(873.95123918,374.41897296)(873.93124023,374.44897308)
\curveto(873.88123925,374.51897286)(873.85123928,374.60397278)(873.84124023,374.70397308)
\lineto(873.84124023,375.03397308)
\lineto(873.84124023,376.18897308)
\lineto(873.84124023,380.34397308)
\lineto(873.84124023,381.37897308)
\lineto(873.84124023,381.67897308)
\curveto(873.85123928,381.7789656)(873.88123925,381.86396552)(873.93124023,381.93397308)
\curveto(873.96123917,381.97396541)(874.01123912,382.00396538)(874.08124023,382.02397308)
\curveto(874.16123897,382.04396534)(874.24623888,382.05396533)(874.33624023,382.05397308)
\curveto(874.4262387,382.06396532)(874.51623861,382.06396532)(874.60624023,382.05397308)
\curveto(874.69623843,382.04396534)(874.76623836,382.02896535)(874.81624023,382.00897308)
\curveto(874.89623823,381.9789654)(874.94623818,381.91896546)(874.96624023,381.82897308)
\curveto(874.99623813,381.74896563)(875.01123812,381.65896572)(875.01124023,381.55897308)
\lineto(875.01124023,381.25897308)
\curveto(875.01123812,381.15896622)(875.0312381,381.06896631)(875.07124023,380.98897308)
\curveto(875.08123805,380.96896641)(875.09123804,380.95396643)(875.10124023,380.94397308)
\lineto(875.14624023,380.89897308)
\curveto(875.25623787,380.89896648)(875.34623778,380.94396644)(875.41624023,381.03397308)
\curveto(875.48623764,381.13396625)(875.54623758,381.21396617)(875.59624023,381.27397308)
\lineto(875.68624023,381.36397308)
\curveto(875.77623735,381.47396591)(875.90123723,381.58896579)(876.06124023,381.70897308)
\curveto(876.22123691,381.82896555)(876.37123676,381.91896546)(876.51124023,381.97897308)
\curveto(876.60123653,382.02896535)(876.69623643,382.06396532)(876.79624023,382.08397308)
\curveto(876.89623623,382.11396527)(877.00123613,382.14396524)(877.11124023,382.17397308)
\curveto(877.17123596,382.1839652)(877.2312359,382.18896519)(877.29124023,382.18897308)
\curveto(877.35123578,382.19896518)(877.40623572,382.20896517)(877.45624023,382.21897308)
}
}
{
\newrgbcolor{curcolor}{0 0 0}
\pscustom[linestyle=none,fillstyle=solid,fillcolor=curcolor]
{
\newpath
\moveto(885.70600586,374.86897308)
\curveto(885.73599803,374.70897267)(885.72099804,374.57397281)(885.66100586,374.46397308)
\curveto(885.60099816,374.36397302)(885.52099824,374.28897309)(885.42100586,374.23897308)
\curveto(885.37099839,374.21897316)(885.31599845,374.20897317)(885.25600586,374.20897308)
\curveto(885.20599856,374.20897317)(885.15099861,374.19897318)(885.09100586,374.17897308)
\curveto(884.87099889,374.12897325)(884.65099911,374.14397324)(884.43100586,374.22397308)
\curveto(884.22099954,374.29397309)(884.07599969,374.383973)(883.99600586,374.49397308)
\curveto(883.94599982,374.56397282)(883.90099986,374.64397274)(883.86100586,374.73397308)
\curveto(883.82099994,374.83397255)(883.77099999,374.91397247)(883.71100586,374.97397308)
\curveto(883.69100007,374.99397239)(883.6660001,375.01397237)(883.63600586,375.03397308)
\curveto(883.61600015,375.05397233)(883.58600018,375.05897232)(883.54600586,375.04897308)
\curveto(883.43600033,375.01897236)(883.33100043,374.96397242)(883.23100586,374.88397308)
\curveto(883.14100062,374.80397258)(883.05100071,374.73397265)(882.96100586,374.67397308)
\curveto(882.83100093,374.59397279)(882.69100107,374.51897286)(882.54100586,374.44897308)
\curveto(882.39100137,374.38897299)(882.23100153,374.33397305)(882.06100586,374.28397308)
\curveto(881.9610018,374.25397313)(881.85100191,374.23397315)(881.73100586,374.22397308)
\curveto(881.62100214,374.21397317)(881.51100225,374.19897318)(881.40100586,374.17897308)
\curveto(881.35100241,374.16897321)(881.30600246,374.16397322)(881.26600586,374.16397308)
\lineto(881.16100586,374.16397308)
\curveto(881.05100271,374.14397324)(880.94600282,374.14397324)(880.84600586,374.16397308)
\lineto(880.71100586,374.16397308)
\curveto(880.6610031,374.17397321)(880.61100315,374.1789732)(880.56100586,374.17897308)
\curveto(880.51100325,374.1789732)(880.4660033,374.18897319)(880.42600586,374.20897308)
\curveto(880.38600338,374.21897316)(880.35100341,374.22397316)(880.32100586,374.22397308)
\curveto(880.30100346,374.21397317)(880.27600349,374.21397317)(880.24600586,374.22397308)
\lineto(880.00600586,374.28397308)
\curveto(879.92600384,374.29397309)(879.85100391,374.31397307)(879.78100586,374.34397308)
\curveto(879.48100428,374.47397291)(879.23600453,374.61897276)(879.04600586,374.77897308)
\curveto(878.8660049,374.94897243)(878.71600505,375.1839722)(878.59600586,375.48397308)
\curveto(878.50600526,375.70397168)(878.4610053,375.96897141)(878.46100586,376.27897308)
\lineto(878.46100586,376.59397308)
\curveto(878.47100529,376.64397074)(878.47600529,376.69397069)(878.47600586,376.74397308)
\lineto(878.50600586,376.92397308)
\lineto(878.62600586,377.25397308)
\curveto(878.6660051,377.36397002)(878.71600505,377.46396992)(878.77600586,377.55397308)
\curveto(878.95600481,377.84396954)(879.20100456,378.05896932)(879.51100586,378.19897308)
\curveto(879.82100394,378.33896904)(880.1610036,378.46396892)(880.53100586,378.57397308)
\curveto(880.67100309,378.61396877)(880.81600295,378.64396874)(880.96600586,378.66397308)
\curveto(881.11600265,378.6839687)(881.2660025,378.70896867)(881.41600586,378.73897308)
\curveto(881.48600228,378.75896862)(881.55100221,378.76896861)(881.61100586,378.76897308)
\curveto(881.68100208,378.76896861)(881.75600201,378.7789686)(881.83600586,378.79897308)
\curveto(881.90600186,378.81896856)(881.97600179,378.82896855)(882.04600586,378.82897308)
\curveto(882.11600165,378.83896854)(882.19100157,378.85396853)(882.27100586,378.87397308)
\curveto(882.52100124,378.93396845)(882.75600101,378.9839684)(882.97600586,379.02397308)
\curveto(883.19600057,379.07396831)(883.37100039,379.18896819)(883.50100586,379.36897308)
\curveto(883.5610002,379.44896793)(883.61100015,379.54896783)(883.65100586,379.66897308)
\curveto(883.69100007,379.79896758)(883.69100007,379.93896744)(883.65100586,380.08897308)
\curveto(883.59100017,380.32896705)(883.50100026,380.51896686)(883.38100586,380.65897308)
\curveto(883.27100049,380.79896658)(883.11100065,380.90896647)(882.90100586,380.98897308)
\curveto(882.78100098,381.03896634)(882.63600113,381.07396631)(882.46600586,381.09397308)
\curveto(882.30600146,381.11396627)(882.13600163,381.12396626)(881.95600586,381.12397308)
\curveto(881.77600199,381.12396626)(881.60100216,381.11396627)(881.43100586,381.09397308)
\curveto(881.2610025,381.07396631)(881.11600265,381.04396634)(880.99600586,381.00397308)
\curveto(880.82600294,380.94396644)(880.6610031,380.85896652)(880.50100586,380.74897308)
\curveto(880.42100334,380.68896669)(880.34600342,380.60896677)(880.27600586,380.50897308)
\curveto(880.21600355,380.41896696)(880.1610036,380.31896706)(880.11100586,380.20897308)
\curveto(880.08100368,380.12896725)(880.05100371,380.04396734)(880.02100586,379.95397308)
\curveto(880.00100376,379.86396752)(879.95600381,379.79396759)(879.88600586,379.74397308)
\curveto(879.84600392,379.71396767)(879.77600399,379.68896769)(879.67600586,379.66897308)
\curveto(879.58600418,379.65896772)(879.49100427,379.65396773)(879.39100586,379.65397308)
\curveto(879.29100447,379.65396773)(879.19100457,379.65896772)(879.09100586,379.66897308)
\curveto(879.00100476,379.68896769)(878.93600483,379.71396767)(878.89600586,379.74397308)
\curveto(878.85600491,379.77396761)(878.82600494,379.82396756)(878.80600586,379.89397308)
\curveto(878.78600498,379.96396742)(878.78600498,380.03896734)(878.80600586,380.11897308)
\curveto(878.83600493,380.24896713)(878.8660049,380.36896701)(878.89600586,380.47897308)
\curveto(878.93600483,380.59896678)(878.98100478,380.71396667)(879.03100586,380.82397308)
\curveto(879.22100454,381.17396621)(879.4610043,381.44396594)(879.75100586,381.63397308)
\curveto(880.04100372,381.83396555)(880.40100336,381.99396539)(880.83100586,382.11397308)
\curveto(880.93100283,382.13396525)(881.03100273,382.14896523)(881.13100586,382.15897308)
\curveto(881.24100252,382.16896521)(881.35100241,382.1839652)(881.46100586,382.20397308)
\curveto(881.50100226,382.21396517)(881.5660022,382.21396517)(881.65600586,382.20397308)
\curveto(881.74600202,382.20396518)(881.80100196,382.21396517)(881.82100586,382.23397308)
\curveto(882.52100124,382.24396514)(883.13100063,382.16396522)(883.65100586,381.99397308)
\curveto(884.17099959,381.82396556)(884.53599923,381.49896588)(884.74600586,381.01897308)
\curveto(884.83599893,380.81896656)(884.88599888,380.5839668)(884.89600586,380.31397308)
\curveto(884.91599885,380.05396733)(884.92599884,379.7789676)(884.92600586,379.48897308)
\lineto(884.92600586,376.17397308)
\curveto(884.92599884,376.03397135)(884.93099883,375.89897148)(884.94100586,375.76897308)
\curveto(884.95099881,375.63897174)(884.98099878,375.53397185)(885.03100586,375.45397308)
\curveto(885.08099868,375.383972)(885.14599862,375.33397205)(885.22600586,375.30397308)
\curveto(885.31599845,375.26397212)(885.40099836,375.23397215)(885.48100586,375.21397308)
\curveto(885.5609982,375.20397218)(885.62099814,375.15897222)(885.66100586,375.07897308)
\curveto(885.68099808,375.04897233)(885.69099807,375.01897236)(885.69100586,374.98897308)
\curveto(885.69099807,374.95897242)(885.69599807,374.91897246)(885.70600586,374.86897308)
\moveto(883.56100586,376.53397308)
\curveto(883.62100014,376.67397071)(883.65100011,376.83397055)(883.65100586,377.01397308)
\curveto(883.6610001,377.20397018)(883.6660001,377.39896998)(883.66600586,377.59897308)
\curveto(883.6660001,377.70896967)(883.6610001,377.80896957)(883.65100586,377.89897308)
\curveto(883.64100012,377.98896939)(883.60100016,378.05896932)(883.53100586,378.10897308)
\curveto(883.50100026,378.12896925)(883.43100033,378.13896924)(883.32100586,378.13897308)
\curveto(883.30100046,378.11896926)(883.2660005,378.10896927)(883.21600586,378.10897308)
\curveto(883.1660006,378.10896927)(883.12100064,378.09896928)(883.08100586,378.07897308)
\curveto(883.00100076,378.05896932)(882.91100085,378.03896934)(882.81100586,378.01897308)
\lineto(882.51100586,377.95897308)
\curveto(882.48100128,377.95896942)(882.44600132,377.95396943)(882.40600586,377.94397308)
\lineto(882.30100586,377.94397308)
\curveto(882.15100161,377.90396948)(881.98600178,377.8789695)(881.80600586,377.86897308)
\curveto(881.63600213,377.86896951)(881.47600229,377.84896953)(881.32600586,377.80897308)
\curveto(881.24600252,377.78896959)(881.17100259,377.76896961)(881.10100586,377.74897308)
\curveto(881.04100272,377.73896964)(880.97100279,377.72396966)(880.89100586,377.70397308)
\curveto(880.73100303,377.65396973)(880.58100318,377.58896979)(880.44100586,377.50897308)
\curveto(880.30100346,377.43896994)(880.18100358,377.34897003)(880.08100586,377.23897308)
\curveto(879.98100378,377.12897025)(879.90600386,376.99397039)(879.85600586,376.83397308)
\curveto(879.80600396,376.6839707)(879.78600398,376.49897088)(879.79600586,376.27897308)
\curveto(879.79600397,376.1789712)(879.81100395,376.0839713)(879.84100586,375.99397308)
\curveto(879.88100388,375.91397147)(879.92600384,375.83897154)(879.97600586,375.76897308)
\curveto(880.05600371,375.65897172)(880.1610036,375.56397182)(880.29100586,375.48397308)
\curveto(880.42100334,375.41397197)(880.5610032,375.35397203)(880.71100586,375.30397308)
\curveto(880.761003,375.29397209)(880.81100295,375.28897209)(880.86100586,375.28897308)
\curveto(880.91100285,375.28897209)(880.9610028,375.2839721)(881.01100586,375.27397308)
\curveto(881.08100268,375.25397213)(881.1660026,375.23897214)(881.26600586,375.22897308)
\curveto(881.37600239,375.22897215)(881.4660023,375.23897214)(881.53600586,375.25897308)
\curveto(881.59600217,375.2789721)(881.65600211,375.2839721)(881.71600586,375.27397308)
\curveto(881.77600199,375.27397211)(881.83600193,375.2839721)(881.89600586,375.30397308)
\curveto(881.97600179,375.32397206)(882.05100171,375.33897204)(882.12100586,375.34897308)
\curveto(882.20100156,375.35897202)(882.27600149,375.378972)(882.34600586,375.40897308)
\curveto(882.63600113,375.52897185)(882.88100088,375.67397171)(883.08100586,375.84397308)
\curveto(883.29100047,376.01397137)(883.45100031,376.24397114)(883.56100586,376.53397308)
}
}
{
\newrgbcolor{curcolor}{0 0 0}
\pscustom[linestyle=none,fillstyle=solid,fillcolor=curcolor]
{
\newpath
\moveto(893.83764648,375.12397308)
\lineto(893.83764648,374.73397308)
\curveto(893.83763861,374.61397277)(893.81263863,374.51397287)(893.76264648,374.43397308)
\curveto(893.71263873,374.36397302)(893.62763882,374.32397306)(893.50764648,374.31397308)
\lineto(893.16264648,374.31397308)
\curveto(893.10263934,374.31397307)(893.0426394,374.30897307)(892.98264648,374.29897308)
\curveto(892.93263951,374.29897308)(892.88763956,374.30897307)(892.84764648,374.32897308)
\curveto(892.75763969,374.34897303)(892.69763975,374.38897299)(892.66764648,374.44897308)
\curveto(892.62763982,374.49897288)(892.60263984,374.55897282)(892.59264648,374.62897308)
\curveto(892.59263985,374.69897268)(892.57763987,374.76897261)(892.54764648,374.83897308)
\curveto(892.53763991,374.85897252)(892.52263992,374.87397251)(892.50264648,374.88397308)
\curveto(892.49263995,374.90397248)(892.47763997,374.92397246)(892.45764648,374.94397308)
\curveto(892.35764009,374.95397243)(892.27764017,374.93397245)(892.21764648,374.88397308)
\curveto(892.16764028,374.83397255)(892.11264033,374.7839726)(892.05264648,374.73397308)
\curveto(891.85264059,374.5839728)(891.65264079,374.46897291)(891.45264648,374.38897308)
\curveto(891.27264117,374.30897307)(891.06264138,374.24897313)(890.82264648,374.20897308)
\curveto(890.59264185,374.16897321)(890.35264209,374.14897323)(890.10264648,374.14897308)
\curveto(889.86264258,374.13897324)(889.62264282,374.15397323)(889.38264648,374.19397308)
\curveto(889.1426433,374.22397316)(888.93264351,374.2789731)(888.75264648,374.35897308)
\curveto(888.23264421,374.5789728)(887.81264463,374.87397251)(887.49264648,375.24397308)
\curveto(887.17264527,375.62397176)(886.92264552,376.09397129)(886.74264648,376.65397308)
\curveto(886.70264574,376.74397064)(886.67264577,376.83397055)(886.65264648,376.92397308)
\curveto(886.6426458,377.02397036)(886.62264582,377.12397026)(886.59264648,377.22397308)
\curveto(886.58264586,377.27397011)(886.57764587,377.32397006)(886.57764648,377.37397308)
\curveto(886.57764587,377.42396996)(886.57264587,377.47396991)(886.56264648,377.52397308)
\curveto(886.5426459,377.57396981)(886.53264591,377.62396976)(886.53264648,377.67397308)
\curveto(886.5426459,377.73396965)(886.5426459,377.78896959)(886.53264648,377.83897308)
\lineto(886.53264648,377.98897308)
\curveto(886.51264593,378.03896934)(886.50264594,378.10396928)(886.50264648,378.18397308)
\curveto(886.50264594,378.26396912)(886.51264593,378.32896905)(886.53264648,378.37897308)
\lineto(886.53264648,378.54397308)
\curveto(886.55264589,378.61396877)(886.55764589,378.6839687)(886.54764648,378.75397308)
\curveto(886.5476459,378.83396855)(886.55764589,378.90896847)(886.57764648,378.97897308)
\curveto(886.58764586,379.02896835)(886.59264585,379.07396831)(886.59264648,379.11397308)
\curveto(886.59264585,379.15396823)(886.59764585,379.19896818)(886.60764648,379.24897308)
\curveto(886.63764581,379.34896803)(886.66264578,379.44396794)(886.68264648,379.53397308)
\curveto(886.70264574,379.63396775)(886.72764572,379.72896765)(886.75764648,379.81897308)
\curveto(886.88764556,380.19896718)(887.05264539,380.53896684)(887.25264648,380.83897308)
\curveto(887.46264498,381.14896623)(887.71264473,381.40396598)(888.00264648,381.60397308)
\curveto(888.17264427,381.72396566)(888.3476441,381.82396556)(888.52764648,381.90397308)
\curveto(888.71764373,381.9839654)(888.92264352,382.05396533)(889.14264648,382.11397308)
\curveto(889.21264323,382.12396526)(889.27764317,382.13396525)(889.33764648,382.14397308)
\curveto(889.40764304,382.15396523)(889.47764297,382.16896521)(889.54764648,382.18897308)
\lineto(889.69764648,382.18897308)
\curveto(889.77764267,382.20896517)(889.89264255,382.21896516)(890.04264648,382.21897308)
\curveto(890.20264224,382.21896516)(890.32264212,382.20896517)(890.40264648,382.18897308)
\curveto(890.442642,382.1789652)(890.49764195,382.17396521)(890.56764648,382.17397308)
\curveto(890.67764177,382.14396524)(890.78764166,382.11896526)(890.89764648,382.09897308)
\curveto(891.00764144,382.08896529)(891.11264133,382.05896532)(891.21264648,382.00897308)
\curveto(891.36264108,381.94896543)(891.50264094,381.8839655)(891.63264648,381.81397308)
\curveto(891.77264067,381.74396564)(891.90264054,381.66396572)(892.02264648,381.57397308)
\curveto(892.08264036,381.52396586)(892.1426403,381.46896591)(892.20264648,381.40897308)
\curveto(892.27264017,381.35896602)(892.36264008,381.34396604)(892.47264648,381.36397308)
\curveto(892.49263995,381.39396599)(892.50763994,381.41896596)(892.51764648,381.43897308)
\curveto(892.53763991,381.45896592)(892.55263989,381.48896589)(892.56264648,381.52897308)
\curveto(892.59263985,381.61896576)(892.60263984,381.73396565)(892.59264648,381.87397308)
\lineto(892.59264648,382.24897308)
\lineto(892.59264648,383.97397308)
\lineto(892.59264648,384.43897308)
\curveto(892.59263985,384.61896276)(892.61763983,384.74896263)(892.66764648,384.82897308)
\curveto(892.70763974,384.89896248)(892.76763968,384.94396244)(892.84764648,384.96397308)
\curveto(892.86763958,384.96396242)(892.89263955,384.96396242)(892.92264648,384.96397308)
\curveto(892.95263949,384.97396241)(892.97763947,384.9789624)(892.99764648,384.97897308)
\curveto(893.13763931,384.98896239)(893.28263916,384.98896239)(893.43264648,384.97897308)
\curveto(893.59263885,384.9789624)(893.70263874,384.93896244)(893.76264648,384.85897308)
\curveto(893.81263863,384.7789626)(893.83763861,384.6789627)(893.83764648,384.55897308)
\lineto(893.83764648,384.18397308)
\lineto(893.83764648,375.12397308)
\moveto(892.62264648,377.95897308)
\curveto(892.6426398,378.00896937)(892.65263979,378.07396931)(892.65264648,378.15397308)
\curveto(892.65263979,378.24396914)(892.6426398,378.31396907)(892.62264648,378.36397308)
\lineto(892.62264648,378.58897308)
\curveto(892.60263984,378.6789687)(892.58763986,378.76896861)(892.57764648,378.85897308)
\curveto(892.56763988,378.95896842)(892.5476399,379.04896833)(892.51764648,379.12897308)
\curveto(892.49763995,379.20896817)(892.47763997,379.2839681)(892.45764648,379.35397308)
\curveto(892.44764,379.42396796)(892.42764002,379.49396789)(892.39764648,379.56397308)
\curveto(892.27764017,379.86396752)(892.12264032,380.12896725)(891.93264648,380.35897308)
\curveto(891.7426407,380.58896679)(891.50264094,380.76896661)(891.21264648,380.89897308)
\curveto(891.11264133,380.94896643)(891.00764144,380.9839664)(890.89764648,381.00397308)
\curveto(890.79764165,381.03396635)(890.68764176,381.05896632)(890.56764648,381.07897308)
\curveto(890.48764196,381.09896628)(890.39764205,381.10896627)(890.29764648,381.10897308)
\lineto(890.02764648,381.10897308)
\curveto(889.97764247,381.09896628)(889.93264251,381.08896629)(889.89264648,381.07897308)
\lineto(889.75764648,381.07897308)
\curveto(889.67764277,381.05896632)(889.59264285,381.03896634)(889.50264648,381.01897308)
\curveto(889.42264302,380.99896638)(889.3426431,380.97396641)(889.26264648,380.94397308)
\curveto(888.9426435,380.80396658)(888.68264376,380.59896678)(888.48264648,380.32897308)
\curveto(888.29264415,380.06896731)(888.13764431,379.76396762)(888.01764648,379.41397308)
\curveto(887.97764447,379.30396808)(887.9476445,379.18896819)(887.92764648,379.06897308)
\curveto(887.91764453,378.95896842)(887.90264454,378.84896853)(887.88264648,378.73897308)
\curveto(887.88264456,378.69896868)(887.87764457,378.65896872)(887.86764648,378.61897308)
\lineto(887.86764648,378.51397308)
\curveto(887.8476446,378.46396892)(887.83764461,378.40896897)(887.83764648,378.34897308)
\curveto(887.8476446,378.28896909)(887.85264459,378.23396915)(887.85264648,378.18397308)
\lineto(887.85264648,377.85397308)
\curveto(887.85264459,377.75396963)(887.86264458,377.65896972)(887.88264648,377.56897308)
\curveto(887.89264455,377.53896984)(887.89764455,377.48896989)(887.89764648,377.41897308)
\curveto(887.91764453,377.34897003)(887.93264451,377.2789701)(887.94264648,377.20897308)
\lineto(888.00264648,376.99897308)
\curveto(888.11264433,376.64897073)(888.26264418,376.34897103)(888.45264648,376.09897308)
\curveto(888.6426438,375.84897153)(888.88264356,375.64397174)(889.17264648,375.48397308)
\curveto(889.26264318,375.43397195)(889.35264309,375.39397199)(889.44264648,375.36397308)
\curveto(889.53264291,375.33397205)(889.63264281,375.30397208)(889.74264648,375.27397308)
\curveto(889.79264265,375.25397213)(889.8426426,375.24897213)(889.89264648,375.25897308)
\curveto(889.95264249,375.26897211)(890.00764244,375.26397212)(890.05764648,375.24397308)
\curveto(890.09764235,375.23397215)(890.13764231,375.22897215)(890.17764648,375.22897308)
\lineto(890.31264648,375.22897308)
\lineto(890.44764648,375.22897308)
\curveto(890.47764197,375.23897214)(890.52764192,375.24397214)(890.59764648,375.24397308)
\curveto(890.67764177,375.26397212)(890.75764169,375.2789721)(890.83764648,375.28897308)
\curveto(890.91764153,375.30897207)(890.99264145,375.33397205)(891.06264648,375.36397308)
\curveto(891.39264105,375.50397188)(891.65764079,375.6789717)(891.85764648,375.88897308)
\curveto(892.06764038,376.10897127)(892.2426402,376.383971)(892.38264648,376.71397308)
\curveto(892.43264001,376.82397056)(892.46763998,376.93397045)(892.48764648,377.04397308)
\curveto(892.50763994,377.15397023)(892.53263991,377.26397012)(892.56264648,377.37397308)
\curveto(892.58263986,377.41396997)(892.59263985,377.44896993)(892.59264648,377.47897308)
\curveto(892.59263985,377.51896986)(892.59763985,377.55896982)(892.60764648,377.59897308)
\curveto(892.61763983,377.65896972)(892.61763983,377.71896966)(892.60764648,377.77897308)
\curveto(892.60763984,377.83896954)(892.61263983,377.89896948)(892.62264648,377.95897308)
}
}
{
\newrgbcolor{curcolor}{0 0 0}
\pscustom[linestyle=none,fillstyle=solid,fillcolor=curcolor]
{
\newpath
\moveto(902.90889648,378.51397308)
\curveto(902.92888842,378.45396893)(902.93888841,378.35896902)(902.93889648,378.22897308)
\curveto(902.93888841,378.10896927)(902.93388842,378.02396936)(902.92389648,377.97397308)
\lineto(902.92389648,377.82397308)
\curveto(902.91388844,377.74396964)(902.90388845,377.66896971)(902.89389648,377.59897308)
\curveto(902.89388846,377.53896984)(902.88888846,377.46896991)(902.87889648,377.38897308)
\curveto(902.85888849,377.32897005)(902.84388851,377.26897011)(902.83389648,377.20897308)
\curveto(902.83388852,377.14897023)(902.82388853,377.08897029)(902.80389648,377.02897308)
\curveto(902.76388859,376.89897048)(902.72888862,376.76897061)(902.69889648,376.63897308)
\curveto(902.66888868,376.50897087)(902.62888872,376.38897099)(902.57889648,376.27897308)
\curveto(902.36888898,375.79897158)(902.08888926,375.39397199)(901.73889648,375.06397308)
\curveto(901.38888996,374.74397264)(900.95889039,374.49897288)(900.44889648,374.32897308)
\curveto(900.33889101,374.28897309)(900.21889113,374.25897312)(900.08889648,374.23897308)
\curveto(899.96889138,374.21897316)(899.84389151,374.19897318)(899.71389648,374.17897308)
\curveto(899.6538917,374.16897321)(899.58889176,374.16397322)(899.51889648,374.16397308)
\curveto(899.45889189,374.15397323)(899.39889195,374.14897323)(899.33889648,374.14897308)
\curveto(899.29889205,374.13897324)(899.23889211,374.13397325)(899.15889648,374.13397308)
\curveto(899.08889226,374.13397325)(899.03889231,374.13897324)(899.00889648,374.14897308)
\curveto(898.96889238,374.15897322)(898.92889242,374.16397322)(898.88889648,374.16397308)
\curveto(898.8488925,374.15397323)(898.81389254,374.15397323)(898.78389648,374.16397308)
\lineto(898.69389648,374.16397308)
\lineto(898.33389648,374.20897308)
\curveto(898.19389316,374.24897313)(898.05889329,374.28897309)(897.92889648,374.32897308)
\curveto(897.79889355,374.36897301)(897.67389368,374.41397297)(897.55389648,374.46397308)
\curveto(897.10389425,374.66397272)(896.73389462,374.92397246)(896.44389648,375.24397308)
\curveto(896.1538952,375.56397182)(895.91389544,375.95397143)(895.72389648,376.41397308)
\curveto(895.67389568,376.51397087)(895.63389572,376.61397077)(895.60389648,376.71397308)
\curveto(895.58389577,376.81397057)(895.56389579,376.91897046)(895.54389648,377.02897308)
\curveto(895.52389583,377.06897031)(895.51389584,377.09897028)(895.51389648,377.11897308)
\curveto(895.52389583,377.14897023)(895.52389583,377.1839702)(895.51389648,377.22397308)
\curveto(895.49389586,377.30397008)(895.47889587,377.38397)(895.46889648,377.46397308)
\curveto(895.46889588,377.55396983)(895.45889589,377.63896974)(895.43889648,377.71897308)
\lineto(895.43889648,377.83897308)
\curveto(895.43889591,377.8789695)(895.43389592,377.92396946)(895.42389648,377.97397308)
\curveto(895.41389594,378.02396936)(895.40889594,378.10896927)(895.40889648,378.22897308)
\curveto(895.40889594,378.35896902)(895.41889593,378.45396893)(895.43889648,378.51397308)
\curveto(895.45889589,378.5839688)(895.46389589,378.65396873)(895.45389648,378.72397308)
\curveto(895.44389591,378.79396859)(895.4488959,378.86396852)(895.46889648,378.93397308)
\curveto(895.47889587,378.9839684)(895.48389587,379.02396836)(895.48389648,379.05397308)
\curveto(895.49389586,379.09396829)(895.50389585,379.13896824)(895.51389648,379.18897308)
\curveto(895.54389581,379.30896807)(895.56889578,379.42896795)(895.58889648,379.54897308)
\curveto(895.61889573,379.66896771)(895.65889569,379.7839676)(895.70889648,379.89397308)
\curveto(895.85889549,380.26396712)(896.03889531,380.59396679)(896.24889648,380.88397308)
\curveto(896.46889488,381.1839662)(896.73389462,381.43396595)(897.04389648,381.63397308)
\curveto(897.16389419,381.71396567)(897.28889406,381.7789656)(897.41889648,381.82897308)
\curveto(897.5488938,381.88896549)(897.68389367,381.94896543)(897.82389648,382.00897308)
\curveto(897.94389341,382.05896532)(898.07389328,382.08896529)(898.21389648,382.09897308)
\curveto(898.353893,382.11896526)(898.49389286,382.14896523)(898.63389648,382.18897308)
\lineto(898.82889648,382.18897308)
\curveto(898.89889245,382.19896518)(898.96389239,382.20896517)(899.02389648,382.21897308)
\curveto(899.91389144,382.22896515)(900.6538907,382.04396534)(901.24389648,381.66397308)
\curveto(901.83388952,381.2839661)(902.25888909,380.78896659)(902.51889648,380.17897308)
\curveto(902.56888878,380.0789673)(902.60888874,379.9789674)(902.63889648,379.87897308)
\curveto(902.66888868,379.7789676)(902.70388865,379.67396771)(902.74389648,379.56397308)
\curveto(902.77388858,379.45396793)(902.79888855,379.33396805)(902.81889648,379.20397308)
\curveto(902.83888851,379.0839683)(902.86388849,378.95896842)(902.89389648,378.82897308)
\curveto(902.90388845,378.7789686)(902.90388845,378.72396866)(902.89389648,378.66397308)
\curveto(902.89388846,378.61396877)(902.89888845,378.56396882)(902.90889648,378.51397308)
\moveto(901.57389648,377.65897308)
\curveto(901.59388976,377.72896965)(901.59888975,377.80896957)(901.58889648,377.89897308)
\lineto(901.58889648,378.15397308)
\curveto(901.58888976,378.54396884)(901.5538898,378.87396851)(901.48389648,379.14397308)
\curveto(901.4538899,379.22396816)(901.42888992,379.30396808)(901.40889648,379.38397308)
\curveto(901.38888996,379.46396792)(901.36388999,379.53896784)(901.33389648,379.60897308)
\curveto(901.0538903,380.25896712)(900.60889074,380.70896667)(899.99889648,380.95897308)
\curveto(899.92889142,380.98896639)(899.8538915,381.00896637)(899.77389648,381.01897308)
\lineto(899.53389648,381.07897308)
\curveto(899.4538919,381.09896628)(899.36889198,381.10896627)(899.27889648,381.10897308)
\lineto(899.00889648,381.10897308)
\lineto(898.73889648,381.06397308)
\curveto(898.63889271,381.04396634)(898.54389281,381.01896636)(898.45389648,380.98897308)
\curveto(898.37389298,380.96896641)(898.29389306,380.93896644)(898.21389648,380.89897308)
\curveto(898.14389321,380.8789665)(898.07889327,380.84896653)(898.01889648,380.80897308)
\curveto(897.95889339,380.76896661)(897.90389345,380.72896665)(897.85389648,380.68897308)
\curveto(897.61389374,380.51896686)(897.41889393,380.31396707)(897.26889648,380.07397308)
\curveto(897.11889423,379.83396755)(896.98889436,379.55396783)(896.87889648,379.23397308)
\curveto(896.8488945,379.13396825)(896.82889452,379.02896835)(896.81889648,378.91897308)
\curveto(896.80889454,378.81896856)(896.79389456,378.71396867)(896.77389648,378.60397308)
\curveto(896.76389459,378.56396882)(896.75889459,378.49896888)(896.75889648,378.40897308)
\curveto(896.7488946,378.378969)(896.74389461,378.34396904)(896.74389648,378.30397308)
\curveto(896.7538946,378.26396912)(896.75889459,378.21896916)(896.75889648,378.16897308)
\lineto(896.75889648,377.86897308)
\curveto(896.75889459,377.76896961)(896.76889458,377.6789697)(896.78889648,377.59897308)
\lineto(896.81889648,377.41897308)
\curveto(896.83889451,377.31897006)(896.8538945,377.21897016)(896.86389648,377.11897308)
\curveto(896.88389447,377.02897035)(896.91389444,376.94397044)(896.95389648,376.86397308)
\curveto(897.0538943,376.62397076)(897.16889418,376.39897098)(897.29889648,376.18897308)
\curveto(897.43889391,375.9789714)(897.60889374,375.80397158)(897.80889648,375.66397308)
\curveto(897.85889349,375.63397175)(897.90389345,375.60897177)(897.94389648,375.58897308)
\curveto(897.98389337,375.56897181)(898.02889332,375.54397184)(898.07889648,375.51397308)
\curveto(898.15889319,375.46397192)(898.24389311,375.41897196)(898.33389648,375.37897308)
\curveto(898.43389292,375.34897203)(898.53889281,375.31897206)(898.64889648,375.28897308)
\curveto(898.69889265,375.26897211)(898.74389261,375.25897212)(898.78389648,375.25897308)
\curveto(898.83389252,375.26897211)(898.88389247,375.26897211)(898.93389648,375.25897308)
\curveto(898.96389239,375.24897213)(899.02389233,375.23897214)(899.11389648,375.22897308)
\curveto(899.21389214,375.21897216)(899.28889206,375.22397216)(899.33889648,375.24397308)
\curveto(899.37889197,375.25397213)(899.41889193,375.25397213)(899.45889648,375.24397308)
\curveto(899.49889185,375.24397214)(899.53889181,375.25397213)(899.57889648,375.27397308)
\curveto(899.65889169,375.29397209)(899.73889161,375.30897207)(899.81889648,375.31897308)
\curveto(899.89889145,375.33897204)(899.97389138,375.36397202)(900.04389648,375.39397308)
\curveto(900.38389097,375.53397185)(900.65889069,375.72897165)(900.86889648,375.97897308)
\curveto(901.07889027,376.22897115)(901.2538901,376.52397086)(901.39389648,376.86397308)
\curveto(901.44388991,376.9839704)(901.47388988,377.10897027)(901.48389648,377.23897308)
\curveto(901.50388985,377.37897)(901.53388982,377.51896986)(901.57389648,377.65897308)
}
}
{
\newrgbcolor{curcolor}{0 0 0}
\pscustom[linestyle=none,fillstyle=solid,fillcolor=curcolor]
{
\newpath
\moveto(906.82717773,382.21897308)
\curveto(907.54717367,382.22896515)(908.15217306,382.14396524)(908.64217773,381.96397308)
\curveto(909.13217208,381.79396559)(909.5121717,381.48896589)(909.78217773,381.04897308)
\curveto(909.85217136,380.93896644)(909.90717131,380.82396656)(909.94717773,380.70397308)
\curveto(909.98717123,380.59396679)(910.02717119,380.46896691)(910.06717773,380.32897308)
\curveto(910.08717113,380.25896712)(910.09217112,380.1839672)(910.08217773,380.10397308)
\curveto(910.07217114,380.03396735)(910.05717116,379.9789674)(910.03717773,379.93897308)
\curveto(910.0171712,379.91896746)(909.99217122,379.89896748)(909.96217773,379.87897308)
\curveto(909.93217128,379.86896751)(909.90717131,379.85396753)(909.88717773,379.83397308)
\curveto(909.83717138,379.81396757)(909.78717143,379.80896757)(909.73717773,379.81897308)
\curveto(909.68717153,379.82896755)(909.63717158,379.82896755)(909.58717773,379.81897308)
\curveto(909.50717171,379.79896758)(909.40217181,379.79396759)(909.27217773,379.80397308)
\curveto(909.14217207,379.82396756)(909.05217216,379.84896753)(909.00217773,379.87897308)
\curveto(908.92217229,379.92896745)(908.86717235,379.99396739)(908.83717773,380.07397308)
\curveto(908.8171724,380.16396722)(908.78217243,380.24896713)(908.73217773,380.32897308)
\curveto(908.64217257,380.48896689)(908.5171727,380.63396675)(908.35717773,380.76397308)
\curveto(908.24717297,380.84396654)(908.12717309,380.90396648)(907.99717773,380.94397308)
\curveto(907.86717335,380.9839664)(907.72717349,381.02396636)(907.57717773,381.06397308)
\curveto(907.52717369,381.0839663)(907.47717374,381.08896629)(907.42717773,381.07897308)
\curveto(907.37717384,381.0789663)(907.32717389,381.0839663)(907.27717773,381.09397308)
\curveto(907.217174,381.11396627)(907.14217407,381.12396626)(907.05217773,381.12397308)
\curveto(906.96217425,381.12396626)(906.88717433,381.11396627)(906.82717773,381.09397308)
\lineto(906.73717773,381.09397308)
\lineto(906.58717773,381.06397308)
\curveto(906.53717468,381.06396632)(906.48717473,381.05896632)(906.43717773,381.04897308)
\curveto(906.17717504,380.98896639)(905.96217525,380.90396648)(905.79217773,380.79397308)
\curveto(905.62217559,380.6839667)(905.50717571,380.49896688)(905.44717773,380.23897308)
\curveto(905.42717579,380.16896721)(905.42217579,380.09896728)(905.43217773,380.02897308)
\curveto(905.45217576,379.95896742)(905.47217574,379.89896748)(905.49217773,379.84897308)
\curveto(905.55217566,379.69896768)(905.62217559,379.58896779)(905.70217773,379.51897308)
\curveto(905.79217542,379.45896792)(905.90217531,379.38896799)(906.03217773,379.30897308)
\curveto(906.19217502,379.20896817)(906.37217484,379.13396825)(906.57217773,379.08397308)
\curveto(906.77217444,379.04396834)(906.97217424,378.99396839)(907.17217773,378.93397308)
\curveto(907.30217391,378.89396849)(907.43217378,378.86396852)(907.56217773,378.84397308)
\curveto(907.69217352,378.82396856)(907.82217339,378.79396859)(907.95217773,378.75397308)
\curveto(908.16217305,378.69396869)(908.36717285,378.63396875)(908.56717773,378.57397308)
\curveto(908.76717245,378.52396886)(908.96717225,378.45896892)(909.16717773,378.37897308)
\lineto(909.31717773,378.31897308)
\curveto(909.36717185,378.29896908)(909.4171718,378.27396911)(909.46717773,378.24397308)
\curveto(909.66717155,378.12396926)(909.84217137,377.98896939)(909.99217773,377.83897308)
\curveto(910.14217107,377.68896969)(910.26717095,377.49896988)(910.36717773,377.26897308)
\curveto(910.38717083,377.19897018)(910.40717081,377.10397028)(910.42717773,376.98397308)
\curveto(910.44717077,376.91397047)(910.45717076,376.83897054)(910.45717773,376.75897308)
\curveto(910.46717075,376.68897069)(910.47217074,376.60897077)(910.47217773,376.51897308)
\lineto(910.47217773,376.36897308)
\curveto(910.45217076,376.29897108)(910.44217077,376.22897115)(910.44217773,376.15897308)
\curveto(910.44217077,376.08897129)(910.43217078,376.01897136)(910.41217773,375.94897308)
\curveto(910.38217083,375.83897154)(910.34717087,375.73397165)(910.30717773,375.63397308)
\curveto(910.26717095,375.53397185)(910.22217099,375.44397194)(910.17217773,375.36397308)
\curveto(910.0121712,375.10397228)(909.80717141,374.89397249)(909.55717773,374.73397308)
\curveto(909.30717191,374.5839728)(909.02717219,374.45397293)(908.71717773,374.34397308)
\curveto(908.62717259,374.31397307)(908.53217268,374.29397309)(908.43217773,374.28397308)
\curveto(908.34217287,374.26397312)(908.25217296,374.23897314)(908.16217773,374.20897308)
\curveto(908.06217315,374.18897319)(907.96217325,374.1789732)(907.86217773,374.17897308)
\curveto(907.76217345,374.1789732)(907.66217355,374.16897321)(907.56217773,374.14897308)
\lineto(907.41217773,374.14897308)
\curveto(907.36217385,374.13897324)(907.29217392,374.13397325)(907.20217773,374.13397308)
\curveto(907.1121741,374.13397325)(907.04217417,374.13897324)(906.99217773,374.14897308)
\lineto(906.82717773,374.14897308)
\curveto(906.76717445,374.16897321)(906.70217451,374.1789732)(906.63217773,374.17897308)
\curveto(906.56217465,374.16897321)(906.50217471,374.17397321)(906.45217773,374.19397308)
\curveto(906.40217481,374.20397318)(906.33717488,374.20897317)(906.25717773,374.20897308)
\lineto(906.01717773,374.26897308)
\curveto(905.94717527,374.2789731)(905.87217534,374.29897308)(905.79217773,374.32897308)
\curveto(905.48217573,374.42897295)(905.212176,374.55397283)(904.98217773,374.70397308)
\curveto(904.75217646,374.85397253)(904.55217666,375.04897233)(904.38217773,375.28897308)
\curveto(904.29217692,375.41897196)(904.217177,375.55397183)(904.15717773,375.69397308)
\curveto(904.09717712,375.83397155)(904.04217717,375.98897139)(903.99217773,376.15897308)
\curveto(903.97217724,376.21897116)(903.96217725,376.28897109)(903.96217773,376.36897308)
\curveto(903.97217724,376.45897092)(903.98717723,376.52897085)(904.00717773,376.57897308)
\curveto(904.03717718,376.61897076)(904.08717713,376.65897072)(904.15717773,376.69897308)
\curveto(904.20717701,376.71897066)(904.27717694,376.72897065)(904.36717773,376.72897308)
\curveto(904.45717676,376.73897064)(904.54717667,376.73897064)(904.63717773,376.72897308)
\curveto(904.72717649,376.71897066)(904.8121764,376.70397068)(904.89217773,376.68397308)
\curveto(904.98217623,376.67397071)(905.04217617,376.65897072)(905.07217773,376.63897308)
\curveto(905.14217607,376.58897079)(905.18717603,376.51397087)(905.20717773,376.41397308)
\curveto(905.23717598,376.32397106)(905.27217594,376.23897114)(905.31217773,376.15897308)
\curveto(905.4121758,375.93897144)(905.54717567,375.76897161)(905.71717773,375.64897308)
\curveto(905.83717538,375.55897182)(905.97217524,375.48897189)(906.12217773,375.43897308)
\curveto(906.27217494,375.38897199)(906.43217478,375.33897204)(906.60217773,375.28897308)
\lineto(906.91717773,375.24397308)
\lineto(907.00717773,375.24397308)
\curveto(907.07717414,375.22397216)(907.16717405,375.21397217)(907.27717773,375.21397308)
\curveto(907.39717382,375.21397217)(907.49717372,375.22397216)(907.57717773,375.24397308)
\curveto(907.64717357,375.24397214)(907.70217351,375.24897213)(907.74217773,375.25897308)
\curveto(907.80217341,375.26897211)(907.86217335,375.27397211)(907.92217773,375.27397308)
\curveto(907.98217323,375.2839721)(908.03717318,375.29397209)(908.08717773,375.30397308)
\curveto(908.37717284,375.383972)(908.60717261,375.48897189)(908.77717773,375.61897308)
\curveto(908.94717227,375.74897163)(909.06717215,375.96897141)(909.13717773,376.27897308)
\curveto(909.15717206,376.32897105)(909.16217205,376.383971)(909.15217773,376.44397308)
\curveto(909.14217207,376.50397088)(909.13217208,376.54897083)(909.12217773,376.57897308)
\curveto(909.07217214,376.76897061)(909.00217221,376.90897047)(908.91217773,376.99897308)
\curveto(908.82217239,377.09897028)(908.70717251,377.18897019)(908.56717773,377.26897308)
\curveto(908.47717274,377.32897005)(908.37717284,377.37897)(908.26717773,377.41897308)
\lineto(907.93717773,377.53897308)
\curveto(907.90717331,377.54896983)(907.87717334,377.55396983)(907.84717773,377.55397308)
\curveto(907.82717339,377.55396983)(907.80217341,377.56396982)(907.77217773,377.58397308)
\curveto(907.43217378,377.69396969)(907.07717414,377.77396961)(906.70717773,377.82397308)
\curveto(906.34717487,377.8839695)(906.00717521,377.9789694)(905.68717773,378.10897308)
\curveto(905.58717563,378.14896923)(905.49217572,378.1839692)(905.40217773,378.21397308)
\curveto(905.3121759,378.24396914)(905.22717599,378.2839691)(905.14717773,378.33397308)
\curveto(904.95717626,378.44396894)(904.78217643,378.56896881)(904.62217773,378.70897308)
\curveto(904.46217675,378.84896853)(904.33717688,379.02396836)(904.24717773,379.23397308)
\curveto(904.217177,379.30396808)(904.19217702,379.37396801)(904.17217773,379.44397308)
\curveto(904.16217705,379.51396787)(904.14717707,379.58896779)(904.12717773,379.66897308)
\curveto(904.09717712,379.78896759)(904.08717713,379.92396746)(904.09717773,380.07397308)
\curveto(904.10717711,380.23396715)(904.12217709,380.36896701)(904.14217773,380.47897308)
\curveto(904.16217705,380.52896685)(904.17217704,380.56896681)(904.17217773,380.59897308)
\curveto(904.18217703,380.63896674)(904.19717702,380.6789667)(904.21717773,380.71897308)
\curveto(904.30717691,380.94896643)(904.42717679,381.14896623)(904.57717773,381.31897308)
\curveto(904.73717648,381.48896589)(904.9171763,381.63896574)(905.11717773,381.76897308)
\curveto(905.26717595,381.85896552)(905.43217578,381.92896545)(905.61217773,381.97897308)
\curveto(905.79217542,382.03896534)(905.98217523,382.09396529)(906.18217773,382.14397308)
\curveto(906.25217496,382.15396523)(906.3171749,382.16396522)(906.37717773,382.17397308)
\curveto(906.44717477,382.1839652)(906.52217469,382.19396519)(906.60217773,382.20397308)
\curveto(906.63217458,382.21396517)(906.67217454,382.21396517)(906.72217773,382.20397308)
\curveto(906.77217444,382.19396519)(906.80717441,382.19896518)(906.82717773,382.21897308)
}
}
{
\newrgbcolor{curcolor}{0 0 0}
\pscustom[linestyle=none,fillstyle=solid,fillcolor=curcolor]
{
\newpath
\moveto(831.96144531,361.67765076)
\curveto(832.03144364,361.67764006)(832.11144356,361.67764006)(832.20144531,361.67765076)
\curveto(832.29144338,361.68764005)(832.37644329,361.68764005)(832.45644531,361.67765076)
\curveto(832.54644312,361.67764006)(832.62644304,361.66764007)(832.69644531,361.64765076)
\curveto(832.7664429,361.62764011)(832.81644285,361.59764014)(832.84644531,361.55765076)
\curveto(832.90644276,361.47764026)(832.93644273,361.37264037)(832.93644531,361.24265076)
\lineto(832.93644531,360.83765076)
\lineto(832.93644531,359.26265076)
\lineto(832.93644531,354.44765076)
\lineto(832.93644531,353.06765076)
\lineto(832.93644531,352.70765076)
\curveto(832.93644273,352.57764916)(832.95144272,352.47264927)(832.98144531,352.39265076)
\curveto(833.01144266,352.32264942)(833.0664426,352.26264948)(833.14644531,352.21265076)
\curveto(833.19644247,352.19264955)(833.26144241,352.17764956)(833.34144531,352.16765076)
\lineto(833.58144531,352.16765076)
\lineto(834.34644531,352.16765076)
\lineto(837.04644531,352.16765076)
\lineto(837.88644531,352.16765076)
\lineto(838.09644531,352.16765076)
\curveto(838.17643749,352.17764956)(838.24643742,352.17264957)(838.30644531,352.15265076)
\curveto(838.43643723,352.11264963)(838.52143715,352.05764968)(838.56144531,351.98765076)
\curveto(838.5714371,351.95764978)(838.58143709,351.90764983)(838.59144531,351.83765076)
\curveto(838.60143707,351.76764997)(838.60643706,351.69265005)(838.60644531,351.61265076)
\curveto(838.61643705,351.5426502)(838.61643705,351.46765027)(838.60644531,351.38765076)
\curveto(838.59643707,351.31765042)(838.58643708,351.26265048)(838.57644531,351.22265076)
\curveto(838.55643711,351.15265059)(838.51143716,351.09765064)(838.44144531,351.05765076)
\curveto(838.39143728,351.01765072)(838.31643735,350.99765074)(838.21644531,350.99765076)
\lineto(837.94644531,350.99765076)
\lineto(836.95644531,350.99765076)
\lineto(833.16144531,350.99765076)
\lineto(832.18644531,350.99765076)
\curveto(832.04644362,350.99765074)(831.92644374,351.00265074)(831.82644531,351.01265076)
\curveto(831.72644394,351.03265071)(831.65144402,351.08265066)(831.60144531,351.16265076)
\curveto(831.56144411,351.22265052)(831.54144413,351.29765044)(831.54144531,351.38765076)
\lineto(831.54144531,351.67265076)
\lineto(831.54144531,352.75265076)
\lineto(831.54144531,356.84765076)
\lineto(831.54144531,360.10265076)
\lineto(831.54144531,361.03265076)
\lineto(831.54144531,361.30265076)
\curveto(831.55144412,361.39264035)(831.5714441,361.46264028)(831.60144531,361.51265076)
\curveto(831.64144403,361.57264017)(831.71644395,361.62264012)(831.82644531,361.66265076)
\curveto(831.84644382,361.67264007)(831.8664438,361.67264007)(831.88644531,361.66265076)
\curveto(831.91644375,361.66264008)(831.94144373,361.66764007)(831.96144531,361.67765076)
}
}
{
\newrgbcolor{curcolor}{0 0 0}
\pscustom[linestyle=none,fillstyle=solid,fillcolor=curcolor]
{
\newpath
\moveto(846.74605469,355.19765076)
\curveto(846.76604663,355.1376466)(846.77604662,355.0426467)(846.77605469,354.91265076)
\curveto(846.77604662,354.79264695)(846.77104662,354.70764703)(846.76105469,354.65765076)
\lineto(846.76105469,354.50765076)
\curveto(846.75104664,354.42764731)(846.74104665,354.35264739)(846.73105469,354.28265076)
\curveto(846.73104666,354.22264752)(846.72604667,354.15264759)(846.71605469,354.07265076)
\curveto(846.6960467,354.01264773)(846.68104671,353.95264779)(846.67105469,353.89265076)
\curveto(846.67104672,353.83264791)(846.66104673,353.77264797)(846.64105469,353.71265076)
\curveto(846.60104679,353.58264816)(846.56604683,353.45264829)(846.53605469,353.32265076)
\curveto(846.50604689,353.19264855)(846.46604693,353.07264867)(846.41605469,352.96265076)
\curveto(846.20604719,352.48264926)(845.92604747,352.07764966)(845.57605469,351.74765076)
\curveto(845.22604817,351.42765031)(844.7960486,351.18265056)(844.28605469,351.01265076)
\curveto(844.17604922,350.97265077)(844.05604934,350.9426508)(843.92605469,350.92265076)
\curveto(843.80604959,350.90265084)(843.68104971,350.88265086)(843.55105469,350.86265076)
\curveto(843.4910499,350.85265089)(843.42604997,350.84765089)(843.35605469,350.84765076)
\curveto(843.2960501,350.8376509)(843.23605016,350.83265091)(843.17605469,350.83265076)
\curveto(843.13605026,350.82265092)(843.07605032,350.81765092)(842.99605469,350.81765076)
\curveto(842.92605047,350.81765092)(842.87605052,350.82265092)(842.84605469,350.83265076)
\curveto(842.80605059,350.8426509)(842.76605063,350.84765089)(842.72605469,350.84765076)
\curveto(842.68605071,350.8376509)(842.65105074,350.8376509)(842.62105469,350.84765076)
\lineto(842.53105469,350.84765076)
\lineto(842.17105469,350.89265076)
\curveto(842.03105136,350.93265081)(841.8960515,350.97265077)(841.76605469,351.01265076)
\curveto(841.63605176,351.05265069)(841.51105188,351.09765064)(841.39105469,351.14765076)
\curveto(840.94105245,351.34765039)(840.57105282,351.60765013)(840.28105469,351.92765076)
\curveto(839.9910534,352.24764949)(839.75105364,352.6376491)(839.56105469,353.09765076)
\curveto(839.51105388,353.19764854)(839.47105392,353.29764844)(839.44105469,353.39765076)
\curveto(839.42105397,353.49764824)(839.40105399,353.60264814)(839.38105469,353.71265076)
\curveto(839.36105403,353.75264799)(839.35105404,353.78264796)(839.35105469,353.80265076)
\curveto(839.36105403,353.83264791)(839.36105403,353.86764787)(839.35105469,353.90765076)
\curveto(839.33105406,353.98764775)(839.31605408,354.06764767)(839.30605469,354.14765076)
\curveto(839.30605409,354.2376475)(839.2960541,354.32264742)(839.27605469,354.40265076)
\lineto(839.27605469,354.52265076)
\curveto(839.27605412,354.56264718)(839.27105412,354.60764713)(839.26105469,354.65765076)
\curveto(839.25105414,354.70764703)(839.24605415,354.79264695)(839.24605469,354.91265076)
\curveto(839.24605415,355.0426467)(839.25605414,355.1376466)(839.27605469,355.19765076)
\curveto(839.2960541,355.26764647)(839.30105409,355.3376464)(839.29105469,355.40765076)
\curveto(839.28105411,355.47764626)(839.28605411,355.54764619)(839.30605469,355.61765076)
\curveto(839.31605408,355.66764607)(839.32105407,355.70764603)(839.32105469,355.73765076)
\curveto(839.33105406,355.77764596)(839.34105405,355.82264592)(839.35105469,355.87265076)
\curveto(839.38105401,355.99264575)(839.40605399,356.11264563)(839.42605469,356.23265076)
\curveto(839.45605394,356.35264539)(839.4960539,356.46764527)(839.54605469,356.57765076)
\curveto(839.6960537,356.94764479)(839.87605352,357.27764446)(840.08605469,357.56765076)
\curveto(840.30605309,357.86764387)(840.57105282,358.11764362)(840.88105469,358.31765076)
\curveto(841.00105239,358.39764334)(841.12605227,358.46264328)(841.25605469,358.51265076)
\curveto(841.38605201,358.57264317)(841.52105187,358.63264311)(841.66105469,358.69265076)
\curveto(841.78105161,358.742643)(841.91105148,358.77264297)(842.05105469,358.78265076)
\curveto(842.1910512,358.80264294)(842.33105106,358.83264291)(842.47105469,358.87265076)
\lineto(842.66605469,358.87265076)
\curveto(842.73605066,358.88264286)(842.80105059,358.89264285)(842.86105469,358.90265076)
\curveto(843.75104964,358.91264283)(844.4910489,358.72764301)(845.08105469,358.34765076)
\curveto(845.67104772,357.96764377)(846.0960473,357.47264427)(846.35605469,356.86265076)
\curveto(846.40604699,356.76264498)(846.44604695,356.66264508)(846.47605469,356.56265076)
\curveto(846.50604689,356.46264528)(846.54104685,356.35764538)(846.58105469,356.24765076)
\curveto(846.61104678,356.1376456)(846.63604676,356.01764572)(846.65605469,355.88765076)
\curveto(846.67604672,355.76764597)(846.70104669,355.6426461)(846.73105469,355.51265076)
\curveto(846.74104665,355.46264628)(846.74104665,355.40764633)(846.73105469,355.34765076)
\curveto(846.73104666,355.29764644)(846.73604666,355.24764649)(846.74605469,355.19765076)
\moveto(845.41105469,354.34265076)
\curveto(845.43104796,354.41264733)(845.43604796,354.49264725)(845.42605469,354.58265076)
\lineto(845.42605469,354.83765076)
\curveto(845.42604797,355.22764651)(845.391048,355.55764618)(845.32105469,355.82765076)
\curveto(845.2910481,355.90764583)(845.26604813,355.98764575)(845.24605469,356.06765076)
\curveto(845.22604817,356.14764559)(845.20104819,356.22264552)(845.17105469,356.29265076)
\curveto(844.8910485,356.9426448)(844.44604895,357.39264435)(843.83605469,357.64265076)
\curveto(843.76604963,357.67264407)(843.6910497,357.69264405)(843.61105469,357.70265076)
\lineto(843.37105469,357.76265076)
\curveto(843.2910501,357.78264396)(843.20605019,357.79264395)(843.11605469,357.79265076)
\lineto(842.84605469,357.79265076)
\lineto(842.57605469,357.74765076)
\curveto(842.47605092,357.72764401)(842.38105101,357.70264404)(842.29105469,357.67265076)
\curveto(842.21105118,357.65264409)(842.13105126,357.62264412)(842.05105469,357.58265076)
\curveto(841.98105141,357.56264418)(841.91605148,357.53264421)(841.85605469,357.49265076)
\curveto(841.7960516,357.45264429)(841.74105165,357.41264433)(841.69105469,357.37265076)
\curveto(841.45105194,357.20264454)(841.25605214,356.99764474)(841.10605469,356.75765076)
\curveto(840.95605244,356.51764522)(840.82605257,356.2376455)(840.71605469,355.91765076)
\curveto(840.68605271,355.81764592)(840.66605273,355.71264603)(840.65605469,355.60265076)
\curveto(840.64605275,355.50264624)(840.63105276,355.39764634)(840.61105469,355.28765076)
\curveto(840.60105279,355.24764649)(840.5960528,355.18264656)(840.59605469,355.09265076)
\curveto(840.58605281,355.06264668)(840.58105281,355.02764671)(840.58105469,354.98765076)
\curveto(840.5910528,354.94764679)(840.5960528,354.90264684)(840.59605469,354.85265076)
\lineto(840.59605469,354.55265076)
\curveto(840.5960528,354.45264729)(840.60605279,354.36264738)(840.62605469,354.28265076)
\lineto(840.65605469,354.10265076)
\curveto(840.67605272,354.00264774)(840.6910527,353.90264784)(840.70105469,353.80265076)
\curveto(840.72105267,353.71264803)(840.75105264,353.62764811)(840.79105469,353.54765076)
\curveto(840.8910525,353.30764843)(841.00605239,353.08264866)(841.13605469,352.87265076)
\curveto(841.27605212,352.66264908)(841.44605195,352.48764925)(841.64605469,352.34765076)
\curveto(841.6960517,352.31764942)(841.74105165,352.29264945)(841.78105469,352.27265076)
\curveto(841.82105157,352.25264949)(841.86605153,352.22764951)(841.91605469,352.19765076)
\curveto(841.9960514,352.14764959)(842.08105131,352.10264964)(842.17105469,352.06265076)
\curveto(842.27105112,352.03264971)(842.37605102,352.00264974)(842.48605469,351.97265076)
\curveto(842.53605086,351.95264979)(842.58105081,351.9426498)(842.62105469,351.94265076)
\curveto(842.67105072,351.95264979)(842.72105067,351.95264979)(842.77105469,351.94265076)
\curveto(842.80105059,351.93264981)(842.86105053,351.92264982)(842.95105469,351.91265076)
\curveto(843.05105034,351.90264984)(843.12605027,351.90764983)(843.17605469,351.92765076)
\curveto(843.21605018,351.9376498)(843.25605014,351.9376498)(843.29605469,351.92765076)
\curveto(843.33605006,351.92764981)(843.37605002,351.9376498)(843.41605469,351.95765076)
\curveto(843.4960499,351.97764976)(843.57604982,351.99264975)(843.65605469,352.00265076)
\curveto(843.73604966,352.02264972)(843.81104958,352.04764969)(843.88105469,352.07765076)
\curveto(844.22104917,352.21764952)(844.4960489,352.41264933)(844.70605469,352.66265076)
\curveto(844.91604848,352.91264883)(845.0910483,353.20764853)(845.23105469,353.54765076)
\curveto(845.28104811,353.66764807)(845.31104808,353.79264795)(845.32105469,353.92265076)
\curveto(845.34104805,354.06264768)(845.37104802,354.20264754)(845.41105469,354.34265076)
}
}
{
\newrgbcolor{curcolor}{0 0 0}
\pscustom[linestyle=none,fillstyle=solid,fillcolor=curcolor]
{
\newpath
\moveto(854.84933594,358.61765076)
\curveto(854.91932834,358.56764317)(854.9543283,358.49264325)(854.95433594,358.39265076)
\curveto(854.96432829,358.29264345)(854.96932829,358.18764355)(854.96933594,358.07765076)
\lineto(854.96933594,351.80765076)
\lineto(854.96933594,351.20765076)
\curveto(854.94932831,351.15765058)(854.94432831,351.10765063)(854.95433594,351.05765076)
\curveto(854.96432829,351.01765072)(854.9593283,350.97265077)(854.93933594,350.92265076)
\curveto(854.91932834,350.82265092)(854.90432835,350.72265102)(854.89433594,350.62265076)
\curveto(854.89432836,350.51265123)(854.87932838,350.40765133)(854.84933594,350.30765076)
\curveto(854.81932844,350.19765154)(854.78932847,350.09265165)(854.75933594,349.99265076)
\curveto(854.73932852,349.89265185)(854.70432855,349.79265195)(854.65433594,349.69265076)
\curveto(854.5543287,349.43265231)(854.42432883,349.19765254)(854.26433594,348.98765076)
\curveto(854.11432914,348.77765296)(853.93432932,348.60265314)(853.72433594,348.46265076)
\curveto(853.5543297,348.3426534)(853.37432988,348.24765349)(853.18433594,348.17765076)
\curveto(852.99433026,348.09765364)(852.78933047,348.02265372)(852.56933594,347.95265076)
\curveto(852.47933078,347.93265381)(852.38933087,347.92265382)(852.29933594,347.92265076)
\curveto(852.20933105,347.91265383)(852.11933114,347.89765384)(852.02933594,347.87765076)
\lineto(851.93933594,347.87765076)
\curveto(851.91933134,347.86765387)(851.89933136,347.86265388)(851.87933594,347.86265076)
\curveto(851.82933143,347.85265389)(851.77933148,347.85265389)(851.72933594,347.86265076)
\curveto(851.68933157,347.87265387)(851.64433161,347.86765387)(851.59433594,347.84765076)
\curveto(851.52433173,347.82765391)(851.41433184,347.82265392)(851.26433594,347.83265076)
\curveto(851.12433213,347.83265391)(851.02433223,347.8426539)(850.96433594,347.86265076)
\curveto(850.93433232,347.86265388)(850.90433235,347.86765387)(850.87433594,347.87765076)
\lineto(850.81433594,347.87765076)
\curveto(850.72433253,347.89765384)(850.63433262,347.91265383)(850.54433594,347.92265076)
\curveto(850.4543328,347.92265382)(850.36933289,347.93265381)(850.28933594,347.95265076)
\curveto(850.20933305,347.97265377)(850.12933313,347.99765374)(850.04933594,348.02765076)
\curveto(849.96933329,348.04765369)(849.88933337,348.07265367)(849.80933594,348.10265076)
\curveto(849.48933377,348.23265351)(849.21933404,348.37765336)(848.99933594,348.53765076)
\curveto(848.78933447,348.69765304)(848.59933466,348.92265282)(848.42933594,349.21265076)
\curveto(848.40933485,349.23265251)(848.39433486,349.25765248)(848.38433594,349.28765076)
\curveto(848.38433487,349.30765243)(848.37433488,349.33265241)(848.35433594,349.36265076)
\curveto(848.32433493,349.4426523)(848.28933497,349.55765218)(848.24933594,349.70765076)
\curveto(848.21933504,349.84765189)(848.24933501,349.95265179)(848.33933594,350.02265076)
\curveto(848.39933486,350.07265167)(848.47933478,350.09765164)(848.57933594,350.09765076)
\lineto(848.90933594,350.09765076)
\lineto(849.07433594,350.09765076)
\curveto(849.13433412,350.09765164)(849.18933407,350.08765165)(849.23933594,350.06765076)
\curveto(849.32933393,350.0376517)(849.39433386,349.98765175)(849.43433594,349.91765076)
\curveto(849.47433378,349.84765189)(849.51933374,349.77265197)(849.56933594,349.69265076)
\lineto(849.68933594,349.51265076)
\curveto(849.73933352,349.4426523)(849.78933347,349.38765235)(849.83933594,349.34765076)
\curveto(850.08933317,349.15765258)(850.38933287,349.01765272)(850.73933594,348.92765076)
\curveto(850.79933246,348.90765283)(850.8593324,348.89765284)(850.91933594,348.89765076)
\curveto(850.98933227,348.88765285)(851.0543322,348.87265287)(851.11433594,348.85265076)
\lineto(851.20433594,348.85265076)
\curveto(851.27433198,348.83265291)(851.3593319,348.82265292)(851.45933594,348.82265076)
\curveto(851.5593317,348.82265292)(851.64933161,348.83265291)(851.72933594,348.85265076)
\curveto(851.7593315,348.86265288)(851.79933146,348.86765287)(851.84933594,348.86765076)
\curveto(851.94933131,348.88765285)(852.04433121,348.90765283)(852.13433594,348.92765076)
\curveto(852.22433103,348.9376528)(852.30933095,348.96265278)(852.38933594,349.00265076)
\curveto(852.67933058,349.12265262)(852.91433034,349.28765245)(853.09433594,349.49765076)
\curveto(853.28432997,349.69765204)(853.43932982,349.9426518)(853.55933594,350.23265076)
\curveto(853.59932966,350.32265142)(853.62432963,350.41765132)(853.63433594,350.51765076)
\curveto(853.6543296,350.61765112)(853.67932958,350.72265102)(853.70933594,350.83265076)
\curveto(853.72932953,350.88265086)(853.73932952,350.93265081)(853.73933594,350.98265076)
\curveto(853.73932952,351.03265071)(853.74432951,351.08265066)(853.75433594,351.13265076)
\curveto(853.76432949,351.16265058)(853.76932949,351.21265053)(853.76933594,351.28265076)
\curveto(853.78932947,351.36265038)(853.78932947,351.44765029)(853.76933594,351.53765076)
\curveto(853.7593295,351.58765015)(853.7543295,351.63265011)(853.75433594,351.67265076)
\curveto(853.76432949,351.71265003)(853.7593295,351.74764999)(853.73933594,351.77765076)
\curveto(853.71932954,351.79764994)(853.70432955,351.80764993)(853.69433594,351.80765076)
\lineto(853.64933594,351.85265076)
\curveto(853.54932971,351.85264989)(853.47432978,351.82264992)(853.42433594,351.76265076)
\curveto(853.38432987,351.71265003)(853.33432992,351.66765007)(853.27433594,351.62765076)
\lineto(853.03433594,351.41765076)
\curveto(852.9543303,351.35765038)(852.86433039,351.30265044)(852.76433594,351.25265076)
\curveto(852.62433063,351.16265058)(852.44933081,351.08765065)(852.23933594,351.02765076)
\curveto(852.02933123,350.97765076)(851.80933145,350.9426508)(851.57933594,350.92265076)
\curveto(851.34933191,350.90265084)(851.11933214,350.90765083)(850.88933594,350.93765076)
\curveto(850.6593326,350.95765078)(850.44933281,350.99765074)(850.25933594,351.05765076)
\curveto(849.31933394,351.36765037)(848.6593346,351.96264978)(848.27933594,352.84265076)
\curveto(848.22933503,352.9426488)(848.18933507,353.0376487)(848.15933594,353.12765076)
\curveto(848.12933513,353.22764851)(848.09433516,353.33264841)(848.05433594,353.44265076)
\curveto(848.03433522,353.49264825)(848.02433523,353.5376482)(848.02433594,353.57765076)
\curveto(848.02433523,353.61764812)(848.01433524,353.66264808)(847.99433594,353.71265076)
\curveto(847.97433528,353.78264796)(847.9593353,353.85264789)(847.94933594,353.92265076)
\curveto(847.94933531,354.00264774)(847.93933532,354.07764766)(847.91933594,354.14765076)
\curveto(847.90933535,354.18764755)(847.90433535,354.22264752)(847.90433594,354.25265076)
\curveto(847.91433534,354.29264745)(847.91433534,354.33264741)(847.90433594,354.37265076)
\curveto(847.90433535,354.41264733)(847.89933536,354.45264729)(847.88933594,354.49265076)
\lineto(847.88933594,354.61265076)
\curveto(847.86933539,354.73264701)(847.86933539,354.85764688)(847.88933594,354.98765076)
\curveto(847.89933536,355.04764669)(847.90433535,355.10764663)(847.90433594,355.16765076)
\lineto(847.90433594,355.33265076)
\curveto(847.91433534,355.38264636)(847.91933534,355.42264632)(847.91933594,355.45265076)
\curveto(847.91933534,355.49264625)(847.92433533,355.5376462)(847.93433594,355.58765076)
\curveto(847.96433529,355.69764604)(847.98433527,355.80264594)(847.99433594,355.90265076)
\curveto(848.00433525,356.01264573)(848.02933523,356.12264562)(848.06933594,356.23265076)
\curveto(848.10933515,356.35264539)(848.14433511,356.46764527)(848.17433594,356.57765076)
\curveto(848.21433504,356.69764504)(848.259335,356.81264493)(848.30933594,356.92265076)
\curveto(848.37933488,357.08264466)(848.4593348,357.22764451)(848.54933594,357.35765076)
\curveto(848.63933462,357.49764424)(848.73433452,357.63264411)(848.83433594,357.76265076)
\curveto(848.90433435,357.87264387)(848.99433426,357.96264378)(849.10433594,358.03265076)
\lineto(849.16433594,358.09265076)
\lineto(849.22433594,358.15265076)
\lineto(849.37433594,358.27265076)
\lineto(849.55433594,358.39265076)
\curveto(849.68433357,358.47264327)(849.81933344,358.5426432)(849.95933594,358.60265076)
\curveto(850.10933315,358.66264308)(850.26933299,358.71764302)(850.43933594,358.76765076)
\curveto(850.53933272,358.79764294)(850.63933262,358.81764292)(850.73933594,358.82765076)
\curveto(850.84933241,358.8376429)(850.9593323,358.85264289)(851.06933594,358.87265076)
\curveto(851.10933215,358.88264286)(851.1593321,358.88264286)(851.21933594,358.87265076)
\curveto(851.28933197,358.86264288)(851.33933192,358.86764287)(851.36933594,358.88765076)
\curveto(851.68933157,358.89764284)(851.97433128,358.86764287)(852.22433594,358.79765076)
\curveto(852.48433077,358.72764301)(852.71433054,358.62764311)(852.91433594,358.49765076)
\curveto(852.98433027,358.45764328)(853.04933021,358.41264333)(853.10933594,358.36265076)
\lineto(853.28933594,358.21265076)
\curveto(853.33932992,358.17264357)(853.38432987,358.12764361)(853.42433594,358.07765076)
\curveto(853.47432978,358.0376437)(853.54932971,358.01764372)(853.64933594,358.01765076)
\lineto(853.69433594,358.06265076)
\curveto(853.71432954,358.08264366)(853.73432952,358.10764363)(853.75433594,358.13765076)
\curveto(853.78432947,358.21764352)(853.79932946,358.29764344)(853.79933594,358.37765076)
\curveto(853.80932945,358.45764328)(853.83932942,358.52764321)(853.88933594,358.58765076)
\curveto(853.91932934,358.62764311)(853.97932928,358.65764308)(854.06933594,358.67765076)
\curveto(854.1593291,358.70764303)(854.254329,358.72264302)(854.35433594,358.72265076)
\curveto(854.4543288,358.72264302)(854.54932871,358.71264303)(854.63933594,358.69265076)
\curveto(854.73932852,358.67264307)(854.80932845,358.64764309)(854.84933594,358.61765076)
\moveto(853.72433594,354.83765076)
\curveto(853.73432952,354.87764686)(853.73932952,354.92764681)(853.73933594,354.98765076)
\curveto(853.73932952,355.05764668)(853.73432952,355.11264663)(853.72433594,355.15265076)
\lineto(853.72433594,355.39265076)
\curveto(853.70432955,355.48264626)(853.68932957,355.56764617)(853.67933594,355.64765076)
\curveto(853.66932959,355.737646)(853.6543296,355.82264592)(853.63433594,355.90265076)
\curveto(853.61432964,355.98264576)(853.59432966,356.05764568)(853.57433594,356.12765076)
\curveto(853.56432969,356.20764553)(853.54432971,356.28264546)(853.51433594,356.35265076)
\curveto(853.40432985,356.63264511)(853.25933,356.88264486)(853.07933594,357.10265076)
\curveto(852.90933035,357.32264442)(852.68933057,357.48764425)(852.41933594,357.59765076)
\curveto(852.33933092,357.6376441)(852.254331,357.66764407)(852.16433594,357.68765076)
\curveto(852.07433118,357.71764402)(851.97933128,357.742644)(851.87933594,357.76265076)
\curveto(851.79933146,357.78264396)(851.70933155,357.78764395)(851.60933594,357.77765076)
\lineto(851.33933594,357.77765076)
\curveto(851.28933197,357.76764397)(851.23933202,357.76264398)(851.18933594,357.76265076)
\curveto(851.14933211,357.76264398)(851.10433215,357.75764398)(851.05433594,357.74765076)
\curveto(850.86433239,357.69764404)(850.70433255,357.64764409)(850.57433594,357.59765076)
\curveto(850.23433302,357.45764428)(849.96933329,357.24764449)(849.77933594,356.96765076)
\curveto(849.58933367,356.68764505)(849.43933382,356.36264538)(849.32933594,355.99265076)
\curveto(849.30933395,355.91264583)(849.29433396,355.83264591)(849.28433594,355.75265076)
\curveto(849.28433397,355.68264606)(849.27433398,355.60764613)(849.25433594,355.52765076)
\curveto(849.23433402,355.49764624)(849.22433403,355.46264628)(849.22433594,355.42265076)
\curveto(849.23433402,355.38264636)(849.23433402,355.34764639)(849.22433594,355.31765076)
\lineto(849.22433594,354.98765076)
\lineto(849.22433594,354.64265076)
\curveto(849.22433403,354.53264721)(849.23433402,354.42764731)(849.25433594,354.32765076)
\lineto(849.25433594,354.25265076)
\curveto(849.26433399,354.22264752)(849.26933399,354.19764754)(849.26933594,354.17765076)
\curveto(849.28933397,354.08764765)(849.30433395,353.99764774)(849.31433594,353.90765076)
\curveto(849.33433392,353.81764792)(849.3593339,353.73264801)(849.38933594,353.65265076)
\curveto(849.46933379,353.39264835)(849.56933369,353.15264859)(849.68933594,352.93265076)
\curveto(849.80933345,352.71264903)(849.96933329,352.53264921)(850.16933594,352.39265076)
\lineto(850.28933594,352.30265076)
\curveto(850.32933293,352.28264946)(850.37433288,352.26264948)(850.42433594,352.24265076)
\curveto(850.50433275,352.19264955)(850.58933267,352.15264959)(850.67933594,352.12265076)
\curveto(850.76933249,352.09264965)(850.86933239,352.06264968)(850.97933594,352.03265076)
\curveto(851.02933223,352.02264972)(851.07433218,352.01764972)(851.11433594,352.01765076)
\curveto(851.16433209,352.02764971)(851.21433204,352.02264972)(851.26433594,352.00265076)
\curveto(851.29433196,351.99264975)(851.34433191,351.98764975)(851.41433594,351.98765076)
\curveto(851.48433177,351.98764975)(851.53433172,351.99264975)(851.56433594,352.00265076)
\curveto(851.59433166,352.01264973)(851.62433163,352.01264973)(851.65433594,352.00265076)
\curveto(851.69433156,352.00264974)(851.73433152,352.00764973)(851.77433594,352.01765076)
\curveto(851.86433139,352.0376497)(851.94933131,352.05764968)(852.02933594,352.07765076)
\curveto(852.10933115,352.09764964)(852.18933107,352.12264962)(852.26933594,352.15265076)
\curveto(852.60933065,352.30264944)(852.87933038,352.51264923)(853.07933594,352.78265076)
\curveto(853.27932998,353.05264869)(853.43932982,353.36764837)(853.55933594,353.72765076)
\curveto(853.58932967,353.81764792)(853.60932965,353.90764783)(853.61933594,353.99765076)
\curveto(853.63932962,354.09764764)(853.6593296,354.19264755)(853.67933594,354.28265076)
\curveto(853.68932957,354.32264742)(853.69432956,354.35764738)(853.69433594,354.38765076)
\curveto(853.69432956,354.42764731)(853.69932956,354.46764727)(853.70933594,354.50765076)
\curveto(853.72932953,354.55764718)(853.72932953,354.60764713)(853.70933594,354.65765076)
\curveto(853.69932956,354.71764702)(853.70432955,354.77764696)(853.72433594,354.83765076)
}
}
{
\newrgbcolor{curcolor}{0 0 0}
\pscustom[linestyle=none,fillstyle=solid,fillcolor=curcolor]
{
\newpath
\moveto(863.59761719,355.16765076)
\curveto(863.6176095,355.06764667)(863.6176095,354.95264679)(863.59761719,354.82265076)
\curveto(863.58760953,354.70264704)(863.55760956,354.61764712)(863.50761719,354.56765076)
\curveto(863.45760966,354.52764721)(863.38260974,354.49764724)(863.28261719,354.47765076)
\curveto(863.19260993,354.46764727)(863.08761003,354.46264728)(862.96761719,354.46265076)
\lineto(862.60761719,354.46265076)
\curveto(862.48761063,354.47264727)(862.38261074,354.47764726)(862.29261719,354.47765076)
\lineto(858.45261719,354.47765076)
\curveto(858.37261475,354.47764726)(858.29261483,354.47264727)(858.21261719,354.46265076)
\curveto(858.13261499,354.46264728)(858.06761505,354.44764729)(858.01761719,354.41765076)
\curveto(857.97761514,354.39764734)(857.93761518,354.35764738)(857.89761719,354.29765076)
\curveto(857.87761524,354.26764747)(857.85761526,354.22264752)(857.83761719,354.16265076)
\curveto(857.8176153,354.11264763)(857.8176153,354.06264768)(857.83761719,354.01265076)
\curveto(857.84761527,353.96264778)(857.85261527,353.91764782)(857.85261719,353.87765076)
\curveto(857.85261527,353.8376479)(857.85761526,353.79764794)(857.86761719,353.75765076)
\curveto(857.88761523,353.67764806)(857.90761521,353.59264815)(857.92761719,353.50265076)
\curveto(857.94761517,353.42264832)(857.97761514,353.3426484)(858.01761719,353.26265076)
\curveto(858.24761487,352.72264902)(858.62761449,352.3376494)(859.15761719,352.10765076)
\curveto(859.2176139,352.07764966)(859.28261384,352.05264969)(859.35261719,352.03265076)
\lineto(859.56261719,351.97265076)
\curveto(859.59261353,351.96264978)(859.64261348,351.95764978)(859.71261719,351.95765076)
\curveto(859.85261327,351.91764982)(860.03761308,351.89764984)(860.26761719,351.89765076)
\curveto(860.49761262,351.89764984)(860.68261244,351.91764982)(860.82261719,351.95765076)
\curveto(860.96261216,351.99764974)(861.08761203,352.0376497)(861.19761719,352.07765076)
\curveto(861.3176118,352.12764961)(861.42761169,352.18764955)(861.52761719,352.25765076)
\curveto(861.63761148,352.32764941)(861.73261139,352.40764933)(861.81261719,352.49765076)
\curveto(861.89261123,352.59764914)(861.96261116,352.70264904)(862.02261719,352.81265076)
\curveto(862.08261104,352.91264883)(862.13261099,353.01764872)(862.17261719,353.12765076)
\curveto(862.2226109,353.2376485)(862.30261082,353.31764842)(862.41261719,353.36765076)
\curveto(862.45261067,353.38764835)(862.5176106,353.40264834)(862.60761719,353.41265076)
\curveto(862.69761042,353.42264832)(862.78761033,353.42264832)(862.87761719,353.41265076)
\curveto(862.96761015,353.41264833)(863.05261007,353.40764833)(863.13261719,353.39765076)
\curveto(863.21260991,353.38764835)(863.26760985,353.36764837)(863.29761719,353.33765076)
\curveto(863.39760972,353.26764847)(863.4226097,353.15264859)(863.37261719,352.99265076)
\curveto(863.29260983,352.72264902)(863.18760993,352.48264926)(863.05761719,352.27265076)
\curveto(862.85761026,351.95264979)(862.62761049,351.68765005)(862.36761719,351.47765076)
\curveto(862.117611,351.27765046)(861.79761132,351.11265063)(861.40761719,350.98265076)
\curveto(861.30761181,350.9426508)(861.20761191,350.91765082)(861.10761719,350.90765076)
\curveto(861.00761211,350.88765085)(860.90261222,350.86765087)(860.79261719,350.84765076)
\curveto(860.74261238,350.8376509)(860.69261243,350.83265091)(860.64261719,350.83265076)
\curveto(860.60261252,350.83265091)(860.55761256,350.82765091)(860.50761719,350.81765076)
\lineto(860.35761719,350.81765076)
\curveto(860.30761281,350.80765093)(860.24761287,350.80265094)(860.17761719,350.80265076)
\curveto(860.117613,350.80265094)(860.06761305,350.80765093)(860.02761719,350.81765076)
\lineto(859.89261719,350.81765076)
\curveto(859.84261328,350.82765091)(859.79761332,350.83265091)(859.75761719,350.83265076)
\curveto(859.7176134,350.83265091)(859.67761344,350.8376509)(859.63761719,350.84765076)
\curveto(859.58761353,350.85765088)(859.53261359,350.86765087)(859.47261719,350.87765076)
\curveto(859.41261371,350.87765086)(859.35761376,350.88265086)(859.30761719,350.89265076)
\curveto(859.2176139,350.91265083)(859.12761399,350.9376508)(859.03761719,350.96765076)
\curveto(858.94761417,350.98765075)(858.86261426,351.01265073)(858.78261719,351.04265076)
\curveto(858.74261438,351.06265068)(858.70761441,351.07265067)(858.67761719,351.07265076)
\curveto(858.64761447,351.08265066)(858.61261451,351.09765064)(858.57261719,351.11765076)
\curveto(858.4226147,351.18765055)(858.26261486,351.27265047)(858.09261719,351.37265076)
\curveto(857.80261532,351.56265018)(857.55261557,351.79264995)(857.34261719,352.06265076)
\curveto(857.14261598,352.3426494)(856.97261615,352.65264909)(856.83261719,352.99265076)
\curveto(856.78261634,353.10264864)(856.74261638,353.21764852)(856.71261719,353.33765076)
\curveto(856.69261643,353.45764828)(856.66261646,353.57764816)(856.62261719,353.69765076)
\curveto(856.61261651,353.737648)(856.60761651,353.77264797)(856.60761719,353.80265076)
\curveto(856.60761651,353.83264791)(856.60261652,353.87264787)(856.59261719,353.92265076)
\curveto(856.57261655,354.00264774)(856.55761656,354.08764765)(856.54761719,354.17765076)
\curveto(856.53761658,354.26764747)(856.5226166,354.35764738)(856.50261719,354.44765076)
\lineto(856.50261719,354.65765076)
\curveto(856.49261663,354.69764704)(856.48261664,354.75264699)(856.47261719,354.82265076)
\curveto(856.47261665,354.90264684)(856.47761664,354.96764677)(856.48761719,355.01765076)
\lineto(856.48761719,355.18265076)
\curveto(856.50761661,355.23264651)(856.51261661,355.28264646)(856.50261719,355.33265076)
\curveto(856.50261662,355.39264635)(856.50761661,355.44764629)(856.51761719,355.49765076)
\curveto(856.55761656,355.65764608)(856.58761653,355.81764592)(856.60761719,355.97765076)
\curveto(856.63761648,356.1376456)(856.68261644,356.28764545)(856.74261719,356.42765076)
\curveto(856.79261633,356.5376452)(856.83761628,356.64764509)(856.87761719,356.75765076)
\curveto(856.92761619,356.87764486)(856.98261614,356.99264475)(857.04261719,357.10265076)
\curveto(857.26261586,357.45264429)(857.51261561,357.75264399)(857.79261719,358.00265076)
\curveto(858.07261505,358.26264348)(858.4176147,358.47764326)(858.82761719,358.64765076)
\curveto(858.94761417,358.69764304)(859.06761405,358.73264301)(859.18761719,358.75265076)
\curveto(859.3176138,358.78264296)(859.45261367,358.81264293)(859.59261719,358.84265076)
\curveto(859.64261348,358.85264289)(859.68761343,358.85764288)(859.72761719,358.85765076)
\curveto(859.76761335,358.86764287)(859.81261331,358.87264287)(859.86261719,358.87265076)
\curveto(859.88261324,358.88264286)(859.90761321,358.88264286)(859.93761719,358.87265076)
\curveto(859.96761315,358.86264288)(859.99261313,358.86764287)(860.01261719,358.88765076)
\curveto(860.43261269,358.89764284)(860.79761232,358.85264289)(861.10761719,358.75265076)
\curveto(861.4176117,358.66264308)(861.69761142,358.5376432)(861.94761719,358.37765076)
\curveto(861.99761112,358.35764338)(862.03761108,358.32764341)(862.06761719,358.28765076)
\curveto(862.09761102,358.25764348)(862.13261099,358.23264351)(862.17261719,358.21265076)
\curveto(862.25261087,358.15264359)(862.33261079,358.08264366)(862.41261719,358.00265076)
\curveto(862.50261062,357.92264382)(862.57761054,357.8426439)(862.63761719,357.76265076)
\curveto(862.79761032,357.55264419)(862.93261019,357.35264439)(863.04261719,357.16265076)
\curveto(863.11261001,357.05264469)(863.16760995,356.93264481)(863.20761719,356.80265076)
\curveto(863.24760987,356.67264507)(863.29260983,356.5426452)(863.34261719,356.41265076)
\curveto(863.39260973,356.28264546)(863.42760969,356.14764559)(863.44761719,356.00765076)
\curveto(863.47760964,355.86764587)(863.51260961,355.72764601)(863.55261719,355.58765076)
\curveto(863.56260956,355.51764622)(863.56760955,355.44764629)(863.56761719,355.37765076)
\lineto(863.59761719,355.16765076)
\moveto(862.14261719,355.67765076)
\curveto(862.17261095,355.71764602)(862.19761092,355.76764597)(862.21761719,355.82765076)
\curveto(862.23761088,355.89764584)(862.23761088,355.96764577)(862.21761719,356.03765076)
\curveto(862.15761096,356.25764548)(862.07261105,356.46264528)(861.96261719,356.65265076)
\curveto(861.8226113,356.88264486)(861.66761145,357.07764466)(861.49761719,357.23765076)
\curveto(861.32761179,357.39764434)(861.10761201,357.53264421)(860.83761719,357.64265076)
\curveto(860.76761235,357.66264408)(860.69761242,357.67764406)(860.62761719,357.68765076)
\curveto(860.55761256,357.70764403)(860.48261264,357.72764401)(860.40261719,357.74765076)
\curveto(860.3226128,357.76764397)(860.23761288,357.77764396)(860.14761719,357.77765076)
\lineto(859.89261719,357.77765076)
\curveto(859.86261326,357.75764398)(859.82761329,357.74764399)(859.78761719,357.74765076)
\curveto(859.74761337,357.75764398)(859.71261341,357.75764398)(859.68261719,357.74765076)
\lineto(859.44261719,357.68765076)
\curveto(859.37261375,357.67764406)(859.30261382,357.66264408)(859.23261719,357.64265076)
\curveto(858.94261418,357.52264422)(858.70761441,357.37264437)(858.52761719,357.19265076)
\curveto(858.35761476,357.01264473)(858.20261492,356.78764495)(858.06261719,356.51765076)
\curveto(858.03261509,356.46764527)(858.00261512,356.40264534)(857.97261719,356.32265076)
\curveto(857.94261518,356.25264549)(857.9176152,356.17264557)(857.89761719,356.08265076)
\curveto(857.87761524,355.99264575)(857.87261525,355.90764583)(857.88261719,355.82765076)
\curveto(857.89261523,355.74764599)(857.92761519,355.68764605)(857.98761719,355.64765076)
\curveto(858.06761505,355.58764615)(858.20261492,355.55764618)(858.39261719,355.55765076)
\curveto(858.59261453,355.56764617)(858.76261436,355.57264617)(858.90261719,355.57265076)
\lineto(861.18261719,355.57265076)
\curveto(861.33261179,355.57264617)(861.51261161,355.56764617)(861.72261719,355.55765076)
\curveto(861.93261119,355.55764618)(862.07261105,355.59764614)(862.14261719,355.67765076)
}
}
{
\newrgbcolor{curcolor}{0 0 0}
\pscustom[linestyle=none,fillstyle=solid,fillcolor=curcolor]
{
\newpath
\moveto(871.78925781,351.55265076)
\curveto(871.81924998,351.39265035)(871.80425,351.25765048)(871.74425781,351.14765076)
\curveto(871.68425012,351.04765069)(871.6042502,350.97265077)(871.50425781,350.92265076)
\curveto(871.45425035,350.90265084)(871.3992504,350.89265085)(871.33925781,350.89265076)
\curveto(871.28925051,350.89265085)(871.23425057,350.88265086)(871.17425781,350.86265076)
\curveto(870.95425085,350.81265093)(870.73425107,350.82765091)(870.51425781,350.90765076)
\curveto(870.3042515,350.97765076)(870.15925164,351.06765067)(870.07925781,351.17765076)
\curveto(870.02925177,351.24765049)(869.98425182,351.32765041)(869.94425781,351.41765076)
\curveto(869.9042519,351.51765022)(869.85425195,351.59765014)(869.79425781,351.65765076)
\curveto(869.77425203,351.67765006)(869.74925205,351.69765004)(869.71925781,351.71765076)
\curveto(869.6992521,351.73765)(869.66925213,351.74265)(869.62925781,351.73265076)
\curveto(869.51925228,351.70265004)(869.41425239,351.64765009)(869.31425781,351.56765076)
\curveto(869.22425258,351.48765025)(869.13425267,351.41765032)(869.04425781,351.35765076)
\curveto(868.91425289,351.27765046)(868.77425303,351.20265054)(868.62425781,351.13265076)
\curveto(868.47425333,351.07265067)(868.31425349,351.01765072)(868.14425781,350.96765076)
\curveto(868.04425376,350.9376508)(867.93425387,350.91765082)(867.81425781,350.90765076)
\curveto(867.7042541,350.89765084)(867.59425421,350.88265086)(867.48425781,350.86265076)
\curveto(867.43425437,350.85265089)(867.38925441,350.84765089)(867.34925781,350.84765076)
\lineto(867.24425781,350.84765076)
\curveto(867.13425467,350.82765091)(867.02925477,350.82765091)(866.92925781,350.84765076)
\lineto(866.79425781,350.84765076)
\curveto(866.74425506,350.85765088)(866.69425511,350.86265088)(866.64425781,350.86265076)
\curveto(866.59425521,350.86265088)(866.54925525,350.87265087)(866.50925781,350.89265076)
\curveto(866.46925533,350.90265084)(866.43425537,350.90765083)(866.40425781,350.90765076)
\curveto(866.38425542,350.89765084)(866.35925544,350.89765084)(866.32925781,350.90765076)
\lineto(866.08925781,350.96765076)
\curveto(866.00925579,350.97765076)(865.93425587,350.99765074)(865.86425781,351.02765076)
\curveto(865.56425624,351.15765058)(865.31925648,351.30265044)(865.12925781,351.46265076)
\curveto(864.94925685,351.63265011)(864.799257,351.86764987)(864.67925781,352.16765076)
\curveto(864.58925721,352.38764935)(864.54425726,352.65264909)(864.54425781,352.96265076)
\lineto(864.54425781,353.27765076)
\curveto(864.55425725,353.32764841)(864.55925724,353.37764836)(864.55925781,353.42765076)
\lineto(864.58925781,353.60765076)
\lineto(864.70925781,353.93765076)
\curveto(864.74925705,354.04764769)(864.799257,354.14764759)(864.85925781,354.23765076)
\curveto(865.03925676,354.52764721)(865.28425652,354.742647)(865.59425781,354.88265076)
\curveto(865.9042559,355.02264672)(866.24425556,355.14764659)(866.61425781,355.25765076)
\curveto(866.75425505,355.29764644)(866.8992549,355.32764641)(867.04925781,355.34765076)
\curveto(867.1992546,355.36764637)(867.34925445,355.39264635)(867.49925781,355.42265076)
\curveto(867.56925423,355.4426463)(867.63425417,355.45264629)(867.69425781,355.45265076)
\curveto(867.76425404,355.45264629)(867.83925396,355.46264628)(867.91925781,355.48265076)
\curveto(867.98925381,355.50264624)(868.05925374,355.51264623)(868.12925781,355.51265076)
\curveto(868.1992536,355.52264622)(868.27425353,355.5376462)(868.35425781,355.55765076)
\curveto(868.6042532,355.61764612)(868.83925296,355.66764607)(869.05925781,355.70765076)
\curveto(869.27925252,355.75764598)(869.45425235,355.87264587)(869.58425781,356.05265076)
\curveto(869.64425216,356.13264561)(869.69425211,356.23264551)(869.73425781,356.35265076)
\curveto(869.77425203,356.48264526)(869.77425203,356.62264512)(869.73425781,356.77265076)
\curveto(869.67425213,357.01264473)(869.58425222,357.20264454)(869.46425781,357.34265076)
\curveto(869.35425245,357.48264426)(869.19425261,357.59264415)(868.98425781,357.67265076)
\curveto(868.86425294,357.72264402)(868.71925308,357.75764398)(868.54925781,357.77765076)
\curveto(868.38925341,357.79764394)(868.21925358,357.80764393)(868.03925781,357.80765076)
\curveto(867.85925394,357.80764393)(867.68425412,357.79764394)(867.51425781,357.77765076)
\curveto(867.34425446,357.75764398)(867.1992546,357.72764401)(867.07925781,357.68765076)
\curveto(866.90925489,357.62764411)(866.74425506,357.5426442)(866.58425781,357.43265076)
\curveto(866.5042553,357.37264437)(866.42925537,357.29264445)(866.35925781,357.19265076)
\curveto(866.2992555,357.10264464)(866.24425556,357.00264474)(866.19425781,356.89265076)
\curveto(866.16425564,356.81264493)(866.13425567,356.72764501)(866.10425781,356.63765076)
\curveto(866.08425572,356.54764519)(866.03925576,356.47764526)(865.96925781,356.42765076)
\curveto(865.92925587,356.39764534)(865.85925594,356.37264537)(865.75925781,356.35265076)
\curveto(865.66925613,356.3426454)(865.57425623,356.3376454)(865.47425781,356.33765076)
\curveto(865.37425643,356.3376454)(865.27425653,356.3426454)(865.17425781,356.35265076)
\curveto(865.08425672,356.37264537)(865.01925678,356.39764534)(864.97925781,356.42765076)
\curveto(864.93925686,356.45764528)(864.90925689,356.50764523)(864.88925781,356.57765076)
\curveto(864.86925693,356.64764509)(864.86925693,356.72264502)(864.88925781,356.80265076)
\curveto(864.91925688,356.93264481)(864.94925685,357.05264469)(864.97925781,357.16265076)
\curveto(865.01925678,357.28264446)(865.06425674,357.39764434)(865.11425781,357.50765076)
\curveto(865.3042565,357.85764388)(865.54425626,358.12764361)(865.83425781,358.31765076)
\curveto(866.12425568,358.51764322)(866.48425532,358.67764306)(866.91425781,358.79765076)
\curveto(867.01425479,358.81764292)(867.11425469,358.83264291)(867.21425781,358.84265076)
\curveto(867.32425448,358.85264289)(867.43425437,358.86764287)(867.54425781,358.88765076)
\curveto(867.58425422,358.89764284)(867.64925415,358.89764284)(867.73925781,358.88765076)
\curveto(867.82925397,358.88764285)(867.88425392,358.89764284)(867.90425781,358.91765076)
\curveto(868.6042532,358.92764281)(869.21425259,358.84764289)(869.73425781,358.67765076)
\curveto(870.25425155,358.50764323)(870.61925118,358.18264356)(870.82925781,357.70265076)
\curveto(870.91925088,357.50264424)(870.96925083,357.26764447)(870.97925781,356.99765076)
\curveto(870.9992508,356.737645)(871.00925079,356.46264528)(871.00925781,356.17265076)
\lineto(871.00925781,352.85765076)
\curveto(871.00925079,352.71764902)(871.01425079,352.58264916)(871.02425781,352.45265076)
\curveto(871.03425077,352.32264942)(871.06425074,352.21764952)(871.11425781,352.13765076)
\curveto(871.16425064,352.06764967)(871.22925057,352.01764972)(871.30925781,351.98765076)
\curveto(871.3992504,351.94764979)(871.48425032,351.91764982)(871.56425781,351.89765076)
\curveto(871.64425016,351.88764985)(871.7042501,351.8426499)(871.74425781,351.76265076)
\curveto(871.76425004,351.73265001)(871.77425003,351.70265004)(871.77425781,351.67265076)
\curveto(871.77425003,351.6426501)(871.77925002,351.60265014)(871.78925781,351.55265076)
\moveto(869.64425781,353.21765076)
\curveto(869.7042521,353.35764838)(869.73425207,353.51764822)(869.73425781,353.69765076)
\curveto(869.74425206,353.88764785)(869.74925205,354.08264766)(869.74925781,354.28265076)
\curveto(869.74925205,354.39264735)(869.74425206,354.49264725)(869.73425781,354.58265076)
\curveto(869.72425208,354.67264707)(869.68425212,354.742647)(869.61425781,354.79265076)
\curveto(869.58425222,354.81264693)(869.51425229,354.82264692)(869.40425781,354.82265076)
\curveto(869.38425242,354.80264694)(869.34925245,354.79264695)(869.29925781,354.79265076)
\curveto(869.24925255,354.79264695)(869.2042526,354.78264696)(869.16425781,354.76265076)
\curveto(869.08425272,354.742647)(868.99425281,354.72264702)(868.89425781,354.70265076)
\lineto(868.59425781,354.64265076)
\curveto(868.56425324,354.6426471)(868.52925327,354.6376471)(868.48925781,354.62765076)
\lineto(868.38425781,354.62765076)
\curveto(868.23425357,354.58764715)(868.06925373,354.56264718)(867.88925781,354.55265076)
\curveto(867.71925408,354.55264719)(867.55925424,354.53264721)(867.40925781,354.49265076)
\curveto(867.32925447,354.47264727)(867.25425455,354.45264729)(867.18425781,354.43265076)
\curveto(867.12425468,354.42264732)(867.05425475,354.40764733)(866.97425781,354.38765076)
\curveto(866.81425499,354.3376474)(866.66425514,354.27264747)(866.52425781,354.19265076)
\curveto(866.38425542,354.12264762)(866.26425554,354.03264771)(866.16425781,353.92265076)
\curveto(866.06425574,353.81264793)(865.98925581,353.67764806)(865.93925781,353.51765076)
\curveto(865.88925591,353.36764837)(865.86925593,353.18264856)(865.87925781,352.96265076)
\curveto(865.87925592,352.86264888)(865.89425591,352.76764897)(865.92425781,352.67765076)
\curveto(865.96425584,352.59764914)(866.00925579,352.52264922)(866.05925781,352.45265076)
\curveto(866.13925566,352.3426494)(866.24425556,352.24764949)(866.37425781,352.16765076)
\curveto(866.5042553,352.09764964)(866.64425516,352.0376497)(866.79425781,351.98765076)
\curveto(866.84425496,351.97764976)(866.89425491,351.97264977)(866.94425781,351.97265076)
\curveto(866.99425481,351.97264977)(867.04425476,351.96764977)(867.09425781,351.95765076)
\curveto(867.16425464,351.9376498)(867.24925455,351.92264982)(867.34925781,351.91265076)
\curveto(867.45925434,351.91264983)(867.54925425,351.92264982)(867.61925781,351.94265076)
\curveto(867.67925412,351.96264978)(867.73925406,351.96764977)(867.79925781,351.95765076)
\curveto(867.85925394,351.95764978)(867.91925388,351.96764977)(867.97925781,351.98765076)
\curveto(868.05925374,352.00764973)(868.13425367,352.02264972)(868.20425781,352.03265076)
\curveto(868.28425352,352.0426497)(868.35925344,352.06264968)(868.42925781,352.09265076)
\curveto(868.71925308,352.21264953)(868.96425284,352.35764938)(869.16425781,352.52765076)
\curveto(869.37425243,352.69764904)(869.53425227,352.92764881)(869.64425781,353.21765076)
}
}
{
\newrgbcolor{curcolor}{0 0 0}
\pscustom[linestyle=none,fillstyle=solid,fillcolor=curcolor]
{
\newpath
\moveto(879.92089844,351.80765076)
\lineto(879.92089844,351.41765076)
\curveto(879.92089056,351.29765044)(879.89589059,351.19765054)(879.84589844,351.11765076)
\curveto(879.79589069,351.04765069)(879.71089077,351.00765073)(879.59089844,350.99765076)
\lineto(879.24589844,350.99765076)
\curveto(879.1858913,350.99765074)(879.12589136,350.99265075)(879.06589844,350.98265076)
\curveto(879.01589147,350.98265076)(878.97089151,350.99265075)(878.93089844,351.01265076)
\curveto(878.84089164,351.03265071)(878.7808917,351.07265067)(878.75089844,351.13265076)
\curveto(878.71089177,351.18265056)(878.6858918,351.2426505)(878.67589844,351.31265076)
\curveto(878.67589181,351.38265036)(878.66089182,351.45265029)(878.63089844,351.52265076)
\curveto(878.62089186,351.5426502)(878.60589188,351.55765018)(878.58589844,351.56765076)
\curveto(878.57589191,351.58765015)(878.56089192,351.60765013)(878.54089844,351.62765076)
\curveto(878.44089204,351.6376501)(878.36089212,351.61765012)(878.30089844,351.56765076)
\curveto(878.25089223,351.51765022)(878.19589229,351.46765027)(878.13589844,351.41765076)
\curveto(877.93589255,351.26765047)(877.73589275,351.15265059)(877.53589844,351.07265076)
\curveto(877.35589313,350.99265075)(877.14589334,350.93265081)(876.90589844,350.89265076)
\curveto(876.67589381,350.85265089)(876.43589405,350.83265091)(876.18589844,350.83265076)
\curveto(875.94589454,350.82265092)(875.70589478,350.8376509)(875.46589844,350.87765076)
\curveto(875.22589526,350.90765083)(875.01589547,350.96265078)(874.83589844,351.04265076)
\curveto(874.31589617,351.26265048)(873.89589659,351.55765018)(873.57589844,351.92765076)
\curveto(873.25589723,352.30764943)(873.00589748,352.77764896)(872.82589844,353.33765076)
\curveto(872.7858977,353.42764831)(872.75589773,353.51764822)(872.73589844,353.60765076)
\curveto(872.72589776,353.70764803)(872.70589778,353.80764793)(872.67589844,353.90765076)
\curveto(872.66589782,353.95764778)(872.66089782,354.00764773)(872.66089844,354.05765076)
\curveto(872.66089782,354.10764763)(872.65589783,354.15764758)(872.64589844,354.20765076)
\curveto(872.62589786,354.25764748)(872.61589787,354.30764743)(872.61589844,354.35765076)
\curveto(872.62589786,354.41764732)(872.62589786,354.47264727)(872.61589844,354.52265076)
\lineto(872.61589844,354.67265076)
\curveto(872.59589789,354.72264702)(872.5858979,354.78764695)(872.58589844,354.86765076)
\curveto(872.5858979,354.94764679)(872.59589789,355.01264673)(872.61589844,355.06265076)
\lineto(872.61589844,355.22765076)
\curveto(872.63589785,355.29764644)(872.64089784,355.36764637)(872.63089844,355.43765076)
\curveto(872.63089785,355.51764622)(872.64089784,355.59264615)(872.66089844,355.66265076)
\curveto(872.67089781,355.71264603)(872.67589781,355.75764598)(872.67589844,355.79765076)
\curveto(872.67589781,355.8376459)(872.6808978,355.88264586)(872.69089844,355.93265076)
\curveto(872.72089776,356.03264571)(872.74589774,356.12764561)(872.76589844,356.21765076)
\curveto(872.7858977,356.31764542)(872.81089767,356.41264533)(872.84089844,356.50265076)
\curveto(872.97089751,356.88264486)(873.13589735,357.22264452)(873.33589844,357.52265076)
\curveto(873.54589694,357.83264391)(873.79589669,358.08764365)(874.08589844,358.28765076)
\curveto(874.25589623,358.40764333)(874.43089605,358.50764323)(874.61089844,358.58765076)
\curveto(874.80089568,358.66764307)(875.00589548,358.737643)(875.22589844,358.79765076)
\curveto(875.29589519,358.80764293)(875.36089512,358.81764292)(875.42089844,358.82765076)
\curveto(875.49089499,358.8376429)(875.56089492,358.85264289)(875.63089844,358.87265076)
\lineto(875.78089844,358.87265076)
\curveto(875.86089462,358.89264285)(875.97589451,358.90264284)(876.12589844,358.90265076)
\curveto(876.2858942,358.90264284)(876.40589408,358.89264285)(876.48589844,358.87265076)
\curveto(876.52589396,358.86264288)(876.5808939,358.85764288)(876.65089844,358.85765076)
\curveto(876.76089372,358.82764291)(876.87089361,358.80264294)(876.98089844,358.78265076)
\curveto(877.09089339,358.77264297)(877.19589329,358.742643)(877.29589844,358.69265076)
\curveto(877.44589304,358.63264311)(877.5858929,358.56764317)(877.71589844,358.49765076)
\curveto(877.85589263,358.42764331)(877.9858925,358.34764339)(878.10589844,358.25765076)
\curveto(878.16589232,358.20764353)(878.22589226,358.15264359)(878.28589844,358.09265076)
\curveto(878.35589213,358.0426437)(878.44589204,358.02764371)(878.55589844,358.04765076)
\curveto(878.57589191,358.07764366)(878.59089189,358.10264364)(878.60089844,358.12265076)
\curveto(878.62089186,358.1426436)(878.63589185,358.17264357)(878.64589844,358.21265076)
\curveto(878.67589181,358.30264344)(878.6858918,358.41764332)(878.67589844,358.55765076)
\lineto(878.67589844,358.93265076)
\lineto(878.67589844,360.65765076)
\lineto(878.67589844,361.12265076)
\curveto(878.67589181,361.30264044)(878.70089178,361.43264031)(878.75089844,361.51265076)
\curveto(878.79089169,361.58264016)(878.85089163,361.62764011)(878.93089844,361.64765076)
\curveto(878.95089153,361.64764009)(878.97589151,361.64764009)(879.00589844,361.64765076)
\curveto(879.03589145,361.65764008)(879.06089142,361.66264008)(879.08089844,361.66265076)
\curveto(879.22089126,361.67264007)(879.36589112,361.67264007)(879.51589844,361.66265076)
\curveto(879.67589081,361.66264008)(879.7858907,361.62264012)(879.84589844,361.54265076)
\curveto(879.89589059,361.46264028)(879.92089056,361.36264038)(879.92089844,361.24265076)
\lineto(879.92089844,360.86765076)
\lineto(879.92089844,351.80765076)
\moveto(878.70589844,354.64265076)
\curveto(878.72589176,354.69264705)(878.73589175,354.75764698)(878.73589844,354.83765076)
\curveto(878.73589175,354.92764681)(878.72589176,354.99764674)(878.70589844,355.04765076)
\lineto(878.70589844,355.27265076)
\curveto(878.6858918,355.36264638)(878.67089181,355.45264629)(878.66089844,355.54265076)
\curveto(878.65089183,355.6426461)(878.63089185,355.73264601)(878.60089844,355.81265076)
\curveto(878.5808919,355.89264585)(878.56089192,355.96764577)(878.54089844,356.03765076)
\curveto(878.53089195,356.10764563)(878.51089197,356.17764556)(878.48089844,356.24765076)
\curveto(878.36089212,356.54764519)(878.20589228,356.81264493)(878.01589844,357.04265076)
\curveto(877.82589266,357.27264447)(877.5858929,357.45264429)(877.29589844,357.58265076)
\curveto(877.19589329,357.63264411)(877.09089339,357.66764407)(876.98089844,357.68765076)
\curveto(876.8808936,357.71764402)(876.77089371,357.742644)(876.65089844,357.76265076)
\curveto(876.57089391,357.78264396)(876.480894,357.79264395)(876.38089844,357.79265076)
\lineto(876.11089844,357.79265076)
\curveto(876.06089442,357.78264396)(876.01589447,357.77264397)(875.97589844,357.76265076)
\lineto(875.84089844,357.76265076)
\curveto(875.76089472,357.742644)(875.67589481,357.72264402)(875.58589844,357.70265076)
\curveto(875.50589498,357.68264406)(875.42589506,357.65764408)(875.34589844,357.62765076)
\curveto(875.02589546,357.48764425)(874.76589572,357.28264446)(874.56589844,357.01265076)
\curveto(874.37589611,356.75264499)(874.22089626,356.44764529)(874.10089844,356.09765076)
\curveto(874.06089642,355.98764575)(874.03089645,355.87264587)(874.01089844,355.75265076)
\curveto(874.00089648,355.6426461)(873.9858965,355.53264621)(873.96589844,355.42265076)
\curveto(873.96589652,355.38264636)(873.96089652,355.3426464)(873.95089844,355.30265076)
\lineto(873.95089844,355.19765076)
\curveto(873.93089655,355.14764659)(873.92089656,355.09264665)(873.92089844,355.03265076)
\curveto(873.93089655,354.97264677)(873.93589655,354.91764682)(873.93589844,354.86765076)
\lineto(873.93589844,354.53765076)
\curveto(873.93589655,354.4376473)(873.94589654,354.3426474)(873.96589844,354.25265076)
\curveto(873.97589651,354.22264752)(873.9808965,354.17264757)(873.98089844,354.10265076)
\curveto(874.00089648,354.03264771)(874.01589647,353.96264778)(874.02589844,353.89265076)
\lineto(874.08589844,353.68265076)
\curveto(874.19589629,353.33264841)(874.34589614,353.03264871)(874.53589844,352.78265076)
\curveto(874.72589576,352.53264921)(874.96589552,352.32764941)(875.25589844,352.16765076)
\curveto(875.34589514,352.11764962)(875.43589505,352.07764966)(875.52589844,352.04765076)
\curveto(875.61589487,352.01764972)(875.71589477,351.98764975)(875.82589844,351.95765076)
\curveto(875.87589461,351.9376498)(875.92589456,351.93264981)(875.97589844,351.94265076)
\curveto(876.03589445,351.95264979)(876.09089439,351.94764979)(876.14089844,351.92765076)
\curveto(876.1808943,351.91764982)(876.22089426,351.91264983)(876.26089844,351.91265076)
\lineto(876.39589844,351.91265076)
\lineto(876.53089844,351.91265076)
\curveto(876.56089392,351.92264982)(876.61089387,351.92764981)(876.68089844,351.92765076)
\curveto(876.76089372,351.94764979)(876.84089364,351.96264978)(876.92089844,351.97265076)
\curveto(877.00089348,351.99264975)(877.07589341,352.01764972)(877.14589844,352.04765076)
\curveto(877.47589301,352.18764955)(877.74089274,352.36264938)(877.94089844,352.57265076)
\curveto(878.15089233,352.79264895)(878.32589216,353.06764867)(878.46589844,353.39765076)
\curveto(878.51589197,353.50764823)(878.55089193,353.61764812)(878.57089844,353.72765076)
\curveto(878.59089189,353.8376479)(878.61589187,353.94764779)(878.64589844,354.05765076)
\curveto(878.66589182,354.09764764)(878.67589181,354.13264761)(878.67589844,354.16265076)
\curveto(878.67589181,354.20264754)(878.6808918,354.2426475)(878.69089844,354.28265076)
\curveto(878.70089178,354.3426474)(878.70089178,354.40264734)(878.69089844,354.46265076)
\curveto(878.69089179,354.52264722)(878.69589179,354.58264716)(878.70589844,354.64265076)
}
}
{
\newrgbcolor{curcolor}{0 0 0}
\pscustom[linestyle=none,fillstyle=solid,fillcolor=curcolor]
{
\newpath
\moveto(888.99214844,355.19765076)
\curveto(889.01214038,355.1376466)(889.02214037,355.0426467)(889.02214844,354.91265076)
\curveto(889.02214037,354.79264695)(889.01714037,354.70764703)(889.00714844,354.65765076)
\lineto(889.00714844,354.50765076)
\curveto(888.99714039,354.42764731)(888.9871404,354.35264739)(888.97714844,354.28265076)
\curveto(888.97714041,354.22264752)(888.97214042,354.15264759)(888.96214844,354.07265076)
\curveto(888.94214045,354.01264773)(888.92714046,353.95264779)(888.91714844,353.89265076)
\curveto(888.91714047,353.83264791)(888.90714048,353.77264797)(888.88714844,353.71265076)
\curveto(888.84714054,353.58264816)(888.81214058,353.45264829)(888.78214844,353.32265076)
\curveto(888.75214064,353.19264855)(888.71214068,353.07264867)(888.66214844,352.96265076)
\curveto(888.45214094,352.48264926)(888.17214122,352.07764966)(887.82214844,351.74765076)
\curveto(887.47214192,351.42765031)(887.04214235,351.18265056)(886.53214844,351.01265076)
\curveto(886.42214297,350.97265077)(886.30214309,350.9426508)(886.17214844,350.92265076)
\curveto(886.05214334,350.90265084)(885.92714346,350.88265086)(885.79714844,350.86265076)
\curveto(885.73714365,350.85265089)(885.67214372,350.84765089)(885.60214844,350.84765076)
\curveto(885.54214385,350.8376509)(885.48214391,350.83265091)(885.42214844,350.83265076)
\curveto(885.38214401,350.82265092)(885.32214407,350.81765092)(885.24214844,350.81765076)
\curveto(885.17214422,350.81765092)(885.12214427,350.82265092)(885.09214844,350.83265076)
\curveto(885.05214434,350.8426509)(885.01214438,350.84765089)(884.97214844,350.84765076)
\curveto(884.93214446,350.8376509)(884.89714449,350.8376509)(884.86714844,350.84765076)
\lineto(884.77714844,350.84765076)
\lineto(884.41714844,350.89265076)
\curveto(884.27714511,350.93265081)(884.14214525,350.97265077)(884.01214844,351.01265076)
\curveto(883.88214551,351.05265069)(883.75714563,351.09765064)(883.63714844,351.14765076)
\curveto(883.1871462,351.34765039)(882.81714657,351.60765013)(882.52714844,351.92765076)
\curveto(882.23714715,352.24764949)(881.99714739,352.6376491)(881.80714844,353.09765076)
\curveto(881.75714763,353.19764854)(881.71714767,353.29764844)(881.68714844,353.39765076)
\curveto(881.66714772,353.49764824)(881.64714774,353.60264814)(881.62714844,353.71265076)
\curveto(881.60714778,353.75264799)(881.59714779,353.78264796)(881.59714844,353.80265076)
\curveto(881.60714778,353.83264791)(881.60714778,353.86764787)(881.59714844,353.90765076)
\curveto(881.57714781,353.98764775)(881.56214783,354.06764767)(881.55214844,354.14765076)
\curveto(881.55214784,354.2376475)(881.54214785,354.32264742)(881.52214844,354.40265076)
\lineto(881.52214844,354.52265076)
\curveto(881.52214787,354.56264718)(881.51714787,354.60764713)(881.50714844,354.65765076)
\curveto(881.49714789,354.70764703)(881.4921479,354.79264695)(881.49214844,354.91265076)
\curveto(881.4921479,355.0426467)(881.50214789,355.1376466)(881.52214844,355.19765076)
\curveto(881.54214785,355.26764647)(881.54714784,355.3376464)(881.53714844,355.40765076)
\curveto(881.52714786,355.47764626)(881.53214786,355.54764619)(881.55214844,355.61765076)
\curveto(881.56214783,355.66764607)(881.56714782,355.70764603)(881.56714844,355.73765076)
\curveto(881.57714781,355.77764596)(881.5871478,355.82264592)(881.59714844,355.87265076)
\curveto(881.62714776,355.99264575)(881.65214774,356.11264563)(881.67214844,356.23265076)
\curveto(881.70214769,356.35264539)(881.74214765,356.46764527)(881.79214844,356.57765076)
\curveto(881.94214745,356.94764479)(882.12214727,357.27764446)(882.33214844,357.56765076)
\curveto(882.55214684,357.86764387)(882.81714657,358.11764362)(883.12714844,358.31765076)
\curveto(883.24714614,358.39764334)(883.37214602,358.46264328)(883.50214844,358.51265076)
\curveto(883.63214576,358.57264317)(883.76714562,358.63264311)(883.90714844,358.69265076)
\curveto(884.02714536,358.742643)(884.15714523,358.77264297)(884.29714844,358.78265076)
\curveto(884.43714495,358.80264294)(884.57714481,358.83264291)(884.71714844,358.87265076)
\lineto(884.91214844,358.87265076)
\curveto(884.98214441,358.88264286)(885.04714434,358.89264285)(885.10714844,358.90265076)
\curveto(885.99714339,358.91264283)(886.73714265,358.72764301)(887.32714844,358.34765076)
\curveto(887.91714147,357.96764377)(888.34214105,357.47264427)(888.60214844,356.86265076)
\curveto(888.65214074,356.76264498)(888.6921407,356.66264508)(888.72214844,356.56265076)
\curveto(888.75214064,356.46264528)(888.7871406,356.35764538)(888.82714844,356.24765076)
\curveto(888.85714053,356.1376456)(888.88214051,356.01764572)(888.90214844,355.88765076)
\curveto(888.92214047,355.76764597)(888.94714044,355.6426461)(888.97714844,355.51265076)
\curveto(888.9871404,355.46264628)(888.9871404,355.40764633)(888.97714844,355.34765076)
\curveto(888.97714041,355.29764644)(888.98214041,355.24764649)(888.99214844,355.19765076)
\moveto(887.65714844,354.34265076)
\curveto(887.67714171,354.41264733)(887.68214171,354.49264725)(887.67214844,354.58265076)
\lineto(887.67214844,354.83765076)
\curveto(887.67214172,355.22764651)(887.63714175,355.55764618)(887.56714844,355.82765076)
\curveto(887.53714185,355.90764583)(887.51214188,355.98764575)(887.49214844,356.06765076)
\curveto(887.47214192,356.14764559)(887.44714194,356.22264552)(887.41714844,356.29265076)
\curveto(887.13714225,356.9426448)(886.6921427,357.39264435)(886.08214844,357.64265076)
\curveto(886.01214338,357.67264407)(885.93714345,357.69264405)(885.85714844,357.70265076)
\lineto(885.61714844,357.76265076)
\curveto(885.53714385,357.78264396)(885.45214394,357.79264395)(885.36214844,357.79265076)
\lineto(885.09214844,357.79265076)
\lineto(884.82214844,357.74765076)
\curveto(884.72214467,357.72764401)(884.62714476,357.70264404)(884.53714844,357.67265076)
\curveto(884.45714493,357.65264409)(884.37714501,357.62264412)(884.29714844,357.58265076)
\curveto(884.22714516,357.56264418)(884.16214523,357.53264421)(884.10214844,357.49265076)
\curveto(884.04214535,357.45264429)(883.9871454,357.41264433)(883.93714844,357.37265076)
\curveto(883.69714569,357.20264454)(883.50214589,356.99764474)(883.35214844,356.75765076)
\curveto(883.20214619,356.51764522)(883.07214632,356.2376455)(882.96214844,355.91765076)
\curveto(882.93214646,355.81764592)(882.91214648,355.71264603)(882.90214844,355.60265076)
\curveto(882.8921465,355.50264624)(882.87714651,355.39764634)(882.85714844,355.28765076)
\curveto(882.84714654,355.24764649)(882.84214655,355.18264656)(882.84214844,355.09265076)
\curveto(882.83214656,355.06264668)(882.82714656,355.02764671)(882.82714844,354.98765076)
\curveto(882.83714655,354.94764679)(882.84214655,354.90264684)(882.84214844,354.85265076)
\lineto(882.84214844,354.55265076)
\curveto(882.84214655,354.45264729)(882.85214654,354.36264738)(882.87214844,354.28265076)
\lineto(882.90214844,354.10265076)
\curveto(882.92214647,354.00264774)(882.93714645,353.90264784)(882.94714844,353.80265076)
\curveto(882.96714642,353.71264803)(882.99714639,353.62764811)(883.03714844,353.54765076)
\curveto(883.13714625,353.30764843)(883.25214614,353.08264866)(883.38214844,352.87265076)
\curveto(883.52214587,352.66264908)(883.6921457,352.48764925)(883.89214844,352.34765076)
\curveto(883.94214545,352.31764942)(883.9871454,352.29264945)(884.02714844,352.27265076)
\curveto(884.06714532,352.25264949)(884.11214528,352.22764951)(884.16214844,352.19765076)
\curveto(884.24214515,352.14764959)(884.32714506,352.10264964)(884.41714844,352.06265076)
\curveto(884.51714487,352.03264971)(884.62214477,352.00264974)(884.73214844,351.97265076)
\curveto(884.78214461,351.95264979)(884.82714456,351.9426498)(884.86714844,351.94265076)
\curveto(884.91714447,351.95264979)(884.96714442,351.95264979)(885.01714844,351.94265076)
\curveto(885.04714434,351.93264981)(885.10714428,351.92264982)(885.19714844,351.91265076)
\curveto(885.29714409,351.90264984)(885.37214402,351.90764983)(885.42214844,351.92765076)
\curveto(885.46214393,351.9376498)(885.50214389,351.9376498)(885.54214844,351.92765076)
\curveto(885.58214381,351.92764981)(885.62214377,351.9376498)(885.66214844,351.95765076)
\curveto(885.74214365,351.97764976)(885.82214357,351.99264975)(885.90214844,352.00265076)
\curveto(885.98214341,352.02264972)(886.05714333,352.04764969)(886.12714844,352.07765076)
\curveto(886.46714292,352.21764952)(886.74214265,352.41264933)(886.95214844,352.66265076)
\curveto(887.16214223,352.91264883)(887.33714205,353.20764853)(887.47714844,353.54765076)
\curveto(887.52714186,353.66764807)(887.55714183,353.79264795)(887.56714844,353.92265076)
\curveto(887.5871418,354.06264768)(887.61714177,354.20264754)(887.65714844,354.34265076)
}
}
{
\newrgbcolor{curcolor}{0 0 0}
\pscustom[linestyle=none,fillstyle=solid,fillcolor=curcolor]
{
\newpath
\moveto(892.91042969,358.90265076)
\curveto(893.63042562,358.91264283)(894.23542502,358.82764291)(894.72542969,358.64765076)
\curveto(895.21542404,358.47764326)(895.59542366,358.17264357)(895.86542969,357.73265076)
\curveto(895.93542332,357.62264412)(895.99042326,357.50764423)(896.03042969,357.38765076)
\curveto(896.07042318,357.27764446)(896.11042314,357.15264459)(896.15042969,357.01265076)
\curveto(896.17042308,356.9426448)(896.17542308,356.86764487)(896.16542969,356.78765076)
\curveto(896.1554231,356.71764502)(896.14042311,356.66264508)(896.12042969,356.62265076)
\curveto(896.10042315,356.60264514)(896.07542318,356.58264516)(896.04542969,356.56265076)
\curveto(896.01542324,356.55264519)(895.99042326,356.5376452)(895.97042969,356.51765076)
\curveto(895.92042333,356.49764524)(895.87042338,356.49264525)(895.82042969,356.50265076)
\curveto(895.77042348,356.51264523)(895.72042353,356.51264523)(895.67042969,356.50265076)
\curveto(895.59042366,356.48264526)(895.48542377,356.47764526)(895.35542969,356.48765076)
\curveto(895.22542403,356.50764523)(895.13542412,356.53264521)(895.08542969,356.56265076)
\curveto(895.00542425,356.61264513)(894.9504243,356.67764506)(894.92042969,356.75765076)
\curveto(894.90042435,356.84764489)(894.86542439,356.93264481)(894.81542969,357.01265076)
\curveto(894.72542453,357.17264457)(894.60042465,357.31764442)(894.44042969,357.44765076)
\curveto(894.33042492,357.52764421)(894.21042504,357.58764415)(894.08042969,357.62765076)
\curveto(893.9504253,357.66764407)(893.81042544,357.70764403)(893.66042969,357.74765076)
\curveto(893.61042564,357.76764397)(893.56042569,357.77264397)(893.51042969,357.76265076)
\curveto(893.46042579,357.76264398)(893.41042584,357.76764397)(893.36042969,357.77765076)
\curveto(893.30042595,357.79764394)(893.22542603,357.80764393)(893.13542969,357.80765076)
\curveto(893.04542621,357.80764393)(892.97042628,357.79764394)(892.91042969,357.77765076)
\lineto(892.82042969,357.77765076)
\lineto(892.67042969,357.74765076)
\curveto(892.62042663,357.74764399)(892.57042668,357.742644)(892.52042969,357.73265076)
\curveto(892.26042699,357.67264407)(892.04542721,357.58764415)(891.87542969,357.47765076)
\curveto(891.70542755,357.36764437)(891.59042766,357.18264456)(891.53042969,356.92265076)
\curveto(891.51042774,356.85264489)(891.50542775,356.78264496)(891.51542969,356.71265076)
\curveto(891.53542772,356.6426451)(891.5554277,356.58264516)(891.57542969,356.53265076)
\curveto(891.63542762,356.38264536)(891.70542755,356.27264547)(891.78542969,356.20265076)
\curveto(891.87542738,356.1426456)(891.98542727,356.07264567)(892.11542969,355.99265076)
\curveto(892.27542698,355.89264585)(892.4554268,355.81764592)(892.65542969,355.76765076)
\curveto(892.8554264,355.72764601)(893.0554262,355.67764606)(893.25542969,355.61765076)
\curveto(893.38542587,355.57764616)(893.51542574,355.54764619)(893.64542969,355.52765076)
\curveto(893.77542548,355.50764623)(893.90542535,355.47764626)(894.03542969,355.43765076)
\curveto(894.24542501,355.37764636)(894.4504248,355.31764642)(894.65042969,355.25765076)
\curveto(894.8504244,355.20764653)(895.0504242,355.1426466)(895.25042969,355.06265076)
\lineto(895.40042969,355.00265076)
\curveto(895.4504238,354.98264676)(895.50042375,354.95764678)(895.55042969,354.92765076)
\curveto(895.7504235,354.80764693)(895.92542333,354.67264707)(896.07542969,354.52265076)
\curveto(896.22542303,354.37264737)(896.3504229,354.18264756)(896.45042969,353.95265076)
\curveto(896.47042278,353.88264786)(896.49042276,353.78764795)(896.51042969,353.66765076)
\curveto(896.53042272,353.59764814)(896.54042271,353.52264822)(896.54042969,353.44265076)
\curveto(896.5504227,353.37264837)(896.5554227,353.29264845)(896.55542969,353.20265076)
\lineto(896.55542969,353.05265076)
\curveto(896.53542272,352.98264876)(896.52542273,352.91264883)(896.52542969,352.84265076)
\curveto(896.52542273,352.77264897)(896.51542274,352.70264904)(896.49542969,352.63265076)
\curveto(896.46542279,352.52264922)(896.43042282,352.41764932)(896.39042969,352.31765076)
\curveto(896.3504229,352.21764952)(896.30542295,352.12764961)(896.25542969,352.04765076)
\curveto(896.09542316,351.78764995)(895.89042336,351.57765016)(895.64042969,351.41765076)
\curveto(895.39042386,351.26765047)(895.11042414,351.1376506)(894.80042969,351.02765076)
\curveto(894.71042454,350.99765074)(894.61542464,350.97765076)(894.51542969,350.96765076)
\curveto(894.42542483,350.94765079)(894.33542492,350.92265082)(894.24542969,350.89265076)
\curveto(894.14542511,350.87265087)(894.04542521,350.86265088)(893.94542969,350.86265076)
\curveto(893.84542541,350.86265088)(893.74542551,350.85265089)(893.64542969,350.83265076)
\lineto(893.49542969,350.83265076)
\curveto(893.44542581,350.82265092)(893.37542588,350.81765092)(893.28542969,350.81765076)
\curveto(893.19542606,350.81765092)(893.12542613,350.82265092)(893.07542969,350.83265076)
\lineto(892.91042969,350.83265076)
\curveto(892.8504264,350.85265089)(892.78542647,350.86265088)(892.71542969,350.86265076)
\curveto(892.64542661,350.85265089)(892.58542667,350.85765088)(892.53542969,350.87765076)
\curveto(892.48542677,350.88765085)(892.42042683,350.89265085)(892.34042969,350.89265076)
\lineto(892.10042969,350.95265076)
\curveto(892.03042722,350.96265078)(891.9554273,350.98265076)(891.87542969,351.01265076)
\curveto(891.56542769,351.11265063)(891.29542796,351.2376505)(891.06542969,351.38765076)
\curveto(890.83542842,351.5376502)(890.63542862,351.73265001)(890.46542969,351.97265076)
\curveto(890.37542888,352.10264964)(890.30042895,352.2376495)(890.24042969,352.37765076)
\curveto(890.18042907,352.51764922)(890.12542913,352.67264907)(890.07542969,352.84265076)
\curveto(890.0554292,352.90264884)(890.04542921,352.97264877)(890.04542969,353.05265076)
\curveto(890.0554292,353.1426486)(890.07042918,353.21264853)(890.09042969,353.26265076)
\curveto(890.12042913,353.30264844)(890.17042908,353.3426484)(890.24042969,353.38265076)
\curveto(890.29042896,353.40264834)(890.36042889,353.41264833)(890.45042969,353.41265076)
\curveto(890.54042871,353.42264832)(890.63042862,353.42264832)(890.72042969,353.41265076)
\curveto(890.81042844,353.40264834)(890.89542836,353.38764835)(890.97542969,353.36765076)
\curveto(891.06542819,353.35764838)(891.12542813,353.3426484)(891.15542969,353.32265076)
\curveto(891.22542803,353.27264847)(891.27042798,353.19764854)(891.29042969,353.09765076)
\curveto(891.32042793,353.00764873)(891.3554279,352.92264882)(891.39542969,352.84265076)
\curveto(891.49542776,352.62264912)(891.63042762,352.45264929)(891.80042969,352.33265076)
\curveto(891.92042733,352.2426495)(892.0554272,352.17264957)(892.20542969,352.12265076)
\curveto(892.3554269,352.07264967)(892.51542674,352.02264972)(892.68542969,351.97265076)
\lineto(893.00042969,351.92765076)
\lineto(893.09042969,351.92765076)
\curveto(893.16042609,351.90764983)(893.250426,351.89764984)(893.36042969,351.89765076)
\curveto(893.48042577,351.89764984)(893.58042567,351.90764983)(893.66042969,351.92765076)
\curveto(893.73042552,351.92764981)(893.78542547,351.93264981)(893.82542969,351.94265076)
\curveto(893.88542537,351.95264979)(893.94542531,351.95764978)(894.00542969,351.95765076)
\curveto(894.06542519,351.96764977)(894.12042513,351.97764976)(894.17042969,351.98765076)
\curveto(894.46042479,352.06764967)(894.69042456,352.17264957)(894.86042969,352.30265076)
\curveto(895.03042422,352.43264931)(895.1504241,352.65264909)(895.22042969,352.96265076)
\curveto(895.24042401,353.01264873)(895.24542401,353.06764867)(895.23542969,353.12765076)
\curveto(895.22542403,353.18764855)(895.21542404,353.23264851)(895.20542969,353.26265076)
\curveto(895.1554241,353.45264829)(895.08542417,353.59264815)(894.99542969,353.68265076)
\curveto(894.90542435,353.78264796)(894.79042446,353.87264787)(894.65042969,353.95265076)
\curveto(894.56042469,354.01264773)(894.46042479,354.06264768)(894.35042969,354.10265076)
\lineto(894.02042969,354.22265076)
\curveto(893.99042526,354.23264751)(893.96042529,354.2376475)(893.93042969,354.23765076)
\curveto(893.91042534,354.2376475)(893.88542537,354.24764749)(893.85542969,354.26765076)
\curveto(893.51542574,354.37764736)(893.16042609,354.45764728)(892.79042969,354.50765076)
\curveto(892.43042682,354.56764717)(892.09042716,354.66264708)(891.77042969,354.79265076)
\curveto(891.67042758,354.83264691)(891.57542768,354.86764687)(891.48542969,354.89765076)
\curveto(891.39542786,354.92764681)(891.31042794,354.96764677)(891.23042969,355.01765076)
\curveto(891.04042821,355.12764661)(890.86542839,355.25264649)(890.70542969,355.39265076)
\curveto(890.54542871,355.53264621)(890.42042883,355.70764603)(890.33042969,355.91765076)
\curveto(890.30042895,355.98764575)(890.27542898,356.05764568)(890.25542969,356.12765076)
\curveto(890.24542901,356.19764554)(890.23042902,356.27264547)(890.21042969,356.35265076)
\curveto(890.18042907,356.47264527)(890.17042908,356.60764513)(890.18042969,356.75765076)
\curveto(890.19042906,356.91764482)(890.20542905,357.05264469)(890.22542969,357.16265076)
\curveto(890.24542901,357.21264453)(890.255429,357.25264449)(890.25542969,357.28265076)
\curveto(890.26542899,357.32264442)(890.28042897,357.36264438)(890.30042969,357.40265076)
\curveto(890.39042886,357.63264411)(890.51042874,357.83264391)(890.66042969,358.00265076)
\curveto(890.82042843,358.17264357)(891.00042825,358.32264342)(891.20042969,358.45265076)
\curveto(891.3504279,358.5426432)(891.51542774,358.61264313)(891.69542969,358.66265076)
\curveto(891.87542738,358.72264302)(892.06542719,358.77764296)(892.26542969,358.82765076)
\curveto(892.33542692,358.8376429)(892.40042685,358.84764289)(892.46042969,358.85765076)
\curveto(892.53042672,358.86764287)(892.60542665,358.87764286)(892.68542969,358.88765076)
\curveto(892.71542654,358.89764284)(892.7554265,358.89764284)(892.80542969,358.88765076)
\curveto(892.8554264,358.87764286)(892.89042636,358.88264286)(892.91042969,358.90265076)
}
}
{
\newrgbcolor{curcolor}{0 0 0}
\pscustom[linestyle=none,fillstyle=solid,fillcolor=curcolor]
{
\newpath
\moveto(836.41642334,338.34634644)
\curveto(837.39641684,338.36633548)(838.21641602,338.20633564)(838.87642334,337.86634644)
\curveto(839.54641469,337.53633631)(840.06641417,337.07633677)(840.43642334,336.48634644)
\curveto(840.5364137,336.32633752)(840.61641362,336.17133768)(840.67642334,336.02134644)
\curveto(840.74641349,335.88133797)(840.81141342,335.71133814)(840.87142334,335.51134644)
\curveto(840.89141334,335.46133839)(840.91141332,335.39133846)(840.93142334,335.30134644)
\curveto(840.95141328,335.22133863)(840.94641329,335.1463387)(840.91642334,335.07634644)
\curveto(840.89641334,335.01633883)(840.85641338,334.97633887)(840.79642334,334.95634644)
\curveto(840.74641349,334.9463389)(840.69141354,334.93133892)(840.63142334,334.91134644)
\lineto(840.48142334,334.91134644)
\curveto(840.45141378,334.90133895)(840.41141382,334.89633895)(840.36142334,334.89634644)
\lineto(840.24142334,334.89634644)
\curveto(840.10141413,334.89633895)(839.97141426,334.90133895)(839.85142334,334.91134644)
\curveto(839.74141449,334.93133892)(839.66141457,334.98133887)(839.61142334,335.06134644)
\curveto(839.54141469,335.16133869)(839.48641475,335.27633857)(839.44642334,335.40634644)
\curveto(839.40641483,335.53633831)(839.35141488,335.65633819)(839.28142334,335.76634644)
\curveto(839.15141508,335.98633786)(839.00141523,336.17633767)(838.83142334,336.33634644)
\curveto(838.67141556,336.49633735)(838.48141575,336.6463372)(838.26142334,336.78634644)
\curveto(838.14141609,336.86633698)(838.00641623,336.92633692)(837.85642334,336.96634644)
\curveto(837.71641652,337.00633684)(837.57141666,337.0463368)(837.42142334,337.08634644)
\curveto(837.31141692,337.11633673)(837.18641705,337.13633671)(837.04642334,337.14634644)
\curveto(836.90641733,337.16633668)(836.75641748,337.17633667)(836.59642334,337.17634644)
\curveto(836.44641779,337.17633667)(836.29641794,337.16633668)(836.14642334,337.14634644)
\curveto(836.00641823,337.13633671)(835.88641835,337.11633673)(835.78642334,337.08634644)
\curveto(835.68641855,337.06633678)(835.59141864,337.0463368)(835.50142334,337.02634644)
\curveto(835.41141882,337.00633684)(835.32141891,336.97633687)(835.23142334,336.93634644)
\curveto(834.39141984,336.58633726)(833.78642045,335.98633786)(833.41642334,335.13634644)
\curveto(833.34642089,334.99633885)(833.28642095,334.846339)(833.23642334,334.68634644)
\curveto(833.19642104,334.53633931)(833.15142108,334.38133947)(833.10142334,334.22134644)
\curveto(833.08142115,334.16133969)(833.07142116,334.09633975)(833.07142334,334.02634644)
\curveto(833.07142116,333.96633988)(833.06142117,333.90633994)(833.04142334,333.84634644)
\curveto(833.0314212,333.80634004)(833.02642121,333.77134008)(833.02642334,333.74134644)
\curveto(833.02642121,333.71134014)(833.02142121,333.67634017)(833.01142334,333.63634644)
\curveto(832.99142124,333.52634032)(832.97642126,333.41134044)(832.96642334,333.29134644)
\lineto(832.96642334,332.94634644)
\curveto(832.96642127,332.87634097)(832.96142127,332.80134105)(832.95142334,332.72134644)
\curveto(832.95142128,332.6513412)(832.95642128,332.58634126)(832.96642334,332.52634644)
\lineto(832.96642334,332.37634644)
\curveto(832.98642125,332.30634154)(832.99142124,332.23634161)(832.98142334,332.16634644)
\curveto(832.98142125,332.09634175)(832.99142124,332.02634182)(833.01142334,331.95634644)
\curveto(833.0314212,331.89634195)(833.0364212,331.83634201)(833.02642334,331.77634644)
\curveto(833.02642121,331.71634213)(833.0364212,331.66134219)(833.05642334,331.61134644)
\curveto(833.08642115,331.48134237)(833.11142112,331.3513425)(833.13142334,331.22134644)
\curveto(833.16142107,331.10134275)(833.19642104,330.98134287)(833.23642334,330.86134644)
\curveto(833.40642083,330.36134349)(833.62642061,329.93134392)(833.89642334,329.57134644)
\curveto(834.16642007,329.22134463)(834.52141971,328.93134492)(834.96142334,328.70134644)
\curveto(835.10141913,328.63134522)(835.24141899,328.57634527)(835.38142334,328.53634644)
\curveto(835.5314187,328.49634535)(835.69141854,328.4513454)(835.86142334,328.40134644)
\curveto(835.9314183,328.38134547)(835.99641824,328.37134548)(836.05642334,328.37134644)
\curveto(836.11641812,328.38134547)(836.18641805,328.37634547)(836.26642334,328.35634644)
\curveto(836.31641792,328.3463455)(836.40641783,328.33634551)(836.53642334,328.32634644)
\curveto(836.66641757,328.32634552)(836.76141747,328.33634551)(836.82142334,328.35634644)
\lineto(836.92642334,328.35634644)
\curveto(836.96641727,328.36634548)(837.00641723,328.36634548)(837.04642334,328.35634644)
\curveto(837.08641715,328.35634549)(837.12641711,328.36634548)(837.16642334,328.38634644)
\curveto(837.26641697,328.40634544)(837.36141687,328.42134543)(837.45142334,328.43134644)
\curveto(837.55141668,328.4513454)(837.64641659,328.48134537)(837.73642334,328.52134644)
\curveto(838.51641572,328.84134501)(839.06641517,329.36634448)(839.38642334,330.09634644)
\curveto(839.46641477,330.27634357)(839.54141469,330.49134336)(839.61142334,330.74134644)
\curveto(839.6314146,330.83134302)(839.64641459,330.92134293)(839.65642334,331.01134644)
\curveto(839.67641456,331.11134274)(839.71141452,331.20134265)(839.76142334,331.28134644)
\curveto(839.81141442,331.36134249)(839.89141434,331.40634244)(840.00142334,331.41634644)
\curveto(840.11141412,331.42634242)(840.231414,331.43134242)(840.36142334,331.43134644)
\lineto(840.51142334,331.43134644)
\curveto(840.56141367,331.43134242)(840.60641363,331.42634242)(840.64642334,331.41634644)
\lineto(840.75142334,331.41634644)
\lineto(840.84142334,331.38634644)
\curveto(840.88141335,331.38634246)(840.91141332,331.37634247)(840.93142334,331.35634644)
\curveto(841.00141323,331.31634253)(841.04141319,331.24134261)(841.05142334,331.13134644)
\curveto(841.06141317,331.03134282)(841.05141318,330.93134292)(841.02142334,330.83134644)
\curveto(840.96141327,330.60134325)(840.90641333,330.38134347)(840.85642334,330.17134644)
\curveto(840.80641343,329.96134389)(840.7314135,329.76134409)(840.63142334,329.57134644)
\curveto(840.55141368,329.44134441)(840.47641376,329.31634453)(840.40642334,329.19634644)
\curveto(840.34641389,329.07634477)(840.27641396,328.95634489)(840.19642334,328.83634644)
\curveto(840.01641422,328.57634527)(839.79141444,328.33634551)(839.52142334,328.11634644)
\curveto(839.26141497,327.90634594)(838.97641526,327.73134612)(838.66642334,327.59134644)
\curveto(838.55641568,327.54134631)(838.44641579,327.50134635)(838.33642334,327.47134644)
\curveto(838.236416,327.44134641)(838.1314161,327.40634644)(838.02142334,327.36634644)
\curveto(837.91141632,327.32634652)(837.79641644,327.30134655)(837.67642334,327.29134644)
\curveto(837.56641667,327.27134658)(837.45141678,327.2513466)(837.33142334,327.23134644)
\curveto(837.28141695,327.21134664)(837.236417,327.20634664)(837.19642334,327.21634644)
\curveto(837.15641708,327.21634663)(837.11641712,327.21134664)(837.07642334,327.20134644)
\curveto(837.01641722,327.19134666)(836.95641728,327.18634666)(836.89642334,327.18634644)
\curveto(836.8364174,327.18634666)(836.77141746,327.18134667)(836.70142334,327.17134644)
\curveto(836.67141756,327.16134669)(836.60141763,327.16134669)(836.49142334,327.17134644)
\curveto(836.39141784,327.17134668)(836.32641791,327.17634667)(836.29642334,327.18634644)
\curveto(836.24641799,327.19634665)(836.19641804,327.20134665)(836.14642334,327.20134644)
\curveto(836.10641813,327.19134666)(836.06141817,327.19134666)(836.01142334,327.20134644)
\lineto(835.86142334,327.20134644)
\curveto(835.78141845,327.22134663)(835.70641853,327.23634661)(835.63642334,327.24634644)
\curveto(835.56641867,327.2463466)(835.49141874,327.25634659)(835.41142334,327.27634644)
\lineto(835.14142334,327.33634644)
\curveto(835.05141918,327.3463465)(834.96641927,327.36634648)(834.88642334,327.39634644)
\curveto(834.67641956,327.45634639)(834.48641975,327.53134632)(834.31642334,327.62134644)
\curveto(833.68642055,327.89134596)(833.17642106,328.27634557)(832.78642334,328.77634644)
\curveto(832.39642184,329.27634457)(832.08642215,329.86634398)(831.85642334,330.54634644)
\curveto(831.81642242,330.66634318)(831.78142245,330.79134306)(831.75142334,330.92134644)
\curveto(831.7314225,331.0513428)(831.70642253,331.18634266)(831.67642334,331.32634644)
\curveto(831.65642258,331.37634247)(831.64642259,331.42134243)(831.64642334,331.46134644)
\curveto(831.65642258,331.50134235)(831.65642258,331.5463423)(831.64642334,331.59634644)
\curveto(831.62642261,331.68634216)(831.61142262,331.78134207)(831.60142334,331.88134644)
\curveto(831.60142263,331.98134187)(831.59142264,332.07634177)(831.57142334,332.16634644)
\lineto(831.57142334,332.45134644)
\curveto(831.55142268,332.50134135)(831.54142269,332.58634126)(831.54142334,332.70634644)
\curveto(831.54142269,332.82634102)(831.55142268,332.91134094)(831.57142334,332.96134644)
\curveto(831.58142265,332.99134086)(831.58142265,333.02134083)(831.57142334,333.05134644)
\curveto(831.56142267,333.09134076)(831.56142267,333.12134073)(831.57142334,333.14134644)
\lineto(831.57142334,333.27634644)
\curveto(831.58142265,333.35634049)(831.58642265,333.43634041)(831.58642334,333.51634644)
\curveto(831.59642264,333.60634024)(831.61142262,333.69134016)(831.63142334,333.77134644)
\curveto(831.65142258,333.83134002)(831.66142257,333.89133996)(831.66142334,333.95134644)
\curveto(831.66142257,334.02133983)(831.67142256,334.09133976)(831.69142334,334.16134644)
\curveto(831.74142249,334.33133952)(831.78142245,334.49633935)(831.81142334,334.65634644)
\curveto(831.84142239,334.81633903)(831.88642235,334.96633888)(831.94642334,335.10634644)
\lineto(832.09642334,335.49634644)
\curveto(832.15642208,335.63633821)(832.22142201,335.76133809)(832.29142334,335.87134644)
\curveto(832.44142179,336.13133772)(832.59142164,336.36633748)(832.74142334,336.57634644)
\curveto(832.77142146,336.62633722)(832.80642143,336.66633718)(832.84642334,336.69634644)
\curveto(832.89642134,336.73633711)(832.9364213,336.78133707)(832.96642334,336.83134644)
\curveto(833.02642121,336.91133694)(833.08642115,336.98133687)(833.14642334,337.04134644)
\lineto(833.35642334,337.22134644)
\curveto(833.41642082,337.27133658)(833.47142076,337.31633653)(833.52142334,337.35634644)
\curveto(833.58142065,337.40633644)(833.64642059,337.45633639)(833.71642334,337.50634644)
\curveto(833.86642037,337.61633623)(834.02142021,337.71133614)(834.18142334,337.79134644)
\curveto(834.35141988,337.87133598)(834.52641971,337.9513359)(834.70642334,338.03134644)
\curveto(834.81641942,338.08133577)(834.9314193,338.11633573)(835.05142334,338.13634644)
\curveto(835.18141905,338.16633568)(835.30641893,338.20133565)(835.42642334,338.24134644)
\curveto(835.49641874,338.2513356)(835.56141867,338.26133559)(835.62142334,338.27134644)
\lineto(835.80142334,338.30134644)
\curveto(835.88141835,338.31133554)(835.95641828,338.31633553)(836.02642334,338.31634644)
\curveto(836.10641813,338.32633552)(836.18641805,338.33633551)(836.26642334,338.34634644)
\curveto(836.28641795,338.35633549)(836.31141792,338.35633549)(836.34142334,338.34634644)
\curveto(836.37141786,338.33633551)(836.39641784,338.33633551)(836.41642334,338.34634644)
}
}
{
\newrgbcolor{curcolor}{0 0 0}
\pscustom[linestyle=none,fillstyle=solid,fillcolor=curcolor]
{
\newpath
\moveto(849.77626709,331.62634644)
\curveto(849.79625903,331.56634228)(849.80625902,331.47134238)(849.80626709,331.34134644)
\curveto(849.80625902,331.22134263)(849.80125902,331.13634271)(849.79126709,331.08634644)
\lineto(849.79126709,330.93634644)
\curveto(849.78125904,330.85634299)(849.77125905,330.78134307)(849.76126709,330.71134644)
\curveto(849.76125906,330.6513432)(849.75625907,330.58134327)(849.74626709,330.50134644)
\curveto(849.7262591,330.44134341)(849.71125911,330.38134347)(849.70126709,330.32134644)
\curveto(849.70125912,330.26134359)(849.69125913,330.20134365)(849.67126709,330.14134644)
\curveto(849.63125919,330.01134384)(849.59625923,329.88134397)(849.56626709,329.75134644)
\curveto(849.53625929,329.62134423)(849.49625933,329.50134435)(849.44626709,329.39134644)
\curveto(849.23625959,328.91134494)(848.95625987,328.50634534)(848.60626709,328.17634644)
\curveto(848.25626057,327.85634599)(847.826261,327.61134624)(847.31626709,327.44134644)
\curveto(847.20626162,327.40134645)(847.08626174,327.37134648)(846.95626709,327.35134644)
\curveto(846.83626199,327.33134652)(846.71126211,327.31134654)(846.58126709,327.29134644)
\curveto(846.5212623,327.28134657)(846.45626237,327.27634657)(846.38626709,327.27634644)
\curveto(846.3262625,327.26634658)(846.26626256,327.26134659)(846.20626709,327.26134644)
\curveto(846.16626266,327.2513466)(846.10626272,327.2463466)(846.02626709,327.24634644)
\curveto(845.95626287,327.2463466)(845.90626292,327.2513466)(845.87626709,327.26134644)
\curveto(845.83626299,327.27134658)(845.79626303,327.27634657)(845.75626709,327.27634644)
\curveto(845.71626311,327.26634658)(845.68126314,327.26634658)(845.65126709,327.27634644)
\lineto(845.56126709,327.27634644)
\lineto(845.20126709,327.32134644)
\curveto(845.06126376,327.36134649)(844.9262639,327.40134645)(844.79626709,327.44134644)
\curveto(844.66626416,327.48134637)(844.54126428,327.52634632)(844.42126709,327.57634644)
\curveto(843.97126485,327.77634607)(843.60126522,328.03634581)(843.31126709,328.35634644)
\curveto(843.0212658,328.67634517)(842.78126604,329.06634478)(842.59126709,329.52634644)
\curveto(842.54126628,329.62634422)(842.50126632,329.72634412)(842.47126709,329.82634644)
\curveto(842.45126637,329.92634392)(842.43126639,330.03134382)(842.41126709,330.14134644)
\curveto(842.39126643,330.18134367)(842.38126644,330.21134364)(842.38126709,330.23134644)
\curveto(842.39126643,330.26134359)(842.39126643,330.29634355)(842.38126709,330.33634644)
\curveto(842.36126646,330.41634343)(842.34626648,330.49634335)(842.33626709,330.57634644)
\curveto(842.33626649,330.66634318)(842.3262665,330.7513431)(842.30626709,330.83134644)
\lineto(842.30626709,330.95134644)
\curveto(842.30626652,330.99134286)(842.30126652,331.03634281)(842.29126709,331.08634644)
\curveto(842.28126654,331.13634271)(842.27626655,331.22134263)(842.27626709,331.34134644)
\curveto(842.27626655,331.47134238)(842.28626654,331.56634228)(842.30626709,331.62634644)
\curveto(842.3262665,331.69634215)(842.33126649,331.76634208)(842.32126709,331.83634644)
\curveto(842.31126651,331.90634194)(842.31626651,331.97634187)(842.33626709,332.04634644)
\curveto(842.34626648,332.09634175)(842.35126647,332.13634171)(842.35126709,332.16634644)
\curveto(842.36126646,332.20634164)(842.37126645,332.2513416)(842.38126709,332.30134644)
\curveto(842.41126641,332.42134143)(842.43626639,332.54134131)(842.45626709,332.66134644)
\curveto(842.48626634,332.78134107)(842.5262663,332.89634095)(842.57626709,333.00634644)
\curveto(842.7262661,333.37634047)(842.90626592,333.70634014)(843.11626709,333.99634644)
\curveto(843.33626549,334.29633955)(843.60126522,334.5463393)(843.91126709,334.74634644)
\curveto(844.03126479,334.82633902)(844.15626467,334.89133896)(844.28626709,334.94134644)
\curveto(844.41626441,335.00133885)(844.55126427,335.06133879)(844.69126709,335.12134644)
\curveto(844.81126401,335.17133868)(844.94126388,335.20133865)(845.08126709,335.21134644)
\curveto(845.2212636,335.23133862)(845.36126346,335.26133859)(845.50126709,335.30134644)
\lineto(845.69626709,335.30134644)
\curveto(845.76626306,335.31133854)(845.83126299,335.32133853)(845.89126709,335.33134644)
\curveto(846.78126204,335.34133851)(847.5212613,335.15633869)(848.11126709,334.77634644)
\curveto(848.70126012,334.39633945)(849.1262597,333.90133995)(849.38626709,333.29134644)
\curveto(849.43625939,333.19134066)(849.47625935,333.09134076)(849.50626709,332.99134644)
\curveto(849.53625929,332.89134096)(849.57125925,332.78634106)(849.61126709,332.67634644)
\curveto(849.64125918,332.56634128)(849.66625916,332.4463414)(849.68626709,332.31634644)
\curveto(849.70625912,332.19634165)(849.73125909,332.07134178)(849.76126709,331.94134644)
\curveto(849.77125905,331.89134196)(849.77125905,331.83634201)(849.76126709,331.77634644)
\curveto(849.76125906,331.72634212)(849.76625906,331.67634217)(849.77626709,331.62634644)
\moveto(848.44126709,330.77134644)
\curveto(848.46126036,330.84134301)(848.46626036,330.92134293)(848.45626709,331.01134644)
\lineto(848.45626709,331.26634644)
\curveto(848.45626037,331.65634219)(848.4212604,331.98634186)(848.35126709,332.25634644)
\curveto(848.3212605,332.33634151)(848.29626053,332.41634143)(848.27626709,332.49634644)
\curveto(848.25626057,332.57634127)(848.23126059,332.6513412)(848.20126709,332.72134644)
\curveto(847.9212609,333.37134048)(847.47626135,333.82134003)(846.86626709,334.07134644)
\curveto(846.79626203,334.10133975)(846.7212621,334.12133973)(846.64126709,334.13134644)
\lineto(846.40126709,334.19134644)
\curveto(846.3212625,334.21133964)(846.23626259,334.22133963)(846.14626709,334.22134644)
\lineto(845.87626709,334.22134644)
\lineto(845.60626709,334.17634644)
\curveto(845.50626332,334.15633969)(845.41126341,334.13133972)(845.32126709,334.10134644)
\curveto(845.24126358,334.08133977)(845.16126366,334.0513398)(845.08126709,334.01134644)
\curveto(845.01126381,333.99133986)(844.94626388,333.96133989)(844.88626709,333.92134644)
\curveto(844.826264,333.88133997)(844.77126405,333.84134001)(844.72126709,333.80134644)
\curveto(844.48126434,333.63134022)(844.28626454,333.42634042)(844.13626709,333.18634644)
\curveto(843.98626484,332.9463409)(843.85626497,332.66634118)(843.74626709,332.34634644)
\curveto(843.71626511,332.2463416)(843.69626513,332.14134171)(843.68626709,332.03134644)
\curveto(843.67626515,331.93134192)(843.66126516,331.82634202)(843.64126709,331.71634644)
\curveto(843.63126519,331.67634217)(843.6262652,331.61134224)(843.62626709,331.52134644)
\curveto(843.61626521,331.49134236)(843.61126521,331.45634239)(843.61126709,331.41634644)
\curveto(843.6212652,331.37634247)(843.6262652,331.33134252)(843.62626709,331.28134644)
\lineto(843.62626709,330.98134644)
\curveto(843.6262652,330.88134297)(843.63626519,330.79134306)(843.65626709,330.71134644)
\lineto(843.68626709,330.53134644)
\curveto(843.70626512,330.43134342)(843.7212651,330.33134352)(843.73126709,330.23134644)
\curveto(843.75126507,330.14134371)(843.78126504,330.05634379)(843.82126709,329.97634644)
\curveto(843.9212649,329.73634411)(844.03626479,329.51134434)(844.16626709,329.30134644)
\curveto(844.30626452,329.09134476)(844.47626435,328.91634493)(844.67626709,328.77634644)
\curveto(844.7262641,328.7463451)(844.77126405,328.72134513)(844.81126709,328.70134644)
\curveto(844.85126397,328.68134517)(844.89626393,328.65634519)(844.94626709,328.62634644)
\curveto(845.0262638,328.57634527)(845.11126371,328.53134532)(845.20126709,328.49134644)
\curveto(845.30126352,328.46134539)(845.40626342,328.43134542)(845.51626709,328.40134644)
\curveto(845.56626326,328.38134547)(845.61126321,328.37134548)(845.65126709,328.37134644)
\curveto(845.70126312,328.38134547)(845.75126307,328.38134547)(845.80126709,328.37134644)
\curveto(845.83126299,328.36134549)(845.89126293,328.3513455)(845.98126709,328.34134644)
\curveto(846.08126274,328.33134552)(846.15626267,328.33634551)(846.20626709,328.35634644)
\curveto(846.24626258,328.36634548)(846.28626254,328.36634548)(846.32626709,328.35634644)
\curveto(846.36626246,328.35634549)(846.40626242,328.36634548)(846.44626709,328.38634644)
\curveto(846.5262623,328.40634544)(846.60626222,328.42134543)(846.68626709,328.43134644)
\curveto(846.76626206,328.4513454)(846.84126198,328.47634537)(846.91126709,328.50634644)
\curveto(847.25126157,328.6463452)(847.5262613,328.84134501)(847.73626709,329.09134644)
\curveto(847.94626088,329.34134451)(848.1212607,329.63634421)(848.26126709,329.97634644)
\curveto(848.31126051,330.09634375)(848.34126048,330.22134363)(848.35126709,330.35134644)
\curveto(848.37126045,330.49134336)(848.40126042,330.63134322)(848.44126709,330.77134644)
}
}
{
\newrgbcolor{curcolor}{0 0 0}
\pscustom[linestyle=none,fillstyle=solid,fillcolor=curcolor]
{
\newpath
\moveto(854.90954834,335.33134644)
\curveto(855.13954355,335.33133852)(855.26954342,335.27133858)(855.29954834,335.15134644)
\curveto(855.32954336,335.04133881)(855.34454334,334.87633897)(855.34454834,334.65634644)
\lineto(855.34454834,334.37134644)
\curveto(855.34454334,334.28133957)(855.31954337,334.20633964)(855.26954834,334.14634644)
\curveto(855.20954348,334.06633978)(855.12454356,334.02133983)(855.01454834,334.01134644)
\curveto(854.90454378,334.01133984)(854.79454389,333.99633985)(854.68454834,333.96634644)
\curveto(854.54454414,333.93633991)(854.40954428,333.90633994)(854.27954834,333.87634644)
\curveto(854.15954453,333.84634)(854.04454464,333.80634004)(853.93454834,333.75634644)
\curveto(853.64454504,333.62634022)(853.40954528,333.4463404)(853.22954834,333.21634644)
\curveto(853.04954564,332.99634085)(852.89454579,332.74134111)(852.76454834,332.45134644)
\curveto(852.72454596,332.34134151)(852.69454599,332.22634162)(852.67454834,332.10634644)
\curveto(852.65454603,331.99634185)(852.62954606,331.88134197)(852.59954834,331.76134644)
\curveto(852.5895461,331.71134214)(852.5845461,331.66134219)(852.58454834,331.61134644)
\curveto(852.59454609,331.56134229)(852.59454609,331.51134234)(852.58454834,331.46134644)
\curveto(852.55454613,331.34134251)(852.53954615,331.20134265)(852.53954834,331.04134644)
\curveto(852.54954614,330.89134296)(852.55454613,330.7463431)(852.55454834,330.60634644)
\lineto(852.55454834,328.76134644)
\lineto(852.55454834,328.41634644)
\curveto(852.55454613,328.29634555)(852.54954614,328.18134567)(852.53954834,328.07134644)
\curveto(852.52954616,327.96134589)(852.52454616,327.86634598)(852.52454834,327.78634644)
\curveto(852.53454615,327.70634614)(852.51454617,327.63634621)(852.46454834,327.57634644)
\curveto(852.41454627,327.50634634)(852.33454635,327.46634638)(852.22454834,327.45634644)
\curveto(852.12454656,327.4463464)(852.01454667,327.44134641)(851.89454834,327.44134644)
\lineto(851.62454834,327.44134644)
\curveto(851.57454711,327.46134639)(851.52454716,327.47634637)(851.47454834,327.48634644)
\curveto(851.43454725,327.50634634)(851.40454728,327.53134632)(851.38454834,327.56134644)
\curveto(851.33454735,327.63134622)(851.30454738,327.71634613)(851.29454834,327.81634644)
\lineto(851.29454834,328.14634644)
\lineto(851.29454834,329.30134644)
\lineto(851.29454834,333.45634644)
\lineto(851.29454834,334.49134644)
\lineto(851.29454834,334.79134644)
\curveto(851.30454738,334.89133896)(851.33454735,334.97633887)(851.38454834,335.04634644)
\curveto(851.41454727,335.08633876)(851.46454722,335.11633873)(851.53454834,335.13634644)
\curveto(851.61454707,335.15633869)(851.69954699,335.16633868)(851.78954834,335.16634644)
\curveto(851.87954681,335.17633867)(851.96954672,335.17633867)(852.05954834,335.16634644)
\curveto(852.14954654,335.15633869)(852.21954647,335.14133871)(852.26954834,335.12134644)
\curveto(852.34954634,335.09133876)(852.39954629,335.03133882)(852.41954834,334.94134644)
\curveto(852.44954624,334.86133899)(852.46454622,334.77133908)(852.46454834,334.67134644)
\lineto(852.46454834,334.37134644)
\curveto(852.46454622,334.27133958)(852.4845462,334.18133967)(852.52454834,334.10134644)
\curveto(852.53454615,334.08133977)(852.54454614,334.06633978)(852.55454834,334.05634644)
\lineto(852.59954834,334.01134644)
\curveto(852.70954598,334.01133984)(852.79954589,334.05633979)(852.86954834,334.14634644)
\curveto(852.93954575,334.2463396)(852.99954569,334.32633952)(853.04954834,334.38634644)
\lineto(853.13954834,334.47634644)
\curveto(853.22954546,334.58633926)(853.35454533,334.70133915)(853.51454834,334.82134644)
\curveto(853.67454501,334.94133891)(853.82454486,335.03133882)(853.96454834,335.09134644)
\curveto(854.05454463,335.14133871)(854.14954454,335.17633867)(854.24954834,335.19634644)
\curveto(854.34954434,335.22633862)(854.45454423,335.25633859)(854.56454834,335.28634644)
\curveto(854.62454406,335.29633855)(854.684544,335.30133855)(854.74454834,335.30134644)
\curveto(854.80454388,335.31133854)(854.85954383,335.32133853)(854.90954834,335.33134644)
}
}
{
\newrgbcolor{curcolor}{0 0 0}
\pscustom[linestyle=none,fillstyle=solid,fillcolor=curcolor]
{
\newpath
\moveto(859.91931396,335.33134644)
\curveto(860.14930917,335.33133852)(860.27930904,335.27133858)(860.30931396,335.15134644)
\curveto(860.33930898,335.04133881)(860.35430897,334.87633897)(860.35431396,334.65634644)
\lineto(860.35431396,334.37134644)
\curveto(860.35430897,334.28133957)(860.32930899,334.20633964)(860.27931396,334.14634644)
\curveto(860.2193091,334.06633978)(860.13430919,334.02133983)(860.02431396,334.01134644)
\curveto(859.91430941,334.01133984)(859.80430952,333.99633985)(859.69431396,333.96634644)
\curveto(859.55430977,333.93633991)(859.4193099,333.90633994)(859.28931396,333.87634644)
\curveto(859.16931015,333.84634)(859.05431027,333.80634004)(858.94431396,333.75634644)
\curveto(858.65431067,333.62634022)(858.4193109,333.4463404)(858.23931396,333.21634644)
\curveto(858.05931126,332.99634085)(857.90431142,332.74134111)(857.77431396,332.45134644)
\curveto(857.73431159,332.34134151)(857.70431162,332.22634162)(857.68431396,332.10634644)
\curveto(857.66431166,331.99634185)(857.63931168,331.88134197)(857.60931396,331.76134644)
\curveto(857.59931172,331.71134214)(857.59431173,331.66134219)(857.59431396,331.61134644)
\curveto(857.60431172,331.56134229)(857.60431172,331.51134234)(857.59431396,331.46134644)
\curveto(857.56431176,331.34134251)(857.54931177,331.20134265)(857.54931396,331.04134644)
\curveto(857.55931176,330.89134296)(857.56431176,330.7463431)(857.56431396,330.60634644)
\lineto(857.56431396,328.76134644)
\lineto(857.56431396,328.41634644)
\curveto(857.56431176,328.29634555)(857.55931176,328.18134567)(857.54931396,328.07134644)
\curveto(857.53931178,327.96134589)(857.53431179,327.86634598)(857.53431396,327.78634644)
\curveto(857.54431178,327.70634614)(857.5243118,327.63634621)(857.47431396,327.57634644)
\curveto(857.4243119,327.50634634)(857.34431198,327.46634638)(857.23431396,327.45634644)
\curveto(857.13431219,327.4463464)(857.0243123,327.44134641)(856.90431396,327.44134644)
\lineto(856.63431396,327.44134644)
\curveto(856.58431274,327.46134639)(856.53431279,327.47634637)(856.48431396,327.48634644)
\curveto(856.44431288,327.50634634)(856.41431291,327.53134632)(856.39431396,327.56134644)
\curveto(856.34431298,327.63134622)(856.31431301,327.71634613)(856.30431396,327.81634644)
\lineto(856.30431396,328.14634644)
\lineto(856.30431396,329.30134644)
\lineto(856.30431396,333.45634644)
\lineto(856.30431396,334.49134644)
\lineto(856.30431396,334.79134644)
\curveto(856.31431301,334.89133896)(856.34431298,334.97633887)(856.39431396,335.04634644)
\curveto(856.4243129,335.08633876)(856.47431285,335.11633873)(856.54431396,335.13634644)
\curveto(856.6243127,335.15633869)(856.70931261,335.16633868)(856.79931396,335.16634644)
\curveto(856.88931243,335.17633867)(856.97931234,335.17633867)(857.06931396,335.16634644)
\curveto(857.15931216,335.15633869)(857.22931209,335.14133871)(857.27931396,335.12134644)
\curveto(857.35931196,335.09133876)(857.40931191,335.03133882)(857.42931396,334.94134644)
\curveto(857.45931186,334.86133899)(857.47431185,334.77133908)(857.47431396,334.67134644)
\lineto(857.47431396,334.37134644)
\curveto(857.47431185,334.27133958)(857.49431183,334.18133967)(857.53431396,334.10134644)
\curveto(857.54431178,334.08133977)(857.55431177,334.06633978)(857.56431396,334.05634644)
\lineto(857.60931396,334.01134644)
\curveto(857.7193116,334.01133984)(857.80931151,334.05633979)(857.87931396,334.14634644)
\curveto(857.94931137,334.2463396)(858.00931131,334.32633952)(858.05931396,334.38634644)
\lineto(858.14931396,334.47634644)
\curveto(858.23931108,334.58633926)(858.36431096,334.70133915)(858.52431396,334.82134644)
\curveto(858.68431064,334.94133891)(858.83431049,335.03133882)(858.97431396,335.09134644)
\curveto(859.06431026,335.14133871)(859.15931016,335.17633867)(859.25931396,335.19634644)
\curveto(859.35930996,335.22633862)(859.46430986,335.25633859)(859.57431396,335.28634644)
\curveto(859.63430969,335.29633855)(859.69430963,335.30133855)(859.75431396,335.30134644)
\curveto(859.81430951,335.31133854)(859.86930945,335.32133853)(859.91931396,335.33134644)
}
}
{
\newrgbcolor{curcolor}{0 0 0}
\pscustom[linestyle=none,fillstyle=solid,fillcolor=curcolor]
{
\newpath
\moveto(868.03407959,331.59634644)
\curveto(868.0540719,331.49634235)(868.0540719,331.38134247)(868.03407959,331.25134644)
\curveto(868.02407193,331.13134272)(867.99407196,331.0463428)(867.94407959,330.99634644)
\curveto(867.89407206,330.95634289)(867.81907214,330.92634292)(867.71907959,330.90634644)
\curveto(867.62907233,330.89634295)(867.52407243,330.89134296)(867.40407959,330.89134644)
\lineto(867.04407959,330.89134644)
\curveto(866.92407303,330.90134295)(866.81907314,330.90634294)(866.72907959,330.90634644)
\lineto(862.88907959,330.90634644)
\curveto(862.80907715,330.90634294)(862.72907723,330.90134295)(862.64907959,330.89134644)
\curveto(862.56907739,330.89134296)(862.50407745,330.87634297)(862.45407959,330.84634644)
\curveto(862.41407754,330.82634302)(862.37407758,330.78634306)(862.33407959,330.72634644)
\curveto(862.31407764,330.69634315)(862.29407766,330.6513432)(862.27407959,330.59134644)
\curveto(862.2540777,330.54134331)(862.2540777,330.49134336)(862.27407959,330.44134644)
\curveto(862.28407767,330.39134346)(862.28907767,330.3463435)(862.28907959,330.30634644)
\curveto(862.28907767,330.26634358)(862.29407766,330.22634362)(862.30407959,330.18634644)
\curveto(862.32407763,330.10634374)(862.34407761,330.02134383)(862.36407959,329.93134644)
\curveto(862.38407757,329.851344)(862.41407754,329.77134408)(862.45407959,329.69134644)
\curveto(862.68407727,329.1513447)(863.06407689,328.76634508)(863.59407959,328.53634644)
\curveto(863.6540763,328.50634534)(863.71907624,328.48134537)(863.78907959,328.46134644)
\lineto(863.99907959,328.40134644)
\curveto(864.02907593,328.39134546)(864.07907588,328.38634546)(864.14907959,328.38634644)
\curveto(864.28907567,328.3463455)(864.47407548,328.32634552)(864.70407959,328.32634644)
\curveto(864.93407502,328.32634552)(865.11907484,328.3463455)(865.25907959,328.38634644)
\curveto(865.39907456,328.42634542)(865.52407443,328.46634538)(865.63407959,328.50634644)
\curveto(865.7540742,328.55634529)(865.86407409,328.61634523)(865.96407959,328.68634644)
\curveto(866.07407388,328.75634509)(866.16907379,328.83634501)(866.24907959,328.92634644)
\curveto(866.32907363,329.02634482)(866.39907356,329.13134472)(866.45907959,329.24134644)
\curveto(866.51907344,329.34134451)(866.56907339,329.4463444)(866.60907959,329.55634644)
\curveto(866.6590733,329.66634418)(866.73907322,329.7463441)(866.84907959,329.79634644)
\curveto(866.88907307,329.81634403)(866.954073,329.83134402)(867.04407959,329.84134644)
\curveto(867.13407282,329.851344)(867.22407273,329.851344)(867.31407959,329.84134644)
\curveto(867.40407255,329.84134401)(867.48907247,329.83634401)(867.56907959,329.82634644)
\curveto(867.64907231,329.81634403)(867.70407225,329.79634405)(867.73407959,329.76634644)
\curveto(867.83407212,329.69634415)(867.8590721,329.58134427)(867.80907959,329.42134644)
\curveto(867.72907223,329.1513447)(867.62407233,328.91134494)(867.49407959,328.70134644)
\curveto(867.29407266,328.38134547)(867.06407289,328.11634573)(866.80407959,327.90634644)
\curveto(866.5540734,327.70634614)(866.23407372,327.54134631)(865.84407959,327.41134644)
\curveto(865.74407421,327.37134648)(865.64407431,327.3463465)(865.54407959,327.33634644)
\curveto(865.44407451,327.31634653)(865.33907462,327.29634655)(865.22907959,327.27634644)
\curveto(865.17907478,327.26634658)(865.12907483,327.26134659)(865.07907959,327.26134644)
\curveto(865.03907492,327.26134659)(864.99407496,327.25634659)(864.94407959,327.24634644)
\lineto(864.79407959,327.24634644)
\curveto(864.74407521,327.23634661)(864.68407527,327.23134662)(864.61407959,327.23134644)
\curveto(864.5540754,327.23134662)(864.50407545,327.23634661)(864.46407959,327.24634644)
\lineto(864.32907959,327.24634644)
\curveto(864.27907568,327.25634659)(864.23407572,327.26134659)(864.19407959,327.26134644)
\curveto(864.1540758,327.26134659)(864.11407584,327.26634658)(864.07407959,327.27634644)
\curveto(864.02407593,327.28634656)(863.96907599,327.29634655)(863.90907959,327.30634644)
\curveto(863.84907611,327.30634654)(863.79407616,327.31134654)(863.74407959,327.32134644)
\curveto(863.6540763,327.34134651)(863.56407639,327.36634648)(863.47407959,327.39634644)
\curveto(863.38407657,327.41634643)(863.29907666,327.44134641)(863.21907959,327.47134644)
\curveto(863.17907678,327.49134636)(863.14407681,327.50134635)(863.11407959,327.50134644)
\curveto(863.08407687,327.51134634)(863.04907691,327.52634632)(863.00907959,327.54634644)
\curveto(862.8590771,327.61634623)(862.69907726,327.70134615)(862.52907959,327.80134644)
\curveto(862.23907772,327.99134586)(861.98907797,328.22134563)(861.77907959,328.49134644)
\curveto(861.57907838,328.77134508)(861.40907855,329.08134477)(861.26907959,329.42134644)
\curveto(861.21907874,329.53134432)(861.17907878,329.6463442)(861.14907959,329.76634644)
\curveto(861.12907883,329.88634396)(861.09907886,330.00634384)(861.05907959,330.12634644)
\curveto(861.04907891,330.16634368)(861.04407891,330.20134365)(861.04407959,330.23134644)
\curveto(861.04407891,330.26134359)(861.03907892,330.30134355)(861.02907959,330.35134644)
\curveto(861.00907895,330.43134342)(860.99407896,330.51634333)(860.98407959,330.60634644)
\curveto(860.97407898,330.69634315)(860.959079,330.78634306)(860.93907959,330.87634644)
\lineto(860.93907959,331.08634644)
\curveto(860.92907903,331.12634272)(860.91907904,331.18134267)(860.90907959,331.25134644)
\curveto(860.90907905,331.33134252)(860.91407904,331.39634245)(860.92407959,331.44634644)
\lineto(860.92407959,331.61134644)
\curveto(860.94407901,331.66134219)(860.94907901,331.71134214)(860.93907959,331.76134644)
\curveto(860.93907902,331.82134203)(860.94407901,331.87634197)(860.95407959,331.92634644)
\curveto(860.99407896,332.08634176)(861.02407893,332.2463416)(861.04407959,332.40634644)
\curveto(861.07407888,332.56634128)(861.11907884,332.71634113)(861.17907959,332.85634644)
\curveto(861.22907873,332.96634088)(861.27407868,333.07634077)(861.31407959,333.18634644)
\curveto(861.36407859,333.30634054)(861.41907854,333.42134043)(861.47907959,333.53134644)
\curveto(861.69907826,333.88133997)(861.94907801,334.18133967)(862.22907959,334.43134644)
\curveto(862.50907745,334.69133916)(862.8540771,334.90633894)(863.26407959,335.07634644)
\curveto(863.38407657,335.12633872)(863.50407645,335.16133869)(863.62407959,335.18134644)
\curveto(863.7540762,335.21133864)(863.88907607,335.24133861)(864.02907959,335.27134644)
\curveto(864.07907588,335.28133857)(864.12407583,335.28633856)(864.16407959,335.28634644)
\curveto(864.20407575,335.29633855)(864.24907571,335.30133855)(864.29907959,335.30134644)
\curveto(864.31907564,335.31133854)(864.34407561,335.31133854)(864.37407959,335.30134644)
\curveto(864.40407555,335.29133856)(864.42907553,335.29633855)(864.44907959,335.31634644)
\curveto(864.86907509,335.32633852)(865.23407472,335.28133857)(865.54407959,335.18134644)
\curveto(865.8540741,335.09133876)(866.13407382,334.96633888)(866.38407959,334.80634644)
\curveto(866.43407352,334.78633906)(866.47407348,334.75633909)(866.50407959,334.71634644)
\curveto(866.53407342,334.68633916)(866.56907339,334.66133919)(866.60907959,334.64134644)
\curveto(866.68907327,334.58133927)(866.76907319,334.51133934)(866.84907959,334.43134644)
\curveto(866.93907302,334.3513395)(867.01407294,334.27133958)(867.07407959,334.19134644)
\curveto(867.23407272,333.98133987)(867.36907259,333.78134007)(867.47907959,333.59134644)
\curveto(867.54907241,333.48134037)(867.60407235,333.36134049)(867.64407959,333.23134644)
\curveto(867.68407227,333.10134075)(867.72907223,332.97134088)(867.77907959,332.84134644)
\curveto(867.82907213,332.71134114)(867.86407209,332.57634127)(867.88407959,332.43634644)
\curveto(867.91407204,332.29634155)(867.94907201,332.15634169)(867.98907959,332.01634644)
\curveto(867.99907196,331.9463419)(868.00407195,331.87634197)(868.00407959,331.80634644)
\lineto(868.03407959,331.59634644)
\moveto(866.57907959,332.10634644)
\curveto(866.60907335,332.1463417)(866.63407332,332.19634165)(866.65407959,332.25634644)
\curveto(866.67407328,332.32634152)(866.67407328,332.39634145)(866.65407959,332.46634644)
\curveto(866.59407336,332.68634116)(866.50907345,332.89134096)(866.39907959,333.08134644)
\curveto(866.2590737,333.31134054)(866.10407385,333.50634034)(865.93407959,333.66634644)
\curveto(865.76407419,333.82634002)(865.54407441,333.96133989)(865.27407959,334.07134644)
\curveto(865.20407475,334.09133976)(865.13407482,334.10633974)(865.06407959,334.11634644)
\curveto(864.99407496,334.13633971)(864.91907504,334.15633969)(864.83907959,334.17634644)
\curveto(864.7590752,334.19633965)(864.67407528,334.20633964)(864.58407959,334.20634644)
\lineto(864.32907959,334.20634644)
\curveto(864.29907566,334.18633966)(864.26407569,334.17633967)(864.22407959,334.17634644)
\curveto(864.18407577,334.18633966)(864.14907581,334.18633966)(864.11907959,334.17634644)
\lineto(863.87907959,334.11634644)
\curveto(863.80907615,334.10633974)(863.73907622,334.09133976)(863.66907959,334.07134644)
\curveto(863.37907658,333.9513399)(863.14407681,333.80134005)(862.96407959,333.62134644)
\curveto(862.79407716,333.44134041)(862.63907732,333.21634063)(862.49907959,332.94634644)
\curveto(862.46907749,332.89634095)(862.43907752,332.83134102)(862.40907959,332.75134644)
\curveto(862.37907758,332.68134117)(862.3540776,332.60134125)(862.33407959,332.51134644)
\curveto(862.31407764,332.42134143)(862.30907765,332.33634151)(862.31907959,332.25634644)
\curveto(862.32907763,332.17634167)(862.36407759,332.11634173)(862.42407959,332.07634644)
\curveto(862.50407745,332.01634183)(862.63907732,331.98634186)(862.82907959,331.98634644)
\curveto(863.02907693,331.99634185)(863.19907676,332.00134185)(863.33907959,332.00134644)
\lineto(865.61907959,332.00134644)
\curveto(865.76907419,332.00134185)(865.94907401,331.99634185)(866.15907959,331.98634644)
\curveto(866.36907359,331.98634186)(866.50907345,332.02634182)(866.57907959,332.10634644)
}
}
{
\newrgbcolor{curcolor}{0 0 0}
\pscustom[linestyle=none,fillstyle=solid,fillcolor=curcolor]
{
\newpath
\moveto(876.46572021,331.62634644)
\curveto(876.48571215,331.56634228)(876.49571214,331.47134238)(876.49572021,331.34134644)
\curveto(876.49571214,331.22134263)(876.49071215,331.13634271)(876.48072021,331.08634644)
\lineto(876.48072021,330.93634644)
\curveto(876.47071217,330.85634299)(876.46071218,330.78134307)(876.45072021,330.71134644)
\curveto(876.45071219,330.6513432)(876.44571219,330.58134327)(876.43572021,330.50134644)
\curveto(876.41571222,330.44134341)(876.40071224,330.38134347)(876.39072021,330.32134644)
\curveto(876.39071225,330.26134359)(876.38071226,330.20134365)(876.36072021,330.14134644)
\curveto(876.32071232,330.01134384)(876.28571235,329.88134397)(876.25572021,329.75134644)
\curveto(876.22571241,329.62134423)(876.18571245,329.50134435)(876.13572021,329.39134644)
\curveto(875.92571271,328.91134494)(875.64571299,328.50634534)(875.29572021,328.17634644)
\curveto(874.94571369,327.85634599)(874.51571412,327.61134624)(874.00572021,327.44134644)
\curveto(873.89571474,327.40134645)(873.77571486,327.37134648)(873.64572021,327.35134644)
\curveto(873.52571511,327.33134652)(873.40071524,327.31134654)(873.27072021,327.29134644)
\curveto(873.21071543,327.28134657)(873.14571549,327.27634657)(873.07572021,327.27634644)
\curveto(873.01571562,327.26634658)(872.95571568,327.26134659)(872.89572021,327.26134644)
\curveto(872.85571578,327.2513466)(872.79571584,327.2463466)(872.71572021,327.24634644)
\curveto(872.64571599,327.2463466)(872.59571604,327.2513466)(872.56572021,327.26134644)
\curveto(872.52571611,327.27134658)(872.48571615,327.27634657)(872.44572021,327.27634644)
\curveto(872.40571623,327.26634658)(872.37071627,327.26634658)(872.34072021,327.27634644)
\lineto(872.25072021,327.27634644)
\lineto(871.89072021,327.32134644)
\curveto(871.75071689,327.36134649)(871.61571702,327.40134645)(871.48572021,327.44134644)
\curveto(871.35571728,327.48134637)(871.23071741,327.52634632)(871.11072021,327.57634644)
\curveto(870.66071798,327.77634607)(870.29071835,328.03634581)(870.00072021,328.35634644)
\curveto(869.71071893,328.67634517)(869.47071917,329.06634478)(869.28072021,329.52634644)
\curveto(869.23071941,329.62634422)(869.19071945,329.72634412)(869.16072021,329.82634644)
\curveto(869.1407195,329.92634392)(869.12071952,330.03134382)(869.10072021,330.14134644)
\curveto(869.08071956,330.18134367)(869.07071957,330.21134364)(869.07072021,330.23134644)
\curveto(869.08071956,330.26134359)(869.08071956,330.29634355)(869.07072021,330.33634644)
\curveto(869.05071959,330.41634343)(869.0357196,330.49634335)(869.02572021,330.57634644)
\curveto(869.02571961,330.66634318)(869.01571962,330.7513431)(868.99572021,330.83134644)
\lineto(868.99572021,330.95134644)
\curveto(868.99571964,330.99134286)(868.99071965,331.03634281)(868.98072021,331.08634644)
\curveto(868.97071967,331.13634271)(868.96571967,331.22134263)(868.96572021,331.34134644)
\curveto(868.96571967,331.47134238)(868.97571966,331.56634228)(868.99572021,331.62634644)
\curveto(869.01571962,331.69634215)(869.02071962,331.76634208)(869.01072021,331.83634644)
\curveto(869.00071964,331.90634194)(869.00571963,331.97634187)(869.02572021,332.04634644)
\curveto(869.0357196,332.09634175)(869.0407196,332.13634171)(869.04072021,332.16634644)
\curveto(869.05071959,332.20634164)(869.06071958,332.2513416)(869.07072021,332.30134644)
\curveto(869.10071954,332.42134143)(869.12571951,332.54134131)(869.14572021,332.66134644)
\curveto(869.17571946,332.78134107)(869.21571942,332.89634095)(869.26572021,333.00634644)
\curveto(869.41571922,333.37634047)(869.59571904,333.70634014)(869.80572021,333.99634644)
\curveto(870.02571861,334.29633955)(870.29071835,334.5463393)(870.60072021,334.74634644)
\curveto(870.72071792,334.82633902)(870.84571779,334.89133896)(870.97572021,334.94134644)
\curveto(871.10571753,335.00133885)(871.2407174,335.06133879)(871.38072021,335.12134644)
\curveto(871.50071714,335.17133868)(871.63071701,335.20133865)(871.77072021,335.21134644)
\curveto(871.91071673,335.23133862)(872.05071659,335.26133859)(872.19072021,335.30134644)
\lineto(872.38572021,335.30134644)
\curveto(872.45571618,335.31133854)(872.52071612,335.32133853)(872.58072021,335.33134644)
\curveto(873.47071517,335.34133851)(874.21071443,335.15633869)(874.80072021,334.77634644)
\curveto(875.39071325,334.39633945)(875.81571282,333.90133995)(876.07572021,333.29134644)
\curveto(876.12571251,333.19134066)(876.16571247,333.09134076)(876.19572021,332.99134644)
\curveto(876.22571241,332.89134096)(876.26071238,332.78634106)(876.30072021,332.67634644)
\curveto(876.33071231,332.56634128)(876.35571228,332.4463414)(876.37572021,332.31634644)
\curveto(876.39571224,332.19634165)(876.42071222,332.07134178)(876.45072021,331.94134644)
\curveto(876.46071218,331.89134196)(876.46071218,331.83634201)(876.45072021,331.77634644)
\curveto(876.45071219,331.72634212)(876.45571218,331.67634217)(876.46572021,331.62634644)
\moveto(875.13072021,330.77134644)
\curveto(875.15071349,330.84134301)(875.15571348,330.92134293)(875.14572021,331.01134644)
\lineto(875.14572021,331.26634644)
\curveto(875.14571349,331.65634219)(875.11071353,331.98634186)(875.04072021,332.25634644)
\curveto(875.01071363,332.33634151)(874.98571365,332.41634143)(874.96572021,332.49634644)
\curveto(874.94571369,332.57634127)(874.92071372,332.6513412)(874.89072021,332.72134644)
\curveto(874.61071403,333.37134048)(874.16571447,333.82134003)(873.55572021,334.07134644)
\curveto(873.48571515,334.10133975)(873.41071523,334.12133973)(873.33072021,334.13134644)
\lineto(873.09072021,334.19134644)
\curveto(873.01071563,334.21133964)(872.92571571,334.22133963)(872.83572021,334.22134644)
\lineto(872.56572021,334.22134644)
\lineto(872.29572021,334.17634644)
\curveto(872.19571644,334.15633969)(872.10071654,334.13133972)(872.01072021,334.10134644)
\curveto(871.93071671,334.08133977)(871.85071679,334.0513398)(871.77072021,334.01134644)
\curveto(871.70071694,333.99133986)(871.635717,333.96133989)(871.57572021,333.92134644)
\curveto(871.51571712,333.88133997)(871.46071718,333.84134001)(871.41072021,333.80134644)
\curveto(871.17071747,333.63134022)(870.97571766,333.42634042)(870.82572021,333.18634644)
\curveto(870.67571796,332.9463409)(870.54571809,332.66634118)(870.43572021,332.34634644)
\curveto(870.40571823,332.2463416)(870.38571825,332.14134171)(870.37572021,332.03134644)
\curveto(870.36571827,331.93134192)(870.35071829,331.82634202)(870.33072021,331.71634644)
\curveto(870.32071832,331.67634217)(870.31571832,331.61134224)(870.31572021,331.52134644)
\curveto(870.30571833,331.49134236)(870.30071834,331.45634239)(870.30072021,331.41634644)
\curveto(870.31071833,331.37634247)(870.31571832,331.33134252)(870.31572021,331.28134644)
\lineto(870.31572021,330.98134644)
\curveto(870.31571832,330.88134297)(870.32571831,330.79134306)(870.34572021,330.71134644)
\lineto(870.37572021,330.53134644)
\curveto(870.39571824,330.43134342)(870.41071823,330.33134352)(870.42072021,330.23134644)
\curveto(870.4407182,330.14134371)(870.47071817,330.05634379)(870.51072021,329.97634644)
\curveto(870.61071803,329.73634411)(870.72571791,329.51134434)(870.85572021,329.30134644)
\curveto(870.99571764,329.09134476)(871.16571747,328.91634493)(871.36572021,328.77634644)
\curveto(871.41571722,328.7463451)(871.46071718,328.72134513)(871.50072021,328.70134644)
\curveto(871.5407171,328.68134517)(871.58571705,328.65634519)(871.63572021,328.62634644)
\curveto(871.71571692,328.57634527)(871.80071684,328.53134532)(871.89072021,328.49134644)
\curveto(871.99071665,328.46134539)(872.09571654,328.43134542)(872.20572021,328.40134644)
\curveto(872.25571638,328.38134547)(872.30071634,328.37134548)(872.34072021,328.37134644)
\curveto(872.39071625,328.38134547)(872.4407162,328.38134547)(872.49072021,328.37134644)
\curveto(872.52071612,328.36134549)(872.58071606,328.3513455)(872.67072021,328.34134644)
\curveto(872.77071587,328.33134552)(872.84571579,328.33634551)(872.89572021,328.35634644)
\curveto(872.9357157,328.36634548)(872.97571566,328.36634548)(873.01572021,328.35634644)
\curveto(873.05571558,328.35634549)(873.09571554,328.36634548)(873.13572021,328.38634644)
\curveto(873.21571542,328.40634544)(873.29571534,328.42134543)(873.37572021,328.43134644)
\curveto(873.45571518,328.4513454)(873.53071511,328.47634537)(873.60072021,328.50634644)
\curveto(873.9407147,328.6463452)(874.21571442,328.84134501)(874.42572021,329.09134644)
\curveto(874.635714,329.34134451)(874.81071383,329.63634421)(874.95072021,329.97634644)
\curveto(875.00071364,330.09634375)(875.03071361,330.22134363)(875.04072021,330.35134644)
\curveto(875.06071358,330.49134336)(875.09071355,330.63134322)(875.13072021,330.77134644)
}
}
{
\newrgbcolor{curcolor}{0 0 0}
\pscustom[linestyle=none,fillstyle=solid,fillcolor=curcolor]
{
\newpath
\moveto(831.96145996,297.38369019)
\curveto(832.14145819,297.38367949)(832.34145799,297.38367949)(832.56145996,297.38369019)
\curveto(832.78145755,297.39367948)(832.94645739,297.35867952)(833.05645996,297.27869019)
\curveto(833.1364572,297.21867966)(833.21145712,297.12867975)(833.28145996,297.00869019)
\curveto(833.35145698,296.89867998)(833.41645692,296.79868008)(833.47645996,296.70869019)
\curveto(833.60645673,296.50868037)(833.7364566,296.30368057)(833.86645996,296.09369019)
\curveto(834.00645633,295.89368098)(834.14145619,295.68868119)(834.27145996,295.47869019)
\lineto(834.48145996,295.14869019)
\curveto(834.56145577,295.04868183)(834.6364557,294.94368193)(834.70645996,294.83369019)
\curveto(835.00645533,294.35368252)(835.31145502,293.873683)(835.62145996,293.39369019)
\curveto(835.9314544,292.92368395)(836.24145409,292.44868443)(836.55145996,291.96869019)
\curveto(836.6314537,291.82868505)(836.71645362,291.69368518)(836.80645996,291.56369019)
\curveto(836.90645343,291.44368543)(836.99645334,291.31368556)(837.07645996,291.17369019)
\lineto(837.58645996,290.36369019)
\curveto(837.76645257,290.10368677)(837.94145239,289.84368703)(838.11145996,289.58369019)
\curveto(838.16145217,289.50368737)(838.22145211,289.40368747)(838.29145996,289.28369019)
\curveto(838.37145196,289.1736877)(838.46645187,289.11868776)(838.57645996,289.11869019)
\curveto(838.62645171,289.13868774)(838.65145168,289.15368772)(838.65145996,289.16369019)
\curveto(838.70145163,289.22368765)(838.72645161,289.30868757)(838.72645996,289.41869019)
\lineto(838.72645996,289.73369019)
\lineto(838.72645996,290.91869019)
\lineto(838.72645996,295.50869019)
\lineto(838.72645996,296.40869019)
\curveto(838.72645161,296.4786804)(838.72145161,296.55368032)(838.71145996,296.63369019)
\curveto(838.70145163,296.71368016)(838.70645163,296.78868009)(838.72645996,296.85869019)
\lineto(838.72645996,297.02369019)
\curveto(838.74645159,297.06367981)(838.75645158,297.10367977)(838.75645996,297.14369019)
\curveto(838.76645157,297.18367969)(838.78145155,297.21867966)(838.80145996,297.24869019)
\curveto(838.86145147,297.32867955)(838.95145138,297.36867951)(839.07145996,297.36869019)
\curveto(839.19145114,297.3786795)(839.32145101,297.38367949)(839.46145996,297.38369019)
\curveto(839.52145081,297.38367949)(839.58145075,297.38367949)(839.64145996,297.38369019)
\curveto(839.71145062,297.38367949)(839.77145056,297.3736795)(839.82145996,297.35369019)
\curveto(839.94145039,297.30367957)(840.00645033,297.21367966)(840.01645996,297.08369019)
\curveto(840.0364503,296.96367991)(840.04645029,296.81868006)(840.04645996,296.64869019)
\lineto(840.04645996,294.99869019)
\lineto(840.04645996,288.71369019)
\lineto(840.04645996,287.45369019)
\lineto(840.04645996,287.12369019)
\curveto(840.05645028,287.01368986)(840.0364503,286.92868995)(839.98645996,286.86869019)
\curveto(839.94645039,286.80869007)(839.89645044,286.76869011)(839.83645996,286.74869019)
\curveto(839.78645055,286.73869014)(839.72145061,286.72369015)(839.64145996,286.70369019)
\lineto(839.25145996,286.70369019)
\lineto(838.87645996,286.70369019)
\curveto(838.75645158,286.70369017)(838.65645168,286.72369015)(838.57645996,286.76369019)
\curveto(838.49645184,286.79369008)(838.4314519,286.84369003)(838.38145996,286.91369019)
\curveto(838.34145199,286.98368989)(838.29645204,287.05368982)(838.24645996,287.12369019)
\curveto(838.16645217,287.24368963)(838.08145225,287.36868951)(837.99145996,287.49869019)
\lineto(837.75145996,287.88869019)
\curveto(837.39145294,288.42868845)(837.0364533,288.96368791)(836.68645996,289.49369019)
\curveto(836.336454,290.02368685)(835.98645435,290.56368631)(835.63645996,291.11369019)
\curveto(835.44645489,291.41368546)(835.25145508,291.70868517)(835.05145996,291.99869019)
\curveto(834.86145547,292.28868459)(834.67145566,292.58368429)(834.48145996,292.88369019)
\curveto(834.15145618,293.40368347)(833.80645653,293.92868295)(833.44645996,294.45869019)
\curveto(833.40645693,294.52868235)(833.36645697,294.59368228)(833.32645996,294.65369019)
\curveto(833.28645705,294.72368215)(833.2314571,294.78368209)(833.16145996,294.83369019)
\curveto(833.14145719,294.84368203)(833.12145721,294.85868202)(833.10145996,294.87869019)
\curveto(833.08145725,294.89868198)(833.05645728,294.90368197)(833.02645996,294.89369019)
\curveto(832.96645737,294.873682)(832.9314574,294.83368204)(832.92145996,294.77369019)
\curveto(832.91145742,294.71368216)(832.89645744,294.65368222)(832.87645996,294.59369019)
\lineto(832.87645996,294.48869019)
\curveto(832.85645748,294.41868246)(832.85145748,294.33868254)(832.86145996,294.24869019)
\curveto(832.87145746,294.16868271)(832.87645746,294.08868279)(832.87645996,294.00869019)
\lineto(832.87645996,293.01869019)
\lineto(832.87645996,288.24869019)
\lineto(832.87645996,287.54369019)
\lineto(832.87645996,287.36369019)
\curveto(832.88645745,287.29368958)(832.88145745,287.23368964)(832.86145996,287.18369019)
\lineto(832.86145996,287.06369019)
\curveto(832.84145749,286.96368991)(832.82145751,286.89368998)(832.80145996,286.85369019)
\curveto(832.78145755,286.80369007)(832.74645759,286.76869011)(832.69645996,286.74869019)
\curveto(832.64645769,286.73869014)(832.59145774,286.72369015)(832.53145996,286.70369019)
\lineto(832.23145996,286.70369019)
\curveto(832.09145824,286.70369017)(831.96645837,286.70869017)(831.85645996,286.71869019)
\curveto(831.74645859,286.72869015)(831.66645867,286.7736901)(831.61645996,286.85369019)
\curveto(831.56645877,286.91368996)(831.54145879,286.99368988)(831.54145996,287.09369019)
\lineto(831.54145996,287.42369019)
\lineto(831.54145996,288.63869019)
\lineto(831.54145996,294.90869019)
\lineto(831.54145996,296.52869019)
\lineto(831.54145996,296.90369019)
\curveto(831.54145879,297.04367983)(831.56645877,297.15367972)(831.61645996,297.23369019)
\curveto(831.64645869,297.28367959)(831.70645863,297.32867955)(831.79645996,297.36869019)
\curveto(831.81645852,297.3786795)(831.84145849,297.3786795)(831.87145996,297.36869019)
\curveto(831.91145842,297.36867951)(831.94145839,297.3736795)(831.96145996,297.38369019)
}
}
{
\newrgbcolor{curcolor}{0 0 0}
\pscustom[linestyle=none,fillstyle=solid,fillcolor=curcolor]
{
\newpath
\moveto(849.22200684,290.90369019)
\curveto(849.24199878,290.84368603)(849.25199877,290.74868613)(849.25200684,290.61869019)
\curveto(849.25199877,290.49868638)(849.24699877,290.41368646)(849.23700684,290.36369019)
\lineto(849.23700684,290.21369019)
\curveto(849.22699879,290.13368674)(849.2169988,290.05868682)(849.20700684,289.98869019)
\curveto(849.20699881,289.92868695)(849.20199882,289.85868702)(849.19200684,289.77869019)
\curveto(849.17199885,289.71868716)(849.15699886,289.65868722)(849.14700684,289.59869019)
\curveto(849.14699887,289.53868734)(849.13699888,289.4786874)(849.11700684,289.41869019)
\curveto(849.07699894,289.28868759)(849.04199898,289.15868772)(849.01200684,289.02869019)
\curveto(848.98199904,288.89868798)(848.94199908,288.7786881)(848.89200684,288.66869019)
\curveto(848.68199934,288.18868869)(848.40199962,287.78368909)(848.05200684,287.45369019)
\curveto(847.70200032,287.13368974)(847.27200075,286.88868999)(846.76200684,286.71869019)
\curveto(846.65200137,286.6786902)(846.53200149,286.64869023)(846.40200684,286.62869019)
\curveto(846.28200174,286.60869027)(846.15700186,286.58869029)(846.02700684,286.56869019)
\curveto(845.96700205,286.55869032)(845.90200212,286.55369032)(845.83200684,286.55369019)
\curveto(845.77200225,286.54369033)(845.71200231,286.53869034)(845.65200684,286.53869019)
\curveto(845.61200241,286.52869035)(845.55200247,286.52369035)(845.47200684,286.52369019)
\curveto(845.40200262,286.52369035)(845.35200267,286.52869035)(845.32200684,286.53869019)
\curveto(845.28200274,286.54869033)(845.24200278,286.55369032)(845.20200684,286.55369019)
\curveto(845.16200286,286.54369033)(845.12700289,286.54369033)(845.09700684,286.55369019)
\lineto(845.00700684,286.55369019)
\lineto(844.64700684,286.59869019)
\curveto(844.50700351,286.63869024)(844.37200365,286.6786902)(844.24200684,286.71869019)
\curveto(844.11200391,286.75869012)(843.98700403,286.80369007)(843.86700684,286.85369019)
\curveto(843.4170046,287.05368982)(843.04700497,287.31368956)(842.75700684,287.63369019)
\curveto(842.46700555,287.95368892)(842.22700579,288.34368853)(842.03700684,288.80369019)
\curveto(841.98700603,288.90368797)(841.94700607,289.00368787)(841.91700684,289.10369019)
\curveto(841.89700612,289.20368767)(841.87700614,289.30868757)(841.85700684,289.41869019)
\curveto(841.83700618,289.45868742)(841.82700619,289.48868739)(841.82700684,289.50869019)
\curveto(841.83700618,289.53868734)(841.83700618,289.5736873)(841.82700684,289.61369019)
\curveto(841.80700621,289.69368718)(841.79200623,289.7736871)(841.78200684,289.85369019)
\curveto(841.78200624,289.94368693)(841.77200625,290.02868685)(841.75200684,290.10869019)
\lineto(841.75200684,290.22869019)
\curveto(841.75200627,290.26868661)(841.74700627,290.31368656)(841.73700684,290.36369019)
\curveto(841.72700629,290.41368646)(841.7220063,290.49868638)(841.72200684,290.61869019)
\curveto(841.7220063,290.74868613)(841.73200629,290.84368603)(841.75200684,290.90369019)
\curveto(841.77200625,290.9736859)(841.77700624,291.04368583)(841.76700684,291.11369019)
\curveto(841.75700626,291.18368569)(841.76200626,291.25368562)(841.78200684,291.32369019)
\curveto(841.79200623,291.3736855)(841.79700622,291.41368546)(841.79700684,291.44369019)
\curveto(841.80700621,291.48368539)(841.8170062,291.52868535)(841.82700684,291.57869019)
\curveto(841.85700616,291.69868518)(841.88200614,291.81868506)(841.90200684,291.93869019)
\curveto(841.93200609,292.05868482)(841.97200605,292.1736847)(842.02200684,292.28369019)
\curveto(842.17200585,292.65368422)(842.35200567,292.98368389)(842.56200684,293.27369019)
\curveto(842.78200524,293.5736833)(843.04700497,293.82368305)(843.35700684,294.02369019)
\curveto(843.47700454,294.10368277)(843.60200442,294.16868271)(843.73200684,294.21869019)
\curveto(843.86200416,294.2786826)(843.99700402,294.33868254)(844.13700684,294.39869019)
\curveto(844.25700376,294.44868243)(844.38700363,294.4786824)(844.52700684,294.48869019)
\curveto(844.66700335,294.50868237)(844.80700321,294.53868234)(844.94700684,294.57869019)
\lineto(845.14200684,294.57869019)
\curveto(845.21200281,294.58868229)(845.27700274,294.59868228)(845.33700684,294.60869019)
\curveto(846.22700179,294.61868226)(846.96700105,294.43368244)(847.55700684,294.05369019)
\curveto(848.14699987,293.6736832)(848.57199945,293.1786837)(848.83200684,292.56869019)
\curveto(848.88199914,292.46868441)(848.9219991,292.36868451)(848.95200684,292.26869019)
\curveto(848.98199904,292.16868471)(849.016999,292.06368481)(849.05700684,291.95369019)
\curveto(849.08699893,291.84368503)(849.11199891,291.72368515)(849.13200684,291.59369019)
\curveto(849.15199887,291.4736854)(849.17699884,291.34868553)(849.20700684,291.21869019)
\curveto(849.2169988,291.16868571)(849.2169988,291.11368576)(849.20700684,291.05369019)
\curveto(849.20699881,291.00368587)(849.21199881,290.95368592)(849.22200684,290.90369019)
\moveto(847.88700684,290.04869019)
\curveto(847.90700011,290.11868676)(847.91200011,290.19868668)(847.90200684,290.28869019)
\lineto(847.90200684,290.54369019)
\curveto(847.90200012,290.93368594)(847.86700015,291.26368561)(847.79700684,291.53369019)
\curveto(847.76700025,291.61368526)(847.74200028,291.69368518)(847.72200684,291.77369019)
\curveto(847.70200032,291.85368502)(847.67700034,291.92868495)(847.64700684,291.99869019)
\curveto(847.36700065,292.64868423)(846.9220011,293.09868378)(846.31200684,293.34869019)
\curveto(846.24200178,293.3786835)(846.16700185,293.39868348)(846.08700684,293.40869019)
\lineto(845.84700684,293.46869019)
\curveto(845.76700225,293.48868339)(845.68200234,293.49868338)(845.59200684,293.49869019)
\lineto(845.32200684,293.49869019)
\lineto(845.05200684,293.45369019)
\curveto(844.95200307,293.43368344)(844.85700316,293.40868347)(844.76700684,293.37869019)
\curveto(844.68700333,293.35868352)(844.60700341,293.32868355)(844.52700684,293.28869019)
\curveto(844.45700356,293.26868361)(844.39200363,293.23868364)(844.33200684,293.19869019)
\curveto(844.27200375,293.15868372)(844.2170038,293.11868376)(844.16700684,293.07869019)
\curveto(843.92700409,292.90868397)(843.73200429,292.70368417)(843.58200684,292.46369019)
\curveto(843.43200459,292.22368465)(843.30200472,291.94368493)(843.19200684,291.62369019)
\curveto(843.16200486,291.52368535)(843.14200488,291.41868546)(843.13200684,291.30869019)
\curveto(843.1220049,291.20868567)(843.10700491,291.10368577)(843.08700684,290.99369019)
\curveto(843.07700494,290.95368592)(843.07200495,290.88868599)(843.07200684,290.79869019)
\curveto(843.06200496,290.76868611)(843.05700496,290.73368614)(843.05700684,290.69369019)
\curveto(843.06700495,290.65368622)(843.07200495,290.60868627)(843.07200684,290.55869019)
\lineto(843.07200684,290.25869019)
\curveto(843.07200495,290.15868672)(843.08200494,290.06868681)(843.10200684,289.98869019)
\lineto(843.13200684,289.80869019)
\curveto(843.15200487,289.70868717)(843.16700485,289.60868727)(843.17700684,289.50869019)
\curveto(843.19700482,289.41868746)(843.22700479,289.33368754)(843.26700684,289.25369019)
\curveto(843.36700465,289.01368786)(843.48200454,288.78868809)(843.61200684,288.57869019)
\curveto(843.75200427,288.36868851)(843.9220041,288.19368868)(844.12200684,288.05369019)
\curveto(844.17200385,288.02368885)(844.2170038,287.99868888)(844.25700684,287.97869019)
\curveto(844.29700372,287.95868892)(844.34200368,287.93368894)(844.39200684,287.90369019)
\curveto(844.47200355,287.85368902)(844.55700346,287.80868907)(844.64700684,287.76869019)
\curveto(844.74700327,287.73868914)(844.85200317,287.70868917)(844.96200684,287.67869019)
\curveto(845.01200301,287.65868922)(845.05700296,287.64868923)(845.09700684,287.64869019)
\curveto(845.14700287,287.65868922)(845.19700282,287.65868922)(845.24700684,287.64869019)
\curveto(845.27700274,287.63868924)(845.33700268,287.62868925)(845.42700684,287.61869019)
\curveto(845.52700249,287.60868927)(845.60200242,287.61368926)(845.65200684,287.63369019)
\curveto(845.69200233,287.64368923)(845.73200229,287.64368923)(845.77200684,287.63369019)
\curveto(845.81200221,287.63368924)(845.85200217,287.64368923)(845.89200684,287.66369019)
\curveto(845.97200205,287.68368919)(846.05200197,287.69868918)(846.13200684,287.70869019)
\curveto(846.21200181,287.72868915)(846.28700173,287.75368912)(846.35700684,287.78369019)
\curveto(846.69700132,287.92368895)(846.97200105,288.11868876)(847.18200684,288.36869019)
\curveto(847.39200063,288.61868826)(847.56700045,288.91368796)(847.70700684,289.25369019)
\curveto(847.75700026,289.3736875)(847.78700023,289.49868738)(847.79700684,289.62869019)
\curveto(847.8170002,289.76868711)(847.84700017,289.90868697)(847.88700684,290.04869019)
}
}
{
\newrgbcolor{curcolor}{0 0 0}
\pscustom[linestyle=none,fillstyle=solid,fillcolor=curcolor]
{
\newpath
\moveto(854.40028809,294.60869019)
\curveto(854.7802831,294.61868226)(855.10028278,294.5786823)(855.36028809,294.48869019)
\curveto(855.63028225,294.39868248)(855.87528201,294.26868261)(856.09528809,294.09869019)
\curveto(856.17528171,294.04868283)(856.24028164,293.9786829)(856.29028809,293.88869019)
\curveto(856.35028153,293.80868307)(856.41528147,293.73368314)(856.48528809,293.66369019)
\curveto(856.50528138,293.64368323)(856.53528135,293.61868326)(856.57528809,293.58869019)
\curveto(856.61528127,293.55868332)(856.66528122,293.54868333)(856.72528809,293.55869019)
\curveto(856.82528106,293.58868329)(856.91028097,293.64868323)(856.98028809,293.73869019)
\curveto(857.06028082,293.83868304)(857.14028074,293.91368296)(857.22028809,293.96369019)
\curveto(857.36028052,294.0736828)(857.50528038,294.16868271)(857.65528809,294.24869019)
\curveto(857.80528008,294.33868254)(857.97027991,294.41368246)(858.15028809,294.47369019)
\curveto(858.23027965,294.50368237)(858.31527957,294.52368235)(858.40528809,294.53369019)
\curveto(858.50527938,294.55368232)(858.60027928,294.5736823)(858.69028809,294.59369019)
\curveto(858.74027914,294.60368227)(858.7852791,294.60868227)(858.82528809,294.60869019)
\lineto(858.97528809,294.60869019)
\curveto(859.02527886,294.62868225)(859.09527879,294.63368224)(859.18528809,294.62369019)
\curveto(859.27527861,294.62368225)(859.34027854,294.61868226)(859.38028809,294.60869019)
\curveto(859.43027845,294.59868228)(859.50527838,294.59368228)(859.60528809,294.59369019)
\curveto(859.69527819,294.5736823)(859.7802781,294.55368232)(859.86028809,294.53369019)
\curveto(859.95027793,294.52368235)(860.03527785,294.50368237)(860.11528809,294.47369019)
\curveto(860.16527772,294.45368242)(860.21027767,294.43868244)(860.25028809,294.42869019)
\curveto(860.30027758,294.42868245)(860.35027753,294.41868246)(860.40028809,294.39869019)
\curveto(860.90027698,294.1786827)(861.24527664,293.83868304)(861.43528809,293.37869019)
\curveto(861.47527641,293.29868358)(861.50527638,293.20868367)(861.52528809,293.10869019)
\curveto(861.54527634,293.01868386)(861.56527632,292.91868396)(861.58528809,292.80869019)
\curveto(861.60527628,292.7786841)(861.61027627,292.74368413)(861.60028809,292.70369019)
\curveto(861.60027628,292.6736842)(861.60527628,292.64368423)(861.61528809,292.61369019)
\lineto(861.61528809,292.47869019)
\curveto(861.62527626,292.43868444)(861.62527626,292.39368448)(861.61528809,292.34369019)
\curveto(861.61527627,292.29368458)(861.61527627,292.24368463)(861.61528809,292.19369019)
\lineto(861.61528809,291.60869019)
\lineto(861.61528809,290.64869019)
\lineto(861.61528809,287.79869019)
\curveto(861.61527627,287.63868924)(861.61527627,287.44868943)(861.61528809,287.22869019)
\curveto(861.62527626,287.00868987)(861.5852763,286.86369001)(861.49528809,286.79369019)
\curveto(861.45527643,286.76369011)(861.39027649,286.73869014)(861.30028809,286.71869019)
\curveto(861.21027667,286.70869017)(861.11527677,286.70369017)(861.01528809,286.70369019)
\curveto(860.91527697,286.70369017)(860.81527707,286.70869017)(860.71528809,286.71869019)
\curveto(860.62527726,286.72869015)(860.56027732,286.74869013)(860.52028809,286.77869019)
\curveto(860.46027742,286.80869007)(860.42027746,286.86869001)(860.40028809,286.95869019)
\curveto(860.3802775,287.01868986)(860.37527751,287.0786898)(860.38528809,287.13869019)
\curveto(860.39527749,287.20868967)(860.39027749,287.2736896)(860.37028809,287.33369019)
\curveto(860.36027752,287.38368949)(860.35527753,287.43868944)(860.35528809,287.49869019)
\curveto(860.36527752,287.56868931)(860.37027751,287.63368924)(860.37028809,287.69369019)
\lineto(860.37028809,288.36869019)
\lineto(860.37028809,291.23369019)
\curveto(860.37027751,291.56368531)(860.36027752,291.873685)(860.34028809,292.16369019)
\curveto(860.33027755,292.46368441)(860.26027762,292.71368416)(860.13028809,292.91369019)
\curveto(859.9802779,293.15368372)(859.75027813,293.32868355)(859.44028809,293.43869019)
\curveto(859.3802785,293.45868342)(859.31527857,293.46868341)(859.24528809,293.46869019)
\curveto(859.1852787,293.4786834)(859.12027876,293.49368338)(859.05028809,293.51369019)
\curveto(859.01027887,293.52368335)(858.94527894,293.52368335)(858.85528809,293.51369019)
\curveto(858.76527912,293.51368336)(858.70527918,293.50868337)(858.67528809,293.49869019)
\curveto(858.62527926,293.48868339)(858.57527931,293.48368339)(858.52528809,293.48369019)
\curveto(858.47527941,293.49368338)(858.42527946,293.48868339)(858.37528809,293.46869019)
\curveto(858.23527965,293.43868344)(858.10027978,293.39868348)(857.97028809,293.34869019)
\curveto(857.45028043,293.12868375)(857.10028078,292.74368413)(856.92028809,292.19369019)
\curveto(856.87028101,292.02368485)(856.84028104,291.82868505)(856.83028809,291.60869019)
\lineto(856.83028809,290.93369019)
\lineto(856.83028809,288.96869019)
\lineto(856.83028809,287.51369019)
\lineto(856.83028809,287.13869019)
\curveto(856.83028105,287.01868986)(856.80528108,286.92368995)(856.75528809,286.85369019)
\curveto(856.70528118,286.7736901)(856.62028126,286.72869015)(856.50028809,286.71869019)
\curveto(856.3802815,286.70869017)(856.25528163,286.70369017)(856.12528809,286.70369019)
\curveto(855.95528193,286.70369017)(855.83028205,286.72369015)(855.75028809,286.76369019)
\curveto(855.66028222,286.81369006)(855.60528228,286.89368998)(855.58528809,287.00369019)
\curveto(855.57528231,287.12368975)(855.57028231,287.25368962)(855.57028809,287.39369019)
\lineto(855.57028809,288.81869019)
\lineto(855.57028809,291.29369019)
\curveto(855.57028231,291.61368526)(855.56028232,291.90868497)(855.54028809,292.17869019)
\curveto(855.52028236,292.45868442)(855.45028243,292.69868418)(855.33028809,292.89869019)
\curveto(855.22028266,293.0786838)(855.09528279,293.20868367)(854.95528809,293.28869019)
\curveto(854.81528307,293.3786835)(854.62528326,293.44868343)(854.38528809,293.49869019)
\curveto(854.34528354,293.50868337)(854.30028358,293.51368336)(854.25028809,293.51369019)
\lineto(854.11528809,293.51369019)
\curveto(853.89528399,293.51368336)(853.70028418,293.48868339)(853.53028809,293.43869019)
\curveto(853.37028451,293.38868349)(853.22528466,293.32368355)(853.09528809,293.24369019)
\curveto(852.5852853,292.93368394)(852.24528564,292.46868441)(852.07528809,291.84869019)
\curveto(852.03528585,291.71868516)(852.01528587,291.56868531)(852.01528809,291.39869019)
\curveto(852.02528586,291.23868564)(852.03028585,291.0786858)(852.03028809,290.91869019)
\lineto(852.03028809,289.22369019)
\lineto(852.03028809,287.57369019)
\lineto(852.03028809,287.16869019)
\curveto(852.03028585,287.02868985)(852.00028588,286.91868996)(851.94028809,286.83869019)
\curveto(851.89028599,286.76869011)(851.81528607,286.72869015)(851.71528809,286.71869019)
\curveto(851.61528627,286.70869017)(851.51028637,286.70369017)(851.40028809,286.70369019)
\lineto(851.17528809,286.70369019)
\curveto(851.11528677,286.72369015)(851.05528683,286.73869014)(850.99528809,286.74869019)
\curveto(850.94528694,286.75869012)(850.90028698,286.78869009)(850.86028809,286.83869019)
\curveto(850.81028707,286.89868998)(850.7852871,286.9736899)(850.78528809,287.06369019)
\lineto(850.78528809,287.37869019)
\lineto(850.78528809,288.35369019)
\lineto(850.78528809,292.64369019)
\lineto(850.78528809,293.75369019)
\lineto(850.78528809,294.03869019)
\curveto(850.7852871,294.13868274)(850.80528708,294.21868266)(850.84528809,294.27869019)
\curveto(850.87528701,294.33868254)(850.92028696,294.3786825)(850.98028809,294.39869019)
\curveto(851.06028682,294.42868245)(851.1852867,294.44368243)(851.35528809,294.44369019)
\curveto(851.53528635,294.44368243)(851.66528622,294.42868245)(851.74528809,294.39869019)
\curveto(851.82528606,294.35868252)(851.880286,294.30868257)(851.91028809,294.24869019)
\curveto(851.93028595,294.19868268)(851.94028594,294.13868274)(851.94028809,294.06869019)
\curveto(851.95028593,293.99868288)(851.96028592,293.93368294)(851.97028809,293.87369019)
\curveto(851.9802859,293.81368306)(852.00028588,293.76368311)(852.03028809,293.72369019)
\curveto(852.06028582,293.68368319)(852.11028577,293.66368321)(852.18028809,293.66369019)
\curveto(852.20028568,293.68368319)(852.22028566,293.69368318)(852.24028809,293.69369019)
\curveto(852.27028561,293.69368318)(852.29528559,293.70368317)(852.31528809,293.72369019)
\curveto(852.37528551,293.7736831)(852.43028545,293.82368305)(852.48028809,293.87369019)
\lineto(852.66028809,294.02369019)
\curveto(852.880285,294.18368269)(853.13028475,294.32368255)(853.41028809,294.44369019)
\curveto(853.51028437,294.48368239)(853.61028427,294.50868237)(853.71028809,294.51869019)
\curveto(853.81028407,294.53868234)(853.91528397,294.56368231)(854.02528809,294.59369019)
\lineto(854.20528809,294.59369019)
\curveto(854.27528361,294.60368227)(854.34028354,294.60868227)(854.40028809,294.60869019)
}
}
{
\newrgbcolor{curcolor}{0 0 0}
\pscustom[linestyle=none,fillstyle=solid,fillcolor=curcolor]
{
\newpath
\moveto(870.93802246,290.73869019)
\curveto(870.94801411,290.68868619)(870.95301411,290.61868626)(870.95302246,290.52869019)
\curveto(870.95301411,290.44868643)(870.94801411,290.38368649)(870.93802246,290.33369019)
\curveto(870.93801412,290.29368658)(870.93301413,290.25368662)(870.92302246,290.21369019)
\lineto(870.92302246,290.09369019)
\curveto(870.90301416,290.01368686)(870.89301417,289.93368694)(870.89302246,289.85369019)
\curveto(870.89301417,289.7736871)(870.88301418,289.69368718)(870.86302246,289.61369019)
\curveto(870.85301421,289.5736873)(870.84801421,289.53368734)(870.84802246,289.49369019)
\curveto(870.84801421,289.46368741)(870.84301422,289.42868745)(870.83302246,289.38869019)
\curveto(870.80301426,289.2786876)(870.77301429,289.1736877)(870.74302246,289.07369019)
\curveto(870.72301434,288.9736879)(870.69301437,288.873688)(870.65302246,288.77369019)
\curveto(870.51301455,288.42368845)(870.34301472,288.10868877)(870.14302246,287.82869019)
\curveto(869.94301512,287.54868933)(869.69301537,287.30868957)(869.39302246,287.10869019)
\curveto(869.24301582,287.00868987)(869.09801596,286.92368995)(868.95802246,286.85369019)
\curveto(868.84801621,286.80369007)(868.73801632,286.76369011)(868.62802246,286.73369019)
\curveto(868.52801653,286.70369017)(868.42301664,286.6736902)(868.31302246,286.64369019)
\curveto(868.24301682,286.62369025)(868.17801688,286.61369026)(868.11802246,286.61369019)
\curveto(868.058017,286.60369027)(867.99801706,286.58869029)(867.93802246,286.56869019)
\lineto(867.78802246,286.56869019)
\curveto(867.73801732,286.54869033)(867.6630174,286.53869034)(867.56302246,286.53869019)
\curveto(867.4630176,286.52869035)(867.38301768,286.53369034)(867.32302246,286.55369019)
\lineto(867.17302246,286.55369019)
\curveto(867.13301793,286.56369031)(867.08801797,286.56869031)(867.03802246,286.56869019)
\curveto(866.99801806,286.56869031)(866.95301811,286.5736903)(866.90302246,286.58369019)
\curveto(866.75301831,286.62369025)(866.60301846,286.65869022)(866.45302246,286.68869019)
\curveto(866.31301875,286.71869016)(866.17301889,286.76369011)(866.03302246,286.82369019)
\curveto(865.83301923,286.90368997)(865.65301941,287.00368987)(865.49302246,287.12369019)
\lineto(865.31302246,287.27369019)
\curveto(865.25301981,287.33368954)(865.18301988,287.3736895)(865.10302246,287.39369019)
\curveto(865.04302002,287.40368947)(864.99302007,287.38868949)(864.95302246,287.34869019)
\curveto(864.92302014,287.31868956)(864.89802016,287.2736896)(864.87802246,287.21369019)
\curveto(864.86802019,287.15368972)(864.8580202,287.08868979)(864.84802246,287.01869019)
\curveto(864.84802021,286.95868992)(864.83802022,286.91368996)(864.81802246,286.88369019)
\curveto(864.77802028,286.83369004)(864.73302033,286.78869009)(864.68302246,286.74869019)
\curveto(864.63302043,286.72869015)(864.5630205,286.71369016)(864.47302246,286.70369019)
\lineto(864.20302246,286.70369019)
\curveto(864.11302095,286.70369017)(864.02802103,286.70869017)(863.94802246,286.71869019)
\curveto(863.86802119,286.73869014)(863.80802125,286.75869012)(863.76802246,286.77869019)
\curveto(863.74802131,286.79869008)(863.72802133,286.82369005)(863.70802246,286.85369019)
\lineto(863.64802246,286.94369019)
\curveto(863.61802144,287.02368985)(863.60302146,287.14368973)(863.60302246,287.30369019)
\curveto(863.61302145,287.46368941)(863.61802144,287.59868928)(863.61802246,287.70869019)
\lineto(863.61802246,296.51369019)
\curveto(863.61802144,296.63368024)(863.61302145,296.75868012)(863.60302246,296.88869019)
\curveto(863.60302146,297.02867985)(863.62802143,297.13867974)(863.67802246,297.21869019)
\curveto(863.71802134,297.2786796)(863.78302128,297.32867955)(863.87302246,297.36869019)
\curveto(863.89302117,297.36867951)(863.91802114,297.36867951)(863.94802246,297.36869019)
\curveto(863.97802108,297.3786795)(864.00302106,297.38367949)(864.02302246,297.38369019)
\curveto(864.1630209,297.39367948)(864.30802075,297.39367948)(864.45802246,297.38369019)
\curveto(864.61802044,297.38367949)(864.72802033,297.34367953)(864.78802246,297.26369019)
\curveto(864.83802022,297.18367969)(864.8630202,297.06867981)(864.86302246,296.91869019)
\lineto(864.86302246,296.51369019)
\lineto(864.86302246,294.75869019)
\lineto(864.86302246,294.50369019)
\lineto(864.86302246,294.21869019)
\curveto(864.87302019,294.12868275)(864.88302018,294.04368283)(864.89302246,293.96369019)
\curveto(864.91302015,293.89368298)(864.94302012,293.84368303)(864.98302246,293.81369019)
\curveto(865.02302004,293.78368309)(865.06801999,293.7786831)(865.11802246,293.79869019)
\curveto(865.16801989,293.81868306)(865.20801985,293.83868304)(865.23802246,293.85869019)
\curveto(865.28801977,293.89868298)(865.33301973,293.93868294)(865.37302246,293.97869019)
\lineto(865.52302246,294.09869019)
\curveto(865.59301947,294.14868273)(865.6630194,294.19368268)(865.73302246,294.23369019)
\lineto(865.97302246,294.35369019)
\curveto(866.15301891,294.44368243)(866.36801869,294.50868237)(866.61802246,294.54869019)
\curveto(866.86801819,294.58868229)(867.12301794,294.60868227)(867.38302246,294.60869019)
\curveto(867.64301742,294.60868227)(867.89801716,294.58368229)(868.14802246,294.53369019)
\curveto(868.39801666,294.49368238)(868.61801644,294.43368244)(868.80802246,294.35369019)
\curveto(869.20801585,294.18368269)(869.55301551,293.94868293)(869.84302246,293.64869019)
\curveto(870.13301493,293.34868353)(870.3630147,292.99868388)(870.53302246,292.59869019)
\curveto(870.58301448,292.48868439)(870.62301444,292.3786845)(870.65302246,292.26869019)
\curveto(870.69301437,292.16868471)(870.73301433,292.06368481)(870.77302246,291.95369019)
\curveto(870.80301426,291.84368503)(870.82301424,291.72868515)(870.83302246,291.60869019)
\lineto(870.89302246,291.27869019)
\curveto(870.90301416,291.24868563)(870.90801415,291.21368566)(870.90802246,291.17369019)
\curveto(870.90801415,291.14368573)(870.91301415,291.11368576)(870.92302246,291.08369019)
\curveto(870.94301412,291.02368585)(870.94301412,290.96368591)(870.92302246,290.90369019)
\curveto(870.91301415,290.85368602)(870.91801414,290.79868608)(870.93802246,290.73869019)
\moveto(869.60302246,290.34869019)
\curveto(869.62301544,290.39868648)(869.62801543,290.45868642)(869.61802246,290.52869019)
\curveto(869.60801545,290.59868628)(869.60301546,290.66368621)(869.60302246,290.72369019)
\curveto(869.60301546,290.89368598)(869.59301547,291.05368582)(869.57302246,291.20369019)
\curveto(869.5630155,291.35368552)(869.53301553,291.48868539)(869.48302246,291.60869019)
\curveto(869.45301561,291.70868517)(869.42801563,291.79868508)(869.40802246,291.87869019)
\curveto(869.38801567,291.95868492)(869.3580157,292.03868484)(869.31802246,292.11869019)
\curveto(869.20801585,292.36868451)(869.058016,292.59868428)(868.86802246,292.80869019)
\curveto(868.67801638,293.02868385)(868.4580166,293.19368368)(868.20802246,293.30369019)
\curveto(868.12801693,293.33368354)(868.04801701,293.35868352)(867.96802246,293.37869019)
\curveto(867.89801716,293.40868347)(867.82301724,293.43368344)(867.74302246,293.45369019)
\curveto(867.63301743,293.48368339)(867.52301754,293.49868338)(867.41302246,293.49869019)
\curveto(867.30301776,293.50868337)(867.18301788,293.51368336)(867.05302246,293.51369019)
\curveto(867.00301806,293.50368337)(866.9580181,293.49368338)(866.91802246,293.48369019)
\lineto(866.78302246,293.48369019)
\lineto(866.51302246,293.42369019)
\curveto(866.43301863,293.40368347)(866.35301871,293.3736835)(866.27302246,293.33369019)
\curveto(865.93301913,293.19368368)(865.6630194,292.98368389)(865.46302246,292.70369019)
\curveto(865.2630198,292.43368444)(865.10301996,292.11368476)(864.98302246,291.74369019)
\curveto(864.94302012,291.63368524)(864.91802014,291.52368535)(864.90802246,291.41369019)
\curveto(864.89802016,291.30368557)(864.87802018,291.18868569)(864.84802246,291.06869019)
\curveto(864.83802022,291.01868586)(864.83802022,290.9736859)(864.84802246,290.93369019)
\curveto(864.8580202,290.89368598)(864.85302021,290.84868603)(864.83302246,290.79869019)
\curveto(864.82302024,290.74868613)(864.81802024,290.6736862)(864.81802246,290.57369019)
\curveto(864.81802024,290.48368639)(864.82302024,290.41368646)(864.83302246,290.36369019)
\lineto(864.83302246,290.24369019)
\curveto(864.84302022,290.20368667)(864.84802021,290.16368671)(864.84802246,290.12369019)
\curveto(864.84802021,290.08368679)(864.85302021,290.04868683)(864.86302246,290.01869019)
\curveto(864.87302019,289.98868689)(864.87802018,289.95368692)(864.87802246,289.91369019)
\curveto(864.87802018,289.88368699)(864.88302018,289.85368702)(864.89302246,289.82369019)
\curveto(864.91302015,289.74368713)(864.92802013,289.66368721)(864.93802246,289.58369019)
\lineto(864.99802246,289.34369019)
\curveto(865.10801995,289.00368787)(865.2580198,288.71368816)(865.44802246,288.47369019)
\curveto(865.64801941,288.23368864)(865.89301917,288.03368884)(866.18302246,287.87369019)
\curveto(866.27301879,287.82368905)(866.36801869,287.78368909)(866.46802246,287.75369019)
\curveto(866.56801849,287.73368914)(866.67301839,287.70868917)(866.78302246,287.67869019)
\curveto(866.83301823,287.65868922)(866.87801818,287.64868923)(866.91802246,287.64869019)
\curveto(866.96801809,287.65868922)(867.01801804,287.65868922)(867.06802246,287.64869019)
\curveto(867.10801795,287.63868924)(867.15301791,287.63368924)(867.20302246,287.63369019)
\lineto(867.33802246,287.63369019)
\lineto(867.47302246,287.63369019)
\curveto(867.51301755,287.64368923)(867.54801751,287.64868923)(867.57802246,287.64869019)
\curveto(867.60801745,287.64868923)(867.64301742,287.65368922)(867.68302246,287.66369019)
\curveto(867.7630173,287.68368919)(867.83801722,287.69868918)(867.90802246,287.70869019)
\curveto(867.97801708,287.72868915)(868.05301701,287.75368912)(868.13302246,287.78369019)
\curveto(868.44301662,287.91368896)(868.69301637,288.08368879)(868.88302246,288.29369019)
\curveto(869.07301599,288.51368836)(869.23301583,288.7786881)(869.36302246,289.08869019)
\curveto(869.41301565,289.22868765)(869.44801561,289.36868751)(869.46802246,289.50869019)
\curveto(869.49801556,289.65868722)(869.53301553,289.80868707)(869.57302246,289.95869019)
\curveto(869.59301547,290.00868687)(869.59801546,290.05368682)(869.58802246,290.09369019)
\curveto(869.58801547,290.14368673)(869.59301547,290.19368668)(869.60302246,290.24369019)
\lineto(869.60302246,290.34869019)
}
}
{
\newrgbcolor{curcolor}{0 0 0}
\pscustom[linestyle=none,fillstyle=solid,fillcolor=curcolor]
{
\newpath
\moveto(876.06427246,294.60869019)
\curveto(876.29426767,294.60868227)(876.42426754,294.54868233)(876.45427246,294.42869019)
\curveto(876.48426748,294.31868256)(876.49926747,294.15368272)(876.49927246,293.93369019)
\lineto(876.49927246,293.64869019)
\curveto(876.49926747,293.55868332)(876.47426749,293.48368339)(876.42427246,293.42369019)
\curveto(876.3642676,293.34368353)(876.27926769,293.29868358)(876.16927246,293.28869019)
\curveto(876.05926791,293.28868359)(875.94926802,293.2736836)(875.83927246,293.24369019)
\curveto(875.69926827,293.21368366)(875.5642684,293.18368369)(875.43427246,293.15369019)
\curveto(875.31426865,293.12368375)(875.19926877,293.08368379)(875.08927246,293.03369019)
\curveto(874.79926917,292.90368397)(874.5642694,292.72368415)(874.38427246,292.49369019)
\curveto(874.20426976,292.2736846)(874.04926992,292.01868486)(873.91927246,291.72869019)
\curveto(873.87927009,291.61868526)(873.84927012,291.50368537)(873.82927246,291.38369019)
\curveto(873.80927016,291.2736856)(873.78427018,291.15868572)(873.75427246,291.03869019)
\curveto(873.74427022,290.98868589)(873.73927023,290.93868594)(873.73927246,290.88869019)
\curveto(873.74927022,290.83868604)(873.74927022,290.78868609)(873.73927246,290.73869019)
\curveto(873.70927026,290.61868626)(873.69427027,290.4786864)(873.69427246,290.31869019)
\curveto(873.70427026,290.16868671)(873.70927026,290.02368685)(873.70927246,289.88369019)
\lineto(873.70927246,288.03869019)
\lineto(873.70927246,287.69369019)
\curveto(873.70927026,287.5736893)(873.70427026,287.45868942)(873.69427246,287.34869019)
\curveto(873.68427028,287.23868964)(873.67927029,287.14368973)(873.67927246,287.06369019)
\curveto(873.68927028,286.98368989)(873.6692703,286.91368996)(873.61927246,286.85369019)
\curveto(873.5692704,286.78369009)(873.48927048,286.74369013)(873.37927246,286.73369019)
\curveto(873.27927069,286.72369015)(873.1692708,286.71869016)(873.04927246,286.71869019)
\lineto(872.77927246,286.71869019)
\curveto(872.72927124,286.73869014)(872.67927129,286.75369012)(872.62927246,286.76369019)
\curveto(872.58927138,286.78369009)(872.55927141,286.80869007)(872.53927246,286.83869019)
\curveto(872.48927148,286.90868997)(872.45927151,286.99368988)(872.44927246,287.09369019)
\lineto(872.44927246,287.42369019)
\lineto(872.44927246,288.57869019)
\lineto(872.44927246,292.73369019)
\lineto(872.44927246,293.76869019)
\lineto(872.44927246,294.06869019)
\curveto(872.45927151,294.16868271)(872.48927148,294.25368262)(872.53927246,294.32369019)
\curveto(872.5692714,294.36368251)(872.61927135,294.39368248)(872.68927246,294.41369019)
\curveto(872.7692712,294.43368244)(872.85427111,294.44368243)(872.94427246,294.44369019)
\curveto(873.03427093,294.45368242)(873.12427084,294.45368242)(873.21427246,294.44369019)
\curveto(873.30427066,294.43368244)(873.37427059,294.41868246)(873.42427246,294.39869019)
\curveto(873.50427046,294.36868251)(873.55427041,294.30868257)(873.57427246,294.21869019)
\curveto(873.60427036,294.13868274)(873.61927035,294.04868283)(873.61927246,293.94869019)
\lineto(873.61927246,293.64869019)
\curveto(873.61927035,293.54868333)(873.63927033,293.45868342)(873.67927246,293.37869019)
\curveto(873.68927028,293.35868352)(873.69927027,293.34368353)(873.70927246,293.33369019)
\lineto(873.75427246,293.28869019)
\curveto(873.8642701,293.28868359)(873.95427001,293.33368354)(874.02427246,293.42369019)
\curveto(874.09426987,293.52368335)(874.15426981,293.60368327)(874.20427246,293.66369019)
\lineto(874.29427246,293.75369019)
\curveto(874.38426958,293.86368301)(874.50926946,293.9786829)(874.66927246,294.09869019)
\curveto(874.82926914,294.21868266)(874.97926899,294.30868257)(875.11927246,294.36869019)
\curveto(875.20926876,294.41868246)(875.30426866,294.45368242)(875.40427246,294.47369019)
\curveto(875.50426846,294.50368237)(875.60926836,294.53368234)(875.71927246,294.56369019)
\curveto(875.77926819,294.5736823)(875.83926813,294.5786823)(875.89927246,294.57869019)
\curveto(875.95926801,294.58868229)(876.01426795,294.59868228)(876.06427246,294.60869019)
}
}
{
\newrgbcolor{curcolor}{0 0 0}
\pscustom[linestyle=none,fillstyle=solid,fillcolor=curcolor]
{
\newpath
\moveto(884.17903809,290.87369019)
\curveto(884.1990304,290.7736861)(884.1990304,290.65868622)(884.17903809,290.52869019)
\curveto(884.16903043,290.40868647)(884.13903046,290.32368655)(884.08903809,290.27369019)
\curveto(884.03903056,290.23368664)(883.96403064,290.20368667)(883.86403809,290.18369019)
\curveto(883.77403083,290.1736867)(883.66903093,290.16868671)(883.54903809,290.16869019)
\lineto(883.18903809,290.16869019)
\curveto(883.06903153,290.1786867)(882.96403164,290.18368669)(882.87403809,290.18369019)
\lineto(879.03403809,290.18369019)
\curveto(878.95403565,290.18368669)(878.87403573,290.1786867)(878.79403809,290.16869019)
\curveto(878.71403589,290.16868671)(878.64903595,290.15368672)(878.59903809,290.12369019)
\curveto(878.55903604,290.10368677)(878.51903608,290.06368681)(878.47903809,290.00369019)
\curveto(878.45903614,289.9736869)(878.43903616,289.92868695)(878.41903809,289.86869019)
\curveto(878.3990362,289.81868706)(878.3990362,289.76868711)(878.41903809,289.71869019)
\curveto(878.42903617,289.66868721)(878.43403617,289.62368725)(878.43403809,289.58369019)
\curveto(878.43403617,289.54368733)(878.43903616,289.50368737)(878.44903809,289.46369019)
\curveto(878.46903613,289.38368749)(878.48903611,289.29868758)(878.50903809,289.20869019)
\curveto(878.52903607,289.12868775)(878.55903604,289.04868783)(878.59903809,288.96869019)
\curveto(878.82903577,288.42868845)(879.20903539,288.04368883)(879.73903809,287.81369019)
\curveto(879.7990348,287.78368909)(879.86403474,287.75868912)(879.93403809,287.73869019)
\lineto(880.14403809,287.67869019)
\curveto(880.17403443,287.66868921)(880.22403438,287.66368921)(880.29403809,287.66369019)
\curveto(880.43403417,287.62368925)(880.61903398,287.60368927)(880.84903809,287.60369019)
\curveto(881.07903352,287.60368927)(881.26403334,287.62368925)(881.40403809,287.66369019)
\curveto(881.54403306,287.70368917)(881.66903293,287.74368913)(881.77903809,287.78369019)
\curveto(881.8990327,287.83368904)(882.00903259,287.89368898)(882.10903809,287.96369019)
\curveto(882.21903238,288.03368884)(882.31403229,288.11368876)(882.39403809,288.20369019)
\curveto(882.47403213,288.30368857)(882.54403206,288.40868847)(882.60403809,288.51869019)
\curveto(882.66403194,288.61868826)(882.71403189,288.72368815)(882.75403809,288.83369019)
\curveto(882.8040318,288.94368793)(882.88403172,289.02368785)(882.99403809,289.07369019)
\curveto(883.03403157,289.09368778)(883.0990315,289.10868777)(883.18903809,289.11869019)
\curveto(883.27903132,289.12868775)(883.36903123,289.12868775)(883.45903809,289.11869019)
\curveto(883.54903105,289.11868776)(883.63403097,289.11368776)(883.71403809,289.10369019)
\curveto(883.79403081,289.09368778)(883.84903075,289.0736878)(883.87903809,289.04369019)
\curveto(883.97903062,288.9736879)(884.0040306,288.85868802)(883.95403809,288.69869019)
\curveto(883.87403073,288.42868845)(883.76903083,288.18868869)(883.63903809,287.97869019)
\curveto(883.43903116,287.65868922)(883.20903139,287.39368948)(882.94903809,287.18369019)
\curveto(882.6990319,286.98368989)(882.37903222,286.81869006)(881.98903809,286.68869019)
\curveto(881.88903271,286.64869023)(881.78903281,286.62369025)(881.68903809,286.61369019)
\curveto(881.58903301,286.59369028)(881.48403312,286.5736903)(881.37403809,286.55369019)
\curveto(881.32403328,286.54369033)(881.27403333,286.53869034)(881.22403809,286.53869019)
\curveto(881.18403342,286.53869034)(881.13903346,286.53369034)(881.08903809,286.52369019)
\lineto(880.93903809,286.52369019)
\curveto(880.88903371,286.51369036)(880.82903377,286.50869037)(880.75903809,286.50869019)
\curveto(880.6990339,286.50869037)(880.64903395,286.51369036)(880.60903809,286.52369019)
\lineto(880.47403809,286.52369019)
\curveto(880.42403418,286.53369034)(880.37903422,286.53869034)(880.33903809,286.53869019)
\curveto(880.2990343,286.53869034)(880.25903434,286.54369033)(880.21903809,286.55369019)
\curveto(880.16903443,286.56369031)(880.11403449,286.5736903)(880.05403809,286.58369019)
\curveto(879.99403461,286.58369029)(879.93903466,286.58869029)(879.88903809,286.59869019)
\curveto(879.7990348,286.61869026)(879.70903489,286.64369023)(879.61903809,286.67369019)
\curveto(879.52903507,286.69369018)(879.44403516,286.71869016)(879.36403809,286.74869019)
\curveto(879.32403528,286.76869011)(879.28903531,286.7786901)(879.25903809,286.77869019)
\curveto(879.22903537,286.78869009)(879.19403541,286.80369007)(879.15403809,286.82369019)
\curveto(879.0040356,286.89368998)(878.84403576,286.9786899)(878.67403809,287.07869019)
\curveto(878.38403622,287.26868961)(878.13403647,287.49868938)(877.92403809,287.76869019)
\curveto(877.72403688,288.04868883)(877.55403705,288.35868852)(877.41403809,288.69869019)
\curveto(877.36403724,288.80868807)(877.32403728,288.92368795)(877.29403809,289.04369019)
\curveto(877.27403733,289.16368771)(877.24403736,289.28368759)(877.20403809,289.40369019)
\curveto(877.19403741,289.44368743)(877.18903741,289.4786874)(877.18903809,289.50869019)
\curveto(877.18903741,289.53868734)(877.18403742,289.5786873)(877.17403809,289.62869019)
\curveto(877.15403745,289.70868717)(877.13903746,289.79368708)(877.12903809,289.88369019)
\curveto(877.11903748,289.9736869)(877.1040375,290.06368681)(877.08403809,290.15369019)
\lineto(877.08403809,290.36369019)
\curveto(877.07403753,290.40368647)(877.06403754,290.45868642)(877.05403809,290.52869019)
\curveto(877.05403755,290.60868627)(877.05903754,290.6736862)(877.06903809,290.72369019)
\lineto(877.06903809,290.88869019)
\curveto(877.08903751,290.93868594)(877.09403751,290.98868589)(877.08403809,291.03869019)
\curveto(877.08403752,291.09868578)(877.08903751,291.15368572)(877.09903809,291.20369019)
\curveto(877.13903746,291.36368551)(877.16903743,291.52368535)(877.18903809,291.68369019)
\curveto(877.21903738,291.84368503)(877.26403734,291.99368488)(877.32403809,292.13369019)
\curveto(877.37403723,292.24368463)(877.41903718,292.35368452)(877.45903809,292.46369019)
\curveto(877.50903709,292.58368429)(877.56403704,292.69868418)(877.62403809,292.80869019)
\curveto(877.84403676,293.15868372)(878.09403651,293.45868342)(878.37403809,293.70869019)
\curveto(878.65403595,293.96868291)(878.9990356,294.18368269)(879.40903809,294.35369019)
\curveto(879.52903507,294.40368247)(879.64903495,294.43868244)(879.76903809,294.45869019)
\curveto(879.8990347,294.48868239)(880.03403457,294.51868236)(880.17403809,294.54869019)
\curveto(880.22403438,294.55868232)(880.26903433,294.56368231)(880.30903809,294.56369019)
\curveto(880.34903425,294.5736823)(880.39403421,294.5786823)(880.44403809,294.57869019)
\curveto(880.46403414,294.58868229)(880.48903411,294.58868229)(880.51903809,294.57869019)
\curveto(880.54903405,294.56868231)(880.57403403,294.5736823)(880.59403809,294.59369019)
\curveto(881.01403359,294.60368227)(881.37903322,294.55868232)(881.68903809,294.45869019)
\curveto(881.9990326,294.36868251)(882.27903232,294.24368263)(882.52903809,294.08369019)
\curveto(882.57903202,294.06368281)(882.61903198,294.03368284)(882.64903809,293.99369019)
\curveto(882.67903192,293.96368291)(882.71403189,293.93868294)(882.75403809,293.91869019)
\curveto(882.83403177,293.85868302)(882.91403169,293.78868309)(882.99403809,293.70869019)
\curveto(883.08403152,293.62868325)(883.15903144,293.54868333)(883.21903809,293.46869019)
\curveto(883.37903122,293.25868362)(883.51403109,293.05868382)(883.62403809,292.86869019)
\curveto(883.69403091,292.75868412)(883.74903085,292.63868424)(883.78903809,292.50869019)
\curveto(883.82903077,292.3786845)(883.87403073,292.24868463)(883.92403809,292.11869019)
\curveto(883.97403063,291.98868489)(884.00903059,291.85368502)(884.02903809,291.71369019)
\curveto(884.05903054,291.5736853)(884.09403051,291.43368544)(884.13403809,291.29369019)
\curveto(884.14403046,291.22368565)(884.14903045,291.15368572)(884.14903809,291.08369019)
\lineto(884.17903809,290.87369019)
\moveto(882.72403809,291.38369019)
\curveto(882.75403185,291.42368545)(882.77903182,291.4736854)(882.79903809,291.53369019)
\curveto(882.81903178,291.60368527)(882.81903178,291.6736852)(882.79903809,291.74369019)
\curveto(882.73903186,291.96368491)(882.65403195,292.16868471)(882.54403809,292.35869019)
\curveto(882.4040322,292.58868429)(882.24903235,292.78368409)(882.07903809,292.94369019)
\curveto(881.90903269,293.10368377)(881.68903291,293.23868364)(881.41903809,293.34869019)
\curveto(881.34903325,293.36868351)(881.27903332,293.38368349)(881.20903809,293.39369019)
\curveto(881.13903346,293.41368346)(881.06403354,293.43368344)(880.98403809,293.45369019)
\curveto(880.9040337,293.4736834)(880.81903378,293.48368339)(880.72903809,293.48369019)
\lineto(880.47403809,293.48369019)
\curveto(880.44403416,293.46368341)(880.40903419,293.45368342)(880.36903809,293.45369019)
\curveto(880.32903427,293.46368341)(880.29403431,293.46368341)(880.26403809,293.45369019)
\lineto(880.02403809,293.39369019)
\curveto(879.95403465,293.38368349)(879.88403472,293.36868351)(879.81403809,293.34869019)
\curveto(879.52403508,293.22868365)(879.28903531,293.0786838)(879.10903809,292.89869019)
\curveto(878.93903566,292.71868416)(878.78403582,292.49368438)(878.64403809,292.22369019)
\curveto(878.61403599,292.1736847)(878.58403602,292.10868477)(878.55403809,292.02869019)
\curveto(878.52403608,291.95868492)(878.4990361,291.878685)(878.47903809,291.78869019)
\curveto(878.45903614,291.69868518)(878.45403615,291.61368526)(878.46403809,291.53369019)
\curveto(878.47403613,291.45368542)(878.50903609,291.39368548)(878.56903809,291.35369019)
\curveto(878.64903595,291.29368558)(878.78403582,291.26368561)(878.97403809,291.26369019)
\curveto(879.17403543,291.2736856)(879.34403526,291.2786856)(879.48403809,291.27869019)
\lineto(881.76403809,291.27869019)
\curveto(881.91403269,291.2786856)(882.09403251,291.2736856)(882.30403809,291.26369019)
\curveto(882.51403209,291.26368561)(882.65403195,291.30368557)(882.72403809,291.38369019)
}
}
{
\newrgbcolor{curcolor}{0 0 0}
\pscustom[linestyle=none,fillstyle=solid,fillcolor=curcolor]
{
}
}
{
\newrgbcolor{curcolor}{0 0 0}
\pscustom[linestyle=none,fillstyle=solid,fillcolor=curcolor]
{
\newpath
\moveto(896.60583496,287.51369019)
\lineto(896.60583496,287.12369019)
\curveto(896.60582709,287.00368987)(896.58082711,286.90368997)(896.53083496,286.82369019)
\curveto(896.48082721,286.75369012)(896.3958273,286.71369016)(896.27583496,286.70369019)
\lineto(895.93083496,286.70369019)
\curveto(895.87082782,286.70369017)(895.81082788,286.69869018)(895.75083496,286.68869019)
\curveto(895.70082799,286.68869019)(895.65582804,286.69869018)(895.61583496,286.71869019)
\curveto(895.52582817,286.73869014)(895.46582823,286.7786901)(895.43583496,286.83869019)
\curveto(895.3958283,286.88868999)(895.37082832,286.94868993)(895.36083496,287.01869019)
\curveto(895.36082833,287.08868979)(895.34582835,287.15868972)(895.31583496,287.22869019)
\curveto(895.30582839,287.24868963)(895.2908284,287.26368961)(895.27083496,287.27369019)
\curveto(895.26082843,287.29368958)(895.24582845,287.31368956)(895.22583496,287.33369019)
\curveto(895.12582857,287.34368953)(895.04582865,287.32368955)(894.98583496,287.27369019)
\curveto(894.93582876,287.22368965)(894.88082881,287.1736897)(894.82083496,287.12369019)
\curveto(894.62082907,286.9736899)(894.42082927,286.85869002)(894.22083496,286.77869019)
\curveto(894.04082965,286.69869018)(893.83082986,286.63869024)(893.59083496,286.59869019)
\curveto(893.36083033,286.55869032)(893.12083057,286.53869034)(892.87083496,286.53869019)
\curveto(892.63083106,286.52869035)(892.3908313,286.54369033)(892.15083496,286.58369019)
\curveto(891.91083178,286.61369026)(891.70083199,286.66869021)(891.52083496,286.74869019)
\curveto(891.00083269,286.96868991)(890.58083311,287.26368961)(890.26083496,287.63369019)
\curveto(889.94083375,288.01368886)(889.690834,288.48368839)(889.51083496,289.04369019)
\curveto(889.47083422,289.13368774)(889.44083425,289.22368765)(889.42083496,289.31369019)
\curveto(889.41083428,289.41368746)(889.3908343,289.51368736)(889.36083496,289.61369019)
\curveto(889.35083434,289.66368721)(889.34583435,289.71368716)(889.34583496,289.76369019)
\curveto(889.34583435,289.81368706)(889.34083435,289.86368701)(889.33083496,289.91369019)
\curveto(889.31083438,289.96368691)(889.30083439,290.01368686)(889.30083496,290.06369019)
\curveto(889.31083438,290.12368675)(889.31083438,290.1786867)(889.30083496,290.22869019)
\lineto(889.30083496,290.37869019)
\curveto(889.28083441,290.42868645)(889.27083442,290.49368638)(889.27083496,290.57369019)
\curveto(889.27083442,290.65368622)(889.28083441,290.71868616)(889.30083496,290.76869019)
\lineto(889.30083496,290.93369019)
\curveto(889.32083437,291.00368587)(889.32583437,291.0736858)(889.31583496,291.14369019)
\curveto(889.31583438,291.22368565)(889.32583437,291.29868558)(889.34583496,291.36869019)
\curveto(889.35583434,291.41868546)(889.36083433,291.46368541)(889.36083496,291.50369019)
\curveto(889.36083433,291.54368533)(889.36583433,291.58868529)(889.37583496,291.63869019)
\curveto(889.40583429,291.73868514)(889.43083426,291.83368504)(889.45083496,291.92369019)
\curveto(889.47083422,292.02368485)(889.4958342,292.11868476)(889.52583496,292.20869019)
\curveto(889.65583404,292.58868429)(889.82083387,292.92868395)(890.02083496,293.22869019)
\curveto(890.23083346,293.53868334)(890.48083321,293.79368308)(890.77083496,293.99369019)
\curveto(890.94083275,294.11368276)(891.11583258,294.21368266)(891.29583496,294.29369019)
\curveto(891.48583221,294.3736825)(891.690832,294.44368243)(891.91083496,294.50369019)
\curveto(891.98083171,294.51368236)(892.04583165,294.52368235)(892.10583496,294.53369019)
\curveto(892.17583152,294.54368233)(892.24583145,294.55868232)(892.31583496,294.57869019)
\lineto(892.46583496,294.57869019)
\curveto(892.54583115,294.59868228)(892.66083103,294.60868227)(892.81083496,294.60869019)
\curveto(892.97083072,294.60868227)(893.0908306,294.59868228)(893.17083496,294.57869019)
\curveto(893.21083048,294.56868231)(893.26583043,294.56368231)(893.33583496,294.56369019)
\curveto(893.44583025,294.53368234)(893.55583014,294.50868237)(893.66583496,294.48869019)
\curveto(893.77582992,294.4786824)(893.88082981,294.44868243)(893.98083496,294.39869019)
\curveto(894.13082956,294.33868254)(894.27082942,294.2736826)(894.40083496,294.20369019)
\curveto(894.54082915,294.13368274)(894.67082902,294.05368282)(894.79083496,293.96369019)
\curveto(894.85082884,293.91368296)(894.91082878,293.85868302)(894.97083496,293.79869019)
\curveto(895.04082865,293.74868313)(895.13082856,293.73368314)(895.24083496,293.75369019)
\curveto(895.26082843,293.78368309)(895.27582842,293.80868307)(895.28583496,293.82869019)
\curveto(895.30582839,293.84868303)(895.32082837,293.878683)(895.33083496,293.91869019)
\curveto(895.36082833,294.00868287)(895.37082832,294.12368275)(895.36083496,294.26369019)
\lineto(895.36083496,294.63869019)
\lineto(895.36083496,296.36369019)
\lineto(895.36083496,296.82869019)
\curveto(895.36082833,297.00867987)(895.38582831,297.13867974)(895.43583496,297.21869019)
\curveto(895.47582822,297.28867959)(895.53582816,297.33367954)(895.61583496,297.35369019)
\curveto(895.63582806,297.35367952)(895.66082803,297.35367952)(895.69083496,297.35369019)
\curveto(895.72082797,297.36367951)(895.74582795,297.36867951)(895.76583496,297.36869019)
\curveto(895.90582779,297.3786795)(896.05082764,297.3786795)(896.20083496,297.36869019)
\curveto(896.36082733,297.36867951)(896.47082722,297.32867955)(896.53083496,297.24869019)
\curveto(896.58082711,297.16867971)(896.60582709,297.06867981)(896.60583496,296.94869019)
\lineto(896.60583496,296.57369019)
\lineto(896.60583496,287.51369019)
\moveto(895.39083496,290.34869019)
\curveto(895.41082828,290.39868648)(895.42082827,290.46368641)(895.42083496,290.54369019)
\curveto(895.42082827,290.63368624)(895.41082828,290.70368617)(895.39083496,290.75369019)
\lineto(895.39083496,290.97869019)
\curveto(895.37082832,291.06868581)(895.35582834,291.15868572)(895.34583496,291.24869019)
\curveto(895.33582836,291.34868553)(895.31582838,291.43868544)(895.28583496,291.51869019)
\curveto(895.26582843,291.59868528)(895.24582845,291.6736852)(895.22583496,291.74369019)
\curveto(895.21582848,291.81368506)(895.1958285,291.88368499)(895.16583496,291.95369019)
\curveto(895.04582865,292.25368462)(894.8908288,292.51868436)(894.70083496,292.74869019)
\curveto(894.51082918,292.9786839)(894.27082942,293.15868372)(893.98083496,293.28869019)
\curveto(893.88082981,293.33868354)(893.77582992,293.3736835)(893.66583496,293.39369019)
\curveto(893.56583013,293.42368345)(893.45583024,293.44868343)(893.33583496,293.46869019)
\curveto(893.25583044,293.48868339)(893.16583053,293.49868338)(893.06583496,293.49869019)
\lineto(892.79583496,293.49869019)
\curveto(892.74583095,293.48868339)(892.70083099,293.4786834)(892.66083496,293.46869019)
\lineto(892.52583496,293.46869019)
\curveto(892.44583125,293.44868343)(892.36083133,293.42868345)(892.27083496,293.40869019)
\curveto(892.1908315,293.38868349)(892.11083158,293.36368351)(892.03083496,293.33369019)
\curveto(891.71083198,293.19368368)(891.45083224,292.98868389)(891.25083496,292.71869019)
\curveto(891.06083263,292.45868442)(890.90583279,292.15368472)(890.78583496,291.80369019)
\curveto(890.74583295,291.69368518)(890.71583298,291.5786853)(890.69583496,291.45869019)
\curveto(890.68583301,291.34868553)(890.67083302,291.23868564)(890.65083496,291.12869019)
\curveto(890.65083304,291.08868579)(890.64583305,291.04868583)(890.63583496,291.00869019)
\lineto(890.63583496,290.90369019)
\curveto(890.61583308,290.85368602)(890.60583309,290.79868608)(890.60583496,290.73869019)
\curveto(890.61583308,290.6786862)(890.62083307,290.62368625)(890.62083496,290.57369019)
\lineto(890.62083496,290.24369019)
\curveto(890.62083307,290.14368673)(890.63083306,290.04868683)(890.65083496,289.95869019)
\curveto(890.66083303,289.92868695)(890.66583303,289.878687)(890.66583496,289.80869019)
\curveto(890.68583301,289.73868714)(890.70083299,289.66868721)(890.71083496,289.59869019)
\lineto(890.77083496,289.38869019)
\curveto(890.88083281,289.03868784)(891.03083266,288.73868814)(891.22083496,288.48869019)
\curveto(891.41083228,288.23868864)(891.65083204,288.03368884)(891.94083496,287.87369019)
\curveto(892.03083166,287.82368905)(892.12083157,287.78368909)(892.21083496,287.75369019)
\curveto(892.30083139,287.72368915)(892.40083129,287.69368918)(892.51083496,287.66369019)
\curveto(892.56083113,287.64368923)(892.61083108,287.63868924)(892.66083496,287.64869019)
\curveto(892.72083097,287.65868922)(892.77583092,287.65368922)(892.82583496,287.63369019)
\curveto(892.86583083,287.62368925)(892.90583079,287.61868926)(892.94583496,287.61869019)
\lineto(893.08083496,287.61869019)
\lineto(893.21583496,287.61869019)
\curveto(893.24583045,287.62868925)(893.2958304,287.63368924)(893.36583496,287.63369019)
\curveto(893.44583025,287.65368922)(893.52583017,287.66868921)(893.60583496,287.67869019)
\curveto(893.68583001,287.69868918)(893.76082993,287.72368915)(893.83083496,287.75369019)
\curveto(894.16082953,287.89368898)(894.42582927,288.06868881)(894.62583496,288.27869019)
\curveto(894.83582886,288.49868838)(895.01082868,288.7736881)(895.15083496,289.10369019)
\curveto(895.20082849,289.21368766)(895.23582846,289.32368755)(895.25583496,289.43369019)
\curveto(895.27582842,289.54368733)(895.30082839,289.65368722)(895.33083496,289.76369019)
\curveto(895.35082834,289.80368707)(895.36082833,289.83868704)(895.36083496,289.86869019)
\curveto(895.36082833,289.90868697)(895.36582833,289.94868693)(895.37583496,289.98869019)
\curveto(895.38582831,290.04868683)(895.38582831,290.10868677)(895.37583496,290.16869019)
\curveto(895.37582832,290.22868665)(895.38082831,290.28868659)(895.39083496,290.34869019)
}
}
{
\newrgbcolor{curcolor}{0 0 0}
\pscustom[linestyle=none,fillstyle=solid,fillcolor=curcolor]
{
\newpath
\moveto(905.30208496,290.87369019)
\curveto(905.32207728,290.7736861)(905.32207728,290.65868622)(905.30208496,290.52869019)
\curveto(905.29207731,290.40868647)(905.26207734,290.32368655)(905.21208496,290.27369019)
\curveto(905.16207744,290.23368664)(905.08707751,290.20368667)(904.98708496,290.18369019)
\curveto(904.8970777,290.1736867)(904.79207781,290.16868671)(904.67208496,290.16869019)
\lineto(904.31208496,290.16869019)
\curveto(904.19207841,290.1786867)(904.08707851,290.18368669)(903.99708496,290.18369019)
\lineto(900.15708496,290.18369019)
\curveto(900.07708252,290.18368669)(899.9970826,290.1786867)(899.91708496,290.16869019)
\curveto(899.83708276,290.16868671)(899.77208283,290.15368672)(899.72208496,290.12369019)
\curveto(899.68208292,290.10368677)(899.64208296,290.06368681)(899.60208496,290.00369019)
\curveto(899.58208302,289.9736869)(899.56208304,289.92868695)(899.54208496,289.86869019)
\curveto(899.52208308,289.81868706)(899.52208308,289.76868711)(899.54208496,289.71869019)
\curveto(899.55208305,289.66868721)(899.55708304,289.62368725)(899.55708496,289.58369019)
\curveto(899.55708304,289.54368733)(899.56208304,289.50368737)(899.57208496,289.46369019)
\curveto(899.59208301,289.38368749)(899.61208299,289.29868758)(899.63208496,289.20869019)
\curveto(899.65208295,289.12868775)(899.68208292,289.04868783)(899.72208496,288.96869019)
\curveto(899.95208265,288.42868845)(900.33208227,288.04368883)(900.86208496,287.81369019)
\curveto(900.92208168,287.78368909)(900.98708161,287.75868912)(901.05708496,287.73869019)
\lineto(901.26708496,287.67869019)
\curveto(901.2970813,287.66868921)(901.34708125,287.66368921)(901.41708496,287.66369019)
\curveto(901.55708104,287.62368925)(901.74208086,287.60368927)(901.97208496,287.60369019)
\curveto(902.2020804,287.60368927)(902.38708021,287.62368925)(902.52708496,287.66369019)
\curveto(902.66707993,287.70368917)(902.79207981,287.74368913)(902.90208496,287.78369019)
\curveto(903.02207958,287.83368904)(903.13207947,287.89368898)(903.23208496,287.96369019)
\curveto(903.34207926,288.03368884)(903.43707916,288.11368876)(903.51708496,288.20369019)
\curveto(903.597079,288.30368857)(903.66707893,288.40868847)(903.72708496,288.51869019)
\curveto(903.78707881,288.61868826)(903.83707876,288.72368815)(903.87708496,288.83369019)
\curveto(903.92707867,288.94368793)(904.00707859,289.02368785)(904.11708496,289.07369019)
\curveto(904.15707844,289.09368778)(904.22207838,289.10868777)(904.31208496,289.11869019)
\curveto(904.4020782,289.12868775)(904.49207811,289.12868775)(904.58208496,289.11869019)
\curveto(904.67207793,289.11868776)(904.75707784,289.11368776)(904.83708496,289.10369019)
\curveto(904.91707768,289.09368778)(904.97207763,289.0736878)(905.00208496,289.04369019)
\curveto(905.1020775,288.9736879)(905.12707747,288.85868802)(905.07708496,288.69869019)
\curveto(904.9970776,288.42868845)(904.89207771,288.18868869)(904.76208496,287.97869019)
\curveto(904.56207804,287.65868922)(904.33207827,287.39368948)(904.07208496,287.18369019)
\curveto(903.82207878,286.98368989)(903.5020791,286.81869006)(903.11208496,286.68869019)
\curveto(903.01207959,286.64869023)(902.91207969,286.62369025)(902.81208496,286.61369019)
\curveto(902.71207989,286.59369028)(902.60707999,286.5736903)(902.49708496,286.55369019)
\curveto(902.44708015,286.54369033)(902.3970802,286.53869034)(902.34708496,286.53869019)
\curveto(902.30708029,286.53869034)(902.26208034,286.53369034)(902.21208496,286.52369019)
\lineto(902.06208496,286.52369019)
\curveto(902.01208059,286.51369036)(901.95208065,286.50869037)(901.88208496,286.50869019)
\curveto(901.82208078,286.50869037)(901.77208083,286.51369036)(901.73208496,286.52369019)
\lineto(901.59708496,286.52369019)
\curveto(901.54708105,286.53369034)(901.5020811,286.53869034)(901.46208496,286.53869019)
\curveto(901.42208118,286.53869034)(901.38208122,286.54369033)(901.34208496,286.55369019)
\curveto(901.29208131,286.56369031)(901.23708136,286.5736903)(901.17708496,286.58369019)
\curveto(901.11708148,286.58369029)(901.06208154,286.58869029)(901.01208496,286.59869019)
\curveto(900.92208168,286.61869026)(900.83208177,286.64369023)(900.74208496,286.67369019)
\curveto(900.65208195,286.69369018)(900.56708203,286.71869016)(900.48708496,286.74869019)
\curveto(900.44708215,286.76869011)(900.41208219,286.7786901)(900.38208496,286.77869019)
\curveto(900.35208225,286.78869009)(900.31708228,286.80369007)(900.27708496,286.82369019)
\curveto(900.12708247,286.89368998)(899.96708263,286.9786899)(899.79708496,287.07869019)
\curveto(899.50708309,287.26868961)(899.25708334,287.49868938)(899.04708496,287.76869019)
\curveto(898.84708375,288.04868883)(898.67708392,288.35868852)(898.53708496,288.69869019)
\curveto(898.48708411,288.80868807)(898.44708415,288.92368795)(898.41708496,289.04369019)
\curveto(898.3970842,289.16368771)(898.36708423,289.28368759)(898.32708496,289.40369019)
\curveto(898.31708428,289.44368743)(898.31208429,289.4786874)(898.31208496,289.50869019)
\curveto(898.31208429,289.53868734)(898.30708429,289.5786873)(898.29708496,289.62869019)
\curveto(898.27708432,289.70868717)(898.26208434,289.79368708)(898.25208496,289.88369019)
\curveto(898.24208436,289.9736869)(898.22708437,290.06368681)(898.20708496,290.15369019)
\lineto(898.20708496,290.36369019)
\curveto(898.1970844,290.40368647)(898.18708441,290.45868642)(898.17708496,290.52869019)
\curveto(898.17708442,290.60868627)(898.18208442,290.6736862)(898.19208496,290.72369019)
\lineto(898.19208496,290.88869019)
\curveto(898.21208439,290.93868594)(898.21708438,290.98868589)(898.20708496,291.03869019)
\curveto(898.20708439,291.09868578)(898.21208439,291.15368572)(898.22208496,291.20369019)
\curveto(898.26208434,291.36368551)(898.29208431,291.52368535)(898.31208496,291.68369019)
\curveto(898.34208426,291.84368503)(898.38708421,291.99368488)(898.44708496,292.13369019)
\curveto(898.4970841,292.24368463)(898.54208406,292.35368452)(898.58208496,292.46369019)
\curveto(898.63208397,292.58368429)(898.68708391,292.69868418)(898.74708496,292.80869019)
\curveto(898.96708363,293.15868372)(899.21708338,293.45868342)(899.49708496,293.70869019)
\curveto(899.77708282,293.96868291)(900.12208248,294.18368269)(900.53208496,294.35369019)
\curveto(900.65208195,294.40368247)(900.77208183,294.43868244)(900.89208496,294.45869019)
\curveto(901.02208158,294.48868239)(901.15708144,294.51868236)(901.29708496,294.54869019)
\curveto(901.34708125,294.55868232)(901.39208121,294.56368231)(901.43208496,294.56369019)
\curveto(901.47208113,294.5736823)(901.51708108,294.5786823)(901.56708496,294.57869019)
\curveto(901.58708101,294.58868229)(901.61208099,294.58868229)(901.64208496,294.57869019)
\curveto(901.67208093,294.56868231)(901.6970809,294.5736823)(901.71708496,294.59369019)
\curveto(902.13708046,294.60368227)(902.5020801,294.55868232)(902.81208496,294.45869019)
\curveto(903.12207948,294.36868251)(903.4020792,294.24368263)(903.65208496,294.08369019)
\curveto(903.7020789,294.06368281)(903.74207886,294.03368284)(903.77208496,293.99369019)
\curveto(903.8020788,293.96368291)(903.83707876,293.93868294)(903.87708496,293.91869019)
\curveto(903.95707864,293.85868302)(904.03707856,293.78868309)(904.11708496,293.70869019)
\curveto(904.20707839,293.62868325)(904.28207832,293.54868333)(904.34208496,293.46869019)
\curveto(904.5020781,293.25868362)(904.63707796,293.05868382)(904.74708496,292.86869019)
\curveto(904.81707778,292.75868412)(904.87207773,292.63868424)(904.91208496,292.50869019)
\curveto(904.95207765,292.3786845)(904.9970776,292.24868463)(905.04708496,292.11869019)
\curveto(905.0970775,291.98868489)(905.13207747,291.85368502)(905.15208496,291.71369019)
\curveto(905.18207742,291.5736853)(905.21707738,291.43368544)(905.25708496,291.29369019)
\curveto(905.26707733,291.22368565)(905.27207733,291.15368572)(905.27208496,291.08369019)
\lineto(905.30208496,290.87369019)
\moveto(903.84708496,291.38369019)
\curveto(903.87707872,291.42368545)(903.9020787,291.4736854)(903.92208496,291.53369019)
\curveto(903.94207866,291.60368527)(903.94207866,291.6736852)(903.92208496,291.74369019)
\curveto(903.86207874,291.96368491)(903.77707882,292.16868471)(903.66708496,292.35869019)
\curveto(903.52707907,292.58868429)(903.37207923,292.78368409)(903.20208496,292.94369019)
\curveto(903.03207957,293.10368377)(902.81207979,293.23868364)(902.54208496,293.34869019)
\curveto(902.47208013,293.36868351)(902.4020802,293.38368349)(902.33208496,293.39369019)
\curveto(902.26208034,293.41368346)(902.18708041,293.43368344)(902.10708496,293.45369019)
\curveto(902.02708057,293.4736834)(901.94208066,293.48368339)(901.85208496,293.48369019)
\lineto(901.59708496,293.48369019)
\curveto(901.56708103,293.46368341)(901.53208107,293.45368342)(901.49208496,293.45369019)
\curveto(901.45208115,293.46368341)(901.41708118,293.46368341)(901.38708496,293.45369019)
\lineto(901.14708496,293.39369019)
\curveto(901.07708152,293.38368349)(901.00708159,293.36868351)(900.93708496,293.34869019)
\curveto(900.64708195,293.22868365)(900.41208219,293.0786838)(900.23208496,292.89869019)
\curveto(900.06208254,292.71868416)(899.90708269,292.49368438)(899.76708496,292.22369019)
\curveto(899.73708286,292.1736847)(899.70708289,292.10868477)(899.67708496,292.02869019)
\curveto(899.64708295,291.95868492)(899.62208298,291.878685)(899.60208496,291.78869019)
\curveto(899.58208302,291.69868518)(899.57708302,291.61368526)(899.58708496,291.53369019)
\curveto(899.597083,291.45368542)(899.63208297,291.39368548)(899.69208496,291.35369019)
\curveto(899.77208283,291.29368558)(899.90708269,291.26368561)(900.09708496,291.26369019)
\curveto(900.2970823,291.2736856)(900.46708213,291.2786856)(900.60708496,291.27869019)
\lineto(902.88708496,291.27869019)
\curveto(903.03707956,291.2786856)(903.21707938,291.2736856)(903.42708496,291.26369019)
\curveto(903.63707896,291.26368561)(903.77707882,291.30368557)(903.84708496,291.38369019)
}
}
{
\newrgbcolor{curcolor}{0 0 0}
\pscustom[linestyle=none,fillstyle=solid,fillcolor=curcolor]
{
\newpath
\moveto(838.66646484,314.23541504)
\curveto(838.68645716,314.13541095)(838.68645716,314.02041107)(838.66646484,313.89041504)
\curveto(838.65645719,313.77041132)(838.62645722,313.6854114)(838.57646484,313.63541504)
\curveto(838.52645732,313.59541149)(838.45145739,313.56541152)(838.35146484,313.54541504)
\curveto(838.26145758,313.53541155)(838.15645769,313.53041156)(838.03646484,313.53041504)
\lineto(837.67646484,313.53041504)
\curveto(837.55645829,313.54041155)(837.45145839,313.54541154)(837.36146484,313.54541504)
\lineto(833.52146484,313.54541504)
\curveto(833.4414624,313.54541154)(833.36146248,313.54041155)(833.28146484,313.53041504)
\curveto(833.20146264,313.53041156)(833.13646271,313.51541157)(833.08646484,313.48541504)
\curveto(833.0464628,313.46541162)(833.00646284,313.42541166)(832.96646484,313.36541504)
\curveto(832.9464629,313.33541175)(832.92646292,313.2904118)(832.90646484,313.23041504)
\curveto(832.88646296,313.18041191)(832.88646296,313.13041196)(832.90646484,313.08041504)
\curveto(832.91646293,313.03041206)(832.92146292,312.9854121)(832.92146484,312.94541504)
\curveto(832.92146292,312.90541218)(832.92646292,312.86541222)(832.93646484,312.82541504)
\curveto(832.95646289,312.74541234)(832.97646287,312.66041243)(832.99646484,312.57041504)
\curveto(833.01646283,312.4904126)(833.0464628,312.41041268)(833.08646484,312.33041504)
\curveto(833.31646253,311.7904133)(833.69646215,311.40541368)(834.22646484,311.17541504)
\curveto(834.28646156,311.14541394)(834.35146149,311.12041397)(834.42146484,311.10041504)
\lineto(834.63146484,311.04041504)
\curveto(834.66146118,311.03041406)(834.71146113,311.02541406)(834.78146484,311.02541504)
\curveto(834.92146092,310.9854141)(835.10646074,310.96541412)(835.33646484,310.96541504)
\curveto(835.56646028,310.96541412)(835.75146009,310.9854141)(835.89146484,311.02541504)
\curveto(836.03145981,311.06541402)(836.15645969,311.10541398)(836.26646484,311.14541504)
\curveto(836.38645946,311.19541389)(836.49645935,311.25541383)(836.59646484,311.32541504)
\curveto(836.70645914,311.39541369)(836.80145904,311.47541361)(836.88146484,311.56541504)
\curveto(836.96145888,311.66541342)(837.03145881,311.77041332)(837.09146484,311.88041504)
\curveto(837.15145869,311.98041311)(837.20145864,312.085413)(837.24146484,312.19541504)
\curveto(837.29145855,312.30541278)(837.37145847,312.3854127)(837.48146484,312.43541504)
\curveto(837.52145832,312.45541263)(837.58645826,312.47041262)(837.67646484,312.48041504)
\curveto(837.76645808,312.4904126)(837.85645799,312.4904126)(837.94646484,312.48041504)
\curveto(838.03645781,312.48041261)(838.12145772,312.47541261)(838.20146484,312.46541504)
\curveto(838.28145756,312.45541263)(838.33645751,312.43541265)(838.36646484,312.40541504)
\curveto(838.46645738,312.33541275)(838.49145735,312.22041287)(838.44146484,312.06041504)
\curveto(838.36145748,311.7904133)(838.25645759,311.55041354)(838.12646484,311.34041504)
\curveto(837.92645792,311.02041407)(837.69645815,310.75541433)(837.43646484,310.54541504)
\curveto(837.18645866,310.34541474)(836.86645898,310.18041491)(836.47646484,310.05041504)
\curveto(836.37645947,310.01041508)(836.27645957,309.9854151)(836.17646484,309.97541504)
\curveto(836.07645977,309.95541513)(835.97145987,309.93541515)(835.86146484,309.91541504)
\curveto(835.81146003,309.90541518)(835.76146008,309.90041519)(835.71146484,309.90041504)
\curveto(835.67146017,309.90041519)(835.62646022,309.89541519)(835.57646484,309.88541504)
\lineto(835.42646484,309.88541504)
\curveto(835.37646047,309.87541521)(835.31646053,309.87041522)(835.24646484,309.87041504)
\curveto(835.18646066,309.87041522)(835.13646071,309.87541521)(835.09646484,309.88541504)
\lineto(834.96146484,309.88541504)
\curveto(834.91146093,309.89541519)(834.86646098,309.90041519)(834.82646484,309.90041504)
\curveto(834.78646106,309.90041519)(834.7464611,309.90541518)(834.70646484,309.91541504)
\curveto(834.65646119,309.92541516)(834.60146124,309.93541515)(834.54146484,309.94541504)
\curveto(834.48146136,309.94541514)(834.42646142,309.95041514)(834.37646484,309.96041504)
\curveto(834.28646156,309.98041511)(834.19646165,310.00541508)(834.10646484,310.03541504)
\curveto(834.01646183,310.05541503)(833.93146191,310.08041501)(833.85146484,310.11041504)
\curveto(833.81146203,310.13041496)(833.77646207,310.14041495)(833.74646484,310.14041504)
\curveto(833.71646213,310.15041494)(833.68146216,310.16541492)(833.64146484,310.18541504)
\curveto(833.49146235,310.25541483)(833.33146251,310.34041475)(833.16146484,310.44041504)
\curveto(832.87146297,310.63041446)(832.62146322,310.86041423)(832.41146484,311.13041504)
\curveto(832.21146363,311.41041368)(832.0414638,311.72041337)(831.90146484,312.06041504)
\curveto(831.85146399,312.17041292)(831.81146403,312.2854128)(831.78146484,312.40541504)
\curveto(831.76146408,312.52541256)(831.73146411,312.64541244)(831.69146484,312.76541504)
\curveto(831.68146416,312.80541228)(831.67646417,312.84041225)(831.67646484,312.87041504)
\curveto(831.67646417,312.90041219)(831.67146417,312.94041215)(831.66146484,312.99041504)
\curveto(831.6414642,313.07041202)(831.62646422,313.15541193)(831.61646484,313.24541504)
\curveto(831.60646424,313.33541175)(831.59146425,313.42541166)(831.57146484,313.51541504)
\lineto(831.57146484,313.72541504)
\curveto(831.56146428,313.76541132)(831.55146429,313.82041127)(831.54146484,313.89041504)
\curveto(831.5414643,313.97041112)(831.5464643,314.03541105)(831.55646484,314.08541504)
\lineto(831.55646484,314.25041504)
\curveto(831.57646427,314.30041079)(831.58146426,314.35041074)(831.57146484,314.40041504)
\curveto(831.57146427,314.46041063)(831.57646427,314.51541057)(831.58646484,314.56541504)
\curveto(831.62646422,314.72541036)(831.65646419,314.8854102)(831.67646484,315.04541504)
\curveto(831.70646414,315.20540988)(831.75146409,315.35540973)(831.81146484,315.49541504)
\curveto(831.86146398,315.60540948)(831.90646394,315.71540937)(831.94646484,315.82541504)
\curveto(831.99646385,315.94540914)(832.05146379,316.06040903)(832.11146484,316.17041504)
\curveto(832.33146351,316.52040857)(832.58146326,316.82040827)(832.86146484,317.07041504)
\curveto(833.1414627,317.33040776)(833.48646236,317.54540754)(833.89646484,317.71541504)
\curveto(834.01646183,317.76540732)(834.13646171,317.80040729)(834.25646484,317.82041504)
\curveto(834.38646146,317.85040724)(834.52146132,317.88040721)(834.66146484,317.91041504)
\curveto(834.71146113,317.92040717)(834.75646109,317.92540716)(834.79646484,317.92541504)
\curveto(834.83646101,317.93540715)(834.88146096,317.94040715)(834.93146484,317.94041504)
\curveto(834.95146089,317.95040714)(834.97646087,317.95040714)(835.00646484,317.94041504)
\curveto(835.03646081,317.93040716)(835.06146078,317.93540715)(835.08146484,317.95541504)
\curveto(835.50146034,317.96540712)(835.86645998,317.92040717)(836.17646484,317.82041504)
\curveto(836.48645936,317.73040736)(836.76645908,317.60540748)(837.01646484,317.44541504)
\curveto(837.06645878,317.42540766)(837.10645874,317.39540769)(837.13646484,317.35541504)
\curveto(837.16645868,317.32540776)(837.20145864,317.30040779)(837.24146484,317.28041504)
\curveto(837.32145852,317.22040787)(837.40145844,317.15040794)(837.48146484,317.07041504)
\curveto(837.57145827,316.9904081)(837.6464582,316.91040818)(837.70646484,316.83041504)
\curveto(837.86645798,316.62040847)(838.00145784,316.42040867)(838.11146484,316.23041504)
\curveto(838.18145766,316.12040897)(838.23645761,316.00040909)(838.27646484,315.87041504)
\curveto(838.31645753,315.74040935)(838.36145748,315.61040948)(838.41146484,315.48041504)
\curveto(838.46145738,315.35040974)(838.49645735,315.21540987)(838.51646484,315.07541504)
\curveto(838.5464573,314.93541015)(838.58145726,314.79541029)(838.62146484,314.65541504)
\curveto(838.63145721,314.5854105)(838.63645721,314.51541057)(838.63646484,314.44541504)
\lineto(838.66646484,314.23541504)
\moveto(837.21146484,314.74541504)
\curveto(837.2414586,314.7854103)(837.26645858,314.83541025)(837.28646484,314.89541504)
\curveto(837.30645854,314.96541012)(837.30645854,315.03541005)(837.28646484,315.10541504)
\curveto(837.22645862,315.32540976)(837.1414587,315.53040956)(837.03146484,315.72041504)
\curveto(836.89145895,315.95040914)(836.73645911,316.14540894)(836.56646484,316.30541504)
\curveto(836.39645945,316.46540862)(836.17645967,316.60040849)(835.90646484,316.71041504)
\curveto(835.83646001,316.73040836)(835.76646008,316.74540834)(835.69646484,316.75541504)
\curveto(835.62646022,316.77540831)(835.55146029,316.79540829)(835.47146484,316.81541504)
\curveto(835.39146045,316.83540825)(835.30646054,316.84540824)(835.21646484,316.84541504)
\lineto(834.96146484,316.84541504)
\curveto(834.93146091,316.82540826)(834.89646095,316.81540827)(834.85646484,316.81541504)
\curveto(834.81646103,316.82540826)(834.78146106,316.82540826)(834.75146484,316.81541504)
\lineto(834.51146484,316.75541504)
\curveto(834.4414614,316.74540834)(834.37146147,316.73040836)(834.30146484,316.71041504)
\curveto(834.01146183,316.5904085)(833.77646207,316.44040865)(833.59646484,316.26041504)
\curveto(833.42646242,316.08040901)(833.27146257,315.85540923)(833.13146484,315.58541504)
\curveto(833.10146274,315.53540955)(833.07146277,315.47040962)(833.04146484,315.39041504)
\curveto(833.01146283,315.32040977)(832.98646286,315.24040985)(832.96646484,315.15041504)
\curveto(832.9464629,315.06041003)(832.9414629,314.97541011)(832.95146484,314.89541504)
\curveto(832.96146288,314.81541027)(832.99646285,314.75541033)(833.05646484,314.71541504)
\curveto(833.13646271,314.65541043)(833.27146257,314.62541046)(833.46146484,314.62541504)
\curveto(833.66146218,314.63541045)(833.83146201,314.64041045)(833.97146484,314.64041504)
\lineto(836.25146484,314.64041504)
\curveto(836.40145944,314.64041045)(836.58145926,314.63541045)(836.79146484,314.62541504)
\curveto(837.00145884,314.62541046)(837.1414587,314.66541042)(837.21146484,314.74541504)
}
}
{
\newrgbcolor{curcolor}{0 0 0}
\pscustom[linestyle=none,fillstyle=solid,fillcolor=curcolor]
{
\newpath
\moveto(840.54310547,320.74541504)
\curveto(840.67310385,320.74540434)(840.80810372,320.74540434)(840.94810547,320.74541504)
\curveto(841.09810343,320.74540434)(841.20810332,320.71040438)(841.27810547,320.64041504)
\curveto(841.3281032,320.57040452)(841.35310317,320.47540461)(841.35310547,320.35541504)
\curveto(841.36310316,320.24540484)(841.36810316,320.13040496)(841.36810547,320.01041504)
\lineto(841.36810547,318.67541504)
\lineto(841.36810547,312.60041504)
\lineto(841.36810547,310.92041504)
\lineto(841.36810547,310.53041504)
\curveto(841.36810316,310.3904147)(841.34310318,310.28041481)(841.29310547,310.20041504)
\curveto(841.26310326,310.15041494)(841.21810331,310.12041497)(841.15810547,310.11041504)
\curveto(841.10810342,310.10041499)(841.04310348,310.085415)(840.96310547,310.06541504)
\lineto(840.75310547,310.06541504)
\lineto(840.43810547,310.06541504)
\curveto(840.33810419,310.07541501)(840.26310426,310.11041498)(840.21310547,310.17041504)
\curveto(840.16310436,310.25041484)(840.13310439,310.35041474)(840.12310547,310.47041504)
\lineto(840.12310547,310.84541504)
\lineto(840.12310547,312.22541504)
\lineto(840.12310547,318.46541504)
\lineto(840.12310547,319.93541504)
\curveto(840.1231044,320.04540504)(840.11810441,320.16040493)(840.10810547,320.28041504)
\curveto(840.10810442,320.41040468)(840.13310439,320.51040458)(840.18310547,320.58041504)
\curveto(840.2231043,320.64040445)(840.29810423,320.6904044)(840.40810547,320.73041504)
\curveto(840.4281041,320.74040435)(840.44810408,320.74040435)(840.46810547,320.73041504)
\curveto(840.49810403,320.73040436)(840.523104,320.73540435)(840.54310547,320.74541504)
}
}
{
\newrgbcolor{curcolor}{0 0 0}
\pscustom[linestyle=none,fillstyle=solid,fillcolor=curcolor]
{
\newpath
\moveto(850.06294922,314.23541504)
\curveto(850.08294153,314.13541095)(850.08294153,314.02041107)(850.06294922,313.89041504)
\curveto(850.05294156,313.77041132)(850.02294159,313.6854114)(849.97294922,313.63541504)
\curveto(849.92294169,313.59541149)(849.84794177,313.56541152)(849.74794922,313.54541504)
\curveto(849.65794196,313.53541155)(849.55294206,313.53041156)(849.43294922,313.53041504)
\lineto(849.07294922,313.53041504)
\curveto(848.95294266,313.54041155)(848.84794277,313.54541154)(848.75794922,313.54541504)
\lineto(844.91794922,313.54541504)
\curveto(844.83794678,313.54541154)(844.75794686,313.54041155)(844.67794922,313.53041504)
\curveto(844.59794702,313.53041156)(844.53294708,313.51541157)(844.48294922,313.48541504)
\curveto(844.44294717,313.46541162)(844.40294721,313.42541166)(844.36294922,313.36541504)
\curveto(844.34294727,313.33541175)(844.32294729,313.2904118)(844.30294922,313.23041504)
\curveto(844.28294733,313.18041191)(844.28294733,313.13041196)(844.30294922,313.08041504)
\curveto(844.3129473,313.03041206)(844.3179473,312.9854121)(844.31794922,312.94541504)
\curveto(844.3179473,312.90541218)(844.32294729,312.86541222)(844.33294922,312.82541504)
\curveto(844.35294726,312.74541234)(844.37294724,312.66041243)(844.39294922,312.57041504)
\curveto(844.4129472,312.4904126)(844.44294717,312.41041268)(844.48294922,312.33041504)
\curveto(844.7129469,311.7904133)(845.09294652,311.40541368)(845.62294922,311.17541504)
\curveto(845.68294593,311.14541394)(845.74794587,311.12041397)(845.81794922,311.10041504)
\lineto(846.02794922,311.04041504)
\curveto(846.05794556,311.03041406)(846.10794551,311.02541406)(846.17794922,311.02541504)
\curveto(846.3179453,310.9854141)(846.50294511,310.96541412)(846.73294922,310.96541504)
\curveto(846.96294465,310.96541412)(847.14794447,310.9854141)(847.28794922,311.02541504)
\curveto(847.42794419,311.06541402)(847.55294406,311.10541398)(847.66294922,311.14541504)
\curveto(847.78294383,311.19541389)(847.89294372,311.25541383)(847.99294922,311.32541504)
\curveto(848.10294351,311.39541369)(848.19794342,311.47541361)(848.27794922,311.56541504)
\curveto(848.35794326,311.66541342)(848.42794319,311.77041332)(848.48794922,311.88041504)
\curveto(848.54794307,311.98041311)(848.59794302,312.085413)(848.63794922,312.19541504)
\curveto(848.68794293,312.30541278)(848.76794285,312.3854127)(848.87794922,312.43541504)
\curveto(848.9179427,312.45541263)(848.98294263,312.47041262)(849.07294922,312.48041504)
\curveto(849.16294245,312.4904126)(849.25294236,312.4904126)(849.34294922,312.48041504)
\curveto(849.43294218,312.48041261)(849.5179421,312.47541261)(849.59794922,312.46541504)
\curveto(849.67794194,312.45541263)(849.73294188,312.43541265)(849.76294922,312.40541504)
\curveto(849.86294175,312.33541275)(849.88794173,312.22041287)(849.83794922,312.06041504)
\curveto(849.75794186,311.7904133)(849.65294196,311.55041354)(849.52294922,311.34041504)
\curveto(849.32294229,311.02041407)(849.09294252,310.75541433)(848.83294922,310.54541504)
\curveto(848.58294303,310.34541474)(848.26294335,310.18041491)(847.87294922,310.05041504)
\curveto(847.77294384,310.01041508)(847.67294394,309.9854151)(847.57294922,309.97541504)
\curveto(847.47294414,309.95541513)(847.36794425,309.93541515)(847.25794922,309.91541504)
\curveto(847.20794441,309.90541518)(847.15794446,309.90041519)(847.10794922,309.90041504)
\curveto(847.06794455,309.90041519)(847.02294459,309.89541519)(846.97294922,309.88541504)
\lineto(846.82294922,309.88541504)
\curveto(846.77294484,309.87541521)(846.7129449,309.87041522)(846.64294922,309.87041504)
\curveto(846.58294503,309.87041522)(846.53294508,309.87541521)(846.49294922,309.88541504)
\lineto(846.35794922,309.88541504)
\curveto(846.30794531,309.89541519)(846.26294535,309.90041519)(846.22294922,309.90041504)
\curveto(846.18294543,309.90041519)(846.14294547,309.90541518)(846.10294922,309.91541504)
\curveto(846.05294556,309.92541516)(845.99794562,309.93541515)(845.93794922,309.94541504)
\curveto(845.87794574,309.94541514)(845.82294579,309.95041514)(845.77294922,309.96041504)
\curveto(845.68294593,309.98041511)(845.59294602,310.00541508)(845.50294922,310.03541504)
\curveto(845.4129462,310.05541503)(845.32794629,310.08041501)(845.24794922,310.11041504)
\curveto(845.20794641,310.13041496)(845.17294644,310.14041495)(845.14294922,310.14041504)
\curveto(845.1129465,310.15041494)(845.07794654,310.16541492)(845.03794922,310.18541504)
\curveto(844.88794673,310.25541483)(844.72794689,310.34041475)(844.55794922,310.44041504)
\curveto(844.26794735,310.63041446)(844.0179476,310.86041423)(843.80794922,311.13041504)
\curveto(843.60794801,311.41041368)(843.43794818,311.72041337)(843.29794922,312.06041504)
\curveto(843.24794837,312.17041292)(843.20794841,312.2854128)(843.17794922,312.40541504)
\curveto(843.15794846,312.52541256)(843.12794849,312.64541244)(843.08794922,312.76541504)
\curveto(843.07794854,312.80541228)(843.07294854,312.84041225)(843.07294922,312.87041504)
\curveto(843.07294854,312.90041219)(843.06794855,312.94041215)(843.05794922,312.99041504)
\curveto(843.03794858,313.07041202)(843.02294859,313.15541193)(843.01294922,313.24541504)
\curveto(843.00294861,313.33541175)(842.98794863,313.42541166)(842.96794922,313.51541504)
\lineto(842.96794922,313.72541504)
\curveto(842.95794866,313.76541132)(842.94794867,313.82041127)(842.93794922,313.89041504)
\curveto(842.93794868,313.97041112)(842.94294867,314.03541105)(842.95294922,314.08541504)
\lineto(842.95294922,314.25041504)
\curveto(842.97294864,314.30041079)(842.97794864,314.35041074)(842.96794922,314.40041504)
\curveto(842.96794865,314.46041063)(842.97294864,314.51541057)(842.98294922,314.56541504)
\curveto(843.02294859,314.72541036)(843.05294856,314.8854102)(843.07294922,315.04541504)
\curveto(843.10294851,315.20540988)(843.14794847,315.35540973)(843.20794922,315.49541504)
\curveto(843.25794836,315.60540948)(843.30294831,315.71540937)(843.34294922,315.82541504)
\curveto(843.39294822,315.94540914)(843.44794817,316.06040903)(843.50794922,316.17041504)
\curveto(843.72794789,316.52040857)(843.97794764,316.82040827)(844.25794922,317.07041504)
\curveto(844.53794708,317.33040776)(844.88294673,317.54540754)(845.29294922,317.71541504)
\curveto(845.4129462,317.76540732)(845.53294608,317.80040729)(845.65294922,317.82041504)
\curveto(845.78294583,317.85040724)(845.9179457,317.88040721)(846.05794922,317.91041504)
\curveto(846.10794551,317.92040717)(846.15294546,317.92540716)(846.19294922,317.92541504)
\curveto(846.23294538,317.93540715)(846.27794534,317.94040715)(846.32794922,317.94041504)
\curveto(846.34794527,317.95040714)(846.37294524,317.95040714)(846.40294922,317.94041504)
\curveto(846.43294518,317.93040716)(846.45794516,317.93540715)(846.47794922,317.95541504)
\curveto(846.89794472,317.96540712)(847.26294435,317.92040717)(847.57294922,317.82041504)
\curveto(847.88294373,317.73040736)(848.16294345,317.60540748)(848.41294922,317.44541504)
\curveto(848.46294315,317.42540766)(848.50294311,317.39540769)(848.53294922,317.35541504)
\curveto(848.56294305,317.32540776)(848.59794302,317.30040779)(848.63794922,317.28041504)
\curveto(848.7179429,317.22040787)(848.79794282,317.15040794)(848.87794922,317.07041504)
\curveto(848.96794265,316.9904081)(849.04294257,316.91040818)(849.10294922,316.83041504)
\curveto(849.26294235,316.62040847)(849.39794222,316.42040867)(849.50794922,316.23041504)
\curveto(849.57794204,316.12040897)(849.63294198,316.00040909)(849.67294922,315.87041504)
\curveto(849.7129419,315.74040935)(849.75794186,315.61040948)(849.80794922,315.48041504)
\curveto(849.85794176,315.35040974)(849.89294172,315.21540987)(849.91294922,315.07541504)
\curveto(849.94294167,314.93541015)(849.97794164,314.79541029)(850.01794922,314.65541504)
\curveto(850.02794159,314.5854105)(850.03294158,314.51541057)(850.03294922,314.44541504)
\lineto(850.06294922,314.23541504)
\moveto(848.60794922,314.74541504)
\curveto(848.63794298,314.7854103)(848.66294295,314.83541025)(848.68294922,314.89541504)
\curveto(848.70294291,314.96541012)(848.70294291,315.03541005)(848.68294922,315.10541504)
\curveto(848.62294299,315.32540976)(848.53794308,315.53040956)(848.42794922,315.72041504)
\curveto(848.28794333,315.95040914)(848.13294348,316.14540894)(847.96294922,316.30541504)
\curveto(847.79294382,316.46540862)(847.57294404,316.60040849)(847.30294922,316.71041504)
\curveto(847.23294438,316.73040836)(847.16294445,316.74540834)(847.09294922,316.75541504)
\curveto(847.02294459,316.77540831)(846.94794467,316.79540829)(846.86794922,316.81541504)
\curveto(846.78794483,316.83540825)(846.70294491,316.84540824)(846.61294922,316.84541504)
\lineto(846.35794922,316.84541504)
\curveto(846.32794529,316.82540826)(846.29294532,316.81540827)(846.25294922,316.81541504)
\curveto(846.2129454,316.82540826)(846.17794544,316.82540826)(846.14794922,316.81541504)
\lineto(845.90794922,316.75541504)
\curveto(845.83794578,316.74540834)(845.76794585,316.73040836)(845.69794922,316.71041504)
\curveto(845.40794621,316.5904085)(845.17294644,316.44040865)(844.99294922,316.26041504)
\curveto(844.82294679,316.08040901)(844.66794695,315.85540923)(844.52794922,315.58541504)
\curveto(844.49794712,315.53540955)(844.46794715,315.47040962)(844.43794922,315.39041504)
\curveto(844.40794721,315.32040977)(844.38294723,315.24040985)(844.36294922,315.15041504)
\curveto(844.34294727,315.06041003)(844.33794728,314.97541011)(844.34794922,314.89541504)
\curveto(844.35794726,314.81541027)(844.39294722,314.75541033)(844.45294922,314.71541504)
\curveto(844.53294708,314.65541043)(844.66794695,314.62541046)(844.85794922,314.62541504)
\curveto(845.05794656,314.63541045)(845.22794639,314.64041045)(845.36794922,314.64041504)
\lineto(847.64794922,314.64041504)
\curveto(847.79794382,314.64041045)(847.97794364,314.63541045)(848.18794922,314.62541504)
\curveto(848.39794322,314.62541046)(848.53794308,314.66541042)(848.60794922,314.74541504)
}
}
{
\newrgbcolor{curcolor}{0 0 0}
\pscustom[linestyle=none,fillstyle=solid,fillcolor=curcolor]
{
\newpath
\moveto(854.50458984,317.97041504)
\curveto(855.24458505,317.98040711)(855.85958444,317.87040722)(856.34958984,317.64041504)
\curveto(856.84958345,317.42040767)(857.24458305,317.085408)(857.53458984,316.63541504)
\curveto(857.66458263,316.43540865)(857.77458252,316.1904089)(857.86458984,315.90041504)
\curveto(857.88458241,315.85040924)(857.8995824,315.7854093)(857.90958984,315.70541504)
\curveto(857.91958238,315.62540946)(857.91458238,315.55540953)(857.89458984,315.49541504)
\curveto(857.86458243,315.44540964)(857.81458248,315.40040969)(857.74458984,315.36041504)
\curveto(857.71458258,315.34040975)(857.68458261,315.33040976)(857.65458984,315.33041504)
\curveto(857.62458267,315.34040975)(857.58958271,315.34040975)(857.54958984,315.33041504)
\curveto(857.50958279,315.32040977)(857.46958283,315.31540977)(857.42958984,315.31541504)
\curveto(857.38958291,315.32540976)(857.34958295,315.33040976)(857.30958984,315.33041504)
\lineto(856.99458984,315.33041504)
\curveto(856.8945834,315.34040975)(856.80958349,315.37040972)(856.73958984,315.42041504)
\curveto(856.65958364,315.48040961)(856.60458369,315.56540952)(856.57458984,315.67541504)
\curveto(856.54458375,315.7854093)(856.50458379,315.88040921)(856.45458984,315.96041504)
\curveto(856.30458399,316.22040887)(856.10958419,316.42540866)(855.86958984,316.57541504)
\curveto(855.78958451,316.62540846)(855.70458459,316.66540842)(855.61458984,316.69541504)
\curveto(855.52458477,316.73540835)(855.42958487,316.77040832)(855.32958984,316.80041504)
\curveto(855.18958511,316.84040825)(855.00458529,316.86040823)(854.77458984,316.86041504)
\curveto(854.54458575,316.87040822)(854.35458594,316.85040824)(854.20458984,316.80041504)
\curveto(854.13458616,316.78040831)(854.06958623,316.76540832)(854.00958984,316.75541504)
\curveto(853.94958635,316.74540834)(853.88458641,316.73040836)(853.81458984,316.71041504)
\curveto(853.55458674,316.60040849)(853.32458697,316.45040864)(853.12458984,316.26041504)
\curveto(852.92458737,316.07040902)(852.76958753,315.84540924)(852.65958984,315.58541504)
\curveto(852.61958768,315.49540959)(852.58458771,315.40040969)(852.55458984,315.30041504)
\curveto(852.52458777,315.21040988)(852.4945878,315.11040998)(852.46458984,315.00041504)
\lineto(852.37458984,314.59541504)
\curveto(852.36458793,314.54541054)(852.35958794,314.4904106)(852.35958984,314.43041504)
\curveto(852.36958793,314.37041072)(852.36458793,314.31541077)(852.34458984,314.26541504)
\lineto(852.34458984,314.14541504)
\curveto(852.33458796,314.10541098)(852.32458797,314.04041105)(852.31458984,313.95041504)
\curveto(852.31458798,313.86041123)(852.32458797,313.79541129)(852.34458984,313.75541504)
\curveto(852.35458794,313.70541138)(852.35458794,313.65541143)(852.34458984,313.60541504)
\curveto(852.33458796,313.55541153)(852.33458796,313.50541158)(852.34458984,313.45541504)
\curveto(852.35458794,313.41541167)(852.35958794,313.34541174)(852.35958984,313.24541504)
\curveto(852.37958792,313.16541192)(852.3945879,313.08041201)(852.40458984,312.99041504)
\curveto(852.42458787,312.90041219)(852.44458785,312.81541227)(852.46458984,312.73541504)
\curveto(852.57458772,312.41541267)(852.6995876,312.13541295)(852.83958984,311.89541504)
\curveto(852.98958731,311.66541342)(853.1945871,311.46541362)(853.45458984,311.29541504)
\curveto(853.54458675,311.24541384)(853.63458666,311.20041389)(853.72458984,311.16041504)
\curveto(853.82458647,311.12041397)(853.92958637,311.08041401)(854.03958984,311.04041504)
\curveto(854.08958621,311.03041406)(854.12958617,311.02541406)(854.15958984,311.02541504)
\curveto(854.18958611,311.02541406)(854.22958607,311.02041407)(854.27958984,311.01041504)
\curveto(854.30958599,311.00041409)(854.35958594,310.99541409)(854.42958984,310.99541504)
\lineto(854.59458984,310.99541504)
\curveto(854.5945857,310.9854141)(854.61458568,310.98041411)(854.65458984,310.98041504)
\curveto(854.67458562,310.9904141)(854.6995856,310.9904141)(854.72958984,310.98041504)
\curveto(854.75958554,310.98041411)(854.78958551,310.9854141)(854.81958984,310.99541504)
\curveto(854.88958541,311.01541407)(854.95458534,311.02041407)(855.01458984,311.01041504)
\curveto(855.08458521,311.01041408)(855.15458514,311.02041407)(855.22458984,311.04041504)
\curveto(855.48458481,311.12041397)(855.70958459,311.22041387)(855.89958984,311.34041504)
\curveto(856.08958421,311.47041362)(856.24958405,311.63541345)(856.37958984,311.83541504)
\curveto(856.42958387,311.91541317)(856.47458382,312.00041309)(856.51458984,312.09041504)
\lineto(856.63458984,312.36041504)
\curveto(856.65458364,312.44041265)(856.67458362,312.51541257)(856.69458984,312.58541504)
\curveto(856.72458357,312.66541242)(856.77458352,312.73041236)(856.84458984,312.78041504)
\curveto(856.87458342,312.81041228)(856.93458336,312.83041226)(857.02458984,312.84041504)
\curveto(857.11458318,312.86041223)(857.20958309,312.87041222)(857.30958984,312.87041504)
\curveto(857.41958288,312.88041221)(857.51958278,312.88041221)(857.60958984,312.87041504)
\curveto(857.70958259,312.86041223)(857.77958252,312.84041225)(857.81958984,312.81041504)
\curveto(857.87958242,312.77041232)(857.91458238,312.71041238)(857.92458984,312.63041504)
\curveto(857.94458235,312.55041254)(857.94458235,312.46541262)(857.92458984,312.37541504)
\curveto(857.87458242,312.22541286)(857.82458247,312.08041301)(857.77458984,311.94041504)
\curveto(857.73458256,311.81041328)(857.67958262,311.68041341)(857.60958984,311.55041504)
\curveto(857.45958284,311.25041384)(857.26958303,310.9854141)(857.03958984,310.75541504)
\curveto(856.81958348,310.52541456)(856.54958375,310.34041475)(856.22958984,310.20041504)
\curveto(856.14958415,310.16041493)(856.06458423,310.12541496)(855.97458984,310.09541504)
\curveto(855.88458441,310.07541501)(855.78958451,310.05041504)(855.68958984,310.02041504)
\curveto(855.57958472,309.98041511)(855.46958483,309.96041513)(855.35958984,309.96041504)
\curveto(855.24958505,309.95041514)(855.13958516,309.93541515)(855.02958984,309.91541504)
\curveto(854.98958531,309.89541519)(854.94958535,309.8904152)(854.90958984,309.90041504)
\curveto(854.86958543,309.91041518)(854.82958547,309.91041518)(854.78958984,309.90041504)
\lineto(854.65458984,309.90041504)
\lineto(854.41458984,309.90041504)
\curveto(854.34458595,309.8904152)(854.27958602,309.89541519)(854.21958984,309.91541504)
\lineto(854.14458984,309.91541504)
\lineto(853.78458984,309.96041504)
\curveto(853.65458664,310.00041509)(853.52958677,310.03541505)(853.40958984,310.06541504)
\curveto(853.28958701,310.09541499)(853.17458712,310.13541495)(853.06458984,310.18541504)
\curveto(852.70458759,310.34541474)(852.40458789,310.53541455)(852.16458984,310.75541504)
\curveto(851.93458836,310.97541411)(851.71958858,311.24541384)(851.51958984,311.56541504)
\curveto(851.46958883,311.64541344)(851.42458887,311.73541335)(851.38458984,311.83541504)
\lineto(851.26458984,312.13541504)
\curveto(851.21458908,312.24541284)(851.17958912,312.36041273)(851.15958984,312.48041504)
\curveto(851.13958916,312.60041249)(851.11458918,312.72041237)(851.08458984,312.84041504)
\curveto(851.07458922,312.88041221)(851.06958923,312.92041217)(851.06958984,312.96041504)
\curveto(851.06958923,313.00041209)(851.06458923,313.04041205)(851.05458984,313.08041504)
\curveto(851.03458926,313.14041195)(851.02458927,313.20541188)(851.02458984,313.27541504)
\curveto(851.03458926,313.34541174)(851.02958927,313.41041168)(851.00958984,313.47041504)
\lineto(851.00958984,313.62041504)
\curveto(850.9995893,313.67041142)(850.9945893,313.74041135)(850.99458984,313.83041504)
\curveto(850.9945893,313.92041117)(850.9995893,313.9904111)(851.00958984,314.04041504)
\curveto(851.01958928,314.090411)(851.01958928,314.13541095)(851.00958984,314.17541504)
\curveto(851.00958929,314.21541087)(851.01458928,314.25541083)(851.02458984,314.29541504)
\curveto(851.04458925,314.36541072)(851.04958925,314.43541065)(851.03958984,314.50541504)
\curveto(851.03958926,314.57541051)(851.04958925,314.64041045)(851.06958984,314.70041504)
\curveto(851.10958919,314.87041022)(851.14458915,315.04041005)(851.17458984,315.21041504)
\curveto(851.20458909,315.38040971)(851.24958905,315.54040955)(851.30958984,315.69041504)
\curveto(851.51958878,316.21040888)(851.77458852,316.63040846)(852.07458984,316.95041504)
\curveto(852.37458792,317.27040782)(852.78458751,317.53540755)(853.30458984,317.74541504)
\curveto(853.41458688,317.79540729)(853.53458676,317.83040726)(853.66458984,317.85041504)
\curveto(853.7945865,317.87040722)(853.92958637,317.89540719)(854.06958984,317.92541504)
\curveto(854.13958616,317.93540715)(854.20958609,317.94040715)(854.27958984,317.94041504)
\curveto(854.34958595,317.95040714)(854.42458587,317.96040713)(854.50458984,317.97041504)
}
}
{
\newrgbcolor{curcolor}{0 0 0}
\pscustom[linestyle=none,fillstyle=solid,fillcolor=curcolor]
{
\newpath
\moveto(860.37123047,320.13041504)
\curveto(860.52122846,320.13040496)(860.67122831,320.12540496)(860.82123047,320.11541504)
\curveto(860.97122801,320.11540497)(861.0762279,320.07540501)(861.13623047,319.99541504)
\curveto(861.18622779,319.93540515)(861.21122777,319.85040524)(861.21123047,319.74041504)
\curveto(861.22122776,319.64040545)(861.22622775,319.53540555)(861.22623047,319.42541504)
\lineto(861.22623047,318.55541504)
\curveto(861.22622775,318.47540661)(861.22122776,318.3904067)(861.21123047,318.30041504)
\curveto(861.21122777,318.22040687)(861.22122776,318.15040694)(861.24123047,318.09041504)
\curveto(861.2812277,317.95040714)(861.37122761,317.86040723)(861.51123047,317.82041504)
\curveto(861.56122742,317.81040728)(861.60622737,317.80540728)(861.64623047,317.80541504)
\lineto(861.79623047,317.80541504)
\lineto(862.20123047,317.80541504)
\curveto(862.36122662,317.81540727)(862.4762265,317.80540728)(862.54623047,317.77541504)
\curveto(862.63622634,317.71540737)(862.69622628,317.65540743)(862.72623047,317.59541504)
\curveto(862.74622623,317.55540753)(862.75622622,317.51040758)(862.75623047,317.46041504)
\lineto(862.75623047,317.31041504)
\curveto(862.75622622,317.20040789)(862.75122623,317.09540799)(862.74123047,316.99541504)
\curveto(862.73122625,316.90540818)(862.69622628,316.83540825)(862.63623047,316.78541504)
\curveto(862.5762264,316.73540835)(862.49122649,316.70540838)(862.38123047,316.69541504)
\lineto(862.05123047,316.69541504)
\curveto(861.94122704,316.70540838)(861.83122715,316.71040838)(861.72123047,316.71041504)
\curveto(861.61122737,316.71040838)(861.51622746,316.69540839)(861.43623047,316.66541504)
\curveto(861.36622761,316.63540845)(861.31622766,316.5854085)(861.28623047,316.51541504)
\curveto(861.25622772,316.44540864)(861.23622774,316.36040873)(861.22623047,316.26041504)
\curveto(861.21622776,316.17040892)(861.21122777,316.07040902)(861.21123047,315.96041504)
\curveto(861.22122776,315.86040923)(861.22622775,315.76040933)(861.22623047,315.66041504)
\lineto(861.22623047,312.69041504)
\curveto(861.22622775,312.47041262)(861.22122776,312.23541285)(861.21123047,311.98541504)
\curveto(861.21122777,311.74541334)(861.25622772,311.56041353)(861.34623047,311.43041504)
\curveto(861.39622758,311.35041374)(861.46122752,311.29541379)(861.54123047,311.26541504)
\curveto(861.62122736,311.23541385)(861.71622726,311.21041388)(861.82623047,311.19041504)
\curveto(861.85622712,311.18041391)(861.88622709,311.17541391)(861.91623047,311.17541504)
\curveto(861.95622702,311.1854139)(861.99122699,311.1854139)(862.02123047,311.17541504)
\lineto(862.21623047,311.17541504)
\curveto(862.31622666,311.17541391)(862.40622657,311.16541392)(862.48623047,311.14541504)
\curveto(862.5762264,311.13541395)(862.64122634,311.10041399)(862.68123047,311.04041504)
\curveto(862.70122628,311.01041408)(862.71622626,310.95541413)(862.72623047,310.87541504)
\curveto(862.74622623,310.80541428)(862.75622622,310.73041436)(862.75623047,310.65041504)
\curveto(862.76622621,310.57041452)(862.76622621,310.4904146)(862.75623047,310.41041504)
\curveto(862.74622623,310.34041475)(862.72622625,310.2854148)(862.69623047,310.24541504)
\curveto(862.65622632,310.17541491)(862.5812264,310.12541496)(862.47123047,310.09541504)
\curveto(862.39122659,310.07541501)(862.30122668,310.06541502)(862.20123047,310.06541504)
\curveto(862.10122688,310.07541501)(862.01122697,310.08041501)(861.93123047,310.08041504)
\curveto(861.87122711,310.08041501)(861.81122717,310.07541501)(861.75123047,310.06541504)
\curveto(861.69122729,310.06541502)(861.63622734,310.07041502)(861.58623047,310.08041504)
\lineto(861.40623047,310.08041504)
\curveto(861.35622762,310.090415)(861.30622767,310.09541499)(861.25623047,310.09541504)
\curveto(861.21622776,310.10541498)(861.17122781,310.11041498)(861.12123047,310.11041504)
\curveto(860.92122806,310.16041493)(860.74622823,310.21541487)(860.59623047,310.27541504)
\curveto(860.45622852,310.33541475)(860.33622864,310.44041465)(860.23623047,310.59041504)
\curveto(860.09622888,310.7904143)(860.01622896,311.04041405)(859.99623047,311.34041504)
\curveto(859.976229,311.65041344)(859.96622901,311.98041311)(859.96623047,312.33041504)
\lineto(859.96623047,316.26041504)
\curveto(859.93622904,316.3904087)(859.90622907,316.4854086)(859.87623047,316.54541504)
\curveto(859.85622912,316.60540848)(859.78622919,316.65540843)(859.66623047,316.69541504)
\curveto(859.62622935,316.70540838)(859.58622939,316.70540838)(859.54623047,316.69541504)
\curveto(859.50622947,316.6854084)(859.46622951,316.6904084)(859.42623047,316.71041504)
\lineto(859.18623047,316.71041504)
\curveto(859.05622992,316.71040838)(858.94623003,316.72040837)(858.85623047,316.74041504)
\curveto(858.7762302,316.77040832)(858.72123026,316.83040826)(858.69123047,316.92041504)
\curveto(858.67123031,316.96040813)(858.65623032,317.00540808)(858.64623047,317.05541504)
\lineto(858.64623047,317.20541504)
\curveto(858.64623033,317.34540774)(858.65623032,317.46040763)(858.67623047,317.55041504)
\curveto(858.69623028,317.65040744)(858.75623022,317.72540736)(858.85623047,317.77541504)
\curveto(858.96623001,317.81540727)(859.10622987,317.82540726)(859.27623047,317.80541504)
\curveto(859.45622952,317.7854073)(859.60622937,317.79540729)(859.72623047,317.83541504)
\curveto(859.81622916,317.8854072)(859.88622909,317.95540713)(859.93623047,318.04541504)
\curveto(859.95622902,318.10540698)(859.96622901,318.18040691)(859.96623047,318.27041504)
\lineto(859.96623047,318.52541504)
\lineto(859.96623047,319.45541504)
\lineto(859.96623047,319.69541504)
\curveto(859.96622901,319.7854053)(859.976229,319.86040523)(859.99623047,319.92041504)
\curveto(860.03622894,320.00040509)(860.11122887,320.06540502)(860.22123047,320.11541504)
\curveto(860.25122873,320.11540497)(860.2762287,320.11540497)(860.29623047,320.11541504)
\curveto(860.32622865,320.12540496)(860.35122863,320.13040496)(860.37123047,320.13041504)
}
}
{
\newrgbcolor{curcolor}{0 0 0}
\pscustom[linestyle=none,fillstyle=solid,fillcolor=curcolor]
{
\newpath
\moveto(867.78802734,317.97041504)
\curveto(868.01802255,317.97040712)(868.14802242,317.91040718)(868.17802734,317.79041504)
\curveto(868.20802236,317.68040741)(868.22302235,317.51540757)(868.22302734,317.29541504)
\lineto(868.22302734,317.01041504)
\curveto(868.22302235,316.92040817)(868.19802237,316.84540824)(868.14802734,316.78541504)
\curveto(868.08802248,316.70540838)(868.00302257,316.66040843)(867.89302734,316.65041504)
\curveto(867.78302279,316.65040844)(867.6730229,316.63540845)(867.56302734,316.60541504)
\curveto(867.42302315,316.57540851)(867.28802328,316.54540854)(867.15802734,316.51541504)
\curveto(867.03802353,316.4854086)(866.92302365,316.44540864)(866.81302734,316.39541504)
\curveto(866.52302405,316.26540882)(866.28802428,316.085409)(866.10802734,315.85541504)
\curveto(865.92802464,315.63540945)(865.7730248,315.38040971)(865.64302734,315.09041504)
\curveto(865.60302497,314.98041011)(865.573025,314.86541022)(865.55302734,314.74541504)
\curveto(865.53302504,314.63541045)(865.50802506,314.52041057)(865.47802734,314.40041504)
\curveto(865.4680251,314.35041074)(865.46302511,314.30041079)(865.46302734,314.25041504)
\curveto(865.4730251,314.20041089)(865.4730251,314.15041094)(865.46302734,314.10041504)
\curveto(865.43302514,313.98041111)(865.41802515,313.84041125)(865.41802734,313.68041504)
\curveto(865.42802514,313.53041156)(865.43302514,313.3854117)(865.43302734,313.24541504)
\lineto(865.43302734,311.40041504)
\lineto(865.43302734,311.05541504)
\curveto(865.43302514,310.93541415)(865.42802514,310.82041427)(865.41802734,310.71041504)
\curveto(865.40802516,310.60041449)(865.40302517,310.50541458)(865.40302734,310.42541504)
\curveto(865.41302516,310.34541474)(865.39302518,310.27541481)(865.34302734,310.21541504)
\curveto(865.29302528,310.14541494)(865.21302536,310.10541498)(865.10302734,310.09541504)
\curveto(865.00302557,310.085415)(864.89302568,310.08041501)(864.77302734,310.08041504)
\lineto(864.50302734,310.08041504)
\curveto(864.45302612,310.10041499)(864.40302617,310.11541497)(864.35302734,310.12541504)
\curveto(864.31302626,310.14541494)(864.28302629,310.17041492)(864.26302734,310.20041504)
\curveto(864.21302636,310.27041482)(864.18302639,310.35541473)(864.17302734,310.45541504)
\lineto(864.17302734,310.78541504)
\lineto(864.17302734,311.94041504)
\lineto(864.17302734,316.09541504)
\lineto(864.17302734,317.13041504)
\lineto(864.17302734,317.43041504)
\curveto(864.18302639,317.53040756)(864.21302636,317.61540747)(864.26302734,317.68541504)
\curveto(864.29302628,317.72540736)(864.34302623,317.75540733)(864.41302734,317.77541504)
\curveto(864.49302608,317.79540729)(864.57802599,317.80540728)(864.66802734,317.80541504)
\curveto(864.75802581,317.81540727)(864.84802572,317.81540727)(864.93802734,317.80541504)
\curveto(865.02802554,317.79540729)(865.09802547,317.78040731)(865.14802734,317.76041504)
\curveto(865.22802534,317.73040736)(865.27802529,317.67040742)(865.29802734,317.58041504)
\curveto(865.32802524,317.50040759)(865.34302523,317.41040768)(865.34302734,317.31041504)
\lineto(865.34302734,317.01041504)
\curveto(865.34302523,316.91040818)(865.36302521,316.82040827)(865.40302734,316.74041504)
\curveto(865.41302516,316.72040837)(865.42302515,316.70540838)(865.43302734,316.69541504)
\lineto(865.47802734,316.65041504)
\curveto(865.58802498,316.65040844)(865.67802489,316.69540839)(865.74802734,316.78541504)
\curveto(865.81802475,316.8854082)(865.87802469,316.96540812)(865.92802734,317.02541504)
\lineto(866.01802734,317.11541504)
\curveto(866.10802446,317.22540786)(866.23302434,317.34040775)(866.39302734,317.46041504)
\curveto(866.55302402,317.58040751)(866.70302387,317.67040742)(866.84302734,317.73041504)
\curveto(866.93302364,317.78040731)(867.02802354,317.81540727)(867.12802734,317.83541504)
\curveto(867.22802334,317.86540722)(867.33302324,317.89540719)(867.44302734,317.92541504)
\curveto(867.50302307,317.93540715)(867.56302301,317.94040715)(867.62302734,317.94041504)
\curveto(867.68302289,317.95040714)(867.73802283,317.96040713)(867.78802734,317.97041504)
}
}
{
\newrgbcolor{curcolor}{0 0 0}
\pscustom[linestyle=none,fillstyle=solid,fillcolor=curcolor]
{
\newpath
\moveto(873.23279297,320.98541504)
\curveto(873.3027879,320.9854041)(873.38778782,320.9854041)(873.48779297,320.98541504)
\curveto(873.59778761,320.99540409)(873.69778751,320.99540409)(873.78779297,320.98541504)
\curveto(873.88778732,320.9854041)(873.97778723,320.97540411)(874.05779297,320.95541504)
\curveto(874.13778707,320.93540415)(874.19278701,320.90540418)(874.22279297,320.86541504)
\curveto(874.23278697,320.82540426)(874.22778698,320.77040432)(874.20779297,320.70041504)
\curveto(874.18778702,320.64040445)(874.14778706,320.58040451)(874.08779297,320.52041504)
\lineto(873.92279297,320.35541504)
\curveto(873.87278733,320.30540478)(873.82278738,320.25040484)(873.77279297,320.19041504)
\curveto(873.73278747,320.14040495)(873.68778752,320.085405)(873.63779297,320.02541504)
\curveto(873.6077876,319.97540511)(873.56778764,319.93040516)(873.51779297,319.89041504)
\curveto(873.47778773,319.86040523)(873.43778777,319.82040527)(873.39779297,319.77041504)
\lineto(873.35279297,319.72541504)
\curveto(873.35278785,319.71540537)(873.34278786,319.70540538)(873.32279297,319.69541504)
\curveto(873.28278792,319.64540544)(873.24278796,319.60040549)(873.20279297,319.56041504)
\curveto(873.16278804,319.53040556)(873.12278808,319.4904056)(873.08279297,319.44041504)
\curveto(873.06278814,319.40040569)(873.03278817,319.36540572)(872.99279297,319.33541504)
\lineto(872.90279297,319.24541504)
\curveto(872.86278834,319.19540589)(872.81778839,319.14540594)(872.76779297,319.09541504)
\curveto(872.72778848,319.04540604)(872.68278852,319.00540608)(872.63279297,318.97541504)
\curveto(872.56278864,318.93540615)(872.44778876,318.90040619)(872.28779297,318.87041504)
\curveto(872.13778907,318.85040624)(872.01778919,318.86540622)(871.92779297,318.91541504)
\curveto(871.89778931,318.93540615)(871.86778934,318.96540612)(871.83779297,319.00541504)
\curveto(871.81778939,319.05540603)(871.81778939,319.11040598)(871.83779297,319.17041504)
\curveto(871.85778935,319.25040584)(871.88778932,319.32040577)(871.92779297,319.38041504)
\curveto(871.96778924,319.45040564)(872.01278919,319.51540557)(872.06279297,319.57541504)
\curveto(872.14278906,319.71540537)(872.22778898,319.86040523)(872.31779297,320.01041504)
\curveto(872.4077888,320.16040493)(872.49778871,320.30540478)(872.58779297,320.44541504)
\lineto(872.70779297,320.65541504)
\curveto(872.74778846,320.73540435)(872.8027884,320.80040429)(872.87279297,320.85041504)
\curveto(872.94278826,320.90040419)(873.01278819,320.94040415)(873.08279297,320.97041504)
\curveto(873.11278809,320.97040412)(873.13778807,320.97040412)(873.15779297,320.97041504)
\curveto(873.18778802,320.98040411)(873.21278799,320.9854041)(873.23279297,320.98541504)
\moveto(876.27779297,314.26541504)
\curveto(876.26778494,314.31541077)(876.26278494,314.36541072)(876.26279297,314.41541504)
\curveto(876.27278493,314.47541061)(876.27278493,314.53041056)(876.26279297,314.58041504)
\curveto(876.23278497,314.71041038)(876.207785,314.83541025)(876.18779297,314.95541504)
\curveto(876.16778504,315.08541)(876.14278506,315.20540988)(876.11279297,315.31541504)
\curveto(876.07278513,315.42540966)(876.03778517,315.53040956)(876.00779297,315.63041504)
\curveto(875.97778523,315.73040936)(875.93778527,315.83040926)(875.88779297,315.93041504)
\curveto(875.62778558,316.54040855)(875.202786,317.03540805)(874.61279297,317.41541504)
\curveto(874.02278718,317.79540729)(873.28278792,317.98040711)(872.39279297,317.97041504)
\curveto(872.33278887,317.96040713)(872.26778894,317.95040714)(872.19779297,317.94041504)
\lineto(872.00279297,317.94041504)
\curveto(871.86278934,317.90040719)(871.72278948,317.87040722)(871.58279297,317.85041504)
\curveto(871.44278976,317.84040725)(871.31278989,317.81040728)(871.19279297,317.76041504)
\curveto(871.05279015,317.70040739)(870.91779029,317.64040745)(870.78779297,317.58041504)
\curveto(870.65779055,317.53040756)(870.53279067,317.46540762)(870.41279297,317.38541504)
\curveto(870.1027911,317.1854079)(869.83779137,316.93540815)(869.61779297,316.63541504)
\curveto(869.4077918,316.34540874)(869.22779198,316.01540907)(869.07779297,315.64541504)
\curveto(869.02779218,315.53540955)(868.98779222,315.42040967)(868.95779297,315.30041504)
\curveto(868.93779227,315.18040991)(868.91279229,315.06041003)(868.88279297,314.94041504)
\curveto(868.87279233,314.8904102)(868.86279234,314.84541024)(868.85279297,314.80541504)
\curveto(868.85279235,314.77541031)(868.84779236,314.73541035)(868.83779297,314.68541504)
\curveto(868.81779239,314.61541047)(868.81279239,314.54541054)(868.82279297,314.47541504)
\curveto(868.83279237,314.40541068)(868.82779238,314.33541075)(868.80779297,314.26541504)
\curveto(868.78779242,314.20541088)(868.77779243,314.11041098)(868.77779297,313.98041504)
\curveto(868.77779243,313.86041123)(868.78279242,313.77541131)(868.79279297,313.72541504)
\curveto(868.8027924,313.67541141)(868.8077924,313.63041146)(868.80779297,313.59041504)
\lineto(868.80779297,313.47041504)
\curveto(868.82779238,313.3904117)(868.83779237,313.30541178)(868.83779297,313.21541504)
\curveto(868.84779236,313.13541195)(868.86279234,313.05541203)(868.88279297,312.97541504)
\curveto(868.89279231,312.93541215)(868.89279231,312.90041219)(868.88279297,312.87041504)
\curveto(868.88279232,312.85041224)(868.89279231,312.82041227)(868.91279297,312.78041504)
\curveto(868.93279227,312.67041242)(868.95279225,312.56541252)(868.97279297,312.46541504)
\curveto(869.0027922,312.36541272)(869.04279216,312.26541282)(869.09279297,312.16541504)
\curveto(869.28279192,311.70541338)(869.52279168,311.31541377)(869.81279297,310.99541504)
\curveto(870.1027911,310.67541441)(870.47279073,310.41541467)(870.92279297,310.21541504)
\curveto(871.04279016,310.16541492)(871.16779004,310.12041497)(871.29779297,310.08041504)
\curveto(871.42778978,310.04041505)(871.56278964,310.00041509)(871.70279297,309.96041504)
\lineto(872.06279297,309.91541504)
\lineto(872.15279297,309.91541504)
\curveto(872.18278902,309.90541518)(872.21778899,309.90541518)(872.25779297,309.91541504)
\curveto(872.29778891,309.91541517)(872.33778887,309.91041518)(872.37779297,309.90041504)
\curveto(872.4077888,309.8904152)(872.45778875,309.8854152)(872.52779297,309.88541504)
\curveto(872.6077886,309.8854152)(872.66778854,309.8904152)(872.70779297,309.90041504)
\curveto(872.76778844,309.90041519)(872.82778838,309.90541518)(872.88779297,309.91541504)
\curveto(872.95778825,309.91541517)(873.02278818,309.92041517)(873.08279297,309.93041504)
\curveto(873.21278799,309.95041514)(873.33778787,309.97041512)(873.45779297,309.99041504)
\curveto(873.58778762,310.01041508)(873.7077875,310.04041505)(873.81779297,310.08041504)
\curveto(874.32778688,310.25041484)(874.75778645,310.49541459)(875.10779297,310.81541504)
\curveto(875.45778575,311.14541394)(875.73778547,311.55041354)(875.94779297,312.03041504)
\curveto(875.99778521,312.14041295)(876.03778517,312.26041283)(876.06779297,312.39041504)
\curveto(876.09778511,312.52041257)(876.13278507,312.65041244)(876.17279297,312.78041504)
\curveto(876.19278501,312.84041225)(876.202785,312.90041219)(876.20279297,312.96041504)
\curveto(876.21278499,313.02041207)(876.22778498,313.08041201)(876.24779297,313.14041504)
\curveto(876.25778495,313.22041187)(876.26278494,313.2904118)(876.26279297,313.35041504)
\curveto(876.27278493,313.42041167)(876.28278492,313.49541159)(876.29279297,313.57541504)
\lineto(876.29279297,313.72541504)
\curveto(876.3027849,313.77541131)(876.3077849,313.86041123)(876.30779297,313.98041504)
\curveto(876.3077849,314.11041098)(876.29778491,314.20541088)(876.27779297,314.26541504)
\moveto(874.94279297,313.41041504)
\curveto(874.9027863,313.27041182)(874.87278633,313.13041196)(874.85279297,312.99041504)
\curveto(874.84278636,312.86041223)(874.81278639,312.73541235)(874.76279297,312.61541504)
\curveto(874.62278658,312.27541281)(874.44778676,311.98041311)(874.23779297,311.73041504)
\curveto(874.02778718,311.48041361)(873.75278745,311.2854138)(873.41279297,311.14541504)
\curveto(873.34278786,311.11541397)(873.26778794,311.090414)(873.18779297,311.07041504)
\curveto(873.1077881,311.06041403)(873.02778818,311.04541404)(872.94779297,311.02541504)
\curveto(872.9077883,311.00541408)(872.86778834,310.99541409)(872.82779297,310.99541504)
\curveto(872.78778842,311.00541408)(872.74778846,311.00541408)(872.70779297,310.99541504)
\curveto(872.65778855,310.97541411)(872.58278862,310.97041412)(872.48279297,310.98041504)
\curveto(872.39278881,310.9904141)(872.33278887,311.00041409)(872.30279297,311.01041504)
\curveto(872.25278895,311.02041407)(872.202789,311.02041407)(872.15279297,311.01041504)
\curveto(872.11278909,311.01041408)(872.06778914,311.02041407)(872.01779297,311.04041504)
\curveto(871.9077893,311.07041402)(871.8027894,311.10041399)(871.70279297,311.13041504)
\curveto(871.61278959,311.17041392)(871.52778968,311.21541387)(871.44779297,311.26541504)
\curveto(871.39778981,311.29541379)(871.35278985,311.32041377)(871.31279297,311.34041504)
\curveto(871.27278993,311.36041373)(871.22778998,311.3854137)(871.17779297,311.41541504)
\curveto(870.97779023,311.55541353)(870.8077904,311.73041336)(870.66779297,311.94041504)
\curveto(870.53779067,312.15041294)(870.42279078,312.37541271)(870.32279297,312.61541504)
\curveto(870.28279092,312.69541239)(870.25279095,312.78041231)(870.23279297,312.87041504)
\curveto(870.22279098,312.97041212)(870.207791,313.07041202)(870.18779297,313.17041504)
\lineto(870.15779297,313.35041504)
\curveto(870.13779107,313.43041166)(870.12779108,313.52041157)(870.12779297,313.62041504)
\lineto(870.12779297,313.92041504)
\curveto(870.12779108,313.97041112)(870.12279108,314.01541107)(870.11279297,314.05541504)
\curveto(870.11279109,314.09541099)(870.11779109,314.13041096)(870.12779297,314.16041504)
\curveto(870.12779108,314.25041084)(870.13279107,314.31541077)(870.14279297,314.35541504)
\curveto(870.16279104,314.46541062)(870.17779103,314.57041052)(870.18779297,314.67041504)
\curveto(870.19779101,314.78041031)(870.21779099,314.8854102)(870.24779297,314.98541504)
\curveto(870.35779085,315.30540978)(870.48779072,315.5854095)(870.63779297,315.82541504)
\curveto(870.78779042,316.06540902)(870.98279022,316.27040882)(871.22279297,316.44041504)
\curveto(871.27278993,316.48040861)(871.32778988,316.52040857)(871.38779297,316.56041504)
\curveto(871.44778976,316.60040849)(871.51278969,316.63040846)(871.58279297,316.65041504)
\curveto(871.66278954,316.6904084)(871.74278946,316.72040837)(871.82279297,316.74041504)
\curveto(871.91278929,316.77040832)(872.0077892,316.79540829)(872.10779297,316.81541504)
\lineto(872.37779297,316.86041504)
\lineto(872.64779297,316.86041504)
\curveto(872.73778847,316.86040823)(872.82278838,316.85040824)(872.90279297,316.83041504)
\lineto(873.14279297,316.77041504)
\curveto(873.22278798,316.76040833)(873.29778791,316.74040835)(873.36779297,316.71041504)
\curveto(873.97778723,316.46040863)(874.42278678,316.01040908)(874.70279297,315.36041504)
\curveto(874.73278647,315.2904098)(874.75778645,315.21540987)(874.77779297,315.13541504)
\curveto(874.79778641,315.05541003)(874.82278638,314.97541011)(874.85279297,314.89541504)
\curveto(874.92278628,314.62541046)(874.95778625,314.29541079)(874.95779297,313.90541504)
\lineto(874.95779297,313.65041504)
\curveto(874.96778624,313.56041153)(874.96278624,313.48041161)(874.94279297,313.41041504)
}
}
{
\newrgbcolor{curcolor}{0 0 0}
\pscustom[linestyle=none,fillstyle=solid,fillcolor=curcolor]
{
\newpath
\moveto(881.45607422,317.94041504)
\curveto(882.08606898,317.96040713)(882.59106848,317.87540721)(882.97107422,317.68541504)
\curveto(883.35106772,317.49540759)(883.65606741,317.21040788)(883.88607422,316.83041504)
\curveto(883.94606712,316.73040836)(883.99106708,316.62040847)(884.02107422,316.50041504)
\curveto(884.06106701,316.3904087)(884.09606697,316.27540881)(884.12607422,316.15541504)
\curveto(884.17606689,315.96540912)(884.20606686,315.76040933)(884.21607422,315.54041504)
\curveto(884.22606684,315.32040977)(884.23106684,315.09540999)(884.23107422,314.86541504)
\lineto(884.23107422,313.26041504)
\lineto(884.23107422,310.92041504)
\curveto(884.23106684,310.75041434)(884.22606684,310.58041451)(884.21607422,310.41041504)
\curveto(884.21606685,310.24041485)(884.15106692,310.13041496)(884.02107422,310.08041504)
\curveto(883.9710671,310.06041503)(883.91606715,310.05041504)(883.85607422,310.05041504)
\curveto(883.80606726,310.04041505)(883.75106732,310.03541505)(883.69107422,310.03541504)
\curveto(883.56106751,310.03541505)(883.43606763,310.04041505)(883.31607422,310.05041504)
\curveto(883.19606787,310.05041504)(883.11106796,310.090415)(883.06107422,310.17041504)
\curveto(883.01106806,310.24041485)(882.98606808,310.33041476)(882.98607422,310.44041504)
\lineto(882.98607422,310.77041504)
\lineto(882.98607422,312.06041504)
\lineto(882.98607422,314.50541504)
\curveto(882.98606808,314.77541031)(882.98106809,315.04041005)(882.97107422,315.30041504)
\curveto(882.96106811,315.57040952)(882.91606815,315.80040929)(882.83607422,315.99041504)
\curveto(882.75606831,316.1904089)(882.63606843,316.35040874)(882.47607422,316.47041504)
\curveto(882.31606875,316.60040849)(882.13106894,316.70040839)(881.92107422,316.77041504)
\curveto(881.86106921,316.7904083)(881.79606927,316.80040829)(881.72607422,316.80041504)
\curveto(881.6660694,316.81040828)(881.60606946,316.82540826)(881.54607422,316.84541504)
\curveto(881.49606957,316.85540823)(881.41606965,316.85540823)(881.30607422,316.84541504)
\curveto(881.20606986,316.84540824)(881.13606993,316.84040825)(881.09607422,316.83041504)
\curveto(881.05607001,316.81040828)(881.02107005,316.80040829)(880.99107422,316.80041504)
\curveto(880.96107011,316.81040828)(880.92607014,316.81040828)(880.88607422,316.80041504)
\curveto(880.75607031,316.77040832)(880.63107044,316.73540835)(880.51107422,316.69541504)
\curveto(880.40107067,316.66540842)(880.29607077,316.62040847)(880.19607422,316.56041504)
\curveto(880.15607091,316.54040855)(880.12107095,316.52040857)(880.09107422,316.50041504)
\curveto(880.06107101,316.48040861)(880.02607104,316.46040863)(879.98607422,316.44041504)
\curveto(879.63607143,316.1904089)(879.38107169,315.81540927)(879.22107422,315.31541504)
\curveto(879.19107188,315.23540985)(879.1710719,315.15040994)(879.16107422,315.06041504)
\curveto(879.15107192,314.98041011)(879.13607193,314.90041019)(879.11607422,314.82041504)
\curveto(879.09607197,314.77041032)(879.09107198,314.72041037)(879.10107422,314.67041504)
\curveto(879.11107196,314.63041046)(879.10607196,314.5904105)(879.08607422,314.55041504)
\lineto(879.08607422,314.23541504)
\curveto(879.07607199,314.20541088)(879.071072,314.17041092)(879.07107422,314.13041504)
\curveto(879.08107199,314.090411)(879.08607198,314.04541104)(879.08607422,313.99541504)
\lineto(879.08607422,313.54541504)
\lineto(879.08607422,312.10541504)
\lineto(879.08607422,310.78541504)
\lineto(879.08607422,310.44041504)
\curveto(879.08607198,310.33041476)(879.06107201,310.24041485)(879.01107422,310.17041504)
\curveto(878.96107211,310.090415)(878.8710722,310.05041504)(878.74107422,310.05041504)
\curveto(878.62107245,310.04041505)(878.49607257,310.03541505)(878.36607422,310.03541504)
\curveto(878.28607278,310.03541505)(878.21107286,310.04041505)(878.14107422,310.05041504)
\curveto(878.071073,310.06041503)(878.01107306,310.085415)(877.96107422,310.12541504)
\curveto(877.88107319,310.17541491)(877.84107323,310.27041482)(877.84107422,310.41041504)
\lineto(877.84107422,310.81541504)
\lineto(877.84107422,312.58541504)
\lineto(877.84107422,316.21541504)
\lineto(877.84107422,317.13041504)
\lineto(877.84107422,317.40041504)
\curveto(877.84107323,317.4904076)(877.86107321,317.56040753)(877.90107422,317.61041504)
\curveto(877.93107314,317.67040742)(877.98107309,317.71040738)(878.05107422,317.73041504)
\curveto(878.09107298,317.74040735)(878.14607292,317.75040734)(878.21607422,317.76041504)
\curveto(878.29607277,317.77040732)(878.37607269,317.77540731)(878.45607422,317.77541504)
\curveto(878.53607253,317.77540731)(878.61107246,317.77040732)(878.68107422,317.76041504)
\curveto(878.76107231,317.75040734)(878.81607225,317.73540735)(878.84607422,317.71541504)
\curveto(878.95607211,317.64540744)(879.00607206,317.55540753)(878.99607422,317.44541504)
\curveto(878.98607208,317.34540774)(879.00107207,317.23040786)(879.04107422,317.10041504)
\curveto(879.06107201,317.04040805)(879.10107197,316.9904081)(879.16107422,316.95041504)
\curveto(879.28107179,316.94040815)(879.37607169,316.9854081)(879.44607422,317.08541504)
\curveto(879.52607154,317.1854079)(879.60607146,317.26540782)(879.68607422,317.32541504)
\curveto(879.82607124,317.42540766)(879.9660711,317.51540757)(880.10607422,317.59541504)
\curveto(880.25607081,317.6854074)(880.42607064,317.76040733)(880.61607422,317.82041504)
\curveto(880.69607037,317.85040724)(880.78107029,317.87040722)(880.87107422,317.88041504)
\curveto(880.9710701,317.8904072)(881.06607,317.90540718)(881.15607422,317.92541504)
\curveto(881.20606986,317.93540715)(881.25606981,317.94040715)(881.30607422,317.94041504)
\lineto(881.45607422,317.94041504)
}
}
{
\newrgbcolor{curcolor}{0 0 0}
\pscustom[linestyle=none,fillstyle=solid,fillcolor=curcolor]
{
\newpath
\moveto(886.40068359,319.29041504)
\curveto(886.32068247,319.35040574)(886.27568252,319.45540563)(886.26568359,319.60541504)
\lineto(886.26568359,320.07041504)
\lineto(886.26568359,320.32541504)
\curveto(886.26568253,320.41540467)(886.28068251,320.4904046)(886.31068359,320.55041504)
\curveto(886.35068244,320.63040446)(886.43068236,320.6904044)(886.55068359,320.73041504)
\curveto(886.57068222,320.74040435)(886.5906822,320.74040435)(886.61068359,320.73041504)
\curveto(886.64068215,320.73040436)(886.66568213,320.73540435)(886.68568359,320.74541504)
\curveto(886.85568194,320.74540434)(887.01568178,320.74040435)(887.16568359,320.73041504)
\curveto(887.31568148,320.72040437)(887.41568138,320.66040443)(887.46568359,320.55041504)
\curveto(887.4956813,320.4904046)(887.51068128,320.41540467)(887.51068359,320.32541504)
\lineto(887.51068359,320.07041504)
\curveto(887.51068128,319.8904052)(887.50568129,319.72040537)(887.49568359,319.56041504)
\curveto(887.4956813,319.40040569)(887.43068136,319.29540579)(887.30068359,319.24541504)
\curveto(887.25068154,319.22540586)(887.1956816,319.21540587)(887.13568359,319.21541504)
\lineto(886.97068359,319.21541504)
\lineto(886.65568359,319.21541504)
\curveto(886.55568224,319.21540587)(886.47068232,319.24040585)(886.40068359,319.29041504)
\moveto(887.51068359,310.78541504)
\lineto(887.51068359,310.47041504)
\curveto(887.52068127,310.37041472)(887.50068129,310.2904148)(887.45068359,310.23041504)
\curveto(887.42068137,310.17041492)(887.37568142,310.13041496)(887.31568359,310.11041504)
\curveto(887.25568154,310.10041499)(887.18568161,310.085415)(887.10568359,310.06541504)
\lineto(886.88068359,310.06541504)
\curveto(886.75068204,310.06541502)(886.63568216,310.07041502)(886.53568359,310.08041504)
\curveto(886.44568235,310.10041499)(886.37568242,310.15041494)(886.32568359,310.23041504)
\curveto(886.28568251,310.2904148)(886.26568253,310.36541472)(886.26568359,310.45541504)
\lineto(886.26568359,310.74041504)
\lineto(886.26568359,317.08541504)
\lineto(886.26568359,317.40041504)
\curveto(886.26568253,317.51040758)(886.2906825,317.59540749)(886.34068359,317.65541504)
\curveto(886.37068242,317.70540738)(886.41068238,317.73540735)(886.46068359,317.74541504)
\curveto(886.51068228,317.75540733)(886.56568223,317.77040732)(886.62568359,317.79041504)
\curveto(886.64568215,317.7904073)(886.66568213,317.7854073)(886.68568359,317.77541504)
\curveto(886.71568208,317.77540731)(886.74068205,317.78040731)(886.76068359,317.79041504)
\curveto(886.8906819,317.7904073)(887.02068177,317.7854073)(887.15068359,317.77541504)
\curveto(887.2906815,317.77540731)(887.38568141,317.73540735)(887.43568359,317.65541504)
\curveto(887.48568131,317.59540749)(887.51068128,317.51540757)(887.51068359,317.41541504)
\lineto(887.51068359,317.13041504)
\lineto(887.51068359,310.78541504)
}
}
{
\newrgbcolor{curcolor}{0 0 0}
\pscustom[linestyle=none,fillstyle=solid,fillcolor=curcolor]
{
\newpath
\moveto(892.59052734,317.97041504)
\curveto(893.33052255,317.98040711)(893.94552194,317.87040722)(894.43552734,317.64041504)
\curveto(894.93552095,317.42040767)(895.33052055,317.085408)(895.62052734,316.63541504)
\curveto(895.75052013,316.43540865)(895.86052002,316.1904089)(895.95052734,315.90041504)
\curveto(895.97051991,315.85040924)(895.9855199,315.7854093)(895.99552734,315.70541504)
\curveto(896.00551988,315.62540946)(896.00051988,315.55540953)(895.98052734,315.49541504)
\curveto(895.95051993,315.44540964)(895.90051998,315.40040969)(895.83052734,315.36041504)
\curveto(895.80052008,315.34040975)(895.77052011,315.33040976)(895.74052734,315.33041504)
\curveto(895.71052017,315.34040975)(895.67552021,315.34040975)(895.63552734,315.33041504)
\curveto(895.59552029,315.32040977)(895.55552033,315.31540977)(895.51552734,315.31541504)
\curveto(895.47552041,315.32540976)(895.43552045,315.33040976)(895.39552734,315.33041504)
\lineto(895.08052734,315.33041504)
\curveto(894.9805209,315.34040975)(894.89552099,315.37040972)(894.82552734,315.42041504)
\curveto(894.74552114,315.48040961)(894.69052119,315.56540952)(894.66052734,315.67541504)
\curveto(894.63052125,315.7854093)(894.59052129,315.88040921)(894.54052734,315.96041504)
\curveto(894.39052149,316.22040887)(894.19552169,316.42540866)(893.95552734,316.57541504)
\curveto(893.87552201,316.62540846)(893.79052209,316.66540842)(893.70052734,316.69541504)
\curveto(893.61052227,316.73540835)(893.51552237,316.77040832)(893.41552734,316.80041504)
\curveto(893.27552261,316.84040825)(893.09052279,316.86040823)(892.86052734,316.86041504)
\curveto(892.63052325,316.87040822)(892.44052344,316.85040824)(892.29052734,316.80041504)
\curveto(892.22052366,316.78040831)(892.15552373,316.76540832)(892.09552734,316.75541504)
\curveto(892.03552385,316.74540834)(891.97052391,316.73040836)(891.90052734,316.71041504)
\curveto(891.64052424,316.60040849)(891.41052447,316.45040864)(891.21052734,316.26041504)
\curveto(891.01052487,316.07040902)(890.85552503,315.84540924)(890.74552734,315.58541504)
\curveto(890.70552518,315.49540959)(890.67052521,315.40040969)(890.64052734,315.30041504)
\curveto(890.61052527,315.21040988)(890.5805253,315.11040998)(890.55052734,315.00041504)
\lineto(890.46052734,314.59541504)
\curveto(890.45052543,314.54541054)(890.44552544,314.4904106)(890.44552734,314.43041504)
\curveto(890.45552543,314.37041072)(890.45052543,314.31541077)(890.43052734,314.26541504)
\lineto(890.43052734,314.14541504)
\curveto(890.42052546,314.10541098)(890.41052547,314.04041105)(890.40052734,313.95041504)
\curveto(890.40052548,313.86041123)(890.41052547,313.79541129)(890.43052734,313.75541504)
\curveto(890.44052544,313.70541138)(890.44052544,313.65541143)(890.43052734,313.60541504)
\curveto(890.42052546,313.55541153)(890.42052546,313.50541158)(890.43052734,313.45541504)
\curveto(890.44052544,313.41541167)(890.44552544,313.34541174)(890.44552734,313.24541504)
\curveto(890.46552542,313.16541192)(890.4805254,313.08041201)(890.49052734,312.99041504)
\curveto(890.51052537,312.90041219)(890.53052535,312.81541227)(890.55052734,312.73541504)
\curveto(890.66052522,312.41541267)(890.7855251,312.13541295)(890.92552734,311.89541504)
\curveto(891.07552481,311.66541342)(891.2805246,311.46541362)(891.54052734,311.29541504)
\curveto(891.63052425,311.24541384)(891.72052416,311.20041389)(891.81052734,311.16041504)
\curveto(891.91052397,311.12041397)(892.01552387,311.08041401)(892.12552734,311.04041504)
\curveto(892.17552371,311.03041406)(892.21552367,311.02541406)(892.24552734,311.02541504)
\curveto(892.27552361,311.02541406)(892.31552357,311.02041407)(892.36552734,311.01041504)
\curveto(892.39552349,311.00041409)(892.44552344,310.99541409)(892.51552734,310.99541504)
\lineto(892.68052734,310.99541504)
\curveto(892.6805232,310.9854141)(892.70052318,310.98041411)(892.74052734,310.98041504)
\curveto(892.76052312,310.9904141)(892.7855231,310.9904141)(892.81552734,310.98041504)
\curveto(892.84552304,310.98041411)(892.87552301,310.9854141)(892.90552734,310.99541504)
\curveto(892.97552291,311.01541407)(893.04052284,311.02041407)(893.10052734,311.01041504)
\curveto(893.17052271,311.01041408)(893.24052264,311.02041407)(893.31052734,311.04041504)
\curveto(893.57052231,311.12041397)(893.79552209,311.22041387)(893.98552734,311.34041504)
\curveto(894.17552171,311.47041362)(894.33552155,311.63541345)(894.46552734,311.83541504)
\curveto(894.51552137,311.91541317)(894.56052132,312.00041309)(894.60052734,312.09041504)
\lineto(894.72052734,312.36041504)
\curveto(894.74052114,312.44041265)(894.76052112,312.51541257)(894.78052734,312.58541504)
\curveto(894.81052107,312.66541242)(894.86052102,312.73041236)(894.93052734,312.78041504)
\curveto(894.96052092,312.81041228)(895.02052086,312.83041226)(895.11052734,312.84041504)
\curveto(895.20052068,312.86041223)(895.29552059,312.87041222)(895.39552734,312.87041504)
\curveto(895.50552038,312.88041221)(895.60552028,312.88041221)(895.69552734,312.87041504)
\curveto(895.79552009,312.86041223)(895.86552002,312.84041225)(895.90552734,312.81041504)
\curveto(895.96551992,312.77041232)(896.00051988,312.71041238)(896.01052734,312.63041504)
\curveto(896.03051985,312.55041254)(896.03051985,312.46541262)(896.01052734,312.37541504)
\curveto(895.96051992,312.22541286)(895.91051997,312.08041301)(895.86052734,311.94041504)
\curveto(895.82052006,311.81041328)(895.76552012,311.68041341)(895.69552734,311.55041504)
\curveto(895.54552034,311.25041384)(895.35552053,310.9854141)(895.12552734,310.75541504)
\curveto(894.90552098,310.52541456)(894.63552125,310.34041475)(894.31552734,310.20041504)
\curveto(894.23552165,310.16041493)(894.15052173,310.12541496)(894.06052734,310.09541504)
\curveto(893.97052191,310.07541501)(893.87552201,310.05041504)(893.77552734,310.02041504)
\curveto(893.66552222,309.98041511)(893.55552233,309.96041513)(893.44552734,309.96041504)
\curveto(893.33552255,309.95041514)(893.22552266,309.93541515)(893.11552734,309.91541504)
\curveto(893.07552281,309.89541519)(893.03552285,309.8904152)(892.99552734,309.90041504)
\curveto(892.95552293,309.91041518)(892.91552297,309.91041518)(892.87552734,309.90041504)
\lineto(892.74052734,309.90041504)
\lineto(892.50052734,309.90041504)
\curveto(892.43052345,309.8904152)(892.36552352,309.89541519)(892.30552734,309.91541504)
\lineto(892.23052734,309.91541504)
\lineto(891.87052734,309.96041504)
\curveto(891.74052414,310.00041509)(891.61552427,310.03541505)(891.49552734,310.06541504)
\curveto(891.37552451,310.09541499)(891.26052462,310.13541495)(891.15052734,310.18541504)
\curveto(890.79052509,310.34541474)(890.49052539,310.53541455)(890.25052734,310.75541504)
\curveto(890.02052586,310.97541411)(889.80552608,311.24541384)(889.60552734,311.56541504)
\curveto(889.55552633,311.64541344)(889.51052637,311.73541335)(889.47052734,311.83541504)
\lineto(889.35052734,312.13541504)
\curveto(889.30052658,312.24541284)(889.26552662,312.36041273)(889.24552734,312.48041504)
\curveto(889.22552666,312.60041249)(889.20052668,312.72041237)(889.17052734,312.84041504)
\curveto(889.16052672,312.88041221)(889.15552673,312.92041217)(889.15552734,312.96041504)
\curveto(889.15552673,313.00041209)(889.15052673,313.04041205)(889.14052734,313.08041504)
\curveto(889.12052676,313.14041195)(889.11052677,313.20541188)(889.11052734,313.27541504)
\curveto(889.12052676,313.34541174)(889.11552677,313.41041168)(889.09552734,313.47041504)
\lineto(889.09552734,313.62041504)
\curveto(889.0855268,313.67041142)(889.0805268,313.74041135)(889.08052734,313.83041504)
\curveto(889.0805268,313.92041117)(889.0855268,313.9904111)(889.09552734,314.04041504)
\curveto(889.10552678,314.090411)(889.10552678,314.13541095)(889.09552734,314.17541504)
\curveto(889.09552679,314.21541087)(889.10052678,314.25541083)(889.11052734,314.29541504)
\curveto(889.13052675,314.36541072)(889.13552675,314.43541065)(889.12552734,314.50541504)
\curveto(889.12552676,314.57541051)(889.13552675,314.64041045)(889.15552734,314.70041504)
\curveto(889.19552669,314.87041022)(889.23052665,315.04041005)(889.26052734,315.21041504)
\curveto(889.29052659,315.38040971)(889.33552655,315.54040955)(889.39552734,315.69041504)
\curveto(889.60552628,316.21040888)(889.86052602,316.63040846)(890.16052734,316.95041504)
\curveto(890.46052542,317.27040782)(890.87052501,317.53540755)(891.39052734,317.74541504)
\curveto(891.50052438,317.79540729)(891.62052426,317.83040726)(891.75052734,317.85041504)
\curveto(891.880524,317.87040722)(892.01552387,317.89540719)(892.15552734,317.92541504)
\curveto(892.22552366,317.93540715)(892.29552359,317.94040715)(892.36552734,317.94041504)
\curveto(892.43552345,317.95040714)(892.51052337,317.96040713)(892.59052734,317.97041504)
}
}
{
\newrgbcolor{curcolor}{0 0 0}
\pscustom[linestyle=none,fillstyle=solid,fillcolor=curcolor]
{
\newpath
\moveto(904.63716797,314.26541504)
\curveto(904.65715991,314.20541088)(904.6671599,314.11041098)(904.66716797,313.98041504)
\curveto(904.6671599,313.86041123)(904.6621599,313.77541131)(904.65216797,313.72541504)
\lineto(904.65216797,313.57541504)
\curveto(904.64215992,313.49541159)(904.63215993,313.42041167)(904.62216797,313.35041504)
\curveto(904.62215994,313.2904118)(904.61715995,313.22041187)(904.60716797,313.14041504)
\curveto(904.58715998,313.08041201)(904.57215999,313.02041207)(904.56216797,312.96041504)
\curveto(904.56216,312.90041219)(904.55216001,312.84041225)(904.53216797,312.78041504)
\curveto(904.49216007,312.65041244)(904.45716011,312.52041257)(904.42716797,312.39041504)
\curveto(904.39716017,312.26041283)(904.35716021,312.14041295)(904.30716797,312.03041504)
\curveto(904.09716047,311.55041354)(903.81716075,311.14541394)(903.46716797,310.81541504)
\curveto(903.11716145,310.49541459)(902.68716188,310.25041484)(902.17716797,310.08041504)
\curveto(902.0671625,310.04041505)(901.94716262,310.01041508)(901.81716797,309.99041504)
\curveto(901.69716287,309.97041512)(901.57216299,309.95041514)(901.44216797,309.93041504)
\curveto(901.38216318,309.92041517)(901.31716325,309.91541517)(901.24716797,309.91541504)
\curveto(901.18716338,309.90541518)(901.12716344,309.90041519)(901.06716797,309.90041504)
\curveto(901.02716354,309.8904152)(900.9671636,309.8854152)(900.88716797,309.88541504)
\curveto(900.81716375,309.8854152)(900.7671638,309.8904152)(900.73716797,309.90041504)
\curveto(900.69716387,309.91041518)(900.65716391,309.91541517)(900.61716797,309.91541504)
\curveto(900.57716399,309.90541518)(900.54216402,309.90541518)(900.51216797,309.91541504)
\lineto(900.42216797,309.91541504)
\lineto(900.06216797,309.96041504)
\curveto(899.92216464,310.00041509)(899.78716478,310.04041505)(899.65716797,310.08041504)
\curveto(899.52716504,310.12041497)(899.40216516,310.16541492)(899.28216797,310.21541504)
\curveto(898.83216573,310.41541467)(898.4621661,310.67541441)(898.17216797,310.99541504)
\curveto(897.88216668,311.31541377)(897.64216692,311.70541338)(897.45216797,312.16541504)
\curveto(897.40216716,312.26541282)(897.3621672,312.36541272)(897.33216797,312.46541504)
\curveto(897.31216725,312.56541252)(897.29216727,312.67041242)(897.27216797,312.78041504)
\curveto(897.25216731,312.82041227)(897.24216732,312.85041224)(897.24216797,312.87041504)
\curveto(897.25216731,312.90041219)(897.25216731,312.93541215)(897.24216797,312.97541504)
\curveto(897.22216734,313.05541203)(897.20716736,313.13541195)(897.19716797,313.21541504)
\curveto(897.19716737,313.30541178)(897.18716738,313.3904117)(897.16716797,313.47041504)
\lineto(897.16716797,313.59041504)
\curveto(897.1671674,313.63041146)(897.1621674,313.67541141)(897.15216797,313.72541504)
\curveto(897.14216742,313.77541131)(897.13716743,313.86041123)(897.13716797,313.98041504)
\curveto(897.13716743,314.11041098)(897.14716742,314.20541088)(897.16716797,314.26541504)
\curveto(897.18716738,314.33541075)(897.19216737,314.40541068)(897.18216797,314.47541504)
\curveto(897.17216739,314.54541054)(897.17716739,314.61541047)(897.19716797,314.68541504)
\curveto(897.20716736,314.73541035)(897.21216735,314.77541031)(897.21216797,314.80541504)
\curveto(897.22216734,314.84541024)(897.23216733,314.8904102)(897.24216797,314.94041504)
\curveto(897.27216729,315.06041003)(897.29716727,315.18040991)(897.31716797,315.30041504)
\curveto(897.34716722,315.42040967)(897.38716718,315.53540955)(897.43716797,315.64541504)
\curveto(897.58716698,316.01540907)(897.7671668,316.34540874)(897.97716797,316.63541504)
\curveto(898.19716637,316.93540815)(898.4621661,317.1854079)(898.77216797,317.38541504)
\curveto(898.89216567,317.46540762)(899.01716555,317.53040756)(899.14716797,317.58041504)
\curveto(899.27716529,317.64040745)(899.41216515,317.70040739)(899.55216797,317.76041504)
\curveto(899.67216489,317.81040728)(899.80216476,317.84040725)(899.94216797,317.85041504)
\curveto(900.08216448,317.87040722)(900.22216434,317.90040719)(900.36216797,317.94041504)
\lineto(900.55716797,317.94041504)
\curveto(900.62716394,317.95040714)(900.69216387,317.96040713)(900.75216797,317.97041504)
\curveto(901.64216292,317.98040711)(902.38216218,317.79540729)(902.97216797,317.41541504)
\curveto(903.562161,317.03540805)(903.98716058,316.54040855)(904.24716797,315.93041504)
\curveto(904.29716027,315.83040926)(904.33716023,315.73040936)(904.36716797,315.63041504)
\curveto(904.39716017,315.53040956)(904.43216013,315.42540966)(904.47216797,315.31541504)
\curveto(904.50216006,315.20540988)(904.52716004,315.08541)(904.54716797,314.95541504)
\curveto(904.56716,314.83541025)(904.59215997,314.71041038)(904.62216797,314.58041504)
\curveto(904.63215993,314.53041056)(904.63215993,314.47541061)(904.62216797,314.41541504)
\curveto(904.62215994,314.36541072)(904.62715994,314.31541077)(904.63716797,314.26541504)
\moveto(903.30216797,313.41041504)
\curveto(903.32216124,313.48041161)(903.32716124,313.56041153)(903.31716797,313.65041504)
\lineto(903.31716797,313.90541504)
\curveto(903.31716125,314.29541079)(903.28216128,314.62541046)(903.21216797,314.89541504)
\curveto(903.18216138,314.97541011)(903.15716141,315.05541003)(903.13716797,315.13541504)
\curveto(903.11716145,315.21540987)(903.09216147,315.2904098)(903.06216797,315.36041504)
\curveto(902.78216178,316.01040908)(902.33716223,316.46040863)(901.72716797,316.71041504)
\curveto(901.65716291,316.74040835)(901.58216298,316.76040833)(901.50216797,316.77041504)
\lineto(901.26216797,316.83041504)
\curveto(901.18216338,316.85040824)(901.09716347,316.86040823)(901.00716797,316.86041504)
\lineto(900.73716797,316.86041504)
\lineto(900.46716797,316.81541504)
\curveto(900.3671642,316.79540829)(900.27216429,316.77040832)(900.18216797,316.74041504)
\curveto(900.10216446,316.72040837)(900.02216454,316.6904084)(899.94216797,316.65041504)
\curveto(899.87216469,316.63040846)(899.80716476,316.60040849)(899.74716797,316.56041504)
\curveto(899.68716488,316.52040857)(899.63216493,316.48040861)(899.58216797,316.44041504)
\curveto(899.34216522,316.27040882)(899.14716542,316.06540902)(898.99716797,315.82541504)
\curveto(898.84716572,315.5854095)(898.71716585,315.30540978)(898.60716797,314.98541504)
\curveto(898.57716599,314.8854102)(898.55716601,314.78041031)(898.54716797,314.67041504)
\curveto(898.53716603,314.57041052)(898.52216604,314.46541062)(898.50216797,314.35541504)
\curveto(898.49216607,314.31541077)(898.48716608,314.25041084)(898.48716797,314.16041504)
\curveto(898.47716609,314.13041096)(898.47216609,314.09541099)(898.47216797,314.05541504)
\curveto(898.48216608,314.01541107)(898.48716608,313.97041112)(898.48716797,313.92041504)
\lineto(898.48716797,313.62041504)
\curveto(898.48716608,313.52041157)(898.49716607,313.43041166)(898.51716797,313.35041504)
\lineto(898.54716797,313.17041504)
\curveto(898.567166,313.07041202)(898.58216598,312.97041212)(898.59216797,312.87041504)
\curveto(898.61216595,312.78041231)(898.64216592,312.69541239)(898.68216797,312.61541504)
\curveto(898.78216578,312.37541271)(898.89716567,312.15041294)(899.02716797,311.94041504)
\curveto(899.1671654,311.73041336)(899.33716523,311.55541353)(899.53716797,311.41541504)
\curveto(899.58716498,311.3854137)(899.63216493,311.36041373)(899.67216797,311.34041504)
\curveto(899.71216485,311.32041377)(899.75716481,311.29541379)(899.80716797,311.26541504)
\curveto(899.88716468,311.21541387)(899.97216459,311.17041392)(900.06216797,311.13041504)
\curveto(900.1621644,311.10041399)(900.2671643,311.07041402)(900.37716797,311.04041504)
\curveto(900.42716414,311.02041407)(900.47216409,311.01041408)(900.51216797,311.01041504)
\curveto(900.562164,311.02041407)(900.61216395,311.02041407)(900.66216797,311.01041504)
\curveto(900.69216387,311.00041409)(900.75216381,310.9904141)(900.84216797,310.98041504)
\curveto(900.94216362,310.97041412)(901.01716355,310.97541411)(901.06716797,310.99541504)
\curveto(901.10716346,311.00541408)(901.14716342,311.00541408)(901.18716797,310.99541504)
\curveto(901.22716334,310.99541409)(901.2671633,311.00541408)(901.30716797,311.02541504)
\curveto(901.38716318,311.04541404)(901.4671631,311.06041403)(901.54716797,311.07041504)
\curveto(901.62716294,311.090414)(901.70216286,311.11541397)(901.77216797,311.14541504)
\curveto(902.11216245,311.2854138)(902.38716218,311.48041361)(902.59716797,311.73041504)
\curveto(902.80716176,311.98041311)(902.98216158,312.27541281)(903.12216797,312.61541504)
\curveto(903.17216139,312.73541235)(903.20216136,312.86041223)(903.21216797,312.99041504)
\curveto(903.23216133,313.13041196)(903.2621613,313.27041182)(903.30216797,313.41041504)
}
}
{
\newrgbcolor{curcolor}{0 0 0}
\pscustom[linestyle=none,fillstyle=solid,fillcolor=curcolor]
{
\newpath
\moveto(831.9614209,276.63770508)
\lineto(832.3964209,276.63770508)
\curveto(832.54641893,276.63769734)(832.65141883,276.59769738)(832.7114209,276.51770508)
\curveto(832.76141872,276.43769754)(832.78641869,276.33769764)(832.7864209,276.21770508)
\curveto(832.79641868,276.09769788)(832.80141868,275.977698)(832.8014209,275.85770508)
\lineto(832.8014209,274.43270508)
\lineto(832.8014209,272.16770508)
\lineto(832.8014209,271.47770508)
\curveto(832.80141868,271.24770273)(832.82641865,271.04770293)(832.8764209,270.87770508)
\curveto(833.03641844,270.42770355)(833.33641814,270.11270386)(833.7764209,269.93270508)
\curveto(833.99641748,269.84270413)(834.26141722,269.80770417)(834.5714209,269.82770508)
\curveto(834.8814166,269.85770412)(835.13141635,269.91270406)(835.3214209,269.99270508)
\curveto(835.65141583,270.13270384)(835.91141557,270.30770367)(836.1014209,270.51770508)
\curveto(836.30141518,270.73770324)(836.45641502,271.02270295)(836.5664209,271.37270508)
\curveto(836.59641488,271.45270252)(836.61641486,271.53270244)(836.6264209,271.61270508)
\curveto(836.63641484,271.69270228)(836.65141483,271.7777022)(836.6714209,271.86770508)
\curveto(836.6814148,271.91770206)(836.6814148,271.96270201)(836.6714209,272.00270508)
\curveto(836.67141481,272.04270193)(836.6814148,272.08770189)(836.7014209,272.13770508)
\lineto(836.7014209,272.45270508)
\curveto(836.72141476,272.53270144)(836.72641475,272.62270135)(836.7164209,272.72270508)
\curveto(836.70641477,272.83270114)(836.70141478,272.93270104)(836.7014209,273.02270508)
\lineto(836.7014209,274.19270508)
\lineto(836.7014209,275.78270508)
\curveto(836.70141478,275.90269807)(836.69641478,276.02769795)(836.6864209,276.15770508)
\curveto(836.68641479,276.29769768)(836.71141477,276.40769757)(836.7614209,276.48770508)
\curveto(836.80141468,276.53769744)(836.84641463,276.56769741)(836.8964209,276.57770508)
\curveto(836.95641452,276.59769738)(837.02641445,276.61769736)(837.1064209,276.63770508)
\lineto(837.3314209,276.63770508)
\curveto(837.45141403,276.63769734)(837.55641392,276.63269734)(837.6464209,276.62270508)
\curveto(837.74641373,276.61269736)(837.82141366,276.56769741)(837.8714209,276.48770508)
\curveto(837.92141356,276.43769754)(837.94641353,276.36269761)(837.9464209,276.26270508)
\lineto(837.9464209,275.97770508)
\lineto(837.9464209,274.95770508)
\lineto(837.9464209,270.92270508)
\lineto(837.9464209,269.57270508)
\curveto(837.94641353,269.45270452)(837.94141354,269.33770464)(837.9314209,269.22770508)
\curveto(837.93141355,269.12770485)(837.89641358,269.05270492)(837.8264209,269.00270508)
\curveto(837.78641369,268.972705)(837.72641375,268.94770503)(837.6464209,268.92770508)
\curveto(837.56641391,268.91770506)(837.476414,268.90770507)(837.3764209,268.89770508)
\curveto(837.28641419,268.89770508)(837.19641428,268.90270507)(837.1064209,268.91270508)
\curveto(837.02641445,268.92270505)(836.96641451,268.94270503)(836.9264209,268.97270508)
\curveto(836.8764146,269.01270496)(836.83141465,269.0777049)(836.7914209,269.16770508)
\curveto(836.7814147,269.20770477)(836.77141471,269.26270471)(836.7614209,269.33270508)
\curveto(836.76141472,269.40270457)(836.75641472,269.46770451)(836.7464209,269.52770508)
\curveto(836.73641474,269.59770438)(836.71641476,269.65270432)(836.6864209,269.69270508)
\curveto(836.65641482,269.73270424)(836.61141487,269.74770423)(836.5514209,269.73770508)
\curveto(836.47141501,269.71770426)(836.39141509,269.65770432)(836.3114209,269.55770508)
\curveto(836.23141525,269.46770451)(836.15641532,269.39770458)(836.0864209,269.34770508)
\curveto(835.86641561,269.18770479)(835.61641586,269.04770493)(835.3364209,268.92770508)
\curveto(835.22641625,268.8777051)(835.11141637,268.84770513)(834.9914209,268.83770508)
\curveto(834.8814166,268.81770516)(834.76641671,268.79270518)(834.6464209,268.76270508)
\curveto(834.59641688,268.75270522)(834.54141694,268.75270522)(834.4814209,268.76270508)
\curveto(834.43141705,268.7727052)(834.3814171,268.76770521)(834.3314209,268.74770508)
\curveto(834.23141725,268.72770525)(834.14141734,268.72770525)(834.0614209,268.74770508)
\lineto(833.9114209,268.74770508)
\curveto(833.86141762,268.76770521)(833.80141768,268.7777052)(833.7314209,268.77770508)
\curveto(833.67141781,268.7777052)(833.61641786,268.78270519)(833.5664209,268.79270508)
\curveto(833.52641795,268.81270516)(833.48641799,268.82270515)(833.4464209,268.82270508)
\curveto(833.41641806,268.81270516)(833.3764181,268.81770516)(833.3264209,268.83770508)
\lineto(833.0864209,268.89770508)
\curveto(833.01641846,268.91770506)(832.94141854,268.94770503)(832.8614209,268.98770508)
\curveto(832.60141888,269.09770488)(832.3814191,269.24270473)(832.2014209,269.42270508)
\curveto(832.03141945,269.61270436)(831.89141959,269.83770414)(831.7814209,270.09770508)
\curveto(831.74141974,270.18770379)(831.71141977,270.2777037)(831.6914209,270.36770508)
\lineto(831.6314209,270.66770508)
\curveto(831.61141987,270.72770325)(831.60141988,270.78270319)(831.6014209,270.83270508)
\curveto(831.61141987,270.89270308)(831.60641987,270.95770302)(831.5864209,271.02770508)
\curveto(831.5764199,271.04770293)(831.57141991,271.0727029)(831.5714209,271.10270508)
\curveto(831.57141991,271.14270283)(831.56641991,271.1777028)(831.5564209,271.20770508)
\lineto(831.5564209,271.35770508)
\curveto(831.54641993,271.39770258)(831.54141994,271.44270253)(831.5414209,271.49270508)
\curveto(831.55141993,271.55270242)(831.55641992,271.60770237)(831.5564209,271.65770508)
\lineto(831.5564209,272.25770508)
\lineto(831.5564209,275.01770508)
\lineto(831.5564209,275.97770508)
\lineto(831.5564209,276.24770508)
\curveto(831.55641992,276.33769764)(831.5764199,276.41269756)(831.6164209,276.47270508)
\curveto(831.65641982,276.54269743)(831.73141975,276.59269738)(831.8414209,276.62270508)
\curveto(831.86141962,276.63269734)(831.8814196,276.63269734)(831.9014209,276.62270508)
\curveto(831.92141956,276.62269735)(831.94141954,276.62769735)(831.9614209,276.63770508)
}
}
{
\newrgbcolor{curcolor}{0 0 0}
\pscustom[linestyle=none,fillstyle=solid,fillcolor=curcolor]
{
\newpath
\moveto(842.27603027,276.81770508)
\curveto(842.99602621,276.82769715)(843.6010256,276.74269723)(844.09103027,276.56270508)
\curveto(844.58102462,276.39269758)(844.96102424,276.08769789)(845.23103027,275.64770508)
\curveto(845.3010239,275.53769844)(845.35602385,275.42269855)(845.39603027,275.30270508)
\curveto(845.43602377,275.19269878)(845.47602373,275.06769891)(845.51603027,274.92770508)
\curveto(845.53602367,274.85769912)(845.54102366,274.78269919)(845.53103027,274.70270508)
\curveto(845.52102368,274.63269934)(845.5060237,274.5776994)(845.48603027,274.53770508)
\curveto(845.46602374,274.51769946)(845.44102376,274.49769948)(845.41103027,274.47770508)
\curveto(845.38102382,274.46769951)(845.35602385,274.45269952)(845.33603027,274.43270508)
\curveto(845.28602392,274.41269956)(845.23602397,274.40769957)(845.18603027,274.41770508)
\curveto(845.13602407,274.42769955)(845.08602412,274.42769955)(845.03603027,274.41770508)
\curveto(844.95602425,274.39769958)(844.85102435,274.39269958)(844.72103027,274.40270508)
\curveto(844.59102461,274.42269955)(844.5010247,274.44769953)(844.45103027,274.47770508)
\curveto(844.37102483,274.52769945)(844.31602489,274.59269938)(844.28603027,274.67270508)
\curveto(844.26602494,274.76269921)(844.23102497,274.84769913)(844.18103027,274.92770508)
\curveto(844.09102511,275.08769889)(843.96602524,275.23269874)(843.80603027,275.36270508)
\curveto(843.69602551,275.44269853)(843.57602563,275.50269847)(843.44603027,275.54270508)
\curveto(843.31602589,275.58269839)(843.17602603,275.62269835)(843.02603027,275.66270508)
\curveto(842.97602623,275.68269829)(842.92602628,275.68769829)(842.87603027,275.67770508)
\curveto(842.82602638,275.6776983)(842.77602643,275.68269829)(842.72603027,275.69270508)
\curveto(842.66602654,275.71269826)(842.59102661,275.72269825)(842.50103027,275.72270508)
\curveto(842.41102679,275.72269825)(842.33602687,275.71269826)(842.27603027,275.69270508)
\lineto(842.18603027,275.69270508)
\lineto(842.03603027,275.66270508)
\curveto(841.98602722,275.66269831)(841.93602727,275.65769832)(841.88603027,275.64770508)
\curveto(841.62602758,275.58769839)(841.41102779,275.50269847)(841.24103027,275.39270508)
\curveto(841.07102813,275.28269869)(840.95602825,275.09769888)(840.89603027,274.83770508)
\curveto(840.87602833,274.76769921)(840.87102833,274.69769928)(840.88103027,274.62770508)
\curveto(840.9010283,274.55769942)(840.92102828,274.49769948)(840.94103027,274.44770508)
\curveto(841.0010282,274.29769968)(841.07102813,274.18769979)(841.15103027,274.11770508)
\curveto(841.24102796,274.05769992)(841.35102785,273.98769999)(841.48103027,273.90770508)
\curveto(841.64102756,273.80770017)(841.82102738,273.73270024)(842.02103027,273.68270508)
\curveto(842.22102698,273.64270033)(842.42102678,273.59270038)(842.62103027,273.53270508)
\curveto(842.75102645,273.49270048)(842.88102632,273.46270051)(843.01103027,273.44270508)
\curveto(843.14102606,273.42270055)(843.27102593,273.39270058)(843.40103027,273.35270508)
\curveto(843.61102559,273.29270068)(843.81602539,273.23270074)(844.01603027,273.17270508)
\curveto(844.21602499,273.12270085)(844.41602479,273.05770092)(844.61603027,272.97770508)
\lineto(844.76603027,272.91770508)
\curveto(844.81602439,272.89770108)(844.86602434,272.8727011)(844.91603027,272.84270508)
\curveto(845.11602409,272.72270125)(845.29102391,272.58770139)(845.44103027,272.43770508)
\curveto(845.59102361,272.28770169)(845.71602349,272.09770188)(845.81603027,271.86770508)
\curveto(845.83602337,271.79770218)(845.85602335,271.70270227)(845.87603027,271.58270508)
\curveto(845.89602331,271.51270246)(845.9060233,271.43770254)(845.90603027,271.35770508)
\curveto(845.91602329,271.28770269)(845.92102328,271.20770277)(845.92103027,271.11770508)
\lineto(845.92103027,270.96770508)
\curveto(845.9010233,270.89770308)(845.89102331,270.82770315)(845.89103027,270.75770508)
\curveto(845.89102331,270.68770329)(845.88102332,270.61770336)(845.86103027,270.54770508)
\curveto(845.83102337,270.43770354)(845.79602341,270.33270364)(845.75603027,270.23270508)
\curveto(845.71602349,270.13270384)(845.67102353,270.04270393)(845.62103027,269.96270508)
\curveto(845.46102374,269.70270427)(845.25602395,269.49270448)(845.00603027,269.33270508)
\curveto(844.75602445,269.18270479)(844.47602473,269.05270492)(844.16603027,268.94270508)
\curveto(844.07602513,268.91270506)(843.98102522,268.89270508)(843.88103027,268.88270508)
\curveto(843.79102541,268.86270511)(843.7010255,268.83770514)(843.61103027,268.80770508)
\curveto(843.51102569,268.78770519)(843.41102579,268.7777052)(843.31103027,268.77770508)
\curveto(843.21102599,268.7777052)(843.11102609,268.76770521)(843.01103027,268.74770508)
\lineto(842.86103027,268.74770508)
\curveto(842.81102639,268.73770524)(842.74102646,268.73270524)(842.65103027,268.73270508)
\curveto(842.56102664,268.73270524)(842.49102671,268.73770524)(842.44103027,268.74770508)
\lineto(842.27603027,268.74770508)
\curveto(842.21602699,268.76770521)(842.15102705,268.7777052)(842.08103027,268.77770508)
\curveto(842.01102719,268.76770521)(841.95102725,268.7727052)(841.90103027,268.79270508)
\curveto(841.85102735,268.80270517)(841.78602742,268.80770517)(841.70603027,268.80770508)
\lineto(841.46603027,268.86770508)
\curveto(841.39602781,268.8777051)(841.32102788,268.89770508)(841.24103027,268.92770508)
\curveto(840.93102827,269.02770495)(840.66102854,269.15270482)(840.43103027,269.30270508)
\curveto(840.201029,269.45270452)(840.0010292,269.64770433)(839.83103027,269.88770508)
\curveto(839.74102946,270.01770396)(839.66602954,270.15270382)(839.60603027,270.29270508)
\curveto(839.54602966,270.43270354)(839.49102971,270.58770339)(839.44103027,270.75770508)
\curveto(839.42102978,270.81770316)(839.41102979,270.88770309)(839.41103027,270.96770508)
\curveto(839.42102978,271.05770292)(839.43602977,271.12770285)(839.45603027,271.17770508)
\curveto(839.48602972,271.21770276)(839.53602967,271.25770272)(839.60603027,271.29770508)
\curveto(839.65602955,271.31770266)(839.72602948,271.32770265)(839.81603027,271.32770508)
\curveto(839.9060293,271.33770264)(839.99602921,271.33770264)(840.08603027,271.32770508)
\curveto(840.17602903,271.31770266)(840.26102894,271.30270267)(840.34103027,271.28270508)
\curveto(840.43102877,271.2727027)(840.49102871,271.25770272)(840.52103027,271.23770508)
\curveto(840.59102861,271.18770279)(840.63602857,271.11270286)(840.65603027,271.01270508)
\curveto(840.68602852,270.92270305)(840.72102848,270.83770314)(840.76103027,270.75770508)
\curveto(840.86102834,270.53770344)(840.99602821,270.36770361)(841.16603027,270.24770508)
\curveto(841.28602792,270.15770382)(841.42102778,270.08770389)(841.57103027,270.03770508)
\curveto(841.72102748,269.98770399)(841.88102732,269.93770404)(842.05103027,269.88770508)
\lineto(842.36603027,269.84270508)
\lineto(842.45603027,269.84270508)
\curveto(842.52602668,269.82270415)(842.61602659,269.81270416)(842.72603027,269.81270508)
\curveto(842.84602636,269.81270416)(842.94602626,269.82270415)(843.02603027,269.84270508)
\curveto(843.09602611,269.84270413)(843.15102605,269.84770413)(843.19103027,269.85770508)
\curveto(843.25102595,269.86770411)(843.31102589,269.8727041)(843.37103027,269.87270508)
\curveto(843.43102577,269.88270409)(843.48602572,269.89270408)(843.53603027,269.90270508)
\curveto(843.82602538,269.98270399)(844.05602515,270.08770389)(844.22603027,270.21770508)
\curveto(844.39602481,270.34770363)(844.51602469,270.56770341)(844.58603027,270.87770508)
\curveto(844.6060246,270.92770305)(844.61102459,270.98270299)(844.60103027,271.04270508)
\curveto(844.59102461,271.10270287)(844.58102462,271.14770283)(844.57103027,271.17770508)
\curveto(844.52102468,271.36770261)(844.45102475,271.50770247)(844.36103027,271.59770508)
\curveto(844.27102493,271.69770228)(844.15602505,271.78770219)(844.01603027,271.86770508)
\curveto(843.92602528,271.92770205)(843.82602538,271.977702)(843.71603027,272.01770508)
\lineto(843.38603027,272.13770508)
\curveto(843.35602585,272.14770183)(843.32602588,272.15270182)(843.29603027,272.15270508)
\curveto(843.27602593,272.15270182)(843.25102595,272.16270181)(843.22103027,272.18270508)
\curveto(842.88102632,272.29270168)(842.52602668,272.3727016)(842.15603027,272.42270508)
\curveto(841.79602741,272.48270149)(841.45602775,272.5777014)(841.13603027,272.70770508)
\curveto(841.03602817,272.74770123)(840.94102826,272.78270119)(840.85103027,272.81270508)
\curveto(840.76102844,272.84270113)(840.67602853,272.88270109)(840.59603027,272.93270508)
\curveto(840.4060288,273.04270093)(840.23102897,273.16770081)(840.07103027,273.30770508)
\curveto(839.91102929,273.44770053)(839.78602942,273.62270035)(839.69603027,273.83270508)
\curveto(839.66602954,273.90270007)(839.64102956,273.9727)(839.62103027,274.04270508)
\curveto(839.61102959,274.11269986)(839.59602961,274.18769979)(839.57603027,274.26770508)
\curveto(839.54602966,274.38769959)(839.53602967,274.52269945)(839.54603027,274.67270508)
\curveto(839.55602965,274.83269914)(839.57102963,274.96769901)(839.59103027,275.07770508)
\curveto(839.61102959,275.12769885)(839.62102958,275.16769881)(839.62103027,275.19770508)
\curveto(839.63102957,275.23769874)(839.64602956,275.2776987)(839.66603027,275.31770508)
\curveto(839.75602945,275.54769843)(839.87602933,275.74769823)(840.02603027,275.91770508)
\curveto(840.18602902,276.08769789)(840.36602884,276.23769774)(840.56603027,276.36770508)
\curveto(840.71602849,276.45769752)(840.88102832,276.52769745)(841.06103027,276.57770508)
\curveto(841.24102796,276.63769734)(841.43102777,276.69269728)(841.63103027,276.74270508)
\curveto(841.7010275,276.75269722)(841.76602744,276.76269721)(841.82603027,276.77270508)
\curveto(841.89602731,276.78269719)(841.97102723,276.79269718)(842.05103027,276.80270508)
\curveto(842.08102712,276.81269716)(842.12102708,276.81269716)(842.17103027,276.80270508)
\curveto(842.22102698,276.79269718)(842.25602695,276.79769718)(842.27603027,276.81770508)
}
}
{
\newrgbcolor{curcolor}{0 0 0}
\pscustom[linestyle=none,fillstyle=solid,fillcolor=curcolor]
{
\newpath
\moveto(847.81103027,276.63770508)
\lineto(848.24603027,276.63770508)
\curveto(848.39602831,276.63769734)(848.5010282,276.59769738)(848.56103027,276.51770508)
\curveto(848.61102809,276.43769754)(848.63602807,276.33769764)(848.63603027,276.21770508)
\curveto(848.64602806,276.09769788)(848.65102805,275.977698)(848.65103027,275.85770508)
\lineto(848.65103027,274.43270508)
\lineto(848.65103027,272.16770508)
\lineto(848.65103027,271.47770508)
\curveto(848.65102805,271.24770273)(848.67602803,271.04770293)(848.72603027,270.87770508)
\curveto(848.88602782,270.42770355)(849.18602752,270.11270386)(849.62603027,269.93270508)
\curveto(849.84602686,269.84270413)(850.11102659,269.80770417)(850.42103027,269.82770508)
\curveto(850.73102597,269.85770412)(850.98102572,269.91270406)(851.17103027,269.99270508)
\curveto(851.5010252,270.13270384)(851.76102494,270.30770367)(851.95103027,270.51770508)
\curveto(852.15102455,270.73770324)(852.3060244,271.02270295)(852.41603027,271.37270508)
\curveto(852.44602426,271.45270252)(852.46602424,271.53270244)(852.47603027,271.61270508)
\curveto(852.48602422,271.69270228)(852.5010242,271.7777022)(852.52103027,271.86770508)
\curveto(852.53102417,271.91770206)(852.53102417,271.96270201)(852.52103027,272.00270508)
\curveto(852.52102418,272.04270193)(852.53102417,272.08770189)(852.55103027,272.13770508)
\lineto(852.55103027,272.45270508)
\curveto(852.57102413,272.53270144)(852.57602413,272.62270135)(852.56603027,272.72270508)
\curveto(852.55602415,272.83270114)(852.55102415,272.93270104)(852.55103027,273.02270508)
\lineto(852.55103027,274.19270508)
\lineto(852.55103027,275.78270508)
\curveto(852.55102415,275.90269807)(852.54602416,276.02769795)(852.53603027,276.15770508)
\curveto(852.53602417,276.29769768)(852.56102414,276.40769757)(852.61103027,276.48770508)
\curveto(852.65102405,276.53769744)(852.69602401,276.56769741)(852.74603027,276.57770508)
\curveto(852.8060239,276.59769738)(852.87602383,276.61769736)(852.95603027,276.63770508)
\lineto(853.18103027,276.63770508)
\curveto(853.3010234,276.63769734)(853.4060233,276.63269734)(853.49603027,276.62270508)
\curveto(853.59602311,276.61269736)(853.67102303,276.56769741)(853.72103027,276.48770508)
\curveto(853.77102293,276.43769754)(853.79602291,276.36269761)(853.79603027,276.26270508)
\lineto(853.79603027,275.97770508)
\lineto(853.79603027,274.95770508)
\lineto(853.79603027,270.92270508)
\lineto(853.79603027,269.57270508)
\curveto(853.79602291,269.45270452)(853.79102291,269.33770464)(853.78103027,269.22770508)
\curveto(853.78102292,269.12770485)(853.74602296,269.05270492)(853.67603027,269.00270508)
\curveto(853.63602307,268.972705)(853.57602313,268.94770503)(853.49603027,268.92770508)
\curveto(853.41602329,268.91770506)(853.32602338,268.90770507)(853.22603027,268.89770508)
\curveto(853.13602357,268.89770508)(853.04602366,268.90270507)(852.95603027,268.91270508)
\curveto(852.87602383,268.92270505)(852.81602389,268.94270503)(852.77603027,268.97270508)
\curveto(852.72602398,269.01270496)(852.68102402,269.0777049)(852.64103027,269.16770508)
\curveto(852.63102407,269.20770477)(852.62102408,269.26270471)(852.61103027,269.33270508)
\curveto(852.61102409,269.40270457)(852.6060241,269.46770451)(852.59603027,269.52770508)
\curveto(852.58602412,269.59770438)(852.56602414,269.65270432)(852.53603027,269.69270508)
\curveto(852.5060242,269.73270424)(852.46102424,269.74770423)(852.40103027,269.73770508)
\curveto(852.32102438,269.71770426)(852.24102446,269.65770432)(852.16103027,269.55770508)
\curveto(852.08102462,269.46770451)(852.0060247,269.39770458)(851.93603027,269.34770508)
\curveto(851.71602499,269.18770479)(851.46602524,269.04770493)(851.18603027,268.92770508)
\curveto(851.07602563,268.8777051)(850.96102574,268.84770513)(850.84103027,268.83770508)
\curveto(850.73102597,268.81770516)(850.61602609,268.79270518)(850.49603027,268.76270508)
\curveto(850.44602626,268.75270522)(850.39102631,268.75270522)(850.33103027,268.76270508)
\curveto(850.28102642,268.7727052)(850.23102647,268.76770521)(850.18103027,268.74770508)
\curveto(850.08102662,268.72770525)(849.99102671,268.72770525)(849.91103027,268.74770508)
\lineto(849.76103027,268.74770508)
\curveto(849.71102699,268.76770521)(849.65102705,268.7777052)(849.58103027,268.77770508)
\curveto(849.52102718,268.7777052)(849.46602724,268.78270519)(849.41603027,268.79270508)
\curveto(849.37602733,268.81270516)(849.33602737,268.82270515)(849.29603027,268.82270508)
\curveto(849.26602744,268.81270516)(849.22602748,268.81770516)(849.17603027,268.83770508)
\lineto(848.93603027,268.89770508)
\curveto(848.86602784,268.91770506)(848.79102791,268.94770503)(848.71103027,268.98770508)
\curveto(848.45102825,269.09770488)(848.23102847,269.24270473)(848.05103027,269.42270508)
\curveto(847.88102882,269.61270436)(847.74102896,269.83770414)(847.63103027,270.09770508)
\curveto(847.59102911,270.18770379)(847.56102914,270.2777037)(847.54103027,270.36770508)
\lineto(847.48103027,270.66770508)
\curveto(847.46102924,270.72770325)(847.45102925,270.78270319)(847.45103027,270.83270508)
\curveto(847.46102924,270.89270308)(847.45602925,270.95770302)(847.43603027,271.02770508)
\curveto(847.42602928,271.04770293)(847.42102928,271.0727029)(847.42103027,271.10270508)
\curveto(847.42102928,271.14270283)(847.41602929,271.1777028)(847.40603027,271.20770508)
\lineto(847.40603027,271.35770508)
\curveto(847.39602931,271.39770258)(847.39102931,271.44270253)(847.39103027,271.49270508)
\curveto(847.4010293,271.55270242)(847.4060293,271.60770237)(847.40603027,271.65770508)
\lineto(847.40603027,272.25770508)
\lineto(847.40603027,275.01770508)
\lineto(847.40603027,275.97770508)
\lineto(847.40603027,276.24770508)
\curveto(847.4060293,276.33769764)(847.42602928,276.41269756)(847.46603027,276.47270508)
\curveto(847.5060292,276.54269743)(847.58102912,276.59269738)(847.69103027,276.62270508)
\curveto(847.71102899,276.63269734)(847.73102897,276.63269734)(847.75103027,276.62270508)
\curveto(847.77102893,276.62269735)(847.79102891,276.62769735)(847.81103027,276.63770508)
}
}
{
\newrgbcolor{curcolor}{0 0 0}
\pscustom[linestyle=none,fillstyle=solid,fillcolor=curcolor]
{
\newpath
\moveto(862.58063965,269.46770508)
\curveto(862.61063182,269.30770467)(862.59563183,269.1727048)(862.53563965,269.06270508)
\curveto(862.47563195,268.96270501)(862.39563203,268.88770509)(862.29563965,268.83770508)
\curveto(862.24563218,268.81770516)(862.19063224,268.80770517)(862.13063965,268.80770508)
\curveto(862.08063235,268.80770517)(862.0256324,268.79770518)(861.96563965,268.77770508)
\curveto(861.74563268,268.72770525)(861.5256329,268.74270523)(861.30563965,268.82270508)
\curveto(861.09563333,268.89270508)(860.95063348,268.98270499)(860.87063965,269.09270508)
\curveto(860.82063361,269.16270481)(860.77563365,269.24270473)(860.73563965,269.33270508)
\curveto(860.69563373,269.43270454)(860.64563378,269.51270446)(860.58563965,269.57270508)
\curveto(860.56563386,269.59270438)(860.54063389,269.61270436)(860.51063965,269.63270508)
\curveto(860.49063394,269.65270432)(860.46063397,269.65770432)(860.42063965,269.64770508)
\curveto(860.31063412,269.61770436)(860.20563422,269.56270441)(860.10563965,269.48270508)
\curveto(860.01563441,269.40270457)(859.9256345,269.33270464)(859.83563965,269.27270508)
\curveto(859.70563472,269.19270478)(859.56563486,269.11770486)(859.41563965,269.04770508)
\curveto(859.26563516,268.98770499)(859.10563532,268.93270504)(858.93563965,268.88270508)
\curveto(858.83563559,268.85270512)(858.7256357,268.83270514)(858.60563965,268.82270508)
\curveto(858.49563593,268.81270516)(858.38563604,268.79770518)(858.27563965,268.77770508)
\curveto(858.2256362,268.76770521)(858.18063625,268.76270521)(858.14063965,268.76270508)
\lineto(858.03563965,268.76270508)
\curveto(857.9256365,268.74270523)(857.82063661,268.74270523)(857.72063965,268.76270508)
\lineto(857.58563965,268.76270508)
\curveto(857.53563689,268.7727052)(857.48563694,268.7777052)(857.43563965,268.77770508)
\curveto(857.38563704,268.7777052)(857.34063709,268.78770519)(857.30063965,268.80770508)
\curveto(857.26063717,268.81770516)(857.2256372,268.82270515)(857.19563965,268.82270508)
\curveto(857.17563725,268.81270516)(857.15063728,268.81270516)(857.12063965,268.82270508)
\lineto(856.88063965,268.88270508)
\curveto(856.80063763,268.89270508)(856.7256377,268.91270506)(856.65563965,268.94270508)
\curveto(856.35563807,269.0727049)(856.11063832,269.21770476)(855.92063965,269.37770508)
\curveto(855.74063869,269.54770443)(855.59063884,269.78270419)(855.47063965,270.08270508)
\curveto(855.38063905,270.30270367)(855.33563909,270.56770341)(855.33563965,270.87770508)
\lineto(855.33563965,271.19270508)
\curveto(855.34563908,271.24270273)(855.35063908,271.29270268)(855.35063965,271.34270508)
\lineto(855.38063965,271.52270508)
\lineto(855.50063965,271.85270508)
\curveto(855.54063889,271.96270201)(855.59063884,272.06270191)(855.65063965,272.15270508)
\curveto(855.8306386,272.44270153)(856.07563835,272.65770132)(856.38563965,272.79770508)
\curveto(856.69563773,272.93770104)(857.03563739,273.06270091)(857.40563965,273.17270508)
\curveto(857.54563688,273.21270076)(857.69063674,273.24270073)(857.84063965,273.26270508)
\curveto(857.99063644,273.28270069)(858.14063629,273.30770067)(858.29063965,273.33770508)
\curveto(858.36063607,273.35770062)(858.425636,273.36770061)(858.48563965,273.36770508)
\curveto(858.55563587,273.36770061)(858.6306358,273.3777006)(858.71063965,273.39770508)
\curveto(858.78063565,273.41770056)(858.85063558,273.42770055)(858.92063965,273.42770508)
\curveto(858.99063544,273.43770054)(859.06563536,273.45270052)(859.14563965,273.47270508)
\curveto(859.39563503,273.53270044)(859.6306348,273.58270039)(859.85063965,273.62270508)
\curveto(860.07063436,273.6727003)(860.24563418,273.78770019)(860.37563965,273.96770508)
\curveto(860.43563399,274.04769993)(860.48563394,274.14769983)(860.52563965,274.26770508)
\curveto(860.56563386,274.39769958)(860.56563386,274.53769944)(860.52563965,274.68770508)
\curveto(860.46563396,274.92769905)(860.37563405,275.11769886)(860.25563965,275.25770508)
\curveto(860.14563428,275.39769858)(859.98563444,275.50769847)(859.77563965,275.58770508)
\curveto(859.65563477,275.63769834)(859.51063492,275.6726983)(859.34063965,275.69270508)
\curveto(859.18063525,275.71269826)(859.01063542,275.72269825)(858.83063965,275.72270508)
\curveto(858.65063578,275.72269825)(858.47563595,275.71269826)(858.30563965,275.69270508)
\curveto(858.13563629,275.6726983)(857.99063644,275.64269833)(857.87063965,275.60270508)
\curveto(857.70063673,275.54269843)(857.53563689,275.45769852)(857.37563965,275.34770508)
\curveto(857.29563713,275.28769869)(857.22063721,275.20769877)(857.15063965,275.10770508)
\curveto(857.09063734,275.01769896)(857.03563739,274.91769906)(856.98563965,274.80770508)
\curveto(856.95563747,274.72769925)(856.9256375,274.64269933)(856.89563965,274.55270508)
\curveto(856.87563755,274.46269951)(856.8306376,274.39269958)(856.76063965,274.34270508)
\curveto(856.72063771,274.31269966)(856.65063778,274.28769969)(856.55063965,274.26770508)
\curveto(856.46063797,274.25769972)(856.36563806,274.25269972)(856.26563965,274.25270508)
\curveto(856.16563826,274.25269972)(856.06563836,274.25769972)(855.96563965,274.26770508)
\curveto(855.87563855,274.28769969)(855.81063862,274.31269966)(855.77063965,274.34270508)
\curveto(855.7306387,274.3726996)(855.70063873,274.42269955)(855.68063965,274.49270508)
\curveto(855.66063877,274.56269941)(855.66063877,274.63769934)(855.68063965,274.71770508)
\curveto(855.71063872,274.84769913)(855.74063869,274.96769901)(855.77063965,275.07770508)
\curveto(855.81063862,275.19769878)(855.85563857,275.31269866)(855.90563965,275.42270508)
\curveto(856.09563833,275.7726982)(856.33563809,276.04269793)(856.62563965,276.23270508)
\curveto(856.91563751,276.43269754)(857.27563715,276.59269738)(857.70563965,276.71270508)
\curveto(857.80563662,276.73269724)(857.90563652,276.74769723)(858.00563965,276.75770508)
\curveto(858.11563631,276.76769721)(858.2256362,276.78269719)(858.33563965,276.80270508)
\curveto(858.37563605,276.81269716)(858.44063599,276.81269716)(858.53063965,276.80270508)
\curveto(858.62063581,276.80269717)(858.67563575,276.81269716)(858.69563965,276.83270508)
\curveto(859.39563503,276.84269713)(860.00563442,276.76269721)(860.52563965,276.59270508)
\curveto(861.04563338,276.42269755)(861.41063302,276.09769788)(861.62063965,275.61770508)
\curveto(861.71063272,275.41769856)(861.76063267,275.18269879)(861.77063965,274.91270508)
\curveto(861.79063264,274.65269932)(861.80063263,274.3776996)(861.80063965,274.08770508)
\lineto(861.80063965,270.77270508)
\curveto(861.80063263,270.63270334)(861.80563262,270.49770348)(861.81563965,270.36770508)
\curveto(861.8256326,270.23770374)(861.85563257,270.13270384)(861.90563965,270.05270508)
\curveto(861.95563247,269.98270399)(862.02063241,269.93270404)(862.10063965,269.90270508)
\curveto(862.19063224,269.86270411)(862.27563215,269.83270414)(862.35563965,269.81270508)
\curveto(862.43563199,269.80270417)(862.49563193,269.75770422)(862.53563965,269.67770508)
\curveto(862.55563187,269.64770433)(862.56563186,269.61770436)(862.56563965,269.58770508)
\curveto(862.56563186,269.55770442)(862.57063186,269.51770446)(862.58063965,269.46770508)
\moveto(860.43563965,271.13270508)
\curveto(860.49563393,271.2727027)(860.5256339,271.43270254)(860.52563965,271.61270508)
\curveto(860.53563389,271.80270217)(860.54063389,271.99770198)(860.54063965,272.19770508)
\curveto(860.54063389,272.30770167)(860.53563389,272.40770157)(860.52563965,272.49770508)
\curveto(860.51563391,272.58770139)(860.47563395,272.65770132)(860.40563965,272.70770508)
\curveto(860.37563405,272.72770125)(860.30563412,272.73770124)(860.19563965,272.73770508)
\curveto(860.17563425,272.71770126)(860.14063429,272.70770127)(860.09063965,272.70770508)
\curveto(860.04063439,272.70770127)(859.99563443,272.69770128)(859.95563965,272.67770508)
\curveto(859.87563455,272.65770132)(859.78563464,272.63770134)(859.68563965,272.61770508)
\lineto(859.38563965,272.55770508)
\curveto(859.35563507,272.55770142)(859.32063511,272.55270142)(859.28063965,272.54270508)
\lineto(859.17563965,272.54270508)
\curveto(859.0256354,272.50270147)(858.86063557,272.4777015)(858.68063965,272.46770508)
\curveto(858.51063592,272.46770151)(858.35063608,272.44770153)(858.20063965,272.40770508)
\curveto(858.12063631,272.38770159)(858.04563638,272.36770161)(857.97563965,272.34770508)
\curveto(857.91563651,272.33770164)(857.84563658,272.32270165)(857.76563965,272.30270508)
\curveto(857.60563682,272.25270172)(857.45563697,272.18770179)(857.31563965,272.10770508)
\curveto(857.17563725,272.03770194)(857.05563737,271.94770203)(856.95563965,271.83770508)
\curveto(856.85563757,271.72770225)(856.78063765,271.59270238)(856.73063965,271.43270508)
\curveto(856.68063775,271.28270269)(856.66063777,271.09770288)(856.67063965,270.87770508)
\curveto(856.67063776,270.7777032)(856.68563774,270.68270329)(856.71563965,270.59270508)
\curveto(856.75563767,270.51270346)(856.80063763,270.43770354)(856.85063965,270.36770508)
\curveto(856.9306375,270.25770372)(857.03563739,270.16270381)(857.16563965,270.08270508)
\curveto(857.29563713,270.01270396)(857.43563699,269.95270402)(857.58563965,269.90270508)
\curveto(857.63563679,269.89270408)(857.68563674,269.88770409)(857.73563965,269.88770508)
\curveto(857.78563664,269.88770409)(857.83563659,269.88270409)(857.88563965,269.87270508)
\curveto(857.95563647,269.85270412)(858.04063639,269.83770414)(858.14063965,269.82770508)
\curveto(858.25063618,269.82770415)(858.34063609,269.83770414)(858.41063965,269.85770508)
\curveto(858.47063596,269.8777041)(858.5306359,269.88270409)(858.59063965,269.87270508)
\curveto(858.65063578,269.8727041)(858.71063572,269.88270409)(858.77063965,269.90270508)
\curveto(858.85063558,269.92270405)(858.9256355,269.93770404)(858.99563965,269.94770508)
\curveto(859.07563535,269.95770402)(859.15063528,269.977704)(859.22063965,270.00770508)
\curveto(859.51063492,270.12770385)(859.75563467,270.2727037)(859.95563965,270.44270508)
\curveto(860.16563426,270.61270336)(860.3256341,270.84270313)(860.43563965,271.13270508)
}
}
{
\newrgbcolor{curcolor}{0 0 0}
\pscustom[linestyle=none,fillstyle=solid,fillcolor=curcolor]
{
\newpath
\moveto(867.39728027,276.81770508)
\curveto(867.62727548,276.81769716)(867.75727535,276.75769722)(867.78728027,276.63770508)
\curveto(867.81727529,276.52769745)(867.83227528,276.36269761)(867.83228027,276.14270508)
\lineto(867.83228027,275.85770508)
\curveto(867.83227528,275.76769821)(867.8072753,275.69269828)(867.75728027,275.63270508)
\curveto(867.69727541,275.55269842)(867.6122755,275.50769847)(867.50228027,275.49770508)
\curveto(867.39227572,275.49769848)(867.28227583,275.48269849)(867.17228027,275.45270508)
\curveto(867.03227608,275.42269855)(866.89727621,275.39269858)(866.76728027,275.36270508)
\curveto(866.64727646,275.33269864)(866.53227658,275.29269868)(866.42228027,275.24270508)
\curveto(866.13227698,275.11269886)(865.89727721,274.93269904)(865.71728027,274.70270508)
\curveto(865.53727757,274.48269949)(865.38227773,274.22769975)(865.25228027,273.93770508)
\curveto(865.2122779,273.82770015)(865.18227793,273.71270026)(865.16228027,273.59270508)
\curveto(865.14227797,273.48270049)(865.11727799,273.36770061)(865.08728027,273.24770508)
\curveto(865.07727803,273.19770078)(865.07227804,273.14770083)(865.07228027,273.09770508)
\curveto(865.08227803,273.04770093)(865.08227803,272.99770098)(865.07228027,272.94770508)
\curveto(865.04227807,272.82770115)(865.02727808,272.68770129)(865.02728027,272.52770508)
\curveto(865.03727807,272.3777016)(865.04227807,272.23270174)(865.04228027,272.09270508)
\lineto(865.04228027,270.24770508)
\lineto(865.04228027,269.90270508)
\curveto(865.04227807,269.78270419)(865.03727807,269.66770431)(865.02728027,269.55770508)
\curveto(865.01727809,269.44770453)(865.0122781,269.35270462)(865.01228027,269.27270508)
\curveto(865.02227809,269.19270478)(865.00227811,269.12270485)(864.95228027,269.06270508)
\curveto(864.90227821,268.99270498)(864.82227829,268.95270502)(864.71228027,268.94270508)
\curveto(864.6122785,268.93270504)(864.50227861,268.92770505)(864.38228027,268.92770508)
\lineto(864.11228027,268.92770508)
\curveto(864.06227905,268.94770503)(864.0122791,268.96270501)(863.96228027,268.97270508)
\curveto(863.92227919,268.99270498)(863.89227922,269.01770496)(863.87228027,269.04770508)
\curveto(863.82227929,269.11770486)(863.79227932,269.20270477)(863.78228027,269.30270508)
\lineto(863.78228027,269.63270508)
\lineto(863.78228027,270.78770508)
\lineto(863.78228027,274.94270508)
\lineto(863.78228027,275.97770508)
\lineto(863.78228027,276.27770508)
\curveto(863.79227932,276.3776976)(863.82227929,276.46269751)(863.87228027,276.53270508)
\curveto(863.90227921,276.5726974)(863.95227916,276.60269737)(864.02228027,276.62270508)
\curveto(864.10227901,276.64269733)(864.18727892,276.65269732)(864.27728027,276.65270508)
\curveto(864.36727874,276.66269731)(864.45727865,276.66269731)(864.54728027,276.65270508)
\curveto(864.63727847,276.64269733)(864.7072784,276.62769735)(864.75728027,276.60770508)
\curveto(864.83727827,276.5776974)(864.88727822,276.51769746)(864.90728027,276.42770508)
\curveto(864.93727817,276.34769763)(864.95227816,276.25769772)(864.95228027,276.15770508)
\lineto(864.95228027,275.85770508)
\curveto(864.95227816,275.75769822)(864.97227814,275.66769831)(865.01228027,275.58770508)
\curveto(865.02227809,275.56769841)(865.03227808,275.55269842)(865.04228027,275.54270508)
\lineto(865.08728027,275.49770508)
\curveto(865.19727791,275.49769848)(865.28727782,275.54269843)(865.35728027,275.63270508)
\curveto(865.42727768,275.73269824)(865.48727762,275.81269816)(865.53728027,275.87270508)
\lineto(865.62728027,275.96270508)
\curveto(865.71727739,276.0726979)(865.84227727,276.18769779)(866.00228027,276.30770508)
\curveto(866.16227695,276.42769755)(866.3122768,276.51769746)(866.45228027,276.57770508)
\curveto(866.54227657,276.62769735)(866.63727647,276.66269731)(866.73728027,276.68270508)
\curveto(866.83727627,276.71269726)(866.94227617,276.74269723)(867.05228027,276.77270508)
\curveto(867.112276,276.78269719)(867.17227594,276.78769719)(867.23228027,276.78770508)
\curveto(867.29227582,276.79769718)(867.34727576,276.80769717)(867.39728027,276.81770508)
}
}
{
\newrgbcolor{curcolor}{0 0 0}
\pscustom[linestyle=none,fillstyle=solid,fillcolor=curcolor]
{
\newpath
\moveto(869.0470459,278.13770508)
\curveto(868.96704478,278.19769578)(868.92204482,278.30269567)(868.9120459,278.45270508)
\lineto(868.9120459,278.91770508)
\lineto(868.9120459,279.17270508)
\curveto(868.91204483,279.26269471)(868.92704482,279.33769464)(868.9570459,279.39770508)
\curveto(868.99704475,279.4776945)(869.07704467,279.53769444)(869.1970459,279.57770508)
\curveto(869.21704453,279.58769439)(869.23704451,279.58769439)(869.2570459,279.57770508)
\curveto(869.28704446,279.5776944)(869.31204443,279.58269439)(869.3320459,279.59270508)
\curveto(869.50204424,279.59269438)(869.66204408,279.58769439)(869.8120459,279.57770508)
\curveto(869.96204378,279.56769441)(870.06204368,279.50769447)(870.1120459,279.39770508)
\curveto(870.1420436,279.33769464)(870.15704359,279.26269471)(870.1570459,279.17270508)
\lineto(870.1570459,278.91770508)
\curveto(870.15704359,278.73769524)(870.15204359,278.56769541)(870.1420459,278.40770508)
\curveto(870.1420436,278.24769573)(870.07704367,278.14269583)(869.9470459,278.09270508)
\curveto(869.89704385,278.0726959)(869.8420439,278.06269591)(869.7820459,278.06270508)
\lineto(869.6170459,278.06270508)
\lineto(869.3020459,278.06270508)
\curveto(869.20204454,278.06269591)(869.11704463,278.08769589)(869.0470459,278.13770508)
\moveto(870.1570459,269.63270508)
\lineto(870.1570459,269.31770508)
\curveto(870.16704358,269.21770476)(870.1470436,269.13770484)(870.0970459,269.07770508)
\curveto(870.06704368,269.01770496)(870.02204372,268.977705)(869.9620459,268.95770508)
\curveto(869.90204384,268.94770503)(869.83204391,268.93270504)(869.7520459,268.91270508)
\lineto(869.5270459,268.91270508)
\curveto(869.39704435,268.91270506)(869.28204446,268.91770506)(869.1820459,268.92770508)
\curveto(869.09204465,268.94770503)(869.02204472,268.99770498)(868.9720459,269.07770508)
\curveto(868.93204481,269.13770484)(868.91204483,269.21270476)(868.9120459,269.30270508)
\lineto(868.9120459,269.58770508)
\lineto(868.9120459,275.93270508)
\lineto(868.9120459,276.24770508)
\curveto(868.91204483,276.35769762)(868.93704481,276.44269753)(868.9870459,276.50270508)
\curveto(869.01704473,276.55269742)(869.05704469,276.58269739)(869.1070459,276.59270508)
\curveto(869.15704459,276.60269737)(869.21204453,276.61769736)(869.2720459,276.63770508)
\curveto(869.29204445,276.63769734)(869.31204443,276.63269734)(869.3320459,276.62270508)
\curveto(869.36204438,276.62269735)(869.38704436,276.62769735)(869.4070459,276.63770508)
\curveto(869.53704421,276.63769734)(869.66704408,276.63269734)(869.7970459,276.62270508)
\curveto(869.93704381,276.62269735)(870.03204371,276.58269739)(870.0820459,276.50270508)
\curveto(870.13204361,276.44269753)(870.15704359,276.36269761)(870.1570459,276.26270508)
\lineto(870.1570459,275.97770508)
\lineto(870.1570459,269.63270508)
}
}
{
\newrgbcolor{curcolor}{0 0 0}
\pscustom[linestyle=none,fillstyle=solid,fillcolor=curcolor]
{
\newpath
\moveto(879.22688965,273.11270508)
\curveto(879.24688159,273.05270092)(879.25688158,272.95770102)(879.25688965,272.82770508)
\curveto(879.25688158,272.70770127)(879.25188158,272.62270135)(879.24188965,272.57270508)
\lineto(879.24188965,272.42270508)
\curveto(879.2318816,272.34270163)(879.22188161,272.26770171)(879.21188965,272.19770508)
\curveto(879.21188162,272.13770184)(879.20688163,272.06770191)(879.19688965,271.98770508)
\curveto(879.17688166,271.92770205)(879.16188167,271.86770211)(879.15188965,271.80770508)
\curveto(879.15188168,271.74770223)(879.14188169,271.68770229)(879.12188965,271.62770508)
\curveto(879.08188175,271.49770248)(879.04688179,271.36770261)(879.01688965,271.23770508)
\curveto(878.98688185,271.10770287)(878.94688189,270.98770299)(878.89688965,270.87770508)
\curveto(878.68688215,270.39770358)(878.40688243,269.99270398)(878.05688965,269.66270508)
\curveto(877.70688313,269.34270463)(877.27688356,269.09770488)(876.76688965,268.92770508)
\curveto(876.65688418,268.88770509)(876.5368843,268.85770512)(876.40688965,268.83770508)
\curveto(876.28688455,268.81770516)(876.16188467,268.79770518)(876.03188965,268.77770508)
\curveto(875.97188486,268.76770521)(875.90688493,268.76270521)(875.83688965,268.76270508)
\curveto(875.77688506,268.75270522)(875.71688512,268.74770523)(875.65688965,268.74770508)
\curveto(875.61688522,268.73770524)(875.55688528,268.73270524)(875.47688965,268.73270508)
\curveto(875.40688543,268.73270524)(875.35688548,268.73770524)(875.32688965,268.74770508)
\curveto(875.28688555,268.75770522)(875.24688559,268.76270521)(875.20688965,268.76270508)
\curveto(875.16688567,268.75270522)(875.1318857,268.75270522)(875.10188965,268.76270508)
\lineto(875.01188965,268.76270508)
\lineto(874.65188965,268.80770508)
\curveto(874.51188632,268.84770513)(874.37688646,268.88770509)(874.24688965,268.92770508)
\curveto(874.11688672,268.96770501)(873.99188684,269.01270496)(873.87188965,269.06270508)
\curveto(873.42188741,269.26270471)(873.05188778,269.52270445)(872.76188965,269.84270508)
\curveto(872.47188836,270.16270381)(872.2318886,270.55270342)(872.04188965,271.01270508)
\curveto(871.99188884,271.11270286)(871.95188888,271.21270276)(871.92188965,271.31270508)
\curveto(871.90188893,271.41270256)(871.88188895,271.51770246)(871.86188965,271.62770508)
\curveto(871.84188899,271.66770231)(871.831889,271.69770228)(871.83188965,271.71770508)
\curveto(871.84188899,271.74770223)(871.84188899,271.78270219)(871.83188965,271.82270508)
\curveto(871.81188902,271.90270207)(871.79688904,271.98270199)(871.78688965,272.06270508)
\curveto(871.78688905,272.15270182)(871.77688906,272.23770174)(871.75688965,272.31770508)
\lineto(871.75688965,272.43770508)
\curveto(871.75688908,272.4777015)(871.75188908,272.52270145)(871.74188965,272.57270508)
\curveto(871.7318891,272.62270135)(871.72688911,272.70770127)(871.72688965,272.82770508)
\curveto(871.72688911,272.95770102)(871.7368891,273.05270092)(871.75688965,273.11270508)
\curveto(871.77688906,273.18270079)(871.78188905,273.25270072)(871.77188965,273.32270508)
\curveto(871.76188907,273.39270058)(871.76688907,273.46270051)(871.78688965,273.53270508)
\curveto(871.79688904,273.58270039)(871.80188903,273.62270035)(871.80188965,273.65270508)
\curveto(871.81188902,273.69270028)(871.82188901,273.73770024)(871.83188965,273.78770508)
\curveto(871.86188897,273.90770007)(871.88688895,274.02769995)(871.90688965,274.14770508)
\curveto(871.9368889,274.26769971)(871.97688886,274.38269959)(872.02688965,274.49270508)
\curveto(872.17688866,274.86269911)(872.35688848,275.19269878)(872.56688965,275.48270508)
\curveto(872.78688805,275.78269819)(873.05188778,276.03269794)(873.36188965,276.23270508)
\curveto(873.48188735,276.31269766)(873.60688723,276.3776976)(873.73688965,276.42770508)
\curveto(873.86688697,276.48769749)(874.00188683,276.54769743)(874.14188965,276.60770508)
\curveto(874.26188657,276.65769732)(874.39188644,276.68769729)(874.53188965,276.69770508)
\curveto(874.67188616,276.71769726)(874.81188602,276.74769723)(874.95188965,276.78770508)
\lineto(875.14688965,276.78770508)
\curveto(875.21688562,276.79769718)(875.28188555,276.80769717)(875.34188965,276.81770508)
\curveto(876.2318846,276.82769715)(876.97188386,276.64269733)(877.56188965,276.26270508)
\curveto(878.15188268,275.88269809)(878.57688226,275.38769859)(878.83688965,274.77770508)
\curveto(878.88688195,274.6776993)(878.92688191,274.5776994)(878.95688965,274.47770508)
\curveto(878.98688185,274.3776996)(879.02188181,274.2726997)(879.06188965,274.16270508)
\curveto(879.09188174,274.05269992)(879.11688172,273.93270004)(879.13688965,273.80270508)
\curveto(879.15688168,273.68270029)(879.18188165,273.55770042)(879.21188965,273.42770508)
\curveto(879.22188161,273.3777006)(879.22188161,273.32270065)(879.21188965,273.26270508)
\curveto(879.21188162,273.21270076)(879.21688162,273.16270081)(879.22688965,273.11270508)
\moveto(877.89188965,272.25770508)
\curveto(877.91188292,272.32770165)(877.91688292,272.40770157)(877.90688965,272.49770508)
\lineto(877.90688965,272.75270508)
\curveto(877.90688293,273.14270083)(877.87188296,273.4727005)(877.80188965,273.74270508)
\curveto(877.77188306,273.82270015)(877.74688309,273.90270007)(877.72688965,273.98270508)
\curveto(877.70688313,274.06269991)(877.68188315,274.13769984)(877.65188965,274.20770508)
\curveto(877.37188346,274.85769912)(876.92688391,275.30769867)(876.31688965,275.55770508)
\curveto(876.24688459,275.58769839)(876.17188466,275.60769837)(876.09188965,275.61770508)
\lineto(875.85188965,275.67770508)
\curveto(875.77188506,275.69769828)(875.68688515,275.70769827)(875.59688965,275.70770508)
\lineto(875.32688965,275.70770508)
\lineto(875.05688965,275.66270508)
\curveto(874.95688588,275.64269833)(874.86188597,275.61769836)(874.77188965,275.58770508)
\curveto(874.69188614,275.56769841)(874.61188622,275.53769844)(874.53188965,275.49770508)
\curveto(874.46188637,275.4776985)(874.39688644,275.44769853)(874.33688965,275.40770508)
\curveto(874.27688656,275.36769861)(874.22188661,275.32769865)(874.17188965,275.28770508)
\curveto(873.9318869,275.11769886)(873.7368871,274.91269906)(873.58688965,274.67270508)
\curveto(873.4368874,274.43269954)(873.30688753,274.15269982)(873.19688965,273.83270508)
\curveto(873.16688767,273.73270024)(873.14688769,273.62770035)(873.13688965,273.51770508)
\curveto(873.12688771,273.41770056)(873.11188772,273.31270066)(873.09188965,273.20270508)
\curveto(873.08188775,273.16270081)(873.07688776,273.09770088)(873.07688965,273.00770508)
\curveto(873.06688777,272.977701)(873.06188777,272.94270103)(873.06188965,272.90270508)
\curveto(873.07188776,272.86270111)(873.07688776,272.81770116)(873.07688965,272.76770508)
\lineto(873.07688965,272.46770508)
\curveto(873.07688776,272.36770161)(873.08688775,272.2777017)(873.10688965,272.19770508)
\lineto(873.13688965,272.01770508)
\curveto(873.15688768,271.91770206)(873.17188766,271.81770216)(873.18188965,271.71770508)
\curveto(873.20188763,271.62770235)(873.2318876,271.54270243)(873.27188965,271.46270508)
\curveto(873.37188746,271.22270275)(873.48688735,270.99770298)(873.61688965,270.78770508)
\curveto(873.75688708,270.5777034)(873.92688691,270.40270357)(874.12688965,270.26270508)
\curveto(874.17688666,270.23270374)(874.22188661,270.20770377)(874.26188965,270.18770508)
\curveto(874.30188653,270.16770381)(874.34688649,270.14270383)(874.39688965,270.11270508)
\curveto(874.47688636,270.06270391)(874.56188627,270.01770396)(874.65188965,269.97770508)
\curveto(874.75188608,269.94770403)(874.85688598,269.91770406)(874.96688965,269.88770508)
\curveto(875.01688582,269.86770411)(875.06188577,269.85770412)(875.10188965,269.85770508)
\curveto(875.15188568,269.86770411)(875.20188563,269.86770411)(875.25188965,269.85770508)
\curveto(875.28188555,269.84770413)(875.34188549,269.83770414)(875.43188965,269.82770508)
\curveto(875.5318853,269.81770416)(875.60688523,269.82270415)(875.65688965,269.84270508)
\curveto(875.69688514,269.85270412)(875.7368851,269.85270412)(875.77688965,269.84270508)
\curveto(875.81688502,269.84270413)(875.85688498,269.85270412)(875.89688965,269.87270508)
\curveto(875.97688486,269.89270408)(876.05688478,269.90770407)(876.13688965,269.91770508)
\curveto(876.21688462,269.93770404)(876.29188454,269.96270401)(876.36188965,269.99270508)
\curveto(876.70188413,270.13270384)(876.97688386,270.32770365)(877.18688965,270.57770508)
\curveto(877.39688344,270.82770315)(877.57188326,271.12270285)(877.71188965,271.46270508)
\curveto(877.76188307,271.58270239)(877.79188304,271.70770227)(877.80188965,271.83770508)
\curveto(877.82188301,271.977702)(877.85188298,272.11770186)(877.89188965,272.25770508)
}
}
{
\newrgbcolor{curcolor}{0 0 0}
\pscustom[linestyle=none,fillstyle=solid,fillcolor=curcolor]
{
\newpath
\moveto(831.97647461,256.60103638)
\lineto(836.61147461,256.60103638)
\lineto(837.82647461,256.60103638)
\curveto(837.93646706,256.60102568)(838.04146695,256.60102568)(838.14147461,256.60103638)
\curveto(838.25146674,256.60102568)(838.33646666,256.5810257)(838.39647461,256.54103638)
\curveto(838.47646652,256.49102579)(838.52146647,256.41602587)(838.53147461,256.31603638)
\curveto(838.55146644,256.22602606)(838.56146643,256.11602617)(838.56147461,255.98603638)
\lineto(838.56147461,255.83603638)
\curveto(838.57146642,255.79602649)(838.56646643,255.75602653)(838.54647461,255.71603638)
\curveto(838.50646649,255.55602673)(838.41646658,255.46602682)(838.27647461,255.44603638)
\curveto(838.14646685,255.43602685)(837.98146701,255.43102685)(837.78147461,255.43103638)
\lineto(836.22147461,255.43103638)
\lineto(834.01647461,255.43103638)
\lineto(833.50647461,255.43103638)
\curveto(833.32647167,255.44102684)(833.1914718,255.41102687)(833.10147461,255.34103638)
\curveto(833.01147198,255.281027)(832.96147203,255.17602711)(832.95147461,255.02603638)
\lineto(832.95147461,254.57603638)
\lineto(832.95147461,253.09103638)
\curveto(832.95147204,253.01102927)(832.94647205,252.91102937)(832.93647461,252.79103638)
\curveto(832.93647206,252.67102961)(832.94647205,252.57102971)(832.96647461,252.49103638)
\lineto(832.96647461,252.37103638)
\curveto(832.98647201,252.31102997)(833.00147199,252.25103003)(833.01147461,252.19103638)
\curveto(833.03147196,252.14103014)(833.06647193,252.10103018)(833.11647461,252.07103638)
\curveto(833.20647179,252.01103027)(833.34647165,251.9810303)(833.53647461,251.98103638)
\curveto(833.72647127,251.99103029)(833.8914711,251.99603029)(834.03147461,251.99603638)
\lineto(836.73147461,251.99603638)
\lineto(837.01647461,251.99603638)
\curveto(837.12646787,252.00603028)(837.23146776,252.00603028)(837.33147461,251.99603638)
\curveto(837.44146755,251.99603029)(837.53646746,251.9860303)(837.61647461,251.96603638)
\curveto(837.70646729,251.94603034)(837.76646723,251.91103037)(837.79647461,251.86103638)
\curveto(837.84646715,251.80103048)(837.87146712,251.72603056)(837.87147461,251.63603638)
\lineto(837.87147461,251.33603638)
\lineto(837.87147461,251.17103638)
\curveto(837.87146712,251.12103116)(837.86146713,251.07603121)(837.84147461,251.03603638)
\curveto(837.80146719,250.93603135)(837.74646725,250.8810314)(837.67647461,250.87103638)
\curveto(837.63646736,250.85103143)(837.5964674,250.84103144)(837.55647461,250.84103638)
\curveto(837.52646747,250.84103144)(837.48646751,250.83603145)(837.43647461,250.82603638)
\curveto(837.3964676,250.81603147)(837.35146764,250.81103147)(837.30147461,250.81103638)
\curveto(837.26146773,250.82103146)(837.22146777,250.82603146)(837.18147461,250.82603638)
\lineto(836.65647461,250.82603638)
\lineto(834.12147461,250.82603638)
\lineto(833.55147461,250.82603638)
\curveto(833.34147165,250.83603145)(833.1914718,250.80603148)(833.10147461,250.73603638)
\curveto(833.05147194,250.69603159)(833.01147198,250.63103165)(832.98147461,250.54103638)
\curveto(832.96147203,250.46103182)(832.94647205,250.36603192)(832.93647461,250.25603638)
\lineto(832.93647461,249.91103638)
\curveto(832.94647205,249.80103248)(832.95147204,249.70103258)(832.95147461,249.61103638)
\lineto(832.95147461,247.03103638)
\curveto(832.95147204,246.86103542)(832.95647204,246.67603561)(832.96647461,246.47603638)
\curveto(832.97647202,246.27603601)(832.94147205,246.12603616)(832.86147461,246.02603638)
\curveto(832.83147216,245.9860363)(832.78647221,245.96103632)(832.72647461,245.95103638)
\curveto(832.66647233,245.95103633)(832.60647239,245.94103634)(832.54647461,245.92103638)
\lineto(832.26147461,245.92103638)
\curveto(832.12147287,245.92103636)(831.991473,245.92603636)(831.87147461,245.93603638)
\curveto(831.75147324,245.94603634)(831.66647333,245.99603629)(831.61647461,246.08603638)
\curveto(831.57647342,246.14603614)(831.55647344,246.22603606)(831.55647461,246.32603638)
\lineto(831.55647461,246.65603638)
\lineto(831.55647461,247.85603638)
\lineto(831.55647461,254.12603638)
\lineto(831.55647461,255.74603638)
\curveto(831.55647344,255.85602643)(831.55147344,255.97602631)(831.54147461,256.10603638)
\curveto(831.54147345,256.24602604)(831.56647343,256.35602593)(831.61647461,256.43603638)
\curveto(831.65647334,256.49602579)(831.73147326,256.54602574)(831.84147461,256.58603638)
\curveto(831.86147313,256.59602569)(831.88147311,256.59602569)(831.90147461,256.58603638)
\curveto(831.93147306,256.5860257)(831.95647304,256.59102569)(831.97647461,256.60103638)
}
}
{
\newrgbcolor{curcolor}{0 0 0}
\pscustom[linestyle=none,fillstyle=solid,fillcolor=curcolor]
{
\newpath
\moveto(847.03975586,250.12103638)
\curveto(847.0597478,250.06103222)(847.06974779,249.96603232)(847.06975586,249.83603638)
\curveto(847.06974779,249.71603257)(847.06474779,249.63103265)(847.05475586,249.58103638)
\lineto(847.05475586,249.43103638)
\curveto(847.04474781,249.35103293)(847.03474782,249.27603301)(847.02475586,249.20603638)
\curveto(847.02474783,249.14603314)(847.01974784,249.07603321)(847.00975586,248.99603638)
\curveto(846.98974787,248.93603335)(846.97474788,248.87603341)(846.96475586,248.81603638)
\curveto(846.96474789,248.75603353)(846.9547479,248.69603359)(846.93475586,248.63603638)
\curveto(846.89474796,248.50603378)(846.859748,248.37603391)(846.82975586,248.24603638)
\curveto(846.79974806,248.11603417)(846.7597481,247.99603429)(846.70975586,247.88603638)
\curveto(846.49974836,247.40603488)(846.21974864,247.00103528)(845.86975586,246.67103638)
\curveto(845.51974934,246.35103593)(845.08974977,246.10603618)(844.57975586,245.93603638)
\curveto(844.46975039,245.89603639)(844.34975051,245.86603642)(844.21975586,245.84603638)
\curveto(844.09975076,245.82603646)(843.97475088,245.80603648)(843.84475586,245.78603638)
\curveto(843.78475107,245.77603651)(843.71975114,245.77103651)(843.64975586,245.77103638)
\curveto(843.58975127,245.76103652)(843.52975133,245.75603653)(843.46975586,245.75603638)
\curveto(843.42975143,245.74603654)(843.36975149,245.74103654)(843.28975586,245.74103638)
\curveto(843.21975164,245.74103654)(843.16975169,245.74603654)(843.13975586,245.75603638)
\curveto(843.09975176,245.76603652)(843.0597518,245.77103651)(843.01975586,245.77103638)
\curveto(842.97975188,245.76103652)(842.94475191,245.76103652)(842.91475586,245.77103638)
\lineto(842.82475586,245.77103638)
\lineto(842.46475586,245.81603638)
\curveto(842.32475253,245.85603643)(842.18975267,245.89603639)(842.05975586,245.93603638)
\curveto(841.92975293,245.97603631)(841.80475305,246.02103626)(841.68475586,246.07103638)
\curveto(841.23475362,246.27103601)(840.86475399,246.53103575)(840.57475586,246.85103638)
\curveto(840.28475457,247.17103511)(840.04475481,247.56103472)(839.85475586,248.02103638)
\curveto(839.80475505,248.12103416)(839.76475509,248.22103406)(839.73475586,248.32103638)
\curveto(839.71475514,248.42103386)(839.69475516,248.52603376)(839.67475586,248.63603638)
\curveto(839.6547552,248.67603361)(839.64475521,248.70603358)(839.64475586,248.72603638)
\curveto(839.6547552,248.75603353)(839.6547552,248.79103349)(839.64475586,248.83103638)
\curveto(839.62475523,248.91103337)(839.60975525,248.99103329)(839.59975586,249.07103638)
\curveto(839.59975526,249.16103312)(839.58975527,249.24603304)(839.56975586,249.32603638)
\lineto(839.56975586,249.44603638)
\curveto(839.56975529,249.4860328)(839.56475529,249.53103275)(839.55475586,249.58103638)
\curveto(839.54475531,249.63103265)(839.53975532,249.71603257)(839.53975586,249.83603638)
\curveto(839.53975532,249.96603232)(839.54975531,250.06103222)(839.56975586,250.12103638)
\curveto(839.58975527,250.19103209)(839.59475526,250.26103202)(839.58475586,250.33103638)
\curveto(839.57475528,250.40103188)(839.57975528,250.47103181)(839.59975586,250.54103638)
\curveto(839.60975525,250.59103169)(839.61475524,250.63103165)(839.61475586,250.66103638)
\curveto(839.62475523,250.70103158)(839.63475522,250.74603154)(839.64475586,250.79603638)
\curveto(839.67475518,250.91603137)(839.69975516,251.03603125)(839.71975586,251.15603638)
\curveto(839.74975511,251.27603101)(839.78975507,251.39103089)(839.83975586,251.50103638)
\curveto(839.98975487,251.87103041)(840.16975469,252.20103008)(840.37975586,252.49103638)
\curveto(840.59975426,252.79102949)(840.86475399,253.04102924)(841.17475586,253.24103638)
\curveto(841.29475356,253.32102896)(841.41975344,253.3860289)(841.54975586,253.43603638)
\curveto(841.67975318,253.49602879)(841.81475304,253.55602873)(841.95475586,253.61603638)
\curveto(842.07475278,253.66602862)(842.20475265,253.69602859)(842.34475586,253.70603638)
\curveto(842.48475237,253.72602856)(842.62475223,253.75602853)(842.76475586,253.79603638)
\lineto(842.95975586,253.79603638)
\curveto(843.02975183,253.80602848)(843.09475176,253.81602847)(843.15475586,253.82603638)
\curveto(844.04475081,253.83602845)(844.78475007,253.65102863)(845.37475586,253.27103638)
\curveto(845.96474889,252.89102939)(846.38974847,252.39602989)(846.64975586,251.78603638)
\curveto(846.69974816,251.6860306)(846.73974812,251.5860307)(846.76975586,251.48603638)
\curveto(846.79974806,251.3860309)(846.83474802,251.281031)(846.87475586,251.17103638)
\curveto(846.90474795,251.06103122)(846.92974793,250.94103134)(846.94975586,250.81103638)
\curveto(846.96974789,250.69103159)(846.99474786,250.56603172)(847.02475586,250.43603638)
\curveto(847.03474782,250.3860319)(847.03474782,250.33103195)(847.02475586,250.27103638)
\curveto(847.02474783,250.22103206)(847.02974783,250.17103211)(847.03975586,250.12103638)
\moveto(845.70475586,249.26603638)
\curveto(845.72474913,249.33603295)(845.72974913,249.41603287)(845.71975586,249.50603638)
\lineto(845.71975586,249.76103638)
\curveto(845.71974914,250.15103213)(845.68474917,250.4810318)(845.61475586,250.75103638)
\curveto(845.58474927,250.83103145)(845.5597493,250.91103137)(845.53975586,250.99103638)
\curveto(845.51974934,251.07103121)(845.49474936,251.14603114)(845.46475586,251.21603638)
\curveto(845.18474967,251.86603042)(844.73975012,252.31602997)(844.12975586,252.56603638)
\curveto(844.0597508,252.59602969)(843.98475087,252.61602967)(843.90475586,252.62603638)
\lineto(843.66475586,252.68603638)
\curveto(843.58475127,252.70602958)(843.49975136,252.71602957)(843.40975586,252.71603638)
\lineto(843.13975586,252.71603638)
\lineto(842.86975586,252.67103638)
\curveto(842.76975209,252.65102963)(842.67475218,252.62602966)(842.58475586,252.59603638)
\curveto(842.50475235,252.57602971)(842.42475243,252.54602974)(842.34475586,252.50603638)
\curveto(842.27475258,252.4860298)(842.20975265,252.45602983)(842.14975586,252.41603638)
\curveto(842.08975277,252.37602991)(842.03475282,252.33602995)(841.98475586,252.29603638)
\curveto(841.74475311,252.12603016)(841.54975331,251.92103036)(841.39975586,251.68103638)
\curveto(841.24975361,251.44103084)(841.11975374,251.16103112)(841.00975586,250.84103638)
\curveto(840.97975388,250.74103154)(840.9597539,250.63603165)(840.94975586,250.52603638)
\curveto(840.93975392,250.42603186)(840.92475393,250.32103196)(840.90475586,250.21103638)
\curveto(840.89475396,250.17103211)(840.88975397,250.10603218)(840.88975586,250.01603638)
\curveto(840.87975398,249.9860323)(840.87475398,249.95103233)(840.87475586,249.91103638)
\curveto(840.88475397,249.87103241)(840.88975397,249.82603246)(840.88975586,249.77603638)
\lineto(840.88975586,249.47603638)
\curveto(840.88975397,249.37603291)(840.89975396,249.286033)(840.91975586,249.20603638)
\lineto(840.94975586,249.02603638)
\curveto(840.96975389,248.92603336)(840.98475387,248.82603346)(840.99475586,248.72603638)
\curveto(841.01475384,248.63603365)(841.04475381,248.55103373)(841.08475586,248.47103638)
\curveto(841.18475367,248.23103405)(841.29975356,248.00603428)(841.42975586,247.79603638)
\curveto(841.56975329,247.5860347)(841.73975312,247.41103487)(841.93975586,247.27103638)
\curveto(841.98975287,247.24103504)(842.03475282,247.21603507)(842.07475586,247.19603638)
\curveto(842.11475274,247.17603511)(842.1597527,247.15103513)(842.20975586,247.12103638)
\curveto(842.28975257,247.07103521)(842.37475248,247.02603526)(842.46475586,246.98603638)
\curveto(842.56475229,246.95603533)(842.66975219,246.92603536)(842.77975586,246.89603638)
\curveto(842.82975203,246.87603541)(842.87475198,246.86603542)(842.91475586,246.86603638)
\curveto(842.96475189,246.87603541)(843.01475184,246.87603541)(843.06475586,246.86603638)
\curveto(843.09475176,246.85603543)(843.1547517,246.84603544)(843.24475586,246.83603638)
\curveto(843.34475151,246.82603546)(843.41975144,246.83103545)(843.46975586,246.85103638)
\curveto(843.50975135,246.86103542)(843.54975131,246.86103542)(843.58975586,246.85103638)
\curveto(843.62975123,246.85103543)(843.66975119,246.86103542)(843.70975586,246.88103638)
\curveto(843.78975107,246.90103538)(843.86975099,246.91603537)(843.94975586,246.92603638)
\curveto(844.02975083,246.94603534)(844.10475075,246.97103531)(844.17475586,247.00103638)
\curveto(844.51475034,247.14103514)(844.78975007,247.33603495)(844.99975586,247.58603638)
\curveto(845.20974965,247.83603445)(845.38474947,248.13103415)(845.52475586,248.47103638)
\curveto(845.57474928,248.59103369)(845.60474925,248.71603357)(845.61475586,248.84603638)
\curveto(845.63474922,248.9860333)(845.66474919,249.12603316)(845.70475586,249.26603638)
}
}
{
\newrgbcolor{curcolor}{0 0 0}
\pscustom[linestyle=none,fillstyle=solid,fillcolor=curcolor]
{
\newpath
\moveto(849.47303711,255.98603638)
\curveto(849.6230351,255.9860263)(849.77303495,255.9810263)(849.92303711,255.97103638)
\curveto(850.07303465,255.97102631)(850.17803454,255.93102635)(850.23803711,255.85103638)
\curveto(850.28803443,255.79102649)(850.31303441,255.70602658)(850.31303711,255.59603638)
\curveto(850.3230344,255.49602679)(850.32803439,255.39102689)(850.32803711,255.28103638)
\lineto(850.32803711,254.41103638)
\curveto(850.32803439,254.33102795)(850.3230344,254.24602804)(850.31303711,254.15603638)
\curveto(850.31303441,254.07602821)(850.3230344,254.00602828)(850.34303711,253.94603638)
\curveto(850.38303434,253.80602848)(850.47303425,253.71602857)(850.61303711,253.67603638)
\curveto(850.66303406,253.66602862)(850.70803401,253.66102862)(850.74803711,253.66103638)
\lineto(850.89803711,253.66103638)
\lineto(851.30303711,253.66103638)
\curveto(851.46303326,253.67102861)(851.57803314,253.66102862)(851.64803711,253.63103638)
\curveto(851.73803298,253.57102871)(851.79803292,253.51102877)(851.82803711,253.45103638)
\curveto(851.84803287,253.41102887)(851.85803286,253.36602892)(851.85803711,253.31603638)
\lineto(851.85803711,253.16603638)
\curveto(851.85803286,253.05602923)(851.85303287,252.95102933)(851.84303711,252.85103638)
\curveto(851.83303289,252.76102952)(851.79803292,252.69102959)(851.73803711,252.64103638)
\curveto(851.67803304,252.59102969)(851.59303313,252.56102972)(851.48303711,252.55103638)
\lineto(851.15303711,252.55103638)
\curveto(851.04303368,252.56102972)(850.93303379,252.56602972)(850.82303711,252.56603638)
\curveto(850.71303401,252.56602972)(850.6180341,252.55102973)(850.53803711,252.52103638)
\curveto(850.46803425,252.49102979)(850.4180343,252.44102984)(850.38803711,252.37103638)
\curveto(850.35803436,252.30102998)(850.33803438,252.21603007)(850.32803711,252.11603638)
\curveto(850.3180344,252.02603026)(850.31303441,251.92603036)(850.31303711,251.81603638)
\curveto(850.3230344,251.71603057)(850.32803439,251.61603067)(850.32803711,251.51603638)
\lineto(850.32803711,248.54603638)
\curveto(850.32803439,248.32603396)(850.3230344,248.09103419)(850.31303711,247.84103638)
\curveto(850.31303441,247.60103468)(850.35803436,247.41603487)(850.44803711,247.28603638)
\curveto(850.49803422,247.20603508)(850.56303416,247.15103513)(850.64303711,247.12103638)
\curveto(850.723034,247.09103519)(850.8180339,247.06603522)(850.92803711,247.04603638)
\curveto(850.95803376,247.03603525)(850.98803373,247.03103525)(851.01803711,247.03103638)
\curveto(851.05803366,247.04103524)(851.09303363,247.04103524)(851.12303711,247.03103638)
\lineto(851.31803711,247.03103638)
\curveto(851.4180333,247.03103525)(851.50803321,247.02103526)(851.58803711,247.00103638)
\curveto(851.67803304,246.99103529)(851.74303298,246.95603533)(851.78303711,246.89603638)
\curveto(851.80303292,246.86603542)(851.8180329,246.81103547)(851.82803711,246.73103638)
\curveto(851.84803287,246.66103562)(851.85803286,246.5860357)(851.85803711,246.50603638)
\curveto(851.86803285,246.42603586)(851.86803285,246.34603594)(851.85803711,246.26603638)
\curveto(851.84803287,246.19603609)(851.82803289,246.14103614)(851.79803711,246.10103638)
\curveto(851.75803296,246.03103625)(851.68303304,245.9810363)(851.57303711,245.95103638)
\curveto(851.49303323,245.93103635)(851.40303332,245.92103636)(851.30303711,245.92103638)
\curveto(851.20303352,245.93103635)(851.11303361,245.93603635)(851.03303711,245.93603638)
\curveto(850.97303375,245.93603635)(850.91303381,245.93103635)(850.85303711,245.92103638)
\curveto(850.79303393,245.92103636)(850.73803398,245.92603636)(850.68803711,245.93603638)
\lineto(850.50803711,245.93603638)
\curveto(850.45803426,245.94603634)(850.40803431,245.95103633)(850.35803711,245.95103638)
\curveto(850.3180344,245.96103632)(850.27303445,245.96603632)(850.22303711,245.96603638)
\curveto(850.0230347,246.01603627)(849.84803487,246.07103621)(849.69803711,246.13103638)
\curveto(849.55803516,246.19103609)(849.43803528,246.29603599)(849.33803711,246.44603638)
\curveto(849.19803552,246.64603564)(849.1180356,246.89603539)(849.09803711,247.19603638)
\curveto(849.07803564,247.50603478)(849.06803565,247.83603445)(849.06803711,248.18603638)
\lineto(849.06803711,252.11603638)
\curveto(849.03803568,252.24603004)(849.00803571,252.34102994)(848.97803711,252.40103638)
\curveto(848.95803576,252.46102982)(848.88803583,252.51102977)(848.76803711,252.55103638)
\curveto(848.72803599,252.56102972)(848.68803603,252.56102972)(848.64803711,252.55103638)
\curveto(848.60803611,252.54102974)(848.56803615,252.54602974)(848.52803711,252.56603638)
\lineto(848.28803711,252.56603638)
\curveto(848.15803656,252.56602972)(848.04803667,252.57602971)(847.95803711,252.59603638)
\curveto(847.87803684,252.62602966)(847.8230369,252.6860296)(847.79303711,252.77603638)
\curveto(847.77303695,252.81602947)(847.75803696,252.86102942)(847.74803711,252.91103638)
\lineto(847.74803711,253.06103638)
\curveto(847.74803697,253.20102908)(847.75803696,253.31602897)(847.77803711,253.40603638)
\curveto(847.79803692,253.50602878)(847.85803686,253.5810287)(847.95803711,253.63103638)
\curveto(848.06803665,253.67102861)(848.20803651,253.6810286)(848.37803711,253.66103638)
\curveto(848.55803616,253.64102864)(848.70803601,253.65102863)(848.82803711,253.69103638)
\curveto(848.9180358,253.74102854)(848.98803573,253.81102847)(849.03803711,253.90103638)
\curveto(849.05803566,253.96102832)(849.06803565,254.03602825)(849.06803711,254.12603638)
\lineto(849.06803711,254.38103638)
\lineto(849.06803711,255.31103638)
\lineto(849.06803711,255.55103638)
\curveto(849.06803565,255.64102664)(849.07803564,255.71602657)(849.09803711,255.77603638)
\curveto(849.13803558,255.85602643)(849.21303551,255.92102636)(849.32303711,255.97103638)
\curveto(849.35303537,255.97102631)(849.37803534,255.97102631)(849.39803711,255.97103638)
\curveto(849.42803529,255.9810263)(849.45303527,255.9860263)(849.47303711,255.98603638)
}
}
{
\newrgbcolor{curcolor}{0 0 0}
\pscustom[linestyle=none,fillstyle=solid,fillcolor=curcolor]
{
\newpath
\moveto(860.36983398,250.12103638)
\curveto(860.38982592,250.06103222)(860.39982591,249.96603232)(860.39983398,249.83603638)
\curveto(860.39982591,249.71603257)(860.39482592,249.63103265)(860.38483398,249.58103638)
\lineto(860.38483398,249.43103638)
\curveto(860.37482594,249.35103293)(860.36482595,249.27603301)(860.35483398,249.20603638)
\curveto(860.35482596,249.14603314)(860.34982596,249.07603321)(860.33983398,248.99603638)
\curveto(860.31982599,248.93603335)(860.30482601,248.87603341)(860.29483398,248.81603638)
\curveto(860.29482602,248.75603353)(860.28482603,248.69603359)(860.26483398,248.63603638)
\curveto(860.22482609,248.50603378)(860.18982612,248.37603391)(860.15983398,248.24603638)
\curveto(860.12982618,248.11603417)(860.08982622,247.99603429)(860.03983398,247.88603638)
\curveto(859.82982648,247.40603488)(859.54982676,247.00103528)(859.19983398,246.67103638)
\curveto(858.84982746,246.35103593)(858.41982789,246.10603618)(857.90983398,245.93603638)
\curveto(857.79982851,245.89603639)(857.67982863,245.86603642)(857.54983398,245.84603638)
\curveto(857.42982888,245.82603646)(857.30482901,245.80603648)(857.17483398,245.78603638)
\curveto(857.1148292,245.77603651)(857.04982926,245.77103651)(856.97983398,245.77103638)
\curveto(856.91982939,245.76103652)(856.85982945,245.75603653)(856.79983398,245.75603638)
\curveto(856.75982955,245.74603654)(856.69982961,245.74103654)(856.61983398,245.74103638)
\curveto(856.54982976,245.74103654)(856.49982981,245.74603654)(856.46983398,245.75603638)
\curveto(856.42982988,245.76603652)(856.38982992,245.77103651)(856.34983398,245.77103638)
\curveto(856.30983,245.76103652)(856.27483004,245.76103652)(856.24483398,245.77103638)
\lineto(856.15483398,245.77103638)
\lineto(855.79483398,245.81603638)
\curveto(855.65483066,245.85603643)(855.51983079,245.89603639)(855.38983398,245.93603638)
\curveto(855.25983105,245.97603631)(855.13483118,246.02103626)(855.01483398,246.07103638)
\curveto(854.56483175,246.27103601)(854.19483212,246.53103575)(853.90483398,246.85103638)
\curveto(853.6148327,247.17103511)(853.37483294,247.56103472)(853.18483398,248.02103638)
\curveto(853.13483318,248.12103416)(853.09483322,248.22103406)(853.06483398,248.32103638)
\curveto(853.04483327,248.42103386)(853.02483329,248.52603376)(853.00483398,248.63603638)
\curveto(852.98483333,248.67603361)(852.97483334,248.70603358)(852.97483398,248.72603638)
\curveto(852.98483333,248.75603353)(852.98483333,248.79103349)(852.97483398,248.83103638)
\curveto(852.95483336,248.91103337)(852.93983337,248.99103329)(852.92983398,249.07103638)
\curveto(852.92983338,249.16103312)(852.91983339,249.24603304)(852.89983398,249.32603638)
\lineto(852.89983398,249.44603638)
\curveto(852.89983341,249.4860328)(852.89483342,249.53103275)(852.88483398,249.58103638)
\curveto(852.87483344,249.63103265)(852.86983344,249.71603257)(852.86983398,249.83603638)
\curveto(852.86983344,249.96603232)(852.87983343,250.06103222)(852.89983398,250.12103638)
\curveto(852.91983339,250.19103209)(852.92483339,250.26103202)(852.91483398,250.33103638)
\curveto(852.90483341,250.40103188)(852.9098334,250.47103181)(852.92983398,250.54103638)
\curveto(852.93983337,250.59103169)(852.94483337,250.63103165)(852.94483398,250.66103638)
\curveto(852.95483336,250.70103158)(852.96483335,250.74603154)(852.97483398,250.79603638)
\curveto(853.00483331,250.91603137)(853.02983328,251.03603125)(853.04983398,251.15603638)
\curveto(853.07983323,251.27603101)(853.11983319,251.39103089)(853.16983398,251.50103638)
\curveto(853.31983299,251.87103041)(853.49983281,252.20103008)(853.70983398,252.49103638)
\curveto(853.92983238,252.79102949)(854.19483212,253.04102924)(854.50483398,253.24103638)
\curveto(854.62483169,253.32102896)(854.74983156,253.3860289)(854.87983398,253.43603638)
\curveto(855.0098313,253.49602879)(855.14483117,253.55602873)(855.28483398,253.61603638)
\curveto(855.40483091,253.66602862)(855.53483078,253.69602859)(855.67483398,253.70603638)
\curveto(855.8148305,253.72602856)(855.95483036,253.75602853)(856.09483398,253.79603638)
\lineto(856.28983398,253.79603638)
\curveto(856.35982995,253.80602848)(856.42482989,253.81602847)(856.48483398,253.82603638)
\curveto(857.37482894,253.83602845)(858.1148282,253.65102863)(858.70483398,253.27103638)
\curveto(859.29482702,252.89102939)(859.71982659,252.39602989)(859.97983398,251.78603638)
\curveto(860.02982628,251.6860306)(860.06982624,251.5860307)(860.09983398,251.48603638)
\curveto(860.12982618,251.3860309)(860.16482615,251.281031)(860.20483398,251.17103638)
\curveto(860.23482608,251.06103122)(860.25982605,250.94103134)(860.27983398,250.81103638)
\curveto(860.29982601,250.69103159)(860.32482599,250.56603172)(860.35483398,250.43603638)
\curveto(860.36482595,250.3860319)(860.36482595,250.33103195)(860.35483398,250.27103638)
\curveto(860.35482596,250.22103206)(860.35982595,250.17103211)(860.36983398,250.12103638)
\moveto(859.03483398,249.26603638)
\curveto(859.05482726,249.33603295)(859.05982725,249.41603287)(859.04983398,249.50603638)
\lineto(859.04983398,249.76103638)
\curveto(859.04982726,250.15103213)(859.0148273,250.4810318)(858.94483398,250.75103638)
\curveto(858.9148274,250.83103145)(858.88982742,250.91103137)(858.86983398,250.99103638)
\curveto(858.84982746,251.07103121)(858.82482749,251.14603114)(858.79483398,251.21603638)
\curveto(858.5148278,251.86603042)(858.06982824,252.31602997)(857.45983398,252.56603638)
\curveto(857.38982892,252.59602969)(857.314829,252.61602967)(857.23483398,252.62603638)
\lineto(856.99483398,252.68603638)
\curveto(856.9148294,252.70602958)(856.82982948,252.71602957)(856.73983398,252.71603638)
\lineto(856.46983398,252.71603638)
\lineto(856.19983398,252.67103638)
\curveto(856.09983021,252.65102963)(856.00483031,252.62602966)(855.91483398,252.59603638)
\curveto(855.83483048,252.57602971)(855.75483056,252.54602974)(855.67483398,252.50603638)
\curveto(855.60483071,252.4860298)(855.53983077,252.45602983)(855.47983398,252.41603638)
\curveto(855.41983089,252.37602991)(855.36483095,252.33602995)(855.31483398,252.29603638)
\curveto(855.07483124,252.12603016)(854.87983143,251.92103036)(854.72983398,251.68103638)
\curveto(854.57983173,251.44103084)(854.44983186,251.16103112)(854.33983398,250.84103638)
\curveto(854.309832,250.74103154)(854.28983202,250.63603165)(854.27983398,250.52603638)
\curveto(854.26983204,250.42603186)(854.25483206,250.32103196)(854.23483398,250.21103638)
\curveto(854.22483209,250.17103211)(854.21983209,250.10603218)(854.21983398,250.01603638)
\curveto(854.2098321,249.9860323)(854.20483211,249.95103233)(854.20483398,249.91103638)
\curveto(854.2148321,249.87103241)(854.21983209,249.82603246)(854.21983398,249.77603638)
\lineto(854.21983398,249.47603638)
\curveto(854.21983209,249.37603291)(854.22983208,249.286033)(854.24983398,249.20603638)
\lineto(854.27983398,249.02603638)
\curveto(854.29983201,248.92603336)(854.314832,248.82603346)(854.32483398,248.72603638)
\curveto(854.34483197,248.63603365)(854.37483194,248.55103373)(854.41483398,248.47103638)
\curveto(854.5148318,248.23103405)(854.62983168,248.00603428)(854.75983398,247.79603638)
\curveto(854.89983141,247.5860347)(855.06983124,247.41103487)(855.26983398,247.27103638)
\curveto(855.31983099,247.24103504)(855.36483095,247.21603507)(855.40483398,247.19603638)
\curveto(855.44483087,247.17603511)(855.48983082,247.15103513)(855.53983398,247.12103638)
\curveto(855.61983069,247.07103521)(855.70483061,247.02603526)(855.79483398,246.98603638)
\curveto(855.89483042,246.95603533)(855.99983031,246.92603536)(856.10983398,246.89603638)
\curveto(856.15983015,246.87603541)(856.20483011,246.86603542)(856.24483398,246.86603638)
\curveto(856.29483002,246.87603541)(856.34482997,246.87603541)(856.39483398,246.86603638)
\curveto(856.42482989,246.85603543)(856.48482983,246.84603544)(856.57483398,246.83603638)
\curveto(856.67482964,246.82603546)(856.74982956,246.83103545)(856.79983398,246.85103638)
\curveto(856.83982947,246.86103542)(856.87982943,246.86103542)(856.91983398,246.85103638)
\curveto(856.95982935,246.85103543)(856.99982931,246.86103542)(857.03983398,246.88103638)
\curveto(857.11982919,246.90103538)(857.19982911,246.91603537)(857.27983398,246.92603638)
\curveto(857.35982895,246.94603534)(857.43482888,246.97103531)(857.50483398,247.00103638)
\curveto(857.84482847,247.14103514)(858.11982819,247.33603495)(858.32983398,247.58603638)
\curveto(858.53982777,247.83603445)(858.7148276,248.13103415)(858.85483398,248.47103638)
\curveto(858.90482741,248.59103369)(858.93482738,248.71603357)(858.94483398,248.84603638)
\curveto(858.96482735,248.9860333)(858.99482732,249.12603316)(859.03483398,249.26603638)
}
}
{
\newrgbcolor{curcolor}{0 0 0}
\pscustom[linestyle=none,fillstyle=solid,fillcolor=curcolor]
{
\newpath
\moveto(868.47311523,253.54103638)
\curveto(868.54310763,253.49102879)(868.5781076,253.41602887)(868.57811523,253.31603638)
\curveto(868.58810759,253.21602907)(868.59310758,253.11102917)(868.59311523,253.00103638)
\lineto(868.59311523,246.73103638)
\lineto(868.59311523,246.13103638)
\curveto(868.5731076,246.0810362)(868.56810761,246.03103625)(868.57811523,245.98103638)
\curveto(868.58810759,245.94103634)(868.58310759,245.89603639)(868.56311523,245.84603638)
\curveto(868.54310763,245.74603654)(868.52810765,245.64603664)(868.51811523,245.54603638)
\curveto(868.51810766,245.43603685)(868.50310767,245.33103695)(868.47311523,245.23103638)
\curveto(868.44310773,245.12103716)(868.41310776,245.01603727)(868.38311523,244.91603638)
\curveto(868.36310781,244.81603747)(868.32810785,244.71603757)(868.27811523,244.61603638)
\curveto(868.178108,244.35603793)(868.04810813,244.12103816)(867.88811523,243.91103638)
\curveto(867.73810844,243.70103858)(867.55810862,243.52603876)(867.34811523,243.38603638)
\curveto(867.178109,243.26603902)(866.99810918,243.17103911)(866.80811523,243.10103638)
\curveto(866.61810956,243.02103926)(866.41310976,242.94603934)(866.19311523,242.87603638)
\curveto(866.10311007,242.85603943)(866.01311016,242.84603944)(865.92311523,242.84603638)
\curveto(865.83311034,242.83603945)(865.74311043,242.82103946)(865.65311523,242.80103638)
\lineto(865.56311523,242.80103638)
\curveto(865.54311063,242.79103949)(865.52311065,242.7860395)(865.50311523,242.78603638)
\curveto(865.45311072,242.77603951)(865.40311077,242.77603951)(865.35311523,242.78603638)
\curveto(865.31311086,242.79603949)(865.26811091,242.79103949)(865.21811523,242.77103638)
\curveto(865.14811103,242.75103953)(865.03811114,242.74603954)(864.88811523,242.75603638)
\curveto(864.74811143,242.75603953)(864.64811153,242.76603952)(864.58811523,242.78603638)
\curveto(864.55811162,242.7860395)(864.52811165,242.79103949)(864.49811523,242.80103638)
\lineto(864.43811523,242.80103638)
\curveto(864.34811183,242.82103946)(864.25811192,242.83603945)(864.16811523,242.84603638)
\curveto(864.0781121,242.84603944)(863.99311218,242.85603943)(863.91311523,242.87603638)
\curveto(863.83311234,242.89603939)(863.75311242,242.92103936)(863.67311523,242.95103638)
\curveto(863.59311258,242.97103931)(863.51311266,242.99603929)(863.43311523,243.02603638)
\curveto(863.11311306,243.15603913)(862.84311333,243.30103898)(862.62311523,243.46103638)
\curveto(862.41311376,243.62103866)(862.22311395,243.84603844)(862.05311523,244.13603638)
\curveto(862.03311414,244.15603813)(862.01811416,244.1810381)(862.00811523,244.21103638)
\curveto(862.00811417,244.23103805)(861.99811418,244.25603803)(861.97811523,244.28603638)
\curveto(861.94811423,244.36603792)(861.91311426,244.4810378)(861.87311523,244.63103638)
\curveto(861.84311433,244.77103751)(861.8731143,244.87603741)(861.96311523,244.94603638)
\curveto(862.02311415,244.99603729)(862.10311407,245.02103726)(862.20311523,245.02103638)
\lineto(862.53311523,245.02103638)
\lineto(862.69811523,245.02103638)
\curveto(862.75811342,245.02103726)(862.81311336,245.01103727)(862.86311523,244.99103638)
\curveto(862.95311322,244.96103732)(863.01811316,244.91103737)(863.05811523,244.84103638)
\curveto(863.09811308,244.77103751)(863.14311303,244.69603759)(863.19311523,244.61603638)
\lineto(863.31311523,244.43603638)
\curveto(863.36311281,244.36603792)(863.41311276,244.31103797)(863.46311523,244.27103638)
\curveto(863.71311246,244.0810382)(864.01311216,243.94103834)(864.36311523,243.85103638)
\curveto(864.42311175,243.83103845)(864.48311169,243.82103846)(864.54311523,243.82103638)
\curveto(864.61311156,243.81103847)(864.6781115,243.79603849)(864.73811523,243.77603638)
\lineto(864.82811523,243.77603638)
\curveto(864.89811128,243.75603853)(864.98311119,243.74603854)(865.08311523,243.74603638)
\curveto(865.18311099,243.74603854)(865.2731109,243.75603853)(865.35311523,243.77603638)
\curveto(865.38311079,243.7860385)(865.42311075,243.79103849)(865.47311523,243.79103638)
\curveto(865.5731106,243.81103847)(865.66811051,243.83103845)(865.75811523,243.85103638)
\curveto(865.84811033,243.86103842)(865.93311024,243.8860384)(866.01311523,243.92603638)
\curveto(866.30310987,244.04603824)(866.53810964,244.21103807)(866.71811523,244.42103638)
\curveto(866.90810927,244.62103766)(867.06310911,244.86603742)(867.18311523,245.15603638)
\curveto(867.22310895,245.24603704)(867.24810893,245.34103694)(867.25811523,245.44103638)
\curveto(867.2781089,245.54103674)(867.30310887,245.64603664)(867.33311523,245.75603638)
\curveto(867.35310882,245.80603648)(867.36310881,245.85603643)(867.36311523,245.90603638)
\curveto(867.36310881,245.95603633)(867.36810881,246.00603628)(867.37811523,246.05603638)
\curveto(867.38810879,246.0860362)(867.39310878,246.13603615)(867.39311523,246.20603638)
\curveto(867.41310876,246.286036)(867.41310876,246.37103591)(867.39311523,246.46103638)
\curveto(867.38310879,246.51103577)(867.3781088,246.55603573)(867.37811523,246.59603638)
\curveto(867.38810879,246.63603565)(867.38310879,246.67103561)(867.36311523,246.70103638)
\curveto(867.34310883,246.72103556)(867.32810885,246.73103555)(867.31811523,246.73103638)
\lineto(867.27311523,246.77603638)
\curveto(867.173109,246.77603551)(867.09810908,246.74603554)(867.04811523,246.68603638)
\curveto(867.00810917,246.63603565)(866.95810922,246.59103569)(866.89811523,246.55103638)
\lineto(866.65811523,246.34103638)
\curveto(866.5781096,246.281036)(866.48810969,246.22603606)(866.38811523,246.17603638)
\curveto(866.24810993,246.0860362)(866.0731101,246.01103627)(865.86311523,245.95103638)
\curveto(865.65311052,245.90103638)(865.43311074,245.86603642)(865.20311523,245.84603638)
\curveto(864.9731112,245.82603646)(864.74311143,245.83103645)(864.51311523,245.86103638)
\curveto(864.28311189,245.8810364)(864.0731121,245.92103636)(863.88311523,245.98103638)
\curveto(862.94311323,246.29103599)(862.28311389,246.8860354)(861.90311523,247.76603638)
\curveto(861.85311432,247.86603442)(861.81311436,247.96103432)(861.78311523,248.05103638)
\curveto(861.75311442,248.15103413)(861.71811446,248.25603403)(861.67811523,248.36603638)
\curveto(861.65811452,248.41603387)(861.64811453,248.46103382)(861.64811523,248.50103638)
\curveto(861.64811453,248.54103374)(861.63811454,248.5860337)(861.61811523,248.63603638)
\curveto(861.59811458,248.70603358)(861.58311459,248.77603351)(861.57311523,248.84603638)
\curveto(861.5731146,248.92603336)(861.56311461,249.00103328)(861.54311523,249.07103638)
\curveto(861.53311464,249.11103317)(861.52811465,249.14603314)(861.52811523,249.17603638)
\curveto(861.53811464,249.21603307)(861.53811464,249.25603303)(861.52811523,249.29603638)
\curveto(861.52811465,249.33603295)(861.52311465,249.37603291)(861.51311523,249.41603638)
\lineto(861.51311523,249.53603638)
\curveto(861.49311468,249.65603263)(861.49311468,249.7810325)(861.51311523,249.91103638)
\curveto(861.52311465,249.97103231)(861.52811465,250.03103225)(861.52811523,250.09103638)
\lineto(861.52811523,250.25603638)
\curveto(861.53811464,250.30603198)(861.54311463,250.34603194)(861.54311523,250.37603638)
\curveto(861.54311463,250.41603187)(861.54811463,250.46103182)(861.55811523,250.51103638)
\curveto(861.58811459,250.62103166)(861.60811457,250.72603156)(861.61811523,250.82603638)
\curveto(861.62811455,250.93603135)(861.65311452,251.04603124)(861.69311523,251.15603638)
\curveto(861.73311444,251.27603101)(861.76811441,251.39103089)(861.79811523,251.50103638)
\curveto(861.83811434,251.62103066)(861.88311429,251.73603055)(861.93311523,251.84603638)
\curveto(862.00311417,252.00603028)(862.08311409,252.15103013)(862.17311523,252.28103638)
\curveto(862.26311391,252.42102986)(862.35811382,252.55602973)(862.45811523,252.68603638)
\curveto(862.52811365,252.79602949)(862.61811356,252.8860294)(862.72811523,252.95603638)
\lineto(862.78811523,253.01603638)
\lineto(862.84811523,253.07603638)
\lineto(862.99811523,253.19603638)
\lineto(863.17811523,253.31603638)
\curveto(863.30811287,253.39602889)(863.44311273,253.46602882)(863.58311523,253.52603638)
\curveto(863.73311244,253.5860287)(863.89311228,253.64102864)(864.06311523,253.69103638)
\curveto(864.16311201,253.72102856)(864.26311191,253.74102854)(864.36311523,253.75103638)
\curveto(864.4731117,253.76102852)(864.58311159,253.77602851)(864.69311523,253.79603638)
\curveto(864.73311144,253.80602848)(864.78311139,253.80602848)(864.84311523,253.79603638)
\curveto(864.91311126,253.7860285)(864.96311121,253.79102849)(864.99311523,253.81103638)
\curveto(865.31311086,253.82102846)(865.59811058,253.79102849)(865.84811523,253.72103638)
\curveto(866.10811007,253.65102863)(866.33810984,253.55102873)(866.53811523,253.42103638)
\curveto(866.60810957,253.3810289)(866.6731095,253.33602895)(866.73311523,253.28603638)
\lineto(866.91311523,253.13603638)
\curveto(866.96310921,253.09602919)(867.00810917,253.05102923)(867.04811523,253.00103638)
\curveto(867.09810908,252.96102932)(867.173109,252.94102934)(867.27311523,252.94103638)
\lineto(867.31811523,252.98603638)
\curveto(867.33810884,253.00602928)(867.35810882,253.03102925)(867.37811523,253.06103638)
\curveto(867.40810877,253.14102914)(867.42310875,253.22102906)(867.42311523,253.30103638)
\curveto(867.43310874,253.3810289)(867.46310871,253.45102883)(867.51311523,253.51103638)
\curveto(867.54310863,253.55102873)(867.60310857,253.5810287)(867.69311523,253.60103638)
\curveto(867.78310839,253.63102865)(867.8781083,253.64602864)(867.97811523,253.64603638)
\curveto(868.0781081,253.64602864)(868.173108,253.63602865)(868.26311523,253.61603638)
\curveto(868.36310781,253.59602869)(868.43310774,253.57102871)(868.47311523,253.54103638)
\moveto(867.34811523,249.76103638)
\curveto(867.35810882,249.80103248)(867.36310881,249.85103243)(867.36311523,249.91103638)
\curveto(867.36310881,249.9810323)(867.35810882,250.03603225)(867.34811523,250.07603638)
\lineto(867.34811523,250.31603638)
\curveto(867.32810885,250.40603188)(867.31310886,250.49103179)(867.30311523,250.57103638)
\curveto(867.29310888,250.66103162)(867.2781089,250.74603154)(867.25811523,250.82603638)
\curveto(867.23810894,250.90603138)(867.21810896,250.9810313)(867.19811523,251.05103638)
\curveto(867.18810899,251.13103115)(867.16810901,251.20603108)(867.13811523,251.27603638)
\curveto(867.02810915,251.55603073)(866.88310929,251.80603048)(866.70311523,252.02603638)
\curveto(866.53310964,252.24603004)(866.31310986,252.41102987)(866.04311523,252.52103638)
\curveto(865.96311021,252.56102972)(865.8781103,252.59102969)(865.78811523,252.61103638)
\curveto(865.69811048,252.64102964)(865.60311057,252.66602962)(865.50311523,252.68603638)
\curveto(865.42311075,252.70602958)(865.33311084,252.71102957)(865.23311523,252.70103638)
\lineto(864.96311523,252.70103638)
\curveto(864.91311126,252.69102959)(864.86311131,252.6860296)(864.81311523,252.68603638)
\curveto(864.7731114,252.6860296)(864.72811145,252.6810296)(864.67811523,252.67103638)
\curveto(864.48811169,252.62102966)(864.32811185,252.57102971)(864.19811523,252.52103638)
\curveto(863.85811232,252.3810299)(863.59311258,252.17103011)(863.40311523,251.89103638)
\curveto(863.21311296,251.61103067)(863.06311311,251.286031)(862.95311523,250.91603638)
\curveto(862.93311324,250.83603145)(862.91811326,250.75603153)(862.90811523,250.67603638)
\curveto(862.90811327,250.60603168)(862.89811328,250.53103175)(862.87811523,250.45103638)
\curveto(862.85811332,250.42103186)(862.84811333,250.3860319)(862.84811523,250.34603638)
\curveto(862.85811332,250.30603198)(862.85811332,250.27103201)(862.84811523,250.24103638)
\lineto(862.84811523,249.91103638)
\lineto(862.84811523,249.56603638)
\curveto(862.84811333,249.45603283)(862.85811332,249.35103293)(862.87811523,249.25103638)
\lineto(862.87811523,249.17603638)
\curveto(862.88811329,249.14603314)(862.89311328,249.12103316)(862.89311523,249.10103638)
\curveto(862.91311326,249.01103327)(862.92811325,248.92103336)(862.93811523,248.83103638)
\curveto(862.95811322,248.74103354)(862.98311319,248.65603363)(863.01311523,248.57603638)
\curveto(863.09311308,248.31603397)(863.19311298,248.07603421)(863.31311523,247.85603638)
\curveto(863.43311274,247.63603465)(863.59311258,247.45603483)(863.79311523,247.31603638)
\lineto(863.91311523,247.22603638)
\curveto(863.95311222,247.20603508)(863.99811218,247.1860351)(864.04811523,247.16603638)
\curveto(864.12811205,247.11603517)(864.21311196,247.07603521)(864.30311523,247.04603638)
\curveto(864.39311178,247.01603527)(864.49311168,246.9860353)(864.60311523,246.95603638)
\curveto(864.65311152,246.94603534)(864.69811148,246.94103534)(864.73811523,246.94103638)
\curveto(864.78811139,246.95103533)(864.83811134,246.94603534)(864.88811523,246.92603638)
\curveto(864.91811126,246.91603537)(864.96811121,246.91103537)(865.03811523,246.91103638)
\curveto(865.10811107,246.91103537)(865.15811102,246.91603537)(865.18811523,246.92603638)
\curveto(865.21811096,246.93603535)(865.24811093,246.93603535)(865.27811523,246.92603638)
\curveto(865.31811086,246.92603536)(865.35811082,246.93103535)(865.39811523,246.94103638)
\curveto(865.48811069,246.96103532)(865.5731106,246.9810353)(865.65311523,247.00103638)
\curveto(865.73311044,247.02103526)(865.81311036,247.04603524)(865.89311523,247.07603638)
\curveto(866.23310994,247.22603506)(866.50310967,247.43603485)(866.70311523,247.70603638)
\curveto(866.90310927,247.97603431)(867.06310911,248.29103399)(867.18311523,248.65103638)
\curveto(867.21310896,248.74103354)(867.23310894,248.83103345)(867.24311523,248.92103638)
\curveto(867.26310891,249.02103326)(867.28310889,249.11603317)(867.30311523,249.20603638)
\curveto(867.31310886,249.24603304)(867.31810886,249.281033)(867.31811523,249.31103638)
\curveto(867.31810886,249.35103293)(867.32310885,249.39103289)(867.33311523,249.43103638)
\curveto(867.35310882,249.4810328)(867.35310882,249.53103275)(867.33311523,249.58103638)
\curveto(867.32310885,249.64103264)(867.32810885,249.70103258)(867.34811523,249.76103638)
}
}
{
\newrgbcolor{curcolor}{0 0 0}
\pscustom[linestyle=none,fillstyle=solid,fillcolor=curcolor]
{
\newpath
\moveto(874.11639648,253.82603638)
\curveto(874.34639169,253.82602846)(874.47639156,253.76602852)(874.50639648,253.64603638)
\curveto(874.5363915,253.53602875)(874.55139149,253.37102891)(874.55139648,253.15103638)
\lineto(874.55139648,252.86603638)
\curveto(874.55139149,252.77602951)(874.52639151,252.70102958)(874.47639648,252.64103638)
\curveto(874.41639162,252.56102972)(874.33139171,252.51602977)(874.22139648,252.50603638)
\curveto(874.11139193,252.50602978)(874.00139204,252.49102979)(873.89139648,252.46103638)
\curveto(873.75139229,252.43102985)(873.61639242,252.40102988)(873.48639648,252.37103638)
\curveto(873.36639267,252.34102994)(873.25139279,252.30102998)(873.14139648,252.25103638)
\curveto(872.85139319,252.12103016)(872.61639342,251.94103034)(872.43639648,251.71103638)
\curveto(872.25639378,251.49103079)(872.10139394,251.23603105)(871.97139648,250.94603638)
\curveto(871.93139411,250.83603145)(871.90139414,250.72103156)(871.88139648,250.60103638)
\curveto(871.86139418,250.49103179)(871.8363942,250.37603191)(871.80639648,250.25603638)
\curveto(871.79639424,250.20603208)(871.79139425,250.15603213)(871.79139648,250.10603638)
\curveto(871.80139424,250.05603223)(871.80139424,250.00603228)(871.79139648,249.95603638)
\curveto(871.76139428,249.83603245)(871.74639429,249.69603259)(871.74639648,249.53603638)
\curveto(871.75639428,249.3860329)(871.76139428,249.24103304)(871.76139648,249.10103638)
\lineto(871.76139648,247.25603638)
\lineto(871.76139648,246.91103638)
\curveto(871.76139428,246.79103549)(871.75639428,246.67603561)(871.74639648,246.56603638)
\curveto(871.7363943,246.45603583)(871.73139431,246.36103592)(871.73139648,246.28103638)
\curveto(871.7413943,246.20103608)(871.72139432,246.13103615)(871.67139648,246.07103638)
\curveto(871.62139442,246.00103628)(871.5413945,245.96103632)(871.43139648,245.95103638)
\curveto(871.33139471,245.94103634)(871.22139482,245.93603635)(871.10139648,245.93603638)
\lineto(870.83139648,245.93603638)
\curveto(870.78139526,245.95603633)(870.73139531,245.97103631)(870.68139648,245.98103638)
\curveto(870.6413954,246.00103628)(870.61139543,246.02603626)(870.59139648,246.05603638)
\curveto(870.5413955,246.12603616)(870.51139553,246.21103607)(870.50139648,246.31103638)
\lineto(870.50139648,246.64103638)
\lineto(870.50139648,247.79603638)
\lineto(870.50139648,251.95103638)
\lineto(870.50139648,252.98603638)
\lineto(870.50139648,253.28603638)
\curveto(870.51139553,253.3860289)(870.5413955,253.47102881)(870.59139648,253.54103638)
\curveto(870.62139542,253.5810287)(870.67139537,253.61102867)(870.74139648,253.63103638)
\curveto(870.82139522,253.65102863)(870.90639513,253.66102862)(870.99639648,253.66103638)
\curveto(871.08639495,253.67102861)(871.17639486,253.67102861)(871.26639648,253.66103638)
\curveto(871.35639468,253.65102863)(871.42639461,253.63602865)(871.47639648,253.61603638)
\curveto(871.55639448,253.5860287)(871.60639443,253.52602876)(871.62639648,253.43603638)
\curveto(871.65639438,253.35602893)(871.67139437,253.26602902)(871.67139648,253.16603638)
\lineto(871.67139648,252.86603638)
\curveto(871.67139437,252.76602952)(871.69139435,252.67602961)(871.73139648,252.59603638)
\curveto(871.7413943,252.57602971)(871.75139429,252.56102972)(871.76139648,252.55103638)
\lineto(871.80639648,252.50603638)
\curveto(871.91639412,252.50602978)(872.00639403,252.55102973)(872.07639648,252.64103638)
\curveto(872.14639389,252.74102954)(872.20639383,252.82102946)(872.25639648,252.88103638)
\lineto(872.34639648,252.97103638)
\curveto(872.4363936,253.0810292)(872.56139348,253.19602909)(872.72139648,253.31603638)
\curveto(872.88139316,253.43602885)(873.03139301,253.52602876)(873.17139648,253.58603638)
\curveto(873.26139278,253.63602865)(873.35639268,253.67102861)(873.45639648,253.69103638)
\curveto(873.55639248,253.72102856)(873.66139238,253.75102853)(873.77139648,253.78103638)
\curveto(873.83139221,253.79102849)(873.89139215,253.79602849)(873.95139648,253.79603638)
\curveto(874.01139203,253.80602848)(874.06639197,253.81602847)(874.11639648,253.82603638)
}
}
{
\newrgbcolor{curcolor}{0 0 0}
\pscustom[linestyle=none,fillstyle=solid,fillcolor=curcolor]
{
\newpath
\moveto(882.36616211,246.47603638)
\curveto(882.39615428,246.31603597)(882.38115429,246.1810361)(882.32116211,246.07103638)
\curveto(882.26115441,245.97103631)(882.18115449,245.89603639)(882.08116211,245.84603638)
\curveto(882.03115464,245.82603646)(881.9761547,245.81603647)(881.91616211,245.81603638)
\curveto(881.86615481,245.81603647)(881.81115486,245.80603648)(881.75116211,245.78603638)
\curveto(881.53115514,245.73603655)(881.31115536,245.75103653)(881.09116211,245.83103638)
\curveto(880.88115579,245.90103638)(880.73615594,245.99103629)(880.65616211,246.10103638)
\curveto(880.60615607,246.17103611)(880.56115611,246.25103603)(880.52116211,246.34103638)
\curveto(880.48115619,246.44103584)(880.43115624,246.52103576)(880.37116211,246.58103638)
\curveto(880.35115632,246.60103568)(880.32615635,246.62103566)(880.29616211,246.64103638)
\curveto(880.2761564,246.66103562)(880.24615643,246.66603562)(880.20616211,246.65603638)
\curveto(880.09615658,246.62603566)(879.99115668,246.57103571)(879.89116211,246.49103638)
\curveto(879.80115687,246.41103587)(879.71115696,246.34103594)(879.62116211,246.28103638)
\curveto(879.49115718,246.20103608)(879.35115732,246.12603616)(879.20116211,246.05603638)
\curveto(879.05115762,245.99603629)(878.89115778,245.94103634)(878.72116211,245.89103638)
\curveto(878.62115805,245.86103642)(878.51115816,245.84103644)(878.39116211,245.83103638)
\curveto(878.28115839,245.82103646)(878.1711585,245.80603648)(878.06116211,245.78603638)
\curveto(878.01115866,245.77603651)(877.96615871,245.77103651)(877.92616211,245.77103638)
\lineto(877.82116211,245.77103638)
\curveto(877.71115896,245.75103653)(877.60615907,245.75103653)(877.50616211,245.77103638)
\lineto(877.37116211,245.77103638)
\curveto(877.32115935,245.7810365)(877.2711594,245.7860365)(877.22116211,245.78603638)
\curveto(877.1711595,245.7860365)(877.12615955,245.79603649)(877.08616211,245.81603638)
\curveto(877.04615963,245.82603646)(877.01115966,245.83103645)(876.98116211,245.83103638)
\curveto(876.96115971,245.82103646)(876.93615974,245.82103646)(876.90616211,245.83103638)
\lineto(876.66616211,245.89103638)
\curveto(876.58616009,245.90103638)(876.51116016,245.92103636)(876.44116211,245.95103638)
\curveto(876.14116053,246.0810362)(875.89616078,246.22603606)(875.70616211,246.38603638)
\curveto(875.52616115,246.55603573)(875.3761613,246.79103549)(875.25616211,247.09103638)
\curveto(875.16616151,247.31103497)(875.12116155,247.57603471)(875.12116211,247.88603638)
\lineto(875.12116211,248.20103638)
\curveto(875.13116154,248.25103403)(875.13616154,248.30103398)(875.13616211,248.35103638)
\lineto(875.16616211,248.53103638)
\lineto(875.28616211,248.86103638)
\curveto(875.32616135,248.97103331)(875.3761613,249.07103321)(875.43616211,249.16103638)
\curveto(875.61616106,249.45103283)(875.86116081,249.66603262)(876.17116211,249.80603638)
\curveto(876.48116019,249.94603234)(876.82115985,250.07103221)(877.19116211,250.18103638)
\curveto(877.33115934,250.22103206)(877.4761592,250.25103203)(877.62616211,250.27103638)
\curveto(877.7761589,250.29103199)(877.92615875,250.31603197)(878.07616211,250.34603638)
\curveto(878.14615853,250.36603192)(878.21115846,250.37603191)(878.27116211,250.37603638)
\curveto(878.34115833,250.37603191)(878.41615826,250.3860319)(878.49616211,250.40603638)
\curveto(878.56615811,250.42603186)(878.63615804,250.43603185)(878.70616211,250.43603638)
\curveto(878.7761579,250.44603184)(878.85115782,250.46103182)(878.93116211,250.48103638)
\curveto(879.18115749,250.54103174)(879.41615726,250.59103169)(879.63616211,250.63103638)
\curveto(879.85615682,250.6810316)(880.03115664,250.79603149)(880.16116211,250.97603638)
\curveto(880.22115645,251.05603123)(880.2711564,251.15603113)(880.31116211,251.27603638)
\curveto(880.35115632,251.40603088)(880.35115632,251.54603074)(880.31116211,251.69603638)
\curveto(880.25115642,251.93603035)(880.16115651,252.12603016)(880.04116211,252.26603638)
\curveto(879.93115674,252.40602988)(879.7711569,252.51602977)(879.56116211,252.59603638)
\curveto(879.44115723,252.64602964)(879.29615738,252.6810296)(879.12616211,252.70103638)
\curveto(878.96615771,252.72102956)(878.79615788,252.73102955)(878.61616211,252.73103638)
\curveto(878.43615824,252.73102955)(878.26115841,252.72102956)(878.09116211,252.70103638)
\curveto(877.92115875,252.6810296)(877.7761589,252.65102963)(877.65616211,252.61103638)
\curveto(877.48615919,252.55102973)(877.32115935,252.46602982)(877.16116211,252.35603638)
\curveto(877.08115959,252.29602999)(877.00615967,252.21603007)(876.93616211,252.11603638)
\curveto(876.8761598,252.02603026)(876.82115985,251.92603036)(876.77116211,251.81603638)
\curveto(876.74115993,251.73603055)(876.71115996,251.65103063)(876.68116211,251.56103638)
\curveto(876.66116001,251.47103081)(876.61616006,251.40103088)(876.54616211,251.35103638)
\curveto(876.50616017,251.32103096)(876.43616024,251.29603099)(876.33616211,251.27603638)
\curveto(876.24616043,251.26603102)(876.15116052,251.26103102)(876.05116211,251.26103638)
\curveto(875.95116072,251.26103102)(875.85116082,251.26603102)(875.75116211,251.27603638)
\curveto(875.66116101,251.29603099)(875.59616108,251.32103096)(875.55616211,251.35103638)
\curveto(875.51616116,251.3810309)(875.48616119,251.43103085)(875.46616211,251.50103638)
\curveto(875.44616123,251.57103071)(875.44616123,251.64603064)(875.46616211,251.72603638)
\curveto(875.49616118,251.85603043)(875.52616115,251.97603031)(875.55616211,252.08603638)
\curveto(875.59616108,252.20603008)(875.64116103,252.32102996)(875.69116211,252.43103638)
\curveto(875.88116079,252.7810295)(876.12116055,253.05102923)(876.41116211,253.24103638)
\curveto(876.70115997,253.44102884)(877.06115961,253.60102868)(877.49116211,253.72103638)
\curveto(877.59115908,253.74102854)(877.69115898,253.75602853)(877.79116211,253.76603638)
\curveto(877.90115877,253.77602851)(878.01115866,253.79102849)(878.12116211,253.81103638)
\curveto(878.16115851,253.82102846)(878.22615845,253.82102846)(878.31616211,253.81103638)
\curveto(878.40615827,253.81102847)(878.46115821,253.82102846)(878.48116211,253.84103638)
\curveto(879.18115749,253.85102843)(879.79115688,253.77102851)(880.31116211,253.60103638)
\curveto(880.83115584,253.43102885)(881.19615548,253.10602918)(881.40616211,252.62603638)
\curveto(881.49615518,252.42602986)(881.54615513,252.19103009)(881.55616211,251.92103638)
\curveto(881.5761551,251.66103062)(881.58615509,251.3860309)(881.58616211,251.09603638)
\lineto(881.58616211,247.78103638)
\curveto(881.58615509,247.64103464)(881.59115508,247.50603478)(881.60116211,247.37603638)
\curveto(881.61115506,247.24603504)(881.64115503,247.14103514)(881.69116211,247.06103638)
\curveto(881.74115493,246.99103529)(881.80615487,246.94103534)(881.88616211,246.91103638)
\curveto(881.9761547,246.87103541)(882.06115461,246.84103544)(882.14116211,246.82103638)
\curveto(882.22115445,246.81103547)(882.28115439,246.76603552)(882.32116211,246.68603638)
\curveto(882.34115433,246.65603563)(882.35115432,246.62603566)(882.35116211,246.59603638)
\curveto(882.35115432,246.56603572)(882.35615432,246.52603576)(882.36616211,246.47603638)
\moveto(880.22116211,248.14103638)
\curveto(880.28115639,248.281034)(880.31115636,248.44103384)(880.31116211,248.62103638)
\curveto(880.32115635,248.81103347)(880.32615635,249.00603328)(880.32616211,249.20603638)
\curveto(880.32615635,249.31603297)(880.32115635,249.41603287)(880.31116211,249.50603638)
\curveto(880.30115637,249.59603269)(880.26115641,249.66603262)(880.19116211,249.71603638)
\curveto(880.16115651,249.73603255)(880.09115658,249.74603254)(879.98116211,249.74603638)
\curveto(879.96115671,249.72603256)(879.92615675,249.71603257)(879.87616211,249.71603638)
\curveto(879.82615685,249.71603257)(879.78115689,249.70603258)(879.74116211,249.68603638)
\curveto(879.66115701,249.66603262)(879.5711571,249.64603264)(879.47116211,249.62603638)
\lineto(879.17116211,249.56603638)
\curveto(879.14115753,249.56603272)(879.10615757,249.56103272)(879.06616211,249.55103638)
\lineto(878.96116211,249.55103638)
\curveto(878.81115786,249.51103277)(878.64615803,249.4860328)(878.46616211,249.47603638)
\curveto(878.29615838,249.47603281)(878.13615854,249.45603283)(877.98616211,249.41603638)
\curveto(877.90615877,249.39603289)(877.83115884,249.37603291)(877.76116211,249.35603638)
\curveto(877.70115897,249.34603294)(877.63115904,249.33103295)(877.55116211,249.31103638)
\curveto(877.39115928,249.26103302)(877.24115943,249.19603309)(877.10116211,249.11603638)
\curveto(876.96115971,249.04603324)(876.84115983,248.95603333)(876.74116211,248.84603638)
\curveto(876.64116003,248.73603355)(876.56616011,248.60103368)(876.51616211,248.44103638)
\curveto(876.46616021,248.29103399)(876.44616023,248.10603418)(876.45616211,247.88603638)
\curveto(876.45616022,247.7860345)(876.4711602,247.69103459)(876.50116211,247.60103638)
\curveto(876.54116013,247.52103476)(876.58616009,247.44603484)(876.63616211,247.37603638)
\curveto(876.71615996,247.26603502)(876.82115985,247.17103511)(876.95116211,247.09103638)
\curveto(877.08115959,247.02103526)(877.22115945,246.96103532)(877.37116211,246.91103638)
\curveto(877.42115925,246.90103538)(877.4711592,246.89603539)(877.52116211,246.89603638)
\curveto(877.5711591,246.89603539)(877.62115905,246.89103539)(877.67116211,246.88103638)
\curveto(877.74115893,246.86103542)(877.82615885,246.84603544)(877.92616211,246.83603638)
\curveto(878.03615864,246.83603545)(878.12615855,246.84603544)(878.19616211,246.86603638)
\curveto(878.25615842,246.8860354)(878.31615836,246.89103539)(878.37616211,246.88103638)
\curveto(878.43615824,246.8810354)(878.49615818,246.89103539)(878.55616211,246.91103638)
\curveto(878.63615804,246.93103535)(878.71115796,246.94603534)(878.78116211,246.95603638)
\curveto(878.86115781,246.96603532)(878.93615774,246.9860353)(879.00616211,247.01603638)
\curveto(879.29615738,247.13603515)(879.54115713,247.281035)(879.74116211,247.45103638)
\curveto(879.95115672,247.62103466)(880.11115656,247.85103443)(880.22116211,248.14103638)
}
}
{
\newrgbcolor{curcolor}{0 0 0}
\pscustom[linestyle=none,fillstyle=solid,fillcolor=curcolor]
{
\newpath
\moveto(885.98280273,256.72103638)
\curveto(886.16279919,256.73102555)(886.352799,256.73102555)(886.55280273,256.72103638)
\curveto(886.7527986,256.71102557)(886.89279846,256.65102563)(886.97280273,256.54103638)
\curveto(887.01279834,256.4810258)(887.03779832,256.40602588)(887.04780273,256.31603638)
\curveto(887.0577983,256.23602605)(887.06279829,256.14602614)(887.06280273,256.04603638)
\curveto(887.06279829,255.91602637)(887.03779832,255.81102647)(886.98780273,255.73103638)
\curveto(886.94779841,255.6810266)(886.88779847,255.64602664)(886.80780273,255.62603638)
\curveto(886.73779862,255.61602667)(886.6577987,255.61102667)(886.56780273,255.61103638)
\lineto(886.28280273,255.61103638)
\curveto(886.19279916,255.62102666)(886.11279924,255.62102666)(886.04280273,255.61103638)
\curveto(885.76279959,255.53102675)(885.57779978,255.40102688)(885.48780273,255.22103638)
\curveto(885.40779995,255.05102723)(885.36779999,254.79102749)(885.36780273,254.44103638)
\curveto(885.36779999,254.37102791)(885.36279999,254.29602799)(885.35280273,254.21603638)
\curveto(885.34280001,254.14602814)(885.34780001,254.0810282)(885.36780273,254.02103638)
\curveto(885.39779996,253.87102841)(885.46279989,253.76602852)(885.56280273,253.70603638)
\curveto(885.64279971,253.67602861)(885.74279961,253.66102862)(885.86280273,253.66103638)
\lineto(886.22280273,253.66103638)
\lineto(886.44780273,253.66103638)
\curveto(886.47779888,253.64102864)(886.50779885,253.63602865)(886.53780273,253.64603638)
\curveto(886.56779879,253.65602863)(886.59779876,253.65102863)(886.62780273,253.63103638)
\curveto(886.72779863,253.60102868)(886.79279856,253.54102874)(886.82280273,253.45103638)
\curveto(886.8527985,253.37102891)(886.86779849,253.26602902)(886.86780273,253.13603638)
\curveto(886.8577985,253.09602919)(886.8527985,253.05602923)(886.85280273,253.01603638)
\lineto(886.85280273,252.89603638)
\curveto(886.82279853,252.74602954)(886.7577986,252.64602964)(886.65780273,252.59603638)
\curveto(886.52779883,252.54602974)(886.357799,252.53102975)(886.14780273,252.55103638)
\curveto(885.94779941,252.5810297)(885.77779958,252.57602971)(885.63780273,252.53603638)
\curveto(885.5577998,252.51602977)(885.49779986,252.47602981)(885.45780273,252.41603638)
\curveto(885.41779994,252.36602992)(885.38779997,252.29602999)(885.36780273,252.20603638)
\curveto(885.34780001,252.13603015)(885.34280001,252.05603023)(885.35280273,251.96603638)
\curveto(885.36279999,251.87603041)(885.36779999,251.79103049)(885.36780273,251.71103638)
\lineto(885.36780273,250.72103638)
\lineto(885.36780273,247.54103638)
\lineto(885.36780273,246.79103638)
\lineto(885.36780273,246.59603638)
\curveto(885.37779998,246.52603576)(885.37279998,246.46603582)(885.35280273,246.41603638)
\lineto(885.35280273,246.29603638)
\lineto(885.32280273,246.17603638)
\curveto(885.31280004,246.13603615)(885.29780006,246.10103618)(885.27780273,246.07103638)
\curveto(885.22780013,246.00103628)(885.1528002,245.96103632)(885.05280273,245.95103638)
\curveto(884.9528004,245.94103634)(884.84280051,245.93603635)(884.72280273,245.93603638)
\lineto(884.43780273,245.93603638)
\curveto(884.38780097,245.95603633)(884.33780102,245.97103631)(884.28780273,245.98103638)
\curveto(884.24780111,246.00103628)(884.21280114,246.03603625)(884.18280273,246.08603638)
\curveto(884.16280119,246.11603617)(884.14280121,246.1810361)(884.12280273,246.28103638)
\lineto(884.12280273,246.38603638)
\curveto(884.10280125,246.43603585)(884.09280126,246.4860358)(884.09280273,246.53603638)
\curveto(884.10280125,246.59603569)(884.10780125,246.65103563)(884.10780273,246.70103638)
\lineto(884.10780273,247.30103638)
\lineto(884.10780273,251.39603638)
\lineto(884.10780273,251.74103638)
\curveto(884.11780124,251.86103042)(884.11780124,251.97103031)(884.10780273,252.07103638)
\curveto(884.10780125,252.1810301)(884.08780127,252.27603001)(884.04780273,252.35603638)
\curveto(884.01780134,252.43602985)(883.96280139,252.49102979)(883.88280273,252.52103638)
\curveto(883.82280153,252.55102973)(883.7528016,252.56602972)(883.67280273,252.56603638)
\lineto(883.44780273,252.56603638)
\lineto(883.20780273,252.56603638)
\curveto(883.13780222,252.56602972)(883.07280228,252.57602971)(883.01280273,252.59603638)
\curveto(882.92280243,252.63602965)(882.8578025,252.72102956)(882.81780273,252.85103638)
\curveto(882.80780255,252.90102938)(882.80280255,252.94602934)(882.80280273,252.98603638)
\lineto(882.80280273,253.12103638)
\curveto(882.80280255,253.26102902)(882.81780254,253.37102891)(882.84780273,253.45103638)
\curveto(882.87780248,253.54102874)(882.94280241,253.60102868)(883.04280273,253.63103638)
\curveto(883.11280224,253.66102862)(883.19280216,253.67102861)(883.28280273,253.66103638)
\lineto(883.56780273,253.66103638)
\curveto(883.66780169,253.66102862)(883.7528016,253.67102861)(883.82280273,253.69103638)
\curveto(883.90280145,253.71102857)(883.96780139,253.75102853)(884.01780273,253.81103638)
\curveto(884.08780127,253.89102839)(884.11780124,254.01602827)(884.10780273,254.18603638)
\lineto(884.10780273,254.66603638)
\curveto(884.10780125,254.86602742)(884.11780124,255.05102723)(884.13780273,255.22103638)
\curveto(884.16780119,255.40102688)(884.21280114,255.56102672)(884.27280273,255.70103638)
\curveto(884.38280097,255.94102634)(884.52780083,256.13602615)(884.70780273,256.28603638)
\curveto(884.89780046,256.43602585)(885.12280023,256.55102573)(885.38280273,256.63103638)
\curveto(885.44279991,256.65102563)(885.50279985,256.66102562)(885.56280273,256.66103638)
\curveto(885.63279972,256.67102561)(885.70279965,256.6860256)(885.77280273,256.70603638)
\curveto(885.79279956,256.71602557)(885.82779953,256.71602557)(885.87780273,256.70603638)
\curveto(885.92779943,256.70602558)(885.96279939,256.71102557)(885.98280273,256.72103638)
}
}
{
\newrgbcolor{curcolor}{0 0 0}
\pscustom[linestyle=none,fillstyle=solid,fillcolor=curcolor]
{
\newpath
\moveto(889.43092773,256.84103638)
\curveto(889.50092531,256.84102544)(889.58592522,256.84102544)(889.68592773,256.84103638)
\curveto(889.79592501,256.85102543)(889.89592491,256.85102543)(889.98592773,256.84103638)
\curveto(890.08592472,256.84102544)(890.17592463,256.83102545)(890.25592773,256.81103638)
\curveto(890.33592447,256.79102549)(890.39092442,256.76102552)(890.42092773,256.72103638)
\curveto(890.43092438,256.6810256)(890.42592438,256.62602566)(890.40592773,256.55603638)
\curveto(890.38592442,256.49602579)(890.34592446,256.43602585)(890.28592773,256.37603638)
\lineto(890.12092773,256.21103638)
\curveto(890.07092474,256.16102612)(890.02092479,256.10602618)(889.97092773,256.04603638)
\curveto(889.93092488,255.99602629)(889.88592492,255.94102634)(889.83592773,255.88103638)
\curveto(889.805925,255.83102645)(889.76592504,255.7860265)(889.71592773,255.74603638)
\curveto(889.67592513,255.71602657)(889.63592517,255.67602661)(889.59592773,255.62603638)
\lineto(889.55092773,255.58103638)
\curveto(889.55092526,255.57102671)(889.54092527,255.56102672)(889.52092773,255.55103638)
\curveto(889.48092533,255.50102678)(889.44092537,255.45602683)(889.40092773,255.41603638)
\curveto(889.36092545,255.3860269)(889.32092549,255.34602694)(889.28092773,255.29603638)
\curveto(889.26092555,255.25602703)(889.23092558,255.22102706)(889.19092773,255.19103638)
\lineto(889.10092773,255.10103638)
\curveto(889.06092575,255.05102723)(889.01592579,255.00102728)(888.96592773,254.95103638)
\curveto(888.92592588,254.90102738)(888.88092593,254.86102742)(888.83092773,254.83103638)
\curveto(888.76092605,254.79102749)(888.64592616,254.75602753)(888.48592773,254.72603638)
\curveto(888.33592647,254.70602758)(888.21592659,254.72102756)(888.12592773,254.77103638)
\curveto(888.09592671,254.79102749)(888.06592674,254.82102746)(888.03592773,254.86103638)
\curveto(888.01592679,254.91102737)(888.01592679,254.96602732)(888.03592773,255.02603638)
\curveto(888.05592675,255.10602718)(888.08592672,255.17602711)(888.12592773,255.23603638)
\curveto(888.16592664,255.30602698)(888.2109266,255.37102691)(888.26092773,255.43103638)
\curveto(888.34092647,255.57102671)(888.42592638,255.71602657)(888.51592773,255.86603638)
\curveto(888.6059262,256.01602627)(888.69592611,256.16102612)(888.78592773,256.30103638)
\lineto(888.90592773,256.51103638)
\curveto(888.94592586,256.59102569)(889.00092581,256.65602563)(889.07092773,256.70603638)
\curveto(889.14092567,256.75602553)(889.2109256,256.79602549)(889.28092773,256.82603638)
\curveto(889.3109255,256.82602546)(889.33592547,256.82602546)(889.35592773,256.82603638)
\curveto(889.38592542,256.83602545)(889.4109254,256.84102544)(889.43092773,256.84103638)
\moveto(889.38592773,252.98603638)
\lineto(889.38592773,253.27103638)
\curveto(889.38592542,253.37102891)(889.36092545,253.45102883)(889.31092773,253.51103638)
\curveto(889.26092555,253.59102869)(889.16592564,253.63102865)(889.02592773,253.63103638)
\curveto(888.89592591,253.64102864)(888.76592604,253.64602864)(888.63592773,253.64603638)
\curveto(888.61592619,253.63602865)(888.59092622,253.63102865)(888.56092773,253.63103638)
\curveto(888.54092627,253.64102864)(888.52092629,253.64602864)(888.50092773,253.64603638)
\curveto(888.44092637,253.62602866)(888.38592642,253.61102867)(888.33592773,253.60103638)
\curveto(888.28592652,253.59102869)(888.24592656,253.56102872)(888.21592773,253.51103638)
\curveto(888.16592664,253.45102883)(888.14092667,253.36602892)(888.14092773,253.25603638)
\lineto(888.14092773,252.94103638)
\lineto(888.14092773,246.59603638)
\lineto(888.14092773,246.31103638)
\curveto(888.14092667,246.22103606)(888.16092665,246.14603614)(888.20092773,246.08603638)
\curveto(888.25092656,246.00603628)(888.32092649,245.95603633)(888.41092773,245.93603638)
\curveto(888.5109263,245.92603636)(888.62592618,245.92103636)(888.75592773,245.92103638)
\lineto(888.98092773,245.92103638)
\curveto(889.06092575,245.94103634)(889.13092568,245.95603633)(889.19092773,245.96603638)
\curveto(889.25092556,245.9860363)(889.29592551,246.02603626)(889.32592773,246.08603638)
\curveto(889.37592543,246.14603614)(889.39592541,246.22603606)(889.38592773,246.32603638)
\lineto(889.38592773,246.64103638)
\lineto(889.38592773,252.98603638)
}
}
{
\newrgbcolor{curcolor}{0 0 0}
\pscustom[linestyle=none,fillstyle=solid,fillcolor=curcolor]
{
\newpath
\moveto(898.21577148,246.47603638)
\curveto(898.24576365,246.31603597)(898.23076367,246.1810361)(898.17077148,246.07103638)
\curveto(898.11076379,245.97103631)(898.03076387,245.89603639)(897.93077148,245.84603638)
\curveto(897.88076402,245.82603646)(897.82576407,245.81603647)(897.76577148,245.81603638)
\curveto(897.71576418,245.81603647)(897.66076424,245.80603648)(897.60077148,245.78603638)
\curveto(897.38076452,245.73603655)(897.16076474,245.75103653)(896.94077148,245.83103638)
\curveto(896.73076517,245.90103638)(896.58576531,245.99103629)(896.50577148,246.10103638)
\curveto(896.45576544,246.17103611)(896.41076549,246.25103603)(896.37077148,246.34103638)
\curveto(896.33076557,246.44103584)(896.28076562,246.52103576)(896.22077148,246.58103638)
\curveto(896.2007657,246.60103568)(896.17576572,246.62103566)(896.14577148,246.64103638)
\curveto(896.12576577,246.66103562)(896.0957658,246.66603562)(896.05577148,246.65603638)
\curveto(895.94576595,246.62603566)(895.84076606,246.57103571)(895.74077148,246.49103638)
\curveto(895.65076625,246.41103587)(895.56076634,246.34103594)(895.47077148,246.28103638)
\curveto(895.34076656,246.20103608)(895.2007667,246.12603616)(895.05077148,246.05603638)
\curveto(894.900767,245.99603629)(894.74076716,245.94103634)(894.57077148,245.89103638)
\curveto(894.47076743,245.86103642)(894.36076754,245.84103644)(894.24077148,245.83103638)
\curveto(894.13076777,245.82103646)(894.02076788,245.80603648)(893.91077148,245.78603638)
\curveto(893.86076804,245.77603651)(893.81576808,245.77103651)(893.77577148,245.77103638)
\lineto(893.67077148,245.77103638)
\curveto(893.56076834,245.75103653)(893.45576844,245.75103653)(893.35577148,245.77103638)
\lineto(893.22077148,245.77103638)
\curveto(893.17076873,245.7810365)(893.12076878,245.7860365)(893.07077148,245.78603638)
\curveto(893.02076888,245.7860365)(892.97576892,245.79603649)(892.93577148,245.81603638)
\curveto(892.895769,245.82603646)(892.86076904,245.83103645)(892.83077148,245.83103638)
\curveto(892.81076909,245.82103646)(892.78576911,245.82103646)(892.75577148,245.83103638)
\lineto(892.51577148,245.89103638)
\curveto(892.43576946,245.90103638)(892.36076954,245.92103636)(892.29077148,245.95103638)
\curveto(891.99076991,246.0810362)(891.74577015,246.22603606)(891.55577148,246.38603638)
\curveto(891.37577052,246.55603573)(891.22577067,246.79103549)(891.10577148,247.09103638)
\curveto(891.01577088,247.31103497)(890.97077093,247.57603471)(890.97077148,247.88603638)
\lineto(890.97077148,248.20103638)
\curveto(890.98077092,248.25103403)(890.98577091,248.30103398)(890.98577148,248.35103638)
\lineto(891.01577148,248.53103638)
\lineto(891.13577148,248.86103638)
\curveto(891.17577072,248.97103331)(891.22577067,249.07103321)(891.28577148,249.16103638)
\curveto(891.46577043,249.45103283)(891.71077019,249.66603262)(892.02077148,249.80603638)
\curveto(892.33076957,249.94603234)(892.67076923,250.07103221)(893.04077148,250.18103638)
\curveto(893.18076872,250.22103206)(893.32576857,250.25103203)(893.47577148,250.27103638)
\curveto(893.62576827,250.29103199)(893.77576812,250.31603197)(893.92577148,250.34603638)
\curveto(893.9957679,250.36603192)(894.06076784,250.37603191)(894.12077148,250.37603638)
\curveto(894.19076771,250.37603191)(894.26576763,250.3860319)(894.34577148,250.40603638)
\curveto(894.41576748,250.42603186)(894.48576741,250.43603185)(894.55577148,250.43603638)
\curveto(894.62576727,250.44603184)(894.7007672,250.46103182)(894.78077148,250.48103638)
\curveto(895.03076687,250.54103174)(895.26576663,250.59103169)(895.48577148,250.63103638)
\curveto(895.70576619,250.6810316)(895.88076602,250.79603149)(896.01077148,250.97603638)
\curveto(896.07076583,251.05603123)(896.12076578,251.15603113)(896.16077148,251.27603638)
\curveto(896.2007657,251.40603088)(896.2007657,251.54603074)(896.16077148,251.69603638)
\curveto(896.1007658,251.93603035)(896.01076589,252.12603016)(895.89077148,252.26603638)
\curveto(895.78076612,252.40602988)(895.62076628,252.51602977)(895.41077148,252.59603638)
\curveto(895.29076661,252.64602964)(895.14576675,252.6810296)(894.97577148,252.70103638)
\curveto(894.81576708,252.72102956)(894.64576725,252.73102955)(894.46577148,252.73103638)
\curveto(894.28576761,252.73102955)(894.11076779,252.72102956)(893.94077148,252.70103638)
\curveto(893.77076813,252.6810296)(893.62576827,252.65102963)(893.50577148,252.61103638)
\curveto(893.33576856,252.55102973)(893.17076873,252.46602982)(893.01077148,252.35603638)
\curveto(892.93076897,252.29602999)(892.85576904,252.21603007)(892.78577148,252.11603638)
\curveto(892.72576917,252.02603026)(892.67076923,251.92603036)(892.62077148,251.81603638)
\curveto(892.59076931,251.73603055)(892.56076934,251.65103063)(892.53077148,251.56103638)
\curveto(892.51076939,251.47103081)(892.46576943,251.40103088)(892.39577148,251.35103638)
\curveto(892.35576954,251.32103096)(892.28576961,251.29603099)(892.18577148,251.27603638)
\curveto(892.0957698,251.26603102)(892.0007699,251.26103102)(891.90077148,251.26103638)
\curveto(891.8007701,251.26103102)(891.7007702,251.26603102)(891.60077148,251.27603638)
\curveto(891.51077039,251.29603099)(891.44577045,251.32103096)(891.40577148,251.35103638)
\curveto(891.36577053,251.3810309)(891.33577056,251.43103085)(891.31577148,251.50103638)
\curveto(891.2957706,251.57103071)(891.2957706,251.64603064)(891.31577148,251.72603638)
\curveto(891.34577055,251.85603043)(891.37577052,251.97603031)(891.40577148,252.08603638)
\curveto(891.44577045,252.20603008)(891.49077041,252.32102996)(891.54077148,252.43103638)
\curveto(891.73077017,252.7810295)(891.97076993,253.05102923)(892.26077148,253.24103638)
\curveto(892.55076935,253.44102884)(892.91076899,253.60102868)(893.34077148,253.72103638)
\curveto(893.44076846,253.74102854)(893.54076836,253.75602853)(893.64077148,253.76603638)
\curveto(893.75076815,253.77602851)(893.86076804,253.79102849)(893.97077148,253.81103638)
\curveto(894.01076789,253.82102846)(894.07576782,253.82102846)(894.16577148,253.81103638)
\curveto(894.25576764,253.81102847)(894.31076759,253.82102846)(894.33077148,253.84103638)
\curveto(895.03076687,253.85102843)(895.64076626,253.77102851)(896.16077148,253.60103638)
\curveto(896.68076522,253.43102885)(897.04576485,253.10602918)(897.25577148,252.62603638)
\curveto(897.34576455,252.42602986)(897.3957645,252.19103009)(897.40577148,251.92103638)
\curveto(897.42576447,251.66103062)(897.43576446,251.3860309)(897.43577148,251.09603638)
\lineto(897.43577148,247.78103638)
\curveto(897.43576446,247.64103464)(897.44076446,247.50603478)(897.45077148,247.37603638)
\curveto(897.46076444,247.24603504)(897.49076441,247.14103514)(897.54077148,247.06103638)
\curveto(897.59076431,246.99103529)(897.65576424,246.94103534)(897.73577148,246.91103638)
\curveto(897.82576407,246.87103541)(897.91076399,246.84103544)(897.99077148,246.82103638)
\curveto(898.07076383,246.81103547)(898.13076377,246.76603552)(898.17077148,246.68603638)
\curveto(898.19076371,246.65603563)(898.2007637,246.62603566)(898.20077148,246.59603638)
\curveto(898.2007637,246.56603572)(898.20576369,246.52603576)(898.21577148,246.47603638)
\moveto(896.07077148,248.14103638)
\curveto(896.13076577,248.281034)(896.16076574,248.44103384)(896.16077148,248.62103638)
\curveto(896.17076573,248.81103347)(896.17576572,249.00603328)(896.17577148,249.20603638)
\curveto(896.17576572,249.31603297)(896.17076573,249.41603287)(896.16077148,249.50603638)
\curveto(896.15076575,249.59603269)(896.11076579,249.66603262)(896.04077148,249.71603638)
\curveto(896.01076589,249.73603255)(895.94076596,249.74603254)(895.83077148,249.74603638)
\curveto(895.81076609,249.72603256)(895.77576612,249.71603257)(895.72577148,249.71603638)
\curveto(895.67576622,249.71603257)(895.63076627,249.70603258)(895.59077148,249.68603638)
\curveto(895.51076639,249.66603262)(895.42076648,249.64603264)(895.32077148,249.62603638)
\lineto(895.02077148,249.56603638)
\curveto(894.99076691,249.56603272)(894.95576694,249.56103272)(894.91577148,249.55103638)
\lineto(894.81077148,249.55103638)
\curveto(894.66076724,249.51103277)(894.4957674,249.4860328)(894.31577148,249.47603638)
\curveto(894.14576775,249.47603281)(893.98576791,249.45603283)(893.83577148,249.41603638)
\curveto(893.75576814,249.39603289)(893.68076822,249.37603291)(893.61077148,249.35603638)
\curveto(893.55076835,249.34603294)(893.48076842,249.33103295)(893.40077148,249.31103638)
\curveto(893.24076866,249.26103302)(893.09076881,249.19603309)(892.95077148,249.11603638)
\curveto(892.81076909,249.04603324)(892.69076921,248.95603333)(892.59077148,248.84603638)
\curveto(892.49076941,248.73603355)(892.41576948,248.60103368)(892.36577148,248.44103638)
\curveto(892.31576958,248.29103399)(892.2957696,248.10603418)(892.30577148,247.88603638)
\curveto(892.30576959,247.7860345)(892.32076958,247.69103459)(892.35077148,247.60103638)
\curveto(892.39076951,247.52103476)(892.43576946,247.44603484)(892.48577148,247.37603638)
\curveto(892.56576933,247.26603502)(892.67076923,247.17103511)(892.80077148,247.09103638)
\curveto(892.93076897,247.02103526)(893.07076883,246.96103532)(893.22077148,246.91103638)
\curveto(893.27076863,246.90103538)(893.32076858,246.89603539)(893.37077148,246.89603638)
\curveto(893.42076848,246.89603539)(893.47076843,246.89103539)(893.52077148,246.88103638)
\curveto(893.59076831,246.86103542)(893.67576822,246.84603544)(893.77577148,246.83603638)
\curveto(893.88576801,246.83603545)(893.97576792,246.84603544)(894.04577148,246.86603638)
\curveto(894.10576779,246.8860354)(894.16576773,246.89103539)(894.22577148,246.88103638)
\curveto(894.28576761,246.8810354)(894.34576755,246.89103539)(894.40577148,246.91103638)
\curveto(894.48576741,246.93103535)(894.56076734,246.94603534)(894.63077148,246.95603638)
\curveto(894.71076719,246.96603532)(894.78576711,246.9860353)(894.85577148,247.01603638)
\curveto(895.14576675,247.13603515)(895.39076651,247.281035)(895.59077148,247.45103638)
\curveto(895.8007661,247.62103466)(895.96076594,247.85103443)(896.07077148,248.14103638)
}
}
{
\newrgbcolor{curcolor}{0.80000001 0.80000001 0.80000001}
\pscustom[linestyle=none,fillstyle=solid,fillcolor=curcolor]
{
\newpath
\moveto(811.9732666,385.02397156)
\lineto(826.9732666,385.02397156)
\lineto(826.9732666,370.02397156)
\lineto(811.9732666,370.02397156)
\closepath
}
}
{
\newrgbcolor{curcolor}{0.7019608 0.7019608 0.7019608}
\pscustom[linestyle=none,fillstyle=solid,fillcolor=curcolor]
{
\newpath
\moveto(811.9732666,361.69264984)
\lineto(826.9732666,361.69264984)
\lineto(826.9732666,346.69264984)
\lineto(811.9732666,346.69264984)
\closepath
}
}
{
\newrgbcolor{curcolor}{0.60000002 0.60000002 0.60000002}
\pscustom[linestyle=none,fillstyle=solid,fillcolor=curcolor]
{
\newpath
\moveto(811.9732666,338.37634277)
\lineto(826.9732666,338.37634277)
\lineto(826.9732666,323.37634277)
\lineto(811.9732666,323.37634277)
\closepath
}
}
{
\newrgbcolor{curcolor}{0.50196081 0.50196081 0.50196081}
\pscustom[linestyle=none,fillstyle=solid,fillcolor=curcolor]
{
\newpath
\moveto(811.9732666,297.3986969)
\lineto(826.9732666,297.3986969)
\lineto(826.9732666,282.3986969)
\lineto(811.9732666,282.3986969)
\closepath
}
}
{
\newrgbcolor{curcolor}{0.40000001 0.40000001 0.40000001}
\pscustom[linestyle=none,fillstyle=solid,fillcolor=curcolor]
{
\newpath
\moveto(811.9732666,256.73603821)
\lineto(826.9732666,256.73603821)
\lineto(826.9732666,241.73603821)
\lineto(811.9732666,241.73603821)
\closepath
}
}
{
\newrgbcolor{curcolor}{0.80000001 0.80000001 0.80000001}
\pscustom[linestyle=none,fillstyle=solid,fillcolor=curcolor]
{
\newpath
\moveto(268.06921387,343.9569397)
\lineto(278.99412441,343.9569397)
\lineto(278.99412441,89.08282471)
\lineto(268.06921387,89.08282471)
\closepath
}
}
{
\newrgbcolor{curcolor}{0.7019608 0.7019608 0.7019608}
\pscustom[linestyle=none,fillstyle=solid,fillcolor=curcolor]
{
\newpath
\moveto(279.05358887,139.0038147)
\lineto(289.97849941,139.0038147)
\lineto(289.97849941,89.08282471)
\lineto(279.05358887,89.08282471)
\closepath
}
}
{
\newrgbcolor{curcolor}{0.60000002 0.60000002 0.60000002}
\pscustom[linestyle=none,fillstyle=solid,fillcolor=curcolor]
{
\newpath
\moveto(290.03796387,111.0194397)
\lineto(300.96287441,111.0194397)
\lineto(300.96287441,89.08282471)
\lineto(290.03796387,89.08282471)
\closepath
}
}
{
\newrgbcolor{curcolor}{0.50196081 0.50196081 0.50196081}
\pscustom[linestyle=none,fillstyle=solid,fillcolor=curcolor]
{
\newpath
\moveto(301.02233887,94.98904419)
\lineto(311.94724941,94.98904419)
\lineto(311.94724941,89.08280468)
\lineto(301.02233887,89.08280468)
\closepath
}
}
{
\newrgbcolor{curcolor}{0.40000001 0.40000001 0.40000001}
\pscustom[linestyle=none,fillstyle=solid,fillcolor=curcolor]
{
\newpath
\moveto(312.00671387,92.98480225)
\lineto(322.93162441,92.98480225)
\lineto(322.93162441,89.08281684)
\lineto(312.00671387,89.08281684)
\closepath
}
}
{
\newrgbcolor{curcolor}{0.80000001 0.80000001 0.80000001}
\pscustom[linestyle=none,fillstyle=solid,fillcolor=curcolor]
{
\newpath
\moveto(177.00671387,92.99966431)
\lineto(187.93162441,92.99966431)
\lineto(187.93162441,89.0828166)
\lineto(177.00671387,89.0828166)
\closepath
}
}
{
\newrgbcolor{curcolor}{0.7019608 0.7019608 0.7019608}
\pscustom[linestyle=none,fillstyle=solid,fillcolor=curcolor]
{
\newpath
\moveto(187.99108887,92.66955566)
\lineto(198.91599941,92.66955566)
\lineto(198.91599941,89.08283734)
\lineto(187.99108887,89.08283734)
\closepath
}
}
{
\newrgbcolor{curcolor}{0.60000002 0.60000002 0.60000002}
\pscustom[linestyle=none,fillstyle=solid,fillcolor=curcolor]
{
\newpath
\moveto(198.97546387,92.9881897)
\lineto(209.90037441,92.9881897)
\lineto(209.90037441,89.08282471)
\lineto(198.97546387,89.08282471)
\closepath
}
}
{
\newrgbcolor{curcolor}{0.50196081 0.50196081 0.50196081}
\pscustom[linestyle=none,fillstyle=solid,fillcolor=curcolor]
{
\newpath
\moveto(209.95983887,92.98904419)
\lineto(220.88474941,92.98904419)
\lineto(220.88474941,89.08280468)
\lineto(209.95983887,89.08280468)
\closepath
}
}
{
\newrgbcolor{curcolor}{0.40000001 0.40000001 0.40000001}
\pscustom[linestyle=none,fillstyle=solid,fillcolor=curcolor]
{
\newpath
\moveto(220.94421387,89.70355225)
\lineto(231.86912441,89.70355225)
\lineto(231.86912441,89.08281684)
\lineto(220.94421387,89.08281684)
\closepath
}
}
{
\newrgbcolor{curcolor}{0.80000001 0.80000001 0.80000001}
\pscustom[linestyle=none,fillstyle=solid,fillcolor=curcolor]
{
\newpath
\moveto(360.01507568,91.98318481)
\lineto(370.93998623,91.98318481)
\lineto(370.93998623,89.08280301)
\lineto(360.01507568,89.08280301)
\closepath
}
}
{
\newrgbcolor{curcolor}{0.7019608 0.7019608 0.7019608}
\pscustom[linestyle=none,fillstyle=solid,fillcolor=curcolor]
{
\newpath
\moveto(370.99945068,91.07855225)
\lineto(381.92436123,91.07855225)
\lineto(381.92436123,89.08282423)
\lineto(370.99945068,89.08282423)
\closepath
}
}
{
\newrgbcolor{curcolor}{0.60000002 0.60000002 0.60000002}
\pscustom[linestyle=none,fillstyle=solid,fillcolor=curcolor]
{
\newpath
\moveto(381.98382568,92.01593018)
\lineto(392.90873623,92.01593018)
\lineto(392.90873623,89.0828371)
\lineto(381.98382568,89.0828371)
\closepath
}
}
{
\newrgbcolor{curcolor}{0.50196081 0.50196081 0.50196081}
\pscustom[linestyle=none,fillstyle=solid,fillcolor=curcolor]
{
\newpath
\moveto(392.96820068,90.51416016)
\lineto(403.89311123,90.51416016)
\lineto(403.89311123,89.08279443)
\lineto(392.96820068,89.08279443)
\closepath
}
}
{
\newrgbcolor{curcolor}{0.40000001 0.40000001 0.40000001}
\pscustom[linestyle=none,fillstyle=solid,fillcolor=curcolor]
{
\newpath
\moveto(403.95257568,89.96871948)
\lineto(414.87748623,89.96871948)
\lineto(414.87748623,89.08281904)
\lineto(403.95257568,89.08281904)
\closepath
}
}
{
\newrgbcolor{curcolor}{0.80000001 0.80000001 0.80000001}
\pscustom[linestyle=none,fillstyle=solid,fillcolor=curcolor]
{
\newpath
\moveto(450.96664429,90.03863525)
\lineto(461.89155483,90.03863525)
\lineto(461.89155483,89.08279711)
\lineto(450.96664429,89.08279711)
\closepath
}
}
{
\newrgbcolor{curcolor}{0.7019608 0.7019608 0.7019608}
\pscustom[linestyle=none,fillstyle=solid,fillcolor=curcolor]
{
\newpath
\moveto(461.95101929,90.03997803)
\lineto(472.87592983,90.03997803)
\lineto(472.87592983,89.08281308)
\lineto(461.95101929,89.08281308)
\closepath
}
}
{
\newrgbcolor{curcolor}{0.60000002 0.60000002 0.60000002}
\pscustom[linestyle=none,fillstyle=solid,fillcolor=curcolor]
{
\newpath
\moveto(472.93539429,90.03634644)
\lineto(483.86030483,90.03634644)
\lineto(483.86030483,89.0828383)
\lineto(472.93539429,89.0828383)
\closepath
}
}
{
\newrgbcolor{curcolor}{0.50196081 0.50196081 0.50196081}
\pscustom[linestyle=none,fillstyle=solid,fillcolor=curcolor]
{
\newpath
\moveto(483.91976929,90.02801514)
\lineto(494.84467983,90.02801514)
\lineto(494.84467983,89.08278531)
\lineto(483.91976929,89.08278531)
\closepath
}
}
{
\newrgbcolor{curcolor}{0.40000001 0.40000001 0.40000001}
\pscustom[linestyle=none,fillstyle=solid,fillcolor=curcolor]
{
\newpath
\moveto(494.90414429,89.47439575)
\lineto(505.82905483,89.47439575)
\lineto(505.82905483,89.08280843)
\lineto(494.90414429,89.08280843)
\closepath
}
}
{
\newrgbcolor{curcolor}{0.80000001 0.80000001 0.80000001}
\pscustom[linestyle=none,fillstyle=solid,fillcolor=curcolor]
{
\newpath
\moveto(542.02270508,89.51934814)
\lineto(552.94761562,89.51934814)
\lineto(552.94761562,89.08279154)
\lineto(542.02270508,89.08279154)
\closepath
}
}
{
\newrgbcolor{curcolor}{0.7019608 0.7019608 0.7019608}
\pscustom[linestyle=none,fillstyle=solid,fillcolor=curcolor]
{
\newpath
\moveto(553.00708008,89.49859619)
\lineto(563.93199062,89.49859619)
\lineto(563.93199062,89.08280987)
\lineto(553.00708008,89.08280987)
\closepath
}
}
{
\newrgbcolor{curcolor}{0.60000002 0.60000002 0.60000002}
\pscustom[linestyle=none,fillstyle=solid,fillcolor=curcolor]
{
\newpath
\moveto(563.99145508,89.48391724)
\lineto(574.91636562,89.48391724)
\lineto(574.91636562,89.08283627)
\lineto(563.99145508,89.08283627)
\closepath
}
}
{
\newrgbcolor{curcolor}{0.50196081 0.50196081 0.50196081}
\pscustom[linestyle=none,fillstyle=solid,fillcolor=curcolor]
{
\newpath
\moveto(574.97583008,89.47802734)
\lineto(585.90074062,89.47802734)
\lineto(585.90074062,89.08275223)
\lineto(574.97583008,89.08275223)
\closepath
}
}
{
\newrgbcolor{curcolor}{0.80000001 0.80000001 0.80000001}
\pscustom[linestyle=none,fillstyle=solid,fillcolor=curcolor]
{
\newpath
\moveto(634.01263428,90.03863525)
\lineto(644.93754482,90.03863525)
\lineto(644.93754482,89.08279711)
\lineto(634.01263428,89.08279711)
\closepath
}
}
{
\newrgbcolor{curcolor}{0.7019608 0.7019608 0.7019608}
\pscustom[linestyle=none,fillstyle=solid,fillcolor=curcolor]
{
\newpath
\moveto(644.99700928,90.03997803)
\lineto(655.92191982,90.03997803)
\lineto(655.92191982,89.08281308)
\lineto(644.99700928,89.08281308)
\closepath
}
}
{
\newrgbcolor{curcolor}{0.60000002 0.60000002 0.60000002}
\pscustom[linestyle=none,fillstyle=solid,fillcolor=curcolor]
{
\newpath
\moveto(655.98138428,90.03634644)
\lineto(666.90629482,90.03634644)
\lineto(666.90629482,89.0828383)
\lineto(655.98138428,89.0828383)
\closepath
}
}
{
\newrgbcolor{curcolor}{0.50196081 0.50196081 0.50196081}
\pscustom[linestyle=none,fillstyle=solid,fillcolor=curcolor]
{
\newpath
\moveto(666.96575928,90.02801514)
\lineto(677.89066982,90.02801514)
\lineto(677.89066982,89.08278531)
\lineto(666.96575928,89.08278531)
\closepath
}
}
{
\newrgbcolor{curcolor}{0.40000001 0.40000001 0.40000001}
\pscustom[linestyle=none,fillstyle=solid,fillcolor=curcolor]
{
\newpath
\moveto(677.95013428,89.7064209)
\lineto(688.87504482,89.7064209)
\lineto(688.87504482,89.08281416)
\lineto(677.95013428,89.08281416)
\closepath
}
}
{
\newrgbcolor{curcolor}{0.80000001 0.80000001 0.80000001}
\pscustom[linestyle=none,fillstyle=solid,fillcolor=curcolor]
{
\newpath
\moveto(724.96020508,89.51934814)
\lineto(735.88511562,89.51934814)
\lineto(735.88511562,89.08279154)
\lineto(724.96020508,89.08279154)
\closepath
}
}
{
\newrgbcolor{curcolor}{0.7019608 0.7019608 0.7019608}
\pscustom[linestyle=none,fillstyle=solid,fillcolor=curcolor]
{
\newpath
\moveto(735.94458008,89.49859619)
\lineto(746.86949062,89.49859619)
\lineto(746.86949062,89.08280987)
\lineto(735.94458008,89.08280987)
\closepath
}
}
{
\newrgbcolor{curcolor}{0.60000002 0.60000002 0.60000002}
\pscustom[linestyle=none,fillstyle=solid,fillcolor=curcolor]
{
\newpath
\moveto(746.92895508,89.48391724)
\lineto(757.85386562,89.48391724)
\lineto(757.85386562,89.08283627)
\lineto(746.92895508,89.08283627)
\closepath
}
}
{
\newrgbcolor{curcolor}{0.50196081 0.50196081 0.50196081}
\pscustom[linestyle=none,fillstyle=solid,fillcolor=curcolor]
{
\newpath
\moveto(757.91333008,89.47802734)
\lineto(768.83824062,89.47802734)
\lineto(768.83824062,89.08275223)
\lineto(757.91333008,89.08275223)
\closepath
}
}
{
\newrgbcolor{curcolor}{0.40000001 0.40000001 0.40000001}
\pscustom[linestyle=none,fillstyle=solid,fillcolor=curcolor]
{
\newpath
\moveto(768.9041748,89.47802734)
\lineto(779.82908535,89.47802734)
\lineto(779.82908535,89.08275223)
\lineto(768.9041748,89.08275223)
\closepath
}
}
\end{pspicture}

\caption{Diagrama de barras de la intención de los usuarios clasificados por
rol}
\label{usuarios_bars_1}
\end{figure}

Puede verse en la figura \ref{usuarios_pie_1} como el predominio en cantidad de
los estudiantes va decayendo progresivamente en intención frente a los otros
roles. Considerando los escasa cantidad de atractivos que posee el sistema es
importante considerar una audiencia de 126 personas como el primer paso hacia
la construcción de un lugar común para el estudio realizado.

\begin{figure}
\centering
%LaTeX with PSTricks extensions
%%Creator: inkscape 0.48.5
%%Please note this file requires PSTricks extensions
\psset{xunit=.5pt,yunit=.5pt,runit=.5pt}
\begin{pspicture}(790,1050)
{
\newrgbcolor{curcolor}{0 0 0}
\pscustom[linestyle=none,fillstyle=solid,fillcolor=curcolor]
{
\newpath
\moveto(20.6807132,1042.99996658)
\lineto(21.9557132,1042.99996658)
\curveto(22.06571042,1042.99995587)(22.17071031,1042.99495588)(22.2707132,1042.98496658)
\curveto(22.3807101,1042.9749559)(22.46071002,1042.93995593)(22.5107132,1042.87996658)
\curveto(22.56070992,1042.79995607)(22.5857099,1042.69495618)(22.5857132,1042.56496658)
\curveto(22.59570989,1042.44495643)(22.60070988,1042.31995655)(22.6007132,1042.18996658)
\lineto(22.6007132,1040.67496658)
\lineto(22.6007132,1037.58496658)
\lineto(22.6007132,1037.05996658)
\curveto(22.60070988,1037.01996185)(22.59570989,1036.9749619)(22.5857132,1036.92496658)
\curveto(22.5857099,1036.88496199)(22.59070989,1036.84496203)(22.6007132,1036.80496658)
\lineto(22.6007132,1036.56496658)
\curveto(22.60070988,1036.4749624)(22.59570989,1036.37996249)(22.5857132,1036.27996658)
\curveto(22.5857099,1036.17996269)(22.59570989,1036.08996278)(22.6157132,1036.00996658)
\curveto(22.61570987,1035.93996293)(22.62070986,1035.88496299)(22.6307132,1035.84496658)
\curveto(22.65070983,1035.73496314)(22.66570982,1035.62496325)(22.6757132,1035.51496658)
\curveto(22.69570979,1035.40496347)(22.72570976,1035.29496358)(22.7657132,1035.18496658)
\curveto(22.87570961,1034.92496395)(23.01570947,1034.70996416)(23.1857132,1034.53996658)
\curveto(23.36570912,1034.3699645)(23.60070888,1034.23496464)(23.8907132,1034.13496658)
\curveto(23.97070851,1034.11496476)(24.05070843,1034.09996477)(24.1307132,1034.08996658)
\curveto(24.21070827,1034.07996479)(24.29070819,1034.06496481)(24.3707132,1034.04496658)
\curveto(24.42070806,1034.02496485)(24.46570802,1034.01496486)(24.5057132,1034.01496658)
\curveto(24.54570794,1034.02496485)(24.59070789,1034.02496485)(24.6407132,1034.01496658)
\curveto(24.6807078,1034.00496487)(24.74570774,1033.99996487)(24.8357132,1033.99996658)
\curveto(24.92570756,1034.00996486)(24.9857075,1034.01996485)(25.0157132,1034.02996658)
\lineto(25.2407132,1034.02996658)
\curveto(25.32070716,1034.04996482)(25.40070708,1034.06496481)(25.4807132,1034.07496658)
\curveto(25.56070692,1034.08496479)(25.63570685,1034.09996477)(25.7057132,1034.11996658)
\curveto(25.84570664,1034.14996472)(25.95570653,1034.18496469)(26.0357132,1034.22496658)
\curveto(26.21570627,1034.30496457)(26.37070611,1034.40996446)(26.5007132,1034.53996658)
\curveto(26.64070584,1034.67996419)(26.75070573,1034.83496404)(26.8307132,1035.00496658)
\curveto(26.94070554,1035.26496361)(27.00570548,1035.5699633)(27.0257132,1035.91996658)
\curveto(27.04570544,1036.27996259)(27.05570543,1036.64996222)(27.0557132,1037.02996658)
\lineto(27.0557132,1040.01496658)
\lineto(27.0557132,1042.02496658)
\curveto(27.05570543,1042.16495671)(27.05070543,1042.31995655)(27.0407132,1042.48996658)
\curveto(27.04070544,1042.65995621)(27.07070541,1042.78495609)(27.1307132,1042.86496658)
\curveto(27.1807053,1042.92495595)(27.25070523,1042.95995591)(27.3407132,1042.96996658)
\curveto(27.43070505,1042.98995588)(27.53070495,1042.99995587)(27.6407132,1042.99996658)
\lineto(28.6007132,1042.99996658)
\curveto(28.6807038,1042.99995587)(28.75570373,1042.99995587)(28.8257132,1042.99996658)
\curveto(28.90570358,1043.00995586)(28.9807035,1043.00495587)(29.0507132,1042.98496658)
\curveto(29.19070329,1042.95495592)(29.2807032,1042.90495597)(29.3207132,1042.83496658)
\curveto(29.37070311,1042.75495612)(29.39070309,1042.63995623)(29.3807132,1042.48996658)
\curveto(29.3807031,1042.34995652)(29.3807031,1042.21995665)(29.3807132,1042.09996658)
\lineto(29.3807132,1040.08996658)
\lineto(29.3807132,1037.05996658)
\curveto(29.3807031,1036.67996219)(29.37570311,1036.30996256)(29.3657132,1035.94996658)
\curveto(29.35570313,1035.58996328)(29.31070317,1035.26496361)(29.2307132,1034.97496658)
\curveto(29.09070339,1034.50496437)(28.91070357,1034.09496478)(28.6907132,1033.74496658)
\curveto(28.480704,1033.40496547)(28.20070428,1033.11496576)(27.8507132,1032.87496658)
\curveto(27.54070494,1032.65496622)(27.17570531,1032.4749664)(26.7557132,1032.33496658)
\curveto(26.66570582,1032.30496657)(26.57070591,1032.27996659)(26.4707132,1032.25996658)
\lineto(26.2007132,1032.19996658)
\curveto(26.14070634,1032.17996669)(26.0807064,1032.1699667)(26.0207132,1032.16996658)
\curveto(25.97070651,1032.1699667)(25.91570657,1032.15996671)(25.8557132,1032.13996658)
\curveto(25.73570675,1032.11996675)(25.60070688,1032.10496677)(25.4507132,1032.09496658)
\curveto(25.30070718,1032.08496679)(25.15570733,1032.07996679)(25.0157132,1032.07996658)
\curveto(24.06570842,1032.0699668)(23.25570923,1032.18496669)(22.5857132,1032.42496658)
\curveto(21.91571057,1032.6749662)(21.39071109,1033.0749658)(21.0107132,1033.62496658)
\curveto(20.8807116,1033.80496507)(20.77071171,1033.98996488)(20.6807132,1034.17996658)
\curveto(20.60071188,1034.37996449)(20.52571196,1034.59496428)(20.4557132,1034.82496658)
\curveto(20.43571205,1034.874964)(20.42571206,1034.91496396)(20.4257132,1034.94496658)
\curveto(20.42571206,1034.98496389)(20.41571207,1035.02996384)(20.3957132,1035.07996658)
\curveto(20.31571217,1035.35996351)(20.27571221,1035.6749632)(20.2757132,1036.02496658)
\lineto(20.2757132,1037.07496658)
\lineto(20.2757132,1041.25996658)
\lineto(20.2757132,1042.30996658)
\lineto(20.2757132,1042.59496658)
\curveto(20.27571221,1042.69495618)(20.29071219,1042.7749561)(20.3207132,1042.83496658)
\curveto(20.3807121,1042.90495597)(20.46071202,1042.95495592)(20.5607132,1042.98496658)
\curveto(20.5807119,1042.98495589)(20.60071188,1042.98495589)(20.6207132,1042.98496658)
\curveto(20.64071184,1042.98495589)(20.66071182,1042.98995588)(20.6807132,1042.99996658)
}
}
{
\newrgbcolor{curcolor}{0 0 0}
\pscustom[linestyle=none,fillstyle=solid,fillcolor=curcolor]
{
\newpath
\moveto(34.13922882,1040.23996658)
\curveto(34.88922432,1040.25995861)(35.53922367,1040.1749587)(36.08922882,1039.98496658)
\curveto(36.64922256,1039.80495907)(37.07422214,1039.48995938)(37.36422882,1039.03996658)
\curveto(37.43422178,1038.92995994)(37.49422172,1038.81496006)(37.54422882,1038.69496658)
\curveto(37.60422161,1038.58496029)(37.65422156,1038.45996041)(37.69422882,1038.31996658)
\curveto(37.7142215,1038.25996061)(37.72422149,1038.19496068)(37.72422882,1038.12496658)
\curveto(37.72422149,1038.05496082)(37.7142215,1037.99496088)(37.69422882,1037.94496658)
\curveto(37.65422156,1037.88496099)(37.59922161,1037.84496103)(37.52922882,1037.82496658)
\curveto(37.47922173,1037.80496107)(37.41922179,1037.79496108)(37.34922882,1037.79496658)
\lineto(37.13922882,1037.79496658)
\lineto(36.47922882,1037.79496658)
\curveto(36.4092228,1037.79496108)(36.33922287,1037.78996108)(36.26922882,1037.77996658)
\curveto(36.19922301,1037.77996109)(36.13422308,1037.78996108)(36.07422882,1037.80996658)
\curveto(35.97422324,1037.82996104)(35.89922331,1037.869961)(35.84922882,1037.92996658)
\curveto(35.79922341,1037.98996088)(35.75422346,1038.04996082)(35.71422882,1038.10996658)
\lineto(35.59422882,1038.31996658)
\curveto(35.56422365,1038.39996047)(35.5142237,1038.46496041)(35.44422882,1038.51496658)
\curveto(35.34422387,1038.59496028)(35.24422397,1038.65496022)(35.14422882,1038.69496658)
\curveto(35.05422416,1038.73496014)(34.93922427,1038.7699601)(34.79922882,1038.79996658)
\curveto(34.72922448,1038.81996005)(34.62422459,1038.83496004)(34.48422882,1038.84496658)
\curveto(34.35422486,1038.85496002)(34.25422496,1038.84996002)(34.18422882,1038.82996658)
\lineto(34.07922882,1038.82996658)
\lineto(33.92922882,1038.79996658)
\curveto(33.88922532,1038.79996007)(33.84422537,1038.79496008)(33.79422882,1038.78496658)
\curveto(33.62422559,1038.73496014)(33.48422573,1038.66496021)(33.37422882,1038.57496658)
\curveto(33.27422594,1038.49496038)(33.20422601,1038.3699605)(33.16422882,1038.19996658)
\curveto(33.14422607,1038.12996074)(33.14422607,1038.06496081)(33.16422882,1038.00496658)
\curveto(33.18422603,1037.94496093)(33.20422601,1037.89496098)(33.22422882,1037.85496658)
\curveto(33.29422592,1037.73496114)(33.37422584,1037.63996123)(33.46422882,1037.56996658)
\curveto(33.56422565,1037.49996137)(33.67922553,1037.43996143)(33.80922882,1037.38996658)
\curveto(33.99922521,1037.30996156)(34.20422501,1037.23996163)(34.42422882,1037.17996658)
\lineto(35.11422882,1037.02996658)
\curveto(35.35422386,1036.98996188)(35.58422363,1036.93996193)(35.80422882,1036.87996658)
\curveto(36.03422318,1036.82996204)(36.24922296,1036.76496211)(36.44922882,1036.68496658)
\curveto(36.53922267,1036.64496223)(36.62422259,1036.60996226)(36.70422882,1036.57996658)
\curveto(36.79422242,1036.55996231)(36.87922233,1036.52496235)(36.95922882,1036.47496658)
\curveto(37.14922206,1036.35496252)(37.31922189,1036.22496265)(37.46922882,1036.08496658)
\curveto(37.62922158,1035.94496293)(37.75422146,1035.7699631)(37.84422882,1035.55996658)
\curveto(37.87422134,1035.48996338)(37.89922131,1035.41996345)(37.91922882,1035.34996658)
\curveto(37.93922127,1035.27996359)(37.95922125,1035.20496367)(37.97922882,1035.12496658)
\curveto(37.98922122,1035.06496381)(37.99422122,1034.9699639)(37.99422882,1034.83996658)
\curveto(38.00422121,1034.71996415)(38.00422121,1034.62496425)(37.99422882,1034.55496658)
\lineto(37.99422882,1034.47996658)
\curveto(37.97422124,1034.41996445)(37.95922125,1034.35996451)(37.94922882,1034.29996658)
\curveto(37.94922126,1034.24996462)(37.94422127,1034.19996467)(37.93422882,1034.14996658)
\curveto(37.86422135,1033.84996502)(37.75422146,1033.58496529)(37.60422882,1033.35496658)
\curveto(37.44422177,1033.11496576)(37.24922196,1032.91996595)(37.01922882,1032.76996658)
\curveto(36.78922242,1032.61996625)(36.52922268,1032.48996638)(36.23922882,1032.37996658)
\curveto(36.12922308,1032.32996654)(36.0092232,1032.29496658)(35.87922882,1032.27496658)
\curveto(35.75922345,1032.25496662)(35.63922357,1032.22996664)(35.51922882,1032.19996658)
\curveto(35.42922378,1032.17996669)(35.33422388,1032.1699667)(35.23422882,1032.16996658)
\curveto(35.14422407,1032.15996671)(35.05422416,1032.14496673)(34.96422882,1032.12496658)
\lineto(34.69422882,1032.12496658)
\curveto(34.63422458,1032.10496677)(34.52922468,1032.09496678)(34.37922882,1032.09496658)
\curveto(34.23922497,1032.09496678)(34.13922507,1032.10496677)(34.07922882,1032.12496658)
\curveto(34.04922516,1032.12496675)(34.0142252,1032.12996674)(33.97422882,1032.13996658)
\lineto(33.86922882,1032.13996658)
\curveto(33.74922546,1032.15996671)(33.62922558,1032.1749667)(33.50922882,1032.18496658)
\curveto(33.38922582,1032.19496668)(33.27422594,1032.21496666)(33.16422882,1032.24496658)
\curveto(32.77422644,1032.35496652)(32.42922678,1032.47996639)(32.12922882,1032.61996658)
\curveto(31.82922738,1032.7699661)(31.57422764,1032.98996588)(31.36422882,1033.27996658)
\curveto(31.22422799,1033.4699654)(31.10422811,1033.68996518)(31.00422882,1033.93996658)
\curveto(30.98422823,1033.99996487)(30.96422825,1034.07996479)(30.94422882,1034.17996658)
\curveto(30.92422829,1034.22996464)(30.9092283,1034.29996457)(30.89922882,1034.38996658)
\curveto(30.88922832,1034.47996439)(30.89422832,1034.55496432)(30.91422882,1034.61496658)
\curveto(30.94422827,1034.68496419)(30.99422822,1034.73496414)(31.06422882,1034.76496658)
\curveto(31.1142281,1034.78496409)(31.17422804,1034.79496408)(31.24422882,1034.79496658)
\lineto(31.46922882,1034.79496658)
\lineto(32.17422882,1034.79496658)
\lineto(32.41422882,1034.79496658)
\curveto(32.49422672,1034.79496408)(32.56422665,1034.78496409)(32.62422882,1034.76496658)
\curveto(32.73422648,1034.72496415)(32.80422641,1034.65996421)(32.83422882,1034.56996658)
\curveto(32.87422634,1034.47996439)(32.91922629,1034.38496449)(32.96922882,1034.28496658)
\curveto(32.98922622,1034.23496464)(33.02422619,1034.1699647)(33.07422882,1034.08996658)
\curveto(33.13422608,1034.00996486)(33.18422603,1033.95996491)(33.22422882,1033.93996658)
\curveto(33.34422587,1033.83996503)(33.45922575,1033.75996511)(33.56922882,1033.69996658)
\curveto(33.67922553,1033.64996522)(33.81922539,1033.59996527)(33.98922882,1033.54996658)
\curveto(34.03922517,1033.52996534)(34.08922512,1033.51996535)(34.13922882,1033.51996658)
\curveto(34.18922502,1033.52996534)(34.23922497,1033.52996534)(34.28922882,1033.51996658)
\curveto(34.36922484,1033.49996537)(34.45422476,1033.48996538)(34.54422882,1033.48996658)
\curveto(34.64422457,1033.49996537)(34.72922448,1033.51496536)(34.79922882,1033.53496658)
\curveto(34.84922436,1033.54496533)(34.89422432,1033.54996532)(34.93422882,1033.54996658)
\curveto(34.98422423,1033.54996532)(35.03422418,1033.55996531)(35.08422882,1033.57996658)
\curveto(35.22422399,1033.62996524)(35.34922386,1033.68996518)(35.45922882,1033.75996658)
\curveto(35.57922363,1033.82996504)(35.67422354,1033.91996495)(35.74422882,1034.02996658)
\curveto(35.79422342,1034.10996476)(35.83422338,1034.23496464)(35.86422882,1034.40496658)
\curveto(35.88422333,1034.4749644)(35.88422333,1034.53996433)(35.86422882,1034.59996658)
\curveto(35.84422337,1034.65996421)(35.82422339,1034.70996416)(35.80422882,1034.74996658)
\curveto(35.73422348,1034.88996398)(35.64422357,1034.99496388)(35.53422882,1035.06496658)
\curveto(35.43422378,1035.13496374)(35.3142239,1035.19996367)(35.17422882,1035.25996658)
\curveto(34.98422423,1035.33996353)(34.78422443,1035.40496347)(34.57422882,1035.45496658)
\curveto(34.36422485,1035.50496337)(34.15422506,1035.55996331)(33.94422882,1035.61996658)
\curveto(33.86422535,1035.63996323)(33.77922543,1035.65496322)(33.68922882,1035.66496658)
\curveto(33.6092256,1035.6749632)(33.52922568,1035.68996318)(33.44922882,1035.70996658)
\curveto(33.12922608,1035.79996307)(32.82422639,1035.88496299)(32.53422882,1035.96496658)
\curveto(32.24422697,1036.05496282)(31.97922723,1036.18496269)(31.73922882,1036.35496658)
\curveto(31.45922775,1036.55496232)(31.25422796,1036.82496205)(31.12422882,1037.16496658)
\curveto(31.10422811,1037.23496164)(31.08422813,1037.32996154)(31.06422882,1037.44996658)
\curveto(31.04422817,1037.51996135)(31.02922818,1037.60496127)(31.01922882,1037.70496658)
\curveto(31.0092282,1037.80496107)(31.0142282,1037.89496098)(31.03422882,1037.97496658)
\curveto(31.05422816,1038.02496085)(31.05922815,1038.06496081)(31.04922882,1038.09496658)
\curveto(31.03922817,1038.13496074)(31.04422817,1038.17996069)(31.06422882,1038.22996658)
\curveto(31.08422813,1038.33996053)(31.10422811,1038.43996043)(31.12422882,1038.52996658)
\curveto(31.15422806,1038.62996024)(31.18922802,1038.72496015)(31.22922882,1038.81496658)
\curveto(31.35922785,1039.10495977)(31.53922767,1039.33995953)(31.76922882,1039.51996658)
\curveto(31.99922721,1039.69995917)(32.25922695,1039.84495903)(32.54922882,1039.95496658)
\curveto(32.65922655,1040.00495887)(32.77422644,1040.03995883)(32.89422882,1040.05996658)
\curveto(33.0142262,1040.08995878)(33.13922607,1040.11995875)(33.26922882,1040.14996658)
\curveto(33.32922588,1040.1699587)(33.38922582,1040.17995869)(33.44922882,1040.17996658)
\lineto(33.62922882,1040.20996658)
\curveto(33.7092255,1040.21995865)(33.79422542,1040.22495865)(33.88422882,1040.22496658)
\curveto(33.97422524,1040.22495865)(34.05922515,1040.22995864)(34.13922882,1040.23996658)
}
}
{
\newrgbcolor{curcolor}{0 0 0}
\pscustom[linestyle=none,fillstyle=solid,fillcolor=curcolor]
{
\newpath
\moveto(39.64586945,1040.01496658)
\lineto(40.77086945,1040.01496658)
\curveto(40.88086701,1040.01495886)(40.98086691,1040.00995886)(41.07086945,1039.99996658)
\curveto(41.16086673,1039.98995888)(41.22586667,1039.95495892)(41.26586945,1039.89496658)
\curveto(41.31586658,1039.83495904)(41.34586655,1039.74995912)(41.35586945,1039.63996658)
\curveto(41.36586653,1039.53995933)(41.37086652,1039.43495944)(41.37086945,1039.32496658)
\lineto(41.37086945,1038.27496658)
\lineto(41.37086945,1036.03996658)
\curveto(41.37086652,1035.67996319)(41.38586651,1035.33996353)(41.41586945,1035.01996658)
\curveto(41.44586645,1034.69996417)(41.53586636,1034.43496444)(41.68586945,1034.22496658)
\curveto(41.82586607,1034.01496486)(42.05086584,1033.86496501)(42.36086945,1033.77496658)
\curveto(42.41086548,1033.76496511)(42.45086544,1033.75996511)(42.48086945,1033.75996658)
\curveto(42.52086537,1033.75996511)(42.56586533,1033.75496512)(42.61586945,1033.74496658)
\curveto(42.66586523,1033.73496514)(42.72086517,1033.72996514)(42.78086945,1033.72996658)
\curveto(42.84086505,1033.72996514)(42.88586501,1033.73496514)(42.91586945,1033.74496658)
\curveto(42.96586493,1033.76496511)(43.00586489,1033.7699651)(43.03586945,1033.75996658)
\curveto(43.07586482,1033.74996512)(43.11586478,1033.75496512)(43.15586945,1033.77496658)
\curveto(43.36586453,1033.82496505)(43.53086436,1033.88996498)(43.65086945,1033.96996658)
\curveto(43.83086406,1034.07996479)(43.97086392,1034.21996465)(44.07086945,1034.38996658)
\curveto(44.18086371,1034.5699643)(44.25586364,1034.76496411)(44.29586945,1034.97496658)
\curveto(44.34586355,1035.19496368)(44.37586352,1035.43496344)(44.38586945,1035.69496658)
\curveto(44.3958635,1035.96496291)(44.40086349,1036.24496263)(44.40086945,1036.53496658)
\lineto(44.40086945,1038.34996658)
\lineto(44.40086945,1039.32496658)
\lineto(44.40086945,1039.59496658)
\curveto(44.40086349,1039.69495918)(44.42086347,1039.7749591)(44.46086945,1039.83496658)
\curveto(44.51086338,1039.92495895)(44.58586331,1039.9749589)(44.68586945,1039.98496658)
\curveto(44.78586311,1040.00495887)(44.90586299,1040.01495886)(45.04586945,1040.01496658)
\lineto(45.84086945,1040.01496658)
\lineto(46.12586945,1040.01496658)
\curveto(46.21586168,1040.01495886)(46.2908616,1039.99495888)(46.35086945,1039.95496658)
\curveto(46.43086146,1039.90495897)(46.47586142,1039.82995904)(46.48586945,1039.72996658)
\curveto(46.4958614,1039.62995924)(46.50086139,1039.51495936)(46.50086945,1039.38496658)
\lineto(46.50086945,1038.24496658)
\lineto(46.50086945,1034.02996658)
\lineto(46.50086945,1032.96496658)
\lineto(46.50086945,1032.66496658)
\curveto(46.50086139,1032.56496631)(46.48086141,1032.48996638)(46.44086945,1032.43996658)
\curveto(46.3908615,1032.35996651)(46.31586158,1032.31496656)(46.21586945,1032.30496658)
\curveto(46.11586178,1032.29496658)(46.01086188,1032.28996658)(45.90086945,1032.28996658)
\lineto(45.09086945,1032.28996658)
\curveto(44.98086291,1032.28996658)(44.88086301,1032.29496658)(44.79086945,1032.30496658)
\curveto(44.71086318,1032.31496656)(44.64586325,1032.35496652)(44.59586945,1032.42496658)
\curveto(44.57586332,1032.45496642)(44.55586334,1032.49996637)(44.53586945,1032.55996658)
\curveto(44.52586337,1032.61996625)(44.51086338,1032.67996619)(44.49086945,1032.73996658)
\curveto(44.48086341,1032.79996607)(44.46586343,1032.85496602)(44.44586945,1032.90496658)
\curveto(44.42586347,1032.95496592)(44.3958635,1032.98496589)(44.35586945,1032.99496658)
\curveto(44.33586356,1033.01496586)(44.31086358,1033.01996585)(44.28086945,1033.00996658)
\curveto(44.25086364,1032.99996587)(44.22586367,1032.98996588)(44.20586945,1032.97996658)
\curveto(44.13586376,1032.93996593)(44.07586382,1032.89496598)(44.02586945,1032.84496658)
\curveto(43.97586392,1032.79496608)(43.92086397,1032.74996612)(43.86086945,1032.70996658)
\curveto(43.82086407,1032.67996619)(43.78086411,1032.64496623)(43.74086945,1032.60496658)
\curveto(43.71086418,1032.5749663)(43.67086422,1032.54496633)(43.62086945,1032.51496658)
\curveto(43.3908645,1032.3749665)(43.12086477,1032.26496661)(42.81086945,1032.18496658)
\curveto(42.74086515,1032.16496671)(42.67086522,1032.15496672)(42.60086945,1032.15496658)
\curveto(42.53086536,1032.14496673)(42.45586544,1032.12996674)(42.37586945,1032.10996658)
\curveto(42.33586556,1032.09996677)(42.2908656,1032.09996677)(42.24086945,1032.10996658)
\curveto(42.20086569,1032.10996676)(42.16086573,1032.10496677)(42.12086945,1032.09496658)
\curveto(42.0908658,1032.08496679)(42.02586587,1032.08496679)(41.92586945,1032.09496658)
\curveto(41.83586606,1032.09496678)(41.77586612,1032.09996677)(41.74586945,1032.10996658)
\curveto(41.6958662,1032.10996676)(41.64586625,1032.11496676)(41.59586945,1032.12496658)
\lineto(41.44586945,1032.12496658)
\curveto(41.32586657,1032.15496672)(41.21086668,1032.17996669)(41.10086945,1032.19996658)
\curveto(40.9908669,1032.21996665)(40.88086701,1032.24996662)(40.77086945,1032.28996658)
\curveto(40.72086717,1032.30996656)(40.67586722,1032.32496655)(40.63586945,1032.33496658)
\curveto(40.60586729,1032.35496652)(40.56586733,1032.3749665)(40.51586945,1032.39496658)
\curveto(40.16586773,1032.58496629)(39.88586801,1032.84996602)(39.67586945,1033.18996658)
\curveto(39.54586835,1033.39996547)(39.45086844,1033.64996522)(39.39086945,1033.93996658)
\curveto(39.33086856,1034.23996463)(39.2908686,1034.55496432)(39.27086945,1034.88496658)
\curveto(39.26086863,1035.22496365)(39.25586864,1035.5699633)(39.25586945,1035.91996658)
\curveto(39.26586863,1036.27996259)(39.27086862,1036.63496224)(39.27086945,1036.98496658)
\lineto(39.27086945,1039.02496658)
\curveto(39.27086862,1039.15495972)(39.26586863,1039.30495957)(39.25586945,1039.47496658)
\curveto(39.25586864,1039.65495922)(39.28086861,1039.78495909)(39.33086945,1039.86496658)
\curveto(39.36086853,1039.91495896)(39.42086847,1039.95995891)(39.51086945,1039.99996658)
\curveto(39.57086832,1039.99995887)(39.61586828,1040.00495887)(39.64586945,1040.01496658)
}
}
{
\newrgbcolor{curcolor}{0 0 0}
\pscustom[linestyle=none,fillstyle=solid,fillcolor=curcolor]
{
\newpath
\moveto(55.18211945,1032.88996658)
\curveto(55.2021116,1032.77996609)(55.21211159,1032.6699662)(55.21211945,1032.55996658)
\curveto(55.22211158,1032.44996642)(55.17211163,1032.3749665)(55.06211945,1032.33496658)
\curveto(55.0021118,1032.30496657)(54.93211187,1032.28996658)(54.85211945,1032.28996658)
\lineto(54.61211945,1032.28996658)
\lineto(53.80211945,1032.28996658)
\lineto(53.53211945,1032.28996658)
\curveto(53.45211335,1032.29996657)(53.38711341,1032.32496655)(53.33711945,1032.36496658)
\curveto(53.26711353,1032.40496647)(53.21211359,1032.45996641)(53.17211945,1032.52996658)
\curveto(53.14211366,1032.60996626)(53.0971137,1032.6749662)(53.03711945,1032.72496658)
\curveto(53.01711378,1032.74496613)(52.99211381,1032.75996611)(52.96211945,1032.76996658)
\curveto(52.93211387,1032.78996608)(52.89211391,1032.79496608)(52.84211945,1032.78496658)
\curveto(52.79211401,1032.76496611)(52.74211406,1032.73996613)(52.69211945,1032.70996658)
\curveto(52.65211415,1032.67996619)(52.60711419,1032.65496622)(52.55711945,1032.63496658)
\curveto(52.50711429,1032.59496628)(52.45211435,1032.55996631)(52.39211945,1032.52996658)
\lineto(52.21211945,1032.43996658)
\curveto(52.08211472,1032.37996649)(51.94711485,1032.32996654)(51.80711945,1032.28996658)
\curveto(51.66711513,1032.25996661)(51.52211528,1032.22496665)(51.37211945,1032.18496658)
\curveto(51.3021155,1032.16496671)(51.23211557,1032.15496672)(51.16211945,1032.15496658)
\curveto(51.1021157,1032.14496673)(51.03711576,1032.13496674)(50.96711945,1032.12496658)
\lineto(50.87711945,1032.12496658)
\curveto(50.84711595,1032.11496676)(50.81711598,1032.10996676)(50.78711945,1032.10996658)
\lineto(50.62211945,1032.10996658)
\curveto(50.52211628,1032.08996678)(50.42211638,1032.08996678)(50.32211945,1032.10996658)
\lineto(50.18711945,1032.10996658)
\curveto(50.11711668,1032.12996674)(50.04711675,1032.13996673)(49.97711945,1032.13996658)
\curveto(49.91711688,1032.12996674)(49.85711694,1032.13496674)(49.79711945,1032.15496658)
\curveto(49.6971171,1032.1749667)(49.6021172,1032.19496668)(49.51211945,1032.21496658)
\curveto(49.42211738,1032.22496665)(49.33711746,1032.24996662)(49.25711945,1032.28996658)
\curveto(48.96711783,1032.39996647)(48.71711808,1032.53996633)(48.50711945,1032.70996658)
\curveto(48.30711849,1032.88996598)(48.14711865,1033.12496575)(48.02711945,1033.41496658)
\curveto(47.9971188,1033.48496539)(47.96711883,1033.55996531)(47.93711945,1033.63996658)
\curveto(47.91711888,1033.71996515)(47.8971189,1033.80496507)(47.87711945,1033.89496658)
\curveto(47.85711894,1033.94496493)(47.84711895,1033.99496488)(47.84711945,1034.04496658)
\curveto(47.85711894,1034.09496478)(47.85711894,1034.14496473)(47.84711945,1034.19496658)
\curveto(47.83711896,1034.22496465)(47.82711897,1034.28496459)(47.81711945,1034.37496658)
\curveto(47.81711898,1034.4749644)(47.82211898,1034.54496433)(47.83211945,1034.58496658)
\curveto(47.85211895,1034.68496419)(47.86211894,1034.7699641)(47.86211945,1034.83996658)
\lineto(47.95211945,1035.16996658)
\curveto(47.98211882,1035.28996358)(48.02211878,1035.39496348)(48.07211945,1035.48496658)
\curveto(48.24211856,1035.7749631)(48.43711836,1035.99496288)(48.65711945,1036.14496658)
\curveto(48.87711792,1036.29496258)(49.15711764,1036.42496245)(49.49711945,1036.53496658)
\curveto(49.62711717,1036.58496229)(49.76211704,1036.61996225)(49.90211945,1036.63996658)
\curveto(50.04211676,1036.65996221)(50.18211662,1036.68496219)(50.32211945,1036.71496658)
\curveto(50.4021164,1036.73496214)(50.48711631,1036.74496213)(50.57711945,1036.74496658)
\curveto(50.66711613,1036.75496212)(50.75711604,1036.7699621)(50.84711945,1036.78996658)
\curveto(50.91711588,1036.80996206)(50.98711581,1036.81496206)(51.05711945,1036.80496658)
\curveto(51.12711567,1036.80496207)(51.2021156,1036.81496206)(51.28211945,1036.83496658)
\curveto(51.35211545,1036.85496202)(51.42211538,1036.86496201)(51.49211945,1036.86496658)
\curveto(51.56211524,1036.86496201)(51.63711516,1036.874962)(51.71711945,1036.89496658)
\curveto(51.92711487,1036.94496193)(52.11711468,1036.98496189)(52.28711945,1037.01496658)
\curveto(52.46711433,1037.05496182)(52.62711417,1037.14496173)(52.76711945,1037.28496658)
\curveto(52.85711394,1037.3749615)(52.91711388,1037.4749614)(52.94711945,1037.58496658)
\curveto(52.95711384,1037.61496126)(52.95711384,1037.63996123)(52.94711945,1037.65996658)
\curveto(52.94711385,1037.67996119)(52.95211385,1037.69996117)(52.96211945,1037.71996658)
\curveto(52.97211383,1037.73996113)(52.97711382,1037.7699611)(52.97711945,1037.80996658)
\lineto(52.97711945,1037.89996658)
\lineto(52.94711945,1038.01996658)
\curveto(52.94711385,1038.05996081)(52.94211386,1038.09496078)(52.93211945,1038.12496658)
\curveto(52.83211397,1038.42496045)(52.62211418,1038.62996024)(52.30211945,1038.73996658)
\curveto(52.21211459,1038.7699601)(52.1021147,1038.78996008)(51.97211945,1038.79996658)
\curveto(51.85211495,1038.81996005)(51.72711507,1038.82496005)(51.59711945,1038.81496658)
\curveto(51.46711533,1038.81496006)(51.34211546,1038.80496007)(51.22211945,1038.78496658)
\curveto(51.1021157,1038.76496011)(50.9971158,1038.73996013)(50.90711945,1038.70996658)
\curveto(50.84711595,1038.68996018)(50.78711601,1038.65996021)(50.72711945,1038.61996658)
\curveto(50.67711612,1038.58996028)(50.62711617,1038.55496032)(50.57711945,1038.51496658)
\curveto(50.52711627,1038.4749604)(50.47211633,1038.41996045)(50.41211945,1038.34996658)
\curveto(50.36211644,1038.27996059)(50.32711647,1038.21496066)(50.30711945,1038.15496658)
\curveto(50.25711654,1038.05496082)(50.21211659,1037.95996091)(50.17211945,1037.86996658)
\curveto(50.14211666,1037.77996109)(50.07211673,1037.71996115)(49.96211945,1037.68996658)
\curveto(49.88211692,1037.6699612)(49.797117,1037.65996121)(49.70711945,1037.65996658)
\lineto(49.43711945,1037.65996658)
\lineto(48.86711945,1037.65996658)
\curveto(48.81711798,1037.65996121)(48.76711803,1037.65496122)(48.71711945,1037.64496658)
\curveto(48.66711813,1037.64496123)(48.62211818,1037.64996122)(48.58211945,1037.65996658)
\lineto(48.44711945,1037.65996658)
\curveto(48.42711837,1037.6699612)(48.4021184,1037.6749612)(48.37211945,1037.67496658)
\curveto(48.34211846,1037.6749612)(48.31711848,1037.68496119)(48.29711945,1037.70496658)
\curveto(48.21711858,1037.72496115)(48.16211864,1037.78996108)(48.13211945,1037.89996658)
\curveto(48.12211868,1037.94996092)(48.12211868,1037.99996087)(48.13211945,1038.04996658)
\curveto(48.14211866,1038.09996077)(48.15211865,1038.14496073)(48.16211945,1038.18496658)
\curveto(48.19211861,1038.29496058)(48.22211858,1038.39496048)(48.25211945,1038.48496658)
\curveto(48.29211851,1038.58496029)(48.33711846,1038.6749602)(48.38711945,1038.75496658)
\lineto(48.47711945,1038.90496658)
\lineto(48.56711945,1039.05496658)
\curveto(48.64711815,1039.16495971)(48.74711805,1039.2699596)(48.86711945,1039.36996658)
\curveto(48.88711791,1039.37995949)(48.91711788,1039.40495947)(48.95711945,1039.44496658)
\curveto(49.00711779,1039.48495939)(49.05211775,1039.51995935)(49.09211945,1039.54996658)
\curveto(49.13211767,1039.57995929)(49.17711762,1039.60995926)(49.22711945,1039.63996658)
\curveto(49.3971174,1039.74995912)(49.57711722,1039.83495904)(49.76711945,1039.89496658)
\curveto(49.95711684,1039.96495891)(50.15211665,1040.02995884)(50.35211945,1040.08996658)
\curveto(50.47211633,1040.11995875)(50.5971162,1040.13995873)(50.72711945,1040.14996658)
\curveto(50.85711594,1040.15995871)(50.98711581,1040.17995869)(51.11711945,1040.20996658)
\curveto(51.15711564,1040.21995865)(51.21711558,1040.21995865)(51.29711945,1040.20996658)
\curveto(51.38711541,1040.19995867)(51.44211536,1040.20495867)(51.46211945,1040.22496658)
\curveto(51.87211493,1040.23495864)(52.26211454,1040.21995865)(52.63211945,1040.17996658)
\curveto(53.01211379,1040.13995873)(53.35211345,1040.06495881)(53.65211945,1039.95496658)
\curveto(53.96211284,1039.84495903)(54.22711257,1039.69495918)(54.44711945,1039.50496658)
\curveto(54.66711213,1039.32495955)(54.83711196,1039.08995978)(54.95711945,1038.79996658)
\curveto(55.02711177,1038.62996024)(55.06711173,1038.43496044)(55.07711945,1038.21496658)
\curveto(55.08711171,1037.99496088)(55.09211171,1037.7699611)(55.09211945,1037.53996658)
\lineto(55.09211945,1034.19496658)
\lineto(55.09211945,1033.60996658)
\curveto(55.09211171,1033.41996545)(55.11211169,1033.24496563)(55.15211945,1033.08496658)
\curveto(55.16211164,1033.05496582)(55.16711163,1033.01996585)(55.16711945,1032.97996658)
\curveto(55.16711163,1032.94996592)(55.17211163,1032.91996595)(55.18211945,1032.88996658)
\moveto(52.97711945,1035.19996658)
\curveto(52.98711381,1035.24996362)(52.99211381,1035.30496357)(52.99211945,1035.36496658)
\curveto(52.99211381,1035.43496344)(52.98711381,1035.49496338)(52.97711945,1035.54496658)
\curveto(52.95711384,1035.60496327)(52.94711385,1035.65996321)(52.94711945,1035.70996658)
\curveto(52.94711385,1035.75996311)(52.92711387,1035.79996307)(52.88711945,1035.82996658)
\curveto(52.83711396,1035.869963)(52.76211404,1035.88996298)(52.66211945,1035.88996658)
\curveto(52.62211418,1035.87996299)(52.58711421,1035.869963)(52.55711945,1035.85996658)
\curveto(52.52711427,1035.85996301)(52.49211431,1035.85496302)(52.45211945,1035.84496658)
\curveto(52.38211442,1035.82496305)(52.30711449,1035.80996306)(52.22711945,1035.79996658)
\curveto(52.14711465,1035.78996308)(52.06711473,1035.7749631)(51.98711945,1035.75496658)
\curveto(51.95711484,1035.74496313)(51.91211489,1035.73996313)(51.85211945,1035.73996658)
\curveto(51.72211508,1035.70996316)(51.59211521,1035.68996318)(51.46211945,1035.67996658)
\curveto(51.33211547,1035.6699632)(51.20711559,1035.64496323)(51.08711945,1035.60496658)
\curveto(51.00711579,1035.58496329)(50.93211587,1035.56496331)(50.86211945,1035.54496658)
\curveto(50.79211601,1035.53496334)(50.72211608,1035.51496336)(50.65211945,1035.48496658)
\curveto(50.44211636,1035.39496348)(50.26211654,1035.25996361)(50.11211945,1035.07996658)
\curveto(49.97211683,1034.89996397)(49.92211688,1034.64996422)(49.96211945,1034.32996658)
\curveto(49.98211682,1034.15996471)(50.03711676,1034.01996485)(50.12711945,1033.90996658)
\curveto(50.1971166,1033.79996507)(50.3021165,1033.70996516)(50.44211945,1033.63996658)
\curveto(50.58211622,1033.57996529)(50.73211607,1033.53496534)(50.89211945,1033.50496658)
\curveto(51.06211574,1033.4749654)(51.23711556,1033.46496541)(51.41711945,1033.47496658)
\curveto(51.60711519,1033.49496538)(51.78211502,1033.52996534)(51.94211945,1033.57996658)
\curveto(52.2021146,1033.65996521)(52.40711439,1033.78496509)(52.55711945,1033.95496658)
\curveto(52.70711409,1034.13496474)(52.82211398,1034.35496452)(52.90211945,1034.61496658)
\curveto(52.92211388,1034.68496419)(52.93211387,1034.75496412)(52.93211945,1034.82496658)
\curveto(52.94211386,1034.90496397)(52.95711384,1034.98496389)(52.97711945,1035.06496658)
\lineto(52.97711945,1035.19996658)
}
}
{
\newrgbcolor{curcolor}{0 0 0}
\pscustom[linestyle=none,fillstyle=solid,fillcolor=curcolor]
{
\newpath
\moveto(61.1704007,1040.22496658)
\curveto(61.28039538,1040.22495865)(61.37539529,1040.21495866)(61.4554007,1040.19496658)
\curveto(61.54539512,1040.1749587)(61.61539505,1040.12995874)(61.6654007,1040.05996658)
\curveto(61.72539494,1039.97995889)(61.75539491,1039.83995903)(61.7554007,1039.63996658)
\lineto(61.7554007,1039.12996658)
\lineto(61.7554007,1038.75496658)
\curveto(61.7653949,1038.61496026)(61.75039491,1038.50496037)(61.7104007,1038.42496658)
\curveto(61.67039499,1038.35496052)(61.61039505,1038.30996056)(61.5304007,1038.28996658)
\curveto(61.4603952,1038.2699606)(61.37539529,1038.25996061)(61.2754007,1038.25996658)
\curveto(61.18539548,1038.25996061)(61.08539558,1038.26496061)(60.9754007,1038.27496658)
\curveto(60.87539579,1038.28496059)(60.78039588,1038.27996059)(60.6904007,1038.25996658)
\curveto(60.62039604,1038.23996063)(60.55039611,1038.22496065)(60.4804007,1038.21496658)
\curveto(60.41039625,1038.21496066)(60.34539632,1038.20496067)(60.2854007,1038.18496658)
\curveto(60.12539654,1038.13496074)(59.9653967,1038.05996081)(59.8054007,1037.95996658)
\curveto(59.64539702,1037.869961)(59.52039714,1037.76496111)(59.4304007,1037.64496658)
\curveto(59.38039728,1037.56496131)(59.32539734,1037.47996139)(59.2654007,1037.38996658)
\curveto(59.21539745,1037.30996156)(59.1653975,1037.22496165)(59.1154007,1037.13496658)
\curveto(59.08539758,1037.05496182)(59.05539761,1036.9699619)(59.0254007,1036.87996658)
\lineto(58.9654007,1036.63996658)
\curveto(58.94539772,1036.5699623)(58.93539773,1036.49496238)(58.9354007,1036.41496658)
\curveto(58.93539773,1036.34496253)(58.92539774,1036.2749626)(58.9054007,1036.20496658)
\curveto(58.89539777,1036.16496271)(58.89039777,1036.12496275)(58.8904007,1036.08496658)
\curveto(58.90039776,1036.05496282)(58.90039776,1036.02496285)(58.8904007,1035.99496658)
\lineto(58.8904007,1035.75496658)
\curveto(58.87039779,1035.68496319)(58.8653978,1035.60496327)(58.8754007,1035.51496658)
\curveto(58.88539778,1035.43496344)(58.89039777,1035.35496352)(58.8904007,1035.27496658)
\lineto(58.8904007,1034.31496658)
\lineto(58.8904007,1033.03996658)
\curveto(58.89039777,1032.90996596)(58.88539778,1032.78996608)(58.8754007,1032.67996658)
\curveto(58.8653978,1032.5699663)(58.83539783,1032.47996639)(58.7854007,1032.40996658)
\curveto(58.7653979,1032.37996649)(58.73039793,1032.35496652)(58.6804007,1032.33496658)
\curveto(58.64039802,1032.32496655)(58.59539807,1032.31496656)(58.5454007,1032.30496658)
\lineto(58.4704007,1032.30496658)
\curveto(58.42039824,1032.29496658)(58.3653983,1032.28996658)(58.3054007,1032.28996658)
\lineto(58.1404007,1032.28996658)
\lineto(57.4954007,1032.28996658)
\curveto(57.43539923,1032.29996657)(57.37039929,1032.30496657)(57.3004007,1032.30496658)
\lineto(57.1054007,1032.30496658)
\curveto(57.05539961,1032.32496655)(57.00539966,1032.33996653)(56.9554007,1032.34996658)
\curveto(56.90539976,1032.3699665)(56.87039979,1032.40496647)(56.8504007,1032.45496658)
\curveto(56.81039985,1032.50496637)(56.78539988,1032.5749663)(56.7754007,1032.66496658)
\lineto(56.7754007,1032.96496658)
\lineto(56.7754007,1033.98496658)
\lineto(56.7754007,1038.21496658)
\lineto(56.7754007,1039.32496658)
\lineto(56.7754007,1039.60996658)
\curveto(56.77539989,1039.70995916)(56.79539987,1039.78995908)(56.8354007,1039.84996658)
\curveto(56.88539978,1039.92995894)(56.9603997,1039.97995889)(57.0604007,1039.99996658)
\curveto(57.1603995,1040.01995885)(57.28039938,1040.02995884)(57.4204007,1040.02996658)
\lineto(58.1854007,1040.02996658)
\curveto(58.30539836,1040.02995884)(58.41039825,1040.01995885)(58.5004007,1039.99996658)
\curveto(58.59039807,1039.98995888)(58.660398,1039.94495893)(58.7104007,1039.86496658)
\curveto(58.74039792,1039.81495906)(58.75539791,1039.74495913)(58.7554007,1039.65496658)
\lineto(58.7854007,1039.38496658)
\curveto(58.79539787,1039.30495957)(58.81039785,1039.22995964)(58.8304007,1039.15996658)
\curveto(58.8603978,1039.08995978)(58.91039775,1039.05495982)(58.9804007,1039.05496658)
\curveto(59.00039766,1039.0749598)(59.02039764,1039.08495979)(59.0404007,1039.08496658)
\curveto(59.0603976,1039.08495979)(59.08039758,1039.09495978)(59.1004007,1039.11496658)
\curveto(59.1603975,1039.16495971)(59.21039745,1039.21995965)(59.2504007,1039.27996658)
\curveto(59.30039736,1039.34995952)(59.3603973,1039.40995946)(59.4304007,1039.45996658)
\curveto(59.47039719,1039.48995938)(59.50539716,1039.51995935)(59.5354007,1039.54996658)
\curveto(59.5653971,1039.58995928)(59.60039706,1039.62495925)(59.6404007,1039.65496658)
\lineto(59.9104007,1039.83496658)
\curveto(60.01039665,1039.89495898)(60.11039655,1039.94995892)(60.2104007,1039.99996658)
\curveto(60.31039635,1040.03995883)(60.41039625,1040.0749588)(60.5104007,1040.10496658)
\lineto(60.8404007,1040.19496658)
\curveto(60.87039579,1040.20495867)(60.92539574,1040.20495867)(61.0054007,1040.19496658)
\curveto(61.09539557,1040.19495868)(61.15039551,1040.20495867)(61.1704007,1040.22496658)
}
}
{
\newrgbcolor{curcolor}{0 0 0}
\pscustom[linestyle=none,fillstyle=solid,fillcolor=curcolor]
{
\newpath
\moveto(64.67547882,1042.87996658)
\curveto(64.74547587,1042.79995607)(64.78047584,1042.67995619)(64.78047882,1042.51996658)
\lineto(64.78047882,1042.05496658)
\lineto(64.78047882,1041.64996658)
\curveto(64.78047584,1041.50995736)(64.74547587,1041.41495746)(64.67547882,1041.36496658)
\curveto(64.615476,1041.31495756)(64.53547608,1041.28495759)(64.43547882,1041.27496658)
\curveto(64.34547627,1041.26495761)(64.24547637,1041.25995761)(64.13547882,1041.25996658)
\lineto(63.29547882,1041.25996658)
\curveto(63.18547743,1041.25995761)(63.08547753,1041.26495761)(62.99547882,1041.27496658)
\curveto(62.9154777,1041.28495759)(62.84547777,1041.31495756)(62.78547882,1041.36496658)
\curveto(62.74547787,1041.39495748)(62.7154779,1041.44995742)(62.69547882,1041.52996658)
\curveto(62.68547793,1041.61995725)(62.67547794,1041.71495716)(62.66547882,1041.81496658)
\lineto(62.66547882,1042.14496658)
\curveto(62.67547794,1042.25495662)(62.68047794,1042.34995652)(62.68047882,1042.42996658)
\lineto(62.68047882,1042.63996658)
\curveto(62.69047793,1042.70995616)(62.71047791,1042.7699561)(62.74047882,1042.81996658)
\curveto(62.76047786,1042.85995601)(62.78547783,1042.88995598)(62.81547882,1042.90996658)
\lineto(62.93547882,1042.96996658)
\curveto(62.95547766,1042.9699559)(62.98047764,1042.9699559)(63.01047882,1042.96996658)
\curveto(63.04047758,1042.97995589)(63.06547755,1042.98495589)(63.08547882,1042.98496658)
\lineto(64.18047882,1042.98496658)
\curveto(64.28047634,1042.98495589)(64.37547624,1042.97995589)(64.46547882,1042.96996658)
\curveto(64.55547606,1042.95995591)(64.62547599,1042.92995594)(64.67547882,1042.87996658)
\moveto(64.78047882,1033.11496658)
\curveto(64.78047584,1032.91496596)(64.77547584,1032.74496613)(64.76547882,1032.60496658)
\curveto(64.75547586,1032.46496641)(64.66547595,1032.3699665)(64.49547882,1032.31996658)
\curveto(64.43547618,1032.29996657)(64.37047625,1032.28996658)(64.30047882,1032.28996658)
\curveto(64.23047639,1032.29996657)(64.15547646,1032.30496657)(64.07547882,1032.30496658)
\lineto(63.23547882,1032.30496658)
\curveto(63.14547747,1032.30496657)(63.05547756,1032.30996656)(62.96547882,1032.31996658)
\curveto(62.88547773,1032.32996654)(62.82547779,1032.35996651)(62.78547882,1032.40996658)
\curveto(62.72547789,1032.47996639)(62.69047793,1032.56496631)(62.68047882,1032.66496658)
\lineto(62.68047882,1033.00996658)
\lineto(62.68047882,1039.33996658)
\lineto(62.68047882,1039.63996658)
\curveto(62.68047794,1039.73995913)(62.70047792,1039.81995905)(62.74047882,1039.87996658)
\curveto(62.80047782,1039.94995892)(62.88547773,1039.99495888)(62.99547882,1040.01496658)
\curveto(63.0154776,1040.02495885)(63.04047758,1040.02495885)(63.07047882,1040.01496658)
\curveto(63.11047751,1040.01495886)(63.14047748,1040.01995885)(63.16047882,1040.02996658)
\lineto(63.91047882,1040.02996658)
\lineto(64.10547882,1040.02996658)
\curveto(64.18547643,1040.03995883)(64.25047637,1040.03995883)(64.30047882,1040.02996658)
\lineto(64.42047882,1040.02996658)
\curveto(64.48047614,1040.00995886)(64.53547608,1039.99495888)(64.58547882,1039.98496658)
\curveto(64.63547598,1039.9749589)(64.67547594,1039.94495893)(64.70547882,1039.89496658)
\curveto(64.74547587,1039.84495903)(64.76547585,1039.7749591)(64.76547882,1039.68496658)
\curveto(64.77547584,1039.59495928)(64.78047584,1039.49995937)(64.78047882,1039.39996658)
\lineto(64.78047882,1033.11496658)
}
}
{
\newrgbcolor{curcolor}{0 0 0}
\pscustom[linestyle=none,fillstyle=solid,fillcolor=curcolor]
{
\newpath
\moveto(74.21266632,1036.47496658)
\curveto(74.23265775,1036.41496246)(74.24265774,1036.32996254)(74.24266632,1036.21996658)
\curveto(74.24265774,1036.10996276)(74.23265775,1036.02496285)(74.21266632,1035.96496658)
\lineto(74.21266632,1035.81496658)
\curveto(74.19265779,1035.73496314)(74.1826578,1035.65496322)(74.18266632,1035.57496658)
\curveto(74.19265779,1035.49496338)(74.1876578,1035.41496346)(74.16766632,1035.33496658)
\curveto(74.14765784,1035.26496361)(74.13265785,1035.19996367)(74.12266632,1035.13996658)
\curveto(74.11265787,1035.07996379)(74.10265788,1035.01496386)(74.09266632,1034.94496658)
\curveto(74.05265793,1034.83496404)(74.01765797,1034.71996415)(73.98766632,1034.59996658)
\curveto(73.95765803,1034.48996438)(73.91765807,1034.38496449)(73.86766632,1034.28496658)
\curveto(73.65765833,1033.80496507)(73.3826586,1033.41496546)(73.04266632,1033.11496658)
\curveto(72.70265928,1032.81496606)(72.29265969,1032.56496631)(71.81266632,1032.36496658)
\curveto(71.69266029,1032.31496656)(71.56766042,1032.27996659)(71.43766632,1032.25996658)
\curveto(71.31766067,1032.22996664)(71.19266079,1032.19996667)(71.06266632,1032.16996658)
\curveto(71.01266097,1032.14996672)(70.95766103,1032.13996673)(70.89766632,1032.13996658)
\curveto(70.83766115,1032.13996673)(70.7826612,1032.13496674)(70.73266632,1032.12496658)
\lineto(70.62766632,1032.12496658)
\curveto(70.59766139,1032.11496676)(70.56766142,1032.10996676)(70.53766632,1032.10996658)
\curveto(70.4876615,1032.09996677)(70.40766158,1032.09496678)(70.29766632,1032.09496658)
\curveto(70.1876618,1032.08496679)(70.10266188,1032.08996678)(70.04266632,1032.10996658)
\lineto(69.89266632,1032.10996658)
\curveto(69.84266214,1032.11996675)(69.7876622,1032.12496675)(69.72766632,1032.12496658)
\curveto(69.67766231,1032.11496676)(69.62766236,1032.11996675)(69.57766632,1032.13996658)
\curveto(69.53766245,1032.14996672)(69.49766249,1032.15496672)(69.45766632,1032.15496658)
\curveto(69.42766256,1032.15496672)(69.3876626,1032.15996671)(69.33766632,1032.16996658)
\curveto(69.23766275,1032.19996667)(69.13766285,1032.22496665)(69.03766632,1032.24496658)
\curveto(68.93766305,1032.26496661)(68.84266314,1032.29496658)(68.75266632,1032.33496658)
\curveto(68.63266335,1032.3749665)(68.51766347,1032.41496646)(68.40766632,1032.45496658)
\curveto(68.30766368,1032.49496638)(68.20266378,1032.54496633)(68.09266632,1032.60496658)
\curveto(67.74266424,1032.81496606)(67.44266454,1033.05996581)(67.19266632,1033.33996658)
\curveto(66.94266504,1033.61996525)(66.73266525,1033.95496492)(66.56266632,1034.34496658)
\curveto(66.51266547,1034.43496444)(66.47266551,1034.52996434)(66.44266632,1034.62996658)
\curveto(66.42266556,1034.72996414)(66.39766559,1034.83496404)(66.36766632,1034.94496658)
\curveto(66.34766564,1034.99496388)(66.33766565,1035.03996383)(66.33766632,1035.07996658)
\curveto(66.33766565,1035.11996375)(66.32766566,1035.16496371)(66.30766632,1035.21496658)
\curveto(66.2876657,1035.29496358)(66.27766571,1035.3749635)(66.27766632,1035.45496658)
\curveto(66.27766571,1035.54496333)(66.26766572,1035.62996324)(66.24766632,1035.70996658)
\curveto(66.23766575,1035.75996311)(66.23266575,1035.80496307)(66.23266632,1035.84496658)
\lineto(66.23266632,1035.97996658)
\curveto(66.21266577,1036.03996283)(66.20266578,1036.12496275)(66.20266632,1036.23496658)
\curveto(66.21266577,1036.34496253)(66.22766576,1036.42996244)(66.24766632,1036.48996658)
\lineto(66.24766632,1036.59496658)
\curveto(66.25766573,1036.64496223)(66.25766573,1036.69496218)(66.24766632,1036.74496658)
\curveto(66.24766574,1036.80496207)(66.25766573,1036.85996201)(66.27766632,1036.90996658)
\curveto(66.2876657,1036.95996191)(66.29266569,1037.00496187)(66.29266632,1037.04496658)
\curveto(66.29266569,1037.09496178)(66.30266568,1037.14496173)(66.32266632,1037.19496658)
\curveto(66.36266562,1037.32496155)(66.39766559,1037.44996142)(66.42766632,1037.56996658)
\curveto(66.45766553,1037.69996117)(66.49766549,1037.82496105)(66.54766632,1037.94496658)
\curveto(66.72766526,1038.35496052)(66.94266504,1038.69496018)(67.19266632,1038.96496658)
\curveto(67.44266454,1039.24495963)(67.74766424,1039.49995937)(68.10766632,1039.72996658)
\curveto(68.20766378,1039.77995909)(68.31266367,1039.82495905)(68.42266632,1039.86496658)
\curveto(68.53266345,1039.90495897)(68.64266334,1039.94995892)(68.75266632,1039.99996658)
\curveto(68.8826631,1040.04995882)(69.01766297,1040.08495879)(69.15766632,1040.10496658)
\curveto(69.29766269,1040.12495875)(69.44266254,1040.15495872)(69.59266632,1040.19496658)
\curveto(69.67266231,1040.20495867)(69.74766224,1040.20995866)(69.81766632,1040.20996658)
\curveto(69.8876621,1040.20995866)(69.95766203,1040.21495866)(70.02766632,1040.22496658)
\curveto(70.60766138,1040.23495864)(71.10766088,1040.1749587)(71.52766632,1040.04496658)
\curveto(71.95766003,1039.91495896)(72.33765965,1039.73495914)(72.66766632,1039.50496658)
\curveto(72.77765921,1039.42495945)(72.8876591,1039.33495954)(72.99766632,1039.23496658)
\curveto(73.11765887,1039.14495973)(73.21765877,1039.04495983)(73.29766632,1038.93496658)
\curveto(73.37765861,1038.83496004)(73.44765854,1038.73496014)(73.50766632,1038.63496658)
\curveto(73.57765841,1038.53496034)(73.64765834,1038.42996044)(73.71766632,1038.31996658)
\curveto(73.7876582,1038.20996066)(73.84265814,1038.08996078)(73.88266632,1037.95996658)
\curveto(73.92265806,1037.83996103)(73.96765802,1037.70996116)(74.01766632,1037.56996658)
\curveto(74.04765794,1037.48996138)(74.07265791,1037.40496147)(74.09266632,1037.31496658)
\lineto(74.15266632,1037.04496658)
\curveto(74.16265782,1037.00496187)(74.16765782,1036.96496191)(74.16766632,1036.92496658)
\curveto(74.16765782,1036.88496199)(74.17265781,1036.84496203)(74.18266632,1036.80496658)
\curveto(74.20265778,1036.75496212)(74.20765778,1036.69996217)(74.19766632,1036.63996658)
\curveto(74.1876578,1036.57996229)(74.19265779,1036.52496235)(74.21266632,1036.47496658)
\moveto(72.11266632,1035.93496658)
\curveto(72.12265986,1035.98496289)(72.12765986,1036.05496282)(72.12766632,1036.14496658)
\curveto(72.12765986,1036.24496263)(72.12265986,1036.31996255)(72.11266632,1036.36996658)
\lineto(72.11266632,1036.48996658)
\curveto(72.09265989,1036.53996233)(72.0826599,1036.59496228)(72.08266632,1036.65496658)
\curveto(72.0826599,1036.71496216)(72.07765991,1036.7699621)(72.06766632,1036.81996658)
\curveto(72.06765992,1036.85996201)(72.06265992,1036.88996198)(72.05266632,1036.90996658)
\lineto(71.99266632,1037.14996658)
\curveto(71.98266,1037.23996163)(71.96266002,1037.32496155)(71.93266632,1037.40496658)
\curveto(71.82266016,1037.66496121)(71.69266029,1037.88496099)(71.54266632,1038.06496658)
\curveto(71.39266059,1038.25496062)(71.19266079,1038.40496047)(70.94266632,1038.51496658)
\curveto(70.8826611,1038.53496034)(70.82266116,1038.54996032)(70.76266632,1038.55996658)
\curveto(70.70266128,1038.57996029)(70.63766135,1038.59996027)(70.56766632,1038.61996658)
\curveto(70.4876615,1038.63996023)(70.40266158,1038.64496023)(70.31266632,1038.63496658)
\lineto(70.04266632,1038.63496658)
\curveto(70.01266197,1038.61496026)(69.97766201,1038.60496027)(69.93766632,1038.60496658)
\curveto(69.89766209,1038.61496026)(69.86266212,1038.61496026)(69.83266632,1038.60496658)
\lineto(69.62266632,1038.54496658)
\curveto(69.56266242,1038.53496034)(69.50766248,1038.51496036)(69.45766632,1038.48496658)
\curveto(69.20766278,1038.3749605)(69.00266298,1038.21496066)(68.84266632,1038.00496658)
\curveto(68.69266329,1037.80496107)(68.57266341,1037.5699613)(68.48266632,1037.29996658)
\curveto(68.45266353,1037.19996167)(68.42766356,1037.09496178)(68.40766632,1036.98496658)
\curveto(68.39766359,1036.874962)(68.3826636,1036.76496211)(68.36266632,1036.65496658)
\curveto(68.35266363,1036.60496227)(68.34766364,1036.55496232)(68.34766632,1036.50496658)
\lineto(68.34766632,1036.35496658)
\curveto(68.32766366,1036.28496259)(68.31766367,1036.17996269)(68.31766632,1036.03996658)
\curveto(68.32766366,1035.89996297)(68.34266364,1035.79496308)(68.36266632,1035.72496658)
\lineto(68.36266632,1035.58996658)
\curveto(68.3826636,1035.50996336)(68.39766359,1035.42996344)(68.40766632,1035.34996658)
\curveto(68.41766357,1035.27996359)(68.43266355,1035.20496367)(68.45266632,1035.12496658)
\curveto(68.55266343,1034.82496405)(68.65766333,1034.57996429)(68.76766632,1034.38996658)
\curveto(68.8876631,1034.20996466)(69.07266291,1034.04496483)(69.32266632,1033.89496658)
\curveto(69.39266259,1033.84496503)(69.46766252,1033.80496507)(69.54766632,1033.77496658)
\curveto(69.63766235,1033.74496513)(69.72766226,1033.71996515)(69.81766632,1033.69996658)
\curveto(69.85766213,1033.68996518)(69.89266209,1033.68496519)(69.92266632,1033.68496658)
\curveto(69.95266203,1033.69496518)(69.987662,1033.69496518)(70.02766632,1033.68496658)
\lineto(70.14766632,1033.65496658)
\curveto(70.19766179,1033.65496522)(70.24266174,1033.65996521)(70.28266632,1033.66996658)
\lineto(70.40266632,1033.66996658)
\curveto(70.4826615,1033.68996518)(70.56266142,1033.70496517)(70.64266632,1033.71496658)
\curveto(70.72266126,1033.72496515)(70.79766119,1033.74496513)(70.86766632,1033.77496658)
\curveto(71.12766086,1033.874965)(71.33766065,1034.00996486)(71.49766632,1034.17996658)
\curveto(71.65766033,1034.34996452)(71.79266019,1034.55996431)(71.90266632,1034.80996658)
\curveto(71.94266004,1034.90996396)(71.97266001,1035.00996386)(71.99266632,1035.10996658)
\curveto(72.01265997,1035.20996366)(72.03765995,1035.31496356)(72.06766632,1035.42496658)
\curveto(72.07765991,1035.46496341)(72.0826599,1035.49996337)(72.08266632,1035.52996658)
\curveto(72.0826599,1035.5699633)(72.0876599,1035.60996326)(72.09766632,1035.64996658)
\lineto(72.09766632,1035.78496658)
\curveto(72.09765989,1035.83496304)(72.10265988,1035.88496299)(72.11266632,1035.93496658)
}
}
{
\newrgbcolor{curcolor}{0 0 0}
\pscustom[linestyle=none,fillstyle=solid,fillcolor=curcolor]
{
\newpath
\moveto(78.5825882,1040.23996658)
\curveto(79.3325837,1040.25995861)(79.98258305,1040.1749587)(80.5325882,1039.98496658)
\curveto(81.09258194,1039.80495907)(81.51758151,1039.48995938)(81.8075882,1039.03996658)
\curveto(81.87758115,1038.92995994)(81.93758109,1038.81496006)(81.9875882,1038.69496658)
\curveto(82.04758098,1038.58496029)(82.09758093,1038.45996041)(82.1375882,1038.31996658)
\curveto(82.15758087,1038.25996061)(82.16758086,1038.19496068)(82.1675882,1038.12496658)
\curveto(82.16758086,1038.05496082)(82.15758087,1037.99496088)(82.1375882,1037.94496658)
\curveto(82.09758093,1037.88496099)(82.04258099,1037.84496103)(81.9725882,1037.82496658)
\curveto(81.92258111,1037.80496107)(81.86258117,1037.79496108)(81.7925882,1037.79496658)
\lineto(81.5825882,1037.79496658)
\lineto(80.9225882,1037.79496658)
\curveto(80.85258218,1037.79496108)(80.78258225,1037.78996108)(80.7125882,1037.77996658)
\curveto(80.64258239,1037.77996109)(80.57758245,1037.78996108)(80.5175882,1037.80996658)
\curveto(80.41758261,1037.82996104)(80.34258269,1037.869961)(80.2925882,1037.92996658)
\curveto(80.24258279,1037.98996088)(80.19758283,1038.04996082)(80.1575882,1038.10996658)
\lineto(80.0375882,1038.31996658)
\curveto(80.00758302,1038.39996047)(79.95758307,1038.46496041)(79.8875882,1038.51496658)
\curveto(79.78758324,1038.59496028)(79.68758334,1038.65496022)(79.5875882,1038.69496658)
\curveto(79.49758353,1038.73496014)(79.38258365,1038.7699601)(79.2425882,1038.79996658)
\curveto(79.17258386,1038.81996005)(79.06758396,1038.83496004)(78.9275882,1038.84496658)
\curveto(78.79758423,1038.85496002)(78.69758433,1038.84996002)(78.6275882,1038.82996658)
\lineto(78.5225882,1038.82996658)
\lineto(78.3725882,1038.79996658)
\curveto(78.3325847,1038.79996007)(78.28758474,1038.79496008)(78.2375882,1038.78496658)
\curveto(78.06758496,1038.73496014)(77.9275851,1038.66496021)(77.8175882,1038.57496658)
\curveto(77.71758531,1038.49496038)(77.64758538,1038.3699605)(77.6075882,1038.19996658)
\curveto(77.58758544,1038.12996074)(77.58758544,1038.06496081)(77.6075882,1038.00496658)
\curveto(77.6275854,1037.94496093)(77.64758538,1037.89496098)(77.6675882,1037.85496658)
\curveto(77.73758529,1037.73496114)(77.81758521,1037.63996123)(77.9075882,1037.56996658)
\curveto(78.00758502,1037.49996137)(78.12258491,1037.43996143)(78.2525882,1037.38996658)
\curveto(78.44258459,1037.30996156)(78.64758438,1037.23996163)(78.8675882,1037.17996658)
\lineto(79.5575882,1037.02996658)
\curveto(79.79758323,1036.98996188)(80.027583,1036.93996193)(80.2475882,1036.87996658)
\curveto(80.47758255,1036.82996204)(80.69258234,1036.76496211)(80.8925882,1036.68496658)
\curveto(80.98258205,1036.64496223)(81.06758196,1036.60996226)(81.1475882,1036.57996658)
\curveto(81.23758179,1036.55996231)(81.32258171,1036.52496235)(81.4025882,1036.47496658)
\curveto(81.59258144,1036.35496252)(81.76258127,1036.22496265)(81.9125882,1036.08496658)
\curveto(82.07258096,1035.94496293)(82.19758083,1035.7699631)(82.2875882,1035.55996658)
\curveto(82.31758071,1035.48996338)(82.34258069,1035.41996345)(82.3625882,1035.34996658)
\curveto(82.38258065,1035.27996359)(82.40258063,1035.20496367)(82.4225882,1035.12496658)
\curveto(82.4325806,1035.06496381)(82.43758059,1034.9699639)(82.4375882,1034.83996658)
\curveto(82.44758058,1034.71996415)(82.44758058,1034.62496425)(82.4375882,1034.55496658)
\lineto(82.4375882,1034.47996658)
\curveto(82.41758061,1034.41996445)(82.40258063,1034.35996451)(82.3925882,1034.29996658)
\curveto(82.39258064,1034.24996462)(82.38758064,1034.19996467)(82.3775882,1034.14996658)
\curveto(82.30758072,1033.84996502)(82.19758083,1033.58496529)(82.0475882,1033.35496658)
\curveto(81.88758114,1033.11496576)(81.69258134,1032.91996595)(81.4625882,1032.76996658)
\curveto(81.2325818,1032.61996625)(80.97258206,1032.48996638)(80.6825882,1032.37996658)
\curveto(80.57258246,1032.32996654)(80.45258258,1032.29496658)(80.3225882,1032.27496658)
\curveto(80.20258283,1032.25496662)(80.08258295,1032.22996664)(79.9625882,1032.19996658)
\curveto(79.87258316,1032.17996669)(79.77758325,1032.1699667)(79.6775882,1032.16996658)
\curveto(79.58758344,1032.15996671)(79.49758353,1032.14496673)(79.4075882,1032.12496658)
\lineto(79.1375882,1032.12496658)
\curveto(79.07758395,1032.10496677)(78.97258406,1032.09496678)(78.8225882,1032.09496658)
\curveto(78.68258435,1032.09496678)(78.58258445,1032.10496677)(78.5225882,1032.12496658)
\curveto(78.49258454,1032.12496675)(78.45758457,1032.12996674)(78.4175882,1032.13996658)
\lineto(78.3125882,1032.13996658)
\curveto(78.19258484,1032.15996671)(78.07258496,1032.1749667)(77.9525882,1032.18496658)
\curveto(77.8325852,1032.19496668)(77.71758531,1032.21496666)(77.6075882,1032.24496658)
\curveto(77.21758581,1032.35496652)(76.87258616,1032.47996639)(76.5725882,1032.61996658)
\curveto(76.27258676,1032.7699661)(76.01758701,1032.98996588)(75.8075882,1033.27996658)
\curveto(75.66758736,1033.4699654)(75.54758748,1033.68996518)(75.4475882,1033.93996658)
\curveto(75.4275876,1033.99996487)(75.40758762,1034.07996479)(75.3875882,1034.17996658)
\curveto(75.36758766,1034.22996464)(75.35258768,1034.29996457)(75.3425882,1034.38996658)
\curveto(75.3325877,1034.47996439)(75.33758769,1034.55496432)(75.3575882,1034.61496658)
\curveto(75.38758764,1034.68496419)(75.43758759,1034.73496414)(75.5075882,1034.76496658)
\curveto(75.55758747,1034.78496409)(75.61758741,1034.79496408)(75.6875882,1034.79496658)
\lineto(75.9125882,1034.79496658)
\lineto(76.6175882,1034.79496658)
\lineto(76.8575882,1034.79496658)
\curveto(76.93758609,1034.79496408)(77.00758602,1034.78496409)(77.0675882,1034.76496658)
\curveto(77.17758585,1034.72496415)(77.24758578,1034.65996421)(77.2775882,1034.56996658)
\curveto(77.31758571,1034.47996439)(77.36258567,1034.38496449)(77.4125882,1034.28496658)
\curveto(77.4325856,1034.23496464)(77.46758556,1034.1699647)(77.5175882,1034.08996658)
\curveto(77.57758545,1034.00996486)(77.6275854,1033.95996491)(77.6675882,1033.93996658)
\curveto(77.78758524,1033.83996503)(77.90258513,1033.75996511)(78.0125882,1033.69996658)
\curveto(78.12258491,1033.64996522)(78.26258477,1033.59996527)(78.4325882,1033.54996658)
\curveto(78.48258455,1033.52996534)(78.5325845,1033.51996535)(78.5825882,1033.51996658)
\curveto(78.6325844,1033.52996534)(78.68258435,1033.52996534)(78.7325882,1033.51996658)
\curveto(78.81258422,1033.49996537)(78.89758413,1033.48996538)(78.9875882,1033.48996658)
\curveto(79.08758394,1033.49996537)(79.17258386,1033.51496536)(79.2425882,1033.53496658)
\curveto(79.29258374,1033.54496533)(79.33758369,1033.54996532)(79.3775882,1033.54996658)
\curveto(79.4275836,1033.54996532)(79.47758355,1033.55996531)(79.5275882,1033.57996658)
\curveto(79.66758336,1033.62996524)(79.79258324,1033.68996518)(79.9025882,1033.75996658)
\curveto(80.02258301,1033.82996504)(80.11758291,1033.91996495)(80.1875882,1034.02996658)
\curveto(80.23758279,1034.10996476)(80.27758275,1034.23496464)(80.3075882,1034.40496658)
\curveto(80.3275827,1034.4749644)(80.3275827,1034.53996433)(80.3075882,1034.59996658)
\curveto(80.28758274,1034.65996421)(80.26758276,1034.70996416)(80.2475882,1034.74996658)
\curveto(80.17758285,1034.88996398)(80.08758294,1034.99496388)(79.9775882,1035.06496658)
\curveto(79.87758315,1035.13496374)(79.75758327,1035.19996367)(79.6175882,1035.25996658)
\curveto(79.4275836,1035.33996353)(79.2275838,1035.40496347)(79.0175882,1035.45496658)
\curveto(78.80758422,1035.50496337)(78.59758443,1035.55996331)(78.3875882,1035.61996658)
\curveto(78.30758472,1035.63996323)(78.22258481,1035.65496322)(78.1325882,1035.66496658)
\curveto(78.05258498,1035.6749632)(77.97258506,1035.68996318)(77.8925882,1035.70996658)
\curveto(77.57258546,1035.79996307)(77.26758576,1035.88496299)(76.9775882,1035.96496658)
\curveto(76.68758634,1036.05496282)(76.42258661,1036.18496269)(76.1825882,1036.35496658)
\curveto(75.90258713,1036.55496232)(75.69758733,1036.82496205)(75.5675882,1037.16496658)
\curveto(75.54758748,1037.23496164)(75.5275875,1037.32996154)(75.5075882,1037.44996658)
\curveto(75.48758754,1037.51996135)(75.47258756,1037.60496127)(75.4625882,1037.70496658)
\curveto(75.45258758,1037.80496107)(75.45758757,1037.89496098)(75.4775882,1037.97496658)
\curveto(75.49758753,1038.02496085)(75.50258753,1038.06496081)(75.4925882,1038.09496658)
\curveto(75.48258755,1038.13496074)(75.48758754,1038.17996069)(75.5075882,1038.22996658)
\curveto(75.5275875,1038.33996053)(75.54758748,1038.43996043)(75.5675882,1038.52996658)
\curveto(75.59758743,1038.62996024)(75.6325874,1038.72496015)(75.6725882,1038.81496658)
\curveto(75.80258723,1039.10495977)(75.98258705,1039.33995953)(76.2125882,1039.51996658)
\curveto(76.44258659,1039.69995917)(76.70258633,1039.84495903)(76.9925882,1039.95496658)
\curveto(77.10258593,1040.00495887)(77.21758581,1040.03995883)(77.3375882,1040.05996658)
\curveto(77.45758557,1040.08995878)(77.58258545,1040.11995875)(77.7125882,1040.14996658)
\curveto(77.77258526,1040.1699587)(77.8325852,1040.17995869)(77.8925882,1040.17996658)
\lineto(78.0725882,1040.20996658)
\curveto(78.15258488,1040.21995865)(78.23758479,1040.22495865)(78.3275882,1040.22496658)
\curveto(78.41758461,1040.22495865)(78.50258453,1040.22995864)(78.5825882,1040.23996658)
}
}
{
\newrgbcolor{curcolor}{0 0 0}
\pscustom[linestyle=none,fillstyle=solid,fillcolor=curcolor]
{
}
}
{
\newrgbcolor{curcolor}{0 0 0}
\pscustom[linestyle=none,fillstyle=solid,fillcolor=curcolor]
{
\newpath
\moveto(92.25438507,1040.22496658)
\curveto(92.36437976,1040.22495865)(92.45937966,1040.21495866)(92.53938507,1040.19496658)
\curveto(92.62937949,1040.1749587)(92.69937942,1040.12995874)(92.74938507,1040.05996658)
\curveto(92.80937931,1039.97995889)(92.83937928,1039.83995903)(92.83938507,1039.63996658)
\lineto(92.83938507,1039.12996658)
\lineto(92.83938507,1038.75496658)
\curveto(92.84937927,1038.61496026)(92.83437929,1038.50496037)(92.79438507,1038.42496658)
\curveto(92.75437937,1038.35496052)(92.69437943,1038.30996056)(92.61438507,1038.28996658)
\curveto(92.54437958,1038.2699606)(92.45937966,1038.25996061)(92.35938507,1038.25996658)
\curveto(92.26937985,1038.25996061)(92.16937995,1038.26496061)(92.05938507,1038.27496658)
\curveto(91.95938016,1038.28496059)(91.86438026,1038.27996059)(91.77438507,1038.25996658)
\curveto(91.70438042,1038.23996063)(91.63438049,1038.22496065)(91.56438507,1038.21496658)
\curveto(91.49438063,1038.21496066)(91.42938069,1038.20496067)(91.36938507,1038.18496658)
\curveto(91.20938091,1038.13496074)(91.04938107,1038.05996081)(90.88938507,1037.95996658)
\curveto(90.72938139,1037.869961)(90.60438152,1037.76496111)(90.51438507,1037.64496658)
\curveto(90.46438166,1037.56496131)(90.40938171,1037.47996139)(90.34938507,1037.38996658)
\curveto(90.29938182,1037.30996156)(90.24938187,1037.22496165)(90.19938507,1037.13496658)
\curveto(90.16938195,1037.05496182)(90.13938198,1036.9699619)(90.10938507,1036.87996658)
\lineto(90.04938507,1036.63996658)
\curveto(90.02938209,1036.5699623)(90.0193821,1036.49496238)(90.01938507,1036.41496658)
\curveto(90.0193821,1036.34496253)(90.00938211,1036.2749626)(89.98938507,1036.20496658)
\curveto(89.97938214,1036.16496271)(89.97438215,1036.12496275)(89.97438507,1036.08496658)
\curveto(89.98438214,1036.05496282)(89.98438214,1036.02496285)(89.97438507,1035.99496658)
\lineto(89.97438507,1035.75496658)
\curveto(89.95438217,1035.68496319)(89.94938217,1035.60496327)(89.95938507,1035.51496658)
\curveto(89.96938215,1035.43496344)(89.97438215,1035.35496352)(89.97438507,1035.27496658)
\lineto(89.97438507,1034.31496658)
\lineto(89.97438507,1033.03996658)
\curveto(89.97438215,1032.90996596)(89.96938215,1032.78996608)(89.95938507,1032.67996658)
\curveto(89.94938217,1032.5699663)(89.9193822,1032.47996639)(89.86938507,1032.40996658)
\curveto(89.84938227,1032.37996649)(89.81438231,1032.35496652)(89.76438507,1032.33496658)
\curveto(89.7243824,1032.32496655)(89.67938244,1032.31496656)(89.62938507,1032.30496658)
\lineto(89.55438507,1032.30496658)
\curveto(89.50438262,1032.29496658)(89.44938267,1032.28996658)(89.38938507,1032.28996658)
\lineto(89.22438507,1032.28996658)
\lineto(88.57938507,1032.28996658)
\curveto(88.5193836,1032.29996657)(88.45438367,1032.30496657)(88.38438507,1032.30496658)
\lineto(88.18938507,1032.30496658)
\curveto(88.13938398,1032.32496655)(88.08938403,1032.33996653)(88.03938507,1032.34996658)
\curveto(87.98938413,1032.3699665)(87.95438417,1032.40496647)(87.93438507,1032.45496658)
\curveto(87.89438423,1032.50496637)(87.86938425,1032.5749663)(87.85938507,1032.66496658)
\lineto(87.85938507,1032.96496658)
\lineto(87.85938507,1033.98496658)
\lineto(87.85938507,1038.21496658)
\lineto(87.85938507,1039.32496658)
\lineto(87.85938507,1039.60996658)
\curveto(87.85938426,1039.70995916)(87.87938424,1039.78995908)(87.91938507,1039.84996658)
\curveto(87.96938415,1039.92995894)(88.04438408,1039.97995889)(88.14438507,1039.99996658)
\curveto(88.24438388,1040.01995885)(88.36438376,1040.02995884)(88.50438507,1040.02996658)
\lineto(89.26938507,1040.02996658)
\curveto(89.38938273,1040.02995884)(89.49438263,1040.01995885)(89.58438507,1039.99996658)
\curveto(89.67438245,1039.98995888)(89.74438238,1039.94495893)(89.79438507,1039.86496658)
\curveto(89.8243823,1039.81495906)(89.83938228,1039.74495913)(89.83938507,1039.65496658)
\lineto(89.86938507,1039.38496658)
\curveto(89.87938224,1039.30495957)(89.89438223,1039.22995964)(89.91438507,1039.15996658)
\curveto(89.94438218,1039.08995978)(89.99438213,1039.05495982)(90.06438507,1039.05496658)
\curveto(90.08438204,1039.0749598)(90.10438202,1039.08495979)(90.12438507,1039.08496658)
\curveto(90.14438198,1039.08495979)(90.16438196,1039.09495978)(90.18438507,1039.11496658)
\curveto(90.24438188,1039.16495971)(90.29438183,1039.21995965)(90.33438507,1039.27996658)
\curveto(90.38438174,1039.34995952)(90.44438168,1039.40995946)(90.51438507,1039.45996658)
\curveto(90.55438157,1039.48995938)(90.58938153,1039.51995935)(90.61938507,1039.54996658)
\curveto(90.64938147,1039.58995928)(90.68438144,1039.62495925)(90.72438507,1039.65496658)
\lineto(90.99438507,1039.83496658)
\curveto(91.09438103,1039.89495898)(91.19438093,1039.94995892)(91.29438507,1039.99996658)
\curveto(91.39438073,1040.03995883)(91.49438063,1040.0749588)(91.59438507,1040.10496658)
\lineto(91.92438507,1040.19496658)
\curveto(91.95438017,1040.20495867)(92.00938011,1040.20495867)(92.08938507,1040.19496658)
\curveto(92.17937994,1040.19495868)(92.23437989,1040.20495867)(92.25438507,1040.22496658)
}
}
{
\newrgbcolor{curcolor}{0 0 0}
\pscustom[linestyle=none,fillstyle=solid,fillcolor=curcolor]
{
\newpath
\moveto(100.76079132,1036.23496658)
\curveto(100.78078316,1036.15496272)(100.78078316,1036.06496281)(100.76079132,1035.96496658)
\curveto(100.7407832,1035.86496301)(100.70578323,1035.79996307)(100.65579132,1035.76996658)
\curveto(100.60578333,1035.72996314)(100.53078341,1035.69996317)(100.43079132,1035.67996658)
\curveto(100.3407836,1035.6699632)(100.2357837,1035.65996321)(100.11579132,1035.64996658)
\lineto(99.77079132,1035.64996658)
\curveto(99.66078428,1035.65996321)(99.56078438,1035.66496321)(99.47079132,1035.66496658)
\lineto(95.81079132,1035.66496658)
\lineto(95.60079132,1035.66496658)
\curveto(95.5407884,1035.66496321)(95.48578845,1035.65496322)(95.43579132,1035.63496658)
\curveto(95.35578858,1035.59496328)(95.30578863,1035.55496332)(95.28579132,1035.51496658)
\curveto(95.26578867,1035.49496338)(95.24578869,1035.45496342)(95.22579132,1035.39496658)
\curveto(95.20578873,1035.34496353)(95.20078874,1035.29496358)(95.21079132,1035.24496658)
\curveto(95.23078871,1035.18496369)(95.2407887,1035.12496375)(95.24079132,1035.06496658)
\curveto(95.25078869,1035.01496386)(95.26578867,1034.95996391)(95.28579132,1034.89996658)
\curveto(95.36578857,1034.65996421)(95.46078848,1034.45996441)(95.57079132,1034.29996658)
\curveto(95.69078825,1034.14996472)(95.85078809,1034.01496486)(96.05079132,1033.89496658)
\curveto(96.13078781,1033.84496503)(96.21078773,1033.80996506)(96.29079132,1033.78996658)
\curveto(96.38078756,1033.77996509)(96.47078747,1033.75996511)(96.56079132,1033.72996658)
\curveto(96.6407873,1033.70996516)(96.75078719,1033.69496518)(96.89079132,1033.68496658)
\curveto(97.03078691,1033.6749652)(97.15078679,1033.67996519)(97.25079132,1033.69996658)
\lineto(97.38579132,1033.69996658)
\curveto(97.48578645,1033.71996515)(97.57578636,1033.73996513)(97.65579132,1033.75996658)
\curveto(97.74578619,1033.78996508)(97.83078611,1033.81996505)(97.91079132,1033.84996658)
\curveto(98.01078593,1033.89996497)(98.12078582,1033.96496491)(98.24079132,1034.04496658)
\curveto(98.37078557,1034.12496475)(98.46578547,1034.20496467)(98.52579132,1034.28496658)
\curveto(98.57578536,1034.35496452)(98.62578531,1034.41996445)(98.67579132,1034.47996658)
\curveto(98.7357852,1034.54996432)(98.80578513,1034.59996427)(98.88579132,1034.62996658)
\curveto(98.98578495,1034.67996419)(99.11078483,1034.69996417)(99.26079132,1034.68996658)
\lineto(99.69579132,1034.68996658)
\lineto(99.87579132,1034.68996658)
\curveto(99.94578399,1034.69996417)(100.00578393,1034.69496418)(100.05579132,1034.67496658)
\lineto(100.20579132,1034.67496658)
\curveto(100.30578363,1034.65496422)(100.37578356,1034.62996424)(100.41579132,1034.59996658)
\curveto(100.45578348,1034.57996429)(100.47578346,1034.53496434)(100.47579132,1034.46496658)
\curveto(100.48578345,1034.39496448)(100.48078346,1034.33496454)(100.46079132,1034.28496658)
\curveto(100.41078353,1034.14496473)(100.35578358,1034.01996485)(100.29579132,1033.90996658)
\curveto(100.2357837,1033.79996507)(100.16578377,1033.68996518)(100.08579132,1033.57996658)
\curveto(99.86578407,1033.24996562)(99.61578432,1032.98496589)(99.33579132,1032.78496658)
\curveto(99.05578488,1032.58496629)(98.70578523,1032.41496646)(98.28579132,1032.27496658)
\curveto(98.17578576,1032.23496664)(98.06578587,1032.20996666)(97.95579132,1032.19996658)
\curveto(97.84578609,1032.18996668)(97.73078621,1032.1699667)(97.61079132,1032.13996658)
\curveto(97.57078637,1032.12996674)(97.52578641,1032.12996674)(97.47579132,1032.13996658)
\curveto(97.4357865,1032.13996673)(97.39578654,1032.13496674)(97.35579132,1032.12496658)
\lineto(97.19079132,1032.12496658)
\curveto(97.1407868,1032.10496677)(97.08078686,1032.09996677)(97.01079132,1032.10996658)
\curveto(96.95078699,1032.10996676)(96.89578704,1032.11496676)(96.84579132,1032.12496658)
\curveto(96.76578717,1032.13496674)(96.69578724,1032.13496674)(96.63579132,1032.12496658)
\curveto(96.57578736,1032.11496676)(96.51078743,1032.11996675)(96.44079132,1032.13996658)
\curveto(96.39078755,1032.15996671)(96.3357876,1032.1699667)(96.27579132,1032.16996658)
\curveto(96.21578772,1032.1699667)(96.16078778,1032.17996669)(96.11079132,1032.19996658)
\curveto(96.00078794,1032.21996665)(95.89078805,1032.24496663)(95.78079132,1032.27496658)
\curveto(95.67078827,1032.29496658)(95.57078837,1032.32996654)(95.48079132,1032.37996658)
\curveto(95.37078857,1032.41996645)(95.26578867,1032.45496642)(95.16579132,1032.48496658)
\curveto(95.07578886,1032.52496635)(94.99078895,1032.5699663)(94.91079132,1032.61996658)
\curveto(94.59078935,1032.81996605)(94.30578963,1033.04996582)(94.05579132,1033.30996658)
\curveto(93.80579013,1033.57996529)(93.60079034,1033.88996498)(93.44079132,1034.23996658)
\curveto(93.39079055,1034.34996452)(93.35079059,1034.45996441)(93.32079132,1034.56996658)
\curveto(93.29079065,1034.68996418)(93.25079069,1034.80996406)(93.20079132,1034.92996658)
\curveto(93.19079075,1034.9699639)(93.18579075,1035.00496387)(93.18579132,1035.03496658)
\curveto(93.18579075,1035.0749638)(93.18079076,1035.11496376)(93.17079132,1035.15496658)
\curveto(93.13079081,1035.2749636)(93.10579083,1035.40496347)(93.09579132,1035.54496658)
\lineto(93.06579132,1035.96496658)
\curveto(93.06579087,1036.01496286)(93.06079088,1036.0699628)(93.05079132,1036.12996658)
\curveto(93.05079089,1036.18996268)(93.05579088,1036.24496263)(93.06579132,1036.29496658)
\lineto(93.06579132,1036.47496658)
\lineto(93.11079132,1036.83496658)
\curveto(93.15079079,1037.00496187)(93.18579075,1037.1699617)(93.21579132,1037.32996658)
\curveto(93.24579069,1037.48996138)(93.29079065,1037.63996123)(93.35079132,1037.77996658)
\curveto(93.78079016,1038.81996005)(94.51078943,1039.55495932)(95.54079132,1039.98496658)
\curveto(95.68078826,1040.04495883)(95.82078812,1040.08495879)(95.96079132,1040.10496658)
\curveto(96.11078783,1040.13495874)(96.26578767,1040.1699587)(96.42579132,1040.20996658)
\curveto(96.50578743,1040.21995865)(96.58078736,1040.22495865)(96.65079132,1040.22496658)
\curveto(96.72078722,1040.22495865)(96.79578714,1040.22995864)(96.87579132,1040.23996658)
\curveto(97.38578655,1040.24995862)(97.82078612,1040.18995868)(98.18079132,1040.05996658)
\curveto(98.55078539,1039.93995893)(98.88078506,1039.77995909)(99.17079132,1039.57996658)
\curveto(99.26078468,1039.51995935)(99.35078459,1039.44995942)(99.44079132,1039.36996658)
\curveto(99.53078441,1039.29995957)(99.61078433,1039.22495965)(99.68079132,1039.14496658)
\curveto(99.71078423,1039.09495978)(99.75078419,1039.05495982)(99.80079132,1039.02496658)
\curveto(99.88078406,1038.91495996)(99.95578398,1038.79996007)(100.02579132,1038.67996658)
\curveto(100.09578384,1038.5699603)(100.17078377,1038.45496042)(100.25079132,1038.33496658)
\curveto(100.30078364,1038.24496063)(100.3407836,1038.14996072)(100.37079132,1038.04996658)
\curveto(100.41078353,1037.95996091)(100.45078349,1037.85996101)(100.49079132,1037.74996658)
\curveto(100.5407834,1037.61996125)(100.58078336,1037.48496139)(100.61079132,1037.34496658)
\curveto(100.6407833,1037.20496167)(100.67578326,1037.06496181)(100.71579132,1036.92496658)
\curveto(100.7357832,1036.84496203)(100.7407832,1036.75496212)(100.73079132,1036.65496658)
\curveto(100.73078321,1036.56496231)(100.7407832,1036.47996239)(100.76079132,1036.39996658)
\lineto(100.76079132,1036.23496658)
\moveto(98.51079132,1037.11996658)
\curveto(98.58078536,1037.21996165)(98.58578535,1037.33996153)(98.52579132,1037.47996658)
\curveto(98.47578546,1037.62996124)(98.4357855,1037.73996113)(98.40579132,1037.80996658)
\curveto(98.26578567,1038.07996079)(98.08078586,1038.28496059)(97.85079132,1038.42496658)
\curveto(97.62078632,1038.5749603)(97.30078664,1038.65496022)(96.89079132,1038.66496658)
\curveto(96.86078708,1038.64496023)(96.82578711,1038.63996023)(96.78579132,1038.64996658)
\curveto(96.74578719,1038.65996021)(96.71078723,1038.65996021)(96.68079132,1038.64996658)
\curveto(96.63078731,1038.62996024)(96.57578736,1038.61496026)(96.51579132,1038.60496658)
\curveto(96.45578748,1038.60496027)(96.40078754,1038.59496028)(96.35079132,1038.57496658)
\curveto(95.91078803,1038.43496044)(95.58578835,1038.15996071)(95.37579132,1037.74996658)
\curveto(95.35578858,1037.70996116)(95.33078861,1037.65496122)(95.30079132,1037.58496658)
\curveto(95.28078866,1037.52496135)(95.26578867,1037.45996141)(95.25579132,1037.38996658)
\curveto(95.24578869,1037.32996154)(95.24578869,1037.2699616)(95.25579132,1037.20996658)
\curveto(95.27578866,1037.14996172)(95.31078863,1037.09996177)(95.36079132,1037.05996658)
\curveto(95.4407885,1037.00996186)(95.55078839,1036.98496189)(95.69079132,1036.98496658)
\lineto(96.09579132,1036.98496658)
\lineto(97.76079132,1036.98496658)
\lineto(98.19579132,1036.98496658)
\curveto(98.35578558,1036.99496188)(98.46078548,1037.03996183)(98.51079132,1037.11996658)
}
}
{
\newrgbcolor{curcolor}{0 0 0}
\pscustom[linestyle=none,fillstyle=solid,fillcolor=curcolor]
{
\newpath
\moveto(109.35907257,1039.93996658)
\curveto(109.42906437,1039.88995898)(109.46406434,1039.80495907)(109.46407257,1039.68496658)
\curveto(109.47406433,1039.5749593)(109.47906432,1039.45995941)(109.47907257,1039.33996658)
\lineto(109.47907257,1032.93496658)
\curveto(109.47906432,1032.85496602)(109.47406433,1032.7749661)(109.46407257,1032.69496658)
\lineto(109.46407257,1032.46996658)
\curveto(109.45406435,1032.38996648)(109.44406436,1032.31996655)(109.43407257,1032.25996658)
\curveto(109.43406437,1032.18996668)(109.42906437,1032.11496676)(109.41907257,1032.03496658)
\curveto(109.37906442,1031.89496698)(109.34406446,1031.76496711)(109.31407257,1031.64496658)
\curveto(109.29406451,1031.51496736)(109.25906454,1031.39496748)(109.20907257,1031.28496658)
\curveto(109.03906476,1030.90496797)(108.81906498,1030.58996828)(108.54907257,1030.33996658)
\curveto(108.28906551,1030.08996878)(107.96906583,1029.88496899)(107.58907257,1029.72496658)
\curveto(107.47906632,1029.6749692)(107.36906643,1029.63496924)(107.25907257,1029.60496658)
\curveto(107.14906665,1029.5749693)(107.03406677,1029.54496933)(106.91407257,1029.51496658)
\curveto(106.804067,1029.48496939)(106.69406711,1029.46496941)(106.58407257,1029.45496658)
\curveto(106.47406733,1029.44496943)(106.36406744,1029.42996944)(106.25407257,1029.40996658)
\lineto(106.13407257,1029.40996658)
\curveto(106.09406771,1029.39996947)(106.04906775,1029.39496948)(105.99907257,1029.39496658)
\curveto(105.95906784,1029.38496949)(105.91406789,1029.38496949)(105.86407257,1029.39496658)
\curveto(105.81406799,1029.39496948)(105.76406804,1029.38996948)(105.71407257,1029.37996658)
\curveto(105.66406814,1029.3699695)(105.5990682,1029.36496951)(105.51907257,1029.36496658)
\curveto(105.43906836,1029.36496951)(105.37406843,1029.3699695)(105.32407257,1029.37996658)
\lineto(105.18907257,1029.37996658)
\curveto(105.14906865,1029.37996949)(105.10906869,1029.38496949)(105.06907257,1029.39496658)
\curveto(104.98906881,1029.41496946)(104.9040689,1029.42496945)(104.81407257,1029.42496658)
\curveto(104.73406907,1029.42496945)(104.65906914,1029.43496944)(104.58907257,1029.45496658)
\curveto(104.56906923,1029.46496941)(104.54406926,1029.4699694)(104.51407257,1029.46996658)
\curveto(104.48406932,1029.4699694)(104.45906934,1029.4749694)(104.43907257,1029.48496658)
\curveto(104.33906946,1029.50496937)(104.23906956,1029.52996934)(104.13907257,1029.55996658)
\curveto(104.04906975,1029.57996929)(103.95906984,1029.60996926)(103.86907257,1029.64996658)
\curveto(103.48907031,1029.80996906)(103.14907065,1030.01496886)(102.84907257,1030.26496658)
\curveto(102.54907125,1030.50496837)(102.32907147,1030.82996804)(102.18907257,1031.23996658)
\curveto(102.16907163,1031.2699676)(102.15907164,1031.29996757)(102.15907257,1031.32996658)
\curveto(102.15907164,1031.35996751)(102.15407165,1031.38496749)(102.14407257,1031.40496658)
\curveto(102.11407169,1031.53496734)(102.12407168,1031.63496724)(102.17407257,1031.70496658)
\curveto(102.23407157,1031.76496711)(102.31407149,1031.80496707)(102.41407257,1031.82496658)
\curveto(102.51407129,1031.84496703)(102.62407118,1031.85496702)(102.74407257,1031.85496658)
\curveto(102.87407093,1031.84496703)(102.99407081,1031.83996703)(103.10407257,1031.83996658)
\lineto(103.61407257,1031.83996658)
\lineto(103.73407257,1031.83996658)
\curveto(103.77407003,1031.82996704)(103.81906998,1031.82496705)(103.86907257,1031.82496658)
\curveto(104.02906977,1031.78496709)(104.12906967,1031.73496714)(104.16907257,1031.67496658)
\curveto(104.20906959,1031.60496727)(104.26906953,1031.51496736)(104.34907257,1031.40496658)
\curveto(104.37906942,1031.36496751)(104.42406938,1031.31496756)(104.48407257,1031.25496658)
\curveto(104.49406931,1031.23496764)(104.5040693,1031.21996765)(104.51407257,1031.20996658)
\curveto(104.52406928,1031.19996767)(104.53406927,1031.18496769)(104.54407257,1031.16496658)
\curveto(104.62406918,1031.10496777)(104.70906909,1031.04996782)(104.79907257,1030.99996658)
\curveto(104.88906891,1030.94996792)(104.98906881,1030.90496797)(105.09907257,1030.86496658)
\curveto(105.16906863,1030.84496803)(105.23906856,1030.83496804)(105.30907257,1030.83496658)
\curveto(105.37906842,1030.82496805)(105.45406835,1030.80996806)(105.53407257,1030.78996658)
\lineto(105.69907257,1030.78996658)
\curveto(105.76906803,1030.7699681)(105.85906794,1030.7699681)(105.96907257,1030.78996658)
\curveto(106.07906772,1030.79996807)(106.16406764,1030.81496806)(106.22407257,1030.83496658)
\curveto(106.27406753,1030.85496802)(106.31406749,1030.86496801)(106.34407257,1030.86496658)
\curveto(106.38406742,1030.86496801)(106.42406738,1030.874968)(106.46407257,1030.89496658)
\curveto(106.67406713,1030.98496789)(106.84906695,1031.10496777)(106.98907257,1031.25496658)
\curveto(107.12906667,1031.40496747)(107.24406656,1031.57996729)(107.33407257,1031.77996658)
\curveto(107.35406645,1031.83996703)(107.36906643,1031.89996697)(107.37907257,1031.95996658)
\curveto(107.38906641,1032.01996685)(107.4040664,1032.08496679)(107.42407257,1032.15496658)
\curveto(107.44406636,1032.24496663)(107.45406635,1032.33996653)(107.45407257,1032.43996658)
\curveto(107.46406634,1032.54996632)(107.46906633,1032.65996621)(107.46907257,1032.76996658)
\lineto(107.46907257,1032.88996658)
\curveto(107.47906632,1032.92996594)(107.47906632,1032.96496591)(107.46907257,1032.99496658)
\curveto(107.44906635,1033.04496583)(107.43906636,1033.08996578)(107.43907257,1033.12996658)
\curveto(107.44906635,1033.1699657)(107.44406636,1033.20996566)(107.42407257,1033.24996658)
\curveto(107.41406639,1033.2699656)(107.3990664,1033.28496559)(107.37907257,1033.29496658)
\lineto(107.33407257,1033.33996658)
\curveto(107.24406656,1033.34996552)(107.16906663,1033.32996554)(107.10907257,1033.27996658)
\curveto(107.05906674,1033.22996564)(107.00906679,1033.18496569)(106.95907257,1033.14496658)
\curveto(106.86906693,1033.0749658)(106.77906702,1033.00996586)(106.68907257,1032.94996658)
\curveto(106.5990672,1032.88996598)(106.4990673,1032.83496604)(106.38907257,1032.78496658)
\curveto(106.27906752,1032.73496614)(106.16906763,1032.69496618)(106.05907257,1032.66496658)
\curveto(105.94906785,1032.63496624)(105.83406797,1032.60496627)(105.71407257,1032.57496658)
\lineto(105.53407257,1032.54496658)
\curveto(105.48406832,1032.54496633)(105.43406837,1032.53996633)(105.38407257,1032.52996658)
\curveto(105.33406847,1032.51996635)(105.25406855,1032.51496636)(105.14407257,1032.51496658)
\curveto(105.03406877,1032.51496636)(104.95406885,1032.51996635)(104.90407257,1032.52996658)
\lineto(104.78407257,1032.52996658)
\curveto(104.75406905,1032.53996633)(104.71906908,1032.54496633)(104.67907257,1032.54496658)
\curveto(104.64906915,1032.54496633)(104.61406919,1032.54996632)(104.57407257,1032.55996658)
\curveto(104.43406937,1032.58996628)(104.2990695,1032.61496626)(104.16907257,1032.63496658)
\curveto(104.03906976,1032.66496621)(103.91906988,1032.70496617)(103.80907257,1032.75496658)
\curveto(103.37907042,1032.92496595)(103.02907077,1033.15996571)(102.75907257,1033.45996658)
\curveto(102.4990713,1033.7699651)(102.27907152,1034.13996473)(102.09907257,1034.56996658)
\curveto(102.04907175,1034.67996419)(102.01407179,1034.79496408)(101.99407257,1034.91496658)
\curveto(101.97407183,1035.03496384)(101.94407186,1035.15496372)(101.90407257,1035.27496658)
\curveto(101.9040719,1035.32496355)(101.8990719,1035.36496351)(101.88907257,1035.39496658)
\curveto(101.86907193,1035.4749634)(101.85907194,1035.55996331)(101.85907257,1035.64996658)
\curveto(101.85907194,1035.74996312)(101.84907195,1035.83996303)(101.82907257,1035.91996658)
\curveto(101.81907198,1035.9699629)(101.81407199,1036.01496286)(101.81407257,1036.05496658)
\lineto(101.81407257,1036.20496658)
\curveto(101.804072,1036.25496262)(101.799072,1036.31496256)(101.79907257,1036.38496658)
\curveto(101.799072,1036.46496241)(101.804072,1036.52996234)(101.81407257,1036.57996658)
\lineto(101.81407257,1036.72996658)
\curveto(101.82407198,1036.7699621)(101.82407198,1036.80996206)(101.81407257,1036.84996658)
\curveto(101.81407199,1036.88996198)(101.82407198,1036.92996194)(101.84407257,1036.96996658)
\curveto(101.86407194,1037.0699618)(101.87907192,1037.16496171)(101.88907257,1037.25496658)
\curveto(101.8990719,1037.35496152)(101.91407189,1037.45496142)(101.93407257,1037.55496658)
\curveto(101.99407181,1037.75496112)(102.05407175,1037.94496093)(102.11407257,1038.12496658)
\curveto(102.18407162,1038.30496057)(102.26907153,1038.4749604)(102.36907257,1038.63496658)
\curveto(102.41907138,1038.73496014)(102.47407133,1038.82496005)(102.53407257,1038.90496658)
\lineto(102.74407257,1039.17496658)
\curveto(102.77407103,1039.22495965)(102.81407099,1039.2749596)(102.86407257,1039.32496658)
\curveto(102.92407088,1039.3749595)(102.97907082,1039.41995945)(103.02907257,1039.45996658)
\lineto(103.11907257,1039.54996658)
\curveto(103.16907063,1039.58995928)(103.21907058,1039.62495925)(103.26907257,1039.65496658)
\curveto(103.31907048,1039.69495918)(103.36907043,1039.72995914)(103.41907257,1039.75996658)
\curveto(103.54907025,1039.83995903)(103.68407012,1039.90995896)(103.82407257,1039.96996658)
\curveto(103.96406984,1040.02995884)(104.11906968,1040.08495879)(104.28907257,1040.13496658)
\curveto(104.36906943,1040.16495871)(104.44906935,1040.17995869)(104.52907257,1040.17996658)
\curveto(104.61906918,1040.18995868)(104.7040691,1040.20495867)(104.78407257,1040.22496658)
\curveto(104.82406898,1040.23495864)(104.87906892,1040.23495864)(104.94907257,1040.22496658)
\curveto(105.01906878,1040.21495866)(105.06406874,1040.21995865)(105.08407257,1040.23996658)
\curveto(105.4040684,1040.24995862)(105.68906811,1040.21995865)(105.93907257,1040.14996658)
\curveto(106.1990676,1040.07995879)(106.42906737,1039.97995889)(106.62907257,1039.84996658)
\curveto(106.65906714,1039.82995904)(106.68906711,1039.80495907)(106.71907257,1039.77496658)
\curveto(106.74906705,1039.75495912)(106.78406702,1039.72995914)(106.82407257,1039.69996658)
\curveto(106.88406692,1039.64995922)(106.93906686,1039.59995927)(106.98907257,1039.54996658)
\curveto(107.03906676,1039.49995937)(107.0990667,1039.45495942)(107.16907257,1039.41496658)
\curveto(107.18906661,1039.40495947)(107.21406659,1039.39495948)(107.24407257,1039.38496658)
\curveto(107.28406652,1039.3749595)(107.31406649,1039.37995949)(107.33407257,1039.39996658)
\curveto(107.38406642,1039.41995945)(107.41406639,1039.45495942)(107.42407257,1039.50496658)
\curveto(107.43406637,1039.55495932)(107.44906635,1039.60495927)(107.46907257,1039.65496658)
\curveto(107.48906631,1039.70495917)(107.5040663,1039.75495912)(107.51407257,1039.80496658)
\curveto(107.53406627,1039.86495901)(107.56406624,1039.91495896)(107.60407257,1039.95496658)
\curveto(107.66406614,1039.99495888)(107.73406607,1040.01495886)(107.81407257,1040.01496658)
\curveto(107.9040659,1040.02495885)(107.99406581,1040.02995884)(108.08407257,1040.02996658)
\lineto(108.84907257,1040.02996658)
\curveto(108.95906484,1040.02995884)(109.05406475,1040.02495885)(109.13407257,1040.01496658)
\curveto(109.22406458,1040.01495886)(109.2990645,1039.98995888)(109.35907257,1039.93996658)
\moveto(107.30407257,1035.30496658)
\curveto(107.34406646,1035.39496348)(107.37906642,1035.50996336)(107.40907257,1035.64996658)
\curveto(107.43906636,1035.78996308)(107.45906634,1035.93496294)(107.46907257,1036.08496658)
\curveto(107.47906632,1036.24496263)(107.47906632,1036.39996247)(107.46907257,1036.54996658)
\curveto(107.46906633,1036.69996217)(107.45406635,1036.83496204)(107.42407257,1036.95496658)
\curveto(107.4040664,1036.99496188)(107.39406641,1037.02496185)(107.39407257,1037.04496658)
\curveto(107.4040664,1037.0749618)(107.4040664,1037.10996176)(107.39407257,1037.14996658)
\lineto(107.33407257,1037.35996658)
\curveto(107.31406649,1037.42996144)(107.28906651,1037.49496138)(107.25907257,1037.55496658)
\curveto(107.11906668,1037.90496097)(106.91906688,1038.1749607)(106.65907257,1038.36496658)
\curveto(106.3990674,1038.55496032)(106.01906778,1038.64996022)(105.51907257,1038.64996658)
\curveto(105.4990683,1038.62996024)(105.46906833,1038.61996025)(105.42907257,1038.61996658)
\curveto(105.3990684,1038.62996024)(105.36906843,1038.62996024)(105.33907257,1038.61996658)
\curveto(105.26906853,1038.59996027)(105.2040686,1038.57996029)(105.14407257,1038.55996658)
\curveto(105.08406872,1038.54996032)(105.02406878,1038.53496034)(104.96407257,1038.51496658)
\curveto(104.7040691,1038.40496047)(104.5040693,1038.23996063)(104.36407257,1038.01996658)
\curveto(104.22406958,1037.79996107)(104.10906969,1037.55496132)(104.01907257,1037.28496658)
\curveto(103.9990698,1037.23496164)(103.98906981,1037.19496168)(103.98907257,1037.16496658)
\curveto(103.98906981,1037.13496174)(103.98406982,1037.09496178)(103.97407257,1037.04496658)
\curveto(103.94406986,1036.93496194)(103.92406988,1036.7749621)(103.91407257,1036.56496658)
\curveto(103.9040699,1036.35496252)(103.91406989,1036.18496269)(103.94407257,1036.05496658)
\lineto(103.94407257,1035.90496658)
\curveto(103.96406984,1035.82496305)(103.97906982,1035.74496313)(103.98907257,1035.66496658)
\curveto(103.9990698,1035.59496328)(104.01406979,1035.51996335)(104.03407257,1035.43996658)
\curveto(104.12406968,1035.17996369)(104.23406957,1034.94996392)(104.36407257,1034.74996658)
\curveto(104.49406931,1034.55996431)(104.67406913,1034.40496447)(104.90407257,1034.28496658)
\curveto(105.0040688,1034.23496464)(105.14406866,1034.18496469)(105.32407257,1034.13496658)
\curveto(105.39406841,1034.13496474)(105.44906835,1034.12996474)(105.48907257,1034.11996658)
\curveto(105.50906829,1034.11996475)(105.53906826,1034.11496476)(105.57907257,1034.10496658)
\curveto(105.61906818,1034.10496477)(105.64906815,1034.10996476)(105.66907257,1034.11996658)
\lineto(105.81907257,1034.11996658)
\curveto(105.90906789,1034.13996473)(105.99406781,1034.15496472)(106.07407257,1034.16496658)
\curveto(106.15406765,1034.1749647)(106.23406757,1034.19996467)(106.31407257,1034.23996658)
\curveto(106.56406724,1034.33996453)(106.76406704,1034.47996439)(106.91407257,1034.65996658)
\curveto(107.07406673,1034.83996403)(107.2040666,1035.05496382)(107.30407257,1035.30496658)
}
}
{
\newrgbcolor{curcolor}{0 0 0}
\pscustom[linestyle=none,fillstyle=solid,fillcolor=curcolor]
{
\newpath
\moveto(113.27899445,1042.87996658)
\curveto(113.3489915,1042.79995607)(113.38399146,1042.67995619)(113.38399445,1042.51996658)
\lineto(113.38399445,1042.05496658)
\lineto(113.38399445,1041.64996658)
\curveto(113.38399146,1041.50995736)(113.3489915,1041.41495746)(113.27899445,1041.36496658)
\curveto(113.21899163,1041.31495756)(113.13899171,1041.28495759)(113.03899445,1041.27496658)
\curveto(112.9489919,1041.26495761)(112.848992,1041.25995761)(112.73899445,1041.25996658)
\lineto(111.89899445,1041.25996658)
\curveto(111.78899306,1041.25995761)(111.68899316,1041.26495761)(111.59899445,1041.27496658)
\curveto(111.51899333,1041.28495759)(111.4489934,1041.31495756)(111.38899445,1041.36496658)
\curveto(111.3489935,1041.39495748)(111.31899353,1041.44995742)(111.29899445,1041.52996658)
\curveto(111.28899356,1041.61995725)(111.27899357,1041.71495716)(111.26899445,1041.81496658)
\lineto(111.26899445,1042.14496658)
\curveto(111.27899357,1042.25495662)(111.28399356,1042.34995652)(111.28399445,1042.42996658)
\lineto(111.28399445,1042.63996658)
\curveto(111.29399355,1042.70995616)(111.31399353,1042.7699561)(111.34399445,1042.81996658)
\curveto(111.36399348,1042.85995601)(111.38899346,1042.88995598)(111.41899445,1042.90996658)
\lineto(111.53899445,1042.96996658)
\curveto(111.55899329,1042.9699559)(111.58399326,1042.9699559)(111.61399445,1042.96996658)
\curveto(111.6439932,1042.97995589)(111.66899318,1042.98495589)(111.68899445,1042.98496658)
\lineto(112.78399445,1042.98496658)
\curveto(112.88399196,1042.98495589)(112.97899187,1042.97995589)(113.06899445,1042.96996658)
\curveto(113.15899169,1042.95995591)(113.22899162,1042.92995594)(113.27899445,1042.87996658)
\moveto(113.38399445,1033.11496658)
\curveto(113.38399146,1032.91496596)(113.37899147,1032.74496613)(113.36899445,1032.60496658)
\curveto(113.35899149,1032.46496641)(113.26899158,1032.3699665)(113.09899445,1032.31996658)
\curveto(113.03899181,1032.29996657)(112.97399187,1032.28996658)(112.90399445,1032.28996658)
\curveto(112.83399201,1032.29996657)(112.75899209,1032.30496657)(112.67899445,1032.30496658)
\lineto(111.83899445,1032.30496658)
\curveto(111.7489931,1032.30496657)(111.65899319,1032.30996656)(111.56899445,1032.31996658)
\curveto(111.48899336,1032.32996654)(111.42899342,1032.35996651)(111.38899445,1032.40996658)
\curveto(111.32899352,1032.47996639)(111.29399355,1032.56496631)(111.28399445,1032.66496658)
\lineto(111.28399445,1033.00996658)
\lineto(111.28399445,1039.33996658)
\lineto(111.28399445,1039.63996658)
\curveto(111.28399356,1039.73995913)(111.30399354,1039.81995905)(111.34399445,1039.87996658)
\curveto(111.40399344,1039.94995892)(111.48899336,1039.99495888)(111.59899445,1040.01496658)
\curveto(111.61899323,1040.02495885)(111.6439932,1040.02495885)(111.67399445,1040.01496658)
\curveto(111.71399313,1040.01495886)(111.7439931,1040.01995885)(111.76399445,1040.02996658)
\lineto(112.51399445,1040.02996658)
\lineto(112.70899445,1040.02996658)
\curveto(112.78899206,1040.03995883)(112.85399199,1040.03995883)(112.90399445,1040.02996658)
\lineto(113.02399445,1040.02996658)
\curveto(113.08399176,1040.00995886)(113.13899171,1039.99495888)(113.18899445,1039.98496658)
\curveto(113.23899161,1039.9749589)(113.27899157,1039.94495893)(113.30899445,1039.89496658)
\curveto(113.3489915,1039.84495903)(113.36899148,1039.7749591)(113.36899445,1039.68496658)
\curveto(113.37899147,1039.59495928)(113.38399146,1039.49995937)(113.38399445,1039.39996658)
\lineto(113.38399445,1033.11496658)
}
}
{
\newrgbcolor{curcolor}{0 0 0}
\pscustom[linestyle=none,fillstyle=solid,fillcolor=curcolor]
{
\newpath
\moveto(118.01618195,1040.23996658)
\curveto(118.76617745,1040.25995861)(119.4161768,1040.1749587)(119.96618195,1039.98496658)
\curveto(120.52617569,1039.80495907)(120.95117526,1039.48995938)(121.24118195,1039.03996658)
\curveto(121.3111749,1038.92995994)(121.37117484,1038.81496006)(121.42118195,1038.69496658)
\curveto(121.48117473,1038.58496029)(121.53117468,1038.45996041)(121.57118195,1038.31996658)
\curveto(121.59117462,1038.25996061)(121.60117461,1038.19496068)(121.60118195,1038.12496658)
\curveto(121.60117461,1038.05496082)(121.59117462,1037.99496088)(121.57118195,1037.94496658)
\curveto(121.53117468,1037.88496099)(121.47617474,1037.84496103)(121.40618195,1037.82496658)
\curveto(121.35617486,1037.80496107)(121.29617492,1037.79496108)(121.22618195,1037.79496658)
\lineto(121.01618195,1037.79496658)
\lineto(120.35618195,1037.79496658)
\curveto(120.28617593,1037.79496108)(120.216176,1037.78996108)(120.14618195,1037.77996658)
\curveto(120.07617614,1037.77996109)(120.0111762,1037.78996108)(119.95118195,1037.80996658)
\curveto(119.85117636,1037.82996104)(119.77617644,1037.869961)(119.72618195,1037.92996658)
\curveto(119.67617654,1037.98996088)(119.63117658,1038.04996082)(119.59118195,1038.10996658)
\lineto(119.47118195,1038.31996658)
\curveto(119.44117677,1038.39996047)(119.39117682,1038.46496041)(119.32118195,1038.51496658)
\curveto(119.22117699,1038.59496028)(119.12117709,1038.65496022)(119.02118195,1038.69496658)
\curveto(118.93117728,1038.73496014)(118.8161774,1038.7699601)(118.67618195,1038.79996658)
\curveto(118.60617761,1038.81996005)(118.50117771,1038.83496004)(118.36118195,1038.84496658)
\curveto(118.23117798,1038.85496002)(118.13117808,1038.84996002)(118.06118195,1038.82996658)
\lineto(117.95618195,1038.82996658)
\lineto(117.80618195,1038.79996658)
\curveto(117.76617845,1038.79996007)(117.72117849,1038.79496008)(117.67118195,1038.78496658)
\curveto(117.50117871,1038.73496014)(117.36117885,1038.66496021)(117.25118195,1038.57496658)
\curveto(117.15117906,1038.49496038)(117.08117913,1038.3699605)(117.04118195,1038.19996658)
\curveto(117.02117919,1038.12996074)(117.02117919,1038.06496081)(117.04118195,1038.00496658)
\curveto(117.06117915,1037.94496093)(117.08117913,1037.89496098)(117.10118195,1037.85496658)
\curveto(117.17117904,1037.73496114)(117.25117896,1037.63996123)(117.34118195,1037.56996658)
\curveto(117.44117877,1037.49996137)(117.55617866,1037.43996143)(117.68618195,1037.38996658)
\curveto(117.87617834,1037.30996156)(118.08117813,1037.23996163)(118.30118195,1037.17996658)
\lineto(118.99118195,1037.02996658)
\curveto(119.23117698,1036.98996188)(119.46117675,1036.93996193)(119.68118195,1036.87996658)
\curveto(119.9111763,1036.82996204)(120.12617609,1036.76496211)(120.32618195,1036.68496658)
\curveto(120.4161758,1036.64496223)(120.50117571,1036.60996226)(120.58118195,1036.57996658)
\curveto(120.67117554,1036.55996231)(120.75617546,1036.52496235)(120.83618195,1036.47496658)
\curveto(121.02617519,1036.35496252)(121.19617502,1036.22496265)(121.34618195,1036.08496658)
\curveto(121.50617471,1035.94496293)(121.63117458,1035.7699631)(121.72118195,1035.55996658)
\curveto(121.75117446,1035.48996338)(121.77617444,1035.41996345)(121.79618195,1035.34996658)
\curveto(121.8161744,1035.27996359)(121.83617438,1035.20496367)(121.85618195,1035.12496658)
\curveto(121.86617435,1035.06496381)(121.87117434,1034.9699639)(121.87118195,1034.83996658)
\curveto(121.88117433,1034.71996415)(121.88117433,1034.62496425)(121.87118195,1034.55496658)
\lineto(121.87118195,1034.47996658)
\curveto(121.85117436,1034.41996445)(121.83617438,1034.35996451)(121.82618195,1034.29996658)
\curveto(121.82617439,1034.24996462)(121.82117439,1034.19996467)(121.81118195,1034.14996658)
\curveto(121.74117447,1033.84996502)(121.63117458,1033.58496529)(121.48118195,1033.35496658)
\curveto(121.32117489,1033.11496576)(121.12617509,1032.91996595)(120.89618195,1032.76996658)
\curveto(120.66617555,1032.61996625)(120.40617581,1032.48996638)(120.11618195,1032.37996658)
\curveto(120.00617621,1032.32996654)(119.88617633,1032.29496658)(119.75618195,1032.27496658)
\curveto(119.63617658,1032.25496662)(119.5161767,1032.22996664)(119.39618195,1032.19996658)
\curveto(119.30617691,1032.17996669)(119.211177,1032.1699667)(119.11118195,1032.16996658)
\curveto(119.02117719,1032.15996671)(118.93117728,1032.14496673)(118.84118195,1032.12496658)
\lineto(118.57118195,1032.12496658)
\curveto(118.5111777,1032.10496677)(118.40617781,1032.09496678)(118.25618195,1032.09496658)
\curveto(118.1161781,1032.09496678)(118.0161782,1032.10496677)(117.95618195,1032.12496658)
\curveto(117.92617829,1032.12496675)(117.89117832,1032.12996674)(117.85118195,1032.13996658)
\lineto(117.74618195,1032.13996658)
\curveto(117.62617859,1032.15996671)(117.50617871,1032.1749667)(117.38618195,1032.18496658)
\curveto(117.26617895,1032.19496668)(117.15117906,1032.21496666)(117.04118195,1032.24496658)
\curveto(116.65117956,1032.35496652)(116.30617991,1032.47996639)(116.00618195,1032.61996658)
\curveto(115.70618051,1032.7699661)(115.45118076,1032.98996588)(115.24118195,1033.27996658)
\curveto(115.10118111,1033.4699654)(114.98118123,1033.68996518)(114.88118195,1033.93996658)
\curveto(114.86118135,1033.99996487)(114.84118137,1034.07996479)(114.82118195,1034.17996658)
\curveto(114.80118141,1034.22996464)(114.78618143,1034.29996457)(114.77618195,1034.38996658)
\curveto(114.76618145,1034.47996439)(114.77118144,1034.55496432)(114.79118195,1034.61496658)
\curveto(114.82118139,1034.68496419)(114.87118134,1034.73496414)(114.94118195,1034.76496658)
\curveto(114.99118122,1034.78496409)(115.05118116,1034.79496408)(115.12118195,1034.79496658)
\lineto(115.34618195,1034.79496658)
\lineto(116.05118195,1034.79496658)
\lineto(116.29118195,1034.79496658)
\curveto(116.37117984,1034.79496408)(116.44117977,1034.78496409)(116.50118195,1034.76496658)
\curveto(116.6111796,1034.72496415)(116.68117953,1034.65996421)(116.71118195,1034.56996658)
\curveto(116.75117946,1034.47996439)(116.79617942,1034.38496449)(116.84618195,1034.28496658)
\curveto(116.86617935,1034.23496464)(116.90117931,1034.1699647)(116.95118195,1034.08996658)
\curveto(117.0111792,1034.00996486)(117.06117915,1033.95996491)(117.10118195,1033.93996658)
\curveto(117.22117899,1033.83996503)(117.33617888,1033.75996511)(117.44618195,1033.69996658)
\curveto(117.55617866,1033.64996522)(117.69617852,1033.59996527)(117.86618195,1033.54996658)
\curveto(117.9161783,1033.52996534)(117.96617825,1033.51996535)(118.01618195,1033.51996658)
\curveto(118.06617815,1033.52996534)(118.1161781,1033.52996534)(118.16618195,1033.51996658)
\curveto(118.24617797,1033.49996537)(118.33117788,1033.48996538)(118.42118195,1033.48996658)
\curveto(118.52117769,1033.49996537)(118.60617761,1033.51496536)(118.67618195,1033.53496658)
\curveto(118.72617749,1033.54496533)(118.77117744,1033.54996532)(118.81118195,1033.54996658)
\curveto(118.86117735,1033.54996532)(118.9111773,1033.55996531)(118.96118195,1033.57996658)
\curveto(119.10117711,1033.62996524)(119.22617699,1033.68996518)(119.33618195,1033.75996658)
\curveto(119.45617676,1033.82996504)(119.55117666,1033.91996495)(119.62118195,1034.02996658)
\curveto(119.67117654,1034.10996476)(119.7111765,1034.23496464)(119.74118195,1034.40496658)
\curveto(119.76117645,1034.4749644)(119.76117645,1034.53996433)(119.74118195,1034.59996658)
\curveto(119.72117649,1034.65996421)(119.70117651,1034.70996416)(119.68118195,1034.74996658)
\curveto(119.6111766,1034.88996398)(119.52117669,1034.99496388)(119.41118195,1035.06496658)
\curveto(119.3111769,1035.13496374)(119.19117702,1035.19996367)(119.05118195,1035.25996658)
\curveto(118.86117735,1035.33996353)(118.66117755,1035.40496347)(118.45118195,1035.45496658)
\curveto(118.24117797,1035.50496337)(118.03117818,1035.55996331)(117.82118195,1035.61996658)
\curveto(117.74117847,1035.63996323)(117.65617856,1035.65496322)(117.56618195,1035.66496658)
\curveto(117.48617873,1035.6749632)(117.40617881,1035.68996318)(117.32618195,1035.70996658)
\curveto(117.00617921,1035.79996307)(116.70117951,1035.88496299)(116.41118195,1035.96496658)
\curveto(116.12118009,1036.05496282)(115.85618036,1036.18496269)(115.61618195,1036.35496658)
\curveto(115.33618088,1036.55496232)(115.13118108,1036.82496205)(115.00118195,1037.16496658)
\curveto(114.98118123,1037.23496164)(114.96118125,1037.32996154)(114.94118195,1037.44996658)
\curveto(114.92118129,1037.51996135)(114.90618131,1037.60496127)(114.89618195,1037.70496658)
\curveto(114.88618133,1037.80496107)(114.89118132,1037.89496098)(114.91118195,1037.97496658)
\curveto(114.93118128,1038.02496085)(114.93618128,1038.06496081)(114.92618195,1038.09496658)
\curveto(114.9161813,1038.13496074)(114.92118129,1038.17996069)(114.94118195,1038.22996658)
\curveto(114.96118125,1038.33996053)(114.98118123,1038.43996043)(115.00118195,1038.52996658)
\curveto(115.03118118,1038.62996024)(115.06618115,1038.72496015)(115.10618195,1038.81496658)
\curveto(115.23618098,1039.10495977)(115.4161808,1039.33995953)(115.64618195,1039.51996658)
\curveto(115.87618034,1039.69995917)(116.13618008,1039.84495903)(116.42618195,1039.95496658)
\curveto(116.53617968,1040.00495887)(116.65117956,1040.03995883)(116.77118195,1040.05996658)
\curveto(116.89117932,1040.08995878)(117.0161792,1040.11995875)(117.14618195,1040.14996658)
\curveto(117.20617901,1040.1699587)(117.26617895,1040.17995869)(117.32618195,1040.17996658)
\lineto(117.50618195,1040.20996658)
\curveto(117.58617863,1040.21995865)(117.67117854,1040.22495865)(117.76118195,1040.22496658)
\curveto(117.85117836,1040.22495865)(117.93617828,1040.22995864)(118.01618195,1040.23996658)
}
}
{
\newrgbcolor{curcolor}{0 0 0}
\pscustom[linestyle=none,fillstyle=solid,fillcolor=curcolor]
{
\newpath
\moveto(124.15282257,1042.33996658)
\lineto(125.15782257,1042.33996658)
\curveto(125.30781959,1042.33995653)(125.43781946,1042.32995654)(125.54782257,1042.30996658)
\curveto(125.66781923,1042.29995657)(125.75281914,1042.23995663)(125.80282257,1042.12996658)
\curveto(125.82281907,1042.07995679)(125.83281906,1042.01995685)(125.83282257,1041.94996658)
\lineto(125.83282257,1041.73996658)
\lineto(125.83282257,1041.06496658)
\curveto(125.83281906,1041.01495786)(125.82781907,1040.95495792)(125.81782257,1040.88496658)
\curveto(125.81781908,1040.82495805)(125.82281907,1040.7699581)(125.83282257,1040.71996658)
\lineto(125.83282257,1040.55496658)
\curveto(125.83281906,1040.4749584)(125.83781906,1040.39995847)(125.84782257,1040.32996658)
\curveto(125.85781904,1040.2699586)(125.88281901,1040.21495866)(125.92282257,1040.16496658)
\curveto(125.9928189,1040.0749588)(126.11781878,1040.02495885)(126.29782257,1040.01496658)
\lineto(126.83782257,1040.01496658)
\lineto(127.01782257,1040.01496658)
\curveto(127.07781782,1040.01495886)(127.13281776,1040.00495887)(127.18282257,1039.98496658)
\curveto(127.2928176,1039.93495894)(127.35281754,1039.84495903)(127.36282257,1039.71496658)
\curveto(127.38281751,1039.58495929)(127.3928175,1039.43995943)(127.39282257,1039.27996658)
\lineto(127.39282257,1039.06996658)
\curveto(127.40281749,1038.99995987)(127.3978175,1038.93995993)(127.37782257,1038.88996658)
\curveto(127.32781757,1038.72996014)(127.22281767,1038.64496023)(127.06282257,1038.63496658)
\curveto(126.90281799,1038.62496025)(126.72281817,1038.61996025)(126.52282257,1038.61996658)
\lineto(126.38782257,1038.61996658)
\curveto(126.34781855,1038.62996024)(126.31281858,1038.62996024)(126.28282257,1038.61996658)
\curveto(126.24281865,1038.60996026)(126.20781869,1038.60496027)(126.17782257,1038.60496658)
\curveto(126.14781875,1038.61496026)(126.11781878,1038.60996026)(126.08782257,1038.58996658)
\curveto(126.00781889,1038.5699603)(125.94781895,1038.52496035)(125.90782257,1038.45496658)
\curveto(125.87781902,1038.39496048)(125.85281904,1038.31996055)(125.83282257,1038.22996658)
\curveto(125.82281907,1038.17996069)(125.82281907,1038.12496075)(125.83282257,1038.06496658)
\curveto(125.84281905,1038.00496087)(125.84281905,1037.94996092)(125.83282257,1037.89996658)
\lineto(125.83282257,1036.96996658)
\lineto(125.83282257,1035.21496658)
\curveto(125.83281906,1034.96496391)(125.83781906,1034.74496413)(125.84782257,1034.55496658)
\curveto(125.86781903,1034.3749645)(125.93281896,1034.21496466)(126.04282257,1034.07496658)
\curveto(126.0928188,1034.01496486)(126.15781874,1033.9699649)(126.23782257,1033.93996658)
\lineto(126.50782257,1033.87996658)
\curveto(126.53781836,1033.869965)(126.56781833,1033.86496501)(126.59782257,1033.86496658)
\curveto(126.63781826,1033.874965)(126.66781823,1033.874965)(126.68782257,1033.86496658)
\lineto(126.85282257,1033.86496658)
\curveto(126.96281793,1033.86496501)(127.05781784,1033.85996501)(127.13782257,1033.84996658)
\curveto(127.21781768,1033.83996503)(127.28281761,1033.79996507)(127.33282257,1033.72996658)
\curveto(127.37281752,1033.6699652)(127.3928175,1033.58996528)(127.39282257,1033.48996658)
\lineto(127.39282257,1033.20496658)
\curveto(127.3928175,1032.99496588)(127.38781751,1032.79996607)(127.37782257,1032.61996658)
\curveto(127.37781752,1032.44996642)(127.2978176,1032.33496654)(127.13782257,1032.27496658)
\curveto(127.08781781,1032.25496662)(127.04281785,1032.24996662)(127.00282257,1032.25996658)
\curveto(126.96281793,1032.25996661)(126.91781798,1032.24996662)(126.86782257,1032.22996658)
\lineto(126.71782257,1032.22996658)
\curveto(126.6978182,1032.22996664)(126.66781823,1032.23496664)(126.62782257,1032.24496658)
\curveto(126.58781831,1032.24496663)(126.55281834,1032.23996663)(126.52282257,1032.22996658)
\curveto(126.47281842,1032.21996665)(126.41781848,1032.21996665)(126.35782257,1032.22996658)
\lineto(126.20782257,1032.22996658)
\lineto(126.05782257,1032.22996658)
\curveto(126.00781889,1032.21996665)(125.96281893,1032.21996665)(125.92282257,1032.22996658)
\lineto(125.75782257,1032.22996658)
\curveto(125.70781919,1032.23996663)(125.65281924,1032.24496663)(125.59282257,1032.24496658)
\curveto(125.53281936,1032.24496663)(125.47781942,1032.24996662)(125.42782257,1032.25996658)
\curveto(125.35781954,1032.2699666)(125.2928196,1032.27996659)(125.23282257,1032.28996658)
\lineto(125.05282257,1032.31996658)
\curveto(124.94281995,1032.34996652)(124.83782006,1032.38496649)(124.73782257,1032.42496658)
\curveto(124.63782026,1032.46496641)(124.54282035,1032.50996636)(124.45282257,1032.55996658)
\lineto(124.36282257,1032.61996658)
\curveto(124.33282056,1032.64996622)(124.2978206,1032.67996619)(124.25782257,1032.70996658)
\curveto(124.23782066,1032.72996614)(124.21282068,1032.74996612)(124.18282257,1032.76996658)
\lineto(124.10782257,1032.84496658)
\curveto(123.96782093,1033.03496584)(123.86282103,1033.24496563)(123.79282257,1033.47496658)
\curveto(123.77282112,1033.51496536)(123.76282113,1033.54996532)(123.76282257,1033.57996658)
\curveto(123.77282112,1033.61996525)(123.77282112,1033.66496521)(123.76282257,1033.71496658)
\curveto(123.75282114,1033.73496514)(123.74782115,1033.75996511)(123.74782257,1033.78996658)
\curveto(123.74782115,1033.81996505)(123.74282115,1033.84496503)(123.73282257,1033.86496658)
\lineto(123.73282257,1034.01496658)
\curveto(123.72282117,1034.05496482)(123.71782118,1034.09996477)(123.71782257,1034.14996658)
\curveto(123.72782117,1034.19996467)(123.73282116,1034.24996462)(123.73282257,1034.29996658)
\lineto(123.73282257,1034.86996658)
\lineto(123.73282257,1037.10496658)
\lineto(123.73282257,1037.89996658)
\lineto(123.73282257,1038.10996658)
\curveto(123.74282115,1038.17996069)(123.73782116,1038.24496063)(123.71782257,1038.30496658)
\curveto(123.67782122,1038.44496043)(123.60782129,1038.53496034)(123.50782257,1038.57496658)
\curveto(123.3978215,1038.62496025)(123.25782164,1038.63996023)(123.08782257,1038.61996658)
\curveto(122.91782198,1038.59996027)(122.77282212,1038.61496026)(122.65282257,1038.66496658)
\curveto(122.57282232,1038.69496018)(122.52282237,1038.73996013)(122.50282257,1038.79996658)
\curveto(122.48282241,1038.85996001)(122.46282243,1038.93495994)(122.44282257,1039.02496658)
\lineto(122.44282257,1039.33996658)
\curveto(122.44282245,1039.51995935)(122.45282244,1039.66495921)(122.47282257,1039.77496658)
\curveto(122.4928224,1039.88495899)(122.57782232,1039.95995891)(122.72782257,1039.99996658)
\curveto(122.76782213,1040.01995885)(122.80782209,1040.02495885)(122.84782257,1040.01496658)
\lineto(122.98282257,1040.01496658)
\curveto(123.13282176,1040.01495886)(123.27282162,1040.01995885)(123.40282257,1040.02996658)
\curveto(123.53282136,1040.04995882)(123.62282127,1040.10995876)(123.67282257,1040.20996658)
\curveto(123.70282119,1040.27995859)(123.71782118,1040.35995851)(123.71782257,1040.44996658)
\curveto(123.72782117,1040.53995833)(123.73282116,1040.62995824)(123.73282257,1040.71996658)
\lineto(123.73282257,1041.64996658)
\lineto(123.73282257,1041.90496658)
\curveto(123.73282116,1041.99495688)(123.74282115,1042.0699568)(123.76282257,1042.12996658)
\curveto(123.81282108,1042.22995664)(123.88782101,1042.29495658)(123.98782257,1042.32496658)
\curveto(124.00782089,1042.33495654)(124.03282086,1042.33495654)(124.06282257,1042.32496658)
\curveto(124.10282079,1042.32495655)(124.13282076,1042.32995654)(124.15282257,1042.33996658)
}
}
{
\newrgbcolor{curcolor}{0 0 0}
\pscustom[linestyle=none,fillstyle=solid,fillcolor=curcolor]
{
\newpath
\moveto(132.80126007,1040.22496658)
\curveto(132.91125476,1040.22495865)(133.00625466,1040.21495866)(133.08626007,1040.19496658)
\curveto(133.17625449,1040.1749587)(133.24625442,1040.12995874)(133.29626007,1040.05996658)
\curveto(133.35625431,1039.97995889)(133.38625428,1039.83995903)(133.38626007,1039.63996658)
\lineto(133.38626007,1039.12996658)
\lineto(133.38626007,1038.75496658)
\curveto(133.39625427,1038.61496026)(133.38125429,1038.50496037)(133.34126007,1038.42496658)
\curveto(133.30125437,1038.35496052)(133.24125443,1038.30996056)(133.16126007,1038.28996658)
\curveto(133.09125458,1038.2699606)(133.00625466,1038.25996061)(132.90626007,1038.25996658)
\curveto(132.81625485,1038.25996061)(132.71625495,1038.26496061)(132.60626007,1038.27496658)
\curveto(132.50625516,1038.28496059)(132.41125526,1038.27996059)(132.32126007,1038.25996658)
\curveto(132.25125542,1038.23996063)(132.18125549,1038.22496065)(132.11126007,1038.21496658)
\curveto(132.04125563,1038.21496066)(131.97625569,1038.20496067)(131.91626007,1038.18496658)
\curveto(131.75625591,1038.13496074)(131.59625607,1038.05996081)(131.43626007,1037.95996658)
\curveto(131.27625639,1037.869961)(131.15125652,1037.76496111)(131.06126007,1037.64496658)
\curveto(131.01125666,1037.56496131)(130.95625671,1037.47996139)(130.89626007,1037.38996658)
\curveto(130.84625682,1037.30996156)(130.79625687,1037.22496165)(130.74626007,1037.13496658)
\curveto(130.71625695,1037.05496182)(130.68625698,1036.9699619)(130.65626007,1036.87996658)
\lineto(130.59626007,1036.63996658)
\curveto(130.57625709,1036.5699623)(130.5662571,1036.49496238)(130.56626007,1036.41496658)
\curveto(130.5662571,1036.34496253)(130.55625711,1036.2749626)(130.53626007,1036.20496658)
\curveto(130.52625714,1036.16496271)(130.52125715,1036.12496275)(130.52126007,1036.08496658)
\curveto(130.53125714,1036.05496282)(130.53125714,1036.02496285)(130.52126007,1035.99496658)
\lineto(130.52126007,1035.75496658)
\curveto(130.50125717,1035.68496319)(130.49625717,1035.60496327)(130.50626007,1035.51496658)
\curveto(130.51625715,1035.43496344)(130.52125715,1035.35496352)(130.52126007,1035.27496658)
\lineto(130.52126007,1034.31496658)
\lineto(130.52126007,1033.03996658)
\curveto(130.52125715,1032.90996596)(130.51625715,1032.78996608)(130.50626007,1032.67996658)
\curveto(130.49625717,1032.5699663)(130.4662572,1032.47996639)(130.41626007,1032.40996658)
\curveto(130.39625727,1032.37996649)(130.36125731,1032.35496652)(130.31126007,1032.33496658)
\curveto(130.2712574,1032.32496655)(130.22625744,1032.31496656)(130.17626007,1032.30496658)
\lineto(130.10126007,1032.30496658)
\curveto(130.05125762,1032.29496658)(129.99625767,1032.28996658)(129.93626007,1032.28996658)
\lineto(129.77126007,1032.28996658)
\lineto(129.12626007,1032.28996658)
\curveto(129.0662586,1032.29996657)(129.00125867,1032.30496657)(128.93126007,1032.30496658)
\lineto(128.73626007,1032.30496658)
\curveto(128.68625898,1032.32496655)(128.63625903,1032.33996653)(128.58626007,1032.34996658)
\curveto(128.53625913,1032.3699665)(128.50125917,1032.40496647)(128.48126007,1032.45496658)
\curveto(128.44125923,1032.50496637)(128.41625925,1032.5749663)(128.40626007,1032.66496658)
\lineto(128.40626007,1032.96496658)
\lineto(128.40626007,1033.98496658)
\lineto(128.40626007,1038.21496658)
\lineto(128.40626007,1039.32496658)
\lineto(128.40626007,1039.60996658)
\curveto(128.40625926,1039.70995916)(128.42625924,1039.78995908)(128.46626007,1039.84996658)
\curveto(128.51625915,1039.92995894)(128.59125908,1039.97995889)(128.69126007,1039.99996658)
\curveto(128.79125888,1040.01995885)(128.91125876,1040.02995884)(129.05126007,1040.02996658)
\lineto(129.81626007,1040.02996658)
\curveto(129.93625773,1040.02995884)(130.04125763,1040.01995885)(130.13126007,1039.99996658)
\curveto(130.22125745,1039.98995888)(130.29125738,1039.94495893)(130.34126007,1039.86496658)
\curveto(130.3712573,1039.81495906)(130.38625728,1039.74495913)(130.38626007,1039.65496658)
\lineto(130.41626007,1039.38496658)
\curveto(130.42625724,1039.30495957)(130.44125723,1039.22995964)(130.46126007,1039.15996658)
\curveto(130.49125718,1039.08995978)(130.54125713,1039.05495982)(130.61126007,1039.05496658)
\curveto(130.63125704,1039.0749598)(130.65125702,1039.08495979)(130.67126007,1039.08496658)
\curveto(130.69125698,1039.08495979)(130.71125696,1039.09495978)(130.73126007,1039.11496658)
\curveto(130.79125688,1039.16495971)(130.84125683,1039.21995965)(130.88126007,1039.27996658)
\curveto(130.93125674,1039.34995952)(130.99125668,1039.40995946)(131.06126007,1039.45996658)
\curveto(131.10125657,1039.48995938)(131.13625653,1039.51995935)(131.16626007,1039.54996658)
\curveto(131.19625647,1039.58995928)(131.23125644,1039.62495925)(131.27126007,1039.65496658)
\lineto(131.54126007,1039.83496658)
\curveto(131.64125603,1039.89495898)(131.74125593,1039.94995892)(131.84126007,1039.99996658)
\curveto(131.94125573,1040.03995883)(132.04125563,1040.0749588)(132.14126007,1040.10496658)
\lineto(132.47126007,1040.19496658)
\curveto(132.50125517,1040.20495867)(132.55625511,1040.20495867)(132.63626007,1040.19496658)
\curveto(132.72625494,1040.19495868)(132.78125489,1040.20495867)(132.80126007,1040.22496658)
}
}
{
\newrgbcolor{curcolor}{0 0 0}
\pscustom[linestyle=none,fillstyle=solid,fillcolor=curcolor]
{
\newpath
\moveto(141.2563382,1032.88996658)
\curveto(141.27633035,1032.77996609)(141.28633034,1032.6699662)(141.2863382,1032.55996658)
\curveto(141.29633033,1032.44996642)(141.24633038,1032.3749665)(141.1363382,1032.33496658)
\curveto(141.07633055,1032.30496657)(141.00633062,1032.28996658)(140.9263382,1032.28996658)
\lineto(140.6863382,1032.28996658)
\lineto(139.8763382,1032.28996658)
\lineto(139.6063382,1032.28996658)
\curveto(139.5263321,1032.29996657)(139.46133216,1032.32496655)(139.4113382,1032.36496658)
\curveto(139.34133228,1032.40496647)(139.28633234,1032.45996641)(139.2463382,1032.52996658)
\curveto(139.21633241,1032.60996626)(139.17133245,1032.6749662)(139.1113382,1032.72496658)
\curveto(139.09133253,1032.74496613)(139.06633256,1032.75996611)(139.0363382,1032.76996658)
\curveto(139.00633262,1032.78996608)(138.96633266,1032.79496608)(138.9163382,1032.78496658)
\curveto(138.86633276,1032.76496611)(138.81633281,1032.73996613)(138.7663382,1032.70996658)
\curveto(138.7263329,1032.67996619)(138.68133294,1032.65496622)(138.6313382,1032.63496658)
\curveto(138.58133304,1032.59496628)(138.5263331,1032.55996631)(138.4663382,1032.52996658)
\lineto(138.2863382,1032.43996658)
\curveto(138.15633347,1032.37996649)(138.0213336,1032.32996654)(137.8813382,1032.28996658)
\curveto(137.74133388,1032.25996661)(137.59633403,1032.22496665)(137.4463382,1032.18496658)
\curveto(137.37633425,1032.16496671)(137.30633432,1032.15496672)(137.2363382,1032.15496658)
\curveto(137.17633445,1032.14496673)(137.11133451,1032.13496674)(137.0413382,1032.12496658)
\lineto(136.9513382,1032.12496658)
\curveto(136.9213347,1032.11496676)(136.89133473,1032.10996676)(136.8613382,1032.10996658)
\lineto(136.6963382,1032.10996658)
\curveto(136.59633503,1032.08996678)(136.49633513,1032.08996678)(136.3963382,1032.10996658)
\lineto(136.2613382,1032.10996658)
\curveto(136.19133543,1032.12996674)(136.1213355,1032.13996673)(136.0513382,1032.13996658)
\curveto(135.99133563,1032.12996674)(135.93133569,1032.13496674)(135.8713382,1032.15496658)
\curveto(135.77133585,1032.1749667)(135.67633595,1032.19496668)(135.5863382,1032.21496658)
\curveto(135.49633613,1032.22496665)(135.41133621,1032.24996662)(135.3313382,1032.28996658)
\curveto(135.04133658,1032.39996647)(134.79133683,1032.53996633)(134.5813382,1032.70996658)
\curveto(134.38133724,1032.88996598)(134.2213374,1033.12496575)(134.1013382,1033.41496658)
\curveto(134.07133755,1033.48496539)(134.04133758,1033.55996531)(134.0113382,1033.63996658)
\curveto(133.99133763,1033.71996515)(133.97133765,1033.80496507)(133.9513382,1033.89496658)
\curveto(133.93133769,1033.94496493)(133.9213377,1033.99496488)(133.9213382,1034.04496658)
\curveto(133.93133769,1034.09496478)(133.93133769,1034.14496473)(133.9213382,1034.19496658)
\curveto(133.91133771,1034.22496465)(133.90133772,1034.28496459)(133.8913382,1034.37496658)
\curveto(133.89133773,1034.4749644)(133.89633773,1034.54496433)(133.9063382,1034.58496658)
\curveto(133.9263377,1034.68496419)(133.93633769,1034.7699641)(133.9363382,1034.83996658)
\lineto(134.0263382,1035.16996658)
\curveto(134.05633757,1035.28996358)(134.09633753,1035.39496348)(134.1463382,1035.48496658)
\curveto(134.31633731,1035.7749631)(134.51133711,1035.99496288)(134.7313382,1036.14496658)
\curveto(134.95133667,1036.29496258)(135.23133639,1036.42496245)(135.5713382,1036.53496658)
\curveto(135.70133592,1036.58496229)(135.83633579,1036.61996225)(135.9763382,1036.63996658)
\curveto(136.11633551,1036.65996221)(136.25633537,1036.68496219)(136.3963382,1036.71496658)
\curveto(136.47633515,1036.73496214)(136.56133506,1036.74496213)(136.6513382,1036.74496658)
\curveto(136.74133488,1036.75496212)(136.83133479,1036.7699621)(136.9213382,1036.78996658)
\curveto(136.99133463,1036.80996206)(137.06133456,1036.81496206)(137.1313382,1036.80496658)
\curveto(137.20133442,1036.80496207)(137.27633435,1036.81496206)(137.3563382,1036.83496658)
\curveto(137.4263342,1036.85496202)(137.49633413,1036.86496201)(137.5663382,1036.86496658)
\curveto(137.63633399,1036.86496201)(137.71133391,1036.874962)(137.7913382,1036.89496658)
\curveto(138.00133362,1036.94496193)(138.19133343,1036.98496189)(138.3613382,1037.01496658)
\curveto(138.54133308,1037.05496182)(138.70133292,1037.14496173)(138.8413382,1037.28496658)
\curveto(138.93133269,1037.3749615)(138.99133263,1037.4749614)(139.0213382,1037.58496658)
\curveto(139.03133259,1037.61496126)(139.03133259,1037.63996123)(139.0213382,1037.65996658)
\curveto(139.0213326,1037.67996119)(139.0263326,1037.69996117)(139.0363382,1037.71996658)
\curveto(139.04633258,1037.73996113)(139.05133257,1037.7699611)(139.0513382,1037.80996658)
\lineto(139.0513382,1037.89996658)
\lineto(139.0213382,1038.01996658)
\curveto(139.0213326,1038.05996081)(139.01633261,1038.09496078)(139.0063382,1038.12496658)
\curveto(138.90633272,1038.42496045)(138.69633293,1038.62996024)(138.3763382,1038.73996658)
\curveto(138.28633334,1038.7699601)(138.17633345,1038.78996008)(138.0463382,1038.79996658)
\curveto(137.9263337,1038.81996005)(137.80133382,1038.82496005)(137.6713382,1038.81496658)
\curveto(137.54133408,1038.81496006)(137.41633421,1038.80496007)(137.2963382,1038.78496658)
\curveto(137.17633445,1038.76496011)(137.07133455,1038.73996013)(136.9813382,1038.70996658)
\curveto(136.9213347,1038.68996018)(136.86133476,1038.65996021)(136.8013382,1038.61996658)
\curveto(136.75133487,1038.58996028)(136.70133492,1038.55496032)(136.6513382,1038.51496658)
\curveto(136.60133502,1038.4749604)(136.54633508,1038.41996045)(136.4863382,1038.34996658)
\curveto(136.43633519,1038.27996059)(136.40133522,1038.21496066)(136.3813382,1038.15496658)
\curveto(136.33133529,1038.05496082)(136.28633534,1037.95996091)(136.2463382,1037.86996658)
\curveto(136.21633541,1037.77996109)(136.14633548,1037.71996115)(136.0363382,1037.68996658)
\curveto(135.95633567,1037.6699612)(135.87133575,1037.65996121)(135.7813382,1037.65996658)
\lineto(135.5113382,1037.65996658)
\lineto(134.9413382,1037.65996658)
\curveto(134.89133673,1037.65996121)(134.84133678,1037.65496122)(134.7913382,1037.64496658)
\curveto(134.74133688,1037.64496123)(134.69633693,1037.64996122)(134.6563382,1037.65996658)
\lineto(134.5213382,1037.65996658)
\curveto(134.50133712,1037.6699612)(134.47633715,1037.6749612)(134.4463382,1037.67496658)
\curveto(134.41633721,1037.6749612)(134.39133723,1037.68496119)(134.3713382,1037.70496658)
\curveto(134.29133733,1037.72496115)(134.23633739,1037.78996108)(134.2063382,1037.89996658)
\curveto(134.19633743,1037.94996092)(134.19633743,1037.99996087)(134.2063382,1038.04996658)
\curveto(134.21633741,1038.09996077)(134.2263374,1038.14496073)(134.2363382,1038.18496658)
\curveto(134.26633736,1038.29496058)(134.29633733,1038.39496048)(134.3263382,1038.48496658)
\curveto(134.36633726,1038.58496029)(134.41133721,1038.6749602)(134.4613382,1038.75496658)
\lineto(134.5513382,1038.90496658)
\lineto(134.6413382,1039.05496658)
\curveto(134.7213369,1039.16495971)(134.8213368,1039.2699596)(134.9413382,1039.36996658)
\curveto(134.96133666,1039.37995949)(134.99133663,1039.40495947)(135.0313382,1039.44496658)
\curveto(135.08133654,1039.48495939)(135.1263365,1039.51995935)(135.1663382,1039.54996658)
\curveto(135.20633642,1039.57995929)(135.25133637,1039.60995926)(135.3013382,1039.63996658)
\curveto(135.47133615,1039.74995912)(135.65133597,1039.83495904)(135.8413382,1039.89496658)
\curveto(136.03133559,1039.96495891)(136.2263354,1040.02995884)(136.4263382,1040.08996658)
\curveto(136.54633508,1040.11995875)(136.67133495,1040.13995873)(136.8013382,1040.14996658)
\curveto(136.93133469,1040.15995871)(137.06133456,1040.17995869)(137.1913382,1040.20996658)
\curveto(137.23133439,1040.21995865)(137.29133433,1040.21995865)(137.3713382,1040.20996658)
\curveto(137.46133416,1040.19995867)(137.51633411,1040.20495867)(137.5363382,1040.22496658)
\curveto(137.94633368,1040.23495864)(138.33633329,1040.21995865)(138.7063382,1040.17996658)
\curveto(139.08633254,1040.13995873)(139.4263322,1040.06495881)(139.7263382,1039.95496658)
\curveto(140.03633159,1039.84495903)(140.30133132,1039.69495918)(140.5213382,1039.50496658)
\curveto(140.74133088,1039.32495955)(140.91133071,1039.08995978)(141.0313382,1038.79996658)
\curveto(141.10133052,1038.62996024)(141.14133048,1038.43496044)(141.1513382,1038.21496658)
\curveto(141.16133046,1037.99496088)(141.16633046,1037.7699611)(141.1663382,1037.53996658)
\lineto(141.1663382,1034.19496658)
\lineto(141.1663382,1033.60996658)
\curveto(141.16633046,1033.41996545)(141.18633044,1033.24496563)(141.2263382,1033.08496658)
\curveto(141.23633039,1033.05496582)(141.24133038,1033.01996585)(141.2413382,1032.97996658)
\curveto(141.24133038,1032.94996592)(141.24633038,1032.91996595)(141.2563382,1032.88996658)
\moveto(139.0513382,1035.19996658)
\curveto(139.06133256,1035.24996362)(139.06633256,1035.30496357)(139.0663382,1035.36496658)
\curveto(139.06633256,1035.43496344)(139.06133256,1035.49496338)(139.0513382,1035.54496658)
\curveto(139.03133259,1035.60496327)(139.0213326,1035.65996321)(139.0213382,1035.70996658)
\curveto(139.0213326,1035.75996311)(139.00133262,1035.79996307)(138.9613382,1035.82996658)
\curveto(138.91133271,1035.869963)(138.83633279,1035.88996298)(138.7363382,1035.88996658)
\curveto(138.69633293,1035.87996299)(138.66133296,1035.869963)(138.6313382,1035.85996658)
\curveto(138.60133302,1035.85996301)(138.56633306,1035.85496302)(138.5263382,1035.84496658)
\curveto(138.45633317,1035.82496305)(138.38133324,1035.80996306)(138.3013382,1035.79996658)
\curveto(138.2213334,1035.78996308)(138.14133348,1035.7749631)(138.0613382,1035.75496658)
\curveto(138.03133359,1035.74496313)(137.98633364,1035.73996313)(137.9263382,1035.73996658)
\curveto(137.79633383,1035.70996316)(137.66633396,1035.68996318)(137.5363382,1035.67996658)
\curveto(137.40633422,1035.6699632)(137.28133434,1035.64496323)(137.1613382,1035.60496658)
\curveto(137.08133454,1035.58496329)(137.00633462,1035.56496331)(136.9363382,1035.54496658)
\curveto(136.86633476,1035.53496334)(136.79633483,1035.51496336)(136.7263382,1035.48496658)
\curveto(136.51633511,1035.39496348)(136.33633529,1035.25996361)(136.1863382,1035.07996658)
\curveto(136.04633558,1034.89996397)(135.99633563,1034.64996422)(136.0363382,1034.32996658)
\curveto(136.05633557,1034.15996471)(136.11133551,1034.01996485)(136.2013382,1033.90996658)
\curveto(136.27133535,1033.79996507)(136.37633525,1033.70996516)(136.5163382,1033.63996658)
\curveto(136.65633497,1033.57996529)(136.80633482,1033.53496534)(136.9663382,1033.50496658)
\curveto(137.13633449,1033.4749654)(137.31133431,1033.46496541)(137.4913382,1033.47496658)
\curveto(137.68133394,1033.49496538)(137.85633377,1033.52996534)(138.0163382,1033.57996658)
\curveto(138.27633335,1033.65996521)(138.48133314,1033.78496509)(138.6313382,1033.95496658)
\curveto(138.78133284,1034.13496474)(138.89633273,1034.35496452)(138.9763382,1034.61496658)
\curveto(138.99633263,1034.68496419)(139.00633262,1034.75496412)(139.0063382,1034.82496658)
\curveto(139.01633261,1034.90496397)(139.03133259,1034.98496389)(139.0513382,1035.06496658)
\lineto(139.0513382,1035.19996658)
}
}
{
\newrgbcolor{curcolor}{0 0 0}
\pscustom[linestyle=none,fillstyle=solid,fillcolor=curcolor]
{
\newpath
\moveto(150.40961945,1033.14496658)
\lineto(150.40961945,1032.72496658)
\curveto(150.40961108,1032.59496628)(150.37961111,1032.48996638)(150.31961945,1032.40996658)
\curveto(150.26961122,1032.35996651)(150.20461128,1032.32496655)(150.12461945,1032.30496658)
\curveto(150.04461144,1032.29496658)(149.95461153,1032.28996658)(149.85461945,1032.28996658)
\lineto(149.02961945,1032.28996658)
\lineto(148.74461945,1032.28996658)
\curveto(148.66461282,1032.29996657)(148.59961289,1032.32496655)(148.54961945,1032.36496658)
\curveto(148.47961301,1032.41496646)(148.43961305,1032.47996639)(148.42961945,1032.55996658)
\curveto(148.41961307,1032.63996623)(148.39961309,1032.71996615)(148.36961945,1032.79996658)
\curveto(148.34961314,1032.81996605)(148.32961316,1032.83496604)(148.30961945,1032.84496658)
\curveto(148.29961319,1032.86496601)(148.2846132,1032.88496599)(148.26461945,1032.90496658)
\curveto(148.15461333,1032.90496597)(148.07461341,1032.87996599)(148.02461945,1032.82996658)
\lineto(147.87461945,1032.67996658)
\curveto(147.80461368,1032.62996624)(147.73961375,1032.58496629)(147.67961945,1032.54496658)
\curveto(147.61961387,1032.51496636)(147.55461393,1032.4749664)(147.48461945,1032.42496658)
\curveto(147.44461404,1032.40496647)(147.39961409,1032.38496649)(147.34961945,1032.36496658)
\curveto(147.30961418,1032.34496653)(147.26461422,1032.32496655)(147.21461945,1032.30496658)
\curveto(147.07461441,1032.25496662)(146.92461456,1032.20996666)(146.76461945,1032.16996658)
\curveto(146.71461477,1032.14996672)(146.66961482,1032.13996673)(146.62961945,1032.13996658)
\curveto(146.5896149,1032.13996673)(146.54961494,1032.13496674)(146.50961945,1032.12496658)
\lineto(146.37461945,1032.12496658)
\curveto(146.34461514,1032.11496676)(146.30461518,1032.10996676)(146.25461945,1032.10996658)
\lineto(146.11961945,1032.10996658)
\curveto(146.05961543,1032.08996678)(145.96961552,1032.08496679)(145.84961945,1032.09496658)
\curveto(145.72961576,1032.09496678)(145.64461584,1032.10496677)(145.59461945,1032.12496658)
\curveto(145.52461596,1032.14496673)(145.45961603,1032.15496672)(145.39961945,1032.15496658)
\curveto(145.34961614,1032.14496673)(145.29461619,1032.14996672)(145.23461945,1032.16996658)
\lineto(144.87461945,1032.28996658)
\curveto(144.76461672,1032.31996655)(144.65461683,1032.35996651)(144.54461945,1032.40996658)
\curveto(144.19461729,1032.55996631)(143.87961761,1032.78996608)(143.59961945,1033.09996658)
\curveto(143.32961816,1033.41996545)(143.11461837,1033.75496512)(142.95461945,1034.10496658)
\curveto(142.90461858,1034.21496466)(142.86461862,1034.31996455)(142.83461945,1034.41996658)
\curveto(142.80461868,1034.52996434)(142.76961872,1034.63996423)(142.72961945,1034.74996658)
\curveto(142.71961877,1034.78996408)(142.71461877,1034.82496405)(142.71461945,1034.85496658)
\curveto(142.71461877,1034.89496398)(142.70461878,1034.93996393)(142.68461945,1034.98996658)
\curveto(142.66461882,1035.0699638)(142.64461884,1035.15496372)(142.62461945,1035.24496658)
\curveto(142.61461887,1035.34496353)(142.59961889,1035.44496343)(142.57961945,1035.54496658)
\curveto(142.56961892,1035.5749633)(142.56461892,1035.60996326)(142.56461945,1035.64996658)
\curveto(142.57461891,1035.68996318)(142.57461891,1035.72496315)(142.56461945,1035.75496658)
\lineto(142.56461945,1035.88996658)
\curveto(142.56461892,1035.93996293)(142.55961893,1035.98996288)(142.54961945,1036.03996658)
\curveto(142.53961895,1036.08996278)(142.53461895,1036.14496273)(142.53461945,1036.20496658)
\curveto(142.53461895,1036.2749626)(142.53961895,1036.32996254)(142.54961945,1036.36996658)
\curveto(142.55961893,1036.41996245)(142.56461892,1036.46496241)(142.56461945,1036.50496658)
\lineto(142.56461945,1036.65496658)
\curveto(142.57461891,1036.70496217)(142.57461891,1036.74996212)(142.56461945,1036.78996658)
\curveto(142.56461892,1036.83996203)(142.57461891,1036.88996198)(142.59461945,1036.93996658)
\curveto(142.61461887,1037.04996182)(142.62961886,1037.15496172)(142.63961945,1037.25496658)
\curveto(142.65961883,1037.35496152)(142.6846188,1037.45496142)(142.71461945,1037.55496658)
\curveto(142.75461873,1037.6749612)(142.7896187,1037.78996108)(142.81961945,1037.89996658)
\curveto(142.84961864,1038.00996086)(142.8896186,1038.11996075)(142.93961945,1038.22996658)
\curveto(143.07961841,1038.52996034)(143.25461823,1038.81496006)(143.46461945,1039.08496658)
\curveto(143.484618,1039.11495976)(143.50961798,1039.13995973)(143.53961945,1039.15996658)
\curveto(143.57961791,1039.18995968)(143.60961788,1039.21995965)(143.62961945,1039.24996658)
\curveto(143.66961782,1039.29995957)(143.70961778,1039.34495953)(143.74961945,1039.38496658)
\curveto(143.7896177,1039.42495945)(143.83461765,1039.46495941)(143.88461945,1039.50496658)
\curveto(143.92461756,1039.52495935)(143.95961753,1039.54995932)(143.98961945,1039.57996658)
\curveto(144.01961747,1039.61995925)(144.05461743,1039.64995922)(144.09461945,1039.66996658)
\curveto(144.34461714,1039.83995903)(144.63461685,1039.97995889)(144.96461945,1040.08996658)
\curveto(145.03461645,1040.10995876)(145.10461638,1040.12495875)(145.17461945,1040.13496658)
\curveto(145.25461623,1040.14495873)(145.33461615,1040.15995871)(145.41461945,1040.17996658)
\curveto(145.484616,1040.19995867)(145.57461591,1040.20995866)(145.68461945,1040.20996658)
\curveto(145.79461569,1040.21995865)(145.90461558,1040.22495865)(146.01461945,1040.22496658)
\curveto(146.12461536,1040.22495865)(146.22961526,1040.21995865)(146.32961945,1040.20996658)
\curveto(146.43961505,1040.19995867)(146.52961496,1040.18495869)(146.59961945,1040.16496658)
\curveto(146.74961474,1040.11495876)(146.89461459,1040.0699588)(147.03461945,1040.02996658)
\curveto(147.17461431,1039.98995888)(147.30461418,1039.93495894)(147.42461945,1039.86496658)
\curveto(147.49461399,1039.81495906)(147.55961393,1039.76495911)(147.61961945,1039.71496658)
\curveto(147.67961381,1039.6749592)(147.74461374,1039.62995924)(147.81461945,1039.57996658)
\curveto(147.85461363,1039.54995932)(147.90961358,1039.50995936)(147.97961945,1039.45996658)
\curveto(148.05961343,1039.40995946)(148.13461335,1039.40995946)(148.20461945,1039.45996658)
\curveto(148.24461324,1039.47995939)(148.26461322,1039.51495936)(148.26461945,1039.56496658)
\curveto(148.26461322,1039.61495926)(148.27461321,1039.66495921)(148.29461945,1039.71496658)
\lineto(148.29461945,1039.86496658)
\curveto(148.30461318,1039.89495898)(148.30961318,1039.92995894)(148.30961945,1039.96996658)
\lineto(148.30961945,1040.08996658)
\lineto(148.30961945,1042.12996658)
\curveto(148.30961318,1042.23995663)(148.30461318,1042.35995651)(148.29461945,1042.48996658)
\curveto(148.29461319,1042.62995624)(148.31961317,1042.73495614)(148.36961945,1042.80496658)
\curveto(148.40961308,1042.88495599)(148.484613,1042.93495594)(148.59461945,1042.95496658)
\curveto(148.61461287,1042.96495591)(148.63461285,1042.96495591)(148.65461945,1042.95496658)
\curveto(148.67461281,1042.95495592)(148.69461279,1042.95995591)(148.71461945,1042.96996658)
\lineto(149.77961945,1042.96996658)
\curveto(149.89961159,1042.9699559)(150.00961148,1042.96495591)(150.10961945,1042.95496658)
\curveto(150.20961128,1042.94495593)(150.2846112,1042.90495597)(150.33461945,1042.83496658)
\curveto(150.3846111,1042.75495612)(150.40961108,1042.64995622)(150.40961945,1042.51996658)
\lineto(150.40961945,1042.15996658)
\lineto(150.40961945,1033.14496658)
\moveto(148.36961945,1036.08496658)
\curveto(148.37961311,1036.12496275)(148.37961311,1036.16496271)(148.36961945,1036.20496658)
\lineto(148.36961945,1036.33996658)
\curveto(148.36961312,1036.43996243)(148.36461312,1036.53996233)(148.35461945,1036.63996658)
\curveto(148.34461314,1036.73996213)(148.32961316,1036.82996204)(148.30961945,1036.90996658)
\curveto(148.2896132,1037.01996185)(148.26961322,1037.11996175)(148.24961945,1037.20996658)
\curveto(148.23961325,1037.29996157)(148.21461327,1037.38496149)(148.17461945,1037.46496658)
\curveto(148.03461345,1037.82496105)(147.82961366,1038.10996076)(147.55961945,1038.31996658)
\curveto(147.29961419,1038.52996034)(146.91961457,1038.63496024)(146.41961945,1038.63496658)
\curveto(146.35961513,1038.63496024)(146.27961521,1038.62496025)(146.17961945,1038.60496658)
\curveto(146.09961539,1038.58496029)(146.02461546,1038.56496031)(145.95461945,1038.54496658)
\curveto(145.89461559,1038.53496034)(145.83461565,1038.51496036)(145.77461945,1038.48496658)
\curveto(145.50461598,1038.3749605)(145.29461619,1038.20496067)(145.14461945,1037.97496658)
\curveto(144.99461649,1037.74496113)(144.87461661,1037.48496139)(144.78461945,1037.19496658)
\curveto(144.75461673,1037.09496178)(144.73461675,1036.99496188)(144.72461945,1036.89496658)
\curveto(144.71461677,1036.79496208)(144.69461679,1036.68996218)(144.66461945,1036.57996658)
\lineto(144.66461945,1036.36996658)
\curveto(144.64461684,1036.27996259)(144.63961685,1036.15496272)(144.64961945,1035.99496658)
\curveto(144.65961683,1035.84496303)(144.67461681,1035.73496314)(144.69461945,1035.66496658)
\lineto(144.69461945,1035.57496658)
\curveto(144.70461678,1035.55496332)(144.70961678,1035.53496334)(144.70961945,1035.51496658)
\curveto(144.72961676,1035.43496344)(144.74461674,1035.35996351)(144.75461945,1035.28996658)
\curveto(144.77461671,1035.21996365)(144.79461669,1035.14496373)(144.81461945,1035.06496658)
\curveto(144.9846165,1034.54496433)(145.27461621,1034.15996471)(145.68461945,1033.90996658)
\curveto(145.81461567,1033.81996505)(145.99461549,1033.74996512)(146.22461945,1033.69996658)
\curveto(146.26461522,1033.68996518)(146.32461516,1033.68496519)(146.40461945,1033.68496658)
\curveto(146.43461505,1033.6749652)(146.47961501,1033.66496521)(146.53961945,1033.65496658)
\curveto(146.60961488,1033.65496522)(146.66461482,1033.65996521)(146.70461945,1033.66996658)
\curveto(146.7846147,1033.68996518)(146.86461462,1033.70496517)(146.94461945,1033.71496658)
\curveto(147.02461446,1033.72496515)(147.10461438,1033.74496513)(147.18461945,1033.77496658)
\curveto(147.43461405,1033.88496499)(147.63461385,1034.02496485)(147.78461945,1034.19496658)
\curveto(147.93461355,1034.36496451)(148.06461342,1034.57996429)(148.17461945,1034.83996658)
\curveto(148.21461327,1034.92996394)(148.24461324,1035.01996385)(148.26461945,1035.10996658)
\curveto(148.2846132,1035.20996366)(148.30461318,1035.31496356)(148.32461945,1035.42496658)
\curveto(148.33461315,1035.4749634)(148.33461315,1035.51996335)(148.32461945,1035.55996658)
\curveto(148.32461316,1035.60996326)(148.33461315,1035.65996321)(148.35461945,1035.70996658)
\curveto(148.36461312,1035.73996313)(148.36961312,1035.7749631)(148.36961945,1035.81496658)
\lineto(148.36961945,1035.94996658)
\lineto(148.36961945,1036.08496658)
}
}
{
\newrgbcolor{curcolor}{0 0 0}
\pscustom[linestyle=none,fillstyle=solid,fillcolor=curcolor]
{
\newpath
\moveto(159.75954132,1036.47496658)
\curveto(159.77953275,1036.41496246)(159.78953274,1036.32996254)(159.78954132,1036.21996658)
\curveto(159.78953274,1036.10996276)(159.77953275,1036.02496285)(159.75954132,1035.96496658)
\lineto(159.75954132,1035.81496658)
\curveto(159.73953279,1035.73496314)(159.7295328,1035.65496322)(159.72954132,1035.57496658)
\curveto(159.73953279,1035.49496338)(159.7345328,1035.41496346)(159.71454132,1035.33496658)
\curveto(159.69453284,1035.26496361)(159.67953285,1035.19996367)(159.66954132,1035.13996658)
\curveto(159.65953287,1035.07996379)(159.64953288,1035.01496386)(159.63954132,1034.94496658)
\curveto(159.59953293,1034.83496404)(159.56453297,1034.71996415)(159.53454132,1034.59996658)
\curveto(159.50453303,1034.48996438)(159.46453307,1034.38496449)(159.41454132,1034.28496658)
\curveto(159.20453333,1033.80496507)(158.9295336,1033.41496546)(158.58954132,1033.11496658)
\curveto(158.24953428,1032.81496606)(157.83953469,1032.56496631)(157.35954132,1032.36496658)
\curveto(157.23953529,1032.31496656)(157.11453542,1032.27996659)(156.98454132,1032.25996658)
\curveto(156.86453567,1032.22996664)(156.73953579,1032.19996667)(156.60954132,1032.16996658)
\curveto(156.55953597,1032.14996672)(156.50453603,1032.13996673)(156.44454132,1032.13996658)
\curveto(156.38453615,1032.13996673)(156.3295362,1032.13496674)(156.27954132,1032.12496658)
\lineto(156.17454132,1032.12496658)
\curveto(156.14453639,1032.11496676)(156.11453642,1032.10996676)(156.08454132,1032.10996658)
\curveto(156.0345365,1032.09996677)(155.95453658,1032.09496678)(155.84454132,1032.09496658)
\curveto(155.7345368,1032.08496679)(155.64953688,1032.08996678)(155.58954132,1032.10996658)
\lineto(155.43954132,1032.10996658)
\curveto(155.38953714,1032.11996675)(155.3345372,1032.12496675)(155.27454132,1032.12496658)
\curveto(155.22453731,1032.11496676)(155.17453736,1032.11996675)(155.12454132,1032.13996658)
\curveto(155.08453745,1032.14996672)(155.04453749,1032.15496672)(155.00454132,1032.15496658)
\curveto(154.97453756,1032.15496672)(154.9345376,1032.15996671)(154.88454132,1032.16996658)
\curveto(154.78453775,1032.19996667)(154.68453785,1032.22496665)(154.58454132,1032.24496658)
\curveto(154.48453805,1032.26496661)(154.38953814,1032.29496658)(154.29954132,1032.33496658)
\curveto(154.17953835,1032.3749665)(154.06453847,1032.41496646)(153.95454132,1032.45496658)
\curveto(153.85453868,1032.49496638)(153.74953878,1032.54496633)(153.63954132,1032.60496658)
\curveto(153.28953924,1032.81496606)(152.98953954,1033.05996581)(152.73954132,1033.33996658)
\curveto(152.48954004,1033.61996525)(152.27954025,1033.95496492)(152.10954132,1034.34496658)
\curveto(152.05954047,1034.43496444)(152.01954051,1034.52996434)(151.98954132,1034.62996658)
\curveto(151.96954056,1034.72996414)(151.94454059,1034.83496404)(151.91454132,1034.94496658)
\curveto(151.89454064,1034.99496388)(151.88454065,1035.03996383)(151.88454132,1035.07996658)
\curveto(151.88454065,1035.11996375)(151.87454066,1035.16496371)(151.85454132,1035.21496658)
\curveto(151.8345407,1035.29496358)(151.82454071,1035.3749635)(151.82454132,1035.45496658)
\curveto(151.82454071,1035.54496333)(151.81454072,1035.62996324)(151.79454132,1035.70996658)
\curveto(151.78454075,1035.75996311)(151.77954075,1035.80496307)(151.77954132,1035.84496658)
\lineto(151.77954132,1035.97996658)
\curveto(151.75954077,1036.03996283)(151.74954078,1036.12496275)(151.74954132,1036.23496658)
\curveto(151.75954077,1036.34496253)(151.77454076,1036.42996244)(151.79454132,1036.48996658)
\lineto(151.79454132,1036.59496658)
\curveto(151.80454073,1036.64496223)(151.80454073,1036.69496218)(151.79454132,1036.74496658)
\curveto(151.79454074,1036.80496207)(151.80454073,1036.85996201)(151.82454132,1036.90996658)
\curveto(151.8345407,1036.95996191)(151.83954069,1037.00496187)(151.83954132,1037.04496658)
\curveto(151.83954069,1037.09496178)(151.84954068,1037.14496173)(151.86954132,1037.19496658)
\curveto(151.90954062,1037.32496155)(151.94454059,1037.44996142)(151.97454132,1037.56996658)
\curveto(152.00454053,1037.69996117)(152.04454049,1037.82496105)(152.09454132,1037.94496658)
\curveto(152.27454026,1038.35496052)(152.48954004,1038.69496018)(152.73954132,1038.96496658)
\curveto(152.98953954,1039.24495963)(153.29453924,1039.49995937)(153.65454132,1039.72996658)
\curveto(153.75453878,1039.77995909)(153.85953867,1039.82495905)(153.96954132,1039.86496658)
\curveto(154.07953845,1039.90495897)(154.18953834,1039.94995892)(154.29954132,1039.99996658)
\curveto(154.4295381,1040.04995882)(154.56453797,1040.08495879)(154.70454132,1040.10496658)
\curveto(154.84453769,1040.12495875)(154.98953754,1040.15495872)(155.13954132,1040.19496658)
\curveto(155.21953731,1040.20495867)(155.29453724,1040.20995866)(155.36454132,1040.20996658)
\curveto(155.4345371,1040.20995866)(155.50453703,1040.21495866)(155.57454132,1040.22496658)
\curveto(156.15453638,1040.23495864)(156.65453588,1040.1749587)(157.07454132,1040.04496658)
\curveto(157.50453503,1039.91495896)(157.88453465,1039.73495914)(158.21454132,1039.50496658)
\curveto(158.32453421,1039.42495945)(158.4345341,1039.33495954)(158.54454132,1039.23496658)
\curveto(158.66453387,1039.14495973)(158.76453377,1039.04495983)(158.84454132,1038.93496658)
\curveto(158.92453361,1038.83496004)(158.99453354,1038.73496014)(159.05454132,1038.63496658)
\curveto(159.12453341,1038.53496034)(159.19453334,1038.42996044)(159.26454132,1038.31996658)
\curveto(159.3345332,1038.20996066)(159.38953314,1038.08996078)(159.42954132,1037.95996658)
\curveto(159.46953306,1037.83996103)(159.51453302,1037.70996116)(159.56454132,1037.56996658)
\curveto(159.59453294,1037.48996138)(159.61953291,1037.40496147)(159.63954132,1037.31496658)
\lineto(159.69954132,1037.04496658)
\curveto(159.70953282,1037.00496187)(159.71453282,1036.96496191)(159.71454132,1036.92496658)
\curveto(159.71453282,1036.88496199)(159.71953281,1036.84496203)(159.72954132,1036.80496658)
\curveto(159.74953278,1036.75496212)(159.75453278,1036.69996217)(159.74454132,1036.63996658)
\curveto(159.7345328,1036.57996229)(159.73953279,1036.52496235)(159.75954132,1036.47496658)
\moveto(157.65954132,1035.93496658)
\curveto(157.66953486,1035.98496289)(157.67453486,1036.05496282)(157.67454132,1036.14496658)
\curveto(157.67453486,1036.24496263)(157.66953486,1036.31996255)(157.65954132,1036.36996658)
\lineto(157.65954132,1036.48996658)
\curveto(157.63953489,1036.53996233)(157.6295349,1036.59496228)(157.62954132,1036.65496658)
\curveto(157.6295349,1036.71496216)(157.62453491,1036.7699621)(157.61454132,1036.81996658)
\curveto(157.61453492,1036.85996201)(157.60953492,1036.88996198)(157.59954132,1036.90996658)
\lineto(157.53954132,1037.14996658)
\curveto(157.529535,1037.23996163)(157.50953502,1037.32496155)(157.47954132,1037.40496658)
\curveto(157.36953516,1037.66496121)(157.23953529,1037.88496099)(157.08954132,1038.06496658)
\curveto(156.93953559,1038.25496062)(156.73953579,1038.40496047)(156.48954132,1038.51496658)
\curveto(156.4295361,1038.53496034)(156.36953616,1038.54996032)(156.30954132,1038.55996658)
\curveto(156.24953628,1038.57996029)(156.18453635,1038.59996027)(156.11454132,1038.61996658)
\curveto(156.0345365,1038.63996023)(155.94953658,1038.64496023)(155.85954132,1038.63496658)
\lineto(155.58954132,1038.63496658)
\curveto(155.55953697,1038.61496026)(155.52453701,1038.60496027)(155.48454132,1038.60496658)
\curveto(155.44453709,1038.61496026)(155.40953712,1038.61496026)(155.37954132,1038.60496658)
\lineto(155.16954132,1038.54496658)
\curveto(155.10953742,1038.53496034)(155.05453748,1038.51496036)(155.00454132,1038.48496658)
\curveto(154.75453778,1038.3749605)(154.54953798,1038.21496066)(154.38954132,1038.00496658)
\curveto(154.23953829,1037.80496107)(154.11953841,1037.5699613)(154.02954132,1037.29996658)
\curveto(153.99953853,1037.19996167)(153.97453856,1037.09496178)(153.95454132,1036.98496658)
\curveto(153.94453859,1036.874962)(153.9295386,1036.76496211)(153.90954132,1036.65496658)
\curveto(153.89953863,1036.60496227)(153.89453864,1036.55496232)(153.89454132,1036.50496658)
\lineto(153.89454132,1036.35496658)
\curveto(153.87453866,1036.28496259)(153.86453867,1036.17996269)(153.86454132,1036.03996658)
\curveto(153.87453866,1035.89996297)(153.88953864,1035.79496308)(153.90954132,1035.72496658)
\lineto(153.90954132,1035.58996658)
\curveto(153.9295386,1035.50996336)(153.94453859,1035.42996344)(153.95454132,1035.34996658)
\curveto(153.96453857,1035.27996359)(153.97953855,1035.20496367)(153.99954132,1035.12496658)
\curveto(154.09953843,1034.82496405)(154.20453833,1034.57996429)(154.31454132,1034.38996658)
\curveto(154.4345381,1034.20996466)(154.61953791,1034.04496483)(154.86954132,1033.89496658)
\curveto(154.93953759,1033.84496503)(155.01453752,1033.80496507)(155.09454132,1033.77496658)
\curveto(155.18453735,1033.74496513)(155.27453726,1033.71996515)(155.36454132,1033.69996658)
\curveto(155.40453713,1033.68996518)(155.43953709,1033.68496519)(155.46954132,1033.68496658)
\curveto(155.49953703,1033.69496518)(155.534537,1033.69496518)(155.57454132,1033.68496658)
\lineto(155.69454132,1033.65496658)
\curveto(155.74453679,1033.65496522)(155.78953674,1033.65996521)(155.82954132,1033.66996658)
\lineto(155.94954132,1033.66996658)
\curveto(156.0295365,1033.68996518)(156.10953642,1033.70496517)(156.18954132,1033.71496658)
\curveto(156.26953626,1033.72496515)(156.34453619,1033.74496513)(156.41454132,1033.77496658)
\curveto(156.67453586,1033.874965)(156.88453565,1034.00996486)(157.04454132,1034.17996658)
\curveto(157.20453533,1034.34996452)(157.33953519,1034.55996431)(157.44954132,1034.80996658)
\curveto(157.48953504,1034.90996396)(157.51953501,1035.00996386)(157.53954132,1035.10996658)
\curveto(157.55953497,1035.20996366)(157.58453495,1035.31496356)(157.61454132,1035.42496658)
\curveto(157.62453491,1035.46496341)(157.6295349,1035.49996337)(157.62954132,1035.52996658)
\curveto(157.6295349,1035.5699633)(157.6345349,1035.60996326)(157.64454132,1035.64996658)
\lineto(157.64454132,1035.78496658)
\curveto(157.64453489,1035.83496304)(157.64953488,1035.88496299)(157.65954132,1035.93496658)
}
}
{
\newrgbcolor{curcolor}{0 0 0}
\pscustom[linestyle=none,fillstyle=solid,fillcolor=curcolor]
{
\newpath
\moveto(164.1294632,1040.23996658)
\curveto(164.8794587,1040.25995861)(165.52945805,1040.1749587)(166.0794632,1039.98496658)
\curveto(166.63945694,1039.80495907)(167.06445651,1039.48995938)(167.3544632,1039.03996658)
\curveto(167.42445615,1038.92995994)(167.48445609,1038.81496006)(167.5344632,1038.69496658)
\curveto(167.59445598,1038.58496029)(167.64445593,1038.45996041)(167.6844632,1038.31996658)
\curveto(167.70445587,1038.25996061)(167.71445586,1038.19496068)(167.7144632,1038.12496658)
\curveto(167.71445586,1038.05496082)(167.70445587,1037.99496088)(167.6844632,1037.94496658)
\curveto(167.64445593,1037.88496099)(167.58945599,1037.84496103)(167.5194632,1037.82496658)
\curveto(167.46945611,1037.80496107)(167.40945617,1037.79496108)(167.3394632,1037.79496658)
\lineto(167.1294632,1037.79496658)
\lineto(166.4694632,1037.79496658)
\curveto(166.39945718,1037.79496108)(166.32945725,1037.78996108)(166.2594632,1037.77996658)
\curveto(166.18945739,1037.77996109)(166.12445745,1037.78996108)(166.0644632,1037.80996658)
\curveto(165.96445761,1037.82996104)(165.88945769,1037.869961)(165.8394632,1037.92996658)
\curveto(165.78945779,1037.98996088)(165.74445783,1038.04996082)(165.7044632,1038.10996658)
\lineto(165.5844632,1038.31996658)
\curveto(165.55445802,1038.39996047)(165.50445807,1038.46496041)(165.4344632,1038.51496658)
\curveto(165.33445824,1038.59496028)(165.23445834,1038.65496022)(165.1344632,1038.69496658)
\curveto(165.04445853,1038.73496014)(164.92945865,1038.7699601)(164.7894632,1038.79996658)
\curveto(164.71945886,1038.81996005)(164.61445896,1038.83496004)(164.4744632,1038.84496658)
\curveto(164.34445923,1038.85496002)(164.24445933,1038.84996002)(164.1744632,1038.82996658)
\lineto(164.0694632,1038.82996658)
\lineto(163.9194632,1038.79996658)
\curveto(163.8794597,1038.79996007)(163.83445974,1038.79496008)(163.7844632,1038.78496658)
\curveto(163.61445996,1038.73496014)(163.4744601,1038.66496021)(163.3644632,1038.57496658)
\curveto(163.26446031,1038.49496038)(163.19446038,1038.3699605)(163.1544632,1038.19996658)
\curveto(163.13446044,1038.12996074)(163.13446044,1038.06496081)(163.1544632,1038.00496658)
\curveto(163.1744604,1037.94496093)(163.19446038,1037.89496098)(163.2144632,1037.85496658)
\curveto(163.28446029,1037.73496114)(163.36446021,1037.63996123)(163.4544632,1037.56996658)
\curveto(163.55446002,1037.49996137)(163.66945991,1037.43996143)(163.7994632,1037.38996658)
\curveto(163.98945959,1037.30996156)(164.19445938,1037.23996163)(164.4144632,1037.17996658)
\lineto(165.1044632,1037.02996658)
\curveto(165.34445823,1036.98996188)(165.574458,1036.93996193)(165.7944632,1036.87996658)
\curveto(166.02445755,1036.82996204)(166.23945734,1036.76496211)(166.4394632,1036.68496658)
\curveto(166.52945705,1036.64496223)(166.61445696,1036.60996226)(166.6944632,1036.57996658)
\curveto(166.78445679,1036.55996231)(166.86945671,1036.52496235)(166.9494632,1036.47496658)
\curveto(167.13945644,1036.35496252)(167.30945627,1036.22496265)(167.4594632,1036.08496658)
\curveto(167.61945596,1035.94496293)(167.74445583,1035.7699631)(167.8344632,1035.55996658)
\curveto(167.86445571,1035.48996338)(167.88945569,1035.41996345)(167.9094632,1035.34996658)
\curveto(167.92945565,1035.27996359)(167.94945563,1035.20496367)(167.9694632,1035.12496658)
\curveto(167.9794556,1035.06496381)(167.98445559,1034.9699639)(167.9844632,1034.83996658)
\curveto(167.99445558,1034.71996415)(167.99445558,1034.62496425)(167.9844632,1034.55496658)
\lineto(167.9844632,1034.47996658)
\curveto(167.96445561,1034.41996445)(167.94945563,1034.35996451)(167.9394632,1034.29996658)
\curveto(167.93945564,1034.24996462)(167.93445564,1034.19996467)(167.9244632,1034.14996658)
\curveto(167.85445572,1033.84996502)(167.74445583,1033.58496529)(167.5944632,1033.35496658)
\curveto(167.43445614,1033.11496576)(167.23945634,1032.91996595)(167.0094632,1032.76996658)
\curveto(166.7794568,1032.61996625)(166.51945706,1032.48996638)(166.2294632,1032.37996658)
\curveto(166.11945746,1032.32996654)(165.99945758,1032.29496658)(165.8694632,1032.27496658)
\curveto(165.74945783,1032.25496662)(165.62945795,1032.22996664)(165.5094632,1032.19996658)
\curveto(165.41945816,1032.17996669)(165.32445825,1032.1699667)(165.2244632,1032.16996658)
\curveto(165.13445844,1032.15996671)(165.04445853,1032.14496673)(164.9544632,1032.12496658)
\lineto(164.6844632,1032.12496658)
\curveto(164.62445895,1032.10496677)(164.51945906,1032.09496678)(164.3694632,1032.09496658)
\curveto(164.22945935,1032.09496678)(164.12945945,1032.10496677)(164.0694632,1032.12496658)
\curveto(164.03945954,1032.12496675)(164.00445957,1032.12996674)(163.9644632,1032.13996658)
\lineto(163.8594632,1032.13996658)
\curveto(163.73945984,1032.15996671)(163.61945996,1032.1749667)(163.4994632,1032.18496658)
\curveto(163.3794602,1032.19496668)(163.26446031,1032.21496666)(163.1544632,1032.24496658)
\curveto(162.76446081,1032.35496652)(162.41946116,1032.47996639)(162.1194632,1032.61996658)
\curveto(161.81946176,1032.7699661)(161.56446201,1032.98996588)(161.3544632,1033.27996658)
\curveto(161.21446236,1033.4699654)(161.09446248,1033.68996518)(160.9944632,1033.93996658)
\curveto(160.9744626,1033.99996487)(160.95446262,1034.07996479)(160.9344632,1034.17996658)
\curveto(160.91446266,1034.22996464)(160.89946268,1034.29996457)(160.8894632,1034.38996658)
\curveto(160.8794627,1034.47996439)(160.88446269,1034.55496432)(160.9044632,1034.61496658)
\curveto(160.93446264,1034.68496419)(160.98446259,1034.73496414)(161.0544632,1034.76496658)
\curveto(161.10446247,1034.78496409)(161.16446241,1034.79496408)(161.2344632,1034.79496658)
\lineto(161.4594632,1034.79496658)
\lineto(162.1644632,1034.79496658)
\lineto(162.4044632,1034.79496658)
\curveto(162.48446109,1034.79496408)(162.55446102,1034.78496409)(162.6144632,1034.76496658)
\curveto(162.72446085,1034.72496415)(162.79446078,1034.65996421)(162.8244632,1034.56996658)
\curveto(162.86446071,1034.47996439)(162.90946067,1034.38496449)(162.9594632,1034.28496658)
\curveto(162.9794606,1034.23496464)(163.01446056,1034.1699647)(163.0644632,1034.08996658)
\curveto(163.12446045,1034.00996486)(163.1744604,1033.95996491)(163.2144632,1033.93996658)
\curveto(163.33446024,1033.83996503)(163.44946013,1033.75996511)(163.5594632,1033.69996658)
\curveto(163.66945991,1033.64996522)(163.80945977,1033.59996527)(163.9794632,1033.54996658)
\curveto(164.02945955,1033.52996534)(164.0794595,1033.51996535)(164.1294632,1033.51996658)
\curveto(164.1794594,1033.52996534)(164.22945935,1033.52996534)(164.2794632,1033.51996658)
\curveto(164.35945922,1033.49996537)(164.44445913,1033.48996538)(164.5344632,1033.48996658)
\curveto(164.63445894,1033.49996537)(164.71945886,1033.51496536)(164.7894632,1033.53496658)
\curveto(164.83945874,1033.54496533)(164.88445869,1033.54996532)(164.9244632,1033.54996658)
\curveto(164.9744586,1033.54996532)(165.02445855,1033.55996531)(165.0744632,1033.57996658)
\curveto(165.21445836,1033.62996524)(165.33945824,1033.68996518)(165.4494632,1033.75996658)
\curveto(165.56945801,1033.82996504)(165.66445791,1033.91996495)(165.7344632,1034.02996658)
\curveto(165.78445779,1034.10996476)(165.82445775,1034.23496464)(165.8544632,1034.40496658)
\curveto(165.8744577,1034.4749644)(165.8744577,1034.53996433)(165.8544632,1034.59996658)
\curveto(165.83445774,1034.65996421)(165.81445776,1034.70996416)(165.7944632,1034.74996658)
\curveto(165.72445785,1034.88996398)(165.63445794,1034.99496388)(165.5244632,1035.06496658)
\curveto(165.42445815,1035.13496374)(165.30445827,1035.19996367)(165.1644632,1035.25996658)
\curveto(164.9744586,1035.33996353)(164.7744588,1035.40496347)(164.5644632,1035.45496658)
\curveto(164.35445922,1035.50496337)(164.14445943,1035.55996331)(163.9344632,1035.61996658)
\curveto(163.85445972,1035.63996323)(163.76945981,1035.65496322)(163.6794632,1035.66496658)
\curveto(163.59945998,1035.6749632)(163.51946006,1035.68996318)(163.4394632,1035.70996658)
\curveto(163.11946046,1035.79996307)(162.81446076,1035.88496299)(162.5244632,1035.96496658)
\curveto(162.23446134,1036.05496282)(161.96946161,1036.18496269)(161.7294632,1036.35496658)
\curveto(161.44946213,1036.55496232)(161.24446233,1036.82496205)(161.1144632,1037.16496658)
\curveto(161.09446248,1037.23496164)(161.0744625,1037.32996154)(161.0544632,1037.44996658)
\curveto(161.03446254,1037.51996135)(161.01946256,1037.60496127)(161.0094632,1037.70496658)
\curveto(160.99946258,1037.80496107)(161.00446257,1037.89496098)(161.0244632,1037.97496658)
\curveto(161.04446253,1038.02496085)(161.04946253,1038.06496081)(161.0394632,1038.09496658)
\curveto(161.02946255,1038.13496074)(161.03446254,1038.17996069)(161.0544632,1038.22996658)
\curveto(161.0744625,1038.33996053)(161.09446248,1038.43996043)(161.1144632,1038.52996658)
\curveto(161.14446243,1038.62996024)(161.1794624,1038.72496015)(161.2194632,1038.81496658)
\curveto(161.34946223,1039.10495977)(161.52946205,1039.33995953)(161.7594632,1039.51996658)
\curveto(161.98946159,1039.69995917)(162.24946133,1039.84495903)(162.5394632,1039.95496658)
\curveto(162.64946093,1040.00495887)(162.76446081,1040.03995883)(162.8844632,1040.05996658)
\curveto(163.00446057,1040.08995878)(163.12946045,1040.11995875)(163.2594632,1040.14996658)
\curveto(163.31946026,1040.1699587)(163.3794602,1040.17995869)(163.4394632,1040.17996658)
\lineto(163.6194632,1040.20996658)
\curveto(163.69945988,1040.21995865)(163.78445979,1040.22495865)(163.8744632,1040.22496658)
\curveto(163.96445961,1040.22495865)(164.04945953,1040.22995864)(164.1294632,1040.23996658)
}
}
{
\newrgbcolor{curcolor}{0 0 0}
\pscustom[linestyle=none,fillstyle=solid,fillcolor=curcolor]
{
\newpath
\moveto(387.77836182,1043.71425148)
\lineto(389.05336182,1043.71425148)
\curveto(389.16335904,1043.71424077)(389.26835893,1043.70924078)(389.36836182,1043.69925148)
\curveto(389.47835872,1043.6892408)(389.55835864,1043.65424083)(389.60836182,1043.59425148)
\curveto(389.65835854,1043.51424097)(389.68335852,1043.40924108)(389.68336182,1043.27925148)
\curveto(389.69335851,1043.15924133)(389.6983585,1043.03424145)(389.69836182,1042.90425148)
\lineto(389.69836182,1041.38925148)
\lineto(389.69836182,1038.29925148)
\lineto(389.69836182,1037.77425148)
\curveto(389.6983585,1037.73424675)(389.69335851,1037.6892468)(389.68336182,1037.63925148)
\curveto(389.68335852,1037.59924689)(389.68835851,1037.55924693)(389.69836182,1037.51925148)
\lineto(389.69836182,1037.27925148)
\curveto(389.6983585,1037.1892473)(389.69335851,1037.09424739)(389.68336182,1036.99425148)
\curveto(389.68335852,1036.89424759)(389.69335851,1036.80424768)(389.71336182,1036.72425148)
\curveto(389.71335849,1036.65424783)(389.71835848,1036.59924789)(389.72836182,1036.55925148)
\curveto(389.74835845,1036.44924804)(389.76335844,1036.33924815)(389.77336182,1036.22925148)
\curveto(389.79335841,1036.11924837)(389.82335838,1036.00924848)(389.86336182,1035.89925148)
\curveto(389.97335823,1035.63924885)(390.11335809,1035.42424906)(390.28336182,1035.25425148)
\curveto(390.46335774,1035.0842494)(390.6983575,1034.94924954)(390.98836182,1034.84925148)
\curveto(391.06835713,1034.82924966)(391.14835705,1034.81424967)(391.22836182,1034.80425148)
\curveto(391.30835689,1034.79424969)(391.38835681,1034.77924971)(391.46836182,1034.75925148)
\curveto(391.51835668,1034.73924975)(391.56335664,1034.72924976)(391.60336182,1034.72925148)
\curveto(391.64335656,1034.73924975)(391.68835651,1034.73924975)(391.73836182,1034.72925148)
\curveto(391.77835642,1034.71924977)(391.84335636,1034.71424977)(391.93336182,1034.71425148)
\curveto(392.02335618,1034.72424976)(392.08335612,1034.73424975)(392.11336182,1034.74425148)
\lineto(392.33836182,1034.74425148)
\curveto(392.41835578,1034.76424972)(392.4983557,1034.77924971)(392.57836182,1034.78925148)
\curveto(392.65835554,1034.79924969)(392.73335547,1034.81424967)(392.80336182,1034.83425148)
\curveto(392.94335526,1034.86424962)(393.05335515,1034.89924959)(393.13336182,1034.93925148)
\curveto(393.31335489,1035.01924947)(393.46835473,1035.12424936)(393.59836182,1035.25425148)
\curveto(393.73835446,1035.39424909)(393.84835435,1035.54924894)(393.92836182,1035.71925148)
\curveto(394.03835416,1035.97924851)(394.1033541,1036.2842482)(394.12336182,1036.63425148)
\curveto(394.14335406,1036.99424749)(394.15335405,1037.36424712)(394.15336182,1037.74425148)
\lineto(394.15336182,1040.72925148)
\lineto(394.15336182,1042.73925148)
\curveto(394.15335405,1042.87924161)(394.14835405,1043.03424145)(394.13836182,1043.20425148)
\curveto(394.13835406,1043.37424111)(394.16835403,1043.49924099)(394.22836182,1043.57925148)
\curveto(394.27835392,1043.63924085)(394.34835385,1043.67424081)(394.43836182,1043.68425148)
\curveto(394.52835367,1043.70424078)(394.62835357,1043.71424077)(394.73836182,1043.71425148)
\lineto(395.69836182,1043.71425148)
\curveto(395.77835242,1043.71424077)(395.85335235,1043.71424077)(395.92336182,1043.71425148)
\curveto(396.0033522,1043.72424076)(396.07835212,1043.71924077)(396.14836182,1043.69925148)
\curveto(396.28835191,1043.66924082)(396.37835182,1043.61924087)(396.41836182,1043.54925148)
\curveto(396.46835173,1043.46924102)(396.48835171,1043.35424113)(396.47836182,1043.20425148)
\curveto(396.47835172,1043.06424142)(396.47835172,1042.93424155)(396.47836182,1042.81425148)
\lineto(396.47836182,1040.80425148)
\lineto(396.47836182,1037.77425148)
\curveto(396.47835172,1037.39424709)(396.47335173,1037.02424746)(396.46336182,1036.66425148)
\curveto(396.45335175,1036.30424818)(396.40835179,1035.97924851)(396.32836182,1035.68925148)
\curveto(396.18835201,1035.21924927)(396.00835219,1034.80924968)(395.78836182,1034.45925148)
\curveto(395.57835262,1034.11925037)(395.2983529,1033.82925066)(394.94836182,1033.58925148)
\curveto(394.63835356,1033.36925112)(394.27335393,1033.1892513)(393.85336182,1033.04925148)
\curveto(393.76335444,1033.01925147)(393.66835453,1032.99425149)(393.56836182,1032.97425148)
\lineto(393.29836182,1032.91425148)
\curveto(393.23835496,1032.89425159)(393.17835502,1032.8842516)(393.11836182,1032.88425148)
\curveto(393.06835513,1032.8842516)(393.01335519,1032.87425161)(392.95336182,1032.85425148)
\curveto(392.83335537,1032.83425165)(392.6983555,1032.81925167)(392.54836182,1032.80925148)
\curveto(392.3983558,1032.79925169)(392.25335595,1032.79425169)(392.11336182,1032.79425148)
\curveto(391.16335704,1032.7842517)(390.35335785,1032.89925159)(389.68336182,1033.13925148)
\curveto(389.01335919,1033.3892511)(388.48835971,1033.7892507)(388.10836182,1034.33925148)
\curveto(387.97836022,1034.51924997)(387.86836033,1034.70424978)(387.77836182,1034.89425148)
\curveto(387.6983605,1035.09424939)(387.62336058,1035.30924918)(387.55336182,1035.53925148)
\curveto(387.53336067,1035.5892489)(387.52336068,1035.62924886)(387.52336182,1035.65925148)
\curveto(387.52336068,1035.69924879)(387.51336069,1035.74424874)(387.49336182,1035.79425148)
\curveto(387.41336079,1036.07424841)(387.37336083,1036.3892481)(387.37336182,1036.73925148)
\lineto(387.37336182,1037.78925148)
\lineto(387.37336182,1041.97425148)
\lineto(387.37336182,1043.02425148)
\lineto(387.37336182,1043.30925148)
\curveto(387.37336083,1043.40924108)(387.38836081,1043.489241)(387.41836182,1043.54925148)
\curveto(387.47836072,1043.61924087)(387.55836064,1043.66924082)(387.65836182,1043.69925148)
\curveto(387.67836052,1043.69924079)(387.6983605,1043.69924079)(387.71836182,1043.69925148)
\curveto(387.73836046,1043.69924079)(387.75836044,1043.70424078)(387.77836182,1043.71425148)
}
}
{
\newrgbcolor{curcolor}{0 0 0}
\pscustom[linestyle=none,fillstyle=solid,fillcolor=curcolor]
{
\newpath
\moveto(401.23687744,1040.95425148)
\curveto(401.98687294,1040.97424351)(402.63687229,1040.8892436)(403.18687744,1040.69925148)
\curveto(403.74687118,1040.51924397)(404.17187076,1040.20424428)(404.46187744,1039.75425148)
\curveto(404.5318704,1039.64424484)(404.59187034,1039.52924496)(404.64187744,1039.40925148)
\curveto(404.70187023,1039.29924519)(404.75187018,1039.17424531)(404.79187744,1039.03425148)
\curveto(404.81187012,1038.97424551)(404.82187011,1038.90924558)(404.82187744,1038.83925148)
\curveto(404.82187011,1038.76924572)(404.81187012,1038.70924578)(404.79187744,1038.65925148)
\curveto(404.75187018,1038.59924589)(404.69687023,1038.55924593)(404.62687744,1038.53925148)
\curveto(404.57687035,1038.51924597)(404.51687041,1038.50924598)(404.44687744,1038.50925148)
\lineto(404.23687744,1038.50925148)
\lineto(403.57687744,1038.50925148)
\curveto(403.50687142,1038.50924598)(403.43687149,1038.50424598)(403.36687744,1038.49425148)
\curveto(403.29687163,1038.49424599)(403.2318717,1038.50424598)(403.17187744,1038.52425148)
\curveto(403.07187186,1038.54424594)(402.99687193,1038.5842459)(402.94687744,1038.64425148)
\curveto(402.89687203,1038.70424578)(402.85187208,1038.76424572)(402.81187744,1038.82425148)
\lineto(402.69187744,1039.03425148)
\curveto(402.66187227,1039.11424537)(402.61187232,1039.17924531)(402.54187744,1039.22925148)
\curveto(402.44187249,1039.30924518)(402.34187259,1039.36924512)(402.24187744,1039.40925148)
\curveto(402.15187278,1039.44924504)(402.03687289,1039.484245)(401.89687744,1039.51425148)
\curveto(401.8268731,1039.53424495)(401.72187321,1039.54924494)(401.58187744,1039.55925148)
\curveto(401.45187348,1039.56924492)(401.35187358,1039.56424492)(401.28187744,1039.54425148)
\lineto(401.17687744,1039.54425148)
\lineto(401.02687744,1039.51425148)
\curveto(400.98687394,1039.51424497)(400.94187399,1039.50924498)(400.89187744,1039.49925148)
\curveto(400.72187421,1039.44924504)(400.58187435,1039.37924511)(400.47187744,1039.28925148)
\curveto(400.37187456,1039.20924528)(400.30187463,1039.0842454)(400.26187744,1038.91425148)
\curveto(400.24187469,1038.84424564)(400.24187469,1038.77924571)(400.26187744,1038.71925148)
\curveto(400.28187465,1038.65924583)(400.30187463,1038.60924588)(400.32187744,1038.56925148)
\curveto(400.39187454,1038.44924604)(400.47187446,1038.35424613)(400.56187744,1038.28425148)
\curveto(400.66187427,1038.21424627)(400.77687415,1038.15424633)(400.90687744,1038.10425148)
\curveto(401.09687383,1038.02424646)(401.30187363,1037.95424653)(401.52187744,1037.89425148)
\lineto(402.21187744,1037.74425148)
\curveto(402.45187248,1037.70424678)(402.68187225,1037.65424683)(402.90187744,1037.59425148)
\curveto(403.1318718,1037.54424694)(403.34687158,1037.47924701)(403.54687744,1037.39925148)
\curveto(403.63687129,1037.35924713)(403.72187121,1037.32424716)(403.80187744,1037.29425148)
\curveto(403.89187104,1037.27424721)(403.97687095,1037.23924725)(404.05687744,1037.18925148)
\curveto(404.24687068,1037.06924742)(404.41687051,1036.93924755)(404.56687744,1036.79925148)
\curveto(404.7268702,1036.65924783)(404.85187008,1036.484248)(404.94187744,1036.27425148)
\curveto(404.97186996,1036.20424828)(404.99686993,1036.13424835)(405.01687744,1036.06425148)
\curveto(405.03686989,1035.99424849)(405.05686987,1035.91924857)(405.07687744,1035.83925148)
\curveto(405.08686984,1035.77924871)(405.09186984,1035.6842488)(405.09187744,1035.55425148)
\curveto(405.10186983,1035.43424905)(405.10186983,1035.33924915)(405.09187744,1035.26925148)
\lineto(405.09187744,1035.19425148)
\curveto(405.07186986,1035.13424935)(405.05686987,1035.07424941)(405.04687744,1035.01425148)
\curveto(405.04686988,1034.96424952)(405.04186989,1034.91424957)(405.03187744,1034.86425148)
\curveto(404.96186997,1034.56424992)(404.85187008,1034.29925019)(404.70187744,1034.06925148)
\curveto(404.54187039,1033.82925066)(404.34687058,1033.63425085)(404.11687744,1033.48425148)
\curveto(403.88687104,1033.33425115)(403.6268713,1033.20425128)(403.33687744,1033.09425148)
\curveto(403.2268717,1033.04425144)(403.10687182,1033.00925148)(402.97687744,1032.98925148)
\curveto(402.85687207,1032.96925152)(402.73687219,1032.94425154)(402.61687744,1032.91425148)
\curveto(402.5268724,1032.89425159)(402.4318725,1032.8842516)(402.33187744,1032.88425148)
\curveto(402.24187269,1032.87425161)(402.15187278,1032.85925163)(402.06187744,1032.83925148)
\lineto(401.79187744,1032.83925148)
\curveto(401.7318732,1032.81925167)(401.6268733,1032.80925168)(401.47687744,1032.80925148)
\curveto(401.33687359,1032.80925168)(401.23687369,1032.81925167)(401.17687744,1032.83925148)
\curveto(401.14687378,1032.83925165)(401.11187382,1032.84425164)(401.07187744,1032.85425148)
\lineto(400.96687744,1032.85425148)
\curveto(400.84687408,1032.87425161)(400.7268742,1032.8892516)(400.60687744,1032.89925148)
\curveto(400.48687444,1032.90925158)(400.37187456,1032.92925156)(400.26187744,1032.95925148)
\curveto(399.87187506,1033.06925142)(399.5268754,1033.19425129)(399.22687744,1033.33425148)
\curveto(398.926876,1033.484251)(398.67187626,1033.70425078)(398.46187744,1033.99425148)
\curveto(398.32187661,1034.1842503)(398.20187673,1034.40425008)(398.10187744,1034.65425148)
\curveto(398.08187685,1034.71424977)(398.06187687,1034.79424969)(398.04187744,1034.89425148)
\curveto(398.02187691,1034.94424954)(398.00687692,1035.01424947)(397.99687744,1035.10425148)
\curveto(397.98687694,1035.19424929)(397.99187694,1035.26924922)(398.01187744,1035.32925148)
\curveto(398.04187689,1035.39924909)(398.09187684,1035.44924904)(398.16187744,1035.47925148)
\curveto(398.21187672,1035.49924899)(398.27187666,1035.50924898)(398.34187744,1035.50925148)
\lineto(398.56687744,1035.50925148)
\lineto(399.27187744,1035.50925148)
\lineto(399.51187744,1035.50925148)
\curveto(399.59187534,1035.50924898)(399.66187527,1035.49924899)(399.72187744,1035.47925148)
\curveto(399.8318751,1035.43924905)(399.90187503,1035.37424911)(399.93187744,1035.28425148)
\curveto(399.97187496,1035.19424929)(400.01687491,1035.09924939)(400.06687744,1034.99925148)
\curveto(400.08687484,1034.94924954)(400.12187481,1034.8842496)(400.17187744,1034.80425148)
\curveto(400.2318747,1034.72424976)(400.28187465,1034.67424981)(400.32187744,1034.65425148)
\curveto(400.44187449,1034.55424993)(400.55687437,1034.47425001)(400.66687744,1034.41425148)
\curveto(400.77687415,1034.36425012)(400.91687401,1034.31425017)(401.08687744,1034.26425148)
\curveto(401.13687379,1034.24425024)(401.18687374,1034.23425025)(401.23687744,1034.23425148)
\curveto(401.28687364,1034.24425024)(401.33687359,1034.24425024)(401.38687744,1034.23425148)
\curveto(401.46687346,1034.21425027)(401.55187338,1034.20425028)(401.64187744,1034.20425148)
\curveto(401.74187319,1034.21425027)(401.8268731,1034.22925026)(401.89687744,1034.24925148)
\curveto(401.94687298,1034.25925023)(401.99187294,1034.26425022)(402.03187744,1034.26425148)
\curveto(402.08187285,1034.26425022)(402.1318728,1034.27425021)(402.18187744,1034.29425148)
\curveto(402.32187261,1034.34425014)(402.44687248,1034.40425008)(402.55687744,1034.47425148)
\curveto(402.67687225,1034.54424994)(402.77187216,1034.63424985)(402.84187744,1034.74425148)
\curveto(402.89187204,1034.82424966)(402.931872,1034.94924954)(402.96187744,1035.11925148)
\curveto(402.98187195,1035.1892493)(402.98187195,1035.25424923)(402.96187744,1035.31425148)
\curveto(402.94187199,1035.37424911)(402.92187201,1035.42424906)(402.90187744,1035.46425148)
\curveto(402.8318721,1035.60424888)(402.74187219,1035.70924878)(402.63187744,1035.77925148)
\curveto(402.5318724,1035.84924864)(402.41187252,1035.91424857)(402.27187744,1035.97425148)
\curveto(402.08187285,1036.05424843)(401.88187305,1036.11924837)(401.67187744,1036.16925148)
\curveto(401.46187347,1036.21924827)(401.25187368,1036.27424821)(401.04187744,1036.33425148)
\curveto(400.96187397,1036.35424813)(400.87687405,1036.36924812)(400.78687744,1036.37925148)
\curveto(400.70687422,1036.3892481)(400.6268743,1036.40424808)(400.54687744,1036.42425148)
\curveto(400.2268747,1036.51424797)(399.92187501,1036.59924789)(399.63187744,1036.67925148)
\curveto(399.34187559,1036.76924772)(399.07687585,1036.89924759)(398.83687744,1037.06925148)
\curveto(398.55687637,1037.26924722)(398.35187658,1037.53924695)(398.22187744,1037.87925148)
\curveto(398.20187673,1037.94924654)(398.18187675,1038.04424644)(398.16187744,1038.16425148)
\curveto(398.14187679,1038.23424625)(398.1268768,1038.31924617)(398.11687744,1038.41925148)
\curveto(398.10687682,1038.51924597)(398.11187682,1038.60924588)(398.13187744,1038.68925148)
\curveto(398.15187678,1038.73924575)(398.15687677,1038.77924571)(398.14687744,1038.80925148)
\curveto(398.13687679,1038.84924564)(398.14187679,1038.89424559)(398.16187744,1038.94425148)
\curveto(398.18187675,1039.05424543)(398.20187673,1039.15424533)(398.22187744,1039.24425148)
\curveto(398.25187668,1039.34424514)(398.28687664,1039.43924505)(398.32687744,1039.52925148)
\curveto(398.45687647,1039.81924467)(398.63687629,1040.05424443)(398.86687744,1040.23425148)
\curveto(399.09687583,1040.41424407)(399.35687557,1040.55924393)(399.64687744,1040.66925148)
\curveto(399.75687517,1040.71924377)(399.87187506,1040.75424373)(399.99187744,1040.77425148)
\curveto(400.11187482,1040.80424368)(400.23687469,1040.83424365)(400.36687744,1040.86425148)
\curveto(400.4268745,1040.8842436)(400.48687444,1040.89424359)(400.54687744,1040.89425148)
\lineto(400.72687744,1040.92425148)
\curveto(400.80687412,1040.93424355)(400.89187404,1040.93924355)(400.98187744,1040.93925148)
\curveto(401.07187386,1040.93924355)(401.15687377,1040.94424354)(401.23687744,1040.95425148)
}
}
{
\newrgbcolor{curcolor}{0 0 0}
\pscustom[linestyle=none,fillstyle=solid,fillcolor=curcolor]
{
\newpath
\moveto(406.74351807,1040.72925148)
\lineto(407.86851807,1040.72925148)
\curveto(407.97851563,1040.72924376)(408.07851553,1040.72424376)(408.16851807,1040.71425148)
\curveto(408.25851535,1040.70424378)(408.32351529,1040.66924382)(408.36351807,1040.60925148)
\curveto(408.4135152,1040.54924394)(408.44351517,1040.46424402)(408.45351807,1040.35425148)
\curveto(408.46351515,1040.25424423)(408.46851514,1040.14924434)(408.46851807,1040.03925148)
\lineto(408.46851807,1038.98925148)
\lineto(408.46851807,1036.75425148)
\curveto(408.46851514,1036.39424809)(408.48351513,1036.05424843)(408.51351807,1035.73425148)
\curveto(408.54351507,1035.41424907)(408.63351498,1035.14924934)(408.78351807,1034.93925148)
\curveto(408.92351469,1034.72924976)(409.14851446,1034.57924991)(409.45851807,1034.48925148)
\curveto(409.5085141,1034.47925001)(409.54851406,1034.47425001)(409.57851807,1034.47425148)
\curveto(409.61851399,1034.47425001)(409.66351395,1034.46925002)(409.71351807,1034.45925148)
\curveto(409.76351385,1034.44925004)(409.81851379,1034.44425004)(409.87851807,1034.44425148)
\curveto(409.93851367,1034.44425004)(409.98351363,1034.44925004)(410.01351807,1034.45925148)
\curveto(410.06351355,1034.47925001)(410.10351351,1034.48425)(410.13351807,1034.47425148)
\curveto(410.17351344,1034.46425002)(410.2135134,1034.46925002)(410.25351807,1034.48925148)
\curveto(410.46351315,1034.53924995)(410.62851298,1034.60424988)(410.74851807,1034.68425148)
\curveto(410.92851268,1034.79424969)(411.06851254,1034.93424955)(411.16851807,1035.10425148)
\curveto(411.27851233,1035.2842492)(411.35351226,1035.47924901)(411.39351807,1035.68925148)
\curveto(411.44351217,1035.90924858)(411.47351214,1036.14924834)(411.48351807,1036.40925148)
\curveto(411.49351212,1036.67924781)(411.49851211,1036.95924753)(411.49851807,1037.24925148)
\lineto(411.49851807,1039.06425148)
\lineto(411.49851807,1040.03925148)
\lineto(411.49851807,1040.30925148)
\curveto(411.49851211,1040.40924408)(411.51851209,1040.489244)(411.55851807,1040.54925148)
\curveto(411.608512,1040.63924385)(411.68351193,1040.6892438)(411.78351807,1040.69925148)
\curveto(411.88351173,1040.71924377)(412.00351161,1040.72924376)(412.14351807,1040.72925148)
\lineto(412.93851807,1040.72925148)
\lineto(413.22351807,1040.72925148)
\curveto(413.3135103,1040.72924376)(413.38851022,1040.70924378)(413.44851807,1040.66925148)
\curveto(413.52851008,1040.61924387)(413.57351004,1040.54424394)(413.58351807,1040.44425148)
\curveto(413.59351002,1040.34424414)(413.59851001,1040.22924426)(413.59851807,1040.09925148)
\lineto(413.59851807,1038.95925148)
\lineto(413.59851807,1034.74425148)
\lineto(413.59851807,1033.67925148)
\lineto(413.59851807,1033.37925148)
\curveto(413.59851001,1033.27925121)(413.57851003,1033.20425128)(413.53851807,1033.15425148)
\curveto(413.48851012,1033.07425141)(413.4135102,1033.02925146)(413.31351807,1033.01925148)
\curveto(413.2135104,1033.00925148)(413.1085105,1033.00425148)(412.99851807,1033.00425148)
\lineto(412.18851807,1033.00425148)
\curveto(412.07851153,1033.00425148)(411.97851163,1033.00925148)(411.88851807,1033.01925148)
\curveto(411.8085118,1033.02925146)(411.74351187,1033.06925142)(411.69351807,1033.13925148)
\curveto(411.67351194,1033.16925132)(411.65351196,1033.21425127)(411.63351807,1033.27425148)
\curveto(411.62351199,1033.33425115)(411.608512,1033.39425109)(411.58851807,1033.45425148)
\curveto(411.57851203,1033.51425097)(411.56351205,1033.56925092)(411.54351807,1033.61925148)
\curveto(411.52351209,1033.66925082)(411.49351212,1033.69925079)(411.45351807,1033.70925148)
\curveto(411.43351218,1033.72925076)(411.4085122,1033.73425075)(411.37851807,1033.72425148)
\curveto(411.34851226,1033.71425077)(411.32351229,1033.70425078)(411.30351807,1033.69425148)
\curveto(411.23351238,1033.65425083)(411.17351244,1033.60925088)(411.12351807,1033.55925148)
\curveto(411.07351254,1033.50925098)(411.01851259,1033.46425102)(410.95851807,1033.42425148)
\curveto(410.91851269,1033.39425109)(410.87851273,1033.35925113)(410.83851807,1033.31925148)
\curveto(410.8085128,1033.2892512)(410.76851284,1033.25925123)(410.71851807,1033.22925148)
\curveto(410.48851312,1033.0892514)(410.21851339,1032.97925151)(409.90851807,1032.89925148)
\curveto(409.83851377,1032.87925161)(409.76851384,1032.86925162)(409.69851807,1032.86925148)
\curveto(409.62851398,1032.85925163)(409.55351406,1032.84425164)(409.47351807,1032.82425148)
\curveto(409.43351418,1032.81425167)(409.38851422,1032.81425167)(409.33851807,1032.82425148)
\curveto(409.29851431,1032.82425166)(409.25851435,1032.81925167)(409.21851807,1032.80925148)
\curveto(409.18851442,1032.79925169)(409.12351449,1032.79925169)(409.02351807,1032.80925148)
\curveto(408.93351468,1032.80925168)(408.87351474,1032.81425167)(408.84351807,1032.82425148)
\curveto(408.79351482,1032.82425166)(408.74351487,1032.82925166)(408.69351807,1032.83925148)
\lineto(408.54351807,1032.83925148)
\curveto(408.42351519,1032.86925162)(408.3085153,1032.89425159)(408.19851807,1032.91425148)
\curveto(408.08851552,1032.93425155)(407.97851563,1032.96425152)(407.86851807,1033.00425148)
\curveto(407.81851579,1033.02425146)(407.77351584,1033.03925145)(407.73351807,1033.04925148)
\curveto(407.70351591,1033.06925142)(407.66351595,1033.0892514)(407.61351807,1033.10925148)
\curveto(407.26351635,1033.29925119)(406.98351663,1033.56425092)(406.77351807,1033.90425148)
\curveto(406.64351697,1034.11425037)(406.54851706,1034.36425012)(406.48851807,1034.65425148)
\curveto(406.42851718,1034.95424953)(406.38851722,1035.26924922)(406.36851807,1035.59925148)
\curveto(406.35851725,1035.93924855)(406.35351726,1036.2842482)(406.35351807,1036.63425148)
\curveto(406.36351725,1036.99424749)(406.36851724,1037.34924714)(406.36851807,1037.69925148)
\lineto(406.36851807,1039.73925148)
\curveto(406.36851724,1039.86924462)(406.36351725,1040.01924447)(406.35351807,1040.18925148)
\curveto(406.35351726,1040.36924412)(406.37851723,1040.49924399)(406.42851807,1040.57925148)
\curveto(406.45851715,1040.62924386)(406.51851709,1040.67424381)(406.60851807,1040.71425148)
\curveto(406.66851694,1040.71424377)(406.7135169,1040.71924377)(406.74351807,1040.72925148)
}
}
{
\newrgbcolor{curcolor}{0 0 0}
\pscustom[linestyle=none,fillstyle=solid,fillcolor=curcolor]
{
\newpath
\moveto(422.27976807,1033.60425148)
\curveto(422.29976022,1033.49425099)(422.30976021,1033.3842511)(422.30976807,1033.27425148)
\curveto(422.3197602,1033.16425132)(422.26976025,1033.0892514)(422.15976807,1033.04925148)
\curveto(422.09976042,1033.01925147)(422.02976049,1033.00425148)(421.94976807,1033.00425148)
\lineto(421.70976807,1033.00425148)
\lineto(420.89976807,1033.00425148)
\lineto(420.62976807,1033.00425148)
\curveto(420.54976197,1033.01425147)(420.48476203,1033.03925145)(420.43476807,1033.07925148)
\curveto(420.36476215,1033.11925137)(420.30976221,1033.17425131)(420.26976807,1033.24425148)
\curveto(420.23976228,1033.32425116)(420.19476232,1033.3892511)(420.13476807,1033.43925148)
\curveto(420.1147624,1033.45925103)(420.08976243,1033.47425101)(420.05976807,1033.48425148)
\curveto(420.02976249,1033.50425098)(419.98976253,1033.50925098)(419.93976807,1033.49925148)
\curveto(419.88976263,1033.47925101)(419.83976268,1033.45425103)(419.78976807,1033.42425148)
\curveto(419.74976277,1033.39425109)(419.70476281,1033.36925112)(419.65476807,1033.34925148)
\curveto(419.60476291,1033.30925118)(419.54976297,1033.27425121)(419.48976807,1033.24425148)
\lineto(419.30976807,1033.15425148)
\curveto(419.17976334,1033.09425139)(419.04476347,1033.04425144)(418.90476807,1033.00425148)
\curveto(418.76476375,1032.97425151)(418.6197639,1032.93925155)(418.46976807,1032.89925148)
\curveto(418.39976412,1032.87925161)(418.32976419,1032.86925162)(418.25976807,1032.86925148)
\curveto(418.19976432,1032.85925163)(418.13476438,1032.84925164)(418.06476807,1032.83925148)
\lineto(417.97476807,1032.83925148)
\curveto(417.94476457,1032.82925166)(417.9147646,1032.82425166)(417.88476807,1032.82425148)
\lineto(417.71976807,1032.82425148)
\curveto(417.6197649,1032.80425168)(417.519765,1032.80425168)(417.41976807,1032.82425148)
\lineto(417.28476807,1032.82425148)
\curveto(417.2147653,1032.84425164)(417.14476537,1032.85425163)(417.07476807,1032.85425148)
\curveto(417.0147655,1032.84425164)(416.95476556,1032.84925164)(416.89476807,1032.86925148)
\curveto(416.79476572,1032.8892516)(416.69976582,1032.90925158)(416.60976807,1032.92925148)
\curveto(416.519766,1032.93925155)(416.43476608,1032.96425152)(416.35476807,1033.00425148)
\curveto(416.06476645,1033.11425137)(415.8147667,1033.25425123)(415.60476807,1033.42425148)
\curveto(415.40476711,1033.60425088)(415.24476727,1033.83925065)(415.12476807,1034.12925148)
\curveto(415.09476742,1034.19925029)(415.06476745,1034.27425021)(415.03476807,1034.35425148)
\curveto(415.0147675,1034.43425005)(414.99476752,1034.51924997)(414.97476807,1034.60925148)
\curveto(414.95476756,1034.65924983)(414.94476757,1034.70924978)(414.94476807,1034.75925148)
\curveto(414.95476756,1034.80924968)(414.95476756,1034.85924963)(414.94476807,1034.90925148)
\curveto(414.93476758,1034.93924955)(414.92476759,1034.99924949)(414.91476807,1035.08925148)
\curveto(414.9147676,1035.1892493)(414.9197676,1035.25924923)(414.92976807,1035.29925148)
\curveto(414.94976757,1035.39924909)(414.95976756,1035.484249)(414.95976807,1035.55425148)
\lineto(415.04976807,1035.88425148)
\curveto(415.07976744,1036.00424848)(415.1197674,1036.10924838)(415.16976807,1036.19925148)
\curveto(415.33976718,1036.489248)(415.53476698,1036.70924778)(415.75476807,1036.85925148)
\curveto(415.97476654,1037.00924748)(416.25476626,1037.13924735)(416.59476807,1037.24925148)
\curveto(416.72476579,1037.29924719)(416.85976566,1037.33424715)(416.99976807,1037.35425148)
\curveto(417.13976538,1037.37424711)(417.27976524,1037.39924709)(417.41976807,1037.42925148)
\curveto(417.49976502,1037.44924704)(417.58476493,1037.45924703)(417.67476807,1037.45925148)
\curveto(417.76476475,1037.46924702)(417.85476466,1037.484247)(417.94476807,1037.50425148)
\curveto(418.0147645,1037.52424696)(418.08476443,1037.52924696)(418.15476807,1037.51925148)
\curveto(418.22476429,1037.51924697)(418.29976422,1037.52924696)(418.37976807,1037.54925148)
\curveto(418.44976407,1037.56924692)(418.519764,1037.57924691)(418.58976807,1037.57925148)
\curveto(418.65976386,1037.57924691)(418.73476378,1037.5892469)(418.81476807,1037.60925148)
\curveto(419.02476349,1037.65924683)(419.2147633,1037.69924679)(419.38476807,1037.72925148)
\curveto(419.56476295,1037.76924672)(419.72476279,1037.85924663)(419.86476807,1037.99925148)
\curveto(419.95476256,1038.0892464)(420.0147625,1038.1892463)(420.04476807,1038.29925148)
\curveto(420.05476246,1038.32924616)(420.05476246,1038.35424613)(420.04476807,1038.37425148)
\curveto(420.04476247,1038.39424609)(420.04976247,1038.41424607)(420.05976807,1038.43425148)
\curveto(420.06976245,1038.45424603)(420.07476244,1038.484246)(420.07476807,1038.52425148)
\lineto(420.07476807,1038.61425148)
\lineto(420.04476807,1038.73425148)
\curveto(420.04476247,1038.77424571)(420.03976248,1038.80924568)(420.02976807,1038.83925148)
\curveto(419.92976259,1039.13924535)(419.7197628,1039.34424514)(419.39976807,1039.45425148)
\curveto(419.30976321,1039.484245)(419.19976332,1039.50424498)(419.06976807,1039.51425148)
\curveto(418.94976357,1039.53424495)(418.82476369,1039.53924495)(418.69476807,1039.52925148)
\curveto(418.56476395,1039.52924496)(418.43976408,1039.51924497)(418.31976807,1039.49925148)
\curveto(418.19976432,1039.47924501)(418.09476442,1039.45424503)(418.00476807,1039.42425148)
\curveto(417.94476457,1039.40424508)(417.88476463,1039.37424511)(417.82476807,1039.33425148)
\curveto(417.77476474,1039.30424518)(417.72476479,1039.26924522)(417.67476807,1039.22925148)
\curveto(417.62476489,1039.1892453)(417.56976495,1039.13424535)(417.50976807,1039.06425148)
\curveto(417.45976506,1038.99424549)(417.42476509,1038.92924556)(417.40476807,1038.86925148)
\curveto(417.35476516,1038.76924572)(417.30976521,1038.67424581)(417.26976807,1038.58425148)
\curveto(417.23976528,1038.49424599)(417.16976535,1038.43424605)(417.05976807,1038.40425148)
\curveto(416.97976554,1038.3842461)(416.89476562,1038.37424611)(416.80476807,1038.37425148)
\lineto(416.53476807,1038.37425148)
\lineto(415.96476807,1038.37425148)
\curveto(415.9147666,1038.37424611)(415.86476665,1038.36924612)(415.81476807,1038.35925148)
\curveto(415.76476675,1038.35924613)(415.7197668,1038.36424612)(415.67976807,1038.37425148)
\lineto(415.54476807,1038.37425148)
\curveto(415.52476699,1038.3842461)(415.49976702,1038.3892461)(415.46976807,1038.38925148)
\curveto(415.43976708,1038.3892461)(415.4147671,1038.39924609)(415.39476807,1038.41925148)
\curveto(415.3147672,1038.43924605)(415.25976726,1038.50424598)(415.22976807,1038.61425148)
\curveto(415.2197673,1038.66424582)(415.2197673,1038.71424577)(415.22976807,1038.76425148)
\curveto(415.23976728,1038.81424567)(415.24976727,1038.85924563)(415.25976807,1038.89925148)
\curveto(415.28976723,1039.00924548)(415.3197672,1039.10924538)(415.34976807,1039.19925148)
\curveto(415.38976713,1039.29924519)(415.43476708,1039.3892451)(415.48476807,1039.46925148)
\lineto(415.57476807,1039.61925148)
\lineto(415.66476807,1039.76925148)
\curveto(415.74476677,1039.87924461)(415.84476667,1039.9842445)(415.96476807,1040.08425148)
\curveto(415.98476653,1040.09424439)(416.0147665,1040.11924437)(416.05476807,1040.15925148)
\curveto(416.10476641,1040.19924429)(416.14976637,1040.23424425)(416.18976807,1040.26425148)
\curveto(416.22976629,1040.29424419)(416.27476624,1040.32424416)(416.32476807,1040.35425148)
\curveto(416.49476602,1040.46424402)(416.67476584,1040.54924394)(416.86476807,1040.60925148)
\curveto(417.05476546,1040.67924381)(417.24976527,1040.74424374)(417.44976807,1040.80425148)
\curveto(417.56976495,1040.83424365)(417.69476482,1040.85424363)(417.82476807,1040.86425148)
\curveto(417.95476456,1040.87424361)(418.08476443,1040.89424359)(418.21476807,1040.92425148)
\curveto(418.25476426,1040.93424355)(418.3147642,1040.93424355)(418.39476807,1040.92425148)
\curveto(418.48476403,1040.91424357)(418.53976398,1040.91924357)(418.55976807,1040.93925148)
\curveto(418.96976355,1040.94924354)(419.35976316,1040.93424355)(419.72976807,1040.89425148)
\curveto(420.10976241,1040.85424363)(420.44976207,1040.77924371)(420.74976807,1040.66925148)
\curveto(421.05976146,1040.55924393)(421.32476119,1040.40924408)(421.54476807,1040.21925148)
\curveto(421.76476075,1040.03924445)(421.93476058,1039.80424468)(422.05476807,1039.51425148)
\curveto(422.12476039,1039.34424514)(422.16476035,1039.14924534)(422.17476807,1038.92925148)
\curveto(422.18476033,1038.70924578)(422.18976033,1038.484246)(422.18976807,1038.25425148)
\lineto(422.18976807,1034.90925148)
\lineto(422.18976807,1034.32425148)
\curveto(422.18976033,1034.13425035)(422.20976031,1033.95925053)(422.24976807,1033.79925148)
\curveto(422.25976026,1033.76925072)(422.26476025,1033.73425075)(422.26476807,1033.69425148)
\curveto(422.26476025,1033.66425082)(422.26976025,1033.63425085)(422.27976807,1033.60425148)
\moveto(420.07476807,1035.91425148)
\curveto(420.08476243,1035.96424852)(420.08976243,1036.01924847)(420.08976807,1036.07925148)
\curveto(420.08976243,1036.14924834)(420.08476243,1036.20924828)(420.07476807,1036.25925148)
\curveto(420.05476246,1036.31924817)(420.04476247,1036.37424811)(420.04476807,1036.42425148)
\curveto(420.04476247,1036.47424801)(420.02476249,1036.51424797)(419.98476807,1036.54425148)
\curveto(419.93476258,1036.5842479)(419.85976266,1036.60424788)(419.75976807,1036.60425148)
\curveto(419.7197628,1036.59424789)(419.68476283,1036.5842479)(419.65476807,1036.57425148)
\curveto(419.62476289,1036.57424791)(419.58976293,1036.56924792)(419.54976807,1036.55925148)
\curveto(419.47976304,1036.53924795)(419.40476311,1036.52424796)(419.32476807,1036.51425148)
\curveto(419.24476327,1036.50424798)(419.16476335,1036.489248)(419.08476807,1036.46925148)
\curveto(419.05476346,1036.45924803)(419.00976351,1036.45424803)(418.94976807,1036.45425148)
\curveto(418.8197637,1036.42424806)(418.68976383,1036.40424808)(418.55976807,1036.39425148)
\curveto(418.42976409,1036.3842481)(418.30476421,1036.35924813)(418.18476807,1036.31925148)
\curveto(418.10476441,1036.29924819)(418.02976449,1036.27924821)(417.95976807,1036.25925148)
\curveto(417.88976463,1036.24924824)(417.8197647,1036.22924826)(417.74976807,1036.19925148)
\curveto(417.53976498,1036.10924838)(417.35976516,1035.97424851)(417.20976807,1035.79425148)
\curveto(417.06976545,1035.61424887)(417.0197655,1035.36424912)(417.05976807,1035.04425148)
\curveto(417.07976544,1034.87424961)(417.13476538,1034.73424975)(417.22476807,1034.62425148)
\curveto(417.29476522,1034.51424997)(417.39976512,1034.42425006)(417.53976807,1034.35425148)
\curveto(417.67976484,1034.29425019)(417.82976469,1034.24925024)(417.98976807,1034.21925148)
\curveto(418.15976436,1034.1892503)(418.33476418,1034.17925031)(418.51476807,1034.18925148)
\curveto(418.70476381,1034.20925028)(418.87976364,1034.24425024)(419.03976807,1034.29425148)
\curveto(419.29976322,1034.37425011)(419.50476301,1034.49924999)(419.65476807,1034.66925148)
\curveto(419.80476271,1034.84924964)(419.9197626,1035.06924942)(419.99976807,1035.32925148)
\curveto(420.0197625,1035.39924909)(420.02976249,1035.46924902)(420.02976807,1035.53925148)
\curveto(420.03976248,1035.61924887)(420.05476246,1035.69924879)(420.07476807,1035.77925148)
\lineto(420.07476807,1035.91425148)
}
}
{
\newrgbcolor{curcolor}{0 0 0}
\pscustom[linestyle=none,fillstyle=solid,fillcolor=curcolor]
{
\newpath
\moveto(428.26804932,1040.93925148)
\curveto(428.378044,1040.93924355)(428.47304391,1040.92924356)(428.55304932,1040.90925148)
\curveto(428.64304374,1040.8892436)(428.71304367,1040.84424364)(428.76304932,1040.77425148)
\curveto(428.82304356,1040.69424379)(428.85304353,1040.55424393)(428.85304932,1040.35425148)
\lineto(428.85304932,1039.84425148)
\lineto(428.85304932,1039.46925148)
\curveto(428.86304352,1039.32924516)(428.84804353,1039.21924527)(428.80804932,1039.13925148)
\curveto(428.76804361,1039.06924542)(428.70804367,1039.02424546)(428.62804932,1039.00425148)
\curveto(428.55804382,1038.9842455)(428.47304391,1038.97424551)(428.37304932,1038.97425148)
\curveto(428.2830441,1038.97424551)(428.1830442,1038.97924551)(428.07304932,1038.98925148)
\curveto(427.97304441,1038.99924549)(427.8780445,1038.99424549)(427.78804932,1038.97425148)
\curveto(427.71804466,1038.95424553)(427.64804473,1038.93924555)(427.57804932,1038.92925148)
\curveto(427.50804487,1038.92924556)(427.44304494,1038.91924557)(427.38304932,1038.89925148)
\curveto(427.22304516,1038.84924564)(427.06304532,1038.77424571)(426.90304932,1038.67425148)
\curveto(426.74304564,1038.5842459)(426.61804576,1038.47924601)(426.52804932,1038.35925148)
\curveto(426.4780459,1038.27924621)(426.42304596,1038.19424629)(426.36304932,1038.10425148)
\curveto(426.31304607,1038.02424646)(426.26304612,1037.93924655)(426.21304932,1037.84925148)
\curveto(426.1830462,1037.76924672)(426.15304623,1037.6842468)(426.12304932,1037.59425148)
\lineto(426.06304932,1037.35425148)
\curveto(426.04304634,1037.2842472)(426.03304635,1037.20924728)(426.03304932,1037.12925148)
\curveto(426.03304635,1037.05924743)(426.02304636,1036.9892475)(426.00304932,1036.91925148)
\curveto(425.99304639,1036.87924761)(425.98804639,1036.83924765)(425.98804932,1036.79925148)
\curveto(425.99804638,1036.76924772)(425.99804638,1036.73924775)(425.98804932,1036.70925148)
\lineto(425.98804932,1036.46925148)
\curveto(425.96804641,1036.39924809)(425.96304642,1036.31924817)(425.97304932,1036.22925148)
\curveto(425.9830464,1036.14924834)(425.98804639,1036.06924842)(425.98804932,1035.98925148)
\lineto(425.98804932,1035.02925148)
\lineto(425.98804932,1033.75425148)
\curveto(425.98804639,1033.62425086)(425.9830464,1033.50425098)(425.97304932,1033.39425148)
\curveto(425.96304642,1033.2842512)(425.93304645,1033.19425129)(425.88304932,1033.12425148)
\curveto(425.86304652,1033.09425139)(425.82804655,1033.06925142)(425.77804932,1033.04925148)
\curveto(425.73804664,1033.03925145)(425.69304669,1033.02925146)(425.64304932,1033.01925148)
\lineto(425.56804932,1033.01925148)
\curveto(425.51804686,1033.00925148)(425.46304692,1033.00425148)(425.40304932,1033.00425148)
\lineto(425.23804932,1033.00425148)
\lineto(424.59304932,1033.00425148)
\curveto(424.53304785,1033.01425147)(424.46804791,1033.01925147)(424.39804932,1033.01925148)
\lineto(424.20304932,1033.01925148)
\curveto(424.15304823,1033.03925145)(424.10304828,1033.05425143)(424.05304932,1033.06425148)
\curveto(424.00304838,1033.0842514)(423.96804841,1033.11925137)(423.94804932,1033.16925148)
\curveto(423.90804847,1033.21925127)(423.8830485,1033.2892512)(423.87304932,1033.37925148)
\lineto(423.87304932,1033.67925148)
\lineto(423.87304932,1034.69925148)
\lineto(423.87304932,1038.92925148)
\lineto(423.87304932,1040.03925148)
\lineto(423.87304932,1040.32425148)
\curveto(423.87304851,1040.42424406)(423.89304849,1040.50424398)(423.93304932,1040.56425148)
\curveto(423.9830484,1040.64424384)(424.05804832,1040.69424379)(424.15804932,1040.71425148)
\curveto(424.25804812,1040.73424375)(424.378048,1040.74424374)(424.51804932,1040.74425148)
\lineto(425.28304932,1040.74425148)
\curveto(425.40304698,1040.74424374)(425.50804687,1040.73424375)(425.59804932,1040.71425148)
\curveto(425.68804669,1040.70424378)(425.75804662,1040.65924383)(425.80804932,1040.57925148)
\curveto(425.83804654,1040.52924396)(425.85304653,1040.45924403)(425.85304932,1040.36925148)
\lineto(425.88304932,1040.09925148)
\curveto(425.89304649,1040.01924447)(425.90804647,1039.94424454)(425.92804932,1039.87425148)
\curveto(425.95804642,1039.80424468)(426.00804637,1039.76924472)(426.07804932,1039.76925148)
\curveto(426.09804628,1039.7892447)(426.11804626,1039.79924469)(426.13804932,1039.79925148)
\curveto(426.15804622,1039.79924469)(426.1780462,1039.80924468)(426.19804932,1039.82925148)
\curveto(426.25804612,1039.87924461)(426.30804607,1039.93424455)(426.34804932,1039.99425148)
\curveto(426.39804598,1040.06424442)(426.45804592,1040.12424436)(426.52804932,1040.17425148)
\curveto(426.56804581,1040.20424428)(426.60304578,1040.23424425)(426.63304932,1040.26425148)
\curveto(426.66304572,1040.30424418)(426.69804568,1040.33924415)(426.73804932,1040.36925148)
\lineto(427.00804932,1040.54925148)
\curveto(427.10804527,1040.60924388)(427.20804517,1040.66424382)(427.30804932,1040.71425148)
\curveto(427.40804497,1040.75424373)(427.50804487,1040.7892437)(427.60804932,1040.81925148)
\lineto(427.93804932,1040.90925148)
\curveto(427.96804441,1040.91924357)(428.02304436,1040.91924357)(428.10304932,1040.90925148)
\curveto(428.19304419,1040.90924358)(428.24804413,1040.91924357)(428.26804932,1040.93925148)
}
}
{
\newrgbcolor{curcolor}{0 0 0}
\pscustom[linestyle=none,fillstyle=solid,fillcolor=curcolor]
{
\newpath
\moveto(431.77312744,1043.59425148)
\curveto(431.84312449,1043.51424097)(431.87812446,1043.39424109)(431.87812744,1043.23425148)
\lineto(431.87812744,1042.76925148)
\lineto(431.87812744,1042.36425148)
\curveto(431.87812446,1042.22424226)(431.84312449,1042.12924236)(431.77312744,1042.07925148)
\curveto(431.71312462,1042.02924246)(431.6331247,1041.99924249)(431.53312744,1041.98925148)
\curveto(431.44312489,1041.97924251)(431.34312499,1041.97424251)(431.23312744,1041.97425148)
\lineto(430.39312744,1041.97425148)
\curveto(430.28312605,1041.97424251)(430.18312615,1041.97924251)(430.09312744,1041.98925148)
\curveto(430.01312632,1041.99924249)(429.94312639,1042.02924246)(429.88312744,1042.07925148)
\curveto(429.84312649,1042.10924238)(429.81312652,1042.16424232)(429.79312744,1042.24425148)
\curveto(429.78312655,1042.33424215)(429.77312656,1042.42924206)(429.76312744,1042.52925148)
\lineto(429.76312744,1042.85925148)
\curveto(429.77312656,1042.96924152)(429.77812656,1043.06424142)(429.77812744,1043.14425148)
\lineto(429.77812744,1043.35425148)
\curveto(429.78812655,1043.42424106)(429.80812653,1043.484241)(429.83812744,1043.53425148)
\curveto(429.85812648,1043.57424091)(429.88312645,1043.60424088)(429.91312744,1043.62425148)
\lineto(430.03312744,1043.68425148)
\curveto(430.05312628,1043.6842408)(430.07812626,1043.6842408)(430.10812744,1043.68425148)
\curveto(430.1381262,1043.69424079)(430.16312617,1043.69924079)(430.18312744,1043.69925148)
\lineto(431.27812744,1043.69925148)
\curveto(431.37812496,1043.69924079)(431.47312486,1043.69424079)(431.56312744,1043.68425148)
\curveto(431.65312468,1043.67424081)(431.72312461,1043.64424084)(431.77312744,1043.59425148)
\moveto(431.87812744,1033.82925148)
\curveto(431.87812446,1033.62925086)(431.87312446,1033.45925103)(431.86312744,1033.31925148)
\curveto(431.85312448,1033.17925131)(431.76312457,1033.0842514)(431.59312744,1033.03425148)
\curveto(431.5331248,1033.01425147)(431.46812487,1033.00425148)(431.39812744,1033.00425148)
\curveto(431.32812501,1033.01425147)(431.25312508,1033.01925147)(431.17312744,1033.01925148)
\lineto(430.33312744,1033.01925148)
\curveto(430.24312609,1033.01925147)(430.15312618,1033.02425146)(430.06312744,1033.03425148)
\curveto(429.98312635,1033.04425144)(429.92312641,1033.07425141)(429.88312744,1033.12425148)
\curveto(429.82312651,1033.19425129)(429.78812655,1033.27925121)(429.77812744,1033.37925148)
\lineto(429.77812744,1033.72425148)
\lineto(429.77812744,1040.05425148)
\lineto(429.77812744,1040.35425148)
\curveto(429.77812656,1040.45424403)(429.79812654,1040.53424395)(429.83812744,1040.59425148)
\curveto(429.89812644,1040.66424382)(429.98312635,1040.70924378)(430.09312744,1040.72925148)
\curveto(430.11312622,1040.73924375)(430.1381262,1040.73924375)(430.16812744,1040.72925148)
\curveto(430.20812613,1040.72924376)(430.2381261,1040.73424375)(430.25812744,1040.74425148)
\lineto(431.00812744,1040.74425148)
\lineto(431.20312744,1040.74425148)
\curveto(431.28312505,1040.75424373)(431.34812499,1040.75424373)(431.39812744,1040.74425148)
\lineto(431.51812744,1040.74425148)
\curveto(431.57812476,1040.72424376)(431.6331247,1040.70924378)(431.68312744,1040.69925148)
\curveto(431.7331246,1040.6892438)(431.77312456,1040.65924383)(431.80312744,1040.60925148)
\curveto(431.84312449,1040.55924393)(431.86312447,1040.489244)(431.86312744,1040.39925148)
\curveto(431.87312446,1040.30924418)(431.87812446,1040.21424427)(431.87812744,1040.11425148)
\lineto(431.87812744,1033.82925148)
}
}
{
\newrgbcolor{curcolor}{0 0 0}
\pscustom[linestyle=none,fillstyle=solid,fillcolor=curcolor]
{
\newpath
\moveto(441.31031494,1037.18925148)
\curveto(441.33030637,1037.12924736)(441.34030636,1037.04424744)(441.34031494,1036.93425148)
\curveto(441.34030636,1036.82424766)(441.33030637,1036.73924775)(441.31031494,1036.67925148)
\lineto(441.31031494,1036.52925148)
\curveto(441.29030641,1036.44924804)(441.28030642,1036.36924812)(441.28031494,1036.28925148)
\curveto(441.29030641,1036.20924828)(441.28530642,1036.12924836)(441.26531494,1036.04925148)
\curveto(441.24530646,1035.97924851)(441.23030647,1035.91424857)(441.22031494,1035.85425148)
\curveto(441.21030649,1035.79424869)(441.2003065,1035.72924876)(441.19031494,1035.65925148)
\curveto(441.15030655,1035.54924894)(441.11530659,1035.43424905)(441.08531494,1035.31425148)
\curveto(441.05530665,1035.20424928)(441.01530669,1035.09924939)(440.96531494,1034.99925148)
\curveto(440.75530695,1034.51924997)(440.48030722,1034.12925036)(440.14031494,1033.82925148)
\curveto(439.8003079,1033.52925096)(439.39030831,1033.27925121)(438.91031494,1033.07925148)
\curveto(438.79030891,1033.02925146)(438.66530904,1032.99425149)(438.53531494,1032.97425148)
\curveto(438.41530929,1032.94425154)(438.29030941,1032.91425157)(438.16031494,1032.88425148)
\curveto(438.11030959,1032.86425162)(438.05530965,1032.85425163)(437.99531494,1032.85425148)
\curveto(437.93530977,1032.85425163)(437.88030982,1032.84925164)(437.83031494,1032.83925148)
\lineto(437.72531494,1032.83925148)
\curveto(437.69531001,1032.82925166)(437.66531004,1032.82425166)(437.63531494,1032.82425148)
\curveto(437.58531012,1032.81425167)(437.5053102,1032.80925168)(437.39531494,1032.80925148)
\curveto(437.28531042,1032.79925169)(437.2003105,1032.80425168)(437.14031494,1032.82425148)
\lineto(436.99031494,1032.82425148)
\curveto(436.94031076,1032.83425165)(436.88531082,1032.83925165)(436.82531494,1032.83925148)
\curveto(436.77531093,1032.82925166)(436.72531098,1032.83425165)(436.67531494,1032.85425148)
\curveto(436.63531107,1032.86425162)(436.59531111,1032.86925162)(436.55531494,1032.86925148)
\curveto(436.52531118,1032.86925162)(436.48531122,1032.87425161)(436.43531494,1032.88425148)
\curveto(436.33531137,1032.91425157)(436.23531147,1032.93925155)(436.13531494,1032.95925148)
\curveto(436.03531167,1032.97925151)(435.94031176,1033.00925148)(435.85031494,1033.04925148)
\curveto(435.73031197,1033.0892514)(435.61531209,1033.12925136)(435.50531494,1033.16925148)
\curveto(435.4053123,1033.20925128)(435.3003124,1033.25925123)(435.19031494,1033.31925148)
\curveto(434.84031286,1033.52925096)(434.54031316,1033.77425071)(434.29031494,1034.05425148)
\curveto(434.04031366,1034.33425015)(433.83031387,1034.66924982)(433.66031494,1035.05925148)
\curveto(433.61031409,1035.14924934)(433.57031413,1035.24424924)(433.54031494,1035.34425148)
\curveto(433.52031418,1035.44424904)(433.49531421,1035.54924894)(433.46531494,1035.65925148)
\curveto(433.44531426,1035.70924878)(433.43531427,1035.75424873)(433.43531494,1035.79425148)
\curveto(433.43531427,1035.83424865)(433.42531428,1035.87924861)(433.40531494,1035.92925148)
\curveto(433.38531432,1036.00924848)(433.37531433,1036.0892484)(433.37531494,1036.16925148)
\curveto(433.37531433,1036.25924823)(433.36531434,1036.34424814)(433.34531494,1036.42425148)
\curveto(433.33531437,1036.47424801)(433.33031437,1036.51924797)(433.33031494,1036.55925148)
\lineto(433.33031494,1036.69425148)
\curveto(433.31031439,1036.75424773)(433.3003144,1036.83924765)(433.30031494,1036.94925148)
\curveto(433.31031439,1037.05924743)(433.32531438,1037.14424734)(433.34531494,1037.20425148)
\lineto(433.34531494,1037.30925148)
\curveto(433.35531435,1037.35924713)(433.35531435,1037.40924708)(433.34531494,1037.45925148)
\curveto(433.34531436,1037.51924697)(433.35531435,1037.57424691)(433.37531494,1037.62425148)
\curveto(433.38531432,1037.67424681)(433.39031431,1037.71924677)(433.39031494,1037.75925148)
\curveto(433.39031431,1037.80924668)(433.4003143,1037.85924663)(433.42031494,1037.90925148)
\curveto(433.46031424,1038.03924645)(433.49531421,1038.16424632)(433.52531494,1038.28425148)
\curveto(433.55531415,1038.41424607)(433.59531411,1038.53924595)(433.64531494,1038.65925148)
\curveto(433.82531388,1039.06924542)(434.04031366,1039.40924508)(434.29031494,1039.67925148)
\curveto(434.54031316,1039.95924453)(434.84531286,1040.21424427)(435.20531494,1040.44425148)
\curveto(435.3053124,1040.49424399)(435.41031229,1040.53924395)(435.52031494,1040.57925148)
\curveto(435.63031207,1040.61924387)(435.74031196,1040.66424382)(435.85031494,1040.71425148)
\curveto(435.98031172,1040.76424372)(436.11531159,1040.79924369)(436.25531494,1040.81925148)
\curveto(436.39531131,1040.83924365)(436.54031116,1040.86924362)(436.69031494,1040.90925148)
\curveto(436.77031093,1040.91924357)(436.84531086,1040.92424356)(436.91531494,1040.92425148)
\curveto(436.98531072,1040.92424356)(437.05531065,1040.92924356)(437.12531494,1040.93925148)
\curveto(437.70531,1040.94924354)(438.2053095,1040.8892436)(438.62531494,1040.75925148)
\curveto(439.05530865,1040.62924386)(439.43530827,1040.44924404)(439.76531494,1040.21925148)
\curveto(439.87530783,1040.13924435)(439.98530772,1040.04924444)(440.09531494,1039.94925148)
\curveto(440.21530749,1039.85924463)(440.31530739,1039.75924473)(440.39531494,1039.64925148)
\curveto(440.47530723,1039.54924494)(440.54530716,1039.44924504)(440.60531494,1039.34925148)
\curveto(440.67530703,1039.24924524)(440.74530696,1039.14424534)(440.81531494,1039.03425148)
\curveto(440.88530682,1038.92424556)(440.94030676,1038.80424568)(440.98031494,1038.67425148)
\curveto(441.02030668,1038.55424593)(441.06530664,1038.42424606)(441.11531494,1038.28425148)
\curveto(441.14530656,1038.20424628)(441.17030653,1038.11924637)(441.19031494,1038.02925148)
\lineto(441.25031494,1037.75925148)
\curveto(441.26030644,1037.71924677)(441.26530644,1037.67924681)(441.26531494,1037.63925148)
\curveto(441.26530644,1037.59924689)(441.27030643,1037.55924693)(441.28031494,1037.51925148)
\curveto(441.3003064,1037.46924702)(441.3053064,1037.41424707)(441.29531494,1037.35425148)
\curveto(441.28530642,1037.29424719)(441.29030641,1037.23924725)(441.31031494,1037.18925148)
\moveto(439.21031494,1036.64925148)
\curveto(439.22030848,1036.69924779)(439.22530848,1036.76924772)(439.22531494,1036.85925148)
\curveto(439.22530848,1036.95924753)(439.22030848,1037.03424745)(439.21031494,1037.08425148)
\lineto(439.21031494,1037.20425148)
\curveto(439.19030851,1037.25424723)(439.18030852,1037.30924718)(439.18031494,1037.36925148)
\curveto(439.18030852,1037.42924706)(439.17530853,1037.484247)(439.16531494,1037.53425148)
\curveto(439.16530854,1037.57424691)(439.16030854,1037.60424688)(439.15031494,1037.62425148)
\lineto(439.09031494,1037.86425148)
\curveto(439.08030862,1037.95424653)(439.06030864,1038.03924645)(439.03031494,1038.11925148)
\curveto(438.92030878,1038.37924611)(438.79030891,1038.59924589)(438.64031494,1038.77925148)
\curveto(438.49030921,1038.96924552)(438.29030941,1039.11924537)(438.04031494,1039.22925148)
\curveto(437.98030972,1039.24924524)(437.92030978,1039.26424522)(437.86031494,1039.27425148)
\curveto(437.8003099,1039.29424519)(437.73530997,1039.31424517)(437.66531494,1039.33425148)
\curveto(437.58531012,1039.35424513)(437.5003102,1039.35924513)(437.41031494,1039.34925148)
\lineto(437.14031494,1039.34925148)
\curveto(437.11031059,1039.32924516)(437.07531063,1039.31924517)(437.03531494,1039.31925148)
\curveto(436.99531071,1039.32924516)(436.96031074,1039.32924516)(436.93031494,1039.31925148)
\lineto(436.72031494,1039.25925148)
\curveto(436.66031104,1039.24924524)(436.6053111,1039.22924526)(436.55531494,1039.19925148)
\curveto(436.3053114,1039.0892454)(436.1003116,1038.92924556)(435.94031494,1038.71925148)
\curveto(435.79031191,1038.51924597)(435.67031203,1038.2842462)(435.58031494,1038.01425148)
\curveto(435.55031215,1037.91424657)(435.52531218,1037.80924668)(435.50531494,1037.69925148)
\curveto(435.49531221,1037.5892469)(435.48031222,1037.47924701)(435.46031494,1037.36925148)
\curveto(435.45031225,1037.31924717)(435.44531226,1037.26924722)(435.44531494,1037.21925148)
\lineto(435.44531494,1037.06925148)
\curveto(435.42531228,1036.99924749)(435.41531229,1036.89424759)(435.41531494,1036.75425148)
\curveto(435.42531228,1036.61424787)(435.44031226,1036.50924798)(435.46031494,1036.43925148)
\lineto(435.46031494,1036.30425148)
\curveto(435.48031222,1036.22424826)(435.49531221,1036.14424834)(435.50531494,1036.06425148)
\curveto(435.51531219,1035.99424849)(435.53031217,1035.91924857)(435.55031494,1035.83925148)
\curveto(435.65031205,1035.53924895)(435.75531195,1035.29424919)(435.86531494,1035.10425148)
\curveto(435.98531172,1034.92424956)(436.17031153,1034.75924973)(436.42031494,1034.60925148)
\curveto(436.49031121,1034.55924993)(436.56531114,1034.51924997)(436.64531494,1034.48925148)
\curveto(436.73531097,1034.45925003)(436.82531088,1034.43425005)(436.91531494,1034.41425148)
\curveto(436.95531075,1034.40425008)(436.99031071,1034.39925009)(437.02031494,1034.39925148)
\curveto(437.05031065,1034.40925008)(437.08531062,1034.40925008)(437.12531494,1034.39925148)
\lineto(437.24531494,1034.36925148)
\curveto(437.29531041,1034.36925012)(437.34031036,1034.37425011)(437.38031494,1034.38425148)
\lineto(437.50031494,1034.38425148)
\curveto(437.58031012,1034.40425008)(437.66031004,1034.41925007)(437.74031494,1034.42925148)
\curveto(437.82030988,1034.43925005)(437.89530981,1034.45925003)(437.96531494,1034.48925148)
\curveto(438.22530948,1034.5892499)(438.43530927,1034.72424976)(438.59531494,1034.89425148)
\curveto(438.75530895,1035.06424942)(438.89030881,1035.27424921)(439.00031494,1035.52425148)
\curveto(439.04030866,1035.62424886)(439.07030863,1035.72424876)(439.09031494,1035.82425148)
\curveto(439.11030859,1035.92424856)(439.13530857,1036.02924846)(439.16531494,1036.13925148)
\curveto(439.17530853,1036.17924831)(439.18030852,1036.21424827)(439.18031494,1036.24425148)
\curveto(439.18030852,1036.2842482)(439.18530852,1036.32424816)(439.19531494,1036.36425148)
\lineto(439.19531494,1036.49925148)
\curveto(439.19530851,1036.54924794)(439.2003085,1036.59924789)(439.21031494,1036.64925148)
}
}
{
\newrgbcolor{curcolor}{0 0 0}
\pscustom[linestyle=none,fillstyle=solid,fillcolor=curcolor]
{
\newpath
\moveto(445.68023682,1040.95425148)
\curveto(446.43023232,1040.97424351)(447.08023167,1040.8892436)(447.63023682,1040.69925148)
\curveto(448.19023056,1040.51924397)(448.61523013,1040.20424428)(448.90523682,1039.75425148)
\curveto(448.97522977,1039.64424484)(449.03522971,1039.52924496)(449.08523682,1039.40925148)
\curveto(449.1452296,1039.29924519)(449.19522955,1039.17424531)(449.23523682,1039.03425148)
\curveto(449.25522949,1038.97424551)(449.26522948,1038.90924558)(449.26523682,1038.83925148)
\curveto(449.26522948,1038.76924572)(449.25522949,1038.70924578)(449.23523682,1038.65925148)
\curveto(449.19522955,1038.59924589)(449.14022961,1038.55924593)(449.07023682,1038.53925148)
\curveto(449.02022973,1038.51924597)(448.96022979,1038.50924598)(448.89023682,1038.50925148)
\lineto(448.68023682,1038.50925148)
\lineto(448.02023682,1038.50925148)
\curveto(447.9502308,1038.50924598)(447.88023087,1038.50424598)(447.81023682,1038.49425148)
\curveto(447.74023101,1038.49424599)(447.67523107,1038.50424598)(447.61523682,1038.52425148)
\curveto(447.51523123,1038.54424594)(447.44023131,1038.5842459)(447.39023682,1038.64425148)
\curveto(447.34023141,1038.70424578)(447.29523145,1038.76424572)(447.25523682,1038.82425148)
\lineto(447.13523682,1039.03425148)
\curveto(447.10523164,1039.11424537)(447.05523169,1039.17924531)(446.98523682,1039.22925148)
\curveto(446.88523186,1039.30924518)(446.78523196,1039.36924512)(446.68523682,1039.40925148)
\curveto(446.59523215,1039.44924504)(446.48023227,1039.484245)(446.34023682,1039.51425148)
\curveto(446.27023248,1039.53424495)(446.16523258,1039.54924494)(446.02523682,1039.55925148)
\curveto(445.89523285,1039.56924492)(445.79523295,1039.56424492)(445.72523682,1039.54425148)
\lineto(445.62023682,1039.54425148)
\lineto(445.47023682,1039.51425148)
\curveto(445.43023332,1039.51424497)(445.38523336,1039.50924498)(445.33523682,1039.49925148)
\curveto(445.16523358,1039.44924504)(445.02523372,1039.37924511)(444.91523682,1039.28925148)
\curveto(444.81523393,1039.20924528)(444.745234,1039.0842454)(444.70523682,1038.91425148)
\curveto(444.68523406,1038.84424564)(444.68523406,1038.77924571)(444.70523682,1038.71925148)
\curveto(444.72523402,1038.65924583)(444.745234,1038.60924588)(444.76523682,1038.56925148)
\curveto(444.83523391,1038.44924604)(444.91523383,1038.35424613)(445.00523682,1038.28425148)
\curveto(445.10523364,1038.21424627)(445.22023353,1038.15424633)(445.35023682,1038.10425148)
\curveto(445.54023321,1038.02424646)(445.745233,1037.95424653)(445.96523682,1037.89425148)
\lineto(446.65523682,1037.74425148)
\curveto(446.89523185,1037.70424678)(447.12523162,1037.65424683)(447.34523682,1037.59425148)
\curveto(447.57523117,1037.54424694)(447.79023096,1037.47924701)(447.99023682,1037.39925148)
\curveto(448.08023067,1037.35924713)(448.16523058,1037.32424716)(448.24523682,1037.29425148)
\curveto(448.33523041,1037.27424721)(448.42023033,1037.23924725)(448.50023682,1037.18925148)
\curveto(448.69023006,1037.06924742)(448.86022989,1036.93924755)(449.01023682,1036.79925148)
\curveto(449.17022958,1036.65924783)(449.29522945,1036.484248)(449.38523682,1036.27425148)
\curveto(449.41522933,1036.20424828)(449.44022931,1036.13424835)(449.46023682,1036.06425148)
\curveto(449.48022927,1035.99424849)(449.50022925,1035.91924857)(449.52023682,1035.83925148)
\curveto(449.53022922,1035.77924871)(449.53522921,1035.6842488)(449.53523682,1035.55425148)
\curveto(449.5452292,1035.43424905)(449.5452292,1035.33924915)(449.53523682,1035.26925148)
\lineto(449.53523682,1035.19425148)
\curveto(449.51522923,1035.13424935)(449.50022925,1035.07424941)(449.49023682,1035.01425148)
\curveto(449.49022926,1034.96424952)(449.48522926,1034.91424957)(449.47523682,1034.86425148)
\curveto(449.40522934,1034.56424992)(449.29522945,1034.29925019)(449.14523682,1034.06925148)
\curveto(448.98522976,1033.82925066)(448.79022996,1033.63425085)(448.56023682,1033.48425148)
\curveto(448.33023042,1033.33425115)(448.07023068,1033.20425128)(447.78023682,1033.09425148)
\curveto(447.67023108,1033.04425144)(447.5502312,1033.00925148)(447.42023682,1032.98925148)
\curveto(447.30023145,1032.96925152)(447.18023157,1032.94425154)(447.06023682,1032.91425148)
\curveto(446.97023178,1032.89425159)(446.87523187,1032.8842516)(446.77523682,1032.88425148)
\curveto(446.68523206,1032.87425161)(446.59523215,1032.85925163)(446.50523682,1032.83925148)
\lineto(446.23523682,1032.83925148)
\curveto(446.17523257,1032.81925167)(446.07023268,1032.80925168)(445.92023682,1032.80925148)
\curveto(445.78023297,1032.80925168)(445.68023307,1032.81925167)(445.62023682,1032.83925148)
\curveto(445.59023316,1032.83925165)(445.55523319,1032.84425164)(445.51523682,1032.85425148)
\lineto(445.41023682,1032.85425148)
\curveto(445.29023346,1032.87425161)(445.17023358,1032.8892516)(445.05023682,1032.89925148)
\curveto(444.93023382,1032.90925158)(444.81523393,1032.92925156)(444.70523682,1032.95925148)
\curveto(444.31523443,1033.06925142)(443.97023478,1033.19425129)(443.67023682,1033.33425148)
\curveto(443.37023538,1033.484251)(443.11523563,1033.70425078)(442.90523682,1033.99425148)
\curveto(442.76523598,1034.1842503)(442.6452361,1034.40425008)(442.54523682,1034.65425148)
\curveto(442.52523622,1034.71424977)(442.50523624,1034.79424969)(442.48523682,1034.89425148)
\curveto(442.46523628,1034.94424954)(442.4502363,1035.01424947)(442.44023682,1035.10425148)
\curveto(442.43023632,1035.19424929)(442.43523631,1035.26924922)(442.45523682,1035.32925148)
\curveto(442.48523626,1035.39924909)(442.53523621,1035.44924904)(442.60523682,1035.47925148)
\curveto(442.65523609,1035.49924899)(442.71523603,1035.50924898)(442.78523682,1035.50925148)
\lineto(443.01023682,1035.50925148)
\lineto(443.71523682,1035.50925148)
\lineto(443.95523682,1035.50925148)
\curveto(444.03523471,1035.50924898)(444.10523464,1035.49924899)(444.16523682,1035.47925148)
\curveto(444.27523447,1035.43924905)(444.3452344,1035.37424911)(444.37523682,1035.28425148)
\curveto(444.41523433,1035.19424929)(444.46023429,1035.09924939)(444.51023682,1034.99925148)
\curveto(444.53023422,1034.94924954)(444.56523418,1034.8842496)(444.61523682,1034.80425148)
\curveto(444.67523407,1034.72424976)(444.72523402,1034.67424981)(444.76523682,1034.65425148)
\curveto(444.88523386,1034.55424993)(445.00023375,1034.47425001)(445.11023682,1034.41425148)
\curveto(445.22023353,1034.36425012)(445.36023339,1034.31425017)(445.53023682,1034.26425148)
\curveto(445.58023317,1034.24425024)(445.63023312,1034.23425025)(445.68023682,1034.23425148)
\curveto(445.73023302,1034.24425024)(445.78023297,1034.24425024)(445.83023682,1034.23425148)
\curveto(445.91023284,1034.21425027)(445.99523275,1034.20425028)(446.08523682,1034.20425148)
\curveto(446.18523256,1034.21425027)(446.27023248,1034.22925026)(446.34023682,1034.24925148)
\curveto(446.39023236,1034.25925023)(446.43523231,1034.26425022)(446.47523682,1034.26425148)
\curveto(446.52523222,1034.26425022)(446.57523217,1034.27425021)(446.62523682,1034.29425148)
\curveto(446.76523198,1034.34425014)(446.89023186,1034.40425008)(447.00023682,1034.47425148)
\curveto(447.12023163,1034.54424994)(447.21523153,1034.63424985)(447.28523682,1034.74425148)
\curveto(447.33523141,1034.82424966)(447.37523137,1034.94924954)(447.40523682,1035.11925148)
\curveto(447.42523132,1035.1892493)(447.42523132,1035.25424923)(447.40523682,1035.31425148)
\curveto(447.38523136,1035.37424911)(447.36523138,1035.42424906)(447.34523682,1035.46425148)
\curveto(447.27523147,1035.60424888)(447.18523156,1035.70924878)(447.07523682,1035.77925148)
\curveto(446.97523177,1035.84924864)(446.85523189,1035.91424857)(446.71523682,1035.97425148)
\curveto(446.52523222,1036.05424843)(446.32523242,1036.11924837)(446.11523682,1036.16925148)
\curveto(445.90523284,1036.21924827)(445.69523305,1036.27424821)(445.48523682,1036.33425148)
\curveto(445.40523334,1036.35424813)(445.32023343,1036.36924812)(445.23023682,1036.37925148)
\curveto(445.1502336,1036.3892481)(445.07023368,1036.40424808)(444.99023682,1036.42425148)
\curveto(444.67023408,1036.51424797)(444.36523438,1036.59924789)(444.07523682,1036.67925148)
\curveto(443.78523496,1036.76924772)(443.52023523,1036.89924759)(443.28023682,1037.06925148)
\curveto(443.00023575,1037.26924722)(442.79523595,1037.53924695)(442.66523682,1037.87925148)
\curveto(442.6452361,1037.94924654)(442.62523612,1038.04424644)(442.60523682,1038.16425148)
\curveto(442.58523616,1038.23424625)(442.57023618,1038.31924617)(442.56023682,1038.41925148)
\curveto(442.5502362,1038.51924597)(442.55523619,1038.60924588)(442.57523682,1038.68925148)
\curveto(442.59523615,1038.73924575)(442.60023615,1038.77924571)(442.59023682,1038.80925148)
\curveto(442.58023617,1038.84924564)(442.58523616,1038.89424559)(442.60523682,1038.94425148)
\curveto(442.62523612,1039.05424543)(442.6452361,1039.15424533)(442.66523682,1039.24425148)
\curveto(442.69523605,1039.34424514)(442.73023602,1039.43924505)(442.77023682,1039.52925148)
\curveto(442.90023585,1039.81924467)(443.08023567,1040.05424443)(443.31023682,1040.23425148)
\curveto(443.54023521,1040.41424407)(443.80023495,1040.55924393)(444.09023682,1040.66925148)
\curveto(444.20023455,1040.71924377)(444.31523443,1040.75424373)(444.43523682,1040.77425148)
\curveto(444.55523419,1040.80424368)(444.68023407,1040.83424365)(444.81023682,1040.86425148)
\curveto(444.87023388,1040.8842436)(444.93023382,1040.89424359)(444.99023682,1040.89425148)
\lineto(445.17023682,1040.92425148)
\curveto(445.2502335,1040.93424355)(445.33523341,1040.93924355)(445.42523682,1040.93925148)
\curveto(445.51523323,1040.93924355)(445.60023315,1040.94424354)(445.68023682,1040.95425148)
}
}
{
\newrgbcolor{curcolor}{0 0 0}
\pscustom[linestyle=none,fillstyle=solid,fillcolor=curcolor]
{
}
}
{
\newrgbcolor{curcolor}{0 0 0}
\pscustom[linestyle=none,fillstyle=solid,fillcolor=curcolor]
{
\newpath
\moveto(462.48703369,1031.15925148)
\curveto(462.48702535,1030.99925349)(462.48202536,1030.84425364)(462.47203369,1030.69425148)
\curveto(462.47202537,1030.53425395)(462.42202542,1030.42425406)(462.32203369,1030.36425148)
\curveto(462.2420256,1030.31425417)(462.12702571,1030.29425419)(461.97703369,1030.30425148)
\lineto(461.55703369,1030.30425148)
\lineto(461.24203369,1030.30425148)
\curveto(461.13202671,1030.29425419)(461.02202682,1030.29425419)(460.91203369,1030.30425148)
\curveto(460.81202703,1030.30425418)(460.71702712,1030.31925417)(460.62703369,1030.34925148)
\curveto(460.54702729,1030.36925412)(460.48702735,1030.40925408)(460.44703369,1030.46925148)
\curveto(460.39702744,1030.54925394)(460.37202747,1030.66425382)(460.37203369,1030.81425148)
\curveto(460.38202746,1030.95425353)(460.38702745,1031.0842534)(460.38703369,1031.20425148)
\lineto(460.38703369,1032.83925148)
\lineto(460.38703369,1033.21425148)
\curveto(460.38702745,1033.35425113)(460.37202747,1033.45925103)(460.34203369,1033.52925148)
\curveto(460.32202752,1033.54925094)(460.30202754,1033.56425092)(460.28203369,1033.57425148)
\curveto(460.27202757,1033.59425089)(460.25702758,1033.61425087)(460.23703369,1033.63425148)
\curveto(460.14702769,1033.64425084)(460.07702776,1033.62425086)(460.02703369,1033.57425148)
\curveto(459.97702786,1033.53425095)(459.92202792,1033.49425099)(459.86203369,1033.45425148)
\curveto(459.77202807,1033.3842511)(459.67702816,1033.31925117)(459.57703369,1033.25925148)
\curveto(459.48702835,1033.19925129)(459.38702845,1033.14425134)(459.27703369,1033.09425148)
\curveto(459.09702874,1033.01425147)(458.89702894,1032.95425153)(458.67703369,1032.91425148)
\curveto(458.45702938,1032.86425162)(458.23202961,1032.83925165)(458.00203369,1032.83925148)
\curveto(457.77203007,1032.82925166)(457.5420303,1032.84425164)(457.31203369,1032.88425148)
\curveto(457.09203075,1032.92425156)(456.89203095,1032.9842515)(456.71203369,1033.06425148)
\curveto(456.26203158,1033.26425122)(455.89703194,1033.51925097)(455.61703369,1033.82925148)
\curveto(455.3370325,1034.14925034)(455.10203274,1034.53924995)(454.91203369,1034.99925148)
\curveto(454.86203298,1035.10924938)(454.82703301,1035.21924927)(454.80703369,1035.32925148)
\curveto(454.78703305,1035.44924904)(454.76203308,1035.56424892)(454.73203369,1035.67425148)
\curveto(454.71203313,1035.71424877)(454.70203314,1035.74924874)(454.70203369,1035.77925148)
\curveto(454.71203313,1035.81924867)(454.71203313,1035.85924863)(454.70203369,1035.89925148)
\curveto(454.68203316,1035.97924851)(454.66703317,1036.06424842)(454.65703369,1036.15425148)
\curveto(454.65703318,1036.25424823)(454.64703319,1036.34924814)(454.62703369,1036.43925148)
\lineto(454.62703369,1036.63425148)
\curveto(454.61703322,1036.6842478)(454.61203323,1036.74424774)(454.61203369,1036.81425148)
\curveto(454.61203323,1036.89424759)(454.61703322,1036.95924753)(454.62703369,1037.00925148)
\curveto(454.6370332,1037.05924743)(454.6420332,1037.10424738)(454.64203369,1037.14425148)
\lineto(454.64203369,1037.27925148)
\curveto(454.65203319,1037.32924716)(454.65203319,1037.37924711)(454.64203369,1037.42925148)
\curveto(454.6420332,1037.47924701)(454.65203319,1037.52924696)(454.67203369,1037.57925148)
\curveto(454.69203315,1037.66924682)(454.70703313,1037.75924673)(454.71703369,1037.84925148)
\curveto(454.72703311,1037.94924654)(454.7420331,1038.04424644)(454.76203369,1038.13425148)
\curveto(454.81203303,1038.30424618)(454.86203298,1038.46424602)(454.91203369,1038.61425148)
\curveto(454.97203287,1038.76424572)(455.03203281,1038.90924558)(455.09203369,1039.04925148)
\curveto(455.15203269,1039.1892453)(455.22703261,1039.32424516)(455.31703369,1039.45425148)
\curveto(455.40703243,1039.5842449)(455.49703234,1039.70924478)(455.58703369,1039.82925148)
\curveto(455.67703216,1039.93924455)(455.77703206,1040.03924445)(455.88703369,1040.12925148)
\curveto(455.91703192,1040.15924433)(455.9370319,1040.1842443)(455.94703369,1040.20425148)
\curveto(455.99703184,1040.23424425)(456.0420318,1040.26424422)(456.08203369,1040.29425148)
\curveto(456.12203172,1040.33424415)(456.16203168,1040.36924412)(456.20203369,1040.39925148)
\curveto(456.3420315,1040.49924399)(456.48703135,1040.57924391)(456.63703369,1040.63925148)
\curveto(456.79703104,1040.70924378)(456.96203088,1040.77424371)(457.13203369,1040.83425148)
\curveto(457.22203062,1040.86424362)(457.31203053,1040.8842436)(457.40203369,1040.89425148)
\curveto(457.49203035,1040.90424358)(457.58203026,1040.91924357)(457.67203369,1040.93925148)
\curveto(457.70203014,1040.94924354)(457.75703008,1040.94924354)(457.83703369,1040.93925148)
\curveto(457.91702992,1040.92924356)(457.96702987,1040.93424355)(457.98703369,1040.95425148)
\curveto(458.30702953,1040.96424352)(458.60702923,1040.93424355)(458.88703369,1040.86425148)
\curveto(459.16702867,1040.80424368)(459.40702843,1040.71424377)(459.60703369,1040.59425148)
\lineto(459.78703369,1040.47425148)
\curveto(459.84702799,1040.43424405)(459.90202794,1040.39424409)(459.95203369,1040.35425148)
\curveto(460.01202783,1040.30424418)(460.06202778,1040.25424423)(460.10203369,1040.20425148)
\curveto(460.15202769,1040.16424432)(460.23202761,1040.14424434)(460.34203369,1040.14425148)
\lineto(460.38703369,1040.18925148)
\lineto(460.44703369,1040.24925148)
\curveto(460.47702736,1040.32924416)(460.49702734,1040.40424408)(460.50703369,1040.47425148)
\curveto(460.51702732,1040.55424393)(460.55702728,1040.61924387)(460.62703369,1040.66925148)
\curveto(460.67702716,1040.70924378)(460.74702709,1040.72924376)(460.83703369,1040.72925148)
\curveto(460.9370269,1040.73924375)(461.0370268,1040.74424374)(461.13703369,1040.74425148)
\lineto(461.85703369,1040.74425148)
\lineto(462.06703369,1040.74425148)
\curveto(462.1370257,1040.74424374)(462.20202564,1040.73424375)(462.26203369,1040.71425148)
\curveto(462.33202551,1040.69424379)(462.38702545,1040.64924384)(462.42703369,1040.57925148)
\curveto(462.47702536,1040.50924398)(462.49702534,1040.41424407)(462.48703369,1040.29425148)
\lineto(462.48703369,1039.94925148)
\lineto(462.48703369,1031.15925148)
\moveto(460.44703369,1036.76925148)
\curveto(460.45702738,1036.7892477)(460.45702738,1036.81424767)(460.44703369,1036.84425148)
\lineto(460.44703369,1036.91925148)
\curveto(460.4370274,1037.01924747)(460.43202741,1037.11424737)(460.43203369,1037.20425148)
\curveto(460.43202741,1037.29424719)(460.42202742,1037.37924711)(460.40203369,1037.45925148)
\curveto(460.39202745,1037.489247)(460.38702745,1037.51424697)(460.38703369,1037.53425148)
\curveto(460.39702744,1037.56424692)(460.39702744,1037.59424689)(460.38703369,1037.62425148)
\curveto(460.36702747,1037.70424678)(460.34702749,1037.77424671)(460.32703369,1037.83425148)
\curveto(460.31702752,1037.90424658)(460.30202754,1037.97424651)(460.28203369,1038.04425148)
\curveto(460.18202766,1038.33424615)(460.04702779,1038.5842459)(459.87703369,1038.79425148)
\curveto(459.70702813,1039.00424548)(459.48702835,1039.16424532)(459.21703369,1039.27425148)
\curveto(459.10702873,1039.32424516)(458.98702885,1039.34924514)(458.85703369,1039.34925148)
\curveto(458.7370291,1039.35924513)(458.60702923,1039.36424512)(458.46703369,1039.36425148)
\curveto(458.4370294,1039.34424514)(458.40202944,1039.33424515)(458.36203369,1039.33425148)
\curveto(458.32202952,1039.34424514)(458.28202956,1039.34424514)(458.24203369,1039.33425148)
\lineto(458.06203369,1039.27425148)
\curveto(458.00202984,1039.26424522)(457.94702989,1039.24924524)(457.89703369,1039.22925148)
\curveto(457.60703023,1039.09924539)(457.37703046,1038.90924558)(457.20703369,1038.65925148)
\curveto(457.04703079,1038.40924608)(456.92203092,1038.11924637)(456.83203369,1037.78925148)
\curveto(456.81203103,1037.70924678)(456.79703104,1037.63424685)(456.78703369,1037.56425148)
\curveto(456.78703105,1037.50424698)(456.77703106,1037.43424705)(456.75703369,1037.35425148)
\curveto(456.75703108,1037.2842472)(456.75203109,1037.23424725)(456.74203369,1037.20425148)
\curveto(456.73203111,1037.15424733)(456.72203112,1037.06424742)(456.71203369,1036.93425148)
\curveto(456.71203113,1036.81424767)(456.72203112,1036.72924776)(456.74203369,1036.67925148)
\lineto(456.74203369,1036.54425148)
\curveto(456.75203109,1036.50424798)(456.75703108,1036.46424802)(456.75703369,1036.42425148)
\curveto(456.75703108,1036.3842481)(456.76203108,1036.34924814)(456.77203369,1036.31925148)
\lineto(456.77203369,1036.24425148)
\curveto(456.78203106,1036.21424827)(456.78703105,1036.1892483)(456.78703369,1036.16925148)
\curveto(456.80703103,1036.0892484)(456.82203102,1036.01424847)(456.83203369,1035.94425148)
\curveto(456.842031,1035.87424861)(456.86203098,1035.80424868)(456.89203369,1035.73425148)
\curveto(456.97203087,1035.484249)(457.07703076,1035.26924922)(457.20703369,1035.08925148)
\curveto(457.3370305,1034.90924958)(457.50203034,1034.75424973)(457.70203369,1034.62425148)
\curveto(457.84203,1034.54424994)(457.99702984,1034.48425)(458.16703369,1034.44425148)
\curveto(458.19702964,1034.43425005)(458.22202962,1034.42925006)(458.24203369,1034.42925148)
\curveto(458.27202957,1034.42925006)(458.30702953,1034.42425006)(458.34703369,1034.41425148)
\curveto(458.37702946,1034.40425008)(458.42202942,1034.39425009)(458.48203369,1034.38425148)
\curveto(458.55202929,1034.3842501)(458.61202923,1034.3892501)(458.66203369,1034.39925148)
\curveto(458.68202916,1034.40925008)(458.70702913,1034.40925008)(458.73703369,1034.39925148)
\curveto(458.77702906,1034.39925009)(458.81202903,1034.40425008)(458.84203369,1034.41425148)
\curveto(458.91202893,1034.43425005)(458.97702886,1034.44925004)(459.03703369,1034.45925148)
\curveto(459.10702873,1034.46925002)(459.17702866,1034.48425)(459.24703369,1034.50425148)
\curveto(459.50702833,1034.61424987)(459.71202813,1034.75924973)(459.86203369,1034.93925148)
\curveto(460.02202782,1035.11924937)(460.15702768,1035.33924915)(460.26703369,1035.59925148)
\curveto(460.29702754,1035.67924881)(460.32202752,1035.76424872)(460.34203369,1035.85425148)
\lineto(460.40203369,1036.12425148)
\lineto(460.40203369,1036.22925148)
\curveto(460.41202743,1036.25924823)(460.41702742,1036.29424819)(460.41703369,1036.33425148)
\curveto(460.4370274,1036.43424805)(460.44702739,1036.51924797)(460.44703369,1036.58925148)
\lineto(460.44703369,1036.76925148)
}
}
{
\newrgbcolor{curcolor}{0 0 0}
\pscustom[linestyle=none,fillstyle=solid,fillcolor=curcolor]
{
\newpath
\moveto(464.51695557,1040.72925148)
\lineto(465.64195557,1040.72925148)
\curveto(465.75195313,1040.72924376)(465.85195303,1040.72424376)(465.94195557,1040.71425148)
\curveto(466.03195285,1040.70424378)(466.09695279,1040.66924382)(466.13695557,1040.60925148)
\curveto(466.1869527,1040.54924394)(466.21695267,1040.46424402)(466.22695557,1040.35425148)
\curveto(466.23695265,1040.25424423)(466.24195264,1040.14924434)(466.24195557,1040.03925148)
\lineto(466.24195557,1038.98925148)
\lineto(466.24195557,1036.75425148)
\curveto(466.24195264,1036.39424809)(466.25695263,1036.05424843)(466.28695557,1035.73425148)
\curveto(466.31695257,1035.41424907)(466.40695248,1035.14924934)(466.55695557,1034.93925148)
\curveto(466.69695219,1034.72924976)(466.92195196,1034.57924991)(467.23195557,1034.48925148)
\curveto(467.2819516,1034.47925001)(467.32195156,1034.47425001)(467.35195557,1034.47425148)
\curveto(467.39195149,1034.47425001)(467.43695145,1034.46925002)(467.48695557,1034.45925148)
\curveto(467.53695135,1034.44925004)(467.59195129,1034.44425004)(467.65195557,1034.44425148)
\curveto(467.71195117,1034.44425004)(467.75695113,1034.44925004)(467.78695557,1034.45925148)
\curveto(467.83695105,1034.47925001)(467.87695101,1034.48425)(467.90695557,1034.47425148)
\curveto(467.94695094,1034.46425002)(467.9869509,1034.46925002)(468.02695557,1034.48925148)
\curveto(468.23695065,1034.53924995)(468.40195048,1034.60424988)(468.52195557,1034.68425148)
\curveto(468.70195018,1034.79424969)(468.84195004,1034.93424955)(468.94195557,1035.10425148)
\curveto(469.05194983,1035.2842492)(469.12694976,1035.47924901)(469.16695557,1035.68925148)
\curveto(469.21694967,1035.90924858)(469.24694964,1036.14924834)(469.25695557,1036.40925148)
\curveto(469.26694962,1036.67924781)(469.27194961,1036.95924753)(469.27195557,1037.24925148)
\lineto(469.27195557,1039.06425148)
\lineto(469.27195557,1040.03925148)
\lineto(469.27195557,1040.30925148)
\curveto(469.27194961,1040.40924408)(469.29194959,1040.489244)(469.33195557,1040.54925148)
\curveto(469.3819495,1040.63924385)(469.45694943,1040.6892438)(469.55695557,1040.69925148)
\curveto(469.65694923,1040.71924377)(469.77694911,1040.72924376)(469.91695557,1040.72925148)
\lineto(470.71195557,1040.72925148)
\lineto(470.99695557,1040.72925148)
\curveto(471.0869478,1040.72924376)(471.16194772,1040.70924378)(471.22195557,1040.66925148)
\curveto(471.30194758,1040.61924387)(471.34694754,1040.54424394)(471.35695557,1040.44425148)
\curveto(471.36694752,1040.34424414)(471.37194751,1040.22924426)(471.37195557,1040.09925148)
\lineto(471.37195557,1038.95925148)
\lineto(471.37195557,1034.74425148)
\lineto(471.37195557,1033.67925148)
\lineto(471.37195557,1033.37925148)
\curveto(471.37194751,1033.27925121)(471.35194753,1033.20425128)(471.31195557,1033.15425148)
\curveto(471.26194762,1033.07425141)(471.1869477,1033.02925146)(471.08695557,1033.01925148)
\curveto(470.9869479,1033.00925148)(470.881948,1033.00425148)(470.77195557,1033.00425148)
\lineto(469.96195557,1033.00425148)
\curveto(469.85194903,1033.00425148)(469.75194913,1033.00925148)(469.66195557,1033.01925148)
\curveto(469.5819493,1033.02925146)(469.51694937,1033.06925142)(469.46695557,1033.13925148)
\curveto(469.44694944,1033.16925132)(469.42694946,1033.21425127)(469.40695557,1033.27425148)
\curveto(469.39694949,1033.33425115)(469.3819495,1033.39425109)(469.36195557,1033.45425148)
\curveto(469.35194953,1033.51425097)(469.33694955,1033.56925092)(469.31695557,1033.61925148)
\curveto(469.29694959,1033.66925082)(469.26694962,1033.69925079)(469.22695557,1033.70925148)
\curveto(469.20694968,1033.72925076)(469.1819497,1033.73425075)(469.15195557,1033.72425148)
\curveto(469.12194976,1033.71425077)(469.09694979,1033.70425078)(469.07695557,1033.69425148)
\curveto(469.00694988,1033.65425083)(468.94694994,1033.60925088)(468.89695557,1033.55925148)
\curveto(468.84695004,1033.50925098)(468.79195009,1033.46425102)(468.73195557,1033.42425148)
\curveto(468.69195019,1033.39425109)(468.65195023,1033.35925113)(468.61195557,1033.31925148)
\curveto(468.5819503,1033.2892512)(468.54195034,1033.25925123)(468.49195557,1033.22925148)
\curveto(468.26195062,1033.0892514)(467.99195089,1032.97925151)(467.68195557,1032.89925148)
\curveto(467.61195127,1032.87925161)(467.54195134,1032.86925162)(467.47195557,1032.86925148)
\curveto(467.40195148,1032.85925163)(467.32695156,1032.84425164)(467.24695557,1032.82425148)
\curveto(467.20695168,1032.81425167)(467.16195172,1032.81425167)(467.11195557,1032.82425148)
\curveto(467.07195181,1032.82425166)(467.03195185,1032.81925167)(466.99195557,1032.80925148)
\curveto(466.96195192,1032.79925169)(466.89695199,1032.79925169)(466.79695557,1032.80925148)
\curveto(466.70695218,1032.80925168)(466.64695224,1032.81425167)(466.61695557,1032.82425148)
\curveto(466.56695232,1032.82425166)(466.51695237,1032.82925166)(466.46695557,1032.83925148)
\lineto(466.31695557,1032.83925148)
\curveto(466.19695269,1032.86925162)(466.0819528,1032.89425159)(465.97195557,1032.91425148)
\curveto(465.86195302,1032.93425155)(465.75195313,1032.96425152)(465.64195557,1033.00425148)
\curveto(465.59195329,1033.02425146)(465.54695334,1033.03925145)(465.50695557,1033.04925148)
\curveto(465.47695341,1033.06925142)(465.43695345,1033.0892514)(465.38695557,1033.10925148)
\curveto(465.03695385,1033.29925119)(464.75695413,1033.56425092)(464.54695557,1033.90425148)
\curveto(464.41695447,1034.11425037)(464.32195456,1034.36425012)(464.26195557,1034.65425148)
\curveto(464.20195468,1034.95424953)(464.16195472,1035.26924922)(464.14195557,1035.59925148)
\curveto(464.13195475,1035.93924855)(464.12695476,1036.2842482)(464.12695557,1036.63425148)
\curveto(464.13695475,1036.99424749)(464.14195474,1037.34924714)(464.14195557,1037.69925148)
\lineto(464.14195557,1039.73925148)
\curveto(464.14195474,1039.86924462)(464.13695475,1040.01924447)(464.12695557,1040.18925148)
\curveto(464.12695476,1040.36924412)(464.15195473,1040.49924399)(464.20195557,1040.57925148)
\curveto(464.23195465,1040.62924386)(464.29195459,1040.67424381)(464.38195557,1040.71425148)
\curveto(464.44195444,1040.71424377)(464.4869544,1040.71924377)(464.51695557,1040.72925148)
}
}
{
\newrgbcolor{curcolor}{0 0 0}
\pscustom[linestyle=none,fillstyle=solid,fillcolor=curcolor]
{
\newpath
\moveto(480.36820557,1036.94925148)
\curveto(480.3881974,1036.86924762)(480.3881974,1036.77924771)(480.36820557,1036.67925148)
\curveto(480.34819744,1036.57924791)(480.31319748,1036.51424797)(480.26320557,1036.48425148)
\curveto(480.21319758,1036.44424804)(480.13819765,1036.41424807)(480.03820557,1036.39425148)
\curveto(479.94819784,1036.3842481)(479.84319795,1036.37424811)(479.72320557,1036.36425148)
\lineto(479.37820557,1036.36425148)
\curveto(479.26819852,1036.37424811)(479.16819862,1036.37924811)(479.07820557,1036.37925148)
\lineto(475.41820557,1036.37925148)
\lineto(475.20820557,1036.37925148)
\curveto(475.14820264,1036.37924811)(475.0932027,1036.36924812)(475.04320557,1036.34925148)
\curveto(474.96320283,1036.30924818)(474.91320288,1036.26924822)(474.89320557,1036.22925148)
\curveto(474.87320292,1036.20924828)(474.85320294,1036.16924832)(474.83320557,1036.10925148)
\curveto(474.81320298,1036.05924843)(474.80820298,1036.00924848)(474.81820557,1035.95925148)
\curveto(474.83820295,1035.89924859)(474.84820294,1035.83924865)(474.84820557,1035.77925148)
\curveto(474.85820293,1035.72924876)(474.87320292,1035.67424881)(474.89320557,1035.61425148)
\curveto(474.97320282,1035.37424911)(475.06820272,1035.17424931)(475.17820557,1035.01425148)
\curveto(475.29820249,1034.86424962)(475.45820233,1034.72924976)(475.65820557,1034.60925148)
\curveto(475.73820205,1034.55924993)(475.81820197,1034.52424996)(475.89820557,1034.50425148)
\curveto(475.9882018,1034.49424999)(476.07820171,1034.47425001)(476.16820557,1034.44425148)
\curveto(476.24820154,1034.42425006)(476.35820143,1034.40925008)(476.49820557,1034.39925148)
\curveto(476.63820115,1034.3892501)(476.75820103,1034.39425009)(476.85820557,1034.41425148)
\lineto(476.99320557,1034.41425148)
\curveto(477.0932007,1034.43425005)(477.18320061,1034.45425003)(477.26320557,1034.47425148)
\curveto(477.35320044,1034.50424998)(477.43820035,1034.53424995)(477.51820557,1034.56425148)
\curveto(477.61820017,1034.61424987)(477.72820006,1034.67924981)(477.84820557,1034.75925148)
\curveto(477.97819981,1034.83924965)(478.07319972,1034.91924957)(478.13320557,1034.99925148)
\curveto(478.18319961,1035.06924942)(478.23319956,1035.13424935)(478.28320557,1035.19425148)
\curveto(478.34319945,1035.26424922)(478.41319938,1035.31424917)(478.49320557,1035.34425148)
\curveto(478.5931992,1035.39424909)(478.71819907,1035.41424907)(478.86820557,1035.40425148)
\lineto(479.30320557,1035.40425148)
\lineto(479.48320557,1035.40425148)
\curveto(479.55319824,1035.41424907)(479.61319818,1035.40924908)(479.66320557,1035.38925148)
\lineto(479.81320557,1035.38925148)
\curveto(479.91319788,1035.36924912)(479.98319781,1035.34424914)(480.02320557,1035.31425148)
\curveto(480.06319773,1035.29424919)(480.08319771,1035.24924924)(480.08320557,1035.17925148)
\curveto(480.0931977,1035.10924938)(480.0881977,1035.04924944)(480.06820557,1034.99925148)
\curveto(480.01819777,1034.85924963)(479.96319783,1034.73424975)(479.90320557,1034.62425148)
\curveto(479.84319795,1034.51424997)(479.77319802,1034.40425008)(479.69320557,1034.29425148)
\curveto(479.47319832,1033.96425052)(479.22319857,1033.69925079)(478.94320557,1033.49925148)
\curveto(478.66319913,1033.29925119)(478.31319948,1033.12925136)(477.89320557,1032.98925148)
\curveto(477.78320001,1032.94925154)(477.67320012,1032.92425156)(477.56320557,1032.91425148)
\curveto(477.45320034,1032.90425158)(477.33820045,1032.8842516)(477.21820557,1032.85425148)
\curveto(477.17820061,1032.84425164)(477.13320066,1032.84425164)(477.08320557,1032.85425148)
\curveto(477.04320075,1032.85425163)(477.00320079,1032.84925164)(476.96320557,1032.83925148)
\lineto(476.79820557,1032.83925148)
\curveto(476.74820104,1032.81925167)(476.6882011,1032.81425167)(476.61820557,1032.82425148)
\curveto(476.55820123,1032.82425166)(476.50320129,1032.82925166)(476.45320557,1032.83925148)
\curveto(476.37320142,1032.84925164)(476.30320149,1032.84925164)(476.24320557,1032.83925148)
\curveto(476.18320161,1032.82925166)(476.11820167,1032.83425165)(476.04820557,1032.85425148)
\curveto(475.99820179,1032.87425161)(475.94320185,1032.8842516)(475.88320557,1032.88425148)
\curveto(475.82320197,1032.8842516)(475.76820202,1032.89425159)(475.71820557,1032.91425148)
\curveto(475.60820218,1032.93425155)(475.49820229,1032.95925153)(475.38820557,1032.98925148)
\curveto(475.27820251,1033.00925148)(475.17820261,1033.04425144)(475.08820557,1033.09425148)
\curveto(474.97820281,1033.13425135)(474.87320292,1033.16925132)(474.77320557,1033.19925148)
\curveto(474.68320311,1033.23925125)(474.59820319,1033.2842512)(474.51820557,1033.33425148)
\curveto(474.19820359,1033.53425095)(473.91320388,1033.76425072)(473.66320557,1034.02425148)
\curveto(473.41320438,1034.29425019)(473.20820458,1034.60424988)(473.04820557,1034.95425148)
\curveto(472.99820479,1035.06424942)(472.95820483,1035.17424931)(472.92820557,1035.28425148)
\curveto(472.89820489,1035.40424908)(472.85820493,1035.52424896)(472.80820557,1035.64425148)
\curveto(472.79820499,1035.6842488)(472.793205,1035.71924877)(472.79320557,1035.74925148)
\curveto(472.793205,1035.7892487)(472.788205,1035.82924866)(472.77820557,1035.86925148)
\curveto(472.73820505,1035.9892485)(472.71320508,1036.11924837)(472.70320557,1036.25925148)
\lineto(472.67320557,1036.67925148)
\curveto(472.67320512,1036.72924776)(472.66820512,1036.7842477)(472.65820557,1036.84425148)
\curveto(472.65820513,1036.90424758)(472.66320513,1036.95924753)(472.67320557,1037.00925148)
\lineto(472.67320557,1037.18925148)
\lineto(472.71820557,1037.54925148)
\curveto(472.75820503,1037.71924677)(472.793205,1037.8842466)(472.82320557,1038.04425148)
\curveto(472.85320494,1038.20424628)(472.89820489,1038.35424613)(472.95820557,1038.49425148)
\curveto(473.3882044,1039.53424495)(474.11820367,1040.26924422)(475.14820557,1040.69925148)
\curveto(475.2882025,1040.75924373)(475.42820236,1040.79924369)(475.56820557,1040.81925148)
\curveto(475.71820207,1040.84924364)(475.87320192,1040.8842436)(476.03320557,1040.92425148)
\curveto(476.11320168,1040.93424355)(476.1882016,1040.93924355)(476.25820557,1040.93925148)
\curveto(476.32820146,1040.93924355)(476.40320139,1040.94424354)(476.48320557,1040.95425148)
\curveto(476.9932008,1040.96424352)(477.42820036,1040.90424358)(477.78820557,1040.77425148)
\curveto(478.15819963,1040.65424383)(478.4881993,1040.49424399)(478.77820557,1040.29425148)
\curveto(478.86819892,1040.23424425)(478.95819883,1040.16424432)(479.04820557,1040.08425148)
\curveto(479.13819865,1040.01424447)(479.21819857,1039.93924455)(479.28820557,1039.85925148)
\curveto(479.31819847,1039.80924468)(479.35819843,1039.76924472)(479.40820557,1039.73925148)
\curveto(479.4881983,1039.62924486)(479.56319823,1039.51424497)(479.63320557,1039.39425148)
\curveto(479.70319809,1039.2842452)(479.77819801,1039.16924532)(479.85820557,1039.04925148)
\curveto(479.90819788,1038.95924553)(479.94819784,1038.86424562)(479.97820557,1038.76425148)
\curveto(480.01819777,1038.67424581)(480.05819773,1038.57424591)(480.09820557,1038.46425148)
\curveto(480.14819764,1038.33424615)(480.1881976,1038.19924629)(480.21820557,1038.05925148)
\curveto(480.24819754,1037.91924657)(480.28319751,1037.77924671)(480.32320557,1037.63925148)
\curveto(480.34319745,1037.55924693)(480.34819744,1037.46924702)(480.33820557,1037.36925148)
\curveto(480.33819745,1037.27924721)(480.34819744,1037.19424729)(480.36820557,1037.11425148)
\lineto(480.36820557,1036.94925148)
\moveto(478.11820557,1037.83425148)
\curveto(478.1881996,1037.93424655)(478.1931996,1038.05424643)(478.13320557,1038.19425148)
\curveto(478.08319971,1038.34424614)(478.04319975,1038.45424603)(478.01320557,1038.52425148)
\curveto(477.87319992,1038.79424569)(477.6882001,1038.99924549)(477.45820557,1039.13925148)
\curveto(477.22820056,1039.2892452)(476.90820088,1039.36924512)(476.49820557,1039.37925148)
\curveto(476.46820132,1039.35924513)(476.43320136,1039.35424513)(476.39320557,1039.36425148)
\curveto(476.35320144,1039.37424511)(476.31820147,1039.37424511)(476.28820557,1039.36425148)
\curveto(476.23820155,1039.34424514)(476.18320161,1039.32924516)(476.12320557,1039.31925148)
\curveto(476.06320173,1039.31924517)(476.00820178,1039.30924518)(475.95820557,1039.28925148)
\curveto(475.51820227,1039.14924534)(475.1932026,1038.87424561)(474.98320557,1038.46425148)
\curveto(474.96320283,1038.42424606)(474.93820285,1038.36924612)(474.90820557,1038.29925148)
\curveto(474.8882029,1038.23924625)(474.87320292,1038.17424631)(474.86320557,1038.10425148)
\curveto(474.85320294,1038.04424644)(474.85320294,1037.9842465)(474.86320557,1037.92425148)
\curveto(474.88320291,1037.86424662)(474.91820287,1037.81424667)(474.96820557,1037.77425148)
\curveto(475.04820274,1037.72424676)(475.15820263,1037.69924679)(475.29820557,1037.69925148)
\lineto(475.70320557,1037.69925148)
\lineto(477.36820557,1037.69925148)
\lineto(477.80320557,1037.69925148)
\curveto(477.96319983,1037.70924678)(478.06819972,1037.75424673)(478.11820557,1037.83425148)
}
}
{
\newrgbcolor{curcolor}{0 0 0}
\pscustom[linestyle=none,fillstyle=solid,fillcolor=curcolor]
{
}
}
{
\newrgbcolor{curcolor}{0 0 0}
\pscustom[linestyle=none,fillstyle=solid,fillcolor=curcolor]
{
\newpath
\moveto(487.87664307,1043.59425148)
\curveto(487.94664012,1043.51424097)(487.98164008,1043.39424109)(487.98164307,1043.23425148)
\lineto(487.98164307,1042.76925148)
\lineto(487.98164307,1042.36425148)
\curveto(487.98164008,1042.22424226)(487.94664012,1042.12924236)(487.87664307,1042.07925148)
\curveto(487.81664025,1042.02924246)(487.73664033,1041.99924249)(487.63664307,1041.98925148)
\curveto(487.54664052,1041.97924251)(487.44664062,1041.97424251)(487.33664307,1041.97425148)
\lineto(486.49664307,1041.97425148)
\curveto(486.38664168,1041.97424251)(486.28664178,1041.97924251)(486.19664307,1041.98925148)
\curveto(486.11664195,1041.99924249)(486.04664202,1042.02924246)(485.98664307,1042.07925148)
\curveto(485.94664212,1042.10924238)(485.91664215,1042.16424232)(485.89664307,1042.24425148)
\curveto(485.88664218,1042.33424215)(485.87664219,1042.42924206)(485.86664307,1042.52925148)
\lineto(485.86664307,1042.85925148)
\curveto(485.87664219,1042.96924152)(485.88164218,1043.06424142)(485.88164307,1043.14425148)
\lineto(485.88164307,1043.35425148)
\curveto(485.89164217,1043.42424106)(485.91164215,1043.484241)(485.94164307,1043.53425148)
\curveto(485.9616421,1043.57424091)(485.98664208,1043.60424088)(486.01664307,1043.62425148)
\lineto(486.13664307,1043.68425148)
\curveto(486.15664191,1043.6842408)(486.18164188,1043.6842408)(486.21164307,1043.68425148)
\curveto(486.24164182,1043.69424079)(486.2666418,1043.69924079)(486.28664307,1043.69925148)
\lineto(487.38164307,1043.69925148)
\curveto(487.48164058,1043.69924079)(487.57664049,1043.69424079)(487.66664307,1043.68425148)
\curveto(487.75664031,1043.67424081)(487.82664024,1043.64424084)(487.87664307,1043.59425148)
\moveto(487.98164307,1033.82925148)
\curveto(487.98164008,1033.62925086)(487.97664009,1033.45925103)(487.96664307,1033.31925148)
\curveto(487.95664011,1033.17925131)(487.8666402,1033.0842514)(487.69664307,1033.03425148)
\curveto(487.63664043,1033.01425147)(487.57164049,1033.00425148)(487.50164307,1033.00425148)
\curveto(487.43164063,1033.01425147)(487.35664071,1033.01925147)(487.27664307,1033.01925148)
\lineto(486.43664307,1033.01925148)
\curveto(486.34664172,1033.01925147)(486.25664181,1033.02425146)(486.16664307,1033.03425148)
\curveto(486.08664198,1033.04425144)(486.02664204,1033.07425141)(485.98664307,1033.12425148)
\curveto(485.92664214,1033.19425129)(485.89164217,1033.27925121)(485.88164307,1033.37925148)
\lineto(485.88164307,1033.72425148)
\lineto(485.88164307,1040.05425148)
\lineto(485.88164307,1040.35425148)
\curveto(485.88164218,1040.45424403)(485.90164216,1040.53424395)(485.94164307,1040.59425148)
\curveto(486.00164206,1040.66424382)(486.08664198,1040.70924378)(486.19664307,1040.72925148)
\curveto(486.21664185,1040.73924375)(486.24164182,1040.73924375)(486.27164307,1040.72925148)
\curveto(486.31164175,1040.72924376)(486.34164172,1040.73424375)(486.36164307,1040.74425148)
\lineto(487.11164307,1040.74425148)
\lineto(487.30664307,1040.74425148)
\curveto(487.38664068,1040.75424373)(487.45164061,1040.75424373)(487.50164307,1040.74425148)
\lineto(487.62164307,1040.74425148)
\curveto(487.68164038,1040.72424376)(487.73664033,1040.70924378)(487.78664307,1040.69925148)
\curveto(487.83664023,1040.6892438)(487.87664019,1040.65924383)(487.90664307,1040.60925148)
\curveto(487.94664012,1040.55924393)(487.9666401,1040.489244)(487.96664307,1040.39925148)
\curveto(487.97664009,1040.30924418)(487.98164008,1040.21424427)(487.98164307,1040.11425148)
\lineto(487.98164307,1033.82925148)
}
}
{
\newrgbcolor{curcolor}{0 0 0}
\pscustom[linestyle=none,fillstyle=solid,fillcolor=curcolor]
{
\newpath
\moveto(494.06883057,1040.93925148)
\curveto(494.66882476,1040.95924353)(495.16882426,1040.87424361)(495.56883057,1040.68425148)
\curveto(495.96882346,1040.49424399)(496.28382315,1040.21424427)(496.51383057,1039.84425148)
\curveto(496.58382285,1039.73424475)(496.63882279,1039.61424487)(496.67883057,1039.48425148)
\curveto(496.71882271,1039.36424512)(496.75882267,1039.23924525)(496.79883057,1039.10925148)
\curveto(496.81882261,1039.02924546)(496.8288226,1038.95424553)(496.82883057,1038.88425148)
\curveto(496.83882259,1038.81424567)(496.85382258,1038.74424574)(496.87383057,1038.67425148)
\curveto(496.87382256,1038.61424587)(496.87882255,1038.57424591)(496.88883057,1038.55425148)
\curveto(496.90882252,1038.41424607)(496.91882251,1038.26924622)(496.91883057,1038.11925148)
\lineto(496.91883057,1037.68425148)
\lineto(496.91883057,1036.34925148)
\lineto(496.91883057,1033.91925148)
\curveto(496.91882251,1033.72925076)(496.91382252,1033.54425094)(496.90383057,1033.36425148)
\curveto(496.90382253,1033.19425129)(496.8338226,1033.0842514)(496.69383057,1033.03425148)
\curveto(496.6338228,1033.01425147)(496.56382287,1033.00425148)(496.48383057,1033.00425148)
\lineto(496.24383057,1033.00425148)
\lineto(495.43383057,1033.00425148)
\curveto(495.31382412,1033.00425148)(495.20382423,1033.00925148)(495.10383057,1033.01925148)
\curveto(495.01382442,1033.03925145)(494.94382449,1033.0842514)(494.89383057,1033.15425148)
\curveto(494.85382458,1033.21425127)(494.8288246,1033.2892512)(494.81883057,1033.37925148)
\lineto(494.81883057,1033.69425148)
\lineto(494.81883057,1034.74425148)
\lineto(494.81883057,1036.97925148)
\curveto(494.81882461,1037.34924714)(494.80382463,1037.6892468)(494.77383057,1037.99925148)
\curveto(494.74382469,1038.31924617)(494.65382478,1038.5892459)(494.50383057,1038.80925148)
\curveto(494.36382507,1039.00924548)(494.15882527,1039.14924534)(493.88883057,1039.22925148)
\curveto(493.83882559,1039.24924524)(493.78382565,1039.25924523)(493.72383057,1039.25925148)
\curveto(493.67382576,1039.25924523)(493.61882581,1039.26924522)(493.55883057,1039.28925148)
\curveto(493.50882592,1039.29924519)(493.44382599,1039.29924519)(493.36383057,1039.28925148)
\curveto(493.29382614,1039.2892452)(493.23882619,1039.2842452)(493.19883057,1039.27425148)
\curveto(493.15882627,1039.26424522)(493.12382631,1039.25924523)(493.09383057,1039.25925148)
\curveto(493.06382637,1039.25924523)(493.0338264,1039.25424523)(493.00383057,1039.24425148)
\curveto(492.77382666,1039.1842453)(492.58882684,1039.10424538)(492.44883057,1039.00425148)
\curveto(492.1288273,1038.77424571)(491.93882749,1038.43924605)(491.87883057,1037.99925148)
\curveto(491.81882761,1037.55924693)(491.78882764,1037.06424742)(491.78883057,1036.51425148)
\lineto(491.78883057,1034.63925148)
\lineto(491.78883057,1033.72425148)
\lineto(491.78883057,1033.45425148)
\curveto(491.78882764,1033.36425112)(491.77382766,1033.2892512)(491.74383057,1033.22925148)
\curveto(491.69382774,1033.11925137)(491.61382782,1033.05425143)(491.50383057,1033.03425148)
\curveto(491.39382804,1033.01425147)(491.25882817,1033.00425148)(491.09883057,1033.00425148)
\lineto(490.34883057,1033.00425148)
\curveto(490.23882919,1033.00425148)(490.1288293,1033.00925148)(490.01883057,1033.01925148)
\curveto(489.90882952,1033.02925146)(489.8288296,1033.06425142)(489.77883057,1033.12425148)
\curveto(489.70882972,1033.21425127)(489.67382976,1033.34425114)(489.67383057,1033.51425148)
\curveto(489.68382975,1033.6842508)(489.68882974,1033.84425064)(489.68883057,1033.99425148)
\lineto(489.68883057,1036.03425148)
\lineto(489.68883057,1039.33425148)
\lineto(489.68883057,1040.09925148)
\lineto(489.68883057,1040.39925148)
\curveto(489.69882973,1040.489244)(489.7288297,1040.56424392)(489.77883057,1040.62425148)
\curveto(489.79882963,1040.65424383)(489.8288296,1040.67424381)(489.86883057,1040.68425148)
\curveto(489.91882951,1040.70424378)(489.96882946,1040.71924377)(490.01883057,1040.72925148)
\lineto(490.09383057,1040.72925148)
\curveto(490.14382929,1040.73924375)(490.19382924,1040.74424374)(490.24383057,1040.74425148)
\lineto(490.40883057,1040.74425148)
\lineto(491.03883057,1040.74425148)
\curveto(491.11882831,1040.74424374)(491.19382824,1040.73924375)(491.26383057,1040.72925148)
\curveto(491.34382809,1040.72924376)(491.41382802,1040.71924377)(491.47383057,1040.69925148)
\curveto(491.54382789,1040.66924382)(491.58882784,1040.62424386)(491.60883057,1040.56425148)
\curveto(491.63882779,1040.50424398)(491.66382777,1040.43424405)(491.68383057,1040.35425148)
\curveto(491.69382774,1040.31424417)(491.69382774,1040.27924421)(491.68383057,1040.24925148)
\curveto(491.68382775,1040.21924427)(491.69382774,1040.1892443)(491.71383057,1040.15925148)
\curveto(491.7338277,1040.10924438)(491.74882768,1040.07924441)(491.75883057,1040.06925148)
\curveto(491.77882765,1040.05924443)(491.80382763,1040.04424444)(491.83383057,1040.02425148)
\curveto(491.94382749,1040.01424447)(492.0338274,1040.04924444)(492.10383057,1040.12925148)
\curveto(492.17382726,1040.21924427)(492.24882718,1040.2892442)(492.32883057,1040.33925148)
\curveto(492.59882683,1040.53924395)(492.89882653,1040.69924379)(493.22883057,1040.81925148)
\curveto(493.31882611,1040.84924364)(493.40882602,1040.86924362)(493.49883057,1040.87925148)
\curveto(493.59882583,1040.8892436)(493.70382573,1040.90424358)(493.81383057,1040.92425148)
\curveto(493.84382559,1040.93424355)(493.88882554,1040.93424355)(493.94883057,1040.92425148)
\curveto(494.00882542,1040.92424356)(494.04882538,1040.92924356)(494.06883057,1040.93925148)
}
}
{
\newrgbcolor{curcolor}{0 0 0}
\pscustom[linestyle=none,fillstyle=solid,fillcolor=curcolor]
{
\newpath
\moveto(505.90008057,1040.65425148)
\curveto(505.97007237,1040.60424388)(506.00507233,1040.51924397)(506.00508057,1040.39925148)
\curveto(506.01507232,1040.2892442)(506.02007232,1040.17424431)(506.02008057,1040.05425148)
\lineto(506.02008057,1033.64925148)
\curveto(506.02007232,1033.56925092)(506.01507232,1033.489251)(506.00508057,1033.40925148)
\lineto(506.00508057,1033.18425148)
\curveto(505.99507234,1033.10425138)(505.98507235,1033.03425145)(505.97508057,1032.97425148)
\curveto(505.97507236,1032.90425158)(505.97007237,1032.82925166)(505.96008057,1032.74925148)
\curveto(505.92007242,1032.60925188)(505.88507245,1032.47925201)(505.85508057,1032.35925148)
\curveto(505.8350725,1032.22925226)(505.80007254,1032.10925238)(505.75008057,1031.99925148)
\curveto(505.58007276,1031.61925287)(505.36007298,1031.30425318)(505.09008057,1031.05425148)
\curveto(504.83007351,1030.80425368)(504.51007383,1030.59925389)(504.13008057,1030.43925148)
\curveto(504.02007432,1030.3892541)(503.91007443,1030.34925414)(503.80008057,1030.31925148)
\curveto(503.69007465,1030.2892542)(503.57507476,1030.25925423)(503.45508057,1030.22925148)
\curveto(503.34507499,1030.19925429)(503.2350751,1030.17925431)(503.12508057,1030.16925148)
\curveto(503.01507532,1030.15925433)(502.90507543,1030.14425434)(502.79508057,1030.12425148)
\lineto(502.67508057,1030.12425148)
\curveto(502.6350757,1030.11425437)(502.59007575,1030.10925438)(502.54008057,1030.10925148)
\curveto(502.50007584,1030.09925439)(502.45507588,1030.09925439)(502.40508057,1030.10925148)
\curveto(502.35507598,1030.10925438)(502.30507603,1030.10425438)(502.25508057,1030.09425148)
\curveto(502.20507613,1030.0842544)(502.1400762,1030.07925441)(502.06008057,1030.07925148)
\curveto(501.98007636,1030.07925441)(501.91507642,1030.0842544)(501.86508057,1030.09425148)
\lineto(501.73008057,1030.09425148)
\curveto(501.69007665,1030.09425439)(501.65007669,1030.09925439)(501.61008057,1030.10925148)
\curveto(501.53007681,1030.12925436)(501.44507689,1030.13925435)(501.35508057,1030.13925148)
\curveto(501.27507706,1030.13925435)(501.20007714,1030.14925434)(501.13008057,1030.16925148)
\curveto(501.11007723,1030.17925431)(501.08507725,1030.1842543)(501.05508057,1030.18425148)
\curveto(501.02507731,1030.1842543)(501.00007734,1030.1892543)(500.98008057,1030.19925148)
\curveto(500.88007746,1030.21925427)(500.78007756,1030.24425424)(500.68008057,1030.27425148)
\curveto(500.59007775,1030.29425419)(500.50007784,1030.32425416)(500.41008057,1030.36425148)
\curveto(500.03007831,1030.52425396)(499.69007865,1030.72925376)(499.39008057,1030.97925148)
\curveto(499.09007925,1031.21925327)(498.87007947,1031.54425294)(498.73008057,1031.95425148)
\curveto(498.71007963,1031.9842525)(498.70007964,1032.01425247)(498.70008057,1032.04425148)
\curveto(498.70007964,1032.07425241)(498.69507964,1032.09925239)(498.68508057,1032.11925148)
\curveto(498.65507968,1032.24925224)(498.66507967,1032.34925214)(498.71508057,1032.41925148)
\curveto(498.77507956,1032.47925201)(498.85507948,1032.51925197)(498.95508057,1032.53925148)
\curveto(499.05507928,1032.55925193)(499.16507917,1032.56925192)(499.28508057,1032.56925148)
\curveto(499.41507892,1032.55925193)(499.5350788,1032.55425193)(499.64508057,1032.55425148)
\lineto(500.15508057,1032.55425148)
\lineto(500.27508057,1032.55425148)
\curveto(500.31507802,1032.54425194)(500.36007798,1032.53925195)(500.41008057,1032.53925148)
\curveto(500.57007777,1032.49925199)(500.67007767,1032.44925204)(500.71008057,1032.38925148)
\curveto(500.75007759,1032.31925217)(500.81007753,1032.22925226)(500.89008057,1032.11925148)
\curveto(500.92007742,1032.07925241)(500.96507737,1032.02925246)(501.02508057,1031.96925148)
\curveto(501.0350773,1031.94925254)(501.04507729,1031.93425255)(501.05508057,1031.92425148)
\curveto(501.06507727,1031.91425257)(501.07507726,1031.89925259)(501.08508057,1031.87925148)
\curveto(501.16507717,1031.81925267)(501.25007709,1031.76425272)(501.34008057,1031.71425148)
\curveto(501.43007691,1031.66425282)(501.53007681,1031.61925287)(501.64008057,1031.57925148)
\curveto(501.71007663,1031.55925293)(501.78007656,1031.54925294)(501.85008057,1031.54925148)
\curveto(501.92007642,1031.53925295)(501.99507634,1031.52425296)(502.07508057,1031.50425148)
\lineto(502.24008057,1031.50425148)
\curveto(502.31007603,1031.484253)(502.40007594,1031.484253)(502.51008057,1031.50425148)
\curveto(502.62007572,1031.51425297)(502.70507563,1031.52925296)(502.76508057,1031.54925148)
\curveto(502.81507552,1031.56925292)(502.85507548,1031.57925291)(502.88508057,1031.57925148)
\curveto(502.92507541,1031.57925291)(502.96507537,1031.5892529)(503.00508057,1031.60925148)
\curveto(503.21507512,1031.69925279)(503.39007495,1031.81925267)(503.53008057,1031.96925148)
\curveto(503.67007467,1032.11925237)(503.78507455,1032.29425219)(503.87508057,1032.49425148)
\curveto(503.89507444,1032.55425193)(503.91007443,1032.61425187)(503.92008057,1032.67425148)
\curveto(503.93007441,1032.73425175)(503.94507439,1032.79925169)(503.96508057,1032.86925148)
\curveto(503.98507435,1032.95925153)(503.99507434,1033.05425143)(503.99508057,1033.15425148)
\curveto(504.00507433,1033.26425122)(504.01007433,1033.37425111)(504.01008057,1033.48425148)
\lineto(504.01008057,1033.60425148)
\curveto(504.02007432,1033.64425084)(504.02007432,1033.67925081)(504.01008057,1033.70925148)
\curveto(503.99007435,1033.75925073)(503.98007436,1033.80425068)(503.98008057,1033.84425148)
\curveto(503.99007435,1033.8842506)(503.98507435,1033.92425056)(503.96508057,1033.96425148)
\curveto(503.95507438,1033.9842505)(503.9400744,1033.99925049)(503.92008057,1034.00925148)
\lineto(503.87508057,1034.05425148)
\curveto(503.78507455,1034.06425042)(503.71007463,1034.04425044)(503.65008057,1033.99425148)
\curveto(503.60007474,1033.94425054)(503.55007479,1033.89925059)(503.50008057,1033.85925148)
\curveto(503.41007493,1033.7892507)(503.32007502,1033.72425076)(503.23008057,1033.66425148)
\curveto(503.1400752,1033.60425088)(503.0400753,1033.54925094)(502.93008057,1033.49925148)
\curveto(502.82007552,1033.44925104)(502.71007563,1033.40925108)(502.60008057,1033.37925148)
\curveto(502.49007585,1033.34925114)(502.37507596,1033.31925117)(502.25508057,1033.28925148)
\lineto(502.07508057,1033.25925148)
\curveto(502.02507631,1033.25925123)(501.97507636,1033.25425123)(501.92508057,1033.24425148)
\curveto(501.87507646,1033.23425125)(501.79507654,1033.22925126)(501.68508057,1033.22925148)
\curveto(501.57507676,1033.22925126)(501.49507684,1033.23425125)(501.44508057,1033.24425148)
\lineto(501.32508057,1033.24425148)
\curveto(501.29507704,1033.25425123)(501.26007708,1033.25925123)(501.22008057,1033.25925148)
\curveto(501.19007715,1033.25925123)(501.15507718,1033.26425122)(501.11508057,1033.27425148)
\curveto(500.97507736,1033.30425118)(500.8400775,1033.32925116)(500.71008057,1033.34925148)
\curveto(500.58007776,1033.37925111)(500.46007788,1033.41925107)(500.35008057,1033.46925148)
\curveto(499.92007842,1033.63925085)(499.57007877,1033.87425061)(499.30008057,1034.17425148)
\curveto(499.0400793,1034.48425)(498.82007952,1034.85424963)(498.64008057,1035.28425148)
\curveto(498.59007975,1035.39424909)(498.55507978,1035.50924898)(498.53508057,1035.62925148)
\curveto(498.51507982,1035.74924874)(498.48507985,1035.86924862)(498.44508057,1035.98925148)
\curveto(498.44507989,1036.03924845)(498.4400799,1036.07924841)(498.43008057,1036.10925148)
\curveto(498.41007993,1036.1892483)(498.40007994,1036.27424821)(498.40008057,1036.36425148)
\curveto(498.40007994,1036.46424802)(498.39007995,1036.55424793)(498.37008057,1036.63425148)
\curveto(498.36007998,1036.6842478)(498.35507998,1036.72924776)(498.35508057,1036.76925148)
\lineto(498.35508057,1036.91925148)
\curveto(498.34507999,1036.96924752)(498.34008,1037.02924746)(498.34008057,1037.09925148)
\curveto(498.34008,1037.17924731)(498.34507999,1037.24424724)(498.35508057,1037.29425148)
\lineto(498.35508057,1037.44425148)
\curveto(498.36507997,1037.484247)(498.36507997,1037.52424696)(498.35508057,1037.56425148)
\curveto(498.35507998,1037.60424688)(498.36507997,1037.64424684)(498.38508057,1037.68425148)
\curveto(498.40507993,1037.7842467)(498.42007992,1037.87924661)(498.43008057,1037.96925148)
\curveto(498.4400799,1038.06924642)(498.45507988,1038.16924632)(498.47508057,1038.26925148)
\curveto(498.5350798,1038.46924602)(498.59507974,1038.65924583)(498.65508057,1038.83925148)
\curveto(498.72507961,1039.01924547)(498.81007953,1039.1892453)(498.91008057,1039.34925148)
\curveto(498.96007938,1039.44924504)(499.01507932,1039.53924495)(499.07508057,1039.61925148)
\lineto(499.28508057,1039.88925148)
\curveto(499.31507902,1039.93924455)(499.35507898,1039.9892445)(499.40508057,1040.03925148)
\curveto(499.46507887,1040.0892444)(499.52007882,1040.13424435)(499.57008057,1040.17425148)
\lineto(499.66008057,1040.26425148)
\curveto(499.71007863,1040.30424418)(499.76007858,1040.33924415)(499.81008057,1040.36925148)
\curveto(499.86007848,1040.40924408)(499.91007843,1040.44424404)(499.96008057,1040.47425148)
\curveto(500.09007825,1040.55424393)(500.22507811,1040.62424386)(500.36508057,1040.68425148)
\curveto(500.50507783,1040.74424374)(500.66007768,1040.79924369)(500.83008057,1040.84925148)
\curveto(500.91007743,1040.87924361)(500.99007735,1040.89424359)(501.07008057,1040.89425148)
\curveto(501.16007718,1040.90424358)(501.24507709,1040.91924357)(501.32508057,1040.93925148)
\curveto(501.36507697,1040.94924354)(501.42007692,1040.94924354)(501.49008057,1040.93925148)
\curveto(501.56007678,1040.92924356)(501.60507673,1040.93424355)(501.62508057,1040.95425148)
\curveto(501.94507639,1040.96424352)(502.23007611,1040.93424355)(502.48008057,1040.86425148)
\curveto(502.7400756,1040.79424369)(502.97007537,1040.69424379)(503.17008057,1040.56425148)
\curveto(503.20007514,1040.54424394)(503.23007511,1040.51924397)(503.26008057,1040.48925148)
\curveto(503.29007505,1040.46924402)(503.32507501,1040.44424404)(503.36508057,1040.41425148)
\curveto(503.42507491,1040.36424412)(503.48007486,1040.31424417)(503.53008057,1040.26425148)
\curveto(503.58007476,1040.21424427)(503.6400747,1040.16924432)(503.71008057,1040.12925148)
\curveto(503.73007461,1040.11924437)(503.75507458,1040.10924438)(503.78508057,1040.09925148)
\curveto(503.82507451,1040.0892444)(503.85507448,1040.09424439)(503.87508057,1040.11425148)
\curveto(503.92507441,1040.13424435)(503.95507438,1040.16924432)(503.96508057,1040.21925148)
\curveto(503.97507436,1040.26924422)(503.99007435,1040.31924417)(504.01008057,1040.36925148)
\curveto(504.03007431,1040.41924407)(504.04507429,1040.46924402)(504.05508057,1040.51925148)
\curveto(504.07507426,1040.57924391)(504.10507423,1040.62924386)(504.14508057,1040.66925148)
\curveto(504.20507413,1040.70924378)(504.27507406,1040.72924376)(504.35508057,1040.72925148)
\curveto(504.44507389,1040.73924375)(504.5350738,1040.74424374)(504.62508057,1040.74425148)
\lineto(505.39008057,1040.74425148)
\curveto(505.50007284,1040.74424374)(505.59507274,1040.73924375)(505.67508057,1040.72925148)
\curveto(505.76507257,1040.72924376)(505.8400725,1040.70424378)(505.90008057,1040.65425148)
\moveto(503.84508057,1036.01925148)
\curveto(503.88507445,1036.10924838)(503.92007442,1036.22424826)(503.95008057,1036.36425148)
\curveto(503.98007436,1036.50424798)(504.00007434,1036.64924784)(504.01008057,1036.79925148)
\curveto(504.02007432,1036.95924753)(504.02007432,1037.11424737)(504.01008057,1037.26425148)
\curveto(504.01007433,1037.41424707)(503.99507434,1037.54924694)(503.96508057,1037.66925148)
\curveto(503.94507439,1037.70924678)(503.9350744,1037.73924675)(503.93508057,1037.75925148)
\curveto(503.94507439,1037.7892467)(503.94507439,1037.82424666)(503.93508057,1037.86425148)
\lineto(503.87508057,1038.07425148)
\curveto(503.85507448,1038.14424634)(503.83007451,1038.20924628)(503.80008057,1038.26925148)
\curveto(503.66007468,1038.61924587)(503.46007488,1038.8892456)(503.20008057,1039.07925148)
\curveto(502.9400754,1039.26924522)(502.56007578,1039.36424512)(502.06008057,1039.36425148)
\curveto(502.0400763,1039.34424514)(502.01007633,1039.33424515)(501.97008057,1039.33425148)
\curveto(501.9400764,1039.34424514)(501.91007643,1039.34424514)(501.88008057,1039.33425148)
\curveto(501.81007653,1039.31424517)(501.74507659,1039.29424519)(501.68508057,1039.27425148)
\curveto(501.62507671,1039.26424522)(501.56507677,1039.24924524)(501.50508057,1039.22925148)
\curveto(501.24507709,1039.11924537)(501.04507729,1038.95424553)(500.90508057,1038.73425148)
\curveto(500.76507757,1038.51424597)(500.65007769,1038.26924622)(500.56008057,1037.99925148)
\curveto(500.5400778,1037.94924654)(500.53007781,1037.90924658)(500.53008057,1037.87925148)
\curveto(500.53007781,1037.84924664)(500.52507781,1037.80924668)(500.51508057,1037.75925148)
\curveto(500.48507785,1037.64924684)(500.46507787,1037.489247)(500.45508057,1037.27925148)
\curveto(500.44507789,1037.06924742)(500.45507788,1036.89924759)(500.48508057,1036.76925148)
\lineto(500.48508057,1036.61925148)
\curveto(500.50507783,1036.53924795)(500.52007782,1036.45924803)(500.53008057,1036.37925148)
\curveto(500.5400778,1036.30924818)(500.55507778,1036.23424825)(500.57508057,1036.15425148)
\curveto(500.66507767,1035.89424859)(500.77507756,1035.66424882)(500.90508057,1035.46425148)
\curveto(501.0350773,1035.27424921)(501.21507712,1035.11924937)(501.44508057,1034.99925148)
\curveto(501.54507679,1034.94924954)(501.68507665,1034.89924959)(501.86508057,1034.84925148)
\curveto(501.9350764,1034.84924964)(501.99007635,1034.84424964)(502.03008057,1034.83425148)
\curveto(502.05007629,1034.83424965)(502.08007626,1034.82924966)(502.12008057,1034.81925148)
\curveto(502.16007618,1034.81924967)(502.19007615,1034.82424966)(502.21008057,1034.83425148)
\lineto(502.36008057,1034.83425148)
\curveto(502.45007589,1034.85424963)(502.5350758,1034.86924962)(502.61508057,1034.87925148)
\curveto(502.69507564,1034.8892496)(502.77507556,1034.91424957)(502.85508057,1034.95425148)
\curveto(503.10507523,1035.05424943)(503.30507503,1035.19424929)(503.45508057,1035.37425148)
\curveto(503.61507472,1035.55424893)(503.74507459,1035.76924872)(503.84508057,1036.01925148)
}
}
{
\newrgbcolor{curcolor}{0 0 0}
\pscustom[linestyle=none,fillstyle=solid,fillcolor=curcolor]
{
\newpath
\moveto(512.14500244,1040.93925148)
\curveto(512.25499713,1040.93924355)(512.34999703,1040.92924356)(512.43000244,1040.90925148)
\curveto(512.51999686,1040.8892436)(512.58999679,1040.84424364)(512.64000244,1040.77425148)
\curveto(512.69999668,1040.69424379)(512.72999665,1040.55424393)(512.73000244,1040.35425148)
\lineto(512.73000244,1039.84425148)
\lineto(512.73000244,1039.46925148)
\curveto(512.73999664,1039.32924516)(512.72499666,1039.21924527)(512.68500244,1039.13925148)
\curveto(512.64499674,1039.06924542)(512.5849968,1039.02424546)(512.50500244,1039.00425148)
\curveto(512.43499695,1038.9842455)(512.34999703,1038.97424551)(512.25000244,1038.97425148)
\curveto(512.15999722,1038.97424551)(512.05999732,1038.97924551)(511.95000244,1038.98925148)
\curveto(511.84999753,1038.99924549)(511.75499763,1038.99424549)(511.66500244,1038.97425148)
\curveto(511.59499779,1038.95424553)(511.52499786,1038.93924555)(511.45500244,1038.92925148)
\curveto(511.384998,1038.92924556)(511.31999806,1038.91924557)(511.26000244,1038.89925148)
\curveto(511.09999828,1038.84924564)(510.93999844,1038.77424571)(510.78000244,1038.67425148)
\curveto(510.61999876,1038.5842459)(510.49499889,1038.47924601)(510.40500244,1038.35925148)
\curveto(510.35499903,1038.27924621)(510.29999908,1038.19424629)(510.24000244,1038.10425148)
\curveto(510.18999919,1038.02424646)(510.13999924,1037.93924655)(510.09000244,1037.84925148)
\curveto(510.05999932,1037.76924672)(510.02999935,1037.6842468)(510.00000244,1037.59425148)
\lineto(509.94000244,1037.35425148)
\curveto(509.91999946,1037.2842472)(509.90999947,1037.20924728)(509.91000244,1037.12925148)
\curveto(509.90999947,1037.05924743)(509.89999948,1036.9892475)(509.88000244,1036.91925148)
\curveto(509.86999951,1036.87924761)(509.86499952,1036.83924765)(509.86500244,1036.79925148)
\curveto(509.87499951,1036.76924772)(509.87499951,1036.73924775)(509.86500244,1036.70925148)
\lineto(509.86500244,1036.46925148)
\curveto(509.84499954,1036.39924809)(509.83999954,1036.31924817)(509.85000244,1036.22925148)
\curveto(509.85999952,1036.14924834)(509.86499952,1036.06924842)(509.86500244,1035.98925148)
\lineto(509.86500244,1035.02925148)
\lineto(509.86500244,1033.75425148)
\curveto(509.86499952,1033.62425086)(509.85999952,1033.50425098)(509.85000244,1033.39425148)
\curveto(509.83999954,1033.2842512)(509.80999957,1033.19425129)(509.76000244,1033.12425148)
\curveto(509.73999964,1033.09425139)(509.70499968,1033.06925142)(509.65500244,1033.04925148)
\curveto(509.61499977,1033.03925145)(509.56999981,1033.02925146)(509.52000244,1033.01925148)
\lineto(509.44500244,1033.01925148)
\curveto(509.39499999,1033.00925148)(509.34000004,1033.00425148)(509.28000244,1033.00425148)
\lineto(509.11500244,1033.00425148)
\lineto(508.47000244,1033.00425148)
\curveto(508.41000097,1033.01425147)(508.34500104,1033.01925147)(508.27500244,1033.01925148)
\lineto(508.08000244,1033.01925148)
\curveto(508.03000135,1033.03925145)(507.9800014,1033.05425143)(507.93000244,1033.06425148)
\curveto(507.8800015,1033.0842514)(507.84500154,1033.11925137)(507.82500244,1033.16925148)
\curveto(507.7850016,1033.21925127)(507.76000162,1033.2892512)(507.75000244,1033.37925148)
\lineto(507.75000244,1033.67925148)
\lineto(507.75000244,1034.69925148)
\lineto(507.75000244,1038.92925148)
\lineto(507.75000244,1040.03925148)
\lineto(507.75000244,1040.32425148)
\curveto(507.75000163,1040.42424406)(507.77000161,1040.50424398)(507.81000244,1040.56425148)
\curveto(507.86000152,1040.64424384)(507.93500145,1040.69424379)(508.03500244,1040.71425148)
\curveto(508.13500125,1040.73424375)(508.25500113,1040.74424374)(508.39500244,1040.74425148)
\lineto(509.16000244,1040.74425148)
\curveto(509.2800001,1040.74424374)(509.385,1040.73424375)(509.47500244,1040.71425148)
\curveto(509.56499982,1040.70424378)(509.63499975,1040.65924383)(509.68500244,1040.57925148)
\curveto(509.71499967,1040.52924396)(509.72999965,1040.45924403)(509.73000244,1040.36925148)
\lineto(509.76000244,1040.09925148)
\curveto(509.76999961,1040.01924447)(509.7849996,1039.94424454)(509.80500244,1039.87425148)
\curveto(509.83499955,1039.80424468)(509.8849995,1039.76924472)(509.95500244,1039.76925148)
\curveto(509.97499941,1039.7892447)(509.99499939,1039.79924469)(510.01500244,1039.79925148)
\curveto(510.03499935,1039.79924469)(510.05499933,1039.80924468)(510.07500244,1039.82925148)
\curveto(510.13499925,1039.87924461)(510.1849992,1039.93424455)(510.22500244,1039.99425148)
\curveto(510.27499911,1040.06424442)(510.33499905,1040.12424436)(510.40500244,1040.17425148)
\curveto(510.44499894,1040.20424428)(510.4799989,1040.23424425)(510.51000244,1040.26425148)
\curveto(510.53999884,1040.30424418)(510.57499881,1040.33924415)(510.61500244,1040.36925148)
\lineto(510.88500244,1040.54925148)
\curveto(510.9849984,1040.60924388)(511.0849983,1040.66424382)(511.18500244,1040.71425148)
\curveto(511.2849981,1040.75424373)(511.384998,1040.7892437)(511.48500244,1040.81925148)
\lineto(511.81500244,1040.90925148)
\curveto(511.84499754,1040.91924357)(511.89999748,1040.91924357)(511.98000244,1040.90925148)
\curveto(512.06999731,1040.90924358)(512.12499726,1040.91924357)(512.14500244,1040.93925148)
}
}
{
\newrgbcolor{curcolor}{0 0 0}
\pscustom[linestyle=none,fillstyle=solid,fillcolor=curcolor]
{
\newpath
\moveto(520.65140869,1036.94925148)
\curveto(520.67140053,1036.86924762)(520.67140053,1036.77924771)(520.65140869,1036.67925148)
\curveto(520.63140057,1036.57924791)(520.5964006,1036.51424797)(520.54640869,1036.48425148)
\curveto(520.4964007,1036.44424804)(520.42140078,1036.41424807)(520.32140869,1036.39425148)
\curveto(520.23140097,1036.3842481)(520.12640107,1036.37424811)(520.00640869,1036.36425148)
\lineto(519.66140869,1036.36425148)
\curveto(519.55140165,1036.37424811)(519.45140175,1036.37924811)(519.36140869,1036.37925148)
\lineto(515.70140869,1036.37925148)
\lineto(515.49140869,1036.37925148)
\curveto(515.43140577,1036.37924811)(515.37640582,1036.36924812)(515.32640869,1036.34925148)
\curveto(515.24640595,1036.30924818)(515.196406,1036.26924822)(515.17640869,1036.22925148)
\curveto(515.15640604,1036.20924828)(515.13640606,1036.16924832)(515.11640869,1036.10925148)
\curveto(515.0964061,1036.05924843)(515.09140611,1036.00924848)(515.10140869,1035.95925148)
\curveto(515.12140608,1035.89924859)(515.13140607,1035.83924865)(515.13140869,1035.77925148)
\curveto(515.14140606,1035.72924876)(515.15640604,1035.67424881)(515.17640869,1035.61425148)
\curveto(515.25640594,1035.37424911)(515.35140585,1035.17424931)(515.46140869,1035.01425148)
\curveto(515.58140562,1034.86424962)(515.74140546,1034.72924976)(515.94140869,1034.60925148)
\curveto(516.02140518,1034.55924993)(516.1014051,1034.52424996)(516.18140869,1034.50425148)
\curveto(516.27140493,1034.49424999)(516.36140484,1034.47425001)(516.45140869,1034.44425148)
\curveto(516.53140467,1034.42425006)(516.64140456,1034.40925008)(516.78140869,1034.39925148)
\curveto(516.92140428,1034.3892501)(517.04140416,1034.39425009)(517.14140869,1034.41425148)
\lineto(517.27640869,1034.41425148)
\curveto(517.37640382,1034.43425005)(517.46640373,1034.45425003)(517.54640869,1034.47425148)
\curveto(517.63640356,1034.50424998)(517.72140348,1034.53424995)(517.80140869,1034.56425148)
\curveto(517.9014033,1034.61424987)(518.01140319,1034.67924981)(518.13140869,1034.75925148)
\curveto(518.26140294,1034.83924965)(518.35640284,1034.91924957)(518.41640869,1034.99925148)
\curveto(518.46640273,1035.06924942)(518.51640268,1035.13424935)(518.56640869,1035.19425148)
\curveto(518.62640257,1035.26424922)(518.6964025,1035.31424917)(518.77640869,1035.34425148)
\curveto(518.87640232,1035.39424909)(519.0014022,1035.41424907)(519.15140869,1035.40425148)
\lineto(519.58640869,1035.40425148)
\lineto(519.76640869,1035.40425148)
\curveto(519.83640136,1035.41424907)(519.8964013,1035.40924908)(519.94640869,1035.38925148)
\lineto(520.09640869,1035.38925148)
\curveto(520.196401,1035.36924912)(520.26640093,1035.34424914)(520.30640869,1035.31425148)
\curveto(520.34640085,1035.29424919)(520.36640083,1035.24924924)(520.36640869,1035.17925148)
\curveto(520.37640082,1035.10924938)(520.37140083,1035.04924944)(520.35140869,1034.99925148)
\curveto(520.3014009,1034.85924963)(520.24640095,1034.73424975)(520.18640869,1034.62425148)
\curveto(520.12640107,1034.51424997)(520.05640114,1034.40425008)(519.97640869,1034.29425148)
\curveto(519.75640144,1033.96425052)(519.50640169,1033.69925079)(519.22640869,1033.49925148)
\curveto(518.94640225,1033.29925119)(518.5964026,1033.12925136)(518.17640869,1032.98925148)
\curveto(518.06640313,1032.94925154)(517.95640324,1032.92425156)(517.84640869,1032.91425148)
\curveto(517.73640346,1032.90425158)(517.62140358,1032.8842516)(517.50140869,1032.85425148)
\curveto(517.46140374,1032.84425164)(517.41640378,1032.84425164)(517.36640869,1032.85425148)
\curveto(517.32640387,1032.85425163)(517.28640391,1032.84925164)(517.24640869,1032.83925148)
\lineto(517.08140869,1032.83925148)
\curveto(517.03140417,1032.81925167)(516.97140423,1032.81425167)(516.90140869,1032.82425148)
\curveto(516.84140436,1032.82425166)(516.78640441,1032.82925166)(516.73640869,1032.83925148)
\curveto(516.65640454,1032.84925164)(516.58640461,1032.84925164)(516.52640869,1032.83925148)
\curveto(516.46640473,1032.82925166)(516.4014048,1032.83425165)(516.33140869,1032.85425148)
\curveto(516.28140492,1032.87425161)(516.22640497,1032.8842516)(516.16640869,1032.88425148)
\curveto(516.10640509,1032.8842516)(516.05140515,1032.89425159)(516.00140869,1032.91425148)
\curveto(515.89140531,1032.93425155)(515.78140542,1032.95925153)(515.67140869,1032.98925148)
\curveto(515.56140564,1033.00925148)(515.46140574,1033.04425144)(515.37140869,1033.09425148)
\curveto(515.26140594,1033.13425135)(515.15640604,1033.16925132)(515.05640869,1033.19925148)
\curveto(514.96640623,1033.23925125)(514.88140632,1033.2842512)(514.80140869,1033.33425148)
\curveto(514.48140672,1033.53425095)(514.196407,1033.76425072)(513.94640869,1034.02425148)
\curveto(513.6964075,1034.29425019)(513.49140771,1034.60424988)(513.33140869,1034.95425148)
\curveto(513.28140792,1035.06424942)(513.24140796,1035.17424931)(513.21140869,1035.28425148)
\curveto(513.18140802,1035.40424908)(513.14140806,1035.52424896)(513.09140869,1035.64425148)
\curveto(513.08140812,1035.6842488)(513.07640812,1035.71924877)(513.07640869,1035.74925148)
\curveto(513.07640812,1035.7892487)(513.07140813,1035.82924866)(513.06140869,1035.86925148)
\curveto(513.02140818,1035.9892485)(512.9964082,1036.11924837)(512.98640869,1036.25925148)
\lineto(512.95640869,1036.67925148)
\curveto(512.95640824,1036.72924776)(512.95140825,1036.7842477)(512.94140869,1036.84425148)
\curveto(512.94140826,1036.90424758)(512.94640825,1036.95924753)(512.95640869,1037.00925148)
\lineto(512.95640869,1037.18925148)
\lineto(513.00140869,1037.54925148)
\curveto(513.04140816,1037.71924677)(513.07640812,1037.8842466)(513.10640869,1038.04425148)
\curveto(513.13640806,1038.20424628)(513.18140802,1038.35424613)(513.24140869,1038.49425148)
\curveto(513.67140753,1039.53424495)(514.4014068,1040.26924422)(515.43140869,1040.69925148)
\curveto(515.57140563,1040.75924373)(515.71140549,1040.79924369)(515.85140869,1040.81925148)
\curveto(516.0014052,1040.84924364)(516.15640504,1040.8842436)(516.31640869,1040.92425148)
\curveto(516.3964048,1040.93424355)(516.47140473,1040.93924355)(516.54140869,1040.93925148)
\curveto(516.61140459,1040.93924355)(516.68640451,1040.94424354)(516.76640869,1040.95425148)
\curveto(517.27640392,1040.96424352)(517.71140349,1040.90424358)(518.07140869,1040.77425148)
\curveto(518.44140276,1040.65424383)(518.77140243,1040.49424399)(519.06140869,1040.29425148)
\curveto(519.15140205,1040.23424425)(519.24140196,1040.16424432)(519.33140869,1040.08425148)
\curveto(519.42140178,1040.01424447)(519.5014017,1039.93924455)(519.57140869,1039.85925148)
\curveto(519.6014016,1039.80924468)(519.64140156,1039.76924472)(519.69140869,1039.73925148)
\curveto(519.77140143,1039.62924486)(519.84640135,1039.51424497)(519.91640869,1039.39425148)
\curveto(519.98640121,1039.2842452)(520.06140114,1039.16924532)(520.14140869,1039.04925148)
\curveto(520.19140101,1038.95924553)(520.23140097,1038.86424562)(520.26140869,1038.76425148)
\curveto(520.3014009,1038.67424581)(520.34140086,1038.57424591)(520.38140869,1038.46425148)
\curveto(520.43140077,1038.33424615)(520.47140073,1038.19924629)(520.50140869,1038.05925148)
\curveto(520.53140067,1037.91924657)(520.56640063,1037.77924671)(520.60640869,1037.63925148)
\curveto(520.62640057,1037.55924693)(520.63140057,1037.46924702)(520.62140869,1037.36925148)
\curveto(520.62140058,1037.27924721)(520.63140057,1037.19424729)(520.65140869,1037.11425148)
\lineto(520.65140869,1036.94925148)
\moveto(518.40140869,1037.83425148)
\curveto(518.47140273,1037.93424655)(518.47640272,1038.05424643)(518.41640869,1038.19425148)
\curveto(518.36640283,1038.34424614)(518.32640287,1038.45424603)(518.29640869,1038.52425148)
\curveto(518.15640304,1038.79424569)(517.97140323,1038.99924549)(517.74140869,1039.13925148)
\curveto(517.51140369,1039.2892452)(517.19140401,1039.36924512)(516.78140869,1039.37925148)
\curveto(516.75140445,1039.35924513)(516.71640448,1039.35424513)(516.67640869,1039.36425148)
\curveto(516.63640456,1039.37424511)(516.6014046,1039.37424511)(516.57140869,1039.36425148)
\curveto(516.52140468,1039.34424514)(516.46640473,1039.32924516)(516.40640869,1039.31925148)
\curveto(516.34640485,1039.31924517)(516.29140491,1039.30924518)(516.24140869,1039.28925148)
\curveto(515.8014054,1039.14924534)(515.47640572,1038.87424561)(515.26640869,1038.46425148)
\curveto(515.24640595,1038.42424606)(515.22140598,1038.36924612)(515.19140869,1038.29925148)
\curveto(515.17140603,1038.23924625)(515.15640604,1038.17424631)(515.14640869,1038.10425148)
\curveto(515.13640606,1038.04424644)(515.13640606,1037.9842465)(515.14640869,1037.92425148)
\curveto(515.16640603,1037.86424662)(515.201406,1037.81424667)(515.25140869,1037.77425148)
\curveto(515.33140587,1037.72424676)(515.44140576,1037.69924679)(515.58140869,1037.69925148)
\lineto(515.98640869,1037.69925148)
\lineto(517.65140869,1037.69925148)
\lineto(518.08640869,1037.69925148)
\curveto(518.24640295,1037.70924678)(518.35140285,1037.75424673)(518.40140869,1037.83425148)
}
}
{
\newrgbcolor{curcolor}{0 0 0}
\pscustom[linestyle=none,fillstyle=solid,fillcolor=curcolor]
{
\newpath
\moveto(524.86968994,1040.95425148)
\curveto(525.61968544,1040.97424351)(526.26968479,1040.8892436)(526.81968994,1040.69925148)
\curveto(527.37968368,1040.51924397)(527.80468326,1040.20424428)(528.09468994,1039.75425148)
\curveto(528.1646829,1039.64424484)(528.22468284,1039.52924496)(528.27468994,1039.40925148)
\curveto(528.33468273,1039.29924519)(528.38468268,1039.17424531)(528.42468994,1039.03425148)
\curveto(528.44468262,1038.97424551)(528.45468261,1038.90924558)(528.45468994,1038.83925148)
\curveto(528.45468261,1038.76924572)(528.44468262,1038.70924578)(528.42468994,1038.65925148)
\curveto(528.38468268,1038.59924589)(528.32968273,1038.55924593)(528.25968994,1038.53925148)
\curveto(528.20968285,1038.51924597)(528.14968291,1038.50924598)(528.07968994,1038.50925148)
\lineto(527.86968994,1038.50925148)
\lineto(527.20968994,1038.50925148)
\curveto(527.13968392,1038.50924598)(527.06968399,1038.50424598)(526.99968994,1038.49425148)
\curveto(526.92968413,1038.49424599)(526.8646842,1038.50424598)(526.80468994,1038.52425148)
\curveto(526.70468436,1038.54424594)(526.62968443,1038.5842459)(526.57968994,1038.64425148)
\curveto(526.52968453,1038.70424578)(526.48468458,1038.76424572)(526.44468994,1038.82425148)
\lineto(526.32468994,1039.03425148)
\curveto(526.29468477,1039.11424537)(526.24468482,1039.17924531)(526.17468994,1039.22925148)
\curveto(526.07468499,1039.30924518)(525.97468509,1039.36924512)(525.87468994,1039.40925148)
\curveto(525.78468528,1039.44924504)(525.66968539,1039.484245)(525.52968994,1039.51425148)
\curveto(525.4596856,1039.53424495)(525.35468571,1039.54924494)(525.21468994,1039.55925148)
\curveto(525.08468598,1039.56924492)(524.98468608,1039.56424492)(524.91468994,1039.54425148)
\lineto(524.80968994,1039.54425148)
\lineto(524.65968994,1039.51425148)
\curveto(524.61968644,1039.51424497)(524.57468649,1039.50924498)(524.52468994,1039.49925148)
\curveto(524.35468671,1039.44924504)(524.21468685,1039.37924511)(524.10468994,1039.28925148)
\curveto(524.00468706,1039.20924528)(523.93468713,1039.0842454)(523.89468994,1038.91425148)
\curveto(523.87468719,1038.84424564)(523.87468719,1038.77924571)(523.89468994,1038.71925148)
\curveto(523.91468715,1038.65924583)(523.93468713,1038.60924588)(523.95468994,1038.56925148)
\curveto(524.02468704,1038.44924604)(524.10468696,1038.35424613)(524.19468994,1038.28425148)
\curveto(524.29468677,1038.21424627)(524.40968665,1038.15424633)(524.53968994,1038.10425148)
\curveto(524.72968633,1038.02424646)(524.93468613,1037.95424653)(525.15468994,1037.89425148)
\lineto(525.84468994,1037.74425148)
\curveto(526.08468498,1037.70424678)(526.31468475,1037.65424683)(526.53468994,1037.59425148)
\curveto(526.7646843,1037.54424694)(526.97968408,1037.47924701)(527.17968994,1037.39925148)
\curveto(527.26968379,1037.35924713)(527.35468371,1037.32424716)(527.43468994,1037.29425148)
\curveto(527.52468354,1037.27424721)(527.60968345,1037.23924725)(527.68968994,1037.18925148)
\curveto(527.87968318,1037.06924742)(528.04968301,1036.93924755)(528.19968994,1036.79925148)
\curveto(528.3596827,1036.65924783)(528.48468258,1036.484248)(528.57468994,1036.27425148)
\curveto(528.60468246,1036.20424828)(528.62968243,1036.13424835)(528.64968994,1036.06425148)
\curveto(528.66968239,1035.99424849)(528.68968237,1035.91924857)(528.70968994,1035.83925148)
\curveto(528.71968234,1035.77924871)(528.72468234,1035.6842488)(528.72468994,1035.55425148)
\curveto(528.73468233,1035.43424905)(528.73468233,1035.33924915)(528.72468994,1035.26925148)
\lineto(528.72468994,1035.19425148)
\curveto(528.70468236,1035.13424935)(528.68968237,1035.07424941)(528.67968994,1035.01425148)
\curveto(528.67968238,1034.96424952)(528.67468239,1034.91424957)(528.66468994,1034.86425148)
\curveto(528.59468247,1034.56424992)(528.48468258,1034.29925019)(528.33468994,1034.06925148)
\curveto(528.17468289,1033.82925066)(527.97968308,1033.63425085)(527.74968994,1033.48425148)
\curveto(527.51968354,1033.33425115)(527.2596838,1033.20425128)(526.96968994,1033.09425148)
\curveto(526.8596842,1033.04425144)(526.73968432,1033.00925148)(526.60968994,1032.98925148)
\curveto(526.48968457,1032.96925152)(526.36968469,1032.94425154)(526.24968994,1032.91425148)
\curveto(526.1596849,1032.89425159)(526.064685,1032.8842516)(525.96468994,1032.88425148)
\curveto(525.87468519,1032.87425161)(525.78468528,1032.85925163)(525.69468994,1032.83925148)
\lineto(525.42468994,1032.83925148)
\curveto(525.3646857,1032.81925167)(525.2596858,1032.80925168)(525.10968994,1032.80925148)
\curveto(524.96968609,1032.80925168)(524.86968619,1032.81925167)(524.80968994,1032.83925148)
\curveto(524.77968628,1032.83925165)(524.74468632,1032.84425164)(524.70468994,1032.85425148)
\lineto(524.59968994,1032.85425148)
\curveto(524.47968658,1032.87425161)(524.3596867,1032.8892516)(524.23968994,1032.89925148)
\curveto(524.11968694,1032.90925158)(524.00468706,1032.92925156)(523.89468994,1032.95925148)
\curveto(523.50468756,1033.06925142)(523.1596879,1033.19425129)(522.85968994,1033.33425148)
\curveto(522.5596885,1033.484251)(522.30468876,1033.70425078)(522.09468994,1033.99425148)
\curveto(521.95468911,1034.1842503)(521.83468923,1034.40425008)(521.73468994,1034.65425148)
\curveto(521.71468935,1034.71424977)(521.69468937,1034.79424969)(521.67468994,1034.89425148)
\curveto(521.65468941,1034.94424954)(521.63968942,1035.01424947)(521.62968994,1035.10425148)
\curveto(521.61968944,1035.19424929)(521.62468944,1035.26924922)(521.64468994,1035.32925148)
\curveto(521.67468939,1035.39924909)(521.72468934,1035.44924904)(521.79468994,1035.47925148)
\curveto(521.84468922,1035.49924899)(521.90468916,1035.50924898)(521.97468994,1035.50925148)
\lineto(522.19968994,1035.50925148)
\lineto(522.90468994,1035.50925148)
\lineto(523.14468994,1035.50925148)
\curveto(523.22468784,1035.50924898)(523.29468777,1035.49924899)(523.35468994,1035.47925148)
\curveto(523.4646876,1035.43924905)(523.53468753,1035.37424911)(523.56468994,1035.28425148)
\curveto(523.60468746,1035.19424929)(523.64968741,1035.09924939)(523.69968994,1034.99925148)
\curveto(523.71968734,1034.94924954)(523.75468731,1034.8842496)(523.80468994,1034.80425148)
\curveto(523.8646872,1034.72424976)(523.91468715,1034.67424981)(523.95468994,1034.65425148)
\curveto(524.07468699,1034.55424993)(524.18968687,1034.47425001)(524.29968994,1034.41425148)
\curveto(524.40968665,1034.36425012)(524.54968651,1034.31425017)(524.71968994,1034.26425148)
\curveto(524.76968629,1034.24425024)(524.81968624,1034.23425025)(524.86968994,1034.23425148)
\curveto(524.91968614,1034.24425024)(524.96968609,1034.24425024)(525.01968994,1034.23425148)
\curveto(525.09968596,1034.21425027)(525.18468588,1034.20425028)(525.27468994,1034.20425148)
\curveto(525.37468569,1034.21425027)(525.4596856,1034.22925026)(525.52968994,1034.24925148)
\curveto(525.57968548,1034.25925023)(525.62468544,1034.26425022)(525.66468994,1034.26425148)
\curveto(525.71468535,1034.26425022)(525.7646853,1034.27425021)(525.81468994,1034.29425148)
\curveto(525.95468511,1034.34425014)(526.07968498,1034.40425008)(526.18968994,1034.47425148)
\curveto(526.30968475,1034.54424994)(526.40468466,1034.63424985)(526.47468994,1034.74425148)
\curveto(526.52468454,1034.82424966)(526.5646845,1034.94924954)(526.59468994,1035.11925148)
\curveto(526.61468445,1035.1892493)(526.61468445,1035.25424923)(526.59468994,1035.31425148)
\curveto(526.57468449,1035.37424911)(526.55468451,1035.42424906)(526.53468994,1035.46425148)
\curveto(526.4646846,1035.60424888)(526.37468469,1035.70924878)(526.26468994,1035.77925148)
\curveto(526.1646849,1035.84924864)(526.04468502,1035.91424857)(525.90468994,1035.97425148)
\curveto(525.71468535,1036.05424843)(525.51468555,1036.11924837)(525.30468994,1036.16925148)
\curveto(525.09468597,1036.21924827)(524.88468618,1036.27424821)(524.67468994,1036.33425148)
\curveto(524.59468647,1036.35424813)(524.50968655,1036.36924812)(524.41968994,1036.37925148)
\curveto(524.33968672,1036.3892481)(524.2596868,1036.40424808)(524.17968994,1036.42425148)
\curveto(523.8596872,1036.51424797)(523.55468751,1036.59924789)(523.26468994,1036.67925148)
\curveto(522.97468809,1036.76924772)(522.70968835,1036.89924759)(522.46968994,1037.06925148)
\curveto(522.18968887,1037.26924722)(521.98468908,1037.53924695)(521.85468994,1037.87925148)
\curveto(521.83468923,1037.94924654)(521.81468925,1038.04424644)(521.79468994,1038.16425148)
\curveto(521.77468929,1038.23424625)(521.7596893,1038.31924617)(521.74968994,1038.41925148)
\curveto(521.73968932,1038.51924597)(521.74468932,1038.60924588)(521.76468994,1038.68925148)
\curveto(521.78468928,1038.73924575)(521.78968927,1038.77924571)(521.77968994,1038.80925148)
\curveto(521.76968929,1038.84924564)(521.77468929,1038.89424559)(521.79468994,1038.94425148)
\curveto(521.81468925,1039.05424543)(521.83468923,1039.15424533)(521.85468994,1039.24425148)
\curveto(521.88468918,1039.34424514)(521.91968914,1039.43924505)(521.95968994,1039.52925148)
\curveto(522.08968897,1039.81924467)(522.26968879,1040.05424443)(522.49968994,1040.23425148)
\curveto(522.72968833,1040.41424407)(522.98968807,1040.55924393)(523.27968994,1040.66925148)
\curveto(523.38968767,1040.71924377)(523.50468756,1040.75424373)(523.62468994,1040.77425148)
\curveto(523.74468732,1040.80424368)(523.86968719,1040.83424365)(523.99968994,1040.86425148)
\curveto(524.059687,1040.8842436)(524.11968694,1040.89424359)(524.17968994,1040.89425148)
\lineto(524.35968994,1040.92425148)
\curveto(524.43968662,1040.93424355)(524.52468654,1040.93924355)(524.61468994,1040.93925148)
\curveto(524.70468636,1040.93924355)(524.78968627,1040.94424354)(524.86968994,1040.95425148)
}
}
{
\newrgbcolor{curcolor}{0 0 0}
\pscustom[linestyle=none,fillstyle=solid,fillcolor=curcolor]
{
\newpath
\moveto(537.00633057,1033.60425148)
\curveto(537.02632272,1033.49425099)(537.03632271,1033.3842511)(537.03633057,1033.27425148)
\curveto(537.0463227,1033.16425132)(536.99632275,1033.0892514)(536.88633057,1033.04925148)
\curveto(536.82632292,1033.01925147)(536.75632299,1033.00425148)(536.67633057,1033.00425148)
\lineto(536.43633057,1033.00425148)
\lineto(535.62633057,1033.00425148)
\lineto(535.35633057,1033.00425148)
\curveto(535.27632447,1033.01425147)(535.21132453,1033.03925145)(535.16133057,1033.07925148)
\curveto(535.09132465,1033.11925137)(535.03632471,1033.17425131)(534.99633057,1033.24425148)
\curveto(534.96632478,1033.32425116)(534.92132482,1033.3892511)(534.86133057,1033.43925148)
\curveto(534.8413249,1033.45925103)(534.81632493,1033.47425101)(534.78633057,1033.48425148)
\curveto(534.75632499,1033.50425098)(534.71632503,1033.50925098)(534.66633057,1033.49925148)
\curveto(534.61632513,1033.47925101)(534.56632518,1033.45425103)(534.51633057,1033.42425148)
\curveto(534.47632527,1033.39425109)(534.43132531,1033.36925112)(534.38133057,1033.34925148)
\curveto(534.33132541,1033.30925118)(534.27632547,1033.27425121)(534.21633057,1033.24425148)
\lineto(534.03633057,1033.15425148)
\curveto(533.90632584,1033.09425139)(533.77132597,1033.04425144)(533.63133057,1033.00425148)
\curveto(533.49132625,1032.97425151)(533.3463264,1032.93925155)(533.19633057,1032.89925148)
\curveto(533.12632662,1032.87925161)(533.05632669,1032.86925162)(532.98633057,1032.86925148)
\curveto(532.92632682,1032.85925163)(532.86132688,1032.84925164)(532.79133057,1032.83925148)
\lineto(532.70133057,1032.83925148)
\curveto(532.67132707,1032.82925166)(532.6413271,1032.82425166)(532.61133057,1032.82425148)
\lineto(532.44633057,1032.82425148)
\curveto(532.3463274,1032.80425168)(532.2463275,1032.80425168)(532.14633057,1032.82425148)
\lineto(532.01133057,1032.82425148)
\curveto(531.9413278,1032.84425164)(531.87132787,1032.85425163)(531.80133057,1032.85425148)
\curveto(531.741328,1032.84425164)(531.68132806,1032.84925164)(531.62133057,1032.86925148)
\curveto(531.52132822,1032.8892516)(531.42632832,1032.90925158)(531.33633057,1032.92925148)
\curveto(531.2463285,1032.93925155)(531.16132858,1032.96425152)(531.08133057,1033.00425148)
\curveto(530.79132895,1033.11425137)(530.5413292,1033.25425123)(530.33133057,1033.42425148)
\curveto(530.13132961,1033.60425088)(529.97132977,1033.83925065)(529.85133057,1034.12925148)
\curveto(529.82132992,1034.19925029)(529.79132995,1034.27425021)(529.76133057,1034.35425148)
\curveto(529.74133,1034.43425005)(529.72133002,1034.51924997)(529.70133057,1034.60925148)
\curveto(529.68133006,1034.65924983)(529.67133007,1034.70924978)(529.67133057,1034.75925148)
\curveto(529.68133006,1034.80924968)(529.68133006,1034.85924963)(529.67133057,1034.90925148)
\curveto(529.66133008,1034.93924955)(529.65133009,1034.99924949)(529.64133057,1035.08925148)
\curveto(529.6413301,1035.1892493)(529.6463301,1035.25924923)(529.65633057,1035.29925148)
\curveto(529.67633007,1035.39924909)(529.68633006,1035.484249)(529.68633057,1035.55425148)
\lineto(529.77633057,1035.88425148)
\curveto(529.80632994,1036.00424848)(529.8463299,1036.10924838)(529.89633057,1036.19925148)
\curveto(530.06632968,1036.489248)(530.26132948,1036.70924778)(530.48133057,1036.85925148)
\curveto(530.70132904,1037.00924748)(530.98132876,1037.13924735)(531.32133057,1037.24925148)
\curveto(531.45132829,1037.29924719)(531.58632816,1037.33424715)(531.72633057,1037.35425148)
\curveto(531.86632788,1037.37424711)(532.00632774,1037.39924709)(532.14633057,1037.42925148)
\curveto(532.22632752,1037.44924704)(532.31132743,1037.45924703)(532.40133057,1037.45925148)
\curveto(532.49132725,1037.46924702)(532.58132716,1037.484247)(532.67133057,1037.50425148)
\curveto(532.741327,1037.52424696)(532.81132693,1037.52924696)(532.88133057,1037.51925148)
\curveto(532.95132679,1037.51924697)(533.02632672,1037.52924696)(533.10633057,1037.54925148)
\curveto(533.17632657,1037.56924692)(533.2463265,1037.57924691)(533.31633057,1037.57925148)
\curveto(533.38632636,1037.57924691)(533.46132628,1037.5892469)(533.54133057,1037.60925148)
\curveto(533.75132599,1037.65924683)(533.9413258,1037.69924679)(534.11133057,1037.72925148)
\curveto(534.29132545,1037.76924672)(534.45132529,1037.85924663)(534.59133057,1037.99925148)
\curveto(534.68132506,1038.0892464)(534.741325,1038.1892463)(534.77133057,1038.29925148)
\curveto(534.78132496,1038.32924616)(534.78132496,1038.35424613)(534.77133057,1038.37425148)
\curveto(534.77132497,1038.39424609)(534.77632497,1038.41424607)(534.78633057,1038.43425148)
\curveto(534.79632495,1038.45424603)(534.80132494,1038.484246)(534.80133057,1038.52425148)
\lineto(534.80133057,1038.61425148)
\lineto(534.77133057,1038.73425148)
\curveto(534.77132497,1038.77424571)(534.76632498,1038.80924568)(534.75633057,1038.83925148)
\curveto(534.65632509,1039.13924535)(534.4463253,1039.34424514)(534.12633057,1039.45425148)
\curveto(534.03632571,1039.484245)(533.92632582,1039.50424498)(533.79633057,1039.51425148)
\curveto(533.67632607,1039.53424495)(533.55132619,1039.53924495)(533.42133057,1039.52925148)
\curveto(533.29132645,1039.52924496)(533.16632658,1039.51924497)(533.04633057,1039.49925148)
\curveto(532.92632682,1039.47924501)(532.82132692,1039.45424503)(532.73133057,1039.42425148)
\curveto(532.67132707,1039.40424508)(532.61132713,1039.37424511)(532.55133057,1039.33425148)
\curveto(532.50132724,1039.30424518)(532.45132729,1039.26924522)(532.40133057,1039.22925148)
\curveto(532.35132739,1039.1892453)(532.29632745,1039.13424535)(532.23633057,1039.06425148)
\curveto(532.18632756,1038.99424549)(532.15132759,1038.92924556)(532.13133057,1038.86925148)
\curveto(532.08132766,1038.76924572)(532.03632771,1038.67424581)(531.99633057,1038.58425148)
\curveto(531.96632778,1038.49424599)(531.89632785,1038.43424605)(531.78633057,1038.40425148)
\curveto(531.70632804,1038.3842461)(531.62132812,1038.37424611)(531.53133057,1038.37425148)
\lineto(531.26133057,1038.37425148)
\lineto(530.69133057,1038.37425148)
\curveto(530.6413291,1038.37424611)(530.59132915,1038.36924612)(530.54133057,1038.35925148)
\curveto(530.49132925,1038.35924613)(530.4463293,1038.36424612)(530.40633057,1038.37425148)
\lineto(530.27133057,1038.37425148)
\curveto(530.25132949,1038.3842461)(530.22632952,1038.3892461)(530.19633057,1038.38925148)
\curveto(530.16632958,1038.3892461)(530.1413296,1038.39924609)(530.12133057,1038.41925148)
\curveto(530.0413297,1038.43924605)(529.98632976,1038.50424598)(529.95633057,1038.61425148)
\curveto(529.9463298,1038.66424582)(529.9463298,1038.71424577)(529.95633057,1038.76425148)
\curveto(529.96632978,1038.81424567)(529.97632977,1038.85924563)(529.98633057,1038.89925148)
\curveto(530.01632973,1039.00924548)(530.0463297,1039.10924538)(530.07633057,1039.19925148)
\curveto(530.11632963,1039.29924519)(530.16132958,1039.3892451)(530.21133057,1039.46925148)
\lineto(530.30133057,1039.61925148)
\lineto(530.39133057,1039.76925148)
\curveto(530.47132927,1039.87924461)(530.57132917,1039.9842445)(530.69133057,1040.08425148)
\curveto(530.71132903,1040.09424439)(530.741329,1040.11924437)(530.78133057,1040.15925148)
\curveto(530.83132891,1040.19924429)(530.87632887,1040.23424425)(530.91633057,1040.26425148)
\curveto(530.95632879,1040.29424419)(531.00132874,1040.32424416)(531.05133057,1040.35425148)
\curveto(531.22132852,1040.46424402)(531.40132834,1040.54924394)(531.59133057,1040.60925148)
\curveto(531.78132796,1040.67924381)(531.97632777,1040.74424374)(532.17633057,1040.80425148)
\curveto(532.29632745,1040.83424365)(532.42132732,1040.85424363)(532.55133057,1040.86425148)
\curveto(532.68132706,1040.87424361)(532.81132693,1040.89424359)(532.94133057,1040.92425148)
\curveto(532.98132676,1040.93424355)(533.0413267,1040.93424355)(533.12133057,1040.92425148)
\curveto(533.21132653,1040.91424357)(533.26632648,1040.91924357)(533.28633057,1040.93925148)
\curveto(533.69632605,1040.94924354)(534.08632566,1040.93424355)(534.45633057,1040.89425148)
\curveto(534.83632491,1040.85424363)(535.17632457,1040.77924371)(535.47633057,1040.66925148)
\curveto(535.78632396,1040.55924393)(536.05132369,1040.40924408)(536.27133057,1040.21925148)
\curveto(536.49132325,1040.03924445)(536.66132308,1039.80424468)(536.78133057,1039.51425148)
\curveto(536.85132289,1039.34424514)(536.89132285,1039.14924534)(536.90133057,1038.92925148)
\curveto(536.91132283,1038.70924578)(536.91632283,1038.484246)(536.91633057,1038.25425148)
\lineto(536.91633057,1034.90925148)
\lineto(536.91633057,1034.32425148)
\curveto(536.91632283,1034.13425035)(536.93632281,1033.95925053)(536.97633057,1033.79925148)
\curveto(536.98632276,1033.76925072)(536.99132275,1033.73425075)(536.99133057,1033.69425148)
\curveto(536.99132275,1033.66425082)(536.99632275,1033.63425085)(537.00633057,1033.60425148)
\moveto(534.80133057,1035.91425148)
\curveto(534.81132493,1035.96424852)(534.81632493,1036.01924847)(534.81633057,1036.07925148)
\curveto(534.81632493,1036.14924834)(534.81132493,1036.20924828)(534.80133057,1036.25925148)
\curveto(534.78132496,1036.31924817)(534.77132497,1036.37424811)(534.77133057,1036.42425148)
\curveto(534.77132497,1036.47424801)(534.75132499,1036.51424797)(534.71133057,1036.54425148)
\curveto(534.66132508,1036.5842479)(534.58632516,1036.60424788)(534.48633057,1036.60425148)
\curveto(534.4463253,1036.59424789)(534.41132533,1036.5842479)(534.38133057,1036.57425148)
\curveto(534.35132539,1036.57424791)(534.31632543,1036.56924792)(534.27633057,1036.55925148)
\curveto(534.20632554,1036.53924795)(534.13132561,1036.52424796)(534.05133057,1036.51425148)
\curveto(533.97132577,1036.50424798)(533.89132585,1036.489248)(533.81133057,1036.46925148)
\curveto(533.78132596,1036.45924803)(533.73632601,1036.45424803)(533.67633057,1036.45425148)
\curveto(533.5463262,1036.42424806)(533.41632633,1036.40424808)(533.28633057,1036.39425148)
\curveto(533.15632659,1036.3842481)(533.03132671,1036.35924813)(532.91133057,1036.31925148)
\curveto(532.83132691,1036.29924819)(532.75632699,1036.27924821)(532.68633057,1036.25925148)
\curveto(532.61632713,1036.24924824)(532.5463272,1036.22924826)(532.47633057,1036.19925148)
\curveto(532.26632748,1036.10924838)(532.08632766,1035.97424851)(531.93633057,1035.79425148)
\curveto(531.79632795,1035.61424887)(531.746328,1035.36424912)(531.78633057,1035.04425148)
\curveto(531.80632794,1034.87424961)(531.86132788,1034.73424975)(531.95133057,1034.62425148)
\curveto(532.02132772,1034.51424997)(532.12632762,1034.42425006)(532.26633057,1034.35425148)
\curveto(532.40632734,1034.29425019)(532.55632719,1034.24925024)(532.71633057,1034.21925148)
\curveto(532.88632686,1034.1892503)(533.06132668,1034.17925031)(533.24133057,1034.18925148)
\curveto(533.43132631,1034.20925028)(533.60632614,1034.24425024)(533.76633057,1034.29425148)
\curveto(534.02632572,1034.37425011)(534.23132551,1034.49924999)(534.38133057,1034.66925148)
\curveto(534.53132521,1034.84924964)(534.6463251,1035.06924942)(534.72633057,1035.32925148)
\curveto(534.746325,1035.39924909)(534.75632499,1035.46924902)(534.75633057,1035.53925148)
\curveto(534.76632498,1035.61924887)(534.78132496,1035.69924879)(534.80133057,1035.77925148)
\lineto(534.80133057,1035.91425148)
}
}
{
\newrgbcolor{curcolor}{0 0 0}
\pscustom[linestyle=none,fillstyle=solid,fillcolor=curcolor]
{
\newpath
\moveto(542.99461182,1040.93925148)
\curveto(543.1046065,1040.93924355)(543.19960641,1040.92924356)(543.27961182,1040.90925148)
\curveto(543.36960624,1040.8892436)(543.43960617,1040.84424364)(543.48961182,1040.77425148)
\curveto(543.54960606,1040.69424379)(543.57960603,1040.55424393)(543.57961182,1040.35425148)
\lineto(543.57961182,1039.84425148)
\lineto(543.57961182,1039.46925148)
\curveto(543.58960602,1039.32924516)(543.57460603,1039.21924527)(543.53461182,1039.13925148)
\curveto(543.49460611,1039.06924542)(543.43460617,1039.02424546)(543.35461182,1039.00425148)
\curveto(543.28460632,1038.9842455)(543.19960641,1038.97424551)(543.09961182,1038.97425148)
\curveto(543.0096066,1038.97424551)(542.9096067,1038.97924551)(542.79961182,1038.98925148)
\curveto(542.69960691,1038.99924549)(542.604607,1038.99424549)(542.51461182,1038.97425148)
\curveto(542.44460716,1038.95424553)(542.37460723,1038.93924555)(542.30461182,1038.92925148)
\curveto(542.23460737,1038.92924556)(542.16960744,1038.91924557)(542.10961182,1038.89925148)
\curveto(541.94960766,1038.84924564)(541.78960782,1038.77424571)(541.62961182,1038.67425148)
\curveto(541.46960814,1038.5842459)(541.34460826,1038.47924601)(541.25461182,1038.35925148)
\curveto(541.2046084,1038.27924621)(541.14960846,1038.19424629)(541.08961182,1038.10425148)
\curveto(541.03960857,1038.02424646)(540.98960862,1037.93924655)(540.93961182,1037.84925148)
\curveto(540.9096087,1037.76924672)(540.87960873,1037.6842468)(540.84961182,1037.59425148)
\lineto(540.78961182,1037.35425148)
\curveto(540.76960884,1037.2842472)(540.75960885,1037.20924728)(540.75961182,1037.12925148)
\curveto(540.75960885,1037.05924743)(540.74960886,1036.9892475)(540.72961182,1036.91925148)
\curveto(540.71960889,1036.87924761)(540.71460889,1036.83924765)(540.71461182,1036.79925148)
\curveto(540.72460888,1036.76924772)(540.72460888,1036.73924775)(540.71461182,1036.70925148)
\lineto(540.71461182,1036.46925148)
\curveto(540.69460891,1036.39924809)(540.68960892,1036.31924817)(540.69961182,1036.22925148)
\curveto(540.7096089,1036.14924834)(540.71460889,1036.06924842)(540.71461182,1035.98925148)
\lineto(540.71461182,1035.02925148)
\lineto(540.71461182,1033.75425148)
\curveto(540.71460889,1033.62425086)(540.7096089,1033.50425098)(540.69961182,1033.39425148)
\curveto(540.68960892,1033.2842512)(540.65960895,1033.19425129)(540.60961182,1033.12425148)
\curveto(540.58960902,1033.09425139)(540.55460905,1033.06925142)(540.50461182,1033.04925148)
\curveto(540.46460914,1033.03925145)(540.41960919,1033.02925146)(540.36961182,1033.01925148)
\lineto(540.29461182,1033.01925148)
\curveto(540.24460936,1033.00925148)(540.18960942,1033.00425148)(540.12961182,1033.00425148)
\lineto(539.96461182,1033.00425148)
\lineto(539.31961182,1033.00425148)
\curveto(539.25961035,1033.01425147)(539.19461041,1033.01925147)(539.12461182,1033.01925148)
\lineto(538.92961182,1033.01925148)
\curveto(538.87961073,1033.03925145)(538.82961078,1033.05425143)(538.77961182,1033.06425148)
\curveto(538.72961088,1033.0842514)(538.69461091,1033.11925137)(538.67461182,1033.16925148)
\curveto(538.63461097,1033.21925127)(538.609611,1033.2892512)(538.59961182,1033.37925148)
\lineto(538.59961182,1033.67925148)
\lineto(538.59961182,1034.69925148)
\lineto(538.59961182,1038.92925148)
\lineto(538.59961182,1040.03925148)
\lineto(538.59961182,1040.32425148)
\curveto(538.59961101,1040.42424406)(538.61961099,1040.50424398)(538.65961182,1040.56425148)
\curveto(538.7096109,1040.64424384)(538.78461082,1040.69424379)(538.88461182,1040.71425148)
\curveto(538.98461062,1040.73424375)(539.1046105,1040.74424374)(539.24461182,1040.74425148)
\lineto(540.00961182,1040.74425148)
\curveto(540.12960948,1040.74424374)(540.23460937,1040.73424375)(540.32461182,1040.71425148)
\curveto(540.41460919,1040.70424378)(540.48460912,1040.65924383)(540.53461182,1040.57925148)
\curveto(540.56460904,1040.52924396)(540.57960903,1040.45924403)(540.57961182,1040.36925148)
\lineto(540.60961182,1040.09925148)
\curveto(540.61960899,1040.01924447)(540.63460897,1039.94424454)(540.65461182,1039.87425148)
\curveto(540.68460892,1039.80424468)(540.73460887,1039.76924472)(540.80461182,1039.76925148)
\curveto(540.82460878,1039.7892447)(540.84460876,1039.79924469)(540.86461182,1039.79925148)
\curveto(540.88460872,1039.79924469)(540.9046087,1039.80924468)(540.92461182,1039.82925148)
\curveto(540.98460862,1039.87924461)(541.03460857,1039.93424455)(541.07461182,1039.99425148)
\curveto(541.12460848,1040.06424442)(541.18460842,1040.12424436)(541.25461182,1040.17425148)
\curveto(541.29460831,1040.20424428)(541.32960828,1040.23424425)(541.35961182,1040.26425148)
\curveto(541.38960822,1040.30424418)(541.42460818,1040.33924415)(541.46461182,1040.36925148)
\lineto(541.73461182,1040.54925148)
\curveto(541.83460777,1040.60924388)(541.93460767,1040.66424382)(542.03461182,1040.71425148)
\curveto(542.13460747,1040.75424373)(542.23460737,1040.7892437)(542.33461182,1040.81925148)
\lineto(542.66461182,1040.90925148)
\curveto(542.69460691,1040.91924357)(542.74960686,1040.91924357)(542.82961182,1040.90925148)
\curveto(542.91960669,1040.90924358)(542.97460663,1040.91924357)(542.99461182,1040.93925148)
}
}
{
\newrgbcolor{curcolor}{0 0 0}
\pscustom[linestyle=none,fillstyle=solid,fillcolor=curcolor]
{
\newpath
\moveto(551.90601807,1037.18925148)
\curveto(551.9260095,1037.12924736)(551.93600949,1037.04424744)(551.93601807,1036.93425148)
\curveto(551.93600949,1036.82424766)(551.9260095,1036.73924775)(551.90601807,1036.67925148)
\lineto(551.90601807,1036.52925148)
\curveto(551.88600954,1036.44924804)(551.87600955,1036.36924812)(551.87601807,1036.28925148)
\curveto(551.88600954,1036.20924828)(551.88100954,1036.12924836)(551.86101807,1036.04925148)
\curveto(551.84100958,1035.97924851)(551.8260096,1035.91424857)(551.81601807,1035.85425148)
\curveto(551.80600962,1035.79424869)(551.79600963,1035.72924876)(551.78601807,1035.65925148)
\curveto(551.74600968,1035.54924894)(551.71100971,1035.43424905)(551.68101807,1035.31425148)
\curveto(551.65100977,1035.20424928)(551.61100981,1035.09924939)(551.56101807,1034.99925148)
\curveto(551.35101007,1034.51924997)(551.07601035,1034.12925036)(550.73601807,1033.82925148)
\curveto(550.39601103,1033.52925096)(549.98601144,1033.27925121)(549.50601807,1033.07925148)
\curveto(549.38601204,1033.02925146)(549.26101216,1032.99425149)(549.13101807,1032.97425148)
\curveto(549.01101241,1032.94425154)(548.88601254,1032.91425157)(548.75601807,1032.88425148)
\curveto(548.70601272,1032.86425162)(548.65101277,1032.85425163)(548.59101807,1032.85425148)
\curveto(548.53101289,1032.85425163)(548.47601295,1032.84925164)(548.42601807,1032.83925148)
\lineto(548.32101807,1032.83925148)
\curveto(548.29101313,1032.82925166)(548.26101316,1032.82425166)(548.23101807,1032.82425148)
\curveto(548.18101324,1032.81425167)(548.10101332,1032.80925168)(547.99101807,1032.80925148)
\curveto(547.88101354,1032.79925169)(547.79601363,1032.80425168)(547.73601807,1032.82425148)
\lineto(547.58601807,1032.82425148)
\curveto(547.53601389,1032.83425165)(547.48101394,1032.83925165)(547.42101807,1032.83925148)
\curveto(547.37101405,1032.82925166)(547.3210141,1032.83425165)(547.27101807,1032.85425148)
\curveto(547.23101419,1032.86425162)(547.19101423,1032.86925162)(547.15101807,1032.86925148)
\curveto(547.1210143,1032.86925162)(547.08101434,1032.87425161)(547.03101807,1032.88425148)
\curveto(546.93101449,1032.91425157)(546.83101459,1032.93925155)(546.73101807,1032.95925148)
\curveto(546.63101479,1032.97925151)(546.53601489,1033.00925148)(546.44601807,1033.04925148)
\curveto(546.3260151,1033.0892514)(546.21101521,1033.12925136)(546.10101807,1033.16925148)
\curveto(546.00101542,1033.20925128)(545.89601553,1033.25925123)(545.78601807,1033.31925148)
\curveto(545.43601599,1033.52925096)(545.13601629,1033.77425071)(544.88601807,1034.05425148)
\curveto(544.63601679,1034.33425015)(544.426017,1034.66924982)(544.25601807,1035.05925148)
\curveto(544.20601722,1035.14924934)(544.16601726,1035.24424924)(544.13601807,1035.34425148)
\curveto(544.11601731,1035.44424904)(544.09101733,1035.54924894)(544.06101807,1035.65925148)
\curveto(544.04101738,1035.70924878)(544.03101739,1035.75424873)(544.03101807,1035.79425148)
\curveto(544.03101739,1035.83424865)(544.0210174,1035.87924861)(544.00101807,1035.92925148)
\curveto(543.98101744,1036.00924848)(543.97101745,1036.0892484)(543.97101807,1036.16925148)
\curveto(543.97101745,1036.25924823)(543.96101746,1036.34424814)(543.94101807,1036.42425148)
\curveto(543.93101749,1036.47424801)(543.9260175,1036.51924797)(543.92601807,1036.55925148)
\lineto(543.92601807,1036.69425148)
\curveto(543.90601752,1036.75424773)(543.89601753,1036.83924765)(543.89601807,1036.94925148)
\curveto(543.90601752,1037.05924743)(543.9210175,1037.14424734)(543.94101807,1037.20425148)
\lineto(543.94101807,1037.30925148)
\curveto(543.95101747,1037.35924713)(543.95101747,1037.40924708)(543.94101807,1037.45925148)
\curveto(543.94101748,1037.51924697)(543.95101747,1037.57424691)(543.97101807,1037.62425148)
\curveto(543.98101744,1037.67424681)(543.98601744,1037.71924677)(543.98601807,1037.75925148)
\curveto(543.98601744,1037.80924668)(543.99601743,1037.85924663)(544.01601807,1037.90925148)
\curveto(544.05601737,1038.03924645)(544.09101733,1038.16424632)(544.12101807,1038.28425148)
\curveto(544.15101727,1038.41424607)(544.19101723,1038.53924595)(544.24101807,1038.65925148)
\curveto(544.421017,1039.06924542)(544.63601679,1039.40924508)(544.88601807,1039.67925148)
\curveto(545.13601629,1039.95924453)(545.44101598,1040.21424427)(545.80101807,1040.44425148)
\curveto(545.90101552,1040.49424399)(546.00601542,1040.53924395)(546.11601807,1040.57925148)
\curveto(546.2260152,1040.61924387)(546.33601509,1040.66424382)(546.44601807,1040.71425148)
\curveto(546.57601485,1040.76424372)(546.71101471,1040.79924369)(546.85101807,1040.81925148)
\curveto(546.99101443,1040.83924365)(547.13601429,1040.86924362)(547.28601807,1040.90925148)
\curveto(547.36601406,1040.91924357)(547.44101398,1040.92424356)(547.51101807,1040.92425148)
\curveto(547.58101384,1040.92424356)(547.65101377,1040.92924356)(547.72101807,1040.93925148)
\curveto(548.30101312,1040.94924354)(548.80101262,1040.8892436)(549.22101807,1040.75925148)
\curveto(549.65101177,1040.62924386)(550.03101139,1040.44924404)(550.36101807,1040.21925148)
\curveto(550.47101095,1040.13924435)(550.58101084,1040.04924444)(550.69101807,1039.94925148)
\curveto(550.81101061,1039.85924463)(550.91101051,1039.75924473)(550.99101807,1039.64925148)
\curveto(551.07101035,1039.54924494)(551.14101028,1039.44924504)(551.20101807,1039.34925148)
\curveto(551.27101015,1039.24924524)(551.34101008,1039.14424534)(551.41101807,1039.03425148)
\curveto(551.48100994,1038.92424556)(551.53600989,1038.80424568)(551.57601807,1038.67425148)
\curveto(551.61600981,1038.55424593)(551.66100976,1038.42424606)(551.71101807,1038.28425148)
\curveto(551.74100968,1038.20424628)(551.76600966,1038.11924637)(551.78601807,1038.02925148)
\lineto(551.84601807,1037.75925148)
\curveto(551.85600957,1037.71924677)(551.86100956,1037.67924681)(551.86101807,1037.63925148)
\curveto(551.86100956,1037.59924689)(551.86600956,1037.55924693)(551.87601807,1037.51925148)
\curveto(551.89600953,1037.46924702)(551.90100952,1037.41424707)(551.89101807,1037.35425148)
\curveto(551.88100954,1037.29424719)(551.88600954,1037.23924725)(551.90601807,1037.18925148)
\moveto(549.80601807,1036.64925148)
\curveto(549.81601161,1036.69924779)(549.8210116,1036.76924772)(549.82101807,1036.85925148)
\curveto(549.8210116,1036.95924753)(549.81601161,1037.03424745)(549.80601807,1037.08425148)
\lineto(549.80601807,1037.20425148)
\curveto(549.78601164,1037.25424723)(549.77601165,1037.30924718)(549.77601807,1037.36925148)
\curveto(549.77601165,1037.42924706)(549.77101165,1037.484247)(549.76101807,1037.53425148)
\curveto(549.76101166,1037.57424691)(549.75601167,1037.60424688)(549.74601807,1037.62425148)
\lineto(549.68601807,1037.86425148)
\curveto(549.67601175,1037.95424653)(549.65601177,1038.03924645)(549.62601807,1038.11925148)
\curveto(549.51601191,1038.37924611)(549.38601204,1038.59924589)(549.23601807,1038.77925148)
\curveto(549.08601234,1038.96924552)(548.88601254,1039.11924537)(548.63601807,1039.22925148)
\curveto(548.57601285,1039.24924524)(548.51601291,1039.26424522)(548.45601807,1039.27425148)
\curveto(548.39601303,1039.29424519)(548.33101309,1039.31424517)(548.26101807,1039.33425148)
\curveto(548.18101324,1039.35424513)(548.09601333,1039.35924513)(548.00601807,1039.34925148)
\lineto(547.73601807,1039.34925148)
\curveto(547.70601372,1039.32924516)(547.67101375,1039.31924517)(547.63101807,1039.31925148)
\curveto(547.59101383,1039.32924516)(547.55601387,1039.32924516)(547.52601807,1039.31925148)
\lineto(547.31601807,1039.25925148)
\curveto(547.25601417,1039.24924524)(547.20101422,1039.22924526)(547.15101807,1039.19925148)
\curveto(546.90101452,1039.0892454)(546.69601473,1038.92924556)(546.53601807,1038.71925148)
\curveto(546.38601504,1038.51924597)(546.26601516,1038.2842462)(546.17601807,1038.01425148)
\curveto(546.14601528,1037.91424657)(546.1210153,1037.80924668)(546.10101807,1037.69925148)
\curveto(546.09101533,1037.5892469)(546.07601535,1037.47924701)(546.05601807,1037.36925148)
\curveto(546.04601538,1037.31924717)(546.04101538,1037.26924722)(546.04101807,1037.21925148)
\lineto(546.04101807,1037.06925148)
\curveto(546.0210154,1036.99924749)(546.01101541,1036.89424759)(546.01101807,1036.75425148)
\curveto(546.0210154,1036.61424787)(546.03601539,1036.50924798)(546.05601807,1036.43925148)
\lineto(546.05601807,1036.30425148)
\curveto(546.07601535,1036.22424826)(546.09101533,1036.14424834)(546.10101807,1036.06425148)
\curveto(546.11101531,1035.99424849)(546.1260153,1035.91924857)(546.14601807,1035.83925148)
\curveto(546.24601518,1035.53924895)(546.35101507,1035.29424919)(546.46101807,1035.10425148)
\curveto(546.58101484,1034.92424956)(546.76601466,1034.75924973)(547.01601807,1034.60925148)
\curveto(547.08601434,1034.55924993)(547.16101426,1034.51924997)(547.24101807,1034.48925148)
\curveto(547.33101409,1034.45925003)(547.421014,1034.43425005)(547.51101807,1034.41425148)
\curveto(547.55101387,1034.40425008)(547.58601384,1034.39925009)(547.61601807,1034.39925148)
\curveto(547.64601378,1034.40925008)(547.68101374,1034.40925008)(547.72101807,1034.39925148)
\lineto(547.84101807,1034.36925148)
\curveto(547.89101353,1034.36925012)(547.93601349,1034.37425011)(547.97601807,1034.38425148)
\lineto(548.09601807,1034.38425148)
\curveto(548.17601325,1034.40425008)(548.25601317,1034.41925007)(548.33601807,1034.42925148)
\curveto(548.41601301,1034.43925005)(548.49101293,1034.45925003)(548.56101807,1034.48925148)
\curveto(548.8210126,1034.5892499)(549.03101239,1034.72424976)(549.19101807,1034.89425148)
\curveto(549.35101207,1035.06424942)(549.48601194,1035.27424921)(549.59601807,1035.52425148)
\curveto(549.63601179,1035.62424886)(549.66601176,1035.72424876)(549.68601807,1035.82425148)
\curveto(549.70601172,1035.92424856)(549.73101169,1036.02924846)(549.76101807,1036.13925148)
\curveto(549.77101165,1036.17924831)(549.77601165,1036.21424827)(549.77601807,1036.24425148)
\curveto(549.77601165,1036.2842482)(549.78101164,1036.32424816)(549.79101807,1036.36425148)
\lineto(549.79101807,1036.49925148)
\curveto(549.79101163,1036.54924794)(549.79601163,1036.59924789)(549.80601807,1036.64925148)
}
}
{
\newrgbcolor{curcolor}{0 0 0}
\pscustom[linestyle=none,fillstyle=solid,fillcolor=curcolor]
{
\newpath
\moveto(557.73093994,1040.93925148)
\curveto(558.33093414,1040.95924353)(558.83093364,1040.87424361)(559.23093994,1040.68425148)
\curveto(559.63093284,1040.49424399)(559.94593252,1040.21424427)(560.17593994,1039.84425148)
\curveto(560.24593222,1039.73424475)(560.30093217,1039.61424487)(560.34093994,1039.48425148)
\curveto(560.38093209,1039.36424512)(560.42093205,1039.23924525)(560.46093994,1039.10925148)
\curveto(560.48093199,1039.02924546)(560.49093198,1038.95424553)(560.49093994,1038.88425148)
\curveto(560.50093197,1038.81424567)(560.51593195,1038.74424574)(560.53593994,1038.67425148)
\curveto(560.53593193,1038.61424587)(560.54093193,1038.57424591)(560.55093994,1038.55425148)
\curveto(560.5709319,1038.41424607)(560.58093189,1038.26924622)(560.58093994,1038.11925148)
\lineto(560.58093994,1037.68425148)
\lineto(560.58093994,1036.34925148)
\lineto(560.58093994,1033.91925148)
\curveto(560.58093189,1033.72925076)(560.57593189,1033.54425094)(560.56593994,1033.36425148)
\curveto(560.5659319,1033.19425129)(560.49593197,1033.0842514)(560.35593994,1033.03425148)
\curveto(560.29593217,1033.01425147)(560.22593224,1033.00425148)(560.14593994,1033.00425148)
\lineto(559.90593994,1033.00425148)
\lineto(559.09593994,1033.00425148)
\curveto(558.97593349,1033.00425148)(558.8659336,1033.00925148)(558.76593994,1033.01925148)
\curveto(558.67593379,1033.03925145)(558.60593386,1033.0842514)(558.55593994,1033.15425148)
\curveto(558.51593395,1033.21425127)(558.49093398,1033.2892512)(558.48093994,1033.37925148)
\lineto(558.48093994,1033.69425148)
\lineto(558.48093994,1034.74425148)
\lineto(558.48093994,1036.97925148)
\curveto(558.48093399,1037.34924714)(558.465934,1037.6892468)(558.43593994,1037.99925148)
\curveto(558.40593406,1038.31924617)(558.31593415,1038.5892459)(558.16593994,1038.80925148)
\curveto(558.02593444,1039.00924548)(557.82093465,1039.14924534)(557.55093994,1039.22925148)
\curveto(557.50093497,1039.24924524)(557.44593502,1039.25924523)(557.38593994,1039.25925148)
\curveto(557.33593513,1039.25924523)(557.28093519,1039.26924522)(557.22093994,1039.28925148)
\curveto(557.1709353,1039.29924519)(557.10593536,1039.29924519)(557.02593994,1039.28925148)
\curveto(556.95593551,1039.2892452)(556.90093557,1039.2842452)(556.86093994,1039.27425148)
\curveto(556.82093565,1039.26424522)(556.78593568,1039.25924523)(556.75593994,1039.25925148)
\curveto(556.72593574,1039.25924523)(556.69593577,1039.25424523)(556.66593994,1039.24425148)
\curveto(556.43593603,1039.1842453)(556.25093622,1039.10424538)(556.11093994,1039.00425148)
\curveto(555.79093668,1038.77424571)(555.60093687,1038.43924605)(555.54093994,1037.99925148)
\curveto(555.48093699,1037.55924693)(555.45093702,1037.06424742)(555.45093994,1036.51425148)
\lineto(555.45093994,1034.63925148)
\lineto(555.45093994,1033.72425148)
\lineto(555.45093994,1033.45425148)
\curveto(555.45093702,1033.36425112)(555.43593703,1033.2892512)(555.40593994,1033.22925148)
\curveto(555.35593711,1033.11925137)(555.27593719,1033.05425143)(555.16593994,1033.03425148)
\curveto(555.05593741,1033.01425147)(554.92093755,1033.00425148)(554.76093994,1033.00425148)
\lineto(554.01093994,1033.00425148)
\curveto(553.90093857,1033.00425148)(553.79093868,1033.00925148)(553.68093994,1033.01925148)
\curveto(553.5709389,1033.02925146)(553.49093898,1033.06425142)(553.44093994,1033.12425148)
\curveto(553.3709391,1033.21425127)(553.33593913,1033.34425114)(553.33593994,1033.51425148)
\curveto(553.34593912,1033.6842508)(553.35093912,1033.84425064)(553.35093994,1033.99425148)
\lineto(553.35093994,1036.03425148)
\lineto(553.35093994,1039.33425148)
\lineto(553.35093994,1040.09925148)
\lineto(553.35093994,1040.39925148)
\curveto(553.36093911,1040.489244)(553.39093908,1040.56424392)(553.44093994,1040.62425148)
\curveto(553.46093901,1040.65424383)(553.49093898,1040.67424381)(553.53093994,1040.68425148)
\curveto(553.58093889,1040.70424378)(553.63093884,1040.71924377)(553.68093994,1040.72925148)
\lineto(553.75593994,1040.72925148)
\curveto(553.80593866,1040.73924375)(553.85593861,1040.74424374)(553.90593994,1040.74425148)
\lineto(554.07093994,1040.74425148)
\lineto(554.70093994,1040.74425148)
\curveto(554.78093769,1040.74424374)(554.85593761,1040.73924375)(554.92593994,1040.72925148)
\curveto(555.00593746,1040.72924376)(555.07593739,1040.71924377)(555.13593994,1040.69925148)
\curveto(555.20593726,1040.66924382)(555.25093722,1040.62424386)(555.27093994,1040.56425148)
\curveto(555.30093717,1040.50424398)(555.32593714,1040.43424405)(555.34593994,1040.35425148)
\curveto(555.35593711,1040.31424417)(555.35593711,1040.27924421)(555.34593994,1040.24925148)
\curveto(555.34593712,1040.21924427)(555.35593711,1040.1892443)(555.37593994,1040.15925148)
\curveto(555.39593707,1040.10924438)(555.41093706,1040.07924441)(555.42093994,1040.06925148)
\curveto(555.44093703,1040.05924443)(555.465937,1040.04424444)(555.49593994,1040.02425148)
\curveto(555.60593686,1040.01424447)(555.69593677,1040.04924444)(555.76593994,1040.12925148)
\curveto(555.83593663,1040.21924427)(555.91093656,1040.2892442)(555.99093994,1040.33925148)
\curveto(556.26093621,1040.53924395)(556.56093591,1040.69924379)(556.89093994,1040.81925148)
\curveto(556.98093549,1040.84924364)(557.0709354,1040.86924362)(557.16093994,1040.87925148)
\curveto(557.26093521,1040.8892436)(557.3659351,1040.90424358)(557.47593994,1040.92425148)
\curveto(557.50593496,1040.93424355)(557.55093492,1040.93424355)(557.61093994,1040.92425148)
\curveto(557.6709348,1040.92424356)(557.71093476,1040.92924356)(557.73093994,1040.93925148)
}
}
{
\newrgbcolor{curcolor}{0 0 0}
\pscustom[linestyle=none,fillstyle=solid,fillcolor=curcolor]
{
}
}
{
\newrgbcolor{curcolor}{0 0 0}
\pscustom[linestyle=none,fillstyle=solid,fillcolor=curcolor]
{
\newpath
\moveto(573.42234619,1033.60425148)
\curveto(573.44233834,1033.49425099)(573.45233833,1033.3842511)(573.45234619,1033.27425148)
\curveto(573.46233832,1033.16425132)(573.41233837,1033.0892514)(573.30234619,1033.04925148)
\curveto(573.24233854,1033.01925147)(573.17233861,1033.00425148)(573.09234619,1033.00425148)
\lineto(572.85234619,1033.00425148)
\lineto(572.04234619,1033.00425148)
\lineto(571.77234619,1033.00425148)
\curveto(571.69234009,1033.01425147)(571.62734016,1033.03925145)(571.57734619,1033.07925148)
\curveto(571.50734028,1033.11925137)(571.45234033,1033.17425131)(571.41234619,1033.24425148)
\curveto(571.3823404,1033.32425116)(571.33734045,1033.3892511)(571.27734619,1033.43925148)
\curveto(571.25734053,1033.45925103)(571.23234055,1033.47425101)(571.20234619,1033.48425148)
\curveto(571.17234061,1033.50425098)(571.13234065,1033.50925098)(571.08234619,1033.49925148)
\curveto(571.03234075,1033.47925101)(570.9823408,1033.45425103)(570.93234619,1033.42425148)
\curveto(570.89234089,1033.39425109)(570.84734094,1033.36925112)(570.79734619,1033.34925148)
\curveto(570.74734104,1033.30925118)(570.69234109,1033.27425121)(570.63234619,1033.24425148)
\lineto(570.45234619,1033.15425148)
\curveto(570.32234146,1033.09425139)(570.1873416,1033.04425144)(570.04734619,1033.00425148)
\curveto(569.90734188,1032.97425151)(569.76234202,1032.93925155)(569.61234619,1032.89925148)
\curveto(569.54234224,1032.87925161)(569.47234231,1032.86925162)(569.40234619,1032.86925148)
\curveto(569.34234244,1032.85925163)(569.27734251,1032.84925164)(569.20734619,1032.83925148)
\lineto(569.11734619,1032.83925148)
\curveto(569.0873427,1032.82925166)(569.05734273,1032.82425166)(569.02734619,1032.82425148)
\lineto(568.86234619,1032.82425148)
\curveto(568.76234302,1032.80425168)(568.66234312,1032.80425168)(568.56234619,1032.82425148)
\lineto(568.42734619,1032.82425148)
\curveto(568.35734343,1032.84425164)(568.2873435,1032.85425163)(568.21734619,1032.85425148)
\curveto(568.15734363,1032.84425164)(568.09734369,1032.84925164)(568.03734619,1032.86925148)
\curveto(567.93734385,1032.8892516)(567.84234394,1032.90925158)(567.75234619,1032.92925148)
\curveto(567.66234412,1032.93925155)(567.57734421,1032.96425152)(567.49734619,1033.00425148)
\curveto(567.20734458,1033.11425137)(566.95734483,1033.25425123)(566.74734619,1033.42425148)
\curveto(566.54734524,1033.60425088)(566.3873454,1033.83925065)(566.26734619,1034.12925148)
\curveto(566.23734555,1034.19925029)(566.20734558,1034.27425021)(566.17734619,1034.35425148)
\curveto(566.15734563,1034.43425005)(566.13734565,1034.51924997)(566.11734619,1034.60925148)
\curveto(566.09734569,1034.65924983)(566.0873457,1034.70924978)(566.08734619,1034.75925148)
\curveto(566.09734569,1034.80924968)(566.09734569,1034.85924963)(566.08734619,1034.90925148)
\curveto(566.07734571,1034.93924955)(566.06734572,1034.99924949)(566.05734619,1035.08925148)
\curveto(566.05734573,1035.1892493)(566.06234572,1035.25924923)(566.07234619,1035.29925148)
\curveto(566.09234569,1035.39924909)(566.10234568,1035.484249)(566.10234619,1035.55425148)
\lineto(566.19234619,1035.88425148)
\curveto(566.22234556,1036.00424848)(566.26234552,1036.10924838)(566.31234619,1036.19925148)
\curveto(566.4823453,1036.489248)(566.67734511,1036.70924778)(566.89734619,1036.85925148)
\curveto(567.11734467,1037.00924748)(567.39734439,1037.13924735)(567.73734619,1037.24925148)
\curveto(567.86734392,1037.29924719)(568.00234378,1037.33424715)(568.14234619,1037.35425148)
\curveto(568.2823435,1037.37424711)(568.42234336,1037.39924709)(568.56234619,1037.42925148)
\curveto(568.64234314,1037.44924704)(568.72734306,1037.45924703)(568.81734619,1037.45925148)
\curveto(568.90734288,1037.46924702)(568.99734279,1037.484247)(569.08734619,1037.50425148)
\curveto(569.15734263,1037.52424696)(569.22734256,1037.52924696)(569.29734619,1037.51925148)
\curveto(569.36734242,1037.51924697)(569.44234234,1037.52924696)(569.52234619,1037.54925148)
\curveto(569.59234219,1037.56924692)(569.66234212,1037.57924691)(569.73234619,1037.57925148)
\curveto(569.80234198,1037.57924691)(569.87734191,1037.5892469)(569.95734619,1037.60925148)
\curveto(570.16734162,1037.65924683)(570.35734143,1037.69924679)(570.52734619,1037.72925148)
\curveto(570.70734108,1037.76924672)(570.86734092,1037.85924663)(571.00734619,1037.99925148)
\curveto(571.09734069,1038.0892464)(571.15734063,1038.1892463)(571.18734619,1038.29925148)
\curveto(571.19734059,1038.32924616)(571.19734059,1038.35424613)(571.18734619,1038.37425148)
\curveto(571.1873406,1038.39424609)(571.19234059,1038.41424607)(571.20234619,1038.43425148)
\curveto(571.21234057,1038.45424603)(571.21734057,1038.484246)(571.21734619,1038.52425148)
\lineto(571.21734619,1038.61425148)
\lineto(571.18734619,1038.73425148)
\curveto(571.1873406,1038.77424571)(571.1823406,1038.80924568)(571.17234619,1038.83925148)
\curveto(571.07234071,1039.13924535)(570.86234092,1039.34424514)(570.54234619,1039.45425148)
\curveto(570.45234133,1039.484245)(570.34234144,1039.50424498)(570.21234619,1039.51425148)
\curveto(570.09234169,1039.53424495)(569.96734182,1039.53924495)(569.83734619,1039.52925148)
\curveto(569.70734208,1039.52924496)(569.5823422,1039.51924497)(569.46234619,1039.49925148)
\curveto(569.34234244,1039.47924501)(569.23734255,1039.45424503)(569.14734619,1039.42425148)
\curveto(569.0873427,1039.40424508)(569.02734276,1039.37424511)(568.96734619,1039.33425148)
\curveto(568.91734287,1039.30424518)(568.86734292,1039.26924522)(568.81734619,1039.22925148)
\curveto(568.76734302,1039.1892453)(568.71234307,1039.13424535)(568.65234619,1039.06425148)
\curveto(568.60234318,1038.99424549)(568.56734322,1038.92924556)(568.54734619,1038.86925148)
\curveto(568.49734329,1038.76924572)(568.45234333,1038.67424581)(568.41234619,1038.58425148)
\curveto(568.3823434,1038.49424599)(568.31234347,1038.43424605)(568.20234619,1038.40425148)
\curveto(568.12234366,1038.3842461)(568.03734375,1038.37424611)(567.94734619,1038.37425148)
\lineto(567.67734619,1038.37425148)
\lineto(567.10734619,1038.37425148)
\curveto(567.05734473,1038.37424611)(567.00734478,1038.36924612)(566.95734619,1038.35925148)
\curveto(566.90734488,1038.35924613)(566.86234492,1038.36424612)(566.82234619,1038.37425148)
\lineto(566.68734619,1038.37425148)
\curveto(566.66734512,1038.3842461)(566.64234514,1038.3892461)(566.61234619,1038.38925148)
\curveto(566.5823452,1038.3892461)(566.55734523,1038.39924609)(566.53734619,1038.41925148)
\curveto(566.45734533,1038.43924605)(566.40234538,1038.50424598)(566.37234619,1038.61425148)
\curveto(566.36234542,1038.66424582)(566.36234542,1038.71424577)(566.37234619,1038.76425148)
\curveto(566.3823454,1038.81424567)(566.39234539,1038.85924563)(566.40234619,1038.89925148)
\curveto(566.43234535,1039.00924548)(566.46234532,1039.10924538)(566.49234619,1039.19925148)
\curveto(566.53234525,1039.29924519)(566.57734521,1039.3892451)(566.62734619,1039.46925148)
\lineto(566.71734619,1039.61925148)
\lineto(566.80734619,1039.76925148)
\curveto(566.8873449,1039.87924461)(566.9873448,1039.9842445)(567.10734619,1040.08425148)
\curveto(567.12734466,1040.09424439)(567.15734463,1040.11924437)(567.19734619,1040.15925148)
\curveto(567.24734454,1040.19924429)(567.29234449,1040.23424425)(567.33234619,1040.26425148)
\curveto(567.37234441,1040.29424419)(567.41734437,1040.32424416)(567.46734619,1040.35425148)
\curveto(567.63734415,1040.46424402)(567.81734397,1040.54924394)(568.00734619,1040.60925148)
\curveto(568.19734359,1040.67924381)(568.39234339,1040.74424374)(568.59234619,1040.80425148)
\curveto(568.71234307,1040.83424365)(568.83734295,1040.85424363)(568.96734619,1040.86425148)
\curveto(569.09734269,1040.87424361)(569.22734256,1040.89424359)(569.35734619,1040.92425148)
\curveto(569.39734239,1040.93424355)(569.45734233,1040.93424355)(569.53734619,1040.92425148)
\curveto(569.62734216,1040.91424357)(569.6823421,1040.91924357)(569.70234619,1040.93925148)
\curveto(570.11234167,1040.94924354)(570.50234128,1040.93424355)(570.87234619,1040.89425148)
\curveto(571.25234053,1040.85424363)(571.59234019,1040.77924371)(571.89234619,1040.66925148)
\curveto(572.20233958,1040.55924393)(572.46733932,1040.40924408)(572.68734619,1040.21925148)
\curveto(572.90733888,1040.03924445)(573.07733871,1039.80424468)(573.19734619,1039.51425148)
\curveto(573.26733852,1039.34424514)(573.30733848,1039.14924534)(573.31734619,1038.92925148)
\curveto(573.32733846,1038.70924578)(573.33233845,1038.484246)(573.33234619,1038.25425148)
\lineto(573.33234619,1034.90925148)
\lineto(573.33234619,1034.32425148)
\curveto(573.33233845,1034.13425035)(573.35233843,1033.95925053)(573.39234619,1033.79925148)
\curveto(573.40233838,1033.76925072)(573.40733838,1033.73425075)(573.40734619,1033.69425148)
\curveto(573.40733838,1033.66425082)(573.41233837,1033.63425085)(573.42234619,1033.60425148)
\moveto(571.21734619,1035.91425148)
\curveto(571.22734056,1035.96424852)(571.23234055,1036.01924847)(571.23234619,1036.07925148)
\curveto(571.23234055,1036.14924834)(571.22734056,1036.20924828)(571.21734619,1036.25925148)
\curveto(571.19734059,1036.31924817)(571.1873406,1036.37424811)(571.18734619,1036.42425148)
\curveto(571.1873406,1036.47424801)(571.16734062,1036.51424797)(571.12734619,1036.54425148)
\curveto(571.07734071,1036.5842479)(571.00234078,1036.60424788)(570.90234619,1036.60425148)
\curveto(570.86234092,1036.59424789)(570.82734096,1036.5842479)(570.79734619,1036.57425148)
\curveto(570.76734102,1036.57424791)(570.73234105,1036.56924792)(570.69234619,1036.55925148)
\curveto(570.62234116,1036.53924795)(570.54734124,1036.52424796)(570.46734619,1036.51425148)
\curveto(570.3873414,1036.50424798)(570.30734148,1036.489248)(570.22734619,1036.46925148)
\curveto(570.19734159,1036.45924803)(570.15234163,1036.45424803)(570.09234619,1036.45425148)
\curveto(569.96234182,1036.42424806)(569.83234195,1036.40424808)(569.70234619,1036.39425148)
\curveto(569.57234221,1036.3842481)(569.44734234,1036.35924813)(569.32734619,1036.31925148)
\curveto(569.24734254,1036.29924819)(569.17234261,1036.27924821)(569.10234619,1036.25925148)
\curveto(569.03234275,1036.24924824)(568.96234282,1036.22924826)(568.89234619,1036.19925148)
\curveto(568.6823431,1036.10924838)(568.50234328,1035.97424851)(568.35234619,1035.79425148)
\curveto(568.21234357,1035.61424887)(568.16234362,1035.36424912)(568.20234619,1035.04425148)
\curveto(568.22234356,1034.87424961)(568.27734351,1034.73424975)(568.36734619,1034.62425148)
\curveto(568.43734335,1034.51424997)(568.54234324,1034.42425006)(568.68234619,1034.35425148)
\curveto(568.82234296,1034.29425019)(568.97234281,1034.24925024)(569.13234619,1034.21925148)
\curveto(569.30234248,1034.1892503)(569.47734231,1034.17925031)(569.65734619,1034.18925148)
\curveto(569.84734194,1034.20925028)(570.02234176,1034.24425024)(570.18234619,1034.29425148)
\curveto(570.44234134,1034.37425011)(570.64734114,1034.49924999)(570.79734619,1034.66925148)
\curveto(570.94734084,1034.84924964)(571.06234072,1035.06924942)(571.14234619,1035.32925148)
\curveto(571.16234062,1035.39924909)(571.17234061,1035.46924902)(571.17234619,1035.53925148)
\curveto(571.1823406,1035.61924887)(571.19734059,1035.69924879)(571.21734619,1035.77925148)
\lineto(571.21734619,1035.91425148)
}
}
{
\newrgbcolor{curcolor}{0 0 0}
\pscustom[linestyle=none,fillstyle=solid,fillcolor=curcolor]
{
\newpath
\moveto(575.49562744,1043.69925148)
\lineto(576.59062744,1043.69925148)
\curveto(576.69062496,1043.69924079)(576.78562486,1043.69424079)(576.87562744,1043.68425148)
\curveto(576.96562468,1043.67424081)(577.03562461,1043.64424084)(577.08562744,1043.59425148)
\curveto(577.1456245,1043.52424096)(577.17562447,1043.42924106)(577.17562744,1043.30925148)
\curveto(577.18562446,1043.19924129)(577.19062446,1043.0842414)(577.19062744,1042.96425148)
\lineto(577.19062744,1041.62925148)
\lineto(577.19062744,1036.24425148)
\lineto(577.19062744,1033.94925148)
\lineto(577.19062744,1033.52925148)
\curveto(577.20062445,1033.37925111)(577.18062447,1033.26425122)(577.13062744,1033.18425148)
\curveto(577.08062457,1033.10425138)(576.99062466,1033.04925144)(576.86062744,1033.01925148)
\curveto(576.80062485,1032.99925149)(576.73062492,1032.99425149)(576.65062744,1033.00425148)
\curveto(576.58062507,1033.01425147)(576.51062514,1033.01925147)(576.44062744,1033.01925148)
\lineto(575.72062744,1033.01925148)
\curveto(575.61062604,1033.01925147)(575.51062614,1033.02425146)(575.42062744,1033.03425148)
\curveto(575.33062632,1033.04425144)(575.25562639,1033.07425141)(575.19562744,1033.12425148)
\curveto(575.13562651,1033.17425131)(575.10062655,1033.24925124)(575.09062744,1033.34925148)
\lineto(575.09062744,1033.67925148)
\lineto(575.09062744,1035.01425148)
\lineto(575.09062744,1040.63925148)
\lineto(575.09062744,1042.67925148)
\curveto(575.09062656,1042.80924168)(575.08562656,1042.96424152)(575.07562744,1043.14425148)
\curveto(575.07562657,1043.32424116)(575.10062655,1043.45424103)(575.15062744,1043.53425148)
\curveto(575.17062648,1043.57424091)(575.19562645,1043.60424088)(575.22562744,1043.62425148)
\lineto(575.34562744,1043.68425148)
\curveto(575.36562628,1043.6842408)(575.39062626,1043.6842408)(575.42062744,1043.68425148)
\curveto(575.4506262,1043.69424079)(575.47562617,1043.69924079)(575.49562744,1043.69925148)
}
}
{
\newrgbcolor{curcolor}{0 0 0}
\pscustom[linestyle=none,fillstyle=solid,fillcolor=curcolor]
{
}
}
{
\newrgbcolor{curcolor}{0 0 0}
\pscustom[linestyle=none,fillstyle=solid,fillcolor=curcolor]
{
\newpath
\moveto(585.98297119,1040.95425148)
\curveto(586.73296669,1040.97424351)(587.38296604,1040.8892436)(587.93297119,1040.69925148)
\curveto(588.49296493,1040.51924397)(588.91796451,1040.20424428)(589.20797119,1039.75425148)
\curveto(589.27796415,1039.64424484)(589.33796409,1039.52924496)(589.38797119,1039.40925148)
\curveto(589.44796398,1039.29924519)(589.49796393,1039.17424531)(589.53797119,1039.03425148)
\curveto(589.55796387,1038.97424551)(589.56796386,1038.90924558)(589.56797119,1038.83925148)
\curveto(589.56796386,1038.76924572)(589.55796387,1038.70924578)(589.53797119,1038.65925148)
\curveto(589.49796393,1038.59924589)(589.44296398,1038.55924593)(589.37297119,1038.53925148)
\curveto(589.3229641,1038.51924597)(589.26296416,1038.50924598)(589.19297119,1038.50925148)
\lineto(588.98297119,1038.50925148)
\lineto(588.32297119,1038.50925148)
\curveto(588.25296517,1038.50924598)(588.18296524,1038.50424598)(588.11297119,1038.49425148)
\curveto(588.04296538,1038.49424599)(587.97796545,1038.50424598)(587.91797119,1038.52425148)
\curveto(587.81796561,1038.54424594)(587.74296568,1038.5842459)(587.69297119,1038.64425148)
\curveto(587.64296578,1038.70424578)(587.59796583,1038.76424572)(587.55797119,1038.82425148)
\lineto(587.43797119,1039.03425148)
\curveto(587.40796602,1039.11424537)(587.35796607,1039.17924531)(587.28797119,1039.22925148)
\curveto(587.18796624,1039.30924518)(587.08796634,1039.36924512)(586.98797119,1039.40925148)
\curveto(586.89796653,1039.44924504)(586.78296664,1039.484245)(586.64297119,1039.51425148)
\curveto(586.57296685,1039.53424495)(586.46796696,1039.54924494)(586.32797119,1039.55925148)
\curveto(586.19796723,1039.56924492)(586.09796733,1039.56424492)(586.02797119,1039.54425148)
\lineto(585.92297119,1039.54425148)
\lineto(585.77297119,1039.51425148)
\curveto(585.73296769,1039.51424497)(585.68796774,1039.50924498)(585.63797119,1039.49925148)
\curveto(585.46796796,1039.44924504)(585.3279681,1039.37924511)(585.21797119,1039.28925148)
\curveto(585.11796831,1039.20924528)(585.04796838,1039.0842454)(585.00797119,1038.91425148)
\curveto(584.98796844,1038.84424564)(584.98796844,1038.77924571)(585.00797119,1038.71925148)
\curveto(585.0279684,1038.65924583)(585.04796838,1038.60924588)(585.06797119,1038.56925148)
\curveto(585.13796829,1038.44924604)(585.21796821,1038.35424613)(585.30797119,1038.28425148)
\curveto(585.40796802,1038.21424627)(585.5229679,1038.15424633)(585.65297119,1038.10425148)
\curveto(585.84296758,1038.02424646)(586.04796738,1037.95424653)(586.26797119,1037.89425148)
\lineto(586.95797119,1037.74425148)
\curveto(587.19796623,1037.70424678)(587.427966,1037.65424683)(587.64797119,1037.59425148)
\curveto(587.87796555,1037.54424694)(588.09296533,1037.47924701)(588.29297119,1037.39925148)
\curveto(588.38296504,1037.35924713)(588.46796496,1037.32424716)(588.54797119,1037.29425148)
\curveto(588.63796479,1037.27424721)(588.7229647,1037.23924725)(588.80297119,1037.18925148)
\curveto(588.99296443,1037.06924742)(589.16296426,1036.93924755)(589.31297119,1036.79925148)
\curveto(589.47296395,1036.65924783)(589.59796383,1036.484248)(589.68797119,1036.27425148)
\curveto(589.71796371,1036.20424828)(589.74296368,1036.13424835)(589.76297119,1036.06425148)
\curveto(589.78296364,1035.99424849)(589.80296362,1035.91924857)(589.82297119,1035.83925148)
\curveto(589.83296359,1035.77924871)(589.83796359,1035.6842488)(589.83797119,1035.55425148)
\curveto(589.84796358,1035.43424905)(589.84796358,1035.33924915)(589.83797119,1035.26925148)
\lineto(589.83797119,1035.19425148)
\curveto(589.81796361,1035.13424935)(589.80296362,1035.07424941)(589.79297119,1035.01425148)
\curveto(589.79296363,1034.96424952)(589.78796364,1034.91424957)(589.77797119,1034.86425148)
\curveto(589.70796372,1034.56424992)(589.59796383,1034.29925019)(589.44797119,1034.06925148)
\curveto(589.28796414,1033.82925066)(589.09296433,1033.63425085)(588.86297119,1033.48425148)
\curveto(588.63296479,1033.33425115)(588.37296505,1033.20425128)(588.08297119,1033.09425148)
\curveto(587.97296545,1033.04425144)(587.85296557,1033.00925148)(587.72297119,1032.98925148)
\curveto(587.60296582,1032.96925152)(587.48296594,1032.94425154)(587.36297119,1032.91425148)
\curveto(587.27296615,1032.89425159)(587.17796625,1032.8842516)(587.07797119,1032.88425148)
\curveto(586.98796644,1032.87425161)(586.89796653,1032.85925163)(586.80797119,1032.83925148)
\lineto(586.53797119,1032.83925148)
\curveto(586.47796695,1032.81925167)(586.37296705,1032.80925168)(586.22297119,1032.80925148)
\curveto(586.08296734,1032.80925168)(585.98296744,1032.81925167)(585.92297119,1032.83925148)
\curveto(585.89296753,1032.83925165)(585.85796757,1032.84425164)(585.81797119,1032.85425148)
\lineto(585.71297119,1032.85425148)
\curveto(585.59296783,1032.87425161)(585.47296795,1032.8892516)(585.35297119,1032.89925148)
\curveto(585.23296819,1032.90925158)(585.11796831,1032.92925156)(585.00797119,1032.95925148)
\curveto(584.61796881,1033.06925142)(584.27296915,1033.19425129)(583.97297119,1033.33425148)
\curveto(583.67296975,1033.484251)(583.41797001,1033.70425078)(583.20797119,1033.99425148)
\curveto(583.06797036,1034.1842503)(582.94797048,1034.40425008)(582.84797119,1034.65425148)
\curveto(582.8279706,1034.71424977)(582.80797062,1034.79424969)(582.78797119,1034.89425148)
\curveto(582.76797066,1034.94424954)(582.75297067,1035.01424947)(582.74297119,1035.10425148)
\curveto(582.73297069,1035.19424929)(582.73797069,1035.26924922)(582.75797119,1035.32925148)
\curveto(582.78797064,1035.39924909)(582.83797059,1035.44924904)(582.90797119,1035.47925148)
\curveto(582.95797047,1035.49924899)(583.01797041,1035.50924898)(583.08797119,1035.50925148)
\lineto(583.31297119,1035.50925148)
\lineto(584.01797119,1035.50925148)
\lineto(584.25797119,1035.50925148)
\curveto(584.33796909,1035.50924898)(584.40796902,1035.49924899)(584.46797119,1035.47925148)
\curveto(584.57796885,1035.43924905)(584.64796878,1035.37424911)(584.67797119,1035.28425148)
\curveto(584.71796871,1035.19424929)(584.76296866,1035.09924939)(584.81297119,1034.99925148)
\curveto(584.83296859,1034.94924954)(584.86796856,1034.8842496)(584.91797119,1034.80425148)
\curveto(584.97796845,1034.72424976)(585.0279684,1034.67424981)(585.06797119,1034.65425148)
\curveto(585.18796824,1034.55424993)(585.30296812,1034.47425001)(585.41297119,1034.41425148)
\curveto(585.5229679,1034.36425012)(585.66296776,1034.31425017)(585.83297119,1034.26425148)
\curveto(585.88296754,1034.24425024)(585.93296749,1034.23425025)(585.98297119,1034.23425148)
\curveto(586.03296739,1034.24425024)(586.08296734,1034.24425024)(586.13297119,1034.23425148)
\curveto(586.21296721,1034.21425027)(586.29796713,1034.20425028)(586.38797119,1034.20425148)
\curveto(586.48796694,1034.21425027)(586.57296685,1034.22925026)(586.64297119,1034.24925148)
\curveto(586.69296673,1034.25925023)(586.73796669,1034.26425022)(586.77797119,1034.26425148)
\curveto(586.8279666,1034.26425022)(586.87796655,1034.27425021)(586.92797119,1034.29425148)
\curveto(587.06796636,1034.34425014)(587.19296623,1034.40425008)(587.30297119,1034.47425148)
\curveto(587.422966,1034.54424994)(587.51796591,1034.63424985)(587.58797119,1034.74425148)
\curveto(587.63796579,1034.82424966)(587.67796575,1034.94924954)(587.70797119,1035.11925148)
\curveto(587.7279657,1035.1892493)(587.7279657,1035.25424923)(587.70797119,1035.31425148)
\curveto(587.68796574,1035.37424911)(587.66796576,1035.42424906)(587.64797119,1035.46425148)
\curveto(587.57796585,1035.60424888)(587.48796594,1035.70924878)(587.37797119,1035.77925148)
\curveto(587.27796615,1035.84924864)(587.15796627,1035.91424857)(587.01797119,1035.97425148)
\curveto(586.8279666,1036.05424843)(586.6279668,1036.11924837)(586.41797119,1036.16925148)
\curveto(586.20796722,1036.21924827)(585.99796743,1036.27424821)(585.78797119,1036.33425148)
\curveto(585.70796772,1036.35424813)(585.6229678,1036.36924812)(585.53297119,1036.37925148)
\curveto(585.45296797,1036.3892481)(585.37296805,1036.40424808)(585.29297119,1036.42425148)
\curveto(584.97296845,1036.51424797)(584.66796876,1036.59924789)(584.37797119,1036.67925148)
\curveto(584.08796934,1036.76924772)(583.8229696,1036.89924759)(583.58297119,1037.06925148)
\curveto(583.30297012,1037.26924722)(583.09797033,1037.53924695)(582.96797119,1037.87925148)
\curveto(582.94797048,1037.94924654)(582.9279705,1038.04424644)(582.90797119,1038.16425148)
\curveto(582.88797054,1038.23424625)(582.87297055,1038.31924617)(582.86297119,1038.41925148)
\curveto(582.85297057,1038.51924597)(582.85797057,1038.60924588)(582.87797119,1038.68925148)
\curveto(582.89797053,1038.73924575)(582.90297052,1038.77924571)(582.89297119,1038.80925148)
\curveto(582.88297054,1038.84924564)(582.88797054,1038.89424559)(582.90797119,1038.94425148)
\curveto(582.9279705,1039.05424543)(582.94797048,1039.15424533)(582.96797119,1039.24425148)
\curveto(582.99797043,1039.34424514)(583.03297039,1039.43924505)(583.07297119,1039.52925148)
\curveto(583.20297022,1039.81924467)(583.38297004,1040.05424443)(583.61297119,1040.23425148)
\curveto(583.84296958,1040.41424407)(584.10296932,1040.55924393)(584.39297119,1040.66925148)
\curveto(584.50296892,1040.71924377)(584.61796881,1040.75424373)(584.73797119,1040.77425148)
\curveto(584.85796857,1040.80424368)(584.98296844,1040.83424365)(585.11297119,1040.86425148)
\curveto(585.17296825,1040.8842436)(585.23296819,1040.89424359)(585.29297119,1040.89425148)
\lineto(585.47297119,1040.92425148)
\curveto(585.55296787,1040.93424355)(585.63796779,1040.93924355)(585.72797119,1040.93925148)
\curveto(585.81796761,1040.93924355)(585.90296752,1040.94424354)(585.98297119,1040.95425148)
}
}
{
\newrgbcolor{curcolor}{0 0 0}
\pscustom[linestyle=none,fillstyle=solid,fillcolor=curcolor]
{
\newpath
\moveto(593.16961182,1043.59425148)
\curveto(593.23960887,1043.51424097)(593.27460883,1043.39424109)(593.27461182,1043.23425148)
\lineto(593.27461182,1042.76925148)
\lineto(593.27461182,1042.36425148)
\curveto(593.27460883,1042.22424226)(593.23960887,1042.12924236)(593.16961182,1042.07925148)
\curveto(593.109609,1042.02924246)(593.02960908,1041.99924249)(592.92961182,1041.98925148)
\curveto(592.83960927,1041.97924251)(592.73960937,1041.97424251)(592.62961182,1041.97425148)
\lineto(591.78961182,1041.97425148)
\curveto(591.67961043,1041.97424251)(591.57961053,1041.97924251)(591.48961182,1041.98925148)
\curveto(591.4096107,1041.99924249)(591.33961077,1042.02924246)(591.27961182,1042.07925148)
\curveto(591.23961087,1042.10924238)(591.2096109,1042.16424232)(591.18961182,1042.24425148)
\curveto(591.17961093,1042.33424215)(591.16961094,1042.42924206)(591.15961182,1042.52925148)
\lineto(591.15961182,1042.85925148)
\curveto(591.16961094,1042.96924152)(591.17461093,1043.06424142)(591.17461182,1043.14425148)
\lineto(591.17461182,1043.35425148)
\curveto(591.18461092,1043.42424106)(591.2046109,1043.484241)(591.23461182,1043.53425148)
\curveto(591.25461085,1043.57424091)(591.27961083,1043.60424088)(591.30961182,1043.62425148)
\lineto(591.42961182,1043.68425148)
\curveto(591.44961066,1043.6842408)(591.47461063,1043.6842408)(591.50461182,1043.68425148)
\curveto(591.53461057,1043.69424079)(591.55961055,1043.69924079)(591.57961182,1043.69925148)
\lineto(592.67461182,1043.69925148)
\curveto(592.77460933,1043.69924079)(592.86960924,1043.69424079)(592.95961182,1043.68425148)
\curveto(593.04960906,1043.67424081)(593.11960899,1043.64424084)(593.16961182,1043.59425148)
\moveto(593.27461182,1033.82925148)
\curveto(593.27460883,1033.62925086)(593.26960884,1033.45925103)(593.25961182,1033.31925148)
\curveto(593.24960886,1033.17925131)(593.15960895,1033.0842514)(592.98961182,1033.03425148)
\curveto(592.92960918,1033.01425147)(592.86460924,1033.00425148)(592.79461182,1033.00425148)
\curveto(592.72460938,1033.01425147)(592.64960946,1033.01925147)(592.56961182,1033.01925148)
\lineto(591.72961182,1033.01925148)
\curveto(591.63961047,1033.01925147)(591.54961056,1033.02425146)(591.45961182,1033.03425148)
\curveto(591.37961073,1033.04425144)(591.31961079,1033.07425141)(591.27961182,1033.12425148)
\curveto(591.21961089,1033.19425129)(591.18461092,1033.27925121)(591.17461182,1033.37925148)
\lineto(591.17461182,1033.72425148)
\lineto(591.17461182,1040.05425148)
\lineto(591.17461182,1040.35425148)
\curveto(591.17461093,1040.45424403)(591.19461091,1040.53424395)(591.23461182,1040.59425148)
\curveto(591.29461081,1040.66424382)(591.37961073,1040.70924378)(591.48961182,1040.72925148)
\curveto(591.5096106,1040.73924375)(591.53461057,1040.73924375)(591.56461182,1040.72925148)
\curveto(591.6046105,1040.72924376)(591.63461047,1040.73424375)(591.65461182,1040.74425148)
\lineto(592.40461182,1040.74425148)
\lineto(592.59961182,1040.74425148)
\curveto(592.67960943,1040.75424373)(592.74460936,1040.75424373)(592.79461182,1040.74425148)
\lineto(592.91461182,1040.74425148)
\curveto(592.97460913,1040.72424376)(593.02960908,1040.70924378)(593.07961182,1040.69925148)
\curveto(593.12960898,1040.6892438)(593.16960894,1040.65924383)(593.19961182,1040.60925148)
\curveto(593.23960887,1040.55924393)(593.25960885,1040.489244)(593.25961182,1040.39925148)
\curveto(593.26960884,1040.30924418)(593.27460883,1040.21424427)(593.27461182,1040.11425148)
\lineto(593.27461182,1033.82925148)
}
}
{
\newrgbcolor{curcolor}{0 0 0}
\pscustom[linestyle=none,fillstyle=solid,fillcolor=curcolor]
{
\newpath
\moveto(597.90679932,1040.95425148)
\curveto(598.65679482,1040.97424351)(599.30679417,1040.8892436)(599.85679932,1040.69925148)
\curveto(600.41679306,1040.51924397)(600.84179263,1040.20424428)(601.13179932,1039.75425148)
\curveto(601.20179227,1039.64424484)(601.26179221,1039.52924496)(601.31179932,1039.40925148)
\curveto(601.3717921,1039.29924519)(601.42179205,1039.17424531)(601.46179932,1039.03425148)
\curveto(601.48179199,1038.97424551)(601.49179198,1038.90924558)(601.49179932,1038.83925148)
\curveto(601.49179198,1038.76924572)(601.48179199,1038.70924578)(601.46179932,1038.65925148)
\curveto(601.42179205,1038.59924589)(601.36679211,1038.55924593)(601.29679932,1038.53925148)
\curveto(601.24679223,1038.51924597)(601.18679229,1038.50924598)(601.11679932,1038.50925148)
\lineto(600.90679932,1038.50925148)
\lineto(600.24679932,1038.50925148)
\curveto(600.1767933,1038.50924598)(600.10679337,1038.50424598)(600.03679932,1038.49425148)
\curveto(599.96679351,1038.49424599)(599.90179357,1038.50424598)(599.84179932,1038.52425148)
\curveto(599.74179373,1038.54424594)(599.66679381,1038.5842459)(599.61679932,1038.64425148)
\curveto(599.56679391,1038.70424578)(599.52179395,1038.76424572)(599.48179932,1038.82425148)
\lineto(599.36179932,1039.03425148)
\curveto(599.33179414,1039.11424537)(599.28179419,1039.17924531)(599.21179932,1039.22925148)
\curveto(599.11179436,1039.30924518)(599.01179446,1039.36924512)(598.91179932,1039.40925148)
\curveto(598.82179465,1039.44924504)(598.70679477,1039.484245)(598.56679932,1039.51425148)
\curveto(598.49679498,1039.53424495)(598.39179508,1039.54924494)(598.25179932,1039.55925148)
\curveto(598.12179535,1039.56924492)(598.02179545,1039.56424492)(597.95179932,1039.54425148)
\lineto(597.84679932,1039.54425148)
\lineto(597.69679932,1039.51425148)
\curveto(597.65679582,1039.51424497)(597.61179586,1039.50924498)(597.56179932,1039.49925148)
\curveto(597.39179608,1039.44924504)(597.25179622,1039.37924511)(597.14179932,1039.28925148)
\curveto(597.04179643,1039.20924528)(596.9717965,1039.0842454)(596.93179932,1038.91425148)
\curveto(596.91179656,1038.84424564)(596.91179656,1038.77924571)(596.93179932,1038.71925148)
\curveto(596.95179652,1038.65924583)(596.9717965,1038.60924588)(596.99179932,1038.56925148)
\curveto(597.06179641,1038.44924604)(597.14179633,1038.35424613)(597.23179932,1038.28425148)
\curveto(597.33179614,1038.21424627)(597.44679603,1038.15424633)(597.57679932,1038.10425148)
\curveto(597.76679571,1038.02424646)(597.9717955,1037.95424653)(598.19179932,1037.89425148)
\lineto(598.88179932,1037.74425148)
\curveto(599.12179435,1037.70424678)(599.35179412,1037.65424683)(599.57179932,1037.59425148)
\curveto(599.80179367,1037.54424694)(600.01679346,1037.47924701)(600.21679932,1037.39925148)
\curveto(600.30679317,1037.35924713)(600.39179308,1037.32424716)(600.47179932,1037.29425148)
\curveto(600.56179291,1037.27424721)(600.64679283,1037.23924725)(600.72679932,1037.18925148)
\curveto(600.91679256,1037.06924742)(601.08679239,1036.93924755)(601.23679932,1036.79925148)
\curveto(601.39679208,1036.65924783)(601.52179195,1036.484248)(601.61179932,1036.27425148)
\curveto(601.64179183,1036.20424828)(601.66679181,1036.13424835)(601.68679932,1036.06425148)
\curveto(601.70679177,1035.99424849)(601.72679175,1035.91924857)(601.74679932,1035.83925148)
\curveto(601.75679172,1035.77924871)(601.76179171,1035.6842488)(601.76179932,1035.55425148)
\curveto(601.7717917,1035.43424905)(601.7717917,1035.33924915)(601.76179932,1035.26925148)
\lineto(601.76179932,1035.19425148)
\curveto(601.74179173,1035.13424935)(601.72679175,1035.07424941)(601.71679932,1035.01425148)
\curveto(601.71679176,1034.96424952)(601.71179176,1034.91424957)(601.70179932,1034.86425148)
\curveto(601.63179184,1034.56424992)(601.52179195,1034.29925019)(601.37179932,1034.06925148)
\curveto(601.21179226,1033.82925066)(601.01679246,1033.63425085)(600.78679932,1033.48425148)
\curveto(600.55679292,1033.33425115)(600.29679318,1033.20425128)(600.00679932,1033.09425148)
\curveto(599.89679358,1033.04425144)(599.7767937,1033.00925148)(599.64679932,1032.98925148)
\curveto(599.52679395,1032.96925152)(599.40679407,1032.94425154)(599.28679932,1032.91425148)
\curveto(599.19679428,1032.89425159)(599.10179437,1032.8842516)(599.00179932,1032.88425148)
\curveto(598.91179456,1032.87425161)(598.82179465,1032.85925163)(598.73179932,1032.83925148)
\lineto(598.46179932,1032.83925148)
\curveto(598.40179507,1032.81925167)(598.29679518,1032.80925168)(598.14679932,1032.80925148)
\curveto(598.00679547,1032.80925168)(597.90679557,1032.81925167)(597.84679932,1032.83925148)
\curveto(597.81679566,1032.83925165)(597.78179569,1032.84425164)(597.74179932,1032.85425148)
\lineto(597.63679932,1032.85425148)
\curveto(597.51679596,1032.87425161)(597.39679608,1032.8892516)(597.27679932,1032.89925148)
\curveto(597.15679632,1032.90925158)(597.04179643,1032.92925156)(596.93179932,1032.95925148)
\curveto(596.54179693,1033.06925142)(596.19679728,1033.19425129)(595.89679932,1033.33425148)
\curveto(595.59679788,1033.484251)(595.34179813,1033.70425078)(595.13179932,1033.99425148)
\curveto(594.99179848,1034.1842503)(594.8717986,1034.40425008)(594.77179932,1034.65425148)
\curveto(594.75179872,1034.71424977)(594.73179874,1034.79424969)(594.71179932,1034.89425148)
\curveto(594.69179878,1034.94424954)(594.6767988,1035.01424947)(594.66679932,1035.10425148)
\curveto(594.65679882,1035.19424929)(594.66179881,1035.26924922)(594.68179932,1035.32925148)
\curveto(594.71179876,1035.39924909)(594.76179871,1035.44924904)(594.83179932,1035.47925148)
\curveto(594.88179859,1035.49924899)(594.94179853,1035.50924898)(595.01179932,1035.50925148)
\lineto(595.23679932,1035.50925148)
\lineto(595.94179932,1035.50925148)
\lineto(596.18179932,1035.50925148)
\curveto(596.26179721,1035.50924898)(596.33179714,1035.49924899)(596.39179932,1035.47925148)
\curveto(596.50179697,1035.43924905)(596.5717969,1035.37424911)(596.60179932,1035.28425148)
\curveto(596.64179683,1035.19424929)(596.68679679,1035.09924939)(596.73679932,1034.99925148)
\curveto(596.75679672,1034.94924954)(596.79179668,1034.8842496)(596.84179932,1034.80425148)
\curveto(596.90179657,1034.72424976)(596.95179652,1034.67424981)(596.99179932,1034.65425148)
\curveto(597.11179636,1034.55424993)(597.22679625,1034.47425001)(597.33679932,1034.41425148)
\curveto(597.44679603,1034.36425012)(597.58679589,1034.31425017)(597.75679932,1034.26425148)
\curveto(597.80679567,1034.24425024)(597.85679562,1034.23425025)(597.90679932,1034.23425148)
\curveto(597.95679552,1034.24425024)(598.00679547,1034.24425024)(598.05679932,1034.23425148)
\curveto(598.13679534,1034.21425027)(598.22179525,1034.20425028)(598.31179932,1034.20425148)
\curveto(598.41179506,1034.21425027)(598.49679498,1034.22925026)(598.56679932,1034.24925148)
\curveto(598.61679486,1034.25925023)(598.66179481,1034.26425022)(598.70179932,1034.26425148)
\curveto(598.75179472,1034.26425022)(598.80179467,1034.27425021)(598.85179932,1034.29425148)
\curveto(598.99179448,1034.34425014)(599.11679436,1034.40425008)(599.22679932,1034.47425148)
\curveto(599.34679413,1034.54424994)(599.44179403,1034.63424985)(599.51179932,1034.74425148)
\curveto(599.56179391,1034.82424966)(599.60179387,1034.94924954)(599.63179932,1035.11925148)
\curveto(599.65179382,1035.1892493)(599.65179382,1035.25424923)(599.63179932,1035.31425148)
\curveto(599.61179386,1035.37424911)(599.59179388,1035.42424906)(599.57179932,1035.46425148)
\curveto(599.50179397,1035.60424888)(599.41179406,1035.70924878)(599.30179932,1035.77925148)
\curveto(599.20179427,1035.84924864)(599.08179439,1035.91424857)(598.94179932,1035.97425148)
\curveto(598.75179472,1036.05424843)(598.55179492,1036.11924837)(598.34179932,1036.16925148)
\curveto(598.13179534,1036.21924827)(597.92179555,1036.27424821)(597.71179932,1036.33425148)
\curveto(597.63179584,1036.35424813)(597.54679593,1036.36924812)(597.45679932,1036.37925148)
\curveto(597.3767961,1036.3892481)(597.29679618,1036.40424808)(597.21679932,1036.42425148)
\curveto(596.89679658,1036.51424797)(596.59179688,1036.59924789)(596.30179932,1036.67925148)
\curveto(596.01179746,1036.76924772)(595.74679773,1036.89924759)(595.50679932,1037.06925148)
\curveto(595.22679825,1037.26924722)(595.02179845,1037.53924695)(594.89179932,1037.87925148)
\curveto(594.8717986,1037.94924654)(594.85179862,1038.04424644)(594.83179932,1038.16425148)
\curveto(594.81179866,1038.23424625)(594.79679868,1038.31924617)(594.78679932,1038.41925148)
\curveto(594.7767987,1038.51924597)(594.78179869,1038.60924588)(594.80179932,1038.68925148)
\curveto(594.82179865,1038.73924575)(594.82679865,1038.77924571)(594.81679932,1038.80925148)
\curveto(594.80679867,1038.84924564)(594.81179866,1038.89424559)(594.83179932,1038.94425148)
\curveto(594.85179862,1039.05424543)(594.8717986,1039.15424533)(594.89179932,1039.24425148)
\curveto(594.92179855,1039.34424514)(594.95679852,1039.43924505)(594.99679932,1039.52925148)
\curveto(595.12679835,1039.81924467)(595.30679817,1040.05424443)(595.53679932,1040.23425148)
\curveto(595.76679771,1040.41424407)(596.02679745,1040.55924393)(596.31679932,1040.66925148)
\curveto(596.42679705,1040.71924377)(596.54179693,1040.75424373)(596.66179932,1040.77425148)
\curveto(596.78179669,1040.80424368)(596.90679657,1040.83424365)(597.03679932,1040.86425148)
\curveto(597.09679638,1040.8842436)(597.15679632,1040.89424359)(597.21679932,1040.89425148)
\lineto(597.39679932,1040.92425148)
\curveto(597.476796,1040.93424355)(597.56179591,1040.93924355)(597.65179932,1040.93925148)
\curveto(597.74179573,1040.93924355)(597.82679565,1040.94424354)(597.90679932,1040.95425148)
}
}
{
\newrgbcolor{curcolor}{0 0 0}
\pscustom[linestyle=none,fillstyle=solid,fillcolor=curcolor]
{
\newpath
\moveto(604.04343994,1043.05425148)
\lineto(605.04843994,1043.05425148)
\curveto(605.19843696,1043.05424143)(605.32843683,1043.04424144)(605.43843994,1043.02425148)
\curveto(605.5584366,1043.01424147)(605.64343651,1042.95424153)(605.69343994,1042.84425148)
\curveto(605.71343644,1042.79424169)(605.72343643,1042.73424175)(605.72343994,1042.66425148)
\lineto(605.72343994,1042.45425148)
\lineto(605.72343994,1041.77925148)
\curveto(605.72343643,1041.72924276)(605.71843644,1041.66924282)(605.70843994,1041.59925148)
\curveto(605.70843645,1041.53924295)(605.71343644,1041.484243)(605.72343994,1041.43425148)
\lineto(605.72343994,1041.26925148)
\curveto(605.72343643,1041.1892433)(605.72843643,1041.11424337)(605.73843994,1041.04425148)
\curveto(605.74843641,1040.9842435)(605.77343638,1040.92924356)(605.81343994,1040.87925148)
\curveto(605.88343627,1040.7892437)(606.00843615,1040.73924375)(606.18843994,1040.72925148)
\lineto(606.72843994,1040.72925148)
\lineto(606.90843994,1040.72925148)
\curveto(606.96843519,1040.72924376)(607.02343513,1040.71924377)(607.07343994,1040.69925148)
\curveto(607.18343497,1040.64924384)(607.24343491,1040.55924393)(607.25343994,1040.42925148)
\curveto(607.27343488,1040.29924419)(607.28343487,1040.15424433)(607.28343994,1039.99425148)
\lineto(607.28343994,1039.78425148)
\curveto(607.29343486,1039.71424477)(607.28843487,1039.65424483)(607.26843994,1039.60425148)
\curveto(607.21843494,1039.44424504)(607.11343504,1039.35924513)(606.95343994,1039.34925148)
\curveto(606.79343536,1039.33924515)(606.61343554,1039.33424515)(606.41343994,1039.33425148)
\lineto(606.27843994,1039.33425148)
\curveto(606.23843592,1039.34424514)(606.20343595,1039.34424514)(606.17343994,1039.33425148)
\curveto(606.13343602,1039.32424516)(606.09843606,1039.31924517)(606.06843994,1039.31925148)
\curveto(606.03843612,1039.32924516)(606.00843615,1039.32424516)(605.97843994,1039.30425148)
\curveto(605.89843626,1039.2842452)(605.83843632,1039.23924525)(605.79843994,1039.16925148)
\curveto(605.76843639,1039.10924538)(605.74343641,1039.03424545)(605.72343994,1038.94425148)
\curveto(605.71343644,1038.89424559)(605.71343644,1038.83924565)(605.72343994,1038.77925148)
\curveto(605.73343642,1038.71924577)(605.73343642,1038.66424582)(605.72343994,1038.61425148)
\lineto(605.72343994,1037.68425148)
\lineto(605.72343994,1035.92925148)
\curveto(605.72343643,1035.67924881)(605.72843643,1035.45924903)(605.73843994,1035.26925148)
\curveto(605.7584364,1035.0892494)(605.82343633,1034.92924956)(605.93343994,1034.78925148)
\curveto(605.98343617,1034.72924976)(606.04843611,1034.6842498)(606.12843994,1034.65425148)
\lineto(606.39843994,1034.59425148)
\curveto(606.42843573,1034.5842499)(606.4584357,1034.57924991)(606.48843994,1034.57925148)
\curveto(606.52843563,1034.5892499)(606.5584356,1034.5892499)(606.57843994,1034.57925148)
\lineto(606.74343994,1034.57925148)
\curveto(606.8534353,1034.57924991)(606.94843521,1034.57424991)(607.02843994,1034.56425148)
\curveto(607.10843505,1034.55424993)(607.17343498,1034.51424997)(607.22343994,1034.44425148)
\curveto(607.26343489,1034.3842501)(607.28343487,1034.30425018)(607.28343994,1034.20425148)
\lineto(607.28343994,1033.91925148)
\curveto(607.28343487,1033.70925078)(607.27843488,1033.51425097)(607.26843994,1033.33425148)
\curveto(607.26843489,1033.16425132)(607.18843497,1033.04925144)(607.02843994,1032.98925148)
\curveto(606.97843518,1032.96925152)(606.93343522,1032.96425152)(606.89343994,1032.97425148)
\curveto(606.8534353,1032.97425151)(606.80843535,1032.96425152)(606.75843994,1032.94425148)
\lineto(606.60843994,1032.94425148)
\curveto(606.58843557,1032.94425154)(606.5584356,1032.94925154)(606.51843994,1032.95925148)
\curveto(606.47843568,1032.95925153)(606.44343571,1032.95425153)(606.41343994,1032.94425148)
\curveto(606.36343579,1032.93425155)(606.30843585,1032.93425155)(606.24843994,1032.94425148)
\lineto(606.09843994,1032.94425148)
\lineto(605.94843994,1032.94425148)
\curveto(605.89843626,1032.93425155)(605.8534363,1032.93425155)(605.81343994,1032.94425148)
\lineto(605.64843994,1032.94425148)
\curveto(605.59843656,1032.95425153)(605.54343661,1032.95925153)(605.48343994,1032.95925148)
\curveto(605.42343673,1032.95925153)(605.36843679,1032.96425152)(605.31843994,1032.97425148)
\curveto(605.24843691,1032.9842515)(605.18343697,1032.99425149)(605.12343994,1033.00425148)
\lineto(604.94343994,1033.03425148)
\curveto(604.83343732,1033.06425142)(604.72843743,1033.09925139)(604.62843994,1033.13925148)
\curveto(604.52843763,1033.17925131)(604.43343772,1033.22425126)(604.34343994,1033.27425148)
\lineto(604.25343994,1033.33425148)
\curveto(604.22343793,1033.36425112)(604.18843797,1033.39425109)(604.14843994,1033.42425148)
\curveto(604.12843803,1033.44425104)(604.10343805,1033.46425102)(604.07343994,1033.48425148)
\lineto(603.99843994,1033.55925148)
\curveto(603.8584383,1033.74925074)(603.7534384,1033.95925053)(603.68343994,1034.18925148)
\curveto(603.66343849,1034.22925026)(603.6534385,1034.26425022)(603.65343994,1034.29425148)
\curveto(603.66343849,1034.33425015)(603.66343849,1034.37925011)(603.65343994,1034.42925148)
\curveto(603.64343851,1034.44925004)(603.63843852,1034.47425001)(603.63843994,1034.50425148)
\curveto(603.63843852,1034.53424995)(603.63343852,1034.55924993)(603.62343994,1034.57925148)
\lineto(603.62343994,1034.72925148)
\curveto(603.61343854,1034.76924972)(603.60843855,1034.81424967)(603.60843994,1034.86425148)
\curveto(603.61843854,1034.91424957)(603.62343853,1034.96424952)(603.62343994,1035.01425148)
\lineto(603.62343994,1035.58425148)
\lineto(603.62343994,1037.81925148)
\lineto(603.62343994,1038.61425148)
\lineto(603.62343994,1038.82425148)
\curveto(603.63343852,1038.89424559)(603.62843853,1038.95924553)(603.60843994,1039.01925148)
\curveto(603.56843859,1039.15924533)(603.49843866,1039.24924524)(603.39843994,1039.28925148)
\curveto(603.28843887,1039.33924515)(603.14843901,1039.35424513)(602.97843994,1039.33425148)
\curveto(602.80843935,1039.31424517)(602.66343949,1039.32924516)(602.54343994,1039.37925148)
\curveto(602.46343969,1039.40924508)(602.41343974,1039.45424503)(602.39343994,1039.51425148)
\curveto(602.37343978,1039.57424491)(602.3534398,1039.64924484)(602.33343994,1039.73925148)
\lineto(602.33343994,1040.05425148)
\curveto(602.33343982,1040.23424425)(602.34343981,1040.37924411)(602.36343994,1040.48925148)
\curveto(602.38343977,1040.59924389)(602.46843969,1040.67424381)(602.61843994,1040.71425148)
\curveto(602.6584395,1040.73424375)(602.69843946,1040.73924375)(602.73843994,1040.72925148)
\lineto(602.87343994,1040.72925148)
\curveto(603.02343913,1040.72924376)(603.16343899,1040.73424375)(603.29343994,1040.74425148)
\curveto(603.42343873,1040.76424372)(603.51343864,1040.82424366)(603.56343994,1040.92425148)
\curveto(603.59343856,1040.99424349)(603.60843855,1041.07424341)(603.60843994,1041.16425148)
\curveto(603.61843854,1041.25424323)(603.62343853,1041.34424314)(603.62343994,1041.43425148)
\lineto(603.62343994,1042.36425148)
\lineto(603.62343994,1042.61925148)
\curveto(603.62343853,1042.70924178)(603.63343852,1042.7842417)(603.65343994,1042.84425148)
\curveto(603.70343845,1042.94424154)(603.77843838,1043.00924148)(603.87843994,1043.03925148)
\curveto(603.89843826,1043.04924144)(603.92343823,1043.04924144)(603.95343994,1043.03925148)
\curveto(603.99343816,1043.03924145)(604.02343813,1043.04424144)(604.04343994,1043.05425148)
}
}
{
\newrgbcolor{curcolor}{0 0 0}
\pscustom[linestyle=none,fillstyle=solid,fillcolor=curcolor]
{
\newpath
\moveto(615.63187744,1036.94925148)
\curveto(615.65186928,1036.86924762)(615.65186928,1036.77924771)(615.63187744,1036.67925148)
\curveto(615.61186932,1036.57924791)(615.57686935,1036.51424797)(615.52687744,1036.48425148)
\curveto(615.47686945,1036.44424804)(615.40186953,1036.41424807)(615.30187744,1036.39425148)
\curveto(615.21186972,1036.3842481)(615.10686982,1036.37424811)(614.98687744,1036.36425148)
\lineto(614.64187744,1036.36425148)
\curveto(614.5318704,1036.37424811)(614.4318705,1036.37924811)(614.34187744,1036.37925148)
\lineto(610.68187744,1036.37925148)
\lineto(610.47187744,1036.37925148)
\curveto(610.41187452,1036.37924811)(610.35687457,1036.36924812)(610.30687744,1036.34925148)
\curveto(610.2268747,1036.30924818)(610.17687475,1036.26924822)(610.15687744,1036.22925148)
\curveto(610.13687479,1036.20924828)(610.11687481,1036.16924832)(610.09687744,1036.10925148)
\curveto(610.07687485,1036.05924843)(610.07187486,1036.00924848)(610.08187744,1035.95925148)
\curveto(610.10187483,1035.89924859)(610.11187482,1035.83924865)(610.11187744,1035.77925148)
\curveto(610.12187481,1035.72924876)(610.13687479,1035.67424881)(610.15687744,1035.61425148)
\curveto(610.23687469,1035.37424911)(610.3318746,1035.17424931)(610.44187744,1035.01425148)
\curveto(610.56187437,1034.86424962)(610.72187421,1034.72924976)(610.92187744,1034.60925148)
\curveto(611.00187393,1034.55924993)(611.08187385,1034.52424996)(611.16187744,1034.50425148)
\curveto(611.25187368,1034.49424999)(611.34187359,1034.47425001)(611.43187744,1034.44425148)
\curveto(611.51187342,1034.42425006)(611.62187331,1034.40925008)(611.76187744,1034.39925148)
\curveto(611.90187303,1034.3892501)(612.02187291,1034.39425009)(612.12187744,1034.41425148)
\lineto(612.25687744,1034.41425148)
\curveto(612.35687257,1034.43425005)(612.44687248,1034.45425003)(612.52687744,1034.47425148)
\curveto(612.61687231,1034.50424998)(612.70187223,1034.53424995)(612.78187744,1034.56425148)
\curveto(612.88187205,1034.61424987)(612.99187194,1034.67924981)(613.11187744,1034.75925148)
\curveto(613.24187169,1034.83924965)(613.33687159,1034.91924957)(613.39687744,1034.99925148)
\curveto(613.44687148,1035.06924942)(613.49687143,1035.13424935)(613.54687744,1035.19425148)
\curveto(613.60687132,1035.26424922)(613.67687125,1035.31424917)(613.75687744,1035.34425148)
\curveto(613.85687107,1035.39424909)(613.98187095,1035.41424907)(614.13187744,1035.40425148)
\lineto(614.56687744,1035.40425148)
\lineto(614.74687744,1035.40425148)
\curveto(614.81687011,1035.41424907)(614.87687005,1035.40924908)(614.92687744,1035.38925148)
\lineto(615.07687744,1035.38925148)
\curveto(615.17686975,1035.36924912)(615.24686968,1035.34424914)(615.28687744,1035.31425148)
\curveto(615.3268696,1035.29424919)(615.34686958,1035.24924924)(615.34687744,1035.17925148)
\curveto(615.35686957,1035.10924938)(615.35186958,1035.04924944)(615.33187744,1034.99925148)
\curveto(615.28186965,1034.85924963)(615.2268697,1034.73424975)(615.16687744,1034.62425148)
\curveto(615.10686982,1034.51424997)(615.03686989,1034.40425008)(614.95687744,1034.29425148)
\curveto(614.73687019,1033.96425052)(614.48687044,1033.69925079)(614.20687744,1033.49925148)
\curveto(613.926871,1033.29925119)(613.57687135,1033.12925136)(613.15687744,1032.98925148)
\curveto(613.04687188,1032.94925154)(612.93687199,1032.92425156)(612.82687744,1032.91425148)
\curveto(612.71687221,1032.90425158)(612.60187233,1032.8842516)(612.48187744,1032.85425148)
\curveto(612.44187249,1032.84425164)(612.39687253,1032.84425164)(612.34687744,1032.85425148)
\curveto(612.30687262,1032.85425163)(612.26687266,1032.84925164)(612.22687744,1032.83925148)
\lineto(612.06187744,1032.83925148)
\curveto(612.01187292,1032.81925167)(611.95187298,1032.81425167)(611.88187744,1032.82425148)
\curveto(611.82187311,1032.82425166)(611.76687316,1032.82925166)(611.71687744,1032.83925148)
\curveto(611.63687329,1032.84925164)(611.56687336,1032.84925164)(611.50687744,1032.83925148)
\curveto(611.44687348,1032.82925166)(611.38187355,1032.83425165)(611.31187744,1032.85425148)
\curveto(611.26187367,1032.87425161)(611.20687372,1032.8842516)(611.14687744,1032.88425148)
\curveto(611.08687384,1032.8842516)(611.0318739,1032.89425159)(610.98187744,1032.91425148)
\curveto(610.87187406,1032.93425155)(610.76187417,1032.95925153)(610.65187744,1032.98925148)
\curveto(610.54187439,1033.00925148)(610.44187449,1033.04425144)(610.35187744,1033.09425148)
\curveto(610.24187469,1033.13425135)(610.13687479,1033.16925132)(610.03687744,1033.19925148)
\curveto(609.94687498,1033.23925125)(609.86187507,1033.2842512)(609.78187744,1033.33425148)
\curveto(609.46187547,1033.53425095)(609.17687575,1033.76425072)(608.92687744,1034.02425148)
\curveto(608.67687625,1034.29425019)(608.47187646,1034.60424988)(608.31187744,1034.95425148)
\curveto(608.26187667,1035.06424942)(608.22187671,1035.17424931)(608.19187744,1035.28425148)
\curveto(608.16187677,1035.40424908)(608.12187681,1035.52424896)(608.07187744,1035.64425148)
\curveto(608.06187687,1035.6842488)(608.05687687,1035.71924877)(608.05687744,1035.74925148)
\curveto(608.05687687,1035.7892487)(608.05187688,1035.82924866)(608.04187744,1035.86925148)
\curveto(608.00187693,1035.9892485)(607.97687695,1036.11924837)(607.96687744,1036.25925148)
\lineto(607.93687744,1036.67925148)
\curveto(607.93687699,1036.72924776)(607.931877,1036.7842477)(607.92187744,1036.84425148)
\curveto(607.92187701,1036.90424758)(607.926877,1036.95924753)(607.93687744,1037.00925148)
\lineto(607.93687744,1037.18925148)
\lineto(607.98187744,1037.54925148)
\curveto(608.02187691,1037.71924677)(608.05687687,1037.8842466)(608.08687744,1038.04425148)
\curveto(608.11687681,1038.20424628)(608.16187677,1038.35424613)(608.22187744,1038.49425148)
\curveto(608.65187628,1039.53424495)(609.38187555,1040.26924422)(610.41187744,1040.69925148)
\curveto(610.55187438,1040.75924373)(610.69187424,1040.79924369)(610.83187744,1040.81925148)
\curveto(610.98187395,1040.84924364)(611.13687379,1040.8842436)(611.29687744,1040.92425148)
\curveto(611.37687355,1040.93424355)(611.45187348,1040.93924355)(611.52187744,1040.93925148)
\curveto(611.59187334,1040.93924355)(611.66687326,1040.94424354)(611.74687744,1040.95425148)
\curveto(612.25687267,1040.96424352)(612.69187224,1040.90424358)(613.05187744,1040.77425148)
\curveto(613.42187151,1040.65424383)(613.75187118,1040.49424399)(614.04187744,1040.29425148)
\curveto(614.1318708,1040.23424425)(614.22187071,1040.16424432)(614.31187744,1040.08425148)
\curveto(614.40187053,1040.01424447)(614.48187045,1039.93924455)(614.55187744,1039.85925148)
\curveto(614.58187035,1039.80924468)(614.62187031,1039.76924472)(614.67187744,1039.73925148)
\curveto(614.75187018,1039.62924486)(614.8268701,1039.51424497)(614.89687744,1039.39425148)
\curveto(614.96686996,1039.2842452)(615.04186989,1039.16924532)(615.12187744,1039.04925148)
\curveto(615.17186976,1038.95924553)(615.21186972,1038.86424562)(615.24187744,1038.76425148)
\curveto(615.28186965,1038.67424581)(615.32186961,1038.57424591)(615.36187744,1038.46425148)
\curveto(615.41186952,1038.33424615)(615.45186948,1038.19924629)(615.48187744,1038.05925148)
\curveto(615.51186942,1037.91924657)(615.54686938,1037.77924671)(615.58687744,1037.63925148)
\curveto(615.60686932,1037.55924693)(615.61186932,1037.46924702)(615.60187744,1037.36925148)
\curveto(615.60186933,1037.27924721)(615.61186932,1037.19424729)(615.63187744,1037.11425148)
\lineto(615.63187744,1036.94925148)
\moveto(613.38187744,1037.83425148)
\curveto(613.45187148,1037.93424655)(613.45687147,1038.05424643)(613.39687744,1038.19425148)
\curveto(613.34687158,1038.34424614)(613.30687162,1038.45424603)(613.27687744,1038.52425148)
\curveto(613.13687179,1038.79424569)(612.95187198,1038.99924549)(612.72187744,1039.13925148)
\curveto(612.49187244,1039.2892452)(612.17187276,1039.36924512)(611.76187744,1039.37925148)
\curveto(611.7318732,1039.35924513)(611.69687323,1039.35424513)(611.65687744,1039.36425148)
\curveto(611.61687331,1039.37424511)(611.58187335,1039.37424511)(611.55187744,1039.36425148)
\curveto(611.50187343,1039.34424514)(611.44687348,1039.32924516)(611.38687744,1039.31925148)
\curveto(611.3268736,1039.31924517)(611.27187366,1039.30924518)(611.22187744,1039.28925148)
\curveto(610.78187415,1039.14924534)(610.45687447,1038.87424561)(610.24687744,1038.46425148)
\curveto(610.2268747,1038.42424606)(610.20187473,1038.36924612)(610.17187744,1038.29925148)
\curveto(610.15187478,1038.23924625)(610.13687479,1038.17424631)(610.12687744,1038.10425148)
\curveto(610.11687481,1038.04424644)(610.11687481,1037.9842465)(610.12687744,1037.92425148)
\curveto(610.14687478,1037.86424662)(610.18187475,1037.81424667)(610.23187744,1037.77425148)
\curveto(610.31187462,1037.72424676)(610.42187451,1037.69924679)(610.56187744,1037.69925148)
\lineto(610.96687744,1037.69925148)
\lineto(612.63187744,1037.69925148)
\lineto(613.06687744,1037.69925148)
\curveto(613.2268717,1037.70924678)(613.3318716,1037.75424673)(613.38187744,1037.83425148)
}
}
{
\newrgbcolor{curcolor}{0 0 0}
\pscustom[linestyle=none,fillstyle=solid,fillcolor=curcolor]
{
\newpath
\moveto(621.33515869,1040.93925148)
\curveto(621.70515309,1040.94924354)(622.03015276,1040.90924358)(622.31015869,1040.81925148)
\curveto(622.5901522,1040.72924376)(622.83515196,1040.60424388)(623.04515869,1040.44425148)
\curveto(623.12515167,1040.3842441)(623.1951516,1040.31424417)(623.25515869,1040.23425148)
\curveto(623.32515147,1040.15424433)(623.40015139,1040.07424441)(623.48015869,1039.99425148)
\curveto(623.50015129,1039.97424451)(623.53015126,1039.94424454)(623.57015869,1039.90425148)
\curveto(623.62015117,1039.87424461)(623.67015112,1039.86924462)(623.72015869,1039.88925148)
\curveto(623.83015096,1039.91924457)(623.93515086,1039.9892445)(624.03515869,1040.09925148)
\curveto(624.13515066,1040.21924427)(624.23015056,1040.30924418)(624.32015869,1040.36925148)
\curveto(624.46015033,1040.47924401)(624.61015018,1040.56924392)(624.77015869,1040.63925148)
\curveto(624.93014986,1040.71924377)(625.11014968,1040.79424369)(625.31015869,1040.86425148)
\curveto(625.3901494,1040.8842436)(625.48514931,1040.89924359)(625.59515869,1040.90925148)
\curveto(625.71514908,1040.92924356)(625.83514896,1040.93924355)(625.95515869,1040.93925148)
\curveto(626.08514871,1040.94924354)(626.20514859,1040.94924354)(626.31515869,1040.93925148)
\curveto(626.43514836,1040.92924356)(626.54014825,1040.91424357)(626.63015869,1040.89425148)
\curveto(626.68014811,1040.8842436)(626.72514807,1040.87924361)(626.76515869,1040.87925148)
\curveto(626.80514799,1040.87924361)(626.85014794,1040.86924362)(626.90015869,1040.84925148)
\curveto(627.04014775,1040.80924368)(627.17514762,1040.76924372)(627.30515869,1040.72925148)
\curveto(627.43514736,1040.6892438)(627.55514724,1040.63424385)(627.66515869,1040.56425148)
\curveto(628.08514671,1040.30424418)(628.40014639,1039.92424456)(628.61015869,1039.42425148)
\curveto(628.65014614,1039.33424515)(628.68014611,1039.23924525)(628.70015869,1039.13925148)
\curveto(628.72014607,1039.04924544)(628.74014605,1038.95924553)(628.76015869,1038.86925148)
\curveto(628.77014602,1038.79924569)(628.77514602,1038.73424575)(628.77515869,1038.67425148)
\curveto(628.78514601,1038.61424587)(628.795146,1038.55424593)(628.80515869,1038.49425148)
\lineto(628.80515869,1038.34425148)
\curveto(628.81514598,1038.2842462)(628.81514598,1038.21424627)(628.80515869,1038.13425148)
\curveto(628.80514599,1038.05424643)(628.80514599,1037.97924651)(628.80515869,1037.90925148)
\lineto(628.80515869,1037.03925148)
\lineto(628.80515869,1034.11425148)
\curveto(628.80514599,1034.03425045)(628.80514599,1033.93925055)(628.80515869,1033.82925148)
\curveto(628.81514598,1033.72925076)(628.81514598,1033.62925086)(628.80515869,1033.52925148)
\curveto(628.80514599,1033.43925105)(628.795146,1033.34925114)(628.77515869,1033.25925148)
\curveto(628.75514604,1033.17925131)(628.72514607,1033.12425136)(628.68515869,1033.09425148)
\curveto(628.62514617,1033.04425144)(628.54514625,1033.01425147)(628.44515869,1033.00425148)
\lineto(628.14515869,1033.00425148)
\lineto(627.35015869,1033.00425148)
\curveto(627.21014758,1033.00425148)(627.08514771,1033.01425147)(626.97515869,1033.03425148)
\curveto(626.86514793,1033.05425143)(626.790148,1033.10925138)(626.75015869,1033.19925148)
\curveto(626.72014807,1033.26925122)(626.70514809,1033.34425114)(626.70515869,1033.42425148)
\curveto(626.70514809,1033.51425097)(626.70514809,1033.59925089)(626.70515869,1033.67925148)
\lineto(626.70515869,1034.51925148)
\lineto(626.70515869,1036.54425148)
\lineto(626.70515869,1037.17425148)
\curveto(626.70514809,1037.22424726)(626.70514809,1037.27924721)(626.70515869,1037.33925148)
\curveto(626.71514808,1037.39924709)(626.71014808,1037.45424703)(626.69015869,1037.50425148)
\lineto(626.69015869,1037.62425148)
\curveto(626.6901481,1037.6842468)(626.6901481,1037.74424674)(626.69015869,1037.80425148)
\curveto(626.6901481,1037.86424662)(626.68514811,1037.92424656)(626.67515869,1037.98425148)
\curveto(626.66514813,1038.02424646)(626.66014813,1038.06424642)(626.66015869,1038.10425148)
\curveto(626.66014813,1038.15424633)(626.65514814,1038.19924629)(626.64515869,1038.23925148)
\curveto(626.60514819,1038.3892461)(626.56014823,1038.51924597)(626.51015869,1038.62925148)
\curveto(626.47014832,1038.74924574)(626.40514839,1038.85424563)(626.31515869,1038.94425148)
\curveto(626.17514862,1039.0842454)(626.00514879,1039.1842453)(625.80515869,1039.24425148)
\curveto(625.76514903,1039.25424523)(625.73014906,1039.25424523)(625.70015869,1039.24425148)
\curveto(625.67014912,1039.24424524)(625.63514916,1039.25424523)(625.59515869,1039.27425148)
\curveto(625.55514924,1039.2842452)(625.50514929,1039.2892452)(625.44515869,1039.28925148)
\curveto(625.3951494,1039.29924519)(625.34514945,1039.29924519)(625.29515869,1039.28925148)
\curveto(625.23514956,1039.26924522)(625.17514962,1039.25924523)(625.11515869,1039.25925148)
\curveto(625.05514974,1039.25924523)(624.9951498,1039.24924524)(624.93515869,1039.22925148)
\curveto(624.64515015,1039.12924536)(624.43515036,1038.97924551)(624.30515869,1038.77925148)
\curveto(624.13515066,1038.54924594)(624.03015076,1038.25924623)(623.99015869,1037.90925148)
\curveto(623.96015083,1037.56924692)(623.94515085,1037.19424729)(623.94515869,1036.78425148)
\lineto(623.94515869,1034.80425148)
\lineto(623.94515869,1033.69425148)
\lineto(623.94515869,1033.39425148)
\curveto(623.94515085,1033.29425119)(623.92015087,1033.21425127)(623.87015869,1033.15425148)
\curveto(623.82015097,1033.0842514)(623.74515105,1033.03925145)(623.64515869,1033.01925148)
\curveto(623.55515124,1033.00925148)(623.45015134,1033.00425148)(623.33015869,1033.00425148)
\lineto(622.52015869,1033.00425148)
\lineto(622.25015869,1033.00425148)
\curveto(622.17015262,1033.01425147)(622.10015269,1033.02925146)(622.04015869,1033.04925148)
\curveto(621.94015285,1033.09925139)(621.88015291,1033.17925131)(621.86015869,1033.28925148)
\curveto(621.85015294,1033.39925109)(621.84515295,1033.52425096)(621.84515869,1033.66425148)
\lineto(621.84515869,1034.93925148)
\lineto(621.84515869,1037.29425148)
\curveto(621.84515295,1037.5842469)(621.83515296,1037.85924663)(621.81515869,1038.11925148)
\curveto(621.795153,1038.37924611)(621.73015306,1038.59424589)(621.62015869,1038.76425148)
\curveto(621.54015325,1038.90424558)(621.43515336,1039.00924548)(621.30515869,1039.07925148)
\curveto(621.18515361,1039.14924534)(621.03515376,1039.20924528)(620.85515869,1039.25925148)
\curveto(620.81515398,1039.26924522)(620.77515402,1039.26924522)(620.73515869,1039.25925148)
\curveto(620.6951541,1039.25924523)(620.65015414,1039.26424522)(620.60015869,1039.27425148)
\curveto(620.4901543,1039.29424519)(620.38515441,1039.2842452)(620.28515869,1039.24425148)
\curveto(620.26515453,1039.24424524)(620.24515455,1039.23924525)(620.22515869,1039.22925148)
\lineto(620.16515869,1039.22925148)
\curveto(620.00515479,1039.17924531)(619.85015494,1039.09424539)(619.70015869,1038.97425148)
\curveto(619.54015525,1038.85424563)(619.41515538,1038.71424577)(619.32515869,1038.55425148)
\curveto(619.24515555,1038.40424608)(619.18515561,1038.22924626)(619.14515869,1038.02925148)
\curveto(619.11515568,1037.83924665)(619.0951557,1037.62924686)(619.08515869,1037.39925148)
\lineto(619.08515869,1036.64925148)
\lineto(619.08515869,1034.62425148)
\lineto(619.08515869,1033.70925148)
\lineto(619.08515869,1033.43925148)
\curveto(619.08515571,1033.34925114)(619.07015572,1033.26925122)(619.04015869,1033.19925148)
\curveto(619.00015579,1033.10925138)(618.92515587,1033.05425143)(618.81515869,1033.03425148)
\curveto(618.70515609,1033.01425147)(618.58015621,1033.00425148)(618.44015869,1033.00425148)
\lineto(617.66015869,1033.00425148)
\lineto(617.36015869,1033.00425148)
\curveto(617.27015752,1033.01425147)(617.1951576,1033.03925145)(617.13515869,1033.07925148)
\curveto(617.04515775,1033.12925136)(616.9951578,1033.21925127)(616.98515869,1033.34925148)
\lineto(616.98515869,1033.78425148)
\lineto(616.98515869,1035.53925148)
\lineto(616.98515869,1039.19925148)
\lineto(616.98515869,1040.09925148)
\lineto(616.98515869,1040.38425148)
\curveto(616.9951578,1040.47424401)(617.02015777,1040.54924394)(617.06015869,1040.60925148)
\curveto(617.11015768,1040.66924382)(617.1901576,1040.70924378)(617.30015869,1040.72925148)
\lineto(617.39015869,1040.72925148)
\curveto(617.44015735,1040.73924375)(617.4901573,1040.74424374)(617.54015869,1040.74425148)
\lineto(617.70515869,1040.74425148)
\lineto(618.32015869,1040.74425148)
\curveto(618.40015639,1040.74424374)(618.47515632,1040.73924375)(618.54515869,1040.72925148)
\curveto(618.62515617,1040.72924376)(618.6951561,1040.71924377)(618.75515869,1040.69925148)
\curveto(618.83515596,1040.66924382)(618.88515591,1040.61924387)(618.90515869,1040.54925148)
\curveto(618.93515586,1040.47924401)(618.96015583,1040.39924409)(618.98015869,1040.30925148)
\curveto(618.9901558,1040.27924421)(618.9901558,1040.24924424)(618.98015869,1040.21925148)
\curveto(618.98015581,1040.19924429)(618.9901558,1040.17924431)(619.01015869,1040.15925148)
\curveto(619.02015577,1040.12924436)(619.03015576,1040.10424438)(619.04015869,1040.08425148)
\curveto(619.06015573,1040.07424441)(619.08015571,1040.05924443)(619.10015869,1040.03925148)
\curveto(619.22015557,1040.02924446)(619.32015547,1040.06424442)(619.40015869,1040.14425148)
\curveto(619.48015531,1040.23424425)(619.55515524,1040.30424418)(619.62515869,1040.35425148)
\curveto(619.76515503,1040.45424403)(619.90515489,1040.54424394)(620.04515869,1040.62425148)
\curveto(620.1951546,1040.70424378)(620.35515444,1040.76924372)(620.52515869,1040.81925148)
\curveto(620.61515418,1040.84924364)(620.70515409,1040.86924362)(620.79515869,1040.87925148)
\curveto(620.88515391,1040.8892436)(620.98015381,1040.90424358)(621.08015869,1040.92425148)
\curveto(621.11015368,1040.93424355)(621.15515364,1040.93424355)(621.21515869,1040.92425148)
\curveto(621.27515352,1040.92424356)(621.31515348,1040.92924356)(621.33515869,1040.93925148)
}
}
{
\newrgbcolor{curcolor}{0 0 0}
\pscustom[linestyle=none,fillstyle=solid,fillcolor=curcolor]
{
\newpath
\moveto(637.52390869,1033.60425148)
\curveto(637.54390084,1033.49425099)(637.55390083,1033.3842511)(637.55390869,1033.27425148)
\curveto(637.56390082,1033.16425132)(637.51390087,1033.0892514)(637.40390869,1033.04925148)
\curveto(637.34390104,1033.01925147)(637.27390111,1033.00425148)(637.19390869,1033.00425148)
\lineto(636.95390869,1033.00425148)
\lineto(636.14390869,1033.00425148)
\lineto(635.87390869,1033.00425148)
\curveto(635.79390259,1033.01425147)(635.72890266,1033.03925145)(635.67890869,1033.07925148)
\curveto(635.60890278,1033.11925137)(635.55390283,1033.17425131)(635.51390869,1033.24425148)
\curveto(635.4839029,1033.32425116)(635.43890295,1033.3892511)(635.37890869,1033.43925148)
\curveto(635.35890303,1033.45925103)(635.33390305,1033.47425101)(635.30390869,1033.48425148)
\curveto(635.27390311,1033.50425098)(635.23390315,1033.50925098)(635.18390869,1033.49925148)
\curveto(635.13390325,1033.47925101)(635.0839033,1033.45425103)(635.03390869,1033.42425148)
\curveto(634.99390339,1033.39425109)(634.94890344,1033.36925112)(634.89890869,1033.34925148)
\curveto(634.84890354,1033.30925118)(634.79390359,1033.27425121)(634.73390869,1033.24425148)
\lineto(634.55390869,1033.15425148)
\curveto(634.42390396,1033.09425139)(634.2889041,1033.04425144)(634.14890869,1033.00425148)
\curveto(634.00890438,1032.97425151)(633.86390452,1032.93925155)(633.71390869,1032.89925148)
\curveto(633.64390474,1032.87925161)(633.57390481,1032.86925162)(633.50390869,1032.86925148)
\curveto(633.44390494,1032.85925163)(633.37890501,1032.84925164)(633.30890869,1032.83925148)
\lineto(633.21890869,1032.83925148)
\curveto(633.1889052,1032.82925166)(633.15890523,1032.82425166)(633.12890869,1032.82425148)
\lineto(632.96390869,1032.82425148)
\curveto(632.86390552,1032.80425168)(632.76390562,1032.80425168)(632.66390869,1032.82425148)
\lineto(632.52890869,1032.82425148)
\curveto(632.45890593,1032.84425164)(632.388906,1032.85425163)(632.31890869,1032.85425148)
\curveto(632.25890613,1032.84425164)(632.19890619,1032.84925164)(632.13890869,1032.86925148)
\curveto(632.03890635,1032.8892516)(631.94390644,1032.90925158)(631.85390869,1032.92925148)
\curveto(631.76390662,1032.93925155)(631.67890671,1032.96425152)(631.59890869,1033.00425148)
\curveto(631.30890708,1033.11425137)(631.05890733,1033.25425123)(630.84890869,1033.42425148)
\curveto(630.64890774,1033.60425088)(630.4889079,1033.83925065)(630.36890869,1034.12925148)
\curveto(630.33890805,1034.19925029)(630.30890808,1034.27425021)(630.27890869,1034.35425148)
\curveto(630.25890813,1034.43425005)(630.23890815,1034.51924997)(630.21890869,1034.60925148)
\curveto(630.19890819,1034.65924983)(630.1889082,1034.70924978)(630.18890869,1034.75925148)
\curveto(630.19890819,1034.80924968)(630.19890819,1034.85924963)(630.18890869,1034.90925148)
\curveto(630.17890821,1034.93924955)(630.16890822,1034.99924949)(630.15890869,1035.08925148)
\curveto(630.15890823,1035.1892493)(630.16390822,1035.25924923)(630.17390869,1035.29925148)
\curveto(630.19390819,1035.39924909)(630.20390818,1035.484249)(630.20390869,1035.55425148)
\lineto(630.29390869,1035.88425148)
\curveto(630.32390806,1036.00424848)(630.36390802,1036.10924838)(630.41390869,1036.19925148)
\curveto(630.5839078,1036.489248)(630.77890761,1036.70924778)(630.99890869,1036.85925148)
\curveto(631.21890717,1037.00924748)(631.49890689,1037.13924735)(631.83890869,1037.24925148)
\curveto(631.96890642,1037.29924719)(632.10390628,1037.33424715)(632.24390869,1037.35425148)
\curveto(632.383906,1037.37424711)(632.52390586,1037.39924709)(632.66390869,1037.42925148)
\curveto(632.74390564,1037.44924704)(632.82890556,1037.45924703)(632.91890869,1037.45925148)
\curveto(633.00890538,1037.46924702)(633.09890529,1037.484247)(633.18890869,1037.50425148)
\curveto(633.25890513,1037.52424696)(633.32890506,1037.52924696)(633.39890869,1037.51925148)
\curveto(633.46890492,1037.51924697)(633.54390484,1037.52924696)(633.62390869,1037.54925148)
\curveto(633.69390469,1037.56924692)(633.76390462,1037.57924691)(633.83390869,1037.57925148)
\curveto(633.90390448,1037.57924691)(633.97890441,1037.5892469)(634.05890869,1037.60925148)
\curveto(634.26890412,1037.65924683)(634.45890393,1037.69924679)(634.62890869,1037.72925148)
\curveto(634.80890358,1037.76924672)(634.96890342,1037.85924663)(635.10890869,1037.99925148)
\curveto(635.19890319,1038.0892464)(635.25890313,1038.1892463)(635.28890869,1038.29925148)
\curveto(635.29890309,1038.32924616)(635.29890309,1038.35424613)(635.28890869,1038.37425148)
\curveto(635.2889031,1038.39424609)(635.29390309,1038.41424607)(635.30390869,1038.43425148)
\curveto(635.31390307,1038.45424603)(635.31890307,1038.484246)(635.31890869,1038.52425148)
\lineto(635.31890869,1038.61425148)
\lineto(635.28890869,1038.73425148)
\curveto(635.2889031,1038.77424571)(635.2839031,1038.80924568)(635.27390869,1038.83925148)
\curveto(635.17390321,1039.13924535)(634.96390342,1039.34424514)(634.64390869,1039.45425148)
\curveto(634.55390383,1039.484245)(634.44390394,1039.50424498)(634.31390869,1039.51425148)
\curveto(634.19390419,1039.53424495)(634.06890432,1039.53924495)(633.93890869,1039.52925148)
\curveto(633.80890458,1039.52924496)(633.6839047,1039.51924497)(633.56390869,1039.49925148)
\curveto(633.44390494,1039.47924501)(633.33890505,1039.45424503)(633.24890869,1039.42425148)
\curveto(633.1889052,1039.40424508)(633.12890526,1039.37424511)(633.06890869,1039.33425148)
\curveto(633.01890537,1039.30424518)(632.96890542,1039.26924522)(632.91890869,1039.22925148)
\curveto(632.86890552,1039.1892453)(632.81390557,1039.13424535)(632.75390869,1039.06425148)
\curveto(632.70390568,1038.99424549)(632.66890572,1038.92924556)(632.64890869,1038.86925148)
\curveto(632.59890579,1038.76924572)(632.55390583,1038.67424581)(632.51390869,1038.58425148)
\curveto(632.4839059,1038.49424599)(632.41390597,1038.43424605)(632.30390869,1038.40425148)
\curveto(632.22390616,1038.3842461)(632.13890625,1038.37424611)(632.04890869,1038.37425148)
\lineto(631.77890869,1038.37425148)
\lineto(631.20890869,1038.37425148)
\curveto(631.15890723,1038.37424611)(631.10890728,1038.36924612)(631.05890869,1038.35925148)
\curveto(631.00890738,1038.35924613)(630.96390742,1038.36424612)(630.92390869,1038.37425148)
\lineto(630.78890869,1038.37425148)
\curveto(630.76890762,1038.3842461)(630.74390764,1038.3892461)(630.71390869,1038.38925148)
\curveto(630.6839077,1038.3892461)(630.65890773,1038.39924609)(630.63890869,1038.41925148)
\curveto(630.55890783,1038.43924605)(630.50390788,1038.50424598)(630.47390869,1038.61425148)
\curveto(630.46390792,1038.66424582)(630.46390792,1038.71424577)(630.47390869,1038.76425148)
\curveto(630.4839079,1038.81424567)(630.49390789,1038.85924563)(630.50390869,1038.89925148)
\curveto(630.53390785,1039.00924548)(630.56390782,1039.10924538)(630.59390869,1039.19925148)
\curveto(630.63390775,1039.29924519)(630.67890771,1039.3892451)(630.72890869,1039.46925148)
\lineto(630.81890869,1039.61925148)
\lineto(630.90890869,1039.76925148)
\curveto(630.9889074,1039.87924461)(631.0889073,1039.9842445)(631.20890869,1040.08425148)
\curveto(631.22890716,1040.09424439)(631.25890713,1040.11924437)(631.29890869,1040.15925148)
\curveto(631.34890704,1040.19924429)(631.39390699,1040.23424425)(631.43390869,1040.26425148)
\curveto(631.47390691,1040.29424419)(631.51890687,1040.32424416)(631.56890869,1040.35425148)
\curveto(631.73890665,1040.46424402)(631.91890647,1040.54924394)(632.10890869,1040.60925148)
\curveto(632.29890609,1040.67924381)(632.49390589,1040.74424374)(632.69390869,1040.80425148)
\curveto(632.81390557,1040.83424365)(632.93890545,1040.85424363)(633.06890869,1040.86425148)
\curveto(633.19890519,1040.87424361)(633.32890506,1040.89424359)(633.45890869,1040.92425148)
\curveto(633.49890489,1040.93424355)(633.55890483,1040.93424355)(633.63890869,1040.92425148)
\curveto(633.72890466,1040.91424357)(633.7839046,1040.91924357)(633.80390869,1040.93925148)
\curveto(634.21390417,1040.94924354)(634.60390378,1040.93424355)(634.97390869,1040.89425148)
\curveto(635.35390303,1040.85424363)(635.69390269,1040.77924371)(635.99390869,1040.66925148)
\curveto(636.30390208,1040.55924393)(636.56890182,1040.40924408)(636.78890869,1040.21925148)
\curveto(637.00890138,1040.03924445)(637.17890121,1039.80424468)(637.29890869,1039.51425148)
\curveto(637.36890102,1039.34424514)(637.40890098,1039.14924534)(637.41890869,1038.92925148)
\curveto(637.42890096,1038.70924578)(637.43390095,1038.484246)(637.43390869,1038.25425148)
\lineto(637.43390869,1034.90925148)
\lineto(637.43390869,1034.32425148)
\curveto(637.43390095,1034.13425035)(637.45390093,1033.95925053)(637.49390869,1033.79925148)
\curveto(637.50390088,1033.76925072)(637.50890088,1033.73425075)(637.50890869,1033.69425148)
\curveto(637.50890088,1033.66425082)(637.51390087,1033.63425085)(637.52390869,1033.60425148)
\moveto(635.31890869,1035.91425148)
\curveto(635.32890306,1035.96424852)(635.33390305,1036.01924847)(635.33390869,1036.07925148)
\curveto(635.33390305,1036.14924834)(635.32890306,1036.20924828)(635.31890869,1036.25925148)
\curveto(635.29890309,1036.31924817)(635.2889031,1036.37424811)(635.28890869,1036.42425148)
\curveto(635.2889031,1036.47424801)(635.26890312,1036.51424797)(635.22890869,1036.54425148)
\curveto(635.17890321,1036.5842479)(635.10390328,1036.60424788)(635.00390869,1036.60425148)
\curveto(634.96390342,1036.59424789)(634.92890346,1036.5842479)(634.89890869,1036.57425148)
\curveto(634.86890352,1036.57424791)(634.83390355,1036.56924792)(634.79390869,1036.55925148)
\curveto(634.72390366,1036.53924795)(634.64890374,1036.52424796)(634.56890869,1036.51425148)
\curveto(634.4889039,1036.50424798)(634.40890398,1036.489248)(634.32890869,1036.46925148)
\curveto(634.29890409,1036.45924803)(634.25390413,1036.45424803)(634.19390869,1036.45425148)
\curveto(634.06390432,1036.42424806)(633.93390445,1036.40424808)(633.80390869,1036.39425148)
\curveto(633.67390471,1036.3842481)(633.54890484,1036.35924813)(633.42890869,1036.31925148)
\curveto(633.34890504,1036.29924819)(633.27390511,1036.27924821)(633.20390869,1036.25925148)
\curveto(633.13390525,1036.24924824)(633.06390532,1036.22924826)(632.99390869,1036.19925148)
\curveto(632.7839056,1036.10924838)(632.60390578,1035.97424851)(632.45390869,1035.79425148)
\curveto(632.31390607,1035.61424887)(632.26390612,1035.36424912)(632.30390869,1035.04425148)
\curveto(632.32390606,1034.87424961)(632.37890601,1034.73424975)(632.46890869,1034.62425148)
\curveto(632.53890585,1034.51424997)(632.64390574,1034.42425006)(632.78390869,1034.35425148)
\curveto(632.92390546,1034.29425019)(633.07390531,1034.24925024)(633.23390869,1034.21925148)
\curveto(633.40390498,1034.1892503)(633.57890481,1034.17925031)(633.75890869,1034.18925148)
\curveto(633.94890444,1034.20925028)(634.12390426,1034.24425024)(634.28390869,1034.29425148)
\curveto(634.54390384,1034.37425011)(634.74890364,1034.49924999)(634.89890869,1034.66925148)
\curveto(635.04890334,1034.84924964)(635.16390322,1035.06924942)(635.24390869,1035.32925148)
\curveto(635.26390312,1035.39924909)(635.27390311,1035.46924902)(635.27390869,1035.53925148)
\curveto(635.2839031,1035.61924887)(635.29890309,1035.69924879)(635.31890869,1035.77925148)
\lineto(635.31890869,1035.91425148)
}
}
{
\newrgbcolor{curcolor}{0 0 0}
\pscustom[linestyle=none,fillstyle=solid,fillcolor=curcolor]
{
\newpath
\moveto(395.05336182,1018.21425148)
\curveto(395.06335314,1018.15424758)(395.06835313,1018.06424767)(395.06836182,1017.94425148)
\curveto(395.06835313,1017.82424791)(395.05835314,1017.739248)(395.03836182,1017.68925148)
\lineto(395.03836182,1017.49425148)
\curveto(395.00835319,1017.38424835)(394.98835321,1017.27924846)(394.97836182,1017.17925148)
\curveto(394.97835322,1017.07924866)(394.96335324,1016.97924876)(394.93336182,1016.87925148)
\curveto(394.91335329,1016.78924895)(394.89335331,1016.69424904)(394.87336182,1016.59425148)
\curveto(394.85335335,1016.50424923)(394.82335338,1016.41424932)(394.78336182,1016.32425148)
\curveto(394.71335349,1016.15424958)(394.64335356,1015.99424974)(394.57336182,1015.84425148)
\curveto(394.5033537,1015.70425003)(394.42335378,1015.56425017)(394.33336182,1015.42425148)
\curveto(394.27335393,1015.3342504)(394.20835399,1015.24925049)(394.13836182,1015.16925148)
\curveto(394.07835412,1015.09925064)(394.00835419,1015.02425071)(393.92836182,1014.94425148)
\lineto(393.82336182,1014.83925148)
\curveto(393.77335443,1014.78925095)(393.71835448,1014.74425099)(393.65836182,1014.70425148)
\lineto(393.50836182,1014.58425148)
\curveto(393.42835477,1014.52425121)(393.33835486,1014.46925127)(393.23836182,1014.41925148)
\curveto(393.14835505,1014.37925136)(393.05335515,1014.3342514)(392.95336182,1014.28425148)
\curveto(392.85335535,1014.2342515)(392.74835545,1014.19925154)(392.63836182,1014.17925148)
\curveto(392.53835566,1014.15925158)(392.43335577,1014.1392516)(392.32336182,1014.11925148)
\curveto(392.26335594,1014.09925164)(392.198356,1014.08925165)(392.12836182,1014.08925148)
\curveto(392.06835613,1014.08925165)(392.0033562,1014.07925166)(391.93336182,1014.05925148)
\lineto(391.79836182,1014.05925148)
\curveto(391.71835648,1014.0392517)(391.64335656,1014.0392517)(391.57336182,1014.05925148)
\lineto(391.42336182,1014.05925148)
\curveto(391.36335684,1014.07925166)(391.2983569,1014.08925165)(391.22836182,1014.08925148)
\curveto(391.16835703,1014.07925166)(391.10835709,1014.08425165)(391.04836182,1014.10425148)
\curveto(390.88835731,1014.15425158)(390.73335747,1014.19925154)(390.58336182,1014.23925148)
\curveto(390.44335776,1014.27925146)(390.31335789,1014.3392514)(390.19336182,1014.41925148)
\curveto(390.12335808,1014.45925128)(390.05835814,1014.49925124)(389.99836182,1014.53925148)
\curveto(389.93835826,1014.58925115)(389.87335833,1014.6392511)(389.80336182,1014.68925148)
\lineto(389.62336182,1014.82425148)
\curveto(389.54335866,1014.88425085)(389.47335873,1014.88925085)(389.41336182,1014.83925148)
\curveto(389.36335884,1014.80925093)(389.33835886,1014.76925097)(389.33836182,1014.71925148)
\curveto(389.33835886,1014.67925106)(389.32835887,1014.62925111)(389.30836182,1014.56925148)
\curveto(389.28835891,1014.46925127)(389.27835892,1014.35425138)(389.27836182,1014.22425148)
\curveto(389.28835891,1014.09425164)(389.29335891,1013.97425176)(389.29336182,1013.86425148)
\lineto(389.29336182,1012.33425148)
\curveto(389.29335891,1012.20425353)(389.28835891,1012.07925366)(389.27836182,1011.95925148)
\curveto(389.27835892,1011.82925391)(389.25335895,1011.72425401)(389.20336182,1011.64425148)
\curveto(389.17335903,1011.60425413)(389.11835908,1011.57425416)(389.03836182,1011.55425148)
\curveto(388.95835924,1011.5342542)(388.86835933,1011.52425421)(388.76836182,1011.52425148)
\curveto(388.66835953,1011.51425422)(388.56835963,1011.51425422)(388.46836182,1011.52425148)
\lineto(388.21336182,1011.52425148)
\lineto(387.80836182,1011.52425148)
\lineto(387.70336182,1011.52425148)
\curveto(387.66336054,1011.52425421)(387.62836057,1011.52925421)(387.59836182,1011.53925148)
\lineto(387.47836182,1011.53925148)
\curveto(387.30836089,1011.58925415)(387.21836098,1011.68925405)(387.20836182,1011.83925148)
\curveto(387.198361,1011.97925376)(387.19336101,1012.14925359)(387.19336182,1012.34925148)
\lineto(387.19336182,1021.15425148)
\curveto(387.19336101,1021.26424447)(387.18836101,1021.37924436)(387.17836182,1021.49925148)
\curveto(387.17836102,1021.62924411)(387.203361,1021.72924401)(387.25336182,1021.79925148)
\curveto(387.29336091,1021.86924387)(387.34836085,1021.91424382)(387.41836182,1021.93425148)
\curveto(387.46836073,1021.95424378)(387.52836067,1021.96424377)(387.59836182,1021.96425148)
\lineto(387.82336182,1021.96425148)
\lineto(388.54336182,1021.96425148)
\lineto(388.82836182,1021.96425148)
\curveto(388.91835928,1021.96424377)(388.99335921,1021.9392438)(389.05336182,1021.88925148)
\curveto(389.12335908,1021.8392439)(389.15835904,1021.77424396)(389.15836182,1021.69425148)
\curveto(389.16835903,1021.62424411)(389.19335901,1021.54924419)(389.23336182,1021.46925148)
\curveto(389.24335896,1021.4392443)(389.25335895,1021.41424432)(389.26336182,1021.39425148)
\curveto(389.28335892,1021.38424435)(389.3033589,1021.36924437)(389.32336182,1021.34925148)
\curveto(389.43335877,1021.3392444)(389.52335868,1021.36924437)(389.59336182,1021.43925148)
\curveto(389.66335854,1021.50924423)(389.73335847,1021.56924417)(389.80336182,1021.61925148)
\curveto(389.93335827,1021.70924403)(390.06835813,1021.78924395)(390.20836182,1021.85925148)
\curveto(390.34835785,1021.9392438)(390.5033577,1022.00424373)(390.67336182,1022.05425148)
\curveto(390.75335745,1022.08424365)(390.83835736,1022.10424363)(390.92836182,1022.11425148)
\curveto(391.02835717,1022.12424361)(391.12335708,1022.1392436)(391.21336182,1022.15925148)
\curveto(391.25335695,1022.16924357)(391.29335691,1022.16924357)(391.33336182,1022.15925148)
\curveto(391.38335682,1022.14924359)(391.42335678,1022.15424358)(391.45336182,1022.17425148)
\curveto(392.02335618,1022.19424354)(392.5033557,1022.11424362)(392.89336182,1021.93425148)
\curveto(393.29335491,1021.76424397)(393.63335457,1021.5392442)(393.91336182,1021.25925148)
\curveto(393.96335424,1021.20924453)(394.00835419,1021.15924458)(394.04836182,1021.10925148)
\curveto(394.08835411,1021.06924467)(394.12835407,1021.02424471)(394.16836182,1020.97425148)
\curveto(394.23835396,1020.88424485)(394.2983539,1020.79424494)(394.34836182,1020.70425148)
\curveto(394.40835379,1020.61424512)(394.46335374,1020.52424521)(394.51336182,1020.43425148)
\curveto(394.53335367,1020.41424532)(394.54335366,1020.38924535)(394.54336182,1020.35925148)
\curveto(394.55335365,1020.32924541)(394.56835363,1020.29424544)(394.58836182,1020.25425148)
\curveto(394.64835355,1020.15424558)(394.7033535,1020.0342457)(394.75336182,1019.89425148)
\curveto(394.77335343,1019.8342459)(394.79335341,1019.76924597)(394.81336182,1019.69925148)
\curveto(394.83335337,1019.6392461)(394.85335335,1019.57424616)(394.87336182,1019.50425148)
\curveto(394.91335329,1019.38424635)(394.93835326,1019.25924648)(394.94836182,1019.12925148)
\curveto(394.96835323,1018.99924674)(394.99335321,1018.86424687)(395.02336182,1018.72425148)
\lineto(395.02336182,1018.55925148)
\lineto(395.05336182,1018.37925148)
\lineto(395.05336182,1018.21425148)
\moveto(392.93836182,1017.86925148)
\curveto(392.94835525,1017.91924782)(392.95335525,1017.98424775)(392.95336182,1018.06425148)
\curveto(392.95335525,1018.15424758)(392.94835525,1018.22424751)(392.93836182,1018.27425148)
\lineto(392.93836182,1018.40925148)
\curveto(392.91835528,1018.46924727)(392.90835529,1018.5342472)(392.90836182,1018.60425148)
\curveto(392.90835529,1018.67424706)(392.8983553,1018.74424699)(392.87836182,1018.81425148)
\curveto(392.85835534,1018.91424682)(392.83835536,1019.00924673)(392.81836182,1019.09925148)
\curveto(392.7983554,1019.19924654)(392.76835543,1019.28924645)(392.72836182,1019.36925148)
\curveto(392.60835559,1019.68924605)(392.45335575,1019.94424579)(392.26336182,1020.13425148)
\curveto(392.07335613,1020.32424541)(391.8033564,1020.46424527)(391.45336182,1020.55425148)
\curveto(391.37335683,1020.57424516)(391.28335692,1020.58424515)(391.18336182,1020.58425148)
\lineto(390.91336182,1020.58425148)
\curveto(390.87335733,1020.57424516)(390.83835736,1020.56924517)(390.80836182,1020.56925148)
\curveto(390.77835742,1020.56924517)(390.74335746,1020.56424517)(390.70336182,1020.55425148)
\lineto(390.49336182,1020.49425148)
\curveto(390.43335777,1020.48424525)(390.37335783,1020.46424527)(390.31336182,1020.43425148)
\curveto(390.05335815,1020.32424541)(389.84835835,1020.15424558)(389.69836182,1019.92425148)
\curveto(389.55835864,1019.69424604)(389.44335876,1019.4392463)(389.35336182,1019.15925148)
\curveto(389.33335887,1019.07924666)(389.31835888,1018.99424674)(389.30836182,1018.90425148)
\curveto(389.2983589,1018.82424691)(389.28335892,1018.74424699)(389.26336182,1018.66425148)
\curveto(389.25335895,1018.62424711)(389.24835895,1018.55924718)(389.24836182,1018.46925148)
\curveto(389.22835897,1018.42924731)(389.22335898,1018.37924736)(389.23336182,1018.31925148)
\curveto(389.24335896,1018.26924747)(389.24335896,1018.21924752)(389.23336182,1018.16925148)
\curveto(389.21335899,1018.10924763)(389.21335899,1018.05424768)(389.23336182,1018.00425148)
\lineto(389.23336182,1017.82425148)
\lineto(389.23336182,1017.68925148)
\curveto(389.23335897,1017.64924809)(389.24335896,1017.60924813)(389.26336182,1017.56925148)
\curveto(389.26335894,1017.49924824)(389.26835893,1017.44424829)(389.27836182,1017.40425148)
\lineto(389.30836182,1017.22425148)
\curveto(389.31835888,1017.16424857)(389.33335887,1017.10424863)(389.35336182,1017.04425148)
\curveto(389.44335876,1016.75424898)(389.54835865,1016.51424922)(389.66836182,1016.32425148)
\curveto(389.7983584,1016.14424959)(389.97835822,1015.98424975)(390.20836182,1015.84425148)
\curveto(390.34835785,1015.76424997)(390.51335769,1015.69925004)(390.70336182,1015.64925148)
\curveto(390.74335746,1015.6392501)(390.77835742,1015.6342501)(390.80836182,1015.63425148)
\curveto(390.83835736,1015.64425009)(390.87335733,1015.64425009)(390.91336182,1015.63425148)
\curveto(390.95335725,1015.62425011)(391.01335719,1015.61425012)(391.09336182,1015.60425148)
\curveto(391.17335703,1015.60425013)(391.23835696,1015.60925013)(391.28836182,1015.61925148)
\curveto(391.36835683,1015.6392501)(391.44835675,1015.65425008)(391.52836182,1015.66425148)
\curveto(391.61835658,1015.68425005)(391.7033565,1015.70925003)(391.78336182,1015.73925148)
\curveto(392.02335618,1015.8392499)(392.21835598,1015.97924976)(392.36836182,1016.15925148)
\curveto(392.51835568,1016.3392494)(392.64335556,1016.54924919)(392.74336182,1016.78925148)
\curveto(392.79335541,1016.90924883)(392.82835537,1017.0342487)(392.84836182,1017.16425148)
\curveto(392.86835533,1017.29424844)(392.89335531,1017.42924831)(392.92336182,1017.56925148)
\lineto(392.92336182,1017.71925148)
\curveto(392.93335527,1017.76924797)(392.93835526,1017.81924792)(392.93836182,1017.86925148)
}
}
{
\newrgbcolor{curcolor}{0 0 0}
\pscustom[linestyle=none,fillstyle=solid,fillcolor=curcolor]
{
\newpath
\moveto(404.10328369,1018.43925148)
\curveto(404.12327512,1018.37924736)(404.13327511,1018.29424744)(404.13328369,1018.18425148)
\curveto(404.13327511,1018.07424766)(404.12327512,1017.98924775)(404.10328369,1017.92925148)
\lineto(404.10328369,1017.77925148)
\curveto(404.08327516,1017.69924804)(404.07327517,1017.61924812)(404.07328369,1017.53925148)
\curveto(404.08327516,1017.45924828)(404.07827517,1017.37924836)(404.05828369,1017.29925148)
\curveto(404.03827521,1017.22924851)(404.02327522,1017.16424857)(404.01328369,1017.10425148)
\curveto(404.00327524,1017.04424869)(403.99327525,1016.97924876)(403.98328369,1016.90925148)
\curveto(403.9432753,1016.79924894)(403.90827534,1016.68424905)(403.87828369,1016.56425148)
\curveto(403.8482754,1016.45424928)(403.80827544,1016.34924939)(403.75828369,1016.24925148)
\curveto(403.5482757,1015.76924997)(403.27327597,1015.37925036)(402.93328369,1015.07925148)
\curveto(402.59327665,1014.77925096)(402.18327706,1014.52925121)(401.70328369,1014.32925148)
\curveto(401.58327766,1014.27925146)(401.45827779,1014.24425149)(401.32828369,1014.22425148)
\curveto(401.20827804,1014.19425154)(401.08327816,1014.16425157)(400.95328369,1014.13425148)
\curveto(400.90327834,1014.11425162)(400.8482784,1014.10425163)(400.78828369,1014.10425148)
\curveto(400.72827852,1014.10425163)(400.67327857,1014.09925164)(400.62328369,1014.08925148)
\lineto(400.51828369,1014.08925148)
\curveto(400.48827876,1014.07925166)(400.45827879,1014.07425166)(400.42828369,1014.07425148)
\curveto(400.37827887,1014.06425167)(400.29827895,1014.05925168)(400.18828369,1014.05925148)
\curveto(400.07827917,1014.04925169)(399.99327925,1014.05425168)(399.93328369,1014.07425148)
\lineto(399.78328369,1014.07425148)
\curveto(399.73327951,1014.08425165)(399.67827957,1014.08925165)(399.61828369,1014.08925148)
\curveto(399.56827968,1014.07925166)(399.51827973,1014.08425165)(399.46828369,1014.10425148)
\curveto(399.42827982,1014.11425162)(399.38827986,1014.11925162)(399.34828369,1014.11925148)
\curveto(399.31827993,1014.11925162)(399.27827997,1014.12425161)(399.22828369,1014.13425148)
\curveto(399.12828012,1014.16425157)(399.02828022,1014.18925155)(398.92828369,1014.20925148)
\curveto(398.82828042,1014.22925151)(398.73328051,1014.25925148)(398.64328369,1014.29925148)
\curveto(398.52328072,1014.3392514)(398.40828084,1014.37925136)(398.29828369,1014.41925148)
\curveto(398.19828105,1014.45925128)(398.09328115,1014.50925123)(397.98328369,1014.56925148)
\curveto(397.63328161,1014.77925096)(397.33328191,1015.02425071)(397.08328369,1015.30425148)
\curveto(396.83328241,1015.58425015)(396.62328262,1015.91924982)(396.45328369,1016.30925148)
\curveto(396.40328284,1016.39924934)(396.36328288,1016.49424924)(396.33328369,1016.59425148)
\curveto(396.31328293,1016.69424904)(396.28828296,1016.79924894)(396.25828369,1016.90925148)
\curveto(396.23828301,1016.95924878)(396.22828302,1017.00424873)(396.22828369,1017.04425148)
\curveto(396.22828302,1017.08424865)(396.21828303,1017.12924861)(396.19828369,1017.17925148)
\curveto(396.17828307,1017.25924848)(396.16828308,1017.3392484)(396.16828369,1017.41925148)
\curveto(396.16828308,1017.50924823)(396.15828309,1017.59424814)(396.13828369,1017.67425148)
\curveto(396.12828312,1017.72424801)(396.12328312,1017.76924797)(396.12328369,1017.80925148)
\lineto(396.12328369,1017.94425148)
\curveto(396.10328314,1018.00424773)(396.09328315,1018.08924765)(396.09328369,1018.19925148)
\curveto(396.10328314,1018.30924743)(396.11828313,1018.39424734)(396.13828369,1018.45425148)
\lineto(396.13828369,1018.55925148)
\curveto(396.1482831,1018.60924713)(396.1482831,1018.65924708)(396.13828369,1018.70925148)
\curveto(396.13828311,1018.76924697)(396.1482831,1018.82424691)(396.16828369,1018.87425148)
\curveto(396.17828307,1018.92424681)(396.18328306,1018.96924677)(396.18328369,1019.00925148)
\curveto(396.18328306,1019.05924668)(396.19328305,1019.10924663)(396.21328369,1019.15925148)
\curveto(396.25328299,1019.28924645)(396.28828296,1019.41424632)(396.31828369,1019.53425148)
\curveto(396.3482829,1019.66424607)(396.38828286,1019.78924595)(396.43828369,1019.90925148)
\curveto(396.61828263,1020.31924542)(396.83328241,1020.65924508)(397.08328369,1020.92925148)
\curveto(397.33328191,1021.20924453)(397.63828161,1021.46424427)(397.99828369,1021.69425148)
\curveto(398.09828115,1021.74424399)(398.20328104,1021.78924395)(398.31328369,1021.82925148)
\curveto(398.42328082,1021.86924387)(398.53328071,1021.91424382)(398.64328369,1021.96425148)
\curveto(398.77328047,1022.01424372)(398.90828034,1022.04924369)(399.04828369,1022.06925148)
\curveto(399.18828006,1022.08924365)(399.33327991,1022.11924362)(399.48328369,1022.15925148)
\curveto(399.56327968,1022.16924357)(399.63827961,1022.17424356)(399.70828369,1022.17425148)
\curveto(399.77827947,1022.17424356)(399.8482794,1022.17924356)(399.91828369,1022.18925148)
\curveto(400.49827875,1022.19924354)(400.99827825,1022.1392436)(401.41828369,1022.00925148)
\curveto(401.8482774,1021.87924386)(402.22827702,1021.69924404)(402.55828369,1021.46925148)
\curveto(402.66827658,1021.38924435)(402.77827647,1021.29924444)(402.88828369,1021.19925148)
\curveto(403.00827624,1021.10924463)(403.10827614,1021.00924473)(403.18828369,1020.89925148)
\curveto(403.26827598,1020.79924494)(403.33827591,1020.69924504)(403.39828369,1020.59925148)
\curveto(403.46827578,1020.49924524)(403.53827571,1020.39424534)(403.60828369,1020.28425148)
\curveto(403.67827557,1020.17424556)(403.73327551,1020.05424568)(403.77328369,1019.92425148)
\curveto(403.81327543,1019.80424593)(403.85827539,1019.67424606)(403.90828369,1019.53425148)
\curveto(403.93827531,1019.45424628)(403.96327528,1019.36924637)(403.98328369,1019.27925148)
\lineto(404.04328369,1019.00925148)
\curveto(404.05327519,1018.96924677)(404.05827519,1018.92924681)(404.05828369,1018.88925148)
\curveto(404.05827519,1018.84924689)(404.06327518,1018.80924693)(404.07328369,1018.76925148)
\curveto(404.09327515,1018.71924702)(404.09827515,1018.66424707)(404.08828369,1018.60425148)
\curveto(404.07827517,1018.54424719)(404.08327516,1018.48924725)(404.10328369,1018.43925148)
\moveto(402.00328369,1017.89925148)
\curveto(402.01327723,1017.94924779)(402.01827723,1018.01924772)(402.01828369,1018.10925148)
\curveto(402.01827723,1018.20924753)(402.01327723,1018.28424745)(402.00328369,1018.33425148)
\lineto(402.00328369,1018.45425148)
\curveto(401.98327726,1018.50424723)(401.97327727,1018.55924718)(401.97328369,1018.61925148)
\curveto(401.97327727,1018.67924706)(401.96827728,1018.734247)(401.95828369,1018.78425148)
\curveto(401.95827729,1018.82424691)(401.95327729,1018.85424688)(401.94328369,1018.87425148)
\lineto(401.88328369,1019.11425148)
\curveto(401.87327737,1019.20424653)(401.85327739,1019.28924645)(401.82328369,1019.36925148)
\curveto(401.71327753,1019.62924611)(401.58327766,1019.84924589)(401.43328369,1020.02925148)
\curveto(401.28327796,1020.21924552)(401.08327816,1020.36924537)(400.83328369,1020.47925148)
\curveto(400.77327847,1020.49924524)(400.71327853,1020.51424522)(400.65328369,1020.52425148)
\curveto(400.59327865,1020.54424519)(400.52827872,1020.56424517)(400.45828369,1020.58425148)
\curveto(400.37827887,1020.60424513)(400.29327895,1020.60924513)(400.20328369,1020.59925148)
\lineto(399.93328369,1020.59925148)
\curveto(399.90327934,1020.57924516)(399.86827938,1020.56924517)(399.82828369,1020.56925148)
\curveto(399.78827946,1020.57924516)(399.75327949,1020.57924516)(399.72328369,1020.56925148)
\lineto(399.51328369,1020.50925148)
\curveto(399.45327979,1020.49924524)(399.39827985,1020.47924526)(399.34828369,1020.44925148)
\curveto(399.09828015,1020.3392454)(398.89328035,1020.17924556)(398.73328369,1019.96925148)
\curveto(398.58328066,1019.76924597)(398.46328078,1019.5342462)(398.37328369,1019.26425148)
\curveto(398.3432809,1019.16424657)(398.31828093,1019.05924668)(398.29828369,1018.94925148)
\curveto(398.28828096,1018.8392469)(398.27328097,1018.72924701)(398.25328369,1018.61925148)
\curveto(398.243281,1018.56924717)(398.23828101,1018.51924722)(398.23828369,1018.46925148)
\lineto(398.23828369,1018.31925148)
\curveto(398.21828103,1018.24924749)(398.20828104,1018.14424759)(398.20828369,1018.00425148)
\curveto(398.21828103,1017.86424787)(398.23328101,1017.75924798)(398.25328369,1017.68925148)
\lineto(398.25328369,1017.55425148)
\curveto(398.27328097,1017.47424826)(398.28828096,1017.39424834)(398.29828369,1017.31425148)
\curveto(398.30828094,1017.24424849)(398.32328092,1017.16924857)(398.34328369,1017.08925148)
\curveto(398.4432808,1016.78924895)(398.5482807,1016.54424919)(398.65828369,1016.35425148)
\curveto(398.77828047,1016.17424956)(398.96328028,1016.00924973)(399.21328369,1015.85925148)
\curveto(399.28327996,1015.80924993)(399.35827989,1015.76924997)(399.43828369,1015.73925148)
\curveto(399.52827972,1015.70925003)(399.61827963,1015.68425005)(399.70828369,1015.66425148)
\curveto(399.7482795,1015.65425008)(399.78327946,1015.64925009)(399.81328369,1015.64925148)
\curveto(399.8432794,1015.65925008)(399.87827937,1015.65925008)(399.91828369,1015.64925148)
\lineto(400.03828369,1015.61925148)
\curveto(400.08827916,1015.61925012)(400.13327911,1015.62425011)(400.17328369,1015.63425148)
\lineto(400.29328369,1015.63425148)
\curveto(400.37327887,1015.65425008)(400.45327879,1015.66925007)(400.53328369,1015.67925148)
\curveto(400.61327863,1015.68925005)(400.68827856,1015.70925003)(400.75828369,1015.73925148)
\curveto(401.01827823,1015.8392499)(401.22827802,1015.97424976)(401.38828369,1016.14425148)
\curveto(401.5482777,1016.31424942)(401.68327756,1016.52424921)(401.79328369,1016.77425148)
\curveto(401.83327741,1016.87424886)(401.86327738,1016.97424876)(401.88328369,1017.07425148)
\curveto(401.90327734,1017.17424856)(401.92827732,1017.27924846)(401.95828369,1017.38925148)
\curveto(401.96827728,1017.42924831)(401.97327727,1017.46424827)(401.97328369,1017.49425148)
\curveto(401.97327727,1017.5342482)(401.97827727,1017.57424816)(401.98828369,1017.61425148)
\lineto(401.98828369,1017.74925148)
\curveto(401.98827726,1017.79924794)(401.99327725,1017.84924789)(402.00328369,1017.89925148)
}
}
{
\newrgbcolor{curcolor}{0 0 0}
\pscustom[linestyle=none,fillstyle=solid,fillcolor=curcolor]
{
\newpath
\moveto(409.92820557,1022.18925148)
\curveto(410.03820025,1022.18924355)(410.13320016,1022.17924356)(410.21320557,1022.15925148)
\curveto(410.30319999,1022.1392436)(410.37319992,1022.09424364)(410.42320557,1022.02425148)
\curveto(410.48319981,1021.94424379)(410.51319978,1021.80424393)(410.51320557,1021.60425148)
\lineto(410.51320557,1021.09425148)
\lineto(410.51320557,1020.71925148)
\curveto(410.52319977,1020.57924516)(410.50819978,1020.46924527)(410.46820557,1020.38925148)
\curveto(410.42819986,1020.31924542)(410.36819992,1020.27424546)(410.28820557,1020.25425148)
\curveto(410.21820007,1020.2342455)(410.13320016,1020.22424551)(410.03320557,1020.22425148)
\curveto(409.94320035,1020.22424551)(409.84320045,1020.22924551)(409.73320557,1020.23925148)
\curveto(409.63320066,1020.24924549)(409.53820075,1020.24424549)(409.44820557,1020.22425148)
\curveto(409.37820091,1020.20424553)(409.30820098,1020.18924555)(409.23820557,1020.17925148)
\curveto(409.16820112,1020.17924556)(409.10320119,1020.16924557)(409.04320557,1020.14925148)
\curveto(408.88320141,1020.09924564)(408.72320157,1020.02424571)(408.56320557,1019.92425148)
\curveto(408.40320189,1019.8342459)(408.27820201,1019.72924601)(408.18820557,1019.60925148)
\curveto(408.13820215,1019.52924621)(408.08320221,1019.44424629)(408.02320557,1019.35425148)
\curveto(407.97320232,1019.27424646)(407.92320237,1019.18924655)(407.87320557,1019.09925148)
\curveto(407.84320245,1019.01924672)(407.81320248,1018.9342468)(407.78320557,1018.84425148)
\lineto(407.72320557,1018.60425148)
\curveto(407.70320259,1018.5342472)(407.6932026,1018.45924728)(407.69320557,1018.37925148)
\curveto(407.6932026,1018.30924743)(407.68320261,1018.2392475)(407.66320557,1018.16925148)
\curveto(407.65320264,1018.12924761)(407.64820264,1018.08924765)(407.64820557,1018.04925148)
\curveto(407.65820263,1018.01924772)(407.65820263,1017.98924775)(407.64820557,1017.95925148)
\lineto(407.64820557,1017.71925148)
\curveto(407.62820266,1017.64924809)(407.62320267,1017.56924817)(407.63320557,1017.47925148)
\curveto(407.64320265,1017.39924834)(407.64820264,1017.31924842)(407.64820557,1017.23925148)
\lineto(407.64820557,1016.27925148)
\lineto(407.64820557,1015.00425148)
\curveto(407.64820264,1014.87425086)(407.64320265,1014.75425098)(407.63320557,1014.64425148)
\curveto(407.62320267,1014.5342512)(407.5932027,1014.44425129)(407.54320557,1014.37425148)
\curveto(407.52320277,1014.34425139)(407.4882028,1014.31925142)(407.43820557,1014.29925148)
\curveto(407.39820289,1014.28925145)(407.35320294,1014.27925146)(407.30320557,1014.26925148)
\lineto(407.22820557,1014.26925148)
\curveto(407.17820311,1014.25925148)(407.12320317,1014.25425148)(407.06320557,1014.25425148)
\lineto(406.89820557,1014.25425148)
\lineto(406.25320557,1014.25425148)
\curveto(406.1932041,1014.26425147)(406.12820416,1014.26925147)(406.05820557,1014.26925148)
\lineto(405.86320557,1014.26925148)
\curveto(405.81320448,1014.28925145)(405.76320453,1014.30425143)(405.71320557,1014.31425148)
\curveto(405.66320463,1014.3342514)(405.62820466,1014.36925137)(405.60820557,1014.41925148)
\curveto(405.56820472,1014.46925127)(405.54320475,1014.5392512)(405.53320557,1014.62925148)
\lineto(405.53320557,1014.92925148)
\lineto(405.53320557,1015.94925148)
\lineto(405.53320557,1020.17925148)
\lineto(405.53320557,1021.28925148)
\lineto(405.53320557,1021.57425148)
\curveto(405.53320476,1021.67424406)(405.55320474,1021.75424398)(405.59320557,1021.81425148)
\curveto(405.64320465,1021.89424384)(405.71820457,1021.94424379)(405.81820557,1021.96425148)
\curveto(405.91820437,1021.98424375)(406.03820425,1021.99424374)(406.17820557,1021.99425148)
\lineto(406.94320557,1021.99425148)
\curveto(407.06320323,1021.99424374)(407.16820312,1021.98424375)(407.25820557,1021.96425148)
\curveto(407.34820294,1021.95424378)(407.41820287,1021.90924383)(407.46820557,1021.82925148)
\curveto(407.49820279,1021.77924396)(407.51320278,1021.70924403)(407.51320557,1021.61925148)
\lineto(407.54320557,1021.34925148)
\curveto(407.55320274,1021.26924447)(407.56820272,1021.19424454)(407.58820557,1021.12425148)
\curveto(407.61820267,1021.05424468)(407.66820262,1021.01924472)(407.73820557,1021.01925148)
\curveto(407.75820253,1021.0392447)(407.77820251,1021.04924469)(407.79820557,1021.04925148)
\curveto(407.81820247,1021.04924469)(407.83820245,1021.05924468)(407.85820557,1021.07925148)
\curveto(407.91820237,1021.12924461)(407.96820232,1021.18424455)(408.00820557,1021.24425148)
\curveto(408.05820223,1021.31424442)(408.11820217,1021.37424436)(408.18820557,1021.42425148)
\curveto(408.22820206,1021.45424428)(408.26320203,1021.48424425)(408.29320557,1021.51425148)
\curveto(408.32320197,1021.55424418)(408.35820193,1021.58924415)(408.39820557,1021.61925148)
\lineto(408.66820557,1021.79925148)
\curveto(408.76820152,1021.85924388)(408.86820142,1021.91424382)(408.96820557,1021.96425148)
\curveto(409.06820122,1022.00424373)(409.16820112,1022.0392437)(409.26820557,1022.06925148)
\lineto(409.59820557,1022.15925148)
\curveto(409.62820066,1022.16924357)(409.68320061,1022.16924357)(409.76320557,1022.15925148)
\curveto(409.85320044,1022.15924358)(409.90820038,1022.16924357)(409.92820557,1022.18925148)
}
}
{
\newrgbcolor{curcolor}{0 0 0}
\pscustom[linestyle=none,fillstyle=solid,fillcolor=curcolor]
{
}
}
{
\newrgbcolor{curcolor}{0 0 0}
\pscustom[linestyle=none,fillstyle=solid,fillcolor=curcolor]
{
\newpath
\moveto(416.00343994,1024.94925148)
\lineto(417.09843994,1024.94925148)
\curveto(417.19843746,1024.94924079)(417.29343736,1024.94424079)(417.38343994,1024.93425148)
\curveto(417.47343718,1024.92424081)(417.54343711,1024.89424084)(417.59343994,1024.84425148)
\curveto(417.653437,1024.77424096)(417.68343697,1024.67924106)(417.68343994,1024.55925148)
\curveto(417.69343696,1024.44924129)(417.69843696,1024.3342414)(417.69843994,1024.21425148)
\lineto(417.69843994,1022.87925148)
\lineto(417.69843994,1017.49425148)
\lineto(417.69843994,1015.19925148)
\lineto(417.69843994,1014.77925148)
\curveto(417.70843695,1014.62925111)(417.68843697,1014.51425122)(417.63843994,1014.43425148)
\curveto(417.58843707,1014.35425138)(417.49843716,1014.29925144)(417.36843994,1014.26925148)
\curveto(417.30843735,1014.24925149)(417.23843742,1014.24425149)(417.15843994,1014.25425148)
\curveto(417.08843757,1014.26425147)(417.01843764,1014.26925147)(416.94843994,1014.26925148)
\lineto(416.22843994,1014.26925148)
\curveto(416.11843854,1014.26925147)(416.01843864,1014.27425146)(415.92843994,1014.28425148)
\curveto(415.83843882,1014.29425144)(415.76343889,1014.32425141)(415.70343994,1014.37425148)
\curveto(415.64343901,1014.42425131)(415.60843905,1014.49925124)(415.59843994,1014.59925148)
\lineto(415.59843994,1014.92925148)
\lineto(415.59843994,1016.26425148)
\lineto(415.59843994,1021.88925148)
\lineto(415.59843994,1023.92925148)
\curveto(415.59843906,1024.05924168)(415.59343906,1024.21424152)(415.58343994,1024.39425148)
\curveto(415.58343907,1024.57424116)(415.60843905,1024.70424103)(415.65843994,1024.78425148)
\curveto(415.67843898,1024.82424091)(415.70343895,1024.85424088)(415.73343994,1024.87425148)
\lineto(415.85343994,1024.93425148)
\curveto(415.87343878,1024.9342408)(415.89843876,1024.9342408)(415.92843994,1024.93425148)
\curveto(415.9584387,1024.94424079)(415.98343867,1024.94924079)(416.00343994,1024.94925148)
}
}
{
\newrgbcolor{curcolor}{0 0 0}
\pscustom[linestyle=none,fillstyle=solid,fillcolor=curcolor]
{
\newpath
\moveto(427.13062744,1018.43925148)
\curveto(427.15061887,1018.37924736)(427.16061886,1018.29424744)(427.16062744,1018.18425148)
\curveto(427.16061886,1018.07424766)(427.15061887,1017.98924775)(427.13062744,1017.92925148)
\lineto(427.13062744,1017.77925148)
\curveto(427.11061891,1017.69924804)(427.10061892,1017.61924812)(427.10062744,1017.53925148)
\curveto(427.11061891,1017.45924828)(427.10561892,1017.37924836)(427.08562744,1017.29925148)
\curveto(427.06561896,1017.22924851)(427.05061897,1017.16424857)(427.04062744,1017.10425148)
\curveto(427.03061899,1017.04424869)(427.020619,1016.97924876)(427.01062744,1016.90925148)
\curveto(426.97061905,1016.79924894)(426.93561909,1016.68424905)(426.90562744,1016.56425148)
\curveto(426.87561915,1016.45424928)(426.83561919,1016.34924939)(426.78562744,1016.24925148)
\curveto(426.57561945,1015.76924997)(426.30061972,1015.37925036)(425.96062744,1015.07925148)
\curveto(425.6206204,1014.77925096)(425.21062081,1014.52925121)(424.73062744,1014.32925148)
\curveto(424.61062141,1014.27925146)(424.48562154,1014.24425149)(424.35562744,1014.22425148)
\curveto(424.23562179,1014.19425154)(424.11062191,1014.16425157)(423.98062744,1014.13425148)
\curveto(423.93062209,1014.11425162)(423.87562215,1014.10425163)(423.81562744,1014.10425148)
\curveto(423.75562227,1014.10425163)(423.70062232,1014.09925164)(423.65062744,1014.08925148)
\lineto(423.54562744,1014.08925148)
\curveto(423.51562251,1014.07925166)(423.48562254,1014.07425166)(423.45562744,1014.07425148)
\curveto(423.40562262,1014.06425167)(423.3256227,1014.05925168)(423.21562744,1014.05925148)
\curveto(423.10562292,1014.04925169)(423.020623,1014.05425168)(422.96062744,1014.07425148)
\lineto(422.81062744,1014.07425148)
\curveto(422.76062326,1014.08425165)(422.70562332,1014.08925165)(422.64562744,1014.08925148)
\curveto(422.59562343,1014.07925166)(422.54562348,1014.08425165)(422.49562744,1014.10425148)
\curveto(422.45562357,1014.11425162)(422.41562361,1014.11925162)(422.37562744,1014.11925148)
\curveto(422.34562368,1014.11925162)(422.30562372,1014.12425161)(422.25562744,1014.13425148)
\curveto(422.15562387,1014.16425157)(422.05562397,1014.18925155)(421.95562744,1014.20925148)
\curveto(421.85562417,1014.22925151)(421.76062426,1014.25925148)(421.67062744,1014.29925148)
\curveto(421.55062447,1014.3392514)(421.43562459,1014.37925136)(421.32562744,1014.41925148)
\curveto(421.2256248,1014.45925128)(421.1206249,1014.50925123)(421.01062744,1014.56925148)
\curveto(420.66062536,1014.77925096)(420.36062566,1015.02425071)(420.11062744,1015.30425148)
\curveto(419.86062616,1015.58425015)(419.65062637,1015.91924982)(419.48062744,1016.30925148)
\curveto(419.43062659,1016.39924934)(419.39062663,1016.49424924)(419.36062744,1016.59425148)
\curveto(419.34062668,1016.69424904)(419.31562671,1016.79924894)(419.28562744,1016.90925148)
\curveto(419.26562676,1016.95924878)(419.25562677,1017.00424873)(419.25562744,1017.04425148)
\curveto(419.25562677,1017.08424865)(419.24562678,1017.12924861)(419.22562744,1017.17925148)
\curveto(419.20562682,1017.25924848)(419.19562683,1017.3392484)(419.19562744,1017.41925148)
\curveto(419.19562683,1017.50924823)(419.18562684,1017.59424814)(419.16562744,1017.67425148)
\curveto(419.15562687,1017.72424801)(419.15062687,1017.76924797)(419.15062744,1017.80925148)
\lineto(419.15062744,1017.94425148)
\curveto(419.13062689,1018.00424773)(419.1206269,1018.08924765)(419.12062744,1018.19925148)
\curveto(419.13062689,1018.30924743)(419.14562688,1018.39424734)(419.16562744,1018.45425148)
\lineto(419.16562744,1018.55925148)
\curveto(419.17562685,1018.60924713)(419.17562685,1018.65924708)(419.16562744,1018.70925148)
\curveto(419.16562686,1018.76924697)(419.17562685,1018.82424691)(419.19562744,1018.87425148)
\curveto(419.20562682,1018.92424681)(419.21062681,1018.96924677)(419.21062744,1019.00925148)
\curveto(419.21062681,1019.05924668)(419.2206268,1019.10924663)(419.24062744,1019.15925148)
\curveto(419.28062674,1019.28924645)(419.31562671,1019.41424632)(419.34562744,1019.53425148)
\curveto(419.37562665,1019.66424607)(419.41562661,1019.78924595)(419.46562744,1019.90925148)
\curveto(419.64562638,1020.31924542)(419.86062616,1020.65924508)(420.11062744,1020.92925148)
\curveto(420.36062566,1021.20924453)(420.66562536,1021.46424427)(421.02562744,1021.69425148)
\curveto(421.1256249,1021.74424399)(421.23062479,1021.78924395)(421.34062744,1021.82925148)
\curveto(421.45062457,1021.86924387)(421.56062446,1021.91424382)(421.67062744,1021.96425148)
\curveto(421.80062422,1022.01424372)(421.93562409,1022.04924369)(422.07562744,1022.06925148)
\curveto(422.21562381,1022.08924365)(422.36062366,1022.11924362)(422.51062744,1022.15925148)
\curveto(422.59062343,1022.16924357)(422.66562336,1022.17424356)(422.73562744,1022.17425148)
\curveto(422.80562322,1022.17424356)(422.87562315,1022.17924356)(422.94562744,1022.18925148)
\curveto(423.5256225,1022.19924354)(424.025622,1022.1392436)(424.44562744,1022.00925148)
\curveto(424.87562115,1021.87924386)(425.25562077,1021.69924404)(425.58562744,1021.46925148)
\curveto(425.69562033,1021.38924435)(425.80562022,1021.29924444)(425.91562744,1021.19925148)
\curveto(426.03561999,1021.10924463)(426.13561989,1021.00924473)(426.21562744,1020.89925148)
\curveto(426.29561973,1020.79924494)(426.36561966,1020.69924504)(426.42562744,1020.59925148)
\curveto(426.49561953,1020.49924524)(426.56561946,1020.39424534)(426.63562744,1020.28425148)
\curveto(426.70561932,1020.17424556)(426.76061926,1020.05424568)(426.80062744,1019.92425148)
\curveto(426.84061918,1019.80424593)(426.88561914,1019.67424606)(426.93562744,1019.53425148)
\curveto(426.96561906,1019.45424628)(426.99061903,1019.36924637)(427.01062744,1019.27925148)
\lineto(427.07062744,1019.00925148)
\curveto(427.08061894,1018.96924677)(427.08561894,1018.92924681)(427.08562744,1018.88925148)
\curveto(427.08561894,1018.84924689)(427.09061893,1018.80924693)(427.10062744,1018.76925148)
\curveto(427.1206189,1018.71924702)(427.1256189,1018.66424707)(427.11562744,1018.60425148)
\curveto(427.10561892,1018.54424719)(427.11061891,1018.48924725)(427.13062744,1018.43925148)
\moveto(425.03062744,1017.89925148)
\curveto(425.04062098,1017.94924779)(425.04562098,1018.01924772)(425.04562744,1018.10925148)
\curveto(425.04562098,1018.20924753)(425.04062098,1018.28424745)(425.03062744,1018.33425148)
\lineto(425.03062744,1018.45425148)
\curveto(425.01062101,1018.50424723)(425.00062102,1018.55924718)(425.00062744,1018.61925148)
\curveto(425.00062102,1018.67924706)(424.99562103,1018.734247)(424.98562744,1018.78425148)
\curveto(424.98562104,1018.82424691)(424.98062104,1018.85424688)(424.97062744,1018.87425148)
\lineto(424.91062744,1019.11425148)
\curveto(424.90062112,1019.20424653)(424.88062114,1019.28924645)(424.85062744,1019.36925148)
\curveto(424.74062128,1019.62924611)(424.61062141,1019.84924589)(424.46062744,1020.02925148)
\curveto(424.31062171,1020.21924552)(424.11062191,1020.36924537)(423.86062744,1020.47925148)
\curveto(423.80062222,1020.49924524)(423.74062228,1020.51424522)(423.68062744,1020.52425148)
\curveto(423.6206224,1020.54424519)(423.55562247,1020.56424517)(423.48562744,1020.58425148)
\curveto(423.40562262,1020.60424513)(423.3206227,1020.60924513)(423.23062744,1020.59925148)
\lineto(422.96062744,1020.59925148)
\curveto(422.93062309,1020.57924516)(422.89562313,1020.56924517)(422.85562744,1020.56925148)
\curveto(422.81562321,1020.57924516)(422.78062324,1020.57924516)(422.75062744,1020.56925148)
\lineto(422.54062744,1020.50925148)
\curveto(422.48062354,1020.49924524)(422.4256236,1020.47924526)(422.37562744,1020.44925148)
\curveto(422.1256239,1020.3392454)(421.9206241,1020.17924556)(421.76062744,1019.96925148)
\curveto(421.61062441,1019.76924597)(421.49062453,1019.5342462)(421.40062744,1019.26425148)
\curveto(421.37062465,1019.16424657)(421.34562468,1019.05924668)(421.32562744,1018.94925148)
\curveto(421.31562471,1018.8392469)(421.30062472,1018.72924701)(421.28062744,1018.61925148)
\curveto(421.27062475,1018.56924717)(421.26562476,1018.51924722)(421.26562744,1018.46925148)
\lineto(421.26562744,1018.31925148)
\curveto(421.24562478,1018.24924749)(421.23562479,1018.14424759)(421.23562744,1018.00425148)
\curveto(421.24562478,1017.86424787)(421.26062476,1017.75924798)(421.28062744,1017.68925148)
\lineto(421.28062744,1017.55425148)
\curveto(421.30062472,1017.47424826)(421.31562471,1017.39424834)(421.32562744,1017.31425148)
\curveto(421.33562469,1017.24424849)(421.35062467,1017.16924857)(421.37062744,1017.08925148)
\curveto(421.47062455,1016.78924895)(421.57562445,1016.54424919)(421.68562744,1016.35425148)
\curveto(421.80562422,1016.17424956)(421.99062403,1016.00924973)(422.24062744,1015.85925148)
\curveto(422.31062371,1015.80924993)(422.38562364,1015.76924997)(422.46562744,1015.73925148)
\curveto(422.55562347,1015.70925003)(422.64562338,1015.68425005)(422.73562744,1015.66425148)
\curveto(422.77562325,1015.65425008)(422.81062321,1015.64925009)(422.84062744,1015.64925148)
\curveto(422.87062315,1015.65925008)(422.90562312,1015.65925008)(422.94562744,1015.64925148)
\lineto(423.06562744,1015.61925148)
\curveto(423.11562291,1015.61925012)(423.16062286,1015.62425011)(423.20062744,1015.63425148)
\lineto(423.32062744,1015.63425148)
\curveto(423.40062262,1015.65425008)(423.48062254,1015.66925007)(423.56062744,1015.67925148)
\curveto(423.64062238,1015.68925005)(423.71562231,1015.70925003)(423.78562744,1015.73925148)
\curveto(424.04562198,1015.8392499)(424.25562177,1015.97424976)(424.41562744,1016.14425148)
\curveto(424.57562145,1016.31424942)(424.71062131,1016.52424921)(424.82062744,1016.77425148)
\curveto(424.86062116,1016.87424886)(424.89062113,1016.97424876)(424.91062744,1017.07425148)
\curveto(424.93062109,1017.17424856)(424.95562107,1017.27924846)(424.98562744,1017.38925148)
\curveto(424.99562103,1017.42924831)(425.00062102,1017.46424827)(425.00062744,1017.49425148)
\curveto(425.00062102,1017.5342482)(425.00562102,1017.57424816)(425.01562744,1017.61425148)
\lineto(425.01562744,1017.74925148)
\curveto(425.01562101,1017.79924794)(425.020621,1017.84924789)(425.03062744,1017.89925148)
}
}
{
\newrgbcolor{curcolor}{0 0 0}
\pscustom[linestyle=none,fillstyle=solid,fillcolor=curcolor]
{
}
}
{
\newrgbcolor{curcolor}{0 0 0}
\pscustom[linestyle=none,fillstyle=solid,fillcolor=curcolor]
{
\newpath
\moveto(437.14570557,1022.18925148)
\curveto(437.51569996,1022.19924354)(437.84069964,1022.15924358)(438.12070557,1022.06925148)
\curveto(438.40069908,1021.97924376)(438.64569883,1021.85424388)(438.85570557,1021.69425148)
\curveto(438.93569854,1021.6342441)(439.00569847,1021.56424417)(439.06570557,1021.48425148)
\curveto(439.13569834,1021.40424433)(439.21069827,1021.32424441)(439.29070557,1021.24425148)
\curveto(439.31069817,1021.22424451)(439.34069814,1021.19424454)(439.38070557,1021.15425148)
\curveto(439.43069805,1021.12424461)(439.480698,1021.11924462)(439.53070557,1021.13925148)
\curveto(439.64069784,1021.16924457)(439.74569773,1021.2392445)(439.84570557,1021.34925148)
\curveto(439.94569753,1021.46924427)(440.04069744,1021.55924418)(440.13070557,1021.61925148)
\curveto(440.27069721,1021.72924401)(440.42069706,1021.81924392)(440.58070557,1021.88925148)
\curveto(440.74069674,1021.96924377)(440.92069656,1022.04424369)(441.12070557,1022.11425148)
\curveto(441.20069628,1022.1342436)(441.29569618,1022.14924359)(441.40570557,1022.15925148)
\curveto(441.52569595,1022.17924356)(441.64569583,1022.18924355)(441.76570557,1022.18925148)
\curveto(441.89569558,1022.19924354)(442.01569546,1022.19924354)(442.12570557,1022.18925148)
\curveto(442.24569523,1022.17924356)(442.35069513,1022.16424357)(442.44070557,1022.14425148)
\curveto(442.49069499,1022.1342436)(442.53569494,1022.12924361)(442.57570557,1022.12925148)
\curveto(442.61569486,1022.12924361)(442.66069482,1022.11924362)(442.71070557,1022.09925148)
\curveto(442.85069463,1022.05924368)(442.98569449,1022.01924372)(443.11570557,1021.97925148)
\curveto(443.24569423,1021.9392438)(443.36569411,1021.88424385)(443.47570557,1021.81425148)
\curveto(443.89569358,1021.55424418)(444.21069327,1021.17424456)(444.42070557,1020.67425148)
\curveto(444.46069302,1020.58424515)(444.49069299,1020.48924525)(444.51070557,1020.38925148)
\curveto(444.53069295,1020.29924544)(444.55069293,1020.20924553)(444.57070557,1020.11925148)
\curveto(444.5806929,1020.04924569)(444.58569289,1019.98424575)(444.58570557,1019.92425148)
\curveto(444.59569288,1019.86424587)(444.60569287,1019.80424593)(444.61570557,1019.74425148)
\lineto(444.61570557,1019.59425148)
\curveto(444.62569285,1019.5342462)(444.62569285,1019.46424627)(444.61570557,1019.38425148)
\curveto(444.61569286,1019.30424643)(444.61569286,1019.22924651)(444.61570557,1019.15925148)
\lineto(444.61570557,1018.28925148)
\lineto(444.61570557,1015.36425148)
\curveto(444.61569286,1015.28425045)(444.61569286,1015.18925055)(444.61570557,1015.07925148)
\curveto(444.62569285,1014.97925076)(444.62569285,1014.87925086)(444.61570557,1014.77925148)
\curveto(444.61569286,1014.68925105)(444.60569287,1014.59925114)(444.58570557,1014.50925148)
\curveto(444.56569291,1014.42925131)(444.53569294,1014.37425136)(444.49570557,1014.34425148)
\curveto(444.43569304,1014.29425144)(444.35569312,1014.26425147)(444.25570557,1014.25425148)
\lineto(443.95570557,1014.25425148)
\lineto(443.16070557,1014.25425148)
\curveto(443.02069446,1014.25425148)(442.89569458,1014.26425147)(442.78570557,1014.28425148)
\curveto(442.6756948,1014.30425143)(442.60069488,1014.35925138)(442.56070557,1014.44925148)
\curveto(442.53069495,1014.51925122)(442.51569496,1014.59425114)(442.51570557,1014.67425148)
\curveto(442.51569496,1014.76425097)(442.51569496,1014.84925089)(442.51570557,1014.92925148)
\lineto(442.51570557,1015.76925148)
\lineto(442.51570557,1017.79425148)
\lineto(442.51570557,1018.42425148)
\curveto(442.51569496,1018.47424726)(442.51569496,1018.52924721)(442.51570557,1018.58925148)
\curveto(442.52569495,1018.64924709)(442.52069496,1018.70424703)(442.50070557,1018.75425148)
\lineto(442.50070557,1018.87425148)
\curveto(442.50069498,1018.9342468)(442.50069498,1018.99424674)(442.50070557,1019.05425148)
\curveto(442.50069498,1019.11424662)(442.49569498,1019.17424656)(442.48570557,1019.23425148)
\curveto(442.475695,1019.27424646)(442.47069501,1019.31424642)(442.47070557,1019.35425148)
\curveto(442.47069501,1019.40424633)(442.46569501,1019.44924629)(442.45570557,1019.48925148)
\curveto(442.41569506,1019.6392461)(442.37069511,1019.76924597)(442.32070557,1019.87925148)
\curveto(442.2806952,1019.99924574)(442.21569526,1020.10424563)(442.12570557,1020.19425148)
\curveto(441.98569549,1020.3342454)(441.81569566,1020.4342453)(441.61570557,1020.49425148)
\curveto(441.5756959,1020.50424523)(441.54069594,1020.50424523)(441.51070557,1020.49425148)
\curveto(441.480696,1020.49424524)(441.44569603,1020.50424523)(441.40570557,1020.52425148)
\curveto(441.36569611,1020.5342452)(441.31569616,1020.5392452)(441.25570557,1020.53925148)
\curveto(441.20569627,1020.54924519)(441.15569632,1020.54924519)(441.10570557,1020.53925148)
\curveto(441.04569643,1020.51924522)(440.98569649,1020.50924523)(440.92570557,1020.50925148)
\curveto(440.86569661,1020.50924523)(440.80569667,1020.49924524)(440.74570557,1020.47925148)
\curveto(440.45569702,1020.37924536)(440.24569723,1020.22924551)(440.11570557,1020.02925148)
\curveto(439.94569753,1019.79924594)(439.84069764,1019.50924623)(439.80070557,1019.15925148)
\curveto(439.77069771,1018.81924692)(439.75569772,1018.44424729)(439.75570557,1018.03425148)
\lineto(439.75570557,1016.05425148)
\lineto(439.75570557,1014.94425148)
\lineto(439.75570557,1014.64425148)
\curveto(439.75569772,1014.54425119)(439.73069775,1014.46425127)(439.68070557,1014.40425148)
\curveto(439.63069785,1014.3342514)(439.55569792,1014.28925145)(439.45570557,1014.26925148)
\curveto(439.36569811,1014.25925148)(439.26069822,1014.25425148)(439.14070557,1014.25425148)
\lineto(438.33070557,1014.25425148)
\lineto(438.06070557,1014.25425148)
\curveto(437.9806995,1014.26425147)(437.91069957,1014.27925146)(437.85070557,1014.29925148)
\curveto(437.75069973,1014.34925139)(437.69069979,1014.42925131)(437.67070557,1014.53925148)
\curveto(437.66069982,1014.64925109)(437.65569982,1014.77425096)(437.65570557,1014.91425148)
\lineto(437.65570557,1016.18925148)
\lineto(437.65570557,1018.54425148)
\curveto(437.65569982,1018.8342469)(437.64569983,1019.10924663)(437.62570557,1019.36925148)
\curveto(437.60569987,1019.62924611)(437.54069994,1019.84424589)(437.43070557,1020.01425148)
\curveto(437.35070013,1020.15424558)(437.24570023,1020.25924548)(437.11570557,1020.32925148)
\curveto(436.99570048,1020.39924534)(436.84570063,1020.45924528)(436.66570557,1020.50925148)
\curveto(436.62570085,1020.51924522)(436.58570089,1020.51924522)(436.54570557,1020.50925148)
\curveto(436.50570097,1020.50924523)(436.46070102,1020.51424522)(436.41070557,1020.52425148)
\curveto(436.30070118,1020.54424519)(436.19570128,1020.5342452)(436.09570557,1020.49425148)
\curveto(436.0757014,1020.49424524)(436.05570142,1020.48924525)(436.03570557,1020.47925148)
\lineto(435.97570557,1020.47925148)
\curveto(435.81570166,1020.42924531)(435.66070182,1020.34424539)(435.51070557,1020.22425148)
\curveto(435.35070213,1020.10424563)(435.22570225,1019.96424577)(435.13570557,1019.80425148)
\curveto(435.05570242,1019.65424608)(434.99570248,1019.47924626)(434.95570557,1019.27925148)
\curveto(434.92570255,1019.08924665)(434.90570257,1018.87924686)(434.89570557,1018.64925148)
\lineto(434.89570557,1017.89925148)
\lineto(434.89570557,1015.87425148)
\lineto(434.89570557,1014.95925148)
\lineto(434.89570557,1014.68925148)
\curveto(434.89570258,1014.59925114)(434.8807026,1014.51925122)(434.85070557,1014.44925148)
\curveto(434.81070267,1014.35925138)(434.73570274,1014.30425143)(434.62570557,1014.28425148)
\curveto(434.51570296,1014.26425147)(434.39070309,1014.25425148)(434.25070557,1014.25425148)
\lineto(433.47070557,1014.25425148)
\lineto(433.17070557,1014.25425148)
\curveto(433.0807044,1014.26425147)(433.00570447,1014.28925145)(432.94570557,1014.32925148)
\curveto(432.85570462,1014.37925136)(432.80570467,1014.46925127)(432.79570557,1014.59925148)
\lineto(432.79570557,1015.03425148)
\lineto(432.79570557,1016.78925148)
\lineto(432.79570557,1020.44925148)
\lineto(432.79570557,1021.34925148)
\lineto(432.79570557,1021.63425148)
\curveto(432.80570467,1021.72424401)(432.83070465,1021.79924394)(432.87070557,1021.85925148)
\curveto(432.92070456,1021.91924382)(433.00070448,1021.95924378)(433.11070557,1021.97925148)
\lineto(433.20070557,1021.97925148)
\curveto(433.25070423,1021.98924375)(433.30070418,1021.99424374)(433.35070557,1021.99425148)
\lineto(433.51570557,1021.99425148)
\lineto(434.13070557,1021.99425148)
\curveto(434.21070327,1021.99424374)(434.28570319,1021.98924375)(434.35570557,1021.97925148)
\curveto(434.43570304,1021.97924376)(434.50570297,1021.96924377)(434.56570557,1021.94925148)
\curveto(434.64570283,1021.91924382)(434.69570278,1021.86924387)(434.71570557,1021.79925148)
\curveto(434.74570273,1021.72924401)(434.77070271,1021.64924409)(434.79070557,1021.55925148)
\curveto(434.80070268,1021.52924421)(434.80070268,1021.49924424)(434.79070557,1021.46925148)
\curveto(434.79070269,1021.44924429)(434.80070268,1021.42924431)(434.82070557,1021.40925148)
\curveto(434.83070265,1021.37924436)(434.84070264,1021.35424438)(434.85070557,1021.33425148)
\curveto(434.87070261,1021.32424441)(434.89070259,1021.30924443)(434.91070557,1021.28925148)
\curveto(435.03070245,1021.27924446)(435.13070235,1021.31424442)(435.21070557,1021.39425148)
\curveto(435.29070219,1021.48424425)(435.36570211,1021.55424418)(435.43570557,1021.60425148)
\curveto(435.5757019,1021.70424403)(435.71570176,1021.79424394)(435.85570557,1021.87425148)
\curveto(436.00570147,1021.95424378)(436.16570131,1022.01924372)(436.33570557,1022.06925148)
\curveto(436.42570105,1022.09924364)(436.51570096,1022.11924362)(436.60570557,1022.12925148)
\curveto(436.69570078,1022.1392436)(436.79070069,1022.15424358)(436.89070557,1022.17425148)
\curveto(436.92070056,1022.18424355)(436.96570051,1022.18424355)(437.02570557,1022.17425148)
\curveto(437.08570039,1022.17424356)(437.12570035,1022.17924356)(437.14570557,1022.18925148)
}
}
{
\newrgbcolor{curcolor}{0 0 0}
\pscustom[linestyle=none,fillstyle=solid,fillcolor=curcolor]
{
\newpath
\moveto(453.64945557,1018.19925148)
\curveto(453.6694474,1018.11924762)(453.6694474,1018.02924771)(453.64945557,1017.92925148)
\curveto(453.62944744,1017.82924791)(453.59444748,1017.76424797)(453.54445557,1017.73425148)
\curveto(453.49444758,1017.69424804)(453.41944765,1017.66424807)(453.31945557,1017.64425148)
\curveto(453.22944784,1017.6342481)(453.12444795,1017.62424811)(453.00445557,1017.61425148)
\lineto(452.65945557,1017.61425148)
\curveto(452.54944852,1017.62424811)(452.44944862,1017.62924811)(452.35945557,1017.62925148)
\lineto(448.69945557,1017.62925148)
\lineto(448.48945557,1017.62925148)
\curveto(448.42945264,1017.62924811)(448.3744527,1017.61924812)(448.32445557,1017.59925148)
\curveto(448.24445283,1017.55924818)(448.19445288,1017.51924822)(448.17445557,1017.47925148)
\curveto(448.15445292,1017.45924828)(448.13445294,1017.41924832)(448.11445557,1017.35925148)
\curveto(448.09445298,1017.30924843)(448.08945298,1017.25924848)(448.09945557,1017.20925148)
\curveto(448.11945295,1017.14924859)(448.12945294,1017.08924865)(448.12945557,1017.02925148)
\curveto(448.13945293,1016.97924876)(448.15445292,1016.92424881)(448.17445557,1016.86425148)
\curveto(448.25445282,1016.62424911)(448.34945272,1016.42424931)(448.45945557,1016.26425148)
\curveto(448.57945249,1016.11424962)(448.73945233,1015.97924976)(448.93945557,1015.85925148)
\curveto(449.01945205,1015.80924993)(449.09945197,1015.77424996)(449.17945557,1015.75425148)
\curveto(449.2694518,1015.74424999)(449.35945171,1015.72425001)(449.44945557,1015.69425148)
\curveto(449.52945154,1015.67425006)(449.63945143,1015.65925008)(449.77945557,1015.64925148)
\curveto(449.91945115,1015.6392501)(450.03945103,1015.64425009)(450.13945557,1015.66425148)
\lineto(450.27445557,1015.66425148)
\curveto(450.3744507,1015.68425005)(450.46445061,1015.70425003)(450.54445557,1015.72425148)
\curveto(450.63445044,1015.75424998)(450.71945035,1015.78424995)(450.79945557,1015.81425148)
\curveto(450.89945017,1015.86424987)(451.00945006,1015.92924981)(451.12945557,1016.00925148)
\curveto(451.25944981,1016.08924965)(451.35444972,1016.16924957)(451.41445557,1016.24925148)
\curveto(451.46444961,1016.31924942)(451.51444956,1016.38424935)(451.56445557,1016.44425148)
\curveto(451.62444945,1016.51424922)(451.69444938,1016.56424917)(451.77445557,1016.59425148)
\curveto(451.8744492,1016.64424909)(451.99944907,1016.66424907)(452.14945557,1016.65425148)
\lineto(452.58445557,1016.65425148)
\lineto(452.76445557,1016.65425148)
\curveto(452.83444824,1016.66424907)(452.89444818,1016.65924908)(452.94445557,1016.63925148)
\lineto(453.09445557,1016.63925148)
\curveto(453.19444788,1016.61924912)(453.26444781,1016.59424914)(453.30445557,1016.56425148)
\curveto(453.34444773,1016.54424919)(453.36444771,1016.49924924)(453.36445557,1016.42925148)
\curveto(453.3744477,1016.35924938)(453.3694477,1016.29924944)(453.34945557,1016.24925148)
\curveto(453.29944777,1016.10924963)(453.24444783,1015.98424975)(453.18445557,1015.87425148)
\curveto(453.12444795,1015.76424997)(453.05444802,1015.65425008)(452.97445557,1015.54425148)
\curveto(452.75444832,1015.21425052)(452.50444857,1014.94925079)(452.22445557,1014.74925148)
\curveto(451.94444913,1014.54925119)(451.59444948,1014.37925136)(451.17445557,1014.23925148)
\curveto(451.06445001,1014.19925154)(450.95445012,1014.17425156)(450.84445557,1014.16425148)
\curveto(450.73445034,1014.15425158)(450.61945045,1014.1342516)(450.49945557,1014.10425148)
\curveto(450.45945061,1014.09425164)(450.41445066,1014.09425164)(450.36445557,1014.10425148)
\curveto(450.32445075,1014.10425163)(450.28445079,1014.09925164)(450.24445557,1014.08925148)
\lineto(450.07945557,1014.08925148)
\curveto(450.02945104,1014.06925167)(449.9694511,1014.06425167)(449.89945557,1014.07425148)
\curveto(449.83945123,1014.07425166)(449.78445129,1014.07925166)(449.73445557,1014.08925148)
\curveto(449.65445142,1014.09925164)(449.58445149,1014.09925164)(449.52445557,1014.08925148)
\curveto(449.46445161,1014.07925166)(449.39945167,1014.08425165)(449.32945557,1014.10425148)
\curveto(449.27945179,1014.12425161)(449.22445185,1014.1342516)(449.16445557,1014.13425148)
\curveto(449.10445197,1014.1342516)(449.04945202,1014.14425159)(448.99945557,1014.16425148)
\curveto(448.88945218,1014.18425155)(448.77945229,1014.20925153)(448.66945557,1014.23925148)
\curveto(448.55945251,1014.25925148)(448.45945261,1014.29425144)(448.36945557,1014.34425148)
\curveto(448.25945281,1014.38425135)(448.15445292,1014.41925132)(448.05445557,1014.44925148)
\curveto(447.96445311,1014.48925125)(447.87945319,1014.5342512)(447.79945557,1014.58425148)
\curveto(447.47945359,1014.78425095)(447.19445388,1015.01425072)(446.94445557,1015.27425148)
\curveto(446.69445438,1015.54425019)(446.48945458,1015.85424988)(446.32945557,1016.20425148)
\curveto(446.27945479,1016.31424942)(446.23945483,1016.42424931)(446.20945557,1016.53425148)
\curveto(446.17945489,1016.65424908)(446.13945493,1016.77424896)(446.08945557,1016.89425148)
\curveto(446.07945499,1016.9342488)(446.074455,1016.96924877)(446.07445557,1016.99925148)
\curveto(446.074455,1017.0392487)(446.069455,1017.07924866)(446.05945557,1017.11925148)
\curveto(446.01945505,1017.2392485)(445.99445508,1017.36924837)(445.98445557,1017.50925148)
\lineto(445.95445557,1017.92925148)
\curveto(445.95445512,1017.97924776)(445.94945512,1018.0342477)(445.93945557,1018.09425148)
\curveto(445.93945513,1018.15424758)(445.94445513,1018.20924753)(445.95445557,1018.25925148)
\lineto(445.95445557,1018.43925148)
\lineto(445.99945557,1018.79925148)
\curveto(446.03945503,1018.96924677)(446.074455,1019.1342466)(446.10445557,1019.29425148)
\curveto(446.13445494,1019.45424628)(446.17945489,1019.60424613)(446.23945557,1019.74425148)
\curveto(446.6694544,1020.78424495)(447.39945367,1021.51924422)(448.42945557,1021.94925148)
\curveto(448.5694525,1022.00924373)(448.70945236,1022.04924369)(448.84945557,1022.06925148)
\curveto(448.99945207,1022.09924364)(449.15445192,1022.1342436)(449.31445557,1022.17425148)
\curveto(449.39445168,1022.18424355)(449.4694516,1022.18924355)(449.53945557,1022.18925148)
\curveto(449.60945146,1022.18924355)(449.68445139,1022.19424354)(449.76445557,1022.20425148)
\curveto(450.2744508,1022.21424352)(450.70945036,1022.15424358)(451.06945557,1022.02425148)
\curveto(451.43944963,1021.90424383)(451.7694493,1021.74424399)(452.05945557,1021.54425148)
\curveto(452.14944892,1021.48424425)(452.23944883,1021.41424432)(452.32945557,1021.33425148)
\curveto(452.41944865,1021.26424447)(452.49944857,1021.18924455)(452.56945557,1021.10925148)
\curveto(452.59944847,1021.05924468)(452.63944843,1021.01924472)(452.68945557,1020.98925148)
\curveto(452.7694483,1020.87924486)(452.84444823,1020.76424497)(452.91445557,1020.64425148)
\curveto(452.98444809,1020.5342452)(453.05944801,1020.41924532)(453.13945557,1020.29925148)
\curveto(453.18944788,1020.20924553)(453.22944784,1020.11424562)(453.25945557,1020.01425148)
\curveto(453.29944777,1019.92424581)(453.33944773,1019.82424591)(453.37945557,1019.71425148)
\curveto(453.42944764,1019.58424615)(453.4694476,1019.44924629)(453.49945557,1019.30925148)
\curveto(453.52944754,1019.16924657)(453.56444751,1019.02924671)(453.60445557,1018.88925148)
\curveto(453.62444745,1018.80924693)(453.62944744,1018.71924702)(453.61945557,1018.61925148)
\curveto(453.61944745,1018.52924721)(453.62944744,1018.44424729)(453.64945557,1018.36425148)
\lineto(453.64945557,1018.19925148)
\moveto(451.39945557,1019.08425148)
\curveto(451.4694496,1019.18424655)(451.4744496,1019.30424643)(451.41445557,1019.44425148)
\curveto(451.36444971,1019.59424614)(451.32444975,1019.70424603)(451.29445557,1019.77425148)
\curveto(451.15444992,1020.04424569)(450.9694501,1020.24924549)(450.73945557,1020.38925148)
\curveto(450.50945056,1020.5392452)(450.18945088,1020.61924512)(449.77945557,1020.62925148)
\curveto(449.74945132,1020.60924513)(449.71445136,1020.60424513)(449.67445557,1020.61425148)
\curveto(449.63445144,1020.62424511)(449.59945147,1020.62424511)(449.56945557,1020.61425148)
\curveto(449.51945155,1020.59424514)(449.46445161,1020.57924516)(449.40445557,1020.56925148)
\curveto(449.34445173,1020.56924517)(449.28945178,1020.55924518)(449.23945557,1020.53925148)
\curveto(448.79945227,1020.39924534)(448.4744526,1020.12424561)(448.26445557,1019.71425148)
\curveto(448.24445283,1019.67424606)(448.21945285,1019.61924612)(448.18945557,1019.54925148)
\curveto(448.1694529,1019.48924625)(448.15445292,1019.42424631)(448.14445557,1019.35425148)
\curveto(448.13445294,1019.29424644)(448.13445294,1019.2342465)(448.14445557,1019.17425148)
\curveto(448.16445291,1019.11424662)(448.19945287,1019.06424667)(448.24945557,1019.02425148)
\curveto(448.32945274,1018.97424676)(448.43945263,1018.94924679)(448.57945557,1018.94925148)
\lineto(448.98445557,1018.94925148)
\lineto(450.64945557,1018.94925148)
\lineto(451.08445557,1018.94925148)
\curveto(451.24444983,1018.95924678)(451.34944972,1019.00424673)(451.39945557,1019.08425148)
}
}
{
\newrgbcolor{curcolor}{0 0 0}
\pscustom[linestyle=none,fillstyle=solid,fillcolor=curcolor]
{
\newpath
\moveto(459.32273682,1022.18925148)
\curveto(459.92273101,1022.20924353)(460.42273051,1022.12424361)(460.82273682,1021.93425148)
\curveto(461.22272971,1021.74424399)(461.5377294,1021.46424427)(461.76773682,1021.09425148)
\curveto(461.8377291,1020.98424475)(461.89272904,1020.86424487)(461.93273682,1020.73425148)
\curveto(461.97272896,1020.61424512)(462.01272892,1020.48924525)(462.05273682,1020.35925148)
\curveto(462.07272886,1020.27924546)(462.08272885,1020.20424553)(462.08273682,1020.13425148)
\curveto(462.09272884,1020.06424567)(462.10772883,1019.99424574)(462.12773682,1019.92425148)
\curveto(462.12772881,1019.86424587)(462.1327288,1019.82424591)(462.14273682,1019.80425148)
\curveto(462.16272877,1019.66424607)(462.17272876,1019.51924622)(462.17273682,1019.36925148)
\lineto(462.17273682,1018.93425148)
\lineto(462.17273682,1017.59925148)
\lineto(462.17273682,1015.16925148)
\curveto(462.17272876,1014.97925076)(462.16772877,1014.79425094)(462.15773682,1014.61425148)
\curveto(462.15772878,1014.44425129)(462.08772885,1014.3342514)(461.94773682,1014.28425148)
\curveto(461.88772905,1014.26425147)(461.81772912,1014.25425148)(461.73773682,1014.25425148)
\lineto(461.49773682,1014.25425148)
\lineto(460.68773682,1014.25425148)
\curveto(460.56773037,1014.25425148)(460.45773048,1014.25925148)(460.35773682,1014.26925148)
\curveto(460.26773067,1014.28925145)(460.19773074,1014.3342514)(460.14773682,1014.40425148)
\curveto(460.10773083,1014.46425127)(460.08273085,1014.5392512)(460.07273682,1014.62925148)
\lineto(460.07273682,1014.94425148)
\lineto(460.07273682,1015.99425148)
\lineto(460.07273682,1018.22925148)
\curveto(460.07273086,1018.59924714)(460.05773088,1018.9392468)(460.02773682,1019.24925148)
\curveto(459.99773094,1019.56924617)(459.90773103,1019.8392459)(459.75773682,1020.05925148)
\curveto(459.61773132,1020.25924548)(459.41273152,1020.39924534)(459.14273682,1020.47925148)
\curveto(459.09273184,1020.49924524)(459.0377319,1020.50924523)(458.97773682,1020.50925148)
\curveto(458.92773201,1020.50924523)(458.87273206,1020.51924522)(458.81273682,1020.53925148)
\curveto(458.76273217,1020.54924519)(458.69773224,1020.54924519)(458.61773682,1020.53925148)
\curveto(458.54773239,1020.5392452)(458.49273244,1020.5342452)(458.45273682,1020.52425148)
\curveto(458.41273252,1020.51424522)(458.37773256,1020.50924523)(458.34773682,1020.50925148)
\curveto(458.31773262,1020.50924523)(458.28773265,1020.50424523)(458.25773682,1020.49425148)
\curveto(458.02773291,1020.4342453)(457.84273309,1020.35424538)(457.70273682,1020.25425148)
\curveto(457.38273355,1020.02424571)(457.19273374,1019.68924605)(457.13273682,1019.24925148)
\curveto(457.07273386,1018.80924693)(457.04273389,1018.31424742)(457.04273682,1017.76425148)
\lineto(457.04273682,1015.88925148)
\lineto(457.04273682,1014.97425148)
\lineto(457.04273682,1014.70425148)
\curveto(457.04273389,1014.61425112)(457.02773391,1014.5392512)(456.99773682,1014.47925148)
\curveto(456.94773399,1014.36925137)(456.86773407,1014.30425143)(456.75773682,1014.28425148)
\curveto(456.64773429,1014.26425147)(456.51273442,1014.25425148)(456.35273682,1014.25425148)
\lineto(455.60273682,1014.25425148)
\curveto(455.49273544,1014.25425148)(455.38273555,1014.25925148)(455.27273682,1014.26925148)
\curveto(455.16273577,1014.27925146)(455.08273585,1014.31425142)(455.03273682,1014.37425148)
\curveto(454.96273597,1014.46425127)(454.92773601,1014.59425114)(454.92773682,1014.76425148)
\curveto(454.937736,1014.9342508)(454.94273599,1015.09425064)(454.94273682,1015.24425148)
\lineto(454.94273682,1017.28425148)
\lineto(454.94273682,1020.58425148)
\lineto(454.94273682,1021.34925148)
\lineto(454.94273682,1021.64925148)
\curveto(454.95273598,1021.739244)(454.98273595,1021.81424392)(455.03273682,1021.87425148)
\curveto(455.05273588,1021.90424383)(455.08273585,1021.92424381)(455.12273682,1021.93425148)
\curveto(455.17273576,1021.95424378)(455.22273571,1021.96924377)(455.27273682,1021.97925148)
\lineto(455.34773682,1021.97925148)
\curveto(455.39773554,1021.98924375)(455.44773549,1021.99424374)(455.49773682,1021.99425148)
\lineto(455.66273682,1021.99425148)
\lineto(456.29273682,1021.99425148)
\curveto(456.37273456,1021.99424374)(456.44773449,1021.98924375)(456.51773682,1021.97925148)
\curveto(456.59773434,1021.97924376)(456.66773427,1021.96924377)(456.72773682,1021.94925148)
\curveto(456.79773414,1021.91924382)(456.84273409,1021.87424386)(456.86273682,1021.81425148)
\curveto(456.89273404,1021.75424398)(456.91773402,1021.68424405)(456.93773682,1021.60425148)
\curveto(456.94773399,1021.56424417)(456.94773399,1021.52924421)(456.93773682,1021.49925148)
\curveto(456.937734,1021.46924427)(456.94773399,1021.4392443)(456.96773682,1021.40925148)
\curveto(456.98773395,1021.35924438)(457.00273393,1021.32924441)(457.01273682,1021.31925148)
\curveto(457.0327339,1021.30924443)(457.05773388,1021.29424444)(457.08773682,1021.27425148)
\curveto(457.19773374,1021.26424447)(457.28773365,1021.29924444)(457.35773682,1021.37925148)
\curveto(457.42773351,1021.46924427)(457.50273343,1021.5392442)(457.58273682,1021.58925148)
\curveto(457.85273308,1021.78924395)(458.15273278,1021.94924379)(458.48273682,1022.06925148)
\curveto(458.57273236,1022.09924364)(458.66273227,1022.11924362)(458.75273682,1022.12925148)
\curveto(458.85273208,1022.1392436)(458.95773198,1022.15424358)(459.06773682,1022.17425148)
\curveto(459.09773184,1022.18424355)(459.14273179,1022.18424355)(459.20273682,1022.17425148)
\curveto(459.26273167,1022.17424356)(459.30273163,1022.17924356)(459.32273682,1022.18925148)
}
}
{
\newrgbcolor{curcolor}{0 0 0}
\pscustom[linestyle=none,fillstyle=solid,fillcolor=curcolor]
{
\newpath
\moveto(471.57398682,1018.43925148)
\curveto(471.59397825,1018.37924736)(471.60397824,1018.29424744)(471.60398682,1018.18425148)
\curveto(471.60397824,1018.07424766)(471.59397825,1017.98924775)(471.57398682,1017.92925148)
\lineto(471.57398682,1017.77925148)
\curveto(471.55397829,1017.69924804)(471.5439783,1017.61924812)(471.54398682,1017.53925148)
\curveto(471.55397829,1017.45924828)(471.54897829,1017.37924836)(471.52898682,1017.29925148)
\curveto(471.50897833,1017.22924851)(471.49397835,1017.16424857)(471.48398682,1017.10425148)
\curveto(471.47397837,1017.04424869)(471.46397838,1016.97924876)(471.45398682,1016.90925148)
\curveto(471.41397843,1016.79924894)(471.37897846,1016.68424905)(471.34898682,1016.56425148)
\curveto(471.31897852,1016.45424928)(471.27897856,1016.34924939)(471.22898682,1016.24925148)
\curveto(471.01897882,1015.76924997)(470.7439791,1015.37925036)(470.40398682,1015.07925148)
\curveto(470.06397978,1014.77925096)(469.65398019,1014.52925121)(469.17398682,1014.32925148)
\curveto(469.05398079,1014.27925146)(468.92898091,1014.24425149)(468.79898682,1014.22425148)
\curveto(468.67898116,1014.19425154)(468.55398129,1014.16425157)(468.42398682,1014.13425148)
\curveto(468.37398147,1014.11425162)(468.31898152,1014.10425163)(468.25898682,1014.10425148)
\curveto(468.19898164,1014.10425163)(468.1439817,1014.09925164)(468.09398682,1014.08925148)
\lineto(467.98898682,1014.08925148)
\curveto(467.95898188,1014.07925166)(467.92898191,1014.07425166)(467.89898682,1014.07425148)
\curveto(467.84898199,1014.06425167)(467.76898207,1014.05925168)(467.65898682,1014.05925148)
\curveto(467.54898229,1014.04925169)(467.46398238,1014.05425168)(467.40398682,1014.07425148)
\lineto(467.25398682,1014.07425148)
\curveto(467.20398264,1014.08425165)(467.14898269,1014.08925165)(467.08898682,1014.08925148)
\curveto(467.0389828,1014.07925166)(466.98898285,1014.08425165)(466.93898682,1014.10425148)
\curveto(466.89898294,1014.11425162)(466.85898298,1014.11925162)(466.81898682,1014.11925148)
\curveto(466.78898305,1014.11925162)(466.74898309,1014.12425161)(466.69898682,1014.13425148)
\curveto(466.59898324,1014.16425157)(466.49898334,1014.18925155)(466.39898682,1014.20925148)
\curveto(466.29898354,1014.22925151)(466.20398364,1014.25925148)(466.11398682,1014.29925148)
\curveto(465.99398385,1014.3392514)(465.87898396,1014.37925136)(465.76898682,1014.41925148)
\curveto(465.66898417,1014.45925128)(465.56398428,1014.50925123)(465.45398682,1014.56925148)
\curveto(465.10398474,1014.77925096)(464.80398504,1015.02425071)(464.55398682,1015.30425148)
\curveto(464.30398554,1015.58425015)(464.09398575,1015.91924982)(463.92398682,1016.30925148)
\curveto(463.87398597,1016.39924934)(463.83398601,1016.49424924)(463.80398682,1016.59425148)
\curveto(463.78398606,1016.69424904)(463.75898608,1016.79924894)(463.72898682,1016.90925148)
\curveto(463.70898613,1016.95924878)(463.69898614,1017.00424873)(463.69898682,1017.04425148)
\curveto(463.69898614,1017.08424865)(463.68898615,1017.12924861)(463.66898682,1017.17925148)
\curveto(463.64898619,1017.25924848)(463.6389862,1017.3392484)(463.63898682,1017.41925148)
\curveto(463.6389862,1017.50924823)(463.62898621,1017.59424814)(463.60898682,1017.67425148)
\curveto(463.59898624,1017.72424801)(463.59398625,1017.76924797)(463.59398682,1017.80925148)
\lineto(463.59398682,1017.94425148)
\curveto(463.57398627,1018.00424773)(463.56398628,1018.08924765)(463.56398682,1018.19925148)
\curveto(463.57398627,1018.30924743)(463.58898625,1018.39424734)(463.60898682,1018.45425148)
\lineto(463.60898682,1018.55925148)
\curveto(463.61898622,1018.60924713)(463.61898622,1018.65924708)(463.60898682,1018.70925148)
\curveto(463.60898623,1018.76924697)(463.61898622,1018.82424691)(463.63898682,1018.87425148)
\curveto(463.64898619,1018.92424681)(463.65398619,1018.96924677)(463.65398682,1019.00925148)
\curveto(463.65398619,1019.05924668)(463.66398618,1019.10924663)(463.68398682,1019.15925148)
\curveto(463.72398612,1019.28924645)(463.75898608,1019.41424632)(463.78898682,1019.53425148)
\curveto(463.81898602,1019.66424607)(463.85898598,1019.78924595)(463.90898682,1019.90925148)
\curveto(464.08898575,1020.31924542)(464.30398554,1020.65924508)(464.55398682,1020.92925148)
\curveto(464.80398504,1021.20924453)(465.10898473,1021.46424427)(465.46898682,1021.69425148)
\curveto(465.56898427,1021.74424399)(465.67398417,1021.78924395)(465.78398682,1021.82925148)
\curveto(465.89398395,1021.86924387)(466.00398384,1021.91424382)(466.11398682,1021.96425148)
\curveto(466.2439836,1022.01424372)(466.37898346,1022.04924369)(466.51898682,1022.06925148)
\curveto(466.65898318,1022.08924365)(466.80398304,1022.11924362)(466.95398682,1022.15925148)
\curveto(467.03398281,1022.16924357)(467.10898273,1022.17424356)(467.17898682,1022.17425148)
\curveto(467.24898259,1022.17424356)(467.31898252,1022.17924356)(467.38898682,1022.18925148)
\curveto(467.96898187,1022.19924354)(468.46898137,1022.1392436)(468.88898682,1022.00925148)
\curveto(469.31898052,1021.87924386)(469.69898014,1021.69924404)(470.02898682,1021.46925148)
\curveto(470.1389797,1021.38924435)(470.24897959,1021.29924444)(470.35898682,1021.19925148)
\curveto(470.47897936,1021.10924463)(470.57897926,1021.00924473)(470.65898682,1020.89925148)
\curveto(470.7389791,1020.79924494)(470.80897903,1020.69924504)(470.86898682,1020.59925148)
\curveto(470.9389789,1020.49924524)(471.00897883,1020.39424534)(471.07898682,1020.28425148)
\curveto(471.14897869,1020.17424556)(471.20397864,1020.05424568)(471.24398682,1019.92425148)
\curveto(471.28397856,1019.80424593)(471.32897851,1019.67424606)(471.37898682,1019.53425148)
\curveto(471.40897843,1019.45424628)(471.43397841,1019.36924637)(471.45398682,1019.27925148)
\lineto(471.51398682,1019.00925148)
\curveto(471.52397832,1018.96924677)(471.52897831,1018.92924681)(471.52898682,1018.88925148)
\curveto(471.52897831,1018.84924689)(471.53397831,1018.80924693)(471.54398682,1018.76925148)
\curveto(471.56397828,1018.71924702)(471.56897827,1018.66424707)(471.55898682,1018.60425148)
\curveto(471.54897829,1018.54424719)(471.55397829,1018.48924725)(471.57398682,1018.43925148)
\moveto(469.47398682,1017.89925148)
\curveto(469.48398036,1017.94924779)(469.48898035,1018.01924772)(469.48898682,1018.10925148)
\curveto(469.48898035,1018.20924753)(469.48398036,1018.28424745)(469.47398682,1018.33425148)
\lineto(469.47398682,1018.45425148)
\curveto(469.45398039,1018.50424723)(469.4439804,1018.55924718)(469.44398682,1018.61925148)
\curveto(469.4439804,1018.67924706)(469.4389804,1018.734247)(469.42898682,1018.78425148)
\curveto(469.42898041,1018.82424691)(469.42398042,1018.85424688)(469.41398682,1018.87425148)
\lineto(469.35398682,1019.11425148)
\curveto(469.3439805,1019.20424653)(469.32398052,1019.28924645)(469.29398682,1019.36925148)
\curveto(469.18398066,1019.62924611)(469.05398079,1019.84924589)(468.90398682,1020.02925148)
\curveto(468.75398109,1020.21924552)(468.55398129,1020.36924537)(468.30398682,1020.47925148)
\curveto(468.2439816,1020.49924524)(468.18398166,1020.51424522)(468.12398682,1020.52425148)
\curveto(468.06398178,1020.54424519)(467.99898184,1020.56424517)(467.92898682,1020.58425148)
\curveto(467.84898199,1020.60424513)(467.76398208,1020.60924513)(467.67398682,1020.59925148)
\lineto(467.40398682,1020.59925148)
\curveto(467.37398247,1020.57924516)(467.3389825,1020.56924517)(467.29898682,1020.56925148)
\curveto(467.25898258,1020.57924516)(467.22398262,1020.57924516)(467.19398682,1020.56925148)
\lineto(466.98398682,1020.50925148)
\curveto(466.92398292,1020.49924524)(466.86898297,1020.47924526)(466.81898682,1020.44925148)
\curveto(466.56898327,1020.3392454)(466.36398348,1020.17924556)(466.20398682,1019.96925148)
\curveto(466.05398379,1019.76924597)(465.93398391,1019.5342462)(465.84398682,1019.26425148)
\curveto(465.81398403,1019.16424657)(465.78898405,1019.05924668)(465.76898682,1018.94925148)
\curveto(465.75898408,1018.8392469)(465.7439841,1018.72924701)(465.72398682,1018.61925148)
\curveto(465.71398413,1018.56924717)(465.70898413,1018.51924722)(465.70898682,1018.46925148)
\lineto(465.70898682,1018.31925148)
\curveto(465.68898415,1018.24924749)(465.67898416,1018.14424759)(465.67898682,1018.00425148)
\curveto(465.68898415,1017.86424787)(465.70398414,1017.75924798)(465.72398682,1017.68925148)
\lineto(465.72398682,1017.55425148)
\curveto(465.7439841,1017.47424826)(465.75898408,1017.39424834)(465.76898682,1017.31425148)
\curveto(465.77898406,1017.24424849)(465.79398405,1017.16924857)(465.81398682,1017.08925148)
\curveto(465.91398393,1016.78924895)(466.01898382,1016.54424919)(466.12898682,1016.35425148)
\curveto(466.24898359,1016.17424956)(466.43398341,1016.00924973)(466.68398682,1015.85925148)
\curveto(466.75398309,1015.80924993)(466.82898301,1015.76924997)(466.90898682,1015.73925148)
\curveto(466.99898284,1015.70925003)(467.08898275,1015.68425005)(467.17898682,1015.66425148)
\curveto(467.21898262,1015.65425008)(467.25398259,1015.64925009)(467.28398682,1015.64925148)
\curveto(467.31398253,1015.65925008)(467.34898249,1015.65925008)(467.38898682,1015.64925148)
\lineto(467.50898682,1015.61925148)
\curveto(467.55898228,1015.61925012)(467.60398224,1015.62425011)(467.64398682,1015.63425148)
\lineto(467.76398682,1015.63425148)
\curveto(467.843982,1015.65425008)(467.92398192,1015.66925007)(468.00398682,1015.67925148)
\curveto(468.08398176,1015.68925005)(468.15898168,1015.70925003)(468.22898682,1015.73925148)
\curveto(468.48898135,1015.8392499)(468.69898114,1015.97424976)(468.85898682,1016.14425148)
\curveto(469.01898082,1016.31424942)(469.15398069,1016.52424921)(469.26398682,1016.77425148)
\curveto(469.30398054,1016.87424886)(469.33398051,1016.97424876)(469.35398682,1017.07425148)
\curveto(469.37398047,1017.17424856)(469.39898044,1017.27924846)(469.42898682,1017.38925148)
\curveto(469.4389804,1017.42924831)(469.4439804,1017.46424827)(469.44398682,1017.49425148)
\curveto(469.4439804,1017.5342482)(469.44898039,1017.57424816)(469.45898682,1017.61425148)
\lineto(469.45898682,1017.74925148)
\curveto(469.45898038,1017.79924794)(469.46398038,1017.84924789)(469.47398682,1017.89925148)
}
}
{
\newrgbcolor{curcolor}{0 0 0}
\pscustom[linestyle=none,fillstyle=solid,fillcolor=curcolor]
{
\newpath
\moveto(475.94390869,1022.20425148)
\curveto(476.69390419,1022.22424351)(477.34390354,1022.1392436)(477.89390869,1021.94925148)
\curveto(478.45390243,1021.76924397)(478.87890201,1021.45424428)(479.16890869,1021.00425148)
\curveto(479.23890165,1020.89424484)(479.29890159,1020.77924496)(479.34890869,1020.65925148)
\curveto(479.40890148,1020.54924519)(479.45890143,1020.42424531)(479.49890869,1020.28425148)
\curveto(479.51890137,1020.22424551)(479.52890136,1020.15924558)(479.52890869,1020.08925148)
\curveto(479.52890136,1020.01924572)(479.51890137,1019.95924578)(479.49890869,1019.90925148)
\curveto(479.45890143,1019.84924589)(479.40390148,1019.80924593)(479.33390869,1019.78925148)
\curveto(479.2839016,1019.76924597)(479.22390166,1019.75924598)(479.15390869,1019.75925148)
\lineto(478.94390869,1019.75925148)
\lineto(478.28390869,1019.75925148)
\curveto(478.21390267,1019.75924598)(478.14390274,1019.75424598)(478.07390869,1019.74425148)
\curveto(478.00390288,1019.74424599)(477.93890295,1019.75424598)(477.87890869,1019.77425148)
\curveto(477.77890311,1019.79424594)(477.70390318,1019.8342459)(477.65390869,1019.89425148)
\curveto(477.60390328,1019.95424578)(477.55890333,1020.01424572)(477.51890869,1020.07425148)
\lineto(477.39890869,1020.28425148)
\curveto(477.36890352,1020.36424537)(477.31890357,1020.42924531)(477.24890869,1020.47925148)
\curveto(477.14890374,1020.55924518)(477.04890384,1020.61924512)(476.94890869,1020.65925148)
\curveto(476.85890403,1020.69924504)(476.74390414,1020.734245)(476.60390869,1020.76425148)
\curveto(476.53390435,1020.78424495)(476.42890446,1020.79924494)(476.28890869,1020.80925148)
\curveto(476.15890473,1020.81924492)(476.05890483,1020.81424492)(475.98890869,1020.79425148)
\lineto(475.88390869,1020.79425148)
\lineto(475.73390869,1020.76425148)
\curveto(475.69390519,1020.76424497)(475.64890524,1020.75924498)(475.59890869,1020.74925148)
\curveto(475.42890546,1020.69924504)(475.2889056,1020.62924511)(475.17890869,1020.53925148)
\curveto(475.07890581,1020.45924528)(475.00890588,1020.3342454)(474.96890869,1020.16425148)
\curveto(474.94890594,1020.09424564)(474.94890594,1020.02924571)(474.96890869,1019.96925148)
\curveto(474.9889059,1019.90924583)(475.00890588,1019.85924588)(475.02890869,1019.81925148)
\curveto(475.09890579,1019.69924604)(475.17890571,1019.60424613)(475.26890869,1019.53425148)
\curveto(475.36890552,1019.46424627)(475.4839054,1019.40424633)(475.61390869,1019.35425148)
\curveto(475.80390508,1019.27424646)(476.00890488,1019.20424653)(476.22890869,1019.14425148)
\lineto(476.91890869,1018.99425148)
\curveto(477.15890373,1018.95424678)(477.3889035,1018.90424683)(477.60890869,1018.84425148)
\curveto(477.83890305,1018.79424694)(478.05390283,1018.72924701)(478.25390869,1018.64925148)
\curveto(478.34390254,1018.60924713)(478.42890246,1018.57424716)(478.50890869,1018.54425148)
\curveto(478.59890229,1018.52424721)(478.6839022,1018.48924725)(478.76390869,1018.43925148)
\curveto(478.95390193,1018.31924742)(479.12390176,1018.18924755)(479.27390869,1018.04925148)
\curveto(479.43390145,1017.90924783)(479.55890133,1017.734248)(479.64890869,1017.52425148)
\curveto(479.67890121,1017.45424828)(479.70390118,1017.38424835)(479.72390869,1017.31425148)
\curveto(479.74390114,1017.24424849)(479.76390112,1017.16924857)(479.78390869,1017.08925148)
\curveto(479.79390109,1017.02924871)(479.79890109,1016.9342488)(479.79890869,1016.80425148)
\curveto(479.80890108,1016.68424905)(479.80890108,1016.58924915)(479.79890869,1016.51925148)
\lineto(479.79890869,1016.44425148)
\curveto(479.77890111,1016.38424935)(479.76390112,1016.32424941)(479.75390869,1016.26425148)
\curveto(479.75390113,1016.21424952)(479.74890114,1016.16424957)(479.73890869,1016.11425148)
\curveto(479.66890122,1015.81424992)(479.55890133,1015.54925019)(479.40890869,1015.31925148)
\curveto(479.24890164,1015.07925066)(479.05390183,1014.88425085)(478.82390869,1014.73425148)
\curveto(478.59390229,1014.58425115)(478.33390255,1014.45425128)(478.04390869,1014.34425148)
\curveto(477.93390295,1014.29425144)(477.81390307,1014.25925148)(477.68390869,1014.23925148)
\curveto(477.56390332,1014.21925152)(477.44390344,1014.19425154)(477.32390869,1014.16425148)
\curveto(477.23390365,1014.14425159)(477.13890375,1014.1342516)(477.03890869,1014.13425148)
\curveto(476.94890394,1014.12425161)(476.85890403,1014.10925163)(476.76890869,1014.08925148)
\lineto(476.49890869,1014.08925148)
\curveto(476.43890445,1014.06925167)(476.33390455,1014.05925168)(476.18390869,1014.05925148)
\curveto(476.04390484,1014.05925168)(475.94390494,1014.06925167)(475.88390869,1014.08925148)
\curveto(475.85390503,1014.08925165)(475.81890507,1014.09425164)(475.77890869,1014.10425148)
\lineto(475.67390869,1014.10425148)
\curveto(475.55390533,1014.12425161)(475.43390545,1014.1392516)(475.31390869,1014.14925148)
\curveto(475.19390569,1014.15925158)(475.07890581,1014.17925156)(474.96890869,1014.20925148)
\curveto(474.57890631,1014.31925142)(474.23390665,1014.44425129)(473.93390869,1014.58425148)
\curveto(473.63390725,1014.734251)(473.37890751,1014.95425078)(473.16890869,1015.24425148)
\curveto(473.02890786,1015.4342503)(472.90890798,1015.65425008)(472.80890869,1015.90425148)
\curveto(472.7889081,1015.96424977)(472.76890812,1016.04424969)(472.74890869,1016.14425148)
\curveto(472.72890816,1016.19424954)(472.71390817,1016.26424947)(472.70390869,1016.35425148)
\curveto(472.69390819,1016.44424929)(472.69890819,1016.51924922)(472.71890869,1016.57925148)
\curveto(472.74890814,1016.64924909)(472.79890809,1016.69924904)(472.86890869,1016.72925148)
\curveto(472.91890797,1016.74924899)(472.97890791,1016.75924898)(473.04890869,1016.75925148)
\lineto(473.27390869,1016.75925148)
\lineto(473.97890869,1016.75925148)
\lineto(474.21890869,1016.75925148)
\curveto(474.29890659,1016.75924898)(474.36890652,1016.74924899)(474.42890869,1016.72925148)
\curveto(474.53890635,1016.68924905)(474.60890628,1016.62424911)(474.63890869,1016.53425148)
\curveto(474.67890621,1016.44424929)(474.72390616,1016.34924939)(474.77390869,1016.24925148)
\curveto(474.79390609,1016.19924954)(474.82890606,1016.1342496)(474.87890869,1016.05425148)
\curveto(474.93890595,1015.97424976)(474.9889059,1015.92424981)(475.02890869,1015.90425148)
\curveto(475.14890574,1015.80424993)(475.26390562,1015.72425001)(475.37390869,1015.66425148)
\curveto(475.4839054,1015.61425012)(475.62390526,1015.56425017)(475.79390869,1015.51425148)
\curveto(475.84390504,1015.49425024)(475.89390499,1015.48425025)(475.94390869,1015.48425148)
\curveto(475.99390489,1015.49425024)(476.04390484,1015.49425024)(476.09390869,1015.48425148)
\curveto(476.17390471,1015.46425027)(476.25890463,1015.45425028)(476.34890869,1015.45425148)
\curveto(476.44890444,1015.46425027)(476.53390435,1015.47925026)(476.60390869,1015.49925148)
\curveto(476.65390423,1015.50925023)(476.69890419,1015.51425022)(476.73890869,1015.51425148)
\curveto(476.7889041,1015.51425022)(476.83890405,1015.52425021)(476.88890869,1015.54425148)
\curveto(477.02890386,1015.59425014)(477.15390373,1015.65425008)(477.26390869,1015.72425148)
\curveto(477.3839035,1015.79424994)(477.47890341,1015.88424985)(477.54890869,1015.99425148)
\curveto(477.59890329,1016.07424966)(477.63890325,1016.19924954)(477.66890869,1016.36925148)
\curveto(477.6889032,1016.4392493)(477.6889032,1016.50424923)(477.66890869,1016.56425148)
\curveto(477.64890324,1016.62424911)(477.62890326,1016.67424906)(477.60890869,1016.71425148)
\curveto(477.53890335,1016.85424888)(477.44890344,1016.95924878)(477.33890869,1017.02925148)
\curveto(477.23890365,1017.09924864)(477.11890377,1017.16424857)(476.97890869,1017.22425148)
\curveto(476.7889041,1017.30424843)(476.5889043,1017.36924837)(476.37890869,1017.41925148)
\curveto(476.16890472,1017.46924827)(475.95890493,1017.52424821)(475.74890869,1017.58425148)
\curveto(475.66890522,1017.60424813)(475.5839053,1017.61924812)(475.49390869,1017.62925148)
\curveto(475.41390547,1017.6392481)(475.33390555,1017.65424808)(475.25390869,1017.67425148)
\curveto(474.93390595,1017.76424797)(474.62890626,1017.84924789)(474.33890869,1017.92925148)
\curveto(474.04890684,1018.01924772)(473.7839071,1018.14924759)(473.54390869,1018.31925148)
\curveto(473.26390762,1018.51924722)(473.05890783,1018.78924695)(472.92890869,1019.12925148)
\curveto(472.90890798,1019.19924654)(472.888908,1019.29424644)(472.86890869,1019.41425148)
\curveto(472.84890804,1019.48424625)(472.83390805,1019.56924617)(472.82390869,1019.66925148)
\curveto(472.81390807,1019.76924597)(472.81890807,1019.85924588)(472.83890869,1019.93925148)
\curveto(472.85890803,1019.98924575)(472.86390802,1020.02924571)(472.85390869,1020.05925148)
\curveto(472.84390804,1020.09924564)(472.84890804,1020.14424559)(472.86890869,1020.19425148)
\curveto(472.888908,1020.30424543)(472.90890798,1020.40424533)(472.92890869,1020.49425148)
\curveto(472.95890793,1020.59424514)(472.99390789,1020.68924505)(473.03390869,1020.77925148)
\curveto(473.16390772,1021.06924467)(473.34390754,1021.30424443)(473.57390869,1021.48425148)
\curveto(473.80390708,1021.66424407)(474.06390682,1021.80924393)(474.35390869,1021.91925148)
\curveto(474.46390642,1021.96924377)(474.57890631,1022.00424373)(474.69890869,1022.02425148)
\curveto(474.81890607,1022.05424368)(474.94390594,1022.08424365)(475.07390869,1022.11425148)
\curveto(475.13390575,1022.1342436)(475.19390569,1022.14424359)(475.25390869,1022.14425148)
\lineto(475.43390869,1022.17425148)
\curveto(475.51390537,1022.18424355)(475.59890529,1022.18924355)(475.68890869,1022.18925148)
\curveto(475.77890511,1022.18924355)(475.86390502,1022.19424354)(475.94390869,1022.20425148)
}
}
{
\newrgbcolor{curcolor}{0 0 0}
\pscustom[linestyle=none,fillstyle=solid,fillcolor=curcolor]
{
}
}
{
\newrgbcolor{curcolor}{0 0 0}
\pscustom[linestyle=none,fillstyle=solid,fillcolor=curcolor]
{
\newpath
\moveto(485.61070557,1021.97925148)
\lineto(486.73570557,1021.97925148)
\curveto(486.84570313,1021.97924376)(486.94570303,1021.97424376)(487.03570557,1021.96425148)
\curveto(487.12570285,1021.95424378)(487.19070279,1021.91924382)(487.23070557,1021.85925148)
\curveto(487.2807027,1021.79924394)(487.31070267,1021.71424402)(487.32070557,1021.60425148)
\curveto(487.33070265,1021.50424423)(487.33570264,1021.39924434)(487.33570557,1021.28925148)
\lineto(487.33570557,1020.23925148)
\lineto(487.33570557,1018.00425148)
\curveto(487.33570264,1017.64424809)(487.35070263,1017.30424843)(487.38070557,1016.98425148)
\curveto(487.41070257,1016.66424907)(487.50070248,1016.39924934)(487.65070557,1016.18925148)
\curveto(487.79070219,1015.97924976)(488.01570196,1015.82924991)(488.32570557,1015.73925148)
\curveto(488.3757016,1015.72925001)(488.41570156,1015.72425001)(488.44570557,1015.72425148)
\curveto(488.48570149,1015.72425001)(488.53070145,1015.71925002)(488.58070557,1015.70925148)
\curveto(488.63070135,1015.69925004)(488.68570129,1015.69425004)(488.74570557,1015.69425148)
\curveto(488.80570117,1015.69425004)(488.85070113,1015.69925004)(488.88070557,1015.70925148)
\curveto(488.93070105,1015.72925001)(488.97070101,1015.73425)(489.00070557,1015.72425148)
\curveto(489.04070094,1015.71425002)(489.0807009,1015.71925002)(489.12070557,1015.73925148)
\curveto(489.33070065,1015.78924995)(489.49570048,1015.85424988)(489.61570557,1015.93425148)
\curveto(489.79570018,1016.04424969)(489.93570004,1016.18424955)(490.03570557,1016.35425148)
\curveto(490.14569983,1016.5342492)(490.22069976,1016.72924901)(490.26070557,1016.93925148)
\curveto(490.31069967,1017.15924858)(490.34069964,1017.39924834)(490.35070557,1017.65925148)
\curveto(490.36069962,1017.92924781)(490.36569961,1018.20924753)(490.36570557,1018.49925148)
\lineto(490.36570557,1020.31425148)
\lineto(490.36570557,1021.28925148)
\lineto(490.36570557,1021.55925148)
\curveto(490.36569961,1021.65924408)(490.38569959,1021.739244)(490.42570557,1021.79925148)
\curveto(490.4756995,1021.88924385)(490.55069943,1021.9392438)(490.65070557,1021.94925148)
\curveto(490.75069923,1021.96924377)(490.87069911,1021.97924376)(491.01070557,1021.97925148)
\lineto(491.80570557,1021.97925148)
\lineto(492.09070557,1021.97925148)
\curveto(492.1806978,1021.97924376)(492.25569772,1021.95924378)(492.31570557,1021.91925148)
\curveto(492.39569758,1021.86924387)(492.44069754,1021.79424394)(492.45070557,1021.69425148)
\curveto(492.46069752,1021.59424414)(492.46569751,1021.47924426)(492.46570557,1021.34925148)
\lineto(492.46570557,1020.20925148)
\lineto(492.46570557,1015.99425148)
\lineto(492.46570557,1014.92925148)
\lineto(492.46570557,1014.62925148)
\curveto(492.46569751,1014.52925121)(492.44569753,1014.45425128)(492.40570557,1014.40425148)
\curveto(492.35569762,1014.32425141)(492.2806977,1014.27925146)(492.18070557,1014.26925148)
\curveto(492.0806979,1014.25925148)(491.975698,1014.25425148)(491.86570557,1014.25425148)
\lineto(491.05570557,1014.25425148)
\curveto(490.94569903,1014.25425148)(490.84569913,1014.25925148)(490.75570557,1014.26925148)
\curveto(490.6756993,1014.27925146)(490.61069937,1014.31925142)(490.56070557,1014.38925148)
\curveto(490.54069944,1014.41925132)(490.52069946,1014.46425127)(490.50070557,1014.52425148)
\curveto(490.49069949,1014.58425115)(490.4756995,1014.64425109)(490.45570557,1014.70425148)
\curveto(490.44569953,1014.76425097)(490.43069955,1014.81925092)(490.41070557,1014.86925148)
\curveto(490.39069959,1014.91925082)(490.36069962,1014.94925079)(490.32070557,1014.95925148)
\curveto(490.30069968,1014.97925076)(490.2756997,1014.98425075)(490.24570557,1014.97425148)
\curveto(490.21569976,1014.96425077)(490.19069979,1014.95425078)(490.17070557,1014.94425148)
\curveto(490.10069988,1014.90425083)(490.04069994,1014.85925088)(489.99070557,1014.80925148)
\curveto(489.94070004,1014.75925098)(489.88570009,1014.71425102)(489.82570557,1014.67425148)
\curveto(489.78570019,1014.64425109)(489.74570023,1014.60925113)(489.70570557,1014.56925148)
\curveto(489.6757003,1014.5392512)(489.63570034,1014.50925123)(489.58570557,1014.47925148)
\curveto(489.35570062,1014.3392514)(489.08570089,1014.22925151)(488.77570557,1014.14925148)
\curveto(488.70570127,1014.12925161)(488.63570134,1014.11925162)(488.56570557,1014.11925148)
\curveto(488.49570148,1014.10925163)(488.42070156,1014.09425164)(488.34070557,1014.07425148)
\curveto(488.30070168,1014.06425167)(488.25570172,1014.06425167)(488.20570557,1014.07425148)
\curveto(488.16570181,1014.07425166)(488.12570185,1014.06925167)(488.08570557,1014.05925148)
\curveto(488.05570192,1014.04925169)(487.99070199,1014.04925169)(487.89070557,1014.05925148)
\curveto(487.80070218,1014.05925168)(487.74070224,1014.06425167)(487.71070557,1014.07425148)
\curveto(487.66070232,1014.07425166)(487.61070237,1014.07925166)(487.56070557,1014.08925148)
\lineto(487.41070557,1014.08925148)
\curveto(487.29070269,1014.11925162)(487.1757028,1014.14425159)(487.06570557,1014.16425148)
\curveto(486.95570302,1014.18425155)(486.84570313,1014.21425152)(486.73570557,1014.25425148)
\curveto(486.68570329,1014.27425146)(486.64070334,1014.28925145)(486.60070557,1014.29925148)
\curveto(486.57070341,1014.31925142)(486.53070345,1014.3392514)(486.48070557,1014.35925148)
\curveto(486.13070385,1014.54925119)(485.85070413,1014.81425092)(485.64070557,1015.15425148)
\curveto(485.51070447,1015.36425037)(485.41570456,1015.61425012)(485.35570557,1015.90425148)
\curveto(485.29570468,1016.20424953)(485.25570472,1016.51924922)(485.23570557,1016.84925148)
\curveto(485.22570475,1017.18924855)(485.22070476,1017.5342482)(485.22070557,1017.88425148)
\curveto(485.23070475,1018.24424749)(485.23570474,1018.59924714)(485.23570557,1018.94925148)
\lineto(485.23570557,1020.98925148)
\curveto(485.23570474,1021.11924462)(485.23070475,1021.26924447)(485.22070557,1021.43925148)
\curveto(485.22070476,1021.61924412)(485.24570473,1021.74924399)(485.29570557,1021.82925148)
\curveto(485.32570465,1021.87924386)(485.38570459,1021.92424381)(485.47570557,1021.96425148)
\curveto(485.53570444,1021.96424377)(485.5807044,1021.96924377)(485.61070557,1021.97925148)
}
}
{
\newrgbcolor{curcolor}{0 0 0}
\pscustom[linestyle=none,fillstyle=solid,fillcolor=curcolor]
{
\newpath
\moveto(498.52195557,1022.18925148)
\curveto(499.12194976,1022.20924353)(499.62194926,1022.12424361)(500.02195557,1021.93425148)
\curveto(500.42194846,1021.74424399)(500.73694815,1021.46424427)(500.96695557,1021.09425148)
\curveto(501.03694785,1020.98424475)(501.09194779,1020.86424487)(501.13195557,1020.73425148)
\curveto(501.17194771,1020.61424512)(501.21194767,1020.48924525)(501.25195557,1020.35925148)
\curveto(501.27194761,1020.27924546)(501.2819476,1020.20424553)(501.28195557,1020.13425148)
\curveto(501.29194759,1020.06424567)(501.30694758,1019.99424574)(501.32695557,1019.92425148)
\curveto(501.32694756,1019.86424587)(501.33194755,1019.82424591)(501.34195557,1019.80425148)
\curveto(501.36194752,1019.66424607)(501.37194751,1019.51924622)(501.37195557,1019.36925148)
\lineto(501.37195557,1018.93425148)
\lineto(501.37195557,1017.59925148)
\lineto(501.37195557,1015.16925148)
\curveto(501.37194751,1014.97925076)(501.36694752,1014.79425094)(501.35695557,1014.61425148)
\curveto(501.35694753,1014.44425129)(501.2869476,1014.3342514)(501.14695557,1014.28425148)
\curveto(501.0869478,1014.26425147)(501.01694787,1014.25425148)(500.93695557,1014.25425148)
\lineto(500.69695557,1014.25425148)
\lineto(499.88695557,1014.25425148)
\curveto(499.76694912,1014.25425148)(499.65694923,1014.25925148)(499.55695557,1014.26925148)
\curveto(499.46694942,1014.28925145)(499.39694949,1014.3342514)(499.34695557,1014.40425148)
\curveto(499.30694958,1014.46425127)(499.2819496,1014.5392512)(499.27195557,1014.62925148)
\lineto(499.27195557,1014.94425148)
\lineto(499.27195557,1015.99425148)
\lineto(499.27195557,1018.22925148)
\curveto(499.27194961,1018.59924714)(499.25694963,1018.9392468)(499.22695557,1019.24925148)
\curveto(499.19694969,1019.56924617)(499.10694978,1019.8392459)(498.95695557,1020.05925148)
\curveto(498.81695007,1020.25924548)(498.61195027,1020.39924534)(498.34195557,1020.47925148)
\curveto(498.29195059,1020.49924524)(498.23695065,1020.50924523)(498.17695557,1020.50925148)
\curveto(498.12695076,1020.50924523)(498.07195081,1020.51924522)(498.01195557,1020.53925148)
\curveto(497.96195092,1020.54924519)(497.89695099,1020.54924519)(497.81695557,1020.53925148)
\curveto(497.74695114,1020.5392452)(497.69195119,1020.5342452)(497.65195557,1020.52425148)
\curveto(497.61195127,1020.51424522)(497.57695131,1020.50924523)(497.54695557,1020.50925148)
\curveto(497.51695137,1020.50924523)(497.4869514,1020.50424523)(497.45695557,1020.49425148)
\curveto(497.22695166,1020.4342453)(497.04195184,1020.35424538)(496.90195557,1020.25425148)
\curveto(496.5819523,1020.02424571)(496.39195249,1019.68924605)(496.33195557,1019.24925148)
\curveto(496.27195261,1018.80924693)(496.24195264,1018.31424742)(496.24195557,1017.76425148)
\lineto(496.24195557,1015.88925148)
\lineto(496.24195557,1014.97425148)
\lineto(496.24195557,1014.70425148)
\curveto(496.24195264,1014.61425112)(496.22695266,1014.5392512)(496.19695557,1014.47925148)
\curveto(496.14695274,1014.36925137)(496.06695282,1014.30425143)(495.95695557,1014.28425148)
\curveto(495.84695304,1014.26425147)(495.71195317,1014.25425148)(495.55195557,1014.25425148)
\lineto(494.80195557,1014.25425148)
\curveto(494.69195419,1014.25425148)(494.5819543,1014.25925148)(494.47195557,1014.26925148)
\curveto(494.36195452,1014.27925146)(494.2819546,1014.31425142)(494.23195557,1014.37425148)
\curveto(494.16195472,1014.46425127)(494.12695476,1014.59425114)(494.12695557,1014.76425148)
\curveto(494.13695475,1014.9342508)(494.14195474,1015.09425064)(494.14195557,1015.24425148)
\lineto(494.14195557,1017.28425148)
\lineto(494.14195557,1020.58425148)
\lineto(494.14195557,1021.34925148)
\lineto(494.14195557,1021.64925148)
\curveto(494.15195473,1021.739244)(494.1819547,1021.81424392)(494.23195557,1021.87425148)
\curveto(494.25195463,1021.90424383)(494.2819546,1021.92424381)(494.32195557,1021.93425148)
\curveto(494.37195451,1021.95424378)(494.42195446,1021.96924377)(494.47195557,1021.97925148)
\lineto(494.54695557,1021.97925148)
\curveto(494.59695429,1021.98924375)(494.64695424,1021.99424374)(494.69695557,1021.99425148)
\lineto(494.86195557,1021.99425148)
\lineto(495.49195557,1021.99425148)
\curveto(495.57195331,1021.99424374)(495.64695324,1021.98924375)(495.71695557,1021.97925148)
\curveto(495.79695309,1021.97924376)(495.86695302,1021.96924377)(495.92695557,1021.94925148)
\curveto(495.99695289,1021.91924382)(496.04195284,1021.87424386)(496.06195557,1021.81425148)
\curveto(496.09195279,1021.75424398)(496.11695277,1021.68424405)(496.13695557,1021.60425148)
\curveto(496.14695274,1021.56424417)(496.14695274,1021.52924421)(496.13695557,1021.49925148)
\curveto(496.13695275,1021.46924427)(496.14695274,1021.4392443)(496.16695557,1021.40925148)
\curveto(496.1869527,1021.35924438)(496.20195268,1021.32924441)(496.21195557,1021.31925148)
\curveto(496.23195265,1021.30924443)(496.25695263,1021.29424444)(496.28695557,1021.27425148)
\curveto(496.39695249,1021.26424447)(496.4869524,1021.29924444)(496.55695557,1021.37925148)
\curveto(496.62695226,1021.46924427)(496.70195218,1021.5392442)(496.78195557,1021.58925148)
\curveto(497.05195183,1021.78924395)(497.35195153,1021.94924379)(497.68195557,1022.06925148)
\curveto(497.77195111,1022.09924364)(497.86195102,1022.11924362)(497.95195557,1022.12925148)
\curveto(498.05195083,1022.1392436)(498.15695073,1022.15424358)(498.26695557,1022.17425148)
\curveto(498.29695059,1022.18424355)(498.34195054,1022.18424355)(498.40195557,1022.17425148)
\curveto(498.46195042,1022.17424356)(498.50195038,1022.17924356)(498.52195557,1022.18925148)
}
}
{
\newrgbcolor{curcolor}{0 0 0}
\pscustom[linestyle=none,fillstyle=solid,fillcolor=curcolor]
{
\newpath
\moveto(510.05320557,1014.85425148)
\curveto(510.07319772,1014.74425099)(510.08319771,1014.6342511)(510.08320557,1014.52425148)
\curveto(510.0931977,1014.41425132)(510.04319775,1014.3392514)(509.93320557,1014.29925148)
\curveto(509.87319792,1014.26925147)(509.80319799,1014.25425148)(509.72320557,1014.25425148)
\lineto(509.48320557,1014.25425148)
\lineto(508.67320557,1014.25425148)
\lineto(508.40320557,1014.25425148)
\curveto(508.32319947,1014.26425147)(508.25819953,1014.28925145)(508.20820557,1014.32925148)
\curveto(508.13819965,1014.36925137)(508.08319971,1014.42425131)(508.04320557,1014.49425148)
\curveto(508.01319978,1014.57425116)(507.96819982,1014.6392511)(507.90820557,1014.68925148)
\curveto(507.8881999,1014.70925103)(507.86319993,1014.72425101)(507.83320557,1014.73425148)
\curveto(507.80319999,1014.75425098)(507.76320003,1014.75925098)(507.71320557,1014.74925148)
\curveto(507.66320013,1014.72925101)(507.61320018,1014.70425103)(507.56320557,1014.67425148)
\curveto(507.52320027,1014.64425109)(507.47820031,1014.61925112)(507.42820557,1014.59925148)
\curveto(507.37820041,1014.55925118)(507.32320047,1014.52425121)(507.26320557,1014.49425148)
\lineto(507.08320557,1014.40425148)
\curveto(506.95320084,1014.34425139)(506.81820097,1014.29425144)(506.67820557,1014.25425148)
\curveto(506.53820125,1014.22425151)(506.3932014,1014.18925155)(506.24320557,1014.14925148)
\curveto(506.17320162,1014.12925161)(506.10320169,1014.11925162)(506.03320557,1014.11925148)
\curveto(505.97320182,1014.10925163)(505.90820188,1014.09925164)(505.83820557,1014.08925148)
\lineto(505.74820557,1014.08925148)
\curveto(505.71820207,1014.07925166)(505.6882021,1014.07425166)(505.65820557,1014.07425148)
\lineto(505.49320557,1014.07425148)
\curveto(505.3932024,1014.05425168)(505.2932025,1014.05425168)(505.19320557,1014.07425148)
\lineto(505.05820557,1014.07425148)
\curveto(504.9882028,1014.09425164)(504.91820287,1014.10425163)(504.84820557,1014.10425148)
\curveto(504.788203,1014.09425164)(504.72820306,1014.09925164)(504.66820557,1014.11925148)
\curveto(504.56820322,1014.1392516)(504.47320332,1014.15925158)(504.38320557,1014.17925148)
\curveto(504.2932035,1014.18925155)(504.20820358,1014.21425152)(504.12820557,1014.25425148)
\curveto(503.83820395,1014.36425137)(503.5882042,1014.50425123)(503.37820557,1014.67425148)
\curveto(503.17820461,1014.85425088)(503.01820477,1015.08925065)(502.89820557,1015.37925148)
\curveto(502.86820492,1015.44925029)(502.83820495,1015.52425021)(502.80820557,1015.60425148)
\curveto(502.788205,1015.68425005)(502.76820502,1015.76924997)(502.74820557,1015.85925148)
\curveto(502.72820506,1015.90924983)(502.71820507,1015.95924978)(502.71820557,1016.00925148)
\curveto(502.72820506,1016.05924968)(502.72820506,1016.10924963)(502.71820557,1016.15925148)
\curveto(502.70820508,1016.18924955)(502.69820509,1016.24924949)(502.68820557,1016.33925148)
\curveto(502.6882051,1016.4392493)(502.6932051,1016.50924923)(502.70320557,1016.54925148)
\curveto(502.72320507,1016.64924909)(502.73320506,1016.734249)(502.73320557,1016.80425148)
\lineto(502.82320557,1017.13425148)
\curveto(502.85320494,1017.25424848)(502.8932049,1017.35924838)(502.94320557,1017.44925148)
\curveto(503.11320468,1017.739248)(503.30820448,1017.95924778)(503.52820557,1018.10925148)
\curveto(503.74820404,1018.25924748)(504.02820376,1018.38924735)(504.36820557,1018.49925148)
\curveto(504.49820329,1018.54924719)(504.63320316,1018.58424715)(504.77320557,1018.60425148)
\curveto(504.91320288,1018.62424711)(505.05320274,1018.64924709)(505.19320557,1018.67925148)
\curveto(505.27320252,1018.69924704)(505.35820243,1018.70924703)(505.44820557,1018.70925148)
\curveto(505.53820225,1018.71924702)(505.62820216,1018.734247)(505.71820557,1018.75425148)
\curveto(505.788202,1018.77424696)(505.85820193,1018.77924696)(505.92820557,1018.76925148)
\curveto(505.99820179,1018.76924697)(506.07320172,1018.77924696)(506.15320557,1018.79925148)
\curveto(506.22320157,1018.81924692)(506.2932015,1018.82924691)(506.36320557,1018.82925148)
\curveto(506.43320136,1018.82924691)(506.50820128,1018.8392469)(506.58820557,1018.85925148)
\curveto(506.79820099,1018.90924683)(506.9882008,1018.94924679)(507.15820557,1018.97925148)
\curveto(507.33820045,1019.01924672)(507.49820029,1019.10924663)(507.63820557,1019.24925148)
\curveto(507.72820006,1019.3392464)(507.7882,1019.4392463)(507.81820557,1019.54925148)
\curveto(507.82819996,1019.57924616)(507.82819996,1019.60424613)(507.81820557,1019.62425148)
\curveto(507.81819997,1019.64424609)(507.82319997,1019.66424607)(507.83320557,1019.68425148)
\curveto(507.84319995,1019.70424603)(507.84819994,1019.734246)(507.84820557,1019.77425148)
\lineto(507.84820557,1019.86425148)
\lineto(507.81820557,1019.98425148)
\curveto(507.81819997,1020.02424571)(507.81319998,1020.05924568)(507.80320557,1020.08925148)
\curveto(507.70320009,1020.38924535)(507.4932003,1020.59424514)(507.17320557,1020.70425148)
\curveto(507.08320071,1020.734245)(506.97320082,1020.75424498)(506.84320557,1020.76425148)
\curveto(506.72320107,1020.78424495)(506.59820119,1020.78924495)(506.46820557,1020.77925148)
\curveto(506.33820145,1020.77924496)(506.21320158,1020.76924497)(506.09320557,1020.74925148)
\curveto(505.97320182,1020.72924501)(505.86820192,1020.70424503)(505.77820557,1020.67425148)
\curveto(505.71820207,1020.65424508)(505.65820213,1020.62424511)(505.59820557,1020.58425148)
\curveto(505.54820224,1020.55424518)(505.49820229,1020.51924522)(505.44820557,1020.47925148)
\curveto(505.39820239,1020.4392453)(505.34320245,1020.38424535)(505.28320557,1020.31425148)
\curveto(505.23320256,1020.24424549)(505.19820259,1020.17924556)(505.17820557,1020.11925148)
\curveto(505.12820266,1020.01924572)(505.08320271,1019.92424581)(505.04320557,1019.83425148)
\curveto(505.01320278,1019.74424599)(504.94320285,1019.68424605)(504.83320557,1019.65425148)
\curveto(504.75320304,1019.6342461)(504.66820312,1019.62424611)(504.57820557,1019.62425148)
\lineto(504.30820557,1019.62425148)
\lineto(503.73820557,1019.62425148)
\curveto(503.6882041,1019.62424611)(503.63820415,1019.61924612)(503.58820557,1019.60925148)
\curveto(503.53820425,1019.60924613)(503.4932043,1019.61424612)(503.45320557,1019.62425148)
\lineto(503.31820557,1019.62425148)
\curveto(503.29820449,1019.6342461)(503.27320452,1019.6392461)(503.24320557,1019.63925148)
\curveto(503.21320458,1019.6392461)(503.1882046,1019.64924609)(503.16820557,1019.66925148)
\curveto(503.0882047,1019.68924605)(503.03320476,1019.75424598)(503.00320557,1019.86425148)
\curveto(502.9932048,1019.91424582)(502.9932048,1019.96424577)(503.00320557,1020.01425148)
\curveto(503.01320478,1020.06424567)(503.02320477,1020.10924563)(503.03320557,1020.14925148)
\curveto(503.06320473,1020.25924548)(503.0932047,1020.35924538)(503.12320557,1020.44925148)
\curveto(503.16320463,1020.54924519)(503.20820458,1020.6392451)(503.25820557,1020.71925148)
\lineto(503.34820557,1020.86925148)
\lineto(503.43820557,1021.01925148)
\curveto(503.51820427,1021.12924461)(503.61820417,1021.2342445)(503.73820557,1021.33425148)
\curveto(503.75820403,1021.34424439)(503.788204,1021.36924437)(503.82820557,1021.40925148)
\curveto(503.87820391,1021.44924429)(503.92320387,1021.48424425)(503.96320557,1021.51425148)
\curveto(504.00320379,1021.54424419)(504.04820374,1021.57424416)(504.09820557,1021.60425148)
\curveto(504.26820352,1021.71424402)(504.44820334,1021.79924394)(504.63820557,1021.85925148)
\curveto(504.82820296,1021.92924381)(505.02320277,1021.99424374)(505.22320557,1022.05425148)
\curveto(505.34320245,1022.08424365)(505.46820232,1022.10424363)(505.59820557,1022.11425148)
\curveto(505.72820206,1022.12424361)(505.85820193,1022.14424359)(505.98820557,1022.17425148)
\curveto(506.02820176,1022.18424355)(506.0882017,1022.18424355)(506.16820557,1022.17425148)
\curveto(506.25820153,1022.16424357)(506.31320148,1022.16924357)(506.33320557,1022.18925148)
\curveto(506.74320105,1022.19924354)(507.13320066,1022.18424355)(507.50320557,1022.14425148)
\curveto(507.88319991,1022.10424363)(508.22319957,1022.02924371)(508.52320557,1021.91925148)
\curveto(508.83319896,1021.80924393)(509.09819869,1021.65924408)(509.31820557,1021.46925148)
\curveto(509.53819825,1021.28924445)(509.70819808,1021.05424468)(509.82820557,1020.76425148)
\curveto(509.89819789,1020.59424514)(509.93819785,1020.39924534)(509.94820557,1020.17925148)
\curveto(509.95819783,1019.95924578)(509.96319783,1019.734246)(509.96320557,1019.50425148)
\lineto(509.96320557,1016.15925148)
\lineto(509.96320557,1015.57425148)
\curveto(509.96319783,1015.38425035)(509.98319781,1015.20925053)(510.02320557,1015.04925148)
\curveto(510.03319776,1015.01925072)(510.03819775,1014.98425075)(510.03820557,1014.94425148)
\curveto(510.03819775,1014.91425082)(510.04319775,1014.88425085)(510.05320557,1014.85425148)
\moveto(507.84820557,1017.16425148)
\curveto(507.85819993,1017.21424852)(507.86319993,1017.26924847)(507.86320557,1017.32925148)
\curveto(507.86319993,1017.39924834)(507.85819993,1017.45924828)(507.84820557,1017.50925148)
\curveto(507.82819996,1017.56924817)(507.81819997,1017.62424811)(507.81820557,1017.67425148)
\curveto(507.81819997,1017.72424801)(507.79819999,1017.76424797)(507.75820557,1017.79425148)
\curveto(507.70820008,1017.8342479)(507.63320016,1017.85424788)(507.53320557,1017.85425148)
\curveto(507.4932003,1017.84424789)(507.45820033,1017.8342479)(507.42820557,1017.82425148)
\curveto(507.39820039,1017.82424791)(507.36320043,1017.81924792)(507.32320557,1017.80925148)
\curveto(507.25320054,1017.78924795)(507.17820061,1017.77424796)(507.09820557,1017.76425148)
\curveto(507.01820077,1017.75424798)(506.93820085,1017.739248)(506.85820557,1017.71925148)
\curveto(506.82820096,1017.70924803)(506.78320101,1017.70424803)(506.72320557,1017.70425148)
\curveto(506.5932012,1017.67424806)(506.46320133,1017.65424808)(506.33320557,1017.64425148)
\curveto(506.20320159,1017.6342481)(506.07820171,1017.60924813)(505.95820557,1017.56925148)
\curveto(505.87820191,1017.54924819)(505.80320199,1017.52924821)(505.73320557,1017.50925148)
\curveto(505.66320213,1017.49924824)(505.5932022,1017.47924826)(505.52320557,1017.44925148)
\curveto(505.31320248,1017.35924838)(505.13320266,1017.22424851)(504.98320557,1017.04425148)
\curveto(504.84320295,1016.86424887)(504.793203,1016.61424912)(504.83320557,1016.29425148)
\curveto(504.85320294,1016.12424961)(504.90820288,1015.98424975)(504.99820557,1015.87425148)
\curveto(505.06820272,1015.76424997)(505.17320262,1015.67425006)(505.31320557,1015.60425148)
\curveto(505.45320234,1015.54425019)(505.60320219,1015.49925024)(505.76320557,1015.46925148)
\curveto(505.93320186,1015.4392503)(506.10820168,1015.42925031)(506.28820557,1015.43925148)
\curveto(506.47820131,1015.45925028)(506.65320114,1015.49425024)(506.81320557,1015.54425148)
\curveto(507.07320072,1015.62425011)(507.27820051,1015.74924999)(507.42820557,1015.91925148)
\curveto(507.57820021,1016.09924964)(507.6932001,1016.31924942)(507.77320557,1016.57925148)
\curveto(507.7932,1016.64924909)(507.80319999,1016.71924902)(507.80320557,1016.78925148)
\curveto(507.81319998,1016.86924887)(507.82819996,1016.94924879)(507.84820557,1017.02925148)
\lineto(507.84820557,1017.16425148)
}
}
{
\newrgbcolor{curcolor}{0 0 0}
\pscustom[linestyle=none,fillstyle=solid,fillcolor=curcolor]
{
}
}
{
\newrgbcolor{curcolor}{0 0 0}
\pscustom[linestyle=none,fillstyle=solid,fillcolor=curcolor]
{
\newpath
\moveto(515.52164307,1021.99425148)
\lineto(516.61664307,1021.99425148)
\curveto(516.72664134,1021.99424374)(516.83164123,1021.98924375)(516.93164307,1021.97925148)
\curveto(517.03164103,1021.97924376)(517.11664095,1021.95924378)(517.18664307,1021.91925148)
\curveto(517.29664077,1021.84924389)(517.37164069,1021.71924402)(517.41164307,1021.52925148)
\curveto(517.4616406,1021.34924439)(517.51164055,1021.18924455)(517.56164307,1021.04925148)
\curveto(517.67164039,1020.71924502)(517.77664029,1020.37924536)(517.87664307,1020.02925148)
\curveto(517.97664009,1019.68924605)(518.08163998,1019.34924639)(518.19164307,1019.00925148)
\curveto(518.27163979,1018.76924697)(518.34663972,1018.52424721)(518.41664307,1018.27425148)
\curveto(518.48663958,1018.02424771)(518.5666395,1017.78424795)(518.65664307,1017.55425148)
\curveto(518.68663938,1017.46424827)(518.71663935,1017.36924837)(518.74664307,1017.26925148)
\curveto(518.78663928,1017.16924857)(518.8616392,1017.11924862)(518.97164307,1017.11925148)
\curveto(518.99163907,1017.1392486)(519.00663906,1017.14924859)(519.01664307,1017.14925148)
\lineto(519.06164307,1017.19425148)
\curveto(519.09163897,1017.24424849)(519.11663895,1017.29424844)(519.13664307,1017.34425148)
\curveto(519.15663891,1017.39424834)(519.17663889,1017.44924829)(519.19664307,1017.50925148)
\curveto(519.24663882,1017.61924812)(519.28163878,1017.734248)(519.30164307,1017.85425148)
\curveto(519.33163873,1017.97424776)(519.3666387,1018.08924765)(519.40664307,1018.19925148)
\curveto(519.54663852,1018.61924712)(519.67663839,1019.0392467)(519.79664307,1019.45925148)
\curveto(519.92663814,1019.88924585)(520.061638,1020.31424542)(520.20164307,1020.73425148)
\curveto(520.25163781,1020.85424488)(520.29163777,1020.97424476)(520.32164307,1021.09425148)
\curveto(520.35163771,1021.22424451)(520.38663768,1021.34424439)(520.42664307,1021.45425148)
\curveto(520.44663762,1021.5342442)(520.47163759,1021.61424412)(520.50164307,1021.69425148)
\curveto(520.53163753,1021.77424396)(520.57663749,1021.8392439)(520.63664307,1021.88925148)
\curveto(520.6666374,1021.90924383)(520.73163733,1021.9392438)(520.83164307,1021.97925148)
\curveto(520.89163717,1021.99924374)(520.95663711,1022.00424373)(521.02664307,1021.99425148)
\lineto(521.22164307,1021.99425148)
\lineto(521.95664307,1021.99425148)
\curveto(522.066636,1021.99424374)(522.1666359,1021.98924375)(522.25664307,1021.97925148)
\curveto(522.35663571,1021.97924376)(522.43163563,1021.95424378)(522.48164307,1021.90425148)
\curveto(522.54163552,1021.85424388)(522.5616355,1021.77924396)(522.54164307,1021.67925148)
\curveto(522.53163553,1021.58924415)(522.51663555,1021.51424422)(522.49664307,1021.45425148)
\curveto(522.44663562,1021.31424442)(522.39663567,1021.16424457)(522.34664307,1021.00425148)
\curveto(522.29663577,1020.85424488)(522.24663582,1020.70924503)(522.19664307,1020.56925148)
\curveto(522.1666359,1020.49924524)(522.14163592,1020.42924531)(522.12164307,1020.35925148)
\curveto(522.11163595,1020.29924544)(522.09663597,1020.2392455)(522.07664307,1020.17925148)
\curveto(522.02663604,1020.06924567)(521.98163608,1019.95924578)(521.94164307,1019.84925148)
\curveto(521.91163615,1019.739246)(521.87663619,1019.62924611)(521.83664307,1019.51925148)
\curveto(521.82663624,1019.48924625)(521.81663625,1019.45424628)(521.80664307,1019.41425148)
\curveto(521.80663626,1019.38424635)(521.79663627,1019.35424638)(521.77664307,1019.32425148)
\curveto(521.71663635,1019.17424656)(521.6616364,1019.01924672)(521.61164307,1018.85925148)
\curveto(521.57163649,1018.70924703)(521.52663654,1018.55924718)(521.47664307,1018.40925148)
\curveto(521.3666367,1018.10924763)(521.2616368,1017.79924794)(521.16164307,1017.47925148)
\curveto(521.061637,1017.16924857)(520.95163711,1016.86424887)(520.83164307,1016.56425148)
\curveto(520.78163728,1016.42424931)(520.73663733,1016.28424945)(520.69664307,1016.14425148)
\curveto(520.65663741,1016.01424972)(520.61163745,1015.87924986)(520.56164307,1015.73925148)
\curveto(520.54163752,1015.68925005)(520.52663754,1015.64425009)(520.51664307,1015.60425148)
\curveto(520.50663756,1015.57425016)(520.49163757,1015.5342502)(520.47164307,1015.48425148)
\curveto(520.43163763,1015.37425036)(520.39163767,1015.25925048)(520.35164307,1015.13925148)
\curveto(520.32163774,1015.02925071)(520.28663778,1014.91925082)(520.24664307,1014.80925148)
\curveto(520.19663787,1014.69925104)(520.15163791,1014.59425114)(520.11164307,1014.49425148)
\curveto(520.07163799,1014.40425133)(519.99663807,1014.3392514)(519.88664307,1014.29925148)
\curveto(519.80663826,1014.26925147)(519.72163834,1014.25425148)(519.63164307,1014.25425148)
\curveto(519.54163852,1014.26425147)(519.45163861,1014.26925147)(519.36164307,1014.26925148)
\lineto(518.38664307,1014.26925148)
\lineto(518.05664307,1014.26925148)
\curveto(517.95664011,1014.26925147)(517.8666402,1014.28925145)(517.78664307,1014.32925148)
\curveto(517.71664035,1014.37925136)(517.6666404,1014.4392513)(517.63664307,1014.50925148)
\curveto(517.60664046,1014.58925115)(517.57664049,1014.67425106)(517.54664307,1014.76425148)
\curveto(517.49664057,1014.88425085)(517.45164061,1015.00925073)(517.41164307,1015.13925148)
\curveto(517.37164069,1015.26925047)(517.32664074,1015.39425034)(517.27664307,1015.51425148)
\curveto(517.25664081,1015.55425018)(517.24164082,1015.58925015)(517.23164307,1015.61925148)
\curveto(517.23164083,1015.65925008)(517.22164084,1015.70425003)(517.20164307,1015.75425148)
\curveto(517.15164091,1015.87424986)(517.10664096,1015.99924974)(517.06664307,1016.12925148)
\curveto(517.02664104,1016.25924948)(516.98164108,1016.38924935)(516.93164307,1016.51925148)
\curveto(516.91164115,1016.56924917)(516.89664117,1016.61424912)(516.88664307,1016.65425148)
\curveto(516.87664119,1016.70424903)(516.8616412,1016.75424898)(516.84164307,1016.80425148)
\curveto(516.80164126,1016.89424884)(516.7666413,1016.98424875)(516.73664307,1017.07425148)
\curveto(516.70664136,1017.17424856)(516.67664139,1017.26924847)(516.64664307,1017.35925148)
\curveto(516.62664144,1017.38924835)(516.61664145,1017.41924832)(516.61664307,1017.44925148)
\curveto(516.61664145,1017.48924825)(516.60664146,1017.52424821)(516.58664307,1017.55425148)
\curveto(516.54664152,1017.64424809)(516.51164155,1017.734248)(516.48164307,1017.82425148)
\curveto(516.4616416,1017.91424782)(516.43164163,1018.00924773)(516.39164307,1018.10925148)
\curveto(516.30164176,1018.3392474)(516.21664185,1018.57424716)(516.13664307,1018.81425148)
\curveto(516.05664201,1019.06424667)(515.97664209,1019.30924643)(515.89664307,1019.54925148)
\curveto(515.78664228,1019.81924592)(515.69164237,1020.08924565)(515.61164307,1020.35925148)
\curveto(515.53164253,1020.6392451)(515.44164262,1020.91424482)(515.34164307,1021.18425148)
\lineto(515.20664307,1021.58925148)
\curveto(515.19664287,1021.61924412)(515.18664288,1021.65424408)(515.17664307,1021.69425148)
\curveto(515.17664289,1021.74424399)(515.18664288,1021.78924395)(515.20664307,1021.82925148)
\curveto(515.23664283,1021.89924384)(515.30664276,1021.94924379)(515.41664307,1021.97925148)
\curveto(515.4666426,1021.97924376)(515.50164256,1021.98424375)(515.52164307,1021.99425148)
}
}
{
\newrgbcolor{curcolor}{0 0 0}
\pscustom[linestyle=none,fillstyle=solid,fillcolor=curcolor]
{
\newpath
\moveto(530.93461182,1018.19925148)
\curveto(530.95460365,1018.11924762)(530.95460365,1018.02924771)(530.93461182,1017.92925148)
\curveto(530.91460369,1017.82924791)(530.87960373,1017.76424797)(530.82961182,1017.73425148)
\curveto(530.77960383,1017.69424804)(530.7046039,1017.66424807)(530.60461182,1017.64425148)
\curveto(530.51460409,1017.6342481)(530.4096042,1017.62424811)(530.28961182,1017.61425148)
\lineto(529.94461182,1017.61425148)
\curveto(529.83460477,1017.62424811)(529.73460487,1017.62924811)(529.64461182,1017.62925148)
\lineto(525.98461182,1017.62925148)
\lineto(525.77461182,1017.62925148)
\curveto(525.71460889,1017.62924811)(525.65960895,1017.61924812)(525.60961182,1017.59925148)
\curveto(525.52960908,1017.55924818)(525.47960913,1017.51924822)(525.45961182,1017.47925148)
\curveto(525.43960917,1017.45924828)(525.41960919,1017.41924832)(525.39961182,1017.35925148)
\curveto(525.37960923,1017.30924843)(525.37460923,1017.25924848)(525.38461182,1017.20925148)
\curveto(525.4046092,1017.14924859)(525.41460919,1017.08924865)(525.41461182,1017.02925148)
\curveto(525.42460918,1016.97924876)(525.43960917,1016.92424881)(525.45961182,1016.86425148)
\curveto(525.53960907,1016.62424911)(525.63460897,1016.42424931)(525.74461182,1016.26425148)
\curveto(525.86460874,1016.11424962)(526.02460858,1015.97924976)(526.22461182,1015.85925148)
\curveto(526.3046083,1015.80924993)(526.38460822,1015.77424996)(526.46461182,1015.75425148)
\curveto(526.55460805,1015.74424999)(526.64460796,1015.72425001)(526.73461182,1015.69425148)
\curveto(526.81460779,1015.67425006)(526.92460768,1015.65925008)(527.06461182,1015.64925148)
\curveto(527.2046074,1015.6392501)(527.32460728,1015.64425009)(527.42461182,1015.66425148)
\lineto(527.55961182,1015.66425148)
\curveto(527.65960695,1015.68425005)(527.74960686,1015.70425003)(527.82961182,1015.72425148)
\curveto(527.91960669,1015.75424998)(528.0046066,1015.78424995)(528.08461182,1015.81425148)
\curveto(528.18460642,1015.86424987)(528.29460631,1015.92924981)(528.41461182,1016.00925148)
\curveto(528.54460606,1016.08924965)(528.63960597,1016.16924957)(528.69961182,1016.24925148)
\curveto(528.74960586,1016.31924942)(528.79960581,1016.38424935)(528.84961182,1016.44425148)
\curveto(528.9096057,1016.51424922)(528.97960563,1016.56424917)(529.05961182,1016.59425148)
\curveto(529.15960545,1016.64424909)(529.28460532,1016.66424907)(529.43461182,1016.65425148)
\lineto(529.86961182,1016.65425148)
\lineto(530.04961182,1016.65425148)
\curveto(530.11960449,1016.66424907)(530.17960443,1016.65924908)(530.22961182,1016.63925148)
\lineto(530.37961182,1016.63925148)
\curveto(530.47960413,1016.61924912)(530.54960406,1016.59424914)(530.58961182,1016.56425148)
\curveto(530.62960398,1016.54424919)(530.64960396,1016.49924924)(530.64961182,1016.42925148)
\curveto(530.65960395,1016.35924938)(530.65460395,1016.29924944)(530.63461182,1016.24925148)
\curveto(530.58460402,1016.10924963)(530.52960408,1015.98424975)(530.46961182,1015.87425148)
\curveto(530.4096042,1015.76424997)(530.33960427,1015.65425008)(530.25961182,1015.54425148)
\curveto(530.03960457,1015.21425052)(529.78960482,1014.94925079)(529.50961182,1014.74925148)
\curveto(529.22960538,1014.54925119)(528.87960573,1014.37925136)(528.45961182,1014.23925148)
\curveto(528.34960626,1014.19925154)(528.23960637,1014.17425156)(528.12961182,1014.16425148)
\curveto(528.01960659,1014.15425158)(527.9046067,1014.1342516)(527.78461182,1014.10425148)
\curveto(527.74460686,1014.09425164)(527.69960691,1014.09425164)(527.64961182,1014.10425148)
\curveto(527.609607,1014.10425163)(527.56960704,1014.09925164)(527.52961182,1014.08925148)
\lineto(527.36461182,1014.08925148)
\curveto(527.31460729,1014.06925167)(527.25460735,1014.06425167)(527.18461182,1014.07425148)
\curveto(527.12460748,1014.07425166)(527.06960754,1014.07925166)(527.01961182,1014.08925148)
\curveto(526.93960767,1014.09925164)(526.86960774,1014.09925164)(526.80961182,1014.08925148)
\curveto(526.74960786,1014.07925166)(526.68460792,1014.08425165)(526.61461182,1014.10425148)
\curveto(526.56460804,1014.12425161)(526.5096081,1014.1342516)(526.44961182,1014.13425148)
\curveto(526.38960822,1014.1342516)(526.33460827,1014.14425159)(526.28461182,1014.16425148)
\curveto(526.17460843,1014.18425155)(526.06460854,1014.20925153)(525.95461182,1014.23925148)
\curveto(525.84460876,1014.25925148)(525.74460886,1014.29425144)(525.65461182,1014.34425148)
\curveto(525.54460906,1014.38425135)(525.43960917,1014.41925132)(525.33961182,1014.44925148)
\curveto(525.24960936,1014.48925125)(525.16460944,1014.5342512)(525.08461182,1014.58425148)
\curveto(524.76460984,1014.78425095)(524.47961013,1015.01425072)(524.22961182,1015.27425148)
\curveto(523.97961063,1015.54425019)(523.77461083,1015.85424988)(523.61461182,1016.20425148)
\curveto(523.56461104,1016.31424942)(523.52461108,1016.42424931)(523.49461182,1016.53425148)
\curveto(523.46461114,1016.65424908)(523.42461118,1016.77424896)(523.37461182,1016.89425148)
\curveto(523.36461124,1016.9342488)(523.35961125,1016.96924877)(523.35961182,1016.99925148)
\curveto(523.35961125,1017.0392487)(523.35461125,1017.07924866)(523.34461182,1017.11925148)
\curveto(523.3046113,1017.2392485)(523.27961133,1017.36924837)(523.26961182,1017.50925148)
\lineto(523.23961182,1017.92925148)
\curveto(523.23961137,1017.97924776)(523.23461137,1018.0342477)(523.22461182,1018.09425148)
\curveto(523.22461138,1018.15424758)(523.22961138,1018.20924753)(523.23961182,1018.25925148)
\lineto(523.23961182,1018.43925148)
\lineto(523.28461182,1018.79925148)
\curveto(523.32461128,1018.96924677)(523.35961125,1019.1342466)(523.38961182,1019.29425148)
\curveto(523.41961119,1019.45424628)(523.46461114,1019.60424613)(523.52461182,1019.74425148)
\curveto(523.95461065,1020.78424495)(524.68460992,1021.51924422)(525.71461182,1021.94925148)
\curveto(525.85460875,1022.00924373)(525.99460861,1022.04924369)(526.13461182,1022.06925148)
\curveto(526.28460832,1022.09924364)(526.43960817,1022.1342436)(526.59961182,1022.17425148)
\curveto(526.67960793,1022.18424355)(526.75460785,1022.18924355)(526.82461182,1022.18925148)
\curveto(526.89460771,1022.18924355)(526.96960764,1022.19424354)(527.04961182,1022.20425148)
\curveto(527.55960705,1022.21424352)(527.99460661,1022.15424358)(528.35461182,1022.02425148)
\curveto(528.72460588,1021.90424383)(529.05460555,1021.74424399)(529.34461182,1021.54425148)
\curveto(529.43460517,1021.48424425)(529.52460508,1021.41424432)(529.61461182,1021.33425148)
\curveto(529.7046049,1021.26424447)(529.78460482,1021.18924455)(529.85461182,1021.10925148)
\curveto(529.88460472,1021.05924468)(529.92460468,1021.01924472)(529.97461182,1020.98925148)
\curveto(530.05460455,1020.87924486)(530.12960448,1020.76424497)(530.19961182,1020.64425148)
\curveto(530.26960434,1020.5342452)(530.34460426,1020.41924532)(530.42461182,1020.29925148)
\curveto(530.47460413,1020.20924553)(530.51460409,1020.11424562)(530.54461182,1020.01425148)
\curveto(530.58460402,1019.92424581)(530.62460398,1019.82424591)(530.66461182,1019.71425148)
\curveto(530.71460389,1019.58424615)(530.75460385,1019.44924629)(530.78461182,1019.30925148)
\curveto(530.81460379,1019.16924657)(530.84960376,1019.02924671)(530.88961182,1018.88925148)
\curveto(530.9096037,1018.80924693)(530.91460369,1018.71924702)(530.90461182,1018.61925148)
\curveto(530.9046037,1018.52924721)(530.91460369,1018.44424729)(530.93461182,1018.36425148)
\lineto(530.93461182,1018.19925148)
\moveto(528.68461182,1019.08425148)
\curveto(528.75460585,1019.18424655)(528.75960585,1019.30424643)(528.69961182,1019.44425148)
\curveto(528.64960596,1019.59424614)(528.609606,1019.70424603)(528.57961182,1019.77425148)
\curveto(528.43960617,1020.04424569)(528.25460635,1020.24924549)(528.02461182,1020.38925148)
\curveto(527.79460681,1020.5392452)(527.47460713,1020.61924512)(527.06461182,1020.62925148)
\curveto(527.03460757,1020.60924513)(526.99960761,1020.60424513)(526.95961182,1020.61425148)
\curveto(526.91960769,1020.62424511)(526.88460772,1020.62424511)(526.85461182,1020.61425148)
\curveto(526.8046078,1020.59424514)(526.74960786,1020.57924516)(526.68961182,1020.56925148)
\curveto(526.62960798,1020.56924517)(526.57460803,1020.55924518)(526.52461182,1020.53925148)
\curveto(526.08460852,1020.39924534)(525.75960885,1020.12424561)(525.54961182,1019.71425148)
\curveto(525.52960908,1019.67424606)(525.5046091,1019.61924612)(525.47461182,1019.54925148)
\curveto(525.45460915,1019.48924625)(525.43960917,1019.42424631)(525.42961182,1019.35425148)
\curveto(525.41960919,1019.29424644)(525.41960919,1019.2342465)(525.42961182,1019.17425148)
\curveto(525.44960916,1019.11424662)(525.48460912,1019.06424667)(525.53461182,1019.02425148)
\curveto(525.61460899,1018.97424676)(525.72460888,1018.94924679)(525.86461182,1018.94925148)
\lineto(526.26961182,1018.94925148)
\lineto(527.93461182,1018.94925148)
\lineto(528.36961182,1018.94925148)
\curveto(528.52960608,1018.95924678)(528.63460597,1019.00424673)(528.68461182,1019.08425148)
}
}
{
\newrgbcolor{curcolor}{0 0 0}
\pscustom[linestyle=none,fillstyle=solid,fillcolor=curcolor]
{
\newpath
\moveto(532.43789307,1021.99425148)
\lineto(537.04289307,1021.99425148)
\lineto(537.91289307,1021.99425148)
\curveto(538.00288647,1021.99424374)(538.09288638,1021.98924375)(538.18289307,1021.97925148)
\curveto(538.2728862,1021.97924376)(538.34288613,1021.95924378)(538.39289307,1021.91925148)
\curveto(538.45288602,1021.87924386)(538.50288597,1021.80424393)(538.54289307,1021.69425148)
\lineto(538.54289307,1021.46925148)
\curveto(538.56288591,1021.41924432)(538.5678859,1021.35924438)(538.55789307,1021.28925148)
\lineto(538.55789307,1021.12425148)
\lineto(538.55789307,1020.73425148)
\curveto(538.55788591,1020.60424513)(538.53788593,1020.49424524)(538.49789307,1020.40425148)
\curveto(538.45788601,1020.32424541)(538.41288606,1020.25424548)(538.36289307,1020.19425148)
\curveto(538.31288616,1020.14424559)(538.26288621,1020.08924565)(538.21289307,1020.02925148)
\curveto(538.15288632,1019.94924579)(538.08788638,1019.86924587)(538.01789307,1019.78925148)
\curveto(537.94788652,1019.71924602)(537.88288659,1019.64424609)(537.82289307,1019.56425148)
\lineto(537.76289307,1019.50425148)
\curveto(537.74288673,1019.49424624)(537.72288675,1019.47924626)(537.70289307,1019.45925148)
\curveto(537.65288682,1019.38924635)(537.59288688,1019.32424641)(537.52289307,1019.26425148)
\curveto(537.46288701,1019.20424653)(537.40788706,1019.1392466)(537.35789307,1019.06925148)
\curveto(537.19788727,1018.85924688)(537.02788744,1018.66424707)(536.84789307,1018.48425148)
\curveto(536.6678878,1018.30424743)(536.49788797,1018.10924763)(536.33789307,1017.89925148)
\lineto(536.21789307,1017.77925148)
\curveto(536.1678883,1017.70924803)(536.10788836,1017.64424809)(536.03789307,1017.58425148)
\curveto(535.97788849,1017.52424821)(535.92288855,1017.45924828)(535.87289307,1017.38925148)
\curveto(535.82288865,1017.31924842)(535.76288871,1017.25424848)(535.69289307,1017.19425148)
\curveto(535.63288884,1017.1342486)(535.57788889,1017.07424866)(535.52789307,1017.01425148)
\curveto(535.47788899,1016.95424878)(535.42788904,1016.89924884)(535.37789307,1016.84925148)
\curveto(535.33788913,1016.79924894)(535.29288918,1016.74424899)(535.24289307,1016.68425148)
\lineto(535.12289307,1016.53425148)
\curveto(535.08288939,1016.49424924)(535.04288943,1016.44924929)(535.00289307,1016.39925148)
\curveto(534.96288951,1016.35924938)(534.92788954,1016.31924942)(534.89789307,1016.27925148)
\curveto(534.8678896,1016.2392495)(534.84288963,1016.19424954)(534.82289307,1016.14425148)
\curveto(534.81288966,1016.12424961)(534.80288967,1016.09924964)(534.79289307,1016.06925148)
\curveto(534.79288968,1016.0392497)(534.80288967,1016.01424972)(534.82289307,1015.99425148)
\curveto(534.84288963,1015.9342498)(534.87788959,1015.89924984)(534.92789307,1015.88925148)
\curveto(534.97788949,1015.88924985)(535.03288944,1015.87924986)(535.09289307,1015.85925148)
\curveto(535.19288928,1015.8392499)(535.30288917,1015.82924991)(535.42289307,1015.82925148)
\curveto(535.55288892,1015.8392499)(535.6728888,1015.84424989)(535.78289307,1015.84425148)
\lineto(537.98789307,1015.84425148)
\curveto(538.1678863,1015.84424989)(538.33788613,1015.8392499)(538.49789307,1015.82925148)
\curveto(538.65788581,1015.81924992)(538.76288571,1015.74424999)(538.81289307,1015.60425148)
\curveto(538.83288564,1015.55425018)(538.84288563,1015.49925024)(538.84289307,1015.43925148)
\lineto(538.84289307,1015.24425148)
\lineto(538.84289307,1014.73425148)
\curveto(538.84288563,1014.54425119)(538.79788567,1014.41425132)(538.70789307,1014.34425148)
\curveto(538.64788582,1014.29425144)(538.5728859,1014.26925147)(538.48289307,1014.26925148)
\lineto(538.18289307,1014.26925148)
\lineto(537.25289307,1014.26925148)
\lineto(533.42789307,1014.26925148)
\lineto(532.40789307,1014.26925148)
\lineto(532.09289307,1014.26925148)
\curveto(532.00289247,1014.26925147)(531.92289255,1014.29425144)(531.85289307,1014.34425148)
\curveto(531.79289268,1014.39425134)(531.75789271,1014.47425126)(531.74789307,1014.58425148)
\curveto(531.73789273,1014.69425104)(531.73289274,1014.80425093)(531.73289307,1014.91425148)
\curveto(531.73289274,1015.05425068)(531.72789274,1015.20425053)(531.71789307,1015.36425148)
\curveto(531.71789275,1015.5342502)(531.73789273,1015.67425006)(531.77789307,1015.78425148)
\curveto(531.79789267,1015.84424989)(531.82289265,1015.89924984)(531.85289307,1015.94925148)
\curveto(531.88289259,1016.00924973)(531.91789255,1016.06424967)(531.95789307,1016.11425148)
\curveto(532.03789243,1016.20424953)(532.11789235,1016.28924945)(532.19789307,1016.36925148)
\curveto(532.27789219,1016.45924928)(532.35789211,1016.55424918)(532.43789307,1016.65425148)
\curveto(532.48789198,1016.70424903)(532.52789194,1016.739249)(532.55789307,1016.75925148)
\curveto(532.69789177,1016.94924879)(532.84789162,1017.12424861)(533.00789307,1017.28425148)
\curveto(533.1678913,1017.44424829)(533.31789115,1017.61424812)(533.45789307,1017.79425148)
\curveto(533.49789097,1017.84424789)(533.53789093,1017.88924785)(533.57789307,1017.92925148)
\curveto(533.61789085,1017.96924777)(533.65289082,1018.01424772)(533.68289307,1018.06425148)
\curveto(533.73289074,1018.1342476)(533.78789068,1018.19424754)(533.84789307,1018.24425148)
\curveto(533.91789055,1018.30424743)(533.97789049,1018.37424736)(534.02789307,1018.45425148)
\curveto(534.04789042,1018.47424726)(534.0678904,1018.48924725)(534.08789307,1018.49925148)
\lineto(534.14789307,1018.55925148)
\curveto(534.19789027,1018.6392471)(534.25289022,1018.70924703)(534.31289307,1018.76925148)
\curveto(534.38289009,1018.82924691)(534.44289003,1018.89424684)(534.49289307,1018.96425148)
\curveto(534.51288996,1018.99424674)(534.53788993,1019.02424671)(534.56789307,1019.05425148)
\curveto(534.60788986,1019.08424665)(534.64288983,1019.11424662)(534.67289307,1019.14425148)
\curveto(534.75288972,1019.25424648)(534.83788963,1019.35424638)(534.92789307,1019.44425148)
\curveto(535.01788945,1019.5342462)(535.10288937,1019.62924611)(535.18289307,1019.72925148)
\curveto(535.23288924,1019.79924594)(535.28788918,1019.86424587)(535.34789307,1019.92425148)
\curveto(535.41788905,1019.98424575)(535.46288901,1020.05924568)(535.48289307,1020.14925148)
\curveto(535.50288897,1020.20924553)(535.49288898,1020.25424548)(535.45289307,1020.28425148)
\curveto(535.42288905,1020.31424542)(535.39288908,1020.3392454)(535.36289307,1020.35925148)
\curveto(535.25288922,1020.40924533)(535.11788935,1020.42924531)(534.95789307,1020.41925148)
\lineto(534.52289307,1020.41925148)
\lineto(532.90289307,1020.41925148)
\curveto(532.78289169,1020.41924532)(532.65289182,1020.41424532)(532.51289307,1020.40425148)
\curveto(532.38289209,1020.40424533)(532.27789219,1020.42924531)(532.19789307,1020.47925148)
\curveto(532.10789236,1020.52924521)(532.05789241,1020.62424511)(532.04789307,1020.76425148)
\curveto(532.03789243,1020.90424483)(532.03289244,1021.04424469)(532.03289307,1021.18425148)
\curveto(532.03289244,1021.2342445)(532.02789244,1021.28924445)(532.01789307,1021.34925148)
\curveto(532.01789245,1021.40924433)(532.02289245,1021.46424427)(532.03289307,1021.51425148)
\lineto(532.03289307,1021.63425148)
\curveto(532.04289243,1021.66424407)(532.04789242,1021.69924404)(532.04789307,1021.73925148)
\curveto(532.05789241,1021.77924396)(532.0728924,1021.80924393)(532.09289307,1021.82925148)
\curveto(532.15289232,1021.90924383)(532.21789225,1021.95424378)(532.28789307,1021.96425148)
\curveto(532.30789216,1021.97424376)(532.33289214,1021.97424376)(532.36289307,1021.96425148)
\curveto(532.39289208,1021.96424377)(532.41789205,1021.97424376)(532.43789307,1021.99425148)
}
}
{
\newrgbcolor{curcolor}{0 0 0}
\pscustom[linestyle=none,fillstyle=solid,fillcolor=curcolor]
{
\newpath
\moveto(20.6807132,694.97712036)
\lineto(21.9557132,694.97712036)
\curveto(22.06571042,694.97710965)(22.17071031,694.97210966)(22.2707132,694.96212036)
\curveto(22.3807101,694.95210968)(22.46071002,694.91710971)(22.5107132,694.85712036)
\curveto(22.56070992,694.77710985)(22.5857099,694.67210996)(22.5857132,694.54212036)
\curveto(22.59570989,694.42211021)(22.60070988,694.29711033)(22.6007132,694.16712036)
\lineto(22.6007132,692.65212036)
\lineto(22.6007132,689.56212036)
\lineto(22.6007132,689.03712036)
\curveto(22.60070988,688.99711563)(22.59570989,688.95211568)(22.5857132,688.90212036)
\curveto(22.5857099,688.86211577)(22.59070989,688.82211581)(22.6007132,688.78212036)
\lineto(22.6007132,688.54212036)
\curveto(22.60070988,688.45211618)(22.59570989,688.35711627)(22.5857132,688.25712036)
\curveto(22.5857099,688.15711647)(22.59570989,688.06711656)(22.6157132,687.98712036)
\curveto(22.61570987,687.91711671)(22.62070986,687.86211677)(22.6307132,687.82212036)
\curveto(22.65070983,687.71211692)(22.66570982,687.60211703)(22.6757132,687.49212036)
\curveto(22.69570979,687.38211725)(22.72570976,687.27211736)(22.7657132,687.16212036)
\curveto(22.87570961,686.90211773)(23.01570947,686.68711794)(23.1857132,686.51712036)
\curveto(23.36570912,686.34711828)(23.60070888,686.21211842)(23.8907132,686.11212036)
\curveto(23.97070851,686.09211854)(24.05070843,686.07711855)(24.1307132,686.06712036)
\curveto(24.21070827,686.05711857)(24.29070819,686.04211859)(24.3707132,686.02212036)
\curveto(24.42070806,686.00211863)(24.46570802,685.99211864)(24.5057132,685.99212036)
\curveto(24.54570794,686.00211863)(24.59070789,686.00211863)(24.6407132,685.99212036)
\curveto(24.6807078,685.98211865)(24.74570774,685.97711865)(24.8357132,685.97712036)
\curveto(24.92570756,685.98711864)(24.9857075,685.99711863)(25.0157132,686.00712036)
\lineto(25.2407132,686.00712036)
\curveto(25.32070716,686.0271186)(25.40070708,686.04211859)(25.4807132,686.05212036)
\curveto(25.56070692,686.06211857)(25.63570685,686.07711855)(25.7057132,686.09712036)
\curveto(25.84570664,686.1271185)(25.95570653,686.16211847)(26.0357132,686.20212036)
\curveto(26.21570627,686.28211835)(26.37070611,686.38711824)(26.5007132,686.51712036)
\curveto(26.64070584,686.65711797)(26.75070573,686.81211782)(26.8307132,686.98212036)
\curveto(26.94070554,687.24211739)(27.00570548,687.54711708)(27.0257132,687.89712036)
\curveto(27.04570544,688.25711637)(27.05570543,688.627116)(27.0557132,689.00712036)
\lineto(27.0557132,691.99212036)
\lineto(27.0557132,694.00212036)
\curveto(27.05570543,694.14211049)(27.05070543,694.29711033)(27.0407132,694.46712036)
\curveto(27.04070544,694.63710999)(27.07070541,694.76210987)(27.1307132,694.84212036)
\curveto(27.1807053,694.90210973)(27.25070523,694.93710969)(27.3407132,694.94712036)
\curveto(27.43070505,694.96710966)(27.53070495,694.97710965)(27.6407132,694.97712036)
\lineto(28.6007132,694.97712036)
\curveto(28.6807038,694.97710965)(28.75570373,694.97710965)(28.8257132,694.97712036)
\curveto(28.90570358,694.98710964)(28.9807035,694.98210965)(29.0507132,694.96212036)
\curveto(29.19070329,694.9321097)(29.2807032,694.88210975)(29.3207132,694.81212036)
\curveto(29.37070311,694.7321099)(29.39070309,694.61711001)(29.3807132,694.46712036)
\curveto(29.3807031,694.3271103)(29.3807031,694.19711043)(29.3807132,694.07712036)
\lineto(29.3807132,692.06712036)
\lineto(29.3807132,689.03712036)
\curveto(29.3807031,688.65711597)(29.37570311,688.28711634)(29.3657132,687.92712036)
\curveto(29.35570313,687.56711706)(29.31070317,687.24211739)(29.2307132,686.95212036)
\curveto(29.09070339,686.48211815)(28.91070357,686.07211856)(28.6907132,685.72212036)
\curveto(28.480704,685.38211925)(28.20070428,685.09211954)(27.8507132,684.85212036)
\curveto(27.54070494,684.63212)(27.17570531,684.45212018)(26.7557132,684.31212036)
\curveto(26.66570582,684.28212035)(26.57070591,684.25712037)(26.4707132,684.23712036)
\lineto(26.2007132,684.17712036)
\curveto(26.14070634,684.15712047)(26.0807064,684.14712048)(26.0207132,684.14712036)
\curveto(25.97070651,684.14712048)(25.91570657,684.13712049)(25.8557132,684.11712036)
\curveto(25.73570675,684.09712053)(25.60070688,684.08212055)(25.4507132,684.07212036)
\curveto(25.30070718,684.06212057)(25.15570733,684.05712057)(25.0157132,684.05712036)
\curveto(24.06570842,684.04712058)(23.25570923,684.16212047)(22.5857132,684.40212036)
\curveto(21.91571057,684.65211998)(21.39071109,685.05211958)(21.0107132,685.60212036)
\curveto(20.8807116,685.78211885)(20.77071171,685.96711866)(20.6807132,686.15712036)
\curveto(20.60071188,686.35711827)(20.52571196,686.57211806)(20.4557132,686.80212036)
\curveto(20.43571205,686.85211778)(20.42571206,686.89211774)(20.4257132,686.92212036)
\curveto(20.42571206,686.96211767)(20.41571207,687.00711762)(20.3957132,687.05712036)
\curveto(20.31571217,687.33711729)(20.27571221,687.65211698)(20.2757132,688.00212036)
\lineto(20.2757132,689.05212036)
\lineto(20.2757132,693.23712036)
\lineto(20.2757132,694.28712036)
\lineto(20.2757132,694.57212036)
\curveto(20.27571221,694.67210996)(20.29071219,694.75210988)(20.3207132,694.81212036)
\curveto(20.3807121,694.88210975)(20.46071202,694.9321097)(20.5607132,694.96212036)
\curveto(20.5807119,694.96210967)(20.60071188,694.96210967)(20.6207132,694.96212036)
\curveto(20.64071184,694.96210967)(20.66071182,694.96710966)(20.6807132,694.97712036)
}
}
{
\newrgbcolor{curcolor}{0 0 0}
\pscustom[linestyle=none,fillstyle=solid,fillcolor=curcolor]
{
\newpath
\moveto(34.13922882,692.21712036)
\curveto(34.88922432,692.23711239)(35.53922367,692.15211248)(36.08922882,691.96212036)
\curveto(36.64922256,691.78211285)(37.07422214,691.46711316)(37.36422882,691.01712036)
\curveto(37.43422178,690.90711372)(37.49422172,690.79211384)(37.54422882,690.67212036)
\curveto(37.60422161,690.56211407)(37.65422156,690.43711419)(37.69422882,690.29712036)
\curveto(37.7142215,690.23711439)(37.72422149,690.17211446)(37.72422882,690.10212036)
\curveto(37.72422149,690.0321146)(37.7142215,689.97211466)(37.69422882,689.92212036)
\curveto(37.65422156,689.86211477)(37.59922161,689.82211481)(37.52922882,689.80212036)
\curveto(37.47922173,689.78211485)(37.41922179,689.77211486)(37.34922882,689.77212036)
\lineto(37.13922882,689.77212036)
\lineto(36.47922882,689.77212036)
\curveto(36.4092228,689.77211486)(36.33922287,689.76711486)(36.26922882,689.75712036)
\curveto(36.19922301,689.75711487)(36.13422308,689.76711486)(36.07422882,689.78712036)
\curveto(35.97422324,689.80711482)(35.89922331,689.84711478)(35.84922882,689.90712036)
\curveto(35.79922341,689.96711466)(35.75422346,690.0271146)(35.71422882,690.08712036)
\lineto(35.59422882,690.29712036)
\curveto(35.56422365,690.37711425)(35.5142237,690.44211419)(35.44422882,690.49212036)
\curveto(35.34422387,690.57211406)(35.24422397,690.632114)(35.14422882,690.67212036)
\curveto(35.05422416,690.71211392)(34.93922427,690.74711388)(34.79922882,690.77712036)
\curveto(34.72922448,690.79711383)(34.62422459,690.81211382)(34.48422882,690.82212036)
\curveto(34.35422486,690.8321138)(34.25422496,690.8271138)(34.18422882,690.80712036)
\lineto(34.07922882,690.80712036)
\lineto(33.92922882,690.77712036)
\curveto(33.88922532,690.77711385)(33.84422537,690.77211386)(33.79422882,690.76212036)
\curveto(33.62422559,690.71211392)(33.48422573,690.64211399)(33.37422882,690.55212036)
\curveto(33.27422594,690.47211416)(33.20422601,690.34711428)(33.16422882,690.17712036)
\curveto(33.14422607,690.10711452)(33.14422607,690.04211459)(33.16422882,689.98212036)
\curveto(33.18422603,689.92211471)(33.20422601,689.87211476)(33.22422882,689.83212036)
\curveto(33.29422592,689.71211492)(33.37422584,689.61711501)(33.46422882,689.54712036)
\curveto(33.56422565,689.47711515)(33.67922553,689.41711521)(33.80922882,689.36712036)
\curveto(33.99922521,689.28711534)(34.20422501,689.21711541)(34.42422882,689.15712036)
\lineto(35.11422882,689.00712036)
\curveto(35.35422386,688.96711566)(35.58422363,688.91711571)(35.80422882,688.85712036)
\curveto(36.03422318,688.80711582)(36.24922296,688.74211589)(36.44922882,688.66212036)
\curveto(36.53922267,688.62211601)(36.62422259,688.58711604)(36.70422882,688.55712036)
\curveto(36.79422242,688.53711609)(36.87922233,688.50211613)(36.95922882,688.45212036)
\curveto(37.14922206,688.3321163)(37.31922189,688.20211643)(37.46922882,688.06212036)
\curveto(37.62922158,687.92211671)(37.75422146,687.74711688)(37.84422882,687.53712036)
\curveto(37.87422134,687.46711716)(37.89922131,687.39711723)(37.91922882,687.32712036)
\curveto(37.93922127,687.25711737)(37.95922125,687.18211745)(37.97922882,687.10212036)
\curveto(37.98922122,687.04211759)(37.99422122,686.94711768)(37.99422882,686.81712036)
\curveto(38.00422121,686.69711793)(38.00422121,686.60211803)(37.99422882,686.53212036)
\lineto(37.99422882,686.45712036)
\curveto(37.97422124,686.39711823)(37.95922125,686.33711829)(37.94922882,686.27712036)
\curveto(37.94922126,686.2271184)(37.94422127,686.17711845)(37.93422882,686.12712036)
\curveto(37.86422135,685.8271188)(37.75422146,685.56211907)(37.60422882,685.33212036)
\curveto(37.44422177,685.09211954)(37.24922196,684.89711973)(37.01922882,684.74712036)
\curveto(36.78922242,684.59712003)(36.52922268,684.46712016)(36.23922882,684.35712036)
\curveto(36.12922308,684.30712032)(36.0092232,684.27212036)(35.87922882,684.25212036)
\curveto(35.75922345,684.2321204)(35.63922357,684.20712042)(35.51922882,684.17712036)
\curveto(35.42922378,684.15712047)(35.33422388,684.14712048)(35.23422882,684.14712036)
\curveto(35.14422407,684.13712049)(35.05422416,684.12212051)(34.96422882,684.10212036)
\lineto(34.69422882,684.10212036)
\curveto(34.63422458,684.08212055)(34.52922468,684.07212056)(34.37922882,684.07212036)
\curveto(34.23922497,684.07212056)(34.13922507,684.08212055)(34.07922882,684.10212036)
\curveto(34.04922516,684.10212053)(34.0142252,684.10712052)(33.97422882,684.11712036)
\lineto(33.86922882,684.11712036)
\curveto(33.74922546,684.13712049)(33.62922558,684.15212048)(33.50922882,684.16212036)
\curveto(33.38922582,684.17212046)(33.27422594,684.19212044)(33.16422882,684.22212036)
\curveto(32.77422644,684.3321203)(32.42922678,684.45712017)(32.12922882,684.59712036)
\curveto(31.82922738,684.74711988)(31.57422764,684.96711966)(31.36422882,685.25712036)
\curveto(31.22422799,685.44711918)(31.10422811,685.66711896)(31.00422882,685.91712036)
\curveto(30.98422823,685.97711865)(30.96422825,686.05711857)(30.94422882,686.15712036)
\curveto(30.92422829,686.20711842)(30.9092283,686.27711835)(30.89922882,686.36712036)
\curveto(30.88922832,686.45711817)(30.89422832,686.5321181)(30.91422882,686.59212036)
\curveto(30.94422827,686.66211797)(30.99422822,686.71211792)(31.06422882,686.74212036)
\curveto(31.1142281,686.76211787)(31.17422804,686.77211786)(31.24422882,686.77212036)
\lineto(31.46922882,686.77212036)
\lineto(32.17422882,686.77212036)
\lineto(32.41422882,686.77212036)
\curveto(32.49422672,686.77211786)(32.56422665,686.76211787)(32.62422882,686.74212036)
\curveto(32.73422648,686.70211793)(32.80422641,686.63711799)(32.83422882,686.54712036)
\curveto(32.87422634,686.45711817)(32.91922629,686.36211827)(32.96922882,686.26212036)
\curveto(32.98922622,686.21211842)(33.02422619,686.14711848)(33.07422882,686.06712036)
\curveto(33.13422608,685.98711864)(33.18422603,685.93711869)(33.22422882,685.91712036)
\curveto(33.34422587,685.81711881)(33.45922575,685.73711889)(33.56922882,685.67712036)
\curveto(33.67922553,685.627119)(33.81922539,685.57711905)(33.98922882,685.52712036)
\curveto(34.03922517,685.50711912)(34.08922512,685.49711913)(34.13922882,685.49712036)
\curveto(34.18922502,685.50711912)(34.23922497,685.50711912)(34.28922882,685.49712036)
\curveto(34.36922484,685.47711915)(34.45422476,685.46711916)(34.54422882,685.46712036)
\curveto(34.64422457,685.47711915)(34.72922448,685.49211914)(34.79922882,685.51212036)
\curveto(34.84922436,685.52211911)(34.89422432,685.5271191)(34.93422882,685.52712036)
\curveto(34.98422423,685.5271191)(35.03422418,685.53711909)(35.08422882,685.55712036)
\curveto(35.22422399,685.60711902)(35.34922386,685.66711896)(35.45922882,685.73712036)
\curveto(35.57922363,685.80711882)(35.67422354,685.89711873)(35.74422882,686.00712036)
\curveto(35.79422342,686.08711854)(35.83422338,686.21211842)(35.86422882,686.38212036)
\curveto(35.88422333,686.45211818)(35.88422333,686.51711811)(35.86422882,686.57712036)
\curveto(35.84422337,686.63711799)(35.82422339,686.68711794)(35.80422882,686.72712036)
\curveto(35.73422348,686.86711776)(35.64422357,686.97211766)(35.53422882,687.04212036)
\curveto(35.43422378,687.11211752)(35.3142239,687.17711745)(35.17422882,687.23712036)
\curveto(34.98422423,687.31711731)(34.78422443,687.38211725)(34.57422882,687.43212036)
\curveto(34.36422485,687.48211715)(34.15422506,687.53711709)(33.94422882,687.59712036)
\curveto(33.86422535,687.61711701)(33.77922543,687.632117)(33.68922882,687.64212036)
\curveto(33.6092256,687.65211698)(33.52922568,687.66711696)(33.44922882,687.68712036)
\curveto(33.12922608,687.77711685)(32.82422639,687.86211677)(32.53422882,687.94212036)
\curveto(32.24422697,688.0321166)(31.97922723,688.16211647)(31.73922882,688.33212036)
\curveto(31.45922775,688.5321161)(31.25422796,688.80211583)(31.12422882,689.14212036)
\curveto(31.10422811,689.21211542)(31.08422813,689.30711532)(31.06422882,689.42712036)
\curveto(31.04422817,689.49711513)(31.02922818,689.58211505)(31.01922882,689.68212036)
\curveto(31.0092282,689.78211485)(31.0142282,689.87211476)(31.03422882,689.95212036)
\curveto(31.05422816,690.00211463)(31.05922815,690.04211459)(31.04922882,690.07212036)
\curveto(31.03922817,690.11211452)(31.04422817,690.15711447)(31.06422882,690.20712036)
\curveto(31.08422813,690.31711431)(31.10422811,690.41711421)(31.12422882,690.50712036)
\curveto(31.15422806,690.60711402)(31.18922802,690.70211393)(31.22922882,690.79212036)
\curveto(31.35922785,691.08211355)(31.53922767,691.31711331)(31.76922882,691.49712036)
\curveto(31.99922721,691.67711295)(32.25922695,691.82211281)(32.54922882,691.93212036)
\curveto(32.65922655,691.98211265)(32.77422644,692.01711261)(32.89422882,692.03712036)
\curveto(33.0142262,692.06711256)(33.13922607,692.09711253)(33.26922882,692.12712036)
\curveto(33.32922588,692.14711248)(33.38922582,692.15711247)(33.44922882,692.15712036)
\lineto(33.62922882,692.18712036)
\curveto(33.7092255,692.19711243)(33.79422542,692.20211243)(33.88422882,692.20212036)
\curveto(33.97422524,692.20211243)(34.05922515,692.20711242)(34.13922882,692.21712036)
}
}
{
\newrgbcolor{curcolor}{0 0 0}
\pscustom[linestyle=none,fillstyle=solid,fillcolor=curcolor]
{
\newpath
\moveto(39.64586945,691.99212036)
\lineto(40.77086945,691.99212036)
\curveto(40.88086701,691.99211264)(40.98086691,691.98711264)(41.07086945,691.97712036)
\curveto(41.16086673,691.96711266)(41.22586667,691.9321127)(41.26586945,691.87212036)
\curveto(41.31586658,691.81211282)(41.34586655,691.7271129)(41.35586945,691.61712036)
\curveto(41.36586653,691.51711311)(41.37086652,691.41211322)(41.37086945,691.30212036)
\lineto(41.37086945,690.25212036)
\lineto(41.37086945,688.01712036)
\curveto(41.37086652,687.65711697)(41.38586651,687.31711731)(41.41586945,686.99712036)
\curveto(41.44586645,686.67711795)(41.53586636,686.41211822)(41.68586945,686.20212036)
\curveto(41.82586607,685.99211864)(42.05086584,685.84211879)(42.36086945,685.75212036)
\curveto(42.41086548,685.74211889)(42.45086544,685.73711889)(42.48086945,685.73712036)
\curveto(42.52086537,685.73711889)(42.56586533,685.7321189)(42.61586945,685.72212036)
\curveto(42.66586523,685.71211892)(42.72086517,685.70711892)(42.78086945,685.70712036)
\curveto(42.84086505,685.70711892)(42.88586501,685.71211892)(42.91586945,685.72212036)
\curveto(42.96586493,685.74211889)(43.00586489,685.74711888)(43.03586945,685.73712036)
\curveto(43.07586482,685.7271189)(43.11586478,685.7321189)(43.15586945,685.75212036)
\curveto(43.36586453,685.80211883)(43.53086436,685.86711876)(43.65086945,685.94712036)
\curveto(43.83086406,686.05711857)(43.97086392,686.19711843)(44.07086945,686.36712036)
\curveto(44.18086371,686.54711808)(44.25586364,686.74211789)(44.29586945,686.95212036)
\curveto(44.34586355,687.17211746)(44.37586352,687.41211722)(44.38586945,687.67212036)
\curveto(44.3958635,687.94211669)(44.40086349,688.22211641)(44.40086945,688.51212036)
\lineto(44.40086945,690.32712036)
\lineto(44.40086945,691.30212036)
\lineto(44.40086945,691.57212036)
\curveto(44.40086349,691.67211296)(44.42086347,691.75211288)(44.46086945,691.81212036)
\curveto(44.51086338,691.90211273)(44.58586331,691.95211268)(44.68586945,691.96212036)
\curveto(44.78586311,691.98211265)(44.90586299,691.99211264)(45.04586945,691.99212036)
\lineto(45.84086945,691.99212036)
\lineto(46.12586945,691.99212036)
\curveto(46.21586168,691.99211264)(46.2908616,691.97211266)(46.35086945,691.93212036)
\curveto(46.43086146,691.88211275)(46.47586142,691.80711282)(46.48586945,691.70712036)
\curveto(46.4958614,691.60711302)(46.50086139,691.49211314)(46.50086945,691.36212036)
\lineto(46.50086945,690.22212036)
\lineto(46.50086945,686.00712036)
\lineto(46.50086945,684.94212036)
\lineto(46.50086945,684.64212036)
\curveto(46.50086139,684.54212009)(46.48086141,684.46712016)(46.44086945,684.41712036)
\curveto(46.3908615,684.33712029)(46.31586158,684.29212034)(46.21586945,684.28212036)
\curveto(46.11586178,684.27212036)(46.01086188,684.26712036)(45.90086945,684.26712036)
\lineto(45.09086945,684.26712036)
\curveto(44.98086291,684.26712036)(44.88086301,684.27212036)(44.79086945,684.28212036)
\curveto(44.71086318,684.29212034)(44.64586325,684.3321203)(44.59586945,684.40212036)
\curveto(44.57586332,684.4321202)(44.55586334,684.47712015)(44.53586945,684.53712036)
\curveto(44.52586337,684.59712003)(44.51086338,684.65711997)(44.49086945,684.71712036)
\curveto(44.48086341,684.77711985)(44.46586343,684.8321198)(44.44586945,684.88212036)
\curveto(44.42586347,684.9321197)(44.3958635,684.96211967)(44.35586945,684.97212036)
\curveto(44.33586356,684.99211964)(44.31086358,684.99711963)(44.28086945,684.98712036)
\curveto(44.25086364,684.97711965)(44.22586367,684.96711966)(44.20586945,684.95712036)
\curveto(44.13586376,684.91711971)(44.07586382,684.87211976)(44.02586945,684.82212036)
\curveto(43.97586392,684.77211986)(43.92086397,684.7271199)(43.86086945,684.68712036)
\curveto(43.82086407,684.65711997)(43.78086411,684.62212001)(43.74086945,684.58212036)
\curveto(43.71086418,684.55212008)(43.67086422,684.52212011)(43.62086945,684.49212036)
\curveto(43.3908645,684.35212028)(43.12086477,684.24212039)(42.81086945,684.16212036)
\curveto(42.74086515,684.14212049)(42.67086522,684.1321205)(42.60086945,684.13212036)
\curveto(42.53086536,684.12212051)(42.45586544,684.10712052)(42.37586945,684.08712036)
\curveto(42.33586556,684.07712055)(42.2908656,684.07712055)(42.24086945,684.08712036)
\curveto(42.20086569,684.08712054)(42.16086573,684.08212055)(42.12086945,684.07212036)
\curveto(42.0908658,684.06212057)(42.02586587,684.06212057)(41.92586945,684.07212036)
\curveto(41.83586606,684.07212056)(41.77586612,684.07712055)(41.74586945,684.08712036)
\curveto(41.6958662,684.08712054)(41.64586625,684.09212054)(41.59586945,684.10212036)
\lineto(41.44586945,684.10212036)
\curveto(41.32586657,684.1321205)(41.21086668,684.15712047)(41.10086945,684.17712036)
\curveto(40.9908669,684.19712043)(40.88086701,684.2271204)(40.77086945,684.26712036)
\curveto(40.72086717,684.28712034)(40.67586722,684.30212033)(40.63586945,684.31212036)
\curveto(40.60586729,684.3321203)(40.56586733,684.35212028)(40.51586945,684.37212036)
\curveto(40.16586773,684.56212007)(39.88586801,684.8271198)(39.67586945,685.16712036)
\curveto(39.54586835,685.37711925)(39.45086844,685.627119)(39.39086945,685.91712036)
\curveto(39.33086856,686.21711841)(39.2908686,686.5321181)(39.27086945,686.86212036)
\curveto(39.26086863,687.20211743)(39.25586864,687.54711708)(39.25586945,687.89712036)
\curveto(39.26586863,688.25711637)(39.27086862,688.61211602)(39.27086945,688.96212036)
\lineto(39.27086945,691.00212036)
\curveto(39.27086862,691.1321135)(39.26586863,691.28211335)(39.25586945,691.45212036)
\curveto(39.25586864,691.632113)(39.28086861,691.76211287)(39.33086945,691.84212036)
\curveto(39.36086853,691.89211274)(39.42086847,691.93711269)(39.51086945,691.97712036)
\curveto(39.57086832,691.97711265)(39.61586828,691.98211265)(39.64586945,691.99212036)
}
}
{
\newrgbcolor{curcolor}{0 0 0}
\pscustom[linestyle=none,fillstyle=solid,fillcolor=curcolor]
{
\newpath
\moveto(55.18211945,684.86712036)
\curveto(55.2021116,684.75711987)(55.21211159,684.64711998)(55.21211945,684.53712036)
\curveto(55.22211158,684.4271202)(55.17211163,684.35212028)(55.06211945,684.31212036)
\curveto(55.0021118,684.28212035)(54.93211187,684.26712036)(54.85211945,684.26712036)
\lineto(54.61211945,684.26712036)
\lineto(53.80211945,684.26712036)
\lineto(53.53211945,684.26712036)
\curveto(53.45211335,684.27712035)(53.38711341,684.30212033)(53.33711945,684.34212036)
\curveto(53.26711353,684.38212025)(53.21211359,684.43712019)(53.17211945,684.50712036)
\curveto(53.14211366,684.58712004)(53.0971137,684.65211998)(53.03711945,684.70212036)
\curveto(53.01711378,684.72211991)(52.99211381,684.73711989)(52.96211945,684.74712036)
\curveto(52.93211387,684.76711986)(52.89211391,684.77211986)(52.84211945,684.76212036)
\curveto(52.79211401,684.74211989)(52.74211406,684.71711991)(52.69211945,684.68712036)
\curveto(52.65211415,684.65711997)(52.60711419,684.63212)(52.55711945,684.61212036)
\curveto(52.50711429,684.57212006)(52.45211435,684.53712009)(52.39211945,684.50712036)
\lineto(52.21211945,684.41712036)
\curveto(52.08211472,684.35712027)(51.94711485,684.30712032)(51.80711945,684.26712036)
\curveto(51.66711513,684.23712039)(51.52211528,684.20212043)(51.37211945,684.16212036)
\curveto(51.3021155,684.14212049)(51.23211557,684.1321205)(51.16211945,684.13212036)
\curveto(51.1021157,684.12212051)(51.03711576,684.11212052)(50.96711945,684.10212036)
\lineto(50.87711945,684.10212036)
\curveto(50.84711595,684.09212054)(50.81711598,684.08712054)(50.78711945,684.08712036)
\lineto(50.62211945,684.08712036)
\curveto(50.52211628,684.06712056)(50.42211638,684.06712056)(50.32211945,684.08712036)
\lineto(50.18711945,684.08712036)
\curveto(50.11711668,684.10712052)(50.04711675,684.11712051)(49.97711945,684.11712036)
\curveto(49.91711688,684.10712052)(49.85711694,684.11212052)(49.79711945,684.13212036)
\curveto(49.6971171,684.15212048)(49.6021172,684.17212046)(49.51211945,684.19212036)
\curveto(49.42211738,684.20212043)(49.33711746,684.2271204)(49.25711945,684.26712036)
\curveto(48.96711783,684.37712025)(48.71711808,684.51712011)(48.50711945,684.68712036)
\curveto(48.30711849,684.86711976)(48.14711865,685.10211953)(48.02711945,685.39212036)
\curveto(47.9971188,685.46211917)(47.96711883,685.53711909)(47.93711945,685.61712036)
\curveto(47.91711888,685.69711893)(47.8971189,685.78211885)(47.87711945,685.87212036)
\curveto(47.85711894,685.92211871)(47.84711895,685.97211866)(47.84711945,686.02212036)
\curveto(47.85711894,686.07211856)(47.85711894,686.12211851)(47.84711945,686.17212036)
\curveto(47.83711896,686.20211843)(47.82711897,686.26211837)(47.81711945,686.35212036)
\curveto(47.81711898,686.45211818)(47.82211898,686.52211811)(47.83211945,686.56212036)
\curveto(47.85211895,686.66211797)(47.86211894,686.74711788)(47.86211945,686.81712036)
\lineto(47.95211945,687.14712036)
\curveto(47.98211882,687.26711736)(48.02211878,687.37211726)(48.07211945,687.46212036)
\curveto(48.24211856,687.75211688)(48.43711836,687.97211666)(48.65711945,688.12212036)
\curveto(48.87711792,688.27211636)(49.15711764,688.40211623)(49.49711945,688.51212036)
\curveto(49.62711717,688.56211607)(49.76211704,688.59711603)(49.90211945,688.61712036)
\curveto(50.04211676,688.63711599)(50.18211662,688.66211597)(50.32211945,688.69212036)
\curveto(50.4021164,688.71211592)(50.48711631,688.72211591)(50.57711945,688.72212036)
\curveto(50.66711613,688.7321159)(50.75711604,688.74711588)(50.84711945,688.76712036)
\curveto(50.91711588,688.78711584)(50.98711581,688.79211584)(51.05711945,688.78212036)
\curveto(51.12711567,688.78211585)(51.2021156,688.79211584)(51.28211945,688.81212036)
\curveto(51.35211545,688.8321158)(51.42211538,688.84211579)(51.49211945,688.84212036)
\curveto(51.56211524,688.84211579)(51.63711516,688.85211578)(51.71711945,688.87212036)
\curveto(51.92711487,688.92211571)(52.11711468,688.96211567)(52.28711945,688.99212036)
\curveto(52.46711433,689.0321156)(52.62711417,689.12211551)(52.76711945,689.26212036)
\curveto(52.85711394,689.35211528)(52.91711388,689.45211518)(52.94711945,689.56212036)
\curveto(52.95711384,689.59211504)(52.95711384,689.61711501)(52.94711945,689.63712036)
\curveto(52.94711385,689.65711497)(52.95211385,689.67711495)(52.96211945,689.69712036)
\curveto(52.97211383,689.71711491)(52.97711382,689.74711488)(52.97711945,689.78712036)
\lineto(52.97711945,689.87712036)
\lineto(52.94711945,689.99712036)
\curveto(52.94711385,690.03711459)(52.94211386,690.07211456)(52.93211945,690.10212036)
\curveto(52.83211397,690.40211423)(52.62211418,690.60711402)(52.30211945,690.71712036)
\curveto(52.21211459,690.74711388)(52.1021147,690.76711386)(51.97211945,690.77712036)
\curveto(51.85211495,690.79711383)(51.72711507,690.80211383)(51.59711945,690.79212036)
\curveto(51.46711533,690.79211384)(51.34211546,690.78211385)(51.22211945,690.76212036)
\curveto(51.1021157,690.74211389)(50.9971158,690.71711391)(50.90711945,690.68712036)
\curveto(50.84711595,690.66711396)(50.78711601,690.63711399)(50.72711945,690.59712036)
\curveto(50.67711612,690.56711406)(50.62711617,690.5321141)(50.57711945,690.49212036)
\curveto(50.52711627,690.45211418)(50.47211633,690.39711423)(50.41211945,690.32712036)
\curveto(50.36211644,690.25711437)(50.32711647,690.19211444)(50.30711945,690.13212036)
\curveto(50.25711654,690.0321146)(50.21211659,689.93711469)(50.17211945,689.84712036)
\curveto(50.14211666,689.75711487)(50.07211673,689.69711493)(49.96211945,689.66712036)
\curveto(49.88211692,689.64711498)(49.797117,689.63711499)(49.70711945,689.63712036)
\lineto(49.43711945,689.63712036)
\lineto(48.86711945,689.63712036)
\curveto(48.81711798,689.63711499)(48.76711803,689.632115)(48.71711945,689.62212036)
\curveto(48.66711813,689.62211501)(48.62211818,689.627115)(48.58211945,689.63712036)
\lineto(48.44711945,689.63712036)
\curveto(48.42711837,689.64711498)(48.4021184,689.65211498)(48.37211945,689.65212036)
\curveto(48.34211846,689.65211498)(48.31711848,689.66211497)(48.29711945,689.68212036)
\curveto(48.21711858,689.70211493)(48.16211864,689.76711486)(48.13211945,689.87712036)
\curveto(48.12211868,689.9271147)(48.12211868,689.97711465)(48.13211945,690.02712036)
\curveto(48.14211866,690.07711455)(48.15211865,690.12211451)(48.16211945,690.16212036)
\curveto(48.19211861,690.27211436)(48.22211858,690.37211426)(48.25211945,690.46212036)
\curveto(48.29211851,690.56211407)(48.33711846,690.65211398)(48.38711945,690.73212036)
\lineto(48.47711945,690.88212036)
\lineto(48.56711945,691.03212036)
\curveto(48.64711815,691.14211349)(48.74711805,691.24711338)(48.86711945,691.34712036)
\curveto(48.88711791,691.35711327)(48.91711788,691.38211325)(48.95711945,691.42212036)
\curveto(49.00711779,691.46211317)(49.05211775,691.49711313)(49.09211945,691.52712036)
\curveto(49.13211767,691.55711307)(49.17711762,691.58711304)(49.22711945,691.61712036)
\curveto(49.3971174,691.7271129)(49.57711722,691.81211282)(49.76711945,691.87212036)
\curveto(49.95711684,691.94211269)(50.15211665,692.00711262)(50.35211945,692.06712036)
\curveto(50.47211633,692.09711253)(50.5971162,692.11711251)(50.72711945,692.12712036)
\curveto(50.85711594,692.13711249)(50.98711581,692.15711247)(51.11711945,692.18712036)
\curveto(51.15711564,692.19711243)(51.21711558,692.19711243)(51.29711945,692.18712036)
\curveto(51.38711541,692.17711245)(51.44211536,692.18211245)(51.46211945,692.20212036)
\curveto(51.87211493,692.21211242)(52.26211454,692.19711243)(52.63211945,692.15712036)
\curveto(53.01211379,692.11711251)(53.35211345,692.04211259)(53.65211945,691.93212036)
\curveto(53.96211284,691.82211281)(54.22711257,691.67211296)(54.44711945,691.48212036)
\curveto(54.66711213,691.30211333)(54.83711196,691.06711356)(54.95711945,690.77712036)
\curveto(55.02711177,690.60711402)(55.06711173,690.41211422)(55.07711945,690.19212036)
\curveto(55.08711171,689.97211466)(55.09211171,689.74711488)(55.09211945,689.51712036)
\lineto(55.09211945,686.17212036)
\lineto(55.09211945,685.58712036)
\curveto(55.09211171,685.39711923)(55.11211169,685.22211941)(55.15211945,685.06212036)
\curveto(55.16211164,685.0321196)(55.16711163,684.99711963)(55.16711945,684.95712036)
\curveto(55.16711163,684.9271197)(55.17211163,684.89711973)(55.18211945,684.86712036)
\moveto(52.97711945,687.17712036)
\curveto(52.98711381,687.2271174)(52.99211381,687.28211735)(52.99211945,687.34212036)
\curveto(52.99211381,687.41211722)(52.98711381,687.47211716)(52.97711945,687.52212036)
\curveto(52.95711384,687.58211705)(52.94711385,687.63711699)(52.94711945,687.68712036)
\curveto(52.94711385,687.73711689)(52.92711387,687.77711685)(52.88711945,687.80712036)
\curveto(52.83711396,687.84711678)(52.76211404,687.86711676)(52.66211945,687.86712036)
\curveto(52.62211418,687.85711677)(52.58711421,687.84711678)(52.55711945,687.83712036)
\curveto(52.52711427,687.83711679)(52.49211431,687.8321168)(52.45211945,687.82212036)
\curveto(52.38211442,687.80211683)(52.30711449,687.78711684)(52.22711945,687.77712036)
\curveto(52.14711465,687.76711686)(52.06711473,687.75211688)(51.98711945,687.73212036)
\curveto(51.95711484,687.72211691)(51.91211489,687.71711691)(51.85211945,687.71712036)
\curveto(51.72211508,687.68711694)(51.59211521,687.66711696)(51.46211945,687.65712036)
\curveto(51.33211547,687.64711698)(51.20711559,687.62211701)(51.08711945,687.58212036)
\curveto(51.00711579,687.56211707)(50.93211587,687.54211709)(50.86211945,687.52212036)
\curveto(50.79211601,687.51211712)(50.72211608,687.49211714)(50.65211945,687.46212036)
\curveto(50.44211636,687.37211726)(50.26211654,687.23711739)(50.11211945,687.05712036)
\curveto(49.97211683,686.87711775)(49.92211688,686.627118)(49.96211945,686.30712036)
\curveto(49.98211682,686.13711849)(50.03711676,685.99711863)(50.12711945,685.88712036)
\curveto(50.1971166,685.77711885)(50.3021165,685.68711894)(50.44211945,685.61712036)
\curveto(50.58211622,685.55711907)(50.73211607,685.51211912)(50.89211945,685.48212036)
\curveto(51.06211574,685.45211918)(51.23711556,685.44211919)(51.41711945,685.45212036)
\curveto(51.60711519,685.47211916)(51.78211502,685.50711912)(51.94211945,685.55712036)
\curveto(52.2021146,685.63711899)(52.40711439,685.76211887)(52.55711945,685.93212036)
\curveto(52.70711409,686.11211852)(52.82211398,686.3321183)(52.90211945,686.59212036)
\curveto(52.92211388,686.66211797)(52.93211387,686.7321179)(52.93211945,686.80212036)
\curveto(52.94211386,686.88211775)(52.95711384,686.96211767)(52.97711945,687.04212036)
\lineto(52.97711945,687.17712036)
}
}
{
\newrgbcolor{curcolor}{0 0 0}
\pscustom[linestyle=none,fillstyle=solid,fillcolor=curcolor]
{
\newpath
\moveto(61.1704007,692.20212036)
\curveto(61.28039538,692.20211243)(61.37539529,692.19211244)(61.4554007,692.17212036)
\curveto(61.54539512,692.15211248)(61.61539505,692.10711252)(61.6654007,692.03712036)
\curveto(61.72539494,691.95711267)(61.75539491,691.81711281)(61.7554007,691.61712036)
\lineto(61.7554007,691.10712036)
\lineto(61.7554007,690.73212036)
\curveto(61.7653949,690.59211404)(61.75039491,690.48211415)(61.7104007,690.40212036)
\curveto(61.67039499,690.3321143)(61.61039505,690.28711434)(61.5304007,690.26712036)
\curveto(61.4603952,690.24711438)(61.37539529,690.23711439)(61.2754007,690.23712036)
\curveto(61.18539548,690.23711439)(61.08539558,690.24211439)(60.9754007,690.25212036)
\curveto(60.87539579,690.26211437)(60.78039588,690.25711437)(60.6904007,690.23712036)
\curveto(60.62039604,690.21711441)(60.55039611,690.20211443)(60.4804007,690.19212036)
\curveto(60.41039625,690.19211444)(60.34539632,690.18211445)(60.2854007,690.16212036)
\curveto(60.12539654,690.11211452)(59.9653967,690.03711459)(59.8054007,689.93712036)
\curveto(59.64539702,689.84711478)(59.52039714,689.74211489)(59.4304007,689.62212036)
\curveto(59.38039728,689.54211509)(59.32539734,689.45711517)(59.2654007,689.36712036)
\curveto(59.21539745,689.28711534)(59.1653975,689.20211543)(59.1154007,689.11212036)
\curveto(59.08539758,689.0321156)(59.05539761,688.94711568)(59.0254007,688.85712036)
\lineto(58.9654007,688.61712036)
\curveto(58.94539772,688.54711608)(58.93539773,688.47211616)(58.9354007,688.39212036)
\curveto(58.93539773,688.32211631)(58.92539774,688.25211638)(58.9054007,688.18212036)
\curveto(58.89539777,688.14211649)(58.89039777,688.10211653)(58.8904007,688.06212036)
\curveto(58.90039776,688.0321166)(58.90039776,688.00211663)(58.8904007,687.97212036)
\lineto(58.8904007,687.73212036)
\curveto(58.87039779,687.66211697)(58.8653978,687.58211705)(58.8754007,687.49212036)
\curveto(58.88539778,687.41211722)(58.89039777,687.3321173)(58.8904007,687.25212036)
\lineto(58.8904007,686.29212036)
\lineto(58.8904007,685.01712036)
\curveto(58.89039777,684.88711974)(58.88539778,684.76711986)(58.8754007,684.65712036)
\curveto(58.8653978,684.54712008)(58.83539783,684.45712017)(58.7854007,684.38712036)
\curveto(58.7653979,684.35712027)(58.73039793,684.3321203)(58.6804007,684.31212036)
\curveto(58.64039802,684.30212033)(58.59539807,684.29212034)(58.5454007,684.28212036)
\lineto(58.4704007,684.28212036)
\curveto(58.42039824,684.27212036)(58.3653983,684.26712036)(58.3054007,684.26712036)
\lineto(58.1404007,684.26712036)
\lineto(57.4954007,684.26712036)
\curveto(57.43539923,684.27712035)(57.37039929,684.28212035)(57.3004007,684.28212036)
\lineto(57.1054007,684.28212036)
\curveto(57.05539961,684.30212033)(57.00539966,684.31712031)(56.9554007,684.32712036)
\curveto(56.90539976,684.34712028)(56.87039979,684.38212025)(56.8504007,684.43212036)
\curveto(56.81039985,684.48212015)(56.78539988,684.55212008)(56.7754007,684.64212036)
\lineto(56.7754007,684.94212036)
\lineto(56.7754007,685.96212036)
\lineto(56.7754007,690.19212036)
\lineto(56.7754007,691.30212036)
\lineto(56.7754007,691.58712036)
\curveto(56.77539989,691.68711294)(56.79539987,691.76711286)(56.8354007,691.82712036)
\curveto(56.88539978,691.90711272)(56.9603997,691.95711267)(57.0604007,691.97712036)
\curveto(57.1603995,691.99711263)(57.28039938,692.00711262)(57.4204007,692.00712036)
\lineto(58.1854007,692.00712036)
\curveto(58.30539836,692.00711262)(58.41039825,691.99711263)(58.5004007,691.97712036)
\curveto(58.59039807,691.96711266)(58.660398,691.92211271)(58.7104007,691.84212036)
\curveto(58.74039792,691.79211284)(58.75539791,691.72211291)(58.7554007,691.63212036)
\lineto(58.7854007,691.36212036)
\curveto(58.79539787,691.28211335)(58.81039785,691.20711342)(58.8304007,691.13712036)
\curveto(58.8603978,691.06711356)(58.91039775,691.0321136)(58.9804007,691.03212036)
\curveto(59.00039766,691.05211358)(59.02039764,691.06211357)(59.0404007,691.06212036)
\curveto(59.0603976,691.06211357)(59.08039758,691.07211356)(59.1004007,691.09212036)
\curveto(59.1603975,691.14211349)(59.21039745,691.19711343)(59.2504007,691.25712036)
\curveto(59.30039736,691.3271133)(59.3603973,691.38711324)(59.4304007,691.43712036)
\curveto(59.47039719,691.46711316)(59.50539716,691.49711313)(59.5354007,691.52712036)
\curveto(59.5653971,691.56711306)(59.60039706,691.60211303)(59.6404007,691.63212036)
\lineto(59.9104007,691.81212036)
\curveto(60.01039665,691.87211276)(60.11039655,691.9271127)(60.2104007,691.97712036)
\curveto(60.31039635,692.01711261)(60.41039625,692.05211258)(60.5104007,692.08212036)
\lineto(60.8404007,692.17212036)
\curveto(60.87039579,692.18211245)(60.92539574,692.18211245)(61.0054007,692.17212036)
\curveto(61.09539557,692.17211246)(61.15039551,692.18211245)(61.1704007,692.20212036)
}
}
{
\newrgbcolor{curcolor}{0 0 0}
\pscustom[linestyle=none,fillstyle=solid,fillcolor=curcolor]
{
\newpath
\moveto(64.67547882,694.85712036)
\curveto(64.74547587,694.77710985)(64.78047584,694.65710997)(64.78047882,694.49712036)
\lineto(64.78047882,694.03212036)
\lineto(64.78047882,693.62712036)
\curveto(64.78047584,693.48711114)(64.74547587,693.39211124)(64.67547882,693.34212036)
\curveto(64.615476,693.29211134)(64.53547608,693.26211137)(64.43547882,693.25212036)
\curveto(64.34547627,693.24211139)(64.24547637,693.23711139)(64.13547882,693.23712036)
\lineto(63.29547882,693.23712036)
\curveto(63.18547743,693.23711139)(63.08547753,693.24211139)(62.99547882,693.25212036)
\curveto(62.9154777,693.26211137)(62.84547777,693.29211134)(62.78547882,693.34212036)
\curveto(62.74547787,693.37211126)(62.7154779,693.4271112)(62.69547882,693.50712036)
\curveto(62.68547793,693.59711103)(62.67547794,693.69211094)(62.66547882,693.79212036)
\lineto(62.66547882,694.12212036)
\curveto(62.67547794,694.2321104)(62.68047794,694.3271103)(62.68047882,694.40712036)
\lineto(62.68047882,694.61712036)
\curveto(62.69047793,694.68710994)(62.71047791,694.74710988)(62.74047882,694.79712036)
\curveto(62.76047786,694.83710979)(62.78547783,694.86710976)(62.81547882,694.88712036)
\lineto(62.93547882,694.94712036)
\curveto(62.95547766,694.94710968)(62.98047764,694.94710968)(63.01047882,694.94712036)
\curveto(63.04047758,694.95710967)(63.06547755,694.96210967)(63.08547882,694.96212036)
\lineto(64.18047882,694.96212036)
\curveto(64.28047634,694.96210967)(64.37547624,694.95710967)(64.46547882,694.94712036)
\curveto(64.55547606,694.93710969)(64.62547599,694.90710972)(64.67547882,694.85712036)
\moveto(64.78047882,685.09212036)
\curveto(64.78047584,684.89211974)(64.77547584,684.72211991)(64.76547882,684.58212036)
\curveto(64.75547586,684.44212019)(64.66547595,684.34712028)(64.49547882,684.29712036)
\curveto(64.43547618,684.27712035)(64.37047625,684.26712036)(64.30047882,684.26712036)
\curveto(64.23047639,684.27712035)(64.15547646,684.28212035)(64.07547882,684.28212036)
\lineto(63.23547882,684.28212036)
\curveto(63.14547747,684.28212035)(63.05547756,684.28712034)(62.96547882,684.29712036)
\curveto(62.88547773,684.30712032)(62.82547779,684.33712029)(62.78547882,684.38712036)
\curveto(62.72547789,684.45712017)(62.69047793,684.54212009)(62.68047882,684.64212036)
\lineto(62.68047882,684.98712036)
\lineto(62.68047882,691.31712036)
\lineto(62.68047882,691.61712036)
\curveto(62.68047794,691.71711291)(62.70047792,691.79711283)(62.74047882,691.85712036)
\curveto(62.80047782,691.9271127)(62.88547773,691.97211266)(62.99547882,691.99212036)
\curveto(63.0154776,692.00211263)(63.04047758,692.00211263)(63.07047882,691.99212036)
\curveto(63.11047751,691.99211264)(63.14047748,691.99711263)(63.16047882,692.00712036)
\lineto(63.91047882,692.00712036)
\lineto(64.10547882,692.00712036)
\curveto(64.18547643,692.01711261)(64.25047637,692.01711261)(64.30047882,692.00712036)
\lineto(64.42047882,692.00712036)
\curveto(64.48047614,691.98711264)(64.53547608,691.97211266)(64.58547882,691.96212036)
\curveto(64.63547598,691.95211268)(64.67547594,691.92211271)(64.70547882,691.87212036)
\curveto(64.74547587,691.82211281)(64.76547585,691.75211288)(64.76547882,691.66212036)
\curveto(64.77547584,691.57211306)(64.78047584,691.47711315)(64.78047882,691.37712036)
\lineto(64.78047882,685.09212036)
}
}
{
\newrgbcolor{curcolor}{0 0 0}
\pscustom[linestyle=none,fillstyle=solid,fillcolor=curcolor]
{
\newpath
\moveto(74.21266632,688.45212036)
\curveto(74.23265775,688.39211624)(74.24265774,688.30711632)(74.24266632,688.19712036)
\curveto(74.24265774,688.08711654)(74.23265775,688.00211663)(74.21266632,687.94212036)
\lineto(74.21266632,687.79212036)
\curveto(74.19265779,687.71211692)(74.1826578,687.632117)(74.18266632,687.55212036)
\curveto(74.19265779,687.47211716)(74.1876578,687.39211724)(74.16766632,687.31212036)
\curveto(74.14765784,687.24211739)(74.13265785,687.17711745)(74.12266632,687.11712036)
\curveto(74.11265787,687.05711757)(74.10265788,686.99211764)(74.09266632,686.92212036)
\curveto(74.05265793,686.81211782)(74.01765797,686.69711793)(73.98766632,686.57712036)
\curveto(73.95765803,686.46711816)(73.91765807,686.36211827)(73.86766632,686.26212036)
\curveto(73.65765833,685.78211885)(73.3826586,685.39211924)(73.04266632,685.09212036)
\curveto(72.70265928,684.79211984)(72.29265969,684.54212009)(71.81266632,684.34212036)
\curveto(71.69266029,684.29212034)(71.56766042,684.25712037)(71.43766632,684.23712036)
\curveto(71.31766067,684.20712042)(71.19266079,684.17712045)(71.06266632,684.14712036)
\curveto(71.01266097,684.1271205)(70.95766103,684.11712051)(70.89766632,684.11712036)
\curveto(70.83766115,684.11712051)(70.7826612,684.11212052)(70.73266632,684.10212036)
\lineto(70.62766632,684.10212036)
\curveto(70.59766139,684.09212054)(70.56766142,684.08712054)(70.53766632,684.08712036)
\curveto(70.4876615,684.07712055)(70.40766158,684.07212056)(70.29766632,684.07212036)
\curveto(70.1876618,684.06212057)(70.10266188,684.06712056)(70.04266632,684.08712036)
\lineto(69.89266632,684.08712036)
\curveto(69.84266214,684.09712053)(69.7876622,684.10212053)(69.72766632,684.10212036)
\curveto(69.67766231,684.09212054)(69.62766236,684.09712053)(69.57766632,684.11712036)
\curveto(69.53766245,684.1271205)(69.49766249,684.1321205)(69.45766632,684.13212036)
\curveto(69.42766256,684.1321205)(69.3876626,684.13712049)(69.33766632,684.14712036)
\curveto(69.23766275,684.17712045)(69.13766285,684.20212043)(69.03766632,684.22212036)
\curveto(68.93766305,684.24212039)(68.84266314,684.27212036)(68.75266632,684.31212036)
\curveto(68.63266335,684.35212028)(68.51766347,684.39212024)(68.40766632,684.43212036)
\curveto(68.30766368,684.47212016)(68.20266378,684.52212011)(68.09266632,684.58212036)
\curveto(67.74266424,684.79211984)(67.44266454,685.03711959)(67.19266632,685.31712036)
\curveto(66.94266504,685.59711903)(66.73266525,685.9321187)(66.56266632,686.32212036)
\curveto(66.51266547,686.41211822)(66.47266551,686.50711812)(66.44266632,686.60712036)
\curveto(66.42266556,686.70711792)(66.39766559,686.81211782)(66.36766632,686.92212036)
\curveto(66.34766564,686.97211766)(66.33766565,687.01711761)(66.33766632,687.05712036)
\curveto(66.33766565,687.09711753)(66.32766566,687.14211749)(66.30766632,687.19212036)
\curveto(66.2876657,687.27211736)(66.27766571,687.35211728)(66.27766632,687.43212036)
\curveto(66.27766571,687.52211711)(66.26766572,687.60711702)(66.24766632,687.68712036)
\curveto(66.23766575,687.73711689)(66.23266575,687.78211685)(66.23266632,687.82212036)
\lineto(66.23266632,687.95712036)
\curveto(66.21266577,688.01711661)(66.20266578,688.10211653)(66.20266632,688.21212036)
\curveto(66.21266577,688.32211631)(66.22766576,688.40711622)(66.24766632,688.46712036)
\lineto(66.24766632,688.57212036)
\curveto(66.25766573,688.62211601)(66.25766573,688.67211596)(66.24766632,688.72212036)
\curveto(66.24766574,688.78211585)(66.25766573,688.83711579)(66.27766632,688.88712036)
\curveto(66.2876657,688.93711569)(66.29266569,688.98211565)(66.29266632,689.02212036)
\curveto(66.29266569,689.07211556)(66.30266568,689.12211551)(66.32266632,689.17212036)
\curveto(66.36266562,689.30211533)(66.39766559,689.4271152)(66.42766632,689.54712036)
\curveto(66.45766553,689.67711495)(66.49766549,689.80211483)(66.54766632,689.92212036)
\curveto(66.72766526,690.3321143)(66.94266504,690.67211396)(67.19266632,690.94212036)
\curveto(67.44266454,691.22211341)(67.74766424,691.47711315)(68.10766632,691.70712036)
\curveto(68.20766378,691.75711287)(68.31266367,691.80211283)(68.42266632,691.84212036)
\curveto(68.53266345,691.88211275)(68.64266334,691.9271127)(68.75266632,691.97712036)
\curveto(68.8826631,692.0271126)(69.01766297,692.06211257)(69.15766632,692.08212036)
\curveto(69.29766269,692.10211253)(69.44266254,692.1321125)(69.59266632,692.17212036)
\curveto(69.67266231,692.18211245)(69.74766224,692.18711244)(69.81766632,692.18712036)
\curveto(69.8876621,692.18711244)(69.95766203,692.19211244)(70.02766632,692.20212036)
\curveto(70.60766138,692.21211242)(71.10766088,692.15211248)(71.52766632,692.02212036)
\curveto(71.95766003,691.89211274)(72.33765965,691.71211292)(72.66766632,691.48212036)
\curveto(72.77765921,691.40211323)(72.8876591,691.31211332)(72.99766632,691.21212036)
\curveto(73.11765887,691.12211351)(73.21765877,691.02211361)(73.29766632,690.91212036)
\curveto(73.37765861,690.81211382)(73.44765854,690.71211392)(73.50766632,690.61212036)
\curveto(73.57765841,690.51211412)(73.64765834,690.40711422)(73.71766632,690.29712036)
\curveto(73.7876582,690.18711444)(73.84265814,690.06711456)(73.88266632,689.93712036)
\curveto(73.92265806,689.81711481)(73.96765802,689.68711494)(74.01766632,689.54712036)
\curveto(74.04765794,689.46711516)(74.07265791,689.38211525)(74.09266632,689.29212036)
\lineto(74.15266632,689.02212036)
\curveto(74.16265782,688.98211565)(74.16765782,688.94211569)(74.16766632,688.90212036)
\curveto(74.16765782,688.86211577)(74.17265781,688.82211581)(74.18266632,688.78212036)
\curveto(74.20265778,688.7321159)(74.20765778,688.67711595)(74.19766632,688.61712036)
\curveto(74.1876578,688.55711607)(74.19265779,688.50211613)(74.21266632,688.45212036)
\moveto(72.11266632,687.91212036)
\curveto(72.12265986,687.96211667)(72.12765986,688.0321166)(72.12766632,688.12212036)
\curveto(72.12765986,688.22211641)(72.12265986,688.29711633)(72.11266632,688.34712036)
\lineto(72.11266632,688.46712036)
\curveto(72.09265989,688.51711611)(72.0826599,688.57211606)(72.08266632,688.63212036)
\curveto(72.0826599,688.69211594)(72.07765991,688.74711588)(72.06766632,688.79712036)
\curveto(72.06765992,688.83711579)(72.06265992,688.86711576)(72.05266632,688.88712036)
\lineto(71.99266632,689.12712036)
\curveto(71.98266,689.21711541)(71.96266002,689.30211533)(71.93266632,689.38212036)
\curveto(71.82266016,689.64211499)(71.69266029,689.86211477)(71.54266632,690.04212036)
\curveto(71.39266059,690.2321144)(71.19266079,690.38211425)(70.94266632,690.49212036)
\curveto(70.8826611,690.51211412)(70.82266116,690.5271141)(70.76266632,690.53712036)
\curveto(70.70266128,690.55711407)(70.63766135,690.57711405)(70.56766632,690.59712036)
\curveto(70.4876615,690.61711401)(70.40266158,690.62211401)(70.31266632,690.61212036)
\lineto(70.04266632,690.61212036)
\curveto(70.01266197,690.59211404)(69.97766201,690.58211405)(69.93766632,690.58212036)
\curveto(69.89766209,690.59211404)(69.86266212,690.59211404)(69.83266632,690.58212036)
\lineto(69.62266632,690.52212036)
\curveto(69.56266242,690.51211412)(69.50766248,690.49211414)(69.45766632,690.46212036)
\curveto(69.20766278,690.35211428)(69.00266298,690.19211444)(68.84266632,689.98212036)
\curveto(68.69266329,689.78211485)(68.57266341,689.54711508)(68.48266632,689.27712036)
\curveto(68.45266353,689.17711545)(68.42766356,689.07211556)(68.40766632,688.96212036)
\curveto(68.39766359,688.85211578)(68.3826636,688.74211589)(68.36266632,688.63212036)
\curveto(68.35266363,688.58211605)(68.34766364,688.5321161)(68.34766632,688.48212036)
\lineto(68.34766632,688.33212036)
\curveto(68.32766366,688.26211637)(68.31766367,688.15711647)(68.31766632,688.01712036)
\curveto(68.32766366,687.87711675)(68.34266364,687.77211686)(68.36266632,687.70212036)
\lineto(68.36266632,687.56712036)
\curveto(68.3826636,687.48711714)(68.39766359,687.40711722)(68.40766632,687.32712036)
\curveto(68.41766357,687.25711737)(68.43266355,687.18211745)(68.45266632,687.10212036)
\curveto(68.55266343,686.80211783)(68.65766333,686.55711807)(68.76766632,686.36712036)
\curveto(68.8876631,686.18711844)(69.07266291,686.02211861)(69.32266632,685.87212036)
\curveto(69.39266259,685.82211881)(69.46766252,685.78211885)(69.54766632,685.75212036)
\curveto(69.63766235,685.72211891)(69.72766226,685.69711893)(69.81766632,685.67712036)
\curveto(69.85766213,685.66711896)(69.89266209,685.66211897)(69.92266632,685.66212036)
\curveto(69.95266203,685.67211896)(69.987662,685.67211896)(70.02766632,685.66212036)
\lineto(70.14766632,685.63212036)
\curveto(70.19766179,685.632119)(70.24266174,685.63711899)(70.28266632,685.64712036)
\lineto(70.40266632,685.64712036)
\curveto(70.4826615,685.66711896)(70.56266142,685.68211895)(70.64266632,685.69212036)
\curveto(70.72266126,685.70211893)(70.79766119,685.72211891)(70.86766632,685.75212036)
\curveto(71.12766086,685.85211878)(71.33766065,685.98711864)(71.49766632,686.15712036)
\curveto(71.65766033,686.3271183)(71.79266019,686.53711809)(71.90266632,686.78712036)
\curveto(71.94266004,686.88711774)(71.97266001,686.98711764)(71.99266632,687.08712036)
\curveto(72.01265997,687.18711744)(72.03765995,687.29211734)(72.06766632,687.40212036)
\curveto(72.07765991,687.44211719)(72.0826599,687.47711715)(72.08266632,687.50712036)
\curveto(72.0826599,687.54711708)(72.0876599,687.58711704)(72.09766632,687.62712036)
\lineto(72.09766632,687.76212036)
\curveto(72.09765989,687.81211682)(72.10265988,687.86211677)(72.11266632,687.91212036)
}
}
{
\newrgbcolor{curcolor}{0 0 0}
\pscustom[linestyle=none,fillstyle=solid,fillcolor=curcolor]
{
\newpath
\moveto(78.5825882,692.21712036)
\curveto(79.3325837,692.23711239)(79.98258305,692.15211248)(80.5325882,691.96212036)
\curveto(81.09258194,691.78211285)(81.51758151,691.46711316)(81.8075882,691.01712036)
\curveto(81.87758115,690.90711372)(81.93758109,690.79211384)(81.9875882,690.67212036)
\curveto(82.04758098,690.56211407)(82.09758093,690.43711419)(82.1375882,690.29712036)
\curveto(82.15758087,690.23711439)(82.16758086,690.17211446)(82.1675882,690.10212036)
\curveto(82.16758086,690.0321146)(82.15758087,689.97211466)(82.1375882,689.92212036)
\curveto(82.09758093,689.86211477)(82.04258099,689.82211481)(81.9725882,689.80212036)
\curveto(81.92258111,689.78211485)(81.86258117,689.77211486)(81.7925882,689.77212036)
\lineto(81.5825882,689.77212036)
\lineto(80.9225882,689.77212036)
\curveto(80.85258218,689.77211486)(80.78258225,689.76711486)(80.7125882,689.75712036)
\curveto(80.64258239,689.75711487)(80.57758245,689.76711486)(80.5175882,689.78712036)
\curveto(80.41758261,689.80711482)(80.34258269,689.84711478)(80.2925882,689.90712036)
\curveto(80.24258279,689.96711466)(80.19758283,690.0271146)(80.1575882,690.08712036)
\lineto(80.0375882,690.29712036)
\curveto(80.00758302,690.37711425)(79.95758307,690.44211419)(79.8875882,690.49212036)
\curveto(79.78758324,690.57211406)(79.68758334,690.632114)(79.5875882,690.67212036)
\curveto(79.49758353,690.71211392)(79.38258365,690.74711388)(79.2425882,690.77712036)
\curveto(79.17258386,690.79711383)(79.06758396,690.81211382)(78.9275882,690.82212036)
\curveto(78.79758423,690.8321138)(78.69758433,690.8271138)(78.6275882,690.80712036)
\lineto(78.5225882,690.80712036)
\lineto(78.3725882,690.77712036)
\curveto(78.3325847,690.77711385)(78.28758474,690.77211386)(78.2375882,690.76212036)
\curveto(78.06758496,690.71211392)(77.9275851,690.64211399)(77.8175882,690.55212036)
\curveto(77.71758531,690.47211416)(77.64758538,690.34711428)(77.6075882,690.17712036)
\curveto(77.58758544,690.10711452)(77.58758544,690.04211459)(77.6075882,689.98212036)
\curveto(77.6275854,689.92211471)(77.64758538,689.87211476)(77.6675882,689.83212036)
\curveto(77.73758529,689.71211492)(77.81758521,689.61711501)(77.9075882,689.54712036)
\curveto(78.00758502,689.47711515)(78.12258491,689.41711521)(78.2525882,689.36712036)
\curveto(78.44258459,689.28711534)(78.64758438,689.21711541)(78.8675882,689.15712036)
\lineto(79.5575882,689.00712036)
\curveto(79.79758323,688.96711566)(80.027583,688.91711571)(80.2475882,688.85712036)
\curveto(80.47758255,688.80711582)(80.69258234,688.74211589)(80.8925882,688.66212036)
\curveto(80.98258205,688.62211601)(81.06758196,688.58711604)(81.1475882,688.55712036)
\curveto(81.23758179,688.53711609)(81.32258171,688.50211613)(81.4025882,688.45212036)
\curveto(81.59258144,688.3321163)(81.76258127,688.20211643)(81.9125882,688.06212036)
\curveto(82.07258096,687.92211671)(82.19758083,687.74711688)(82.2875882,687.53712036)
\curveto(82.31758071,687.46711716)(82.34258069,687.39711723)(82.3625882,687.32712036)
\curveto(82.38258065,687.25711737)(82.40258063,687.18211745)(82.4225882,687.10212036)
\curveto(82.4325806,687.04211759)(82.43758059,686.94711768)(82.4375882,686.81712036)
\curveto(82.44758058,686.69711793)(82.44758058,686.60211803)(82.4375882,686.53212036)
\lineto(82.4375882,686.45712036)
\curveto(82.41758061,686.39711823)(82.40258063,686.33711829)(82.3925882,686.27712036)
\curveto(82.39258064,686.2271184)(82.38758064,686.17711845)(82.3775882,686.12712036)
\curveto(82.30758072,685.8271188)(82.19758083,685.56211907)(82.0475882,685.33212036)
\curveto(81.88758114,685.09211954)(81.69258134,684.89711973)(81.4625882,684.74712036)
\curveto(81.2325818,684.59712003)(80.97258206,684.46712016)(80.6825882,684.35712036)
\curveto(80.57258246,684.30712032)(80.45258258,684.27212036)(80.3225882,684.25212036)
\curveto(80.20258283,684.2321204)(80.08258295,684.20712042)(79.9625882,684.17712036)
\curveto(79.87258316,684.15712047)(79.77758325,684.14712048)(79.6775882,684.14712036)
\curveto(79.58758344,684.13712049)(79.49758353,684.12212051)(79.4075882,684.10212036)
\lineto(79.1375882,684.10212036)
\curveto(79.07758395,684.08212055)(78.97258406,684.07212056)(78.8225882,684.07212036)
\curveto(78.68258435,684.07212056)(78.58258445,684.08212055)(78.5225882,684.10212036)
\curveto(78.49258454,684.10212053)(78.45758457,684.10712052)(78.4175882,684.11712036)
\lineto(78.3125882,684.11712036)
\curveto(78.19258484,684.13712049)(78.07258496,684.15212048)(77.9525882,684.16212036)
\curveto(77.8325852,684.17212046)(77.71758531,684.19212044)(77.6075882,684.22212036)
\curveto(77.21758581,684.3321203)(76.87258616,684.45712017)(76.5725882,684.59712036)
\curveto(76.27258676,684.74711988)(76.01758701,684.96711966)(75.8075882,685.25712036)
\curveto(75.66758736,685.44711918)(75.54758748,685.66711896)(75.4475882,685.91712036)
\curveto(75.4275876,685.97711865)(75.40758762,686.05711857)(75.3875882,686.15712036)
\curveto(75.36758766,686.20711842)(75.35258768,686.27711835)(75.3425882,686.36712036)
\curveto(75.3325877,686.45711817)(75.33758769,686.5321181)(75.3575882,686.59212036)
\curveto(75.38758764,686.66211797)(75.43758759,686.71211792)(75.5075882,686.74212036)
\curveto(75.55758747,686.76211787)(75.61758741,686.77211786)(75.6875882,686.77212036)
\lineto(75.9125882,686.77212036)
\lineto(76.6175882,686.77212036)
\lineto(76.8575882,686.77212036)
\curveto(76.93758609,686.77211786)(77.00758602,686.76211787)(77.0675882,686.74212036)
\curveto(77.17758585,686.70211793)(77.24758578,686.63711799)(77.2775882,686.54712036)
\curveto(77.31758571,686.45711817)(77.36258567,686.36211827)(77.4125882,686.26212036)
\curveto(77.4325856,686.21211842)(77.46758556,686.14711848)(77.5175882,686.06712036)
\curveto(77.57758545,685.98711864)(77.6275854,685.93711869)(77.6675882,685.91712036)
\curveto(77.78758524,685.81711881)(77.90258513,685.73711889)(78.0125882,685.67712036)
\curveto(78.12258491,685.627119)(78.26258477,685.57711905)(78.4325882,685.52712036)
\curveto(78.48258455,685.50711912)(78.5325845,685.49711913)(78.5825882,685.49712036)
\curveto(78.6325844,685.50711912)(78.68258435,685.50711912)(78.7325882,685.49712036)
\curveto(78.81258422,685.47711915)(78.89758413,685.46711916)(78.9875882,685.46712036)
\curveto(79.08758394,685.47711915)(79.17258386,685.49211914)(79.2425882,685.51212036)
\curveto(79.29258374,685.52211911)(79.33758369,685.5271191)(79.3775882,685.52712036)
\curveto(79.4275836,685.5271191)(79.47758355,685.53711909)(79.5275882,685.55712036)
\curveto(79.66758336,685.60711902)(79.79258324,685.66711896)(79.9025882,685.73712036)
\curveto(80.02258301,685.80711882)(80.11758291,685.89711873)(80.1875882,686.00712036)
\curveto(80.23758279,686.08711854)(80.27758275,686.21211842)(80.3075882,686.38212036)
\curveto(80.3275827,686.45211818)(80.3275827,686.51711811)(80.3075882,686.57712036)
\curveto(80.28758274,686.63711799)(80.26758276,686.68711794)(80.2475882,686.72712036)
\curveto(80.17758285,686.86711776)(80.08758294,686.97211766)(79.9775882,687.04212036)
\curveto(79.87758315,687.11211752)(79.75758327,687.17711745)(79.6175882,687.23712036)
\curveto(79.4275836,687.31711731)(79.2275838,687.38211725)(79.0175882,687.43212036)
\curveto(78.80758422,687.48211715)(78.59758443,687.53711709)(78.3875882,687.59712036)
\curveto(78.30758472,687.61711701)(78.22258481,687.632117)(78.1325882,687.64212036)
\curveto(78.05258498,687.65211698)(77.97258506,687.66711696)(77.8925882,687.68712036)
\curveto(77.57258546,687.77711685)(77.26758576,687.86211677)(76.9775882,687.94212036)
\curveto(76.68758634,688.0321166)(76.42258661,688.16211647)(76.1825882,688.33212036)
\curveto(75.90258713,688.5321161)(75.69758733,688.80211583)(75.5675882,689.14212036)
\curveto(75.54758748,689.21211542)(75.5275875,689.30711532)(75.5075882,689.42712036)
\curveto(75.48758754,689.49711513)(75.47258756,689.58211505)(75.4625882,689.68212036)
\curveto(75.45258758,689.78211485)(75.45758757,689.87211476)(75.4775882,689.95212036)
\curveto(75.49758753,690.00211463)(75.50258753,690.04211459)(75.4925882,690.07212036)
\curveto(75.48258755,690.11211452)(75.48758754,690.15711447)(75.5075882,690.20712036)
\curveto(75.5275875,690.31711431)(75.54758748,690.41711421)(75.5675882,690.50712036)
\curveto(75.59758743,690.60711402)(75.6325874,690.70211393)(75.6725882,690.79212036)
\curveto(75.80258723,691.08211355)(75.98258705,691.31711331)(76.2125882,691.49712036)
\curveto(76.44258659,691.67711295)(76.70258633,691.82211281)(76.9925882,691.93212036)
\curveto(77.10258593,691.98211265)(77.21758581,692.01711261)(77.3375882,692.03712036)
\curveto(77.45758557,692.06711256)(77.58258545,692.09711253)(77.7125882,692.12712036)
\curveto(77.77258526,692.14711248)(77.8325852,692.15711247)(77.8925882,692.15712036)
\lineto(78.0725882,692.18712036)
\curveto(78.15258488,692.19711243)(78.23758479,692.20211243)(78.3275882,692.20212036)
\curveto(78.41758461,692.20211243)(78.50258453,692.20711242)(78.5825882,692.21712036)
}
}
{
\newrgbcolor{curcolor}{0 0 0}
\pscustom[linestyle=none,fillstyle=solid,fillcolor=curcolor]
{
}
}
{
\newrgbcolor{curcolor}{0 0 0}
\pscustom[linestyle=none,fillstyle=solid,fillcolor=curcolor]
{
\newpath
\moveto(95.38938507,682.42212036)
\curveto(95.38937673,682.26212237)(95.38437674,682.10712252)(95.37438507,681.95712036)
\curveto(95.37437675,681.79712283)(95.3243768,681.68712294)(95.22438507,681.62712036)
\curveto(95.14437698,681.57712305)(95.02937709,681.55712307)(94.87938507,681.56712036)
\lineto(94.45938507,681.56712036)
\lineto(94.14438507,681.56712036)
\curveto(94.03437809,681.55712307)(93.9243782,681.55712307)(93.81438507,681.56712036)
\curveto(93.71437841,681.56712306)(93.6193785,681.58212305)(93.52938507,681.61212036)
\curveto(93.44937867,681.632123)(93.38937873,681.67212296)(93.34938507,681.73212036)
\curveto(93.29937882,681.81212282)(93.27437885,681.9271227)(93.27438507,682.07712036)
\curveto(93.28437884,682.21712241)(93.28937883,682.34712228)(93.28938507,682.46712036)
\lineto(93.28938507,684.10212036)
\lineto(93.28938507,684.47712036)
\curveto(93.28937883,684.61712001)(93.27437885,684.72211991)(93.24438507,684.79212036)
\curveto(93.2243789,684.81211982)(93.20437892,684.8271198)(93.18438507,684.83712036)
\curveto(93.17437895,684.85711977)(93.15937896,684.87711975)(93.13938507,684.89712036)
\curveto(93.04937907,684.90711972)(92.97937914,684.88711974)(92.92938507,684.83712036)
\curveto(92.87937924,684.79711983)(92.8243793,684.75711987)(92.76438507,684.71712036)
\curveto(92.67437945,684.64711998)(92.57937954,684.58212005)(92.47938507,684.52212036)
\curveto(92.38937973,684.46212017)(92.28937983,684.40712022)(92.17938507,684.35712036)
\curveto(91.99938012,684.27712035)(91.79938032,684.21712041)(91.57938507,684.17712036)
\curveto(91.35938076,684.1271205)(91.13438099,684.10212053)(90.90438507,684.10212036)
\curveto(90.67438145,684.09212054)(90.44438168,684.10712052)(90.21438507,684.14712036)
\curveto(89.99438213,684.18712044)(89.79438233,684.24712038)(89.61438507,684.32712036)
\curveto(89.16438296,684.5271201)(88.79938332,684.78211985)(88.51938507,685.09212036)
\curveto(88.23938388,685.41211922)(88.00438412,685.80211883)(87.81438507,686.26212036)
\curveto(87.76438436,686.37211826)(87.72938439,686.48211815)(87.70938507,686.59212036)
\curveto(87.68938443,686.71211792)(87.66438446,686.8271178)(87.63438507,686.93712036)
\curveto(87.61438451,686.97711765)(87.60438452,687.01211762)(87.60438507,687.04212036)
\curveto(87.61438451,687.08211755)(87.61438451,687.12211751)(87.60438507,687.16212036)
\curveto(87.58438454,687.24211739)(87.56938455,687.3271173)(87.55938507,687.41712036)
\curveto(87.55938456,687.51711711)(87.54938457,687.61211702)(87.52938507,687.70212036)
\lineto(87.52938507,687.89712036)
\curveto(87.5193846,687.94711668)(87.51438461,688.00711662)(87.51438507,688.07712036)
\curveto(87.51438461,688.15711647)(87.5193846,688.22211641)(87.52938507,688.27212036)
\curveto(87.53938458,688.32211631)(87.54438458,688.36711626)(87.54438507,688.40712036)
\lineto(87.54438507,688.54212036)
\curveto(87.55438457,688.59211604)(87.55438457,688.64211599)(87.54438507,688.69212036)
\curveto(87.54438458,688.74211589)(87.55438457,688.79211584)(87.57438507,688.84212036)
\curveto(87.59438453,688.9321157)(87.60938451,689.02211561)(87.61938507,689.11212036)
\curveto(87.62938449,689.21211542)(87.64438448,689.30711532)(87.66438507,689.39712036)
\curveto(87.71438441,689.56711506)(87.76438436,689.7271149)(87.81438507,689.87712036)
\curveto(87.87438425,690.0271146)(87.93438419,690.17211446)(87.99438507,690.31212036)
\curveto(88.05438407,690.45211418)(88.12938399,690.58711404)(88.21938507,690.71712036)
\curveto(88.30938381,690.84711378)(88.39938372,690.97211366)(88.48938507,691.09212036)
\curveto(88.57938354,691.20211343)(88.67938344,691.30211333)(88.78938507,691.39212036)
\curveto(88.8193833,691.42211321)(88.83938328,691.44711318)(88.84938507,691.46712036)
\curveto(88.89938322,691.49711313)(88.94438318,691.5271131)(88.98438507,691.55712036)
\curveto(89.0243831,691.59711303)(89.06438306,691.632113)(89.10438507,691.66212036)
\curveto(89.24438288,691.76211287)(89.38938273,691.84211279)(89.53938507,691.90212036)
\curveto(89.69938242,691.97211266)(89.86438226,692.03711259)(90.03438507,692.09712036)
\curveto(90.124382,692.1271125)(90.21438191,692.14711248)(90.30438507,692.15712036)
\curveto(90.39438173,692.16711246)(90.48438164,692.18211245)(90.57438507,692.20212036)
\curveto(90.60438152,692.21211242)(90.65938146,692.21211242)(90.73938507,692.20212036)
\curveto(90.8193813,692.19211244)(90.86938125,692.19711243)(90.88938507,692.21712036)
\curveto(91.20938091,692.2271124)(91.50938061,692.19711243)(91.78938507,692.12712036)
\curveto(92.06938005,692.06711256)(92.30937981,691.97711265)(92.50938507,691.85712036)
\lineto(92.68938507,691.73712036)
\curveto(92.74937937,691.69711293)(92.80437932,691.65711297)(92.85438507,691.61712036)
\curveto(92.91437921,691.56711306)(92.96437916,691.51711311)(93.00438507,691.46712036)
\curveto(93.05437907,691.4271132)(93.13437899,691.40711322)(93.24438507,691.40712036)
\lineto(93.28938507,691.45212036)
\lineto(93.34938507,691.51212036)
\curveto(93.37937874,691.59211304)(93.39937872,691.66711296)(93.40938507,691.73712036)
\curveto(93.4193787,691.81711281)(93.45937866,691.88211275)(93.52938507,691.93212036)
\curveto(93.57937854,691.97211266)(93.64937847,691.99211264)(93.73938507,691.99212036)
\curveto(93.83937828,692.00211263)(93.93937818,692.00711262)(94.03938507,692.00712036)
\lineto(94.75938507,692.00712036)
\lineto(94.96938507,692.00712036)
\curveto(95.03937708,692.00711262)(95.10437702,691.99711263)(95.16438507,691.97712036)
\curveto(95.23437689,691.95711267)(95.28937683,691.91211272)(95.32938507,691.84212036)
\curveto(95.37937674,691.77211286)(95.39937672,691.67711295)(95.38938507,691.55712036)
\lineto(95.38938507,691.21212036)
\lineto(95.38938507,682.42212036)
\moveto(93.34938507,688.03212036)
\curveto(93.35937876,688.05211658)(93.35937876,688.07711655)(93.34938507,688.10712036)
\lineto(93.34938507,688.18212036)
\curveto(93.33937878,688.28211635)(93.33437879,688.37711625)(93.33438507,688.46712036)
\curveto(93.33437879,688.55711607)(93.3243788,688.64211599)(93.30438507,688.72212036)
\curveto(93.29437883,688.75211588)(93.28937883,688.77711585)(93.28938507,688.79712036)
\curveto(93.29937882,688.8271158)(93.29937882,688.85711577)(93.28938507,688.88712036)
\curveto(93.26937885,688.96711566)(93.24937887,689.03711559)(93.22938507,689.09712036)
\curveto(93.2193789,689.16711546)(93.20437892,689.23711539)(93.18438507,689.30712036)
\curveto(93.08437904,689.59711503)(92.94937917,689.84711478)(92.77938507,690.05712036)
\curveto(92.60937951,690.26711436)(92.38937973,690.4271142)(92.11938507,690.53712036)
\curveto(92.00938011,690.58711404)(91.88938023,690.61211402)(91.75938507,690.61212036)
\curveto(91.63938048,690.62211401)(91.50938061,690.627114)(91.36938507,690.62712036)
\curveto(91.33938078,690.60711402)(91.30438082,690.59711403)(91.26438507,690.59712036)
\curveto(91.2243809,690.60711402)(91.18438094,690.60711402)(91.14438507,690.59712036)
\lineto(90.96438507,690.53712036)
\curveto(90.90438122,690.5271141)(90.84938127,690.51211412)(90.79938507,690.49212036)
\curveto(90.50938161,690.36211427)(90.27938184,690.17211446)(90.10938507,689.92212036)
\curveto(89.94938217,689.67211496)(89.8243823,689.38211525)(89.73438507,689.05212036)
\curveto(89.71438241,688.97211566)(89.69938242,688.89711573)(89.68938507,688.82712036)
\curveto(89.68938243,688.76711586)(89.67938244,688.69711593)(89.65938507,688.61712036)
\curveto(89.65938246,688.54711608)(89.65438247,688.49711613)(89.64438507,688.46712036)
\curveto(89.63438249,688.41711621)(89.6243825,688.3271163)(89.61438507,688.19712036)
\curveto(89.61438251,688.07711655)(89.6243825,687.99211664)(89.64438507,687.94212036)
\lineto(89.64438507,687.80712036)
\curveto(89.65438247,687.76711686)(89.65938246,687.7271169)(89.65938507,687.68712036)
\curveto(89.65938246,687.64711698)(89.66438246,687.61211702)(89.67438507,687.58212036)
\lineto(89.67438507,687.50712036)
\curveto(89.68438244,687.47711715)(89.68938243,687.45211718)(89.68938507,687.43212036)
\curveto(89.70938241,687.35211728)(89.7243824,687.27711735)(89.73438507,687.20712036)
\curveto(89.74438238,687.13711749)(89.76438236,687.06711756)(89.79438507,686.99712036)
\curveto(89.87438225,686.74711788)(89.97938214,686.5321181)(90.10938507,686.35212036)
\curveto(90.23938188,686.17211846)(90.40438172,686.01711861)(90.60438507,685.88712036)
\curveto(90.74438138,685.80711882)(90.89938122,685.74711888)(91.06938507,685.70712036)
\curveto(91.09938102,685.69711893)(91.124381,685.69211894)(91.14438507,685.69212036)
\curveto(91.17438095,685.69211894)(91.20938091,685.68711894)(91.24938507,685.67712036)
\curveto(91.27938084,685.66711896)(91.3243808,685.65711897)(91.38438507,685.64712036)
\curveto(91.45438067,685.64711898)(91.51438061,685.65211898)(91.56438507,685.66212036)
\curveto(91.58438054,685.67211896)(91.60938051,685.67211896)(91.63938507,685.66212036)
\curveto(91.67938044,685.66211897)(91.71438041,685.66711896)(91.74438507,685.67712036)
\curveto(91.81438031,685.69711893)(91.87938024,685.71211892)(91.93938507,685.72212036)
\curveto(92.00938011,685.7321189)(92.07938004,685.74711888)(92.14938507,685.76712036)
\curveto(92.40937971,685.87711875)(92.61437951,686.02211861)(92.76438507,686.20212036)
\curveto(92.9243792,686.38211825)(93.05937906,686.60211803)(93.16938507,686.86212036)
\curveto(93.19937892,686.94211769)(93.2243789,687.0271176)(93.24438507,687.11712036)
\lineto(93.30438507,687.38712036)
\lineto(93.30438507,687.49212036)
\curveto(93.31437881,687.52211711)(93.3193788,687.55711707)(93.31938507,687.59712036)
\curveto(93.33937878,687.69711693)(93.34937877,687.78211685)(93.34938507,687.85212036)
\lineto(93.34938507,688.03212036)
}
}
{
\newrgbcolor{curcolor}{0 0 0}
\pscustom[linestyle=none,fillstyle=solid,fillcolor=curcolor]
{
\newpath
\moveto(97.41930695,691.99212036)
\lineto(98.54430695,691.99212036)
\curveto(98.65430451,691.99211264)(98.75430441,691.98711264)(98.84430695,691.97712036)
\curveto(98.93430423,691.96711266)(98.99930417,691.9321127)(99.03930695,691.87212036)
\curveto(99.08930408,691.81211282)(99.11930405,691.7271129)(99.12930695,691.61712036)
\curveto(99.13930403,691.51711311)(99.14430402,691.41211322)(99.14430695,691.30212036)
\lineto(99.14430695,690.25212036)
\lineto(99.14430695,688.01712036)
\curveto(99.14430402,687.65711697)(99.15930401,687.31711731)(99.18930695,686.99712036)
\curveto(99.21930395,686.67711795)(99.30930386,686.41211822)(99.45930695,686.20212036)
\curveto(99.59930357,685.99211864)(99.82430334,685.84211879)(100.13430695,685.75212036)
\curveto(100.18430298,685.74211889)(100.22430294,685.73711889)(100.25430695,685.73712036)
\curveto(100.29430287,685.73711889)(100.33930283,685.7321189)(100.38930695,685.72212036)
\curveto(100.43930273,685.71211892)(100.49430267,685.70711892)(100.55430695,685.70712036)
\curveto(100.61430255,685.70711892)(100.65930251,685.71211892)(100.68930695,685.72212036)
\curveto(100.73930243,685.74211889)(100.77930239,685.74711888)(100.80930695,685.73712036)
\curveto(100.84930232,685.7271189)(100.88930228,685.7321189)(100.92930695,685.75212036)
\curveto(101.13930203,685.80211883)(101.30430186,685.86711876)(101.42430695,685.94712036)
\curveto(101.60430156,686.05711857)(101.74430142,686.19711843)(101.84430695,686.36712036)
\curveto(101.95430121,686.54711808)(102.02930114,686.74211789)(102.06930695,686.95212036)
\curveto(102.11930105,687.17211746)(102.14930102,687.41211722)(102.15930695,687.67212036)
\curveto(102.169301,687.94211669)(102.17430099,688.22211641)(102.17430695,688.51212036)
\lineto(102.17430695,690.32712036)
\lineto(102.17430695,691.30212036)
\lineto(102.17430695,691.57212036)
\curveto(102.17430099,691.67211296)(102.19430097,691.75211288)(102.23430695,691.81212036)
\curveto(102.28430088,691.90211273)(102.35930081,691.95211268)(102.45930695,691.96212036)
\curveto(102.55930061,691.98211265)(102.67930049,691.99211264)(102.81930695,691.99212036)
\lineto(103.61430695,691.99212036)
\lineto(103.89930695,691.99212036)
\curveto(103.98929918,691.99211264)(104.0642991,691.97211266)(104.12430695,691.93212036)
\curveto(104.20429896,691.88211275)(104.24929892,691.80711282)(104.25930695,691.70712036)
\curveto(104.2692989,691.60711302)(104.27429889,691.49211314)(104.27430695,691.36212036)
\lineto(104.27430695,690.22212036)
\lineto(104.27430695,686.00712036)
\lineto(104.27430695,684.94212036)
\lineto(104.27430695,684.64212036)
\curveto(104.27429889,684.54212009)(104.25429891,684.46712016)(104.21430695,684.41712036)
\curveto(104.164299,684.33712029)(104.08929908,684.29212034)(103.98930695,684.28212036)
\curveto(103.88929928,684.27212036)(103.78429938,684.26712036)(103.67430695,684.26712036)
\lineto(102.86430695,684.26712036)
\curveto(102.75430041,684.26712036)(102.65430051,684.27212036)(102.56430695,684.28212036)
\curveto(102.48430068,684.29212034)(102.41930075,684.3321203)(102.36930695,684.40212036)
\curveto(102.34930082,684.4321202)(102.32930084,684.47712015)(102.30930695,684.53712036)
\curveto(102.29930087,684.59712003)(102.28430088,684.65711997)(102.26430695,684.71712036)
\curveto(102.25430091,684.77711985)(102.23930093,684.8321198)(102.21930695,684.88212036)
\curveto(102.19930097,684.9321197)(102.169301,684.96211967)(102.12930695,684.97212036)
\curveto(102.10930106,684.99211964)(102.08430108,684.99711963)(102.05430695,684.98712036)
\curveto(102.02430114,684.97711965)(101.99930117,684.96711966)(101.97930695,684.95712036)
\curveto(101.90930126,684.91711971)(101.84930132,684.87211976)(101.79930695,684.82212036)
\curveto(101.74930142,684.77211986)(101.69430147,684.7271199)(101.63430695,684.68712036)
\curveto(101.59430157,684.65711997)(101.55430161,684.62212001)(101.51430695,684.58212036)
\curveto(101.48430168,684.55212008)(101.44430172,684.52212011)(101.39430695,684.49212036)
\curveto(101.164302,684.35212028)(100.89430227,684.24212039)(100.58430695,684.16212036)
\curveto(100.51430265,684.14212049)(100.44430272,684.1321205)(100.37430695,684.13212036)
\curveto(100.30430286,684.12212051)(100.22930294,684.10712052)(100.14930695,684.08712036)
\curveto(100.10930306,684.07712055)(100.0643031,684.07712055)(100.01430695,684.08712036)
\curveto(99.97430319,684.08712054)(99.93430323,684.08212055)(99.89430695,684.07212036)
\curveto(99.8643033,684.06212057)(99.79930337,684.06212057)(99.69930695,684.07212036)
\curveto(99.60930356,684.07212056)(99.54930362,684.07712055)(99.51930695,684.08712036)
\curveto(99.4693037,684.08712054)(99.41930375,684.09212054)(99.36930695,684.10212036)
\lineto(99.21930695,684.10212036)
\curveto(99.09930407,684.1321205)(98.98430418,684.15712047)(98.87430695,684.17712036)
\curveto(98.7643044,684.19712043)(98.65430451,684.2271204)(98.54430695,684.26712036)
\curveto(98.49430467,684.28712034)(98.44930472,684.30212033)(98.40930695,684.31212036)
\curveto(98.37930479,684.3321203)(98.33930483,684.35212028)(98.28930695,684.37212036)
\curveto(97.93930523,684.56212007)(97.65930551,684.8271198)(97.44930695,685.16712036)
\curveto(97.31930585,685.37711925)(97.22430594,685.627119)(97.16430695,685.91712036)
\curveto(97.10430606,686.21711841)(97.0643061,686.5321181)(97.04430695,686.86212036)
\curveto(97.03430613,687.20211743)(97.02930614,687.54711708)(97.02930695,687.89712036)
\curveto(97.03930613,688.25711637)(97.04430612,688.61211602)(97.04430695,688.96212036)
\lineto(97.04430695,691.00212036)
\curveto(97.04430612,691.1321135)(97.03930613,691.28211335)(97.02930695,691.45212036)
\curveto(97.02930614,691.632113)(97.05430611,691.76211287)(97.10430695,691.84212036)
\curveto(97.13430603,691.89211274)(97.19430597,691.93711269)(97.28430695,691.97712036)
\curveto(97.34430582,691.97711265)(97.38930578,691.98211265)(97.41930695,691.99212036)
}
}
{
\newrgbcolor{curcolor}{0 0 0}
\pscustom[linestyle=none,fillstyle=solid,fillcolor=curcolor]
{
\newpath
\moveto(113.27055695,688.21212036)
\curveto(113.29054878,688.1321165)(113.29054878,688.04211659)(113.27055695,687.94212036)
\curveto(113.25054882,687.84211679)(113.21554886,687.77711685)(113.16555695,687.74712036)
\curveto(113.11554896,687.70711692)(113.04054903,687.67711695)(112.94055695,687.65712036)
\curveto(112.85054922,687.64711698)(112.74554933,687.63711699)(112.62555695,687.62712036)
\lineto(112.28055695,687.62712036)
\curveto(112.1705499,687.63711699)(112.07055,687.64211699)(111.98055695,687.64212036)
\lineto(108.32055695,687.64212036)
\lineto(108.11055695,687.64212036)
\curveto(108.05055402,687.64211699)(107.99555408,687.632117)(107.94555695,687.61212036)
\curveto(107.86555421,687.57211706)(107.81555426,687.5321171)(107.79555695,687.49212036)
\curveto(107.7755543,687.47211716)(107.75555432,687.4321172)(107.73555695,687.37212036)
\curveto(107.71555436,687.32211731)(107.71055436,687.27211736)(107.72055695,687.22212036)
\curveto(107.74055433,687.16211747)(107.75055432,687.10211753)(107.75055695,687.04212036)
\curveto(107.76055431,686.99211764)(107.7755543,686.93711769)(107.79555695,686.87712036)
\curveto(107.8755542,686.63711799)(107.9705541,686.43711819)(108.08055695,686.27712036)
\curveto(108.20055387,686.1271185)(108.36055371,685.99211864)(108.56055695,685.87212036)
\curveto(108.64055343,685.82211881)(108.72055335,685.78711884)(108.80055695,685.76712036)
\curveto(108.89055318,685.75711887)(108.98055309,685.73711889)(109.07055695,685.70712036)
\curveto(109.15055292,685.68711894)(109.26055281,685.67211896)(109.40055695,685.66212036)
\curveto(109.54055253,685.65211898)(109.66055241,685.65711897)(109.76055695,685.67712036)
\lineto(109.89555695,685.67712036)
\curveto(109.99555208,685.69711893)(110.08555199,685.71711891)(110.16555695,685.73712036)
\curveto(110.25555182,685.76711886)(110.34055173,685.79711883)(110.42055695,685.82712036)
\curveto(110.52055155,685.87711875)(110.63055144,685.94211869)(110.75055695,686.02212036)
\curveto(110.88055119,686.10211853)(110.9755511,686.18211845)(111.03555695,686.26212036)
\curveto(111.08555099,686.3321183)(111.13555094,686.39711823)(111.18555695,686.45712036)
\curveto(111.24555083,686.5271181)(111.31555076,686.57711805)(111.39555695,686.60712036)
\curveto(111.49555058,686.65711797)(111.62055045,686.67711795)(111.77055695,686.66712036)
\lineto(112.20555695,686.66712036)
\lineto(112.38555695,686.66712036)
\curveto(112.45554962,686.67711795)(112.51554956,686.67211796)(112.56555695,686.65212036)
\lineto(112.71555695,686.65212036)
\curveto(112.81554926,686.632118)(112.88554919,686.60711802)(112.92555695,686.57712036)
\curveto(112.96554911,686.55711807)(112.98554909,686.51211812)(112.98555695,686.44212036)
\curveto(112.99554908,686.37211826)(112.99054908,686.31211832)(112.97055695,686.26212036)
\curveto(112.92054915,686.12211851)(112.86554921,685.99711863)(112.80555695,685.88712036)
\curveto(112.74554933,685.77711885)(112.6755494,685.66711896)(112.59555695,685.55712036)
\curveto(112.3755497,685.2271194)(112.12554995,684.96211967)(111.84555695,684.76212036)
\curveto(111.56555051,684.56212007)(111.21555086,684.39212024)(110.79555695,684.25212036)
\curveto(110.68555139,684.21212042)(110.5755515,684.18712044)(110.46555695,684.17712036)
\curveto(110.35555172,684.16712046)(110.24055183,684.14712048)(110.12055695,684.11712036)
\curveto(110.08055199,684.10712052)(110.03555204,684.10712052)(109.98555695,684.11712036)
\curveto(109.94555213,684.11712051)(109.90555217,684.11212052)(109.86555695,684.10212036)
\lineto(109.70055695,684.10212036)
\curveto(109.65055242,684.08212055)(109.59055248,684.07712055)(109.52055695,684.08712036)
\curveto(109.46055261,684.08712054)(109.40555267,684.09212054)(109.35555695,684.10212036)
\curveto(109.2755528,684.11212052)(109.20555287,684.11212052)(109.14555695,684.10212036)
\curveto(109.08555299,684.09212054)(109.02055305,684.09712053)(108.95055695,684.11712036)
\curveto(108.90055317,684.13712049)(108.84555323,684.14712048)(108.78555695,684.14712036)
\curveto(108.72555335,684.14712048)(108.6705534,684.15712047)(108.62055695,684.17712036)
\curveto(108.51055356,684.19712043)(108.40055367,684.22212041)(108.29055695,684.25212036)
\curveto(108.18055389,684.27212036)(108.08055399,684.30712032)(107.99055695,684.35712036)
\curveto(107.88055419,684.39712023)(107.7755543,684.4321202)(107.67555695,684.46212036)
\curveto(107.58555449,684.50212013)(107.50055457,684.54712008)(107.42055695,684.59712036)
\curveto(107.10055497,684.79711983)(106.81555526,685.0271196)(106.56555695,685.28712036)
\curveto(106.31555576,685.55711907)(106.11055596,685.86711876)(105.95055695,686.21712036)
\curveto(105.90055617,686.3271183)(105.86055621,686.43711819)(105.83055695,686.54712036)
\curveto(105.80055627,686.66711796)(105.76055631,686.78711784)(105.71055695,686.90712036)
\curveto(105.70055637,686.94711768)(105.69555638,686.98211765)(105.69555695,687.01212036)
\curveto(105.69555638,687.05211758)(105.69055638,687.09211754)(105.68055695,687.13212036)
\curveto(105.64055643,687.25211738)(105.61555646,687.38211725)(105.60555695,687.52212036)
\lineto(105.57555695,687.94212036)
\curveto(105.5755565,687.99211664)(105.5705565,688.04711658)(105.56055695,688.10712036)
\curveto(105.56055651,688.16711646)(105.56555651,688.22211641)(105.57555695,688.27212036)
\lineto(105.57555695,688.45212036)
\lineto(105.62055695,688.81212036)
\curveto(105.66055641,688.98211565)(105.69555638,689.14711548)(105.72555695,689.30712036)
\curveto(105.75555632,689.46711516)(105.80055627,689.61711501)(105.86055695,689.75712036)
\curveto(106.29055578,690.79711383)(107.02055505,691.5321131)(108.05055695,691.96212036)
\curveto(108.19055388,692.02211261)(108.33055374,692.06211257)(108.47055695,692.08212036)
\curveto(108.62055345,692.11211252)(108.7755533,692.14711248)(108.93555695,692.18712036)
\curveto(109.01555306,692.19711243)(109.09055298,692.20211243)(109.16055695,692.20212036)
\curveto(109.23055284,692.20211243)(109.30555277,692.20711242)(109.38555695,692.21712036)
\curveto(109.89555218,692.2271124)(110.33055174,692.16711246)(110.69055695,692.03712036)
\curveto(111.06055101,691.91711271)(111.39055068,691.75711287)(111.68055695,691.55712036)
\curveto(111.7705503,691.49711313)(111.86055021,691.4271132)(111.95055695,691.34712036)
\curveto(112.04055003,691.27711335)(112.12054995,691.20211343)(112.19055695,691.12212036)
\curveto(112.22054985,691.07211356)(112.26054981,691.0321136)(112.31055695,691.00212036)
\curveto(112.39054968,690.89211374)(112.46554961,690.77711385)(112.53555695,690.65712036)
\curveto(112.60554947,690.54711408)(112.68054939,690.4321142)(112.76055695,690.31212036)
\curveto(112.81054926,690.22211441)(112.85054922,690.1271145)(112.88055695,690.02712036)
\curveto(112.92054915,689.93711469)(112.96054911,689.83711479)(113.00055695,689.72712036)
\curveto(113.05054902,689.59711503)(113.09054898,689.46211517)(113.12055695,689.32212036)
\curveto(113.15054892,689.18211545)(113.18554889,689.04211559)(113.22555695,688.90212036)
\curveto(113.24554883,688.82211581)(113.25054882,688.7321159)(113.24055695,688.63212036)
\curveto(113.24054883,688.54211609)(113.25054882,688.45711617)(113.27055695,688.37712036)
\lineto(113.27055695,688.21212036)
\moveto(111.02055695,689.09712036)
\curveto(111.09055098,689.19711543)(111.09555098,689.31711531)(111.03555695,689.45712036)
\curveto(110.98555109,689.60711502)(110.94555113,689.71711491)(110.91555695,689.78712036)
\curveto(110.7755513,690.05711457)(110.59055148,690.26211437)(110.36055695,690.40212036)
\curveto(110.13055194,690.55211408)(109.81055226,690.632114)(109.40055695,690.64212036)
\curveto(109.3705527,690.62211401)(109.33555274,690.61711401)(109.29555695,690.62712036)
\curveto(109.25555282,690.63711399)(109.22055285,690.63711399)(109.19055695,690.62712036)
\curveto(109.14055293,690.60711402)(109.08555299,690.59211404)(109.02555695,690.58212036)
\curveto(108.96555311,690.58211405)(108.91055316,690.57211406)(108.86055695,690.55212036)
\curveto(108.42055365,690.41211422)(108.09555398,690.13711449)(107.88555695,689.72712036)
\curveto(107.86555421,689.68711494)(107.84055423,689.632115)(107.81055695,689.56212036)
\curveto(107.79055428,689.50211513)(107.7755543,689.43711519)(107.76555695,689.36712036)
\curveto(107.75555432,689.30711532)(107.75555432,689.24711538)(107.76555695,689.18712036)
\curveto(107.78555429,689.1271155)(107.82055425,689.07711555)(107.87055695,689.03712036)
\curveto(107.95055412,688.98711564)(108.06055401,688.96211567)(108.20055695,688.96212036)
\lineto(108.60555695,688.96212036)
\lineto(110.27055695,688.96212036)
\lineto(110.70555695,688.96212036)
\curveto(110.86555121,688.97211566)(110.9705511,689.01711561)(111.02055695,689.09712036)
}
}
{
\newrgbcolor{curcolor}{0 0 0}
\pscustom[linestyle=none,fillstyle=solid,fillcolor=curcolor]
{
}
}
{
\newrgbcolor{curcolor}{0 0 0}
\pscustom[linestyle=none,fillstyle=solid,fillcolor=curcolor]
{
\newpath
\moveto(126.04399445,688.21212036)
\curveto(126.06398628,688.1321165)(126.06398628,688.04211659)(126.04399445,687.94212036)
\curveto(126.02398632,687.84211679)(125.98898636,687.77711685)(125.93899445,687.74712036)
\curveto(125.88898646,687.70711692)(125.81398653,687.67711695)(125.71399445,687.65712036)
\curveto(125.62398672,687.64711698)(125.51898683,687.63711699)(125.39899445,687.62712036)
\lineto(125.05399445,687.62712036)
\curveto(124.9439874,687.63711699)(124.8439875,687.64211699)(124.75399445,687.64212036)
\lineto(121.09399445,687.64212036)
\lineto(120.88399445,687.64212036)
\curveto(120.82399152,687.64211699)(120.76899158,687.632117)(120.71899445,687.61212036)
\curveto(120.63899171,687.57211706)(120.58899176,687.5321171)(120.56899445,687.49212036)
\curveto(120.5489918,687.47211716)(120.52899182,687.4321172)(120.50899445,687.37212036)
\curveto(120.48899186,687.32211731)(120.48399186,687.27211736)(120.49399445,687.22212036)
\curveto(120.51399183,687.16211747)(120.52399182,687.10211753)(120.52399445,687.04212036)
\curveto(120.53399181,686.99211764)(120.5489918,686.93711769)(120.56899445,686.87712036)
\curveto(120.6489917,686.63711799)(120.7439916,686.43711819)(120.85399445,686.27712036)
\curveto(120.97399137,686.1271185)(121.13399121,685.99211864)(121.33399445,685.87212036)
\curveto(121.41399093,685.82211881)(121.49399085,685.78711884)(121.57399445,685.76712036)
\curveto(121.66399068,685.75711887)(121.75399059,685.73711889)(121.84399445,685.70712036)
\curveto(121.92399042,685.68711894)(122.03399031,685.67211896)(122.17399445,685.66212036)
\curveto(122.31399003,685.65211898)(122.43398991,685.65711897)(122.53399445,685.67712036)
\lineto(122.66899445,685.67712036)
\curveto(122.76898958,685.69711893)(122.85898949,685.71711891)(122.93899445,685.73712036)
\curveto(123.02898932,685.76711886)(123.11398923,685.79711883)(123.19399445,685.82712036)
\curveto(123.29398905,685.87711875)(123.40398894,685.94211869)(123.52399445,686.02212036)
\curveto(123.65398869,686.10211853)(123.7489886,686.18211845)(123.80899445,686.26212036)
\curveto(123.85898849,686.3321183)(123.90898844,686.39711823)(123.95899445,686.45712036)
\curveto(124.01898833,686.5271181)(124.08898826,686.57711805)(124.16899445,686.60712036)
\curveto(124.26898808,686.65711797)(124.39398795,686.67711795)(124.54399445,686.66712036)
\lineto(124.97899445,686.66712036)
\lineto(125.15899445,686.66712036)
\curveto(125.22898712,686.67711795)(125.28898706,686.67211796)(125.33899445,686.65212036)
\lineto(125.48899445,686.65212036)
\curveto(125.58898676,686.632118)(125.65898669,686.60711802)(125.69899445,686.57712036)
\curveto(125.73898661,686.55711807)(125.75898659,686.51211812)(125.75899445,686.44212036)
\curveto(125.76898658,686.37211826)(125.76398658,686.31211832)(125.74399445,686.26212036)
\curveto(125.69398665,686.12211851)(125.63898671,685.99711863)(125.57899445,685.88712036)
\curveto(125.51898683,685.77711885)(125.4489869,685.66711896)(125.36899445,685.55712036)
\curveto(125.1489872,685.2271194)(124.89898745,684.96211967)(124.61899445,684.76212036)
\curveto(124.33898801,684.56212007)(123.98898836,684.39212024)(123.56899445,684.25212036)
\curveto(123.45898889,684.21212042)(123.348989,684.18712044)(123.23899445,684.17712036)
\curveto(123.12898922,684.16712046)(123.01398933,684.14712048)(122.89399445,684.11712036)
\curveto(122.85398949,684.10712052)(122.80898954,684.10712052)(122.75899445,684.11712036)
\curveto(122.71898963,684.11712051)(122.67898967,684.11212052)(122.63899445,684.10212036)
\lineto(122.47399445,684.10212036)
\curveto(122.42398992,684.08212055)(122.36398998,684.07712055)(122.29399445,684.08712036)
\curveto(122.23399011,684.08712054)(122.17899017,684.09212054)(122.12899445,684.10212036)
\curveto(122.0489903,684.11212052)(121.97899037,684.11212052)(121.91899445,684.10212036)
\curveto(121.85899049,684.09212054)(121.79399055,684.09712053)(121.72399445,684.11712036)
\curveto(121.67399067,684.13712049)(121.61899073,684.14712048)(121.55899445,684.14712036)
\curveto(121.49899085,684.14712048)(121.4439909,684.15712047)(121.39399445,684.17712036)
\curveto(121.28399106,684.19712043)(121.17399117,684.22212041)(121.06399445,684.25212036)
\curveto(120.95399139,684.27212036)(120.85399149,684.30712032)(120.76399445,684.35712036)
\curveto(120.65399169,684.39712023)(120.5489918,684.4321202)(120.44899445,684.46212036)
\curveto(120.35899199,684.50212013)(120.27399207,684.54712008)(120.19399445,684.59712036)
\curveto(119.87399247,684.79711983)(119.58899276,685.0271196)(119.33899445,685.28712036)
\curveto(119.08899326,685.55711907)(118.88399346,685.86711876)(118.72399445,686.21712036)
\curveto(118.67399367,686.3271183)(118.63399371,686.43711819)(118.60399445,686.54712036)
\curveto(118.57399377,686.66711796)(118.53399381,686.78711784)(118.48399445,686.90712036)
\curveto(118.47399387,686.94711768)(118.46899388,686.98211765)(118.46899445,687.01212036)
\curveto(118.46899388,687.05211758)(118.46399388,687.09211754)(118.45399445,687.13212036)
\curveto(118.41399393,687.25211738)(118.38899396,687.38211725)(118.37899445,687.52212036)
\lineto(118.34899445,687.94212036)
\curveto(118.348994,687.99211664)(118.343994,688.04711658)(118.33399445,688.10712036)
\curveto(118.33399401,688.16711646)(118.33899401,688.22211641)(118.34899445,688.27212036)
\lineto(118.34899445,688.45212036)
\lineto(118.39399445,688.81212036)
\curveto(118.43399391,688.98211565)(118.46899388,689.14711548)(118.49899445,689.30712036)
\curveto(118.52899382,689.46711516)(118.57399377,689.61711501)(118.63399445,689.75712036)
\curveto(119.06399328,690.79711383)(119.79399255,691.5321131)(120.82399445,691.96212036)
\curveto(120.96399138,692.02211261)(121.10399124,692.06211257)(121.24399445,692.08212036)
\curveto(121.39399095,692.11211252)(121.5489908,692.14711248)(121.70899445,692.18712036)
\curveto(121.78899056,692.19711243)(121.86399048,692.20211243)(121.93399445,692.20212036)
\curveto(122.00399034,692.20211243)(122.07899027,692.20711242)(122.15899445,692.21712036)
\curveto(122.66898968,692.2271124)(123.10398924,692.16711246)(123.46399445,692.03712036)
\curveto(123.83398851,691.91711271)(124.16398818,691.75711287)(124.45399445,691.55712036)
\curveto(124.5439878,691.49711313)(124.63398771,691.4271132)(124.72399445,691.34712036)
\curveto(124.81398753,691.27711335)(124.89398745,691.20211343)(124.96399445,691.12212036)
\curveto(124.99398735,691.07211356)(125.03398731,691.0321136)(125.08399445,691.00212036)
\curveto(125.16398718,690.89211374)(125.23898711,690.77711385)(125.30899445,690.65712036)
\curveto(125.37898697,690.54711408)(125.45398689,690.4321142)(125.53399445,690.31212036)
\curveto(125.58398676,690.22211441)(125.62398672,690.1271145)(125.65399445,690.02712036)
\curveto(125.69398665,689.93711469)(125.73398661,689.83711479)(125.77399445,689.72712036)
\curveto(125.82398652,689.59711503)(125.86398648,689.46211517)(125.89399445,689.32212036)
\curveto(125.92398642,689.18211545)(125.95898639,689.04211559)(125.99899445,688.90212036)
\curveto(126.01898633,688.82211581)(126.02398632,688.7321159)(126.01399445,688.63212036)
\curveto(126.01398633,688.54211609)(126.02398632,688.45711617)(126.04399445,688.37712036)
\lineto(126.04399445,688.21212036)
\moveto(123.79399445,689.09712036)
\curveto(123.86398848,689.19711543)(123.86898848,689.31711531)(123.80899445,689.45712036)
\curveto(123.75898859,689.60711502)(123.71898863,689.71711491)(123.68899445,689.78712036)
\curveto(123.5489888,690.05711457)(123.36398898,690.26211437)(123.13399445,690.40212036)
\curveto(122.90398944,690.55211408)(122.58398976,690.632114)(122.17399445,690.64212036)
\curveto(122.1439902,690.62211401)(122.10899024,690.61711401)(122.06899445,690.62712036)
\curveto(122.02899032,690.63711399)(121.99399035,690.63711399)(121.96399445,690.62712036)
\curveto(121.91399043,690.60711402)(121.85899049,690.59211404)(121.79899445,690.58212036)
\curveto(121.73899061,690.58211405)(121.68399066,690.57211406)(121.63399445,690.55212036)
\curveto(121.19399115,690.41211422)(120.86899148,690.13711449)(120.65899445,689.72712036)
\curveto(120.63899171,689.68711494)(120.61399173,689.632115)(120.58399445,689.56212036)
\curveto(120.56399178,689.50211513)(120.5489918,689.43711519)(120.53899445,689.36712036)
\curveto(120.52899182,689.30711532)(120.52899182,689.24711538)(120.53899445,689.18712036)
\curveto(120.55899179,689.1271155)(120.59399175,689.07711555)(120.64399445,689.03712036)
\curveto(120.72399162,688.98711564)(120.83399151,688.96211567)(120.97399445,688.96212036)
\lineto(121.37899445,688.96212036)
\lineto(123.04399445,688.96212036)
\lineto(123.47899445,688.96212036)
\curveto(123.63898871,688.97211566)(123.7439886,689.01711561)(123.79399445,689.09712036)
}
}
{
\newrgbcolor{curcolor}{0 0 0}
\pscustom[linestyle=none,fillstyle=solid,fillcolor=curcolor]
{
\newpath
\moveto(130.2622757,692.21712036)
\curveto(131.0122712,692.23711239)(131.66227055,692.15211248)(132.2122757,691.96212036)
\curveto(132.77226944,691.78211285)(133.19726901,691.46711316)(133.4872757,691.01712036)
\curveto(133.55726865,690.90711372)(133.61726859,690.79211384)(133.6672757,690.67212036)
\curveto(133.72726848,690.56211407)(133.77726843,690.43711419)(133.8172757,690.29712036)
\curveto(133.83726837,690.23711439)(133.84726836,690.17211446)(133.8472757,690.10212036)
\curveto(133.84726836,690.0321146)(133.83726837,689.97211466)(133.8172757,689.92212036)
\curveto(133.77726843,689.86211477)(133.72226849,689.82211481)(133.6522757,689.80212036)
\curveto(133.60226861,689.78211485)(133.54226867,689.77211486)(133.4722757,689.77212036)
\lineto(133.2622757,689.77212036)
\lineto(132.6022757,689.77212036)
\curveto(132.53226968,689.77211486)(132.46226975,689.76711486)(132.3922757,689.75712036)
\curveto(132.32226989,689.75711487)(132.25726995,689.76711486)(132.1972757,689.78712036)
\curveto(132.09727011,689.80711482)(132.02227019,689.84711478)(131.9722757,689.90712036)
\curveto(131.92227029,689.96711466)(131.87727033,690.0271146)(131.8372757,690.08712036)
\lineto(131.7172757,690.29712036)
\curveto(131.68727052,690.37711425)(131.63727057,690.44211419)(131.5672757,690.49212036)
\curveto(131.46727074,690.57211406)(131.36727084,690.632114)(131.2672757,690.67212036)
\curveto(131.17727103,690.71211392)(131.06227115,690.74711388)(130.9222757,690.77712036)
\curveto(130.85227136,690.79711383)(130.74727146,690.81211382)(130.6072757,690.82212036)
\curveto(130.47727173,690.8321138)(130.37727183,690.8271138)(130.3072757,690.80712036)
\lineto(130.2022757,690.80712036)
\lineto(130.0522757,690.77712036)
\curveto(130.0122722,690.77711385)(129.96727224,690.77211386)(129.9172757,690.76212036)
\curveto(129.74727246,690.71211392)(129.6072726,690.64211399)(129.4972757,690.55212036)
\curveto(129.39727281,690.47211416)(129.32727288,690.34711428)(129.2872757,690.17712036)
\curveto(129.26727294,690.10711452)(129.26727294,690.04211459)(129.2872757,689.98212036)
\curveto(129.3072729,689.92211471)(129.32727288,689.87211476)(129.3472757,689.83212036)
\curveto(129.41727279,689.71211492)(129.49727271,689.61711501)(129.5872757,689.54712036)
\curveto(129.68727252,689.47711515)(129.80227241,689.41711521)(129.9322757,689.36712036)
\curveto(130.12227209,689.28711534)(130.32727188,689.21711541)(130.5472757,689.15712036)
\lineto(131.2372757,689.00712036)
\curveto(131.47727073,688.96711566)(131.7072705,688.91711571)(131.9272757,688.85712036)
\curveto(132.15727005,688.80711582)(132.37226984,688.74211589)(132.5722757,688.66212036)
\curveto(132.66226955,688.62211601)(132.74726946,688.58711604)(132.8272757,688.55712036)
\curveto(132.91726929,688.53711609)(133.00226921,688.50211613)(133.0822757,688.45212036)
\curveto(133.27226894,688.3321163)(133.44226877,688.20211643)(133.5922757,688.06212036)
\curveto(133.75226846,687.92211671)(133.87726833,687.74711688)(133.9672757,687.53712036)
\curveto(133.99726821,687.46711716)(134.02226819,687.39711723)(134.0422757,687.32712036)
\curveto(134.06226815,687.25711737)(134.08226813,687.18211745)(134.1022757,687.10212036)
\curveto(134.1122681,687.04211759)(134.11726809,686.94711768)(134.1172757,686.81712036)
\curveto(134.12726808,686.69711793)(134.12726808,686.60211803)(134.1172757,686.53212036)
\lineto(134.1172757,686.45712036)
\curveto(134.09726811,686.39711823)(134.08226813,686.33711829)(134.0722757,686.27712036)
\curveto(134.07226814,686.2271184)(134.06726814,686.17711845)(134.0572757,686.12712036)
\curveto(133.98726822,685.8271188)(133.87726833,685.56211907)(133.7272757,685.33212036)
\curveto(133.56726864,685.09211954)(133.37226884,684.89711973)(133.1422757,684.74712036)
\curveto(132.9122693,684.59712003)(132.65226956,684.46712016)(132.3622757,684.35712036)
\curveto(132.25226996,684.30712032)(132.13227008,684.27212036)(132.0022757,684.25212036)
\curveto(131.88227033,684.2321204)(131.76227045,684.20712042)(131.6422757,684.17712036)
\curveto(131.55227066,684.15712047)(131.45727075,684.14712048)(131.3572757,684.14712036)
\curveto(131.26727094,684.13712049)(131.17727103,684.12212051)(131.0872757,684.10212036)
\lineto(130.8172757,684.10212036)
\curveto(130.75727145,684.08212055)(130.65227156,684.07212056)(130.5022757,684.07212036)
\curveto(130.36227185,684.07212056)(130.26227195,684.08212055)(130.2022757,684.10212036)
\curveto(130.17227204,684.10212053)(130.13727207,684.10712052)(130.0972757,684.11712036)
\lineto(129.9922757,684.11712036)
\curveto(129.87227234,684.13712049)(129.75227246,684.15212048)(129.6322757,684.16212036)
\curveto(129.5122727,684.17212046)(129.39727281,684.19212044)(129.2872757,684.22212036)
\curveto(128.89727331,684.3321203)(128.55227366,684.45712017)(128.2522757,684.59712036)
\curveto(127.95227426,684.74711988)(127.69727451,684.96711966)(127.4872757,685.25712036)
\curveto(127.34727486,685.44711918)(127.22727498,685.66711896)(127.1272757,685.91712036)
\curveto(127.1072751,685.97711865)(127.08727512,686.05711857)(127.0672757,686.15712036)
\curveto(127.04727516,686.20711842)(127.03227518,686.27711835)(127.0222757,686.36712036)
\curveto(127.0122752,686.45711817)(127.01727519,686.5321181)(127.0372757,686.59212036)
\curveto(127.06727514,686.66211797)(127.11727509,686.71211792)(127.1872757,686.74212036)
\curveto(127.23727497,686.76211787)(127.29727491,686.77211786)(127.3672757,686.77212036)
\lineto(127.5922757,686.77212036)
\lineto(128.2972757,686.77212036)
\lineto(128.5372757,686.77212036)
\curveto(128.61727359,686.77211786)(128.68727352,686.76211787)(128.7472757,686.74212036)
\curveto(128.85727335,686.70211793)(128.92727328,686.63711799)(128.9572757,686.54712036)
\curveto(128.99727321,686.45711817)(129.04227317,686.36211827)(129.0922757,686.26212036)
\curveto(129.1122731,686.21211842)(129.14727306,686.14711848)(129.1972757,686.06712036)
\curveto(129.25727295,685.98711864)(129.3072729,685.93711869)(129.3472757,685.91712036)
\curveto(129.46727274,685.81711881)(129.58227263,685.73711889)(129.6922757,685.67712036)
\curveto(129.80227241,685.627119)(129.94227227,685.57711905)(130.1122757,685.52712036)
\curveto(130.16227205,685.50711912)(130.212272,685.49711913)(130.2622757,685.49712036)
\curveto(130.3122719,685.50711912)(130.36227185,685.50711912)(130.4122757,685.49712036)
\curveto(130.49227172,685.47711915)(130.57727163,685.46711916)(130.6672757,685.46712036)
\curveto(130.76727144,685.47711915)(130.85227136,685.49211914)(130.9222757,685.51212036)
\curveto(130.97227124,685.52211911)(131.01727119,685.5271191)(131.0572757,685.52712036)
\curveto(131.1072711,685.5271191)(131.15727105,685.53711909)(131.2072757,685.55712036)
\curveto(131.34727086,685.60711902)(131.47227074,685.66711896)(131.5822757,685.73712036)
\curveto(131.70227051,685.80711882)(131.79727041,685.89711873)(131.8672757,686.00712036)
\curveto(131.91727029,686.08711854)(131.95727025,686.21211842)(131.9872757,686.38212036)
\curveto(132.0072702,686.45211818)(132.0072702,686.51711811)(131.9872757,686.57712036)
\curveto(131.96727024,686.63711799)(131.94727026,686.68711794)(131.9272757,686.72712036)
\curveto(131.85727035,686.86711776)(131.76727044,686.97211766)(131.6572757,687.04212036)
\curveto(131.55727065,687.11211752)(131.43727077,687.17711745)(131.2972757,687.23712036)
\curveto(131.1072711,687.31711731)(130.9072713,687.38211725)(130.6972757,687.43212036)
\curveto(130.48727172,687.48211715)(130.27727193,687.53711709)(130.0672757,687.59712036)
\curveto(129.98727222,687.61711701)(129.90227231,687.632117)(129.8122757,687.64212036)
\curveto(129.73227248,687.65211698)(129.65227256,687.66711696)(129.5722757,687.68712036)
\curveto(129.25227296,687.77711685)(128.94727326,687.86211677)(128.6572757,687.94212036)
\curveto(128.36727384,688.0321166)(128.10227411,688.16211647)(127.8622757,688.33212036)
\curveto(127.58227463,688.5321161)(127.37727483,688.80211583)(127.2472757,689.14212036)
\curveto(127.22727498,689.21211542)(127.207275,689.30711532)(127.1872757,689.42712036)
\curveto(127.16727504,689.49711513)(127.15227506,689.58211505)(127.1422757,689.68212036)
\curveto(127.13227508,689.78211485)(127.13727507,689.87211476)(127.1572757,689.95212036)
\curveto(127.17727503,690.00211463)(127.18227503,690.04211459)(127.1722757,690.07212036)
\curveto(127.16227505,690.11211452)(127.16727504,690.15711447)(127.1872757,690.20712036)
\curveto(127.207275,690.31711431)(127.22727498,690.41711421)(127.2472757,690.50712036)
\curveto(127.27727493,690.60711402)(127.3122749,690.70211393)(127.3522757,690.79212036)
\curveto(127.48227473,691.08211355)(127.66227455,691.31711331)(127.8922757,691.49712036)
\curveto(128.12227409,691.67711295)(128.38227383,691.82211281)(128.6722757,691.93212036)
\curveto(128.78227343,691.98211265)(128.89727331,692.01711261)(129.0172757,692.03712036)
\curveto(129.13727307,692.06711256)(129.26227295,692.09711253)(129.3922757,692.12712036)
\curveto(129.45227276,692.14711248)(129.5122727,692.15711247)(129.5722757,692.15712036)
\lineto(129.7522757,692.18712036)
\curveto(129.83227238,692.19711243)(129.91727229,692.20211243)(130.0072757,692.20212036)
\curveto(130.09727211,692.20211243)(130.18227203,692.20711242)(130.2622757,692.21712036)
}
}
{
\newrgbcolor{curcolor}{0 0 0}
\pscustom[linestyle=none,fillstyle=solid,fillcolor=curcolor]
{
\newpath
\moveto(136.39891632,694.31712036)
\lineto(137.40391632,694.31712036)
\curveto(137.55391334,694.31711031)(137.68391321,694.30711032)(137.79391632,694.28712036)
\curveto(137.91391298,694.27711035)(137.99891289,694.21711041)(138.04891632,694.10712036)
\curveto(138.06891282,694.05711057)(138.07891281,693.99711063)(138.07891632,693.92712036)
\lineto(138.07891632,693.71712036)
\lineto(138.07891632,693.04212036)
\curveto(138.07891281,692.99211164)(138.07391282,692.9321117)(138.06391632,692.86212036)
\curveto(138.06391283,692.80211183)(138.06891282,692.74711188)(138.07891632,692.69712036)
\lineto(138.07891632,692.53212036)
\curveto(138.07891281,692.45211218)(138.08391281,692.37711225)(138.09391632,692.30712036)
\curveto(138.10391279,692.24711238)(138.12891276,692.19211244)(138.16891632,692.14212036)
\curveto(138.23891265,692.05211258)(138.36391253,692.00211263)(138.54391632,691.99212036)
\lineto(139.08391632,691.99212036)
\lineto(139.26391632,691.99212036)
\curveto(139.32391157,691.99211264)(139.37891151,691.98211265)(139.42891632,691.96212036)
\curveto(139.53891135,691.91211272)(139.59891129,691.82211281)(139.60891632,691.69212036)
\curveto(139.62891126,691.56211307)(139.63891125,691.41711321)(139.63891632,691.25712036)
\lineto(139.63891632,691.04712036)
\curveto(139.64891124,690.97711365)(139.64391125,690.91711371)(139.62391632,690.86712036)
\curveto(139.57391132,690.70711392)(139.46891142,690.62211401)(139.30891632,690.61212036)
\curveto(139.14891174,690.60211403)(138.96891192,690.59711403)(138.76891632,690.59712036)
\lineto(138.63391632,690.59712036)
\curveto(138.5939123,690.60711402)(138.55891233,690.60711402)(138.52891632,690.59712036)
\curveto(138.4889124,690.58711404)(138.45391244,690.58211405)(138.42391632,690.58212036)
\curveto(138.3939125,690.59211404)(138.36391253,690.58711404)(138.33391632,690.56712036)
\curveto(138.25391264,690.54711408)(138.1939127,690.50211413)(138.15391632,690.43212036)
\curveto(138.12391277,690.37211426)(138.09891279,690.29711433)(138.07891632,690.20712036)
\curveto(138.06891282,690.15711447)(138.06891282,690.10211453)(138.07891632,690.04212036)
\curveto(138.0889128,689.98211465)(138.0889128,689.9271147)(138.07891632,689.87712036)
\lineto(138.07891632,688.94712036)
\lineto(138.07891632,687.19212036)
\curveto(138.07891281,686.94211769)(138.08391281,686.72211791)(138.09391632,686.53212036)
\curveto(138.11391278,686.35211828)(138.17891271,686.19211844)(138.28891632,686.05212036)
\curveto(138.33891255,685.99211864)(138.40391249,685.94711868)(138.48391632,685.91712036)
\lineto(138.75391632,685.85712036)
\curveto(138.78391211,685.84711878)(138.81391208,685.84211879)(138.84391632,685.84212036)
\curveto(138.88391201,685.85211878)(138.91391198,685.85211878)(138.93391632,685.84212036)
\lineto(139.09891632,685.84212036)
\curveto(139.20891168,685.84211879)(139.30391159,685.83711879)(139.38391632,685.82712036)
\curveto(139.46391143,685.81711881)(139.52891136,685.77711885)(139.57891632,685.70712036)
\curveto(139.61891127,685.64711898)(139.63891125,685.56711906)(139.63891632,685.46712036)
\lineto(139.63891632,685.18212036)
\curveto(139.63891125,684.97211966)(139.63391126,684.77711985)(139.62391632,684.59712036)
\curveto(139.62391127,684.4271202)(139.54391135,684.31212032)(139.38391632,684.25212036)
\curveto(139.33391156,684.2321204)(139.2889116,684.2271204)(139.24891632,684.23712036)
\curveto(139.20891168,684.23712039)(139.16391173,684.2271204)(139.11391632,684.20712036)
\lineto(138.96391632,684.20712036)
\curveto(138.94391195,684.20712042)(138.91391198,684.21212042)(138.87391632,684.22212036)
\curveto(138.83391206,684.22212041)(138.79891209,684.21712041)(138.76891632,684.20712036)
\curveto(138.71891217,684.19712043)(138.66391223,684.19712043)(138.60391632,684.20712036)
\lineto(138.45391632,684.20712036)
\lineto(138.30391632,684.20712036)
\curveto(138.25391264,684.19712043)(138.20891268,684.19712043)(138.16891632,684.20712036)
\lineto(138.00391632,684.20712036)
\curveto(137.95391294,684.21712041)(137.89891299,684.22212041)(137.83891632,684.22212036)
\curveto(137.77891311,684.22212041)(137.72391317,684.2271204)(137.67391632,684.23712036)
\curveto(137.60391329,684.24712038)(137.53891335,684.25712037)(137.47891632,684.26712036)
\lineto(137.29891632,684.29712036)
\curveto(137.1889137,684.3271203)(137.08391381,684.36212027)(136.98391632,684.40212036)
\curveto(136.88391401,684.44212019)(136.7889141,684.48712014)(136.69891632,684.53712036)
\lineto(136.60891632,684.59712036)
\curveto(136.57891431,684.62712)(136.54391435,684.65711997)(136.50391632,684.68712036)
\curveto(136.48391441,684.70711992)(136.45891443,684.7271199)(136.42891632,684.74712036)
\lineto(136.35391632,684.82212036)
\curveto(136.21391468,685.01211962)(136.10891478,685.22211941)(136.03891632,685.45212036)
\curveto(136.01891487,685.49211914)(136.00891488,685.5271191)(136.00891632,685.55712036)
\curveto(136.01891487,685.59711903)(136.01891487,685.64211899)(136.00891632,685.69212036)
\curveto(135.99891489,685.71211892)(135.9939149,685.73711889)(135.99391632,685.76712036)
\curveto(135.9939149,685.79711883)(135.9889149,685.82211881)(135.97891632,685.84212036)
\lineto(135.97891632,685.99212036)
\curveto(135.96891492,686.0321186)(135.96391493,686.07711855)(135.96391632,686.12712036)
\curveto(135.97391492,686.17711845)(135.97891491,686.2271184)(135.97891632,686.27712036)
\lineto(135.97891632,686.84712036)
\lineto(135.97891632,689.08212036)
\lineto(135.97891632,689.87712036)
\lineto(135.97891632,690.08712036)
\curveto(135.9889149,690.15711447)(135.98391491,690.22211441)(135.96391632,690.28212036)
\curveto(135.92391497,690.42211421)(135.85391504,690.51211412)(135.75391632,690.55212036)
\curveto(135.64391525,690.60211403)(135.50391539,690.61711401)(135.33391632,690.59712036)
\curveto(135.16391573,690.57711405)(135.01891587,690.59211404)(134.89891632,690.64212036)
\curveto(134.81891607,690.67211396)(134.76891612,690.71711391)(134.74891632,690.77712036)
\curveto(134.72891616,690.83711379)(134.70891618,690.91211372)(134.68891632,691.00212036)
\lineto(134.68891632,691.31712036)
\curveto(134.6889162,691.49711313)(134.69891619,691.64211299)(134.71891632,691.75212036)
\curveto(134.73891615,691.86211277)(134.82391607,691.93711269)(134.97391632,691.97712036)
\curveto(135.01391588,691.99711263)(135.05391584,692.00211263)(135.09391632,691.99212036)
\lineto(135.22891632,691.99212036)
\curveto(135.37891551,691.99211264)(135.51891537,691.99711263)(135.64891632,692.00712036)
\curveto(135.77891511,692.0271126)(135.86891502,692.08711254)(135.91891632,692.18712036)
\curveto(135.94891494,692.25711237)(135.96391493,692.33711229)(135.96391632,692.42712036)
\curveto(135.97391492,692.51711211)(135.97891491,692.60711202)(135.97891632,692.69712036)
\lineto(135.97891632,693.62712036)
\lineto(135.97891632,693.88212036)
\curveto(135.97891491,693.97211066)(135.9889149,694.04711058)(136.00891632,694.10712036)
\curveto(136.05891483,694.20711042)(136.13391476,694.27211036)(136.23391632,694.30212036)
\curveto(136.25391464,694.31211032)(136.27891461,694.31211032)(136.30891632,694.30212036)
\curveto(136.34891454,694.30211033)(136.37891451,694.30711032)(136.39891632,694.31712036)
}
}
{
\newrgbcolor{curcolor}{0 0 0}
\pscustom[linestyle=none,fillstyle=solid,fillcolor=curcolor]
{
\newpath
\moveto(147.67235382,684.86712036)
\curveto(147.69234597,684.75711987)(147.70234596,684.64711998)(147.70235382,684.53712036)
\curveto(147.71234595,684.4271202)(147.662346,684.35212028)(147.55235382,684.31212036)
\curveto(147.49234617,684.28212035)(147.42234624,684.26712036)(147.34235382,684.26712036)
\lineto(147.10235382,684.26712036)
\lineto(146.29235382,684.26712036)
\lineto(146.02235382,684.26712036)
\curveto(145.94234772,684.27712035)(145.87734779,684.30212033)(145.82735382,684.34212036)
\curveto(145.75734791,684.38212025)(145.70234796,684.43712019)(145.66235382,684.50712036)
\curveto(145.63234803,684.58712004)(145.58734808,684.65211998)(145.52735382,684.70212036)
\curveto(145.50734816,684.72211991)(145.48234818,684.73711989)(145.45235382,684.74712036)
\curveto(145.42234824,684.76711986)(145.38234828,684.77211986)(145.33235382,684.76212036)
\curveto(145.28234838,684.74211989)(145.23234843,684.71711991)(145.18235382,684.68712036)
\curveto(145.14234852,684.65711997)(145.09734857,684.63212)(145.04735382,684.61212036)
\curveto(144.99734867,684.57212006)(144.94234872,684.53712009)(144.88235382,684.50712036)
\lineto(144.70235382,684.41712036)
\curveto(144.57234909,684.35712027)(144.43734923,684.30712032)(144.29735382,684.26712036)
\curveto(144.15734951,684.23712039)(144.01234965,684.20212043)(143.86235382,684.16212036)
\curveto(143.79234987,684.14212049)(143.72234994,684.1321205)(143.65235382,684.13212036)
\curveto(143.59235007,684.12212051)(143.52735014,684.11212052)(143.45735382,684.10212036)
\lineto(143.36735382,684.10212036)
\curveto(143.33735033,684.09212054)(143.30735036,684.08712054)(143.27735382,684.08712036)
\lineto(143.11235382,684.08712036)
\curveto(143.01235065,684.06712056)(142.91235075,684.06712056)(142.81235382,684.08712036)
\lineto(142.67735382,684.08712036)
\curveto(142.60735106,684.10712052)(142.53735113,684.11712051)(142.46735382,684.11712036)
\curveto(142.40735126,684.10712052)(142.34735132,684.11212052)(142.28735382,684.13212036)
\curveto(142.18735148,684.15212048)(142.09235157,684.17212046)(142.00235382,684.19212036)
\curveto(141.91235175,684.20212043)(141.82735184,684.2271204)(141.74735382,684.26712036)
\curveto(141.45735221,684.37712025)(141.20735246,684.51712011)(140.99735382,684.68712036)
\curveto(140.79735287,684.86711976)(140.63735303,685.10211953)(140.51735382,685.39212036)
\curveto(140.48735318,685.46211917)(140.45735321,685.53711909)(140.42735382,685.61712036)
\curveto(140.40735326,685.69711893)(140.38735328,685.78211885)(140.36735382,685.87212036)
\curveto(140.34735332,685.92211871)(140.33735333,685.97211866)(140.33735382,686.02212036)
\curveto(140.34735332,686.07211856)(140.34735332,686.12211851)(140.33735382,686.17212036)
\curveto(140.32735334,686.20211843)(140.31735335,686.26211837)(140.30735382,686.35212036)
\curveto(140.30735336,686.45211818)(140.31235335,686.52211811)(140.32235382,686.56212036)
\curveto(140.34235332,686.66211797)(140.35235331,686.74711788)(140.35235382,686.81712036)
\lineto(140.44235382,687.14712036)
\curveto(140.47235319,687.26711736)(140.51235315,687.37211726)(140.56235382,687.46212036)
\curveto(140.73235293,687.75211688)(140.92735274,687.97211666)(141.14735382,688.12212036)
\curveto(141.3673523,688.27211636)(141.64735202,688.40211623)(141.98735382,688.51212036)
\curveto(142.11735155,688.56211607)(142.25235141,688.59711603)(142.39235382,688.61712036)
\curveto(142.53235113,688.63711599)(142.67235099,688.66211597)(142.81235382,688.69212036)
\curveto(142.89235077,688.71211592)(142.97735069,688.72211591)(143.06735382,688.72212036)
\curveto(143.15735051,688.7321159)(143.24735042,688.74711588)(143.33735382,688.76712036)
\curveto(143.40735026,688.78711584)(143.47735019,688.79211584)(143.54735382,688.78212036)
\curveto(143.61735005,688.78211585)(143.69234997,688.79211584)(143.77235382,688.81212036)
\curveto(143.84234982,688.8321158)(143.91234975,688.84211579)(143.98235382,688.84212036)
\curveto(144.05234961,688.84211579)(144.12734954,688.85211578)(144.20735382,688.87212036)
\curveto(144.41734925,688.92211571)(144.60734906,688.96211567)(144.77735382,688.99212036)
\curveto(144.95734871,689.0321156)(145.11734855,689.12211551)(145.25735382,689.26212036)
\curveto(145.34734832,689.35211528)(145.40734826,689.45211518)(145.43735382,689.56212036)
\curveto(145.44734822,689.59211504)(145.44734822,689.61711501)(145.43735382,689.63712036)
\curveto(145.43734823,689.65711497)(145.44234822,689.67711495)(145.45235382,689.69712036)
\curveto(145.4623482,689.71711491)(145.4673482,689.74711488)(145.46735382,689.78712036)
\lineto(145.46735382,689.87712036)
\lineto(145.43735382,689.99712036)
\curveto(145.43734823,690.03711459)(145.43234823,690.07211456)(145.42235382,690.10212036)
\curveto(145.32234834,690.40211423)(145.11234855,690.60711402)(144.79235382,690.71712036)
\curveto(144.70234896,690.74711388)(144.59234907,690.76711386)(144.46235382,690.77712036)
\curveto(144.34234932,690.79711383)(144.21734945,690.80211383)(144.08735382,690.79212036)
\curveto(143.95734971,690.79211384)(143.83234983,690.78211385)(143.71235382,690.76212036)
\curveto(143.59235007,690.74211389)(143.48735018,690.71711391)(143.39735382,690.68712036)
\curveto(143.33735033,690.66711396)(143.27735039,690.63711399)(143.21735382,690.59712036)
\curveto(143.1673505,690.56711406)(143.11735055,690.5321141)(143.06735382,690.49212036)
\curveto(143.01735065,690.45211418)(142.9623507,690.39711423)(142.90235382,690.32712036)
\curveto(142.85235081,690.25711437)(142.81735085,690.19211444)(142.79735382,690.13212036)
\curveto(142.74735092,690.0321146)(142.70235096,689.93711469)(142.66235382,689.84712036)
\curveto(142.63235103,689.75711487)(142.5623511,689.69711493)(142.45235382,689.66712036)
\curveto(142.37235129,689.64711498)(142.28735138,689.63711499)(142.19735382,689.63712036)
\lineto(141.92735382,689.63712036)
\lineto(141.35735382,689.63712036)
\curveto(141.30735236,689.63711499)(141.25735241,689.632115)(141.20735382,689.62212036)
\curveto(141.15735251,689.62211501)(141.11235255,689.627115)(141.07235382,689.63712036)
\lineto(140.93735382,689.63712036)
\curveto(140.91735275,689.64711498)(140.89235277,689.65211498)(140.86235382,689.65212036)
\curveto(140.83235283,689.65211498)(140.80735286,689.66211497)(140.78735382,689.68212036)
\curveto(140.70735296,689.70211493)(140.65235301,689.76711486)(140.62235382,689.87712036)
\curveto(140.61235305,689.9271147)(140.61235305,689.97711465)(140.62235382,690.02712036)
\curveto(140.63235303,690.07711455)(140.64235302,690.12211451)(140.65235382,690.16212036)
\curveto(140.68235298,690.27211436)(140.71235295,690.37211426)(140.74235382,690.46212036)
\curveto(140.78235288,690.56211407)(140.82735284,690.65211398)(140.87735382,690.73212036)
\lineto(140.96735382,690.88212036)
\lineto(141.05735382,691.03212036)
\curveto(141.13735253,691.14211349)(141.23735243,691.24711338)(141.35735382,691.34712036)
\curveto(141.37735229,691.35711327)(141.40735226,691.38211325)(141.44735382,691.42212036)
\curveto(141.49735217,691.46211317)(141.54235212,691.49711313)(141.58235382,691.52712036)
\curveto(141.62235204,691.55711307)(141.667352,691.58711304)(141.71735382,691.61712036)
\curveto(141.88735178,691.7271129)(142.0673516,691.81211282)(142.25735382,691.87212036)
\curveto(142.44735122,691.94211269)(142.64235102,692.00711262)(142.84235382,692.06712036)
\curveto(142.9623507,692.09711253)(143.08735058,692.11711251)(143.21735382,692.12712036)
\curveto(143.34735032,692.13711249)(143.47735019,692.15711247)(143.60735382,692.18712036)
\curveto(143.64735002,692.19711243)(143.70734996,692.19711243)(143.78735382,692.18712036)
\curveto(143.87734979,692.17711245)(143.93234973,692.18211245)(143.95235382,692.20212036)
\curveto(144.3623493,692.21211242)(144.75234891,692.19711243)(145.12235382,692.15712036)
\curveto(145.50234816,692.11711251)(145.84234782,692.04211259)(146.14235382,691.93212036)
\curveto(146.45234721,691.82211281)(146.71734695,691.67211296)(146.93735382,691.48212036)
\curveto(147.15734651,691.30211333)(147.32734634,691.06711356)(147.44735382,690.77712036)
\curveto(147.51734615,690.60711402)(147.55734611,690.41211422)(147.56735382,690.19212036)
\curveto(147.57734609,689.97211466)(147.58234608,689.74711488)(147.58235382,689.51712036)
\lineto(147.58235382,686.17212036)
\lineto(147.58235382,685.58712036)
\curveto(147.58234608,685.39711923)(147.60234606,685.22211941)(147.64235382,685.06212036)
\curveto(147.65234601,685.0321196)(147.65734601,684.99711963)(147.65735382,684.95712036)
\curveto(147.65734601,684.9271197)(147.662346,684.89711973)(147.67235382,684.86712036)
\moveto(145.46735382,687.17712036)
\curveto(145.47734819,687.2271174)(145.48234818,687.28211735)(145.48235382,687.34212036)
\curveto(145.48234818,687.41211722)(145.47734819,687.47211716)(145.46735382,687.52212036)
\curveto(145.44734822,687.58211705)(145.43734823,687.63711699)(145.43735382,687.68712036)
\curveto(145.43734823,687.73711689)(145.41734825,687.77711685)(145.37735382,687.80712036)
\curveto(145.32734834,687.84711678)(145.25234841,687.86711676)(145.15235382,687.86712036)
\curveto(145.11234855,687.85711677)(145.07734859,687.84711678)(145.04735382,687.83712036)
\curveto(145.01734865,687.83711679)(144.98234868,687.8321168)(144.94235382,687.82212036)
\curveto(144.87234879,687.80211683)(144.79734887,687.78711684)(144.71735382,687.77712036)
\curveto(144.63734903,687.76711686)(144.55734911,687.75211688)(144.47735382,687.73212036)
\curveto(144.44734922,687.72211691)(144.40234926,687.71711691)(144.34235382,687.71712036)
\curveto(144.21234945,687.68711694)(144.08234958,687.66711696)(143.95235382,687.65712036)
\curveto(143.82234984,687.64711698)(143.69734997,687.62211701)(143.57735382,687.58212036)
\curveto(143.49735017,687.56211707)(143.42235024,687.54211709)(143.35235382,687.52212036)
\curveto(143.28235038,687.51211712)(143.21235045,687.49211714)(143.14235382,687.46212036)
\curveto(142.93235073,687.37211726)(142.75235091,687.23711739)(142.60235382,687.05712036)
\curveto(142.4623512,686.87711775)(142.41235125,686.627118)(142.45235382,686.30712036)
\curveto(142.47235119,686.13711849)(142.52735114,685.99711863)(142.61735382,685.88712036)
\curveto(142.68735098,685.77711885)(142.79235087,685.68711894)(142.93235382,685.61712036)
\curveto(143.07235059,685.55711907)(143.22235044,685.51211912)(143.38235382,685.48212036)
\curveto(143.55235011,685.45211918)(143.72734994,685.44211919)(143.90735382,685.45212036)
\curveto(144.09734957,685.47211916)(144.27234939,685.50711912)(144.43235382,685.55712036)
\curveto(144.69234897,685.63711899)(144.89734877,685.76211887)(145.04735382,685.93212036)
\curveto(145.19734847,686.11211852)(145.31234835,686.3321183)(145.39235382,686.59212036)
\curveto(145.41234825,686.66211797)(145.42234824,686.7321179)(145.42235382,686.80212036)
\curveto(145.43234823,686.88211775)(145.44734822,686.96211767)(145.46735382,687.04212036)
\lineto(145.46735382,687.17712036)
}
}
{
\newrgbcolor{curcolor}{0 0 0}
\pscustom[linestyle=none,fillstyle=solid,fillcolor=curcolor]
{
\newpath
\moveto(157.03563507,688.52712036)
\curveto(157.05562647,688.46711616)(157.06562646,688.36211627)(157.06563507,688.21212036)
\curveto(157.06562646,688.07211656)(157.06062647,687.97211666)(157.05063507,687.91212036)
\curveto(157.05062648,687.86211677)(157.04562648,687.81711681)(157.03563507,687.77712036)
\lineto(157.03563507,687.65712036)
\curveto(157.01562651,687.57711705)(157.00562652,687.49711713)(157.00563507,687.41712036)
\curveto(157.00562652,687.34711728)(156.99562653,687.27211736)(156.97563507,687.19212036)
\curveto(156.97562655,687.15211748)(156.96562656,687.08211755)(156.94563507,686.98212036)
\curveto(156.91562661,686.86211777)(156.88562664,686.73711789)(156.85563507,686.60712036)
\curveto(156.83562669,686.48711814)(156.80062673,686.37211826)(156.75063507,686.26212036)
\curveto(156.57062696,685.81211882)(156.34562718,685.42211921)(156.07563507,685.09212036)
\curveto(155.80562772,684.76211987)(155.45062808,684.50212013)(155.01063507,684.31212036)
\curveto(154.92062861,684.27212036)(154.8256287,684.24212039)(154.72563507,684.22212036)
\curveto(154.63562889,684.19212044)(154.53562899,684.16212047)(154.42563507,684.13212036)
\curveto(154.36562916,684.11212052)(154.30062923,684.10212053)(154.23063507,684.10212036)
\curveto(154.17062936,684.10212053)(154.11062942,684.09712053)(154.05063507,684.08712036)
\lineto(153.91563507,684.08712036)
\curveto(153.85562967,684.06712056)(153.77562975,684.06212057)(153.67563507,684.07212036)
\curveto(153.57562995,684.07212056)(153.49563003,684.08212055)(153.43563507,684.10212036)
\lineto(153.34563507,684.10212036)
\curveto(153.29563023,684.11212052)(153.24063029,684.12212051)(153.18063507,684.13212036)
\curveto(153.12063041,684.1321205)(153.06063047,684.13712049)(153.00063507,684.14712036)
\curveto(152.81063072,684.19712043)(152.63563089,684.24712038)(152.47563507,684.29712036)
\curveto(152.31563121,684.34712028)(152.16563136,684.41712021)(152.02563507,684.50712036)
\lineto(151.84563507,684.62712036)
\curveto(151.79563173,684.66711996)(151.74563178,684.71211992)(151.69563507,684.76212036)
\lineto(151.60563507,684.82212036)
\curveto(151.57563195,684.84211979)(151.54563198,684.85711977)(151.51563507,684.86712036)
\curveto(151.4256321,684.89711973)(151.37063216,684.87711975)(151.35063507,684.80712036)
\curveto(151.30063223,684.73711989)(151.26563226,684.65211998)(151.24563507,684.55212036)
\curveto(151.23563229,684.46212017)(151.20063233,684.39212024)(151.14063507,684.34212036)
\curveto(151.08063245,684.30212033)(151.01063252,684.27712035)(150.93063507,684.26712036)
\lineto(150.66063507,684.26712036)
\lineto(149.94063507,684.26712036)
\lineto(149.71563507,684.26712036)
\curveto(149.64563388,684.25712037)(149.58063395,684.26212037)(149.52063507,684.28212036)
\curveto(149.38063415,684.3321203)(149.30063423,684.42212021)(149.28063507,684.55212036)
\curveto(149.27063426,684.69211994)(149.26563426,684.84711978)(149.26563507,685.01712036)
\lineto(149.26563507,694.16712036)
\lineto(149.26563507,694.51212036)
\curveto(149.26563426,694.63211)(149.29063424,694.7271099)(149.34063507,694.79712036)
\curveto(149.38063415,694.86710976)(149.45063408,694.91210972)(149.55063507,694.93212036)
\curveto(149.57063396,694.94210969)(149.59063394,694.94210969)(149.61063507,694.93212036)
\curveto(149.64063389,694.9321097)(149.66563386,694.93710969)(149.68563507,694.94712036)
\lineto(150.63063507,694.94712036)
\curveto(150.81063272,694.94710968)(150.96563256,694.93710969)(151.09563507,694.91712036)
\curveto(151.2256323,694.90710972)(151.31063222,694.8321098)(151.35063507,694.69212036)
\curveto(151.38063215,694.59211004)(151.39063214,694.45711017)(151.38063507,694.28712036)
\curveto(151.37063216,694.1271105)(151.36563216,693.98711064)(151.36563507,693.86712036)
\lineto(151.36563507,692.23212036)
\lineto(151.36563507,691.90212036)
\curveto(151.36563216,691.79211284)(151.37563215,691.69711293)(151.39563507,691.61712036)
\curveto(151.40563212,691.56711306)(151.41563211,691.52211311)(151.42563507,691.48212036)
\curveto(151.43563209,691.45211318)(151.46063207,691.4321132)(151.50063507,691.42212036)
\curveto(151.52063201,691.40211323)(151.54563198,691.39211324)(151.57563507,691.39212036)
\curveto(151.61563191,691.39211324)(151.64563188,691.39711323)(151.66563507,691.40712036)
\curveto(151.73563179,691.44711318)(151.80063173,691.48711314)(151.86063507,691.52712036)
\curveto(151.92063161,691.57711305)(151.98563154,691.627113)(152.05563507,691.67712036)
\curveto(152.18563134,691.76711286)(152.32063121,691.84211279)(152.46063507,691.90212036)
\curveto(152.60063093,691.97211266)(152.75563077,692.0321126)(152.92563507,692.08212036)
\curveto(153.00563052,692.11211252)(153.08563044,692.1271125)(153.16563507,692.12712036)
\curveto(153.24563028,692.13711249)(153.3256302,692.15211248)(153.40563507,692.17212036)
\curveto(153.47563005,692.19211244)(153.55062998,692.20211243)(153.63063507,692.20212036)
\lineto(153.87063507,692.20212036)
\lineto(154.02063507,692.20212036)
\curveto(154.05062948,692.19211244)(154.08562944,692.18711244)(154.12563507,692.18712036)
\curveto(154.16562936,692.19711243)(154.20562932,692.19711243)(154.24563507,692.18712036)
\curveto(154.35562917,692.15711247)(154.45562907,692.1321125)(154.54563507,692.11212036)
\curveto(154.64562888,692.10211253)(154.74062879,692.07711255)(154.83063507,692.03712036)
\curveto(155.29062824,691.84711278)(155.66562786,691.60211303)(155.95563507,691.30212036)
\curveto(156.24562728,691.00211363)(156.49062704,690.627114)(156.69063507,690.17712036)
\curveto(156.74062679,690.05711457)(156.78062675,689.9321147)(156.81063507,689.80212036)
\curveto(156.85062668,689.67211496)(156.89062664,689.53711509)(156.93063507,689.39712036)
\curveto(156.95062658,689.3271153)(156.96062657,689.25711537)(156.96063507,689.18712036)
\curveto(156.97062656,689.1271155)(156.98562654,689.05711557)(157.00563507,688.97712036)
\curveto(157.0256265,688.9271157)(157.0306265,688.87211576)(157.02063507,688.81212036)
\curveto(157.02062651,688.75211588)(157.0256265,688.69211594)(157.03563507,688.63212036)
\lineto(157.03563507,688.52712036)
\moveto(154.81563507,687.11712036)
\curveto(154.84562868,687.21711741)(154.87062866,687.34211729)(154.89063507,687.49212036)
\curveto(154.92062861,687.64211699)(154.93562859,687.79211684)(154.93563507,687.94212036)
\curveto(154.94562858,688.10211653)(154.94562858,688.25711637)(154.93563507,688.40712036)
\curveto(154.93562859,688.56711606)(154.92062861,688.70211593)(154.89063507,688.81212036)
\curveto(154.86062867,688.91211572)(154.84062869,689.00711562)(154.83063507,689.09712036)
\curveto(154.82062871,689.18711544)(154.79562873,689.27211536)(154.75563507,689.35212036)
\curveto(154.61562891,689.70211493)(154.41562911,689.99711463)(154.15563507,690.23712036)
\curveto(153.90562962,690.48711414)(153.53562999,690.61211402)(153.04563507,690.61212036)
\curveto(153.00563052,690.61211402)(152.97063056,690.60711402)(152.94063507,690.59712036)
\lineto(152.83563507,690.59712036)
\curveto(152.76563076,690.57711405)(152.70063083,690.55711407)(152.64063507,690.53712036)
\curveto(152.58063095,690.5271141)(152.52063101,690.51211412)(152.46063507,690.49212036)
\curveto(152.17063136,690.36211427)(151.95063158,690.17711445)(151.80063507,689.93712036)
\curveto(151.65063188,689.70711492)(151.525632,689.44211519)(151.42563507,689.14212036)
\curveto(151.39563213,689.06211557)(151.37563215,688.97711565)(151.36563507,688.88712036)
\curveto(151.36563216,688.80711582)(151.35563217,688.7271159)(151.33563507,688.64712036)
\curveto(151.3256322,688.61711601)(151.32063221,688.56711606)(151.32063507,688.49712036)
\curveto(151.31063222,688.45711617)(151.30563222,688.41711621)(151.30563507,688.37712036)
\curveto(151.31563221,688.33711629)(151.31563221,688.29711633)(151.30563507,688.25712036)
\curveto(151.28563224,688.17711645)(151.28063225,688.06711656)(151.29063507,687.92712036)
\curveto(151.30063223,687.78711684)(151.31563221,687.68711694)(151.33563507,687.62712036)
\curveto(151.35563217,687.53711709)(151.36563216,687.45211718)(151.36563507,687.37212036)
\curveto(151.37563215,687.29211734)(151.39563213,687.21211742)(151.42563507,687.13212036)
\curveto(151.51563201,686.85211778)(151.62063191,686.60711802)(151.74063507,686.39712036)
\curveto(151.87063166,686.19711843)(152.05063148,686.0271186)(152.28063507,685.88712036)
\curveto(152.44063109,685.78711884)(152.60563092,685.71711891)(152.77563507,685.67712036)
\curveto(152.79563073,685.67711895)(152.81563071,685.67211896)(152.83563507,685.66212036)
\lineto(152.92563507,685.66212036)
\curveto(152.95563057,685.65211898)(153.00563052,685.64211899)(153.07563507,685.63212036)
\curveto(153.14563038,685.632119)(153.20563032,685.63711899)(153.25563507,685.64712036)
\curveto(153.35563017,685.66711896)(153.44563008,685.68211895)(153.52563507,685.69212036)
\curveto(153.61562991,685.71211892)(153.70062983,685.73711889)(153.78063507,685.76712036)
\curveto(154.06062947,685.89711873)(154.27562925,686.07711855)(154.42563507,686.30712036)
\curveto(154.58562894,686.53711809)(154.71562881,686.80711782)(154.81563507,687.11712036)
}
}
{
\newrgbcolor{curcolor}{0 0 0}
\pscustom[linestyle=none,fillstyle=solid,fillcolor=curcolor]
{
\newpath
\moveto(158.91555695,694.96212036)
\lineto(160.01055695,694.96212036)
\curveto(160.11055446,694.96210967)(160.20555437,694.95710967)(160.29555695,694.94712036)
\curveto(160.38555419,694.93710969)(160.45555412,694.90710972)(160.50555695,694.85712036)
\curveto(160.56555401,694.78710984)(160.59555398,694.69210994)(160.59555695,694.57212036)
\curveto(160.60555397,694.46211017)(160.61055396,694.34711028)(160.61055695,694.22712036)
\lineto(160.61055695,692.89212036)
\lineto(160.61055695,687.50712036)
\lineto(160.61055695,685.21212036)
\lineto(160.61055695,684.79212036)
\curveto(160.62055395,684.64211999)(160.60055397,684.5271201)(160.55055695,684.44712036)
\curveto(160.50055407,684.36712026)(160.41055416,684.31212032)(160.28055695,684.28212036)
\curveto(160.22055435,684.26212037)(160.15055442,684.25712037)(160.07055695,684.26712036)
\curveto(160.00055457,684.27712035)(159.93055464,684.28212035)(159.86055695,684.28212036)
\lineto(159.14055695,684.28212036)
\curveto(159.03055554,684.28212035)(158.93055564,684.28712034)(158.84055695,684.29712036)
\curveto(158.75055582,684.30712032)(158.6755559,684.33712029)(158.61555695,684.38712036)
\curveto(158.55555602,684.43712019)(158.52055605,684.51212012)(158.51055695,684.61212036)
\lineto(158.51055695,684.94212036)
\lineto(158.51055695,686.27712036)
\lineto(158.51055695,691.90212036)
\lineto(158.51055695,693.94212036)
\curveto(158.51055606,694.07211056)(158.50555607,694.2271104)(158.49555695,694.40712036)
\curveto(158.49555608,694.58711004)(158.52055605,694.71710991)(158.57055695,694.79712036)
\curveto(158.59055598,694.83710979)(158.61555596,694.86710976)(158.64555695,694.88712036)
\lineto(158.76555695,694.94712036)
\curveto(158.78555579,694.94710968)(158.81055576,694.94710968)(158.84055695,694.94712036)
\curveto(158.8705557,694.95710967)(158.89555568,694.96210967)(158.91555695,694.96212036)
}
}
{
\newrgbcolor{curcolor}{0 0 0}
\pscustom[linestyle=none,fillstyle=solid,fillcolor=curcolor]
{
\newpath
\moveto(169.63774445,688.21212036)
\curveto(169.65773628,688.1321165)(169.65773628,688.04211659)(169.63774445,687.94212036)
\curveto(169.61773632,687.84211679)(169.58273636,687.77711685)(169.53274445,687.74712036)
\curveto(169.48273646,687.70711692)(169.40773653,687.67711695)(169.30774445,687.65712036)
\curveto(169.21773672,687.64711698)(169.11273683,687.63711699)(168.99274445,687.62712036)
\lineto(168.64774445,687.62712036)
\curveto(168.5377374,687.63711699)(168.4377375,687.64211699)(168.34774445,687.64212036)
\lineto(164.68774445,687.64212036)
\lineto(164.47774445,687.64212036)
\curveto(164.41774152,687.64211699)(164.36274158,687.632117)(164.31274445,687.61212036)
\curveto(164.23274171,687.57211706)(164.18274176,687.5321171)(164.16274445,687.49212036)
\curveto(164.1427418,687.47211716)(164.12274182,687.4321172)(164.10274445,687.37212036)
\curveto(164.08274186,687.32211731)(164.07774186,687.27211736)(164.08774445,687.22212036)
\curveto(164.10774183,687.16211747)(164.11774182,687.10211753)(164.11774445,687.04212036)
\curveto(164.12774181,686.99211764)(164.1427418,686.93711769)(164.16274445,686.87712036)
\curveto(164.2427417,686.63711799)(164.3377416,686.43711819)(164.44774445,686.27712036)
\curveto(164.56774137,686.1271185)(164.72774121,685.99211864)(164.92774445,685.87212036)
\curveto(165.00774093,685.82211881)(165.08774085,685.78711884)(165.16774445,685.76712036)
\curveto(165.25774068,685.75711887)(165.34774059,685.73711889)(165.43774445,685.70712036)
\curveto(165.51774042,685.68711894)(165.62774031,685.67211896)(165.76774445,685.66212036)
\curveto(165.90774003,685.65211898)(166.02773991,685.65711897)(166.12774445,685.67712036)
\lineto(166.26274445,685.67712036)
\curveto(166.36273958,685.69711893)(166.45273949,685.71711891)(166.53274445,685.73712036)
\curveto(166.62273932,685.76711886)(166.70773923,685.79711883)(166.78774445,685.82712036)
\curveto(166.88773905,685.87711875)(166.99773894,685.94211869)(167.11774445,686.02212036)
\curveto(167.24773869,686.10211853)(167.3427386,686.18211845)(167.40274445,686.26212036)
\curveto(167.45273849,686.3321183)(167.50273844,686.39711823)(167.55274445,686.45712036)
\curveto(167.61273833,686.5271181)(167.68273826,686.57711805)(167.76274445,686.60712036)
\curveto(167.86273808,686.65711797)(167.98773795,686.67711795)(168.13774445,686.66712036)
\lineto(168.57274445,686.66712036)
\lineto(168.75274445,686.66712036)
\curveto(168.82273712,686.67711795)(168.88273706,686.67211796)(168.93274445,686.65212036)
\lineto(169.08274445,686.65212036)
\curveto(169.18273676,686.632118)(169.25273669,686.60711802)(169.29274445,686.57712036)
\curveto(169.33273661,686.55711807)(169.35273659,686.51211812)(169.35274445,686.44212036)
\curveto(169.36273658,686.37211826)(169.35773658,686.31211832)(169.33774445,686.26212036)
\curveto(169.28773665,686.12211851)(169.23273671,685.99711863)(169.17274445,685.88712036)
\curveto(169.11273683,685.77711885)(169.0427369,685.66711896)(168.96274445,685.55712036)
\curveto(168.7427372,685.2271194)(168.49273745,684.96211967)(168.21274445,684.76212036)
\curveto(167.93273801,684.56212007)(167.58273836,684.39212024)(167.16274445,684.25212036)
\curveto(167.05273889,684.21212042)(166.942739,684.18712044)(166.83274445,684.17712036)
\curveto(166.72273922,684.16712046)(166.60773933,684.14712048)(166.48774445,684.11712036)
\curveto(166.44773949,684.10712052)(166.40273954,684.10712052)(166.35274445,684.11712036)
\curveto(166.31273963,684.11712051)(166.27273967,684.11212052)(166.23274445,684.10212036)
\lineto(166.06774445,684.10212036)
\curveto(166.01773992,684.08212055)(165.95773998,684.07712055)(165.88774445,684.08712036)
\curveto(165.82774011,684.08712054)(165.77274017,684.09212054)(165.72274445,684.10212036)
\curveto(165.6427403,684.11212052)(165.57274037,684.11212052)(165.51274445,684.10212036)
\curveto(165.45274049,684.09212054)(165.38774055,684.09712053)(165.31774445,684.11712036)
\curveto(165.26774067,684.13712049)(165.21274073,684.14712048)(165.15274445,684.14712036)
\curveto(165.09274085,684.14712048)(165.0377409,684.15712047)(164.98774445,684.17712036)
\curveto(164.87774106,684.19712043)(164.76774117,684.22212041)(164.65774445,684.25212036)
\curveto(164.54774139,684.27212036)(164.44774149,684.30712032)(164.35774445,684.35712036)
\curveto(164.24774169,684.39712023)(164.1427418,684.4321202)(164.04274445,684.46212036)
\curveto(163.95274199,684.50212013)(163.86774207,684.54712008)(163.78774445,684.59712036)
\curveto(163.46774247,684.79711983)(163.18274276,685.0271196)(162.93274445,685.28712036)
\curveto(162.68274326,685.55711907)(162.47774346,685.86711876)(162.31774445,686.21712036)
\curveto(162.26774367,686.3271183)(162.22774371,686.43711819)(162.19774445,686.54712036)
\curveto(162.16774377,686.66711796)(162.12774381,686.78711784)(162.07774445,686.90712036)
\curveto(162.06774387,686.94711768)(162.06274388,686.98211765)(162.06274445,687.01212036)
\curveto(162.06274388,687.05211758)(162.05774388,687.09211754)(162.04774445,687.13212036)
\curveto(162.00774393,687.25211738)(161.98274396,687.38211725)(161.97274445,687.52212036)
\lineto(161.94274445,687.94212036)
\curveto(161.942744,687.99211664)(161.937744,688.04711658)(161.92774445,688.10712036)
\curveto(161.92774401,688.16711646)(161.93274401,688.22211641)(161.94274445,688.27212036)
\lineto(161.94274445,688.45212036)
\lineto(161.98774445,688.81212036)
\curveto(162.02774391,688.98211565)(162.06274388,689.14711548)(162.09274445,689.30712036)
\curveto(162.12274382,689.46711516)(162.16774377,689.61711501)(162.22774445,689.75712036)
\curveto(162.65774328,690.79711383)(163.38774255,691.5321131)(164.41774445,691.96212036)
\curveto(164.55774138,692.02211261)(164.69774124,692.06211257)(164.83774445,692.08212036)
\curveto(164.98774095,692.11211252)(165.1427408,692.14711248)(165.30274445,692.18712036)
\curveto(165.38274056,692.19711243)(165.45774048,692.20211243)(165.52774445,692.20212036)
\curveto(165.59774034,692.20211243)(165.67274027,692.20711242)(165.75274445,692.21712036)
\curveto(166.26273968,692.2271124)(166.69773924,692.16711246)(167.05774445,692.03712036)
\curveto(167.42773851,691.91711271)(167.75773818,691.75711287)(168.04774445,691.55712036)
\curveto(168.1377378,691.49711313)(168.22773771,691.4271132)(168.31774445,691.34712036)
\curveto(168.40773753,691.27711335)(168.48773745,691.20211343)(168.55774445,691.12212036)
\curveto(168.58773735,691.07211356)(168.62773731,691.0321136)(168.67774445,691.00212036)
\curveto(168.75773718,690.89211374)(168.83273711,690.77711385)(168.90274445,690.65712036)
\curveto(168.97273697,690.54711408)(169.04773689,690.4321142)(169.12774445,690.31212036)
\curveto(169.17773676,690.22211441)(169.21773672,690.1271145)(169.24774445,690.02712036)
\curveto(169.28773665,689.93711469)(169.32773661,689.83711479)(169.36774445,689.72712036)
\curveto(169.41773652,689.59711503)(169.45773648,689.46211517)(169.48774445,689.32212036)
\curveto(169.51773642,689.18211545)(169.55273639,689.04211559)(169.59274445,688.90212036)
\curveto(169.61273633,688.82211581)(169.61773632,688.7321159)(169.60774445,688.63212036)
\curveto(169.60773633,688.54211609)(169.61773632,688.45711617)(169.63774445,688.37712036)
\lineto(169.63774445,688.21212036)
\moveto(167.38774445,689.09712036)
\curveto(167.45773848,689.19711543)(167.46273848,689.31711531)(167.40274445,689.45712036)
\curveto(167.35273859,689.60711502)(167.31273863,689.71711491)(167.28274445,689.78712036)
\curveto(167.1427388,690.05711457)(166.95773898,690.26211437)(166.72774445,690.40212036)
\curveto(166.49773944,690.55211408)(166.17773976,690.632114)(165.76774445,690.64212036)
\curveto(165.7377402,690.62211401)(165.70274024,690.61711401)(165.66274445,690.62712036)
\curveto(165.62274032,690.63711399)(165.58774035,690.63711399)(165.55774445,690.62712036)
\curveto(165.50774043,690.60711402)(165.45274049,690.59211404)(165.39274445,690.58212036)
\curveto(165.33274061,690.58211405)(165.27774066,690.57211406)(165.22774445,690.55212036)
\curveto(164.78774115,690.41211422)(164.46274148,690.13711449)(164.25274445,689.72712036)
\curveto(164.23274171,689.68711494)(164.20774173,689.632115)(164.17774445,689.56212036)
\curveto(164.15774178,689.50211513)(164.1427418,689.43711519)(164.13274445,689.36712036)
\curveto(164.12274182,689.30711532)(164.12274182,689.24711538)(164.13274445,689.18712036)
\curveto(164.15274179,689.1271155)(164.18774175,689.07711555)(164.23774445,689.03712036)
\curveto(164.31774162,688.98711564)(164.42774151,688.96211567)(164.56774445,688.96212036)
\lineto(164.97274445,688.96212036)
\lineto(166.63774445,688.96212036)
\lineto(167.07274445,688.96212036)
\curveto(167.23273871,688.97211566)(167.3377386,689.01711561)(167.38774445,689.09712036)
}
}
{
\newrgbcolor{curcolor}{0 0 0}
\pscustom[linestyle=none,fillstyle=solid,fillcolor=curcolor]
{
\newpath
\moveto(174.4560257,692.21712036)
\curveto(175.26602054,692.23711239)(175.94101986,692.11711251)(176.4810257,691.85712036)
\curveto(177.03101877,691.59711303)(177.46601834,691.2271134)(177.7860257,690.74712036)
\curveto(177.94601786,690.50711412)(178.06601774,690.2321144)(178.1460257,689.92212036)
\curveto(178.16601764,689.87211476)(178.18101762,689.80711482)(178.1910257,689.72712036)
\curveto(178.21101759,689.64711498)(178.21101759,689.57711505)(178.1910257,689.51712036)
\curveto(178.15101765,689.40711522)(178.08101772,689.34211529)(177.9810257,689.32212036)
\curveto(177.88101792,689.31211532)(177.76101804,689.30711532)(177.6210257,689.30712036)
\lineto(176.8410257,689.30712036)
\lineto(176.5560257,689.30712036)
\curveto(176.46601934,689.30711532)(176.39101941,689.3271153)(176.3310257,689.36712036)
\curveto(176.25101955,689.40711522)(176.19601961,689.46711516)(176.1660257,689.54712036)
\curveto(176.13601967,689.63711499)(176.09601971,689.7271149)(176.0460257,689.81712036)
\curveto(175.98601982,689.9271147)(175.92101988,690.0271146)(175.8510257,690.11712036)
\curveto(175.78102002,690.20711442)(175.7010201,690.28711434)(175.6110257,690.35712036)
\curveto(175.47102033,690.44711418)(175.31602049,690.51711411)(175.1460257,690.56712036)
\curveto(175.08602072,690.58711404)(175.02602078,690.59711403)(174.9660257,690.59712036)
\curveto(174.9060209,690.59711403)(174.85102095,690.60711402)(174.8010257,690.62712036)
\lineto(174.6510257,690.62712036)
\curveto(174.45102135,690.627114)(174.29102151,690.60711402)(174.1710257,690.56712036)
\curveto(173.88102192,690.47711415)(173.64602216,690.33711429)(173.4660257,690.14712036)
\curveto(173.28602252,689.96711466)(173.14102266,689.74711488)(173.0310257,689.48712036)
\curveto(172.98102282,689.37711525)(172.94102286,689.25711537)(172.9110257,689.12712036)
\curveto(172.89102291,689.00711562)(172.86602294,688.87711575)(172.8360257,688.73712036)
\curveto(172.82602298,688.69711593)(172.82102298,688.65711597)(172.8210257,688.61712036)
\curveto(172.82102298,688.57711605)(172.81602299,688.53711609)(172.8060257,688.49712036)
\curveto(172.78602302,688.39711623)(172.77602303,688.25711637)(172.7760257,688.07712036)
\curveto(172.78602302,687.89711673)(172.801023,687.75711687)(172.8210257,687.65712036)
\curveto(172.82102298,687.57711705)(172.82602298,687.52211711)(172.8360257,687.49212036)
\curveto(172.85602295,687.42211721)(172.86602294,687.35211728)(172.8660257,687.28212036)
\curveto(172.87602293,687.21211742)(172.89102291,687.14211749)(172.9110257,687.07212036)
\curveto(172.99102281,686.84211779)(173.08602272,686.632118)(173.1960257,686.44212036)
\curveto(173.3060225,686.25211838)(173.44602236,686.09211854)(173.6160257,685.96212036)
\curveto(173.65602215,685.9321187)(173.71602209,685.89711873)(173.7960257,685.85712036)
\curveto(173.9060219,685.78711884)(174.01602179,685.74211889)(174.1260257,685.72212036)
\curveto(174.24602156,685.70211893)(174.39102141,685.68211895)(174.5610257,685.66212036)
\lineto(174.6510257,685.66212036)
\curveto(174.69102111,685.66211897)(174.72102108,685.66711896)(174.7410257,685.67712036)
\lineto(174.8760257,685.67712036)
\curveto(174.94602086,685.69711893)(175.01102079,685.71211892)(175.0710257,685.72212036)
\curveto(175.14102066,685.74211889)(175.2060206,685.76211887)(175.2660257,685.78212036)
\curveto(175.56602024,685.91211872)(175.79602001,686.10211853)(175.9560257,686.35212036)
\curveto(175.99601981,686.40211823)(176.03101977,686.45711817)(176.0610257,686.51712036)
\curveto(176.09101971,686.58711804)(176.11601969,686.64711798)(176.1360257,686.69712036)
\curveto(176.17601963,686.80711782)(176.21101959,686.90211773)(176.2410257,686.98212036)
\curveto(176.27101953,687.07211756)(176.34101946,687.14211749)(176.4510257,687.19212036)
\curveto(176.54101926,687.2321174)(176.68601912,687.24711738)(176.8860257,687.23712036)
\lineto(177.3810257,687.23712036)
\lineto(177.5910257,687.23712036)
\curveto(177.67101813,687.24711738)(177.73601807,687.24211739)(177.7860257,687.22212036)
\lineto(177.9060257,687.22212036)
\lineto(178.0260257,687.19212036)
\curveto(178.06601774,687.19211744)(178.09601771,687.18211745)(178.1160257,687.16212036)
\curveto(178.16601764,687.12211751)(178.19601761,687.06211757)(178.2060257,686.98212036)
\curveto(178.22601758,686.91211772)(178.22601758,686.83711779)(178.2060257,686.75712036)
\curveto(178.11601769,686.4271182)(178.0060178,686.1321185)(177.8760257,685.87212036)
\curveto(177.46601834,685.10211953)(176.81101899,684.56712006)(175.9110257,684.26712036)
\curveto(175.81101999,684.23712039)(175.7060201,684.21712041)(175.5960257,684.20712036)
\curveto(175.48602032,684.18712044)(175.37602043,684.16212047)(175.2660257,684.13212036)
\curveto(175.2060206,684.12212051)(175.14602066,684.11712051)(175.0860257,684.11712036)
\curveto(175.02602078,684.11712051)(174.96602084,684.11212052)(174.9060257,684.10212036)
\lineto(174.7410257,684.10212036)
\curveto(174.69102111,684.08212055)(174.61602119,684.07712055)(174.5160257,684.08712036)
\curveto(174.41602139,684.08712054)(174.34102146,684.09212054)(174.2910257,684.10212036)
\curveto(174.21102159,684.12212051)(174.13602167,684.1321205)(174.0660257,684.13212036)
\curveto(174.0060218,684.12212051)(173.94102186,684.1271205)(173.8710257,684.14712036)
\lineto(173.7210257,684.17712036)
\curveto(173.67102213,684.17712045)(173.62102218,684.18212045)(173.5710257,684.19212036)
\curveto(173.46102234,684.22212041)(173.35602245,684.25212038)(173.2560257,684.28212036)
\curveto(173.15602265,684.31212032)(173.06102274,684.34712028)(172.9710257,684.38712036)
\curveto(172.5010233,684.58712004)(172.1060237,684.84211979)(171.7860257,685.15212036)
\curveto(171.46602434,685.47211916)(171.2060246,685.86711876)(171.0060257,686.33712036)
\curveto(170.95602485,686.4271182)(170.91602489,686.52211811)(170.8860257,686.62212036)
\lineto(170.7960257,686.95212036)
\curveto(170.78602502,686.99211764)(170.78102502,687.0271176)(170.7810257,687.05712036)
\curveto(170.78102502,687.09711753)(170.77102503,687.14211749)(170.7510257,687.19212036)
\curveto(170.73102507,687.26211737)(170.72102508,687.3321173)(170.7210257,687.40212036)
\curveto(170.72102508,687.48211715)(170.71102509,687.55711707)(170.6910257,687.62712036)
\lineto(170.6910257,687.88212036)
\curveto(170.67102513,687.9321167)(170.66102514,687.98711664)(170.6610257,688.04712036)
\curveto(170.66102514,688.11711651)(170.67102513,688.17711645)(170.6910257,688.22712036)
\curveto(170.7010251,688.27711635)(170.7010251,688.32211631)(170.6910257,688.36212036)
\curveto(170.68102512,688.40211623)(170.68102512,688.44211619)(170.6910257,688.48212036)
\curveto(170.71102509,688.55211608)(170.71602509,688.61711601)(170.7060257,688.67712036)
\curveto(170.7060251,688.73711589)(170.71602509,688.79711583)(170.7360257,688.85712036)
\curveto(170.78602502,689.03711559)(170.82602498,689.20711542)(170.8560257,689.36712036)
\curveto(170.88602492,689.53711509)(170.93102487,689.70211493)(170.9910257,689.86212036)
\curveto(171.21102459,690.37211426)(171.48602432,690.79711383)(171.8160257,691.13712036)
\curveto(172.15602365,691.47711315)(172.58602322,691.75211288)(173.1060257,691.96212036)
\curveto(173.24602256,692.02211261)(173.39102241,692.06211257)(173.5410257,692.08212036)
\curveto(173.69102211,692.11211252)(173.84602196,692.14711248)(174.0060257,692.18712036)
\curveto(174.08602172,692.19711243)(174.16102164,692.20211243)(174.2310257,692.20212036)
\curveto(174.3010215,692.20211243)(174.37602143,692.20711242)(174.4560257,692.21712036)
}
}
{
\newrgbcolor{curcolor}{0 0 0}
\pscustom[linestyle=none,fillstyle=solid,fillcolor=curcolor]
{
\newpath
\moveto(181.59930695,694.85712036)
\curveto(181.669304,694.77710985)(181.70430396,694.65710997)(181.70430695,694.49712036)
\lineto(181.70430695,694.03212036)
\lineto(181.70430695,693.62712036)
\curveto(181.70430396,693.48711114)(181.669304,693.39211124)(181.59930695,693.34212036)
\curveto(181.53930413,693.29211134)(181.45930421,693.26211137)(181.35930695,693.25212036)
\curveto(181.2693044,693.24211139)(181.1693045,693.23711139)(181.05930695,693.23712036)
\lineto(180.21930695,693.23712036)
\curveto(180.10930556,693.23711139)(180.00930566,693.24211139)(179.91930695,693.25212036)
\curveto(179.83930583,693.26211137)(179.7693059,693.29211134)(179.70930695,693.34212036)
\curveto(179.669306,693.37211126)(179.63930603,693.4271112)(179.61930695,693.50712036)
\curveto(179.60930606,693.59711103)(179.59930607,693.69211094)(179.58930695,693.79212036)
\lineto(179.58930695,694.12212036)
\curveto(179.59930607,694.2321104)(179.60430606,694.3271103)(179.60430695,694.40712036)
\lineto(179.60430695,694.61712036)
\curveto(179.61430605,694.68710994)(179.63430603,694.74710988)(179.66430695,694.79712036)
\curveto(179.68430598,694.83710979)(179.70930596,694.86710976)(179.73930695,694.88712036)
\lineto(179.85930695,694.94712036)
\curveto(179.87930579,694.94710968)(179.90430576,694.94710968)(179.93430695,694.94712036)
\curveto(179.9643057,694.95710967)(179.98930568,694.96210967)(180.00930695,694.96212036)
\lineto(181.10430695,694.96212036)
\curveto(181.20430446,694.96210967)(181.29930437,694.95710967)(181.38930695,694.94712036)
\curveto(181.47930419,694.93710969)(181.54930412,694.90710972)(181.59930695,694.85712036)
\moveto(181.70430695,685.09212036)
\curveto(181.70430396,684.89211974)(181.69930397,684.72211991)(181.68930695,684.58212036)
\curveto(181.67930399,684.44212019)(181.58930408,684.34712028)(181.41930695,684.29712036)
\curveto(181.35930431,684.27712035)(181.29430437,684.26712036)(181.22430695,684.26712036)
\curveto(181.15430451,684.27712035)(181.07930459,684.28212035)(180.99930695,684.28212036)
\lineto(180.15930695,684.28212036)
\curveto(180.0693056,684.28212035)(179.97930569,684.28712034)(179.88930695,684.29712036)
\curveto(179.80930586,684.30712032)(179.74930592,684.33712029)(179.70930695,684.38712036)
\curveto(179.64930602,684.45712017)(179.61430605,684.54212009)(179.60430695,684.64212036)
\lineto(179.60430695,684.98712036)
\lineto(179.60430695,691.31712036)
\lineto(179.60430695,691.61712036)
\curveto(179.60430606,691.71711291)(179.62430604,691.79711283)(179.66430695,691.85712036)
\curveto(179.72430594,691.9271127)(179.80930586,691.97211266)(179.91930695,691.99212036)
\curveto(179.93930573,692.00211263)(179.9643057,692.00211263)(179.99430695,691.99212036)
\curveto(180.03430563,691.99211264)(180.0643056,691.99711263)(180.08430695,692.00712036)
\lineto(180.83430695,692.00712036)
\lineto(181.02930695,692.00712036)
\curveto(181.10930456,692.01711261)(181.17430449,692.01711261)(181.22430695,692.00712036)
\lineto(181.34430695,692.00712036)
\curveto(181.40430426,691.98711264)(181.45930421,691.97211266)(181.50930695,691.96212036)
\curveto(181.55930411,691.95211268)(181.59930407,691.92211271)(181.62930695,691.87212036)
\curveto(181.669304,691.82211281)(181.68930398,691.75211288)(181.68930695,691.66212036)
\curveto(181.69930397,691.57211306)(181.70430396,691.47711315)(181.70430695,691.37712036)
\lineto(181.70430695,685.09212036)
}
}
{
\newrgbcolor{curcolor}{0 0 0}
\pscustom[linestyle=none,fillstyle=solid,fillcolor=curcolor]
{
\newpath
\moveto(190.73149445,688.21212036)
\curveto(190.75148628,688.1321165)(190.75148628,688.04211659)(190.73149445,687.94212036)
\curveto(190.71148632,687.84211679)(190.67648636,687.77711685)(190.62649445,687.74712036)
\curveto(190.57648646,687.70711692)(190.50148653,687.67711695)(190.40149445,687.65712036)
\curveto(190.31148672,687.64711698)(190.20648683,687.63711699)(190.08649445,687.62712036)
\lineto(189.74149445,687.62712036)
\curveto(189.6314874,687.63711699)(189.5314875,687.64211699)(189.44149445,687.64212036)
\lineto(185.78149445,687.64212036)
\lineto(185.57149445,687.64212036)
\curveto(185.51149152,687.64211699)(185.45649158,687.632117)(185.40649445,687.61212036)
\curveto(185.32649171,687.57211706)(185.27649176,687.5321171)(185.25649445,687.49212036)
\curveto(185.2364918,687.47211716)(185.21649182,687.4321172)(185.19649445,687.37212036)
\curveto(185.17649186,687.32211731)(185.17149186,687.27211736)(185.18149445,687.22212036)
\curveto(185.20149183,687.16211747)(185.21149182,687.10211753)(185.21149445,687.04212036)
\curveto(185.22149181,686.99211764)(185.2364918,686.93711769)(185.25649445,686.87712036)
\curveto(185.3364917,686.63711799)(185.4314916,686.43711819)(185.54149445,686.27712036)
\curveto(185.66149137,686.1271185)(185.82149121,685.99211864)(186.02149445,685.87212036)
\curveto(186.10149093,685.82211881)(186.18149085,685.78711884)(186.26149445,685.76712036)
\curveto(186.35149068,685.75711887)(186.44149059,685.73711889)(186.53149445,685.70712036)
\curveto(186.61149042,685.68711894)(186.72149031,685.67211896)(186.86149445,685.66212036)
\curveto(187.00149003,685.65211898)(187.12148991,685.65711897)(187.22149445,685.67712036)
\lineto(187.35649445,685.67712036)
\curveto(187.45648958,685.69711893)(187.54648949,685.71711891)(187.62649445,685.73712036)
\curveto(187.71648932,685.76711886)(187.80148923,685.79711883)(187.88149445,685.82712036)
\curveto(187.98148905,685.87711875)(188.09148894,685.94211869)(188.21149445,686.02212036)
\curveto(188.34148869,686.10211853)(188.4364886,686.18211845)(188.49649445,686.26212036)
\curveto(188.54648849,686.3321183)(188.59648844,686.39711823)(188.64649445,686.45712036)
\curveto(188.70648833,686.5271181)(188.77648826,686.57711805)(188.85649445,686.60712036)
\curveto(188.95648808,686.65711797)(189.08148795,686.67711795)(189.23149445,686.66712036)
\lineto(189.66649445,686.66712036)
\lineto(189.84649445,686.66712036)
\curveto(189.91648712,686.67711795)(189.97648706,686.67211796)(190.02649445,686.65212036)
\lineto(190.17649445,686.65212036)
\curveto(190.27648676,686.632118)(190.34648669,686.60711802)(190.38649445,686.57712036)
\curveto(190.42648661,686.55711807)(190.44648659,686.51211812)(190.44649445,686.44212036)
\curveto(190.45648658,686.37211826)(190.45148658,686.31211832)(190.43149445,686.26212036)
\curveto(190.38148665,686.12211851)(190.32648671,685.99711863)(190.26649445,685.88712036)
\curveto(190.20648683,685.77711885)(190.1364869,685.66711896)(190.05649445,685.55712036)
\curveto(189.8364872,685.2271194)(189.58648745,684.96211967)(189.30649445,684.76212036)
\curveto(189.02648801,684.56212007)(188.67648836,684.39212024)(188.25649445,684.25212036)
\curveto(188.14648889,684.21212042)(188.036489,684.18712044)(187.92649445,684.17712036)
\curveto(187.81648922,684.16712046)(187.70148933,684.14712048)(187.58149445,684.11712036)
\curveto(187.54148949,684.10712052)(187.49648954,684.10712052)(187.44649445,684.11712036)
\curveto(187.40648963,684.11712051)(187.36648967,684.11212052)(187.32649445,684.10212036)
\lineto(187.16149445,684.10212036)
\curveto(187.11148992,684.08212055)(187.05148998,684.07712055)(186.98149445,684.08712036)
\curveto(186.92149011,684.08712054)(186.86649017,684.09212054)(186.81649445,684.10212036)
\curveto(186.7364903,684.11212052)(186.66649037,684.11212052)(186.60649445,684.10212036)
\curveto(186.54649049,684.09212054)(186.48149055,684.09712053)(186.41149445,684.11712036)
\curveto(186.36149067,684.13712049)(186.30649073,684.14712048)(186.24649445,684.14712036)
\curveto(186.18649085,684.14712048)(186.1314909,684.15712047)(186.08149445,684.17712036)
\curveto(185.97149106,684.19712043)(185.86149117,684.22212041)(185.75149445,684.25212036)
\curveto(185.64149139,684.27212036)(185.54149149,684.30712032)(185.45149445,684.35712036)
\curveto(185.34149169,684.39712023)(185.2364918,684.4321202)(185.13649445,684.46212036)
\curveto(185.04649199,684.50212013)(184.96149207,684.54712008)(184.88149445,684.59712036)
\curveto(184.56149247,684.79711983)(184.27649276,685.0271196)(184.02649445,685.28712036)
\curveto(183.77649326,685.55711907)(183.57149346,685.86711876)(183.41149445,686.21712036)
\curveto(183.36149367,686.3271183)(183.32149371,686.43711819)(183.29149445,686.54712036)
\curveto(183.26149377,686.66711796)(183.22149381,686.78711784)(183.17149445,686.90712036)
\curveto(183.16149387,686.94711768)(183.15649388,686.98211765)(183.15649445,687.01212036)
\curveto(183.15649388,687.05211758)(183.15149388,687.09211754)(183.14149445,687.13212036)
\curveto(183.10149393,687.25211738)(183.07649396,687.38211725)(183.06649445,687.52212036)
\lineto(183.03649445,687.94212036)
\curveto(183.036494,687.99211664)(183.031494,688.04711658)(183.02149445,688.10712036)
\curveto(183.02149401,688.16711646)(183.02649401,688.22211641)(183.03649445,688.27212036)
\lineto(183.03649445,688.45212036)
\lineto(183.08149445,688.81212036)
\curveto(183.12149391,688.98211565)(183.15649388,689.14711548)(183.18649445,689.30712036)
\curveto(183.21649382,689.46711516)(183.26149377,689.61711501)(183.32149445,689.75712036)
\curveto(183.75149328,690.79711383)(184.48149255,691.5321131)(185.51149445,691.96212036)
\curveto(185.65149138,692.02211261)(185.79149124,692.06211257)(185.93149445,692.08212036)
\curveto(186.08149095,692.11211252)(186.2364908,692.14711248)(186.39649445,692.18712036)
\curveto(186.47649056,692.19711243)(186.55149048,692.20211243)(186.62149445,692.20212036)
\curveto(186.69149034,692.20211243)(186.76649027,692.20711242)(186.84649445,692.21712036)
\curveto(187.35648968,692.2271124)(187.79148924,692.16711246)(188.15149445,692.03712036)
\curveto(188.52148851,691.91711271)(188.85148818,691.75711287)(189.14149445,691.55712036)
\curveto(189.2314878,691.49711313)(189.32148771,691.4271132)(189.41149445,691.34712036)
\curveto(189.50148753,691.27711335)(189.58148745,691.20211343)(189.65149445,691.12212036)
\curveto(189.68148735,691.07211356)(189.72148731,691.0321136)(189.77149445,691.00212036)
\curveto(189.85148718,690.89211374)(189.92648711,690.77711385)(189.99649445,690.65712036)
\curveto(190.06648697,690.54711408)(190.14148689,690.4321142)(190.22149445,690.31212036)
\curveto(190.27148676,690.22211441)(190.31148672,690.1271145)(190.34149445,690.02712036)
\curveto(190.38148665,689.93711469)(190.42148661,689.83711479)(190.46149445,689.72712036)
\curveto(190.51148652,689.59711503)(190.55148648,689.46211517)(190.58149445,689.32212036)
\curveto(190.61148642,689.18211545)(190.64648639,689.04211559)(190.68649445,688.90212036)
\curveto(190.70648633,688.82211581)(190.71148632,688.7321159)(190.70149445,688.63212036)
\curveto(190.70148633,688.54211609)(190.71148632,688.45711617)(190.73149445,688.37712036)
\lineto(190.73149445,688.21212036)
\moveto(188.48149445,689.09712036)
\curveto(188.55148848,689.19711543)(188.55648848,689.31711531)(188.49649445,689.45712036)
\curveto(188.44648859,689.60711502)(188.40648863,689.71711491)(188.37649445,689.78712036)
\curveto(188.2364888,690.05711457)(188.05148898,690.26211437)(187.82149445,690.40212036)
\curveto(187.59148944,690.55211408)(187.27148976,690.632114)(186.86149445,690.64212036)
\curveto(186.8314902,690.62211401)(186.79649024,690.61711401)(186.75649445,690.62712036)
\curveto(186.71649032,690.63711399)(186.68149035,690.63711399)(186.65149445,690.62712036)
\curveto(186.60149043,690.60711402)(186.54649049,690.59211404)(186.48649445,690.58212036)
\curveto(186.42649061,690.58211405)(186.37149066,690.57211406)(186.32149445,690.55212036)
\curveto(185.88149115,690.41211422)(185.55649148,690.13711449)(185.34649445,689.72712036)
\curveto(185.32649171,689.68711494)(185.30149173,689.632115)(185.27149445,689.56212036)
\curveto(185.25149178,689.50211513)(185.2364918,689.43711519)(185.22649445,689.36712036)
\curveto(185.21649182,689.30711532)(185.21649182,689.24711538)(185.22649445,689.18712036)
\curveto(185.24649179,689.1271155)(185.28149175,689.07711555)(185.33149445,689.03712036)
\curveto(185.41149162,688.98711564)(185.52149151,688.96211567)(185.66149445,688.96212036)
\lineto(186.06649445,688.96212036)
\lineto(187.73149445,688.96212036)
\lineto(188.16649445,688.96212036)
\curveto(188.32648871,688.97211566)(188.4314886,689.01711561)(188.48149445,689.09712036)
}
}
{
\newrgbcolor{curcolor}{0 0 0}
\pscustom[linestyle=none,fillstyle=solid,fillcolor=curcolor]
{
\newpath
\moveto(196.4047757,692.20212036)
\curveto(196.51477038,692.20211243)(196.60977029,692.19211244)(196.6897757,692.17212036)
\curveto(196.77977012,692.15211248)(196.84977005,692.10711252)(196.8997757,692.03712036)
\curveto(196.95976994,691.95711267)(196.98976991,691.81711281)(196.9897757,691.61712036)
\lineto(196.9897757,691.10712036)
\lineto(196.9897757,690.73212036)
\curveto(196.9997699,690.59211404)(196.98476991,690.48211415)(196.9447757,690.40212036)
\curveto(196.90476999,690.3321143)(196.84477005,690.28711434)(196.7647757,690.26712036)
\curveto(196.6947702,690.24711438)(196.60977029,690.23711439)(196.5097757,690.23712036)
\curveto(196.41977048,690.23711439)(196.31977058,690.24211439)(196.2097757,690.25212036)
\curveto(196.10977079,690.26211437)(196.01477088,690.25711437)(195.9247757,690.23712036)
\curveto(195.85477104,690.21711441)(195.78477111,690.20211443)(195.7147757,690.19212036)
\curveto(195.64477125,690.19211444)(195.57977132,690.18211445)(195.5197757,690.16212036)
\curveto(195.35977154,690.11211452)(195.1997717,690.03711459)(195.0397757,689.93712036)
\curveto(194.87977202,689.84711478)(194.75477214,689.74211489)(194.6647757,689.62212036)
\curveto(194.61477228,689.54211509)(194.55977234,689.45711517)(194.4997757,689.36712036)
\curveto(194.44977245,689.28711534)(194.3997725,689.20211543)(194.3497757,689.11212036)
\curveto(194.31977258,689.0321156)(194.28977261,688.94711568)(194.2597757,688.85712036)
\lineto(194.1997757,688.61712036)
\curveto(194.17977272,688.54711608)(194.16977273,688.47211616)(194.1697757,688.39212036)
\curveto(194.16977273,688.32211631)(194.15977274,688.25211638)(194.1397757,688.18212036)
\curveto(194.12977277,688.14211649)(194.12477277,688.10211653)(194.1247757,688.06212036)
\curveto(194.13477276,688.0321166)(194.13477276,688.00211663)(194.1247757,687.97212036)
\lineto(194.1247757,687.73212036)
\curveto(194.10477279,687.66211697)(194.0997728,687.58211705)(194.1097757,687.49212036)
\curveto(194.11977278,687.41211722)(194.12477277,687.3321173)(194.1247757,687.25212036)
\lineto(194.1247757,686.29212036)
\lineto(194.1247757,685.01712036)
\curveto(194.12477277,684.88711974)(194.11977278,684.76711986)(194.1097757,684.65712036)
\curveto(194.0997728,684.54712008)(194.06977283,684.45712017)(194.0197757,684.38712036)
\curveto(193.9997729,684.35712027)(193.96477293,684.3321203)(193.9147757,684.31212036)
\curveto(193.87477302,684.30212033)(193.82977307,684.29212034)(193.7797757,684.28212036)
\lineto(193.7047757,684.28212036)
\curveto(193.65477324,684.27212036)(193.5997733,684.26712036)(193.5397757,684.26712036)
\lineto(193.3747757,684.26712036)
\lineto(192.7297757,684.26712036)
\curveto(192.66977423,684.27712035)(192.60477429,684.28212035)(192.5347757,684.28212036)
\lineto(192.3397757,684.28212036)
\curveto(192.28977461,684.30212033)(192.23977466,684.31712031)(192.1897757,684.32712036)
\curveto(192.13977476,684.34712028)(192.10477479,684.38212025)(192.0847757,684.43212036)
\curveto(192.04477485,684.48212015)(192.01977488,684.55212008)(192.0097757,684.64212036)
\lineto(192.0097757,684.94212036)
\lineto(192.0097757,685.96212036)
\lineto(192.0097757,690.19212036)
\lineto(192.0097757,691.30212036)
\lineto(192.0097757,691.58712036)
\curveto(192.00977489,691.68711294)(192.02977487,691.76711286)(192.0697757,691.82712036)
\curveto(192.11977478,691.90711272)(192.1947747,691.95711267)(192.2947757,691.97712036)
\curveto(192.3947745,691.99711263)(192.51477438,692.00711262)(192.6547757,692.00712036)
\lineto(193.4197757,692.00712036)
\curveto(193.53977336,692.00711262)(193.64477325,691.99711263)(193.7347757,691.97712036)
\curveto(193.82477307,691.96711266)(193.894773,691.92211271)(193.9447757,691.84212036)
\curveto(193.97477292,691.79211284)(193.98977291,691.72211291)(193.9897757,691.63212036)
\lineto(194.0197757,691.36212036)
\curveto(194.02977287,691.28211335)(194.04477285,691.20711342)(194.0647757,691.13712036)
\curveto(194.0947728,691.06711356)(194.14477275,691.0321136)(194.2147757,691.03212036)
\curveto(194.23477266,691.05211358)(194.25477264,691.06211357)(194.2747757,691.06212036)
\curveto(194.2947726,691.06211357)(194.31477258,691.07211356)(194.3347757,691.09212036)
\curveto(194.3947725,691.14211349)(194.44477245,691.19711343)(194.4847757,691.25712036)
\curveto(194.53477236,691.3271133)(194.5947723,691.38711324)(194.6647757,691.43712036)
\curveto(194.70477219,691.46711316)(194.73977216,691.49711313)(194.7697757,691.52712036)
\curveto(194.7997721,691.56711306)(194.83477206,691.60211303)(194.8747757,691.63212036)
\lineto(195.1447757,691.81212036)
\curveto(195.24477165,691.87211276)(195.34477155,691.9271127)(195.4447757,691.97712036)
\curveto(195.54477135,692.01711261)(195.64477125,692.05211258)(195.7447757,692.08212036)
\lineto(196.0747757,692.17212036)
\curveto(196.10477079,692.18211245)(196.15977074,692.18211245)(196.2397757,692.17212036)
\curveto(196.32977057,692.17211246)(196.38477051,692.18211245)(196.4047757,692.20212036)
}
}
{
\newrgbcolor{curcolor}{0 0 0}
\pscustom[linestyle=none,fillstyle=solid,fillcolor=curcolor]
{
\newpath
\moveto(205.31618195,688.45212036)
\curveto(205.33617338,688.39211624)(205.34617337,688.30711632)(205.34618195,688.19712036)
\curveto(205.34617337,688.08711654)(205.33617338,688.00211663)(205.31618195,687.94212036)
\lineto(205.31618195,687.79212036)
\curveto(205.29617342,687.71211692)(205.28617343,687.632117)(205.28618195,687.55212036)
\curveto(205.29617342,687.47211716)(205.29117342,687.39211724)(205.27118195,687.31212036)
\curveto(205.25117346,687.24211739)(205.23617348,687.17711745)(205.22618195,687.11712036)
\curveto(205.2161735,687.05711757)(205.20617351,686.99211764)(205.19618195,686.92212036)
\curveto(205.15617356,686.81211782)(205.12117359,686.69711793)(205.09118195,686.57712036)
\curveto(205.06117365,686.46711816)(205.02117369,686.36211827)(204.97118195,686.26212036)
\curveto(204.76117395,685.78211885)(204.48617423,685.39211924)(204.14618195,685.09212036)
\curveto(203.80617491,684.79211984)(203.39617532,684.54212009)(202.91618195,684.34212036)
\curveto(202.79617592,684.29212034)(202.67117604,684.25712037)(202.54118195,684.23712036)
\curveto(202.42117629,684.20712042)(202.29617642,684.17712045)(202.16618195,684.14712036)
\curveto(202.1161766,684.1271205)(202.06117665,684.11712051)(202.00118195,684.11712036)
\curveto(201.94117677,684.11712051)(201.88617683,684.11212052)(201.83618195,684.10212036)
\lineto(201.73118195,684.10212036)
\curveto(201.70117701,684.09212054)(201.67117704,684.08712054)(201.64118195,684.08712036)
\curveto(201.59117712,684.07712055)(201.5111772,684.07212056)(201.40118195,684.07212036)
\curveto(201.29117742,684.06212057)(201.20617751,684.06712056)(201.14618195,684.08712036)
\lineto(200.99618195,684.08712036)
\curveto(200.94617777,684.09712053)(200.89117782,684.10212053)(200.83118195,684.10212036)
\curveto(200.78117793,684.09212054)(200.73117798,684.09712053)(200.68118195,684.11712036)
\curveto(200.64117807,684.1271205)(200.60117811,684.1321205)(200.56118195,684.13212036)
\curveto(200.53117818,684.1321205)(200.49117822,684.13712049)(200.44118195,684.14712036)
\curveto(200.34117837,684.17712045)(200.24117847,684.20212043)(200.14118195,684.22212036)
\curveto(200.04117867,684.24212039)(199.94617877,684.27212036)(199.85618195,684.31212036)
\curveto(199.73617898,684.35212028)(199.62117909,684.39212024)(199.51118195,684.43212036)
\curveto(199.4111793,684.47212016)(199.30617941,684.52212011)(199.19618195,684.58212036)
\curveto(198.84617987,684.79211984)(198.54618017,685.03711959)(198.29618195,685.31712036)
\curveto(198.04618067,685.59711903)(197.83618088,685.9321187)(197.66618195,686.32212036)
\curveto(197.6161811,686.41211822)(197.57618114,686.50711812)(197.54618195,686.60712036)
\curveto(197.52618119,686.70711792)(197.50118121,686.81211782)(197.47118195,686.92212036)
\curveto(197.45118126,686.97211766)(197.44118127,687.01711761)(197.44118195,687.05712036)
\curveto(197.44118127,687.09711753)(197.43118128,687.14211749)(197.41118195,687.19212036)
\curveto(197.39118132,687.27211736)(197.38118133,687.35211728)(197.38118195,687.43212036)
\curveto(197.38118133,687.52211711)(197.37118134,687.60711702)(197.35118195,687.68712036)
\curveto(197.34118137,687.73711689)(197.33618138,687.78211685)(197.33618195,687.82212036)
\lineto(197.33618195,687.95712036)
\curveto(197.3161814,688.01711661)(197.30618141,688.10211653)(197.30618195,688.21212036)
\curveto(197.3161814,688.32211631)(197.33118138,688.40711622)(197.35118195,688.46712036)
\lineto(197.35118195,688.57212036)
\curveto(197.36118135,688.62211601)(197.36118135,688.67211596)(197.35118195,688.72212036)
\curveto(197.35118136,688.78211585)(197.36118135,688.83711579)(197.38118195,688.88712036)
\curveto(197.39118132,688.93711569)(197.39618132,688.98211565)(197.39618195,689.02212036)
\curveto(197.39618132,689.07211556)(197.40618131,689.12211551)(197.42618195,689.17212036)
\curveto(197.46618125,689.30211533)(197.50118121,689.4271152)(197.53118195,689.54712036)
\curveto(197.56118115,689.67711495)(197.60118111,689.80211483)(197.65118195,689.92212036)
\curveto(197.83118088,690.3321143)(198.04618067,690.67211396)(198.29618195,690.94212036)
\curveto(198.54618017,691.22211341)(198.85117986,691.47711315)(199.21118195,691.70712036)
\curveto(199.3111794,691.75711287)(199.4161793,691.80211283)(199.52618195,691.84212036)
\curveto(199.63617908,691.88211275)(199.74617897,691.9271127)(199.85618195,691.97712036)
\curveto(199.98617873,692.0271126)(200.12117859,692.06211257)(200.26118195,692.08212036)
\curveto(200.40117831,692.10211253)(200.54617817,692.1321125)(200.69618195,692.17212036)
\curveto(200.77617794,692.18211245)(200.85117786,692.18711244)(200.92118195,692.18712036)
\curveto(200.99117772,692.18711244)(201.06117765,692.19211244)(201.13118195,692.20212036)
\curveto(201.711177,692.21211242)(202.2111765,692.15211248)(202.63118195,692.02212036)
\curveto(203.06117565,691.89211274)(203.44117527,691.71211292)(203.77118195,691.48212036)
\curveto(203.88117483,691.40211323)(203.99117472,691.31211332)(204.10118195,691.21212036)
\curveto(204.22117449,691.12211351)(204.32117439,691.02211361)(204.40118195,690.91212036)
\curveto(204.48117423,690.81211382)(204.55117416,690.71211392)(204.61118195,690.61212036)
\curveto(204.68117403,690.51211412)(204.75117396,690.40711422)(204.82118195,690.29712036)
\curveto(204.89117382,690.18711444)(204.94617377,690.06711456)(204.98618195,689.93712036)
\curveto(205.02617369,689.81711481)(205.07117364,689.68711494)(205.12118195,689.54712036)
\curveto(205.15117356,689.46711516)(205.17617354,689.38211525)(205.19618195,689.29212036)
\lineto(205.25618195,689.02212036)
\curveto(205.26617345,688.98211565)(205.27117344,688.94211569)(205.27118195,688.90212036)
\curveto(205.27117344,688.86211577)(205.27617344,688.82211581)(205.28618195,688.78212036)
\curveto(205.30617341,688.7321159)(205.3111734,688.67711595)(205.30118195,688.61712036)
\curveto(205.29117342,688.55711607)(205.29617342,688.50211613)(205.31618195,688.45212036)
\moveto(203.21618195,687.91212036)
\curveto(203.22617549,687.96211667)(203.23117548,688.0321166)(203.23118195,688.12212036)
\curveto(203.23117548,688.22211641)(203.22617549,688.29711633)(203.21618195,688.34712036)
\lineto(203.21618195,688.46712036)
\curveto(203.19617552,688.51711611)(203.18617553,688.57211606)(203.18618195,688.63212036)
\curveto(203.18617553,688.69211594)(203.18117553,688.74711588)(203.17118195,688.79712036)
\curveto(203.17117554,688.83711579)(203.16617555,688.86711576)(203.15618195,688.88712036)
\lineto(203.09618195,689.12712036)
\curveto(203.08617563,689.21711541)(203.06617565,689.30211533)(203.03618195,689.38212036)
\curveto(202.92617579,689.64211499)(202.79617592,689.86211477)(202.64618195,690.04212036)
\curveto(202.49617622,690.2321144)(202.29617642,690.38211425)(202.04618195,690.49212036)
\curveto(201.98617673,690.51211412)(201.92617679,690.5271141)(201.86618195,690.53712036)
\curveto(201.80617691,690.55711407)(201.74117697,690.57711405)(201.67118195,690.59712036)
\curveto(201.59117712,690.61711401)(201.50617721,690.62211401)(201.41618195,690.61212036)
\lineto(201.14618195,690.61212036)
\curveto(201.1161776,690.59211404)(201.08117763,690.58211405)(201.04118195,690.58212036)
\curveto(201.00117771,690.59211404)(200.96617775,690.59211404)(200.93618195,690.58212036)
\lineto(200.72618195,690.52212036)
\curveto(200.66617805,690.51211412)(200.6111781,690.49211414)(200.56118195,690.46212036)
\curveto(200.3111784,690.35211428)(200.10617861,690.19211444)(199.94618195,689.98212036)
\curveto(199.79617892,689.78211485)(199.67617904,689.54711508)(199.58618195,689.27712036)
\curveto(199.55617916,689.17711545)(199.53117918,689.07211556)(199.51118195,688.96212036)
\curveto(199.50117921,688.85211578)(199.48617923,688.74211589)(199.46618195,688.63212036)
\curveto(199.45617926,688.58211605)(199.45117926,688.5321161)(199.45118195,688.48212036)
\lineto(199.45118195,688.33212036)
\curveto(199.43117928,688.26211637)(199.42117929,688.15711647)(199.42118195,688.01712036)
\curveto(199.43117928,687.87711675)(199.44617927,687.77211686)(199.46618195,687.70212036)
\lineto(199.46618195,687.56712036)
\curveto(199.48617923,687.48711714)(199.50117921,687.40711722)(199.51118195,687.32712036)
\curveto(199.52117919,687.25711737)(199.53617918,687.18211745)(199.55618195,687.10212036)
\curveto(199.65617906,686.80211783)(199.76117895,686.55711807)(199.87118195,686.36712036)
\curveto(199.99117872,686.18711844)(200.17617854,686.02211861)(200.42618195,685.87212036)
\curveto(200.49617822,685.82211881)(200.57117814,685.78211885)(200.65118195,685.75212036)
\curveto(200.74117797,685.72211891)(200.83117788,685.69711893)(200.92118195,685.67712036)
\curveto(200.96117775,685.66711896)(200.99617772,685.66211897)(201.02618195,685.66212036)
\curveto(201.05617766,685.67211896)(201.09117762,685.67211896)(201.13118195,685.66212036)
\lineto(201.25118195,685.63212036)
\curveto(201.30117741,685.632119)(201.34617737,685.63711899)(201.38618195,685.64712036)
\lineto(201.50618195,685.64712036)
\curveto(201.58617713,685.66711896)(201.66617705,685.68211895)(201.74618195,685.69212036)
\curveto(201.82617689,685.70211893)(201.90117681,685.72211891)(201.97118195,685.75212036)
\curveto(202.23117648,685.85211878)(202.44117627,685.98711864)(202.60118195,686.15712036)
\curveto(202.76117595,686.3271183)(202.89617582,686.53711809)(203.00618195,686.78712036)
\curveto(203.04617567,686.88711774)(203.07617564,686.98711764)(203.09618195,687.08712036)
\curveto(203.1161756,687.18711744)(203.14117557,687.29211734)(203.17118195,687.40212036)
\curveto(203.18117553,687.44211719)(203.18617553,687.47711715)(203.18618195,687.50712036)
\curveto(203.18617553,687.54711708)(203.19117552,687.58711704)(203.20118195,687.62712036)
\lineto(203.20118195,687.76212036)
\curveto(203.20117551,687.81211682)(203.20617551,687.86211677)(203.21618195,687.91212036)
}
}
{
\newrgbcolor{curcolor}{0 0 0}
\pscustom[linestyle=none,fillstyle=solid,fillcolor=curcolor]
{
\newpath
\moveto(211.14110382,692.20212036)
\curveto(211.74109802,692.22211241)(212.24109752,692.13711249)(212.64110382,691.94712036)
\curveto(213.04109672,691.75711287)(213.3560964,691.47711315)(213.58610382,691.10712036)
\curveto(213.6560961,690.99711363)(213.71109605,690.87711375)(213.75110382,690.74712036)
\curveto(213.79109597,690.627114)(213.83109593,690.50211413)(213.87110382,690.37212036)
\curveto(213.89109587,690.29211434)(213.90109586,690.21711441)(213.90110382,690.14712036)
\curveto(213.91109585,690.07711455)(213.92609583,690.00711462)(213.94610382,689.93712036)
\curveto(213.94609581,689.87711475)(213.95109581,689.83711479)(213.96110382,689.81712036)
\curveto(213.98109578,689.67711495)(213.99109577,689.5321151)(213.99110382,689.38212036)
\lineto(213.99110382,688.94712036)
\lineto(213.99110382,687.61212036)
\lineto(213.99110382,685.18212036)
\curveto(213.99109577,684.99211964)(213.98609577,684.80711982)(213.97610382,684.62712036)
\curveto(213.97609578,684.45712017)(213.90609585,684.34712028)(213.76610382,684.29712036)
\curveto(213.70609605,684.27712035)(213.63609612,684.26712036)(213.55610382,684.26712036)
\lineto(213.31610382,684.26712036)
\lineto(212.50610382,684.26712036)
\curveto(212.38609737,684.26712036)(212.27609748,684.27212036)(212.17610382,684.28212036)
\curveto(212.08609767,684.30212033)(212.01609774,684.34712028)(211.96610382,684.41712036)
\curveto(211.92609783,684.47712015)(211.90109786,684.55212008)(211.89110382,684.64212036)
\lineto(211.89110382,684.95712036)
\lineto(211.89110382,686.00712036)
\lineto(211.89110382,688.24212036)
\curveto(211.89109787,688.61211602)(211.87609788,688.95211568)(211.84610382,689.26212036)
\curveto(211.81609794,689.58211505)(211.72609803,689.85211478)(211.57610382,690.07212036)
\curveto(211.43609832,690.27211436)(211.23109853,690.41211422)(210.96110382,690.49212036)
\curveto(210.91109885,690.51211412)(210.8560989,690.52211411)(210.79610382,690.52212036)
\curveto(210.74609901,690.52211411)(210.69109907,690.5321141)(210.63110382,690.55212036)
\curveto(210.58109918,690.56211407)(210.51609924,690.56211407)(210.43610382,690.55212036)
\curveto(210.36609939,690.55211408)(210.31109945,690.54711408)(210.27110382,690.53712036)
\curveto(210.23109953,690.5271141)(210.19609956,690.52211411)(210.16610382,690.52212036)
\curveto(210.13609962,690.52211411)(210.10609965,690.51711411)(210.07610382,690.50712036)
\curveto(209.84609991,690.44711418)(209.6611001,690.36711426)(209.52110382,690.26712036)
\curveto(209.20110056,690.03711459)(209.01110075,689.70211493)(208.95110382,689.26212036)
\curveto(208.89110087,688.82211581)(208.8611009,688.3271163)(208.86110382,687.77712036)
\lineto(208.86110382,685.90212036)
\lineto(208.86110382,684.98712036)
\lineto(208.86110382,684.71712036)
\curveto(208.8611009,684.62712)(208.84610091,684.55212008)(208.81610382,684.49212036)
\curveto(208.76610099,684.38212025)(208.68610107,684.31712031)(208.57610382,684.29712036)
\curveto(208.46610129,684.27712035)(208.33110143,684.26712036)(208.17110382,684.26712036)
\lineto(207.42110382,684.26712036)
\curveto(207.31110245,684.26712036)(207.20110256,684.27212036)(207.09110382,684.28212036)
\curveto(206.98110278,684.29212034)(206.90110286,684.3271203)(206.85110382,684.38712036)
\curveto(206.78110298,684.47712015)(206.74610301,684.60712002)(206.74610382,684.77712036)
\curveto(206.756103,684.94711968)(206.761103,685.10711952)(206.76110382,685.25712036)
\lineto(206.76110382,687.29712036)
\lineto(206.76110382,690.59712036)
\lineto(206.76110382,691.36212036)
\lineto(206.76110382,691.66212036)
\curveto(206.77110299,691.75211288)(206.80110296,691.8271128)(206.85110382,691.88712036)
\curveto(206.87110289,691.91711271)(206.90110286,691.93711269)(206.94110382,691.94712036)
\curveto(206.99110277,691.96711266)(207.04110272,691.98211265)(207.09110382,691.99212036)
\lineto(207.16610382,691.99212036)
\curveto(207.21610254,692.00211263)(207.26610249,692.00711262)(207.31610382,692.00712036)
\lineto(207.48110382,692.00712036)
\lineto(208.11110382,692.00712036)
\curveto(208.19110157,692.00711262)(208.26610149,692.00211263)(208.33610382,691.99212036)
\curveto(208.41610134,691.99211264)(208.48610127,691.98211265)(208.54610382,691.96212036)
\curveto(208.61610114,691.9321127)(208.6611011,691.88711274)(208.68110382,691.82712036)
\curveto(208.71110105,691.76711286)(208.73610102,691.69711293)(208.75610382,691.61712036)
\curveto(208.76610099,691.57711305)(208.76610099,691.54211309)(208.75610382,691.51212036)
\curveto(208.756101,691.48211315)(208.76610099,691.45211318)(208.78610382,691.42212036)
\curveto(208.80610095,691.37211326)(208.82110094,691.34211329)(208.83110382,691.33212036)
\curveto(208.85110091,691.32211331)(208.87610088,691.30711332)(208.90610382,691.28712036)
\curveto(209.01610074,691.27711335)(209.10610065,691.31211332)(209.17610382,691.39212036)
\curveto(209.24610051,691.48211315)(209.32110044,691.55211308)(209.40110382,691.60212036)
\curveto(209.67110009,691.80211283)(209.97109979,691.96211267)(210.30110382,692.08212036)
\curveto(210.39109937,692.11211252)(210.48109928,692.1321125)(210.57110382,692.14212036)
\curveto(210.67109909,692.15211248)(210.77609898,692.16711246)(210.88610382,692.18712036)
\curveto(210.91609884,692.19711243)(210.9610988,692.19711243)(211.02110382,692.18712036)
\curveto(211.08109868,692.18711244)(211.12109864,692.19211244)(211.14110382,692.20212036)
}
}
{
\newrgbcolor{curcolor}{0 0 0}
\pscustom[linestyle=none,fillstyle=solid,fillcolor=curcolor]
{
\newpath
\moveto(23.0357132,673.46712036)
\curveto(23.7857087,673.48711239)(24.43570805,673.40211248)(24.9857132,673.21212036)
\curveto(25.54570694,673.03211285)(25.97070651,672.71711316)(26.2607132,672.26712036)
\curveto(26.33070615,672.15711372)(26.39070609,672.04211384)(26.4407132,671.92212036)
\curveto(26.50070598,671.81211407)(26.55070593,671.68711419)(26.5907132,671.54712036)
\curveto(26.61070587,671.48711439)(26.62070586,671.42211446)(26.6207132,671.35212036)
\curveto(26.62070586,671.2821146)(26.61070587,671.22211466)(26.5907132,671.17212036)
\curveto(26.55070593,671.11211477)(26.49570599,671.07211481)(26.4257132,671.05212036)
\curveto(26.37570611,671.03211485)(26.31570617,671.02211486)(26.2457132,671.02212036)
\lineto(26.0357132,671.02212036)
\lineto(25.3757132,671.02212036)
\curveto(25.30570718,671.02211486)(25.23570725,671.01711486)(25.1657132,671.00712036)
\curveto(25.09570739,671.00711487)(25.03070745,671.01711486)(24.9707132,671.03712036)
\curveto(24.87070761,671.05711482)(24.79570769,671.09711478)(24.7457132,671.15712036)
\curveto(24.69570779,671.21711466)(24.65070783,671.2771146)(24.6107132,671.33712036)
\lineto(24.4907132,671.54712036)
\curveto(24.46070802,671.62711425)(24.41070807,671.69211419)(24.3407132,671.74212036)
\curveto(24.24070824,671.82211406)(24.14070834,671.882114)(24.0407132,671.92212036)
\curveto(23.95070853,671.96211392)(23.83570865,671.99711388)(23.6957132,672.02712036)
\curveto(23.62570886,672.04711383)(23.52070896,672.06211382)(23.3807132,672.07212036)
\curveto(23.25070923,672.0821138)(23.15070933,672.0771138)(23.0807132,672.05712036)
\lineto(22.9757132,672.05712036)
\lineto(22.8257132,672.02712036)
\curveto(22.7857097,672.02711385)(22.74070974,672.02211386)(22.6907132,672.01212036)
\curveto(22.52070996,671.96211392)(22.3807101,671.89211399)(22.2707132,671.80212036)
\curveto(22.17071031,671.72211416)(22.10071038,671.59711428)(22.0607132,671.42712036)
\curveto(22.04071044,671.35711452)(22.04071044,671.29211459)(22.0607132,671.23212036)
\curveto(22.0807104,671.17211471)(22.10071038,671.12211476)(22.1207132,671.08212036)
\curveto(22.19071029,670.96211492)(22.27071021,670.86711501)(22.3607132,670.79712036)
\curveto(22.46071002,670.72711515)(22.57570991,670.66711521)(22.7057132,670.61712036)
\curveto(22.89570959,670.53711534)(23.10070938,670.46711541)(23.3207132,670.40712036)
\lineto(24.0107132,670.25712036)
\curveto(24.25070823,670.21711566)(24.480708,670.16711571)(24.7007132,670.10712036)
\curveto(24.93070755,670.05711582)(25.14570734,669.99211589)(25.3457132,669.91212036)
\curveto(25.43570705,669.87211601)(25.52070696,669.83711604)(25.6007132,669.80712036)
\curveto(25.69070679,669.78711609)(25.77570671,669.75211613)(25.8557132,669.70212036)
\curveto(26.04570644,669.5821163)(26.21570627,669.45211643)(26.3657132,669.31212036)
\curveto(26.52570596,669.17211671)(26.65070583,668.99711688)(26.7407132,668.78712036)
\curveto(26.77070571,668.71711716)(26.79570569,668.64711723)(26.8157132,668.57712036)
\curveto(26.83570565,668.50711737)(26.85570563,668.43211745)(26.8757132,668.35212036)
\curveto(26.8857056,668.29211759)(26.89070559,668.19711768)(26.8907132,668.06712036)
\curveto(26.90070558,667.94711793)(26.90070558,667.85211803)(26.8907132,667.78212036)
\lineto(26.8907132,667.70712036)
\curveto(26.87070561,667.64711823)(26.85570563,667.58711829)(26.8457132,667.52712036)
\curveto(26.84570564,667.4771184)(26.84070564,667.42711845)(26.8307132,667.37712036)
\curveto(26.76070572,667.0771188)(26.65070583,666.81211907)(26.5007132,666.58212036)
\curveto(26.34070614,666.34211954)(26.14570634,666.14711973)(25.9157132,665.99712036)
\curveto(25.6857068,665.84712003)(25.42570706,665.71712016)(25.1357132,665.60712036)
\curveto(25.02570746,665.55712032)(24.90570758,665.52212036)(24.7757132,665.50212036)
\curveto(24.65570783,665.4821204)(24.53570795,665.45712042)(24.4157132,665.42712036)
\curveto(24.32570816,665.40712047)(24.23070825,665.39712048)(24.1307132,665.39712036)
\curveto(24.04070844,665.38712049)(23.95070853,665.37212051)(23.8607132,665.35212036)
\lineto(23.5907132,665.35212036)
\curveto(23.53070895,665.33212055)(23.42570906,665.32212056)(23.2757132,665.32212036)
\curveto(23.13570935,665.32212056)(23.03570945,665.33212055)(22.9757132,665.35212036)
\curveto(22.94570954,665.35212053)(22.91070957,665.35712052)(22.8707132,665.36712036)
\lineto(22.7657132,665.36712036)
\curveto(22.64570984,665.38712049)(22.52570996,665.40212048)(22.4057132,665.41212036)
\curveto(22.2857102,665.42212046)(22.17071031,665.44212044)(22.0607132,665.47212036)
\curveto(21.67071081,665.5821203)(21.32571116,665.70712017)(21.0257132,665.84712036)
\curveto(20.72571176,665.99711988)(20.47071201,666.21711966)(20.2607132,666.50712036)
\curveto(20.12071236,666.69711918)(20.00071248,666.91711896)(19.9007132,667.16712036)
\curveto(19.8807126,667.22711865)(19.86071262,667.30711857)(19.8407132,667.40712036)
\curveto(19.82071266,667.45711842)(19.80571268,667.52711835)(19.7957132,667.61712036)
\curveto(19.7857127,667.70711817)(19.79071269,667.7821181)(19.8107132,667.84212036)
\curveto(19.84071264,667.91211797)(19.89071259,667.96211792)(19.9607132,667.99212036)
\curveto(20.01071247,668.01211787)(20.07071241,668.02211786)(20.1407132,668.02212036)
\lineto(20.3657132,668.02212036)
\lineto(21.0707132,668.02212036)
\lineto(21.3107132,668.02212036)
\curveto(21.39071109,668.02211786)(21.46071102,668.01211787)(21.5207132,667.99212036)
\curveto(21.63071085,667.95211793)(21.70071078,667.88711799)(21.7307132,667.79712036)
\curveto(21.77071071,667.70711817)(21.81571067,667.61211827)(21.8657132,667.51212036)
\curveto(21.8857106,667.46211842)(21.92071056,667.39711848)(21.9707132,667.31712036)
\curveto(22.03071045,667.23711864)(22.0807104,667.18711869)(22.1207132,667.16712036)
\curveto(22.24071024,667.06711881)(22.35571013,666.98711889)(22.4657132,666.92712036)
\curveto(22.57570991,666.877119)(22.71570977,666.82711905)(22.8857132,666.77712036)
\curveto(22.93570955,666.75711912)(22.9857095,666.74711913)(23.0357132,666.74712036)
\curveto(23.0857094,666.75711912)(23.13570935,666.75711912)(23.1857132,666.74712036)
\curveto(23.26570922,666.72711915)(23.35070913,666.71711916)(23.4407132,666.71712036)
\curveto(23.54070894,666.72711915)(23.62570886,666.74211914)(23.6957132,666.76212036)
\curveto(23.74570874,666.77211911)(23.79070869,666.7771191)(23.8307132,666.77712036)
\curveto(23.8807086,666.7771191)(23.93070855,666.78711909)(23.9807132,666.80712036)
\curveto(24.12070836,666.85711902)(24.24570824,666.91711896)(24.3557132,666.98712036)
\curveto(24.47570801,667.05711882)(24.57070791,667.14711873)(24.6407132,667.25712036)
\curveto(24.69070779,667.33711854)(24.73070775,667.46211842)(24.7607132,667.63212036)
\curveto(24.7807077,667.70211818)(24.7807077,667.76711811)(24.7607132,667.82712036)
\curveto(24.74070774,667.88711799)(24.72070776,667.93711794)(24.7007132,667.97712036)
\curveto(24.63070785,668.11711776)(24.54070794,668.22211766)(24.4307132,668.29212036)
\curveto(24.33070815,668.36211752)(24.21070827,668.42711745)(24.0707132,668.48712036)
\curveto(23.8807086,668.56711731)(23.6807088,668.63211725)(23.4707132,668.68212036)
\curveto(23.26070922,668.73211715)(23.05070943,668.78711709)(22.8407132,668.84712036)
\curveto(22.76070972,668.86711701)(22.67570981,668.882117)(22.5857132,668.89212036)
\curveto(22.50570998,668.90211698)(22.42571006,668.91711696)(22.3457132,668.93712036)
\curveto(22.02571046,669.02711685)(21.72071076,669.11211677)(21.4307132,669.19212036)
\curveto(21.14071134,669.2821166)(20.87571161,669.41211647)(20.6357132,669.58212036)
\curveto(20.35571213,669.7821161)(20.15071233,670.05211583)(20.0207132,670.39212036)
\curveto(20.00071248,670.46211542)(19.9807125,670.55711532)(19.9607132,670.67712036)
\curveto(19.94071254,670.74711513)(19.92571256,670.83211505)(19.9157132,670.93212036)
\curveto(19.90571258,671.03211485)(19.91071257,671.12211476)(19.9307132,671.20212036)
\curveto(19.95071253,671.25211463)(19.95571253,671.29211459)(19.9457132,671.32212036)
\curveto(19.93571255,671.36211452)(19.94071254,671.40711447)(19.9607132,671.45712036)
\curveto(19.9807125,671.56711431)(20.00071248,671.66711421)(20.0207132,671.75712036)
\curveto(20.05071243,671.85711402)(20.0857124,671.95211393)(20.1257132,672.04212036)
\curveto(20.25571223,672.33211355)(20.43571205,672.56711331)(20.6657132,672.74712036)
\curveto(20.89571159,672.92711295)(21.15571133,673.07211281)(21.4457132,673.18212036)
\curveto(21.55571093,673.23211265)(21.67071081,673.26711261)(21.7907132,673.28712036)
\curveto(21.91071057,673.31711256)(22.03571045,673.34711253)(22.1657132,673.37712036)
\curveto(22.22571026,673.39711248)(22.2857102,673.40711247)(22.3457132,673.40712036)
\lineto(22.5257132,673.43712036)
\curveto(22.60570988,673.44711243)(22.69070979,673.45211243)(22.7807132,673.45212036)
\curveto(22.87070961,673.45211243)(22.95570953,673.45711242)(23.0357132,673.46712036)
}
}
{
\newrgbcolor{curcolor}{0 0 0}
\pscustom[linestyle=none,fillstyle=solid,fillcolor=curcolor]
{
\newpath
\moveto(28.54235382,673.24212036)
\lineto(29.66735382,673.24212036)
\curveto(29.77735139,673.24211264)(29.87735129,673.23711264)(29.96735382,673.22712036)
\curveto(30.05735111,673.21711266)(30.12235104,673.1821127)(30.16235382,673.12212036)
\curveto(30.21235095,673.06211282)(30.24235092,672.9771129)(30.25235382,672.86712036)
\curveto(30.2623509,672.76711311)(30.2673509,672.66211322)(30.26735382,672.55212036)
\lineto(30.26735382,671.50212036)
\lineto(30.26735382,669.26712036)
\curveto(30.2673509,668.90711697)(30.28235088,668.56711731)(30.31235382,668.24712036)
\curveto(30.34235082,667.92711795)(30.43235073,667.66211822)(30.58235382,667.45212036)
\curveto(30.72235044,667.24211864)(30.94735022,667.09211879)(31.25735382,667.00212036)
\curveto(31.30734986,666.99211889)(31.34734982,666.98711889)(31.37735382,666.98712036)
\curveto(31.41734975,666.98711889)(31.4623497,666.9821189)(31.51235382,666.97212036)
\curveto(31.5623496,666.96211892)(31.61734955,666.95711892)(31.67735382,666.95712036)
\curveto(31.73734943,666.95711892)(31.78234938,666.96211892)(31.81235382,666.97212036)
\curveto(31.8623493,666.99211889)(31.90234926,666.99711888)(31.93235382,666.98712036)
\curveto(31.97234919,666.9771189)(32.01234915,666.9821189)(32.05235382,667.00212036)
\curveto(32.2623489,667.05211883)(32.42734874,667.11711876)(32.54735382,667.19712036)
\curveto(32.72734844,667.30711857)(32.8673483,667.44711843)(32.96735382,667.61712036)
\curveto(33.07734809,667.79711808)(33.15234801,667.99211789)(33.19235382,668.20212036)
\curveto(33.24234792,668.42211746)(33.27234789,668.66211722)(33.28235382,668.92212036)
\curveto(33.29234787,669.19211669)(33.29734787,669.47211641)(33.29735382,669.76212036)
\lineto(33.29735382,671.57712036)
\lineto(33.29735382,672.55212036)
\lineto(33.29735382,672.82212036)
\curveto(33.29734787,672.92211296)(33.31734785,673.00211288)(33.35735382,673.06212036)
\curveto(33.40734776,673.15211273)(33.48234768,673.20211268)(33.58235382,673.21212036)
\curveto(33.68234748,673.23211265)(33.80234736,673.24211264)(33.94235382,673.24212036)
\lineto(34.73735382,673.24212036)
\lineto(35.02235382,673.24212036)
\curveto(35.11234605,673.24211264)(35.18734598,673.22211266)(35.24735382,673.18212036)
\curveto(35.32734584,673.13211275)(35.37234579,673.05711282)(35.38235382,672.95712036)
\curveto(35.39234577,672.85711302)(35.39734577,672.74211314)(35.39735382,672.61212036)
\lineto(35.39735382,671.47212036)
\lineto(35.39735382,667.25712036)
\lineto(35.39735382,666.19212036)
\lineto(35.39735382,665.89212036)
\curveto(35.39734577,665.79212009)(35.37734579,665.71712016)(35.33735382,665.66712036)
\curveto(35.28734588,665.58712029)(35.21234595,665.54212034)(35.11235382,665.53212036)
\curveto(35.01234615,665.52212036)(34.90734626,665.51712036)(34.79735382,665.51712036)
\lineto(33.98735382,665.51712036)
\curveto(33.87734729,665.51712036)(33.77734739,665.52212036)(33.68735382,665.53212036)
\curveto(33.60734756,665.54212034)(33.54234762,665.5821203)(33.49235382,665.65212036)
\curveto(33.47234769,665.6821202)(33.45234771,665.72712015)(33.43235382,665.78712036)
\curveto(33.42234774,665.84712003)(33.40734776,665.90711997)(33.38735382,665.96712036)
\curveto(33.37734779,666.02711985)(33.3623478,666.0821198)(33.34235382,666.13212036)
\curveto(33.32234784,666.1821197)(33.29234787,666.21211967)(33.25235382,666.22212036)
\curveto(33.23234793,666.24211964)(33.20734796,666.24711963)(33.17735382,666.23712036)
\curveto(33.14734802,666.22711965)(33.12234804,666.21711966)(33.10235382,666.20712036)
\curveto(33.03234813,666.16711971)(32.97234819,666.12211976)(32.92235382,666.07212036)
\curveto(32.87234829,666.02211986)(32.81734835,665.9771199)(32.75735382,665.93712036)
\curveto(32.71734845,665.90711997)(32.67734849,665.87212001)(32.63735382,665.83212036)
\curveto(32.60734856,665.80212008)(32.5673486,665.77212011)(32.51735382,665.74212036)
\curveto(32.28734888,665.60212028)(32.01734915,665.49212039)(31.70735382,665.41212036)
\curveto(31.63734953,665.39212049)(31.5673496,665.3821205)(31.49735382,665.38212036)
\curveto(31.42734974,665.37212051)(31.35234981,665.35712052)(31.27235382,665.33712036)
\curveto(31.23234993,665.32712055)(31.18734998,665.32712055)(31.13735382,665.33712036)
\curveto(31.09735007,665.33712054)(31.05735011,665.33212055)(31.01735382,665.32212036)
\curveto(30.98735018,665.31212057)(30.92235024,665.31212057)(30.82235382,665.32212036)
\curveto(30.73235043,665.32212056)(30.67235049,665.32712055)(30.64235382,665.33712036)
\curveto(30.59235057,665.33712054)(30.54235062,665.34212054)(30.49235382,665.35212036)
\lineto(30.34235382,665.35212036)
\curveto(30.22235094,665.3821205)(30.10735106,665.40712047)(29.99735382,665.42712036)
\curveto(29.88735128,665.44712043)(29.77735139,665.4771204)(29.66735382,665.51712036)
\curveto(29.61735155,665.53712034)(29.57235159,665.55212033)(29.53235382,665.56212036)
\curveto(29.50235166,665.5821203)(29.4623517,665.60212028)(29.41235382,665.62212036)
\curveto(29.0623521,665.81212007)(28.78235238,666.0771198)(28.57235382,666.41712036)
\curveto(28.44235272,666.62711925)(28.34735282,666.877119)(28.28735382,667.16712036)
\curveto(28.22735294,667.46711841)(28.18735298,667.7821181)(28.16735382,668.11212036)
\curveto(28.15735301,668.45211743)(28.15235301,668.79711708)(28.15235382,669.14712036)
\curveto(28.162353,669.50711637)(28.167353,669.86211602)(28.16735382,670.21212036)
\lineto(28.16735382,672.25212036)
\curveto(28.167353,672.3821135)(28.162353,672.53211335)(28.15235382,672.70212036)
\curveto(28.15235301,672.882113)(28.17735299,673.01211287)(28.22735382,673.09212036)
\curveto(28.25735291,673.14211274)(28.31735285,673.18711269)(28.40735382,673.22712036)
\curveto(28.4673527,673.22711265)(28.51235265,673.23211265)(28.54235382,673.24212036)
}
}
{
\newrgbcolor{curcolor}{0 0 0}
\pscustom[linestyle=none,fillstyle=solid,fillcolor=curcolor]
{
}
}
{
\newrgbcolor{curcolor}{0 0 0}
\pscustom[linestyle=none,fillstyle=solid,fillcolor=curcolor]
{
\newpath
\moveto(44.75876007,673.46712036)
\curveto(45.56875491,673.48711239)(46.24375424,673.36711251)(46.78376007,673.10712036)
\curveto(47.33375315,672.84711303)(47.76875271,672.4771134)(48.08876007,671.99712036)
\curveto(48.24875223,671.75711412)(48.36875211,671.4821144)(48.44876007,671.17212036)
\curveto(48.46875201,671.12211476)(48.483752,671.05711482)(48.49376007,670.97712036)
\curveto(48.51375197,670.89711498)(48.51375197,670.82711505)(48.49376007,670.76712036)
\curveto(48.45375203,670.65711522)(48.3837521,670.59211529)(48.28376007,670.57212036)
\curveto(48.1837523,670.56211532)(48.06375242,670.55711532)(47.92376007,670.55712036)
\lineto(47.14376007,670.55712036)
\lineto(46.85876007,670.55712036)
\curveto(46.76875371,670.55711532)(46.69375379,670.5771153)(46.63376007,670.61712036)
\curveto(46.55375393,670.65711522)(46.49875398,670.71711516)(46.46876007,670.79712036)
\curveto(46.43875404,670.88711499)(46.39875408,670.9771149)(46.34876007,671.06712036)
\curveto(46.28875419,671.1771147)(46.22375426,671.2771146)(46.15376007,671.36712036)
\curveto(46.0837544,671.45711442)(46.00375448,671.53711434)(45.91376007,671.60712036)
\curveto(45.77375471,671.69711418)(45.61875486,671.76711411)(45.44876007,671.81712036)
\curveto(45.38875509,671.83711404)(45.32875515,671.84711403)(45.26876007,671.84712036)
\curveto(45.20875527,671.84711403)(45.15375533,671.85711402)(45.10376007,671.87712036)
\lineto(44.95376007,671.87712036)
\curveto(44.75375573,671.877114)(44.59375589,671.85711402)(44.47376007,671.81712036)
\curveto(44.1837563,671.72711415)(43.94875653,671.58711429)(43.76876007,671.39712036)
\curveto(43.58875689,671.21711466)(43.44375704,670.99711488)(43.33376007,670.73712036)
\curveto(43.2837572,670.62711525)(43.24375724,670.50711537)(43.21376007,670.37712036)
\curveto(43.19375729,670.25711562)(43.16875731,670.12711575)(43.13876007,669.98712036)
\curveto(43.12875735,669.94711593)(43.12375736,669.90711597)(43.12376007,669.86712036)
\curveto(43.12375736,669.82711605)(43.11875736,669.78711609)(43.10876007,669.74712036)
\curveto(43.08875739,669.64711623)(43.0787574,669.50711637)(43.07876007,669.32712036)
\curveto(43.08875739,669.14711673)(43.10375738,669.00711687)(43.12376007,668.90712036)
\curveto(43.12375736,668.82711705)(43.12875735,668.77211711)(43.13876007,668.74212036)
\curveto(43.15875732,668.67211721)(43.16875731,668.60211728)(43.16876007,668.53212036)
\curveto(43.1787573,668.46211742)(43.19375729,668.39211749)(43.21376007,668.32212036)
\curveto(43.29375719,668.09211779)(43.38875709,667.882118)(43.49876007,667.69212036)
\curveto(43.60875687,667.50211838)(43.74875673,667.34211854)(43.91876007,667.21212036)
\curveto(43.95875652,667.1821187)(44.01875646,667.14711873)(44.09876007,667.10712036)
\curveto(44.20875627,667.03711884)(44.31875616,666.99211889)(44.42876007,666.97212036)
\curveto(44.54875593,666.95211893)(44.69375579,666.93211895)(44.86376007,666.91212036)
\lineto(44.95376007,666.91212036)
\curveto(44.99375549,666.91211897)(45.02375546,666.91711896)(45.04376007,666.92712036)
\lineto(45.17876007,666.92712036)
\curveto(45.24875523,666.94711893)(45.31375517,666.96211892)(45.37376007,666.97212036)
\curveto(45.44375504,666.99211889)(45.50875497,667.01211887)(45.56876007,667.03212036)
\curveto(45.86875461,667.16211872)(46.09875438,667.35211853)(46.25876007,667.60212036)
\curveto(46.29875418,667.65211823)(46.33375415,667.70711817)(46.36376007,667.76712036)
\curveto(46.39375409,667.83711804)(46.41875406,667.89711798)(46.43876007,667.94712036)
\curveto(46.478754,668.05711782)(46.51375397,668.15211773)(46.54376007,668.23212036)
\curveto(46.57375391,668.32211756)(46.64375384,668.39211749)(46.75376007,668.44212036)
\curveto(46.84375364,668.4821174)(46.98875349,668.49711738)(47.18876007,668.48712036)
\lineto(47.68376007,668.48712036)
\lineto(47.89376007,668.48712036)
\curveto(47.97375251,668.49711738)(48.03875244,668.49211739)(48.08876007,668.47212036)
\lineto(48.20876007,668.47212036)
\lineto(48.32876007,668.44212036)
\curveto(48.36875211,668.44211744)(48.39875208,668.43211745)(48.41876007,668.41212036)
\curveto(48.46875201,668.37211751)(48.49875198,668.31211757)(48.50876007,668.23212036)
\curveto(48.52875195,668.16211772)(48.52875195,668.08711779)(48.50876007,668.00712036)
\curveto(48.41875206,667.6771182)(48.30875217,667.3821185)(48.17876007,667.12212036)
\curveto(47.76875271,666.35211953)(47.11375337,665.81712006)(46.21376007,665.51712036)
\curveto(46.11375437,665.48712039)(46.00875447,665.46712041)(45.89876007,665.45712036)
\curveto(45.78875469,665.43712044)(45.6787548,665.41212047)(45.56876007,665.38212036)
\curveto(45.50875497,665.37212051)(45.44875503,665.36712051)(45.38876007,665.36712036)
\curveto(45.32875515,665.36712051)(45.26875521,665.36212052)(45.20876007,665.35212036)
\lineto(45.04376007,665.35212036)
\curveto(44.99375549,665.33212055)(44.91875556,665.32712055)(44.81876007,665.33712036)
\curveto(44.71875576,665.33712054)(44.64375584,665.34212054)(44.59376007,665.35212036)
\curveto(44.51375597,665.37212051)(44.43875604,665.3821205)(44.36876007,665.38212036)
\curveto(44.30875617,665.37212051)(44.24375624,665.3771205)(44.17376007,665.39712036)
\lineto(44.02376007,665.42712036)
\curveto(43.97375651,665.42712045)(43.92375656,665.43212045)(43.87376007,665.44212036)
\curveto(43.76375672,665.47212041)(43.65875682,665.50212038)(43.55876007,665.53212036)
\curveto(43.45875702,665.56212032)(43.36375712,665.59712028)(43.27376007,665.63712036)
\curveto(42.80375768,665.83712004)(42.40875807,666.09211979)(42.08876007,666.40212036)
\curveto(41.76875871,666.72211916)(41.50875897,667.11711876)(41.30876007,667.58712036)
\curveto(41.25875922,667.6771182)(41.21875926,667.77211811)(41.18876007,667.87212036)
\lineto(41.09876007,668.20212036)
\curveto(41.08875939,668.24211764)(41.0837594,668.2771176)(41.08376007,668.30712036)
\curveto(41.0837594,668.34711753)(41.07375941,668.39211749)(41.05376007,668.44212036)
\curveto(41.03375945,668.51211737)(41.02375946,668.5821173)(41.02376007,668.65212036)
\curveto(41.02375946,668.73211715)(41.01375947,668.80711707)(40.99376007,668.87712036)
\lineto(40.99376007,669.13212036)
\curveto(40.97375951,669.1821167)(40.96375952,669.23711664)(40.96376007,669.29712036)
\curveto(40.96375952,669.36711651)(40.97375951,669.42711645)(40.99376007,669.47712036)
\curveto(41.00375948,669.52711635)(41.00375948,669.57211631)(40.99376007,669.61212036)
\curveto(40.9837595,669.65211623)(40.9837595,669.69211619)(40.99376007,669.73212036)
\curveto(41.01375947,669.80211608)(41.01875946,669.86711601)(41.00876007,669.92712036)
\curveto(41.00875947,669.98711589)(41.01875946,670.04711583)(41.03876007,670.10712036)
\curveto(41.08875939,670.28711559)(41.12875935,670.45711542)(41.15876007,670.61712036)
\curveto(41.18875929,670.78711509)(41.23375925,670.95211493)(41.29376007,671.11212036)
\curveto(41.51375897,671.62211426)(41.78875869,672.04711383)(42.11876007,672.38712036)
\curveto(42.45875802,672.72711315)(42.88875759,673.00211288)(43.40876007,673.21212036)
\curveto(43.54875693,673.27211261)(43.69375679,673.31211257)(43.84376007,673.33212036)
\curveto(43.99375649,673.36211252)(44.14875633,673.39711248)(44.30876007,673.43712036)
\curveto(44.38875609,673.44711243)(44.46375602,673.45211243)(44.53376007,673.45212036)
\curveto(44.60375588,673.45211243)(44.6787558,673.45711242)(44.75876007,673.46712036)
}
}
{
\newrgbcolor{curcolor}{0 0 0}
\pscustom[linestyle=none,fillstyle=solid,fillcolor=curcolor]
{
\newpath
\moveto(57.57204132,669.70212036)
\curveto(57.59203275,669.64211624)(57.60203274,669.55711632)(57.60204132,669.44712036)
\curveto(57.60203274,669.33711654)(57.59203275,669.25211663)(57.57204132,669.19212036)
\lineto(57.57204132,669.04212036)
\curveto(57.55203279,668.96211692)(57.5420328,668.882117)(57.54204132,668.80212036)
\curveto(57.55203279,668.72211716)(57.5470328,668.64211724)(57.52704132,668.56212036)
\curveto(57.50703284,668.49211739)(57.49203285,668.42711745)(57.48204132,668.36712036)
\curveto(57.47203287,668.30711757)(57.46203288,668.24211764)(57.45204132,668.17212036)
\curveto(57.41203293,668.06211782)(57.37703297,667.94711793)(57.34704132,667.82712036)
\curveto(57.31703303,667.71711816)(57.27703307,667.61211827)(57.22704132,667.51212036)
\curveto(57.01703333,667.03211885)(56.7420336,666.64211924)(56.40204132,666.34212036)
\curveto(56.06203428,666.04211984)(55.65203469,665.79212009)(55.17204132,665.59212036)
\curveto(55.05203529,665.54212034)(54.92703542,665.50712037)(54.79704132,665.48712036)
\curveto(54.67703567,665.45712042)(54.55203579,665.42712045)(54.42204132,665.39712036)
\curveto(54.37203597,665.3771205)(54.31703603,665.36712051)(54.25704132,665.36712036)
\curveto(54.19703615,665.36712051)(54.1420362,665.36212052)(54.09204132,665.35212036)
\lineto(53.98704132,665.35212036)
\curveto(53.95703639,665.34212054)(53.92703642,665.33712054)(53.89704132,665.33712036)
\curveto(53.8470365,665.32712055)(53.76703658,665.32212056)(53.65704132,665.32212036)
\curveto(53.5470368,665.31212057)(53.46203688,665.31712056)(53.40204132,665.33712036)
\lineto(53.25204132,665.33712036)
\curveto(53.20203714,665.34712053)(53.1470372,665.35212053)(53.08704132,665.35212036)
\curveto(53.03703731,665.34212054)(52.98703736,665.34712053)(52.93704132,665.36712036)
\curveto(52.89703745,665.3771205)(52.85703749,665.3821205)(52.81704132,665.38212036)
\curveto(52.78703756,665.3821205)(52.7470376,665.38712049)(52.69704132,665.39712036)
\curveto(52.59703775,665.42712045)(52.49703785,665.45212043)(52.39704132,665.47212036)
\curveto(52.29703805,665.49212039)(52.20203814,665.52212036)(52.11204132,665.56212036)
\curveto(51.99203835,665.60212028)(51.87703847,665.64212024)(51.76704132,665.68212036)
\curveto(51.66703868,665.72212016)(51.56203878,665.77212011)(51.45204132,665.83212036)
\curveto(51.10203924,666.04211984)(50.80203954,666.28711959)(50.55204132,666.56712036)
\curveto(50.30204004,666.84711903)(50.09204025,667.1821187)(49.92204132,667.57212036)
\curveto(49.87204047,667.66211822)(49.83204051,667.75711812)(49.80204132,667.85712036)
\curveto(49.78204056,667.95711792)(49.75704059,668.06211782)(49.72704132,668.17212036)
\curveto(49.70704064,668.22211766)(49.69704065,668.26711761)(49.69704132,668.30712036)
\curveto(49.69704065,668.34711753)(49.68704066,668.39211749)(49.66704132,668.44212036)
\curveto(49.6470407,668.52211736)(49.63704071,668.60211728)(49.63704132,668.68212036)
\curveto(49.63704071,668.77211711)(49.62704072,668.85711702)(49.60704132,668.93712036)
\curveto(49.59704075,668.98711689)(49.59204075,669.03211685)(49.59204132,669.07212036)
\lineto(49.59204132,669.20712036)
\curveto(49.57204077,669.26711661)(49.56204078,669.35211653)(49.56204132,669.46212036)
\curveto(49.57204077,669.57211631)(49.58704076,669.65711622)(49.60704132,669.71712036)
\lineto(49.60704132,669.82212036)
\curveto(49.61704073,669.87211601)(49.61704073,669.92211596)(49.60704132,669.97212036)
\curveto(49.60704074,670.03211585)(49.61704073,670.08711579)(49.63704132,670.13712036)
\curveto(49.6470407,670.18711569)(49.65204069,670.23211565)(49.65204132,670.27212036)
\curveto(49.65204069,670.32211556)(49.66204068,670.37211551)(49.68204132,670.42212036)
\curveto(49.72204062,670.55211533)(49.75704059,670.6771152)(49.78704132,670.79712036)
\curveto(49.81704053,670.92711495)(49.85704049,671.05211483)(49.90704132,671.17212036)
\curveto(50.08704026,671.5821143)(50.30204004,671.92211396)(50.55204132,672.19212036)
\curveto(50.80203954,672.47211341)(51.10703924,672.72711315)(51.46704132,672.95712036)
\curveto(51.56703878,673.00711287)(51.67203867,673.05211283)(51.78204132,673.09212036)
\curveto(51.89203845,673.13211275)(52.00203834,673.1771127)(52.11204132,673.22712036)
\curveto(52.2420381,673.2771126)(52.37703797,673.31211257)(52.51704132,673.33212036)
\curveto(52.65703769,673.35211253)(52.80203754,673.3821125)(52.95204132,673.42212036)
\curveto(53.03203731,673.43211245)(53.10703724,673.43711244)(53.17704132,673.43712036)
\curveto(53.2470371,673.43711244)(53.31703703,673.44211244)(53.38704132,673.45212036)
\curveto(53.96703638,673.46211242)(54.46703588,673.40211248)(54.88704132,673.27212036)
\curveto(55.31703503,673.14211274)(55.69703465,672.96211292)(56.02704132,672.73212036)
\curveto(56.13703421,672.65211323)(56.2470341,672.56211332)(56.35704132,672.46212036)
\curveto(56.47703387,672.37211351)(56.57703377,672.27211361)(56.65704132,672.16212036)
\curveto(56.73703361,672.06211382)(56.80703354,671.96211392)(56.86704132,671.86212036)
\curveto(56.93703341,671.76211412)(57.00703334,671.65711422)(57.07704132,671.54712036)
\curveto(57.1470332,671.43711444)(57.20203314,671.31711456)(57.24204132,671.18712036)
\curveto(57.28203306,671.06711481)(57.32703302,670.93711494)(57.37704132,670.79712036)
\curveto(57.40703294,670.71711516)(57.43203291,670.63211525)(57.45204132,670.54212036)
\lineto(57.51204132,670.27212036)
\curveto(57.52203282,670.23211565)(57.52703282,670.19211569)(57.52704132,670.15212036)
\curveto(57.52703282,670.11211577)(57.53203281,670.07211581)(57.54204132,670.03212036)
\curveto(57.56203278,669.9821159)(57.56703278,669.92711595)(57.55704132,669.86712036)
\curveto(57.5470328,669.80711607)(57.55203279,669.75211613)(57.57204132,669.70212036)
\moveto(55.47204132,669.16212036)
\curveto(55.48203486,669.21211667)(55.48703486,669.2821166)(55.48704132,669.37212036)
\curveto(55.48703486,669.47211641)(55.48203486,669.54711633)(55.47204132,669.59712036)
\lineto(55.47204132,669.71712036)
\curveto(55.45203489,669.76711611)(55.4420349,669.82211606)(55.44204132,669.88212036)
\curveto(55.4420349,669.94211594)(55.43703491,669.99711588)(55.42704132,670.04712036)
\curveto(55.42703492,670.08711579)(55.42203492,670.11711576)(55.41204132,670.13712036)
\lineto(55.35204132,670.37712036)
\curveto(55.342035,670.46711541)(55.32203502,670.55211533)(55.29204132,670.63212036)
\curveto(55.18203516,670.89211499)(55.05203529,671.11211477)(54.90204132,671.29212036)
\curveto(54.75203559,671.4821144)(54.55203579,671.63211425)(54.30204132,671.74212036)
\curveto(54.2420361,671.76211412)(54.18203616,671.7771141)(54.12204132,671.78712036)
\curveto(54.06203628,671.80711407)(53.99703635,671.82711405)(53.92704132,671.84712036)
\curveto(53.8470365,671.86711401)(53.76203658,671.87211401)(53.67204132,671.86212036)
\lineto(53.40204132,671.86212036)
\curveto(53.37203697,671.84211404)(53.33703701,671.83211405)(53.29704132,671.83212036)
\curveto(53.25703709,671.84211404)(53.22203712,671.84211404)(53.19204132,671.83212036)
\lineto(52.98204132,671.77212036)
\curveto(52.92203742,671.76211412)(52.86703748,671.74211414)(52.81704132,671.71212036)
\curveto(52.56703778,671.60211428)(52.36203798,671.44211444)(52.20204132,671.23212036)
\curveto(52.05203829,671.03211485)(51.93203841,670.79711508)(51.84204132,670.52712036)
\curveto(51.81203853,670.42711545)(51.78703856,670.32211556)(51.76704132,670.21212036)
\curveto(51.75703859,670.10211578)(51.7420386,669.99211589)(51.72204132,669.88212036)
\curveto(51.71203863,669.83211605)(51.70703864,669.7821161)(51.70704132,669.73212036)
\lineto(51.70704132,669.58212036)
\curveto(51.68703866,669.51211637)(51.67703867,669.40711647)(51.67704132,669.26712036)
\curveto(51.68703866,669.12711675)(51.70203864,669.02211686)(51.72204132,668.95212036)
\lineto(51.72204132,668.81712036)
\curveto(51.7420386,668.73711714)(51.75703859,668.65711722)(51.76704132,668.57712036)
\curveto(51.77703857,668.50711737)(51.79203855,668.43211745)(51.81204132,668.35212036)
\curveto(51.91203843,668.05211783)(52.01703833,667.80711807)(52.12704132,667.61712036)
\curveto(52.2470381,667.43711844)(52.43203791,667.27211861)(52.68204132,667.12212036)
\curveto(52.75203759,667.07211881)(52.82703752,667.03211885)(52.90704132,667.00212036)
\curveto(52.99703735,666.97211891)(53.08703726,666.94711893)(53.17704132,666.92712036)
\curveto(53.21703713,666.91711896)(53.25203709,666.91211897)(53.28204132,666.91212036)
\curveto(53.31203703,666.92211896)(53.347037,666.92211896)(53.38704132,666.91212036)
\lineto(53.50704132,666.88212036)
\curveto(53.55703679,666.882119)(53.60203674,666.88711899)(53.64204132,666.89712036)
\lineto(53.76204132,666.89712036)
\curveto(53.8420365,666.91711896)(53.92203642,666.93211895)(54.00204132,666.94212036)
\curveto(54.08203626,666.95211893)(54.15703619,666.97211891)(54.22704132,667.00212036)
\curveto(54.48703586,667.10211878)(54.69703565,667.23711864)(54.85704132,667.40712036)
\curveto(55.01703533,667.5771183)(55.15203519,667.78711809)(55.26204132,668.03712036)
\curveto(55.30203504,668.13711774)(55.33203501,668.23711764)(55.35204132,668.33712036)
\curveto(55.37203497,668.43711744)(55.39703495,668.54211734)(55.42704132,668.65212036)
\curveto(55.43703491,668.69211719)(55.4420349,668.72711715)(55.44204132,668.75712036)
\curveto(55.4420349,668.79711708)(55.4470349,668.83711704)(55.45704132,668.87712036)
\lineto(55.45704132,669.01212036)
\curveto(55.45703489,669.06211682)(55.46203488,669.11211677)(55.47204132,669.16212036)
}
}
{
\newrgbcolor{curcolor}{0 0 0}
\pscustom[linestyle=none,fillstyle=solid,fillcolor=curcolor]
{
\newpath
\moveto(63.3969632,673.45212036)
\curveto(63.50695788,673.45211243)(63.60195779,673.44211244)(63.6819632,673.42212036)
\curveto(63.77195762,673.40211248)(63.84195755,673.35711252)(63.8919632,673.28712036)
\curveto(63.95195744,673.20711267)(63.98195741,673.06711281)(63.9819632,672.86712036)
\lineto(63.9819632,672.35712036)
\lineto(63.9819632,671.98212036)
\curveto(63.9919574,671.84211404)(63.97695741,671.73211415)(63.9369632,671.65212036)
\curveto(63.89695749,671.5821143)(63.83695755,671.53711434)(63.7569632,671.51712036)
\curveto(63.6869577,671.49711438)(63.60195779,671.48711439)(63.5019632,671.48712036)
\curveto(63.41195798,671.48711439)(63.31195808,671.49211439)(63.2019632,671.50212036)
\curveto(63.10195829,671.51211437)(63.00695838,671.50711437)(62.9169632,671.48712036)
\curveto(62.84695854,671.46711441)(62.77695861,671.45211443)(62.7069632,671.44212036)
\curveto(62.63695875,671.44211444)(62.57195882,671.43211445)(62.5119632,671.41212036)
\curveto(62.35195904,671.36211452)(62.1919592,671.28711459)(62.0319632,671.18712036)
\curveto(61.87195952,671.09711478)(61.74695964,670.99211489)(61.6569632,670.87212036)
\curveto(61.60695978,670.79211509)(61.55195984,670.70711517)(61.4919632,670.61712036)
\curveto(61.44195995,670.53711534)(61.39196,670.45211543)(61.3419632,670.36212036)
\curveto(61.31196008,670.2821156)(61.28196011,670.19711568)(61.2519632,670.10712036)
\lineto(61.1919632,669.86712036)
\curveto(61.17196022,669.79711608)(61.16196023,669.72211616)(61.1619632,669.64212036)
\curveto(61.16196023,669.57211631)(61.15196024,669.50211638)(61.1319632,669.43212036)
\curveto(61.12196027,669.39211649)(61.11696027,669.35211653)(61.1169632,669.31212036)
\curveto(61.12696026,669.2821166)(61.12696026,669.25211663)(61.1169632,669.22212036)
\lineto(61.1169632,668.98212036)
\curveto(61.09696029,668.91211697)(61.0919603,668.83211705)(61.1019632,668.74212036)
\curveto(61.11196028,668.66211722)(61.11696027,668.5821173)(61.1169632,668.50212036)
\lineto(61.1169632,667.54212036)
\lineto(61.1169632,666.26712036)
\curveto(61.11696027,666.13711974)(61.11196028,666.01711986)(61.1019632,665.90712036)
\curveto(61.0919603,665.79712008)(61.06196033,665.70712017)(61.0119632,665.63712036)
\curveto(60.9919604,665.60712027)(60.95696043,665.5821203)(60.9069632,665.56212036)
\curveto(60.86696052,665.55212033)(60.82196057,665.54212034)(60.7719632,665.53212036)
\lineto(60.6969632,665.53212036)
\curveto(60.64696074,665.52212036)(60.5919608,665.51712036)(60.5319632,665.51712036)
\lineto(60.3669632,665.51712036)
\lineto(59.7219632,665.51712036)
\curveto(59.66196173,665.52712035)(59.59696179,665.53212035)(59.5269632,665.53212036)
\lineto(59.3319632,665.53212036)
\curveto(59.28196211,665.55212033)(59.23196216,665.56712031)(59.1819632,665.57712036)
\curveto(59.13196226,665.59712028)(59.09696229,665.63212025)(59.0769632,665.68212036)
\curveto(59.03696235,665.73212015)(59.01196238,665.80212008)(59.0019632,665.89212036)
\lineto(59.0019632,666.19212036)
\lineto(59.0019632,667.21212036)
\lineto(59.0019632,671.44212036)
\lineto(59.0019632,672.55212036)
\lineto(59.0019632,672.83712036)
\curveto(59.00196239,672.93711294)(59.02196237,673.01711286)(59.0619632,673.07712036)
\curveto(59.11196228,673.15711272)(59.1869622,673.20711267)(59.2869632,673.22712036)
\curveto(59.386962,673.24711263)(59.50696188,673.25711262)(59.6469632,673.25712036)
\lineto(60.4119632,673.25712036)
\curveto(60.53196086,673.25711262)(60.63696075,673.24711263)(60.7269632,673.22712036)
\curveto(60.81696057,673.21711266)(60.8869605,673.17211271)(60.9369632,673.09212036)
\curveto(60.96696042,673.04211284)(60.98196041,672.97211291)(60.9819632,672.88212036)
\lineto(61.0119632,672.61212036)
\curveto(61.02196037,672.53211335)(61.03696035,672.45711342)(61.0569632,672.38712036)
\curveto(61.0869603,672.31711356)(61.13696025,672.2821136)(61.2069632,672.28212036)
\curveto(61.22696016,672.30211358)(61.24696014,672.31211357)(61.2669632,672.31212036)
\curveto(61.2869601,672.31211357)(61.30696008,672.32211356)(61.3269632,672.34212036)
\curveto(61.38696,672.39211349)(61.43695995,672.44711343)(61.4769632,672.50712036)
\curveto(61.52695986,672.5771133)(61.5869598,672.63711324)(61.6569632,672.68712036)
\curveto(61.69695969,672.71711316)(61.73195966,672.74711313)(61.7619632,672.77712036)
\curveto(61.7919596,672.81711306)(61.82695956,672.85211303)(61.8669632,672.88212036)
\lineto(62.1369632,673.06212036)
\curveto(62.23695915,673.12211276)(62.33695905,673.1771127)(62.4369632,673.22712036)
\curveto(62.53695885,673.26711261)(62.63695875,673.30211258)(62.7369632,673.33212036)
\lineto(63.0669632,673.42212036)
\curveto(63.09695829,673.43211245)(63.15195824,673.43211245)(63.2319632,673.42212036)
\curveto(63.32195807,673.42211246)(63.37695801,673.43211245)(63.3969632,673.45212036)
}
}
{
\newrgbcolor{curcolor}{0 0 0}
\pscustom[linestyle=none,fillstyle=solid,fillcolor=curcolor]
{
\newpath
\moveto(69.22704132,673.45212036)
\curveto(69.33703601,673.45211243)(69.43203591,673.44211244)(69.51204132,673.42212036)
\curveto(69.60203574,673.40211248)(69.67203567,673.35711252)(69.72204132,673.28712036)
\curveto(69.78203556,673.20711267)(69.81203553,673.06711281)(69.81204132,672.86712036)
\lineto(69.81204132,672.35712036)
\lineto(69.81204132,671.98212036)
\curveto(69.82203552,671.84211404)(69.80703554,671.73211415)(69.76704132,671.65212036)
\curveto(69.72703562,671.5821143)(69.66703568,671.53711434)(69.58704132,671.51712036)
\curveto(69.51703583,671.49711438)(69.43203591,671.48711439)(69.33204132,671.48712036)
\curveto(69.2420361,671.48711439)(69.1420362,671.49211439)(69.03204132,671.50212036)
\curveto(68.93203641,671.51211437)(68.83703651,671.50711437)(68.74704132,671.48712036)
\curveto(68.67703667,671.46711441)(68.60703674,671.45211443)(68.53704132,671.44212036)
\curveto(68.46703688,671.44211444)(68.40203694,671.43211445)(68.34204132,671.41212036)
\curveto(68.18203716,671.36211452)(68.02203732,671.28711459)(67.86204132,671.18712036)
\curveto(67.70203764,671.09711478)(67.57703777,670.99211489)(67.48704132,670.87212036)
\curveto(67.43703791,670.79211509)(67.38203796,670.70711517)(67.32204132,670.61712036)
\curveto(67.27203807,670.53711534)(67.22203812,670.45211543)(67.17204132,670.36212036)
\curveto(67.1420382,670.2821156)(67.11203823,670.19711568)(67.08204132,670.10712036)
\lineto(67.02204132,669.86712036)
\curveto(67.00203834,669.79711608)(66.99203835,669.72211616)(66.99204132,669.64212036)
\curveto(66.99203835,669.57211631)(66.98203836,669.50211638)(66.96204132,669.43212036)
\curveto(66.95203839,669.39211649)(66.9470384,669.35211653)(66.94704132,669.31212036)
\curveto(66.95703839,669.2821166)(66.95703839,669.25211663)(66.94704132,669.22212036)
\lineto(66.94704132,668.98212036)
\curveto(66.92703842,668.91211697)(66.92203842,668.83211705)(66.93204132,668.74212036)
\curveto(66.9420384,668.66211722)(66.9470384,668.5821173)(66.94704132,668.50212036)
\lineto(66.94704132,667.54212036)
\lineto(66.94704132,666.26712036)
\curveto(66.9470384,666.13711974)(66.9420384,666.01711986)(66.93204132,665.90712036)
\curveto(66.92203842,665.79712008)(66.89203845,665.70712017)(66.84204132,665.63712036)
\curveto(66.82203852,665.60712027)(66.78703856,665.5821203)(66.73704132,665.56212036)
\curveto(66.69703865,665.55212033)(66.65203869,665.54212034)(66.60204132,665.53212036)
\lineto(66.52704132,665.53212036)
\curveto(66.47703887,665.52212036)(66.42203892,665.51712036)(66.36204132,665.51712036)
\lineto(66.19704132,665.51712036)
\lineto(65.55204132,665.51712036)
\curveto(65.49203985,665.52712035)(65.42703992,665.53212035)(65.35704132,665.53212036)
\lineto(65.16204132,665.53212036)
\curveto(65.11204023,665.55212033)(65.06204028,665.56712031)(65.01204132,665.57712036)
\curveto(64.96204038,665.59712028)(64.92704042,665.63212025)(64.90704132,665.68212036)
\curveto(64.86704048,665.73212015)(64.8420405,665.80212008)(64.83204132,665.89212036)
\lineto(64.83204132,666.19212036)
\lineto(64.83204132,667.21212036)
\lineto(64.83204132,671.44212036)
\lineto(64.83204132,672.55212036)
\lineto(64.83204132,672.83712036)
\curveto(64.83204051,672.93711294)(64.85204049,673.01711286)(64.89204132,673.07712036)
\curveto(64.9420404,673.15711272)(65.01704033,673.20711267)(65.11704132,673.22712036)
\curveto(65.21704013,673.24711263)(65.33704001,673.25711262)(65.47704132,673.25712036)
\lineto(66.24204132,673.25712036)
\curveto(66.36203898,673.25711262)(66.46703888,673.24711263)(66.55704132,673.22712036)
\curveto(66.6470387,673.21711266)(66.71703863,673.17211271)(66.76704132,673.09212036)
\curveto(66.79703855,673.04211284)(66.81203853,672.97211291)(66.81204132,672.88212036)
\lineto(66.84204132,672.61212036)
\curveto(66.85203849,672.53211335)(66.86703848,672.45711342)(66.88704132,672.38712036)
\curveto(66.91703843,672.31711356)(66.96703838,672.2821136)(67.03704132,672.28212036)
\curveto(67.05703829,672.30211358)(67.07703827,672.31211357)(67.09704132,672.31212036)
\curveto(67.11703823,672.31211357)(67.13703821,672.32211356)(67.15704132,672.34212036)
\curveto(67.21703813,672.39211349)(67.26703808,672.44711343)(67.30704132,672.50712036)
\curveto(67.35703799,672.5771133)(67.41703793,672.63711324)(67.48704132,672.68712036)
\curveto(67.52703782,672.71711316)(67.56203778,672.74711313)(67.59204132,672.77712036)
\curveto(67.62203772,672.81711306)(67.65703769,672.85211303)(67.69704132,672.88212036)
\lineto(67.96704132,673.06212036)
\curveto(68.06703728,673.12211276)(68.16703718,673.1771127)(68.26704132,673.22712036)
\curveto(68.36703698,673.26711261)(68.46703688,673.30211258)(68.56704132,673.33212036)
\lineto(68.89704132,673.42212036)
\curveto(68.92703642,673.43211245)(68.98203636,673.43211245)(69.06204132,673.42212036)
\curveto(69.15203619,673.42211246)(69.20703614,673.43211245)(69.22704132,673.45212036)
}
}
{
\newrgbcolor{curcolor}{0 0 0}
\pscustom[linestyle=none,fillstyle=solid,fillcolor=curcolor]
{
\newpath
\moveto(77.73344757,669.46212036)
\curveto(77.75343941,669.3821165)(77.75343941,669.29211659)(77.73344757,669.19212036)
\curveto(77.71343945,669.09211679)(77.67843948,669.02711685)(77.62844757,668.99712036)
\curveto(77.57843958,668.95711692)(77.50343966,668.92711695)(77.40344757,668.90712036)
\curveto(77.31343985,668.89711698)(77.20843995,668.88711699)(77.08844757,668.87712036)
\lineto(76.74344757,668.87712036)
\curveto(76.63344053,668.88711699)(76.53344063,668.89211699)(76.44344757,668.89212036)
\lineto(72.78344757,668.89212036)
\lineto(72.57344757,668.89212036)
\curveto(72.51344465,668.89211699)(72.4584447,668.882117)(72.40844757,668.86212036)
\curveto(72.32844483,668.82211706)(72.27844488,668.7821171)(72.25844757,668.74212036)
\curveto(72.23844492,668.72211716)(72.21844494,668.6821172)(72.19844757,668.62212036)
\curveto(72.17844498,668.57211731)(72.17344499,668.52211736)(72.18344757,668.47212036)
\curveto(72.20344496,668.41211747)(72.21344495,668.35211753)(72.21344757,668.29212036)
\curveto(72.22344494,668.24211764)(72.23844492,668.18711769)(72.25844757,668.12712036)
\curveto(72.33844482,667.88711799)(72.43344473,667.68711819)(72.54344757,667.52712036)
\curveto(72.6634445,667.3771185)(72.82344434,667.24211864)(73.02344757,667.12212036)
\curveto(73.10344406,667.07211881)(73.18344398,667.03711884)(73.26344757,667.01712036)
\curveto(73.35344381,667.00711887)(73.44344372,666.98711889)(73.53344757,666.95712036)
\curveto(73.61344355,666.93711894)(73.72344344,666.92211896)(73.86344757,666.91212036)
\curveto(74.00344316,666.90211898)(74.12344304,666.90711897)(74.22344757,666.92712036)
\lineto(74.35844757,666.92712036)
\curveto(74.4584427,666.94711893)(74.54844261,666.96711891)(74.62844757,666.98712036)
\curveto(74.71844244,667.01711886)(74.80344236,667.04711883)(74.88344757,667.07712036)
\curveto(74.98344218,667.12711875)(75.09344207,667.19211869)(75.21344757,667.27212036)
\curveto(75.34344182,667.35211853)(75.43844172,667.43211845)(75.49844757,667.51212036)
\curveto(75.54844161,667.5821183)(75.59844156,667.64711823)(75.64844757,667.70712036)
\curveto(75.70844145,667.7771181)(75.77844138,667.82711805)(75.85844757,667.85712036)
\curveto(75.9584412,667.90711797)(76.08344108,667.92711795)(76.23344757,667.91712036)
\lineto(76.66844757,667.91712036)
\lineto(76.84844757,667.91712036)
\curveto(76.91844024,667.92711795)(76.97844018,667.92211796)(77.02844757,667.90212036)
\lineto(77.17844757,667.90212036)
\curveto(77.27843988,667.882118)(77.34843981,667.85711802)(77.38844757,667.82712036)
\curveto(77.42843973,667.80711807)(77.44843971,667.76211812)(77.44844757,667.69212036)
\curveto(77.4584397,667.62211826)(77.45343971,667.56211832)(77.43344757,667.51212036)
\curveto(77.38343978,667.37211851)(77.32843983,667.24711863)(77.26844757,667.13712036)
\curveto(77.20843995,667.02711885)(77.13844002,666.91711896)(77.05844757,666.80712036)
\curveto(76.83844032,666.4771194)(76.58844057,666.21211967)(76.30844757,666.01212036)
\curveto(76.02844113,665.81212007)(75.67844148,665.64212024)(75.25844757,665.50212036)
\curveto(75.14844201,665.46212042)(75.03844212,665.43712044)(74.92844757,665.42712036)
\curveto(74.81844234,665.41712046)(74.70344246,665.39712048)(74.58344757,665.36712036)
\curveto(74.54344262,665.35712052)(74.49844266,665.35712052)(74.44844757,665.36712036)
\curveto(74.40844275,665.36712051)(74.36844279,665.36212052)(74.32844757,665.35212036)
\lineto(74.16344757,665.35212036)
\curveto(74.11344305,665.33212055)(74.05344311,665.32712055)(73.98344757,665.33712036)
\curveto(73.92344324,665.33712054)(73.86844329,665.34212054)(73.81844757,665.35212036)
\curveto(73.73844342,665.36212052)(73.66844349,665.36212052)(73.60844757,665.35212036)
\curveto(73.54844361,665.34212054)(73.48344368,665.34712053)(73.41344757,665.36712036)
\curveto(73.3634438,665.38712049)(73.30844385,665.39712048)(73.24844757,665.39712036)
\curveto(73.18844397,665.39712048)(73.13344403,665.40712047)(73.08344757,665.42712036)
\curveto(72.97344419,665.44712043)(72.8634443,665.47212041)(72.75344757,665.50212036)
\curveto(72.64344452,665.52212036)(72.54344462,665.55712032)(72.45344757,665.60712036)
\curveto(72.34344482,665.64712023)(72.23844492,665.6821202)(72.13844757,665.71212036)
\curveto(72.04844511,665.75212013)(71.9634452,665.79712008)(71.88344757,665.84712036)
\curveto(71.5634456,666.04711983)(71.27844588,666.2771196)(71.02844757,666.53712036)
\curveto(70.77844638,666.80711907)(70.57344659,667.11711876)(70.41344757,667.46712036)
\curveto(70.3634468,667.5771183)(70.32344684,667.68711819)(70.29344757,667.79712036)
\curveto(70.2634469,667.91711796)(70.22344694,668.03711784)(70.17344757,668.15712036)
\curveto(70.163447,668.19711768)(70.158447,668.23211765)(70.15844757,668.26212036)
\curveto(70.158447,668.30211758)(70.15344701,668.34211754)(70.14344757,668.38212036)
\curveto(70.10344706,668.50211738)(70.07844708,668.63211725)(70.06844757,668.77212036)
\lineto(70.03844757,669.19212036)
\curveto(70.03844712,669.24211664)(70.03344713,669.29711658)(70.02344757,669.35712036)
\curveto(70.02344714,669.41711646)(70.02844713,669.47211641)(70.03844757,669.52212036)
\lineto(70.03844757,669.70212036)
\lineto(70.08344757,670.06212036)
\curveto(70.12344704,670.23211565)(70.158447,670.39711548)(70.18844757,670.55712036)
\curveto(70.21844694,670.71711516)(70.2634469,670.86711501)(70.32344757,671.00712036)
\curveto(70.75344641,672.04711383)(71.48344568,672.7821131)(72.51344757,673.21212036)
\curveto(72.65344451,673.27211261)(72.79344437,673.31211257)(72.93344757,673.33212036)
\curveto(73.08344408,673.36211252)(73.23844392,673.39711248)(73.39844757,673.43712036)
\curveto(73.47844368,673.44711243)(73.55344361,673.45211243)(73.62344757,673.45212036)
\curveto(73.69344347,673.45211243)(73.76844339,673.45711242)(73.84844757,673.46712036)
\curveto(74.3584428,673.4771124)(74.79344237,673.41711246)(75.15344757,673.28712036)
\curveto(75.52344164,673.16711271)(75.85344131,673.00711287)(76.14344757,672.80712036)
\curveto(76.23344093,672.74711313)(76.32344084,672.6771132)(76.41344757,672.59712036)
\curveto(76.50344066,672.52711335)(76.58344058,672.45211343)(76.65344757,672.37212036)
\curveto(76.68344048,672.32211356)(76.72344044,672.2821136)(76.77344757,672.25212036)
\curveto(76.85344031,672.14211374)(76.92844023,672.02711385)(76.99844757,671.90712036)
\curveto(77.06844009,671.79711408)(77.14344002,671.6821142)(77.22344757,671.56212036)
\curveto(77.27343989,671.47211441)(77.31343985,671.3771145)(77.34344757,671.27712036)
\curveto(77.38343978,671.18711469)(77.42343974,671.08711479)(77.46344757,670.97712036)
\curveto(77.51343965,670.84711503)(77.55343961,670.71211517)(77.58344757,670.57212036)
\curveto(77.61343955,670.43211545)(77.64843951,670.29211559)(77.68844757,670.15212036)
\curveto(77.70843945,670.07211581)(77.71343945,669.9821159)(77.70344757,669.88212036)
\curveto(77.70343946,669.79211609)(77.71343945,669.70711617)(77.73344757,669.62712036)
\lineto(77.73344757,669.46212036)
\moveto(75.48344757,670.34712036)
\curveto(75.55344161,670.44711543)(75.5584416,670.56711531)(75.49844757,670.70712036)
\curveto(75.44844171,670.85711502)(75.40844175,670.96711491)(75.37844757,671.03712036)
\curveto(75.23844192,671.30711457)(75.05344211,671.51211437)(74.82344757,671.65212036)
\curveto(74.59344257,671.80211408)(74.27344289,671.882114)(73.86344757,671.89212036)
\curveto(73.83344333,671.87211401)(73.79844336,671.86711401)(73.75844757,671.87712036)
\curveto(73.71844344,671.88711399)(73.68344348,671.88711399)(73.65344757,671.87712036)
\curveto(73.60344356,671.85711402)(73.54844361,671.84211404)(73.48844757,671.83212036)
\curveto(73.42844373,671.83211405)(73.37344379,671.82211406)(73.32344757,671.80212036)
\curveto(72.88344428,671.66211422)(72.5584446,671.38711449)(72.34844757,670.97712036)
\curveto(72.32844483,670.93711494)(72.30344486,670.882115)(72.27344757,670.81212036)
\curveto(72.25344491,670.75211513)(72.23844492,670.68711519)(72.22844757,670.61712036)
\curveto(72.21844494,670.55711532)(72.21844494,670.49711538)(72.22844757,670.43712036)
\curveto(72.24844491,670.3771155)(72.28344488,670.32711555)(72.33344757,670.28712036)
\curveto(72.41344475,670.23711564)(72.52344464,670.21211567)(72.66344757,670.21212036)
\lineto(73.06844757,670.21212036)
\lineto(74.73344757,670.21212036)
\lineto(75.16844757,670.21212036)
\curveto(75.32844183,670.22211566)(75.43344173,670.26711561)(75.48344757,670.34712036)
}
}
{
\newrgbcolor{curcolor}{0 0 0}
\pscustom[linestyle=none,fillstyle=solid,fillcolor=curcolor]
{
\newpath
\moveto(86.75172882,669.70212036)
\curveto(86.77172025,669.64211624)(86.78172024,669.55711632)(86.78172882,669.44712036)
\curveto(86.78172024,669.33711654)(86.77172025,669.25211663)(86.75172882,669.19212036)
\lineto(86.75172882,669.04212036)
\curveto(86.73172029,668.96211692)(86.7217203,668.882117)(86.72172882,668.80212036)
\curveto(86.73172029,668.72211716)(86.7267203,668.64211724)(86.70672882,668.56212036)
\curveto(86.68672034,668.49211739)(86.67172035,668.42711745)(86.66172882,668.36712036)
\curveto(86.65172037,668.30711757)(86.64172038,668.24211764)(86.63172882,668.17212036)
\curveto(86.59172043,668.06211782)(86.55672047,667.94711793)(86.52672882,667.82712036)
\curveto(86.49672053,667.71711816)(86.45672057,667.61211827)(86.40672882,667.51212036)
\curveto(86.19672083,667.03211885)(85.9217211,666.64211924)(85.58172882,666.34212036)
\curveto(85.24172178,666.04211984)(84.83172219,665.79212009)(84.35172882,665.59212036)
\curveto(84.23172279,665.54212034)(84.10672292,665.50712037)(83.97672882,665.48712036)
\curveto(83.85672317,665.45712042)(83.73172329,665.42712045)(83.60172882,665.39712036)
\curveto(83.55172347,665.3771205)(83.49672353,665.36712051)(83.43672882,665.36712036)
\curveto(83.37672365,665.36712051)(83.3217237,665.36212052)(83.27172882,665.35212036)
\lineto(83.16672882,665.35212036)
\curveto(83.13672389,665.34212054)(83.10672392,665.33712054)(83.07672882,665.33712036)
\curveto(83.026724,665.32712055)(82.94672408,665.32212056)(82.83672882,665.32212036)
\curveto(82.7267243,665.31212057)(82.64172438,665.31712056)(82.58172882,665.33712036)
\lineto(82.43172882,665.33712036)
\curveto(82.38172464,665.34712053)(82.3267247,665.35212053)(82.26672882,665.35212036)
\curveto(82.21672481,665.34212054)(82.16672486,665.34712053)(82.11672882,665.36712036)
\curveto(82.07672495,665.3771205)(82.03672499,665.3821205)(81.99672882,665.38212036)
\curveto(81.96672506,665.3821205)(81.9267251,665.38712049)(81.87672882,665.39712036)
\curveto(81.77672525,665.42712045)(81.67672535,665.45212043)(81.57672882,665.47212036)
\curveto(81.47672555,665.49212039)(81.38172564,665.52212036)(81.29172882,665.56212036)
\curveto(81.17172585,665.60212028)(81.05672597,665.64212024)(80.94672882,665.68212036)
\curveto(80.84672618,665.72212016)(80.74172628,665.77212011)(80.63172882,665.83212036)
\curveto(80.28172674,666.04211984)(79.98172704,666.28711959)(79.73172882,666.56712036)
\curveto(79.48172754,666.84711903)(79.27172775,667.1821187)(79.10172882,667.57212036)
\curveto(79.05172797,667.66211822)(79.01172801,667.75711812)(78.98172882,667.85712036)
\curveto(78.96172806,667.95711792)(78.93672809,668.06211782)(78.90672882,668.17212036)
\curveto(78.88672814,668.22211766)(78.87672815,668.26711761)(78.87672882,668.30712036)
\curveto(78.87672815,668.34711753)(78.86672816,668.39211749)(78.84672882,668.44212036)
\curveto(78.8267282,668.52211736)(78.81672821,668.60211728)(78.81672882,668.68212036)
\curveto(78.81672821,668.77211711)(78.80672822,668.85711702)(78.78672882,668.93712036)
\curveto(78.77672825,668.98711689)(78.77172825,669.03211685)(78.77172882,669.07212036)
\lineto(78.77172882,669.20712036)
\curveto(78.75172827,669.26711661)(78.74172828,669.35211653)(78.74172882,669.46212036)
\curveto(78.75172827,669.57211631)(78.76672826,669.65711622)(78.78672882,669.71712036)
\lineto(78.78672882,669.82212036)
\curveto(78.79672823,669.87211601)(78.79672823,669.92211596)(78.78672882,669.97212036)
\curveto(78.78672824,670.03211585)(78.79672823,670.08711579)(78.81672882,670.13712036)
\curveto(78.8267282,670.18711569)(78.83172819,670.23211565)(78.83172882,670.27212036)
\curveto(78.83172819,670.32211556)(78.84172818,670.37211551)(78.86172882,670.42212036)
\curveto(78.90172812,670.55211533)(78.93672809,670.6771152)(78.96672882,670.79712036)
\curveto(78.99672803,670.92711495)(79.03672799,671.05211483)(79.08672882,671.17212036)
\curveto(79.26672776,671.5821143)(79.48172754,671.92211396)(79.73172882,672.19212036)
\curveto(79.98172704,672.47211341)(80.28672674,672.72711315)(80.64672882,672.95712036)
\curveto(80.74672628,673.00711287)(80.85172617,673.05211283)(80.96172882,673.09212036)
\curveto(81.07172595,673.13211275)(81.18172584,673.1771127)(81.29172882,673.22712036)
\curveto(81.4217256,673.2771126)(81.55672547,673.31211257)(81.69672882,673.33212036)
\curveto(81.83672519,673.35211253)(81.98172504,673.3821125)(82.13172882,673.42212036)
\curveto(82.21172481,673.43211245)(82.28672474,673.43711244)(82.35672882,673.43712036)
\curveto(82.4267246,673.43711244)(82.49672453,673.44211244)(82.56672882,673.45212036)
\curveto(83.14672388,673.46211242)(83.64672338,673.40211248)(84.06672882,673.27212036)
\curveto(84.49672253,673.14211274)(84.87672215,672.96211292)(85.20672882,672.73212036)
\curveto(85.31672171,672.65211323)(85.4267216,672.56211332)(85.53672882,672.46212036)
\curveto(85.65672137,672.37211351)(85.75672127,672.27211361)(85.83672882,672.16212036)
\curveto(85.91672111,672.06211382)(85.98672104,671.96211392)(86.04672882,671.86212036)
\curveto(86.11672091,671.76211412)(86.18672084,671.65711422)(86.25672882,671.54712036)
\curveto(86.3267207,671.43711444)(86.38172064,671.31711456)(86.42172882,671.18712036)
\curveto(86.46172056,671.06711481)(86.50672052,670.93711494)(86.55672882,670.79712036)
\curveto(86.58672044,670.71711516)(86.61172041,670.63211525)(86.63172882,670.54212036)
\lineto(86.69172882,670.27212036)
\curveto(86.70172032,670.23211565)(86.70672032,670.19211569)(86.70672882,670.15212036)
\curveto(86.70672032,670.11211577)(86.71172031,670.07211581)(86.72172882,670.03212036)
\curveto(86.74172028,669.9821159)(86.74672028,669.92711595)(86.73672882,669.86712036)
\curveto(86.7267203,669.80711607)(86.73172029,669.75211613)(86.75172882,669.70212036)
\moveto(84.65172882,669.16212036)
\curveto(84.66172236,669.21211667)(84.66672236,669.2821166)(84.66672882,669.37212036)
\curveto(84.66672236,669.47211641)(84.66172236,669.54711633)(84.65172882,669.59712036)
\lineto(84.65172882,669.71712036)
\curveto(84.63172239,669.76711611)(84.6217224,669.82211606)(84.62172882,669.88212036)
\curveto(84.6217224,669.94211594)(84.61672241,669.99711588)(84.60672882,670.04712036)
\curveto(84.60672242,670.08711579)(84.60172242,670.11711576)(84.59172882,670.13712036)
\lineto(84.53172882,670.37712036)
\curveto(84.5217225,670.46711541)(84.50172252,670.55211533)(84.47172882,670.63212036)
\curveto(84.36172266,670.89211499)(84.23172279,671.11211477)(84.08172882,671.29212036)
\curveto(83.93172309,671.4821144)(83.73172329,671.63211425)(83.48172882,671.74212036)
\curveto(83.4217236,671.76211412)(83.36172366,671.7771141)(83.30172882,671.78712036)
\curveto(83.24172378,671.80711407)(83.17672385,671.82711405)(83.10672882,671.84712036)
\curveto(83.026724,671.86711401)(82.94172408,671.87211401)(82.85172882,671.86212036)
\lineto(82.58172882,671.86212036)
\curveto(82.55172447,671.84211404)(82.51672451,671.83211405)(82.47672882,671.83212036)
\curveto(82.43672459,671.84211404)(82.40172462,671.84211404)(82.37172882,671.83212036)
\lineto(82.16172882,671.77212036)
\curveto(82.10172492,671.76211412)(82.04672498,671.74211414)(81.99672882,671.71212036)
\curveto(81.74672528,671.60211428)(81.54172548,671.44211444)(81.38172882,671.23212036)
\curveto(81.23172579,671.03211485)(81.11172591,670.79711508)(81.02172882,670.52712036)
\curveto(80.99172603,670.42711545)(80.96672606,670.32211556)(80.94672882,670.21212036)
\curveto(80.93672609,670.10211578)(80.9217261,669.99211589)(80.90172882,669.88212036)
\curveto(80.89172613,669.83211605)(80.88672614,669.7821161)(80.88672882,669.73212036)
\lineto(80.88672882,669.58212036)
\curveto(80.86672616,669.51211637)(80.85672617,669.40711647)(80.85672882,669.26712036)
\curveto(80.86672616,669.12711675)(80.88172614,669.02211686)(80.90172882,668.95212036)
\lineto(80.90172882,668.81712036)
\curveto(80.9217261,668.73711714)(80.93672609,668.65711722)(80.94672882,668.57712036)
\curveto(80.95672607,668.50711737)(80.97172605,668.43211745)(80.99172882,668.35212036)
\curveto(81.09172593,668.05211783)(81.19672583,667.80711807)(81.30672882,667.61712036)
\curveto(81.4267256,667.43711844)(81.61172541,667.27211861)(81.86172882,667.12212036)
\curveto(81.93172509,667.07211881)(82.00672502,667.03211885)(82.08672882,667.00212036)
\curveto(82.17672485,666.97211891)(82.26672476,666.94711893)(82.35672882,666.92712036)
\curveto(82.39672463,666.91711896)(82.43172459,666.91211897)(82.46172882,666.91212036)
\curveto(82.49172453,666.92211896)(82.5267245,666.92211896)(82.56672882,666.91212036)
\lineto(82.68672882,666.88212036)
\curveto(82.73672429,666.882119)(82.78172424,666.88711899)(82.82172882,666.89712036)
\lineto(82.94172882,666.89712036)
\curveto(83.021724,666.91711896)(83.10172392,666.93211895)(83.18172882,666.94212036)
\curveto(83.26172376,666.95211893)(83.33672369,666.97211891)(83.40672882,667.00212036)
\curveto(83.66672336,667.10211878)(83.87672315,667.23711864)(84.03672882,667.40712036)
\curveto(84.19672283,667.5771183)(84.33172269,667.78711809)(84.44172882,668.03712036)
\curveto(84.48172254,668.13711774)(84.51172251,668.23711764)(84.53172882,668.33712036)
\curveto(84.55172247,668.43711744)(84.57672245,668.54211734)(84.60672882,668.65212036)
\curveto(84.61672241,668.69211719)(84.6217224,668.72711715)(84.62172882,668.75712036)
\curveto(84.6217224,668.79711708)(84.6267224,668.83711704)(84.63672882,668.87712036)
\lineto(84.63672882,669.01212036)
\curveto(84.63672239,669.06211682)(84.64172238,669.11211677)(84.65172882,669.16212036)
}
}
{
\newrgbcolor{curcolor}{0 0 0}
\pscustom[linestyle=none,fillstyle=solid,fillcolor=curcolor]
{
}
}
{
\newrgbcolor{curcolor}{0 0 0}
\pscustom[linestyle=none,fillstyle=solid,fillcolor=curcolor]
{
\newpath
\moveto(99.67680695,669.46212036)
\curveto(99.69679878,669.3821165)(99.69679878,669.29211659)(99.67680695,669.19212036)
\curveto(99.65679882,669.09211679)(99.62179886,669.02711685)(99.57180695,668.99712036)
\curveto(99.52179896,668.95711692)(99.44679903,668.92711695)(99.34680695,668.90712036)
\curveto(99.25679922,668.89711698)(99.15179933,668.88711699)(99.03180695,668.87712036)
\lineto(98.68680695,668.87712036)
\curveto(98.5767999,668.88711699)(98.4768,668.89211699)(98.38680695,668.89212036)
\lineto(94.72680695,668.89212036)
\lineto(94.51680695,668.89212036)
\curveto(94.45680402,668.89211699)(94.40180408,668.882117)(94.35180695,668.86212036)
\curveto(94.27180421,668.82211706)(94.22180426,668.7821171)(94.20180695,668.74212036)
\curveto(94.1818043,668.72211716)(94.16180432,668.6821172)(94.14180695,668.62212036)
\curveto(94.12180436,668.57211731)(94.11680436,668.52211736)(94.12680695,668.47212036)
\curveto(94.14680433,668.41211747)(94.15680432,668.35211753)(94.15680695,668.29212036)
\curveto(94.16680431,668.24211764)(94.1818043,668.18711769)(94.20180695,668.12712036)
\curveto(94.2818042,667.88711799)(94.3768041,667.68711819)(94.48680695,667.52712036)
\curveto(94.60680387,667.3771185)(94.76680371,667.24211864)(94.96680695,667.12212036)
\curveto(95.04680343,667.07211881)(95.12680335,667.03711884)(95.20680695,667.01712036)
\curveto(95.29680318,667.00711887)(95.38680309,666.98711889)(95.47680695,666.95712036)
\curveto(95.55680292,666.93711894)(95.66680281,666.92211896)(95.80680695,666.91212036)
\curveto(95.94680253,666.90211898)(96.06680241,666.90711897)(96.16680695,666.92712036)
\lineto(96.30180695,666.92712036)
\curveto(96.40180208,666.94711893)(96.49180199,666.96711891)(96.57180695,666.98712036)
\curveto(96.66180182,667.01711886)(96.74680173,667.04711883)(96.82680695,667.07712036)
\curveto(96.92680155,667.12711875)(97.03680144,667.19211869)(97.15680695,667.27212036)
\curveto(97.28680119,667.35211853)(97.3818011,667.43211845)(97.44180695,667.51212036)
\curveto(97.49180099,667.5821183)(97.54180094,667.64711823)(97.59180695,667.70712036)
\curveto(97.65180083,667.7771181)(97.72180076,667.82711805)(97.80180695,667.85712036)
\curveto(97.90180058,667.90711797)(98.02680045,667.92711795)(98.17680695,667.91712036)
\lineto(98.61180695,667.91712036)
\lineto(98.79180695,667.91712036)
\curveto(98.86179962,667.92711795)(98.92179956,667.92211796)(98.97180695,667.90212036)
\lineto(99.12180695,667.90212036)
\curveto(99.22179926,667.882118)(99.29179919,667.85711802)(99.33180695,667.82712036)
\curveto(99.37179911,667.80711807)(99.39179909,667.76211812)(99.39180695,667.69212036)
\curveto(99.40179908,667.62211826)(99.39679908,667.56211832)(99.37680695,667.51212036)
\curveto(99.32679915,667.37211851)(99.27179921,667.24711863)(99.21180695,667.13712036)
\curveto(99.15179933,667.02711885)(99.0817994,666.91711896)(99.00180695,666.80712036)
\curveto(98.7817997,666.4771194)(98.53179995,666.21211967)(98.25180695,666.01212036)
\curveto(97.97180051,665.81212007)(97.62180086,665.64212024)(97.20180695,665.50212036)
\curveto(97.09180139,665.46212042)(96.9818015,665.43712044)(96.87180695,665.42712036)
\curveto(96.76180172,665.41712046)(96.64680183,665.39712048)(96.52680695,665.36712036)
\curveto(96.48680199,665.35712052)(96.44180204,665.35712052)(96.39180695,665.36712036)
\curveto(96.35180213,665.36712051)(96.31180217,665.36212052)(96.27180695,665.35212036)
\lineto(96.10680695,665.35212036)
\curveto(96.05680242,665.33212055)(95.99680248,665.32712055)(95.92680695,665.33712036)
\curveto(95.86680261,665.33712054)(95.81180267,665.34212054)(95.76180695,665.35212036)
\curveto(95.6818028,665.36212052)(95.61180287,665.36212052)(95.55180695,665.35212036)
\curveto(95.49180299,665.34212054)(95.42680305,665.34712053)(95.35680695,665.36712036)
\curveto(95.30680317,665.38712049)(95.25180323,665.39712048)(95.19180695,665.39712036)
\curveto(95.13180335,665.39712048)(95.0768034,665.40712047)(95.02680695,665.42712036)
\curveto(94.91680356,665.44712043)(94.80680367,665.47212041)(94.69680695,665.50212036)
\curveto(94.58680389,665.52212036)(94.48680399,665.55712032)(94.39680695,665.60712036)
\curveto(94.28680419,665.64712023)(94.1818043,665.6821202)(94.08180695,665.71212036)
\curveto(93.99180449,665.75212013)(93.90680457,665.79712008)(93.82680695,665.84712036)
\curveto(93.50680497,666.04711983)(93.22180526,666.2771196)(92.97180695,666.53712036)
\curveto(92.72180576,666.80711907)(92.51680596,667.11711876)(92.35680695,667.46712036)
\curveto(92.30680617,667.5771183)(92.26680621,667.68711819)(92.23680695,667.79712036)
\curveto(92.20680627,667.91711796)(92.16680631,668.03711784)(92.11680695,668.15712036)
\curveto(92.10680637,668.19711768)(92.10180638,668.23211765)(92.10180695,668.26212036)
\curveto(92.10180638,668.30211758)(92.09680638,668.34211754)(92.08680695,668.38212036)
\curveto(92.04680643,668.50211738)(92.02180646,668.63211725)(92.01180695,668.77212036)
\lineto(91.98180695,669.19212036)
\curveto(91.9818065,669.24211664)(91.9768065,669.29711658)(91.96680695,669.35712036)
\curveto(91.96680651,669.41711646)(91.97180651,669.47211641)(91.98180695,669.52212036)
\lineto(91.98180695,669.70212036)
\lineto(92.02680695,670.06212036)
\curveto(92.06680641,670.23211565)(92.10180638,670.39711548)(92.13180695,670.55712036)
\curveto(92.16180632,670.71711516)(92.20680627,670.86711501)(92.26680695,671.00712036)
\curveto(92.69680578,672.04711383)(93.42680505,672.7821131)(94.45680695,673.21212036)
\curveto(94.59680388,673.27211261)(94.73680374,673.31211257)(94.87680695,673.33212036)
\curveto(95.02680345,673.36211252)(95.1818033,673.39711248)(95.34180695,673.43712036)
\curveto(95.42180306,673.44711243)(95.49680298,673.45211243)(95.56680695,673.45212036)
\curveto(95.63680284,673.45211243)(95.71180277,673.45711242)(95.79180695,673.46712036)
\curveto(96.30180218,673.4771124)(96.73680174,673.41711246)(97.09680695,673.28712036)
\curveto(97.46680101,673.16711271)(97.79680068,673.00711287)(98.08680695,672.80712036)
\curveto(98.1768003,672.74711313)(98.26680021,672.6771132)(98.35680695,672.59712036)
\curveto(98.44680003,672.52711335)(98.52679995,672.45211343)(98.59680695,672.37212036)
\curveto(98.62679985,672.32211356)(98.66679981,672.2821136)(98.71680695,672.25212036)
\curveto(98.79679968,672.14211374)(98.87179961,672.02711385)(98.94180695,671.90712036)
\curveto(99.01179947,671.79711408)(99.08679939,671.6821142)(99.16680695,671.56212036)
\curveto(99.21679926,671.47211441)(99.25679922,671.3771145)(99.28680695,671.27712036)
\curveto(99.32679915,671.18711469)(99.36679911,671.08711479)(99.40680695,670.97712036)
\curveto(99.45679902,670.84711503)(99.49679898,670.71211517)(99.52680695,670.57212036)
\curveto(99.55679892,670.43211545)(99.59179889,670.29211559)(99.63180695,670.15212036)
\curveto(99.65179883,670.07211581)(99.65679882,669.9821159)(99.64680695,669.88212036)
\curveto(99.64679883,669.79211609)(99.65679882,669.70711617)(99.67680695,669.62712036)
\lineto(99.67680695,669.46212036)
\moveto(97.42680695,670.34712036)
\curveto(97.49680098,670.44711543)(97.50180098,670.56711531)(97.44180695,670.70712036)
\curveto(97.39180109,670.85711502)(97.35180113,670.96711491)(97.32180695,671.03712036)
\curveto(97.1818013,671.30711457)(96.99680148,671.51211437)(96.76680695,671.65212036)
\curveto(96.53680194,671.80211408)(96.21680226,671.882114)(95.80680695,671.89212036)
\curveto(95.7768027,671.87211401)(95.74180274,671.86711401)(95.70180695,671.87712036)
\curveto(95.66180282,671.88711399)(95.62680285,671.88711399)(95.59680695,671.87712036)
\curveto(95.54680293,671.85711402)(95.49180299,671.84211404)(95.43180695,671.83212036)
\curveto(95.37180311,671.83211405)(95.31680316,671.82211406)(95.26680695,671.80212036)
\curveto(94.82680365,671.66211422)(94.50180398,671.38711449)(94.29180695,670.97712036)
\curveto(94.27180421,670.93711494)(94.24680423,670.882115)(94.21680695,670.81212036)
\curveto(94.19680428,670.75211513)(94.1818043,670.68711519)(94.17180695,670.61712036)
\curveto(94.16180432,670.55711532)(94.16180432,670.49711538)(94.17180695,670.43712036)
\curveto(94.19180429,670.3771155)(94.22680425,670.32711555)(94.27680695,670.28712036)
\curveto(94.35680412,670.23711564)(94.46680401,670.21211567)(94.60680695,670.21212036)
\lineto(95.01180695,670.21212036)
\lineto(96.67680695,670.21212036)
\lineto(97.11180695,670.21212036)
\curveto(97.27180121,670.22211566)(97.3768011,670.26711561)(97.42680695,670.34712036)
}
}
{
\newrgbcolor{curcolor}{0 0 0}
\pscustom[linestyle=none,fillstyle=solid,fillcolor=curcolor]
{
\newpath
\moveto(101.4350882,676.21212036)
\lineto(102.5300882,676.21212036)
\curveto(102.63008571,676.21210967)(102.72508562,676.20710967)(102.8150882,676.19712036)
\curveto(102.90508544,676.18710969)(102.97508537,676.15710972)(103.0250882,676.10712036)
\curveto(103.08508526,676.03710984)(103.11508523,675.94210994)(103.1150882,675.82212036)
\curveto(103.12508522,675.71211017)(103.13008521,675.59711028)(103.1300882,675.47712036)
\lineto(103.1300882,674.14212036)
\lineto(103.1300882,668.75712036)
\lineto(103.1300882,666.46212036)
\lineto(103.1300882,666.04212036)
\curveto(103.1400852,665.89211999)(103.12008522,665.7771201)(103.0700882,665.69712036)
\curveto(103.02008532,665.61712026)(102.93008541,665.56212032)(102.8000882,665.53212036)
\curveto(102.7400856,665.51212037)(102.67008567,665.50712037)(102.5900882,665.51712036)
\curveto(102.52008582,665.52712035)(102.45008589,665.53212035)(102.3800882,665.53212036)
\lineto(101.6600882,665.53212036)
\curveto(101.55008679,665.53212035)(101.45008689,665.53712034)(101.3600882,665.54712036)
\curveto(101.27008707,665.55712032)(101.19508715,665.58712029)(101.1350882,665.63712036)
\curveto(101.07508727,665.68712019)(101.0400873,665.76212012)(101.0300882,665.86212036)
\lineto(101.0300882,666.19212036)
\lineto(101.0300882,667.52712036)
\lineto(101.0300882,673.15212036)
\lineto(101.0300882,675.19212036)
\curveto(101.03008731,675.32211056)(101.02508732,675.4771104)(101.0150882,675.65712036)
\curveto(101.01508733,675.83711004)(101.0400873,675.96710991)(101.0900882,676.04712036)
\curveto(101.11008723,676.08710979)(101.13508721,676.11710976)(101.1650882,676.13712036)
\lineto(101.2850882,676.19712036)
\curveto(101.30508704,676.19710968)(101.33008701,676.19710968)(101.3600882,676.19712036)
\curveto(101.39008695,676.20710967)(101.41508693,676.21210967)(101.4350882,676.21212036)
}
}
{
\newrgbcolor{curcolor}{0 0 0}
\pscustom[linestyle=none,fillstyle=solid,fillcolor=curcolor]
{
\newpath
\moveto(112.1572757,669.46212036)
\curveto(112.17726753,669.3821165)(112.17726753,669.29211659)(112.1572757,669.19212036)
\curveto(112.13726757,669.09211679)(112.10226761,669.02711685)(112.0522757,668.99712036)
\curveto(112.00226771,668.95711692)(111.92726778,668.92711695)(111.8272757,668.90712036)
\curveto(111.73726797,668.89711698)(111.63226808,668.88711699)(111.5122757,668.87712036)
\lineto(111.1672757,668.87712036)
\curveto(111.05726865,668.88711699)(110.95726875,668.89211699)(110.8672757,668.89212036)
\lineto(107.2072757,668.89212036)
\lineto(106.9972757,668.89212036)
\curveto(106.93727277,668.89211699)(106.88227283,668.882117)(106.8322757,668.86212036)
\curveto(106.75227296,668.82211706)(106.70227301,668.7821171)(106.6822757,668.74212036)
\curveto(106.66227305,668.72211716)(106.64227307,668.6821172)(106.6222757,668.62212036)
\curveto(106.60227311,668.57211731)(106.59727311,668.52211736)(106.6072757,668.47212036)
\curveto(106.62727308,668.41211747)(106.63727307,668.35211753)(106.6372757,668.29212036)
\curveto(106.64727306,668.24211764)(106.66227305,668.18711769)(106.6822757,668.12712036)
\curveto(106.76227295,667.88711799)(106.85727285,667.68711819)(106.9672757,667.52712036)
\curveto(107.08727262,667.3771185)(107.24727246,667.24211864)(107.4472757,667.12212036)
\curveto(107.52727218,667.07211881)(107.6072721,667.03711884)(107.6872757,667.01712036)
\curveto(107.77727193,667.00711887)(107.86727184,666.98711889)(107.9572757,666.95712036)
\curveto(108.03727167,666.93711894)(108.14727156,666.92211896)(108.2872757,666.91212036)
\curveto(108.42727128,666.90211898)(108.54727116,666.90711897)(108.6472757,666.92712036)
\lineto(108.7822757,666.92712036)
\curveto(108.88227083,666.94711893)(108.97227074,666.96711891)(109.0522757,666.98712036)
\curveto(109.14227057,667.01711886)(109.22727048,667.04711883)(109.3072757,667.07712036)
\curveto(109.4072703,667.12711875)(109.51727019,667.19211869)(109.6372757,667.27212036)
\curveto(109.76726994,667.35211853)(109.86226985,667.43211845)(109.9222757,667.51212036)
\curveto(109.97226974,667.5821183)(110.02226969,667.64711823)(110.0722757,667.70712036)
\curveto(110.13226958,667.7771181)(110.20226951,667.82711805)(110.2822757,667.85712036)
\curveto(110.38226933,667.90711797)(110.5072692,667.92711795)(110.6572757,667.91712036)
\lineto(111.0922757,667.91712036)
\lineto(111.2722757,667.91712036)
\curveto(111.34226837,667.92711795)(111.40226831,667.92211796)(111.4522757,667.90212036)
\lineto(111.6022757,667.90212036)
\curveto(111.70226801,667.882118)(111.77226794,667.85711802)(111.8122757,667.82712036)
\curveto(111.85226786,667.80711807)(111.87226784,667.76211812)(111.8722757,667.69212036)
\curveto(111.88226783,667.62211826)(111.87726783,667.56211832)(111.8572757,667.51212036)
\curveto(111.8072679,667.37211851)(111.75226796,667.24711863)(111.6922757,667.13712036)
\curveto(111.63226808,667.02711885)(111.56226815,666.91711896)(111.4822757,666.80712036)
\curveto(111.26226845,666.4771194)(111.0122687,666.21211967)(110.7322757,666.01212036)
\curveto(110.45226926,665.81212007)(110.10226961,665.64212024)(109.6822757,665.50212036)
\curveto(109.57227014,665.46212042)(109.46227025,665.43712044)(109.3522757,665.42712036)
\curveto(109.24227047,665.41712046)(109.12727058,665.39712048)(109.0072757,665.36712036)
\curveto(108.96727074,665.35712052)(108.92227079,665.35712052)(108.8722757,665.36712036)
\curveto(108.83227088,665.36712051)(108.79227092,665.36212052)(108.7522757,665.35212036)
\lineto(108.5872757,665.35212036)
\curveto(108.53727117,665.33212055)(108.47727123,665.32712055)(108.4072757,665.33712036)
\curveto(108.34727136,665.33712054)(108.29227142,665.34212054)(108.2422757,665.35212036)
\curveto(108.16227155,665.36212052)(108.09227162,665.36212052)(108.0322757,665.35212036)
\curveto(107.97227174,665.34212054)(107.9072718,665.34712053)(107.8372757,665.36712036)
\curveto(107.78727192,665.38712049)(107.73227198,665.39712048)(107.6722757,665.39712036)
\curveto(107.6122721,665.39712048)(107.55727215,665.40712047)(107.5072757,665.42712036)
\curveto(107.39727231,665.44712043)(107.28727242,665.47212041)(107.1772757,665.50212036)
\curveto(107.06727264,665.52212036)(106.96727274,665.55712032)(106.8772757,665.60712036)
\curveto(106.76727294,665.64712023)(106.66227305,665.6821202)(106.5622757,665.71212036)
\curveto(106.47227324,665.75212013)(106.38727332,665.79712008)(106.3072757,665.84712036)
\curveto(105.98727372,666.04711983)(105.70227401,666.2771196)(105.4522757,666.53712036)
\curveto(105.20227451,666.80711907)(104.99727471,667.11711876)(104.8372757,667.46712036)
\curveto(104.78727492,667.5771183)(104.74727496,667.68711819)(104.7172757,667.79712036)
\curveto(104.68727502,667.91711796)(104.64727506,668.03711784)(104.5972757,668.15712036)
\curveto(104.58727512,668.19711768)(104.58227513,668.23211765)(104.5822757,668.26212036)
\curveto(104.58227513,668.30211758)(104.57727513,668.34211754)(104.5672757,668.38212036)
\curveto(104.52727518,668.50211738)(104.50227521,668.63211725)(104.4922757,668.77212036)
\lineto(104.4622757,669.19212036)
\curveto(104.46227525,669.24211664)(104.45727525,669.29711658)(104.4472757,669.35712036)
\curveto(104.44727526,669.41711646)(104.45227526,669.47211641)(104.4622757,669.52212036)
\lineto(104.4622757,669.70212036)
\lineto(104.5072757,670.06212036)
\curveto(104.54727516,670.23211565)(104.58227513,670.39711548)(104.6122757,670.55712036)
\curveto(104.64227507,670.71711516)(104.68727502,670.86711501)(104.7472757,671.00712036)
\curveto(105.17727453,672.04711383)(105.9072738,672.7821131)(106.9372757,673.21212036)
\curveto(107.07727263,673.27211261)(107.21727249,673.31211257)(107.3572757,673.33212036)
\curveto(107.5072722,673.36211252)(107.66227205,673.39711248)(107.8222757,673.43712036)
\curveto(107.90227181,673.44711243)(107.97727173,673.45211243)(108.0472757,673.45212036)
\curveto(108.11727159,673.45211243)(108.19227152,673.45711242)(108.2722757,673.46712036)
\curveto(108.78227093,673.4771124)(109.21727049,673.41711246)(109.5772757,673.28712036)
\curveto(109.94726976,673.16711271)(110.27726943,673.00711287)(110.5672757,672.80712036)
\curveto(110.65726905,672.74711313)(110.74726896,672.6771132)(110.8372757,672.59712036)
\curveto(110.92726878,672.52711335)(111.0072687,672.45211343)(111.0772757,672.37212036)
\curveto(111.1072686,672.32211356)(111.14726856,672.2821136)(111.1972757,672.25212036)
\curveto(111.27726843,672.14211374)(111.35226836,672.02711385)(111.4222757,671.90712036)
\curveto(111.49226822,671.79711408)(111.56726814,671.6821142)(111.6472757,671.56212036)
\curveto(111.69726801,671.47211441)(111.73726797,671.3771145)(111.7672757,671.27712036)
\curveto(111.8072679,671.18711469)(111.84726786,671.08711479)(111.8872757,670.97712036)
\curveto(111.93726777,670.84711503)(111.97726773,670.71211517)(112.0072757,670.57212036)
\curveto(112.03726767,670.43211545)(112.07226764,670.29211559)(112.1122757,670.15212036)
\curveto(112.13226758,670.07211581)(112.13726757,669.9821159)(112.1272757,669.88212036)
\curveto(112.12726758,669.79211609)(112.13726757,669.70711617)(112.1572757,669.62712036)
\lineto(112.1572757,669.46212036)
\moveto(109.9072757,670.34712036)
\curveto(109.97726973,670.44711543)(109.98226973,670.56711531)(109.9222757,670.70712036)
\curveto(109.87226984,670.85711502)(109.83226988,670.96711491)(109.8022757,671.03712036)
\curveto(109.66227005,671.30711457)(109.47727023,671.51211437)(109.2472757,671.65212036)
\curveto(109.01727069,671.80211408)(108.69727101,671.882114)(108.2872757,671.89212036)
\curveto(108.25727145,671.87211401)(108.22227149,671.86711401)(108.1822757,671.87712036)
\curveto(108.14227157,671.88711399)(108.1072716,671.88711399)(108.0772757,671.87712036)
\curveto(108.02727168,671.85711402)(107.97227174,671.84211404)(107.9122757,671.83212036)
\curveto(107.85227186,671.83211405)(107.79727191,671.82211406)(107.7472757,671.80212036)
\curveto(107.3072724,671.66211422)(106.98227273,671.38711449)(106.7722757,670.97712036)
\curveto(106.75227296,670.93711494)(106.72727298,670.882115)(106.6972757,670.81212036)
\curveto(106.67727303,670.75211513)(106.66227305,670.68711519)(106.6522757,670.61712036)
\curveto(106.64227307,670.55711532)(106.64227307,670.49711538)(106.6522757,670.43712036)
\curveto(106.67227304,670.3771155)(106.707273,670.32711555)(106.7572757,670.28712036)
\curveto(106.83727287,670.23711564)(106.94727276,670.21211567)(107.0872757,670.21212036)
\lineto(107.4922757,670.21212036)
\lineto(109.1572757,670.21212036)
\lineto(109.5922757,670.21212036)
\curveto(109.75226996,670.22211566)(109.85726985,670.26711561)(109.9072757,670.34712036)
}
}
{
\newrgbcolor{curcolor}{0 0 0}
\pscustom[linestyle=none,fillstyle=solid,fillcolor=curcolor]
{
\newpath
\moveto(116.97555695,673.46712036)
\curveto(117.78555179,673.48711239)(118.46055111,673.36711251)(119.00055695,673.10712036)
\curveto(119.55055002,672.84711303)(119.98554959,672.4771134)(120.30555695,671.99712036)
\curveto(120.46554911,671.75711412)(120.58554899,671.4821144)(120.66555695,671.17212036)
\curveto(120.68554889,671.12211476)(120.70054887,671.05711482)(120.71055695,670.97712036)
\curveto(120.73054884,670.89711498)(120.73054884,670.82711505)(120.71055695,670.76712036)
\curveto(120.6705489,670.65711522)(120.60054897,670.59211529)(120.50055695,670.57212036)
\curveto(120.40054917,670.56211532)(120.28054929,670.55711532)(120.14055695,670.55712036)
\lineto(119.36055695,670.55712036)
\lineto(119.07555695,670.55712036)
\curveto(118.98555059,670.55711532)(118.91055066,670.5771153)(118.85055695,670.61712036)
\curveto(118.7705508,670.65711522)(118.71555086,670.71711516)(118.68555695,670.79712036)
\curveto(118.65555092,670.88711499)(118.61555096,670.9771149)(118.56555695,671.06712036)
\curveto(118.50555107,671.1771147)(118.44055113,671.2771146)(118.37055695,671.36712036)
\curveto(118.30055127,671.45711442)(118.22055135,671.53711434)(118.13055695,671.60712036)
\curveto(117.99055158,671.69711418)(117.83555174,671.76711411)(117.66555695,671.81712036)
\curveto(117.60555197,671.83711404)(117.54555203,671.84711403)(117.48555695,671.84712036)
\curveto(117.42555215,671.84711403)(117.3705522,671.85711402)(117.32055695,671.87712036)
\lineto(117.17055695,671.87712036)
\curveto(116.9705526,671.877114)(116.81055276,671.85711402)(116.69055695,671.81712036)
\curveto(116.40055317,671.72711415)(116.16555341,671.58711429)(115.98555695,671.39712036)
\curveto(115.80555377,671.21711466)(115.66055391,670.99711488)(115.55055695,670.73712036)
\curveto(115.50055407,670.62711525)(115.46055411,670.50711537)(115.43055695,670.37712036)
\curveto(115.41055416,670.25711562)(115.38555419,670.12711575)(115.35555695,669.98712036)
\curveto(115.34555423,669.94711593)(115.34055423,669.90711597)(115.34055695,669.86712036)
\curveto(115.34055423,669.82711605)(115.33555424,669.78711609)(115.32555695,669.74712036)
\curveto(115.30555427,669.64711623)(115.29555428,669.50711637)(115.29555695,669.32712036)
\curveto(115.30555427,669.14711673)(115.32055425,669.00711687)(115.34055695,668.90712036)
\curveto(115.34055423,668.82711705)(115.34555423,668.77211711)(115.35555695,668.74212036)
\curveto(115.3755542,668.67211721)(115.38555419,668.60211728)(115.38555695,668.53212036)
\curveto(115.39555418,668.46211742)(115.41055416,668.39211749)(115.43055695,668.32212036)
\curveto(115.51055406,668.09211779)(115.60555397,667.882118)(115.71555695,667.69212036)
\curveto(115.82555375,667.50211838)(115.96555361,667.34211854)(116.13555695,667.21212036)
\curveto(116.1755534,667.1821187)(116.23555334,667.14711873)(116.31555695,667.10712036)
\curveto(116.42555315,667.03711884)(116.53555304,666.99211889)(116.64555695,666.97212036)
\curveto(116.76555281,666.95211893)(116.91055266,666.93211895)(117.08055695,666.91212036)
\lineto(117.17055695,666.91212036)
\curveto(117.21055236,666.91211897)(117.24055233,666.91711896)(117.26055695,666.92712036)
\lineto(117.39555695,666.92712036)
\curveto(117.46555211,666.94711893)(117.53055204,666.96211892)(117.59055695,666.97212036)
\curveto(117.66055191,666.99211889)(117.72555185,667.01211887)(117.78555695,667.03212036)
\curveto(118.08555149,667.16211872)(118.31555126,667.35211853)(118.47555695,667.60212036)
\curveto(118.51555106,667.65211823)(118.55055102,667.70711817)(118.58055695,667.76712036)
\curveto(118.61055096,667.83711804)(118.63555094,667.89711798)(118.65555695,667.94712036)
\curveto(118.69555088,668.05711782)(118.73055084,668.15211773)(118.76055695,668.23212036)
\curveto(118.79055078,668.32211756)(118.86055071,668.39211749)(118.97055695,668.44212036)
\curveto(119.06055051,668.4821174)(119.20555037,668.49711738)(119.40555695,668.48712036)
\lineto(119.90055695,668.48712036)
\lineto(120.11055695,668.48712036)
\curveto(120.19054938,668.49711738)(120.25554932,668.49211739)(120.30555695,668.47212036)
\lineto(120.42555695,668.47212036)
\lineto(120.54555695,668.44212036)
\curveto(120.58554899,668.44211744)(120.61554896,668.43211745)(120.63555695,668.41212036)
\curveto(120.68554889,668.37211751)(120.71554886,668.31211757)(120.72555695,668.23212036)
\curveto(120.74554883,668.16211772)(120.74554883,668.08711779)(120.72555695,668.00712036)
\curveto(120.63554894,667.6771182)(120.52554905,667.3821185)(120.39555695,667.12212036)
\curveto(119.98554959,666.35211953)(119.33055024,665.81712006)(118.43055695,665.51712036)
\curveto(118.33055124,665.48712039)(118.22555135,665.46712041)(118.11555695,665.45712036)
\curveto(118.00555157,665.43712044)(117.89555168,665.41212047)(117.78555695,665.38212036)
\curveto(117.72555185,665.37212051)(117.66555191,665.36712051)(117.60555695,665.36712036)
\curveto(117.54555203,665.36712051)(117.48555209,665.36212052)(117.42555695,665.35212036)
\lineto(117.26055695,665.35212036)
\curveto(117.21055236,665.33212055)(117.13555244,665.32712055)(117.03555695,665.33712036)
\curveto(116.93555264,665.33712054)(116.86055271,665.34212054)(116.81055695,665.35212036)
\curveto(116.73055284,665.37212051)(116.65555292,665.3821205)(116.58555695,665.38212036)
\curveto(116.52555305,665.37212051)(116.46055311,665.3771205)(116.39055695,665.39712036)
\lineto(116.24055695,665.42712036)
\curveto(116.19055338,665.42712045)(116.14055343,665.43212045)(116.09055695,665.44212036)
\curveto(115.98055359,665.47212041)(115.8755537,665.50212038)(115.77555695,665.53212036)
\curveto(115.6755539,665.56212032)(115.58055399,665.59712028)(115.49055695,665.63712036)
\curveto(115.02055455,665.83712004)(114.62555495,666.09211979)(114.30555695,666.40212036)
\curveto(113.98555559,666.72211916)(113.72555585,667.11711876)(113.52555695,667.58712036)
\curveto(113.4755561,667.6771182)(113.43555614,667.77211811)(113.40555695,667.87212036)
\lineto(113.31555695,668.20212036)
\curveto(113.30555627,668.24211764)(113.30055627,668.2771176)(113.30055695,668.30712036)
\curveto(113.30055627,668.34711753)(113.29055628,668.39211749)(113.27055695,668.44212036)
\curveto(113.25055632,668.51211737)(113.24055633,668.5821173)(113.24055695,668.65212036)
\curveto(113.24055633,668.73211715)(113.23055634,668.80711707)(113.21055695,668.87712036)
\lineto(113.21055695,669.13212036)
\curveto(113.19055638,669.1821167)(113.18055639,669.23711664)(113.18055695,669.29712036)
\curveto(113.18055639,669.36711651)(113.19055638,669.42711645)(113.21055695,669.47712036)
\curveto(113.22055635,669.52711635)(113.22055635,669.57211631)(113.21055695,669.61212036)
\curveto(113.20055637,669.65211623)(113.20055637,669.69211619)(113.21055695,669.73212036)
\curveto(113.23055634,669.80211608)(113.23555634,669.86711601)(113.22555695,669.92712036)
\curveto(113.22555635,669.98711589)(113.23555634,670.04711583)(113.25555695,670.10712036)
\curveto(113.30555627,670.28711559)(113.34555623,670.45711542)(113.37555695,670.61712036)
\curveto(113.40555617,670.78711509)(113.45055612,670.95211493)(113.51055695,671.11212036)
\curveto(113.73055584,671.62211426)(114.00555557,672.04711383)(114.33555695,672.38712036)
\curveto(114.6755549,672.72711315)(115.10555447,673.00211288)(115.62555695,673.21212036)
\curveto(115.76555381,673.27211261)(115.91055366,673.31211257)(116.06055695,673.33212036)
\curveto(116.21055336,673.36211252)(116.36555321,673.39711248)(116.52555695,673.43712036)
\curveto(116.60555297,673.44711243)(116.68055289,673.45211243)(116.75055695,673.45212036)
\curveto(116.82055275,673.45211243)(116.89555268,673.45711242)(116.97555695,673.46712036)
}
}
{
\newrgbcolor{curcolor}{0 0 0}
\pscustom[linestyle=none,fillstyle=solid,fillcolor=curcolor]
{
\newpath
\moveto(123.0688382,675.56712036)
\lineto(124.0738382,675.56712036)
\curveto(124.22383521,675.56711031)(124.35383508,675.55711032)(124.4638382,675.53712036)
\curveto(124.58383485,675.52711035)(124.66883477,675.46711041)(124.7188382,675.35712036)
\curveto(124.7388347,675.30711057)(124.74883469,675.24711063)(124.7488382,675.17712036)
\lineto(124.7488382,674.96712036)
\lineto(124.7488382,674.29212036)
\curveto(124.74883469,674.24211164)(124.74383469,674.1821117)(124.7338382,674.11212036)
\curveto(124.7338347,674.05211183)(124.7388347,673.99711188)(124.7488382,673.94712036)
\lineto(124.7488382,673.78212036)
\curveto(124.74883469,673.70211218)(124.75383468,673.62711225)(124.7638382,673.55712036)
\curveto(124.77383466,673.49711238)(124.79883464,673.44211244)(124.8388382,673.39212036)
\curveto(124.90883453,673.30211258)(125.0338344,673.25211263)(125.2138382,673.24212036)
\lineto(125.7538382,673.24212036)
\lineto(125.9338382,673.24212036)
\curveto(125.99383344,673.24211264)(126.04883339,673.23211265)(126.0988382,673.21212036)
\curveto(126.20883323,673.16211272)(126.26883317,673.07211281)(126.2788382,672.94212036)
\curveto(126.29883314,672.81211307)(126.30883313,672.66711321)(126.3088382,672.50712036)
\lineto(126.3088382,672.29712036)
\curveto(126.31883312,672.22711365)(126.31383312,672.16711371)(126.2938382,672.11712036)
\curveto(126.24383319,671.95711392)(126.1388333,671.87211401)(125.9788382,671.86212036)
\curveto(125.81883362,671.85211403)(125.6388338,671.84711403)(125.4388382,671.84712036)
\lineto(125.3038382,671.84712036)
\curveto(125.26383417,671.85711402)(125.22883421,671.85711402)(125.1988382,671.84712036)
\curveto(125.15883428,671.83711404)(125.12383431,671.83211405)(125.0938382,671.83212036)
\curveto(125.06383437,671.84211404)(125.0338344,671.83711404)(125.0038382,671.81712036)
\curveto(124.92383451,671.79711408)(124.86383457,671.75211413)(124.8238382,671.68212036)
\curveto(124.79383464,671.62211426)(124.76883467,671.54711433)(124.7488382,671.45712036)
\curveto(124.7388347,671.40711447)(124.7388347,671.35211453)(124.7488382,671.29212036)
\curveto(124.75883468,671.23211465)(124.75883468,671.1771147)(124.7488382,671.12712036)
\lineto(124.7488382,670.19712036)
\lineto(124.7488382,668.44212036)
\curveto(124.74883469,668.19211769)(124.75383468,667.97211791)(124.7638382,667.78212036)
\curveto(124.78383465,667.60211828)(124.84883459,667.44211844)(124.9588382,667.30212036)
\curveto(125.00883443,667.24211864)(125.07383436,667.19711868)(125.1538382,667.16712036)
\lineto(125.4238382,667.10712036)
\curveto(125.45383398,667.09711878)(125.48383395,667.09211879)(125.5138382,667.09212036)
\curveto(125.55383388,667.10211878)(125.58383385,667.10211878)(125.6038382,667.09212036)
\lineto(125.7688382,667.09212036)
\curveto(125.87883356,667.09211879)(125.97383346,667.08711879)(126.0538382,667.07712036)
\curveto(126.1338333,667.06711881)(126.19883324,667.02711885)(126.2488382,666.95712036)
\curveto(126.28883315,666.89711898)(126.30883313,666.81711906)(126.3088382,666.71712036)
\lineto(126.3088382,666.43212036)
\curveto(126.30883313,666.22211966)(126.30383313,666.02711985)(126.2938382,665.84712036)
\curveto(126.29383314,665.6771202)(126.21383322,665.56212032)(126.0538382,665.50212036)
\curveto(126.00383343,665.4821204)(125.95883348,665.4771204)(125.9188382,665.48712036)
\curveto(125.87883356,665.48712039)(125.8338336,665.4771204)(125.7838382,665.45712036)
\lineto(125.6338382,665.45712036)
\curveto(125.61383382,665.45712042)(125.58383385,665.46212042)(125.5438382,665.47212036)
\curveto(125.50383393,665.47212041)(125.46883397,665.46712041)(125.4388382,665.45712036)
\curveto(125.38883405,665.44712043)(125.3338341,665.44712043)(125.2738382,665.45712036)
\lineto(125.1238382,665.45712036)
\lineto(124.9738382,665.45712036)
\curveto(124.92383451,665.44712043)(124.87883456,665.44712043)(124.8388382,665.45712036)
\lineto(124.6738382,665.45712036)
\curveto(124.62383481,665.46712041)(124.56883487,665.47212041)(124.5088382,665.47212036)
\curveto(124.44883499,665.47212041)(124.39383504,665.4771204)(124.3438382,665.48712036)
\curveto(124.27383516,665.49712038)(124.20883523,665.50712037)(124.1488382,665.51712036)
\lineto(123.9688382,665.54712036)
\curveto(123.85883558,665.5771203)(123.75383568,665.61212027)(123.6538382,665.65212036)
\curveto(123.55383588,665.69212019)(123.45883598,665.73712014)(123.3688382,665.78712036)
\lineto(123.2788382,665.84712036)
\curveto(123.24883619,665.87712)(123.21383622,665.90711997)(123.1738382,665.93712036)
\curveto(123.15383628,665.95711992)(123.12883631,665.9771199)(123.0988382,665.99712036)
\lineto(123.0238382,666.07212036)
\curveto(122.88383655,666.26211962)(122.77883666,666.47211941)(122.7088382,666.70212036)
\curveto(122.68883675,666.74211914)(122.67883676,666.7771191)(122.6788382,666.80712036)
\curveto(122.68883675,666.84711903)(122.68883675,666.89211899)(122.6788382,666.94212036)
\curveto(122.66883677,666.96211892)(122.66383677,666.98711889)(122.6638382,667.01712036)
\curveto(122.66383677,667.04711883)(122.65883678,667.07211881)(122.6488382,667.09212036)
\lineto(122.6488382,667.24212036)
\curveto(122.6388368,667.2821186)(122.6338368,667.32711855)(122.6338382,667.37712036)
\curveto(122.64383679,667.42711845)(122.64883679,667.4771184)(122.6488382,667.52712036)
\lineto(122.6488382,668.09712036)
\lineto(122.6488382,670.33212036)
\lineto(122.6488382,671.12712036)
\lineto(122.6488382,671.33712036)
\curveto(122.65883678,671.40711447)(122.65383678,671.47211441)(122.6338382,671.53212036)
\curveto(122.59383684,671.67211421)(122.52383691,671.76211412)(122.4238382,671.80212036)
\curveto(122.31383712,671.85211403)(122.17383726,671.86711401)(122.0038382,671.84712036)
\curveto(121.8338376,671.82711405)(121.68883775,671.84211404)(121.5688382,671.89212036)
\curveto(121.48883795,671.92211396)(121.438838,671.96711391)(121.4188382,672.02712036)
\curveto(121.39883804,672.08711379)(121.37883806,672.16211372)(121.3588382,672.25212036)
\lineto(121.3588382,672.56712036)
\curveto(121.35883808,672.74711313)(121.36883807,672.89211299)(121.3888382,673.00212036)
\curveto(121.40883803,673.11211277)(121.49383794,673.18711269)(121.6438382,673.22712036)
\curveto(121.68383775,673.24711263)(121.72383771,673.25211263)(121.7638382,673.24212036)
\lineto(121.8988382,673.24212036)
\curveto(122.04883739,673.24211264)(122.18883725,673.24711263)(122.3188382,673.25712036)
\curveto(122.44883699,673.2771126)(122.5388369,673.33711254)(122.5888382,673.43712036)
\curveto(122.61883682,673.50711237)(122.6338368,673.58711229)(122.6338382,673.67712036)
\curveto(122.64383679,673.76711211)(122.64883679,673.85711202)(122.6488382,673.94712036)
\lineto(122.6488382,674.87712036)
\lineto(122.6488382,675.13212036)
\curveto(122.64883679,675.22211066)(122.65883678,675.29711058)(122.6788382,675.35712036)
\curveto(122.72883671,675.45711042)(122.80383663,675.52211036)(122.9038382,675.55212036)
\curveto(122.92383651,675.56211032)(122.94883649,675.56211032)(122.9788382,675.55212036)
\curveto(123.01883642,675.55211033)(123.04883639,675.55711032)(123.0688382,675.56712036)
}
}
{
\newrgbcolor{curcolor}{0 0 0}
\pscustom[linestyle=none,fillstyle=solid,fillcolor=curcolor]
{
\newpath
\moveto(131.7172757,673.45212036)
\curveto(131.82727038,673.45211243)(131.92227029,673.44211244)(132.0022757,673.42212036)
\curveto(132.09227012,673.40211248)(132.16227005,673.35711252)(132.2122757,673.28712036)
\curveto(132.27226994,673.20711267)(132.30226991,673.06711281)(132.3022757,672.86712036)
\lineto(132.3022757,672.35712036)
\lineto(132.3022757,671.98212036)
\curveto(132.3122699,671.84211404)(132.29726991,671.73211415)(132.2572757,671.65212036)
\curveto(132.21726999,671.5821143)(132.15727005,671.53711434)(132.0772757,671.51712036)
\curveto(132.0072702,671.49711438)(131.92227029,671.48711439)(131.8222757,671.48712036)
\curveto(131.73227048,671.48711439)(131.63227058,671.49211439)(131.5222757,671.50212036)
\curveto(131.42227079,671.51211437)(131.32727088,671.50711437)(131.2372757,671.48712036)
\curveto(131.16727104,671.46711441)(131.09727111,671.45211443)(131.0272757,671.44212036)
\curveto(130.95727125,671.44211444)(130.89227132,671.43211445)(130.8322757,671.41212036)
\curveto(130.67227154,671.36211452)(130.5122717,671.28711459)(130.3522757,671.18712036)
\curveto(130.19227202,671.09711478)(130.06727214,670.99211489)(129.9772757,670.87212036)
\curveto(129.92727228,670.79211509)(129.87227234,670.70711517)(129.8122757,670.61712036)
\curveto(129.76227245,670.53711534)(129.7122725,670.45211543)(129.6622757,670.36212036)
\curveto(129.63227258,670.2821156)(129.60227261,670.19711568)(129.5722757,670.10712036)
\lineto(129.5122757,669.86712036)
\curveto(129.49227272,669.79711608)(129.48227273,669.72211616)(129.4822757,669.64212036)
\curveto(129.48227273,669.57211631)(129.47227274,669.50211638)(129.4522757,669.43212036)
\curveto(129.44227277,669.39211649)(129.43727277,669.35211653)(129.4372757,669.31212036)
\curveto(129.44727276,669.2821166)(129.44727276,669.25211663)(129.4372757,669.22212036)
\lineto(129.4372757,668.98212036)
\curveto(129.41727279,668.91211697)(129.4122728,668.83211705)(129.4222757,668.74212036)
\curveto(129.43227278,668.66211722)(129.43727277,668.5821173)(129.4372757,668.50212036)
\lineto(129.4372757,667.54212036)
\lineto(129.4372757,666.26712036)
\curveto(129.43727277,666.13711974)(129.43227278,666.01711986)(129.4222757,665.90712036)
\curveto(129.4122728,665.79712008)(129.38227283,665.70712017)(129.3322757,665.63712036)
\curveto(129.3122729,665.60712027)(129.27727293,665.5821203)(129.2272757,665.56212036)
\curveto(129.18727302,665.55212033)(129.14227307,665.54212034)(129.0922757,665.53212036)
\lineto(129.0172757,665.53212036)
\curveto(128.96727324,665.52212036)(128.9122733,665.51712036)(128.8522757,665.51712036)
\lineto(128.6872757,665.51712036)
\lineto(128.0422757,665.51712036)
\curveto(127.98227423,665.52712035)(127.91727429,665.53212035)(127.8472757,665.53212036)
\lineto(127.6522757,665.53212036)
\curveto(127.60227461,665.55212033)(127.55227466,665.56712031)(127.5022757,665.57712036)
\curveto(127.45227476,665.59712028)(127.41727479,665.63212025)(127.3972757,665.68212036)
\curveto(127.35727485,665.73212015)(127.33227488,665.80212008)(127.3222757,665.89212036)
\lineto(127.3222757,666.19212036)
\lineto(127.3222757,667.21212036)
\lineto(127.3222757,671.44212036)
\lineto(127.3222757,672.55212036)
\lineto(127.3222757,672.83712036)
\curveto(127.32227489,672.93711294)(127.34227487,673.01711286)(127.3822757,673.07712036)
\curveto(127.43227478,673.15711272)(127.5072747,673.20711267)(127.6072757,673.22712036)
\curveto(127.7072745,673.24711263)(127.82727438,673.25711262)(127.9672757,673.25712036)
\lineto(128.7322757,673.25712036)
\curveto(128.85227336,673.25711262)(128.95727325,673.24711263)(129.0472757,673.22712036)
\curveto(129.13727307,673.21711266)(129.207273,673.17211271)(129.2572757,673.09212036)
\curveto(129.28727292,673.04211284)(129.30227291,672.97211291)(129.3022757,672.88212036)
\lineto(129.3322757,672.61212036)
\curveto(129.34227287,672.53211335)(129.35727285,672.45711342)(129.3772757,672.38712036)
\curveto(129.4072728,672.31711356)(129.45727275,672.2821136)(129.5272757,672.28212036)
\curveto(129.54727266,672.30211358)(129.56727264,672.31211357)(129.5872757,672.31212036)
\curveto(129.6072726,672.31211357)(129.62727258,672.32211356)(129.6472757,672.34212036)
\curveto(129.7072725,672.39211349)(129.75727245,672.44711343)(129.7972757,672.50712036)
\curveto(129.84727236,672.5771133)(129.9072723,672.63711324)(129.9772757,672.68712036)
\curveto(130.01727219,672.71711316)(130.05227216,672.74711313)(130.0822757,672.77712036)
\curveto(130.1122721,672.81711306)(130.14727206,672.85211303)(130.1872757,672.88212036)
\lineto(130.4572757,673.06212036)
\curveto(130.55727165,673.12211276)(130.65727155,673.1771127)(130.7572757,673.22712036)
\curveto(130.85727135,673.26711261)(130.95727125,673.30211258)(131.0572757,673.33212036)
\lineto(131.3872757,673.42212036)
\curveto(131.41727079,673.43211245)(131.47227074,673.43211245)(131.5522757,673.42212036)
\curveto(131.64227057,673.42211246)(131.69727051,673.43211245)(131.7172757,673.45212036)
}
}
{
\newrgbcolor{curcolor}{0 0 0}
\pscustom[linestyle=none,fillstyle=solid,fillcolor=curcolor]
{
\newpath
\moveto(140.89235382,669.70212036)
\curveto(140.87234529,669.75211613)(140.8673453,669.80711607)(140.87735382,669.86712036)
\curveto(140.88734528,669.92711595)(140.88234528,669.9821159)(140.86235382,670.03212036)
\curveto(140.85234531,670.07211581)(140.84734532,670.11211577)(140.84735382,670.15212036)
\curveto(140.84734532,670.19211569)(140.84234532,670.23211565)(140.83235382,670.27212036)
\lineto(140.77235382,670.54212036)
\curveto(140.75234541,670.63211525)(140.72734544,670.71711516)(140.69735382,670.79712036)
\curveto(140.64734552,670.93711494)(140.60234556,671.06711481)(140.56235382,671.18712036)
\curveto(140.52234564,671.31711456)(140.4673457,671.43711444)(140.39735382,671.54712036)
\curveto(140.32734584,671.65711422)(140.25734591,671.76211412)(140.18735382,671.86212036)
\curveto(140.12734604,671.96211392)(140.05734611,672.06211382)(139.97735382,672.16212036)
\curveto(139.89734627,672.27211361)(139.79734637,672.37211351)(139.67735382,672.46212036)
\curveto(139.5673466,672.56211332)(139.45734671,672.65211323)(139.34735382,672.73212036)
\curveto(139.01734715,672.96211292)(138.63734753,673.14211274)(138.20735382,673.27212036)
\curveto(137.78734838,673.40211248)(137.28734888,673.46211242)(136.70735382,673.45212036)
\curveto(136.63734953,673.44211244)(136.5673496,673.43711244)(136.49735382,673.43712036)
\curveto(136.42734974,673.43711244)(136.35234981,673.43211245)(136.27235382,673.42212036)
\curveto(136.12235004,673.3821125)(135.97735019,673.35211253)(135.83735382,673.33212036)
\curveto(135.69735047,673.31211257)(135.5623506,673.2771126)(135.43235382,673.22712036)
\curveto(135.32235084,673.1771127)(135.21235095,673.13211275)(135.10235382,673.09212036)
\curveto(134.99235117,673.05211283)(134.88735128,673.00711287)(134.78735382,672.95712036)
\curveto(134.42735174,672.72711315)(134.12235204,672.47211341)(133.87235382,672.19212036)
\curveto(133.62235254,671.92211396)(133.40735276,671.5821143)(133.22735382,671.17212036)
\curveto(133.17735299,671.05211483)(133.13735303,670.92711495)(133.10735382,670.79712036)
\curveto(133.07735309,670.6771152)(133.04235312,670.55211533)(133.00235382,670.42212036)
\curveto(132.98235318,670.37211551)(132.97235319,670.32211556)(132.97235382,670.27212036)
\curveto(132.97235319,670.23211565)(132.9673532,670.18711569)(132.95735382,670.13712036)
\curveto(132.93735323,670.08711579)(132.92735324,670.03211585)(132.92735382,669.97212036)
\curveto(132.93735323,669.92211596)(132.93735323,669.87211601)(132.92735382,669.82212036)
\lineto(132.92735382,669.71712036)
\curveto(132.90735326,669.65711622)(132.89235327,669.57211631)(132.88235382,669.46212036)
\curveto(132.88235328,669.35211653)(132.89235327,669.26711661)(132.91235382,669.20712036)
\lineto(132.91235382,669.07212036)
\curveto(132.91235325,669.03211685)(132.91735325,668.98711689)(132.92735382,668.93712036)
\curveto(132.94735322,668.85711702)(132.95735321,668.77211711)(132.95735382,668.68212036)
\curveto(132.95735321,668.60211728)(132.9673532,668.52211736)(132.98735382,668.44212036)
\curveto(133.00735316,668.39211749)(133.01735315,668.34711753)(133.01735382,668.30712036)
\curveto(133.01735315,668.26711761)(133.02735314,668.22211766)(133.04735382,668.17212036)
\curveto(133.07735309,668.06211782)(133.10235306,667.95711792)(133.12235382,667.85712036)
\curveto(133.15235301,667.75711812)(133.19235297,667.66211822)(133.24235382,667.57212036)
\curveto(133.41235275,667.1821187)(133.62235254,666.84711903)(133.87235382,666.56712036)
\curveto(134.12235204,666.28711959)(134.42235174,666.04211984)(134.77235382,665.83212036)
\curveto(134.88235128,665.77212011)(134.98735118,665.72212016)(135.08735382,665.68212036)
\curveto(135.19735097,665.64212024)(135.31235085,665.60212028)(135.43235382,665.56212036)
\curveto(135.52235064,665.52212036)(135.61735055,665.49212039)(135.71735382,665.47212036)
\curveto(135.81735035,665.45212043)(135.91735025,665.42712045)(136.01735382,665.39712036)
\curveto(136.0673501,665.38712049)(136.10735006,665.3821205)(136.13735382,665.38212036)
\curveto(136.17734999,665.3821205)(136.21734995,665.3771205)(136.25735382,665.36712036)
\curveto(136.30734986,665.34712053)(136.35734981,665.34212054)(136.40735382,665.35212036)
\curveto(136.4673497,665.35212053)(136.52234964,665.34712053)(136.57235382,665.33712036)
\lineto(136.72235382,665.33712036)
\curveto(136.78234938,665.31712056)(136.8673493,665.31212057)(136.97735382,665.32212036)
\curveto(137.08734908,665.32212056)(137.167349,665.32712055)(137.21735382,665.33712036)
\curveto(137.24734892,665.33712054)(137.27734889,665.34212054)(137.30735382,665.35212036)
\lineto(137.41235382,665.35212036)
\curveto(137.4623487,665.36212052)(137.51734865,665.36712051)(137.57735382,665.36712036)
\curveto(137.63734853,665.36712051)(137.69234847,665.3771205)(137.74235382,665.39712036)
\curveto(137.87234829,665.42712045)(137.99734817,665.45712042)(138.11735382,665.48712036)
\curveto(138.24734792,665.50712037)(138.37234779,665.54212034)(138.49235382,665.59212036)
\curveto(138.97234719,665.79212009)(139.38234678,666.04211984)(139.72235382,666.34212036)
\curveto(140.0623461,666.64211924)(140.33734583,667.03211885)(140.54735382,667.51212036)
\curveto(140.59734557,667.61211827)(140.63734553,667.71711816)(140.66735382,667.82712036)
\curveto(140.69734547,667.94711793)(140.73234543,668.06211782)(140.77235382,668.17212036)
\curveto(140.78234538,668.24211764)(140.79234537,668.30711757)(140.80235382,668.36712036)
\curveto(140.81234535,668.42711745)(140.82734534,668.49211739)(140.84735382,668.56212036)
\curveto(140.8673453,668.64211724)(140.87234529,668.72211716)(140.86235382,668.80212036)
\curveto(140.8623453,668.882117)(140.87234529,668.96211692)(140.89235382,669.04212036)
\lineto(140.89235382,669.19212036)
\curveto(140.91234525,669.25211663)(140.92234524,669.33711654)(140.92235382,669.44712036)
\curveto(140.92234524,669.55711632)(140.91234525,669.64211624)(140.89235382,669.70212036)
\moveto(138.79235382,669.16212036)
\curveto(138.78234738,669.11211677)(138.77734739,669.06211682)(138.77735382,669.01212036)
\lineto(138.77735382,668.87712036)
\curveto(138.7673474,668.83711704)(138.7623474,668.79711708)(138.76235382,668.75712036)
\curveto(138.7623474,668.72711715)(138.75734741,668.69211719)(138.74735382,668.65212036)
\curveto(138.71734745,668.54211734)(138.69234747,668.43711744)(138.67235382,668.33712036)
\curveto(138.65234751,668.23711764)(138.62234754,668.13711774)(138.58235382,668.03712036)
\curveto(138.47234769,667.78711809)(138.33734783,667.5771183)(138.17735382,667.40712036)
\curveto(138.01734815,667.23711864)(137.80734836,667.10211878)(137.54735382,667.00212036)
\curveto(137.47734869,666.97211891)(137.40234876,666.95211893)(137.32235382,666.94212036)
\curveto(137.24234892,666.93211895)(137.162349,666.91711896)(137.08235382,666.89712036)
\lineto(136.96235382,666.89712036)
\curveto(136.92234924,666.88711899)(136.87734929,666.882119)(136.82735382,666.88212036)
\lineto(136.70735382,666.91212036)
\curveto(136.6673495,666.92211896)(136.63234953,666.92211896)(136.60235382,666.91212036)
\curveto(136.57234959,666.91211897)(136.53734963,666.91711896)(136.49735382,666.92712036)
\curveto(136.40734976,666.94711893)(136.31734985,666.97211891)(136.22735382,667.00212036)
\curveto(136.14735002,667.03211885)(136.07235009,667.07211881)(136.00235382,667.12212036)
\curveto(135.75235041,667.27211861)(135.5673506,667.43711844)(135.44735382,667.61712036)
\curveto(135.33735083,667.80711807)(135.23235093,668.05211783)(135.13235382,668.35212036)
\curveto(135.11235105,668.43211745)(135.09735107,668.50711737)(135.08735382,668.57712036)
\curveto(135.07735109,668.65711722)(135.0623511,668.73711714)(135.04235382,668.81712036)
\lineto(135.04235382,668.95212036)
\curveto(135.02235114,669.02211686)(135.00735116,669.12711675)(134.99735382,669.26712036)
\curveto(134.99735117,669.40711647)(135.00735116,669.51211637)(135.02735382,669.58212036)
\lineto(135.02735382,669.73212036)
\curveto(135.02735114,669.7821161)(135.03235113,669.83211605)(135.04235382,669.88212036)
\curveto(135.0623511,669.99211589)(135.07735109,670.10211578)(135.08735382,670.21212036)
\curveto(135.10735106,670.32211556)(135.13235103,670.42711545)(135.16235382,670.52712036)
\curveto(135.25235091,670.79711508)(135.37235079,671.03211485)(135.52235382,671.23212036)
\curveto(135.68235048,671.44211444)(135.88735028,671.60211428)(136.13735382,671.71212036)
\curveto(136.18734998,671.74211414)(136.24234992,671.76211412)(136.30235382,671.77212036)
\lineto(136.51235382,671.83212036)
\curveto(136.54234962,671.84211404)(136.57734959,671.84211404)(136.61735382,671.83212036)
\curveto(136.65734951,671.83211405)(136.69234947,671.84211404)(136.72235382,671.86212036)
\lineto(136.99235382,671.86212036)
\curveto(137.08234908,671.87211401)(137.167349,671.86711401)(137.24735382,671.84712036)
\curveto(137.31734885,671.82711405)(137.38234878,671.80711407)(137.44235382,671.78712036)
\curveto(137.50234866,671.7771141)(137.5623486,671.76211412)(137.62235382,671.74212036)
\curveto(137.87234829,671.63211425)(138.07234809,671.4821144)(138.22235382,671.29212036)
\curveto(138.37234779,671.11211477)(138.50234766,670.89211499)(138.61235382,670.63212036)
\curveto(138.64234752,670.55211533)(138.6623475,670.46711541)(138.67235382,670.37712036)
\lineto(138.73235382,670.13712036)
\curveto(138.74234742,670.11711576)(138.74734742,670.08711579)(138.74735382,670.04712036)
\curveto(138.75734741,669.99711588)(138.7623474,669.94211594)(138.76235382,669.88212036)
\curveto(138.7623474,669.82211606)(138.77234739,669.76711611)(138.79235382,669.71712036)
\lineto(138.79235382,669.59712036)
\curveto(138.80234736,669.54711633)(138.80734736,669.47211641)(138.80735382,669.37212036)
\curveto(138.80734736,669.2821166)(138.80234736,669.21211667)(138.79235382,669.16212036)
\moveto(137.56235382,676.33212036)
\lineto(138.62735382,676.33212036)
\curveto(138.70734746,676.33210955)(138.80234736,676.33210955)(138.91235382,676.33212036)
\curveto(139.02234714,676.33210955)(139.10234706,676.31710956)(139.15235382,676.28712036)
\curveto(139.17234699,676.2771096)(139.18234698,676.26210962)(139.18235382,676.24212036)
\curveto(139.19234697,676.23210965)(139.20734696,676.22210966)(139.22735382,676.21212036)
\curveto(139.23734693,676.09210979)(139.18734698,675.98710989)(139.07735382,675.89712036)
\curveto(138.97734719,675.80711007)(138.89234727,675.72711015)(138.82235382,675.65712036)
\curveto(138.74234742,675.58711029)(138.6623475,675.51211037)(138.58235382,675.43212036)
\curveto(138.51234765,675.36211052)(138.43734773,675.29711058)(138.35735382,675.23712036)
\curveto(138.31734785,675.20711067)(138.28234788,675.17211071)(138.25235382,675.13212036)
\curveto(138.23234793,675.10211078)(138.20234796,675.0771108)(138.16235382,675.05712036)
\curveto(138.14234802,675.02711085)(138.11734805,675.00211088)(138.08735382,674.98212036)
\lineto(137.93735382,674.83212036)
\lineto(137.78735382,674.71212036)
\lineto(137.74235382,674.66712036)
\curveto(137.74234842,674.65711122)(137.73234843,674.64211124)(137.71235382,674.62212036)
\curveto(137.63234853,674.56211132)(137.55234861,674.49711138)(137.47235382,674.42712036)
\curveto(137.40234876,674.35711152)(137.31234885,674.30211158)(137.20235382,674.26212036)
\curveto(137.162349,674.25211163)(137.12234904,674.24711163)(137.08235382,674.24712036)
\curveto(137.05234911,674.24711163)(137.01234915,674.24211164)(136.96235382,674.23212036)
\curveto(136.93234923,674.22211166)(136.89234927,674.21711166)(136.84235382,674.21712036)
\curveto(136.79234937,674.22711165)(136.74734942,674.23211165)(136.70735382,674.23212036)
\lineto(136.36235382,674.23212036)
\curveto(136.24234992,674.23211165)(136.15235001,674.25711162)(136.09235382,674.30712036)
\curveto(136.03235013,674.34711153)(136.01735015,674.41711146)(136.04735382,674.51712036)
\curveto(136.0673501,674.59711128)(136.10235006,674.66711121)(136.15235382,674.72712036)
\curveto(136.20234996,674.79711108)(136.24734992,674.86711101)(136.28735382,674.93712036)
\curveto(136.38734978,675.0771108)(136.48234968,675.21211067)(136.57235382,675.34212036)
\curveto(136.6623495,675.47211041)(136.75234941,675.60711027)(136.84235382,675.74712036)
\curveto(136.89234927,675.82711005)(136.94234922,675.91210997)(136.99235382,676.00212036)
\curveto(137.05234911,676.09210979)(137.11734905,676.16210972)(137.18735382,676.21212036)
\curveto(137.22734894,676.24210964)(137.29734887,676.2771096)(137.39735382,676.31712036)
\curveto(137.41734875,676.32710955)(137.44234872,676.32710955)(137.47235382,676.31712036)
\curveto(137.51234865,676.31710956)(137.54234862,676.32210956)(137.56235382,676.33212036)
}
}
{
\newrgbcolor{curcolor}{0 0 0}
\pscustom[linestyle=none,fillstyle=solid,fillcolor=curcolor]
{
\newpath
\moveto(146.7172757,673.45212036)
\curveto(147.31726989,673.47211241)(147.81726939,673.38711249)(148.2172757,673.19712036)
\curveto(148.61726859,673.00711287)(148.93226828,672.72711315)(149.1622757,672.35712036)
\curveto(149.23226798,672.24711363)(149.28726792,672.12711375)(149.3272757,671.99712036)
\curveto(149.36726784,671.877114)(149.4072678,671.75211413)(149.4472757,671.62212036)
\curveto(149.46726774,671.54211434)(149.47726773,671.46711441)(149.4772757,671.39712036)
\curveto(149.48726772,671.32711455)(149.50226771,671.25711462)(149.5222757,671.18712036)
\curveto(149.52226769,671.12711475)(149.52726768,671.08711479)(149.5372757,671.06712036)
\curveto(149.55726765,670.92711495)(149.56726764,670.7821151)(149.5672757,670.63212036)
\lineto(149.5672757,670.19712036)
\lineto(149.5672757,668.86212036)
\lineto(149.5672757,666.43212036)
\curveto(149.56726764,666.24211964)(149.56226765,666.05711982)(149.5522757,665.87712036)
\curveto(149.55226766,665.70712017)(149.48226773,665.59712028)(149.3422757,665.54712036)
\curveto(149.28226793,665.52712035)(149.212268,665.51712036)(149.1322757,665.51712036)
\lineto(148.8922757,665.51712036)
\lineto(148.0822757,665.51712036)
\curveto(147.96226925,665.51712036)(147.85226936,665.52212036)(147.7522757,665.53212036)
\curveto(147.66226955,665.55212033)(147.59226962,665.59712028)(147.5422757,665.66712036)
\curveto(147.50226971,665.72712015)(147.47726973,665.80212008)(147.4672757,665.89212036)
\lineto(147.4672757,666.20712036)
\lineto(147.4672757,667.25712036)
\lineto(147.4672757,669.49212036)
\curveto(147.46726974,669.86211602)(147.45226976,670.20211568)(147.4222757,670.51212036)
\curveto(147.39226982,670.83211505)(147.30226991,671.10211478)(147.1522757,671.32212036)
\curveto(147.0122702,671.52211436)(146.8072704,671.66211422)(146.5372757,671.74212036)
\curveto(146.48727072,671.76211412)(146.43227078,671.77211411)(146.3722757,671.77212036)
\curveto(146.32227089,671.77211411)(146.26727094,671.7821141)(146.2072757,671.80212036)
\curveto(146.15727105,671.81211407)(146.09227112,671.81211407)(146.0122757,671.80212036)
\curveto(145.94227127,671.80211408)(145.88727132,671.79711408)(145.8472757,671.78712036)
\curveto(145.8072714,671.7771141)(145.77227144,671.77211411)(145.7422757,671.77212036)
\curveto(145.7122715,671.77211411)(145.68227153,671.76711411)(145.6522757,671.75712036)
\curveto(145.42227179,671.69711418)(145.23727197,671.61711426)(145.0972757,671.51712036)
\curveto(144.77727243,671.28711459)(144.58727262,670.95211493)(144.5272757,670.51212036)
\curveto(144.46727274,670.07211581)(144.43727277,669.5771163)(144.4372757,669.02712036)
\lineto(144.4372757,667.15212036)
\lineto(144.4372757,666.23712036)
\lineto(144.4372757,665.96712036)
\curveto(144.43727277,665.87712)(144.42227279,665.80212008)(144.3922757,665.74212036)
\curveto(144.34227287,665.63212025)(144.26227295,665.56712031)(144.1522757,665.54712036)
\curveto(144.04227317,665.52712035)(143.9072733,665.51712036)(143.7472757,665.51712036)
\lineto(142.9972757,665.51712036)
\curveto(142.88727432,665.51712036)(142.77727443,665.52212036)(142.6672757,665.53212036)
\curveto(142.55727465,665.54212034)(142.47727473,665.5771203)(142.4272757,665.63712036)
\curveto(142.35727485,665.72712015)(142.32227489,665.85712002)(142.3222757,666.02712036)
\curveto(142.33227488,666.19711968)(142.33727487,666.35711952)(142.3372757,666.50712036)
\lineto(142.3372757,668.54712036)
\lineto(142.3372757,671.84712036)
\lineto(142.3372757,672.61212036)
\lineto(142.3372757,672.91212036)
\curveto(142.34727486,673.00211288)(142.37727483,673.0771128)(142.4272757,673.13712036)
\curveto(142.44727476,673.16711271)(142.47727473,673.18711269)(142.5172757,673.19712036)
\curveto(142.56727464,673.21711266)(142.61727459,673.23211265)(142.6672757,673.24212036)
\lineto(142.7422757,673.24212036)
\curveto(142.79227442,673.25211263)(142.84227437,673.25711262)(142.8922757,673.25712036)
\lineto(143.0572757,673.25712036)
\lineto(143.6872757,673.25712036)
\curveto(143.76727344,673.25711262)(143.84227337,673.25211263)(143.9122757,673.24212036)
\curveto(143.99227322,673.24211264)(144.06227315,673.23211265)(144.1222757,673.21212036)
\curveto(144.19227302,673.1821127)(144.23727297,673.13711274)(144.2572757,673.07712036)
\curveto(144.28727292,673.01711286)(144.3122729,672.94711293)(144.3322757,672.86712036)
\curveto(144.34227287,672.82711305)(144.34227287,672.79211309)(144.3322757,672.76212036)
\curveto(144.33227288,672.73211315)(144.34227287,672.70211318)(144.3622757,672.67212036)
\curveto(144.38227283,672.62211326)(144.39727281,672.59211329)(144.4072757,672.58212036)
\curveto(144.42727278,672.57211331)(144.45227276,672.55711332)(144.4822757,672.53712036)
\curveto(144.59227262,672.52711335)(144.68227253,672.56211332)(144.7522757,672.64212036)
\curveto(144.82227239,672.73211315)(144.89727231,672.80211308)(144.9772757,672.85212036)
\curveto(145.24727196,673.05211283)(145.54727166,673.21211267)(145.8772757,673.33212036)
\curveto(145.96727124,673.36211252)(146.05727115,673.3821125)(146.1472757,673.39212036)
\curveto(146.24727096,673.40211248)(146.35227086,673.41711246)(146.4622757,673.43712036)
\curveto(146.49227072,673.44711243)(146.53727067,673.44711243)(146.5972757,673.43712036)
\curveto(146.65727055,673.43711244)(146.69727051,673.44211244)(146.7172757,673.45212036)
}
}
{
\newrgbcolor{curcolor}{0 0 0}
\pscustom[linestyle=none,fillstyle=solid,fillcolor=curcolor]
{
\newpath
\moveto(153.2985257,676.10712036)
\curveto(153.36852275,676.02710985)(153.40352271,675.90710997)(153.4035257,675.74712036)
\lineto(153.4035257,675.28212036)
\lineto(153.4035257,674.87712036)
\curveto(153.40352271,674.73711114)(153.36852275,674.64211124)(153.2985257,674.59212036)
\curveto(153.23852288,674.54211134)(153.15852296,674.51211137)(153.0585257,674.50212036)
\curveto(152.96852315,674.49211139)(152.86852325,674.48711139)(152.7585257,674.48712036)
\lineto(151.9185257,674.48712036)
\curveto(151.80852431,674.48711139)(151.70852441,674.49211139)(151.6185257,674.50212036)
\curveto(151.53852458,674.51211137)(151.46852465,674.54211134)(151.4085257,674.59212036)
\curveto(151.36852475,674.62211126)(151.33852478,674.6771112)(151.3185257,674.75712036)
\curveto(151.30852481,674.84711103)(151.29852482,674.94211094)(151.2885257,675.04212036)
\lineto(151.2885257,675.37212036)
\curveto(151.29852482,675.4821104)(151.30352481,675.5771103)(151.3035257,675.65712036)
\lineto(151.3035257,675.86712036)
\curveto(151.3135248,675.93710994)(151.33352478,675.99710988)(151.3635257,676.04712036)
\curveto(151.38352473,676.08710979)(151.40852471,676.11710976)(151.4385257,676.13712036)
\lineto(151.5585257,676.19712036)
\curveto(151.57852454,676.19710968)(151.60352451,676.19710968)(151.6335257,676.19712036)
\curveto(151.66352445,676.20710967)(151.68852443,676.21210967)(151.7085257,676.21212036)
\lineto(152.8035257,676.21212036)
\curveto(152.90352321,676.21210967)(152.99852312,676.20710967)(153.0885257,676.19712036)
\curveto(153.17852294,676.18710969)(153.24852287,676.15710972)(153.2985257,676.10712036)
\moveto(153.4035257,666.34212036)
\curveto(153.40352271,666.14211974)(153.39852272,665.97211991)(153.3885257,665.83212036)
\curveto(153.37852274,665.69212019)(153.28852283,665.59712028)(153.1185257,665.54712036)
\curveto(153.05852306,665.52712035)(152.99352312,665.51712036)(152.9235257,665.51712036)
\curveto(152.85352326,665.52712035)(152.77852334,665.53212035)(152.6985257,665.53212036)
\lineto(151.8585257,665.53212036)
\curveto(151.76852435,665.53212035)(151.67852444,665.53712034)(151.5885257,665.54712036)
\curveto(151.50852461,665.55712032)(151.44852467,665.58712029)(151.4085257,665.63712036)
\curveto(151.34852477,665.70712017)(151.3135248,665.79212009)(151.3035257,665.89212036)
\lineto(151.3035257,666.23712036)
\lineto(151.3035257,672.56712036)
\lineto(151.3035257,672.86712036)
\curveto(151.30352481,672.96711291)(151.32352479,673.04711283)(151.3635257,673.10712036)
\curveto(151.42352469,673.1771127)(151.50852461,673.22211266)(151.6185257,673.24212036)
\curveto(151.63852448,673.25211263)(151.66352445,673.25211263)(151.6935257,673.24212036)
\curveto(151.73352438,673.24211264)(151.76352435,673.24711263)(151.7835257,673.25712036)
\lineto(152.5335257,673.25712036)
\lineto(152.7285257,673.25712036)
\curveto(152.80852331,673.26711261)(152.87352324,673.26711261)(152.9235257,673.25712036)
\lineto(153.0435257,673.25712036)
\curveto(153.10352301,673.23711264)(153.15852296,673.22211266)(153.2085257,673.21212036)
\curveto(153.25852286,673.20211268)(153.29852282,673.17211271)(153.3285257,673.12212036)
\curveto(153.36852275,673.07211281)(153.38852273,673.00211288)(153.3885257,672.91212036)
\curveto(153.39852272,672.82211306)(153.40352271,672.72711315)(153.4035257,672.62712036)
\lineto(153.4035257,666.34212036)
}
}
{
\newrgbcolor{curcolor}{0 0 0}
\pscustom[linestyle=none,fillstyle=solid,fillcolor=curcolor]
{
\newpath
\moveto(158.6357132,673.46712036)
\curveto(159.44570804,673.48711239)(160.12070736,673.36711251)(160.6607132,673.10712036)
\curveto(161.21070627,672.84711303)(161.64570584,672.4771134)(161.9657132,671.99712036)
\curveto(162.12570536,671.75711412)(162.24570524,671.4821144)(162.3257132,671.17212036)
\curveto(162.34570514,671.12211476)(162.36070512,671.05711482)(162.3707132,670.97712036)
\curveto(162.39070509,670.89711498)(162.39070509,670.82711505)(162.3707132,670.76712036)
\curveto(162.33070515,670.65711522)(162.26070522,670.59211529)(162.1607132,670.57212036)
\curveto(162.06070542,670.56211532)(161.94070554,670.55711532)(161.8007132,670.55712036)
\lineto(161.0207132,670.55712036)
\lineto(160.7357132,670.55712036)
\curveto(160.64570684,670.55711532)(160.57070691,670.5771153)(160.5107132,670.61712036)
\curveto(160.43070705,670.65711522)(160.37570711,670.71711516)(160.3457132,670.79712036)
\curveto(160.31570717,670.88711499)(160.27570721,670.9771149)(160.2257132,671.06712036)
\curveto(160.16570732,671.1771147)(160.10070738,671.2771146)(160.0307132,671.36712036)
\curveto(159.96070752,671.45711442)(159.8807076,671.53711434)(159.7907132,671.60712036)
\curveto(159.65070783,671.69711418)(159.49570799,671.76711411)(159.3257132,671.81712036)
\curveto(159.26570822,671.83711404)(159.20570828,671.84711403)(159.1457132,671.84712036)
\curveto(159.0857084,671.84711403)(159.03070845,671.85711402)(158.9807132,671.87712036)
\lineto(158.8307132,671.87712036)
\curveto(158.63070885,671.877114)(158.47070901,671.85711402)(158.3507132,671.81712036)
\curveto(158.06070942,671.72711415)(157.82570966,671.58711429)(157.6457132,671.39712036)
\curveto(157.46571002,671.21711466)(157.32071016,670.99711488)(157.2107132,670.73712036)
\curveto(157.16071032,670.62711525)(157.12071036,670.50711537)(157.0907132,670.37712036)
\curveto(157.07071041,670.25711562)(157.04571044,670.12711575)(157.0157132,669.98712036)
\curveto(157.00571048,669.94711593)(157.00071048,669.90711597)(157.0007132,669.86712036)
\curveto(157.00071048,669.82711605)(156.99571049,669.78711609)(156.9857132,669.74712036)
\curveto(156.96571052,669.64711623)(156.95571053,669.50711637)(156.9557132,669.32712036)
\curveto(156.96571052,669.14711673)(156.9807105,669.00711687)(157.0007132,668.90712036)
\curveto(157.00071048,668.82711705)(157.00571048,668.77211711)(157.0157132,668.74212036)
\curveto(157.03571045,668.67211721)(157.04571044,668.60211728)(157.0457132,668.53212036)
\curveto(157.05571043,668.46211742)(157.07071041,668.39211749)(157.0907132,668.32212036)
\curveto(157.17071031,668.09211779)(157.26571022,667.882118)(157.3757132,667.69212036)
\curveto(157.48571,667.50211838)(157.62570986,667.34211854)(157.7957132,667.21212036)
\curveto(157.83570965,667.1821187)(157.89570959,667.14711873)(157.9757132,667.10712036)
\curveto(158.0857094,667.03711884)(158.19570929,666.99211889)(158.3057132,666.97212036)
\curveto(158.42570906,666.95211893)(158.57070891,666.93211895)(158.7407132,666.91212036)
\lineto(158.8307132,666.91212036)
\curveto(158.87070861,666.91211897)(158.90070858,666.91711896)(158.9207132,666.92712036)
\lineto(159.0557132,666.92712036)
\curveto(159.12570836,666.94711893)(159.19070829,666.96211892)(159.2507132,666.97212036)
\curveto(159.32070816,666.99211889)(159.3857081,667.01211887)(159.4457132,667.03212036)
\curveto(159.74570774,667.16211872)(159.97570751,667.35211853)(160.1357132,667.60212036)
\curveto(160.17570731,667.65211823)(160.21070727,667.70711817)(160.2407132,667.76712036)
\curveto(160.27070721,667.83711804)(160.29570719,667.89711798)(160.3157132,667.94712036)
\curveto(160.35570713,668.05711782)(160.39070709,668.15211773)(160.4207132,668.23212036)
\curveto(160.45070703,668.32211756)(160.52070696,668.39211749)(160.6307132,668.44212036)
\curveto(160.72070676,668.4821174)(160.86570662,668.49711738)(161.0657132,668.48712036)
\lineto(161.5607132,668.48712036)
\lineto(161.7707132,668.48712036)
\curveto(161.85070563,668.49711738)(161.91570557,668.49211739)(161.9657132,668.47212036)
\lineto(162.0857132,668.47212036)
\lineto(162.2057132,668.44212036)
\curveto(162.24570524,668.44211744)(162.27570521,668.43211745)(162.2957132,668.41212036)
\curveto(162.34570514,668.37211751)(162.37570511,668.31211757)(162.3857132,668.23212036)
\curveto(162.40570508,668.16211772)(162.40570508,668.08711779)(162.3857132,668.00712036)
\curveto(162.29570519,667.6771182)(162.1857053,667.3821185)(162.0557132,667.12212036)
\curveto(161.64570584,666.35211953)(160.99070649,665.81712006)(160.0907132,665.51712036)
\curveto(159.99070749,665.48712039)(159.8857076,665.46712041)(159.7757132,665.45712036)
\curveto(159.66570782,665.43712044)(159.55570793,665.41212047)(159.4457132,665.38212036)
\curveto(159.3857081,665.37212051)(159.32570816,665.36712051)(159.2657132,665.36712036)
\curveto(159.20570828,665.36712051)(159.14570834,665.36212052)(159.0857132,665.35212036)
\lineto(158.9207132,665.35212036)
\curveto(158.87070861,665.33212055)(158.79570869,665.32712055)(158.6957132,665.33712036)
\curveto(158.59570889,665.33712054)(158.52070896,665.34212054)(158.4707132,665.35212036)
\curveto(158.39070909,665.37212051)(158.31570917,665.3821205)(158.2457132,665.38212036)
\curveto(158.1857093,665.37212051)(158.12070936,665.3771205)(158.0507132,665.39712036)
\lineto(157.9007132,665.42712036)
\curveto(157.85070963,665.42712045)(157.80070968,665.43212045)(157.7507132,665.44212036)
\curveto(157.64070984,665.47212041)(157.53570995,665.50212038)(157.4357132,665.53212036)
\curveto(157.33571015,665.56212032)(157.24071024,665.59712028)(157.1507132,665.63712036)
\curveto(156.6807108,665.83712004)(156.2857112,666.09211979)(155.9657132,666.40212036)
\curveto(155.64571184,666.72211916)(155.3857121,667.11711876)(155.1857132,667.58712036)
\curveto(155.13571235,667.6771182)(155.09571239,667.77211811)(155.0657132,667.87212036)
\lineto(154.9757132,668.20212036)
\curveto(154.96571252,668.24211764)(154.96071252,668.2771176)(154.9607132,668.30712036)
\curveto(154.96071252,668.34711753)(154.95071253,668.39211749)(154.9307132,668.44212036)
\curveto(154.91071257,668.51211737)(154.90071258,668.5821173)(154.9007132,668.65212036)
\curveto(154.90071258,668.73211715)(154.89071259,668.80711707)(154.8707132,668.87712036)
\lineto(154.8707132,669.13212036)
\curveto(154.85071263,669.1821167)(154.84071264,669.23711664)(154.8407132,669.29712036)
\curveto(154.84071264,669.36711651)(154.85071263,669.42711645)(154.8707132,669.47712036)
\curveto(154.8807126,669.52711635)(154.8807126,669.57211631)(154.8707132,669.61212036)
\curveto(154.86071262,669.65211623)(154.86071262,669.69211619)(154.8707132,669.73212036)
\curveto(154.89071259,669.80211608)(154.89571259,669.86711601)(154.8857132,669.92712036)
\curveto(154.8857126,669.98711589)(154.89571259,670.04711583)(154.9157132,670.10712036)
\curveto(154.96571252,670.28711559)(155.00571248,670.45711542)(155.0357132,670.61712036)
\curveto(155.06571242,670.78711509)(155.11071237,670.95211493)(155.1707132,671.11212036)
\curveto(155.39071209,671.62211426)(155.66571182,672.04711383)(155.9957132,672.38712036)
\curveto(156.33571115,672.72711315)(156.76571072,673.00211288)(157.2857132,673.21212036)
\curveto(157.42571006,673.27211261)(157.57070991,673.31211257)(157.7207132,673.33212036)
\curveto(157.87070961,673.36211252)(158.02570946,673.39711248)(158.1857132,673.43712036)
\curveto(158.26570922,673.44711243)(158.34070914,673.45211243)(158.4107132,673.45212036)
\curveto(158.480709,673.45211243)(158.55570893,673.45711242)(158.6357132,673.46712036)
}
}
{
\newrgbcolor{curcolor}{0 0 0}
\pscustom[linestyle=none,fillstyle=solid,fillcolor=curcolor]
{
\newpath
\moveto(171.44899445,669.70212036)
\curveto(171.46898588,669.64211624)(171.47898587,669.55711632)(171.47899445,669.44712036)
\curveto(171.47898587,669.33711654)(171.46898588,669.25211663)(171.44899445,669.19212036)
\lineto(171.44899445,669.04212036)
\curveto(171.42898592,668.96211692)(171.41898593,668.882117)(171.41899445,668.80212036)
\curveto(171.42898592,668.72211716)(171.42398592,668.64211724)(171.40399445,668.56212036)
\curveto(171.38398596,668.49211739)(171.36898598,668.42711745)(171.35899445,668.36712036)
\curveto(171.348986,668.30711757)(171.33898601,668.24211764)(171.32899445,668.17212036)
\curveto(171.28898606,668.06211782)(171.25398609,667.94711793)(171.22399445,667.82712036)
\curveto(171.19398615,667.71711816)(171.15398619,667.61211827)(171.10399445,667.51212036)
\curveto(170.89398645,667.03211885)(170.61898673,666.64211924)(170.27899445,666.34212036)
\curveto(169.93898741,666.04211984)(169.52898782,665.79212009)(169.04899445,665.59212036)
\curveto(168.92898842,665.54212034)(168.80398854,665.50712037)(168.67399445,665.48712036)
\curveto(168.55398879,665.45712042)(168.42898892,665.42712045)(168.29899445,665.39712036)
\curveto(168.2489891,665.3771205)(168.19398915,665.36712051)(168.13399445,665.36712036)
\curveto(168.07398927,665.36712051)(168.01898933,665.36212052)(167.96899445,665.35212036)
\lineto(167.86399445,665.35212036)
\curveto(167.83398951,665.34212054)(167.80398954,665.33712054)(167.77399445,665.33712036)
\curveto(167.72398962,665.32712055)(167.6439897,665.32212056)(167.53399445,665.32212036)
\curveto(167.42398992,665.31212057)(167.33899001,665.31712056)(167.27899445,665.33712036)
\lineto(167.12899445,665.33712036)
\curveto(167.07899027,665.34712053)(167.02399032,665.35212053)(166.96399445,665.35212036)
\curveto(166.91399043,665.34212054)(166.86399048,665.34712053)(166.81399445,665.36712036)
\curveto(166.77399057,665.3771205)(166.73399061,665.3821205)(166.69399445,665.38212036)
\curveto(166.66399068,665.3821205)(166.62399072,665.38712049)(166.57399445,665.39712036)
\curveto(166.47399087,665.42712045)(166.37399097,665.45212043)(166.27399445,665.47212036)
\curveto(166.17399117,665.49212039)(166.07899127,665.52212036)(165.98899445,665.56212036)
\curveto(165.86899148,665.60212028)(165.75399159,665.64212024)(165.64399445,665.68212036)
\curveto(165.5439918,665.72212016)(165.43899191,665.77212011)(165.32899445,665.83212036)
\curveto(164.97899237,666.04211984)(164.67899267,666.28711959)(164.42899445,666.56712036)
\curveto(164.17899317,666.84711903)(163.96899338,667.1821187)(163.79899445,667.57212036)
\curveto(163.7489936,667.66211822)(163.70899364,667.75711812)(163.67899445,667.85712036)
\curveto(163.65899369,667.95711792)(163.63399371,668.06211782)(163.60399445,668.17212036)
\curveto(163.58399376,668.22211766)(163.57399377,668.26711761)(163.57399445,668.30712036)
\curveto(163.57399377,668.34711753)(163.56399378,668.39211749)(163.54399445,668.44212036)
\curveto(163.52399382,668.52211736)(163.51399383,668.60211728)(163.51399445,668.68212036)
\curveto(163.51399383,668.77211711)(163.50399384,668.85711702)(163.48399445,668.93712036)
\curveto(163.47399387,668.98711689)(163.46899388,669.03211685)(163.46899445,669.07212036)
\lineto(163.46899445,669.20712036)
\curveto(163.4489939,669.26711661)(163.43899391,669.35211653)(163.43899445,669.46212036)
\curveto(163.4489939,669.57211631)(163.46399388,669.65711622)(163.48399445,669.71712036)
\lineto(163.48399445,669.82212036)
\curveto(163.49399385,669.87211601)(163.49399385,669.92211596)(163.48399445,669.97212036)
\curveto(163.48399386,670.03211585)(163.49399385,670.08711579)(163.51399445,670.13712036)
\curveto(163.52399382,670.18711569)(163.52899382,670.23211565)(163.52899445,670.27212036)
\curveto(163.52899382,670.32211556)(163.53899381,670.37211551)(163.55899445,670.42212036)
\curveto(163.59899375,670.55211533)(163.63399371,670.6771152)(163.66399445,670.79712036)
\curveto(163.69399365,670.92711495)(163.73399361,671.05211483)(163.78399445,671.17212036)
\curveto(163.96399338,671.5821143)(164.17899317,671.92211396)(164.42899445,672.19212036)
\curveto(164.67899267,672.47211341)(164.98399236,672.72711315)(165.34399445,672.95712036)
\curveto(165.4439919,673.00711287)(165.5489918,673.05211283)(165.65899445,673.09212036)
\curveto(165.76899158,673.13211275)(165.87899147,673.1771127)(165.98899445,673.22712036)
\curveto(166.11899123,673.2771126)(166.25399109,673.31211257)(166.39399445,673.33212036)
\curveto(166.53399081,673.35211253)(166.67899067,673.3821125)(166.82899445,673.42212036)
\curveto(166.90899044,673.43211245)(166.98399036,673.43711244)(167.05399445,673.43712036)
\curveto(167.12399022,673.43711244)(167.19399015,673.44211244)(167.26399445,673.45212036)
\curveto(167.8439895,673.46211242)(168.343989,673.40211248)(168.76399445,673.27212036)
\curveto(169.19398815,673.14211274)(169.57398777,672.96211292)(169.90399445,672.73212036)
\curveto(170.01398733,672.65211323)(170.12398722,672.56211332)(170.23399445,672.46212036)
\curveto(170.35398699,672.37211351)(170.45398689,672.27211361)(170.53399445,672.16212036)
\curveto(170.61398673,672.06211382)(170.68398666,671.96211392)(170.74399445,671.86212036)
\curveto(170.81398653,671.76211412)(170.88398646,671.65711422)(170.95399445,671.54712036)
\curveto(171.02398632,671.43711444)(171.07898627,671.31711456)(171.11899445,671.18712036)
\curveto(171.15898619,671.06711481)(171.20398614,670.93711494)(171.25399445,670.79712036)
\curveto(171.28398606,670.71711516)(171.30898604,670.63211525)(171.32899445,670.54212036)
\lineto(171.38899445,670.27212036)
\curveto(171.39898595,670.23211565)(171.40398594,670.19211569)(171.40399445,670.15212036)
\curveto(171.40398594,670.11211577)(171.40898594,670.07211581)(171.41899445,670.03212036)
\curveto(171.43898591,669.9821159)(171.4439859,669.92711595)(171.43399445,669.86712036)
\curveto(171.42398592,669.80711607)(171.42898592,669.75211613)(171.44899445,669.70212036)
\moveto(169.34899445,669.16212036)
\curveto(169.35898799,669.21211667)(169.36398798,669.2821166)(169.36399445,669.37212036)
\curveto(169.36398798,669.47211641)(169.35898799,669.54711633)(169.34899445,669.59712036)
\lineto(169.34899445,669.71712036)
\curveto(169.32898802,669.76711611)(169.31898803,669.82211606)(169.31899445,669.88212036)
\curveto(169.31898803,669.94211594)(169.31398803,669.99711588)(169.30399445,670.04712036)
\curveto(169.30398804,670.08711579)(169.29898805,670.11711576)(169.28899445,670.13712036)
\lineto(169.22899445,670.37712036)
\curveto(169.21898813,670.46711541)(169.19898815,670.55211533)(169.16899445,670.63212036)
\curveto(169.05898829,670.89211499)(168.92898842,671.11211477)(168.77899445,671.29212036)
\curveto(168.62898872,671.4821144)(168.42898892,671.63211425)(168.17899445,671.74212036)
\curveto(168.11898923,671.76211412)(168.05898929,671.7771141)(167.99899445,671.78712036)
\curveto(167.93898941,671.80711407)(167.87398947,671.82711405)(167.80399445,671.84712036)
\curveto(167.72398962,671.86711401)(167.63898971,671.87211401)(167.54899445,671.86212036)
\lineto(167.27899445,671.86212036)
\curveto(167.2489901,671.84211404)(167.21399013,671.83211405)(167.17399445,671.83212036)
\curveto(167.13399021,671.84211404)(167.09899025,671.84211404)(167.06899445,671.83212036)
\lineto(166.85899445,671.77212036)
\curveto(166.79899055,671.76211412)(166.7439906,671.74211414)(166.69399445,671.71212036)
\curveto(166.4439909,671.60211428)(166.23899111,671.44211444)(166.07899445,671.23212036)
\curveto(165.92899142,671.03211485)(165.80899154,670.79711508)(165.71899445,670.52712036)
\curveto(165.68899166,670.42711545)(165.66399168,670.32211556)(165.64399445,670.21212036)
\curveto(165.63399171,670.10211578)(165.61899173,669.99211589)(165.59899445,669.88212036)
\curveto(165.58899176,669.83211605)(165.58399176,669.7821161)(165.58399445,669.73212036)
\lineto(165.58399445,669.58212036)
\curveto(165.56399178,669.51211637)(165.55399179,669.40711647)(165.55399445,669.26712036)
\curveto(165.56399178,669.12711675)(165.57899177,669.02211686)(165.59899445,668.95212036)
\lineto(165.59899445,668.81712036)
\curveto(165.61899173,668.73711714)(165.63399171,668.65711722)(165.64399445,668.57712036)
\curveto(165.65399169,668.50711737)(165.66899168,668.43211745)(165.68899445,668.35212036)
\curveto(165.78899156,668.05211783)(165.89399145,667.80711807)(166.00399445,667.61712036)
\curveto(166.12399122,667.43711844)(166.30899104,667.27211861)(166.55899445,667.12212036)
\curveto(166.62899072,667.07211881)(166.70399064,667.03211885)(166.78399445,667.00212036)
\curveto(166.87399047,666.97211891)(166.96399038,666.94711893)(167.05399445,666.92712036)
\curveto(167.09399025,666.91711896)(167.12899022,666.91211897)(167.15899445,666.91212036)
\curveto(167.18899016,666.92211896)(167.22399012,666.92211896)(167.26399445,666.91212036)
\lineto(167.38399445,666.88212036)
\curveto(167.43398991,666.882119)(167.47898987,666.88711899)(167.51899445,666.89712036)
\lineto(167.63899445,666.89712036)
\curveto(167.71898963,666.91711896)(167.79898955,666.93211895)(167.87899445,666.94212036)
\curveto(167.95898939,666.95211893)(168.03398931,666.97211891)(168.10399445,667.00212036)
\curveto(168.36398898,667.10211878)(168.57398877,667.23711864)(168.73399445,667.40712036)
\curveto(168.89398845,667.5771183)(169.02898832,667.78711809)(169.13899445,668.03712036)
\curveto(169.17898817,668.13711774)(169.20898814,668.23711764)(169.22899445,668.33712036)
\curveto(169.2489881,668.43711744)(169.27398807,668.54211734)(169.30399445,668.65212036)
\curveto(169.31398803,668.69211719)(169.31898803,668.72711715)(169.31899445,668.75712036)
\curveto(169.31898803,668.79711708)(169.32398802,668.83711704)(169.33399445,668.87712036)
\lineto(169.33399445,669.01212036)
\curveto(169.33398801,669.06211682)(169.33898801,669.11211677)(169.34899445,669.16212036)
}
}
{
\newrgbcolor{curcolor}{0 0 0}
\pscustom[linestyle=none,fillstyle=solid,fillcolor=curcolor]
{
\newpath
\moveto(387.77836182,694.26267334)
\lineto(389.05336182,694.26267334)
\curveto(389.16335904,694.26266263)(389.26835893,694.25766263)(389.36836182,694.24767334)
\curveto(389.47835872,694.23766265)(389.55835864,694.20266269)(389.60836182,694.14267334)
\curveto(389.65835854,694.06266283)(389.68335852,693.95766293)(389.68336182,693.82767334)
\curveto(389.69335851,693.70766318)(389.6983585,693.58266331)(389.69836182,693.45267334)
\lineto(389.69836182,691.93767334)
\lineto(389.69836182,688.84767334)
\lineto(389.69836182,688.32267334)
\curveto(389.6983585,688.28266861)(389.69335851,688.23766865)(389.68336182,688.18767334)
\curveto(389.68335852,688.14766874)(389.68835851,688.10766878)(389.69836182,688.06767334)
\lineto(389.69836182,687.82767334)
\curveto(389.6983585,687.73766915)(389.69335851,687.64266925)(389.68336182,687.54267334)
\curveto(389.68335852,687.44266945)(389.69335851,687.35266954)(389.71336182,687.27267334)
\curveto(389.71335849,687.20266969)(389.71835848,687.14766974)(389.72836182,687.10767334)
\curveto(389.74835845,686.99766989)(389.76335844,686.88767)(389.77336182,686.77767334)
\curveto(389.79335841,686.66767022)(389.82335838,686.55767033)(389.86336182,686.44767334)
\curveto(389.97335823,686.1876707)(390.11335809,685.97267092)(390.28336182,685.80267334)
\curveto(390.46335774,685.63267126)(390.6983575,685.49767139)(390.98836182,685.39767334)
\curveto(391.06835713,685.37767151)(391.14835705,685.36267153)(391.22836182,685.35267334)
\curveto(391.30835689,685.34267155)(391.38835681,685.32767156)(391.46836182,685.30767334)
\curveto(391.51835668,685.2876716)(391.56335664,685.27767161)(391.60336182,685.27767334)
\curveto(391.64335656,685.2876716)(391.68835651,685.2876716)(391.73836182,685.27767334)
\curveto(391.77835642,685.26767162)(391.84335636,685.26267163)(391.93336182,685.26267334)
\curveto(392.02335618,685.27267162)(392.08335612,685.28267161)(392.11336182,685.29267334)
\lineto(392.33836182,685.29267334)
\curveto(392.41835578,685.31267158)(392.4983557,685.32767156)(392.57836182,685.33767334)
\curveto(392.65835554,685.34767154)(392.73335547,685.36267153)(392.80336182,685.38267334)
\curveto(392.94335526,685.41267148)(393.05335515,685.44767144)(393.13336182,685.48767334)
\curveto(393.31335489,685.56767132)(393.46835473,685.67267122)(393.59836182,685.80267334)
\curveto(393.73835446,685.94267095)(393.84835435,686.09767079)(393.92836182,686.26767334)
\curveto(394.03835416,686.52767036)(394.1033541,686.83267006)(394.12336182,687.18267334)
\curveto(394.14335406,687.54266935)(394.15335405,687.91266898)(394.15336182,688.29267334)
\lineto(394.15336182,691.27767334)
\lineto(394.15336182,693.28767334)
\curveto(394.15335405,693.42766346)(394.14835405,693.58266331)(394.13836182,693.75267334)
\curveto(394.13835406,693.92266297)(394.16835403,694.04766284)(394.22836182,694.12767334)
\curveto(394.27835392,694.1876627)(394.34835385,694.22266267)(394.43836182,694.23267334)
\curveto(394.52835367,694.25266264)(394.62835357,694.26266263)(394.73836182,694.26267334)
\lineto(395.69836182,694.26267334)
\curveto(395.77835242,694.26266263)(395.85335235,694.26266263)(395.92336182,694.26267334)
\curveto(396.0033522,694.27266262)(396.07835212,694.26766262)(396.14836182,694.24767334)
\curveto(396.28835191,694.21766267)(396.37835182,694.16766272)(396.41836182,694.09767334)
\curveto(396.46835173,694.01766287)(396.48835171,693.90266299)(396.47836182,693.75267334)
\curveto(396.47835172,693.61266328)(396.47835172,693.48266341)(396.47836182,693.36267334)
\lineto(396.47836182,691.35267334)
\lineto(396.47836182,688.32267334)
\curveto(396.47835172,687.94266895)(396.47335173,687.57266932)(396.46336182,687.21267334)
\curveto(396.45335175,686.85267004)(396.40835179,686.52767036)(396.32836182,686.23767334)
\curveto(396.18835201,685.76767112)(396.00835219,685.35767153)(395.78836182,685.00767334)
\curveto(395.57835262,684.66767222)(395.2983529,684.37767251)(394.94836182,684.13767334)
\curveto(394.63835356,683.91767297)(394.27335393,683.73767315)(393.85336182,683.59767334)
\curveto(393.76335444,683.56767332)(393.66835453,683.54267335)(393.56836182,683.52267334)
\lineto(393.29836182,683.46267334)
\curveto(393.23835496,683.44267345)(393.17835502,683.43267346)(393.11836182,683.43267334)
\curveto(393.06835513,683.43267346)(393.01335519,683.42267347)(392.95336182,683.40267334)
\curveto(392.83335537,683.38267351)(392.6983555,683.36767352)(392.54836182,683.35767334)
\curveto(392.3983558,683.34767354)(392.25335595,683.34267355)(392.11336182,683.34267334)
\curveto(391.16335704,683.33267356)(390.35335785,683.44767344)(389.68336182,683.68767334)
\curveto(389.01335919,683.93767295)(388.48835971,684.33767255)(388.10836182,684.88767334)
\curveto(387.97836022,685.06767182)(387.86836033,685.25267164)(387.77836182,685.44267334)
\curveto(387.6983605,685.64267125)(387.62336058,685.85767103)(387.55336182,686.08767334)
\curveto(387.53336067,686.13767075)(387.52336068,686.17767071)(387.52336182,686.20767334)
\curveto(387.52336068,686.24767064)(387.51336069,686.2926706)(387.49336182,686.34267334)
\curveto(387.41336079,686.62267027)(387.37336083,686.93766995)(387.37336182,687.28767334)
\lineto(387.37336182,688.33767334)
\lineto(387.37336182,692.52267334)
\lineto(387.37336182,693.57267334)
\lineto(387.37336182,693.85767334)
\curveto(387.37336083,693.95766293)(387.38836081,694.03766285)(387.41836182,694.09767334)
\curveto(387.47836072,694.16766272)(387.55836064,694.21766267)(387.65836182,694.24767334)
\curveto(387.67836052,694.24766264)(387.6983605,694.24766264)(387.71836182,694.24767334)
\curveto(387.73836046,694.24766264)(387.75836044,694.25266264)(387.77836182,694.26267334)
}
}
{
\newrgbcolor{curcolor}{0 0 0}
\pscustom[linestyle=none,fillstyle=solid,fillcolor=curcolor]
{
\newpath
\moveto(401.23687744,691.50267334)
\curveto(401.98687294,691.52266537)(402.63687229,691.43766545)(403.18687744,691.24767334)
\curveto(403.74687118,691.06766582)(404.17187076,690.75266614)(404.46187744,690.30267334)
\curveto(404.5318704,690.1926667)(404.59187034,690.07766681)(404.64187744,689.95767334)
\curveto(404.70187023,689.84766704)(404.75187018,689.72266717)(404.79187744,689.58267334)
\curveto(404.81187012,689.52266737)(404.82187011,689.45766743)(404.82187744,689.38767334)
\curveto(404.82187011,689.31766757)(404.81187012,689.25766763)(404.79187744,689.20767334)
\curveto(404.75187018,689.14766774)(404.69687023,689.10766778)(404.62687744,689.08767334)
\curveto(404.57687035,689.06766782)(404.51687041,689.05766783)(404.44687744,689.05767334)
\lineto(404.23687744,689.05767334)
\lineto(403.57687744,689.05767334)
\curveto(403.50687142,689.05766783)(403.43687149,689.05266784)(403.36687744,689.04267334)
\curveto(403.29687163,689.04266785)(403.2318717,689.05266784)(403.17187744,689.07267334)
\curveto(403.07187186,689.0926678)(402.99687193,689.13266776)(402.94687744,689.19267334)
\curveto(402.89687203,689.25266764)(402.85187208,689.31266758)(402.81187744,689.37267334)
\lineto(402.69187744,689.58267334)
\curveto(402.66187227,689.66266723)(402.61187232,689.72766716)(402.54187744,689.77767334)
\curveto(402.44187249,689.85766703)(402.34187259,689.91766697)(402.24187744,689.95767334)
\curveto(402.15187278,689.99766689)(402.03687289,690.03266686)(401.89687744,690.06267334)
\curveto(401.8268731,690.08266681)(401.72187321,690.09766679)(401.58187744,690.10767334)
\curveto(401.45187348,690.11766677)(401.35187358,690.11266678)(401.28187744,690.09267334)
\lineto(401.17687744,690.09267334)
\lineto(401.02687744,690.06267334)
\curveto(400.98687394,690.06266683)(400.94187399,690.05766683)(400.89187744,690.04767334)
\curveto(400.72187421,689.99766689)(400.58187435,689.92766696)(400.47187744,689.83767334)
\curveto(400.37187456,689.75766713)(400.30187463,689.63266726)(400.26187744,689.46267334)
\curveto(400.24187469,689.3926675)(400.24187469,689.32766756)(400.26187744,689.26767334)
\curveto(400.28187465,689.20766768)(400.30187463,689.15766773)(400.32187744,689.11767334)
\curveto(400.39187454,688.99766789)(400.47187446,688.90266799)(400.56187744,688.83267334)
\curveto(400.66187427,688.76266813)(400.77687415,688.70266819)(400.90687744,688.65267334)
\curveto(401.09687383,688.57266832)(401.30187363,688.50266839)(401.52187744,688.44267334)
\lineto(402.21187744,688.29267334)
\curveto(402.45187248,688.25266864)(402.68187225,688.20266869)(402.90187744,688.14267334)
\curveto(403.1318718,688.0926688)(403.34687158,688.02766886)(403.54687744,687.94767334)
\curveto(403.63687129,687.90766898)(403.72187121,687.87266902)(403.80187744,687.84267334)
\curveto(403.89187104,687.82266907)(403.97687095,687.7876691)(404.05687744,687.73767334)
\curveto(404.24687068,687.61766927)(404.41687051,687.4876694)(404.56687744,687.34767334)
\curveto(404.7268702,687.20766968)(404.85187008,687.03266986)(404.94187744,686.82267334)
\curveto(404.97186996,686.75267014)(404.99686993,686.68267021)(405.01687744,686.61267334)
\curveto(405.03686989,686.54267035)(405.05686987,686.46767042)(405.07687744,686.38767334)
\curveto(405.08686984,686.32767056)(405.09186984,686.23267066)(405.09187744,686.10267334)
\curveto(405.10186983,685.98267091)(405.10186983,685.887671)(405.09187744,685.81767334)
\lineto(405.09187744,685.74267334)
\curveto(405.07186986,685.68267121)(405.05686987,685.62267127)(405.04687744,685.56267334)
\curveto(405.04686988,685.51267138)(405.04186989,685.46267143)(405.03187744,685.41267334)
\curveto(404.96186997,685.11267178)(404.85187008,684.84767204)(404.70187744,684.61767334)
\curveto(404.54187039,684.37767251)(404.34687058,684.18267271)(404.11687744,684.03267334)
\curveto(403.88687104,683.88267301)(403.6268713,683.75267314)(403.33687744,683.64267334)
\curveto(403.2268717,683.5926733)(403.10687182,683.55767333)(402.97687744,683.53767334)
\curveto(402.85687207,683.51767337)(402.73687219,683.4926734)(402.61687744,683.46267334)
\curveto(402.5268724,683.44267345)(402.4318725,683.43267346)(402.33187744,683.43267334)
\curveto(402.24187269,683.42267347)(402.15187278,683.40767348)(402.06187744,683.38767334)
\lineto(401.79187744,683.38767334)
\curveto(401.7318732,683.36767352)(401.6268733,683.35767353)(401.47687744,683.35767334)
\curveto(401.33687359,683.35767353)(401.23687369,683.36767352)(401.17687744,683.38767334)
\curveto(401.14687378,683.3876735)(401.11187382,683.3926735)(401.07187744,683.40267334)
\lineto(400.96687744,683.40267334)
\curveto(400.84687408,683.42267347)(400.7268742,683.43767345)(400.60687744,683.44767334)
\curveto(400.48687444,683.45767343)(400.37187456,683.47767341)(400.26187744,683.50767334)
\curveto(399.87187506,683.61767327)(399.5268754,683.74267315)(399.22687744,683.88267334)
\curveto(398.926876,684.03267286)(398.67187626,684.25267264)(398.46187744,684.54267334)
\curveto(398.32187661,684.73267216)(398.20187673,684.95267194)(398.10187744,685.20267334)
\curveto(398.08187685,685.26267163)(398.06187687,685.34267155)(398.04187744,685.44267334)
\curveto(398.02187691,685.4926714)(398.00687692,685.56267133)(397.99687744,685.65267334)
\curveto(397.98687694,685.74267115)(397.99187694,685.81767107)(398.01187744,685.87767334)
\curveto(398.04187689,685.94767094)(398.09187684,685.99767089)(398.16187744,686.02767334)
\curveto(398.21187672,686.04767084)(398.27187666,686.05767083)(398.34187744,686.05767334)
\lineto(398.56687744,686.05767334)
\lineto(399.27187744,686.05767334)
\lineto(399.51187744,686.05767334)
\curveto(399.59187534,686.05767083)(399.66187527,686.04767084)(399.72187744,686.02767334)
\curveto(399.8318751,685.9876709)(399.90187503,685.92267097)(399.93187744,685.83267334)
\curveto(399.97187496,685.74267115)(400.01687491,685.64767124)(400.06687744,685.54767334)
\curveto(400.08687484,685.49767139)(400.12187481,685.43267146)(400.17187744,685.35267334)
\curveto(400.2318747,685.27267162)(400.28187465,685.22267167)(400.32187744,685.20267334)
\curveto(400.44187449,685.10267179)(400.55687437,685.02267187)(400.66687744,684.96267334)
\curveto(400.77687415,684.91267198)(400.91687401,684.86267203)(401.08687744,684.81267334)
\curveto(401.13687379,684.7926721)(401.18687374,684.78267211)(401.23687744,684.78267334)
\curveto(401.28687364,684.7926721)(401.33687359,684.7926721)(401.38687744,684.78267334)
\curveto(401.46687346,684.76267213)(401.55187338,684.75267214)(401.64187744,684.75267334)
\curveto(401.74187319,684.76267213)(401.8268731,684.77767211)(401.89687744,684.79767334)
\curveto(401.94687298,684.80767208)(401.99187294,684.81267208)(402.03187744,684.81267334)
\curveto(402.08187285,684.81267208)(402.1318728,684.82267207)(402.18187744,684.84267334)
\curveto(402.32187261,684.892672)(402.44687248,684.95267194)(402.55687744,685.02267334)
\curveto(402.67687225,685.0926718)(402.77187216,685.18267171)(402.84187744,685.29267334)
\curveto(402.89187204,685.37267152)(402.931872,685.49767139)(402.96187744,685.66767334)
\curveto(402.98187195,685.73767115)(402.98187195,685.80267109)(402.96187744,685.86267334)
\curveto(402.94187199,685.92267097)(402.92187201,685.97267092)(402.90187744,686.01267334)
\curveto(402.8318721,686.15267074)(402.74187219,686.25767063)(402.63187744,686.32767334)
\curveto(402.5318724,686.39767049)(402.41187252,686.46267043)(402.27187744,686.52267334)
\curveto(402.08187285,686.60267029)(401.88187305,686.66767022)(401.67187744,686.71767334)
\curveto(401.46187347,686.76767012)(401.25187368,686.82267007)(401.04187744,686.88267334)
\curveto(400.96187397,686.90266999)(400.87687405,686.91766997)(400.78687744,686.92767334)
\curveto(400.70687422,686.93766995)(400.6268743,686.95266994)(400.54687744,686.97267334)
\curveto(400.2268747,687.06266983)(399.92187501,687.14766974)(399.63187744,687.22767334)
\curveto(399.34187559,687.31766957)(399.07687585,687.44766944)(398.83687744,687.61767334)
\curveto(398.55687637,687.81766907)(398.35187658,688.0876688)(398.22187744,688.42767334)
\curveto(398.20187673,688.49766839)(398.18187675,688.5926683)(398.16187744,688.71267334)
\curveto(398.14187679,688.78266811)(398.1268768,688.86766802)(398.11687744,688.96767334)
\curveto(398.10687682,689.06766782)(398.11187682,689.15766773)(398.13187744,689.23767334)
\curveto(398.15187678,689.2876676)(398.15687677,689.32766756)(398.14687744,689.35767334)
\curveto(398.13687679,689.39766749)(398.14187679,689.44266745)(398.16187744,689.49267334)
\curveto(398.18187675,689.60266729)(398.20187673,689.70266719)(398.22187744,689.79267334)
\curveto(398.25187668,689.892667)(398.28687664,689.9876669)(398.32687744,690.07767334)
\curveto(398.45687647,690.36766652)(398.63687629,690.60266629)(398.86687744,690.78267334)
\curveto(399.09687583,690.96266593)(399.35687557,691.10766578)(399.64687744,691.21767334)
\curveto(399.75687517,691.26766562)(399.87187506,691.30266559)(399.99187744,691.32267334)
\curveto(400.11187482,691.35266554)(400.23687469,691.38266551)(400.36687744,691.41267334)
\curveto(400.4268745,691.43266546)(400.48687444,691.44266545)(400.54687744,691.44267334)
\lineto(400.72687744,691.47267334)
\curveto(400.80687412,691.48266541)(400.89187404,691.4876654)(400.98187744,691.48767334)
\curveto(401.07187386,691.4876654)(401.15687377,691.4926654)(401.23687744,691.50267334)
}
}
{
\newrgbcolor{curcolor}{0 0 0}
\pscustom[linestyle=none,fillstyle=solid,fillcolor=curcolor]
{
\newpath
\moveto(406.74351807,691.27767334)
\lineto(407.86851807,691.27767334)
\curveto(407.97851563,691.27766561)(408.07851553,691.27266562)(408.16851807,691.26267334)
\curveto(408.25851535,691.25266564)(408.32351529,691.21766567)(408.36351807,691.15767334)
\curveto(408.4135152,691.09766579)(408.44351517,691.01266588)(408.45351807,690.90267334)
\curveto(408.46351515,690.80266609)(408.46851514,690.69766619)(408.46851807,690.58767334)
\lineto(408.46851807,689.53767334)
\lineto(408.46851807,687.30267334)
\curveto(408.46851514,686.94266995)(408.48351513,686.60267029)(408.51351807,686.28267334)
\curveto(408.54351507,685.96267093)(408.63351498,685.69767119)(408.78351807,685.48767334)
\curveto(408.92351469,685.27767161)(409.14851446,685.12767176)(409.45851807,685.03767334)
\curveto(409.5085141,685.02767186)(409.54851406,685.02267187)(409.57851807,685.02267334)
\curveto(409.61851399,685.02267187)(409.66351395,685.01767187)(409.71351807,685.00767334)
\curveto(409.76351385,684.99767189)(409.81851379,684.9926719)(409.87851807,684.99267334)
\curveto(409.93851367,684.9926719)(409.98351363,684.99767189)(410.01351807,685.00767334)
\curveto(410.06351355,685.02767186)(410.10351351,685.03267186)(410.13351807,685.02267334)
\curveto(410.17351344,685.01267188)(410.2135134,685.01767187)(410.25351807,685.03767334)
\curveto(410.46351315,685.0876718)(410.62851298,685.15267174)(410.74851807,685.23267334)
\curveto(410.92851268,685.34267155)(411.06851254,685.48267141)(411.16851807,685.65267334)
\curveto(411.27851233,685.83267106)(411.35351226,686.02767086)(411.39351807,686.23767334)
\curveto(411.44351217,686.45767043)(411.47351214,686.69767019)(411.48351807,686.95767334)
\curveto(411.49351212,687.22766966)(411.49851211,687.50766938)(411.49851807,687.79767334)
\lineto(411.49851807,689.61267334)
\lineto(411.49851807,690.58767334)
\lineto(411.49851807,690.85767334)
\curveto(411.49851211,690.95766593)(411.51851209,691.03766585)(411.55851807,691.09767334)
\curveto(411.608512,691.1876657)(411.68351193,691.23766565)(411.78351807,691.24767334)
\curveto(411.88351173,691.26766562)(412.00351161,691.27766561)(412.14351807,691.27767334)
\lineto(412.93851807,691.27767334)
\lineto(413.22351807,691.27767334)
\curveto(413.3135103,691.27766561)(413.38851022,691.25766563)(413.44851807,691.21767334)
\curveto(413.52851008,691.16766572)(413.57351004,691.0926658)(413.58351807,690.99267334)
\curveto(413.59351002,690.892666)(413.59851001,690.77766611)(413.59851807,690.64767334)
\lineto(413.59851807,689.50767334)
\lineto(413.59851807,685.29267334)
\lineto(413.59851807,684.22767334)
\lineto(413.59851807,683.92767334)
\curveto(413.59851001,683.82767306)(413.57851003,683.75267314)(413.53851807,683.70267334)
\curveto(413.48851012,683.62267327)(413.4135102,683.57767331)(413.31351807,683.56767334)
\curveto(413.2135104,683.55767333)(413.1085105,683.55267334)(412.99851807,683.55267334)
\lineto(412.18851807,683.55267334)
\curveto(412.07851153,683.55267334)(411.97851163,683.55767333)(411.88851807,683.56767334)
\curveto(411.8085118,683.57767331)(411.74351187,683.61767327)(411.69351807,683.68767334)
\curveto(411.67351194,683.71767317)(411.65351196,683.76267313)(411.63351807,683.82267334)
\curveto(411.62351199,683.88267301)(411.608512,683.94267295)(411.58851807,684.00267334)
\curveto(411.57851203,684.06267283)(411.56351205,684.11767277)(411.54351807,684.16767334)
\curveto(411.52351209,684.21767267)(411.49351212,684.24767264)(411.45351807,684.25767334)
\curveto(411.43351218,684.27767261)(411.4085122,684.28267261)(411.37851807,684.27267334)
\curveto(411.34851226,684.26267263)(411.32351229,684.25267264)(411.30351807,684.24267334)
\curveto(411.23351238,684.20267269)(411.17351244,684.15767273)(411.12351807,684.10767334)
\curveto(411.07351254,684.05767283)(411.01851259,684.01267288)(410.95851807,683.97267334)
\curveto(410.91851269,683.94267295)(410.87851273,683.90767298)(410.83851807,683.86767334)
\curveto(410.8085128,683.83767305)(410.76851284,683.80767308)(410.71851807,683.77767334)
\curveto(410.48851312,683.63767325)(410.21851339,683.52767336)(409.90851807,683.44767334)
\curveto(409.83851377,683.42767346)(409.76851384,683.41767347)(409.69851807,683.41767334)
\curveto(409.62851398,683.40767348)(409.55351406,683.3926735)(409.47351807,683.37267334)
\curveto(409.43351418,683.36267353)(409.38851422,683.36267353)(409.33851807,683.37267334)
\curveto(409.29851431,683.37267352)(409.25851435,683.36767352)(409.21851807,683.35767334)
\curveto(409.18851442,683.34767354)(409.12351449,683.34767354)(409.02351807,683.35767334)
\curveto(408.93351468,683.35767353)(408.87351474,683.36267353)(408.84351807,683.37267334)
\curveto(408.79351482,683.37267352)(408.74351487,683.37767351)(408.69351807,683.38767334)
\lineto(408.54351807,683.38767334)
\curveto(408.42351519,683.41767347)(408.3085153,683.44267345)(408.19851807,683.46267334)
\curveto(408.08851552,683.48267341)(407.97851563,683.51267338)(407.86851807,683.55267334)
\curveto(407.81851579,683.57267332)(407.77351584,683.5876733)(407.73351807,683.59767334)
\curveto(407.70351591,683.61767327)(407.66351595,683.63767325)(407.61351807,683.65767334)
\curveto(407.26351635,683.84767304)(406.98351663,684.11267278)(406.77351807,684.45267334)
\curveto(406.64351697,684.66267223)(406.54851706,684.91267198)(406.48851807,685.20267334)
\curveto(406.42851718,685.50267139)(406.38851722,685.81767107)(406.36851807,686.14767334)
\curveto(406.35851725,686.4876704)(406.35351726,686.83267006)(406.35351807,687.18267334)
\curveto(406.36351725,687.54266935)(406.36851724,687.89766899)(406.36851807,688.24767334)
\lineto(406.36851807,690.28767334)
\curveto(406.36851724,690.41766647)(406.36351725,690.56766632)(406.35351807,690.73767334)
\curveto(406.35351726,690.91766597)(406.37851723,691.04766584)(406.42851807,691.12767334)
\curveto(406.45851715,691.17766571)(406.51851709,691.22266567)(406.60851807,691.26267334)
\curveto(406.66851694,691.26266563)(406.7135169,691.26766562)(406.74351807,691.27767334)
}
}
{
\newrgbcolor{curcolor}{0 0 0}
\pscustom[linestyle=none,fillstyle=solid,fillcolor=curcolor]
{
\newpath
\moveto(422.27976807,684.15267334)
\curveto(422.29976022,684.04267285)(422.30976021,683.93267296)(422.30976807,683.82267334)
\curveto(422.3197602,683.71267318)(422.26976025,683.63767325)(422.15976807,683.59767334)
\curveto(422.09976042,683.56767332)(422.02976049,683.55267334)(421.94976807,683.55267334)
\lineto(421.70976807,683.55267334)
\lineto(420.89976807,683.55267334)
\lineto(420.62976807,683.55267334)
\curveto(420.54976197,683.56267333)(420.48476203,683.5876733)(420.43476807,683.62767334)
\curveto(420.36476215,683.66767322)(420.30976221,683.72267317)(420.26976807,683.79267334)
\curveto(420.23976228,683.87267302)(420.19476232,683.93767295)(420.13476807,683.98767334)
\curveto(420.1147624,684.00767288)(420.08976243,684.02267287)(420.05976807,684.03267334)
\curveto(420.02976249,684.05267284)(419.98976253,684.05767283)(419.93976807,684.04767334)
\curveto(419.88976263,684.02767286)(419.83976268,684.00267289)(419.78976807,683.97267334)
\curveto(419.74976277,683.94267295)(419.70476281,683.91767297)(419.65476807,683.89767334)
\curveto(419.60476291,683.85767303)(419.54976297,683.82267307)(419.48976807,683.79267334)
\lineto(419.30976807,683.70267334)
\curveto(419.17976334,683.64267325)(419.04476347,683.5926733)(418.90476807,683.55267334)
\curveto(418.76476375,683.52267337)(418.6197639,683.4876734)(418.46976807,683.44767334)
\curveto(418.39976412,683.42767346)(418.32976419,683.41767347)(418.25976807,683.41767334)
\curveto(418.19976432,683.40767348)(418.13476438,683.39767349)(418.06476807,683.38767334)
\lineto(417.97476807,683.38767334)
\curveto(417.94476457,683.37767351)(417.9147646,683.37267352)(417.88476807,683.37267334)
\lineto(417.71976807,683.37267334)
\curveto(417.6197649,683.35267354)(417.519765,683.35267354)(417.41976807,683.37267334)
\lineto(417.28476807,683.37267334)
\curveto(417.2147653,683.3926735)(417.14476537,683.40267349)(417.07476807,683.40267334)
\curveto(417.0147655,683.3926735)(416.95476556,683.39767349)(416.89476807,683.41767334)
\curveto(416.79476572,683.43767345)(416.69976582,683.45767343)(416.60976807,683.47767334)
\curveto(416.519766,683.4876734)(416.43476608,683.51267338)(416.35476807,683.55267334)
\curveto(416.06476645,683.66267323)(415.8147667,683.80267309)(415.60476807,683.97267334)
\curveto(415.40476711,684.15267274)(415.24476727,684.3876725)(415.12476807,684.67767334)
\curveto(415.09476742,684.74767214)(415.06476745,684.82267207)(415.03476807,684.90267334)
\curveto(415.0147675,684.98267191)(414.99476752,685.06767182)(414.97476807,685.15767334)
\curveto(414.95476756,685.20767168)(414.94476757,685.25767163)(414.94476807,685.30767334)
\curveto(414.95476756,685.35767153)(414.95476756,685.40767148)(414.94476807,685.45767334)
\curveto(414.93476758,685.4876714)(414.92476759,685.54767134)(414.91476807,685.63767334)
\curveto(414.9147676,685.73767115)(414.9197676,685.80767108)(414.92976807,685.84767334)
\curveto(414.94976757,685.94767094)(414.95976756,686.03267086)(414.95976807,686.10267334)
\lineto(415.04976807,686.43267334)
\curveto(415.07976744,686.55267034)(415.1197674,686.65767023)(415.16976807,686.74767334)
\curveto(415.33976718,687.03766985)(415.53476698,687.25766963)(415.75476807,687.40767334)
\curveto(415.97476654,687.55766933)(416.25476626,687.6876692)(416.59476807,687.79767334)
\curveto(416.72476579,687.84766904)(416.85976566,687.88266901)(416.99976807,687.90267334)
\curveto(417.13976538,687.92266897)(417.27976524,687.94766894)(417.41976807,687.97767334)
\curveto(417.49976502,687.99766889)(417.58476493,688.00766888)(417.67476807,688.00767334)
\curveto(417.76476475,688.01766887)(417.85476466,688.03266886)(417.94476807,688.05267334)
\curveto(418.0147645,688.07266882)(418.08476443,688.07766881)(418.15476807,688.06767334)
\curveto(418.22476429,688.06766882)(418.29976422,688.07766881)(418.37976807,688.09767334)
\curveto(418.44976407,688.11766877)(418.519764,688.12766876)(418.58976807,688.12767334)
\curveto(418.65976386,688.12766876)(418.73476378,688.13766875)(418.81476807,688.15767334)
\curveto(419.02476349,688.20766868)(419.2147633,688.24766864)(419.38476807,688.27767334)
\curveto(419.56476295,688.31766857)(419.72476279,688.40766848)(419.86476807,688.54767334)
\curveto(419.95476256,688.63766825)(420.0147625,688.73766815)(420.04476807,688.84767334)
\curveto(420.05476246,688.87766801)(420.05476246,688.90266799)(420.04476807,688.92267334)
\curveto(420.04476247,688.94266795)(420.04976247,688.96266793)(420.05976807,688.98267334)
\curveto(420.06976245,689.00266789)(420.07476244,689.03266786)(420.07476807,689.07267334)
\lineto(420.07476807,689.16267334)
\lineto(420.04476807,689.28267334)
\curveto(420.04476247,689.32266757)(420.03976248,689.35766753)(420.02976807,689.38767334)
\curveto(419.92976259,689.6876672)(419.7197628,689.892667)(419.39976807,690.00267334)
\curveto(419.30976321,690.03266686)(419.19976332,690.05266684)(419.06976807,690.06267334)
\curveto(418.94976357,690.08266681)(418.82476369,690.0876668)(418.69476807,690.07767334)
\curveto(418.56476395,690.07766681)(418.43976408,690.06766682)(418.31976807,690.04767334)
\curveto(418.19976432,690.02766686)(418.09476442,690.00266689)(418.00476807,689.97267334)
\curveto(417.94476457,689.95266694)(417.88476463,689.92266697)(417.82476807,689.88267334)
\curveto(417.77476474,689.85266704)(417.72476479,689.81766707)(417.67476807,689.77767334)
\curveto(417.62476489,689.73766715)(417.56976495,689.68266721)(417.50976807,689.61267334)
\curveto(417.45976506,689.54266735)(417.42476509,689.47766741)(417.40476807,689.41767334)
\curveto(417.35476516,689.31766757)(417.30976521,689.22266767)(417.26976807,689.13267334)
\curveto(417.23976528,689.04266785)(417.16976535,688.98266791)(417.05976807,688.95267334)
\curveto(416.97976554,688.93266796)(416.89476562,688.92266797)(416.80476807,688.92267334)
\lineto(416.53476807,688.92267334)
\lineto(415.96476807,688.92267334)
\curveto(415.9147666,688.92266797)(415.86476665,688.91766797)(415.81476807,688.90767334)
\curveto(415.76476675,688.90766798)(415.7197668,688.91266798)(415.67976807,688.92267334)
\lineto(415.54476807,688.92267334)
\curveto(415.52476699,688.93266796)(415.49976702,688.93766795)(415.46976807,688.93767334)
\curveto(415.43976708,688.93766795)(415.4147671,688.94766794)(415.39476807,688.96767334)
\curveto(415.3147672,688.9876679)(415.25976726,689.05266784)(415.22976807,689.16267334)
\curveto(415.2197673,689.21266768)(415.2197673,689.26266763)(415.22976807,689.31267334)
\curveto(415.23976728,689.36266753)(415.24976727,689.40766748)(415.25976807,689.44767334)
\curveto(415.28976723,689.55766733)(415.3197672,689.65766723)(415.34976807,689.74767334)
\curveto(415.38976713,689.84766704)(415.43476708,689.93766695)(415.48476807,690.01767334)
\lineto(415.57476807,690.16767334)
\lineto(415.66476807,690.31767334)
\curveto(415.74476677,690.42766646)(415.84476667,690.53266636)(415.96476807,690.63267334)
\curveto(415.98476653,690.64266625)(416.0147665,690.66766622)(416.05476807,690.70767334)
\curveto(416.10476641,690.74766614)(416.14976637,690.78266611)(416.18976807,690.81267334)
\curveto(416.22976629,690.84266605)(416.27476624,690.87266602)(416.32476807,690.90267334)
\curveto(416.49476602,691.01266588)(416.67476584,691.09766579)(416.86476807,691.15767334)
\curveto(417.05476546,691.22766566)(417.24976527,691.2926656)(417.44976807,691.35267334)
\curveto(417.56976495,691.38266551)(417.69476482,691.40266549)(417.82476807,691.41267334)
\curveto(417.95476456,691.42266547)(418.08476443,691.44266545)(418.21476807,691.47267334)
\curveto(418.25476426,691.48266541)(418.3147642,691.48266541)(418.39476807,691.47267334)
\curveto(418.48476403,691.46266543)(418.53976398,691.46766542)(418.55976807,691.48767334)
\curveto(418.96976355,691.49766539)(419.35976316,691.48266541)(419.72976807,691.44267334)
\curveto(420.10976241,691.40266549)(420.44976207,691.32766556)(420.74976807,691.21767334)
\curveto(421.05976146,691.10766578)(421.32476119,690.95766593)(421.54476807,690.76767334)
\curveto(421.76476075,690.5876663)(421.93476058,690.35266654)(422.05476807,690.06267334)
\curveto(422.12476039,689.892667)(422.16476035,689.69766719)(422.17476807,689.47767334)
\curveto(422.18476033,689.25766763)(422.18976033,689.03266786)(422.18976807,688.80267334)
\lineto(422.18976807,685.45767334)
\lineto(422.18976807,684.87267334)
\curveto(422.18976033,684.68267221)(422.20976031,684.50767238)(422.24976807,684.34767334)
\curveto(422.25976026,684.31767257)(422.26476025,684.28267261)(422.26476807,684.24267334)
\curveto(422.26476025,684.21267268)(422.26976025,684.18267271)(422.27976807,684.15267334)
\moveto(420.07476807,686.46267334)
\curveto(420.08476243,686.51267038)(420.08976243,686.56767032)(420.08976807,686.62767334)
\curveto(420.08976243,686.69767019)(420.08476243,686.75767013)(420.07476807,686.80767334)
\curveto(420.05476246,686.86767002)(420.04476247,686.92266997)(420.04476807,686.97267334)
\curveto(420.04476247,687.02266987)(420.02476249,687.06266983)(419.98476807,687.09267334)
\curveto(419.93476258,687.13266976)(419.85976266,687.15266974)(419.75976807,687.15267334)
\curveto(419.7197628,687.14266975)(419.68476283,687.13266976)(419.65476807,687.12267334)
\curveto(419.62476289,687.12266977)(419.58976293,687.11766977)(419.54976807,687.10767334)
\curveto(419.47976304,687.0876698)(419.40476311,687.07266982)(419.32476807,687.06267334)
\curveto(419.24476327,687.05266984)(419.16476335,687.03766985)(419.08476807,687.01767334)
\curveto(419.05476346,687.00766988)(419.00976351,687.00266989)(418.94976807,687.00267334)
\curveto(418.8197637,686.97266992)(418.68976383,686.95266994)(418.55976807,686.94267334)
\curveto(418.42976409,686.93266996)(418.30476421,686.90766998)(418.18476807,686.86767334)
\curveto(418.10476441,686.84767004)(418.02976449,686.82767006)(417.95976807,686.80767334)
\curveto(417.88976463,686.79767009)(417.8197647,686.77767011)(417.74976807,686.74767334)
\curveto(417.53976498,686.65767023)(417.35976516,686.52267037)(417.20976807,686.34267334)
\curveto(417.06976545,686.16267073)(417.0197655,685.91267098)(417.05976807,685.59267334)
\curveto(417.07976544,685.42267147)(417.13476538,685.28267161)(417.22476807,685.17267334)
\curveto(417.29476522,685.06267183)(417.39976512,684.97267192)(417.53976807,684.90267334)
\curveto(417.67976484,684.84267205)(417.82976469,684.79767209)(417.98976807,684.76767334)
\curveto(418.15976436,684.73767215)(418.33476418,684.72767216)(418.51476807,684.73767334)
\curveto(418.70476381,684.75767213)(418.87976364,684.7926721)(419.03976807,684.84267334)
\curveto(419.29976322,684.92267197)(419.50476301,685.04767184)(419.65476807,685.21767334)
\curveto(419.80476271,685.39767149)(419.9197626,685.61767127)(419.99976807,685.87767334)
\curveto(420.0197625,685.94767094)(420.02976249,686.01767087)(420.02976807,686.08767334)
\curveto(420.03976248,686.16767072)(420.05476246,686.24767064)(420.07476807,686.32767334)
\lineto(420.07476807,686.46267334)
}
}
{
\newrgbcolor{curcolor}{0 0 0}
\pscustom[linestyle=none,fillstyle=solid,fillcolor=curcolor]
{
\newpath
\moveto(428.26804932,691.48767334)
\curveto(428.378044,691.4876654)(428.47304391,691.47766541)(428.55304932,691.45767334)
\curveto(428.64304374,691.43766545)(428.71304367,691.3926655)(428.76304932,691.32267334)
\curveto(428.82304356,691.24266565)(428.85304353,691.10266579)(428.85304932,690.90267334)
\lineto(428.85304932,690.39267334)
\lineto(428.85304932,690.01767334)
\curveto(428.86304352,689.87766701)(428.84804353,689.76766712)(428.80804932,689.68767334)
\curveto(428.76804361,689.61766727)(428.70804367,689.57266732)(428.62804932,689.55267334)
\curveto(428.55804382,689.53266736)(428.47304391,689.52266737)(428.37304932,689.52267334)
\curveto(428.2830441,689.52266737)(428.1830442,689.52766736)(428.07304932,689.53767334)
\curveto(427.97304441,689.54766734)(427.8780445,689.54266735)(427.78804932,689.52267334)
\curveto(427.71804466,689.50266739)(427.64804473,689.4876674)(427.57804932,689.47767334)
\curveto(427.50804487,689.47766741)(427.44304494,689.46766742)(427.38304932,689.44767334)
\curveto(427.22304516,689.39766749)(427.06304532,689.32266757)(426.90304932,689.22267334)
\curveto(426.74304564,689.13266776)(426.61804576,689.02766786)(426.52804932,688.90767334)
\curveto(426.4780459,688.82766806)(426.42304596,688.74266815)(426.36304932,688.65267334)
\curveto(426.31304607,688.57266832)(426.26304612,688.4876684)(426.21304932,688.39767334)
\curveto(426.1830462,688.31766857)(426.15304623,688.23266866)(426.12304932,688.14267334)
\lineto(426.06304932,687.90267334)
\curveto(426.04304634,687.83266906)(426.03304635,687.75766913)(426.03304932,687.67767334)
\curveto(426.03304635,687.60766928)(426.02304636,687.53766935)(426.00304932,687.46767334)
\curveto(425.99304639,687.42766946)(425.98804639,687.3876695)(425.98804932,687.34767334)
\curveto(425.99804638,687.31766957)(425.99804638,687.2876696)(425.98804932,687.25767334)
\lineto(425.98804932,687.01767334)
\curveto(425.96804641,686.94766994)(425.96304642,686.86767002)(425.97304932,686.77767334)
\curveto(425.9830464,686.69767019)(425.98804639,686.61767027)(425.98804932,686.53767334)
\lineto(425.98804932,685.57767334)
\lineto(425.98804932,684.30267334)
\curveto(425.98804639,684.17267272)(425.9830464,684.05267284)(425.97304932,683.94267334)
\curveto(425.96304642,683.83267306)(425.93304645,683.74267315)(425.88304932,683.67267334)
\curveto(425.86304652,683.64267325)(425.82804655,683.61767327)(425.77804932,683.59767334)
\curveto(425.73804664,683.5876733)(425.69304669,683.57767331)(425.64304932,683.56767334)
\lineto(425.56804932,683.56767334)
\curveto(425.51804686,683.55767333)(425.46304692,683.55267334)(425.40304932,683.55267334)
\lineto(425.23804932,683.55267334)
\lineto(424.59304932,683.55267334)
\curveto(424.53304785,683.56267333)(424.46804791,683.56767332)(424.39804932,683.56767334)
\lineto(424.20304932,683.56767334)
\curveto(424.15304823,683.5876733)(424.10304828,683.60267329)(424.05304932,683.61267334)
\curveto(424.00304838,683.63267326)(423.96804841,683.66767322)(423.94804932,683.71767334)
\curveto(423.90804847,683.76767312)(423.8830485,683.83767305)(423.87304932,683.92767334)
\lineto(423.87304932,684.22767334)
\lineto(423.87304932,685.24767334)
\lineto(423.87304932,689.47767334)
\lineto(423.87304932,690.58767334)
\lineto(423.87304932,690.87267334)
\curveto(423.87304851,690.97266592)(423.89304849,691.05266584)(423.93304932,691.11267334)
\curveto(423.9830484,691.1926657)(424.05804832,691.24266565)(424.15804932,691.26267334)
\curveto(424.25804812,691.28266561)(424.378048,691.2926656)(424.51804932,691.29267334)
\lineto(425.28304932,691.29267334)
\curveto(425.40304698,691.2926656)(425.50804687,691.28266561)(425.59804932,691.26267334)
\curveto(425.68804669,691.25266564)(425.75804662,691.20766568)(425.80804932,691.12767334)
\curveto(425.83804654,691.07766581)(425.85304653,691.00766588)(425.85304932,690.91767334)
\lineto(425.88304932,690.64767334)
\curveto(425.89304649,690.56766632)(425.90804647,690.4926664)(425.92804932,690.42267334)
\curveto(425.95804642,690.35266654)(426.00804637,690.31766657)(426.07804932,690.31767334)
\curveto(426.09804628,690.33766655)(426.11804626,690.34766654)(426.13804932,690.34767334)
\curveto(426.15804622,690.34766654)(426.1780462,690.35766653)(426.19804932,690.37767334)
\curveto(426.25804612,690.42766646)(426.30804607,690.48266641)(426.34804932,690.54267334)
\curveto(426.39804598,690.61266628)(426.45804592,690.67266622)(426.52804932,690.72267334)
\curveto(426.56804581,690.75266614)(426.60304578,690.78266611)(426.63304932,690.81267334)
\curveto(426.66304572,690.85266604)(426.69804568,690.887666)(426.73804932,690.91767334)
\lineto(427.00804932,691.09767334)
\curveto(427.10804527,691.15766573)(427.20804517,691.21266568)(427.30804932,691.26267334)
\curveto(427.40804497,691.30266559)(427.50804487,691.33766555)(427.60804932,691.36767334)
\lineto(427.93804932,691.45767334)
\curveto(427.96804441,691.46766542)(428.02304436,691.46766542)(428.10304932,691.45767334)
\curveto(428.19304419,691.45766543)(428.24804413,691.46766542)(428.26804932,691.48767334)
}
}
{
\newrgbcolor{curcolor}{0 0 0}
\pscustom[linestyle=none,fillstyle=solid,fillcolor=curcolor]
{
\newpath
\moveto(431.77312744,694.14267334)
\curveto(431.84312449,694.06266283)(431.87812446,693.94266295)(431.87812744,693.78267334)
\lineto(431.87812744,693.31767334)
\lineto(431.87812744,692.91267334)
\curveto(431.87812446,692.77266412)(431.84312449,692.67766421)(431.77312744,692.62767334)
\curveto(431.71312462,692.57766431)(431.6331247,692.54766434)(431.53312744,692.53767334)
\curveto(431.44312489,692.52766436)(431.34312499,692.52266437)(431.23312744,692.52267334)
\lineto(430.39312744,692.52267334)
\curveto(430.28312605,692.52266437)(430.18312615,692.52766436)(430.09312744,692.53767334)
\curveto(430.01312632,692.54766434)(429.94312639,692.57766431)(429.88312744,692.62767334)
\curveto(429.84312649,692.65766423)(429.81312652,692.71266418)(429.79312744,692.79267334)
\curveto(429.78312655,692.88266401)(429.77312656,692.97766391)(429.76312744,693.07767334)
\lineto(429.76312744,693.40767334)
\curveto(429.77312656,693.51766337)(429.77812656,693.61266328)(429.77812744,693.69267334)
\lineto(429.77812744,693.90267334)
\curveto(429.78812655,693.97266292)(429.80812653,694.03266286)(429.83812744,694.08267334)
\curveto(429.85812648,694.12266277)(429.88312645,694.15266274)(429.91312744,694.17267334)
\lineto(430.03312744,694.23267334)
\curveto(430.05312628,694.23266266)(430.07812626,694.23266266)(430.10812744,694.23267334)
\curveto(430.1381262,694.24266265)(430.16312617,694.24766264)(430.18312744,694.24767334)
\lineto(431.27812744,694.24767334)
\curveto(431.37812496,694.24766264)(431.47312486,694.24266265)(431.56312744,694.23267334)
\curveto(431.65312468,694.22266267)(431.72312461,694.1926627)(431.77312744,694.14267334)
\moveto(431.87812744,684.37767334)
\curveto(431.87812446,684.17767271)(431.87312446,684.00767288)(431.86312744,683.86767334)
\curveto(431.85312448,683.72767316)(431.76312457,683.63267326)(431.59312744,683.58267334)
\curveto(431.5331248,683.56267333)(431.46812487,683.55267334)(431.39812744,683.55267334)
\curveto(431.32812501,683.56267333)(431.25312508,683.56767332)(431.17312744,683.56767334)
\lineto(430.33312744,683.56767334)
\curveto(430.24312609,683.56767332)(430.15312618,683.57267332)(430.06312744,683.58267334)
\curveto(429.98312635,683.5926733)(429.92312641,683.62267327)(429.88312744,683.67267334)
\curveto(429.82312651,683.74267315)(429.78812655,683.82767306)(429.77812744,683.92767334)
\lineto(429.77812744,684.27267334)
\lineto(429.77812744,690.60267334)
\lineto(429.77812744,690.90267334)
\curveto(429.77812656,691.00266589)(429.79812654,691.08266581)(429.83812744,691.14267334)
\curveto(429.89812644,691.21266568)(429.98312635,691.25766563)(430.09312744,691.27767334)
\curveto(430.11312622,691.2876656)(430.1381262,691.2876656)(430.16812744,691.27767334)
\curveto(430.20812613,691.27766561)(430.2381261,691.28266561)(430.25812744,691.29267334)
\lineto(431.00812744,691.29267334)
\lineto(431.20312744,691.29267334)
\curveto(431.28312505,691.30266559)(431.34812499,691.30266559)(431.39812744,691.29267334)
\lineto(431.51812744,691.29267334)
\curveto(431.57812476,691.27266562)(431.6331247,691.25766563)(431.68312744,691.24767334)
\curveto(431.7331246,691.23766565)(431.77312456,691.20766568)(431.80312744,691.15767334)
\curveto(431.84312449,691.10766578)(431.86312447,691.03766585)(431.86312744,690.94767334)
\curveto(431.87312446,690.85766603)(431.87812446,690.76266613)(431.87812744,690.66267334)
\lineto(431.87812744,684.37767334)
}
}
{
\newrgbcolor{curcolor}{0 0 0}
\pscustom[linestyle=none,fillstyle=solid,fillcolor=curcolor]
{
\newpath
\moveto(441.31031494,687.73767334)
\curveto(441.33030637,687.67766921)(441.34030636,687.5926693)(441.34031494,687.48267334)
\curveto(441.34030636,687.37266952)(441.33030637,687.2876696)(441.31031494,687.22767334)
\lineto(441.31031494,687.07767334)
\curveto(441.29030641,686.99766989)(441.28030642,686.91766997)(441.28031494,686.83767334)
\curveto(441.29030641,686.75767013)(441.28530642,686.67767021)(441.26531494,686.59767334)
\curveto(441.24530646,686.52767036)(441.23030647,686.46267043)(441.22031494,686.40267334)
\curveto(441.21030649,686.34267055)(441.2003065,686.27767061)(441.19031494,686.20767334)
\curveto(441.15030655,686.09767079)(441.11530659,685.98267091)(441.08531494,685.86267334)
\curveto(441.05530665,685.75267114)(441.01530669,685.64767124)(440.96531494,685.54767334)
\curveto(440.75530695,685.06767182)(440.48030722,684.67767221)(440.14031494,684.37767334)
\curveto(439.8003079,684.07767281)(439.39030831,683.82767306)(438.91031494,683.62767334)
\curveto(438.79030891,683.57767331)(438.66530904,683.54267335)(438.53531494,683.52267334)
\curveto(438.41530929,683.4926734)(438.29030941,683.46267343)(438.16031494,683.43267334)
\curveto(438.11030959,683.41267348)(438.05530965,683.40267349)(437.99531494,683.40267334)
\curveto(437.93530977,683.40267349)(437.88030982,683.39767349)(437.83031494,683.38767334)
\lineto(437.72531494,683.38767334)
\curveto(437.69531001,683.37767351)(437.66531004,683.37267352)(437.63531494,683.37267334)
\curveto(437.58531012,683.36267353)(437.5053102,683.35767353)(437.39531494,683.35767334)
\curveto(437.28531042,683.34767354)(437.2003105,683.35267354)(437.14031494,683.37267334)
\lineto(436.99031494,683.37267334)
\curveto(436.94031076,683.38267351)(436.88531082,683.3876735)(436.82531494,683.38767334)
\curveto(436.77531093,683.37767351)(436.72531098,683.38267351)(436.67531494,683.40267334)
\curveto(436.63531107,683.41267348)(436.59531111,683.41767347)(436.55531494,683.41767334)
\curveto(436.52531118,683.41767347)(436.48531122,683.42267347)(436.43531494,683.43267334)
\curveto(436.33531137,683.46267343)(436.23531147,683.4876734)(436.13531494,683.50767334)
\curveto(436.03531167,683.52767336)(435.94031176,683.55767333)(435.85031494,683.59767334)
\curveto(435.73031197,683.63767325)(435.61531209,683.67767321)(435.50531494,683.71767334)
\curveto(435.4053123,683.75767313)(435.3003124,683.80767308)(435.19031494,683.86767334)
\curveto(434.84031286,684.07767281)(434.54031316,684.32267257)(434.29031494,684.60267334)
\curveto(434.04031366,684.88267201)(433.83031387,685.21767167)(433.66031494,685.60767334)
\curveto(433.61031409,685.69767119)(433.57031413,685.7926711)(433.54031494,685.89267334)
\curveto(433.52031418,685.9926709)(433.49531421,686.09767079)(433.46531494,686.20767334)
\curveto(433.44531426,686.25767063)(433.43531427,686.30267059)(433.43531494,686.34267334)
\curveto(433.43531427,686.38267051)(433.42531428,686.42767046)(433.40531494,686.47767334)
\curveto(433.38531432,686.55767033)(433.37531433,686.63767025)(433.37531494,686.71767334)
\curveto(433.37531433,686.80767008)(433.36531434,686.89267)(433.34531494,686.97267334)
\curveto(433.33531437,687.02266987)(433.33031437,687.06766982)(433.33031494,687.10767334)
\lineto(433.33031494,687.24267334)
\curveto(433.31031439,687.30266959)(433.3003144,687.3876695)(433.30031494,687.49767334)
\curveto(433.31031439,687.60766928)(433.32531438,687.6926692)(433.34531494,687.75267334)
\lineto(433.34531494,687.85767334)
\curveto(433.35531435,687.90766898)(433.35531435,687.95766893)(433.34531494,688.00767334)
\curveto(433.34531436,688.06766882)(433.35531435,688.12266877)(433.37531494,688.17267334)
\curveto(433.38531432,688.22266867)(433.39031431,688.26766862)(433.39031494,688.30767334)
\curveto(433.39031431,688.35766853)(433.4003143,688.40766848)(433.42031494,688.45767334)
\curveto(433.46031424,688.5876683)(433.49531421,688.71266818)(433.52531494,688.83267334)
\curveto(433.55531415,688.96266793)(433.59531411,689.0876678)(433.64531494,689.20767334)
\curveto(433.82531388,689.61766727)(434.04031366,689.95766693)(434.29031494,690.22767334)
\curveto(434.54031316,690.50766638)(434.84531286,690.76266613)(435.20531494,690.99267334)
\curveto(435.3053124,691.04266585)(435.41031229,691.0876658)(435.52031494,691.12767334)
\curveto(435.63031207,691.16766572)(435.74031196,691.21266568)(435.85031494,691.26267334)
\curveto(435.98031172,691.31266558)(436.11531159,691.34766554)(436.25531494,691.36767334)
\curveto(436.39531131,691.3876655)(436.54031116,691.41766547)(436.69031494,691.45767334)
\curveto(436.77031093,691.46766542)(436.84531086,691.47266542)(436.91531494,691.47267334)
\curveto(436.98531072,691.47266542)(437.05531065,691.47766541)(437.12531494,691.48767334)
\curveto(437.70531,691.49766539)(438.2053095,691.43766545)(438.62531494,691.30767334)
\curveto(439.05530865,691.17766571)(439.43530827,690.99766589)(439.76531494,690.76767334)
\curveto(439.87530783,690.6876662)(439.98530772,690.59766629)(440.09531494,690.49767334)
\curveto(440.21530749,690.40766648)(440.31530739,690.30766658)(440.39531494,690.19767334)
\curveto(440.47530723,690.09766679)(440.54530716,689.99766689)(440.60531494,689.89767334)
\curveto(440.67530703,689.79766709)(440.74530696,689.6926672)(440.81531494,689.58267334)
\curveto(440.88530682,689.47266742)(440.94030676,689.35266754)(440.98031494,689.22267334)
\curveto(441.02030668,689.10266779)(441.06530664,688.97266792)(441.11531494,688.83267334)
\curveto(441.14530656,688.75266814)(441.17030653,688.66766822)(441.19031494,688.57767334)
\lineto(441.25031494,688.30767334)
\curveto(441.26030644,688.26766862)(441.26530644,688.22766866)(441.26531494,688.18767334)
\curveto(441.26530644,688.14766874)(441.27030643,688.10766878)(441.28031494,688.06767334)
\curveto(441.3003064,688.01766887)(441.3053064,687.96266893)(441.29531494,687.90267334)
\curveto(441.28530642,687.84266905)(441.29030641,687.7876691)(441.31031494,687.73767334)
\moveto(439.21031494,687.19767334)
\curveto(439.22030848,687.24766964)(439.22530848,687.31766957)(439.22531494,687.40767334)
\curveto(439.22530848,687.50766938)(439.22030848,687.58266931)(439.21031494,687.63267334)
\lineto(439.21031494,687.75267334)
\curveto(439.19030851,687.80266909)(439.18030852,687.85766903)(439.18031494,687.91767334)
\curveto(439.18030852,687.97766891)(439.17530853,688.03266886)(439.16531494,688.08267334)
\curveto(439.16530854,688.12266877)(439.16030854,688.15266874)(439.15031494,688.17267334)
\lineto(439.09031494,688.41267334)
\curveto(439.08030862,688.50266839)(439.06030864,688.5876683)(439.03031494,688.66767334)
\curveto(438.92030878,688.92766796)(438.79030891,689.14766774)(438.64031494,689.32767334)
\curveto(438.49030921,689.51766737)(438.29030941,689.66766722)(438.04031494,689.77767334)
\curveto(437.98030972,689.79766709)(437.92030978,689.81266708)(437.86031494,689.82267334)
\curveto(437.8003099,689.84266705)(437.73530997,689.86266703)(437.66531494,689.88267334)
\curveto(437.58531012,689.90266699)(437.5003102,689.90766698)(437.41031494,689.89767334)
\lineto(437.14031494,689.89767334)
\curveto(437.11031059,689.87766701)(437.07531063,689.86766702)(437.03531494,689.86767334)
\curveto(436.99531071,689.87766701)(436.96031074,689.87766701)(436.93031494,689.86767334)
\lineto(436.72031494,689.80767334)
\curveto(436.66031104,689.79766709)(436.6053111,689.77766711)(436.55531494,689.74767334)
\curveto(436.3053114,689.63766725)(436.1003116,689.47766741)(435.94031494,689.26767334)
\curveto(435.79031191,689.06766782)(435.67031203,688.83266806)(435.58031494,688.56267334)
\curveto(435.55031215,688.46266843)(435.52531218,688.35766853)(435.50531494,688.24767334)
\curveto(435.49531221,688.13766875)(435.48031222,688.02766886)(435.46031494,687.91767334)
\curveto(435.45031225,687.86766902)(435.44531226,687.81766907)(435.44531494,687.76767334)
\lineto(435.44531494,687.61767334)
\curveto(435.42531228,687.54766934)(435.41531229,687.44266945)(435.41531494,687.30267334)
\curveto(435.42531228,687.16266973)(435.44031226,687.05766983)(435.46031494,686.98767334)
\lineto(435.46031494,686.85267334)
\curveto(435.48031222,686.77267012)(435.49531221,686.6926702)(435.50531494,686.61267334)
\curveto(435.51531219,686.54267035)(435.53031217,686.46767042)(435.55031494,686.38767334)
\curveto(435.65031205,686.0876708)(435.75531195,685.84267105)(435.86531494,685.65267334)
\curveto(435.98531172,685.47267142)(436.17031153,685.30767158)(436.42031494,685.15767334)
\curveto(436.49031121,685.10767178)(436.56531114,685.06767182)(436.64531494,685.03767334)
\curveto(436.73531097,685.00767188)(436.82531088,684.98267191)(436.91531494,684.96267334)
\curveto(436.95531075,684.95267194)(436.99031071,684.94767194)(437.02031494,684.94767334)
\curveto(437.05031065,684.95767193)(437.08531062,684.95767193)(437.12531494,684.94767334)
\lineto(437.24531494,684.91767334)
\curveto(437.29531041,684.91767197)(437.34031036,684.92267197)(437.38031494,684.93267334)
\lineto(437.50031494,684.93267334)
\curveto(437.58031012,684.95267194)(437.66031004,684.96767192)(437.74031494,684.97767334)
\curveto(437.82030988,684.9876719)(437.89530981,685.00767188)(437.96531494,685.03767334)
\curveto(438.22530948,685.13767175)(438.43530927,685.27267162)(438.59531494,685.44267334)
\curveto(438.75530895,685.61267128)(438.89030881,685.82267107)(439.00031494,686.07267334)
\curveto(439.04030866,686.17267072)(439.07030863,686.27267062)(439.09031494,686.37267334)
\curveto(439.11030859,686.47267042)(439.13530857,686.57767031)(439.16531494,686.68767334)
\curveto(439.17530853,686.72767016)(439.18030852,686.76267013)(439.18031494,686.79267334)
\curveto(439.18030852,686.83267006)(439.18530852,686.87267002)(439.19531494,686.91267334)
\lineto(439.19531494,687.04767334)
\curveto(439.19530851,687.09766979)(439.2003085,687.14766974)(439.21031494,687.19767334)
}
}
{
\newrgbcolor{curcolor}{0 0 0}
\pscustom[linestyle=none,fillstyle=solid,fillcolor=curcolor]
{
\newpath
\moveto(445.68023682,691.50267334)
\curveto(446.43023232,691.52266537)(447.08023167,691.43766545)(447.63023682,691.24767334)
\curveto(448.19023056,691.06766582)(448.61523013,690.75266614)(448.90523682,690.30267334)
\curveto(448.97522977,690.1926667)(449.03522971,690.07766681)(449.08523682,689.95767334)
\curveto(449.1452296,689.84766704)(449.19522955,689.72266717)(449.23523682,689.58267334)
\curveto(449.25522949,689.52266737)(449.26522948,689.45766743)(449.26523682,689.38767334)
\curveto(449.26522948,689.31766757)(449.25522949,689.25766763)(449.23523682,689.20767334)
\curveto(449.19522955,689.14766774)(449.14022961,689.10766778)(449.07023682,689.08767334)
\curveto(449.02022973,689.06766782)(448.96022979,689.05766783)(448.89023682,689.05767334)
\lineto(448.68023682,689.05767334)
\lineto(448.02023682,689.05767334)
\curveto(447.9502308,689.05766783)(447.88023087,689.05266784)(447.81023682,689.04267334)
\curveto(447.74023101,689.04266785)(447.67523107,689.05266784)(447.61523682,689.07267334)
\curveto(447.51523123,689.0926678)(447.44023131,689.13266776)(447.39023682,689.19267334)
\curveto(447.34023141,689.25266764)(447.29523145,689.31266758)(447.25523682,689.37267334)
\lineto(447.13523682,689.58267334)
\curveto(447.10523164,689.66266723)(447.05523169,689.72766716)(446.98523682,689.77767334)
\curveto(446.88523186,689.85766703)(446.78523196,689.91766697)(446.68523682,689.95767334)
\curveto(446.59523215,689.99766689)(446.48023227,690.03266686)(446.34023682,690.06267334)
\curveto(446.27023248,690.08266681)(446.16523258,690.09766679)(446.02523682,690.10767334)
\curveto(445.89523285,690.11766677)(445.79523295,690.11266678)(445.72523682,690.09267334)
\lineto(445.62023682,690.09267334)
\lineto(445.47023682,690.06267334)
\curveto(445.43023332,690.06266683)(445.38523336,690.05766683)(445.33523682,690.04767334)
\curveto(445.16523358,689.99766689)(445.02523372,689.92766696)(444.91523682,689.83767334)
\curveto(444.81523393,689.75766713)(444.745234,689.63266726)(444.70523682,689.46267334)
\curveto(444.68523406,689.3926675)(444.68523406,689.32766756)(444.70523682,689.26767334)
\curveto(444.72523402,689.20766768)(444.745234,689.15766773)(444.76523682,689.11767334)
\curveto(444.83523391,688.99766789)(444.91523383,688.90266799)(445.00523682,688.83267334)
\curveto(445.10523364,688.76266813)(445.22023353,688.70266819)(445.35023682,688.65267334)
\curveto(445.54023321,688.57266832)(445.745233,688.50266839)(445.96523682,688.44267334)
\lineto(446.65523682,688.29267334)
\curveto(446.89523185,688.25266864)(447.12523162,688.20266869)(447.34523682,688.14267334)
\curveto(447.57523117,688.0926688)(447.79023096,688.02766886)(447.99023682,687.94767334)
\curveto(448.08023067,687.90766898)(448.16523058,687.87266902)(448.24523682,687.84267334)
\curveto(448.33523041,687.82266907)(448.42023033,687.7876691)(448.50023682,687.73767334)
\curveto(448.69023006,687.61766927)(448.86022989,687.4876694)(449.01023682,687.34767334)
\curveto(449.17022958,687.20766968)(449.29522945,687.03266986)(449.38523682,686.82267334)
\curveto(449.41522933,686.75267014)(449.44022931,686.68267021)(449.46023682,686.61267334)
\curveto(449.48022927,686.54267035)(449.50022925,686.46767042)(449.52023682,686.38767334)
\curveto(449.53022922,686.32767056)(449.53522921,686.23267066)(449.53523682,686.10267334)
\curveto(449.5452292,685.98267091)(449.5452292,685.887671)(449.53523682,685.81767334)
\lineto(449.53523682,685.74267334)
\curveto(449.51522923,685.68267121)(449.50022925,685.62267127)(449.49023682,685.56267334)
\curveto(449.49022926,685.51267138)(449.48522926,685.46267143)(449.47523682,685.41267334)
\curveto(449.40522934,685.11267178)(449.29522945,684.84767204)(449.14523682,684.61767334)
\curveto(448.98522976,684.37767251)(448.79022996,684.18267271)(448.56023682,684.03267334)
\curveto(448.33023042,683.88267301)(448.07023068,683.75267314)(447.78023682,683.64267334)
\curveto(447.67023108,683.5926733)(447.5502312,683.55767333)(447.42023682,683.53767334)
\curveto(447.30023145,683.51767337)(447.18023157,683.4926734)(447.06023682,683.46267334)
\curveto(446.97023178,683.44267345)(446.87523187,683.43267346)(446.77523682,683.43267334)
\curveto(446.68523206,683.42267347)(446.59523215,683.40767348)(446.50523682,683.38767334)
\lineto(446.23523682,683.38767334)
\curveto(446.17523257,683.36767352)(446.07023268,683.35767353)(445.92023682,683.35767334)
\curveto(445.78023297,683.35767353)(445.68023307,683.36767352)(445.62023682,683.38767334)
\curveto(445.59023316,683.3876735)(445.55523319,683.3926735)(445.51523682,683.40267334)
\lineto(445.41023682,683.40267334)
\curveto(445.29023346,683.42267347)(445.17023358,683.43767345)(445.05023682,683.44767334)
\curveto(444.93023382,683.45767343)(444.81523393,683.47767341)(444.70523682,683.50767334)
\curveto(444.31523443,683.61767327)(443.97023478,683.74267315)(443.67023682,683.88267334)
\curveto(443.37023538,684.03267286)(443.11523563,684.25267264)(442.90523682,684.54267334)
\curveto(442.76523598,684.73267216)(442.6452361,684.95267194)(442.54523682,685.20267334)
\curveto(442.52523622,685.26267163)(442.50523624,685.34267155)(442.48523682,685.44267334)
\curveto(442.46523628,685.4926714)(442.4502363,685.56267133)(442.44023682,685.65267334)
\curveto(442.43023632,685.74267115)(442.43523631,685.81767107)(442.45523682,685.87767334)
\curveto(442.48523626,685.94767094)(442.53523621,685.99767089)(442.60523682,686.02767334)
\curveto(442.65523609,686.04767084)(442.71523603,686.05767083)(442.78523682,686.05767334)
\lineto(443.01023682,686.05767334)
\lineto(443.71523682,686.05767334)
\lineto(443.95523682,686.05767334)
\curveto(444.03523471,686.05767083)(444.10523464,686.04767084)(444.16523682,686.02767334)
\curveto(444.27523447,685.9876709)(444.3452344,685.92267097)(444.37523682,685.83267334)
\curveto(444.41523433,685.74267115)(444.46023429,685.64767124)(444.51023682,685.54767334)
\curveto(444.53023422,685.49767139)(444.56523418,685.43267146)(444.61523682,685.35267334)
\curveto(444.67523407,685.27267162)(444.72523402,685.22267167)(444.76523682,685.20267334)
\curveto(444.88523386,685.10267179)(445.00023375,685.02267187)(445.11023682,684.96267334)
\curveto(445.22023353,684.91267198)(445.36023339,684.86267203)(445.53023682,684.81267334)
\curveto(445.58023317,684.7926721)(445.63023312,684.78267211)(445.68023682,684.78267334)
\curveto(445.73023302,684.7926721)(445.78023297,684.7926721)(445.83023682,684.78267334)
\curveto(445.91023284,684.76267213)(445.99523275,684.75267214)(446.08523682,684.75267334)
\curveto(446.18523256,684.76267213)(446.27023248,684.77767211)(446.34023682,684.79767334)
\curveto(446.39023236,684.80767208)(446.43523231,684.81267208)(446.47523682,684.81267334)
\curveto(446.52523222,684.81267208)(446.57523217,684.82267207)(446.62523682,684.84267334)
\curveto(446.76523198,684.892672)(446.89023186,684.95267194)(447.00023682,685.02267334)
\curveto(447.12023163,685.0926718)(447.21523153,685.18267171)(447.28523682,685.29267334)
\curveto(447.33523141,685.37267152)(447.37523137,685.49767139)(447.40523682,685.66767334)
\curveto(447.42523132,685.73767115)(447.42523132,685.80267109)(447.40523682,685.86267334)
\curveto(447.38523136,685.92267097)(447.36523138,685.97267092)(447.34523682,686.01267334)
\curveto(447.27523147,686.15267074)(447.18523156,686.25767063)(447.07523682,686.32767334)
\curveto(446.97523177,686.39767049)(446.85523189,686.46267043)(446.71523682,686.52267334)
\curveto(446.52523222,686.60267029)(446.32523242,686.66767022)(446.11523682,686.71767334)
\curveto(445.90523284,686.76767012)(445.69523305,686.82267007)(445.48523682,686.88267334)
\curveto(445.40523334,686.90266999)(445.32023343,686.91766997)(445.23023682,686.92767334)
\curveto(445.1502336,686.93766995)(445.07023368,686.95266994)(444.99023682,686.97267334)
\curveto(444.67023408,687.06266983)(444.36523438,687.14766974)(444.07523682,687.22767334)
\curveto(443.78523496,687.31766957)(443.52023523,687.44766944)(443.28023682,687.61767334)
\curveto(443.00023575,687.81766907)(442.79523595,688.0876688)(442.66523682,688.42767334)
\curveto(442.6452361,688.49766839)(442.62523612,688.5926683)(442.60523682,688.71267334)
\curveto(442.58523616,688.78266811)(442.57023618,688.86766802)(442.56023682,688.96767334)
\curveto(442.5502362,689.06766782)(442.55523619,689.15766773)(442.57523682,689.23767334)
\curveto(442.59523615,689.2876676)(442.60023615,689.32766756)(442.59023682,689.35767334)
\curveto(442.58023617,689.39766749)(442.58523616,689.44266745)(442.60523682,689.49267334)
\curveto(442.62523612,689.60266729)(442.6452361,689.70266719)(442.66523682,689.79267334)
\curveto(442.69523605,689.892667)(442.73023602,689.9876669)(442.77023682,690.07767334)
\curveto(442.90023585,690.36766652)(443.08023567,690.60266629)(443.31023682,690.78267334)
\curveto(443.54023521,690.96266593)(443.80023495,691.10766578)(444.09023682,691.21767334)
\curveto(444.20023455,691.26766562)(444.31523443,691.30266559)(444.43523682,691.32267334)
\curveto(444.55523419,691.35266554)(444.68023407,691.38266551)(444.81023682,691.41267334)
\curveto(444.87023388,691.43266546)(444.93023382,691.44266545)(444.99023682,691.44267334)
\lineto(445.17023682,691.47267334)
\curveto(445.2502335,691.48266541)(445.33523341,691.4876654)(445.42523682,691.48767334)
\curveto(445.51523323,691.4876654)(445.60023315,691.4926654)(445.68023682,691.50267334)
}
}
{
\newrgbcolor{curcolor}{0 0 0}
\pscustom[linestyle=none,fillstyle=solid,fillcolor=curcolor]
{
}
}
{
\newrgbcolor{curcolor}{0 0 0}
\pscustom[linestyle=none,fillstyle=solid,fillcolor=curcolor]
{
\newpath
\moveto(462.48703369,681.70767334)
\curveto(462.48702535,681.54767534)(462.48202536,681.3926755)(462.47203369,681.24267334)
\curveto(462.47202537,681.08267581)(462.42202542,680.97267592)(462.32203369,680.91267334)
\curveto(462.2420256,680.86267603)(462.12702571,680.84267605)(461.97703369,680.85267334)
\lineto(461.55703369,680.85267334)
\lineto(461.24203369,680.85267334)
\curveto(461.13202671,680.84267605)(461.02202682,680.84267605)(460.91203369,680.85267334)
\curveto(460.81202703,680.85267604)(460.71702712,680.86767602)(460.62703369,680.89767334)
\curveto(460.54702729,680.91767597)(460.48702735,680.95767593)(460.44703369,681.01767334)
\curveto(460.39702744,681.09767579)(460.37202747,681.21267568)(460.37203369,681.36267334)
\curveto(460.38202746,681.50267539)(460.38702745,681.63267526)(460.38703369,681.75267334)
\lineto(460.38703369,683.38767334)
\lineto(460.38703369,683.76267334)
\curveto(460.38702745,683.90267299)(460.37202747,684.00767288)(460.34203369,684.07767334)
\curveto(460.32202752,684.09767279)(460.30202754,684.11267278)(460.28203369,684.12267334)
\curveto(460.27202757,684.14267275)(460.25702758,684.16267273)(460.23703369,684.18267334)
\curveto(460.14702769,684.1926727)(460.07702776,684.17267272)(460.02703369,684.12267334)
\curveto(459.97702786,684.08267281)(459.92202792,684.04267285)(459.86203369,684.00267334)
\curveto(459.77202807,683.93267296)(459.67702816,683.86767302)(459.57703369,683.80767334)
\curveto(459.48702835,683.74767314)(459.38702845,683.6926732)(459.27703369,683.64267334)
\curveto(459.09702874,683.56267333)(458.89702894,683.50267339)(458.67703369,683.46267334)
\curveto(458.45702938,683.41267348)(458.23202961,683.3876735)(458.00203369,683.38767334)
\curveto(457.77203007,683.37767351)(457.5420303,683.3926735)(457.31203369,683.43267334)
\curveto(457.09203075,683.47267342)(456.89203095,683.53267336)(456.71203369,683.61267334)
\curveto(456.26203158,683.81267308)(455.89703194,684.06767282)(455.61703369,684.37767334)
\curveto(455.3370325,684.69767219)(455.10203274,685.0876718)(454.91203369,685.54767334)
\curveto(454.86203298,685.65767123)(454.82703301,685.76767112)(454.80703369,685.87767334)
\curveto(454.78703305,685.99767089)(454.76203308,686.11267078)(454.73203369,686.22267334)
\curveto(454.71203313,686.26267063)(454.70203314,686.29767059)(454.70203369,686.32767334)
\curveto(454.71203313,686.36767052)(454.71203313,686.40767048)(454.70203369,686.44767334)
\curveto(454.68203316,686.52767036)(454.66703317,686.61267028)(454.65703369,686.70267334)
\curveto(454.65703318,686.80267009)(454.64703319,686.89766999)(454.62703369,686.98767334)
\lineto(454.62703369,687.18267334)
\curveto(454.61703322,687.23266966)(454.61203323,687.2926696)(454.61203369,687.36267334)
\curveto(454.61203323,687.44266945)(454.61703322,687.50766938)(454.62703369,687.55767334)
\curveto(454.6370332,687.60766928)(454.6420332,687.65266924)(454.64203369,687.69267334)
\lineto(454.64203369,687.82767334)
\curveto(454.65203319,687.87766901)(454.65203319,687.92766896)(454.64203369,687.97767334)
\curveto(454.6420332,688.02766886)(454.65203319,688.07766881)(454.67203369,688.12767334)
\curveto(454.69203315,688.21766867)(454.70703313,688.30766858)(454.71703369,688.39767334)
\curveto(454.72703311,688.49766839)(454.7420331,688.5926683)(454.76203369,688.68267334)
\curveto(454.81203303,688.85266804)(454.86203298,689.01266788)(454.91203369,689.16267334)
\curveto(454.97203287,689.31266758)(455.03203281,689.45766743)(455.09203369,689.59767334)
\curveto(455.15203269,689.73766715)(455.22703261,689.87266702)(455.31703369,690.00267334)
\curveto(455.40703243,690.13266676)(455.49703234,690.25766663)(455.58703369,690.37767334)
\curveto(455.67703216,690.4876664)(455.77703206,690.5876663)(455.88703369,690.67767334)
\curveto(455.91703192,690.70766618)(455.9370319,690.73266616)(455.94703369,690.75267334)
\curveto(455.99703184,690.78266611)(456.0420318,690.81266608)(456.08203369,690.84267334)
\curveto(456.12203172,690.88266601)(456.16203168,690.91766597)(456.20203369,690.94767334)
\curveto(456.3420315,691.04766584)(456.48703135,691.12766576)(456.63703369,691.18767334)
\curveto(456.79703104,691.25766563)(456.96203088,691.32266557)(457.13203369,691.38267334)
\curveto(457.22203062,691.41266548)(457.31203053,691.43266546)(457.40203369,691.44267334)
\curveto(457.49203035,691.45266544)(457.58203026,691.46766542)(457.67203369,691.48767334)
\curveto(457.70203014,691.49766539)(457.75703008,691.49766539)(457.83703369,691.48767334)
\curveto(457.91702992,691.47766541)(457.96702987,691.48266541)(457.98703369,691.50267334)
\curveto(458.30702953,691.51266538)(458.60702923,691.48266541)(458.88703369,691.41267334)
\curveto(459.16702867,691.35266554)(459.40702843,691.26266563)(459.60703369,691.14267334)
\lineto(459.78703369,691.02267334)
\curveto(459.84702799,690.98266591)(459.90202794,690.94266595)(459.95203369,690.90267334)
\curveto(460.01202783,690.85266604)(460.06202778,690.80266609)(460.10203369,690.75267334)
\curveto(460.15202769,690.71266618)(460.23202761,690.6926662)(460.34203369,690.69267334)
\lineto(460.38703369,690.73767334)
\lineto(460.44703369,690.79767334)
\curveto(460.47702736,690.87766601)(460.49702734,690.95266594)(460.50703369,691.02267334)
\curveto(460.51702732,691.10266579)(460.55702728,691.16766572)(460.62703369,691.21767334)
\curveto(460.67702716,691.25766563)(460.74702709,691.27766561)(460.83703369,691.27767334)
\curveto(460.9370269,691.2876656)(461.0370268,691.2926656)(461.13703369,691.29267334)
\lineto(461.85703369,691.29267334)
\lineto(462.06703369,691.29267334)
\curveto(462.1370257,691.2926656)(462.20202564,691.28266561)(462.26203369,691.26267334)
\curveto(462.33202551,691.24266565)(462.38702545,691.19766569)(462.42703369,691.12767334)
\curveto(462.47702536,691.05766583)(462.49702534,690.96266593)(462.48703369,690.84267334)
\lineto(462.48703369,690.49767334)
\lineto(462.48703369,681.70767334)
\moveto(460.44703369,687.31767334)
\curveto(460.45702738,687.33766955)(460.45702738,687.36266953)(460.44703369,687.39267334)
\lineto(460.44703369,687.46767334)
\curveto(460.4370274,687.56766932)(460.43202741,687.66266923)(460.43203369,687.75267334)
\curveto(460.43202741,687.84266905)(460.42202742,687.92766896)(460.40203369,688.00767334)
\curveto(460.39202745,688.03766885)(460.38702745,688.06266883)(460.38703369,688.08267334)
\curveto(460.39702744,688.11266878)(460.39702744,688.14266875)(460.38703369,688.17267334)
\curveto(460.36702747,688.25266864)(460.34702749,688.32266857)(460.32703369,688.38267334)
\curveto(460.31702752,688.45266844)(460.30202754,688.52266837)(460.28203369,688.59267334)
\curveto(460.18202766,688.88266801)(460.04702779,689.13266776)(459.87703369,689.34267334)
\curveto(459.70702813,689.55266734)(459.48702835,689.71266718)(459.21703369,689.82267334)
\curveto(459.10702873,689.87266702)(458.98702885,689.89766699)(458.85703369,689.89767334)
\curveto(458.7370291,689.90766698)(458.60702923,689.91266698)(458.46703369,689.91267334)
\curveto(458.4370294,689.892667)(458.40202944,689.88266701)(458.36203369,689.88267334)
\curveto(458.32202952,689.892667)(458.28202956,689.892667)(458.24203369,689.88267334)
\lineto(458.06203369,689.82267334)
\curveto(458.00202984,689.81266708)(457.94702989,689.79766709)(457.89703369,689.77767334)
\curveto(457.60703023,689.64766724)(457.37703046,689.45766743)(457.20703369,689.20767334)
\curveto(457.04703079,688.95766793)(456.92203092,688.66766822)(456.83203369,688.33767334)
\curveto(456.81203103,688.25766863)(456.79703104,688.18266871)(456.78703369,688.11267334)
\curveto(456.78703105,688.05266884)(456.77703106,687.98266891)(456.75703369,687.90267334)
\curveto(456.75703108,687.83266906)(456.75203109,687.78266911)(456.74203369,687.75267334)
\curveto(456.73203111,687.70266919)(456.72203112,687.61266928)(456.71203369,687.48267334)
\curveto(456.71203113,687.36266953)(456.72203112,687.27766961)(456.74203369,687.22767334)
\lineto(456.74203369,687.09267334)
\curveto(456.75203109,687.05266984)(456.75703108,687.01266988)(456.75703369,686.97267334)
\curveto(456.75703108,686.93266996)(456.76203108,686.89766999)(456.77203369,686.86767334)
\lineto(456.77203369,686.79267334)
\curveto(456.78203106,686.76267013)(456.78703105,686.73767015)(456.78703369,686.71767334)
\curveto(456.80703103,686.63767025)(456.82203102,686.56267033)(456.83203369,686.49267334)
\curveto(456.842031,686.42267047)(456.86203098,686.35267054)(456.89203369,686.28267334)
\curveto(456.97203087,686.03267086)(457.07703076,685.81767107)(457.20703369,685.63767334)
\curveto(457.3370305,685.45767143)(457.50203034,685.30267159)(457.70203369,685.17267334)
\curveto(457.84203,685.0926718)(457.99702984,685.03267186)(458.16703369,684.99267334)
\curveto(458.19702964,684.98267191)(458.22202962,684.97767191)(458.24203369,684.97767334)
\curveto(458.27202957,684.97767191)(458.30702953,684.97267192)(458.34703369,684.96267334)
\curveto(458.37702946,684.95267194)(458.42202942,684.94267195)(458.48203369,684.93267334)
\curveto(458.55202929,684.93267196)(458.61202923,684.93767195)(458.66203369,684.94767334)
\curveto(458.68202916,684.95767193)(458.70702913,684.95767193)(458.73703369,684.94767334)
\curveto(458.77702906,684.94767194)(458.81202903,684.95267194)(458.84203369,684.96267334)
\curveto(458.91202893,684.98267191)(458.97702886,684.99767189)(459.03703369,685.00767334)
\curveto(459.10702873,685.01767187)(459.17702866,685.03267186)(459.24703369,685.05267334)
\curveto(459.50702833,685.16267173)(459.71202813,685.30767158)(459.86203369,685.48767334)
\curveto(460.02202782,685.66767122)(460.15702768,685.887671)(460.26703369,686.14767334)
\curveto(460.29702754,686.22767066)(460.32202752,686.31267058)(460.34203369,686.40267334)
\lineto(460.40203369,686.67267334)
\lineto(460.40203369,686.77767334)
\curveto(460.41202743,686.80767008)(460.41702742,686.84267005)(460.41703369,686.88267334)
\curveto(460.4370274,686.98266991)(460.44702739,687.06766982)(460.44703369,687.13767334)
\lineto(460.44703369,687.31767334)
}
}
{
\newrgbcolor{curcolor}{0 0 0}
\pscustom[linestyle=none,fillstyle=solid,fillcolor=curcolor]
{
\newpath
\moveto(464.51695557,691.27767334)
\lineto(465.64195557,691.27767334)
\curveto(465.75195313,691.27766561)(465.85195303,691.27266562)(465.94195557,691.26267334)
\curveto(466.03195285,691.25266564)(466.09695279,691.21766567)(466.13695557,691.15767334)
\curveto(466.1869527,691.09766579)(466.21695267,691.01266588)(466.22695557,690.90267334)
\curveto(466.23695265,690.80266609)(466.24195264,690.69766619)(466.24195557,690.58767334)
\lineto(466.24195557,689.53767334)
\lineto(466.24195557,687.30267334)
\curveto(466.24195264,686.94266995)(466.25695263,686.60267029)(466.28695557,686.28267334)
\curveto(466.31695257,685.96267093)(466.40695248,685.69767119)(466.55695557,685.48767334)
\curveto(466.69695219,685.27767161)(466.92195196,685.12767176)(467.23195557,685.03767334)
\curveto(467.2819516,685.02767186)(467.32195156,685.02267187)(467.35195557,685.02267334)
\curveto(467.39195149,685.02267187)(467.43695145,685.01767187)(467.48695557,685.00767334)
\curveto(467.53695135,684.99767189)(467.59195129,684.9926719)(467.65195557,684.99267334)
\curveto(467.71195117,684.9926719)(467.75695113,684.99767189)(467.78695557,685.00767334)
\curveto(467.83695105,685.02767186)(467.87695101,685.03267186)(467.90695557,685.02267334)
\curveto(467.94695094,685.01267188)(467.9869509,685.01767187)(468.02695557,685.03767334)
\curveto(468.23695065,685.0876718)(468.40195048,685.15267174)(468.52195557,685.23267334)
\curveto(468.70195018,685.34267155)(468.84195004,685.48267141)(468.94195557,685.65267334)
\curveto(469.05194983,685.83267106)(469.12694976,686.02767086)(469.16695557,686.23767334)
\curveto(469.21694967,686.45767043)(469.24694964,686.69767019)(469.25695557,686.95767334)
\curveto(469.26694962,687.22766966)(469.27194961,687.50766938)(469.27195557,687.79767334)
\lineto(469.27195557,689.61267334)
\lineto(469.27195557,690.58767334)
\lineto(469.27195557,690.85767334)
\curveto(469.27194961,690.95766593)(469.29194959,691.03766585)(469.33195557,691.09767334)
\curveto(469.3819495,691.1876657)(469.45694943,691.23766565)(469.55695557,691.24767334)
\curveto(469.65694923,691.26766562)(469.77694911,691.27766561)(469.91695557,691.27767334)
\lineto(470.71195557,691.27767334)
\lineto(470.99695557,691.27767334)
\curveto(471.0869478,691.27766561)(471.16194772,691.25766563)(471.22195557,691.21767334)
\curveto(471.30194758,691.16766572)(471.34694754,691.0926658)(471.35695557,690.99267334)
\curveto(471.36694752,690.892666)(471.37194751,690.77766611)(471.37195557,690.64767334)
\lineto(471.37195557,689.50767334)
\lineto(471.37195557,685.29267334)
\lineto(471.37195557,684.22767334)
\lineto(471.37195557,683.92767334)
\curveto(471.37194751,683.82767306)(471.35194753,683.75267314)(471.31195557,683.70267334)
\curveto(471.26194762,683.62267327)(471.1869477,683.57767331)(471.08695557,683.56767334)
\curveto(470.9869479,683.55767333)(470.881948,683.55267334)(470.77195557,683.55267334)
\lineto(469.96195557,683.55267334)
\curveto(469.85194903,683.55267334)(469.75194913,683.55767333)(469.66195557,683.56767334)
\curveto(469.5819493,683.57767331)(469.51694937,683.61767327)(469.46695557,683.68767334)
\curveto(469.44694944,683.71767317)(469.42694946,683.76267313)(469.40695557,683.82267334)
\curveto(469.39694949,683.88267301)(469.3819495,683.94267295)(469.36195557,684.00267334)
\curveto(469.35194953,684.06267283)(469.33694955,684.11767277)(469.31695557,684.16767334)
\curveto(469.29694959,684.21767267)(469.26694962,684.24767264)(469.22695557,684.25767334)
\curveto(469.20694968,684.27767261)(469.1819497,684.28267261)(469.15195557,684.27267334)
\curveto(469.12194976,684.26267263)(469.09694979,684.25267264)(469.07695557,684.24267334)
\curveto(469.00694988,684.20267269)(468.94694994,684.15767273)(468.89695557,684.10767334)
\curveto(468.84695004,684.05767283)(468.79195009,684.01267288)(468.73195557,683.97267334)
\curveto(468.69195019,683.94267295)(468.65195023,683.90767298)(468.61195557,683.86767334)
\curveto(468.5819503,683.83767305)(468.54195034,683.80767308)(468.49195557,683.77767334)
\curveto(468.26195062,683.63767325)(467.99195089,683.52767336)(467.68195557,683.44767334)
\curveto(467.61195127,683.42767346)(467.54195134,683.41767347)(467.47195557,683.41767334)
\curveto(467.40195148,683.40767348)(467.32695156,683.3926735)(467.24695557,683.37267334)
\curveto(467.20695168,683.36267353)(467.16195172,683.36267353)(467.11195557,683.37267334)
\curveto(467.07195181,683.37267352)(467.03195185,683.36767352)(466.99195557,683.35767334)
\curveto(466.96195192,683.34767354)(466.89695199,683.34767354)(466.79695557,683.35767334)
\curveto(466.70695218,683.35767353)(466.64695224,683.36267353)(466.61695557,683.37267334)
\curveto(466.56695232,683.37267352)(466.51695237,683.37767351)(466.46695557,683.38767334)
\lineto(466.31695557,683.38767334)
\curveto(466.19695269,683.41767347)(466.0819528,683.44267345)(465.97195557,683.46267334)
\curveto(465.86195302,683.48267341)(465.75195313,683.51267338)(465.64195557,683.55267334)
\curveto(465.59195329,683.57267332)(465.54695334,683.5876733)(465.50695557,683.59767334)
\curveto(465.47695341,683.61767327)(465.43695345,683.63767325)(465.38695557,683.65767334)
\curveto(465.03695385,683.84767304)(464.75695413,684.11267278)(464.54695557,684.45267334)
\curveto(464.41695447,684.66267223)(464.32195456,684.91267198)(464.26195557,685.20267334)
\curveto(464.20195468,685.50267139)(464.16195472,685.81767107)(464.14195557,686.14767334)
\curveto(464.13195475,686.4876704)(464.12695476,686.83267006)(464.12695557,687.18267334)
\curveto(464.13695475,687.54266935)(464.14195474,687.89766899)(464.14195557,688.24767334)
\lineto(464.14195557,690.28767334)
\curveto(464.14195474,690.41766647)(464.13695475,690.56766632)(464.12695557,690.73767334)
\curveto(464.12695476,690.91766597)(464.15195473,691.04766584)(464.20195557,691.12767334)
\curveto(464.23195465,691.17766571)(464.29195459,691.22266567)(464.38195557,691.26267334)
\curveto(464.44195444,691.26266563)(464.4869544,691.26766562)(464.51695557,691.27767334)
}
}
{
\newrgbcolor{curcolor}{0 0 0}
\pscustom[linestyle=none,fillstyle=solid,fillcolor=curcolor]
{
\newpath
\moveto(480.36820557,687.49767334)
\curveto(480.3881974,687.41766947)(480.3881974,687.32766956)(480.36820557,687.22767334)
\curveto(480.34819744,687.12766976)(480.31319748,687.06266983)(480.26320557,687.03267334)
\curveto(480.21319758,686.9926699)(480.13819765,686.96266993)(480.03820557,686.94267334)
\curveto(479.94819784,686.93266996)(479.84319795,686.92266997)(479.72320557,686.91267334)
\lineto(479.37820557,686.91267334)
\curveto(479.26819852,686.92266997)(479.16819862,686.92766996)(479.07820557,686.92767334)
\lineto(475.41820557,686.92767334)
\lineto(475.20820557,686.92767334)
\curveto(475.14820264,686.92766996)(475.0932027,686.91766997)(475.04320557,686.89767334)
\curveto(474.96320283,686.85767003)(474.91320288,686.81767007)(474.89320557,686.77767334)
\curveto(474.87320292,686.75767013)(474.85320294,686.71767017)(474.83320557,686.65767334)
\curveto(474.81320298,686.60767028)(474.80820298,686.55767033)(474.81820557,686.50767334)
\curveto(474.83820295,686.44767044)(474.84820294,686.3876705)(474.84820557,686.32767334)
\curveto(474.85820293,686.27767061)(474.87320292,686.22267067)(474.89320557,686.16267334)
\curveto(474.97320282,685.92267097)(475.06820272,685.72267117)(475.17820557,685.56267334)
\curveto(475.29820249,685.41267148)(475.45820233,685.27767161)(475.65820557,685.15767334)
\curveto(475.73820205,685.10767178)(475.81820197,685.07267182)(475.89820557,685.05267334)
\curveto(475.9882018,685.04267185)(476.07820171,685.02267187)(476.16820557,684.99267334)
\curveto(476.24820154,684.97267192)(476.35820143,684.95767193)(476.49820557,684.94767334)
\curveto(476.63820115,684.93767195)(476.75820103,684.94267195)(476.85820557,684.96267334)
\lineto(476.99320557,684.96267334)
\curveto(477.0932007,684.98267191)(477.18320061,685.00267189)(477.26320557,685.02267334)
\curveto(477.35320044,685.05267184)(477.43820035,685.08267181)(477.51820557,685.11267334)
\curveto(477.61820017,685.16267173)(477.72820006,685.22767166)(477.84820557,685.30767334)
\curveto(477.97819981,685.3876715)(478.07319972,685.46767142)(478.13320557,685.54767334)
\curveto(478.18319961,685.61767127)(478.23319956,685.68267121)(478.28320557,685.74267334)
\curveto(478.34319945,685.81267108)(478.41319938,685.86267103)(478.49320557,685.89267334)
\curveto(478.5931992,685.94267095)(478.71819907,685.96267093)(478.86820557,685.95267334)
\lineto(479.30320557,685.95267334)
\lineto(479.48320557,685.95267334)
\curveto(479.55319824,685.96267093)(479.61319818,685.95767093)(479.66320557,685.93767334)
\lineto(479.81320557,685.93767334)
\curveto(479.91319788,685.91767097)(479.98319781,685.892671)(480.02320557,685.86267334)
\curveto(480.06319773,685.84267105)(480.08319771,685.79767109)(480.08320557,685.72767334)
\curveto(480.0931977,685.65767123)(480.0881977,685.59767129)(480.06820557,685.54767334)
\curveto(480.01819777,685.40767148)(479.96319783,685.28267161)(479.90320557,685.17267334)
\curveto(479.84319795,685.06267183)(479.77319802,684.95267194)(479.69320557,684.84267334)
\curveto(479.47319832,684.51267238)(479.22319857,684.24767264)(478.94320557,684.04767334)
\curveto(478.66319913,683.84767304)(478.31319948,683.67767321)(477.89320557,683.53767334)
\curveto(477.78320001,683.49767339)(477.67320012,683.47267342)(477.56320557,683.46267334)
\curveto(477.45320034,683.45267344)(477.33820045,683.43267346)(477.21820557,683.40267334)
\curveto(477.17820061,683.3926735)(477.13320066,683.3926735)(477.08320557,683.40267334)
\curveto(477.04320075,683.40267349)(477.00320079,683.39767349)(476.96320557,683.38767334)
\lineto(476.79820557,683.38767334)
\curveto(476.74820104,683.36767352)(476.6882011,683.36267353)(476.61820557,683.37267334)
\curveto(476.55820123,683.37267352)(476.50320129,683.37767351)(476.45320557,683.38767334)
\curveto(476.37320142,683.39767349)(476.30320149,683.39767349)(476.24320557,683.38767334)
\curveto(476.18320161,683.37767351)(476.11820167,683.38267351)(476.04820557,683.40267334)
\curveto(475.99820179,683.42267347)(475.94320185,683.43267346)(475.88320557,683.43267334)
\curveto(475.82320197,683.43267346)(475.76820202,683.44267345)(475.71820557,683.46267334)
\curveto(475.60820218,683.48267341)(475.49820229,683.50767338)(475.38820557,683.53767334)
\curveto(475.27820251,683.55767333)(475.17820261,683.5926733)(475.08820557,683.64267334)
\curveto(474.97820281,683.68267321)(474.87320292,683.71767317)(474.77320557,683.74767334)
\curveto(474.68320311,683.7876731)(474.59820319,683.83267306)(474.51820557,683.88267334)
\curveto(474.19820359,684.08267281)(473.91320388,684.31267258)(473.66320557,684.57267334)
\curveto(473.41320438,684.84267205)(473.20820458,685.15267174)(473.04820557,685.50267334)
\curveto(472.99820479,685.61267128)(472.95820483,685.72267117)(472.92820557,685.83267334)
\curveto(472.89820489,685.95267094)(472.85820493,686.07267082)(472.80820557,686.19267334)
\curveto(472.79820499,686.23267066)(472.793205,686.26767062)(472.79320557,686.29767334)
\curveto(472.793205,686.33767055)(472.788205,686.37767051)(472.77820557,686.41767334)
\curveto(472.73820505,686.53767035)(472.71320508,686.66767022)(472.70320557,686.80767334)
\lineto(472.67320557,687.22767334)
\curveto(472.67320512,687.27766961)(472.66820512,687.33266956)(472.65820557,687.39267334)
\curveto(472.65820513,687.45266944)(472.66320513,687.50766938)(472.67320557,687.55767334)
\lineto(472.67320557,687.73767334)
\lineto(472.71820557,688.09767334)
\curveto(472.75820503,688.26766862)(472.793205,688.43266846)(472.82320557,688.59267334)
\curveto(472.85320494,688.75266814)(472.89820489,688.90266799)(472.95820557,689.04267334)
\curveto(473.3882044,690.08266681)(474.11820367,690.81766607)(475.14820557,691.24767334)
\curveto(475.2882025,691.30766558)(475.42820236,691.34766554)(475.56820557,691.36767334)
\curveto(475.71820207,691.39766549)(475.87320192,691.43266546)(476.03320557,691.47267334)
\curveto(476.11320168,691.48266541)(476.1882016,691.4876654)(476.25820557,691.48767334)
\curveto(476.32820146,691.4876654)(476.40320139,691.4926654)(476.48320557,691.50267334)
\curveto(476.9932008,691.51266538)(477.42820036,691.45266544)(477.78820557,691.32267334)
\curveto(478.15819963,691.20266569)(478.4881993,691.04266585)(478.77820557,690.84267334)
\curveto(478.86819892,690.78266611)(478.95819883,690.71266618)(479.04820557,690.63267334)
\curveto(479.13819865,690.56266633)(479.21819857,690.4876664)(479.28820557,690.40767334)
\curveto(479.31819847,690.35766653)(479.35819843,690.31766657)(479.40820557,690.28767334)
\curveto(479.4881983,690.17766671)(479.56319823,690.06266683)(479.63320557,689.94267334)
\curveto(479.70319809,689.83266706)(479.77819801,689.71766717)(479.85820557,689.59767334)
\curveto(479.90819788,689.50766738)(479.94819784,689.41266748)(479.97820557,689.31267334)
\curveto(480.01819777,689.22266767)(480.05819773,689.12266777)(480.09820557,689.01267334)
\curveto(480.14819764,688.88266801)(480.1881976,688.74766814)(480.21820557,688.60767334)
\curveto(480.24819754,688.46766842)(480.28319751,688.32766856)(480.32320557,688.18767334)
\curveto(480.34319745,688.10766878)(480.34819744,688.01766887)(480.33820557,687.91767334)
\curveto(480.33819745,687.82766906)(480.34819744,687.74266915)(480.36820557,687.66267334)
\lineto(480.36820557,687.49767334)
\moveto(478.11820557,688.38267334)
\curveto(478.1881996,688.48266841)(478.1931996,688.60266829)(478.13320557,688.74267334)
\curveto(478.08319971,688.892668)(478.04319975,689.00266789)(478.01320557,689.07267334)
\curveto(477.87319992,689.34266755)(477.6882001,689.54766734)(477.45820557,689.68767334)
\curveto(477.22820056,689.83766705)(476.90820088,689.91766697)(476.49820557,689.92767334)
\curveto(476.46820132,689.90766698)(476.43320136,689.90266699)(476.39320557,689.91267334)
\curveto(476.35320144,689.92266697)(476.31820147,689.92266697)(476.28820557,689.91267334)
\curveto(476.23820155,689.892667)(476.18320161,689.87766701)(476.12320557,689.86767334)
\curveto(476.06320173,689.86766702)(476.00820178,689.85766703)(475.95820557,689.83767334)
\curveto(475.51820227,689.69766719)(475.1932026,689.42266747)(474.98320557,689.01267334)
\curveto(474.96320283,688.97266792)(474.93820285,688.91766797)(474.90820557,688.84767334)
\curveto(474.8882029,688.7876681)(474.87320292,688.72266817)(474.86320557,688.65267334)
\curveto(474.85320294,688.5926683)(474.85320294,688.53266836)(474.86320557,688.47267334)
\curveto(474.88320291,688.41266848)(474.91820287,688.36266853)(474.96820557,688.32267334)
\curveto(475.04820274,688.27266862)(475.15820263,688.24766864)(475.29820557,688.24767334)
\lineto(475.70320557,688.24767334)
\lineto(477.36820557,688.24767334)
\lineto(477.80320557,688.24767334)
\curveto(477.96319983,688.25766863)(478.06819972,688.30266859)(478.11820557,688.38267334)
}
}
{
\newrgbcolor{curcolor}{0 0 0}
\pscustom[linestyle=none,fillstyle=solid,fillcolor=curcolor]
{
}
}
{
\newrgbcolor{curcolor}{0 0 0}
\pscustom[linestyle=none,fillstyle=solid,fillcolor=curcolor]
{
\newpath
\moveto(493.14164307,687.49767334)
\curveto(493.1616349,687.41766947)(493.1616349,687.32766956)(493.14164307,687.22767334)
\curveto(493.12163494,687.12766976)(493.08663498,687.06266983)(493.03664307,687.03267334)
\curveto(492.98663508,686.9926699)(492.91163515,686.96266993)(492.81164307,686.94267334)
\curveto(492.72163534,686.93266996)(492.61663545,686.92266997)(492.49664307,686.91267334)
\lineto(492.15164307,686.91267334)
\curveto(492.04163602,686.92266997)(491.94163612,686.92766996)(491.85164307,686.92767334)
\lineto(488.19164307,686.92767334)
\lineto(487.98164307,686.92767334)
\curveto(487.92164014,686.92766996)(487.8666402,686.91766997)(487.81664307,686.89767334)
\curveto(487.73664033,686.85767003)(487.68664038,686.81767007)(487.66664307,686.77767334)
\curveto(487.64664042,686.75767013)(487.62664044,686.71767017)(487.60664307,686.65767334)
\curveto(487.58664048,686.60767028)(487.58164048,686.55767033)(487.59164307,686.50767334)
\curveto(487.61164045,686.44767044)(487.62164044,686.3876705)(487.62164307,686.32767334)
\curveto(487.63164043,686.27767061)(487.64664042,686.22267067)(487.66664307,686.16267334)
\curveto(487.74664032,685.92267097)(487.84164022,685.72267117)(487.95164307,685.56267334)
\curveto(488.07163999,685.41267148)(488.23163983,685.27767161)(488.43164307,685.15767334)
\curveto(488.51163955,685.10767178)(488.59163947,685.07267182)(488.67164307,685.05267334)
\curveto(488.7616393,685.04267185)(488.85163921,685.02267187)(488.94164307,684.99267334)
\curveto(489.02163904,684.97267192)(489.13163893,684.95767193)(489.27164307,684.94767334)
\curveto(489.41163865,684.93767195)(489.53163853,684.94267195)(489.63164307,684.96267334)
\lineto(489.76664307,684.96267334)
\curveto(489.8666382,684.98267191)(489.95663811,685.00267189)(490.03664307,685.02267334)
\curveto(490.12663794,685.05267184)(490.21163785,685.08267181)(490.29164307,685.11267334)
\curveto(490.39163767,685.16267173)(490.50163756,685.22767166)(490.62164307,685.30767334)
\curveto(490.75163731,685.3876715)(490.84663722,685.46767142)(490.90664307,685.54767334)
\curveto(490.95663711,685.61767127)(491.00663706,685.68267121)(491.05664307,685.74267334)
\curveto(491.11663695,685.81267108)(491.18663688,685.86267103)(491.26664307,685.89267334)
\curveto(491.3666367,685.94267095)(491.49163657,685.96267093)(491.64164307,685.95267334)
\lineto(492.07664307,685.95267334)
\lineto(492.25664307,685.95267334)
\curveto(492.32663574,685.96267093)(492.38663568,685.95767093)(492.43664307,685.93767334)
\lineto(492.58664307,685.93767334)
\curveto(492.68663538,685.91767097)(492.75663531,685.892671)(492.79664307,685.86267334)
\curveto(492.83663523,685.84267105)(492.85663521,685.79767109)(492.85664307,685.72767334)
\curveto(492.8666352,685.65767123)(492.8616352,685.59767129)(492.84164307,685.54767334)
\curveto(492.79163527,685.40767148)(492.73663533,685.28267161)(492.67664307,685.17267334)
\curveto(492.61663545,685.06267183)(492.54663552,684.95267194)(492.46664307,684.84267334)
\curveto(492.24663582,684.51267238)(491.99663607,684.24767264)(491.71664307,684.04767334)
\curveto(491.43663663,683.84767304)(491.08663698,683.67767321)(490.66664307,683.53767334)
\curveto(490.55663751,683.49767339)(490.44663762,683.47267342)(490.33664307,683.46267334)
\curveto(490.22663784,683.45267344)(490.11163795,683.43267346)(489.99164307,683.40267334)
\curveto(489.95163811,683.3926735)(489.90663816,683.3926735)(489.85664307,683.40267334)
\curveto(489.81663825,683.40267349)(489.77663829,683.39767349)(489.73664307,683.38767334)
\lineto(489.57164307,683.38767334)
\curveto(489.52163854,683.36767352)(489.4616386,683.36267353)(489.39164307,683.37267334)
\curveto(489.33163873,683.37267352)(489.27663879,683.37767351)(489.22664307,683.38767334)
\curveto(489.14663892,683.39767349)(489.07663899,683.39767349)(489.01664307,683.38767334)
\curveto(488.95663911,683.37767351)(488.89163917,683.38267351)(488.82164307,683.40267334)
\curveto(488.77163929,683.42267347)(488.71663935,683.43267346)(488.65664307,683.43267334)
\curveto(488.59663947,683.43267346)(488.54163952,683.44267345)(488.49164307,683.46267334)
\curveto(488.38163968,683.48267341)(488.27163979,683.50767338)(488.16164307,683.53767334)
\curveto(488.05164001,683.55767333)(487.95164011,683.5926733)(487.86164307,683.64267334)
\curveto(487.75164031,683.68267321)(487.64664042,683.71767317)(487.54664307,683.74767334)
\curveto(487.45664061,683.7876731)(487.37164069,683.83267306)(487.29164307,683.88267334)
\curveto(486.97164109,684.08267281)(486.68664138,684.31267258)(486.43664307,684.57267334)
\curveto(486.18664188,684.84267205)(485.98164208,685.15267174)(485.82164307,685.50267334)
\curveto(485.77164229,685.61267128)(485.73164233,685.72267117)(485.70164307,685.83267334)
\curveto(485.67164239,685.95267094)(485.63164243,686.07267082)(485.58164307,686.19267334)
\curveto(485.57164249,686.23267066)(485.5666425,686.26767062)(485.56664307,686.29767334)
\curveto(485.5666425,686.33767055)(485.5616425,686.37767051)(485.55164307,686.41767334)
\curveto(485.51164255,686.53767035)(485.48664258,686.66767022)(485.47664307,686.80767334)
\lineto(485.44664307,687.22767334)
\curveto(485.44664262,687.27766961)(485.44164262,687.33266956)(485.43164307,687.39267334)
\curveto(485.43164263,687.45266944)(485.43664263,687.50766938)(485.44664307,687.55767334)
\lineto(485.44664307,687.73767334)
\lineto(485.49164307,688.09767334)
\curveto(485.53164253,688.26766862)(485.5666425,688.43266846)(485.59664307,688.59267334)
\curveto(485.62664244,688.75266814)(485.67164239,688.90266799)(485.73164307,689.04267334)
\curveto(486.1616419,690.08266681)(486.89164117,690.81766607)(487.92164307,691.24767334)
\curveto(488.06164,691.30766558)(488.20163986,691.34766554)(488.34164307,691.36767334)
\curveto(488.49163957,691.39766549)(488.64663942,691.43266546)(488.80664307,691.47267334)
\curveto(488.88663918,691.48266541)(488.9616391,691.4876654)(489.03164307,691.48767334)
\curveto(489.10163896,691.4876654)(489.17663889,691.4926654)(489.25664307,691.50267334)
\curveto(489.7666383,691.51266538)(490.20163786,691.45266544)(490.56164307,691.32267334)
\curveto(490.93163713,691.20266569)(491.2616368,691.04266585)(491.55164307,690.84267334)
\curveto(491.64163642,690.78266611)(491.73163633,690.71266618)(491.82164307,690.63267334)
\curveto(491.91163615,690.56266633)(491.99163607,690.4876664)(492.06164307,690.40767334)
\curveto(492.09163597,690.35766653)(492.13163593,690.31766657)(492.18164307,690.28767334)
\curveto(492.2616358,690.17766671)(492.33663573,690.06266683)(492.40664307,689.94267334)
\curveto(492.47663559,689.83266706)(492.55163551,689.71766717)(492.63164307,689.59767334)
\curveto(492.68163538,689.50766738)(492.72163534,689.41266748)(492.75164307,689.31267334)
\curveto(492.79163527,689.22266767)(492.83163523,689.12266777)(492.87164307,689.01267334)
\curveto(492.92163514,688.88266801)(492.9616351,688.74766814)(492.99164307,688.60767334)
\curveto(493.02163504,688.46766842)(493.05663501,688.32766856)(493.09664307,688.18767334)
\curveto(493.11663495,688.10766878)(493.12163494,688.01766887)(493.11164307,687.91767334)
\curveto(493.11163495,687.82766906)(493.12163494,687.74266915)(493.14164307,687.66267334)
\lineto(493.14164307,687.49767334)
\moveto(490.89164307,688.38267334)
\curveto(490.9616371,688.48266841)(490.9666371,688.60266829)(490.90664307,688.74267334)
\curveto(490.85663721,688.892668)(490.81663725,689.00266789)(490.78664307,689.07267334)
\curveto(490.64663742,689.34266755)(490.4616376,689.54766734)(490.23164307,689.68767334)
\curveto(490.00163806,689.83766705)(489.68163838,689.91766697)(489.27164307,689.92767334)
\curveto(489.24163882,689.90766698)(489.20663886,689.90266699)(489.16664307,689.91267334)
\curveto(489.12663894,689.92266697)(489.09163897,689.92266697)(489.06164307,689.91267334)
\curveto(489.01163905,689.892667)(488.95663911,689.87766701)(488.89664307,689.86767334)
\curveto(488.83663923,689.86766702)(488.78163928,689.85766703)(488.73164307,689.83767334)
\curveto(488.29163977,689.69766719)(487.9666401,689.42266747)(487.75664307,689.01267334)
\curveto(487.73664033,688.97266792)(487.71164035,688.91766797)(487.68164307,688.84767334)
\curveto(487.6616404,688.7876681)(487.64664042,688.72266817)(487.63664307,688.65267334)
\curveto(487.62664044,688.5926683)(487.62664044,688.53266836)(487.63664307,688.47267334)
\curveto(487.65664041,688.41266848)(487.69164037,688.36266853)(487.74164307,688.32267334)
\curveto(487.82164024,688.27266862)(487.93164013,688.24766864)(488.07164307,688.24767334)
\lineto(488.47664307,688.24767334)
\lineto(490.14164307,688.24767334)
\lineto(490.57664307,688.24767334)
\curveto(490.73663733,688.25766863)(490.84163722,688.30266859)(490.89164307,688.38267334)
}
}
{
\newrgbcolor{curcolor}{0 0 0}
\pscustom[linestyle=none,fillstyle=solid,fillcolor=curcolor]
{
\newpath
\moveto(497.35992432,691.50267334)
\curveto(498.10991982,691.52266537)(498.75991917,691.43766545)(499.30992432,691.24767334)
\curveto(499.86991806,691.06766582)(500.29491763,690.75266614)(500.58492432,690.30267334)
\curveto(500.65491727,690.1926667)(500.71491721,690.07766681)(500.76492432,689.95767334)
\curveto(500.8249171,689.84766704)(500.87491705,689.72266717)(500.91492432,689.58267334)
\curveto(500.93491699,689.52266737)(500.94491698,689.45766743)(500.94492432,689.38767334)
\curveto(500.94491698,689.31766757)(500.93491699,689.25766763)(500.91492432,689.20767334)
\curveto(500.87491705,689.14766774)(500.81991711,689.10766778)(500.74992432,689.08767334)
\curveto(500.69991723,689.06766782)(500.63991729,689.05766783)(500.56992432,689.05767334)
\lineto(500.35992432,689.05767334)
\lineto(499.69992432,689.05767334)
\curveto(499.6299183,689.05766783)(499.55991837,689.05266784)(499.48992432,689.04267334)
\curveto(499.41991851,689.04266785)(499.35491857,689.05266784)(499.29492432,689.07267334)
\curveto(499.19491873,689.0926678)(499.11991881,689.13266776)(499.06992432,689.19267334)
\curveto(499.01991891,689.25266764)(498.97491895,689.31266758)(498.93492432,689.37267334)
\lineto(498.81492432,689.58267334)
\curveto(498.78491914,689.66266723)(498.73491919,689.72766716)(498.66492432,689.77767334)
\curveto(498.56491936,689.85766703)(498.46491946,689.91766697)(498.36492432,689.95767334)
\curveto(498.27491965,689.99766689)(498.15991977,690.03266686)(498.01992432,690.06267334)
\curveto(497.94991998,690.08266681)(497.84492008,690.09766679)(497.70492432,690.10767334)
\curveto(497.57492035,690.11766677)(497.47492045,690.11266678)(497.40492432,690.09267334)
\lineto(497.29992432,690.09267334)
\lineto(497.14992432,690.06267334)
\curveto(497.10992082,690.06266683)(497.06492086,690.05766683)(497.01492432,690.04767334)
\curveto(496.84492108,689.99766689)(496.70492122,689.92766696)(496.59492432,689.83767334)
\curveto(496.49492143,689.75766713)(496.4249215,689.63266726)(496.38492432,689.46267334)
\curveto(496.36492156,689.3926675)(496.36492156,689.32766756)(496.38492432,689.26767334)
\curveto(496.40492152,689.20766768)(496.4249215,689.15766773)(496.44492432,689.11767334)
\curveto(496.51492141,688.99766789)(496.59492133,688.90266799)(496.68492432,688.83267334)
\curveto(496.78492114,688.76266813)(496.89992103,688.70266819)(497.02992432,688.65267334)
\curveto(497.21992071,688.57266832)(497.4249205,688.50266839)(497.64492432,688.44267334)
\lineto(498.33492432,688.29267334)
\curveto(498.57491935,688.25266864)(498.80491912,688.20266869)(499.02492432,688.14267334)
\curveto(499.25491867,688.0926688)(499.46991846,688.02766886)(499.66992432,687.94767334)
\curveto(499.75991817,687.90766898)(499.84491808,687.87266902)(499.92492432,687.84267334)
\curveto(500.01491791,687.82266907)(500.09991783,687.7876691)(500.17992432,687.73767334)
\curveto(500.36991756,687.61766927)(500.53991739,687.4876694)(500.68992432,687.34767334)
\curveto(500.84991708,687.20766968)(500.97491695,687.03266986)(501.06492432,686.82267334)
\curveto(501.09491683,686.75267014)(501.11991681,686.68267021)(501.13992432,686.61267334)
\curveto(501.15991677,686.54267035)(501.17991675,686.46767042)(501.19992432,686.38767334)
\curveto(501.20991672,686.32767056)(501.21491671,686.23267066)(501.21492432,686.10267334)
\curveto(501.2249167,685.98267091)(501.2249167,685.887671)(501.21492432,685.81767334)
\lineto(501.21492432,685.74267334)
\curveto(501.19491673,685.68267121)(501.17991675,685.62267127)(501.16992432,685.56267334)
\curveto(501.16991676,685.51267138)(501.16491676,685.46267143)(501.15492432,685.41267334)
\curveto(501.08491684,685.11267178)(500.97491695,684.84767204)(500.82492432,684.61767334)
\curveto(500.66491726,684.37767251)(500.46991746,684.18267271)(500.23992432,684.03267334)
\curveto(500.00991792,683.88267301)(499.74991818,683.75267314)(499.45992432,683.64267334)
\curveto(499.34991858,683.5926733)(499.2299187,683.55767333)(499.09992432,683.53767334)
\curveto(498.97991895,683.51767337)(498.85991907,683.4926734)(498.73992432,683.46267334)
\curveto(498.64991928,683.44267345)(498.55491937,683.43267346)(498.45492432,683.43267334)
\curveto(498.36491956,683.42267347)(498.27491965,683.40767348)(498.18492432,683.38767334)
\lineto(497.91492432,683.38767334)
\curveto(497.85492007,683.36767352)(497.74992018,683.35767353)(497.59992432,683.35767334)
\curveto(497.45992047,683.35767353)(497.35992057,683.36767352)(497.29992432,683.38767334)
\curveto(497.26992066,683.3876735)(497.23492069,683.3926735)(497.19492432,683.40267334)
\lineto(497.08992432,683.40267334)
\curveto(496.96992096,683.42267347)(496.84992108,683.43767345)(496.72992432,683.44767334)
\curveto(496.60992132,683.45767343)(496.49492143,683.47767341)(496.38492432,683.50767334)
\curveto(495.99492193,683.61767327)(495.64992228,683.74267315)(495.34992432,683.88267334)
\curveto(495.04992288,684.03267286)(494.79492313,684.25267264)(494.58492432,684.54267334)
\curveto(494.44492348,684.73267216)(494.3249236,684.95267194)(494.22492432,685.20267334)
\curveto(494.20492372,685.26267163)(494.18492374,685.34267155)(494.16492432,685.44267334)
\curveto(494.14492378,685.4926714)(494.1299238,685.56267133)(494.11992432,685.65267334)
\curveto(494.10992382,685.74267115)(494.11492381,685.81767107)(494.13492432,685.87767334)
\curveto(494.16492376,685.94767094)(494.21492371,685.99767089)(494.28492432,686.02767334)
\curveto(494.33492359,686.04767084)(494.39492353,686.05767083)(494.46492432,686.05767334)
\lineto(494.68992432,686.05767334)
\lineto(495.39492432,686.05767334)
\lineto(495.63492432,686.05767334)
\curveto(495.71492221,686.05767083)(495.78492214,686.04767084)(495.84492432,686.02767334)
\curveto(495.95492197,685.9876709)(496.0249219,685.92267097)(496.05492432,685.83267334)
\curveto(496.09492183,685.74267115)(496.13992179,685.64767124)(496.18992432,685.54767334)
\curveto(496.20992172,685.49767139)(496.24492168,685.43267146)(496.29492432,685.35267334)
\curveto(496.35492157,685.27267162)(496.40492152,685.22267167)(496.44492432,685.20267334)
\curveto(496.56492136,685.10267179)(496.67992125,685.02267187)(496.78992432,684.96267334)
\curveto(496.89992103,684.91267198)(497.03992089,684.86267203)(497.20992432,684.81267334)
\curveto(497.25992067,684.7926721)(497.30992062,684.78267211)(497.35992432,684.78267334)
\curveto(497.40992052,684.7926721)(497.45992047,684.7926721)(497.50992432,684.78267334)
\curveto(497.58992034,684.76267213)(497.67492025,684.75267214)(497.76492432,684.75267334)
\curveto(497.86492006,684.76267213)(497.94991998,684.77767211)(498.01992432,684.79767334)
\curveto(498.06991986,684.80767208)(498.11491981,684.81267208)(498.15492432,684.81267334)
\curveto(498.20491972,684.81267208)(498.25491967,684.82267207)(498.30492432,684.84267334)
\curveto(498.44491948,684.892672)(498.56991936,684.95267194)(498.67992432,685.02267334)
\curveto(498.79991913,685.0926718)(498.89491903,685.18267171)(498.96492432,685.29267334)
\curveto(499.01491891,685.37267152)(499.05491887,685.49767139)(499.08492432,685.66767334)
\curveto(499.10491882,685.73767115)(499.10491882,685.80267109)(499.08492432,685.86267334)
\curveto(499.06491886,685.92267097)(499.04491888,685.97267092)(499.02492432,686.01267334)
\curveto(498.95491897,686.15267074)(498.86491906,686.25767063)(498.75492432,686.32767334)
\curveto(498.65491927,686.39767049)(498.53491939,686.46267043)(498.39492432,686.52267334)
\curveto(498.20491972,686.60267029)(498.00491992,686.66767022)(497.79492432,686.71767334)
\curveto(497.58492034,686.76767012)(497.37492055,686.82267007)(497.16492432,686.88267334)
\curveto(497.08492084,686.90266999)(496.99992093,686.91766997)(496.90992432,686.92767334)
\curveto(496.8299211,686.93766995)(496.74992118,686.95266994)(496.66992432,686.97267334)
\curveto(496.34992158,687.06266983)(496.04492188,687.14766974)(495.75492432,687.22767334)
\curveto(495.46492246,687.31766957)(495.19992273,687.44766944)(494.95992432,687.61767334)
\curveto(494.67992325,687.81766907)(494.47492345,688.0876688)(494.34492432,688.42767334)
\curveto(494.3249236,688.49766839)(494.30492362,688.5926683)(494.28492432,688.71267334)
\curveto(494.26492366,688.78266811)(494.24992368,688.86766802)(494.23992432,688.96767334)
\curveto(494.2299237,689.06766782)(494.23492369,689.15766773)(494.25492432,689.23767334)
\curveto(494.27492365,689.2876676)(494.27992365,689.32766756)(494.26992432,689.35767334)
\curveto(494.25992367,689.39766749)(494.26492366,689.44266745)(494.28492432,689.49267334)
\curveto(494.30492362,689.60266729)(494.3249236,689.70266719)(494.34492432,689.79267334)
\curveto(494.37492355,689.892667)(494.40992352,689.9876669)(494.44992432,690.07767334)
\curveto(494.57992335,690.36766652)(494.75992317,690.60266629)(494.98992432,690.78267334)
\curveto(495.21992271,690.96266593)(495.47992245,691.10766578)(495.76992432,691.21767334)
\curveto(495.87992205,691.26766562)(495.99492193,691.30266559)(496.11492432,691.32267334)
\curveto(496.23492169,691.35266554)(496.35992157,691.38266551)(496.48992432,691.41267334)
\curveto(496.54992138,691.43266546)(496.60992132,691.44266545)(496.66992432,691.44267334)
\lineto(496.84992432,691.47267334)
\curveto(496.929921,691.48266541)(497.01492091,691.4876654)(497.10492432,691.48767334)
\curveto(497.19492073,691.4876654)(497.27992065,691.4926654)(497.35992432,691.50267334)
}
}
{
\newrgbcolor{curcolor}{0 0 0}
\pscustom[linestyle=none,fillstyle=solid,fillcolor=curcolor]
{
\newpath
\moveto(503.49656494,693.60267334)
\lineto(504.50156494,693.60267334)
\curveto(504.65156196,693.60266329)(504.78156183,693.5926633)(504.89156494,693.57267334)
\curveto(505.0115616,693.56266333)(505.09656151,693.50266339)(505.14656494,693.39267334)
\curveto(505.16656144,693.34266355)(505.17656143,693.28266361)(505.17656494,693.21267334)
\lineto(505.17656494,693.00267334)
\lineto(505.17656494,692.32767334)
\curveto(505.17656143,692.27766461)(505.17156144,692.21766467)(505.16156494,692.14767334)
\curveto(505.16156145,692.0876648)(505.16656144,692.03266486)(505.17656494,691.98267334)
\lineto(505.17656494,691.81767334)
\curveto(505.17656143,691.73766515)(505.18156143,691.66266523)(505.19156494,691.59267334)
\curveto(505.20156141,691.53266536)(505.22656138,691.47766541)(505.26656494,691.42767334)
\curveto(505.33656127,691.33766555)(505.46156115,691.2876656)(505.64156494,691.27767334)
\lineto(506.18156494,691.27767334)
\lineto(506.36156494,691.27767334)
\curveto(506.42156019,691.27766561)(506.47656013,691.26766562)(506.52656494,691.24767334)
\curveto(506.63655997,691.19766569)(506.69655991,691.10766578)(506.70656494,690.97767334)
\curveto(506.72655988,690.84766604)(506.73655987,690.70266619)(506.73656494,690.54267334)
\lineto(506.73656494,690.33267334)
\curveto(506.74655986,690.26266663)(506.74155987,690.20266669)(506.72156494,690.15267334)
\curveto(506.67155994,689.9926669)(506.56656004,689.90766698)(506.40656494,689.89767334)
\curveto(506.24656036,689.887667)(506.06656054,689.88266701)(505.86656494,689.88267334)
\lineto(505.73156494,689.88267334)
\curveto(505.69156092,689.892667)(505.65656095,689.892667)(505.62656494,689.88267334)
\curveto(505.58656102,689.87266702)(505.55156106,689.86766702)(505.52156494,689.86767334)
\curveto(505.49156112,689.87766701)(505.46156115,689.87266702)(505.43156494,689.85267334)
\curveto(505.35156126,689.83266706)(505.29156132,689.7876671)(505.25156494,689.71767334)
\curveto(505.22156139,689.65766723)(505.19656141,689.58266731)(505.17656494,689.49267334)
\curveto(505.16656144,689.44266745)(505.16656144,689.3876675)(505.17656494,689.32767334)
\curveto(505.18656142,689.26766762)(505.18656142,689.21266768)(505.17656494,689.16267334)
\lineto(505.17656494,688.23267334)
\lineto(505.17656494,686.47767334)
\curveto(505.17656143,686.22767066)(505.18156143,686.00767088)(505.19156494,685.81767334)
\curveto(505.2115614,685.63767125)(505.27656133,685.47767141)(505.38656494,685.33767334)
\curveto(505.43656117,685.27767161)(505.50156111,685.23267166)(505.58156494,685.20267334)
\lineto(505.85156494,685.14267334)
\curveto(505.88156073,685.13267176)(505.9115607,685.12767176)(505.94156494,685.12767334)
\curveto(505.98156063,685.13767175)(506.0115606,685.13767175)(506.03156494,685.12767334)
\lineto(506.19656494,685.12767334)
\curveto(506.3065603,685.12767176)(506.40156021,685.12267177)(506.48156494,685.11267334)
\curveto(506.56156005,685.10267179)(506.62655998,685.06267183)(506.67656494,684.99267334)
\curveto(506.71655989,684.93267196)(506.73655987,684.85267204)(506.73656494,684.75267334)
\lineto(506.73656494,684.46767334)
\curveto(506.73655987,684.25767263)(506.73155988,684.06267283)(506.72156494,683.88267334)
\curveto(506.72155989,683.71267318)(506.64155997,683.59767329)(506.48156494,683.53767334)
\curveto(506.43156018,683.51767337)(506.38656022,683.51267338)(506.34656494,683.52267334)
\curveto(506.3065603,683.52267337)(506.26156035,683.51267338)(506.21156494,683.49267334)
\lineto(506.06156494,683.49267334)
\curveto(506.04156057,683.4926734)(506.0115606,683.49767339)(505.97156494,683.50767334)
\curveto(505.93156068,683.50767338)(505.89656071,683.50267339)(505.86656494,683.49267334)
\curveto(505.81656079,683.48267341)(505.76156085,683.48267341)(505.70156494,683.49267334)
\lineto(505.55156494,683.49267334)
\lineto(505.40156494,683.49267334)
\curveto(505.35156126,683.48267341)(505.3065613,683.48267341)(505.26656494,683.49267334)
\lineto(505.10156494,683.49267334)
\curveto(505.05156156,683.50267339)(504.99656161,683.50767338)(504.93656494,683.50767334)
\curveto(504.87656173,683.50767338)(504.82156179,683.51267338)(504.77156494,683.52267334)
\curveto(504.70156191,683.53267336)(504.63656197,683.54267335)(504.57656494,683.55267334)
\lineto(504.39656494,683.58267334)
\curveto(504.28656232,683.61267328)(504.18156243,683.64767324)(504.08156494,683.68767334)
\curveto(503.98156263,683.72767316)(503.88656272,683.77267312)(503.79656494,683.82267334)
\lineto(503.70656494,683.88267334)
\curveto(503.67656293,683.91267298)(503.64156297,683.94267295)(503.60156494,683.97267334)
\curveto(503.58156303,683.9926729)(503.55656305,684.01267288)(503.52656494,684.03267334)
\lineto(503.45156494,684.10767334)
\curveto(503.3115633,684.29767259)(503.2065634,684.50767238)(503.13656494,684.73767334)
\curveto(503.11656349,684.77767211)(503.1065635,684.81267208)(503.10656494,684.84267334)
\curveto(503.11656349,684.88267201)(503.11656349,684.92767196)(503.10656494,684.97767334)
\curveto(503.09656351,684.99767189)(503.09156352,685.02267187)(503.09156494,685.05267334)
\curveto(503.09156352,685.08267181)(503.08656352,685.10767178)(503.07656494,685.12767334)
\lineto(503.07656494,685.27767334)
\curveto(503.06656354,685.31767157)(503.06156355,685.36267153)(503.06156494,685.41267334)
\curveto(503.07156354,685.46267143)(503.07656353,685.51267138)(503.07656494,685.56267334)
\lineto(503.07656494,686.13267334)
\lineto(503.07656494,688.36767334)
\lineto(503.07656494,689.16267334)
\lineto(503.07656494,689.37267334)
\curveto(503.08656352,689.44266745)(503.08156353,689.50766738)(503.06156494,689.56767334)
\curveto(503.02156359,689.70766718)(502.95156366,689.79766709)(502.85156494,689.83767334)
\curveto(502.74156387,689.887667)(502.60156401,689.90266699)(502.43156494,689.88267334)
\curveto(502.26156435,689.86266703)(502.11656449,689.87766701)(501.99656494,689.92767334)
\curveto(501.91656469,689.95766693)(501.86656474,690.00266689)(501.84656494,690.06267334)
\curveto(501.82656478,690.12266677)(501.8065648,690.19766669)(501.78656494,690.28767334)
\lineto(501.78656494,690.60267334)
\curveto(501.78656482,690.78266611)(501.79656481,690.92766596)(501.81656494,691.03767334)
\curveto(501.83656477,691.14766574)(501.92156469,691.22266567)(502.07156494,691.26267334)
\curveto(502.1115645,691.28266561)(502.15156446,691.2876656)(502.19156494,691.27767334)
\lineto(502.32656494,691.27767334)
\curveto(502.47656413,691.27766561)(502.61656399,691.28266561)(502.74656494,691.29267334)
\curveto(502.87656373,691.31266558)(502.96656364,691.37266552)(503.01656494,691.47267334)
\curveto(503.04656356,691.54266535)(503.06156355,691.62266527)(503.06156494,691.71267334)
\curveto(503.07156354,691.80266509)(503.07656353,691.892665)(503.07656494,691.98267334)
\lineto(503.07656494,692.91267334)
\lineto(503.07656494,693.16767334)
\curveto(503.07656353,693.25766363)(503.08656352,693.33266356)(503.10656494,693.39267334)
\curveto(503.15656345,693.4926634)(503.23156338,693.55766333)(503.33156494,693.58767334)
\curveto(503.35156326,693.59766329)(503.37656323,693.59766329)(503.40656494,693.58767334)
\curveto(503.44656316,693.5876633)(503.47656313,693.5926633)(503.49656494,693.60267334)
}
}
{
\newrgbcolor{curcolor}{0 0 0}
\pscustom[linestyle=none,fillstyle=solid,fillcolor=curcolor]
{
\newpath
\moveto(514.77000244,684.15267334)
\curveto(514.78999459,684.04267285)(514.79999458,683.93267296)(514.80000244,683.82267334)
\curveto(514.80999457,683.71267318)(514.75999462,683.63767325)(514.65000244,683.59767334)
\curveto(514.58999479,683.56767332)(514.51999486,683.55267334)(514.44000244,683.55267334)
\lineto(514.20000244,683.55267334)
\lineto(513.39000244,683.55267334)
\lineto(513.12000244,683.55267334)
\curveto(513.03999634,683.56267333)(512.97499641,683.5876733)(512.92500244,683.62767334)
\curveto(512.85499653,683.66767322)(512.79999658,683.72267317)(512.76000244,683.79267334)
\curveto(512.72999665,683.87267302)(512.6849967,683.93767295)(512.62500244,683.98767334)
\curveto(512.60499678,684.00767288)(512.5799968,684.02267287)(512.55000244,684.03267334)
\curveto(512.51999686,684.05267284)(512.4799969,684.05767283)(512.43000244,684.04767334)
\curveto(512.379997,684.02767286)(512.32999705,684.00267289)(512.28000244,683.97267334)
\curveto(512.23999714,683.94267295)(512.19499719,683.91767297)(512.14500244,683.89767334)
\curveto(512.09499729,683.85767303)(512.03999734,683.82267307)(511.98000244,683.79267334)
\lineto(511.80000244,683.70267334)
\curveto(511.66999771,683.64267325)(511.53499785,683.5926733)(511.39500244,683.55267334)
\curveto(511.25499813,683.52267337)(511.10999827,683.4876734)(510.96000244,683.44767334)
\curveto(510.88999849,683.42767346)(510.81999856,683.41767347)(510.75000244,683.41767334)
\curveto(510.68999869,683.40767348)(510.62499876,683.39767349)(510.55500244,683.38767334)
\lineto(510.46500244,683.38767334)
\curveto(510.43499895,683.37767351)(510.40499898,683.37267352)(510.37500244,683.37267334)
\lineto(510.21000244,683.37267334)
\curveto(510.10999927,683.35267354)(510.00999937,683.35267354)(509.91000244,683.37267334)
\lineto(509.77500244,683.37267334)
\curveto(509.70499968,683.3926735)(509.63499975,683.40267349)(509.56500244,683.40267334)
\curveto(509.50499988,683.3926735)(509.44499994,683.39767349)(509.38500244,683.41767334)
\curveto(509.2850001,683.43767345)(509.19000019,683.45767343)(509.10000244,683.47767334)
\curveto(509.01000037,683.4876734)(508.92500046,683.51267338)(508.84500244,683.55267334)
\curveto(508.55500083,683.66267323)(508.30500108,683.80267309)(508.09500244,683.97267334)
\curveto(507.89500149,684.15267274)(507.73500165,684.3876725)(507.61500244,684.67767334)
\curveto(507.5850018,684.74767214)(507.55500183,684.82267207)(507.52500244,684.90267334)
\curveto(507.50500188,684.98267191)(507.4850019,685.06767182)(507.46500244,685.15767334)
\curveto(507.44500194,685.20767168)(507.43500195,685.25767163)(507.43500244,685.30767334)
\curveto(507.44500194,685.35767153)(507.44500194,685.40767148)(507.43500244,685.45767334)
\curveto(507.42500196,685.4876714)(507.41500197,685.54767134)(507.40500244,685.63767334)
\curveto(507.40500198,685.73767115)(507.41000197,685.80767108)(507.42000244,685.84767334)
\curveto(507.44000194,685.94767094)(507.45000193,686.03267086)(507.45000244,686.10267334)
\lineto(507.54000244,686.43267334)
\curveto(507.57000181,686.55267034)(507.61000177,686.65767023)(507.66000244,686.74767334)
\curveto(507.83000155,687.03766985)(508.02500136,687.25766963)(508.24500244,687.40767334)
\curveto(508.46500092,687.55766933)(508.74500064,687.6876692)(509.08500244,687.79767334)
\curveto(509.21500017,687.84766904)(509.35000003,687.88266901)(509.49000244,687.90267334)
\curveto(509.62999975,687.92266897)(509.76999961,687.94766894)(509.91000244,687.97767334)
\curveto(509.98999939,687.99766889)(510.07499931,688.00766888)(510.16500244,688.00767334)
\curveto(510.25499913,688.01766887)(510.34499904,688.03266886)(510.43500244,688.05267334)
\curveto(510.50499888,688.07266882)(510.57499881,688.07766881)(510.64500244,688.06767334)
\curveto(510.71499867,688.06766882)(510.78999859,688.07766881)(510.87000244,688.09767334)
\curveto(510.93999844,688.11766877)(511.00999837,688.12766876)(511.08000244,688.12767334)
\curveto(511.14999823,688.12766876)(511.22499816,688.13766875)(511.30500244,688.15767334)
\curveto(511.51499787,688.20766868)(511.70499768,688.24766864)(511.87500244,688.27767334)
\curveto(512.05499733,688.31766857)(512.21499717,688.40766848)(512.35500244,688.54767334)
\curveto(512.44499694,688.63766825)(512.50499688,688.73766815)(512.53500244,688.84767334)
\curveto(512.54499684,688.87766801)(512.54499684,688.90266799)(512.53500244,688.92267334)
\curveto(512.53499685,688.94266795)(512.53999684,688.96266793)(512.55000244,688.98267334)
\curveto(512.55999682,689.00266789)(512.56499682,689.03266786)(512.56500244,689.07267334)
\lineto(512.56500244,689.16267334)
\lineto(512.53500244,689.28267334)
\curveto(512.53499685,689.32266757)(512.52999685,689.35766753)(512.52000244,689.38767334)
\curveto(512.41999696,689.6876672)(512.20999717,689.892667)(511.89000244,690.00267334)
\curveto(511.79999758,690.03266686)(511.68999769,690.05266684)(511.56000244,690.06267334)
\curveto(511.43999794,690.08266681)(511.31499807,690.0876668)(511.18500244,690.07767334)
\curveto(511.05499833,690.07766681)(510.92999845,690.06766682)(510.81000244,690.04767334)
\curveto(510.68999869,690.02766686)(510.5849988,690.00266689)(510.49500244,689.97267334)
\curveto(510.43499895,689.95266694)(510.37499901,689.92266697)(510.31500244,689.88267334)
\curveto(510.26499912,689.85266704)(510.21499917,689.81766707)(510.16500244,689.77767334)
\curveto(510.11499927,689.73766715)(510.05999932,689.68266721)(510.00000244,689.61267334)
\curveto(509.94999943,689.54266735)(509.91499947,689.47766741)(509.89500244,689.41767334)
\curveto(509.84499954,689.31766757)(509.79999958,689.22266767)(509.76000244,689.13267334)
\curveto(509.72999965,689.04266785)(509.65999972,688.98266791)(509.55000244,688.95267334)
\curveto(509.46999991,688.93266796)(509.385,688.92266797)(509.29500244,688.92267334)
\lineto(509.02500244,688.92267334)
\lineto(508.45500244,688.92267334)
\curveto(508.40500098,688.92266797)(508.35500103,688.91766797)(508.30500244,688.90767334)
\curveto(508.25500113,688.90766798)(508.21000117,688.91266798)(508.17000244,688.92267334)
\lineto(508.03500244,688.92267334)
\curveto(508.01500137,688.93266796)(507.99000139,688.93766795)(507.96000244,688.93767334)
\curveto(507.93000145,688.93766795)(507.90500148,688.94766794)(507.88500244,688.96767334)
\curveto(507.80500158,688.9876679)(507.75000163,689.05266784)(507.72000244,689.16267334)
\curveto(507.71000167,689.21266768)(507.71000167,689.26266763)(507.72000244,689.31267334)
\curveto(507.73000165,689.36266753)(507.74000164,689.40766748)(507.75000244,689.44767334)
\curveto(507.7800016,689.55766733)(507.81000157,689.65766723)(507.84000244,689.74767334)
\curveto(507.8800015,689.84766704)(507.92500146,689.93766695)(507.97500244,690.01767334)
\lineto(508.06500244,690.16767334)
\lineto(508.15500244,690.31767334)
\curveto(508.23500115,690.42766646)(508.33500105,690.53266636)(508.45500244,690.63267334)
\curveto(508.47500091,690.64266625)(508.50500088,690.66766622)(508.54500244,690.70767334)
\curveto(508.59500079,690.74766614)(508.64000074,690.78266611)(508.68000244,690.81267334)
\curveto(508.72000066,690.84266605)(508.76500062,690.87266602)(508.81500244,690.90267334)
\curveto(508.9850004,691.01266588)(509.16500022,691.09766579)(509.35500244,691.15767334)
\curveto(509.54499984,691.22766566)(509.73999964,691.2926656)(509.94000244,691.35267334)
\curveto(510.05999932,691.38266551)(510.1849992,691.40266549)(510.31500244,691.41267334)
\curveto(510.44499894,691.42266547)(510.57499881,691.44266545)(510.70500244,691.47267334)
\curveto(510.74499864,691.48266541)(510.80499858,691.48266541)(510.88500244,691.47267334)
\curveto(510.97499841,691.46266543)(511.02999835,691.46766542)(511.05000244,691.48767334)
\curveto(511.45999792,691.49766539)(511.84999753,691.48266541)(512.22000244,691.44267334)
\curveto(512.59999678,691.40266549)(512.93999644,691.32766556)(513.24000244,691.21767334)
\curveto(513.54999583,691.10766578)(513.81499557,690.95766593)(514.03500244,690.76767334)
\curveto(514.25499513,690.5876663)(514.42499496,690.35266654)(514.54500244,690.06267334)
\curveto(514.61499477,689.892667)(514.65499473,689.69766719)(514.66500244,689.47767334)
\curveto(514.67499471,689.25766763)(514.6799947,689.03266786)(514.68000244,688.80267334)
\lineto(514.68000244,685.45767334)
\lineto(514.68000244,684.87267334)
\curveto(514.6799947,684.68267221)(514.69999468,684.50767238)(514.74000244,684.34767334)
\curveto(514.74999463,684.31767257)(514.75499463,684.28267261)(514.75500244,684.24267334)
\curveto(514.75499463,684.21267268)(514.75999462,684.18267271)(514.77000244,684.15267334)
\moveto(512.56500244,686.46267334)
\curveto(512.57499681,686.51267038)(512.5799968,686.56767032)(512.58000244,686.62767334)
\curveto(512.5799968,686.69767019)(512.57499681,686.75767013)(512.56500244,686.80767334)
\curveto(512.54499684,686.86767002)(512.53499685,686.92266997)(512.53500244,686.97267334)
\curveto(512.53499685,687.02266987)(512.51499687,687.06266983)(512.47500244,687.09267334)
\curveto(512.42499696,687.13266976)(512.34999703,687.15266974)(512.25000244,687.15267334)
\curveto(512.20999717,687.14266975)(512.17499721,687.13266976)(512.14500244,687.12267334)
\curveto(512.11499727,687.12266977)(512.0799973,687.11766977)(512.04000244,687.10767334)
\curveto(511.96999741,687.0876698)(511.89499749,687.07266982)(511.81500244,687.06267334)
\curveto(511.73499765,687.05266984)(511.65499773,687.03766985)(511.57500244,687.01767334)
\curveto(511.54499784,687.00766988)(511.49999788,687.00266989)(511.44000244,687.00267334)
\curveto(511.30999807,686.97266992)(511.1799982,686.95266994)(511.05000244,686.94267334)
\curveto(510.91999846,686.93266996)(510.79499859,686.90766998)(510.67500244,686.86767334)
\curveto(510.59499879,686.84767004)(510.51999886,686.82767006)(510.45000244,686.80767334)
\curveto(510.379999,686.79767009)(510.30999907,686.77767011)(510.24000244,686.74767334)
\curveto(510.02999935,686.65767023)(509.84999953,686.52267037)(509.70000244,686.34267334)
\curveto(509.55999982,686.16267073)(509.50999987,685.91267098)(509.55000244,685.59267334)
\curveto(509.56999981,685.42267147)(509.62499976,685.28267161)(509.71500244,685.17267334)
\curveto(509.7849996,685.06267183)(509.88999949,684.97267192)(510.03000244,684.90267334)
\curveto(510.16999921,684.84267205)(510.31999906,684.79767209)(510.48000244,684.76767334)
\curveto(510.64999873,684.73767215)(510.82499856,684.72767216)(511.00500244,684.73767334)
\curveto(511.19499819,684.75767213)(511.36999801,684.7926721)(511.53000244,684.84267334)
\curveto(511.78999759,684.92267197)(511.99499739,685.04767184)(512.14500244,685.21767334)
\curveto(512.29499709,685.39767149)(512.40999697,685.61767127)(512.49000244,685.87767334)
\curveto(512.50999687,685.94767094)(512.51999686,686.01767087)(512.52000244,686.08767334)
\curveto(512.52999685,686.16767072)(512.54499684,686.24767064)(512.56500244,686.32767334)
\lineto(512.56500244,686.46267334)
}
}
{
\newrgbcolor{curcolor}{0 0 0}
\pscustom[linestyle=none,fillstyle=solid,fillcolor=curcolor]
{
\newpath
\moveto(524.13328369,687.81267334)
\curveto(524.15327509,687.75266914)(524.16327508,687.64766924)(524.16328369,687.49767334)
\curveto(524.16327508,687.35766953)(524.15827509,687.25766963)(524.14828369,687.19767334)
\curveto(524.1482751,687.14766974)(524.1432751,687.10266979)(524.13328369,687.06267334)
\lineto(524.13328369,686.94267334)
\curveto(524.11327513,686.86267003)(524.10327514,686.78267011)(524.10328369,686.70267334)
\curveto(524.10327514,686.63267026)(524.09327515,686.55767033)(524.07328369,686.47767334)
\curveto(524.07327517,686.43767045)(524.06327518,686.36767052)(524.04328369,686.26767334)
\curveto(524.01327523,686.14767074)(523.98327526,686.02267087)(523.95328369,685.89267334)
\curveto(523.93327531,685.77267112)(523.89827535,685.65767123)(523.84828369,685.54767334)
\curveto(523.66827558,685.09767179)(523.4432758,684.70767218)(523.17328369,684.37767334)
\curveto(522.90327634,684.04767284)(522.5482767,683.7876731)(522.10828369,683.59767334)
\curveto(522.01827723,683.55767333)(521.92327732,683.52767336)(521.82328369,683.50767334)
\curveto(521.73327751,683.47767341)(521.63327761,683.44767344)(521.52328369,683.41767334)
\curveto(521.46327778,683.39767349)(521.39827785,683.3876735)(521.32828369,683.38767334)
\curveto(521.26827798,683.3876735)(521.20827804,683.38267351)(521.14828369,683.37267334)
\lineto(521.01328369,683.37267334)
\curveto(520.95327829,683.35267354)(520.87327837,683.34767354)(520.77328369,683.35767334)
\curveto(520.67327857,683.35767353)(520.59327865,683.36767352)(520.53328369,683.38767334)
\lineto(520.44328369,683.38767334)
\curveto(520.39327885,683.39767349)(520.33827891,683.40767348)(520.27828369,683.41767334)
\curveto(520.21827903,683.41767347)(520.15827909,683.42267347)(520.09828369,683.43267334)
\curveto(519.90827934,683.48267341)(519.73327951,683.53267336)(519.57328369,683.58267334)
\curveto(519.41327983,683.63267326)(519.26327998,683.70267319)(519.12328369,683.79267334)
\lineto(518.94328369,683.91267334)
\curveto(518.89328035,683.95267294)(518.8432804,683.99767289)(518.79328369,684.04767334)
\lineto(518.70328369,684.10767334)
\curveto(518.67328057,684.12767276)(518.6432806,684.14267275)(518.61328369,684.15267334)
\curveto(518.52328072,684.18267271)(518.46828078,684.16267273)(518.44828369,684.09267334)
\curveto(518.39828085,684.02267287)(518.36328088,683.93767295)(518.34328369,683.83767334)
\curveto(518.33328091,683.74767314)(518.29828095,683.67767321)(518.23828369,683.62767334)
\curveto(518.17828107,683.5876733)(518.10828114,683.56267333)(518.02828369,683.55267334)
\lineto(517.75828369,683.55267334)
\lineto(517.03828369,683.55267334)
\lineto(516.81328369,683.55267334)
\curveto(516.7432825,683.54267335)(516.67828257,683.54767334)(516.61828369,683.56767334)
\curveto(516.47828277,683.61767327)(516.39828285,683.70767318)(516.37828369,683.83767334)
\curveto(516.36828288,683.97767291)(516.36328288,684.13267276)(516.36328369,684.30267334)
\lineto(516.36328369,693.45267334)
\lineto(516.36328369,693.79767334)
\curveto(516.36328288,693.91766297)(516.38828286,694.01266288)(516.43828369,694.08267334)
\curveto(516.47828277,694.15266274)(516.5482827,694.19766269)(516.64828369,694.21767334)
\curveto(516.66828258,694.22766266)(516.68828256,694.22766266)(516.70828369,694.21767334)
\curveto(516.73828251,694.21766267)(516.76328248,694.22266267)(516.78328369,694.23267334)
\lineto(517.72828369,694.23267334)
\curveto(517.90828134,694.23266266)(518.06328118,694.22266267)(518.19328369,694.20267334)
\curveto(518.32328092,694.1926627)(518.40828084,694.11766277)(518.44828369,693.97767334)
\curveto(518.47828077,693.87766301)(518.48828076,693.74266315)(518.47828369,693.57267334)
\curveto(518.46828078,693.41266348)(518.46328078,693.27266362)(518.46328369,693.15267334)
\lineto(518.46328369,691.51767334)
\lineto(518.46328369,691.18767334)
\curveto(518.46328078,691.07766581)(518.47328077,690.98266591)(518.49328369,690.90267334)
\curveto(518.50328074,690.85266604)(518.51328073,690.80766608)(518.52328369,690.76767334)
\curveto(518.53328071,690.73766615)(518.55828069,690.71766617)(518.59828369,690.70767334)
\curveto(518.61828063,690.6876662)(518.6432806,690.67766621)(518.67328369,690.67767334)
\curveto(518.71328053,690.67766621)(518.7432805,690.68266621)(518.76328369,690.69267334)
\curveto(518.83328041,690.73266616)(518.89828035,690.77266612)(518.95828369,690.81267334)
\curveto(519.01828023,690.86266603)(519.08328016,690.91266598)(519.15328369,690.96267334)
\curveto(519.28327996,691.05266584)(519.41827983,691.12766576)(519.55828369,691.18767334)
\curveto(519.69827955,691.25766563)(519.85327939,691.31766557)(520.02328369,691.36767334)
\curveto(520.10327914,691.39766549)(520.18327906,691.41266548)(520.26328369,691.41267334)
\curveto(520.3432789,691.42266547)(520.42327882,691.43766545)(520.50328369,691.45767334)
\curveto(520.57327867,691.47766541)(520.6482786,691.4876654)(520.72828369,691.48767334)
\lineto(520.96828369,691.48767334)
\lineto(521.11828369,691.48767334)
\curveto(521.1482781,691.47766541)(521.18327806,691.47266542)(521.22328369,691.47267334)
\curveto(521.26327798,691.48266541)(521.30327794,691.48266541)(521.34328369,691.47267334)
\curveto(521.45327779,691.44266545)(521.55327769,691.41766547)(521.64328369,691.39767334)
\curveto(521.7432775,691.3876655)(521.83827741,691.36266553)(521.92828369,691.32267334)
\curveto(522.38827686,691.13266576)(522.76327648,690.887666)(523.05328369,690.58767334)
\curveto(523.3432759,690.2876666)(523.58827566,689.91266698)(523.78828369,689.46267334)
\curveto(523.83827541,689.34266755)(523.87827537,689.21766767)(523.90828369,689.08767334)
\curveto(523.9482753,688.95766793)(523.98827526,688.82266807)(524.02828369,688.68267334)
\curveto(524.0482752,688.61266828)(524.05827519,688.54266835)(524.05828369,688.47267334)
\curveto(524.06827518,688.41266848)(524.08327516,688.34266855)(524.10328369,688.26267334)
\curveto(524.12327512,688.21266868)(524.12827512,688.15766873)(524.11828369,688.09767334)
\curveto(524.11827513,688.03766885)(524.12327512,687.97766891)(524.13328369,687.91767334)
\lineto(524.13328369,687.81267334)
\moveto(521.91328369,686.40267334)
\curveto(521.9432773,686.50267039)(521.96827728,686.62767026)(521.98828369,686.77767334)
\curveto(522.01827723,686.92766996)(522.03327721,687.07766981)(522.03328369,687.22767334)
\curveto(522.0432772,687.3876695)(522.0432772,687.54266935)(522.03328369,687.69267334)
\curveto(522.03327721,687.85266904)(522.01827723,687.9876689)(521.98828369,688.09767334)
\curveto(521.95827729,688.19766869)(521.93827731,688.2926686)(521.92828369,688.38267334)
\curveto(521.91827733,688.47266842)(521.89327735,688.55766833)(521.85328369,688.63767334)
\curveto(521.71327753,688.9876679)(521.51327773,689.28266761)(521.25328369,689.52267334)
\curveto(521.00327824,689.77266712)(520.63327861,689.89766699)(520.14328369,689.89767334)
\curveto(520.10327914,689.89766699)(520.06827918,689.892667)(520.03828369,689.88267334)
\lineto(519.93328369,689.88267334)
\curveto(519.86327938,689.86266703)(519.79827945,689.84266705)(519.73828369,689.82267334)
\curveto(519.67827957,689.81266708)(519.61827963,689.79766709)(519.55828369,689.77767334)
\curveto(519.26827998,689.64766724)(519.0482802,689.46266743)(518.89828369,689.22267334)
\curveto(518.7482805,688.9926679)(518.62328062,688.72766816)(518.52328369,688.42767334)
\curveto(518.49328075,688.34766854)(518.47328077,688.26266863)(518.46328369,688.17267334)
\curveto(518.46328078,688.0926688)(518.45328079,688.01266888)(518.43328369,687.93267334)
\curveto(518.42328082,687.90266899)(518.41828083,687.85266904)(518.41828369,687.78267334)
\curveto(518.40828084,687.74266915)(518.40328084,687.70266919)(518.40328369,687.66267334)
\curveto(518.41328083,687.62266927)(518.41328083,687.58266931)(518.40328369,687.54267334)
\curveto(518.38328086,687.46266943)(518.37828087,687.35266954)(518.38828369,687.21267334)
\curveto(518.39828085,687.07266982)(518.41328083,686.97266992)(518.43328369,686.91267334)
\curveto(518.45328079,686.82267007)(518.46328078,686.73767015)(518.46328369,686.65767334)
\curveto(518.47328077,686.57767031)(518.49328075,686.49767039)(518.52328369,686.41767334)
\curveto(518.61328063,686.13767075)(518.71828053,685.892671)(518.83828369,685.68267334)
\curveto(518.96828028,685.48267141)(519.1482801,685.31267158)(519.37828369,685.17267334)
\curveto(519.53827971,685.07267182)(519.70327954,685.00267189)(519.87328369,684.96267334)
\curveto(519.89327935,684.96267193)(519.91327933,684.95767193)(519.93328369,684.94767334)
\lineto(520.02328369,684.94767334)
\curveto(520.05327919,684.93767195)(520.10327914,684.92767196)(520.17328369,684.91767334)
\curveto(520.243279,684.91767197)(520.30327894,684.92267197)(520.35328369,684.93267334)
\curveto(520.45327879,684.95267194)(520.5432787,684.96767192)(520.62328369,684.97767334)
\curveto(520.71327853,684.99767189)(520.79827845,685.02267187)(520.87828369,685.05267334)
\curveto(521.15827809,685.18267171)(521.37327787,685.36267153)(521.52328369,685.59267334)
\curveto(521.68327756,685.82267107)(521.81327743,686.0926708)(521.91328369,686.40267334)
}
}
{
\newrgbcolor{curcolor}{0 0 0}
\pscustom[linestyle=none,fillstyle=solid,fillcolor=curcolor]
{
\newpath
\moveto(526.01320557,694.24767334)
\lineto(527.10820557,694.24767334)
\curveto(527.20820308,694.24766264)(527.30320299,694.24266265)(527.39320557,694.23267334)
\curveto(527.48320281,694.22266267)(527.55320274,694.1926627)(527.60320557,694.14267334)
\curveto(527.66320263,694.07266282)(527.6932026,693.97766291)(527.69320557,693.85767334)
\curveto(527.70320259,693.74766314)(527.70820258,693.63266326)(527.70820557,693.51267334)
\lineto(527.70820557,692.17767334)
\lineto(527.70820557,686.79267334)
\lineto(527.70820557,684.49767334)
\lineto(527.70820557,684.07767334)
\curveto(527.71820257,683.92767296)(527.69820259,683.81267308)(527.64820557,683.73267334)
\curveto(527.59820269,683.65267324)(527.50820278,683.59767329)(527.37820557,683.56767334)
\curveto(527.31820297,683.54767334)(527.24820304,683.54267335)(527.16820557,683.55267334)
\curveto(527.09820319,683.56267333)(527.02820326,683.56767332)(526.95820557,683.56767334)
\lineto(526.23820557,683.56767334)
\curveto(526.12820416,683.56767332)(526.02820426,683.57267332)(525.93820557,683.58267334)
\curveto(525.84820444,683.5926733)(525.77320452,683.62267327)(525.71320557,683.67267334)
\curveto(525.65320464,683.72267317)(525.61820467,683.79767309)(525.60820557,683.89767334)
\lineto(525.60820557,684.22767334)
\lineto(525.60820557,685.56267334)
\lineto(525.60820557,691.18767334)
\lineto(525.60820557,693.22767334)
\curveto(525.60820468,693.35766353)(525.60320469,693.51266338)(525.59320557,693.69267334)
\curveto(525.5932047,693.87266302)(525.61820467,694.00266289)(525.66820557,694.08267334)
\curveto(525.6882046,694.12266277)(525.71320458,694.15266274)(525.74320557,694.17267334)
\lineto(525.86320557,694.23267334)
\curveto(525.88320441,694.23266266)(525.90820438,694.23266266)(525.93820557,694.23267334)
\curveto(525.96820432,694.24266265)(525.9932043,694.24766264)(526.01320557,694.24767334)
}
}
{
\newrgbcolor{curcolor}{0 0 0}
\pscustom[linestyle=none,fillstyle=solid,fillcolor=curcolor]
{
\newpath
\moveto(536.73539307,687.49767334)
\curveto(536.7553849,687.41766947)(536.7553849,687.32766956)(536.73539307,687.22767334)
\curveto(536.71538494,687.12766976)(536.68038498,687.06266983)(536.63039307,687.03267334)
\curveto(536.58038508,686.9926699)(536.50538515,686.96266993)(536.40539307,686.94267334)
\curveto(536.31538534,686.93266996)(536.21038545,686.92266997)(536.09039307,686.91267334)
\lineto(535.74539307,686.91267334)
\curveto(535.63538602,686.92266997)(535.53538612,686.92766996)(535.44539307,686.92767334)
\lineto(531.78539307,686.92767334)
\lineto(531.57539307,686.92767334)
\curveto(531.51539014,686.92766996)(531.4603902,686.91766997)(531.41039307,686.89767334)
\curveto(531.33039033,686.85767003)(531.28039038,686.81767007)(531.26039307,686.77767334)
\curveto(531.24039042,686.75767013)(531.22039044,686.71767017)(531.20039307,686.65767334)
\curveto(531.18039048,686.60767028)(531.17539048,686.55767033)(531.18539307,686.50767334)
\curveto(531.20539045,686.44767044)(531.21539044,686.3876705)(531.21539307,686.32767334)
\curveto(531.22539043,686.27767061)(531.24039042,686.22267067)(531.26039307,686.16267334)
\curveto(531.34039032,685.92267097)(531.43539022,685.72267117)(531.54539307,685.56267334)
\curveto(531.66538999,685.41267148)(531.82538983,685.27767161)(532.02539307,685.15767334)
\curveto(532.10538955,685.10767178)(532.18538947,685.07267182)(532.26539307,685.05267334)
\curveto(532.3553893,685.04267185)(532.44538921,685.02267187)(532.53539307,684.99267334)
\curveto(532.61538904,684.97267192)(532.72538893,684.95767193)(532.86539307,684.94767334)
\curveto(533.00538865,684.93767195)(533.12538853,684.94267195)(533.22539307,684.96267334)
\lineto(533.36039307,684.96267334)
\curveto(533.4603882,684.98267191)(533.55038811,685.00267189)(533.63039307,685.02267334)
\curveto(533.72038794,685.05267184)(533.80538785,685.08267181)(533.88539307,685.11267334)
\curveto(533.98538767,685.16267173)(534.09538756,685.22767166)(534.21539307,685.30767334)
\curveto(534.34538731,685.3876715)(534.44038722,685.46767142)(534.50039307,685.54767334)
\curveto(534.55038711,685.61767127)(534.60038706,685.68267121)(534.65039307,685.74267334)
\curveto(534.71038695,685.81267108)(534.78038688,685.86267103)(534.86039307,685.89267334)
\curveto(534.9603867,685.94267095)(535.08538657,685.96267093)(535.23539307,685.95267334)
\lineto(535.67039307,685.95267334)
\lineto(535.85039307,685.95267334)
\curveto(535.92038574,685.96267093)(535.98038568,685.95767093)(536.03039307,685.93767334)
\lineto(536.18039307,685.93767334)
\curveto(536.28038538,685.91767097)(536.35038531,685.892671)(536.39039307,685.86267334)
\curveto(536.43038523,685.84267105)(536.45038521,685.79767109)(536.45039307,685.72767334)
\curveto(536.4603852,685.65767123)(536.4553852,685.59767129)(536.43539307,685.54767334)
\curveto(536.38538527,685.40767148)(536.33038533,685.28267161)(536.27039307,685.17267334)
\curveto(536.21038545,685.06267183)(536.14038552,684.95267194)(536.06039307,684.84267334)
\curveto(535.84038582,684.51267238)(535.59038607,684.24767264)(535.31039307,684.04767334)
\curveto(535.03038663,683.84767304)(534.68038698,683.67767321)(534.26039307,683.53767334)
\curveto(534.15038751,683.49767339)(534.04038762,683.47267342)(533.93039307,683.46267334)
\curveto(533.82038784,683.45267344)(533.70538795,683.43267346)(533.58539307,683.40267334)
\curveto(533.54538811,683.3926735)(533.50038816,683.3926735)(533.45039307,683.40267334)
\curveto(533.41038825,683.40267349)(533.37038829,683.39767349)(533.33039307,683.38767334)
\lineto(533.16539307,683.38767334)
\curveto(533.11538854,683.36767352)(533.0553886,683.36267353)(532.98539307,683.37267334)
\curveto(532.92538873,683.37267352)(532.87038879,683.37767351)(532.82039307,683.38767334)
\curveto(532.74038892,683.39767349)(532.67038899,683.39767349)(532.61039307,683.38767334)
\curveto(532.55038911,683.37767351)(532.48538917,683.38267351)(532.41539307,683.40267334)
\curveto(532.36538929,683.42267347)(532.31038935,683.43267346)(532.25039307,683.43267334)
\curveto(532.19038947,683.43267346)(532.13538952,683.44267345)(532.08539307,683.46267334)
\curveto(531.97538968,683.48267341)(531.86538979,683.50767338)(531.75539307,683.53767334)
\curveto(531.64539001,683.55767333)(531.54539011,683.5926733)(531.45539307,683.64267334)
\curveto(531.34539031,683.68267321)(531.24039042,683.71767317)(531.14039307,683.74767334)
\curveto(531.05039061,683.7876731)(530.96539069,683.83267306)(530.88539307,683.88267334)
\curveto(530.56539109,684.08267281)(530.28039138,684.31267258)(530.03039307,684.57267334)
\curveto(529.78039188,684.84267205)(529.57539208,685.15267174)(529.41539307,685.50267334)
\curveto(529.36539229,685.61267128)(529.32539233,685.72267117)(529.29539307,685.83267334)
\curveto(529.26539239,685.95267094)(529.22539243,686.07267082)(529.17539307,686.19267334)
\curveto(529.16539249,686.23267066)(529.1603925,686.26767062)(529.16039307,686.29767334)
\curveto(529.1603925,686.33767055)(529.1553925,686.37767051)(529.14539307,686.41767334)
\curveto(529.10539255,686.53767035)(529.08039258,686.66767022)(529.07039307,686.80767334)
\lineto(529.04039307,687.22767334)
\curveto(529.04039262,687.27766961)(529.03539262,687.33266956)(529.02539307,687.39267334)
\curveto(529.02539263,687.45266944)(529.03039263,687.50766938)(529.04039307,687.55767334)
\lineto(529.04039307,687.73767334)
\lineto(529.08539307,688.09767334)
\curveto(529.12539253,688.26766862)(529.1603925,688.43266846)(529.19039307,688.59267334)
\curveto(529.22039244,688.75266814)(529.26539239,688.90266799)(529.32539307,689.04267334)
\curveto(529.7553919,690.08266681)(530.48539117,690.81766607)(531.51539307,691.24767334)
\curveto(531.65539,691.30766558)(531.79538986,691.34766554)(531.93539307,691.36767334)
\curveto(532.08538957,691.39766549)(532.24038942,691.43266546)(532.40039307,691.47267334)
\curveto(532.48038918,691.48266541)(532.5553891,691.4876654)(532.62539307,691.48767334)
\curveto(532.69538896,691.4876654)(532.77038889,691.4926654)(532.85039307,691.50267334)
\curveto(533.3603883,691.51266538)(533.79538786,691.45266544)(534.15539307,691.32267334)
\curveto(534.52538713,691.20266569)(534.8553868,691.04266585)(535.14539307,690.84267334)
\curveto(535.23538642,690.78266611)(535.32538633,690.71266618)(535.41539307,690.63267334)
\curveto(535.50538615,690.56266633)(535.58538607,690.4876664)(535.65539307,690.40767334)
\curveto(535.68538597,690.35766653)(535.72538593,690.31766657)(535.77539307,690.28767334)
\curveto(535.8553858,690.17766671)(535.93038573,690.06266683)(536.00039307,689.94267334)
\curveto(536.07038559,689.83266706)(536.14538551,689.71766717)(536.22539307,689.59767334)
\curveto(536.27538538,689.50766738)(536.31538534,689.41266748)(536.34539307,689.31267334)
\curveto(536.38538527,689.22266767)(536.42538523,689.12266777)(536.46539307,689.01267334)
\curveto(536.51538514,688.88266801)(536.5553851,688.74766814)(536.58539307,688.60767334)
\curveto(536.61538504,688.46766842)(536.65038501,688.32766856)(536.69039307,688.18767334)
\curveto(536.71038495,688.10766878)(536.71538494,688.01766887)(536.70539307,687.91767334)
\curveto(536.70538495,687.82766906)(536.71538494,687.74266915)(536.73539307,687.66267334)
\lineto(536.73539307,687.49767334)
\moveto(534.48539307,688.38267334)
\curveto(534.5553871,688.48266841)(534.5603871,688.60266829)(534.50039307,688.74267334)
\curveto(534.45038721,688.892668)(534.41038725,689.00266789)(534.38039307,689.07267334)
\curveto(534.24038742,689.34266755)(534.0553876,689.54766734)(533.82539307,689.68767334)
\curveto(533.59538806,689.83766705)(533.27538838,689.91766697)(532.86539307,689.92767334)
\curveto(532.83538882,689.90766698)(532.80038886,689.90266699)(532.76039307,689.91267334)
\curveto(532.72038894,689.92266697)(532.68538897,689.92266697)(532.65539307,689.91267334)
\curveto(532.60538905,689.892667)(532.55038911,689.87766701)(532.49039307,689.86767334)
\curveto(532.43038923,689.86766702)(532.37538928,689.85766703)(532.32539307,689.83767334)
\curveto(531.88538977,689.69766719)(531.5603901,689.42266747)(531.35039307,689.01267334)
\curveto(531.33039033,688.97266792)(531.30539035,688.91766797)(531.27539307,688.84767334)
\curveto(531.2553904,688.7876681)(531.24039042,688.72266817)(531.23039307,688.65267334)
\curveto(531.22039044,688.5926683)(531.22039044,688.53266836)(531.23039307,688.47267334)
\curveto(531.25039041,688.41266848)(531.28539037,688.36266853)(531.33539307,688.32267334)
\curveto(531.41539024,688.27266862)(531.52539013,688.24766864)(531.66539307,688.24767334)
\lineto(532.07039307,688.24767334)
\lineto(533.73539307,688.24767334)
\lineto(534.17039307,688.24767334)
\curveto(534.33038733,688.25766863)(534.43538722,688.30266859)(534.48539307,688.38267334)
}
}
{
\newrgbcolor{curcolor}{0 0 0}
\pscustom[linestyle=none,fillstyle=solid,fillcolor=curcolor]
{
\newpath
\moveto(541.55367432,691.50267334)
\curveto(542.36366916,691.52266537)(543.03866848,691.40266549)(543.57867432,691.14267334)
\curveto(544.12866739,690.88266601)(544.56366696,690.51266638)(544.88367432,690.03267334)
\curveto(545.04366648,689.7926671)(545.16366636,689.51766737)(545.24367432,689.20767334)
\curveto(545.26366626,689.15766773)(545.27866624,689.0926678)(545.28867432,689.01267334)
\curveto(545.30866621,688.93266796)(545.30866621,688.86266803)(545.28867432,688.80267334)
\curveto(545.24866627,688.6926682)(545.17866634,688.62766826)(545.07867432,688.60767334)
\curveto(544.97866654,688.59766829)(544.85866666,688.5926683)(544.71867432,688.59267334)
\lineto(543.93867432,688.59267334)
\lineto(543.65367432,688.59267334)
\curveto(543.56366796,688.5926683)(543.48866803,688.61266828)(543.42867432,688.65267334)
\curveto(543.34866817,688.6926682)(543.29366823,688.75266814)(543.26367432,688.83267334)
\curveto(543.23366829,688.92266797)(543.19366833,689.01266788)(543.14367432,689.10267334)
\curveto(543.08366844,689.21266768)(543.0186685,689.31266758)(542.94867432,689.40267334)
\curveto(542.87866864,689.4926674)(542.79866872,689.57266732)(542.70867432,689.64267334)
\curveto(542.56866895,689.73266716)(542.41366911,689.80266709)(542.24367432,689.85267334)
\curveto(542.18366934,689.87266702)(542.1236694,689.88266701)(542.06367432,689.88267334)
\curveto(542.00366952,689.88266701)(541.94866957,689.892667)(541.89867432,689.91267334)
\lineto(541.74867432,689.91267334)
\curveto(541.54866997,689.91266698)(541.38867013,689.892667)(541.26867432,689.85267334)
\curveto(540.97867054,689.76266713)(540.74367078,689.62266727)(540.56367432,689.43267334)
\curveto(540.38367114,689.25266764)(540.23867128,689.03266786)(540.12867432,688.77267334)
\curveto(540.07867144,688.66266823)(540.03867148,688.54266835)(540.00867432,688.41267334)
\curveto(539.98867153,688.2926686)(539.96367156,688.16266873)(539.93367432,688.02267334)
\curveto(539.9236716,687.98266891)(539.9186716,687.94266895)(539.91867432,687.90267334)
\curveto(539.9186716,687.86266903)(539.91367161,687.82266907)(539.90367432,687.78267334)
\curveto(539.88367164,687.68266921)(539.87367165,687.54266935)(539.87367432,687.36267334)
\curveto(539.88367164,687.18266971)(539.89867162,687.04266985)(539.91867432,686.94267334)
\curveto(539.9186716,686.86267003)(539.9236716,686.80767008)(539.93367432,686.77767334)
\curveto(539.95367157,686.70767018)(539.96367156,686.63767025)(539.96367432,686.56767334)
\curveto(539.97367155,686.49767039)(539.98867153,686.42767046)(540.00867432,686.35767334)
\curveto(540.08867143,686.12767076)(540.18367134,685.91767097)(540.29367432,685.72767334)
\curveto(540.40367112,685.53767135)(540.54367098,685.37767151)(540.71367432,685.24767334)
\curveto(540.75367077,685.21767167)(540.81367071,685.18267171)(540.89367432,685.14267334)
\curveto(541.00367052,685.07267182)(541.11367041,685.02767186)(541.22367432,685.00767334)
\curveto(541.34367018,684.9876719)(541.48867003,684.96767192)(541.65867432,684.94767334)
\lineto(541.74867432,684.94767334)
\curveto(541.78866973,684.94767194)(541.8186697,684.95267194)(541.83867432,684.96267334)
\lineto(541.97367432,684.96267334)
\curveto(542.04366948,684.98267191)(542.10866941,684.99767189)(542.16867432,685.00767334)
\curveto(542.23866928,685.02767186)(542.30366922,685.04767184)(542.36367432,685.06767334)
\curveto(542.66366886,685.19767169)(542.89366863,685.3876715)(543.05367432,685.63767334)
\curveto(543.09366843,685.6876712)(543.12866839,685.74267115)(543.15867432,685.80267334)
\curveto(543.18866833,685.87267102)(543.21366831,685.93267096)(543.23367432,685.98267334)
\curveto(543.27366825,686.0926708)(543.30866821,686.1876707)(543.33867432,686.26767334)
\curveto(543.36866815,686.35767053)(543.43866808,686.42767046)(543.54867432,686.47767334)
\curveto(543.63866788,686.51767037)(543.78366774,686.53267036)(543.98367432,686.52267334)
\lineto(544.47867432,686.52267334)
\lineto(544.68867432,686.52267334)
\curveto(544.76866675,686.53267036)(544.83366669,686.52767036)(544.88367432,686.50767334)
\lineto(545.00367432,686.50767334)
\lineto(545.12367432,686.47767334)
\curveto(545.16366636,686.47767041)(545.19366633,686.46767042)(545.21367432,686.44767334)
\curveto(545.26366626,686.40767048)(545.29366623,686.34767054)(545.30367432,686.26767334)
\curveto(545.3236662,686.19767069)(545.3236662,686.12267077)(545.30367432,686.04267334)
\curveto(545.21366631,685.71267118)(545.10366642,685.41767147)(544.97367432,685.15767334)
\curveto(544.56366696,684.3876725)(543.90866761,683.85267304)(543.00867432,683.55267334)
\curveto(542.90866861,683.52267337)(542.80366872,683.50267339)(542.69367432,683.49267334)
\curveto(542.58366894,683.47267342)(542.47366905,683.44767344)(542.36367432,683.41767334)
\curveto(542.30366922,683.40767348)(542.24366928,683.40267349)(542.18367432,683.40267334)
\curveto(542.1236694,683.40267349)(542.06366946,683.39767349)(542.00367432,683.38767334)
\lineto(541.83867432,683.38767334)
\curveto(541.78866973,683.36767352)(541.71366981,683.36267353)(541.61367432,683.37267334)
\curveto(541.51367001,683.37267352)(541.43867008,683.37767351)(541.38867432,683.38767334)
\curveto(541.30867021,683.40767348)(541.23367029,683.41767347)(541.16367432,683.41767334)
\curveto(541.10367042,683.40767348)(541.03867048,683.41267348)(540.96867432,683.43267334)
\lineto(540.81867432,683.46267334)
\curveto(540.76867075,683.46267343)(540.7186708,683.46767342)(540.66867432,683.47767334)
\curveto(540.55867096,683.50767338)(540.45367107,683.53767335)(540.35367432,683.56767334)
\curveto(540.25367127,683.59767329)(540.15867136,683.63267326)(540.06867432,683.67267334)
\curveto(539.59867192,683.87267302)(539.20367232,684.12767276)(538.88367432,684.43767334)
\curveto(538.56367296,684.75767213)(538.30367322,685.15267174)(538.10367432,685.62267334)
\curveto(538.05367347,685.71267118)(538.01367351,685.80767108)(537.98367432,685.90767334)
\lineto(537.89367432,686.23767334)
\curveto(537.88367364,686.27767061)(537.87867364,686.31267058)(537.87867432,686.34267334)
\curveto(537.87867364,686.38267051)(537.86867365,686.42767046)(537.84867432,686.47767334)
\curveto(537.82867369,686.54767034)(537.8186737,686.61767027)(537.81867432,686.68767334)
\curveto(537.8186737,686.76767012)(537.80867371,686.84267005)(537.78867432,686.91267334)
\lineto(537.78867432,687.16767334)
\curveto(537.76867375,687.21766967)(537.75867376,687.27266962)(537.75867432,687.33267334)
\curveto(537.75867376,687.40266949)(537.76867375,687.46266943)(537.78867432,687.51267334)
\curveto(537.79867372,687.56266933)(537.79867372,687.60766928)(537.78867432,687.64767334)
\curveto(537.77867374,687.6876692)(537.77867374,687.72766916)(537.78867432,687.76767334)
\curveto(537.80867371,687.83766905)(537.81367371,687.90266899)(537.80367432,687.96267334)
\curveto(537.80367372,688.02266887)(537.81367371,688.08266881)(537.83367432,688.14267334)
\curveto(537.88367364,688.32266857)(537.9236736,688.4926684)(537.95367432,688.65267334)
\curveto(537.98367354,688.82266807)(538.02867349,688.9876679)(538.08867432,689.14767334)
\curveto(538.30867321,689.65766723)(538.58367294,690.08266681)(538.91367432,690.42267334)
\curveto(539.25367227,690.76266613)(539.68367184,691.03766585)(540.20367432,691.24767334)
\curveto(540.34367118,691.30766558)(540.48867103,691.34766554)(540.63867432,691.36767334)
\curveto(540.78867073,691.39766549)(540.94367058,691.43266546)(541.10367432,691.47267334)
\curveto(541.18367034,691.48266541)(541.25867026,691.4876654)(541.32867432,691.48767334)
\curveto(541.39867012,691.4876654)(541.47367005,691.4926654)(541.55367432,691.50267334)
}
}
{
\newrgbcolor{curcolor}{0 0 0}
\pscustom[linestyle=none,fillstyle=solid,fillcolor=curcolor]
{
\newpath
\moveto(548.69695557,694.14267334)
\curveto(548.76695262,694.06266283)(548.80195258,693.94266295)(548.80195557,693.78267334)
\lineto(548.80195557,693.31767334)
\lineto(548.80195557,692.91267334)
\curveto(548.80195258,692.77266412)(548.76695262,692.67766421)(548.69695557,692.62767334)
\curveto(548.63695275,692.57766431)(548.55695283,692.54766434)(548.45695557,692.53767334)
\curveto(548.36695302,692.52766436)(548.26695312,692.52266437)(548.15695557,692.52267334)
\lineto(547.31695557,692.52267334)
\curveto(547.20695418,692.52266437)(547.10695428,692.52766436)(547.01695557,692.53767334)
\curveto(546.93695445,692.54766434)(546.86695452,692.57766431)(546.80695557,692.62767334)
\curveto(546.76695462,692.65766423)(546.73695465,692.71266418)(546.71695557,692.79267334)
\curveto(546.70695468,692.88266401)(546.69695469,692.97766391)(546.68695557,693.07767334)
\lineto(546.68695557,693.40767334)
\curveto(546.69695469,693.51766337)(546.70195468,693.61266328)(546.70195557,693.69267334)
\lineto(546.70195557,693.90267334)
\curveto(546.71195467,693.97266292)(546.73195465,694.03266286)(546.76195557,694.08267334)
\curveto(546.7819546,694.12266277)(546.80695458,694.15266274)(546.83695557,694.17267334)
\lineto(546.95695557,694.23267334)
\curveto(546.97695441,694.23266266)(547.00195438,694.23266266)(547.03195557,694.23267334)
\curveto(547.06195432,694.24266265)(547.0869543,694.24766264)(547.10695557,694.24767334)
\lineto(548.20195557,694.24767334)
\curveto(548.30195308,694.24766264)(548.39695299,694.24266265)(548.48695557,694.23267334)
\curveto(548.57695281,694.22266267)(548.64695274,694.1926627)(548.69695557,694.14267334)
\moveto(548.80195557,684.37767334)
\curveto(548.80195258,684.17767271)(548.79695259,684.00767288)(548.78695557,683.86767334)
\curveto(548.77695261,683.72767316)(548.6869527,683.63267326)(548.51695557,683.58267334)
\curveto(548.45695293,683.56267333)(548.39195299,683.55267334)(548.32195557,683.55267334)
\curveto(548.25195313,683.56267333)(548.17695321,683.56767332)(548.09695557,683.56767334)
\lineto(547.25695557,683.56767334)
\curveto(547.16695422,683.56767332)(547.07695431,683.57267332)(546.98695557,683.58267334)
\curveto(546.90695448,683.5926733)(546.84695454,683.62267327)(546.80695557,683.67267334)
\curveto(546.74695464,683.74267315)(546.71195467,683.82767306)(546.70195557,683.92767334)
\lineto(546.70195557,684.27267334)
\lineto(546.70195557,690.60267334)
\lineto(546.70195557,690.90267334)
\curveto(546.70195468,691.00266589)(546.72195466,691.08266581)(546.76195557,691.14267334)
\curveto(546.82195456,691.21266568)(546.90695448,691.25766563)(547.01695557,691.27767334)
\curveto(547.03695435,691.2876656)(547.06195432,691.2876656)(547.09195557,691.27767334)
\curveto(547.13195425,691.27766561)(547.16195422,691.28266561)(547.18195557,691.29267334)
\lineto(547.93195557,691.29267334)
\lineto(548.12695557,691.29267334)
\curveto(548.20695318,691.30266559)(548.27195311,691.30266559)(548.32195557,691.29267334)
\lineto(548.44195557,691.29267334)
\curveto(548.50195288,691.27266562)(548.55695283,691.25766563)(548.60695557,691.24767334)
\curveto(548.65695273,691.23766565)(548.69695269,691.20766568)(548.72695557,691.15767334)
\curveto(548.76695262,691.10766578)(548.7869526,691.03766585)(548.78695557,690.94767334)
\curveto(548.79695259,690.85766603)(548.80195258,690.76266613)(548.80195557,690.66267334)
\lineto(548.80195557,684.37767334)
}
}
{
\newrgbcolor{curcolor}{0 0 0}
\pscustom[linestyle=none,fillstyle=solid,fillcolor=curcolor]
{
\newpath
\moveto(557.82914307,687.49767334)
\curveto(557.8491349,687.41766947)(557.8491349,687.32766956)(557.82914307,687.22767334)
\curveto(557.80913494,687.12766976)(557.77413498,687.06266983)(557.72414307,687.03267334)
\curveto(557.67413508,686.9926699)(557.59913515,686.96266993)(557.49914307,686.94267334)
\curveto(557.40913534,686.93266996)(557.30413545,686.92266997)(557.18414307,686.91267334)
\lineto(556.83914307,686.91267334)
\curveto(556.72913602,686.92266997)(556.62913612,686.92766996)(556.53914307,686.92767334)
\lineto(552.87914307,686.92767334)
\lineto(552.66914307,686.92767334)
\curveto(552.60914014,686.92766996)(552.5541402,686.91766997)(552.50414307,686.89767334)
\curveto(552.42414033,686.85767003)(552.37414038,686.81767007)(552.35414307,686.77767334)
\curveto(552.33414042,686.75767013)(552.31414044,686.71767017)(552.29414307,686.65767334)
\curveto(552.27414048,686.60767028)(552.26914048,686.55767033)(552.27914307,686.50767334)
\curveto(552.29914045,686.44767044)(552.30914044,686.3876705)(552.30914307,686.32767334)
\curveto(552.31914043,686.27767061)(552.33414042,686.22267067)(552.35414307,686.16267334)
\curveto(552.43414032,685.92267097)(552.52914022,685.72267117)(552.63914307,685.56267334)
\curveto(552.75913999,685.41267148)(552.91913983,685.27767161)(553.11914307,685.15767334)
\curveto(553.19913955,685.10767178)(553.27913947,685.07267182)(553.35914307,685.05267334)
\curveto(553.4491393,685.04267185)(553.53913921,685.02267187)(553.62914307,684.99267334)
\curveto(553.70913904,684.97267192)(553.81913893,684.95767193)(553.95914307,684.94767334)
\curveto(554.09913865,684.93767195)(554.21913853,684.94267195)(554.31914307,684.96267334)
\lineto(554.45414307,684.96267334)
\curveto(554.5541382,684.98267191)(554.64413811,685.00267189)(554.72414307,685.02267334)
\curveto(554.81413794,685.05267184)(554.89913785,685.08267181)(554.97914307,685.11267334)
\curveto(555.07913767,685.16267173)(555.18913756,685.22767166)(555.30914307,685.30767334)
\curveto(555.43913731,685.3876715)(555.53413722,685.46767142)(555.59414307,685.54767334)
\curveto(555.64413711,685.61767127)(555.69413706,685.68267121)(555.74414307,685.74267334)
\curveto(555.80413695,685.81267108)(555.87413688,685.86267103)(555.95414307,685.89267334)
\curveto(556.0541367,685.94267095)(556.17913657,685.96267093)(556.32914307,685.95267334)
\lineto(556.76414307,685.95267334)
\lineto(556.94414307,685.95267334)
\curveto(557.01413574,685.96267093)(557.07413568,685.95767093)(557.12414307,685.93767334)
\lineto(557.27414307,685.93767334)
\curveto(557.37413538,685.91767097)(557.44413531,685.892671)(557.48414307,685.86267334)
\curveto(557.52413523,685.84267105)(557.54413521,685.79767109)(557.54414307,685.72767334)
\curveto(557.5541352,685.65767123)(557.5491352,685.59767129)(557.52914307,685.54767334)
\curveto(557.47913527,685.40767148)(557.42413533,685.28267161)(557.36414307,685.17267334)
\curveto(557.30413545,685.06267183)(557.23413552,684.95267194)(557.15414307,684.84267334)
\curveto(556.93413582,684.51267238)(556.68413607,684.24767264)(556.40414307,684.04767334)
\curveto(556.12413663,683.84767304)(555.77413698,683.67767321)(555.35414307,683.53767334)
\curveto(555.24413751,683.49767339)(555.13413762,683.47267342)(555.02414307,683.46267334)
\curveto(554.91413784,683.45267344)(554.79913795,683.43267346)(554.67914307,683.40267334)
\curveto(554.63913811,683.3926735)(554.59413816,683.3926735)(554.54414307,683.40267334)
\curveto(554.50413825,683.40267349)(554.46413829,683.39767349)(554.42414307,683.38767334)
\lineto(554.25914307,683.38767334)
\curveto(554.20913854,683.36767352)(554.1491386,683.36267353)(554.07914307,683.37267334)
\curveto(554.01913873,683.37267352)(553.96413879,683.37767351)(553.91414307,683.38767334)
\curveto(553.83413892,683.39767349)(553.76413899,683.39767349)(553.70414307,683.38767334)
\curveto(553.64413911,683.37767351)(553.57913917,683.38267351)(553.50914307,683.40267334)
\curveto(553.45913929,683.42267347)(553.40413935,683.43267346)(553.34414307,683.43267334)
\curveto(553.28413947,683.43267346)(553.22913952,683.44267345)(553.17914307,683.46267334)
\curveto(553.06913968,683.48267341)(552.95913979,683.50767338)(552.84914307,683.53767334)
\curveto(552.73914001,683.55767333)(552.63914011,683.5926733)(552.54914307,683.64267334)
\curveto(552.43914031,683.68267321)(552.33414042,683.71767317)(552.23414307,683.74767334)
\curveto(552.14414061,683.7876731)(552.05914069,683.83267306)(551.97914307,683.88267334)
\curveto(551.65914109,684.08267281)(551.37414138,684.31267258)(551.12414307,684.57267334)
\curveto(550.87414188,684.84267205)(550.66914208,685.15267174)(550.50914307,685.50267334)
\curveto(550.45914229,685.61267128)(550.41914233,685.72267117)(550.38914307,685.83267334)
\curveto(550.35914239,685.95267094)(550.31914243,686.07267082)(550.26914307,686.19267334)
\curveto(550.25914249,686.23267066)(550.2541425,686.26767062)(550.25414307,686.29767334)
\curveto(550.2541425,686.33767055)(550.2491425,686.37767051)(550.23914307,686.41767334)
\curveto(550.19914255,686.53767035)(550.17414258,686.66767022)(550.16414307,686.80767334)
\lineto(550.13414307,687.22767334)
\curveto(550.13414262,687.27766961)(550.12914262,687.33266956)(550.11914307,687.39267334)
\curveto(550.11914263,687.45266944)(550.12414263,687.50766938)(550.13414307,687.55767334)
\lineto(550.13414307,687.73767334)
\lineto(550.17914307,688.09767334)
\curveto(550.21914253,688.26766862)(550.2541425,688.43266846)(550.28414307,688.59267334)
\curveto(550.31414244,688.75266814)(550.35914239,688.90266799)(550.41914307,689.04267334)
\curveto(550.8491419,690.08266681)(551.57914117,690.81766607)(552.60914307,691.24767334)
\curveto(552.74914,691.30766558)(552.88913986,691.34766554)(553.02914307,691.36767334)
\curveto(553.17913957,691.39766549)(553.33413942,691.43266546)(553.49414307,691.47267334)
\curveto(553.57413918,691.48266541)(553.6491391,691.4876654)(553.71914307,691.48767334)
\curveto(553.78913896,691.4876654)(553.86413889,691.4926654)(553.94414307,691.50267334)
\curveto(554.4541383,691.51266538)(554.88913786,691.45266544)(555.24914307,691.32267334)
\curveto(555.61913713,691.20266569)(555.9491368,691.04266585)(556.23914307,690.84267334)
\curveto(556.32913642,690.78266611)(556.41913633,690.71266618)(556.50914307,690.63267334)
\curveto(556.59913615,690.56266633)(556.67913607,690.4876664)(556.74914307,690.40767334)
\curveto(556.77913597,690.35766653)(556.81913593,690.31766657)(556.86914307,690.28767334)
\curveto(556.9491358,690.17766671)(557.02413573,690.06266683)(557.09414307,689.94267334)
\curveto(557.16413559,689.83266706)(557.23913551,689.71766717)(557.31914307,689.59767334)
\curveto(557.36913538,689.50766738)(557.40913534,689.41266748)(557.43914307,689.31267334)
\curveto(557.47913527,689.22266767)(557.51913523,689.12266777)(557.55914307,689.01267334)
\curveto(557.60913514,688.88266801)(557.6491351,688.74766814)(557.67914307,688.60767334)
\curveto(557.70913504,688.46766842)(557.74413501,688.32766856)(557.78414307,688.18767334)
\curveto(557.80413495,688.10766878)(557.80913494,688.01766887)(557.79914307,687.91767334)
\curveto(557.79913495,687.82766906)(557.80913494,687.74266915)(557.82914307,687.66267334)
\lineto(557.82914307,687.49767334)
\moveto(555.57914307,688.38267334)
\curveto(555.6491371,688.48266841)(555.6541371,688.60266829)(555.59414307,688.74267334)
\curveto(555.54413721,688.892668)(555.50413725,689.00266789)(555.47414307,689.07267334)
\curveto(555.33413742,689.34266755)(555.1491376,689.54766734)(554.91914307,689.68767334)
\curveto(554.68913806,689.83766705)(554.36913838,689.91766697)(553.95914307,689.92767334)
\curveto(553.92913882,689.90766698)(553.89413886,689.90266699)(553.85414307,689.91267334)
\curveto(553.81413894,689.92266697)(553.77913897,689.92266697)(553.74914307,689.91267334)
\curveto(553.69913905,689.892667)(553.64413911,689.87766701)(553.58414307,689.86767334)
\curveto(553.52413923,689.86766702)(553.46913928,689.85766703)(553.41914307,689.83767334)
\curveto(552.97913977,689.69766719)(552.6541401,689.42266747)(552.44414307,689.01267334)
\curveto(552.42414033,688.97266792)(552.39914035,688.91766797)(552.36914307,688.84767334)
\curveto(552.3491404,688.7876681)(552.33414042,688.72266817)(552.32414307,688.65267334)
\curveto(552.31414044,688.5926683)(552.31414044,688.53266836)(552.32414307,688.47267334)
\curveto(552.34414041,688.41266848)(552.37914037,688.36266853)(552.42914307,688.32267334)
\curveto(552.50914024,688.27266862)(552.61914013,688.24766864)(552.75914307,688.24767334)
\lineto(553.16414307,688.24767334)
\lineto(554.82914307,688.24767334)
\lineto(555.26414307,688.24767334)
\curveto(555.42413733,688.25766863)(555.52913722,688.30266859)(555.57914307,688.38267334)
}
}
{
\newrgbcolor{curcolor}{0 0 0}
\pscustom[linestyle=none,fillstyle=solid,fillcolor=curcolor]
{
\newpath
\moveto(563.50242432,691.48767334)
\curveto(563.612419,691.4876654)(563.70741891,691.47766541)(563.78742432,691.45767334)
\curveto(563.87741874,691.43766545)(563.94741867,691.3926655)(563.99742432,691.32267334)
\curveto(564.05741856,691.24266565)(564.08741853,691.10266579)(564.08742432,690.90267334)
\lineto(564.08742432,690.39267334)
\lineto(564.08742432,690.01767334)
\curveto(564.09741852,689.87766701)(564.08241853,689.76766712)(564.04242432,689.68767334)
\curveto(564.00241861,689.61766727)(563.94241867,689.57266732)(563.86242432,689.55267334)
\curveto(563.79241882,689.53266736)(563.70741891,689.52266737)(563.60742432,689.52267334)
\curveto(563.5174191,689.52266737)(563.4174192,689.52766736)(563.30742432,689.53767334)
\curveto(563.20741941,689.54766734)(563.1124195,689.54266735)(563.02242432,689.52267334)
\curveto(562.95241966,689.50266739)(562.88241973,689.4876674)(562.81242432,689.47767334)
\curveto(562.74241987,689.47766741)(562.67741994,689.46766742)(562.61742432,689.44767334)
\curveto(562.45742016,689.39766749)(562.29742032,689.32266757)(562.13742432,689.22267334)
\curveto(561.97742064,689.13266776)(561.85242076,689.02766786)(561.76242432,688.90767334)
\curveto(561.7124209,688.82766806)(561.65742096,688.74266815)(561.59742432,688.65267334)
\curveto(561.54742107,688.57266832)(561.49742112,688.4876684)(561.44742432,688.39767334)
\curveto(561.4174212,688.31766857)(561.38742123,688.23266866)(561.35742432,688.14267334)
\lineto(561.29742432,687.90267334)
\curveto(561.27742134,687.83266906)(561.26742135,687.75766913)(561.26742432,687.67767334)
\curveto(561.26742135,687.60766928)(561.25742136,687.53766935)(561.23742432,687.46767334)
\curveto(561.22742139,687.42766946)(561.22242139,687.3876695)(561.22242432,687.34767334)
\curveto(561.23242138,687.31766957)(561.23242138,687.2876696)(561.22242432,687.25767334)
\lineto(561.22242432,687.01767334)
\curveto(561.20242141,686.94766994)(561.19742142,686.86767002)(561.20742432,686.77767334)
\curveto(561.2174214,686.69767019)(561.22242139,686.61767027)(561.22242432,686.53767334)
\lineto(561.22242432,685.57767334)
\lineto(561.22242432,684.30267334)
\curveto(561.22242139,684.17267272)(561.2174214,684.05267284)(561.20742432,683.94267334)
\curveto(561.19742142,683.83267306)(561.16742145,683.74267315)(561.11742432,683.67267334)
\curveto(561.09742152,683.64267325)(561.06242155,683.61767327)(561.01242432,683.59767334)
\curveto(560.97242164,683.5876733)(560.92742169,683.57767331)(560.87742432,683.56767334)
\lineto(560.80242432,683.56767334)
\curveto(560.75242186,683.55767333)(560.69742192,683.55267334)(560.63742432,683.55267334)
\lineto(560.47242432,683.55267334)
\lineto(559.82742432,683.55267334)
\curveto(559.76742285,683.56267333)(559.70242291,683.56767332)(559.63242432,683.56767334)
\lineto(559.43742432,683.56767334)
\curveto(559.38742323,683.5876733)(559.33742328,683.60267329)(559.28742432,683.61267334)
\curveto(559.23742338,683.63267326)(559.20242341,683.66767322)(559.18242432,683.71767334)
\curveto(559.14242347,683.76767312)(559.1174235,683.83767305)(559.10742432,683.92767334)
\lineto(559.10742432,684.22767334)
\lineto(559.10742432,685.24767334)
\lineto(559.10742432,689.47767334)
\lineto(559.10742432,690.58767334)
\lineto(559.10742432,690.87267334)
\curveto(559.10742351,690.97266592)(559.12742349,691.05266584)(559.16742432,691.11267334)
\curveto(559.2174234,691.1926657)(559.29242332,691.24266565)(559.39242432,691.26267334)
\curveto(559.49242312,691.28266561)(559.612423,691.2926656)(559.75242432,691.29267334)
\lineto(560.51742432,691.29267334)
\curveto(560.63742198,691.2926656)(560.74242187,691.28266561)(560.83242432,691.26267334)
\curveto(560.92242169,691.25266564)(560.99242162,691.20766568)(561.04242432,691.12767334)
\curveto(561.07242154,691.07766581)(561.08742153,691.00766588)(561.08742432,690.91767334)
\lineto(561.11742432,690.64767334)
\curveto(561.12742149,690.56766632)(561.14242147,690.4926664)(561.16242432,690.42267334)
\curveto(561.19242142,690.35266654)(561.24242137,690.31766657)(561.31242432,690.31767334)
\curveto(561.33242128,690.33766655)(561.35242126,690.34766654)(561.37242432,690.34767334)
\curveto(561.39242122,690.34766654)(561.4124212,690.35766653)(561.43242432,690.37767334)
\curveto(561.49242112,690.42766646)(561.54242107,690.48266641)(561.58242432,690.54267334)
\curveto(561.63242098,690.61266628)(561.69242092,690.67266622)(561.76242432,690.72267334)
\curveto(561.80242081,690.75266614)(561.83742078,690.78266611)(561.86742432,690.81267334)
\curveto(561.89742072,690.85266604)(561.93242068,690.887666)(561.97242432,690.91767334)
\lineto(562.24242432,691.09767334)
\curveto(562.34242027,691.15766573)(562.44242017,691.21266568)(562.54242432,691.26267334)
\curveto(562.64241997,691.30266559)(562.74241987,691.33766555)(562.84242432,691.36767334)
\lineto(563.17242432,691.45767334)
\curveto(563.20241941,691.46766542)(563.25741936,691.46766542)(563.33742432,691.45767334)
\curveto(563.42741919,691.45766543)(563.48241913,691.46766542)(563.50242432,691.48767334)
}
}
{
\newrgbcolor{curcolor}{0 0 0}
\pscustom[linestyle=none,fillstyle=solid,fillcolor=curcolor]
{
\newpath
\moveto(572.41383057,687.73767334)
\curveto(572.433822,687.67766921)(572.44382199,687.5926693)(572.44383057,687.48267334)
\curveto(572.44382199,687.37266952)(572.433822,687.2876696)(572.41383057,687.22767334)
\lineto(572.41383057,687.07767334)
\curveto(572.39382204,686.99766989)(572.38382205,686.91766997)(572.38383057,686.83767334)
\curveto(572.39382204,686.75767013)(572.38882204,686.67767021)(572.36883057,686.59767334)
\curveto(572.34882208,686.52767036)(572.3338221,686.46267043)(572.32383057,686.40267334)
\curveto(572.31382212,686.34267055)(572.30382213,686.27767061)(572.29383057,686.20767334)
\curveto(572.25382218,686.09767079)(572.21882221,685.98267091)(572.18883057,685.86267334)
\curveto(572.15882227,685.75267114)(572.11882231,685.64767124)(572.06883057,685.54767334)
\curveto(571.85882257,685.06767182)(571.58382285,684.67767221)(571.24383057,684.37767334)
\curveto(570.90382353,684.07767281)(570.49382394,683.82767306)(570.01383057,683.62767334)
\curveto(569.89382454,683.57767331)(569.76882466,683.54267335)(569.63883057,683.52267334)
\curveto(569.51882491,683.4926734)(569.39382504,683.46267343)(569.26383057,683.43267334)
\curveto(569.21382522,683.41267348)(569.15882527,683.40267349)(569.09883057,683.40267334)
\curveto(569.03882539,683.40267349)(568.98382545,683.39767349)(568.93383057,683.38767334)
\lineto(568.82883057,683.38767334)
\curveto(568.79882563,683.37767351)(568.76882566,683.37267352)(568.73883057,683.37267334)
\curveto(568.68882574,683.36267353)(568.60882582,683.35767353)(568.49883057,683.35767334)
\curveto(568.38882604,683.34767354)(568.30382613,683.35267354)(568.24383057,683.37267334)
\lineto(568.09383057,683.37267334)
\curveto(568.04382639,683.38267351)(567.98882644,683.3876735)(567.92883057,683.38767334)
\curveto(567.87882655,683.37767351)(567.8288266,683.38267351)(567.77883057,683.40267334)
\curveto(567.73882669,683.41267348)(567.69882673,683.41767347)(567.65883057,683.41767334)
\curveto(567.6288268,683.41767347)(567.58882684,683.42267347)(567.53883057,683.43267334)
\curveto(567.43882699,683.46267343)(567.33882709,683.4876734)(567.23883057,683.50767334)
\curveto(567.13882729,683.52767336)(567.04382739,683.55767333)(566.95383057,683.59767334)
\curveto(566.8338276,683.63767325)(566.71882771,683.67767321)(566.60883057,683.71767334)
\curveto(566.50882792,683.75767313)(566.40382803,683.80767308)(566.29383057,683.86767334)
\curveto(565.94382849,684.07767281)(565.64382879,684.32267257)(565.39383057,684.60267334)
\curveto(565.14382929,684.88267201)(564.9338295,685.21767167)(564.76383057,685.60767334)
\curveto(564.71382972,685.69767119)(564.67382976,685.7926711)(564.64383057,685.89267334)
\curveto(564.62382981,685.9926709)(564.59882983,686.09767079)(564.56883057,686.20767334)
\curveto(564.54882988,686.25767063)(564.53882989,686.30267059)(564.53883057,686.34267334)
\curveto(564.53882989,686.38267051)(564.5288299,686.42767046)(564.50883057,686.47767334)
\curveto(564.48882994,686.55767033)(564.47882995,686.63767025)(564.47883057,686.71767334)
\curveto(564.47882995,686.80767008)(564.46882996,686.89267)(564.44883057,686.97267334)
\curveto(564.43882999,687.02266987)(564.43383,687.06766982)(564.43383057,687.10767334)
\lineto(564.43383057,687.24267334)
\curveto(564.41383002,687.30266959)(564.40383003,687.3876695)(564.40383057,687.49767334)
\curveto(564.41383002,687.60766928)(564.42883,687.6926692)(564.44883057,687.75267334)
\lineto(564.44883057,687.85767334)
\curveto(564.45882997,687.90766898)(564.45882997,687.95766893)(564.44883057,688.00767334)
\curveto(564.44882998,688.06766882)(564.45882997,688.12266877)(564.47883057,688.17267334)
\curveto(564.48882994,688.22266867)(564.49382994,688.26766862)(564.49383057,688.30767334)
\curveto(564.49382994,688.35766853)(564.50382993,688.40766848)(564.52383057,688.45767334)
\curveto(564.56382987,688.5876683)(564.59882983,688.71266818)(564.62883057,688.83267334)
\curveto(564.65882977,688.96266793)(564.69882973,689.0876678)(564.74883057,689.20767334)
\curveto(564.9288295,689.61766727)(565.14382929,689.95766693)(565.39383057,690.22767334)
\curveto(565.64382879,690.50766638)(565.94882848,690.76266613)(566.30883057,690.99267334)
\curveto(566.40882802,691.04266585)(566.51382792,691.0876658)(566.62383057,691.12767334)
\curveto(566.7338277,691.16766572)(566.84382759,691.21266568)(566.95383057,691.26267334)
\curveto(567.08382735,691.31266558)(567.21882721,691.34766554)(567.35883057,691.36767334)
\curveto(567.49882693,691.3876655)(567.64382679,691.41766547)(567.79383057,691.45767334)
\curveto(567.87382656,691.46766542)(567.94882648,691.47266542)(568.01883057,691.47267334)
\curveto(568.08882634,691.47266542)(568.15882627,691.47766541)(568.22883057,691.48767334)
\curveto(568.80882562,691.49766539)(569.30882512,691.43766545)(569.72883057,691.30767334)
\curveto(570.15882427,691.17766571)(570.53882389,690.99766589)(570.86883057,690.76767334)
\curveto(570.97882345,690.6876662)(571.08882334,690.59766629)(571.19883057,690.49767334)
\curveto(571.31882311,690.40766648)(571.41882301,690.30766658)(571.49883057,690.19767334)
\curveto(571.57882285,690.09766679)(571.64882278,689.99766689)(571.70883057,689.89767334)
\curveto(571.77882265,689.79766709)(571.84882258,689.6926672)(571.91883057,689.58267334)
\curveto(571.98882244,689.47266742)(572.04382239,689.35266754)(572.08383057,689.22267334)
\curveto(572.12382231,689.10266779)(572.16882226,688.97266792)(572.21883057,688.83267334)
\curveto(572.24882218,688.75266814)(572.27382216,688.66766822)(572.29383057,688.57767334)
\lineto(572.35383057,688.30767334)
\curveto(572.36382207,688.26766862)(572.36882206,688.22766866)(572.36883057,688.18767334)
\curveto(572.36882206,688.14766874)(572.37382206,688.10766878)(572.38383057,688.06767334)
\curveto(572.40382203,688.01766887)(572.40882202,687.96266893)(572.39883057,687.90267334)
\curveto(572.38882204,687.84266905)(572.39382204,687.7876691)(572.41383057,687.73767334)
\moveto(570.31383057,687.19767334)
\curveto(570.32382411,687.24766964)(570.3288241,687.31766957)(570.32883057,687.40767334)
\curveto(570.3288241,687.50766938)(570.32382411,687.58266931)(570.31383057,687.63267334)
\lineto(570.31383057,687.75267334)
\curveto(570.29382414,687.80266909)(570.28382415,687.85766903)(570.28383057,687.91767334)
\curveto(570.28382415,687.97766891)(570.27882415,688.03266886)(570.26883057,688.08267334)
\curveto(570.26882416,688.12266877)(570.26382417,688.15266874)(570.25383057,688.17267334)
\lineto(570.19383057,688.41267334)
\curveto(570.18382425,688.50266839)(570.16382427,688.5876683)(570.13383057,688.66767334)
\curveto(570.02382441,688.92766796)(569.89382454,689.14766774)(569.74383057,689.32767334)
\curveto(569.59382484,689.51766737)(569.39382504,689.66766722)(569.14383057,689.77767334)
\curveto(569.08382535,689.79766709)(569.02382541,689.81266708)(568.96383057,689.82267334)
\curveto(568.90382553,689.84266705)(568.83882559,689.86266703)(568.76883057,689.88267334)
\curveto(568.68882574,689.90266699)(568.60382583,689.90766698)(568.51383057,689.89767334)
\lineto(568.24383057,689.89767334)
\curveto(568.21382622,689.87766701)(568.17882625,689.86766702)(568.13883057,689.86767334)
\curveto(568.09882633,689.87766701)(568.06382637,689.87766701)(568.03383057,689.86767334)
\lineto(567.82383057,689.80767334)
\curveto(567.76382667,689.79766709)(567.70882672,689.77766711)(567.65883057,689.74767334)
\curveto(567.40882702,689.63766725)(567.20382723,689.47766741)(567.04383057,689.26767334)
\curveto(566.89382754,689.06766782)(566.77382766,688.83266806)(566.68383057,688.56267334)
\curveto(566.65382778,688.46266843)(566.6288278,688.35766853)(566.60883057,688.24767334)
\curveto(566.59882783,688.13766875)(566.58382785,688.02766886)(566.56383057,687.91767334)
\curveto(566.55382788,687.86766902)(566.54882788,687.81766907)(566.54883057,687.76767334)
\lineto(566.54883057,687.61767334)
\curveto(566.5288279,687.54766934)(566.51882791,687.44266945)(566.51883057,687.30267334)
\curveto(566.5288279,687.16266973)(566.54382789,687.05766983)(566.56383057,686.98767334)
\lineto(566.56383057,686.85267334)
\curveto(566.58382785,686.77267012)(566.59882783,686.6926702)(566.60883057,686.61267334)
\curveto(566.61882781,686.54267035)(566.6338278,686.46767042)(566.65383057,686.38767334)
\curveto(566.75382768,686.0876708)(566.85882757,685.84267105)(566.96883057,685.65267334)
\curveto(567.08882734,685.47267142)(567.27382716,685.30767158)(567.52383057,685.15767334)
\curveto(567.59382684,685.10767178)(567.66882676,685.06767182)(567.74883057,685.03767334)
\curveto(567.83882659,685.00767188)(567.9288265,684.98267191)(568.01883057,684.96267334)
\curveto(568.05882637,684.95267194)(568.09382634,684.94767194)(568.12383057,684.94767334)
\curveto(568.15382628,684.95767193)(568.18882624,684.95767193)(568.22883057,684.94767334)
\lineto(568.34883057,684.91767334)
\curveto(568.39882603,684.91767197)(568.44382599,684.92267197)(568.48383057,684.93267334)
\lineto(568.60383057,684.93267334)
\curveto(568.68382575,684.95267194)(568.76382567,684.96767192)(568.84383057,684.97767334)
\curveto(568.92382551,684.9876719)(568.99882543,685.00767188)(569.06883057,685.03767334)
\curveto(569.3288251,685.13767175)(569.53882489,685.27267162)(569.69883057,685.44267334)
\curveto(569.85882457,685.61267128)(569.99382444,685.82267107)(570.10383057,686.07267334)
\curveto(570.14382429,686.17267072)(570.17382426,686.27267062)(570.19383057,686.37267334)
\curveto(570.21382422,686.47267042)(570.23882419,686.57767031)(570.26883057,686.68767334)
\curveto(570.27882415,686.72767016)(570.28382415,686.76267013)(570.28383057,686.79267334)
\curveto(570.28382415,686.83267006)(570.28882414,686.87267002)(570.29883057,686.91267334)
\lineto(570.29883057,687.04767334)
\curveto(570.29882413,687.09766979)(570.30382413,687.14766974)(570.31383057,687.19767334)
}
}
{
\newrgbcolor{curcolor}{0 0 0}
\pscustom[linestyle=none,fillstyle=solid,fillcolor=curcolor]
{
\newpath
\moveto(578.23875244,691.48767334)
\curveto(578.83874664,691.50766538)(579.33874614,691.42266547)(579.73875244,691.23267334)
\curveto(580.13874534,691.04266585)(580.45374502,690.76266613)(580.68375244,690.39267334)
\curveto(580.75374472,690.28266661)(580.80874467,690.16266673)(580.84875244,690.03267334)
\curveto(580.88874459,689.91266698)(580.92874455,689.7876671)(580.96875244,689.65767334)
\curveto(580.98874449,689.57766731)(580.99874448,689.50266739)(580.99875244,689.43267334)
\curveto(581.00874447,689.36266753)(581.02374445,689.2926676)(581.04375244,689.22267334)
\curveto(581.04374443,689.16266773)(581.04874443,689.12266777)(581.05875244,689.10267334)
\curveto(581.0787444,688.96266793)(581.08874439,688.81766807)(581.08875244,688.66767334)
\lineto(581.08875244,688.23267334)
\lineto(581.08875244,686.89767334)
\lineto(581.08875244,684.46767334)
\curveto(581.08874439,684.27767261)(581.08374439,684.0926728)(581.07375244,683.91267334)
\curveto(581.0737444,683.74267315)(581.00374447,683.63267326)(580.86375244,683.58267334)
\curveto(580.80374467,683.56267333)(580.73374474,683.55267334)(580.65375244,683.55267334)
\lineto(580.41375244,683.55267334)
\lineto(579.60375244,683.55267334)
\curveto(579.48374599,683.55267334)(579.3737461,683.55767333)(579.27375244,683.56767334)
\curveto(579.18374629,683.5876733)(579.11374636,683.63267326)(579.06375244,683.70267334)
\curveto(579.02374645,683.76267313)(578.99874648,683.83767305)(578.98875244,683.92767334)
\lineto(578.98875244,684.24267334)
\lineto(578.98875244,685.29267334)
\lineto(578.98875244,687.52767334)
\curveto(578.98874649,687.89766899)(578.9737465,688.23766865)(578.94375244,688.54767334)
\curveto(578.91374656,688.86766802)(578.82374665,689.13766775)(578.67375244,689.35767334)
\curveto(578.53374694,689.55766733)(578.32874715,689.69766719)(578.05875244,689.77767334)
\curveto(578.00874747,689.79766709)(577.95374752,689.80766708)(577.89375244,689.80767334)
\curveto(577.84374763,689.80766708)(577.78874769,689.81766707)(577.72875244,689.83767334)
\curveto(577.6787478,689.84766704)(577.61374786,689.84766704)(577.53375244,689.83767334)
\curveto(577.46374801,689.83766705)(577.40874807,689.83266706)(577.36875244,689.82267334)
\curveto(577.32874815,689.81266708)(577.29374818,689.80766708)(577.26375244,689.80767334)
\curveto(577.23374824,689.80766708)(577.20374827,689.80266709)(577.17375244,689.79267334)
\curveto(576.94374853,689.73266716)(576.75874872,689.65266724)(576.61875244,689.55267334)
\curveto(576.29874918,689.32266757)(576.10874937,688.9876679)(576.04875244,688.54767334)
\curveto(575.98874949,688.10766878)(575.95874952,687.61266928)(575.95875244,687.06267334)
\lineto(575.95875244,685.18767334)
\lineto(575.95875244,684.27267334)
\lineto(575.95875244,684.00267334)
\curveto(575.95874952,683.91267298)(575.94374953,683.83767305)(575.91375244,683.77767334)
\curveto(575.86374961,683.66767322)(575.78374969,683.60267329)(575.67375244,683.58267334)
\curveto(575.56374991,683.56267333)(575.42875005,683.55267334)(575.26875244,683.55267334)
\lineto(574.51875244,683.55267334)
\curveto(574.40875107,683.55267334)(574.29875118,683.55767333)(574.18875244,683.56767334)
\curveto(574.0787514,683.57767331)(573.99875148,683.61267328)(573.94875244,683.67267334)
\curveto(573.8787516,683.76267313)(573.84375163,683.892673)(573.84375244,684.06267334)
\curveto(573.85375162,684.23267266)(573.85875162,684.3926725)(573.85875244,684.54267334)
\lineto(573.85875244,686.58267334)
\lineto(573.85875244,689.88267334)
\lineto(573.85875244,690.64767334)
\lineto(573.85875244,690.94767334)
\curveto(573.86875161,691.03766585)(573.89875158,691.11266578)(573.94875244,691.17267334)
\curveto(573.96875151,691.20266569)(573.99875148,691.22266567)(574.03875244,691.23267334)
\curveto(574.08875139,691.25266564)(574.13875134,691.26766562)(574.18875244,691.27767334)
\lineto(574.26375244,691.27767334)
\curveto(574.31375116,691.2876656)(574.36375111,691.2926656)(574.41375244,691.29267334)
\lineto(574.57875244,691.29267334)
\lineto(575.20875244,691.29267334)
\curveto(575.28875019,691.2926656)(575.36375011,691.2876656)(575.43375244,691.27767334)
\curveto(575.51374996,691.27766561)(575.58374989,691.26766562)(575.64375244,691.24767334)
\curveto(575.71374976,691.21766567)(575.75874972,691.17266572)(575.77875244,691.11267334)
\curveto(575.80874967,691.05266584)(575.83374964,690.98266591)(575.85375244,690.90267334)
\curveto(575.86374961,690.86266603)(575.86374961,690.82766606)(575.85375244,690.79767334)
\curveto(575.85374962,690.76766612)(575.86374961,690.73766615)(575.88375244,690.70767334)
\curveto(575.90374957,690.65766623)(575.91874956,690.62766626)(575.92875244,690.61767334)
\curveto(575.94874953,690.60766628)(575.9737495,690.5926663)(576.00375244,690.57267334)
\curveto(576.11374936,690.56266633)(576.20374927,690.59766629)(576.27375244,690.67767334)
\curveto(576.34374913,690.76766612)(576.41874906,690.83766605)(576.49875244,690.88767334)
\curveto(576.76874871,691.0876658)(577.06874841,691.24766564)(577.39875244,691.36767334)
\curveto(577.48874799,691.39766549)(577.5787479,691.41766547)(577.66875244,691.42767334)
\curveto(577.76874771,691.43766545)(577.8737476,691.45266544)(577.98375244,691.47267334)
\curveto(578.01374746,691.48266541)(578.05874742,691.48266541)(578.11875244,691.47267334)
\curveto(578.1787473,691.47266542)(578.21874726,691.47766541)(578.23875244,691.48767334)
}
}
{
\newrgbcolor{curcolor}{0 0 0}
\pscustom[linestyle=none,fillstyle=solid,fillcolor=curcolor]
{
}
}
{
\newrgbcolor{curcolor}{0 0 0}
\pscustom[linestyle=none,fillstyle=solid,fillcolor=curcolor]
{
\newpath
\moveto(587.30015869,691.27767334)
\lineto(588.42515869,691.27767334)
\curveto(588.53515626,691.27766561)(588.63515616,691.27266562)(588.72515869,691.26267334)
\curveto(588.81515598,691.25266564)(588.88015591,691.21766567)(588.92015869,691.15767334)
\curveto(588.97015582,691.09766579)(589.00015579,691.01266588)(589.01015869,690.90267334)
\curveto(589.02015577,690.80266609)(589.02515577,690.69766619)(589.02515869,690.58767334)
\lineto(589.02515869,689.53767334)
\lineto(589.02515869,687.30267334)
\curveto(589.02515577,686.94266995)(589.04015575,686.60267029)(589.07015869,686.28267334)
\curveto(589.10015569,685.96267093)(589.1901556,685.69767119)(589.34015869,685.48767334)
\curveto(589.48015531,685.27767161)(589.70515509,685.12767176)(590.01515869,685.03767334)
\curveto(590.06515473,685.02767186)(590.10515469,685.02267187)(590.13515869,685.02267334)
\curveto(590.17515462,685.02267187)(590.22015457,685.01767187)(590.27015869,685.00767334)
\curveto(590.32015447,684.99767189)(590.37515442,684.9926719)(590.43515869,684.99267334)
\curveto(590.4951543,684.9926719)(590.54015425,684.99767189)(590.57015869,685.00767334)
\curveto(590.62015417,685.02767186)(590.66015413,685.03267186)(590.69015869,685.02267334)
\curveto(590.73015406,685.01267188)(590.77015402,685.01767187)(590.81015869,685.03767334)
\curveto(591.02015377,685.0876718)(591.18515361,685.15267174)(591.30515869,685.23267334)
\curveto(591.48515331,685.34267155)(591.62515317,685.48267141)(591.72515869,685.65267334)
\curveto(591.83515296,685.83267106)(591.91015288,686.02767086)(591.95015869,686.23767334)
\curveto(592.00015279,686.45767043)(592.03015276,686.69767019)(592.04015869,686.95767334)
\curveto(592.05015274,687.22766966)(592.05515274,687.50766938)(592.05515869,687.79767334)
\lineto(592.05515869,689.61267334)
\lineto(592.05515869,690.58767334)
\lineto(592.05515869,690.85767334)
\curveto(592.05515274,690.95766593)(592.07515272,691.03766585)(592.11515869,691.09767334)
\curveto(592.16515263,691.1876657)(592.24015255,691.23766565)(592.34015869,691.24767334)
\curveto(592.44015235,691.26766562)(592.56015223,691.27766561)(592.70015869,691.27767334)
\lineto(593.49515869,691.27767334)
\lineto(593.78015869,691.27767334)
\curveto(593.87015092,691.27766561)(593.94515085,691.25766563)(594.00515869,691.21767334)
\curveto(594.08515071,691.16766572)(594.13015066,691.0926658)(594.14015869,690.99267334)
\curveto(594.15015064,690.892666)(594.15515064,690.77766611)(594.15515869,690.64767334)
\lineto(594.15515869,689.50767334)
\lineto(594.15515869,685.29267334)
\lineto(594.15515869,684.22767334)
\lineto(594.15515869,683.92767334)
\curveto(594.15515064,683.82767306)(594.13515066,683.75267314)(594.09515869,683.70267334)
\curveto(594.04515075,683.62267327)(593.97015082,683.57767331)(593.87015869,683.56767334)
\curveto(593.77015102,683.55767333)(593.66515113,683.55267334)(593.55515869,683.55267334)
\lineto(592.74515869,683.55267334)
\curveto(592.63515216,683.55267334)(592.53515226,683.55767333)(592.44515869,683.56767334)
\curveto(592.36515243,683.57767331)(592.30015249,683.61767327)(592.25015869,683.68767334)
\curveto(592.23015256,683.71767317)(592.21015258,683.76267313)(592.19015869,683.82267334)
\curveto(592.18015261,683.88267301)(592.16515263,683.94267295)(592.14515869,684.00267334)
\curveto(592.13515266,684.06267283)(592.12015267,684.11767277)(592.10015869,684.16767334)
\curveto(592.08015271,684.21767267)(592.05015274,684.24767264)(592.01015869,684.25767334)
\curveto(591.9901528,684.27767261)(591.96515283,684.28267261)(591.93515869,684.27267334)
\curveto(591.90515289,684.26267263)(591.88015291,684.25267264)(591.86015869,684.24267334)
\curveto(591.790153,684.20267269)(591.73015306,684.15767273)(591.68015869,684.10767334)
\curveto(591.63015316,684.05767283)(591.57515322,684.01267288)(591.51515869,683.97267334)
\curveto(591.47515332,683.94267295)(591.43515336,683.90767298)(591.39515869,683.86767334)
\curveto(591.36515343,683.83767305)(591.32515347,683.80767308)(591.27515869,683.77767334)
\curveto(591.04515375,683.63767325)(590.77515402,683.52767336)(590.46515869,683.44767334)
\curveto(590.3951544,683.42767346)(590.32515447,683.41767347)(590.25515869,683.41767334)
\curveto(590.18515461,683.40767348)(590.11015468,683.3926735)(590.03015869,683.37267334)
\curveto(589.9901548,683.36267353)(589.94515485,683.36267353)(589.89515869,683.37267334)
\curveto(589.85515494,683.37267352)(589.81515498,683.36767352)(589.77515869,683.35767334)
\curveto(589.74515505,683.34767354)(589.68015511,683.34767354)(589.58015869,683.35767334)
\curveto(589.4901553,683.35767353)(589.43015536,683.36267353)(589.40015869,683.37267334)
\curveto(589.35015544,683.37267352)(589.30015549,683.37767351)(589.25015869,683.38767334)
\lineto(589.10015869,683.38767334)
\curveto(588.98015581,683.41767347)(588.86515593,683.44267345)(588.75515869,683.46267334)
\curveto(588.64515615,683.48267341)(588.53515626,683.51267338)(588.42515869,683.55267334)
\curveto(588.37515642,683.57267332)(588.33015646,683.5876733)(588.29015869,683.59767334)
\curveto(588.26015653,683.61767327)(588.22015657,683.63767325)(588.17015869,683.65767334)
\curveto(587.82015697,683.84767304)(587.54015725,684.11267278)(587.33015869,684.45267334)
\curveto(587.20015759,684.66267223)(587.10515769,684.91267198)(587.04515869,685.20267334)
\curveto(586.98515781,685.50267139)(586.94515785,685.81767107)(586.92515869,686.14767334)
\curveto(586.91515788,686.4876704)(586.91015788,686.83267006)(586.91015869,687.18267334)
\curveto(586.92015787,687.54266935)(586.92515787,687.89766899)(586.92515869,688.24767334)
\lineto(586.92515869,690.28767334)
\curveto(586.92515787,690.41766647)(586.92015787,690.56766632)(586.91015869,690.73767334)
\curveto(586.91015788,690.91766597)(586.93515786,691.04766584)(586.98515869,691.12767334)
\curveto(587.01515778,691.17766571)(587.07515772,691.22266567)(587.16515869,691.26267334)
\curveto(587.22515757,691.26266563)(587.27015752,691.26766562)(587.30015869,691.27767334)
}
}
{
\newrgbcolor{curcolor}{0 0 0}
\pscustom[linestyle=none,fillstyle=solid,fillcolor=curcolor]
{
\newpath
\moveto(600.21140869,691.48767334)
\curveto(600.81140289,691.50766538)(601.31140239,691.42266547)(601.71140869,691.23267334)
\curveto(602.11140159,691.04266585)(602.42640127,690.76266613)(602.65640869,690.39267334)
\curveto(602.72640097,690.28266661)(602.78140092,690.16266673)(602.82140869,690.03267334)
\curveto(602.86140084,689.91266698)(602.9014008,689.7876671)(602.94140869,689.65767334)
\curveto(602.96140074,689.57766731)(602.97140073,689.50266739)(602.97140869,689.43267334)
\curveto(602.98140072,689.36266753)(602.9964007,689.2926676)(603.01640869,689.22267334)
\curveto(603.01640068,689.16266773)(603.02140068,689.12266777)(603.03140869,689.10267334)
\curveto(603.05140065,688.96266793)(603.06140064,688.81766807)(603.06140869,688.66767334)
\lineto(603.06140869,688.23267334)
\lineto(603.06140869,686.89767334)
\lineto(603.06140869,684.46767334)
\curveto(603.06140064,684.27767261)(603.05640064,684.0926728)(603.04640869,683.91267334)
\curveto(603.04640065,683.74267315)(602.97640072,683.63267326)(602.83640869,683.58267334)
\curveto(602.77640092,683.56267333)(602.70640099,683.55267334)(602.62640869,683.55267334)
\lineto(602.38640869,683.55267334)
\lineto(601.57640869,683.55267334)
\curveto(601.45640224,683.55267334)(601.34640235,683.55767333)(601.24640869,683.56767334)
\curveto(601.15640254,683.5876733)(601.08640261,683.63267326)(601.03640869,683.70267334)
\curveto(600.9964027,683.76267313)(600.97140273,683.83767305)(600.96140869,683.92767334)
\lineto(600.96140869,684.24267334)
\lineto(600.96140869,685.29267334)
\lineto(600.96140869,687.52767334)
\curveto(600.96140274,687.89766899)(600.94640275,688.23766865)(600.91640869,688.54767334)
\curveto(600.88640281,688.86766802)(600.7964029,689.13766775)(600.64640869,689.35767334)
\curveto(600.50640319,689.55766733)(600.3014034,689.69766719)(600.03140869,689.77767334)
\curveto(599.98140372,689.79766709)(599.92640377,689.80766708)(599.86640869,689.80767334)
\curveto(599.81640388,689.80766708)(599.76140394,689.81766707)(599.70140869,689.83767334)
\curveto(599.65140405,689.84766704)(599.58640411,689.84766704)(599.50640869,689.83767334)
\curveto(599.43640426,689.83766705)(599.38140432,689.83266706)(599.34140869,689.82267334)
\curveto(599.3014044,689.81266708)(599.26640443,689.80766708)(599.23640869,689.80767334)
\curveto(599.20640449,689.80766708)(599.17640452,689.80266709)(599.14640869,689.79267334)
\curveto(598.91640478,689.73266716)(598.73140497,689.65266724)(598.59140869,689.55267334)
\curveto(598.27140543,689.32266757)(598.08140562,688.9876679)(598.02140869,688.54767334)
\curveto(597.96140574,688.10766878)(597.93140577,687.61266928)(597.93140869,687.06267334)
\lineto(597.93140869,685.18767334)
\lineto(597.93140869,684.27267334)
\lineto(597.93140869,684.00267334)
\curveto(597.93140577,683.91267298)(597.91640578,683.83767305)(597.88640869,683.77767334)
\curveto(597.83640586,683.66767322)(597.75640594,683.60267329)(597.64640869,683.58267334)
\curveto(597.53640616,683.56267333)(597.4014063,683.55267334)(597.24140869,683.55267334)
\lineto(596.49140869,683.55267334)
\curveto(596.38140732,683.55267334)(596.27140743,683.55767333)(596.16140869,683.56767334)
\curveto(596.05140765,683.57767331)(595.97140773,683.61267328)(595.92140869,683.67267334)
\curveto(595.85140785,683.76267313)(595.81640788,683.892673)(595.81640869,684.06267334)
\curveto(595.82640787,684.23267266)(595.83140787,684.3926725)(595.83140869,684.54267334)
\lineto(595.83140869,686.58267334)
\lineto(595.83140869,689.88267334)
\lineto(595.83140869,690.64767334)
\lineto(595.83140869,690.94767334)
\curveto(595.84140786,691.03766585)(595.87140783,691.11266578)(595.92140869,691.17267334)
\curveto(595.94140776,691.20266569)(595.97140773,691.22266567)(596.01140869,691.23267334)
\curveto(596.06140764,691.25266564)(596.11140759,691.26766562)(596.16140869,691.27767334)
\lineto(596.23640869,691.27767334)
\curveto(596.28640741,691.2876656)(596.33640736,691.2926656)(596.38640869,691.29267334)
\lineto(596.55140869,691.29267334)
\lineto(597.18140869,691.29267334)
\curveto(597.26140644,691.2926656)(597.33640636,691.2876656)(597.40640869,691.27767334)
\curveto(597.48640621,691.27766561)(597.55640614,691.26766562)(597.61640869,691.24767334)
\curveto(597.68640601,691.21766567)(597.73140597,691.17266572)(597.75140869,691.11267334)
\curveto(597.78140592,691.05266584)(597.80640589,690.98266591)(597.82640869,690.90267334)
\curveto(597.83640586,690.86266603)(597.83640586,690.82766606)(597.82640869,690.79767334)
\curveto(597.82640587,690.76766612)(597.83640586,690.73766615)(597.85640869,690.70767334)
\curveto(597.87640582,690.65766623)(597.89140581,690.62766626)(597.90140869,690.61767334)
\curveto(597.92140578,690.60766628)(597.94640575,690.5926663)(597.97640869,690.57267334)
\curveto(598.08640561,690.56266633)(598.17640552,690.59766629)(598.24640869,690.67767334)
\curveto(598.31640538,690.76766612)(598.39140531,690.83766605)(598.47140869,690.88767334)
\curveto(598.74140496,691.0876658)(599.04140466,691.24766564)(599.37140869,691.36767334)
\curveto(599.46140424,691.39766549)(599.55140415,691.41766547)(599.64140869,691.42767334)
\curveto(599.74140396,691.43766545)(599.84640385,691.45266544)(599.95640869,691.47267334)
\curveto(599.98640371,691.48266541)(600.03140367,691.48266541)(600.09140869,691.47267334)
\curveto(600.15140355,691.47266542)(600.19140351,691.47766541)(600.21140869,691.48767334)
}
}
{
\newrgbcolor{curcolor}{0 0 0}
\pscustom[linestyle=none,fillstyle=solid,fillcolor=curcolor]
{
\newpath
\moveto(391.58836182,672.73767334)
\curveto(392.18835601,672.75766538)(392.68835551,672.67266547)(393.08836182,672.48267334)
\curveto(393.48835471,672.29266585)(393.8033544,672.01266613)(394.03336182,671.64267334)
\curveto(394.1033541,671.53266661)(394.15835404,671.41266673)(394.19836182,671.28267334)
\curveto(394.23835396,671.16266698)(394.27835392,671.0376671)(394.31836182,670.90767334)
\curveto(394.33835386,670.82766731)(394.34835385,670.75266739)(394.34836182,670.68267334)
\curveto(394.35835384,670.61266753)(394.37335383,670.5426676)(394.39336182,670.47267334)
\curveto(394.39335381,670.41266773)(394.3983538,670.37266777)(394.40836182,670.35267334)
\curveto(394.42835377,670.21266793)(394.43835376,670.06766807)(394.43836182,669.91767334)
\lineto(394.43836182,669.48267334)
\lineto(394.43836182,668.14767334)
\lineto(394.43836182,665.71767334)
\curveto(394.43835376,665.52767261)(394.43335377,665.3426728)(394.42336182,665.16267334)
\curveto(394.42335378,664.99267315)(394.35335385,664.88267326)(394.21336182,664.83267334)
\curveto(394.15335405,664.81267333)(394.08335412,664.80267334)(394.00336182,664.80267334)
\lineto(393.76336182,664.80267334)
\lineto(392.95336182,664.80267334)
\curveto(392.83335537,664.80267334)(392.72335548,664.80767333)(392.62336182,664.81767334)
\curveto(392.53335567,664.8376733)(392.46335574,664.88267326)(392.41336182,664.95267334)
\curveto(392.37335583,665.01267313)(392.34835585,665.08767305)(392.33836182,665.17767334)
\lineto(392.33836182,665.49267334)
\lineto(392.33836182,666.54267334)
\lineto(392.33836182,668.77767334)
\curveto(392.33835586,669.14766899)(392.32335588,669.48766865)(392.29336182,669.79767334)
\curveto(392.26335594,670.11766802)(392.17335603,670.38766775)(392.02336182,670.60767334)
\curveto(391.88335632,670.80766733)(391.67835652,670.94766719)(391.40836182,671.02767334)
\curveto(391.35835684,671.04766709)(391.3033569,671.05766708)(391.24336182,671.05767334)
\curveto(391.19335701,671.05766708)(391.13835706,671.06766707)(391.07836182,671.08767334)
\curveto(391.02835717,671.09766704)(390.96335724,671.09766704)(390.88336182,671.08767334)
\curveto(390.81335739,671.08766705)(390.75835744,671.08266706)(390.71836182,671.07267334)
\curveto(390.67835752,671.06266708)(390.64335756,671.05766708)(390.61336182,671.05767334)
\curveto(390.58335762,671.05766708)(390.55335765,671.05266709)(390.52336182,671.04267334)
\curveto(390.29335791,670.98266716)(390.10835809,670.90266724)(389.96836182,670.80267334)
\curveto(389.64835855,670.57266757)(389.45835874,670.2376679)(389.39836182,669.79767334)
\curveto(389.33835886,669.35766878)(389.30835889,668.86266928)(389.30836182,668.31267334)
\lineto(389.30836182,666.43767334)
\lineto(389.30836182,665.52267334)
\lineto(389.30836182,665.25267334)
\curveto(389.30835889,665.16267298)(389.29335891,665.08767305)(389.26336182,665.02767334)
\curveto(389.21335899,664.91767322)(389.13335907,664.85267329)(389.02336182,664.83267334)
\curveto(388.91335929,664.81267333)(388.77835942,664.80267334)(388.61836182,664.80267334)
\lineto(387.86836182,664.80267334)
\curveto(387.75836044,664.80267334)(387.64836055,664.80767333)(387.53836182,664.81767334)
\curveto(387.42836077,664.82767331)(387.34836085,664.86267328)(387.29836182,664.92267334)
\curveto(387.22836097,665.01267313)(387.19336101,665.142673)(387.19336182,665.31267334)
\curveto(387.203361,665.48267266)(387.20836099,665.6426725)(387.20836182,665.79267334)
\lineto(387.20836182,667.83267334)
\lineto(387.20836182,671.13267334)
\lineto(387.20836182,671.89767334)
\lineto(387.20836182,672.19767334)
\curveto(387.21836098,672.28766585)(387.24836095,672.36266578)(387.29836182,672.42267334)
\curveto(387.31836088,672.45266569)(387.34836085,672.47266567)(387.38836182,672.48267334)
\curveto(387.43836076,672.50266564)(387.48836071,672.51766562)(387.53836182,672.52767334)
\lineto(387.61336182,672.52767334)
\curveto(387.66336054,672.5376656)(387.71336049,672.5426656)(387.76336182,672.54267334)
\lineto(387.92836182,672.54267334)
\lineto(388.55836182,672.54267334)
\curveto(388.63835956,672.5426656)(388.71335949,672.5376656)(388.78336182,672.52767334)
\curveto(388.86335934,672.52766561)(388.93335927,672.51766562)(388.99336182,672.49767334)
\curveto(389.06335914,672.46766567)(389.10835909,672.42266572)(389.12836182,672.36267334)
\curveto(389.15835904,672.30266584)(389.18335902,672.23266591)(389.20336182,672.15267334)
\curveto(389.21335899,672.11266603)(389.21335899,672.07766606)(389.20336182,672.04767334)
\curveto(389.203359,672.01766612)(389.21335899,671.98766615)(389.23336182,671.95767334)
\curveto(389.25335895,671.90766623)(389.26835893,671.87766626)(389.27836182,671.86767334)
\curveto(389.2983589,671.85766628)(389.32335888,671.8426663)(389.35336182,671.82267334)
\curveto(389.46335874,671.81266633)(389.55335865,671.84766629)(389.62336182,671.92767334)
\curveto(389.69335851,672.01766612)(389.76835843,672.08766605)(389.84836182,672.13767334)
\curveto(390.11835808,672.3376658)(390.41835778,672.49766564)(390.74836182,672.61767334)
\curveto(390.83835736,672.64766549)(390.92835727,672.66766547)(391.01836182,672.67767334)
\curveto(391.11835708,672.68766545)(391.22335698,672.70266544)(391.33336182,672.72267334)
\curveto(391.36335684,672.73266541)(391.40835679,672.73266541)(391.46836182,672.72267334)
\curveto(391.52835667,672.72266542)(391.56835663,672.72766541)(391.58836182,672.73767334)
}
}
{
\newrgbcolor{curcolor}{0 0 0}
\pscustom[linestyle=none,fillstyle=solid,fillcolor=curcolor]
{
\newpath
\moveto(403.83961182,668.98767334)
\curveto(403.85960325,668.92766921)(403.86960324,668.8426693)(403.86961182,668.73267334)
\curveto(403.86960324,668.62266952)(403.85960325,668.5376696)(403.83961182,668.47767334)
\lineto(403.83961182,668.32767334)
\curveto(403.81960329,668.24766989)(403.8096033,668.16766997)(403.80961182,668.08767334)
\curveto(403.81960329,668.00767013)(403.81460329,667.92767021)(403.79461182,667.84767334)
\curveto(403.77460333,667.77767036)(403.75960335,667.71267043)(403.74961182,667.65267334)
\curveto(403.73960337,667.59267055)(403.72960338,667.52767061)(403.71961182,667.45767334)
\curveto(403.67960343,667.34767079)(403.64460346,667.23267091)(403.61461182,667.11267334)
\curveto(403.58460352,667.00267114)(403.54460356,666.89767124)(403.49461182,666.79767334)
\curveto(403.28460382,666.31767182)(403.0096041,665.92767221)(402.66961182,665.62767334)
\curveto(402.32960478,665.32767281)(401.91960519,665.07767306)(401.43961182,664.87767334)
\curveto(401.31960579,664.82767331)(401.19460591,664.79267335)(401.06461182,664.77267334)
\curveto(400.94460616,664.7426734)(400.81960629,664.71267343)(400.68961182,664.68267334)
\curveto(400.63960647,664.66267348)(400.58460652,664.65267349)(400.52461182,664.65267334)
\curveto(400.46460664,664.65267349)(400.4096067,664.64767349)(400.35961182,664.63767334)
\lineto(400.25461182,664.63767334)
\curveto(400.22460688,664.62767351)(400.19460691,664.62267352)(400.16461182,664.62267334)
\curveto(400.11460699,664.61267353)(400.03460707,664.60767353)(399.92461182,664.60767334)
\curveto(399.81460729,664.59767354)(399.72960738,664.60267354)(399.66961182,664.62267334)
\lineto(399.51961182,664.62267334)
\curveto(399.46960764,664.63267351)(399.41460769,664.6376735)(399.35461182,664.63767334)
\curveto(399.3046078,664.62767351)(399.25460785,664.63267351)(399.20461182,664.65267334)
\curveto(399.16460794,664.66267348)(399.12460798,664.66767347)(399.08461182,664.66767334)
\curveto(399.05460805,664.66767347)(399.01460809,664.67267347)(398.96461182,664.68267334)
\curveto(398.86460824,664.71267343)(398.76460834,664.7376734)(398.66461182,664.75767334)
\curveto(398.56460854,664.77767336)(398.46960864,664.80767333)(398.37961182,664.84767334)
\curveto(398.25960885,664.88767325)(398.14460896,664.92767321)(398.03461182,664.96767334)
\curveto(397.93460917,665.00767313)(397.82960928,665.05767308)(397.71961182,665.11767334)
\curveto(397.36960974,665.32767281)(397.06961004,665.57267257)(396.81961182,665.85267334)
\curveto(396.56961054,666.13267201)(396.35961075,666.46767167)(396.18961182,666.85767334)
\curveto(396.13961097,666.94767119)(396.09961101,667.0426711)(396.06961182,667.14267334)
\curveto(396.04961106,667.2426709)(396.02461108,667.34767079)(395.99461182,667.45767334)
\curveto(395.97461113,667.50767063)(395.96461114,667.55267059)(395.96461182,667.59267334)
\curveto(395.96461114,667.63267051)(395.95461115,667.67767046)(395.93461182,667.72767334)
\curveto(395.91461119,667.80767033)(395.9046112,667.88767025)(395.90461182,667.96767334)
\curveto(395.9046112,668.05767008)(395.89461121,668.14267)(395.87461182,668.22267334)
\curveto(395.86461124,668.27266987)(395.85961125,668.31766982)(395.85961182,668.35767334)
\lineto(395.85961182,668.49267334)
\curveto(395.83961127,668.55266959)(395.82961128,668.6376695)(395.82961182,668.74767334)
\curveto(395.83961127,668.85766928)(395.85461125,668.9426692)(395.87461182,669.00267334)
\lineto(395.87461182,669.10767334)
\curveto(395.88461122,669.15766898)(395.88461122,669.20766893)(395.87461182,669.25767334)
\curveto(395.87461123,669.31766882)(395.88461122,669.37266877)(395.90461182,669.42267334)
\curveto(395.91461119,669.47266867)(395.91961119,669.51766862)(395.91961182,669.55767334)
\curveto(395.91961119,669.60766853)(395.92961118,669.65766848)(395.94961182,669.70767334)
\curveto(395.98961112,669.8376683)(396.02461108,669.96266818)(396.05461182,670.08267334)
\curveto(396.08461102,670.21266793)(396.12461098,670.3376678)(396.17461182,670.45767334)
\curveto(396.35461075,670.86766727)(396.56961054,671.20766693)(396.81961182,671.47767334)
\curveto(397.06961004,671.75766638)(397.37460973,672.01266613)(397.73461182,672.24267334)
\curveto(397.83460927,672.29266585)(397.93960917,672.3376658)(398.04961182,672.37767334)
\curveto(398.15960895,672.41766572)(398.26960884,672.46266568)(398.37961182,672.51267334)
\curveto(398.5096086,672.56266558)(398.64460846,672.59766554)(398.78461182,672.61767334)
\curveto(398.92460818,672.6376655)(399.06960804,672.66766547)(399.21961182,672.70767334)
\curveto(399.29960781,672.71766542)(399.37460773,672.72266542)(399.44461182,672.72267334)
\curveto(399.51460759,672.72266542)(399.58460752,672.72766541)(399.65461182,672.73767334)
\curveto(400.23460687,672.74766539)(400.73460637,672.68766545)(401.15461182,672.55767334)
\curveto(401.58460552,672.42766571)(401.96460514,672.24766589)(402.29461182,672.01767334)
\curveto(402.4046047,671.9376662)(402.51460459,671.84766629)(402.62461182,671.74767334)
\curveto(402.74460436,671.65766648)(402.84460426,671.55766658)(402.92461182,671.44767334)
\curveto(403.0046041,671.34766679)(403.07460403,671.24766689)(403.13461182,671.14767334)
\curveto(403.2046039,671.04766709)(403.27460383,670.9426672)(403.34461182,670.83267334)
\curveto(403.41460369,670.72266742)(403.46960364,670.60266754)(403.50961182,670.47267334)
\curveto(403.54960356,670.35266779)(403.59460351,670.22266792)(403.64461182,670.08267334)
\curveto(403.67460343,670.00266814)(403.69960341,669.91766822)(403.71961182,669.82767334)
\lineto(403.77961182,669.55767334)
\curveto(403.78960332,669.51766862)(403.79460331,669.47766866)(403.79461182,669.43767334)
\curveto(403.79460331,669.39766874)(403.79960331,669.35766878)(403.80961182,669.31767334)
\curveto(403.82960328,669.26766887)(403.83460327,669.21266893)(403.82461182,669.15267334)
\curveto(403.81460329,669.09266905)(403.81960329,669.0376691)(403.83961182,668.98767334)
\moveto(401.73961182,668.44767334)
\curveto(401.74960536,668.49766964)(401.75460535,668.56766957)(401.75461182,668.65767334)
\curveto(401.75460535,668.75766938)(401.74960536,668.83266931)(401.73961182,668.88267334)
\lineto(401.73961182,669.00267334)
\curveto(401.71960539,669.05266909)(401.7096054,669.10766903)(401.70961182,669.16767334)
\curveto(401.7096054,669.22766891)(401.7046054,669.28266886)(401.69461182,669.33267334)
\curveto(401.69460541,669.37266877)(401.68960542,669.40266874)(401.67961182,669.42267334)
\lineto(401.61961182,669.66267334)
\curveto(401.6096055,669.75266839)(401.58960552,669.8376683)(401.55961182,669.91767334)
\curveto(401.44960566,670.17766796)(401.31960579,670.39766774)(401.16961182,670.57767334)
\curveto(401.01960609,670.76766737)(400.81960629,670.91766722)(400.56961182,671.02767334)
\curveto(400.5096066,671.04766709)(400.44960666,671.06266708)(400.38961182,671.07267334)
\curveto(400.32960678,671.09266705)(400.26460684,671.11266703)(400.19461182,671.13267334)
\curveto(400.11460699,671.15266699)(400.02960708,671.15766698)(399.93961182,671.14767334)
\lineto(399.66961182,671.14767334)
\curveto(399.63960747,671.12766701)(399.6046075,671.11766702)(399.56461182,671.11767334)
\curveto(399.52460758,671.12766701)(399.48960762,671.12766701)(399.45961182,671.11767334)
\lineto(399.24961182,671.05767334)
\curveto(399.18960792,671.04766709)(399.13460797,671.02766711)(399.08461182,670.99767334)
\curveto(398.83460827,670.88766725)(398.62960848,670.72766741)(398.46961182,670.51767334)
\curveto(398.31960879,670.31766782)(398.19960891,670.08266806)(398.10961182,669.81267334)
\curveto(398.07960903,669.71266843)(398.05460905,669.60766853)(398.03461182,669.49767334)
\curveto(398.02460908,669.38766875)(398.0096091,669.27766886)(397.98961182,669.16767334)
\curveto(397.97960913,669.11766902)(397.97460913,669.06766907)(397.97461182,669.01767334)
\lineto(397.97461182,668.86767334)
\curveto(397.95460915,668.79766934)(397.94460916,668.69266945)(397.94461182,668.55267334)
\curveto(397.95460915,668.41266973)(397.96960914,668.30766983)(397.98961182,668.23767334)
\lineto(397.98961182,668.10267334)
\curveto(398.0096091,668.02267012)(398.02460908,667.9426702)(398.03461182,667.86267334)
\curveto(398.04460906,667.79267035)(398.05960905,667.71767042)(398.07961182,667.63767334)
\curveto(398.17960893,667.3376708)(398.28460882,667.09267105)(398.39461182,666.90267334)
\curveto(398.51460859,666.72267142)(398.69960841,666.55767158)(398.94961182,666.40767334)
\curveto(399.01960809,666.35767178)(399.09460801,666.31767182)(399.17461182,666.28767334)
\curveto(399.26460784,666.25767188)(399.35460775,666.23267191)(399.44461182,666.21267334)
\curveto(399.48460762,666.20267194)(399.51960759,666.19767194)(399.54961182,666.19767334)
\curveto(399.57960753,666.20767193)(399.61460749,666.20767193)(399.65461182,666.19767334)
\lineto(399.77461182,666.16767334)
\curveto(399.82460728,666.16767197)(399.86960724,666.17267197)(399.90961182,666.18267334)
\lineto(400.02961182,666.18267334)
\curveto(400.109607,666.20267194)(400.18960692,666.21767192)(400.26961182,666.22767334)
\curveto(400.34960676,666.2376719)(400.42460668,666.25767188)(400.49461182,666.28767334)
\curveto(400.75460635,666.38767175)(400.96460614,666.52267162)(401.12461182,666.69267334)
\curveto(401.28460582,666.86267128)(401.41960569,667.07267107)(401.52961182,667.32267334)
\curveto(401.56960554,667.42267072)(401.59960551,667.52267062)(401.61961182,667.62267334)
\curveto(401.63960547,667.72267042)(401.66460544,667.82767031)(401.69461182,667.93767334)
\curveto(401.7046054,667.97767016)(401.7096054,668.01267013)(401.70961182,668.04267334)
\curveto(401.7096054,668.08267006)(401.71460539,668.12267002)(401.72461182,668.16267334)
\lineto(401.72461182,668.29767334)
\curveto(401.72460538,668.34766979)(401.72960538,668.39766974)(401.73961182,668.44767334)
}
}
{
\newrgbcolor{curcolor}{0 0 0}
\pscustom[linestyle=none,fillstyle=solid,fillcolor=curcolor]
{
\newpath
\moveto(409.69453369,672.73767334)
\curveto(410.06452809,672.74766539)(410.38952776,672.70766543)(410.66953369,672.61767334)
\curveto(410.9495272,672.52766561)(411.19452696,672.40266574)(411.40453369,672.24267334)
\curveto(411.48452667,672.18266596)(411.5545266,672.11266603)(411.61453369,672.03267334)
\curveto(411.68452647,671.95266619)(411.75952639,671.87266627)(411.83953369,671.79267334)
\curveto(411.85952629,671.77266637)(411.88952626,671.7426664)(411.92953369,671.70267334)
\curveto(411.97952617,671.67266647)(412.02952612,671.66766647)(412.07953369,671.68767334)
\curveto(412.18952596,671.71766642)(412.29452586,671.78766635)(412.39453369,671.89767334)
\curveto(412.49452566,672.01766612)(412.58952556,672.10766603)(412.67953369,672.16767334)
\curveto(412.81952533,672.27766586)(412.96952518,672.36766577)(413.12953369,672.43767334)
\curveto(413.28952486,672.51766562)(413.46952468,672.59266555)(413.66953369,672.66267334)
\curveto(413.7495244,672.68266546)(413.84452431,672.69766544)(413.95453369,672.70767334)
\curveto(414.07452408,672.72766541)(414.19452396,672.7376654)(414.31453369,672.73767334)
\curveto(414.44452371,672.74766539)(414.56452359,672.74766539)(414.67453369,672.73767334)
\curveto(414.79452336,672.72766541)(414.89952325,672.71266543)(414.98953369,672.69267334)
\curveto(415.03952311,672.68266546)(415.08452307,672.67766546)(415.12453369,672.67767334)
\curveto(415.16452299,672.67766546)(415.20952294,672.66766547)(415.25953369,672.64767334)
\curveto(415.39952275,672.60766553)(415.53452262,672.56766557)(415.66453369,672.52767334)
\curveto(415.79452236,672.48766565)(415.91452224,672.43266571)(416.02453369,672.36267334)
\curveto(416.44452171,672.10266604)(416.75952139,671.72266642)(416.96953369,671.22267334)
\curveto(417.00952114,671.13266701)(417.03952111,671.0376671)(417.05953369,670.93767334)
\curveto(417.07952107,670.84766729)(417.09952105,670.75766738)(417.11953369,670.66767334)
\curveto(417.12952102,670.59766754)(417.13452102,670.53266761)(417.13453369,670.47267334)
\curveto(417.14452101,670.41266773)(417.154521,670.35266779)(417.16453369,670.29267334)
\lineto(417.16453369,670.14267334)
\curveto(417.17452098,670.08266806)(417.17452098,670.01266813)(417.16453369,669.93267334)
\curveto(417.16452099,669.85266829)(417.16452099,669.77766836)(417.16453369,669.70767334)
\lineto(417.16453369,668.83767334)
\lineto(417.16453369,665.91267334)
\curveto(417.16452099,665.83267231)(417.16452099,665.7376724)(417.16453369,665.62767334)
\curveto(417.17452098,665.52767261)(417.17452098,665.42767271)(417.16453369,665.32767334)
\curveto(417.16452099,665.2376729)(417.154521,665.14767299)(417.13453369,665.05767334)
\curveto(417.11452104,664.97767316)(417.08452107,664.92267322)(417.04453369,664.89267334)
\curveto(416.98452117,664.8426733)(416.90452125,664.81267333)(416.80453369,664.80267334)
\lineto(416.50453369,664.80267334)
\lineto(415.70953369,664.80267334)
\curveto(415.56952258,664.80267334)(415.44452271,664.81267333)(415.33453369,664.83267334)
\curveto(415.22452293,664.85267329)(415.149523,664.90767323)(415.10953369,664.99767334)
\curveto(415.07952307,665.06767307)(415.06452309,665.142673)(415.06453369,665.22267334)
\curveto(415.06452309,665.31267283)(415.06452309,665.39767274)(415.06453369,665.47767334)
\lineto(415.06453369,666.31767334)
\lineto(415.06453369,668.34267334)
\lineto(415.06453369,668.97267334)
\curveto(415.06452309,669.02266912)(415.06452309,669.07766906)(415.06453369,669.13767334)
\curveto(415.07452308,669.19766894)(415.06952308,669.25266889)(415.04953369,669.30267334)
\lineto(415.04953369,669.42267334)
\curveto(415.0495231,669.48266866)(415.0495231,669.5426686)(415.04953369,669.60267334)
\curveto(415.0495231,669.66266848)(415.04452311,669.72266842)(415.03453369,669.78267334)
\curveto(415.02452313,669.82266832)(415.01952313,669.86266828)(415.01953369,669.90267334)
\curveto(415.01952313,669.95266819)(415.01452314,669.99766814)(415.00453369,670.03767334)
\curveto(414.96452319,670.18766795)(414.91952323,670.31766782)(414.86953369,670.42767334)
\curveto(414.82952332,670.54766759)(414.76452339,670.65266749)(414.67453369,670.74267334)
\curveto(414.53452362,670.88266726)(414.36452379,670.98266716)(414.16453369,671.04267334)
\curveto(414.12452403,671.05266709)(414.08952406,671.05266709)(414.05953369,671.04267334)
\curveto(414.02952412,671.0426671)(413.99452416,671.05266709)(413.95453369,671.07267334)
\curveto(413.91452424,671.08266706)(413.86452429,671.08766705)(413.80453369,671.08767334)
\curveto(413.7545244,671.09766704)(413.70452445,671.09766704)(413.65453369,671.08767334)
\curveto(413.59452456,671.06766707)(413.53452462,671.05766708)(413.47453369,671.05767334)
\curveto(413.41452474,671.05766708)(413.3545248,671.04766709)(413.29453369,671.02767334)
\curveto(413.00452515,670.92766721)(412.79452536,670.77766736)(412.66453369,670.57767334)
\curveto(412.49452566,670.34766779)(412.38952576,670.05766808)(412.34953369,669.70767334)
\curveto(412.31952583,669.36766877)(412.30452585,668.99266915)(412.30453369,668.58267334)
\lineto(412.30453369,666.60267334)
\lineto(412.30453369,665.49267334)
\lineto(412.30453369,665.19267334)
\curveto(412.30452585,665.09267305)(412.27952587,665.01267313)(412.22953369,664.95267334)
\curveto(412.17952597,664.88267326)(412.10452605,664.8376733)(412.00453369,664.81767334)
\curveto(411.91452624,664.80767333)(411.80952634,664.80267334)(411.68953369,664.80267334)
\lineto(410.87953369,664.80267334)
\lineto(410.60953369,664.80267334)
\curveto(410.52952762,664.81267333)(410.45952769,664.82767331)(410.39953369,664.84767334)
\curveto(410.29952785,664.89767324)(410.23952791,664.97767316)(410.21953369,665.08767334)
\curveto(410.20952794,665.19767294)(410.20452795,665.32267282)(410.20453369,665.46267334)
\lineto(410.20453369,666.73767334)
\lineto(410.20453369,669.09267334)
\curveto(410.20452795,669.38266876)(410.19452796,669.65766848)(410.17453369,669.91767334)
\curveto(410.154528,670.17766796)(410.08952806,670.39266775)(409.97953369,670.56267334)
\curveto(409.89952825,670.70266744)(409.79452836,670.80766733)(409.66453369,670.87767334)
\curveto(409.54452861,670.94766719)(409.39452876,671.00766713)(409.21453369,671.05767334)
\curveto(409.17452898,671.06766707)(409.13452902,671.06766707)(409.09453369,671.05767334)
\curveto(409.0545291,671.05766708)(409.00952914,671.06266708)(408.95953369,671.07267334)
\curveto(408.8495293,671.09266705)(408.74452941,671.08266706)(408.64453369,671.04267334)
\curveto(408.62452953,671.0426671)(408.60452955,671.0376671)(408.58453369,671.02767334)
\lineto(408.52453369,671.02767334)
\curveto(408.36452979,670.97766716)(408.20952994,670.89266725)(408.05953369,670.77267334)
\curveto(407.89953025,670.65266749)(407.77453038,670.51266763)(407.68453369,670.35267334)
\curveto(407.60453055,670.20266794)(407.54453061,670.02766811)(407.50453369,669.82767334)
\curveto(407.47453068,669.6376685)(407.4545307,669.42766871)(407.44453369,669.19767334)
\lineto(407.44453369,668.44767334)
\lineto(407.44453369,666.42267334)
\lineto(407.44453369,665.50767334)
\lineto(407.44453369,665.23767334)
\curveto(407.44453071,665.14767299)(407.42953072,665.06767307)(407.39953369,664.99767334)
\curveto(407.35953079,664.90767323)(407.28453087,664.85267329)(407.17453369,664.83267334)
\curveto(407.06453109,664.81267333)(406.93953121,664.80267334)(406.79953369,664.80267334)
\lineto(406.01953369,664.80267334)
\lineto(405.71953369,664.80267334)
\curveto(405.62953252,664.81267333)(405.5545326,664.8376733)(405.49453369,664.87767334)
\curveto(405.40453275,664.92767321)(405.3545328,665.01767312)(405.34453369,665.14767334)
\lineto(405.34453369,665.58267334)
\lineto(405.34453369,667.33767334)
\lineto(405.34453369,670.99767334)
\lineto(405.34453369,671.89767334)
\lineto(405.34453369,672.18267334)
\curveto(405.3545328,672.27266587)(405.37953277,672.34766579)(405.41953369,672.40767334)
\curveto(405.46953268,672.46766567)(405.5495326,672.50766563)(405.65953369,672.52767334)
\lineto(405.74953369,672.52767334)
\curveto(405.79953235,672.5376656)(405.8495323,672.5426656)(405.89953369,672.54267334)
\lineto(406.06453369,672.54267334)
\lineto(406.67953369,672.54267334)
\curveto(406.75953139,672.5426656)(406.83453132,672.5376656)(406.90453369,672.52767334)
\curveto(406.98453117,672.52766561)(407.0545311,672.51766562)(407.11453369,672.49767334)
\curveto(407.19453096,672.46766567)(407.24453091,672.41766572)(407.26453369,672.34767334)
\curveto(407.29453086,672.27766586)(407.31953083,672.19766594)(407.33953369,672.10767334)
\curveto(407.3495308,672.07766606)(407.3495308,672.04766609)(407.33953369,672.01767334)
\curveto(407.33953081,671.99766614)(407.3495308,671.97766616)(407.36953369,671.95767334)
\curveto(407.37953077,671.92766621)(407.38953076,671.90266624)(407.39953369,671.88267334)
\curveto(407.41953073,671.87266627)(407.43953071,671.85766628)(407.45953369,671.83767334)
\curveto(407.57953057,671.82766631)(407.67953047,671.86266628)(407.75953369,671.94267334)
\curveto(407.83953031,672.03266611)(407.91453024,672.10266604)(407.98453369,672.15267334)
\curveto(408.12453003,672.25266589)(408.26452989,672.3426658)(408.40453369,672.42267334)
\curveto(408.5545296,672.50266564)(408.71452944,672.56766557)(408.88453369,672.61767334)
\curveto(408.97452918,672.64766549)(409.06452909,672.66766547)(409.15453369,672.67767334)
\curveto(409.24452891,672.68766545)(409.33952881,672.70266544)(409.43953369,672.72267334)
\curveto(409.46952868,672.73266541)(409.51452864,672.73266541)(409.57453369,672.72267334)
\curveto(409.63452852,672.72266542)(409.67452848,672.72766541)(409.69453369,672.73767334)
}
}
{
\newrgbcolor{curcolor}{0 0 0}
\pscustom[linestyle=none,fillstyle=solid,fillcolor=curcolor]
{
\newpath
\moveto(426.63328369,669.06267334)
\curveto(426.65327509,669.00266914)(426.66327508,668.89766924)(426.66328369,668.74767334)
\curveto(426.66327508,668.60766953)(426.65827509,668.50766963)(426.64828369,668.44767334)
\curveto(426.6482751,668.39766974)(426.6432751,668.35266979)(426.63328369,668.31267334)
\lineto(426.63328369,668.19267334)
\curveto(426.61327513,668.11267003)(426.60327514,668.03267011)(426.60328369,667.95267334)
\curveto(426.60327514,667.88267026)(426.59327515,667.80767033)(426.57328369,667.72767334)
\curveto(426.57327517,667.68767045)(426.56327518,667.61767052)(426.54328369,667.51767334)
\curveto(426.51327523,667.39767074)(426.48327526,667.27267087)(426.45328369,667.14267334)
\curveto(426.43327531,667.02267112)(426.39827535,666.90767123)(426.34828369,666.79767334)
\curveto(426.16827558,666.34767179)(425.9432758,665.95767218)(425.67328369,665.62767334)
\curveto(425.40327634,665.29767284)(425.0482767,665.0376731)(424.60828369,664.84767334)
\curveto(424.51827723,664.80767333)(424.42327732,664.77767336)(424.32328369,664.75767334)
\curveto(424.23327751,664.72767341)(424.13327761,664.69767344)(424.02328369,664.66767334)
\curveto(423.96327778,664.64767349)(423.89827785,664.6376735)(423.82828369,664.63767334)
\curveto(423.76827798,664.6376735)(423.70827804,664.63267351)(423.64828369,664.62267334)
\lineto(423.51328369,664.62267334)
\curveto(423.45327829,664.60267354)(423.37327837,664.59767354)(423.27328369,664.60767334)
\curveto(423.17327857,664.60767353)(423.09327865,664.61767352)(423.03328369,664.63767334)
\lineto(422.94328369,664.63767334)
\curveto(422.89327885,664.64767349)(422.83827891,664.65767348)(422.77828369,664.66767334)
\curveto(422.71827903,664.66767347)(422.65827909,664.67267347)(422.59828369,664.68267334)
\curveto(422.40827934,664.73267341)(422.23327951,664.78267336)(422.07328369,664.83267334)
\curveto(421.91327983,664.88267326)(421.76327998,664.95267319)(421.62328369,665.04267334)
\lineto(421.44328369,665.16267334)
\curveto(421.39328035,665.20267294)(421.3432804,665.24767289)(421.29328369,665.29767334)
\lineto(421.20328369,665.35767334)
\curveto(421.17328057,665.37767276)(421.1432806,665.39267275)(421.11328369,665.40267334)
\curveto(421.02328072,665.43267271)(420.96828078,665.41267273)(420.94828369,665.34267334)
\curveto(420.89828085,665.27267287)(420.86328088,665.18767295)(420.84328369,665.08767334)
\curveto(420.83328091,664.99767314)(420.79828095,664.92767321)(420.73828369,664.87767334)
\curveto(420.67828107,664.8376733)(420.60828114,664.81267333)(420.52828369,664.80267334)
\lineto(420.25828369,664.80267334)
\lineto(419.53828369,664.80267334)
\lineto(419.31328369,664.80267334)
\curveto(419.2432825,664.79267335)(419.17828257,664.79767334)(419.11828369,664.81767334)
\curveto(418.97828277,664.86767327)(418.89828285,664.95767318)(418.87828369,665.08767334)
\curveto(418.86828288,665.22767291)(418.86328288,665.38267276)(418.86328369,665.55267334)
\lineto(418.86328369,674.70267334)
\lineto(418.86328369,675.04767334)
\curveto(418.86328288,675.16766297)(418.88828286,675.26266288)(418.93828369,675.33267334)
\curveto(418.97828277,675.40266274)(419.0482827,675.44766269)(419.14828369,675.46767334)
\curveto(419.16828258,675.47766266)(419.18828256,675.47766266)(419.20828369,675.46767334)
\curveto(419.23828251,675.46766267)(419.26328248,675.47266267)(419.28328369,675.48267334)
\lineto(420.22828369,675.48267334)
\curveto(420.40828134,675.48266266)(420.56328118,675.47266267)(420.69328369,675.45267334)
\curveto(420.82328092,675.4426627)(420.90828084,675.36766277)(420.94828369,675.22767334)
\curveto(420.97828077,675.12766301)(420.98828076,674.99266315)(420.97828369,674.82267334)
\curveto(420.96828078,674.66266348)(420.96328078,674.52266362)(420.96328369,674.40267334)
\lineto(420.96328369,672.76767334)
\lineto(420.96328369,672.43767334)
\curveto(420.96328078,672.32766581)(420.97328077,672.23266591)(420.99328369,672.15267334)
\curveto(421.00328074,672.10266604)(421.01328073,672.05766608)(421.02328369,672.01767334)
\curveto(421.03328071,671.98766615)(421.05828069,671.96766617)(421.09828369,671.95767334)
\curveto(421.11828063,671.9376662)(421.1432806,671.92766621)(421.17328369,671.92767334)
\curveto(421.21328053,671.92766621)(421.2432805,671.93266621)(421.26328369,671.94267334)
\curveto(421.33328041,671.98266616)(421.39828035,672.02266612)(421.45828369,672.06267334)
\curveto(421.51828023,672.11266603)(421.58328016,672.16266598)(421.65328369,672.21267334)
\curveto(421.78327996,672.30266584)(421.91827983,672.37766576)(422.05828369,672.43767334)
\curveto(422.19827955,672.50766563)(422.35327939,672.56766557)(422.52328369,672.61767334)
\curveto(422.60327914,672.64766549)(422.68327906,672.66266548)(422.76328369,672.66267334)
\curveto(422.8432789,672.67266547)(422.92327882,672.68766545)(423.00328369,672.70767334)
\curveto(423.07327867,672.72766541)(423.1482786,672.7376654)(423.22828369,672.73767334)
\lineto(423.46828369,672.73767334)
\lineto(423.61828369,672.73767334)
\curveto(423.6482781,672.72766541)(423.68327806,672.72266542)(423.72328369,672.72267334)
\curveto(423.76327798,672.73266541)(423.80327794,672.73266541)(423.84328369,672.72267334)
\curveto(423.95327779,672.69266545)(424.05327769,672.66766547)(424.14328369,672.64767334)
\curveto(424.2432775,672.6376655)(424.33827741,672.61266553)(424.42828369,672.57267334)
\curveto(424.88827686,672.38266576)(425.26327648,672.137666)(425.55328369,671.83767334)
\curveto(425.8432759,671.5376666)(426.08827566,671.16266698)(426.28828369,670.71267334)
\curveto(426.33827541,670.59266755)(426.37827537,670.46766767)(426.40828369,670.33767334)
\curveto(426.4482753,670.20766793)(426.48827526,670.07266807)(426.52828369,669.93267334)
\curveto(426.5482752,669.86266828)(426.55827519,669.79266835)(426.55828369,669.72267334)
\curveto(426.56827518,669.66266848)(426.58327516,669.59266855)(426.60328369,669.51267334)
\curveto(426.62327512,669.46266868)(426.62827512,669.40766873)(426.61828369,669.34767334)
\curveto(426.61827513,669.28766885)(426.62327512,669.22766891)(426.63328369,669.16767334)
\lineto(426.63328369,669.06267334)
\moveto(424.41328369,667.65267334)
\curveto(424.4432773,667.75267039)(424.46827728,667.87767026)(424.48828369,668.02767334)
\curveto(424.51827723,668.17766996)(424.53327721,668.32766981)(424.53328369,668.47767334)
\curveto(424.5432772,668.6376695)(424.5432772,668.79266935)(424.53328369,668.94267334)
\curveto(424.53327721,669.10266904)(424.51827723,669.2376689)(424.48828369,669.34767334)
\curveto(424.45827729,669.44766869)(424.43827731,669.5426686)(424.42828369,669.63267334)
\curveto(424.41827733,669.72266842)(424.39327735,669.80766833)(424.35328369,669.88767334)
\curveto(424.21327753,670.2376679)(424.01327773,670.53266761)(423.75328369,670.77267334)
\curveto(423.50327824,671.02266712)(423.13327861,671.14766699)(422.64328369,671.14767334)
\curveto(422.60327914,671.14766699)(422.56827918,671.142667)(422.53828369,671.13267334)
\lineto(422.43328369,671.13267334)
\curveto(422.36327938,671.11266703)(422.29827945,671.09266705)(422.23828369,671.07267334)
\curveto(422.17827957,671.06266708)(422.11827963,671.04766709)(422.05828369,671.02767334)
\curveto(421.76827998,670.89766724)(421.5482802,670.71266743)(421.39828369,670.47267334)
\curveto(421.2482805,670.2426679)(421.12328062,669.97766816)(421.02328369,669.67767334)
\curveto(420.99328075,669.59766854)(420.97328077,669.51266863)(420.96328369,669.42267334)
\curveto(420.96328078,669.3426688)(420.95328079,669.26266888)(420.93328369,669.18267334)
\curveto(420.92328082,669.15266899)(420.91828083,669.10266904)(420.91828369,669.03267334)
\curveto(420.90828084,668.99266915)(420.90328084,668.95266919)(420.90328369,668.91267334)
\curveto(420.91328083,668.87266927)(420.91328083,668.83266931)(420.90328369,668.79267334)
\curveto(420.88328086,668.71266943)(420.87828087,668.60266954)(420.88828369,668.46267334)
\curveto(420.89828085,668.32266982)(420.91328083,668.22266992)(420.93328369,668.16267334)
\curveto(420.95328079,668.07267007)(420.96328078,667.98767015)(420.96328369,667.90767334)
\curveto(420.97328077,667.82767031)(420.99328075,667.74767039)(421.02328369,667.66767334)
\curveto(421.11328063,667.38767075)(421.21828053,667.142671)(421.33828369,666.93267334)
\curveto(421.46828028,666.73267141)(421.6482801,666.56267158)(421.87828369,666.42267334)
\curveto(422.03827971,666.32267182)(422.20327954,666.25267189)(422.37328369,666.21267334)
\curveto(422.39327935,666.21267193)(422.41327933,666.20767193)(422.43328369,666.19767334)
\lineto(422.52328369,666.19767334)
\curveto(422.55327919,666.18767195)(422.60327914,666.17767196)(422.67328369,666.16767334)
\curveto(422.743279,666.16767197)(422.80327894,666.17267197)(422.85328369,666.18267334)
\curveto(422.95327879,666.20267194)(423.0432787,666.21767192)(423.12328369,666.22767334)
\curveto(423.21327853,666.24767189)(423.29827845,666.27267187)(423.37828369,666.30267334)
\curveto(423.65827809,666.43267171)(423.87327787,666.61267153)(424.02328369,666.84267334)
\curveto(424.18327756,667.07267107)(424.31327743,667.3426708)(424.41328369,667.65267334)
}
}
{
\newrgbcolor{curcolor}{0 0 0}
\pscustom[linestyle=none,fillstyle=solid,fillcolor=curcolor]
{
\newpath
\moveto(432.42820557,672.73767334)
\curveto(432.53820025,672.7376654)(432.63320016,672.72766541)(432.71320557,672.70767334)
\curveto(432.80319999,672.68766545)(432.87319992,672.6426655)(432.92320557,672.57267334)
\curveto(432.98319981,672.49266565)(433.01319978,672.35266579)(433.01320557,672.15267334)
\lineto(433.01320557,671.64267334)
\lineto(433.01320557,671.26767334)
\curveto(433.02319977,671.12766701)(433.00819978,671.01766712)(432.96820557,670.93767334)
\curveto(432.92819986,670.86766727)(432.86819992,670.82266732)(432.78820557,670.80267334)
\curveto(432.71820007,670.78266736)(432.63320016,670.77266737)(432.53320557,670.77267334)
\curveto(432.44320035,670.77266737)(432.34320045,670.77766736)(432.23320557,670.78767334)
\curveto(432.13320066,670.79766734)(432.03820075,670.79266735)(431.94820557,670.77267334)
\curveto(431.87820091,670.75266739)(431.80820098,670.7376674)(431.73820557,670.72767334)
\curveto(431.66820112,670.72766741)(431.60320119,670.71766742)(431.54320557,670.69767334)
\curveto(431.38320141,670.64766749)(431.22320157,670.57266757)(431.06320557,670.47267334)
\curveto(430.90320189,670.38266776)(430.77820201,670.27766786)(430.68820557,670.15767334)
\curveto(430.63820215,670.07766806)(430.58320221,669.99266815)(430.52320557,669.90267334)
\curveto(430.47320232,669.82266832)(430.42320237,669.7376684)(430.37320557,669.64767334)
\curveto(430.34320245,669.56766857)(430.31320248,669.48266866)(430.28320557,669.39267334)
\lineto(430.22320557,669.15267334)
\curveto(430.20320259,669.08266906)(430.1932026,669.00766913)(430.19320557,668.92767334)
\curveto(430.1932026,668.85766928)(430.18320261,668.78766935)(430.16320557,668.71767334)
\curveto(430.15320264,668.67766946)(430.14820264,668.6376695)(430.14820557,668.59767334)
\curveto(430.15820263,668.56766957)(430.15820263,668.5376696)(430.14820557,668.50767334)
\lineto(430.14820557,668.26767334)
\curveto(430.12820266,668.19766994)(430.12320267,668.11767002)(430.13320557,668.02767334)
\curveto(430.14320265,667.94767019)(430.14820264,667.86767027)(430.14820557,667.78767334)
\lineto(430.14820557,666.82767334)
\lineto(430.14820557,665.55267334)
\curveto(430.14820264,665.42267272)(430.14320265,665.30267284)(430.13320557,665.19267334)
\curveto(430.12320267,665.08267306)(430.0932027,664.99267315)(430.04320557,664.92267334)
\curveto(430.02320277,664.89267325)(429.9882028,664.86767327)(429.93820557,664.84767334)
\curveto(429.89820289,664.8376733)(429.85320294,664.82767331)(429.80320557,664.81767334)
\lineto(429.72820557,664.81767334)
\curveto(429.67820311,664.80767333)(429.62320317,664.80267334)(429.56320557,664.80267334)
\lineto(429.39820557,664.80267334)
\lineto(428.75320557,664.80267334)
\curveto(428.6932041,664.81267333)(428.62820416,664.81767332)(428.55820557,664.81767334)
\lineto(428.36320557,664.81767334)
\curveto(428.31320448,664.8376733)(428.26320453,664.85267329)(428.21320557,664.86267334)
\curveto(428.16320463,664.88267326)(428.12820466,664.91767322)(428.10820557,664.96767334)
\curveto(428.06820472,665.01767312)(428.04320475,665.08767305)(428.03320557,665.17767334)
\lineto(428.03320557,665.47767334)
\lineto(428.03320557,666.49767334)
\lineto(428.03320557,670.72767334)
\lineto(428.03320557,671.83767334)
\lineto(428.03320557,672.12267334)
\curveto(428.03320476,672.22266592)(428.05320474,672.30266584)(428.09320557,672.36267334)
\curveto(428.14320465,672.4426657)(428.21820457,672.49266565)(428.31820557,672.51267334)
\curveto(428.41820437,672.53266561)(428.53820425,672.5426656)(428.67820557,672.54267334)
\lineto(429.44320557,672.54267334)
\curveto(429.56320323,672.5426656)(429.66820312,672.53266561)(429.75820557,672.51267334)
\curveto(429.84820294,672.50266564)(429.91820287,672.45766568)(429.96820557,672.37767334)
\curveto(429.99820279,672.32766581)(430.01320278,672.25766588)(430.01320557,672.16767334)
\lineto(430.04320557,671.89767334)
\curveto(430.05320274,671.81766632)(430.06820272,671.7426664)(430.08820557,671.67267334)
\curveto(430.11820267,671.60266654)(430.16820262,671.56766657)(430.23820557,671.56767334)
\curveto(430.25820253,671.58766655)(430.27820251,671.59766654)(430.29820557,671.59767334)
\curveto(430.31820247,671.59766654)(430.33820245,671.60766653)(430.35820557,671.62767334)
\curveto(430.41820237,671.67766646)(430.46820232,671.73266641)(430.50820557,671.79267334)
\curveto(430.55820223,671.86266628)(430.61820217,671.92266622)(430.68820557,671.97267334)
\curveto(430.72820206,672.00266614)(430.76320203,672.03266611)(430.79320557,672.06267334)
\curveto(430.82320197,672.10266604)(430.85820193,672.137666)(430.89820557,672.16767334)
\lineto(431.16820557,672.34767334)
\curveto(431.26820152,672.40766573)(431.36820142,672.46266568)(431.46820557,672.51267334)
\curveto(431.56820122,672.55266559)(431.66820112,672.58766555)(431.76820557,672.61767334)
\lineto(432.09820557,672.70767334)
\curveto(432.12820066,672.71766542)(432.18320061,672.71766542)(432.26320557,672.70767334)
\curveto(432.35320044,672.70766543)(432.40820038,672.71766542)(432.42820557,672.73767334)
}
}
{
\newrgbcolor{curcolor}{0 0 0}
\pscustom[linestyle=none,fillstyle=solid,fillcolor=curcolor]
{
\newpath
\moveto(440.93461182,668.74767334)
\curveto(440.95460365,668.66766947)(440.95460365,668.57766956)(440.93461182,668.47767334)
\curveto(440.91460369,668.37766976)(440.87960373,668.31266983)(440.82961182,668.28267334)
\curveto(440.77960383,668.2426699)(440.7046039,668.21266993)(440.60461182,668.19267334)
\curveto(440.51460409,668.18266996)(440.4096042,668.17266997)(440.28961182,668.16267334)
\lineto(439.94461182,668.16267334)
\curveto(439.83460477,668.17266997)(439.73460487,668.17766996)(439.64461182,668.17767334)
\lineto(435.98461182,668.17767334)
\lineto(435.77461182,668.17767334)
\curveto(435.71460889,668.17766996)(435.65960895,668.16766997)(435.60961182,668.14767334)
\curveto(435.52960908,668.10767003)(435.47960913,668.06767007)(435.45961182,668.02767334)
\curveto(435.43960917,668.00767013)(435.41960919,667.96767017)(435.39961182,667.90767334)
\curveto(435.37960923,667.85767028)(435.37460923,667.80767033)(435.38461182,667.75767334)
\curveto(435.4046092,667.69767044)(435.41460919,667.6376705)(435.41461182,667.57767334)
\curveto(435.42460918,667.52767061)(435.43960917,667.47267067)(435.45961182,667.41267334)
\curveto(435.53960907,667.17267097)(435.63460897,666.97267117)(435.74461182,666.81267334)
\curveto(435.86460874,666.66267148)(436.02460858,666.52767161)(436.22461182,666.40767334)
\curveto(436.3046083,666.35767178)(436.38460822,666.32267182)(436.46461182,666.30267334)
\curveto(436.55460805,666.29267185)(436.64460796,666.27267187)(436.73461182,666.24267334)
\curveto(436.81460779,666.22267192)(436.92460768,666.20767193)(437.06461182,666.19767334)
\curveto(437.2046074,666.18767195)(437.32460728,666.19267195)(437.42461182,666.21267334)
\lineto(437.55961182,666.21267334)
\curveto(437.65960695,666.23267191)(437.74960686,666.25267189)(437.82961182,666.27267334)
\curveto(437.91960669,666.30267184)(438.0046066,666.33267181)(438.08461182,666.36267334)
\curveto(438.18460642,666.41267173)(438.29460631,666.47767166)(438.41461182,666.55767334)
\curveto(438.54460606,666.6376715)(438.63960597,666.71767142)(438.69961182,666.79767334)
\curveto(438.74960586,666.86767127)(438.79960581,666.93267121)(438.84961182,666.99267334)
\curveto(438.9096057,667.06267108)(438.97960563,667.11267103)(439.05961182,667.14267334)
\curveto(439.15960545,667.19267095)(439.28460532,667.21267093)(439.43461182,667.20267334)
\lineto(439.86961182,667.20267334)
\lineto(440.04961182,667.20267334)
\curveto(440.11960449,667.21267093)(440.17960443,667.20767093)(440.22961182,667.18767334)
\lineto(440.37961182,667.18767334)
\curveto(440.47960413,667.16767097)(440.54960406,667.142671)(440.58961182,667.11267334)
\curveto(440.62960398,667.09267105)(440.64960396,667.04767109)(440.64961182,666.97767334)
\curveto(440.65960395,666.90767123)(440.65460395,666.84767129)(440.63461182,666.79767334)
\curveto(440.58460402,666.65767148)(440.52960408,666.53267161)(440.46961182,666.42267334)
\curveto(440.4096042,666.31267183)(440.33960427,666.20267194)(440.25961182,666.09267334)
\curveto(440.03960457,665.76267238)(439.78960482,665.49767264)(439.50961182,665.29767334)
\curveto(439.22960538,665.09767304)(438.87960573,664.92767321)(438.45961182,664.78767334)
\curveto(438.34960626,664.74767339)(438.23960637,664.72267342)(438.12961182,664.71267334)
\curveto(438.01960659,664.70267344)(437.9046067,664.68267346)(437.78461182,664.65267334)
\curveto(437.74460686,664.6426735)(437.69960691,664.6426735)(437.64961182,664.65267334)
\curveto(437.609607,664.65267349)(437.56960704,664.64767349)(437.52961182,664.63767334)
\lineto(437.36461182,664.63767334)
\curveto(437.31460729,664.61767352)(437.25460735,664.61267353)(437.18461182,664.62267334)
\curveto(437.12460748,664.62267352)(437.06960754,664.62767351)(437.01961182,664.63767334)
\curveto(436.93960767,664.64767349)(436.86960774,664.64767349)(436.80961182,664.63767334)
\curveto(436.74960786,664.62767351)(436.68460792,664.63267351)(436.61461182,664.65267334)
\curveto(436.56460804,664.67267347)(436.5096081,664.68267346)(436.44961182,664.68267334)
\curveto(436.38960822,664.68267346)(436.33460827,664.69267345)(436.28461182,664.71267334)
\curveto(436.17460843,664.73267341)(436.06460854,664.75767338)(435.95461182,664.78767334)
\curveto(435.84460876,664.80767333)(435.74460886,664.8426733)(435.65461182,664.89267334)
\curveto(435.54460906,664.93267321)(435.43960917,664.96767317)(435.33961182,664.99767334)
\curveto(435.24960936,665.0376731)(435.16460944,665.08267306)(435.08461182,665.13267334)
\curveto(434.76460984,665.33267281)(434.47961013,665.56267258)(434.22961182,665.82267334)
\curveto(433.97961063,666.09267205)(433.77461083,666.40267174)(433.61461182,666.75267334)
\curveto(433.56461104,666.86267128)(433.52461108,666.97267117)(433.49461182,667.08267334)
\curveto(433.46461114,667.20267094)(433.42461118,667.32267082)(433.37461182,667.44267334)
\curveto(433.36461124,667.48267066)(433.35961125,667.51767062)(433.35961182,667.54767334)
\curveto(433.35961125,667.58767055)(433.35461125,667.62767051)(433.34461182,667.66767334)
\curveto(433.3046113,667.78767035)(433.27961133,667.91767022)(433.26961182,668.05767334)
\lineto(433.23961182,668.47767334)
\curveto(433.23961137,668.52766961)(433.23461137,668.58266956)(433.22461182,668.64267334)
\curveto(433.22461138,668.70266944)(433.22961138,668.75766938)(433.23961182,668.80767334)
\lineto(433.23961182,668.98767334)
\lineto(433.28461182,669.34767334)
\curveto(433.32461128,669.51766862)(433.35961125,669.68266846)(433.38961182,669.84267334)
\curveto(433.41961119,670.00266814)(433.46461114,670.15266799)(433.52461182,670.29267334)
\curveto(433.95461065,671.33266681)(434.68460992,672.06766607)(435.71461182,672.49767334)
\curveto(435.85460875,672.55766558)(435.99460861,672.59766554)(436.13461182,672.61767334)
\curveto(436.28460832,672.64766549)(436.43960817,672.68266546)(436.59961182,672.72267334)
\curveto(436.67960793,672.73266541)(436.75460785,672.7376654)(436.82461182,672.73767334)
\curveto(436.89460771,672.7376654)(436.96960764,672.7426654)(437.04961182,672.75267334)
\curveto(437.55960705,672.76266538)(437.99460661,672.70266544)(438.35461182,672.57267334)
\curveto(438.72460588,672.45266569)(439.05460555,672.29266585)(439.34461182,672.09267334)
\curveto(439.43460517,672.03266611)(439.52460508,671.96266618)(439.61461182,671.88267334)
\curveto(439.7046049,671.81266633)(439.78460482,671.7376664)(439.85461182,671.65767334)
\curveto(439.88460472,671.60766653)(439.92460468,671.56766657)(439.97461182,671.53767334)
\curveto(440.05460455,671.42766671)(440.12960448,671.31266683)(440.19961182,671.19267334)
\curveto(440.26960434,671.08266706)(440.34460426,670.96766717)(440.42461182,670.84767334)
\curveto(440.47460413,670.75766738)(440.51460409,670.66266748)(440.54461182,670.56267334)
\curveto(440.58460402,670.47266767)(440.62460398,670.37266777)(440.66461182,670.26267334)
\curveto(440.71460389,670.13266801)(440.75460385,669.99766814)(440.78461182,669.85767334)
\curveto(440.81460379,669.71766842)(440.84960376,669.57766856)(440.88961182,669.43767334)
\curveto(440.9096037,669.35766878)(440.91460369,669.26766887)(440.90461182,669.16767334)
\curveto(440.9046037,669.07766906)(440.91460369,668.99266915)(440.93461182,668.91267334)
\lineto(440.93461182,668.74767334)
\moveto(438.68461182,669.63267334)
\curveto(438.75460585,669.73266841)(438.75960585,669.85266829)(438.69961182,669.99267334)
\curveto(438.64960596,670.142668)(438.609606,670.25266789)(438.57961182,670.32267334)
\curveto(438.43960617,670.59266755)(438.25460635,670.79766734)(438.02461182,670.93767334)
\curveto(437.79460681,671.08766705)(437.47460713,671.16766697)(437.06461182,671.17767334)
\curveto(437.03460757,671.15766698)(436.99960761,671.15266699)(436.95961182,671.16267334)
\curveto(436.91960769,671.17266697)(436.88460772,671.17266697)(436.85461182,671.16267334)
\curveto(436.8046078,671.142667)(436.74960786,671.12766701)(436.68961182,671.11767334)
\curveto(436.62960798,671.11766702)(436.57460803,671.10766703)(436.52461182,671.08767334)
\curveto(436.08460852,670.94766719)(435.75960885,670.67266747)(435.54961182,670.26267334)
\curveto(435.52960908,670.22266792)(435.5046091,670.16766797)(435.47461182,670.09767334)
\curveto(435.45460915,670.0376681)(435.43960917,669.97266817)(435.42961182,669.90267334)
\curveto(435.41960919,669.8426683)(435.41960919,669.78266836)(435.42961182,669.72267334)
\curveto(435.44960916,669.66266848)(435.48460912,669.61266853)(435.53461182,669.57267334)
\curveto(435.61460899,669.52266862)(435.72460888,669.49766864)(435.86461182,669.49767334)
\lineto(436.26961182,669.49767334)
\lineto(437.93461182,669.49767334)
\lineto(438.36961182,669.49767334)
\curveto(438.52960608,669.50766863)(438.63460597,669.55266859)(438.68461182,669.63267334)
}
}
{
\newrgbcolor{curcolor}{0 0 0}
\pscustom[linestyle=none,fillstyle=solid,fillcolor=curcolor]
{
}
}
{
\newrgbcolor{curcolor}{0 0 0}
\pscustom[linestyle=none,fillstyle=solid,fillcolor=curcolor]
{
\newpath
\moveto(453.93304932,665.65767334)
\lineto(453.93304932,665.23767334)
\curveto(453.93304095,665.10767303)(453.90304098,665.00267314)(453.84304932,664.92267334)
\curveto(453.79304109,664.87267327)(453.72804115,664.8376733)(453.64804932,664.81767334)
\curveto(453.56804131,664.80767333)(453.4780414,664.80267334)(453.37804932,664.80267334)
\lineto(452.55304932,664.80267334)
\lineto(452.26804932,664.80267334)
\curveto(452.18804269,664.81267333)(452.12304276,664.8376733)(452.07304932,664.87767334)
\curveto(452.00304288,664.92767321)(451.96304292,664.99267315)(451.95304932,665.07267334)
\curveto(451.94304294,665.15267299)(451.92304296,665.23267291)(451.89304932,665.31267334)
\curveto(451.87304301,665.33267281)(451.85304303,665.34767279)(451.83304932,665.35767334)
\curveto(451.82304306,665.37767276)(451.80804307,665.39767274)(451.78804932,665.41767334)
\curveto(451.6780432,665.41767272)(451.59804328,665.39267275)(451.54804932,665.34267334)
\lineto(451.39804932,665.19267334)
\curveto(451.32804355,665.142673)(451.26304362,665.09767304)(451.20304932,665.05767334)
\curveto(451.14304374,665.02767311)(451.0780438,664.98767315)(451.00804932,664.93767334)
\curveto(450.96804391,664.91767322)(450.92304396,664.89767324)(450.87304932,664.87767334)
\curveto(450.83304405,664.85767328)(450.78804409,664.8376733)(450.73804932,664.81767334)
\curveto(450.59804428,664.76767337)(450.44804443,664.72267342)(450.28804932,664.68267334)
\curveto(450.23804464,664.66267348)(450.19304469,664.65267349)(450.15304932,664.65267334)
\curveto(450.11304477,664.65267349)(450.07304481,664.64767349)(450.03304932,664.63767334)
\lineto(449.89804932,664.63767334)
\curveto(449.86804501,664.62767351)(449.82804505,664.62267352)(449.77804932,664.62267334)
\lineto(449.64304932,664.62267334)
\curveto(449.5830453,664.60267354)(449.49304539,664.59767354)(449.37304932,664.60767334)
\curveto(449.25304563,664.60767353)(449.16804571,664.61767352)(449.11804932,664.63767334)
\curveto(449.04804583,664.65767348)(448.9830459,664.66767347)(448.92304932,664.66767334)
\curveto(448.87304601,664.65767348)(448.81804606,664.66267348)(448.75804932,664.68267334)
\lineto(448.39804932,664.80267334)
\curveto(448.28804659,664.83267331)(448.1780467,664.87267327)(448.06804932,664.92267334)
\curveto(447.71804716,665.07267307)(447.40304748,665.30267284)(447.12304932,665.61267334)
\curveto(446.85304803,665.93267221)(446.63804824,666.26767187)(446.47804932,666.61767334)
\curveto(446.42804845,666.72767141)(446.38804849,666.83267131)(446.35804932,666.93267334)
\curveto(446.32804855,667.0426711)(446.29304859,667.15267099)(446.25304932,667.26267334)
\curveto(446.24304864,667.30267084)(446.23804864,667.3376708)(446.23804932,667.36767334)
\curveto(446.23804864,667.40767073)(446.22804865,667.45267069)(446.20804932,667.50267334)
\curveto(446.18804869,667.58267056)(446.16804871,667.66767047)(446.14804932,667.75767334)
\curveto(446.13804874,667.85767028)(446.12304876,667.95767018)(446.10304932,668.05767334)
\curveto(446.09304879,668.08767005)(446.08804879,668.12267002)(446.08804932,668.16267334)
\curveto(446.09804878,668.20266994)(446.09804878,668.2376699)(446.08804932,668.26767334)
\lineto(446.08804932,668.40267334)
\curveto(446.08804879,668.45266969)(446.0830488,668.50266964)(446.07304932,668.55267334)
\curveto(446.06304882,668.60266954)(446.05804882,668.65766948)(446.05804932,668.71767334)
\curveto(446.05804882,668.78766935)(446.06304882,668.8426693)(446.07304932,668.88267334)
\curveto(446.0830488,668.93266921)(446.08804879,668.97766916)(446.08804932,669.01767334)
\lineto(446.08804932,669.16767334)
\curveto(446.09804878,669.21766892)(446.09804878,669.26266888)(446.08804932,669.30267334)
\curveto(446.08804879,669.35266879)(446.09804878,669.40266874)(446.11804932,669.45267334)
\curveto(446.13804874,669.56266858)(446.15304873,669.66766847)(446.16304932,669.76767334)
\curveto(446.1830487,669.86766827)(446.20804867,669.96766817)(446.23804932,670.06767334)
\curveto(446.2780486,670.18766795)(446.31304857,670.30266784)(446.34304932,670.41267334)
\curveto(446.37304851,670.52266762)(446.41304847,670.63266751)(446.46304932,670.74267334)
\curveto(446.60304828,671.0426671)(446.7780481,671.32766681)(446.98804932,671.59767334)
\curveto(447.00804787,671.62766651)(447.03304785,671.65266649)(447.06304932,671.67267334)
\curveto(447.10304778,671.70266644)(447.13304775,671.73266641)(447.15304932,671.76267334)
\curveto(447.19304769,671.81266633)(447.23304765,671.85766628)(447.27304932,671.89767334)
\curveto(447.31304757,671.9376662)(447.35804752,671.97766616)(447.40804932,672.01767334)
\curveto(447.44804743,672.0376661)(447.4830474,672.06266608)(447.51304932,672.09267334)
\curveto(447.54304734,672.13266601)(447.5780473,672.16266598)(447.61804932,672.18267334)
\curveto(447.86804701,672.35266579)(448.15804672,672.49266565)(448.48804932,672.60267334)
\curveto(448.55804632,672.62266552)(448.62804625,672.6376655)(448.69804932,672.64767334)
\curveto(448.7780461,672.65766548)(448.85804602,672.67266547)(448.93804932,672.69267334)
\curveto(449.00804587,672.71266543)(449.09804578,672.72266542)(449.20804932,672.72267334)
\curveto(449.31804556,672.73266541)(449.42804545,672.7376654)(449.53804932,672.73767334)
\curveto(449.64804523,672.7376654)(449.75304513,672.73266541)(449.85304932,672.72267334)
\curveto(449.96304492,672.71266543)(450.05304483,672.69766544)(450.12304932,672.67767334)
\curveto(450.27304461,672.62766551)(450.41804446,672.58266556)(450.55804932,672.54267334)
\curveto(450.69804418,672.50266564)(450.82804405,672.44766569)(450.94804932,672.37767334)
\curveto(451.01804386,672.32766581)(451.0830438,672.27766586)(451.14304932,672.22767334)
\curveto(451.20304368,672.18766595)(451.26804361,672.142666)(451.33804932,672.09267334)
\curveto(451.3780435,672.06266608)(451.43304345,672.02266612)(451.50304932,671.97267334)
\curveto(451.5830433,671.92266622)(451.65804322,671.92266622)(451.72804932,671.97267334)
\curveto(451.76804311,671.99266615)(451.78804309,672.02766611)(451.78804932,672.07767334)
\curveto(451.78804309,672.12766601)(451.79804308,672.17766596)(451.81804932,672.22767334)
\lineto(451.81804932,672.37767334)
\curveto(451.82804305,672.40766573)(451.83304305,672.4426657)(451.83304932,672.48267334)
\lineto(451.83304932,672.60267334)
\lineto(451.83304932,674.64267334)
\curveto(451.83304305,674.75266339)(451.82804305,674.87266327)(451.81804932,675.00267334)
\curveto(451.81804306,675.142663)(451.84304304,675.24766289)(451.89304932,675.31767334)
\curveto(451.93304295,675.39766274)(452.00804287,675.44766269)(452.11804932,675.46767334)
\curveto(452.13804274,675.47766266)(452.15804272,675.47766266)(452.17804932,675.46767334)
\curveto(452.19804268,675.46766267)(452.21804266,675.47266267)(452.23804932,675.48267334)
\lineto(453.30304932,675.48267334)
\curveto(453.42304146,675.48266266)(453.53304135,675.47766266)(453.63304932,675.46767334)
\curveto(453.73304115,675.45766268)(453.80804107,675.41766272)(453.85804932,675.34767334)
\curveto(453.90804097,675.26766287)(453.93304095,675.16266298)(453.93304932,675.03267334)
\lineto(453.93304932,674.67267334)
\lineto(453.93304932,665.65767334)
\moveto(451.89304932,668.59767334)
\curveto(451.90304298,668.6376695)(451.90304298,668.67766946)(451.89304932,668.71767334)
\lineto(451.89304932,668.85267334)
\curveto(451.89304299,668.95266919)(451.88804299,669.05266909)(451.87804932,669.15267334)
\curveto(451.86804301,669.25266889)(451.85304303,669.3426688)(451.83304932,669.42267334)
\curveto(451.81304307,669.53266861)(451.79304309,669.63266851)(451.77304932,669.72267334)
\curveto(451.76304312,669.81266833)(451.73804314,669.89766824)(451.69804932,669.97767334)
\curveto(451.55804332,670.3376678)(451.35304353,670.62266752)(451.08304932,670.83267334)
\curveto(450.82304406,671.0426671)(450.44304444,671.14766699)(449.94304932,671.14767334)
\curveto(449.883045,671.14766699)(449.80304508,671.137667)(449.70304932,671.11767334)
\curveto(449.62304526,671.09766704)(449.54804533,671.07766706)(449.47804932,671.05767334)
\curveto(449.41804546,671.04766709)(449.35804552,671.02766711)(449.29804932,670.99767334)
\curveto(449.02804585,670.88766725)(448.81804606,670.71766742)(448.66804932,670.48767334)
\curveto(448.51804636,670.25766788)(448.39804648,669.99766814)(448.30804932,669.70767334)
\curveto(448.2780466,669.60766853)(448.25804662,669.50766863)(448.24804932,669.40767334)
\curveto(448.23804664,669.30766883)(448.21804666,669.20266894)(448.18804932,669.09267334)
\lineto(448.18804932,668.88267334)
\curveto(448.16804671,668.79266935)(448.16304672,668.66766947)(448.17304932,668.50767334)
\curveto(448.1830467,668.35766978)(448.19804668,668.24766989)(448.21804932,668.17767334)
\lineto(448.21804932,668.08767334)
\curveto(448.22804665,668.06767007)(448.23304665,668.04767009)(448.23304932,668.02767334)
\curveto(448.25304663,667.94767019)(448.26804661,667.87267027)(448.27804932,667.80267334)
\curveto(448.29804658,667.73267041)(448.31804656,667.65767048)(448.33804932,667.57767334)
\curveto(448.50804637,667.05767108)(448.79804608,666.67267147)(449.20804932,666.42267334)
\curveto(449.33804554,666.33267181)(449.51804536,666.26267188)(449.74804932,666.21267334)
\curveto(449.78804509,666.20267194)(449.84804503,666.19767194)(449.92804932,666.19767334)
\curveto(449.95804492,666.18767195)(450.00304488,666.17767196)(450.06304932,666.16767334)
\curveto(450.13304475,666.16767197)(450.18804469,666.17267197)(450.22804932,666.18267334)
\curveto(450.30804457,666.20267194)(450.38804449,666.21767192)(450.46804932,666.22767334)
\curveto(450.54804433,666.2376719)(450.62804425,666.25767188)(450.70804932,666.28767334)
\curveto(450.95804392,666.39767174)(451.15804372,666.5376716)(451.30804932,666.70767334)
\curveto(451.45804342,666.87767126)(451.58804329,667.09267105)(451.69804932,667.35267334)
\curveto(451.73804314,667.4426707)(451.76804311,667.53267061)(451.78804932,667.62267334)
\curveto(451.80804307,667.72267042)(451.82804305,667.82767031)(451.84804932,667.93767334)
\curveto(451.85804302,667.98767015)(451.85804302,668.03267011)(451.84804932,668.07267334)
\curveto(451.84804303,668.12267002)(451.85804302,668.17266997)(451.87804932,668.22267334)
\curveto(451.88804299,668.25266989)(451.89304299,668.28766985)(451.89304932,668.32767334)
\lineto(451.89304932,668.46267334)
\lineto(451.89304932,668.59767334)
}
}
{
\newrgbcolor{curcolor}{0 0 0}
\pscustom[linestyle=none,fillstyle=solid,fillcolor=curcolor]
{
\newpath
\moveto(462.87797119,668.74767334)
\curveto(462.89796303,668.66766947)(462.89796303,668.57766956)(462.87797119,668.47767334)
\curveto(462.85796307,668.37766976)(462.8229631,668.31266983)(462.77297119,668.28267334)
\curveto(462.7229632,668.2426699)(462.64796328,668.21266993)(462.54797119,668.19267334)
\curveto(462.45796347,668.18266996)(462.35296357,668.17266997)(462.23297119,668.16267334)
\lineto(461.88797119,668.16267334)
\curveto(461.77796415,668.17266997)(461.67796425,668.17766996)(461.58797119,668.17767334)
\lineto(457.92797119,668.17767334)
\lineto(457.71797119,668.17767334)
\curveto(457.65796827,668.17766996)(457.60296832,668.16766997)(457.55297119,668.14767334)
\curveto(457.47296845,668.10767003)(457.4229685,668.06767007)(457.40297119,668.02767334)
\curveto(457.38296854,668.00767013)(457.36296856,667.96767017)(457.34297119,667.90767334)
\curveto(457.3229686,667.85767028)(457.31796861,667.80767033)(457.32797119,667.75767334)
\curveto(457.34796858,667.69767044)(457.35796857,667.6376705)(457.35797119,667.57767334)
\curveto(457.36796856,667.52767061)(457.38296854,667.47267067)(457.40297119,667.41267334)
\curveto(457.48296844,667.17267097)(457.57796835,666.97267117)(457.68797119,666.81267334)
\curveto(457.80796812,666.66267148)(457.96796796,666.52767161)(458.16797119,666.40767334)
\curveto(458.24796768,666.35767178)(458.3279676,666.32267182)(458.40797119,666.30267334)
\curveto(458.49796743,666.29267185)(458.58796734,666.27267187)(458.67797119,666.24267334)
\curveto(458.75796717,666.22267192)(458.86796706,666.20767193)(459.00797119,666.19767334)
\curveto(459.14796678,666.18767195)(459.26796666,666.19267195)(459.36797119,666.21267334)
\lineto(459.50297119,666.21267334)
\curveto(459.60296632,666.23267191)(459.69296623,666.25267189)(459.77297119,666.27267334)
\curveto(459.86296606,666.30267184)(459.94796598,666.33267181)(460.02797119,666.36267334)
\curveto(460.1279658,666.41267173)(460.23796569,666.47767166)(460.35797119,666.55767334)
\curveto(460.48796544,666.6376715)(460.58296534,666.71767142)(460.64297119,666.79767334)
\curveto(460.69296523,666.86767127)(460.74296518,666.93267121)(460.79297119,666.99267334)
\curveto(460.85296507,667.06267108)(460.922965,667.11267103)(461.00297119,667.14267334)
\curveto(461.10296482,667.19267095)(461.2279647,667.21267093)(461.37797119,667.20267334)
\lineto(461.81297119,667.20267334)
\lineto(461.99297119,667.20267334)
\curveto(462.06296386,667.21267093)(462.1229638,667.20767093)(462.17297119,667.18767334)
\lineto(462.32297119,667.18767334)
\curveto(462.4229635,667.16767097)(462.49296343,667.142671)(462.53297119,667.11267334)
\curveto(462.57296335,667.09267105)(462.59296333,667.04767109)(462.59297119,666.97767334)
\curveto(462.60296332,666.90767123)(462.59796333,666.84767129)(462.57797119,666.79767334)
\curveto(462.5279634,666.65767148)(462.47296345,666.53267161)(462.41297119,666.42267334)
\curveto(462.35296357,666.31267183)(462.28296364,666.20267194)(462.20297119,666.09267334)
\curveto(461.98296394,665.76267238)(461.73296419,665.49767264)(461.45297119,665.29767334)
\curveto(461.17296475,665.09767304)(460.8229651,664.92767321)(460.40297119,664.78767334)
\curveto(460.29296563,664.74767339)(460.18296574,664.72267342)(460.07297119,664.71267334)
\curveto(459.96296596,664.70267344)(459.84796608,664.68267346)(459.72797119,664.65267334)
\curveto(459.68796624,664.6426735)(459.64296628,664.6426735)(459.59297119,664.65267334)
\curveto(459.55296637,664.65267349)(459.51296641,664.64767349)(459.47297119,664.63767334)
\lineto(459.30797119,664.63767334)
\curveto(459.25796667,664.61767352)(459.19796673,664.61267353)(459.12797119,664.62267334)
\curveto(459.06796686,664.62267352)(459.01296691,664.62767351)(458.96297119,664.63767334)
\curveto(458.88296704,664.64767349)(458.81296711,664.64767349)(458.75297119,664.63767334)
\curveto(458.69296723,664.62767351)(458.6279673,664.63267351)(458.55797119,664.65267334)
\curveto(458.50796742,664.67267347)(458.45296747,664.68267346)(458.39297119,664.68267334)
\curveto(458.33296759,664.68267346)(458.27796765,664.69267345)(458.22797119,664.71267334)
\curveto(458.11796781,664.73267341)(458.00796792,664.75767338)(457.89797119,664.78767334)
\curveto(457.78796814,664.80767333)(457.68796824,664.8426733)(457.59797119,664.89267334)
\curveto(457.48796844,664.93267321)(457.38296854,664.96767317)(457.28297119,664.99767334)
\curveto(457.19296873,665.0376731)(457.10796882,665.08267306)(457.02797119,665.13267334)
\curveto(456.70796922,665.33267281)(456.4229695,665.56267258)(456.17297119,665.82267334)
\curveto(455.92297,666.09267205)(455.71797021,666.40267174)(455.55797119,666.75267334)
\curveto(455.50797042,666.86267128)(455.46797046,666.97267117)(455.43797119,667.08267334)
\curveto(455.40797052,667.20267094)(455.36797056,667.32267082)(455.31797119,667.44267334)
\curveto(455.30797062,667.48267066)(455.30297062,667.51767062)(455.30297119,667.54767334)
\curveto(455.30297062,667.58767055)(455.29797063,667.62767051)(455.28797119,667.66767334)
\curveto(455.24797068,667.78767035)(455.2229707,667.91767022)(455.21297119,668.05767334)
\lineto(455.18297119,668.47767334)
\curveto(455.18297074,668.52766961)(455.17797075,668.58266956)(455.16797119,668.64267334)
\curveto(455.16797076,668.70266944)(455.17297075,668.75766938)(455.18297119,668.80767334)
\lineto(455.18297119,668.98767334)
\lineto(455.22797119,669.34767334)
\curveto(455.26797066,669.51766862)(455.30297062,669.68266846)(455.33297119,669.84267334)
\curveto(455.36297056,670.00266814)(455.40797052,670.15266799)(455.46797119,670.29267334)
\curveto(455.89797003,671.33266681)(456.6279693,672.06766607)(457.65797119,672.49767334)
\curveto(457.79796813,672.55766558)(457.93796799,672.59766554)(458.07797119,672.61767334)
\curveto(458.2279677,672.64766549)(458.38296754,672.68266546)(458.54297119,672.72267334)
\curveto(458.6229673,672.73266541)(458.69796723,672.7376654)(458.76797119,672.73767334)
\curveto(458.83796709,672.7376654)(458.91296701,672.7426654)(458.99297119,672.75267334)
\curveto(459.50296642,672.76266538)(459.93796599,672.70266544)(460.29797119,672.57267334)
\curveto(460.66796526,672.45266569)(460.99796493,672.29266585)(461.28797119,672.09267334)
\curveto(461.37796455,672.03266611)(461.46796446,671.96266618)(461.55797119,671.88267334)
\curveto(461.64796428,671.81266633)(461.7279642,671.7376664)(461.79797119,671.65767334)
\curveto(461.8279641,671.60766653)(461.86796406,671.56766657)(461.91797119,671.53767334)
\curveto(461.99796393,671.42766671)(462.07296385,671.31266683)(462.14297119,671.19267334)
\curveto(462.21296371,671.08266706)(462.28796364,670.96766717)(462.36797119,670.84767334)
\curveto(462.41796351,670.75766738)(462.45796347,670.66266748)(462.48797119,670.56267334)
\curveto(462.5279634,670.47266767)(462.56796336,670.37266777)(462.60797119,670.26267334)
\curveto(462.65796327,670.13266801)(462.69796323,669.99766814)(462.72797119,669.85767334)
\curveto(462.75796317,669.71766842)(462.79296313,669.57766856)(462.83297119,669.43767334)
\curveto(462.85296307,669.35766878)(462.85796307,669.26766887)(462.84797119,669.16767334)
\curveto(462.84796308,669.07766906)(462.85796307,668.99266915)(462.87797119,668.91267334)
\lineto(462.87797119,668.74767334)
\moveto(460.62797119,669.63267334)
\curveto(460.69796523,669.73266841)(460.70296522,669.85266829)(460.64297119,669.99267334)
\curveto(460.59296533,670.142668)(460.55296537,670.25266789)(460.52297119,670.32267334)
\curveto(460.38296554,670.59266755)(460.19796573,670.79766734)(459.96797119,670.93767334)
\curveto(459.73796619,671.08766705)(459.41796651,671.16766697)(459.00797119,671.17767334)
\curveto(458.97796695,671.15766698)(458.94296698,671.15266699)(458.90297119,671.16267334)
\curveto(458.86296706,671.17266697)(458.8279671,671.17266697)(458.79797119,671.16267334)
\curveto(458.74796718,671.142667)(458.69296723,671.12766701)(458.63297119,671.11767334)
\curveto(458.57296735,671.11766702)(458.51796741,671.10766703)(458.46797119,671.08767334)
\curveto(458.0279679,670.94766719)(457.70296822,670.67266747)(457.49297119,670.26267334)
\curveto(457.47296845,670.22266792)(457.44796848,670.16766797)(457.41797119,670.09767334)
\curveto(457.39796853,670.0376681)(457.38296854,669.97266817)(457.37297119,669.90267334)
\curveto(457.36296856,669.8426683)(457.36296856,669.78266836)(457.37297119,669.72267334)
\curveto(457.39296853,669.66266848)(457.4279685,669.61266853)(457.47797119,669.57267334)
\curveto(457.55796837,669.52266862)(457.66796826,669.49766864)(457.80797119,669.49767334)
\lineto(458.21297119,669.49767334)
\lineto(459.87797119,669.49767334)
\lineto(460.31297119,669.49767334)
\curveto(460.47296545,669.50766863)(460.57796535,669.55266859)(460.62797119,669.63267334)
}
}
{
\newrgbcolor{curcolor}{0 0 0}
\pscustom[linestyle=none,fillstyle=solid,fillcolor=curcolor]
{
}
}
{
\newrgbcolor{curcolor}{0 0 0}
\pscustom[linestyle=none,fillstyle=solid,fillcolor=curcolor]
{
\newpath
\moveto(468.70640869,672.52767334)
\lineto(469.83140869,672.52767334)
\curveto(469.94140626,672.52766561)(470.04140616,672.52266562)(470.13140869,672.51267334)
\curveto(470.22140598,672.50266564)(470.28640591,672.46766567)(470.32640869,672.40767334)
\curveto(470.37640582,672.34766579)(470.40640579,672.26266588)(470.41640869,672.15267334)
\curveto(470.42640577,672.05266609)(470.43140577,671.94766619)(470.43140869,671.83767334)
\lineto(470.43140869,670.78767334)
\lineto(470.43140869,668.55267334)
\curveto(470.43140577,668.19266995)(470.44640575,667.85267029)(470.47640869,667.53267334)
\curveto(470.50640569,667.21267093)(470.5964056,666.94767119)(470.74640869,666.73767334)
\curveto(470.88640531,666.52767161)(471.11140509,666.37767176)(471.42140869,666.28767334)
\curveto(471.47140473,666.27767186)(471.51140469,666.27267187)(471.54140869,666.27267334)
\curveto(471.58140462,666.27267187)(471.62640457,666.26767187)(471.67640869,666.25767334)
\curveto(471.72640447,666.24767189)(471.78140442,666.2426719)(471.84140869,666.24267334)
\curveto(471.9014043,666.2426719)(471.94640425,666.24767189)(471.97640869,666.25767334)
\curveto(472.02640417,666.27767186)(472.06640413,666.28267186)(472.09640869,666.27267334)
\curveto(472.13640406,666.26267188)(472.17640402,666.26767187)(472.21640869,666.28767334)
\curveto(472.42640377,666.3376718)(472.59140361,666.40267174)(472.71140869,666.48267334)
\curveto(472.89140331,666.59267155)(473.03140317,666.73267141)(473.13140869,666.90267334)
\curveto(473.24140296,667.08267106)(473.31640288,667.27767086)(473.35640869,667.48767334)
\curveto(473.40640279,667.70767043)(473.43640276,667.94767019)(473.44640869,668.20767334)
\curveto(473.45640274,668.47766966)(473.46140274,668.75766938)(473.46140869,669.04767334)
\lineto(473.46140869,670.86267334)
\lineto(473.46140869,671.83767334)
\lineto(473.46140869,672.10767334)
\curveto(473.46140274,672.20766593)(473.48140272,672.28766585)(473.52140869,672.34767334)
\curveto(473.57140263,672.4376657)(473.64640255,672.48766565)(473.74640869,672.49767334)
\curveto(473.84640235,672.51766562)(473.96640223,672.52766561)(474.10640869,672.52767334)
\lineto(474.90140869,672.52767334)
\lineto(475.18640869,672.52767334)
\curveto(475.27640092,672.52766561)(475.35140085,672.50766563)(475.41140869,672.46767334)
\curveto(475.49140071,672.41766572)(475.53640066,672.3426658)(475.54640869,672.24267334)
\curveto(475.55640064,672.142666)(475.56140064,672.02766611)(475.56140869,671.89767334)
\lineto(475.56140869,670.75767334)
\lineto(475.56140869,666.54267334)
\lineto(475.56140869,665.47767334)
\lineto(475.56140869,665.17767334)
\curveto(475.56140064,665.07767306)(475.54140066,665.00267314)(475.50140869,664.95267334)
\curveto(475.45140075,664.87267327)(475.37640082,664.82767331)(475.27640869,664.81767334)
\curveto(475.17640102,664.80767333)(475.07140113,664.80267334)(474.96140869,664.80267334)
\lineto(474.15140869,664.80267334)
\curveto(474.04140216,664.80267334)(473.94140226,664.80767333)(473.85140869,664.81767334)
\curveto(473.77140243,664.82767331)(473.70640249,664.86767327)(473.65640869,664.93767334)
\curveto(473.63640256,664.96767317)(473.61640258,665.01267313)(473.59640869,665.07267334)
\curveto(473.58640261,665.13267301)(473.57140263,665.19267295)(473.55140869,665.25267334)
\curveto(473.54140266,665.31267283)(473.52640267,665.36767277)(473.50640869,665.41767334)
\curveto(473.48640271,665.46767267)(473.45640274,665.49767264)(473.41640869,665.50767334)
\curveto(473.3964028,665.52767261)(473.37140283,665.53267261)(473.34140869,665.52267334)
\curveto(473.31140289,665.51267263)(473.28640291,665.50267264)(473.26640869,665.49267334)
\curveto(473.196403,665.45267269)(473.13640306,665.40767273)(473.08640869,665.35767334)
\curveto(473.03640316,665.30767283)(472.98140322,665.26267288)(472.92140869,665.22267334)
\curveto(472.88140332,665.19267295)(472.84140336,665.15767298)(472.80140869,665.11767334)
\curveto(472.77140343,665.08767305)(472.73140347,665.05767308)(472.68140869,665.02767334)
\curveto(472.45140375,664.88767325)(472.18140402,664.77767336)(471.87140869,664.69767334)
\curveto(471.8014044,664.67767346)(471.73140447,664.66767347)(471.66140869,664.66767334)
\curveto(471.59140461,664.65767348)(471.51640468,664.6426735)(471.43640869,664.62267334)
\curveto(471.3964048,664.61267353)(471.35140485,664.61267353)(471.30140869,664.62267334)
\curveto(471.26140494,664.62267352)(471.22140498,664.61767352)(471.18140869,664.60767334)
\curveto(471.15140505,664.59767354)(471.08640511,664.59767354)(470.98640869,664.60767334)
\curveto(470.8964053,664.60767353)(470.83640536,664.61267353)(470.80640869,664.62267334)
\curveto(470.75640544,664.62267352)(470.70640549,664.62767351)(470.65640869,664.63767334)
\lineto(470.50640869,664.63767334)
\curveto(470.38640581,664.66767347)(470.27140593,664.69267345)(470.16140869,664.71267334)
\curveto(470.05140615,664.73267341)(469.94140626,664.76267338)(469.83140869,664.80267334)
\curveto(469.78140642,664.82267332)(469.73640646,664.8376733)(469.69640869,664.84767334)
\curveto(469.66640653,664.86767327)(469.62640657,664.88767325)(469.57640869,664.90767334)
\curveto(469.22640697,665.09767304)(468.94640725,665.36267278)(468.73640869,665.70267334)
\curveto(468.60640759,665.91267223)(468.51140769,666.16267198)(468.45140869,666.45267334)
\curveto(468.39140781,666.75267139)(468.35140785,667.06767107)(468.33140869,667.39767334)
\curveto(468.32140788,667.7376704)(468.31640788,668.08267006)(468.31640869,668.43267334)
\curveto(468.32640787,668.79266935)(468.33140787,669.14766899)(468.33140869,669.49767334)
\lineto(468.33140869,671.53767334)
\curveto(468.33140787,671.66766647)(468.32640787,671.81766632)(468.31640869,671.98767334)
\curveto(468.31640788,672.16766597)(468.34140786,672.29766584)(468.39140869,672.37767334)
\curveto(468.42140778,672.42766571)(468.48140772,672.47266567)(468.57140869,672.51267334)
\curveto(468.63140757,672.51266563)(468.67640752,672.51766562)(468.70640869,672.52767334)
}
}
{
\newrgbcolor{curcolor}{0 0 0}
\pscustom[linestyle=none,fillstyle=solid,fillcolor=curcolor]
{
\newpath
\moveto(480.16265869,672.75267334)
\curveto(480.91265419,672.77266537)(481.56265354,672.68766545)(482.11265869,672.49767334)
\curveto(482.67265243,672.31766582)(483.09765201,672.00266614)(483.38765869,671.55267334)
\curveto(483.45765165,671.4426667)(483.51765159,671.32766681)(483.56765869,671.20767334)
\curveto(483.62765148,671.09766704)(483.67765143,670.97266717)(483.71765869,670.83267334)
\curveto(483.73765137,670.77266737)(483.74765136,670.70766743)(483.74765869,670.63767334)
\curveto(483.74765136,670.56766757)(483.73765137,670.50766763)(483.71765869,670.45767334)
\curveto(483.67765143,670.39766774)(483.62265148,670.35766778)(483.55265869,670.33767334)
\curveto(483.5026516,670.31766782)(483.44265166,670.30766783)(483.37265869,670.30767334)
\lineto(483.16265869,670.30767334)
\lineto(482.50265869,670.30767334)
\curveto(482.43265267,670.30766783)(482.36265274,670.30266784)(482.29265869,670.29267334)
\curveto(482.22265288,670.29266785)(482.15765295,670.30266784)(482.09765869,670.32267334)
\curveto(481.99765311,670.3426678)(481.92265318,670.38266776)(481.87265869,670.44267334)
\curveto(481.82265328,670.50266764)(481.77765333,670.56266758)(481.73765869,670.62267334)
\lineto(481.61765869,670.83267334)
\curveto(481.58765352,670.91266723)(481.53765357,670.97766716)(481.46765869,671.02767334)
\curveto(481.36765374,671.10766703)(481.26765384,671.16766697)(481.16765869,671.20767334)
\curveto(481.07765403,671.24766689)(480.96265414,671.28266686)(480.82265869,671.31267334)
\curveto(480.75265435,671.33266681)(480.64765446,671.34766679)(480.50765869,671.35767334)
\curveto(480.37765473,671.36766677)(480.27765483,671.36266678)(480.20765869,671.34267334)
\lineto(480.10265869,671.34267334)
\lineto(479.95265869,671.31267334)
\curveto(479.91265519,671.31266683)(479.86765524,671.30766683)(479.81765869,671.29767334)
\curveto(479.64765546,671.24766689)(479.5076556,671.17766696)(479.39765869,671.08767334)
\curveto(479.29765581,671.00766713)(479.22765588,670.88266726)(479.18765869,670.71267334)
\curveto(479.16765594,670.6426675)(479.16765594,670.57766756)(479.18765869,670.51767334)
\curveto(479.2076559,670.45766768)(479.22765588,670.40766773)(479.24765869,670.36767334)
\curveto(479.31765579,670.24766789)(479.39765571,670.15266799)(479.48765869,670.08267334)
\curveto(479.58765552,670.01266813)(479.7026554,669.95266819)(479.83265869,669.90267334)
\curveto(480.02265508,669.82266832)(480.22765488,669.75266839)(480.44765869,669.69267334)
\lineto(481.13765869,669.54267334)
\curveto(481.37765373,669.50266864)(481.6076535,669.45266869)(481.82765869,669.39267334)
\curveto(482.05765305,669.3426688)(482.27265283,669.27766886)(482.47265869,669.19767334)
\curveto(482.56265254,669.15766898)(482.64765246,669.12266902)(482.72765869,669.09267334)
\curveto(482.81765229,669.07266907)(482.9026522,669.0376691)(482.98265869,668.98767334)
\curveto(483.17265193,668.86766927)(483.34265176,668.7376694)(483.49265869,668.59767334)
\curveto(483.65265145,668.45766968)(483.77765133,668.28266986)(483.86765869,668.07267334)
\curveto(483.89765121,668.00267014)(483.92265118,667.93267021)(483.94265869,667.86267334)
\curveto(483.96265114,667.79267035)(483.98265112,667.71767042)(484.00265869,667.63767334)
\curveto(484.01265109,667.57767056)(484.01765109,667.48267066)(484.01765869,667.35267334)
\curveto(484.02765108,667.23267091)(484.02765108,667.137671)(484.01765869,667.06767334)
\lineto(484.01765869,666.99267334)
\curveto(483.99765111,666.93267121)(483.98265112,666.87267127)(483.97265869,666.81267334)
\curveto(483.97265113,666.76267138)(483.96765114,666.71267143)(483.95765869,666.66267334)
\curveto(483.88765122,666.36267178)(483.77765133,666.09767204)(483.62765869,665.86767334)
\curveto(483.46765164,665.62767251)(483.27265183,665.43267271)(483.04265869,665.28267334)
\curveto(482.81265229,665.13267301)(482.55265255,665.00267314)(482.26265869,664.89267334)
\curveto(482.15265295,664.8426733)(482.03265307,664.80767333)(481.90265869,664.78767334)
\curveto(481.78265332,664.76767337)(481.66265344,664.7426734)(481.54265869,664.71267334)
\curveto(481.45265365,664.69267345)(481.35765375,664.68267346)(481.25765869,664.68267334)
\curveto(481.16765394,664.67267347)(481.07765403,664.65767348)(480.98765869,664.63767334)
\lineto(480.71765869,664.63767334)
\curveto(480.65765445,664.61767352)(480.55265455,664.60767353)(480.40265869,664.60767334)
\curveto(480.26265484,664.60767353)(480.16265494,664.61767352)(480.10265869,664.63767334)
\curveto(480.07265503,664.6376735)(480.03765507,664.6426735)(479.99765869,664.65267334)
\lineto(479.89265869,664.65267334)
\curveto(479.77265533,664.67267347)(479.65265545,664.68767345)(479.53265869,664.69767334)
\curveto(479.41265569,664.70767343)(479.29765581,664.72767341)(479.18765869,664.75767334)
\curveto(478.79765631,664.86767327)(478.45265665,664.99267315)(478.15265869,665.13267334)
\curveto(477.85265725,665.28267286)(477.59765751,665.50267264)(477.38765869,665.79267334)
\curveto(477.24765786,665.98267216)(477.12765798,666.20267194)(477.02765869,666.45267334)
\curveto(477.0076581,666.51267163)(476.98765812,666.59267155)(476.96765869,666.69267334)
\curveto(476.94765816,666.7426714)(476.93265817,666.81267133)(476.92265869,666.90267334)
\curveto(476.91265819,666.99267115)(476.91765819,667.06767107)(476.93765869,667.12767334)
\curveto(476.96765814,667.19767094)(477.01765809,667.24767089)(477.08765869,667.27767334)
\curveto(477.13765797,667.29767084)(477.19765791,667.30767083)(477.26765869,667.30767334)
\lineto(477.49265869,667.30767334)
\lineto(478.19765869,667.30767334)
\lineto(478.43765869,667.30767334)
\curveto(478.51765659,667.30767083)(478.58765652,667.29767084)(478.64765869,667.27767334)
\curveto(478.75765635,667.2376709)(478.82765628,667.17267097)(478.85765869,667.08267334)
\curveto(478.89765621,666.99267115)(478.94265616,666.89767124)(478.99265869,666.79767334)
\curveto(479.01265609,666.74767139)(479.04765606,666.68267146)(479.09765869,666.60267334)
\curveto(479.15765595,666.52267162)(479.2076559,666.47267167)(479.24765869,666.45267334)
\curveto(479.36765574,666.35267179)(479.48265562,666.27267187)(479.59265869,666.21267334)
\curveto(479.7026554,666.16267198)(479.84265526,666.11267203)(480.01265869,666.06267334)
\curveto(480.06265504,666.0426721)(480.11265499,666.03267211)(480.16265869,666.03267334)
\curveto(480.21265489,666.0426721)(480.26265484,666.0426721)(480.31265869,666.03267334)
\curveto(480.39265471,666.01267213)(480.47765463,666.00267214)(480.56765869,666.00267334)
\curveto(480.66765444,666.01267213)(480.75265435,666.02767211)(480.82265869,666.04767334)
\curveto(480.87265423,666.05767208)(480.91765419,666.06267208)(480.95765869,666.06267334)
\curveto(481.0076541,666.06267208)(481.05765405,666.07267207)(481.10765869,666.09267334)
\curveto(481.24765386,666.142672)(481.37265373,666.20267194)(481.48265869,666.27267334)
\curveto(481.6026535,666.3426718)(481.69765341,666.43267171)(481.76765869,666.54267334)
\curveto(481.81765329,666.62267152)(481.85765325,666.74767139)(481.88765869,666.91767334)
\curveto(481.9076532,666.98767115)(481.9076532,667.05267109)(481.88765869,667.11267334)
\curveto(481.86765324,667.17267097)(481.84765326,667.22267092)(481.82765869,667.26267334)
\curveto(481.75765335,667.40267074)(481.66765344,667.50767063)(481.55765869,667.57767334)
\curveto(481.45765365,667.64767049)(481.33765377,667.71267043)(481.19765869,667.77267334)
\curveto(481.0076541,667.85267029)(480.8076543,667.91767022)(480.59765869,667.96767334)
\curveto(480.38765472,668.01767012)(480.17765493,668.07267007)(479.96765869,668.13267334)
\curveto(479.88765522,668.15266999)(479.8026553,668.16766997)(479.71265869,668.17767334)
\curveto(479.63265547,668.18766995)(479.55265555,668.20266994)(479.47265869,668.22267334)
\curveto(479.15265595,668.31266983)(478.84765626,668.39766974)(478.55765869,668.47767334)
\curveto(478.26765684,668.56766957)(478.0026571,668.69766944)(477.76265869,668.86767334)
\curveto(477.48265762,669.06766907)(477.27765783,669.3376688)(477.14765869,669.67767334)
\curveto(477.12765798,669.74766839)(477.107658,669.8426683)(477.08765869,669.96267334)
\curveto(477.06765804,670.03266811)(477.05265805,670.11766802)(477.04265869,670.21767334)
\curveto(477.03265807,670.31766782)(477.03765807,670.40766773)(477.05765869,670.48767334)
\curveto(477.07765803,670.5376676)(477.08265802,670.57766756)(477.07265869,670.60767334)
\curveto(477.06265804,670.64766749)(477.06765804,670.69266745)(477.08765869,670.74267334)
\curveto(477.107658,670.85266729)(477.12765798,670.95266719)(477.14765869,671.04267334)
\curveto(477.17765793,671.142667)(477.21265789,671.2376669)(477.25265869,671.32767334)
\curveto(477.38265772,671.61766652)(477.56265754,671.85266629)(477.79265869,672.03267334)
\curveto(478.02265708,672.21266593)(478.28265682,672.35766578)(478.57265869,672.46767334)
\curveto(478.68265642,672.51766562)(478.79765631,672.55266559)(478.91765869,672.57267334)
\curveto(479.03765607,672.60266554)(479.16265594,672.63266551)(479.29265869,672.66267334)
\curveto(479.35265575,672.68266546)(479.41265569,672.69266545)(479.47265869,672.69267334)
\lineto(479.65265869,672.72267334)
\curveto(479.73265537,672.73266541)(479.81765529,672.7376654)(479.90765869,672.73767334)
\curveto(479.99765511,672.7376654)(480.08265502,672.7426654)(480.16265869,672.75267334)
}
}
{
\newrgbcolor{curcolor}{0 0 0}
\pscustom[linestyle=none,fillstyle=solid,fillcolor=curcolor]
{
\newpath
\moveto(485.66929932,672.52767334)
\lineto(486.79429932,672.52767334)
\curveto(486.90429688,672.52766561)(487.00429678,672.52266562)(487.09429932,672.51267334)
\curveto(487.1842966,672.50266564)(487.24929654,672.46766567)(487.28929932,672.40767334)
\curveto(487.33929645,672.34766579)(487.36929642,672.26266588)(487.37929932,672.15267334)
\curveto(487.3892964,672.05266609)(487.39429639,671.94766619)(487.39429932,671.83767334)
\lineto(487.39429932,670.78767334)
\lineto(487.39429932,668.55267334)
\curveto(487.39429639,668.19266995)(487.40929638,667.85267029)(487.43929932,667.53267334)
\curveto(487.46929632,667.21267093)(487.55929623,666.94767119)(487.70929932,666.73767334)
\curveto(487.84929594,666.52767161)(488.07429571,666.37767176)(488.38429932,666.28767334)
\curveto(488.43429535,666.27767186)(488.47429531,666.27267187)(488.50429932,666.27267334)
\curveto(488.54429524,666.27267187)(488.5892952,666.26767187)(488.63929932,666.25767334)
\curveto(488.6892951,666.24767189)(488.74429504,666.2426719)(488.80429932,666.24267334)
\curveto(488.86429492,666.2426719)(488.90929488,666.24767189)(488.93929932,666.25767334)
\curveto(488.9892948,666.27767186)(489.02929476,666.28267186)(489.05929932,666.27267334)
\curveto(489.09929469,666.26267188)(489.13929465,666.26767187)(489.17929932,666.28767334)
\curveto(489.3892944,666.3376718)(489.55429423,666.40267174)(489.67429932,666.48267334)
\curveto(489.85429393,666.59267155)(489.99429379,666.73267141)(490.09429932,666.90267334)
\curveto(490.20429358,667.08267106)(490.27929351,667.27767086)(490.31929932,667.48767334)
\curveto(490.36929342,667.70767043)(490.39929339,667.94767019)(490.40929932,668.20767334)
\curveto(490.41929337,668.47766966)(490.42429336,668.75766938)(490.42429932,669.04767334)
\lineto(490.42429932,670.86267334)
\lineto(490.42429932,671.83767334)
\lineto(490.42429932,672.10767334)
\curveto(490.42429336,672.20766593)(490.44429334,672.28766585)(490.48429932,672.34767334)
\curveto(490.53429325,672.4376657)(490.60929318,672.48766565)(490.70929932,672.49767334)
\curveto(490.80929298,672.51766562)(490.92929286,672.52766561)(491.06929932,672.52767334)
\lineto(491.86429932,672.52767334)
\lineto(492.14929932,672.52767334)
\curveto(492.23929155,672.52766561)(492.31429147,672.50766563)(492.37429932,672.46767334)
\curveto(492.45429133,672.41766572)(492.49929129,672.3426658)(492.50929932,672.24267334)
\curveto(492.51929127,672.142666)(492.52429126,672.02766611)(492.52429932,671.89767334)
\lineto(492.52429932,670.75767334)
\lineto(492.52429932,666.54267334)
\lineto(492.52429932,665.47767334)
\lineto(492.52429932,665.17767334)
\curveto(492.52429126,665.07767306)(492.50429128,665.00267314)(492.46429932,664.95267334)
\curveto(492.41429137,664.87267327)(492.33929145,664.82767331)(492.23929932,664.81767334)
\curveto(492.13929165,664.80767333)(492.03429175,664.80267334)(491.92429932,664.80267334)
\lineto(491.11429932,664.80267334)
\curveto(491.00429278,664.80267334)(490.90429288,664.80767333)(490.81429932,664.81767334)
\curveto(490.73429305,664.82767331)(490.66929312,664.86767327)(490.61929932,664.93767334)
\curveto(490.59929319,664.96767317)(490.57929321,665.01267313)(490.55929932,665.07267334)
\curveto(490.54929324,665.13267301)(490.53429325,665.19267295)(490.51429932,665.25267334)
\curveto(490.50429328,665.31267283)(490.4892933,665.36767277)(490.46929932,665.41767334)
\curveto(490.44929334,665.46767267)(490.41929337,665.49767264)(490.37929932,665.50767334)
\curveto(490.35929343,665.52767261)(490.33429345,665.53267261)(490.30429932,665.52267334)
\curveto(490.27429351,665.51267263)(490.24929354,665.50267264)(490.22929932,665.49267334)
\curveto(490.15929363,665.45267269)(490.09929369,665.40767273)(490.04929932,665.35767334)
\curveto(489.99929379,665.30767283)(489.94429384,665.26267288)(489.88429932,665.22267334)
\curveto(489.84429394,665.19267295)(489.80429398,665.15767298)(489.76429932,665.11767334)
\curveto(489.73429405,665.08767305)(489.69429409,665.05767308)(489.64429932,665.02767334)
\curveto(489.41429437,664.88767325)(489.14429464,664.77767336)(488.83429932,664.69767334)
\curveto(488.76429502,664.67767346)(488.69429509,664.66767347)(488.62429932,664.66767334)
\curveto(488.55429523,664.65767348)(488.47929531,664.6426735)(488.39929932,664.62267334)
\curveto(488.35929543,664.61267353)(488.31429547,664.61267353)(488.26429932,664.62267334)
\curveto(488.22429556,664.62267352)(488.1842956,664.61767352)(488.14429932,664.60767334)
\curveto(488.11429567,664.59767354)(488.04929574,664.59767354)(487.94929932,664.60767334)
\curveto(487.85929593,664.60767353)(487.79929599,664.61267353)(487.76929932,664.62267334)
\curveto(487.71929607,664.62267352)(487.66929612,664.62767351)(487.61929932,664.63767334)
\lineto(487.46929932,664.63767334)
\curveto(487.34929644,664.66767347)(487.23429655,664.69267345)(487.12429932,664.71267334)
\curveto(487.01429677,664.73267341)(486.90429688,664.76267338)(486.79429932,664.80267334)
\curveto(486.74429704,664.82267332)(486.69929709,664.8376733)(486.65929932,664.84767334)
\curveto(486.62929716,664.86767327)(486.5892972,664.88767325)(486.53929932,664.90767334)
\curveto(486.1892976,665.09767304)(485.90929788,665.36267278)(485.69929932,665.70267334)
\curveto(485.56929822,665.91267223)(485.47429831,666.16267198)(485.41429932,666.45267334)
\curveto(485.35429843,666.75267139)(485.31429847,667.06767107)(485.29429932,667.39767334)
\curveto(485.2842985,667.7376704)(485.27929851,668.08267006)(485.27929932,668.43267334)
\curveto(485.2892985,668.79266935)(485.29429849,669.14766899)(485.29429932,669.49767334)
\lineto(485.29429932,671.53767334)
\curveto(485.29429849,671.66766647)(485.2892985,671.81766632)(485.27929932,671.98767334)
\curveto(485.27929851,672.16766597)(485.30429848,672.29766584)(485.35429932,672.37767334)
\curveto(485.3842984,672.42766571)(485.44429834,672.47266567)(485.53429932,672.51267334)
\curveto(485.59429819,672.51266563)(485.63929815,672.51766562)(485.66929932,672.52767334)
}
}
{
\newrgbcolor{curcolor}{0 0 0}
\pscustom[linestyle=none,fillstyle=solid,fillcolor=curcolor]
{
\newpath
\moveto(501.20554932,665.40267334)
\curveto(501.22554147,665.29267285)(501.23554146,665.18267296)(501.23554932,665.07267334)
\curveto(501.24554145,664.96267318)(501.1955415,664.88767325)(501.08554932,664.84767334)
\curveto(501.02554167,664.81767332)(500.95554174,664.80267334)(500.87554932,664.80267334)
\lineto(500.63554932,664.80267334)
\lineto(499.82554932,664.80267334)
\lineto(499.55554932,664.80267334)
\curveto(499.47554322,664.81267333)(499.41054328,664.8376733)(499.36054932,664.87767334)
\curveto(499.2905434,664.91767322)(499.23554346,664.97267317)(499.19554932,665.04267334)
\curveto(499.16554353,665.12267302)(499.12054357,665.18767295)(499.06054932,665.23767334)
\curveto(499.04054365,665.25767288)(499.01554368,665.27267287)(498.98554932,665.28267334)
\curveto(498.95554374,665.30267284)(498.91554378,665.30767283)(498.86554932,665.29767334)
\curveto(498.81554388,665.27767286)(498.76554393,665.25267289)(498.71554932,665.22267334)
\curveto(498.67554402,665.19267295)(498.63054406,665.16767297)(498.58054932,665.14767334)
\curveto(498.53054416,665.10767303)(498.47554422,665.07267307)(498.41554932,665.04267334)
\lineto(498.23554932,664.95267334)
\curveto(498.10554459,664.89267325)(497.97054472,664.8426733)(497.83054932,664.80267334)
\curveto(497.690545,664.77267337)(497.54554515,664.7376734)(497.39554932,664.69767334)
\curveto(497.32554537,664.67767346)(497.25554544,664.66767347)(497.18554932,664.66767334)
\curveto(497.12554557,664.65767348)(497.06054563,664.64767349)(496.99054932,664.63767334)
\lineto(496.90054932,664.63767334)
\curveto(496.87054582,664.62767351)(496.84054585,664.62267352)(496.81054932,664.62267334)
\lineto(496.64554932,664.62267334)
\curveto(496.54554615,664.60267354)(496.44554625,664.60267354)(496.34554932,664.62267334)
\lineto(496.21054932,664.62267334)
\curveto(496.14054655,664.6426735)(496.07054662,664.65267349)(496.00054932,664.65267334)
\curveto(495.94054675,664.6426735)(495.88054681,664.64767349)(495.82054932,664.66767334)
\curveto(495.72054697,664.68767345)(495.62554707,664.70767343)(495.53554932,664.72767334)
\curveto(495.44554725,664.7376734)(495.36054733,664.76267338)(495.28054932,664.80267334)
\curveto(494.9905477,664.91267323)(494.74054795,665.05267309)(494.53054932,665.22267334)
\curveto(494.33054836,665.40267274)(494.17054852,665.6376725)(494.05054932,665.92767334)
\curveto(494.02054867,665.99767214)(493.9905487,666.07267207)(493.96054932,666.15267334)
\curveto(493.94054875,666.23267191)(493.92054877,666.31767182)(493.90054932,666.40767334)
\curveto(493.88054881,666.45767168)(493.87054882,666.50767163)(493.87054932,666.55767334)
\curveto(493.88054881,666.60767153)(493.88054881,666.65767148)(493.87054932,666.70767334)
\curveto(493.86054883,666.7376714)(493.85054884,666.79767134)(493.84054932,666.88767334)
\curveto(493.84054885,666.98767115)(493.84554885,667.05767108)(493.85554932,667.09767334)
\curveto(493.87554882,667.19767094)(493.88554881,667.28267086)(493.88554932,667.35267334)
\lineto(493.97554932,667.68267334)
\curveto(494.00554869,667.80267034)(494.04554865,667.90767023)(494.09554932,667.99767334)
\curveto(494.26554843,668.28766985)(494.46054823,668.50766963)(494.68054932,668.65767334)
\curveto(494.90054779,668.80766933)(495.18054751,668.9376692)(495.52054932,669.04767334)
\curveto(495.65054704,669.09766904)(495.78554691,669.13266901)(495.92554932,669.15267334)
\curveto(496.06554663,669.17266897)(496.20554649,669.19766894)(496.34554932,669.22767334)
\curveto(496.42554627,669.24766889)(496.51054618,669.25766888)(496.60054932,669.25767334)
\curveto(496.690546,669.26766887)(496.78054591,669.28266886)(496.87054932,669.30267334)
\curveto(496.94054575,669.32266882)(497.01054568,669.32766881)(497.08054932,669.31767334)
\curveto(497.15054554,669.31766882)(497.22554547,669.32766881)(497.30554932,669.34767334)
\curveto(497.37554532,669.36766877)(497.44554525,669.37766876)(497.51554932,669.37767334)
\curveto(497.58554511,669.37766876)(497.66054503,669.38766875)(497.74054932,669.40767334)
\curveto(497.95054474,669.45766868)(498.14054455,669.49766864)(498.31054932,669.52767334)
\curveto(498.4905442,669.56766857)(498.65054404,669.65766848)(498.79054932,669.79767334)
\curveto(498.88054381,669.88766825)(498.94054375,669.98766815)(498.97054932,670.09767334)
\curveto(498.98054371,670.12766801)(498.98054371,670.15266799)(498.97054932,670.17267334)
\curveto(498.97054372,670.19266795)(498.97554372,670.21266793)(498.98554932,670.23267334)
\curveto(498.9955437,670.25266789)(499.00054369,670.28266786)(499.00054932,670.32267334)
\lineto(499.00054932,670.41267334)
\lineto(498.97054932,670.53267334)
\curveto(498.97054372,670.57266757)(498.96554373,670.60766753)(498.95554932,670.63767334)
\curveto(498.85554384,670.9376672)(498.64554405,671.142667)(498.32554932,671.25267334)
\curveto(498.23554446,671.28266686)(498.12554457,671.30266684)(497.99554932,671.31267334)
\curveto(497.87554482,671.33266681)(497.75054494,671.3376668)(497.62054932,671.32767334)
\curveto(497.4905452,671.32766681)(497.36554533,671.31766682)(497.24554932,671.29767334)
\curveto(497.12554557,671.27766686)(497.02054567,671.25266689)(496.93054932,671.22267334)
\curveto(496.87054582,671.20266694)(496.81054588,671.17266697)(496.75054932,671.13267334)
\curveto(496.70054599,671.10266704)(496.65054604,671.06766707)(496.60054932,671.02767334)
\curveto(496.55054614,670.98766715)(496.4955462,670.93266721)(496.43554932,670.86267334)
\curveto(496.38554631,670.79266735)(496.35054634,670.72766741)(496.33054932,670.66767334)
\curveto(496.28054641,670.56766757)(496.23554646,670.47266767)(496.19554932,670.38267334)
\curveto(496.16554653,670.29266785)(496.0955466,670.23266791)(495.98554932,670.20267334)
\curveto(495.90554679,670.18266796)(495.82054687,670.17266797)(495.73054932,670.17267334)
\lineto(495.46054932,670.17267334)
\lineto(494.89054932,670.17267334)
\curveto(494.84054785,670.17266797)(494.7905479,670.16766797)(494.74054932,670.15767334)
\curveto(494.690548,670.15766798)(494.64554805,670.16266798)(494.60554932,670.17267334)
\lineto(494.47054932,670.17267334)
\curveto(494.45054824,670.18266796)(494.42554827,670.18766795)(494.39554932,670.18767334)
\curveto(494.36554833,670.18766795)(494.34054835,670.19766794)(494.32054932,670.21767334)
\curveto(494.24054845,670.2376679)(494.18554851,670.30266784)(494.15554932,670.41267334)
\curveto(494.14554855,670.46266768)(494.14554855,670.51266763)(494.15554932,670.56267334)
\curveto(494.16554853,670.61266753)(494.17554852,670.65766748)(494.18554932,670.69767334)
\curveto(494.21554848,670.80766733)(494.24554845,670.90766723)(494.27554932,670.99767334)
\curveto(494.31554838,671.09766704)(494.36054833,671.18766695)(494.41054932,671.26767334)
\lineto(494.50054932,671.41767334)
\lineto(494.59054932,671.56767334)
\curveto(494.67054802,671.67766646)(494.77054792,671.78266636)(494.89054932,671.88267334)
\curveto(494.91054778,671.89266625)(494.94054775,671.91766622)(494.98054932,671.95767334)
\curveto(495.03054766,671.99766614)(495.07554762,672.03266611)(495.11554932,672.06267334)
\curveto(495.15554754,672.09266605)(495.20054749,672.12266602)(495.25054932,672.15267334)
\curveto(495.42054727,672.26266588)(495.60054709,672.34766579)(495.79054932,672.40767334)
\curveto(495.98054671,672.47766566)(496.17554652,672.5426656)(496.37554932,672.60267334)
\curveto(496.4955462,672.63266551)(496.62054607,672.65266549)(496.75054932,672.66267334)
\curveto(496.88054581,672.67266547)(497.01054568,672.69266545)(497.14054932,672.72267334)
\curveto(497.18054551,672.73266541)(497.24054545,672.73266541)(497.32054932,672.72267334)
\curveto(497.41054528,672.71266543)(497.46554523,672.71766542)(497.48554932,672.73767334)
\curveto(497.8955448,672.74766539)(498.28554441,672.73266541)(498.65554932,672.69267334)
\curveto(499.03554366,672.65266549)(499.37554332,672.57766556)(499.67554932,672.46767334)
\curveto(499.98554271,672.35766578)(500.25054244,672.20766593)(500.47054932,672.01767334)
\curveto(500.690542,671.8376663)(500.86054183,671.60266654)(500.98054932,671.31267334)
\curveto(501.05054164,671.142667)(501.0905416,670.94766719)(501.10054932,670.72767334)
\curveto(501.11054158,670.50766763)(501.11554158,670.28266786)(501.11554932,670.05267334)
\lineto(501.11554932,666.70767334)
\lineto(501.11554932,666.12267334)
\curveto(501.11554158,665.93267221)(501.13554156,665.75767238)(501.17554932,665.59767334)
\curveto(501.18554151,665.56767257)(501.1905415,665.53267261)(501.19054932,665.49267334)
\curveto(501.1905415,665.46267268)(501.1955415,665.43267271)(501.20554932,665.40267334)
\moveto(499.00054932,667.71267334)
\curveto(499.01054368,667.76267038)(499.01554368,667.81767032)(499.01554932,667.87767334)
\curveto(499.01554368,667.94767019)(499.01054368,668.00767013)(499.00054932,668.05767334)
\curveto(498.98054371,668.11767002)(498.97054372,668.17266997)(498.97054932,668.22267334)
\curveto(498.97054372,668.27266987)(498.95054374,668.31266983)(498.91054932,668.34267334)
\curveto(498.86054383,668.38266976)(498.78554391,668.40266974)(498.68554932,668.40267334)
\curveto(498.64554405,668.39266975)(498.61054408,668.38266976)(498.58054932,668.37267334)
\curveto(498.55054414,668.37266977)(498.51554418,668.36766977)(498.47554932,668.35767334)
\curveto(498.40554429,668.3376698)(498.33054436,668.32266982)(498.25054932,668.31267334)
\curveto(498.17054452,668.30266984)(498.0905446,668.28766985)(498.01054932,668.26767334)
\curveto(497.98054471,668.25766988)(497.93554476,668.25266989)(497.87554932,668.25267334)
\curveto(497.74554495,668.22266992)(497.61554508,668.20266994)(497.48554932,668.19267334)
\curveto(497.35554534,668.18266996)(497.23054546,668.15766998)(497.11054932,668.11767334)
\curveto(497.03054566,668.09767004)(496.95554574,668.07767006)(496.88554932,668.05767334)
\curveto(496.81554588,668.04767009)(496.74554595,668.02767011)(496.67554932,667.99767334)
\curveto(496.46554623,667.90767023)(496.28554641,667.77267037)(496.13554932,667.59267334)
\curveto(495.9955467,667.41267073)(495.94554675,667.16267098)(495.98554932,666.84267334)
\curveto(496.00554669,666.67267147)(496.06054663,666.53267161)(496.15054932,666.42267334)
\curveto(496.22054647,666.31267183)(496.32554637,666.22267192)(496.46554932,666.15267334)
\curveto(496.60554609,666.09267205)(496.75554594,666.04767209)(496.91554932,666.01767334)
\curveto(497.08554561,665.98767215)(497.26054543,665.97767216)(497.44054932,665.98767334)
\curveto(497.63054506,666.00767213)(497.80554489,666.0426721)(497.96554932,666.09267334)
\curveto(498.22554447,666.17267197)(498.43054426,666.29767184)(498.58054932,666.46767334)
\curveto(498.73054396,666.64767149)(498.84554385,666.86767127)(498.92554932,667.12767334)
\curveto(498.94554375,667.19767094)(498.95554374,667.26767087)(498.95554932,667.33767334)
\curveto(498.96554373,667.41767072)(498.98054371,667.49767064)(499.00054932,667.57767334)
\lineto(499.00054932,667.71267334)
}
}
{
\newrgbcolor{curcolor}{0 0 0}
\pscustom[linestyle=none,fillstyle=solid,fillcolor=curcolor]
{
\newpath
\moveto(507.19383057,672.73767334)
\curveto(507.30382525,672.7376654)(507.39882516,672.72766541)(507.47883057,672.70767334)
\curveto(507.56882499,672.68766545)(507.63882492,672.6426655)(507.68883057,672.57267334)
\curveto(507.74882481,672.49266565)(507.77882478,672.35266579)(507.77883057,672.15267334)
\lineto(507.77883057,671.64267334)
\lineto(507.77883057,671.26767334)
\curveto(507.78882477,671.12766701)(507.77382478,671.01766712)(507.73383057,670.93767334)
\curveto(507.69382486,670.86766727)(507.63382492,670.82266732)(507.55383057,670.80267334)
\curveto(507.48382507,670.78266736)(507.39882516,670.77266737)(507.29883057,670.77267334)
\curveto(507.20882535,670.77266737)(507.10882545,670.77766736)(506.99883057,670.78767334)
\curveto(506.89882566,670.79766734)(506.80382575,670.79266735)(506.71383057,670.77267334)
\curveto(506.64382591,670.75266739)(506.57382598,670.7376674)(506.50383057,670.72767334)
\curveto(506.43382612,670.72766741)(506.36882619,670.71766742)(506.30883057,670.69767334)
\curveto(506.14882641,670.64766749)(505.98882657,670.57266757)(505.82883057,670.47267334)
\curveto(505.66882689,670.38266776)(505.54382701,670.27766786)(505.45383057,670.15767334)
\curveto(505.40382715,670.07766806)(505.34882721,669.99266815)(505.28883057,669.90267334)
\curveto(505.23882732,669.82266832)(505.18882737,669.7376684)(505.13883057,669.64767334)
\curveto(505.10882745,669.56766857)(505.07882748,669.48266866)(505.04883057,669.39267334)
\lineto(504.98883057,669.15267334)
\curveto(504.96882759,669.08266906)(504.9588276,669.00766913)(504.95883057,668.92767334)
\curveto(504.9588276,668.85766928)(504.94882761,668.78766935)(504.92883057,668.71767334)
\curveto(504.91882764,668.67766946)(504.91382764,668.6376695)(504.91383057,668.59767334)
\curveto(504.92382763,668.56766957)(504.92382763,668.5376696)(504.91383057,668.50767334)
\lineto(504.91383057,668.26767334)
\curveto(504.89382766,668.19766994)(504.88882767,668.11767002)(504.89883057,668.02767334)
\curveto(504.90882765,667.94767019)(504.91382764,667.86767027)(504.91383057,667.78767334)
\lineto(504.91383057,666.82767334)
\lineto(504.91383057,665.55267334)
\curveto(504.91382764,665.42267272)(504.90882765,665.30267284)(504.89883057,665.19267334)
\curveto(504.88882767,665.08267306)(504.8588277,664.99267315)(504.80883057,664.92267334)
\curveto(504.78882777,664.89267325)(504.7538278,664.86767327)(504.70383057,664.84767334)
\curveto(504.66382789,664.8376733)(504.61882794,664.82767331)(504.56883057,664.81767334)
\lineto(504.49383057,664.81767334)
\curveto(504.44382811,664.80767333)(504.38882817,664.80267334)(504.32883057,664.80267334)
\lineto(504.16383057,664.80267334)
\lineto(503.51883057,664.80267334)
\curveto(503.4588291,664.81267333)(503.39382916,664.81767332)(503.32383057,664.81767334)
\lineto(503.12883057,664.81767334)
\curveto(503.07882948,664.8376733)(503.02882953,664.85267329)(502.97883057,664.86267334)
\curveto(502.92882963,664.88267326)(502.89382966,664.91767322)(502.87383057,664.96767334)
\curveto(502.83382972,665.01767312)(502.80882975,665.08767305)(502.79883057,665.17767334)
\lineto(502.79883057,665.47767334)
\lineto(502.79883057,666.49767334)
\lineto(502.79883057,670.72767334)
\lineto(502.79883057,671.83767334)
\lineto(502.79883057,672.12267334)
\curveto(502.79882976,672.22266592)(502.81882974,672.30266584)(502.85883057,672.36267334)
\curveto(502.90882965,672.4426657)(502.98382957,672.49266565)(503.08383057,672.51267334)
\curveto(503.18382937,672.53266561)(503.30382925,672.5426656)(503.44383057,672.54267334)
\lineto(504.20883057,672.54267334)
\curveto(504.32882823,672.5426656)(504.43382812,672.53266561)(504.52383057,672.51267334)
\curveto(504.61382794,672.50266564)(504.68382787,672.45766568)(504.73383057,672.37767334)
\curveto(504.76382779,672.32766581)(504.77882778,672.25766588)(504.77883057,672.16767334)
\lineto(504.80883057,671.89767334)
\curveto(504.81882774,671.81766632)(504.83382772,671.7426664)(504.85383057,671.67267334)
\curveto(504.88382767,671.60266654)(504.93382762,671.56766657)(505.00383057,671.56767334)
\curveto(505.02382753,671.58766655)(505.04382751,671.59766654)(505.06383057,671.59767334)
\curveto(505.08382747,671.59766654)(505.10382745,671.60766653)(505.12383057,671.62767334)
\curveto(505.18382737,671.67766646)(505.23382732,671.73266641)(505.27383057,671.79267334)
\curveto(505.32382723,671.86266628)(505.38382717,671.92266622)(505.45383057,671.97267334)
\curveto(505.49382706,672.00266614)(505.52882703,672.03266611)(505.55883057,672.06267334)
\curveto(505.58882697,672.10266604)(505.62382693,672.137666)(505.66383057,672.16767334)
\lineto(505.93383057,672.34767334)
\curveto(506.03382652,672.40766573)(506.13382642,672.46266568)(506.23383057,672.51267334)
\curveto(506.33382622,672.55266559)(506.43382612,672.58766555)(506.53383057,672.61767334)
\lineto(506.86383057,672.70767334)
\curveto(506.89382566,672.71766542)(506.94882561,672.71766542)(507.02883057,672.70767334)
\curveto(507.11882544,672.70766543)(507.17382538,672.71766542)(507.19383057,672.73767334)
}
}
{
\newrgbcolor{curcolor}{0 0 0}
\pscustom[linestyle=none,fillstyle=solid,fillcolor=curcolor]
{
\newpath
\moveto(510.69890869,675.39267334)
\curveto(510.76890574,675.31266283)(510.80390571,675.19266295)(510.80390869,675.03267334)
\lineto(510.80390869,674.56767334)
\lineto(510.80390869,674.16267334)
\curveto(510.80390571,674.02266412)(510.76890574,673.92766421)(510.69890869,673.87767334)
\curveto(510.63890587,673.82766431)(510.55890595,673.79766434)(510.45890869,673.78767334)
\curveto(510.36890614,673.77766436)(510.26890624,673.77266437)(510.15890869,673.77267334)
\lineto(509.31890869,673.77267334)
\curveto(509.2089073,673.77266437)(509.1089074,673.77766436)(509.01890869,673.78767334)
\curveto(508.93890757,673.79766434)(508.86890764,673.82766431)(508.80890869,673.87767334)
\curveto(508.76890774,673.90766423)(508.73890777,673.96266418)(508.71890869,674.04267334)
\curveto(508.7089078,674.13266401)(508.69890781,674.22766391)(508.68890869,674.32767334)
\lineto(508.68890869,674.65767334)
\curveto(508.69890781,674.76766337)(508.70390781,674.86266328)(508.70390869,674.94267334)
\lineto(508.70390869,675.15267334)
\curveto(508.7139078,675.22266292)(508.73390778,675.28266286)(508.76390869,675.33267334)
\curveto(508.78390773,675.37266277)(508.8089077,675.40266274)(508.83890869,675.42267334)
\lineto(508.95890869,675.48267334)
\curveto(508.97890753,675.48266266)(509.00390751,675.48266266)(509.03390869,675.48267334)
\curveto(509.06390745,675.49266265)(509.08890742,675.49766264)(509.10890869,675.49767334)
\lineto(510.20390869,675.49767334)
\curveto(510.30390621,675.49766264)(510.39890611,675.49266265)(510.48890869,675.48267334)
\curveto(510.57890593,675.47266267)(510.64890586,675.4426627)(510.69890869,675.39267334)
\moveto(510.80390869,665.62767334)
\curveto(510.80390571,665.42767271)(510.79890571,665.25767288)(510.78890869,665.11767334)
\curveto(510.77890573,664.97767316)(510.68890582,664.88267326)(510.51890869,664.83267334)
\curveto(510.45890605,664.81267333)(510.39390612,664.80267334)(510.32390869,664.80267334)
\curveto(510.25390626,664.81267333)(510.17890633,664.81767332)(510.09890869,664.81767334)
\lineto(509.25890869,664.81767334)
\curveto(509.16890734,664.81767332)(509.07890743,664.82267332)(508.98890869,664.83267334)
\curveto(508.9089076,664.8426733)(508.84890766,664.87267327)(508.80890869,664.92267334)
\curveto(508.74890776,664.99267315)(508.7139078,665.07767306)(508.70390869,665.17767334)
\lineto(508.70390869,665.52267334)
\lineto(508.70390869,671.85267334)
\lineto(508.70390869,672.15267334)
\curveto(508.70390781,672.25266589)(508.72390779,672.33266581)(508.76390869,672.39267334)
\curveto(508.82390769,672.46266568)(508.9089076,672.50766563)(509.01890869,672.52767334)
\curveto(509.03890747,672.5376656)(509.06390745,672.5376656)(509.09390869,672.52767334)
\curveto(509.13390738,672.52766561)(509.16390735,672.53266561)(509.18390869,672.54267334)
\lineto(509.93390869,672.54267334)
\lineto(510.12890869,672.54267334)
\curveto(510.2089063,672.55266559)(510.27390624,672.55266559)(510.32390869,672.54267334)
\lineto(510.44390869,672.54267334)
\curveto(510.50390601,672.52266562)(510.55890595,672.50766563)(510.60890869,672.49767334)
\curveto(510.65890585,672.48766565)(510.69890581,672.45766568)(510.72890869,672.40767334)
\curveto(510.76890574,672.35766578)(510.78890572,672.28766585)(510.78890869,672.19767334)
\curveto(510.79890571,672.10766603)(510.80390571,672.01266613)(510.80390869,671.91267334)
\lineto(510.80390869,665.62767334)
}
}
{
\newrgbcolor{curcolor}{0 0 0}
\pscustom[linestyle=none,fillstyle=solid,fillcolor=curcolor]
{
\newpath
\moveto(520.23609619,668.98767334)
\curveto(520.25608762,668.92766921)(520.26608761,668.8426693)(520.26609619,668.73267334)
\curveto(520.26608761,668.62266952)(520.25608762,668.5376696)(520.23609619,668.47767334)
\lineto(520.23609619,668.32767334)
\curveto(520.21608766,668.24766989)(520.20608767,668.16766997)(520.20609619,668.08767334)
\curveto(520.21608766,668.00767013)(520.21108767,667.92767021)(520.19109619,667.84767334)
\curveto(520.17108771,667.77767036)(520.15608772,667.71267043)(520.14609619,667.65267334)
\curveto(520.13608774,667.59267055)(520.12608775,667.52767061)(520.11609619,667.45767334)
\curveto(520.0760878,667.34767079)(520.04108784,667.23267091)(520.01109619,667.11267334)
\curveto(519.9810879,667.00267114)(519.94108794,666.89767124)(519.89109619,666.79767334)
\curveto(519.6810882,666.31767182)(519.40608847,665.92767221)(519.06609619,665.62767334)
\curveto(518.72608915,665.32767281)(518.31608956,665.07767306)(517.83609619,664.87767334)
\curveto(517.71609016,664.82767331)(517.59109029,664.79267335)(517.46109619,664.77267334)
\curveto(517.34109054,664.7426734)(517.21609066,664.71267343)(517.08609619,664.68267334)
\curveto(517.03609084,664.66267348)(516.9810909,664.65267349)(516.92109619,664.65267334)
\curveto(516.86109102,664.65267349)(516.80609107,664.64767349)(516.75609619,664.63767334)
\lineto(516.65109619,664.63767334)
\curveto(516.62109126,664.62767351)(516.59109129,664.62267352)(516.56109619,664.62267334)
\curveto(516.51109137,664.61267353)(516.43109145,664.60767353)(516.32109619,664.60767334)
\curveto(516.21109167,664.59767354)(516.12609175,664.60267354)(516.06609619,664.62267334)
\lineto(515.91609619,664.62267334)
\curveto(515.86609201,664.63267351)(515.81109207,664.6376735)(515.75109619,664.63767334)
\curveto(515.70109218,664.62767351)(515.65109223,664.63267351)(515.60109619,664.65267334)
\curveto(515.56109232,664.66267348)(515.52109236,664.66767347)(515.48109619,664.66767334)
\curveto(515.45109243,664.66767347)(515.41109247,664.67267347)(515.36109619,664.68267334)
\curveto(515.26109262,664.71267343)(515.16109272,664.7376734)(515.06109619,664.75767334)
\curveto(514.96109292,664.77767336)(514.86609301,664.80767333)(514.77609619,664.84767334)
\curveto(514.65609322,664.88767325)(514.54109334,664.92767321)(514.43109619,664.96767334)
\curveto(514.33109355,665.00767313)(514.22609365,665.05767308)(514.11609619,665.11767334)
\curveto(513.76609411,665.32767281)(513.46609441,665.57267257)(513.21609619,665.85267334)
\curveto(512.96609491,666.13267201)(512.75609512,666.46767167)(512.58609619,666.85767334)
\curveto(512.53609534,666.94767119)(512.49609538,667.0426711)(512.46609619,667.14267334)
\curveto(512.44609543,667.2426709)(512.42109546,667.34767079)(512.39109619,667.45767334)
\curveto(512.37109551,667.50767063)(512.36109552,667.55267059)(512.36109619,667.59267334)
\curveto(512.36109552,667.63267051)(512.35109553,667.67767046)(512.33109619,667.72767334)
\curveto(512.31109557,667.80767033)(512.30109558,667.88767025)(512.30109619,667.96767334)
\curveto(512.30109558,668.05767008)(512.29109559,668.14267)(512.27109619,668.22267334)
\curveto(512.26109562,668.27266987)(512.25609562,668.31766982)(512.25609619,668.35767334)
\lineto(512.25609619,668.49267334)
\curveto(512.23609564,668.55266959)(512.22609565,668.6376695)(512.22609619,668.74767334)
\curveto(512.23609564,668.85766928)(512.25109563,668.9426692)(512.27109619,669.00267334)
\lineto(512.27109619,669.10767334)
\curveto(512.2810956,669.15766898)(512.2810956,669.20766893)(512.27109619,669.25767334)
\curveto(512.27109561,669.31766882)(512.2810956,669.37266877)(512.30109619,669.42267334)
\curveto(512.31109557,669.47266867)(512.31609556,669.51766862)(512.31609619,669.55767334)
\curveto(512.31609556,669.60766853)(512.32609555,669.65766848)(512.34609619,669.70767334)
\curveto(512.38609549,669.8376683)(512.42109546,669.96266818)(512.45109619,670.08267334)
\curveto(512.4810954,670.21266793)(512.52109536,670.3376678)(512.57109619,670.45767334)
\curveto(512.75109513,670.86766727)(512.96609491,671.20766693)(513.21609619,671.47767334)
\curveto(513.46609441,671.75766638)(513.77109411,672.01266613)(514.13109619,672.24267334)
\curveto(514.23109365,672.29266585)(514.33609354,672.3376658)(514.44609619,672.37767334)
\curveto(514.55609332,672.41766572)(514.66609321,672.46266568)(514.77609619,672.51267334)
\curveto(514.90609297,672.56266558)(515.04109284,672.59766554)(515.18109619,672.61767334)
\curveto(515.32109256,672.6376655)(515.46609241,672.66766547)(515.61609619,672.70767334)
\curveto(515.69609218,672.71766542)(515.77109211,672.72266542)(515.84109619,672.72267334)
\curveto(515.91109197,672.72266542)(515.9810919,672.72766541)(516.05109619,672.73767334)
\curveto(516.63109125,672.74766539)(517.13109075,672.68766545)(517.55109619,672.55767334)
\curveto(517.9810899,672.42766571)(518.36108952,672.24766589)(518.69109619,672.01767334)
\curveto(518.80108908,671.9376662)(518.91108897,671.84766629)(519.02109619,671.74767334)
\curveto(519.14108874,671.65766648)(519.24108864,671.55766658)(519.32109619,671.44767334)
\curveto(519.40108848,671.34766679)(519.47108841,671.24766689)(519.53109619,671.14767334)
\curveto(519.60108828,671.04766709)(519.67108821,670.9426672)(519.74109619,670.83267334)
\curveto(519.81108807,670.72266742)(519.86608801,670.60266754)(519.90609619,670.47267334)
\curveto(519.94608793,670.35266779)(519.99108789,670.22266792)(520.04109619,670.08267334)
\curveto(520.07108781,670.00266814)(520.09608778,669.91766822)(520.11609619,669.82767334)
\lineto(520.17609619,669.55767334)
\curveto(520.18608769,669.51766862)(520.19108769,669.47766866)(520.19109619,669.43767334)
\curveto(520.19108769,669.39766874)(520.19608768,669.35766878)(520.20609619,669.31767334)
\curveto(520.22608765,669.26766887)(520.23108765,669.21266893)(520.22109619,669.15267334)
\curveto(520.21108767,669.09266905)(520.21608766,669.0376691)(520.23609619,668.98767334)
\moveto(518.13609619,668.44767334)
\curveto(518.14608973,668.49766964)(518.15108973,668.56766957)(518.15109619,668.65767334)
\curveto(518.15108973,668.75766938)(518.14608973,668.83266931)(518.13609619,668.88267334)
\lineto(518.13609619,669.00267334)
\curveto(518.11608976,669.05266909)(518.10608977,669.10766903)(518.10609619,669.16767334)
\curveto(518.10608977,669.22766891)(518.10108978,669.28266886)(518.09109619,669.33267334)
\curveto(518.09108979,669.37266877)(518.08608979,669.40266874)(518.07609619,669.42267334)
\lineto(518.01609619,669.66267334)
\curveto(518.00608987,669.75266839)(517.98608989,669.8376683)(517.95609619,669.91767334)
\curveto(517.84609003,670.17766796)(517.71609016,670.39766774)(517.56609619,670.57767334)
\curveto(517.41609046,670.76766737)(517.21609066,670.91766722)(516.96609619,671.02767334)
\curveto(516.90609097,671.04766709)(516.84609103,671.06266708)(516.78609619,671.07267334)
\curveto(516.72609115,671.09266705)(516.66109122,671.11266703)(516.59109619,671.13267334)
\curveto(516.51109137,671.15266699)(516.42609145,671.15766698)(516.33609619,671.14767334)
\lineto(516.06609619,671.14767334)
\curveto(516.03609184,671.12766701)(516.00109188,671.11766702)(515.96109619,671.11767334)
\curveto(515.92109196,671.12766701)(515.88609199,671.12766701)(515.85609619,671.11767334)
\lineto(515.64609619,671.05767334)
\curveto(515.58609229,671.04766709)(515.53109235,671.02766711)(515.48109619,670.99767334)
\curveto(515.23109265,670.88766725)(515.02609285,670.72766741)(514.86609619,670.51767334)
\curveto(514.71609316,670.31766782)(514.59609328,670.08266806)(514.50609619,669.81267334)
\curveto(514.4760934,669.71266843)(514.45109343,669.60766853)(514.43109619,669.49767334)
\curveto(514.42109346,669.38766875)(514.40609347,669.27766886)(514.38609619,669.16767334)
\curveto(514.3760935,669.11766902)(514.37109351,669.06766907)(514.37109619,669.01767334)
\lineto(514.37109619,668.86767334)
\curveto(514.35109353,668.79766934)(514.34109354,668.69266945)(514.34109619,668.55267334)
\curveto(514.35109353,668.41266973)(514.36609351,668.30766983)(514.38609619,668.23767334)
\lineto(514.38609619,668.10267334)
\curveto(514.40609347,668.02267012)(514.42109346,667.9426702)(514.43109619,667.86267334)
\curveto(514.44109344,667.79267035)(514.45609342,667.71767042)(514.47609619,667.63767334)
\curveto(514.5760933,667.3376708)(514.6810932,667.09267105)(514.79109619,666.90267334)
\curveto(514.91109297,666.72267142)(515.09609278,666.55767158)(515.34609619,666.40767334)
\curveto(515.41609246,666.35767178)(515.49109239,666.31767182)(515.57109619,666.28767334)
\curveto(515.66109222,666.25767188)(515.75109213,666.23267191)(515.84109619,666.21267334)
\curveto(515.881092,666.20267194)(515.91609196,666.19767194)(515.94609619,666.19767334)
\curveto(515.9760919,666.20767193)(516.01109187,666.20767193)(516.05109619,666.19767334)
\lineto(516.17109619,666.16767334)
\curveto(516.22109166,666.16767197)(516.26609161,666.17267197)(516.30609619,666.18267334)
\lineto(516.42609619,666.18267334)
\curveto(516.50609137,666.20267194)(516.58609129,666.21767192)(516.66609619,666.22767334)
\curveto(516.74609113,666.2376719)(516.82109106,666.25767188)(516.89109619,666.28767334)
\curveto(517.15109073,666.38767175)(517.36109052,666.52267162)(517.52109619,666.69267334)
\curveto(517.6810902,666.86267128)(517.81609006,667.07267107)(517.92609619,667.32267334)
\curveto(517.96608991,667.42267072)(517.99608988,667.52267062)(518.01609619,667.62267334)
\curveto(518.03608984,667.72267042)(518.06108982,667.82767031)(518.09109619,667.93767334)
\curveto(518.10108978,667.97767016)(518.10608977,668.01267013)(518.10609619,668.04267334)
\curveto(518.10608977,668.08267006)(518.11108977,668.12267002)(518.12109619,668.16267334)
\lineto(518.12109619,668.29767334)
\curveto(518.12108976,668.34766979)(518.12608975,668.39766974)(518.13609619,668.44767334)
}
}
{
\newrgbcolor{curcolor}{0 0 0}
\pscustom[linestyle=none,fillstyle=solid,fillcolor=curcolor]
{
\newpath
\moveto(20.6807132,349.97696777)
\lineto(21.9557132,349.97696777)
\curveto(22.06571042,349.97695706)(22.17071031,349.97195707)(22.2707132,349.96196777)
\curveto(22.3807101,349.95195709)(22.46071002,349.91695712)(22.5107132,349.85696777)
\curveto(22.56070992,349.77695726)(22.5857099,349.67195737)(22.5857132,349.54196777)
\curveto(22.59570989,349.42195762)(22.60070988,349.29695774)(22.6007132,349.16696777)
\lineto(22.6007132,347.65196777)
\lineto(22.6007132,344.56196777)
\lineto(22.6007132,344.03696777)
\curveto(22.60070988,343.99696304)(22.59570989,343.95196309)(22.5857132,343.90196777)
\curveto(22.5857099,343.86196318)(22.59070989,343.82196322)(22.6007132,343.78196777)
\lineto(22.6007132,343.54196777)
\curveto(22.60070988,343.45196359)(22.59570989,343.35696368)(22.5857132,343.25696777)
\curveto(22.5857099,343.15696388)(22.59570989,343.06696397)(22.6157132,342.98696777)
\curveto(22.61570987,342.91696412)(22.62070986,342.86196418)(22.6307132,342.82196777)
\curveto(22.65070983,342.71196433)(22.66570982,342.60196444)(22.6757132,342.49196777)
\curveto(22.69570979,342.38196466)(22.72570976,342.27196477)(22.7657132,342.16196777)
\curveto(22.87570961,341.90196514)(23.01570947,341.68696535)(23.1857132,341.51696777)
\curveto(23.36570912,341.34696569)(23.60070888,341.21196583)(23.8907132,341.11196777)
\curveto(23.97070851,341.09196595)(24.05070843,341.07696596)(24.1307132,341.06696777)
\curveto(24.21070827,341.05696598)(24.29070819,341.041966)(24.3707132,341.02196777)
\curveto(24.42070806,341.00196604)(24.46570802,340.99196605)(24.5057132,340.99196777)
\curveto(24.54570794,341.00196604)(24.59070789,341.00196604)(24.6407132,340.99196777)
\curveto(24.6807078,340.98196606)(24.74570774,340.97696606)(24.8357132,340.97696777)
\curveto(24.92570756,340.98696605)(24.9857075,340.99696604)(25.0157132,341.00696777)
\lineto(25.2407132,341.00696777)
\curveto(25.32070716,341.02696601)(25.40070708,341.041966)(25.4807132,341.05196777)
\curveto(25.56070692,341.06196598)(25.63570685,341.07696596)(25.7057132,341.09696777)
\curveto(25.84570664,341.12696591)(25.95570653,341.16196588)(26.0357132,341.20196777)
\curveto(26.21570627,341.28196576)(26.37070611,341.38696565)(26.5007132,341.51696777)
\curveto(26.64070584,341.65696538)(26.75070573,341.81196523)(26.8307132,341.98196777)
\curveto(26.94070554,342.2419648)(27.00570548,342.54696449)(27.0257132,342.89696777)
\curveto(27.04570544,343.25696378)(27.05570543,343.62696341)(27.0557132,344.00696777)
\lineto(27.0557132,346.99196777)
\lineto(27.0557132,349.00196777)
\curveto(27.05570543,349.1419579)(27.05070543,349.29695774)(27.0407132,349.46696777)
\curveto(27.04070544,349.6369574)(27.07070541,349.76195728)(27.1307132,349.84196777)
\curveto(27.1807053,349.90195714)(27.25070523,349.9369571)(27.3407132,349.94696777)
\curveto(27.43070505,349.96695707)(27.53070495,349.97695706)(27.6407132,349.97696777)
\lineto(28.6007132,349.97696777)
\curveto(28.6807038,349.97695706)(28.75570373,349.97695706)(28.8257132,349.97696777)
\curveto(28.90570358,349.98695705)(28.9807035,349.98195706)(29.0507132,349.96196777)
\curveto(29.19070329,349.93195711)(29.2807032,349.88195716)(29.3207132,349.81196777)
\curveto(29.37070311,349.73195731)(29.39070309,349.61695742)(29.3807132,349.46696777)
\curveto(29.3807031,349.32695771)(29.3807031,349.19695784)(29.3807132,349.07696777)
\lineto(29.3807132,347.06696777)
\lineto(29.3807132,344.03696777)
\curveto(29.3807031,343.65696338)(29.37570311,343.28696375)(29.3657132,342.92696777)
\curveto(29.35570313,342.56696447)(29.31070317,342.2419648)(29.2307132,341.95196777)
\curveto(29.09070339,341.48196556)(28.91070357,341.07196597)(28.6907132,340.72196777)
\curveto(28.480704,340.38196666)(28.20070428,340.09196695)(27.8507132,339.85196777)
\curveto(27.54070494,339.63196741)(27.17570531,339.45196759)(26.7557132,339.31196777)
\curveto(26.66570582,339.28196776)(26.57070591,339.25696778)(26.4707132,339.23696777)
\lineto(26.2007132,339.17696777)
\curveto(26.14070634,339.15696788)(26.0807064,339.14696789)(26.0207132,339.14696777)
\curveto(25.97070651,339.14696789)(25.91570657,339.1369679)(25.8557132,339.11696777)
\curveto(25.73570675,339.09696794)(25.60070688,339.08196796)(25.4507132,339.07196777)
\curveto(25.30070718,339.06196798)(25.15570733,339.05696798)(25.0157132,339.05696777)
\curveto(24.06570842,339.04696799)(23.25570923,339.16196788)(22.5857132,339.40196777)
\curveto(21.91571057,339.65196739)(21.39071109,340.05196699)(21.0107132,340.60196777)
\curveto(20.8807116,340.78196626)(20.77071171,340.96696607)(20.6807132,341.15696777)
\curveto(20.60071188,341.35696568)(20.52571196,341.57196547)(20.4557132,341.80196777)
\curveto(20.43571205,341.85196519)(20.42571206,341.89196515)(20.4257132,341.92196777)
\curveto(20.42571206,341.96196508)(20.41571207,342.00696503)(20.3957132,342.05696777)
\curveto(20.31571217,342.3369647)(20.27571221,342.65196439)(20.2757132,343.00196777)
\lineto(20.2757132,344.05196777)
\lineto(20.2757132,348.23696777)
\lineto(20.2757132,349.28696777)
\lineto(20.2757132,349.57196777)
\curveto(20.27571221,349.67195737)(20.29071219,349.75195729)(20.3207132,349.81196777)
\curveto(20.3807121,349.88195716)(20.46071202,349.93195711)(20.5607132,349.96196777)
\curveto(20.5807119,349.96195708)(20.60071188,349.96195708)(20.6207132,349.96196777)
\curveto(20.64071184,349.96195708)(20.66071182,349.96695707)(20.6807132,349.97696777)
}
}
{
\newrgbcolor{curcolor}{0 0 0}
\pscustom[linestyle=none,fillstyle=solid,fillcolor=curcolor]
{
\newpath
\moveto(34.13922882,347.21696777)
\curveto(34.88922432,347.2369598)(35.53922367,347.15195989)(36.08922882,346.96196777)
\curveto(36.64922256,346.78196026)(37.07422214,346.46696057)(37.36422882,346.01696777)
\curveto(37.43422178,345.90696113)(37.49422172,345.79196125)(37.54422882,345.67196777)
\curveto(37.60422161,345.56196148)(37.65422156,345.4369616)(37.69422882,345.29696777)
\curveto(37.7142215,345.2369618)(37.72422149,345.17196187)(37.72422882,345.10196777)
\curveto(37.72422149,345.03196201)(37.7142215,344.97196207)(37.69422882,344.92196777)
\curveto(37.65422156,344.86196218)(37.59922161,344.82196222)(37.52922882,344.80196777)
\curveto(37.47922173,344.78196226)(37.41922179,344.77196227)(37.34922882,344.77196777)
\lineto(37.13922882,344.77196777)
\lineto(36.47922882,344.77196777)
\curveto(36.4092228,344.77196227)(36.33922287,344.76696227)(36.26922882,344.75696777)
\curveto(36.19922301,344.75696228)(36.13422308,344.76696227)(36.07422882,344.78696777)
\curveto(35.97422324,344.80696223)(35.89922331,344.84696219)(35.84922882,344.90696777)
\curveto(35.79922341,344.96696207)(35.75422346,345.02696201)(35.71422882,345.08696777)
\lineto(35.59422882,345.29696777)
\curveto(35.56422365,345.37696166)(35.5142237,345.4419616)(35.44422882,345.49196777)
\curveto(35.34422387,345.57196147)(35.24422397,345.63196141)(35.14422882,345.67196777)
\curveto(35.05422416,345.71196133)(34.93922427,345.74696129)(34.79922882,345.77696777)
\curveto(34.72922448,345.79696124)(34.62422459,345.81196123)(34.48422882,345.82196777)
\curveto(34.35422486,345.83196121)(34.25422496,345.82696121)(34.18422882,345.80696777)
\lineto(34.07922882,345.80696777)
\lineto(33.92922882,345.77696777)
\curveto(33.88922532,345.77696126)(33.84422537,345.77196127)(33.79422882,345.76196777)
\curveto(33.62422559,345.71196133)(33.48422573,345.6419614)(33.37422882,345.55196777)
\curveto(33.27422594,345.47196157)(33.20422601,345.34696169)(33.16422882,345.17696777)
\curveto(33.14422607,345.10696193)(33.14422607,345.041962)(33.16422882,344.98196777)
\curveto(33.18422603,344.92196212)(33.20422601,344.87196217)(33.22422882,344.83196777)
\curveto(33.29422592,344.71196233)(33.37422584,344.61696242)(33.46422882,344.54696777)
\curveto(33.56422565,344.47696256)(33.67922553,344.41696262)(33.80922882,344.36696777)
\curveto(33.99922521,344.28696275)(34.20422501,344.21696282)(34.42422882,344.15696777)
\lineto(35.11422882,344.00696777)
\curveto(35.35422386,343.96696307)(35.58422363,343.91696312)(35.80422882,343.85696777)
\curveto(36.03422318,343.80696323)(36.24922296,343.7419633)(36.44922882,343.66196777)
\curveto(36.53922267,343.62196342)(36.62422259,343.58696345)(36.70422882,343.55696777)
\curveto(36.79422242,343.5369635)(36.87922233,343.50196354)(36.95922882,343.45196777)
\curveto(37.14922206,343.33196371)(37.31922189,343.20196384)(37.46922882,343.06196777)
\curveto(37.62922158,342.92196412)(37.75422146,342.74696429)(37.84422882,342.53696777)
\curveto(37.87422134,342.46696457)(37.89922131,342.39696464)(37.91922882,342.32696777)
\curveto(37.93922127,342.25696478)(37.95922125,342.18196486)(37.97922882,342.10196777)
\curveto(37.98922122,342.041965)(37.99422122,341.94696509)(37.99422882,341.81696777)
\curveto(38.00422121,341.69696534)(38.00422121,341.60196544)(37.99422882,341.53196777)
\lineto(37.99422882,341.45696777)
\curveto(37.97422124,341.39696564)(37.95922125,341.3369657)(37.94922882,341.27696777)
\curveto(37.94922126,341.22696581)(37.94422127,341.17696586)(37.93422882,341.12696777)
\curveto(37.86422135,340.82696621)(37.75422146,340.56196648)(37.60422882,340.33196777)
\curveto(37.44422177,340.09196695)(37.24922196,339.89696714)(37.01922882,339.74696777)
\curveto(36.78922242,339.59696744)(36.52922268,339.46696757)(36.23922882,339.35696777)
\curveto(36.12922308,339.30696773)(36.0092232,339.27196777)(35.87922882,339.25196777)
\curveto(35.75922345,339.23196781)(35.63922357,339.20696783)(35.51922882,339.17696777)
\curveto(35.42922378,339.15696788)(35.33422388,339.14696789)(35.23422882,339.14696777)
\curveto(35.14422407,339.1369679)(35.05422416,339.12196792)(34.96422882,339.10196777)
\lineto(34.69422882,339.10196777)
\curveto(34.63422458,339.08196796)(34.52922468,339.07196797)(34.37922882,339.07196777)
\curveto(34.23922497,339.07196797)(34.13922507,339.08196796)(34.07922882,339.10196777)
\curveto(34.04922516,339.10196794)(34.0142252,339.10696793)(33.97422882,339.11696777)
\lineto(33.86922882,339.11696777)
\curveto(33.74922546,339.1369679)(33.62922558,339.15196789)(33.50922882,339.16196777)
\curveto(33.38922582,339.17196787)(33.27422594,339.19196785)(33.16422882,339.22196777)
\curveto(32.77422644,339.33196771)(32.42922678,339.45696758)(32.12922882,339.59696777)
\curveto(31.82922738,339.74696729)(31.57422764,339.96696707)(31.36422882,340.25696777)
\curveto(31.22422799,340.44696659)(31.10422811,340.66696637)(31.00422882,340.91696777)
\curveto(30.98422823,340.97696606)(30.96422825,341.05696598)(30.94422882,341.15696777)
\curveto(30.92422829,341.20696583)(30.9092283,341.27696576)(30.89922882,341.36696777)
\curveto(30.88922832,341.45696558)(30.89422832,341.53196551)(30.91422882,341.59196777)
\curveto(30.94422827,341.66196538)(30.99422822,341.71196533)(31.06422882,341.74196777)
\curveto(31.1142281,341.76196528)(31.17422804,341.77196527)(31.24422882,341.77196777)
\lineto(31.46922882,341.77196777)
\lineto(32.17422882,341.77196777)
\lineto(32.41422882,341.77196777)
\curveto(32.49422672,341.77196527)(32.56422665,341.76196528)(32.62422882,341.74196777)
\curveto(32.73422648,341.70196534)(32.80422641,341.6369654)(32.83422882,341.54696777)
\curveto(32.87422634,341.45696558)(32.91922629,341.36196568)(32.96922882,341.26196777)
\curveto(32.98922622,341.21196583)(33.02422619,341.14696589)(33.07422882,341.06696777)
\curveto(33.13422608,340.98696605)(33.18422603,340.9369661)(33.22422882,340.91696777)
\curveto(33.34422587,340.81696622)(33.45922575,340.7369663)(33.56922882,340.67696777)
\curveto(33.67922553,340.62696641)(33.81922539,340.57696646)(33.98922882,340.52696777)
\curveto(34.03922517,340.50696653)(34.08922512,340.49696654)(34.13922882,340.49696777)
\curveto(34.18922502,340.50696653)(34.23922497,340.50696653)(34.28922882,340.49696777)
\curveto(34.36922484,340.47696656)(34.45422476,340.46696657)(34.54422882,340.46696777)
\curveto(34.64422457,340.47696656)(34.72922448,340.49196655)(34.79922882,340.51196777)
\curveto(34.84922436,340.52196652)(34.89422432,340.52696651)(34.93422882,340.52696777)
\curveto(34.98422423,340.52696651)(35.03422418,340.5369665)(35.08422882,340.55696777)
\curveto(35.22422399,340.60696643)(35.34922386,340.66696637)(35.45922882,340.73696777)
\curveto(35.57922363,340.80696623)(35.67422354,340.89696614)(35.74422882,341.00696777)
\curveto(35.79422342,341.08696595)(35.83422338,341.21196583)(35.86422882,341.38196777)
\curveto(35.88422333,341.45196559)(35.88422333,341.51696552)(35.86422882,341.57696777)
\curveto(35.84422337,341.6369654)(35.82422339,341.68696535)(35.80422882,341.72696777)
\curveto(35.73422348,341.86696517)(35.64422357,341.97196507)(35.53422882,342.04196777)
\curveto(35.43422378,342.11196493)(35.3142239,342.17696486)(35.17422882,342.23696777)
\curveto(34.98422423,342.31696472)(34.78422443,342.38196466)(34.57422882,342.43196777)
\curveto(34.36422485,342.48196456)(34.15422506,342.5369645)(33.94422882,342.59696777)
\curveto(33.86422535,342.61696442)(33.77922543,342.63196441)(33.68922882,342.64196777)
\curveto(33.6092256,342.65196439)(33.52922568,342.66696437)(33.44922882,342.68696777)
\curveto(33.12922608,342.77696426)(32.82422639,342.86196418)(32.53422882,342.94196777)
\curveto(32.24422697,343.03196401)(31.97922723,343.16196388)(31.73922882,343.33196777)
\curveto(31.45922775,343.53196351)(31.25422796,343.80196324)(31.12422882,344.14196777)
\curveto(31.10422811,344.21196283)(31.08422813,344.30696273)(31.06422882,344.42696777)
\curveto(31.04422817,344.49696254)(31.02922818,344.58196246)(31.01922882,344.68196777)
\curveto(31.0092282,344.78196226)(31.0142282,344.87196217)(31.03422882,344.95196777)
\curveto(31.05422816,345.00196204)(31.05922815,345.041962)(31.04922882,345.07196777)
\curveto(31.03922817,345.11196193)(31.04422817,345.15696188)(31.06422882,345.20696777)
\curveto(31.08422813,345.31696172)(31.10422811,345.41696162)(31.12422882,345.50696777)
\curveto(31.15422806,345.60696143)(31.18922802,345.70196134)(31.22922882,345.79196777)
\curveto(31.35922785,346.08196096)(31.53922767,346.31696072)(31.76922882,346.49696777)
\curveto(31.99922721,346.67696036)(32.25922695,346.82196022)(32.54922882,346.93196777)
\curveto(32.65922655,346.98196006)(32.77422644,347.01696002)(32.89422882,347.03696777)
\curveto(33.0142262,347.06695997)(33.13922607,347.09695994)(33.26922882,347.12696777)
\curveto(33.32922588,347.14695989)(33.38922582,347.15695988)(33.44922882,347.15696777)
\lineto(33.62922882,347.18696777)
\curveto(33.7092255,347.19695984)(33.79422542,347.20195984)(33.88422882,347.20196777)
\curveto(33.97422524,347.20195984)(34.05922515,347.20695983)(34.13922882,347.21696777)
}
}
{
\newrgbcolor{curcolor}{0 0 0}
\pscustom[linestyle=none,fillstyle=solid,fillcolor=curcolor]
{
\newpath
\moveto(39.64586945,346.99196777)
\lineto(40.77086945,346.99196777)
\curveto(40.88086701,346.99196005)(40.98086691,346.98696005)(41.07086945,346.97696777)
\curveto(41.16086673,346.96696007)(41.22586667,346.93196011)(41.26586945,346.87196777)
\curveto(41.31586658,346.81196023)(41.34586655,346.72696031)(41.35586945,346.61696777)
\curveto(41.36586653,346.51696052)(41.37086652,346.41196063)(41.37086945,346.30196777)
\lineto(41.37086945,345.25196777)
\lineto(41.37086945,343.01696777)
\curveto(41.37086652,342.65696438)(41.38586651,342.31696472)(41.41586945,341.99696777)
\curveto(41.44586645,341.67696536)(41.53586636,341.41196563)(41.68586945,341.20196777)
\curveto(41.82586607,340.99196605)(42.05086584,340.8419662)(42.36086945,340.75196777)
\curveto(42.41086548,340.7419663)(42.45086544,340.7369663)(42.48086945,340.73696777)
\curveto(42.52086537,340.7369663)(42.56586533,340.73196631)(42.61586945,340.72196777)
\curveto(42.66586523,340.71196633)(42.72086517,340.70696633)(42.78086945,340.70696777)
\curveto(42.84086505,340.70696633)(42.88586501,340.71196633)(42.91586945,340.72196777)
\curveto(42.96586493,340.7419663)(43.00586489,340.74696629)(43.03586945,340.73696777)
\curveto(43.07586482,340.72696631)(43.11586478,340.73196631)(43.15586945,340.75196777)
\curveto(43.36586453,340.80196624)(43.53086436,340.86696617)(43.65086945,340.94696777)
\curveto(43.83086406,341.05696598)(43.97086392,341.19696584)(44.07086945,341.36696777)
\curveto(44.18086371,341.54696549)(44.25586364,341.7419653)(44.29586945,341.95196777)
\curveto(44.34586355,342.17196487)(44.37586352,342.41196463)(44.38586945,342.67196777)
\curveto(44.3958635,342.9419641)(44.40086349,343.22196382)(44.40086945,343.51196777)
\lineto(44.40086945,345.32696777)
\lineto(44.40086945,346.30196777)
\lineto(44.40086945,346.57196777)
\curveto(44.40086349,346.67196037)(44.42086347,346.75196029)(44.46086945,346.81196777)
\curveto(44.51086338,346.90196014)(44.58586331,346.95196009)(44.68586945,346.96196777)
\curveto(44.78586311,346.98196006)(44.90586299,346.99196005)(45.04586945,346.99196777)
\lineto(45.84086945,346.99196777)
\lineto(46.12586945,346.99196777)
\curveto(46.21586168,346.99196005)(46.2908616,346.97196007)(46.35086945,346.93196777)
\curveto(46.43086146,346.88196016)(46.47586142,346.80696023)(46.48586945,346.70696777)
\curveto(46.4958614,346.60696043)(46.50086139,346.49196055)(46.50086945,346.36196777)
\lineto(46.50086945,345.22196777)
\lineto(46.50086945,341.00696777)
\lineto(46.50086945,339.94196777)
\lineto(46.50086945,339.64196777)
\curveto(46.50086139,339.5419675)(46.48086141,339.46696757)(46.44086945,339.41696777)
\curveto(46.3908615,339.3369677)(46.31586158,339.29196775)(46.21586945,339.28196777)
\curveto(46.11586178,339.27196777)(46.01086188,339.26696777)(45.90086945,339.26696777)
\lineto(45.09086945,339.26696777)
\curveto(44.98086291,339.26696777)(44.88086301,339.27196777)(44.79086945,339.28196777)
\curveto(44.71086318,339.29196775)(44.64586325,339.33196771)(44.59586945,339.40196777)
\curveto(44.57586332,339.43196761)(44.55586334,339.47696756)(44.53586945,339.53696777)
\curveto(44.52586337,339.59696744)(44.51086338,339.65696738)(44.49086945,339.71696777)
\curveto(44.48086341,339.77696726)(44.46586343,339.83196721)(44.44586945,339.88196777)
\curveto(44.42586347,339.93196711)(44.3958635,339.96196708)(44.35586945,339.97196777)
\curveto(44.33586356,339.99196705)(44.31086358,339.99696704)(44.28086945,339.98696777)
\curveto(44.25086364,339.97696706)(44.22586367,339.96696707)(44.20586945,339.95696777)
\curveto(44.13586376,339.91696712)(44.07586382,339.87196717)(44.02586945,339.82196777)
\curveto(43.97586392,339.77196727)(43.92086397,339.72696731)(43.86086945,339.68696777)
\curveto(43.82086407,339.65696738)(43.78086411,339.62196742)(43.74086945,339.58196777)
\curveto(43.71086418,339.55196749)(43.67086422,339.52196752)(43.62086945,339.49196777)
\curveto(43.3908645,339.35196769)(43.12086477,339.2419678)(42.81086945,339.16196777)
\curveto(42.74086515,339.1419679)(42.67086522,339.13196791)(42.60086945,339.13196777)
\curveto(42.53086536,339.12196792)(42.45586544,339.10696793)(42.37586945,339.08696777)
\curveto(42.33586556,339.07696796)(42.2908656,339.07696796)(42.24086945,339.08696777)
\curveto(42.20086569,339.08696795)(42.16086573,339.08196796)(42.12086945,339.07196777)
\curveto(42.0908658,339.06196798)(42.02586587,339.06196798)(41.92586945,339.07196777)
\curveto(41.83586606,339.07196797)(41.77586612,339.07696796)(41.74586945,339.08696777)
\curveto(41.6958662,339.08696795)(41.64586625,339.09196795)(41.59586945,339.10196777)
\lineto(41.44586945,339.10196777)
\curveto(41.32586657,339.13196791)(41.21086668,339.15696788)(41.10086945,339.17696777)
\curveto(40.9908669,339.19696784)(40.88086701,339.22696781)(40.77086945,339.26696777)
\curveto(40.72086717,339.28696775)(40.67586722,339.30196774)(40.63586945,339.31196777)
\curveto(40.60586729,339.33196771)(40.56586733,339.35196769)(40.51586945,339.37196777)
\curveto(40.16586773,339.56196748)(39.88586801,339.82696721)(39.67586945,340.16696777)
\curveto(39.54586835,340.37696666)(39.45086844,340.62696641)(39.39086945,340.91696777)
\curveto(39.33086856,341.21696582)(39.2908686,341.53196551)(39.27086945,341.86196777)
\curveto(39.26086863,342.20196484)(39.25586864,342.54696449)(39.25586945,342.89696777)
\curveto(39.26586863,343.25696378)(39.27086862,343.61196343)(39.27086945,343.96196777)
\lineto(39.27086945,346.00196777)
\curveto(39.27086862,346.13196091)(39.26586863,346.28196076)(39.25586945,346.45196777)
\curveto(39.25586864,346.63196041)(39.28086861,346.76196028)(39.33086945,346.84196777)
\curveto(39.36086853,346.89196015)(39.42086847,346.9369601)(39.51086945,346.97696777)
\curveto(39.57086832,346.97696006)(39.61586828,346.98196006)(39.64586945,346.99196777)
}
}
{
\newrgbcolor{curcolor}{0 0 0}
\pscustom[linestyle=none,fillstyle=solid,fillcolor=curcolor]
{
\newpath
\moveto(55.18211945,339.86696777)
\curveto(55.2021116,339.75696728)(55.21211159,339.64696739)(55.21211945,339.53696777)
\curveto(55.22211158,339.42696761)(55.17211163,339.35196769)(55.06211945,339.31196777)
\curveto(55.0021118,339.28196776)(54.93211187,339.26696777)(54.85211945,339.26696777)
\lineto(54.61211945,339.26696777)
\lineto(53.80211945,339.26696777)
\lineto(53.53211945,339.26696777)
\curveto(53.45211335,339.27696776)(53.38711341,339.30196774)(53.33711945,339.34196777)
\curveto(53.26711353,339.38196766)(53.21211359,339.4369676)(53.17211945,339.50696777)
\curveto(53.14211366,339.58696745)(53.0971137,339.65196739)(53.03711945,339.70196777)
\curveto(53.01711378,339.72196732)(52.99211381,339.7369673)(52.96211945,339.74696777)
\curveto(52.93211387,339.76696727)(52.89211391,339.77196727)(52.84211945,339.76196777)
\curveto(52.79211401,339.7419673)(52.74211406,339.71696732)(52.69211945,339.68696777)
\curveto(52.65211415,339.65696738)(52.60711419,339.63196741)(52.55711945,339.61196777)
\curveto(52.50711429,339.57196747)(52.45211435,339.5369675)(52.39211945,339.50696777)
\lineto(52.21211945,339.41696777)
\curveto(52.08211472,339.35696768)(51.94711485,339.30696773)(51.80711945,339.26696777)
\curveto(51.66711513,339.2369678)(51.52211528,339.20196784)(51.37211945,339.16196777)
\curveto(51.3021155,339.1419679)(51.23211557,339.13196791)(51.16211945,339.13196777)
\curveto(51.1021157,339.12196792)(51.03711576,339.11196793)(50.96711945,339.10196777)
\lineto(50.87711945,339.10196777)
\curveto(50.84711595,339.09196795)(50.81711598,339.08696795)(50.78711945,339.08696777)
\lineto(50.62211945,339.08696777)
\curveto(50.52211628,339.06696797)(50.42211638,339.06696797)(50.32211945,339.08696777)
\lineto(50.18711945,339.08696777)
\curveto(50.11711668,339.10696793)(50.04711675,339.11696792)(49.97711945,339.11696777)
\curveto(49.91711688,339.10696793)(49.85711694,339.11196793)(49.79711945,339.13196777)
\curveto(49.6971171,339.15196789)(49.6021172,339.17196787)(49.51211945,339.19196777)
\curveto(49.42211738,339.20196784)(49.33711746,339.22696781)(49.25711945,339.26696777)
\curveto(48.96711783,339.37696766)(48.71711808,339.51696752)(48.50711945,339.68696777)
\curveto(48.30711849,339.86696717)(48.14711865,340.10196694)(48.02711945,340.39196777)
\curveto(47.9971188,340.46196658)(47.96711883,340.5369665)(47.93711945,340.61696777)
\curveto(47.91711888,340.69696634)(47.8971189,340.78196626)(47.87711945,340.87196777)
\curveto(47.85711894,340.92196612)(47.84711895,340.97196607)(47.84711945,341.02196777)
\curveto(47.85711894,341.07196597)(47.85711894,341.12196592)(47.84711945,341.17196777)
\curveto(47.83711896,341.20196584)(47.82711897,341.26196578)(47.81711945,341.35196777)
\curveto(47.81711898,341.45196559)(47.82211898,341.52196552)(47.83211945,341.56196777)
\curveto(47.85211895,341.66196538)(47.86211894,341.74696529)(47.86211945,341.81696777)
\lineto(47.95211945,342.14696777)
\curveto(47.98211882,342.26696477)(48.02211878,342.37196467)(48.07211945,342.46196777)
\curveto(48.24211856,342.75196429)(48.43711836,342.97196407)(48.65711945,343.12196777)
\curveto(48.87711792,343.27196377)(49.15711764,343.40196364)(49.49711945,343.51196777)
\curveto(49.62711717,343.56196348)(49.76211704,343.59696344)(49.90211945,343.61696777)
\curveto(50.04211676,343.6369634)(50.18211662,343.66196338)(50.32211945,343.69196777)
\curveto(50.4021164,343.71196333)(50.48711631,343.72196332)(50.57711945,343.72196777)
\curveto(50.66711613,343.73196331)(50.75711604,343.74696329)(50.84711945,343.76696777)
\curveto(50.91711588,343.78696325)(50.98711581,343.79196325)(51.05711945,343.78196777)
\curveto(51.12711567,343.78196326)(51.2021156,343.79196325)(51.28211945,343.81196777)
\curveto(51.35211545,343.83196321)(51.42211538,343.8419632)(51.49211945,343.84196777)
\curveto(51.56211524,343.8419632)(51.63711516,343.85196319)(51.71711945,343.87196777)
\curveto(51.92711487,343.92196312)(52.11711468,343.96196308)(52.28711945,343.99196777)
\curveto(52.46711433,344.03196301)(52.62711417,344.12196292)(52.76711945,344.26196777)
\curveto(52.85711394,344.35196269)(52.91711388,344.45196259)(52.94711945,344.56196777)
\curveto(52.95711384,344.59196245)(52.95711384,344.61696242)(52.94711945,344.63696777)
\curveto(52.94711385,344.65696238)(52.95211385,344.67696236)(52.96211945,344.69696777)
\curveto(52.97211383,344.71696232)(52.97711382,344.74696229)(52.97711945,344.78696777)
\lineto(52.97711945,344.87696777)
\lineto(52.94711945,344.99696777)
\curveto(52.94711385,345.036962)(52.94211386,345.07196197)(52.93211945,345.10196777)
\curveto(52.83211397,345.40196164)(52.62211418,345.60696143)(52.30211945,345.71696777)
\curveto(52.21211459,345.74696129)(52.1021147,345.76696127)(51.97211945,345.77696777)
\curveto(51.85211495,345.79696124)(51.72711507,345.80196124)(51.59711945,345.79196777)
\curveto(51.46711533,345.79196125)(51.34211546,345.78196126)(51.22211945,345.76196777)
\curveto(51.1021157,345.7419613)(50.9971158,345.71696132)(50.90711945,345.68696777)
\curveto(50.84711595,345.66696137)(50.78711601,345.6369614)(50.72711945,345.59696777)
\curveto(50.67711612,345.56696147)(50.62711617,345.53196151)(50.57711945,345.49196777)
\curveto(50.52711627,345.45196159)(50.47211633,345.39696164)(50.41211945,345.32696777)
\curveto(50.36211644,345.25696178)(50.32711647,345.19196185)(50.30711945,345.13196777)
\curveto(50.25711654,345.03196201)(50.21211659,344.9369621)(50.17211945,344.84696777)
\curveto(50.14211666,344.75696228)(50.07211673,344.69696234)(49.96211945,344.66696777)
\curveto(49.88211692,344.64696239)(49.797117,344.6369624)(49.70711945,344.63696777)
\lineto(49.43711945,344.63696777)
\lineto(48.86711945,344.63696777)
\curveto(48.81711798,344.6369624)(48.76711803,344.63196241)(48.71711945,344.62196777)
\curveto(48.66711813,344.62196242)(48.62211818,344.62696241)(48.58211945,344.63696777)
\lineto(48.44711945,344.63696777)
\curveto(48.42711837,344.64696239)(48.4021184,344.65196239)(48.37211945,344.65196777)
\curveto(48.34211846,344.65196239)(48.31711848,344.66196238)(48.29711945,344.68196777)
\curveto(48.21711858,344.70196234)(48.16211864,344.76696227)(48.13211945,344.87696777)
\curveto(48.12211868,344.92696211)(48.12211868,344.97696206)(48.13211945,345.02696777)
\curveto(48.14211866,345.07696196)(48.15211865,345.12196192)(48.16211945,345.16196777)
\curveto(48.19211861,345.27196177)(48.22211858,345.37196167)(48.25211945,345.46196777)
\curveto(48.29211851,345.56196148)(48.33711846,345.65196139)(48.38711945,345.73196777)
\lineto(48.47711945,345.88196777)
\lineto(48.56711945,346.03196777)
\curveto(48.64711815,346.1419609)(48.74711805,346.24696079)(48.86711945,346.34696777)
\curveto(48.88711791,346.35696068)(48.91711788,346.38196066)(48.95711945,346.42196777)
\curveto(49.00711779,346.46196058)(49.05211775,346.49696054)(49.09211945,346.52696777)
\curveto(49.13211767,346.55696048)(49.17711762,346.58696045)(49.22711945,346.61696777)
\curveto(49.3971174,346.72696031)(49.57711722,346.81196023)(49.76711945,346.87196777)
\curveto(49.95711684,346.9419601)(50.15211665,347.00696003)(50.35211945,347.06696777)
\curveto(50.47211633,347.09695994)(50.5971162,347.11695992)(50.72711945,347.12696777)
\curveto(50.85711594,347.1369599)(50.98711581,347.15695988)(51.11711945,347.18696777)
\curveto(51.15711564,347.19695984)(51.21711558,347.19695984)(51.29711945,347.18696777)
\curveto(51.38711541,347.17695986)(51.44211536,347.18195986)(51.46211945,347.20196777)
\curveto(51.87211493,347.21195983)(52.26211454,347.19695984)(52.63211945,347.15696777)
\curveto(53.01211379,347.11695992)(53.35211345,347.04196)(53.65211945,346.93196777)
\curveto(53.96211284,346.82196022)(54.22711257,346.67196037)(54.44711945,346.48196777)
\curveto(54.66711213,346.30196074)(54.83711196,346.06696097)(54.95711945,345.77696777)
\curveto(55.02711177,345.60696143)(55.06711173,345.41196163)(55.07711945,345.19196777)
\curveto(55.08711171,344.97196207)(55.09211171,344.74696229)(55.09211945,344.51696777)
\lineto(55.09211945,341.17196777)
\lineto(55.09211945,340.58696777)
\curveto(55.09211171,340.39696664)(55.11211169,340.22196682)(55.15211945,340.06196777)
\curveto(55.16211164,340.03196701)(55.16711163,339.99696704)(55.16711945,339.95696777)
\curveto(55.16711163,339.92696711)(55.17211163,339.89696714)(55.18211945,339.86696777)
\moveto(52.97711945,342.17696777)
\curveto(52.98711381,342.22696481)(52.99211381,342.28196476)(52.99211945,342.34196777)
\curveto(52.99211381,342.41196463)(52.98711381,342.47196457)(52.97711945,342.52196777)
\curveto(52.95711384,342.58196446)(52.94711385,342.6369644)(52.94711945,342.68696777)
\curveto(52.94711385,342.7369643)(52.92711387,342.77696426)(52.88711945,342.80696777)
\curveto(52.83711396,342.84696419)(52.76211404,342.86696417)(52.66211945,342.86696777)
\curveto(52.62211418,342.85696418)(52.58711421,342.84696419)(52.55711945,342.83696777)
\curveto(52.52711427,342.8369642)(52.49211431,342.83196421)(52.45211945,342.82196777)
\curveto(52.38211442,342.80196424)(52.30711449,342.78696425)(52.22711945,342.77696777)
\curveto(52.14711465,342.76696427)(52.06711473,342.75196429)(51.98711945,342.73196777)
\curveto(51.95711484,342.72196432)(51.91211489,342.71696432)(51.85211945,342.71696777)
\curveto(51.72211508,342.68696435)(51.59211521,342.66696437)(51.46211945,342.65696777)
\curveto(51.33211547,342.64696439)(51.20711559,342.62196442)(51.08711945,342.58196777)
\curveto(51.00711579,342.56196448)(50.93211587,342.5419645)(50.86211945,342.52196777)
\curveto(50.79211601,342.51196453)(50.72211608,342.49196455)(50.65211945,342.46196777)
\curveto(50.44211636,342.37196467)(50.26211654,342.2369648)(50.11211945,342.05696777)
\curveto(49.97211683,341.87696516)(49.92211688,341.62696541)(49.96211945,341.30696777)
\curveto(49.98211682,341.1369659)(50.03711676,340.99696604)(50.12711945,340.88696777)
\curveto(50.1971166,340.77696626)(50.3021165,340.68696635)(50.44211945,340.61696777)
\curveto(50.58211622,340.55696648)(50.73211607,340.51196653)(50.89211945,340.48196777)
\curveto(51.06211574,340.45196659)(51.23711556,340.4419666)(51.41711945,340.45196777)
\curveto(51.60711519,340.47196657)(51.78211502,340.50696653)(51.94211945,340.55696777)
\curveto(52.2021146,340.6369664)(52.40711439,340.76196628)(52.55711945,340.93196777)
\curveto(52.70711409,341.11196593)(52.82211398,341.33196571)(52.90211945,341.59196777)
\curveto(52.92211388,341.66196538)(52.93211387,341.73196531)(52.93211945,341.80196777)
\curveto(52.94211386,341.88196516)(52.95711384,341.96196508)(52.97711945,342.04196777)
\lineto(52.97711945,342.17696777)
}
}
{
\newrgbcolor{curcolor}{0 0 0}
\pscustom[linestyle=none,fillstyle=solid,fillcolor=curcolor]
{
\newpath
\moveto(61.1704007,347.20196777)
\curveto(61.28039538,347.20195984)(61.37539529,347.19195985)(61.4554007,347.17196777)
\curveto(61.54539512,347.15195989)(61.61539505,347.10695993)(61.6654007,347.03696777)
\curveto(61.72539494,346.95696008)(61.75539491,346.81696022)(61.7554007,346.61696777)
\lineto(61.7554007,346.10696777)
\lineto(61.7554007,345.73196777)
\curveto(61.7653949,345.59196145)(61.75039491,345.48196156)(61.7104007,345.40196777)
\curveto(61.67039499,345.33196171)(61.61039505,345.28696175)(61.5304007,345.26696777)
\curveto(61.4603952,345.24696179)(61.37539529,345.2369618)(61.2754007,345.23696777)
\curveto(61.18539548,345.2369618)(61.08539558,345.2419618)(60.9754007,345.25196777)
\curveto(60.87539579,345.26196178)(60.78039588,345.25696178)(60.6904007,345.23696777)
\curveto(60.62039604,345.21696182)(60.55039611,345.20196184)(60.4804007,345.19196777)
\curveto(60.41039625,345.19196185)(60.34539632,345.18196186)(60.2854007,345.16196777)
\curveto(60.12539654,345.11196193)(59.9653967,345.036962)(59.8054007,344.93696777)
\curveto(59.64539702,344.84696219)(59.52039714,344.7419623)(59.4304007,344.62196777)
\curveto(59.38039728,344.5419625)(59.32539734,344.45696258)(59.2654007,344.36696777)
\curveto(59.21539745,344.28696275)(59.1653975,344.20196284)(59.1154007,344.11196777)
\curveto(59.08539758,344.03196301)(59.05539761,343.94696309)(59.0254007,343.85696777)
\lineto(58.9654007,343.61696777)
\curveto(58.94539772,343.54696349)(58.93539773,343.47196357)(58.9354007,343.39196777)
\curveto(58.93539773,343.32196372)(58.92539774,343.25196379)(58.9054007,343.18196777)
\curveto(58.89539777,343.1419639)(58.89039777,343.10196394)(58.8904007,343.06196777)
\curveto(58.90039776,343.03196401)(58.90039776,343.00196404)(58.8904007,342.97196777)
\lineto(58.8904007,342.73196777)
\curveto(58.87039779,342.66196438)(58.8653978,342.58196446)(58.8754007,342.49196777)
\curveto(58.88539778,342.41196463)(58.89039777,342.33196471)(58.8904007,342.25196777)
\lineto(58.8904007,341.29196777)
\lineto(58.8904007,340.01696777)
\curveto(58.89039777,339.88696715)(58.88539778,339.76696727)(58.8754007,339.65696777)
\curveto(58.8653978,339.54696749)(58.83539783,339.45696758)(58.7854007,339.38696777)
\curveto(58.7653979,339.35696768)(58.73039793,339.33196771)(58.6804007,339.31196777)
\curveto(58.64039802,339.30196774)(58.59539807,339.29196775)(58.5454007,339.28196777)
\lineto(58.4704007,339.28196777)
\curveto(58.42039824,339.27196777)(58.3653983,339.26696777)(58.3054007,339.26696777)
\lineto(58.1404007,339.26696777)
\lineto(57.4954007,339.26696777)
\curveto(57.43539923,339.27696776)(57.37039929,339.28196776)(57.3004007,339.28196777)
\lineto(57.1054007,339.28196777)
\curveto(57.05539961,339.30196774)(57.00539966,339.31696772)(56.9554007,339.32696777)
\curveto(56.90539976,339.34696769)(56.87039979,339.38196766)(56.8504007,339.43196777)
\curveto(56.81039985,339.48196756)(56.78539988,339.55196749)(56.7754007,339.64196777)
\lineto(56.7754007,339.94196777)
\lineto(56.7754007,340.96196777)
\lineto(56.7754007,345.19196777)
\lineto(56.7754007,346.30196777)
\lineto(56.7754007,346.58696777)
\curveto(56.77539989,346.68696035)(56.79539987,346.76696027)(56.8354007,346.82696777)
\curveto(56.88539978,346.90696013)(56.9603997,346.95696008)(57.0604007,346.97696777)
\curveto(57.1603995,346.99696004)(57.28039938,347.00696003)(57.4204007,347.00696777)
\lineto(58.1854007,347.00696777)
\curveto(58.30539836,347.00696003)(58.41039825,346.99696004)(58.5004007,346.97696777)
\curveto(58.59039807,346.96696007)(58.660398,346.92196012)(58.7104007,346.84196777)
\curveto(58.74039792,346.79196025)(58.75539791,346.72196032)(58.7554007,346.63196777)
\lineto(58.7854007,346.36196777)
\curveto(58.79539787,346.28196076)(58.81039785,346.20696083)(58.8304007,346.13696777)
\curveto(58.8603978,346.06696097)(58.91039775,346.03196101)(58.9804007,346.03196777)
\curveto(59.00039766,346.05196099)(59.02039764,346.06196098)(59.0404007,346.06196777)
\curveto(59.0603976,346.06196098)(59.08039758,346.07196097)(59.1004007,346.09196777)
\curveto(59.1603975,346.1419609)(59.21039745,346.19696084)(59.2504007,346.25696777)
\curveto(59.30039736,346.32696071)(59.3603973,346.38696065)(59.4304007,346.43696777)
\curveto(59.47039719,346.46696057)(59.50539716,346.49696054)(59.5354007,346.52696777)
\curveto(59.5653971,346.56696047)(59.60039706,346.60196044)(59.6404007,346.63196777)
\lineto(59.9104007,346.81196777)
\curveto(60.01039665,346.87196017)(60.11039655,346.92696011)(60.2104007,346.97696777)
\curveto(60.31039635,347.01696002)(60.41039625,347.05195999)(60.5104007,347.08196777)
\lineto(60.8404007,347.17196777)
\curveto(60.87039579,347.18195986)(60.92539574,347.18195986)(61.0054007,347.17196777)
\curveto(61.09539557,347.17195987)(61.15039551,347.18195986)(61.1704007,347.20196777)
}
}
{
\newrgbcolor{curcolor}{0 0 0}
\pscustom[linestyle=none,fillstyle=solid,fillcolor=curcolor]
{
\newpath
\moveto(64.67547882,349.85696777)
\curveto(64.74547587,349.77695726)(64.78047584,349.65695738)(64.78047882,349.49696777)
\lineto(64.78047882,349.03196777)
\lineto(64.78047882,348.62696777)
\curveto(64.78047584,348.48695855)(64.74547587,348.39195865)(64.67547882,348.34196777)
\curveto(64.615476,348.29195875)(64.53547608,348.26195878)(64.43547882,348.25196777)
\curveto(64.34547627,348.2419588)(64.24547637,348.2369588)(64.13547882,348.23696777)
\lineto(63.29547882,348.23696777)
\curveto(63.18547743,348.2369588)(63.08547753,348.2419588)(62.99547882,348.25196777)
\curveto(62.9154777,348.26195878)(62.84547777,348.29195875)(62.78547882,348.34196777)
\curveto(62.74547787,348.37195867)(62.7154779,348.42695861)(62.69547882,348.50696777)
\curveto(62.68547793,348.59695844)(62.67547794,348.69195835)(62.66547882,348.79196777)
\lineto(62.66547882,349.12196777)
\curveto(62.67547794,349.23195781)(62.68047794,349.32695771)(62.68047882,349.40696777)
\lineto(62.68047882,349.61696777)
\curveto(62.69047793,349.68695735)(62.71047791,349.74695729)(62.74047882,349.79696777)
\curveto(62.76047786,349.8369572)(62.78547783,349.86695717)(62.81547882,349.88696777)
\lineto(62.93547882,349.94696777)
\curveto(62.95547766,349.94695709)(62.98047764,349.94695709)(63.01047882,349.94696777)
\curveto(63.04047758,349.95695708)(63.06547755,349.96195708)(63.08547882,349.96196777)
\lineto(64.18047882,349.96196777)
\curveto(64.28047634,349.96195708)(64.37547624,349.95695708)(64.46547882,349.94696777)
\curveto(64.55547606,349.9369571)(64.62547599,349.90695713)(64.67547882,349.85696777)
\moveto(64.78047882,340.09196777)
\curveto(64.78047584,339.89196715)(64.77547584,339.72196732)(64.76547882,339.58196777)
\curveto(64.75547586,339.4419676)(64.66547595,339.34696769)(64.49547882,339.29696777)
\curveto(64.43547618,339.27696776)(64.37047625,339.26696777)(64.30047882,339.26696777)
\curveto(64.23047639,339.27696776)(64.15547646,339.28196776)(64.07547882,339.28196777)
\lineto(63.23547882,339.28196777)
\curveto(63.14547747,339.28196776)(63.05547756,339.28696775)(62.96547882,339.29696777)
\curveto(62.88547773,339.30696773)(62.82547779,339.3369677)(62.78547882,339.38696777)
\curveto(62.72547789,339.45696758)(62.69047793,339.5419675)(62.68047882,339.64196777)
\lineto(62.68047882,339.98696777)
\lineto(62.68047882,346.31696777)
\lineto(62.68047882,346.61696777)
\curveto(62.68047794,346.71696032)(62.70047792,346.79696024)(62.74047882,346.85696777)
\curveto(62.80047782,346.92696011)(62.88547773,346.97196007)(62.99547882,346.99196777)
\curveto(63.0154776,347.00196004)(63.04047758,347.00196004)(63.07047882,346.99196777)
\curveto(63.11047751,346.99196005)(63.14047748,346.99696004)(63.16047882,347.00696777)
\lineto(63.91047882,347.00696777)
\lineto(64.10547882,347.00696777)
\curveto(64.18547643,347.01696002)(64.25047637,347.01696002)(64.30047882,347.00696777)
\lineto(64.42047882,347.00696777)
\curveto(64.48047614,346.98696005)(64.53547608,346.97196007)(64.58547882,346.96196777)
\curveto(64.63547598,346.95196009)(64.67547594,346.92196012)(64.70547882,346.87196777)
\curveto(64.74547587,346.82196022)(64.76547585,346.75196029)(64.76547882,346.66196777)
\curveto(64.77547584,346.57196047)(64.78047584,346.47696056)(64.78047882,346.37696777)
\lineto(64.78047882,340.09196777)
}
}
{
\newrgbcolor{curcolor}{0 0 0}
\pscustom[linestyle=none,fillstyle=solid,fillcolor=curcolor]
{
\newpath
\moveto(74.21266632,343.45196777)
\curveto(74.23265775,343.39196365)(74.24265774,343.30696373)(74.24266632,343.19696777)
\curveto(74.24265774,343.08696395)(74.23265775,343.00196404)(74.21266632,342.94196777)
\lineto(74.21266632,342.79196777)
\curveto(74.19265779,342.71196433)(74.1826578,342.63196441)(74.18266632,342.55196777)
\curveto(74.19265779,342.47196457)(74.1876578,342.39196465)(74.16766632,342.31196777)
\curveto(74.14765784,342.2419648)(74.13265785,342.17696486)(74.12266632,342.11696777)
\curveto(74.11265787,342.05696498)(74.10265788,341.99196505)(74.09266632,341.92196777)
\curveto(74.05265793,341.81196523)(74.01765797,341.69696534)(73.98766632,341.57696777)
\curveto(73.95765803,341.46696557)(73.91765807,341.36196568)(73.86766632,341.26196777)
\curveto(73.65765833,340.78196626)(73.3826586,340.39196665)(73.04266632,340.09196777)
\curveto(72.70265928,339.79196725)(72.29265969,339.5419675)(71.81266632,339.34196777)
\curveto(71.69266029,339.29196775)(71.56766042,339.25696778)(71.43766632,339.23696777)
\curveto(71.31766067,339.20696783)(71.19266079,339.17696786)(71.06266632,339.14696777)
\curveto(71.01266097,339.12696791)(70.95766103,339.11696792)(70.89766632,339.11696777)
\curveto(70.83766115,339.11696792)(70.7826612,339.11196793)(70.73266632,339.10196777)
\lineto(70.62766632,339.10196777)
\curveto(70.59766139,339.09196795)(70.56766142,339.08696795)(70.53766632,339.08696777)
\curveto(70.4876615,339.07696796)(70.40766158,339.07196797)(70.29766632,339.07196777)
\curveto(70.1876618,339.06196798)(70.10266188,339.06696797)(70.04266632,339.08696777)
\lineto(69.89266632,339.08696777)
\curveto(69.84266214,339.09696794)(69.7876622,339.10196794)(69.72766632,339.10196777)
\curveto(69.67766231,339.09196795)(69.62766236,339.09696794)(69.57766632,339.11696777)
\curveto(69.53766245,339.12696791)(69.49766249,339.13196791)(69.45766632,339.13196777)
\curveto(69.42766256,339.13196791)(69.3876626,339.1369679)(69.33766632,339.14696777)
\curveto(69.23766275,339.17696786)(69.13766285,339.20196784)(69.03766632,339.22196777)
\curveto(68.93766305,339.2419678)(68.84266314,339.27196777)(68.75266632,339.31196777)
\curveto(68.63266335,339.35196769)(68.51766347,339.39196765)(68.40766632,339.43196777)
\curveto(68.30766368,339.47196757)(68.20266378,339.52196752)(68.09266632,339.58196777)
\curveto(67.74266424,339.79196725)(67.44266454,340.036967)(67.19266632,340.31696777)
\curveto(66.94266504,340.59696644)(66.73266525,340.93196611)(66.56266632,341.32196777)
\curveto(66.51266547,341.41196563)(66.47266551,341.50696553)(66.44266632,341.60696777)
\curveto(66.42266556,341.70696533)(66.39766559,341.81196523)(66.36766632,341.92196777)
\curveto(66.34766564,341.97196507)(66.33766565,342.01696502)(66.33766632,342.05696777)
\curveto(66.33766565,342.09696494)(66.32766566,342.1419649)(66.30766632,342.19196777)
\curveto(66.2876657,342.27196477)(66.27766571,342.35196469)(66.27766632,342.43196777)
\curveto(66.27766571,342.52196452)(66.26766572,342.60696443)(66.24766632,342.68696777)
\curveto(66.23766575,342.7369643)(66.23266575,342.78196426)(66.23266632,342.82196777)
\lineto(66.23266632,342.95696777)
\curveto(66.21266577,343.01696402)(66.20266578,343.10196394)(66.20266632,343.21196777)
\curveto(66.21266577,343.32196372)(66.22766576,343.40696363)(66.24766632,343.46696777)
\lineto(66.24766632,343.57196777)
\curveto(66.25766573,343.62196342)(66.25766573,343.67196337)(66.24766632,343.72196777)
\curveto(66.24766574,343.78196326)(66.25766573,343.8369632)(66.27766632,343.88696777)
\curveto(66.2876657,343.9369631)(66.29266569,343.98196306)(66.29266632,344.02196777)
\curveto(66.29266569,344.07196297)(66.30266568,344.12196292)(66.32266632,344.17196777)
\curveto(66.36266562,344.30196274)(66.39766559,344.42696261)(66.42766632,344.54696777)
\curveto(66.45766553,344.67696236)(66.49766549,344.80196224)(66.54766632,344.92196777)
\curveto(66.72766526,345.33196171)(66.94266504,345.67196137)(67.19266632,345.94196777)
\curveto(67.44266454,346.22196082)(67.74766424,346.47696056)(68.10766632,346.70696777)
\curveto(68.20766378,346.75696028)(68.31266367,346.80196024)(68.42266632,346.84196777)
\curveto(68.53266345,346.88196016)(68.64266334,346.92696011)(68.75266632,346.97696777)
\curveto(68.8826631,347.02696001)(69.01766297,347.06195998)(69.15766632,347.08196777)
\curveto(69.29766269,347.10195994)(69.44266254,347.13195991)(69.59266632,347.17196777)
\curveto(69.67266231,347.18195986)(69.74766224,347.18695985)(69.81766632,347.18696777)
\curveto(69.8876621,347.18695985)(69.95766203,347.19195985)(70.02766632,347.20196777)
\curveto(70.60766138,347.21195983)(71.10766088,347.15195989)(71.52766632,347.02196777)
\curveto(71.95766003,346.89196015)(72.33765965,346.71196033)(72.66766632,346.48196777)
\curveto(72.77765921,346.40196064)(72.8876591,346.31196073)(72.99766632,346.21196777)
\curveto(73.11765887,346.12196092)(73.21765877,346.02196102)(73.29766632,345.91196777)
\curveto(73.37765861,345.81196123)(73.44765854,345.71196133)(73.50766632,345.61196777)
\curveto(73.57765841,345.51196153)(73.64765834,345.40696163)(73.71766632,345.29696777)
\curveto(73.7876582,345.18696185)(73.84265814,345.06696197)(73.88266632,344.93696777)
\curveto(73.92265806,344.81696222)(73.96765802,344.68696235)(74.01766632,344.54696777)
\curveto(74.04765794,344.46696257)(74.07265791,344.38196266)(74.09266632,344.29196777)
\lineto(74.15266632,344.02196777)
\curveto(74.16265782,343.98196306)(74.16765782,343.9419631)(74.16766632,343.90196777)
\curveto(74.16765782,343.86196318)(74.17265781,343.82196322)(74.18266632,343.78196777)
\curveto(74.20265778,343.73196331)(74.20765778,343.67696336)(74.19766632,343.61696777)
\curveto(74.1876578,343.55696348)(74.19265779,343.50196354)(74.21266632,343.45196777)
\moveto(72.11266632,342.91196777)
\curveto(72.12265986,342.96196408)(72.12765986,343.03196401)(72.12766632,343.12196777)
\curveto(72.12765986,343.22196382)(72.12265986,343.29696374)(72.11266632,343.34696777)
\lineto(72.11266632,343.46696777)
\curveto(72.09265989,343.51696352)(72.0826599,343.57196347)(72.08266632,343.63196777)
\curveto(72.0826599,343.69196335)(72.07765991,343.74696329)(72.06766632,343.79696777)
\curveto(72.06765992,343.8369632)(72.06265992,343.86696317)(72.05266632,343.88696777)
\lineto(71.99266632,344.12696777)
\curveto(71.98266,344.21696282)(71.96266002,344.30196274)(71.93266632,344.38196777)
\curveto(71.82266016,344.6419624)(71.69266029,344.86196218)(71.54266632,345.04196777)
\curveto(71.39266059,345.23196181)(71.19266079,345.38196166)(70.94266632,345.49196777)
\curveto(70.8826611,345.51196153)(70.82266116,345.52696151)(70.76266632,345.53696777)
\curveto(70.70266128,345.55696148)(70.63766135,345.57696146)(70.56766632,345.59696777)
\curveto(70.4876615,345.61696142)(70.40266158,345.62196142)(70.31266632,345.61196777)
\lineto(70.04266632,345.61196777)
\curveto(70.01266197,345.59196145)(69.97766201,345.58196146)(69.93766632,345.58196777)
\curveto(69.89766209,345.59196145)(69.86266212,345.59196145)(69.83266632,345.58196777)
\lineto(69.62266632,345.52196777)
\curveto(69.56266242,345.51196153)(69.50766248,345.49196155)(69.45766632,345.46196777)
\curveto(69.20766278,345.35196169)(69.00266298,345.19196185)(68.84266632,344.98196777)
\curveto(68.69266329,344.78196226)(68.57266341,344.54696249)(68.48266632,344.27696777)
\curveto(68.45266353,344.17696286)(68.42766356,344.07196297)(68.40766632,343.96196777)
\curveto(68.39766359,343.85196319)(68.3826636,343.7419633)(68.36266632,343.63196777)
\curveto(68.35266363,343.58196346)(68.34766364,343.53196351)(68.34766632,343.48196777)
\lineto(68.34766632,343.33196777)
\curveto(68.32766366,343.26196378)(68.31766367,343.15696388)(68.31766632,343.01696777)
\curveto(68.32766366,342.87696416)(68.34266364,342.77196427)(68.36266632,342.70196777)
\lineto(68.36266632,342.56696777)
\curveto(68.3826636,342.48696455)(68.39766359,342.40696463)(68.40766632,342.32696777)
\curveto(68.41766357,342.25696478)(68.43266355,342.18196486)(68.45266632,342.10196777)
\curveto(68.55266343,341.80196524)(68.65766333,341.55696548)(68.76766632,341.36696777)
\curveto(68.8876631,341.18696585)(69.07266291,341.02196602)(69.32266632,340.87196777)
\curveto(69.39266259,340.82196622)(69.46766252,340.78196626)(69.54766632,340.75196777)
\curveto(69.63766235,340.72196632)(69.72766226,340.69696634)(69.81766632,340.67696777)
\curveto(69.85766213,340.66696637)(69.89266209,340.66196638)(69.92266632,340.66196777)
\curveto(69.95266203,340.67196637)(69.987662,340.67196637)(70.02766632,340.66196777)
\lineto(70.14766632,340.63196777)
\curveto(70.19766179,340.63196641)(70.24266174,340.6369664)(70.28266632,340.64696777)
\lineto(70.40266632,340.64696777)
\curveto(70.4826615,340.66696637)(70.56266142,340.68196636)(70.64266632,340.69196777)
\curveto(70.72266126,340.70196634)(70.79766119,340.72196632)(70.86766632,340.75196777)
\curveto(71.12766086,340.85196619)(71.33766065,340.98696605)(71.49766632,341.15696777)
\curveto(71.65766033,341.32696571)(71.79266019,341.5369655)(71.90266632,341.78696777)
\curveto(71.94266004,341.88696515)(71.97266001,341.98696505)(71.99266632,342.08696777)
\curveto(72.01265997,342.18696485)(72.03765995,342.29196475)(72.06766632,342.40196777)
\curveto(72.07765991,342.4419646)(72.0826599,342.47696456)(72.08266632,342.50696777)
\curveto(72.0826599,342.54696449)(72.0876599,342.58696445)(72.09766632,342.62696777)
\lineto(72.09766632,342.76196777)
\curveto(72.09765989,342.81196423)(72.10265988,342.86196418)(72.11266632,342.91196777)
}
}
{
\newrgbcolor{curcolor}{0 0 0}
\pscustom[linestyle=none,fillstyle=solid,fillcolor=curcolor]
{
\newpath
\moveto(78.5825882,347.21696777)
\curveto(79.3325837,347.2369598)(79.98258305,347.15195989)(80.5325882,346.96196777)
\curveto(81.09258194,346.78196026)(81.51758151,346.46696057)(81.8075882,346.01696777)
\curveto(81.87758115,345.90696113)(81.93758109,345.79196125)(81.9875882,345.67196777)
\curveto(82.04758098,345.56196148)(82.09758093,345.4369616)(82.1375882,345.29696777)
\curveto(82.15758087,345.2369618)(82.16758086,345.17196187)(82.1675882,345.10196777)
\curveto(82.16758086,345.03196201)(82.15758087,344.97196207)(82.1375882,344.92196777)
\curveto(82.09758093,344.86196218)(82.04258099,344.82196222)(81.9725882,344.80196777)
\curveto(81.92258111,344.78196226)(81.86258117,344.77196227)(81.7925882,344.77196777)
\lineto(81.5825882,344.77196777)
\lineto(80.9225882,344.77196777)
\curveto(80.85258218,344.77196227)(80.78258225,344.76696227)(80.7125882,344.75696777)
\curveto(80.64258239,344.75696228)(80.57758245,344.76696227)(80.5175882,344.78696777)
\curveto(80.41758261,344.80696223)(80.34258269,344.84696219)(80.2925882,344.90696777)
\curveto(80.24258279,344.96696207)(80.19758283,345.02696201)(80.1575882,345.08696777)
\lineto(80.0375882,345.29696777)
\curveto(80.00758302,345.37696166)(79.95758307,345.4419616)(79.8875882,345.49196777)
\curveto(79.78758324,345.57196147)(79.68758334,345.63196141)(79.5875882,345.67196777)
\curveto(79.49758353,345.71196133)(79.38258365,345.74696129)(79.2425882,345.77696777)
\curveto(79.17258386,345.79696124)(79.06758396,345.81196123)(78.9275882,345.82196777)
\curveto(78.79758423,345.83196121)(78.69758433,345.82696121)(78.6275882,345.80696777)
\lineto(78.5225882,345.80696777)
\lineto(78.3725882,345.77696777)
\curveto(78.3325847,345.77696126)(78.28758474,345.77196127)(78.2375882,345.76196777)
\curveto(78.06758496,345.71196133)(77.9275851,345.6419614)(77.8175882,345.55196777)
\curveto(77.71758531,345.47196157)(77.64758538,345.34696169)(77.6075882,345.17696777)
\curveto(77.58758544,345.10696193)(77.58758544,345.041962)(77.6075882,344.98196777)
\curveto(77.6275854,344.92196212)(77.64758538,344.87196217)(77.6675882,344.83196777)
\curveto(77.73758529,344.71196233)(77.81758521,344.61696242)(77.9075882,344.54696777)
\curveto(78.00758502,344.47696256)(78.12258491,344.41696262)(78.2525882,344.36696777)
\curveto(78.44258459,344.28696275)(78.64758438,344.21696282)(78.8675882,344.15696777)
\lineto(79.5575882,344.00696777)
\curveto(79.79758323,343.96696307)(80.027583,343.91696312)(80.2475882,343.85696777)
\curveto(80.47758255,343.80696323)(80.69258234,343.7419633)(80.8925882,343.66196777)
\curveto(80.98258205,343.62196342)(81.06758196,343.58696345)(81.1475882,343.55696777)
\curveto(81.23758179,343.5369635)(81.32258171,343.50196354)(81.4025882,343.45196777)
\curveto(81.59258144,343.33196371)(81.76258127,343.20196384)(81.9125882,343.06196777)
\curveto(82.07258096,342.92196412)(82.19758083,342.74696429)(82.2875882,342.53696777)
\curveto(82.31758071,342.46696457)(82.34258069,342.39696464)(82.3625882,342.32696777)
\curveto(82.38258065,342.25696478)(82.40258063,342.18196486)(82.4225882,342.10196777)
\curveto(82.4325806,342.041965)(82.43758059,341.94696509)(82.4375882,341.81696777)
\curveto(82.44758058,341.69696534)(82.44758058,341.60196544)(82.4375882,341.53196777)
\lineto(82.4375882,341.45696777)
\curveto(82.41758061,341.39696564)(82.40258063,341.3369657)(82.3925882,341.27696777)
\curveto(82.39258064,341.22696581)(82.38758064,341.17696586)(82.3775882,341.12696777)
\curveto(82.30758072,340.82696621)(82.19758083,340.56196648)(82.0475882,340.33196777)
\curveto(81.88758114,340.09196695)(81.69258134,339.89696714)(81.4625882,339.74696777)
\curveto(81.2325818,339.59696744)(80.97258206,339.46696757)(80.6825882,339.35696777)
\curveto(80.57258246,339.30696773)(80.45258258,339.27196777)(80.3225882,339.25196777)
\curveto(80.20258283,339.23196781)(80.08258295,339.20696783)(79.9625882,339.17696777)
\curveto(79.87258316,339.15696788)(79.77758325,339.14696789)(79.6775882,339.14696777)
\curveto(79.58758344,339.1369679)(79.49758353,339.12196792)(79.4075882,339.10196777)
\lineto(79.1375882,339.10196777)
\curveto(79.07758395,339.08196796)(78.97258406,339.07196797)(78.8225882,339.07196777)
\curveto(78.68258435,339.07196797)(78.58258445,339.08196796)(78.5225882,339.10196777)
\curveto(78.49258454,339.10196794)(78.45758457,339.10696793)(78.4175882,339.11696777)
\lineto(78.3125882,339.11696777)
\curveto(78.19258484,339.1369679)(78.07258496,339.15196789)(77.9525882,339.16196777)
\curveto(77.8325852,339.17196787)(77.71758531,339.19196785)(77.6075882,339.22196777)
\curveto(77.21758581,339.33196771)(76.87258616,339.45696758)(76.5725882,339.59696777)
\curveto(76.27258676,339.74696729)(76.01758701,339.96696707)(75.8075882,340.25696777)
\curveto(75.66758736,340.44696659)(75.54758748,340.66696637)(75.4475882,340.91696777)
\curveto(75.4275876,340.97696606)(75.40758762,341.05696598)(75.3875882,341.15696777)
\curveto(75.36758766,341.20696583)(75.35258768,341.27696576)(75.3425882,341.36696777)
\curveto(75.3325877,341.45696558)(75.33758769,341.53196551)(75.3575882,341.59196777)
\curveto(75.38758764,341.66196538)(75.43758759,341.71196533)(75.5075882,341.74196777)
\curveto(75.55758747,341.76196528)(75.61758741,341.77196527)(75.6875882,341.77196777)
\lineto(75.9125882,341.77196777)
\lineto(76.6175882,341.77196777)
\lineto(76.8575882,341.77196777)
\curveto(76.93758609,341.77196527)(77.00758602,341.76196528)(77.0675882,341.74196777)
\curveto(77.17758585,341.70196534)(77.24758578,341.6369654)(77.2775882,341.54696777)
\curveto(77.31758571,341.45696558)(77.36258567,341.36196568)(77.4125882,341.26196777)
\curveto(77.4325856,341.21196583)(77.46758556,341.14696589)(77.5175882,341.06696777)
\curveto(77.57758545,340.98696605)(77.6275854,340.9369661)(77.6675882,340.91696777)
\curveto(77.78758524,340.81696622)(77.90258513,340.7369663)(78.0125882,340.67696777)
\curveto(78.12258491,340.62696641)(78.26258477,340.57696646)(78.4325882,340.52696777)
\curveto(78.48258455,340.50696653)(78.5325845,340.49696654)(78.5825882,340.49696777)
\curveto(78.6325844,340.50696653)(78.68258435,340.50696653)(78.7325882,340.49696777)
\curveto(78.81258422,340.47696656)(78.89758413,340.46696657)(78.9875882,340.46696777)
\curveto(79.08758394,340.47696656)(79.17258386,340.49196655)(79.2425882,340.51196777)
\curveto(79.29258374,340.52196652)(79.33758369,340.52696651)(79.3775882,340.52696777)
\curveto(79.4275836,340.52696651)(79.47758355,340.5369665)(79.5275882,340.55696777)
\curveto(79.66758336,340.60696643)(79.79258324,340.66696637)(79.9025882,340.73696777)
\curveto(80.02258301,340.80696623)(80.11758291,340.89696614)(80.1875882,341.00696777)
\curveto(80.23758279,341.08696595)(80.27758275,341.21196583)(80.3075882,341.38196777)
\curveto(80.3275827,341.45196559)(80.3275827,341.51696552)(80.3075882,341.57696777)
\curveto(80.28758274,341.6369654)(80.26758276,341.68696535)(80.2475882,341.72696777)
\curveto(80.17758285,341.86696517)(80.08758294,341.97196507)(79.9775882,342.04196777)
\curveto(79.87758315,342.11196493)(79.75758327,342.17696486)(79.6175882,342.23696777)
\curveto(79.4275836,342.31696472)(79.2275838,342.38196466)(79.0175882,342.43196777)
\curveto(78.80758422,342.48196456)(78.59758443,342.5369645)(78.3875882,342.59696777)
\curveto(78.30758472,342.61696442)(78.22258481,342.63196441)(78.1325882,342.64196777)
\curveto(78.05258498,342.65196439)(77.97258506,342.66696437)(77.8925882,342.68696777)
\curveto(77.57258546,342.77696426)(77.26758576,342.86196418)(76.9775882,342.94196777)
\curveto(76.68758634,343.03196401)(76.42258661,343.16196388)(76.1825882,343.33196777)
\curveto(75.90258713,343.53196351)(75.69758733,343.80196324)(75.5675882,344.14196777)
\curveto(75.54758748,344.21196283)(75.5275875,344.30696273)(75.5075882,344.42696777)
\curveto(75.48758754,344.49696254)(75.47258756,344.58196246)(75.4625882,344.68196777)
\curveto(75.45258758,344.78196226)(75.45758757,344.87196217)(75.4775882,344.95196777)
\curveto(75.49758753,345.00196204)(75.50258753,345.041962)(75.4925882,345.07196777)
\curveto(75.48258755,345.11196193)(75.48758754,345.15696188)(75.5075882,345.20696777)
\curveto(75.5275875,345.31696172)(75.54758748,345.41696162)(75.5675882,345.50696777)
\curveto(75.59758743,345.60696143)(75.6325874,345.70196134)(75.6725882,345.79196777)
\curveto(75.80258723,346.08196096)(75.98258705,346.31696072)(76.2125882,346.49696777)
\curveto(76.44258659,346.67696036)(76.70258633,346.82196022)(76.9925882,346.93196777)
\curveto(77.10258593,346.98196006)(77.21758581,347.01696002)(77.3375882,347.03696777)
\curveto(77.45758557,347.06695997)(77.58258545,347.09695994)(77.7125882,347.12696777)
\curveto(77.77258526,347.14695989)(77.8325852,347.15695988)(77.8925882,347.15696777)
\lineto(78.0725882,347.18696777)
\curveto(78.15258488,347.19695984)(78.23758479,347.20195984)(78.3275882,347.20196777)
\curveto(78.41758461,347.20195984)(78.50258453,347.20695983)(78.5825882,347.21696777)
}
}
{
\newrgbcolor{curcolor}{0 0 0}
\pscustom[linestyle=none,fillstyle=solid,fillcolor=curcolor]
{
}
}
{
\newrgbcolor{curcolor}{0 0 0}
\pscustom[linestyle=none,fillstyle=solid,fillcolor=curcolor]
{
\newpath
\moveto(95.38938507,337.42196777)
\curveto(95.38937673,337.26196978)(95.38437674,337.10696993)(95.37438507,336.95696777)
\curveto(95.37437675,336.79697024)(95.3243768,336.68697035)(95.22438507,336.62696777)
\curveto(95.14437698,336.57697046)(95.02937709,336.55697048)(94.87938507,336.56696777)
\lineto(94.45938507,336.56696777)
\lineto(94.14438507,336.56696777)
\curveto(94.03437809,336.55697048)(93.9243782,336.55697048)(93.81438507,336.56696777)
\curveto(93.71437841,336.56697047)(93.6193785,336.58197046)(93.52938507,336.61196777)
\curveto(93.44937867,336.63197041)(93.38937873,336.67197037)(93.34938507,336.73196777)
\curveto(93.29937882,336.81197023)(93.27437885,336.92697011)(93.27438507,337.07696777)
\curveto(93.28437884,337.21696982)(93.28937883,337.34696969)(93.28938507,337.46696777)
\lineto(93.28938507,339.10196777)
\lineto(93.28938507,339.47696777)
\curveto(93.28937883,339.61696742)(93.27437885,339.72196732)(93.24438507,339.79196777)
\curveto(93.2243789,339.81196723)(93.20437892,339.82696721)(93.18438507,339.83696777)
\curveto(93.17437895,339.85696718)(93.15937896,339.87696716)(93.13938507,339.89696777)
\curveto(93.04937907,339.90696713)(92.97937914,339.88696715)(92.92938507,339.83696777)
\curveto(92.87937924,339.79696724)(92.8243793,339.75696728)(92.76438507,339.71696777)
\curveto(92.67437945,339.64696739)(92.57937954,339.58196746)(92.47938507,339.52196777)
\curveto(92.38937973,339.46196758)(92.28937983,339.40696763)(92.17938507,339.35696777)
\curveto(91.99938012,339.27696776)(91.79938032,339.21696782)(91.57938507,339.17696777)
\curveto(91.35938076,339.12696791)(91.13438099,339.10196794)(90.90438507,339.10196777)
\curveto(90.67438145,339.09196795)(90.44438168,339.10696793)(90.21438507,339.14696777)
\curveto(89.99438213,339.18696785)(89.79438233,339.24696779)(89.61438507,339.32696777)
\curveto(89.16438296,339.52696751)(88.79938332,339.78196726)(88.51938507,340.09196777)
\curveto(88.23938388,340.41196663)(88.00438412,340.80196624)(87.81438507,341.26196777)
\curveto(87.76438436,341.37196567)(87.72938439,341.48196556)(87.70938507,341.59196777)
\curveto(87.68938443,341.71196533)(87.66438446,341.82696521)(87.63438507,341.93696777)
\curveto(87.61438451,341.97696506)(87.60438452,342.01196503)(87.60438507,342.04196777)
\curveto(87.61438451,342.08196496)(87.61438451,342.12196492)(87.60438507,342.16196777)
\curveto(87.58438454,342.2419648)(87.56938455,342.32696471)(87.55938507,342.41696777)
\curveto(87.55938456,342.51696452)(87.54938457,342.61196443)(87.52938507,342.70196777)
\lineto(87.52938507,342.89696777)
\curveto(87.5193846,342.94696409)(87.51438461,343.00696403)(87.51438507,343.07696777)
\curveto(87.51438461,343.15696388)(87.5193846,343.22196382)(87.52938507,343.27196777)
\curveto(87.53938458,343.32196372)(87.54438458,343.36696367)(87.54438507,343.40696777)
\lineto(87.54438507,343.54196777)
\curveto(87.55438457,343.59196345)(87.55438457,343.6419634)(87.54438507,343.69196777)
\curveto(87.54438458,343.7419633)(87.55438457,343.79196325)(87.57438507,343.84196777)
\curveto(87.59438453,343.93196311)(87.60938451,344.02196302)(87.61938507,344.11196777)
\curveto(87.62938449,344.21196283)(87.64438448,344.30696273)(87.66438507,344.39696777)
\curveto(87.71438441,344.56696247)(87.76438436,344.72696231)(87.81438507,344.87696777)
\curveto(87.87438425,345.02696201)(87.93438419,345.17196187)(87.99438507,345.31196777)
\curveto(88.05438407,345.45196159)(88.12938399,345.58696145)(88.21938507,345.71696777)
\curveto(88.30938381,345.84696119)(88.39938372,345.97196107)(88.48938507,346.09196777)
\curveto(88.57938354,346.20196084)(88.67938344,346.30196074)(88.78938507,346.39196777)
\curveto(88.8193833,346.42196062)(88.83938328,346.44696059)(88.84938507,346.46696777)
\curveto(88.89938322,346.49696054)(88.94438318,346.52696051)(88.98438507,346.55696777)
\curveto(89.0243831,346.59696044)(89.06438306,346.63196041)(89.10438507,346.66196777)
\curveto(89.24438288,346.76196028)(89.38938273,346.8419602)(89.53938507,346.90196777)
\curveto(89.69938242,346.97196007)(89.86438226,347.03696)(90.03438507,347.09696777)
\curveto(90.124382,347.12695991)(90.21438191,347.14695989)(90.30438507,347.15696777)
\curveto(90.39438173,347.16695987)(90.48438164,347.18195986)(90.57438507,347.20196777)
\curveto(90.60438152,347.21195983)(90.65938146,347.21195983)(90.73938507,347.20196777)
\curveto(90.8193813,347.19195985)(90.86938125,347.19695984)(90.88938507,347.21696777)
\curveto(91.20938091,347.22695981)(91.50938061,347.19695984)(91.78938507,347.12696777)
\curveto(92.06938005,347.06695997)(92.30937981,346.97696006)(92.50938507,346.85696777)
\lineto(92.68938507,346.73696777)
\curveto(92.74937937,346.69696034)(92.80437932,346.65696038)(92.85438507,346.61696777)
\curveto(92.91437921,346.56696047)(92.96437916,346.51696052)(93.00438507,346.46696777)
\curveto(93.05437907,346.42696061)(93.13437899,346.40696063)(93.24438507,346.40696777)
\lineto(93.28938507,346.45196777)
\lineto(93.34938507,346.51196777)
\curveto(93.37937874,346.59196045)(93.39937872,346.66696037)(93.40938507,346.73696777)
\curveto(93.4193787,346.81696022)(93.45937866,346.88196016)(93.52938507,346.93196777)
\curveto(93.57937854,346.97196007)(93.64937847,346.99196005)(93.73938507,346.99196777)
\curveto(93.83937828,347.00196004)(93.93937818,347.00696003)(94.03938507,347.00696777)
\lineto(94.75938507,347.00696777)
\lineto(94.96938507,347.00696777)
\curveto(95.03937708,347.00696003)(95.10437702,346.99696004)(95.16438507,346.97696777)
\curveto(95.23437689,346.95696008)(95.28937683,346.91196013)(95.32938507,346.84196777)
\curveto(95.37937674,346.77196027)(95.39937672,346.67696036)(95.38938507,346.55696777)
\lineto(95.38938507,346.21196777)
\lineto(95.38938507,337.42196777)
\moveto(93.34938507,343.03196777)
\curveto(93.35937876,343.05196399)(93.35937876,343.07696396)(93.34938507,343.10696777)
\lineto(93.34938507,343.18196777)
\curveto(93.33937878,343.28196376)(93.33437879,343.37696366)(93.33438507,343.46696777)
\curveto(93.33437879,343.55696348)(93.3243788,343.6419634)(93.30438507,343.72196777)
\curveto(93.29437883,343.75196329)(93.28937883,343.77696326)(93.28938507,343.79696777)
\curveto(93.29937882,343.82696321)(93.29937882,343.85696318)(93.28938507,343.88696777)
\curveto(93.26937885,343.96696307)(93.24937887,344.036963)(93.22938507,344.09696777)
\curveto(93.2193789,344.16696287)(93.20437892,344.2369628)(93.18438507,344.30696777)
\curveto(93.08437904,344.59696244)(92.94937917,344.84696219)(92.77938507,345.05696777)
\curveto(92.60937951,345.26696177)(92.38937973,345.42696161)(92.11938507,345.53696777)
\curveto(92.00938011,345.58696145)(91.88938023,345.61196143)(91.75938507,345.61196777)
\curveto(91.63938048,345.62196142)(91.50938061,345.62696141)(91.36938507,345.62696777)
\curveto(91.33938078,345.60696143)(91.30438082,345.59696144)(91.26438507,345.59696777)
\curveto(91.2243809,345.60696143)(91.18438094,345.60696143)(91.14438507,345.59696777)
\lineto(90.96438507,345.53696777)
\curveto(90.90438122,345.52696151)(90.84938127,345.51196153)(90.79938507,345.49196777)
\curveto(90.50938161,345.36196168)(90.27938184,345.17196187)(90.10938507,344.92196777)
\curveto(89.94938217,344.67196237)(89.8243823,344.38196266)(89.73438507,344.05196777)
\curveto(89.71438241,343.97196307)(89.69938242,343.89696314)(89.68938507,343.82696777)
\curveto(89.68938243,343.76696327)(89.67938244,343.69696334)(89.65938507,343.61696777)
\curveto(89.65938246,343.54696349)(89.65438247,343.49696354)(89.64438507,343.46696777)
\curveto(89.63438249,343.41696362)(89.6243825,343.32696371)(89.61438507,343.19696777)
\curveto(89.61438251,343.07696396)(89.6243825,342.99196405)(89.64438507,342.94196777)
\lineto(89.64438507,342.80696777)
\curveto(89.65438247,342.76696427)(89.65938246,342.72696431)(89.65938507,342.68696777)
\curveto(89.65938246,342.64696439)(89.66438246,342.61196443)(89.67438507,342.58196777)
\lineto(89.67438507,342.50696777)
\curveto(89.68438244,342.47696456)(89.68938243,342.45196459)(89.68938507,342.43196777)
\curveto(89.70938241,342.35196469)(89.7243824,342.27696476)(89.73438507,342.20696777)
\curveto(89.74438238,342.1369649)(89.76438236,342.06696497)(89.79438507,341.99696777)
\curveto(89.87438225,341.74696529)(89.97938214,341.53196551)(90.10938507,341.35196777)
\curveto(90.23938188,341.17196587)(90.40438172,341.01696602)(90.60438507,340.88696777)
\curveto(90.74438138,340.80696623)(90.89938122,340.74696629)(91.06938507,340.70696777)
\curveto(91.09938102,340.69696634)(91.124381,340.69196635)(91.14438507,340.69196777)
\curveto(91.17438095,340.69196635)(91.20938091,340.68696635)(91.24938507,340.67696777)
\curveto(91.27938084,340.66696637)(91.3243808,340.65696638)(91.38438507,340.64696777)
\curveto(91.45438067,340.64696639)(91.51438061,340.65196639)(91.56438507,340.66196777)
\curveto(91.58438054,340.67196637)(91.60938051,340.67196637)(91.63938507,340.66196777)
\curveto(91.67938044,340.66196638)(91.71438041,340.66696637)(91.74438507,340.67696777)
\curveto(91.81438031,340.69696634)(91.87938024,340.71196633)(91.93938507,340.72196777)
\curveto(92.00938011,340.73196631)(92.07938004,340.74696629)(92.14938507,340.76696777)
\curveto(92.40937971,340.87696616)(92.61437951,341.02196602)(92.76438507,341.20196777)
\curveto(92.9243792,341.38196566)(93.05937906,341.60196544)(93.16938507,341.86196777)
\curveto(93.19937892,341.9419651)(93.2243789,342.02696501)(93.24438507,342.11696777)
\lineto(93.30438507,342.38696777)
\lineto(93.30438507,342.49196777)
\curveto(93.31437881,342.52196452)(93.3193788,342.55696448)(93.31938507,342.59696777)
\curveto(93.33937878,342.69696434)(93.34937877,342.78196426)(93.34938507,342.85196777)
\lineto(93.34938507,343.03196777)
}
}
{
\newrgbcolor{curcolor}{0 0 0}
\pscustom[linestyle=none,fillstyle=solid,fillcolor=curcolor]
{
\newpath
\moveto(97.41930695,346.99196777)
\lineto(98.54430695,346.99196777)
\curveto(98.65430451,346.99196005)(98.75430441,346.98696005)(98.84430695,346.97696777)
\curveto(98.93430423,346.96696007)(98.99930417,346.93196011)(99.03930695,346.87196777)
\curveto(99.08930408,346.81196023)(99.11930405,346.72696031)(99.12930695,346.61696777)
\curveto(99.13930403,346.51696052)(99.14430402,346.41196063)(99.14430695,346.30196777)
\lineto(99.14430695,345.25196777)
\lineto(99.14430695,343.01696777)
\curveto(99.14430402,342.65696438)(99.15930401,342.31696472)(99.18930695,341.99696777)
\curveto(99.21930395,341.67696536)(99.30930386,341.41196563)(99.45930695,341.20196777)
\curveto(99.59930357,340.99196605)(99.82430334,340.8419662)(100.13430695,340.75196777)
\curveto(100.18430298,340.7419663)(100.22430294,340.7369663)(100.25430695,340.73696777)
\curveto(100.29430287,340.7369663)(100.33930283,340.73196631)(100.38930695,340.72196777)
\curveto(100.43930273,340.71196633)(100.49430267,340.70696633)(100.55430695,340.70696777)
\curveto(100.61430255,340.70696633)(100.65930251,340.71196633)(100.68930695,340.72196777)
\curveto(100.73930243,340.7419663)(100.77930239,340.74696629)(100.80930695,340.73696777)
\curveto(100.84930232,340.72696631)(100.88930228,340.73196631)(100.92930695,340.75196777)
\curveto(101.13930203,340.80196624)(101.30430186,340.86696617)(101.42430695,340.94696777)
\curveto(101.60430156,341.05696598)(101.74430142,341.19696584)(101.84430695,341.36696777)
\curveto(101.95430121,341.54696549)(102.02930114,341.7419653)(102.06930695,341.95196777)
\curveto(102.11930105,342.17196487)(102.14930102,342.41196463)(102.15930695,342.67196777)
\curveto(102.169301,342.9419641)(102.17430099,343.22196382)(102.17430695,343.51196777)
\lineto(102.17430695,345.32696777)
\lineto(102.17430695,346.30196777)
\lineto(102.17430695,346.57196777)
\curveto(102.17430099,346.67196037)(102.19430097,346.75196029)(102.23430695,346.81196777)
\curveto(102.28430088,346.90196014)(102.35930081,346.95196009)(102.45930695,346.96196777)
\curveto(102.55930061,346.98196006)(102.67930049,346.99196005)(102.81930695,346.99196777)
\lineto(103.61430695,346.99196777)
\lineto(103.89930695,346.99196777)
\curveto(103.98929918,346.99196005)(104.0642991,346.97196007)(104.12430695,346.93196777)
\curveto(104.20429896,346.88196016)(104.24929892,346.80696023)(104.25930695,346.70696777)
\curveto(104.2692989,346.60696043)(104.27429889,346.49196055)(104.27430695,346.36196777)
\lineto(104.27430695,345.22196777)
\lineto(104.27430695,341.00696777)
\lineto(104.27430695,339.94196777)
\lineto(104.27430695,339.64196777)
\curveto(104.27429889,339.5419675)(104.25429891,339.46696757)(104.21430695,339.41696777)
\curveto(104.164299,339.3369677)(104.08929908,339.29196775)(103.98930695,339.28196777)
\curveto(103.88929928,339.27196777)(103.78429938,339.26696777)(103.67430695,339.26696777)
\lineto(102.86430695,339.26696777)
\curveto(102.75430041,339.26696777)(102.65430051,339.27196777)(102.56430695,339.28196777)
\curveto(102.48430068,339.29196775)(102.41930075,339.33196771)(102.36930695,339.40196777)
\curveto(102.34930082,339.43196761)(102.32930084,339.47696756)(102.30930695,339.53696777)
\curveto(102.29930087,339.59696744)(102.28430088,339.65696738)(102.26430695,339.71696777)
\curveto(102.25430091,339.77696726)(102.23930093,339.83196721)(102.21930695,339.88196777)
\curveto(102.19930097,339.93196711)(102.169301,339.96196708)(102.12930695,339.97196777)
\curveto(102.10930106,339.99196705)(102.08430108,339.99696704)(102.05430695,339.98696777)
\curveto(102.02430114,339.97696706)(101.99930117,339.96696707)(101.97930695,339.95696777)
\curveto(101.90930126,339.91696712)(101.84930132,339.87196717)(101.79930695,339.82196777)
\curveto(101.74930142,339.77196727)(101.69430147,339.72696731)(101.63430695,339.68696777)
\curveto(101.59430157,339.65696738)(101.55430161,339.62196742)(101.51430695,339.58196777)
\curveto(101.48430168,339.55196749)(101.44430172,339.52196752)(101.39430695,339.49196777)
\curveto(101.164302,339.35196769)(100.89430227,339.2419678)(100.58430695,339.16196777)
\curveto(100.51430265,339.1419679)(100.44430272,339.13196791)(100.37430695,339.13196777)
\curveto(100.30430286,339.12196792)(100.22930294,339.10696793)(100.14930695,339.08696777)
\curveto(100.10930306,339.07696796)(100.0643031,339.07696796)(100.01430695,339.08696777)
\curveto(99.97430319,339.08696795)(99.93430323,339.08196796)(99.89430695,339.07196777)
\curveto(99.8643033,339.06196798)(99.79930337,339.06196798)(99.69930695,339.07196777)
\curveto(99.60930356,339.07196797)(99.54930362,339.07696796)(99.51930695,339.08696777)
\curveto(99.4693037,339.08696795)(99.41930375,339.09196795)(99.36930695,339.10196777)
\lineto(99.21930695,339.10196777)
\curveto(99.09930407,339.13196791)(98.98430418,339.15696788)(98.87430695,339.17696777)
\curveto(98.7643044,339.19696784)(98.65430451,339.22696781)(98.54430695,339.26696777)
\curveto(98.49430467,339.28696775)(98.44930472,339.30196774)(98.40930695,339.31196777)
\curveto(98.37930479,339.33196771)(98.33930483,339.35196769)(98.28930695,339.37196777)
\curveto(97.93930523,339.56196748)(97.65930551,339.82696721)(97.44930695,340.16696777)
\curveto(97.31930585,340.37696666)(97.22430594,340.62696641)(97.16430695,340.91696777)
\curveto(97.10430606,341.21696582)(97.0643061,341.53196551)(97.04430695,341.86196777)
\curveto(97.03430613,342.20196484)(97.02930614,342.54696449)(97.02930695,342.89696777)
\curveto(97.03930613,343.25696378)(97.04430612,343.61196343)(97.04430695,343.96196777)
\lineto(97.04430695,346.00196777)
\curveto(97.04430612,346.13196091)(97.03930613,346.28196076)(97.02930695,346.45196777)
\curveto(97.02930614,346.63196041)(97.05430611,346.76196028)(97.10430695,346.84196777)
\curveto(97.13430603,346.89196015)(97.19430597,346.9369601)(97.28430695,346.97696777)
\curveto(97.34430582,346.97696006)(97.38930578,346.98196006)(97.41930695,346.99196777)
}
}
{
\newrgbcolor{curcolor}{0 0 0}
\pscustom[linestyle=none,fillstyle=solid,fillcolor=curcolor]
{
\newpath
\moveto(113.27055695,343.21196777)
\curveto(113.29054878,343.13196391)(113.29054878,343.041964)(113.27055695,342.94196777)
\curveto(113.25054882,342.8419642)(113.21554886,342.77696426)(113.16555695,342.74696777)
\curveto(113.11554896,342.70696433)(113.04054903,342.67696436)(112.94055695,342.65696777)
\curveto(112.85054922,342.64696439)(112.74554933,342.6369644)(112.62555695,342.62696777)
\lineto(112.28055695,342.62696777)
\curveto(112.1705499,342.6369644)(112.07055,342.6419644)(111.98055695,342.64196777)
\lineto(108.32055695,342.64196777)
\lineto(108.11055695,342.64196777)
\curveto(108.05055402,342.6419644)(107.99555408,342.63196441)(107.94555695,342.61196777)
\curveto(107.86555421,342.57196447)(107.81555426,342.53196451)(107.79555695,342.49196777)
\curveto(107.7755543,342.47196457)(107.75555432,342.43196461)(107.73555695,342.37196777)
\curveto(107.71555436,342.32196472)(107.71055436,342.27196477)(107.72055695,342.22196777)
\curveto(107.74055433,342.16196488)(107.75055432,342.10196494)(107.75055695,342.04196777)
\curveto(107.76055431,341.99196505)(107.7755543,341.9369651)(107.79555695,341.87696777)
\curveto(107.8755542,341.6369654)(107.9705541,341.4369656)(108.08055695,341.27696777)
\curveto(108.20055387,341.12696591)(108.36055371,340.99196605)(108.56055695,340.87196777)
\curveto(108.64055343,340.82196622)(108.72055335,340.78696625)(108.80055695,340.76696777)
\curveto(108.89055318,340.75696628)(108.98055309,340.7369663)(109.07055695,340.70696777)
\curveto(109.15055292,340.68696635)(109.26055281,340.67196637)(109.40055695,340.66196777)
\curveto(109.54055253,340.65196639)(109.66055241,340.65696638)(109.76055695,340.67696777)
\lineto(109.89555695,340.67696777)
\curveto(109.99555208,340.69696634)(110.08555199,340.71696632)(110.16555695,340.73696777)
\curveto(110.25555182,340.76696627)(110.34055173,340.79696624)(110.42055695,340.82696777)
\curveto(110.52055155,340.87696616)(110.63055144,340.9419661)(110.75055695,341.02196777)
\curveto(110.88055119,341.10196594)(110.9755511,341.18196586)(111.03555695,341.26196777)
\curveto(111.08555099,341.33196571)(111.13555094,341.39696564)(111.18555695,341.45696777)
\curveto(111.24555083,341.52696551)(111.31555076,341.57696546)(111.39555695,341.60696777)
\curveto(111.49555058,341.65696538)(111.62055045,341.67696536)(111.77055695,341.66696777)
\lineto(112.20555695,341.66696777)
\lineto(112.38555695,341.66696777)
\curveto(112.45554962,341.67696536)(112.51554956,341.67196537)(112.56555695,341.65196777)
\lineto(112.71555695,341.65196777)
\curveto(112.81554926,341.63196541)(112.88554919,341.60696543)(112.92555695,341.57696777)
\curveto(112.96554911,341.55696548)(112.98554909,341.51196553)(112.98555695,341.44196777)
\curveto(112.99554908,341.37196567)(112.99054908,341.31196573)(112.97055695,341.26196777)
\curveto(112.92054915,341.12196592)(112.86554921,340.99696604)(112.80555695,340.88696777)
\curveto(112.74554933,340.77696626)(112.6755494,340.66696637)(112.59555695,340.55696777)
\curveto(112.3755497,340.22696681)(112.12554995,339.96196708)(111.84555695,339.76196777)
\curveto(111.56555051,339.56196748)(111.21555086,339.39196765)(110.79555695,339.25196777)
\curveto(110.68555139,339.21196783)(110.5755515,339.18696785)(110.46555695,339.17696777)
\curveto(110.35555172,339.16696787)(110.24055183,339.14696789)(110.12055695,339.11696777)
\curveto(110.08055199,339.10696793)(110.03555204,339.10696793)(109.98555695,339.11696777)
\curveto(109.94555213,339.11696792)(109.90555217,339.11196793)(109.86555695,339.10196777)
\lineto(109.70055695,339.10196777)
\curveto(109.65055242,339.08196796)(109.59055248,339.07696796)(109.52055695,339.08696777)
\curveto(109.46055261,339.08696795)(109.40555267,339.09196795)(109.35555695,339.10196777)
\curveto(109.2755528,339.11196793)(109.20555287,339.11196793)(109.14555695,339.10196777)
\curveto(109.08555299,339.09196795)(109.02055305,339.09696794)(108.95055695,339.11696777)
\curveto(108.90055317,339.1369679)(108.84555323,339.14696789)(108.78555695,339.14696777)
\curveto(108.72555335,339.14696789)(108.6705534,339.15696788)(108.62055695,339.17696777)
\curveto(108.51055356,339.19696784)(108.40055367,339.22196782)(108.29055695,339.25196777)
\curveto(108.18055389,339.27196777)(108.08055399,339.30696773)(107.99055695,339.35696777)
\curveto(107.88055419,339.39696764)(107.7755543,339.43196761)(107.67555695,339.46196777)
\curveto(107.58555449,339.50196754)(107.50055457,339.54696749)(107.42055695,339.59696777)
\curveto(107.10055497,339.79696724)(106.81555526,340.02696701)(106.56555695,340.28696777)
\curveto(106.31555576,340.55696648)(106.11055596,340.86696617)(105.95055695,341.21696777)
\curveto(105.90055617,341.32696571)(105.86055621,341.4369656)(105.83055695,341.54696777)
\curveto(105.80055627,341.66696537)(105.76055631,341.78696525)(105.71055695,341.90696777)
\curveto(105.70055637,341.94696509)(105.69555638,341.98196506)(105.69555695,342.01196777)
\curveto(105.69555638,342.05196499)(105.69055638,342.09196495)(105.68055695,342.13196777)
\curveto(105.64055643,342.25196479)(105.61555646,342.38196466)(105.60555695,342.52196777)
\lineto(105.57555695,342.94196777)
\curveto(105.5755565,342.99196405)(105.5705565,343.04696399)(105.56055695,343.10696777)
\curveto(105.56055651,343.16696387)(105.56555651,343.22196382)(105.57555695,343.27196777)
\lineto(105.57555695,343.45196777)
\lineto(105.62055695,343.81196777)
\curveto(105.66055641,343.98196306)(105.69555638,344.14696289)(105.72555695,344.30696777)
\curveto(105.75555632,344.46696257)(105.80055627,344.61696242)(105.86055695,344.75696777)
\curveto(106.29055578,345.79696124)(107.02055505,346.53196051)(108.05055695,346.96196777)
\curveto(108.19055388,347.02196002)(108.33055374,347.06195998)(108.47055695,347.08196777)
\curveto(108.62055345,347.11195993)(108.7755533,347.14695989)(108.93555695,347.18696777)
\curveto(109.01555306,347.19695984)(109.09055298,347.20195984)(109.16055695,347.20196777)
\curveto(109.23055284,347.20195984)(109.30555277,347.20695983)(109.38555695,347.21696777)
\curveto(109.89555218,347.22695981)(110.33055174,347.16695987)(110.69055695,347.03696777)
\curveto(111.06055101,346.91696012)(111.39055068,346.75696028)(111.68055695,346.55696777)
\curveto(111.7705503,346.49696054)(111.86055021,346.42696061)(111.95055695,346.34696777)
\curveto(112.04055003,346.27696076)(112.12054995,346.20196084)(112.19055695,346.12196777)
\curveto(112.22054985,346.07196097)(112.26054981,346.03196101)(112.31055695,346.00196777)
\curveto(112.39054968,345.89196115)(112.46554961,345.77696126)(112.53555695,345.65696777)
\curveto(112.60554947,345.54696149)(112.68054939,345.43196161)(112.76055695,345.31196777)
\curveto(112.81054926,345.22196182)(112.85054922,345.12696191)(112.88055695,345.02696777)
\curveto(112.92054915,344.9369621)(112.96054911,344.8369622)(113.00055695,344.72696777)
\curveto(113.05054902,344.59696244)(113.09054898,344.46196258)(113.12055695,344.32196777)
\curveto(113.15054892,344.18196286)(113.18554889,344.041963)(113.22555695,343.90196777)
\curveto(113.24554883,343.82196322)(113.25054882,343.73196331)(113.24055695,343.63196777)
\curveto(113.24054883,343.5419635)(113.25054882,343.45696358)(113.27055695,343.37696777)
\lineto(113.27055695,343.21196777)
\moveto(111.02055695,344.09696777)
\curveto(111.09055098,344.19696284)(111.09555098,344.31696272)(111.03555695,344.45696777)
\curveto(110.98555109,344.60696243)(110.94555113,344.71696232)(110.91555695,344.78696777)
\curveto(110.7755513,345.05696198)(110.59055148,345.26196178)(110.36055695,345.40196777)
\curveto(110.13055194,345.55196149)(109.81055226,345.63196141)(109.40055695,345.64196777)
\curveto(109.3705527,345.62196142)(109.33555274,345.61696142)(109.29555695,345.62696777)
\curveto(109.25555282,345.6369614)(109.22055285,345.6369614)(109.19055695,345.62696777)
\curveto(109.14055293,345.60696143)(109.08555299,345.59196145)(109.02555695,345.58196777)
\curveto(108.96555311,345.58196146)(108.91055316,345.57196147)(108.86055695,345.55196777)
\curveto(108.42055365,345.41196163)(108.09555398,345.1369619)(107.88555695,344.72696777)
\curveto(107.86555421,344.68696235)(107.84055423,344.63196241)(107.81055695,344.56196777)
\curveto(107.79055428,344.50196254)(107.7755543,344.4369626)(107.76555695,344.36696777)
\curveto(107.75555432,344.30696273)(107.75555432,344.24696279)(107.76555695,344.18696777)
\curveto(107.78555429,344.12696291)(107.82055425,344.07696296)(107.87055695,344.03696777)
\curveto(107.95055412,343.98696305)(108.06055401,343.96196308)(108.20055695,343.96196777)
\lineto(108.60555695,343.96196777)
\lineto(110.27055695,343.96196777)
\lineto(110.70555695,343.96196777)
\curveto(110.86555121,343.97196307)(110.9705511,344.01696302)(111.02055695,344.09696777)
}
}
{
\newrgbcolor{curcolor}{0 0 0}
\pscustom[linestyle=none,fillstyle=solid,fillcolor=curcolor]
{
}
}
{
\newrgbcolor{curcolor}{0 0 0}
\pscustom[linestyle=none,fillstyle=solid,fillcolor=curcolor]
{
\newpath
\moveto(126.04399445,343.21196777)
\curveto(126.06398628,343.13196391)(126.06398628,343.041964)(126.04399445,342.94196777)
\curveto(126.02398632,342.8419642)(125.98898636,342.77696426)(125.93899445,342.74696777)
\curveto(125.88898646,342.70696433)(125.81398653,342.67696436)(125.71399445,342.65696777)
\curveto(125.62398672,342.64696439)(125.51898683,342.6369644)(125.39899445,342.62696777)
\lineto(125.05399445,342.62696777)
\curveto(124.9439874,342.6369644)(124.8439875,342.6419644)(124.75399445,342.64196777)
\lineto(121.09399445,342.64196777)
\lineto(120.88399445,342.64196777)
\curveto(120.82399152,342.6419644)(120.76899158,342.63196441)(120.71899445,342.61196777)
\curveto(120.63899171,342.57196447)(120.58899176,342.53196451)(120.56899445,342.49196777)
\curveto(120.5489918,342.47196457)(120.52899182,342.43196461)(120.50899445,342.37196777)
\curveto(120.48899186,342.32196472)(120.48399186,342.27196477)(120.49399445,342.22196777)
\curveto(120.51399183,342.16196488)(120.52399182,342.10196494)(120.52399445,342.04196777)
\curveto(120.53399181,341.99196505)(120.5489918,341.9369651)(120.56899445,341.87696777)
\curveto(120.6489917,341.6369654)(120.7439916,341.4369656)(120.85399445,341.27696777)
\curveto(120.97399137,341.12696591)(121.13399121,340.99196605)(121.33399445,340.87196777)
\curveto(121.41399093,340.82196622)(121.49399085,340.78696625)(121.57399445,340.76696777)
\curveto(121.66399068,340.75696628)(121.75399059,340.7369663)(121.84399445,340.70696777)
\curveto(121.92399042,340.68696635)(122.03399031,340.67196637)(122.17399445,340.66196777)
\curveto(122.31399003,340.65196639)(122.43398991,340.65696638)(122.53399445,340.67696777)
\lineto(122.66899445,340.67696777)
\curveto(122.76898958,340.69696634)(122.85898949,340.71696632)(122.93899445,340.73696777)
\curveto(123.02898932,340.76696627)(123.11398923,340.79696624)(123.19399445,340.82696777)
\curveto(123.29398905,340.87696616)(123.40398894,340.9419661)(123.52399445,341.02196777)
\curveto(123.65398869,341.10196594)(123.7489886,341.18196586)(123.80899445,341.26196777)
\curveto(123.85898849,341.33196571)(123.90898844,341.39696564)(123.95899445,341.45696777)
\curveto(124.01898833,341.52696551)(124.08898826,341.57696546)(124.16899445,341.60696777)
\curveto(124.26898808,341.65696538)(124.39398795,341.67696536)(124.54399445,341.66696777)
\lineto(124.97899445,341.66696777)
\lineto(125.15899445,341.66696777)
\curveto(125.22898712,341.67696536)(125.28898706,341.67196537)(125.33899445,341.65196777)
\lineto(125.48899445,341.65196777)
\curveto(125.58898676,341.63196541)(125.65898669,341.60696543)(125.69899445,341.57696777)
\curveto(125.73898661,341.55696548)(125.75898659,341.51196553)(125.75899445,341.44196777)
\curveto(125.76898658,341.37196567)(125.76398658,341.31196573)(125.74399445,341.26196777)
\curveto(125.69398665,341.12196592)(125.63898671,340.99696604)(125.57899445,340.88696777)
\curveto(125.51898683,340.77696626)(125.4489869,340.66696637)(125.36899445,340.55696777)
\curveto(125.1489872,340.22696681)(124.89898745,339.96196708)(124.61899445,339.76196777)
\curveto(124.33898801,339.56196748)(123.98898836,339.39196765)(123.56899445,339.25196777)
\curveto(123.45898889,339.21196783)(123.348989,339.18696785)(123.23899445,339.17696777)
\curveto(123.12898922,339.16696787)(123.01398933,339.14696789)(122.89399445,339.11696777)
\curveto(122.85398949,339.10696793)(122.80898954,339.10696793)(122.75899445,339.11696777)
\curveto(122.71898963,339.11696792)(122.67898967,339.11196793)(122.63899445,339.10196777)
\lineto(122.47399445,339.10196777)
\curveto(122.42398992,339.08196796)(122.36398998,339.07696796)(122.29399445,339.08696777)
\curveto(122.23399011,339.08696795)(122.17899017,339.09196795)(122.12899445,339.10196777)
\curveto(122.0489903,339.11196793)(121.97899037,339.11196793)(121.91899445,339.10196777)
\curveto(121.85899049,339.09196795)(121.79399055,339.09696794)(121.72399445,339.11696777)
\curveto(121.67399067,339.1369679)(121.61899073,339.14696789)(121.55899445,339.14696777)
\curveto(121.49899085,339.14696789)(121.4439909,339.15696788)(121.39399445,339.17696777)
\curveto(121.28399106,339.19696784)(121.17399117,339.22196782)(121.06399445,339.25196777)
\curveto(120.95399139,339.27196777)(120.85399149,339.30696773)(120.76399445,339.35696777)
\curveto(120.65399169,339.39696764)(120.5489918,339.43196761)(120.44899445,339.46196777)
\curveto(120.35899199,339.50196754)(120.27399207,339.54696749)(120.19399445,339.59696777)
\curveto(119.87399247,339.79696724)(119.58899276,340.02696701)(119.33899445,340.28696777)
\curveto(119.08899326,340.55696648)(118.88399346,340.86696617)(118.72399445,341.21696777)
\curveto(118.67399367,341.32696571)(118.63399371,341.4369656)(118.60399445,341.54696777)
\curveto(118.57399377,341.66696537)(118.53399381,341.78696525)(118.48399445,341.90696777)
\curveto(118.47399387,341.94696509)(118.46899388,341.98196506)(118.46899445,342.01196777)
\curveto(118.46899388,342.05196499)(118.46399388,342.09196495)(118.45399445,342.13196777)
\curveto(118.41399393,342.25196479)(118.38899396,342.38196466)(118.37899445,342.52196777)
\lineto(118.34899445,342.94196777)
\curveto(118.348994,342.99196405)(118.343994,343.04696399)(118.33399445,343.10696777)
\curveto(118.33399401,343.16696387)(118.33899401,343.22196382)(118.34899445,343.27196777)
\lineto(118.34899445,343.45196777)
\lineto(118.39399445,343.81196777)
\curveto(118.43399391,343.98196306)(118.46899388,344.14696289)(118.49899445,344.30696777)
\curveto(118.52899382,344.46696257)(118.57399377,344.61696242)(118.63399445,344.75696777)
\curveto(119.06399328,345.79696124)(119.79399255,346.53196051)(120.82399445,346.96196777)
\curveto(120.96399138,347.02196002)(121.10399124,347.06195998)(121.24399445,347.08196777)
\curveto(121.39399095,347.11195993)(121.5489908,347.14695989)(121.70899445,347.18696777)
\curveto(121.78899056,347.19695984)(121.86399048,347.20195984)(121.93399445,347.20196777)
\curveto(122.00399034,347.20195984)(122.07899027,347.20695983)(122.15899445,347.21696777)
\curveto(122.66898968,347.22695981)(123.10398924,347.16695987)(123.46399445,347.03696777)
\curveto(123.83398851,346.91696012)(124.16398818,346.75696028)(124.45399445,346.55696777)
\curveto(124.5439878,346.49696054)(124.63398771,346.42696061)(124.72399445,346.34696777)
\curveto(124.81398753,346.27696076)(124.89398745,346.20196084)(124.96399445,346.12196777)
\curveto(124.99398735,346.07196097)(125.03398731,346.03196101)(125.08399445,346.00196777)
\curveto(125.16398718,345.89196115)(125.23898711,345.77696126)(125.30899445,345.65696777)
\curveto(125.37898697,345.54696149)(125.45398689,345.43196161)(125.53399445,345.31196777)
\curveto(125.58398676,345.22196182)(125.62398672,345.12696191)(125.65399445,345.02696777)
\curveto(125.69398665,344.9369621)(125.73398661,344.8369622)(125.77399445,344.72696777)
\curveto(125.82398652,344.59696244)(125.86398648,344.46196258)(125.89399445,344.32196777)
\curveto(125.92398642,344.18196286)(125.95898639,344.041963)(125.99899445,343.90196777)
\curveto(126.01898633,343.82196322)(126.02398632,343.73196331)(126.01399445,343.63196777)
\curveto(126.01398633,343.5419635)(126.02398632,343.45696358)(126.04399445,343.37696777)
\lineto(126.04399445,343.21196777)
\moveto(123.79399445,344.09696777)
\curveto(123.86398848,344.19696284)(123.86898848,344.31696272)(123.80899445,344.45696777)
\curveto(123.75898859,344.60696243)(123.71898863,344.71696232)(123.68899445,344.78696777)
\curveto(123.5489888,345.05696198)(123.36398898,345.26196178)(123.13399445,345.40196777)
\curveto(122.90398944,345.55196149)(122.58398976,345.63196141)(122.17399445,345.64196777)
\curveto(122.1439902,345.62196142)(122.10899024,345.61696142)(122.06899445,345.62696777)
\curveto(122.02899032,345.6369614)(121.99399035,345.6369614)(121.96399445,345.62696777)
\curveto(121.91399043,345.60696143)(121.85899049,345.59196145)(121.79899445,345.58196777)
\curveto(121.73899061,345.58196146)(121.68399066,345.57196147)(121.63399445,345.55196777)
\curveto(121.19399115,345.41196163)(120.86899148,345.1369619)(120.65899445,344.72696777)
\curveto(120.63899171,344.68696235)(120.61399173,344.63196241)(120.58399445,344.56196777)
\curveto(120.56399178,344.50196254)(120.5489918,344.4369626)(120.53899445,344.36696777)
\curveto(120.52899182,344.30696273)(120.52899182,344.24696279)(120.53899445,344.18696777)
\curveto(120.55899179,344.12696291)(120.59399175,344.07696296)(120.64399445,344.03696777)
\curveto(120.72399162,343.98696305)(120.83399151,343.96196308)(120.97399445,343.96196777)
\lineto(121.37899445,343.96196777)
\lineto(123.04399445,343.96196777)
\lineto(123.47899445,343.96196777)
\curveto(123.63898871,343.97196307)(123.7439886,344.01696302)(123.79399445,344.09696777)
}
}
{
\newrgbcolor{curcolor}{0 0 0}
\pscustom[linestyle=none,fillstyle=solid,fillcolor=curcolor]
{
\newpath
\moveto(130.2622757,347.21696777)
\curveto(131.0122712,347.2369598)(131.66227055,347.15195989)(132.2122757,346.96196777)
\curveto(132.77226944,346.78196026)(133.19726901,346.46696057)(133.4872757,346.01696777)
\curveto(133.55726865,345.90696113)(133.61726859,345.79196125)(133.6672757,345.67196777)
\curveto(133.72726848,345.56196148)(133.77726843,345.4369616)(133.8172757,345.29696777)
\curveto(133.83726837,345.2369618)(133.84726836,345.17196187)(133.8472757,345.10196777)
\curveto(133.84726836,345.03196201)(133.83726837,344.97196207)(133.8172757,344.92196777)
\curveto(133.77726843,344.86196218)(133.72226849,344.82196222)(133.6522757,344.80196777)
\curveto(133.60226861,344.78196226)(133.54226867,344.77196227)(133.4722757,344.77196777)
\lineto(133.2622757,344.77196777)
\lineto(132.6022757,344.77196777)
\curveto(132.53226968,344.77196227)(132.46226975,344.76696227)(132.3922757,344.75696777)
\curveto(132.32226989,344.75696228)(132.25726995,344.76696227)(132.1972757,344.78696777)
\curveto(132.09727011,344.80696223)(132.02227019,344.84696219)(131.9722757,344.90696777)
\curveto(131.92227029,344.96696207)(131.87727033,345.02696201)(131.8372757,345.08696777)
\lineto(131.7172757,345.29696777)
\curveto(131.68727052,345.37696166)(131.63727057,345.4419616)(131.5672757,345.49196777)
\curveto(131.46727074,345.57196147)(131.36727084,345.63196141)(131.2672757,345.67196777)
\curveto(131.17727103,345.71196133)(131.06227115,345.74696129)(130.9222757,345.77696777)
\curveto(130.85227136,345.79696124)(130.74727146,345.81196123)(130.6072757,345.82196777)
\curveto(130.47727173,345.83196121)(130.37727183,345.82696121)(130.3072757,345.80696777)
\lineto(130.2022757,345.80696777)
\lineto(130.0522757,345.77696777)
\curveto(130.0122722,345.77696126)(129.96727224,345.77196127)(129.9172757,345.76196777)
\curveto(129.74727246,345.71196133)(129.6072726,345.6419614)(129.4972757,345.55196777)
\curveto(129.39727281,345.47196157)(129.32727288,345.34696169)(129.2872757,345.17696777)
\curveto(129.26727294,345.10696193)(129.26727294,345.041962)(129.2872757,344.98196777)
\curveto(129.3072729,344.92196212)(129.32727288,344.87196217)(129.3472757,344.83196777)
\curveto(129.41727279,344.71196233)(129.49727271,344.61696242)(129.5872757,344.54696777)
\curveto(129.68727252,344.47696256)(129.80227241,344.41696262)(129.9322757,344.36696777)
\curveto(130.12227209,344.28696275)(130.32727188,344.21696282)(130.5472757,344.15696777)
\lineto(131.2372757,344.00696777)
\curveto(131.47727073,343.96696307)(131.7072705,343.91696312)(131.9272757,343.85696777)
\curveto(132.15727005,343.80696323)(132.37226984,343.7419633)(132.5722757,343.66196777)
\curveto(132.66226955,343.62196342)(132.74726946,343.58696345)(132.8272757,343.55696777)
\curveto(132.91726929,343.5369635)(133.00226921,343.50196354)(133.0822757,343.45196777)
\curveto(133.27226894,343.33196371)(133.44226877,343.20196384)(133.5922757,343.06196777)
\curveto(133.75226846,342.92196412)(133.87726833,342.74696429)(133.9672757,342.53696777)
\curveto(133.99726821,342.46696457)(134.02226819,342.39696464)(134.0422757,342.32696777)
\curveto(134.06226815,342.25696478)(134.08226813,342.18196486)(134.1022757,342.10196777)
\curveto(134.1122681,342.041965)(134.11726809,341.94696509)(134.1172757,341.81696777)
\curveto(134.12726808,341.69696534)(134.12726808,341.60196544)(134.1172757,341.53196777)
\lineto(134.1172757,341.45696777)
\curveto(134.09726811,341.39696564)(134.08226813,341.3369657)(134.0722757,341.27696777)
\curveto(134.07226814,341.22696581)(134.06726814,341.17696586)(134.0572757,341.12696777)
\curveto(133.98726822,340.82696621)(133.87726833,340.56196648)(133.7272757,340.33196777)
\curveto(133.56726864,340.09196695)(133.37226884,339.89696714)(133.1422757,339.74696777)
\curveto(132.9122693,339.59696744)(132.65226956,339.46696757)(132.3622757,339.35696777)
\curveto(132.25226996,339.30696773)(132.13227008,339.27196777)(132.0022757,339.25196777)
\curveto(131.88227033,339.23196781)(131.76227045,339.20696783)(131.6422757,339.17696777)
\curveto(131.55227066,339.15696788)(131.45727075,339.14696789)(131.3572757,339.14696777)
\curveto(131.26727094,339.1369679)(131.17727103,339.12196792)(131.0872757,339.10196777)
\lineto(130.8172757,339.10196777)
\curveto(130.75727145,339.08196796)(130.65227156,339.07196797)(130.5022757,339.07196777)
\curveto(130.36227185,339.07196797)(130.26227195,339.08196796)(130.2022757,339.10196777)
\curveto(130.17227204,339.10196794)(130.13727207,339.10696793)(130.0972757,339.11696777)
\lineto(129.9922757,339.11696777)
\curveto(129.87227234,339.1369679)(129.75227246,339.15196789)(129.6322757,339.16196777)
\curveto(129.5122727,339.17196787)(129.39727281,339.19196785)(129.2872757,339.22196777)
\curveto(128.89727331,339.33196771)(128.55227366,339.45696758)(128.2522757,339.59696777)
\curveto(127.95227426,339.74696729)(127.69727451,339.96696707)(127.4872757,340.25696777)
\curveto(127.34727486,340.44696659)(127.22727498,340.66696637)(127.1272757,340.91696777)
\curveto(127.1072751,340.97696606)(127.08727512,341.05696598)(127.0672757,341.15696777)
\curveto(127.04727516,341.20696583)(127.03227518,341.27696576)(127.0222757,341.36696777)
\curveto(127.0122752,341.45696558)(127.01727519,341.53196551)(127.0372757,341.59196777)
\curveto(127.06727514,341.66196538)(127.11727509,341.71196533)(127.1872757,341.74196777)
\curveto(127.23727497,341.76196528)(127.29727491,341.77196527)(127.3672757,341.77196777)
\lineto(127.5922757,341.77196777)
\lineto(128.2972757,341.77196777)
\lineto(128.5372757,341.77196777)
\curveto(128.61727359,341.77196527)(128.68727352,341.76196528)(128.7472757,341.74196777)
\curveto(128.85727335,341.70196534)(128.92727328,341.6369654)(128.9572757,341.54696777)
\curveto(128.99727321,341.45696558)(129.04227317,341.36196568)(129.0922757,341.26196777)
\curveto(129.1122731,341.21196583)(129.14727306,341.14696589)(129.1972757,341.06696777)
\curveto(129.25727295,340.98696605)(129.3072729,340.9369661)(129.3472757,340.91696777)
\curveto(129.46727274,340.81696622)(129.58227263,340.7369663)(129.6922757,340.67696777)
\curveto(129.80227241,340.62696641)(129.94227227,340.57696646)(130.1122757,340.52696777)
\curveto(130.16227205,340.50696653)(130.212272,340.49696654)(130.2622757,340.49696777)
\curveto(130.3122719,340.50696653)(130.36227185,340.50696653)(130.4122757,340.49696777)
\curveto(130.49227172,340.47696656)(130.57727163,340.46696657)(130.6672757,340.46696777)
\curveto(130.76727144,340.47696656)(130.85227136,340.49196655)(130.9222757,340.51196777)
\curveto(130.97227124,340.52196652)(131.01727119,340.52696651)(131.0572757,340.52696777)
\curveto(131.1072711,340.52696651)(131.15727105,340.5369665)(131.2072757,340.55696777)
\curveto(131.34727086,340.60696643)(131.47227074,340.66696637)(131.5822757,340.73696777)
\curveto(131.70227051,340.80696623)(131.79727041,340.89696614)(131.8672757,341.00696777)
\curveto(131.91727029,341.08696595)(131.95727025,341.21196583)(131.9872757,341.38196777)
\curveto(132.0072702,341.45196559)(132.0072702,341.51696552)(131.9872757,341.57696777)
\curveto(131.96727024,341.6369654)(131.94727026,341.68696535)(131.9272757,341.72696777)
\curveto(131.85727035,341.86696517)(131.76727044,341.97196507)(131.6572757,342.04196777)
\curveto(131.55727065,342.11196493)(131.43727077,342.17696486)(131.2972757,342.23696777)
\curveto(131.1072711,342.31696472)(130.9072713,342.38196466)(130.6972757,342.43196777)
\curveto(130.48727172,342.48196456)(130.27727193,342.5369645)(130.0672757,342.59696777)
\curveto(129.98727222,342.61696442)(129.90227231,342.63196441)(129.8122757,342.64196777)
\curveto(129.73227248,342.65196439)(129.65227256,342.66696437)(129.5722757,342.68696777)
\curveto(129.25227296,342.77696426)(128.94727326,342.86196418)(128.6572757,342.94196777)
\curveto(128.36727384,343.03196401)(128.10227411,343.16196388)(127.8622757,343.33196777)
\curveto(127.58227463,343.53196351)(127.37727483,343.80196324)(127.2472757,344.14196777)
\curveto(127.22727498,344.21196283)(127.207275,344.30696273)(127.1872757,344.42696777)
\curveto(127.16727504,344.49696254)(127.15227506,344.58196246)(127.1422757,344.68196777)
\curveto(127.13227508,344.78196226)(127.13727507,344.87196217)(127.1572757,344.95196777)
\curveto(127.17727503,345.00196204)(127.18227503,345.041962)(127.1722757,345.07196777)
\curveto(127.16227505,345.11196193)(127.16727504,345.15696188)(127.1872757,345.20696777)
\curveto(127.207275,345.31696172)(127.22727498,345.41696162)(127.2472757,345.50696777)
\curveto(127.27727493,345.60696143)(127.3122749,345.70196134)(127.3522757,345.79196777)
\curveto(127.48227473,346.08196096)(127.66227455,346.31696072)(127.8922757,346.49696777)
\curveto(128.12227409,346.67696036)(128.38227383,346.82196022)(128.6722757,346.93196777)
\curveto(128.78227343,346.98196006)(128.89727331,347.01696002)(129.0172757,347.03696777)
\curveto(129.13727307,347.06695997)(129.26227295,347.09695994)(129.3922757,347.12696777)
\curveto(129.45227276,347.14695989)(129.5122727,347.15695988)(129.5722757,347.15696777)
\lineto(129.7522757,347.18696777)
\curveto(129.83227238,347.19695984)(129.91727229,347.20195984)(130.0072757,347.20196777)
\curveto(130.09727211,347.20195984)(130.18227203,347.20695983)(130.2622757,347.21696777)
}
}
{
\newrgbcolor{curcolor}{0 0 0}
\pscustom[linestyle=none,fillstyle=solid,fillcolor=curcolor]
{
\newpath
\moveto(136.39891632,349.31696777)
\lineto(137.40391632,349.31696777)
\curveto(137.55391334,349.31695772)(137.68391321,349.30695773)(137.79391632,349.28696777)
\curveto(137.91391298,349.27695776)(137.99891289,349.21695782)(138.04891632,349.10696777)
\curveto(138.06891282,349.05695798)(138.07891281,348.99695804)(138.07891632,348.92696777)
\lineto(138.07891632,348.71696777)
\lineto(138.07891632,348.04196777)
\curveto(138.07891281,347.99195905)(138.07391282,347.93195911)(138.06391632,347.86196777)
\curveto(138.06391283,347.80195924)(138.06891282,347.74695929)(138.07891632,347.69696777)
\lineto(138.07891632,347.53196777)
\curveto(138.07891281,347.45195959)(138.08391281,347.37695966)(138.09391632,347.30696777)
\curveto(138.10391279,347.24695979)(138.12891276,347.19195985)(138.16891632,347.14196777)
\curveto(138.23891265,347.05195999)(138.36391253,347.00196004)(138.54391632,346.99196777)
\lineto(139.08391632,346.99196777)
\lineto(139.26391632,346.99196777)
\curveto(139.32391157,346.99196005)(139.37891151,346.98196006)(139.42891632,346.96196777)
\curveto(139.53891135,346.91196013)(139.59891129,346.82196022)(139.60891632,346.69196777)
\curveto(139.62891126,346.56196048)(139.63891125,346.41696062)(139.63891632,346.25696777)
\lineto(139.63891632,346.04696777)
\curveto(139.64891124,345.97696106)(139.64391125,345.91696112)(139.62391632,345.86696777)
\curveto(139.57391132,345.70696133)(139.46891142,345.62196142)(139.30891632,345.61196777)
\curveto(139.14891174,345.60196144)(138.96891192,345.59696144)(138.76891632,345.59696777)
\lineto(138.63391632,345.59696777)
\curveto(138.5939123,345.60696143)(138.55891233,345.60696143)(138.52891632,345.59696777)
\curveto(138.4889124,345.58696145)(138.45391244,345.58196146)(138.42391632,345.58196777)
\curveto(138.3939125,345.59196145)(138.36391253,345.58696145)(138.33391632,345.56696777)
\curveto(138.25391264,345.54696149)(138.1939127,345.50196154)(138.15391632,345.43196777)
\curveto(138.12391277,345.37196167)(138.09891279,345.29696174)(138.07891632,345.20696777)
\curveto(138.06891282,345.15696188)(138.06891282,345.10196194)(138.07891632,345.04196777)
\curveto(138.0889128,344.98196206)(138.0889128,344.92696211)(138.07891632,344.87696777)
\lineto(138.07891632,343.94696777)
\lineto(138.07891632,342.19196777)
\curveto(138.07891281,341.9419651)(138.08391281,341.72196532)(138.09391632,341.53196777)
\curveto(138.11391278,341.35196569)(138.17891271,341.19196585)(138.28891632,341.05196777)
\curveto(138.33891255,340.99196605)(138.40391249,340.94696609)(138.48391632,340.91696777)
\lineto(138.75391632,340.85696777)
\curveto(138.78391211,340.84696619)(138.81391208,340.8419662)(138.84391632,340.84196777)
\curveto(138.88391201,340.85196619)(138.91391198,340.85196619)(138.93391632,340.84196777)
\lineto(139.09891632,340.84196777)
\curveto(139.20891168,340.8419662)(139.30391159,340.8369662)(139.38391632,340.82696777)
\curveto(139.46391143,340.81696622)(139.52891136,340.77696626)(139.57891632,340.70696777)
\curveto(139.61891127,340.64696639)(139.63891125,340.56696647)(139.63891632,340.46696777)
\lineto(139.63891632,340.18196777)
\curveto(139.63891125,339.97196707)(139.63391126,339.77696726)(139.62391632,339.59696777)
\curveto(139.62391127,339.42696761)(139.54391135,339.31196773)(139.38391632,339.25196777)
\curveto(139.33391156,339.23196781)(139.2889116,339.22696781)(139.24891632,339.23696777)
\curveto(139.20891168,339.2369678)(139.16391173,339.22696781)(139.11391632,339.20696777)
\lineto(138.96391632,339.20696777)
\curveto(138.94391195,339.20696783)(138.91391198,339.21196783)(138.87391632,339.22196777)
\curveto(138.83391206,339.22196782)(138.79891209,339.21696782)(138.76891632,339.20696777)
\curveto(138.71891217,339.19696784)(138.66391223,339.19696784)(138.60391632,339.20696777)
\lineto(138.45391632,339.20696777)
\lineto(138.30391632,339.20696777)
\curveto(138.25391264,339.19696784)(138.20891268,339.19696784)(138.16891632,339.20696777)
\lineto(138.00391632,339.20696777)
\curveto(137.95391294,339.21696782)(137.89891299,339.22196782)(137.83891632,339.22196777)
\curveto(137.77891311,339.22196782)(137.72391317,339.22696781)(137.67391632,339.23696777)
\curveto(137.60391329,339.24696779)(137.53891335,339.25696778)(137.47891632,339.26696777)
\lineto(137.29891632,339.29696777)
\curveto(137.1889137,339.32696771)(137.08391381,339.36196768)(136.98391632,339.40196777)
\curveto(136.88391401,339.4419676)(136.7889141,339.48696755)(136.69891632,339.53696777)
\lineto(136.60891632,339.59696777)
\curveto(136.57891431,339.62696741)(136.54391435,339.65696738)(136.50391632,339.68696777)
\curveto(136.48391441,339.70696733)(136.45891443,339.72696731)(136.42891632,339.74696777)
\lineto(136.35391632,339.82196777)
\curveto(136.21391468,340.01196703)(136.10891478,340.22196682)(136.03891632,340.45196777)
\curveto(136.01891487,340.49196655)(136.00891488,340.52696651)(136.00891632,340.55696777)
\curveto(136.01891487,340.59696644)(136.01891487,340.6419664)(136.00891632,340.69196777)
\curveto(135.99891489,340.71196633)(135.9939149,340.7369663)(135.99391632,340.76696777)
\curveto(135.9939149,340.79696624)(135.9889149,340.82196622)(135.97891632,340.84196777)
\lineto(135.97891632,340.99196777)
\curveto(135.96891492,341.03196601)(135.96391493,341.07696596)(135.96391632,341.12696777)
\curveto(135.97391492,341.17696586)(135.97891491,341.22696581)(135.97891632,341.27696777)
\lineto(135.97891632,341.84696777)
\lineto(135.97891632,344.08196777)
\lineto(135.97891632,344.87696777)
\lineto(135.97891632,345.08696777)
\curveto(135.9889149,345.15696188)(135.98391491,345.22196182)(135.96391632,345.28196777)
\curveto(135.92391497,345.42196162)(135.85391504,345.51196153)(135.75391632,345.55196777)
\curveto(135.64391525,345.60196144)(135.50391539,345.61696142)(135.33391632,345.59696777)
\curveto(135.16391573,345.57696146)(135.01891587,345.59196145)(134.89891632,345.64196777)
\curveto(134.81891607,345.67196137)(134.76891612,345.71696132)(134.74891632,345.77696777)
\curveto(134.72891616,345.8369612)(134.70891618,345.91196113)(134.68891632,346.00196777)
\lineto(134.68891632,346.31696777)
\curveto(134.6889162,346.49696054)(134.69891619,346.6419604)(134.71891632,346.75196777)
\curveto(134.73891615,346.86196018)(134.82391607,346.9369601)(134.97391632,346.97696777)
\curveto(135.01391588,346.99696004)(135.05391584,347.00196004)(135.09391632,346.99196777)
\lineto(135.22891632,346.99196777)
\curveto(135.37891551,346.99196005)(135.51891537,346.99696004)(135.64891632,347.00696777)
\curveto(135.77891511,347.02696001)(135.86891502,347.08695995)(135.91891632,347.18696777)
\curveto(135.94891494,347.25695978)(135.96391493,347.3369597)(135.96391632,347.42696777)
\curveto(135.97391492,347.51695952)(135.97891491,347.60695943)(135.97891632,347.69696777)
\lineto(135.97891632,348.62696777)
\lineto(135.97891632,348.88196777)
\curveto(135.97891491,348.97195807)(135.9889149,349.04695799)(136.00891632,349.10696777)
\curveto(136.05891483,349.20695783)(136.13391476,349.27195777)(136.23391632,349.30196777)
\curveto(136.25391464,349.31195773)(136.27891461,349.31195773)(136.30891632,349.30196777)
\curveto(136.34891454,349.30195774)(136.37891451,349.30695773)(136.39891632,349.31696777)
}
}
{
\newrgbcolor{curcolor}{0 0 0}
\pscustom[linestyle=none,fillstyle=solid,fillcolor=curcolor]
{
\newpath
\moveto(147.67235382,339.86696777)
\curveto(147.69234597,339.75696728)(147.70234596,339.64696739)(147.70235382,339.53696777)
\curveto(147.71234595,339.42696761)(147.662346,339.35196769)(147.55235382,339.31196777)
\curveto(147.49234617,339.28196776)(147.42234624,339.26696777)(147.34235382,339.26696777)
\lineto(147.10235382,339.26696777)
\lineto(146.29235382,339.26696777)
\lineto(146.02235382,339.26696777)
\curveto(145.94234772,339.27696776)(145.87734779,339.30196774)(145.82735382,339.34196777)
\curveto(145.75734791,339.38196766)(145.70234796,339.4369676)(145.66235382,339.50696777)
\curveto(145.63234803,339.58696745)(145.58734808,339.65196739)(145.52735382,339.70196777)
\curveto(145.50734816,339.72196732)(145.48234818,339.7369673)(145.45235382,339.74696777)
\curveto(145.42234824,339.76696727)(145.38234828,339.77196727)(145.33235382,339.76196777)
\curveto(145.28234838,339.7419673)(145.23234843,339.71696732)(145.18235382,339.68696777)
\curveto(145.14234852,339.65696738)(145.09734857,339.63196741)(145.04735382,339.61196777)
\curveto(144.99734867,339.57196747)(144.94234872,339.5369675)(144.88235382,339.50696777)
\lineto(144.70235382,339.41696777)
\curveto(144.57234909,339.35696768)(144.43734923,339.30696773)(144.29735382,339.26696777)
\curveto(144.15734951,339.2369678)(144.01234965,339.20196784)(143.86235382,339.16196777)
\curveto(143.79234987,339.1419679)(143.72234994,339.13196791)(143.65235382,339.13196777)
\curveto(143.59235007,339.12196792)(143.52735014,339.11196793)(143.45735382,339.10196777)
\lineto(143.36735382,339.10196777)
\curveto(143.33735033,339.09196795)(143.30735036,339.08696795)(143.27735382,339.08696777)
\lineto(143.11235382,339.08696777)
\curveto(143.01235065,339.06696797)(142.91235075,339.06696797)(142.81235382,339.08696777)
\lineto(142.67735382,339.08696777)
\curveto(142.60735106,339.10696793)(142.53735113,339.11696792)(142.46735382,339.11696777)
\curveto(142.40735126,339.10696793)(142.34735132,339.11196793)(142.28735382,339.13196777)
\curveto(142.18735148,339.15196789)(142.09235157,339.17196787)(142.00235382,339.19196777)
\curveto(141.91235175,339.20196784)(141.82735184,339.22696781)(141.74735382,339.26696777)
\curveto(141.45735221,339.37696766)(141.20735246,339.51696752)(140.99735382,339.68696777)
\curveto(140.79735287,339.86696717)(140.63735303,340.10196694)(140.51735382,340.39196777)
\curveto(140.48735318,340.46196658)(140.45735321,340.5369665)(140.42735382,340.61696777)
\curveto(140.40735326,340.69696634)(140.38735328,340.78196626)(140.36735382,340.87196777)
\curveto(140.34735332,340.92196612)(140.33735333,340.97196607)(140.33735382,341.02196777)
\curveto(140.34735332,341.07196597)(140.34735332,341.12196592)(140.33735382,341.17196777)
\curveto(140.32735334,341.20196584)(140.31735335,341.26196578)(140.30735382,341.35196777)
\curveto(140.30735336,341.45196559)(140.31235335,341.52196552)(140.32235382,341.56196777)
\curveto(140.34235332,341.66196538)(140.35235331,341.74696529)(140.35235382,341.81696777)
\lineto(140.44235382,342.14696777)
\curveto(140.47235319,342.26696477)(140.51235315,342.37196467)(140.56235382,342.46196777)
\curveto(140.73235293,342.75196429)(140.92735274,342.97196407)(141.14735382,343.12196777)
\curveto(141.3673523,343.27196377)(141.64735202,343.40196364)(141.98735382,343.51196777)
\curveto(142.11735155,343.56196348)(142.25235141,343.59696344)(142.39235382,343.61696777)
\curveto(142.53235113,343.6369634)(142.67235099,343.66196338)(142.81235382,343.69196777)
\curveto(142.89235077,343.71196333)(142.97735069,343.72196332)(143.06735382,343.72196777)
\curveto(143.15735051,343.73196331)(143.24735042,343.74696329)(143.33735382,343.76696777)
\curveto(143.40735026,343.78696325)(143.47735019,343.79196325)(143.54735382,343.78196777)
\curveto(143.61735005,343.78196326)(143.69234997,343.79196325)(143.77235382,343.81196777)
\curveto(143.84234982,343.83196321)(143.91234975,343.8419632)(143.98235382,343.84196777)
\curveto(144.05234961,343.8419632)(144.12734954,343.85196319)(144.20735382,343.87196777)
\curveto(144.41734925,343.92196312)(144.60734906,343.96196308)(144.77735382,343.99196777)
\curveto(144.95734871,344.03196301)(145.11734855,344.12196292)(145.25735382,344.26196777)
\curveto(145.34734832,344.35196269)(145.40734826,344.45196259)(145.43735382,344.56196777)
\curveto(145.44734822,344.59196245)(145.44734822,344.61696242)(145.43735382,344.63696777)
\curveto(145.43734823,344.65696238)(145.44234822,344.67696236)(145.45235382,344.69696777)
\curveto(145.4623482,344.71696232)(145.4673482,344.74696229)(145.46735382,344.78696777)
\lineto(145.46735382,344.87696777)
\lineto(145.43735382,344.99696777)
\curveto(145.43734823,345.036962)(145.43234823,345.07196197)(145.42235382,345.10196777)
\curveto(145.32234834,345.40196164)(145.11234855,345.60696143)(144.79235382,345.71696777)
\curveto(144.70234896,345.74696129)(144.59234907,345.76696127)(144.46235382,345.77696777)
\curveto(144.34234932,345.79696124)(144.21734945,345.80196124)(144.08735382,345.79196777)
\curveto(143.95734971,345.79196125)(143.83234983,345.78196126)(143.71235382,345.76196777)
\curveto(143.59235007,345.7419613)(143.48735018,345.71696132)(143.39735382,345.68696777)
\curveto(143.33735033,345.66696137)(143.27735039,345.6369614)(143.21735382,345.59696777)
\curveto(143.1673505,345.56696147)(143.11735055,345.53196151)(143.06735382,345.49196777)
\curveto(143.01735065,345.45196159)(142.9623507,345.39696164)(142.90235382,345.32696777)
\curveto(142.85235081,345.25696178)(142.81735085,345.19196185)(142.79735382,345.13196777)
\curveto(142.74735092,345.03196201)(142.70235096,344.9369621)(142.66235382,344.84696777)
\curveto(142.63235103,344.75696228)(142.5623511,344.69696234)(142.45235382,344.66696777)
\curveto(142.37235129,344.64696239)(142.28735138,344.6369624)(142.19735382,344.63696777)
\lineto(141.92735382,344.63696777)
\lineto(141.35735382,344.63696777)
\curveto(141.30735236,344.6369624)(141.25735241,344.63196241)(141.20735382,344.62196777)
\curveto(141.15735251,344.62196242)(141.11235255,344.62696241)(141.07235382,344.63696777)
\lineto(140.93735382,344.63696777)
\curveto(140.91735275,344.64696239)(140.89235277,344.65196239)(140.86235382,344.65196777)
\curveto(140.83235283,344.65196239)(140.80735286,344.66196238)(140.78735382,344.68196777)
\curveto(140.70735296,344.70196234)(140.65235301,344.76696227)(140.62235382,344.87696777)
\curveto(140.61235305,344.92696211)(140.61235305,344.97696206)(140.62235382,345.02696777)
\curveto(140.63235303,345.07696196)(140.64235302,345.12196192)(140.65235382,345.16196777)
\curveto(140.68235298,345.27196177)(140.71235295,345.37196167)(140.74235382,345.46196777)
\curveto(140.78235288,345.56196148)(140.82735284,345.65196139)(140.87735382,345.73196777)
\lineto(140.96735382,345.88196777)
\lineto(141.05735382,346.03196777)
\curveto(141.13735253,346.1419609)(141.23735243,346.24696079)(141.35735382,346.34696777)
\curveto(141.37735229,346.35696068)(141.40735226,346.38196066)(141.44735382,346.42196777)
\curveto(141.49735217,346.46196058)(141.54235212,346.49696054)(141.58235382,346.52696777)
\curveto(141.62235204,346.55696048)(141.667352,346.58696045)(141.71735382,346.61696777)
\curveto(141.88735178,346.72696031)(142.0673516,346.81196023)(142.25735382,346.87196777)
\curveto(142.44735122,346.9419601)(142.64235102,347.00696003)(142.84235382,347.06696777)
\curveto(142.9623507,347.09695994)(143.08735058,347.11695992)(143.21735382,347.12696777)
\curveto(143.34735032,347.1369599)(143.47735019,347.15695988)(143.60735382,347.18696777)
\curveto(143.64735002,347.19695984)(143.70734996,347.19695984)(143.78735382,347.18696777)
\curveto(143.87734979,347.17695986)(143.93234973,347.18195986)(143.95235382,347.20196777)
\curveto(144.3623493,347.21195983)(144.75234891,347.19695984)(145.12235382,347.15696777)
\curveto(145.50234816,347.11695992)(145.84234782,347.04196)(146.14235382,346.93196777)
\curveto(146.45234721,346.82196022)(146.71734695,346.67196037)(146.93735382,346.48196777)
\curveto(147.15734651,346.30196074)(147.32734634,346.06696097)(147.44735382,345.77696777)
\curveto(147.51734615,345.60696143)(147.55734611,345.41196163)(147.56735382,345.19196777)
\curveto(147.57734609,344.97196207)(147.58234608,344.74696229)(147.58235382,344.51696777)
\lineto(147.58235382,341.17196777)
\lineto(147.58235382,340.58696777)
\curveto(147.58234608,340.39696664)(147.60234606,340.22196682)(147.64235382,340.06196777)
\curveto(147.65234601,340.03196701)(147.65734601,339.99696704)(147.65735382,339.95696777)
\curveto(147.65734601,339.92696711)(147.662346,339.89696714)(147.67235382,339.86696777)
\moveto(145.46735382,342.17696777)
\curveto(145.47734819,342.22696481)(145.48234818,342.28196476)(145.48235382,342.34196777)
\curveto(145.48234818,342.41196463)(145.47734819,342.47196457)(145.46735382,342.52196777)
\curveto(145.44734822,342.58196446)(145.43734823,342.6369644)(145.43735382,342.68696777)
\curveto(145.43734823,342.7369643)(145.41734825,342.77696426)(145.37735382,342.80696777)
\curveto(145.32734834,342.84696419)(145.25234841,342.86696417)(145.15235382,342.86696777)
\curveto(145.11234855,342.85696418)(145.07734859,342.84696419)(145.04735382,342.83696777)
\curveto(145.01734865,342.8369642)(144.98234868,342.83196421)(144.94235382,342.82196777)
\curveto(144.87234879,342.80196424)(144.79734887,342.78696425)(144.71735382,342.77696777)
\curveto(144.63734903,342.76696427)(144.55734911,342.75196429)(144.47735382,342.73196777)
\curveto(144.44734922,342.72196432)(144.40234926,342.71696432)(144.34235382,342.71696777)
\curveto(144.21234945,342.68696435)(144.08234958,342.66696437)(143.95235382,342.65696777)
\curveto(143.82234984,342.64696439)(143.69734997,342.62196442)(143.57735382,342.58196777)
\curveto(143.49735017,342.56196448)(143.42235024,342.5419645)(143.35235382,342.52196777)
\curveto(143.28235038,342.51196453)(143.21235045,342.49196455)(143.14235382,342.46196777)
\curveto(142.93235073,342.37196467)(142.75235091,342.2369648)(142.60235382,342.05696777)
\curveto(142.4623512,341.87696516)(142.41235125,341.62696541)(142.45235382,341.30696777)
\curveto(142.47235119,341.1369659)(142.52735114,340.99696604)(142.61735382,340.88696777)
\curveto(142.68735098,340.77696626)(142.79235087,340.68696635)(142.93235382,340.61696777)
\curveto(143.07235059,340.55696648)(143.22235044,340.51196653)(143.38235382,340.48196777)
\curveto(143.55235011,340.45196659)(143.72734994,340.4419666)(143.90735382,340.45196777)
\curveto(144.09734957,340.47196657)(144.27234939,340.50696653)(144.43235382,340.55696777)
\curveto(144.69234897,340.6369664)(144.89734877,340.76196628)(145.04735382,340.93196777)
\curveto(145.19734847,341.11196593)(145.31234835,341.33196571)(145.39235382,341.59196777)
\curveto(145.41234825,341.66196538)(145.42234824,341.73196531)(145.42235382,341.80196777)
\curveto(145.43234823,341.88196516)(145.44734822,341.96196508)(145.46735382,342.04196777)
\lineto(145.46735382,342.17696777)
}
}
{
\newrgbcolor{curcolor}{0 0 0}
\pscustom[linestyle=none,fillstyle=solid,fillcolor=curcolor]
{
\newpath
\moveto(157.03563507,343.52696777)
\curveto(157.05562647,343.46696357)(157.06562646,343.36196368)(157.06563507,343.21196777)
\curveto(157.06562646,343.07196397)(157.06062647,342.97196407)(157.05063507,342.91196777)
\curveto(157.05062648,342.86196418)(157.04562648,342.81696422)(157.03563507,342.77696777)
\lineto(157.03563507,342.65696777)
\curveto(157.01562651,342.57696446)(157.00562652,342.49696454)(157.00563507,342.41696777)
\curveto(157.00562652,342.34696469)(156.99562653,342.27196477)(156.97563507,342.19196777)
\curveto(156.97562655,342.15196489)(156.96562656,342.08196496)(156.94563507,341.98196777)
\curveto(156.91562661,341.86196518)(156.88562664,341.7369653)(156.85563507,341.60696777)
\curveto(156.83562669,341.48696555)(156.80062673,341.37196567)(156.75063507,341.26196777)
\curveto(156.57062696,340.81196623)(156.34562718,340.42196662)(156.07563507,340.09196777)
\curveto(155.80562772,339.76196728)(155.45062808,339.50196754)(155.01063507,339.31196777)
\curveto(154.92062861,339.27196777)(154.8256287,339.2419678)(154.72563507,339.22196777)
\curveto(154.63562889,339.19196785)(154.53562899,339.16196788)(154.42563507,339.13196777)
\curveto(154.36562916,339.11196793)(154.30062923,339.10196794)(154.23063507,339.10196777)
\curveto(154.17062936,339.10196794)(154.11062942,339.09696794)(154.05063507,339.08696777)
\lineto(153.91563507,339.08696777)
\curveto(153.85562967,339.06696797)(153.77562975,339.06196798)(153.67563507,339.07196777)
\curveto(153.57562995,339.07196797)(153.49563003,339.08196796)(153.43563507,339.10196777)
\lineto(153.34563507,339.10196777)
\curveto(153.29563023,339.11196793)(153.24063029,339.12196792)(153.18063507,339.13196777)
\curveto(153.12063041,339.13196791)(153.06063047,339.1369679)(153.00063507,339.14696777)
\curveto(152.81063072,339.19696784)(152.63563089,339.24696779)(152.47563507,339.29696777)
\curveto(152.31563121,339.34696769)(152.16563136,339.41696762)(152.02563507,339.50696777)
\lineto(151.84563507,339.62696777)
\curveto(151.79563173,339.66696737)(151.74563178,339.71196733)(151.69563507,339.76196777)
\lineto(151.60563507,339.82196777)
\curveto(151.57563195,339.8419672)(151.54563198,339.85696718)(151.51563507,339.86696777)
\curveto(151.4256321,339.89696714)(151.37063216,339.87696716)(151.35063507,339.80696777)
\curveto(151.30063223,339.7369673)(151.26563226,339.65196739)(151.24563507,339.55196777)
\curveto(151.23563229,339.46196758)(151.20063233,339.39196765)(151.14063507,339.34196777)
\curveto(151.08063245,339.30196774)(151.01063252,339.27696776)(150.93063507,339.26696777)
\lineto(150.66063507,339.26696777)
\lineto(149.94063507,339.26696777)
\lineto(149.71563507,339.26696777)
\curveto(149.64563388,339.25696778)(149.58063395,339.26196778)(149.52063507,339.28196777)
\curveto(149.38063415,339.33196771)(149.30063423,339.42196762)(149.28063507,339.55196777)
\curveto(149.27063426,339.69196735)(149.26563426,339.84696719)(149.26563507,340.01696777)
\lineto(149.26563507,349.16696777)
\lineto(149.26563507,349.51196777)
\curveto(149.26563426,349.63195741)(149.29063424,349.72695731)(149.34063507,349.79696777)
\curveto(149.38063415,349.86695717)(149.45063408,349.91195713)(149.55063507,349.93196777)
\curveto(149.57063396,349.9419571)(149.59063394,349.9419571)(149.61063507,349.93196777)
\curveto(149.64063389,349.93195711)(149.66563386,349.9369571)(149.68563507,349.94696777)
\lineto(150.63063507,349.94696777)
\curveto(150.81063272,349.94695709)(150.96563256,349.9369571)(151.09563507,349.91696777)
\curveto(151.2256323,349.90695713)(151.31063222,349.83195721)(151.35063507,349.69196777)
\curveto(151.38063215,349.59195745)(151.39063214,349.45695758)(151.38063507,349.28696777)
\curveto(151.37063216,349.12695791)(151.36563216,348.98695805)(151.36563507,348.86696777)
\lineto(151.36563507,347.23196777)
\lineto(151.36563507,346.90196777)
\curveto(151.36563216,346.79196025)(151.37563215,346.69696034)(151.39563507,346.61696777)
\curveto(151.40563212,346.56696047)(151.41563211,346.52196052)(151.42563507,346.48196777)
\curveto(151.43563209,346.45196059)(151.46063207,346.43196061)(151.50063507,346.42196777)
\curveto(151.52063201,346.40196064)(151.54563198,346.39196065)(151.57563507,346.39196777)
\curveto(151.61563191,346.39196065)(151.64563188,346.39696064)(151.66563507,346.40696777)
\curveto(151.73563179,346.44696059)(151.80063173,346.48696055)(151.86063507,346.52696777)
\curveto(151.92063161,346.57696046)(151.98563154,346.62696041)(152.05563507,346.67696777)
\curveto(152.18563134,346.76696027)(152.32063121,346.8419602)(152.46063507,346.90196777)
\curveto(152.60063093,346.97196007)(152.75563077,347.03196001)(152.92563507,347.08196777)
\curveto(153.00563052,347.11195993)(153.08563044,347.12695991)(153.16563507,347.12696777)
\curveto(153.24563028,347.1369599)(153.3256302,347.15195989)(153.40563507,347.17196777)
\curveto(153.47563005,347.19195985)(153.55062998,347.20195984)(153.63063507,347.20196777)
\lineto(153.87063507,347.20196777)
\lineto(154.02063507,347.20196777)
\curveto(154.05062948,347.19195985)(154.08562944,347.18695985)(154.12563507,347.18696777)
\curveto(154.16562936,347.19695984)(154.20562932,347.19695984)(154.24563507,347.18696777)
\curveto(154.35562917,347.15695988)(154.45562907,347.13195991)(154.54563507,347.11196777)
\curveto(154.64562888,347.10195994)(154.74062879,347.07695996)(154.83063507,347.03696777)
\curveto(155.29062824,346.84696019)(155.66562786,346.60196044)(155.95563507,346.30196777)
\curveto(156.24562728,346.00196104)(156.49062704,345.62696141)(156.69063507,345.17696777)
\curveto(156.74062679,345.05696198)(156.78062675,344.93196211)(156.81063507,344.80196777)
\curveto(156.85062668,344.67196237)(156.89062664,344.5369625)(156.93063507,344.39696777)
\curveto(156.95062658,344.32696271)(156.96062657,344.25696278)(156.96063507,344.18696777)
\curveto(156.97062656,344.12696291)(156.98562654,344.05696298)(157.00563507,343.97696777)
\curveto(157.0256265,343.92696311)(157.0306265,343.87196317)(157.02063507,343.81196777)
\curveto(157.02062651,343.75196329)(157.0256265,343.69196335)(157.03563507,343.63196777)
\lineto(157.03563507,343.52696777)
\moveto(154.81563507,342.11696777)
\curveto(154.84562868,342.21696482)(154.87062866,342.3419647)(154.89063507,342.49196777)
\curveto(154.92062861,342.6419644)(154.93562859,342.79196425)(154.93563507,342.94196777)
\curveto(154.94562858,343.10196394)(154.94562858,343.25696378)(154.93563507,343.40696777)
\curveto(154.93562859,343.56696347)(154.92062861,343.70196334)(154.89063507,343.81196777)
\curveto(154.86062867,343.91196313)(154.84062869,344.00696303)(154.83063507,344.09696777)
\curveto(154.82062871,344.18696285)(154.79562873,344.27196277)(154.75563507,344.35196777)
\curveto(154.61562891,344.70196234)(154.41562911,344.99696204)(154.15563507,345.23696777)
\curveto(153.90562962,345.48696155)(153.53562999,345.61196143)(153.04563507,345.61196777)
\curveto(153.00563052,345.61196143)(152.97063056,345.60696143)(152.94063507,345.59696777)
\lineto(152.83563507,345.59696777)
\curveto(152.76563076,345.57696146)(152.70063083,345.55696148)(152.64063507,345.53696777)
\curveto(152.58063095,345.52696151)(152.52063101,345.51196153)(152.46063507,345.49196777)
\curveto(152.17063136,345.36196168)(151.95063158,345.17696186)(151.80063507,344.93696777)
\curveto(151.65063188,344.70696233)(151.525632,344.4419626)(151.42563507,344.14196777)
\curveto(151.39563213,344.06196298)(151.37563215,343.97696306)(151.36563507,343.88696777)
\curveto(151.36563216,343.80696323)(151.35563217,343.72696331)(151.33563507,343.64696777)
\curveto(151.3256322,343.61696342)(151.32063221,343.56696347)(151.32063507,343.49696777)
\curveto(151.31063222,343.45696358)(151.30563222,343.41696362)(151.30563507,343.37696777)
\curveto(151.31563221,343.3369637)(151.31563221,343.29696374)(151.30563507,343.25696777)
\curveto(151.28563224,343.17696386)(151.28063225,343.06696397)(151.29063507,342.92696777)
\curveto(151.30063223,342.78696425)(151.31563221,342.68696435)(151.33563507,342.62696777)
\curveto(151.35563217,342.5369645)(151.36563216,342.45196459)(151.36563507,342.37196777)
\curveto(151.37563215,342.29196475)(151.39563213,342.21196483)(151.42563507,342.13196777)
\curveto(151.51563201,341.85196519)(151.62063191,341.60696543)(151.74063507,341.39696777)
\curveto(151.87063166,341.19696584)(152.05063148,341.02696601)(152.28063507,340.88696777)
\curveto(152.44063109,340.78696625)(152.60563092,340.71696632)(152.77563507,340.67696777)
\curveto(152.79563073,340.67696636)(152.81563071,340.67196637)(152.83563507,340.66196777)
\lineto(152.92563507,340.66196777)
\curveto(152.95563057,340.65196639)(153.00563052,340.6419664)(153.07563507,340.63196777)
\curveto(153.14563038,340.63196641)(153.20563032,340.6369664)(153.25563507,340.64696777)
\curveto(153.35563017,340.66696637)(153.44563008,340.68196636)(153.52563507,340.69196777)
\curveto(153.61562991,340.71196633)(153.70062983,340.7369663)(153.78063507,340.76696777)
\curveto(154.06062947,340.89696614)(154.27562925,341.07696596)(154.42563507,341.30696777)
\curveto(154.58562894,341.5369655)(154.71562881,341.80696523)(154.81563507,342.11696777)
}
}
{
\newrgbcolor{curcolor}{0 0 0}
\pscustom[linestyle=none,fillstyle=solid,fillcolor=curcolor]
{
\newpath
\moveto(158.91555695,349.96196777)
\lineto(160.01055695,349.96196777)
\curveto(160.11055446,349.96195708)(160.20555437,349.95695708)(160.29555695,349.94696777)
\curveto(160.38555419,349.9369571)(160.45555412,349.90695713)(160.50555695,349.85696777)
\curveto(160.56555401,349.78695725)(160.59555398,349.69195735)(160.59555695,349.57196777)
\curveto(160.60555397,349.46195758)(160.61055396,349.34695769)(160.61055695,349.22696777)
\lineto(160.61055695,347.89196777)
\lineto(160.61055695,342.50696777)
\lineto(160.61055695,340.21196777)
\lineto(160.61055695,339.79196777)
\curveto(160.62055395,339.6419674)(160.60055397,339.52696751)(160.55055695,339.44696777)
\curveto(160.50055407,339.36696767)(160.41055416,339.31196773)(160.28055695,339.28196777)
\curveto(160.22055435,339.26196778)(160.15055442,339.25696778)(160.07055695,339.26696777)
\curveto(160.00055457,339.27696776)(159.93055464,339.28196776)(159.86055695,339.28196777)
\lineto(159.14055695,339.28196777)
\curveto(159.03055554,339.28196776)(158.93055564,339.28696775)(158.84055695,339.29696777)
\curveto(158.75055582,339.30696773)(158.6755559,339.3369677)(158.61555695,339.38696777)
\curveto(158.55555602,339.4369676)(158.52055605,339.51196753)(158.51055695,339.61196777)
\lineto(158.51055695,339.94196777)
\lineto(158.51055695,341.27696777)
\lineto(158.51055695,346.90196777)
\lineto(158.51055695,348.94196777)
\curveto(158.51055606,349.07195797)(158.50555607,349.22695781)(158.49555695,349.40696777)
\curveto(158.49555608,349.58695745)(158.52055605,349.71695732)(158.57055695,349.79696777)
\curveto(158.59055598,349.8369572)(158.61555596,349.86695717)(158.64555695,349.88696777)
\lineto(158.76555695,349.94696777)
\curveto(158.78555579,349.94695709)(158.81055576,349.94695709)(158.84055695,349.94696777)
\curveto(158.8705557,349.95695708)(158.89555568,349.96195708)(158.91555695,349.96196777)
}
}
{
\newrgbcolor{curcolor}{0 0 0}
\pscustom[linestyle=none,fillstyle=solid,fillcolor=curcolor]
{
\newpath
\moveto(169.63774445,343.21196777)
\curveto(169.65773628,343.13196391)(169.65773628,343.041964)(169.63774445,342.94196777)
\curveto(169.61773632,342.8419642)(169.58273636,342.77696426)(169.53274445,342.74696777)
\curveto(169.48273646,342.70696433)(169.40773653,342.67696436)(169.30774445,342.65696777)
\curveto(169.21773672,342.64696439)(169.11273683,342.6369644)(168.99274445,342.62696777)
\lineto(168.64774445,342.62696777)
\curveto(168.5377374,342.6369644)(168.4377375,342.6419644)(168.34774445,342.64196777)
\lineto(164.68774445,342.64196777)
\lineto(164.47774445,342.64196777)
\curveto(164.41774152,342.6419644)(164.36274158,342.63196441)(164.31274445,342.61196777)
\curveto(164.23274171,342.57196447)(164.18274176,342.53196451)(164.16274445,342.49196777)
\curveto(164.1427418,342.47196457)(164.12274182,342.43196461)(164.10274445,342.37196777)
\curveto(164.08274186,342.32196472)(164.07774186,342.27196477)(164.08774445,342.22196777)
\curveto(164.10774183,342.16196488)(164.11774182,342.10196494)(164.11774445,342.04196777)
\curveto(164.12774181,341.99196505)(164.1427418,341.9369651)(164.16274445,341.87696777)
\curveto(164.2427417,341.6369654)(164.3377416,341.4369656)(164.44774445,341.27696777)
\curveto(164.56774137,341.12696591)(164.72774121,340.99196605)(164.92774445,340.87196777)
\curveto(165.00774093,340.82196622)(165.08774085,340.78696625)(165.16774445,340.76696777)
\curveto(165.25774068,340.75696628)(165.34774059,340.7369663)(165.43774445,340.70696777)
\curveto(165.51774042,340.68696635)(165.62774031,340.67196637)(165.76774445,340.66196777)
\curveto(165.90774003,340.65196639)(166.02773991,340.65696638)(166.12774445,340.67696777)
\lineto(166.26274445,340.67696777)
\curveto(166.36273958,340.69696634)(166.45273949,340.71696632)(166.53274445,340.73696777)
\curveto(166.62273932,340.76696627)(166.70773923,340.79696624)(166.78774445,340.82696777)
\curveto(166.88773905,340.87696616)(166.99773894,340.9419661)(167.11774445,341.02196777)
\curveto(167.24773869,341.10196594)(167.3427386,341.18196586)(167.40274445,341.26196777)
\curveto(167.45273849,341.33196571)(167.50273844,341.39696564)(167.55274445,341.45696777)
\curveto(167.61273833,341.52696551)(167.68273826,341.57696546)(167.76274445,341.60696777)
\curveto(167.86273808,341.65696538)(167.98773795,341.67696536)(168.13774445,341.66696777)
\lineto(168.57274445,341.66696777)
\lineto(168.75274445,341.66696777)
\curveto(168.82273712,341.67696536)(168.88273706,341.67196537)(168.93274445,341.65196777)
\lineto(169.08274445,341.65196777)
\curveto(169.18273676,341.63196541)(169.25273669,341.60696543)(169.29274445,341.57696777)
\curveto(169.33273661,341.55696548)(169.35273659,341.51196553)(169.35274445,341.44196777)
\curveto(169.36273658,341.37196567)(169.35773658,341.31196573)(169.33774445,341.26196777)
\curveto(169.28773665,341.12196592)(169.23273671,340.99696604)(169.17274445,340.88696777)
\curveto(169.11273683,340.77696626)(169.0427369,340.66696637)(168.96274445,340.55696777)
\curveto(168.7427372,340.22696681)(168.49273745,339.96196708)(168.21274445,339.76196777)
\curveto(167.93273801,339.56196748)(167.58273836,339.39196765)(167.16274445,339.25196777)
\curveto(167.05273889,339.21196783)(166.942739,339.18696785)(166.83274445,339.17696777)
\curveto(166.72273922,339.16696787)(166.60773933,339.14696789)(166.48774445,339.11696777)
\curveto(166.44773949,339.10696793)(166.40273954,339.10696793)(166.35274445,339.11696777)
\curveto(166.31273963,339.11696792)(166.27273967,339.11196793)(166.23274445,339.10196777)
\lineto(166.06774445,339.10196777)
\curveto(166.01773992,339.08196796)(165.95773998,339.07696796)(165.88774445,339.08696777)
\curveto(165.82774011,339.08696795)(165.77274017,339.09196795)(165.72274445,339.10196777)
\curveto(165.6427403,339.11196793)(165.57274037,339.11196793)(165.51274445,339.10196777)
\curveto(165.45274049,339.09196795)(165.38774055,339.09696794)(165.31774445,339.11696777)
\curveto(165.26774067,339.1369679)(165.21274073,339.14696789)(165.15274445,339.14696777)
\curveto(165.09274085,339.14696789)(165.0377409,339.15696788)(164.98774445,339.17696777)
\curveto(164.87774106,339.19696784)(164.76774117,339.22196782)(164.65774445,339.25196777)
\curveto(164.54774139,339.27196777)(164.44774149,339.30696773)(164.35774445,339.35696777)
\curveto(164.24774169,339.39696764)(164.1427418,339.43196761)(164.04274445,339.46196777)
\curveto(163.95274199,339.50196754)(163.86774207,339.54696749)(163.78774445,339.59696777)
\curveto(163.46774247,339.79696724)(163.18274276,340.02696701)(162.93274445,340.28696777)
\curveto(162.68274326,340.55696648)(162.47774346,340.86696617)(162.31774445,341.21696777)
\curveto(162.26774367,341.32696571)(162.22774371,341.4369656)(162.19774445,341.54696777)
\curveto(162.16774377,341.66696537)(162.12774381,341.78696525)(162.07774445,341.90696777)
\curveto(162.06774387,341.94696509)(162.06274388,341.98196506)(162.06274445,342.01196777)
\curveto(162.06274388,342.05196499)(162.05774388,342.09196495)(162.04774445,342.13196777)
\curveto(162.00774393,342.25196479)(161.98274396,342.38196466)(161.97274445,342.52196777)
\lineto(161.94274445,342.94196777)
\curveto(161.942744,342.99196405)(161.937744,343.04696399)(161.92774445,343.10696777)
\curveto(161.92774401,343.16696387)(161.93274401,343.22196382)(161.94274445,343.27196777)
\lineto(161.94274445,343.45196777)
\lineto(161.98774445,343.81196777)
\curveto(162.02774391,343.98196306)(162.06274388,344.14696289)(162.09274445,344.30696777)
\curveto(162.12274382,344.46696257)(162.16774377,344.61696242)(162.22774445,344.75696777)
\curveto(162.65774328,345.79696124)(163.38774255,346.53196051)(164.41774445,346.96196777)
\curveto(164.55774138,347.02196002)(164.69774124,347.06195998)(164.83774445,347.08196777)
\curveto(164.98774095,347.11195993)(165.1427408,347.14695989)(165.30274445,347.18696777)
\curveto(165.38274056,347.19695984)(165.45774048,347.20195984)(165.52774445,347.20196777)
\curveto(165.59774034,347.20195984)(165.67274027,347.20695983)(165.75274445,347.21696777)
\curveto(166.26273968,347.22695981)(166.69773924,347.16695987)(167.05774445,347.03696777)
\curveto(167.42773851,346.91696012)(167.75773818,346.75696028)(168.04774445,346.55696777)
\curveto(168.1377378,346.49696054)(168.22773771,346.42696061)(168.31774445,346.34696777)
\curveto(168.40773753,346.27696076)(168.48773745,346.20196084)(168.55774445,346.12196777)
\curveto(168.58773735,346.07196097)(168.62773731,346.03196101)(168.67774445,346.00196777)
\curveto(168.75773718,345.89196115)(168.83273711,345.77696126)(168.90274445,345.65696777)
\curveto(168.97273697,345.54696149)(169.04773689,345.43196161)(169.12774445,345.31196777)
\curveto(169.17773676,345.22196182)(169.21773672,345.12696191)(169.24774445,345.02696777)
\curveto(169.28773665,344.9369621)(169.32773661,344.8369622)(169.36774445,344.72696777)
\curveto(169.41773652,344.59696244)(169.45773648,344.46196258)(169.48774445,344.32196777)
\curveto(169.51773642,344.18196286)(169.55273639,344.041963)(169.59274445,343.90196777)
\curveto(169.61273633,343.82196322)(169.61773632,343.73196331)(169.60774445,343.63196777)
\curveto(169.60773633,343.5419635)(169.61773632,343.45696358)(169.63774445,343.37696777)
\lineto(169.63774445,343.21196777)
\moveto(167.38774445,344.09696777)
\curveto(167.45773848,344.19696284)(167.46273848,344.31696272)(167.40274445,344.45696777)
\curveto(167.35273859,344.60696243)(167.31273863,344.71696232)(167.28274445,344.78696777)
\curveto(167.1427388,345.05696198)(166.95773898,345.26196178)(166.72774445,345.40196777)
\curveto(166.49773944,345.55196149)(166.17773976,345.63196141)(165.76774445,345.64196777)
\curveto(165.7377402,345.62196142)(165.70274024,345.61696142)(165.66274445,345.62696777)
\curveto(165.62274032,345.6369614)(165.58774035,345.6369614)(165.55774445,345.62696777)
\curveto(165.50774043,345.60696143)(165.45274049,345.59196145)(165.39274445,345.58196777)
\curveto(165.33274061,345.58196146)(165.27774066,345.57196147)(165.22774445,345.55196777)
\curveto(164.78774115,345.41196163)(164.46274148,345.1369619)(164.25274445,344.72696777)
\curveto(164.23274171,344.68696235)(164.20774173,344.63196241)(164.17774445,344.56196777)
\curveto(164.15774178,344.50196254)(164.1427418,344.4369626)(164.13274445,344.36696777)
\curveto(164.12274182,344.30696273)(164.12274182,344.24696279)(164.13274445,344.18696777)
\curveto(164.15274179,344.12696291)(164.18774175,344.07696296)(164.23774445,344.03696777)
\curveto(164.31774162,343.98696305)(164.42774151,343.96196308)(164.56774445,343.96196777)
\lineto(164.97274445,343.96196777)
\lineto(166.63774445,343.96196777)
\lineto(167.07274445,343.96196777)
\curveto(167.23273871,343.97196307)(167.3377386,344.01696302)(167.38774445,344.09696777)
}
}
{
\newrgbcolor{curcolor}{0 0 0}
\pscustom[linestyle=none,fillstyle=solid,fillcolor=curcolor]
{
\newpath
\moveto(174.4560257,347.21696777)
\curveto(175.26602054,347.2369598)(175.94101986,347.11695992)(176.4810257,346.85696777)
\curveto(177.03101877,346.59696044)(177.46601834,346.22696081)(177.7860257,345.74696777)
\curveto(177.94601786,345.50696153)(178.06601774,345.23196181)(178.1460257,344.92196777)
\curveto(178.16601764,344.87196217)(178.18101762,344.80696223)(178.1910257,344.72696777)
\curveto(178.21101759,344.64696239)(178.21101759,344.57696246)(178.1910257,344.51696777)
\curveto(178.15101765,344.40696263)(178.08101772,344.3419627)(177.9810257,344.32196777)
\curveto(177.88101792,344.31196273)(177.76101804,344.30696273)(177.6210257,344.30696777)
\lineto(176.8410257,344.30696777)
\lineto(176.5560257,344.30696777)
\curveto(176.46601934,344.30696273)(176.39101941,344.32696271)(176.3310257,344.36696777)
\curveto(176.25101955,344.40696263)(176.19601961,344.46696257)(176.1660257,344.54696777)
\curveto(176.13601967,344.6369624)(176.09601971,344.72696231)(176.0460257,344.81696777)
\curveto(175.98601982,344.92696211)(175.92101988,345.02696201)(175.8510257,345.11696777)
\curveto(175.78102002,345.20696183)(175.7010201,345.28696175)(175.6110257,345.35696777)
\curveto(175.47102033,345.44696159)(175.31602049,345.51696152)(175.1460257,345.56696777)
\curveto(175.08602072,345.58696145)(175.02602078,345.59696144)(174.9660257,345.59696777)
\curveto(174.9060209,345.59696144)(174.85102095,345.60696143)(174.8010257,345.62696777)
\lineto(174.6510257,345.62696777)
\curveto(174.45102135,345.62696141)(174.29102151,345.60696143)(174.1710257,345.56696777)
\curveto(173.88102192,345.47696156)(173.64602216,345.3369617)(173.4660257,345.14696777)
\curveto(173.28602252,344.96696207)(173.14102266,344.74696229)(173.0310257,344.48696777)
\curveto(172.98102282,344.37696266)(172.94102286,344.25696278)(172.9110257,344.12696777)
\curveto(172.89102291,344.00696303)(172.86602294,343.87696316)(172.8360257,343.73696777)
\curveto(172.82602298,343.69696334)(172.82102298,343.65696338)(172.8210257,343.61696777)
\curveto(172.82102298,343.57696346)(172.81602299,343.5369635)(172.8060257,343.49696777)
\curveto(172.78602302,343.39696364)(172.77602303,343.25696378)(172.7760257,343.07696777)
\curveto(172.78602302,342.89696414)(172.801023,342.75696428)(172.8210257,342.65696777)
\curveto(172.82102298,342.57696446)(172.82602298,342.52196452)(172.8360257,342.49196777)
\curveto(172.85602295,342.42196462)(172.86602294,342.35196469)(172.8660257,342.28196777)
\curveto(172.87602293,342.21196483)(172.89102291,342.1419649)(172.9110257,342.07196777)
\curveto(172.99102281,341.8419652)(173.08602272,341.63196541)(173.1960257,341.44196777)
\curveto(173.3060225,341.25196579)(173.44602236,341.09196595)(173.6160257,340.96196777)
\curveto(173.65602215,340.93196611)(173.71602209,340.89696614)(173.7960257,340.85696777)
\curveto(173.9060219,340.78696625)(174.01602179,340.7419663)(174.1260257,340.72196777)
\curveto(174.24602156,340.70196634)(174.39102141,340.68196636)(174.5610257,340.66196777)
\lineto(174.6510257,340.66196777)
\curveto(174.69102111,340.66196638)(174.72102108,340.66696637)(174.7410257,340.67696777)
\lineto(174.8760257,340.67696777)
\curveto(174.94602086,340.69696634)(175.01102079,340.71196633)(175.0710257,340.72196777)
\curveto(175.14102066,340.7419663)(175.2060206,340.76196628)(175.2660257,340.78196777)
\curveto(175.56602024,340.91196613)(175.79602001,341.10196594)(175.9560257,341.35196777)
\curveto(175.99601981,341.40196564)(176.03101977,341.45696558)(176.0610257,341.51696777)
\curveto(176.09101971,341.58696545)(176.11601969,341.64696539)(176.1360257,341.69696777)
\curveto(176.17601963,341.80696523)(176.21101959,341.90196514)(176.2410257,341.98196777)
\curveto(176.27101953,342.07196497)(176.34101946,342.1419649)(176.4510257,342.19196777)
\curveto(176.54101926,342.23196481)(176.68601912,342.24696479)(176.8860257,342.23696777)
\lineto(177.3810257,342.23696777)
\lineto(177.5910257,342.23696777)
\curveto(177.67101813,342.24696479)(177.73601807,342.2419648)(177.7860257,342.22196777)
\lineto(177.9060257,342.22196777)
\lineto(178.0260257,342.19196777)
\curveto(178.06601774,342.19196485)(178.09601771,342.18196486)(178.1160257,342.16196777)
\curveto(178.16601764,342.12196492)(178.19601761,342.06196498)(178.2060257,341.98196777)
\curveto(178.22601758,341.91196513)(178.22601758,341.8369652)(178.2060257,341.75696777)
\curveto(178.11601769,341.42696561)(178.0060178,341.13196591)(177.8760257,340.87196777)
\curveto(177.46601834,340.10196694)(176.81101899,339.56696747)(175.9110257,339.26696777)
\curveto(175.81101999,339.2369678)(175.7060201,339.21696782)(175.5960257,339.20696777)
\curveto(175.48602032,339.18696785)(175.37602043,339.16196788)(175.2660257,339.13196777)
\curveto(175.2060206,339.12196792)(175.14602066,339.11696792)(175.0860257,339.11696777)
\curveto(175.02602078,339.11696792)(174.96602084,339.11196793)(174.9060257,339.10196777)
\lineto(174.7410257,339.10196777)
\curveto(174.69102111,339.08196796)(174.61602119,339.07696796)(174.5160257,339.08696777)
\curveto(174.41602139,339.08696795)(174.34102146,339.09196795)(174.2910257,339.10196777)
\curveto(174.21102159,339.12196792)(174.13602167,339.13196791)(174.0660257,339.13196777)
\curveto(174.0060218,339.12196792)(173.94102186,339.12696791)(173.8710257,339.14696777)
\lineto(173.7210257,339.17696777)
\curveto(173.67102213,339.17696786)(173.62102218,339.18196786)(173.5710257,339.19196777)
\curveto(173.46102234,339.22196782)(173.35602245,339.25196779)(173.2560257,339.28196777)
\curveto(173.15602265,339.31196773)(173.06102274,339.34696769)(172.9710257,339.38696777)
\curveto(172.5010233,339.58696745)(172.1060237,339.8419672)(171.7860257,340.15196777)
\curveto(171.46602434,340.47196657)(171.2060246,340.86696617)(171.0060257,341.33696777)
\curveto(170.95602485,341.42696561)(170.91602489,341.52196552)(170.8860257,341.62196777)
\lineto(170.7960257,341.95196777)
\curveto(170.78602502,341.99196505)(170.78102502,342.02696501)(170.7810257,342.05696777)
\curveto(170.78102502,342.09696494)(170.77102503,342.1419649)(170.7510257,342.19196777)
\curveto(170.73102507,342.26196478)(170.72102508,342.33196471)(170.7210257,342.40196777)
\curveto(170.72102508,342.48196456)(170.71102509,342.55696448)(170.6910257,342.62696777)
\lineto(170.6910257,342.88196777)
\curveto(170.67102513,342.93196411)(170.66102514,342.98696405)(170.6610257,343.04696777)
\curveto(170.66102514,343.11696392)(170.67102513,343.17696386)(170.6910257,343.22696777)
\curveto(170.7010251,343.27696376)(170.7010251,343.32196372)(170.6910257,343.36196777)
\curveto(170.68102512,343.40196364)(170.68102512,343.4419636)(170.6910257,343.48196777)
\curveto(170.71102509,343.55196349)(170.71602509,343.61696342)(170.7060257,343.67696777)
\curveto(170.7060251,343.7369633)(170.71602509,343.79696324)(170.7360257,343.85696777)
\curveto(170.78602502,344.036963)(170.82602498,344.20696283)(170.8560257,344.36696777)
\curveto(170.88602492,344.5369625)(170.93102487,344.70196234)(170.9910257,344.86196777)
\curveto(171.21102459,345.37196167)(171.48602432,345.79696124)(171.8160257,346.13696777)
\curveto(172.15602365,346.47696056)(172.58602322,346.75196029)(173.1060257,346.96196777)
\curveto(173.24602256,347.02196002)(173.39102241,347.06195998)(173.5410257,347.08196777)
\curveto(173.69102211,347.11195993)(173.84602196,347.14695989)(174.0060257,347.18696777)
\curveto(174.08602172,347.19695984)(174.16102164,347.20195984)(174.2310257,347.20196777)
\curveto(174.3010215,347.20195984)(174.37602143,347.20695983)(174.4560257,347.21696777)
}
}
{
\newrgbcolor{curcolor}{0 0 0}
\pscustom[linestyle=none,fillstyle=solid,fillcolor=curcolor]
{
\newpath
\moveto(181.59930695,349.85696777)
\curveto(181.669304,349.77695726)(181.70430396,349.65695738)(181.70430695,349.49696777)
\lineto(181.70430695,349.03196777)
\lineto(181.70430695,348.62696777)
\curveto(181.70430396,348.48695855)(181.669304,348.39195865)(181.59930695,348.34196777)
\curveto(181.53930413,348.29195875)(181.45930421,348.26195878)(181.35930695,348.25196777)
\curveto(181.2693044,348.2419588)(181.1693045,348.2369588)(181.05930695,348.23696777)
\lineto(180.21930695,348.23696777)
\curveto(180.10930556,348.2369588)(180.00930566,348.2419588)(179.91930695,348.25196777)
\curveto(179.83930583,348.26195878)(179.7693059,348.29195875)(179.70930695,348.34196777)
\curveto(179.669306,348.37195867)(179.63930603,348.42695861)(179.61930695,348.50696777)
\curveto(179.60930606,348.59695844)(179.59930607,348.69195835)(179.58930695,348.79196777)
\lineto(179.58930695,349.12196777)
\curveto(179.59930607,349.23195781)(179.60430606,349.32695771)(179.60430695,349.40696777)
\lineto(179.60430695,349.61696777)
\curveto(179.61430605,349.68695735)(179.63430603,349.74695729)(179.66430695,349.79696777)
\curveto(179.68430598,349.8369572)(179.70930596,349.86695717)(179.73930695,349.88696777)
\lineto(179.85930695,349.94696777)
\curveto(179.87930579,349.94695709)(179.90430576,349.94695709)(179.93430695,349.94696777)
\curveto(179.9643057,349.95695708)(179.98930568,349.96195708)(180.00930695,349.96196777)
\lineto(181.10430695,349.96196777)
\curveto(181.20430446,349.96195708)(181.29930437,349.95695708)(181.38930695,349.94696777)
\curveto(181.47930419,349.9369571)(181.54930412,349.90695713)(181.59930695,349.85696777)
\moveto(181.70430695,340.09196777)
\curveto(181.70430396,339.89196715)(181.69930397,339.72196732)(181.68930695,339.58196777)
\curveto(181.67930399,339.4419676)(181.58930408,339.34696769)(181.41930695,339.29696777)
\curveto(181.35930431,339.27696776)(181.29430437,339.26696777)(181.22430695,339.26696777)
\curveto(181.15430451,339.27696776)(181.07930459,339.28196776)(180.99930695,339.28196777)
\lineto(180.15930695,339.28196777)
\curveto(180.0693056,339.28196776)(179.97930569,339.28696775)(179.88930695,339.29696777)
\curveto(179.80930586,339.30696773)(179.74930592,339.3369677)(179.70930695,339.38696777)
\curveto(179.64930602,339.45696758)(179.61430605,339.5419675)(179.60430695,339.64196777)
\lineto(179.60430695,339.98696777)
\lineto(179.60430695,346.31696777)
\lineto(179.60430695,346.61696777)
\curveto(179.60430606,346.71696032)(179.62430604,346.79696024)(179.66430695,346.85696777)
\curveto(179.72430594,346.92696011)(179.80930586,346.97196007)(179.91930695,346.99196777)
\curveto(179.93930573,347.00196004)(179.9643057,347.00196004)(179.99430695,346.99196777)
\curveto(180.03430563,346.99196005)(180.0643056,346.99696004)(180.08430695,347.00696777)
\lineto(180.83430695,347.00696777)
\lineto(181.02930695,347.00696777)
\curveto(181.10930456,347.01696002)(181.17430449,347.01696002)(181.22430695,347.00696777)
\lineto(181.34430695,347.00696777)
\curveto(181.40430426,346.98696005)(181.45930421,346.97196007)(181.50930695,346.96196777)
\curveto(181.55930411,346.95196009)(181.59930407,346.92196012)(181.62930695,346.87196777)
\curveto(181.669304,346.82196022)(181.68930398,346.75196029)(181.68930695,346.66196777)
\curveto(181.69930397,346.57196047)(181.70430396,346.47696056)(181.70430695,346.37696777)
\lineto(181.70430695,340.09196777)
}
}
{
\newrgbcolor{curcolor}{0 0 0}
\pscustom[linestyle=none,fillstyle=solid,fillcolor=curcolor]
{
\newpath
\moveto(190.73149445,343.21196777)
\curveto(190.75148628,343.13196391)(190.75148628,343.041964)(190.73149445,342.94196777)
\curveto(190.71148632,342.8419642)(190.67648636,342.77696426)(190.62649445,342.74696777)
\curveto(190.57648646,342.70696433)(190.50148653,342.67696436)(190.40149445,342.65696777)
\curveto(190.31148672,342.64696439)(190.20648683,342.6369644)(190.08649445,342.62696777)
\lineto(189.74149445,342.62696777)
\curveto(189.6314874,342.6369644)(189.5314875,342.6419644)(189.44149445,342.64196777)
\lineto(185.78149445,342.64196777)
\lineto(185.57149445,342.64196777)
\curveto(185.51149152,342.6419644)(185.45649158,342.63196441)(185.40649445,342.61196777)
\curveto(185.32649171,342.57196447)(185.27649176,342.53196451)(185.25649445,342.49196777)
\curveto(185.2364918,342.47196457)(185.21649182,342.43196461)(185.19649445,342.37196777)
\curveto(185.17649186,342.32196472)(185.17149186,342.27196477)(185.18149445,342.22196777)
\curveto(185.20149183,342.16196488)(185.21149182,342.10196494)(185.21149445,342.04196777)
\curveto(185.22149181,341.99196505)(185.2364918,341.9369651)(185.25649445,341.87696777)
\curveto(185.3364917,341.6369654)(185.4314916,341.4369656)(185.54149445,341.27696777)
\curveto(185.66149137,341.12696591)(185.82149121,340.99196605)(186.02149445,340.87196777)
\curveto(186.10149093,340.82196622)(186.18149085,340.78696625)(186.26149445,340.76696777)
\curveto(186.35149068,340.75696628)(186.44149059,340.7369663)(186.53149445,340.70696777)
\curveto(186.61149042,340.68696635)(186.72149031,340.67196637)(186.86149445,340.66196777)
\curveto(187.00149003,340.65196639)(187.12148991,340.65696638)(187.22149445,340.67696777)
\lineto(187.35649445,340.67696777)
\curveto(187.45648958,340.69696634)(187.54648949,340.71696632)(187.62649445,340.73696777)
\curveto(187.71648932,340.76696627)(187.80148923,340.79696624)(187.88149445,340.82696777)
\curveto(187.98148905,340.87696616)(188.09148894,340.9419661)(188.21149445,341.02196777)
\curveto(188.34148869,341.10196594)(188.4364886,341.18196586)(188.49649445,341.26196777)
\curveto(188.54648849,341.33196571)(188.59648844,341.39696564)(188.64649445,341.45696777)
\curveto(188.70648833,341.52696551)(188.77648826,341.57696546)(188.85649445,341.60696777)
\curveto(188.95648808,341.65696538)(189.08148795,341.67696536)(189.23149445,341.66696777)
\lineto(189.66649445,341.66696777)
\lineto(189.84649445,341.66696777)
\curveto(189.91648712,341.67696536)(189.97648706,341.67196537)(190.02649445,341.65196777)
\lineto(190.17649445,341.65196777)
\curveto(190.27648676,341.63196541)(190.34648669,341.60696543)(190.38649445,341.57696777)
\curveto(190.42648661,341.55696548)(190.44648659,341.51196553)(190.44649445,341.44196777)
\curveto(190.45648658,341.37196567)(190.45148658,341.31196573)(190.43149445,341.26196777)
\curveto(190.38148665,341.12196592)(190.32648671,340.99696604)(190.26649445,340.88696777)
\curveto(190.20648683,340.77696626)(190.1364869,340.66696637)(190.05649445,340.55696777)
\curveto(189.8364872,340.22696681)(189.58648745,339.96196708)(189.30649445,339.76196777)
\curveto(189.02648801,339.56196748)(188.67648836,339.39196765)(188.25649445,339.25196777)
\curveto(188.14648889,339.21196783)(188.036489,339.18696785)(187.92649445,339.17696777)
\curveto(187.81648922,339.16696787)(187.70148933,339.14696789)(187.58149445,339.11696777)
\curveto(187.54148949,339.10696793)(187.49648954,339.10696793)(187.44649445,339.11696777)
\curveto(187.40648963,339.11696792)(187.36648967,339.11196793)(187.32649445,339.10196777)
\lineto(187.16149445,339.10196777)
\curveto(187.11148992,339.08196796)(187.05148998,339.07696796)(186.98149445,339.08696777)
\curveto(186.92149011,339.08696795)(186.86649017,339.09196795)(186.81649445,339.10196777)
\curveto(186.7364903,339.11196793)(186.66649037,339.11196793)(186.60649445,339.10196777)
\curveto(186.54649049,339.09196795)(186.48149055,339.09696794)(186.41149445,339.11696777)
\curveto(186.36149067,339.1369679)(186.30649073,339.14696789)(186.24649445,339.14696777)
\curveto(186.18649085,339.14696789)(186.1314909,339.15696788)(186.08149445,339.17696777)
\curveto(185.97149106,339.19696784)(185.86149117,339.22196782)(185.75149445,339.25196777)
\curveto(185.64149139,339.27196777)(185.54149149,339.30696773)(185.45149445,339.35696777)
\curveto(185.34149169,339.39696764)(185.2364918,339.43196761)(185.13649445,339.46196777)
\curveto(185.04649199,339.50196754)(184.96149207,339.54696749)(184.88149445,339.59696777)
\curveto(184.56149247,339.79696724)(184.27649276,340.02696701)(184.02649445,340.28696777)
\curveto(183.77649326,340.55696648)(183.57149346,340.86696617)(183.41149445,341.21696777)
\curveto(183.36149367,341.32696571)(183.32149371,341.4369656)(183.29149445,341.54696777)
\curveto(183.26149377,341.66696537)(183.22149381,341.78696525)(183.17149445,341.90696777)
\curveto(183.16149387,341.94696509)(183.15649388,341.98196506)(183.15649445,342.01196777)
\curveto(183.15649388,342.05196499)(183.15149388,342.09196495)(183.14149445,342.13196777)
\curveto(183.10149393,342.25196479)(183.07649396,342.38196466)(183.06649445,342.52196777)
\lineto(183.03649445,342.94196777)
\curveto(183.036494,342.99196405)(183.031494,343.04696399)(183.02149445,343.10696777)
\curveto(183.02149401,343.16696387)(183.02649401,343.22196382)(183.03649445,343.27196777)
\lineto(183.03649445,343.45196777)
\lineto(183.08149445,343.81196777)
\curveto(183.12149391,343.98196306)(183.15649388,344.14696289)(183.18649445,344.30696777)
\curveto(183.21649382,344.46696257)(183.26149377,344.61696242)(183.32149445,344.75696777)
\curveto(183.75149328,345.79696124)(184.48149255,346.53196051)(185.51149445,346.96196777)
\curveto(185.65149138,347.02196002)(185.79149124,347.06195998)(185.93149445,347.08196777)
\curveto(186.08149095,347.11195993)(186.2364908,347.14695989)(186.39649445,347.18696777)
\curveto(186.47649056,347.19695984)(186.55149048,347.20195984)(186.62149445,347.20196777)
\curveto(186.69149034,347.20195984)(186.76649027,347.20695983)(186.84649445,347.21696777)
\curveto(187.35648968,347.22695981)(187.79148924,347.16695987)(188.15149445,347.03696777)
\curveto(188.52148851,346.91696012)(188.85148818,346.75696028)(189.14149445,346.55696777)
\curveto(189.2314878,346.49696054)(189.32148771,346.42696061)(189.41149445,346.34696777)
\curveto(189.50148753,346.27696076)(189.58148745,346.20196084)(189.65149445,346.12196777)
\curveto(189.68148735,346.07196097)(189.72148731,346.03196101)(189.77149445,346.00196777)
\curveto(189.85148718,345.89196115)(189.92648711,345.77696126)(189.99649445,345.65696777)
\curveto(190.06648697,345.54696149)(190.14148689,345.43196161)(190.22149445,345.31196777)
\curveto(190.27148676,345.22196182)(190.31148672,345.12696191)(190.34149445,345.02696777)
\curveto(190.38148665,344.9369621)(190.42148661,344.8369622)(190.46149445,344.72696777)
\curveto(190.51148652,344.59696244)(190.55148648,344.46196258)(190.58149445,344.32196777)
\curveto(190.61148642,344.18196286)(190.64648639,344.041963)(190.68649445,343.90196777)
\curveto(190.70648633,343.82196322)(190.71148632,343.73196331)(190.70149445,343.63196777)
\curveto(190.70148633,343.5419635)(190.71148632,343.45696358)(190.73149445,343.37696777)
\lineto(190.73149445,343.21196777)
\moveto(188.48149445,344.09696777)
\curveto(188.55148848,344.19696284)(188.55648848,344.31696272)(188.49649445,344.45696777)
\curveto(188.44648859,344.60696243)(188.40648863,344.71696232)(188.37649445,344.78696777)
\curveto(188.2364888,345.05696198)(188.05148898,345.26196178)(187.82149445,345.40196777)
\curveto(187.59148944,345.55196149)(187.27148976,345.63196141)(186.86149445,345.64196777)
\curveto(186.8314902,345.62196142)(186.79649024,345.61696142)(186.75649445,345.62696777)
\curveto(186.71649032,345.6369614)(186.68149035,345.6369614)(186.65149445,345.62696777)
\curveto(186.60149043,345.60696143)(186.54649049,345.59196145)(186.48649445,345.58196777)
\curveto(186.42649061,345.58196146)(186.37149066,345.57196147)(186.32149445,345.55196777)
\curveto(185.88149115,345.41196163)(185.55649148,345.1369619)(185.34649445,344.72696777)
\curveto(185.32649171,344.68696235)(185.30149173,344.63196241)(185.27149445,344.56196777)
\curveto(185.25149178,344.50196254)(185.2364918,344.4369626)(185.22649445,344.36696777)
\curveto(185.21649182,344.30696273)(185.21649182,344.24696279)(185.22649445,344.18696777)
\curveto(185.24649179,344.12696291)(185.28149175,344.07696296)(185.33149445,344.03696777)
\curveto(185.41149162,343.98696305)(185.52149151,343.96196308)(185.66149445,343.96196777)
\lineto(186.06649445,343.96196777)
\lineto(187.73149445,343.96196777)
\lineto(188.16649445,343.96196777)
\curveto(188.32648871,343.97196307)(188.4314886,344.01696302)(188.48149445,344.09696777)
}
}
{
\newrgbcolor{curcolor}{0 0 0}
\pscustom[linestyle=none,fillstyle=solid,fillcolor=curcolor]
{
\newpath
\moveto(196.4047757,347.20196777)
\curveto(196.51477038,347.20195984)(196.60977029,347.19195985)(196.6897757,347.17196777)
\curveto(196.77977012,347.15195989)(196.84977005,347.10695993)(196.8997757,347.03696777)
\curveto(196.95976994,346.95696008)(196.98976991,346.81696022)(196.9897757,346.61696777)
\lineto(196.9897757,346.10696777)
\lineto(196.9897757,345.73196777)
\curveto(196.9997699,345.59196145)(196.98476991,345.48196156)(196.9447757,345.40196777)
\curveto(196.90476999,345.33196171)(196.84477005,345.28696175)(196.7647757,345.26696777)
\curveto(196.6947702,345.24696179)(196.60977029,345.2369618)(196.5097757,345.23696777)
\curveto(196.41977048,345.2369618)(196.31977058,345.2419618)(196.2097757,345.25196777)
\curveto(196.10977079,345.26196178)(196.01477088,345.25696178)(195.9247757,345.23696777)
\curveto(195.85477104,345.21696182)(195.78477111,345.20196184)(195.7147757,345.19196777)
\curveto(195.64477125,345.19196185)(195.57977132,345.18196186)(195.5197757,345.16196777)
\curveto(195.35977154,345.11196193)(195.1997717,345.036962)(195.0397757,344.93696777)
\curveto(194.87977202,344.84696219)(194.75477214,344.7419623)(194.6647757,344.62196777)
\curveto(194.61477228,344.5419625)(194.55977234,344.45696258)(194.4997757,344.36696777)
\curveto(194.44977245,344.28696275)(194.3997725,344.20196284)(194.3497757,344.11196777)
\curveto(194.31977258,344.03196301)(194.28977261,343.94696309)(194.2597757,343.85696777)
\lineto(194.1997757,343.61696777)
\curveto(194.17977272,343.54696349)(194.16977273,343.47196357)(194.1697757,343.39196777)
\curveto(194.16977273,343.32196372)(194.15977274,343.25196379)(194.1397757,343.18196777)
\curveto(194.12977277,343.1419639)(194.12477277,343.10196394)(194.1247757,343.06196777)
\curveto(194.13477276,343.03196401)(194.13477276,343.00196404)(194.1247757,342.97196777)
\lineto(194.1247757,342.73196777)
\curveto(194.10477279,342.66196438)(194.0997728,342.58196446)(194.1097757,342.49196777)
\curveto(194.11977278,342.41196463)(194.12477277,342.33196471)(194.1247757,342.25196777)
\lineto(194.1247757,341.29196777)
\lineto(194.1247757,340.01696777)
\curveto(194.12477277,339.88696715)(194.11977278,339.76696727)(194.1097757,339.65696777)
\curveto(194.0997728,339.54696749)(194.06977283,339.45696758)(194.0197757,339.38696777)
\curveto(193.9997729,339.35696768)(193.96477293,339.33196771)(193.9147757,339.31196777)
\curveto(193.87477302,339.30196774)(193.82977307,339.29196775)(193.7797757,339.28196777)
\lineto(193.7047757,339.28196777)
\curveto(193.65477324,339.27196777)(193.5997733,339.26696777)(193.5397757,339.26696777)
\lineto(193.3747757,339.26696777)
\lineto(192.7297757,339.26696777)
\curveto(192.66977423,339.27696776)(192.60477429,339.28196776)(192.5347757,339.28196777)
\lineto(192.3397757,339.28196777)
\curveto(192.28977461,339.30196774)(192.23977466,339.31696772)(192.1897757,339.32696777)
\curveto(192.13977476,339.34696769)(192.10477479,339.38196766)(192.0847757,339.43196777)
\curveto(192.04477485,339.48196756)(192.01977488,339.55196749)(192.0097757,339.64196777)
\lineto(192.0097757,339.94196777)
\lineto(192.0097757,340.96196777)
\lineto(192.0097757,345.19196777)
\lineto(192.0097757,346.30196777)
\lineto(192.0097757,346.58696777)
\curveto(192.00977489,346.68696035)(192.02977487,346.76696027)(192.0697757,346.82696777)
\curveto(192.11977478,346.90696013)(192.1947747,346.95696008)(192.2947757,346.97696777)
\curveto(192.3947745,346.99696004)(192.51477438,347.00696003)(192.6547757,347.00696777)
\lineto(193.4197757,347.00696777)
\curveto(193.53977336,347.00696003)(193.64477325,346.99696004)(193.7347757,346.97696777)
\curveto(193.82477307,346.96696007)(193.894773,346.92196012)(193.9447757,346.84196777)
\curveto(193.97477292,346.79196025)(193.98977291,346.72196032)(193.9897757,346.63196777)
\lineto(194.0197757,346.36196777)
\curveto(194.02977287,346.28196076)(194.04477285,346.20696083)(194.0647757,346.13696777)
\curveto(194.0947728,346.06696097)(194.14477275,346.03196101)(194.2147757,346.03196777)
\curveto(194.23477266,346.05196099)(194.25477264,346.06196098)(194.2747757,346.06196777)
\curveto(194.2947726,346.06196098)(194.31477258,346.07196097)(194.3347757,346.09196777)
\curveto(194.3947725,346.1419609)(194.44477245,346.19696084)(194.4847757,346.25696777)
\curveto(194.53477236,346.32696071)(194.5947723,346.38696065)(194.6647757,346.43696777)
\curveto(194.70477219,346.46696057)(194.73977216,346.49696054)(194.7697757,346.52696777)
\curveto(194.7997721,346.56696047)(194.83477206,346.60196044)(194.8747757,346.63196777)
\lineto(195.1447757,346.81196777)
\curveto(195.24477165,346.87196017)(195.34477155,346.92696011)(195.4447757,346.97696777)
\curveto(195.54477135,347.01696002)(195.64477125,347.05195999)(195.7447757,347.08196777)
\lineto(196.0747757,347.17196777)
\curveto(196.10477079,347.18195986)(196.15977074,347.18195986)(196.2397757,347.17196777)
\curveto(196.32977057,347.17195987)(196.38477051,347.18195986)(196.4047757,347.20196777)
}
}
{
\newrgbcolor{curcolor}{0 0 0}
\pscustom[linestyle=none,fillstyle=solid,fillcolor=curcolor]
{
\newpath
\moveto(205.31618195,343.45196777)
\curveto(205.33617338,343.39196365)(205.34617337,343.30696373)(205.34618195,343.19696777)
\curveto(205.34617337,343.08696395)(205.33617338,343.00196404)(205.31618195,342.94196777)
\lineto(205.31618195,342.79196777)
\curveto(205.29617342,342.71196433)(205.28617343,342.63196441)(205.28618195,342.55196777)
\curveto(205.29617342,342.47196457)(205.29117342,342.39196465)(205.27118195,342.31196777)
\curveto(205.25117346,342.2419648)(205.23617348,342.17696486)(205.22618195,342.11696777)
\curveto(205.2161735,342.05696498)(205.20617351,341.99196505)(205.19618195,341.92196777)
\curveto(205.15617356,341.81196523)(205.12117359,341.69696534)(205.09118195,341.57696777)
\curveto(205.06117365,341.46696557)(205.02117369,341.36196568)(204.97118195,341.26196777)
\curveto(204.76117395,340.78196626)(204.48617423,340.39196665)(204.14618195,340.09196777)
\curveto(203.80617491,339.79196725)(203.39617532,339.5419675)(202.91618195,339.34196777)
\curveto(202.79617592,339.29196775)(202.67117604,339.25696778)(202.54118195,339.23696777)
\curveto(202.42117629,339.20696783)(202.29617642,339.17696786)(202.16618195,339.14696777)
\curveto(202.1161766,339.12696791)(202.06117665,339.11696792)(202.00118195,339.11696777)
\curveto(201.94117677,339.11696792)(201.88617683,339.11196793)(201.83618195,339.10196777)
\lineto(201.73118195,339.10196777)
\curveto(201.70117701,339.09196795)(201.67117704,339.08696795)(201.64118195,339.08696777)
\curveto(201.59117712,339.07696796)(201.5111772,339.07196797)(201.40118195,339.07196777)
\curveto(201.29117742,339.06196798)(201.20617751,339.06696797)(201.14618195,339.08696777)
\lineto(200.99618195,339.08696777)
\curveto(200.94617777,339.09696794)(200.89117782,339.10196794)(200.83118195,339.10196777)
\curveto(200.78117793,339.09196795)(200.73117798,339.09696794)(200.68118195,339.11696777)
\curveto(200.64117807,339.12696791)(200.60117811,339.13196791)(200.56118195,339.13196777)
\curveto(200.53117818,339.13196791)(200.49117822,339.1369679)(200.44118195,339.14696777)
\curveto(200.34117837,339.17696786)(200.24117847,339.20196784)(200.14118195,339.22196777)
\curveto(200.04117867,339.2419678)(199.94617877,339.27196777)(199.85618195,339.31196777)
\curveto(199.73617898,339.35196769)(199.62117909,339.39196765)(199.51118195,339.43196777)
\curveto(199.4111793,339.47196757)(199.30617941,339.52196752)(199.19618195,339.58196777)
\curveto(198.84617987,339.79196725)(198.54618017,340.036967)(198.29618195,340.31696777)
\curveto(198.04618067,340.59696644)(197.83618088,340.93196611)(197.66618195,341.32196777)
\curveto(197.6161811,341.41196563)(197.57618114,341.50696553)(197.54618195,341.60696777)
\curveto(197.52618119,341.70696533)(197.50118121,341.81196523)(197.47118195,341.92196777)
\curveto(197.45118126,341.97196507)(197.44118127,342.01696502)(197.44118195,342.05696777)
\curveto(197.44118127,342.09696494)(197.43118128,342.1419649)(197.41118195,342.19196777)
\curveto(197.39118132,342.27196477)(197.38118133,342.35196469)(197.38118195,342.43196777)
\curveto(197.38118133,342.52196452)(197.37118134,342.60696443)(197.35118195,342.68696777)
\curveto(197.34118137,342.7369643)(197.33618138,342.78196426)(197.33618195,342.82196777)
\lineto(197.33618195,342.95696777)
\curveto(197.3161814,343.01696402)(197.30618141,343.10196394)(197.30618195,343.21196777)
\curveto(197.3161814,343.32196372)(197.33118138,343.40696363)(197.35118195,343.46696777)
\lineto(197.35118195,343.57196777)
\curveto(197.36118135,343.62196342)(197.36118135,343.67196337)(197.35118195,343.72196777)
\curveto(197.35118136,343.78196326)(197.36118135,343.8369632)(197.38118195,343.88696777)
\curveto(197.39118132,343.9369631)(197.39618132,343.98196306)(197.39618195,344.02196777)
\curveto(197.39618132,344.07196297)(197.40618131,344.12196292)(197.42618195,344.17196777)
\curveto(197.46618125,344.30196274)(197.50118121,344.42696261)(197.53118195,344.54696777)
\curveto(197.56118115,344.67696236)(197.60118111,344.80196224)(197.65118195,344.92196777)
\curveto(197.83118088,345.33196171)(198.04618067,345.67196137)(198.29618195,345.94196777)
\curveto(198.54618017,346.22196082)(198.85117986,346.47696056)(199.21118195,346.70696777)
\curveto(199.3111794,346.75696028)(199.4161793,346.80196024)(199.52618195,346.84196777)
\curveto(199.63617908,346.88196016)(199.74617897,346.92696011)(199.85618195,346.97696777)
\curveto(199.98617873,347.02696001)(200.12117859,347.06195998)(200.26118195,347.08196777)
\curveto(200.40117831,347.10195994)(200.54617817,347.13195991)(200.69618195,347.17196777)
\curveto(200.77617794,347.18195986)(200.85117786,347.18695985)(200.92118195,347.18696777)
\curveto(200.99117772,347.18695985)(201.06117765,347.19195985)(201.13118195,347.20196777)
\curveto(201.711177,347.21195983)(202.2111765,347.15195989)(202.63118195,347.02196777)
\curveto(203.06117565,346.89196015)(203.44117527,346.71196033)(203.77118195,346.48196777)
\curveto(203.88117483,346.40196064)(203.99117472,346.31196073)(204.10118195,346.21196777)
\curveto(204.22117449,346.12196092)(204.32117439,346.02196102)(204.40118195,345.91196777)
\curveto(204.48117423,345.81196123)(204.55117416,345.71196133)(204.61118195,345.61196777)
\curveto(204.68117403,345.51196153)(204.75117396,345.40696163)(204.82118195,345.29696777)
\curveto(204.89117382,345.18696185)(204.94617377,345.06696197)(204.98618195,344.93696777)
\curveto(205.02617369,344.81696222)(205.07117364,344.68696235)(205.12118195,344.54696777)
\curveto(205.15117356,344.46696257)(205.17617354,344.38196266)(205.19618195,344.29196777)
\lineto(205.25618195,344.02196777)
\curveto(205.26617345,343.98196306)(205.27117344,343.9419631)(205.27118195,343.90196777)
\curveto(205.27117344,343.86196318)(205.27617344,343.82196322)(205.28618195,343.78196777)
\curveto(205.30617341,343.73196331)(205.3111734,343.67696336)(205.30118195,343.61696777)
\curveto(205.29117342,343.55696348)(205.29617342,343.50196354)(205.31618195,343.45196777)
\moveto(203.21618195,342.91196777)
\curveto(203.22617549,342.96196408)(203.23117548,343.03196401)(203.23118195,343.12196777)
\curveto(203.23117548,343.22196382)(203.22617549,343.29696374)(203.21618195,343.34696777)
\lineto(203.21618195,343.46696777)
\curveto(203.19617552,343.51696352)(203.18617553,343.57196347)(203.18618195,343.63196777)
\curveto(203.18617553,343.69196335)(203.18117553,343.74696329)(203.17118195,343.79696777)
\curveto(203.17117554,343.8369632)(203.16617555,343.86696317)(203.15618195,343.88696777)
\lineto(203.09618195,344.12696777)
\curveto(203.08617563,344.21696282)(203.06617565,344.30196274)(203.03618195,344.38196777)
\curveto(202.92617579,344.6419624)(202.79617592,344.86196218)(202.64618195,345.04196777)
\curveto(202.49617622,345.23196181)(202.29617642,345.38196166)(202.04618195,345.49196777)
\curveto(201.98617673,345.51196153)(201.92617679,345.52696151)(201.86618195,345.53696777)
\curveto(201.80617691,345.55696148)(201.74117697,345.57696146)(201.67118195,345.59696777)
\curveto(201.59117712,345.61696142)(201.50617721,345.62196142)(201.41618195,345.61196777)
\lineto(201.14618195,345.61196777)
\curveto(201.1161776,345.59196145)(201.08117763,345.58196146)(201.04118195,345.58196777)
\curveto(201.00117771,345.59196145)(200.96617775,345.59196145)(200.93618195,345.58196777)
\lineto(200.72618195,345.52196777)
\curveto(200.66617805,345.51196153)(200.6111781,345.49196155)(200.56118195,345.46196777)
\curveto(200.3111784,345.35196169)(200.10617861,345.19196185)(199.94618195,344.98196777)
\curveto(199.79617892,344.78196226)(199.67617904,344.54696249)(199.58618195,344.27696777)
\curveto(199.55617916,344.17696286)(199.53117918,344.07196297)(199.51118195,343.96196777)
\curveto(199.50117921,343.85196319)(199.48617923,343.7419633)(199.46618195,343.63196777)
\curveto(199.45617926,343.58196346)(199.45117926,343.53196351)(199.45118195,343.48196777)
\lineto(199.45118195,343.33196777)
\curveto(199.43117928,343.26196378)(199.42117929,343.15696388)(199.42118195,343.01696777)
\curveto(199.43117928,342.87696416)(199.44617927,342.77196427)(199.46618195,342.70196777)
\lineto(199.46618195,342.56696777)
\curveto(199.48617923,342.48696455)(199.50117921,342.40696463)(199.51118195,342.32696777)
\curveto(199.52117919,342.25696478)(199.53617918,342.18196486)(199.55618195,342.10196777)
\curveto(199.65617906,341.80196524)(199.76117895,341.55696548)(199.87118195,341.36696777)
\curveto(199.99117872,341.18696585)(200.17617854,341.02196602)(200.42618195,340.87196777)
\curveto(200.49617822,340.82196622)(200.57117814,340.78196626)(200.65118195,340.75196777)
\curveto(200.74117797,340.72196632)(200.83117788,340.69696634)(200.92118195,340.67696777)
\curveto(200.96117775,340.66696637)(200.99617772,340.66196638)(201.02618195,340.66196777)
\curveto(201.05617766,340.67196637)(201.09117762,340.67196637)(201.13118195,340.66196777)
\lineto(201.25118195,340.63196777)
\curveto(201.30117741,340.63196641)(201.34617737,340.6369664)(201.38618195,340.64696777)
\lineto(201.50618195,340.64696777)
\curveto(201.58617713,340.66696637)(201.66617705,340.68196636)(201.74618195,340.69196777)
\curveto(201.82617689,340.70196634)(201.90117681,340.72196632)(201.97118195,340.75196777)
\curveto(202.23117648,340.85196619)(202.44117627,340.98696605)(202.60118195,341.15696777)
\curveto(202.76117595,341.32696571)(202.89617582,341.5369655)(203.00618195,341.78696777)
\curveto(203.04617567,341.88696515)(203.07617564,341.98696505)(203.09618195,342.08696777)
\curveto(203.1161756,342.18696485)(203.14117557,342.29196475)(203.17118195,342.40196777)
\curveto(203.18117553,342.4419646)(203.18617553,342.47696456)(203.18618195,342.50696777)
\curveto(203.18617553,342.54696449)(203.19117552,342.58696445)(203.20118195,342.62696777)
\lineto(203.20118195,342.76196777)
\curveto(203.20117551,342.81196423)(203.20617551,342.86196418)(203.21618195,342.91196777)
}
}
{
\newrgbcolor{curcolor}{0 0 0}
\pscustom[linestyle=none,fillstyle=solid,fillcolor=curcolor]
{
\newpath
\moveto(211.14110382,347.20196777)
\curveto(211.74109802,347.22195982)(212.24109752,347.1369599)(212.64110382,346.94696777)
\curveto(213.04109672,346.75696028)(213.3560964,346.47696056)(213.58610382,346.10696777)
\curveto(213.6560961,345.99696104)(213.71109605,345.87696116)(213.75110382,345.74696777)
\curveto(213.79109597,345.62696141)(213.83109593,345.50196154)(213.87110382,345.37196777)
\curveto(213.89109587,345.29196175)(213.90109586,345.21696182)(213.90110382,345.14696777)
\curveto(213.91109585,345.07696196)(213.92609583,345.00696203)(213.94610382,344.93696777)
\curveto(213.94609581,344.87696216)(213.95109581,344.8369622)(213.96110382,344.81696777)
\curveto(213.98109578,344.67696236)(213.99109577,344.53196251)(213.99110382,344.38196777)
\lineto(213.99110382,343.94696777)
\lineto(213.99110382,342.61196777)
\lineto(213.99110382,340.18196777)
\curveto(213.99109577,339.99196705)(213.98609577,339.80696723)(213.97610382,339.62696777)
\curveto(213.97609578,339.45696758)(213.90609585,339.34696769)(213.76610382,339.29696777)
\curveto(213.70609605,339.27696776)(213.63609612,339.26696777)(213.55610382,339.26696777)
\lineto(213.31610382,339.26696777)
\lineto(212.50610382,339.26696777)
\curveto(212.38609737,339.26696777)(212.27609748,339.27196777)(212.17610382,339.28196777)
\curveto(212.08609767,339.30196774)(212.01609774,339.34696769)(211.96610382,339.41696777)
\curveto(211.92609783,339.47696756)(211.90109786,339.55196749)(211.89110382,339.64196777)
\lineto(211.89110382,339.95696777)
\lineto(211.89110382,341.00696777)
\lineto(211.89110382,343.24196777)
\curveto(211.89109787,343.61196343)(211.87609788,343.95196309)(211.84610382,344.26196777)
\curveto(211.81609794,344.58196246)(211.72609803,344.85196219)(211.57610382,345.07196777)
\curveto(211.43609832,345.27196177)(211.23109853,345.41196163)(210.96110382,345.49196777)
\curveto(210.91109885,345.51196153)(210.8560989,345.52196152)(210.79610382,345.52196777)
\curveto(210.74609901,345.52196152)(210.69109907,345.53196151)(210.63110382,345.55196777)
\curveto(210.58109918,345.56196148)(210.51609924,345.56196148)(210.43610382,345.55196777)
\curveto(210.36609939,345.55196149)(210.31109945,345.54696149)(210.27110382,345.53696777)
\curveto(210.23109953,345.52696151)(210.19609956,345.52196152)(210.16610382,345.52196777)
\curveto(210.13609962,345.52196152)(210.10609965,345.51696152)(210.07610382,345.50696777)
\curveto(209.84609991,345.44696159)(209.6611001,345.36696167)(209.52110382,345.26696777)
\curveto(209.20110056,345.036962)(209.01110075,344.70196234)(208.95110382,344.26196777)
\curveto(208.89110087,343.82196322)(208.8611009,343.32696371)(208.86110382,342.77696777)
\lineto(208.86110382,340.90196777)
\lineto(208.86110382,339.98696777)
\lineto(208.86110382,339.71696777)
\curveto(208.8611009,339.62696741)(208.84610091,339.55196749)(208.81610382,339.49196777)
\curveto(208.76610099,339.38196766)(208.68610107,339.31696772)(208.57610382,339.29696777)
\curveto(208.46610129,339.27696776)(208.33110143,339.26696777)(208.17110382,339.26696777)
\lineto(207.42110382,339.26696777)
\curveto(207.31110245,339.26696777)(207.20110256,339.27196777)(207.09110382,339.28196777)
\curveto(206.98110278,339.29196775)(206.90110286,339.32696771)(206.85110382,339.38696777)
\curveto(206.78110298,339.47696756)(206.74610301,339.60696743)(206.74610382,339.77696777)
\curveto(206.756103,339.94696709)(206.761103,340.10696693)(206.76110382,340.25696777)
\lineto(206.76110382,342.29696777)
\lineto(206.76110382,345.59696777)
\lineto(206.76110382,346.36196777)
\lineto(206.76110382,346.66196777)
\curveto(206.77110299,346.75196029)(206.80110296,346.82696021)(206.85110382,346.88696777)
\curveto(206.87110289,346.91696012)(206.90110286,346.9369601)(206.94110382,346.94696777)
\curveto(206.99110277,346.96696007)(207.04110272,346.98196006)(207.09110382,346.99196777)
\lineto(207.16610382,346.99196777)
\curveto(207.21610254,347.00196004)(207.26610249,347.00696003)(207.31610382,347.00696777)
\lineto(207.48110382,347.00696777)
\lineto(208.11110382,347.00696777)
\curveto(208.19110157,347.00696003)(208.26610149,347.00196004)(208.33610382,346.99196777)
\curveto(208.41610134,346.99196005)(208.48610127,346.98196006)(208.54610382,346.96196777)
\curveto(208.61610114,346.93196011)(208.6611011,346.88696015)(208.68110382,346.82696777)
\curveto(208.71110105,346.76696027)(208.73610102,346.69696034)(208.75610382,346.61696777)
\curveto(208.76610099,346.57696046)(208.76610099,346.5419605)(208.75610382,346.51196777)
\curveto(208.756101,346.48196056)(208.76610099,346.45196059)(208.78610382,346.42196777)
\curveto(208.80610095,346.37196067)(208.82110094,346.3419607)(208.83110382,346.33196777)
\curveto(208.85110091,346.32196072)(208.87610088,346.30696073)(208.90610382,346.28696777)
\curveto(209.01610074,346.27696076)(209.10610065,346.31196073)(209.17610382,346.39196777)
\curveto(209.24610051,346.48196056)(209.32110044,346.55196049)(209.40110382,346.60196777)
\curveto(209.67110009,346.80196024)(209.97109979,346.96196008)(210.30110382,347.08196777)
\curveto(210.39109937,347.11195993)(210.48109928,347.13195991)(210.57110382,347.14196777)
\curveto(210.67109909,347.15195989)(210.77609898,347.16695987)(210.88610382,347.18696777)
\curveto(210.91609884,347.19695984)(210.9610988,347.19695984)(211.02110382,347.18696777)
\curveto(211.08109868,347.18695985)(211.12109864,347.19195985)(211.14110382,347.20196777)
}
}
{
\newrgbcolor{curcolor}{0 0 0}
\pscustom[linestyle=none,fillstyle=solid,fillcolor=curcolor]
{
\newpath
\moveto(20.4857132,328.24196777)
\lineto(21.6107132,328.24196777)
\curveto(21.72071076,328.24196005)(21.82071066,328.23696005)(21.9107132,328.22696777)
\curveto(22.00071048,328.21696007)(22.06571042,328.18196011)(22.1057132,328.12196777)
\curveto(22.15571033,328.06196023)(22.1857103,327.97696031)(22.1957132,327.86696777)
\curveto(22.20571028,327.76696052)(22.21071027,327.66196063)(22.2107132,327.55196777)
\lineto(22.2107132,326.50196777)
\lineto(22.2107132,324.26696777)
\curveto(22.21071027,323.90696438)(22.22571026,323.56696472)(22.2557132,323.24696777)
\curveto(22.2857102,322.92696536)(22.37571011,322.66196563)(22.5257132,322.45196777)
\curveto(22.66570982,322.24196605)(22.89070959,322.0919662)(23.2007132,322.00196777)
\curveto(23.25070923,321.9919663)(23.29070919,321.9869663)(23.3207132,321.98696777)
\curveto(23.36070912,321.9869663)(23.40570908,321.98196631)(23.4557132,321.97196777)
\curveto(23.50570898,321.96196633)(23.56070892,321.95696633)(23.6207132,321.95696777)
\curveto(23.6807088,321.95696633)(23.72570876,321.96196633)(23.7557132,321.97196777)
\curveto(23.80570868,321.9919663)(23.84570864,321.99696629)(23.8757132,321.98696777)
\curveto(23.91570857,321.97696631)(23.95570853,321.98196631)(23.9957132,322.00196777)
\curveto(24.20570828,322.05196624)(24.37070811,322.11696617)(24.4907132,322.19696777)
\curveto(24.67070781,322.30696598)(24.81070767,322.44696584)(24.9107132,322.61696777)
\curveto(25.02070746,322.79696549)(25.09570739,322.9919653)(25.1357132,323.20196777)
\curveto(25.1857073,323.42196487)(25.21570727,323.66196463)(25.2257132,323.92196777)
\curveto(25.23570725,324.1919641)(25.24070724,324.47196382)(25.2407132,324.76196777)
\lineto(25.2407132,326.57696777)
\lineto(25.2407132,327.55196777)
\lineto(25.2407132,327.82196777)
\curveto(25.24070724,327.92196037)(25.26070722,328.00196029)(25.3007132,328.06196777)
\curveto(25.35070713,328.15196014)(25.42570706,328.20196009)(25.5257132,328.21196777)
\curveto(25.62570686,328.23196006)(25.74570674,328.24196005)(25.8857132,328.24196777)
\lineto(26.6807132,328.24196777)
\lineto(26.9657132,328.24196777)
\curveto(27.05570543,328.24196005)(27.13070535,328.22196007)(27.1907132,328.18196777)
\curveto(27.27070521,328.13196016)(27.31570517,328.05696023)(27.3257132,327.95696777)
\curveto(27.33570515,327.85696043)(27.34070514,327.74196055)(27.3407132,327.61196777)
\lineto(27.3407132,326.47196777)
\lineto(27.3407132,322.25696777)
\lineto(27.3407132,321.19196777)
\lineto(27.3407132,320.89196777)
\curveto(27.34070514,320.7919675)(27.32070516,320.71696757)(27.2807132,320.66696777)
\curveto(27.23070525,320.5869677)(27.15570533,320.54196775)(27.0557132,320.53196777)
\curveto(26.95570553,320.52196777)(26.85070563,320.51696777)(26.7407132,320.51696777)
\lineto(25.9307132,320.51696777)
\curveto(25.82070666,320.51696777)(25.72070676,320.52196777)(25.6307132,320.53196777)
\curveto(25.55070693,320.54196775)(25.485707,320.58196771)(25.4357132,320.65196777)
\curveto(25.41570707,320.68196761)(25.39570709,320.72696756)(25.3757132,320.78696777)
\curveto(25.36570712,320.84696744)(25.35070713,320.90696738)(25.3307132,320.96696777)
\curveto(25.32070716,321.02696726)(25.30570718,321.08196721)(25.2857132,321.13196777)
\curveto(25.26570722,321.18196711)(25.23570725,321.21196708)(25.1957132,321.22196777)
\curveto(25.17570731,321.24196705)(25.15070733,321.24696704)(25.1207132,321.23696777)
\curveto(25.09070739,321.22696706)(25.06570742,321.21696707)(25.0457132,321.20696777)
\curveto(24.97570751,321.16696712)(24.91570757,321.12196717)(24.8657132,321.07196777)
\curveto(24.81570767,321.02196727)(24.76070772,320.97696731)(24.7007132,320.93696777)
\curveto(24.66070782,320.90696738)(24.62070786,320.87196742)(24.5807132,320.83196777)
\curveto(24.55070793,320.80196749)(24.51070797,320.77196752)(24.4607132,320.74196777)
\curveto(24.23070825,320.60196769)(23.96070852,320.4919678)(23.6507132,320.41196777)
\curveto(23.5807089,320.3919679)(23.51070897,320.38196791)(23.4407132,320.38196777)
\curveto(23.37070911,320.37196792)(23.29570919,320.35696793)(23.2157132,320.33696777)
\curveto(23.17570931,320.32696796)(23.13070935,320.32696796)(23.0807132,320.33696777)
\curveto(23.04070944,320.33696795)(23.00070948,320.33196796)(22.9607132,320.32196777)
\curveto(22.93070955,320.31196798)(22.86570962,320.31196798)(22.7657132,320.32196777)
\curveto(22.67570981,320.32196797)(22.61570987,320.32696796)(22.5857132,320.33696777)
\curveto(22.53570995,320.33696795)(22.48571,320.34196795)(22.4357132,320.35196777)
\lineto(22.2857132,320.35196777)
\curveto(22.16571032,320.38196791)(22.05071043,320.40696788)(21.9407132,320.42696777)
\curveto(21.83071065,320.44696784)(21.72071076,320.47696781)(21.6107132,320.51696777)
\curveto(21.56071092,320.53696775)(21.51571097,320.55196774)(21.4757132,320.56196777)
\curveto(21.44571104,320.58196771)(21.40571108,320.60196769)(21.3557132,320.62196777)
\curveto(21.00571148,320.81196748)(20.72571176,321.07696721)(20.5157132,321.41696777)
\curveto(20.3857121,321.62696666)(20.29071219,321.87696641)(20.2307132,322.16696777)
\curveto(20.17071231,322.46696582)(20.13071235,322.78196551)(20.1107132,323.11196777)
\curveto(20.10071238,323.45196484)(20.09571239,323.79696449)(20.0957132,324.14696777)
\curveto(20.10571238,324.50696378)(20.11071237,324.86196343)(20.1107132,325.21196777)
\lineto(20.1107132,327.25196777)
\curveto(20.11071237,327.38196091)(20.10571238,327.53196076)(20.0957132,327.70196777)
\curveto(20.09571239,327.88196041)(20.12071236,328.01196028)(20.1707132,328.09196777)
\curveto(20.20071228,328.14196015)(20.26071222,328.1869601)(20.3507132,328.22696777)
\curveto(20.41071207,328.22696006)(20.45571203,328.23196006)(20.4857132,328.24196777)
}
}
{
\newrgbcolor{curcolor}{0 0 0}
\pscustom[linestyle=none,fillstyle=solid,fillcolor=curcolor]
{
\newpath
\moveto(33.3969632,328.45196777)
\curveto(33.99695739,328.47195982)(34.49695689,328.3869599)(34.8969632,328.19696777)
\curveto(35.29695609,328.00696028)(35.61195578,327.72696056)(35.8419632,327.35696777)
\curveto(35.91195548,327.24696104)(35.96695542,327.12696116)(36.0069632,326.99696777)
\curveto(36.04695534,326.87696141)(36.0869553,326.75196154)(36.1269632,326.62196777)
\curveto(36.14695524,326.54196175)(36.15695523,326.46696182)(36.1569632,326.39696777)
\curveto(36.16695522,326.32696196)(36.18195521,326.25696203)(36.2019632,326.18696777)
\curveto(36.20195519,326.12696216)(36.20695518,326.0869622)(36.2169632,326.06696777)
\curveto(36.23695515,325.92696236)(36.24695514,325.78196251)(36.2469632,325.63196777)
\lineto(36.2469632,325.19696777)
\lineto(36.2469632,323.86196777)
\lineto(36.2469632,321.43196777)
\curveto(36.24695514,321.24196705)(36.24195515,321.05696723)(36.2319632,320.87696777)
\curveto(36.23195516,320.70696758)(36.16195523,320.59696769)(36.0219632,320.54696777)
\curveto(35.96195543,320.52696776)(35.8919555,320.51696777)(35.8119632,320.51696777)
\lineto(35.5719632,320.51696777)
\lineto(34.7619632,320.51696777)
\curveto(34.64195675,320.51696777)(34.53195686,320.52196777)(34.4319632,320.53196777)
\curveto(34.34195705,320.55196774)(34.27195712,320.59696769)(34.2219632,320.66696777)
\curveto(34.18195721,320.72696756)(34.15695723,320.80196749)(34.1469632,320.89196777)
\lineto(34.1469632,321.20696777)
\lineto(34.1469632,322.25696777)
\lineto(34.1469632,324.49196777)
\curveto(34.14695724,324.86196343)(34.13195726,325.20196309)(34.1019632,325.51196777)
\curveto(34.07195732,325.83196246)(33.98195741,326.10196219)(33.8319632,326.32196777)
\curveto(33.6919577,326.52196177)(33.4869579,326.66196163)(33.2169632,326.74196777)
\curveto(33.16695822,326.76196153)(33.11195828,326.77196152)(33.0519632,326.77196777)
\curveto(33.00195839,326.77196152)(32.94695844,326.78196151)(32.8869632,326.80196777)
\curveto(32.83695855,326.81196148)(32.77195862,326.81196148)(32.6919632,326.80196777)
\curveto(32.62195877,326.80196149)(32.56695882,326.79696149)(32.5269632,326.78696777)
\curveto(32.4869589,326.77696151)(32.45195894,326.77196152)(32.4219632,326.77196777)
\curveto(32.391959,326.77196152)(32.36195903,326.76696152)(32.3319632,326.75696777)
\curveto(32.10195929,326.69696159)(31.91695947,326.61696167)(31.7769632,326.51696777)
\curveto(31.45695993,326.286962)(31.26696012,325.95196234)(31.2069632,325.51196777)
\curveto(31.14696024,325.07196322)(31.11696027,324.57696371)(31.1169632,324.02696777)
\lineto(31.1169632,322.15196777)
\lineto(31.1169632,321.23696777)
\lineto(31.1169632,320.96696777)
\curveto(31.11696027,320.87696741)(31.10196029,320.80196749)(31.0719632,320.74196777)
\curveto(31.02196037,320.63196766)(30.94196045,320.56696772)(30.8319632,320.54696777)
\curveto(30.72196067,320.52696776)(30.5869608,320.51696777)(30.4269632,320.51696777)
\lineto(29.6769632,320.51696777)
\curveto(29.56696182,320.51696777)(29.45696193,320.52196777)(29.3469632,320.53196777)
\curveto(29.23696215,320.54196775)(29.15696223,320.57696771)(29.1069632,320.63696777)
\curveto(29.03696235,320.72696756)(29.00196239,320.85696743)(29.0019632,321.02696777)
\curveto(29.01196238,321.19696709)(29.01696237,321.35696693)(29.0169632,321.50696777)
\lineto(29.0169632,323.54696777)
\lineto(29.0169632,326.84696777)
\lineto(29.0169632,327.61196777)
\lineto(29.0169632,327.91196777)
\curveto(29.02696236,328.00196029)(29.05696233,328.07696021)(29.1069632,328.13696777)
\curveto(29.12696226,328.16696012)(29.15696223,328.1869601)(29.1969632,328.19696777)
\curveto(29.24696214,328.21696007)(29.29696209,328.23196006)(29.3469632,328.24196777)
\lineto(29.4219632,328.24196777)
\curveto(29.47196192,328.25196004)(29.52196187,328.25696003)(29.5719632,328.25696777)
\lineto(29.7369632,328.25696777)
\lineto(30.3669632,328.25696777)
\curveto(30.44696094,328.25696003)(30.52196087,328.25196004)(30.5919632,328.24196777)
\curveto(30.67196072,328.24196005)(30.74196065,328.23196006)(30.8019632,328.21196777)
\curveto(30.87196052,328.18196011)(30.91696047,328.13696015)(30.9369632,328.07696777)
\curveto(30.96696042,328.01696027)(30.9919604,327.94696034)(31.0119632,327.86696777)
\curveto(31.02196037,327.82696046)(31.02196037,327.7919605)(31.0119632,327.76196777)
\curveto(31.01196038,327.73196056)(31.02196037,327.70196059)(31.0419632,327.67196777)
\curveto(31.06196033,327.62196067)(31.07696031,327.5919607)(31.0869632,327.58196777)
\curveto(31.10696028,327.57196072)(31.13196026,327.55696073)(31.1619632,327.53696777)
\curveto(31.27196012,327.52696076)(31.36196003,327.56196073)(31.4319632,327.64196777)
\curveto(31.50195989,327.73196056)(31.57695981,327.80196049)(31.6569632,327.85196777)
\curveto(31.92695946,328.05196024)(32.22695916,328.21196008)(32.5569632,328.33196777)
\curveto(32.64695874,328.36195993)(32.73695865,328.38195991)(32.8269632,328.39196777)
\curveto(32.92695846,328.40195989)(33.03195836,328.41695987)(33.1419632,328.43696777)
\curveto(33.17195822,328.44695984)(33.21695817,328.44695984)(33.2769632,328.43696777)
\curveto(33.33695805,328.43695985)(33.37695801,328.44195985)(33.3969632,328.45196777)
}
}
{
\newrgbcolor{curcolor}{0 0 0}
\pscustom[linestyle=none,fillstyle=solid,fillcolor=curcolor]
{
\newpath
\moveto(44.9282132,321.11696777)
\curveto(44.94820535,321.00696728)(44.95820534,320.89696739)(44.9582132,320.78696777)
\curveto(44.96820533,320.67696761)(44.91820538,320.60196769)(44.8082132,320.56196777)
\curveto(44.74820555,320.53196776)(44.67820562,320.51696777)(44.5982132,320.51696777)
\lineto(44.3582132,320.51696777)
\lineto(43.5482132,320.51696777)
\lineto(43.2782132,320.51696777)
\curveto(43.1982071,320.52696776)(43.13320716,320.55196774)(43.0832132,320.59196777)
\curveto(43.01320728,320.63196766)(42.95820734,320.6869676)(42.9182132,320.75696777)
\curveto(42.88820741,320.83696745)(42.84320745,320.90196739)(42.7832132,320.95196777)
\curveto(42.76320753,320.97196732)(42.73820756,320.9869673)(42.7082132,320.99696777)
\curveto(42.67820762,321.01696727)(42.63820766,321.02196727)(42.5882132,321.01196777)
\curveto(42.53820776,320.9919673)(42.48820781,320.96696732)(42.4382132,320.93696777)
\curveto(42.3982079,320.90696738)(42.35320794,320.88196741)(42.3032132,320.86196777)
\curveto(42.25320804,320.82196747)(42.1982081,320.7869675)(42.1382132,320.75696777)
\lineto(41.9582132,320.66696777)
\curveto(41.82820847,320.60696768)(41.6932086,320.55696773)(41.5532132,320.51696777)
\curveto(41.41320888,320.4869678)(41.26820903,320.45196784)(41.1182132,320.41196777)
\curveto(41.04820925,320.3919679)(40.97820932,320.38196791)(40.9082132,320.38196777)
\curveto(40.84820945,320.37196792)(40.78320951,320.36196793)(40.7132132,320.35196777)
\lineto(40.6232132,320.35196777)
\curveto(40.5932097,320.34196795)(40.56320973,320.33696795)(40.5332132,320.33696777)
\lineto(40.3682132,320.33696777)
\curveto(40.26821003,320.31696797)(40.16821013,320.31696797)(40.0682132,320.33696777)
\lineto(39.9332132,320.33696777)
\curveto(39.86321043,320.35696793)(39.7932105,320.36696792)(39.7232132,320.36696777)
\curveto(39.66321063,320.35696793)(39.60321069,320.36196793)(39.5432132,320.38196777)
\curveto(39.44321085,320.40196789)(39.34821095,320.42196787)(39.2582132,320.44196777)
\curveto(39.16821113,320.45196784)(39.08321121,320.47696781)(39.0032132,320.51696777)
\curveto(38.71321158,320.62696766)(38.46321183,320.76696752)(38.2532132,320.93696777)
\curveto(38.05321224,321.11696717)(37.8932124,321.35196694)(37.7732132,321.64196777)
\curveto(37.74321255,321.71196658)(37.71321258,321.7869665)(37.6832132,321.86696777)
\curveto(37.66321263,321.94696634)(37.64321265,322.03196626)(37.6232132,322.12196777)
\curveto(37.60321269,322.17196612)(37.5932127,322.22196607)(37.5932132,322.27196777)
\curveto(37.60321269,322.32196597)(37.60321269,322.37196592)(37.5932132,322.42196777)
\curveto(37.58321271,322.45196584)(37.57321272,322.51196578)(37.5632132,322.60196777)
\curveto(37.56321273,322.70196559)(37.56821273,322.77196552)(37.5782132,322.81196777)
\curveto(37.5982127,322.91196538)(37.60821269,322.99696529)(37.6082132,323.06696777)
\lineto(37.6982132,323.39696777)
\curveto(37.72821257,323.51696477)(37.76821253,323.62196467)(37.8182132,323.71196777)
\curveto(37.98821231,324.00196429)(38.18321211,324.22196407)(38.4032132,324.37196777)
\curveto(38.62321167,324.52196377)(38.90321139,324.65196364)(39.2432132,324.76196777)
\curveto(39.37321092,324.81196348)(39.50821079,324.84696344)(39.6482132,324.86696777)
\curveto(39.78821051,324.8869634)(39.92821037,324.91196338)(40.0682132,324.94196777)
\curveto(40.14821015,324.96196333)(40.23321006,324.97196332)(40.3232132,324.97196777)
\curveto(40.41320988,324.98196331)(40.50320979,324.99696329)(40.5932132,325.01696777)
\curveto(40.66320963,325.03696325)(40.73320956,325.04196325)(40.8032132,325.03196777)
\curveto(40.87320942,325.03196326)(40.94820935,325.04196325)(41.0282132,325.06196777)
\curveto(41.0982092,325.08196321)(41.16820913,325.0919632)(41.2382132,325.09196777)
\curveto(41.30820899,325.0919632)(41.38320891,325.10196319)(41.4632132,325.12196777)
\curveto(41.67320862,325.17196312)(41.86320843,325.21196308)(42.0332132,325.24196777)
\curveto(42.21320808,325.28196301)(42.37320792,325.37196292)(42.5132132,325.51196777)
\curveto(42.60320769,325.60196269)(42.66320763,325.70196259)(42.6932132,325.81196777)
\curveto(42.70320759,325.84196245)(42.70320759,325.86696242)(42.6932132,325.88696777)
\curveto(42.6932076,325.90696238)(42.6982076,325.92696236)(42.7082132,325.94696777)
\curveto(42.71820758,325.96696232)(42.72320757,325.99696229)(42.7232132,326.03696777)
\lineto(42.7232132,326.12696777)
\lineto(42.6932132,326.24696777)
\curveto(42.6932076,326.286962)(42.68820761,326.32196197)(42.6782132,326.35196777)
\curveto(42.57820772,326.65196164)(42.36820793,326.85696143)(42.0482132,326.96696777)
\curveto(41.95820834,326.99696129)(41.84820845,327.01696127)(41.7182132,327.02696777)
\curveto(41.5982087,327.04696124)(41.47320882,327.05196124)(41.3432132,327.04196777)
\curveto(41.21320908,327.04196125)(41.08820921,327.03196126)(40.9682132,327.01196777)
\curveto(40.84820945,326.9919613)(40.74320955,326.96696132)(40.6532132,326.93696777)
\curveto(40.5932097,326.91696137)(40.53320976,326.8869614)(40.4732132,326.84696777)
\curveto(40.42320987,326.81696147)(40.37320992,326.78196151)(40.3232132,326.74196777)
\curveto(40.27321002,326.70196159)(40.21821008,326.64696164)(40.1582132,326.57696777)
\curveto(40.10821019,326.50696178)(40.07321022,326.44196185)(40.0532132,326.38196777)
\curveto(40.00321029,326.28196201)(39.95821034,326.1869621)(39.9182132,326.09696777)
\curveto(39.88821041,326.00696228)(39.81821048,325.94696234)(39.7082132,325.91696777)
\curveto(39.62821067,325.89696239)(39.54321075,325.8869624)(39.4532132,325.88696777)
\lineto(39.1832132,325.88696777)
\lineto(38.6132132,325.88696777)
\curveto(38.56321173,325.8869624)(38.51321178,325.88196241)(38.4632132,325.87196777)
\curveto(38.41321188,325.87196242)(38.36821193,325.87696241)(38.3282132,325.88696777)
\lineto(38.1932132,325.88696777)
\curveto(38.17321212,325.89696239)(38.14821215,325.90196239)(38.1182132,325.90196777)
\curveto(38.08821221,325.90196239)(38.06321223,325.91196238)(38.0432132,325.93196777)
\curveto(37.96321233,325.95196234)(37.90821239,326.01696227)(37.8782132,326.12696777)
\curveto(37.86821243,326.17696211)(37.86821243,326.22696206)(37.8782132,326.27696777)
\curveto(37.88821241,326.32696196)(37.8982124,326.37196192)(37.9082132,326.41196777)
\curveto(37.93821236,326.52196177)(37.96821233,326.62196167)(37.9982132,326.71196777)
\curveto(38.03821226,326.81196148)(38.08321221,326.90196139)(38.1332132,326.98196777)
\lineto(38.2232132,327.13196777)
\lineto(38.3132132,327.28196777)
\curveto(38.3932119,327.3919609)(38.4932118,327.49696079)(38.6132132,327.59696777)
\curveto(38.63321166,327.60696068)(38.66321163,327.63196066)(38.7032132,327.67196777)
\curveto(38.75321154,327.71196058)(38.7982115,327.74696054)(38.8382132,327.77696777)
\curveto(38.87821142,327.80696048)(38.92321137,327.83696045)(38.9732132,327.86696777)
\curveto(39.14321115,327.97696031)(39.32321097,328.06196023)(39.5132132,328.12196777)
\curveto(39.70321059,328.1919601)(39.8982104,328.25696003)(40.0982132,328.31696777)
\curveto(40.21821008,328.34695994)(40.34320995,328.36695992)(40.4732132,328.37696777)
\curveto(40.60320969,328.3869599)(40.73320956,328.40695988)(40.8632132,328.43696777)
\curveto(40.90320939,328.44695984)(40.96320933,328.44695984)(41.0432132,328.43696777)
\curveto(41.13320916,328.42695986)(41.18820911,328.43195986)(41.2082132,328.45196777)
\curveto(41.61820868,328.46195983)(42.00820829,328.44695984)(42.3782132,328.40696777)
\curveto(42.75820754,328.36695992)(43.0982072,328.29196)(43.3982132,328.18196777)
\curveto(43.70820659,328.07196022)(43.97320632,327.92196037)(44.1932132,327.73196777)
\curveto(44.41320588,327.55196074)(44.58320571,327.31696097)(44.7032132,327.02696777)
\curveto(44.77320552,326.85696143)(44.81320548,326.66196163)(44.8232132,326.44196777)
\curveto(44.83320546,326.22196207)(44.83820546,325.99696229)(44.8382132,325.76696777)
\lineto(44.8382132,322.42196777)
\lineto(44.8382132,321.83696777)
\curveto(44.83820546,321.64696664)(44.85820544,321.47196682)(44.8982132,321.31196777)
\curveto(44.90820539,321.28196701)(44.91320538,321.24696704)(44.9132132,321.20696777)
\curveto(44.91320538,321.17696711)(44.91820538,321.14696714)(44.9282132,321.11696777)
\moveto(42.7232132,323.42696777)
\curveto(42.73320756,323.47696481)(42.73820756,323.53196476)(42.7382132,323.59196777)
\curveto(42.73820756,323.66196463)(42.73320756,323.72196457)(42.7232132,323.77196777)
\curveto(42.70320759,323.83196446)(42.6932076,323.8869644)(42.6932132,323.93696777)
\curveto(42.6932076,323.9869643)(42.67320762,324.02696426)(42.6332132,324.05696777)
\curveto(42.58320771,324.09696419)(42.50820779,324.11696417)(42.4082132,324.11696777)
\curveto(42.36820793,324.10696418)(42.33320796,324.09696419)(42.3032132,324.08696777)
\curveto(42.27320802,324.0869642)(42.23820806,324.08196421)(42.1982132,324.07196777)
\curveto(42.12820817,324.05196424)(42.05320824,324.03696425)(41.9732132,324.02696777)
\curveto(41.8932084,324.01696427)(41.81320848,324.00196429)(41.7332132,323.98196777)
\curveto(41.70320859,323.97196432)(41.65820864,323.96696432)(41.5982132,323.96696777)
\curveto(41.46820883,323.93696435)(41.33820896,323.91696437)(41.2082132,323.90696777)
\curveto(41.07820922,323.89696439)(40.95320934,323.87196442)(40.8332132,323.83196777)
\curveto(40.75320954,323.81196448)(40.67820962,323.7919645)(40.6082132,323.77196777)
\curveto(40.53820976,323.76196453)(40.46820983,323.74196455)(40.3982132,323.71196777)
\curveto(40.18821011,323.62196467)(40.00821029,323.4869648)(39.8582132,323.30696777)
\curveto(39.71821058,323.12696516)(39.66821063,322.87696541)(39.7082132,322.55696777)
\curveto(39.72821057,322.3869659)(39.78321051,322.24696604)(39.8732132,322.13696777)
\curveto(39.94321035,322.02696626)(40.04821025,321.93696635)(40.1882132,321.86696777)
\curveto(40.32820997,321.80696648)(40.47820982,321.76196653)(40.6382132,321.73196777)
\curveto(40.80820949,321.70196659)(40.98320931,321.6919666)(41.1632132,321.70196777)
\curveto(41.35320894,321.72196657)(41.52820877,321.75696653)(41.6882132,321.80696777)
\curveto(41.94820835,321.8869664)(42.15320814,322.01196628)(42.3032132,322.18196777)
\curveto(42.45320784,322.36196593)(42.56820773,322.58196571)(42.6482132,322.84196777)
\curveto(42.66820763,322.91196538)(42.67820762,322.98196531)(42.6782132,323.05196777)
\curveto(42.68820761,323.13196516)(42.70320759,323.21196508)(42.7232132,323.29196777)
\lineto(42.7232132,323.42696777)
}
}
{
\newrgbcolor{curcolor}{0 0 0}
\pscustom[linestyle=none,fillstyle=solid,fillcolor=curcolor]
{
}
}
{
\newrgbcolor{curcolor}{0 0 0}
\pscustom[linestyle=none,fillstyle=solid,fillcolor=curcolor]
{
\newpath
\moveto(53.6216507,331.21196777)
\curveto(53.71164686,331.21195708)(53.81164676,331.21195708)(53.9216507,331.21196777)
\curveto(54.04164653,331.21195708)(54.15664641,331.20695708)(54.2666507,331.19696777)
\curveto(54.38664618,331.1869571)(54.49164608,331.16695712)(54.5816507,331.13696777)
\curveto(54.6716459,331.11695717)(54.73164584,331.08195721)(54.7616507,331.03196777)
\curveto(54.82164575,330.95195734)(54.85164572,330.83695745)(54.8516507,330.68696777)
\lineto(54.8516507,330.28196777)
\curveto(54.85164572,330.18195811)(54.84664572,330.08195821)(54.8366507,329.98196777)
\curveto(54.83664573,329.88195841)(54.81664575,329.80695848)(54.7766507,329.75696777)
\curveto(54.73664583,329.69695859)(54.68664588,329.65695863)(54.6266507,329.63696777)
\curveto(54.566646,329.62695866)(54.49664607,329.62195867)(54.4166507,329.62196777)
\lineto(54.1916507,329.62196777)
\curveto(54.12164645,329.63195866)(54.05164652,329.63195866)(53.9816507,329.62196777)
\curveto(53.80164677,329.58195871)(53.66164691,329.53195876)(53.5616507,329.47196777)
\curveto(53.46164711,329.42195887)(53.38164719,329.31195898)(53.3216507,329.14196777)
\curveto(53.30164727,329.11195918)(53.29164728,329.08195921)(53.2916507,329.05196777)
\curveto(53.30164727,329.03195926)(53.30164727,329.00695928)(53.2916507,328.97696777)
\curveto(53.28164729,328.93695935)(53.2716473,328.87695941)(53.2616507,328.79696777)
\curveto(53.25164732,328.71695957)(53.25164732,328.65195964)(53.2616507,328.60196777)
\curveto(53.28164729,328.53195976)(53.30664726,328.47195982)(53.3366507,328.42196777)
\curveto(53.3666472,328.37195992)(53.41164716,328.33195996)(53.4716507,328.30196777)
\curveto(53.571647,328.25196004)(53.69164688,328.23696005)(53.8316507,328.25696777)
\curveto(53.9716466,328.27696001)(54.10164647,328.27696001)(54.2216507,328.25696777)
\curveto(54.2716463,328.24696004)(54.31164626,328.24196005)(54.3416507,328.24196777)
\curveto(54.38164619,328.25196004)(54.42164615,328.25196004)(54.4616507,328.24196777)
\curveto(54.55164602,328.20196009)(54.61664595,328.15696013)(54.6566507,328.10696777)
\curveto(54.67664589,328.07696021)(54.69164588,328.02696026)(54.7016507,327.95696777)
\curveto(54.71164586,327.89696039)(54.72164585,327.82696046)(54.7316507,327.74696777)
\curveto(54.74164583,327.67696061)(54.74164583,327.60196069)(54.7316507,327.52196777)
\curveto(54.73164584,327.45196084)(54.72664584,327.39696089)(54.7166507,327.35696777)
\curveto(54.70664586,327.31696097)(54.70664586,327.27696101)(54.7166507,327.23696777)
\curveto(54.72664584,327.20696108)(54.72164585,327.17196112)(54.7016507,327.13196777)
\curveto(54.68164589,327.01196128)(54.62164595,326.93696135)(54.5216507,326.90696777)
\curveto(54.44164613,326.86696142)(54.34664622,326.84696144)(54.2366507,326.84696777)
\curveto(54.12664644,326.85696143)(54.01664655,326.86196143)(53.9066507,326.86196777)
\lineto(53.8016507,326.86196777)
\curveto(53.76164681,326.86196143)(53.72664684,326.85696143)(53.6966507,326.84696777)
\lineto(53.5766507,326.84696777)
\curveto(53.40664716,326.80696148)(53.30164727,326.69696159)(53.2616507,326.51696777)
\curveto(53.24164733,326.45696183)(53.23664733,326.3869619)(53.2466507,326.30696777)
\curveto(53.25664731,326.22696206)(53.26164731,326.14696214)(53.2616507,326.06696777)
\lineto(53.2616507,325.15196777)
\lineto(53.2616507,322.22696777)
\lineto(53.2616507,321.52196777)
\lineto(53.2616507,321.32696777)
\curveto(53.2716473,321.26696702)(53.2666473,321.21196708)(53.2466507,321.16196777)
\lineto(53.2466507,320.99696777)
\curveto(53.24664732,320.83696745)(53.22164735,320.72196757)(53.1716507,320.65196777)
\curveto(53.15164742,320.62196767)(53.11664745,320.59696769)(53.0666507,320.57696777)
\curveto(53.01664755,320.56696772)(52.9666476,320.55196774)(52.9166507,320.53196777)
\lineto(52.8416507,320.53196777)
\curveto(52.79164778,320.52196777)(52.73664783,320.51696777)(52.6766507,320.51696777)
\curveto(52.61664795,320.52696776)(52.56164801,320.53196776)(52.5116507,320.53196777)
\lineto(51.8516507,320.53196777)
\curveto(51.78164879,320.53196776)(51.70664886,320.52696776)(51.6266507,320.51696777)
\curveto(51.55664901,320.51696777)(51.49664907,320.52696776)(51.4466507,320.54696777)
\curveto(51.32664924,320.57696771)(51.24664932,320.62696766)(51.2066507,320.69696777)
\curveto(51.17664939,320.74696754)(51.15664941,320.81196748)(51.1466507,320.89196777)
\lineto(51.1466507,321.13196777)
\lineto(51.1466507,321.91196777)
\lineto(51.1466507,326.11196777)
\curveto(51.14664942,326.28196201)(51.13664943,326.42696186)(51.1166507,326.54696777)
\curveto(51.09664947,326.67696161)(51.02664954,326.76696152)(50.9066507,326.81696777)
\curveto(50.79664977,326.86696142)(50.66164991,326.87696141)(50.5016507,326.84696777)
\curveto(50.34165023,326.82696146)(50.20665036,326.84196145)(50.0966507,326.89196777)
\curveto(49.98665058,326.94196135)(49.91665065,327.02696126)(49.8866507,327.14696777)
\curveto(49.8666507,327.19696109)(49.86165071,327.25696103)(49.8716507,327.32696777)
\lineto(49.8716507,327.53696777)
\curveto(49.8716507,327.71696057)(49.88165069,327.86696042)(49.9016507,327.98696777)
\curveto(49.92165065,328.10696018)(50.00665056,328.1919601)(50.1566507,328.24196777)
\curveto(50.23665033,328.26196003)(50.32165025,328.27196002)(50.4116507,328.27196777)
\lineto(50.6666507,328.27196777)
\curveto(50.75664981,328.27196002)(50.83664973,328.27696001)(50.9066507,328.28696777)
\curveto(50.97664959,328.30695998)(51.03164954,328.34695994)(51.0716507,328.40696777)
\curveto(51.14164943,328.50695978)(51.1666494,328.63195966)(51.1466507,328.78196777)
\curveto(51.13664943,328.94195935)(51.14664942,329.0919592)(51.1766507,329.23196777)
\curveto(51.18664938,329.27195902)(51.19164938,329.31195898)(51.1916507,329.35196777)
\curveto(51.20164937,329.3919589)(51.21164936,329.43695885)(51.2216507,329.48696777)
\curveto(51.26164931,329.62695866)(51.30164927,329.75195854)(51.3416507,329.86196777)
\curveto(51.38164919,329.98195831)(51.43664913,330.0919582)(51.5066507,330.19196777)
\curveto(51.64664892,330.43195786)(51.83164874,330.62195767)(52.0616507,330.76196777)
\curveto(52.29164828,330.91195738)(52.55164802,331.02695726)(52.8416507,331.10696777)
\curveto(52.92164765,331.13695715)(53.00664756,331.15195714)(53.0966507,331.15196777)
\curveto(53.18664738,331.16195713)(53.27664729,331.17695711)(53.3666507,331.19696777)
\curveto(53.39664717,331.20695708)(53.44164713,331.20695708)(53.5016507,331.19696777)
\curveto(53.56164701,331.1869571)(53.60164697,331.1919571)(53.6216507,331.21196777)
}
}
{
\newrgbcolor{curcolor}{0 0 0}
\pscustom[linestyle=none,fillstyle=solid,fillcolor=curcolor]
{
\newpath
\moveto(63.43141632,324.70196777)
\curveto(63.45140775,324.64196365)(63.46140774,324.55696373)(63.46141632,324.44696777)
\curveto(63.46140774,324.33696395)(63.45140775,324.25196404)(63.43141632,324.19196777)
\lineto(63.43141632,324.04196777)
\curveto(63.41140779,323.96196433)(63.4014078,323.88196441)(63.40141632,323.80196777)
\curveto(63.41140779,323.72196457)(63.4064078,323.64196465)(63.38641632,323.56196777)
\curveto(63.36640784,323.4919648)(63.35140785,323.42696486)(63.34141632,323.36696777)
\curveto(63.33140787,323.30696498)(63.32140788,323.24196505)(63.31141632,323.17196777)
\curveto(63.27140793,323.06196523)(63.23640797,322.94696534)(63.20641632,322.82696777)
\curveto(63.17640803,322.71696557)(63.13640807,322.61196568)(63.08641632,322.51196777)
\curveto(62.87640833,322.03196626)(62.6014086,321.64196665)(62.26141632,321.34196777)
\curveto(61.92140928,321.04196725)(61.51140969,320.7919675)(61.03141632,320.59196777)
\curveto(60.91141029,320.54196775)(60.78641042,320.50696778)(60.65641632,320.48696777)
\curveto(60.53641067,320.45696783)(60.41141079,320.42696786)(60.28141632,320.39696777)
\curveto(60.23141097,320.37696791)(60.17641103,320.36696792)(60.11641632,320.36696777)
\curveto(60.05641115,320.36696792)(60.0014112,320.36196793)(59.95141632,320.35196777)
\lineto(59.84641632,320.35196777)
\curveto(59.81641139,320.34196795)(59.78641142,320.33696795)(59.75641632,320.33696777)
\curveto(59.7064115,320.32696796)(59.62641158,320.32196797)(59.51641632,320.32196777)
\curveto(59.4064118,320.31196798)(59.32141188,320.31696797)(59.26141632,320.33696777)
\lineto(59.11141632,320.33696777)
\curveto(59.06141214,320.34696794)(59.0064122,320.35196794)(58.94641632,320.35196777)
\curveto(58.89641231,320.34196795)(58.84641236,320.34696794)(58.79641632,320.36696777)
\curveto(58.75641245,320.37696791)(58.71641249,320.38196791)(58.67641632,320.38196777)
\curveto(58.64641256,320.38196791)(58.6064126,320.3869679)(58.55641632,320.39696777)
\curveto(58.45641275,320.42696786)(58.35641285,320.45196784)(58.25641632,320.47196777)
\curveto(58.15641305,320.4919678)(58.06141314,320.52196777)(57.97141632,320.56196777)
\curveto(57.85141335,320.60196769)(57.73641347,320.64196765)(57.62641632,320.68196777)
\curveto(57.52641368,320.72196757)(57.42141378,320.77196752)(57.31141632,320.83196777)
\curveto(56.96141424,321.04196725)(56.66141454,321.286967)(56.41141632,321.56696777)
\curveto(56.16141504,321.84696644)(55.95141525,322.18196611)(55.78141632,322.57196777)
\curveto(55.73141547,322.66196563)(55.69141551,322.75696553)(55.66141632,322.85696777)
\curveto(55.64141556,322.95696533)(55.61641559,323.06196523)(55.58641632,323.17196777)
\curveto(55.56641564,323.22196507)(55.55641565,323.26696502)(55.55641632,323.30696777)
\curveto(55.55641565,323.34696494)(55.54641566,323.3919649)(55.52641632,323.44196777)
\curveto(55.5064157,323.52196477)(55.49641571,323.60196469)(55.49641632,323.68196777)
\curveto(55.49641571,323.77196452)(55.48641572,323.85696443)(55.46641632,323.93696777)
\curveto(55.45641575,323.9869643)(55.45141575,324.03196426)(55.45141632,324.07196777)
\lineto(55.45141632,324.20696777)
\curveto(55.43141577,324.26696402)(55.42141578,324.35196394)(55.42141632,324.46196777)
\curveto(55.43141577,324.57196372)(55.44641576,324.65696363)(55.46641632,324.71696777)
\lineto(55.46641632,324.82196777)
\curveto(55.47641573,324.87196342)(55.47641573,324.92196337)(55.46641632,324.97196777)
\curveto(55.46641574,325.03196326)(55.47641573,325.0869632)(55.49641632,325.13696777)
\curveto(55.5064157,325.1869631)(55.51141569,325.23196306)(55.51141632,325.27196777)
\curveto(55.51141569,325.32196297)(55.52141568,325.37196292)(55.54141632,325.42196777)
\curveto(55.58141562,325.55196274)(55.61641559,325.67696261)(55.64641632,325.79696777)
\curveto(55.67641553,325.92696236)(55.71641549,326.05196224)(55.76641632,326.17196777)
\curveto(55.94641526,326.58196171)(56.16141504,326.92196137)(56.41141632,327.19196777)
\curveto(56.66141454,327.47196082)(56.96641424,327.72696056)(57.32641632,327.95696777)
\curveto(57.42641378,328.00696028)(57.53141367,328.05196024)(57.64141632,328.09196777)
\curveto(57.75141345,328.13196016)(57.86141334,328.17696011)(57.97141632,328.22696777)
\curveto(58.1014131,328.27696001)(58.23641297,328.31195998)(58.37641632,328.33196777)
\curveto(58.51641269,328.35195994)(58.66141254,328.38195991)(58.81141632,328.42196777)
\curveto(58.89141231,328.43195986)(58.96641224,328.43695985)(59.03641632,328.43696777)
\curveto(59.1064121,328.43695985)(59.17641203,328.44195985)(59.24641632,328.45196777)
\curveto(59.82641138,328.46195983)(60.32641088,328.40195989)(60.74641632,328.27196777)
\curveto(61.17641003,328.14196015)(61.55640965,327.96196033)(61.88641632,327.73196777)
\curveto(61.99640921,327.65196064)(62.1064091,327.56196073)(62.21641632,327.46196777)
\curveto(62.33640887,327.37196092)(62.43640877,327.27196102)(62.51641632,327.16196777)
\curveto(62.59640861,327.06196123)(62.66640854,326.96196133)(62.72641632,326.86196777)
\curveto(62.79640841,326.76196153)(62.86640834,326.65696163)(62.93641632,326.54696777)
\curveto(63.0064082,326.43696185)(63.06140814,326.31696197)(63.10141632,326.18696777)
\curveto(63.14140806,326.06696222)(63.18640802,325.93696235)(63.23641632,325.79696777)
\curveto(63.26640794,325.71696257)(63.29140791,325.63196266)(63.31141632,325.54196777)
\lineto(63.37141632,325.27196777)
\curveto(63.38140782,325.23196306)(63.38640782,325.1919631)(63.38641632,325.15196777)
\curveto(63.38640782,325.11196318)(63.39140781,325.07196322)(63.40141632,325.03196777)
\curveto(63.42140778,324.98196331)(63.42640778,324.92696336)(63.41641632,324.86696777)
\curveto(63.4064078,324.80696348)(63.41140779,324.75196354)(63.43141632,324.70196777)
\moveto(61.33141632,324.16196777)
\curveto(61.34140986,324.21196408)(61.34640986,324.28196401)(61.34641632,324.37196777)
\curveto(61.34640986,324.47196382)(61.34140986,324.54696374)(61.33141632,324.59696777)
\lineto(61.33141632,324.71696777)
\curveto(61.31140989,324.76696352)(61.3014099,324.82196347)(61.30141632,324.88196777)
\curveto(61.3014099,324.94196335)(61.29640991,324.99696329)(61.28641632,325.04696777)
\curveto(61.28640992,325.0869632)(61.28140992,325.11696317)(61.27141632,325.13696777)
\lineto(61.21141632,325.37696777)
\curveto(61.20141,325.46696282)(61.18141002,325.55196274)(61.15141632,325.63196777)
\curveto(61.04141016,325.8919624)(60.91141029,326.11196218)(60.76141632,326.29196777)
\curveto(60.61141059,326.48196181)(60.41141079,326.63196166)(60.16141632,326.74196777)
\curveto(60.1014111,326.76196153)(60.04141116,326.77696151)(59.98141632,326.78696777)
\curveto(59.92141128,326.80696148)(59.85641135,326.82696146)(59.78641632,326.84696777)
\curveto(59.7064115,326.86696142)(59.62141158,326.87196142)(59.53141632,326.86196777)
\lineto(59.26141632,326.86196777)
\curveto(59.23141197,326.84196145)(59.19641201,326.83196146)(59.15641632,326.83196777)
\curveto(59.11641209,326.84196145)(59.08141212,326.84196145)(59.05141632,326.83196777)
\lineto(58.84141632,326.77196777)
\curveto(58.78141242,326.76196153)(58.72641248,326.74196155)(58.67641632,326.71196777)
\curveto(58.42641278,326.60196169)(58.22141298,326.44196185)(58.06141632,326.23196777)
\curveto(57.91141329,326.03196226)(57.79141341,325.79696249)(57.70141632,325.52696777)
\curveto(57.67141353,325.42696286)(57.64641356,325.32196297)(57.62641632,325.21196777)
\curveto(57.61641359,325.10196319)(57.6014136,324.9919633)(57.58141632,324.88196777)
\curveto(57.57141363,324.83196346)(57.56641364,324.78196351)(57.56641632,324.73196777)
\lineto(57.56641632,324.58196777)
\curveto(57.54641366,324.51196378)(57.53641367,324.40696388)(57.53641632,324.26696777)
\curveto(57.54641366,324.12696416)(57.56141364,324.02196427)(57.58141632,323.95196777)
\lineto(57.58141632,323.81696777)
\curveto(57.6014136,323.73696455)(57.61641359,323.65696463)(57.62641632,323.57696777)
\curveto(57.63641357,323.50696478)(57.65141355,323.43196486)(57.67141632,323.35196777)
\curveto(57.77141343,323.05196524)(57.87641333,322.80696548)(57.98641632,322.61696777)
\curveto(58.1064131,322.43696585)(58.29141291,322.27196602)(58.54141632,322.12196777)
\curveto(58.61141259,322.07196622)(58.68641252,322.03196626)(58.76641632,322.00196777)
\curveto(58.85641235,321.97196632)(58.94641226,321.94696634)(59.03641632,321.92696777)
\curveto(59.07641213,321.91696637)(59.11141209,321.91196638)(59.14141632,321.91196777)
\curveto(59.17141203,321.92196637)(59.206412,321.92196637)(59.24641632,321.91196777)
\lineto(59.36641632,321.88196777)
\curveto(59.41641179,321.88196641)(59.46141174,321.8869664)(59.50141632,321.89696777)
\lineto(59.62141632,321.89696777)
\curveto(59.7014115,321.91696637)(59.78141142,321.93196636)(59.86141632,321.94196777)
\curveto(59.94141126,321.95196634)(60.01641119,321.97196632)(60.08641632,322.00196777)
\curveto(60.34641086,322.10196619)(60.55641065,322.23696605)(60.71641632,322.40696777)
\curveto(60.87641033,322.57696571)(61.01141019,322.7869655)(61.12141632,323.03696777)
\curveto(61.16141004,323.13696515)(61.19141001,323.23696505)(61.21141632,323.33696777)
\curveto(61.23140997,323.43696485)(61.25640995,323.54196475)(61.28641632,323.65196777)
\curveto(61.29640991,323.6919646)(61.3014099,323.72696456)(61.30141632,323.75696777)
\curveto(61.3014099,323.79696449)(61.3064099,323.83696445)(61.31641632,323.87696777)
\lineto(61.31641632,324.01196777)
\curveto(61.31640989,324.06196423)(61.32140988,324.11196418)(61.33141632,324.16196777)
}
}
{
\newrgbcolor{curcolor}{0 0 0}
\pscustom[linestyle=none,fillstyle=solid,fillcolor=curcolor]
{
\newpath
\moveto(65.8813382,330.56696777)
\lineto(66.8863382,330.56696777)
\curveto(67.03633521,330.56695772)(67.16633508,330.55695773)(67.2763382,330.53696777)
\curveto(67.39633485,330.52695776)(67.48133477,330.46695782)(67.5313382,330.35696777)
\curveto(67.5513347,330.30695798)(67.56133469,330.24695804)(67.5613382,330.17696777)
\lineto(67.5613382,329.96696777)
\lineto(67.5613382,329.29196777)
\curveto(67.56133469,329.24195905)(67.55633469,329.18195911)(67.5463382,329.11196777)
\curveto(67.5463347,329.05195924)(67.5513347,328.99695929)(67.5613382,328.94696777)
\lineto(67.5613382,328.78196777)
\curveto(67.56133469,328.70195959)(67.56633468,328.62695966)(67.5763382,328.55696777)
\curveto(67.58633466,328.49695979)(67.61133464,328.44195985)(67.6513382,328.39196777)
\curveto(67.72133453,328.30195999)(67.8463344,328.25196004)(68.0263382,328.24196777)
\lineto(68.5663382,328.24196777)
\lineto(68.7463382,328.24196777)
\curveto(68.80633344,328.24196005)(68.86133339,328.23196006)(68.9113382,328.21196777)
\curveto(69.02133323,328.16196013)(69.08133317,328.07196022)(69.0913382,327.94196777)
\curveto(69.11133314,327.81196048)(69.12133313,327.66696062)(69.1213382,327.50696777)
\lineto(69.1213382,327.29696777)
\curveto(69.13133312,327.22696106)(69.12633312,327.16696112)(69.1063382,327.11696777)
\curveto(69.05633319,326.95696133)(68.9513333,326.87196142)(68.7913382,326.86196777)
\curveto(68.63133362,326.85196144)(68.4513338,326.84696144)(68.2513382,326.84696777)
\lineto(68.1163382,326.84696777)
\curveto(68.07633417,326.85696143)(68.04133421,326.85696143)(68.0113382,326.84696777)
\curveto(67.97133428,326.83696145)(67.93633431,326.83196146)(67.9063382,326.83196777)
\curveto(67.87633437,326.84196145)(67.8463344,326.83696145)(67.8163382,326.81696777)
\curveto(67.73633451,326.79696149)(67.67633457,326.75196154)(67.6363382,326.68196777)
\curveto(67.60633464,326.62196167)(67.58133467,326.54696174)(67.5613382,326.45696777)
\curveto(67.5513347,326.40696188)(67.5513347,326.35196194)(67.5613382,326.29196777)
\curveto(67.57133468,326.23196206)(67.57133468,326.17696211)(67.5613382,326.12696777)
\lineto(67.5613382,325.19696777)
\lineto(67.5613382,323.44196777)
\curveto(67.56133469,323.1919651)(67.56633468,322.97196532)(67.5763382,322.78196777)
\curveto(67.59633465,322.60196569)(67.66133459,322.44196585)(67.7713382,322.30196777)
\curveto(67.82133443,322.24196605)(67.88633436,322.19696609)(67.9663382,322.16696777)
\lineto(68.2363382,322.10696777)
\curveto(68.26633398,322.09696619)(68.29633395,322.0919662)(68.3263382,322.09196777)
\curveto(68.36633388,322.10196619)(68.39633385,322.10196619)(68.4163382,322.09196777)
\lineto(68.5813382,322.09196777)
\curveto(68.69133356,322.0919662)(68.78633346,322.0869662)(68.8663382,322.07696777)
\curveto(68.9463333,322.06696622)(69.01133324,322.02696626)(69.0613382,321.95696777)
\curveto(69.10133315,321.89696639)(69.12133313,321.81696647)(69.1213382,321.71696777)
\lineto(69.1213382,321.43196777)
\curveto(69.12133313,321.22196707)(69.11633313,321.02696726)(69.1063382,320.84696777)
\curveto(69.10633314,320.67696761)(69.02633322,320.56196773)(68.8663382,320.50196777)
\curveto(68.81633343,320.48196781)(68.77133348,320.47696781)(68.7313382,320.48696777)
\curveto(68.69133356,320.4869678)(68.6463336,320.47696781)(68.5963382,320.45696777)
\lineto(68.4463382,320.45696777)
\curveto(68.42633382,320.45696783)(68.39633385,320.46196783)(68.3563382,320.47196777)
\curveto(68.31633393,320.47196782)(68.28133397,320.46696782)(68.2513382,320.45696777)
\curveto(68.20133405,320.44696784)(68.1463341,320.44696784)(68.0863382,320.45696777)
\lineto(67.9363382,320.45696777)
\lineto(67.7863382,320.45696777)
\curveto(67.73633451,320.44696784)(67.69133456,320.44696784)(67.6513382,320.45696777)
\lineto(67.4863382,320.45696777)
\curveto(67.43633481,320.46696782)(67.38133487,320.47196782)(67.3213382,320.47196777)
\curveto(67.26133499,320.47196782)(67.20633504,320.47696781)(67.1563382,320.48696777)
\curveto(67.08633516,320.49696779)(67.02133523,320.50696778)(66.9613382,320.51696777)
\lineto(66.7813382,320.54696777)
\curveto(66.67133558,320.57696771)(66.56633568,320.61196768)(66.4663382,320.65196777)
\curveto(66.36633588,320.6919676)(66.27133598,320.73696755)(66.1813382,320.78696777)
\lineto(66.0913382,320.84696777)
\curveto(66.06133619,320.87696741)(66.02633622,320.90696738)(65.9863382,320.93696777)
\curveto(65.96633628,320.95696733)(65.94133631,320.97696731)(65.9113382,320.99696777)
\lineto(65.8363382,321.07196777)
\curveto(65.69633655,321.26196703)(65.59133666,321.47196682)(65.5213382,321.70196777)
\curveto(65.50133675,321.74196655)(65.49133676,321.77696651)(65.4913382,321.80696777)
\curveto(65.50133675,321.84696644)(65.50133675,321.8919664)(65.4913382,321.94196777)
\curveto(65.48133677,321.96196633)(65.47633677,321.9869663)(65.4763382,322.01696777)
\curveto(65.47633677,322.04696624)(65.47133678,322.07196622)(65.4613382,322.09196777)
\lineto(65.4613382,322.24196777)
\curveto(65.4513368,322.28196601)(65.4463368,322.32696596)(65.4463382,322.37696777)
\curveto(65.45633679,322.42696586)(65.46133679,322.47696581)(65.4613382,322.52696777)
\lineto(65.4613382,323.09696777)
\lineto(65.4613382,325.33196777)
\lineto(65.4613382,326.12696777)
\lineto(65.4613382,326.33696777)
\curveto(65.47133678,326.40696188)(65.46633678,326.47196182)(65.4463382,326.53196777)
\curveto(65.40633684,326.67196162)(65.33633691,326.76196153)(65.2363382,326.80196777)
\curveto(65.12633712,326.85196144)(64.98633726,326.86696142)(64.8163382,326.84696777)
\curveto(64.6463376,326.82696146)(64.50133775,326.84196145)(64.3813382,326.89196777)
\curveto(64.30133795,326.92196137)(64.251338,326.96696132)(64.2313382,327.02696777)
\curveto(64.21133804,327.0869612)(64.19133806,327.16196113)(64.1713382,327.25196777)
\lineto(64.1713382,327.56696777)
\curveto(64.17133808,327.74696054)(64.18133807,327.8919604)(64.2013382,328.00196777)
\curveto(64.22133803,328.11196018)(64.30633794,328.1869601)(64.4563382,328.22696777)
\curveto(64.49633775,328.24696004)(64.53633771,328.25196004)(64.5763382,328.24196777)
\lineto(64.7113382,328.24196777)
\curveto(64.86133739,328.24196005)(65.00133725,328.24696004)(65.1313382,328.25696777)
\curveto(65.26133699,328.27696001)(65.3513369,328.33695995)(65.4013382,328.43696777)
\curveto(65.43133682,328.50695978)(65.4463368,328.5869597)(65.4463382,328.67696777)
\curveto(65.45633679,328.76695952)(65.46133679,328.85695943)(65.4613382,328.94696777)
\lineto(65.4613382,329.87696777)
\lineto(65.4613382,330.13196777)
\curveto(65.46133679,330.22195807)(65.47133678,330.29695799)(65.4913382,330.35696777)
\curveto(65.54133671,330.45695783)(65.61633663,330.52195777)(65.7163382,330.55196777)
\curveto(65.73633651,330.56195773)(65.76133649,330.56195773)(65.7913382,330.55196777)
\curveto(65.83133642,330.55195774)(65.86133639,330.55695773)(65.8813382,330.56696777)
}
}
{
\newrgbcolor{curcolor}{0 0 0}
\pscustom[linestyle=none,fillstyle=solid,fillcolor=curcolor]
{
\newpath
\moveto(77.8747757,324.70196777)
\curveto(77.89476713,324.64196365)(77.90476712,324.55696373)(77.9047757,324.44696777)
\curveto(77.90476712,324.33696395)(77.89476713,324.25196404)(77.8747757,324.19196777)
\lineto(77.8747757,324.04196777)
\curveto(77.85476717,323.96196433)(77.84476718,323.88196441)(77.8447757,323.80196777)
\curveto(77.85476717,323.72196457)(77.84976717,323.64196465)(77.8297757,323.56196777)
\curveto(77.80976721,323.4919648)(77.79476723,323.42696486)(77.7847757,323.36696777)
\curveto(77.77476725,323.30696498)(77.76476726,323.24196505)(77.7547757,323.17196777)
\curveto(77.71476731,323.06196523)(77.67976734,322.94696534)(77.6497757,322.82696777)
\curveto(77.6197674,322.71696557)(77.57976744,322.61196568)(77.5297757,322.51196777)
\curveto(77.3197677,322.03196626)(77.04476798,321.64196665)(76.7047757,321.34196777)
\curveto(76.36476866,321.04196725)(75.95476907,320.7919675)(75.4747757,320.59196777)
\curveto(75.35476967,320.54196775)(75.22976979,320.50696778)(75.0997757,320.48696777)
\curveto(74.97977004,320.45696783)(74.85477017,320.42696786)(74.7247757,320.39696777)
\curveto(74.67477035,320.37696791)(74.6197704,320.36696792)(74.5597757,320.36696777)
\curveto(74.49977052,320.36696792)(74.44477058,320.36196793)(74.3947757,320.35196777)
\lineto(74.2897757,320.35196777)
\curveto(74.25977076,320.34196795)(74.22977079,320.33696795)(74.1997757,320.33696777)
\curveto(74.14977087,320.32696796)(74.06977095,320.32196797)(73.9597757,320.32196777)
\curveto(73.84977117,320.31196798)(73.76477126,320.31696797)(73.7047757,320.33696777)
\lineto(73.5547757,320.33696777)
\curveto(73.50477152,320.34696794)(73.44977157,320.35196794)(73.3897757,320.35196777)
\curveto(73.33977168,320.34196795)(73.28977173,320.34696794)(73.2397757,320.36696777)
\curveto(73.19977182,320.37696791)(73.15977186,320.38196791)(73.1197757,320.38196777)
\curveto(73.08977193,320.38196791)(73.04977197,320.3869679)(72.9997757,320.39696777)
\curveto(72.89977212,320.42696786)(72.79977222,320.45196784)(72.6997757,320.47196777)
\curveto(72.59977242,320.4919678)(72.50477252,320.52196777)(72.4147757,320.56196777)
\curveto(72.29477273,320.60196769)(72.17977284,320.64196765)(72.0697757,320.68196777)
\curveto(71.96977305,320.72196757)(71.86477316,320.77196752)(71.7547757,320.83196777)
\curveto(71.40477362,321.04196725)(71.10477392,321.286967)(70.8547757,321.56696777)
\curveto(70.60477442,321.84696644)(70.39477463,322.18196611)(70.2247757,322.57196777)
\curveto(70.17477485,322.66196563)(70.13477489,322.75696553)(70.1047757,322.85696777)
\curveto(70.08477494,322.95696533)(70.05977496,323.06196523)(70.0297757,323.17196777)
\curveto(70.00977501,323.22196507)(69.99977502,323.26696502)(69.9997757,323.30696777)
\curveto(69.99977502,323.34696494)(69.98977503,323.3919649)(69.9697757,323.44196777)
\curveto(69.94977507,323.52196477)(69.93977508,323.60196469)(69.9397757,323.68196777)
\curveto(69.93977508,323.77196452)(69.92977509,323.85696443)(69.9097757,323.93696777)
\curveto(69.89977512,323.9869643)(69.89477513,324.03196426)(69.8947757,324.07196777)
\lineto(69.8947757,324.20696777)
\curveto(69.87477515,324.26696402)(69.86477516,324.35196394)(69.8647757,324.46196777)
\curveto(69.87477515,324.57196372)(69.88977513,324.65696363)(69.9097757,324.71696777)
\lineto(69.9097757,324.82196777)
\curveto(69.9197751,324.87196342)(69.9197751,324.92196337)(69.9097757,324.97196777)
\curveto(69.90977511,325.03196326)(69.9197751,325.0869632)(69.9397757,325.13696777)
\curveto(69.94977507,325.1869631)(69.95477507,325.23196306)(69.9547757,325.27196777)
\curveto(69.95477507,325.32196297)(69.96477506,325.37196292)(69.9847757,325.42196777)
\curveto(70.024775,325.55196274)(70.05977496,325.67696261)(70.0897757,325.79696777)
\curveto(70.1197749,325.92696236)(70.15977486,326.05196224)(70.2097757,326.17196777)
\curveto(70.38977463,326.58196171)(70.60477442,326.92196137)(70.8547757,327.19196777)
\curveto(71.10477392,327.47196082)(71.40977361,327.72696056)(71.7697757,327.95696777)
\curveto(71.86977315,328.00696028)(71.97477305,328.05196024)(72.0847757,328.09196777)
\curveto(72.19477283,328.13196016)(72.30477272,328.17696011)(72.4147757,328.22696777)
\curveto(72.54477248,328.27696001)(72.67977234,328.31195998)(72.8197757,328.33196777)
\curveto(72.95977206,328.35195994)(73.10477192,328.38195991)(73.2547757,328.42196777)
\curveto(73.33477169,328.43195986)(73.40977161,328.43695985)(73.4797757,328.43696777)
\curveto(73.54977147,328.43695985)(73.6197714,328.44195985)(73.6897757,328.45196777)
\curveto(74.26977075,328.46195983)(74.76977025,328.40195989)(75.1897757,328.27196777)
\curveto(75.6197694,328.14196015)(75.99976902,327.96196033)(76.3297757,327.73196777)
\curveto(76.43976858,327.65196064)(76.54976847,327.56196073)(76.6597757,327.46196777)
\curveto(76.77976824,327.37196092)(76.87976814,327.27196102)(76.9597757,327.16196777)
\curveto(77.03976798,327.06196123)(77.10976791,326.96196133)(77.1697757,326.86196777)
\curveto(77.23976778,326.76196153)(77.30976771,326.65696163)(77.3797757,326.54696777)
\curveto(77.44976757,326.43696185)(77.50476752,326.31696197)(77.5447757,326.18696777)
\curveto(77.58476744,326.06696222)(77.62976739,325.93696235)(77.6797757,325.79696777)
\curveto(77.70976731,325.71696257)(77.73476729,325.63196266)(77.7547757,325.54196777)
\lineto(77.8147757,325.27196777)
\curveto(77.8247672,325.23196306)(77.82976719,325.1919631)(77.8297757,325.15196777)
\curveto(77.82976719,325.11196318)(77.83476719,325.07196322)(77.8447757,325.03196777)
\curveto(77.86476716,324.98196331)(77.86976715,324.92696336)(77.8597757,324.86696777)
\curveto(77.84976717,324.80696348)(77.85476717,324.75196354)(77.8747757,324.70196777)
\moveto(75.7747757,324.16196777)
\curveto(75.78476924,324.21196408)(75.78976923,324.28196401)(75.7897757,324.37196777)
\curveto(75.78976923,324.47196382)(75.78476924,324.54696374)(75.7747757,324.59696777)
\lineto(75.7747757,324.71696777)
\curveto(75.75476927,324.76696352)(75.74476928,324.82196347)(75.7447757,324.88196777)
\curveto(75.74476928,324.94196335)(75.73976928,324.99696329)(75.7297757,325.04696777)
\curveto(75.72976929,325.0869632)(75.7247693,325.11696317)(75.7147757,325.13696777)
\lineto(75.6547757,325.37696777)
\curveto(75.64476938,325.46696282)(75.6247694,325.55196274)(75.5947757,325.63196777)
\curveto(75.48476954,325.8919624)(75.35476967,326.11196218)(75.2047757,326.29196777)
\curveto(75.05476997,326.48196181)(74.85477017,326.63196166)(74.6047757,326.74196777)
\curveto(74.54477048,326.76196153)(74.48477054,326.77696151)(74.4247757,326.78696777)
\curveto(74.36477066,326.80696148)(74.29977072,326.82696146)(74.2297757,326.84696777)
\curveto(74.14977087,326.86696142)(74.06477096,326.87196142)(73.9747757,326.86196777)
\lineto(73.7047757,326.86196777)
\curveto(73.67477135,326.84196145)(73.63977138,326.83196146)(73.5997757,326.83196777)
\curveto(73.55977146,326.84196145)(73.5247715,326.84196145)(73.4947757,326.83196777)
\lineto(73.2847757,326.77196777)
\curveto(73.2247718,326.76196153)(73.16977185,326.74196155)(73.1197757,326.71196777)
\curveto(72.86977215,326.60196169)(72.66477236,326.44196185)(72.5047757,326.23196777)
\curveto(72.35477267,326.03196226)(72.23477279,325.79696249)(72.1447757,325.52696777)
\curveto(72.11477291,325.42696286)(72.08977293,325.32196297)(72.0697757,325.21196777)
\curveto(72.05977296,325.10196319)(72.04477298,324.9919633)(72.0247757,324.88196777)
\curveto(72.01477301,324.83196346)(72.00977301,324.78196351)(72.0097757,324.73196777)
\lineto(72.0097757,324.58196777)
\curveto(71.98977303,324.51196378)(71.97977304,324.40696388)(71.9797757,324.26696777)
\curveto(71.98977303,324.12696416)(72.00477302,324.02196427)(72.0247757,323.95196777)
\lineto(72.0247757,323.81696777)
\curveto(72.04477298,323.73696455)(72.05977296,323.65696463)(72.0697757,323.57696777)
\curveto(72.07977294,323.50696478)(72.09477293,323.43196486)(72.1147757,323.35196777)
\curveto(72.21477281,323.05196524)(72.3197727,322.80696548)(72.4297757,322.61696777)
\curveto(72.54977247,322.43696585)(72.73477229,322.27196602)(72.9847757,322.12196777)
\curveto(73.05477197,322.07196622)(73.12977189,322.03196626)(73.2097757,322.00196777)
\curveto(73.29977172,321.97196632)(73.38977163,321.94696634)(73.4797757,321.92696777)
\curveto(73.5197715,321.91696637)(73.55477147,321.91196638)(73.5847757,321.91196777)
\curveto(73.61477141,321.92196637)(73.64977137,321.92196637)(73.6897757,321.91196777)
\lineto(73.8097757,321.88196777)
\curveto(73.85977116,321.88196641)(73.90477112,321.8869664)(73.9447757,321.89696777)
\lineto(74.0647757,321.89696777)
\curveto(74.14477088,321.91696637)(74.2247708,321.93196636)(74.3047757,321.94196777)
\curveto(74.38477064,321.95196634)(74.45977056,321.97196632)(74.5297757,322.00196777)
\curveto(74.78977023,322.10196619)(74.99977002,322.23696605)(75.1597757,322.40696777)
\curveto(75.3197697,322.57696571)(75.45476957,322.7869655)(75.5647757,323.03696777)
\curveto(75.60476942,323.13696515)(75.63476939,323.23696505)(75.6547757,323.33696777)
\curveto(75.67476935,323.43696485)(75.69976932,323.54196475)(75.7297757,323.65196777)
\curveto(75.73976928,323.6919646)(75.74476928,323.72696456)(75.7447757,323.75696777)
\curveto(75.74476928,323.79696449)(75.74976927,323.83696445)(75.7597757,323.87696777)
\lineto(75.7597757,324.01196777)
\curveto(75.75976926,324.06196423)(75.76476926,324.11196418)(75.7747757,324.16196777)
}
}
{
\newrgbcolor{curcolor}{0 0 0}
\pscustom[linestyle=none,fillstyle=solid,fillcolor=curcolor]
{
\newpath
\moveto(86.62469757,328.16696777)
\curveto(86.69468937,328.11696017)(86.72968934,328.03196026)(86.72969757,327.91196777)
\curveto(86.73968933,327.80196049)(86.74468932,327.6869606)(86.74469757,327.56696777)
\lineto(86.74469757,321.16196777)
\curveto(86.74468932,321.08196721)(86.73968933,321.00196729)(86.72969757,320.92196777)
\lineto(86.72969757,320.69696777)
\curveto(86.71968935,320.61696767)(86.70968936,320.54696774)(86.69969757,320.48696777)
\curveto(86.69968937,320.41696787)(86.69468937,320.34196795)(86.68469757,320.26196777)
\curveto(86.64468942,320.12196817)(86.60968946,319.9919683)(86.57969757,319.87196777)
\curveto(86.55968951,319.74196855)(86.52468954,319.62196867)(86.47469757,319.51196777)
\curveto(86.30468976,319.13196916)(86.08468998,318.81696947)(85.81469757,318.56696777)
\curveto(85.55469051,318.31696997)(85.23469083,318.11197018)(84.85469757,317.95196777)
\curveto(84.74469132,317.90197039)(84.63469143,317.86197043)(84.52469757,317.83196777)
\curveto(84.41469165,317.80197049)(84.29969177,317.77197052)(84.17969757,317.74196777)
\curveto(84.069692,317.71197058)(83.95969211,317.6919706)(83.84969757,317.68196777)
\curveto(83.73969233,317.67197062)(83.62969244,317.65697063)(83.51969757,317.63696777)
\lineto(83.39969757,317.63696777)
\curveto(83.35969271,317.62697066)(83.31469275,317.62197067)(83.26469757,317.62196777)
\curveto(83.22469284,317.61197068)(83.17969289,317.61197068)(83.12969757,317.62196777)
\curveto(83.07969299,317.62197067)(83.02969304,317.61697067)(82.97969757,317.60696777)
\curveto(82.92969314,317.59697069)(82.8646932,317.5919707)(82.78469757,317.59196777)
\curveto(82.70469336,317.5919707)(82.63969343,317.59697069)(82.58969757,317.60696777)
\lineto(82.45469757,317.60696777)
\curveto(82.41469365,317.60697068)(82.37469369,317.61197068)(82.33469757,317.62196777)
\curveto(82.25469381,317.64197065)(82.1696939,317.65197064)(82.07969757,317.65196777)
\curveto(81.99969407,317.65197064)(81.92469414,317.66197063)(81.85469757,317.68196777)
\curveto(81.83469423,317.6919706)(81.80969426,317.69697059)(81.77969757,317.69696777)
\curveto(81.74969432,317.69697059)(81.72469434,317.70197059)(81.70469757,317.71196777)
\curveto(81.60469446,317.73197056)(81.50469456,317.75697053)(81.40469757,317.78696777)
\curveto(81.31469475,317.80697048)(81.22469484,317.83697045)(81.13469757,317.87696777)
\curveto(80.75469531,318.03697025)(80.41469565,318.24197005)(80.11469757,318.49196777)
\curveto(79.81469625,318.73196956)(79.59469647,319.05696923)(79.45469757,319.46696777)
\curveto(79.43469663,319.49696879)(79.42469664,319.52696876)(79.42469757,319.55696777)
\curveto(79.42469664,319.5869687)(79.41969665,319.61196868)(79.40969757,319.63196777)
\curveto(79.37969669,319.76196853)(79.38969668,319.86196843)(79.43969757,319.93196777)
\curveto(79.49969657,319.9919683)(79.57969649,320.03196826)(79.67969757,320.05196777)
\curveto(79.77969629,320.07196822)(79.88969618,320.08196821)(80.00969757,320.08196777)
\curveto(80.13969593,320.07196822)(80.25969581,320.06696822)(80.36969757,320.06696777)
\lineto(80.87969757,320.06696777)
\lineto(80.99969757,320.06696777)
\curveto(81.03969503,320.05696823)(81.08469498,320.05196824)(81.13469757,320.05196777)
\curveto(81.29469477,320.01196828)(81.39469467,319.96196833)(81.43469757,319.90196777)
\curveto(81.47469459,319.83196846)(81.53469453,319.74196855)(81.61469757,319.63196777)
\curveto(81.64469442,319.5919687)(81.68969438,319.54196875)(81.74969757,319.48196777)
\curveto(81.75969431,319.46196883)(81.7696943,319.44696884)(81.77969757,319.43696777)
\curveto(81.78969428,319.42696886)(81.79969427,319.41196888)(81.80969757,319.39196777)
\curveto(81.88969418,319.33196896)(81.97469409,319.27696901)(82.06469757,319.22696777)
\curveto(82.15469391,319.17696911)(82.25469381,319.13196916)(82.36469757,319.09196777)
\curveto(82.43469363,319.07196922)(82.50469356,319.06196923)(82.57469757,319.06196777)
\curveto(82.64469342,319.05196924)(82.71969335,319.03696925)(82.79969757,319.01696777)
\lineto(82.96469757,319.01696777)
\curveto(83.03469303,318.99696929)(83.12469294,318.99696929)(83.23469757,319.01696777)
\curveto(83.34469272,319.02696926)(83.42969264,319.04196925)(83.48969757,319.06196777)
\curveto(83.53969253,319.08196921)(83.57969249,319.0919692)(83.60969757,319.09196777)
\curveto(83.64969242,319.0919692)(83.68969238,319.10196919)(83.72969757,319.12196777)
\curveto(83.93969213,319.21196908)(84.11469195,319.33196896)(84.25469757,319.48196777)
\curveto(84.39469167,319.63196866)(84.50969156,319.80696848)(84.59969757,320.00696777)
\curveto(84.61969145,320.06696822)(84.63469143,320.12696816)(84.64469757,320.18696777)
\curveto(84.65469141,320.24696804)(84.6696914,320.31196798)(84.68969757,320.38196777)
\curveto(84.70969136,320.47196782)(84.71969135,320.56696772)(84.71969757,320.66696777)
\curveto(84.72969134,320.77696751)(84.73469133,320.8869674)(84.73469757,320.99696777)
\lineto(84.73469757,321.11696777)
\curveto(84.74469132,321.15696713)(84.74469132,321.1919671)(84.73469757,321.22196777)
\curveto(84.71469135,321.27196702)(84.70469136,321.31696697)(84.70469757,321.35696777)
\curveto(84.71469135,321.39696689)(84.70969136,321.43696685)(84.68969757,321.47696777)
\curveto(84.67969139,321.49696679)(84.6646914,321.51196678)(84.64469757,321.52196777)
\lineto(84.59969757,321.56696777)
\curveto(84.50969156,321.57696671)(84.43469163,321.55696673)(84.37469757,321.50696777)
\curveto(84.32469174,321.45696683)(84.27469179,321.41196688)(84.22469757,321.37196777)
\curveto(84.13469193,321.30196699)(84.04469202,321.23696705)(83.95469757,321.17696777)
\curveto(83.8646922,321.11696717)(83.7646923,321.06196723)(83.65469757,321.01196777)
\curveto(83.54469252,320.96196733)(83.43469263,320.92196737)(83.32469757,320.89196777)
\curveto(83.21469285,320.86196743)(83.09969297,320.83196746)(82.97969757,320.80196777)
\lineto(82.79969757,320.77196777)
\curveto(82.74969332,320.77196752)(82.69969337,320.76696752)(82.64969757,320.75696777)
\curveto(82.59969347,320.74696754)(82.51969355,320.74196755)(82.40969757,320.74196777)
\curveto(82.29969377,320.74196755)(82.21969385,320.74696754)(82.16969757,320.75696777)
\lineto(82.04969757,320.75696777)
\curveto(82.01969405,320.76696752)(81.98469408,320.77196752)(81.94469757,320.77196777)
\curveto(81.91469415,320.77196752)(81.87969419,320.77696751)(81.83969757,320.78696777)
\curveto(81.69969437,320.81696747)(81.5646945,320.84196745)(81.43469757,320.86196777)
\curveto(81.30469476,320.8919674)(81.18469488,320.93196736)(81.07469757,320.98196777)
\curveto(80.64469542,321.15196714)(80.29469577,321.3869669)(80.02469757,321.68696777)
\curveto(79.7646963,321.99696629)(79.54469652,322.36696592)(79.36469757,322.79696777)
\curveto(79.31469675,322.90696538)(79.27969679,323.02196527)(79.25969757,323.14196777)
\curveto(79.23969683,323.26196503)(79.20969686,323.38196491)(79.16969757,323.50196777)
\curveto(79.1696969,323.55196474)(79.1646969,323.5919647)(79.15469757,323.62196777)
\curveto(79.13469693,323.70196459)(79.12469694,323.7869645)(79.12469757,323.87696777)
\curveto(79.12469694,323.97696431)(79.11469695,324.06696422)(79.09469757,324.14696777)
\curveto(79.08469698,324.19696409)(79.07969699,324.24196405)(79.07969757,324.28196777)
\lineto(79.07969757,324.43196777)
\curveto(79.069697,324.48196381)(79.064697,324.54196375)(79.06469757,324.61196777)
\curveto(79.064697,324.6919636)(79.069697,324.75696353)(79.07969757,324.80696777)
\lineto(79.07969757,324.95696777)
\curveto(79.08969698,324.99696329)(79.08969698,325.03696325)(79.07969757,325.07696777)
\curveto(79.07969699,325.11696317)(79.08969698,325.15696313)(79.10969757,325.19696777)
\curveto(79.12969694,325.29696299)(79.14469692,325.3919629)(79.15469757,325.48196777)
\curveto(79.1646969,325.58196271)(79.17969689,325.68196261)(79.19969757,325.78196777)
\curveto(79.25969681,325.98196231)(79.31969675,326.17196212)(79.37969757,326.35196777)
\curveto(79.44969662,326.53196176)(79.53469653,326.70196159)(79.63469757,326.86196777)
\curveto(79.68469638,326.96196133)(79.73969633,327.05196124)(79.79969757,327.13196777)
\lineto(80.00969757,327.40196777)
\curveto(80.03969603,327.45196084)(80.07969599,327.50196079)(80.12969757,327.55196777)
\curveto(80.18969588,327.60196069)(80.24469582,327.64696064)(80.29469757,327.68696777)
\lineto(80.38469757,327.77696777)
\curveto(80.43469563,327.81696047)(80.48469558,327.85196044)(80.53469757,327.88196777)
\curveto(80.58469548,327.92196037)(80.63469543,327.95696033)(80.68469757,327.98696777)
\curveto(80.81469525,328.06696022)(80.94969512,328.13696015)(81.08969757,328.19696777)
\curveto(81.22969484,328.25696003)(81.38469468,328.31195998)(81.55469757,328.36196777)
\curveto(81.63469443,328.3919599)(81.71469435,328.40695988)(81.79469757,328.40696777)
\curveto(81.88469418,328.41695987)(81.9696941,328.43195986)(82.04969757,328.45196777)
\curveto(82.08969398,328.46195983)(82.14469392,328.46195983)(82.21469757,328.45196777)
\curveto(82.28469378,328.44195985)(82.32969374,328.44695984)(82.34969757,328.46696777)
\curveto(82.6696934,328.47695981)(82.95469311,328.44695984)(83.20469757,328.37696777)
\curveto(83.4646926,328.30695998)(83.69469237,328.20696008)(83.89469757,328.07696777)
\curveto(83.92469214,328.05696023)(83.95469211,328.03196026)(83.98469757,328.00196777)
\curveto(84.01469205,327.98196031)(84.04969202,327.95696033)(84.08969757,327.92696777)
\curveto(84.14969192,327.87696041)(84.20469186,327.82696046)(84.25469757,327.77696777)
\curveto(84.30469176,327.72696056)(84.3646917,327.68196061)(84.43469757,327.64196777)
\curveto(84.45469161,327.63196066)(84.47969159,327.62196067)(84.50969757,327.61196777)
\curveto(84.54969152,327.60196069)(84.57969149,327.60696068)(84.59969757,327.62696777)
\curveto(84.64969142,327.64696064)(84.67969139,327.68196061)(84.68969757,327.73196777)
\curveto(84.69969137,327.78196051)(84.71469135,327.83196046)(84.73469757,327.88196777)
\curveto(84.75469131,327.93196036)(84.7696913,327.98196031)(84.77969757,328.03196777)
\curveto(84.79969127,328.0919602)(84.82969124,328.14196015)(84.86969757,328.18196777)
\curveto(84.92969114,328.22196007)(84.99969107,328.24196005)(85.07969757,328.24196777)
\curveto(85.1696909,328.25196004)(85.25969081,328.25696003)(85.34969757,328.25696777)
\lineto(86.11469757,328.25696777)
\curveto(86.22468984,328.25696003)(86.31968975,328.25196004)(86.39969757,328.24196777)
\curveto(86.48968958,328.24196005)(86.5646895,328.21696007)(86.62469757,328.16696777)
\moveto(84.56969757,323.53196777)
\curveto(84.60969146,323.62196467)(84.64469142,323.73696455)(84.67469757,323.87696777)
\curveto(84.70469136,324.01696427)(84.72469134,324.16196413)(84.73469757,324.31196777)
\curveto(84.74469132,324.47196382)(84.74469132,324.62696366)(84.73469757,324.77696777)
\curveto(84.73469133,324.92696336)(84.71969135,325.06196323)(84.68969757,325.18196777)
\curveto(84.6696914,325.22196307)(84.65969141,325.25196304)(84.65969757,325.27196777)
\curveto(84.6696914,325.30196299)(84.6696914,325.33696295)(84.65969757,325.37696777)
\lineto(84.59969757,325.58696777)
\curveto(84.57969149,325.65696263)(84.55469151,325.72196257)(84.52469757,325.78196777)
\curveto(84.38469168,326.13196216)(84.18469188,326.40196189)(83.92469757,326.59196777)
\curveto(83.6646924,326.78196151)(83.28469278,326.87696141)(82.78469757,326.87696777)
\curveto(82.7646933,326.85696143)(82.73469333,326.84696144)(82.69469757,326.84696777)
\curveto(82.6646934,326.85696143)(82.63469343,326.85696143)(82.60469757,326.84696777)
\curveto(82.53469353,326.82696146)(82.4696936,326.80696148)(82.40969757,326.78696777)
\curveto(82.34969372,326.77696151)(82.28969378,326.76196153)(82.22969757,326.74196777)
\curveto(81.9696941,326.63196166)(81.7696943,326.46696182)(81.62969757,326.24696777)
\curveto(81.48969458,326.02696226)(81.37469469,325.78196251)(81.28469757,325.51196777)
\curveto(81.2646948,325.46196283)(81.25469481,325.42196287)(81.25469757,325.39196777)
\curveto(81.25469481,325.36196293)(81.24969482,325.32196297)(81.23969757,325.27196777)
\curveto(81.20969486,325.16196313)(81.18969488,325.00196329)(81.17969757,324.79196777)
\curveto(81.1696949,324.58196371)(81.17969489,324.41196388)(81.20969757,324.28196777)
\lineto(81.20969757,324.13196777)
\curveto(81.22969484,324.05196424)(81.24469482,323.97196432)(81.25469757,323.89196777)
\curveto(81.2646948,323.82196447)(81.27969479,323.74696454)(81.29969757,323.66696777)
\curveto(81.38969468,323.40696488)(81.49969457,323.17696511)(81.62969757,322.97696777)
\curveto(81.75969431,322.7869655)(81.93969413,322.63196566)(82.16969757,322.51196777)
\curveto(82.2696938,322.46196583)(82.40969366,322.41196588)(82.58969757,322.36196777)
\curveto(82.65969341,322.36196593)(82.71469335,322.35696593)(82.75469757,322.34696777)
\curveto(82.77469329,322.34696594)(82.80469326,322.34196595)(82.84469757,322.33196777)
\curveto(82.88469318,322.33196596)(82.91469315,322.33696595)(82.93469757,322.34696777)
\lineto(83.08469757,322.34696777)
\curveto(83.17469289,322.36696592)(83.25969281,322.38196591)(83.33969757,322.39196777)
\curveto(83.41969265,322.40196589)(83.49969257,322.42696586)(83.57969757,322.46696777)
\curveto(83.82969224,322.56696572)(84.02969204,322.70696558)(84.17969757,322.88696777)
\curveto(84.33969173,323.06696522)(84.4696916,323.28196501)(84.56969757,323.53196777)
}
}
{
\newrgbcolor{curcolor}{0 0 0}
\pscustom[linestyle=none,fillstyle=solid,fillcolor=curcolor]
{
\newpath
\moveto(92.86961945,328.45196777)
\curveto(92.97961413,328.45195984)(93.07461404,328.44195985)(93.15461945,328.42196777)
\curveto(93.24461387,328.40195989)(93.3146138,328.35695993)(93.36461945,328.28696777)
\curveto(93.42461369,328.20696008)(93.45461366,328.06696022)(93.45461945,327.86696777)
\lineto(93.45461945,327.35696777)
\lineto(93.45461945,326.98196777)
\curveto(93.46461365,326.84196145)(93.44961366,326.73196156)(93.40961945,326.65196777)
\curveto(93.36961374,326.58196171)(93.3096138,326.53696175)(93.22961945,326.51696777)
\curveto(93.15961395,326.49696179)(93.07461404,326.4869618)(92.97461945,326.48696777)
\curveto(92.88461423,326.4869618)(92.78461433,326.4919618)(92.67461945,326.50196777)
\curveto(92.57461454,326.51196178)(92.47961463,326.50696178)(92.38961945,326.48696777)
\curveto(92.31961479,326.46696182)(92.24961486,326.45196184)(92.17961945,326.44196777)
\curveto(92.109615,326.44196185)(92.04461507,326.43196186)(91.98461945,326.41196777)
\curveto(91.82461529,326.36196193)(91.66461545,326.286962)(91.50461945,326.18696777)
\curveto(91.34461577,326.09696219)(91.21961589,325.9919623)(91.12961945,325.87196777)
\curveto(91.07961603,325.7919625)(91.02461609,325.70696258)(90.96461945,325.61696777)
\curveto(90.9146162,325.53696275)(90.86461625,325.45196284)(90.81461945,325.36196777)
\curveto(90.78461633,325.28196301)(90.75461636,325.19696309)(90.72461945,325.10696777)
\lineto(90.66461945,324.86696777)
\curveto(90.64461647,324.79696349)(90.63461648,324.72196357)(90.63461945,324.64196777)
\curveto(90.63461648,324.57196372)(90.62461649,324.50196379)(90.60461945,324.43196777)
\curveto(90.59461652,324.3919639)(90.58961652,324.35196394)(90.58961945,324.31196777)
\curveto(90.59961651,324.28196401)(90.59961651,324.25196404)(90.58961945,324.22196777)
\lineto(90.58961945,323.98196777)
\curveto(90.56961654,323.91196438)(90.56461655,323.83196446)(90.57461945,323.74196777)
\curveto(90.58461653,323.66196463)(90.58961652,323.58196471)(90.58961945,323.50196777)
\lineto(90.58961945,322.54196777)
\lineto(90.58961945,321.26696777)
\curveto(90.58961652,321.13696715)(90.58461653,321.01696727)(90.57461945,320.90696777)
\curveto(90.56461655,320.79696749)(90.53461658,320.70696758)(90.48461945,320.63696777)
\curveto(90.46461665,320.60696768)(90.42961668,320.58196771)(90.37961945,320.56196777)
\curveto(90.33961677,320.55196774)(90.29461682,320.54196775)(90.24461945,320.53196777)
\lineto(90.16961945,320.53196777)
\curveto(90.11961699,320.52196777)(90.06461705,320.51696777)(90.00461945,320.51696777)
\lineto(89.83961945,320.51696777)
\lineto(89.19461945,320.51696777)
\curveto(89.13461798,320.52696776)(89.06961804,320.53196776)(88.99961945,320.53196777)
\lineto(88.80461945,320.53196777)
\curveto(88.75461836,320.55196774)(88.70461841,320.56696772)(88.65461945,320.57696777)
\curveto(88.60461851,320.59696769)(88.56961854,320.63196766)(88.54961945,320.68196777)
\curveto(88.5096186,320.73196756)(88.48461863,320.80196749)(88.47461945,320.89196777)
\lineto(88.47461945,321.19196777)
\lineto(88.47461945,322.21196777)
\lineto(88.47461945,326.44196777)
\lineto(88.47461945,327.55196777)
\lineto(88.47461945,327.83696777)
\curveto(88.47461864,327.93696035)(88.49461862,328.01696027)(88.53461945,328.07696777)
\curveto(88.58461853,328.15696013)(88.65961845,328.20696008)(88.75961945,328.22696777)
\curveto(88.85961825,328.24696004)(88.97961813,328.25696003)(89.11961945,328.25696777)
\lineto(89.88461945,328.25696777)
\curveto(90.00461711,328.25696003)(90.109617,328.24696004)(90.19961945,328.22696777)
\curveto(90.28961682,328.21696007)(90.35961675,328.17196012)(90.40961945,328.09196777)
\curveto(90.43961667,328.04196025)(90.45461666,327.97196032)(90.45461945,327.88196777)
\lineto(90.48461945,327.61196777)
\curveto(90.49461662,327.53196076)(90.5096166,327.45696083)(90.52961945,327.38696777)
\curveto(90.55961655,327.31696097)(90.6096165,327.28196101)(90.67961945,327.28196777)
\curveto(90.69961641,327.30196099)(90.71961639,327.31196098)(90.73961945,327.31196777)
\curveto(90.75961635,327.31196098)(90.77961633,327.32196097)(90.79961945,327.34196777)
\curveto(90.85961625,327.3919609)(90.9096162,327.44696084)(90.94961945,327.50696777)
\curveto(90.99961611,327.57696071)(91.05961605,327.63696065)(91.12961945,327.68696777)
\curveto(91.16961594,327.71696057)(91.20461591,327.74696054)(91.23461945,327.77696777)
\curveto(91.26461585,327.81696047)(91.29961581,327.85196044)(91.33961945,327.88196777)
\lineto(91.60961945,328.06196777)
\curveto(91.7096154,328.12196017)(91.8096153,328.17696011)(91.90961945,328.22696777)
\curveto(92.0096151,328.26696002)(92.109615,328.30195999)(92.20961945,328.33196777)
\lineto(92.53961945,328.42196777)
\curveto(92.56961454,328.43195986)(92.62461449,328.43195986)(92.70461945,328.42196777)
\curveto(92.79461432,328.42195987)(92.84961426,328.43195986)(92.86961945,328.45196777)
}
}
{
\newrgbcolor{curcolor}{0 0 0}
\pscustom[linestyle=none,fillstyle=solid,fillcolor=curcolor]
{
\newpath
\moveto(101.32469757,321.11696777)
\curveto(101.34468972,321.00696728)(101.35468971,320.89696739)(101.35469757,320.78696777)
\curveto(101.3646897,320.67696761)(101.31468975,320.60196769)(101.20469757,320.56196777)
\curveto(101.14468992,320.53196776)(101.07468999,320.51696777)(100.99469757,320.51696777)
\lineto(100.75469757,320.51696777)
\lineto(99.94469757,320.51696777)
\lineto(99.67469757,320.51696777)
\curveto(99.59469147,320.52696776)(99.52969154,320.55196774)(99.47969757,320.59196777)
\curveto(99.40969166,320.63196766)(99.35469171,320.6869676)(99.31469757,320.75696777)
\curveto(99.28469178,320.83696745)(99.23969183,320.90196739)(99.17969757,320.95196777)
\curveto(99.15969191,320.97196732)(99.13469193,320.9869673)(99.10469757,320.99696777)
\curveto(99.07469199,321.01696727)(99.03469203,321.02196727)(98.98469757,321.01196777)
\curveto(98.93469213,320.9919673)(98.88469218,320.96696732)(98.83469757,320.93696777)
\curveto(98.79469227,320.90696738)(98.74969232,320.88196741)(98.69969757,320.86196777)
\curveto(98.64969242,320.82196747)(98.59469247,320.7869675)(98.53469757,320.75696777)
\lineto(98.35469757,320.66696777)
\curveto(98.22469284,320.60696768)(98.08969298,320.55696773)(97.94969757,320.51696777)
\curveto(97.80969326,320.4869678)(97.6646934,320.45196784)(97.51469757,320.41196777)
\curveto(97.44469362,320.3919679)(97.37469369,320.38196791)(97.30469757,320.38196777)
\curveto(97.24469382,320.37196792)(97.17969389,320.36196793)(97.10969757,320.35196777)
\lineto(97.01969757,320.35196777)
\curveto(96.98969408,320.34196795)(96.95969411,320.33696795)(96.92969757,320.33696777)
\lineto(96.76469757,320.33696777)
\curveto(96.6646944,320.31696797)(96.5646945,320.31696797)(96.46469757,320.33696777)
\lineto(96.32969757,320.33696777)
\curveto(96.25969481,320.35696793)(96.18969488,320.36696792)(96.11969757,320.36696777)
\curveto(96.05969501,320.35696793)(95.99969507,320.36196793)(95.93969757,320.38196777)
\curveto(95.83969523,320.40196789)(95.74469532,320.42196787)(95.65469757,320.44196777)
\curveto(95.5646955,320.45196784)(95.47969559,320.47696781)(95.39969757,320.51696777)
\curveto(95.10969596,320.62696766)(94.85969621,320.76696752)(94.64969757,320.93696777)
\curveto(94.44969662,321.11696717)(94.28969678,321.35196694)(94.16969757,321.64196777)
\curveto(94.13969693,321.71196658)(94.10969696,321.7869665)(94.07969757,321.86696777)
\curveto(94.05969701,321.94696634)(94.03969703,322.03196626)(94.01969757,322.12196777)
\curveto(93.99969707,322.17196612)(93.98969708,322.22196607)(93.98969757,322.27196777)
\curveto(93.99969707,322.32196597)(93.99969707,322.37196592)(93.98969757,322.42196777)
\curveto(93.97969709,322.45196584)(93.9696971,322.51196578)(93.95969757,322.60196777)
\curveto(93.95969711,322.70196559)(93.9646971,322.77196552)(93.97469757,322.81196777)
\curveto(93.99469707,322.91196538)(94.00469706,322.99696529)(94.00469757,323.06696777)
\lineto(94.09469757,323.39696777)
\curveto(94.12469694,323.51696477)(94.1646969,323.62196467)(94.21469757,323.71196777)
\curveto(94.38469668,324.00196429)(94.57969649,324.22196407)(94.79969757,324.37196777)
\curveto(95.01969605,324.52196377)(95.29969577,324.65196364)(95.63969757,324.76196777)
\curveto(95.7696953,324.81196348)(95.90469516,324.84696344)(96.04469757,324.86696777)
\curveto(96.18469488,324.8869634)(96.32469474,324.91196338)(96.46469757,324.94196777)
\curveto(96.54469452,324.96196333)(96.62969444,324.97196332)(96.71969757,324.97196777)
\curveto(96.80969426,324.98196331)(96.89969417,324.99696329)(96.98969757,325.01696777)
\curveto(97.05969401,325.03696325)(97.12969394,325.04196325)(97.19969757,325.03196777)
\curveto(97.2696938,325.03196326)(97.34469372,325.04196325)(97.42469757,325.06196777)
\curveto(97.49469357,325.08196321)(97.5646935,325.0919632)(97.63469757,325.09196777)
\curveto(97.70469336,325.0919632)(97.77969329,325.10196319)(97.85969757,325.12196777)
\curveto(98.069693,325.17196312)(98.25969281,325.21196308)(98.42969757,325.24196777)
\curveto(98.60969246,325.28196301)(98.7696923,325.37196292)(98.90969757,325.51196777)
\curveto(98.99969207,325.60196269)(99.05969201,325.70196259)(99.08969757,325.81196777)
\curveto(99.09969197,325.84196245)(99.09969197,325.86696242)(99.08969757,325.88696777)
\curveto(99.08969198,325.90696238)(99.09469197,325.92696236)(99.10469757,325.94696777)
\curveto(99.11469195,325.96696232)(99.11969195,325.99696229)(99.11969757,326.03696777)
\lineto(99.11969757,326.12696777)
\lineto(99.08969757,326.24696777)
\curveto(99.08969198,326.286962)(99.08469198,326.32196197)(99.07469757,326.35196777)
\curveto(98.97469209,326.65196164)(98.7646923,326.85696143)(98.44469757,326.96696777)
\curveto(98.35469271,326.99696129)(98.24469282,327.01696127)(98.11469757,327.02696777)
\curveto(97.99469307,327.04696124)(97.8696932,327.05196124)(97.73969757,327.04196777)
\curveto(97.60969346,327.04196125)(97.48469358,327.03196126)(97.36469757,327.01196777)
\curveto(97.24469382,326.9919613)(97.13969393,326.96696132)(97.04969757,326.93696777)
\curveto(96.98969408,326.91696137)(96.92969414,326.8869614)(96.86969757,326.84696777)
\curveto(96.81969425,326.81696147)(96.7696943,326.78196151)(96.71969757,326.74196777)
\curveto(96.6696944,326.70196159)(96.61469445,326.64696164)(96.55469757,326.57696777)
\curveto(96.50469456,326.50696178)(96.4696946,326.44196185)(96.44969757,326.38196777)
\curveto(96.39969467,326.28196201)(96.35469471,326.1869621)(96.31469757,326.09696777)
\curveto(96.28469478,326.00696228)(96.21469485,325.94696234)(96.10469757,325.91696777)
\curveto(96.02469504,325.89696239)(95.93969513,325.8869624)(95.84969757,325.88696777)
\lineto(95.57969757,325.88696777)
\lineto(95.00969757,325.88696777)
\curveto(94.95969611,325.8869624)(94.90969616,325.88196241)(94.85969757,325.87196777)
\curveto(94.80969626,325.87196242)(94.7646963,325.87696241)(94.72469757,325.88696777)
\lineto(94.58969757,325.88696777)
\curveto(94.5696965,325.89696239)(94.54469652,325.90196239)(94.51469757,325.90196777)
\curveto(94.48469658,325.90196239)(94.45969661,325.91196238)(94.43969757,325.93196777)
\curveto(94.35969671,325.95196234)(94.30469676,326.01696227)(94.27469757,326.12696777)
\curveto(94.2646968,326.17696211)(94.2646968,326.22696206)(94.27469757,326.27696777)
\curveto(94.28469678,326.32696196)(94.29469677,326.37196192)(94.30469757,326.41196777)
\curveto(94.33469673,326.52196177)(94.3646967,326.62196167)(94.39469757,326.71196777)
\curveto(94.43469663,326.81196148)(94.47969659,326.90196139)(94.52969757,326.98196777)
\lineto(94.61969757,327.13196777)
\lineto(94.70969757,327.28196777)
\curveto(94.78969628,327.3919609)(94.88969618,327.49696079)(95.00969757,327.59696777)
\curveto(95.02969604,327.60696068)(95.05969601,327.63196066)(95.09969757,327.67196777)
\curveto(95.14969592,327.71196058)(95.19469587,327.74696054)(95.23469757,327.77696777)
\curveto(95.27469579,327.80696048)(95.31969575,327.83696045)(95.36969757,327.86696777)
\curveto(95.53969553,327.97696031)(95.71969535,328.06196023)(95.90969757,328.12196777)
\curveto(96.09969497,328.1919601)(96.29469477,328.25696003)(96.49469757,328.31696777)
\curveto(96.61469445,328.34695994)(96.73969433,328.36695992)(96.86969757,328.37696777)
\curveto(96.99969407,328.3869599)(97.12969394,328.40695988)(97.25969757,328.43696777)
\curveto(97.29969377,328.44695984)(97.35969371,328.44695984)(97.43969757,328.43696777)
\curveto(97.52969354,328.42695986)(97.58469348,328.43195986)(97.60469757,328.45196777)
\curveto(98.01469305,328.46195983)(98.40469266,328.44695984)(98.77469757,328.40696777)
\curveto(99.15469191,328.36695992)(99.49469157,328.29196)(99.79469757,328.18196777)
\curveto(100.10469096,328.07196022)(100.3696907,327.92196037)(100.58969757,327.73196777)
\curveto(100.80969026,327.55196074)(100.97969009,327.31696097)(101.09969757,327.02696777)
\curveto(101.1696899,326.85696143)(101.20968986,326.66196163)(101.21969757,326.44196777)
\curveto(101.22968984,326.22196207)(101.23468983,325.99696229)(101.23469757,325.76696777)
\lineto(101.23469757,322.42196777)
\lineto(101.23469757,321.83696777)
\curveto(101.23468983,321.64696664)(101.25468981,321.47196682)(101.29469757,321.31196777)
\curveto(101.30468976,321.28196701)(101.30968976,321.24696704)(101.30969757,321.20696777)
\curveto(101.30968976,321.17696711)(101.31468975,321.14696714)(101.32469757,321.11696777)
\moveto(99.11969757,323.42696777)
\curveto(99.12969194,323.47696481)(99.13469193,323.53196476)(99.13469757,323.59196777)
\curveto(99.13469193,323.66196463)(99.12969194,323.72196457)(99.11969757,323.77196777)
\curveto(99.09969197,323.83196446)(99.08969198,323.8869644)(99.08969757,323.93696777)
\curveto(99.08969198,323.9869643)(99.069692,324.02696426)(99.02969757,324.05696777)
\curveto(98.97969209,324.09696419)(98.90469216,324.11696417)(98.80469757,324.11696777)
\curveto(98.7646923,324.10696418)(98.72969234,324.09696419)(98.69969757,324.08696777)
\curveto(98.6696924,324.0869642)(98.63469243,324.08196421)(98.59469757,324.07196777)
\curveto(98.52469254,324.05196424)(98.44969262,324.03696425)(98.36969757,324.02696777)
\curveto(98.28969278,324.01696427)(98.20969286,324.00196429)(98.12969757,323.98196777)
\curveto(98.09969297,323.97196432)(98.05469301,323.96696432)(97.99469757,323.96696777)
\curveto(97.8646932,323.93696435)(97.73469333,323.91696437)(97.60469757,323.90696777)
\curveto(97.47469359,323.89696439)(97.34969372,323.87196442)(97.22969757,323.83196777)
\curveto(97.14969392,323.81196448)(97.07469399,323.7919645)(97.00469757,323.77196777)
\curveto(96.93469413,323.76196453)(96.8646942,323.74196455)(96.79469757,323.71196777)
\curveto(96.58469448,323.62196467)(96.40469466,323.4869648)(96.25469757,323.30696777)
\curveto(96.11469495,323.12696516)(96.064695,322.87696541)(96.10469757,322.55696777)
\curveto(96.12469494,322.3869659)(96.17969489,322.24696604)(96.26969757,322.13696777)
\curveto(96.33969473,322.02696626)(96.44469462,321.93696635)(96.58469757,321.86696777)
\curveto(96.72469434,321.80696648)(96.87469419,321.76196653)(97.03469757,321.73196777)
\curveto(97.20469386,321.70196659)(97.37969369,321.6919666)(97.55969757,321.70196777)
\curveto(97.74969332,321.72196657)(97.92469314,321.75696653)(98.08469757,321.80696777)
\curveto(98.34469272,321.8869664)(98.54969252,322.01196628)(98.69969757,322.18196777)
\curveto(98.84969222,322.36196593)(98.9646921,322.58196571)(99.04469757,322.84196777)
\curveto(99.064692,322.91196538)(99.07469199,322.98196531)(99.07469757,323.05196777)
\curveto(99.08469198,323.13196516)(99.09969197,323.21196508)(99.11969757,323.29196777)
\lineto(99.11969757,323.42696777)
}
}
{
\newrgbcolor{curcolor}{0 0 0}
\pscustom[linestyle=none,fillstyle=solid,fillcolor=curcolor]
{
\newpath
\moveto(105.85797882,331.21196777)
\curveto(105.94797498,331.21195708)(106.04797488,331.21195708)(106.15797882,331.21196777)
\curveto(106.27797465,331.21195708)(106.39297454,331.20695708)(106.50297882,331.19696777)
\curveto(106.62297431,331.1869571)(106.7279742,331.16695712)(106.81797882,331.13696777)
\curveto(106.90797402,331.11695717)(106.96797396,331.08195721)(106.99797882,331.03196777)
\curveto(107.05797387,330.95195734)(107.08797384,330.83695745)(107.08797882,330.68696777)
\lineto(107.08797882,330.28196777)
\curveto(107.08797384,330.18195811)(107.08297385,330.08195821)(107.07297882,329.98196777)
\curveto(107.07297386,329.88195841)(107.05297388,329.80695848)(107.01297882,329.75696777)
\curveto(106.97297396,329.69695859)(106.92297401,329.65695863)(106.86297882,329.63696777)
\curveto(106.80297413,329.62695866)(106.7329742,329.62195867)(106.65297882,329.62196777)
\lineto(106.42797882,329.62196777)
\curveto(106.35797457,329.63195866)(106.28797464,329.63195866)(106.21797882,329.62196777)
\curveto(106.03797489,329.58195871)(105.89797503,329.53195876)(105.79797882,329.47196777)
\curveto(105.69797523,329.42195887)(105.61797531,329.31195898)(105.55797882,329.14196777)
\curveto(105.53797539,329.11195918)(105.5279754,329.08195921)(105.52797882,329.05196777)
\curveto(105.53797539,329.03195926)(105.53797539,329.00695928)(105.52797882,328.97696777)
\curveto(105.51797541,328.93695935)(105.50797542,328.87695941)(105.49797882,328.79696777)
\curveto(105.48797544,328.71695957)(105.48797544,328.65195964)(105.49797882,328.60196777)
\curveto(105.51797541,328.53195976)(105.54297539,328.47195982)(105.57297882,328.42196777)
\curveto(105.60297533,328.37195992)(105.64797528,328.33195996)(105.70797882,328.30196777)
\curveto(105.80797512,328.25196004)(105.927975,328.23696005)(106.06797882,328.25696777)
\curveto(106.20797472,328.27696001)(106.33797459,328.27696001)(106.45797882,328.25696777)
\curveto(106.50797442,328.24696004)(106.54797438,328.24196005)(106.57797882,328.24196777)
\curveto(106.61797431,328.25196004)(106.65797427,328.25196004)(106.69797882,328.24196777)
\curveto(106.78797414,328.20196009)(106.85297408,328.15696013)(106.89297882,328.10696777)
\curveto(106.91297402,328.07696021)(106.927974,328.02696026)(106.93797882,327.95696777)
\curveto(106.94797398,327.89696039)(106.95797397,327.82696046)(106.96797882,327.74696777)
\curveto(106.97797395,327.67696061)(106.97797395,327.60196069)(106.96797882,327.52196777)
\curveto(106.96797396,327.45196084)(106.96297397,327.39696089)(106.95297882,327.35696777)
\curveto(106.94297399,327.31696097)(106.94297399,327.27696101)(106.95297882,327.23696777)
\curveto(106.96297397,327.20696108)(106.95797397,327.17196112)(106.93797882,327.13196777)
\curveto(106.91797401,327.01196128)(106.85797407,326.93696135)(106.75797882,326.90696777)
\curveto(106.67797425,326.86696142)(106.58297435,326.84696144)(106.47297882,326.84696777)
\curveto(106.36297457,326.85696143)(106.25297468,326.86196143)(106.14297882,326.86196777)
\lineto(106.03797882,326.86196777)
\curveto(105.99797493,326.86196143)(105.96297497,326.85696143)(105.93297882,326.84696777)
\lineto(105.81297882,326.84696777)
\curveto(105.64297529,326.80696148)(105.53797539,326.69696159)(105.49797882,326.51696777)
\curveto(105.47797545,326.45696183)(105.47297546,326.3869619)(105.48297882,326.30696777)
\curveto(105.49297544,326.22696206)(105.49797543,326.14696214)(105.49797882,326.06696777)
\lineto(105.49797882,325.15196777)
\lineto(105.49797882,322.22696777)
\lineto(105.49797882,321.52196777)
\lineto(105.49797882,321.32696777)
\curveto(105.50797542,321.26696702)(105.50297543,321.21196708)(105.48297882,321.16196777)
\lineto(105.48297882,320.99696777)
\curveto(105.48297545,320.83696745)(105.45797547,320.72196757)(105.40797882,320.65196777)
\curveto(105.38797554,320.62196767)(105.35297558,320.59696769)(105.30297882,320.57696777)
\curveto(105.25297568,320.56696772)(105.20297573,320.55196774)(105.15297882,320.53196777)
\lineto(105.07797882,320.53196777)
\curveto(105.0279759,320.52196777)(104.97297596,320.51696777)(104.91297882,320.51696777)
\curveto(104.85297608,320.52696776)(104.79797613,320.53196776)(104.74797882,320.53196777)
\lineto(104.08797882,320.53196777)
\curveto(104.01797691,320.53196776)(103.94297699,320.52696776)(103.86297882,320.51696777)
\curveto(103.79297714,320.51696777)(103.7329772,320.52696776)(103.68297882,320.54696777)
\curveto(103.56297737,320.57696771)(103.48297745,320.62696766)(103.44297882,320.69696777)
\curveto(103.41297752,320.74696754)(103.39297754,320.81196748)(103.38297882,320.89196777)
\lineto(103.38297882,321.13196777)
\lineto(103.38297882,321.91196777)
\lineto(103.38297882,326.11196777)
\curveto(103.38297755,326.28196201)(103.37297756,326.42696186)(103.35297882,326.54696777)
\curveto(103.3329776,326.67696161)(103.26297767,326.76696152)(103.14297882,326.81696777)
\curveto(103.0329779,326.86696142)(102.89797803,326.87696141)(102.73797882,326.84696777)
\curveto(102.57797835,326.82696146)(102.44297849,326.84196145)(102.33297882,326.89196777)
\curveto(102.22297871,326.94196135)(102.15297878,327.02696126)(102.12297882,327.14696777)
\curveto(102.10297883,327.19696109)(102.09797883,327.25696103)(102.10797882,327.32696777)
\lineto(102.10797882,327.53696777)
\curveto(102.10797882,327.71696057)(102.11797881,327.86696042)(102.13797882,327.98696777)
\curveto(102.15797877,328.10696018)(102.24297869,328.1919601)(102.39297882,328.24196777)
\curveto(102.47297846,328.26196003)(102.55797837,328.27196002)(102.64797882,328.27196777)
\lineto(102.90297882,328.27196777)
\curveto(102.99297794,328.27196002)(103.07297786,328.27696001)(103.14297882,328.28696777)
\curveto(103.21297772,328.30695998)(103.26797766,328.34695994)(103.30797882,328.40696777)
\curveto(103.37797755,328.50695978)(103.40297753,328.63195966)(103.38297882,328.78196777)
\curveto(103.37297756,328.94195935)(103.38297755,329.0919592)(103.41297882,329.23196777)
\curveto(103.42297751,329.27195902)(103.4279775,329.31195898)(103.42797882,329.35196777)
\curveto(103.43797749,329.3919589)(103.44797748,329.43695885)(103.45797882,329.48696777)
\curveto(103.49797743,329.62695866)(103.53797739,329.75195854)(103.57797882,329.86196777)
\curveto(103.61797731,329.98195831)(103.67297726,330.0919582)(103.74297882,330.19196777)
\curveto(103.88297705,330.43195786)(104.06797686,330.62195767)(104.29797882,330.76196777)
\curveto(104.5279764,330.91195738)(104.78797614,331.02695726)(105.07797882,331.10696777)
\curveto(105.15797577,331.13695715)(105.24297569,331.15195714)(105.33297882,331.15196777)
\curveto(105.42297551,331.16195713)(105.51297542,331.17695711)(105.60297882,331.19696777)
\curveto(105.6329753,331.20695708)(105.67797525,331.20695708)(105.73797882,331.19696777)
\curveto(105.79797513,331.1869571)(105.83797509,331.1919571)(105.85797882,331.21196777)
}
}
{
\newrgbcolor{curcolor}{0 0 0}
\pscustom[linestyle=none,fillstyle=solid,fillcolor=curcolor]
{
\newpath
\moveto(110.10274445,327.62696777)
\curveto(110.10274146,327.72696056)(110.09774147,327.82196047)(110.08774445,327.91196777)
\curveto(110.08774148,328.00196029)(110.0677415,328.07196022)(110.02774445,328.12196777)
\curveto(109.99774157,328.17196012)(109.95774161,328.20196009)(109.90774445,328.21196777)
\curveto(109.85774171,328.22196007)(109.80274176,328.23696005)(109.74274445,328.25696777)
\lineto(109.62274445,328.25696777)
\curveto(109.57274199,328.26696002)(109.50774206,328.26696002)(109.42774445,328.25696777)
\lineto(109.23274445,328.25696777)
\lineto(108.48274445,328.25696777)
\curveto(108.4627431,328.24696004)(108.43274313,328.24196005)(108.39274445,328.24196777)
\curveto(108.3627432,328.25196004)(108.33774323,328.25196004)(108.31774445,328.24196777)
\curveto(108.20774336,328.22196007)(108.12274344,328.17696011)(108.06274445,328.10696777)
\curveto(108.02274354,328.04696024)(108.00274356,327.96696032)(108.00274445,327.86696777)
\lineto(108.00274445,327.56696777)
\lineto(108.00274445,321.23696777)
\lineto(108.00274445,320.89196777)
\curveto(108.01274355,320.7919675)(108.04774352,320.70696758)(108.10774445,320.63696777)
\curveto(108.14774342,320.5869677)(108.20774336,320.55696773)(108.28774445,320.54696777)
\curveto(108.37774319,320.53696775)(108.4677431,320.53196776)(108.55774445,320.53196777)
\lineto(109.39774445,320.53196777)
\curveto(109.47774209,320.53196776)(109.55274201,320.52696776)(109.62274445,320.51696777)
\curveto(109.69274187,320.51696777)(109.75774181,320.52696776)(109.81774445,320.54696777)
\curveto(109.98774158,320.59696769)(110.07774149,320.6919676)(110.08774445,320.83196777)
\curveto(110.09774147,320.97196732)(110.10274146,321.14196715)(110.10274445,321.34196777)
\lineto(110.10274445,327.62696777)
\moveto(109.68274445,331.33196777)
\lineto(110.74774445,331.33196777)
\curveto(110.82774074,331.33195696)(110.92274064,331.33195696)(111.03274445,331.33196777)
\curveto(111.14274042,331.33195696)(111.22274034,331.31695697)(111.27274445,331.28696777)
\curveto(111.29274027,331.27695701)(111.30274026,331.26195703)(111.30274445,331.24196777)
\curveto(111.31274025,331.23195706)(111.32774024,331.22195707)(111.34774445,331.21196777)
\curveto(111.35774021,331.0919572)(111.30774026,330.9869573)(111.19774445,330.89696777)
\curveto(111.09774047,330.80695748)(111.01274055,330.72695756)(110.94274445,330.65696777)
\curveto(110.8627407,330.5869577)(110.78274078,330.51195778)(110.70274445,330.43196777)
\curveto(110.63274093,330.36195793)(110.55774101,330.29695799)(110.47774445,330.23696777)
\curveto(110.43774113,330.20695808)(110.40274116,330.17195812)(110.37274445,330.13196777)
\curveto(110.35274121,330.10195819)(110.32274124,330.07695821)(110.28274445,330.05696777)
\curveto(110.2627413,330.02695826)(110.23774133,330.00195829)(110.20774445,329.98196777)
\lineto(110.05774445,329.83196777)
\lineto(109.90774445,329.71196777)
\lineto(109.86274445,329.66696777)
\curveto(109.8627417,329.65695863)(109.85274171,329.64195865)(109.83274445,329.62196777)
\curveto(109.75274181,329.56195873)(109.67274189,329.49695879)(109.59274445,329.42696777)
\curveto(109.52274204,329.35695893)(109.43274213,329.30195899)(109.32274445,329.26196777)
\curveto(109.28274228,329.25195904)(109.24274232,329.24695904)(109.20274445,329.24696777)
\curveto(109.17274239,329.24695904)(109.13274243,329.24195905)(109.08274445,329.23196777)
\curveto(109.05274251,329.22195907)(109.01274255,329.21695907)(108.96274445,329.21696777)
\curveto(108.91274265,329.22695906)(108.8677427,329.23195906)(108.82774445,329.23196777)
\lineto(108.48274445,329.23196777)
\curveto(108.3627432,329.23195906)(108.27274329,329.25695903)(108.21274445,329.30696777)
\curveto(108.15274341,329.34695894)(108.13774343,329.41695887)(108.16774445,329.51696777)
\curveto(108.18774338,329.59695869)(108.22274334,329.66695862)(108.27274445,329.72696777)
\curveto(108.32274324,329.79695849)(108.3677432,329.86695842)(108.40774445,329.93696777)
\curveto(108.50774306,330.07695821)(108.60274296,330.21195808)(108.69274445,330.34196777)
\curveto(108.78274278,330.47195782)(108.87274269,330.60695768)(108.96274445,330.74696777)
\curveto(109.01274255,330.82695746)(109.0627425,330.91195738)(109.11274445,331.00196777)
\curveto(109.17274239,331.0919572)(109.23774233,331.16195713)(109.30774445,331.21196777)
\curveto(109.34774222,331.24195705)(109.41774215,331.27695701)(109.51774445,331.31696777)
\curveto(109.53774203,331.32695696)(109.562742,331.32695696)(109.59274445,331.31696777)
\curveto(109.63274193,331.31695697)(109.6627419,331.32195697)(109.68274445,331.33196777)
}
}
{
\newrgbcolor{curcolor}{0 0 0}
\pscustom[linestyle=none,fillstyle=solid,fillcolor=curcolor]
{
\newpath
\moveto(118.81493195,321.11696777)
\curveto(118.8349241,321.00696728)(118.84492409,320.89696739)(118.84493195,320.78696777)
\curveto(118.85492408,320.67696761)(118.80492413,320.60196769)(118.69493195,320.56196777)
\curveto(118.6349243,320.53196776)(118.56492437,320.51696777)(118.48493195,320.51696777)
\lineto(118.24493195,320.51696777)
\lineto(117.43493195,320.51696777)
\lineto(117.16493195,320.51696777)
\curveto(117.08492585,320.52696776)(117.01992591,320.55196774)(116.96993195,320.59196777)
\curveto(116.89992603,320.63196766)(116.84492609,320.6869676)(116.80493195,320.75696777)
\curveto(116.77492616,320.83696745)(116.7299262,320.90196739)(116.66993195,320.95196777)
\curveto(116.64992628,320.97196732)(116.62492631,320.9869673)(116.59493195,320.99696777)
\curveto(116.56492637,321.01696727)(116.52492641,321.02196727)(116.47493195,321.01196777)
\curveto(116.42492651,320.9919673)(116.37492656,320.96696732)(116.32493195,320.93696777)
\curveto(116.28492665,320.90696738)(116.23992669,320.88196741)(116.18993195,320.86196777)
\curveto(116.13992679,320.82196747)(116.08492685,320.7869675)(116.02493195,320.75696777)
\lineto(115.84493195,320.66696777)
\curveto(115.71492722,320.60696768)(115.57992735,320.55696773)(115.43993195,320.51696777)
\curveto(115.29992763,320.4869678)(115.15492778,320.45196784)(115.00493195,320.41196777)
\curveto(114.934928,320.3919679)(114.86492807,320.38196791)(114.79493195,320.38196777)
\curveto(114.7349282,320.37196792)(114.66992826,320.36196793)(114.59993195,320.35196777)
\lineto(114.50993195,320.35196777)
\curveto(114.47992845,320.34196795)(114.44992848,320.33696795)(114.41993195,320.33696777)
\lineto(114.25493195,320.33696777)
\curveto(114.15492878,320.31696797)(114.05492888,320.31696797)(113.95493195,320.33696777)
\lineto(113.81993195,320.33696777)
\curveto(113.74992918,320.35696793)(113.67992925,320.36696792)(113.60993195,320.36696777)
\curveto(113.54992938,320.35696793)(113.48992944,320.36196793)(113.42993195,320.38196777)
\curveto(113.3299296,320.40196789)(113.2349297,320.42196787)(113.14493195,320.44196777)
\curveto(113.05492988,320.45196784)(112.96992996,320.47696781)(112.88993195,320.51696777)
\curveto(112.59993033,320.62696766)(112.34993058,320.76696752)(112.13993195,320.93696777)
\curveto(111.93993099,321.11696717)(111.77993115,321.35196694)(111.65993195,321.64196777)
\curveto(111.6299313,321.71196658)(111.59993133,321.7869665)(111.56993195,321.86696777)
\curveto(111.54993138,321.94696634)(111.5299314,322.03196626)(111.50993195,322.12196777)
\curveto(111.48993144,322.17196612)(111.47993145,322.22196607)(111.47993195,322.27196777)
\curveto(111.48993144,322.32196597)(111.48993144,322.37196592)(111.47993195,322.42196777)
\curveto(111.46993146,322.45196584)(111.45993147,322.51196578)(111.44993195,322.60196777)
\curveto(111.44993148,322.70196559)(111.45493148,322.77196552)(111.46493195,322.81196777)
\curveto(111.48493145,322.91196538)(111.49493144,322.99696529)(111.49493195,323.06696777)
\lineto(111.58493195,323.39696777)
\curveto(111.61493132,323.51696477)(111.65493128,323.62196467)(111.70493195,323.71196777)
\curveto(111.87493106,324.00196429)(112.06993086,324.22196407)(112.28993195,324.37196777)
\curveto(112.50993042,324.52196377)(112.78993014,324.65196364)(113.12993195,324.76196777)
\curveto(113.25992967,324.81196348)(113.39492954,324.84696344)(113.53493195,324.86696777)
\curveto(113.67492926,324.8869634)(113.81492912,324.91196338)(113.95493195,324.94196777)
\curveto(114.0349289,324.96196333)(114.11992881,324.97196332)(114.20993195,324.97196777)
\curveto(114.29992863,324.98196331)(114.38992854,324.99696329)(114.47993195,325.01696777)
\curveto(114.54992838,325.03696325)(114.61992831,325.04196325)(114.68993195,325.03196777)
\curveto(114.75992817,325.03196326)(114.8349281,325.04196325)(114.91493195,325.06196777)
\curveto(114.98492795,325.08196321)(115.05492788,325.0919632)(115.12493195,325.09196777)
\curveto(115.19492774,325.0919632)(115.26992766,325.10196319)(115.34993195,325.12196777)
\curveto(115.55992737,325.17196312)(115.74992718,325.21196308)(115.91993195,325.24196777)
\curveto(116.09992683,325.28196301)(116.25992667,325.37196292)(116.39993195,325.51196777)
\curveto(116.48992644,325.60196269)(116.54992638,325.70196259)(116.57993195,325.81196777)
\curveto(116.58992634,325.84196245)(116.58992634,325.86696242)(116.57993195,325.88696777)
\curveto(116.57992635,325.90696238)(116.58492635,325.92696236)(116.59493195,325.94696777)
\curveto(116.60492633,325.96696232)(116.60992632,325.99696229)(116.60993195,326.03696777)
\lineto(116.60993195,326.12696777)
\lineto(116.57993195,326.24696777)
\curveto(116.57992635,326.286962)(116.57492636,326.32196197)(116.56493195,326.35196777)
\curveto(116.46492647,326.65196164)(116.25492668,326.85696143)(115.93493195,326.96696777)
\curveto(115.84492709,326.99696129)(115.7349272,327.01696127)(115.60493195,327.02696777)
\curveto(115.48492745,327.04696124)(115.35992757,327.05196124)(115.22993195,327.04196777)
\curveto(115.09992783,327.04196125)(114.97492796,327.03196126)(114.85493195,327.01196777)
\curveto(114.7349282,326.9919613)(114.6299283,326.96696132)(114.53993195,326.93696777)
\curveto(114.47992845,326.91696137)(114.41992851,326.8869614)(114.35993195,326.84696777)
\curveto(114.30992862,326.81696147)(114.25992867,326.78196151)(114.20993195,326.74196777)
\curveto(114.15992877,326.70196159)(114.10492883,326.64696164)(114.04493195,326.57696777)
\curveto(113.99492894,326.50696178)(113.95992897,326.44196185)(113.93993195,326.38196777)
\curveto(113.88992904,326.28196201)(113.84492909,326.1869621)(113.80493195,326.09696777)
\curveto(113.77492916,326.00696228)(113.70492923,325.94696234)(113.59493195,325.91696777)
\curveto(113.51492942,325.89696239)(113.4299295,325.8869624)(113.33993195,325.88696777)
\lineto(113.06993195,325.88696777)
\lineto(112.49993195,325.88696777)
\curveto(112.44993048,325.8869624)(112.39993053,325.88196241)(112.34993195,325.87196777)
\curveto(112.29993063,325.87196242)(112.25493068,325.87696241)(112.21493195,325.88696777)
\lineto(112.07993195,325.88696777)
\curveto(112.05993087,325.89696239)(112.0349309,325.90196239)(112.00493195,325.90196777)
\curveto(111.97493096,325.90196239)(111.94993098,325.91196238)(111.92993195,325.93196777)
\curveto(111.84993108,325.95196234)(111.79493114,326.01696227)(111.76493195,326.12696777)
\curveto(111.75493118,326.17696211)(111.75493118,326.22696206)(111.76493195,326.27696777)
\curveto(111.77493116,326.32696196)(111.78493115,326.37196192)(111.79493195,326.41196777)
\curveto(111.82493111,326.52196177)(111.85493108,326.62196167)(111.88493195,326.71196777)
\curveto(111.92493101,326.81196148)(111.96993096,326.90196139)(112.01993195,326.98196777)
\lineto(112.10993195,327.13196777)
\lineto(112.19993195,327.28196777)
\curveto(112.27993065,327.3919609)(112.37993055,327.49696079)(112.49993195,327.59696777)
\curveto(112.51993041,327.60696068)(112.54993038,327.63196066)(112.58993195,327.67196777)
\curveto(112.63993029,327.71196058)(112.68493025,327.74696054)(112.72493195,327.77696777)
\curveto(112.76493017,327.80696048)(112.80993012,327.83696045)(112.85993195,327.86696777)
\curveto(113.0299299,327.97696031)(113.20992972,328.06196023)(113.39993195,328.12196777)
\curveto(113.58992934,328.1919601)(113.78492915,328.25696003)(113.98493195,328.31696777)
\curveto(114.10492883,328.34695994)(114.2299287,328.36695992)(114.35993195,328.37696777)
\curveto(114.48992844,328.3869599)(114.61992831,328.40695988)(114.74993195,328.43696777)
\curveto(114.78992814,328.44695984)(114.84992808,328.44695984)(114.92993195,328.43696777)
\curveto(115.01992791,328.42695986)(115.07492786,328.43195986)(115.09493195,328.45196777)
\curveto(115.50492743,328.46195983)(115.89492704,328.44695984)(116.26493195,328.40696777)
\curveto(116.64492629,328.36695992)(116.98492595,328.29196)(117.28493195,328.18196777)
\curveto(117.59492534,328.07196022)(117.85992507,327.92196037)(118.07993195,327.73196777)
\curveto(118.29992463,327.55196074)(118.46992446,327.31696097)(118.58993195,327.02696777)
\curveto(118.65992427,326.85696143)(118.69992423,326.66196163)(118.70993195,326.44196777)
\curveto(118.71992421,326.22196207)(118.72492421,325.99696229)(118.72493195,325.76696777)
\lineto(118.72493195,322.42196777)
\lineto(118.72493195,321.83696777)
\curveto(118.72492421,321.64696664)(118.74492419,321.47196682)(118.78493195,321.31196777)
\curveto(118.79492414,321.28196701)(118.79992413,321.24696704)(118.79993195,321.20696777)
\curveto(118.79992413,321.17696711)(118.80492413,321.14696714)(118.81493195,321.11696777)
\moveto(116.60993195,323.42696777)
\curveto(116.61992631,323.47696481)(116.62492631,323.53196476)(116.62493195,323.59196777)
\curveto(116.62492631,323.66196463)(116.61992631,323.72196457)(116.60993195,323.77196777)
\curveto(116.58992634,323.83196446)(116.57992635,323.8869644)(116.57993195,323.93696777)
\curveto(116.57992635,323.9869643)(116.55992637,324.02696426)(116.51993195,324.05696777)
\curveto(116.46992646,324.09696419)(116.39492654,324.11696417)(116.29493195,324.11696777)
\curveto(116.25492668,324.10696418)(116.21992671,324.09696419)(116.18993195,324.08696777)
\curveto(116.15992677,324.0869642)(116.12492681,324.08196421)(116.08493195,324.07196777)
\curveto(116.01492692,324.05196424)(115.93992699,324.03696425)(115.85993195,324.02696777)
\curveto(115.77992715,324.01696427)(115.69992723,324.00196429)(115.61993195,323.98196777)
\curveto(115.58992734,323.97196432)(115.54492739,323.96696432)(115.48493195,323.96696777)
\curveto(115.35492758,323.93696435)(115.22492771,323.91696437)(115.09493195,323.90696777)
\curveto(114.96492797,323.89696439)(114.83992809,323.87196442)(114.71993195,323.83196777)
\curveto(114.63992829,323.81196448)(114.56492837,323.7919645)(114.49493195,323.77196777)
\curveto(114.42492851,323.76196453)(114.35492858,323.74196455)(114.28493195,323.71196777)
\curveto(114.07492886,323.62196467)(113.89492904,323.4869648)(113.74493195,323.30696777)
\curveto(113.60492933,323.12696516)(113.55492938,322.87696541)(113.59493195,322.55696777)
\curveto(113.61492932,322.3869659)(113.66992926,322.24696604)(113.75993195,322.13696777)
\curveto(113.8299291,322.02696626)(113.934929,321.93696635)(114.07493195,321.86696777)
\curveto(114.21492872,321.80696648)(114.36492857,321.76196653)(114.52493195,321.73196777)
\curveto(114.69492824,321.70196659)(114.86992806,321.6919666)(115.04993195,321.70196777)
\curveto(115.23992769,321.72196657)(115.41492752,321.75696653)(115.57493195,321.80696777)
\curveto(115.8349271,321.8869664)(116.03992689,322.01196628)(116.18993195,322.18196777)
\curveto(116.33992659,322.36196593)(116.45492648,322.58196571)(116.53493195,322.84196777)
\curveto(116.55492638,322.91196538)(116.56492637,322.98196531)(116.56493195,323.05196777)
\curveto(116.57492636,323.13196516)(116.58992634,323.21196508)(116.60993195,323.29196777)
\lineto(116.60993195,323.42696777)
}
}
{
\newrgbcolor{curcolor}{0 0 0}
\pscustom[linestyle=none,fillstyle=solid,fillcolor=curcolor]
{
}
}
{
\newrgbcolor{curcolor}{0 0 0}
\pscustom[linestyle=none,fillstyle=solid,fillcolor=curcolor]
{
\newpath
\moveto(132.42836945,324.47696777)
\curveto(132.43836077,324.41696387)(132.44336076,324.32696396)(132.44336945,324.20696777)
\curveto(132.44336076,324.0869642)(132.43336077,324.00196429)(132.41336945,323.95196777)
\lineto(132.41336945,323.75696777)
\curveto(132.38336082,323.64696464)(132.36336084,323.54196475)(132.35336945,323.44196777)
\curveto(132.35336085,323.34196495)(132.33836087,323.24196505)(132.30836945,323.14196777)
\curveto(132.28836092,323.05196524)(132.26836094,322.95696533)(132.24836945,322.85696777)
\curveto(132.22836098,322.76696552)(132.19836101,322.67696561)(132.15836945,322.58696777)
\curveto(132.08836112,322.41696587)(132.01836119,322.25696603)(131.94836945,322.10696777)
\curveto(131.87836133,321.96696632)(131.79836141,321.82696646)(131.70836945,321.68696777)
\curveto(131.64836156,321.59696669)(131.58336162,321.51196678)(131.51336945,321.43196777)
\curveto(131.45336175,321.36196693)(131.38336182,321.286967)(131.30336945,321.20696777)
\lineto(131.19836945,321.10196777)
\curveto(131.14836206,321.05196724)(131.09336211,321.00696728)(131.03336945,320.96696777)
\lineto(130.88336945,320.84696777)
\curveto(130.8033624,320.7869675)(130.71336249,320.73196756)(130.61336945,320.68196777)
\curveto(130.52336268,320.64196765)(130.42836278,320.59696769)(130.32836945,320.54696777)
\curveto(130.22836298,320.49696779)(130.12336308,320.46196783)(130.01336945,320.44196777)
\curveto(129.91336329,320.42196787)(129.8083634,320.40196789)(129.69836945,320.38196777)
\curveto(129.63836357,320.36196793)(129.57336363,320.35196794)(129.50336945,320.35196777)
\curveto(129.44336376,320.35196794)(129.37836383,320.34196795)(129.30836945,320.32196777)
\lineto(129.17336945,320.32196777)
\curveto(129.09336411,320.30196799)(129.01836419,320.30196799)(128.94836945,320.32196777)
\lineto(128.79836945,320.32196777)
\curveto(128.73836447,320.34196795)(128.67336453,320.35196794)(128.60336945,320.35196777)
\curveto(128.54336466,320.34196795)(128.48336472,320.34696794)(128.42336945,320.36696777)
\curveto(128.26336494,320.41696787)(128.1083651,320.46196783)(127.95836945,320.50196777)
\curveto(127.81836539,320.54196775)(127.68836552,320.60196769)(127.56836945,320.68196777)
\curveto(127.49836571,320.72196757)(127.43336577,320.76196753)(127.37336945,320.80196777)
\curveto(127.31336589,320.85196744)(127.24836596,320.90196739)(127.17836945,320.95196777)
\lineto(126.99836945,321.08696777)
\curveto(126.91836629,321.14696714)(126.84836636,321.15196714)(126.78836945,321.10196777)
\curveto(126.73836647,321.07196722)(126.71336649,321.03196726)(126.71336945,320.98196777)
\curveto(126.71336649,320.94196735)(126.7033665,320.8919674)(126.68336945,320.83196777)
\curveto(126.66336654,320.73196756)(126.65336655,320.61696767)(126.65336945,320.48696777)
\curveto(126.66336654,320.35696793)(126.66836654,320.23696805)(126.66836945,320.12696777)
\lineto(126.66836945,318.59696777)
\curveto(126.66836654,318.46696982)(126.66336654,318.34196995)(126.65336945,318.22196777)
\curveto(126.65336655,318.0919702)(126.62836658,317.9869703)(126.57836945,317.90696777)
\curveto(126.54836666,317.86697042)(126.49336671,317.83697045)(126.41336945,317.81696777)
\curveto(126.33336687,317.79697049)(126.24336696,317.7869705)(126.14336945,317.78696777)
\curveto(126.04336716,317.77697051)(125.94336726,317.77697051)(125.84336945,317.78696777)
\lineto(125.58836945,317.78696777)
\lineto(125.18336945,317.78696777)
\lineto(125.07836945,317.78696777)
\curveto(125.03836817,317.7869705)(125.0033682,317.7919705)(124.97336945,317.80196777)
\lineto(124.85336945,317.80196777)
\curveto(124.68336852,317.85197044)(124.59336861,317.95197034)(124.58336945,318.10196777)
\curveto(124.57336863,318.24197005)(124.56836864,318.41196988)(124.56836945,318.61196777)
\lineto(124.56836945,327.41696777)
\curveto(124.56836864,327.52696076)(124.56336864,327.64196065)(124.55336945,327.76196777)
\curveto(124.55336865,327.8919604)(124.57836863,327.9919603)(124.62836945,328.06196777)
\curveto(124.66836854,328.13196016)(124.72336848,328.17696011)(124.79336945,328.19696777)
\curveto(124.84336836,328.21696007)(124.9033683,328.22696006)(124.97336945,328.22696777)
\lineto(125.19836945,328.22696777)
\lineto(125.91836945,328.22696777)
\lineto(126.20336945,328.22696777)
\curveto(126.29336691,328.22696006)(126.36836684,328.20196009)(126.42836945,328.15196777)
\curveto(126.49836671,328.10196019)(126.53336667,328.03696025)(126.53336945,327.95696777)
\curveto(126.54336666,327.8869604)(126.56836664,327.81196048)(126.60836945,327.73196777)
\curveto(126.61836659,327.70196059)(126.62836658,327.67696061)(126.63836945,327.65696777)
\curveto(126.65836655,327.64696064)(126.67836653,327.63196066)(126.69836945,327.61196777)
\curveto(126.8083664,327.60196069)(126.89836631,327.63196066)(126.96836945,327.70196777)
\curveto(127.03836617,327.77196052)(127.1083661,327.83196046)(127.17836945,327.88196777)
\curveto(127.3083659,327.97196032)(127.44336576,328.05196024)(127.58336945,328.12196777)
\curveto(127.72336548,328.20196009)(127.87836533,328.26696002)(128.04836945,328.31696777)
\curveto(128.12836508,328.34695994)(128.21336499,328.36695992)(128.30336945,328.37696777)
\curveto(128.4033648,328.3869599)(128.49836471,328.40195989)(128.58836945,328.42196777)
\curveto(128.62836458,328.43195986)(128.66836454,328.43195986)(128.70836945,328.42196777)
\curveto(128.75836445,328.41195988)(128.79836441,328.41695987)(128.82836945,328.43696777)
\curveto(129.39836381,328.45695983)(129.87836333,328.37695991)(130.26836945,328.19696777)
\curveto(130.66836254,328.02696026)(131.0083622,327.80196049)(131.28836945,327.52196777)
\curveto(131.33836187,327.47196082)(131.38336182,327.42196087)(131.42336945,327.37196777)
\curveto(131.46336174,327.33196096)(131.5033617,327.286961)(131.54336945,327.23696777)
\curveto(131.61336159,327.14696114)(131.67336153,327.05696123)(131.72336945,326.96696777)
\curveto(131.78336142,326.87696141)(131.83836137,326.7869615)(131.88836945,326.69696777)
\curveto(131.9083613,326.67696161)(131.91836129,326.65196164)(131.91836945,326.62196777)
\curveto(131.92836128,326.5919617)(131.94336126,326.55696173)(131.96336945,326.51696777)
\curveto(132.02336118,326.41696187)(132.07836113,326.29696199)(132.12836945,326.15696777)
\curveto(132.14836106,326.09696219)(132.16836104,326.03196226)(132.18836945,325.96196777)
\curveto(132.208361,325.90196239)(132.22836098,325.83696245)(132.24836945,325.76696777)
\curveto(132.28836092,325.64696264)(132.31336089,325.52196277)(132.32336945,325.39196777)
\curveto(132.34336086,325.26196303)(132.36836084,325.12696316)(132.39836945,324.98696777)
\lineto(132.39836945,324.82196777)
\lineto(132.42836945,324.64196777)
\lineto(132.42836945,324.47696777)
\moveto(130.31336945,324.13196777)
\curveto(130.32336288,324.18196411)(130.32836288,324.24696404)(130.32836945,324.32696777)
\curveto(130.32836288,324.41696387)(130.32336288,324.4869638)(130.31336945,324.53696777)
\lineto(130.31336945,324.67196777)
\curveto(130.29336291,324.73196356)(130.28336292,324.79696349)(130.28336945,324.86696777)
\curveto(130.28336292,324.93696335)(130.27336293,325.00696328)(130.25336945,325.07696777)
\curveto(130.23336297,325.17696311)(130.21336299,325.27196302)(130.19336945,325.36196777)
\curveto(130.17336303,325.46196283)(130.14336306,325.55196274)(130.10336945,325.63196777)
\curveto(129.98336322,325.95196234)(129.82836338,326.20696208)(129.63836945,326.39696777)
\curveto(129.44836376,326.5869617)(129.17836403,326.72696156)(128.82836945,326.81696777)
\curveto(128.74836446,326.83696145)(128.65836455,326.84696144)(128.55836945,326.84696777)
\lineto(128.28836945,326.84696777)
\curveto(128.24836496,326.83696145)(128.21336499,326.83196146)(128.18336945,326.83196777)
\curveto(128.15336505,326.83196146)(128.11836509,326.82696146)(128.07836945,326.81696777)
\lineto(127.86836945,326.75696777)
\curveto(127.8083654,326.74696154)(127.74836546,326.72696156)(127.68836945,326.69696777)
\curveto(127.42836578,326.5869617)(127.22336598,326.41696187)(127.07336945,326.18696777)
\curveto(126.93336627,325.95696233)(126.81836639,325.70196259)(126.72836945,325.42196777)
\curveto(126.7083665,325.34196295)(126.69336651,325.25696303)(126.68336945,325.16696777)
\curveto(126.67336653,325.0869632)(126.65836655,325.00696328)(126.63836945,324.92696777)
\curveto(126.62836658,324.8869634)(126.62336658,324.82196347)(126.62336945,324.73196777)
\curveto(126.6033666,324.6919636)(126.59836661,324.64196365)(126.60836945,324.58196777)
\curveto(126.61836659,324.53196376)(126.61836659,324.48196381)(126.60836945,324.43196777)
\curveto(126.58836662,324.37196392)(126.58836662,324.31696397)(126.60836945,324.26696777)
\lineto(126.60836945,324.08696777)
\lineto(126.60836945,323.95196777)
\curveto(126.6083666,323.91196438)(126.61836659,323.87196442)(126.63836945,323.83196777)
\curveto(126.63836657,323.76196453)(126.64336656,323.70696458)(126.65336945,323.66696777)
\lineto(126.68336945,323.48696777)
\curveto(126.69336651,323.42696486)(126.7083665,323.36696492)(126.72836945,323.30696777)
\curveto(126.81836639,323.01696527)(126.92336628,322.77696551)(127.04336945,322.58696777)
\curveto(127.17336603,322.40696588)(127.35336585,322.24696604)(127.58336945,322.10696777)
\curveto(127.72336548,322.02696626)(127.88836532,321.96196633)(128.07836945,321.91196777)
\curveto(128.11836509,321.90196639)(128.15336505,321.89696639)(128.18336945,321.89696777)
\curveto(128.21336499,321.90696638)(128.24836496,321.90696638)(128.28836945,321.89696777)
\curveto(128.32836488,321.8869664)(128.38836482,321.87696641)(128.46836945,321.86696777)
\curveto(128.54836466,321.86696642)(128.61336459,321.87196642)(128.66336945,321.88196777)
\curveto(128.74336446,321.90196639)(128.82336438,321.91696637)(128.90336945,321.92696777)
\curveto(128.99336421,321.94696634)(129.07836413,321.97196632)(129.15836945,322.00196777)
\curveto(129.39836381,322.10196619)(129.59336361,322.24196605)(129.74336945,322.42196777)
\curveto(129.89336331,322.60196569)(130.01836319,322.81196548)(130.11836945,323.05196777)
\curveto(130.16836304,323.17196512)(130.203363,323.29696499)(130.22336945,323.42696777)
\curveto(130.24336296,323.55696473)(130.26836294,323.6919646)(130.29836945,323.83196777)
\lineto(130.29836945,323.98196777)
\curveto(130.3083629,324.03196426)(130.31336289,324.08196421)(130.31336945,324.13196777)
}
}
{
\newrgbcolor{curcolor}{0 0 0}
\pscustom[linestyle=none,fillstyle=solid,fillcolor=curcolor]
{
\newpath
\moveto(141.07329132,324.46196777)
\curveto(141.09328316,324.38196391)(141.09328316,324.291964)(141.07329132,324.19196777)
\curveto(141.0532832,324.0919642)(141.01828323,324.02696426)(140.96829132,323.99696777)
\curveto(140.91828333,323.95696433)(140.84328341,323.92696436)(140.74329132,323.90696777)
\curveto(140.6532836,323.89696439)(140.5482837,323.8869644)(140.42829132,323.87696777)
\lineto(140.08329132,323.87696777)
\curveto(139.97328428,323.8869644)(139.87328438,323.8919644)(139.78329132,323.89196777)
\lineto(136.12329132,323.89196777)
\lineto(135.91329132,323.89196777)
\curveto(135.8532884,323.8919644)(135.79828845,323.88196441)(135.74829132,323.86196777)
\curveto(135.66828858,323.82196447)(135.61828863,323.78196451)(135.59829132,323.74196777)
\curveto(135.57828867,323.72196457)(135.55828869,323.68196461)(135.53829132,323.62196777)
\curveto(135.51828873,323.57196472)(135.51328874,323.52196477)(135.52329132,323.47196777)
\curveto(135.54328871,323.41196488)(135.5532887,323.35196494)(135.55329132,323.29196777)
\curveto(135.56328869,323.24196505)(135.57828867,323.1869651)(135.59829132,323.12696777)
\curveto(135.67828857,322.8869654)(135.77328848,322.6869656)(135.88329132,322.52696777)
\curveto(136.00328825,322.37696591)(136.16328809,322.24196605)(136.36329132,322.12196777)
\curveto(136.44328781,322.07196622)(136.52328773,322.03696625)(136.60329132,322.01696777)
\curveto(136.69328756,322.00696628)(136.78328747,321.9869663)(136.87329132,321.95696777)
\curveto(136.9532873,321.93696635)(137.06328719,321.92196637)(137.20329132,321.91196777)
\curveto(137.34328691,321.90196639)(137.46328679,321.90696638)(137.56329132,321.92696777)
\lineto(137.69829132,321.92696777)
\curveto(137.79828645,321.94696634)(137.88828636,321.96696632)(137.96829132,321.98696777)
\curveto(138.05828619,322.01696627)(138.14328611,322.04696624)(138.22329132,322.07696777)
\curveto(138.32328593,322.12696616)(138.43328582,322.1919661)(138.55329132,322.27196777)
\curveto(138.68328557,322.35196594)(138.77828547,322.43196586)(138.83829132,322.51196777)
\curveto(138.88828536,322.58196571)(138.93828531,322.64696564)(138.98829132,322.70696777)
\curveto(139.0482852,322.77696551)(139.11828513,322.82696546)(139.19829132,322.85696777)
\curveto(139.29828495,322.90696538)(139.42328483,322.92696536)(139.57329132,322.91696777)
\lineto(140.00829132,322.91696777)
\lineto(140.18829132,322.91696777)
\curveto(140.25828399,322.92696536)(140.31828393,322.92196537)(140.36829132,322.90196777)
\lineto(140.51829132,322.90196777)
\curveto(140.61828363,322.88196541)(140.68828356,322.85696543)(140.72829132,322.82696777)
\curveto(140.76828348,322.80696548)(140.78828346,322.76196553)(140.78829132,322.69196777)
\curveto(140.79828345,322.62196567)(140.79328346,322.56196573)(140.77329132,322.51196777)
\curveto(140.72328353,322.37196592)(140.66828358,322.24696604)(140.60829132,322.13696777)
\curveto(140.5482837,322.02696626)(140.47828377,321.91696637)(140.39829132,321.80696777)
\curveto(140.17828407,321.47696681)(139.92828432,321.21196708)(139.64829132,321.01196777)
\curveto(139.36828488,320.81196748)(139.01828523,320.64196765)(138.59829132,320.50196777)
\curveto(138.48828576,320.46196783)(138.37828587,320.43696785)(138.26829132,320.42696777)
\curveto(138.15828609,320.41696787)(138.04328621,320.39696789)(137.92329132,320.36696777)
\curveto(137.88328637,320.35696793)(137.83828641,320.35696793)(137.78829132,320.36696777)
\curveto(137.7482865,320.36696792)(137.70828654,320.36196793)(137.66829132,320.35196777)
\lineto(137.50329132,320.35196777)
\curveto(137.4532868,320.33196796)(137.39328686,320.32696796)(137.32329132,320.33696777)
\curveto(137.26328699,320.33696795)(137.20828704,320.34196795)(137.15829132,320.35196777)
\curveto(137.07828717,320.36196793)(137.00828724,320.36196793)(136.94829132,320.35196777)
\curveto(136.88828736,320.34196795)(136.82328743,320.34696794)(136.75329132,320.36696777)
\curveto(136.70328755,320.3869679)(136.6482876,320.39696789)(136.58829132,320.39696777)
\curveto(136.52828772,320.39696789)(136.47328778,320.40696788)(136.42329132,320.42696777)
\curveto(136.31328794,320.44696784)(136.20328805,320.47196782)(136.09329132,320.50196777)
\curveto(135.98328827,320.52196777)(135.88328837,320.55696773)(135.79329132,320.60696777)
\curveto(135.68328857,320.64696764)(135.57828867,320.68196761)(135.47829132,320.71196777)
\curveto(135.38828886,320.75196754)(135.30328895,320.79696749)(135.22329132,320.84696777)
\curveto(134.90328935,321.04696724)(134.61828963,321.27696701)(134.36829132,321.53696777)
\curveto(134.11829013,321.80696648)(133.91329034,322.11696617)(133.75329132,322.46696777)
\curveto(133.70329055,322.57696571)(133.66329059,322.6869656)(133.63329132,322.79696777)
\curveto(133.60329065,322.91696537)(133.56329069,323.03696525)(133.51329132,323.15696777)
\curveto(133.50329075,323.19696509)(133.49829075,323.23196506)(133.49829132,323.26196777)
\curveto(133.49829075,323.30196499)(133.49329076,323.34196495)(133.48329132,323.38196777)
\curveto(133.44329081,323.50196479)(133.41829083,323.63196466)(133.40829132,323.77196777)
\lineto(133.37829132,324.19196777)
\curveto(133.37829087,324.24196405)(133.37329088,324.29696399)(133.36329132,324.35696777)
\curveto(133.36329089,324.41696387)(133.36829088,324.47196382)(133.37829132,324.52196777)
\lineto(133.37829132,324.70196777)
\lineto(133.42329132,325.06196777)
\curveto(133.46329079,325.23196306)(133.49829075,325.39696289)(133.52829132,325.55696777)
\curveto(133.55829069,325.71696257)(133.60329065,325.86696242)(133.66329132,326.00696777)
\curveto(134.09329016,327.04696124)(134.82328943,327.78196051)(135.85329132,328.21196777)
\curveto(135.99328826,328.27196002)(136.13328812,328.31195998)(136.27329132,328.33196777)
\curveto(136.42328783,328.36195993)(136.57828767,328.39695989)(136.73829132,328.43696777)
\curveto(136.81828743,328.44695984)(136.89328736,328.45195984)(136.96329132,328.45196777)
\curveto(137.03328722,328.45195984)(137.10828714,328.45695983)(137.18829132,328.46696777)
\curveto(137.69828655,328.47695981)(138.13328612,328.41695987)(138.49329132,328.28696777)
\curveto(138.86328539,328.16696012)(139.19328506,328.00696028)(139.48329132,327.80696777)
\curveto(139.57328468,327.74696054)(139.66328459,327.67696061)(139.75329132,327.59696777)
\curveto(139.84328441,327.52696076)(139.92328433,327.45196084)(139.99329132,327.37196777)
\curveto(140.02328423,327.32196097)(140.06328419,327.28196101)(140.11329132,327.25196777)
\curveto(140.19328406,327.14196115)(140.26828398,327.02696126)(140.33829132,326.90696777)
\curveto(140.40828384,326.79696149)(140.48328377,326.68196161)(140.56329132,326.56196777)
\curveto(140.61328364,326.47196182)(140.6532836,326.37696191)(140.68329132,326.27696777)
\curveto(140.72328353,326.1869621)(140.76328349,326.0869622)(140.80329132,325.97696777)
\curveto(140.8532834,325.84696244)(140.89328336,325.71196258)(140.92329132,325.57196777)
\curveto(140.9532833,325.43196286)(140.98828326,325.291963)(141.02829132,325.15196777)
\curveto(141.0482832,325.07196322)(141.0532832,324.98196331)(141.04329132,324.88196777)
\curveto(141.04328321,324.7919635)(141.0532832,324.70696358)(141.07329132,324.62696777)
\lineto(141.07329132,324.46196777)
\moveto(138.82329132,325.34696777)
\curveto(138.89328536,325.44696284)(138.89828535,325.56696272)(138.83829132,325.70696777)
\curveto(138.78828546,325.85696243)(138.7482855,325.96696232)(138.71829132,326.03696777)
\curveto(138.57828567,326.30696198)(138.39328586,326.51196178)(138.16329132,326.65196777)
\curveto(137.93328632,326.80196149)(137.61328664,326.88196141)(137.20329132,326.89196777)
\curveto(137.17328708,326.87196142)(137.13828711,326.86696142)(137.09829132,326.87696777)
\curveto(137.05828719,326.8869614)(137.02328723,326.8869614)(136.99329132,326.87696777)
\curveto(136.94328731,326.85696143)(136.88828736,326.84196145)(136.82829132,326.83196777)
\curveto(136.76828748,326.83196146)(136.71328754,326.82196147)(136.66329132,326.80196777)
\curveto(136.22328803,326.66196163)(135.89828835,326.3869619)(135.68829132,325.97696777)
\curveto(135.66828858,325.93696235)(135.64328861,325.88196241)(135.61329132,325.81196777)
\curveto(135.59328866,325.75196254)(135.57828867,325.6869626)(135.56829132,325.61696777)
\curveto(135.55828869,325.55696273)(135.55828869,325.49696279)(135.56829132,325.43696777)
\curveto(135.58828866,325.37696291)(135.62328863,325.32696296)(135.67329132,325.28696777)
\curveto(135.7532885,325.23696305)(135.86328839,325.21196308)(136.00329132,325.21196777)
\lineto(136.40829132,325.21196777)
\lineto(138.07329132,325.21196777)
\lineto(138.50829132,325.21196777)
\curveto(138.66828558,325.22196307)(138.77328548,325.26696302)(138.82329132,325.34696777)
}
}
{
\newrgbcolor{curcolor}{0 0 0}
\pscustom[linestyle=none,fillstyle=solid,fillcolor=curcolor]
{
\newpath
\moveto(146.74657257,328.45196777)
\curveto(146.85656726,328.45195984)(146.95156716,328.44195985)(147.03157257,328.42196777)
\curveto(147.12156699,328.40195989)(147.19156692,328.35695993)(147.24157257,328.28696777)
\curveto(147.30156681,328.20696008)(147.33156678,328.06696022)(147.33157257,327.86696777)
\lineto(147.33157257,327.35696777)
\lineto(147.33157257,326.98196777)
\curveto(147.34156677,326.84196145)(147.32656679,326.73196156)(147.28657257,326.65196777)
\curveto(147.24656687,326.58196171)(147.18656693,326.53696175)(147.10657257,326.51696777)
\curveto(147.03656708,326.49696179)(146.95156716,326.4869618)(146.85157257,326.48696777)
\curveto(146.76156735,326.4869618)(146.66156745,326.4919618)(146.55157257,326.50196777)
\curveto(146.45156766,326.51196178)(146.35656776,326.50696178)(146.26657257,326.48696777)
\curveto(146.19656792,326.46696182)(146.12656799,326.45196184)(146.05657257,326.44196777)
\curveto(145.98656813,326.44196185)(145.92156819,326.43196186)(145.86157257,326.41196777)
\curveto(145.70156841,326.36196193)(145.54156857,326.286962)(145.38157257,326.18696777)
\curveto(145.22156889,326.09696219)(145.09656902,325.9919623)(145.00657257,325.87196777)
\curveto(144.95656916,325.7919625)(144.90156921,325.70696258)(144.84157257,325.61696777)
\curveto(144.79156932,325.53696275)(144.74156937,325.45196284)(144.69157257,325.36196777)
\curveto(144.66156945,325.28196301)(144.63156948,325.19696309)(144.60157257,325.10696777)
\lineto(144.54157257,324.86696777)
\curveto(144.52156959,324.79696349)(144.5115696,324.72196357)(144.51157257,324.64196777)
\curveto(144.5115696,324.57196372)(144.50156961,324.50196379)(144.48157257,324.43196777)
\curveto(144.47156964,324.3919639)(144.46656965,324.35196394)(144.46657257,324.31196777)
\curveto(144.47656964,324.28196401)(144.47656964,324.25196404)(144.46657257,324.22196777)
\lineto(144.46657257,323.98196777)
\curveto(144.44656967,323.91196438)(144.44156967,323.83196446)(144.45157257,323.74196777)
\curveto(144.46156965,323.66196463)(144.46656965,323.58196471)(144.46657257,323.50196777)
\lineto(144.46657257,322.54196777)
\lineto(144.46657257,321.26696777)
\curveto(144.46656965,321.13696715)(144.46156965,321.01696727)(144.45157257,320.90696777)
\curveto(144.44156967,320.79696749)(144.4115697,320.70696758)(144.36157257,320.63696777)
\curveto(144.34156977,320.60696768)(144.30656981,320.58196771)(144.25657257,320.56196777)
\curveto(144.2165699,320.55196774)(144.17156994,320.54196775)(144.12157257,320.53196777)
\lineto(144.04657257,320.53196777)
\curveto(143.99657012,320.52196777)(143.94157017,320.51696777)(143.88157257,320.51696777)
\lineto(143.71657257,320.51696777)
\lineto(143.07157257,320.51696777)
\curveto(143.0115711,320.52696776)(142.94657117,320.53196776)(142.87657257,320.53196777)
\lineto(142.68157257,320.53196777)
\curveto(142.63157148,320.55196774)(142.58157153,320.56696772)(142.53157257,320.57696777)
\curveto(142.48157163,320.59696769)(142.44657167,320.63196766)(142.42657257,320.68196777)
\curveto(142.38657173,320.73196756)(142.36157175,320.80196749)(142.35157257,320.89196777)
\lineto(142.35157257,321.19196777)
\lineto(142.35157257,322.21196777)
\lineto(142.35157257,326.44196777)
\lineto(142.35157257,327.55196777)
\lineto(142.35157257,327.83696777)
\curveto(142.35157176,327.93696035)(142.37157174,328.01696027)(142.41157257,328.07696777)
\curveto(142.46157165,328.15696013)(142.53657158,328.20696008)(142.63657257,328.22696777)
\curveto(142.73657138,328.24696004)(142.85657126,328.25696003)(142.99657257,328.25696777)
\lineto(143.76157257,328.25696777)
\curveto(143.88157023,328.25696003)(143.98657013,328.24696004)(144.07657257,328.22696777)
\curveto(144.16656995,328.21696007)(144.23656988,328.17196012)(144.28657257,328.09196777)
\curveto(144.3165698,328.04196025)(144.33156978,327.97196032)(144.33157257,327.88196777)
\lineto(144.36157257,327.61196777)
\curveto(144.37156974,327.53196076)(144.38656973,327.45696083)(144.40657257,327.38696777)
\curveto(144.43656968,327.31696097)(144.48656963,327.28196101)(144.55657257,327.28196777)
\curveto(144.57656954,327.30196099)(144.59656952,327.31196098)(144.61657257,327.31196777)
\curveto(144.63656948,327.31196098)(144.65656946,327.32196097)(144.67657257,327.34196777)
\curveto(144.73656938,327.3919609)(144.78656933,327.44696084)(144.82657257,327.50696777)
\curveto(144.87656924,327.57696071)(144.93656918,327.63696065)(145.00657257,327.68696777)
\curveto(145.04656907,327.71696057)(145.08156903,327.74696054)(145.11157257,327.77696777)
\curveto(145.14156897,327.81696047)(145.17656894,327.85196044)(145.21657257,327.88196777)
\lineto(145.48657257,328.06196777)
\curveto(145.58656853,328.12196017)(145.68656843,328.17696011)(145.78657257,328.22696777)
\curveto(145.88656823,328.26696002)(145.98656813,328.30195999)(146.08657257,328.33196777)
\lineto(146.41657257,328.42196777)
\curveto(146.44656767,328.43195986)(146.50156761,328.43195986)(146.58157257,328.42196777)
\curveto(146.67156744,328.42195987)(146.72656739,328.43195986)(146.74657257,328.45196777)
}
}
{
\newrgbcolor{curcolor}{0 0 0}
\pscustom[linestyle=none,fillstyle=solid,fillcolor=curcolor]
{
\newpath
\moveto(151.1216507,328.46696777)
\curveto(151.8716462,328.4869598)(152.52164555,328.40195989)(153.0716507,328.21196777)
\curveto(153.63164444,328.03196026)(154.05664401,327.71696057)(154.3466507,327.26696777)
\curveto(154.41664365,327.15696113)(154.47664359,327.04196125)(154.5266507,326.92196777)
\curveto(154.58664348,326.81196148)(154.63664343,326.6869616)(154.6766507,326.54696777)
\curveto(154.69664337,326.4869618)(154.70664336,326.42196187)(154.7066507,326.35196777)
\curveto(154.70664336,326.28196201)(154.69664337,326.22196207)(154.6766507,326.17196777)
\curveto(154.63664343,326.11196218)(154.58164349,326.07196222)(154.5116507,326.05196777)
\curveto(154.46164361,326.03196226)(154.40164367,326.02196227)(154.3316507,326.02196777)
\lineto(154.1216507,326.02196777)
\lineto(153.4616507,326.02196777)
\curveto(153.39164468,326.02196227)(153.32164475,326.01696227)(153.2516507,326.00696777)
\curveto(153.18164489,326.00696228)(153.11664495,326.01696227)(153.0566507,326.03696777)
\curveto(152.95664511,326.05696223)(152.88164519,326.09696219)(152.8316507,326.15696777)
\curveto(152.78164529,326.21696207)(152.73664533,326.27696201)(152.6966507,326.33696777)
\lineto(152.5766507,326.54696777)
\curveto(152.54664552,326.62696166)(152.49664557,326.6919616)(152.4266507,326.74196777)
\curveto(152.32664574,326.82196147)(152.22664584,326.88196141)(152.1266507,326.92196777)
\curveto(152.03664603,326.96196133)(151.92164615,326.99696129)(151.7816507,327.02696777)
\curveto(151.71164636,327.04696124)(151.60664646,327.06196123)(151.4666507,327.07196777)
\curveto(151.33664673,327.08196121)(151.23664683,327.07696121)(151.1666507,327.05696777)
\lineto(151.0616507,327.05696777)
\lineto(150.9116507,327.02696777)
\curveto(150.8716472,327.02696126)(150.82664724,327.02196127)(150.7766507,327.01196777)
\curveto(150.60664746,326.96196133)(150.4666476,326.8919614)(150.3566507,326.80196777)
\curveto(150.25664781,326.72196157)(150.18664788,326.59696169)(150.1466507,326.42696777)
\curveto(150.12664794,326.35696193)(150.12664794,326.291962)(150.1466507,326.23196777)
\curveto(150.1666479,326.17196212)(150.18664788,326.12196217)(150.2066507,326.08196777)
\curveto(150.27664779,325.96196233)(150.35664771,325.86696242)(150.4466507,325.79696777)
\curveto(150.54664752,325.72696256)(150.66164741,325.66696262)(150.7916507,325.61696777)
\curveto(150.98164709,325.53696275)(151.18664688,325.46696282)(151.4066507,325.40696777)
\lineto(152.0966507,325.25696777)
\curveto(152.33664573,325.21696307)(152.5666455,325.16696312)(152.7866507,325.10696777)
\curveto(153.01664505,325.05696323)(153.23164484,324.9919633)(153.4316507,324.91196777)
\curveto(153.52164455,324.87196342)(153.60664446,324.83696345)(153.6866507,324.80696777)
\curveto(153.77664429,324.7869635)(153.86164421,324.75196354)(153.9416507,324.70196777)
\curveto(154.13164394,324.58196371)(154.30164377,324.45196384)(154.4516507,324.31196777)
\curveto(154.61164346,324.17196412)(154.73664333,323.99696429)(154.8266507,323.78696777)
\curveto(154.85664321,323.71696457)(154.88164319,323.64696464)(154.9016507,323.57696777)
\curveto(154.92164315,323.50696478)(154.94164313,323.43196486)(154.9616507,323.35196777)
\curveto(154.9716431,323.291965)(154.97664309,323.19696509)(154.9766507,323.06696777)
\curveto(154.98664308,322.94696534)(154.98664308,322.85196544)(154.9766507,322.78196777)
\lineto(154.9766507,322.70696777)
\curveto(154.95664311,322.64696564)(154.94164313,322.5869657)(154.9316507,322.52696777)
\curveto(154.93164314,322.47696581)(154.92664314,322.42696586)(154.9166507,322.37696777)
\curveto(154.84664322,322.07696621)(154.73664333,321.81196648)(154.5866507,321.58196777)
\curveto(154.42664364,321.34196695)(154.23164384,321.14696714)(154.0016507,320.99696777)
\curveto(153.7716443,320.84696744)(153.51164456,320.71696757)(153.2216507,320.60696777)
\curveto(153.11164496,320.55696773)(152.99164508,320.52196777)(152.8616507,320.50196777)
\curveto(152.74164533,320.48196781)(152.62164545,320.45696783)(152.5016507,320.42696777)
\curveto(152.41164566,320.40696788)(152.31664575,320.39696789)(152.2166507,320.39696777)
\curveto(152.12664594,320.3869679)(152.03664603,320.37196792)(151.9466507,320.35196777)
\lineto(151.6766507,320.35196777)
\curveto(151.61664645,320.33196796)(151.51164656,320.32196797)(151.3616507,320.32196777)
\curveto(151.22164685,320.32196797)(151.12164695,320.33196796)(151.0616507,320.35196777)
\curveto(151.03164704,320.35196794)(150.99664707,320.35696793)(150.9566507,320.36696777)
\lineto(150.8516507,320.36696777)
\curveto(150.73164734,320.3869679)(150.61164746,320.40196789)(150.4916507,320.41196777)
\curveto(150.3716477,320.42196787)(150.25664781,320.44196785)(150.1466507,320.47196777)
\curveto(149.75664831,320.58196771)(149.41164866,320.70696758)(149.1116507,320.84696777)
\curveto(148.81164926,320.99696729)(148.55664951,321.21696707)(148.3466507,321.50696777)
\curveto(148.20664986,321.69696659)(148.08664998,321.91696637)(147.9866507,322.16696777)
\curveto(147.9666501,322.22696606)(147.94665012,322.30696598)(147.9266507,322.40696777)
\curveto(147.90665016,322.45696583)(147.89165018,322.52696576)(147.8816507,322.61696777)
\curveto(147.8716502,322.70696558)(147.87665019,322.78196551)(147.8966507,322.84196777)
\curveto(147.92665014,322.91196538)(147.97665009,322.96196533)(148.0466507,322.99196777)
\curveto(148.09664997,323.01196528)(148.15664991,323.02196527)(148.2266507,323.02196777)
\lineto(148.4516507,323.02196777)
\lineto(149.1566507,323.02196777)
\lineto(149.3966507,323.02196777)
\curveto(149.47664859,323.02196527)(149.54664852,323.01196528)(149.6066507,322.99196777)
\curveto(149.71664835,322.95196534)(149.78664828,322.8869654)(149.8166507,322.79696777)
\curveto(149.85664821,322.70696558)(149.90164817,322.61196568)(149.9516507,322.51196777)
\curveto(149.9716481,322.46196583)(150.00664806,322.39696589)(150.0566507,322.31696777)
\curveto(150.11664795,322.23696605)(150.1666479,322.1869661)(150.2066507,322.16696777)
\curveto(150.32664774,322.06696622)(150.44164763,321.9869663)(150.5516507,321.92696777)
\curveto(150.66164741,321.87696641)(150.80164727,321.82696646)(150.9716507,321.77696777)
\curveto(151.02164705,321.75696653)(151.071647,321.74696654)(151.1216507,321.74696777)
\curveto(151.1716469,321.75696653)(151.22164685,321.75696653)(151.2716507,321.74696777)
\curveto(151.35164672,321.72696656)(151.43664663,321.71696657)(151.5266507,321.71696777)
\curveto(151.62664644,321.72696656)(151.71164636,321.74196655)(151.7816507,321.76196777)
\curveto(151.83164624,321.77196652)(151.87664619,321.77696651)(151.9166507,321.77696777)
\curveto(151.9666461,321.77696651)(152.01664605,321.7869665)(152.0666507,321.80696777)
\curveto(152.20664586,321.85696643)(152.33164574,321.91696637)(152.4416507,321.98696777)
\curveto(152.56164551,322.05696623)(152.65664541,322.14696614)(152.7266507,322.25696777)
\curveto(152.77664529,322.33696595)(152.81664525,322.46196583)(152.8466507,322.63196777)
\curveto(152.8666452,322.70196559)(152.8666452,322.76696552)(152.8466507,322.82696777)
\curveto(152.82664524,322.8869654)(152.80664526,322.93696535)(152.7866507,322.97696777)
\curveto(152.71664535,323.11696517)(152.62664544,323.22196507)(152.5166507,323.29196777)
\curveto(152.41664565,323.36196493)(152.29664577,323.42696486)(152.1566507,323.48696777)
\curveto(151.9666461,323.56696472)(151.7666463,323.63196466)(151.5566507,323.68196777)
\curveto(151.34664672,323.73196456)(151.13664693,323.7869645)(150.9266507,323.84696777)
\curveto(150.84664722,323.86696442)(150.76164731,323.88196441)(150.6716507,323.89196777)
\curveto(150.59164748,323.90196439)(150.51164756,323.91696437)(150.4316507,323.93696777)
\curveto(150.11164796,324.02696426)(149.80664826,324.11196418)(149.5166507,324.19196777)
\curveto(149.22664884,324.28196401)(148.96164911,324.41196388)(148.7216507,324.58196777)
\curveto(148.44164963,324.78196351)(148.23664983,325.05196324)(148.1066507,325.39196777)
\curveto(148.08664998,325.46196283)(148.06665,325.55696273)(148.0466507,325.67696777)
\curveto(148.02665004,325.74696254)(148.01165006,325.83196246)(148.0016507,325.93196777)
\curveto(147.99165008,326.03196226)(147.99665007,326.12196217)(148.0166507,326.20196777)
\curveto(148.03665003,326.25196204)(148.04165003,326.291962)(148.0316507,326.32196777)
\curveto(148.02165005,326.36196193)(148.02665004,326.40696188)(148.0466507,326.45696777)
\curveto(148.06665,326.56696172)(148.08664998,326.66696162)(148.1066507,326.75696777)
\curveto(148.13664993,326.85696143)(148.1716499,326.95196134)(148.2116507,327.04196777)
\curveto(148.34164973,327.33196096)(148.52164955,327.56696072)(148.7516507,327.74696777)
\curveto(148.98164909,327.92696036)(149.24164883,328.07196022)(149.5316507,328.18196777)
\curveto(149.64164843,328.23196006)(149.75664831,328.26696002)(149.8766507,328.28696777)
\curveto(149.99664807,328.31695997)(150.12164795,328.34695994)(150.2516507,328.37696777)
\curveto(150.31164776,328.39695989)(150.3716477,328.40695988)(150.4316507,328.40696777)
\lineto(150.6116507,328.43696777)
\curveto(150.69164738,328.44695984)(150.77664729,328.45195984)(150.8666507,328.45196777)
\curveto(150.95664711,328.45195984)(151.04164703,328.45695983)(151.1216507,328.46696777)
}
}
{
\newrgbcolor{curcolor}{0 0 0}
\pscustom[linestyle=none,fillstyle=solid,fillcolor=curcolor]
{
\newpath
\moveto(163.97829132,324.70196777)
\curveto(163.99828275,324.64196365)(164.00828274,324.55696373)(164.00829132,324.44696777)
\curveto(164.00828274,324.33696395)(163.99828275,324.25196404)(163.97829132,324.19196777)
\lineto(163.97829132,324.04196777)
\curveto(163.95828279,323.96196433)(163.9482828,323.88196441)(163.94829132,323.80196777)
\curveto(163.95828279,323.72196457)(163.9532828,323.64196465)(163.93329132,323.56196777)
\curveto(163.91328284,323.4919648)(163.89828285,323.42696486)(163.88829132,323.36696777)
\curveto(163.87828287,323.30696498)(163.86828288,323.24196505)(163.85829132,323.17196777)
\curveto(163.81828293,323.06196523)(163.78328297,322.94696534)(163.75329132,322.82696777)
\curveto(163.72328303,322.71696557)(163.68328307,322.61196568)(163.63329132,322.51196777)
\curveto(163.42328333,322.03196626)(163.1482836,321.64196665)(162.80829132,321.34196777)
\curveto(162.46828428,321.04196725)(162.05828469,320.7919675)(161.57829132,320.59196777)
\curveto(161.45828529,320.54196775)(161.33328542,320.50696778)(161.20329132,320.48696777)
\curveto(161.08328567,320.45696783)(160.95828579,320.42696786)(160.82829132,320.39696777)
\curveto(160.77828597,320.37696791)(160.72328603,320.36696792)(160.66329132,320.36696777)
\curveto(160.60328615,320.36696792)(160.5482862,320.36196793)(160.49829132,320.35196777)
\lineto(160.39329132,320.35196777)
\curveto(160.36328639,320.34196795)(160.33328642,320.33696795)(160.30329132,320.33696777)
\curveto(160.2532865,320.32696796)(160.17328658,320.32196797)(160.06329132,320.32196777)
\curveto(159.9532868,320.31196798)(159.86828688,320.31696797)(159.80829132,320.33696777)
\lineto(159.65829132,320.33696777)
\curveto(159.60828714,320.34696794)(159.5532872,320.35196794)(159.49329132,320.35196777)
\curveto(159.44328731,320.34196795)(159.39328736,320.34696794)(159.34329132,320.36696777)
\curveto(159.30328745,320.37696791)(159.26328749,320.38196791)(159.22329132,320.38196777)
\curveto(159.19328756,320.38196791)(159.1532876,320.3869679)(159.10329132,320.39696777)
\curveto(159.00328775,320.42696786)(158.90328785,320.45196784)(158.80329132,320.47196777)
\curveto(158.70328805,320.4919678)(158.60828814,320.52196777)(158.51829132,320.56196777)
\curveto(158.39828835,320.60196769)(158.28328847,320.64196765)(158.17329132,320.68196777)
\curveto(158.07328868,320.72196757)(157.96828878,320.77196752)(157.85829132,320.83196777)
\curveto(157.50828924,321.04196725)(157.20828954,321.286967)(156.95829132,321.56696777)
\curveto(156.70829004,321.84696644)(156.49829025,322.18196611)(156.32829132,322.57196777)
\curveto(156.27829047,322.66196563)(156.23829051,322.75696553)(156.20829132,322.85696777)
\curveto(156.18829056,322.95696533)(156.16329059,323.06196523)(156.13329132,323.17196777)
\curveto(156.11329064,323.22196507)(156.10329065,323.26696502)(156.10329132,323.30696777)
\curveto(156.10329065,323.34696494)(156.09329066,323.3919649)(156.07329132,323.44196777)
\curveto(156.0532907,323.52196477)(156.04329071,323.60196469)(156.04329132,323.68196777)
\curveto(156.04329071,323.77196452)(156.03329072,323.85696443)(156.01329132,323.93696777)
\curveto(156.00329075,323.9869643)(155.99829075,324.03196426)(155.99829132,324.07196777)
\lineto(155.99829132,324.20696777)
\curveto(155.97829077,324.26696402)(155.96829078,324.35196394)(155.96829132,324.46196777)
\curveto(155.97829077,324.57196372)(155.99329076,324.65696363)(156.01329132,324.71696777)
\lineto(156.01329132,324.82196777)
\curveto(156.02329073,324.87196342)(156.02329073,324.92196337)(156.01329132,324.97196777)
\curveto(156.01329074,325.03196326)(156.02329073,325.0869632)(156.04329132,325.13696777)
\curveto(156.0532907,325.1869631)(156.05829069,325.23196306)(156.05829132,325.27196777)
\curveto(156.05829069,325.32196297)(156.06829068,325.37196292)(156.08829132,325.42196777)
\curveto(156.12829062,325.55196274)(156.16329059,325.67696261)(156.19329132,325.79696777)
\curveto(156.22329053,325.92696236)(156.26329049,326.05196224)(156.31329132,326.17196777)
\curveto(156.49329026,326.58196171)(156.70829004,326.92196137)(156.95829132,327.19196777)
\curveto(157.20828954,327.47196082)(157.51328924,327.72696056)(157.87329132,327.95696777)
\curveto(157.97328878,328.00696028)(158.07828867,328.05196024)(158.18829132,328.09196777)
\curveto(158.29828845,328.13196016)(158.40828834,328.17696011)(158.51829132,328.22696777)
\curveto(158.6482881,328.27696001)(158.78328797,328.31195998)(158.92329132,328.33196777)
\curveto(159.06328769,328.35195994)(159.20828754,328.38195991)(159.35829132,328.42196777)
\curveto(159.43828731,328.43195986)(159.51328724,328.43695985)(159.58329132,328.43696777)
\curveto(159.6532871,328.43695985)(159.72328703,328.44195985)(159.79329132,328.45196777)
\curveto(160.37328638,328.46195983)(160.87328588,328.40195989)(161.29329132,328.27196777)
\curveto(161.72328503,328.14196015)(162.10328465,327.96196033)(162.43329132,327.73196777)
\curveto(162.54328421,327.65196064)(162.6532841,327.56196073)(162.76329132,327.46196777)
\curveto(162.88328387,327.37196092)(162.98328377,327.27196102)(163.06329132,327.16196777)
\curveto(163.14328361,327.06196123)(163.21328354,326.96196133)(163.27329132,326.86196777)
\curveto(163.34328341,326.76196153)(163.41328334,326.65696163)(163.48329132,326.54696777)
\curveto(163.5532832,326.43696185)(163.60828314,326.31696197)(163.64829132,326.18696777)
\curveto(163.68828306,326.06696222)(163.73328302,325.93696235)(163.78329132,325.79696777)
\curveto(163.81328294,325.71696257)(163.83828291,325.63196266)(163.85829132,325.54196777)
\lineto(163.91829132,325.27196777)
\curveto(163.92828282,325.23196306)(163.93328282,325.1919631)(163.93329132,325.15196777)
\curveto(163.93328282,325.11196318)(163.93828281,325.07196322)(163.94829132,325.03196777)
\curveto(163.96828278,324.98196331)(163.97328278,324.92696336)(163.96329132,324.86696777)
\curveto(163.9532828,324.80696348)(163.95828279,324.75196354)(163.97829132,324.70196777)
\moveto(161.87829132,324.16196777)
\curveto(161.88828486,324.21196408)(161.89328486,324.28196401)(161.89329132,324.37196777)
\curveto(161.89328486,324.47196382)(161.88828486,324.54696374)(161.87829132,324.59696777)
\lineto(161.87829132,324.71696777)
\curveto(161.85828489,324.76696352)(161.8482849,324.82196347)(161.84829132,324.88196777)
\curveto(161.8482849,324.94196335)(161.84328491,324.99696329)(161.83329132,325.04696777)
\curveto(161.83328492,325.0869632)(161.82828492,325.11696317)(161.81829132,325.13696777)
\lineto(161.75829132,325.37696777)
\curveto(161.748285,325.46696282)(161.72828502,325.55196274)(161.69829132,325.63196777)
\curveto(161.58828516,325.8919624)(161.45828529,326.11196218)(161.30829132,326.29196777)
\curveto(161.15828559,326.48196181)(160.95828579,326.63196166)(160.70829132,326.74196777)
\curveto(160.6482861,326.76196153)(160.58828616,326.77696151)(160.52829132,326.78696777)
\curveto(160.46828628,326.80696148)(160.40328635,326.82696146)(160.33329132,326.84696777)
\curveto(160.2532865,326.86696142)(160.16828658,326.87196142)(160.07829132,326.86196777)
\lineto(159.80829132,326.86196777)
\curveto(159.77828697,326.84196145)(159.74328701,326.83196146)(159.70329132,326.83196777)
\curveto(159.66328709,326.84196145)(159.62828712,326.84196145)(159.59829132,326.83196777)
\lineto(159.38829132,326.77196777)
\curveto(159.32828742,326.76196153)(159.27328748,326.74196155)(159.22329132,326.71196777)
\curveto(158.97328778,326.60196169)(158.76828798,326.44196185)(158.60829132,326.23196777)
\curveto(158.45828829,326.03196226)(158.33828841,325.79696249)(158.24829132,325.52696777)
\curveto(158.21828853,325.42696286)(158.19328856,325.32196297)(158.17329132,325.21196777)
\curveto(158.16328859,325.10196319)(158.1482886,324.9919633)(158.12829132,324.88196777)
\curveto(158.11828863,324.83196346)(158.11328864,324.78196351)(158.11329132,324.73196777)
\lineto(158.11329132,324.58196777)
\curveto(158.09328866,324.51196378)(158.08328867,324.40696388)(158.08329132,324.26696777)
\curveto(158.09328866,324.12696416)(158.10828864,324.02196427)(158.12829132,323.95196777)
\lineto(158.12829132,323.81696777)
\curveto(158.1482886,323.73696455)(158.16328859,323.65696463)(158.17329132,323.57696777)
\curveto(158.18328857,323.50696478)(158.19828855,323.43196486)(158.21829132,323.35196777)
\curveto(158.31828843,323.05196524)(158.42328833,322.80696548)(158.53329132,322.61696777)
\curveto(158.6532881,322.43696585)(158.83828791,322.27196602)(159.08829132,322.12196777)
\curveto(159.15828759,322.07196622)(159.23328752,322.03196626)(159.31329132,322.00196777)
\curveto(159.40328735,321.97196632)(159.49328726,321.94696634)(159.58329132,321.92696777)
\curveto(159.62328713,321.91696637)(159.65828709,321.91196638)(159.68829132,321.91196777)
\curveto(159.71828703,321.92196637)(159.753287,321.92196637)(159.79329132,321.91196777)
\lineto(159.91329132,321.88196777)
\curveto(159.96328679,321.88196641)(160.00828674,321.8869664)(160.04829132,321.89696777)
\lineto(160.16829132,321.89696777)
\curveto(160.2482865,321.91696637)(160.32828642,321.93196636)(160.40829132,321.94196777)
\curveto(160.48828626,321.95196634)(160.56328619,321.97196632)(160.63329132,322.00196777)
\curveto(160.89328586,322.10196619)(161.10328565,322.23696605)(161.26329132,322.40696777)
\curveto(161.42328533,322.57696571)(161.55828519,322.7869655)(161.66829132,323.03696777)
\curveto(161.70828504,323.13696515)(161.73828501,323.23696505)(161.75829132,323.33696777)
\curveto(161.77828497,323.43696485)(161.80328495,323.54196475)(161.83329132,323.65196777)
\curveto(161.84328491,323.6919646)(161.8482849,323.72696456)(161.84829132,323.75696777)
\curveto(161.8482849,323.79696449)(161.8532849,323.83696445)(161.86329132,323.87696777)
\lineto(161.86329132,324.01196777)
\curveto(161.86328489,324.06196423)(161.86828488,324.11196418)(161.87829132,324.16196777)
}
}
{
\newrgbcolor{curcolor}{0 0 0}
\pscustom[linestyle=none,fillstyle=solid,fillcolor=curcolor]
{
\newpath
\moveto(169.8032132,328.45196777)
\curveto(170.40320739,328.47195982)(170.90320689,328.3869599)(171.3032132,328.19696777)
\curveto(171.70320609,328.00696028)(172.01820578,327.72696056)(172.2482132,327.35696777)
\curveto(172.31820548,327.24696104)(172.37320542,327.12696116)(172.4132132,326.99696777)
\curveto(172.45320534,326.87696141)(172.4932053,326.75196154)(172.5332132,326.62196777)
\curveto(172.55320524,326.54196175)(172.56320523,326.46696182)(172.5632132,326.39696777)
\curveto(172.57320522,326.32696196)(172.58820521,326.25696203)(172.6082132,326.18696777)
\curveto(172.60820519,326.12696216)(172.61320518,326.0869622)(172.6232132,326.06696777)
\curveto(172.64320515,325.92696236)(172.65320514,325.78196251)(172.6532132,325.63196777)
\lineto(172.6532132,325.19696777)
\lineto(172.6532132,323.86196777)
\lineto(172.6532132,321.43196777)
\curveto(172.65320514,321.24196705)(172.64820515,321.05696723)(172.6382132,320.87696777)
\curveto(172.63820516,320.70696758)(172.56820523,320.59696769)(172.4282132,320.54696777)
\curveto(172.36820543,320.52696776)(172.2982055,320.51696777)(172.2182132,320.51696777)
\lineto(171.9782132,320.51696777)
\lineto(171.1682132,320.51696777)
\curveto(171.04820675,320.51696777)(170.93820686,320.52196777)(170.8382132,320.53196777)
\curveto(170.74820705,320.55196774)(170.67820712,320.59696769)(170.6282132,320.66696777)
\curveto(170.58820721,320.72696756)(170.56320723,320.80196749)(170.5532132,320.89196777)
\lineto(170.5532132,321.20696777)
\lineto(170.5532132,322.25696777)
\lineto(170.5532132,324.49196777)
\curveto(170.55320724,324.86196343)(170.53820726,325.20196309)(170.5082132,325.51196777)
\curveto(170.47820732,325.83196246)(170.38820741,326.10196219)(170.2382132,326.32196777)
\curveto(170.0982077,326.52196177)(169.8932079,326.66196163)(169.6232132,326.74196777)
\curveto(169.57320822,326.76196153)(169.51820828,326.77196152)(169.4582132,326.77196777)
\curveto(169.40820839,326.77196152)(169.35320844,326.78196151)(169.2932132,326.80196777)
\curveto(169.24320855,326.81196148)(169.17820862,326.81196148)(169.0982132,326.80196777)
\curveto(169.02820877,326.80196149)(168.97320882,326.79696149)(168.9332132,326.78696777)
\curveto(168.8932089,326.77696151)(168.85820894,326.77196152)(168.8282132,326.77196777)
\curveto(168.798209,326.77196152)(168.76820903,326.76696152)(168.7382132,326.75696777)
\curveto(168.50820929,326.69696159)(168.32320947,326.61696167)(168.1832132,326.51696777)
\curveto(167.86320993,326.286962)(167.67321012,325.95196234)(167.6132132,325.51196777)
\curveto(167.55321024,325.07196322)(167.52321027,324.57696371)(167.5232132,324.02696777)
\lineto(167.5232132,322.15196777)
\lineto(167.5232132,321.23696777)
\lineto(167.5232132,320.96696777)
\curveto(167.52321027,320.87696741)(167.50821029,320.80196749)(167.4782132,320.74196777)
\curveto(167.42821037,320.63196766)(167.34821045,320.56696772)(167.2382132,320.54696777)
\curveto(167.12821067,320.52696776)(166.9932108,320.51696777)(166.8332132,320.51696777)
\lineto(166.0832132,320.51696777)
\curveto(165.97321182,320.51696777)(165.86321193,320.52196777)(165.7532132,320.53196777)
\curveto(165.64321215,320.54196775)(165.56321223,320.57696771)(165.5132132,320.63696777)
\curveto(165.44321235,320.72696756)(165.40821239,320.85696743)(165.4082132,321.02696777)
\curveto(165.41821238,321.19696709)(165.42321237,321.35696693)(165.4232132,321.50696777)
\lineto(165.4232132,323.54696777)
\lineto(165.4232132,326.84696777)
\lineto(165.4232132,327.61196777)
\lineto(165.4232132,327.91196777)
\curveto(165.43321236,328.00196029)(165.46321233,328.07696021)(165.5132132,328.13696777)
\curveto(165.53321226,328.16696012)(165.56321223,328.1869601)(165.6032132,328.19696777)
\curveto(165.65321214,328.21696007)(165.70321209,328.23196006)(165.7532132,328.24196777)
\lineto(165.8282132,328.24196777)
\curveto(165.87821192,328.25196004)(165.92821187,328.25696003)(165.9782132,328.25696777)
\lineto(166.1432132,328.25696777)
\lineto(166.7732132,328.25696777)
\curveto(166.85321094,328.25696003)(166.92821087,328.25196004)(166.9982132,328.24196777)
\curveto(167.07821072,328.24196005)(167.14821065,328.23196006)(167.2082132,328.21196777)
\curveto(167.27821052,328.18196011)(167.32321047,328.13696015)(167.3432132,328.07696777)
\curveto(167.37321042,328.01696027)(167.3982104,327.94696034)(167.4182132,327.86696777)
\curveto(167.42821037,327.82696046)(167.42821037,327.7919605)(167.4182132,327.76196777)
\curveto(167.41821038,327.73196056)(167.42821037,327.70196059)(167.4482132,327.67196777)
\curveto(167.46821033,327.62196067)(167.48321031,327.5919607)(167.4932132,327.58196777)
\curveto(167.51321028,327.57196072)(167.53821026,327.55696073)(167.5682132,327.53696777)
\curveto(167.67821012,327.52696076)(167.76821003,327.56196073)(167.8382132,327.64196777)
\curveto(167.90820989,327.73196056)(167.98320981,327.80196049)(168.0632132,327.85196777)
\curveto(168.33320946,328.05196024)(168.63320916,328.21196008)(168.9632132,328.33196777)
\curveto(169.05320874,328.36195993)(169.14320865,328.38195991)(169.2332132,328.39196777)
\curveto(169.33320846,328.40195989)(169.43820836,328.41695987)(169.5482132,328.43696777)
\curveto(169.57820822,328.44695984)(169.62320817,328.44695984)(169.6832132,328.43696777)
\curveto(169.74320805,328.43695985)(169.78320801,328.44195985)(169.8032132,328.45196777)
}
}
{
\newrgbcolor{curcolor}{0 0 0}
\pscustom[linestyle=none,fillstyle=solid,fillcolor=curcolor]
{
\newpath
\moveto(181.3344632,321.11696777)
\curveto(181.35445535,321.00696728)(181.36445534,320.89696739)(181.3644632,320.78696777)
\curveto(181.37445533,320.67696761)(181.32445538,320.60196769)(181.2144632,320.56196777)
\curveto(181.15445555,320.53196776)(181.08445562,320.51696777)(181.0044632,320.51696777)
\lineto(180.7644632,320.51696777)
\lineto(179.9544632,320.51696777)
\lineto(179.6844632,320.51696777)
\curveto(179.6044571,320.52696776)(179.53945716,320.55196774)(179.4894632,320.59196777)
\curveto(179.41945728,320.63196766)(179.36445734,320.6869676)(179.3244632,320.75696777)
\curveto(179.29445741,320.83696745)(179.24945745,320.90196739)(179.1894632,320.95196777)
\curveto(179.16945753,320.97196732)(179.14445756,320.9869673)(179.1144632,320.99696777)
\curveto(179.08445762,321.01696727)(179.04445766,321.02196727)(178.9944632,321.01196777)
\curveto(178.94445776,320.9919673)(178.89445781,320.96696732)(178.8444632,320.93696777)
\curveto(178.8044579,320.90696738)(178.75945794,320.88196741)(178.7094632,320.86196777)
\curveto(178.65945804,320.82196747)(178.6044581,320.7869675)(178.5444632,320.75696777)
\lineto(178.3644632,320.66696777)
\curveto(178.23445847,320.60696768)(178.0994586,320.55696773)(177.9594632,320.51696777)
\curveto(177.81945888,320.4869678)(177.67445903,320.45196784)(177.5244632,320.41196777)
\curveto(177.45445925,320.3919679)(177.38445932,320.38196791)(177.3144632,320.38196777)
\curveto(177.25445945,320.37196792)(177.18945951,320.36196793)(177.1194632,320.35196777)
\lineto(177.0294632,320.35196777)
\curveto(176.9994597,320.34196795)(176.96945973,320.33696795)(176.9394632,320.33696777)
\lineto(176.7744632,320.33696777)
\curveto(176.67446003,320.31696797)(176.57446013,320.31696797)(176.4744632,320.33696777)
\lineto(176.3394632,320.33696777)
\curveto(176.26946043,320.35696793)(176.1994605,320.36696792)(176.1294632,320.36696777)
\curveto(176.06946063,320.35696793)(176.00946069,320.36196793)(175.9494632,320.38196777)
\curveto(175.84946085,320.40196789)(175.75446095,320.42196787)(175.6644632,320.44196777)
\curveto(175.57446113,320.45196784)(175.48946121,320.47696781)(175.4094632,320.51696777)
\curveto(175.11946158,320.62696766)(174.86946183,320.76696752)(174.6594632,320.93696777)
\curveto(174.45946224,321.11696717)(174.2994624,321.35196694)(174.1794632,321.64196777)
\curveto(174.14946255,321.71196658)(174.11946258,321.7869665)(174.0894632,321.86696777)
\curveto(174.06946263,321.94696634)(174.04946265,322.03196626)(174.0294632,322.12196777)
\curveto(174.00946269,322.17196612)(173.9994627,322.22196607)(173.9994632,322.27196777)
\curveto(174.00946269,322.32196597)(174.00946269,322.37196592)(173.9994632,322.42196777)
\curveto(173.98946271,322.45196584)(173.97946272,322.51196578)(173.9694632,322.60196777)
\curveto(173.96946273,322.70196559)(173.97446273,322.77196552)(173.9844632,322.81196777)
\curveto(174.0044627,322.91196538)(174.01446269,322.99696529)(174.0144632,323.06696777)
\lineto(174.1044632,323.39696777)
\curveto(174.13446257,323.51696477)(174.17446253,323.62196467)(174.2244632,323.71196777)
\curveto(174.39446231,324.00196429)(174.58946211,324.22196407)(174.8094632,324.37196777)
\curveto(175.02946167,324.52196377)(175.30946139,324.65196364)(175.6494632,324.76196777)
\curveto(175.77946092,324.81196348)(175.91446079,324.84696344)(176.0544632,324.86696777)
\curveto(176.19446051,324.8869634)(176.33446037,324.91196338)(176.4744632,324.94196777)
\curveto(176.55446015,324.96196333)(176.63946006,324.97196332)(176.7294632,324.97196777)
\curveto(176.81945988,324.98196331)(176.90945979,324.99696329)(176.9994632,325.01696777)
\curveto(177.06945963,325.03696325)(177.13945956,325.04196325)(177.2094632,325.03196777)
\curveto(177.27945942,325.03196326)(177.35445935,325.04196325)(177.4344632,325.06196777)
\curveto(177.5044592,325.08196321)(177.57445913,325.0919632)(177.6444632,325.09196777)
\curveto(177.71445899,325.0919632)(177.78945891,325.10196319)(177.8694632,325.12196777)
\curveto(178.07945862,325.17196312)(178.26945843,325.21196308)(178.4394632,325.24196777)
\curveto(178.61945808,325.28196301)(178.77945792,325.37196292)(178.9194632,325.51196777)
\curveto(179.00945769,325.60196269)(179.06945763,325.70196259)(179.0994632,325.81196777)
\curveto(179.10945759,325.84196245)(179.10945759,325.86696242)(179.0994632,325.88696777)
\curveto(179.0994576,325.90696238)(179.1044576,325.92696236)(179.1144632,325.94696777)
\curveto(179.12445758,325.96696232)(179.12945757,325.99696229)(179.1294632,326.03696777)
\lineto(179.1294632,326.12696777)
\lineto(179.0994632,326.24696777)
\curveto(179.0994576,326.286962)(179.09445761,326.32196197)(179.0844632,326.35196777)
\curveto(178.98445772,326.65196164)(178.77445793,326.85696143)(178.4544632,326.96696777)
\curveto(178.36445834,326.99696129)(178.25445845,327.01696127)(178.1244632,327.02696777)
\curveto(178.0044587,327.04696124)(177.87945882,327.05196124)(177.7494632,327.04196777)
\curveto(177.61945908,327.04196125)(177.49445921,327.03196126)(177.3744632,327.01196777)
\curveto(177.25445945,326.9919613)(177.14945955,326.96696132)(177.0594632,326.93696777)
\curveto(176.9994597,326.91696137)(176.93945976,326.8869614)(176.8794632,326.84696777)
\curveto(176.82945987,326.81696147)(176.77945992,326.78196151)(176.7294632,326.74196777)
\curveto(176.67946002,326.70196159)(176.62446008,326.64696164)(176.5644632,326.57696777)
\curveto(176.51446019,326.50696178)(176.47946022,326.44196185)(176.4594632,326.38196777)
\curveto(176.40946029,326.28196201)(176.36446034,326.1869621)(176.3244632,326.09696777)
\curveto(176.29446041,326.00696228)(176.22446048,325.94696234)(176.1144632,325.91696777)
\curveto(176.03446067,325.89696239)(175.94946075,325.8869624)(175.8594632,325.88696777)
\lineto(175.5894632,325.88696777)
\lineto(175.0194632,325.88696777)
\curveto(174.96946173,325.8869624)(174.91946178,325.88196241)(174.8694632,325.87196777)
\curveto(174.81946188,325.87196242)(174.77446193,325.87696241)(174.7344632,325.88696777)
\lineto(174.5994632,325.88696777)
\curveto(174.57946212,325.89696239)(174.55446215,325.90196239)(174.5244632,325.90196777)
\curveto(174.49446221,325.90196239)(174.46946223,325.91196238)(174.4494632,325.93196777)
\curveto(174.36946233,325.95196234)(174.31446239,326.01696227)(174.2844632,326.12696777)
\curveto(174.27446243,326.17696211)(174.27446243,326.22696206)(174.2844632,326.27696777)
\curveto(174.29446241,326.32696196)(174.3044624,326.37196192)(174.3144632,326.41196777)
\curveto(174.34446236,326.52196177)(174.37446233,326.62196167)(174.4044632,326.71196777)
\curveto(174.44446226,326.81196148)(174.48946221,326.90196139)(174.5394632,326.98196777)
\lineto(174.6294632,327.13196777)
\lineto(174.7194632,327.28196777)
\curveto(174.7994619,327.3919609)(174.8994618,327.49696079)(175.0194632,327.59696777)
\curveto(175.03946166,327.60696068)(175.06946163,327.63196066)(175.1094632,327.67196777)
\curveto(175.15946154,327.71196058)(175.2044615,327.74696054)(175.2444632,327.77696777)
\curveto(175.28446142,327.80696048)(175.32946137,327.83696045)(175.3794632,327.86696777)
\curveto(175.54946115,327.97696031)(175.72946097,328.06196023)(175.9194632,328.12196777)
\curveto(176.10946059,328.1919601)(176.3044604,328.25696003)(176.5044632,328.31696777)
\curveto(176.62446008,328.34695994)(176.74945995,328.36695992)(176.8794632,328.37696777)
\curveto(177.00945969,328.3869599)(177.13945956,328.40695988)(177.2694632,328.43696777)
\curveto(177.30945939,328.44695984)(177.36945933,328.44695984)(177.4494632,328.43696777)
\curveto(177.53945916,328.42695986)(177.59445911,328.43195986)(177.6144632,328.45196777)
\curveto(178.02445868,328.46195983)(178.41445829,328.44695984)(178.7844632,328.40696777)
\curveto(179.16445754,328.36695992)(179.5044572,328.29196)(179.8044632,328.18196777)
\curveto(180.11445659,328.07196022)(180.37945632,327.92196037)(180.5994632,327.73196777)
\curveto(180.81945588,327.55196074)(180.98945571,327.31696097)(181.1094632,327.02696777)
\curveto(181.17945552,326.85696143)(181.21945548,326.66196163)(181.2294632,326.44196777)
\curveto(181.23945546,326.22196207)(181.24445546,325.99696229)(181.2444632,325.76696777)
\lineto(181.2444632,322.42196777)
\lineto(181.2444632,321.83696777)
\curveto(181.24445546,321.64696664)(181.26445544,321.47196682)(181.3044632,321.31196777)
\curveto(181.31445539,321.28196701)(181.31945538,321.24696704)(181.3194632,321.20696777)
\curveto(181.31945538,321.17696711)(181.32445538,321.14696714)(181.3344632,321.11696777)
\moveto(179.1294632,323.42696777)
\curveto(179.13945756,323.47696481)(179.14445756,323.53196476)(179.1444632,323.59196777)
\curveto(179.14445756,323.66196463)(179.13945756,323.72196457)(179.1294632,323.77196777)
\curveto(179.10945759,323.83196446)(179.0994576,323.8869644)(179.0994632,323.93696777)
\curveto(179.0994576,323.9869643)(179.07945762,324.02696426)(179.0394632,324.05696777)
\curveto(178.98945771,324.09696419)(178.91445779,324.11696417)(178.8144632,324.11696777)
\curveto(178.77445793,324.10696418)(178.73945796,324.09696419)(178.7094632,324.08696777)
\curveto(178.67945802,324.0869642)(178.64445806,324.08196421)(178.6044632,324.07196777)
\curveto(178.53445817,324.05196424)(178.45945824,324.03696425)(178.3794632,324.02696777)
\curveto(178.2994584,324.01696427)(178.21945848,324.00196429)(178.1394632,323.98196777)
\curveto(178.10945859,323.97196432)(178.06445864,323.96696432)(178.0044632,323.96696777)
\curveto(177.87445883,323.93696435)(177.74445896,323.91696437)(177.6144632,323.90696777)
\curveto(177.48445922,323.89696439)(177.35945934,323.87196442)(177.2394632,323.83196777)
\curveto(177.15945954,323.81196448)(177.08445962,323.7919645)(177.0144632,323.77196777)
\curveto(176.94445976,323.76196453)(176.87445983,323.74196455)(176.8044632,323.71196777)
\curveto(176.59446011,323.62196467)(176.41446029,323.4869648)(176.2644632,323.30696777)
\curveto(176.12446058,323.12696516)(176.07446063,322.87696541)(176.1144632,322.55696777)
\curveto(176.13446057,322.3869659)(176.18946051,322.24696604)(176.2794632,322.13696777)
\curveto(176.34946035,322.02696626)(176.45446025,321.93696635)(176.5944632,321.86696777)
\curveto(176.73445997,321.80696648)(176.88445982,321.76196653)(177.0444632,321.73196777)
\curveto(177.21445949,321.70196659)(177.38945931,321.6919666)(177.5694632,321.70196777)
\curveto(177.75945894,321.72196657)(177.93445877,321.75696653)(178.0944632,321.80696777)
\curveto(178.35445835,321.8869664)(178.55945814,322.01196628)(178.7094632,322.18196777)
\curveto(178.85945784,322.36196593)(178.97445773,322.58196571)(179.0544632,322.84196777)
\curveto(179.07445763,322.91196538)(179.08445762,322.98196531)(179.0844632,323.05196777)
\curveto(179.09445761,323.13196516)(179.10945759,323.21196508)(179.1294632,323.29196777)
\lineto(179.1294632,323.42696777)
}
}
{
\newrgbcolor{curcolor}{0 0 0}
\pscustom[linestyle=none,fillstyle=solid,fillcolor=curcolor]
{
\newpath
\moveto(183.40774445,331.21196777)
\lineto(184.50274445,331.21196777)
\curveto(184.60274196,331.21195708)(184.69774187,331.20695708)(184.78774445,331.19696777)
\curveto(184.87774169,331.1869571)(184.94774162,331.15695713)(184.99774445,331.10696777)
\curveto(185.05774151,331.03695725)(185.08774148,330.94195735)(185.08774445,330.82196777)
\curveto(185.09774147,330.71195758)(185.10274146,330.59695769)(185.10274445,330.47696777)
\lineto(185.10274445,329.14196777)
\lineto(185.10274445,323.75696777)
\lineto(185.10274445,321.46196777)
\lineto(185.10274445,321.04196777)
\curveto(185.11274145,320.8919674)(185.09274147,320.77696751)(185.04274445,320.69696777)
\curveto(184.99274157,320.61696767)(184.90274166,320.56196773)(184.77274445,320.53196777)
\curveto(184.71274185,320.51196778)(184.64274192,320.50696778)(184.56274445,320.51696777)
\curveto(184.49274207,320.52696776)(184.42274214,320.53196776)(184.35274445,320.53196777)
\lineto(183.63274445,320.53196777)
\curveto(183.52274304,320.53196776)(183.42274314,320.53696775)(183.33274445,320.54696777)
\curveto(183.24274332,320.55696773)(183.1677434,320.5869677)(183.10774445,320.63696777)
\curveto(183.04774352,320.6869676)(183.01274355,320.76196753)(183.00274445,320.86196777)
\lineto(183.00274445,321.19196777)
\lineto(183.00274445,322.52696777)
\lineto(183.00274445,328.15196777)
\lineto(183.00274445,330.19196777)
\curveto(183.00274356,330.32195797)(182.99774357,330.47695781)(182.98774445,330.65696777)
\curveto(182.98774358,330.83695745)(183.01274355,330.96695732)(183.06274445,331.04696777)
\curveto(183.08274348,331.0869572)(183.10774346,331.11695717)(183.13774445,331.13696777)
\lineto(183.25774445,331.19696777)
\curveto(183.27774329,331.19695709)(183.30274326,331.19695709)(183.33274445,331.19696777)
\curveto(183.3627432,331.20695708)(183.38774318,331.21195708)(183.40774445,331.21196777)
}
}
{
\newrgbcolor{curcolor}{0.80000001 0.80000001 0.80000001}
\pscustom[linestyle=none,fillstyle=solid,fillcolor=curcolor]
{
\newpath
\moveto(235.99027368,1004.64501626)
\curveto(316.63232855,995.37371349)(374.4897547,922.48453554)(365.21845193,841.84248067)
\curveto(355.94714916,761.2004258)(283.05797121,703.34299964)(202.41591635,712.61430241)
\curveto(121.77386148,721.88560518)(63.91643532,794.77478313)(73.18773809,875.416838)
\curveto(80.54441562,939.40542815)(128.70695345,991.14910154)(192.0045985,1003.06835087)
\lineto(219.20309501,858.62965934)
\closepath
}
}
{
\newrgbcolor{curcolor}{0.90196079 0.90196079 0.90196079}
\pscustom[linestyle=none,fillstyle=solid,fillcolor=curcolor]
{
\newpath
\moveto(219.20309789,1005.60684745)
\curveto(224.8052036,1005.60684734)(230.40272756,1005.28655704)(235.96826897,1004.64754443)
\lineto(219.20309501,858.62965934)
\closepath
}
}
{
\newrgbcolor{curcolor}{0.7019608 0.7019608 0.7019608}
\pscustom[linestyle=none,fillstyle=solid,fillcolor=curcolor]
{
\newpath
\moveto(191.98930363,1003.06546994)
\curveto(196.30021915,1003.87770864)(200.64533726,1004.49654103)(205.0116074,1004.92010846)
\lineto(219.20309501,858.62965934)
\closepath
}
}
{
\newrgbcolor{curcolor}{0.60000002 0.60000002 0.60000002}
\pscustom[linestyle=none,fillstyle=solid,fillcolor=curcolor]
{
\newpath
\moveto(204.99174129,1004.9181799)
\curveto(206.65941254,1005.0801876)(208.32973218,1005.21366658)(210.0019677,1005.31855831)
\lineto(219.20309501,858.62965934)
\closepath
}
}
{
\newrgbcolor{curcolor}{0.50196081 0.50196081 0.50196081}
\pscustom[linestyle=none,fillstyle=solid,fillcolor=curcolor]
{
\newpath
\moveto(209.96399025,1005.31617122)
\curveto(210.65273517,1005.35955207)(211.34177707,1005.39808154)(212.0310647,1005.43175675)
\lineto(219.20309501,858.62965934)
\closepath
}
}
{
\newrgbcolor{curcolor}{0.40000001 0.40000001 0.40000001}
\pscustom[linestyle=none,fillstyle=solid,fillcolor=curcolor]
{
\newpath
\moveto(212.01904164,1005.43116887)
\curveto(213.45578445,1005.50147903)(214.89348204,1005.55069335)(216.33166968,1005.57879592)
\lineto(219.20309501,858.62965934)
\closepath
}
}
{
\newrgbcolor{curcolor}{0.3019608 0.3019608 0.3019608}
\pscustom[linestyle=none,fillstyle=solid,fillcolor=curcolor]
{
\newpath
\moveto(216.30339672,1005.57824073)
\curveto(217.26984619,1005.59731143)(218.23646028,1005.60684747)(219.20309789,1005.60684745)
\lineto(219.20309501,858.62965934)
\closepath
}
}
{
\newrgbcolor{curcolor}{0 0 0}
\pscustom[linestyle=none,fillstyle=solid,fillcolor=curcolor]
{
\newpath
\moveto(626.45375732,188.69002441)
\curveto(626.52375559,188.69001375)(626.60375551,188.69001375)(626.69375732,188.69002441)
\curveto(626.78375533,188.70001374)(626.86875524,188.70001374)(626.94875732,188.69002441)
\curveto(627.03875507,188.69001375)(627.11875499,188.68001376)(627.18875732,188.66002441)
\curveto(627.25875485,188.6400138)(627.3087548,188.61001383)(627.33875732,188.57002441)
\curveto(627.39875471,188.50001394)(627.42875468,188.40001404)(627.42875732,188.27002441)
\curveto(627.43875467,188.15001429)(627.44375467,188.02501441)(627.44375732,187.89502441)
\lineto(627.44375732,186.44002441)
\lineto(627.44375732,180.65002441)
\lineto(627.44375732,178.89502441)
\lineto(627.44375732,178.47502441)
\curveto(627.44375467,178.3350241)(627.41875469,178.22502421)(627.36875732,178.14502441)
\curveto(627.32875478,178.09502434)(627.27875483,178.06502437)(627.21875732,178.05502441)
\curveto(627.16875494,178.04502439)(627.10375501,178.03002441)(627.02375732,178.01002441)
\lineto(626.73875732,178.01002441)
\curveto(626.59875551,178.01002443)(626.46875564,178.01502442)(626.34875732,178.02502441)
\curveto(626.22875588,178.0350244)(626.14375597,178.08502435)(626.09375732,178.17502441)
\curveto(626.05375606,178.2350242)(626.03375608,178.31502412)(626.03375732,178.41502441)
\lineto(626.03375732,178.74502441)
\lineto(626.03375732,179.94502441)
\lineto(626.03375732,186.21502441)
\lineto(626.03375732,187.83502441)
\curveto(626.03375608,187.94501449)(626.02875608,188.06501437)(626.01875732,188.19502441)
\curveto(626.01875609,188.3350141)(626.04375607,188.44501399)(626.09375732,188.52502441)
\curveto(626.13375598,188.59501384)(626.2137559,188.64501379)(626.33375732,188.67502441)
\curveto(626.35375576,188.68501375)(626.37375574,188.68501375)(626.39375732,188.67502441)
\curveto(626.4137557,188.67501376)(626.43375568,188.68001376)(626.45375732,188.69002441)
}
}
{
\newrgbcolor{curcolor}{0 0 0}
\pscustom[linestyle=none,fillstyle=solid,fillcolor=curcolor]
{
\newpath
\moveto(633.2902417,185.91502441)
\curveto(633.92023646,185.9350165)(634.42523596,185.85001659)(634.8052417,185.66002441)
\curveto(635.1852352,185.47001697)(635.49023489,185.18501725)(635.7202417,184.80502441)
\curveto(635.7802346,184.70501773)(635.82523456,184.59501784)(635.8552417,184.47502441)
\curveto(635.89523449,184.36501807)(635.93023445,184.25001819)(635.9602417,184.13002441)
\curveto(636.01023437,183.9400185)(636.04023434,183.7350187)(636.0502417,183.51502441)
\curveto(636.06023432,183.29501914)(636.06523432,183.07001937)(636.0652417,182.84002441)
\lineto(636.0652417,181.23502441)
\lineto(636.0652417,178.89502441)
\curveto(636.06523432,178.72502371)(636.06023432,178.55502388)(636.0502417,178.38502441)
\curveto(636.05023433,178.21502422)(635.9852344,178.10502433)(635.8552417,178.05502441)
\curveto(635.80523458,178.0350244)(635.75023463,178.02502441)(635.6902417,178.02502441)
\curveto(635.64023474,178.01502442)(635.5852348,178.01002443)(635.5252417,178.01002441)
\curveto(635.39523499,178.01002443)(635.27023511,178.01502442)(635.1502417,178.02502441)
\curveto(635.03023535,178.02502441)(634.94523544,178.06502437)(634.8952417,178.14502441)
\curveto(634.84523554,178.21502422)(634.82023556,178.30502413)(634.8202417,178.41502441)
\lineto(634.8202417,178.74502441)
\lineto(634.8202417,180.03502441)
\lineto(634.8202417,182.48002441)
\curveto(634.82023556,182.75001969)(634.81523557,183.01501942)(634.8052417,183.27502441)
\curveto(634.79523559,183.54501889)(634.75023563,183.77501866)(634.6702417,183.96502441)
\curveto(634.59023579,184.16501827)(634.47023591,184.32501811)(634.3102417,184.44502441)
\curveto(634.15023623,184.57501786)(633.96523642,184.67501776)(633.7552417,184.74502441)
\curveto(633.69523669,184.76501767)(633.63023675,184.77501766)(633.5602417,184.77502441)
\curveto(633.50023688,184.78501765)(633.44023694,184.80001764)(633.3802417,184.82002441)
\curveto(633.33023705,184.83001761)(633.25023713,184.83001761)(633.1402417,184.82002441)
\curveto(633.04023734,184.82001762)(632.97023741,184.81501762)(632.9302417,184.80502441)
\curveto(632.89023749,184.78501765)(632.85523753,184.77501766)(632.8252417,184.77502441)
\curveto(632.79523759,184.78501765)(632.76023762,184.78501765)(632.7202417,184.77502441)
\curveto(632.59023779,184.74501769)(632.46523792,184.71001773)(632.3452417,184.67002441)
\curveto(632.23523815,184.6400178)(632.13023825,184.59501784)(632.0302417,184.53502441)
\curveto(631.99023839,184.51501792)(631.95523843,184.49501794)(631.9252417,184.47502441)
\curveto(631.89523849,184.45501798)(631.86023852,184.435018)(631.8202417,184.41502441)
\curveto(631.47023891,184.16501827)(631.21523917,183.79001865)(631.0552417,183.29002441)
\curveto(631.02523936,183.21001923)(631.00523938,183.12501931)(630.9952417,183.03502441)
\curveto(630.9852394,182.95501948)(630.97023941,182.87501956)(630.9502417,182.79502441)
\curveto(630.93023945,182.74501969)(630.92523946,182.69501974)(630.9352417,182.64502441)
\curveto(630.94523944,182.60501983)(630.94023944,182.56501987)(630.9202417,182.52502441)
\lineto(630.9202417,182.21002441)
\curveto(630.91023947,182.18002026)(630.90523948,182.14502029)(630.9052417,182.10502441)
\curveto(630.91523947,182.06502037)(630.92023946,182.02002042)(630.9202417,181.97002441)
\lineto(630.9202417,181.52002441)
\lineto(630.9202417,180.08002441)
\lineto(630.9202417,178.76002441)
\lineto(630.9202417,178.41502441)
\curveto(630.92023946,178.30502413)(630.89523949,178.21502422)(630.8452417,178.14502441)
\curveto(630.79523959,178.06502437)(630.70523968,178.02502441)(630.5752417,178.02502441)
\curveto(630.45523993,178.01502442)(630.33024005,178.01002443)(630.2002417,178.01002441)
\curveto(630.12024026,178.01002443)(630.04524034,178.01502442)(629.9752417,178.02502441)
\curveto(629.90524048,178.0350244)(629.84524054,178.06002438)(629.7952417,178.10002441)
\curveto(629.71524067,178.15002429)(629.67524071,178.24502419)(629.6752417,178.38502441)
\lineto(629.6752417,178.79002441)
\lineto(629.6752417,180.56002441)
\lineto(629.6752417,184.19002441)
\lineto(629.6752417,185.10502441)
\lineto(629.6752417,185.37502441)
\curveto(629.67524071,185.46501697)(629.69524069,185.5350169)(629.7352417,185.58502441)
\curveto(629.76524062,185.64501679)(629.81524057,185.68501675)(629.8852417,185.70502441)
\curveto(629.92524046,185.71501672)(629.9802404,185.72501671)(630.0502417,185.73502441)
\curveto(630.13024025,185.74501669)(630.21024017,185.75001669)(630.2902417,185.75002441)
\curveto(630.37024001,185.75001669)(630.44523994,185.74501669)(630.5152417,185.73502441)
\curveto(630.59523979,185.72501671)(630.65023973,185.71001673)(630.6802417,185.69002441)
\curveto(630.79023959,185.62001682)(630.84023954,185.53001691)(630.8302417,185.42002441)
\curveto(630.82023956,185.32001712)(630.83523955,185.20501723)(630.8752417,185.07502441)
\curveto(630.89523949,185.01501742)(630.93523945,184.96501747)(630.9952417,184.92502441)
\curveto(631.11523927,184.91501752)(631.21023917,184.96001748)(631.2802417,185.06002441)
\curveto(631.36023902,185.16001728)(631.44023894,185.2400172)(631.5202417,185.30002441)
\curveto(631.66023872,185.40001704)(631.80023858,185.49001695)(631.9402417,185.57002441)
\curveto(632.09023829,185.66001678)(632.26023812,185.7350167)(632.4502417,185.79502441)
\curveto(632.53023785,185.82501661)(632.61523777,185.84501659)(632.7052417,185.85502441)
\curveto(632.80523758,185.86501657)(632.90023748,185.88001656)(632.9902417,185.90002441)
\curveto(633.04023734,185.91001653)(633.09023729,185.91501652)(633.1402417,185.91502441)
\lineto(633.2902417,185.91502441)
}
}
{
\newrgbcolor{curcolor}{0 0 0}
\pscustom[linestyle=none,fillstyle=solid,fillcolor=curcolor]
{
\newpath
\moveto(637.72485107,185.76502441)
\lineto(638.20485107,185.76502441)
\curveto(638.37484973,185.76501667)(638.5048496,185.7350167)(638.59485107,185.67502441)
\curveto(638.66484944,185.62501681)(638.7098494,185.56001688)(638.72985107,185.48002441)
\curveto(638.75984935,185.41001703)(638.78984932,185.3350171)(638.81985107,185.25502441)
\curveto(638.87984923,185.11501732)(638.92984918,184.97501746)(638.96985107,184.83502441)
\curveto(639.0098491,184.69501774)(639.05484905,184.55501788)(639.10485107,184.41502441)
\curveto(639.3048488,183.87501856)(639.48984862,183.33001911)(639.65985107,182.78002441)
\curveto(639.82984828,182.2400202)(640.01484809,181.70002074)(640.21485107,181.16002441)
\curveto(640.28484782,180.98002146)(640.34484776,180.79502164)(640.39485107,180.60502441)
\curveto(640.44484766,180.42502201)(640.5098476,180.24502219)(640.58985107,180.06502441)
\curveto(640.6098475,179.99502244)(640.63484747,179.92002252)(640.66485107,179.84002441)
\curveto(640.69484741,179.76002268)(640.74484736,179.71002273)(640.81485107,179.69002441)
\curveto(640.89484721,179.67002277)(640.95484715,179.70502273)(640.99485107,179.79502441)
\curveto(641.04484706,179.88502255)(641.07984703,179.95502248)(641.09985107,180.00502441)
\curveto(641.17984693,180.19502224)(641.24484686,180.38502205)(641.29485107,180.57502441)
\curveto(641.35484675,180.77502166)(641.41984669,180.97502146)(641.48985107,181.17502441)
\curveto(641.61984649,181.55502088)(641.74484636,181.93002051)(641.86485107,182.30002441)
\curveto(641.98484612,182.68001976)(642.109846,183.06001938)(642.23985107,183.44002441)
\curveto(642.28984582,183.61001883)(642.33984577,183.77501866)(642.38985107,183.93502441)
\curveto(642.43984567,184.10501833)(642.49984561,184.27001817)(642.56985107,184.43002441)
\curveto(642.61984549,184.57001787)(642.66484544,184.71001773)(642.70485107,184.85002441)
\curveto(642.74484536,184.99001745)(642.78984532,185.13001731)(642.83985107,185.27002441)
\curveto(642.85984525,185.3400171)(642.88484522,185.41001703)(642.91485107,185.48002441)
\curveto(642.94484516,185.55001689)(642.98484512,185.61001683)(643.03485107,185.66002441)
\curveto(643.11484499,185.71001673)(643.2048449,185.7400167)(643.30485107,185.75002441)
\curveto(643.4048447,185.76001668)(643.52484458,185.76501667)(643.66485107,185.76502441)
\curveto(643.73484437,185.76501667)(643.79984431,185.76001668)(643.85985107,185.75002441)
\curveto(643.91984419,185.75001669)(643.97484413,185.7400167)(644.02485107,185.72002441)
\curveto(644.11484399,185.68001676)(644.15984395,185.61501682)(644.15985107,185.52502441)
\curveto(644.16984394,185.435017)(644.15484395,185.34501709)(644.11485107,185.25502441)
\curveto(644.05484405,185.08501735)(643.99484411,184.91001753)(643.93485107,184.73002441)
\curveto(643.87484423,184.55001789)(643.8048443,184.37501806)(643.72485107,184.20502441)
\curveto(643.7048444,184.15501828)(643.68984442,184.10501833)(643.67985107,184.05502441)
\curveto(643.66984444,184.01501842)(643.65484445,183.97001847)(643.63485107,183.92002441)
\curveto(643.55484455,183.75001869)(643.48984462,183.57501886)(643.43985107,183.39502441)
\curveto(643.38984472,183.21501922)(643.32484478,183.0350194)(643.24485107,182.85502441)
\curveto(643.19484491,182.72501971)(643.14484496,182.59001985)(643.09485107,182.45002441)
\curveto(643.05484505,182.32002012)(643.0048451,182.19002025)(642.94485107,182.06002441)
\curveto(642.77484533,181.65002079)(642.61984549,181.2350212)(642.47985107,180.81502441)
\curveto(642.34984576,180.39502204)(642.19984591,179.98002246)(642.02985107,179.57002441)
\curveto(641.96984614,179.41002303)(641.91484619,179.25002319)(641.86485107,179.09002441)
\curveto(641.81484629,178.93002351)(641.75484635,178.77002367)(641.68485107,178.61002441)
\curveto(641.63484647,178.50002394)(641.58984652,178.39502404)(641.54985107,178.29502441)
\curveto(641.51984659,178.20502423)(641.44984666,178.1350243)(641.33985107,178.08502441)
\curveto(641.27984683,178.05502438)(641.2098469,178.0400244)(641.12985107,178.04002441)
\lineto(640.90485107,178.04002441)
\lineto(640.43985107,178.04002441)
\curveto(640.28984782,178.05002439)(640.17984793,178.10002434)(640.10985107,178.19002441)
\curveto(640.03984807,178.27002417)(639.98984812,178.36502407)(639.95985107,178.47502441)
\curveto(639.92984818,178.59502384)(639.88984822,178.71002373)(639.83985107,178.82002441)
\curveto(639.77984833,178.96002348)(639.71984839,179.10502333)(639.65985107,179.25502441)
\curveto(639.6098485,179.41502302)(639.55984855,179.56502287)(639.50985107,179.70502441)
\curveto(639.48984862,179.75502268)(639.47484863,179.79502264)(639.46485107,179.82502441)
\curveto(639.45484865,179.86502257)(639.43984867,179.91002253)(639.41985107,179.96002441)
\curveto(639.21984889,180.440022)(639.03484907,180.92502151)(638.86485107,181.41502441)
\curveto(638.7048494,181.90502053)(638.52484958,182.39002005)(638.32485107,182.87002441)
\curveto(638.26484984,183.03001941)(638.2048499,183.18501925)(638.14485107,183.33502441)
\curveto(638.09485001,183.49501894)(638.03985007,183.65501878)(637.97985107,183.81502441)
\lineto(637.91985107,183.96502441)
\curveto(637.9098502,184.02501841)(637.89485021,184.08001836)(637.87485107,184.13002441)
\curveto(637.79485031,184.30001814)(637.72485038,184.47001797)(637.66485107,184.64002441)
\curveto(637.61485049,184.81001763)(637.55485055,184.98001746)(637.48485107,185.15002441)
\curveto(637.46485064,185.21001723)(637.43985067,185.29001715)(637.40985107,185.39002441)
\curveto(637.37985073,185.49001695)(637.38485072,185.57501686)(637.42485107,185.64502441)
\curveto(637.47485063,185.69501674)(637.53485057,185.73001671)(637.60485107,185.75002441)
\curveto(637.67485043,185.75001669)(637.71485039,185.75501668)(637.72485107,185.76502441)
}
}
{
\newrgbcolor{curcolor}{0 0 0}
\pscustom[linestyle=none,fillstyle=solid,fillcolor=curcolor]
{
\newpath
\moveto(645.73485107,187.26502441)
\curveto(645.65484995,187.32501511)(645.60985,187.43001501)(645.59985107,187.58002441)
\lineto(645.59985107,188.04502441)
\lineto(645.59985107,188.30002441)
\curveto(645.59985001,188.39001405)(645.61484999,188.46501397)(645.64485107,188.52502441)
\curveto(645.68484992,188.60501383)(645.76484984,188.66501377)(645.88485107,188.70502441)
\curveto(645.9048497,188.71501372)(645.92484968,188.71501372)(645.94485107,188.70502441)
\curveto(645.97484963,188.70501373)(645.99984961,188.71001373)(646.01985107,188.72002441)
\curveto(646.18984942,188.72001372)(646.34984926,188.71501372)(646.49985107,188.70502441)
\curveto(646.64984896,188.69501374)(646.74984886,188.6350138)(646.79985107,188.52502441)
\curveto(646.82984878,188.46501397)(646.84484876,188.39001405)(646.84485107,188.30002441)
\lineto(646.84485107,188.04502441)
\curveto(646.84484876,187.86501457)(646.83984877,187.69501474)(646.82985107,187.53502441)
\curveto(646.82984878,187.37501506)(646.76484884,187.27001517)(646.63485107,187.22002441)
\curveto(646.58484902,187.20001524)(646.52984908,187.19001525)(646.46985107,187.19002441)
\lineto(646.30485107,187.19002441)
\lineto(645.98985107,187.19002441)
\curveto(645.88984972,187.19001525)(645.8048498,187.21501522)(645.73485107,187.26502441)
\moveto(646.84485107,178.76002441)
\lineto(646.84485107,178.44502441)
\curveto(646.85484875,178.34502409)(646.83484877,178.26502417)(646.78485107,178.20502441)
\curveto(646.75484885,178.14502429)(646.7098489,178.10502433)(646.64985107,178.08502441)
\curveto(646.58984902,178.07502436)(646.51984909,178.06002438)(646.43985107,178.04002441)
\lineto(646.21485107,178.04002441)
\curveto(646.08484952,178.0400244)(645.96984964,178.04502439)(645.86985107,178.05502441)
\curveto(645.77984983,178.07502436)(645.7098499,178.12502431)(645.65985107,178.20502441)
\curveto(645.61984999,178.26502417)(645.59985001,178.3400241)(645.59985107,178.43002441)
\lineto(645.59985107,178.71502441)
\lineto(645.59985107,185.06002441)
\lineto(645.59985107,185.37502441)
\curveto(645.59985001,185.48501695)(645.62484998,185.57001687)(645.67485107,185.63002441)
\curveto(645.7048499,185.68001676)(645.74484986,185.71001673)(645.79485107,185.72002441)
\curveto(645.84484976,185.73001671)(645.89984971,185.74501669)(645.95985107,185.76502441)
\curveto(645.97984963,185.76501667)(645.99984961,185.76001668)(646.01985107,185.75002441)
\curveto(646.04984956,185.75001669)(646.07484953,185.75501668)(646.09485107,185.76502441)
\curveto(646.22484938,185.76501667)(646.35484925,185.76001668)(646.48485107,185.75002441)
\curveto(646.62484898,185.75001669)(646.71984889,185.71001673)(646.76985107,185.63002441)
\curveto(646.81984879,185.57001687)(646.84484876,185.49001695)(646.84485107,185.39002441)
\lineto(646.84485107,185.10502441)
\lineto(646.84485107,178.76002441)
}
}
{
\newrgbcolor{curcolor}{0 0 0}
\pscustom[linestyle=none,fillstyle=solid,fillcolor=curcolor]
{
\newpath
\moveto(649.73469482,188.10502441)
\curveto(649.88469281,188.10501433)(650.03469266,188.10001434)(650.18469482,188.09002441)
\curveto(650.33469236,188.09001435)(650.43969226,188.05001439)(650.49969482,187.97002441)
\curveto(650.54969215,187.91001453)(650.57469212,187.82501461)(650.57469482,187.71502441)
\curveto(650.58469211,187.61501482)(650.58969211,187.51001493)(650.58969482,187.40002441)
\lineto(650.58969482,186.53002441)
\curveto(650.58969211,186.45001599)(650.58469211,186.36501607)(650.57469482,186.27502441)
\curveto(650.57469212,186.19501624)(650.58469211,186.12501631)(650.60469482,186.06502441)
\curveto(650.64469205,185.92501651)(650.73469196,185.8350166)(650.87469482,185.79502441)
\curveto(650.92469177,185.78501665)(650.96969173,185.78001666)(651.00969482,185.78002441)
\lineto(651.15969482,185.78002441)
\lineto(651.56469482,185.78002441)
\curveto(651.72469097,185.79001665)(651.83969086,185.78001666)(651.90969482,185.75002441)
\curveto(651.9996907,185.69001675)(652.05969064,185.63001681)(652.08969482,185.57002441)
\curveto(652.10969059,185.53001691)(652.11969058,185.48501695)(652.11969482,185.43502441)
\lineto(652.11969482,185.28502441)
\curveto(652.11969058,185.17501726)(652.11469058,185.07001737)(652.10469482,184.97002441)
\curveto(652.0946906,184.88001756)(652.05969064,184.81001763)(651.99969482,184.76002441)
\curveto(651.93969076,184.71001773)(651.85469084,184.68001776)(651.74469482,184.67002441)
\lineto(651.41469482,184.67002441)
\curveto(651.30469139,184.68001776)(651.1946915,184.68501775)(651.08469482,184.68502441)
\curveto(650.97469172,184.68501775)(650.87969182,184.67001777)(650.79969482,184.64002441)
\curveto(650.72969197,184.61001783)(650.67969202,184.56001788)(650.64969482,184.49002441)
\curveto(650.61969208,184.42001802)(650.5996921,184.3350181)(650.58969482,184.23502441)
\curveto(650.57969212,184.14501829)(650.57469212,184.04501839)(650.57469482,183.93502441)
\curveto(650.58469211,183.8350186)(650.58969211,183.7350187)(650.58969482,183.63502441)
\lineto(650.58969482,180.66502441)
\curveto(650.58969211,180.44502199)(650.58469211,180.21002223)(650.57469482,179.96002441)
\curveto(650.57469212,179.72002272)(650.61969208,179.5350229)(650.70969482,179.40502441)
\curveto(650.75969194,179.32502311)(650.82469187,179.27002317)(650.90469482,179.24002441)
\curveto(650.98469171,179.21002323)(651.07969162,179.18502325)(651.18969482,179.16502441)
\curveto(651.21969148,179.15502328)(651.24969145,179.15002329)(651.27969482,179.15002441)
\curveto(651.31969138,179.16002328)(651.35469134,179.16002328)(651.38469482,179.15002441)
\lineto(651.57969482,179.15002441)
\curveto(651.67969102,179.15002329)(651.76969093,179.1400233)(651.84969482,179.12002441)
\curveto(651.93969076,179.11002333)(652.00469069,179.07502336)(652.04469482,179.01502441)
\curveto(652.06469063,178.98502345)(652.07969062,178.93002351)(652.08969482,178.85002441)
\curveto(652.10969059,178.78002366)(652.11969058,178.70502373)(652.11969482,178.62502441)
\curveto(652.12969057,178.54502389)(652.12969057,178.46502397)(652.11969482,178.38502441)
\curveto(652.10969059,178.31502412)(652.08969061,178.26002418)(652.05969482,178.22002441)
\curveto(652.01969068,178.15002429)(651.94469075,178.10002434)(651.83469482,178.07002441)
\curveto(651.75469094,178.05002439)(651.66469103,178.0400244)(651.56469482,178.04002441)
\curveto(651.46469123,178.05002439)(651.37469132,178.05502438)(651.29469482,178.05502441)
\curveto(651.23469146,178.05502438)(651.17469152,178.05002439)(651.11469482,178.04002441)
\curveto(651.05469164,178.0400244)(650.9996917,178.04502439)(650.94969482,178.05502441)
\lineto(650.76969482,178.05502441)
\curveto(650.71969198,178.06502437)(650.66969203,178.07002437)(650.61969482,178.07002441)
\curveto(650.57969212,178.08002436)(650.53469216,178.08502435)(650.48469482,178.08502441)
\curveto(650.28469241,178.1350243)(650.10969259,178.19002425)(649.95969482,178.25002441)
\curveto(649.81969288,178.31002413)(649.699693,178.41502402)(649.59969482,178.56502441)
\curveto(649.45969324,178.76502367)(649.37969332,179.01502342)(649.35969482,179.31502441)
\curveto(649.33969336,179.62502281)(649.32969337,179.95502248)(649.32969482,180.30502441)
\lineto(649.32969482,184.23502441)
\curveto(649.2996934,184.36501807)(649.26969343,184.46001798)(649.23969482,184.52002441)
\curveto(649.21969348,184.58001786)(649.14969355,184.63001781)(649.02969482,184.67002441)
\curveto(648.98969371,184.68001776)(648.94969375,184.68001776)(648.90969482,184.67002441)
\curveto(648.86969383,184.66001778)(648.82969387,184.66501777)(648.78969482,184.68502441)
\lineto(648.54969482,184.68502441)
\curveto(648.41969428,184.68501775)(648.30969439,184.69501774)(648.21969482,184.71502441)
\curveto(648.13969456,184.74501769)(648.08469461,184.80501763)(648.05469482,184.89502441)
\curveto(648.03469466,184.9350175)(648.01969468,184.98001746)(648.00969482,185.03002441)
\lineto(648.00969482,185.18002441)
\curveto(648.00969469,185.32001712)(648.01969468,185.435017)(648.03969482,185.52502441)
\curveto(648.05969464,185.62501681)(648.11969458,185.70001674)(648.21969482,185.75002441)
\curveto(648.32969437,185.79001665)(648.46969423,185.80001664)(648.63969482,185.78002441)
\curveto(648.81969388,185.76001668)(648.96969373,185.77001667)(649.08969482,185.81002441)
\curveto(649.17969352,185.86001658)(649.24969345,185.93001651)(649.29969482,186.02002441)
\curveto(649.31969338,186.08001636)(649.32969337,186.15501628)(649.32969482,186.24502441)
\lineto(649.32969482,186.50002441)
\lineto(649.32969482,187.43002441)
\lineto(649.32969482,187.67002441)
\curveto(649.32969337,187.76001468)(649.33969336,187.8350146)(649.35969482,187.89502441)
\curveto(649.3996933,187.97501446)(649.47469322,188.0400144)(649.58469482,188.09002441)
\curveto(649.61469308,188.09001435)(649.63969306,188.09001435)(649.65969482,188.09002441)
\curveto(649.68969301,188.10001434)(649.71469298,188.10501433)(649.73469482,188.10502441)
}
}
{
\newrgbcolor{curcolor}{0 0 0}
\pscustom[linestyle=none,fillstyle=solid,fillcolor=curcolor]
{
\newpath
\moveto(660.3914917,178.59502441)
\curveto(660.42148387,178.435024)(660.40648388,178.30002414)(660.3464917,178.19002441)
\curveto(660.286484,178.09002435)(660.20648408,178.01502442)(660.1064917,177.96502441)
\curveto(660.05648423,177.94502449)(660.00148429,177.9350245)(659.9414917,177.93502441)
\curveto(659.8914844,177.9350245)(659.83648445,177.92502451)(659.7764917,177.90502441)
\curveto(659.55648473,177.85502458)(659.33648495,177.87002457)(659.1164917,177.95002441)
\curveto(658.90648538,178.02002442)(658.76148553,178.11002433)(658.6814917,178.22002441)
\curveto(658.63148566,178.29002415)(658.5864857,178.37002407)(658.5464917,178.46002441)
\curveto(658.50648578,178.56002388)(658.45648583,178.6400238)(658.3964917,178.70002441)
\curveto(658.37648591,178.72002372)(658.35148594,178.7400237)(658.3214917,178.76002441)
\curveto(658.30148599,178.78002366)(658.27148602,178.78502365)(658.2314917,178.77502441)
\curveto(658.12148617,178.74502369)(658.01648627,178.69002375)(657.9164917,178.61002441)
\curveto(657.82648646,178.53002391)(657.73648655,178.46002398)(657.6464917,178.40002441)
\curveto(657.51648677,178.32002412)(657.37648691,178.24502419)(657.2264917,178.17502441)
\curveto(657.07648721,178.11502432)(656.91648737,178.06002438)(656.7464917,178.01002441)
\curveto(656.64648764,177.98002446)(656.53648775,177.96002448)(656.4164917,177.95002441)
\curveto(656.30648798,177.9400245)(656.19648809,177.92502451)(656.0864917,177.90502441)
\curveto(656.03648825,177.89502454)(655.9914883,177.89002455)(655.9514917,177.89002441)
\lineto(655.8464917,177.89002441)
\curveto(655.73648855,177.87002457)(655.63148866,177.87002457)(655.5314917,177.89002441)
\lineto(655.3964917,177.89002441)
\curveto(655.34648894,177.90002454)(655.29648899,177.90502453)(655.2464917,177.90502441)
\curveto(655.19648909,177.90502453)(655.15148914,177.91502452)(655.1114917,177.93502441)
\curveto(655.07148922,177.94502449)(655.03648925,177.95002449)(655.0064917,177.95002441)
\curveto(654.9864893,177.9400245)(654.96148933,177.9400245)(654.9314917,177.95002441)
\lineto(654.6914917,178.01002441)
\curveto(654.61148968,178.02002442)(654.53648975,178.0400244)(654.4664917,178.07002441)
\curveto(654.16649012,178.20002424)(653.92149037,178.34502409)(653.7314917,178.50502441)
\curveto(653.55149074,178.67502376)(653.40149089,178.91002353)(653.2814917,179.21002441)
\curveto(653.1914911,179.43002301)(653.14649114,179.69502274)(653.1464917,180.00502441)
\lineto(653.1464917,180.32002441)
\curveto(653.15649113,180.37002207)(653.16149113,180.42002202)(653.1614917,180.47002441)
\lineto(653.1914917,180.65002441)
\lineto(653.3114917,180.98002441)
\curveto(653.35149094,181.09002135)(653.40149089,181.19002125)(653.4614917,181.28002441)
\curveto(653.64149065,181.57002087)(653.8864904,181.78502065)(654.1964917,181.92502441)
\curveto(654.50648978,182.06502037)(654.84648944,182.19002025)(655.2164917,182.30002441)
\curveto(655.35648893,182.3400201)(655.50148879,182.37002007)(655.6514917,182.39002441)
\curveto(655.80148849,182.41002003)(655.95148834,182.43502)(656.1014917,182.46502441)
\curveto(656.17148812,182.48501995)(656.23648805,182.49501994)(656.2964917,182.49502441)
\curveto(656.36648792,182.49501994)(656.44148785,182.50501993)(656.5214917,182.52502441)
\curveto(656.5914877,182.54501989)(656.66148763,182.55501988)(656.7314917,182.55502441)
\curveto(656.80148749,182.56501987)(656.87648741,182.58001986)(656.9564917,182.60002441)
\curveto(657.20648708,182.66001978)(657.44148685,182.71001973)(657.6614917,182.75002441)
\curveto(657.88148641,182.80001964)(658.05648623,182.91501952)(658.1864917,183.09502441)
\curveto(658.24648604,183.17501926)(658.29648599,183.27501916)(658.3364917,183.39502441)
\curveto(658.37648591,183.52501891)(658.37648591,183.66501877)(658.3364917,183.81502441)
\curveto(658.27648601,184.05501838)(658.1864861,184.24501819)(658.0664917,184.38502441)
\curveto(657.95648633,184.52501791)(657.79648649,184.6350178)(657.5864917,184.71502441)
\curveto(657.46648682,184.76501767)(657.32148697,184.80001764)(657.1514917,184.82002441)
\curveto(656.9914873,184.8400176)(656.82148747,184.85001759)(656.6414917,184.85002441)
\curveto(656.46148783,184.85001759)(656.286488,184.8400176)(656.1164917,184.82002441)
\curveto(655.94648834,184.80001764)(655.80148849,184.77001767)(655.6814917,184.73002441)
\curveto(655.51148878,184.67001777)(655.34648894,184.58501785)(655.1864917,184.47502441)
\curveto(655.10648918,184.41501802)(655.03148926,184.3350181)(654.9614917,184.23502441)
\curveto(654.90148939,184.14501829)(654.84648944,184.04501839)(654.7964917,183.93502441)
\curveto(654.76648952,183.85501858)(654.73648955,183.77001867)(654.7064917,183.68002441)
\curveto(654.6864896,183.59001885)(654.64148965,183.52001892)(654.5714917,183.47002441)
\curveto(654.53148976,183.440019)(654.46148983,183.41501902)(654.3614917,183.39502441)
\curveto(654.27149002,183.38501905)(654.17649011,183.38001906)(654.0764917,183.38002441)
\curveto(653.97649031,183.38001906)(653.87649041,183.38501905)(653.7764917,183.39502441)
\curveto(653.6864906,183.41501902)(653.62149067,183.440019)(653.5814917,183.47002441)
\curveto(653.54149075,183.50001894)(653.51149078,183.55001889)(653.4914917,183.62002441)
\curveto(653.47149082,183.69001875)(653.47149082,183.76501867)(653.4914917,183.84502441)
\curveto(653.52149077,183.97501846)(653.55149074,184.09501834)(653.5814917,184.20502441)
\curveto(653.62149067,184.32501811)(653.66649062,184.440018)(653.7164917,184.55002441)
\curveto(653.90649038,184.90001754)(654.14649014,185.17001727)(654.4364917,185.36002441)
\curveto(654.72648956,185.56001688)(655.0864892,185.72001672)(655.5164917,185.84002441)
\curveto(655.61648867,185.86001658)(655.71648857,185.87501656)(655.8164917,185.88502441)
\curveto(655.92648836,185.89501654)(656.03648825,185.91001653)(656.1464917,185.93002441)
\curveto(656.1864881,185.9400165)(656.25148804,185.9400165)(656.3414917,185.93002441)
\curveto(656.43148786,185.93001651)(656.4864878,185.9400165)(656.5064917,185.96002441)
\curveto(657.20648708,185.97001647)(657.81648647,185.89001655)(658.3364917,185.72002441)
\curveto(658.85648543,185.55001689)(659.22148507,185.22501721)(659.4314917,184.74502441)
\curveto(659.52148477,184.54501789)(659.57148472,184.31001813)(659.5814917,184.04002441)
\curveto(659.60148469,183.78001866)(659.61148468,183.50501893)(659.6114917,183.21502441)
\lineto(659.6114917,179.90002441)
\curveto(659.61148468,179.76002268)(659.61648467,179.62502281)(659.6264917,179.49502441)
\curveto(659.63648465,179.36502307)(659.66648462,179.26002318)(659.7164917,179.18002441)
\curveto(659.76648452,179.11002333)(659.83148446,179.06002338)(659.9114917,179.03002441)
\curveto(660.00148429,178.99002345)(660.0864842,178.96002348)(660.1664917,178.94002441)
\curveto(660.24648404,178.93002351)(660.30648398,178.88502355)(660.3464917,178.80502441)
\curveto(660.36648392,178.77502366)(660.37648391,178.74502369)(660.3764917,178.71502441)
\curveto(660.37648391,178.68502375)(660.38148391,178.64502379)(660.3914917,178.59502441)
\moveto(658.2464917,180.26002441)
\curveto(658.30648598,180.40002204)(658.33648595,180.56002188)(658.3364917,180.74002441)
\curveto(658.34648594,180.93002151)(658.35148594,181.12502131)(658.3514917,181.32502441)
\curveto(658.35148594,181.435021)(658.34648594,181.5350209)(658.3364917,181.62502441)
\curveto(658.32648596,181.71502072)(658.286486,181.78502065)(658.2164917,181.83502441)
\curveto(658.1864861,181.85502058)(658.11648617,181.86502057)(658.0064917,181.86502441)
\curveto(657.9864863,181.84502059)(657.95148634,181.8350206)(657.9014917,181.83502441)
\curveto(657.85148644,181.8350206)(657.80648648,181.82502061)(657.7664917,181.80502441)
\curveto(657.6864866,181.78502065)(657.59648669,181.76502067)(657.4964917,181.74502441)
\lineto(657.1964917,181.68502441)
\curveto(657.16648712,181.68502075)(657.13148716,181.68002076)(657.0914917,181.67002441)
\lineto(656.9864917,181.67002441)
\curveto(656.83648745,181.63002081)(656.67148762,181.60502083)(656.4914917,181.59502441)
\curveto(656.32148797,181.59502084)(656.16148813,181.57502086)(656.0114917,181.53502441)
\curveto(655.93148836,181.51502092)(655.85648843,181.49502094)(655.7864917,181.47502441)
\curveto(655.72648856,181.46502097)(655.65648863,181.45002099)(655.5764917,181.43002441)
\curveto(655.41648887,181.38002106)(655.26648902,181.31502112)(655.1264917,181.23502441)
\curveto(654.9864893,181.16502127)(654.86648942,181.07502136)(654.7664917,180.96502441)
\curveto(654.66648962,180.85502158)(654.5914897,180.72002172)(654.5414917,180.56002441)
\curveto(654.4914898,180.41002203)(654.47148982,180.22502221)(654.4814917,180.00502441)
\curveto(654.48148981,179.90502253)(654.49648979,179.81002263)(654.5264917,179.72002441)
\curveto(654.56648972,179.6400228)(654.61148968,179.56502287)(654.6614917,179.49502441)
\curveto(654.74148955,179.38502305)(654.84648944,179.29002315)(654.9764917,179.21002441)
\curveto(655.10648918,179.1400233)(655.24648904,179.08002336)(655.3964917,179.03002441)
\curveto(655.44648884,179.02002342)(655.49648879,179.01502342)(655.5464917,179.01502441)
\curveto(655.59648869,179.01502342)(655.64648864,179.01002343)(655.6964917,179.00002441)
\curveto(655.76648852,178.98002346)(655.85148844,178.96502347)(655.9514917,178.95502441)
\curveto(656.06148823,178.95502348)(656.15148814,178.96502347)(656.2214917,178.98502441)
\curveto(656.28148801,179.00502343)(656.34148795,179.01002343)(656.4014917,179.00002441)
\curveto(656.46148783,179.00002344)(656.52148777,179.01002343)(656.5814917,179.03002441)
\curveto(656.66148763,179.05002339)(656.73648755,179.06502337)(656.8064917,179.07502441)
\curveto(656.8864874,179.08502335)(656.96148733,179.10502333)(657.0314917,179.13502441)
\curveto(657.32148697,179.25502318)(657.56648672,179.40002304)(657.7664917,179.57002441)
\curveto(657.97648631,179.7400227)(658.13648615,179.97002247)(658.2464917,180.26002441)
}
}
{
\newrgbcolor{curcolor}{0 0 0}
\pscustom[linestyle=none,fillstyle=solid,fillcolor=curcolor]
{
\newpath
\moveto(668.52313232,178.85002441)
\lineto(668.52313232,178.46002441)
\curveto(668.52312445,178.3400241)(668.49812447,178.2400242)(668.44813232,178.16002441)
\curveto(668.39812457,178.09002435)(668.31312466,178.05002439)(668.19313232,178.04002441)
\lineto(667.84813232,178.04002441)
\curveto(667.78812518,178.0400244)(667.72812524,178.0350244)(667.66813232,178.02502441)
\curveto(667.61812535,178.02502441)(667.5731254,178.0350244)(667.53313232,178.05502441)
\curveto(667.44312553,178.07502436)(667.38312559,178.11502432)(667.35313232,178.17502441)
\curveto(667.31312566,178.22502421)(667.28812568,178.28502415)(667.27813232,178.35502441)
\curveto(667.27812569,178.42502401)(667.26312571,178.49502394)(667.23313232,178.56502441)
\curveto(667.22312575,178.58502385)(667.20812576,178.60002384)(667.18813232,178.61002441)
\curveto(667.17812579,178.63002381)(667.16312581,178.65002379)(667.14313232,178.67002441)
\curveto(667.04312593,178.68002376)(666.96312601,178.66002378)(666.90313232,178.61002441)
\curveto(666.85312612,178.56002388)(666.79812617,178.51002393)(666.73813232,178.46002441)
\curveto(666.53812643,178.31002413)(666.33812663,178.19502424)(666.13813232,178.11502441)
\curveto(665.95812701,178.0350244)(665.74812722,177.97502446)(665.50813232,177.93502441)
\curveto(665.27812769,177.89502454)(665.03812793,177.87502456)(664.78813232,177.87502441)
\curveto(664.54812842,177.86502457)(664.30812866,177.88002456)(664.06813232,177.92002441)
\curveto(663.82812914,177.95002449)(663.61812935,178.00502443)(663.43813232,178.08502441)
\curveto(662.91813005,178.30502413)(662.49813047,178.60002384)(662.17813232,178.97002441)
\curveto(661.85813111,179.35002309)(661.60813136,179.82002262)(661.42813232,180.38002441)
\curveto(661.38813158,180.47002197)(661.35813161,180.56002188)(661.33813232,180.65002441)
\curveto(661.32813164,180.75002169)(661.30813166,180.85002159)(661.27813232,180.95002441)
\curveto(661.2681317,181.00002144)(661.26313171,181.05002139)(661.26313232,181.10002441)
\curveto(661.26313171,181.15002129)(661.25813171,181.20002124)(661.24813232,181.25002441)
\curveto(661.22813174,181.30002114)(661.21813175,181.35002109)(661.21813232,181.40002441)
\curveto(661.22813174,181.46002098)(661.22813174,181.51502092)(661.21813232,181.56502441)
\lineto(661.21813232,181.71502441)
\curveto(661.19813177,181.76502067)(661.18813178,181.83002061)(661.18813232,181.91002441)
\curveto(661.18813178,181.99002045)(661.19813177,182.05502038)(661.21813232,182.10502441)
\lineto(661.21813232,182.27002441)
\curveto(661.23813173,182.3400201)(661.24313173,182.41002003)(661.23313232,182.48002441)
\curveto(661.23313174,182.56001988)(661.24313173,182.6350198)(661.26313232,182.70502441)
\curveto(661.2731317,182.75501968)(661.27813169,182.80001964)(661.27813232,182.84002441)
\curveto(661.27813169,182.88001956)(661.28313169,182.92501951)(661.29313232,182.97502441)
\curveto(661.32313165,183.07501936)(661.34813162,183.17001927)(661.36813232,183.26002441)
\curveto(661.38813158,183.36001908)(661.41313156,183.45501898)(661.44313232,183.54502441)
\curveto(661.5731314,183.92501851)(661.73813123,184.26501817)(661.93813232,184.56502441)
\curveto(662.14813082,184.87501756)(662.39813057,185.13001731)(662.68813232,185.33002441)
\curveto(662.85813011,185.45001699)(663.03312994,185.55001689)(663.21313232,185.63002441)
\curveto(663.40312957,185.71001673)(663.60812936,185.78001666)(663.82813232,185.84002441)
\curveto(663.89812907,185.85001659)(663.96312901,185.86001658)(664.02313232,185.87002441)
\curveto(664.09312888,185.88001656)(664.16312881,185.89501654)(664.23313232,185.91502441)
\lineto(664.38313232,185.91502441)
\curveto(664.46312851,185.9350165)(664.57812839,185.94501649)(664.72813232,185.94502441)
\curveto(664.88812808,185.94501649)(665.00812796,185.9350165)(665.08813232,185.91502441)
\curveto(665.12812784,185.90501653)(665.18312779,185.90001654)(665.25313232,185.90002441)
\curveto(665.36312761,185.87001657)(665.4731275,185.84501659)(665.58313232,185.82502441)
\curveto(665.69312728,185.81501662)(665.79812717,185.78501665)(665.89813232,185.73502441)
\curveto(666.04812692,185.67501676)(666.18812678,185.61001683)(666.31813232,185.54002441)
\curveto(666.45812651,185.47001697)(666.58812638,185.39001705)(666.70813232,185.30002441)
\curveto(666.7681262,185.25001719)(666.82812614,185.19501724)(666.88813232,185.13502441)
\curveto(666.95812601,185.08501735)(667.04812592,185.07001737)(667.15813232,185.09002441)
\curveto(667.17812579,185.12001732)(667.19312578,185.14501729)(667.20313232,185.16502441)
\curveto(667.22312575,185.18501725)(667.23812573,185.21501722)(667.24813232,185.25502441)
\curveto(667.27812569,185.34501709)(667.28812568,185.46001698)(667.27813232,185.60002441)
\lineto(667.27813232,185.97502441)
\lineto(667.27813232,187.70002441)
\lineto(667.27813232,188.16502441)
\curveto(667.27812569,188.34501409)(667.30312567,188.47501396)(667.35313232,188.55502441)
\curveto(667.39312558,188.62501381)(667.45312552,188.67001377)(667.53313232,188.69002441)
\curveto(667.55312542,188.69001375)(667.57812539,188.69001375)(667.60813232,188.69002441)
\curveto(667.63812533,188.70001374)(667.66312531,188.70501373)(667.68313232,188.70502441)
\curveto(667.82312515,188.71501372)(667.968125,188.71501372)(668.11813232,188.70502441)
\curveto(668.27812469,188.70501373)(668.38812458,188.66501377)(668.44813232,188.58502441)
\curveto(668.49812447,188.50501393)(668.52312445,188.40501403)(668.52313232,188.28502441)
\lineto(668.52313232,187.91002441)
\lineto(668.52313232,178.85002441)
\moveto(667.30813232,181.68502441)
\curveto(667.32812564,181.7350207)(667.33812563,181.80002064)(667.33813232,181.88002441)
\curveto(667.33812563,181.97002047)(667.32812564,182.0400204)(667.30813232,182.09002441)
\lineto(667.30813232,182.31502441)
\curveto(667.28812568,182.40502003)(667.2731257,182.49501994)(667.26313232,182.58502441)
\curveto(667.25312572,182.68501975)(667.23312574,182.77501966)(667.20313232,182.85502441)
\curveto(667.18312579,182.9350195)(667.16312581,183.01001943)(667.14313232,183.08002441)
\curveto(667.13312584,183.15001929)(667.11312586,183.22001922)(667.08313232,183.29002441)
\curveto(666.96312601,183.59001885)(666.80812616,183.85501858)(666.61813232,184.08502441)
\curveto(666.42812654,184.31501812)(666.18812678,184.49501794)(665.89813232,184.62502441)
\curveto(665.79812717,184.67501776)(665.69312728,184.71001773)(665.58313232,184.73002441)
\curveto(665.48312749,184.76001768)(665.3731276,184.78501765)(665.25313232,184.80502441)
\curveto(665.1731278,184.82501761)(665.08312789,184.8350176)(664.98313232,184.83502441)
\lineto(664.71313232,184.83502441)
\curveto(664.66312831,184.82501761)(664.61812835,184.81501762)(664.57813232,184.80502441)
\lineto(664.44313232,184.80502441)
\curveto(664.36312861,184.78501765)(664.27812869,184.76501767)(664.18813232,184.74502441)
\curveto(664.10812886,184.72501771)(664.02812894,184.70001774)(663.94813232,184.67002441)
\curveto(663.62812934,184.53001791)(663.3681296,184.32501811)(663.16813232,184.05502441)
\curveto(662.97812999,183.79501864)(662.82313015,183.49001895)(662.70313232,183.14002441)
\curveto(662.66313031,183.03001941)(662.63313034,182.91501952)(662.61313232,182.79502441)
\curveto(662.60313037,182.68501975)(662.58813038,182.57501986)(662.56813232,182.46502441)
\curveto(662.5681304,182.42502001)(662.56313041,182.38502005)(662.55313232,182.34502441)
\lineto(662.55313232,182.24002441)
\curveto(662.53313044,182.19002025)(662.52313045,182.1350203)(662.52313232,182.07502441)
\curveto(662.53313044,182.01502042)(662.53813043,181.96002048)(662.53813232,181.91002441)
\lineto(662.53813232,181.58002441)
\curveto(662.53813043,181.48002096)(662.54813042,181.38502105)(662.56813232,181.29502441)
\curveto(662.57813039,181.26502117)(662.58313039,181.21502122)(662.58313232,181.14502441)
\curveto(662.60313037,181.07502136)(662.61813035,181.00502143)(662.62813232,180.93502441)
\lineto(662.68813232,180.72502441)
\curveto(662.79813017,180.37502206)(662.94813002,180.07502236)(663.13813232,179.82502441)
\curveto(663.32812964,179.57502286)(663.5681294,179.37002307)(663.85813232,179.21002441)
\curveto(663.94812902,179.16002328)(664.03812893,179.12002332)(664.12813232,179.09002441)
\curveto(664.21812875,179.06002338)(664.31812865,179.03002341)(664.42813232,179.00002441)
\curveto(664.47812849,178.98002346)(664.52812844,178.97502346)(664.57813232,178.98502441)
\curveto(664.63812833,178.99502344)(664.69312828,178.99002345)(664.74313232,178.97002441)
\curveto(664.78312819,178.96002348)(664.82312815,178.95502348)(664.86313232,178.95502441)
\lineto(664.99813232,178.95502441)
\lineto(665.13313232,178.95502441)
\curveto(665.16312781,178.96502347)(665.21312776,178.97002347)(665.28313232,178.97002441)
\curveto(665.36312761,178.99002345)(665.44312753,179.00502343)(665.52313232,179.01502441)
\curveto(665.60312737,179.0350234)(665.67812729,179.06002338)(665.74813232,179.09002441)
\curveto(666.07812689,179.23002321)(666.34312663,179.40502303)(666.54313232,179.61502441)
\curveto(666.75312622,179.8350226)(666.92812604,180.11002233)(667.06813232,180.44002441)
\curveto(667.11812585,180.55002189)(667.15312582,180.66002178)(667.17313232,180.77002441)
\curveto(667.19312578,180.88002156)(667.21812575,180.99002145)(667.24813232,181.10002441)
\curveto(667.2681257,181.1400213)(667.27812569,181.17502126)(667.27813232,181.20502441)
\curveto(667.27812569,181.24502119)(667.28312569,181.28502115)(667.29313232,181.32502441)
\curveto(667.30312567,181.38502105)(667.30312567,181.44502099)(667.29313232,181.50502441)
\curveto(667.29312568,181.56502087)(667.29812567,181.62502081)(667.30813232,181.68502441)
}
}
{
\newrgbcolor{curcolor}{0 0 0}
\pscustom[linestyle=none,fillstyle=solid,fillcolor=curcolor]
{
\newpath
\moveto(677.59438232,182.24002441)
\curveto(677.61437426,182.18002026)(677.62437425,182.08502035)(677.62438232,181.95502441)
\curveto(677.62437425,181.8350206)(677.61937426,181.75002069)(677.60938232,181.70002441)
\lineto(677.60938232,181.55002441)
\curveto(677.59937428,181.47002097)(677.58937429,181.39502104)(677.57938232,181.32502441)
\curveto(677.5793743,181.26502117)(677.5743743,181.19502124)(677.56438232,181.11502441)
\curveto(677.54437433,181.05502138)(677.52937435,180.99502144)(677.51938232,180.93502441)
\curveto(677.51937436,180.87502156)(677.50937437,180.81502162)(677.48938232,180.75502441)
\curveto(677.44937443,180.62502181)(677.41437446,180.49502194)(677.38438232,180.36502441)
\curveto(677.35437452,180.2350222)(677.31437456,180.11502232)(677.26438232,180.00502441)
\curveto(677.05437482,179.52502291)(676.7743751,179.12002332)(676.42438232,178.79002441)
\curveto(676.0743758,178.47002397)(675.64437623,178.22502421)(675.13438232,178.05502441)
\curveto(675.02437685,178.01502442)(674.90437697,177.98502445)(674.77438232,177.96502441)
\curveto(674.65437722,177.94502449)(674.52937735,177.92502451)(674.39938232,177.90502441)
\curveto(674.33937754,177.89502454)(674.2743776,177.89002455)(674.20438232,177.89002441)
\curveto(674.14437773,177.88002456)(674.08437779,177.87502456)(674.02438232,177.87502441)
\curveto(673.98437789,177.86502457)(673.92437795,177.86002458)(673.84438232,177.86002441)
\curveto(673.7743781,177.86002458)(673.72437815,177.86502457)(673.69438232,177.87502441)
\curveto(673.65437822,177.88502455)(673.61437826,177.89002455)(673.57438232,177.89002441)
\curveto(673.53437834,177.88002456)(673.49937838,177.88002456)(673.46938232,177.89002441)
\lineto(673.37938232,177.89002441)
\lineto(673.01938232,177.93502441)
\curveto(672.879379,177.97502446)(672.74437913,178.01502442)(672.61438232,178.05502441)
\curveto(672.48437939,178.09502434)(672.35937952,178.1400243)(672.23938232,178.19002441)
\curveto(671.78938009,178.39002405)(671.41938046,178.65002379)(671.12938232,178.97002441)
\curveto(670.83938104,179.29002315)(670.59938128,179.68002276)(670.40938232,180.14002441)
\curveto(670.35938152,180.2400222)(670.31938156,180.3400221)(670.28938232,180.44002441)
\curveto(670.26938161,180.5400219)(670.24938163,180.64502179)(670.22938232,180.75502441)
\curveto(670.20938167,180.79502164)(670.19938168,180.82502161)(670.19938232,180.84502441)
\curveto(670.20938167,180.87502156)(670.20938167,180.91002153)(670.19938232,180.95002441)
\curveto(670.1793817,181.03002141)(670.16438171,181.11002133)(670.15438232,181.19002441)
\curveto(670.15438172,181.28002116)(670.14438173,181.36502107)(670.12438232,181.44502441)
\lineto(670.12438232,181.56502441)
\curveto(670.12438175,181.60502083)(670.11938176,181.65002079)(670.10938232,181.70002441)
\curveto(670.09938178,181.75002069)(670.09438178,181.8350206)(670.09438232,181.95502441)
\curveto(670.09438178,182.08502035)(670.10438177,182.18002026)(670.12438232,182.24002441)
\curveto(670.14438173,182.31002013)(670.14938173,182.38002006)(670.13938232,182.45002441)
\curveto(670.12938175,182.52001992)(670.13438174,182.59001985)(670.15438232,182.66002441)
\curveto(670.16438171,182.71001973)(670.16938171,182.75001969)(670.16938232,182.78002441)
\curveto(670.1793817,182.82001962)(670.18938169,182.86501957)(670.19938232,182.91502441)
\curveto(670.22938165,183.0350194)(670.25438162,183.15501928)(670.27438232,183.27502441)
\curveto(670.30438157,183.39501904)(670.34438153,183.51001893)(670.39438232,183.62002441)
\curveto(670.54438133,183.99001845)(670.72438115,184.32001812)(670.93438232,184.61002441)
\curveto(671.15438072,184.91001753)(671.41938046,185.16001728)(671.72938232,185.36002441)
\curveto(671.84938003,185.440017)(671.9743799,185.50501693)(672.10438232,185.55502441)
\curveto(672.23437964,185.61501682)(672.36937951,185.67501676)(672.50938232,185.73502441)
\curveto(672.62937925,185.78501665)(672.75937912,185.81501662)(672.89938232,185.82502441)
\curveto(673.03937884,185.84501659)(673.1793787,185.87501656)(673.31938232,185.91502441)
\lineto(673.51438232,185.91502441)
\curveto(673.58437829,185.92501651)(673.64937823,185.9350165)(673.70938232,185.94502441)
\curveto(674.59937728,185.95501648)(675.33937654,185.77001667)(675.92938232,185.39002441)
\curveto(676.51937536,185.01001743)(676.94437493,184.51501792)(677.20438232,183.90502441)
\curveto(677.25437462,183.80501863)(677.29437458,183.70501873)(677.32438232,183.60502441)
\curveto(677.35437452,183.50501893)(677.38937449,183.40001904)(677.42938232,183.29002441)
\curveto(677.45937442,183.18001926)(677.48437439,183.06001938)(677.50438232,182.93002441)
\curveto(677.52437435,182.81001963)(677.54937433,182.68501975)(677.57938232,182.55502441)
\curveto(677.58937429,182.50501993)(677.58937429,182.45001999)(677.57938232,182.39002441)
\curveto(677.5793743,182.3400201)(677.58437429,182.29002015)(677.59438232,182.24002441)
\moveto(676.25938232,181.38502441)
\curveto(676.2793756,181.45502098)(676.28437559,181.5350209)(676.27438232,181.62502441)
\lineto(676.27438232,181.88002441)
\curveto(676.2743756,182.27002017)(676.23937564,182.60001984)(676.16938232,182.87002441)
\curveto(676.13937574,182.95001949)(676.11437576,183.03001941)(676.09438232,183.11002441)
\curveto(676.0743758,183.19001925)(676.04937583,183.26501917)(676.01938232,183.33502441)
\curveto(675.73937614,183.98501845)(675.29437658,184.435018)(674.68438232,184.68502441)
\curveto(674.61437726,184.71501772)(674.53937734,184.7350177)(674.45938232,184.74502441)
\lineto(674.21938232,184.80502441)
\curveto(674.13937774,184.82501761)(674.05437782,184.8350176)(673.96438232,184.83502441)
\lineto(673.69438232,184.83502441)
\lineto(673.42438232,184.79002441)
\curveto(673.32437855,184.77001767)(673.22937865,184.74501769)(673.13938232,184.71502441)
\curveto(673.05937882,184.69501774)(672.9793789,184.66501777)(672.89938232,184.62502441)
\curveto(672.82937905,184.60501783)(672.76437911,184.57501786)(672.70438232,184.53502441)
\curveto(672.64437923,184.49501794)(672.58937929,184.45501798)(672.53938232,184.41502441)
\curveto(672.29937958,184.24501819)(672.10437977,184.0400184)(671.95438232,183.80002441)
\curveto(671.80438007,183.56001888)(671.6743802,183.28001916)(671.56438232,182.96002441)
\curveto(671.53438034,182.86001958)(671.51438036,182.75501968)(671.50438232,182.64502441)
\curveto(671.49438038,182.54501989)(671.4793804,182.44002)(671.45938232,182.33002441)
\curveto(671.44938043,182.29002015)(671.44438043,182.22502021)(671.44438232,182.13502441)
\curveto(671.43438044,182.10502033)(671.42938045,182.07002037)(671.42938232,182.03002441)
\curveto(671.43938044,181.99002045)(671.44438043,181.94502049)(671.44438232,181.89502441)
\lineto(671.44438232,181.59502441)
\curveto(671.44438043,181.49502094)(671.45438042,181.40502103)(671.47438232,181.32502441)
\lineto(671.50438232,181.14502441)
\curveto(671.52438035,181.04502139)(671.53938034,180.94502149)(671.54938232,180.84502441)
\curveto(671.56938031,180.75502168)(671.59938028,180.67002177)(671.63938232,180.59002441)
\curveto(671.73938014,180.35002209)(671.85438002,180.12502231)(671.98438232,179.91502441)
\curveto(672.12437975,179.70502273)(672.29437958,179.53002291)(672.49438232,179.39002441)
\curveto(672.54437933,179.36002308)(672.58937929,179.3350231)(672.62938232,179.31502441)
\curveto(672.66937921,179.29502314)(672.71437916,179.27002317)(672.76438232,179.24002441)
\curveto(672.84437903,179.19002325)(672.92937895,179.14502329)(673.01938232,179.10502441)
\curveto(673.11937876,179.07502336)(673.22437865,179.04502339)(673.33438232,179.01502441)
\curveto(673.38437849,178.99502344)(673.42937845,178.98502345)(673.46938232,178.98502441)
\curveto(673.51937836,178.99502344)(673.56937831,178.99502344)(673.61938232,178.98502441)
\curveto(673.64937823,178.97502346)(673.70937817,178.96502347)(673.79938232,178.95502441)
\curveto(673.89937798,178.94502349)(673.9743779,178.95002349)(674.02438232,178.97002441)
\curveto(674.06437781,178.98002346)(674.10437777,178.98002346)(674.14438232,178.97002441)
\curveto(674.18437769,178.97002347)(674.22437765,178.98002346)(674.26438232,179.00002441)
\curveto(674.34437753,179.02002342)(674.42437745,179.0350234)(674.50438232,179.04502441)
\curveto(674.58437729,179.06502337)(674.65937722,179.09002335)(674.72938232,179.12002441)
\curveto(675.06937681,179.26002318)(675.34437653,179.45502298)(675.55438232,179.70502441)
\curveto(675.76437611,179.95502248)(675.93937594,180.25002219)(676.07938232,180.59002441)
\curveto(676.12937575,180.71002173)(676.15937572,180.8350216)(676.16938232,180.96502441)
\curveto(676.18937569,181.10502133)(676.21937566,181.24502119)(676.25938232,181.38502441)
}
}
{
\newrgbcolor{curcolor}{0.90196079 0.90196079 0.90196079}
\pscustom[linestyle=none,fillstyle=solid,fillcolor=curcolor]
{
\newpath
\moveto(606.37554932,188.75006104)
\lineto(621.37554932,188.75006104)
\lineto(621.37554932,173.75006104)
\lineto(606.37554932,173.75006104)
\closepath
}
}
{
\newrgbcolor{curcolor}{0 0 0}
\pscustom[linestyle=none,fillstyle=solid,fillcolor=curcolor]
{
\newpath
\moveto(626.36375732,165.68431885)
\lineto(631.26875732,165.68431885)
\lineto(632.55875732,165.68431885)
\curveto(632.66874944,165.68430815)(632.77874933,165.68430815)(632.88875732,165.68431885)
\curveto(632.99874911,165.69430814)(633.08874902,165.67430816)(633.15875732,165.62431885)
\curveto(633.18874892,165.60430823)(633.2137489,165.57930826)(633.23375732,165.54931885)
\curveto(633.25374886,165.51930832)(633.27374884,165.48930835)(633.29375732,165.45931885)
\curveto(633.3137488,165.38930845)(633.32374879,165.27430856)(633.32375732,165.11431885)
\curveto(633.32374879,164.96430887)(633.3137488,164.84930899)(633.29375732,164.76931885)
\curveto(633.25374886,164.62930921)(633.16874894,164.54930929)(633.03875732,164.52931885)
\curveto(632.9087492,164.51930932)(632.75374936,164.51430932)(632.57375732,164.51431885)
\lineto(631.07375732,164.51431885)
\lineto(628.55375732,164.51431885)
\lineto(627.98375732,164.51431885)
\curveto(627.77375434,164.52430931)(627.61875449,164.49930934)(627.51875732,164.43931885)
\curveto(627.41875469,164.37930946)(627.36375475,164.27430956)(627.35375732,164.12431885)
\lineto(627.35375732,163.65931885)
\lineto(627.35375732,162.12931885)
\curveto(627.35375476,162.01931182)(627.34875476,161.88931195)(627.33875732,161.73931885)
\curveto(627.33875477,161.58931225)(627.34875476,161.46931237)(627.36875732,161.37931885)
\curveto(627.39875471,161.25931258)(627.45875465,161.17931266)(627.54875732,161.13931885)
\curveto(627.58875452,161.11931272)(627.65875445,161.09931274)(627.75875732,161.07931885)
\lineto(627.90875732,161.07931885)
\curveto(627.94875416,161.06931277)(627.98875412,161.06431277)(628.02875732,161.06431885)
\curveto(628.07875403,161.07431276)(628.12875398,161.07931276)(628.17875732,161.07931885)
\lineto(628.68875732,161.07931885)
\lineto(631.62875732,161.07931885)
\lineto(631.92875732,161.07931885)
\curveto(632.03875007,161.08931275)(632.14874996,161.08931275)(632.25875732,161.07931885)
\curveto(632.37874973,161.07931276)(632.48374963,161.06931277)(632.57375732,161.04931885)
\curveto(632.67374944,161.0393128)(632.74874936,161.01931282)(632.79875732,160.98931885)
\curveto(632.82874928,160.96931287)(632.85374926,160.92431291)(632.87375732,160.85431885)
\curveto(632.89374922,160.78431305)(632.9087492,160.70931313)(632.91875732,160.62931885)
\curveto(632.92874918,160.54931329)(632.92874918,160.46431337)(632.91875732,160.37431885)
\curveto(632.91874919,160.29431354)(632.9087492,160.22431361)(632.88875732,160.16431885)
\curveto(632.86874924,160.07431376)(632.82374929,160.00931383)(632.75375732,159.96931885)
\curveto(632.73374938,159.94931389)(632.70374941,159.9343139)(632.66375732,159.92431885)
\curveto(632.63374948,159.92431391)(632.60374951,159.91931392)(632.57375732,159.90931885)
\lineto(632.48375732,159.90931885)
\curveto(632.43374968,159.89931394)(632.38374973,159.89431394)(632.33375732,159.89431885)
\curveto(632.28374983,159.90431393)(632.23374988,159.90931393)(632.18375732,159.90931885)
\lineto(631.62875732,159.90931885)
\lineto(628.46375732,159.90931885)
\lineto(628.10375732,159.90931885)
\curveto(627.99375412,159.91931392)(627.88875422,159.91431392)(627.78875732,159.89431885)
\curveto(627.68875442,159.88431395)(627.59875451,159.85931398)(627.51875732,159.81931885)
\curveto(627.44875466,159.77931406)(627.39875471,159.70931413)(627.36875732,159.60931885)
\curveto(627.34875476,159.54931429)(627.33875477,159.47931436)(627.33875732,159.39931885)
\curveto(627.34875476,159.31931452)(627.35375476,159.2393146)(627.35375732,159.15931885)
\lineto(627.35375732,158.31931885)
\lineto(627.35375732,156.89431885)
\curveto(627.35375476,156.75431708)(627.35875475,156.62431721)(627.36875732,156.50431885)
\curveto(627.37875473,156.39431744)(627.41875469,156.31431752)(627.48875732,156.26431885)
\curveto(627.55875455,156.21431762)(627.63875447,156.18431765)(627.72875732,156.17431885)
\lineto(628.02875732,156.17431885)
\lineto(628.98875732,156.17431885)
\lineto(631.76375732,156.17431885)
\lineto(632.61875732,156.17431885)
\lineto(632.85875732,156.17431885)
\curveto(632.93874917,156.18431765)(633.0087491,156.17931766)(633.06875732,156.15931885)
\curveto(633.18874892,156.11931772)(633.26874884,156.06431777)(633.30875732,155.99431885)
\curveto(633.32874878,155.96431787)(633.34374877,155.91431792)(633.35375732,155.84431885)
\curveto(633.36374875,155.77431806)(633.36874874,155.69931814)(633.36875732,155.61931885)
\curveto(633.37874873,155.54931829)(633.37874873,155.47431836)(633.36875732,155.39431885)
\curveto(633.35874875,155.32431851)(633.34874876,155.26931857)(633.33875732,155.22931885)
\curveto(633.29874881,155.14931869)(633.25374886,155.09431874)(633.20375732,155.06431885)
\curveto(633.14374897,155.02431881)(633.06374905,155.00431883)(632.96375732,155.00431885)
\lineto(632.69375732,155.00431885)
\lineto(631.64375732,155.00431885)
\lineto(627.65375732,155.00431885)
\lineto(626.60375732,155.00431885)
\curveto(626.46375565,155.00431883)(626.34375577,155.00931883)(626.24375732,155.01931885)
\curveto(626.14375597,155.0393188)(626.06875604,155.08931875)(626.01875732,155.16931885)
\curveto(625.97875613,155.22931861)(625.95875615,155.30431853)(625.95875732,155.39431885)
\lineto(625.95875732,155.67931885)
\lineto(625.95875732,156.72931885)
\lineto(625.95875732,160.74931885)
\lineto(625.95875732,164.10931885)
\lineto(625.95875732,165.03931885)
\lineto(625.95875732,165.30931885)
\curveto(625.95875615,165.39930844)(625.97875613,165.46930837)(626.01875732,165.51931885)
\curveto(626.05875605,165.58930825)(626.13375598,165.6393082)(626.24375732,165.66931885)
\curveto(626.26375585,165.67930816)(626.28375583,165.67930816)(626.30375732,165.66931885)
\curveto(626.32375579,165.66930817)(626.34375577,165.67430816)(626.36375732,165.68431885)
}
}
{
\newrgbcolor{curcolor}{0 0 0}
\pscustom[linestyle=none,fillstyle=solid,fillcolor=curcolor]
{
\newpath
\moveto(637.3036792,162.90931885)
\curveto(638.02367513,162.91931092)(638.62867453,162.834311)(639.1186792,162.65431885)
\curveto(639.60867355,162.48431135)(639.98867317,162.17931166)(640.2586792,161.73931885)
\curveto(640.32867283,161.62931221)(640.38367277,161.51431232)(640.4236792,161.39431885)
\curveto(640.46367269,161.28431255)(640.50367265,161.15931268)(640.5436792,161.01931885)
\curveto(640.56367259,160.94931289)(640.56867259,160.87431296)(640.5586792,160.79431885)
\curveto(640.54867261,160.72431311)(640.53367262,160.66931317)(640.5136792,160.62931885)
\curveto(640.49367266,160.60931323)(640.46867269,160.58931325)(640.4386792,160.56931885)
\curveto(640.40867275,160.55931328)(640.38367277,160.54431329)(640.3636792,160.52431885)
\curveto(640.31367284,160.50431333)(640.26367289,160.49931334)(640.2136792,160.50931885)
\curveto(640.16367299,160.51931332)(640.11367304,160.51931332)(640.0636792,160.50931885)
\curveto(639.98367317,160.48931335)(639.87867328,160.48431335)(639.7486792,160.49431885)
\curveto(639.61867354,160.51431332)(639.52867363,160.5393133)(639.4786792,160.56931885)
\curveto(639.39867376,160.61931322)(639.34367381,160.68431315)(639.3136792,160.76431885)
\curveto(639.29367386,160.85431298)(639.2586739,160.9393129)(639.2086792,161.01931885)
\curveto(639.11867404,161.17931266)(638.99367416,161.32431251)(638.8336792,161.45431885)
\curveto(638.72367443,161.5343123)(638.60367455,161.59431224)(638.4736792,161.63431885)
\curveto(638.34367481,161.67431216)(638.20367495,161.71431212)(638.0536792,161.75431885)
\curveto(638.00367515,161.77431206)(637.9536752,161.77931206)(637.9036792,161.76931885)
\curveto(637.8536753,161.76931207)(637.80367535,161.77431206)(637.7536792,161.78431885)
\curveto(637.69367546,161.80431203)(637.61867554,161.81431202)(637.5286792,161.81431885)
\curveto(637.43867572,161.81431202)(637.36367579,161.80431203)(637.3036792,161.78431885)
\lineto(637.2136792,161.78431885)
\lineto(637.0636792,161.75431885)
\curveto(637.01367614,161.75431208)(636.96367619,161.74931209)(636.9136792,161.73931885)
\curveto(636.6536765,161.67931216)(636.43867672,161.59431224)(636.2686792,161.48431885)
\curveto(636.09867706,161.37431246)(635.98367717,161.18931265)(635.9236792,160.92931885)
\curveto(635.90367725,160.85931298)(635.89867726,160.78931305)(635.9086792,160.71931885)
\curveto(635.92867723,160.64931319)(635.94867721,160.58931325)(635.9686792,160.53931885)
\curveto(636.02867713,160.38931345)(636.09867706,160.27931356)(636.1786792,160.20931885)
\curveto(636.26867689,160.14931369)(636.37867678,160.07931376)(636.5086792,159.99931885)
\curveto(636.66867649,159.89931394)(636.84867631,159.82431401)(637.0486792,159.77431885)
\curveto(637.24867591,159.7343141)(637.44867571,159.68431415)(637.6486792,159.62431885)
\curveto(637.77867538,159.58431425)(637.90867525,159.55431428)(638.0386792,159.53431885)
\curveto(638.16867499,159.51431432)(638.29867486,159.48431435)(638.4286792,159.44431885)
\curveto(638.63867452,159.38431445)(638.84367431,159.32431451)(639.0436792,159.26431885)
\curveto(639.24367391,159.21431462)(639.44367371,159.14931469)(639.6436792,159.06931885)
\lineto(639.7936792,159.00931885)
\curveto(639.84367331,158.98931485)(639.89367326,158.96431487)(639.9436792,158.93431885)
\curveto(640.14367301,158.81431502)(640.31867284,158.67931516)(640.4686792,158.52931885)
\curveto(640.61867254,158.37931546)(640.74367241,158.18931565)(640.8436792,157.95931885)
\curveto(640.86367229,157.88931595)(640.88367227,157.79431604)(640.9036792,157.67431885)
\curveto(640.92367223,157.60431623)(640.93367222,157.52931631)(640.9336792,157.44931885)
\curveto(640.94367221,157.37931646)(640.94867221,157.29931654)(640.9486792,157.20931885)
\lineto(640.9486792,157.05931885)
\curveto(640.92867223,156.98931685)(640.91867224,156.91931692)(640.9186792,156.84931885)
\curveto(640.91867224,156.77931706)(640.90867225,156.70931713)(640.8886792,156.63931885)
\curveto(640.8586723,156.52931731)(640.82367233,156.42431741)(640.7836792,156.32431885)
\curveto(640.74367241,156.22431761)(640.69867246,156.1343177)(640.6486792,156.05431885)
\curveto(640.48867267,155.79431804)(640.28367287,155.58431825)(640.0336792,155.42431885)
\curveto(639.78367337,155.27431856)(639.50367365,155.14431869)(639.1936792,155.03431885)
\curveto(639.10367405,155.00431883)(639.00867415,154.98431885)(638.9086792,154.97431885)
\curveto(638.81867434,154.95431888)(638.72867443,154.92931891)(638.6386792,154.89931885)
\curveto(638.53867462,154.87931896)(638.43867472,154.86931897)(638.3386792,154.86931885)
\curveto(638.23867492,154.86931897)(638.13867502,154.85931898)(638.0386792,154.83931885)
\lineto(637.8886792,154.83931885)
\curveto(637.83867532,154.82931901)(637.76867539,154.82431901)(637.6786792,154.82431885)
\curveto(637.58867557,154.82431901)(637.51867564,154.82931901)(637.4686792,154.83931885)
\lineto(637.3036792,154.83931885)
\curveto(637.24367591,154.85931898)(637.17867598,154.86931897)(637.1086792,154.86931885)
\curveto(637.03867612,154.85931898)(636.97867618,154.86431897)(636.9286792,154.88431885)
\curveto(636.87867628,154.89431894)(636.81367634,154.89931894)(636.7336792,154.89931885)
\lineto(636.4936792,154.95931885)
\curveto(636.42367673,154.96931887)(636.34867681,154.98931885)(636.2686792,155.01931885)
\curveto(635.9586772,155.11931872)(635.68867747,155.24431859)(635.4586792,155.39431885)
\curveto(635.22867793,155.54431829)(635.02867813,155.7393181)(634.8586792,155.97931885)
\curveto(634.76867839,156.10931773)(634.69367846,156.24431759)(634.6336792,156.38431885)
\curveto(634.57367858,156.52431731)(634.51867864,156.67931716)(634.4686792,156.84931885)
\curveto(634.44867871,156.90931693)(634.43867872,156.97931686)(634.4386792,157.05931885)
\curveto(634.44867871,157.14931669)(634.46367869,157.21931662)(634.4836792,157.26931885)
\curveto(634.51367864,157.30931653)(634.56367859,157.34931649)(634.6336792,157.38931885)
\curveto(634.68367847,157.40931643)(634.7536784,157.41931642)(634.8436792,157.41931885)
\curveto(634.93367822,157.42931641)(635.02367813,157.42931641)(635.1136792,157.41931885)
\curveto(635.20367795,157.40931643)(635.28867787,157.39431644)(635.3686792,157.37431885)
\curveto(635.4586777,157.36431647)(635.51867764,157.34931649)(635.5486792,157.32931885)
\curveto(635.61867754,157.27931656)(635.66367749,157.20431663)(635.6836792,157.10431885)
\curveto(635.71367744,157.01431682)(635.74867741,156.92931691)(635.7886792,156.84931885)
\curveto(635.88867727,156.62931721)(636.02367713,156.45931738)(636.1936792,156.33931885)
\curveto(636.31367684,156.24931759)(636.44867671,156.17931766)(636.5986792,156.12931885)
\curveto(636.74867641,156.07931776)(636.90867625,156.02931781)(637.0786792,155.97931885)
\lineto(637.3936792,155.93431885)
\lineto(637.4836792,155.93431885)
\curveto(637.5536756,155.91431792)(637.64367551,155.90431793)(637.7536792,155.90431885)
\curveto(637.87367528,155.90431793)(637.97367518,155.91431792)(638.0536792,155.93431885)
\curveto(638.12367503,155.9343179)(638.17867498,155.9393179)(638.2186792,155.94931885)
\curveto(638.27867488,155.95931788)(638.33867482,155.96431787)(638.3986792,155.96431885)
\curveto(638.4586747,155.97431786)(638.51367464,155.98431785)(638.5636792,155.99431885)
\curveto(638.8536743,156.07431776)(639.08367407,156.17931766)(639.2536792,156.30931885)
\curveto(639.42367373,156.4393174)(639.54367361,156.65931718)(639.6136792,156.96931885)
\curveto(639.63367352,157.01931682)(639.63867352,157.07431676)(639.6286792,157.13431885)
\curveto(639.61867354,157.19431664)(639.60867355,157.2393166)(639.5986792,157.26931885)
\curveto(639.54867361,157.45931638)(639.47867368,157.59931624)(639.3886792,157.68931885)
\curveto(639.29867386,157.78931605)(639.18367397,157.87931596)(639.0436792,157.95931885)
\curveto(638.9536742,158.01931582)(638.8536743,158.06931577)(638.7436792,158.10931885)
\lineto(638.4136792,158.22931885)
\curveto(638.38367477,158.2393156)(638.3536748,158.24431559)(638.3236792,158.24431885)
\curveto(638.30367485,158.24431559)(638.27867488,158.25431558)(638.2486792,158.27431885)
\curveto(637.90867525,158.38431545)(637.5536756,158.46431537)(637.1836792,158.51431885)
\curveto(636.82367633,158.57431526)(636.48367667,158.66931517)(636.1636792,158.79931885)
\curveto(636.06367709,158.839315)(635.96867719,158.87431496)(635.8786792,158.90431885)
\curveto(635.78867737,158.9343149)(635.70367745,158.97431486)(635.6236792,159.02431885)
\curveto(635.43367772,159.1343147)(635.2586779,159.25931458)(635.0986792,159.39931885)
\curveto(634.93867822,159.5393143)(634.81367834,159.71431412)(634.7236792,159.92431885)
\curveto(634.69367846,159.99431384)(634.66867849,160.06431377)(634.6486792,160.13431885)
\curveto(634.63867852,160.20431363)(634.62367853,160.27931356)(634.6036792,160.35931885)
\curveto(634.57367858,160.47931336)(634.56367859,160.61431322)(634.5736792,160.76431885)
\curveto(634.58367857,160.92431291)(634.59867856,161.05931278)(634.6186792,161.16931885)
\curveto(634.63867852,161.21931262)(634.64867851,161.25931258)(634.6486792,161.28931885)
\curveto(634.6586785,161.32931251)(634.67367848,161.36931247)(634.6936792,161.40931885)
\curveto(634.78367837,161.6393122)(634.90367825,161.839312)(635.0536792,162.00931885)
\curveto(635.21367794,162.17931166)(635.39367776,162.32931151)(635.5936792,162.45931885)
\curveto(635.74367741,162.54931129)(635.90867725,162.61931122)(636.0886792,162.66931885)
\curveto(636.26867689,162.72931111)(636.4586767,162.78431105)(636.6586792,162.83431885)
\curveto(636.72867643,162.84431099)(636.79367636,162.85431098)(636.8536792,162.86431885)
\curveto(636.92367623,162.87431096)(636.99867616,162.88431095)(637.0786792,162.89431885)
\curveto(637.10867605,162.90431093)(637.14867601,162.90431093)(637.1986792,162.89431885)
\curveto(637.24867591,162.88431095)(637.28367587,162.88931095)(637.3036792,162.90931885)
}
}
{
\newrgbcolor{curcolor}{0 0 0}
\pscustom[linestyle=none,fillstyle=solid,fillcolor=curcolor]
{
\newpath
\moveto(643.3186792,165.06931885)
\curveto(643.46867719,165.06930877)(643.61867704,165.06430877)(643.7686792,165.05431885)
\curveto(643.91867674,165.05430878)(644.02367663,165.01430882)(644.0836792,164.93431885)
\curveto(644.13367652,164.87430896)(644.1586765,164.78930905)(644.1586792,164.67931885)
\curveto(644.16867649,164.57930926)(644.17367648,164.47430936)(644.1736792,164.36431885)
\lineto(644.1736792,163.49431885)
\curveto(644.17367648,163.41431042)(644.16867649,163.32931051)(644.1586792,163.23931885)
\curveto(644.1586765,163.15931068)(644.16867649,163.08931075)(644.1886792,163.02931885)
\curveto(644.22867643,162.88931095)(644.31867634,162.79931104)(644.4586792,162.75931885)
\curveto(644.50867615,162.74931109)(644.5536761,162.74431109)(644.5936792,162.74431885)
\lineto(644.7436792,162.74431885)
\lineto(645.1486792,162.74431885)
\curveto(645.30867535,162.75431108)(645.42367523,162.74431109)(645.4936792,162.71431885)
\curveto(645.58367507,162.65431118)(645.64367501,162.59431124)(645.6736792,162.53431885)
\curveto(645.69367496,162.49431134)(645.70367495,162.44931139)(645.7036792,162.39931885)
\lineto(645.7036792,162.24931885)
\curveto(645.70367495,162.1393117)(645.69867496,162.0343118)(645.6886792,161.93431885)
\curveto(645.67867498,161.84431199)(645.64367501,161.77431206)(645.5836792,161.72431885)
\curveto(645.52367513,161.67431216)(645.43867522,161.64431219)(645.3286792,161.63431885)
\lineto(644.9986792,161.63431885)
\curveto(644.88867577,161.64431219)(644.77867588,161.64931219)(644.6686792,161.64931885)
\curveto(644.5586761,161.64931219)(644.46367619,161.6343122)(644.3836792,161.60431885)
\curveto(644.31367634,161.57431226)(644.26367639,161.52431231)(644.2336792,161.45431885)
\curveto(644.20367645,161.38431245)(644.18367647,161.29931254)(644.1736792,161.19931885)
\curveto(644.16367649,161.10931273)(644.1586765,161.00931283)(644.1586792,160.89931885)
\curveto(644.16867649,160.79931304)(644.17367648,160.69931314)(644.1736792,160.59931885)
\lineto(644.1736792,157.62931885)
\curveto(644.17367648,157.40931643)(644.16867649,157.17431666)(644.1586792,156.92431885)
\curveto(644.1586765,156.68431715)(644.20367645,156.49931734)(644.2936792,156.36931885)
\curveto(644.34367631,156.28931755)(644.40867625,156.2343176)(644.4886792,156.20431885)
\curveto(644.56867609,156.17431766)(644.66367599,156.14931769)(644.7736792,156.12931885)
\curveto(644.80367585,156.11931772)(644.83367582,156.11431772)(644.8636792,156.11431885)
\curveto(644.90367575,156.12431771)(644.93867572,156.12431771)(644.9686792,156.11431885)
\lineto(645.1636792,156.11431885)
\curveto(645.26367539,156.11431772)(645.3536753,156.10431773)(645.4336792,156.08431885)
\curveto(645.52367513,156.07431776)(645.58867507,156.0393178)(645.6286792,155.97931885)
\curveto(645.64867501,155.94931789)(645.66367499,155.89431794)(645.6736792,155.81431885)
\curveto(645.69367496,155.74431809)(645.70367495,155.66931817)(645.7036792,155.58931885)
\curveto(645.71367494,155.50931833)(645.71367494,155.42931841)(645.7036792,155.34931885)
\curveto(645.69367496,155.27931856)(645.67367498,155.22431861)(645.6436792,155.18431885)
\curveto(645.60367505,155.11431872)(645.52867513,155.06431877)(645.4186792,155.03431885)
\curveto(645.33867532,155.01431882)(645.24867541,155.00431883)(645.1486792,155.00431885)
\curveto(645.04867561,155.01431882)(644.9586757,155.01931882)(644.8786792,155.01931885)
\curveto(644.81867584,155.01931882)(644.7586759,155.01431882)(644.6986792,155.00431885)
\curveto(644.63867602,155.00431883)(644.58367607,155.00931883)(644.5336792,155.01931885)
\lineto(644.3536792,155.01931885)
\curveto(644.30367635,155.02931881)(644.2536764,155.0343188)(644.2036792,155.03431885)
\curveto(644.16367649,155.04431879)(644.11867654,155.04931879)(644.0686792,155.04931885)
\curveto(643.86867679,155.09931874)(643.69367696,155.15431868)(643.5436792,155.21431885)
\curveto(643.40367725,155.27431856)(643.28367737,155.37931846)(643.1836792,155.52931885)
\curveto(643.04367761,155.72931811)(642.96367769,155.97931786)(642.9436792,156.27931885)
\curveto(642.92367773,156.58931725)(642.91367774,156.91931692)(642.9136792,157.26931885)
\lineto(642.9136792,161.19931885)
\curveto(642.88367777,161.32931251)(642.8536778,161.42431241)(642.8236792,161.48431885)
\curveto(642.80367785,161.54431229)(642.73367792,161.59431224)(642.6136792,161.63431885)
\curveto(642.57367808,161.64431219)(642.53367812,161.64431219)(642.4936792,161.63431885)
\curveto(642.4536782,161.62431221)(642.41367824,161.62931221)(642.3736792,161.64931885)
\lineto(642.1336792,161.64931885)
\curveto(642.00367865,161.64931219)(641.89367876,161.65931218)(641.8036792,161.67931885)
\curveto(641.72367893,161.70931213)(641.66867899,161.76931207)(641.6386792,161.85931885)
\curveto(641.61867904,161.89931194)(641.60367905,161.94431189)(641.5936792,161.99431885)
\lineto(641.5936792,162.14431885)
\curveto(641.59367906,162.28431155)(641.60367905,162.39931144)(641.6236792,162.48931885)
\curveto(641.64367901,162.58931125)(641.70367895,162.66431117)(641.8036792,162.71431885)
\curveto(641.91367874,162.75431108)(642.0536786,162.76431107)(642.2236792,162.74431885)
\curveto(642.40367825,162.72431111)(642.5536781,162.7343111)(642.6736792,162.77431885)
\curveto(642.76367789,162.82431101)(642.83367782,162.89431094)(642.8836792,162.98431885)
\curveto(642.90367775,163.04431079)(642.91367774,163.11931072)(642.9136792,163.20931885)
\lineto(642.9136792,163.46431885)
\lineto(642.9136792,164.39431885)
\lineto(642.9136792,164.63431885)
\curveto(642.91367774,164.72430911)(642.92367773,164.79930904)(642.9436792,164.85931885)
\curveto(642.98367767,164.9393089)(643.0586776,165.00430883)(643.1686792,165.05431885)
\curveto(643.19867746,165.05430878)(643.22367743,165.05430878)(643.2436792,165.05431885)
\curveto(643.27367738,165.06430877)(643.29867736,165.06930877)(643.3186792,165.06931885)
}
}
{
\newrgbcolor{curcolor}{0 0 0}
\pscustom[linestyle=none,fillstyle=solid,fillcolor=curcolor]
{
\newpath
\moveto(647.55547607,162.72931885)
\lineto(647.99047607,162.72931885)
\curveto(648.14047411,162.72931111)(648.245474,162.68931115)(648.30547607,162.60931885)
\curveto(648.35547389,162.52931131)(648.38047387,162.42931141)(648.38047607,162.30931885)
\curveto(648.39047386,162.18931165)(648.39547385,162.06931177)(648.39547607,161.94931885)
\lineto(648.39547607,160.52431885)
\lineto(648.39547607,158.25931885)
\lineto(648.39547607,157.56931885)
\curveto(648.39547385,157.3393165)(648.42047383,157.1393167)(648.47047607,156.96931885)
\curveto(648.63047362,156.51931732)(648.93047332,156.20431763)(649.37047607,156.02431885)
\curveto(649.59047266,155.9343179)(649.85547239,155.89931794)(650.16547607,155.91931885)
\curveto(650.47547177,155.94931789)(650.72547152,156.00431783)(650.91547607,156.08431885)
\curveto(651.245471,156.22431761)(651.50547074,156.39931744)(651.69547607,156.60931885)
\curveto(651.89547035,156.82931701)(652.0504702,157.11431672)(652.16047607,157.46431885)
\curveto(652.19047006,157.54431629)(652.21047004,157.62431621)(652.22047607,157.70431885)
\curveto(652.23047002,157.78431605)(652.24547,157.86931597)(652.26547607,157.95931885)
\curveto(652.27546997,158.00931583)(652.27546997,158.05431578)(652.26547607,158.09431885)
\curveto(652.26546998,158.1343157)(652.27546997,158.17931566)(652.29547607,158.22931885)
\lineto(652.29547607,158.54431885)
\curveto(652.31546993,158.62431521)(652.32046993,158.71431512)(652.31047607,158.81431885)
\curveto(652.30046995,158.92431491)(652.29546995,159.02431481)(652.29547607,159.11431885)
\lineto(652.29547607,160.28431885)
\lineto(652.29547607,161.87431885)
\curveto(652.29546995,161.99431184)(652.29046996,162.11931172)(652.28047607,162.24931885)
\curveto(652.28046997,162.38931145)(652.30546994,162.49931134)(652.35547607,162.57931885)
\curveto(652.39546985,162.62931121)(652.44046981,162.65931118)(652.49047607,162.66931885)
\curveto(652.5504697,162.68931115)(652.62046963,162.70931113)(652.70047607,162.72931885)
\lineto(652.92547607,162.72931885)
\curveto(653.0454692,162.72931111)(653.1504691,162.72431111)(653.24047607,162.71431885)
\curveto(653.34046891,162.70431113)(653.41546883,162.65931118)(653.46547607,162.57931885)
\curveto(653.51546873,162.52931131)(653.54046871,162.45431138)(653.54047607,162.35431885)
\lineto(653.54047607,162.06931885)
\lineto(653.54047607,161.04931885)
\lineto(653.54047607,157.01431885)
\lineto(653.54047607,155.66431885)
\curveto(653.54046871,155.54431829)(653.53546871,155.42931841)(653.52547607,155.31931885)
\curveto(653.52546872,155.21931862)(653.49046876,155.14431869)(653.42047607,155.09431885)
\curveto(653.38046887,155.06431877)(653.32046893,155.0393188)(653.24047607,155.01931885)
\curveto(653.16046909,155.00931883)(653.07046918,154.99931884)(652.97047607,154.98931885)
\curveto(652.88046937,154.98931885)(652.79046946,154.99431884)(652.70047607,155.00431885)
\curveto(652.62046963,155.01431882)(652.56046969,155.0343188)(652.52047607,155.06431885)
\curveto(652.47046978,155.10431873)(652.42546982,155.16931867)(652.38547607,155.25931885)
\curveto(652.37546987,155.29931854)(652.36546988,155.35431848)(652.35547607,155.42431885)
\curveto(652.35546989,155.49431834)(652.3504699,155.55931828)(652.34047607,155.61931885)
\curveto(652.33046992,155.68931815)(652.31046994,155.74431809)(652.28047607,155.78431885)
\curveto(652.25047,155.82431801)(652.20547004,155.839318)(652.14547607,155.82931885)
\curveto(652.06547018,155.80931803)(651.98547026,155.74931809)(651.90547607,155.64931885)
\curveto(651.82547042,155.55931828)(651.7504705,155.48931835)(651.68047607,155.43931885)
\curveto(651.46047079,155.27931856)(651.21047104,155.1393187)(650.93047607,155.01931885)
\curveto(650.82047143,154.96931887)(650.70547154,154.9393189)(650.58547607,154.92931885)
\curveto(650.47547177,154.90931893)(650.36047189,154.88431895)(650.24047607,154.85431885)
\curveto(650.19047206,154.84431899)(650.13547211,154.84431899)(650.07547607,154.85431885)
\curveto(650.02547222,154.86431897)(649.97547227,154.85931898)(649.92547607,154.83931885)
\curveto(649.82547242,154.81931902)(649.73547251,154.81931902)(649.65547607,154.83931885)
\lineto(649.50547607,154.83931885)
\curveto(649.45547279,154.85931898)(649.39547285,154.86931897)(649.32547607,154.86931885)
\curveto(649.26547298,154.86931897)(649.21047304,154.87431896)(649.16047607,154.88431885)
\curveto(649.12047313,154.90431893)(649.08047317,154.91431892)(649.04047607,154.91431885)
\curveto(649.01047324,154.90431893)(648.97047328,154.90931893)(648.92047607,154.92931885)
\lineto(648.68047607,154.98931885)
\curveto(648.61047364,155.00931883)(648.53547371,155.0393188)(648.45547607,155.07931885)
\curveto(648.19547405,155.18931865)(647.97547427,155.3343185)(647.79547607,155.51431885)
\curveto(647.62547462,155.70431813)(647.48547476,155.92931791)(647.37547607,156.18931885)
\curveto(647.33547491,156.27931756)(647.30547494,156.36931747)(647.28547607,156.45931885)
\lineto(647.22547607,156.75931885)
\curveto(647.20547504,156.81931702)(647.19547505,156.87431696)(647.19547607,156.92431885)
\curveto(647.20547504,156.98431685)(647.20047505,157.04931679)(647.18047607,157.11931885)
\curveto(647.17047508,157.1393167)(647.16547508,157.16431667)(647.16547607,157.19431885)
\curveto(647.16547508,157.2343166)(647.16047509,157.26931657)(647.15047607,157.29931885)
\lineto(647.15047607,157.44931885)
\curveto(647.14047511,157.48931635)(647.13547511,157.5343163)(647.13547607,157.58431885)
\curveto(647.1454751,157.64431619)(647.1504751,157.69931614)(647.15047607,157.74931885)
\lineto(647.15047607,158.34931885)
\lineto(647.15047607,161.10931885)
\lineto(647.15047607,162.06931885)
\lineto(647.15047607,162.33931885)
\curveto(647.1504751,162.42931141)(647.17047508,162.50431133)(647.21047607,162.56431885)
\curveto(647.250475,162.6343112)(647.32547492,162.68431115)(647.43547607,162.71431885)
\curveto(647.45547479,162.72431111)(647.47547477,162.72431111)(647.49547607,162.71431885)
\curveto(647.51547473,162.71431112)(647.53547471,162.71931112)(647.55547607,162.72931885)
}
}
{
\newrgbcolor{curcolor}{0 0 0}
\pscustom[linestyle=none,fillstyle=solid,fillcolor=curcolor]
{
\newpath
\moveto(662.40008545,155.81431885)
\lineto(662.40008545,155.42431885)
\curveto(662.40007757,155.30431853)(662.3750776,155.20431863)(662.32508545,155.12431885)
\curveto(662.2750777,155.05431878)(662.19007778,155.01431882)(662.07008545,155.00431885)
\lineto(661.72508545,155.00431885)
\curveto(661.66507831,155.00431883)(661.60507837,154.99931884)(661.54508545,154.98931885)
\curveto(661.49507848,154.98931885)(661.45007852,154.99931884)(661.41008545,155.01931885)
\curveto(661.32007865,155.0393188)(661.26007871,155.07931876)(661.23008545,155.13931885)
\curveto(661.19007878,155.18931865)(661.16507881,155.24931859)(661.15508545,155.31931885)
\curveto(661.15507882,155.38931845)(661.14007883,155.45931838)(661.11008545,155.52931885)
\curveto(661.10007887,155.54931829)(661.08507889,155.56431827)(661.06508545,155.57431885)
\curveto(661.05507892,155.59431824)(661.04007893,155.61431822)(661.02008545,155.63431885)
\curveto(660.92007905,155.64431819)(660.84007913,155.62431821)(660.78008545,155.57431885)
\curveto(660.73007924,155.52431831)(660.6750793,155.47431836)(660.61508545,155.42431885)
\curveto(660.41507956,155.27431856)(660.21507976,155.15931868)(660.01508545,155.07931885)
\curveto(659.83508014,154.99931884)(659.62508035,154.9393189)(659.38508545,154.89931885)
\curveto(659.15508082,154.85931898)(658.91508106,154.839319)(658.66508545,154.83931885)
\curveto(658.42508155,154.82931901)(658.18508179,154.84431899)(657.94508545,154.88431885)
\curveto(657.70508227,154.91431892)(657.49508248,154.96931887)(657.31508545,155.04931885)
\curveto(656.79508318,155.26931857)(656.3750836,155.56431827)(656.05508545,155.93431885)
\curveto(655.73508424,156.31431752)(655.48508449,156.78431705)(655.30508545,157.34431885)
\curveto(655.26508471,157.4343164)(655.23508474,157.52431631)(655.21508545,157.61431885)
\curveto(655.20508477,157.71431612)(655.18508479,157.81431602)(655.15508545,157.91431885)
\curveto(655.14508483,157.96431587)(655.14008483,158.01431582)(655.14008545,158.06431885)
\curveto(655.14008483,158.11431572)(655.13508484,158.16431567)(655.12508545,158.21431885)
\curveto(655.10508487,158.26431557)(655.09508488,158.31431552)(655.09508545,158.36431885)
\curveto(655.10508487,158.42431541)(655.10508487,158.47931536)(655.09508545,158.52931885)
\lineto(655.09508545,158.67931885)
\curveto(655.0750849,158.72931511)(655.06508491,158.79431504)(655.06508545,158.87431885)
\curveto(655.06508491,158.95431488)(655.0750849,159.01931482)(655.09508545,159.06931885)
\lineto(655.09508545,159.23431885)
\curveto(655.11508486,159.30431453)(655.12008485,159.37431446)(655.11008545,159.44431885)
\curveto(655.11008486,159.52431431)(655.12008485,159.59931424)(655.14008545,159.66931885)
\curveto(655.15008482,159.71931412)(655.15508482,159.76431407)(655.15508545,159.80431885)
\curveto(655.15508482,159.84431399)(655.16008481,159.88931395)(655.17008545,159.93931885)
\curveto(655.20008477,160.0393138)(655.22508475,160.1343137)(655.24508545,160.22431885)
\curveto(655.26508471,160.32431351)(655.29008468,160.41931342)(655.32008545,160.50931885)
\curveto(655.45008452,160.88931295)(655.61508436,161.22931261)(655.81508545,161.52931885)
\curveto(656.02508395,161.839312)(656.2750837,162.09431174)(656.56508545,162.29431885)
\curveto(656.73508324,162.41431142)(656.91008306,162.51431132)(657.09008545,162.59431885)
\curveto(657.28008269,162.67431116)(657.48508249,162.74431109)(657.70508545,162.80431885)
\curveto(657.7750822,162.81431102)(657.84008213,162.82431101)(657.90008545,162.83431885)
\curveto(657.970082,162.84431099)(658.04008193,162.85931098)(658.11008545,162.87931885)
\lineto(658.26008545,162.87931885)
\curveto(658.34008163,162.89931094)(658.45508152,162.90931093)(658.60508545,162.90931885)
\curveto(658.76508121,162.90931093)(658.88508109,162.89931094)(658.96508545,162.87931885)
\curveto(659.00508097,162.86931097)(659.06008091,162.86431097)(659.13008545,162.86431885)
\curveto(659.24008073,162.834311)(659.35008062,162.80931103)(659.46008545,162.78931885)
\curveto(659.5700804,162.77931106)(659.6750803,162.74931109)(659.77508545,162.69931885)
\curveto(659.92508005,162.6393112)(660.06507991,162.57431126)(660.19508545,162.50431885)
\curveto(660.33507964,162.4343114)(660.46507951,162.35431148)(660.58508545,162.26431885)
\curveto(660.64507933,162.21431162)(660.70507927,162.15931168)(660.76508545,162.09931885)
\curveto(660.83507914,162.04931179)(660.92507905,162.0343118)(661.03508545,162.05431885)
\curveto(661.05507892,162.08431175)(661.0700789,162.10931173)(661.08008545,162.12931885)
\curveto(661.10007887,162.14931169)(661.11507886,162.17931166)(661.12508545,162.21931885)
\curveto(661.15507882,162.30931153)(661.16507881,162.42431141)(661.15508545,162.56431885)
\lineto(661.15508545,162.93931885)
\lineto(661.15508545,164.66431885)
\lineto(661.15508545,165.12931885)
\curveto(661.15507882,165.30930853)(661.18007879,165.4393084)(661.23008545,165.51931885)
\curveto(661.2700787,165.58930825)(661.33007864,165.6343082)(661.41008545,165.65431885)
\curveto(661.43007854,165.65430818)(661.45507852,165.65430818)(661.48508545,165.65431885)
\curveto(661.51507846,165.66430817)(661.54007843,165.66930817)(661.56008545,165.66931885)
\curveto(661.70007827,165.67930816)(661.84507813,165.67930816)(661.99508545,165.66931885)
\curveto(662.15507782,165.66930817)(662.26507771,165.62930821)(662.32508545,165.54931885)
\curveto(662.3750776,165.46930837)(662.40007757,165.36930847)(662.40008545,165.24931885)
\lineto(662.40008545,164.87431885)
\lineto(662.40008545,155.81431885)
\moveto(661.18508545,158.64931885)
\curveto(661.20507877,158.69931514)(661.21507876,158.76431507)(661.21508545,158.84431885)
\curveto(661.21507876,158.9343149)(661.20507877,159.00431483)(661.18508545,159.05431885)
\lineto(661.18508545,159.27931885)
\curveto(661.16507881,159.36931447)(661.15007882,159.45931438)(661.14008545,159.54931885)
\curveto(661.13007884,159.64931419)(661.11007886,159.7393141)(661.08008545,159.81931885)
\curveto(661.06007891,159.89931394)(661.04007893,159.97431386)(661.02008545,160.04431885)
\curveto(661.01007896,160.11431372)(660.99007898,160.18431365)(660.96008545,160.25431885)
\curveto(660.84007913,160.55431328)(660.68507929,160.81931302)(660.49508545,161.04931885)
\curveto(660.30507967,161.27931256)(660.06507991,161.45931238)(659.77508545,161.58931885)
\curveto(659.6750803,161.6393122)(659.5700804,161.67431216)(659.46008545,161.69431885)
\curveto(659.36008061,161.72431211)(659.25008072,161.74931209)(659.13008545,161.76931885)
\curveto(659.05008092,161.78931205)(658.96008101,161.79931204)(658.86008545,161.79931885)
\lineto(658.59008545,161.79931885)
\curveto(658.54008143,161.78931205)(658.49508148,161.77931206)(658.45508545,161.76931885)
\lineto(658.32008545,161.76931885)
\curveto(658.24008173,161.74931209)(658.15508182,161.72931211)(658.06508545,161.70931885)
\curveto(657.98508199,161.68931215)(657.90508207,161.66431217)(657.82508545,161.63431885)
\curveto(657.50508247,161.49431234)(657.24508273,161.28931255)(657.04508545,161.01931885)
\curveto(656.85508312,160.75931308)(656.70008327,160.45431338)(656.58008545,160.10431885)
\curveto(656.54008343,159.99431384)(656.51008346,159.87931396)(656.49008545,159.75931885)
\curveto(656.48008349,159.64931419)(656.46508351,159.5393143)(656.44508545,159.42931885)
\curveto(656.44508353,159.38931445)(656.44008353,159.34931449)(656.43008545,159.30931885)
\lineto(656.43008545,159.20431885)
\curveto(656.41008356,159.15431468)(656.40008357,159.09931474)(656.40008545,159.03931885)
\curveto(656.41008356,158.97931486)(656.41508356,158.92431491)(656.41508545,158.87431885)
\lineto(656.41508545,158.54431885)
\curveto(656.41508356,158.44431539)(656.42508355,158.34931549)(656.44508545,158.25931885)
\curveto(656.45508352,158.22931561)(656.46008351,158.17931566)(656.46008545,158.10931885)
\curveto(656.48008349,158.0393158)(656.49508348,157.96931587)(656.50508545,157.89931885)
\lineto(656.56508545,157.68931885)
\curveto(656.6750833,157.3393165)(656.82508315,157.0393168)(657.01508545,156.78931885)
\curveto(657.20508277,156.5393173)(657.44508253,156.3343175)(657.73508545,156.17431885)
\curveto(657.82508215,156.12431771)(657.91508206,156.08431775)(658.00508545,156.05431885)
\curveto(658.09508188,156.02431781)(658.19508178,155.99431784)(658.30508545,155.96431885)
\curveto(658.35508162,155.94431789)(658.40508157,155.9393179)(658.45508545,155.94931885)
\curveto(658.51508146,155.95931788)(658.5700814,155.95431788)(658.62008545,155.93431885)
\curveto(658.66008131,155.92431791)(658.70008127,155.91931792)(658.74008545,155.91931885)
\lineto(658.87508545,155.91931885)
\lineto(659.01008545,155.91931885)
\curveto(659.04008093,155.92931791)(659.09008088,155.9343179)(659.16008545,155.93431885)
\curveto(659.24008073,155.95431788)(659.32008065,155.96931787)(659.40008545,155.97931885)
\curveto(659.48008049,155.99931784)(659.55508042,156.02431781)(659.62508545,156.05431885)
\curveto(659.95508002,156.19431764)(660.22007975,156.36931747)(660.42008545,156.57931885)
\curveto(660.63007934,156.79931704)(660.80507917,157.07431676)(660.94508545,157.40431885)
\curveto(660.99507898,157.51431632)(661.03007894,157.62431621)(661.05008545,157.73431885)
\curveto(661.0700789,157.84431599)(661.09507888,157.95431588)(661.12508545,158.06431885)
\curveto(661.14507883,158.10431573)(661.15507882,158.1393157)(661.15508545,158.16931885)
\curveto(661.15507882,158.20931563)(661.16007881,158.24931559)(661.17008545,158.28931885)
\curveto(661.18007879,158.34931549)(661.18007879,158.40931543)(661.17008545,158.46931885)
\curveto(661.1700788,158.52931531)(661.1750788,158.58931525)(661.18508545,158.64931885)
}
}
{
\newrgbcolor{curcolor}{0 0 0}
\pscustom[linestyle=none,fillstyle=solid,fillcolor=curcolor]
{
\newpath
\moveto(664.63133545,164.22931885)
\curveto(664.55133433,164.28930955)(664.50633437,164.39430944)(664.49633545,164.54431885)
\lineto(664.49633545,165.00931885)
\lineto(664.49633545,165.26431885)
\curveto(664.49633438,165.35430848)(664.51133437,165.42930841)(664.54133545,165.48931885)
\curveto(664.5813343,165.56930827)(664.66133422,165.62930821)(664.78133545,165.66931885)
\curveto(664.80133408,165.67930816)(664.82133406,165.67930816)(664.84133545,165.66931885)
\curveto(664.87133401,165.66930817)(664.89633398,165.67430816)(664.91633545,165.68431885)
\curveto(665.08633379,165.68430815)(665.24633363,165.67930816)(665.39633545,165.66931885)
\curveto(665.54633333,165.65930818)(665.64633323,165.59930824)(665.69633545,165.48931885)
\curveto(665.72633315,165.42930841)(665.74133314,165.35430848)(665.74133545,165.26431885)
\lineto(665.74133545,165.00931885)
\curveto(665.74133314,164.82930901)(665.73633314,164.65930918)(665.72633545,164.49931885)
\curveto(665.72633315,164.3393095)(665.66133322,164.2343096)(665.53133545,164.18431885)
\curveto(665.4813334,164.16430967)(665.42633345,164.15430968)(665.36633545,164.15431885)
\lineto(665.20133545,164.15431885)
\lineto(664.88633545,164.15431885)
\curveto(664.78633409,164.15430968)(664.70133418,164.17930966)(664.63133545,164.22931885)
\moveto(665.74133545,155.72431885)
\lineto(665.74133545,155.40931885)
\curveto(665.75133313,155.30931853)(665.73133315,155.22931861)(665.68133545,155.16931885)
\curveto(665.65133323,155.10931873)(665.60633327,155.06931877)(665.54633545,155.04931885)
\curveto(665.48633339,155.0393188)(665.41633346,155.02431881)(665.33633545,155.00431885)
\lineto(665.11133545,155.00431885)
\curveto(664.9813339,155.00431883)(664.86633401,155.00931883)(664.76633545,155.01931885)
\curveto(664.6763342,155.0393188)(664.60633427,155.08931875)(664.55633545,155.16931885)
\curveto(664.51633436,155.22931861)(664.49633438,155.30431853)(664.49633545,155.39431885)
\lineto(664.49633545,155.67931885)
\lineto(664.49633545,162.02431885)
\lineto(664.49633545,162.33931885)
\curveto(664.49633438,162.44931139)(664.52133436,162.5343113)(664.57133545,162.59431885)
\curveto(664.60133428,162.64431119)(664.64133424,162.67431116)(664.69133545,162.68431885)
\curveto(664.74133414,162.69431114)(664.79633408,162.70931113)(664.85633545,162.72931885)
\curveto(664.876334,162.72931111)(664.89633398,162.72431111)(664.91633545,162.71431885)
\curveto(664.94633393,162.71431112)(664.97133391,162.71931112)(664.99133545,162.72931885)
\curveto(665.12133376,162.72931111)(665.25133363,162.72431111)(665.38133545,162.71431885)
\curveto(665.52133336,162.71431112)(665.61633326,162.67431116)(665.66633545,162.59431885)
\curveto(665.71633316,162.5343113)(665.74133314,162.45431138)(665.74133545,162.35431885)
\lineto(665.74133545,162.06931885)
\lineto(665.74133545,155.72431885)
}
}
{
\newrgbcolor{curcolor}{0 0 0}
\pscustom[linestyle=none,fillstyle=solid,fillcolor=curcolor]
{
\newpath
\moveto(674.5711792,155.55931885)
\curveto(674.60117137,155.39931844)(674.58617138,155.26431857)(674.5261792,155.15431885)
\curveto(674.4661715,155.05431878)(674.38617158,154.97931886)(674.2861792,154.92931885)
\curveto(674.23617173,154.90931893)(674.18117179,154.89931894)(674.1211792,154.89931885)
\curveto(674.0711719,154.89931894)(674.01617195,154.88931895)(673.9561792,154.86931885)
\curveto(673.73617223,154.81931902)(673.51617245,154.834319)(673.2961792,154.91431885)
\curveto(673.08617288,154.98431885)(672.94117303,155.07431876)(672.8611792,155.18431885)
\curveto(672.81117316,155.25431858)(672.7661732,155.3343185)(672.7261792,155.42431885)
\curveto(672.68617328,155.52431831)(672.63617333,155.60431823)(672.5761792,155.66431885)
\curveto(672.55617341,155.68431815)(672.53117344,155.70431813)(672.5011792,155.72431885)
\curveto(672.48117349,155.74431809)(672.45117352,155.74931809)(672.4111792,155.73931885)
\curveto(672.30117367,155.70931813)(672.19617377,155.65431818)(672.0961792,155.57431885)
\curveto(672.00617396,155.49431834)(671.91617405,155.42431841)(671.8261792,155.36431885)
\curveto(671.69617427,155.28431855)(671.55617441,155.20931863)(671.4061792,155.13931885)
\curveto(671.25617471,155.07931876)(671.09617487,155.02431881)(670.9261792,154.97431885)
\curveto(670.82617514,154.94431889)(670.71617525,154.92431891)(670.5961792,154.91431885)
\curveto(670.48617548,154.90431893)(670.37617559,154.88931895)(670.2661792,154.86931885)
\curveto(670.21617575,154.85931898)(670.1711758,154.85431898)(670.1311792,154.85431885)
\lineto(670.0261792,154.85431885)
\curveto(669.91617605,154.834319)(669.81117616,154.834319)(669.7111792,154.85431885)
\lineto(669.5761792,154.85431885)
\curveto(669.52617644,154.86431897)(669.47617649,154.86931897)(669.4261792,154.86931885)
\curveto(669.37617659,154.86931897)(669.33117664,154.87931896)(669.2911792,154.89931885)
\curveto(669.25117672,154.90931893)(669.21617675,154.91431892)(669.1861792,154.91431885)
\curveto(669.1661768,154.90431893)(669.14117683,154.90431893)(669.1111792,154.91431885)
\lineto(668.8711792,154.97431885)
\curveto(668.79117718,154.98431885)(668.71617725,155.00431883)(668.6461792,155.03431885)
\curveto(668.34617762,155.16431867)(668.10117787,155.30931853)(667.9111792,155.46931885)
\curveto(667.73117824,155.6393182)(667.58117839,155.87431796)(667.4611792,156.17431885)
\curveto(667.3711786,156.39431744)(667.32617864,156.65931718)(667.3261792,156.96931885)
\lineto(667.3261792,157.28431885)
\curveto(667.33617863,157.3343165)(667.34117863,157.38431645)(667.3411792,157.43431885)
\lineto(667.3711792,157.61431885)
\lineto(667.4911792,157.94431885)
\curveto(667.53117844,158.05431578)(667.58117839,158.15431568)(667.6411792,158.24431885)
\curveto(667.82117815,158.5343153)(668.0661779,158.74931509)(668.3761792,158.88931885)
\curveto(668.68617728,159.02931481)(669.02617694,159.15431468)(669.3961792,159.26431885)
\curveto(669.53617643,159.30431453)(669.68117629,159.3343145)(669.8311792,159.35431885)
\curveto(669.98117599,159.37431446)(670.13117584,159.39931444)(670.2811792,159.42931885)
\curveto(670.35117562,159.44931439)(670.41617555,159.45931438)(670.4761792,159.45931885)
\curveto(670.54617542,159.45931438)(670.62117535,159.46931437)(670.7011792,159.48931885)
\curveto(670.7711752,159.50931433)(670.84117513,159.51931432)(670.9111792,159.51931885)
\curveto(670.98117499,159.52931431)(671.05617491,159.54431429)(671.1361792,159.56431885)
\curveto(671.38617458,159.62431421)(671.62117435,159.67431416)(671.8411792,159.71431885)
\curveto(672.06117391,159.76431407)(672.23617373,159.87931396)(672.3661792,160.05931885)
\curveto(672.42617354,160.1393137)(672.47617349,160.2393136)(672.5161792,160.35931885)
\curveto(672.55617341,160.48931335)(672.55617341,160.62931321)(672.5161792,160.77931885)
\curveto(672.45617351,161.01931282)(672.3661736,161.20931263)(672.2461792,161.34931885)
\curveto(672.13617383,161.48931235)(671.97617399,161.59931224)(671.7661792,161.67931885)
\curveto(671.64617432,161.72931211)(671.50117447,161.76431207)(671.3311792,161.78431885)
\curveto(671.1711748,161.80431203)(671.00117497,161.81431202)(670.8211792,161.81431885)
\curveto(670.64117533,161.81431202)(670.4661755,161.80431203)(670.2961792,161.78431885)
\curveto(670.12617584,161.76431207)(669.98117599,161.7343121)(669.8611792,161.69431885)
\curveto(669.69117628,161.6343122)(669.52617644,161.54931229)(669.3661792,161.43931885)
\curveto(669.28617668,161.37931246)(669.21117676,161.29931254)(669.1411792,161.19931885)
\curveto(669.08117689,161.10931273)(669.02617694,161.00931283)(668.9761792,160.89931885)
\curveto(668.94617702,160.81931302)(668.91617705,160.7343131)(668.8861792,160.64431885)
\curveto(668.8661771,160.55431328)(668.82117715,160.48431335)(668.7511792,160.43431885)
\curveto(668.71117726,160.40431343)(668.64117733,160.37931346)(668.5411792,160.35931885)
\curveto(668.45117752,160.34931349)(668.35617761,160.34431349)(668.2561792,160.34431885)
\curveto(668.15617781,160.34431349)(668.05617791,160.34931349)(667.9561792,160.35931885)
\curveto(667.8661781,160.37931346)(667.80117817,160.40431343)(667.7611792,160.43431885)
\curveto(667.72117825,160.46431337)(667.69117828,160.51431332)(667.6711792,160.58431885)
\curveto(667.65117832,160.65431318)(667.65117832,160.72931311)(667.6711792,160.80931885)
\curveto(667.70117827,160.9393129)(667.73117824,161.05931278)(667.7611792,161.16931885)
\curveto(667.80117817,161.28931255)(667.84617812,161.40431243)(667.8961792,161.51431885)
\curveto(668.08617788,161.86431197)(668.32617764,162.1343117)(668.6161792,162.32431885)
\curveto(668.90617706,162.52431131)(669.2661767,162.68431115)(669.6961792,162.80431885)
\curveto(669.79617617,162.82431101)(669.89617607,162.839311)(669.9961792,162.84931885)
\curveto(670.10617586,162.85931098)(670.21617575,162.87431096)(670.3261792,162.89431885)
\curveto(670.3661756,162.90431093)(670.43117554,162.90431093)(670.5211792,162.89431885)
\curveto(670.61117536,162.89431094)(670.6661753,162.90431093)(670.6861792,162.92431885)
\curveto(671.38617458,162.9343109)(671.99617397,162.85431098)(672.5161792,162.68431885)
\curveto(673.03617293,162.51431132)(673.40117257,162.18931165)(673.6111792,161.70931885)
\curveto(673.70117227,161.50931233)(673.75117222,161.27431256)(673.7611792,161.00431885)
\curveto(673.78117219,160.74431309)(673.79117218,160.46931337)(673.7911792,160.17931885)
\lineto(673.7911792,156.86431885)
\curveto(673.79117218,156.72431711)(673.79617217,156.58931725)(673.8061792,156.45931885)
\curveto(673.81617215,156.32931751)(673.84617212,156.22431761)(673.8961792,156.14431885)
\curveto(673.94617202,156.07431776)(674.01117196,156.02431781)(674.0911792,155.99431885)
\curveto(674.18117179,155.95431788)(674.2661717,155.92431791)(674.3461792,155.90431885)
\curveto(674.42617154,155.89431794)(674.48617148,155.84931799)(674.5261792,155.76931885)
\curveto(674.54617142,155.7393181)(674.55617141,155.70931813)(674.5561792,155.67931885)
\curveto(674.55617141,155.64931819)(674.56117141,155.60931823)(674.5711792,155.55931885)
\moveto(672.4261792,157.22431885)
\curveto(672.48617348,157.36431647)(672.51617345,157.52431631)(672.5161792,157.70431885)
\curveto(672.52617344,157.89431594)(672.53117344,158.08931575)(672.5311792,158.28931885)
\curveto(672.53117344,158.39931544)(672.52617344,158.49931534)(672.5161792,158.58931885)
\curveto(672.50617346,158.67931516)(672.4661735,158.74931509)(672.3961792,158.79931885)
\curveto(672.3661736,158.81931502)(672.29617367,158.82931501)(672.1861792,158.82931885)
\curveto(672.1661738,158.80931503)(672.13117384,158.79931504)(672.0811792,158.79931885)
\curveto(672.03117394,158.79931504)(671.98617398,158.78931505)(671.9461792,158.76931885)
\curveto(671.8661741,158.74931509)(671.77617419,158.72931511)(671.6761792,158.70931885)
\lineto(671.3761792,158.64931885)
\curveto(671.34617462,158.64931519)(671.31117466,158.64431519)(671.2711792,158.63431885)
\lineto(671.1661792,158.63431885)
\curveto(671.01617495,158.59431524)(670.85117512,158.56931527)(670.6711792,158.55931885)
\curveto(670.50117547,158.55931528)(670.34117563,158.5393153)(670.1911792,158.49931885)
\curveto(670.11117586,158.47931536)(670.03617593,158.45931538)(669.9661792,158.43931885)
\curveto(669.90617606,158.42931541)(669.83617613,158.41431542)(669.7561792,158.39431885)
\curveto(669.59617637,158.34431549)(669.44617652,158.27931556)(669.3061792,158.19931885)
\curveto(669.1661768,158.12931571)(669.04617692,158.0393158)(668.9461792,157.92931885)
\curveto(668.84617712,157.81931602)(668.7711772,157.68431615)(668.7211792,157.52431885)
\curveto(668.6711773,157.37431646)(668.65117732,157.18931665)(668.6611792,156.96931885)
\curveto(668.66117731,156.86931697)(668.67617729,156.77431706)(668.7061792,156.68431885)
\curveto(668.74617722,156.60431723)(668.79117718,156.52931731)(668.8411792,156.45931885)
\curveto(668.92117705,156.34931749)(669.02617694,156.25431758)(669.1561792,156.17431885)
\curveto(669.28617668,156.10431773)(669.42617654,156.04431779)(669.5761792,155.99431885)
\curveto(669.62617634,155.98431785)(669.67617629,155.97931786)(669.7261792,155.97931885)
\curveto(669.77617619,155.97931786)(669.82617614,155.97431786)(669.8761792,155.96431885)
\curveto(669.94617602,155.94431789)(670.03117594,155.92931791)(670.1311792,155.91931885)
\curveto(670.24117573,155.91931792)(670.33117564,155.92931791)(670.4011792,155.94931885)
\curveto(670.46117551,155.96931787)(670.52117545,155.97431786)(670.5811792,155.96431885)
\curveto(670.64117533,155.96431787)(670.70117527,155.97431786)(670.7611792,155.99431885)
\curveto(670.84117513,156.01431782)(670.91617505,156.02931781)(670.9861792,156.03931885)
\curveto(671.0661749,156.04931779)(671.14117483,156.06931777)(671.2111792,156.09931885)
\curveto(671.50117447,156.21931762)(671.74617422,156.36431747)(671.9461792,156.53431885)
\curveto(672.15617381,156.70431713)(672.31617365,156.9343169)(672.4261792,157.22431885)
}
}
{
\newrgbcolor{curcolor}{0 0 0}
\pscustom[linestyle=none,fillstyle=solid,fillcolor=curcolor]
{
\newpath
\moveto(679.43281982,162.87931885)
\curveto(680.06281459,162.89931094)(680.56781408,162.81431102)(680.94781982,162.62431885)
\curveto(681.32781332,162.4343114)(681.63281302,162.14931169)(681.86281982,161.76931885)
\curveto(681.92281273,161.66931217)(681.96781268,161.55931228)(681.99781982,161.43931885)
\curveto(682.03781261,161.32931251)(682.07281258,161.21431262)(682.10281982,161.09431885)
\curveto(682.1528125,160.90431293)(682.18281247,160.69931314)(682.19281982,160.47931885)
\curveto(682.20281245,160.25931358)(682.20781244,160.0343138)(682.20781982,159.80431885)
\lineto(682.20781982,158.19931885)
\lineto(682.20781982,155.85931885)
\curveto(682.20781244,155.68931815)(682.20281245,155.51931832)(682.19281982,155.34931885)
\curveto(682.19281246,155.17931866)(682.12781252,155.06931877)(681.99781982,155.01931885)
\curveto(681.9478127,154.99931884)(681.89281276,154.98931885)(681.83281982,154.98931885)
\curveto(681.78281287,154.97931886)(681.72781292,154.97431886)(681.66781982,154.97431885)
\curveto(681.53781311,154.97431886)(681.41281324,154.97931886)(681.29281982,154.98931885)
\curveto(681.17281348,154.98931885)(681.08781356,155.02931881)(681.03781982,155.10931885)
\curveto(680.98781366,155.17931866)(680.96281369,155.26931857)(680.96281982,155.37931885)
\lineto(680.96281982,155.70931885)
\lineto(680.96281982,156.99931885)
\lineto(680.96281982,159.44431885)
\curveto(680.96281369,159.71431412)(680.95781369,159.97931386)(680.94781982,160.23931885)
\curveto(680.93781371,160.50931333)(680.89281376,160.7393131)(680.81281982,160.92931885)
\curveto(680.73281392,161.12931271)(680.61281404,161.28931255)(680.45281982,161.40931885)
\curveto(680.29281436,161.5393123)(680.10781454,161.6393122)(679.89781982,161.70931885)
\curveto(679.83781481,161.72931211)(679.77281488,161.7393121)(679.70281982,161.73931885)
\curveto(679.64281501,161.74931209)(679.58281507,161.76431207)(679.52281982,161.78431885)
\curveto(679.47281518,161.79431204)(679.39281526,161.79431204)(679.28281982,161.78431885)
\curveto(679.18281547,161.78431205)(679.11281554,161.77931206)(679.07281982,161.76931885)
\curveto(679.03281562,161.74931209)(678.99781565,161.7393121)(678.96781982,161.73931885)
\curveto(678.93781571,161.74931209)(678.90281575,161.74931209)(678.86281982,161.73931885)
\curveto(678.73281592,161.70931213)(678.60781604,161.67431216)(678.48781982,161.63431885)
\curveto(678.37781627,161.60431223)(678.27281638,161.55931228)(678.17281982,161.49931885)
\curveto(678.13281652,161.47931236)(678.09781655,161.45931238)(678.06781982,161.43931885)
\curveto(678.03781661,161.41931242)(678.00281665,161.39931244)(677.96281982,161.37931885)
\curveto(677.61281704,161.12931271)(677.35781729,160.75431308)(677.19781982,160.25431885)
\curveto(677.16781748,160.17431366)(677.1478175,160.08931375)(677.13781982,159.99931885)
\curveto(677.12781752,159.91931392)(677.11281754,159.839314)(677.09281982,159.75931885)
\curveto(677.07281758,159.70931413)(677.06781758,159.65931418)(677.07781982,159.60931885)
\curveto(677.08781756,159.56931427)(677.08281757,159.52931431)(677.06281982,159.48931885)
\lineto(677.06281982,159.17431885)
\curveto(677.0528176,159.14431469)(677.0478176,159.10931473)(677.04781982,159.06931885)
\curveto(677.05781759,159.02931481)(677.06281759,158.98431485)(677.06281982,158.93431885)
\lineto(677.06281982,158.48431885)
\lineto(677.06281982,157.04431885)
\lineto(677.06281982,155.72431885)
\lineto(677.06281982,155.37931885)
\curveto(677.06281759,155.26931857)(677.03781761,155.17931866)(676.98781982,155.10931885)
\curveto(676.93781771,155.02931881)(676.8478178,154.98931885)(676.71781982,154.98931885)
\curveto(676.59781805,154.97931886)(676.47281818,154.97431886)(676.34281982,154.97431885)
\curveto(676.26281839,154.97431886)(676.18781846,154.97931886)(676.11781982,154.98931885)
\curveto(676.0478186,154.99931884)(675.98781866,155.02431881)(675.93781982,155.06431885)
\curveto(675.85781879,155.11431872)(675.81781883,155.20931863)(675.81781982,155.34931885)
\lineto(675.81781982,155.75431885)
\lineto(675.81781982,157.52431885)
\lineto(675.81781982,161.15431885)
\lineto(675.81781982,162.06931885)
\lineto(675.81781982,162.33931885)
\curveto(675.81781883,162.42931141)(675.83781881,162.49931134)(675.87781982,162.54931885)
\curveto(675.90781874,162.60931123)(675.95781869,162.64931119)(676.02781982,162.66931885)
\curveto(676.06781858,162.67931116)(676.12281853,162.68931115)(676.19281982,162.69931885)
\curveto(676.27281838,162.70931113)(676.3528183,162.71431112)(676.43281982,162.71431885)
\curveto(676.51281814,162.71431112)(676.58781806,162.70931113)(676.65781982,162.69931885)
\curveto(676.73781791,162.68931115)(676.79281786,162.67431116)(676.82281982,162.65431885)
\curveto(676.93281772,162.58431125)(676.98281767,162.49431134)(676.97281982,162.38431885)
\curveto(676.96281769,162.28431155)(676.97781767,162.16931167)(677.01781982,162.03931885)
\curveto(677.03781761,161.97931186)(677.07781757,161.92931191)(677.13781982,161.88931885)
\curveto(677.25781739,161.87931196)(677.3528173,161.92431191)(677.42281982,162.02431885)
\curveto(677.50281715,162.12431171)(677.58281707,162.20431163)(677.66281982,162.26431885)
\curveto(677.80281685,162.36431147)(677.94281671,162.45431138)(678.08281982,162.53431885)
\curveto(678.23281642,162.62431121)(678.40281625,162.69931114)(678.59281982,162.75931885)
\curveto(678.67281598,162.78931105)(678.75781589,162.80931103)(678.84781982,162.81931885)
\curveto(678.9478157,162.82931101)(679.04281561,162.84431099)(679.13281982,162.86431885)
\curveto(679.18281547,162.87431096)(679.23281542,162.87931096)(679.28281982,162.87931885)
\lineto(679.43281982,162.87931885)
}
}
{
\newrgbcolor{curcolor}{0 0 0}
\pscustom[linestyle=none,fillstyle=solid,fillcolor=curcolor]
{
\newpath
\moveto(685.0374292,165.06931885)
\curveto(685.18742719,165.06930877)(685.33742704,165.06430877)(685.4874292,165.05431885)
\curveto(685.63742674,165.05430878)(685.74242663,165.01430882)(685.8024292,164.93431885)
\curveto(685.85242652,164.87430896)(685.8774265,164.78930905)(685.8774292,164.67931885)
\curveto(685.88742649,164.57930926)(685.89242648,164.47430936)(685.8924292,164.36431885)
\lineto(685.8924292,163.49431885)
\curveto(685.89242648,163.41431042)(685.88742649,163.32931051)(685.8774292,163.23931885)
\curveto(685.8774265,163.15931068)(685.88742649,163.08931075)(685.9074292,163.02931885)
\curveto(685.94742643,162.88931095)(686.03742634,162.79931104)(686.1774292,162.75931885)
\curveto(686.22742615,162.74931109)(686.2724261,162.74431109)(686.3124292,162.74431885)
\lineto(686.4624292,162.74431885)
\lineto(686.8674292,162.74431885)
\curveto(687.02742535,162.75431108)(687.14242523,162.74431109)(687.2124292,162.71431885)
\curveto(687.30242507,162.65431118)(687.36242501,162.59431124)(687.3924292,162.53431885)
\curveto(687.41242496,162.49431134)(687.42242495,162.44931139)(687.4224292,162.39931885)
\lineto(687.4224292,162.24931885)
\curveto(687.42242495,162.1393117)(687.41742496,162.0343118)(687.4074292,161.93431885)
\curveto(687.39742498,161.84431199)(687.36242501,161.77431206)(687.3024292,161.72431885)
\curveto(687.24242513,161.67431216)(687.15742522,161.64431219)(687.0474292,161.63431885)
\lineto(686.7174292,161.63431885)
\curveto(686.60742577,161.64431219)(686.49742588,161.64931219)(686.3874292,161.64931885)
\curveto(686.2774261,161.64931219)(686.18242619,161.6343122)(686.1024292,161.60431885)
\curveto(686.03242634,161.57431226)(685.98242639,161.52431231)(685.9524292,161.45431885)
\curveto(685.92242645,161.38431245)(685.90242647,161.29931254)(685.8924292,161.19931885)
\curveto(685.88242649,161.10931273)(685.8774265,161.00931283)(685.8774292,160.89931885)
\curveto(685.88742649,160.79931304)(685.89242648,160.69931314)(685.8924292,160.59931885)
\lineto(685.8924292,157.62931885)
\curveto(685.89242648,157.40931643)(685.88742649,157.17431666)(685.8774292,156.92431885)
\curveto(685.8774265,156.68431715)(685.92242645,156.49931734)(686.0124292,156.36931885)
\curveto(686.06242631,156.28931755)(686.12742625,156.2343176)(686.2074292,156.20431885)
\curveto(686.28742609,156.17431766)(686.38242599,156.14931769)(686.4924292,156.12931885)
\curveto(686.52242585,156.11931772)(686.55242582,156.11431772)(686.5824292,156.11431885)
\curveto(686.62242575,156.12431771)(686.65742572,156.12431771)(686.6874292,156.11431885)
\lineto(686.8824292,156.11431885)
\curveto(686.98242539,156.11431772)(687.0724253,156.10431773)(687.1524292,156.08431885)
\curveto(687.24242513,156.07431776)(687.30742507,156.0393178)(687.3474292,155.97931885)
\curveto(687.36742501,155.94931789)(687.38242499,155.89431794)(687.3924292,155.81431885)
\curveto(687.41242496,155.74431809)(687.42242495,155.66931817)(687.4224292,155.58931885)
\curveto(687.43242494,155.50931833)(687.43242494,155.42931841)(687.4224292,155.34931885)
\curveto(687.41242496,155.27931856)(687.39242498,155.22431861)(687.3624292,155.18431885)
\curveto(687.32242505,155.11431872)(687.24742513,155.06431877)(687.1374292,155.03431885)
\curveto(687.05742532,155.01431882)(686.96742541,155.00431883)(686.8674292,155.00431885)
\curveto(686.76742561,155.01431882)(686.6774257,155.01931882)(686.5974292,155.01931885)
\curveto(686.53742584,155.01931882)(686.4774259,155.01431882)(686.4174292,155.00431885)
\curveto(686.35742602,155.00431883)(686.30242607,155.00931883)(686.2524292,155.01931885)
\lineto(686.0724292,155.01931885)
\curveto(686.02242635,155.02931881)(685.9724264,155.0343188)(685.9224292,155.03431885)
\curveto(685.88242649,155.04431879)(685.83742654,155.04931879)(685.7874292,155.04931885)
\curveto(685.58742679,155.09931874)(685.41242696,155.15431868)(685.2624292,155.21431885)
\curveto(685.12242725,155.27431856)(685.00242737,155.37931846)(684.9024292,155.52931885)
\curveto(684.76242761,155.72931811)(684.68242769,155.97931786)(684.6624292,156.27931885)
\curveto(684.64242773,156.58931725)(684.63242774,156.91931692)(684.6324292,157.26931885)
\lineto(684.6324292,161.19931885)
\curveto(684.60242777,161.32931251)(684.5724278,161.42431241)(684.5424292,161.48431885)
\curveto(684.52242785,161.54431229)(684.45242792,161.59431224)(684.3324292,161.63431885)
\curveto(684.29242808,161.64431219)(684.25242812,161.64431219)(684.2124292,161.63431885)
\curveto(684.1724282,161.62431221)(684.13242824,161.62931221)(684.0924292,161.64931885)
\lineto(683.8524292,161.64931885)
\curveto(683.72242865,161.64931219)(683.61242876,161.65931218)(683.5224292,161.67931885)
\curveto(683.44242893,161.70931213)(683.38742899,161.76931207)(683.3574292,161.85931885)
\curveto(683.33742904,161.89931194)(683.32242905,161.94431189)(683.3124292,161.99431885)
\lineto(683.3124292,162.14431885)
\curveto(683.31242906,162.28431155)(683.32242905,162.39931144)(683.3424292,162.48931885)
\curveto(683.36242901,162.58931125)(683.42242895,162.66431117)(683.5224292,162.71431885)
\curveto(683.63242874,162.75431108)(683.7724286,162.76431107)(683.9424292,162.74431885)
\curveto(684.12242825,162.72431111)(684.2724281,162.7343111)(684.3924292,162.77431885)
\curveto(684.48242789,162.82431101)(684.55242782,162.89431094)(684.6024292,162.98431885)
\curveto(684.62242775,163.04431079)(684.63242774,163.11931072)(684.6324292,163.20931885)
\lineto(684.6324292,163.46431885)
\lineto(684.6324292,164.39431885)
\lineto(684.6324292,164.63431885)
\curveto(684.63242774,164.72430911)(684.64242773,164.79930904)(684.6624292,164.85931885)
\curveto(684.70242767,164.9393089)(684.7774276,165.00430883)(684.8874292,165.05431885)
\curveto(684.91742746,165.05430878)(684.94242743,165.05430878)(684.9624292,165.05431885)
\curveto(684.99242738,165.06430877)(685.01742736,165.06930877)(685.0374292,165.06931885)
}
}
{
\newrgbcolor{curcolor}{0 0 0}
\pscustom[linestyle=none,fillstyle=solid,fillcolor=curcolor]
{
\newpath
\moveto(695.55922607,159.17431885)
\curveto(695.57921839,159.07431476)(695.57921839,158.95931488)(695.55922607,158.82931885)
\curveto(695.54921842,158.70931513)(695.51921845,158.62431521)(695.46922607,158.57431885)
\curveto(695.41921855,158.5343153)(695.34421862,158.50431533)(695.24422607,158.48431885)
\curveto(695.15421881,158.47431536)(695.04921892,158.46931537)(694.92922607,158.46931885)
\lineto(694.56922607,158.46931885)
\curveto(694.44921952,158.47931536)(694.34421962,158.48431535)(694.25422607,158.48431885)
\lineto(690.41422607,158.48431885)
\curveto(690.33422363,158.48431535)(690.25422371,158.47931536)(690.17422607,158.46931885)
\curveto(690.09422387,158.46931537)(690.02922394,158.45431538)(689.97922607,158.42431885)
\curveto(689.93922403,158.40431543)(689.89922407,158.36431547)(689.85922607,158.30431885)
\curveto(689.83922413,158.27431556)(689.81922415,158.22931561)(689.79922607,158.16931885)
\curveto(689.77922419,158.11931572)(689.77922419,158.06931577)(689.79922607,158.01931885)
\curveto(689.80922416,157.96931587)(689.81422415,157.92431591)(689.81422607,157.88431885)
\curveto(689.81422415,157.84431599)(689.81922415,157.80431603)(689.82922607,157.76431885)
\curveto(689.84922412,157.68431615)(689.8692241,157.59931624)(689.88922607,157.50931885)
\curveto(689.90922406,157.42931641)(689.93922403,157.34931649)(689.97922607,157.26931885)
\curveto(690.20922376,156.72931711)(690.58922338,156.34431749)(691.11922607,156.11431885)
\curveto(691.17922279,156.08431775)(691.24422272,156.05931778)(691.31422607,156.03931885)
\lineto(691.52422607,155.97931885)
\curveto(691.55422241,155.96931787)(691.60422236,155.96431787)(691.67422607,155.96431885)
\curveto(691.81422215,155.92431791)(691.99922197,155.90431793)(692.22922607,155.90431885)
\curveto(692.45922151,155.90431793)(692.64422132,155.92431791)(692.78422607,155.96431885)
\curveto(692.92422104,156.00431783)(693.04922092,156.04431779)(693.15922607,156.08431885)
\curveto(693.27922069,156.1343177)(693.38922058,156.19431764)(693.48922607,156.26431885)
\curveto(693.59922037,156.3343175)(693.69422027,156.41431742)(693.77422607,156.50431885)
\curveto(693.85422011,156.60431723)(693.92422004,156.70931713)(693.98422607,156.81931885)
\curveto(694.04421992,156.91931692)(694.09421987,157.02431681)(694.13422607,157.13431885)
\curveto(694.18421978,157.24431659)(694.2642197,157.32431651)(694.37422607,157.37431885)
\curveto(694.41421955,157.39431644)(694.47921949,157.40931643)(694.56922607,157.41931885)
\curveto(694.65921931,157.42931641)(694.74921922,157.42931641)(694.83922607,157.41931885)
\curveto(694.92921904,157.41931642)(695.01421895,157.41431642)(695.09422607,157.40431885)
\curveto(695.17421879,157.39431644)(695.22921874,157.37431646)(695.25922607,157.34431885)
\curveto(695.35921861,157.27431656)(695.38421858,157.15931668)(695.33422607,156.99931885)
\curveto(695.25421871,156.72931711)(695.14921882,156.48931735)(695.01922607,156.27931885)
\curveto(694.81921915,155.95931788)(694.58921938,155.69431814)(694.32922607,155.48431885)
\curveto(694.07921989,155.28431855)(693.75922021,155.11931872)(693.36922607,154.98931885)
\curveto(693.2692207,154.94931889)(693.1692208,154.92431891)(693.06922607,154.91431885)
\curveto(692.969221,154.89431894)(692.8642211,154.87431896)(692.75422607,154.85431885)
\curveto(692.70422126,154.84431899)(692.65422131,154.839319)(692.60422607,154.83931885)
\curveto(692.5642214,154.839319)(692.51922145,154.834319)(692.46922607,154.82431885)
\lineto(692.31922607,154.82431885)
\curveto(692.2692217,154.81431902)(692.20922176,154.80931903)(692.13922607,154.80931885)
\curveto(692.07922189,154.80931903)(692.02922194,154.81431902)(691.98922607,154.82431885)
\lineto(691.85422607,154.82431885)
\curveto(691.80422216,154.834319)(691.75922221,154.839319)(691.71922607,154.83931885)
\curveto(691.67922229,154.839319)(691.63922233,154.84431899)(691.59922607,154.85431885)
\curveto(691.54922242,154.86431897)(691.49422247,154.87431896)(691.43422607,154.88431885)
\curveto(691.37422259,154.88431895)(691.31922265,154.88931895)(691.26922607,154.89931885)
\curveto(691.17922279,154.91931892)(691.08922288,154.94431889)(690.99922607,154.97431885)
\curveto(690.90922306,154.99431884)(690.82422314,155.01931882)(690.74422607,155.04931885)
\curveto(690.70422326,155.06931877)(690.6692233,155.07931876)(690.63922607,155.07931885)
\curveto(690.60922336,155.08931875)(690.57422339,155.10431873)(690.53422607,155.12431885)
\curveto(690.38422358,155.19431864)(690.22422374,155.27931856)(690.05422607,155.37931885)
\curveto(689.7642242,155.56931827)(689.51422445,155.79931804)(689.30422607,156.06931885)
\curveto(689.10422486,156.34931749)(688.93422503,156.65931718)(688.79422607,156.99931885)
\curveto(688.74422522,157.10931673)(688.70422526,157.22431661)(688.67422607,157.34431885)
\curveto(688.65422531,157.46431637)(688.62422534,157.58431625)(688.58422607,157.70431885)
\curveto(688.57422539,157.74431609)(688.5692254,157.77931606)(688.56922607,157.80931885)
\curveto(688.5692254,157.839316)(688.5642254,157.87931596)(688.55422607,157.92931885)
\curveto(688.53422543,158.00931583)(688.51922545,158.09431574)(688.50922607,158.18431885)
\curveto(688.49922547,158.27431556)(688.48422548,158.36431547)(688.46422607,158.45431885)
\lineto(688.46422607,158.66431885)
\curveto(688.45422551,158.70431513)(688.44422552,158.75931508)(688.43422607,158.82931885)
\curveto(688.43422553,158.90931493)(688.43922553,158.97431486)(688.44922607,159.02431885)
\lineto(688.44922607,159.18931885)
\curveto(688.4692255,159.2393146)(688.47422549,159.28931455)(688.46422607,159.33931885)
\curveto(688.4642255,159.39931444)(688.4692255,159.45431438)(688.47922607,159.50431885)
\curveto(688.51922545,159.66431417)(688.54922542,159.82431401)(688.56922607,159.98431885)
\curveto(688.59922537,160.14431369)(688.64422532,160.29431354)(688.70422607,160.43431885)
\curveto(688.75422521,160.54431329)(688.79922517,160.65431318)(688.83922607,160.76431885)
\curveto(688.88922508,160.88431295)(688.94422502,160.99931284)(689.00422607,161.10931885)
\curveto(689.22422474,161.45931238)(689.47422449,161.75931208)(689.75422607,162.00931885)
\curveto(690.03422393,162.26931157)(690.37922359,162.48431135)(690.78922607,162.65431885)
\curveto(690.90922306,162.70431113)(691.02922294,162.7393111)(691.14922607,162.75931885)
\curveto(691.27922269,162.78931105)(691.41422255,162.81931102)(691.55422607,162.84931885)
\curveto(691.60422236,162.85931098)(691.64922232,162.86431097)(691.68922607,162.86431885)
\curveto(691.72922224,162.87431096)(691.77422219,162.87931096)(691.82422607,162.87931885)
\curveto(691.84422212,162.88931095)(691.8692221,162.88931095)(691.89922607,162.87931885)
\curveto(691.92922204,162.86931097)(691.95422201,162.87431096)(691.97422607,162.89431885)
\curveto(692.39422157,162.90431093)(692.75922121,162.85931098)(693.06922607,162.75931885)
\curveto(693.37922059,162.66931117)(693.65922031,162.54431129)(693.90922607,162.38431885)
\curveto(693.95922001,162.36431147)(693.99921997,162.3343115)(694.02922607,162.29431885)
\curveto(694.05921991,162.26431157)(694.09421987,162.2393116)(694.13422607,162.21931885)
\curveto(694.21421975,162.15931168)(694.29421967,162.08931175)(694.37422607,162.00931885)
\curveto(694.4642195,161.92931191)(694.53921943,161.84931199)(694.59922607,161.76931885)
\curveto(694.75921921,161.55931228)(694.89421907,161.35931248)(695.00422607,161.16931885)
\curveto(695.07421889,161.05931278)(695.12921884,160.9393129)(695.16922607,160.80931885)
\curveto(695.20921876,160.67931316)(695.25421871,160.54931329)(695.30422607,160.41931885)
\curveto(695.35421861,160.28931355)(695.38921858,160.15431368)(695.40922607,160.01431885)
\curveto(695.43921853,159.87431396)(695.47421849,159.7343141)(695.51422607,159.59431885)
\curveto(695.52421844,159.52431431)(695.52921844,159.45431438)(695.52922607,159.38431885)
\lineto(695.55922607,159.17431885)
\moveto(694.10422607,159.68431885)
\curveto(694.13421983,159.72431411)(694.15921981,159.77431406)(694.17922607,159.83431885)
\curveto(694.19921977,159.90431393)(694.19921977,159.97431386)(694.17922607,160.04431885)
\curveto(694.11921985,160.26431357)(694.03421993,160.46931337)(693.92422607,160.65931885)
\curveto(693.78422018,160.88931295)(693.62922034,161.08431275)(693.45922607,161.24431885)
\curveto(693.28922068,161.40431243)(693.0692209,161.5393123)(692.79922607,161.64931885)
\curveto(692.72922124,161.66931217)(692.65922131,161.68431215)(692.58922607,161.69431885)
\curveto(692.51922145,161.71431212)(692.44422152,161.7343121)(692.36422607,161.75431885)
\curveto(692.28422168,161.77431206)(692.19922177,161.78431205)(692.10922607,161.78431885)
\lineto(691.85422607,161.78431885)
\curveto(691.82422214,161.76431207)(691.78922218,161.75431208)(691.74922607,161.75431885)
\curveto(691.70922226,161.76431207)(691.67422229,161.76431207)(691.64422607,161.75431885)
\lineto(691.40422607,161.69431885)
\curveto(691.33422263,161.68431215)(691.2642227,161.66931217)(691.19422607,161.64931885)
\curveto(690.90422306,161.52931231)(690.6692233,161.37931246)(690.48922607,161.19931885)
\curveto(690.31922365,161.01931282)(690.1642238,160.79431304)(690.02422607,160.52431885)
\curveto(689.99422397,160.47431336)(689.964224,160.40931343)(689.93422607,160.32931885)
\curveto(689.90422406,160.25931358)(689.87922409,160.17931366)(689.85922607,160.08931885)
\curveto(689.83922413,159.99931384)(689.83422413,159.91431392)(689.84422607,159.83431885)
\curveto(689.85422411,159.75431408)(689.88922408,159.69431414)(689.94922607,159.65431885)
\curveto(690.02922394,159.59431424)(690.1642238,159.56431427)(690.35422607,159.56431885)
\curveto(690.55422341,159.57431426)(690.72422324,159.57931426)(690.86422607,159.57931885)
\lineto(693.14422607,159.57931885)
\curveto(693.29422067,159.57931426)(693.47422049,159.57431426)(693.68422607,159.56431885)
\curveto(693.89422007,159.56431427)(694.03421993,159.60431423)(694.10422607,159.68431885)
}
}
{
\newrgbcolor{curcolor}{0.80000001 0.80000001 0.80000001}
\pscustom[linestyle=none,fillstyle=solid,fillcolor=curcolor]
{
\newpath
\moveto(606.37554932,165.71435547)
\lineto(621.37554932,165.71435547)
\lineto(621.37554932,150.71435547)
\lineto(606.37554932,150.71435547)
\closepath
}
}
{
\newrgbcolor{curcolor}{0 0 0}
\pscustom[linestyle=none,fillstyle=solid,fillcolor=curcolor]
{
\newpath
\moveto(634.31375732,132.92714111)
\curveto(634.33374778,132.87714037)(634.35874775,132.81714043)(634.38875732,132.74714111)
\curveto(634.41874769,132.67714057)(634.43874767,132.60214064)(634.44875732,132.52214111)
\curveto(634.46874764,132.45214079)(634.46874764,132.38214086)(634.44875732,132.31214111)
\curveto(634.43874767,132.25214099)(634.39874771,132.20714104)(634.32875732,132.17714111)
\curveto(634.27874783,132.15714109)(634.21874789,132.1471411)(634.14875732,132.14714111)
\lineto(633.93875732,132.14714111)
\lineto(633.48875732,132.14714111)
\curveto(633.33874877,132.1471411)(633.21874889,132.17214107)(633.12875732,132.22214111)
\curveto(633.02874908,132.28214096)(632.95374916,132.38714086)(632.90375732,132.53714111)
\curveto(632.86374925,132.68714056)(632.81874929,132.82214042)(632.76875732,132.94214111)
\curveto(632.65874945,133.20214004)(632.55874955,133.47213977)(632.46875732,133.75214111)
\curveto(632.37874973,134.03213921)(632.27874983,134.30713894)(632.16875732,134.57714111)
\curveto(632.13874997,134.66713858)(632.10875,134.75213849)(632.07875732,134.83214111)
\curveto(632.05875005,134.91213833)(632.02875008,134.98713826)(631.98875732,135.05714111)
\curveto(631.95875015,135.12713812)(631.9137502,135.18713806)(631.85375732,135.23714111)
\curveto(631.79375032,135.28713796)(631.7137504,135.32713792)(631.61375732,135.35714111)
\curveto(631.56375055,135.37713787)(631.50375061,135.38213786)(631.43375732,135.37214111)
\lineto(631.23875732,135.37214111)
\lineto(628.40375732,135.37214111)
\lineto(628.10375732,135.37214111)
\curveto(627.99375412,135.38213786)(627.88875422,135.38213786)(627.78875732,135.37214111)
\curveto(627.68875442,135.36213788)(627.59375452,135.3471379)(627.50375732,135.32714111)
\curveto(627.42375469,135.30713794)(627.36375475,135.26713798)(627.32375732,135.20714111)
\curveto(627.24375487,135.10713814)(627.18375493,134.99213825)(627.14375732,134.86214111)
\curveto(627.113755,134.7421385)(627.07375504,134.61713863)(627.02375732,134.48714111)
\curveto(626.92375519,134.25713899)(626.82875528,134.01713923)(626.73875732,133.76714111)
\curveto(626.65875545,133.51713973)(626.56875554,133.27713997)(626.46875732,133.04714111)
\curveto(626.44875566,132.98714026)(626.42375569,132.91714033)(626.39375732,132.83714111)
\curveto(626.37375574,132.76714048)(626.34875576,132.69214055)(626.31875732,132.61214111)
\curveto(626.28875582,132.53214071)(626.25375586,132.45714079)(626.21375732,132.38714111)
\curveto(626.18375593,132.32714092)(626.14875596,132.28214096)(626.10875732,132.25214111)
\curveto(626.02875608,132.19214105)(625.91875619,132.15714109)(625.77875732,132.14714111)
\lineto(625.35875732,132.14714111)
\lineto(625.11875732,132.14714111)
\curveto(625.04875706,132.15714109)(624.98875712,132.18214106)(624.93875732,132.22214111)
\curveto(624.88875722,132.25214099)(624.85875725,132.29714095)(624.84875732,132.35714111)
\curveto(624.84875726,132.41714083)(624.85375726,132.47714077)(624.86375732,132.53714111)
\curveto(624.88375723,132.60714064)(624.90375721,132.67214057)(624.92375732,132.73214111)
\curveto(624.95375716,132.80214044)(624.97875713,132.85214039)(624.99875732,132.88214111)
\curveto(625.13875697,133.20214004)(625.26375685,133.51713973)(625.37375732,133.82714111)
\curveto(625.48375663,134.1471391)(625.60375651,134.46713878)(625.73375732,134.78714111)
\curveto(625.82375629,135.00713824)(625.9087562,135.22213802)(625.98875732,135.43214111)
\curveto(626.06875604,135.65213759)(626.15375596,135.87213737)(626.24375732,136.09214111)
\curveto(626.54375557,136.81213643)(626.82875528,137.53713571)(627.09875732,138.26714111)
\curveto(627.36875474,139.00713424)(627.65375446,139.7421335)(627.95375732,140.47214111)
\curveto(628.06375405,140.73213251)(628.16375395,140.99713225)(628.25375732,141.26714111)
\curveto(628.35375376,141.53713171)(628.45875365,141.80213144)(628.56875732,142.06214111)
\curveto(628.61875349,142.17213107)(628.66375345,142.29213095)(628.70375732,142.42214111)
\curveto(628.75375336,142.56213068)(628.82375329,142.66213058)(628.91375732,142.72214111)
\curveto(628.95375316,142.76213048)(629.01875309,142.79213045)(629.10875732,142.81214111)
\curveto(629.12875298,142.82213042)(629.14875296,142.82213042)(629.16875732,142.81214111)
\curveto(629.19875291,142.81213043)(629.22375289,142.81713043)(629.24375732,142.82714111)
\curveto(629.42375269,142.82713042)(629.63375248,142.82713042)(629.87375732,142.82714111)
\curveto(630.113752,142.83713041)(630.28875182,142.80213044)(630.39875732,142.72214111)
\curveto(630.47875163,142.66213058)(630.53875157,142.56213068)(630.57875732,142.42214111)
\curveto(630.62875148,142.29213095)(630.67875143,142.17213107)(630.72875732,142.06214111)
\curveto(630.82875128,141.83213141)(630.91875119,141.60213164)(630.99875732,141.37214111)
\curveto(631.07875103,141.1421321)(631.16875094,140.91213233)(631.26875732,140.68214111)
\curveto(631.34875076,140.48213276)(631.42375069,140.27713297)(631.49375732,140.06714111)
\curveto(631.57375054,139.85713339)(631.65875045,139.65213359)(631.74875732,139.45214111)
\curveto(632.04875006,138.72213452)(632.33374978,137.98213526)(632.60375732,137.23214111)
\curveto(632.88374923,136.49213675)(633.17874893,135.75713749)(633.48875732,135.02714111)
\curveto(633.52874858,134.93713831)(633.55874855,134.85213839)(633.57875732,134.77214111)
\curveto(633.6087485,134.69213855)(633.63874847,134.60713864)(633.66875732,134.51714111)
\curveto(633.77874833,134.25713899)(633.88374823,133.99213925)(633.98375732,133.72214111)
\curveto(634.09374802,133.45213979)(634.20374791,133.18714006)(634.31375732,132.92714111)
\moveto(631.10375732,136.57214111)
\curveto(631.19375092,136.60213664)(631.24875086,136.65213659)(631.26875732,136.72214111)
\curveto(631.29875081,136.79213645)(631.30375081,136.86713638)(631.28375732,136.94714111)
\curveto(631.27375084,137.03713621)(631.24875086,137.12213612)(631.20875732,137.20214111)
\curveto(631.17875093,137.29213595)(631.14875096,137.36713588)(631.11875732,137.42714111)
\curveto(631.09875101,137.46713578)(631.08875102,137.50213574)(631.08875732,137.53214111)
\curveto(631.08875102,137.56213568)(631.07875103,137.59713565)(631.05875732,137.63714111)
\lineto(630.96875732,137.87714111)
\curveto(630.94875116,137.96713528)(630.91875119,138.05713519)(630.87875732,138.14714111)
\curveto(630.72875138,138.50713474)(630.59375152,138.87213437)(630.47375732,139.24214111)
\curveto(630.36375175,139.62213362)(630.23375188,139.99213325)(630.08375732,140.35214111)
\curveto(630.03375208,140.46213278)(629.98875212,140.57213267)(629.94875732,140.68214111)
\curveto(629.91875219,140.79213245)(629.87875223,140.89713235)(629.82875732,140.99714111)
\curveto(629.8087523,141.0471322)(629.78375233,141.09213215)(629.75375732,141.13214111)
\curveto(629.73375238,141.18213206)(629.68375243,141.20713204)(629.60375732,141.20714111)
\curveto(629.58375253,141.18713206)(629.56375255,141.17213207)(629.54375732,141.16214111)
\curveto(629.52375259,141.15213209)(629.50375261,141.13713211)(629.48375732,141.11714111)
\curveto(629.44375267,141.06713218)(629.4137527,141.01213223)(629.39375732,140.95214111)
\curveto(629.37375274,140.90213234)(629.35375276,140.8471324)(629.33375732,140.78714111)
\curveto(629.28375283,140.67713257)(629.24375287,140.56713268)(629.21375732,140.45714111)
\curveto(629.18375293,140.3471329)(629.14375297,140.23713301)(629.09375732,140.12714111)
\curveto(628.92375319,139.73713351)(628.77375334,139.3421339)(628.64375732,138.94214111)
\curveto(628.52375359,138.5421347)(628.38375373,138.15213509)(628.22375732,137.77214111)
\lineto(628.16375732,137.62214111)
\curveto(628.15375396,137.57213567)(628.13875397,137.52213572)(628.11875732,137.47214111)
\lineto(628.02875732,137.23214111)
\curveto(627.99875411,137.15213609)(627.97375414,137.07213617)(627.95375732,136.99214111)
\curveto(627.93375418,136.9421363)(627.92375419,136.88713636)(627.92375732,136.82714111)
\curveto(627.93375418,136.76713648)(627.94875416,136.71713653)(627.96875732,136.67714111)
\curveto(628.01875409,136.59713665)(628.12375399,136.55213669)(628.28375732,136.54214111)
\lineto(628.73375732,136.54214111)
\lineto(630.33875732,136.54214111)
\curveto(630.44875166,136.5421367)(630.58375153,136.53713671)(630.74375732,136.52714111)
\curveto(630.90375121,136.52713672)(631.02375109,136.5421367)(631.10375732,136.57214111)
}
}
{
\newrgbcolor{curcolor}{0 0 0}
\pscustom[linestyle=none,fillstyle=solid,fillcolor=curcolor]
{
\newpath
\moveto(635.89531982,139.87214111)
\lineto(636.33031982,139.87214111)
\curveto(636.48031786,139.87213337)(636.58531775,139.83213341)(636.64531982,139.75214111)
\curveto(636.69531764,139.67213357)(636.72031762,139.57213367)(636.72031982,139.45214111)
\curveto(636.73031761,139.33213391)(636.7353176,139.21213403)(636.73531982,139.09214111)
\lineto(636.73531982,137.66714111)
\lineto(636.73531982,135.40214111)
\lineto(636.73531982,134.71214111)
\curveto(636.7353176,134.48213876)(636.76031758,134.28213896)(636.81031982,134.11214111)
\curveto(636.97031737,133.66213958)(637.27031707,133.3471399)(637.71031982,133.16714111)
\curveto(637.93031641,133.07714017)(638.19531614,133.0421402)(638.50531982,133.06214111)
\curveto(638.81531552,133.09214015)(639.06531527,133.1471401)(639.25531982,133.22714111)
\curveto(639.58531475,133.36713988)(639.84531449,133.5421397)(640.03531982,133.75214111)
\curveto(640.2353141,133.97213927)(640.39031395,134.25713899)(640.50031982,134.60714111)
\curveto(640.53031381,134.68713856)(640.55031379,134.76713848)(640.56031982,134.84714111)
\curveto(640.57031377,134.92713832)(640.58531375,135.01213823)(640.60531982,135.10214111)
\curveto(640.61531372,135.15213809)(640.61531372,135.19713805)(640.60531982,135.23714111)
\curveto(640.60531373,135.27713797)(640.61531372,135.32213792)(640.63531982,135.37214111)
\lineto(640.63531982,135.68714111)
\curveto(640.65531368,135.76713748)(640.66031368,135.85713739)(640.65031982,135.95714111)
\curveto(640.6403137,136.06713718)(640.6353137,136.16713708)(640.63531982,136.25714111)
\lineto(640.63531982,137.42714111)
\lineto(640.63531982,139.01714111)
\curveto(640.6353137,139.13713411)(640.63031371,139.26213398)(640.62031982,139.39214111)
\curveto(640.62031372,139.53213371)(640.64531369,139.6421336)(640.69531982,139.72214111)
\curveto(640.7353136,139.77213347)(640.78031356,139.80213344)(640.83031982,139.81214111)
\curveto(640.89031345,139.83213341)(640.96031338,139.85213339)(641.04031982,139.87214111)
\lineto(641.26531982,139.87214111)
\curveto(641.38531295,139.87213337)(641.49031285,139.86713338)(641.58031982,139.85714111)
\curveto(641.68031266,139.8471334)(641.75531258,139.80213344)(641.80531982,139.72214111)
\curveto(641.85531248,139.67213357)(641.88031246,139.59713365)(641.88031982,139.49714111)
\lineto(641.88031982,139.21214111)
\lineto(641.88031982,138.19214111)
\lineto(641.88031982,134.15714111)
\lineto(641.88031982,132.80714111)
\curveto(641.88031246,132.68714056)(641.87531246,132.57214067)(641.86531982,132.46214111)
\curveto(641.86531247,132.36214088)(641.83031251,132.28714096)(641.76031982,132.23714111)
\curveto(641.72031262,132.20714104)(641.66031268,132.18214106)(641.58031982,132.16214111)
\curveto(641.50031284,132.15214109)(641.41031293,132.1421411)(641.31031982,132.13214111)
\curveto(641.22031312,132.13214111)(641.13031321,132.13714111)(641.04031982,132.14714111)
\curveto(640.96031338,132.15714109)(640.90031344,132.17714107)(640.86031982,132.20714111)
\curveto(640.81031353,132.247141)(640.76531357,132.31214093)(640.72531982,132.40214111)
\curveto(640.71531362,132.4421408)(640.70531363,132.49714075)(640.69531982,132.56714111)
\curveto(640.69531364,132.63714061)(640.69031365,132.70214054)(640.68031982,132.76214111)
\curveto(640.67031367,132.83214041)(640.65031369,132.88714036)(640.62031982,132.92714111)
\curveto(640.59031375,132.96714028)(640.54531379,132.98214026)(640.48531982,132.97214111)
\curveto(640.40531393,132.95214029)(640.32531401,132.89214035)(640.24531982,132.79214111)
\curveto(640.16531417,132.70214054)(640.09031425,132.63214061)(640.02031982,132.58214111)
\curveto(639.80031454,132.42214082)(639.55031479,132.28214096)(639.27031982,132.16214111)
\curveto(639.16031518,132.11214113)(639.04531529,132.08214116)(638.92531982,132.07214111)
\curveto(638.81531552,132.05214119)(638.70031564,132.02714122)(638.58031982,131.99714111)
\curveto(638.53031581,131.98714126)(638.47531586,131.98714126)(638.41531982,131.99714111)
\curveto(638.36531597,132.00714124)(638.31531602,132.00214124)(638.26531982,131.98214111)
\curveto(638.16531617,131.96214128)(638.07531626,131.96214128)(637.99531982,131.98214111)
\lineto(637.84531982,131.98214111)
\curveto(637.79531654,132.00214124)(637.7353166,132.01214123)(637.66531982,132.01214111)
\curveto(637.60531673,132.01214123)(637.55031679,132.01714123)(637.50031982,132.02714111)
\curveto(637.46031688,132.0471412)(637.42031692,132.05714119)(637.38031982,132.05714111)
\curveto(637.35031699,132.0471412)(637.31031703,132.05214119)(637.26031982,132.07214111)
\lineto(637.02031982,132.13214111)
\curveto(636.95031739,132.15214109)(636.87531746,132.18214106)(636.79531982,132.22214111)
\curveto(636.5353178,132.33214091)(636.31531802,132.47714077)(636.13531982,132.65714111)
\curveto(635.96531837,132.8471404)(635.82531851,133.07214017)(635.71531982,133.33214111)
\curveto(635.67531866,133.42213982)(635.64531869,133.51213973)(635.62531982,133.60214111)
\lineto(635.56531982,133.90214111)
\curveto(635.54531879,133.96213928)(635.5353188,134.01713923)(635.53531982,134.06714111)
\curveto(635.54531879,134.12713912)(635.5403188,134.19213905)(635.52031982,134.26214111)
\curveto(635.51031883,134.28213896)(635.50531883,134.30713894)(635.50531982,134.33714111)
\curveto(635.50531883,134.37713887)(635.50031884,134.41213883)(635.49031982,134.44214111)
\lineto(635.49031982,134.59214111)
\curveto(635.48031886,134.63213861)(635.47531886,134.67713857)(635.47531982,134.72714111)
\curveto(635.48531885,134.78713846)(635.49031885,134.8421384)(635.49031982,134.89214111)
\lineto(635.49031982,135.49214111)
\lineto(635.49031982,138.25214111)
\lineto(635.49031982,139.21214111)
\lineto(635.49031982,139.48214111)
\curveto(635.49031885,139.57213367)(635.51031883,139.6471336)(635.55031982,139.70714111)
\curveto(635.59031875,139.77713347)(635.66531867,139.82713342)(635.77531982,139.85714111)
\curveto(635.79531854,139.86713338)(635.81531852,139.86713338)(635.83531982,139.85714111)
\curveto(635.85531848,139.85713339)(635.87531846,139.86213338)(635.89531982,139.87214111)
}
}
{
\newrgbcolor{curcolor}{0 0 0}
\pscustom[linestyle=none,fillstyle=solid,fillcolor=curcolor]
{
\newpath
\moveto(643.7949292,139.87214111)
\lineto(644.3199292,139.87214111)
\curveto(644.51992754,139.88213336)(644.66992739,139.86213338)(644.7699292,139.81214111)
\curveto(644.88992717,139.76213348)(644.98492708,139.68213356)(645.0549292,139.57214111)
\curveto(645.13492693,139.46213378)(645.20992685,139.35213389)(645.2799292,139.24214111)
\curveto(645.40992665,139.0421342)(645.53992652,138.8471344)(645.6699292,138.65714111)
\curveto(645.79992626,138.47713477)(645.93492613,138.28713496)(646.0749292,138.08714111)
\curveto(646.12492594,138.00713524)(646.17492589,137.93213531)(646.2249292,137.86214111)
\curveto(646.28492578,137.79213545)(646.33992572,137.72213552)(646.3899292,137.65214111)
\curveto(646.42992563,137.59213565)(646.46992559,137.53713571)(646.5099292,137.48714111)
\curveto(646.54992551,137.43713581)(646.60992545,137.40213584)(646.6899292,137.38214111)
\curveto(646.73992532,137.36213588)(646.77992528,137.36213588)(646.8099292,137.38214111)
\curveto(646.84992521,137.41213583)(646.87992518,137.43713581)(646.8999292,137.45714111)
\curveto(646.97992508,137.50713574)(647.04492502,137.57713567)(647.0949292,137.66714111)
\curveto(647.15492491,137.75713549)(647.20992485,137.8421354)(647.2599292,137.92214111)
\curveto(647.40992465,138.12213512)(647.5599245,138.32713492)(647.7099292,138.53714111)
\lineto(648.1599292,139.16714111)
\curveto(648.23992382,139.27713397)(648.31992374,139.39213385)(648.3999292,139.51214111)
\curveto(648.47992358,139.63213361)(648.57492349,139.72713352)(648.6849292,139.79714111)
\curveto(648.7649233,139.8471334)(648.8599232,139.87213337)(648.9699292,139.87214111)
\lineto(649.3149292,139.87214111)
\lineto(649.4499292,139.87214111)
\curveto(649.49992256,139.87213337)(649.54992251,139.86713338)(649.5999292,139.85714111)
\lineto(649.6749292,139.85714111)
\curveto(649.79492227,139.83713341)(649.87492219,139.79713345)(649.9149292,139.73714111)
\curveto(649.93492213,139.68713356)(649.92992213,139.63213361)(649.8999292,139.57214111)
\curveto(649.87992218,139.52213372)(649.8599222,139.48213376)(649.8399292,139.45214111)
\lineto(649.6299292,139.15214111)
\curveto(649.5599225,139.06213418)(649.48492258,138.96713428)(649.4049292,138.86714111)
\curveto(649.17492289,138.5471347)(648.93992312,138.23213501)(648.6999292,137.92214111)
\curveto(648.46992359,137.62213562)(648.23992382,137.31213593)(648.0099292,136.99214111)
\curveto(647.9599241,136.91213633)(647.90492416,136.83213641)(647.8449292,136.75214111)
\curveto(647.78492428,136.68213656)(647.72992433,136.60213664)(647.6799292,136.51214111)
\curveto(647.6599244,136.48213676)(647.63992442,136.4421368)(647.6199292,136.39214111)
\curveto(647.59992446,136.35213689)(647.59992446,136.30213694)(647.6199292,136.24214111)
\curveto(647.63992442,136.15213709)(647.66992439,136.07713717)(647.7099292,136.01714111)
\curveto(647.7599243,135.95713729)(647.80992425,135.89213735)(647.8599292,135.82214111)
\lineto(648.0399292,135.55214111)
\curveto(648.10992395,135.46213778)(648.17492389,135.37213787)(648.2349292,135.28214111)
\lineto(648.9249292,134.32214111)
\lineto(649.6149292,133.36214111)
\curveto(649.69492237,133.25213999)(649.77492229,133.13714011)(649.8549292,133.01714111)
\lineto(650.0949292,132.68714111)
\curveto(650.14492192,132.61714063)(650.18492188,132.55214069)(650.2149292,132.49214111)
\curveto(650.25492181,132.4421408)(650.2649218,132.36214088)(650.2449292,132.25214111)
\curveto(650.22492184,132.242141)(650.20492186,132.22714102)(650.1849292,132.20714111)
\curveto(650.17492189,132.19714105)(650.1599219,132.18714106)(650.1399292,132.17714111)
\curveto(650.08992197,132.15714109)(650.02492204,132.1471411)(649.9449292,132.14714111)
\lineto(649.7049292,132.14714111)
\lineto(649.1949292,132.14714111)
\curveto(649.05492301,132.15714109)(648.92992313,132.20214104)(648.8199292,132.28214111)
\curveto(648.76992329,132.31214093)(648.72992333,132.3471409)(648.6999292,132.38714111)
\curveto(648.67992338,132.43714081)(648.65492341,132.48714076)(648.6249292,132.53714111)
\lineto(648.4749292,132.74714111)
\curveto(648.42492364,132.81714043)(648.37492369,132.89214035)(648.3249292,132.97214111)
\lineto(647.3799292,134.36714111)
\curveto(647.32992473,134.4471388)(647.27992478,134.52213872)(647.2299292,134.59214111)
\curveto(647.17992488,134.66213858)(647.12992493,134.73713851)(647.0799292,134.81714111)
\curveto(647.02992503,134.88713836)(646.97992508,134.9471383)(646.9299292,134.99714111)
\curveto(646.88992517,135.05713819)(646.82992523,135.09713815)(646.7499292,135.11714111)
\curveto(646.69992536,135.13713811)(646.64992541,135.12713812)(646.5999292,135.08714111)
\curveto(646.5599255,135.05713819)(646.52992553,135.03213821)(646.5099292,135.01214111)
\curveto(646.42992563,134.93213831)(646.3599257,134.8421384)(646.2999292,134.74214111)
\curveto(646.23992582,134.6421386)(646.17992588,134.5471387)(646.1199292,134.45714111)
\curveto(645.94992611,134.19713905)(645.77492629,133.93713931)(645.5949292,133.67714111)
\curveto(645.42492664,133.42713982)(645.24992681,133.17714007)(645.0699292,132.92714111)
\curveto(645.01992704,132.8471404)(644.9649271,132.76714048)(644.9049292,132.68714111)
\lineto(644.7549292,132.44714111)
\curveto(644.73492733,132.41714083)(644.70992735,132.38214086)(644.6799292,132.34214111)
\curveto(644.6599274,132.31214093)(644.63492743,132.28714096)(644.6049292,132.26714111)
\curveto(644.50492756,132.19714105)(644.38492768,132.15714109)(644.2449292,132.14714111)
\lineto(643.7949292,132.14714111)
\lineto(643.5699292,132.14714111)
\curveto(643.49992856,132.1471411)(643.43992862,132.15714109)(643.3899292,132.17714111)
\curveto(643.3599287,132.19714105)(643.33492873,132.21214103)(643.3149292,132.22214111)
\curveto(643.30492876,132.242141)(643.28992877,132.26214098)(643.2699292,132.28214111)
\curveto(643.2599288,132.39214085)(643.27492879,132.47714077)(643.3149292,132.53714111)
\curveto(643.3649287,132.59714065)(643.41492865,132.66214058)(643.4649292,132.73214111)
\curveto(643.54492852,132.8421404)(643.61992844,132.9421403)(643.6899292,133.03214111)
\curveto(643.7599283,133.13214011)(643.82992823,133.23714001)(643.8999292,133.34714111)
\curveto(644.11992794,133.6471396)(644.33492773,133.9471393)(644.5449292,134.24714111)
\lineto(645.1749292,135.14714111)
\curveto(645.24492682,135.23713801)(645.30992675,135.32713792)(645.3699292,135.41714111)
\curveto(645.43992662,135.50713774)(645.50492656,135.60213764)(645.5649292,135.70214111)
\curveto(645.61492645,135.77213747)(645.6649264,135.83713741)(645.7149292,135.89714111)
\curveto(645.7649263,135.96713728)(645.79992626,136.05713719)(645.8199292,136.16714111)
\curveto(645.83992622,136.21713703)(645.83492623,136.26713698)(645.8049292,136.31714111)
\curveto(645.78492628,136.36713688)(645.7649263,136.40713684)(645.7449292,136.43714111)
\curveto(645.69492637,136.52713672)(645.63992642,136.61213663)(645.5799292,136.69214111)
\lineto(645.3999292,136.93214111)
\curveto(645.16992689,137.25213599)(644.93492713,137.57213567)(644.6949292,137.89214111)
\lineto(644.0049292,138.85214111)
\curveto(643.92492814,138.96213428)(643.84492822,139.06213418)(643.7649292,139.15214111)
\curveto(643.69492837,139.242134)(643.62492844,139.3421339)(643.5549292,139.45214111)
\curveto(643.53492853,139.48213376)(643.51492855,139.52213372)(643.4949292,139.57214111)
\curveto(643.47492859,139.63213361)(643.47492859,139.68213356)(643.4949292,139.72214111)
\curveto(643.51492855,139.77213347)(643.54492852,139.80213344)(643.5849292,139.81214111)
\curveto(643.62492844,139.83213341)(643.66992839,139.8471334)(643.7199292,139.85714111)
\curveto(643.73992832,139.86713338)(643.75492831,139.86713338)(643.7649292,139.85714111)
\curveto(643.77492829,139.85713339)(643.78492828,139.86213338)(643.7949292,139.87214111)
}
}
{
\newrgbcolor{curcolor}{0 0 0}
\pscustom[linestyle=none,fillstyle=solid,fillcolor=curcolor]
{
\newpath
\moveto(651.82860107,141.37214111)
\curveto(651.74859995,141.43213181)(651.7036,141.53713171)(651.69360107,141.68714111)
\lineto(651.69360107,142.15214111)
\lineto(651.69360107,142.40714111)
\curveto(651.69360001,142.49713075)(651.70859999,142.57213067)(651.73860107,142.63214111)
\curveto(651.77859992,142.71213053)(651.85859984,142.77213047)(651.97860107,142.81214111)
\curveto(651.9985997,142.82213042)(652.01859968,142.82213042)(652.03860107,142.81214111)
\curveto(652.06859963,142.81213043)(652.09359961,142.81713043)(652.11360107,142.82714111)
\curveto(652.28359942,142.82713042)(652.44359926,142.82213042)(652.59360107,142.81214111)
\curveto(652.74359896,142.80213044)(652.84359886,142.7421305)(652.89360107,142.63214111)
\curveto(652.92359878,142.57213067)(652.93859876,142.49713075)(652.93860107,142.40714111)
\lineto(652.93860107,142.15214111)
\curveto(652.93859876,141.97213127)(652.93359877,141.80213144)(652.92360107,141.64214111)
\curveto(652.92359878,141.48213176)(652.85859884,141.37713187)(652.72860107,141.32714111)
\curveto(652.67859902,141.30713194)(652.62359908,141.29713195)(652.56360107,141.29714111)
\lineto(652.39860107,141.29714111)
\lineto(652.08360107,141.29714111)
\curveto(651.98359972,141.29713195)(651.8985998,141.32213192)(651.82860107,141.37214111)
\moveto(652.93860107,132.86714111)
\lineto(652.93860107,132.55214111)
\curveto(652.94859875,132.45214079)(652.92859877,132.37214087)(652.87860107,132.31214111)
\curveto(652.84859885,132.25214099)(652.8035989,132.21214103)(652.74360107,132.19214111)
\curveto(652.68359902,132.18214106)(652.61359909,132.16714108)(652.53360107,132.14714111)
\lineto(652.30860107,132.14714111)
\curveto(652.17859952,132.1471411)(652.06359964,132.15214109)(651.96360107,132.16214111)
\curveto(651.87359983,132.18214106)(651.8035999,132.23214101)(651.75360107,132.31214111)
\curveto(651.71359999,132.37214087)(651.69360001,132.4471408)(651.69360107,132.53714111)
\lineto(651.69360107,132.82214111)
\lineto(651.69360107,139.16714111)
\lineto(651.69360107,139.48214111)
\curveto(651.69360001,139.59213365)(651.71859998,139.67713357)(651.76860107,139.73714111)
\curveto(651.7985999,139.78713346)(651.83859986,139.81713343)(651.88860107,139.82714111)
\curveto(651.93859976,139.83713341)(651.99359971,139.85213339)(652.05360107,139.87214111)
\curveto(652.07359963,139.87213337)(652.09359961,139.86713338)(652.11360107,139.85714111)
\curveto(652.14359956,139.85713339)(652.16859953,139.86213338)(652.18860107,139.87214111)
\curveto(652.31859938,139.87213337)(652.44859925,139.86713338)(652.57860107,139.85714111)
\curveto(652.71859898,139.85713339)(652.81359889,139.81713343)(652.86360107,139.73714111)
\curveto(652.91359879,139.67713357)(652.93859876,139.59713365)(652.93860107,139.49714111)
\lineto(652.93860107,139.21214111)
\lineto(652.93860107,132.86714111)
}
}
{
\newrgbcolor{curcolor}{0 0 0}
\pscustom[linestyle=none,fillstyle=solid,fillcolor=curcolor]
{
\newpath
\moveto(655.45344482,142.82714111)
\curveto(655.58344321,142.82713042)(655.71844307,142.82713042)(655.85844482,142.82714111)
\curveto(656.00844278,142.82713042)(656.11844267,142.79213045)(656.18844482,142.72214111)
\curveto(656.23844255,142.65213059)(656.26344253,142.55713069)(656.26344482,142.43714111)
\curveto(656.27344252,142.32713092)(656.27844251,142.21213103)(656.27844482,142.09214111)
\lineto(656.27844482,140.75714111)
\lineto(656.27844482,134.68214111)
\lineto(656.27844482,133.00214111)
\lineto(656.27844482,132.61214111)
\curveto(656.27844251,132.47214077)(656.25344254,132.36214088)(656.20344482,132.28214111)
\curveto(656.17344262,132.23214101)(656.12844266,132.20214104)(656.06844482,132.19214111)
\curveto(656.01844277,132.18214106)(655.95344284,132.16714108)(655.87344482,132.14714111)
\lineto(655.66344482,132.14714111)
\lineto(655.34844482,132.14714111)
\curveto(655.24844354,132.15714109)(655.17344362,132.19214105)(655.12344482,132.25214111)
\curveto(655.07344372,132.33214091)(655.04344375,132.43214081)(655.03344482,132.55214111)
\lineto(655.03344482,132.92714111)
\lineto(655.03344482,134.30714111)
\lineto(655.03344482,140.54714111)
\lineto(655.03344482,142.01714111)
\curveto(655.03344376,142.12713112)(655.02844376,142.242131)(655.01844482,142.36214111)
\curveto(655.01844377,142.49213075)(655.04344375,142.59213065)(655.09344482,142.66214111)
\curveto(655.13344366,142.72213052)(655.20844358,142.77213047)(655.31844482,142.81214111)
\curveto(655.33844345,142.82213042)(655.35844343,142.82213042)(655.37844482,142.81214111)
\curveto(655.40844338,142.81213043)(655.43344336,142.81713043)(655.45344482,142.82714111)
}
}
{
\newrgbcolor{curcolor}{0 0 0}
\pscustom[linestyle=none,fillstyle=solid,fillcolor=curcolor]
{
\newpath
\moveto(658.50828857,141.37214111)
\curveto(658.42828745,141.43213181)(658.3832875,141.53713171)(658.37328857,141.68714111)
\lineto(658.37328857,142.15214111)
\lineto(658.37328857,142.40714111)
\curveto(658.37328751,142.49713075)(658.38828749,142.57213067)(658.41828857,142.63214111)
\curveto(658.45828742,142.71213053)(658.53828734,142.77213047)(658.65828857,142.81214111)
\curveto(658.6782872,142.82213042)(658.69828718,142.82213042)(658.71828857,142.81214111)
\curveto(658.74828713,142.81213043)(658.77328711,142.81713043)(658.79328857,142.82714111)
\curveto(658.96328692,142.82713042)(659.12328676,142.82213042)(659.27328857,142.81214111)
\curveto(659.42328646,142.80213044)(659.52328636,142.7421305)(659.57328857,142.63214111)
\curveto(659.60328628,142.57213067)(659.61828626,142.49713075)(659.61828857,142.40714111)
\lineto(659.61828857,142.15214111)
\curveto(659.61828626,141.97213127)(659.61328627,141.80213144)(659.60328857,141.64214111)
\curveto(659.60328628,141.48213176)(659.53828634,141.37713187)(659.40828857,141.32714111)
\curveto(659.35828652,141.30713194)(659.30328658,141.29713195)(659.24328857,141.29714111)
\lineto(659.07828857,141.29714111)
\lineto(658.76328857,141.29714111)
\curveto(658.66328722,141.29713195)(658.5782873,141.32213192)(658.50828857,141.37214111)
\moveto(659.61828857,132.86714111)
\lineto(659.61828857,132.55214111)
\curveto(659.62828625,132.45214079)(659.60828627,132.37214087)(659.55828857,132.31214111)
\curveto(659.52828635,132.25214099)(659.4832864,132.21214103)(659.42328857,132.19214111)
\curveto(659.36328652,132.18214106)(659.29328659,132.16714108)(659.21328857,132.14714111)
\lineto(658.98828857,132.14714111)
\curveto(658.85828702,132.1471411)(658.74328714,132.15214109)(658.64328857,132.16214111)
\curveto(658.55328733,132.18214106)(658.4832874,132.23214101)(658.43328857,132.31214111)
\curveto(658.39328749,132.37214087)(658.37328751,132.4471408)(658.37328857,132.53714111)
\lineto(658.37328857,132.82214111)
\lineto(658.37328857,139.16714111)
\lineto(658.37328857,139.48214111)
\curveto(658.37328751,139.59213365)(658.39828748,139.67713357)(658.44828857,139.73714111)
\curveto(658.4782874,139.78713346)(658.51828736,139.81713343)(658.56828857,139.82714111)
\curveto(658.61828726,139.83713341)(658.67328721,139.85213339)(658.73328857,139.87214111)
\curveto(658.75328713,139.87213337)(658.77328711,139.86713338)(658.79328857,139.85714111)
\curveto(658.82328706,139.85713339)(658.84828703,139.86213338)(658.86828857,139.87214111)
\curveto(658.99828688,139.87213337)(659.12828675,139.86713338)(659.25828857,139.85714111)
\curveto(659.39828648,139.85713339)(659.49328639,139.81713343)(659.54328857,139.73714111)
\curveto(659.59328629,139.67713357)(659.61828626,139.59713365)(659.61828857,139.49714111)
\lineto(659.61828857,139.21214111)
\lineto(659.61828857,132.86714111)
}
}
{
\newrgbcolor{curcolor}{0 0 0}
\pscustom[linestyle=none,fillstyle=solid,fillcolor=curcolor]
{
\newpath
\moveto(668.44813232,132.70214111)
\curveto(668.47812449,132.5421407)(668.46312451,132.40714084)(668.40313232,132.29714111)
\curveto(668.34312463,132.19714105)(668.26312471,132.12214112)(668.16313232,132.07214111)
\curveto(668.11312486,132.05214119)(668.05812491,132.0421412)(667.99813232,132.04214111)
\curveto(667.94812502,132.0421412)(667.89312508,132.03214121)(667.83313232,132.01214111)
\curveto(667.61312536,131.96214128)(667.39312558,131.97714127)(667.17313232,132.05714111)
\curveto(666.96312601,132.12714112)(666.81812615,132.21714103)(666.73813232,132.32714111)
\curveto(666.68812628,132.39714085)(666.64312633,132.47714077)(666.60313232,132.56714111)
\curveto(666.56312641,132.66714058)(666.51312646,132.7471405)(666.45313232,132.80714111)
\curveto(666.43312654,132.82714042)(666.40812656,132.8471404)(666.37813232,132.86714111)
\curveto(666.35812661,132.88714036)(666.32812664,132.89214035)(666.28813232,132.88214111)
\curveto(666.17812679,132.85214039)(666.0731269,132.79714045)(665.97313232,132.71714111)
\curveto(665.88312709,132.63714061)(665.79312718,132.56714068)(665.70313232,132.50714111)
\curveto(665.5731274,132.42714082)(665.43312754,132.35214089)(665.28313232,132.28214111)
\curveto(665.13312784,132.22214102)(664.973128,132.16714108)(664.80313232,132.11714111)
\curveto(664.70312827,132.08714116)(664.59312838,132.06714118)(664.47313232,132.05714111)
\curveto(664.36312861,132.0471412)(664.25312872,132.03214121)(664.14313232,132.01214111)
\curveto(664.09312888,132.00214124)(664.04812892,131.99714125)(664.00813232,131.99714111)
\lineto(663.90313232,131.99714111)
\curveto(663.79312918,131.97714127)(663.68812928,131.97714127)(663.58813232,131.99714111)
\lineto(663.45313232,131.99714111)
\curveto(663.40312957,132.00714124)(663.35312962,132.01214123)(663.30313232,132.01214111)
\curveto(663.25312972,132.01214123)(663.20812976,132.02214122)(663.16813232,132.04214111)
\curveto(663.12812984,132.05214119)(663.09312988,132.05714119)(663.06313232,132.05714111)
\curveto(663.04312993,132.0471412)(663.01812995,132.0471412)(662.98813232,132.05714111)
\lineto(662.74813232,132.11714111)
\curveto(662.6681303,132.12714112)(662.59313038,132.1471411)(662.52313232,132.17714111)
\curveto(662.22313075,132.30714094)(661.97813099,132.45214079)(661.78813232,132.61214111)
\curveto(661.60813136,132.78214046)(661.45813151,133.01714023)(661.33813232,133.31714111)
\curveto(661.24813172,133.53713971)(661.20313177,133.80213944)(661.20313232,134.11214111)
\lineto(661.20313232,134.42714111)
\curveto(661.21313176,134.47713877)(661.21813175,134.52713872)(661.21813232,134.57714111)
\lineto(661.24813232,134.75714111)
\lineto(661.36813232,135.08714111)
\curveto(661.40813156,135.19713805)(661.45813151,135.29713795)(661.51813232,135.38714111)
\curveto(661.69813127,135.67713757)(661.94313103,135.89213735)(662.25313232,136.03214111)
\curveto(662.56313041,136.17213707)(662.90313007,136.29713695)(663.27313232,136.40714111)
\curveto(663.41312956,136.4471368)(663.55812941,136.47713677)(663.70813232,136.49714111)
\curveto(663.85812911,136.51713673)(664.00812896,136.5421367)(664.15813232,136.57214111)
\curveto(664.22812874,136.59213665)(664.29312868,136.60213664)(664.35313232,136.60214111)
\curveto(664.42312855,136.60213664)(664.49812847,136.61213663)(664.57813232,136.63214111)
\curveto(664.64812832,136.65213659)(664.71812825,136.66213658)(664.78813232,136.66214111)
\curveto(664.85812811,136.67213657)(664.93312804,136.68713656)(665.01313232,136.70714111)
\curveto(665.26312771,136.76713648)(665.49812747,136.81713643)(665.71813232,136.85714111)
\curveto(665.93812703,136.90713634)(666.11312686,137.02213622)(666.24313232,137.20214111)
\curveto(666.30312667,137.28213596)(666.35312662,137.38213586)(666.39313232,137.50214111)
\curveto(666.43312654,137.63213561)(666.43312654,137.77213547)(666.39313232,137.92214111)
\curveto(666.33312664,138.16213508)(666.24312673,138.35213489)(666.12313232,138.49214111)
\curveto(666.01312696,138.63213461)(665.85312712,138.7421345)(665.64313232,138.82214111)
\curveto(665.52312745,138.87213437)(665.37812759,138.90713434)(665.20813232,138.92714111)
\curveto(665.04812792,138.9471343)(664.87812809,138.95713429)(664.69813232,138.95714111)
\curveto(664.51812845,138.95713429)(664.34312863,138.9471343)(664.17313232,138.92714111)
\curveto(664.00312897,138.90713434)(663.85812911,138.87713437)(663.73813232,138.83714111)
\curveto(663.5681294,138.77713447)(663.40312957,138.69213455)(663.24313232,138.58214111)
\curveto(663.16312981,138.52213472)(663.08812988,138.4421348)(663.01813232,138.34214111)
\curveto(662.95813001,138.25213499)(662.90313007,138.15213509)(662.85313232,138.04214111)
\curveto(662.82313015,137.96213528)(662.79313018,137.87713537)(662.76313232,137.78714111)
\curveto(662.74313023,137.69713555)(662.69813027,137.62713562)(662.62813232,137.57714111)
\curveto(662.58813038,137.5471357)(662.51813045,137.52213572)(662.41813232,137.50214111)
\curveto(662.32813064,137.49213575)(662.23313074,137.48713576)(662.13313232,137.48714111)
\curveto(662.03313094,137.48713576)(661.93313104,137.49213575)(661.83313232,137.50214111)
\curveto(661.74313123,137.52213572)(661.67813129,137.5471357)(661.63813232,137.57714111)
\curveto(661.59813137,137.60713564)(661.5681314,137.65713559)(661.54813232,137.72714111)
\curveto(661.52813144,137.79713545)(661.52813144,137.87213537)(661.54813232,137.95214111)
\curveto(661.57813139,138.08213516)(661.60813136,138.20213504)(661.63813232,138.31214111)
\curveto(661.67813129,138.43213481)(661.72313125,138.5471347)(661.77313232,138.65714111)
\curveto(661.96313101,139.00713424)(662.20313077,139.27713397)(662.49313232,139.46714111)
\curveto(662.78313019,139.66713358)(663.14312983,139.82713342)(663.57313232,139.94714111)
\curveto(663.6731293,139.96713328)(663.7731292,139.98213326)(663.87313232,139.99214111)
\curveto(663.98312899,140.00213324)(664.09312888,140.01713323)(664.20313232,140.03714111)
\curveto(664.24312873,140.0471332)(664.30812866,140.0471332)(664.39813232,140.03714111)
\curveto(664.48812848,140.03713321)(664.54312843,140.0471332)(664.56313232,140.06714111)
\curveto(665.26312771,140.07713317)(665.8731271,139.99713325)(666.39313232,139.82714111)
\curveto(666.91312606,139.65713359)(667.27812569,139.33213391)(667.48813232,138.85214111)
\curveto(667.57812539,138.65213459)(667.62812534,138.41713483)(667.63813232,138.14714111)
\curveto(667.65812531,137.88713536)(667.6681253,137.61213563)(667.66813232,137.32214111)
\lineto(667.66813232,134.00714111)
\curveto(667.6681253,133.86713938)(667.6731253,133.73213951)(667.68313232,133.60214111)
\curveto(667.69312528,133.47213977)(667.72312525,133.36713988)(667.77313232,133.28714111)
\curveto(667.82312515,133.21714003)(667.88812508,133.16714008)(667.96813232,133.13714111)
\curveto(668.05812491,133.09714015)(668.14312483,133.06714018)(668.22313232,133.04714111)
\curveto(668.30312467,133.03714021)(668.36312461,132.99214025)(668.40313232,132.91214111)
\curveto(668.42312455,132.88214036)(668.43312454,132.85214039)(668.43313232,132.82214111)
\curveto(668.43312454,132.79214045)(668.43812453,132.75214049)(668.44813232,132.70214111)
\moveto(666.30313232,134.36714111)
\curveto(666.36312661,134.50713874)(666.39312658,134.66713858)(666.39313232,134.84714111)
\curveto(666.40312657,135.03713821)(666.40812656,135.23213801)(666.40813232,135.43214111)
\curveto(666.40812656,135.5421377)(666.40312657,135.6421376)(666.39313232,135.73214111)
\curveto(666.38312659,135.82213742)(666.34312663,135.89213735)(666.27313232,135.94214111)
\curveto(666.24312673,135.96213728)(666.1731268,135.97213727)(666.06313232,135.97214111)
\curveto(666.04312693,135.95213729)(666.00812696,135.9421373)(665.95813232,135.94214111)
\curveto(665.90812706,135.9421373)(665.86312711,135.93213731)(665.82313232,135.91214111)
\curveto(665.74312723,135.89213735)(665.65312732,135.87213737)(665.55313232,135.85214111)
\lineto(665.25313232,135.79214111)
\curveto(665.22312775,135.79213745)(665.18812778,135.78713746)(665.14813232,135.77714111)
\lineto(665.04313232,135.77714111)
\curveto(664.89312808,135.73713751)(664.72812824,135.71213753)(664.54813232,135.70214111)
\curveto(664.37812859,135.70213754)(664.21812875,135.68213756)(664.06813232,135.64214111)
\curveto(663.98812898,135.62213762)(663.91312906,135.60213764)(663.84313232,135.58214111)
\curveto(663.78312919,135.57213767)(663.71312926,135.55713769)(663.63313232,135.53714111)
\curveto(663.4731295,135.48713776)(663.32312965,135.42213782)(663.18313232,135.34214111)
\curveto(663.04312993,135.27213797)(662.92313005,135.18213806)(662.82313232,135.07214111)
\curveto(662.72313025,134.96213828)(662.64813032,134.82713842)(662.59813232,134.66714111)
\curveto(662.54813042,134.51713873)(662.52813044,134.33213891)(662.53813232,134.11214111)
\curveto(662.53813043,134.01213923)(662.55313042,133.91713933)(662.58313232,133.82714111)
\curveto(662.62313035,133.7471395)(662.6681303,133.67213957)(662.71813232,133.60214111)
\curveto(662.79813017,133.49213975)(662.90313007,133.39713985)(663.03313232,133.31714111)
\curveto(663.16312981,133.24714)(663.30312967,133.18714006)(663.45313232,133.13714111)
\curveto(663.50312947,133.12714012)(663.55312942,133.12214012)(663.60313232,133.12214111)
\curveto(663.65312932,133.12214012)(663.70312927,133.11714013)(663.75313232,133.10714111)
\curveto(663.82312915,133.08714016)(663.90812906,133.07214017)(664.00813232,133.06214111)
\curveto(664.11812885,133.06214018)(664.20812876,133.07214017)(664.27813232,133.09214111)
\curveto(664.33812863,133.11214013)(664.39812857,133.11714013)(664.45813232,133.10714111)
\curveto(664.51812845,133.10714014)(664.57812839,133.11714013)(664.63813232,133.13714111)
\curveto(664.71812825,133.15714009)(664.79312818,133.17214007)(664.86313232,133.18214111)
\curveto(664.94312803,133.19214005)(665.01812795,133.21214003)(665.08813232,133.24214111)
\curveto(665.37812759,133.36213988)(665.62312735,133.50713974)(665.82313232,133.67714111)
\curveto(666.03312694,133.8471394)(666.19312678,134.07713917)(666.30313232,134.36714111)
}
}
{
\newrgbcolor{curcolor}{0 0 0}
\pscustom[linestyle=none,fillstyle=solid,fillcolor=curcolor]
{
\newpath
\moveto(673.26477295,140.05214111)
\curveto(673.49476816,140.05213319)(673.62476803,139.99213325)(673.65477295,139.87214111)
\curveto(673.68476797,139.76213348)(673.69976795,139.59713365)(673.69977295,139.37714111)
\lineto(673.69977295,139.09214111)
\curveto(673.69976795,139.00213424)(673.67476798,138.92713432)(673.62477295,138.86714111)
\curveto(673.56476809,138.78713446)(673.47976817,138.7421345)(673.36977295,138.73214111)
\curveto(673.25976839,138.73213451)(673.1497685,138.71713453)(673.03977295,138.68714111)
\curveto(672.89976875,138.65713459)(672.76476889,138.62713462)(672.63477295,138.59714111)
\curveto(672.51476914,138.56713468)(672.39976925,138.52713472)(672.28977295,138.47714111)
\curveto(671.99976965,138.3471349)(671.76476989,138.16713508)(671.58477295,137.93714111)
\curveto(671.40477025,137.71713553)(671.2497704,137.46213578)(671.11977295,137.17214111)
\curveto(671.07977057,137.06213618)(671.0497706,136.9471363)(671.02977295,136.82714111)
\curveto(671.00977064,136.71713653)(670.98477067,136.60213664)(670.95477295,136.48214111)
\curveto(670.94477071,136.43213681)(670.93977071,136.38213686)(670.93977295,136.33214111)
\curveto(670.9497707,136.28213696)(670.9497707,136.23213701)(670.93977295,136.18214111)
\curveto(670.90977074,136.06213718)(670.89477076,135.92213732)(670.89477295,135.76214111)
\curveto(670.90477075,135.61213763)(670.90977074,135.46713778)(670.90977295,135.32714111)
\lineto(670.90977295,133.48214111)
\lineto(670.90977295,133.13714111)
\curveto(670.90977074,133.01714023)(670.90477075,132.90214034)(670.89477295,132.79214111)
\curveto(670.88477077,132.68214056)(670.87977077,132.58714066)(670.87977295,132.50714111)
\curveto(670.88977076,132.42714082)(670.86977078,132.35714089)(670.81977295,132.29714111)
\curveto(670.76977088,132.22714102)(670.68977096,132.18714106)(670.57977295,132.17714111)
\curveto(670.47977117,132.16714108)(670.36977128,132.16214108)(670.24977295,132.16214111)
\lineto(669.97977295,132.16214111)
\curveto(669.92977172,132.18214106)(669.87977177,132.19714105)(669.82977295,132.20714111)
\curveto(669.78977186,132.22714102)(669.75977189,132.25214099)(669.73977295,132.28214111)
\curveto(669.68977196,132.35214089)(669.65977199,132.43714081)(669.64977295,132.53714111)
\lineto(669.64977295,132.86714111)
\lineto(669.64977295,134.02214111)
\lineto(669.64977295,138.17714111)
\lineto(669.64977295,139.21214111)
\lineto(669.64977295,139.51214111)
\curveto(669.65977199,139.61213363)(669.68977196,139.69713355)(669.73977295,139.76714111)
\curveto(669.76977188,139.80713344)(669.81977183,139.83713341)(669.88977295,139.85714111)
\curveto(669.96977168,139.87713337)(670.0547716,139.88713336)(670.14477295,139.88714111)
\curveto(670.23477142,139.89713335)(670.32477133,139.89713335)(670.41477295,139.88714111)
\curveto(670.50477115,139.87713337)(670.57477108,139.86213338)(670.62477295,139.84214111)
\curveto(670.70477095,139.81213343)(670.7547709,139.75213349)(670.77477295,139.66214111)
\curveto(670.80477085,139.58213366)(670.81977083,139.49213375)(670.81977295,139.39214111)
\lineto(670.81977295,139.09214111)
\curveto(670.81977083,138.99213425)(670.83977081,138.90213434)(670.87977295,138.82214111)
\curveto(670.88977076,138.80213444)(670.89977075,138.78713446)(670.90977295,138.77714111)
\lineto(670.95477295,138.73214111)
\curveto(671.06477059,138.73213451)(671.1547705,138.77713447)(671.22477295,138.86714111)
\curveto(671.29477036,138.96713428)(671.3547703,139.0471342)(671.40477295,139.10714111)
\lineto(671.49477295,139.19714111)
\curveto(671.58477007,139.30713394)(671.70976994,139.42213382)(671.86977295,139.54214111)
\curveto(672.02976962,139.66213358)(672.17976947,139.75213349)(672.31977295,139.81214111)
\curveto(672.40976924,139.86213338)(672.50476915,139.89713335)(672.60477295,139.91714111)
\curveto(672.70476895,139.9471333)(672.80976884,139.97713327)(672.91977295,140.00714111)
\curveto(672.97976867,140.01713323)(673.03976861,140.02213322)(673.09977295,140.02214111)
\curveto(673.15976849,140.03213321)(673.21476844,140.0421332)(673.26477295,140.05214111)
}
}
{
\newrgbcolor{curcolor}{0.7019608 0.7019608 0.7019608}
\pscustom[linestyle=none,fillstyle=solid,fillcolor=curcolor]
{
\newpath
\moveto(606.37554932,142.85717773)
\lineto(621.37554932,142.85717773)
\lineto(621.37554932,127.85717773)
\lineto(606.37554932,127.85717773)
\closepath
}
}
{
\newrgbcolor{curcolor}{0 0 0}
\pscustom[linestyle=none,fillstyle=solid,fillcolor=curcolor]
{
\newpath
\moveto(634.74875732,114.70637451)
\lineto(634.74875732,114.43637451)
\curveto(634.75874735,114.34636926)(634.75374736,114.26636934)(634.73375732,114.19637451)
\lineto(634.73375732,114.04637451)
\curveto(634.72374739,114.01636959)(634.71874739,113.98136963)(634.71875732,113.94137451)
\curveto(634.72874738,113.90136971)(634.72874738,113.87136974)(634.71875732,113.85137451)
\curveto(634.7087474,113.80136981)(634.70374741,113.74636986)(634.70375732,113.68637451)
\curveto(634.70374741,113.63636997)(634.69874741,113.58637002)(634.68875732,113.53637451)
\curveto(634.65874745,113.39637021)(634.63874747,113.24637036)(634.62875732,113.08637451)
\curveto(634.61874749,112.93637067)(634.58874752,112.79137082)(634.53875732,112.65137451)
\curveto(634.5087476,112.53137108)(634.47374764,112.4063712)(634.43375732,112.27637451)
\curveto(634.40374771,112.15637145)(634.36374775,112.03637157)(634.31375732,111.91637451)
\curveto(634.14374797,111.48637212)(633.92874818,111.09637251)(633.66875732,110.74637451)
\curveto(633.41874869,110.4063732)(633.10374901,110.11637349)(632.72375732,109.87637451)
\curveto(632.53374958,109.75637385)(632.32874978,109.65137396)(632.10875732,109.56137451)
\curveto(631.89875021,109.48137413)(631.66875044,109.40137421)(631.41875732,109.32137451)
\curveto(631.3087508,109.28137433)(631.18875092,109.25137436)(631.05875732,109.23137451)
\curveto(630.93875117,109.22137439)(630.81875129,109.20137441)(630.69875732,109.17137451)
\curveto(630.58875152,109.15137446)(630.47875163,109.14137447)(630.36875732,109.14137451)
\curveto(630.26875184,109.14137447)(630.16875194,109.13137448)(630.06875732,109.11137451)
\lineto(629.85875732,109.11137451)
\curveto(629.82875228,109.10137451)(629.79375232,109.09637451)(629.75375732,109.09637451)
\curveto(629.7137524,109.1063745)(629.67375244,109.1113745)(629.63375732,109.11137451)
\lineto(626.63375732,109.11137451)
\curveto(626.48375563,109.1113745)(626.34875576,109.11637449)(626.22875732,109.12637451)
\curveto(626.11875599,109.14637446)(626.04375607,109.2113744)(626.00375732,109.32137451)
\curveto(625.96375615,109.40137421)(625.94375617,109.51637409)(625.94375732,109.66637451)
\curveto(625.95375616,109.81637379)(625.95875615,109.95137366)(625.95875732,110.07137451)
\lineto(625.95875732,118.93637451)
\curveto(625.95875615,119.05636455)(625.95375616,119.18136443)(625.94375732,119.31137451)
\curveto(625.94375617,119.45136416)(625.96875614,119.56136405)(626.01875732,119.64137451)
\curveto(626.05875605,119.7113639)(626.13375598,119.75636385)(626.24375732,119.77637451)
\curveto(626.26375585,119.78636382)(626.28375583,119.78636382)(626.30375732,119.77637451)
\curveto(626.32375579,119.77636383)(626.34375577,119.78136383)(626.36375732,119.79137451)
\lineto(629.61875732,119.79137451)
\curveto(629.66875244,119.79136382)(629.7137524,119.79136382)(629.75375732,119.79137451)
\curveto(629.80375231,119.80136381)(629.84875226,119.80136381)(629.88875732,119.79137451)
\curveto(629.93875217,119.77136384)(629.98875212,119.76636384)(630.03875732,119.77637451)
\curveto(630.09875201,119.78636382)(630.15375196,119.78636382)(630.20375732,119.77637451)
\curveto(630.25375186,119.76636384)(630.3087518,119.76136385)(630.36875732,119.76137451)
\curveto(630.42875168,119.76136385)(630.48375163,119.75636385)(630.53375732,119.74637451)
\curveto(630.58375153,119.73636387)(630.62875148,119.73136388)(630.66875732,119.73137451)
\curveto(630.71875139,119.73136388)(630.76875134,119.72636388)(630.81875732,119.71637451)
\curveto(630.92875118,119.69636391)(631.03375108,119.67636393)(631.13375732,119.65637451)
\curveto(631.23375088,119.64636396)(631.33375078,119.62636398)(631.43375732,119.59637451)
\curveto(631.65375046,119.52636408)(631.86375025,119.45636415)(632.06375732,119.38637451)
\curveto(632.26374985,119.32636428)(632.44874966,119.24136437)(632.61875732,119.13137451)
\curveto(632.75874935,119.05136456)(632.88374923,118.97136464)(632.99375732,118.89137451)
\curveto(633.02374909,118.87136474)(633.05374906,118.84636476)(633.08375732,118.81637451)
\curveto(633.113749,118.79636481)(633.14374897,118.77636483)(633.17375732,118.75637451)
\curveto(633.23374888,118.7063649)(633.28874882,118.65636495)(633.33875732,118.60637451)
\curveto(633.38874872,118.55636505)(633.43874867,118.5063651)(633.48875732,118.45637451)
\curveto(633.53874857,118.4063652)(633.57874853,118.37136524)(633.60875732,118.35137451)
\curveto(633.64874846,118.29136532)(633.68874842,118.23636537)(633.72875732,118.18637451)
\curveto(633.77874833,118.13636547)(633.82374829,118.08136553)(633.86375732,118.02137451)
\curveto(633.9137482,117.96136565)(633.95374816,117.89636571)(633.98375732,117.82637451)
\curveto(634.02374809,117.76636584)(634.06874804,117.70136591)(634.11875732,117.63137451)
\curveto(634.13874797,117.59136602)(634.15374796,117.55636605)(634.16375732,117.52637451)
\curveto(634.17374794,117.49636611)(634.18874792,117.46136615)(634.20875732,117.42137451)
\curveto(634.24874786,117.34136627)(634.28374783,117.26136635)(634.31375732,117.18137451)
\curveto(634.34374777,117.1113665)(634.37874773,117.03636657)(634.41875732,116.95637451)
\curveto(634.45874765,116.84636676)(634.48874762,116.73136688)(634.50875732,116.61137451)
\curveto(634.53874757,116.50136711)(634.56874754,116.39136722)(634.59875732,116.28137451)
\curveto(634.61874749,116.22136739)(634.62874748,116.16136745)(634.62875732,116.10137451)
\curveto(634.62874748,116.05136756)(634.63874747,115.99636761)(634.65875732,115.93637451)
\curveto(634.7087474,115.75636785)(634.73374738,115.55636805)(634.73375732,115.33637451)
\curveto(634.74374737,115.12636848)(634.74874736,114.91636869)(634.74875732,114.70637451)
\moveto(633.32375732,113.92637451)
\curveto(633.34374877,114.02636958)(633.35374876,114.13136948)(633.35375732,114.24137451)
\lineto(633.35375732,114.58637451)
\lineto(633.35375732,114.81137451)
\curveto(633.36374875,114.89136872)(633.35874875,114.96636864)(633.33875732,115.03637451)
\curveto(633.33874877,115.06636854)(633.33374878,115.09636851)(633.32375732,115.12637451)
\lineto(633.32375732,115.23137451)
\curveto(633.30374881,115.34136827)(633.28874882,115.45136816)(633.27875732,115.56137451)
\curveto(633.27874883,115.67136794)(633.26374885,115.78136783)(633.23375732,115.89137451)
\curveto(633.2137489,115.97136764)(633.19374892,116.04636756)(633.17375732,116.11637451)
\curveto(633.16374895,116.19636741)(633.14874896,116.27636733)(633.12875732,116.35637451)
\curveto(633.01874909,116.71636689)(632.87874923,117.03136658)(632.70875732,117.30137451)
\curveto(632.42874968,117.75136586)(632.0137501,118.09136552)(631.46375732,118.32137451)
\curveto(631.37375074,118.37136524)(631.27875083,118.4063652)(631.17875732,118.42637451)
\curveto(631.07875103,118.45636515)(630.97375114,118.48636512)(630.86375732,118.51637451)
\curveto(630.75375136,118.54636506)(630.63875147,118.56136505)(630.51875732,118.56137451)
\curveto(630.4087517,118.57136504)(630.29875181,118.58636502)(630.18875732,118.60637451)
\lineto(629.87375732,118.60637451)
\curveto(629.84375227,118.61636499)(629.8087523,118.62136499)(629.76875732,118.62137451)
\lineto(629.64875732,118.62137451)
\lineto(627.81875732,118.62137451)
\curveto(627.79875431,118.611365)(627.77375434,118.606365)(627.74375732,118.60637451)
\curveto(627.7137544,118.61636499)(627.68875442,118.61636499)(627.66875732,118.60637451)
\lineto(627.51875732,118.54637451)
\curveto(627.47875463,118.52636508)(627.44875466,118.49636511)(627.42875732,118.45637451)
\curveto(627.4087547,118.41636519)(627.38875472,118.34636526)(627.36875732,118.24637451)
\lineto(627.36875732,118.12637451)
\curveto(627.35875475,118.08636552)(627.35375476,118.04136557)(627.35375732,117.99137451)
\lineto(627.35375732,117.85637451)
\lineto(627.35375732,111.04637451)
\lineto(627.35375732,110.89637451)
\curveto(627.35375476,110.85637275)(627.35875475,110.81637279)(627.36875732,110.77637451)
\lineto(627.36875732,110.65637451)
\curveto(627.38875472,110.55637305)(627.4087547,110.48637312)(627.42875732,110.44637451)
\curveto(627.5087546,110.32637328)(627.65875445,110.26637334)(627.87875732,110.26637451)
\curveto(628.09875401,110.27637333)(628.3087538,110.28137333)(628.50875732,110.28137451)
\lineto(629.37875732,110.28137451)
\curveto(629.44875266,110.28137333)(629.52375259,110.27637333)(629.60375732,110.26637451)
\curveto(629.68375243,110.26637334)(629.75375236,110.27637333)(629.81375732,110.29637451)
\lineto(629.97875732,110.29637451)
\curveto(630.02875208,110.3063733)(630.08375203,110.3063733)(630.14375732,110.29637451)
\curveto(630.20375191,110.29637331)(630.26375185,110.30137331)(630.32375732,110.31137451)
\curveto(630.38375173,110.33137328)(630.44375167,110.34137327)(630.50375732,110.34137451)
\curveto(630.56375155,110.35137326)(630.62875148,110.36637324)(630.69875732,110.38637451)
\curveto(630.8087513,110.41637319)(630.9137512,110.44637316)(631.01375732,110.47637451)
\curveto(631.12375099,110.5063731)(631.23375088,110.54637306)(631.34375732,110.59637451)
\curveto(631.7137504,110.75637285)(632.02875008,110.97137264)(632.28875732,111.24137451)
\curveto(632.55874955,111.52137209)(632.77874933,111.85137176)(632.94875732,112.23137451)
\curveto(632.99874911,112.34137127)(633.03874907,112.45637115)(633.06875732,112.57637451)
\lineto(633.18875732,112.96637451)
\curveto(633.21874889,113.07637053)(633.23874887,113.19137042)(633.24875732,113.31137451)
\curveto(633.26874884,113.44137017)(633.28874882,113.56637004)(633.30875732,113.68637451)
\curveto(633.31874879,113.73636987)(633.32374879,113.77636983)(633.32375732,113.80637451)
\lineto(633.32375732,113.92637451)
}
}
{
\newrgbcolor{curcolor}{0 0 0}
\pscustom[linestyle=none,fillstyle=solid,fillcolor=curcolor]
{
\newpath
\moveto(643.37563232,113.31137451)
\curveto(643.39562426,113.25137036)(643.40562425,113.15637045)(643.40563232,113.02637451)
\curveto(643.40562425,112.9063707)(643.40062426,112.82137079)(643.39063232,112.77137451)
\lineto(643.39063232,112.62137451)
\curveto(643.38062428,112.54137107)(643.37062429,112.46637114)(643.36063232,112.39637451)
\curveto(643.3606243,112.33637127)(643.3556243,112.26637134)(643.34563232,112.18637451)
\curveto(643.32562433,112.12637148)(643.31062435,112.06637154)(643.30063232,112.00637451)
\curveto(643.30062436,111.94637166)(643.29062437,111.88637172)(643.27063232,111.82637451)
\curveto(643.23062443,111.69637191)(643.19562446,111.56637204)(643.16563232,111.43637451)
\curveto(643.13562452,111.3063723)(643.09562456,111.18637242)(643.04563232,111.07637451)
\curveto(642.83562482,110.59637301)(642.5556251,110.19137342)(642.20563232,109.86137451)
\curveto(641.8556258,109.54137407)(641.42562623,109.29637431)(640.91563232,109.12637451)
\curveto(640.80562685,109.08637452)(640.68562697,109.05637455)(640.55563232,109.03637451)
\curveto(640.43562722,109.01637459)(640.31062735,108.99637461)(640.18063232,108.97637451)
\curveto(640.12062754,108.96637464)(640.0556276,108.96137465)(639.98563232,108.96137451)
\curveto(639.92562773,108.95137466)(639.86562779,108.94637466)(639.80563232,108.94637451)
\curveto(639.76562789,108.93637467)(639.70562795,108.93137468)(639.62563232,108.93137451)
\curveto(639.5556281,108.93137468)(639.50562815,108.93637467)(639.47563232,108.94637451)
\curveto(639.43562822,108.95637465)(639.39562826,108.96137465)(639.35563232,108.96137451)
\curveto(639.31562834,108.95137466)(639.28062838,108.95137466)(639.25063232,108.96137451)
\lineto(639.16063232,108.96137451)
\lineto(638.80063232,109.00637451)
\curveto(638.660629,109.04637456)(638.52562913,109.08637452)(638.39563232,109.12637451)
\curveto(638.26562939,109.16637444)(638.14062952,109.2113744)(638.02063232,109.26137451)
\curveto(637.57063009,109.46137415)(637.20063046,109.72137389)(636.91063232,110.04137451)
\curveto(636.62063104,110.36137325)(636.38063128,110.75137286)(636.19063232,111.21137451)
\curveto(636.14063152,111.3113723)(636.10063156,111.4113722)(636.07063232,111.51137451)
\curveto(636.05063161,111.611372)(636.03063163,111.71637189)(636.01063232,111.82637451)
\curveto(635.99063167,111.86637174)(635.98063168,111.89637171)(635.98063232,111.91637451)
\curveto(635.99063167,111.94637166)(635.99063167,111.98137163)(635.98063232,112.02137451)
\curveto(635.9606317,112.10137151)(635.94563171,112.18137143)(635.93563232,112.26137451)
\curveto(635.93563172,112.35137126)(635.92563173,112.43637117)(635.90563232,112.51637451)
\lineto(635.90563232,112.63637451)
\curveto(635.90563175,112.67637093)(635.90063176,112.72137089)(635.89063232,112.77137451)
\curveto(635.88063178,112.82137079)(635.87563178,112.9063707)(635.87563232,113.02637451)
\curveto(635.87563178,113.15637045)(635.88563177,113.25137036)(635.90563232,113.31137451)
\curveto(635.92563173,113.38137023)(635.93063173,113.45137016)(635.92063232,113.52137451)
\curveto(635.91063175,113.59137002)(635.91563174,113.66136995)(635.93563232,113.73137451)
\curveto(635.94563171,113.78136983)(635.95063171,113.82136979)(635.95063232,113.85137451)
\curveto(635.9606317,113.89136972)(635.97063169,113.93636967)(635.98063232,113.98637451)
\curveto(636.01063165,114.1063695)(636.03563162,114.22636938)(636.05563232,114.34637451)
\curveto(636.08563157,114.46636914)(636.12563153,114.58136903)(636.17563232,114.69137451)
\curveto(636.32563133,115.06136855)(636.50563115,115.39136822)(636.71563232,115.68137451)
\curveto(636.93563072,115.98136763)(637.20063046,116.23136738)(637.51063232,116.43137451)
\curveto(637.63063003,116.5113671)(637.7556299,116.57636703)(637.88563232,116.62637451)
\curveto(638.01562964,116.68636692)(638.15062951,116.74636686)(638.29063232,116.80637451)
\curveto(638.41062925,116.85636675)(638.54062912,116.88636672)(638.68063232,116.89637451)
\curveto(638.82062884,116.91636669)(638.9606287,116.94636666)(639.10063232,116.98637451)
\lineto(639.29563232,116.98637451)
\curveto(639.36562829,116.99636661)(639.43062823,117.0063666)(639.49063232,117.01637451)
\curveto(640.38062728,117.02636658)(641.12062654,116.84136677)(641.71063232,116.46137451)
\curveto(642.30062536,116.08136753)(642.72562493,115.58636802)(642.98563232,114.97637451)
\curveto(643.03562462,114.87636873)(643.07562458,114.77636883)(643.10563232,114.67637451)
\curveto(643.13562452,114.57636903)(643.17062449,114.47136914)(643.21063232,114.36137451)
\curveto(643.24062442,114.25136936)(643.26562439,114.13136948)(643.28563232,114.00137451)
\curveto(643.30562435,113.88136973)(643.33062433,113.75636985)(643.36063232,113.62637451)
\curveto(643.37062429,113.57637003)(643.37062429,113.52137009)(643.36063232,113.46137451)
\curveto(643.3606243,113.4113702)(643.36562429,113.36137025)(643.37563232,113.31137451)
\moveto(642.04063232,112.45637451)
\curveto(642.0606256,112.52637108)(642.06562559,112.606371)(642.05563232,112.69637451)
\lineto(642.05563232,112.95137451)
\curveto(642.0556256,113.34137027)(642.02062564,113.67136994)(641.95063232,113.94137451)
\curveto(641.92062574,114.02136959)(641.89562576,114.10136951)(641.87563232,114.18137451)
\curveto(641.8556258,114.26136935)(641.83062583,114.33636927)(641.80063232,114.40637451)
\curveto(641.52062614,115.05636855)(641.07562658,115.5063681)(640.46563232,115.75637451)
\curveto(640.39562726,115.78636782)(640.32062734,115.8063678)(640.24063232,115.81637451)
\lineto(640.00063232,115.87637451)
\curveto(639.92062774,115.89636771)(639.83562782,115.9063677)(639.74563232,115.90637451)
\lineto(639.47563232,115.90637451)
\lineto(639.20563232,115.86137451)
\curveto(639.10562855,115.84136777)(639.01062865,115.81636779)(638.92063232,115.78637451)
\curveto(638.84062882,115.76636784)(638.7606289,115.73636787)(638.68063232,115.69637451)
\curveto(638.61062905,115.67636793)(638.54562911,115.64636796)(638.48563232,115.60637451)
\curveto(638.42562923,115.56636804)(638.37062929,115.52636808)(638.32063232,115.48637451)
\curveto(638.08062958,115.31636829)(637.88562977,115.1113685)(637.73563232,114.87137451)
\curveto(637.58563007,114.63136898)(637.4556302,114.35136926)(637.34563232,114.03137451)
\curveto(637.31563034,113.93136968)(637.29563036,113.82636978)(637.28563232,113.71637451)
\curveto(637.27563038,113.61636999)(637.2606304,113.5113701)(637.24063232,113.40137451)
\curveto(637.23063043,113.36137025)(637.22563043,113.29637031)(637.22563232,113.20637451)
\curveto(637.21563044,113.17637043)(637.21063045,113.14137047)(637.21063232,113.10137451)
\curveto(637.22063044,113.06137055)(637.22563043,113.01637059)(637.22563232,112.96637451)
\lineto(637.22563232,112.66637451)
\curveto(637.22563043,112.56637104)(637.23563042,112.47637113)(637.25563232,112.39637451)
\lineto(637.28563232,112.21637451)
\curveto(637.30563035,112.11637149)(637.32063034,112.01637159)(637.33063232,111.91637451)
\curveto(637.35063031,111.82637178)(637.38063028,111.74137187)(637.42063232,111.66137451)
\curveto(637.52063014,111.42137219)(637.63563002,111.19637241)(637.76563232,110.98637451)
\curveto(637.90562975,110.77637283)(638.07562958,110.60137301)(638.27563232,110.46137451)
\curveto(638.32562933,110.43137318)(638.37062929,110.4063732)(638.41063232,110.38637451)
\curveto(638.45062921,110.36637324)(638.49562916,110.34137327)(638.54563232,110.31137451)
\curveto(638.62562903,110.26137335)(638.71062895,110.21637339)(638.80063232,110.17637451)
\curveto(638.90062876,110.14637346)(639.00562865,110.11637349)(639.11563232,110.08637451)
\curveto(639.16562849,110.06637354)(639.21062845,110.05637355)(639.25063232,110.05637451)
\curveto(639.30062836,110.06637354)(639.35062831,110.06637354)(639.40063232,110.05637451)
\curveto(639.43062823,110.04637356)(639.49062817,110.03637357)(639.58063232,110.02637451)
\curveto(639.68062798,110.01637359)(639.7556279,110.02137359)(639.80563232,110.04137451)
\curveto(639.84562781,110.05137356)(639.88562777,110.05137356)(639.92563232,110.04137451)
\curveto(639.96562769,110.04137357)(640.00562765,110.05137356)(640.04563232,110.07137451)
\curveto(640.12562753,110.09137352)(640.20562745,110.1063735)(640.28563232,110.11637451)
\curveto(640.36562729,110.13637347)(640.44062722,110.16137345)(640.51063232,110.19137451)
\curveto(640.85062681,110.33137328)(641.12562653,110.52637308)(641.33563232,110.77637451)
\curveto(641.54562611,111.02637258)(641.72062594,111.32137229)(641.86063232,111.66137451)
\curveto(641.91062575,111.78137183)(641.94062572,111.9063717)(641.95063232,112.03637451)
\curveto(641.97062569,112.17637143)(642.00062566,112.31637129)(642.04063232,112.45637451)
}
}
{
\newrgbcolor{curcolor}{0 0 0}
\pscustom[linestyle=none,fillstyle=solid,fillcolor=curcolor]
{
\newpath
\moveto(647.99891357,117.01637451)
\curveto(648.73890878,117.02636658)(649.35390817,116.91636669)(649.84391357,116.68637451)
\curveto(650.34390718,116.46636714)(650.73890678,116.13136748)(651.02891357,115.68137451)
\curveto(651.15890636,115.48136813)(651.26890625,115.23636837)(651.35891357,114.94637451)
\curveto(651.37890614,114.89636871)(651.39390613,114.83136878)(651.40391357,114.75137451)
\curveto(651.41390611,114.67136894)(651.40890611,114.60136901)(651.38891357,114.54137451)
\curveto(651.35890616,114.49136912)(651.30890621,114.44636916)(651.23891357,114.40637451)
\curveto(651.20890631,114.38636922)(651.17890634,114.37636923)(651.14891357,114.37637451)
\curveto(651.1189064,114.38636922)(651.08390644,114.38636922)(651.04391357,114.37637451)
\curveto(651.00390652,114.36636924)(650.96390656,114.36136925)(650.92391357,114.36137451)
\curveto(650.88390664,114.37136924)(650.84390668,114.37636923)(650.80391357,114.37637451)
\lineto(650.48891357,114.37637451)
\curveto(650.38890713,114.38636922)(650.30390722,114.41636919)(650.23391357,114.46637451)
\curveto(650.15390737,114.52636908)(650.09890742,114.611369)(650.06891357,114.72137451)
\curveto(650.03890748,114.83136878)(649.99890752,114.92636868)(649.94891357,115.00637451)
\curveto(649.79890772,115.26636834)(649.60390792,115.47136814)(649.36391357,115.62137451)
\curveto(649.28390824,115.67136794)(649.19890832,115.7113679)(649.10891357,115.74137451)
\curveto(649.0189085,115.78136783)(648.9239086,115.81636779)(648.82391357,115.84637451)
\curveto(648.68390884,115.88636772)(648.49890902,115.9063677)(648.26891357,115.90637451)
\curveto(648.03890948,115.91636769)(647.84890967,115.89636771)(647.69891357,115.84637451)
\curveto(647.62890989,115.82636778)(647.56390996,115.8113678)(647.50391357,115.80137451)
\curveto(647.44391008,115.79136782)(647.37891014,115.77636783)(647.30891357,115.75637451)
\curveto(647.04891047,115.64636796)(646.8189107,115.49636811)(646.61891357,115.30637451)
\curveto(646.4189111,115.11636849)(646.26391126,114.89136872)(646.15391357,114.63137451)
\curveto(646.11391141,114.54136907)(646.07891144,114.44636916)(646.04891357,114.34637451)
\curveto(646.0189115,114.25636935)(645.98891153,114.15636945)(645.95891357,114.04637451)
\lineto(645.86891357,113.64137451)
\curveto(645.85891166,113.59137002)(645.85391167,113.53637007)(645.85391357,113.47637451)
\curveto(645.86391166,113.41637019)(645.85891166,113.36137025)(645.83891357,113.31137451)
\lineto(645.83891357,113.19137451)
\curveto(645.82891169,113.15137046)(645.8189117,113.08637052)(645.80891357,112.99637451)
\curveto(645.80891171,112.9063707)(645.8189117,112.84137077)(645.83891357,112.80137451)
\curveto(645.84891167,112.75137086)(645.84891167,112.70137091)(645.83891357,112.65137451)
\curveto(645.82891169,112.60137101)(645.82891169,112.55137106)(645.83891357,112.50137451)
\curveto(645.84891167,112.46137115)(645.85391167,112.39137122)(645.85391357,112.29137451)
\curveto(645.87391165,112.2113714)(645.88891163,112.12637148)(645.89891357,112.03637451)
\curveto(645.9189116,111.94637166)(645.93891158,111.86137175)(645.95891357,111.78137451)
\curveto(646.06891145,111.46137215)(646.19391133,111.18137243)(646.33391357,110.94137451)
\curveto(646.48391104,110.7113729)(646.68891083,110.5113731)(646.94891357,110.34137451)
\curveto(647.03891048,110.29137332)(647.12891039,110.24637336)(647.21891357,110.20637451)
\curveto(647.3189102,110.16637344)(647.4239101,110.12637348)(647.53391357,110.08637451)
\curveto(647.58390994,110.07637353)(647.6239099,110.07137354)(647.65391357,110.07137451)
\curveto(647.68390984,110.07137354)(647.7239098,110.06637354)(647.77391357,110.05637451)
\curveto(647.80390972,110.04637356)(647.85390967,110.04137357)(647.92391357,110.04137451)
\lineto(648.08891357,110.04137451)
\curveto(648.08890943,110.03137358)(648.10890941,110.02637358)(648.14891357,110.02637451)
\curveto(648.16890935,110.03637357)(648.19390933,110.03637357)(648.22391357,110.02637451)
\curveto(648.25390927,110.02637358)(648.28390924,110.03137358)(648.31391357,110.04137451)
\curveto(648.38390914,110.06137355)(648.44890907,110.06637354)(648.50891357,110.05637451)
\curveto(648.57890894,110.05637355)(648.64890887,110.06637354)(648.71891357,110.08637451)
\curveto(648.97890854,110.16637344)(649.20390832,110.26637334)(649.39391357,110.38637451)
\curveto(649.58390794,110.51637309)(649.74390778,110.68137293)(649.87391357,110.88137451)
\curveto(649.9239076,110.96137265)(649.96890755,111.04637256)(650.00891357,111.13637451)
\lineto(650.12891357,111.40637451)
\curveto(650.14890737,111.48637212)(650.16890735,111.56137205)(650.18891357,111.63137451)
\curveto(650.2189073,111.7113719)(650.26890725,111.77637183)(650.33891357,111.82637451)
\curveto(650.36890715,111.85637175)(650.42890709,111.87637173)(650.51891357,111.88637451)
\curveto(650.60890691,111.9063717)(650.70390682,111.91637169)(650.80391357,111.91637451)
\curveto(650.91390661,111.92637168)(651.01390651,111.92637168)(651.10391357,111.91637451)
\curveto(651.20390632,111.9063717)(651.27390625,111.88637172)(651.31391357,111.85637451)
\curveto(651.37390615,111.81637179)(651.40890611,111.75637185)(651.41891357,111.67637451)
\curveto(651.43890608,111.59637201)(651.43890608,111.5113721)(651.41891357,111.42137451)
\curveto(651.36890615,111.27137234)(651.3189062,111.12637248)(651.26891357,110.98637451)
\curveto(651.22890629,110.85637275)(651.17390635,110.72637288)(651.10391357,110.59637451)
\curveto(650.95390657,110.29637331)(650.76390676,110.03137358)(650.53391357,109.80137451)
\curveto(650.31390721,109.57137404)(650.04390748,109.38637422)(649.72391357,109.24637451)
\curveto(649.64390788,109.2063744)(649.55890796,109.17137444)(649.46891357,109.14137451)
\curveto(649.37890814,109.12137449)(649.28390824,109.09637451)(649.18391357,109.06637451)
\curveto(649.07390845,109.02637458)(648.96390856,109.0063746)(648.85391357,109.00637451)
\curveto(648.74390878,108.99637461)(648.63390889,108.98137463)(648.52391357,108.96137451)
\curveto(648.48390904,108.94137467)(648.44390908,108.93637467)(648.40391357,108.94637451)
\curveto(648.36390916,108.95637465)(648.3239092,108.95637465)(648.28391357,108.94637451)
\lineto(648.14891357,108.94637451)
\lineto(647.90891357,108.94637451)
\curveto(647.83890968,108.93637467)(647.77390975,108.94137467)(647.71391357,108.96137451)
\lineto(647.63891357,108.96137451)
\lineto(647.27891357,109.00637451)
\curveto(647.14891037,109.04637456)(647.0239105,109.08137453)(646.90391357,109.11137451)
\curveto(646.78391074,109.14137447)(646.66891085,109.18137443)(646.55891357,109.23137451)
\curveto(646.19891132,109.39137422)(645.89891162,109.58137403)(645.65891357,109.80137451)
\curveto(645.42891209,110.02137359)(645.21391231,110.29137332)(645.01391357,110.61137451)
\curveto(644.96391256,110.69137292)(644.9189126,110.78137283)(644.87891357,110.88137451)
\lineto(644.75891357,111.18137451)
\curveto(644.70891281,111.29137232)(644.67391285,111.4063722)(644.65391357,111.52637451)
\curveto(644.63391289,111.64637196)(644.60891291,111.76637184)(644.57891357,111.88637451)
\curveto(644.56891295,111.92637168)(644.56391296,111.96637164)(644.56391357,112.00637451)
\curveto(644.56391296,112.04637156)(644.55891296,112.08637152)(644.54891357,112.12637451)
\curveto(644.52891299,112.18637142)(644.518913,112.25137136)(644.51891357,112.32137451)
\curveto(644.52891299,112.39137122)(644.523913,112.45637115)(644.50391357,112.51637451)
\lineto(644.50391357,112.66637451)
\curveto(644.49391303,112.71637089)(644.48891303,112.78637082)(644.48891357,112.87637451)
\curveto(644.48891303,112.96637064)(644.49391303,113.03637057)(644.50391357,113.08637451)
\curveto(644.51391301,113.13637047)(644.51391301,113.18137043)(644.50391357,113.22137451)
\curveto(644.50391302,113.26137035)(644.50891301,113.30137031)(644.51891357,113.34137451)
\curveto(644.53891298,113.4113702)(644.54391298,113.48137013)(644.53391357,113.55137451)
\curveto(644.53391299,113.62136999)(644.54391298,113.68636992)(644.56391357,113.74637451)
\curveto(644.60391292,113.91636969)(644.63891288,114.08636952)(644.66891357,114.25637451)
\curveto(644.69891282,114.42636918)(644.74391278,114.58636902)(644.80391357,114.73637451)
\curveto(645.01391251,115.25636835)(645.26891225,115.67636793)(645.56891357,115.99637451)
\curveto(645.86891165,116.31636729)(646.27891124,116.58136703)(646.79891357,116.79137451)
\curveto(646.90891061,116.84136677)(647.02891049,116.87636673)(647.15891357,116.89637451)
\curveto(647.28891023,116.91636669)(647.4239101,116.94136667)(647.56391357,116.97137451)
\curveto(647.63390989,116.98136663)(647.70390982,116.98636662)(647.77391357,116.98637451)
\curveto(647.84390968,116.99636661)(647.9189096,117.0063666)(647.99891357,117.01637451)
}
}
{
\newrgbcolor{curcolor}{0 0 0}
\pscustom[linestyle=none,fillstyle=solid,fillcolor=curcolor]
{
\newpath
\moveto(659.6705542,113.28137451)
\curveto(659.69054651,113.18137043)(659.69054651,113.06637054)(659.6705542,112.93637451)
\curveto(659.66054654,112.81637079)(659.63054657,112.73137088)(659.5805542,112.68137451)
\curveto(659.53054667,112.64137097)(659.45554675,112.611371)(659.3555542,112.59137451)
\curveto(659.26554694,112.58137103)(659.16054704,112.57637103)(659.0405542,112.57637451)
\lineto(658.6805542,112.57637451)
\curveto(658.56054764,112.58637102)(658.45554775,112.59137102)(658.3655542,112.59137451)
\lineto(654.5255542,112.59137451)
\curveto(654.44555176,112.59137102)(654.36555184,112.58637102)(654.2855542,112.57637451)
\curveto(654.205552,112.57637103)(654.14055206,112.56137105)(654.0905542,112.53137451)
\curveto(654.05055215,112.5113711)(654.01055219,112.47137114)(653.9705542,112.41137451)
\curveto(653.95055225,112.38137123)(653.93055227,112.33637127)(653.9105542,112.27637451)
\curveto(653.89055231,112.22637138)(653.89055231,112.17637143)(653.9105542,112.12637451)
\curveto(653.92055228,112.07637153)(653.92555228,112.03137158)(653.9255542,111.99137451)
\curveto(653.92555228,111.95137166)(653.93055227,111.9113717)(653.9405542,111.87137451)
\curveto(653.96055224,111.79137182)(653.98055222,111.7063719)(654.0005542,111.61637451)
\curveto(654.02055218,111.53637207)(654.05055215,111.45637215)(654.0905542,111.37637451)
\curveto(654.32055188,110.83637277)(654.7005515,110.45137316)(655.2305542,110.22137451)
\curveto(655.29055091,110.19137342)(655.35555085,110.16637344)(655.4255542,110.14637451)
\lineto(655.6355542,110.08637451)
\curveto(655.66555054,110.07637353)(655.71555049,110.07137354)(655.7855542,110.07137451)
\curveto(655.92555028,110.03137358)(656.11055009,110.0113736)(656.3405542,110.01137451)
\curveto(656.57054963,110.0113736)(656.75554945,110.03137358)(656.8955542,110.07137451)
\curveto(657.03554917,110.1113735)(657.16054904,110.15137346)(657.2705542,110.19137451)
\curveto(657.39054881,110.24137337)(657.5005487,110.30137331)(657.6005542,110.37137451)
\curveto(657.71054849,110.44137317)(657.8055484,110.52137309)(657.8855542,110.61137451)
\curveto(657.96554824,110.7113729)(658.03554817,110.81637279)(658.0955542,110.92637451)
\curveto(658.15554805,111.02637258)(658.205548,111.13137248)(658.2455542,111.24137451)
\curveto(658.29554791,111.35137226)(658.37554783,111.43137218)(658.4855542,111.48137451)
\curveto(658.52554768,111.50137211)(658.59054761,111.51637209)(658.6805542,111.52637451)
\curveto(658.77054743,111.53637207)(658.86054734,111.53637207)(658.9505542,111.52637451)
\curveto(659.04054716,111.52637208)(659.12554708,111.52137209)(659.2055542,111.51137451)
\curveto(659.28554692,111.50137211)(659.34054686,111.48137213)(659.3705542,111.45137451)
\curveto(659.47054673,111.38137223)(659.49554671,111.26637234)(659.4455542,111.10637451)
\curveto(659.36554684,110.83637277)(659.26054694,110.59637301)(659.1305542,110.38637451)
\curveto(658.93054727,110.06637354)(658.7005475,109.80137381)(658.4405542,109.59137451)
\curveto(658.19054801,109.39137422)(657.87054833,109.22637438)(657.4805542,109.09637451)
\curveto(657.38054882,109.05637455)(657.28054892,109.03137458)(657.1805542,109.02137451)
\curveto(657.08054912,109.00137461)(656.97554923,108.98137463)(656.8655542,108.96137451)
\curveto(656.81554939,108.95137466)(656.76554944,108.94637466)(656.7155542,108.94637451)
\curveto(656.67554953,108.94637466)(656.63054957,108.94137467)(656.5805542,108.93137451)
\lineto(656.4305542,108.93137451)
\curveto(656.38054982,108.92137469)(656.32054988,108.91637469)(656.2505542,108.91637451)
\curveto(656.19055001,108.91637469)(656.14055006,108.92137469)(656.1005542,108.93137451)
\lineto(655.9655542,108.93137451)
\curveto(655.91555029,108.94137467)(655.87055033,108.94637466)(655.8305542,108.94637451)
\curveto(655.79055041,108.94637466)(655.75055045,108.95137466)(655.7105542,108.96137451)
\curveto(655.66055054,108.97137464)(655.6055506,108.98137463)(655.5455542,108.99137451)
\curveto(655.48555072,108.99137462)(655.43055077,108.99637461)(655.3805542,109.00637451)
\curveto(655.29055091,109.02637458)(655.200551,109.05137456)(655.1105542,109.08137451)
\curveto(655.02055118,109.10137451)(654.93555127,109.12637448)(654.8555542,109.15637451)
\curveto(654.81555139,109.17637443)(654.78055142,109.18637442)(654.7505542,109.18637451)
\curveto(654.72055148,109.19637441)(654.68555152,109.2113744)(654.6455542,109.23137451)
\curveto(654.49555171,109.30137431)(654.33555187,109.38637422)(654.1655542,109.48637451)
\curveto(653.87555233,109.67637393)(653.62555258,109.9063737)(653.4155542,110.17637451)
\curveto(653.21555299,110.45637315)(653.04555316,110.76637284)(652.9055542,111.10637451)
\curveto(652.85555335,111.21637239)(652.81555339,111.33137228)(652.7855542,111.45137451)
\curveto(652.76555344,111.57137204)(652.73555347,111.69137192)(652.6955542,111.81137451)
\curveto(652.68555352,111.85137176)(652.68055352,111.88637172)(652.6805542,111.91637451)
\curveto(652.68055352,111.94637166)(652.67555353,111.98637162)(652.6655542,112.03637451)
\curveto(652.64555356,112.11637149)(652.63055357,112.20137141)(652.6205542,112.29137451)
\curveto(652.61055359,112.38137123)(652.59555361,112.47137114)(652.5755542,112.56137451)
\lineto(652.5755542,112.77137451)
\curveto(652.56555364,112.8113708)(652.55555365,112.86637074)(652.5455542,112.93637451)
\curveto(652.54555366,113.01637059)(652.55055365,113.08137053)(652.5605542,113.13137451)
\lineto(652.5605542,113.29637451)
\curveto(652.58055362,113.34637026)(652.58555362,113.39637021)(652.5755542,113.44637451)
\curveto(652.57555363,113.5063701)(652.58055362,113.56137005)(652.5905542,113.61137451)
\curveto(652.63055357,113.77136984)(652.66055354,113.93136968)(652.6805542,114.09137451)
\curveto(652.71055349,114.25136936)(652.75555345,114.40136921)(652.8155542,114.54137451)
\curveto(652.86555334,114.65136896)(652.91055329,114.76136885)(652.9505542,114.87137451)
\curveto(653.0005532,114.99136862)(653.05555315,115.1063685)(653.1155542,115.21637451)
\curveto(653.33555287,115.56636804)(653.58555262,115.86636774)(653.8655542,116.11637451)
\curveto(654.14555206,116.37636723)(654.49055171,116.59136702)(654.9005542,116.76137451)
\curveto(655.02055118,116.8113668)(655.14055106,116.84636676)(655.2605542,116.86637451)
\curveto(655.39055081,116.89636671)(655.52555068,116.92636668)(655.6655542,116.95637451)
\curveto(655.71555049,116.96636664)(655.76055044,116.97136664)(655.8005542,116.97137451)
\curveto(655.84055036,116.98136663)(655.88555032,116.98636662)(655.9355542,116.98637451)
\curveto(655.95555025,116.99636661)(655.98055022,116.99636661)(656.0105542,116.98637451)
\curveto(656.04055016,116.97636663)(656.06555014,116.98136663)(656.0855542,117.00137451)
\curveto(656.5055497,117.0113666)(656.87054933,116.96636664)(657.1805542,116.86637451)
\curveto(657.49054871,116.77636683)(657.77054843,116.65136696)(658.0205542,116.49137451)
\curveto(658.07054813,116.47136714)(658.11054809,116.44136717)(658.1405542,116.40137451)
\curveto(658.17054803,116.37136724)(658.205548,116.34636726)(658.2455542,116.32637451)
\curveto(658.32554788,116.26636734)(658.4055478,116.19636741)(658.4855542,116.11637451)
\curveto(658.57554763,116.03636757)(658.65054755,115.95636765)(658.7105542,115.87637451)
\curveto(658.87054733,115.66636794)(659.0055472,115.46636814)(659.1155542,115.27637451)
\curveto(659.18554702,115.16636844)(659.24054696,115.04636856)(659.2805542,114.91637451)
\curveto(659.32054688,114.78636882)(659.36554684,114.65636895)(659.4155542,114.52637451)
\curveto(659.46554674,114.39636921)(659.5005467,114.26136935)(659.5205542,114.12137451)
\curveto(659.55054665,113.98136963)(659.58554662,113.84136977)(659.6255542,113.70137451)
\curveto(659.63554657,113.63136998)(659.64054656,113.56137005)(659.6405542,113.49137451)
\lineto(659.6705542,113.28137451)
\moveto(658.2155542,113.79137451)
\curveto(658.24554796,113.83136978)(658.27054793,113.88136973)(658.2905542,113.94137451)
\curveto(658.31054789,114.0113696)(658.31054789,114.08136953)(658.2905542,114.15137451)
\curveto(658.23054797,114.37136924)(658.14554806,114.57636903)(658.0355542,114.76637451)
\curveto(657.89554831,114.99636861)(657.74054846,115.19136842)(657.5705542,115.35137451)
\curveto(657.4005488,115.5113681)(657.18054902,115.64636796)(656.9105542,115.75637451)
\curveto(656.84054936,115.77636783)(656.77054943,115.79136782)(656.7005542,115.80137451)
\curveto(656.63054957,115.82136779)(656.55554965,115.84136777)(656.4755542,115.86137451)
\curveto(656.39554981,115.88136773)(656.31054989,115.89136772)(656.2205542,115.89137451)
\lineto(655.9655542,115.89137451)
\curveto(655.93555027,115.87136774)(655.9005503,115.86136775)(655.8605542,115.86137451)
\curveto(655.82055038,115.87136774)(655.78555042,115.87136774)(655.7555542,115.86137451)
\lineto(655.5155542,115.80137451)
\curveto(655.44555076,115.79136782)(655.37555083,115.77636783)(655.3055542,115.75637451)
\curveto(655.01555119,115.63636797)(654.78055142,115.48636812)(654.6005542,115.30637451)
\curveto(654.43055177,115.12636848)(654.27555193,114.90136871)(654.1355542,114.63137451)
\curveto(654.1055521,114.58136903)(654.07555213,114.51636909)(654.0455542,114.43637451)
\curveto(654.01555219,114.36636924)(653.99055221,114.28636932)(653.9705542,114.19637451)
\curveto(653.95055225,114.1063695)(653.94555226,114.02136959)(653.9555542,113.94137451)
\curveto(653.96555224,113.86136975)(654.0005522,113.80136981)(654.0605542,113.76137451)
\curveto(654.14055206,113.70136991)(654.27555193,113.67136994)(654.4655542,113.67137451)
\curveto(654.66555154,113.68136993)(654.83555137,113.68636992)(654.9755542,113.68637451)
\lineto(657.2555542,113.68637451)
\curveto(657.4055488,113.68636992)(657.58554862,113.68136993)(657.7955542,113.67137451)
\curveto(658.0055482,113.67136994)(658.14554806,113.7113699)(658.2155542,113.79137451)
}
}
{
\newrgbcolor{curcolor}{0 0 0}
\pscustom[linestyle=none,fillstyle=solid,fillcolor=curcolor]
{
\newpath
\moveto(664.66719482,116.98637451)
\curveto(665.29718959,117.0063666)(665.80218908,116.92136669)(666.18219482,116.73137451)
\curveto(666.56218832,116.54136707)(666.86718802,116.25636735)(667.09719482,115.87637451)
\curveto(667.15718773,115.77636783)(667.20218768,115.66636794)(667.23219482,115.54637451)
\curveto(667.27218761,115.43636817)(667.30718758,115.32136829)(667.33719482,115.20137451)
\curveto(667.3871875,115.0113686)(667.41718747,114.8063688)(667.42719482,114.58637451)
\curveto(667.43718745,114.36636924)(667.44218744,114.14136947)(667.44219482,113.91137451)
\lineto(667.44219482,112.30637451)
\lineto(667.44219482,109.96637451)
\curveto(667.44218744,109.79637381)(667.43718745,109.62637398)(667.42719482,109.45637451)
\curveto(667.42718746,109.28637432)(667.36218752,109.17637443)(667.23219482,109.12637451)
\curveto(667.1821877,109.1063745)(667.12718776,109.09637451)(667.06719482,109.09637451)
\curveto(667.01718787,109.08637452)(666.96218792,109.08137453)(666.90219482,109.08137451)
\curveto(666.77218811,109.08137453)(666.64718824,109.08637452)(666.52719482,109.09637451)
\curveto(666.40718848,109.09637451)(666.32218856,109.13637447)(666.27219482,109.21637451)
\curveto(666.22218866,109.28637432)(666.19718869,109.37637423)(666.19719482,109.48637451)
\lineto(666.19719482,109.81637451)
\lineto(666.19719482,111.10637451)
\lineto(666.19719482,113.55137451)
\curveto(666.19718869,113.82136979)(666.19218869,114.08636952)(666.18219482,114.34637451)
\curveto(666.17218871,114.61636899)(666.12718876,114.84636876)(666.04719482,115.03637451)
\curveto(665.96718892,115.23636837)(665.84718904,115.39636821)(665.68719482,115.51637451)
\curveto(665.52718936,115.64636796)(665.34218954,115.74636786)(665.13219482,115.81637451)
\curveto(665.07218981,115.83636777)(665.00718988,115.84636776)(664.93719482,115.84637451)
\curveto(664.87719001,115.85636775)(664.81719007,115.87136774)(664.75719482,115.89137451)
\curveto(664.70719018,115.90136771)(664.62719026,115.90136771)(664.51719482,115.89137451)
\curveto(664.41719047,115.89136772)(664.34719054,115.88636772)(664.30719482,115.87637451)
\curveto(664.26719062,115.85636775)(664.23219065,115.84636776)(664.20219482,115.84637451)
\curveto(664.17219071,115.85636775)(664.13719075,115.85636775)(664.09719482,115.84637451)
\curveto(663.96719092,115.81636779)(663.84219104,115.78136783)(663.72219482,115.74137451)
\curveto(663.61219127,115.7113679)(663.50719138,115.66636794)(663.40719482,115.60637451)
\curveto(663.36719152,115.58636802)(663.33219155,115.56636804)(663.30219482,115.54637451)
\curveto(663.27219161,115.52636808)(663.23719165,115.5063681)(663.19719482,115.48637451)
\curveto(662.84719204,115.23636837)(662.59219229,114.86136875)(662.43219482,114.36137451)
\curveto(662.40219248,114.28136933)(662.3821925,114.19636941)(662.37219482,114.10637451)
\curveto(662.36219252,114.02636958)(662.34719254,113.94636966)(662.32719482,113.86637451)
\curveto(662.30719258,113.81636979)(662.30219258,113.76636984)(662.31219482,113.71637451)
\curveto(662.32219256,113.67636993)(662.31719257,113.63636997)(662.29719482,113.59637451)
\lineto(662.29719482,113.28137451)
\curveto(662.2871926,113.25137036)(662.2821926,113.21637039)(662.28219482,113.17637451)
\curveto(662.29219259,113.13637047)(662.29719259,113.09137052)(662.29719482,113.04137451)
\lineto(662.29719482,112.59137451)
\lineto(662.29719482,111.15137451)
\lineto(662.29719482,109.83137451)
\lineto(662.29719482,109.48637451)
\curveto(662.29719259,109.37637423)(662.27219261,109.28637432)(662.22219482,109.21637451)
\curveto(662.17219271,109.13637447)(662.0821928,109.09637451)(661.95219482,109.09637451)
\curveto(661.83219305,109.08637452)(661.70719318,109.08137453)(661.57719482,109.08137451)
\curveto(661.49719339,109.08137453)(661.42219346,109.08637452)(661.35219482,109.09637451)
\curveto(661.2821936,109.1063745)(661.22219366,109.13137448)(661.17219482,109.17137451)
\curveto(661.09219379,109.22137439)(661.05219383,109.31637429)(661.05219482,109.45637451)
\lineto(661.05219482,109.86137451)
\lineto(661.05219482,111.63137451)
\lineto(661.05219482,115.26137451)
\lineto(661.05219482,116.17637451)
\lineto(661.05219482,116.44637451)
\curveto(661.05219383,116.53636707)(661.07219381,116.606367)(661.11219482,116.65637451)
\curveto(661.14219374,116.71636689)(661.19219369,116.75636685)(661.26219482,116.77637451)
\curveto(661.30219358,116.78636682)(661.35719353,116.79636681)(661.42719482,116.80637451)
\curveto(661.50719338,116.81636679)(661.5871933,116.82136679)(661.66719482,116.82137451)
\curveto(661.74719314,116.82136679)(661.82219306,116.81636679)(661.89219482,116.80637451)
\curveto(661.97219291,116.79636681)(662.02719286,116.78136683)(662.05719482,116.76137451)
\curveto(662.16719272,116.69136692)(662.21719267,116.60136701)(662.20719482,116.49137451)
\curveto(662.19719269,116.39136722)(662.21219267,116.27636733)(662.25219482,116.14637451)
\curveto(662.27219261,116.08636752)(662.31219257,116.03636757)(662.37219482,115.99637451)
\curveto(662.49219239,115.98636762)(662.5871923,116.03136758)(662.65719482,116.13137451)
\curveto(662.73719215,116.23136738)(662.81719207,116.3113673)(662.89719482,116.37137451)
\curveto(663.03719185,116.47136714)(663.17719171,116.56136705)(663.31719482,116.64137451)
\curveto(663.46719142,116.73136688)(663.63719125,116.8063668)(663.82719482,116.86637451)
\curveto(663.90719098,116.89636671)(663.99219089,116.91636669)(664.08219482,116.92637451)
\curveto(664.1821907,116.93636667)(664.27719061,116.95136666)(664.36719482,116.97137451)
\curveto(664.41719047,116.98136663)(664.46719042,116.98636662)(664.51719482,116.98637451)
\lineto(664.66719482,116.98637451)
}
}
{
\newrgbcolor{curcolor}{0 0 0}
\pscustom[linestyle=none,fillstyle=solid,fillcolor=curcolor]
{
\newpath
\moveto(670.2718042,119.17637451)
\curveto(670.42180219,119.17636443)(670.57180204,119.17136444)(670.7218042,119.16137451)
\curveto(670.87180174,119.16136445)(670.97680163,119.12136449)(671.0368042,119.04137451)
\curveto(671.08680152,118.98136463)(671.1118015,118.89636471)(671.1118042,118.78637451)
\curveto(671.12180149,118.68636492)(671.12680148,118.58136503)(671.1268042,118.47137451)
\lineto(671.1268042,117.60137451)
\curveto(671.12680148,117.52136609)(671.12180149,117.43636617)(671.1118042,117.34637451)
\curveto(671.1118015,117.26636634)(671.12180149,117.19636641)(671.1418042,117.13637451)
\curveto(671.18180143,116.99636661)(671.27180134,116.9063667)(671.4118042,116.86637451)
\curveto(671.46180115,116.85636675)(671.5068011,116.85136676)(671.5468042,116.85137451)
\lineto(671.6968042,116.85137451)
\lineto(672.1018042,116.85137451)
\curveto(672.26180035,116.86136675)(672.37680023,116.85136676)(672.4468042,116.82137451)
\curveto(672.53680007,116.76136685)(672.59680001,116.70136691)(672.6268042,116.64137451)
\curveto(672.64679996,116.60136701)(672.65679995,116.55636705)(672.6568042,116.50637451)
\lineto(672.6568042,116.35637451)
\curveto(672.65679995,116.24636736)(672.65179996,116.14136747)(672.6418042,116.04137451)
\curveto(672.63179998,115.95136766)(672.59680001,115.88136773)(672.5368042,115.83137451)
\curveto(672.47680013,115.78136783)(672.39180022,115.75136786)(672.2818042,115.74137451)
\lineto(671.9518042,115.74137451)
\curveto(671.84180077,115.75136786)(671.73180088,115.75636785)(671.6218042,115.75637451)
\curveto(671.5118011,115.75636785)(671.41680119,115.74136787)(671.3368042,115.71137451)
\curveto(671.26680134,115.68136793)(671.21680139,115.63136798)(671.1868042,115.56137451)
\curveto(671.15680145,115.49136812)(671.13680147,115.4063682)(671.1268042,115.30637451)
\curveto(671.11680149,115.21636839)(671.1118015,115.11636849)(671.1118042,115.00637451)
\curveto(671.12180149,114.9063687)(671.12680148,114.8063688)(671.1268042,114.70637451)
\lineto(671.1268042,111.73637451)
\curveto(671.12680148,111.51637209)(671.12180149,111.28137233)(671.1118042,111.03137451)
\curveto(671.1118015,110.79137282)(671.15680145,110.606373)(671.2468042,110.47637451)
\curveto(671.29680131,110.39637321)(671.36180125,110.34137327)(671.4418042,110.31137451)
\curveto(671.52180109,110.28137333)(671.61680099,110.25637335)(671.7268042,110.23637451)
\curveto(671.75680085,110.22637338)(671.78680082,110.22137339)(671.8168042,110.22137451)
\curveto(671.85680075,110.23137338)(671.89180072,110.23137338)(671.9218042,110.22137451)
\lineto(672.1168042,110.22137451)
\curveto(672.21680039,110.22137339)(672.3068003,110.2113734)(672.3868042,110.19137451)
\curveto(672.47680013,110.18137343)(672.54180007,110.14637346)(672.5818042,110.08637451)
\curveto(672.60180001,110.05637355)(672.61679999,110.00137361)(672.6268042,109.92137451)
\curveto(672.64679996,109.85137376)(672.65679995,109.77637383)(672.6568042,109.69637451)
\curveto(672.66679994,109.61637399)(672.66679994,109.53637407)(672.6568042,109.45637451)
\curveto(672.64679996,109.38637422)(672.62679998,109.33137428)(672.5968042,109.29137451)
\curveto(672.55680005,109.22137439)(672.48180013,109.17137444)(672.3718042,109.14137451)
\curveto(672.29180032,109.12137449)(672.20180041,109.1113745)(672.1018042,109.11137451)
\curveto(672.00180061,109.12137449)(671.9118007,109.12637448)(671.8318042,109.12637451)
\curveto(671.77180084,109.12637448)(671.7118009,109.12137449)(671.6518042,109.11137451)
\curveto(671.59180102,109.1113745)(671.53680107,109.11637449)(671.4868042,109.12637451)
\lineto(671.3068042,109.12637451)
\curveto(671.25680135,109.13637447)(671.2068014,109.14137447)(671.1568042,109.14137451)
\curveto(671.11680149,109.15137446)(671.07180154,109.15637445)(671.0218042,109.15637451)
\curveto(670.82180179,109.2063744)(670.64680196,109.26137435)(670.4968042,109.32137451)
\curveto(670.35680225,109.38137423)(670.23680237,109.48637412)(670.1368042,109.63637451)
\curveto(669.99680261,109.83637377)(669.91680269,110.08637352)(669.8968042,110.38637451)
\curveto(669.87680273,110.69637291)(669.86680274,111.02637258)(669.8668042,111.37637451)
\lineto(669.8668042,115.30637451)
\curveto(669.83680277,115.43636817)(669.8068028,115.53136808)(669.7768042,115.59137451)
\curveto(669.75680285,115.65136796)(669.68680292,115.70136791)(669.5668042,115.74137451)
\curveto(669.52680308,115.75136786)(669.48680312,115.75136786)(669.4468042,115.74137451)
\curveto(669.4068032,115.73136788)(669.36680324,115.73636787)(669.3268042,115.75637451)
\lineto(669.0868042,115.75637451)
\curveto(668.95680365,115.75636785)(668.84680376,115.76636784)(668.7568042,115.78637451)
\curveto(668.67680393,115.81636779)(668.62180399,115.87636773)(668.5918042,115.96637451)
\curveto(668.57180404,116.0063676)(668.55680405,116.05136756)(668.5468042,116.10137451)
\lineto(668.5468042,116.25137451)
\curveto(668.54680406,116.39136722)(668.55680405,116.5063671)(668.5768042,116.59637451)
\curveto(668.59680401,116.69636691)(668.65680395,116.77136684)(668.7568042,116.82137451)
\curveto(668.86680374,116.86136675)(669.0068036,116.87136674)(669.1768042,116.85137451)
\curveto(669.35680325,116.83136678)(669.5068031,116.84136677)(669.6268042,116.88137451)
\curveto(669.71680289,116.93136668)(669.78680282,117.00136661)(669.8368042,117.09137451)
\curveto(669.85680275,117.15136646)(669.86680274,117.22636638)(669.8668042,117.31637451)
\lineto(669.8668042,117.57137451)
\lineto(669.8668042,118.50137451)
\lineto(669.8668042,118.74137451)
\curveto(669.86680274,118.83136478)(669.87680273,118.9063647)(669.8968042,118.96637451)
\curveto(669.93680267,119.04636456)(670.0118026,119.1113645)(670.1218042,119.16137451)
\curveto(670.15180246,119.16136445)(670.17680243,119.16136445)(670.1968042,119.16137451)
\curveto(670.22680238,119.17136444)(670.25180236,119.17636443)(670.2718042,119.17637451)
}
}
{
\newrgbcolor{curcolor}{0 0 0}
\pscustom[linestyle=none,fillstyle=solid,fillcolor=curcolor]
{
\newpath
\moveto(680.79360107,113.28137451)
\curveto(680.81359339,113.18137043)(680.81359339,113.06637054)(680.79360107,112.93637451)
\curveto(680.78359342,112.81637079)(680.75359345,112.73137088)(680.70360107,112.68137451)
\curveto(680.65359355,112.64137097)(680.57859362,112.611371)(680.47860107,112.59137451)
\curveto(680.38859381,112.58137103)(680.28359392,112.57637103)(680.16360107,112.57637451)
\lineto(679.80360107,112.57637451)
\curveto(679.68359452,112.58637102)(679.57859462,112.59137102)(679.48860107,112.59137451)
\lineto(675.64860107,112.59137451)
\curveto(675.56859863,112.59137102)(675.48859871,112.58637102)(675.40860107,112.57637451)
\curveto(675.32859887,112.57637103)(675.26359894,112.56137105)(675.21360107,112.53137451)
\curveto(675.17359903,112.5113711)(675.13359907,112.47137114)(675.09360107,112.41137451)
\curveto(675.07359913,112.38137123)(675.05359915,112.33637127)(675.03360107,112.27637451)
\curveto(675.01359919,112.22637138)(675.01359919,112.17637143)(675.03360107,112.12637451)
\curveto(675.04359916,112.07637153)(675.04859915,112.03137158)(675.04860107,111.99137451)
\curveto(675.04859915,111.95137166)(675.05359915,111.9113717)(675.06360107,111.87137451)
\curveto(675.08359912,111.79137182)(675.1035991,111.7063719)(675.12360107,111.61637451)
\curveto(675.14359906,111.53637207)(675.17359903,111.45637215)(675.21360107,111.37637451)
\curveto(675.44359876,110.83637277)(675.82359838,110.45137316)(676.35360107,110.22137451)
\curveto(676.41359779,110.19137342)(676.47859772,110.16637344)(676.54860107,110.14637451)
\lineto(676.75860107,110.08637451)
\curveto(676.78859741,110.07637353)(676.83859736,110.07137354)(676.90860107,110.07137451)
\curveto(677.04859715,110.03137358)(677.23359697,110.0113736)(677.46360107,110.01137451)
\curveto(677.69359651,110.0113736)(677.87859632,110.03137358)(678.01860107,110.07137451)
\curveto(678.15859604,110.1113735)(678.28359592,110.15137346)(678.39360107,110.19137451)
\curveto(678.51359569,110.24137337)(678.62359558,110.30137331)(678.72360107,110.37137451)
\curveto(678.83359537,110.44137317)(678.92859527,110.52137309)(679.00860107,110.61137451)
\curveto(679.08859511,110.7113729)(679.15859504,110.81637279)(679.21860107,110.92637451)
\curveto(679.27859492,111.02637258)(679.32859487,111.13137248)(679.36860107,111.24137451)
\curveto(679.41859478,111.35137226)(679.4985947,111.43137218)(679.60860107,111.48137451)
\curveto(679.64859455,111.50137211)(679.71359449,111.51637209)(679.80360107,111.52637451)
\curveto(679.89359431,111.53637207)(679.98359422,111.53637207)(680.07360107,111.52637451)
\curveto(680.16359404,111.52637208)(680.24859395,111.52137209)(680.32860107,111.51137451)
\curveto(680.40859379,111.50137211)(680.46359374,111.48137213)(680.49360107,111.45137451)
\curveto(680.59359361,111.38137223)(680.61859358,111.26637234)(680.56860107,111.10637451)
\curveto(680.48859371,110.83637277)(680.38359382,110.59637301)(680.25360107,110.38637451)
\curveto(680.05359415,110.06637354)(679.82359438,109.80137381)(679.56360107,109.59137451)
\curveto(679.31359489,109.39137422)(678.99359521,109.22637438)(678.60360107,109.09637451)
\curveto(678.5035957,109.05637455)(678.4035958,109.03137458)(678.30360107,109.02137451)
\curveto(678.203596,109.00137461)(678.0985961,108.98137463)(677.98860107,108.96137451)
\curveto(677.93859626,108.95137466)(677.88859631,108.94637466)(677.83860107,108.94637451)
\curveto(677.7985964,108.94637466)(677.75359645,108.94137467)(677.70360107,108.93137451)
\lineto(677.55360107,108.93137451)
\curveto(677.5035967,108.92137469)(677.44359676,108.91637469)(677.37360107,108.91637451)
\curveto(677.31359689,108.91637469)(677.26359694,108.92137469)(677.22360107,108.93137451)
\lineto(677.08860107,108.93137451)
\curveto(677.03859716,108.94137467)(676.99359721,108.94637466)(676.95360107,108.94637451)
\curveto(676.91359729,108.94637466)(676.87359733,108.95137466)(676.83360107,108.96137451)
\curveto(676.78359742,108.97137464)(676.72859747,108.98137463)(676.66860107,108.99137451)
\curveto(676.60859759,108.99137462)(676.55359765,108.99637461)(676.50360107,109.00637451)
\curveto(676.41359779,109.02637458)(676.32359788,109.05137456)(676.23360107,109.08137451)
\curveto(676.14359806,109.10137451)(676.05859814,109.12637448)(675.97860107,109.15637451)
\curveto(675.93859826,109.17637443)(675.9035983,109.18637442)(675.87360107,109.18637451)
\curveto(675.84359836,109.19637441)(675.80859839,109.2113744)(675.76860107,109.23137451)
\curveto(675.61859858,109.30137431)(675.45859874,109.38637422)(675.28860107,109.48637451)
\curveto(674.9985992,109.67637393)(674.74859945,109.9063737)(674.53860107,110.17637451)
\curveto(674.33859986,110.45637315)(674.16860003,110.76637284)(674.02860107,111.10637451)
\curveto(673.97860022,111.21637239)(673.93860026,111.33137228)(673.90860107,111.45137451)
\curveto(673.88860031,111.57137204)(673.85860034,111.69137192)(673.81860107,111.81137451)
\curveto(673.80860039,111.85137176)(673.8036004,111.88637172)(673.80360107,111.91637451)
\curveto(673.8036004,111.94637166)(673.7986004,111.98637162)(673.78860107,112.03637451)
\curveto(673.76860043,112.11637149)(673.75360045,112.20137141)(673.74360107,112.29137451)
\curveto(673.73360047,112.38137123)(673.71860048,112.47137114)(673.69860107,112.56137451)
\lineto(673.69860107,112.77137451)
\curveto(673.68860051,112.8113708)(673.67860052,112.86637074)(673.66860107,112.93637451)
\curveto(673.66860053,113.01637059)(673.67360053,113.08137053)(673.68360107,113.13137451)
\lineto(673.68360107,113.29637451)
\curveto(673.7036005,113.34637026)(673.70860049,113.39637021)(673.69860107,113.44637451)
\curveto(673.6986005,113.5063701)(673.7036005,113.56137005)(673.71360107,113.61137451)
\curveto(673.75360045,113.77136984)(673.78360042,113.93136968)(673.80360107,114.09137451)
\curveto(673.83360037,114.25136936)(673.87860032,114.40136921)(673.93860107,114.54137451)
\curveto(673.98860021,114.65136896)(674.03360017,114.76136885)(674.07360107,114.87137451)
\curveto(674.12360008,114.99136862)(674.17860002,115.1063685)(674.23860107,115.21637451)
\curveto(674.45859974,115.56636804)(674.70859949,115.86636774)(674.98860107,116.11637451)
\curveto(675.26859893,116.37636723)(675.61359859,116.59136702)(676.02360107,116.76137451)
\curveto(676.14359806,116.8113668)(676.26359794,116.84636676)(676.38360107,116.86637451)
\curveto(676.51359769,116.89636671)(676.64859755,116.92636668)(676.78860107,116.95637451)
\curveto(676.83859736,116.96636664)(676.88359732,116.97136664)(676.92360107,116.97137451)
\curveto(676.96359724,116.98136663)(677.00859719,116.98636662)(677.05860107,116.98637451)
\curveto(677.07859712,116.99636661)(677.1035971,116.99636661)(677.13360107,116.98637451)
\curveto(677.16359704,116.97636663)(677.18859701,116.98136663)(677.20860107,117.00137451)
\curveto(677.62859657,117.0113666)(677.99359621,116.96636664)(678.30360107,116.86637451)
\curveto(678.61359559,116.77636683)(678.89359531,116.65136696)(679.14360107,116.49137451)
\curveto(679.19359501,116.47136714)(679.23359497,116.44136717)(679.26360107,116.40137451)
\curveto(679.29359491,116.37136724)(679.32859487,116.34636726)(679.36860107,116.32637451)
\curveto(679.44859475,116.26636734)(679.52859467,116.19636741)(679.60860107,116.11637451)
\curveto(679.6985945,116.03636757)(679.77359443,115.95636765)(679.83360107,115.87637451)
\curveto(679.99359421,115.66636794)(680.12859407,115.46636814)(680.23860107,115.27637451)
\curveto(680.30859389,115.16636844)(680.36359384,115.04636856)(680.40360107,114.91637451)
\curveto(680.44359376,114.78636882)(680.48859371,114.65636895)(680.53860107,114.52637451)
\curveto(680.58859361,114.39636921)(680.62359358,114.26136935)(680.64360107,114.12137451)
\curveto(680.67359353,113.98136963)(680.70859349,113.84136977)(680.74860107,113.70137451)
\curveto(680.75859344,113.63136998)(680.76359344,113.56137005)(680.76360107,113.49137451)
\lineto(680.79360107,113.28137451)
\moveto(679.33860107,113.79137451)
\curveto(679.36859483,113.83136978)(679.39359481,113.88136973)(679.41360107,113.94137451)
\curveto(679.43359477,114.0113696)(679.43359477,114.08136953)(679.41360107,114.15137451)
\curveto(679.35359485,114.37136924)(679.26859493,114.57636903)(679.15860107,114.76637451)
\curveto(679.01859518,114.99636861)(678.86359534,115.19136842)(678.69360107,115.35137451)
\curveto(678.52359568,115.5113681)(678.3035959,115.64636796)(678.03360107,115.75637451)
\curveto(677.96359624,115.77636783)(677.89359631,115.79136782)(677.82360107,115.80137451)
\curveto(677.75359645,115.82136779)(677.67859652,115.84136777)(677.59860107,115.86137451)
\curveto(677.51859668,115.88136773)(677.43359677,115.89136772)(677.34360107,115.89137451)
\lineto(677.08860107,115.89137451)
\curveto(677.05859714,115.87136774)(677.02359718,115.86136775)(676.98360107,115.86137451)
\curveto(676.94359726,115.87136774)(676.90859729,115.87136774)(676.87860107,115.86137451)
\lineto(676.63860107,115.80137451)
\curveto(676.56859763,115.79136782)(676.4985977,115.77636783)(676.42860107,115.75637451)
\curveto(676.13859806,115.63636797)(675.9035983,115.48636812)(675.72360107,115.30637451)
\curveto(675.55359865,115.12636848)(675.3985988,114.90136871)(675.25860107,114.63137451)
\curveto(675.22859897,114.58136903)(675.198599,114.51636909)(675.16860107,114.43637451)
\curveto(675.13859906,114.36636924)(675.11359909,114.28636932)(675.09360107,114.19637451)
\curveto(675.07359913,114.1063695)(675.06859913,114.02136959)(675.07860107,113.94137451)
\curveto(675.08859911,113.86136975)(675.12359908,113.80136981)(675.18360107,113.76137451)
\curveto(675.26359894,113.70136991)(675.3985988,113.67136994)(675.58860107,113.67137451)
\curveto(675.78859841,113.68136993)(675.95859824,113.68636992)(676.09860107,113.68637451)
\lineto(678.37860107,113.68637451)
\curveto(678.52859567,113.68636992)(678.70859549,113.68136993)(678.91860107,113.67137451)
\curveto(679.12859507,113.67136994)(679.26859493,113.7113699)(679.33860107,113.79137451)
}
}
{
\newrgbcolor{curcolor}{0.60000002 0.60000002 0.60000002}
\pscustom[linestyle=none,fillstyle=solid,fillcolor=curcolor]
{
\newpath
\moveto(606.37554932,119.82141113)
\lineto(621.37554932,119.82141113)
\lineto(621.37554932,104.82141113)
\lineto(606.37554932,104.82141113)
\closepath
}
}
{
\newrgbcolor{curcolor}{0 0 0}
\pscustom[linestyle=none,fillstyle=solid,fillcolor=curcolor]
{
\newpath
\moveto(626.42375732,96.75566895)
\lineto(627.33875732,96.75566895)
\curveto(627.43875467,96.75565825)(627.53375458,96.75565825)(627.62375732,96.75566895)
\curveto(627.7137544,96.75565825)(627.78875432,96.73565827)(627.84875732,96.69566895)
\curveto(627.93875417,96.63565837)(627.99875411,96.55565845)(628.02875732,96.45566895)
\curveto(628.06875404,96.35565865)(628.113754,96.25065876)(628.16375732,96.14066895)
\curveto(628.24375387,95.95065906)(628.3137538,95.76065925)(628.37375732,95.57066895)
\curveto(628.44375367,95.38065963)(628.51875359,95.19065982)(628.59875732,95.00066895)
\curveto(628.66875344,94.82066019)(628.73375338,94.63566037)(628.79375732,94.44566895)
\curveto(628.85375326,94.26566074)(628.92375319,94.08566092)(629.00375732,93.90566895)
\curveto(629.06375305,93.76566124)(629.11875299,93.62066139)(629.16875732,93.47066895)
\curveto(629.21875289,93.32066169)(629.27375284,93.17566183)(629.33375732,93.03566895)
\curveto(629.5137526,92.58566242)(629.68375243,92.13066288)(629.84375732,91.67066895)
\curveto(630.00375211,91.22066379)(630.17375194,90.77066424)(630.35375732,90.32066895)
\curveto(630.37375174,90.27066474)(630.38875172,90.22066479)(630.39875732,90.17066895)
\lineto(630.45875732,90.02066895)
\curveto(630.54875156,89.80066521)(630.63375148,89.57566543)(630.71375732,89.34566895)
\curveto(630.79375132,89.12566588)(630.87875123,88.9056661)(630.96875732,88.68566895)
\curveto(631.0087511,88.59566641)(631.04875106,88.48566652)(631.08875732,88.35566895)
\curveto(631.12875098,88.23566677)(631.19375092,88.16566684)(631.28375732,88.14566895)
\curveto(631.32375079,88.13566687)(631.35375076,88.13566687)(631.37375732,88.14566895)
\lineto(631.43375732,88.20566895)
\curveto(631.48375063,88.25566675)(631.51875059,88.3106667)(631.53875732,88.37066895)
\curveto(631.56875054,88.43066658)(631.59875051,88.49566651)(631.62875732,88.56566895)
\lineto(631.86875732,89.19566895)
\curveto(631.94875016,89.41566559)(632.02875008,89.63066538)(632.10875732,89.84066895)
\lineto(632.16875732,89.99066895)
\lineto(632.22875732,90.17066895)
\curveto(632.3087498,90.36066465)(632.37874973,90.55066446)(632.43875732,90.74066895)
\curveto(632.5087496,90.94066407)(632.58374953,91.14066387)(632.66375732,91.34066895)
\curveto(632.90374921,91.92066309)(633.12374899,92.5056625)(633.32375732,93.09566895)
\curveto(633.53374858,93.68566132)(633.75874835,94.27066074)(633.99875732,94.85066895)
\curveto(634.07874803,95.05065996)(634.15374796,95.25565975)(634.22375732,95.46566895)
\curveto(634.30374781,95.67565933)(634.38374773,95.88065913)(634.46375732,96.08066895)
\curveto(634.50374761,96.16065885)(634.53874757,96.26065875)(634.56875732,96.38066895)
\curveto(634.6087475,96.50065851)(634.66374745,96.58565842)(634.73375732,96.63566895)
\curveto(634.79374732,96.67565833)(634.86874724,96.7056583)(634.95875732,96.72566895)
\curveto(635.05874705,96.74565826)(635.16874694,96.75565825)(635.28875732,96.75566895)
\curveto(635.4087467,96.76565824)(635.52874658,96.76565824)(635.64875732,96.75566895)
\curveto(635.76874634,96.75565825)(635.87874623,96.75565825)(635.97875732,96.75566895)
\curveto(636.06874604,96.75565825)(636.15874595,96.75565825)(636.24875732,96.75566895)
\curveto(636.34874576,96.75565825)(636.42374569,96.73565827)(636.47375732,96.69566895)
\curveto(636.56374555,96.64565836)(636.6137455,96.55565845)(636.62375732,96.42566895)
\curveto(636.63374548,96.29565871)(636.63874547,96.15565885)(636.63875732,96.00566895)
\lineto(636.63875732,94.35566895)
\lineto(636.63875732,88.08566895)
\lineto(636.63875732,86.82566895)
\curveto(636.63874547,86.71566829)(636.63874547,86.6056684)(636.63875732,86.49566895)
\curveto(636.64874546,86.38566862)(636.62874548,86.30066871)(636.57875732,86.24066895)
\curveto(636.54874556,86.18066883)(636.50374561,86.14066887)(636.44375732,86.12066895)
\curveto(636.38374573,86.1106689)(636.3137458,86.09566891)(636.23375732,86.07566895)
\lineto(635.99375732,86.07566895)
\lineto(635.63375732,86.07566895)
\curveto(635.52374659,86.08566892)(635.44374667,86.13066888)(635.39375732,86.21066895)
\curveto(635.37374674,86.24066877)(635.35874675,86.27066874)(635.34875732,86.30066895)
\curveto(635.34874676,86.34066867)(635.33874677,86.38566862)(635.31875732,86.43566895)
\lineto(635.31875732,86.60066895)
\curveto(635.3087468,86.66066835)(635.30374681,86.73066828)(635.30375732,86.81066895)
\curveto(635.3137468,86.89066812)(635.31874679,86.96566804)(635.31875732,87.03566895)
\lineto(635.31875732,87.87566895)
\lineto(635.31875732,92.30066895)
\curveto(635.31874679,92.55066246)(635.31874679,92.80066221)(635.31875732,93.05066895)
\curveto(635.31874679,93.3106617)(635.3137468,93.56066145)(635.30375732,93.80066895)
\curveto(635.30374681,93.90066111)(635.29874681,94.010661)(635.28875732,94.13066895)
\curveto(635.27874683,94.25066076)(635.22374689,94.3106607)(635.12375732,94.31066895)
\lineto(635.12375732,94.29566895)
\curveto(635.05374706,94.27566073)(634.99374712,94.2106608)(634.94375732,94.10066895)
\curveto(634.90374721,93.99066102)(634.86874724,93.89566111)(634.83875732,93.81566895)
\curveto(634.76874734,93.64566136)(634.70374741,93.47066154)(634.64375732,93.29066895)
\curveto(634.58374753,93.12066189)(634.5137476,92.95066206)(634.43375732,92.78066895)
\curveto(634.4137477,92.73066228)(634.39874771,92.68566232)(634.38875732,92.64566895)
\curveto(634.37874773,92.6056624)(634.36374775,92.56066245)(634.34375732,92.51066895)
\curveto(634.26374785,92.33066268)(634.19374792,92.14566286)(634.13375732,91.95566895)
\curveto(634.08374803,91.77566323)(634.01874809,91.59566341)(633.93875732,91.41566895)
\curveto(633.86874824,91.26566374)(633.8087483,91.1106639)(633.75875732,90.95066895)
\curveto(633.7087484,90.80066421)(633.65374846,90.65066436)(633.59375732,90.50066895)
\curveto(633.39374872,90.03066498)(633.2137489,89.55566545)(633.05375732,89.07566895)
\curveto(632.89374922,88.6056664)(632.71874939,88.14066687)(632.52875732,87.68066895)
\curveto(632.44874966,87.50066751)(632.37874973,87.32066769)(632.31875732,87.14066895)
\curveto(632.25874985,86.96066805)(632.19374992,86.78066823)(632.12375732,86.60066895)
\curveto(632.07375004,86.49066852)(632.02375009,86.38566862)(631.97375732,86.28566895)
\curveto(631.93375018,86.19566881)(631.84875026,86.13066888)(631.71875732,86.09066895)
\curveto(631.69875041,86.08066893)(631.67375044,86.07566893)(631.64375732,86.07566895)
\curveto(631.62375049,86.08566892)(631.59875051,86.08566892)(631.56875732,86.07566895)
\curveto(631.53875057,86.06566894)(631.50375061,86.06066895)(631.46375732,86.06066895)
\curveto(631.42375069,86.07066894)(631.38375073,86.07566893)(631.34375732,86.07566895)
\lineto(631.04375732,86.07566895)
\curveto(630.94375117,86.07566893)(630.86375125,86.10066891)(630.80375732,86.15066895)
\curveto(630.72375139,86.20066881)(630.66375145,86.27066874)(630.62375732,86.36066895)
\curveto(630.59375152,86.46066855)(630.55375156,86.56066845)(630.50375732,86.66066895)
\curveto(630.42375169,86.86066815)(630.34375177,87.06566794)(630.26375732,87.27566895)
\curveto(630.19375192,87.49566751)(630.11875199,87.7056673)(630.03875732,87.90566895)
\curveto(629.95875215,88.08566692)(629.88875222,88.26566674)(629.82875732,88.44566895)
\curveto(629.77875233,88.63566637)(629.7137524,88.82066619)(629.63375732,89.00066895)
\curveto(629.40375271,89.56066545)(629.18875292,90.12566488)(628.98875732,90.69566895)
\curveto(628.78875332,91.26566374)(628.57375354,91.83066318)(628.34375732,92.39066895)
\lineto(628.10375732,93.02066895)
\curveto(628.03375408,93.24066177)(627.95875415,93.45066156)(627.87875732,93.65066895)
\curveto(627.82875428,93.76066125)(627.78375433,93.86566114)(627.74375732,93.96566895)
\curveto(627.7137544,94.07566093)(627.66375445,94.17066084)(627.59375732,94.25066895)
\curveto(627.58375453,94.27066074)(627.57375454,94.28066073)(627.56375732,94.28066895)
\lineto(627.53375732,94.31066895)
\lineto(627.45875732,94.31066895)
\lineto(627.42875732,94.28066895)
\curveto(627.41875469,94.28066073)(627.4087547,94.27566073)(627.39875732,94.26566895)
\curveto(627.37875473,94.21566079)(627.36875474,94.16066085)(627.36875732,94.10066895)
\curveto(627.36875474,94.04066097)(627.35875475,93.98066103)(627.33875732,93.92066895)
\lineto(627.33875732,93.75566895)
\curveto(627.31875479,93.69566131)(627.3137548,93.63066138)(627.32375732,93.56066895)
\curveto(627.33375478,93.49066152)(627.33875477,93.42066159)(627.33875732,93.35066895)
\lineto(627.33875732,92.54066895)
\lineto(627.33875732,87.98066895)
\lineto(627.33875732,86.79566895)
\curveto(627.33875477,86.68566832)(627.33375478,86.57566843)(627.32375732,86.46566895)
\curveto(627.32375479,86.35566865)(627.29875481,86.27066874)(627.24875732,86.21066895)
\curveto(627.19875491,86.13066888)(627.108755,86.08566892)(626.97875732,86.07566895)
\lineto(626.58875732,86.07566895)
\lineto(626.39375732,86.07566895)
\curveto(626.34375577,86.07566893)(626.29375582,86.08566892)(626.24375732,86.10566895)
\curveto(626.113756,86.14566886)(626.03875607,86.23066878)(626.01875732,86.36066895)
\curveto(626.0087561,86.49066852)(626.00375611,86.64066837)(626.00375732,86.81066895)
\lineto(626.00375732,88.55066895)
\lineto(626.00375732,94.55066895)
\lineto(626.00375732,95.96066895)
\curveto(626.00375611,96.07065894)(625.99875611,96.18565882)(625.98875732,96.30566895)
\curveto(625.98875612,96.42565858)(626.0137561,96.52065849)(626.06375732,96.59066895)
\curveto(626.10375601,96.65065836)(626.17875593,96.70065831)(626.28875732,96.74066895)
\curveto(626.3087558,96.75065826)(626.32875578,96.75065826)(626.34875732,96.74066895)
\curveto(626.37875573,96.74065827)(626.40375571,96.74565826)(626.42375732,96.75566895)
}
}
{
\newrgbcolor{curcolor}{0 0 0}
\pscustom[linestyle=none,fillstyle=solid,fillcolor=curcolor]
{
\newpath
\moveto(645.8658667,90.27566895)
\curveto(645.88585864,90.21566479)(645.89585863,90.12066489)(645.8958667,89.99066895)
\curveto(645.89585863,89.87066514)(645.89085863,89.78566522)(645.8808667,89.73566895)
\lineto(645.8808667,89.58566895)
\curveto(645.87085865,89.5056655)(645.86085866,89.43066558)(645.8508667,89.36066895)
\curveto(645.85085867,89.30066571)(645.84585868,89.23066578)(645.8358667,89.15066895)
\curveto(645.81585871,89.09066592)(645.80085872,89.03066598)(645.7908667,88.97066895)
\curveto(645.79085873,88.9106661)(645.78085874,88.85066616)(645.7608667,88.79066895)
\curveto(645.7208588,88.66066635)(645.68585884,88.53066648)(645.6558667,88.40066895)
\curveto(645.6258589,88.27066674)(645.58585894,88.15066686)(645.5358667,88.04066895)
\curveto(645.3258592,87.56066745)(645.04585948,87.15566785)(644.6958667,86.82566895)
\curveto(644.34586018,86.5056685)(643.91586061,86.26066875)(643.4058667,86.09066895)
\curveto(643.29586123,86.05066896)(643.17586135,86.02066899)(643.0458667,86.00066895)
\curveto(642.9258616,85.98066903)(642.80086172,85.96066905)(642.6708667,85.94066895)
\curveto(642.61086191,85.93066908)(642.54586198,85.92566908)(642.4758667,85.92566895)
\curveto(642.41586211,85.91566909)(642.35586217,85.9106691)(642.2958667,85.91066895)
\curveto(642.25586227,85.90066911)(642.19586233,85.89566911)(642.1158667,85.89566895)
\curveto(642.04586248,85.89566911)(641.99586253,85.90066911)(641.9658667,85.91066895)
\curveto(641.9258626,85.92066909)(641.88586264,85.92566908)(641.8458667,85.92566895)
\curveto(641.80586272,85.91566909)(641.77086275,85.91566909)(641.7408667,85.92566895)
\lineto(641.6508667,85.92566895)
\lineto(641.2908667,85.97066895)
\curveto(641.15086337,86.010669)(641.01586351,86.05066896)(640.8858667,86.09066895)
\curveto(640.75586377,86.13066888)(640.63086389,86.17566883)(640.5108667,86.22566895)
\curveto(640.06086446,86.42566858)(639.69086483,86.68566832)(639.4008667,87.00566895)
\curveto(639.11086541,87.32566768)(638.87086565,87.71566729)(638.6808667,88.17566895)
\curveto(638.63086589,88.27566673)(638.59086593,88.37566663)(638.5608667,88.47566895)
\curveto(638.54086598,88.57566643)(638.520866,88.68066633)(638.5008667,88.79066895)
\curveto(638.48086604,88.83066618)(638.47086605,88.86066615)(638.4708667,88.88066895)
\curveto(638.48086604,88.9106661)(638.48086604,88.94566606)(638.4708667,88.98566895)
\curveto(638.45086607,89.06566594)(638.43586609,89.14566586)(638.4258667,89.22566895)
\curveto(638.4258661,89.31566569)(638.41586611,89.40066561)(638.3958667,89.48066895)
\lineto(638.3958667,89.60066895)
\curveto(638.39586613,89.64066537)(638.39086613,89.68566532)(638.3808667,89.73566895)
\curveto(638.37086615,89.78566522)(638.36586616,89.87066514)(638.3658667,89.99066895)
\curveto(638.36586616,90.12066489)(638.37586615,90.21566479)(638.3958667,90.27566895)
\curveto(638.41586611,90.34566466)(638.4208661,90.41566459)(638.4108667,90.48566895)
\curveto(638.40086612,90.55566445)(638.40586612,90.62566438)(638.4258667,90.69566895)
\curveto(638.43586609,90.74566426)(638.44086608,90.78566422)(638.4408667,90.81566895)
\curveto(638.45086607,90.85566415)(638.46086606,90.90066411)(638.4708667,90.95066895)
\curveto(638.50086602,91.07066394)(638.525866,91.19066382)(638.5458667,91.31066895)
\curveto(638.57586595,91.43066358)(638.61586591,91.54566346)(638.6658667,91.65566895)
\curveto(638.81586571,92.02566298)(638.99586553,92.35566265)(639.2058667,92.64566895)
\curveto(639.4258651,92.94566206)(639.69086483,93.19566181)(640.0008667,93.39566895)
\curveto(640.1208644,93.47566153)(640.24586428,93.54066147)(640.3758667,93.59066895)
\curveto(640.50586402,93.65066136)(640.64086388,93.7106613)(640.7808667,93.77066895)
\curveto(640.90086362,93.82066119)(641.03086349,93.85066116)(641.1708667,93.86066895)
\curveto(641.31086321,93.88066113)(641.45086307,93.9106611)(641.5908667,93.95066895)
\lineto(641.7858667,93.95066895)
\curveto(641.85586267,93.96066105)(641.9208626,93.97066104)(641.9808667,93.98066895)
\curveto(642.87086165,93.99066102)(643.61086091,93.8056612)(644.2008667,93.42566895)
\curveto(644.79085973,93.04566196)(645.21585931,92.55066246)(645.4758667,91.94066895)
\curveto(645.525859,91.84066317)(645.56585896,91.74066327)(645.5958667,91.64066895)
\curveto(645.6258589,91.54066347)(645.66085886,91.43566357)(645.7008667,91.32566895)
\curveto(645.73085879,91.21566379)(645.75585877,91.09566391)(645.7758667,90.96566895)
\curveto(645.79585873,90.84566416)(645.8208587,90.72066429)(645.8508667,90.59066895)
\curveto(645.86085866,90.54066447)(645.86085866,90.48566452)(645.8508667,90.42566895)
\curveto(645.85085867,90.37566463)(645.85585867,90.32566468)(645.8658667,90.27566895)
\moveto(644.5308667,89.42066895)
\curveto(644.55085997,89.49066552)(644.55585997,89.57066544)(644.5458667,89.66066895)
\lineto(644.5458667,89.91566895)
\curveto(644.54585998,90.3056647)(644.51086001,90.63566437)(644.4408667,90.90566895)
\curveto(644.41086011,90.98566402)(644.38586014,91.06566394)(644.3658667,91.14566895)
\curveto(644.34586018,91.22566378)(644.3208602,91.30066371)(644.2908667,91.37066895)
\curveto(644.01086051,92.02066299)(643.56586096,92.47066254)(642.9558667,92.72066895)
\curveto(642.88586164,92.75066226)(642.81086171,92.77066224)(642.7308667,92.78066895)
\lineto(642.4908667,92.84066895)
\curveto(642.41086211,92.86066215)(642.3258622,92.87066214)(642.2358667,92.87066895)
\lineto(641.9658667,92.87066895)
\lineto(641.6958667,92.82566895)
\curveto(641.59586293,92.8056622)(641.50086302,92.78066223)(641.4108667,92.75066895)
\curveto(641.33086319,92.73066228)(641.25086327,92.70066231)(641.1708667,92.66066895)
\curveto(641.10086342,92.64066237)(641.03586349,92.6106624)(640.9758667,92.57066895)
\curveto(640.91586361,92.53066248)(640.86086366,92.49066252)(640.8108667,92.45066895)
\curveto(640.57086395,92.28066273)(640.37586415,92.07566293)(640.2258667,91.83566895)
\curveto(640.07586445,91.59566341)(639.94586458,91.31566369)(639.8358667,90.99566895)
\curveto(639.80586472,90.89566411)(639.78586474,90.79066422)(639.7758667,90.68066895)
\curveto(639.76586476,90.58066443)(639.75086477,90.47566453)(639.7308667,90.36566895)
\curveto(639.7208648,90.32566468)(639.71586481,90.26066475)(639.7158667,90.17066895)
\curveto(639.70586482,90.14066487)(639.70086482,90.1056649)(639.7008667,90.06566895)
\curveto(639.71086481,90.02566498)(639.71586481,89.98066503)(639.7158667,89.93066895)
\lineto(639.7158667,89.63066895)
\curveto(639.71586481,89.53066548)(639.7258648,89.44066557)(639.7458667,89.36066895)
\lineto(639.7758667,89.18066895)
\curveto(639.79586473,89.08066593)(639.81086471,88.98066603)(639.8208667,88.88066895)
\curveto(639.84086468,88.79066622)(639.87086465,88.7056663)(639.9108667,88.62566895)
\curveto(640.01086451,88.38566662)(640.1258644,88.16066685)(640.2558667,87.95066895)
\curveto(640.39586413,87.74066727)(640.56586396,87.56566744)(640.7658667,87.42566895)
\curveto(640.81586371,87.39566761)(640.86086366,87.37066764)(640.9008667,87.35066895)
\curveto(640.94086358,87.33066768)(640.98586354,87.3056677)(641.0358667,87.27566895)
\curveto(641.11586341,87.22566778)(641.20086332,87.18066783)(641.2908667,87.14066895)
\curveto(641.39086313,87.1106679)(641.49586303,87.08066793)(641.6058667,87.05066895)
\curveto(641.65586287,87.03066798)(641.70086282,87.02066799)(641.7408667,87.02066895)
\curveto(641.79086273,87.03066798)(641.84086268,87.03066798)(641.8908667,87.02066895)
\curveto(641.9208626,87.010668)(641.98086254,87.00066801)(642.0708667,86.99066895)
\curveto(642.17086235,86.98066803)(642.24586228,86.98566802)(642.2958667,87.00566895)
\curveto(642.33586219,87.01566799)(642.37586215,87.01566799)(642.4158667,87.00566895)
\curveto(642.45586207,87.005668)(642.49586203,87.01566799)(642.5358667,87.03566895)
\curveto(642.61586191,87.05566795)(642.69586183,87.07066794)(642.7758667,87.08066895)
\curveto(642.85586167,87.10066791)(642.93086159,87.12566788)(643.0008667,87.15566895)
\curveto(643.34086118,87.29566771)(643.61586091,87.49066752)(643.8258667,87.74066895)
\curveto(644.03586049,87.99066702)(644.21086031,88.28566672)(644.3508667,88.62566895)
\curveto(644.40086012,88.74566626)(644.43086009,88.87066614)(644.4408667,89.00066895)
\curveto(644.46086006,89.14066587)(644.49086003,89.28066573)(644.5308667,89.42066895)
}
}
{
\newrgbcolor{curcolor}{0 0 0}
\pscustom[linestyle=none,fillstyle=solid,fillcolor=curcolor]
{
\newpath
\moveto(654.31414795,86.88566895)
\lineto(654.31414795,86.49566895)
\curveto(654.31414007,86.37566863)(654.2891401,86.27566873)(654.23914795,86.19566895)
\curveto(654.1891402,86.12566888)(654.10414028,86.08566892)(653.98414795,86.07566895)
\lineto(653.63914795,86.07566895)
\curveto(653.57914081,86.07566893)(653.51914087,86.07066894)(653.45914795,86.06066895)
\curveto(653.40914098,86.06066895)(653.36414102,86.07066894)(653.32414795,86.09066895)
\curveto(653.23414115,86.1106689)(653.17414121,86.15066886)(653.14414795,86.21066895)
\curveto(653.10414128,86.26066875)(653.07914131,86.32066869)(653.06914795,86.39066895)
\curveto(653.06914132,86.46066855)(653.05414133,86.53066848)(653.02414795,86.60066895)
\curveto(653.01414137,86.62066839)(652.99914139,86.63566837)(652.97914795,86.64566895)
\curveto(652.96914142,86.66566834)(652.95414143,86.68566832)(652.93414795,86.70566895)
\curveto(652.83414155,86.71566829)(652.75414163,86.69566831)(652.69414795,86.64566895)
\curveto(652.64414174,86.59566841)(652.5891418,86.54566846)(652.52914795,86.49566895)
\curveto(652.32914206,86.34566866)(652.12914226,86.23066878)(651.92914795,86.15066895)
\curveto(651.74914264,86.07066894)(651.53914285,86.010669)(651.29914795,85.97066895)
\curveto(651.06914332,85.93066908)(650.82914356,85.9106691)(650.57914795,85.91066895)
\curveto(650.33914405,85.90066911)(650.09914429,85.91566909)(649.85914795,85.95566895)
\curveto(649.61914477,85.98566902)(649.40914498,86.04066897)(649.22914795,86.12066895)
\curveto(648.70914568,86.34066867)(648.2891461,86.63566837)(647.96914795,87.00566895)
\curveto(647.64914674,87.38566762)(647.39914699,87.85566715)(647.21914795,88.41566895)
\curveto(647.17914721,88.5056665)(647.14914724,88.59566641)(647.12914795,88.68566895)
\curveto(647.11914727,88.78566622)(647.09914729,88.88566612)(647.06914795,88.98566895)
\curveto(647.05914733,89.03566597)(647.05414733,89.08566592)(647.05414795,89.13566895)
\curveto(647.05414733,89.18566582)(647.04914734,89.23566577)(647.03914795,89.28566895)
\curveto(647.01914737,89.33566567)(647.00914738,89.38566562)(647.00914795,89.43566895)
\curveto(647.01914737,89.49566551)(647.01914737,89.55066546)(647.00914795,89.60066895)
\lineto(647.00914795,89.75066895)
\curveto(646.9891474,89.80066521)(646.97914741,89.86566514)(646.97914795,89.94566895)
\curveto(646.97914741,90.02566498)(646.9891474,90.09066492)(647.00914795,90.14066895)
\lineto(647.00914795,90.30566895)
\curveto(647.02914736,90.37566463)(647.03414735,90.44566456)(647.02414795,90.51566895)
\curveto(647.02414736,90.59566441)(647.03414735,90.67066434)(647.05414795,90.74066895)
\curveto(647.06414732,90.79066422)(647.06914732,90.83566417)(647.06914795,90.87566895)
\curveto(647.06914732,90.91566409)(647.07414731,90.96066405)(647.08414795,91.01066895)
\curveto(647.11414727,91.1106639)(647.13914725,91.2056638)(647.15914795,91.29566895)
\curveto(647.17914721,91.39566361)(647.20414718,91.49066352)(647.23414795,91.58066895)
\curveto(647.36414702,91.96066305)(647.52914686,92.30066271)(647.72914795,92.60066895)
\curveto(647.93914645,92.9106621)(648.1891462,93.16566184)(648.47914795,93.36566895)
\curveto(648.64914574,93.48566152)(648.82414556,93.58566142)(649.00414795,93.66566895)
\curveto(649.19414519,93.74566126)(649.39914499,93.81566119)(649.61914795,93.87566895)
\curveto(649.6891447,93.88566112)(649.75414463,93.89566111)(649.81414795,93.90566895)
\curveto(649.8841445,93.91566109)(649.95414443,93.93066108)(650.02414795,93.95066895)
\lineto(650.17414795,93.95066895)
\curveto(650.25414413,93.97066104)(650.36914402,93.98066103)(650.51914795,93.98066895)
\curveto(650.67914371,93.98066103)(650.79914359,93.97066104)(650.87914795,93.95066895)
\curveto(650.91914347,93.94066107)(650.97414341,93.93566107)(651.04414795,93.93566895)
\curveto(651.15414323,93.9056611)(651.26414312,93.88066113)(651.37414795,93.86066895)
\curveto(651.4841429,93.85066116)(651.5891428,93.82066119)(651.68914795,93.77066895)
\curveto(651.83914255,93.7106613)(651.97914241,93.64566136)(652.10914795,93.57566895)
\curveto(652.24914214,93.5056615)(652.37914201,93.42566158)(652.49914795,93.33566895)
\curveto(652.55914183,93.28566172)(652.61914177,93.23066178)(652.67914795,93.17066895)
\curveto(652.74914164,93.12066189)(652.83914155,93.1056619)(652.94914795,93.12566895)
\curveto(652.96914142,93.15566185)(652.9841414,93.18066183)(652.99414795,93.20066895)
\curveto(653.01414137,93.22066179)(653.02914136,93.25066176)(653.03914795,93.29066895)
\curveto(653.06914132,93.38066163)(653.07914131,93.49566151)(653.06914795,93.63566895)
\lineto(653.06914795,94.01066895)
\lineto(653.06914795,95.73566895)
\lineto(653.06914795,96.20066895)
\curveto(653.06914132,96.38065863)(653.09414129,96.5106585)(653.14414795,96.59066895)
\curveto(653.1841412,96.66065835)(653.24414114,96.7056583)(653.32414795,96.72566895)
\curveto(653.34414104,96.72565828)(653.36914102,96.72565828)(653.39914795,96.72566895)
\curveto(653.42914096,96.73565827)(653.45414093,96.74065827)(653.47414795,96.74066895)
\curveto(653.61414077,96.75065826)(653.75914063,96.75065826)(653.90914795,96.74066895)
\curveto(654.06914032,96.74065827)(654.17914021,96.70065831)(654.23914795,96.62066895)
\curveto(654.2891401,96.54065847)(654.31414007,96.44065857)(654.31414795,96.32066895)
\lineto(654.31414795,95.94566895)
\lineto(654.31414795,86.88566895)
\moveto(653.09914795,89.72066895)
\curveto(653.11914127,89.77066524)(653.12914126,89.83566517)(653.12914795,89.91566895)
\curveto(653.12914126,90.005665)(653.11914127,90.07566493)(653.09914795,90.12566895)
\lineto(653.09914795,90.35066895)
\curveto(653.07914131,90.44066457)(653.06414132,90.53066448)(653.05414795,90.62066895)
\curveto(653.04414134,90.72066429)(653.02414136,90.8106642)(652.99414795,90.89066895)
\curveto(652.97414141,90.97066404)(652.95414143,91.04566396)(652.93414795,91.11566895)
\curveto(652.92414146,91.18566382)(652.90414148,91.25566375)(652.87414795,91.32566895)
\curveto(652.75414163,91.62566338)(652.59914179,91.89066312)(652.40914795,92.12066895)
\curveto(652.21914217,92.35066266)(651.97914241,92.53066248)(651.68914795,92.66066895)
\curveto(651.5891428,92.7106623)(651.4841429,92.74566226)(651.37414795,92.76566895)
\curveto(651.27414311,92.79566221)(651.16414322,92.82066219)(651.04414795,92.84066895)
\curveto(650.96414342,92.86066215)(650.87414351,92.87066214)(650.77414795,92.87066895)
\lineto(650.50414795,92.87066895)
\curveto(650.45414393,92.86066215)(650.40914398,92.85066216)(650.36914795,92.84066895)
\lineto(650.23414795,92.84066895)
\curveto(650.15414423,92.82066219)(650.06914432,92.80066221)(649.97914795,92.78066895)
\curveto(649.89914449,92.76066225)(649.81914457,92.73566227)(649.73914795,92.70566895)
\curveto(649.41914497,92.56566244)(649.15914523,92.36066265)(648.95914795,92.09066895)
\curveto(648.76914562,91.83066318)(648.61414577,91.52566348)(648.49414795,91.17566895)
\curveto(648.45414593,91.06566394)(648.42414596,90.95066406)(648.40414795,90.83066895)
\curveto(648.39414599,90.72066429)(648.37914601,90.6106644)(648.35914795,90.50066895)
\curveto(648.35914603,90.46066455)(648.35414603,90.42066459)(648.34414795,90.38066895)
\lineto(648.34414795,90.27566895)
\curveto(648.32414606,90.22566478)(648.31414607,90.17066484)(648.31414795,90.11066895)
\curveto(648.32414606,90.05066496)(648.32914606,89.99566501)(648.32914795,89.94566895)
\lineto(648.32914795,89.61566895)
\curveto(648.32914606,89.51566549)(648.33914605,89.42066559)(648.35914795,89.33066895)
\curveto(648.36914602,89.30066571)(648.37414601,89.25066576)(648.37414795,89.18066895)
\curveto(648.39414599,89.1106659)(648.40914598,89.04066597)(648.41914795,88.97066895)
\lineto(648.47914795,88.76066895)
\curveto(648.5891458,88.4106666)(648.73914565,88.1106669)(648.92914795,87.86066895)
\curveto(649.11914527,87.6106674)(649.35914503,87.4056676)(649.64914795,87.24566895)
\curveto(649.73914465,87.19566781)(649.82914456,87.15566785)(649.91914795,87.12566895)
\curveto(650.00914438,87.09566791)(650.10914428,87.06566794)(650.21914795,87.03566895)
\curveto(650.26914412,87.01566799)(650.31914407,87.010668)(650.36914795,87.02066895)
\curveto(650.42914396,87.03066798)(650.4841439,87.02566798)(650.53414795,87.00566895)
\curveto(650.57414381,86.99566801)(650.61414377,86.99066802)(650.65414795,86.99066895)
\lineto(650.78914795,86.99066895)
\lineto(650.92414795,86.99066895)
\curveto(650.95414343,87.00066801)(651.00414338,87.005668)(651.07414795,87.00566895)
\curveto(651.15414323,87.02566798)(651.23414315,87.04066797)(651.31414795,87.05066895)
\curveto(651.39414299,87.07066794)(651.46914292,87.09566791)(651.53914795,87.12566895)
\curveto(651.86914252,87.26566774)(652.13414225,87.44066757)(652.33414795,87.65066895)
\curveto(652.54414184,87.87066714)(652.71914167,88.14566686)(652.85914795,88.47566895)
\curveto(652.90914148,88.58566642)(652.94414144,88.69566631)(652.96414795,88.80566895)
\curveto(652.9841414,88.91566609)(653.00914138,89.02566598)(653.03914795,89.13566895)
\curveto(653.05914133,89.17566583)(653.06914132,89.2106658)(653.06914795,89.24066895)
\curveto(653.06914132,89.28066573)(653.07414131,89.32066569)(653.08414795,89.36066895)
\curveto(653.09414129,89.42066559)(653.09414129,89.48066553)(653.08414795,89.54066895)
\curveto(653.0841413,89.60066541)(653.0891413,89.66066535)(653.09914795,89.72066895)
}
}
{
\newrgbcolor{curcolor}{0 0 0}
\pscustom[linestyle=none,fillstyle=solid,fillcolor=curcolor]
{
\newpath
\moveto(663.01039795,90.24566895)
\curveto(663.03039026,90.14566486)(663.03039026,90.03066498)(663.01039795,89.90066895)
\curveto(663.00039029,89.78066523)(662.97039032,89.69566531)(662.92039795,89.64566895)
\curveto(662.87039042,89.6056654)(662.7953905,89.57566543)(662.69539795,89.55566895)
\curveto(662.60539069,89.54566546)(662.50039079,89.54066547)(662.38039795,89.54066895)
\lineto(662.02039795,89.54066895)
\curveto(661.90039139,89.55066546)(661.7953915,89.55566545)(661.70539795,89.55566895)
\lineto(657.86539795,89.55566895)
\curveto(657.78539551,89.55566545)(657.70539559,89.55066546)(657.62539795,89.54066895)
\curveto(657.54539575,89.54066547)(657.48039581,89.52566548)(657.43039795,89.49566895)
\curveto(657.3903959,89.47566553)(657.35039594,89.43566557)(657.31039795,89.37566895)
\curveto(657.290396,89.34566566)(657.27039602,89.30066571)(657.25039795,89.24066895)
\curveto(657.23039606,89.19066582)(657.23039606,89.14066587)(657.25039795,89.09066895)
\curveto(657.26039603,89.04066597)(657.26539603,88.99566601)(657.26539795,88.95566895)
\curveto(657.26539603,88.91566609)(657.27039602,88.87566613)(657.28039795,88.83566895)
\curveto(657.30039599,88.75566625)(657.32039597,88.67066634)(657.34039795,88.58066895)
\curveto(657.36039593,88.50066651)(657.3903959,88.42066659)(657.43039795,88.34066895)
\curveto(657.66039563,87.80066721)(658.04039525,87.41566759)(658.57039795,87.18566895)
\curveto(658.63039466,87.15566785)(658.6953946,87.13066788)(658.76539795,87.11066895)
\lineto(658.97539795,87.05066895)
\curveto(659.00539429,87.04066797)(659.05539424,87.03566797)(659.12539795,87.03566895)
\curveto(659.26539403,86.99566801)(659.45039384,86.97566803)(659.68039795,86.97566895)
\curveto(659.91039338,86.97566803)(660.0953932,86.99566801)(660.23539795,87.03566895)
\curveto(660.37539292,87.07566793)(660.50039279,87.11566789)(660.61039795,87.15566895)
\curveto(660.73039256,87.2056678)(660.84039245,87.26566774)(660.94039795,87.33566895)
\curveto(661.05039224,87.4056676)(661.14539215,87.48566752)(661.22539795,87.57566895)
\curveto(661.30539199,87.67566733)(661.37539192,87.78066723)(661.43539795,87.89066895)
\curveto(661.4953918,87.99066702)(661.54539175,88.09566691)(661.58539795,88.20566895)
\curveto(661.63539166,88.31566669)(661.71539158,88.39566661)(661.82539795,88.44566895)
\curveto(661.86539143,88.46566654)(661.93039136,88.48066653)(662.02039795,88.49066895)
\curveto(662.11039118,88.50066651)(662.20039109,88.50066651)(662.29039795,88.49066895)
\curveto(662.38039091,88.49066652)(662.46539083,88.48566652)(662.54539795,88.47566895)
\curveto(662.62539067,88.46566654)(662.68039061,88.44566656)(662.71039795,88.41566895)
\curveto(662.81039048,88.34566666)(662.83539046,88.23066678)(662.78539795,88.07066895)
\curveto(662.70539059,87.80066721)(662.60039069,87.56066745)(662.47039795,87.35066895)
\curveto(662.27039102,87.03066798)(662.04039125,86.76566824)(661.78039795,86.55566895)
\curveto(661.53039176,86.35566865)(661.21039208,86.19066882)(660.82039795,86.06066895)
\curveto(660.72039257,86.02066899)(660.62039267,85.99566901)(660.52039795,85.98566895)
\curveto(660.42039287,85.96566904)(660.31539298,85.94566906)(660.20539795,85.92566895)
\curveto(660.15539314,85.91566909)(660.10539319,85.9106691)(660.05539795,85.91066895)
\curveto(660.01539328,85.9106691)(659.97039332,85.9056691)(659.92039795,85.89566895)
\lineto(659.77039795,85.89566895)
\curveto(659.72039357,85.88566912)(659.66039363,85.88066913)(659.59039795,85.88066895)
\curveto(659.53039376,85.88066913)(659.48039381,85.88566912)(659.44039795,85.89566895)
\lineto(659.30539795,85.89566895)
\curveto(659.25539404,85.9056691)(659.21039408,85.9106691)(659.17039795,85.91066895)
\curveto(659.13039416,85.9106691)(659.0903942,85.91566909)(659.05039795,85.92566895)
\curveto(659.00039429,85.93566907)(658.94539435,85.94566906)(658.88539795,85.95566895)
\curveto(658.82539447,85.95566905)(658.77039452,85.96066905)(658.72039795,85.97066895)
\curveto(658.63039466,85.99066902)(658.54039475,86.01566899)(658.45039795,86.04566895)
\curveto(658.36039493,86.06566894)(658.27539502,86.09066892)(658.19539795,86.12066895)
\curveto(658.15539514,86.14066887)(658.12039517,86.15066886)(658.09039795,86.15066895)
\curveto(658.06039523,86.16066885)(658.02539527,86.17566883)(657.98539795,86.19566895)
\curveto(657.83539546,86.26566874)(657.67539562,86.35066866)(657.50539795,86.45066895)
\curveto(657.21539608,86.64066837)(656.96539633,86.87066814)(656.75539795,87.14066895)
\curveto(656.55539674,87.42066759)(656.38539691,87.73066728)(656.24539795,88.07066895)
\curveto(656.1953971,88.18066683)(656.15539714,88.29566671)(656.12539795,88.41566895)
\curveto(656.10539719,88.53566647)(656.07539722,88.65566635)(656.03539795,88.77566895)
\curveto(656.02539727,88.81566619)(656.02039727,88.85066616)(656.02039795,88.88066895)
\curveto(656.02039727,88.9106661)(656.01539728,88.95066606)(656.00539795,89.00066895)
\curveto(655.98539731,89.08066593)(655.97039732,89.16566584)(655.96039795,89.25566895)
\curveto(655.95039734,89.34566566)(655.93539736,89.43566557)(655.91539795,89.52566895)
\lineto(655.91539795,89.73566895)
\curveto(655.90539739,89.77566523)(655.8953974,89.83066518)(655.88539795,89.90066895)
\curveto(655.88539741,89.98066503)(655.8903974,90.04566496)(655.90039795,90.09566895)
\lineto(655.90039795,90.26066895)
\curveto(655.92039737,90.3106647)(655.92539737,90.36066465)(655.91539795,90.41066895)
\curveto(655.91539738,90.47066454)(655.92039737,90.52566448)(655.93039795,90.57566895)
\curveto(655.97039732,90.73566427)(656.00039729,90.89566411)(656.02039795,91.05566895)
\curveto(656.05039724,91.21566379)(656.0953972,91.36566364)(656.15539795,91.50566895)
\curveto(656.20539709,91.61566339)(656.25039704,91.72566328)(656.29039795,91.83566895)
\curveto(656.34039695,91.95566305)(656.3953969,92.07066294)(656.45539795,92.18066895)
\curveto(656.67539662,92.53066248)(656.92539637,92.83066218)(657.20539795,93.08066895)
\curveto(657.48539581,93.34066167)(657.83039546,93.55566145)(658.24039795,93.72566895)
\curveto(658.36039493,93.77566123)(658.48039481,93.8106612)(658.60039795,93.83066895)
\curveto(658.73039456,93.86066115)(658.86539443,93.89066112)(659.00539795,93.92066895)
\curveto(659.05539424,93.93066108)(659.10039419,93.93566107)(659.14039795,93.93566895)
\curveto(659.18039411,93.94566106)(659.22539407,93.95066106)(659.27539795,93.95066895)
\curveto(659.295394,93.96066105)(659.32039397,93.96066105)(659.35039795,93.95066895)
\curveto(659.38039391,93.94066107)(659.40539389,93.94566106)(659.42539795,93.96566895)
\curveto(659.84539345,93.97566103)(660.21039308,93.93066108)(660.52039795,93.83066895)
\curveto(660.83039246,93.74066127)(661.11039218,93.61566139)(661.36039795,93.45566895)
\curveto(661.41039188,93.43566157)(661.45039184,93.4056616)(661.48039795,93.36566895)
\curveto(661.51039178,93.33566167)(661.54539175,93.3106617)(661.58539795,93.29066895)
\curveto(661.66539163,93.23066178)(661.74539155,93.16066185)(661.82539795,93.08066895)
\curveto(661.91539138,93.00066201)(661.9903913,92.92066209)(662.05039795,92.84066895)
\curveto(662.21039108,92.63066238)(662.34539095,92.43066258)(662.45539795,92.24066895)
\curveto(662.52539077,92.13066288)(662.58039071,92.010663)(662.62039795,91.88066895)
\curveto(662.66039063,91.75066326)(662.70539059,91.62066339)(662.75539795,91.49066895)
\curveto(662.80539049,91.36066365)(662.84039045,91.22566378)(662.86039795,91.08566895)
\curveto(662.8903904,90.94566406)(662.92539037,90.8056642)(662.96539795,90.66566895)
\curveto(662.97539032,90.59566441)(662.98039031,90.52566448)(662.98039795,90.45566895)
\lineto(663.01039795,90.24566895)
\moveto(661.55539795,90.75566895)
\curveto(661.58539171,90.79566421)(661.61039168,90.84566416)(661.63039795,90.90566895)
\curveto(661.65039164,90.97566403)(661.65039164,91.04566396)(661.63039795,91.11566895)
\curveto(661.57039172,91.33566367)(661.48539181,91.54066347)(661.37539795,91.73066895)
\curveto(661.23539206,91.96066305)(661.08039221,92.15566285)(660.91039795,92.31566895)
\curveto(660.74039255,92.47566253)(660.52039277,92.6106624)(660.25039795,92.72066895)
\curveto(660.18039311,92.74066227)(660.11039318,92.75566225)(660.04039795,92.76566895)
\curveto(659.97039332,92.78566222)(659.8953934,92.8056622)(659.81539795,92.82566895)
\curveto(659.73539356,92.84566216)(659.65039364,92.85566215)(659.56039795,92.85566895)
\lineto(659.30539795,92.85566895)
\curveto(659.27539402,92.83566217)(659.24039405,92.82566218)(659.20039795,92.82566895)
\curveto(659.16039413,92.83566217)(659.12539417,92.83566217)(659.09539795,92.82566895)
\lineto(658.85539795,92.76566895)
\curveto(658.78539451,92.75566225)(658.71539458,92.74066227)(658.64539795,92.72066895)
\curveto(658.35539494,92.60066241)(658.12039517,92.45066256)(657.94039795,92.27066895)
\curveto(657.77039552,92.09066292)(657.61539568,91.86566314)(657.47539795,91.59566895)
\curveto(657.44539585,91.54566346)(657.41539588,91.48066353)(657.38539795,91.40066895)
\curveto(657.35539594,91.33066368)(657.33039596,91.25066376)(657.31039795,91.16066895)
\curveto(657.290396,91.07066394)(657.28539601,90.98566402)(657.29539795,90.90566895)
\curveto(657.30539599,90.82566418)(657.34039595,90.76566424)(657.40039795,90.72566895)
\curveto(657.48039581,90.66566434)(657.61539568,90.63566437)(657.80539795,90.63566895)
\curveto(658.00539529,90.64566436)(658.17539512,90.65066436)(658.31539795,90.65066895)
\lineto(660.59539795,90.65066895)
\curveto(660.74539255,90.65066436)(660.92539237,90.64566436)(661.13539795,90.63566895)
\curveto(661.34539195,90.63566437)(661.48539181,90.67566433)(661.55539795,90.75566895)
}
}
{
\newrgbcolor{curcolor}{0 0 0}
\pscustom[linestyle=none,fillstyle=solid,fillcolor=curcolor]
{
\newpath
\moveto(667.96203857,93.98066895)
\curveto(668.19203378,93.98066103)(668.32203365,93.92066109)(668.35203857,93.80066895)
\curveto(668.38203359,93.69066132)(668.39703358,93.52566148)(668.39703857,93.30566895)
\lineto(668.39703857,93.02066895)
\curveto(668.39703358,92.93066208)(668.3720336,92.85566215)(668.32203857,92.79566895)
\curveto(668.26203371,92.71566229)(668.1770338,92.67066234)(668.06703857,92.66066895)
\curveto(667.95703402,92.66066235)(667.84703413,92.64566236)(667.73703857,92.61566895)
\curveto(667.59703438,92.58566242)(667.46203451,92.55566245)(667.33203857,92.52566895)
\curveto(667.21203476,92.49566251)(667.09703488,92.45566255)(666.98703857,92.40566895)
\curveto(666.69703528,92.27566273)(666.46203551,92.09566291)(666.28203857,91.86566895)
\curveto(666.10203587,91.64566336)(665.94703603,91.39066362)(665.81703857,91.10066895)
\curveto(665.7770362,90.99066402)(665.74703623,90.87566413)(665.72703857,90.75566895)
\curveto(665.70703627,90.64566436)(665.68203629,90.53066448)(665.65203857,90.41066895)
\curveto(665.64203633,90.36066465)(665.63703634,90.3106647)(665.63703857,90.26066895)
\curveto(665.64703633,90.2106648)(665.64703633,90.16066485)(665.63703857,90.11066895)
\curveto(665.60703637,89.99066502)(665.59203638,89.85066516)(665.59203857,89.69066895)
\curveto(665.60203637,89.54066547)(665.60703637,89.39566561)(665.60703857,89.25566895)
\lineto(665.60703857,87.41066895)
\lineto(665.60703857,87.06566895)
\curveto(665.60703637,86.94566806)(665.60203637,86.83066818)(665.59203857,86.72066895)
\curveto(665.58203639,86.6106684)(665.5770364,86.51566849)(665.57703857,86.43566895)
\curveto(665.58703639,86.35566865)(665.56703641,86.28566872)(665.51703857,86.22566895)
\curveto(665.46703651,86.15566885)(665.38703659,86.11566889)(665.27703857,86.10566895)
\curveto(665.1770368,86.09566891)(665.06703691,86.09066892)(664.94703857,86.09066895)
\lineto(664.67703857,86.09066895)
\curveto(664.62703735,86.1106689)(664.5770374,86.12566888)(664.52703857,86.13566895)
\curveto(664.48703749,86.15566885)(664.45703752,86.18066883)(664.43703857,86.21066895)
\curveto(664.38703759,86.28066873)(664.35703762,86.36566864)(664.34703857,86.46566895)
\lineto(664.34703857,86.79566895)
\lineto(664.34703857,87.95066895)
\lineto(664.34703857,92.10566895)
\lineto(664.34703857,93.14066895)
\lineto(664.34703857,93.44066895)
\curveto(664.35703762,93.54066147)(664.38703759,93.62566138)(664.43703857,93.69566895)
\curveto(664.46703751,93.73566127)(664.51703746,93.76566124)(664.58703857,93.78566895)
\curveto(664.66703731,93.8056612)(664.75203722,93.81566119)(664.84203857,93.81566895)
\curveto(664.93203704,93.82566118)(665.02203695,93.82566118)(665.11203857,93.81566895)
\curveto(665.20203677,93.8056612)(665.2720367,93.79066122)(665.32203857,93.77066895)
\curveto(665.40203657,93.74066127)(665.45203652,93.68066133)(665.47203857,93.59066895)
\curveto(665.50203647,93.5106615)(665.51703646,93.42066159)(665.51703857,93.32066895)
\lineto(665.51703857,93.02066895)
\curveto(665.51703646,92.92066209)(665.53703644,92.83066218)(665.57703857,92.75066895)
\curveto(665.58703639,92.73066228)(665.59703638,92.71566229)(665.60703857,92.70566895)
\lineto(665.65203857,92.66066895)
\curveto(665.76203621,92.66066235)(665.85203612,92.7056623)(665.92203857,92.79566895)
\curveto(665.99203598,92.89566211)(666.05203592,92.97566203)(666.10203857,93.03566895)
\lineto(666.19203857,93.12566895)
\curveto(666.28203569,93.23566177)(666.40703557,93.35066166)(666.56703857,93.47066895)
\curveto(666.72703525,93.59066142)(666.8770351,93.68066133)(667.01703857,93.74066895)
\curveto(667.10703487,93.79066122)(667.20203477,93.82566118)(667.30203857,93.84566895)
\curveto(667.40203457,93.87566113)(667.50703447,93.9056611)(667.61703857,93.93566895)
\curveto(667.6770343,93.94566106)(667.73703424,93.95066106)(667.79703857,93.95066895)
\curveto(667.85703412,93.96066105)(667.91203406,93.97066104)(667.96203857,93.98066895)
}
}
{
\newrgbcolor{curcolor}{0 0 0}
\pscustom[linestyle=none,fillstyle=solid,fillcolor=curcolor]
{
\newpath
\moveto(676.2118042,86.63066895)
\curveto(676.24179637,86.47066854)(676.22679638,86.33566867)(676.1668042,86.22566895)
\curveto(676.1067965,86.12566888)(676.02679658,86.05066896)(675.9268042,86.00066895)
\curveto(675.87679673,85.98066903)(675.82179679,85.97066904)(675.7618042,85.97066895)
\curveto(675.7117969,85.97066904)(675.65679695,85.96066905)(675.5968042,85.94066895)
\curveto(675.37679723,85.89066912)(675.15679745,85.9056691)(674.9368042,85.98566895)
\curveto(674.72679788,86.05566895)(674.58179803,86.14566886)(674.5018042,86.25566895)
\curveto(674.45179816,86.32566868)(674.4067982,86.4056686)(674.3668042,86.49566895)
\curveto(674.32679828,86.59566841)(674.27679833,86.67566833)(674.2168042,86.73566895)
\curveto(674.19679841,86.75566825)(674.17179844,86.77566823)(674.1418042,86.79566895)
\curveto(674.12179849,86.81566819)(674.09179852,86.82066819)(674.0518042,86.81066895)
\curveto(673.94179867,86.78066823)(673.83679877,86.72566828)(673.7368042,86.64566895)
\curveto(673.64679896,86.56566844)(673.55679905,86.49566851)(673.4668042,86.43566895)
\curveto(673.33679927,86.35566865)(673.19679941,86.28066873)(673.0468042,86.21066895)
\curveto(672.89679971,86.15066886)(672.73679987,86.09566891)(672.5668042,86.04566895)
\curveto(672.46680014,86.01566899)(672.35680025,85.99566901)(672.2368042,85.98566895)
\curveto(672.12680048,85.97566903)(672.01680059,85.96066905)(671.9068042,85.94066895)
\curveto(671.85680075,85.93066908)(671.8118008,85.92566908)(671.7718042,85.92566895)
\lineto(671.6668042,85.92566895)
\curveto(671.55680105,85.9056691)(671.45180116,85.9056691)(671.3518042,85.92566895)
\lineto(671.2168042,85.92566895)
\curveto(671.16680144,85.93566907)(671.11680149,85.94066907)(671.0668042,85.94066895)
\curveto(671.01680159,85.94066907)(670.97180164,85.95066906)(670.9318042,85.97066895)
\curveto(670.89180172,85.98066903)(670.85680175,85.98566902)(670.8268042,85.98566895)
\curveto(670.8068018,85.97566903)(670.78180183,85.97566903)(670.7518042,85.98566895)
\lineto(670.5118042,86.04566895)
\curveto(670.43180218,86.05566895)(670.35680225,86.07566893)(670.2868042,86.10566895)
\curveto(669.98680262,86.23566877)(669.74180287,86.38066863)(669.5518042,86.54066895)
\curveto(669.37180324,86.7106683)(669.22180339,86.94566806)(669.1018042,87.24566895)
\curveto(669.0118036,87.46566754)(668.96680364,87.73066728)(668.9668042,88.04066895)
\lineto(668.9668042,88.35566895)
\curveto(668.97680363,88.4056666)(668.98180363,88.45566655)(668.9818042,88.50566895)
\lineto(669.0118042,88.68566895)
\lineto(669.1318042,89.01566895)
\curveto(669.17180344,89.12566588)(669.22180339,89.22566578)(669.2818042,89.31566895)
\curveto(669.46180315,89.6056654)(669.7068029,89.82066519)(670.0168042,89.96066895)
\curveto(670.32680228,90.10066491)(670.66680194,90.22566478)(671.0368042,90.33566895)
\curveto(671.17680143,90.37566463)(671.32180129,90.4056646)(671.4718042,90.42566895)
\curveto(671.62180099,90.44566456)(671.77180084,90.47066454)(671.9218042,90.50066895)
\curveto(671.99180062,90.52066449)(672.05680055,90.53066448)(672.1168042,90.53066895)
\curveto(672.18680042,90.53066448)(672.26180035,90.54066447)(672.3418042,90.56066895)
\curveto(672.4118002,90.58066443)(672.48180013,90.59066442)(672.5518042,90.59066895)
\curveto(672.62179999,90.60066441)(672.69679991,90.61566439)(672.7768042,90.63566895)
\curveto(673.02679958,90.69566431)(673.26179935,90.74566426)(673.4818042,90.78566895)
\curveto(673.70179891,90.83566417)(673.87679873,90.95066406)(674.0068042,91.13066895)
\curveto(674.06679854,91.2106638)(674.11679849,91.3106637)(674.1568042,91.43066895)
\curveto(674.19679841,91.56066345)(674.19679841,91.70066331)(674.1568042,91.85066895)
\curveto(674.09679851,92.09066292)(674.0067986,92.28066273)(673.8868042,92.42066895)
\curveto(673.77679883,92.56066245)(673.61679899,92.67066234)(673.4068042,92.75066895)
\curveto(673.28679932,92.80066221)(673.14179947,92.83566217)(672.9718042,92.85566895)
\curveto(672.8117998,92.87566213)(672.64179997,92.88566212)(672.4618042,92.88566895)
\curveto(672.28180033,92.88566212)(672.1068005,92.87566213)(671.9368042,92.85566895)
\curveto(671.76680084,92.83566217)(671.62180099,92.8056622)(671.5018042,92.76566895)
\curveto(671.33180128,92.7056623)(671.16680144,92.62066239)(671.0068042,92.51066895)
\curveto(670.92680168,92.45066256)(670.85180176,92.37066264)(670.7818042,92.27066895)
\curveto(670.72180189,92.18066283)(670.66680194,92.08066293)(670.6168042,91.97066895)
\curveto(670.58680202,91.89066312)(670.55680205,91.8056632)(670.5268042,91.71566895)
\curveto(670.5068021,91.62566338)(670.46180215,91.55566345)(670.3918042,91.50566895)
\curveto(670.35180226,91.47566353)(670.28180233,91.45066356)(670.1818042,91.43066895)
\curveto(670.09180252,91.42066359)(669.99680261,91.41566359)(669.8968042,91.41566895)
\curveto(669.79680281,91.41566359)(669.69680291,91.42066359)(669.5968042,91.43066895)
\curveto(669.5068031,91.45066356)(669.44180317,91.47566353)(669.4018042,91.50566895)
\curveto(669.36180325,91.53566347)(669.33180328,91.58566342)(669.3118042,91.65566895)
\curveto(669.29180332,91.72566328)(669.29180332,91.80066321)(669.3118042,91.88066895)
\curveto(669.34180327,92.010663)(669.37180324,92.13066288)(669.4018042,92.24066895)
\curveto(669.44180317,92.36066265)(669.48680312,92.47566253)(669.5368042,92.58566895)
\curveto(669.72680288,92.93566207)(669.96680264,93.2056618)(670.2568042,93.39566895)
\curveto(670.54680206,93.59566141)(670.9068017,93.75566125)(671.3368042,93.87566895)
\curveto(671.43680117,93.89566111)(671.53680107,93.9106611)(671.6368042,93.92066895)
\curveto(671.74680086,93.93066108)(671.85680075,93.94566106)(671.9668042,93.96566895)
\curveto(672.0068006,93.97566103)(672.07180054,93.97566103)(672.1618042,93.96566895)
\curveto(672.25180036,93.96566104)(672.3068003,93.97566103)(672.3268042,93.99566895)
\curveto(673.02679958,94.005661)(673.63679897,93.92566108)(674.1568042,93.75566895)
\curveto(674.67679793,93.58566142)(675.04179757,93.26066175)(675.2518042,92.78066895)
\curveto(675.34179727,92.58066243)(675.39179722,92.34566266)(675.4018042,92.07566895)
\curveto(675.42179719,91.81566319)(675.43179718,91.54066347)(675.4318042,91.25066895)
\lineto(675.4318042,87.93566895)
\curveto(675.43179718,87.79566721)(675.43679717,87.66066735)(675.4468042,87.53066895)
\curveto(675.45679715,87.40066761)(675.48679712,87.29566771)(675.5368042,87.21566895)
\curveto(675.58679702,87.14566786)(675.65179696,87.09566791)(675.7318042,87.06566895)
\curveto(675.82179679,87.02566798)(675.9067967,86.99566801)(675.9868042,86.97566895)
\curveto(676.06679654,86.96566804)(676.12679648,86.92066809)(676.1668042,86.84066895)
\curveto(676.18679642,86.8106682)(676.19679641,86.78066823)(676.1968042,86.75066895)
\curveto(676.19679641,86.72066829)(676.20179641,86.68066833)(676.2118042,86.63066895)
\moveto(674.0668042,88.29566895)
\curveto(674.12679848,88.43566657)(674.15679845,88.59566641)(674.1568042,88.77566895)
\curveto(674.16679844,88.96566604)(674.17179844,89.16066585)(674.1718042,89.36066895)
\curveto(674.17179844,89.47066554)(674.16679844,89.57066544)(674.1568042,89.66066895)
\curveto(674.14679846,89.75066526)(674.1067985,89.82066519)(674.0368042,89.87066895)
\curveto(674.0067986,89.89066512)(673.93679867,89.90066511)(673.8268042,89.90066895)
\curveto(673.8067988,89.88066513)(673.77179884,89.87066514)(673.7218042,89.87066895)
\curveto(673.67179894,89.87066514)(673.62679898,89.86066515)(673.5868042,89.84066895)
\curveto(673.5067991,89.82066519)(673.41679919,89.80066521)(673.3168042,89.78066895)
\lineto(673.0168042,89.72066895)
\curveto(672.98679962,89.72066529)(672.95179966,89.71566529)(672.9118042,89.70566895)
\lineto(672.8068042,89.70566895)
\curveto(672.65679995,89.66566534)(672.49180012,89.64066537)(672.3118042,89.63066895)
\curveto(672.14180047,89.63066538)(671.98180063,89.6106654)(671.8318042,89.57066895)
\curveto(671.75180086,89.55066546)(671.67680093,89.53066548)(671.6068042,89.51066895)
\curveto(671.54680106,89.50066551)(671.47680113,89.48566552)(671.3968042,89.46566895)
\curveto(671.23680137,89.41566559)(671.08680152,89.35066566)(670.9468042,89.27066895)
\curveto(670.8068018,89.20066581)(670.68680192,89.1106659)(670.5868042,89.00066895)
\curveto(670.48680212,88.89066612)(670.4118022,88.75566625)(670.3618042,88.59566895)
\curveto(670.3118023,88.44566656)(670.29180232,88.26066675)(670.3018042,88.04066895)
\curveto(670.30180231,87.94066707)(670.31680229,87.84566716)(670.3468042,87.75566895)
\curveto(670.38680222,87.67566733)(670.43180218,87.60066741)(670.4818042,87.53066895)
\curveto(670.56180205,87.42066759)(670.66680194,87.32566768)(670.7968042,87.24566895)
\curveto(670.92680168,87.17566783)(671.06680154,87.11566789)(671.2168042,87.06566895)
\curveto(671.26680134,87.05566795)(671.31680129,87.05066796)(671.3668042,87.05066895)
\curveto(671.41680119,87.05066796)(671.46680114,87.04566796)(671.5168042,87.03566895)
\curveto(671.58680102,87.01566799)(671.67180094,87.00066801)(671.7718042,86.99066895)
\curveto(671.88180073,86.99066802)(671.97180064,87.00066801)(672.0418042,87.02066895)
\curveto(672.10180051,87.04066797)(672.16180045,87.04566796)(672.2218042,87.03566895)
\curveto(672.28180033,87.03566797)(672.34180027,87.04566796)(672.4018042,87.06566895)
\curveto(672.48180013,87.08566792)(672.55680005,87.10066791)(672.6268042,87.11066895)
\curveto(672.7067999,87.12066789)(672.78179983,87.14066787)(672.8518042,87.17066895)
\curveto(673.14179947,87.29066772)(673.38679922,87.43566757)(673.5868042,87.60566895)
\curveto(673.79679881,87.77566723)(673.95679865,88.005667)(674.0668042,88.29566895)
}
}
{
\newrgbcolor{curcolor}{0 0 0}
\pscustom[linestyle=none,fillstyle=solid,fillcolor=curcolor]
{
\newpath
\moveto(684.34344482,86.88566895)
\lineto(684.34344482,86.49566895)
\curveto(684.34343695,86.37566863)(684.31843697,86.27566873)(684.26844482,86.19566895)
\curveto(684.21843707,86.12566888)(684.13343716,86.08566892)(684.01344482,86.07566895)
\lineto(683.66844482,86.07566895)
\curveto(683.60843768,86.07566893)(683.54843774,86.07066894)(683.48844482,86.06066895)
\curveto(683.43843785,86.06066895)(683.3934379,86.07066894)(683.35344482,86.09066895)
\curveto(683.26343803,86.1106689)(683.20343809,86.15066886)(683.17344482,86.21066895)
\curveto(683.13343816,86.26066875)(683.10843818,86.32066869)(683.09844482,86.39066895)
\curveto(683.09843819,86.46066855)(683.08343821,86.53066848)(683.05344482,86.60066895)
\curveto(683.04343825,86.62066839)(683.02843826,86.63566837)(683.00844482,86.64566895)
\curveto(682.99843829,86.66566834)(682.98343831,86.68566832)(682.96344482,86.70566895)
\curveto(682.86343843,86.71566829)(682.78343851,86.69566831)(682.72344482,86.64566895)
\curveto(682.67343862,86.59566841)(682.61843867,86.54566846)(682.55844482,86.49566895)
\curveto(682.35843893,86.34566866)(682.15843913,86.23066878)(681.95844482,86.15066895)
\curveto(681.77843951,86.07066894)(681.56843972,86.010669)(681.32844482,85.97066895)
\curveto(681.09844019,85.93066908)(680.85844043,85.9106691)(680.60844482,85.91066895)
\curveto(680.36844092,85.90066911)(680.12844116,85.91566909)(679.88844482,85.95566895)
\curveto(679.64844164,85.98566902)(679.43844185,86.04066897)(679.25844482,86.12066895)
\curveto(678.73844255,86.34066867)(678.31844297,86.63566837)(677.99844482,87.00566895)
\curveto(677.67844361,87.38566762)(677.42844386,87.85566715)(677.24844482,88.41566895)
\curveto(677.20844408,88.5056665)(677.17844411,88.59566641)(677.15844482,88.68566895)
\curveto(677.14844414,88.78566622)(677.12844416,88.88566612)(677.09844482,88.98566895)
\curveto(677.0884442,89.03566597)(677.08344421,89.08566592)(677.08344482,89.13566895)
\curveto(677.08344421,89.18566582)(677.07844421,89.23566577)(677.06844482,89.28566895)
\curveto(677.04844424,89.33566567)(677.03844425,89.38566562)(677.03844482,89.43566895)
\curveto(677.04844424,89.49566551)(677.04844424,89.55066546)(677.03844482,89.60066895)
\lineto(677.03844482,89.75066895)
\curveto(677.01844427,89.80066521)(677.00844428,89.86566514)(677.00844482,89.94566895)
\curveto(677.00844428,90.02566498)(677.01844427,90.09066492)(677.03844482,90.14066895)
\lineto(677.03844482,90.30566895)
\curveto(677.05844423,90.37566463)(677.06344423,90.44566456)(677.05344482,90.51566895)
\curveto(677.05344424,90.59566441)(677.06344423,90.67066434)(677.08344482,90.74066895)
\curveto(677.0934442,90.79066422)(677.09844419,90.83566417)(677.09844482,90.87566895)
\curveto(677.09844419,90.91566409)(677.10344419,90.96066405)(677.11344482,91.01066895)
\curveto(677.14344415,91.1106639)(677.16844412,91.2056638)(677.18844482,91.29566895)
\curveto(677.20844408,91.39566361)(677.23344406,91.49066352)(677.26344482,91.58066895)
\curveto(677.3934439,91.96066305)(677.55844373,92.30066271)(677.75844482,92.60066895)
\curveto(677.96844332,92.9106621)(678.21844307,93.16566184)(678.50844482,93.36566895)
\curveto(678.67844261,93.48566152)(678.85344244,93.58566142)(679.03344482,93.66566895)
\curveto(679.22344207,93.74566126)(679.42844186,93.81566119)(679.64844482,93.87566895)
\curveto(679.71844157,93.88566112)(679.78344151,93.89566111)(679.84344482,93.90566895)
\curveto(679.91344138,93.91566109)(679.98344131,93.93066108)(680.05344482,93.95066895)
\lineto(680.20344482,93.95066895)
\curveto(680.28344101,93.97066104)(680.39844089,93.98066103)(680.54844482,93.98066895)
\curveto(680.70844058,93.98066103)(680.82844046,93.97066104)(680.90844482,93.95066895)
\curveto(680.94844034,93.94066107)(681.00344029,93.93566107)(681.07344482,93.93566895)
\curveto(681.18344011,93.9056611)(681.29344,93.88066113)(681.40344482,93.86066895)
\curveto(681.51343978,93.85066116)(681.61843967,93.82066119)(681.71844482,93.77066895)
\curveto(681.86843942,93.7106613)(682.00843928,93.64566136)(682.13844482,93.57566895)
\curveto(682.27843901,93.5056615)(682.40843888,93.42566158)(682.52844482,93.33566895)
\curveto(682.5884387,93.28566172)(682.64843864,93.23066178)(682.70844482,93.17066895)
\curveto(682.77843851,93.12066189)(682.86843842,93.1056619)(682.97844482,93.12566895)
\curveto(682.99843829,93.15566185)(683.01343828,93.18066183)(683.02344482,93.20066895)
\curveto(683.04343825,93.22066179)(683.05843823,93.25066176)(683.06844482,93.29066895)
\curveto(683.09843819,93.38066163)(683.10843818,93.49566151)(683.09844482,93.63566895)
\lineto(683.09844482,94.01066895)
\lineto(683.09844482,95.73566895)
\lineto(683.09844482,96.20066895)
\curveto(683.09843819,96.38065863)(683.12343817,96.5106585)(683.17344482,96.59066895)
\curveto(683.21343808,96.66065835)(683.27343802,96.7056583)(683.35344482,96.72566895)
\curveto(683.37343792,96.72565828)(683.39843789,96.72565828)(683.42844482,96.72566895)
\curveto(683.45843783,96.73565827)(683.48343781,96.74065827)(683.50344482,96.74066895)
\curveto(683.64343765,96.75065826)(683.7884375,96.75065826)(683.93844482,96.74066895)
\curveto(684.09843719,96.74065827)(684.20843708,96.70065831)(684.26844482,96.62066895)
\curveto(684.31843697,96.54065847)(684.34343695,96.44065857)(684.34344482,96.32066895)
\lineto(684.34344482,95.94566895)
\lineto(684.34344482,86.88566895)
\moveto(683.12844482,89.72066895)
\curveto(683.14843814,89.77066524)(683.15843813,89.83566517)(683.15844482,89.91566895)
\curveto(683.15843813,90.005665)(683.14843814,90.07566493)(683.12844482,90.12566895)
\lineto(683.12844482,90.35066895)
\curveto(683.10843818,90.44066457)(683.0934382,90.53066448)(683.08344482,90.62066895)
\curveto(683.07343822,90.72066429)(683.05343824,90.8106642)(683.02344482,90.89066895)
\curveto(683.00343829,90.97066404)(682.98343831,91.04566396)(682.96344482,91.11566895)
\curveto(682.95343834,91.18566382)(682.93343836,91.25566375)(682.90344482,91.32566895)
\curveto(682.78343851,91.62566338)(682.62843866,91.89066312)(682.43844482,92.12066895)
\curveto(682.24843904,92.35066266)(682.00843928,92.53066248)(681.71844482,92.66066895)
\curveto(681.61843967,92.7106623)(681.51343978,92.74566226)(681.40344482,92.76566895)
\curveto(681.30343999,92.79566221)(681.1934401,92.82066219)(681.07344482,92.84066895)
\curveto(680.9934403,92.86066215)(680.90344039,92.87066214)(680.80344482,92.87066895)
\lineto(680.53344482,92.87066895)
\curveto(680.48344081,92.86066215)(680.43844085,92.85066216)(680.39844482,92.84066895)
\lineto(680.26344482,92.84066895)
\curveto(680.18344111,92.82066219)(680.09844119,92.80066221)(680.00844482,92.78066895)
\curveto(679.92844136,92.76066225)(679.84844144,92.73566227)(679.76844482,92.70566895)
\curveto(679.44844184,92.56566244)(679.1884421,92.36066265)(678.98844482,92.09066895)
\curveto(678.79844249,91.83066318)(678.64344265,91.52566348)(678.52344482,91.17566895)
\curveto(678.48344281,91.06566394)(678.45344284,90.95066406)(678.43344482,90.83066895)
\curveto(678.42344287,90.72066429)(678.40844288,90.6106644)(678.38844482,90.50066895)
\curveto(678.3884429,90.46066455)(678.38344291,90.42066459)(678.37344482,90.38066895)
\lineto(678.37344482,90.27566895)
\curveto(678.35344294,90.22566478)(678.34344295,90.17066484)(678.34344482,90.11066895)
\curveto(678.35344294,90.05066496)(678.35844293,89.99566501)(678.35844482,89.94566895)
\lineto(678.35844482,89.61566895)
\curveto(678.35844293,89.51566549)(678.36844292,89.42066559)(678.38844482,89.33066895)
\curveto(678.39844289,89.30066571)(678.40344289,89.25066576)(678.40344482,89.18066895)
\curveto(678.42344287,89.1106659)(678.43844285,89.04066597)(678.44844482,88.97066895)
\lineto(678.50844482,88.76066895)
\curveto(678.61844267,88.4106666)(678.76844252,88.1106669)(678.95844482,87.86066895)
\curveto(679.14844214,87.6106674)(679.3884419,87.4056676)(679.67844482,87.24566895)
\curveto(679.76844152,87.19566781)(679.85844143,87.15566785)(679.94844482,87.12566895)
\curveto(680.03844125,87.09566791)(680.13844115,87.06566794)(680.24844482,87.03566895)
\curveto(680.29844099,87.01566799)(680.34844094,87.010668)(680.39844482,87.02066895)
\curveto(680.45844083,87.03066798)(680.51344078,87.02566798)(680.56344482,87.00566895)
\curveto(680.60344069,86.99566801)(680.64344065,86.99066802)(680.68344482,86.99066895)
\lineto(680.81844482,86.99066895)
\lineto(680.95344482,86.99066895)
\curveto(680.98344031,87.00066801)(681.03344026,87.005668)(681.10344482,87.00566895)
\curveto(681.18344011,87.02566798)(681.26344003,87.04066797)(681.34344482,87.05066895)
\curveto(681.42343987,87.07066794)(681.49843979,87.09566791)(681.56844482,87.12566895)
\curveto(681.89843939,87.26566774)(682.16343913,87.44066757)(682.36344482,87.65066895)
\curveto(682.57343872,87.87066714)(682.74843854,88.14566686)(682.88844482,88.47566895)
\curveto(682.93843835,88.58566642)(682.97343832,88.69566631)(682.99344482,88.80566895)
\curveto(683.01343828,88.91566609)(683.03843825,89.02566598)(683.06844482,89.13566895)
\curveto(683.0884382,89.17566583)(683.09843819,89.2106658)(683.09844482,89.24066895)
\curveto(683.09843819,89.28066573)(683.10343819,89.32066569)(683.11344482,89.36066895)
\curveto(683.12343817,89.42066559)(683.12343817,89.48066553)(683.11344482,89.54066895)
\curveto(683.11343818,89.60066541)(683.11843817,89.66066535)(683.12844482,89.72066895)
}
}
{
\newrgbcolor{curcolor}{0 0 0}
\pscustom[linestyle=none,fillstyle=solid,fillcolor=curcolor]
{
\newpath
\moveto(693.41469482,90.27566895)
\curveto(693.43468676,90.21566479)(693.44468675,90.12066489)(693.44469482,89.99066895)
\curveto(693.44468675,89.87066514)(693.43968676,89.78566522)(693.42969482,89.73566895)
\lineto(693.42969482,89.58566895)
\curveto(693.41968678,89.5056655)(693.40968679,89.43066558)(693.39969482,89.36066895)
\curveto(693.3996868,89.30066571)(693.3946868,89.23066578)(693.38469482,89.15066895)
\curveto(693.36468683,89.09066592)(693.34968685,89.03066598)(693.33969482,88.97066895)
\curveto(693.33968686,88.9106661)(693.32968687,88.85066616)(693.30969482,88.79066895)
\curveto(693.26968693,88.66066635)(693.23468696,88.53066648)(693.20469482,88.40066895)
\curveto(693.17468702,88.27066674)(693.13468706,88.15066686)(693.08469482,88.04066895)
\curveto(692.87468732,87.56066745)(692.5946876,87.15566785)(692.24469482,86.82566895)
\curveto(691.8946883,86.5056685)(691.46468873,86.26066875)(690.95469482,86.09066895)
\curveto(690.84468935,86.05066896)(690.72468947,86.02066899)(690.59469482,86.00066895)
\curveto(690.47468972,85.98066903)(690.34968985,85.96066905)(690.21969482,85.94066895)
\curveto(690.15969004,85.93066908)(690.0946901,85.92566908)(690.02469482,85.92566895)
\curveto(689.96469023,85.91566909)(689.90469029,85.9106691)(689.84469482,85.91066895)
\curveto(689.80469039,85.90066911)(689.74469045,85.89566911)(689.66469482,85.89566895)
\curveto(689.5946906,85.89566911)(689.54469065,85.90066911)(689.51469482,85.91066895)
\curveto(689.47469072,85.92066909)(689.43469076,85.92566908)(689.39469482,85.92566895)
\curveto(689.35469084,85.91566909)(689.31969088,85.91566909)(689.28969482,85.92566895)
\lineto(689.19969482,85.92566895)
\lineto(688.83969482,85.97066895)
\curveto(688.6996915,86.010669)(688.56469163,86.05066896)(688.43469482,86.09066895)
\curveto(688.30469189,86.13066888)(688.17969202,86.17566883)(688.05969482,86.22566895)
\curveto(687.60969259,86.42566858)(687.23969296,86.68566832)(686.94969482,87.00566895)
\curveto(686.65969354,87.32566768)(686.41969378,87.71566729)(686.22969482,88.17566895)
\curveto(686.17969402,88.27566673)(686.13969406,88.37566663)(686.10969482,88.47566895)
\curveto(686.08969411,88.57566643)(686.06969413,88.68066633)(686.04969482,88.79066895)
\curveto(686.02969417,88.83066618)(686.01969418,88.86066615)(686.01969482,88.88066895)
\curveto(686.02969417,88.9106661)(686.02969417,88.94566606)(686.01969482,88.98566895)
\curveto(685.9996942,89.06566594)(685.98469421,89.14566586)(685.97469482,89.22566895)
\curveto(685.97469422,89.31566569)(685.96469423,89.40066561)(685.94469482,89.48066895)
\lineto(685.94469482,89.60066895)
\curveto(685.94469425,89.64066537)(685.93969426,89.68566532)(685.92969482,89.73566895)
\curveto(685.91969428,89.78566522)(685.91469428,89.87066514)(685.91469482,89.99066895)
\curveto(685.91469428,90.12066489)(685.92469427,90.21566479)(685.94469482,90.27566895)
\curveto(685.96469423,90.34566466)(685.96969423,90.41566459)(685.95969482,90.48566895)
\curveto(685.94969425,90.55566445)(685.95469424,90.62566438)(685.97469482,90.69566895)
\curveto(685.98469421,90.74566426)(685.98969421,90.78566422)(685.98969482,90.81566895)
\curveto(685.9996942,90.85566415)(686.00969419,90.90066411)(686.01969482,90.95066895)
\curveto(686.04969415,91.07066394)(686.07469412,91.19066382)(686.09469482,91.31066895)
\curveto(686.12469407,91.43066358)(686.16469403,91.54566346)(686.21469482,91.65566895)
\curveto(686.36469383,92.02566298)(686.54469365,92.35566265)(686.75469482,92.64566895)
\curveto(686.97469322,92.94566206)(687.23969296,93.19566181)(687.54969482,93.39566895)
\curveto(687.66969253,93.47566153)(687.7946924,93.54066147)(687.92469482,93.59066895)
\curveto(688.05469214,93.65066136)(688.18969201,93.7106613)(688.32969482,93.77066895)
\curveto(688.44969175,93.82066119)(688.57969162,93.85066116)(688.71969482,93.86066895)
\curveto(688.85969134,93.88066113)(688.9996912,93.9106611)(689.13969482,93.95066895)
\lineto(689.33469482,93.95066895)
\curveto(689.40469079,93.96066105)(689.46969073,93.97066104)(689.52969482,93.98066895)
\curveto(690.41968978,93.99066102)(691.15968904,93.8056612)(691.74969482,93.42566895)
\curveto(692.33968786,93.04566196)(692.76468743,92.55066246)(693.02469482,91.94066895)
\curveto(693.07468712,91.84066317)(693.11468708,91.74066327)(693.14469482,91.64066895)
\curveto(693.17468702,91.54066347)(693.20968699,91.43566357)(693.24969482,91.32566895)
\curveto(693.27968692,91.21566379)(693.30468689,91.09566391)(693.32469482,90.96566895)
\curveto(693.34468685,90.84566416)(693.36968683,90.72066429)(693.39969482,90.59066895)
\curveto(693.40968679,90.54066447)(693.40968679,90.48566452)(693.39969482,90.42566895)
\curveto(693.3996868,90.37566463)(693.40468679,90.32566468)(693.41469482,90.27566895)
\moveto(692.07969482,89.42066895)
\curveto(692.0996881,89.49066552)(692.10468809,89.57066544)(692.09469482,89.66066895)
\lineto(692.09469482,89.91566895)
\curveto(692.0946881,90.3056647)(692.05968814,90.63566437)(691.98969482,90.90566895)
\curveto(691.95968824,90.98566402)(691.93468826,91.06566394)(691.91469482,91.14566895)
\curveto(691.8946883,91.22566378)(691.86968833,91.30066371)(691.83969482,91.37066895)
\curveto(691.55968864,92.02066299)(691.11468908,92.47066254)(690.50469482,92.72066895)
\curveto(690.43468976,92.75066226)(690.35968984,92.77066224)(690.27969482,92.78066895)
\lineto(690.03969482,92.84066895)
\curveto(689.95969024,92.86066215)(689.87469032,92.87066214)(689.78469482,92.87066895)
\lineto(689.51469482,92.87066895)
\lineto(689.24469482,92.82566895)
\curveto(689.14469105,92.8056622)(689.04969115,92.78066223)(688.95969482,92.75066895)
\curveto(688.87969132,92.73066228)(688.7996914,92.70066231)(688.71969482,92.66066895)
\curveto(688.64969155,92.64066237)(688.58469161,92.6106624)(688.52469482,92.57066895)
\curveto(688.46469173,92.53066248)(688.40969179,92.49066252)(688.35969482,92.45066895)
\curveto(688.11969208,92.28066273)(687.92469227,92.07566293)(687.77469482,91.83566895)
\curveto(687.62469257,91.59566341)(687.4946927,91.31566369)(687.38469482,90.99566895)
\curveto(687.35469284,90.89566411)(687.33469286,90.79066422)(687.32469482,90.68066895)
\curveto(687.31469288,90.58066443)(687.2996929,90.47566453)(687.27969482,90.36566895)
\curveto(687.26969293,90.32566468)(687.26469293,90.26066475)(687.26469482,90.17066895)
\curveto(687.25469294,90.14066487)(687.24969295,90.1056649)(687.24969482,90.06566895)
\curveto(687.25969294,90.02566498)(687.26469293,89.98066503)(687.26469482,89.93066895)
\lineto(687.26469482,89.63066895)
\curveto(687.26469293,89.53066548)(687.27469292,89.44066557)(687.29469482,89.36066895)
\lineto(687.32469482,89.18066895)
\curveto(687.34469285,89.08066593)(687.35969284,88.98066603)(687.36969482,88.88066895)
\curveto(687.38969281,88.79066622)(687.41969278,88.7056663)(687.45969482,88.62566895)
\curveto(687.55969264,88.38566662)(687.67469252,88.16066685)(687.80469482,87.95066895)
\curveto(687.94469225,87.74066727)(688.11469208,87.56566744)(688.31469482,87.42566895)
\curveto(688.36469183,87.39566761)(688.40969179,87.37066764)(688.44969482,87.35066895)
\curveto(688.48969171,87.33066768)(688.53469166,87.3056677)(688.58469482,87.27566895)
\curveto(688.66469153,87.22566778)(688.74969145,87.18066783)(688.83969482,87.14066895)
\curveto(688.93969126,87.1106679)(689.04469115,87.08066793)(689.15469482,87.05066895)
\curveto(689.20469099,87.03066798)(689.24969095,87.02066799)(689.28969482,87.02066895)
\curveto(689.33969086,87.03066798)(689.38969081,87.03066798)(689.43969482,87.02066895)
\curveto(689.46969073,87.010668)(689.52969067,87.00066801)(689.61969482,86.99066895)
\curveto(689.71969048,86.98066803)(689.7946904,86.98566802)(689.84469482,87.00566895)
\curveto(689.88469031,87.01566799)(689.92469027,87.01566799)(689.96469482,87.00566895)
\curveto(690.00469019,87.005668)(690.04469015,87.01566799)(690.08469482,87.03566895)
\curveto(690.16469003,87.05566795)(690.24468995,87.07066794)(690.32469482,87.08066895)
\curveto(690.40468979,87.10066791)(690.47968972,87.12566788)(690.54969482,87.15566895)
\curveto(690.88968931,87.29566771)(691.16468903,87.49066752)(691.37469482,87.74066895)
\curveto(691.58468861,87.99066702)(691.75968844,88.28566672)(691.89969482,88.62566895)
\curveto(691.94968825,88.74566626)(691.97968822,88.87066614)(691.98969482,89.00066895)
\curveto(692.00968819,89.14066587)(692.03968816,89.28066573)(692.07969482,89.42066895)
}
}
{
\newrgbcolor{curcolor}{0 0 0}
\pscustom[linestyle=none,fillstyle=solid,fillcolor=curcolor]
{
\newpath
\moveto(698.54797607,93.98066895)
\curveto(698.77797128,93.98066103)(698.90797115,93.92066109)(698.93797607,93.80066895)
\curveto(698.96797109,93.69066132)(698.98297108,93.52566148)(698.98297607,93.30566895)
\lineto(698.98297607,93.02066895)
\curveto(698.98297108,92.93066208)(698.9579711,92.85566215)(698.90797607,92.79566895)
\curveto(698.84797121,92.71566229)(698.7629713,92.67066234)(698.65297607,92.66066895)
\curveto(698.54297152,92.66066235)(698.43297163,92.64566236)(698.32297607,92.61566895)
\curveto(698.18297188,92.58566242)(698.04797201,92.55566245)(697.91797607,92.52566895)
\curveto(697.79797226,92.49566251)(697.68297238,92.45566255)(697.57297607,92.40566895)
\curveto(697.28297278,92.27566273)(697.04797301,92.09566291)(696.86797607,91.86566895)
\curveto(696.68797337,91.64566336)(696.53297353,91.39066362)(696.40297607,91.10066895)
\curveto(696.3629737,90.99066402)(696.33297373,90.87566413)(696.31297607,90.75566895)
\curveto(696.29297377,90.64566436)(696.26797379,90.53066448)(696.23797607,90.41066895)
\curveto(696.22797383,90.36066465)(696.22297384,90.3106647)(696.22297607,90.26066895)
\curveto(696.23297383,90.2106648)(696.23297383,90.16066485)(696.22297607,90.11066895)
\curveto(696.19297387,89.99066502)(696.17797388,89.85066516)(696.17797607,89.69066895)
\curveto(696.18797387,89.54066547)(696.19297387,89.39566561)(696.19297607,89.25566895)
\lineto(696.19297607,87.41066895)
\lineto(696.19297607,87.06566895)
\curveto(696.19297387,86.94566806)(696.18797387,86.83066818)(696.17797607,86.72066895)
\curveto(696.16797389,86.6106684)(696.1629739,86.51566849)(696.16297607,86.43566895)
\curveto(696.17297389,86.35566865)(696.15297391,86.28566872)(696.10297607,86.22566895)
\curveto(696.05297401,86.15566885)(695.97297409,86.11566889)(695.86297607,86.10566895)
\curveto(695.7629743,86.09566891)(695.65297441,86.09066892)(695.53297607,86.09066895)
\lineto(695.26297607,86.09066895)
\curveto(695.21297485,86.1106689)(695.1629749,86.12566888)(695.11297607,86.13566895)
\curveto(695.07297499,86.15566885)(695.04297502,86.18066883)(695.02297607,86.21066895)
\curveto(694.97297509,86.28066873)(694.94297512,86.36566864)(694.93297607,86.46566895)
\lineto(694.93297607,86.79566895)
\lineto(694.93297607,87.95066895)
\lineto(694.93297607,92.10566895)
\lineto(694.93297607,93.14066895)
\lineto(694.93297607,93.44066895)
\curveto(694.94297512,93.54066147)(694.97297509,93.62566138)(695.02297607,93.69566895)
\curveto(695.05297501,93.73566127)(695.10297496,93.76566124)(695.17297607,93.78566895)
\curveto(695.25297481,93.8056612)(695.33797472,93.81566119)(695.42797607,93.81566895)
\curveto(695.51797454,93.82566118)(695.60797445,93.82566118)(695.69797607,93.81566895)
\curveto(695.78797427,93.8056612)(695.8579742,93.79066122)(695.90797607,93.77066895)
\curveto(695.98797407,93.74066127)(696.03797402,93.68066133)(696.05797607,93.59066895)
\curveto(696.08797397,93.5106615)(696.10297396,93.42066159)(696.10297607,93.32066895)
\lineto(696.10297607,93.02066895)
\curveto(696.10297396,92.92066209)(696.12297394,92.83066218)(696.16297607,92.75066895)
\curveto(696.17297389,92.73066228)(696.18297388,92.71566229)(696.19297607,92.70566895)
\lineto(696.23797607,92.66066895)
\curveto(696.34797371,92.66066235)(696.43797362,92.7056623)(696.50797607,92.79566895)
\curveto(696.57797348,92.89566211)(696.63797342,92.97566203)(696.68797607,93.03566895)
\lineto(696.77797607,93.12566895)
\curveto(696.86797319,93.23566177)(696.99297307,93.35066166)(697.15297607,93.47066895)
\curveto(697.31297275,93.59066142)(697.4629726,93.68066133)(697.60297607,93.74066895)
\curveto(697.69297237,93.79066122)(697.78797227,93.82566118)(697.88797607,93.84566895)
\curveto(697.98797207,93.87566113)(698.09297197,93.9056611)(698.20297607,93.93566895)
\curveto(698.2629718,93.94566106)(698.32297174,93.95066106)(698.38297607,93.95066895)
\curveto(698.44297162,93.96066105)(698.49797156,93.97066104)(698.54797607,93.98066895)
}
}
{
\newrgbcolor{curcolor}{0.50196081 0.50196081 0.50196081}
\pscustom[linestyle=none,fillstyle=solid,fillcolor=curcolor]
{
\newpath
\moveto(606.37554932,96.78570557)
\lineto(621.37554932,96.78570557)
\lineto(621.37554932,81.78570557)
\lineto(606.37554932,81.78570557)
\closepath
}
}
{
\newrgbcolor{curcolor}{0 0 0}
\pscustom[linestyle=none,fillstyle=solid,fillcolor=curcolor]
{
\newpath
\moveto(634.74875732,68.81349121)
\lineto(634.74875732,68.54349121)
\curveto(634.75874735,68.45348596)(634.75374736,68.37348604)(634.73375732,68.30349121)
\lineto(634.73375732,68.15349121)
\curveto(634.72374739,68.12348629)(634.71874739,68.08848633)(634.71875732,68.04849121)
\curveto(634.72874738,68.00848641)(634.72874738,67.97848644)(634.71875732,67.95849121)
\curveto(634.7087474,67.90848651)(634.70374741,67.85348656)(634.70375732,67.79349121)
\curveto(634.70374741,67.74348667)(634.69874741,67.69348672)(634.68875732,67.64349121)
\curveto(634.65874745,67.50348691)(634.63874747,67.35348706)(634.62875732,67.19349121)
\curveto(634.61874749,67.04348737)(634.58874752,66.89848752)(634.53875732,66.75849121)
\curveto(634.5087476,66.63848778)(634.47374764,66.5134879)(634.43375732,66.38349121)
\curveto(634.40374771,66.26348815)(634.36374775,66.14348827)(634.31375732,66.02349121)
\curveto(634.14374797,65.59348882)(633.92874818,65.20348921)(633.66875732,64.85349121)
\curveto(633.41874869,64.5134899)(633.10374901,64.22349019)(632.72375732,63.98349121)
\curveto(632.53374958,63.86349055)(632.32874978,63.75849066)(632.10875732,63.66849121)
\curveto(631.89875021,63.58849083)(631.66875044,63.50849091)(631.41875732,63.42849121)
\curveto(631.3087508,63.38849103)(631.18875092,63.35849106)(631.05875732,63.33849121)
\curveto(630.93875117,63.32849109)(630.81875129,63.30849111)(630.69875732,63.27849121)
\curveto(630.58875152,63.25849116)(630.47875163,63.24849117)(630.36875732,63.24849121)
\curveto(630.26875184,63.24849117)(630.16875194,63.23849118)(630.06875732,63.21849121)
\lineto(629.85875732,63.21849121)
\curveto(629.82875228,63.20849121)(629.79375232,63.20349121)(629.75375732,63.20349121)
\curveto(629.7137524,63.2134912)(629.67375244,63.2184912)(629.63375732,63.21849121)
\lineto(626.63375732,63.21849121)
\curveto(626.48375563,63.2184912)(626.34875576,63.22349119)(626.22875732,63.23349121)
\curveto(626.11875599,63.25349116)(626.04375607,63.3184911)(626.00375732,63.42849121)
\curveto(625.96375615,63.50849091)(625.94375617,63.62349079)(625.94375732,63.77349121)
\curveto(625.95375616,63.92349049)(625.95875615,64.05849036)(625.95875732,64.17849121)
\lineto(625.95875732,73.04349121)
\curveto(625.95875615,73.16348125)(625.95375616,73.28848113)(625.94375732,73.41849121)
\curveto(625.94375617,73.55848086)(625.96875614,73.66848075)(626.01875732,73.74849121)
\curveto(626.05875605,73.8184806)(626.13375598,73.86348055)(626.24375732,73.88349121)
\curveto(626.26375585,73.89348052)(626.28375583,73.89348052)(626.30375732,73.88349121)
\curveto(626.32375579,73.88348053)(626.34375577,73.88848053)(626.36375732,73.89849121)
\lineto(629.61875732,73.89849121)
\curveto(629.66875244,73.89848052)(629.7137524,73.89848052)(629.75375732,73.89849121)
\curveto(629.80375231,73.90848051)(629.84875226,73.90848051)(629.88875732,73.89849121)
\curveto(629.93875217,73.87848054)(629.98875212,73.87348054)(630.03875732,73.88349121)
\curveto(630.09875201,73.89348052)(630.15375196,73.89348052)(630.20375732,73.88349121)
\curveto(630.25375186,73.87348054)(630.3087518,73.86848055)(630.36875732,73.86849121)
\curveto(630.42875168,73.86848055)(630.48375163,73.86348055)(630.53375732,73.85349121)
\curveto(630.58375153,73.84348057)(630.62875148,73.83848058)(630.66875732,73.83849121)
\curveto(630.71875139,73.83848058)(630.76875134,73.83348058)(630.81875732,73.82349121)
\curveto(630.92875118,73.80348061)(631.03375108,73.78348063)(631.13375732,73.76349121)
\curveto(631.23375088,73.75348066)(631.33375078,73.73348068)(631.43375732,73.70349121)
\curveto(631.65375046,73.63348078)(631.86375025,73.56348085)(632.06375732,73.49349121)
\curveto(632.26374985,73.43348098)(632.44874966,73.34848107)(632.61875732,73.23849121)
\curveto(632.75874935,73.15848126)(632.88374923,73.07848134)(632.99375732,72.99849121)
\curveto(633.02374909,72.97848144)(633.05374906,72.95348146)(633.08375732,72.92349121)
\curveto(633.113749,72.90348151)(633.14374897,72.88348153)(633.17375732,72.86349121)
\curveto(633.23374888,72.8134816)(633.28874882,72.76348165)(633.33875732,72.71349121)
\curveto(633.38874872,72.66348175)(633.43874867,72.6134818)(633.48875732,72.56349121)
\curveto(633.53874857,72.5134819)(633.57874853,72.47848194)(633.60875732,72.45849121)
\curveto(633.64874846,72.39848202)(633.68874842,72.34348207)(633.72875732,72.29349121)
\curveto(633.77874833,72.24348217)(633.82374829,72.18848223)(633.86375732,72.12849121)
\curveto(633.9137482,72.06848235)(633.95374816,72.00348241)(633.98375732,71.93349121)
\curveto(634.02374809,71.87348254)(634.06874804,71.80848261)(634.11875732,71.73849121)
\curveto(634.13874797,71.69848272)(634.15374796,71.66348275)(634.16375732,71.63349121)
\curveto(634.17374794,71.60348281)(634.18874792,71.56848285)(634.20875732,71.52849121)
\curveto(634.24874786,71.44848297)(634.28374783,71.36848305)(634.31375732,71.28849121)
\curveto(634.34374777,71.2184832)(634.37874773,71.14348327)(634.41875732,71.06349121)
\curveto(634.45874765,70.95348346)(634.48874762,70.83848358)(634.50875732,70.71849121)
\curveto(634.53874757,70.60848381)(634.56874754,70.49848392)(634.59875732,70.38849121)
\curveto(634.61874749,70.32848409)(634.62874748,70.26848415)(634.62875732,70.20849121)
\curveto(634.62874748,70.15848426)(634.63874747,70.10348431)(634.65875732,70.04349121)
\curveto(634.7087474,69.86348455)(634.73374738,69.66348475)(634.73375732,69.44349121)
\curveto(634.74374737,69.23348518)(634.74874736,69.02348539)(634.74875732,68.81349121)
\moveto(633.32375732,68.03349121)
\curveto(633.34374877,68.13348628)(633.35374876,68.23848618)(633.35375732,68.34849121)
\lineto(633.35375732,68.69349121)
\lineto(633.35375732,68.91849121)
\curveto(633.36374875,68.99848542)(633.35874875,69.07348534)(633.33875732,69.14349121)
\curveto(633.33874877,69.17348524)(633.33374878,69.20348521)(633.32375732,69.23349121)
\lineto(633.32375732,69.33849121)
\curveto(633.30374881,69.44848497)(633.28874882,69.55848486)(633.27875732,69.66849121)
\curveto(633.27874883,69.77848464)(633.26374885,69.88848453)(633.23375732,69.99849121)
\curveto(633.2137489,70.07848434)(633.19374892,70.15348426)(633.17375732,70.22349121)
\curveto(633.16374895,70.30348411)(633.14874896,70.38348403)(633.12875732,70.46349121)
\curveto(633.01874909,70.82348359)(632.87874923,71.13848328)(632.70875732,71.40849121)
\curveto(632.42874968,71.85848256)(632.0137501,72.19848222)(631.46375732,72.42849121)
\curveto(631.37375074,72.47848194)(631.27875083,72.5134819)(631.17875732,72.53349121)
\curveto(631.07875103,72.56348185)(630.97375114,72.59348182)(630.86375732,72.62349121)
\curveto(630.75375136,72.65348176)(630.63875147,72.66848175)(630.51875732,72.66849121)
\curveto(630.4087517,72.67848174)(630.29875181,72.69348172)(630.18875732,72.71349121)
\lineto(629.87375732,72.71349121)
\curveto(629.84375227,72.72348169)(629.8087523,72.72848169)(629.76875732,72.72849121)
\lineto(629.64875732,72.72849121)
\lineto(627.81875732,72.72849121)
\curveto(627.79875431,72.7184817)(627.77375434,72.7134817)(627.74375732,72.71349121)
\curveto(627.7137544,72.72348169)(627.68875442,72.72348169)(627.66875732,72.71349121)
\lineto(627.51875732,72.65349121)
\curveto(627.47875463,72.63348178)(627.44875466,72.60348181)(627.42875732,72.56349121)
\curveto(627.4087547,72.52348189)(627.38875472,72.45348196)(627.36875732,72.35349121)
\lineto(627.36875732,72.23349121)
\curveto(627.35875475,72.19348222)(627.35375476,72.14848227)(627.35375732,72.09849121)
\lineto(627.35375732,71.96349121)
\lineto(627.35375732,65.15349121)
\lineto(627.35375732,65.00349121)
\curveto(627.35375476,64.96348945)(627.35875475,64.92348949)(627.36875732,64.88349121)
\lineto(627.36875732,64.76349121)
\curveto(627.38875472,64.66348975)(627.4087547,64.59348982)(627.42875732,64.55349121)
\curveto(627.5087546,64.43348998)(627.65875445,64.37349004)(627.87875732,64.37349121)
\curveto(628.09875401,64.38349003)(628.3087538,64.38849003)(628.50875732,64.38849121)
\lineto(629.37875732,64.38849121)
\curveto(629.44875266,64.38849003)(629.52375259,64.38349003)(629.60375732,64.37349121)
\curveto(629.68375243,64.37349004)(629.75375236,64.38349003)(629.81375732,64.40349121)
\lineto(629.97875732,64.40349121)
\curveto(630.02875208,64.41349)(630.08375203,64.41349)(630.14375732,64.40349121)
\curveto(630.20375191,64.40349001)(630.26375185,64.40849001)(630.32375732,64.41849121)
\curveto(630.38375173,64.43848998)(630.44375167,64.44848997)(630.50375732,64.44849121)
\curveto(630.56375155,64.45848996)(630.62875148,64.47348994)(630.69875732,64.49349121)
\curveto(630.8087513,64.52348989)(630.9137512,64.55348986)(631.01375732,64.58349121)
\curveto(631.12375099,64.6134898)(631.23375088,64.65348976)(631.34375732,64.70349121)
\curveto(631.7137504,64.86348955)(632.02875008,65.07848934)(632.28875732,65.34849121)
\curveto(632.55874955,65.62848879)(632.77874933,65.95848846)(632.94875732,66.33849121)
\curveto(632.99874911,66.44848797)(633.03874907,66.56348785)(633.06875732,66.68349121)
\lineto(633.18875732,67.07349121)
\curveto(633.21874889,67.18348723)(633.23874887,67.29848712)(633.24875732,67.41849121)
\curveto(633.26874884,67.54848687)(633.28874882,67.67348674)(633.30875732,67.79349121)
\curveto(633.31874879,67.84348657)(633.32374879,67.88348653)(633.32375732,67.91349121)
\lineto(633.32375732,68.03349121)
}
}
{
\newrgbcolor{curcolor}{0 0 0}
\pscustom[linestyle=none,fillstyle=solid,fillcolor=curcolor]
{
\newpath
\moveto(643.00063232,67.38849121)
\curveto(643.02062464,67.28848713)(643.02062464,67.17348724)(643.00063232,67.04349121)
\curveto(642.99062467,66.92348749)(642.9606247,66.83848758)(642.91063232,66.78849121)
\curveto(642.8606248,66.74848767)(642.78562487,66.7184877)(642.68563232,66.69849121)
\curveto(642.59562506,66.68848773)(642.49062517,66.68348773)(642.37063232,66.68349121)
\lineto(642.01063232,66.68349121)
\curveto(641.89062577,66.69348772)(641.78562587,66.69848772)(641.69563232,66.69849121)
\lineto(637.85563232,66.69849121)
\curveto(637.77562988,66.69848772)(637.69562996,66.69348772)(637.61563232,66.68349121)
\curveto(637.53563012,66.68348773)(637.47063019,66.66848775)(637.42063232,66.63849121)
\curveto(637.38063028,66.6184878)(637.34063032,66.57848784)(637.30063232,66.51849121)
\curveto(637.28063038,66.48848793)(637.2606304,66.44348797)(637.24063232,66.38349121)
\curveto(637.22063044,66.33348808)(637.22063044,66.28348813)(637.24063232,66.23349121)
\curveto(637.25063041,66.18348823)(637.2556304,66.13848828)(637.25563232,66.09849121)
\curveto(637.2556304,66.05848836)(637.2606304,66.0184884)(637.27063232,65.97849121)
\curveto(637.29063037,65.89848852)(637.31063035,65.8134886)(637.33063232,65.72349121)
\curveto(637.35063031,65.64348877)(637.38063028,65.56348885)(637.42063232,65.48349121)
\curveto(637.65063001,64.94348947)(638.03062963,64.55848986)(638.56063232,64.32849121)
\curveto(638.62062904,64.29849012)(638.68562897,64.27349014)(638.75563232,64.25349121)
\lineto(638.96563232,64.19349121)
\curveto(638.99562866,64.18349023)(639.04562861,64.17849024)(639.11563232,64.17849121)
\curveto(639.2556284,64.13849028)(639.44062822,64.1184903)(639.67063232,64.11849121)
\curveto(639.90062776,64.1184903)(640.08562757,64.13849028)(640.22563232,64.17849121)
\curveto(640.36562729,64.2184902)(640.49062717,64.25849016)(640.60063232,64.29849121)
\curveto(640.72062694,64.34849007)(640.83062683,64.40849001)(640.93063232,64.47849121)
\curveto(641.04062662,64.54848987)(641.13562652,64.62848979)(641.21563232,64.71849121)
\curveto(641.29562636,64.8184896)(641.36562629,64.92348949)(641.42563232,65.03349121)
\curveto(641.48562617,65.13348928)(641.53562612,65.23848918)(641.57563232,65.34849121)
\curveto(641.62562603,65.45848896)(641.70562595,65.53848888)(641.81563232,65.58849121)
\curveto(641.8556258,65.60848881)(641.92062574,65.62348879)(642.01063232,65.63349121)
\curveto(642.10062556,65.64348877)(642.19062547,65.64348877)(642.28063232,65.63349121)
\curveto(642.37062529,65.63348878)(642.4556252,65.62848879)(642.53563232,65.61849121)
\curveto(642.61562504,65.60848881)(642.67062499,65.58848883)(642.70063232,65.55849121)
\curveto(642.80062486,65.48848893)(642.82562483,65.37348904)(642.77563232,65.21349121)
\curveto(642.69562496,64.94348947)(642.59062507,64.70348971)(642.46063232,64.49349121)
\curveto(642.2606254,64.17349024)(642.03062563,63.90849051)(641.77063232,63.69849121)
\curveto(641.52062614,63.49849092)(641.20062646,63.33349108)(640.81063232,63.20349121)
\curveto(640.71062695,63.16349125)(640.61062705,63.13849128)(640.51063232,63.12849121)
\curveto(640.41062725,63.10849131)(640.30562735,63.08849133)(640.19563232,63.06849121)
\curveto(640.14562751,63.05849136)(640.09562756,63.05349136)(640.04563232,63.05349121)
\curveto(640.00562765,63.05349136)(639.9606277,63.04849137)(639.91063232,63.03849121)
\lineto(639.76063232,63.03849121)
\curveto(639.71062795,63.02849139)(639.65062801,63.02349139)(639.58063232,63.02349121)
\curveto(639.52062814,63.02349139)(639.47062819,63.02849139)(639.43063232,63.03849121)
\lineto(639.29563232,63.03849121)
\curveto(639.24562841,63.04849137)(639.20062846,63.05349136)(639.16063232,63.05349121)
\curveto(639.12062854,63.05349136)(639.08062858,63.05849136)(639.04063232,63.06849121)
\curveto(638.99062867,63.07849134)(638.93562872,63.08849133)(638.87563232,63.09849121)
\curveto(638.81562884,63.09849132)(638.7606289,63.10349131)(638.71063232,63.11349121)
\curveto(638.62062904,63.13349128)(638.53062913,63.15849126)(638.44063232,63.18849121)
\curveto(638.35062931,63.20849121)(638.26562939,63.23349118)(638.18563232,63.26349121)
\curveto(638.14562951,63.28349113)(638.11062955,63.29349112)(638.08063232,63.29349121)
\curveto(638.05062961,63.30349111)(638.01562964,63.3184911)(637.97563232,63.33849121)
\curveto(637.82562983,63.40849101)(637.66562999,63.49349092)(637.49563232,63.59349121)
\curveto(637.20563045,63.78349063)(636.9556307,64.0134904)(636.74563232,64.28349121)
\curveto(636.54563111,64.56348985)(636.37563128,64.87348954)(636.23563232,65.21349121)
\curveto(636.18563147,65.32348909)(636.14563151,65.43848898)(636.11563232,65.55849121)
\curveto(636.09563156,65.67848874)(636.06563159,65.79848862)(636.02563232,65.91849121)
\curveto(636.01563164,65.95848846)(636.01063165,65.99348842)(636.01063232,66.02349121)
\curveto(636.01063165,66.05348836)(636.00563165,66.09348832)(635.99563232,66.14349121)
\curveto(635.97563168,66.22348819)(635.9606317,66.30848811)(635.95063232,66.39849121)
\curveto(635.94063172,66.48848793)(635.92563173,66.57848784)(635.90563232,66.66849121)
\lineto(635.90563232,66.87849121)
\curveto(635.89563176,66.9184875)(635.88563177,66.97348744)(635.87563232,67.04349121)
\curveto(635.87563178,67.12348729)(635.88063178,67.18848723)(635.89063232,67.23849121)
\lineto(635.89063232,67.40349121)
\curveto(635.91063175,67.45348696)(635.91563174,67.50348691)(635.90563232,67.55349121)
\curveto(635.90563175,67.6134868)(635.91063175,67.66848675)(635.92063232,67.71849121)
\curveto(635.9606317,67.87848654)(635.99063167,68.03848638)(636.01063232,68.19849121)
\curveto(636.04063162,68.35848606)(636.08563157,68.50848591)(636.14563232,68.64849121)
\curveto(636.19563146,68.75848566)(636.24063142,68.86848555)(636.28063232,68.97849121)
\curveto(636.33063133,69.09848532)(636.38563127,69.2134852)(636.44563232,69.32349121)
\curveto(636.66563099,69.67348474)(636.91563074,69.97348444)(637.19563232,70.22349121)
\curveto(637.47563018,70.48348393)(637.82062984,70.69848372)(638.23063232,70.86849121)
\curveto(638.35062931,70.9184835)(638.47062919,70.95348346)(638.59063232,70.97349121)
\curveto(638.72062894,71.00348341)(638.8556288,71.03348338)(638.99563232,71.06349121)
\curveto(639.04562861,71.07348334)(639.09062857,71.07848334)(639.13063232,71.07849121)
\curveto(639.17062849,71.08848333)(639.21562844,71.09348332)(639.26563232,71.09349121)
\curveto(639.28562837,71.10348331)(639.31062835,71.10348331)(639.34063232,71.09349121)
\curveto(639.37062829,71.08348333)(639.39562826,71.08848333)(639.41563232,71.10849121)
\curveto(639.83562782,71.1184833)(640.20062746,71.07348334)(640.51063232,70.97349121)
\curveto(640.82062684,70.88348353)(641.10062656,70.75848366)(641.35063232,70.59849121)
\curveto(641.40062626,70.57848384)(641.44062622,70.54848387)(641.47063232,70.50849121)
\curveto(641.50062616,70.47848394)(641.53562612,70.45348396)(641.57563232,70.43349121)
\curveto(641.655626,70.37348404)(641.73562592,70.30348411)(641.81563232,70.22349121)
\curveto(641.90562575,70.14348427)(641.98062568,70.06348435)(642.04063232,69.98349121)
\curveto(642.20062546,69.77348464)(642.33562532,69.57348484)(642.44563232,69.38349121)
\curveto(642.51562514,69.27348514)(642.57062509,69.15348526)(642.61063232,69.02349121)
\curveto(642.65062501,68.89348552)(642.69562496,68.76348565)(642.74563232,68.63349121)
\curveto(642.79562486,68.50348591)(642.83062483,68.36848605)(642.85063232,68.22849121)
\curveto(642.88062478,68.08848633)(642.91562474,67.94848647)(642.95563232,67.80849121)
\curveto(642.96562469,67.73848668)(642.97062469,67.66848675)(642.97063232,67.59849121)
\lineto(643.00063232,67.38849121)
\moveto(641.54563232,67.89849121)
\curveto(641.57562608,67.93848648)(641.60062606,67.98848643)(641.62063232,68.04849121)
\curveto(641.64062602,68.1184863)(641.64062602,68.18848623)(641.62063232,68.25849121)
\curveto(641.5606261,68.47848594)(641.47562618,68.68348573)(641.36563232,68.87349121)
\curveto(641.22562643,69.10348531)(641.07062659,69.29848512)(640.90063232,69.45849121)
\curveto(640.73062693,69.6184848)(640.51062715,69.75348466)(640.24063232,69.86349121)
\curveto(640.17062749,69.88348453)(640.10062756,69.89848452)(640.03063232,69.90849121)
\curveto(639.9606277,69.92848449)(639.88562777,69.94848447)(639.80563232,69.96849121)
\curveto(639.72562793,69.98848443)(639.64062802,69.99848442)(639.55063232,69.99849121)
\lineto(639.29563232,69.99849121)
\curveto(639.26562839,69.97848444)(639.23062843,69.96848445)(639.19063232,69.96849121)
\curveto(639.15062851,69.97848444)(639.11562854,69.97848444)(639.08563232,69.96849121)
\lineto(638.84563232,69.90849121)
\curveto(638.77562888,69.89848452)(638.70562895,69.88348453)(638.63563232,69.86349121)
\curveto(638.34562931,69.74348467)(638.11062955,69.59348482)(637.93063232,69.41349121)
\curveto(637.7606299,69.23348518)(637.60563005,69.00848541)(637.46563232,68.73849121)
\curveto(637.43563022,68.68848573)(637.40563025,68.62348579)(637.37563232,68.54349121)
\curveto(637.34563031,68.47348594)(637.32063034,68.39348602)(637.30063232,68.30349121)
\curveto(637.28063038,68.2134862)(637.27563038,68.12848629)(637.28563232,68.04849121)
\curveto(637.29563036,67.96848645)(637.33063033,67.90848651)(637.39063232,67.86849121)
\curveto(637.47063019,67.80848661)(637.60563005,67.77848664)(637.79563232,67.77849121)
\curveto(637.99562966,67.78848663)(638.16562949,67.79348662)(638.30563232,67.79349121)
\lineto(640.58563232,67.79349121)
\curveto(640.73562692,67.79348662)(640.91562674,67.78848663)(641.12563232,67.77849121)
\curveto(641.33562632,67.77848664)(641.47562618,67.8184866)(641.54563232,67.89849121)
}
}
{
\newrgbcolor{curcolor}{0 0 0}
\pscustom[linestyle=none,fillstyle=solid,fillcolor=curcolor]
{
\newpath
\moveto(646.73727295,71.12349121)
\curveto(647.45726888,71.13348328)(648.06226828,71.04848337)(648.55227295,70.86849121)
\curveto(649.0422673,70.69848372)(649.42226692,70.39348402)(649.69227295,69.95349121)
\curveto(649.76226658,69.84348457)(649.81726652,69.72848469)(649.85727295,69.60849121)
\curveto(649.89726644,69.49848492)(649.9372664,69.37348504)(649.97727295,69.23349121)
\curveto(649.99726634,69.16348525)(650.00226634,69.08848533)(649.99227295,69.00849121)
\curveto(649.98226636,68.93848548)(649.96726637,68.88348553)(649.94727295,68.84349121)
\curveto(649.92726641,68.82348559)(649.90226644,68.80348561)(649.87227295,68.78349121)
\curveto(649.8422665,68.77348564)(649.81726652,68.75848566)(649.79727295,68.73849121)
\curveto(649.74726659,68.7184857)(649.69726664,68.7134857)(649.64727295,68.72349121)
\curveto(649.59726674,68.73348568)(649.54726679,68.73348568)(649.49727295,68.72349121)
\curveto(649.41726692,68.70348571)(649.31226703,68.69848572)(649.18227295,68.70849121)
\curveto(649.05226729,68.72848569)(648.96226738,68.75348566)(648.91227295,68.78349121)
\curveto(648.83226751,68.83348558)(648.77726756,68.89848552)(648.74727295,68.97849121)
\curveto(648.72726761,69.06848535)(648.69226765,69.15348526)(648.64227295,69.23349121)
\curveto(648.55226779,69.39348502)(648.42726791,69.53848488)(648.26727295,69.66849121)
\curveto(648.15726818,69.74848467)(648.0372683,69.80848461)(647.90727295,69.84849121)
\curveto(647.77726856,69.88848453)(647.6372687,69.92848449)(647.48727295,69.96849121)
\curveto(647.4372689,69.98848443)(647.38726895,69.99348442)(647.33727295,69.98349121)
\curveto(647.28726905,69.98348443)(647.2372691,69.98848443)(647.18727295,69.99849121)
\curveto(647.12726921,70.0184844)(647.05226929,70.02848439)(646.96227295,70.02849121)
\curveto(646.87226947,70.02848439)(646.79726954,70.0184844)(646.73727295,69.99849121)
\lineto(646.64727295,69.99849121)
\lineto(646.49727295,69.96849121)
\curveto(646.44726989,69.96848445)(646.39726994,69.96348445)(646.34727295,69.95349121)
\curveto(646.08727025,69.89348452)(645.87227047,69.80848461)(645.70227295,69.69849121)
\curveto(645.53227081,69.58848483)(645.41727092,69.40348501)(645.35727295,69.14349121)
\curveto(645.337271,69.07348534)(645.33227101,69.00348541)(645.34227295,68.93349121)
\curveto(645.36227098,68.86348555)(645.38227096,68.80348561)(645.40227295,68.75349121)
\curveto(645.46227088,68.60348581)(645.53227081,68.49348592)(645.61227295,68.42349121)
\curveto(645.70227064,68.36348605)(645.81227053,68.29348612)(645.94227295,68.21349121)
\curveto(646.10227024,68.1134863)(646.28227006,68.03848638)(646.48227295,67.98849121)
\curveto(646.68226966,67.94848647)(646.88226946,67.89848652)(647.08227295,67.83849121)
\curveto(647.21226913,67.79848662)(647.342269,67.76848665)(647.47227295,67.74849121)
\curveto(647.60226874,67.72848669)(647.73226861,67.69848672)(647.86227295,67.65849121)
\curveto(648.07226827,67.59848682)(648.27726806,67.53848688)(648.47727295,67.47849121)
\curveto(648.67726766,67.42848699)(648.87726746,67.36348705)(649.07727295,67.28349121)
\lineto(649.22727295,67.22349121)
\curveto(649.27726706,67.20348721)(649.32726701,67.17848724)(649.37727295,67.14849121)
\curveto(649.57726676,67.02848739)(649.75226659,66.89348752)(649.90227295,66.74349121)
\curveto(650.05226629,66.59348782)(650.17726616,66.40348801)(650.27727295,66.17349121)
\curveto(650.29726604,66.10348831)(650.31726602,66.00848841)(650.33727295,65.88849121)
\curveto(650.35726598,65.8184886)(650.36726597,65.74348867)(650.36727295,65.66349121)
\curveto(650.37726596,65.59348882)(650.38226596,65.5134889)(650.38227295,65.42349121)
\lineto(650.38227295,65.27349121)
\curveto(650.36226598,65.20348921)(650.35226599,65.13348928)(650.35227295,65.06349121)
\curveto(650.35226599,64.99348942)(650.342266,64.92348949)(650.32227295,64.85349121)
\curveto(650.29226605,64.74348967)(650.25726608,64.63848978)(650.21727295,64.53849121)
\curveto(650.17726616,64.43848998)(650.13226621,64.34849007)(650.08227295,64.26849121)
\curveto(649.92226642,64.00849041)(649.71726662,63.79849062)(649.46727295,63.63849121)
\curveto(649.21726712,63.48849093)(648.9372674,63.35849106)(648.62727295,63.24849121)
\curveto(648.5372678,63.2184912)(648.4422679,63.19849122)(648.34227295,63.18849121)
\curveto(648.25226809,63.16849125)(648.16226818,63.14349127)(648.07227295,63.11349121)
\curveto(647.97226837,63.09349132)(647.87226847,63.08349133)(647.77227295,63.08349121)
\curveto(647.67226867,63.08349133)(647.57226877,63.07349134)(647.47227295,63.05349121)
\lineto(647.32227295,63.05349121)
\curveto(647.27226907,63.04349137)(647.20226914,63.03849138)(647.11227295,63.03849121)
\curveto(647.02226932,63.03849138)(646.95226939,63.04349137)(646.90227295,63.05349121)
\lineto(646.73727295,63.05349121)
\curveto(646.67726966,63.07349134)(646.61226973,63.08349133)(646.54227295,63.08349121)
\curveto(646.47226987,63.07349134)(646.41226993,63.07849134)(646.36227295,63.09849121)
\curveto(646.31227003,63.10849131)(646.24727009,63.1134913)(646.16727295,63.11349121)
\lineto(645.92727295,63.17349121)
\curveto(645.85727048,63.18349123)(645.78227056,63.20349121)(645.70227295,63.23349121)
\curveto(645.39227095,63.33349108)(645.12227122,63.45849096)(644.89227295,63.60849121)
\curveto(644.66227168,63.75849066)(644.46227188,63.95349046)(644.29227295,64.19349121)
\curveto(644.20227214,64.32349009)(644.12727221,64.45848996)(644.06727295,64.59849121)
\curveto(644.00727233,64.73848968)(643.95227239,64.89348952)(643.90227295,65.06349121)
\curveto(643.88227246,65.12348929)(643.87227247,65.19348922)(643.87227295,65.27349121)
\curveto(643.88227246,65.36348905)(643.89727244,65.43348898)(643.91727295,65.48349121)
\curveto(643.94727239,65.52348889)(643.99727234,65.56348885)(644.06727295,65.60349121)
\curveto(644.11727222,65.62348879)(644.18727215,65.63348878)(644.27727295,65.63349121)
\curveto(644.36727197,65.64348877)(644.45727188,65.64348877)(644.54727295,65.63349121)
\curveto(644.6372717,65.62348879)(644.72227162,65.60848881)(644.80227295,65.58849121)
\curveto(644.89227145,65.57848884)(644.95227139,65.56348885)(644.98227295,65.54349121)
\curveto(645.05227129,65.49348892)(645.09727124,65.418489)(645.11727295,65.31849121)
\curveto(645.14727119,65.22848919)(645.18227116,65.14348927)(645.22227295,65.06349121)
\curveto(645.32227102,64.84348957)(645.45727088,64.67348974)(645.62727295,64.55349121)
\curveto(645.74727059,64.46348995)(645.88227046,64.39349002)(646.03227295,64.34349121)
\curveto(646.18227016,64.29349012)(646.34227,64.24349017)(646.51227295,64.19349121)
\lineto(646.82727295,64.14849121)
\lineto(646.91727295,64.14849121)
\curveto(646.98726935,64.12849029)(647.07726926,64.1184903)(647.18727295,64.11849121)
\curveto(647.30726903,64.1184903)(647.40726893,64.12849029)(647.48727295,64.14849121)
\curveto(647.55726878,64.14849027)(647.61226873,64.15349026)(647.65227295,64.16349121)
\curveto(647.71226863,64.17349024)(647.77226857,64.17849024)(647.83227295,64.17849121)
\curveto(647.89226845,64.18849023)(647.94726839,64.19849022)(647.99727295,64.20849121)
\curveto(648.28726805,64.28849013)(648.51726782,64.39349002)(648.68727295,64.52349121)
\curveto(648.85726748,64.65348976)(648.97726736,64.87348954)(649.04727295,65.18349121)
\curveto(649.06726727,65.23348918)(649.07226727,65.28848913)(649.06227295,65.34849121)
\curveto(649.05226729,65.40848901)(649.0422673,65.45348896)(649.03227295,65.48349121)
\curveto(648.98226736,65.67348874)(648.91226743,65.8134886)(648.82227295,65.90349121)
\curveto(648.73226761,66.00348841)(648.61726772,66.09348832)(648.47727295,66.17349121)
\curveto(648.38726795,66.23348818)(648.28726805,66.28348813)(648.17727295,66.32349121)
\lineto(647.84727295,66.44349121)
\curveto(647.81726852,66.45348796)(647.78726855,66.45848796)(647.75727295,66.45849121)
\curveto(647.7372686,66.45848796)(647.71226863,66.46848795)(647.68227295,66.48849121)
\curveto(647.342269,66.59848782)(646.98726935,66.67848774)(646.61727295,66.72849121)
\curveto(646.25727008,66.78848763)(645.91727042,66.88348753)(645.59727295,67.01349121)
\curveto(645.49727084,67.05348736)(645.40227094,67.08848733)(645.31227295,67.11849121)
\curveto(645.22227112,67.14848727)(645.1372712,67.18848723)(645.05727295,67.23849121)
\curveto(644.86727147,67.34848707)(644.69227165,67.47348694)(644.53227295,67.61349121)
\curveto(644.37227197,67.75348666)(644.24727209,67.92848649)(644.15727295,68.13849121)
\curveto(644.12727221,68.20848621)(644.10227224,68.27848614)(644.08227295,68.34849121)
\curveto(644.07227227,68.418486)(644.05727228,68.49348592)(644.03727295,68.57349121)
\curveto(644.00727233,68.69348572)(643.99727234,68.82848559)(644.00727295,68.97849121)
\curveto(644.01727232,69.13848528)(644.03227231,69.27348514)(644.05227295,69.38349121)
\curveto(644.07227227,69.43348498)(644.08227226,69.47348494)(644.08227295,69.50349121)
\curveto(644.09227225,69.54348487)(644.10727223,69.58348483)(644.12727295,69.62349121)
\curveto(644.21727212,69.85348456)(644.337272,70.05348436)(644.48727295,70.22349121)
\curveto(644.64727169,70.39348402)(644.82727151,70.54348387)(645.02727295,70.67349121)
\curveto(645.17727116,70.76348365)(645.342271,70.83348358)(645.52227295,70.88349121)
\curveto(645.70227064,70.94348347)(645.89227045,70.99848342)(646.09227295,71.04849121)
\curveto(646.16227018,71.05848336)(646.22727011,71.06848335)(646.28727295,71.07849121)
\curveto(646.35726998,71.08848333)(646.43226991,71.09848332)(646.51227295,71.10849121)
\curveto(646.5422698,71.1184833)(646.58226976,71.1184833)(646.63227295,71.10849121)
\curveto(646.68226966,71.09848332)(646.71726962,71.10348331)(646.73727295,71.12349121)
}
}
{
\newrgbcolor{curcolor}{0 0 0}
\pscustom[linestyle=none,fillstyle=solid,fillcolor=curcolor]
{
\newpath
\moveto(658.69227295,63.77349121)
\curveto(658.72226512,63.6134908)(658.70726513,63.47849094)(658.64727295,63.36849121)
\curveto(658.58726525,63.26849115)(658.50726533,63.19349122)(658.40727295,63.14349121)
\curveto(658.35726548,63.12349129)(658.30226554,63.1134913)(658.24227295,63.11349121)
\curveto(658.19226565,63.1134913)(658.1372657,63.10349131)(658.07727295,63.08349121)
\curveto(657.85726598,63.03349138)(657.6372662,63.04849137)(657.41727295,63.12849121)
\curveto(657.20726663,63.19849122)(657.06226678,63.28849113)(656.98227295,63.39849121)
\curveto(656.93226691,63.46849095)(656.88726695,63.54849087)(656.84727295,63.63849121)
\curveto(656.80726703,63.73849068)(656.75726708,63.8184906)(656.69727295,63.87849121)
\curveto(656.67726716,63.89849052)(656.65226719,63.9184905)(656.62227295,63.93849121)
\curveto(656.60226724,63.95849046)(656.57226727,63.96349045)(656.53227295,63.95349121)
\curveto(656.42226742,63.92349049)(656.31726752,63.86849055)(656.21727295,63.78849121)
\curveto(656.12726771,63.70849071)(656.0372678,63.63849078)(655.94727295,63.57849121)
\curveto(655.81726802,63.49849092)(655.67726816,63.42349099)(655.52727295,63.35349121)
\curveto(655.37726846,63.29349112)(655.21726862,63.23849118)(655.04727295,63.18849121)
\curveto(654.94726889,63.15849126)(654.837269,63.13849128)(654.71727295,63.12849121)
\curveto(654.60726923,63.1184913)(654.49726934,63.10349131)(654.38727295,63.08349121)
\curveto(654.3372695,63.07349134)(654.29226955,63.06849135)(654.25227295,63.06849121)
\lineto(654.14727295,63.06849121)
\curveto(654.0372698,63.04849137)(653.93226991,63.04849137)(653.83227295,63.06849121)
\lineto(653.69727295,63.06849121)
\curveto(653.64727019,63.07849134)(653.59727024,63.08349133)(653.54727295,63.08349121)
\curveto(653.49727034,63.08349133)(653.45227039,63.09349132)(653.41227295,63.11349121)
\curveto(653.37227047,63.12349129)(653.3372705,63.12849129)(653.30727295,63.12849121)
\curveto(653.28727055,63.1184913)(653.26227058,63.1184913)(653.23227295,63.12849121)
\lineto(652.99227295,63.18849121)
\curveto(652.91227093,63.19849122)(652.837271,63.2184912)(652.76727295,63.24849121)
\curveto(652.46727137,63.37849104)(652.22227162,63.52349089)(652.03227295,63.68349121)
\curveto(651.85227199,63.85349056)(651.70227214,64.08849033)(651.58227295,64.38849121)
\curveto(651.49227235,64.60848981)(651.44727239,64.87348954)(651.44727295,65.18349121)
\lineto(651.44727295,65.49849121)
\curveto(651.45727238,65.54848887)(651.46227238,65.59848882)(651.46227295,65.64849121)
\lineto(651.49227295,65.82849121)
\lineto(651.61227295,66.15849121)
\curveto(651.65227219,66.26848815)(651.70227214,66.36848805)(651.76227295,66.45849121)
\curveto(651.9422719,66.74848767)(652.18727165,66.96348745)(652.49727295,67.10349121)
\curveto(652.80727103,67.24348717)(653.14727069,67.36848705)(653.51727295,67.47849121)
\curveto(653.65727018,67.5184869)(653.80227004,67.54848687)(653.95227295,67.56849121)
\curveto(654.10226974,67.58848683)(654.25226959,67.6134868)(654.40227295,67.64349121)
\curveto(654.47226937,67.66348675)(654.5372693,67.67348674)(654.59727295,67.67349121)
\curveto(654.66726917,67.67348674)(654.7422691,67.68348673)(654.82227295,67.70349121)
\curveto(654.89226895,67.72348669)(654.96226888,67.73348668)(655.03227295,67.73349121)
\curveto(655.10226874,67.74348667)(655.17726866,67.75848666)(655.25727295,67.77849121)
\curveto(655.50726833,67.83848658)(655.7422681,67.88848653)(655.96227295,67.92849121)
\curveto(656.18226766,67.97848644)(656.35726748,68.09348632)(656.48727295,68.27349121)
\curveto(656.54726729,68.35348606)(656.59726724,68.45348596)(656.63727295,68.57349121)
\curveto(656.67726716,68.70348571)(656.67726716,68.84348557)(656.63727295,68.99349121)
\curveto(656.57726726,69.23348518)(656.48726735,69.42348499)(656.36727295,69.56349121)
\curveto(656.25726758,69.70348471)(656.09726774,69.8134846)(655.88727295,69.89349121)
\curveto(655.76726807,69.94348447)(655.62226822,69.97848444)(655.45227295,69.99849121)
\curveto(655.29226855,70.0184844)(655.12226872,70.02848439)(654.94227295,70.02849121)
\curveto(654.76226908,70.02848439)(654.58726925,70.0184844)(654.41727295,69.99849121)
\curveto(654.24726959,69.97848444)(654.10226974,69.94848447)(653.98227295,69.90849121)
\curveto(653.81227003,69.84848457)(653.64727019,69.76348465)(653.48727295,69.65349121)
\curveto(653.40727043,69.59348482)(653.33227051,69.5134849)(653.26227295,69.41349121)
\curveto(653.20227064,69.32348509)(653.14727069,69.22348519)(653.09727295,69.11349121)
\curveto(653.06727077,69.03348538)(653.0372708,68.94848547)(653.00727295,68.85849121)
\curveto(652.98727085,68.76848565)(652.9422709,68.69848572)(652.87227295,68.64849121)
\curveto(652.83227101,68.6184858)(652.76227108,68.59348582)(652.66227295,68.57349121)
\curveto(652.57227127,68.56348585)(652.47727136,68.55848586)(652.37727295,68.55849121)
\curveto(652.27727156,68.55848586)(652.17727166,68.56348585)(652.07727295,68.57349121)
\curveto(651.98727185,68.59348582)(651.92227192,68.6184858)(651.88227295,68.64849121)
\curveto(651.842272,68.67848574)(651.81227203,68.72848569)(651.79227295,68.79849121)
\curveto(651.77227207,68.86848555)(651.77227207,68.94348547)(651.79227295,69.02349121)
\curveto(651.82227202,69.15348526)(651.85227199,69.27348514)(651.88227295,69.38349121)
\curveto(651.92227192,69.50348491)(651.96727187,69.6184848)(652.01727295,69.72849121)
\curveto(652.20727163,70.07848434)(652.44727139,70.34848407)(652.73727295,70.53849121)
\curveto(653.02727081,70.73848368)(653.38727045,70.89848352)(653.81727295,71.01849121)
\curveto(653.91726992,71.03848338)(654.01726982,71.05348336)(654.11727295,71.06349121)
\curveto(654.22726961,71.07348334)(654.3372695,71.08848333)(654.44727295,71.10849121)
\curveto(654.48726935,71.1184833)(654.55226929,71.1184833)(654.64227295,71.10849121)
\curveto(654.73226911,71.10848331)(654.78726905,71.1184833)(654.80727295,71.13849121)
\curveto(655.50726833,71.14848327)(656.11726772,71.06848335)(656.63727295,70.89849121)
\curveto(657.15726668,70.72848369)(657.52226632,70.40348401)(657.73227295,69.92349121)
\curveto(657.82226602,69.72348469)(657.87226597,69.48848493)(657.88227295,69.21849121)
\curveto(657.90226594,68.95848546)(657.91226593,68.68348573)(657.91227295,68.39349121)
\lineto(657.91227295,65.07849121)
\curveto(657.91226593,64.93848948)(657.91726592,64.80348961)(657.92727295,64.67349121)
\curveto(657.9372659,64.54348987)(657.96726587,64.43848998)(658.01727295,64.35849121)
\curveto(658.06726577,64.28849013)(658.13226571,64.23849018)(658.21227295,64.20849121)
\curveto(658.30226554,64.16849025)(658.38726545,64.13849028)(658.46727295,64.11849121)
\curveto(658.54726529,64.10849031)(658.60726523,64.06349035)(658.64727295,63.98349121)
\curveto(658.66726517,63.95349046)(658.67726516,63.92349049)(658.67727295,63.89349121)
\curveto(658.67726516,63.86349055)(658.68226516,63.82349059)(658.69227295,63.77349121)
\moveto(656.54727295,65.43849121)
\curveto(656.60726723,65.57848884)(656.6372672,65.73848868)(656.63727295,65.91849121)
\curveto(656.64726719,66.10848831)(656.65226719,66.30348811)(656.65227295,66.50349121)
\curveto(656.65226719,66.6134878)(656.64726719,66.7134877)(656.63727295,66.80349121)
\curveto(656.62726721,66.89348752)(656.58726725,66.96348745)(656.51727295,67.01349121)
\curveto(656.48726735,67.03348738)(656.41726742,67.04348737)(656.30727295,67.04349121)
\curveto(656.28726755,67.02348739)(656.25226759,67.0134874)(656.20227295,67.01349121)
\curveto(656.15226769,67.0134874)(656.10726773,67.00348741)(656.06727295,66.98349121)
\curveto(655.98726785,66.96348745)(655.89726794,66.94348747)(655.79727295,66.92349121)
\lineto(655.49727295,66.86349121)
\curveto(655.46726837,66.86348755)(655.43226841,66.85848756)(655.39227295,66.84849121)
\lineto(655.28727295,66.84849121)
\curveto(655.1372687,66.80848761)(654.97226887,66.78348763)(654.79227295,66.77349121)
\curveto(654.62226922,66.77348764)(654.46226938,66.75348766)(654.31227295,66.71349121)
\curveto(654.23226961,66.69348772)(654.15726968,66.67348774)(654.08727295,66.65349121)
\curveto(654.02726981,66.64348777)(653.95726988,66.62848779)(653.87727295,66.60849121)
\curveto(653.71727012,66.55848786)(653.56727027,66.49348792)(653.42727295,66.41349121)
\curveto(653.28727055,66.34348807)(653.16727067,66.25348816)(653.06727295,66.14349121)
\curveto(652.96727087,66.03348838)(652.89227095,65.89848852)(652.84227295,65.73849121)
\curveto(652.79227105,65.58848883)(652.77227107,65.40348901)(652.78227295,65.18349121)
\curveto(652.78227106,65.08348933)(652.79727104,64.98848943)(652.82727295,64.89849121)
\curveto(652.86727097,64.8184896)(652.91227093,64.74348967)(652.96227295,64.67349121)
\curveto(653.0422708,64.56348985)(653.14727069,64.46848995)(653.27727295,64.38849121)
\curveto(653.40727043,64.3184901)(653.54727029,64.25849016)(653.69727295,64.20849121)
\curveto(653.74727009,64.19849022)(653.79727004,64.19349022)(653.84727295,64.19349121)
\curveto(653.89726994,64.19349022)(653.94726989,64.18849023)(653.99727295,64.17849121)
\curveto(654.06726977,64.15849026)(654.15226969,64.14349027)(654.25227295,64.13349121)
\curveto(654.36226948,64.13349028)(654.45226939,64.14349027)(654.52227295,64.16349121)
\curveto(654.58226926,64.18349023)(654.6422692,64.18849023)(654.70227295,64.17849121)
\curveto(654.76226908,64.17849024)(654.82226902,64.18849023)(654.88227295,64.20849121)
\curveto(654.96226888,64.22849019)(655.0372688,64.24349017)(655.10727295,64.25349121)
\curveto(655.18726865,64.26349015)(655.26226858,64.28349013)(655.33227295,64.31349121)
\curveto(655.62226822,64.43348998)(655.86726797,64.57848984)(656.06727295,64.74849121)
\curveto(656.27726756,64.9184895)(656.4372674,65.14848927)(656.54727295,65.43849121)
}
}
{
\newrgbcolor{curcolor}{0 0 0}
\pscustom[linestyle=none,fillstyle=solid,fillcolor=curcolor]
{
\newpath
\moveto(663.50891357,71.12349121)
\curveto(663.73890878,71.12348329)(663.86890865,71.06348335)(663.89891357,70.94349121)
\curveto(663.92890859,70.83348358)(663.94390858,70.66848375)(663.94391357,70.44849121)
\lineto(663.94391357,70.16349121)
\curveto(663.94390858,70.07348434)(663.9189086,69.99848442)(663.86891357,69.93849121)
\curveto(663.80890871,69.85848456)(663.7239088,69.8134846)(663.61391357,69.80349121)
\curveto(663.50390902,69.80348461)(663.39390913,69.78848463)(663.28391357,69.75849121)
\curveto(663.14390938,69.72848469)(663.00890951,69.69848472)(662.87891357,69.66849121)
\curveto(662.75890976,69.63848478)(662.64390988,69.59848482)(662.53391357,69.54849121)
\curveto(662.24391028,69.418485)(662.00891051,69.23848518)(661.82891357,69.00849121)
\curveto(661.64891087,68.78848563)(661.49391103,68.53348588)(661.36391357,68.24349121)
\curveto(661.3239112,68.13348628)(661.29391123,68.0184864)(661.27391357,67.89849121)
\curveto(661.25391127,67.78848663)(661.22891129,67.67348674)(661.19891357,67.55349121)
\curveto(661.18891133,67.50348691)(661.18391134,67.45348696)(661.18391357,67.40349121)
\curveto(661.19391133,67.35348706)(661.19391133,67.30348711)(661.18391357,67.25349121)
\curveto(661.15391137,67.13348728)(661.13891138,66.99348742)(661.13891357,66.83349121)
\curveto(661.14891137,66.68348773)(661.15391137,66.53848788)(661.15391357,66.39849121)
\lineto(661.15391357,64.55349121)
\lineto(661.15391357,64.20849121)
\curveto(661.15391137,64.08849033)(661.14891137,63.97349044)(661.13891357,63.86349121)
\curveto(661.12891139,63.75349066)(661.1239114,63.65849076)(661.12391357,63.57849121)
\curveto(661.13391139,63.49849092)(661.11391141,63.42849099)(661.06391357,63.36849121)
\curveto(661.01391151,63.29849112)(660.93391159,63.25849116)(660.82391357,63.24849121)
\curveto(660.7239118,63.23849118)(660.61391191,63.23349118)(660.49391357,63.23349121)
\lineto(660.22391357,63.23349121)
\curveto(660.17391235,63.25349116)(660.1239124,63.26849115)(660.07391357,63.27849121)
\curveto(660.03391249,63.29849112)(660.00391252,63.32349109)(659.98391357,63.35349121)
\curveto(659.93391259,63.42349099)(659.90391262,63.50849091)(659.89391357,63.60849121)
\lineto(659.89391357,63.93849121)
\lineto(659.89391357,65.09349121)
\lineto(659.89391357,69.24849121)
\lineto(659.89391357,70.28349121)
\lineto(659.89391357,70.58349121)
\curveto(659.90391262,70.68348373)(659.93391259,70.76848365)(659.98391357,70.83849121)
\curveto(660.01391251,70.87848354)(660.06391246,70.90848351)(660.13391357,70.92849121)
\curveto(660.21391231,70.94848347)(660.29891222,70.95848346)(660.38891357,70.95849121)
\curveto(660.47891204,70.96848345)(660.56891195,70.96848345)(660.65891357,70.95849121)
\curveto(660.74891177,70.94848347)(660.8189117,70.93348348)(660.86891357,70.91349121)
\curveto(660.94891157,70.88348353)(660.99891152,70.82348359)(661.01891357,70.73349121)
\curveto(661.04891147,70.65348376)(661.06391146,70.56348385)(661.06391357,70.46349121)
\lineto(661.06391357,70.16349121)
\curveto(661.06391146,70.06348435)(661.08391144,69.97348444)(661.12391357,69.89349121)
\curveto(661.13391139,69.87348454)(661.14391138,69.85848456)(661.15391357,69.84849121)
\lineto(661.19891357,69.80349121)
\curveto(661.30891121,69.80348461)(661.39891112,69.84848457)(661.46891357,69.93849121)
\curveto(661.53891098,70.03848438)(661.59891092,70.1184843)(661.64891357,70.17849121)
\lineto(661.73891357,70.26849121)
\curveto(661.82891069,70.37848404)(661.95391057,70.49348392)(662.11391357,70.61349121)
\curveto(662.27391025,70.73348368)(662.4239101,70.82348359)(662.56391357,70.88349121)
\curveto(662.65390987,70.93348348)(662.74890977,70.96848345)(662.84891357,70.98849121)
\curveto(662.94890957,71.0184834)(663.05390947,71.04848337)(663.16391357,71.07849121)
\curveto(663.2239093,71.08848333)(663.28390924,71.09348332)(663.34391357,71.09349121)
\curveto(663.40390912,71.10348331)(663.45890906,71.1134833)(663.50891357,71.12349121)
}
}
{
\newrgbcolor{curcolor}{0 0 0}
\pscustom[linestyle=none,fillstyle=solid,fillcolor=curcolor]
{
\newpath
\moveto(668.5186792,71.12349121)
\curveto(668.74867441,71.12348329)(668.87867428,71.06348335)(668.9086792,70.94349121)
\curveto(668.93867422,70.83348358)(668.9536742,70.66848375)(668.9536792,70.44849121)
\lineto(668.9536792,70.16349121)
\curveto(668.9536742,70.07348434)(668.92867423,69.99848442)(668.8786792,69.93849121)
\curveto(668.81867434,69.85848456)(668.73367442,69.8134846)(668.6236792,69.80349121)
\curveto(668.51367464,69.80348461)(668.40367475,69.78848463)(668.2936792,69.75849121)
\curveto(668.153675,69.72848469)(668.01867514,69.69848472)(667.8886792,69.66849121)
\curveto(667.76867539,69.63848478)(667.6536755,69.59848482)(667.5436792,69.54849121)
\curveto(667.2536759,69.418485)(667.01867614,69.23848518)(666.8386792,69.00849121)
\curveto(666.6586765,68.78848563)(666.50367665,68.53348588)(666.3736792,68.24349121)
\curveto(666.33367682,68.13348628)(666.30367685,68.0184864)(666.2836792,67.89849121)
\curveto(666.26367689,67.78848663)(666.23867692,67.67348674)(666.2086792,67.55349121)
\curveto(666.19867696,67.50348691)(666.19367696,67.45348696)(666.1936792,67.40349121)
\curveto(666.20367695,67.35348706)(666.20367695,67.30348711)(666.1936792,67.25349121)
\curveto(666.16367699,67.13348728)(666.14867701,66.99348742)(666.1486792,66.83349121)
\curveto(666.158677,66.68348773)(666.16367699,66.53848788)(666.1636792,66.39849121)
\lineto(666.1636792,64.55349121)
\lineto(666.1636792,64.20849121)
\curveto(666.16367699,64.08849033)(666.158677,63.97349044)(666.1486792,63.86349121)
\curveto(666.13867702,63.75349066)(666.13367702,63.65849076)(666.1336792,63.57849121)
\curveto(666.14367701,63.49849092)(666.12367703,63.42849099)(666.0736792,63.36849121)
\curveto(666.02367713,63.29849112)(665.94367721,63.25849116)(665.8336792,63.24849121)
\curveto(665.73367742,63.23849118)(665.62367753,63.23349118)(665.5036792,63.23349121)
\lineto(665.2336792,63.23349121)
\curveto(665.18367797,63.25349116)(665.13367802,63.26849115)(665.0836792,63.27849121)
\curveto(665.04367811,63.29849112)(665.01367814,63.32349109)(664.9936792,63.35349121)
\curveto(664.94367821,63.42349099)(664.91367824,63.50849091)(664.9036792,63.60849121)
\lineto(664.9036792,63.93849121)
\lineto(664.9036792,65.09349121)
\lineto(664.9036792,69.24849121)
\lineto(664.9036792,70.28349121)
\lineto(664.9036792,70.58349121)
\curveto(664.91367824,70.68348373)(664.94367821,70.76848365)(664.9936792,70.83849121)
\curveto(665.02367813,70.87848354)(665.07367808,70.90848351)(665.1436792,70.92849121)
\curveto(665.22367793,70.94848347)(665.30867785,70.95848346)(665.3986792,70.95849121)
\curveto(665.48867767,70.96848345)(665.57867758,70.96848345)(665.6686792,70.95849121)
\curveto(665.7586774,70.94848347)(665.82867733,70.93348348)(665.8786792,70.91349121)
\curveto(665.9586772,70.88348353)(666.00867715,70.82348359)(666.0286792,70.73349121)
\curveto(666.0586771,70.65348376)(666.07367708,70.56348385)(666.0736792,70.46349121)
\lineto(666.0736792,70.16349121)
\curveto(666.07367708,70.06348435)(666.09367706,69.97348444)(666.1336792,69.89349121)
\curveto(666.14367701,69.87348454)(666.153677,69.85848456)(666.1636792,69.84849121)
\lineto(666.2086792,69.80349121)
\curveto(666.31867684,69.80348461)(666.40867675,69.84848457)(666.4786792,69.93849121)
\curveto(666.54867661,70.03848438)(666.60867655,70.1184843)(666.6586792,70.17849121)
\lineto(666.7486792,70.26849121)
\curveto(666.83867632,70.37848404)(666.96367619,70.49348392)(667.1236792,70.61349121)
\curveto(667.28367587,70.73348368)(667.43367572,70.82348359)(667.5736792,70.88349121)
\curveto(667.66367549,70.93348348)(667.7586754,70.96848345)(667.8586792,70.98849121)
\curveto(667.9586752,71.0184834)(668.06367509,71.04848337)(668.1736792,71.07849121)
\curveto(668.23367492,71.08848333)(668.29367486,71.09348332)(668.3536792,71.09349121)
\curveto(668.41367474,71.10348331)(668.46867469,71.1134833)(668.5186792,71.12349121)
}
}
{
\newrgbcolor{curcolor}{0 0 0}
\pscustom[linestyle=none,fillstyle=solid,fillcolor=curcolor]
{
\newpath
\moveto(677.00844482,67.41849121)
\curveto(677.02843676,67.35848706)(677.03843675,67.26348715)(677.03844482,67.13349121)
\curveto(677.03843675,67.0134874)(677.03343676,66.92848749)(677.02344482,66.87849121)
\lineto(677.02344482,66.72849121)
\curveto(677.01343678,66.64848777)(677.00343679,66.57348784)(676.99344482,66.50349121)
\curveto(676.9934368,66.44348797)(676.9884368,66.37348804)(676.97844482,66.29349121)
\curveto(676.95843683,66.23348818)(676.94343685,66.17348824)(676.93344482,66.11349121)
\curveto(676.93343686,66.05348836)(676.92343687,65.99348842)(676.90344482,65.93349121)
\curveto(676.86343693,65.80348861)(676.82843696,65.67348874)(676.79844482,65.54349121)
\curveto(676.76843702,65.413489)(676.72843706,65.29348912)(676.67844482,65.18349121)
\curveto(676.46843732,64.70348971)(676.1884376,64.29849012)(675.83844482,63.96849121)
\curveto(675.4884383,63.64849077)(675.05843873,63.40349101)(674.54844482,63.23349121)
\curveto(674.43843935,63.19349122)(674.31843947,63.16349125)(674.18844482,63.14349121)
\curveto(674.06843972,63.12349129)(673.94343985,63.10349131)(673.81344482,63.08349121)
\curveto(673.75344004,63.07349134)(673.6884401,63.06849135)(673.61844482,63.06849121)
\curveto(673.55844023,63.05849136)(673.49844029,63.05349136)(673.43844482,63.05349121)
\curveto(673.39844039,63.04349137)(673.33844045,63.03849138)(673.25844482,63.03849121)
\curveto(673.1884406,63.03849138)(673.13844065,63.04349137)(673.10844482,63.05349121)
\curveto(673.06844072,63.06349135)(673.02844076,63.06849135)(672.98844482,63.06849121)
\curveto(672.94844084,63.05849136)(672.91344088,63.05849136)(672.88344482,63.06849121)
\lineto(672.79344482,63.06849121)
\lineto(672.43344482,63.11349121)
\curveto(672.2934415,63.15349126)(672.15844163,63.19349122)(672.02844482,63.23349121)
\curveto(671.89844189,63.27349114)(671.77344202,63.3184911)(671.65344482,63.36849121)
\curveto(671.20344259,63.56849085)(670.83344296,63.82849059)(670.54344482,64.14849121)
\curveto(670.25344354,64.46848995)(670.01344378,64.85848956)(669.82344482,65.31849121)
\curveto(669.77344402,65.418489)(669.73344406,65.5184889)(669.70344482,65.61849121)
\curveto(669.68344411,65.7184887)(669.66344413,65.82348859)(669.64344482,65.93349121)
\curveto(669.62344417,65.97348844)(669.61344418,66.00348841)(669.61344482,66.02349121)
\curveto(669.62344417,66.05348836)(669.62344417,66.08848833)(669.61344482,66.12849121)
\curveto(669.5934442,66.20848821)(669.57844421,66.28848813)(669.56844482,66.36849121)
\curveto(669.56844422,66.45848796)(669.55844423,66.54348787)(669.53844482,66.62349121)
\lineto(669.53844482,66.74349121)
\curveto(669.53844425,66.78348763)(669.53344426,66.82848759)(669.52344482,66.87849121)
\curveto(669.51344428,66.92848749)(669.50844428,67.0134874)(669.50844482,67.13349121)
\curveto(669.50844428,67.26348715)(669.51844427,67.35848706)(669.53844482,67.41849121)
\curveto(669.55844423,67.48848693)(669.56344423,67.55848686)(669.55344482,67.62849121)
\curveto(669.54344425,67.69848672)(669.54844424,67.76848665)(669.56844482,67.83849121)
\curveto(669.57844421,67.88848653)(669.58344421,67.92848649)(669.58344482,67.95849121)
\curveto(669.5934442,67.99848642)(669.60344419,68.04348637)(669.61344482,68.09349121)
\curveto(669.64344415,68.2134862)(669.66844412,68.33348608)(669.68844482,68.45349121)
\curveto(669.71844407,68.57348584)(669.75844403,68.68848573)(669.80844482,68.79849121)
\curveto(669.95844383,69.16848525)(670.13844365,69.49848492)(670.34844482,69.78849121)
\curveto(670.56844322,70.08848433)(670.83344296,70.33848408)(671.14344482,70.53849121)
\curveto(671.26344253,70.6184838)(671.3884424,70.68348373)(671.51844482,70.73349121)
\curveto(671.64844214,70.79348362)(671.78344201,70.85348356)(671.92344482,70.91349121)
\curveto(672.04344175,70.96348345)(672.17344162,70.99348342)(672.31344482,71.00349121)
\curveto(672.45344134,71.02348339)(672.5934412,71.05348336)(672.73344482,71.09349121)
\lineto(672.92844482,71.09349121)
\curveto(672.99844079,71.10348331)(673.06344073,71.1134833)(673.12344482,71.12349121)
\curveto(674.01343978,71.13348328)(674.75343904,70.94848347)(675.34344482,70.56849121)
\curveto(675.93343786,70.18848423)(676.35843743,69.69348472)(676.61844482,69.08349121)
\curveto(676.66843712,68.98348543)(676.70843708,68.88348553)(676.73844482,68.78349121)
\curveto(676.76843702,68.68348573)(676.80343699,68.57848584)(676.84344482,68.46849121)
\curveto(676.87343692,68.35848606)(676.89843689,68.23848618)(676.91844482,68.10849121)
\curveto(676.93843685,67.98848643)(676.96343683,67.86348655)(676.99344482,67.73349121)
\curveto(677.00343679,67.68348673)(677.00343679,67.62848679)(676.99344482,67.56849121)
\curveto(676.9934368,67.5184869)(676.99843679,67.46848695)(677.00844482,67.41849121)
\moveto(675.67344482,66.56349121)
\curveto(675.6934381,66.63348778)(675.69843809,66.7134877)(675.68844482,66.80349121)
\lineto(675.68844482,67.05849121)
\curveto(675.6884381,67.44848697)(675.65343814,67.77848664)(675.58344482,68.04849121)
\curveto(675.55343824,68.12848629)(675.52843826,68.20848621)(675.50844482,68.28849121)
\curveto(675.4884383,68.36848605)(675.46343833,68.44348597)(675.43344482,68.51349121)
\curveto(675.15343864,69.16348525)(674.70843908,69.6134848)(674.09844482,69.86349121)
\curveto(674.02843976,69.89348452)(673.95343984,69.9134845)(673.87344482,69.92349121)
\lineto(673.63344482,69.98349121)
\curveto(673.55344024,70.00348441)(673.46844032,70.0134844)(673.37844482,70.01349121)
\lineto(673.10844482,70.01349121)
\lineto(672.83844482,69.96849121)
\curveto(672.73844105,69.94848447)(672.64344115,69.92348449)(672.55344482,69.89349121)
\curveto(672.47344132,69.87348454)(672.3934414,69.84348457)(672.31344482,69.80349121)
\curveto(672.24344155,69.78348463)(672.17844161,69.75348466)(672.11844482,69.71349121)
\curveto(672.05844173,69.67348474)(672.00344179,69.63348478)(671.95344482,69.59349121)
\curveto(671.71344208,69.42348499)(671.51844227,69.2184852)(671.36844482,68.97849121)
\curveto(671.21844257,68.73848568)(671.0884427,68.45848596)(670.97844482,68.13849121)
\curveto(670.94844284,68.03848638)(670.92844286,67.93348648)(670.91844482,67.82349121)
\curveto(670.90844288,67.72348669)(670.8934429,67.6184868)(670.87344482,67.50849121)
\curveto(670.86344293,67.46848695)(670.85844293,67.40348701)(670.85844482,67.31349121)
\curveto(670.84844294,67.28348713)(670.84344295,67.24848717)(670.84344482,67.20849121)
\curveto(670.85344294,67.16848725)(670.85844293,67.12348729)(670.85844482,67.07349121)
\lineto(670.85844482,66.77349121)
\curveto(670.85844293,66.67348774)(670.86844292,66.58348783)(670.88844482,66.50349121)
\lineto(670.91844482,66.32349121)
\curveto(670.93844285,66.22348819)(670.95344284,66.12348829)(670.96344482,66.02349121)
\curveto(670.98344281,65.93348848)(671.01344278,65.84848857)(671.05344482,65.76849121)
\curveto(671.15344264,65.52848889)(671.26844252,65.30348911)(671.39844482,65.09349121)
\curveto(671.53844225,64.88348953)(671.70844208,64.70848971)(671.90844482,64.56849121)
\curveto(671.95844183,64.53848988)(672.00344179,64.5134899)(672.04344482,64.49349121)
\curveto(672.08344171,64.47348994)(672.12844166,64.44848997)(672.17844482,64.41849121)
\curveto(672.25844153,64.36849005)(672.34344145,64.32349009)(672.43344482,64.28349121)
\curveto(672.53344126,64.25349016)(672.63844115,64.22349019)(672.74844482,64.19349121)
\curveto(672.79844099,64.17349024)(672.84344095,64.16349025)(672.88344482,64.16349121)
\curveto(672.93344086,64.17349024)(672.98344081,64.17349024)(673.03344482,64.16349121)
\curveto(673.06344073,64.15349026)(673.12344067,64.14349027)(673.21344482,64.13349121)
\curveto(673.31344048,64.12349029)(673.3884404,64.12849029)(673.43844482,64.14849121)
\curveto(673.47844031,64.15849026)(673.51844027,64.15849026)(673.55844482,64.14849121)
\curveto(673.59844019,64.14849027)(673.63844015,64.15849026)(673.67844482,64.17849121)
\curveto(673.75844003,64.19849022)(673.83843995,64.2134902)(673.91844482,64.22349121)
\curveto(673.99843979,64.24349017)(674.07343972,64.26849015)(674.14344482,64.29849121)
\curveto(674.48343931,64.43848998)(674.75843903,64.63348978)(674.96844482,64.88349121)
\curveto(675.17843861,65.13348928)(675.35343844,65.42848899)(675.49344482,65.76849121)
\curveto(675.54343825,65.88848853)(675.57343822,66.0134884)(675.58344482,66.14349121)
\curveto(675.60343819,66.28348813)(675.63343816,66.42348799)(675.67344482,66.56349121)
}
}
{
\newrgbcolor{curcolor}{0 0 0}
\pscustom[linestyle=none,fillstyle=solid,fillcolor=curcolor]
{
\newpath
\moveto(679.06672607,73.89849121)
\curveto(679.19672446,73.89848052)(679.33172432,73.89848052)(679.47172607,73.89849121)
\curveto(679.62172403,73.89848052)(679.73172392,73.86348055)(679.80172607,73.79349121)
\curveto(679.8517238,73.72348069)(679.87672378,73.62848079)(679.87672607,73.50849121)
\curveto(679.88672377,73.39848102)(679.89172376,73.28348113)(679.89172607,73.16349121)
\lineto(679.89172607,71.82849121)
\lineto(679.89172607,65.75349121)
\lineto(679.89172607,64.07349121)
\lineto(679.89172607,63.68349121)
\curveto(679.89172376,63.54349087)(679.86672379,63.43349098)(679.81672607,63.35349121)
\curveto(679.78672387,63.30349111)(679.74172391,63.27349114)(679.68172607,63.26349121)
\curveto(679.63172402,63.25349116)(679.56672409,63.23849118)(679.48672607,63.21849121)
\lineto(679.27672607,63.21849121)
\lineto(678.96172607,63.21849121)
\curveto(678.86172479,63.22849119)(678.78672487,63.26349115)(678.73672607,63.32349121)
\curveto(678.68672497,63.40349101)(678.656725,63.50349091)(678.64672607,63.62349121)
\lineto(678.64672607,63.99849121)
\lineto(678.64672607,65.37849121)
\lineto(678.64672607,71.61849121)
\lineto(678.64672607,73.08849121)
\curveto(678.64672501,73.19848122)(678.64172501,73.3134811)(678.63172607,73.43349121)
\curveto(678.63172502,73.56348085)(678.656725,73.66348075)(678.70672607,73.73349121)
\curveto(678.74672491,73.79348062)(678.82172483,73.84348057)(678.93172607,73.88349121)
\curveto(678.9517247,73.89348052)(678.97172468,73.89348052)(678.99172607,73.88349121)
\curveto(679.02172463,73.88348053)(679.04672461,73.88848053)(679.06672607,73.89849121)
}
}
{
\newrgbcolor{curcolor}{0 0 0}
\pscustom[linestyle=none,fillstyle=solid,fillcolor=curcolor]
{
\newpath
\moveto(682.40656982,73.89849121)
\curveto(682.53656821,73.89848052)(682.67156807,73.89848052)(682.81156982,73.89849121)
\curveto(682.96156778,73.89848052)(683.07156767,73.86348055)(683.14156982,73.79349121)
\curveto(683.19156755,73.72348069)(683.21656753,73.62848079)(683.21656982,73.50849121)
\curveto(683.22656752,73.39848102)(683.23156751,73.28348113)(683.23156982,73.16349121)
\lineto(683.23156982,71.82849121)
\lineto(683.23156982,65.75349121)
\lineto(683.23156982,64.07349121)
\lineto(683.23156982,63.68349121)
\curveto(683.23156751,63.54349087)(683.20656754,63.43349098)(683.15656982,63.35349121)
\curveto(683.12656762,63.30349111)(683.08156766,63.27349114)(683.02156982,63.26349121)
\curveto(682.97156777,63.25349116)(682.90656784,63.23849118)(682.82656982,63.21849121)
\lineto(682.61656982,63.21849121)
\lineto(682.30156982,63.21849121)
\curveto(682.20156854,63.22849119)(682.12656862,63.26349115)(682.07656982,63.32349121)
\curveto(682.02656872,63.40349101)(681.99656875,63.50349091)(681.98656982,63.62349121)
\lineto(681.98656982,63.99849121)
\lineto(681.98656982,65.37849121)
\lineto(681.98656982,71.61849121)
\lineto(681.98656982,73.08849121)
\curveto(681.98656876,73.19848122)(681.98156876,73.3134811)(681.97156982,73.43349121)
\curveto(681.97156877,73.56348085)(681.99656875,73.66348075)(682.04656982,73.73349121)
\curveto(682.08656866,73.79348062)(682.16156858,73.84348057)(682.27156982,73.88349121)
\curveto(682.29156845,73.89348052)(682.31156843,73.89348052)(682.33156982,73.88349121)
\curveto(682.36156838,73.88348053)(682.38656836,73.88848053)(682.40656982,73.89849121)
}
}
{
\newrgbcolor{curcolor}{0 0 0}
\pscustom[linestyle=none,fillstyle=solid,fillcolor=curcolor]
{
\newpath
\moveto(692.06141357,63.77349121)
\curveto(692.09140574,63.6134908)(692.07640576,63.47849094)(692.01641357,63.36849121)
\curveto(691.95640588,63.26849115)(691.87640596,63.19349122)(691.77641357,63.14349121)
\curveto(691.72640611,63.12349129)(691.67140616,63.1134913)(691.61141357,63.11349121)
\curveto(691.56140627,63.1134913)(691.50640633,63.10349131)(691.44641357,63.08349121)
\curveto(691.22640661,63.03349138)(691.00640683,63.04849137)(690.78641357,63.12849121)
\curveto(690.57640726,63.19849122)(690.4314074,63.28849113)(690.35141357,63.39849121)
\curveto(690.30140753,63.46849095)(690.25640758,63.54849087)(690.21641357,63.63849121)
\curveto(690.17640766,63.73849068)(690.12640771,63.8184906)(690.06641357,63.87849121)
\curveto(690.04640779,63.89849052)(690.02140781,63.9184905)(689.99141357,63.93849121)
\curveto(689.97140786,63.95849046)(689.94140789,63.96349045)(689.90141357,63.95349121)
\curveto(689.79140804,63.92349049)(689.68640815,63.86849055)(689.58641357,63.78849121)
\curveto(689.49640834,63.70849071)(689.40640843,63.63849078)(689.31641357,63.57849121)
\curveto(689.18640865,63.49849092)(689.04640879,63.42349099)(688.89641357,63.35349121)
\curveto(688.74640909,63.29349112)(688.58640925,63.23849118)(688.41641357,63.18849121)
\curveto(688.31640952,63.15849126)(688.20640963,63.13849128)(688.08641357,63.12849121)
\curveto(687.97640986,63.1184913)(687.86640997,63.10349131)(687.75641357,63.08349121)
\curveto(687.70641013,63.07349134)(687.66141017,63.06849135)(687.62141357,63.06849121)
\lineto(687.51641357,63.06849121)
\curveto(687.40641043,63.04849137)(687.30141053,63.04849137)(687.20141357,63.06849121)
\lineto(687.06641357,63.06849121)
\curveto(687.01641082,63.07849134)(686.96641087,63.08349133)(686.91641357,63.08349121)
\curveto(686.86641097,63.08349133)(686.82141101,63.09349132)(686.78141357,63.11349121)
\curveto(686.74141109,63.12349129)(686.70641113,63.12849129)(686.67641357,63.12849121)
\curveto(686.65641118,63.1184913)(686.6314112,63.1184913)(686.60141357,63.12849121)
\lineto(686.36141357,63.18849121)
\curveto(686.28141155,63.19849122)(686.20641163,63.2184912)(686.13641357,63.24849121)
\curveto(685.836412,63.37849104)(685.59141224,63.52349089)(685.40141357,63.68349121)
\curveto(685.22141261,63.85349056)(685.07141276,64.08849033)(684.95141357,64.38849121)
\curveto(684.86141297,64.60848981)(684.81641302,64.87348954)(684.81641357,65.18349121)
\lineto(684.81641357,65.49849121)
\curveto(684.82641301,65.54848887)(684.831413,65.59848882)(684.83141357,65.64849121)
\lineto(684.86141357,65.82849121)
\lineto(684.98141357,66.15849121)
\curveto(685.02141281,66.26848815)(685.07141276,66.36848805)(685.13141357,66.45849121)
\curveto(685.31141252,66.74848767)(685.55641228,66.96348745)(685.86641357,67.10349121)
\curveto(686.17641166,67.24348717)(686.51641132,67.36848705)(686.88641357,67.47849121)
\curveto(687.02641081,67.5184869)(687.17141066,67.54848687)(687.32141357,67.56849121)
\curveto(687.47141036,67.58848683)(687.62141021,67.6134868)(687.77141357,67.64349121)
\curveto(687.84140999,67.66348675)(687.90640993,67.67348674)(687.96641357,67.67349121)
\curveto(688.0364098,67.67348674)(688.11140972,67.68348673)(688.19141357,67.70349121)
\curveto(688.26140957,67.72348669)(688.3314095,67.73348668)(688.40141357,67.73349121)
\curveto(688.47140936,67.74348667)(688.54640929,67.75848666)(688.62641357,67.77849121)
\curveto(688.87640896,67.83848658)(689.11140872,67.88848653)(689.33141357,67.92849121)
\curveto(689.55140828,67.97848644)(689.72640811,68.09348632)(689.85641357,68.27349121)
\curveto(689.91640792,68.35348606)(689.96640787,68.45348596)(690.00641357,68.57349121)
\curveto(690.04640779,68.70348571)(690.04640779,68.84348557)(690.00641357,68.99349121)
\curveto(689.94640789,69.23348518)(689.85640798,69.42348499)(689.73641357,69.56349121)
\curveto(689.62640821,69.70348471)(689.46640837,69.8134846)(689.25641357,69.89349121)
\curveto(689.1364087,69.94348447)(688.99140884,69.97848444)(688.82141357,69.99849121)
\curveto(688.66140917,70.0184844)(688.49140934,70.02848439)(688.31141357,70.02849121)
\curveto(688.1314097,70.02848439)(687.95640988,70.0184844)(687.78641357,69.99849121)
\curveto(687.61641022,69.97848444)(687.47141036,69.94848447)(687.35141357,69.90849121)
\curveto(687.18141065,69.84848457)(687.01641082,69.76348465)(686.85641357,69.65349121)
\curveto(686.77641106,69.59348482)(686.70141113,69.5134849)(686.63141357,69.41349121)
\curveto(686.57141126,69.32348509)(686.51641132,69.22348519)(686.46641357,69.11349121)
\curveto(686.4364114,69.03348538)(686.40641143,68.94848547)(686.37641357,68.85849121)
\curveto(686.35641148,68.76848565)(686.31141152,68.69848572)(686.24141357,68.64849121)
\curveto(686.20141163,68.6184858)(686.1314117,68.59348582)(686.03141357,68.57349121)
\curveto(685.94141189,68.56348585)(685.84641199,68.55848586)(685.74641357,68.55849121)
\curveto(685.64641219,68.55848586)(685.54641229,68.56348585)(685.44641357,68.57349121)
\curveto(685.35641248,68.59348582)(685.29141254,68.6184858)(685.25141357,68.64849121)
\curveto(685.21141262,68.67848574)(685.18141265,68.72848569)(685.16141357,68.79849121)
\curveto(685.14141269,68.86848555)(685.14141269,68.94348547)(685.16141357,69.02349121)
\curveto(685.19141264,69.15348526)(685.22141261,69.27348514)(685.25141357,69.38349121)
\curveto(685.29141254,69.50348491)(685.3364125,69.6184848)(685.38641357,69.72849121)
\curveto(685.57641226,70.07848434)(685.81641202,70.34848407)(686.10641357,70.53849121)
\curveto(686.39641144,70.73848368)(686.75641108,70.89848352)(687.18641357,71.01849121)
\curveto(687.28641055,71.03848338)(687.38641045,71.05348336)(687.48641357,71.06349121)
\curveto(687.59641024,71.07348334)(687.70641013,71.08848333)(687.81641357,71.10849121)
\curveto(687.85640998,71.1184833)(687.92140991,71.1184833)(688.01141357,71.10849121)
\curveto(688.10140973,71.10848331)(688.15640968,71.1184833)(688.17641357,71.13849121)
\curveto(688.87640896,71.14848327)(689.48640835,71.06848335)(690.00641357,70.89849121)
\curveto(690.52640731,70.72848369)(690.89140694,70.40348401)(691.10141357,69.92349121)
\curveto(691.19140664,69.72348469)(691.24140659,69.48848493)(691.25141357,69.21849121)
\curveto(691.27140656,68.95848546)(691.28140655,68.68348573)(691.28141357,68.39349121)
\lineto(691.28141357,65.07849121)
\curveto(691.28140655,64.93848948)(691.28640655,64.80348961)(691.29641357,64.67349121)
\curveto(691.30640653,64.54348987)(691.3364065,64.43848998)(691.38641357,64.35849121)
\curveto(691.4364064,64.28849013)(691.50140633,64.23849018)(691.58141357,64.20849121)
\curveto(691.67140616,64.16849025)(691.75640608,64.13849028)(691.83641357,64.11849121)
\curveto(691.91640592,64.10849031)(691.97640586,64.06349035)(692.01641357,63.98349121)
\curveto(692.0364058,63.95349046)(692.04640579,63.92349049)(692.04641357,63.89349121)
\curveto(692.04640579,63.86349055)(692.05140578,63.82349059)(692.06141357,63.77349121)
\moveto(689.91641357,65.43849121)
\curveto(689.97640786,65.57848884)(690.00640783,65.73848868)(690.00641357,65.91849121)
\curveto(690.01640782,66.10848831)(690.02140781,66.30348811)(690.02141357,66.50349121)
\curveto(690.02140781,66.6134878)(690.01640782,66.7134877)(690.00641357,66.80349121)
\curveto(689.99640784,66.89348752)(689.95640788,66.96348745)(689.88641357,67.01349121)
\curveto(689.85640798,67.03348738)(689.78640805,67.04348737)(689.67641357,67.04349121)
\curveto(689.65640818,67.02348739)(689.62140821,67.0134874)(689.57141357,67.01349121)
\curveto(689.52140831,67.0134874)(689.47640836,67.00348741)(689.43641357,66.98349121)
\curveto(689.35640848,66.96348745)(689.26640857,66.94348747)(689.16641357,66.92349121)
\lineto(688.86641357,66.86349121)
\curveto(688.836409,66.86348755)(688.80140903,66.85848756)(688.76141357,66.84849121)
\lineto(688.65641357,66.84849121)
\curveto(688.50640933,66.80848761)(688.34140949,66.78348763)(688.16141357,66.77349121)
\curveto(687.99140984,66.77348764)(687.83141,66.75348766)(687.68141357,66.71349121)
\curveto(687.60141023,66.69348772)(687.52641031,66.67348774)(687.45641357,66.65349121)
\curveto(687.39641044,66.64348777)(687.32641051,66.62848779)(687.24641357,66.60849121)
\curveto(687.08641075,66.55848786)(686.9364109,66.49348792)(686.79641357,66.41349121)
\curveto(686.65641118,66.34348807)(686.5364113,66.25348816)(686.43641357,66.14349121)
\curveto(686.3364115,66.03348838)(686.26141157,65.89848852)(686.21141357,65.73849121)
\curveto(686.16141167,65.58848883)(686.14141169,65.40348901)(686.15141357,65.18349121)
\curveto(686.15141168,65.08348933)(686.16641167,64.98848943)(686.19641357,64.89849121)
\curveto(686.2364116,64.8184896)(686.28141155,64.74348967)(686.33141357,64.67349121)
\curveto(686.41141142,64.56348985)(686.51641132,64.46848995)(686.64641357,64.38849121)
\curveto(686.77641106,64.3184901)(686.91641092,64.25849016)(687.06641357,64.20849121)
\curveto(687.11641072,64.19849022)(687.16641067,64.19349022)(687.21641357,64.19349121)
\curveto(687.26641057,64.19349022)(687.31641052,64.18849023)(687.36641357,64.17849121)
\curveto(687.4364104,64.15849026)(687.52141031,64.14349027)(687.62141357,64.13349121)
\curveto(687.7314101,64.13349028)(687.82141001,64.14349027)(687.89141357,64.16349121)
\curveto(687.95140988,64.18349023)(688.01140982,64.18849023)(688.07141357,64.17849121)
\curveto(688.1314097,64.17849024)(688.19140964,64.18849023)(688.25141357,64.20849121)
\curveto(688.3314095,64.22849019)(688.40640943,64.24349017)(688.47641357,64.25349121)
\curveto(688.55640928,64.26349015)(688.6314092,64.28349013)(688.70141357,64.31349121)
\curveto(688.99140884,64.43348998)(689.2364086,64.57848984)(689.43641357,64.74849121)
\curveto(689.64640819,64.9184895)(689.80640803,65.14848927)(689.91641357,65.43849121)
}
}
{
\newrgbcolor{curcolor}{0 0 0}
\pscustom[linestyle=none,fillstyle=solid,fillcolor=curcolor]
{
\newpath
\moveto(700.1930542,64.02849121)
\lineto(700.1930542,63.63849121)
\curveto(700.19304632,63.5184909)(700.16804635,63.418491)(700.1180542,63.33849121)
\curveto(700.06804645,63.26849115)(699.98304653,63.22849119)(699.8630542,63.21849121)
\lineto(699.5180542,63.21849121)
\curveto(699.45804706,63.2184912)(699.39804712,63.2134912)(699.3380542,63.20349121)
\curveto(699.28804723,63.20349121)(699.24304727,63.2134912)(699.2030542,63.23349121)
\curveto(699.1130474,63.25349116)(699.05304746,63.29349112)(699.0230542,63.35349121)
\curveto(698.98304753,63.40349101)(698.95804756,63.46349095)(698.9480542,63.53349121)
\curveto(698.94804757,63.60349081)(698.93304758,63.67349074)(698.9030542,63.74349121)
\curveto(698.89304762,63.76349065)(698.87804764,63.77849064)(698.8580542,63.78849121)
\curveto(698.84804767,63.80849061)(698.83304768,63.82849059)(698.8130542,63.84849121)
\curveto(698.7130478,63.85849056)(698.63304788,63.83849058)(698.5730542,63.78849121)
\curveto(698.52304799,63.73849068)(698.46804805,63.68849073)(698.4080542,63.63849121)
\curveto(698.20804831,63.48849093)(698.00804851,63.37349104)(697.8080542,63.29349121)
\curveto(697.62804889,63.2134912)(697.4180491,63.15349126)(697.1780542,63.11349121)
\curveto(696.94804957,63.07349134)(696.70804981,63.05349136)(696.4580542,63.05349121)
\curveto(696.2180503,63.04349137)(695.97805054,63.05849136)(695.7380542,63.09849121)
\curveto(695.49805102,63.12849129)(695.28805123,63.18349123)(695.1080542,63.26349121)
\curveto(694.58805193,63.48349093)(694.16805235,63.77849064)(693.8480542,64.14849121)
\curveto(693.52805299,64.52848989)(693.27805324,64.99848942)(693.0980542,65.55849121)
\curveto(693.05805346,65.64848877)(693.02805349,65.73848868)(693.0080542,65.82849121)
\curveto(692.99805352,65.92848849)(692.97805354,66.02848839)(692.9480542,66.12849121)
\curveto(692.93805358,66.17848824)(692.93305358,66.22848819)(692.9330542,66.27849121)
\curveto(692.93305358,66.32848809)(692.92805359,66.37848804)(692.9180542,66.42849121)
\curveto(692.89805362,66.47848794)(692.88805363,66.52848789)(692.8880542,66.57849121)
\curveto(692.89805362,66.63848778)(692.89805362,66.69348772)(692.8880542,66.74349121)
\lineto(692.8880542,66.89349121)
\curveto(692.86805365,66.94348747)(692.85805366,67.00848741)(692.8580542,67.08849121)
\curveto(692.85805366,67.16848725)(692.86805365,67.23348718)(692.8880542,67.28349121)
\lineto(692.8880542,67.44849121)
\curveto(692.90805361,67.5184869)(692.9130536,67.58848683)(692.9030542,67.65849121)
\curveto(692.90305361,67.73848668)(692.9130536,67.8134866)(692.9330542,67.88349121)
\curveto(692.94305357,67.93348648)(692.94805357,67.97848644)(692.9480542,68.01849121)
\curveto(692.94805357,68.05848636)(692.95305356,68.10348631)(692.9630542,68.15349121)
\curveto(692.99305352,68.25348616)(693.0180535,68.34848607)(693.0380542,68.43849121)
\curveto(693.05805346,68.53848588)(693.08305343,68.63348578)(693.1130542,68.72349121)
\curveto(693.24305327,69.10348531)(693.40805311,69.44348497)(693.6080542,69.74349121)
\curveto(693.8180527,70.05348436)(694.06805245,70.30848411)(694.3580542,70.50849121)
\curveto(694.52805199,70.62848379)(694.70305181,70.72848369)(694.8830542,70.80849121)
\curveto(695.07305144,70.88848353)(695.27805124,70.95848346)(695.4980542,71.01849121)
\curveto(695.56805095,71.02848339)(695.63305088,71.03848338)(695.6930542,71.04849121)
\curveto(695.76305075,71.05848336)(695.83305068,71.07348334)(695.9030542,71.09349121)
\lineto(696.0530542,71.09349121)
\curveto(696.13305038,71.1134833)(696.24805027,71.12348329)(696.3980542,71.12349121)
\curveto(696.55804996,71.12348329)(696.67804984,71.1134833)(696.7580542,71.09349121)
\curveto(696.79804972,71.08348333)(696.85304966,71.07848334)(696.9230542,71.07849121)
\curveto(697.03304948,71.04848337)(697.14304937,71.02348339)(697.2530542,71.00349121)
\curveto(697.36304915,70.99348342)(697.46804905,70.96348345)(697.5680542,70.91349121)
\curveto(697.7180488,70.85348356)(697.85804866,70.78848363)(697.9880542,70.71849121)
\curveto(698.12804839,70.64848377)(698.25804826,70.56848385)(698.3780542,70.47849121)
\curveto(698.43804808,70.42848399)(698.49804802,70.37348404)(698.5580542,70.31349121)
\curveto(698.62804789,70.26348415)(698.7180478,70.24848417)(698.8280542,70.26849121)
\curveto(698.84804767,70.29848412)(698.86304765,70.32348409)(698.8730542,70.34349121)
\curveto(698.89304762,70.36348405)(698.90804761,70.39348402)(698.9180542,70.43349121)
\curveto(698.94804757,70.52348389)(698.95804756,70.63848378)(698.9480542,70.77849121)
\lineto(698.9480542,71.15349121)
\lineto(698.9480542,72.87849121)
\lineto(698.9480542,73.34349121)
\curveto(698.94804757,73.52348089)(698.97304754,73.65348076)(699.0230542,73.73349121)
\curveto(699.06304745,73.80348061)(699.12304739,73.84848057)(699.2030542,73.86849121)
\curveto(699.22304729,73.86848055)(699.24804727,73.86848055)(699.2780542,73.86849121)
\curveto(699.30804721,73.87848054)(699.33304718,73.88348053)(699.3530542,73.88349121)
\curveto(699.49304702,73.89348052)(699.63804688,73.89348052)(699.7880542,73.88349121)
\curveto(699.94804657,73.88348053)(700.05804646,73.84348057)(700.1180542,73.76349121)
\curveto(700.16804635,73.68348073)(700.19304632,73.58348083)(700.1930542,73.46349121)
\lineto(700.1930542,73.08849121)
\lineto(700.1930542,64.02849121)
\moveto(698.9780542,66.86349121)
\curveto(698.99804752,66.9134875)(699.00804751,66.97848744)(699.0080542,67.05849121)
\curveto(699.00804751,67.14848727)(698.99804752,67.2184872)(698.9780542,67.26849121)
\lineto(698.9780542,67.49349121)
\curveto(698.95804756,67.58348683)(698.94304757,67.67348674)(698.9330542,67.76349121)
\curveto(698.92304759,67.86348655)(698.90304761,67.95348646)(698.8730542,68.03349121)
\curveto(698.85304766,68.1134863)(698.83304768,68.18848623)(698.8130542,68.25849121)
\curveto(698.80304771,68.32848609)(698.78304773,68.39848602)(698.7530542,68.46849121)
\curveto(698.63304788,68.76848565)(698.47804804,69.03348538)(698.2880542,69.26349121)
\curveto(698.09804842,69.49348492)(697.85804866,69.67348474)(697.5680542,69.80349121)
\curveto(697.46804905,69.85348456)(697.36304915,69.88848453)(697.2530542,69.90849121)
\curveto(697.15304936,69.93848448)(697.04304947,69.96348445)(696.9230542,69.98349121)
\curveto(696.84304967,70.00348441)(696.75304976,70.0134844)(696.6530542,70.01349121)
\lineto(696.3830542,70.01349121)
\curveto(696.33305018,70.00348441)(696.28805023,69.99348442)(696.2480542,69.98349121)
\lineto(696.1130542,69.98349121)
\curveto(696.03305048,69.96348445)(695.94805057,69.94348447)(695.8580542,69.92349121)
\curveto(695.77805074,69.90348451)(695.69805082,69.87848454)(695.6180542,69.84849121)
\curveto(695.29805122,69.70848471)(695.03805148,69.50348491)(694.8380542,69.23349121)
\curveto(694.64805187,68.97348544)(694.49305202,68.66848575)(694.3730542,68.31849121)
\curveto(694.33305218,68.20848621)(694.30305221,68.09348632)(694.2830542,67.97349121)
\curveto(694.27305224,67.86348655)(694.25805226,67.75348666)(694.2380542,67.64349121)
\curveto(694.23805228,67.60348681)(694.23305228,67.56348685)(694.2230542,67.52349121)
\lineto(694.2230542,67.41849121)
\curveto(694.20305231,67.36848705)(694.19305232,67.3134871)(694.1930542,67.25349121)
\curveto(694.20305231,67.19348722)(694.20805231,67.13848728)(694.2080542,67.08849121)
\lineto(694.2080542,66.75849121)
\curveto(694.20805231,66.65848776)(694.2180523,66.56348785)(694.2380542,66.47349121)
\curveto(694.24805227,66.44348797)(694.25305226,66.39348802)(694.2530542,66.32349121)
\curveto(694.27305224,66.25348816)(694.28805223,66.18348823)(694.2980542,66.11349121)
\lineto(694.3580542,65.90349121)
\curveto(694.46805205,65.55348886)(694.6180519,65.25348916)(694.8080542,65.00349121)
\curveto(694.99805152,64.75348966)(695.23805128,64.54848987)(695.5280542,64.38849121)
\curveto(695.6180509,64.33849008)(695.70805081,64.29849012)(695.7980542,64.26849121)
\curveto(695.88805063,64.23849018)(695.98805053,64.20849021)(696.0980542,64.17849121)
\curveto(696.14805037,64.15849026)(696.19805032,64.15349026)(696.2480542,64.16349121)
\curveto(696.30805021,64.17349024)(696.36305015,64.16849025)(696.4130542,64.14849121)
\curveto(696.45305006,64.13849028)(696.49305002,64.13349028)(696.5330542,64.13349121)
\lineto(696.6680542,64.13349121)
\lineto(696.8030542,64.13349121)
\curveto(696.83304968,64.14349027)(696.88304963,64.14849027)(696.9530542,64.14849121)
\curveto(697.03304948,64.16849025)(697.1130494,64.18349023)(697.1930542,64.19349121)
\curveto(697.27304924,64.2134902)(697.34804917,64.23849018)(697.4180542,64.26849121)
\curveto(697.74804877,64.40849001)(698.0130485,64.58348983)(698.2130542,64.79349121)
\curveto(698.42304809,65.0134894)(698.59804792,65.28848913)(698.7380542,65.61849121)
\curveto(698.78804773,65.72848869)(698.82304769,65.83848858)(698.8430542,65.94849121)
\curveto(698.86304765,66.05848836)(698.88804763,66.16848825)(698.9180542,66.27849121)
\curveto(698.93804758,66.3184881)(698.94804757,66.35348806)(698.9480542,66.38349121)
\curveto(698.94804757,66.42348799)(698.95304756,66.46348795)(698.9630542,66.50349121)
\curveto(698.97304754,66.56348785)(698.97304754,66.62348779)(698.9630542,66.68349121)
\curveto(698.96304755,66.74348767)(698.96804755,66.80348761)(698.9780542,66.86349121)
}
}
{
\newrgbcolor{curcolor}{0 0 0}
\pscustom[linestyle=none,fillstyle=solid,fillcolor=curcolor]
{
\newpath
\moveto(709.2643042,67.41849121)
\curveto(709.28429614,67.35848706)(709.29429613,67.26348715)(709.2943042,67.13349121)
\curveto(709.29429613,67.0134874)(709.28929613,66.92848749)(709.2793042,66.87849121)
\lineto(709.2793042,66.72849121)
\curveto(709.26929615,66.64848777)(709.25929616,66.57348784)(709.2493042,66.50349121)
\curveto(709.24929617,66.44348797)(709.24429618,66.37348804)(709.2343042,66.29349121)
\curveto(709.21429621,66.23348818)(709.19929622,66.17348824)(709.1893042,66.11349121)
\curveto(709.18929623,66.05348836)(709.17929624,65.99348842)(709.1593042,65.93349121)
\curveto(709.1192963,65.80348861)(709.08429634,65.67348874)(709.0543042,65.54349121)
\curveto(709.0242964,65.413489)(708.98429644,65.29348912)(708.9343042,65.18349121)
\curveto(708.7242967,64.70348971)(708.44429698,64.29849012)(708.0943042,63.96849121)
\curveto(707.74429768,63.64849077)(707.31429811,63.40349101)(706.8043042,63.23349121)
\curveto(706.69429873,63.19349122)(706.57429885,63.16349125)(706.4443042,63.14349121)
\curveto(706.3242991,63.12349129)(706.19929922,63.10349131)(706.0693042,63.08349121)
\curveto(706.00929941,63.07349134)(705.94429948,63.06849135)(705.8743042,63.06849121)
\curveto(705.81429961,63.05849136)(705.75429967,63.05349136)(705.6943042,63.05349121)
\curveto(705.65429977,63.04349137)(705.59429983,63.03849138)(705.5143042,63.03849121)
\curveto(705.44429998,63.03849138)(705.39430003,63.04349137)(705.3643042,63.05349121)
\curveto(705.3243001,63.06349135)(705.28430014,63.06849135)(705.2443042,63.06849121)
\curveto(705.20430022,63.05849136)(705.16930025,63.05849136)(705.1393042,63.06849121)
\lineto(705.0493042,63.06849121)
\lineto(704.6893042,63.11349121)
\curveto(704.54930087,63.15349126)(704.41430101,63.19349122)(704.2843042,63.23349121)
\curveto(704.15430127,63.27349114)(704.02930139,63.3184911)(703.9093042,63.36849121)
\curveto(703.45930196,63.56849085)(703.08930233,63.82849059)(702.7993042,64.14849121)
\curveto(702.50930291,64.46848995)(702.26930315,64.85848956)(702.0793042,65.31849121)
\curveto(702.02930339,65.418489)(701.98930343,65.5184889)(701.9593042,65.61849121)
\curveto(701.93930348,65.7184887)(701.9193035,65.82348859)(701.8993042,65.93349121)
\curveto(701.87930354,65.97348844)(701.86930355,66.00348841)(701.8693042,66.02349121)
\curveto(701.87930354,66.05348836)(701.87930354,66.08848833)(701.8693042,66.12849121)
\curveto(701.84930357,66.20848821)(701.83430359,66.28848813)(701.8243042,66.36849121)
\curveto(701.8243036,66.45848796)(701.81430361,66.54348787)(701.7943042,66.62349121)
\lineto(701.7943042,66.74349121)
\curveto(701.79430363,66.78348763)(701.78930363,66.82848759)(701.7793042,66.87849121)
\curveto(701.76930365,66.92848749)(701.76430366,67.0134874)(701.7643042,67.13349121)
\curveto(701.76430366,67.26348715)(701.77430365,67.35848706)(701.7943042,67.41849121)
\curveto(701.81430361,67.48848693)(701.8193036,67.55848686)(701.8093042,67.62849121)
\curveto(701.79930362,67.69848672)(701.80430362,67.76848665)(701.8243042,67.83849121)
\curveto(701.83430359,67.88848653)(701.83930358,67.92848649)(701.8393042,67.95849121)
\curveto(701.84930357,67.99848642)(701.85930356,68.04348637)(701.8693042,68.09349121)
\curveto(701.89930352,68.2134862)(701.9243035,68.33348608)(701.9443042,68.45349121)
\curveto(701.97430345,68.57348584)(702.01430341,68.68848573)(702.0643042,68.79849121)
\curveto(702.21430321,69.16848525)(702.39430303,69.49848492)(702.6043042,69.78849121)
\curveto(702.8243026,70.08848433)(703.08930233,70.33848408)(703.3993042,70.53849121)
\curveto(703.5193019,70.6184838)(703.64430178,70.68348373)(703.7743042,70.73349121)
\curveto(703.90430152,70.79348362)(704.03930138,70.85348356)(704.1793042,70.91349121)
\curveto(704.29930112,70.96348345)(704.42930099,70.99348342)(704.5693042,71.00349121)
\curveto(704.70930071,71.02348339)(704.84930057,71.05348336)(704.9893042,71.09349121)
\lineto(705.1843042,71.09349121)
\curveto(705.25430017,71.10348331)(705.3193001,71.1134833)(705.3793042,71.12349121)
\curveto(706.26929915,71.13348328)(707.00929841,70.94848347)(707.5993042,70.56849121)
\curveto(708.18929723,70.18848423)(708.61429681,69.69348472)(708.8743042,69.08349121)
\curveto(708.9242965,68.98348543)(708.96429646,68.88348553)(708.9943042,68.78349121)
\curveto(709.0242964,68.68348573)(709.05929636,68.57848584)(709.0993042,68.46849121)
\curveto(709.12929629,68.35848606)(709.15429627,68.23848618)(709.1743042,68.10849121)
\curveto(709.19429623,67.98848643)(709.2192962,67.86348655)(709.2493042,67.73349121)
\curveto(709.25929616,67.68348673)(709.25929616,67.62848679)(709.2493042,67.56849121)
\curveto(709.24929617,67.5184869)(709.25429617,67.46848695)(709.2643042,67.41849121)
\moveto(707.9293042,66.56349121)
\curveto(707.94929747,66.63348778)(707.95429747,66.7134877)(707.9443042,66.80349121)
\lineto(707.9443042,67.05849121)
\curveto(707.94429748,67.44848697)(707.90929751,67.77848664)(707.8393042,68.04849121)
\curveto(707.80929761,68.12848629)(707.78429764,68.20848621)(707.7643042,68.28849121)
\curveto(707.74429768,68.36848605)(707.7192977,68.44348597)(707.6893042,68.51349121)
\curveto(707.40929801,69.16348525)(706.96429846,69.6134848)(706.3543042,69.86349121)
\curveto(706.28429914,69.89348452)(706.20929921,69.9134845)(706.1293042,69.92349121)
\lineto(705.8893042,69.98349121)
\curveto(705.80929961,70.00348441)(705.7242997,70.0134844)(705.6343042,70.01349121)
\lineto(705.3643042,70.01349121)
\lineto(705.0943042,69.96849121)
\curveto(704.99430043,69.94848447)(704.89930052,69.92348449)(704.8093042,69.89349121)
\curveto(704.72930069,69.87348454)(704.64930077,69.84348457)(704.5693042,69.80349121)
\curveto(704.49930092,69.78348463)(704.43430099,69.75348466)(704.3743042,69.71349121)
\curveto(704.31430111,69.67348474)(704.25930116,69.63348478)(704.2093042,69.59349121)
\curveto(703.96930145,69.42348499)(703.77430165,69.2184852)(703.6243042,68.97849121)
\curveto(703.47430195,68.73848568)(703.34430208,68.45848596)(703.2343042,68.13849121)
\curveto(703.20430222,68.03848638)(703.18430224,67.93348648)(703.1743042,67.82349121)
\curveto(703.16430226,67.72348669)(703.14930227,67.6184868)(703.1293042,67.50849121)
\curveto(703.1193023,67.46848695)(703.11430231,67.40348701)(703.1143042,67.31349121)
\curveto(703.10430232,67.28348713)(703.09930232,67.24848717)(703.0993042,67.20849121)
\curveto(703.10930231,67.16848725)(703.11430231,67.12348729)(703.1143042,67.07349121)
\lineto(703.1143042,66.77349121)
\curveto(703.11430231,66.67348774)(703.1243023,66.58348783)(703.1443042,66.50349121)
\lineto(703.1743042,66.32349121)
\curveto(703.19430223,66.22348819)(703.20930221,66.12348829)(703.2193042,66.02349121)
\curveto(703.23930218,65.93348848)(703.26930215,65.84848857)(703.3093042,65.76849121)
\curveto(703.40930201,65.52848889)(703.5243019,65.30348911)(703.6543042,65.09349121)
\curveto(703.79430163,64.88348953)(703.96430146,64.70848971)(704.1643042,64.56849121)
\curveto(704.21430121,64.53848988)(704.25930116,64.5134899)(704.2993042,64.49349121)
\curveto(704.33930108,64.47348994)(704.38430104,64.44848997)(704.4343042,64.41849121)
\curveto(704.51430091,64.36849005)(704.59930082,64.32349009)(704.6893042,64.28349121)
\curveto(704.78930063,64.25349016)(704.89430053,64.22349019)(705.0043042,64.19349121)
\curveto(705.05430037,64.17349024)(705.09930032,64.16349025)(705.1393042,64.16349121)
\curveto(705.18930023,64.17349024)(705.23930018,64.17349024)(705.2893042,64.16349121)
\curveto(705.3193001,64.15349026)(705.37930004,64.14349027)(705.4693042,64.13349121)
\curveto(705.56929985,64.12349029)(705.64429978,64.12849029)(705.6943042,64.14849121)
\curveto(705.73429969,64.15849026)(705.77429965,64.15849026)(705.8143042,64.14849121)
\curveto(705.85429957,64.14849027)(705.89429953,64.15849026)(705.9343042,64.17849121)
\curveto(706.01429941,64.19849022)(706.09429933,64.2134902)(706.1743042,64.22349121)
\curveto(706.25429917,64.24349017)(706.32929909,64.26849015)(706.3993042,64.29849121)
\curveto(706.73929868,64.43848998)(707.01429841,64.63348978)(707.2243042,64.88349121)
\curveto(707.43429799,65.13348928)(707.60929781,65.42848899)(707.7493042,65.76849121)
\curveto(707.79929762,65.88848853)(707.82929759,66.0134884)(707.8393042,66.14349121)
\curveto(707.85929756,66.28348813)(707.88929753,66.42348799)(707.9293042,66.56349121)
}
}
{
\newrgbcolor{curcolor}{0 0 0}
\pscustom[linestyle=none,fillstyle=solid,fillcolor=curcolor]
{
\newpath
\moveto(714.39758545,71.12349121)
\curveto(714.62758066,71.12348329)(714.75758053,71.06348335)(714.78758545,70.94349121)
\curveto(714.81758047,70.83348358)(714.83258045,70.66848375)(714.83258545,70.44849121)
\lineto(714.83258545,70.16349121)
\curveto(714.83258045,70.07348434)(714.80758048,69.99848442)(714.75758545,69.93849121)
\curveto(714.69758059,69.85848456)(714.61258067,69.8134846)(714.50258545,69.80349121)
\curveto(714.39258089,69.80348461)(714.282581,69.78848463)(714.17258545,69.75849121)
\curveto(714.03258125,69.72848469)(713.89758139,69.69848472)(713.76758545,69.66849121)
\curveto(713.64758164,69.63848478)(713.53258175,69.59848482)(713.42258545,69.54849121)
\curveto(713.13258215,69.418485)(712.89758239,69.23848518)(712.71758545,69.00849121)
\curveto(712.53758275,68.78848563)(712.3825829,68.53348588)(712.25258545,68.24349121)
\curveto(712.21258307,68.13348628)(712.1825831,68.0184864)(712.16258545,67.89849121)
\curveto(712.14258314,67.78848663)(712.11758317,67.67348674)(712.08758545,67.55349121)
\curveto(712.07758321,67.50348691)(712.07258321,67.45348696)(712.07258545,67.40349121)
\curveto(712.0825832,67.35348706)(712.0825832,67.30348711)(712.07258545,67.25349121)
\curveto(712.04258324,67.13348728)(712.02758326,66.99348742)(712.02758545,66.83349121)
\curveto(712.03758325,66.68348773)(712.04258324,66.53848788)(712.04258545,66.39849121)
\lineto(712.04258545,64.55349121)
\lineto(712.04258545,64.20849121)
\curveto(712.04258324,64.08849033)(712.03758325,63.97349044)(712.02758545,63.86349121)
\curveto(712.01758327,63.75349066)(712.01258327,63.65849076)(712.01258545,63.57849121)
\curveto(712.02258326,63.49849092)(712.00258328,63.42849099)(711.95258545,63.36849121)
\curveto(711.90258338,63.29849112)(711.82258346,63.25849116)(711.71258545,63.24849121)
\curveto(711.61258367,63.23849118)(711.50258378,63.23349118)(711.38258545,63.23349121)
\lineto(711.11258545,63.23349121)
\curveto(711.06258422,63.25349116)(711.01258427,63.26849115)(710.96258545,63.27849121)
\curveto(710.92258436,63.29849112)(710.89258439,63.32349109)(710.87258545,63.35349121)
\curveto(710.82258446,63.42349099)(710.79258449,63.50849091)(710.78258545,63.60849121)
\lineto(710.78258545,63.93849121)
\lineto(710.78258545,65.09349121)
\lineto(710.78258545,69.24849121)
\lineto(710.78258545,70.28349121)
\lineto(710.78258545,70.58349121)
\curveto(710.79258449,70.68348373)(710.82258446,70.76848365)(710.87258545,70.83849121)
\curveto(710.90258438,70.87848354)(710.95258433,70.90848351)(711.02258545,70.92849121)
\curveto(711.10258418,70.94848347)(711.1875841,70.95848346)(711.27758545,70.95849121)
\curveto(711.36758392,70.96848345)(711.45758383,70.96848345)(711.54758545,70.95849121)
\curveto(711.63758365,70.94848347)(711.70758358,70.93348348)(711.75758545,70.91349121)
\curveto(711.83758345,70.88348353)(711.8875834,70.82348359)(711.90758545,70.73349121)
\curveto(711.93758335,70.65348376)(711.95258333,70.56348385)(711.95258545,70.46349121)
\lineto(711.95258545,70.16349121)
\curveto(711.95258333,70.06348435)(711.97258331,69.97348444)(712.01258545,69.89349121)
\curveto(712.02258326,69.87348454)(712.03258325,69.85848456)(712.04258545,69.84849121)
\lineto(712.08758545,69.80349121)
\curveto(712.19758309,69.80348461)(712.287583,69.84848457)(712.35758545,69.93849121)
\curveto(712.42758286,70.03848438)(712.4875828,70.1184843)(712.53758545,70.17849121)
\lineto(712.62758545,70.26849121)
\curveto(712.71758257,70.37848404)(712.84258244,70.49348392)(713.00258545,70.61349121)
\curveto(713.16258212,70.73348368)(713.31258197,70.82348359)(713.45258545,70.88349121)
\curveto(713.54258174,70.93348348)(713.63758165,70.96848345)(713.73758545,70.98849121)
\curveto(713.83758145,71.0184834)(713.94258134,71.04848337)(714.05258545,71.07849121)
\curveto(714.11258117,71.08848333)(714.17258111,71.09348332)(714.23258545,71.09349121)
\curveto(714.29258099,71.10348331)(714.34758094,71.1134833)(714.39758545,71.12349121)
}
}
{
\newrgbcolor{curcolor}{0.40000001 0.40000001 0.40000001}
\pscustom[linestyle=none,fillstyle=solid,fillcolor=curcolor]
{
\newpath
\moveto(606.37554932,73.92852783)
\lineto(621.37554932,73.92852783)
\lineto(621.37554932,58.92852783)
\lineto(606.37554932,58.92852783)
\closepath
}
}
{
\newrgbcolor{curcolor}{0 0 0}
\pscustom[linestyle=none,fillstyle=solid,fillcolor=curcolor]
{
\newpath
\moveto(634.31375732,40.78431885)
\curveto(634.33374778,40.7343181)(634.35874775,40.67431816)(634.38875732,40.60431885)
\curveto(634.41874769,40.5343183)(634.43874767,40.45931838)(634.44875732,40.37931885)
\curveto(634.46874764,40.30931853)(634.46874764,40.2393186)(634.44875732,40.16931885)
\curveto(634.43874767,40.10931873)(634.39874771,40.06431877)(634.32875732,40.03431885)
\curveto(634.27874783,40.01431882)(634.21874789,40.00431883)(634.14875732,40.00431885)
\lineto(633.93875732,40.00431885)
\lineto(633.48875732,40.00431885)
\curveto(633.33874877,40.00431883)(633.21874889,40.02931881)(633.12875732,40.07931885)
\curveto(633.02874908,40.1393187)(632.95374916,40.24431859)(632.90375732,40.39431885)
\curveto(632.86374925,40.54431829)(632.81874929,40.67931816)(632.76875732,40.79931885)
\curveto(632.65874945,41.05931778)(632.55874955,41.32931751)(632.46875732,41.60931885)
\curveto(632.37874973,41.88931695)(632.27874983,42.16431667)(632.16875732,42.43431885)
\curveto(632.13874997,42.52431631)(632.10875,42.60931623)(632.07875732,42.68931885)
\curveto(632.05875005,42.76931607)(632.02875008,42.84431599)(631.98875732,42.91431885)
\curveto(631.95875015,42.98431585)(631.9137502,43.04431579)(631.85375732,43.09431885)
\curveto(631.79375032,43.14431569)(631.7137504,43.18431565)(631.61375732,43.21431885)
\curveto(631.56375055,43.2343156)(631.50375061,43.2393156)(631.43375732,43.22931885)
\lineto(631.23875732,43.22931885)
\lineto(628.40375732,43.22931885)
\lineto(628.10375732,43.22931885)
\curveto(627.99375412,43.2393156)(627.88875422,43.2393156)(627.78875732,43.22931885)
\curveto(627.68875442,43.21931562)(627.59375452,43.20431563)(627.50375732,43.18431885)
\curveto(627.42375469,43.16431567)(627.36375475,43.12431571)(627.32375732,43.06431885)
\curveto(627.24375487,42.96431587)(627.18375493,42.84931599)(627.14375732,42.71931885)
\curveto(627.113755,42.59931624)(627.07375504,42.47431636)(627.02375732,42.34431885)
\curveto(626.92375519,42.11431672)(626.82875528,41.87431696)(626.73875732,41.62431885)
\curveto(626.65875545,41.37431746)(626.56875554,41.1343177)(626.46875732,40.90431885)
\curveto(626.44875566,40.84431799)(626.42375569,40.77431806)(626.39375732,40.69431885)
\curveto(626.37375574,40.62431821)(626.34875576,40.54931829)(626.31875732,40.46931885)
\curveto(626.28875582,40.38931845)(626.25375586,40.31431852)(626.21375732,40.24431885)
\curveto(626.18375593,40.18431865)(626.14875596,40.1393187)(626.10875732,40.10931885)
\curveto(626.02875608,40.04931879)(625.91875619,40.01431882)(625.77875732,40.00431885)
\lineto(625.35875732,40.00431885)
\lineto(625.11875732,40.00431885)
\curveto(625.04875706,40.01431882)(624.98875712,40.0393188)(624.93875732,40.07931885)
\curveto(624.88875722,40.10931873)(624.85875725,40.15431868)(624.84875732,40.21431885)
\curveto(624.84875726,40.27431856)(624.85375726,40.3343185)(624.86375732,40.39431885)
\curveto(624.88375723,40.46431837)(624.90375721,40.52931831)(624.92375732,40.58931885)
\curveto(624.95375716,40.65931818)(624.97875713,40.70931813)(624.99875732,40.73931885)
\curveto(625.13875697,41.05931778)(625.26375685,41.37431746)(625.37375732,41.68431885)
\curveto(625.48375663,42.00431683)(625.60375651,42.32431651)(625.73375732,42.64431885)
\curveto(625.82375629,42.86431597)(625.9087562,43.07931576)(625.98875732,43.28931885)
\curveto(626.06875604,43.50931533)(626.15375596,43.72931511)(626.24375732,43.94931885)
\curveto(626.54375557,44.66931417)(626.82875528,45.39431344)(627.09875732,46.12431885)
\curveto(627.36875474,46.86431197)(627.65375446,47.59931124)(627.95375732,48.32931885)
\curveto(628.06375405,48.58931025)(628.16375395,48.85430998)(628.25375732,49.12431885)
\curveto(628.35375376,49.39430944)(628.45875365,49.65930918)(628.56875732,49.91931885)
\curveto(628.61875349,50.02930881)(628.66375345,50.14930869)(628.70375732,50.27931885)
\curveto(628.75375336,50.41930842)(628.82375329,50.51930832)(628.91375732,50.57931885)
\curveto(628.95375316,50.61930822)(629.01875309,50.64930819)(629.10875732,50.66931885)
\curveto(629.12875298,50.67930816)(629.14875296,50.67930816)(629.16875732,50.66931885)
\curveto(629.19875291,50.66930817)(629.22375289,50.67430816)(629.24375732,50.68431885)
\curveto(629.42375269,50.68430815)(629.63375248,50.68430815)(629.87375732,50.68431885)
\curveto(630.113752,50.69430814)(630.28875182,50.65930818)(630.39875732,50.57931885)
\curveto(630.47875163,50.51930832)(630.53875157,50.41930842)(630.57875732,50.27931885)
\curveto(630.62875148,50.14930869)(630.67875143,50.02930881)(630.72875732,49.91931885)
\curveto(630.82875128,49.68930915)(630.91875119,49.45930938)(630.99875732,49.22931885)
\curveto(631.07875103,48.99930984)(631.16875094,48.76931007)(631.26875732,48.53931885)
\curveto(631.34875076,48.3393105)(631.42375069,48.1343107)(631.49375732,47.92431885)
\curveto(631.57375054,47.71431112)(631.65875045,47.50931133)(631.74875732,47.30931885)
\curveto(632.04875006,46.57931226)(632.33374978,45.839313)(632.60375732,45.08931885)
\curveto(632.88374923,44.34931449)(633.17874893,43.61431522)(633.48875732,42.88431885)
\curveto(633.52874858,42.79431604)(633.55874855,42.70931613)(633.57875732,42.62931885)
\curveto(633.6087485,42.54931629)(633.63874847,42.46431637)(633.66875732,42.37431885)
\curveto(633.77874833,42.11431672)(633.88374823,41.84931699)(633.98375732,41.57931885)
\curveto(634.09374802,41.30931753)(634.20374791,41.04431779)(634.31375732,40.78431885)
\moveto(631.10375732,44.42931885)
\curveto(631.19375092,44.45931438)(631.24875086,44.50931433)(631.26875732,44.57931885)
\curveto(631.29875081,44.64931419)(631.30375081,44.72431411)(631.28375732,44.80431885)
\curveto(631.27375084,44.89431394)(631.24875086,44.97931386)(631.20875732,45.05931885)
\curveto(631.17875093,45.14931369)(631.14875096,45.22431361)(631.11875732,45.28431885)
\curveto(631.09875101,45.32431351)(631.08875102,45.35931348)(631.08875732,45.38931885)
\curveto(631.08875102,45.41931342)(631.07875103,45.45431338)(631.05875732,45.49431885)
\lineto(630.96875732,45.73431885)
\curveto(630.94875116,45.82431301)(630.91875119,45.91431292)(630.87875732,46.00431885)
\curveto(630.72875138,46.36431247)(630.59375152,46.72931211)(630.47375732,47.09931885)
\curveto(630.36375175,47.47931136)(630.23375188,47.84931099)(630.08375732,48.20931885)
\curveto(630.03375208,48.31931052)(629.98875212,48.42931041)(629.94875732,48.53931885)
\curveto(629.91875219,48.64931019)(629.87875223,48.75431008)(629.82875732,48.85431885)
\curveto(629.8087523,48.90430993)(629.78375233,48.94930989)(629.75375732,48.98931885)
\curveto(629.73375238,49.0393098)(629.68375243,49.06430977)(629.60375732,49.06431885)
\curveto(629.58375253,49.04430979)(629.56375255,49.02930981)(629.54375732,49.01931885)
\curveto(629.52375259,49.00930983)(629.50375261,48.99430984)(629.48375732,48.97431885)
\curveto(629.44375267,48.92430991)(629.4137527,48.86930997)(629.39375732,48.80931885)
\curveto(629.37375274,48.75931008)(629.35375276,48.70431013)(629.33375732,48.64431885)
\curveto(629.28375283,48.5343103)(629.24375287,48.42431041)(629.21375732,48.31431885)
\curveto(629.18375293,48.20431063)(629.14375297,48.09431074)(629.09375732,47.98431885)
\curveto(628.92375319,47.59431124)(628.77375334,47.19931164)(628.64375732,46.79931885)
\curveto(628.52375359,46.39931244)(628.38375373,46.00931283)(628.22375732,45.62931885)
\lineto(628.16375732,45.47931885)
\curveto(628.15375396,45.42931341)(628.13875397,45.37931346)(628.11875732,45.32931885)
\lineto(628.02875732,45.08931885)
\curveto(627.99875411,45.00931383)(627.97375414,44.92931391)(627.95375732,44.84931885)
\curveto(627.93375418,44.79931404)(627.92375419,44.74431409)(627.92375732,44.68431885)
\curveto(627.93375418,44.62431421)(627.94875416,44.57431426)(627.96875732,44.53431885)
\curveto(628.01875409,44.45431438)(628.12375399,44.40931443)(628.28375732,44.39931885)
\lineto(628.73375732,44.39931885)
\lineto(630.33875732,44.39931885)
\curveto(630.44875166,44.39931444)(630.58375153,44.39431444)(630.74375732,44.38431885)
\curveto(630.90375121,44.38431445)(631.02375109,44.39931444)(631.10375732,44.42931885)
}
}
{
\newrgbcolor{curcolor}{0 0 0}
\pscustom[linestyle=none,fillstyle=solid,fillcolor=curcolor]
{
\newpath
\moveto(642.39031982,40.81431885)
\lineto(642.39031982,40.42431885)
\curveto(642.39031195,40.30431853)(642.36531197,40.20431863)(642.31531982,40.12431885)
\curveto(642.26531207,40.05431878)(642.18031216,40.01431882)(642.06031982,40.00431885)
\lineto(641.71531982,40.00431885)
\curveto(641.65531268,40.00431883)(641.59531274,39.99931884)(641.53531982,39.98931885)
\curveto(641.48531285,39.98931885)(641.4403129,39.99931884)(641.40031982,40.01931885)
\curveto(641.31031303,40.0393188)(641.25031309,40.07931876)(641.22031982,40.13931885)
\curveto(641.18031316,40.18931865)(641.15531318,40.24931859)(641.14531982,40.31931885)
\curveto(641.14531319,40.38931845)(641.13031321,40.45931838)(641.10031982,40.52931885)
\curveto(641.09031325,40.54931829)(641.07531326,40.56431827)(641.05531982,40.57431885)
\curveto(641.04531329,40.59431824)(641.03031331,40.61431822)(641.01031982,40.63431885)
\curveto(640.91031343,40.64431819)(640.83031351,40.62431821)(640.77031982,40.57431885)
\curveto(640.72031362,40.52431831)(640.66531367,40.47431836)(640.60531982,40.42431885)
\curveto(640.40531393,40.27431856)(640.20531413,40.15931868)(640.00531982,40.07931885)
\curveto(639.82531451,39.99931884)(639.61531472,39.9393189)(639.37531982,39.89931885)
\curveto(639.14531519,39.85931898)(638.90531543,39.839319)(638.65531982,39.83931885)
\curveto(638.41531592,39.82931901)(638.17531616,39.84431899)(637.93531982,39.88431885)
\curveto(637.69531664,39.91431892)(637.48531685,39.96931887)(637.30531982,40.04931885)
\curveto(636.78531755,40.26931857)(636.36531797,40.56431827)(636.04531982,40.93431885)
\curveto(635.72531861,41.31431752)(635.47531886,41.78431705)(635.29531982,42.34431885)
\curveto(635.25531908,42.4343164)(635.22531911,42.52431631)(635.20531982,42.61431885)
\curveto(635.19531914,42.71431612)(635.17531916,42.81431602)(635.14531982,42.91431885)
\curveto(635.1353192,42.96431587)(635.13031921,43.01431582)(635.13031982,43.06431885)
\curveto(635.13031921,43.11431572)(635.12531921,43.16431567)(635.11531982,43.21431885)
\curveto(635.09531924,43.26431557)(635.08531925,43.31431552)(635.08531982,43.36431885)
\curveto(635.09531924,43.42431541)(635.09531924,43.47931536)(635.08531982,43.52931885)
\lineto(635.08531982,43.67931885)
\curveto(635.06531927,43.72931511)(635.05531928,43.79431504)(635.05531982,43.87431885)
\curveto(635.05531928,43.95431488)(635.06531927,44.01931482)(635.08531982,44.06931885)
\lineto(635.08531982,44.23431885)
\curveto(635.10531923,44.30431453)(635.11031923,44.37431446)(635.10031982,44.44431885)
\curveto(635.10031924,44.52431431)(635.11031923,44.59931424)(635.13031982,44.66931885)
\curveto(635.1403192,44.71931412)(635.14531919,44.76431407)(635.14531982,44.80431885)
\curveto(635.14531919,44.84431399)(635.15031919,44.88931395)(635.16031982,44.93931885)
\curveto(635.19031915,45.0393138)(635.21531912,45.1343137)(635.23531982,45.22431885)
\curveto(635.25531908,45.32431351)(635.28031906,45.41931342)(635.31031982,45.50931885)
\curveto(635.4403189,45.88931295)(635.60531873,46.22931261)(635.80531982,46.52931885)
\curveto(636.01531832,46.839312)(636.26531807,47.09431174)(636.55531982,47.29431885)
\curveto(636.72531761,47.41431142)(636.90031744,47.51431132)(637.08031982,47.59431885)
\curveto(637.27031707,47.67431116)(637.47531686,47.74431109)(637.69531982,47.80431885)
\curveto(637.76531657,47.81431102)(637.83031651,47.82431101)(637.89031982,47.83431885)
\curveto(637.96031638,47.84431099)(638.03031631,47.85931098)(638.10031982,47.87931885)
\lineto(638.25031982,47.87931885)
\curveto(638.33031601,47.89931094)(638.44531589,47.90931093)(638.59531982,47.90931885)
\curveto(638.75531558,47.90931093)(638.87531546,47.89931094)(638.95531982,47.87931885)
\curveto(638.99531534,47.86931097)(639.05031529,47.86431097)(639.12031982,47.86431885)
\curveto(639.23031511,47.834311)(639.340315,47.80931103)(639.45031982,47.78931885)
\curveto(639.56031478,47.77931106)(639.66531467,47.74931109)(639.76531982,47.69931885)
\curveto(639.91531442,47.6393112)(640.05531428,47.57431126)(640.18531982,47.50431885)
\curveto(640.32531401,47.4343114)(640.45531388,47.35431148)(640.57531982,47.26431885)
\curveto(640.6353137,47.21431162)(640.69531364,47.15931168)(640.75531982,47.09931885)
\curveto(640.82531351,47.04931179)(640.91531342,47.0343118)(641.02531982,47.05431885)
\curveto(641.04531329,47.08431175)(641.06031328,47.10931173)(641.07031982,47.12931885)
\curveto(641.09031325,47.14931169)(641.10531323,47.17931166)(641.11531982,47.21931885)
\curveto(641.14531319,47.30931153)(641.15531318,47.42431141)(641.14531982,47.56431885)
\lineto(641.14531982,47.93931885)
\lineto(641.14531982,49.66431885)
\lineto(641.14531982,50.12931885)
\curveto(641.14531319,50.30930853)(641.17031317,50.4393084)(641.22031982,50.51931885)
\curveto(641.26031308,50.58930825)(641.32031302,50.6343082)(641.40031982,50.65431885)
\curveto(641.42031292,50.65430818)(641.44531289,50.65430818)(641.47531982,50.65431885)
\curveto(641.50531283,50.66430817)(641.53031281,50.66930817)(641.55031982,50.66931885)
\curveto(641.69031265,50.67930816)(641.8353125,50.67930816)(641.98531982,50.66931885)
\curveto(642.14531219,50.66930817)(642.25531208,50.62930821)(642.31531982,50.54931885)
\curveto(642.36531197,50.46930837)(642.39031195,50.36930847)(642.39031982,50.24931885)
\lineto(642.39031982,49.87431885)
\lineto(642.39031982,40.81431885)
\moveto(641.17531982,43.64931885)
\curveto(641.19531314,43.69931514)(641.20531313,43.76431507)(641.20531982,43.84431885)
\curveto(641.20531313,43.9343149)(641.19531314,44.00431483)(641.17531982,44.05431885)
\lineto(641.17531982,44.27931885)
\curveto(641.15531318,44.36931447)(641.1403132,44.45931438)(641.13031982,44.54931885)
\curveto(641.12031322,44.64931419)(641.10031324,44.7393141)(641.07031982,44.81931885)
\curveto(641.05031329,44.89931394)(641.03031331,44.97431386)(641.01031982,45.04431885)
\curveto(641.00031334,45.11431372)(640.98031336,45.18431365)(640.95031982,45.25431885)
\curveto(640.83031351,45.55431328)(640.67531366,45.81931302)(640.48531982,46.04931885)
\curveto(640.29531404,46.27931256)(640.05531428,46.45931238)(639.76531982,46.58931885)
\curveto(639.66531467,46.6393122)(639.56031478,46.67431216)(639.45031982,46.69431885)
\curveto(639.35031499,46.72431211)(639.2403151,46.74931209)(639.12031982,46.76931885)
\curveto(639.0403153,46.78931205)(638.95031539,46.79931204)(638.85031982,46.79931885)
\lineto(638.58031982,46.79931885)
\curveto(638.53031581,46.78931205)(638.48531585,46.77931206)(638.44531982,46.76931885)
\lineto(638.31031982,46.76931885)
\curveto(638.23031611,46.74931209)(638.14531619,46.72931211)(638.05531982,46.70931885)
\curveto(637.97531636,46.68931215)(637.89531644,46.66431217)(637.81531982,46.63431885)
\curveto(637.49531684,46.49431234)(637.2353171,46.28931255)(637.03531982,46.01931885)
\curveto(636.84531749,45.75931308)(636.69031765,45.45431338)(636.57031982,45.10431885)
\curveto(636.53031781,44.99431384)(636.50031784,44.87931396)(636.48031982,44.75931885)
\curveto(636.47031787,44.64931419)(636.45531788,44.5393143)(636.43531982,44.42931885)
\curveto(636.4353179,44.38931445)(636.43031791,44.34931449)(636.42031982,44.30931885)
\lineto(636.42031982,44.20431885)
\curveto(636.40031794,44.15431468)(636.39031795,44.09931474)(636.39031982,44.03931885)
\curveto(636.40031794,43.97931486)(636.40531793,43.92431491)(636.40531982,43.87431885)
\lineto(636.40531982,43.54431885)
\curveto(636.40531793,43.44431539)(636.41531792,43.34931549)(636.43531982,43.25931885)
\curveto(636.44531789,43.22931561)(636.45031789,43.17931566)(636.45031982,43.10931885)
\curveto(636.47031787,43.0393158)(636.48531785,42.96931587)(636.49531982,42.89931885)
\lineto(636.55531982,42.68931885)
\curveto(636.66531767,42.3393165)(636.81531752,42.0393168)(637.00531982,41.78931885)
\curveto(637.19531714,41.5393173)(637.4353169,41.3343175)(637.72531982,41.17431885)
\curveto(637.81531652,41.12431771)(637.90531643,41.08431775)(637.99531982,41.05431885)
\curveto(638.08531625,41.02431781)(638.18531615,40.99431784)(638.29531982,40.96431885)
\curveto(638.34531599,40.94431789)(638.39531594,40.9393179)(638.44531982,40.94931885)
\curveto(638.50531583,40.95931788)(638.56031578,40.95431788)(638.61031982,40.93431885)
\curveto(638.65031569,40.92431791)(638.69031565,40.91931792)(638.73031982,40.91931885)
\lineto(638.86531982,40.91931885)
\lineto(639.00031982,40.91931885)
\curveto(639.03031531,40.92931791)(639.08031526,40.9343179)(639.15031982,40.93431885)
\curveto(639.23031511,40.95431788)(639.31031503,40.96931787)(639.39031982,40.97931885)
\curveto(639.47031487,40.99931784)(639.54531479,41.02431781)(639.61531982,41.05431885)
\curveto(639.94531439,41.19431764)(640.21031413,41.36931747)(640.41031982,41.57931885)
\curveto(640.62031372,41.79931704)(640.79531354,42.07431676)(640.93531982,42.40431885)
\curveto(640.98531335,42.51431632)(641.02031332,42.62431621)(641.04031982,42.73431885)
\curveto(641.06031328,42.84431599)(641.08531325,42.95431588)(641.11531982,43.06431885)
\curveto(641.1353132,43.10431573)(641.14531319,43.1393157)(641.14531982,43.16931885)
\curveto(641.14531319,43.20931563)(641.15031319,43.24931559)(641.16031982,43.28931885)
\curveto(641.17031317,43.34931549)(641.17031317,43.40931543)(641.16031982,43.46931885)
\curveto(641.16031318,43.52931531)(641.16531317,43.58931525)(641.17531982,43.64931885)
}
}
{
\newrgbcolor{curcolor}{0 0 0}
\pscustom[linestyle=none,fillstyle=solid,fillcolor=curcolor]
{
\newpath
\moveto(648.02656982,47.90931885)
\curveto(648.40656484,47.91931092)(648.72656452,47.87931096)(648.98656982,47.78931885)
\curveto(649.25656399,47.69931114)(649.50156374,47.56931127)(649.72156982,47.39931885)
\curveto(649.80156344,47.34931149)(649.86656338,47.27931156)(649.91656982,47.18931885)
\curveto(649.97656327,47.10931173)(650.0415632,47.0343118)(650.11156982,46.96431885)
\curveto(650.13156311,46.94431189)(650.16156308,46.91931192)(650.20156982,46.88931885)
\curveto(650.241563,46.85931198)(650.29156295,46.84931199)(650.35156982,46.85931885)
\curveto(650.45156279,46.88931195)(650.53656271,46.94931189)(650.60656982,47.03931885)
\curveto(650.68656256,47.1393117)(650.76656248,47.21431162)(650.84656982,47.26431885)
\curveto(650.98656226,47.37431146)(651.13156211,47.46931137)(651.28156982,47.54931885)
\curveto(651.43156181,47.6393112)(651.59656165,47.71431112)(651.77656982,47.77431885)
\curveto(651.85656139,47.80431103)(651.9415613,47.82431101)(652.03156982,47.83431885)
\curveto(652.13156111,47.85431098)(652.22656102,47.87431096)(652.31656982,47.89431885)
\curveto(652.36656088,47.90431093)(652.41156083,47.90931093)(652.45156982,47.90931885)
\lineto(652.60156982,47.90931885)
\curveto(652.65156059,47.92931091)(652.72156052,47.9343109)(652.81156982,47.92431885)
\curveto(652.90156034,47.92431091)(652.96656028,47.91931092)(653.00656982,47.90931885)
\curveto(653.05656019,47.89931094)(653.13156011,47.89431094)(653.23156982,47.89431885)
\curveto(653.32155992,47.87431096)(653.40655984,47.85431098)(653.48656982,47.83431885)
\curveto(653.57655967,47.82431101)(653.66155958,47.80431103)(653.74156982,47.77431885)
\curveto(653.79155945,47.75431108)(653.83655941,47.7393111)(653.87656982,47.72931885)
\curveto(653.92655932,47.72931111)(653.97655927,47.71931112)(654.02656982,47.69931885)
\curveto(654.52655872,47.47931136)(654.87155837,47.1393117)(655.06156982,46.67931885)
\curveto(655.10155814,46.59931224)(655.13155811,46.50931233)(655.15156982,46.40931885)
\curveto(655.17155807,46.31931252)(655.19155805,46.21931262)(655.21156982,46.10931885)
\curveto(655.23155801,46.07931276)(655.23655801,46.04431279)(655.22656982,46.00431885)
\curveto(655.22655802,45.97431286)(655.23155801,45.94431289)(655.24156982,45.91431885)
\lineto(655.24156982,45.77931885)
\curveto(655.25155799,45.7393131)(655.25155799,45.69431314)(655.24156982,45.64431885)
\curveto(655.241558,45.59431324)(655.241558,45.54431329)(655.24156982,45.49431885)
\lineto(655.24156982,44.90931885)
\lineto(655.24156982,43.94931885)
\lineto(655.24156982,41.09931885)
\curveto(655.241558,40.9393179)(655.241558,40.74931809)(655.24156982,40.52931885)
\curveto(655.25155799,40.30931853)(655.21155803,40.16431867)(655.12156982,40.09431885)
\curveto(655.08155816,40.06431877)(655.01655823,40.0393188)(654.92656982,40.01931885)
\curveto(654.83655841,40.00931883)(654.7415585,40.00431883)(654.64156982,40.00431885)
\curveto(654.5415587,40.00431883)(654.4415588,40.00931883)(654.34156982,40.01931885)
\curveto(654.25155899,40.02931881)(654.18655906,40.04931879)(654.14656982,40.07931885)
\curveto(654.08655916,40.10931873)(654.0465592,40.16931867)(654.02656982,40.25931885)
\curveto(654.00655924,40.31931852)(654.00155924,40.37931846)(654.01156982,40.43931885)
\curveto(654.02155922,40.50931833)(654.01655923,40.57431826)(653.99656982,40.63431885)
\curveto(653.98655926,40.68431815)(653.98155926,40.7393181)(653.98156982,40.79931885)
\curveto(653.99155925,40.86931797)(653.99655925,40.9343179)(653.99656982,40.99431885)
\lineto(653.99656982,41.66931885)
\lineto(653.99656982,44.53431885)
\curveto(653.99655925,44.86431397)(653.98655926,45.17431366)(653.96656982,45.46431885)
\curveto(653.95655929,45.76431307)(653.88655936,46.01431282)(653.75656982,46.21431885)
\curveto(653.60655964,46.45431238)(653.37655987,46.62931221)(653.06656982,46.73931885)
\curveto(653.00656024,46.75931208)(652.9415603,46.76931207)(652.87156982,46.76931885)
\curveto(652.81156043,46.77931206)(652.7465605,46.79431204)(652.67656982,46.81431885)
\curveto(652.63656061,46.82431201)(652.57156067,46.82431201)(652.48156982,46.81431885)
\curveto(652.39156085,46.81431202)(652.33156091,46.80931203)(652.30156982,46.79931885)
\curveto(652.25156099,46.78931205)(652.20156104,46.78431205)(652.15156982,46.78431885)
\curveto(652.10156114,46.79431204)(652.05156119,46.78931205)(652.00156982,46.76931885)
\curveto(651.86156138,46.7393121)(651.72656152,46.69931214)(651.59656982,46.64931885)
\curveto(651.07656217,46.42931241)(650.72656252,46.04431279)(650.54656982,45.49431885)
\curveto(650.49656275,45.32431351)(650.46656278,45.12931371)(650.45656982,44.90931885)
\lineto(650.45656982,44.23431885)
\lineto(650.45656982,42.26931885)
\lineto(650.45656982,40.81431885)
\lineto(650.45656982,40.43931885)
\curveto(650.45656279,40.31931852)(650.43156281,40.22431861)(650.38156982,40.15431885)
\curveto(650.33156291,40.07431876)(650.246563,40.02931881)(650.12656982,40.01931885)
\curveto(650.00656324,40.00931883)(649.88156336,40.00431883)(649.75156982,40.00431885)
\curveto(649.58156366,40.00431883)(649.45656379,40.02431881)(649.37656982,40.06431885)
\curveto(649.28656396,40.11431872)(649.23156401,40.19431864)(649.21156982,40.30431885)
\curveto(649.20156404,40.42431841)(649.19656405,40.55431828)(649.19656982,40.69431885)
\lineto(649.19656982,42.11931885)
\lineto(649.19656982,44.59431885)
\curveto(649.19656405,44.91431392)(649.18656406,45.20931363)(649.16656982,45.47931885)
\curveto(649.1465641,45.75931308)(649.07656417,45.99931284)(648.95656982,46.19931885)
\curveto(648.8465644,46.37931246)(648.72156452,46.50931233)(648.58156982,46.58931885)
\curveto(648.4415648,46.67931216)(648.25156499,46.74931209)(648.01156982,46.79931885)
\curveto(647.97156527,46.80931203)(647.92656532,46.81431202)(647.87656982,46.81431885)
\lineto(647.74156982,46.81431885)
\curveto(647.52156572,46.81431202)(647.32656592,46.78931205)(647.15656982,46.73931885)
\curveto(646.99656625,46.68931215)(646.85156639,46.62431221)(646.72156982,46.54431885)
\curveto(646.21156703,46.2343126)(645.87156737,45.76931307)(645.70156982,45.14931885)
\curveto(645.66156758,45.01931382)(645.6415676,44.86931397)(645.64156982,44.69931885)
\curveto(645.65156759,44.5393143)(645.65656759,44.37931446)(645.65656982,44.21931885)
\lineto(645.65656982,42.52431885)
\lineto(645.65656982,40.87431885)
\lineto(645.65656982,40.46931885)
\curveto(645.65656759,40.32931851)(645.62656762,40.21931862)(645.56656982,40.13931885)
\curveto(645.51656773,40.06931877)(645.4415678,40.02931881)(645.34156982,40.01931885)
\curveto(645.241568,40.00931883)(645.13656811,40.00431883)(645.02656982,40.00431885)
\lineto(644.80156982,40.00431885)
\curveto(644.7415685,40.02431881)(644.68156856,40.0393188)(644.62156982,40.04931885)
\curveto(644.57156867,40.05931878)(644.52656872,40.08931875)(644.48656982,40.13931885)
\curveto(644.43656881,40.19931864)(644.41156883,40.27431856)(644.41156982,40.36431885)
\lineto(644.41156982,40.67931885)
\lineto(644.41156982,41.65431885)
\lineto(644.41156982,45.94431885)
\lineto(644.41156982,47.05431885)
\lineto(644.41156982,47.33931885)
\curveto(644.41156883,47.4393114)(644.43156881,47.51931132)(644.47156982,47.57931885)
\curveto(644.50156874,47.6393112)(644.5465687,47.67931116)(644.60656982,47.69931885)
\curveto(644.68656856,47.72931111)(644.81156843,47.74431109)(644.98156982,47.74431885)
\curveto(645.16156808,47.74431109)(645.29156795,47.72931111)(645.37156982,47.69931885)
\curveto(645.45156779,47.65931118)(645.50656774,47.60931123)(645.53656982,47.54931885)
\curveto(645.55656769,47.49931134)(645.56656768,47.4393114)(645.56656982,47.36931885)
\curveto(645.57656767,47.29931154)(645.58656766,47.2343116)(645.59656982,47.17431885)
\curveto(645.60656764,47.11431172)(645.62656762,47.06431177)(645.65656982,47.02431885)
\curveto(645.68656756,46.98431185)(645.73656751,46.96431187)(645.80656982,46.96431885)
\curveto(645.82656742,46.98431185)(645.8465674,46.99431184)(645.86656982,46.99431885)
\curveto(645.89656735,46.99431184)(645.92156732,47.00431183)(645.94156982,47.02431885)
\curveto(646.00156724,47.07431176)(646.05656719,47.12431171)(646.10656982,47.17431885)
\lineto(646.28656982,47.32431885)
\curveto(646.50656674,47.48431135)(646.75656649,47.62431121)(647.03656982,47.74431885)
\curveto(647.13656611,47.78431105)(647.23656601,47.80931103)(647.33656982,47.81931885)
\curveto(647.43656581,47.839311)(647.5415657,47.86431097)(647.65156982,47.89431885)
\lineto(647.83156982,47.89431885)
\curveto(647.90156534,47.90431093)(647.96656528,47.90931093)(648.02656982,47.90931885)
}
}
{
\newrgbcolor{curcolor}{0 0 0}
\pscustom[linestyle=none,fillstyle=solid,fillcolor=curcolor]
{
\newpath
\moveto(657.4243042,49.22931885)
\curveto(657.34430308,49.28930955)(657.29930312,49.39430944)(657.2893042,49.54431885)
\lineto(657.2893042,50.00931885)
\lineto(657.2893042,50.26431885)
\curveto(657.28930313,50.35430848)(657.30430312,50.42930841)(657.3343042,50.48931885)
\curveto(657.37430305,50.56930827)(657.45430297,50.62930821)(657.5743042,50.66931885)
\curveto(657.59430283,50.67930816)(657.61430281,50.67930816)(657.6343042,50.66931885)
\curveto(657.66430276,50.66930817)(657.68930273,50.67430816)(657.7093042,50.68431885)
\curveto(657.87930254,50.68430815)(658.03930238,50.67930816)(658.1893042,50.66931885)
\curveto(658.33930208,50.65930818)(658.43930198,50.59930824)(658.4893042,50.48931885)
\curveto(658.5193019,50.42930841)(658.53430189,50.35430848)(658.5343042,50.26431885)
\lineto(658.5343042,50.00931885)
\curveto(658.53430189,49.82930901)(658.52930189,49.65930918)(658.5193042,49.49931885)
\curveto(658.5193019,49.3393095)(658.45430197,49.2343096)(658.3243042,49.18431885)
\curveto(658.27430215,49.16430967)(658.2193022,49.15430968)(658.1593042,49.15431885)
\lineto(657.9943042,49.15431885)
\lineto(657.6793042,49.15431885)
\curveto(657.57930284,49.15430968)(657.49430293,49.17930966)(657.4243042,49.22931885)
\moveto(658.5343042,40.72431885)
\lineto(658.5343042,40.40931885)
\curveto(658.54430188,40.30931853)(658.5243019,40.22931861)(658.4743042,40.16931885)
\curveto(658.44430198,40.10931873)(658.39930202,40.06931877)(658.3393042,40.04931885)
\curveto(658.27930214,40.0393188)(658.20930221,40.02431881)(658.1293042,40.00431885)
\lineto(657.9043042,40.00431885)
\curveto(657.77430265,40.00431883)(657.65930276,40.00931883)(657.5593042,40.01931885)
\curveto(657.46930295,40.0393188)(657.39930302,40.08931875)(657.3493042,40.16931885)
\curveto(657.30930311,40.22931861)(657.28930313,40.30431853)(657.2893042,40.39431885)
\lineto(657.2893042,40.67931885)
\lineto(657.2893042,47.02431885)
\lineto(657.2893042,47.33931885)
\curveto(657.28930313,47.44931139)(657.31430311,47.5343113)(657.3643042,47.59431885)
\curveto(657.39430303,47.64431119)(657.43430299,47.67431116)(657.4843042,47.68431885)
\curveto(657.53430289,47.69431114)(657.58930283,47.70931113)(657.6493042,47.72931885)
\curveto(657.66930275,47.72931111)(657.68930273,47.72431111)(657.7093042,47.71431885)
\curveto(657.73930268,47.71431112)(657.76430266,47.71931112)(657.7843042,47.72931885)
\curveto(657.91430251,47.72931111)(658.04430238,47.72431111)(658.1743042,47.71431885)
\curveto(658.31430211,47.71431112)(658.40930201,47.67431116)(658.4593042,47.59431885)
\curveto(658.50930191,47.5343113)(658.53430189,47.45431138)(658.5343042,47.35431885)
\lineto(658.5343042,47.06931885)
\lineto(658.5343042,40.72431885)
}
}
{
\newrgbcolor{curcolor}{0 0 0}
\pscustom[linestyle=none,fillstyle=solid,fillcolor=curcolor]
{
\newpath
\moveto(664.16914795,47.87931885)
\curveto(664.79914271,47.89931094)(665.30414221,47.81431102)(665.68414795,47.62431885)
\curveto(666.06414145,47.4343114)(666.36914114,47.14931169)(666.59914795,46.76931885)
\curveto(666.65914085,46.66931217)(666.70414081,46.55931228)(666.73414795,46.43931885)
\curveto(666.77414074,46.32931251)(666.8091407,46.21431262)(666.83914795,46.09431885)
\curveto(666.88914062,45.90431293)(666.91914059,45.69931314)(666.92914795,45.47931885)
\curveto(666.93914057,45.25931358)(666.94414057,45.0343138)(666.94414795,44.80431885)
\lineto(666.94414795,43.19931885)
\lineto(666.94414795,40.85931885)
\curveto(666.94414057,40.68931815)(666.93914057,40.51931832)(666.92914795,40.34931885)
\curveto(666.92914058,40.17931866)(666.86414065,40.06931877)(666.73414795,40.01931885)
\curveto(666.68414083,39.99931884)(666.62914088,39.98931885)(666.56914795,39.98931885)
\curveto(666.51914099,39.97931886)(666.46414105,39.97431886)(666.40414795,39.97431885)
\curveto(666.27414124,39.97431886)(666.14914136,39.97931886)(666.02914795,39.98931885)
\curveto(665.9091416,39.98931885)(665.82414169,40.02931881)(665.77414795,40.10931885)
\curveto(665.72414179,40.17931866)(665.69914181,40.26931857)(665.69914795,40.37931885)
\lineto(665.69914795,40.70931885)
\lineto(665.69914795,41.99931885)
\lineto(665.69914795,44.44431885)
\curveto(665.69914181,44.71431412)(665.69414182,44.97931386)(665.68414795,45.23931885)
\curveto(665.67414184,45.50931333)(665.62914188,45.7393131)(665.54914795,45.92931885)
\curveto(665.46914204,46.12931271)(665.34914216,46.28931255)(665.18914795,46.40931885)
\curveto(665.02914248,46.5393123)(664.84414267,46.6393122)(664.63414795,46.70931885)
\curveto(664.57414294,46.72931211)(664.509143,46.7393121)(664.43914795,46.73931885)
\curveto(664.37914313,46.74931209)(664.31914319,46.76431207)(664.25914795,46.78431885)
\curveto(664.2091433,46.79431204)(664.12914338,46.79431204)(664.01914795,46.78431885)
\curveto(663.91914359,46.78431205)(663.84914366,46.77931206)(663.80914795,46.76931885)
\curveto(663.76914374,46.74931209)(663.73414378,46.7393121)(663.70414795,46.73931885)
\curveto(663.67414384,46.74931209)(663.63914387,46.74931209)(663.59914795,46.73931885)
\curveto(663.46914404,46.70931213)(663.34414417,46.67431216)(663.22414795,46.63431885)
\curveto(663.1141444,46.60431223)(663.0091445,46.55931228)(662.90914795,46.49931885)
\curveto(662.86914464,46.47931236)(662.83414468,46.45931238)(662.80414795,46.43931885)
\curveto(662.77414474,46.41931242)(662.73914477,46.39931244)(662.69914795,46.37931885)
\curveto(662.34914516,46.12931271)(662.09414542,45.75431308)(661.93414795,45.25431885)
\curveto(661.90414561,45.17431366)(661.88414563,45.08931375)(661.87414795,44.99931885)
\curveto(661.86414565,44.91931392)(661.84914566,44.839314)(661.82914795,44.75931885)
\curveto(661.8091457,44.70931413)(661.80414571,44.65931418)(661.81414795,44.60931885)
\curveto(661.82414569,44.56931427)(661.81914569,44.52931431)(661.79914795,44.48931885)
\lineto(661.79914795,44.17431885)
\curveto(661.78914572,44.14431469)(661.78414573,44.10931473)(661.78414795,44.06931885)
\curveto(661.79414572,44.02931481)(661.79914571,43.98431485)(661.79914795,43.93431885)
\lineto(661.79914795,43.48431885)
\lineto(661.79914795,42.04431885)
\lineto(661.79914795,40.72431885)
\lineto(661.79914795,40.37931885)
\curveto(661.79914571,40.26931857)(661.77414574,40.17931866)(661.72414795,40.10931885)
\curveto(661.67414584,40.02931881)(661.58414593,39.98931885)(661.45414795,39.98931885)
\curveto(661.33414618,39.97931886)(661.2091463,39.97431886)(661.07914795,39.97431885)
\curveto(660.99914651,39.97431886)(660.92414659,39.97931886)(660.85414795,39.98931885)
\curveto(660.78414673,39.99931884)(660.72414679,40.02431881)(660.67414795,40.06431885)
\curveto(660.59414692,40.11431872)(660.55414696,40.20931863)(660.55414795,40.34931885)
\lineto(660.55414795,40.75431885)
\lineto(660.55414795,42.52431885)
\lineto(660.55414795,46.15431885)
\lineto(660.55414795,47.06931885)
\lineto(660.55414795,47.33931885)
\curveto(660.55414696,47.42931141)(660.57414694,47.49931134)(660.61414795,47.54931885)
\curveto(660.64414687,47.60931123)(660.69414682,47.64931119)(660.76414795,47.66931885)
\curveto(660.80414671,47.67931116)(660.85914665,47.68931115)(660.92914795,47.69931885)
\curveto(661.0091465,47.70931113)(661.08914642,47.71431112)(661.16914795,47.71431885)
\curveto(661.24914626,47.71431112)(661.32414619,47.70931113)(661.39414795,47.69931885)
\curveto(661.47414604,47.68931115)(661.52914598,47.67431116)(661.55914795,47.65431885)
\curveto(661.66914584,47.58431125)(661.71914579,47.49431134)(661.70914795,47.38431885)
\curveto(661.69914581,47.28431155)(661.7141458,47.16931167)(661.75414795,47.03931885)
\curveto(661.77414574,46.97931186)(661.8141457,46.92931191)(661.87414795,46.88931885)
\curveto(661.99414552,46.87931196)(662.08914542,46.92431191)(662.15914795,47.02431885)
\curveto(662.23914527,47.12431171)(662.31914519,47.20431163)(662.39914795,47.26431885)
\curveto(662.53914497,47.36431147)(662.67914483,47.45431138)(662.81914795,47.53431885)
\curveto(662.96914454,47.62431121)(663.13914437,47.69931114)(663.32914795,47.75931885)
\curveto(663.4091441,47.78931105)(663.49414402,47.80931103)(663.58414795,47.81931885)
\curveto(663.68414383,47.82931101)(663.77914373,47.84431099)(663.86914795,47.86431885)
\curveto(663.91914359,47.87431096)(663.96914354,47.87931096)(664.01914795,47.87931885)
\lineto(664.16914795,47.87931885)
}
}
{
\newrgbcolor{curcolor}{0 0 0}
\pscustom[linestyle=none,fillstyle=solid,fillcolor=curcolor]
{
\newpath
\moveto(669.11375732,49.22931885)
\curveto(669.0337562,49.28930955)(668.98875625,49.39430944)(668.97875732,49.54431885)
\lineto(668.97875732,50.00931885)
\lineto(668.97875732,50.26431885)
\curveto(668.97875626,50.35430848)(668.99375624,50.42930841)(669.02375732,50.48931885)
\curveto(669.06375617,50.56930827)(669.14375609,50.62930821)(669.26375732,50.66931885)
\curveto(669.28375595,50.67930816)(669.30375593,50.67930816)(669.32375732,50.66931885)
\curveto(669.35375588,50.66930817)(669.37875586,50.67430816)(669.39875732,50.68431885)
\curveto(669.56875567,50.68430815)(669.72875551,50.67930816)(669.87875732,50.66931885)
\curveto(670.02875521,50.65930818)(670.12875511,50.59930824)(670.17875732,50.48931885)
\curveto(670.20875503,50.42930841)(670.22375501,50.35430848)(670.22375732,50.26431885)
\lineto(670.22375732,50.00931885)
\curveto(670.22375501,49.82930901)(670.21875502,49.65930918)(670.20875732,49.49931885)
\curveto(670.20875503,49.3393095)(670.14375509,49.2343096)(670.01375732,49.18431885)
\curveto(669.96375527,49.16430967)(669.90875533,49.15430968)(669.84875732,49.15431885)
\lineto(669.68375732,49.15431885)
\lineto(669.36875732,49.15431885)
\curveto(669.26875597,49.15430968)(669.18375605,49.17930966)(669.11375732,49.22931885)
\moveto(670.22375732,40.72431885)
\lineto(670.22375732,40.40931885)
\curveto(670.233755,40.30931853)(670.21375502,40.22931861)(670.16375732,40.16931885)
\curveto(670.1337551,40.10931873)(670.08875515,40.06931877)(670.02875732,40.04931885)
\curveto(669.96875527,40.0393188)(669.89875534,40.02431881)(669.81875732,40.00431885)
\lineto(669.59375732,40.00431885)
\curveto(669.46375577,40.00431883)(669.34875589,40.00931883)(669.24875732,40.01931885)
\curveto(669.15875608,40.0393188)(669.08875615,40.08931875)(669.03875732,40.16931885)
\curveto(668.99875624,40.22931861)(668.97875626,40.30431853)(668.97875732,40.39431885)
\lineto(668.97875732,40.67931885)
\lineto(668.97875732,47.02431885)
\lineto(668.97875732,47.33931885)
\curveto(668.97875626,47.44931139)(669.00375623,47.5343113)(669.05375732,47.59431885)
\curveto(669.08375615,47.64431119)(669.12375611,47.67431116)(669.17375732,47.68431885)
\curveto(669.22375601,47.69431114)(669.27875596,47.70931113)(669.33875732,47.72931885)
\curveto(669.35875588,47.72931111)(669.37875586,47.72431111)(669.39875732,47.71431885)
\curveto(669.42875581,47.71431112)(669.45375578,47.71931112)(669.47375732,47.72931885)
\curveto(669.60375563,47.72931111)(669.7337555,47.72431111)(669.86375732,47.71431885)
\curveto(670.00375523,47.71431112)(670.09875514,47.67431116)(670.14875732,47.59431885)
\curveto(670.19875504,47.5343113)(670.22375501,47.45431138)(670.22375732,47.35431885)
\lineto(670.22375732,47.06931885)
\lineto(670.22375732,40.72431885)
}
}
{
\newrgbcolor{curcolor}{0 0 0}
\pscustom[linestyle=none,fillstyle=solid,fillcolor=curcolor]
{
\newpath
\moveto(674.59860107,47.90931885)
\curveto(675.31859701,47.91931092)(675.9235964,47.834311)(676.41360107,47.65431885)
\curveto(676.90359542,47.48431135)(677.28359504,47.17931166)(677.55360107,46.73931885)
\curveto(677.6235947,46.62931221)(677.67859465,46.51431232)(677.71860107,46.39431885)
\curveto(677.75859457,46.28431255)(677.79859453,46.15931268)(677.83860107,46.01931885)
\curveto(677.85859447,45.94931289)(677.86359446,45.87431296)(677.85360107,45.79431885)
\curveto(677.84359448,45.72431311)(677.8285945,45.66931317)(677.80860107,45.62931885)
\curveto(677.78859454,45.60931323)(677.76359456,45.58931325)(677.73360107,45.56931885)
\curveto(677.70359462,45.55931328)(677.67859465,45.54431329)(677.65860107,45.52431885)
\curveto(677.60859472,45.50431333)(677.55859477,45.49931334)(677.50860107,45.50931885)
\curveto(677.45859487,45.51931332)(677.40859492,45.51931332)(677.35860107,45.50931885)
\curveto(677.27859505,45.48931335)(677.17359515,45.48431335)(677.04360107,45.49431885)
\curveto(676.91359541,45.51431332)(676.8235955,45.5393133)(676.77360107,45.56931885)
\curveto(676.69359563,45.61931322)(676.63859569,45.68431315)(676.60860107,45.76431885)
\curveto(676.58859574,45.85431298)(676.55359577,45.9393129)(676.50360107,46.01931885)
\curveto(676.41359591,46.17931266)(676.28859604,46.32431251)(676.12860107,46.45431885)
\curveto(676.01859631,46.5343123)(675.89859643,46.59431224)(675.76860107,46.63431885)
\curveto(675.63859669,46.67431216)(675.49859683,46.71431212)(675.34860107,46.75431885)
\curveto(675.29859703,46.77431206)(675.24859708,46.77931206)(675.19860107,46.76931885)
\curveto(675.14859718,46.76931207)(675.09859723,46.77431206)(675.04860107,46.78431885)
\curveto(674.98859734,46.80431203)(674.91359741,46.81431202)(674.82360107,46.81431885)
\curveto(674.73359759,46.81431202)(674.65859767,46.80431203)(674.59860107,46.78431885)
\lineto(674.50860107,46.78431885)
\lineto(674.35860107,46.75431885)
\curveto(674.30859802,46.75431208)(674.25859807,46.74931209)(674.20860107,46.73931885)
\curveto(673.94859838,46.67931216)(673.73359859,46.59431224)(673.56360107,46.48431885)
\curveto(673.39359893,46.37431246)(673.27859905,46.18931265)(673.21860107,45.92931885)
\curveto(673.19859913,45.85931298)(673.19359913,45.78931305)(673.20360107,45.71931885)
\curveto(673.2235991,45.64931319)(673.24359908,45.58931325)(673.26360107,45.53931885)
\curveto(673.323599,45.38931345)(673.39359893,45.27931356)(673.47360107,45.20931885)
\curveto(673.56359876,45.14931369)(673.67359865,45.07931376)(673.80360107,44.99931885)
\curveto(673.96359836,44.89931394)(674.14359818,44.82431401)(674.34360107,44.77431885)
\curveto(674.54359778,44.7343141)(674.74359758,44.68431415)(674.94360107,44.62431885)
\curveto(675.07359725,44.58431425)(675.20359712,44.55431428)(675.33360107,44.53431885)
\curveto(675.46359686,44.51431432)(675.59359673,44.48431435)(675.72360107,44.44431885)
\curveto(675.93359639,44.38431445)(676.13859619,44.32431451)(676.33860107,44.26431885)
\curveto(676.53859579,44.21431462)(676.73859559,44.14931469)(676.93860107,44.06931885)
\lineto(677.08860107,44.00931885)
\curveto(677.13859519,43.98931485)(677.18859514,43.96431487)(677.23860107,43.93431885)
\curveto(677.43859489,43.81431502)(677.61359471,43.67931516)(677.76360107,43.52931885)
\curveto(677.91359441,43.37931546)(678.03859429,43.18931565)(678.13860107,42.95931885)
\curveto(678.15859417,42.88931595)(678.17859415,42.79431604)(678.19860107,42.67431885)
\curveto(678.21859411,42.60431623)(678.2285941,42.52931631)(678.22860107,42.44931885)
\curveto(678.23859409,42.37931646)(678.24359408,42.29931654)(678.24360107,42.20931885)
\lineto(678.24360107,42.05931885)
\curveto(678.2235941,41.98931685)(678.21359411,41.91931692)(678.21360107,41.84931885)
\curveto(678.21359411,41.77931706)(678.20359412,41.70931713)(678.18360107,41.63931885)
\curveto(678.15359417,41.52931731)(678.11859421,41.42431741)(678.07860107,41.32431885)
\curveto(678.03859429,41.22431761)(677.99359433,41.1343177)(677.94360107,41.05431885)
\curveto(677.78359454,40.79431804)(677.57859475,40.58431825)(677.32860107,40.42431885)
\curveto(677.07859525,40.27431856)(676.79859553,40.14431869)(676.48860107,40.03431885)
\curveto(676.39859593,40.00431883)(676.30359602,39.98431885)(676.20360107,39.97431885)
\curveto(676.11359621,39.95431888)(676.0235963,39.92931891)(675.93360107,39.89931885)
\curveto(675.83359649,39.87931896)(675.73359659,39.86931897)(675.63360107,39.86931885)
\curveto(675.53359679,39.86931897)(675.43359689,39.85931898)(675.33360107,39.83931885)
\lineto(675.18360107,39.83931885)
\curveto(675.13359719,39.82931901)(675.06359726,39.82431901)(674.97360107,39.82431885)
\curveto(674.88359744,39.82431901)(674.81359751,39.82931901)(674.76360107,39.83931885)
\lineto(674.59860107,39.83931885)
\curveto(674.53859779,39.85931898)(674.47359785,39.86931897)(674.40360107,39.86931885)
\curveto(674.33359799,39.85931898)(674.27359805,39.86431897)(674.22360107,39.88431885)
\curveto(674.17359815,39.89431894)(674.10859822,39.89931894)(674.02860107,39.89931885)
\lineto(673.78860107,39.95931885)
\curveto(673.71859861,39.96931887)(673.64359868,39.98931885)(673.56360107,40.01931885)
\curveto(673.25359907,40.11931872)(672.98359934,40.24431859)(672.75360107,40.39431885)
\curveto(672.5235998,40.54431829)(672.3236,40.7393181)(672.15360107,40.97931885)
\curveto(672.06360026,41.10931773)(671.98860034,41.24431759)(671.92860107,41.38431885)
\curveto(671.86860046,41.52431731)(671.81360051,41.67931716)(671.76360107,41.84931885)
\curveto(671.74360058,41.90931693)(671.73360059,41.97931686)(671.73360107,42.05931885)
\curveto(671.74360058,42.14931669)(671.75860057,42.21931662)(671.77860107,42.26931885)
\curveto(671.80860052,42.30931653)(671.85860047,42.34931649)(671.92860107,42.38931885)
\curveto(671.97860035,42.40931643)(672.04860028,42.41931642)(672.13860107,42.41931885)
\curveto(672.2286001,42.42931641)(672.31860001,42.42931641)(672.40860107,42.41931885)
\curveto(672.49859983,42.40931643)(672.58359974,42.39431644)(672.66360107,42.37431885)
\curveto(672.75359957,42.36431647)(672.81359951,42.34931649)(672.84360107,42.32931885)
\curveto(672.91359941,42.27931656)(672.95859937,42.20431663)(672.97860107,42.10431885)
\curveto(673.00859932,42.01431682)(673.04359928,41.92931691)(673.08360107,41.84931885)
\curveto(673.18359914,41.62931721)(673.31859901,41.45931738)(673.48860107,41.33931885)
\curveto(673.60859872,41.24931759)(673.74359858,41.17931766)(673.89360107,41.12931885)
\curveto(674.04359828,41.07931776)(674.20359812,41.02931781)(674.37360107,40.97931885)
\lineto(674.68860107,40.93431885)
\lineto(674.77860107,40.93431885)
\curveto(674.84859748,40.91431792)(674.93859739,40.90431793)(675.04860107,40.90431885)
\curveto(675.16859716,40.90431793)(675.26859706,40.91431792)(675.34860107,40.93431885)
\curveto(675.41859691,40.9343179)(675.47359685,40.9393179)(675.51360107,40.94931885)
\curveto(675.57359675,40.95931788)(675.63359669,40.96431787)(675.69360107,40.96431885)
\curveto(675.75359657,40.97431786)(675.80859652,40.98431785)(675.85860107,40.99431885)
\curveto(676.14859618,41.07431776)(676.37859595,41.17931766)(676.54860107,41.30931885)
\curveto(676.71859561,41.4393174)(676.83859549,41.65931718)(676.90860107,41.96931885)
\curveto(676.9285954,42.01931682)(676.93359539,42.07431676)(676.92360107,42.13431885)
\curveto(676.91359541,42.19431664)(676.90359542,42.2393166)(676.89360107,42.26931885)
\curveto(676.84359548,42.45931638)(676.77359555,42.59931624)(676.68360107,42.68931885)
\curveto(676.59359573,42.78931605)(676.47859585,42.87931596)(676.33860107,42.95931885)
\curveto(676.24859608,43.01931582)(676.14859618,43.06931577)(676.03860107,43.10931885)
\lineto(675.70860107,43.22931885)
\curveto(675.67859665,43.2393156)(675.64859668,43.24431559)(675.61860107,43.24431885)
\curveto(675.59859673,43.24431559)(675.57359675,43.25431558)(675.54360107,43.27431885)
\curveto(675.20359712,43.38431545)(674.84859748,43.46431537)(674.47860107,43.51431885)
\curveto(674.11859821,43.57431526)(673.77859855,43.66931517)(673.45860107,43.79931885)
\curveto(673.35859897,43.839315)(673.26359906,43.87431496)(673.17360107,43.90431885)
\curveto(673.08359924,43.9343149)(672.99859933,43.97431486)(672.91860107,44.02431885)
\curveto(672.7285996,44.1343147)(672.55359977,44.25931458)(672.39360107,44.39931885)
\curveto(672.23360009,44.5393143)(672.10860022,44.71431412)(672.01860107,44.92431885)
\curveto(671.98860034,44.99431384)(671.96360036,45.06431377)(671.94360107,45.13431885)
\curveto(671.93360039,45.20431363)(671.91860041,45.27931356)(671.89860107,45.35931885)
\curveto(671.86860046,45.47931336)(671.85860047,45.61431322)(671.86860107,45.76431885)
\curveto(671.87860045,45.92431291)(671.89360043,46.05931278)(671.91360107,46.16931885)
\curveto(671.93360039,46.21931262)(671.94360038,46.25931258)(671.94360107,46.28931885)
\curveto(671.95360037,46.32931251)(671.96860036,46.36931247)(671.98860107,46.40931885)
\curveto(672.07860025,46.6393122)(672.19860013,46.839312)(672.34860107,47.00931885)
\curveto(672.50859982,47.17931166)(672.68859964,47.32931151)(672.88860107,47.45931885)
\curveto(673.03859929,47.54931129)(673.20359912,47.61931122)(673.38360107,47.66931885)
\curveto(673.56359876,47.72931111)(673.75359857,47.78431105)(673.95360107,47.83431885)
\curveto(674.0235983,47.84431099)(674.08859824,47.85431098)(674.14860107,47.86431885)
\curveto(674.21859811,47.87431096)(674.29359803,47.88431095)(674.37360107,47.89431885)
\curveto(674.40359792,47.90431093)(674.44359788,47.90431093)(674.49360107,47.89431885)
\curveto(674.54359778,47.88431095)(674.57859775,47.88931095)(674.59860107,47.90931885)
}
}
{
\newrgbcolor{curcolor}{0 0 0}
\pscustom[linestyle=none,fillstyle=solid,fillcolor=curcolor]
{
\newpath
\moveto(680.61360107,50.06931885)
\curveto(680.76359906,50.06930877)(680.91359891,50.06430877)(681.06360107,50.05431885)
\curveto(681.21359861,50.05430878)(681.31859851,50.01430882)(681.37860107,49.93431885)
\curveto(681.4285984,49.87430896)(681.45359837,49.78930905)(681.45360107,49.67931885)
\curveto(681.46359836,49.57930926)(681.46859836,49.47430936)(681.46860107,49.36431885)
\lineto(681.46860107,48.49431885)
\curveto(681.46859836,48.41431042)(681.46359836,48.32931051)(681.45360107,48.23931885)
\curveto(681.45359837,48.15931068)(681.46359836,48.08931075)(681.48360107,48.02931885)
\curveto(681.5235983,47.88931095)(681.61359821,47.79931104)(681.75360107,47.75931885)
\curveto(681.80359802,47.74931109)(681.84859798,47.74431109)(681.88860107,47.74431885)
\lineto(682.03860107,47.74431885)
\lineto(682.44360107,47.74431885)
\curveto(682.60359722,47.75431108)(682.71859711,47.74431109)(682.78860107,47.71431885)
\curveto(682.87859695,47.65431118)(682.93859689,47.59431124)(682.96860107,47.53431885)
\curveto(682.98859684,47.49431134)(682.99859683,47.44931139)(682.99860107,47.39931885)
\lineto(682.99860107,47.24931885)
\curveto(682.99859683,47.1393117)(682.99359683,47.0343118)(682.98360107,46.93431885)
\curveto(682.97359685,46.84431199)(682.93859689,46.77431206)(682.87860107,46.72431885)
\curveto(682.81859701,46.67431216)(682.73359709,46.64431219)(682.62360107,46.63431885)
\lineto(682.29360107,46.63431885)
\curveto(682.18359764,46.64431219)(682.07359775,46.64931219)(681.96360107,46.64931885)
\curveto(681.85359797,46.64931219)(681.75859807,46.6343122)(681.67860107,46.60431885)
\curveto(681.60859822,46.57431226)(681.55859827,46.52431231)(681.52860107,46.45431885)
\curveto(681.49859833,46.38431245)(681.47859835,46.29931254)(681.46860107,46.19931885)
\curveto(681.45859837,46.10931273)(681.45359837,46.00931283)(681.45360107,45.89931885)
\curveto(681.46359836,45.79931304)(681.46859836,45.69931314)(681.46860107,45.59931885)
\lineto(681.46860107,42.62931885)
\curveto(681.46859836,42.40931643)(681.46359836,42.17431666)(681.45360107,41.92431885)
\curveto(681.45359837,41.68431715)(681.49859833,41.49931734)(681.58860107,41.36931885)
\curveto(681.63859819,41.28931755)(681.70359812,41.2343176)(681.78360107,41.20431885)
\curveto(681.86359796,41.17431766)(681.95859787,41.14931769)(682.06860107,41.12931885)
\curveto(682.09859773,41.11931772)(682.1285977,41.11431772)(682.15860107,41.11431885)
\curveto(682.19859763,41.12431771)(682.23359759,41.12431771)(682.26360107,41.11431885)
\lineto(682.45860107,41.11431885)
\curveto(682.55859727,41.11431772)(682.64859718,41.10431773)(682.72860107,41.08431885)
\curveto(682.81859701,41.07431776)(682.88359694,41.0393178)(682.92360107,40.97931885)
\curveto(682.94359688,40.94931789)(682.95859687,40.89431794)(682.96860107,40.81431885)
\curveto(682.98859684,40.74431809)(682.99859683,40.66931817)(682.99860107,40.58931885)
\curveto(683.00859682,40.50931833)(683.00859682,40.42931841)(682.99860107,40.34931885)
\curveto(682.98859684,40.27931856)(682.96859686,40.22431861)(682.93860107,40.18431885)
\curveto(682.89859693,40.11431872)(682.823597,40.06431877)(682.71360107,40.03431885)
\curveto(682.63359719,40.01431882)(682.54359728,40.00431883)(682.44360107,40.00431885)
\curveto(682.34359748,40.01431882)(682.25359757,40.01931882)(682.17360107,40.01931885)
\curveto(682.11359771,40.01931882)(682.05359777,40.01431882)(681.99360107,40.00431885)
\curveto(681.93359789,40.00431883)(681.87859795,40.00931883)(681.82860107,40.01931885)
\lineto(681.64860107,40.01931885)
\curveto(681.59859823,40.02931881)(681.54859828,40.0343188)(681.49860107,40.03431885)
\curveto(681.45859837,40.04431879)(681.41359841,40.04931879)(681.36360107,40.04931885)
\curveto(681.16359866,40.09931874)(680.98859884,40.15431868)(680.83860107,40.21431885)
\curveto(680.69859913,40.27431856)(680.57859925,40.37931846)(680.47860107,40.52931885)
\curveto(680.33859949,40.72931811)(680.25859957,40.97931786)(680.23860107,41.27931885)
\curveto(680.21859961,41.58931725)(680.20859962,41.91931692)(680.20860107,42.26931885)
\lineto(680.20860107,46.19931885)
\curveto(680.17859965,46.32931251)(680.14859968,46.42431241)(680.11860107,46.48431885)
\curveto(680.09859973,46.54431229)(680.0285998,46.59431224)(679.90860107,46.63431885)
\curveto(679.86859996,46.64431219)(679.8286,46.64431219)(679.78860107,46.63431885)
\curveto(679.74860008,46.62431221)(679.70860012,46.62931221)(679.66860107,46.64931885)
\lineto(679.42860107,46.64931885)
\curveto(679.29860053,46.64931219)(679.18860064,46.65931218)(679.09860107,46.67931885)
\curveto(679.01860081,46.70931213)(678.96360086,46.76931207)(678.93360107,46.85931885)
\curveto(678.91360091,46.89931194)(678.89860093,46.94431189)(678.88860107,46.99431885)
\lineto(678.88860107,47.14431885)
\curveto(678.88860094,47.28431155)(678.89860093,47.39931144)(678.91860107,47.48931885)
\curveto(678.93860089,47.58931125)(678.99860083,47.66431117)(679.09860107,47.71431885)
\curveto(679.20860062,47.75431108)(679.34860048,47.76431107)(679.51860107,47.74431885)
\curveto(679.69860013,47.72431111)(679.84859998,47.7343111)(679.96860107,47.77431885)
\curveto(680.05859977,47.82431101)(680.1285997,47.89431094)(680.17860107,47.98431885)
\curveto(680.19859963,48.04431079)(680.20859962,48.11931072)(680.20860107,48.20931885)
\lineto(680.20860107,48.46431885)
\lineto(680.20860107,49.39431885)
\lineto(680.20860107,49.63431885)
\curveto(680.20859962,49.72430911)(680.21859961,49.79930904)(680.23860107,49.85931885)
\curveto(680.27859955,49.9393089)(680.35359947,50.00430883)(680.46360107,50.05431885)
\curveto(680.49359933,50.05430878)(680.51859931,50.05430878)(680.53860107,50.05431885)
\curveto(680.56859926,50.06430877)(680.59359923,50.06930877)(680.61360107,50.06931885)
}
}
{
\newrgbcolor{curcolor}{0 0 0}
\pscustom[linestyle=none,fillstyle=solid,fillcolor=curcolor]
{
\newpath
\moveto(688.03039795,47.90931885)
\curveto(688.26039316,47.90931093)(688.39039303,47.84931099)(688.42039795,47.72931885)
\curveto(688.45039297,47.61931122)(688.46539295,47.45431138)(688.46539795,47.23431885)
\lineto(688.46539795,46.94931885)
\curveto(688.46539295,46.85931198)(688.44039298,46.78431205)(688.39039795,46.72431885)
\curveto(688.33039309,46.64431219)(688.24539317,46.59931224)(688.13539795,46.58931885)
\curveto(688.02539339,46.58931225)(687.9153935,46.57431226)(687.80539795,46.54431885)
\curveto(687.66539375,46.51431232)(687.53039389,46.48431235)(687.40039795,46.45431885)
\curveto(687.28039414,46.42431241)(687.16539425,46.38431245)(687.05539795,46.33431885)
\curveto(686.76539465,46.20431263)(686.53039489,46.02431281)(686.35039795,45.79431885)
\curveto(686.17039525,45.57431326)(686.0153954,45.31931352)(685.88539795,45.02931885)
\curveto(685.84539557,44.91931392)(685.8153956,44.80431403)(685.79539795,44.68431885)
\curveto(685.77539564,44.57431426)(685.75039567,44.45931438)(685.72039795,44.33931885)
\curveto(685.71039571,44.28931455)(685.70539571,44.2393146)(685.70539795,44.18931885)
\curveto(685.7153957,44.1393147)(685.7153957,44.08931475)(685.70539795,44.03931885)
\curveto(685.67539574,43.91931492)(685.66039576,43.77931506)(685.66039795,43.61931885)
\curveto(685.67039575,43.46931537)(685.67539574,43.32431551)(685.67539795,43.18431885)
\lineto(685.67539795,41.33931885)
\lineto(685.67539795,40.99431885)
\curveto(685.67539574,40.87431796)(685.67039575,40.75931808)(685.66039795,40.64931885)
\curveto(685.65039577,40.5393183)(685.64539577,40.44431839)(685.64539795,40.36431885)
\curveto(685.65539576,40.28431855)(685.63539578,40.21431862)(685.58539795,40.15431885)
\curveto(685.53539588,40.08431875)(685.45539596,40.04431879)(685.34539795,40.03431885)
\curveto(685.24539617,40.02431881)(685.13539628,40.01931882)(685.01539795,40.01931885)
\lineto(684.74539795,40.01931885)
\curveto(684.69539672,40.0393188)(684.64539677,40.05431878)(684.59539795,40.06431885)
\curveto(684.55539686,40.08431875)(684.52539689,40.10931873)(684.50539795,40.13931885)
\curveto(684.45539696,40.20931863)(684.42539699,40.29431854)(684.41539795,40.39431885)
\lineto(684.41539795,40.72431885)
\lineto(684.41539795,41.87931885)
\lineto(684.41539795,46.03431885)
\lineto(684.41539795,47.06931885)
\lineto(684.41539795,47.36931885)
\curveto(684.42539699,47.46931137)(684.45539696,47.55431128)(684.50539795,47.62431885)
\curveto(684.53539688,47.66431117)(684.58539683,47.69431114)(684.65539795,47.71431885)
\curveto(684.73539668,47.7343111)(684.8203966,47.74431109)(684.91039795,47.74431885)
\curveto(685.00039642,47.75431108)(685.09039633,47.75431108)(685.18039795,47.74431885)
\curveto(685.27039615,47.7343111)(685.34039608,47.71931112)(685.39039795,47.69931885)
\curveto(685.47039595,47.66931117)(685.5203959,47.60931123)(685.54039795,47.51931885)
\curveto(685.57039585,47.4393114)(685.58539583,47.34931149)(685.58539795,47.24931885)
\lineto(685.58539795,46.94931885)
\curveto(685.58539583,46.84931199)(685.60539581,46.75931208)(685.64539795,46.67931885)
\curveto(685.65539576,46.65931218)(685.66539575,46.64431219)(685.67539795,46.63431885)
\lineto(685.72039795,46.58931885)
\curveto(685.83039559,46.58931225)(685.9203955,46.6343122)(685.99039795,46.72431885)
\curveto(686.06039536,46.82431201)(686.1203953,46.90431193)(686.17039795,46.96431885)
\lineto(686.26039795,47.05431885)
\curveto(686.35039507,47.16431167)(686.47539494,47.27931156)(686.63539795,47.39931885)
\curveto(686.79539462,47.51931132)(686.94539447,47.60931123)(687.08539795,47.66931885)
\curveto(687.17539424,47.71931112)(687.27039415,47.75431108)(687.37039795,47.77431885)
\curveto(687.47039395,47.80431103)(687.57539384,47.834311)(687.68539795,47.86431885)
\curveto(687.74539367,47.87431096)(687.80539361,47.87931096)(687.86539795,47.87931885)
\curveto(687.92539349,47.88931095)(687.98039344,47.89931094)(688.03039795,47.90931885)
}
}
{
\newrgbcolor{curcolor}{0 0 0}
\pscustom[linestyle=none,fillstyle=solid,fillcolor=curcolor]
{
\newpath
\moveto(696.28016357,40.55931885)
\curveto(696.31015574,40.39931844)(696.29515576,40.26431857)(696.23516357,40.15431885)
\curveto(696.17515588,40.05431878)(696.09515596,39.97931886)(695.99516357,39.92931885)
\curveto(695.94515611,39.90931893)(695.89015616,39.89931894)(695.83016357,39.89931885)
\curveto(695.78015627,39.89931894)(695.72515633,39.88931895)(695.66516357,39.86931885)
\curveto(695.44515661,39.81931902)(695.22515683,39.834319)(695.00516357,39.91431885)
\curveto(694.79515726,39.98431885)(694.6501574,40.07431876)(694.57016357,40.18431885)
\curveto(694.52015753,40.25431858)(694.47515758,40.3343185)(694.43516357,40.42431885)
\curveto(694.39515766,40.52431831)(694.34515771,40.60431823)(694.28516357,40.66431885)
\curveto(694.26515779,40.68431815)(694.24015781,40.70431813)(694.21016357,40.72431885)
\curveto(694.19015786,40.74431809)(694.16015789,40.74931809)(694.12016357,40.73931885)
\curveto(694.01015804,40.70931813)(693.90515815,40.65431818)(693.80516357,40.57431885)
\curveto(693.71515834,40.49431834)(693.62515843,40.42431841)(693.53516357,40.36431885)
\curveto(693.40515865,40.28431855)(693.26515879,40.20931863)(693.11516357,40.13931885)
\curveto(692.96515909,40.07931876)(692.80515925,40.02431881)(692.63516357,39.97431885)
\curveto(692.53515952,39.94431889)(692.42515963,39.92431891)(692.30516357,39.91431885)
\curveto(692.19515986,39.90431893)(692.08515997,39.88931895)(691.97516357,39.86931885)
\curveto(691.92516013,39.85931898)(691.88016017,39.85431898)(691.84016357,39.85431885)
\lineto(691.73516357,39.85431885)
\curveto(691.62516043,39.834319)(691.52016053,39.834319)(691.42016357,39.85431885)
\lineto(691.28516357,39.85431885)
\curveto(691.23516082,39.86431897)(691.18516087,39.86931897)(691.13516357,39.86931885)
\curveto(691.08516097,39.86931897)(691.04016101,39.87931896)(691.00016357,39.89931885)
\curveto(690.96016109,39.90931893)(690.92516113,39.91431892)(690.89516357,39.91431885)
\curveto(690.87516118,39.90431893)(690.8501612,39.90431893)(690.82016357,39.91431885)
\lineto(690.58016357,39.97431885)
\curveto(690.50016155,39.98431885)(690.42516163,40.00431883)(690.35516357,40.03431885)
\curveto(690.055162,40.16431867)(689.81016224,40.30931853)(689.62016357,40.46931885)
\curveto(689.44016261,40.6393182)(689.29016276,40.87431796)(689.17016357,41.17431885)
\curveto(689.08016297,41.39431744)(689.03516302,41.65931718)(689.03516357,41.96931885)
\lineto(689.03516357,42.28431885)
\curveto(689.04516301,42.3343165)(689.050163,42.38431645)(689.05016357,42.43431885)
\lineto(689.08016357,42.61431885)
\lineto(689.20016357,42.94431885)
\curveto(689.24016281,43.05431578)(689.29016276,43.15431568)(689.35016357,43.24431885)
\curveto(689.53016252,43.5343153)(689.77516228,43.74931509)(690.08516357,43.88931885)
\curveto(690.39516166,44.02931481)(690.73516132,44.15431468)(691.10516357,44.26431885)
\curveto(691.24516081,44.30431453)(691.39016066,44.3343145)(691.54016357,44.35431885)
\curveto(691.69016036,44.37431446)(691.84016021,44.39931444)(691.99016357,44.42931885)
\curveto(692.06015999,44.44931439)(692.12515993,44.45931438)(692.18516357,44.45931885)
\curveto(692.2551598,44.45931438)(692.33015972,44.46931437)(692.41016357,44.48931885)
\curveto(692.48015957,44.50931433)(692.5501595,44.51931432)(692.62016357,44.51931885)
\curveto(692.69015936,44.52931431)(692.76515929,44.54431429)(692.84516357,44.56431885)
\curveto(693.09515896,44.62431421)(693.33015872,44.67431416)(693.55016357,44.71431885)
\curveto(693.77015828,44.76431407)(693.94515811,44.87931396)(694.07516357,45.05931885)
\curveto(694.13515792,45.1393137)(694.18515787,45.2393136)(694.22516357,45.35931885)
\curveto(694.26515779,45.48931335)(694.26515779,45.62931321)(694.22516357,45.77931885)
\curveto(694.16515789,46.01931282)(694.07515798,46.20931263)(693.95516357,46.34931885)
\curveto(693.84515821,46.48931235)(693.68515837,46.59931224)(693.47516357,46.67931885)
\curveto(693.3551587,46.72931211)(693.21015884,46.76431207)(693.04016357,46.78431885)
\curveto(692.88015917,46.80431203)(692.71015934,46.81431202)(692.53016357,46.81431885)
\curveto(692.3501597,46.81431202)(692.17515988,46.80431203)(692.00516357,46.78431885)
\curveto(691.83516022,46.76431207)(691.69016036,46.7343121)(691.57016357,46.69431885)
\curveto(691.40016065,46.6343122)(691.23516082,46.54931229)(691.07516357,46.43931885)
\curveto(690.99516106,46.37931246)(690.92016113,46.29931254)(690.85016357,46.19931885)
\curveto(690.79016126,46.10931273)(690.73516132,46.00931283)(690.68516357,45.89931885)
\curveto(690.6551614,45.81931302)(690.62516143,45.7343131)(690.59516357,45.64431885)
\curveto(690.57516148,45.55431328)(690.53016152,45.48431335)(690.46016357,45.43431885)
\curveto(690.42016163,45.40431343)(690.3501617,45.37931346)(690.25016357,45.35931885)
\curveto(690.16016189,45.34931349)(690.06516199,45.34431349)(689.96516357,45.34431885)
\curveto(689.86516219,45.34431349)(689.76516229,45.34931349)(689.66516357,45.35931885)
\curveto(689.57516248,45.37931346)(689.51016254,45.40431343)(689.47016357,45.43431885)
\curveto(689.43016262,45.46431337)(689.40016265,45.51431332)(689.38016357,45.58431885)
\curveto(689.36016269,45.65431318)(689.36016269,45.72931311)(689.38016357,45.80931885)
\curveto(689.41016264,45.9393129)(689.44016261,46.05931278)(689.47016357,46.16931885)
\curveto(689.51016254,46.28931255)(689.5551625,46.40431243)(689.60516357,46.51431885)
\curveto(689.79516226,46.86431197)(690.03516202,47.1343117)(690.32516357,47.32431885)
\curveto(690.61516144,47.52431131)(690.97516108,47.68431115)(691.40516357,47.80431885)
\curveto(691.50516055,47.82431101)(691.60516045,47.839311)(691.70516357,47.84931885)
\curveto(691.81516024,47.85931098)(691.92516013,47.87431096)(692.03516357,47.89431885)
\curveto(692.07515998,47.90431093)(692.14015991,47.90431093)(692.23016357,47.89431885)
\curveto(692.32015973,47.89431094)(692.37515968,47.90431093)(692.39516357,47.92431885)
\curveto(693.09515896,47.9343109)(693.70515835,47.85431098)(694.22516357,47.68431885)
\curveto(694.74515731,47.51431132)(695.11015694,47.18931165)(695.32016357,46.70931885)
\curveto(695.41015664,46.50931233)(695.46015659,46.27431256)(695.47016357,46.00431885)
\curveto(695.49015656,45.74431309)(695.50015655,45.46931337)(695.50016357,45.17931885)
\lineto(695.50016357,41.86431885)
\curveto(695.50015655,41.72431711)(695.50515655,41.58931725)(695.51516357,41.45931885)
\curveto(695.52515653,41.32931751)(695.5551565,41.22431761)(695.60516357,41.14431885)
\curveto(695.6551564,41.07431776)(695.72015633,41.02431781)(695.80016357,40.99431885)
\curveto(695.89015616,40.95431788)(695.97515608,40.92431791)(696.05516357,40.90431885)
\curveto(696.13515592,40.89431794)(696.19515586,40.84931799)(696.23516357,40.76931885)
\curveto(696.2551558,40.7393181)(696.26515579,40.70931813)(696.26516357,40.67931885)
\curveto(696.26515579,40.64931819)(696.27015578,40.60931823)(696.28016357,40.55931885)
\moveto(694.13516357,42.22431885)
\curveto(694.19515786,42.36431647)(694.22515783,42.52431631)(694.22516357,42.70431885)
\curveto(694.23515782,42.89431594)(694.24015781,43.08931575)(694.24016357,43.28931885)
\curveto(694.24015781,43.39931544)(694.23515782,43.49931534)(694.22516357,43.58931885)
\curveto(694.21515784,43.67931516)(694.17515788,43.74931509)(694.10516357,43.79931885)
\curveto(694.07515798,43.81931502)(694.00515805,43.82931501)(693.89516357,43.82931885)
\curveto(693.87515818,43.80931503)(693.84015821,43.79931504)(693.79016357,43.79931885)
\curveto(693.74015831,43.79931504)(693.69515836,43.78931505)(693.65516357,43.76931885)
\curveto(693.57515848,43.74931509)(693.48515857,43.72931511)(693.38516357,43.70931885)
\lineto(693.08516357,43.64931885)
\curveto(693.055159,43.64931519)(693.02015903,43.64431519)(692.98016357,43.63431885)
\lineto(692.87516357,43.63431885)
\curveto(692.72515933,43.59431524)(692.56015949,43.56931527)(692.38016357,43.55931885)
\curveto(692.21015984,43.55931528)(692.05016,43.5393153)(691.90016357,43.49931885)
\curveto(691.82016023,43.47931536)(691.74516031,43.45931538)(691.67516357,43.43931885)
\curveto(691.61516044,43.42931541)(691.54516051,43.41431542)(691.46516357,43.39431885)
\curveto(691.30516075,43.34431549)(691.1551609,43.27931556)(691.01516357,43.19931885)
\curveto(690.87516118,43.12931571)(690.7551613,43.0393158)(690.65516357,42.92931885)
\curveto(690.5551615,42.81931602)(690.48016157,42.68431615)(690.43016357,42.52431885)
\curveto(690.38016167,42.37431646)(690.36016169,42.18931665)(690.37016357,41.96931885)
\curveto(690.37016168,41.86931697)(690.38516167,41.77431706)(690.41516357,41.68431885)
\curveto(690.4551616,41.60431723)(690.50016155,41.52931731)(690.55016357,41.45931885)
\curveto(690.63016142,41.34931749)(690.73516132,41.25431758)(690.86516357,41.17431885)
\curveto(690.99516106,41.10431773)(691.13516092,41.04431779)(691.28516357,40.99431885)
\curveto(691.33516072,40.98431785)(691.38516067,40.97931786)(691.43516357,40.97931885)
\curveto(691.48516057,40.97931786)(691.53516052,40.97431786)(691.58516357,40.96431885)
\curveto(691.6551604,40.94431789)(691.74016031,40.92931791)(691.84016357,40.91931885)
\curveto(691.9501601,40.91931792)(692.04016001,40.92931791)(692.11016357,40.94931885)
\curveto(692.17015988,40.96931787)(692.23015982,40.97431786)(692.29016357,40.96431885)
\curveto(692.3501597,40.96431787)(692.41015964,40.97431786)(692.47016357,40.99431885)
\curveto(692.5501595,41.01431782)(692.62515943,41.02931781)(692.69516357,41.03931885)
\curveto(692.77515928,41.04931779)(692.8501592,41.06931777)(692.92016357,41.09931885)
\curveto(693.21015884,41.21931762)(693.4551586,41.36431747)(693.65516357,41.53431885)
\curveto(693.86515819,41.70431713)(694.02515803,41.9343169)(694.13516357,42.22431885)
}
}
{
\newrgbcolor{curcolor}{0 0 0}
\pscustom[linestyle=none,fillstyle=solid,fillcolor=curcolor]
{
\newpath
\moveto(704.4118042,40.81431885)
\lineto(704.4118042,40.42431885)
\curveto(704.41179632,40.30431853)(704.38679635,40.20431863)(704.3368042,40.12431885)
\curveto(704.28679645,40.05431878)(704.20179653,40.01431882)(704.0818042,40.00431885)
\lineto(703.7368042,40.00431885)
\curveto(703.67679706,40.00431883)(703.61679712,39.99931884)(703.5568042,39.98931885)
\curveto(703.50679723,39.98931885)(703.46179727,39.99931884)(703.4218042,40.01931885)
\curveto(703.3317974,40.0393188)(703.27179746,40.07931876)(703.2418042,40.13931885)
\curveto(703.20179753,40.18931865)(703.17679756,40.24931859)(703.1668042,40.31931885)
\curveto(703.16679757,40.38931845)(703.15179758,40.45931838)(703.1218042,40.52931885)
\curveto(703.11179762,40.54931829)(703.09679764,40.56431827)(703.0768042,40.57431885)
\curveto(703.06679767,40.59431824)(703.05179768,40.61431822)(703.0318042,40.63431885)
\curveto(702.9317978,40.64431819)(702.85179788,40.62431821)(702.7918042,40.57431885)
\curveto(702.74179799,40.52431831)(702.68679805,40.47431836)(702.6268042,40.42431885)
\curveto(702.42679831,40.27431856)(702.22679851,40.15931868)(702.0268042,40.07931885)
\curveto(701.84679889,39.99931884)(701.6367991,39.9393189)(701.3968042,39.89931885)
\curveto(701.16679957,39.85931898)(700.92679981,39.839319)(700.6768042,39.83931885)
\curveto(700.4368003,39.82931901)(700.19680054,39.84431899)(699.9568042,39.88431885)
\curveto(699.71680102,39.91431892)(699.50680123,39.96931887)(699.3268042,40.04931885)
\curveto(698.80680193,40.26931857)(698.38680235,40.56431827)(698.0668042,40.93431885)
\curveto(697.74680299,41.31431752)(697.49680324,41.78431705)(697.3168042,42.34431885)
\curveto(697.27680346,42.4343164)(697.24680349,42.52431631)(697.2268042,42.61431885)
\curveto(697.21680352,42.71431612)(697.19680354,42.81431602)(697.1668042,42.91431885)
\curveto(697.15680358,42.96431587)(697.15180358,43.01431582)(697.1518042,43.06431885)
\curveto(697.15180358,43.11431572)(697.14680359,43.16431567)(697.1368042,43.21431885)
\curveto(697.11680362,43.26431557)(697.10680363,43.31431552)(697.1068042,43.36431885)
\curveto(697.11680362,43.42431541)(697.11680362,43.47931536)(697.1068042,43.52931885)
\lineto(697.1068042,43.67931885)
\curveto(697.08680365,43.72931511)(697.07680366,43.79431504)(697.0768042,43.87431885)
\curveto(697.07680366,43.95431488)(697.08680365,44.01931482)(697.1068042,44.06931885)
\lineto(697.1068042,44.23431885)
\curveto(697.12680361,44.30431453)(697.1318036,44.37431446)(697.1218042,44.44431885)
\curveto(697.12180361,44.52431431)(697.1318036,44.59931424)(697.1518042,44.66931885)
\curveto(697.16180357,44.71931412)(697.16680357,44.76431407)(697.1668042,44.80431885)
\curveto(697.16680357,44.84431399)(697.17180356,44.88931395)(697.1818042,44.93931885)
\curveto(697.21180352,45.0393138)(697.2368035,45.1343137)(697.2568042,45.22431885)
\curveto(697.27680346,45.32431351)(697.30180343,45.41931342)(697.3318042,45.50931885)
\curveto(697.46180327,45.88931295)(697.62680311,46.22931261)(697.8268042,46.52931885)
\curveto(698.0368027,46.839312)(698.28680245,47.09431174)(698.5768042,47.29431885)
\curveto(698.74680199,47.41431142)(698.92180181,47.51431132)(699.1018042,47.59431885)
\curveto(699.29180144,47.67431116)(699.49680124,47.74431109)(699.7168042,47.80431885)
\curveto(699.78680095,47.81431102)(699.85180088,47.82431101)(699.9118042,47.83431885)
\curveto(699.98180075,47.84431099)(700.05180068,47.85931098)(700.1218042,47.87931885)
\lineto(700.2718042,47.87931885)
\curveto(700.35180038,47.89931094)(700.46680027,47.90931093)(700.6168042,47.90931885)
\curveto(700.77679996,47.90931093)(700.89679984,47.89931094)(700.9768042,47.87931885)
\curveto(701.01679972,47.86931097)(701.07179966,47.86431097)(701.1418042,47.86431885)
\curveto(701.25179948,47.834311)(701.36179937,47.80931103)(701.4718042,47.78931885)
\curveto(701.58179915,47.77931106)(701.68679905,47.74931109)(701.7868042,47.69931885)
\curveto(701.9367988,47.6393112)(702.07679866,47.57431126)(702.2068042,47.50431885)
\curveto(702.34679839,47.4343114)(702.47679826,47.35431148)(702.5968042,47.26431885)
\curveto(702.65679808,47.21431162)(702.71679802,47.15931168)(702.7768042,47.09931885)
\curveto(702.84679789,47.04931179)(702.9367978,47.0343118)(703.0468042,47.05431885)
\curveto(703.06679767,47.08431175)(703.08179765,47.10931173)(703.0918042,47.12931885)
\curveto(703.11179762,47.14931169)(703.12679761,47.17931166)(703.1368042,47.21931885)
\curveto(703.16679757,47.30931153)(703.17679756,47.42431141)(703.1668042,47.56431885)
\lineto(703.1668042,47.93931885)
\lineto(703.1668042,49.66431885)
\lineto(703.1668042,50.12931885)
\curveto(703.16679757,50.30930853)(703.19179754,50.4393084)(703.2418042,50.51931885)
\curveto(703.28179745,50.58930825)(703.34179739,50.6343082)(703.4218042,50.65431885)
\curveto(703.44179729,50.65430818)(703.46679727,50.65430818)(703.4968042,50.65431885)
\curveto(703.52679721,50.66430817)(703.55179718,50.66930817)(703.5718042,50.66931885)
\curveto(703.71179702,50.67930816)(703.85679688,50.67930816)(704.0068042,50.66931885)
\curveto(704.16679657,50.66930817)(704.27679646,50.62930821)(704.3368042,50.54931885)
\curveto(704.38679635,50.46930837)(704.41179632,50.36930847)(704.4118042,50.24931885)
\lineto(704.4118042,49.87431885)
\lineto(704.4118042,40.81431885)
\moveto(703.1968042,43.64931885)
\curveto(703.21679752,43.69931514)(703.22679751,43.76431507)(703.2268042,43.84431885)
\curveto(703.22679751,43.9343149)(703.21679752,44.00431483)(703.1968042,44.05431885)
\lineto(703.1968042,44.27931885)
\curveto(703.17679756,44.36931447)(703.16179757,44.45931438)(703.1518042,44.54931885)
\curveto(703.14179759,44.64931419)(703.12179761,44.7393141)(703.0918042,44.81931885)
\curveto(703.07179766,44.89931394)(703.05179768,44.97431386)(703.0318042,45.04431885)
\curveto(703.02179771,45.11431372)(703.00179773,45.18431365)(702.9718042,45.25431885)
\curveto(702.85179788,45.55431328)(702.69679804,45.81931302)(702.5068042,46.04931885)
\curveto(702.31679842,46.27931256)(702.07679866,46.45931238)(701.7868042,46.58931885)
\curveto(701.68679905,46.6393122)(701.58179915,46.67431216)(701.4718042,46.69431885)
\curveto(701.37179936,46.72431211)(701.26179947,46.74931209)(701.1418042,46.76931885)
\curveto(701.06179967,46.78931205)(700.97179976,46.79931204)(700.8718042,46.79931885)
\lineto(700.6018042,46.79931885)
\curveto(700.55180018,46.78931205)(700.50680023,46.77931206)(700.4668042,46.76931885)
\lineto(700.3318042,46.76931885)
\curveto(700.25180048,46.74931209)(700.16680057,46.72931211)(700.0768042,46.70931885)
\curveto(699.99680074,46.68931215)(699.91680082,46.66431217)(699.8368042,46.63431885)
\curveto(699.51680122,46.49431234)(699.25680148,46.28931255)(699.0568042,46.01931885)
\curveto(698.86680187,45.75931308)(698.71180202,45.45431338)(698.5918042,45.10431885)
\curveto(698.55180218,44.99431384)(698.52180221,44.87931396)(698.5018042,44.75931885)
\curveto(698.49180224,44.64931419)(698.47680226,44.5393143)(698.4568042,44.42931885)
\curveto(698.45680228,44.38931445)(698.45180228,44.34931449)(698.4418042,44.30931885)
\lineto(698.4418042,44.20431885)
\curveto(698.42180231,44.15431468)(698.41180232,44.09931474)(698.4118042,44.03931885)
\curveto(698.42180231,43.97931486)(698.42680231,43.92431491)(698.4268042,43.87431885)
\lineto(698.4268042,43.54431885)
\curveto(698.42680231,43.44431539)(698.4368023,43.34931549)(698.4568042,43.25931885)
\curveto(698.46680227,43.22931561)(698.47180226,43.17931566)(698.4718042,43.10931885)
\curveto(698.49180224,43.0393158)(698.50680223,42.96931587)(698.5168042,42.89931885)
\lineto(698.5768042,42.68931885)
\curveto(698.68680205,42.3393165)(698.8368019,42.0393168)(699.0268042,41.78931885)
\curveto(699.21680152,41.5393173)(699.45680128,41.3343175)(699.7468042,41.17431885)
\curveto(699.8368009,41.12431771)(699.92680081,41.08431775)(700.0168042,41.05431885)
\curveto(700.10680063,41.02431781)(700.20680053,40.99431784)(700.3168042,40.96431885)
\curveto(700.36680037,40.94431789)(700.41680032,40.9393179)(700.4668042,40.94931885)
\curveto(700.52680021,40.95931788)(700.58180015,40.95431788)(700.6318042,40.93431885)
\curveto(700.67180006,40.92431791)(700.71180002,40.91931792)(700.7518042,40.91931885)
\lineto(700.8868042,40.91931885)
\lineto(701.0218042,40.91931885)
\curveto(701.05179968,40.92931791)(701.10179963,40.9343179)(701.1718042,40.93431885)
\curveto(701.25179948,40.95431788)(701.3317994,40.96931787)(701.4118042,40.97931885)
\curveto(701.49179924,40.99931784)(701.56679917,41.02431781)(701.6368042,41.05431885)
\curveto(701.96679877,41.19431764)(702.2317985,41.36931747)(702.4318042,41.57931885)
\curveto(702.64179809,41.79931704)(702.81679792,42.07431676)(702.9568042,42.40431885)
\curveto(703.00679773,42.51431632)(703.04179769,42.62431621)(703.0618042,42.73431885)
\curveto(703.08179765,42.84431599)(703.10679763,42.95431588)(703.1368042,43.06431885)
\curveto(703.15679758,43.10431573)(703.16679757,43.1393157)(703.1668042,43.16931885)
\curveto(703.16679757,43.20931563)(703.17179756,43.24931559)(703.1818042,43.28931885)
\curveto(703.19179754,43.34931549)(703.19179754,43.40931543)(703.1818042,43.46931885)
\curveto(703.18179755,43.52931531)(703.18679755,43.58931525)(703.1968042,43.64931885)
}
}
{
\newrgbcolor{curcolor}{0 0 0}
\pscustom[linestyle=none,fillstyle=solid,fillcolor=curcolor]
{
\newpath
\moveto(713.4830542,44.20431885)
\curveto(713.50304614,44.14431469)(713.51304613,44.04931479)(713.5130542,43.91931885)
\curveto(713.51304613,43.79931504)(713.50804613,43.71431512)(713.4980542,43.66431885)
\lineto(713.4980542,43.51431885)
\curveto(713.48804615,43.4343154)(713.47804616,43.35931548)(713.4680542,43.28931885)
\curveto(713.46804617,43.22931561)(713.46304618,43.15931568)(713.4530542,43.07931885)
\curveto(713.43304621,43.01931582)(713.41804622,42.95931588)(713.4080542,42.89931885)
\curveto(713.40804623,42.839316)(713.39804624,42.77931606)(713.3780542,42.71931885)
\curveto(713.3380463,42.58931625)(713.30304634,42.45931638)(713.2730542,42.32931885)
\curveto(713.2430464,42.19931664)(713.20304644,42.07931676)(713.1530542,41.96931885)
\curveto(712.9430467,41.48931735)(712.66304698,41.08431775)(712.3130542,40.75431885)
\curveto(711.96304768,40.4343184)(711.53304811,40.18931865)(711.0230542,40.01931885)
\curveto(710.91304873,39.97931886)(710.79304885,39.94931889)(710.6630542,39.92931885)
\curveto(710.5430491,39.90931893)(710.41804922,39.88931895)(710.2880542,39.86931885)
\curveto(710.22804941,39.85931898)(710.16304948,39.85431898)(710.0930542,39.85431885)
\curveto(710.03304961,39.84431899)(709.97304967,39.839319)(709.9130542,39.83931885)
\curveto(709.87304977,39.82931901)(709.81304983,39.82431901)(709.7330542,39.82431885)
\curveto(709.66304998,39.82431901)(709.61305003,39.82931901)(709.5830542,39.83931885)
\curveto(709.5430501,39.84931899)(709.50305014,39.85431898)(709.4630542,39.85431885)
\curveto(709.42305022,39.84431899)(709.38805025,39.84431899)(709.3580542,39.85431885)
\lineto(709.2680542,39.85431885)
\lineto(708.9080542,39.89931885)
\curveto(708.76805087,39.9393189)(708.63305101,39.97931886)(708.5030542,40.01931885)
\curveto(708.37305127,40.05931878)(708.24805139,40.10431873)(708.1280542,40.15431885)
\curveto(707.67805196,40.35431848)(707.30805233,40.61431822)(707.0180542,40.93431885)
\curveto(706.72805291,41.25431758)(706.48805315,41.64431719)(706.2980542,42.10431885)
\curveto(706.24805339,42.20431663)(706.20805343,42.30431653)(706.1780542,42.40431885)
\curveto(706.15805348,42.50431633)(706.1380535,42.60931623)(706.1180542,42.71931885)
\curveto(706.09805354,42.75931608)(706.08805355,42.78931605)(706.0880542,42.80931885)
\curveto(706.09805354,42.839316)(706.09805354,42.87431596)(706.0880542,42.91431885)
\curveto(706.06805357,42.99431584)(706.05305359,43.07431576)(706.0430542,43.15431885)
\curveto(706.0430536,43.24431559)(706.03305361,43.32931551)(706.0130542,43.40931885)
\lineto(706.0130542,43.52931885)
\curveto(706.01305363,43.56931527)(706.00805363,43.61431522)(705.9980542,43.66431885)
\curveto(705.98805365,43.71431512)(705.98305366,43.79931504)(705.9830542,43.91931885)
\curveto(705.98305366,44.04931479)(705.99305365,44.14431469)(706.0130542,44.20431885)
\curveto(706.03305361,44.27431456)(706.0380536,44.34431449)(706.0280542,44.41431885)
\curveto(706.01805362,44.48431435)(706.02305362,44.55431428)(706.0430542,44.62431885)
\curveto(706.05305359,44.67431416)(706.05805358,44.71431412)(706.0580542,44.74431885)
\curveto(706.06805357,44.78431405)(706.07805356,44.82931401)(706.0880542,44.87931885)
\curveto(706.11805352,44.99931384)(706.1430535,45.11931372)(706.1630542,45.23931885)
\curveto(706.19305345,45.35931348)(706.23305341,45.47431336)(706.2830542,45.58431885)
\curveto(706.43305321,45.95431288)(706.61305303,46.28431255)(706.8230542,46.57431885)
\curveto(707.0430526,46.87431196)(707.30805233,47.12431171)(707.6180542,47.32431885)
\curveto(707.7380519,47.40431143)(707.86305178,47.46931137)(707.9930542,47.51931885)
\curveto(708.12305152,47.57931126)(708.25805138,47.6393112)(708.3980542,47.69931885)
\curveto(708.51805112,47.74931109)(708.64805099,47.77931106)(708.7880542,47.78931885)
\curveto(708.92805071,47.80931103)(709.06805057,47.839311)(709.2080542,47.87931885)
\lineto(709.4030542,47.87931885)
\curveto(709.47305017,47.88931095)(709.5380501,47.89931094)(709.5980542,47.90931885)
\curveto(710.48804915,47.91931092)(711.22804841,47.7343111)(711.8180542,47.35431885)
\curveto(712.40804723,46.97431186)(712.83304681,46.47931236)(713.0930542,45.86931885)
\curveto(713.1430465,45.76931307)(713.18304646,45.66931317)(713.2130542,45.56931885)
\curveto(713.2430464,45.46931337)(713.27804636,45.36431347)(713.3180542,45.25431885)
\curveto(713.34804629,45.14431369)(713.37304627,45.02431381)(713.3930542,44.89431885)
\curveto(713.41304623,44.77431406)(713.4380462,44.64931419)(713.4680542,44.51931885)
\curveto(713.47804616,44.46931437)(713.47804616,44.41431442)(713.4680542,44.35431885)
\curveto(713.46804617,44.30431453)(713.47304617,44.25431458)(713.4830542,44.20431885)
\moveto(712.1480542,43.34931885)
\curveto(712.16804747,43.41931542)(712.17304747,43.49931534)(712.1630542,43.58931885)
\lineto(712.1630542,43.84431885)
\curveto(712.16304748,44.2343146)(712.12804751,44.56431427)(712.0580542,44.83431885)
\curveto(712.02804761,44.91431392)(712.00304764,44.99431384)(711.9830542,45.07431885)
\curveto(711.96304768,45.15431368)(711.9380477,45.22931361)(711.9080542,45.29931885)
\curveto(711.62804801,45.94931289)(711.18304846,46.39931244)(710.5730542,46.64931885)
\curveto(710.50304914,46.67931216)(710.42804921,46.69931214)(710.3480542,46.70931885)
\lineto(710.1080542,46.76931885)
\curveto(710.02804961,46.78931205)(709.9430497,46.79931204)(709.8530542,46.79931885)
\lineto(709.5830542,46.79931885)
\lineto(709.3130542,46.75431885)
\curveto(709.21305043,46.7343121)(709.11805052,46.70931213)(709.0280542,46.67931885)
\curveto(708.94805069,46.65931218)(708.86805077,46.62931221)(708.7880542,46.58931885)
\curveto(708.71805092,46.56931227)(708.65305099,46.5393123)(708.5930542,46.49931885)
\curveto(708.53305111,46.45931238)(708.47805116,46.41931242)(708.4280542,46.37931885)
\curveto(708.18805145,46.20931263)(707.99305165,46.00431283)(707.8430542,45.76431885)
\curveto(707.69305195,45.52431331)(707.56305208,45.24431359)(707.4530542,44.92431885)
\curveto(707.42305222,44.82431401)(707.40305224,44.71931412)(707.3930542,44.60931885)
\curveto(707.38305226,44.50931433)(707.36805227,44.40431443)(707.3480542,44.29431885)
\curveto(707.3380523,44.25431458)(707.33305231,44.18931465)(707.3330542,44.09931885)
\curveto(707.32305232,44.06931477)(707.31805232,44.0343148)(707.3180542,43.99431885)
\curveto(707.32805231,43.95431488)(707.33305231,43.90931493)(707.3330542,43.85931885)
\lineto(707.3330542,43.55931885)
\curveto(707.33305231,43.45931538)(707.3430523,43.36931547)(707.3630542,43.28931885)
\lineto(707.3930542,43.10931885)
\curveto(707.41305223,43.00931583)(707.42805221,42.90931593)(707.4380542,42.80931885)
\curveto(707.45805218,42.71931612)(707.48805215,42.6343162)(707.5280542,42.55431885)
\curveto(707.62805201,42.31431652)(707.7430519,42.08931675)(707.8730542,41.87931885)
\curveto(708.01305163,41.66931717)(708.18305146,41.49431734)(708.3830542,41.35431885)
\curveto(708.43305121,41.32431751)(708.47805116,41.29931754)(708.5180542,41.27931885)
\curveto(708.55805108,41.25931758)(708.60305104,41.2343176)(708.6530542,41.20431885)
\curveto(708.73305091,41.15431768)(708.81805082,41.10931773)(708.9080542,41.06931885)
\curveto(709.00805063,41.0393178)(709.11305053,41.00931783)(709.2230542,40.97931885)
\curveto(709.27305037,40.95931788)(709.31805032,40.94931789)(709.3580542,40.94931885)
\curveto(709.40805023,40.95931788)(709.45805018,40.95931788)(709.5080542,40.94931885)
\curveto(709.5380501,40.9393179)(709.59805004,40.92931791)(709.6880542,40.91931885)
\curveto(709.78804985,40.90931793)(709.86304978,40.91431792)(709.9130542,40.93431885)
\curveto(709.95304969,40.94431789)(709.99304965,40.94431789)(710.0330542,40.93431885)
\curveto(710.07304957,40.9343179)(710.11304953,40.94431789)(710.1530542,40.96431885)
\curveto(710.23304941,40.98431785)(710.31304933,40.99931784)(710.3930542,41.00931885)
\curveto(710.47304917,41.02931781)(710.54804909,41.05431778)(710.6180542,41.08431885)
\curveto(710.95804868,41.22431761)(711.23304841,41.41931742)(711.4430542,41.66931885)
\curveto(711.65304799,41.91931692)(711.82804781,42.21431662)(711.9680542,42.55431885)
\curveto(712.01804762,42.67431616)(712.04804759,42.79931604)(712.0580542,42.92931885)
\curveto(712.07804756,43.06931577)(712.10804753,43.20931563)(712.1480542,43.34931885)
}
}
{
\newrgbcolor{curcolor}{0 0 0}
\pscustom[linestyle=none,fillstyle=solid,fillcolor=curcolor]
{
\newpath
\moveto(718.61633545,47.90931885)
\curveto(718.84633066,47.90931093)(718.97633053,47.84931099)(719.00633545,47.72931885)
\curveto(719.03633047,47.61931122)(719.05133045,47.45431138)(719.05133545,47.23431885)
\lineto(719.05133545,46.94931885)
\curveto(719.05133045,46.85931198)(719.02633048,46.78431205)(718.97633545,46.72431885)
\curveto(718.91633059,46.64431219)(718.83133067,46.59931224)(718.72133545,46.58931885)
\curveto(718.61133089,46.58931225)(718.501331,46.57431226)(718.39133545,46.54431885)
\curveto(718.25133125,46.51431232)(718.11633139,46.48431235)(717.98633545,46.45431885)
\curveto(717.86633164,46.42431241)(717.75133175,46.38431245)(717.64133545,46.33431885)
\curveto(717.35133215,46.20431263)(717.11633239,46.02431281)(716.93633545,45.79431885)
\curveto(716.75633275,45.57431326)(716.6013329,45.31931352)(716.47133545,45.02931885)
\curveto(716.43133307,44.91931392)(716.4013331,44.80431403)(716.38133545,44.68431885)
\curveto(716.36133314,44.57431426)(716.33633317,44.45931438)(716.30633545,44.33931885)
\curveto(716.29633321,44.28931455)(716.29133321,44.2393146)(716.29133545,44.18931885)
\curveto(716.3013332,44.1393147)(716.3013332,44.08931475)(716.29133545,44.03931885)
\curveto(716.26133324,43.91931492)(716.24633326,43.77931506)(716.24633545,43.61931885)
\curveto(716.25633325,43.46931537)(716.26133324,43.32431551)(716.26133545,43.18431885)
\lineto(716.26133545,41.33931885)
\lineto(716.26133545,40.99431885)
\curveto(716.26133324,40.87431796)(716.25633325,40.75931808)(716.24633545,40.64931885)
\curveto(716.23633327,40.5393183)(716.23133327,40.44431839)(716.23133545,40.36431885)
\curveto(716.24133326,40.28431855)(716.22133328,40.21431862)(716.17133545,40.15431885)
\curveto(716.12133338,40.08431875)(716.04133346,40.04431879)(715.93133545,40.03431885)
\curveto(715.83133367,40.02431881)(715.72133378,40.01931882)(715.60133545,40.01931885)
\lineto(715.33133545,40.01931885)
\curveto(715.28133422,40.0393188)(715.23133427,40.05431878)(715.18133545,40.06431885)
\curveto(715.14133436,40.08431875)(715.11133439,40.10931873)(715.09133545,40.13931885)
\curveto(715.04133446,40.20931863)(715.01133449,40.29431854)(715.00133545,40.39431885)
\lineto(715.00133545,40.72431885)
\lineto(715.00133545,41.87931885)
\lineto(715.00133545,46.03431885)
\lineto(715.00133545,47.06931885)
\lineto(715.00133545,47.36931885)
\curveto(715.01133449,47.46931137)(715.04133446,47.55431128)(715.09133545,47.62431885)
\curveto(715.12133438,47.66431117)(715.17133433,47.69431114)(715.24133545,47.71431885)
\curveto(715.32133418,47.7343111)(715.4063341,47.74431109)(715.49633545,47.74431885)
\curveto(715.58633392,47.75431108)(715.67633383,47.75431108)(715.76633545,47.74431885)
\curveto(715.85633365,47.7343111)(715.92633358,47.71931112)(715.97633545,47.69931885)
\curveto(716.05633345,47.66931117)(716.1063334,47.60931123)(716.12633545,47.51931885)
\curveto(716.15633335,47.4393114)(716.17133333,47.34931149)(716.17133545,47.24931885)
\lineto(716.17133545,46.94931885)
\curveto(716.17133333,46.84931199)(716.19133331,46.75931208)(716.23133545,46.67931885)
\curveto(716.24133326,46.65931218)(716.25133325,46.64431219)(716.26133545,46.63431885)
\lineto(716.30633545,46.58931885)
\curveto(716.41633309,46.58931225)(716.506333,46.6343122)(716.57633545,46.72431885)
\curveto(716.64633286,46.82431201)(716.7063328,46.90431193)(716.75633545,46.96431885)
\lineto(716.84633545,47.05431885)
\curveto(716.93633257,47.16431167)(717.06133244,47.27931156)(717.22133545,47.39931885)
\curveto(717.38133212,47.51931132)(717.53133197,47.60931123)(717.67133545,47.66931885)
\curveto(717.76133174,47.71931112)(717.85633165,47.75431108)(717.95633545,47.77431885)
\curveto(718.05633145,47.80431103)(718.16133134,47.834311)(718.27133545,47.86431885)
\curveto(718.33133117,47.87431096)(718.39133111,47.87931096)(718.45133545,47.87931885)
\curveto(718.51133099,47.88931095)(718.56633094,47.89931094)(718.61633545,47.90931885)
}
}
{
\newrgbcolor{curcolor}{0.3019608 0.3019608 0.3019608}
\pscustom[linestyle=none,fillstyle=solid,fillcolor=curcolor]
{
\newpath
\moveto(606.37554932,50.71429443)
\lineto(621.37554932,50.71429443)
\lineto(621.37554932,35.71429443)
\lineto(606.37554932,35.71429443)
\closepath
}
}
{
\newrgbcolor{curcolor}{0.80000001 0.80000001 0.80000001}
\pscustom[linestyle=none,fillstyle=solid,fillcolor=curcolor]
{
\newpath
\moveto(649.54242278,991.5884822)
\curveto(722.973553,956.99126868)(754.45466259,869.41699299)(719.85744907,795.98586277)
\curveto(685.26023556,722.55473255)(597.68595986,691.07362296)(524.25482964,725.67083648)
\curveto(450.82369942,760.26804999)(419.34258983,847.84232568)(453.93980335,921.2734559)
\curveto(462.8118844,940.10407469)(475.58267228,956.83535291)(491.40670295,970.35963032)
\lineto(586.89862621,858.62965934)
\closepath
}
}
{
\newrgbcolor{curcolor}{0.90196079 0.90196079 0.90196079}
\pscustom[linestyle=none,fillstyle=solid,fillcolor=curcolor]
{
\newpath
\moveto(586.89862909,1005.60684745)
\curveto(608.5233611,1005.60684702)(629.88168698,1000.83507184)(649.45028058,991.63185616)
\lineto(586.89862621,858.62965934)
\closepath
}
}
{
\newrgbcolor{curcolor}{0.7019608 0.7019608 0.7019608}
\pscustom[linestyle=none,fillstyle=solid,fillcolor=curcolor]
{
\newpath
\moveto(491.37827156,970.33532469)
\curveto(502.57398285,979.90886163)(515.14530058,987.74492043)(528.670793,993.58081074)
\lineto(586.89862621,858.62965934)
\closepath
}
}
{
\newrgbcolor{curcolor}{0.60000002 0.60000002 0.60000002}
\pscustom[linestyle=none,fillstyle=solid,fillcolor=curcolor]
{
\newpath
\moveto(528.33034865,993.43340813)
\curveto(535.12938729,996.38739113)(542.14216593,998.82276198)(549.30864827,1000.71867152)
\lineto(586.89862621,858.62965934)
\closepath
}
}
{
\newrgbcolor{curcolor}{0.50196081 0.50196081 0.50196081}
\pscustom[linestyle=none,fillstyle=solid,fillcolor=curcolor]
{
\newpath
\moveto(549.19724784,1000.68915354)
\curveto(551.66568549,1001.34425584)(554.1507785,1001.9349092)(556.64999542,1002.46051194)
\lineto(586.89862621,858.62965934)
\closepath
}
}
{
\newrgbcolor{curcolor}{0.40000001 0.40000001 0.40000001}
\pscustom[linestyle=none,fillstyle=solid,fillcolor=curcolor]
{
\newpath
\moveto(556.57197164,1002.44408089)
\curveto(563.72685938,1003.95285732)(570.98436589,1004.92558877)(578.28403309,1005.35417198)
\lineto(586.89862621,858.62965934)
\closepath
}
}
{
\newrgbcolor{curcolor}{0.3019608 0.3019608 0.3019608}
\pscustom[linestyle=none,fillstyle=solid,fillcolor=curcolor]
{
\newpath
\moveto(578.21859482,1005.35031528)
\curveto(581.10878139,1005.52129945)(584.0033892,1005.6068475)(586.89862909,1005.60684745)
\lineto(586.89862621,858.62965934)
\closepath
}
}
{
\newrgbcolor{curcolor}{0.80000001 0.80000001 0.80000001}
\pscustom[linestyle=none,fillstyle=solid,fillcolor=curcolor]
{
\newpath
\moveto(331.18484993,604.350054)
\curveto(383.65526853,542.41498122)(375.98265035,449.67100592)(314.04757757,397.20058732)
\curveto(252.11250479,344.73016871)(159.3685295,352.40278689)(106.89811089,414.33785967)
\curveto(75.73539457,451.12173057)(64.5930913,500.84943077)(77.08027889,547.41377542)
\lineto(219.04148041,509.34395684)
\closepath
}
}
{
\newrgbcolor{curcolor}{0.90196079 0.90196079 0.90196079}
\pscustom[linestyle=none,fillstyle=solid,fillcolor=curcolor]
{
\newpath
\moveto(219.04148329,656.32114495)
\curveto(262.23014731,656.3211441)(303.23300497,637.32575913)(331.15909916,604.38044136)
\lineto(219.04148041,509.34395684)
\closepath
}
}
{
\newrgbcolor{curcolor}{0.7019608 0.7019608 0.7019608}
\pscustom[linestyle=none,fillstyle=solid,fillcolor=curcolor]
{
\newpath
\moveto(77.08703879,547.43897395)
\curveto(84.8955478,576.53602181)(101.45691576,602.53073158)(124.52874463,621.90343711)
\lineto(219.04148041,509.34395684)
\closepath
}
}
{
\newrgbcolor{curcolor}{0.60000002 0.60000002 0.60000002}
\pscustom[linestyle=none,fillstyle=solid,fillcolor=curcolor]
{
\newpath
\moveto(124.46819006,621.85256352)
\curveto(134.31748257,630.13175235)(145.21128791,637.08177438)(156.87024645,642.52440395)
\lineto(219.04148041,509.34395684)
\closepath
}
}
{
\newrgbcolor{curcolor}{0.50196081 0.50196081 0.50196081}
\pscustom[linestyle=none,fillstyle=solid,fillcolor=curcolor]
{
\newpath
\moveto(156.75566771,642.4708563)
\curveto(160.64433618,644.29023974)(164.61085433,645.93836619)(168.6438108,647.41049414)
\lineto(219.04148041,509.34395684)
\closepath
}
}
{
\newrgbcolor{curcolor}{0.40000001 0.40000001 0.40000001}
\pscustom[linestyle=none,fillstyle=solid,fillcolor=curcolor]
{
\newpath
\moveto(168.6614285,647.41692376)
\curveto(180.5124384,651.7411192)(192.87085594,654.52156303)(205.43192988,655.68969117)
\lineto(219.04148041,509.34395684)
\closepath
}
}
{
\newrgbcolor{curcolor}{0.3019608 0.3019608 0.3019608}
\pscustom[linestyle=none,fillstyle=solid,fillcolor=curcolor]
{
\newpath
\moveto(205.41619779,655.6882273)
\curveto(209.9457175,656.10994512)(214.49237407,656.32114504)(219.04148329,656.32114495)
\lineto(219.04148041,509.34395684)
\closepath
}
}
{
\newrgbcolor{curcolor}{0.80000001 0.80000001 0.80000001}
\pscustom[linestyle=none,fillstyle=solid,fillcolor=curcolor]
{
\newpath
\moveto(731.53230326,490.82193115)
\curveto(721.30287094,410.2958063)(647.73095537,353.30907746)(567.20483052,363.53850979)
\curveto(518.72607235,369.69688605)(476.47859875,399.54765609)(454.48155674,443.18532027)
\lineto(585.72685621,509.34395684)
\closepath
}
}
{
\newrgbcolor{curcolor}{0.90196079 0.90196079 0.90196079}
\pscustom[linestyle=none,fillstyle=solid,fillcolor=curcolor]
{
\newpath
\moveto(585.72685909,656.32114495)
\curveto(666.90011866,656.32114335)(732.70404591,590.51721352)(732.70404432,509.34395395)
\curveto(732.7040442,503.08791772)(732.30461527,496.83826356)(731.50819493,490.63312827)
\lineto(585.72685621,509.34395684)
\closepath
}
}
{
\newrgbcolor{curcolor}{0.7019608 0.7019608 0.7019608}
\pscustom[linestyle=none,fillstyle=solid,fillcolor=curcolor]
{
\newpath
\moveto(454.60950719,442.93210088)
\curveto(439.3846331,472.99066651)(434.9498606,507.36770324)(442.04789108,540.30601107)
\lineto(585.72685621,509.34395684)
\closepath
}
}
{
\newrgbcolor{curcolor}{0.60000002 0.60000002 0.60000002}
\pscustom[linestyle=none,fillstyle=solid,fillcolor=curcolor]
{
\newpath
\moveto(441.9903722,540.03788112)
\curveto(447.23085712,564.57853193)(458.66730879,587.36876993)(475.21032159,606.23765839)
\lineto(585.72685621,509.34395684)
\closepath
}
}
{
\newrgbcolor{curcolor}{0.50196081 0.50196081 0.50196081}
\pscustom[linestyle=none,fillstyle=solid,fillcolor=curcolor]
{
\newpath
\moveto(474.93193892,605.91921312)
\curveto(480.43182828,612.22890166)(486.46066229,618.05723809)(492.95269005,623.34065819)
\lineto(585.72685621,509.34395684)
\closepath
}
}
{
\newrgbcolor{curcolor}{0.40000001 0.40000001 0.40000001}
\pscustom[linestyle=none,fillstyle=solid,fillcolor=curcolor]
{
\newpath
\moveto(492.87016937,623.27345053)
\curveto(512.30622873,639.1145468)(535.47176654,649.71995379)(560.16350785,654.08099095)
\lineto(585.72685621,509.34395684)
\closepath
}
}
{
\newrgbcolor{curcolor}{0.3019608 0.3019608 0.3019608}
\pscustom[linestyle=none,fillstyle=solid,fillcolor=curcolor]
{
\newpath
\moveto(560.16674996,654.08156354)
\curveto(568.6052103,655.57176327)(577.15782717,656.32114511)(585.72685909,656.32114495)
\lineto(585.72685621,509.34395684)
\closepath
}
}
{
\newrgbcolor{curcolor}{0.80000001 0.80000001 0.80000001}
\pscustom[linestyle=none,fillstyle=solid,fillcolor=curcolor]
{
\newpath
\moveto(265.64891159,303.26347006)
\curveto(342.63281249,277.52289758)(384.17373791,194.24825835)(358.43316543,117.26435745)
\curveto(332.69259296,40.28045656)(249.41795373,-1.26046886)(172.43405283,24.48010361)
\curveto(148.60024608,32.44924811)(127.20788838,46.3965023)(110.30055858,64.98950656)
\lineto(219.04148221,163.87178684)
\closepath
}
}
{
\newrgbcolor{curcolor}{0.90196079 0.90196079 0.90196079}
\pscustom[linestyle=none,fillstyle=solid,fillcolor=curcolor]
{
\newpath
\moveto(219.04148509,310.84897495)
\curveto(234.87878178,310.84897464)(250.61198138,308.28929746)(265.63250487,303.26895479)
\lineto(219.04148221,163.87178684)
\closepath
}
}
{
\newrgbcolor{curcolor}{0.7019608 0.7019608 0.7019608}
\pscustom[linestyle=none,fillstyle=solid,fillcolor=curcolor]
{
\newpath
\moveto(110.43594044,64.84083147)
\curveto(77.03681331,101.46907849)(64.35047018,152.46499957)(76.69452771,200.47285744)
\lineto(219.04148221,163.87178684)
\closepath
}
}
{
\newrgbcolor{curcolor}{0.60000002 0.60000002 0.60000002}
\pscustom[linestyle=none,fillstyle=solid,fillcolor=curcolor]
{
\newpath
\moveto(76.6502317,200.30014946)
\curveto(80.68680468,216.07831329)(87.31771886,231.07586683)(96.27193293,244.67977898)
\lineto(219.04148221,163.87178684)
\closepath
}
}
{
\newrgbcolor{curcolor}{0.50196081 0.50196081 0.50196081}
\pscustom[linestyle=none,fillstyle=solid,fillcolor=curcolor]
{
\newpath
\moveto(96.28363522,244.69755516)
\curveto(96.3517956,244.80107687)(96.4200869,244.90451232)(96.48850896,245.00786125)
\lineto(219.04148221,163.87178684)
\closepath
}
}
{
\newrgbcolor{curcolor}{0.40000001 0.40000001 0.40000001}
\pscustom[linestyle=none,fillstyle=solid,fillcolor=curcolor]
{
\newpath
\moveto(96.33628689,244.77746566)
\curveto(114.08617419,271.69774394)(140.19639639,292.03381768)(170.64334447,302.65187939)
\lineto(219.04148221,163.87178684)
\closepath
}
}
{
\newrgbcolor{curcolor}{0.3019608 0.3019608 0.3019608}
\pscustom[linestyle=none,fillstyle=solid,fillcolor=curcolor]
{
\newpath
\moveto(170.54293587,302.61682216)
\curveto(186.13185456,308.06595332)(202.52762981,310.84897527)(219.04148509,310.84897495)
\lineto(219.04148221,163.87178684)
\closepath
}
}
{
\newrgbcolor{curcolor}{0 0 0}
\pscustom[linestyle=none,fillstyle=solid,fillcolor=curcolor]
{
\newpath
\moveto(110.24504639,574.28858887)
\curveto(110.24503875,574.20858334)(110.25003875,574.12858342)(110.26004639,574.04858887)
\curveto(110.27003873,573.96858358)(110.26503873,573.89358365)(110.24504639,573.82358887)
\curveto(110.22503877,573.78358376)(110.22003878,573.73858381)(110.23004639,573.68858887)
\curveto(110.24003876,573.6485839)(110.24003876,573.60858394)(110.23004639,573.56858887)
\lineto(110.23004639,573.41858887)
\curveto(110.22003878,573.32858422)(110.21503878,573.23858431)(110.21504639,573.14858887)
\curveto(110.21503878,573.06858448)(110.21003879,572.98858456)(110.20004639,572.90858887)
\lineto(110.17004639,572.66858887)
\curveto(110.16003884,572.59858495)(110.15003885,572.52358502)(110.14004639,572.44358887)
\curveto(110.13003887,572.40358514)(110.12503887,572.36358518)(110.12504639,572.32358887)
\curveto(110.12503887,572.28358526)(110.12003888,572.23858531)(110.11004639,572.18858887)
\curveto(110.07003893,572.0485855)(110.04003896,571.90858564)(110.02004639,571.76858887)
\curveto(110.01003899,571.62858592)(109.98003902,571.49358605)(109.93004639,571.36358887)
\curveto(109.88003912,571.19358635)(109.82503917,571.02858652)(109.76504639,570.86858887)
\curveto(109.71503928,570.70858684)(109.65503934,570.55358699)(109.58504639,570.40358887)
\curveto(109.56503943,570.3435872)(109.53503946,570.28358726)(109.49504639,570.22358887)
\lineto(109.40504639,570.07358887)
\curveto(109.20503979,569.75358779)(108.99004001,569.48858806)(108.76004639,569.27858887)
\curveto(108.53004047,569.06858848)(108.23504076,568.88858866)(107.87504639,568.73858887)
\curveto(107.75504124,568.68858886)(107.62504137,568.65358889)(107.48504639,568.63358887)
\curveto(107.35504164,568.61358893)(107.22004178,568.58858896)(107.08004639,568.55858887)
\curveto(107.02004198,568.548589)(106.96004204,568.543589)(106.90004639,568.54358887)
\curveto(106.84004216,568.543589)(106.77504222,568.53858901)(106.70504639,568.52858887)
\curveto(106.67504232,568.51858903)(106.62504237,568.51858903)(106.55504639,568.52858887)
\lineto(106.40504639,568.52858887)
\lineto(106.25504639,568.52858887)
\curveto(106.17504282,568.548589)(106.09004291,568.56358898)(106.00004639,568.57358887)
\curveto(105.92004308,568.57358897)(105.84504315,568.58358896)(105.77504639,568.60358887)
\curveto(105.73504326,568.61358893)(105.7000433,568.61858893)(105.67004639,568.61858887)
\curveto(105.65004335,568.60858894)(105.62504337,568.61358893)(105.59504639,568.63358887)
\lineto(105.32504639,568.69358887)
\curveto(105.23504376,568.72358882)(105.15004385,568.75358879)(105.07004639,568.78358887)
\curveto(104.49004451,569.02358852)(104.05504494,569.39358815)(103.76504639,569.89358887)
\curveto(103.68504531,570.02358752)(103.62004538,570.15858739)(103.57004639,570.29858887)
\curveto(103.53004547,570.43858711)(103.48504551,570.58858696)(103.43504639,570.74858887)
\curveto(103.41504558,570.82858672)(103.41004559,570.90858664)(103.42004639,570.98858887)
\curveto(103.44004556,571.06858648)(103.47504552,571.12358642)(103.52504639,571.15358887)
\curveto(103.55504544,571.17358637)(103.61004539,571.18858636)(103.69004639,571.19858887)
\curveto(103.77004523,571.21858633)(103.85504514,571.22858632)(103.94504639,571.22858887)
\curveto(104.03504496,571.23858631)(104.12004488,571.23858631)(104.20004639,571.22858887)
\curveto(104.29004471,571.21858633)(104.36004464,571.20858634)(104.41004639,571.19858887)
\curveto(104.43004457,571.18858636)(104.45504454,571.17358637)(104.48504639,571.15358887)
\curveto(104.52504447,571.13358641)(104.55504444,571.11358643)(104.57504639,571.09358887)
\curveto(104.63504436,571.01358653)(104.68004432,570.91858663)(104.71004639,570.80858887)
\curveto(104.75004425,570.69858685)(104.7950442,570.59858695)(104.84504639,570.50858887)
\curveto(105.0950439,570.11858743)(105.46504353,569.8485877)(105.95504639,569.69858887)
\curveto(106.02504297,569.67858787)(106.0950429,569.66358788)(106.16504639,569.65358887)
\curveto(106.24504275,569.65358789)(106.32504267,569.6435879)(106.40504639,569.62358887)
\curveto(106.44504255,569.61358793)(106.5000425,569.60858794)(106.57004639,569.60858887)
\curveto(106.65004235,569.60858794)(106.70504229,569.61358793)(106.73504639,569.62358887)
\curveto(106.76504223,569.63358791)(106.7950422,569.63858791)(106.82504639,569.63858887)
\lineto(106.93004639,569.63858887)
\curveto(107.01004199,569.65858789)(107.08504191,569.67858787)(107.15504639,569.69858887)
\curveto(107.23504176,569.71858783)(107.31004169,569.7435878)(107.38004639,569.77358887)
\curveto(107.73004127,569.92358762)(108.000041,570.13858741)(108.19004639,570.41858887)
\curveto(108.38004062,570.69858685)(108.53504046,571.02358652)(108.65504639,571.39358887)
\curveto(108.68504031,571.47358607)(108.70504029,571.548586)(108.71504639,571.61858887)
\curveto(108.73504026,571.68858586)(108.75504024,571.76358578)(108.77504639,571.84358887)
\curveto(108.7950402,571.93358561)(108.81004019,572.02858552)(108.82004639,572.12858887)
\curveto(108.84004016,572.23858531)(108.86004014,572.3435852)(108.88004639,572.44358887)
\curveto(108.89004011,572.49358505)(108.8950401,572.543585)(108.89504639,572.59358887)
\curveto(108.90504009,572.65358489)(108.91004009,572.70858484)(108.91004639,572.75858887)
\curveto(108.93004007,572.81858473)(108.94004006,572.89358465)(108.94004639,572.98358887)
\curveto(108.94004006,573.08358446)(108.93004007,573.16358438)(108.91004639,573.22358887)
\curveto(108.88004012,573.31358423)(108.83004017,573.35358419)(108.76004639,573.34358887)
\curveto(108.7000403,573.33358421)(108.64504035,573.30358424)(108.59504639,573.25358887)
\curveto(108.51504048,573.20358434)(108.44504055,573.1435844)(108.38504639,573.07358887)
\curveto(108.33504066,573.00358454)(108.27004073,572.9435846)(108.19004639,572.89358887)
\curveto(108.03004097,572.78358476)(107.86504113,572.68358486)(107.69504639,572.59358887)
\curveto(107.52504147,572.51358503)(107.33004167,572.4435851)(107.11004639,572.38358887)
\curveto(107.01004199,572.35358519)(106.91004209,572.33858521)(106.81004639,572.33858887)
\curveto(106.72004228,572.33858521)(106.62004238,572.32858522)(106.51004639,572.30858887)
\lineto(106.36004639,572.30858887)
\curveto(106.31004269,572.32858522)(106.26004274,572.33358521)(106.21004639,572.32358887)
\curveto(106.17004283,572.31358523)(106.13004287,572.31358523)(106.09004639,572.32358887)
\curveto(106.06004294,572.33358521)(106.01504298,572.33858521)(105.95504639,572.33858887)
\curveto(105.8950431,572.3485852)(105.83004317,572.35858519)(105.76004639,572.36858887)
\lineto(105.58004639,572.39858887)
\curveto(105.13004387,572.51858503)(104.75004425,572.68358486)(104.44004639,572.89358887)
\curveto(104.17004483,573.08358446)(103.94004506,573.31358423)(103.75004639,573.58358887)
\curveto(103.57004543,573.86358368)(103.42504557,574.17858337)(103.31504639,574.52858887)
\lineto(103.25504639,574.73858887)
\curveto(103.24504575,574.81858273)(103.23004577,574.89858265)(103.21004639,574.97858887)
\curveto(103.2000458,575.00858254)(103.1950458,575.03858251)(103.19504639,575.06858887)
\curveto(103.1950458,575.09858245)(103.19004581,575.12858242)(103.18004639,575.15858887)
\curveto(103.17004583,575.21858233)(103.16504583,575.27858227)(103.16504639,575.33858887)
\curveto(103.16504583,575.40858214)(103.15504584,575.46858208)(103.13504639,575.51858887)
\lineto(103.13504639,575.69858887)
\curveto(103.12504587,575.7485818)(103.12004588,575.81858173)(103.12004639,575.90858887)
\curveto(103.12004588,575.99858155)(103.13004587,576.06858148)(103.15004639,576.11858887)
\lineto(103.15004639,576.28358887)
\curveto(103.17004583,576.36358118)(103.18004582,576.43858111)(103.18004639,576.50858887)
\curveto(103.19004581,576.57858097)(103.20504579,576.6485809)(103.22504639,576.71858887)
\curveto(103.28504571,576.91858063)(103.34504565,577.10858044)(103.40504639,577.28858887)
\curveto(103.47504552,577.46858008)(103.56504543,577.63857991)(103.67504639,577.79858887)
\curveto(103.71504528,577.86857968)(103.75504524,577.93357961)(103.79504639,577.99358887)
\lineto(103.94504639,578.17358887)
\curveto(103.96504503,578.18357936)(103.98504501,578.19857935)(104.00504639,578.21858887)
\curveto(104.0950449,578.3485792)(104.20504479,578.45857909)(104.33504639,578.54858887)
\curveto(104.5950444,578.7485788)(104.86004414,578.90357864)(105.13004639,579.01358887)
\curveto(105.21004379,579.05357849)(105.29004371,579.08357846)(105.37004639,579.10358887)
\curveto(105.46004354,579.13357841)(105.55004345,579.15857839)(105.64004639,579.17858887)
\curveto(105.74004326,579.20857834)(105.84004316,579.22857832)(105.94004639,579.23858887)
\curveto(106.04004296,579.2485783)(106.14504285,579.26357828)(106.25504639,579.28358887)
\curveto(106.28504271,579.29357825)(106.32504267,579.29357825)(106.37504639,579.28358887)
\curveto(106.43504256,579.27357827)(106.47504252,579.27857827)(106.49504639,579.29858887)
\curveto(107.21504178,579.31857823)(107.81504118,579.20357834)(108.29504639,578.95358887)
\curveto(108.77504022,578.70357884)(109.15003985,578.36357918)(109.42004639,577.93358887)
\curveto(109.51003949,577.79357975)(109.59003941,577.6485799)(109.66004639,577.49858887)
\curveto(109.73003927,577.3485802)(109.8000392,577.18858036)(109.87004639,577.01858887)
\curveto(109.92003908,576.87858067)(109.96003904,576.72858082)(109.99004639,576.56858887)
\curveto(110.02003898,576.40858114)(110.05503894,576.2485813)(110.09504639,576.08858887)
\curveto(110.11503888,576.03858151)(110.12503887,575.98358156)(110.12504639,575.92358887)
\curveto(110.12503887,575.87358167)(110.13003887,575.82358172)(110.14004639,575.77358887)
\curveto(110.16003884,575.71358183)(110.17003883,575.6485819)(110.17004639,575.57858887)
\curveto(110.17003883,575.51858203)(110.18003882,575.46358208)(110.20004639,575.41358887)
\lineto(110.20004639,575.24858887)
\curveto(110.22003878,575.19858235)(110.22503877,575.1485824)(110.21504639,575.09858887)
\curveto(110.20503879,575.0485825)(110.21003879,574.99858255)(110.23004639,574.94858887)
\curveto(110.23003877,574.92858262)(110.22503877,574.90358264)(110.21504639,574.87358887)
\curveto(110.21503878,574.8435827)(110.22003878,574.81858273)(110.23004639,574.79858887)
\curveto(110.24003876,574.76858278)(110.24003876,574.73358281)(110.23004639,574.69358887)
\curveto(110.23003877,574.65358289)(110.23503876,574.61358293)(110.24504639,574.57358887)
\curveto(110.25503874,574.53358301)(110.25503874,574.48858306)(110.24504639,574.43858887)
\lineto(110.24504639,574.28858887)
\moveto(108.74504639,575.59358887)
\curveto(108.75504024,575.6435819)(108.76004024,575.70358184)(108.76004639,575.77358887)
\curveto(108.76004024,575.8435817)(108.75504024,575.90358164)(108.74504639,575.95358887)
\curveto(108.73504026,576.00358154)(108.73004027,576.07858147)(108.73004639,576.17858887)
\curveto(108.71004029,576.25858129)(108.69004031,576.33358121)(108.67004639,576.40358887)
\curveto(108.66004034,576.47358107)(108.64504035,576.543581)(108.62504639,576.61358887)
\curveto(108.48504051,577.0435805)(108.29004071,577.37858017)(108.04004639,577.61858887)
\curveto(107.8000412,577.85857969)(107.45504154,578.03857951)(107.00504639,578.15858887)
\curveto(106.91504208,578.17857937)(106.81504218,578.18857936)(106.70504639,578.18858887)
\lineto(106.37504639,578.18858887)
\curveto(106.35504264,578.16857938)(106.32004268,578.15857939)(106.27004639,578.15858887)
\curveto(106.22004278,578.16857938)(106.17504282,578.16857938)(106.13504639,578.15858887)
\curveto(106.05504294,578.13857941)(105.98004302,578.11857943)(105.91004639,578.09858887)
\lineto(105.70004639,578.03858887)
\curveto(105.41004359,577.90857964)(105.18004382,577.72857982)(105.01004639,577.49858887)
\curveto(104.84004416,577.27858027)(104.70504429,577.01858053)(104.60504639,576.71858887)
\curveto(104.57504442,576.62858092)(104.55004445,576.53358101)(104.53004639,576.43358887)
\curveto(104.52004448,576.3435812)(104.50504449,576.2485813)(104.48504639,576.14858887)
\lineto(104.48504639,576.01358887)
\curveto(104.45504454,575.90358164)(104.44504455,575.76358178)(104.45504639,575.59358887)
\curveto(104.47504452,575.43358211)(104.4950445,575.30358224)(104.51504639,575.20358887)
\curveto(104.53504446,575.1435824)(104.55004445,575.08358246)(104.56004639,575.02358887)
\curveto(104.57004443,574.97358257)(104.58504441,574.92358262)(104.60504639,574.87358887)
\curveto(104.68504431,574.67358287)(104.78004422,574.48358306)(104.89004639,574.30358887)
\curveto(105.01004399,574.12358342)(105.15004385,573.97858357)(105.31004639,573.86858887)
\curveto(105.36004364,573.81858373)(105.41504358,573.77858377)(105.47504639,573.74858887)
\curveto(105.53504346,573.71858383)(105.5950434,573.68358386)(105.65504639,573.64358887)
\curveto(105.80504319,573.56358398)(105.99004301,573.49858405)(106.21004639,573.44858887)
\curveto(106.26004274,573.42858412)(106.3000427,573.42358412)(106.33004639,573.43358887)
\curveto(106.37004263,573.4435841)(106.41504258,573.43858411)(106.46504639,573.41858887)
\curveto(106.50504249,573.40858414)(106.56004244,573.40358414)(106.63004639,573.40358887)
\curveto(106.7000423,573.40358414)(106.76004224,573.40858414)(106.81004639,573.41858887)
\curveto(106.91004209,573.43858411)(107.00504199,573.45358409)(107.09504639,573.46358887)
\curveto(107.18504181,573.48358406)(107.27504172,573.51358403)(107.36504639,573.55358887)
\curveto(107.90504109,573.77358377)(108.3000407,574.16858338)(108.55004639,574.73858887)
\curveto(108.6000404,574.83858271)(108.63504036,574.93858261)(108.65504639,575.03858887)
\curveto(108.67504032,575.1485824)(108.7000403,575.25858229)(108.73004639,575.36858887)
\curveto(108.73004027,575.46858208)(108.73504026,575.543582)(108.74504639,575.59358887)
}
}
{
\newrgbcolor{curcolor}{0 0 0}
\pscustom[linestyle=none,fillstyle=solid,fillcolor=curcolor]
{
\newpath
\moveto(112.60965576,570.32858887)
\lineto(112.90965576,570.32858887)
\curveto(113.0196537,570.33858721)(113.1246536,570.33858721)(113.22465576,570.32858887)
\curveto(113.33465339,570.32858722)(113.43465329,570.31858723)(113.52465576,570.29858887)
\curveto(113.61465311,570.28858726)(113.68465304,570.26358728)(113.73465576,570.22358887)
\curveto(113.75465297,570.20358734)(113.76965295,570.17358737)(113.77965576,570.13358887)
\curveto(113.79965292,570.09358745)(113.8196529,570.0485875)(113.83965576,569.99858887)
\lineto(113.83965576,569.92358887)
\curveto(113.84965287,569.87358767)(113.84965287,569.81858773)(113.83965576,569.75858887)
\lineto(113.83965576,569.60858887)
\lineto(113.83965576,569.12858887)
\curveto(113.83965288,568.95858859)(113.79965292,568.83858871)(113.71965576,568.76858887)
\curveto(113.64965307,568.71858883)(113.55965316,568.69358885)(113.44965576,568.69358887)
\lineto(113.11965576,568.69358887)
\lineto(112.66965576,568.69358887)
\curveto(112.5196542,568.69358885)(112.40465432,568.72358882)(112.32465576,568.78358887)
\curveto(112.28465444,568.81358873)(112.25465447,568.86358868)(112.23465576,568.93358887)
\curveto(112.21465451,569.01358853)(112.19965452,569.09858845)(112.18965576,569.18858887)
\lineto(112.18965576,569.47358887)
\curveto(112.19965452,569.57358797)(112.20465452,569.65858789)(112.20465576,569.72858887)
\lineto(112.20465576,569.92358887)
\curveto(112.20465452,569.98358756)(112.21465451,570.03858751)(112.23465576,570.08858887)
\curveto(112.27465445,570.19858735)(112.34465438,570.26858728)(112.44465576,570.29858887)
\curveto(112.47465425,570.29858725)(112.52965419,570.30858724)(112.60965576,570.32858887)
}
}
{
\newrgbcolor{curcolor}{0 0 0}
\pscustom[linestyle=none,fillstyle=solid,fillcolor=curcolor]
{
\newpath
\moveto(116.27481201,579.10358887)
\lineto(121.07481201,579.10358887)
\lineto(122.07981201,579.10358887)
\curveto(122.21980491,579.10357844)(122.33980479,579.09357845)(122.43981201,579.07358887)
\curveto(122.54980458,579.06357848)(122.6298045,579.01857853)(122.67981201,578.93858887)
\curveto(122.69980443,578.89857865)(122.70980442,578.8485787)(122.70981201,578.78858887)
\curveto(122.71980441,578.72857882)(122.72480441,578.66357888)(122.72481201,578.59358887)
\lineto(122.72481201,578.32358887)
\curveto(122.72480441,578.23357931)(122.71480442,578.15357939)(122.69481201,578.08358887)
\curveto(122.65480448,578.00357954)(122.60980452,577.93357961)(122.55981201,577.87358887)
\lineto(122.40981201,577.69358887)
\curveto(122.37980475,577.6435799)(122.34480479,577.60357994)(122.30481201,577.57358887)
\curveto(122.26480487,577.54358)(122.22480491,577.50358004)(122.18481201,577.45358887)
\curveto(122.10480503,577.3435802)(122.01980511,577.23358031)(121.92981201,577.12358887)
\curveto(121.83980529,577.02358052)(121.75480538,576.91858063)(121.67481201,576.80858887)
\curveto(121.5348056,576.60858094)(121.39480574,576.39858115)(121.25481201,576.17858887)
\curveto(121.11480602,575.96858158)(120.97480616,575.75358179)(120.83481201,575.53358887)
\curveto(120.78480635,575.4435821)(120.7348064,575.3485822)(120.68481201,575.24858887)
\curveto(120.6348065,575.1485824)(120.57980655,575.05358249)(120.51981201,574.96358887)
\curveto(120.49980663,574.9435826)(120.48980664,574.91858263)(120.48981201,574.88858887)
\curveto(120.48980664,574.85858269)(120.47980665,574.83358271)(120.45981201,574.81358887)
\curveto(120.38980674,574.71358283)(120.32480681,574.59858295)(120.26481201,574.46858887)
\curveto(120.20480693,574.3485832)(120.14980698,574.23358331)(120.09981201,574.12358887)
\curveto(119.99980713,573.89358365)(119.90480723,573.65858389)(119.81481201,573.41858887)
\curveto(119.72480741,573.17858437)(119.62480751,572.93858461)(119.51481201,572.69858887)
\curveto(119.49480764,572.6485849)(119.47980765,572.60358494)(119.46981201,572.56358887)
\curveto(119.46980766,572.52358502)(119.45980767,572.47858507)(119.43981201,572.42858887)
\curveto(119.38980774,572.30858524)(119.34480779,572.18358536)(119.30481201,572.05358887)
\curveto(119.27480786,571.93358561)(119.23980789,571.81358573)(119.19981201,571.69358887)
\curveto(119.11980801,571.46358608)(119.05480808,571.22358632)(119.00481201,570.97358887)
\curveto(118.96480817,570.73358681)(118.91480822,570.49358705)(118.85481201,570.25358887)
\curveto(118.81480832,570.10358744)(118.78980834,569.95358759)(118.77981201,569.80358887)
\curveto(118.76980836,569.65358789)(118.74980838,569.50358804)(118.71981201,569.35358887)
\curveto(118.70980842,569.31358823)(118.70480843,569.25358829)(118.70481201,569.17358887)
\curveto(118.67480846,569.05358849)(118.64480849,568.95358859)(118.61481201,568.87358887)
\curveto(118.58480855,568.79358875)(118.51480862,568.73858881)(118.40481201,568.70858887)
\curveto(118.35480878,568.68858886)(118.29980883,568.67858887)(118.23981201,568.67858887)
\lineto(118.04481201,568.67858887)
\curveto(117.90480923,568.67858887)(117.76480937,568.68358886)(117.62481201,568.69358887)
\curveto(117.49480964,568.70358884)(117.39980973,568.7485888)(117.33981201,568.82858887)
\curveto(117.29980983,568.88858866)(117.27980985,568.97358857)(117.27981201,569.08358887)
\curveto(117.28980984,569.19358835)(117.30480983,569.28858826)(117.32481201,569.36858887)
\lineto(117.32481201,569.44358887)
\curveto(117.3348098,569.47358807)(117.33980979,569.50358804)(117.33981201,569.53358887)
\curveto(117.35980977,569.61358793)(117.36980976,569.68858786)(117.36981201,569.75858887)
\curveto(117.36980976,569.82858772)(117.37980975,569.89858765)(117.39981201,569.96858887)
\curveto(117.44980968,570.15858739)(117.48980964,570.3435872)(117.51981201,570.52358887)
\curveto(117.54980958,570.71358683)(117.58980954,570.89358665)(117.63981201,571.06358887)
\curveto(117.65980947,571.11358643)(117.66980946,571.15358639)(117.66981201,571.18358887)
\curveto(117.66980946,571.21358633)(117.67480946,571.2485863)(117.68481201,571.28858887)
\curveto(117.78480935,571.58858596)(117.87480926,571.88358566)(117.95481201,572.17358887)
\curveto(118.04480909,572.46358508)(118.14980898,572.7435848)(118.26981201,573.01358887)
\curveto(118.5298086,573.59358395)(118.79980833,574.1435834)(119.07981201,574.66358887)
\curveto(119.35980777,575.19358235)(119.66980746,575.69858185)(120.00981201,576.17858887)
\curveto(120.14980698,576.37858117)(120.29980683,576.56858098)(120.45981201,576.74858887)
\curveto(120.61980651,576.93858061)(120.76980636,577.12858042)(120.90981201,577.31858887)
\curveto(120.94980618,577.36858018)(120.98480615,577.41358013)(121.01481201,577.45358887)
\curveto(121.05480608,577.50358004)(121.08980604,577.55357999)(121.11981201,577.60358887)
\curveto(121.129806,577.62357992)(121.13980599,577.6485799)(121.14981201,577.67858887)
\curveto(121.16980596,577.70857984)(121.16980596,577.73857981)(121.14981201,577.76858887)
\curveto(121.129806,577.82857972)(121.09480604,577.86357968)(121.04481201,577.87358887)
\curveto(120.99480614,577.89357965)(120.94480619,577.91357963)(120.89481201,577.93358887)
\lineto(120.78981201,577.93358887)
\curveto(120.74980638,577.9435796)(120.69980643,577.9435796)(120.63981201,577.93358887)
\lineto(120.48981201,577.93358887)
\lineto(119.88981201,577.93358887)
\lineto(117.24981201,577.93358887)
\lineto(116.51481201,577.93358887)
\lineto(116.27481201,577.93358887)
\curveto(116.20481093,577.9435796)(116.14481099,577.95857959)(116.09481201,577.97858887)
\curveto(116.00481113,578.01857953)(115.94481119,578.07857947)(115.91481201,578.15858887)
\curveto(115.86481127,578.25857929)(115.84981128,578.40357914)(115.86981201,578.59358887)
\curveto(115.88981124,578.79357875)(115.92481121,578.92857862)(115.97481201,578.99858887)
\curveto(115.99481114,579.01857853)(116.01981111,579.03357851)(116.04981201,579.04358887)
\lineto(116.16981201,579.10358887)
\curveto(116.18981094,579.10357844)(116.20481093,579.09857845)(116.21481201,579.08858887)
\curveto(116.2348109,579.08857846)(116.25481088,579.09357845)(116.27481201,579.10358887)
}
}
{
\newrgbcolor{curcolor}{0 0 0}
\pscustom[linestyle=none,fillstyle=solid,fillcolor=curcolor]
{
\newpath
\moveto(133.96942139,577.21358887)
\curveto(133.76941109,576.92358062)(133.5594113,576.63858091)(133.33942139,576.35858887)
\curveto(133.12941173,576.07858147)(132.92441193,575.79358175)(132.72442139,575.50358887)
\curveto(132.12441273,574.65358289)(131.51941334,573.81358373)(130.90942139,572.98358887)
\curveto(130.29941456,572.16358538)(129.69441516,571.32858622)(129.09442139,570.47858887)
\lineto(128.58442139,569.75858887)
\lineto(128.07442139,569.06858887)
\curveto(127.99441686,568.95858859)(127.91441694,568.8435887)(127.83442139,568.72358887)
\curveto(127.7544171,568.60358894)(127.6594172,568.50858904)(127.54942139,568.43858887)
\curveto(127.50941735,568.41858913)(127.44441741,568.40358914)(127.35442139,568.39358887)
\curveto(127.27441758,568.37358917)(127.18441767,568.36358918)(127.08442139,568.36358887)
\curveto(126.98441787,568.36358918)(126.88941797,568.36858918)(126.79942139,568.37858887)
\curveto(126.71941814,568.38858916)(126.6594182,568.40858914)(126.61942139,568.43858887)
\curveto(126.58941827,568.45858909)(126.56441829,568.49358905)(126.54442139,568.54358887)
\curveto(126.53441832,568.58358896)(126.53941832,568.62858892)(126.55942139,568.67858887)
\curveto(126.59941826,568.75858879)(126.64441821,568.83358871)(126.69442139,568.90358887)
\curveto(126.7544181,568.98358856)(126.80941805,569.06358848)(126.85942139,569.14358887)
\curveto(127.09941776,569.48358806)(127.34441751,569.81858773)(127.59442139,570.14858887)
\curveto(127.84441701,570.47858707)(128.08441677,570.81358673)(128.31442139,571.15358887)
\curveto(128.47441638,571.37358617)(128.63441622,571.58858596)(128.79442139,571.79858887)
\curveto(128.9544159,572.00858554)(129.11441574,572.22358532)(129.27442139,572.44358887)
\curveto(129.63441522,572.96358458)(129.99941486,573.47358407)(130.36942139,573.97358887)
\curveto(130.73941412,574.47358307)(131.10941375,574.98358256)(131.47942139,575.50358887)
\curveto(131.61941324,575.70358184)(131.7594131,575.89858165)(131.89942139,576.08858887)
\curveto(132.04941281,576.27858127)(132.19441266,576.47358107)(132.33442139,576.67358887)
\curveto(132.54441231,576.97358057)(132.7594121,577.27358027)(132.97942139,577.57358887)
\lineto(133.63942139,578.47358887)
\lineto(133.81942139,578.74358887)
\lineto(134.02942139,579.01358887)
\lineto(134.14942139,579.19358887)
\curveto(134.19941066,579.25357829)(134.24941061,579.30857824)(134.29942139,579.35858887)
\curveto(134.36941049,579.40857814)(134.44441041,579.4435781)(134.52442139,579.46358887)
\curveto(134.54441031,579.47357807)(134.56941029,579.47357807)(134.59942139,579.46358887)
\curveto(134.63941022,579.46357808)(134.66941019,579.47357807)(134.68942139,579.49358887)
\curveto(134.80941005,579.49357805)(134.94440991,579.48857806)(135.09442139,579.47858887)
\curveto(135.24440961,579.47857807)(135.33440952,579.43357811)(135.36442139,579.34358887)
\curveto(135.38440947,579.31357823)(135.38940947,579.27857827)(135.37942139,579.23858887)
\curveto(135.36940949,579.19857835)(135.3544095,579.16857838)(135.33442139,579.14858887)
\curveto(135.29440956,579.06857848)(135.2544096,578.99857855)(135.21442139,578.93858887)
\curveto(135.17440968,578.87857867)(135.12940973,578.81857873)(135.07942139,578.75858887)
\lineto(134.50942139,577.97858887)
\curveto(134.32941053,577.72857982)(134.14941071,577.47358007)(133.96942139,577.21358887)
\moveto(127.11442139,573.31358887)
\curveto(127.06441779,573.33358421)(127.01441784,573.33858421)(126.96442139,573.32858887)
\curveto(126.91441794,573.31858423)(126.86441799,573.32358422)(126.81442139,573.34358887)
\curveto(126.70441815,573.36358418)(126.59941826,573.38358416)(126.49942139,573.40358887)
\curveto(126.40941845,573.43358411)(126.31441854,573.47358407)(126.21442139,573.52358887)
\curveto(125.88441897,573.66358388)(125.62941923,573.85858369)(125.44942139,574.10858887)
\curveto(125.26941959,574.36858318)(125.12441973,574.67858287)(125.01442139,575.03858887)
\curveto(124.98441987,575.11858243)(124.96441989,575.19858235)(124.95442139,575.27858887)
\curveto(124.94441991,575.36858218)(124.92941993,575.45358209)(124.90942139,575.53358887)
\curveto(124.89941996,575.58358196)(124.89441996,575.6485819)(124.89442139,575.72858887)
\curveto(124.88441997,575.75858179)(124.87941998,575.78858176)(124.87942139,575.81858887)
\curveto(124.87941998,575.85858169)(124.87441998,575.89358165)(124.86442139,575.92358887)
\lineto(124.86442139,576.07358887)
\curveto(124.85442,576.12358142)(124.84942001,576.18358136)(124.84942139,576.25358887)
\curveto(124.84942001,576.33358121)(124.85442,576.39858115)(124.86442139,576.44858887)
\lineto(124.86442139,576.61358887)
\curveto(124.88441997,576.66358088)(124.88941997,576.70858084)(124.87942139,576.74858887)
\curveto(124.87941998,576.79858075)(124.88441997,576.8435807)(124.89442139,576.88358887)
\curveto(124.90441995,576.92358062)(124.90941995,576.95858059)(124.90942139,576.98858887)
\curveto(124.90941995,577.02858052)(124.91441994,577.06858048)(124.92442139,577.10858887)
\curveto(124.9544199,577.21858033)(124.97441988,577.32858022)(124.98442139,577.43858887)
\curveto(125.00441985,577.55857999)(125.03941982,577.67357987)(125.08942139,577.78358887)
\curveto(125.22941963,578.12357942)(125.38941947,578.39857915)(125.56942139,578.60858887)
\curveto(125.7594191,578.82857872)(126.02941883,579.00857854)(126.37942139,579.14858887)
\curveto(126.4594184,579.17857837)(126.54441831,579.19857835)(126.63442139,579.20858887)
\curveto(126.72441813,579.22857832)(126.81941804,579.2485783)(126.91942139,579.26858887)
\curveto(126.94941791,579.27857827)(127.00441785,579.27857827)(127.08442139,579.26858887)
\curveto(127.16441769,579.26857828)(127.21441764,579.27857827)(127.23442139,579.29858887)
\curveto(127.79441706,579.30857824)(128.24441661,579.19857835)(128.58442139,578.96858887)
\curveto(128.93441592,578.73857881)(129.19441566,578.43357911)(129.36442139,578.05358887)
\curveto(129.40441545,577.96357958)(129.43941542,577.86857968)(129.46942139,577.76858887)
\curveto(129.49941536,577.66857988)(129.52441533,577.56857998)(129.54442139,577.46858887)
\curveto(129.56441529,577.43858011)(129.56941529,577.40858014)(129.55942139,577.37858887)
\curveto(129.5594153,577.3485802)(129.56441529,577.31858023)(129.57442139,577.28858887)
\curveto(129.60441525,577.17858037)(129.62441523,577.05358049)(129.63442139,576.91358887)
\curveto(129.64441521,576.78358076)(129.6544152,576.6485809)(129.66442139,576.50858887)
\lineto(129.66442139,576.34358887)
\curveto(129.67441518,576.28358126)(129.67441518,576.22858132)(129.66442139,576.17858887)
\curveto(129.6544152,576.12858142)(129.64941521,576.07858147)(129.64942139,576.02858887)
\lineto(129.64942139,575.89358887)
\curveto(129.63941522,575.85358169)(129.63441522,575.81358173)(129.63442139,575.77358887)
\curveto(129.64441521,575.73358181)(129.63941522,575.68858186)(129.61942139,575.63858887)
\curveto(129.59941526,575.52858202)(129.57941528,575.42358212)(129.55942139,575.32358887)
\curveto(129.54941531,575.22358232)(129.52941533,575.12358242)(129.49942139,575.02358887)
\curveto(129.36941549,574.66358288)(129.20441565,574.3485832)(129.00442139,574.07858887)
\curveto(128.80441605,573.80858374)(128.52941633,573.60358394)(128.17942139,573.46358887)
\curveto(128.09941676,573.43358411)(128.01441684,573.40858414)(127.92442139,573.38858887)
\lineto(127.65442139,573.32858887)
\curveto(127.60441725,573.31858423)(127.5594173,573.31358423)(127.51942139,573.31358887)
\curveto(127.47941738,573.32358422)(127.43941742,573.32358422)(127.39942139,573.31358887)
\curveto(127.29941756,573.29358425)(127.20441765,573.29358425)(127.11442139,573.31358887)
\moveto(126.27442139,574.70858887)
\curveto(126.31441854,574.63858291)(126.3544185,574.57358297)(126.39442139,574.51358887)
\curveto(126.43441842,574.46358308)(126.48441837,574.41358313)(126.54442139,574.36358887)
\lineto(126.69442139,574.24358887)
\curveto(126.7544181,574.21358333)(126.81941804,574.18858336)(126.88942139,574.16858887)
\curveto(126.92941793,574.1485834)(126.96441789,574.13858341)(126.99442139,574.13858887)
\curveto(127.03441782,574.1485834)(127.07441778,574.1435834)(127.11442139,574.12358887)
\curveto(127.14441771,574.12358342)(127.18441767,574.11858343)(127.23442139,574.10858887)
\curveto(127.28441757,574.10858344)(127.32441753,574.11358343)(127.35442139,574.12358887)
\lineto(127.57942139,574.16858887)
\curveto(127.82941703,574.2485833)(128.01441684,574.37358317)(128.13442139,574.54358887)
\curveto(128.21441664,574.6435829)(128.28441657,574.77358277)(128.34442139,574.93358887)
\curveto(128.42441643,575.11358243)(128.48441637,575.33858221)(128.52442139,575.60858887)
\curveto(128.56441629,575.88858166)(128.57941628,576.16858138)(128.56942139,576.44858887)
\curveto(128.5594163,576.73858081)(128.52941633,577.01358053)(128.47942139,577.27358887)
\curveto(128.42941643,577.53358001)(128.3544165,577.7435798)(128.25442139,577.90358887)
\curveto(128.13441672,578.10357944)(127.98441687,578.25357929)(127.80442139,578.35358887)
\curveto(127.72441713,578.40357914)(127.63441722,578.43357911)(127.53442139,578.44358887)
\curveto(127.43441742,578.46357908)(127.32941753,578.47357907)(127.21942139,578.47358887)
\curveto(127.19941766,578.46357908)(127.17441768,578.45857909)(127.14442139,578.45858887)
\curveto(127.12441773,578.46857908)(127.10441775,578.46857908)(127.08442139,578.45858887)
\curveto(127.03441782,578.4485791)(126.98941787,578.43857911)(126.94942139,578.42858887)
\curveto(126.90941795,578.42857912)(126.86941799,578.41857913)(126.82942139,578.39858887)
\curveto(126.64941821,578.31857923)(126.49941836,578.19857935)(126.37942139,578.03858887)
\curveto(126.26941859,577.87857967)(126.17941868,577.69857985)(126.10942139,577.49858887)
\curveto(126.04941881,577.30858024)(126.00441885,577.08358046)(125.97442139,576.82358887)
\curveto(125.9544189,576.56358098)(125.94941891,576.29858125)(125.95942139,576.02858887)
\curveto(125.96941889,575.76858178)(125.99941886,575.51858203)(126.04942139,575.27858887)
\curveto(126.10941875,575.0485825)(126.18441867,574.85858269)(126.27442139,574.70858887)
\moveto(137.07442139,571.72358887)
\curveto(137.08440777,571.67358587)(137.08940777,571.58358596)(137.08942139,571.45358887)
\curveto(137.08940777,571.32358622)(137.07940778,571.23358631)(137.05942139,571.18358887)
\curveto(137.03940782,571.13358641)(137.03440782,571.07858647)(137.04442139,571.01858887)
\curveto(137.0544078,570.96858658)(137.0544078,570.91858663)(137.04442139,570.86858887)
\curveto(137.00440785,570.72858682)(136.97440788,570.59358695)(136.95442139,570.46358887)
\curveto(136.94440791,570.33358721)(136.91440794,570.21358733)(136.86442139,570.10358887)
\curveto(136.72440813,569.75358779)(136.5594083,569.45858809)(136.36942139,569.21858887)
\curveto(136.17940868,568.98858856)(135.90940895,568.80358874)(135.55942139,568.66358887)
\curveto(135.47940938,568.63358891)(135.39440946,568.61358893)(135.30442139,568.60358887)
\curveto(135.21440964,568.58358896)(135.12940973,568.56358898)(135.04942139,568.54358887)
\curveto(134.99940986,568.53358901)(134.94940991,568.52858902)(134.89942139,568.52858887)
\curveto(134.84941001,568.52858902)(134.79941006,568.52358902)(134.74942139,568.51358887)
\curveto(134.71941014,568.50358904)(134.66941019,568.50358904)(134.59942139,568.51358887)
\curveto(134.52941033,568.51358903)(134.47941038,568.51858903)(134.44942139,568.52858887)
\curveto(134.38941047,568.548589)(134.32941053,568.55858899)(134.26942139,568.55858887)
\curveto(134.21941064,568.548589)(134.16941069,568.55358899)(134.11942139,568.57358887)
\curveto(134.02941083,568.59358895)(133.93941092,568.61858893)(133.84942139,568.64858887)
\curveto(133.76941109,568.66858888)(133.68941117,568.69858885)(133.60942139,568.73858887)
\curveto(133.28941157,568.87858867)(133.03941182,569.07358847)(132.85942139,569.32358887)
\curveto(132.67941218,569.58358796)(132.52941233,569.88858766)(132.40942139,570.23858887)
\curveto(132.38941247,570.31858723)(132.37441248,570.40358714)(132.36442139,570.49358887)
\curveto(132.3544125,570.58358696)(132.33941252,570.66858688)(132.31942139,570.74858887)
\curveto(132.30941255,570.77858677)(132.30441255,570.80858674)(132.30442139,570.83858887)
\lineto(132.30442139,570.94358887)
\curveto(132.28441257,571.02358652)(132.27441258,571.10358644)(132.27442139,571.18358887)
\lineto(132.27442139,571.31858887)
\curveto(132.2544126,571.41858613)(132.2544126,571.51858603)(132.27442139,571.61858887)
\lineto(132.27442139,571.79858887)
\curveto(132.28441257,571.8485857)(132.28941257,571.89358565)(132.28942139,571.93358887)
\curveto(132.28941257,571.98358556)(132.29441256,572.02858552)(132.30442139,572.06858887)
\curveto(132.31441254,572.10858544)(132.31941254,572.1435854)(132.31942139,572.17358887)
\curveto(132.31941254,572.21358533)(132.32441253,572.25358529)(132.33442139,572.29358887)
\lineto(132.39442139,572.62358887)
\curveto(132.41441244,572.7435848)(132.44441241,572.85358469)(132.48442139,572.95358887)
\curveto(132.62441223,573.28358426)(132.78441207,573.55858399)(132.96442139,573.77858887)
\curveto(133.1544117,574.00858354)(133.41441144,574.19358335)(133.74442139,574.33358887)
\curveto(133.82441103,574.37358317)(133.90941095,574.39858315)(133.99942139,574.40858887)
\lineto(134.29942139,574.46858887)
\lineto(134.43442139,574.46858887)
\curveto(134.48441037,574.47858307)(134.53441032,574.48358306)(134.58442139,574.48358887)
\curveto(135.1544097,574.50358304)(135.61440924,574.39858315)(135.96442139,574.16858887)
\curveto(136.32440853,573.9485836)(136.58940827,573.6485839)(136.75942139,573.26858887)
\curveto(136.80940805,573.16858438)(136.84940801,573.06858448)(136.87942139,572.96858887)
\curveto(136.90940795,572.86858468)(136.93940792,572.76358478)(136.96942139,572.65358887)
\curveto(136.97940788,572.61358493)(136.98440787,572.57858497)(136.98442139,572.54858887)
\curveto(136.98440787,572.52858502)(136.98940787,572.49858505)(136.99942139,572.45858887)
\curveto(137.01940784,572.38858516)(137.02940783,572.31358523)(137.02942139,572.23358887)
\curveto(137.02940783,572.15358539)(137.03940782,572.07358547)(137.05942139,571.99358887)
\curveto(137.0594078,571.9435856)(137.0594078,571.89858565)(137.05942139,571.85858887)
\curveto(137.0594078,571.81858573)(137.06440779,571.77358577)(137.07442139,571.72358887)
\moveto(135.96442139,571.28858887)
\curveto(135.97440888,571.33858621)(135.97940888,571.41358613)(135.97942139,571.51358887)
\curveto(135.98940887,571.61358593)(135.98440887,571.68858586)(135.96442139,571.73858887)
\curveto(135.94440891,571.79858575)(135.93940892,571.85358569)(135.94942139,571.90358887)
\curveto(135.96940889,571.96358558)(135.96940889,572.02358552)(135.94942139,572.08358887)
\curveto(135.93940892,572.11358543)(135.93440892,572.1485854)(135.93442139,572.18858887)
\curveto(135.93440892,572.22858532)(135.92940893,572.26858528)(135.91942139,572.30858887)
\curveto(135.89940896,572.38858516)(135.87940898,572.46358508)(135.85942139,572.53358887)
\curveto(135.84940901,572.61358493)(135.83440902,572.69358485)(135.81442139,572.77358887)
\curveto(135.78440907,572.83358471)(135.7594091,572.89358465)(135.73942139,572.95358887)
\curveto(135.71940914,573.01358453)(135.68940917,573.07358447)(135.64942139,573.13358887)
\curveto(135.54940931,573.30358424)(135.41940944,573.43858411)(135.25942139,573.53858887)
\curveto(135.17940968,573.58858396)(135.08440977,573.62358392)(134.97442139,573.64358887)
\curveto(134.86440999,573.66358388)(134.73941012,573.67358387)(134.59942139,573.67358887)
\curveto(134.57941028,573.66358388)(134.5544103,573.65858389)(134.52442139,573.65858887)
\curveto(134.49441036,573.66858388)(134.46441039,573.66858388)(134.43442139,573.65858887)
\lineto(134.28442139,573.59858887)
\curveto(134.23441062,573.58858396)(134.18941067,573.57358397)(134.14942139,573.55358887)
\curveto(133.9594109,573.4435841)(133.81441104,573.29858425)(133.71442139,573.11858887)
\curveto(133.62441123,572.93858461)(133.54441131,572.73358481)(133.47442139,572.50358887)
\curveto(133.43441142,572.37358517)(133.41441144,572.23858531)(133.41442139,572.09858887)
\curveto(133.41441144,571.96858558)(133.40441145,571.82358572)(133.38442139,571.66358887)
\curveto(133.37441148,571.61358593)(133.36441149,571.55358599)(133.35442139,571.48358887)
\curveto(133.3544115,571.41358613)(133.36441149,571.35358619)(133.38442139,571.30358887)
\lineto(133.38442139,571.13858887)
\lineto(133.38442139,570.95858887)
\curveto(133.39441146,570.90858664)(133.40441145,570.85358669)(133.41442139,570.79358887)
\curveto(133.42441143,570.7435868)(133.42941143,570.68858686)(133.42942139,570.62858887)
\curveto(133.43941142,570.56858698)(133.4544114,570.51358703)(133.47442139,570.46358887)
\curveto(133.52441133,570.27358727)(133.58441127,570.09858745)(133.65442139,569.93858887)
\curveto(133.72441113,569.77858777)(133.82941103,569.6485879)(133.96942139,569.54858887)
\curveto(134.09941076,569.4485881)(134.23941062,569.37858817)(134.38942139,569.33858887)
\curveto(134.41941044,569.32858822)(134.44441041,569.32358822)(134.46442139,569.32358887)
\curveto(134.49441036,569.33358821)(134.52441033,569.33358821)(134.55442139,569.32358887)
\curveto(134.57441028,569.32358822)(134.60441025,569.31858823)(134.64442139,569.30858887)
\curveto(134.68441017,569.30858824)(134.71941014,569.31358823)(134.74942139,569.32358887)
\curveto(134.78941007,569.33358821)(134.82941003,569.33858821)(134.86942139,569.33858887)
\curveto(134.90940995,569.33858821)(134.94940991,569.3485882)(134.98942139,569.36858887)
\curveto(135.22940963,569.4485881)(135.42440943,569.58358796)(135.57442139,569.77358887)
\curveto(135.69440916,569.95358759)(135.78440907,570.15858739)(135.84442139,570.38858887)
\curveto(135.86440899,570.45858709)(135.87940898,570.52858702)(135.88942139,570.59858887)
\curveto(135.89940896,570.67858687)(135.91440894,570.75858679)(135.93442139,570.83858887)
\curveto(135.93440892,570.89858665)(135.93940892,570.9435866)(135.94942139,570.97358887)
\curveto(135.94940891,570.99358655)(135.94940891,571.01858653)(135.94942139,571.04858887)
\curveto(135.94940891,571.08858646)(135.9544089,571.11858643)(135.96442139,571.13858887)
\lineto(135.96442139,571.28858887)
}
}
{
\newrgbcolor{curcolor}{0 0 0}
\pscustom[linestyle=none,fillstyle=solid,fillcolor=curcolor]
{
\newpath
\moveto(231.01932556,390.15641113)
\curveto(231.04931784,390.03640692)(231.07431781,389.89640706)(231.09432556,389.73641113)
\curveto(231.11431777,389.57640738)(231.12431776,389.41140754)(231.12432556,389.24141113)
\curveto(231.12431776,389.07140788)(231.11431777,388.90640805)(231.09432556,388.74641113)
\curveto(231.07431781,388.58640837)(231.04931784,388.44640851)(231.01932556,388.32641113)
\curveto(230.97931791,388.18640877)(230.94431794,388.06140889)(230.91432556,387.95141113)
\curveto(230.884318,387.84140911)(230.84431804,387.73140922)(230.79432556,387.62141113)
\curveto(230.52431836,386.98140997)(230.10931878,386.49641046)(229.54932556,386.16641113)
\curveto(229.46931942,386.10641085)(229.3843195,386.0564109)(229.29432556,386.01641113)
\curveto(229.20431968,385.98641097)(229.10431978,385.951411)(228.99432556,385.91141113)
\curveto(228.88432,385.86141109)(228.76432012,385.82641113)(228.63432556,385.80641113)
\curveto(228.51432037,385.77641118)(228.3843205,385.74641121)(228.24432556,385.71641113)
\curveto(228.1843207,385.69641126)(228.12432076,385.69141126)(228.06432556,385.70141113)
\curveto(228.01432087,385.71141124)(227.95432093,385.70641125)(227.88432556,385.68641113)
\curveto(227.86432102,385.67641128)(227.83932105,385.67641128)(227.80932556,385.68641113)
\curveto(227.77932111,385.68641127)(227.75432113,385.68141127)(227.73432556,385.67141113)
\lineto(227.58432556,385.67141113)
\curveto(227.51432137,385.66141129)(227.46432142,385.66141129)(227.43432556,385.67141113)
\curveto(227.39432149,385.68141127)(227.34932154,385.68641127)(227.29932556,385.68641113)
\curveto(227.25932163,385.67641128)(227.21932167,385.67641128)(227.17932556,385.68641113)
\curveto(227.0893218,385.70641125)(226.99932189,385.72141123)(226.90932556,385.73141113)
\curveto(226.81932207,385.73141122)(226.72932216,385.74141121)(226.63932556,385.76141113)
\curveto(226.54932234,385.79141116)(226.45932243,385.81641114)(226.36932556,385.83641113)
\curveto(226.27932261,385.8564111)(226.19432269,385.88641107)(226.11432556,385.92641113)
\curveto(225.87432301,386.03641092)(225.64932324,386.16641079)(225.43932556,386.31641113)
\curveto(225.22932366,386.47641048)(225.04932384,386.6564103)(224.89932556,386.85641113)
\curveto(224.77932411,387.02640993)(224.67432421,387.20140975)(224.58432556,387.38141113)
\curveto(224.49432439,387.56140939)(224.40432448,387.7514092)(224.31432556,387.95141113)
\curveto(224.27432461,388.0514089)(224.23932465,388.1514088)(224.20932556,388.25141113)
\curveto(224.1893247,388.36140859)(224.16432472,388.47140848)(224.13432556,388.58141113)
\curveto(224.09432479,388.72140823)(224.06932482,388.86140809)(224.05932556,389.00141113)
\curveto(224.04932484,389.14140781)(224.02932486,389.28140767)(223.99932556,389.42141113)
\curveto(223.9893249,389.53140742)(223.97932491,389.63140732)(223.96932556,389.72141113)
\curveto(223.96932492,389.82140713)(223.95932493,389.92140703)(223.93932556,390.02141113)
\lineto(223.93932556,390.11141113)
\curveto(223.94932494,390.14140681)(223.94932494,390.16640679)(223.93932556,390.18641113)
\lineto(223.93932556,390.39641113)
\curveto(223.91932497,390.4564065)(223.90932498,390.52140643)(223.90932556,390.59141113)
\curveto(223.91932497,390.67140628)(223.92432496,390.74640621)(223.92432556,390.81641113)
\lineto(223.92432556,390.96641113)
\curveto(223.92432496,391.01640594)(223.92932496,391.06640589)(223.93932556,391.11641113)
\lineto(223.93932556,391.49141113)
\curveto(223.94932494,391.52140543)(223.94932494,391.5564054)(223.93932556,391.59641113)
\curveto(223.93932495,391.63640532)(223.94432494,391.67640528)(223.95432556,391.71641113)
\curveto(223.97432491,391.82640513)(223.9893249,391.93640502)(223.99932556,392.04641113)
\curveto(224.00932488,392.16640479)(224.01932487,392.28140467)(224.02932556,392.39141113)
\curveto(224.06932482,392.54140441)(224.09432479,392.68640427)(224.10432556,392.82641113)
\curveto(224.12432476,392.97640398)(224.15432473,393.12140383)(224.19432556,393.26141113)
\curveto(224.2843246,393.56140339)(224.37932451,393.84640311)(224.47932556,394.11641113)
\curveto(224.57932431,394.38640257)(224.70432418,394.63640232)(224.85432556,394.86641113)
\curveto(225.05432383,395.18640177)(225.29932359,395.46640149)(225.58932556,395.70641113)
\curveto(225.87932301,395.94640101)(226.21932267,396.13140082)(226.60932556,396.26141113)
\curveto(226.71932217,396.30140065)(226.82932206,396.32640063)(226.93932556,396.33641113)
\curveto(227.05932183,396.3564006)(227.17932171,396.38140057)(227.29932556,396.41141113)
\curveto(227.36932152,396.42140053)(227.43432145,396.42640053)(227.49432556,396.42641113)
\curveto(227.55432133,396.42640053)(227.61932127,396.43140052)(227.68932556,396.44141113)
\curveto(228.3893205,396.46140049)(228.96431992,396.34640061)(229.41432556,396.09641113)
\curveto(229.86431902,395.84640111)(230.20931868,395.49640146)(230.44932556,395.04641113)
\curveto(230.55931833,394.81640214)(230.65931823,394.54140241)(230.74932556,394.22141113)
\curveto(230.76931812,394.1514028)(230.76931812,394.07640288)(230.74932556,393.99641113)
\curveto(230.73931815,393.92640303)(230.71431817,393.87640308)(230.67432556,393.84641113)
\curveto(230.64431824,393.81640314)(230.5843183,393.79140316)(230.49432556,393.77141113)
\curveto(230.40431848,393.76140319)(230.30431858,393.7514032)(230.19432556,393.74141113)
\curveto(230.09431879,393.74140321)(229.99431889,393.74640321)(229.89432556,393.75641113)
\curveto(229.80431908,393.76640319)(229.73931915,393.78640317)(229.69932556,393.81641113)
\curveto(229.5893193,393.88640307)(229.50931938,393.99640296)(229.45932556,394.14641113)
\curveto(229.41931947,394.29640266)(229.36431952,394.42640253)(229.29432556,394.53641113)
\curveto(229.10431978,394.84640211)(228.82432006,395.07640188)(228.45432556,395.22641113)
\curveto(228.3843205,395.2564017)(228.30932058,395.27640168)(228.22932556,395.28641113)
\curveto(228.15932073,395.29640166)(228.0843208,395.31140164)(228.00432556,395.33141113)
\curveto(227.95432093,395.34140161)(227.884321,395.34640161)(227.79432556,395.34641113)
\curveto(227.71432117,395.34640161)(227.64932124,395.34140161)(227.59932556,395.33141113)
\curveto(227.55932133,395.31140164)(227.52432136,395.30640165)(227.49432556,395.31641113)
\curveto(227.46432142,395.32640163)(227.42932146,395.32640163)(227.38932556,395.31641113)
\lineto(227.14932556,395.25641113)
\curveto(227.07932181,395.23640172)(227.00932188,395.21140174)(226.93932556,395.18141113)
\curveto(226.55932233,395.02140193)(226.26932262,394.81140214)(226.06932556,394.55141113)
\curveto(225.87932301,394.29140266)(225.70432318,393.97640298)(225.54432556,393.60641113)
\curveto(225.51432337,393.52640343)(225.4893234,393.44640351)(225.46932556,393.36641113)
\curveto(225.45932343,393.28640367)(225.43932345,393.20640375)(225.40932556,393.12641113)
\curveto(225.37932351,393.01640394)(225.35432353,392.90140405)(225.33432556,392.78141113)
\curveto(225.32432356,392.66140429)(225.30432358,392.54140441)(225.27432556,392.42141113)
\curveto(225.25432363,392.37140458)(225.24432364,392.32140463)(225.24432556,392.27141113)
\curveto(225.25432363,392.22140473)(225.24932364,392.17140478)(225.22932556,392.12141113)
\curveto(225.21932367,392.06140489)(225.21932367,391.98140497)(225.22932556,391.88141113)
\curveto(225.23932365,391.79140516)(225.25432363,391.73640522)(225.27432556,391.71641113)
\curveto(225.29432359,391.67640528)(225.32432356,391.6564053)(225.36432556,391.65641113)
\curveto(225.41432347,391.6564053)(225.45932343,391.66640529)(225.49932556,391.68641113)
\curveto(225.56932332,391.72640523)(225.62932326,391.77140518)(225.67932556,391.82141113)
\curveto(225.72932316,391.87140508)(225.7893231,391.92140503)(225.85932556,391.97141113)
\lineto(225.91932556,392.03141113)
\curveto(225.94932294,392.06140489)(225.97932291,392.08640487)(226.00932556,392.10641113)
\curveto(226.23932265,392.26640469)(226.51432237,392.40140455)(226.83432556,392.51141113)
\curveto(226.90432198,392.53140442)(226.97432191,392.54640441)(227.04432556,392.55641113)
\curveto(227.11432177,392.56640439)(227.1893217,392.58140437)(227.26932556,392.60141113)
\curveto(227.30932158,392.60140435)(227.34432154,392.60640435)(227.37432556,392.61641113)
\curveto(227.40432148,392.62640433)(227.43932145,392.62640433)(227.47932556,392.61641113)
\curveto(227.52932136,392.61640434)(227.56932132,392.62640433)(227.59932556,392.64641113)
\lineto(227.76432556,392.64641113)
\lineto(227.85432556,392.64641113)
\curveto(227.90432098,392.6564043)(227.94432094,392.6564043)(227.97432556,392.64641113)
\curveto(228.02432086,392.63640432)(228.07432081,392.63140432)(228.12432556,392.63141113)
\curveto(228.1843207,392.64140431)(228.23932065,392.64140431)(228.28932556,392.63141113)
\curveto(228.39932049,392.60140435)(228.50432038,392.58140437)(228.60432556,392.57141113)
\curveto(228.71432017,392.56140439)(228.81932007,392.53640442)(228.91932556,392.49641113)
\curveto(229.33931955,392.3564046)(229.6843192,392.17140478)(229.95432556,391.94141113)
\curveto(230.22431866,391.72140523)(230.46431842,391.43640552)(230.67432556,391.08641113)
\curveto(230.75431813,390.94640601)(230.81931807,390.79640616)(230.86932556,390.63641113)
\curveto(230.91931797,390.48640647)(230.96931792,390.32640663)(231.01932556,390.15641113)
\moveto(229.77432556,388.85141113)
\curveto(229.7843191,388.90140805)(229.7893191,388.94640801)(229.78932556,388.98641113)
\lineto(229.78932556,389.13641113)
\curveto(229.7893191,389.44640751)(229.74931914,389.73140722)(229.66932556,389.99141113)
\curveto(229.64931924,390.0514069)(229.62931926,390.10640685)(229.60932556,390.15641113)
\curveto(229.59931929,390.21640674)(229.5843193,390.27140668)(229.56432556,390.32141113)
\curveto(229.34431954,390.81140614)(228.99931989,391.16140579)(228.52932556,391.37141113)
\curveto(228.44932044,391.40140555)(228.36932052,391.42640553)(228.28932556,391.44641113)
\lineto(228.04932556,391.50641113)
\curveto(227.96932092,391.52640543)(227.87932101,391.53640542)(227.77932556,391.53641113)
\lineto(227.46432556,391.53641113)
\curveto(227.44432144,391.51640544)(227.40432148,391.50640545)(227.34432556,391.50641113)
\curveto(227.29432159,391.51640544)(227.24932164,391.51640544)(227.20932556,391.50641113)
\lineto(226.96932556,391.44641113)
\curveto(226.89932199,391.43640552)(226.82932206,391.41640554)(226.75932556,391.38641113)
\curveto(226.15932273,391.12640583)(225.75432313,390.66140629)(225.54432556,389.99141113)
\curveto(225.51432337,389.91140704)(225.49432339,389.83140712)(225.48432556,389.75141113)
\curveto(225.47432341,389.67140728)(225.45932343,389.58640737)(225.43932556,389.49641113)
\lineto(225.43932556,389.34641113)
\curveto(225.42932346,389.30640765)(225.42432346,389.23640772)(225.42432556,389.13641113)
\curveto(225.42432346,388.90640805)(225.44432344,388.71140824)(225.48432556,388.55141113)
\curveto(225.50432338,388.48140847)(225.51932337,388.41640854)(225.52932556,388.35641113)
\curveto(225.53932335,388.29640866)(225.55932333,388.23140872)(225.58932556,388.16141113)
\curveto(225.69932319,387.88140907)(225.84432304,387.63640932)(226.02432556,387.42641113)
\curveto(226.20432268,387.22640973)(226.43932245,387.06640989)(226.72932556,386.94641113)
\lineto(226.96932556,386.85641113)
\lineto(227.20932556,386.79641113)
\curveto(227.25932163,386.77641018)(227.29932159,386.77141018)(227.32932556,386.78141113)
\curveto(227.36932152,386.79141016)(227.41432147,386.78641017)(227.46432556,386.76641113)
\curveto(227.49432139,386.7564102)(227.54932134,386.7514102)(227.62932556,386.75141113)
\curveto(227.70932118,386.7514102)(227.76932112,386.7564102)(227.80932556,386.76641113)
\curveto(227.91932097,386.78641017)(228.02432086,386.80141015)(228.12432556,386.81141113)
\curveto(228.22432066,386.82141013)(228.31932057,386.8514101)(228.40932556,386.90141113)
\curveto(228.93931995,387.10140985)(229.32931956,387.47640948)(229.57932556,388.02641113)
\curveto(229.61931927,388.12640883)(229.64931924,388.23140872)(229.66932556,388.34141113)
\lineto(229.75932556,388.67141113)
\curveto(229.75931913,388.7514082)(229.76431912,388.81140814)(229.77432556,388.85141113)
}
}
{
\newrgbcolor{curcolor}{0 0 0}
\pscustom[linestyle=none,fillstyle=solid,fillcolor=curcolor]
{
\newpath
\moveto(233.87893494,396.24641113)
\lineto(237.47893494,396.24641113)
\lineto(238.12393494,396.24641113)
\curveto(238.20392841,396.24640071)(238.27892833,396.24140071)(238.34893494,396.23141113)
\curveto(238.41892819,396.23140072)(238.47892813,396.22140073)(238.52893494,396.20141113)
\curveto(238.59892801,396.17140078)(238.65392796,396.11140084)(238.69393494,396.02141113)
\curveto(238.7139279,395.99140096)(238.72392789,395.951401)(238.72393494,395.90141113)
\lineto(238.72393494,395.76641113)
\curveto(238.73392788,395.6564013)(238.72892788,395.5514014)(238.70893494,395.45141113)
\curveto(238.69892791,395.3514016)(238.66392795,395.28140167)(238.60393494,395.24141113)
\curveto(238.5139281,395.17140178)(238.37892823,395.13640182)(238.19893494,395.13641113)
\curveto(238.01892859,395.14640181)(237.85392876,395.1514018)(237.70393494,395.15141113)
\lineto(235.70893494,395.15141113)
\lineto(235.21393494,395.15141113)
\lineto(235.07893494,395.15141113)
\curveto(235.03893157,395.1514018)(234.99893161,395.14640181)(234.95893494,395.13641113)
\lineto(234.74893494,395.13641113)
\curveto(234.63893197,395.10640185)(234.55893205,395.06640189)(234.50893494,395.01641113)
\curveto(234.45893215,394.97640198)(234.42393219,394.92140203)(234.40393494,394.85141113)
\curveto(234.38393223,394.79140216)(234.36893224,394.72140223)(234.35893494,394.64141113)
\curveto(234.34893226,394.56140239)(234.32893228,394.47140248)(234.29893494,394.37141113)
\curveto(234.24893236,394.17140278)(234.2089324,393.96640299)(234.17893494,393.75641113)
\curveto(234.14893246,393.54640341)(234.1089325,393.34140361)(234.05893494,393.14141113)
\curveto(234.03893257,393.07140388)(234.02893258,393.00140395)(234.02893494,392.93141113)
\curveto(234.02893258,392.87140408)(234.01893259,392.80640415)(233.99893494,392.73641113)
\curveto(233.98893262,392.70640425)(233.97893263,392.66640429)(233.96893494,392.61641113)
\curveto(233.96893264,392.57640438)(233.97393264,392.53640442)(233.98393494,392.49641113)
\curveto(234.00393261,392.44640451)(234.02893258,392.40140455)(234.05893494,392.36141113)
\curveto(234.09893251,392.33140462)(234.15893245,392.32640463)(234.23893494,392.34641113)
\curveto(234.29893231,392.36640459)(234.35893225,392.39140456)(234.41893494,392.42141113)
\curveto(234.47893213,392.46140449)(234.53893207,392.49640446)(234.59893494,392.52641113)
\curveto(234.65893195,392.54640441)(234.7089319,392.56140439)(234.74893494,392.57141113)
\curveto(234.93893167,392.6514043)(235.14393147,392.70640425)(235.36393494,392.73641113)
\curveto(235.59393102,392.76640419)(235.82393079,392.77640418)(236.05393494,392.76641113)
\curveto(236.29393032,392.76640419)(236.52393009,392.74140421)(236.74393494,392.69141113)
\curveto(236.96392965,392.6514043)(237.16392945,392.59140436)(237.34393494,392.51141113)
\curveto(237.39392922,392.49140446)(237.43892917,392.47140448)(237.47893494,392.45141113)
\curveto(237.52892908,392.43140452)(237.57892903,392.40640455)(237.62893494,392.37641113)
\curveto(237.97892863,392.16640479)(238.25892835,391.93640502)(238.46893494,391.68641113)
\curveto(238.68892792,391.43640552)(238.88392773,391.11140584)(239.05393494,390.71141113)
\curveto(239.10392751,390.60140635)(239.13892747,390.49140646)(239.15893494,390.38141113)
\curveto(239.17892743,390.27140668)(239.20392741,390.1564068)(239.23393494,390.03641113)
\curveto(239.24392737,390.00640695)(239.24892736,389.96140699)(239.24893494,389.90141113)
\curveto(239.26892734,389.84140711)(239.27892733,389.77140718)(239.27893494,389.69141113)
\curveto(239.27892733,389.62140733)(239.28892732,389.5564074)(239.30893494,389.49641113)
\lineto(239.30893494,389.33141113)
\curveto(239.31892729,389.28140767)(239.32392729,389.21140774)(239.32393494,389.12141113)
\curveto(239.32392729,389.03140792)(239.3139273,388.96140799)(239.29393494,388.91141113)
\curveto(239.27392734,388.8514081)(239.26892734,388.79140816)(239.27893494,388.73141113)
\curveto(239.28892732,388.68140827)(239.28392733,388.63140832)(239.26393494,388.58141113)
\curveto(239.22392739,388.42140853)(239.18892742,388.27140868)(239.15893494,388.13141113)
\curveto(239.12892748,387.99140896)(239.08392753,387.8564091)(239.02393494,387.72641113)
\curveto(238.86392775,387.3564096)(238.64392797,387.02140993)(238.36393494,386.72141113)
\curveto(238.08392853,386.42141053)(237.76392885,386.19141076)(237.40393494,386.03141113)
\curveto(237.23392938,385.951411)(237.03392958,385.87641108)(236.80393494,385.80641113)
\curveto(236.69392992,385.76641119)(236.57893003,385.74141121)(236.45893494,385.73141113)
\curveto(236.33893027,385.72141123)(236.21893039,385.70141125)(236.09893494,385.67141113)
\curveto(236.04893056,385.6514113)(235.99393062,385.6514113)(235.93393494,385.67141113)
\curveto(235.87393074,385.68141127)(235.8139308,385.67641128)(235.75393494,385.65641113)
\curveto(235.65393096,385.63641132)(235.55393106,385.63641132)(235.45393494,385.65641113)
\lineto(235.31893494,385.65641113)
\curveto(235.26893134,385.67641128)(235.2089314,385.68641127)(235.13893494,385.68641113)
\curveto(235.07893153,385.67641128)(235.02393159,385.68141127)(234.97393494,385.70141113)
\curveto(234.93393168,385.71141124)(234.89893171,385.71641124)(234.86893494,385.71641113)
\curveto(234.83893177,385.71641124)(234.80393181,385.72141123)(234.76393494,385.73141113)
\lineto(234.49393494,385.79141113)
\curveto(234.40393221,385.81141114)(234.31893229,385.84141111)(234.23893494,385.88141113)
\curveto(233.89893271,386.02141093)(233.608933,386.17641078)(233.36893494,386.34641113)
\curveto(233.12893348,386.52641043)(232.9089337,386.7564102)(232.70893494,387.03641113)
\curveto(232.55893405,387.26640969)(232.44393417,387.50640945)(232.36393494,387.75641113)
\curveto(232.34393427,387.80640915)(232.33393428,387.8514091)(232.33393494,387.89141113)
\curveto(232.33393428,387.94140901)(232.32393429,387.99140896)(232.30393494,388.04141113)
\curveto(232.28393433,388.10140885)(232.26893434,388.18140877)(232.25893494,388.28141113)
\curveto(232.25893435,388.38140857)(232.27893433,388.4564085)(232.31893494,388.50641113)
\curveto(232.36893424,388.58640837)(232.44893416,388.63140832)(232.55893494,388.64141113)
\curveto(232.66893394,388.6514083)(232.78393383,388.6564083)(232.90393494,388.65641113)
\lineto(233.06893494,388.65641113)
\curveto(233.12893348,388.6564083)(233.18393343,388.64640831)(233.23393494,388.62641113)
\curveto(233.32393329,388.60640835)(233.39393322,388.56640839)(233.44393494,388.50641113)
\curveto(233.5139331,388.41640854)(233.55893305,388.30640865)(233.57893494,388.17641113)
\curveto(233.608933,388.0564089)(233.65393296,387.951409)(233.71393494,387.86141113)
\curveto(233.90393271,387.52140943)(234.16393245,387.2514097)(234.49393494,387.05141113)
\curveto(234.59393202,386.99140996)(234.69893191,386.94141001)(234.80893494,386.90141113)
\curveto(234.92893168,386.87141008)(235.04893156,386.83641012)(235.16893494,386.79641113)
\curveto(235.33893127,386.74641021)(235.54393107,386.72641023)(235.78393494,386.73641113)
\curveto(236.03393058,386.7564102)(236.23393038,386.79141016)(236.38393494,386.84141113)
\curveto(236.75392986,386.96140999)(237.04392957,387.12140983)(237.25393494,387.32141113)
\curveto(237.47392914,387.53140942)(237.65392896,387.81140914)(237.79393494,388.16141113)
\curveto(237.84392877,388.26140869)(237.87392874,388.36640859)(237.88393494,388.47641113)
\curveto(237.90392871,388.58640837)(237.92892868,388.70140825)(237.95893494,388.82141113)
\lineto(237.95893494,388.92641113)
\curveto(237.96892864,388.96640799)(237.97392864,389.00640795)(237.97393494,389.04641113)
\curveto(237.98392863,389.07640788)(237.98392863,389.11140784)(237.97393494,389.15141113)
\lineto(237.97393494,389.27141113)
\curveto(237.97392864,389.53140742)(237.94392867,389.77640718)(237.88393494,390.00641113)
\curveto(237.77392884,390.3564066)(237.61892899,390.6514063)(237.41893494,390.89141113)
\curveto(237.21892939,391.14140581)(236.95892965,391.33640562)(236.63893494,391.47641113)
\lineto(236.45893494,391.53641113)
\curveto(236.4089302,391.5564054)(236.34893026,391.57640538)(236.27893494,391.59641113)
\curveto(236.22893038,391.61640534)(236.16893044,391.62640533)(236.09893494,391.62641113)
\curveto(236.03893057,391.63640532)(235.97393064,391.6514053)(235.90393494,391.67141113)
\lineto(235.75393494,391.67141113)
\curveto(235.7139309,391.69140526)(235.65893095,391.70140525)(235.58893494,391.70141113)
\curveto(235.52893108,391.70140525)(235.47393114,391.69140526)(235.42393494,391.67141113)
\lineto(235.31893494,391.67141113)
\curveto(235.28893132,391.67140528)(235.25393136,391.66640529)(235.21393494,391.65641113)
\lineto(234.97393494,391.59641113)
\curveto(234.89393172,391.58640537)(234.8139318,391.56640539)(234.73393494,391.53641113)
\curveto(234.49393212,391.43640552)(234.26393235,391.30140565)(234.04393494,391.13141113)
\curveto(233.95393266,391.06140589)(233.86893274,390.98640597)(233.78893494,390.90641113)
\curveto(233.7089329,390.83640612)(233.608933,390.78140617)(233.48893494,390.74141113)
\curveto(233.39893321,390.71140624)(233.25893335,390.70140625)(233.06893494,390.71141113)
\curveto(232.88893372,390.72140623)(232.76893384,390.74640621)(232.70893494,390.78641113)
\curveto(232.65893395,390.82640613)(232.61893399,390.88640607)(232.58893494,390.96641113)
\curveto(232.56893404,391.04640591)(232.56893404,391.13140582)(232.58893494,391.22141113)
\curveto(232.61893399,391.34140561)(232.63893397,391.46140549)(232.64893494,391.58141113)
\curveto(232.66893394,391.71140524)(232.69393392,391.83640512)(232.72393494,391.95641113)
\curveto(232.74393387,391.99640496)(232.74893386,392.03140492)(232.73893494,392.06141113)
\curveto(232.73893387,392.10140485)(232.74893386,392.14640481)(232.76893494,392.19641113)
\curveto(232.78893382,392.28640467)(232.80393381,392.37640458)(232.81393494,392.46641113)
\curveto(232.82393379,392.56640439)(232.84393377,392.66140429)(232.87393494,392.75141113)
\curveto(232.88393373,392.81140414)(232.88893372,392.87140408)(232.88893494,392.93141113)
\curveto(232.89893371,392.99140396)(232.9139337,393.0514039)(232.93393494,393.11141113)
\curveto(232.98393363,393.31140364)(233.01893359,393.51640344)(233.03893494,393.72641113)
\curveto(233.06893354,393.94640301)(233.1089335,394.1564028)(233.15893494,394.35641113)
\curveto(233.18893342,394.4564025)(233.2089334,394.5564024)(233.21893494,394.65641113)
\curveto(233.22893338,394.7564022)(233.24393337,394.8564021)(233.26393494,394.95641113)
\curveto(233.27393334,394.98640197)(233.27893333,395.02640193)(233.27893494,395.07641113)
\curveto(233.3089333,395.18640177)(233.32893328,395.29140166)(233.33893494,395.39141113)
\curveto(233.35893325,395.50140145)(233.38393323,395.61140134)(233.41393494,395.72141113)
\curveto(233.43393318,395.80140115)(233.44893316,395.87140108)(233.45893494,395.93141113)
\curveto(233.46893314,396.00140095)(233.49393312,396.06140089)(233.53393494,396.11141113)
\curveto(233.55393306,396.14140081)(233.58393303,396.16140079)(233.62393494,396.17141113)
\curveto(233.66393295,396.19140076)(233.7089329,396.21140074)(233.75893494,396.23141113)
\curveto(233.81893279,396.23140072)(233.85893275,396.23640072)(233.87893494,396.24641113)
}
}
{
\newrgbcolor{curcolor}{0 0 0}
\pscustom[linestyle=none,fillstyle=solid,fillcolor=curcolor]
{
\newpath
\moveto(241.67354431,387.47141113)
\lineto(241.97354431,387.47141113)
\curveto(242.08354225,387.48140947)(242.18854215,387.48140947)(242.28854431,387.47141113)
\curveto(242.39854194,387.47140948)(242.49854184,387.46140949)(242.58854431,387.44141113)
\curveto(242.67854166,387.43140952)(242.74854159,387.40640955)(242.79854431,387.36641113)
\curveto(242.81854152,387.34640961)(242.8335415,387.31640964)(242.84354431,387.27641113)
\curveto(242.86354147,387.23640972)(242.88354145,387.19140976)(242.90354431,387.14141113)
\lineto(242.90354431,387.06641113)
\curveto(242.91354142,387.01640994)(242.91354142,386.96140999)(242.90354431,386.90141113)
\lineto(242.90354431,386.75141113)
\lineto(242.90354431,386.27141113)
\curveto(242.90354143,386.10141085)(242.86354147,385.98141097)(242.78354431,385.91141113)
\curveto(242.71354162,385.86141109)(242.62354171,385.83641112)(242.51354431,385.83641113)
\lineto(242.18354431,385.83641113)
\lineto(241.73354431,385.83641113)
\curveto(241.58354275,385.83641112)(241.46854287,385.86641109)(241.38854431,385.92641113)
\curveto(241.34854299,385.956411)(241.31854302,386.00641095)(241.29854431,386.07641113)
\curveto(241.27854306,386.1564108)(241.26354307,386.24141071)(241.25354431,386.33141113)
\lineto(241.25354431,386.61641113)
\curveto(241.26354307,386.71641024)(241.26854307,386.80141015)(241.26854431,386.87141113)
\lineto(241.26854431,387.06641113)
\curveto(241.26854307,387.12640983)(241.27854306,387.18140977)(241.29854431,387.23141113)
\curveto(241.338543,387.34140961)(241.40854293,387.41140954)(241.50854431,387.44141113)
\curveto(241.5385428,387.44140951)(241.59354274,387.4514095)(241.67354431,387.47141113)
}
}
{
\newrgbcolor{curcolor}{0 0 0}
\pscustom[linestyle=none,fillstyle=solid,fillcolor=curcolor]
{
\newpath
\moveto(247.99370056,396.44141113)
\curveto(249.62369512,396.47140048)(250.67369407,395.91640104)(251.14370056,394.77641113)
\curveto(251.2436935,394.54640241)(251.30869344,394.2564027)(251.33870056,393.90641113)
\curveto(251.37869337,393.56640339)(251.35369339,393.2564037)(251.26370056,392.97641113)
\curveto(251.17369357,392.71640424)(251.05369369,392.49140446)(250.90370056,392.30141113)
\curveto(250.88369386,392.26140469)(250.85869389,392.22640473)(250.82870056,392.19641113)
\curveto(250.79869395,392.17640478)(250.77369397,392.1514048)(250.75370056,392.12141113)
\lineto(250.66370056,392.00141113)
\curveto(250.63369411,391.97140498)(250.59869415,391.94640501)(250.55870056,391.92641113)
\curveto(250.50869424,391.87640508)(250.45369429,391.83140512)(250.39370056,391.79141113)
\curveto(250.3436944,391.7514052)(250.29869445,391.70140525)(250.25870056,391.64141113)
\curveto(250.21869453,391.60140535)(250.20369454,391.5514054)(250.21370056,391.49141113)
\curveto(250.22369452,391.44140551)(250.25369449,391.39640556)(250.30370056,391.35641113)
\curveto(250.35369439,391.31640564)(250.40869434,391.27640568)(250.46870056,391.23641113)
\curveto(250.53869421,391.20640575)(250.60369414,391.17640578)(250.66370056,391.14641113)
\curveto(250.72369402,391.11640584)(250.77369397,391.08640587)(250.81370056,391.05641113)
\curveto(251.13369361,390.83640612)(251.38869336,390.52640643)(251.57870056,390.12641113)
\curveto(251.61869313,390.03640692)(251.6486931,389.94140701)(251.66870056,389.84141113)
\curveto(251.69869305,389.7514072)(251.72369302,389.66140729)(251.74370056,389.57141113)
\curveto(251.75369299,389.52140743)(251.75869299,389.47140748)(251.75870056,389.42141113)
\curveto(251.76869298,389.38140757)(251.77869297,389.33640762)(251.78870056,389.28641113)
\curveto(251.79869295,389.23640772)(251.79869295,389.18640777)(251.78870056,389.13641113)
\curveto(251.77869297,389.08640787)(251.78369296,389.03640792)(251.80370056,388.98641113)
\curveto(251.81369293,388.93640802)(251.81869293,388.87640808)(251.81870056,388.80641113)
\curveto(251.81869293,388.73640822)(251.80869294,388.67640828)(251.78870056,388.62641113)
\lineto(251.78870056,388.40141113)
\lineto(251.72870056,388.16141113)
\curveto(251.71869303,388.09140886)(251.70369304,388.02140893)(251.68370056,387.95141113)
\curveto(251.65369309,387.86140909)(251.62369312,387.77640918)(251.59370056,387.69641113)
\curveto(251.57369317,387.61640934)(251.5436932,387.53640942)(251.50370056,387.45641113)
\curveto(251.48369326,387.39640956)(251.45369329,387.33640962)(251.41370056,387.27641113)
\curveto(251.38369336,387.22640973)(251.3486934,387.17640978)(251.30870056,387.12641113)
\curveto(251.10869364,386.81641014)(250.85869389,386.5564104)(250.55870056,386.34641113)
\curveto(250.25869449,386.14641081)(249.91369483,385.98141097)(249.52370056,385.85141113)
\curveto(249.40369534,385.81141114)(249.27369547,385.78641117)(249.13370056,385.77641113)
\curveto(249.00369574,385.7564112)(248.86869588,385.73141122)(248.72870056,385.70141113)
\curveto(248.65869609,385.69141126)(248.58869616,385.68641127)(248.51870056,385.68641113)
\curveto(248.45869629,385.68641127)(248.39369635,385.68141127)(248.32370056,385.67141113)
\curveto(248.28369646,385.66141129)(248.22369652,385.6564113)(248.14370056,385.65641113)
\curveto(248.07369667,385.6564113)(248.02369672,385.66141129)(247.99370056,385.67141113)
\curveto(247.9436968,385.68141127)(247.89869685,385.68641127)(247.85870056,385.68641113)
\lineto(247.73870056,385.68641113)
\curveto(247.63869711,385.70641125)(247.53869721,385.72141123)(247.43870056,385.73141113)
\curveto(247.33869741,385.74141121)(247.2436975,385.7564112)(247.15370056,385.77641113)
\curveto(247.0436977,385.80641115)(246.93369781,385.83141112)(246.82370056,385.85141113)
\curveto(246.72369802,385.88141107)(246.61869813,385.92141103)(246.50870056,385.97141113)
\curveto(246.13869861,386.13141082)(245.82369892,386.33141062)(245.56370056,386.57141113)
\curveto(245.30369944,386.82141013)(245.09369965,387.13140982)(244.93370056,387.50141113)
\curveto(244.89369985,387.59140936)(244.85869989,387.68640927)(244.82870056,387.78641113)
\curveto(244.79869995,387.88640907)(244.76869998,387.99140896)(244.73870056,388.10141113)
\curveto(244.71870003,388.1514088)(244.70870004,388.20140875)(244.70870056,388.25141113)
\curveto(244.70870004,388.31140864)(244.69870005,388.37140858)(244.67870056,388.43141113)
\curveto(244.65870009,388.49140846)(244.6487001,388.57140838)(244.64870056,388.67141113)
\curveto(244.6487001,388.77140818)(244.66370008,388.84640811)(244.69370056,388.89641113)
\curveto(244.70370004,388.92640803)(244.71870003,388.951408)(244.73870056,388.97141113)
\lineto(244.79870056,389.03141113)
\curveto(244.83869991,389.0514079)(244.89869985,389.06640789)(244.97870056,389.07641113)
\curveto(245.06869968,389.08640787)(245.15869959,389.09140786)(245.24870056,389.09141113)
\curveto(245.33869941,389.09140786)(245.42369932,389.08640787)(245.50370056,389.07641113)
\curveto(245.59369915,389.06640789)(245.65869909,389.0564079)(245.69870056,389.04641113)
\curveto(245.71869903,389.02640793)(245.73869901,389.01140794)(245.75870056,389.00141113)
\curveto(245.77869897,389.00140795)(245.79869895,388.99140796)(245.81870056,388.97141113)
\curveto(245.88869886,388.88140807)(245.92869882,388.76640819)(245.93870056,388.62641113)
\curveto(245.95869879,388.48640847)(245.98869876,388.36140859)(246.02870056,388.25141113)
\lineto(246.17870056,387.89141113)
\curveto(246.22869852,387.78140917)(246.29369845,387.67640928)(246.37370056,387.57641113)
\curveto(246.39369835,387.54640941)(246.41369833,387.52140943)(246.43370056,387.50141113)
\curveto(246.46369828,387.48140947)(246.48869826,387.4564095)(246.50870056,387.42641113)
\curveto(246.5486982,387.36640959)(246.58369816,387.32140963)(246.61370056,387.29141113)
\curveto(246.65369809,387.26140969)(246.68869806,387.23140972)(246.71870056,387.20141113)
\curveto(246.75869799,387.17140978)(246.80369794,387.14140981)(246.85370056,387.11141113)
\curveto(246.9436978,387.0514099)(247.03869771,387.00140995)(247.13870056,386.96141113)
\lineto(247.46870056,386.84141113)
\curveto(247.61869713,386.79141016)(247.81869693,386.76141019)(248.06870056,386.75141113)
\curveto(248.31869643,386.74141021)(248.52869622,386.76141019)(248.69870056,386.81141113)
\curveto(248.77869597,386.83141012)(248.8486959,386.84641011)(248.90870056,386.85641113)
\lineto(249.11870056,386.91641113)
\curveto(249.39869535,387.03640992)(249.63869511,387.18640977)(249.83870056,387.36641113)
\curveto(250.0486947,387.54640941)(250.21369453,387.77640918)(250.33370056,388.05641113)
\curveto(250.36369438,388.12640883)(250.38369436,388.19640876)(250.39370056,388.26641113)
\lineto(250.45370056,388.50641113)
\curveto(250.49369425,388.64640831)(250.50369424,388.80640815)(250.48370056,388.98641113)
\curveto(250.46369428,389.17640778)(250.43369431,389.32640763)(250.39370056,389.43641113)
\curveto(250.26369448,389.81640714)(250.07869467,390.10640685)(249.83870056,390.30641113)
\curveto(249.60869514,390.50640645)(249.29869545,390.66640629)(248.90870056,390.78641113)
\curveto(248.79869595,390.81640614)(248.67869607,390.83640612)(248.54870056,390.84641113)
\curveto(248.42869632,390.8564061)(248.30369644,390.86140609)(248.17370056,390.86141113)
\curveto(248.01369673,390.86140609)(247.87369687,390.86640609)(247.75370056,390.87641113)
\curveto(247.63369711,390.88640607)(247.5486972,390.94640601)(247.49870056,391.05641113)
\curveto(247.47869727,391.08640587)(247.46869728,391.12140583)(247.46870056,391.16141113)
\lineto(247.46870056,391.29641113)
\curveto(247.45869729,391.39640556)(247.45869729,391.49140546)(247.46870056,391.58141113)
\curveto(247.48869726,391.67140528)(247.52869722,391.73640522)(247.58870056,391.77641113)
\curveto(247.62869712,391.80640515)(247.66869708,391.82640513)(247.70870056,391.83641113)
\curveto(247.75869699,391.84640511)(247.81369693,391.8564051)(247.87370056,391.86641113)
\curveto(247.89369685,391.87640508)(247.91869683,391.87640508)(247.94870056,391.86641113)
\curveto(247.97869677,391.86640509)(248.00369674,391.87140508)(248.02370056,391.88141113)
\lineto(248.15870056,391.88141113)
\curveto(248.26869648,391.90140505)(248.36869638,391.91140504)(248.45870056,391.91141113)
\curveto(248.55869619,391.92140503)(248.65369609,391.94140501)(248.74370056,391.97141113)
\curveto(249.06369568,392.08140487)(249.31869543,392.22640473)(249.50870056,392.40641113)
\curveto(249.69869505,392.58640437)(249.8486949,392.83640412)(249.95870056,393.15641113)
\curveto(249.98869476,393.2564037)(250.00869474,393.38140357)(250.01870056,393.53141113)
\curveto(250.03869471,393.69140326)(250.03369471,393.83640312)(250.00370056,393.96641113)
\curveto(249.98369476,394.03640292)(249.96369478,394.10140285)(249.94370056,394.16141113)
\curveto(249.93369481,394.23140272)(249.91369483,394.29640266)(249.88370056,394.35641113)
\curveto(249.78369496,394.59640236)(249.63869511,394.78640217)(249.44870056,394.92641113)
\curveto(249.25869549,395.06640189)(249.03369571,395.17640178)(248.77370056,395.25641113)
\curveto(248.71369603,395.27640168)(248.65369609,395.28640167)(248.59370056,395.28641113)
\curveto(248.53369621,395.28640167)(248.46869628,395.29640166)(248.39870056,395.31641113)
\curveto(248.31869643,395.33640162)(248.22369652,395.34640161)(248.11370056,395.34641113)
\curveto(248.00369674,395.34640161)(247.90869684,395.33640162)(247.82870056,395.31641113)
\curveto(247.77869697,395.29640166)(247.72869702,395.28640167)(247.67870056,395.28641113)
\curveto(247.63869711,395.28640167)(247.59369715,395.27640168)(247.54370056,395.25641113)
\curveto(247.36369738,395.20640175)(247.19369755,395.13140182)(247.03370056,395.03141113)
\curveto(246.88369786,394.94140201)(246.75369799,394.82640213)(246.64370056,394.68641113)
\curveto(246.55369819,394.56640239)(246.47369827,394.43640252)(246.40370056,394.29641113)
\curveto(246.33369841,394.1564028)(246.26869848,394.00140295)(246.20870056,393.83141113)
\curveto(246.17869857,393.72140323)(246.15869859,393.60140335)(246.14870056,393.47141113)
\curveto(246.13869861,393.3514036)(246.10369864,393.2514037)(246.04370056,393.17141113)
\curveto(246.02369872,393.13140382)(245.96369878,393.09140386)(245.86370056,393.05141113)
\curveto(245.82369892,393.04140391)(245.76369898,393.03140392)(245.68370056,393.02141113)
\lineto(245.42870056,393.02141113)
\curveto(245.33869941,393.03140392)(245.25369949,393.04140391)(245.17370056,393.05141113)
\curveto(245.10369964,393.06140389)(245.05369969,393.07640388)(245.02370056,393.09641113)
\curveto(244.98369976,393.12640383)(244.9486998,393.18140377)(244.91870056,393.26141113)
\curveto(244.88869986,393.34140361)(244.88369986,393.42640353)(244.90370056,393.51641113)
\curveto(244.91369983,393.56640339)(244.91869983,393.61640334)(244.91870056,393.66641113)
\lineto(244.94870056,393.84641113)
\curveto(244.97869977,393.94640301)(245.00369974,394.04640291)(245.02370056,394.14641113)
\curveto(245.05369969,394.24640271)(245.08869966,394.33640262)(245.12870056,394.41641113)
\curveto(245.17869957,394.52640243)(245.22369952,394.63140232)(245.26370056,394.73141113)
\curveto(245.30369944,394.84140211)(245.35369939,394.94640201)(245.41370056,395.04641113)
\curveto(245.743699,395.58640137)(246.21369853,395.98140097)(246.82370056,396.23141113)
\curveto(246.9436978,396.28140067)(247.06869768,396.31640064)(247.19870056,396.33641113)
\curveto(247.33869741,396.3564006)(247.47869727,396.38140057)(247.61870056,396.41141113)
\curveto(247.67869707,396.42140053)(247.73869701,396.42640053)(247.79870056,396.42641113)
\curveto(247.86869688,396.42640053)(247.93369681,396.43140052)(247.99370056,396.44141113)
}
}
{
\newrgbcolor{curcolor}{0 0 0}
\pscustom[linestyle=none,fillstyle=solid,fillcolor=curcolor]
{
\newpath
\moveto(263.03330994,394.35641113)
\curveto(262.83329964,394.06640289)(262.62329985,393.78140317)(262.40330994,393.50141113)
\curveto(262.19330028,393.22140373)(261.98830048,392.93640402)(261.78830994,392.64641113)
\curveto(261.18830128,391.79640516)(260.58330189,390.956406)(259.97330994,390.12641113)
\curveto(259.36330311,389.30640765)(258.75830371,388.47140848)(258.15830994,387.62141113)
\lineto(257.64830994,386.90141113)
\lineto(257.13830994,386.21141113)
\curveto(257.05830541,386.10141085)(256.97830549,385.98641097)(256.89830994,385.86641113)
\curveto(256.81830565,385.74641121)(256.72330575,385.6514113)(256.61330994,385.58141113)
\curveto(256.5733059,385.56141139)(256.50830596,385.54641141)(256.41830994,385.53641113)
\curveto(256.33830613,385.51641144)(256.24830622,385.50641145)(256.14830994,385.50641113)
\curveto(256.04830642,385.50641145)(255.95330652,385.51141144)(255.86330994,385.52141113)
\curveto(255.78330669,385.53141142)(255.72330675,385.5514114)(255.68330994,385.58141113)
\curveto(255.65330682,385.60141135)(255.62830684,385.63641132)(255.60830994,385.68641113)
\curveto(255.59830687,385.72641123)(255.60330687,385.77141118)(255.62330994,385.82141113)
\curveto(255.66330681,385.90141105)(255.70830676,385.97641098)(255.75830994,386.04641113)
\curveto(255.81830665,386.12641083)(255.8733066,386.20641075)(255.92330994,386.28641113)
\curveto(256.16330631,386.62641033)(256.40830606,386.96140999)(256.65830994,387.29141113)
\curveto(256.90830556,387.62140933)(257.14830532,387.956409)(257.37830994,388.29641113)
\curveto(257.53830493,388.51640844)(257.69830477,388.73140822)(257.85830994,388.94141113)
\curveto(258.01830445,389.1514078)(258.17830429,389.36640759)(258.33830994,389.58641113)
\curveto(258.69830377,390.10640685)(259.06330341,390.61640634)(259.43330994,391.11641113)
\curveto(259.80330267,391.61640534)(260.1733023,392.12640483)(260.54330994,392.64641113)
\curveto(260.68330179,392.84640411)(260.82330165,393.04140391)(260.96330994,393.23141113)
\curveto(261.11330136,393.42140353)(261.25830121,393.61640334)(261.39830994,393.81641113)
\curveto(261.60830086,394.11640284)(261.82330065,394.41640254)(262.04330994,394.71641113)
\lineto(262.70330994,395.61641113)
\lineto(262.88330994,395.88641113)
\lineto(263.09330994,396.15641113)
\lineto(263.21330994,396.33641113)
\curveto(263.26329921,396.39640056)(263.31329916,396.4514005)(263.36330994,396.50141113)
\curveto(263.43329904,396.5514004)(263.50829896,396.58640037)(263.58830994,396.60641113)
\curveto(263.60829886,396.61640034)(263.63329884,396.61640034)(263.66330994,396.60641113)
\curveto(263.70329877,396.60640035)(263.73329874,396.61640034)(263.75330994,396.63641113)
\curveto(263.8732986,396.63640032)(264.00829846,396.63140032)(264.15830994,396.62141113)
\curveto(264.30829816,396.62140033)(264.39829807,396.57640038)(264.42830994,396.48641113)
\curveto(264.44829802,396.4564005)(264.45329802,396.42140053)(264.44330994,396.38141113)
\curveto(264.43329804,396.34140061)(264.41829805,396.31140064)(264.39830994,396.29141113)
\curveto(264.35829811,396.21140074)(264.31829815,396.14140081)(264.27830994,396.08141113)
\curveto(264.23829823,396.02140093)(264.19329828,395.96140099)(264.14330994,395.90141113)
\lineto(263.57330994,395.12141113)
\curveto(263.39329908,394.87140208)(263.21329926,394.61640234)(263.03330994,394.35641113)
\moveto(256.17830994,390.45641113)
\curveto(256.12830634,390.47640648)(256.07830639,390.48140647)(256.02830994,390.47141113)
\curveto(255.97830649,390.46140649)(255.92830654,390.46640649)(255.87830994,390.48641113)
\curveto(255.7683067,390.50640645)(255.66330681,390.52640643)(255.56330994,390.54641113)
\curveto(255.473307,390.57640638)(255.37830709,390.61640634)(255.27830994,390.66641113)
\curveto(254.94830752,390.80640615)(254.69330778,391.00140595)(254.51330994,391.25141113)
\curveto(254.33330814,391.51140544)(254.18830828,391.82140513)(254.07830994,392.18141113)
\curveto(254.04830842,392.26140469)(254.02830844,392.34140461)(254.01830994,392.42141113)
\curveto(254.00830846,392.51140444)(253.99330848,392.59640436)(253.97330994,392.67641113)
\curveto(253.96330851,392.72640423)(253.95830851,392.79140416)(253.95830994,392.87141113)
\curveto(253.94830852,392.90140405)(253.94330853,392.93140402)(253.94330994,392.96141113)
\curveto(253.94330853,393.00140395)(253.93830853,393.03640392)(253.92830994,393.06641113)
\lineto(253.92830994,393.21641113)
\curveto(253.91830855,393.26640369)(253.91330856,393.32640363)(253.91330994,393.39641113)
\curveto(253.91330856,393.47640348)(253.91830855,393.54140341)(253.92830994,393.59141113)
\lineto(253.92830994,393.75641113)
\curveto(253.94830852,393.80640315)(253.95330852,393.8514031)(253.94330994,393.89141113)
\curveto(253.94330853,393.94140301)(253.94830852,393.98640297)(253.95830994,394.02641113)
\curveto(253.9683085,394.06640289)(253.9733085,394.10140285)(253.97330994,394.13141113)
\curveto(253.9733085,394.17140278)(253.97830849,394.21140274)(253.98830994,394.25141113)
\curveto(254.01830845,394.36140259)(254.03830843,394.47140248)(254.04830994,394.58141113)
\curveto(254.0683084,394.70140225)(254.10330837,394.81640214)(254.15330994,394.92641113)
\curveto(254.29330818,395.26640169)(254.45330802,395.54140141)(254.63330994,395.75141113)
\curveto(254.82330765,395.97140098)(255.09330738,396.1514008)(255.44330994,396.29141113)
\curveto(255.52330695,396.32140063)(255.60830686,396.34140061)(255.69830994,396.35141113)
\curveto(255.78830668,396.37140058)(255.88330659,396.39140056)(255.98330994,396.41141113)
\curveto(256.01330646,396.42140053)(256.0683064,396.42140053)(256.14830994,396.41141113)
\curveto(256.22830624,396.41140054)(256.27830619,396.42140053)(256.29830994,396.44141113)
\curveto(256.85830561,396.4514005)(257.30830516,396.34140061)(257.64830994,396.11141113)
\curveto(257.99830447,395.88140107)(258.25830421,395.57640138)(258.42830994,395.19641113)
\curveto(258.468304,395.10640185)(258.50330397,395.01140194)(258.53330994,394.91141113)
\curveto(258.56330391,394.81140214)(258.58830388,394.71140224)(258.60830994,394.61141113)
\curveto(258.62830384,394.58140237)(258.63330384,394.5514024)(258.62330994,394.52141113)
\curveto(258.62330385,394.49140246)(258.62830384,394.46140249)(258.63830994,394.43141113)
\curveto(258.6683038,394.32140263)(258.68830378,394.19640276)(258.69830994,394.05641113)
\curveto(258.70830376,393.92640303)(258.71830375,393.79140316)(258.72830994,393.65141113)
\lineto(258.72830994,393.48641113)
\curveto(258.73830373,393.42640353)(258.73830373,393.37140358)(258.72830994,393.32141113)
\curveto(258.71830375,393.27140368)(258.71330376,393.22140373)(258.71330994,393.17141113)
\lineto(258.71330994,393.03641113)
\curveto(258.70330377,392.99640396)(258.69830377,392.956404)(258.69830994,392.91641113)
\curveto(258.70830376,392.87640408)(258.70330377,392.83140412)(258.68330994,392.78141113)
\curveto(258.66330381,392.67140428)(258.64330383,392.56640439)(258.62330994,392.46641113)
\curveto(258.61330386,392.36640459)(258.59330388,392.26640469)(258.56330994,392.16641113)
\curveto(258.43330404,391.80640515)(258.2683042,391.49140546)(258.06830994,391.22141113)
\curveto(257.8683046,390.951406)(257.59330488,390.74640621)(257.24330994,390.60641113)
\curveto(257.16330531,390.57640638)(257.07830539,390.5514064)(256.98830994,390.53141113)
\lineto(256.71830994,390.47141113)
\curveto(256.6683058,390.46140649)(256.62330585,390.4564065)(256.58330994,390.45641113)
\curveto(256.54330593,390.46640649)(256.50330597,390.46640649)(256.46330994,390.45641113)
\curveto(256.36330611,390.43640652)(256.2683062,390.43640652)(256.17830994,390.45641113)
\moveto(255.33830994,391.85141113)
\curveto(255.37830709,391.78140517)(255.41830705,391.71640524)(255.45830994,391.65641113)
\curveto(255.49830697,391.60640535)(255.54830692,391.5564054)(255.60830994,391.50641113)
\lineto(255.75830994,391.38641113)
\curveto(255.81830665,391.3564056)(255.88330659,391.33140562)(255.95330994,391.31141113)
\curveto(255.99330648,391.29140566)(256.02830644,391.28140567)(256.05830994,391.28141113)
\curveto(256.09830637,391.29140566)(256.13830633,391.28640567)(256.17830994,391.26641113)
\curveto(256.20830626,391.26640569)(256.24830622,391.26140569)(256.29830994,391.25141113)
\curveto(256.34830612,391.2514057)(256.38830608,391.2564057)(256.41830994,391.26641113)
\lineto(256.64330994,391.31141113)
\curveto(256.89330558,391.39140556)(257.07830539,391.51640544)(257.19830994,391.68641113)
\curveto(257.27830519,391.78640517)(257.34830512,391.91640504)(257.40830994,392.07641113)
\curveto(257.48830498,392.2564047)(257.54830492,392.48140447)(257.58830994,392.75141113)
\curveto(257.62830484,393.03140392)(257.64330483,393.31140364)(257.63330994,393.59141113)
\curveto(257.62330485,393.88140307)(257.59330488,394.1564028)(257.54330994,394.41641113)
\curveto(257.49330498,394.67640228)(257.41830505,394.88640207)(257.31830994,395.04641113)
\curveto(257.19830527,395.24640171)(257.04830542,395.39640156)(256.86830994,395.49641113)
\curveto(256.78830568,395.54640141)(256.69830577,395.57640138)(256.59830994,395.58641113)
\curveto(256.49830597,395.60640135)(256.39330608,395.61640134)(256.28330994,395.61641113)
\curveto(256.26330621,395.60640135)(256.23830623,395.60140135)(256.20830994,395.60141113)
\curveto(256.18830628,395.61140134)(256.1683063,395.61140134)(256.14830994,395.60141113)
\curveto(256.09830637,395.59140136)(256.05330642,395.58140137)(256.01330994,395.57141113)
\curveto(255.9733065,395.57140138)(255.93330654,395.56140139)(255.89330994,395.54141113)
\curveto(255.71330676,395.46140149)(255.56330691,395.34140161)(255.44330994,395.18141113)
\curveto(255.33330714,395.02140193)(255.24330723,394.84140211)(255.17330994,394.64141113)
\curveto(255.11330736,394.4514025)(255.0683074,394.22640273)(255.03830994,393.96641113)
\curveto(255.01830745,393.70640325)(255.01330746,393.44140351)(255.02330994,393.17141113)
\curveto(255.03330744,392.91140404)(255.06330741,392.66140429)(255.11330994,392.42141113)
\curveto(255.1733073,392.19140476)(255.24830722,392.00140495)(255.33830994,391.85141113)
\moveto(266.13830994,388.86641113)
\curveto(266.14829632,388.81640814)(266.15329632,388.72640823)(266.15330994,388.59641113)
\curveto(266.15329632,388.46640849)(266.14329633,388.37640858)(266.12330994,388.32641113)
\curveto(266.10329637,388.27640868)(266.09829637,388.22140873)(266.10830994,388.16141113)
\curveto(266.11829635,388.11140884)(266.11829635,388.06140889)(266.10830994,388.01141113)
\curveto(266.0682964,387.87140908)(266.03829643,387.73640922)(266.01830994,387.60641113)
\curveto(266.00829646,387.47640948)(265.97829649,387.3564096)(265.92830994,387.24641113)
\curveto(265.78829668,386.89641006)(265.62329685,386.60141035)(265.43330994,386.36141113)
\curveto(265.24329723,386.13141082)(264.9732975,385.94641101)(264.62330994,385.80641113)
\curveto(264.54329793,385.77641118)(264.45829801,385.7564112)(264.36830994,385.74641113)
\curveto(264.27829819,385.72641123)(264.19329828,385.70641125)(264.11330994,385.68641113)
\curveto(264.06329841,385.67641128)(264.01329846,385.67141128)(263.96330994,385.67141113)
\curveto(263.91329856,385.67141128)(263.86329861,385.66641129)(263.81330994,385.65641113)
\curveto(263.78329869,385.64641131)(263.73329874,385.64641131)(263.66330994,385.65641113)
\curveto(263.59329888,385.6564113)(263.54329893,385.66141129)(263.51330994,385.67141113)
\curveto(263.45329902,385.69141126)(263.39329908,385.70141125)(263.33330994,385.70141113)
\curveto(263.28329919,385.69141126)(263.23329924,385.69641126)(263.18330994,385.71641113)
\curveto(263.09329938,385.73641122)(263.00329947,385.76141119)(262.91330994,385.79141113)
\curveto(262.83329964,385.81141114)(262.75329972,385.84141111)(262.67330994,385.88141113)
\curveto(262.35330012,386.02141093)(262.10330037,386.21641074)(261.92330994,386.46641113)
\curveto(261.74330073,386.72641023)(261.59330088,387.03140992)(261.47330994,387.38141113)
\curveto(261.45330102,387.46140949)(261.43830103,387.54640941)(261.42830994,387.63641113)
\curveto(261.41830105,387.72640923)(261.40330107,387.81140914)(261.38330994,387.89141113)
\curveto(261.3733011,387.92140903)(261.3683011,387.951409)(261.36830994,387.98141113)
\lineto(261.36830994,388.08641113)
\curveto(261.34830112,388.16640879)(261.33830113,388.24640871)(261.33830994,388.32641113)
\lineto(261.33830994,388.46141113)
\curveto(261.31830115,388.56140839)(261.31830115,388.66140829)(261.33830994,388.76141113)
\lineto(261.33830994,388.94141113)
\curveto(261.34830112,388.99140796)(261.35330112,389.03640792)(261.35330994,389.07641113)
\curveto(261.35330112,389.12640783)(261.35830111,389.17140778)(261.36830994,389.21141113)
\curveto(261.37830109,389.2514077)(261.38330109,389.28640767)(261.38330994,389.31641113)
\curveto(261.38330109,389.3564076)(261.38830108,389.39640756)(261.39830994,389.43641113)
\lineto(261.45830994,389.76641113)
\curveto(261.47830099,389.88640707)(261.50830096,389.99640696)(261.54830994,390.09641113)
\curveto(261.68830078,390.42640653)(261.84830062,390.70140625)(262.02830994,390.92141113)
\curveto(262.21830025,391.1514058)(262.47829999,391.33640562)(262.80830994,391.47641113)
\curveto(262.88829958,391.51640544)(262.9732995,391.54140541)(263.06330994,391.55141113)
\lineto(263.36330994,391.61141113)
\lineto(263.49830994,391.61141113)
\curveto(263.54829892,391.62140533)(263.59829887,391.62640533)(263.64830994,391.62641113)
\curveto(264.21829825,391.64640531)(264.67829779,391.54140541)(265.02830994,391.31141113)
\curveto(265.38829708,391.09140586)(265.65329682,390.79140616)(265.82330994,390.41141113)
\curveto(265.8732966,390.31140664)(265.91329656,390.21140674)(265.94330994,390.11141113)
\curveto(265.9732965,390.01140694)(266.00329647,389.90640705)(266.03330994,389.79641113)
\curveto(266.04329643,389.7564072)(266.04829642,389.72140723)(266.04830994,389.69141113)
\curveto(266.04829642,389.67140728)(266.05329642,389.64140731)(266.06330994,389.60141113)
\curveto(266.08329639,389.53140742)(266.09329638,389.4564075)(266.09330994,389.37641113)
\curveto(266.09329638,389.29640766)(266.10329637,389.21640774)(266.12330994,389.13641113)
\curveto(266.12329635,389.08640787)(266.12329635,389.04140791)(266.12330994,389.00141113)
\curveto(266.12329635,388.96140799)(266.12829634,388.91640804)(266.13830994,388.86641113)
\moveto(265.02830994,388.43141113)
\curveto(265.03829743,388.48140847)(265.04329743,388.5564084)(265.04330994,388.65641113)
\curveto(265.05329742,388.7564082)(265.04829742,388.83140812)(265.02830994,388.88141113)
\curveto(265.00829746,388.94140801)(265.00329747,388.99640796)(265.01330994,389.04641113)
\curveto(265.03329744,389.10640785)(265.03329744,389.16640779)(265.01330994,389.22641113)
\curveto(265.00329747,389.2564077)(264.99829747,389.29140766)(264.99830994,389.33141113)
\curveto(264.99829747,389.37140758)(264.99329748,389.41140754)(264.98330994,389.45141113)
\curveto(264.96329751,389.53140742)(264.94329753,389.60640735)(264.92330994,389.67641113)
\curveto(264.91329756,389.7564072)(264.89829757,389.83640712)(264.87830994,389.91641113)
\curveto(264.84829762,389.97640698)(264.82329765,390.03640692)(264.80330994,390.09641113)
\curveto(264.78329769,390.1564068)(264.75329772,390.21640674)(264.71330994,390.27641113)
\curveto(264.61329786,390.44640651)(264.48329799,390.58140637)(264.32330994,390.68141113)
\curveto(264.24329823,390.73140622)(264.14829832,390.76640619)(264.03830994,390.78641113)
\curveto(263.92829854,390.80640615)(263.80329867,390.81640614)(263.66330994,390.81641113)
\curveto(263.64329883,390.80640615)(263.61829885,390.80140615)(263.58830994,390.80141113)
\curveto(263.55829891,390.81140614)(263.52829894,390.81140614)(263.49830994,390.80141113)
\lineto(263.34830994,390.74141113)
\curveto(263.29829917,390.73140622)(263.25329922,390.71640624)(263.21330994,390.69641113)
\curveto(263.02329945,390.58640637)(262.87829959,390.44140651)(262.77830994,390.26141113)
\curveto(262.68829978,390.08140687)(262.60829986,389.87640708)(262.53830994,389.64641113)
\curveto(262.49829997,389.51640744)(262.47829999,389.38140757)(262.47830994,389.24141113)
\curveto(262.47829999,389.11140784)(262.4683,388.96640799)(262.44830994,388.80641113)
\curveto(262.43830003,388.7564082)(262.42830004,388.69640826)(262.41830994,388.62641113)
\curveto(262.41830005,388.5564084)(262.42830004,388.49640846)(262.44830994,388.44641113)
\lineto(262.44830994,388.28141113)
\lineto(262.44830994,388.10141113)
\curveto(262.45830001,388.0514089)(262.4683,387.99640896)(262.47830994,387.93641113)
\curveto(262.48829998,387.88640907)(262.49329998,387.83140912)(262.49330994,387.77141113)
\curveto(262.50329997,387.71140924)(262.51829995,387.6564093)(262.53830994,387.60641113)
\curveto(262.58829988,387.41640954)(262.64829982,387.24140971)(262.71830994,387.08141113)
\curveto(262.78829968,386.92141003)(262.89329958,386.79141016)(263.03330994,386.69141113)
\curveto(263.16329931,386.59141036)(263.30329917,386.52141043)(263.45330994,386.48141113)
\curveto(263.48329899,386.47141048)(263.50829896,386.46641049)(263.52830994,386.46641113)
\curveto(263.55829891,386.47641048)(263.58829888,386.47641048)(263.61830994,386.46641113)
\curveto(263.63829883,386.46641049)(263.6682988,386.46141049)(263.70830994,386.45141113)
\curveto(263.74829872,386.4514105)(263.78329869,386.4564105)(263.81330994,386.46641113)
\curveto(263.85329862,386.47641048)(263.89329858,386.48141047)(263.93330994,386.48141113)
\curveto(263.9732985,386.48141047)(264.01329846,386.49141046)(264.05330994,386.51141113)
\curveto(264.29329818,386.59141036)(264.48829798,386.72641023)(264.63830994,386.91641113)
\curveto(264.75829771,387.09640986)(264.84829762,387.30140965)(264.90830994,387.53141113)
\curveto(264.92829754,387.60140935)(264.94329753,387.67140928)(264.95330994,387.74141113)
\curveto(264.96329751,387.82140913)(264.97829749,387.90140905)(264.99830994,387.98141113)
\curveto(264.99829747,388.04140891)(265.00329747,388.08640887)(265.01330994,388.11641113)
\curveto(265.01329746,388.13640882)(265.01329746,388.16140879)(265.01330994,388.19141113)
\curveto(265.01329746,388.23140872)(265.01829745,388.26140869)(265.02830994,388.28141113)
\lineto(265.02830994,388.43141113)
}
}
{
\newrgbcolor{curcolor}{0 0 0}
\pscustom[linestyle=none,fillstyle=solid,fillcolor=curcolor]
{
\newpath
\moveto(663.69792969,593.58429443)
\curveto(664.38792505,593.5942838)(664.98792445,593.47428392)(665.49792969,593.22429443)
\curveto(666.01792342,592.97428442)(666.41292303,592.63928476)(666.68292969,592.21929443)
\curveto(666.73292271,592.13928526)(666.77792266,592.04928535)(666.81792969,591.94929443)
\curveto(666.85792258,591.85928554)(666.90292254,591.76428563)(666.95292969,591.66429443)
\curveto(666.99292245,591.56428583)(667.02292242,591.46428593)(667.04292969,591.36429443)
\curveto(667.06292238,591.26428613)(667.08292236,591.15928624)(667.10292969,591.04929443)
\curveto(667.12292232,590.9992864)(667.12792231,590.95428644)(667.11792969,590.91429443)
\curveto(667.10792233,590.87428652)(667.11292233,590.82928657)(667.13292969,590.77929443)
\curveto(667.1429223,590.72928667)(667.14792229,590.64428675)(667.14792969,590.52429443)
\curveto(667.14792229,590.41428698)(667.1429223,590.32928707)(667.13292969,590.26929443)
\curveto(667.11292233,590.20928719)(667.10292234,590.14928725)(667.10292969,590.08929443)
\curveto(667.11292233,590.02928737)(667.10792233,589.96928743)(667.08792969,589.90929443)
\curveto(667.04792239,589.76928763)(667.01292243,589.63428776)(666.98292969,589.50429443)
\curveto(666.95292249,589.37428802)(666.91292253,589.24928815)(666.86292969,589.12929443)
\curveto(666.80292264,588.98928841)(666.73292271,588.86428853)(666.65292969,588.75429443)
\curveto(666.58292286,588.64428875)(666.50792293,588.53428886)(666.42792969,588.42429443)
\lineto(666.36792969,588.36429443)
\curveto(666.35792308,588.34428905)(666.3429231,588.32428907)(666.32292969,588.30429443)
\curveto(666.20292324,588.14428925)(666.06792337,587.9992894)(665.91792969,587.86929443)
\curveto(665.76792367,587.73928966)(665.60792383,587.61428978)(665.43792969,587.49429443)
\curveto(665.12792431,587.27429012)(664.83292461,587.06929033)(664.55292969,586.87929443)
\curveto(664.32292512,586.73929066)(664.09292535,586.60429079)(663.86292969,586.47429443)
\curveto(663.6429258,586.34429105)(663.42292602,586.20929119)(663.20292969,586.06929443)
\curveto(662.95292649,585.8992915)(662.71292673,585.71929168)(662.48292969,585.52929443)
\curveto(662.26292718,585.33929206)(662.07292737,585.11429228)(661.91292969,584.85429443)
\curveto(661.87292757,584.7942926)(661.8379276,584.73429266)(661.80792969,584.67429443)
\curveto(661.77792766,584.62429277)(661.74792769,584.55929284)(661.71792969,584.47929443)
\curveto(661.69792774,584.40929299)(661.69292775,584.34929305)(661.70292969,584.29929443)
\curveto(661.72292772,584.22929317)(661.75792768,584.17429322)(661.80792969,584.13429443)
\curveto(661.85792758,584.10429329)(661.91792752,584.08429331)(661.98792969,584.07429443)
\lineto(662.22792969,584.07429443)
\lineto(662.97792969,584.07429443)
\lineto(665.78292969,584.07429443)
\lineto(666.44292969,584.07429443)
\curveto(666.53292291,584.07429332)(666.61792282,584.06929333)(666.69792969,584.05929443)
\curveto(666.77792266,584.05929334)(666.8429226,584.03929336)(666.89292969,583.99929443)
\curveto(666.9429225,583.95929344)(666.98292246,583.88429351)(667.01292969,583.77429443)
\curveto(667.05292239,583.67429372)(667.06292238,583.57429382)(667.04292969,583.47429443)
\lineto(667.04292969,583.33929443)
\curveto(667.02292242,583.26929413)(667.00292244,583.20929419)(666.98292969,583.15929443)
\curveto(666.96292248,583.10929429)(666.92792251,583.06929433)(666.87792969,583.03929443)
\curveto(666.82792261,582.9992944)(666.75792268,582.97929442)(666.66792969,582.97929443)
\lineto(666.39792969,582.97929443)
\lineto(665.49792969,582.97929443)
\lineto(661.98792969,582.97929443)
\lineto(660.92292969,582.97929443)
\curveto(660.8429286,582.97929442)(660.75292869,582.97429442)(660.65292969,582.96429443)
\curveto(660.55292889,582.96429443)(660.46792897,582.97429442)(660.39792969,582.99429443)
\curveto(660.18792925,583.06429433)(660.12292932,583.24429415)(660.20292969,583.53429443)
\curveto(660.21292923,583.57429382)(660.21292923,583.60929379)(660.20292969,583.63929443)
\curveto(660.20292924,583.67929372)(660.21292923,583.72429367)(660.23292969,583.77429443)
\curveto(660.25292919,583.85429354)(660.27292917,583.93929346)(660.29292969,584.02929443)
\curveto(660.31292913,584.11929328)(660.3379291,584.20429319)(660.36792969,584.28429443)
\curveto(660.52792891,584.77429262)(660.72792871,585.18929221)(660.96792969,585.52929443)
\curveto(661.14792829,585.77929162)(661.35292809,586.00429139)(661.58292969,586.20429443)
\curveto(661.81292763,586.41429098)(662.05292739,586.60929079)(662.30292969,586.78929443)
\curveto(662.56292688,586.96929043)(662.82792661,587.13929026)(663.09792969,587.29929443)
\curveto(663.37792606,587.46928993)(663.64792579,587.64428975)(663.90792969,587.82429443)
\curveto(664.01792542,587.90428949)(664.12292532,587.97928942)(664.22292969,588.04929443)
\curveto(664.33292511,588.11928928)(664.442925,588.1942892)(664.55292969,588.27429443)
\curveto(664.59292485,588.30428909)(664.62792481,588.33428906)(664.65792969,588.36429443)
\curveto(664.69792474,588.40428899)(664.7379247,588.43428896)(664.77792969,588.45429443)
\curveto(664.91792452,588.56428883)(665.0429244,588.68928871)(665.15292969,588.82929443)
\curveto(665.17292427,588.85928854)(665.19792424,588.88428851)(665.22792969,588.90429443)
\curveto(665.25792418,588.93428846)(665.28292416,588.96428843)(665.30292969,588.99429443)
\curveto(665.38292406,589.0942883)(665.44792399,589.1942882)(665.49792969,589.29429443)
\curveto(665.55792388,589.394288)(665.61292383,589.50428789)(665.66292969,589.62429443)
\curveto(665.69292375,589.6942877)(665.71292373,589.76928763)(665.72292969,589.84929443)
\lineto(665.78292969,590.08929443)
\lineto(665.78292969,590.17929443)
\curveto(665.79292365,590.20928719)(665.79792364,590.23928716)(665.79792969,590.26929443)
\curveto(665.81792362,590.33928706)(665.82292362,590.43428696)(665.81292969,590.55429443)
\curveto(665.81292363,590.68428671)(665.80292364,590.78428661)(665.78292969,590.85429443)
\curveto(665.76292368,590.93428646)(665.7429237,591.00928639)(665.72292969,591.07929443)
\curveto(665.71292373,591.15928624)(665.69292375,591.23928616)(665.66292969,591.31929443)
\curveto(665.55292389,591.55928584)(665.40292404,591.75928564)(665.21292969,591.91929443)
\curveto(665.03292441,592.08928531)(664.81292463,592.22928517)(664.55292969,592.33929443)
\curveto(664.48292496,592.35928504)(664.41292503,592.37428502)(664.34292969,592.38429443)
\curveto(664.27292517,592.40428499)(664.19792524,592.42428497)(664.11792969,592.44429443)
\curveto(664.0379254,592.46428493)(663.92792551,592.47428492)(663.78792969,592.47429443)
\curveto(663.65792578,592.47428492)(663.55292589,592.46428493)(663.47292969,592.44429443)
\curveto(663.41292603,592.43428496)(663.35792608,592.42928497)(663.30792969,592.42929443)
\curveto(663.25792618,592.42928497)(663.20792623,592.41928498)(663.15792969,592.39929443)
\curveto(663.05792638,592.35928504)(662.96292648,592.31928508)(662.87292969,592.27929443)
\curveto(662.79292665,592.23928516)(662.71292673,592.1942852)(662.63292969,592.14429443)
\curveto(662.60292684,592.12428527)(662.57292687,592.0992853)(662.54292969,592.06929443)
\curveto(662.52292692,592.03928536)(662.49792694,592.01428538)(662.46792969,591.99429443)
\lineto(662.39292969,591.91929443)
\curveto(662.36292708,591.8992855)(662.3379271,591.87928552)(662.31792969,591.85929443)
\lineto(662.16792969,591.64929443)
\curveto(662.12792731,591.58928581)(662.08292736,591.52428587)(662.03292969,591.45429443)
\curveto(661.97292747,591.36428603)(661.92292752,591.25928614)(661.88292969,591.13929443)
\curveto(661.85292759,591.02928637)(661.81792762,590.91928648)(661.77792969,590.80929443)
\curveto(661.7379277,590.6992867)(661.71292773,590.55428684)(661.70292969,590.37429443)
\curveto(661.69292775,590.20428719)(661.66292778,590.07928732)(661.61292969,589.99929443)
\curveto(661.56292788,589.91928748)(661.48792795,589.87428752)(661.38792969,589.86429443)
\curveto(661.28792815,589.85428754)(661.17792826,589.84928755)(661.05792969,589.84929443)
\curveto(661.01792842,589.84928755)(660.97792846,589.84428755)(660.93792969,589.83429443)
\curveto(660.89792854,589.83428756)(660.86292858,589.83928756)(660.83292969,589.84929443)
\curveto(660.78292866,589.86928753)(660.73292871,589.87928752)(660.68292969,589.87929443)
\curveto(660.6429288,589.87928752)(660.60292884,589.88928751)(660.56292969,589.90929443)
\curveto(660.47292897,589.96928743)(660.42792901,590.10428729)(660.42792969,590.31429443)
\lineto(660.42792969,590.43429443)
\curveto(660.437929,590.4942869)(660.442929,590.55428684)(660.44292969,590.61429443)
\curveto(660.45292899,590.68428671)(660.46292898,590.74928665)(660.47292969,590.80929443)
\curveto(660.49292895,590.91928648)(660.51292893,591.01928638)(660.53292969,591.10929443)
\curveto(660.55292889,591.20928619)(660.58292886,591.30428609)(660.62292969,591.39429443)
\curveto(660.6429288,591.46428593)(660.66292878,591.52428587)(660.68292969,591.57429443)
\lineto(660.74292969,591.75429443)
\curveto(660.86292858,592.01428538)(661.01792842,592.25928514)(661.20792969,592.48929443)
\curveto(661.40792803,592.71928468)(661.62292782,592.90428449)(661.85292969,593.04429443)
\curveto(661.96292748,593.12428427)(662.07792736,593.18928421)(662.19792969,593.23929443)
\lineto(662.58792969,593.38929443)
\curveto(662.69792674,593.43928396)(662.81292663,593.46928393)(662.93292969,593.47929443)
\curveto(663.05292639,593.4992839)(663.17792626,593.52428387)(663.30792969,593.55429443)
\curveto(663.37792606,593.55428384)(663.442926,593.55428384)(663.50292969,593.55429443)
\curveto(663.56292588,593.56428383)(663.62792581,593.57428382)(663.69792969,593.58429443)
}
}
{
\newrgbcolor{curcolor}{0 0 0}
\pscustom[linestyle=none,fillstyle=solid,fillcolor=curcolor]
{
\newpath
\moveto(669.25753906,593.38929443)
\lineto(674.05753906,593.38929443)
\lineto(675.06253906,593.38929443)
\curveto(675.20253196,593.38928401)(675.32253184,593.37928402)(675.42253906,593.35929443)
\curveto(675.53253163,593.34928405)(675.61253155,593.30428409)(675.66253906,593.22429443)
\curveto(675.68253148,593.18428421)(675.69253147,593.13428426)(675.69253906,593.07429443)
\curveto(675.70253146,593.01428438)(675.70753146,592.94928445)(675.70753906,592.87929443)
\lineto(675.70753906,592.60929443)
\curveto(675.70753146,592.51928488)(675.69753147,592.43928496)(675.67753906,592.36929443)
\curveto(675.63753153,592.28928511)(675.59253157,592.21928518)(675.54253906,592.15929443)
\lineto(675.39253906,591.97929443)
\curveto(675.3625318,591.92928547)(675.32753184,591.88928551)(675.28753906,591.85929443)
\curveto(675.24753192,591.82928557)(675.20753196,591.78928561)(675.16753906,591.73929443)
\curveto(675.08753208,591.62928577)(675.00253216,591.51928588)(674.91253906,591.40929443)
\curveto(674.82253234,591.30928609)(674.73753243,591.20428619)(674.65753906,591.09429443)
\curveto(674.51753265,590.8942865)(674.37753279,590.68428671)(674.23753906,590.46429443)
\curveto(674.09753307,590.25428714)(673.95753321,590.03928736)(673.81753906,589.81929443)
\curveto(673.7675334,589.72928767)(673.71753345,589.63428776)(673.66753906,589.53429443)
\curveto(673.61753355,589.43428796)(673.5625336,589.33928806)(673.50253906,589.24929443)
\curveto(673.48253368,589.22928817)(673.47253369,589.20428819)(673.47253906,589.17429443)
\curveto(673.47253369,589.14428825)(673.4625337,589.11928828)(673.44253906,589.09929443)
\curveto(673.37253379,588.9992884)(673.30753386,588.88428851)(673.24753906,588.75429443)
\curveto(673.18753398,588.63428876)(673.13253403,588.51928888)(673.08253906,588.40929443)
\curveto(672.98253418,588.17928922)(672.88753428,587.94428945)(672.79753906,587.70429443)
\curveto(672.70753446,587.46428993)(672.60753456,587.22429017)(672.49753906,586.98429443)
\curveto(672.47753469,586.93429046)(672.4625347,586.88929051)(672.45253906,586.84929443)
\curveto(672.45253471,586.80929059)(672.44253472,586.76429063)(672.42253906,586.71429443)
\curveto(672.37253479,586.5942908)(672.32753484,586.46929093)(672.28753906,586.33929443)
\curveto(672.25753491,586.21929118)(672.22253494,586.0992913)(672.18253906,585.97929443)
\curveto(672.10253506,585.74929165)(672.03753513,585.50929189)(671.98753906,585.25929443)
\curveto(671.94753522,585.01929238)(671.89753527,584.77929262)(671.83753906,584.53929443)
\curveto(671.79753537,584.38929301)(671.77253539,584.23929316)(671.76253906,584.08929443)
\curveto(671.75253541,583.93929346)(671.73253543,583.78929361)(671.70253906,583.63929443)
\curveto(671.69253547,583.5992938)(671.68753548,583.53929386)(671.68753906,583.45929443)
\curveto(671.65753551,583.33929406)(671.62753554,583.23929416)(671.59753906,583.15929443)
\curveto(671.5675356,583.07929432)(671.49753567,583.02429437)(671.38753906,582.99429443)
\curveto(671.33753583,582.97429442)(671.28253588,582.96429443)(671.22253906,582.96429443)
\lineto(671.02753906,582.96429443)
\curveto(670.88753628,582.96429443)(670.74753642,582.96929443)(670.60753906,582.97929443)
\curveto(670.47753669,582.98929441)(670.38253678,583.03429436)(670.32253906,583.11429443)
\curveto(670.28253688,583.17429422)(670.2625369,583.25929414)(670.26253906,583.36929443)
\curveto(670.27253689,583.47929392)(670.28753688,583.57429382)(670.30753906,583.65429443)
\lineto(670.30753906,583.72929443)
\curveto(670.31753685,583.75929364)(670.32253684,583.78929361)(670.32253906,583.81929443)
\curveto(670.34253682,583.8992935)(670.35253681,583.97429342)(670.35253906,584.04429443)
\curveto(670.35253681,584.11429328)(670.3625368,584.18429321)(670.38253906,584.25429443)
\curveto(670.43253673,584.44429295)(670.47253669,584.62929277)(670.50253906,584.80929443)
\curveto(670.53253663,584.9992924)(670.57253659,585.17929222)(670.62253906,585.34929443)
\curveto(670.64253652,585.399292)(670.65253651,585.43929196)(670.65253906,585.46929443)
\curveto(670.65253651,585.4992919)(670.65753651,585.53429186)(670.66753906,585.57429443)
\curveto(670.7675364,585.87429152)(670.85753631,586.16929123)(670.93753906,586.45929443)
\curveto(671.02753614,586.74929065)(671.13253603,587.02929037)(671.25253906,587.29929443)
\curveto(671.51253565,587.87928952)(671.78253538,588.42928897)(672.06253906,588.94929443)
\curveto(672.34253482,589.47928792)(672.65253451,589.98428741)(672.99253906,590.46429443)
\curveto(673.13253403,590.66428673)(673.28253388,590.85428654)(673.44253906,591.03429443)
\curveto(673.60253356,591.22428617)(673.75253341,591.41428598)(673.89253906,591.60429443)
\curveto(673.93253323,591.65428574)(673.9675332,591.6992857)(673.99753906,591.73929443)
\curveto(674.03753313,591.78928561)(674.07253309,591.83928556)(674.10253906,591.88929443)
\curveto(674.11253305,591.90928549)(674.12253304,591.93428546)(674.13253906,591.96429443)
\curveto(674.15253301,591.9942854)(674.15253301,592.02428537)(674.13253906,592.05429443)
\curveto(674.11253305,592.11428528)(674.07753309,592.14928525)(674.02753906,592.15929443)
\curveto(673.97753319,592.17928522)(673.92753324,592.1992852)(673.87753906,592.21929443)
\lineto(673.77253906,592.21929443)
\curveto(673.73253343,592.22928517)(673.68253348,592.22928517)(673.62253906,592.21929443)
\lineto(673.47253906,592.21929443)
\lineto(672.87253906,592.21929443)
\lineto(670.23253906,592.21929443)
\lineto(669.49753906,592.21929443)
\lineto(669.25753906,592.21929443)
\curveto(669.18753798,592.22928517)(669.12753804,592.24428515)(669.07753906,592.26429443)
\curveto(668.98753818,592.30428509)(668.92753824,592.36428503)(668.89753906,592.44429443)
\curveto(668.84753832,592.54428485)(668.83253833,592.68928471)(668.85253906,592.87929443)
\curveto(668.87253829,593.07928432)(668.90753826,593.21428418)(668.95753906,593.28429443)
\curveto(668.97753819,593.30428409)(669.00253816,593.31928408)(669.03253906,593.32929443)
\lineto(669.15253906,593.38929443)
\curveto(669.17253799,593.38928401)(669.18753798,593.38428401)(669.19753906,593.37429443)
\curveto(669.21753795,593.37428402)(669.23753793,593.37928402)(669.25753906,593.38929443)
}
}
{
\newrgbcolor{curcolor}{0 0 0}
\pscustom[linestyle=none,fillstyle=solid,fillcolor=curcolor]
{
\newpath
\moveto(686.95214844,591.49929443)
\curveto(686.75213814,591.20928619)(686.54213835,590.92428647)(686.32214844,590.64429443)
\curveto(686.11213878,590.36428703)(685.90713898,590.07928732)(685.70714844,589.78929443)
\curveto(685.10713978,588.93928846)(684.50214039,588.0992893)(683.89214844,587.26929443)
\curveto(683.28214161,586.44929095)(682.67714221,585.61429178)(682.07714844,584.76429443)
\lineto(681.56714844,584.04429443)
\lineto(681.05714844,583.35429443)
\curveto(680.97714391,583.24429415)(680.89714399,583.12929427)(680.81714844,583.00929443)
\curveto(680.73714415,582.88929451)(680.64214425,582.7942946)(680.53214844,582.72429443)
\curveto(680.4921444,582.70429469)(680.42714446,582.68929471)(680.33714844,582.67929443)
\curveto(680.25714463,582.65929474)(680.16714472,582.64929475)(680.06714844,582.64929443)
\curveto(679.96714492,582.64929475)(679.87214502,582.65429474)(679.78214844,582.66429443)
\curveto(679.70214519,582.67429472)(679.64214525,582.6942947)(679.60214844,582.72429443)
\curveto(679.57214532,582.74429465)(679.54714534,582.77929462)(679.52714844,582.82929443)
\curveto(679.51714537,582.86929453)(679.52214537,582.91429448)(679.54214844,582.96429443)
\curveto(679.58214531,583.04429435)(679.62714526,583.11929428)(679.67714844,583.18929443)
\curveto(679.73714515,583.26929413)(679.7921451,583.34929405)(679.84214844,583.42929443)
\curveto(680.08214481,583.76929363)(680.32714456,584.10429329)(680.57714844,584.43429443)
\curveto(680.82714406,584.76429263)(681.06714382,585.0992923)(681.29714844,585.43929443)
\curveto(681.45714343,585.65929174)(681.61714327,585.87429152)(681.77714844,586.08429443)
\curveto(681.93714295,586.2942911)(682.09714279,586.50929089)(682.25714844,586.72929443)
\curveto(682.61714227,587.24929015)(682.98214191,587.75928964)(683.35214844,588.25929443)
\curveto(683.72214117,588.75928864)(684.0921408,589.26928813)(684.46214844,589.78929443)
\curveto(684.60214029,589.98928741)(684.74214015,590.18428721)(684.88214844,590.37429443)
\curveto(685.03213986,590.56428683)(685.17713971,590.75928664)(685.31714844,590.95929443)
\curveto(685.52713936,591.25928614)(685.74213915,591.55928584)(685.96214844,591.85929443)
\lineto(686.62214844,592.75929443)
\lineto(686.80214844,593.02929443)
\lineto(687.01214844,593.29929443)
\lineto(687.13214844,593.47929443)
\curveto(687.18213771,593.53928386)(687.23213766,593.5942838)(687.28214844,593.64429443)
\curveto(687.35213754,593.6942837)(687.42713746,593.72928367)(687.50714844,593.74929443)
\curveto(687.52713736,593.75928364)(687.55213734,593.75928364)(687.58214844,593.74929443)
\curveto(687.62213727,593.74928365)(687.65213724,593.75928364)(687.67214844,593.77929443)
\curveto(687.7921371,593.77928362)(687.92713696,593.77428362)(688.07714844,593.76429443)
\curveto(688.22713666,593.76428363)(688.31713657,593.71928368)(688.34714844,593.62929443)
\curveto(688.36713652,593.5992838)(688.37213652,593.56428383)(688.36214844,593.52429443)
\curveto(688.35213654,593.48428391)(688.33713655,593.45428394)(688.31714844,593.43429443)
\curveto(688.27713661,593.35428404)(688.23713665,593.28428411)(688.19714844,593.22429443)
\curveto(688.15713673,593.16428423)(688.11213678,593.10428429)(688.06214844,593.04429443)
\lineto(687.49214844,592.26429443)
\curveto(687.31213758,592.01428538)(687.13213776,591.75928564)(686.95214844,591.49929443)
\moveto(680.09714844,587.59929443)
\curveto(680.04714484,587.61928978)(679.99714489,587.62428977)(679.94714844,587.61429443)
\curveto(679.89714499,587.60428979)(679.84714504,587.60928979)(679.79714844,587.62929443)
\curveto(679.6871452,587.64928975)(679.58214531,587.66928973)(679.48214844,587.68929443)
\curveto(679.3921455,587.71928968)(679.29714559,587.75928964)(679.19714844,587.80929443)
\curveto(678.86714602,587.94928945)(678.61214628,588.14428925)(678.43214844,588.39429443)
\curveto(678.25214664,588.65428874)(678.10714678,588.96428843)(677.99714844,589.32429443)
\curveto(677.96714692,589.40428799)(677.94714694,589.48428791)(677.93714844,589.56429443)
\curveto(677.92714696,589.65428774)(677.91214698,589.73928766)(677.89214844,589.81929443)
\curveto(677.88214701,589.86928753)(677.87714701,589.93428746)(677.87714844,590.01429443)
\curveto(677.86714702,590.04428735)(677.86214703,590.07428732)(677.86214844,590.10429443)
\curveto(677.86214703,590.14428725)(677.85714703,590.17928722)(677.84714844,590.20929443)
\lineto(677.84714844,590.35929443)
\curveto(677.83714705,590.40928699)(677.83214706,590.46928693)(677.83214844,590.53929443)
\curveto(677.83214706,590.61928678)(677.83714705,590.68428671)(677.84714844,590.73429443)
\lineto(677.84714844,590.89929443)
\curveto(677.86714702,590.94928645)(677.87214702,590.9942864)(677.86214844,591.03429443)
\curveto(677.86214703,591.08428631)(677.86714702,591.12928627)(677.87714844,591.16929443)
\curveto(677.887147,591.20928619)(677.892147,591.24428615)(677.89214844,591.27429443)
\curveto(677.892147,591.31428608)(677.89714699,591.35428604)(677.90714844,591.39429443)
\curveto(677.93714695,591.50428589)(677.95714693,591.61428578)(677.96714844,591.72429443)
\curveto(677.9871469,591.84428555)(678.02214687,591.95928544)(678.07214844,592.06929443)
\curveto(678.21214668,592.40928499)(678.37214652,592.68428471)(678.55214844,592.89429443)
\curveto(678.74214615,593.11428428)(679.01214588,593.2942841)(679.36214844,593.43429443)
\curveto(679.44214545,593.46428393)(679.52714536,593.48428391)(679.61714844,593.49429443)
\curveto(679.70714518,593.51428388)(679.80214509,593.53428386)(679.90214844,593.55429443)
\curveto(679.93214496,593.56428383)(679.9871449,593.56428383)(680.06714844,593.55429443)
\curveto(680.14714474,593.55428384)(680.19714469,593.56428383)(680.21714844,593.58429443)
\curveto(680.77714411,593.5942838)(681.22714366,593.48428391)(681.56714844,593.25429443)
\curveto(681.91714297,593.02428437)(682.17714271,592.71928468)(682.34714844,592.33929443)
\curveto(682.3871425,592.24928515)(682.42214247,592.15428524)(682.45214844,592.05429443)
\curveto(682.48214241,591.95428544)(682.50714238,591.85428554)(682.52714844,591.75429443)
\curveto(682.54714234,591.72428567)(682.55214234,591.6942857)(682.54214844,591.66429443)
\curveto(682.54214235,591.63428576)(682.54714234,591.60428579)(682.55714844,591.57429443)
\curveto(682.5871423,591.46428593)(682.60714228,591.33928606)(682.61714844,591.19929443)
\curveto(682.62714226,591.06928633)(682.63714225,590.93428646)(682.64714844,590.79429443)
\lineto(682.64714844,590.62929443)
\curveto(682.65714223,590.56928683)(682.65714223,590.51428688)(682.64714844,590.46429443)
\curveto(682.63714225,590.41428698)(682.63214226,590.36428703)(682.63214844,590.31429443)
\lineto(682.63214844,590.17929443)
\curveto(682.62214227,590.13928726)(682.61714227,590.0992873)(682.61714844,590.05929443)
\curveto(682.62714226,590.01928738)(682.62214227,589.97428742)(682.60214844,589.92429443)
\curveto(682.58214231,589.81428758)(682.56214233,589.70928769)(682.54214844,589.60929443)
\curveto(682.53214236,589.50928789)(682.51214238,589.40928799)(682.48214844,589.30929443)
\curveto(682.35214254,588.94928845)(682.1871427,588.63428876)(681.98714844,588.36429443)
\curveto(681.7871431,588.0942893)(681.51214338,587.88928951)(681.16214844,587.74929443)
\curveto(681.08214381,587.71928968)(680.99714389,587.6942897)(680.90714844,587.67429443)
\lineto(680.63714844,587.61429443)
\curveto(680.5871443,587.60428979)(680.54214435,587.5992898)(680.50214844,587.59929443)
\curveto(680.46214443,587.60928979)(680.42214447,587.60928979)(680.38214844,587.59929443)
\curveto(680.28214461,587.57928982)(680.1871447,587.57928982)(680.09714844,587.59929443)
\moveto(679.25714844,588.99429443)
\curveto(679.29714559,588.92428847)(679.33714555,588.85928854)(679.37714844,588.79929443)
\curveto(679.41714547,588.74928865)(679.46714542,588.6992887)(679.52714844,588.64929443)
\lineto(679.67714844,588.52929443)
\curveto(679.73714515,588.4992889)(679.80214509,588.47428892)(679.87214844,588.45429443)
\curveto(679.91214498,588.43428896)(679.94714494,588.42428897)(679.97714844,588.42429443)
\curveto(680.01714487,588.43428896)(680.05714483,588.42928897)(680.09714844,588.40929443)
\curveto(680.12714476,588.40928899)(680.16714472,588.40428899)(680.21714844,588.39429443)
\curveto(680.26714462,588.394289)(680.30714458,588.399289)(680.33714844,588.40929443)
\lineto(680.56214844,588.45429443)
\curveto(680.81214408,588.53428886)(680.99714389,588.65928874)(681.11714844,588.82929443)
\curveto(681.19714369,588.92928847)(681.26714362,589.05928834)(681.32714844,589.21929443)
\curveto(681.40714348,589.399288)(681.46714342,589.62428777)(681.50714844,589.89429443)
\curveto(681.54714334,590.17428722)(681.56214333,590.45428694)(681.55214844,590.73429443)
\curveto(681.54214335,591.02428637)(681.51214338,591.2992861)(681.46214844,591.55929443)
\curveto(681.41214348,591.81928558)(681.33714355,592.02928537)(681.23714844,592.18929443)
\curveto(681.11714377,592.38928501)(680.96714392,592.53928486)(680.78714844,592.63929443)
\curveto(680.70714418,592.68928471)(680.61714427,592.71928468)(680.51714844,592.72929443)
\curveto(680.41714447,592.74928465)(680.31214458,592.75928464)(680.20214844,592.75929443)
\curveto(680.18214471,592.74928465)(680.15714473,592.74428465)(680.12714844,592.74429443)
\curveto(680.10714478,592.75428464)(680.0871448,592.75428464)(680.06714844,592.74429443)
\curveto(680.01714487,592.73428466)(679.97214492,592.72428467)(679.93214844,592.71429443)
\curveto(679.892145,592.71428468)(679.85214504,592.70428469)(679.81214844,592.68429443)
\curveto(679.63214526,592.60428479)(679.48214541,592.48428491)(679.36214844,592.32429443)
\curveto(679.25214564,592.16428523)(679.16214573,591.98428541)(679.09214844,591.78429443)
\curveto(679.03214586,591.5942858)(678.9871459,591.36928603)(678.95714844,591.10929443)
\curveto(678.93714595,590.84928655)(678.93214596,590.58428681)(678.94214844,590.31429443)
\curveto(678.95214594,590.05428734)(678.98214591,589.80428759)(679.03214844,589.56429443)
\curveto(679.0921458,589.33428806)(679.16714572,589.14428825)(679.25714844,588.99429443)
\moveto(690.05714844,586.00929443)
\curveto(690.06713482,585.95929144)(690.07213482,585.86929153)(690.07214844,585.73929443)
\curveto(690.07213482,585.60929179)(690.06213483,585.51929188)(690.04214844,585.46929443)
\curveto(690.02213487,585.41929198)(690.01713487,585.36429203)(690.02714844,585.30429443)
\curveto(690.03713485,585.25429214)(690.03713485,585.20429219)(690.02714844,585.15429443)
\curveto(689.9871349,585.01429238)(689.95713493,584.87929252)(689.93714844,584.74929443)
\curveto(689.92713496,584.61929278)(689.89713499,584.4992929)(689.84714844,584.38929443)
\curveto(689.70713518,584.03929336)(689.54213535,583.74429365)(689.35214844,583.50429443)
\curveto(689.16213573,583.27429412)(688.892136,583.08929431)(688.54214844,582.94929443)
\curveto(688.46213643,582.91929448)(688.37713651,582.8992945)(688.28714844,582.88929443)
\curveto(688.19713669,582.86929453)(688.11213678,582.84929455)(688.03214844,582.82929443)
\curveto(687.98213691,582.81929458)(687.93213696,582.81429458)(687.88214844,582.81429443)
\curveto(687.83213706,582.81429458)(687.78213711,582.80929459)(687.73214844,582.79929443)
\curveto(687.70213719,582.78929461)(687.65213724,582.78929461)(687.58214844,582.79929443)
\curveto(687.51213738,582.7992946)(687.46213743,582.80429459)(687.43214844,582.81429443)
\curveto(687.37213752,582.83429456)(687.31213758,582.84429455)(687.25214844,582.84429443)
\curveto(687.20213769,582.83429456)(687.15213774,582.83929456)(687.10214844,582.85929443)
\curveto(687.01213788,582.87929452)(686.92213797,582.90429449)(686.83214844,582.93429443)
\curveto(686.75213814,582.95429444)(686.67213822,582.98429441)(686.59214844,583.02429443)
\curveto(686.27213862,583.16429423)(686.02213887,583.35929404)(685.84214844,583.60929443)
\curveto(685.66213923,583.86929353)(685.51213938,584.17429322)(685.39214844,584.52429443)
\curveto(685.37213952,584.60429279)(685.35713953,584.68929271)(685.34714844,584.77929443)
\curveto(685.33713955,584.86929253)(685.32213957,584.95429244)(685.30214844,585.03429443)
\curveto(685.2921396,585.06429233)(685.2871396,585.0942923)(685.28714844,585.12429443)
\lineto(685.28714844,585.22929443)
\curveto(685.26713962,585.30929209)(685.25713963,585.38929201)(685.25714844,585.46929443)
\lineto(685.25714844,585.60429443)
\curveto(685.23713965,585.70429169)(685.23713965,585.80429159)(685.25714844,585.90429443)
\lineto(685.25714844,586.08429443)
\curveto(685.26713962,586.13429126)(685.27213962,586.17929122)(685.27214844,586.21929443)
\curveto(685.27213962,586.26929113)(685.27713961,586.31429108)(685.28714844,586.35429443)
\curveto(685.29713959,586.394291)(685.30213959,586.42929097)(685.30214844,586.45929443)
\curveto(685.30213959,586.4992909)(685.30713958,586.53929086)(685.31714844,586.57929443)
\lineto(685.37714844,586.90929443)
\curveto(685.39713949,587.02929037)(685.42713946,587.13929026)(685.46714844,587.23929443)
\curveto(685.60713928,587.56928983)(685.76713912,587.84428955)(685.94714844,588.06429443)
\curveto(686.13713875,588.2942891)(686.39713849,588.47928892)(686.72714844,588.61929443)
\curveto(686.80713808,588.65928874)(686.892138,588.68428871)(686.98214844,588.69429443)
\lineto(687.28214844,588.75429443)
\lineto(687.41714844,588.75429443)
\curveto(687.46713742,588.76428863)(687.51713737,588.76928863)(687.56714844,588.76929443)
\curveto(688.13713675,588.78928861)(688.59713629,588.68428871)(688.94714844,588.45429443)
\curveto(689.30713558,588.23428916)(689.57213532,587.93428946)(689.74214844,587.55429443)
\curveto(689.7921351,587.45428994)(689.83213506,587.35429004)(689.86214844,587.25429443)
\curveto(689.892135,587.15429024)(689.92213497,587.04929035)(689.95214844,586.93929443)
\curveto(689.96213493,586.8992905)(689.96713492,586.86429053)(689.96714844,586.83429443)
\curveto(689.96713492,586.81429058)(689.97213492,586.78429061)(689.98214844,586.74429443)
\curveto(690.00213489,586.67429072)(690.01213488,586.5992908)(690.01214844,586.51929443)
\curveto(690.01213488,586.43929096)(690.02213487,586.35929104)(690.04214844,586.27929443)
\curveto(690.04213485,586.22929117)(690.04213485,586.18429121)(690.04214844,586.14429443)
\curveto(690.04213485,586.10429129)(690.04713484,586.05929134)(690.05714844,586.00929443)
\moveto(688.94714844,585.57429443)
\curveto(688.95713593,585.62429177)(688.96213593,585.6992917)(688.96214844,585.79929443)
\curveto(688.97213592,585.8992915)(688.96713592,585.97429142)(688.94714844,586.02429443)
\curveto(688.92713596,586.08429131)(688.92213597,586.13929126)(688.93214844,586.18929443)
\curveto(688.95213594,586.24929115)(688.95213594,586.30929109)(688.93214844,586.36929443)
\curveto(688.92213597,586.399291)(688.91713597,586.43429096)(688.91714844,586.47429443)
\curveto(688.91713597,586.51429088)(688.91213598,586.55429084)(688.90214844,586.59429443)
\curveto(688.88213601,586.67429072)(688.86213603,586.74929065)(688.84214844,586.81929443)
\curveto(688.83213606,586.8992905)(688.81713607,586.97929042)(688.79714844,587.05929443)
\curveto(688.76713612,587.11929028)(688.74213615,587.17929022)(688.72214844,587.23929443)
\curveto(688.70213619,587.2992901)(688.67213622,587.35929004)(688.63214844,587.41929443)
\curveto(688.53213636,587.58928981)(688.40213649,587.72428967)(688.24214844,587.82429443)
\curveto(688.16213673,587.87428952)(688.06713682,587.90928949)(687.95714844,587.92929443)
\curveto(687.84713704,587.94928945)(687.72213717,587.95928944)(687.58214844,587.95929443)
\curveto(687.56213733,587.94928945)(687.53713735,587.94428945)(687.50714844,587.94429443)
\curveto(687.47713741,587.95428944)(687.44713744,587.95428944)(687.41714844,587.94429443)
\lineto(687.26714844,587.88429443)
\curveto(687.21713767,587.87428952)(687.17213772,587.85928954)(687.13214844,587.83929443)
\curveto(686.94213795,587.72928967)(686.79713809,587.58428981)(686.69714844,587.40429443)
\curveto(686.60713828,587.22429017)(686.52713836,587.01929038)(686.45714844,586.78929443)
\curveto(686.41713847,586.65929074)(686.39713849,586.52429087)(686.39714844,586.38429443)
\curveto(686.39713849,586.25429114)(686.3871385,586.10929129)(686.36714844,585.94929443)
\curveto(686.35713853,585.8992915)(686.34713854,585.83929156)(686.33714844,585.76929443)
\curveto(686.33713855,585.6992917)(686.34713854,585.63929176)(686.36714844,585.58929443)
\lineto(686.36714844,585.42429443)
\lineto(686.36714844,585.24429443)
\curveto(686.37713851,585.1942922)(686.3871385,585.13929226)(686.39714844,585.07929443)
\curveto(686.40713848,585.02929237)(686.41213848,584.97429242)(686.41214844,584.91429443)
\curveto(686.42213847,584.85429254)(686.43713845,584.7992926)(686.45714844,584.74929443)
\curveto(686.50713838,584.55929284)(686.56713832,584.38429301)(686.63714844,584.22429443)
\curveto(686.70713818,584.06429333)(686.81213808,583.93429346)(686.95214844,583.83429443)
\curveto(687.08213781,583.73429366)(687.22213767,583.66429373)(687.37214844,583.62429443)
\curveto(687.40213749,583.61429378)(687.42713746,583.60929379)(687.44714844,583.60929443)
\curveto(687.47713741,583.61929378)(687.50713738,583.61929378)(687.53714844,583.60929443)
\curveto(687.55713733,583.60929379)(687.5871373,583.60429379)(687.62714844,583.59429443)
\curveto(687.66713722,583.5942938)(687.70213719,583.5992938)(687.73214844,583.60929443)
\curveto(687.77213712,583.61929378)(687.81213708,583.62429377)(687.85214844,583.62429443)
\curveto(687.892137,583.62429377)(687.93213696,583.63429376)(687.97214844,583.65429443)
\curveto(688.21213668,583.73429366)(688.40713648,583.86929353)(688.55714844,584.05929443)
\curveto(688.67713621,584.23929316)(688.76713612,584.44429295)(688.82714844,584.67429443)
\curveto(688.84713604,584.74429265)(688.86213603,584.81429258)(688.87214844,584.88429443)
\curveto(688.88213601,584.96429243)(688.89713599,585.04429235)(688.91714844,585.12429443)
\curveto(688.91713597,585.18429221)(688.92213597,585.22929217)(688.93214844,585.25929443)
\curveto(688.93213596,585.27929212)(688.93213596,585.30429209)(688.93214844,585.33429443)
\curveto(688.93213596,585.37429202)(688.93713595,585.40429199)(688.94714844,585.42429443)
\lineto(688.94714844,585.57429443)
}
}
{
\newrgbcolor{curcolor}{0 0 0}
\pscustom[linestyle=none,fillstyle=solid,fillcolor=curcolor]
{
\newpath
\moveto(454.50789917,500.0128833)
\curveto(454.60789431,500.01287268)(454.70289422,500.00287269)(454.79289917,499.9828833)
\curveto(454.88289404,499.97287272)(454.94789397,499.94287275)(454.98789917,499.8928833)
\curveto(455.04789387,499.81287288)(455.07789384,499.70787299)(455.07789917,499.5778833)
\lineto(455.07789917,499.1878833)
\lineto(455.07789917,497.6878833)
\lineto(455.07789917,491.2978833)
\lineto(455.07789917,490.1278833)
\lineto(455.07789917,489.8128833)
\curveto(455.08789383,489.71288298)(455.07289385,489.63288306)(455.03289917,489.5728833)
\curveto(454.98289394,489.4928832)(454.90789401,489.44288325)(454.80789917,489.4228833)
\curveto(454.7178942,489.41288328)(454.60789431,489.40788329)(454.47789917,489.4078833)
\lineto(454.25289917,489.4078833)
\curveto(454.17289475,489.42788327)(454.10289482,489.44288325)(454.04289917,489.4528833)
\curveto(453.98289494,489.47288322)(453.93289499,489.51288318)(453.89289917,489.5728833)
\curveto(453.85289507,489.63288306)(453.83289509,489.70788299)(453.83289917,489.7978833)
\lineto(453.83289917,490.0978833)
\lineto(453.83289917,491.1928833)
\lineto(453.83289917,496.5328833)
\curveto(453.81289511,496.62287607)(453.79789512,496.697876)(453.78789917,496.7578833)
\curveto(453.78789513,496.82787587)(453.75789516,496.88787581)(453.69789917,496.9378833)
\curveto(453.62789529,496.98787571)(453.53789538,497.01287568)(453.42789917,497.0128833)
\curveto(453.32789559,497.02287567)(453.2178957,497.02787567)(453.09789917,497.0278833)
\lineto(451.95789917,497.0278833)
\lineto(451.46289917,497.0278833)
\curveto(451.30289762,497.03787566)(451.19289773,497.0978756)(451.13289917,497.2078833)
\curveto(451.11289781,497.23787546)(451.10289782,497.26787543)(451.10289917,497.2978833)
\curveto(451.10289782,497.33787536)(451.09789782,497.38287531)(451.08789917,497.4328833)
\curveto(451.06789785,497.55287514)(451.07289785,497.66287503)(451.10289917,497.7628833)
\curveto(451.14289778,497.86287483)(451.19789772,497.93287476)(451.26789917,497.9728833)
\curveto(451.34789757,498.02287467)(451.46789745,498.04787465)(451.62789917,498.0478833)
\curveto(451.78789713,498.04787465)(451.922897,498.06287463)(452.03289917,498.0928833)
\curveto(452.08289684,498.10287459)(452.13789678,498.10787459)(452.19789917,498.1078833)
\curveto(452.25789666,498.11787458)(452.3178966,498.13287456)(452.37789917,498.1528833)
\curveto(452.52789639,498.20287449)(452.67289625,498.25287444)(452.81289917,498.3028833)
\curveto(452.95289597,498.36287433)(453.08789583,498.43287426)(453.21789917,498.5128833)
\curveto(453.35789556,498.60287409)(453.47789544,498.70787399)(453.57789917,498.8278833)
\curveto(453.67789524,498.94787375)(453.77289515,499.07787362)(453.86289917,499.2178833)
\curveto(453.922895,499.31787338)(453.96789495,499.42787327)(453.99789917,499.5478833)
\curveto(454.03789488,499.66787303)(454.08789483,499.77287292)(454.14789917,499.8628833)
\curveto(454.19789472,499.92287277)(454.26789465,499.96287273)(454.35789917,499.9828833)
\curveto(454.37789454,499.9928727)(454.40289452,499.9978727)(454.43289917,499.9978833)
\curveto(454.46289446,499.9978727)(454.48789443,500.00287269)(454.50789917,500.0128833)
}
}
{
\newrgbcolor{curcolor}{0 0 0}
\pscustom[linestyle=none,fillstyle=solid,fillcolor=curcolor]
{
\newpath
\moveto(465.79750854,494.4928833)
\lineto(465.79750854,494.2378833)
\curveto(465.80750084,494.15787854)(465.80250084,494.08287861)(465.78250854,494.0128833)
\lineto(465.78250854,493.7728833)
\lineto(465.78250854,493.6078833)
\curveto(465.76250088,493.50787919)(465.75250089,493.40287929)(465.75250854,493.2928833)
\curveto(465.75250089,493.1928795)(465.7425009,493.0928796)(465.72250854,492.9928833)
\lineto(465.72250854,492.8428833)
\curveto(465.69250095,492.70287999)(465.67250097,492.56288013)(465.66250854,492.4228833)
\curveto(465.65250099,492.2928804)(465.62750102,492.16288053)(465.58750854,492.0328833)
\curveto(465.56750108,491.95288074)(465.5475011,491.86788083)(465.52750854,491.7778833)
\lineto(465.46750854,491.5378833)
\lineto(465.34750854,491.2378833)
\curveto(465.31750133,491.14788155)(465.28250136,491.05788164)(465.24250854,490.9678833)
\curveto(465.1425015,490.74788195)(465.00750164,490.53288216)(464.83750854,490.3228833)
\curveto(464.67750197,490.11288258)(464.50250214,489.94288275)(464.31250854,489.8128833)
\curveto(464.26250238,489.77288292)(464.20250244,489.73288296)(464.13250854,489.6928833)
\curveto(464.07250257,489.66288303)(464.01250263,489.62788307)(463.95250854,489.5878833)
\curveto(463.87250277,489.53788316)(463.77750287,489.4978832)(463.66750854,489.4678833)
\curveto(463.55750309,489.43788326)(463.45250319,489.40788329)(463.35250854,489.3778833)
\curveto(463.2425034,489.33788336)(463.13250351,489.31288338)(463.02250854,489.3028833)
\curveto(462.91250373,489.2928834)(462.79750385,489.27788342)(462.67750854,489.2578833)
\curveto(462.63750401,489.24788345)(462.59250405,489.24788345)(462.54250854,489.2578833)
\curveto(462.50250414,489.25788344)(462.46250418,489.25288344)(462.42250854,489.2428833)
\curveto(462.38250426,489.23288346)(462.32750432,489.22788347)(462.25750854,489.2278833)
\curveto(462.18750446,489.22788347)(462.13750451,489.23288346)(462.10750854,489.2428833)
\curveto(462.05750459,489.26288343)(462.01250463,489.26788343)(461.97250854,489.2578833)
\curveto(461.93250471,489.24788345)(461.89750475,489.24788345)(461.86750854,489.2578833)
\lineto(461.77750854,489.2578833)
\curveto(461.71750493,489.27788342)(461.65250499,489.2928834)(461.58250854,489.3028833)
\curveto(461.52250512,489.30288339)(461.45750519,489.30788339)(461.38750854,489.3178833)
\curveto(461.21750543,489.36788333)(461.05750559,489.41788328)(460.90750854,489.4678833)
\curveto(460.75750589,489.51788318)(460.61250603,489.58288311)(460.47250854,489.6628833)
\curveto(460.42250622,489.70288299)(460.36750628,489.73288296)(460.30750854,489.7528833)
\curveto(460.25750639,489.78288291)(460.20750644,489.81788288)(460.15750854,489.8578833)
\curveto(459.91750673,490.03788266)(459.71750693,490.25788244)(459.55750854,490.5178833)
\curveto(459.39750725,490.77788192)(459.25750739,491.06288163)(459.13750854,491.3728833)
\curveto(459.07750757,491.51288118)(459.03250761,491.65288104)(459.00250854,491.7928833)
\curveto(458.97250767,491.94288075)(458.93750771,492.0978806)(458.89750854,492.2578833)
\curveto(458.87750777,492.36788033)(458.86250778,492.47788022)(458.85250854,492.5878833)
\curveto(458.8425078,492.69788)(458.82750782,492.80787989)(458.80750854,492.9178833)
\curveto(458.79750785,492.95787974)(458.79250785,492.9978797)(458.79250854,493.0378833)
\curveto(458.80250784,493.07787962)(458.80250784,493.11787958)(458.79250854,493.1578833)
\curveto(458.78250786,493.20787949)(458.77750787,493.25787944)(458.77750854,493.3078833)
\lineto(458.77750854,493.4728833)
\curveto(458.75750789,493.52287917)(458.75250789,493.57287912)(458.76250854,493.6228833)
\curveto(458.77250787,493.68287901)(458.77250787,493.73787896)(458.76250854,493.7878833)
\curveto(458.75250789,493.82787887)(458.75250789,493.87287882)(458.76250854,493.9228833)
\curveto(458.77250787,493.97287872)(458.76750788,494.02287867)(458.74750854,494.0728833)
\curveto(458.72750792,494.14287855)(458.72250792,494.21787848)(458.73250854,494.2978833)
\curveto(458.7425079,494.38787831)(458.7475079,494.47287822)(458.74750854,494.5528833)
\curveto(458.7475079,494.64287805)(458.7425079,494.74287795)(458.73250854,494.8528833)
\curveto(458.72250792,494.97287772)(458.72750792,495.07287762)(458.74750854,495.1528833)
\lineto(458.74750854,495.4378833)
\lineto(458.79250854,496.0678833)
\curveto(458.80250784,496.16787653)(458.81250783,496.26287643)(458.82250854,496.3528833)
\lineto(458.85250854,496.6528833)
\curveto(458.87250777,496.70287599)(458.87750777,496.75287594)(458.86750854,496.8028833)
\curveto(458.86750778,496.86287583)(458.87750777,496.91787578)(458.89750854,496.9678833)
\curveto(458.9475077,497.13787556)(458.98750766,497.30287539)(459.01750854,497.4628833)
\curveto(459.0475076,497.63287506)(459.09750755,497.7928749)(459.16750854,497.9428833)
\curveto(459.35750729,498.40287429)(459.57750707,498.77787392)(459.82750854,499.0678833)
\curveto(460.08750656,499.35787334)(460.4475062,499.60287309)(460.90750854,499.8028833)
\curveto(461.03750561,499.85287284)(461.16750548,499.88787281)(461.29750854,499.9078833)
\curveto(461.43750521,499.92787277)(461.57750507,499.95287274)(461.71750854,499.9828833)
\curveto(461.78750486,499.9928727)(461.85250479,499.9978727)(461.91250854,499.9978833)
\curveto(461.97250467,499.9978727)(462.03750461,500.00287269)(462.10750854,500.0128833)
\curveto(462.93750371,500.03287266)(463.60750304,499.88287281)(464.11750854,499.5628833)
\curveto(464.62750202,499.25287344)(465.00750164,498.81287388)(465.25750854,498.2428833)
\curveto(465.30750134,498.12287457)(465.35250129,497.9978747)(465.39250854,497.8678833)
\curveto(465.43250121,497.73787496)(465.47750117,497.60287509)(465.52750854,497.4628833)
\curveto(465.5475011,497.38287531)(465.56250108,497.2978754)(465.57250854,497.2078833)
\lineto(465.63250854,496.9678833)
\curveto(465.66250098,496.85787584)(465.67750097,496.74787595)(465.67750854,496.6378833)
\curveto(465.68750096,496.52787617)(465.70250094,496.41787628)(465.72250854,496.3078833)
\curveto(465.7425009,496.25787644)(465.7475009,496.21287648)(465.73750854,496.1728833)
\curveto(465.73750091,496.13287656)(465.7425009,496.0928766)(465.75250854,496.0528833)
\curveto(465.76250088,496.00287669)(465.76250088,495.94787675)(465.75250854,495.8878833)
\curveto(465.75250089,495.83787686)(465.75750089,495.78787691)(465.76750854,495.7378833)
\lineto(465.76750854,495.6028833)
\curveto(465.78750086,495.54287715)(465.78750086,495.47287722)(465.76750854,495.3928833)
\curveto(465.75750089,495.32287737)(465.76250088,495.25787744)(465.78250854,495.1978833)
\curveto(465.79250085,495.16787753)(465.79750085,495.12787757)(465.79750854,495.0778833)
\lineto(465.79750854,494.9578833)
\lineto(465.79750854,494.4928833)
\moveto(464.25250854,492.1678833)
\curveto(464.35250229,492.48788021)(464.41250223,492.85287984)(464.43250854,493.2628833)
\curveto(464.45250219,493.67287902)(464.46250218,494.08287861)(464.46250854,494.4928833)
\curveto(464.46250218,494.92287777)(464.45250219,495.34287735)(464.43250854,495.7528833)
\curveto(464.41250223,496.16287653)(464.36750228,496.54787615)(464.29750854,496.9078833)
\curveto(464.22750242,497.26787543)(464.11750253,497.58787511)(463.96750854,497.8678833)
\curveto(463.82750282,498.15787454)(463.63250301,498.3928743)(463.38250854,498.5728833)
\curveto(463.22250342,498.68287401)(463.0425036,498.76287393)(462.84250854,498.8128833)
\curveto(462.642504,498.87287382)(462.39750425,498.90287379)(462.10750854,498.9028833)
\curveto(462.08750456,498.88287381)(462.05250459,498.87287382)(462.00250854,498.8728833)
\curveto(461.95250469,498.88287381)(461.91250473,498.88287381)(461.88250854,498.8728833)
\curveto(461.80250484,498.85287384)(461.72750492,498.83287386)(461.65750854,498.8128833)
\curveto(461.59750505,498.80287389)(461.53250511,498.78287391)(461.46250854,498.7528833)
\curveto(461.19250545,498.63287406)(460.97250567,498.46287423)(460.80250854,498.2428833)
\curveto(460.642506,498.03287466)(460.50750614,497.78787491)(460.39750854,497.5078833)
\curveto(460.3475063,497.3978753)(460.30750634,497.27787542)(460.27750854,497.1478833)
\curveto(460.25750639,497.02787567)(460.23250641,496.90287579)(460.20250854,496.7728833)
\curveto(460.18250646,496.72287597)(460.17250647,496.66787603)(460.17250854,496.6078833)
\curveto(460.17250647,496.55787614)(460.16750648,496.50787619)(460.15750854,496.4578833)
\curveto(460.1475065,496.36787633)(460.13750651,496.27287642)(460.12750854,496.1728833)
\curveto(460.11750653,496.08287661)(460.10750654,495.98787671)(460.09750854,495.8878833)
\curveto(460.09750655,495.80787689)(460.09250655,495.72287697)(460.08250854,495.6328833)
\lineto(460.08250854,495.3928833)
\lineto(460.08250854,495.2128833)
\curveto(460.07250657,495.18287751)(460.06750658,495.14787755)(460.06750854,495.1078833)
\lineto(460.06750854,494.9728833)
\lineto(460.06750854,494.5228833)
\curveto(460.06750658,494.44287825)(460.06250658,494.35787834)(460.05250854,494.2678833)
\curveto(460.05250659,494.18787851)(460.06250658,494.11287858)(460.08250854,494.0428833)
\lineto(460.08250854,493.7728833)
\curveto(460.08250656,493.75287894)(460.07750657,493.72287897)(460.06750854,493.6828833)
\curveto(460.06750658,493.65287904)(460.07250657,493.62787907)(460.08250854,493.6078833)
\curveto(460.09250655,493.50787919)(460.09750655,493.40787929)(460.09750854,493.3078833)
\curveto(460.10750654,493.21787948)(460.11750653,493.11787958)(460.12750854,493.0078833)
\curveto(460.15750649,492.88787981)(460.17250647,492.76287993)(460.17250854,492.6328833)
\curveto(460.18250646,492.51288018)(460.20750644,492.3978803)(460.24750854,492.2878833)
\curveto(460.32750632,491.98788071)(460.41250623,491.72288097)(460.50250854,491.4928833)
\curveto(460.60250604,491.26288143)(460.7475059,491.04788165)(460.93750854,490.8478833)
\curveto(461.1475055,490.64788205)(461.41250523,490.4978822)(461.73250854,490.3978833)
\curveto(461.77250487,490.37788232)(461.80750484,490.36788233)(461.83750854,490.3678833)
\curveto(461.87750477,490.37788232)(461.92250472,490.37288232)(461.97250854,490.3528833)
\curveto(462.01250463,490.34288235)(462.08250456,490.33288236)(462.18250854,490.3228833)
\curveto(462.29250435,490.31288238)(462.37750427,490.31788238)(462.43750854,490.3378833)
\curveto(462.50750414,490.35788234)(462.57750407,490.36788233)(462.64750854,490.3678833)
\curveto(462.71750393,490.37788232)(462.78250386,490.3928823)(462.84250854,490.4128833)
\curveto(463.0425036,490.47288222)(463.22250342,490.55788214)(463.38250854,490.6678833)
\curveto(463.41250323,490.68788201)(463.43750321,490.70788199)(463.45750854,490.7278833)
\lineto(463.51750854,490.7878833)
\curveto(463.55750309,490.80788189)(463.60750304,490.84788185)(463.66750854,490.9078833)
\curveto(463.76750288,491.04788165)(463.85250279,491.17788152)(463.92250854,491.2978833)
\curveto(463.99250265,491.41788128)(464.06250258,491.56288113)(464.13250854,491.7328833)
\curveto(464.16250248,491.80288089)(464.18250246,491.87288082)(464.19250854,491.9428833)
\curveto(464.21250243,492.01288068)(464.23250241,492.08788061)(464.25250854,492.1678833)
}
}
{
\newrgbcolor{curcolor}{0 0 0}
\pscustom[linestyle=none,fillstyle=solid,fillcolor=curcolor]
{
\newpath
\moveto(468.10211792,491.0428833)
\lineto(468.40211792,491.0428833)
\curveto(468.51211586,491.05288164)(468.61711575,491.05288164)(468.71711792,491.0428833)
\curveto(468.82711554,491.04288165)(468.92711544,491.03288166)(469.01711792,491.0128833)
\curveto(469.10711526,491.00288169)(469.17711519,490.97788172)(469.22711792,490.9378833)
\curveto(469.24711512,490.91788178)(469.26211511,490.88788181)(469.27211792,490.8478833)
\curveto(469.29211508,490.80788189)(469.31211506,490.76288193)(469.33211792,490.7128833)
\lineto(469.33211792,490.6378833)
\curveto(469.34211503,490.58788211)(469.34211503,490.53288216)(469.33211792,490.4728833)
\lineto(469.33211792,490.3228833)
\lineto(469.33211792,489.8428833)
\curveto(469.33211504,489.67288302)(469.29211508,489.55288314)(469.21211792,489.4828833)
\curveto(469.14211523,489.43288326)(469.05211532,489.40788329)(468.94211792,489.4078833)
\lineto(468.61211792,489.4078833)
\lineto(468.16211792,489.4078833)
\curveto(468.01211636,489.40788329)(467.89711647,489.43788326)(467.81711792,489.4978833)
\curveto(467.77711659,489.52788317)(467.74711662,489.57788312)(467.72711792,489.6478833)
\curveto(467.70711666,489.72788297)(467.69211668,489.81288288)(467.68211792,489.9028833)
\lineto(467.68211792,490.1878833)
\curveto(467.69211668,490.28788241)(467.69711667,490.37288232)(467.69711792,490.4428833)
\lineto(467.69711792,490.6378833)
\curveto(467.69711667,490.697882)(467.70711666,490.75288194)(467.72711792,490.8028833)
\curveto(467.7671166,490.91288178)(467.83711653,490.98288171)(467.93711792,491.0128833)
\curveto(467.9671164,491.01288168)(468.02211635,491.02288167)(468.10211792,491.0428833)
}
}
{
\newrgbcolor{curcolor}{0 0 0}
\pscustom[linestyle=none,fillstyle=solid,fillcolor=curcolor]
{
\newpath
\moveto(478.32227417,492.4828833)
\curveto(478.33226645,492.44288025)(478.33226645,492.3928803)(478.32227417,492.3328833)
\curveto(478.32226646,492.27288042)(478.31726646,492.22288047)(478.30727417,492.1828833)
\curveto(478.30726647,492.14288055)(478.30226648,492.10288059)(478.29227417,492.0628833)
\lineto(478.29227417,491.9578833)
\curveto(478.27226651,491.87788082)(478.25726652,491.7978809)(478.24727417,491.7178833)
\curveto(478.23726654,491.63788106)(478.21726656,491.56288113)(478.18727417,491.4928833)
\curveto(478.16726661,491.41288128)(478.14726663,491.33788136)(478.12727417,491.2678833)
\curveto(478.10726667,491.1978815)(478.0772667,491.12288157)(478.03727417,491.0428833)
\curveto(477.85726692,490.62288207)(477.60226718,490.28288241)(477.27227417,490.0228833)
\curveto(476.94226784,489.76288293)(476.55226823,489.55788314)(476.10227417,489.4078833)
\curveto(475.9822688,489.36788333)(475.85726892,489.34288335)(475.72727417,489.3328833)
\curveto(475.60726917,489.31288338)(475.4822693,489.28788341)(475.35227417,489.2578833)
\curveto(475.29226949,489.24788345)(475.22726955,489.24288345)(475.15727417,489.2428833)
\curveto(475.09726968,489.24288345)(475.03226975,489.23788346)(474.96227417,489.2278833)
\lineto(474.84227417,489.2278833)
\lineto(474.64727417,489.2278833)
\curveto(474.58727019,489.21788348)(474.53227025,489.22288347)(474.48227417,489.2428833)
\curveto(474.41227037,489.26288343)(474.34727043,489.26788343)(474.28727417,489.2578833)
\curveto(474.22727055,489.24788345)(474.16727061,489.25288344)(474.10727417,489.2728833)
\curveto(474.05727072,489.28288341)(474.01227077,489.28788341)(473.97227417,489.2878833)
\curveto(473.93227085,489.28788341)(473.88727089,489.2978834)(473.83727417,489.3178833)
\curveto(473.75727102,489.33788336)(473.6822711,489.35788334)(473.61227417,489.3778833)
\curveto(473.54227124,489.38788331)(473.47227131,489.40288329)(473.40227417,489.4228833)
\curveto(472.92227186,489.5928831)(472.52227226,489.80288289)(472.20227417,490.0528833)
\curveto(471.89227289,490.31288238)(471.64227314,490.66788203)(471.45227417,491.1178833)
\curveto(471.42227336,491.17788152)(471.39727338,491.23788146)(471.37727417,491.2978833)
\curveto(471.36727341,491.36788133)(471.35227343,491.44288125)(471.33227417,491.5228833)
\curveto(471.31227347,491.58288111)(471.29727348,491.64788105)(471.28727417,491.7178833)
\curveto(471.2772735,491.78788091)(471.26227352,491.85788084)(471.24227417,491.9278833)
\curveto(471.23227355,491.97788072)(471.22727355,492.01788068)(471.22727417,492.0478833)
\lineto(471.22727417,492.1678833)
\curveto(471.21727356,492.20788049)(471.20727357,492.25788044)(471.19727417,492.3178833)
\curveto(471.19727358,492.37788032)(471.20227358,492.42788027)(471.21227417,492.4678833)
\lineto(471.21227417,492.6028833)
\curveto(471.22227356,492.65288004)(471.22727355,492.70287999)(471.22727417,492.7528833)
\curveto(471.24727353,492.85287984)(471.26227352,492.94787975)(471.27227417,493.0378833)
\curveto(471.2822735,493.13787956)(471.30227348,493.23287946)(471.33227417,493.3228833)
\curveto(471.3822734,493.47287922)(471.43727334,493.61287908)(471.49727417,493.7428833)
\curveto(471.55727322,493.87287882)(471.62727315,493.9928787)(471.70727417,494.1028833)
\curveto(471.73727304,494.15287854)(471.76727301,494.1928785)(471.79727417,494.2228833)
\curveto(471.83727294,494.25287844)(471.87227291,494.28787841)(471.90227417,494.3278833)
\curveto(471.96227282,494.40787829)(472.03227275,494.47787822)(472.11227417,494.5378833)
\curveto(472.17227261,494.58787811)(472.23227255,494.63287806)(472.29227417,494.6728833)
\lineto(472.50227417,494.8228833)
\curveto(472.55227223,494.86287783)(472.60227218,494.8978778)(472.65227417,494.9278833)
\curveto(472.70227208,494.96787773)(472.73727204,495.02287767)(472.75727417,495.0928833)
\curveto(472.75727202,495.12287757)(472.74727203,495.14787755)(472.72727417,495.1678833)
\curveto(472.71727206,495.1978775)(472.70727207,495.22287747)(472.69727417,495.2428833)
\curveto(472.65727212,495.2928774)(472.60727217,495.33787736)(472.54727417,495.3778833)
\curveto(472.49727228,495.42787727)(472.44727233,495.47287722)(472.39727417,495.5128833)
\curveto(472.35727242,495.54287715)(472.30727247,495.5978771)(472.24727417,495.6778833)
\curveto(472.22727255,495.70787699)(472.19727258,495.73287696)(472.15727417,495.7528833)
\curveto(472.12727265,495.78287691)(472.10227268,495.81787688)(472.08227417,495.8578833)
\curveto(471.91227287,496.06787663)(471.782273,496.31287638)(471.69227417,496.5928833)
\curveto(471.67227311,496.67287602)(471.65727312,496.75287594)(471.64727417,496.8328833)
\curveto(471.63727314,496.91287578)(471.62227316,496.9928757)(471.60227417,497.0728833)
\curveto(471.5822732,497.12287557)(471.57227321,497.18787551)(471.57227417,497.2678833)
\curveto(471.57227321,497.35787534)(471.5822732,497.42787527)(471.60227417,497.4778833)
\curveto(471.60227318,497.57787512)(471.60727317,497.64787505)(471.61727417,497.6878833)
\curveto(471.63727314,497.76787493)(471.65227313,497.83787486)(471.66227417,497.8978833)
\curveto(471.67227311,497.96787473)(471.68727309,498.03787466)(471.70727417,498.1078833)
\curveto(471.85727292,498.53787416)(472.07227271,498.88287381)(472.35227417,499.1428833)
\curveto(472.64227214,499.40287329)(472.99227179,499.61787308)(473.40227417,499.7878833)
\curveto(473.51227127,499.83787286)(473.62727115,499.86787283)(473.74727417,499.8778833)
\curveto(473.8772709,499.8978728)(474.00727077,499.92787277)(474.13727417,499.9678833)
\curveto(474.21727056,499.96787273)(474.28727049,499.96787273)(474.34727417,499.9678833)
\curveto(474.41727036,499.97787272)(474.49227029,499.98787271)(474.57227417,499.9978833)
\curveto(475.36226942,500.01787268)(476.01726876,499.88787281)(476.53727417,499.6078833)
\curveto(477.06726771,499.32787337)(477.44726733,498.91787378)(477.67727417,498.3778833)
\curveto(477.78726699,498.14787455)(477.85726692,497.86287483)(477.88727417,497.5228833)
\curveto(477.92726685,497.1928755)(477.89726688,496.88787581)(477.79727417,496.6078833)
\curveto(477.75726702,496.47787622)(477.70726707,496.35787634)(477.64727417,496.2478833)
\curveto(477.59726718,496.13787656)(477.53726724,496.03287666)(477.46727417,495.9328833)
\curveto(477.44726733,495.8928768)(477.41726736,495.85787684)(477.37727417,495.8278833)
\lineto(477.28727417,495.7378833)
\curveto(477.23726754,495.64787705)(477.1772676,495.58287711)(477.10727417,495.5428833)
\curveto(477.05726772,495.4928772)(477.00226778,495.44287725)(476.94227417,495.3928833)
\curveto(476.89226789,495.35287734)(476.84726793,495.30787739)(476.80727417,495.2578833)
\curveto(476.78726799,495.23787746)(476.76726801,495.21287748)(476.74727417,495.1828833)
\curveto(476.73726804,495.16287753)(476.73726804,495.13787756)(476.74727417,495.1078833)
\curveto(476.75726802,495.05787764)(476.78726799,495.00787769)(476.83727417,494.9578833)
\curveto(476.88726789,494.91787778)(476.94226784,494.87787782)(477.00227417,494.8378833)
\lineto(477.18227417,494.7178833)
\curveto(477.24226754,494.68787801)(477.29226749,494.65787804)(477.33227417,494.6278833)
\curveto(477.66226712,494.38787831)(477.91226687,494.07787862)(478.08227417,493.6978833)
\curveto(478.12226666,493.61787908)(478.15226663,493.53287916)(478.17227417,493.4428833)
\curveto(478.20226658,493.35287934)(478.22726655,493.26287943)(478.24727417,493.1728833)
\curveto(478.25726652,493.12287957)(478.26726651,493.06787963)(478.27727417,493.0078833)
\lineto(478.30727417,492.8578833)
\curveto(478.31726646,492.7978799)(478.31726646,492.73287996)(478.30727417,492.6628833)
\curveto(478.29726648,492.60288009)(478.30226648,492.54288015)(478.32227417,492.4828833)
\moveto(472.93727417,497.5228833)
\curveto(472.90727187,497.41287528)(472.90227188,497.27287542)(472.92227417,497.1028833)
\curveto(472.94227184,496.94287575)(472.96727181,496.81787588)(472.99727417,496.7278833)
\curveto(473.10727167,496.40787629)(473.25727152,496.16287653)(473.44727417,495.9928833)
\curveto(473.63727114,495.83287686)(473.90227088,495.70287699)(474.24227417,495.6028833)
\curveto(474.37227041,495.57287712)(474.53727024,495.54787715)(474.73727417,495.5278833)
\curveto(474.93726984,495.51787718)(475.10726967,495.53287716)(475.24727417,495.5728833)
\curveto(475.53726924,495.65287704)(475.777269,495.76287693)(475.96727417,495.9028833)
\curveto(476.16726861,496.05287664)(476.32226846,496.25287644)(476.43227417,496.5028833)
\curveto(476.45226833,496.55287614)(476.46226832,496.5978761)(476.46227417,496.6378833)
\curveto(476.47226831,496.67787602)(476.48726829,496.72287597)(476.50727417,496.7728833)
\curveto(476.53726824,496.88287581)(476.55726822,497.02287567)(476.56727417,497.1928833)
\curveto(476.5772682,497.36287533)(476.56726821,497.50787519)(476.53727417,497.6278833)
\curveto(476.51726826,497.71787498)(476.49226829,497.80287489)(476.46227417,497.8828833)
\curveto(476.44226834,497.96287473)(476.40726837,498.04287465)(476.35727417,498.1228833)
\curveto(476.18726859,498.3928743)(475.96226882,498.58787411)(475.68227417,498.7078833)
\curveto(475.41226937,498.82787387)(475.05226973,498.88787381)(474.60227417,498.8878833)
\curveto(474.5822702,498.86787383)(474.55227023,498.86287383)(474.51227417,498.8728833)
\curveto(474.47227031,498.88287381)(474.43727034,498.88287381)(474.40727417,498.8728833)
\curveto(474.35727042,498.85287384)(474.30227048,498.83787386)(474.24227417,498.8278833)
\curveto(474.19227059,498.82787387)(474.14227064,498.81787388)(474.09227417,498.7978833)
\curveto(473.85227093,498.70787399)(473.64227114,498.5928741)(473.46227417,498.4528833)
\curveto(473.2822715,498.32287437)(473.14227164,498.14287455)(473.04227417,497.9128833)
\curveto(473.02227176,497.85287484)(473.00227178,497.78787491)(472.98227417,497.7178833)
\curveto(472.97227181,497.65787504)(472.95727182,497.5928751)(472.93727417,497.5228833)
\moveto(476.95727417,491.9878833)
\curveto(477.00726777,492.17788052)(477.01226777,492.38288031)(476.97227417,492.6028833)
\curveto(476.94226784,492.82287987)(476.89726788,493.00287969)(476.83727417,493.1428833)
\curveto(476.66726811,493.51287918)(476.40726837,493.81787888)(476.05727417,494.0578833)
\curveto(475.71726906,494.2978784)(475.2822695,494.41787828)(474.75227417,494.4178833)
\curveto(474.72227006,494.3978783)(474.6822701,494.3928783)(474.63227417,494.4028833)
\curveto(474.5822702,494.42287827)(474.54227024,494.42787827)(474.51227417,494.4178833)
\lineto(474.24227417,494.3578833)
\curveto(474.16227062,494.34787835)(474.0822707,494.33287836)(474.00227417,494.3128833)
\curveto(473.70227108,494.20287849)(473.43727134,494.05787864)(473.20727417,493.8778833)
\curveto(472.98727179,493.697879)(472.81727196,493.46787923)(472.69727417,493.1878833)
\curveto(472.66727211,493.10787959)(472.64227214,493.02787967)(472.62227417,492.9478833)
\curveto(472.60227218,492.86787983)(472.5822722,492.78287991)(472.56227417,492.6928833)
\curveto(472.53227225,492.57288012)(472.52227226,492.42288027)(472.53227417,492.2428833)
\curveto(472.55227223,492.06288063)(472.5772722,491.92288077)(472.60727417,491.8228833)
\curveto(472.62727215,491.77288092)(472.63727214,491.72788097)(472.63727417,491.6878833)
\curveto(472.64727213,491.65788104)(472.66227212,491.61788108)(472.68227417,491.5678833)
\curveto(472.782272,491.34788135)(472.91227187,491.14788155)(473.07227417,490.9678833)
\curveto(473.24227154,490.78788191)(473.43727134,490.65288204)(473.65727417,490.5628833)
\curveto(473.72727105,490.52288217)(473.82227096,490.48788221)(473.94227417,490.4578833)
\curveto(474.16227062,490.36788233)(474.41727036,490.32288237)(474.70727417,490.3228833)
\lineto(474.99227417,490.3228833)
\curveto(475.09226969,490.34288235)(475.18726959,490.35788234)(475.27727417,490.3678833)
\curveto(475.36726941,490.37788232)(475.45726932,490.3978823)(475.54727417,490.4278833)
\curveto(475.80726897,490.50788219)(476.04726873,490.63788206)(476.26727417,490.8178833)
\curveto(476.49726828,491.00788169)(476.66726811,491.22288147)(476.77727417,491.4628833)
\curveto(476.81726796,491.54288115)(476.84726793,491.62288107)(476.86727417,491.7028833)
\curveto(476.89726788,491.7928809)(476.92726785,491.88788081)(476.95727417,491.9878833)
}
}
{
\newrgbcolor{curcolor}{0 0 0}
\pscustom[linestyle=none,fillstyle=solid,fillcolor=curcolor]
{
\newpath
\moveto(489.46188354,497.9278833)
\curveto(489.26187324,497.63787506)(489.05187345,497.35287534)(488.83188354,497.0728833)
\curveto(488.62187388,496.7928759)(488.41687409,496.50787619)(488.21688354,496.2178833)
\curveto(487.61687489,495.36787733)(487.01187549,494.52787817)(486.40188354,493.6978833)
\curveto(485.79187671,492.87787982)(485.18687732,492.04288065)(484.58688354,491.1928833)
\lineto(484.07688354,490.4728833)
\lineto(483.56688354,489.7828833)
\curveto(483.48687902,489.67288302)(483.4068791,489.55788314)(483.32688354,489.4378833)
\curveto(483.24687926,489.31788338)(483.15187935,489.22288347)(483.04188354,489.1528833)
\curveto(483.0018795,489.13288356)(482.93687957,489.11788358)(482.84688354,489.1078833)
\curveto(482.76687974,489.08788361)(482.67687983,489.07788362)(482.57688354,489.0778833)
\curveto(482.47688003,489.07788362)(482.38188012,489.08288361)(482.29188354,489.0928833)
\curveto(482.21188029,489.10288359)(482.15188035,489.12288357)(482.11188354,489.1528833)
\curveto(482.08188042,489.17288352)(482.05688045,489.20788349)(482.03688354,489.2578833)
\curveto(482.02688048,489.2978834)(482.03188047,489.34288335)(482.05188354,489.3928833)
\curveto(482.09188041,489.47288322)(482.13688037,489.54788315)(482.18688354,489.6178833)
\curveto(482.24688026,489.697883)(482.3018802,489.77788292)(482.35188354,489.8578833)
\curveto(482.59187991,490.1978825)(482.83687967,490.53288216)(483.08688354,490.8628833)
\curveto(483.33687917,491.1928815)(483.57687893,491.52788117)(483.80688354,491.8678833)
\curveto(483.96687854,492.08788061)(484.12687838,492.30288039)(484.28688354,492.5128833)
\curveto(484.44687806,492.72287997)(484.6068779,492.93787976)(484.76688354,493.1578833)
\curveto(485.12687738,493.67787902)(485.49187701,494.18787851)(485.86188354,494.6878833)
\curveto(486.23187627,495.18787751)(486.6018759,495.697877)(486.97188354,496.2178833)
\curveto(487.11187539,496.41787628)(487.25187525,496.61287608)(487.39188354,496.8028833)
\curveto(487.54187496,496.9928757)(487.68687482,497.18787551)(487.82688354,497.3878833)
\curveto(488.03687447,497.68787501)(488.25187425,497.98787471)(488.47188354,498.2878833)
\lineto(489.13188354,499.1878833)
\lineto(489.31188354,499.4578833)
\lineto(489.52188354,499.7278833)
\lineto(489.64188354,499.9078833)
\curveto(489.69187281,499.96787273)(489.74187276,500.02287267)(489.79188354,500.0728833)
\curveto(489.86187264,500.12287257)(489.93687257,500.15787254)(490.01688354,500.1778833)
\curveto(490.03687247,500.18787251)(490.06187244,500.18787251)(490.09188354,500.1778833)
\curveto(490.13187237,500.17787252)(490.16187234,500.18787251)(490.18188354,500.2078833)
\curveto(490.3018722,500.20787249)(490.43687207,500.20287249)(490.58688354,500.1928833)
\curveto(490.73687177,500.1928725)(490.82687168,500.14787255)(490.85688354,500.0578833)
\curveto(490.87687163,500.02787267)(490.88187162,499.9928727)(490.87188354,499.9528833)
\curveto(490.86187164,499.91287278)(490.84687166,499.88287281)(490.82688354,499.8628833)
\curveto(490.78687172,499.78287291)(490.74687176,499.71287298)(490.70688354,499.6528833)
\curveto(490.66687184,499.5928731)(490.62187188,499.53287316)(490.57188354,499.4728833)
\lineto(490.00188354,498.6928833)
\curveto(489.82187268,498.44287425)(489.64187286,498.18787451)(489.46188354,497.9278833)
\moveto(482.60688354,494.0278833)
\curveto(482.55687995,494.04787865)(482.50688,494.05287864)(482.45688354,494.0428833)
\curveto(482.4068801,494.03287866)(482.35688015,494.03787866)(482.30688354,494.0578833)
\curveto(482.19688031,494.07787862)(482.09188041,494.0978786)(481.99188354,494.1178833)
\curveto(481.9018806,494.14787855)(481.8068807,494.18787851)(481.70688354,494.2378833)
\curveto(481.37688113,494.37787832)(481.12188138,494.57287812)(480.94188354,494.8228833)
\curveto(480.76188174,495.08287761)(480.61688189,495.3928773)(480.50688354,495.7528833)
\curveto(480.47688203,495.83287686)(480.45688205,495.91287678)(480.44688354,495.9928833)
\curveto(480.43688207,496.08287661)(480.42188208,496.16787653)(480.40188354,496.2478833)
\curveto(480.39188211,496.2978764)(480.38688212,496.36287633)(480.38688354,496.4428833)
\curveto(480.37688213,496.47287622)(480.37188213,496.50287619)(480.37188354,496.5328833)
\curveto(480.37188213,496.57287612)(480.36688214,496.60787609)(480.35688354,496.6378833)
\lineto(480.35688354,496.7878833)
\curveto(480.34688216,496.83787586)(480.34188216,496.8978758)(480.34188354,496.9678833)
\curveto(480.34188216,497.04787565)(480.34688216,497.11287558)(480.35688354,497.1628833)
\lineto(480.35688354,497.3278833)
\curveto(480.37688213,497.37787532)(480.38188212,497.42287527)(480.37188354,497.4628833)
\curveto(480.37188213,497.51287518)(480.37688213,497.55787514)(480.38688354,497.5978833)
\curveto(480.39688211,497.63787506)(480.4018821,497.67287502)(480.40188354,497.7028833)
\curveto(480.4018821,497.74287495)(480.4068821,497.78287491)(480.41688354,497.8228833)
\curveto(480.44688206,497.93287476)(480.46688204,498.04287465)(480.47688354,498.1528833)
\curveto(480.49688201,498.27287442)(480.53188197,498.38787431)(480.58188354,498.4978833)
\curveto(480.72188178,498.83787386)(480.88188162,499.11287358)(481.06188354,499.3228833)
\curveto(481.25188125,499.54287315)(481.52188098,499.72287297)(481.87188354,499.8628833)
\curveto(481.95188055,499.8928728)(482.03688047,499.91287278)(482.12688354,499.9228833)
\curveto(482.21688029,499.94287275)(482.31188019,499.96287273)(482.41188354,499.9828833)
\curveto(482.44188006,499.9928727)(482.49688001,499.9928727)(482.57688354,499.9828833)
\curveto(482.65687985,499.98287271)(482.7068798,499.9928727)(482.72688354,500.0128833)
\curveto(483.28687922,500.02287267)(483.73687877,499.91287278)(484.07688354,499.6828833)
\curveto(484.42687808,499.45287324)(484.68687782,499.14787355)(484.85688354,498.7678833)
\curveto(484.89687761,498.67787402)(484.93187757,498.58287411)(484.96188354,498.4828833)
\curveto(484.99187751,498.38287431)(485.01687749,498.28287441)(485.03688354,498.1828833)
\curveto(485.05687745,498.15287454)(485.06187744,498.12287457)(485.05188354,498.0928833)
\curveto(485.05187745,498.06287463)(485.05687745,498.03287466)(485.06688354,498.0028833)
\curveto(485.09687741,497.8928748)(485.11687739,497.76787493)(485.12688354,497.6278833)
\curveto(485.13687737,497.4978752)(485.14687736,497.36287533)(485.15688354,497.2228833)
\lineto(485.15688354,497.0578833)
\curveto(485.16687734,496.9978757)(485.16687734,496.94287575)(485.15688354,496.8928833)
\curveto(485.14687736,496.84287585)(485.14187736,496.7928759)(485.14188354,496.7428833)
\lineto(485.14188354,496.6078833)
\curveto(485.13187737,496.56787613)(485.12687738,496.52787617)(485.12688354,496.4878833)
\curveto(485.13687737,496.44787625)(485.13187737,496.40287629)(485.11188354,496.3528833)
\curveto(485.09187741,496.24287645)(485.07187743,496.13787656)(485.05188354,496.0378833)
\curveto(485.04187746,495.93787676)(485.02187748,495.83787686)(484.99188354,495.7378833)
\curveto(484.86187764,495.37787732)(484.69687781,495.06287763)(484.49688354,494.7928833)
\curveto(484.29687821,494.52287817)(484.02187848,494.31787838)(483.67188354,494.1778833)
\curveto(483.59187891,494.14787855)(483.506879,494.12287857)(483.41688354,494.1028833)
\lineto(483.14688354,494.0428833)
\curveto(483.09687941,494.03287866)(483.05187945,494.02787867)(483.01188354,494.0278833)
\curveto(482.97187953,494.03787866)(482.93187957,494.03787866)(482.89188354,494.0278833)
\curveto(482.79187971,494.00787869)(482.69687981,494.00787869)(482.60688354,494.0278833)
\moveto(481.76688354,495.4228833)
\curveto(481.8068807,495.35287734)(481.84688066,495.28787741)(481.88688354,495.2278833)
\curveto(481.92688058,495.17787752)(481.97688053,495.12787757)(482.03688354,495.0778833)
\lineto(482.18688354,494.9578833)
\curveto(482.24688026,494.92787777)(482.31188019,494.90287779)(482.38188354,494.8828833)
\curveto(482.42188008,494.86287783)(482.45688005,494.85287784)(482.48688354,494.8528833)
\curveto(482.52687998,494.86287783)(482.56687994,494.85787784)(482.60688354,494.8378833)
\curveto(482.63687987,494.83787786)(482.67687983,494.83287786)(482.72688354,494.8228833)
\curveto(482.77687973,494.82287787)(482.81687969,494.82787787)(482.84688354,494.8378833)
\lineto(483.07188354,494.8828833)
\curveto(483.32187918,494.96287773)(483.506879,495.08787761)(483.62688354,495.2578833)
\curveto(483.7068788,495.35787734)(483.77687873,495.48787721)(483.83688354,495.6478833)
\curveto(483.91687859,495.82787687)(483.97687853,496.05287664)(484.01688354,496.3228833)
\curveto(484.05687845,496.60287609)(484.07187843,496.88287581)(484.06188354,497.1628833)
\curveto(484.05187845,497.45287524)(484.02187848,497.72787497)(483.97188354,497.9878833)
\curveto(483.92187858,498.24787445)(483.84687866,498.45787424)(483.74688354,498.6178833)
\curveto(483.62687888,498.81787388)(483.47687903,498.96787373)(483.29688354,499.0678833)
\curveto(483.21687929,499.11787358)(483.12687938,499.14787355)(483.02688354,499.1578833)
\curveto(482.92687958,499.17787352)(482.82187968,499.18787351)(482.71188354,499.1878833)
\curveto(482.69187981,499.17787352)(482.66687984,499.17287352)(482.63688354,499.1728833)
\curveto(482.61687989,499.18287351)(482.59687991,499.18287351)(482.57688354,499.1728833)
\curveto(482.52687998,499.16287353)(482.48188002,499.15287354)(482.44188354,499.1428833)
\curveto(482.4018801,499.14287355)(482.36188014,499.13287356)(482.32188354,499.1128833)
\curveto(482.14188036,499.03287366)(481.99188051,498.91287378)(481.87188354,498.7528833)
\curveto(481.76188074,498.5928741)(481.67188083,498.41287428)(481.60188354,498.2128833)
\curveto(481.54188096,498.02287467)(481.49688101,497.7978749)(481.46688354,497.5378833)
\curveto(481.44688106,497.27787542)(481.44188106,497.01287568)(481.45188354,496.7428833)
\curveto(481.46188104,496.48287621)(481.49188101,496.23287646)(481.54188354,495.9928833)
\curveto(481.6018809,495.76287693)(481.67688083,495.57287712)(481.76688354,495.4228833)
\moveto(492.56688354,492.4378833)
\curveto(492.57686993,492.38788031)(492.58186992,492.2978804)(492.58188354,492.1678833)
\curveto(492.58186992,492.03788066)(492.57186993,491.94788075)(492.55188354,491.8978833)
\curveto(492.53186997,491.84788085)(492.52686998,491.7928809)(492.53688354,491.7328833)
\curveto(492.54686996,491.68288101)(492.54686996,491.63288106)(492.53688354,491.5828833)
\curveto(492.49687001,491.44288125)(492.46687004,491.30788139)(492.44688354,491.1778833)
\curveto(492.43687007,491.04788165)(492.4068701,490.92788177)(492.35688354,490.8178833)
\curveto(492.21687029,490.46788223)(492.05187045,490.17288252)(491.86188354,489.9328833)
\curveto(491.67187083,489.70288299)(491.4018711,489.51788318)(491.05188354,489.3778833)
\curveto(490.97187153,489.34788335)(490.88687162,489.32788337)(490.79688354,489.3178833)
\curveto(490.7068718,489.2978834)(490.62187188,489.27788342)(490.54188354,489.2578833)
\curveto(490.49187201,489.24788345)(490.44187206,489.24288345)(490.39188354,489.2428833)
\curveto(490.34187216,489.24288345)(490.29187221,489.23788346)(490.24188354,489.2278833)
\curveto(490.21187229,489.21788348)(490.16187234,489.21788348)(490.09188354,489.2278833)
\curveto(490.02187248,489.22788347)(489.97187253,489.23288346)(489.94188354,489.2428833)
\curveto(489.88187262,489.26288343)(489.82187268,489.27288342)(489.76188354,489.2728833)
\curveto(489.71187279,489.26288343)(489.66187284,489.26788343)(489.61188354,489.2878833)
\curveto(489.52187298,489.30788339)(489.43187307,489.33288336)(489.34188354,489.3628833)
\curveto(489.26187324,489.38288331)(489.18187332,489.41288328)(489.10188354,489.4528833)
\curveto(488.78187372,489.5928831)(488.53187397,489.78788291)(488.35188354,490.0378833)
\curveto(488.17187433,490.2978824)(488.02187448,490.60288209)(487.90188354,490.9528833)
\curveto(487.88187462,491.03288166)(487.86687464,491.11788158)(487.85688354,491.2078833)
\curveto(487.84687466,491.2978814)(487.83187467,491.38288131)(487.81188354,491.4628833)
\curveto(487.8018747,491.4928812)(487.79687471,491.52288117)(487.79688354,491.5528833)
\lineto(487.79688354,491.6578833)
\curveto(487.77687473,491.73788096)(487.76687474,491.81788088)(487.76688354,491.8978833)
\lineto(487.76688354,492.0328833)
\curveto(487.74687476,492.13288056)(487.74687476,492.23288046)(487.76688354,492.3328833)
\lineto(487.76688354,492.5128833)
\curveto(487.77687473,492.56288013)(487.78187472,492.60788009)(487.78188354,492.6478833)
\curveto(487.78187472,492.69788)(487.78687472,492.74287995)(487.79688354,492.7828833)
\curveto(487.8068747,492.82287987)(487.81187469,492.85787984)(487.81188354,492.8878833)
\curveto(487.81187469,492.92787977)(487.81687469,492.96787973)(487.82688354,493.0078833)
\lineto(487.88688354,493.3378833)
\curveto(487.9068746,493.45787924)(487.93687457,493.56787913)(487.97688354,493.6678833)
\curveto(488.11687439,493.9978787)(488.27687423,494.27287842)(488.45688354,494.4928833)
\curveto(488.64687386,494.72287797)(488.9068736,494.90787779)(489.23688354,495.0478833)
\curveto(489.31687319,495.08787761)(489.4018731,495.11287758)(489.49188354,495.1228833)
\lineto(489.79188354,495.1828833)
\lineto(489.92688354,495.1828833)
\curveto(489.97687253,495.1928775)(490.02687248,495.1978775)(490.07688354,495.1978833)
\curveto(490.64687186,495.21787748)(491.1068714,495.11287758)(491.45688354,494.8828833)
\curveto(491.81687069,494.66287803)(492.08187042,494.36287833)(492.25188354,493.9828833)
\curveto(492.3018702,493.88287881)(492.34187016,493.78287891)(492.37188354,493.6828833)
\curveto(492.4018701,493.58287911)(492.43187007,493.47787922)(492.46188354,493.3678833)
\curveto(492.47187003,493.32787937)(492.47687003,493.2928794)(492.47688354,493.2628833)
\curveto(492.47687003,493.24287945)(492.48187002,493.21287948)(492.49188354,493.1728833)
\curveto(492.51186999,493.10287959)(492.52186998,493.02787967)(492.52188354,492.9478833)
\curveto(492.52186998,492.86787983)(492.53186997,492.78787991)(492.55188354,492.7078833)
\curveto(492.55186995,492.65788004)(492.55186995,492.61288008)(492.55188354,492.5728833)
\curveto(492.55186995,492.53288016)(492.55686995,492.48788021)(492.56688354,492.4378833)
\moveto(491.45688354,492.0028833)
\curveto(491.46687104,492.05288064)(491.47187103,492.12788057)(491.47188354,492.2278833)
\curveto(491.48187102,492.32788037)(491.47687103,492.40288029)(491.45688354,492.4528833)
\curveto(491.43687107,492.51288018)(491.43187107,492.56788013)(491.44188354,492.6178833)
\curveto(491.46187104,492.67788002)(491.46187104,492.73787996)(491.44188354,492.7978833)
\curveto(491.43187107,492.82787987)(491.42687108,492.86287983)(491.42688354,492.9028833)
\curveto(491.42687108,492.94287975)(491.42187108,492.98287971)(491.41188354,493.0228833)
\curveto(491.39187111,493.10287959)(491.37187113,493.17787952)(491.35188354,493.2478833)
\curveto(491.34187116,493.32787937)(491.32687118,493.40787929)(491.30688354,493.4878833)
\curveto(491.27687123,493.54787915)(491.25187125,493.60787909)(491.23188354,493.6678833)
\curveto(491.21187129,493.72787897)(491.18187132,493.78787891)(491.14188354,493.8478833)
\curveto(491.04187146,494.01787868)(490.91187159,494.15287854)(490.75188354,494.2528833)
\curveto(490.67187183,494.30287839)(490.57687193,494.33787836)(490.46688354,494.3578833)
\curveto(490.35687215,494.37787832)(490.23187227,494.38787831)(490.09188354,494.3878833)
\curveto(490.07187243,494.37787832)(490.04687246,494.37287832)(490.01688354,494.3728833)
\curveto(489.98687252,494.38287831)(489.95687255,494.38287831)(489.92688354,494.3728833)
\lineto(489.77688354,494.3128833)
\curveto(489.72687278,494.30287839)(489.68187282,494.28787841)(489.64188354,494.2678833)
\curveto(489.45187305,494.15787854)(489.3068732,494.01287868)(489.20688354,493.8328833)
\curveto(489.11687339,493.65287904)(489.03687347,493.44787925)(488.96688354,493.2178833)
\curveto(488.92687358,493.08787961)(488.9068736,492.95287974)(488.90688354,492.8128833)
\curveto(488.9068736,492.68288001)(488.89687361,492.53788016)(488.87688354,492.3778833)
\curveto(488.86687364,492.32788037)(488.85687365,492.26788043)(488.84688354,492.1978833)
\curveto(488.84687366,492.12788057)(488.85687365,492.06788063)(488.87688354,492.0178833)
\lineto(488.87688354,491.8528833)
\lineto(488.87688354,491.6728833)
\curveto(488.88687362,491.62288107)(488.89687361,491.56788113)(488.90688354,491.5078833)
\curveto(488.91687359,491.45788124)(488.92187358,491.40288129)(488.92188354,491.3428833)
\curveto(488.93187357,491.28288141)(488.94687356,491.22788147)(488.96688354,491.1778833)
\curveto(489.01687349,490.98788171)(489.07687343,490.81288188)(489.14688354,490.6528833)
\curveto(489.21687329,490.4928822)(489.32187318,490.36288233)(489.46188354,490.2628833)
\curveto(489.59187291,490.16288253)(489.73187277,490.0928826)(489.88188354,490.0528833)
\curveto(489.91187259,490.04288265)(489.93687257,490.03788266)(489.95688354,490.0378833)
\curveto(489.98687252,490.04788265)(490.01687249,490.04788265)(490.04688354,490.0378833)
\curveto(490.06687244,490.03788266)(490.09687241,490.03288266)(490.13688354,490.0228833)
\curveto(490.17687233,490.02288267)(490.21187229,490.02788267)(490.24188354,490.0378833)
\curveto(490.28187222,490.04788265)(490.32187218,490.05288264)(490.36188354,490.0528833)
\curveto(490.4018721,490.05288264)(490.44187206,490.06288263)(490.48188354,490.0828833)
\curveto(490.72187178,490.16288253)(490.91687159,490.2978824)(491.06688354,490.4878833)
\curveto(491.18687132,490.66788203)(491.27687123,490.87288182)(491.33688354,491.1028833)
\curveto(491.35687115,491.17288152)(491.37187113,491.24288145)(491.38188354,491.3128833)
\curveto(491.39187111,491.3928813)(491.4068711,491.47288122)(491.42688354,491.5528833)
\curveto(491.42687108,491.61288108)(491.43187107,491.65788104)(491.44188354,491.6878833)
\curveto(491.44187106,491.70788099)(491.44187106,491.73288096)(491.44188354,491.7628833)
\curveto(491.44187106,491.80288089)(491.44687106,491.83288086)(491.45688354,491.8528833)
\lineto(491.45688354,492.0028833)
}
}
{
\newrgbcolor{curcolor}{0 0 0}
\pscustom[linestyle=none,fillstyle=solid,fillcolor=curcolor]
{
\newpath
\moveto(471.03427979,558.91138062)
\curveto(471.04427207,558.87137757)(471.04427207,558.82137762)(471.03427979,558.76138062)
\curveto(471.03427208,558.70137774)(471.02927208,558.65137779)(471.01927979,558.61138062)
\curveto(471.01927209,558.57137787)(471.0142721,558.53137791)(471.00427979,558.49138062)
\lineto(471.00427979,558.38638062)
\curveto(470.98427213,558.30637813)(470.96927214,558.22637821)(470.95927979,558.14638062)
\curveto(470.94927216,558.06637837)(470.92927218,557.99137845)(470.89927979,557.92138062)
\curveto(470.87927223,557.8413786)(470.85927225,557.76637867)(470.83927979,557.69638062)
\curveto(470.81927229,557.62637881)(470.78927232,557.55137889)(470.74927979,557.47138062)
\curveto(470.56927254,557.05137939)(470.3142728,556.71137973)(469.98427979,556.45138062)
\curveto(469.65427346,556.19138025)(469.26427385,555.98638045)(468.81427979,555.83638062)
\curveto(468.69427442,555.79638064)(468.56927454,555.77138067)(468.43927979,555.76138062)
\curveto(468.31927479,555.7413807)(468.19427492,555.71638072)(468.06427979,555.68638062)
\curveto(468.00427511,555.67638076)(467.93927517,555.67138077)(467.86927979,555.67138062)
\curveto(467.8092753,555.67138077)(467.74427537,555.66638077)(467.67427979,555.65638062)
\lineto(467.55427979,555.65638062)
\lineto(467.35927979,555.65638062)
\curveto(467.29927581,555.64638079)(467.24427587,555.65138079)(467.19427979,555.67138062)
\curveto(467.12427599,555.69138075)(467.05927605,555.69638074)(466.99927979,555.68638062)
\curveto(466.93927617,555.67638076)(466.87927623,555.68138076)(466.81927979,555.70138062)
\curveto(466.76927634,555.71138073)(466.72427639,555.71638072)(466.68427979,555.71638062)
\curveto(466.64427647,555.71638072)(466.59927651,555.72638071)(466.54927979,555.74638062)
\curveto(466.46927664,555.76638067)(466.39427672,555.78638065)(466.32427979,555.80638062)
\curveto(466.25427686,555.81638062)(466.18427693,555.83138061)(466.11427979,555.85138062)
\curveto(465.63427748,556.02138042)(465.23427788,556.23138021)(464.91427979,556.48138062)
\curveto(464.60427851,556.7413797)(464.35427876,557.09637934)(464.16427979,557.54638062)
\curveto(464.13427898,557.60637883)(464.109279,557.66637877)(464.08927979,557.72638062)
\curveto(464.07927903,557.79637864)(464.06427905,557.87137857)(464.04427979,557.95138062)
\curveto(464.02427909,558.01137843)(464.0092791,558.07637836)(463.99927979,558.14638062)
\curveto(463.98927912,558.21637822)(463.97427914,558.28637815)(463.95427979,558.35638062)
\curveto(463.94427917,558.40637803)(463.93927917,558.44637799)(463.93927979,558.47638062)
\lineto(463.93927979,558.59638062)
\curveto(463.92927918,558.6363778)(463.91927919,558.68637775)(463.90927979,558.74638062)
\curveto(463.9092792,558.80637763)(463.9142792,558.85637758)(463.92427979,558.89638062)
\lineto(463.92427979,559.03138062)
\curveto(463.93427918,559.08137736)(463.93927917,559.13137731)(463.93927979,559.18138062)
\curveto(463.95927915,559.28137716)(463.97427914,559.37637706)(463.98427979,559.46638062)
\curveto(463.99427912,559.56637687)(464.0142791,559.66137678)(464.04427979,559.75138062)
\curveto(464.09427902,559.90137654)(464.14927896,560.0413764)(464.20927979,560.17138062)
\curveto(464.26927884,560.30137614)(464.33927877,560.42137602)(464.41927979,560.53138062)
\curveto(464.44927866,560.58137586)(464.47927863,560.62137582)(464.50927979,560.65138062)
\curveto(464.54927856,560.68137576)(464.58427853,560.71637572)(464.61427979,560.75638062)
\curveto(464.67427844,560.8363756)(464.74427837,560.90637553)(464.82427979,560.96638062)
\curveto(464.88427823,561.01637542)(464.94427817,561.06137538)(465.00427979,561.10138062)
\lineto(465.21427979,561.25138062)
\curveto(465.26427785,561.29137515)(465.3142778,561.32637511)(465.36427979,561.35638062)
\curveto(465.4142777,561.39637504)(465.44927766,561.45137499)(465.46927979,561.52138062)
\curveto(465.46927764,561.55137489)(465.45927765,561.57637486)(465.43927979,561.59638062)
\curveto(465.42927768,561.62637481)(465.41927769,561.65137479)(465.40927979,561.67138062)
\curveto(465.36927774,561.72137472)(465.31927779,561.76637467)(465.25927979,561.80638062)
\curveto(465.2092779,561.85637458)(465.15927795,561.90137454)(465.10927979,561.94138062)
\curveto(465.06927804,561.97137447)(465.01927809,562.02637441)(464.95927979,562.10638062)
\curveto(464.93927817,562.1363743)(464.9092782,562.16137428)(464.86927979,562.18138062)
\curveto(464.83927827,562.21137423)(464.8142783,562.24637419)(464.79427979,562.28638062)
\curveto(464.62427849,562.49637394)(464.49427862,562.7413737)(464.40427979,563.02138062)
\curveto(464.38427873,563.10137334)(464.36927874,563.18137326)(464.35927979,563.26138062)
\curveto(464.34927876,563.3413731)(464.33427878,563.42137302)(464.31427979,563.50138062)
\curveto(464.29427882,563.55137289)(464.28427883,563.61637282)(464.28427979,563.69638062)
\curveto(464.28427883,563.78637265)(464.29427882,563.85637258)(464.31427979,563.90638062)
\curveto(464.3142788,564.00637243)(464.31927879,564.07637236)(464.32927979,564.11638062)
\curveto(464.34927876,564.19637224)(464.36427875,564.26637217)(464.37427979,564.32638062)
\curveto(464.38427873,564.39637204)(464.39927871,564.46637197)(464.41927979,564.53638062)
\curveto(464.56927854,564.96637147)(464.78427833,565.31137113)(465.06427979,565.57138062)
\curveto(465.35427776,565.83137061)(465.70427741,566.04637039)(466.11427979,566.21638062)
\curveto(466.22427689,566.26637017)(466.33927677,566.29637014)(466.45927979,566.30638062)
\curveto(466.58927652,566.32637011)(466.71927639,566.35637008)(466.84927979,566.39638062)
\curveto(466.92927618,566.39637004)(466.99927611,566.39637004)(467.05927979,566.39638062)
\curveto(467.12927598,566.40637003)(467.20427591,566.41637002)(467.28427979,566.42638062)
\curveto(468.07427504,566.44636999)(468.72927438,566.31637012)(469.24927979,566.03638062)
\curveto(469.77927333,565.75637068)(470.15927295,565.34637109)(470.38927979,564.80638062)
\curveto(470.49927261,564.57637186)(470.56927254,564.29137215)(470.59927979,563.95138062)
\curveto(470.63927247,563.62137282)(470.6092725,563.31637312)(470.50927979,563.03638062)
\curveto(470.46927264,562.90637353)(470.41927269,562.78637365)(470.35927979,562.67638062)
\curveto(470.3092728,562.56637387)(470.24927286,562.46137398)(470.17927979,562.36138062)
\curveto(470.15927295,562.32137412)(470.12927298,562.28637415)(470.08927979,562.25638062)
\lineto(469.99927979,562.16638062)
\curveto(469.94927316,562.07637436)(469.88927322,562.01137443)(469.81927979,561.97138062)
\curveto(469.76927334,561.92137452)(469.7142734,561.87137457)(469.65427979,561.82138062)
\curveto(469.60427351,561.78137466)(469.55927355,561.7363747)(469.51927979,561.68638062)
\curveto(469.49927361,561.66637477)(469.47927363,561.6413748)(469.45927979,561.61138062)
\curveto(469.44927366,561.59137485)(469.44927366,561.56637487)(469.45927979,561.53638062)
\curveto(469.46927364,561.48637495)(469.49927361,561.436375)(469.54927979,561.38638062)
\curveto(469.59927351,561.34637509)(469.65427346,561.30637513)(469.71427979,561.26638062)
\lineto(469.89427979,561.14638062)
\curveto(469.95427316,561.11637532)(470.00427311,561.08637535)(470.04427979,561.05638062)
\curveto(470.37427274,560.81637562)(470.62427249,560.50637593)(470.79427979,560.12638062)
\curveto(470.83427228,560.04637639)(470.86427225,559.96137648)(470.88427979,559.87138062)
\curveto(470.9142722,559.78137666)(470.93927217,559.69137675)(470.95927979,559.60138062)
\curveto(470.96927214,559.55137689)(470.97927213,559.49637694)(470.98927979,559.43638062)
\lineto(471.01927979,559.28638062)
\curveto(471.02927208,559.22637721)(471.02927208,559.16137728)(471.01927979,559.09138062)
\curveto(471.0092721,559.03137741)(471.0142721,558.97137747)(471.03427979,558.91138062)
\moveto(465.64927979,563.95138062)
\curveto(465.61927749,563.8413726)(465.6142775,563.70137274)(465.63427979,563.53138062)
\curveto(465.65427746,563.37137307)(465.67927743,563.24637319)(465.70927979,563.15638062)
\curveto(465.81927729,562.8363736)(465.96927714,562.59137385)(466.15927979,562.42138062)
\curveto(466.34927676,562.26137418)(466.6142765,562.13137431)(466.95427979,562.03138062)
\curveto(467.08427603,562.00137444)(467.24927586,561.97637446)(467.44927979,561.95638062)
\curveto(467.64927546,561.94637449)(467.81927529,561.96137448)(467.95927979,562.00138062)
\curveto(468.24927486,562.08137436)(468.48927462,562.19137425)(468.67927979,562.33138062)
\curveto(468.87927423,562.48137396)(469.03427408,562.68137376)(469.14427979,562.93138062)
\curveto(469.16427395,562.98137346)(469.17427394,563.02637341)(469.17427979,563.06638062)
\curveto(469.18427393,563.10637333)(469.19927391,563.15137329)(469.21927979,563.20138062)
\curveto(469.24927386,563.31137313)(469.26927384,563.45137299)(469.27927979,563.62138062)
\curveto(469.28927382,563.79137265)(469.27927383,563.9363725)(469.24927979,564.05638062)
\curveto(469.22927388,564.14637229)(469.20427391,564.23137221)(469.17427979,564.31138062)
\curveto(469.15427396,564.39137205)(469.11927399,564.47137197)(469.06927979,564.55138062)
\curveto(468.89927421,564.82137162)(468.67427444,565.01637142)(468.39427979,565.13638062)
\curveto(468.12427499,565.25637118)(467.76427535,565.31637112)(467.31427979,565.31638062)
\curveto(467.29427582,565.29637114)(467.26427585,565.29137115)(467.22427979,565.30138062)
\curveto(467.18427593,565.31137113)(467.14927596,565.31137113)(467.11927979,565.30138062)
\curveto(467.06927604,565.28137116)(467.0142761,565.26637117)(466.95427979,565.25638062)
\curveto(466.90427621,565.25637118)(466.85427626,565.24637119)(466.80427979,565.22638062)
\curveto(466.56427655,565.1363713)(466.35427676,565.02137142)(466.17427979,564.88138062)
\curveto(465.99427712,564.75137169)(465.85427726,564.57137187)(465.75427979,564.34138062)
\curveto(465.73427738,564.28137216)(465.7142774,564.21637222)(465.69427979,564.14638062)
\curveto(465.68427743,564.08637235)(465.66927744,564.02137242)(465.64927979,563.95138062)
\moveto(469.66927979,558.41638062)
\curveto(469.71927339,558.60637783)(469.72427339,558.81137763)(469.68427979,559.03138062)
\curveto(469.65427346,559.25137719)(469.6092735,559.43137701)(469.54927979,559.57138062)
\curveto(469.37927373,559.9413765)(469.11927399,560.24637619)(468.76927979,560.48638062)
\curveto(468.42927468,560.72637571)(467.99427512,560.84637559)(467.46427979,560.84638062)
\curveto(467.43427568,560.82637561)(467.39427572,560.82137562)(467.34427979,560.83138062)
\curveto(467.29427582,560.85137559)(467.25427586,560.85637558)(467.22427979,560.84638062)
\lineto(466.95427979,560.78638062)
\curveto(466.87427624,560.77637566)(466.79427632,560.76137568)(466.71427979,560.74138062)
\curveto(466.4142767,560.63137581)(466.14927696,560.48637595)(465.91927979,560.30638062)
\curveto(465.69927741,560.12637631)(465.52927758,559.89637654)(465.40927979,559.61638062)
\curveto(465.37927773,559.5363769)(465.35427776,559.45637698)(465.33427979,559.37638062)
\curveto(465.3142778,559.29637714)(465.29427782,559.21137723)(465.27427979,559.12138062)
\curveto(465.24427787,559.00137744)(465.23427788,558.85137759)(465.24427979,558.67138062)
\curveto(465.26427785,558.49137795)(465.28927782,558.35137809)(465.31927979,558.25138062)
\curveto(465.33927777,558.20137824)(465.34927776,558.15637828)(465.34927979,558.11638062)
\curveto(465.35927775,558.08637835)(465.37427774,558.04637839)(465.39427979,557.99638062)
\curveto(465.49427762,557.77637866)(465.62427749,557.57637886)(465.78427979,557.39638062)
\curveto(465.95427716,557.21637922)(466.14927696,557.08137936)(466.36927979,556.99138062)
\curveto(466.43927667,556.95137949)(466.53427658,556.91637952)(466.65427979,556.88638062)
\curveto(466.87427624,556.79637964)(467.12927598,556.75137969)(467.41927979,556.75138062)
\lineto(467.70427979,556.75138062)
\curveto(467.80427531,556.77137967)(467.89927521,556.78637965)(467.98927979,556.79638062)
\curveto(468.07927503,556.80637963)(468.16927494,556.82637961)(468.25927979,556.85638062)
\curveto(468.51927459,556.9363795)(468.75927435,557.06637937)(468.97927979,557.24638062)
\curveto(469.2092739,557.436379)(469.37927373,557.65137879)(469.48927979,557.89138062)
\curveto(469.52927358,557.97137847)(469.55927355,558.05137839)(469.57927979,558.13138062)
\curveto(469.6092735,558.22137822)(469.63927347,558.31637812)(469.66927979,558.41638062)
}
}
{
\newrgbcolor{curcolor}{0 0 0}
\pscustom[linestyle=none,fillstyle=solid,fillcolor=curcolor]
{
\newpath
\moveto(473.32388916,557.47138062)
\lineto(473.62388916,557.47138062)
\curveto(473.7338871,557.48137896)(473.838887,557.48137896)(473.93888916,557.47138062)
\curveto(474.04888679,557.47137897)(474.14888669,557.46137898)(474.23888916,557.44138062)
\curveto(474.32888651,557.43137901)(474.39888644,557.40637903)(474.44888916,557.36638062)
\curveto(474.46888637,557.34637909)(474.48388635,557.31637912)(474.49388916,557.27638062)
\curveto(474.51388632,557.2363792)(474.5338863,557.19137925)(474.55388916,557.14138062)
\lineto(474.55388916,557.06638062)
\curveto(474.56388627,557.01637942)(474.56388627,556.96137948)(474.55388916,556.90138062)
\lineto(474.55388916,556.75138062)
\lineto(474.55388916,556.27138062)
\curveto(474.55388628,556.10138034)(474.51388632,555.98138046)(474.43388916,555.91138062)
\curveto(474.36388647,555.86138058)(474.27388656,555.8363806)(474.16388916,555.83638062)
\lineto(473.83388916,555.83638062)
\lineto(473.38388916,555.83638062)
\curveto(473.2338876,555.8363806)(473.11888772,555.86638057)(473.03888916,555.92638062)
\curveto(472.99888784,555.95638048)(472.96888787,556.00638043)(472.94888916,556.07638062)
\curveto(472.92888791,556.15638028)(472.91388792,556.2413802)(472.90388916,556.33138062)
\lineto(472.90388916,556.61638062)
\curveto(472.91388792,556.71637972)(472.91888792,556.80137964)(472.91888916,556.87138062)
\lineto(472.91888916,557.06638062)
\curveto(472.91888792,557.12637931)(472.92888791,557.18137926)(472.94888916,557.23138062)
\curveto(472.98888785,557.3413791)(473.05888778,557.41137903)(473.15888916,557.44138062)
\curveto(473.18888765,557.441379)(473.24388759,557.45137899)(473.32388916,557.47138062)
}
}
{
\newrgbcolor{curcolor}{0 0 0}
\pscustom[linestyle=none,fillstyle=solid,fillcolor=curcolor]
{
\newpath
\moveto(480.58904541,566.44138062)
\curveto(480.68904056,566.44137)(480.78404046,566.43137001)(480.87404541,566.41138062)
\curveto(480.96404028,566.40137004)(481.02904022,566.37137007)(481.06904541,566.32138062)
\curveto(481.12904012,566.2413702)(481.15904009,566.1363703)(481.15904541,566.00638062)
\lineto(481.15904541,565.61638062)
\lineto(481.15904541,564.11638062)
\lineto(481.15904541,557.72638062)
\lineto(481.15904541,556.55638062)
\lineto(481.15904541,556.24138062)
\curveto(481.16904008,556.1413803)(481.15404009,556.06138038)(481.11404541,556.00138062)
\curveto(481.06404018,555.92138052)(480.98904026,555.87138057)(480.88904541,555.85138062)
\curveto(480.79904045,555.8413806)(480.68904056,555.8363806)(480.55904541,555.83638062)
\lineto(480.33404541,555.83638062)
\curveto(480.25404099,555.85638058)(480.18404106,555.87138057)(480.12404541,555.88138062)
\curveto(480.06404118,555.90138054)(480.01404123,555.9413805)(479.97404541,556.00138062)
\curveto(479.93404131,556.06138038)(479.91404133,556.1363803)(479.91404541,556.22638062)
\lineto(479.91404541,556.52638062)
\lineto(479.91404541,557.62138062)
\lineto(479.91404541,562.96138062)
\curveto(479.89404135,563.05137339)(479.87904137,563.12637331)(479.86904541,563.18638062)
\curveto(479.86904138,563.25637318)(479.83904141,563.31637312)(479.77904541,563.36638062)
\curveto(479.70904154,563.41637302)(479.61904163,563.441373)(479.50904541,563.44138062)
\curveto(479.40904184,563.45137299)(479.29904195,563.45637298)(479.17904541,563.45638062)
\lineto(478.03904541,563.45638062)
\lineto(477.54404541,563.45638062)
\curveto(477.38404386,563.46637297)(477.27404397,563.52637291)(477.21404541,563.63638062)
\curveto(477.19404405,563.66637277)(477.18404406,563.69637274)(477.18404541,563.72638062)
\curveto(477.18404406,563.76637267)(477.17904407,563.81137263)(477.16904541,563.86138062)
\curveto(477.1490441,563.98137246)(477.15404409,564.09137235)(477.18404541,564.19138062)
\curveto(477.22404402,564.29137215)(477.27904397,564.36137208)(477.34904541,564.40138062)
\curveto(477.42904382,564.45137199)(477.5490437,564.47637196)(477.70904541,564.47638062)
\curveto(477.86904338,564.47637196)(478.00404324,564.49137195)(478.11404541,564.52138062)
\curveto(478.16404308,564.53137191)(478.21904303,564.5363719)(478.27904541,564.53638062)
\curveto(478.33904291,564.54637189)(478.39904285,564.56137188)(478.45904541,564.58138062)
\curveto(478.60904264,564.63137181)(478.75404249,564.68137176)(478.89404541,564.73138062)
\curveto(479.03404221,564.79137165)(479.16904208,564.86137158)(479.29904541,564.94138062)
\curveto(479.43904181,565.03137141)(479.55904169,565.1363713)(479.65904541,565.25638062)
\curveto(479.75904149,565.37637106)(479.85404139,565.50637093)(479.94404541,565.64638062)
\curveto(480.00404124,565.74637069)(480.0490412,565.85637058)(480.07904541,565.97638062)
\curveto(480.11904113,566.09637034)(480.16904108,566.20137024)(480.22904541,566.29138062)
\curveto(480.27904097,566.35137009)(480.3490409,566.39137005)(480.43904541,566.41138062)
\curveto(480.45904079,566.42137002)(480.48404076,566.42637001)(480.51404541,566.42638062)
\curveto(480.5440407,566.42637001)(480.56904068,566.43137001)(480.58904541,566.44138062)
}
}
{
\newrgbcolor{curcolor}{0 0 0}
\pscustom[linestyle=none,fillstyle=solid,fillcolor=curcolor]
{
\newpath
\moveto(494.68365479,564.35638062)
\curveto(494.48364449,564.06637237)(494.2736447,563.78137266)(494.05365479,563.50138062)
\curveto(493.84364513,563.22137322)(493.63864533,562.9363735)(493.43865479,562.64638062)
\curveto(492.83864613,561.79637464)(492.23364674,560.95637548)(491.62365479,560.12638062)
\curveto(491.01364796,559.30637713)(490.40864856,558.47137797)(489.80865479,557.62138062)
\lineto(489.29865479,556.90138062)
\lineto(488.78865479,556.21138062)
\curveto(488.70865026,556.10138034)(488.62865034,555.98638045)(488.54865479,555.86638062)
\curveto(488.4686505,555.74638069)(488.3736506,555.65138079)(488.26365479,555.58138062)
\curveto(488.22365075,555.56138088)(488.15865081,555.54638089)(488.06865479,555.53638062)
\curveto(487.98865098,555.51638092)(487.89865107,555.50638093)(487.79865479,555.50638062)
\curveto(487.69865127,555.50638093)(487.60365137,555.51138093)(487.51365479,555.52138062)
\curveto(487.43365154,555.53138091)(487.3736516,555.55138089)(487.33365479,555.58138062)
\curveto(487.30365167,555.60138084)(487.27865169,555.6363808)(487.25865479,555.68638062)
\curveto(487.24865172,555.72638071)(487.25365172,555.77138067)(487.27365479,555.82138062)
\curveto(487.31365166,555.90138054)(487.35865161,555.97638046)(487.40865479,556.04638062)
\curveto(487.4686515,556.12638031)(487.52365145,556.20638023)(487.57365479,556.28638062)
\curveto(487.81365116,556.62637981)(488.05865091,556.96137948)(488.30865479,557.29138062)
\curveto(488.55865041,557.62137882)(488.79865017,557.95637848)(489.02865479,558.29638062)
\curveto(489.18864978,558.51637792)(489.34864962,558.73137771)(489.50865479,558.94138062)
\curveto(489.6686493,559.15137729)(489.82864914,559.36637707)(489.98865479,559.58638062)
\curveto(490.34864862,560.10637633)(490.71364826,560.61637582)(491.08365479,561.11638062)
\curveto(491.45364752,561.61637482)(491.82364715,562.12637431)(492.19365479,562.64638062)
\curveto(492.33364664,562.84637359)(492.4736465,563.0413734)(492.61365479,563.23138062)
\curveto(492.76364621,563.42137302)(492.90864606,563.61637282)(493.04865479,563.81638062)
\curveto(493.25864571,564.11637232)(493.4736455,564.41637202)(493.69365479,564.71638062)
\lineto(494.35365479,565.61638062)
\lineto(494.53365479,565.88638062)
\lineto(494.74365479,566.15638062)
\lineto(494.86365479,566.33638062)
\curveto(494.91364406,566.39637004)(494.96364401,566.45136999)(495.01365479,566.50138062)
\curveto(495.08364389,566.55136989)(495.15864381,566.58636985)(495.23865479,566.60638062)
\curveto(495.25864371,566.61636982)(495.28364369,566.61636982)(495.31365479,566.60638062)
\curveto(495.35364362,566.60636983)(495.38364359,566.61636982)(495.40365479,566.63638062)
\curveto(495.52364345,566.6363698)(495.65864331,566.63136981)(495.80865479,566.62138062)
\curveto(495.95864301,566.62136982)(496.04864292,566.57636986)(496.07865479,566.48638062)
\curveto(496.09864287,566.45636998)(496.10364287,566.42137002)(496.09365479,566.38138062)
\curveto(496.08364289,566.3413701)(496.0686429,566.31137013)(496.04865479,566.29138062)
\curveto(496.00864296,566.21137023)(495.968643,566.1413703)(495.92865479,566.08138062)
\curveto(495.88864308,566.02137042)(495.84364313,565.96137048)(495.79365479,565.90138062)
\lineto(495.22365479,565.12138062)
\curveto(495.04364393,564.87137157)(494.86364411,564.61637182)(494.68365479,564.35638062)
\moveto(487.82865479,560.45638062)
\curveto(487.77865119,560.47637596)(487.72865124,560.48137596)(487.67865479,560.47138062)
\curveto(487.62865134,560.46137598)(487.57865139,560.46637597)(487.52865479,560.48638062)
\curveto(487.41865155,560.50637593)(487.31365166,560.52637591)(487.21365479,560.54638062)
\curveto(487.12365185,560.57637586)(487.02865194,560.61637582)(486.92865479,560.66638062)
\curveto(486.59865237,560.80637563)(486.34365263,561.00137544)(486.16365479,561.25138062)
\curveto(485.98365299,561.51137493)(485.83865313,561.82137462)(485.72865479,562.18138062)
\curveto(485.69865327,562.26137418)(485.67865329,562.3413741)(485.66865479,562.42138062)
\curveto(485.65865331,562.51137393)(485.64365333,562.59637384)(485.62365479,562.67638062)
\curveto(485.61365336,562.72637371)(485.60865336,562.79137365)(485.60865479,562.87138062)
\curveto(485.59865337,562.90137354)(485.59365338,562.93137351)(485.59365479,562.96138062)
\curveto(485.59365338,563.00137344)(485.58865338,563.0363734)(485.57865479,563.06638062)
\lineto(485.57865479,563.21638062)
\curveto(485.5686534,563.26637317)(485.56365341,563.32637311)(485.56365479,563.39638062)
\curveto(485.56365341,563.47637296)(485.5686534,563.5413729)(485.57865479,563.59138062)
\lineto(485.57865479,563.75638062)
\curveto(485.59865337,563.80637263)(485.60365337,563.85137259)(485.59365479,563.89138062)
\curveto(485.59365338,563.9413725)(485.59865337,563.98637245)(485.60865479,564.02638062)
\curveto(485.61865335,564.06637237)(485.62365335,564.10137234)(485.62365479,564.13138062)
\curveto(485.62365335,564.17137227)(485.62865334,564.21137223)(485.63865479,564.25138062)
\curveto(485.6686533,564.36137208)(485.68865328,564.47137197)(485.69865479,564.58138062)
\curveto(485.71865325,564.70137174)(485.75365322,564.81637162)(485.80365479,564.92638062)
\curveto(485.94365303,565.26637117)(486.10365287,565.5413709)(486.28365479,565.75138062)
\curveto(486.4736525,565.97137047)(486.74365223,566.15137029)(487.09365479,566.29138062)
\curveto(487.1736518,566.32137012)(487.25865171,566.3413701)(487.34865479,566.35138062)
\curveto(487.43865153,566.37137007)(487.53365144,566.39137005)(487.63365479,566.41138062)
\curveto(487.66365131,566.42137002)(487.71865125,566.42137002)(487.79865479,566.41138062)
\curveto(487.87865109,566.41137003)(487.92865104,566.42137002)(487.94865479,566.44138062)
\curveto(488.50865046,566.45136999)(488.95865001,566.3413701)(489.29865479,566.11138062)
\curveto(489.64864932,565.88137056)(489.90864906,565.57637086)(490.07865479,565.19638062)
\curveto(490.11864885,565.10637133)(490.15364882,565.01137143)(490.18365479,564.91138062)
\curveto(490.21364876,564.81137163)(490.23864873,564.71137173)(490.25865479,564.61138062)
\curveto(490.27864869,564.58137186)(490.28364869,564.55137189)(490.27365479,564.52138062)
\curveto(490.2736487,564.49137195)(490.27864869,564.46137198)(490.28865479,564.43138062)
\curveto(490.31864865,564.32137212)(490.33864863,564.19637224)(490.34865479,564.05638062)
\curveto(490.35864861,563.92637251)(490.3686486,563.79137265)(490.37865479,563.65138062)
\lineto(490.37865479,563.48638062)
\curveto(490.38864858,563.42637301)(490.38864858,563.37137307)(490.37865479,563.32138062)
\curveto(490.3686486,563.27137317)(490.36364861,563.22137322)(490.36365479,563.17138062)
\lineto(490.36365479,563.03638062)
\curveto(490.35364862,562.99637344)(490.34864862,562.95637348)(490.34865479,562.91638062)
\curveto(490.35864861,562.87637356)(490.35364862,562.83137361)(490.33365479,562.78138062)
\curveto(490.31364866,562.67137377)(490.29364868,562.56637387)(490.27365479,562.46638062)
\curveto(490.26364871,562.36637407)(490.24364873,562.26637417)(490.21365479,562.16638062)
\curveto(490.08364889,561.80637463)(489.91864905,561.49137495)(489.71865479,561.22138062)
\curveto(489.51864945,560.95137549)(489.24364973,560.74637569)(488.89365479,560.60638062)
\curveto(488.81365016,560.57637586)(488.72865024,560.55137589)(488.63865479,560.53138062)
\lineto(488.36865479,560.47138062)
\curveto(488.31865065,560.46137598)(488.2736507,560.45637598)(488.23365479,560.45638062)
\curveto(488.19365078,560.46637597)(488.15365082,560.46637597)(488.11365479,560.45638062)
\curveto(488.01365096,560.436376)(487.91865105,560.436376)(487.82865479,560.45638062)
\moveto(486.98865479,561.85138062)
\curveto(487.02865194,561.78137466)(487.0686519,561.71637472)(487.10865479,561.65638062)
\curveto(487.14865182,561.60637483)(487.19865177,561.55637488)(487.25865479,561.50638062)
\lineto(487.40865479,561.38638062)
\curveto(487.4686515,561.35637508)(487.53365144,561.33137511)(487.60365479,561.31138062)
\curveto(487.64365133,561.29137515)(487.67865129,561.28137516)(487.70865479,561.28138062)
\curveto(487.74865122,561.29137515)(487.78865118,561.28637515)(487.82865479,561.26638062)
\curveto(487.85865111,561.26637517)(487.89865107,561.26137518)(487.94865479,561.25138062)
\curveto(487.99865097,561.25137519)(488.03865093,561.25637518)(488.06865479,561.26638062)
\lineto(488.29365479,561.31138062)
\curveto(488.54365043,561.39137505)(488.72865024,561.51637492)(488.84865479,561.68638062)
\curveto(488.92865004,561.78637465)(488.99864997,561.91637452)(489.05865479,562.07638062)
\curveto(489.13864983,562.25637418)(489.19864977,562.48137396)(489.23865479,562.75138062)
\curveto(489.27864969,563.03137341)(489.29364968,563.31137313)(489.28365479,563.59138062)
\curveto(489.2736497,563.88137256)(489.24364973,564.15637228)(489.19365479,564.41638062)
\curveto(489.14364983,564.67637176)(489.0686499,564.88637155)(488.96865479,565.04638062)
\curveto(488.84865012,565.24637119)(488.69865027,565.39637104)(488.51865479,565.49638062)
\curveto(488.43865053,565.54637089)(488.34865062,565.57637086)(488.24865479,565.58638062)
\curveto(488.14865082,565.60637083)(488.04365093,565.61637082)(487.93365479,565.61638062)
\curveto(487.91365106,565.60637083)(487.88865108,565.60137084)(487.85865479,565.60138062)
\curveto(487.83865113,565.61137083)(487.81865115,565.61137083)(487.79865479,565.60138062)
\curveto(487.74865122,565.59137085)(487.70365127,565.58137086)(487.66365479,565.57138062)
\curveto(487.62365135,565.57137087)(487.58365139,565.56137088)(487.54365479,565.54138062)
\curveto(487.36365161,565.46137098)(487.21365176,565.3413711)(487.09365479,565.18138062)
\curveto(486.98365199,565.02137142)(486.89365208,564.8413716)(486.82365479,564.64138062)
\curveto(486.76365221,564.45137199)(486.71865225,564.22637221)(486.68865479,563.96638062)
\curveto(486.6686523,563.70637273)(486.66365231,563.441373)(486.67365479,563.17138062)
\curveto(486.68365229,562.91137353)(486.71365226,562.66137378)(486.76365479,562.42138062)
\curveto(486.82365215,562.19137425)(486.89865207,562.00137444)(486.98865479,561.85138062)
\moveto(497.78865479,558.86638062)
\curveto(497.79864117,558.81637762)(497.80364117,558.72637771)(497.80365479,558.59638062)
\curveto(497.80364117,558.46637797)(497.79364118,558.37637806)(497.77365479,558.32638062)
\curveto(497.75364122,558.27637816)(497.74864122,558.22137822)(497.75865479,558.16138062)
\curveto(497.7686412,558.11137833)(497.7686412,558.06137838)(497.75865479,558.01138062)
\curveto(497.71864125,557.87137857)(497.68864128,557.7363787)(497.66865479,557.60638062)
\curveto(497.65864131,557.47637896)(497.62864134,557.35637908)(497.57865479,557.24638062)
\curveto(497.43864153,556.89637954)(497.2736417,556.60137984)(497.08365479,556.36138062)
\curveto(496.89364208,556.13138031)(496.62364235,555.94638049)(496.27365479,555.80638062)
\curveto(496.19364278,555.77638066)(496.10864286,555.75638068)(496.01865479,555.74638062)
\curveto(495.92864304,555.72638071)(495.84364313,555.70638073)(495.76365479,555.68638062)
\curveto(495.71364326,555.67638076)(495.66364331,555.67138077)(495.61365479,555.67138062)
\curveto(495.56364341,555.67138077)(495.51364346,555.66638077)(495.46365479,555.65638062)
\curveto(495.43364354,555.64638079)(495.38364359,555.64638079)(495.31365479,555.65638062)
\curveto(495.24364373,555.65638078)(495.19364378,555.66138078)(495.16365479,555.67138062)
\curveto(495.10364387,555.69138075)(495.04364393,555.70138074)(494.98365479,555.70138062)
\curveto(494.93364404,555.69138075)(494.88364409,555.69638074)(494.83365479,555.71638062)
\curveto(494.74364423,555.7363807)(494.65364432,555.76138068)(494.56365479,555.79138062)
\curveto(494.48364449,555.81138063)(494.40364457,555.8413806)(494.32365479,555.88138062)
\curveto(494.00364497,556.02138042)(493.75364522,556.21638022)(493.57365479,556.46638062)
\curveto(493.39364558,556.72637971)(493.24364573,557.03137941)(493.12365479,557.38138062)
\curveto(493.10364587,557.46137898)(493.08864588,557.54637889)(493.07865479,557.63638062)
\curveto(493.0686459,557.72637871)(493.05364592,557.81137863)(493.03365479,557.89138062)
\curveto(493.02364595,557.92137852)(493.01864595,557.95137849)(493.01865479,557.98138062)
\lineto(493.01865479,558.08638062)
\curveto(492.99864597,558.16637827)(492.98864598,558.24637819)(492.98865479,558.32638062)
\lineto(492.98865479,558.46138062)
\curveto(492.968646,558.56137788)(492.968646,558.66137778)(492.98865479,558.76138062)
\lineto(492.98865479,558.94138062)
\curveto(492.99864597,558.99137745)(493.00364597,559.0363774)(493.00365479,559.07638062)
\curveto(493.00364597,559.12637731)(493.00864596,559.17137727)(493.01865479,559.21138062)
\curveto(493.02864594,559.25137719)(493.03364594,559.28637715)(493.03365479,559.31638062)
\curveto(493.03364594,559.35637708)(493.03864593,559.39637704)(493.04865479,559.43638062)
\lineto(493.10865479,559.76638062)
\curveto(493.12864584,559.88637655)(493.15864581,559.99637644)(493.19865479,560.09638062)
\curveto(493.33864563,560.42637601)(493.49864547,560.70137574)(493.67865479,560.92138062)
\curveto(493.8686451,561.15137529)(494.12864484,561.3363751)(494.45865479,561.47638062)
\curveto(494.53864443,561.51637492)(494.62364435,561.5413749)(494.71365479,561.55138062)
\lineto(495.01365479,561.61138062)
\lineto(495.14865479,561.61138062)
\curveto(495.19864377,561.62137482)(495.24864372,561.62637481)(495.29865479,561.62638062)
\curveto(495.8686431,561.64637479)(496.32864264,561.5413749)(496.67865479,561.31138062)
\curveto(497.03864193,561.09137535)(497.30364167,560.79137565)(497.47365479,560.41138062)
\curveto(497.52364145,560.31137613)(497.56364141,560.21137623)(497.59365479,560.11138062)
\curveto(497.62364135,560.01137643)(497.65364132,559.90637653)(497.68365479,559.79638062)
\curveto(497.69364128,559.75637668)(497.69864127,559.72137672)(497.69865479,559.69138062)
\curveto(497.69864127,559.67137677)(497.70364127,559.6413768)(497.71365479,559.60138062)
\curveto(497.73364124,559.53137691)(497.74364123,559.45637698)(497.74365479,559.37638062)
\curveto(497.74364123,559.29637714)(497.75364122,559.21637722)(497.77365479,559.13638062)
\curveto(497.7736412,559.08637735)(497.7736412,559.0413774)(497.77365479,559.00138062)
\curveto(497.7736412,558.96137748)(497.77864119,558.91637752)(497.78865479,558.86638062)
\moveto(496.67865479,558.43138062)
\curveto(496.68864228,558.48137796)(496.69364228,558.55637788)(496.69365479,558.65638062)
\curveto(496.70364227,558.75637768)(496.69864227,558.83137761)(496.67865479,558.88138062)
\curveto(496.65864231,558.9413775)(496.65364232,558.99637744)(496.66365479,559.04638062)
\curveto(496.68364229,559.10637733)(496.68364229,559.16637727)(496.66365479,559.22638062)
\curveto(496.65364232,559.25637718)(496.64864232,559.29137715)(496.64865479,559.33138062)
\curveto(496.64864232,559.37137707)(496.64364233,559.41137703)(496.63365479,559.45138062)
\curveto(496.61364236,559.53137691)(496.59364238,559.60637683)(496.57365479,559.67638062)
\curveto(496.56364241,559.75637668)(496.54864242,559.8363766)(496.52865479,559.91638062)
\curveto(496.49864247,559.97637646)(496.4736425,560.0363764)(496.45365479,560.09638062)
\curveto(496.43364254,560.15637628)(496.40364257,560.21637622)(496.36365479,560.27638062)
\curveto(496.26364271,560.44637599)(496.13364284,560.58137586)(495.97365479,560.68138062)
\curveto(495.89364308,560.73137571)(495.79864317,560.76637567)(495.68865479,560.78638062)
\curveto(495.57864339,560.80637563)(495.45364352,560.81637562)(495.31365479,560.81638062)
\curveto(495.29364368,560.80637563)(495.2686437,560.80137564)(495.23865479,560.80138062)
\curveto(495.20864376,560.81137563)(495.17864379,560.81137563)(495.14865479,560.80138062)
\lineto(494.99865479,560.74138062)
\curveto(494.94864402,560.73137571)(494.90364407,560.71637572)(494.86365479,560.69638062)
\curveto(494.6736443,560.58637585)(494.52864444,560.441376)(494.42865479,560.26138062)
\curveto(494.33864463,560.08137636)(494.25864471,559.87637656)(494.18865479,559.64638062)
\curveto(494.14864482,559.51637692)(494.12864484,559.38137706)(494.12865479,559.24138062)
\curveto(494.12864484,559.11137733)(494.11864485,558.96637747)(494.09865479,558.80638062)
\curveto(494.08864488,558.75637768)(494.07864489,558.69637774)(494.06865479,558.62638062)
\curveto(494.0686449,558.55637788)(494.07864489,558.49637794)(494.09865479,558.44638062)
\lineto(494.09865479,558.28138062)
\lineto(494.09865479,558.10138062)
\curveto(494.10864486,558.05137839)(494.11864485,557.99637844)(494.12865479,557.93638062)
\curveto(494.13864483,557.88637855)(494.14364483,557.83137861)(494.14365479,557.77138062)
\curveto(494.15364482,557.71137873)(494.1686448,557.65637878)(494.18865479,557.60638062)
\curveto(494.23864473,557.41637902)(494.29864467,557.2413792)(494.36865479,557.08138062)
\curveto(494.43864453,556.92137952)(494.54364443,556.79137965)(494.68365479,556.69138062)
\curveto(494.81364416,556.59137985)(494.95364402,556.52137992)(495.10365479,556.48138062)
\curveto(495.13364384,556.47137997)(495.15864381,556.46637997)(495.17865479,556.46638062)
\curveto(495.20864376,556.47637996)(495.23864373,556.47637996)(495.26865479,556.46638062)
\curveto(495.28864368,556.46637997)(495.31864365,556.46137998)(495.35865479,556.45138062)
\curveto(495.39864357,556.45137999)(495.43364354,556.45637998)(495.46365479,556.46638062)
\curveto(495.50364347,556.47637996)(495.54364343,556.48137996)(495.58365479,556.48138062)
\curveto(495.62364335,556.48137996)(495.66364331,556.49137995)(495.70365479,556.51138062)
\curveto(495.94364303,556.59137985)(496.13864283,556.72637971)(496.28865479,556.91638062)
\curveto(496.40864256,557.09637934)(496.49864247,557.30137914)(496.55865479,557.53138062)
\curveto(496.57864239,557.60137884)(496.59364238,557.67137877)(496.60365479,557.74138062)
\curveto(496.61364236,557.82137862)(496.62864234,557.90137854)(496.64865479,557.98138062)
\curveto(496.64864232,558.0413784)(496.65364232,558.08637835)(496.66365479,558.11638062)
\curveto(496.66364231,558.1363783)(496.66364231,558.16137828)(496.66365479,558.19138062)
\curveto(496.66364231,558.23137821)(496.6686423,558.26137818)(496.67865479,558.28138062)
\lineto(496.67865479,558.43138062)
}
}
{
\newrgbcolor{curcolor}{0 0 0}
\pscustom[linestyle=none,fillstyle=solid,fillcolor=curcolor]
{
\newpath
\moveto(591.61357422,387.18858887)
\curveto(591.68356657,387.13858541)(591.72356653,387.06858548)(591.73357422,386.97858887)
\curveto(591.7535665,386.88858566)(591.76356649,386.78358576)(591.76357422,386.66358887)
\curveto(591.76356649,386.61358593)(591.7585665,386.56358598)(591.74857422,386.51358887)
\curveto(591.74856651,386.46358608)(591.73856652,386.41858613)(591.71857422,386.37858887)
\curveto(591.68856657,386.28858626)(591.62856663,386.22858632)(591.53857422,386.19858887)
\curveto(591.4585668,386.17858637)(591.36356689,386.16858638)(591.25357422,386.16858887)
\lineto(590.93857422,386.16858887)
\curveto(590.82856743,386.17858637)(590.72356753,386.16858638)(590.62357422,386.13858887)
\curveto(590.48356777,386.10858644)(590.39356786,386.02858652)(590.35357422,385.89858887)
\curveto(590.33356792,385.82858672)(590.32356793,385.7435868)(590.32357422,385.64358887)
\lineto(590.32357422,385.37358887)
\lineto(590.32357422,384.42858887)
\lineto(590.32357422,384.09858887)
\curveto(590.32356793,383.98858856)(590.30356795,383.90358864)(590.26357422,383.84358887)
\curveto(590.22356803,383.78358876)(590.17356808,383.7435888)(590.11357422,383.72358887)
\curveto(590.06356819,383.71358883)(589.99856826,383.69858885)(589.91857422,383.67858887)
\lineto(589.72357422,383.67858887)
\curveto(589.60356865,383.67858887)(589.49856876,383.68358886)(589.40857422,383.69358887)
\curveto(589.31856894,383.71358883)(589.24856901,383.76358878)(589.19857422,383.84358887)
\curveto(589.16856909,383.89358865)(589.1535691,383.96358858)(589.15357422,384.05358887)
\lineto(589.15357422,384.35358887)
\lineto(589.15357422,385.38858887)
\curveto(589.1535691,385.548587)(589.14356911,385.69358685)(589.12357422,385.82358887)
\curveto(589.11356914,385.96358658)(589.0585692,386.05858649)(588.95857422,386.10858887)
\curveto(588.90856935,386.12858642)(588.83856942,386.1435864)(588.74857422,386.15358887)
\curveto(588.66856959,386.16358638)(588.57856968,386.16858638)(588.47857422,386.16858887)
\lineto(588.19357422,386.16858887)
\lineto(587.95357422,386.16858887)
\lineto(585.68857422,386.16858887)
\curveto(585.59857266,386.16858638)(585.49357276,386.16358638)(585.37357422,386.15358887)
\lineto(585.04357422,386.15358887)
\curveto(584.93357332,386.15358639)(584.83357342,386.16358638)(584.74357422,386.18358887)
\curveto(584.6535736,386.20358634)(584.59357366,386.23858631)(584.56357422,386.28858887)
\curveto(584.51357374,386.35858619)(584.48857377,386.45358609)(584.48857422,386.57358887)
\lineto(584.48857422,386.91858887)
\lineto(584.48857422,387.18858887)
\curveto(584.52857373,387.35858519)(584.58357367,387.49858505)(584.65357422,387.60858887)
\curveto(584.72357353,387.71858483)(584.80357345,387.83358471)(584.89357422,387.95358887)
\lineto(585.25357422,388.49358887)
\curveto(585.69357256,389.12358342)(586.12857213,389.7435828)(586.55857422,390.35358887)
\lineto(587.87857422,392.21358887)
\curveto(588.03857022,392.4435801)(588.19357006,392.66357988)(588.34357422,392.87358887)
\curveto(588.49356976,393.09357945)(588.64856961,393.31857923)(588.80857422,393.54858887)
\curveto(588.8585694,393.61857893)(588.90856935,393.68357886)(588.95857422,393.74358887)
\curveto(589.00856925,393.81357873)(589.0585692,393.88857866)(589.10857422,393.96858887)
\lineto(589.16857422,394.05858887)
\curveto(589.19856906,394.09857845)(589.22856903,394.12857842)(589.25857422,394.14858887)
\curveto(589.29856896,394.17857837)(589.33856892,394.19857835)(589.37857422,394.20858887)
\curveto(589.41856884,394.22857832)(589.46356879,394.2485783)(589.51357422,394.26858887)
\curveto(589.53356872,394.26857828)(589.5535687,394.26357828)(589.57357422,394.25358887)
\curveto(589.60356865,394.25357829)(589.62856863,394.26357828)(589.64857422,394.28358887)
\curveto(589.77856848,394.28357826)(589.89856836,394.27857827)(590.00857422,394.26858887)
\curveto(590.11856814,394.25857829)(590.19856806,394.21357833)(590.24857422,394.13358887)
\curveto(590.28856797,394.08357846)(590.30856795,394.01357853)(590.30857422,393.92358887)
\curveto(590.31856794,393.83357871)(590.32356793,393.73857881)(590.32357422,393.63858887)
\lineto(590.32357422,388.17858887)
\curveto(590.32356793,388.10858444)(590.31856794,388.03358451)(590.30857422,387.95358887)
\curveto(590.30856795,387.88358466)(590.31356794,387.81358473)(590.32357422,387.74358887)
\lineto(590.32357422,387.63858887)
\curveto(590.34356791,387.58858496)(590.3585679,387.53358501)(590.36857422,387.47358887)
\curveto(590.37856788,387.42358512)(590.40356785,387.38358516)(590.44357422,387.35358887)
\curveto(590.51356774,387.30358524)(590.59856766,387.27358527)(590.69857422,387.26358887)
\lineto(591.02857422,387.26358887)
\curveto(591.13856712,387.26358528)(591.24356701,387.25858529)(591.34357422,387.24858887)
\curveto(591.4535668,387.2485853)(591.54356671,387.22858532)(591.61357422,387.18858887)
\moveto(589.04857422,387.38358887)
\curveto(589.12856913,387.49358505)(589.16356909,387.66358488)(589.15357422,387.89358887)
\lineto(589.15357422,388.50858887)
\lineto(589.15357422,390.98358887)
\lineto(589.15357422,391.29858887)
\curveto(589.16356909,391.41858113)(589.1585691,391.51858103)(589.13857422,391.59858887)
\lineto(589.13857422,391.74858887)
\curveto(589.13856912,391.83858071)(589.12356913,391.92358062)(589.09357422,392.00358887)
\curveto(589.08356917,392.02358052)(589.07356918,392.03358051)(589.06357422,392.03358887)
\lineto(589.01857422,392.07858887)
\curveto(588.99856926,392.08858046)(588.96856929,392.09358045)(588.92857422,392.09358887)
\curveto(588.90856935,392.07358047)(588.88856937,392.05858049)(588.86857422,392.04858887)
\curveto(588.8585694,392.0485805)(588.84356941,392.0435805)(588.82357422,392.03358887)
\curveto(588.76356949,391.98358056)(588.70356955,391.91358063)(588.64357422,391.82358887)
\curveto(588.58356967,391.73358081)(588.52856973,391.65358089)(588.47857422,391.58358887)
\curveto(588.37856988,391.4435811)(588.28356997,391.29858125)(588.19357422,391.14858887)
\curveto(588.10357015,391.00858154)(588.00857025,390.86858168)(587.90857422,390.72858887)
\lineto(587.36857422,389.94858887)
\curveto(587.19857106,389.68858286)(587.02357123,389.42858312)(586.84357422,389.16858887)
\curveto(586.76357149,389.05858349)(586.68857157,388.95358359)(586.61857422,388.85358887)
\lineto(586.40857422,388.55358887)
\curveto(586.3585719,388.47358407)(586.30857195,388.39858415)(586.25857422,388.32858887)
\curveto(586.21857204,388.25858429)(586.17357208,388.18358436)(586.12357422,388.10358887)
\curveto(586.07357218,388.0435845)(586.02357223,387.97858457)(585.97357422,387.90858887)
\curveto(585.93357232,387.8485847)(585.89357236,387.77858477)(585.85357422,387.69858887)
\curveto(585.81357244,387.63858491)(585.78857247,387.56858498)(585.77857422,387.48858887)
\curveto(585.76857249,387.41858513)(585.80357245,387.36358518)(585.88357422,387.32358887)
\curveto(585.9535723,387.27358527)(586.06357219,387.2485853)(586.21357422,387.24858887)
\curveto(586.37357188,387.25858529)(586.50857175,387.26358528)(586.61857422,387.26358887)
\lineto(588.29857422,387.26358887)
\lineto(588.73357422,387.26358887)
\curveto(588.88356937,387.26358528)(588.98856927,387.30358524)(589.04857422,387.38358887)
}
}
{
\newrgbcolor{curcolor}{0 0 0}
\pscustom[linestyle=none,fillstyle=solid,fillcolor=curcolor]
{
\newpath
\moveto(600.08318359,388.77858887)
\lineto(600.08318359,388.52358887)
\curveto(600.09317589,388.4435841)(600.08817589,388.36858418)(600.06818359,388.29858887)
\lineto(600.06818359,388.05858887)
\lineto(600.06818359,387.89358887)
\curveto(600.04817593,387.79358475)(600.03817594,387.68858486)(600.03818359,387.57858887)
\curveto(600.03817594,387.47858507)(600.02817595,387.37858517)(600.00818359,387.27858887)
\lineto(600.00818359,387.12858887)
\curveto(599.978176,386.98858556)(599.95817602,386.8485857)(599.94818359,386.70858887)
\curveto(599.93817604,386.57858597)(599.91317607,386.4485861)(599.87318359,386.31858887)
\curveto(599.85317613,386.23858631)(599.83317615,386.15358639)(599.81318359,386.06358887)
\lineto(599.75318359,385.82358887)
\lineto(599.63318359,385.52358887)
\curveto(599.60317638,385.43358711)(599.56817641,385.3435872)(599.52818359,385.25358887)
\curveto(599.42817655,385.03358751)(599.29317669,384.81858773)(599.12318359,384.60858887)
\curveto(598.96317702,384.39858815)(598.78817719,384.22858832)(598.59818359,384.09858887)
\curveto(598.54817743,384.05858849)(598.48817749,384.01858853)(598.41818359,383.97858887)
\curveto(598.35817762,383.9485886)(598.29817768,383.91358863)(598.23818359,383.87358887)
\curveto(598.15817782,383.82358872)(598.06317792,383.78358876)(597.95318359,383.75358887)
\curveto(597.84317814,383.72358882)(597.73817824,383.69358885)(597.63818359,383.66358887)
\curveto(597.52817845,383.62358892)(597.41817856,383.59858895)(597.30818359,383.58858887)
\curveto(597.19817878,383.57858897)(597.0831789,383.56358898)(596.96318359,383.54358887)
\curveto(596.92317906,383.53358901)(596.8781791,383.53358901)(596.82818359,383.54358887)
\curveto(596.78817919,383.543589)(596.74817923,383.53858901)(596.70818359,383.52858887)
\curveto(596.66817931,383.51858903)(596.61317937,383.51358903)(596.54318359,383.51358887)
\curveto(596.47317951,383.51358903)(596.42317956,383.51858903)(596.39318359,383.52858887)
\curveto(596.34317964,383.548589)(596.29817968,383.55358899)(596.25818359,383.54358887)
\curveto(596.21817976,383.53358901)(596.1831798,383.53358901)(596.15318359,383.54358887)
\lineto(596.06318359,383.54358887)
\curveto(596.00317998,383.56358898)(595.93818004,383.57858897)(595.86818359,383.58858887)
\curveto(595.80818017,383.58858896)(595.74318024,383.59358895)(595.67318359,383.60358887)
\curveto(595.50318048,383.65358889)(595.34318064,383.70358884)(595.19318359,383.75358887)
\curveto(595.04318094,383.80358874)(594.89818108,383.86858868)(594.75818359,383.94858887)
\curveto(594.70818127,383.98858856)(594.65318133,384.01858853)(594.59318359,384.03858887)
\curveto(594.54318144,384.06858848)(594.49318149,384.10358844)(594.44318359,384.14358887)
\curveto(594.20318178,384.32358822)(594.00318198,384.543588)(593.84318359,384.80358887)
\curveto(593.6831823,385.06358748)(593.54318244,385.3485872)(593.42318359,385.65858887)
\curveto(593.36318262,385.79858675)(593.31818266,385.93858661)(593.28818359,386.07858887)
\curveto(593.25818272,386.22858632)(593.22318276,386.38358616)(593.18318359,386.54358887)
\curveto(593.16318282,386.65358589)(593.14818283,386.76358578)(593.13818359,386.87358887)
\curveto(593.12818285,386.98358556)(593.11318287,387.09358545)(593.09318359,387.20358887)
\curveto(593.0831829,387.2435853)(593.0781829,387.28358526)(593.07818359,387.32358887)
\curveto(593.08818289,387.36358518)(593.08818289,387.40358514)(593.07818359,387.44358887)
\curveto(593.06818291,387.49358505)(593.06318292,387.543585)(593.06318359,387.59358887)
\lineto(593.06318359,387.75858887)
\curveto(593.04318294,387.80858474)(593.03818294,387.85858469)(593.04818359,387.90858887)
\curveto(593.05818292,387.96858458)(593.05818292,388.02358452)(593.04818359,388.07358887)
\curveto(593.03818294,388.11358443)(593.03818294,388.15858439)(593.04818359,388.20858887)
\curveto(593.05818292,388.25858429)(593.05318293,388.30858424)(593.03318359,388.35858887)
\curveto(593.01318297,388.42858412)(593.00818297,388.50358404)(593.01818359,388.58358887)
\curveto(593.02818295,388.67358387)(593.03318295,388.75858379)(593.03318359,388.83858887)
\curveto(593.03318295,388.92858362)(593.02818295,389.02858352)(593.01818359,389.13858887)
\curveto(593.00818297,389.25858329)(593.01318297,389.35858319)(593.03318359,389.43858887)
\lineto(593.03318359,389.72358887)
\lineto(593.07818359,390.35358887)
\curveto(593.08818289,390.45358209)(593.09818288,390.548582)(593.10818359,390.63858887)
\lineto(593.13818359,390.93858887)
\curveto(593.15818282,390.98858156)(593.16318282,391.03858151)(593.15318359,391.08858887)
\curveto(593.15318283,391.1485814)(593.16318282,391.20358134)(593.18318359,391.25358887)
\curveto(593.23318275,391.42358112)(593.27318271,391.58858096)(593.30318359,391.74858887)
\curveto(593.33318265,391.91858063)(593.3831826,392.07858047)(593.45318359,392.22858887)
\curveto(593.64318234,392.68857986)(593.86318212,393.06357948)(594.11318359,393.35358887)
\curveto(594.37318161,393.6435789)(594.73318125,393.88857866)(595.19318359,394.08858887)
\curveto(595.32318066,394.13857841)(595.45318053,394.17357837)(595.58318359,394.19358887)
\curveto(595.72318026,394.21357833)(595.86318012,394.23857831)(596.00318359,394.26858887)
\curveto(596.07317991,394.27857827)(596.13817984,394.28357826)(596.19818359,394.28358887)
\curveto(596.25817972,394.28357826)(596.32317966,394.28857826)(596.39318359,394.29858887)
\curveto(597.22317876,394.31857823)(597.89317809,394.16857838)(598.40318359,393.84858887)
\curveto(598.91317707,393.53857901)(599.29317669,393.09857945)(599.54318359,392.52858887)
\curveto(599.59317639,392.40858014)(599.63817634,392.28358026)(599.67818359,392.15358887)
\curveto(599.71817626,392.02358052)(599.76317622,391.88858066)(599.81318359,391.74858887)
\curveto(599.83317615,391.66858088)(599.84817613,391.58358096)(599.85818359,391.49358887)
\lineto(599.91818359,391.25358887)
\curveto(599.94817603,391.1435814)(599.96317602,391.03358151)(599.96318359,390.92358887)
\curveto(599.97317601,390.81358173)(599.98817599,390.70358184)(600.00818359,390.59358887)
\curveto(600.02817595,390.543582)(600.03317595,390.49858205)(600.02318359,390.45858887)
\curveto(600.02317596,390.41858213)(600.02817595,390.37858217)(600.03818359,390.33858887)
\curveto(600.04817593,390.28858226)(600.04817593,390.23358231)(600.03818359,390.17358887)
\curveto(600.03817594,390.12358242)(600.04317594,390.07358247)(600.05318359,390.02358887)
\lineto(600.05318359,389.88858887)
\curveto(600.07317591,389.82858272)(600.07317591,389.75858279)(600.05318359,389.67858887)
\curveto(600.04317594,389.60858294)(600.04817593,389.543583)(600.06818359,389.48358887)
\curveto(600.0781759,389.45358309)(600.0831759,389.41358313)(600.08318359,389.36358887)
\lineto(600.08318359,389.24358887)
\lineto(600.08318359,388.77858887)
\moveto(598.53818359,386.45358887)
\curveto(598.63817734,386.77358577)(598.69817728,387.13858541)(598.71818359,387.54858887)
\curveto(598.73817724,387.95858459)(598.74817723,388.36858418)(598.74818359,388.77858887)
\curveto(598.74817723,389.20858334)(598.73817724,389.62858292)(598.71818359,390.03858887)
\curveto(598.69817728,390.4485821)(598.65317733,390.83358171)(598.58318359,391.19358887)
\curveto(598.51317747,391.55358099)(598.40317758,391.87358067)(598.25318359,392.15358887)
\curveto(598.11317787,392.4435801)(597.91817806,392.67857987)(597.66818359,392.85858887)
\curveto(597.50817847,392.96857958)(597.32817865,393.0485795)(597.12818359,393.09858887)
\curveto(596.92817905,393.15857939)(596.6831793,393.18857936)(596.39318359,393.18858887)
\curveto(596.37317961,393.16857938)(596.33817964,393.15857939)(596.28818359,393.15858887)
\curveto(596.23817974,393.16857938)(596.19817978,393.16857938)(596.16818359,393.15858887)
\curveto(596.08817989,393.13857941)(596.01317997,393.11857943)(595.94318359,393.09858887)
\curveto(595.8831801,393.08857946)(595.81818016,393.06857948)(595.74818359,393.03858887)
\curveto(595.4781805,392.91857963)(595.25818072,392.7485798)(595.08818359,392.52858887)
\curveto(594.92818105,392.31858023)(594.79318119,392.07358047)(594.68318359,391.79358887)
\curveto(594.63318135,391.68358086)(594.59318139,391.56358098)(594.56318359,391.43358887)
\curveto(594.54318144,391.31358123)(594.51818146,391.18858136)(594.48818359,391.05858887)
\curveto(594.46818151,391.00858154)(594.45818152,390.95358159)(594.45818359,390.89358887)
\curveto(594.45818152,390.8435817)(594.45318153,390.79358175)(594.44318359,390.74358887)
\curveto(594.43318155,390.65358189)(594.42318156,390.55858199)(594.41318359,390.45858887)
\curveto(594.40318158,390.36858218)(594.39318159,390.27358227)(594.38318359,390.17358887)
\curveto(594.3831816,390.09358245)(594.3781816,390.00858254)(594.36818359,389.91858887)
\lineto(594.36818359,389.67858887)
\lineto(594.36818359,389.49858887)
\curveto(594.35818162,389.46858308)(594.35318163,389.43358311)(594.35318359,389.39358887)
\lineto(594.35318359,389.25858887)
\lineto(594.35318359,388.80858887)
\curveto(594.35318163,388.72858382)(594.34818163,388.6435839)(594.33818359,388.55358887)
\curveto(594.33818164,388.47358407)(594.34818163,388.39858415)(594.36818359,388.32858887)
\lineto(594.36818359,388.05858887)
\curveto(594.36818161,388.03858451)(594.36318162,388.00858454)(594.35318359,387.96858887)
\curveto(594.35318163,387.93858461)(594.35818162,387.91358463)(594.36818359,387.89358887)
\curveto(594.3781816,387.79358475)(594.3831816,387.69358485)(594.38318359,387.59358887)
\curveto(594.39318159,387.50358504)(594.40318158,387.40358514)(594.41318359,387.29358887)
\curveto(594.44318154,387.17358537)(594.45818152,387.0485855)(594.45818359,386.91858887)
\curveto(594.46818151,386.79858575)(594.49318149,386.68358586)(594.53318359,386.57358887)
\curveto(594.61318137,386.27358627)(594.69818128,386.00858654)(594.78818359,385.77858887)
\curveto(594.88818109,385.548587)(595.03318095,385.33358721)(595.22318359,385.13358887)
\curveto(595.43318055,384.93358761)(595.69818028,384.78358776)(596.01818359,384.68358887)
\curveto(596.05817992,384.66358788)(596.09317989,384.65358789)(596.12318359,384.65358887)
\curveto(596.16317982,384.66358788)(596.20817977,384.65858789)(596.25818359,384.63858887)
\curveto(596.29817968,384.62858792)(596.36817961,384.61858793)(596.46818359,384.60858887)
\curveto(596.5781794,384.59858795)(596.66317932,384.60358794)(596.72318359,384.62358887)
\curveto(596.79317919,384.6435879)(596.86317912,384.65358789)(596.93318359,384.65358887)
\curveto(597.00317898,384.66358788)(597.06817891,384.67858787)(597.12818359,384.69858887)
\curveto(597.32817865,384.75858779)(597.50817847,384.8435877)(597.66818359,384.95358887)
\curveto(597.69817828,384.97358757)(597.72317826,384.99358755)(597.74318359,385.01358887)
\lineto(597.80318359,385.07358887)
\curveto(597.84317814,385.09358745)(597.89317809,385.13358741)(597.95318359,385.19358887)
\curveto(598.05317793,385.33358721)(598.13817784,385.46358708)(598.20818359,385.58358887)
\curveto(598.2781777,385.70358684)(598.34817763,385.8485867)(598.41818359,386.01858887)
\curveto(598.44817753,386.08858646)(598.46817751,386.15858639)(598.47818359,386.22858887)
\curveto(598.49817748,386.29858625)(598.51817746,386.37358617)(598.53818359,386.45358887)
}
}
{
\newrgbcolor{curcolor}{0 0 0}
\pscustom[linestyle=none,fillstyle=solid,fillcolor=curcolor]
{
\newpath
\moveto(602.38779297,385.32858887)
\lineto(602.68779297,385.32858887)
\curveto(602.79779091,385.33858721)(602.9027908,385.33858721)(603.00279297,385.32858887)
\curveto(603.11279059,385.32858722)(603.21279049,385.31858723)(603.30279297,385.29858887)
\curveto(603.39279031,385.28858726)(603.46279024,385.26358728)(603.51279297,385.22358887)
\curveto(603.53279017,385.20358734)(603.54779016,385.17358737)(603.55779297,385.13358887)
\curveto(603.57779013,385.09358745)(603.59779011,385.0485875)(603.61779297,384.99858887)
\lineto(603.61779297,384.92358887)
\curveto(603.62779008,384.87358767)(603.62779008,384.81858773)(603.61779297,384.75858887)
\lineto(603.61779297,384.60858887)
\lineto(603.61779297,384.12858887)
\curveto(603.61779009,383.95858859)(603.57779013,383.83858871)(603.49779297,383.76858887)
\curveto(603.42779028,383.71858883)(603.33779037,383.69358885)(603.22779297,383.69358887)
\lineto(602.89779297,383.69358887)
\lineto(602.44779297,383.69358887)
\curveto(602.29779141,383.69358885)(602.18279152,383.72358882)(602.10279297,383.78358887)
\curveto(602.06279164,383.81358873)(602.03279167,383.86358868)(602.01279297,383.93358887)
\curveto(601.99279171,384.01358853)(601.97779173,384.09858845)(601.96779297,384.18858887)
\lineto(601.96779297,384.47358887)
\curveto(601.97779173,384.57358797)(601.98279172,384.65858789)(601.98279297,384.72858887)
\lineto(601.98279297,384.92358887)
\curveto(601.98279172,384.98358756)(601.99279171,385.03858751)(602.01279297,385.08858887)
\curveto(602.05279165,385.19858735)(602.12279158,385.26858728)(602.22279297,385.29858887)
\curveto(602.25279145,385.29858725)(602.3077914,385.30858724)(602.38779297,385.32858887)
}
}
{
\newrgbcolor{curcolor}{0 0 0}
\pscustom[linestyle=none,fillstyle=solid,fillcolor=curcolor]
{
\newpath
\moveto(607.10294922,394.10358887)
\lineto(610.70294922,394.10358887)
\lineto(611.34794922,394.10358887)
\curveto(611.42794269,394.10357844)(611.50294261,394.09857845)(611.57294922,394.08858887)
\curveto(611.64294247,394.08857846)(611.70294241,394.07857847)(611.75294922,394.05858887)
\curveto(611.82294229,394.02857852)(611.87794224,393.96857858)(611.91794922,393.87858887)
\curveto(611.93794218,393.8485787)(611.94794217,393.80857874)(611.94794922,393.75858887)
\lineto(611.94794922,393.62358887)
\curveto(611.95794216,393.51357903)(611.95294216,393.40857914)(611.93294922,393.30858887)
\curveto(611.92294219,393.20857934)(611.88794223,393.13857941)(611.82794922,393.09858887)
\curveto(611.73794238,393.02857952)(611.60294251,392.99357955)(611.42294922,392.99358887)
\curveto(611.24294287,393.00357954)(611.07794304,393.00857954)(610.92794922,393.00858887)
\lineto(608.93294922,393.00858887)
\lineto(608.43794922,393.00858887)
\lineto(608.30294922,393.00858887)
\curveto(608.26294585,393.00857954)(608.22294589,393.00357954)(608.18294922,392.99358887)
\lineto(607.97294922,392.99358887)
\curveto(607.86294625,392.96357958)(607.78294633,392.92357962)(607.73294922,392.87358887)
\curveto(607.68294643,392.83357971)(607.64794647,392.77857977)(607.62794922,392.70858887)
\curveto(607.60794651,392.6485799)(607.59294652,392.57857997)(607.58294922,392.49858887)
\curveto(607.57294654,392.41858013)(607.55294656,392.32858022)(607.52294922,392.22858887)
\curveto(607.47294664,392.02858052)(607.43294668,391.82358072)(607.40294922,391.61358887)
\curveto(607.37294674,391.40358114)(607.33294678,391.19858135)(607.28294922,390.99858887)
\curveto(607.26294685,390.92858162)(607.25294686,390.85858169)(607.25294922,390.78858887)
\curveto(607.25294686,390.72858182)(607.24294687,390.66358188)(607.22294922,390.59358887)
\curveto(607.2129469,390.56358198)(607.20294691,390.52358202)(607.19294922,390.47358887)
\curveto(607.19294692,390.43358211)(607.19794692,390.39358215)(607.20794922,390.35358887)
\curveto(607.22794689,390.30358224)(607.25294686,390.25858229)(607.28294922,390.21858887)
\curveto(607.32294679,390.18858236)(607.38294673,390.18358236)(607.46294922,390.20358887)
\curveto(607.52294659,390.22358232)(607.58294653,390.2485823)(607.64294922,390.27858887)
\curveto(607.70294641,390.31858223)(607.76294635,390.35358219)(607.82294922,390.38358887)
\curveto(607.88294623,390.40358214)(607.93294618,390.41858213)(607.97294922,390.42858887)
\curveto(608.16294595,390.50858204)(608.36794575,390.56358198)(608.58794922,390.59358887)
\curveto(608.8179453,390.62358192)(609.04794507,390.63358191)(609.27794922,390.62358887)
\curveto(609.5179446,390.62358192)(609.74794437,390.59858195)(609.96794922,390.54858887)
\curveto(610.18794393,390.50858204)(610.38794373,390.4485821)(610.56794922,390.36858887)
\curveto(610.6179435,390.3485822)(610.66294345,390.32858222)(610.70294922,390.30858887)
\curveto(610.75294336,390.28858226)(610.80294331,390.26358228)(610.85294922,390.23358887)
\curveto(611.20294291,390.02358252)(611.48294263,389.79358275)(611.69294922,389.54358887)
\curveto(611.9129422,389.29358325)(612.10794201,388.96858358)(612.27794922,388.56858887)
\curveto(612.32794179,388.45858409)(612.36294175,388.3485842)(612.38294922,388.23858887)
\curveto(612.40294171,388.12858442)(612.42794169,388.01358453)(612.45794922,387.89358887)
\curveto(612.46794165,387.86358468)(612.47294164,387.81858473)(612.47294922,387.75858887)
\curveto(612.49294162,387.69858485)(612.50294161,387.62858492)(612.50294922,387.54858887)
\curveto(612.50294161,387.47858507)(612.5129416,387.41358513)(612.53294922,387.35358887)
\lineto(612.53294922,387.18858887)
\curveto(612.54294157,387.13858541)(612.54794157,387.06858548)(612.54794922,386.97858887)
\curveto(612.54794157,386.88858566)(612.53794158,386.81858573)(612.51794922,386.76858887)
\curveto(612.49794162,386.70858584)(612.49294162,386.6485859)(612.50294922,386.58858887)
\curveto(612.5129416,386.53858601)(612.50794161,386.48858606)(612.48794922,386.43858887)
\curveto(612.44794167,386.27858627)(612.4129417,386.12858642)(612.38294922,385.98858887)
\curveto(612.35294176,385.8485867)(612.30794181,385.71358683)(612.24794922,385.58358887)
\curveto(612.08794203,385.21358733)(611.86794225,384.87858767)(611.58794922,384.57858887)
\curveto(611.30794281,384.27858827)(610.98794313,384.0485885)(610.62794922,383.88858887)
\curveto(610.45794366,383.80858874)(610.25794386,383.73358881)(610.02794922,383.66358887)
\curveto(609.9179442,383.62358892)(609.80294431,383.59858895)(609.68294922,383.58858887)
\curveto(609.56294455,383.57858897)(609.44294467,383.55858899)(609.32294922,383.52858887)
\curveto(609.27294484,383.50858904)(609.2179449,383.50858904)(609.15794922,383.52858887)
\curveto(609.09794502,383.53858901)(609.03794508,383.53358901)(608.97794922,383.51358887)
\curveto(608.87794524,383.49358905)(608.77794534,383.49358905)(608.67794922,383.51358887)
\lineto(608.54294922,383.51358887)
\curveto(608.49294562,383.53358901)(608.43294568,383.543589)(608.36294922,383.54358887)
\curveto(608.30294581,383.53358901)(608.24794587,383.53858901)(608.19794922,383.55858887)
\curveto(608.15794596,383.56858898)(608.12294599,383.57358897)(608.09294922,383.57358887)
\curveto(608.06294605,383.57358897)(608.02794609,383.57858897)(607.98794922,383.58858887)
\lineto(607.71794922,383.64858887)
\curveto(607.62794649,383.66858888)(607.54294657,383.69858885)(607.46294922,383.73858887)
\curveto(607.12294699,383.87858867)(606.83294728,384.03358851)(606.59294922,384.20358887)
\curveto(606.35294776,384.38358816)(606.13294798,384.61358793)(605.93294922,384.89358887)
\curveto(605.78294833,385.12358742)(605.66794845,385.36358718)(605.58794922,385.61358887)
\curveto(605.56794855,385.66358688)(605.55794856,385.70858684)(605.55794922,385.74858887)
\curveto(605.55794856,385.79858675)(605.54794857,385.8485867)(605.52794922,385.89858887)
\curveto(605.50794861,385.95858659)(605.49294862,386.03858651)(605.48294922,386.13858887)
\curveto(605.48294863,386.23858631)(605.50294861,386.31358623)(605.54294922,386.36358887)
\curveto(605.59294852,386.4435861)(605.67294844,386.48858606)(605.78294922,386.49858887)
\curveto(605.89294822,386.50858604)(606.00794811,386.51358603)(606.12794922,386.51358887)
\lineto(606.29294922,386.51358887)
\curveto(606.35294776,386.51358603)(606.40794771,386.50358604)(606.45794922,386.48358887)
\curveto(606.54794757,386.46358608)(606.6179475,386.42358612)(606.66794922,386.36358887)
\curveto(606.73794738,386.27358627)(606.78294733,386.16358638)(606.80294922,386.03358887)
\curveto(606.83294728,385.91358663)(606.87794724,385.80858674)(606.93794922,385.71858887)
\curveto(607.12794699,385.37858717)(607.38794673,385.10858744)(607.71794922,384.90858887)
\curveto(607.8179463,384.8485877)(607.92294619,384.79858775)(608.03294922,384.75858887)
\curveto(608.15294596,384.72858782)(608.27294584,384.69358785)(608.39294922,384.65358887)
\curveto(608.56294555,384.60358794)(608.76794535,384.58358796)(609.00794922,384.59358887)
\curveto(609.25794486,384.61358793)(609.45794466,384.6485879)(609.60794922,384.69858887)
\curveto(609.97794414,384.81858773)(610.26794385,384.97858757)(610.47794922,385.17858887)
\curveto(610.69794342,385.38858716)(610.87794324,385.66858688)(611.01794922,386.01858887)
\curveto(611.06794305,386.11858643)(611.09794302,386.22358632)(611.10794922,386.33358887)
\curveto(611.12794299,386.4435861)(611.15294296,386.55858599)(611.18294922,386.67858887)
\lineto(611.18294922,386.78358887)
\curveto(611.19294292,386.82358572)(611.19794292,386.86358568)(611.19794922,386.90358887)
\curveto(611.20794291,386.93358561)(611.20794291,386.96858558)(611.19794922,387.00858887)
\lineto(611.19794922,387.12858887)
\curveto(611.19794292,387.38858516)(611.16794295,387.63358491)(611.10794922,387.86358887)
\curveto(610.99794312,388.21358433)(610.84294327,388.50858404)(610.64294922,388.74858887)
\curveto(610.44294367,388.99858355)(610.18294393,389.19358335)(609.86294922,389.33358887)
\lineto(609.68294922,389.39358887)
\curveto(609.63294448,389.41358313)(609.57294454,389.43358311)(609.50294922,389.45358887)
\curveto(609.45294466,389.47358307)(609.39294472,389.48358306)(609.32294922,389.48358887)
\curveto(609.26294485,389.49358305)(609.19794492,389.50858304)(609.12794922,389.52858887)
\lineto(608.97794922,389.52858887)
\curveto(608.93794518,389.548583)(608.88294523,389.55858299)(608.81294922,389.55858887)
\curveto(608.75294536,389.55858299)(608.69794542,389.548583)(608.64794922,389.52858887)
\lineto(608.54294922,389.52858887)
\curveto(608.5129456,389.52858302)(608.47794564,389.52358302)(608.43794922,389.51358887)
\lineto(608.19794922,389.45358887)
\curveto(608.117946,389.4435831)(608.03794608,389.42358312)(607.95794922,389.39358887)
\curveto(607.7179464,389.29358325)(607.48794663,389.15858339)(607.26794922,388.98858887)
\curveto(607.17794694,388.91858363)(607.09294702,388.8435837)(607.01294922,388.76358887)
\curveto(606.93294718,388.69358385)(606.83294728,388.63858391)(606.71294922,388.59858887)
\curveto(606.62294749,388.56858398)(606.48294763,388.55858399)(606.29294922,388.56858887)
\curveto(606.112948,388.57858397)(605.99294812,388.60358394)(605.93294922,388.64358887)
\curveto(605.88294823,388.68358386)(605.84294827,388.7435838)(605.81294922,388.82358887)
\curveto(605.79294832,388.90358364)(605.79294832,388.98858356)(605.81294922,389.07858887)
\curveto(605.84294827,389.19858335)(605.86294825,389.31858323)(605.87294922,389.43858887)
\curveto(605.89294822,389.56858298)(605.9179482,389.69358285)(605.94794922,389.81358887)
\curveto(605.96794815,389.85358269)(605.97294814,389.88858266)(605.96294922,389.91858887)
\curveto(605.96294815,389.95858259)(605.97294814,390.00358254)(605.99294922,390.05358887)
\curveto(606.0129481,390.1435824)(606.02794809,390.23358231)(606.03794922,390.32358887)
\curveto(606.04794807,390.42358212)(606.06794805,390.51858203)(606.09794922,390.60858887)
\curveto(606.10794801,390.66858188)(606.112948,390.72858182)(606.11294922,390.78858887)
\curveto(606.12294799,390.8485817)(606.13794798,390.90858164)(606.15794922,390.96858887)
\curveto(606.20794791,391.16858138)(606.24294787,391.37358117)(606.26294922,391.58358887)
\curveto(606.29294782,391.80358074)(606.33294778,392.01358053)(606.38294922,392.21358887)
\curveto(606.4129477,392.31358023)(606.43294768,392.41358013)(606.44294922,392.51358887)
\curveto(606.45294766,392.61357993)(606.46794765,392.71357983)(606.48794922,392.81358887)
\curveto(606.49794762,392.8435797)(606.50294761,392.88357966)(606.50294922,392.93358887)
\curveto(606.53294758,393.0435795)(606.55294756,393.1485794)(606.56294922,393.24858887)
\curveto(606.58294753,393.35857919)(606.60794751,393.46857908)(606.63794922,393.57858887)
\curveto(606.65794746,393.65857889)(606.67294744,393.72857882)(606.68294922,393.78858887)
\curveto(606.69294742,393.85857869)(606.7179474,393.91857863)(606.75794922,393.96858887)
\curveto(606.77794734,393.99857855)(606.80794731,394.01857853)(606.84794922,394.02858887)
\curveto(606.88794723,394.0485785)(606.93294718,394.06857848)(606.98294922,394.08858887)
\curveto(607.04294707,394.08857846)(607.08294703,394.09357845)(607.10294922,394.10358887)
}
}
{
\newrgbcolor{curcolor}{0 0 0}
\pscustom[linestyle=none,fillstyle=solid,fillcolor=curcolor]
{
\newpath
\moveto(623.74755859,392.21358887)
\curveto(623.54754829,391.92358062)(623.3375485,391.63858091)(623.11755859,391.35858887)
\curveto(622.90754893,391.07858147)(622.70254914,390.79358175)(622.50255859,390.50358887)
\curveto(621.90254994,389.65358289)(621.29755054,388.81358373)(620.68755859,387.98358887)
\curveto(620.07755176,387.16358538)(619.47255237,386.32858622)(618.87255859,385.47858887)
\lineto(618.36255859,384.75858887)
\lineto(617.85255859,384.06858887)
\curveto(617.77255407,383.95858859)(617.69255415,383.8435887)(617.61255859,383.72358887)
\curveto(617.53255431,383.60358894)(617.4375544,383.50858904)(617.32755859,383.43858887)
\curveto(617.28755455,383.41858913)(617.22255462,383.40358914)(617.13255859,383.39358887)
\curveto(617.05255479,383.37358917)(616.96255488,383.36358918)(616.86255859,383.36358887)
\curveto(616.76255508,383.36358918)(616.66755517,383.36858918)(616.57755859,383.37858887)
\curveto(616.49755534,383.38858916)(616.4375554,383.40858914)(616.39755859,383.43858887)
\curveto(616.36755547,383.45858909)(616.3425555,383.49358905)(616.32255859,383.54358887)
\curveto(616.31255553,383.58358896)(616.31755552,383.62858892)(616.33755859,383.67858887)
\curveto(616.37755546,383.75858879)(616.42255542,383.83358871)(616.47255859,383.90358887)
\curveto(616.53255531,383.98358856)(616.58755525,384.06358848)(616.63755859,384.14358887)
\curveto(616.87755496,384.48358806)(617.12255472,384.81858773)(617.37255859,385.14858887)
\curveto(617.62255422,385.47858707)(617.86255398,385.81358673)(618.09255859,386.15358887)
\curveto(618.25255359,386.37358617)(618.41255343,386.58858596)(618.57255859,386.79858887)
\curveto(618.73255311,387.00858554)(618.89255295,387.22358532)(619.05255859,387.44358887)
\curveto(619.41255243,387.96358458)(619.77755206,388.47358407)(620.14755859,388.97358887)
\curveto(620.51755132,389.47358307)(620.88755095,389.98358256)(621.25755859,390.50358887)
\curveto(621.39755044,390.70358184)(621.5375503,390.89858165)(621.67755859,391.08858887)
\curveto(621.82755001,391.27858127)(621.97254987,391.47358107)(622.11255859,391.67358887)
\curveto(622.32254952,391.97358057)(622.5375493,392.27358027)(622.75755859,392.57358887)
\lineto(623.41755859,393.47358887)
\lineto(623.59755859,393.74358887)
\lineto(623.80755859,394.01358887)
\lineto(623.92755859,394.19358887)
\curveto(623.97754786,394.25357829)(624.02754781,394.30857824)(624.07755859,394.35858887)
\curveto(624.14754769,394.40857814)(624.22254762,394.4435781)(624.30255859,394.46358887)
\curveto(624.32254752,394.47357807)(624.34754749,394.47357807)(624.37755859,394.46358887)
\curveto(624.41754742,394.46357808)(624.44754739,394.47357807)(624.46755859,394.49358887)
\curveto(624.58754725,394.49357805)(624.72254712,394.48857806)(624.87255859,394.47858887)
\curveto(625.02254682,394.47857807)(625.11254673,394.43357811)(625.14255859,394.34358887)
\curveto(625.16254668,394.31357823)(625.16754667,394.27857827)(625.15755859,394.23858887)
\curveto(625.14754669,394.19857835)(625.13254671,394.16857838)(625.11255859,394.14858887)
\curveto(625.07254677,394.06857848)(625.03254681,393.99857855)(624.99255859,393.93858887)
\curveto(624.95254689,393.87857867)(624.90754693,393.81857873)(624.85755859,393.75858887)
\lineto(624.28755859,392.97858887)
\curveto(624.10754773,392.72857982)(623.92754791,392.47358007)(623.74755859,392.21358887)
\moveto(616.89255859,388.31358887)
\curveto(616.842555,388.33358421)(616.79255505,388.33858421)(616.74255859,388.32858887)
\curveto(616.69255515,388.31858423)(616.6425552,388.32358422)(616.59255859,388.34358887)
\curveto(616.48255536,388.36358418)(616.37755546,388.38358416)(616.27755859,388.40358887)
\curveto(616.18755565,388.43358411)(616.09255575,388.47358407)(615.99255859,388.52358887)
\curveto(615.66255618,388.66358388)(615.40755643,388.85858369)(615.22755859,389.10858887)
\curveto(615.04755679,389.36858318)(614.90255694,389.67858287)(614.79255859,390.03858887)
\curveto(614.76255708,390.11858243)(614.7425571,390.19858235)(614.73255859,390.27858887)
\curveto(614.72255712,390.36858218)(614.70755713,390.45358209)(614.68755859,390.53358887)
\curveto(614.67755716,390.58358196)(614.67255717,390.6485819)(614.67255859,390.72858887)
\curveto(614.66255718,390.75858179)(614.65755718,390.78858176)(614.65755859,390.81858887)
\curveto(614.65755718,390.85858169)(614.65255719,390.89358165)(614.64255859,390.92358887)
\lineto(614.64255859,391.07358887)
\curveto(614.63255721,391.12358142)(614.62755721,391.18358136)(614.62755859,391.25358887)
\curveto(614.62755721,391.33358121)(614.63255721,391.39858115)(614.64255859,391.44858887)
\lineto(614.64255859,391.61358887)
\curveto(614.66255718,391.66358088)(614.66755717,391.70858084)(614.65755859,391.74858887)
\curveto(614.65755718,391.79858075)(614.66255718,391.8435807)(614.67255859,391.88358887)
\curveto(614.68255716,391.92358062)(614.68755715,391.95858059)(614.68755859,391.98858887)
\curveto(614.68755715,392.02858052)(614.69255715,392.06858048)(614.70255859,392.10858887)
\curveto(614.73255711,392.21858033)(614.75255709,392.32858022)(614.76255859,392.43858887)
\curveto(614.78255706,392.55857999)(614.81755702,392.67357987)(614.86755859,392.78358887)
\curveto(615.00755683,393.12357942)(615.16755667,393.39857915)(615.34755859,393.60858887)
\curveto(615.5375563,393.82857872)(615.80755603,394.00857854)(616.15755859,394.14858887)
\curveto(616.2375556,394.17857837)(616.32255552,394.19857835)(616.41255859,394.20858887)
\curveto(616.50255534,394.22857832)(616.59755524,394.2485783)(616.69755859,394.26858887)
\curveto(616.72755511,394.27857827)(616.78255506,394.27857827)(616.86255859,394.26858887)
\curveto(616.9425549,394.26857828)(616.99255485,394.27857827)(617.01255859,394.29858887)
\curveto(617.57255427,394.30857824)(618.02255382,394.19857835)(618.36255859,393.96858887)
\curveto(618.71255313,393.73857881)(618.97255287,393.43357911)(619.14255859,393.05358887)
\curveto(619.18255266,392.96357958)(619.21755262,392.86857968)(619.24755859,392.76858887)
\curveto(619.27755256,392.66857988)(619.30255254,392.56857998)(619.32255859,392.46858887)
\curveto(619.3425525,392.43858011)(619.34755249,392.40858014)(619.33755859,392.37858887)
\curveto(619.3375525,392.3485802)(619.3425525,392.31858023)(619.35255859,392.28858887)
\curveto(619.38255246,392.17858037)(619.40255244,392.05358049)(619.41255859,391.91358887)
\curveto(619.42255242,391.78358076)(619.43255241,391.6485809)(619.44255859,391.50858887)
\lineto(619.44255859,391.34358887)
\curveto(619.45255239,391.28358126)(619.45255239,391.22858132)(619.44255859,391.17858887)
\curveto(619.43255241,391.12858142)(619.42755241,391.07858147)(619.42755859,391.02858887)
\lineto(619.42755859,390.89358887)
\curveto(619.41755242,390.85358169)(619.41255243,390.81358173)(619.41255859,390.77358887)
\curveto(619.42255242,390.73358181)(619.41755242,390.68858186)(619.39755859,390.63858887)
\curveto(619.37755246,390.52858202)(619.35755248,390.42358212)(619.33755859,390.32358887)
\curveto(619.32755251,390.22358232)(619.30755253,390.12358242)(619.27755859,390.02358887)
\curveto(619.14755269,389.66358288)(618.98255286,389.3485832)(618.78255859,389.07858887)
\curveto(618.58255326,388.80858374)(618.30755353,388.60358394)(617.95755859,388.46358887)
\curveto(617.87755396,388.43358411)(617.79255405,388.40858414)(617.70255859,388.38858887)
\lineto(617.43255859,388.32858887)
\curveto(617.38255446,388.31858423)(617.3375545,388.31358423)(617.29755859,388.31358887)
\curveto(617.25755458,388.32358422)(617.21755462,388.32358422)(617.17755859,388.31358887)
\curveto(617.07755476,388.29358425)(616.98255486,388.29358425)(616.89255859,388.31358887)
\moveto(616.05255859,389.70858887)
\curveto(616.09255575,389.63858291)(616.13255571,389.57358297)(616.17255859,389.51358887)
\curveto(616.21255563,389.46358308)(616.26255558,389.41358313)(616.32255859,389.36358887)
\lineto(616.47255859,389.24358887)
\curveto(616.53255531,389.21358333)(616.59755524,389.18858336)(616.66755859,389.16858887)
\curveto(616.70755513,389.1485834)(616.7425551,389.13858341)(616.77255859,389.13858887)
\curveto(616.81255503,389.1485834)(616.85255499,389.1435834)(616.89255859,389.12358887)
\curveto(616.92255492,389.12358342)(616.96255488,389.11858343)(617.01255859,389.10858887)
\curveto(617.06255478,389.10858344)(617.10255474,389.11358343)(617.13255859,389.12358887)
\lineto(617.35755859,389.16858887)
\curveto(617.60755423,389.2485833)(617.79255405,389.37358317)(617.91255859,389.54358887)
\curveto(617.99255385,389.6435829)(618.06255378,389.77358277)(618.12255859,389.93358887)
\curveto(618.20255364,390.11358243)(618.26255358,390.33858221)(618.30255859,390.60858887)
\curveto(618.3425535,390.88858166)(618.35755348,391.16858138)(618.34755859,391.44858887)
\curveto(618.3375535,391.73858081)(618.30755353,392.01358053)(618.25755859,392.27358887)
\curveto(618.20755363,392.53358001)(618.13255371,392.7435798)(618.03255859,392.90358887)
\curveto(617.91255393,393.10357944)(617.76255408,393.25357929)(617.58255859,393.35358887)
\curveto(617.50255434,393.40357914)(617.41255443,393.43357911)(617.31255859,393.44358887)
\curveto(617.21255463,393.46357908)(617.10755473,393.47357907)(616.99755859,393.47358887)
\curveto(616.97755486,393.46357908)(616.95255489,393.45857909)(616.92255859,393.45858887)
\curveto(616.90255494,393.46857908)(616.88255496,393.46857908)(616.86255859,393.45858887)
\curveto(616.81255503,393.4485791)(616.76755507,393.43857911)(616.72755859,393.42858887)
\curveto(616.68755515,393.42857912)(616.64755519,393.41857913)(616.60755859,393.39858887)
\curveto(616.42755541,393.31857923)(616.27755556,393.19857935)(616.15755859,393.03858887)
\curveto(616.04755579,392.87857967)(615.95755588,392.69857985)(615.88755859,392.49858887)
\curveto(615.82755601,392.30858024)(615.78255606,392.08358046)(615.75255859,391.82358887)
\curveto(615.73255611,391.56358098)(615.72755611,391.29858125)(615.73755859,391.02858887)
\curveto(615.74755609,390.76858178)(615.77755606,390.51858203)(615.82755859,390.27858887)
\curveto(615.88755595,390.0485825)(615.96255588,389.85858269)(616.05255859,389.70858887)
\moveto(626.85255859,386.72358887)
\curveto(626.86254498,386.67358587)(626.86754497,386.58358596)(626.86755859,386.45358887)
\curveto(626.86754497,386.32358622)(626.85754498,386.23358631)(626.83755859,386.18358887)
\curveto(626.81754502,386.13358641)(626.81254503,386.07858647)(626.82255859,386.01858887)
\curveto(626.83254501,385.96858658)(626.83254501,385.91858663)(626.82255859,385.86858887)
\curveto(626.78254506,385.72858682)(626.75254509,385.59358695)(626.73255859,385.46358887)
\curveto(626.72254512,385.33358721)(626.69254515,385.21358733)(626.64255859,385.10358887)
\curveto(626.50254534,384.75358779)(626.3375455,384.45858809)(626.14755859,384.21858887)
\curveto(625.95754588,383.98858856)(625.68754615,383.80358874)(625.33755859,383.66358887)
\curveto(625.25754658,383.63358891)(625.17254667,383.61358893)(625.08255859,383.60358887)
\curveto(624.99254685,383.58358896)(624.90754693,383.56358898)(624.82755859,383.54358887)
\curveto(624.77754706,383.53358901)(624.72754711,383.52858902)(624.67755859,383.52858887)
\curveto(624.62754721,383.52858902)(624.57754726,383.52358902)(624.52755859,383.51358887)
\curveto(624.49754734,383.50358904)(624.44754739,383.50358904)(624.37755859,383.51358887)
\curveto(624.30754753,383.51358903)(624.25754758,383.51858903)(624.22755859,383.52858887)
\curveto(624.16754767,383.548589)(624.10754773,383.55858899)(624.04755859,383.55858887)
\curveto(623.99754784,383.548589)(623.94754789,383.55358899)(623.89755859,383.57358887)
\curveto(623.80754803,383.59358895)(623.71754812,383.61858893)(623.62755859,383.64858887)
\curveto(623.54754829,383.66858888)(623.46754837,383.69858885)(623.38755859,383.73858887)
\curveto(623.06754877,383.87858867)(622.81754902,384.07358847)(622.63755859,384.32358887)
\curveto(622.45754938,384.58358796)(622.30754953,384.88858766)(622.18755859,385.23858887)
\curveto(622.16754967,385.31858723)(622.15254969,385.40358714)(622.14255859,385.49358887)
\curveto(622.13254971,385.58358696)(622.11754972,385.66858688)(622.09755859,385.74858887)
\curveto(622.08754975,385.77858677)(622.08254976,385.80858674)(622.08255859,385.83858887)
\lineto(622.08255859,385.94358887)
\curveto(622.06254978,386.02358652)(622.05254979,386.10358644)(622.05255859,386.18358887)
\lineto(622.05255859,386.31858887)
\curveto(622.03254981,386.41858613)(622.03254981,386.51858603)(622.05255859,386.61858887)
\lineto(622.05255859,386.79858887)
\curveto(622.06254978,386.8485857)(622.06754977,386.89358565)(622.06755859,386.93358887)
\curveto(622.06754977,386.98358556)(622.07254977,387.02858552)(622.08255859,387.06858887)
\curveto(622.09254975,387.10858544)(622.09754974,387.1435854)(622.09755859,387.17358887)
\curveto(622.09754974,387.21358533)(622.10254974,387.25358529)(622.11255859,387.29358887)
\lineto(622.17255859,387.62358887)
\curveto(622.19254965,387.7435848)(622.22254962,387.85358469)(622.26255859,387.95358887)
\curveto(622.40254944,388.28358426)(622.56254928,388.55858399)(622.74255859,388.77858887)
\curveto(622.93254891,389.00858354)(623.19254865,389.19358335)(623.52255859,389.33358887)
\curveto(623.60254824,389.37358317)(623.68754815,389.39858315)(623.77755859,389.40858887)
\lineto(624.07755859,389.46858887)
\lineto(624.21255859,389.46858887)
\curveto(624.26254758,389.47858307)(624.31254753,389.48358306)(624.36255859,389.48358887)
\curveto(624.93254691,389.50358304)(625.39254645,389.39858315)(625.74255859,389.16858887)
\curveto(626.10254574,388.9485836)(626.36754547,388.6485839)(626.53755859,388.26858887)
\curveto(626.58754525,388.16858438)(626.62754521,388.06858448)(626.65755859,387.96858887)
\curveto(626.68754515,387.86858468)(626.71754512,387.76358478)(626.74755859,387.65358887)
\curveto(626.75754508,387.61358493)(626.76254508,387.57858497)(626.76255859,387.54858887)
\curveto(626.76254508,387.52858502)(626.76754507,387.49858505)(626.77755859,387.45858887)
\curveto(626.79754504,387.38858516)(626.80754503,387.31358523)(626.80755859,387.23358887)
\curveto(626.80754503,387.15358539)(626.81754502,387.07358547)(626.83755859,386.99358887)
\curveto(626.837545,386.9435856)(626.837545,386.89858565)(626.83755859,386.85858887)
\curveto(626.837545,386.81858573)(626.842545,386.77358577)(626.85255859,386.72358887)
\moveto(625.74255859,386.28858887)
\curveto(625.75254609,386.33858621)(625.75754608,386.41358613)(625.75755859,386.51358887)
\curveto(625.76754607,386.61358593)(625.76254608,386.68858586)(625.74255859,386.73858887)
\curveto(625.72254612,386.79858575)(625.71754612,386.85358569)(625.72755859,386.90358887)
\curveto(625.74754609,386.96358558)(625.74754609,387.02358552)(625.72755859,387.08358887)
\curveto(625.71754612,387.11358543)(625.71254613,387.1485854)(625.71255859,387.18858887)
\curveto(625.71254613,387.22858532)(625.70754613,387.26858528)(625.69755859,387.30858887)
\curveto(625.67754616,387.38858516)(625.65754618,387.46358508)(625.63755859,387.53358887)
\curveto(625.62754621,387.61358493)(625.61254623,387.69358485)(625.59255859,387.77358887)
\curveto(625.56254628,387.83358471)(625.5375463,387.89358465)(625.51755859,387.95358887)
\curveto(625.49754634,388.01358453)(625.46754637,388.07358447)(625.42755859,388.13358887)
\curveto(625.32754651,388.30358424)(625.19754664,388.43858411)(625.03755859,388.53858887)
\curveto(624.95754688,388.58858396)(624.86254698,388.62358392)(624.75255859,388.64358887)
\curveto(624.6425472,388.66358388)(624.51754732,388.67358387)(624.37755859,388.67358887)
\curveto(624.35754748,388.66358388)(624.33254751,388.65858389)(624.30255859,388.65858887)
\curveto(624.27254757,388.66858388)(624.2425476,388.66858388)(624.21255859,388.65858887)
\lineto(624.06255859,388.59858887)
\curveto(624.01254783,388.58858396)(623.96754787,388.57358397)(623.92755859,388.55358887)
\curveto(623.7375481,388.4435841)(623.59254825,388.29858425)(623.49255859,388.11858887)
\curveto(623.40254844,387.93858461)(623.32254852,387.73358481)(623.25255859,387.50358887)
\curveto(623.21254863,387.37358517)(623.19254865,387.23858531)(623.19255859,387.09858887)
\curveto(623.19254865,386.96858558)(623.18254866,386.82358572)(623.16255859,386.66358887)
\curveto(623.15254869,386.61358593)(623.1425487,386.55358599)(623.13255859,386.48358887)
\curveto(623.13254871,386.41358613)(623.1425487,386.35358619)(623.16255859,386.30358887)
\lineto(623.16255859,386.13858887)
\lineto(623.16255859,385.95858887)
\curveto(623.17254867,385.90858664)(623.18254866,385.85358669)(623.19255859,385.79358887)
\curveto(623.20254864,385.7435868)(623.20754863,385.68858686)(623.20755859,385.62858887)
\curveto(623.21754862,385.56858698)(623.23254861,385.51358703)(623.25255859,385.46358887)
\curveto(623.30254854,385.27358727)(623.36254848,385.09858745)(623.43255859,384.93858887)
\curveto(623.50254834,384.77858777)(623.60754823,384.6485879)(623.74755859,384.54858887)
\curveto(623.87754796,384.4485881)(624.01754782,384.37858817)(624.16755859,384.33858887)
\curveto(624.19754764,384.32858822)(624.22254762,384.32358822)(624.24255859,384.32358887)
\curveto(624.27254757,384.33358821)(624.30254754,384.33358821)(624.33255859,384.32358887)
\curveto(624.35254749,384.32358822)(624.38254746,384.31858823)(624.42255859,384.30858887)
\curveto(624.46254738,384.30858824)(624.49754734,384.31358823)(624.52755859,384.32358887)
\curveto(624.56754727,384.33358821)(624.60754723,384.33858821)(624.64755859,384.33858887)
\curveto(624.68754715,384.33858821)(624.72754711,384.3485882)(624.76755859,384.36858887)
\curveto(625.00754683,384.4485881)(625.20254664,384.58358796)(625.35255859,384.77358887)
\curveto(625.47254637,384.95358759)(625.56254628,385.15858739)(625.62255859,385.38858887)
\curveto(625.6425462,385.45858709)(625.65754618,385.52858702)(625.66755859,385.59858887)
\curveto(625.67754616,385.67858687)(625.69254615,385.75858679)(625.71255859,385.83858887)
\curveto(625.71254613,385.89858665)(625.71754612,385.9435866)(625.72755859,385.97358887)
\curveto(625.72754611,385.99358655)(625.72754611,386.01858653)(625.72755859,386.04858887)
\curveto(625.72754611,386.08858646)(625.73254611,386.11858643)(625.74255859,386.13858887)
\lineto(625.74255859,386.28858887)
}
}
{
\newrgbcolor{curcolor}{0 0 0}
\pscustom[linestyle=none,fillstyle=solid,fillcolor=curcolor]
{
\newpath
\moveto(209.53076721,734.28858887)
\curveto(209.53075958,734.20858334)(209.53575957,734.12858342)(209.54576721,734.04858887)
\curveto(209.55575955,733.96858358)(209.55075956,733.89358365)(209.53076721,733.82358887)
\curveto(209.5107596,733.78358376)(209.5057596,733.73858381)(209.51576721,733.68858887)
\curveto(209.52575958,733.6485839)(209.52575958,733.60858394)(209.51576721,733.56858887)
\lineto(209.51576721,733.41858887)
\curveto(209.5057596,733.32858422)(209.50075961,733.23858431)(209.50076721,733.14858887)
\curveto(209.50075961,733.06858448)(209.49575961,732.98858456)(209.48576721,732.90858887)
\lineto(209.45576721,732.66858887)
\curveto(209.44575966,732.59858495)(209.43575967,732.52358502)(209.42576721,732.44358887)
\curveto(209.41575969,732.40358514)(209.4107597,732.36358518)(209.41076721,732.32358887)
\curveto(209.4107597,732.28358526)(209.4057597,732.23858531)(209.39576721,732.18858887)
\curveto(209.35575975,732.0485855)(209.32575978,731.90858564)(209.30576721,731.76858887)
\curveto(209.29575981,731.62858592)(209.26575984,731.49358605)(209.21576721,731.36358887)
\curveto(209.16575994,731.19358635)(209.11076,731.02858652)(209.05076721,730.86858887)
\curveto(209.00076011,730.70858684)(208.94076017,730.55358699)(208.87076721,730.40358887)
\curveto(208.85076026,730.3435872)(208.82076029,730.28358726)(208.78076721,730.22358887)
\lineto(208.69076721,730.07358887)
\curveto(208.49076062,729.75358779)(208.27576083,729.48858806)(208.04576721,729.27858887)
\curveto(207.81576129,729.06858848)(207.52076159,728.88858866)(207.16076721,728.73858887)
\curveto(207.04076207,728.68858886)(206.9107622,728.65358889)(206.77076721,728.63358887)
\curveto(206.64076247,728.61358893)(206.5057626,728.58858896)(206.36576721,728.55858887)
\curveto(206.3057628,728.548589)(206.24576286,728.543589)(206.18576721,728.54358887)
\curveto(206.12576298,728.543589)(206.06076305,728.53858901)(205.99076721,728.52858887)
\curveto(205.96076315,728.51858903)(205.9107632,728.51858903)(205.84076721,728.52858887)
\lineto(205.69076721,728.52858887)
\lineto(205.54076721,728.52858887)
\curveto(205.46076365,728.548589)(205.37576373,728.56358898)(205.28576721,728.57358887)
\curveto(205.2057639,728.57358897)(205.13076398,728.58358896)(205.06076721,728.60358887)
\curveto(205.02076409,728.61358893)(204.98576412,728.61858893)(204.95576721,728.61858887)
\curveto(204.93576417,728.60858894)(204.9107642,728.61358893)(204.88076721,728.63358887)
\lineto(204.61076721,728.69358887)
\curveto(204.52076459,728.72358882)(204.43576467,728.75358879)(204.35576721,728.78358887)
\curveto(203.77576533,729.02358852)(203.34076577,729.39358815)(203.05076721,729.89358887)
\curveto(202.97076614,730.02358752)(202.9057662,730.15858739)(202.85576721,730.29858887)
\curveto(202.81576629,730.43858711)(202.77076634,730.58858696)(202.72076721,730.74858887)
\curveto(202.70076641,730.82858672)(202.69576641,730.90858664)(202.70576721,730.98858887)
\curveto(202.72576638,731.06858648)(202.76076635,731.12358642)(202.81076721,731.15358887)
\curveto(202.84076627,731.17358637)(202.89576621,731.18858636)(202.97576721,731.19858887)
\curveto(203.05576605,731.21858633)(203.14076597,731.22858632)(203.23076721,731.22858887)
\curveto(203.32076579,731.23858631)(203.4057657,731.23858631)(203.48576721,731.22858887)
\curveto(203.57576553,731.21858633)(203.64576546,731.20858634)(203.69576721,731.19858887)
\curveto(203.71576539,731.18858636)(203.74076537,731.17358637)(203.77076721,731.15358887)
\curveto(203.8107653,731.13358641)(203.84076527,731.11358643)(203.86076721,731.09358887)
\curveto(203.92076519,731.01358653)(203.96576514,730.91858663)(203.99576721,730.80858887)
\curveto(204.03576507,730.69858685)(204.08076503,730.59858695)(204.13076721,730.50858887)
\curveto(204.38076473,730.11858743)(204.75076436,729.8485877)(205.24076721,729.69858887)
\curveto(205.3107638,729.67858787)(205.38076373,729.66358788)(205.45076721,729.65358887)
\curveto(205.53076358,729.65358789)(205.6107635,729.6435879)(205.69076721,729.62358887)
\curveto(205.73076338,729.61358793)(205.78576332,729.60858794)(205.85576721,729.60858887)
\curveto(205.93576317,729.60858794)(205.99076312,729.61358793)(206.02076721,729.62358887)
\curveto(206.05076306,729.63358791)(206.08076303,729.63858791)(206.11076721,729.63858887)
\lineto(206.21576721,729.63858887)
\curveto(206.29576281,729.65858789)(206.37076274,729.67858787)(206.44076721,729.69858887)
\curveto(206.52076259,729.71858783)(206.59576251,729.7435878)(206.66576721,729.77358887)
\curveto(207.01576209,729.92358762)(207.28576182,730.13858741)(207.47576721,730.41858887)
\curveto(207.66576144,730.69858685)(207.82076129,731.02358652)(207.94076721,731.39358887)
\curveto(207.97076114,731.47358607)(207.99076112,731.548586)(208.00076721,731.61858887)
\curveto(208.02076109,731.68858586)(208.04076107,731.76358578)(208.06076721,731.84358887)
\curveto(208.08076103,731.93358561)(208.09576101,732.02858552)(208.10576721,732.12858887)
\curveto(208.12576098,732.23858531)(208.14576096,732.3435852)(208.16576721,732.44358887)
\curveto(208.17576093,732.49358505)(208.18076093,732.543585)(208.18076721,732.59358887)
\curveto(208.19076092,732.65358489)(208.19576091,732.70858484)(208.19576721,732.75858887)
\curveto(208.21576089,732.81858473)(208.22576088,732.89358465)(208.22576721,732.98358887)
\curveto(208.22576088,733.08358446)(208.21576089,733.16358438)(208.19576721,733.22358887)
\curveto(208.16576094,733.31358423)(208.11576099,733.35358419)(208.04576721,733.34358887)
\curveto(207.98576112,733.33358421)(207.93076118,733.30358424)(207.88076721,733.25358887)
\curveto(207.80076131,733.20358434)(207.73076138,733.1435844)(207.67076721,733.07358887)
\curveto(207.62076149,733.00358454)(207.55576155,732.9435846)(207.47576721,732.89358887)
\curveto(207.31576179,732.78358476)(207.15076196,732.68358486)(206.98076721,732.59358887)
\curveto(206.8107623,732.51358503)(206.61576249,732.4435851)(206.39576721,732.38358887)
\curveto(206.29576281,732.35358519)(206.19576291,732.33858521)(206.09576721,732.33858887)
\curveto(206.0057631,732.33858521)(205.9057632,732.32858522)(205.79576721,732.30858887)
\lineto(205.64576721,732.30858887)
\curveto(205.59576351,732.32858522)(205.54576356,732.33358521)(205.49576721,732.32358887)
\curveto(205.45576365,732.31358523)(205.41576369,732.31358523)(205.37576721,732.32358887)
\curveto(205.34576376,732.33358521)(205.30076381,732.33858521)(205.24076721,732.33858887)
\curveto(205.18076393,732.3485852)(205.11576399,732.35858519)(205.04576721,732.36858887)
\lineto(204.86576721,732.39858887)
\curveto(204.41576469,732.51858503)(204.03576507,732.68358486)(203.72576721,732.89358887)
\curveto(203.45576565,733.08358446)(203.22576588,733.31358423)(203.03576721,733.58358887)
\curveto(202.85576625,733.86358368)(202.7107664,734.17858337)(202.60076721,734.52858887)
\lineto(202.54076721,734.73858887)
\curveto(202.53076658,734.81858273)(202.51576659,734.89858265)(202.49576721,734.97858887)
\curveto(202.48576662,735.00858254)(202.48076663,735.03858251)(202.48076721,735.06858887)
\curveto(202.48076663,735.09858245)(202.47576663,735.12858242)(202.46576721,735.15858887)
\curveto(202.45576665,735.21858233)(202.45076666,735.27858227)(202.45076721,735.33858887)
\curveto(202.45076666,735.40858214)(202.44076667,735.46858208)(202.42076721,735.51858887)
\lineto(202.42076721,735.69858887)
\curveto(202.4107667,735.7485818)(202.4057667,735.81858173)(202.40576721,735.90858887)
\curveto(202.4057667,735.99858155)(202.41576669,736.06858148)(202.43576721,736.11858887)
\lineto(202.43576721,736.28358887)
\curveto(202.45576665,736.36358118)(202.46576664,736.43858111)(202.46576721,736.50858887)
\curveto(202.47576663,736.57858097)(202.49076662,736.6485809)(202.51076721,736.71858887)
\curveto(202.57076654,736.91858063)(202.63076648,737.10858044)(202.69076721,737.28858887)
\curveto(202.76076635,737.46858008)(202.85076626,737.63857991)(202.96076721,737.79858887)
\curveto(203.00076611,737.86857968)(203.04076607,737.93357961)(203.08076721,737.99358887)
\lineto(203.23076721,738.17358887)
\curveto(203.25076586,738.18357936)(203.27076584,738.19857935)(203.29076721,738.21858887)
\curveto(203.38076573,738.3485792)(203.49076562,738.45857909)(203.62076721,738.54858887)
\curveto(203.88076523,738.7485788)(204.14576496,738.90357864)(204.41576721,739.01358887)
\curveto(204.49576461,739.05357849)(204.57576453,739.08357846)(204.65576721,739.10358887)
\curveto(204.74576436,739.13357841)(204.83576427,739.15857839)(204.92576721,739.17858887)
\curveto(205.02576408,739.20857834)(205.12576398,739.22857832)(205.22576721,739.23858887)
\curveto(205.32576378,739.2485783)(205.43076368,739.26357828)(205.54076721,739.28358887)
\curveto(205.57076354,739.29357825)(205.6107635,739.29357825)(205.66076721,739.28358887)
\curveto(205.72076339,739.27357827)(205.76076335,739.27857827)(205.78076721,739.29858887)
\curveto(206.50076261,739.31857823)(207.10076201,739.20357834)(207.58076721,738.95358887)
\curveto(208.06076105,738.70357884)(208.43576067,738.36357918)(208.70576721,737.93358887)
\curveto(208.79576031,737.79357975)(208.87576023,737.6485799)(208.94576721,737.49858887)
\curveto(209.01576009,737.3485802)(209.08576002,737.18858036)(209.15576721,737.01858887)
\curveto(209.2057599,736.87858067)(209.24575986,736.72858082)(209.27576721,736.56858887)
\curveto(209.3057598,736.40858114)(209.34075977,736.2485813)(209.38076721,736.08858887)
\curveto(209.40075971,736.03858151)(209.4107597,735.98358156)(209.41076721,735.92358887)
\curveto(209.4107597,735.87358167)(209.41575969,735.82358172)(209.42576721,735.77358887)
\curveto(209.44575966,735.71358183)(209.45575965,735.6485819)(209.45576721,735.57858887)
\curveto(209.45575965,735.51858203)(209.46575964,735.46358208)(209.48576721,735.41358887)
\lineto(209.48576721,735.24858887)
\curveto(209.5057596,735.19858235)(209.5107596,735.1485824)(209.50076721,735.09858887)
\curveto(209.49075962,735.0485825)(209.49575961,734.99858255)(209.51576721,734.94858887)
\curveto(209.51575959,734.92858262)(209.5107596,734.90358264)(209.50076721,734.87358887)
\curveto(209.50075961,734.8435827)(209.5057596,734.81858273)(209.51576721,734.79858887)
\curveto(209.52575958,734.76858278)(209.52575958,734.73358281)(209.51576721,734.69358887)
\curveto(209.51575959,734.65358289)(209.52075959,734.61358293)(209.53076721,734.57358887)
\curveto(209.54075957,734.53358301)(209.54075957,734.48858306)(209.53076721,734.43858887)
\lineto(209.53076721,734.28858887)
\moveto(208.03076721,735.59358887)
\curveto(208.04076107,735.6435819)(208.04576106,735.70358184)(208.04576721,735.77358887)
\curveto(208.04576106,735.8435817)(208.04076107,735.90358164)(208.03076721,735.95358887)
\curveto(208.02076109,736.00358154)(208.01576109,736.07858147)(208.01576721,736.17858887)
\curveto(207.99576111,736.25858129)(207.97576113,736.33358121)(207.95576721,736.40358887)
\curveto(207.94576116,736.47358107)(207.93076118,736.543581)(207.91076721,736.61358887)
\curveto(207.77076134,737.0435805)(207.57576153,737.37858017)(207.32576721,737.61858887)
\curveto(207.08576202,737.85857969)(206.74076237,738.03857951)(206.29076721,738.15858887)
\curveto(206.20076291,738.17857937)(206.10076301,738.18857936)(205.99076721,738.18858887)
\lineto(205.66076721,738.18858887)
\curveto(205.64076347,738.16857938)(205.6057635,738.15857939)(205.55576721,738.15858887)
\curveto(205.5057636,738.16857938)(205.46076365,738.16857938)(205.42076721,738.15858887)
\curveto(205.34076377,738.13857941)(205.26576384,738.11857943)(205.19576721,738.09858887)
\lineto(204.98576721,738.03858887)
\curveto(204.69576441,737.90857964)(204.46576464,737.72857982)(204.29576721,737.49858887)
\curveto(204.12576498,737.27858027)(203.99076512,737.01858053)(203.89076721,736.71858887)
\curveto(203.86076525,736.62858092)(203.83576527,736.53358101)(203.81576721,736.43358887)
\curveto(203.8057653,736.3435812)(203.79076532,736.2485813)(203.77076721,736.14858887)
\lineto(203.77076721,736.01358887)
\curveto(203.74076537,735.90358164)(203.73076538,735.76358178)(203.74076721,735.59358887)
\curveto(203.76076535,735.43358211)(203.78076533,735.30358224)(203.80076721,735.20358887)
\curveto(203.82076529,735.1435824)(203.83576527,735.08358246)(203.84576721,735.02358887)
\curveto(203.85576525,734.97358257)(203.87076524,734.92358262)(203.89076721,734.87358887)
\curveto(203.97076514,734.67358287)(204.06576504,734.48358306)(204.17576721,734.30358887)
\curveto(204.29576481,734.12358342)(204.43576467,733.97858357)(204.59576721,733.86858887)
\curveto(204.64576446,733.81858373)(204.70076441,733.77858377)(204.76076721,733.74858887)
\curveto(204.82076429,733.71858383)(204.88076423,733.68358386)(204.94076721,733.64358887)
\curveto(205.09076402,733.56358398)(205.27576383,733.49858405)(205.49576721,733.44858887)
\curveto(205.54576356,733.42858412)(205.58576352,733.42358412)(205.61576721,733.43358887)
\curveto(205.65576345,733.4435841)(205.70076341,733.43858411)(205.75076721,733.41858887)
\curveto(205.79076332,733.40858414)(205.84576326,733.40358414)(205.91576721,733.40358887)
\curveto(205.98576312,733.40358414)(206.04576306,733.40858414)(206.09576721,733.41858887)
\curveto(206.19576291,733.43858411)(206.29076282,733.45358409)(206.38076721,733.46358887)
\curveto(206.47076264,733.48358406)(206.56076255,733.51358403)(206.65076721,733.55358887)
\curveto(207.19076192,733.77358377)(207.58576152,734.16858338)(207.83576721,734.73858887)
\curveto(207.88576122,734.83858271)(207.92076119,734.93858261)(207.94076721,735.03858887)
\curveto(207.96076115,735.1485824)(207.98576112,735.25858229)(208.01576721,735.36858887)
\curveto(208.01576109,735.46858208)(208.02076109,735.543582)(208.03076721,735.59358887)
}
}
{
\newrgbcolor{curcolor}{0 0 0}
\pscustom[linestyle=none,fillstyle=solid,fillcolor=curcolor]
{
\newpath
\moveto(212.45037659,739.10358887)
\lineto(216.05037659,739.10358887)
\lineto(216.69537659,739.10358887)
\curveto(216.77537006,739.10357844)(216.85036998,739.09857845)(216.92037659,739.08858887)
\curveto(216.99036984,739.08857846)(217.05036978,739.07857847)(217.10037659,739.05858887)
\curveto(217.17036966,739.02857852)(217.22536961,738.96857858)(217.26537659,738.87858887)
\curveto(217.28536955,738.8485787)(217.29536954,738.80857874)(217.29537659,738.75858887)
\lineto(217.29537659,738.62358887)
\curveto(217.30536953,738.51357903)(217.30036953,738.40857914)(217.28037659,738.30858887)
\curveto(217.27036956,738.20857934)(217.2353696,738.13857941)(217.17537659,738.09858887)
\curveto(217.08536975,738.02857952)(216.95036988,737.99357955)(216.77037659,737.99358887)
\curveto(216.59037024,738.00357954)(216.42537041,738.00857954)(216.27537659,738.00858887)
\lineto(214.28037659,738.00858887)
\lineto(213.78537659,738.00858887)
\lineto(213.65037659,738.00858887)
\curveto(213.61037322,738.00857954)(213.57037326,738.00357954)(213.53037659,737.99358887)
\lineto(213.32037659,737.99358887)
\curveto(213.21037362,737.96357958)(213.1303737,737.92357962)(213.08037659,737.87358887)
\curveto(213.0303738,737.83357971)(212.99537384,737.77857977)(212.97537659,737.70858887)
\curveto(212.95537388,737.6485799)(212.94037389,737.57857997)(212.93037659,737.49858887)
\curveto(212.92037391,737.41858013)(212.90037393,737.32858022)(212.87037659,737.22858887)
\curveto(212.82037401,737.02858052)(212.78037405,736.82358072)(212.75037659,736.61358887)
\curveto(212.72037411,736.40358114)(212.68037415,736.19858135)(212.63037659,735.99858887)
\curveto(212.61037422,735.92858162)(212.60037423,735.85858169)(212.60037659,735.78858887)
\curveto(212.60037423,735.72858182)(212.59037424,735.66358188)(212.57037659,735.59358887)
\curveto(212.56037427,735.56358198)(212.55037428,735.52358202)(212.54037659,735.47358887)
\curveto(212.54037429,735.43358211)(212.54537429,735.39358215)(212.55537659,735.35358887)
\curveto(212.57537426,735.30358224)(212.60037423,735.25858229)(212.63037659,735.21858887)
\curveto(212.67037416,735.18858236)(212.7303741,735.18358236)(212.81037659,735.20358887)
\curveto(212.87037396,735.22358232)(212.9303739,735.2485823)(212.99037659,735.27858887)
\curveto(213.05037378,735.31858223)(213.11037372,735.35358219)(213.17037659,735.38358887)
\curveto(213.2303736,735.40358214)(213.28037355,735.41858213)(213.32037659,735.42858887)
\curveto(213.51037332,735.50858204)(213.71537312,735.56358198)(213.93537659,735.59358887)
\curveto(214.16537267,735.62358192)(214.39537244,735.63358191)(214.62537659,735.62358887)
\curveto(214.86537197,735.62358192)(215.09537174,735.59858195)(215.31537659,735.54858887)
\curveto(215.5353713,735.50858204)(215.7353711,735.4485821)(215.91537659,735.36858887)
\curveto(215.96537087,735.3485822)(216.01037082,735.32858222)(216.05037659,735.30858887)
\curveto(216.10037073,735.28858226)(216.15037068,735.26358228)(216.20037659,735.23358887)
\curveto(216.55037028,735.02358252)(216.83037,734.79358275)(217.04037659,734.54358887)
\curveto(217.26036957,734.29358325)(217.45536938,733.96858358)(217.62537659,733.56858887)
\curveto(217.67536916,733.45858409)(217.71036912,733.3485842)(217.73037659,733.23858887)
\curveto(217.75036908,733.12858442)(217.77536906,733.01358453)(217.80537659,732.89358887)
\curveto(217.81536902,732.86358468)(217.82036901,732.81858473)(217.82037659,732.75858887)
\curveto(217.84036899,732.69858485)(217.85036898,732.62858492)(217.85037659,732.54858887)
\curveto(217.85036898,732.47858507)(217.86036897,732.41358513)(217.88037659,732.35358887)
\lineto(217.88037659,732.18858887)
\curveto(217.89036894,732.13858541)(217.89536894,732.06858548)(217.89537659,731.97858887)
\curveto(217.89536894,731.88858566)(217.88536895,731.81858573)(217.86537659,731.76858887)
\curveto(217.84536899,731.70858584)(217.84036899,731.6485859)(217.85037659,731.58858887)
\curveto(217.86036897,731.53858601)(217.85536898,731.48858606)(217.83537659,731.43858887)
\curveto(217.79536904,731.27858627)(217.76036907,731.12858642)(217.73037659,730.98858887)
\curveto(217.70036913,730.8485867)(217.65536918,730.71358683)(217.59537659,730.58358887)
\curveto(217.4353694,730.21358733)(217.21536962,729.87858767)(216.93537659,729.57858887)
\curveto(216.65537018,729.27858827)(216.3353705,729.0485885)(215.97537659,728.88858887)
\curveto(215.80537103,728.80858874)(215.60537123,728.73358881)(215.37537659,728.66358887)
\curveto(215.26537157,728.62358892)(215.15037168,728.59858895)(215.03037659,728.58858887)
\curveto(214.91037192,728.57858897)(214.79037204,728.55858899)(214.67037659,728.52858887)
\curveto(214.62037221,728.50858904)(214.56537227,728.50858904)(214.50537659,728.52858887)
\curveto(214.44537239,728.53858901)(214.38537245,728.53358901)(214.32537659,728.51358887)
\curveto(214.22537261,728.49358905)(214.12537271,728.49358905)(214.02537659,728.51358887)
\lineto(213.89037659,728.51358887)
\curveto(213.84037299,728.53358901)(213.78037305,728.543589)(213.71037659,728.54358887)
\curveto(213.65037318,728.53358901)(213.59537324,728.53858901)(213.54537659,728.55858887)
\curveto(213.50537333,728.56858898)(213.47037336,728.57358897)(213.44037659,728.57358887)
\curveto(213.41037342,728.57358897)(213.37537346,728.57858897)(213.33537659,728.58858887)
\lineto(213.06537659,728.64858887)
\curveto(212.97537386,728.66858888)(212.89037394,728.69858885)(212.81037659,728.73858887)
\curveto(212.47037436,728.87858867)(212.18037465,729.03358851)(211.94037659,729.20358887)
\curveto(211.70037513,729.38358816)(211.48037535,729.61358793)(211.28037659,729.89358887)
\curveto(211.1303757,730.12358742)(211.01537582,730.36358718)(210.93537659,730.61358887)
\curveto(210.91537592,730.66358688)(210.90537593,730.70858684)(210.90537659,730.74858887)
\curveto(210.90537593,730.79858675)(210.89537594,730.8485867)(210.87537659,730.89858887)
\curveto(210.85537598,730.95858659)(210.84037599,731.03858651)(210.83037659,731.13858887)
\curveto(210.830376,731.23858631)(210.85037598,731.31358623)(210.89037659,731.36358887)
\curveto(210.94037589,731.4435861)(211.02037581,731.48858606)(211.13037659,731.49858887)
\curveto(211.24037559,731.50858604)(211.35537548,731.51358603)(211.47537659,731.51358887)
\lineto(211.64037659,731.51358887)
\curveto(211.70037513,731.51358603)(211.75537508,731.50358604)(211.80537659,731.48358887)
\curveto(211.89537494,731.46358608)(211.96537487,731.42358612)(212.01537659,731.36358887)
\curveto(212.08537475,731.27358627)(212.1303747,731.16358638)(212.15037659,731.03358887)
\curveto(212.18037465,730.91358663)(212.22537461,730.80858674)(212.28537659,730.71858887)
\curveto(212.47537436,730.37858717)(212.7353741,730.10858744)(213.06537659,729.90858887)
\curveto(213.16537367,729.8485877)(213.27037356,729.79858775)(213.38037659,729.75858887)
\curveto(213.50037333,729.72858782)(213.62037321,729.69358785)(213.74037659,729.65358887)
\curveto(213.91037292,729.60358794)(214.11537272,729.58358796)(214.35537659,729.59358887)
\curveto(214.60537223,729.61358793)(214.80537203,729.6485879)(214.95537659,729.69858887)
\curveto(215.32537151,729.81858773)(215.61537122,729.97858757)(215.82537659,730.17858887)
\curveto(216.04537079,730.38858716)(216.22537061,730.66858688)(216.36537659,731.01858887)
\curveto(216.41537042,731.11858643)(216.44537039,731.22358632)(216.45537659,731.33358887)
\curveto(216.47537036,731.4435861)(216.50037033,731.55858599)(216.53037659,731.67858887)
\lineto(216.53037659,731.78358887)
\curveto(216.54037029,731.82358572)(216.54537029,731.86358568)(216.54537659,731.90358887)
\curveto(216.55537028,731.93358561)(216.55537028,731.96858558)(216.54537659,732.00858887)
\lineto(216.54537659,732.12858887)
\curveto(216.54537029,732.38858516)(216.51537032,732.63358491)(216.45537659,732.86358887)
\curveto(216.34537049,733.21358433)(216.19037064,733.50858404)(215.99037659,733.74858887)
\curveto(215.79037104,733.99858355)(215.5303713,734.19358335)(215.21037659,734.33358887)
\lineto(215.03037659,734.39358887)
\curveto(214.98037185,734.41358313)(214.92037191,734.43358311)(214.85037659,734.45358887)
\curveto(214.80037203,734.47358307)(214.74037209,734.48358306)(214.67037659,734.48358887)
\curveto(214.61037222,734.49358305)(214.54537229,734.50858304)(214.47537659,734.52858887)
\lineto(214.32537659,734.52858887)
\curveto(214.28537255,734.548583)(214.2303726,734.55858299)(214.16037659,734.55858887)
\curveto(214.10037273,734.55858299)(214.04537279,734.548583)(213.99537659,734.52858887)
\lineto(213.89037659,734.52858887)
\curveto(213.86037297,734.52858302)(213.82537301,734.52358302)(213.78537659,734.51358887)
\lineto(213.54537659,734.45358887)
\curveto(213.46537337,734.4435831)(213.38537345,734.42358312)(213.30537659,734.39358887)
\curveto(213.06537377,734.29358325)(212.835374,734.15858339)(212.61537659,733.98858887)
\curveto(212.52537431,733.91858363)(212.44037439,733.8435837)(212.36037659,733.76358887)
\curveto(212.28037455,733.69358385)(212.18037465,733.63858391)(212.06037659,733.59858887)
\curveto(211.97037486,733.56858398)(211.830375,733.55858399)(211.64037659,733.56858887)
\curveto(211.46037537,733.57858397)(211.34037549,733.60358394)(211.28037659,733.64358887)
\curveto(211.2303756,733.68358386)(211.19037564,733.7435838)(211.16037659,733.82358887)
\curveto(211.14037569,733.90358364)(211.14037569,733.98858356)(211.16037659,734.07858887)
\curveto(211.19037564,734.19858335)(211.21037562,734.31858323)(211.22037659,734.43858887)
\curveto(211.24037559,734.56858298)(211.26537557,734.69358285)(211.29537659,734.81358887)
\curveto(211.31537552,734.85358269)(211.32037551,734.88858266)(211.31037659,734.91858887)
\curveto(211.31037552,734.95858259)(211.32037551,735.00358254)(211.34037659,735.05358887)
\curveto(211.36037547,735.1435824)(211.37537546,735.23358231)(211.38537659,735.32358887)
\curveto(211.39537544,735.42358212)(211.41537542,735.51858203)(211.44537659,735.60858887)
\curveto(211.45537538,735.66858188)(211.46037537,735.72858182)(211.46037659,735.78858887)
\curveto(211.47037536,735.8485817)(211.48537535,735.90858164)(211.50537659,735.96858887)
\curveto(211.55537528,736.16858138)(211.59037524,736.37358117)(211.61037659,736.58358887)
\curveto(211.64037519,736.80358074)(211.68037515,737.01358053)(211.73037659,737.21358887)
\curveto(211.76037507,737.31358023)(211.78037505,737.41358013)(211.79037659,737.51358887)
\curveto(211.80037503,737.61357993)(211.81537502,737.71357983)(211.83537659,737.81358887)
\curveto(211.84537499,737.8435797)(211.85037498,737.88357966)(211.85037659,737.93358887)
\curveto(211.88037495,738.0435795)(211.90037493,738.1485794)(211.91037659,738.24858887)
\curveto(211.9303749,738.35857919)(211.95537488,738.46857908)(211.98537659,738.57858887)
\curveto(212.00537483,738.65857889)(212.02037481,738.72857882)(212.03037659,738.78858887)
\curveto(212.04037479,738.85857869)(212.06537477,738.91857863)(212.10537659,738.96858887)
\curveto(212.12537471,738.99857855)(212.15537468,739.01857853)(212.19537659,739.02858887)
\curveto(212.2353746,739.0485785)(212.28037455,739.06857848)(212.33037659,739.08858887)
\curveto(212.39037444,739.08857846)(212.4303744,739.09357845)(212.45037659,739.10358887)
}
}
{
\newrgbcolor{curcolor}{0 0 0}
\pscustom[linestyle=none,fillstyle=solid,fillcolor=curcolor]
{
\newpath
\moveto(220.24498596,730.32858887)
\lineto(220.54498596,730.32858887)
\curveto(220.6549839,730.33858721)(220.7599838,730.33858721)(220.85998596,730.32858887)
\curveto(220.96998359,730.32858722)(221.06998349,730.31858723)(221.15998596,730.29858887)
\curveto(221.24998331,730.28858726)(221.31998324,730.26358728)(221.36998596,730.22358887)
\curveto(221.38998317,730.20358734)(221.40498315,730.17358737)(221.41498596,730.13358887)
\curveto(221.43498312,730.09358745)(221.4549831,730.0485875)(221.47498596,729.99858887)
\lineto(221.47498596,729.92358887)
\curveto(221.48498307,729.87358767)(221.48498307,729.81858773)(221.47498596,729.75858887)
\lineto(221.47498596,729.60858887)
\lineto(221.47498596,729.12858887)
\curveto(221.47498308,728.95858859)(221.43498312,728.83858871)(221.35498596,728.76858887)
\curveto(221.28498327,728.71858883)(221.19498336,728.69358885)(221.08498596,728.69358887)
\lineto(220.75498596,728.69358887)
\lineto(220.30498596,728.69358887)
\curveto(220.1549844,728.69358885)(220.03998452,728.72358882)(219.95998596,728.78358887)
\curveto(219.91998464,728.81358873)(219.88998467,728.86358868)(219.86998596,728.93358887)
\curveto(219.84998471,729.01358853)(219.83498472,729.09858845)(219.82498596,729.18858887)
\lineto(219.82498596,729.47358887)
\curveto(219.83498472,729.57358797)(219.83998472,729.65858789)(219.83998596,729.72858887)
\lineto(219.83998596,729.92358887)
\curveto(219.83998472,729.98358756)(219.84998471,730.03858751)(219.86998596,730.08858887)
\curveto(219.90998465,730.19858735)(219.97998458,730.26858728)(220.07998596,730.29858887)
\curveto(220.10998445,730.29858725)(220.16498439,730.30858724)(220.24498596,730.32858887)
}
}
{
\newrgbcolor{curcolor}{0 0 0}
\pscustom[linestyle=none,fillstyle=solid,fillcolor=curcolor]
{
\newpath
\moveto(230.33014221,732.18858887)
\curveto(230.40013457,732.13858541)(230.44013453,732.06858548)(230.45014221,731.97858887)
\curveto(230.4701345,731.88858566)(230.48013449,731.78358576)(230.48014221,731.66358887)
\curveto(230.48013449,731.61358593)(230.47513449,731.56358598)(230.46514221,731.51358887)
\curveto(230.4651345,731.46358608)(230.45513451,731.41858613)(230.43514221,731.37858887)
\curveto(230.40513456,731.28858626)(230.34513462,731.22858632)(230.25514221,731.19858887)
\curveto(230.17513479,731.17858637)(230.08013489,731.16858638)(229.97014221,731.16858887)
\lineto(229.65514221,731.16858887)
\curveto(229.54513542,731.17858637)(229.44013553,731.16858638)(229.34014221,731.13858887)
\curveto(229.20013577,731.10858644)(229.11013586,731.02858652)(229.07014221,730.89858887)
\curveto(229.05013592,730.82858672)(229.04013593,730.7435868)(229.04014221,730.64358887)
\lineto(229.04014221,730.37358887)
\lineto(229.04014221,729.42858887)
\lineto(229.04014221,729.09858887)
\curveto(229.04013593,728.98858856)(229.02013595,728.90358864)(228.98014221,728.84358887)
\curveto(228.94013603,728.78358876)(228.89013608,728.7435888)(228.83014221,728.72358887)
\curveto(228.78013619,728.71358883)(228.71513625,728.69858885)(228.63514221,728.67858887)
\lineto(228.44014221,728.67858887)
\curveto(228.32013665,728.67858887)(228.21513675,728.68358886)(228.12514221,728.69358887)
\curveto(228.03513693,728.71358883)(227.965137,728.76358878)(227.91514221,728.84358887)
\curveto(227.88513708,728.89358865)(227.8701371,728.96358858)(227.87014221,729.05358887)
\lineto(227.87014221,729.35358887)
\lineto(227.87014221,730.38858887)
\curveto(227.8701371,730.548587)(227.86013711,730.69358685)(227.84014221,730.82358887)
\curveto(227.83013714,730.96358658)(227.77513719,731.05858649)(227.67514221,731.10858887)
\curveto(227.62513734,731.12858642)(227.55513741,731.1435864)(227.46514221,731.15358887)
\curveto(227.38513758,731.16358638)(227.29513767,731.16858638)(227.19514221,731.16858887)
\lineto(226.91014221,731.16858887)
\lineto(226.67014221,731.16858887)
\lineto(224.40514221,731.16858887)
\curveto(224.31514065,731.16858638)(224.21014076,731.16358638)(224.09014221,731.15358887)
\lineto(223.76014221,731.15358887)
\curveto(223.65014132,731.15358639)(223.55014142,731.16358638)(223.46014221,731.18358887)
\curveto(223.3701416,731.20358634)(223.31014166,731.23858631)(223.28014221,731.28858887)
\curveto(223.23014174,731.35858619)(223.20514176,731.45358609)(223.20514221,731.57358887)
\lineto(223.20514221,731.91858887)
\lineto(223.20514221,732.18858887)
\curveto(223.24514172,732.35858519)(223.30014167,732.49858505)(223.37014221,732.60858887)
\curveto(223.44014153,732.71858483)(223.52014145,732.83358471)(223.61014221,732.95358887)
\lineto(223.97014221,733.49358887)
\curveto(224.41014056,734.12358342)(224.84514012,734.7435828)(225.27514221,735.35358887)
\lineto(226.59514221,737.21358887)
\curveto(226.75513821,737.4435801)(226.91013806,737.66357988)(227.06014221,737.87358887)
\curveto(227.21013776,738.09357945)(227.3651376,738.31857923)(227.52514221,738.54858887)
\curveto(227.57513739,738.61857893)(227.62513734,738.68357886)(227.67514221,738.74358887)
\curveto(227.72513724,738.81357873)(227.77513719,738.88857866)(227.82514221,738.96858887)
\lineto(227.88514221,739.05858887)
\curveto(227.91513705,739.09857845)(227.94513702,739.12857842)(227.97514221,739.14858887)
\curveto(228.01513695,739.17857837)(228.05513691,739.19857835)(228.09514221,739.20858887)
\curveto(228.13513683,739.22857832)(228.18013679,739.2485783)(228.23014221,739.26858887)
\curveto(228.25013672,739.26857828)(228.2701367,739.26357828)(228.29014221,739.25358887)
\curveto(228.32013665,739.25357829)(228.34513662,739.26357828)(228.36514221,739.28358887)
\curveto(228.49513647,739.28357826)(228.61513635,739.27857827)(228.72514221,739.26858887)
\curveto(228.83513613,739.25857829)(228.91513605,739.21357833)(228.96514221,739.13358887)
\curveto(229.00513596,739.08357846)(229.02513594,739.01357853)(229.02514221,738.92358887)
\curveto(229.03513593,738.83357871)(229.04013593,738.73857881)(229.04014221,738.63858887)
\lineto(229.04014221,733.17858887)
\curveto(229.04013593,733.10858444)(229.03513593,733.03358451)(229.02514221,732.95358887)
\curveto(229.02513594,732.88358466)(229.03013594,732.81358473)(229.04014221,732.74358887)
\lineto(229.04014221,732.63858887)
\curveto(229.06013591,732.58858496)(229.07513589,732.53358501)(229.08514221,732.47358887)
\curveto(229.09513587,732.42358512)(229.12013585,732.38358516)(229.16014221,732.35358887)
\curveto(229.23013574,732.30358524)(229.31513565,732.27358527)(229.41514221,732.26358887)
\lineto(229.74514221,732.26358887)
\curveto(229.85513511,732.26358528)(229.96013501,732.25858529)(230.06014221,732.24858887)
\curveto(230.1701348,732.2485853)(230.26013471,732.22858532)(230.33014221,732.18858887)
\moveto(227.76514221,732.38358887)
\curveto(227.84513712,732.49358505)(227.88013709,732.66358488)(227.87014221,732.89358887)
\lineto(227.87014221,733.50858887)
\lineto(227.87014221,735.98358887)
\lineto(227.87014221,736.29858887)
\curveto(227.88013709,736.41858113)(227.87513709,736.51858103)(227.85514221,736.59858887)
\lineto(227.85514221,736.74858887)
\curveto(227.85513711,736.83858071)(227.84013713,736.92358062)(227.81014221,737.00358887)
\curveto(227.80013717,737.02358052)(227.79013718,737.03358051)(227.78014221,737.03358887)
\lineto(227.73514221,737.07858887)
\curveto(227.71513725,737.08858046)(227.68513728,737.09358045)(227.64514221,737.09358887)
\curveto(227.62513734,737.07358047)(227.60513736,737.05858049)(227.58514221,737.04858887)
\curveto(227.57513739,737.0485805)(227.56013741,737.0435805)(227.54014221,737.03358887)
\curveto(227.48013749,736.98358056)(227.42013755,736.91358063)(227.36014221,736.82358887)
\curveto(227.30013767,736.73358081)(227.24513772,736.65358089)(227.19514221,736.58358887)
\curveto(227.09513787,736.4435811)(227.00013797,736.29858125)(226.91014221,736.14858887)
\curveto(226.82013815,736.00858154)(226.72513824,735.86858168)(226.62514221,735.72858887)
\lineto(226.08514221,734.94858887)
\curveto(225.91513905,734.68858286)(225.74013923,734.42858312)(225.56014221,734.16858887)
\curveto(225.48013949,734.05858349)(225.40513956,733.95358359)(225.33514221,733.85358887)
\lineto(225.12514221,733.55358887)
\curveto(225.07513989,733.47358407)(225.02513994,733.39858415)(224.97514221,733.32858887)
\curveto(224.93514003,733.25858429)(224.89014008,733.18358436)(224.84014221,733.10358887)
\curveto(224.79014018,733.0435845)(224.74014023,732.97858457)(224.69014221,732.90858887)
\curveto(224.65014032,732.8485847)(224.61014036,732.77858477)(224.57014221,732.69858887)
\curveto(224.53014044,732.63858491)(224.50514046,732.56858498)(224.49514221,732.48858887)
\curveto(224.48514048,732.41858513)(224.52014045,732.36358518)(224.60014221,732.32358887)
\curveto(224.6701403,732.27358527)(224.78014019,732.2485853)(224.93014221,732.24858887)
\curveto(225.09013988,732.25858529)(225.22513974,732.26358528)(225.33514221,732.26358887)
\lineto(227.01514221,732.26358887)
\lineto(227.45014221,732.26358887)
\curveto(227.60013737,732.26358528)(227.70513726,732.30358524)(227.76514221,732.38358887)
}
}
{
\newrgbcolor{curcolor}{0 0 0}
\pscustom[linestyle=none,fillstyle=solid,fillcolor=curcolor]
{
\newpath
\moveto(241.60475159,737.21358887)
\curveto(241.40474129,736.92358062)(241.1947415,736.63858091)(240.97475159,736.35858887)
\curveto(240.76474193,736.07858147)(240.55974213,735.79358175)(240.35975159,735.50358887)
\curveto(239.75974293,734.65358289)(239.15474354,733.81358373)(238.54475159,732.98358887)
\curveto(237.93474476,732.16358538)(237.32974536,731.32858622)(236.72975159,730.47858887)
\lineto(236.21975159,729.75858887)
\lineto(235.70975159,729.06858887)
\curveto(235.62974706,728.95858859)(235.54974714,728.8435887)(235.46975159,728.72358887)
\curveto(235.3897473,728.60358894)(235.2947474,728.50858904)(235.18475159,728.43858887)
\curveto(235.14474755,728.41858913)(235.07974761,728.40358914)(234.98975159,728.39358887)
\curveto(234.90974778,728.37358917)(234.81974787,728.36358918)(234.71975159,728.36358887)
\curveto(234.61974807,728.36358918)(234.52474817,728.36858918)(234.43475159,728.37858887)
\curveto(234.35474834,728.38858916)(234.2947484,728.40858914)(234.25475159,728.43858887)
\curveto(234.22474847,728.45858909)(234.19974849,728.49358905)(234.17975159,728.54358887)
\curveto(234.16974852,728.58358896)(234.17474852,728.62858892)(234.19475159,728.67858887)
\curveto(234.23474846,728.75858879)(234.27974841,728.83358871)(234.32975159,728.90358887)
\curveto(234.3897483,728.98358856)(234.44474825,729.06358848)(234.49475159,729.14358887)
\curveto(234.73474796,729.48358806)(234.97974771,729.81858773)(235.22975159,730.14858887)
\curveto(235.47974721,730.47858707)(235.71974697,730.81358673)(235.94975159,731.15358887)
\curveto(236.10974658,731.37358617)(236.26974642,731.58858596)(236.42975159,731.79858887)
\curveto(236.5897461,732.00858554)(236.74974594,732.22358532)(236.90975159,732.44358887)
\curveto(237.26974542,732.96358458)(237.63474506,733.47358407)(238.00475159,733.97358887)
\curveto(238.37474432,734.47358307)(238.74474395,734.98358256)(239.11475159,735.50358887)
\curveto(239.25474344,735.70358184)(239.3947433,735.89858165)(239.53475159,736.08858887)
\curveto(239.68474301,736.27858127)(239.82974286,736.47358107)(239.96975159,736.67358887)
\curveto(240.17974251,736.97358057)(240.3947423,737.27358027)(240.61475159,737.57358887)
\lineto(241.27475159,738.47358887)
\lineto(241.45475159,738.74358887)
\lineto(241.66475159,739.01358887)
\lineto(241.78475159,739.19358887)
\curveto(241.83474086,739.25357829)(241.88474081,739.30857824)(241.93475159,739.35858887)
\curveto(242.00474069,739.40857814)(242.07974061,739.4435781)(242.15975159,739.46358887)
\curveto(242.17974051,739.47357807)(242.20474049,739.47357807)(242.23475159,739.46358887)
\curveto(242.27474042,739.46357808)(242.30474039,739.47357807)(242.32475159,739.49358887)
\curveto(242.44474025,739.49357805)(242.57974011,739.48857806)(242.72975159,739.47858887)
\curveto(242.87973981,739.47857807)(242.96973972,739.43357811)(242.99975159,739.34358887)
\curveto(243.01973967,739.31357823)(243.02473967,739.27857827)(243.01475159,739.23858887)
\curveto(243.00473969,739.19857835)(242.9897397,739.16857838)(242.96975159,739.14858887)
\curveto(242.92973976,739.06857848)(242.8897398,738.99857855)(242.84975159,738.93858887)
\curveto(242.80973988,738.87857867)(242.76473993,738.81857873)(242.71475159,738.75858887)
\lineto(242.14475159,737.97858887)
\curveto(241.96474073,737.72857982)(241.78474091,737.47358007)(241.60475159,737.21358887)
\moveto(234.74975159,733.31358887)
\curveto(234.69974799,733.33358421)(234.64974804,733.33858421)(234.59975159,733.32858887)
\curveto(234.54974814,733.31858423)(234.49974819,733.32358422)(234.44975159,733.34358887)
\curveto(234.33974835,733.36358418)(234.23474846,733.38358416)(234.13475159,733.40358887)
\curveto(234.04474865,733.43358411)(233.94974874,733.47358407)(233.84975159,733.52358887)
\curveto(233.51974917,733.66358388)(233.26474943,733.85858369)(233.08475159,734.10858887)
\curveto(232.90474979,734.36858318)(232.75974993,734.67858287)(232.64975159,735.03858887)
\curveto(232.61975007,735.11858243)(232.59975009,735.19858235)(232.58975159,735.27858887)
\curveto(232.57975011,735.36858218)(232.56475013,735.45358209)(232.54475159,735.53358887)
\curveto(232.53475016,735.58358196)(232.52975016,735.6485819)(232.52975159,735.72858887)
\curveto(232.51975017,735.75858179)(232.51475018,735.78858176)(232.51475159,735.81858887)
\curveto(232.51475018,735.85858169)(232.50975018,735.89358165)(232.49975159,735.92358887)
\lineto(232.49975159,736.07358887)
\curveto(232.4897502,736.12358142)(232.48475021,736.18358136)(232.48475159,736.25358887)
\curveto(232.48475021,736.33358121)(232.4897502,736.39858115)(232.49975159,736.44858887)
\lineto(232.49975159,736.61358887)
\curveto(232.51975017,736.66358088)(232.52475017,736.70858084)(232.51475159,736.74858887)
\curveto(232.51475018,736.79858075)(232.51975017,736.8435807)(232.52975159,736.88358887)
\curveto(232.53975015,736.92358062)(232.54475015,736.95858059)(232.54475159,736.98858887)
\curveto(232.54475015,737.02858052)(232.54975014,737.06858048)(232.55975159,737.10858887)
\curveto(232.5897501,737.21858033)(232.60975008,737.32858022)(232.61975159,737.43858887)
\curveto(232.63975005,737.55857999)(232.67475002,737.67357987)(232.72475159,737.78358887)
\curveto(232.86474983,738.12357942)(233.02474967,738.39857915)(233.20475159,738.60858887)
\curveto(233.3947493,738.82857872)(233.66474903,739.00857854)(234.01475159,739.14858887)
\curveto(234.0947486,739.17857837)(234.17974851,739.19857835)(234.26975159,739.20858887)
\curveto(234.35974833,739.22857832)(234.45474824,739.2485783)(234.55475159,739.26858887)
\curveto(234.58474811,739.27857827)(234.63974805,739.27857827)(234.71975159,739.26858887)
\curveto(234.79974789,739.26857828)(234.84974784,739.27857827)(234.86975159,739.29858887)
\curveto(235.42974726,739.30857824)(235.87974681,739.19857835)(236.21975159,738.96858887)
\curveto(236.56974612,738.73857881)(236.82974586,738.43357911)(236.99975159,738.05358887)
\curveto(237.03974565,737.96357958)(237.07474562,737.86857968)(237.10475159,737.76858887)
\curveto(237.13474556,737.66857988)(237.15974553,737.56857998)(237.17975159,737.46858887)
\curveto(237.19974549,737.43858011)(237.20474549,737.40858014)(237.19475159,737.37858887)
\curveto(237.1947455,737.3485802)(237.19974549,737.31858023)(237.20975159,737.28858887)
\curveto(237.23974545,737.17858037)(237.25974543,737.05358049)(237.26975159,736.91358887)
\curveto(237.27974541,736.78358076)(237.2897454,736.6485809)(237.29975159,736.50858887)
\lineto(237.29975159,736.34358887)
\curveto(237.30974538,736.28358126)(237.30974538,736.22858132)(237.29975159,736.17858887)
\curveto(237.2897454,736.12858142)(237.28474541,736.07858147)(237.28475159,736.02858887)
\lineto(237.28475159,735.89358887)
\curveto(237.27474542,735.85358169)(237.26974542,735.81358173)(237.26975159,735.77358887)
\curveto(237.27974541,735.73358181)(237.27474542,735.68858186)(237.25475159,735.63858887)
\curveto(237.23474546,735.52858202)(237.21474548,735.42358212)(237.19475159,735.32358887)
\curveto(237.18474551,735.22358232)(237.16474553,735.12358242)(237.13475159,735.02358887)
\curveto(237.00474569,734.66358288)(236.83974585,734.3485832)(236.63975159,734.07858887)
\curveto(236.43974625,733.80858374)(236.16474653,733.60358394)(235.81475159,733.46358887)
\curveto(235.73474696,733.43358411)(235.64974704,733.40858414)(235.55975159,733.38858887)
\lineto(235.28975159,733.32858887)
\curveto(235.23974745,733.31858423)(235.1947475,733.31358423)(235.15475159,733.31358887)
\curveto(235.11474758,733.32358422)(235.07474762,733.32358422)(235.03475159,733.31358887)
\curveto(234.93474776,733.29358425)(234.83974785,733.29358425)(234.74975159,733.31358887)
\moveto(233.90975159,734.70858887)
\curveto(233.94974874,734.63858291)(233.9897487,734.57358297)(234.02975159,734.51358887)
\curveto(234.06974862,734.46358308)(234.11974857,734.41358313)(234.17975159,734.36358887)
\lineto(234.32975159,734.24358887)
\curveto(234.3897483,734.21358333)(234.45474824,734.18858336)(234.52475159,734.16858887)
\curveto(234.56474813,734.1485834)(234.59974809,734.13858341)(234.62975159,734.13858887)
\curveto(234.66974802,734.1485834)(234.70974798,734.1435834)(234.74975159,734.12358887)
\curveto(234.77974791,734.12358342)(234.81974787,734.11858343)(234.86975159,734.10858887)
\curveto(234.91974777,734.10858344)(234.95974773,734.11358343)(234.98975159,734.12358887)
\lineto(235.21475159,734.16858887)
\curveto(235.46474723,734.2485833)(235.64974704,734.37358317)(235.76975159,734.54358887)
\curveto(235.84974684,734.6435829)(235.91974677,734.77358277)(235.97975159,734.93358887)
\curveto(236.05974663,735.11358243)(236.11974657,735.33858221)(236.15975159,735.60858887)
\curveto(236.19974649,735.88858166)(236.21474648,736.16858138)(236.20475159,736.44858887)
\curveto(236.1947465,736.73858081)(236.16474653,737.01358053)(236.11475159,737.27358887)
\curveto(236.06474663,737.53358001)(235.9897467,737.7435798)(235.88975159,737.90358887)
\curveto(235.76974692,738.10357944)(235.61974707,738.25357929)(235.43975159,738.35358887)
\curveto(235.35974733,738.40357914)(235.26974742,738.43357911)(235.16975159,738.44358887)
\curveto(235.06974762,738.46357908)(234.96474773,738.47357907)(234.85475159,738.47358887)
\curveto(234.83474786,738.46357908)(234.80974788,738.45857909)(234.77975159,738.45858887)
\curveto(234.75974793,738.46857908)(234.73974795,738.46857908)(234.71975159,738.45858887)
\curveto(234.66974802,738.4485791)(234.62474807,738.43857911)(234.58475159,738.42858887)
\curveto(234.54474815,738.42857912)(234.50474819,738.41857913)(234.46475159,738.39858887)
\curveto(234.28474841,738.31857923)(234.13474856,738.19857935)(234.01475159,738.03858887)
\curveto(233.90474879,737.87857967)(233.81474888,737.69857985)(233.74475159,737.49858887)
\curveto(233.68474901,737.30858024)(233.63974905,737.08358046)(233.60975159,736.82358887)
\curveto(233.5897491,736.56358098)(233.58474911,736.29858125)(233.59475159,736.02858887)
\curveto(233.60474909,735.76858178)(233.63474906,735.51858203)(233.68475159,735.27858887)
\curveto(233.74474895,735.0485825)(233.81974887,734.85858269)(233.90975159,734.70858887)
\moveto(244.70975159,731.72358887)
\curveto(244.71973797,731.67358587)(244.72473797,731.58358596)(244.72475159,731.45358887)
\curveto(244.72473797,731.32358622)(244.71473798,731.23358631)(244.69475159,731.18358887)
\curveto(244.67473802,731.13358641)(244.66973802,731.07858647)(244.67975159,731.01858887)
\curveto(244.689738,730.96858658)(244.689738,730.91858663)(244.67975159,730.86858887)
\curveto(244.63973805,730.72858682)(244.60973808,730.59358695)(244.58975159,730.46358887)
\curveto(244.57973811,730.33358721)(244.54973814,730.21358733)(244.49975159,730.10358887)
\curveto(244.35973833,729.75358779)(244.1947385,729.45858809)(244.00475159,729.21858887)
\curveto(243.81473888,728.98858856)(243.54473915,728.80358874)(243.19475159,728.66358887)
\curveto(243.11473958,728.63358891)(243.02973966,728.61358893)(242.93975159,728.60358887)
\curveto(242.84973984,728.58358896)(242.76473993,728.56358898)(242.68475159,728.54358887)
\curveto(242.63474006,728.53358901)(242.58474011,728.52858902)(242.53475159,728.52858887)
\curveto(242.48474021,728.52858902)(242.43474026,728.52358902)(242.38475159,728.51358887)
\curveto(242.35474034,728.50358904)(242.30474039,728.50358904)(242.23475159,728.51358887)
\curveto(242.16474053,728.51358903)(242.11474058,728.51858903)(242.08475159,728.52858887)
\curveto(242.02474067,728.548589)(241.96474073,728.55858899)(241.90475159,728.55858887)
\curveto(241.85474084,728.548589)(241.80474089,728.55358899)(241.75475159,728.57358887)
\curveto(241.66474103,728.59358895)(241.57474112,728.61858893)(241.48475159,728.64858887)
\curveto(241.40474129,728.66858888)(241.32474137,728.69858885)(241.24475159,728.73858887)
\curveto(240.92474177,728.87858867)(240.67474202,729.07358847)(240.49475159,729.32358887)
\curveto(240.31474238,729.58358796)(240.16474253,729.88858766)(240.04475159,730.23858887)
\curveto(240.02474267,730.31858723)(240.00974268,730.40358714)(239.99975159,730.49358887)
\curveto(239.9897427,730.58358696)(239.97474272,730.66858688)(239.95475159,730.74858887)
\curveto(239.94474275,730.77858677)(239.93974275,730.80858674)(239.93975159,730.83858887)
\lineto(239.93975159,730.94358887)
\curveto(239.91974277,731.02358652)(239.90974278,731.10358644)(239.90975159,731.18358887)
\lineto(239.90975159,731.31858887)
\curveto(239.8897428,731.41858613)(239.8897428,731.51858603)(239.90975159,731.61858887)
\lineto(239.90975159,731.79858887)
\curveto(239.91974277,731.8485857)(239.92474277,731.89358565)(239.92475159,731.93358887)
\curveto(239.92474277,731.98358556)(239.92974276,732.02858552)(239.93975159,732.06858887)
\curveto(239.94974274,732.10858544)(239.95474274,732.1435854)(239.95475159,732.17358887)
\curveto(239.95474274,732.21358533)(239.95974273,732.25358529)(239.96975159,732.29358887)
\lineto(240.02975159,732.62358887)
\curveto(240.04974264,732.7435848)(240.07974261,732.85358469)(240.11975159,732.95358887)
\curveto(240.25974243,733.28358426)(240.41974227,733.55858399)(240.59975159,733.77858887)
\curveto(240.7897419,734.00858354)(241.04974164,734.19358335)(241.37975159,734.33358887)
\curveto(241.45974123,734.37358317)(241.54474115,734.39858315)(241.63475159,734.40858887)
\lineto(241.93475159,734.46858887)
\lineto(242.06975159,734.46858887)
\curveto(242.11974057,734.47858307)(242.16974052,734.48358306)(242.21975159,734.48358887)
\curveto(242.7897399,734.50358304)(243.24973944,734.39858315)(243.59975159,734.16858887)
\curveto(243.95973873,733.9485836)(244.22473847,733.6485839)(244.39475159,733.26858887)
\curveto(244.44473825,733.16858438)(244.48473821,733.06858448)(244.51475159,732.96858887)
\curveto(244.54473815,732.86858468)(244.57473812,732.76358478)(244.60475159,732.65358887)
\curveto(244.61473808,732.61358493)(244.61973807,732.57858497)(244.61975159,732.54858887)
\curveto(244.61973807,732.52858502)(244.62473807,732.49858505)(244.63475159,732.45858887)
\curveto(244.65473804,732.38858516)(244.66473803,732.31358523)(244.66475159,732.23358887)
\curveto(244.66473803,732.15358539)(244.67473802,732.07358547)(244.69475159,731.99358887)
\curveto(244.694738,731.9435856)(244.694738,731.89858565)(244.69475159,731.85858887)
\curveto(244.694738,731.81858573)(244.69973799,731.77358577)(244.70975159,731.72358887)
\moveto(243.59975159,731.28858887)
\curveto(243.60973908,731.33858621)(243.61473908,731.41358613)(243.61475159,731.51358887)
\curveto(243.62473907,731.61358593)(243.61973907,731.68858586)(243.59975159,731.73858887)
\curveto(243.57973911,731.79858575)(243.57473912,731.85358569)(243.58475159,731.90358887)
\curveto(243.60473909,731.96358558)(243.60473909,732.02358552)(243.58475159,732.08358887)
\curveto(243.57473912,732.11358543)(243.56973912,732.1485854)(243.56975159,732.18858887)
\curveto(243.56973912,732.22858532)(243.56473913,732.26858528)(243.55475159,732.30858887)
\curveto(243.53473916,732.38858516)(243.51473918,732.46358508)(243.49475159,732.53358887)
\curveto(243.48473921,732.61358493)(243.46973922,732.69358485)(243.44975159,732.77358887)
\curveto(243.41973927,732.83358471)(243.3947393,732.89358465)(243.37475159,732.95358887)
\curveto(243.35473934,733.01358453)(243.32473937,733.07358447)(243.28475159,733.13358887)
\curveto(243.18473951,733.30358424)(243.05473964,733.43858411)(242.89475159,733.53858887)
\curveto(242.81473988,733.58858396)(242.71973997,733.62358392)(242.60975159,733.64358887)
\curveto(242.49974019,733.66358388)(242.37474032,733.67358387)(242.23475159,733.67358887)
\curveto(242.21474048,733.66358388)(242.1897405,733.65858389)(242.15975159,733.65858887)
\curveto(242.12974056,733.66858388)(242.09974059,733.66858388)(242.06975159,733.65858887)
\lineto(241.91975159,733.59858887)
\curveto(241.86974082,733.58858396)(241.82474087,733.57358397)(241.78475159,733.55358887)
\curveto(241.5947411,733.4435841)(241.44974124,733.29858425)(241.34975159,733.11858887)
\curveto(241.25974143,732.93858461)(241.17974151,732.73358481)(241.10975159,732.50358887)
\curveto(241.06974162,732.37358517)(241.04974164,732.23858531)(241.04975159,732.09858887)
\curveto(241.04974164,731.96858558)(241.03974165,731.82358572)(241.01975159,731.66358887)
\curveto(241.00974168,731.61358593)(240.99974169,731.55358599)(240.98975159,731.48358887)
\curveto(240.9897417,731.41358613)(240.99974169,731.35358619)(241.01975159,731.30358887)
\lineto(241.01975159,731.13858887)
\lineto(241.01975159,730.95858887)
\curveto(241.02974166,730.90858664)(241.03974165,730.85358669)(241.04975159,730.79358887)
\curveto(241.05974163,730.7435868)(241.06474163,730.68858686)(241.06475159,730.62858887)
\curveto(241.07474162,730.56858698)(241.0897416,730.51358703)(241.10975159,730.46358887)
\curveto(241.15974153,730.27358727)(241.21974147,730.09858745)(241.28975159,729.93858887)
\curveto(241.35974133,729.77858777)(241.46474123,729.6485879)(241.60475159,729.54858887)
\curveto(241.73474096,729.4485881)(241.87474082,729.37858817)(242.02475159,729.33858887)
\curveto(242.05474064,729.32858822)(242.07974061,729.32358822)(242.09975159,729.32358887)
\curveto(242.12974056,729.33358821)(242.15974053,729.33358821)(242.18975159,729.32358887)
\curveto(242.20974048,729.32358822)(242.23974045,729.31858823)(242.27975159,729.30858887)
\curveto(242.31974037,729.30858824)(242.35474034,729.31358823)(242.38475159,729.32358887)
\curveto(242.42474027,729.33358821)(242.46474023,729.33858821)(242.50475159,729.33858887)
\curveto(242.54474015,729.33858821)(242.58474011,729.3485882)(242.62475159,729.36858887)
\curveto(242.86473983,729.4485881)(243.05973963,729.58358796)(243.20975159,729.77358887)
\curveto(243.32973936,729.95358759)(243.41973927,730.15858739)(243.47975159,730.38858887)
\curveto(243.49973919,730.45858709)(243.51473918,730.52858702)(243.52475159,730.59858887)
\curveto(243.53473916,730.67858687)(243.54973914,730.75858679)(243.56975159,730.83858887)
\curveto(243.56973912,730.89858665)(243.57473912,730.9435866)(243.58475159,730.97358887)
\curveto(243.58473911,730.99358655)(243.58473911,731.01858653)(243.58475159,731.04858887)
\curveto(243.58473911,731.08858646)(243.5897391,731.11858643)(243.59975159,731.13858887)
\lineto(243.59975159,731.28858887)
}
}
{
\newrgbcolor{curcolor}{0 0 0}
\pscustom[linestyle=none,fillstyle=solid,fillcolor=curcolor]
{
\newpath
\moveto(598.05075195,986.24641876)
\lineto(602.85075195,986.24641876)
\lineto(603.85575195,986.24641876)
\curveto(603.99574485,986.24640834)(604.11574473,986.23640835)(604.21575195,986.21641876)
\curveto(604.32574452,986.20640838)(604.40574444,986.16140842)(604.45575195,986.08141876)
\curveto(604.47574437,986.04140854)(604.48574436,985.99140859)(604.48575195,985.93141876)
\curveto(604.49574435,985.87140871)(604.50074435,985.80640878)(604.50075195,985.73641876)
\lineto(604.50075195,985.46641876)
\curveto(604.50074435,985.37640921)(604.49074436,985.29640929)(604.47075195,985.22641876)
\curveto(604.43074442,985.14640944)(604.38574446,985.07640951)(604.33575195,985.01641876)
\lineto(604.18575195,984.83641876)
\curveto(604.15574469,984.7864098)(604.12074473,984.74640984)(604.08075195,984.71641876)
\curveto(604.04074481,984.6864099)(604.00074485,984.64640994)(603.96075195,984.59641876)
\curveto(603.88074497,984.4864101)(603.79574505,984.37641021)(603.70575195,984.26641876)
\curveto(603.61574523,984.16641042)(603.53074532,984.06141052)(603.45075195,983.95141876)
\curveto(603.31074554,983.75141083)(603.17074568,983.54141104)(603.03075195,983.32141876)
\curveto(602.89074596,983.11141147)(602.7507461,982.89641169)(602.61075195,982.67641876)
\curveto(602.56074629,982.586412)(602.51074634,982.49141209)(602.46075195,982.39141876)
\curveto(602.41074644,982.29141229)(602.35574649,982.19641239)(602.29575195,982.10641876)
\curveto(602.27574657,982.0864125)(602.26574658,982.06141252)(602.26575195,982.03141876)
\curveto(602.26574658,982.00141258)(602.25574659,981.97641261)(602.23575195,981.95641876)
\curveto(602.16574668,981.85641273)(602.10074675,981.74141284)(602.04075195,981.61141876)
\curveto(601.98074687,981.49141309)(601.92574692,981.37641321)(601.87575195,981.26641876)
\curveto(601.77574707,981.03641355)(601.68074717,980.80141378)(601.59075195,980.56141876)
\curveto(601.50074735,980.32141426)(601.40074745,980.0814145)(601.29075195,979.84141876)
\curveto(601.27074758,979.79141479)(601.25574759,979.74641484)(601.24575195,979.70641876)
\curveto(601.2457476,979.66641492)(601.23574761,979.62141496)(601.21575195,979.57141876)
\curveto(601.16574768,979.45141513)(601.12074773,979.32641526)(601.08075195,979.19641876)
\curveto(601.0507478,979.07641551)(601.01574783,978.95641563)(600.97575195,978.83641876)
\curveto(600.89574795,978.60641598)(600.83074802,978.36641622)(600.78075195,978.11641876)
\curveto(600.74074811,977.87641671)(600.69074816,977.63641695)(600.63075195,977.39641876)
\curveto(600.59074826,977.24641734)(600.56574828,977.09641749)(600.55575195,976.94641876)
\curveto(600.5457483,976.79641779)(600.52574832,976.64641794)(600.49575195,976.49641876)
\curveto(600.48574836,976.45641813)(600.48074837,976.39641819)(600.48075195,976.31641876)
\curveto(600.4507484,976.19641839)(600.42074843,976.09641849)(600.39075195,976.01641876)
\curveto(600.36074849,975.93641865)(600.29074856,975.8814187)(600.18075195,975.85141876)
\curveto(600.13074872,975.83141875)(600.07574877,975.82141876)(600.01575195,975.82141876)
\lineto(599.82075195,975.82141876)
\curveto(599.68074917,975.82141876)(599.54074931,975.82641876)(599.40075195,975.83641876)
\curveto(599.27074958,975.84641874)(599.17574967,975.89141869)(599.11575195,975.97141876)
\curveto(599.07574977,976.03141855)(599.05574979,976.11641847)(599.05575195,976.22641876)
\curveto(599.06574978,976.33641825)(599.08074977,976.43141815)(599.10075195,976.51141876)
\lineto(599.10075195,976.58641876)
\curveto(599.11074974,976.61641797)(599.11574973,976.64641794)(599.11575195,976.67641876)
\curveto(599.13574971,976.75641783)(599.1457497,976.83141775)(599.14575195,976.90141876)
\curveto(599.1457497,976.97141761)(599.15574969,977.04141754)(599.17575195,977.11141876)
\curveto(599.22574962,977.30141728)(599.26574958,977.4864171)(599.29575195,977.66641876)
\curveto(599.32574952,977.85641673)(599.36574948,978.03641655)(599.41575195,978.20641876)
\curveto(599.43574941,978.25641633)(599.4457494,978.29641629)(599.44575195,978.32641876)
\curveto(599.4457494,978.35641623)(599.4507494,978.39141619)(599.46075195,978.43141876)
\curveto(599.56074929,978.73141585)(599.6507492,979.02641556)(599.73075195,979.31641876)
\curveto(599.82074903,979.60641498)(599.92574892,979.8864147)(600.04575195,980.15641876)
\curveto(600.30574854,980.73641385)(600.57574827,981.2864133)(600.85575195,981.80641876)
\curveto(601.13574771,982.33641225)(601.4457474,982.84141174)(601.78575195,983.32141876)
\curveto(601.92574692,983.52141106)(602.07574677,983.71141087)(602.23575195,983.89141876)
\curveto(602.39574645,984.0814105)(602.5457463,984.27141031)(602.68575195,984.46141876)
\curveto(602.72574612,984.51141007)(602.76074609,984.55641003)(602.79075195,984.59641876)
\curveto(602.83074602,984.64640994)(602.86574598,984.69640989)(602.89575195,984.74641876)
\curveto(602.90574594,984.76640982)(602.91574593,984.79140979)(602.92575195,984.82141876)
\curveto(602.9457459,984.85140973)(602.9457459,984.8814097)(602.92575195,984.91141876)
\curveto(602.90574594,984.97140961)(602.87074598,985.00640958)(602.82075195,985.01641876)
\curveto(602.77074608,985.03640955)(602.72074613,985.05640953)(602.67075195,985.07641876)
\lineto(602.56575195,985.07641876)
\curveto(602.52574632,985.0864095)(602.47574637,985.0864095)(602.41575195,985.07641876)
\lineto(602.26575195,985.07641876)
\lineto(601.66575195,985.07641876)
\lineto(599.02575195,985.07641876)
\lineto(598.29075195,985.07641876)
\lineto(598.05075195,985.07641876)
\curveto(597.98075087,985.0864095)(597.92075093,985.10140948)(597.87075195,985.12141876)
\curveto(597.78075107,985.16140942)(597.72075113,985.22140936)(597.69075195,985.30141876)
\curveto(597.64075121,985.40140918)(597.62575122,985.54640904)(597.64575195,985.73641876)
\curveto(597.66575118,985.93640865)(597.70075115,986.07140851)(597.75075195,986.14141876)
\curveto(597.77075108,986.16140842)(597.79575105,986.17640841)(597.82575195,986.18641876)
\lineto(597.94575195,986.24641876)
\curveto(597.96575088,986.24640834)(597.98075087,986.24140834)(597.99075195,986.23141876)
\curveto(598.01075084,986.23140835)(598.03075082,986.23640835)(598.05075195,986.24641876)
}
}
{
\newrgbcolor{curcolor}{0 0 0}
\pscustom[linestyle=none,fillstyle=solid,fillcolor=curcolor]
{
\newpath
\moveto(606.89536133,977.47141876)
\lineto(607.19536133,977.47141876)
\curveto(607.30535927,977.4814171)(607.41035916,977.4814171)(607.51036133,977.47141876)
\curveto(607.62035895,977.47141711)(607.72035885,977.46141712)(607.81036133,977.44141876)
\curveto(607.90035867,977.43141715)(607.9703586,977.40641718)(608.02036133,977.36641876)
\curveto(608.04035853,977.34641724)(608.05535852,977.31641727)(608.06536133,977.27641876)
\curveto(608.08535849,977.23641735)(608.10535847,977.19141739)(608.12536133,977.14141876)
\lineto(608.12536133,977.06641876)
\curveto(608.13535844,977.01641757)(608.13535844,976.96141762)(608.12536133,976.90141876)
\lineto(608.12536133,976.75141876)
\lineto(608.12536133,976.27141876)
\curveto(608.12535845,976.10141848)(608.08535849,975.9814186)(608.00536133,975.91141876)
\curveto(607.93535864,975.86141872)(607.84535873,975.83641875)(607.73536133,975.83641876)
\lineto(607.40536133,975.83641876)
\lineto(606.95536133,975.83641876)
\curveto(606.80535977,975.83641875)(606.69035988,975.86641872)(606.61036133,975.92641876)
\curveto(606.57036,975.95641863)(606.54036003,976.00641858)(606.52036133,976.07641876)
\curveto(606.50036007,976.15641843)(606.48536009,976.24141834)(606.47536133,976.33141876)
\lineto(606.47536133,976.61641876)
\curveto(606.48536009,976.71641787)(606.49036008,976.80141778)(606.49036133,976.87141876)
\lineto(606.49036133,977.06641876)
\curveto(606.49036008,977.12641746)(606.50036007,977.1814174)(606.52036133,977.23141876)
\curveto(606.56036001,977.34141724)(606.63035994,977.41141717)(606.73036133,977.44141876)
\curveto(606.76035981,977.44141714)(606.81535976,977.45141713)(606.89536133,977.47141876)
}
}
{
\newrgbcolor{curcolor}{0 0 0}
\pscustom[linestyle=none,fillstyle=solid,fillcolor=curcolor]
{
\newpath
\moveto(614.16051758,986.44141876)
\curveto(614.26051272,986.44140814)(614.35551263,986.43140815)(614.44551758,986.41141876)
\curveto(614.53551245,986.40140818)(614.60051238,986.37140821)(614.64051758,986.32141876)
\curveto(614.70051228,986.24140834)(614.73051225,986.13640845)(614.73051758,986.00641876)
\lineto(614.73051758,985.61641876)
\lineto(614.73051758,984.11641876)
\lineto(614.73051758,977.72641876)
\lineto(614.73051758,976.55641876)
\lineto(614.73051758,976.24141876)
\curveto(614.74051224,976.14141844)(614.72551226,976.06141852)(614.68551758,976.00141876)
\curveto(614.63551235,975.92141866)(614.56051242,975.87141871)(614.46051758,975.85141876)
\curveto(614.37051261,975.84141874)(614.26051272,975.83641875)(614.13051758,975.83641876)
\lineto(613.90551758,975.83641876)
\curveto(613.82551316,975.85641873)(613.75551323,975.87141871)(613.69551758,975.88141876)
\curveto(613.63551335,975.90141868)(613.5855134,975.94141864)(613.54551758,976.00141876)
\curveto(613.50551348,976.06141852)(613.4855135,976.13641845)(613.48551758,976.22641876)
\lineto(613.48551758,976.52641876)
\lineto(613.48551758,977.62141876)
\lineto(613.48551758,982.96141876)
\curveto(613.46551352,983.05141153)(613.45051353,983.12641146)(613.44051758,983.18641876)
\curveto(613.44051354,983.25641133)(613.41051357,983.31641127)(613.35051758,983.36641876)
\curveto(613.2805137,983.41641117)(613.19051379,983.44141114)(613.08051758,983.44141876)
\curveto(612.980514,983.45141113)(612.87051411,983.45641113)(612.75051758,983.45641876)
\lineto(611.61051758,983.45641876)
\lineto(611.11551758,983.45641876)
\curveto(610.95551603,983.46641112)(610.84551614,983.52641106)(610.78551758,983.63641876)
\curveto(610.76551622,983.66641092)(610.75551623,983.69641089)(610.75551758,983.72641876)
\curveto(610.75551623,983.76641082)(610.75051623,983.81141077)(610.74051758,983.86141876)
\curveto(610.72051626,983.9814106)(610.72551626,984.09141049)(610.75551758,984.19141876)
\curveto(610.79551619,984.29141029)(610.85051613,984.36141022)(610.92051758,984.40141876)
\curveto(611.00051598,984.45141013)(611.12051586,984.47641011)(611.28051758,984.47641876)
\curveto(611.44051554,984.47641011)(611.57551541,984.49141009)(611.68551758,984.52141876)
\curveto(611.73551525,984.53141005)(611.79051519,984.53641005)(611.85051758,984.53641876)
\curveto(611.91051507,984.54641004)(611.97051501,984.56141002)(612.03051758,984.58141876)
\curveto(612.1805148,984.63140995)(612.32551466,984.6814099)(612.46551758,984.73141876)
\curveto(612.60551438,984.79140979)(612.74051424,984.86140972)(612.87051758,984.94141876)
\curveto(613.01051397,985.03140955)(613.13051385,985.13640945)(613.23051758,985.25641876)
\curveto(613.33051365,985.37640921)(613.42551356,985.50640908)(613.51551758,985.64641876)
\curveto(613.57551341,985.74640884)(613.62051336,985.85640873)(613.65051758,985.97641876)
\curveto(613.69051329,986.09640849)(613.74051324,986.20140838)(613.80051758,986.29141876)
\curveto(613.85051313,986.35140823)(613.92051306,986.39140819)(614.01051758,986.41141876)
\curveto(614.03051295,986.42140816)(614.05551293,986.42640816)(614.08551758,986.42641876)
\curveto(614.11551287,986.42640816)(614.14051284,986.43140815)(614.16051758,986.44141876)
}
}
{
\newrgbcolor{curcolor}{0 0 0}
\pscustom[linestyle=none,fillstyle=solid,fillcolor=curcolor]
{
\newpath
\moveto(628.25512695,984.35641876)
\curveto(628.05511665,984.06641052)(627.84511686,983.7814108)(627.62512695,983.50141876)
\curveto(627.41511729,983.22141136)(627.2101175,982.93641165)(627.01012695,982.64641876)
\curveto(626.4101183,981.79641279)(625.8051189,980.95641363)(625.19512695,980.12641876)
\curveto(624.58512012,979.30641528)(623.98012073,978.47141611)(623.38012695,977.62141876)
\lineto(622.87012695,976.90141876)
\lineto(622.36012695,976.21141876)
\curveto(622.28012243,976.10141848)(622.20012251,975.9864186)(622.12012695,975.86641876)
\curveto(622.04012267,975.74641884)(621.94512276,975.65141893)(621.83512695,975.58141876)
\curveto(621.79512291,975.56141902)(621.73012298,975.54641904)(621.64012695,975.53641876)
\curveto(621.56012315,975.51641907)(621.47012324,975.50641908)(621.37012695,975.50641876)
\curveto(621.27012344,975.50641908)(621.17512353,975.51141907)(621.08512695,975.52141876)
\curveto(621.0051237,975.53141905)(620.94512376,975.55141903)(620.90512695,975.58141876)
\curveto(620.87512383,975.60141898)(620.85012386,975.63641895)(620.83012695,975.68641876)
\curveto(620.82012389,975.72641886)(620.82512388,975.77141881)(620.84512695,975.82141876)
\curveto(620.88512382,975.90141868)(620.93012378,975.97641861)(620.98012695,976.04641876)
\curveto(621.04012367,976.12641846)(621.09512361,976.20641838)(621.14512695,976.28641876)
\curveto(621.38512332,976.62641796)(621.63012308,976.96141762)(621.88012695,977.29141876)
\curveto(622.13012258,977.62141696)(622.37012234,977.95641663)(622.60012695,978.29641876)
\curveto(622.76012195,978.51641607)(622.92012179,978.73141585)(623.08012695,978.94141876)
\curveto(623.24012147,979.15141543)(623.40012131,979.36641522)(623.56012695,979.58641876)
\curveto(623.92012079,980.10641448)(624.28512042,980.61641397)(624.65512695,981.11641876)
\curveto(625.02511968,981.61641297)(625.39511931,982.12641246)(625.76512695,982.64641876)
\curveto(625.9051188,982.84641174)(626.04511866,983.04141154)(626.18512695,983.23141876)
\curveto(626.33511837,983.42141116)(626.48011823,983.61641097)(626.62012695,983.81641876)
\curveto(626.83011788,984.11641047)(627.04511766,984.41641017)(627.26512695,984.71641876)
\lineto(627.92512695,985.61641876)
\lineto(628.10512695,985.88641876)
\lineto(628.31512695,986.15641876)
\lineto(628.43512695,986.33641876)
\curveto(628.48511622,986.39640819)(628.53511617,986.45140813)(628.58512695,986.50141876)
\curveto(628.65511605,986.55140803)(628.73011598,986.586408)(628.81012695,986.60641876)
\curveto(628.83011588,986.61640797)(628.85511585,986.61640797)(628.88512695,986.60641876)
\curveto(628.92511578,986.60640798)(628.95511575,986.61640797)(628.97512695,986.63641876)
\curveto(629.09511561,986.63640795)(629.23011548,986.63140795)(629.38012695,986.62141876)
\curveto(629.53011518,986.62140796)(629.62011509,986.57640801)(629.65012695,986.48641876)
\curveto(629.67011504,986.45640813)(629.67511503,986.42140816)(629.66512695,986.38141876)
\curveto(629.65511505,986.34140824)(629.64011507,986.31140827)(629.62012695,986.29141876)
\curveto(629.58011513,986.21140837)(629.54011517,986.14140844)(629.50012695,986.08141876)
\curveto(629.46011525,986.02140856)(629.41511529,985.96140862)(629.36512695,985.90141876)
\lineto(628.79512695,985.12141876)
\curveto(628.61511609,984.87140971)(628.43511627,984.61640997)(628.25512695,984.35641876)
\moveto(621.40012695,980.45641876)
\curveto(621.35012336,980.47641411)(621.30012341,980.4814141)(621.25012695,980.47141876)
\curveto(621.20012351,980.46141412)(621.15012356,980.46641412)(621.10012695,980.48641876)
\curveto(620.99012372,980.50641408)(620.88512382,980.52641406)(620.78512695,980.54641876)
\curveto(620.69512401,980.57641401)(620.60012411,980.61641397)(620.50012695,980.66641876)
\curveto(620.17012454,980.80641378)(619.91512479,981.00141358)(619.73512695,981.25141876)
\curveto(619.55512515,981.51141307)(619.4101253,981.82141276)(619.30012695,982.18141876)
\curveto(619.27012544,982.26141232)(619.25012546,982.34141224)(619.24012695,982.42141876)
\curveto(619.23012548,982.51141207)(619.21512549,982.59641199)(619.19512695,982.67641876)
\curveto(619.18512552,982.72641186)(619.18012553,982.79141179)(619.18012695,982.87141876)
\curveto(619.17012554,982.90141168)(619.16512554,982.93141165)(619.16512695,982.96141876)
\curveto(619.16512554,983.00141158)(619.16012555,983.03641155)(619.15012695,983.06641876)
\lineto(619.15012695,983.21641876)
\curveto(619.14012557,983.26641132)(619.13512557,983.32641126)(619.13512695,983.39641876)
\curveto(619.13512557,983.47641111)(619.14012557,983.54141104)(619.15012695,983.59141876)
\lineto(619.15012695,983.75641876)
\curveto(619.17012554,983.80641078)(619.17512553,983.85141073)(619.16512695,983.89141876)
\curveto(619.16512554,983.94141064)(619.17012554,983.9864106)(619.18012695,984.02641876)
\curveto(619.19012552,984.06641052)(619.19512551,984.10141048)(619.19512695,984.13141876)
\curveto(619.19512551,984.17141041)(619.20012551,984.21141037)(619.21012695,984.25141876)
\curveto(619.24012547,984.36141022)(619.26012545,984.47141011)(619.27012695,984.58141876)
\curveto(619.29012542,984.70140988)(619.32512538,984.81640977)(619.37512695,984.92641876)
\curveto(619.51512519,985.26640932)(619.67512503,985.54140904)(619.85512695,985.75141876)
\curveto(620.04512466,985.97140861)(620.31512439,986.15140843)(620.66512695,986.29141876)
\curveto(620.74512396,986.32140826)(620.83012388,986.34140824)(620.92012695,986.35141876)
\curveto(621.0101237,986.37140821)(621.1051236,986.39140819)(621.20512695,986.41141876)
\curveto(621.23512347,986.42140816)(621.29012342,986.42140816)(621.37012695,986.41141876)
\curveto(621.45012326,986.41140817)(621.50012321,986.42140816)(621.52012695,986.44141876)
\curveto(622.08012263,986.45140813)(622.53012218,986.34140824)(622.87012695,986.11141876)
\curveto(623.22012149,985.8814087)(623.48012123,985.57640901)(623.65012695,985.19641876)
\curveto(623.69012102,985.10640948)(623.72512098,985.01140957)(623.75512695,984.91141876)
\curveto(623.78512092,984.81140977)(623.8101209,984.71140987)(623.83012695,984.61141876)
\curveto(623.85012086,984.58141)(623.85512085,984.55141003)(623.84512695,984.52141876)
\curveto(623.84512086,984.49141009)(623.85012086,984.46141012)(623.86012695,984.43141876)
\curveto(623.89012082,984.32141026)(623.9101208,984.19641039)(623.92012695,984.05641876)
\curveto(623.93012078,983.92641066)(623.94012077,983.79141079)(623.95012695,983.65141876)
\lineto(623.95012695,983.48641876)
\curveto(623.96012075,983.42641116)(623.96012075,983.37141121)(623.95012695,983.32141876)
\curveto(623.94012077,983.27141131)(623.93512077,983.22141136)(623.93512695,983.17141876)
\lineto(623.93512695,983.03641876)
\curveto(623.92512078,982.99641159)(623.92012079,982.95641163)(623.92012695,982.91641876)
\curveto(623.93012078,982.87641171)(623.92512078,982.83141175)(623.90512695,982.78141876)
\curveto(623.88512082,982.67141191)(623.86512084,982.56641202)(623.84512695,982.46641876)
\curveto(623.83512087,982.36641222)(623.81512089,982.26641232)(623.78512695,982.16641876)
\curveto(623.65512105,981.80641278)(623.49012122,981.49141309)(623.29012695,981.22141876)
\curveto(623.09012162,980.95141363)(622.81512189,980.74641384)(622.46512695,980.60641876)
\curveto(622.38512232,980.57641401)(622.30012241,980.55141403)(622.21012695,980.53141876)
\lineto(621.94012695,980.47141876)
\curveto(621.89012282,980.46141412)(621.84512286,980.45641413)(621.80512695,980.45641876)
\curveto(621.76512294,980.46641412)(621.72512298,980.46641412)(621.68512695,980.45641876)
\curveto(621.58512312,980.43641415)(621.49012322,980.43641415)(621.40012695,980.45641876)
\moveto(620.56012695,981.85141876)
\curveto(620.60012411,981.7814128)(620.64012407,981.71641287)(620.68012695,981.65641876)
\curveto(620.72012399,981.60641298)(620.77012394,981.55641303)(620.83012695,981.50641876)
\lineto(620.98012695,981.38641876)
\curveto(621.04012367,981.35641323)(621.1051236,981.33141325)(621.17512695,981.31141876)
\curveto(621.21512349,981.29141329)(621.25012346,981.2814133)(621.28012695,981.28141876)
\curveto(621.32012339,981.29141329)(621.36012335,981.2864133)(621.40012695,981.26641876)
\curveto(621.43012328,981.26641332)(621.47012324,981.26141332)(621.52012695,981.25141876)
\curveto(621.57012314,981.25141333)(621.6101231,981.25641333)(621.64012695,981.26641876)
\lineto(621.86512695,981.31141876)
\curveto(622.11512259,981.39141319)(622.30012241,981.51641307)(622.42012695,981.68641876)
\curveto(622.50012221,981.7864128)(622.57012214,981.91641267)(622.63012695,982.07641876)
\curveto(622.710122,982.25641233)(622.77012194,982.4814121)(622.81012695,982.75141876)
\curveto(622.85012186,983.03141155)(622.86512184,983.31141127)(622.85512695,983.59141876)
\curveto(622.84512186,983.8814107)(622.81512189,984.15641043)(622.76512695,984.41641876)
\curveto(622.71512199,984.67640991)(622.64012207,984.8864097)(622.54012695,985.04641876)
\curveto(622.42012229,985.24640934)(622.27012244,985.39640919)(622.09012695,985.49641876)
\curveto(622.0101227,985.54640904)(621.92012279,985.57640901)(621.82012695,985.58641876)
\curveto(621.72012299,985.60640898)(621.61512309,985.61640897)(621.50512695,985.61641876)
\curveto(621.48512322,985.60640898)(621.46012325,985.60140898)(621.43012695,985.60141876)
\curveto(621.4101233,985.61140897)(621.39012332,985.61140897)(621.37012695,985.60141876)
\curveto(621.32012339,985.59140899)(621.27512343,985.581409)(621.23512695,985.57141876)
\curveto(621.19512351,985.57140901)(621.15512355,985.56140902)(621.11512695,985.54141876)
\curveto(620.93512377,985.46140912)(620.78512392,985.34140924)(620.66512695,985.18141876)
\curveto(620.55512415,985.02140956)(620.46512424,984.84140974)(620.39512695,984.64141876)
\curveto(620.33512437,984.45141013)(620.29012442,984.22641036)(620.26012695,983.96641876)
\curveto(620.24012447,983.70641088)(620.23512447,983.44141114)(620.24512695,983.17141876)
\curveto(620.25512445,982.91141167)(620.28512442,982.66141192)(620.33512695,982.42141876)
\curveto(620.39512431,982.19141239)(620.47012424,982.00141258)(620.56012695,981.85141876)
\moveto(631.36012695,978.86641876)
\curveto(631.37011334,978.81641577)(631.37511333,978.72641586)(631.37512695,978.59641876)
\curveto(631.37511333,978.46641612)(631.36511334,978.37641621)(631.34512695,978.32641876)
\curveto(631.32511338,978.27641631)(631.32011339,978.22141636)(631.33012695,978.16141876)
\curveto(631.34011337,978.11141647)(631.34011337,978.06141652)(631.33012695,978.01141876)
\curveto(631.29011342,977.87141671)(631.26011345,977.73641685)(631.24012695,977.60641876)
\curveto(631.23011348,977.47641711)(631.20011351,977.35641723)(631.15012695,977.24641876)
\curveto(631.0101137,976.89641769)(630.84511386,976.60141798)(630.65512695,976.36141876)
\curveto(630.46511424,976.13141845)(630.19511451,975.94641864)(629.84512695,975.80641876)
\curveto(629.76511494,975.77641881)(629.68011503,975.75641883)(629.59012695,975.74641876)
\curveto(629.50011521,975.72641886)(629.41511529,975.70641888)(629.33512695,975.68641876)
\curveto(629.28511542,975.67641891)(629.23511547,975.67141891)(629.18512695,975.67141876)
\curveto(629.13511557,975.67141891)(629.08511562,975.66641892)(629.03512695,975.65641876)
\curveto(629.0051157,975.64641894)(628.95511575,975.64641894)(628.88512695,975.65641876)
\curveto(628.81511589,975.65641893)(628.76511594,975.66141892)(628.73512695,975.67141876)
\curveto(628.67511603,975.69141889)(628.61511609,975.70141888)(628.55512695,975.70141876)
\curveto(628.5051162,975.69141889)(628.45511625,975.69641889)(628.40512695,975.71641876)
\curveto(628.31511639,975.73641885)(628.22511648,975.76141882)(628.13512695,975.79141876)
\curveto(628.05511665,975.81141877)(627.97511673,975.84141874)(627.89512695,975.88141876)
\curveto(627.57511713,976.02141856)(627.32511738,976.21641837)(627.14512695,976.46641876)
\curveto(626.96511774,976.72641786)(626.81511789,977.03141755)(626.69512695,977.38141876)
\curveto(626.67511803,977.46141712)(626.66011805,977.54641704)(626.65012695,977.63641876)
\curveto(626.64011807,977.72641686)(626.62511808,977.81141677)(626.60512695,977.89141876)
\curveto(626.59511811,977.92141666)(626.59011812,977.95141663)(626.59012695,977.98141876)
\lineto(626.59012695,978.08641876)
\curveto(626.57011814,978.16641642)(626.56011815,978.24641634)(626.56012695,978.32641876)
\lineto(626.56012695,978.46141876)
\curveto(626.54011817,978.56141602)(626.54011817,978.66141592)(626.56012695,978.76141876)
\lineto(626.56012695,978.94141876)
\curveto(626.57011814,978.99141559)(626.57511813,979.03641555)(626.57512695,979.07641876)
\curveto(626.57511813,979.12641546)(626.58011813,979.17141541)(626.59012695,979.21141876)
\curveto(626.60011811,979.25141533)(626.6051181,979.2864153)(626.60512695,979.31641876)
\curveto(626.6051181,979.35641523)(626.6101181,979.39641519)(626.62012695,979.43641876)
\lineto(626.68012695,979.76641876)
\curveto(626.70011801,979.8864147)(626.73011798,979.99641459)(626.77012695,980.09641876)
\curveto(626.9101178,980.42641416)(627.07011764,980.70141388)(627.25012695,980.92141876)
\curveto(627.44011727,981.15141343)(627.70011701,981.33641325)(628.03012695,981.47641876)
\curveto(628.1101166,981.51641307)(628.19511651,981.54141304)(628.28512695,981.55141876)
\lineto(628.58512695,981.61141876)
\lineto(628.72012695,981.61141876)
\curveto(628.77011594,981.62141296)(628.82011589,981.62641296)(628.87012695,981.62641876)
\curveto(629.44011527,981.64641294)(629.90011481,981.54141304)(630.25012695,981.31141876)
\curveto(630.6101141,981.09141349)(630.87511383,980.79141379)(631.04512695,980.41141876)
\curveto(631.09511361,980.31141427)(631.13511357,980.21141437)(631.16512695,980.11141876)
\curveto(631.19511351,980.01141457)(631.22511348,979.90641468)(631.25512695,979.79641876)
\curveto(631.26511344,979.75641483)(631.27011344,979.72141486)(631.27012695,979.69141876)
\curveto(631.27011344,979.67141491)(631.27511343,979.64141494)(631.28512695,979.60141876)
\curveto(631.3051134,979.53141505)(631.31511339,979.45641513)(631.31512695,979.37641876)
\curveto(631.31511339,979.29641529)(631.32511338,979.21641537)(631.34512695,979.13641876)
\curveto(631.34511336,979.0864155)(631.34511336,979.04141554)(631.34512695,979.00141876)
\curveto(631.34511336,978.96141562)(631.35011336,978.91641567)(631.36012695,978.86641876)
\moveto(630.25012695,978.43141876)
\curveto(630.26011445,978.4814161)(630.26511444,978.55641603)(630.26512695,978.65641876)
\curveto(630.27511443,978.75641583)(630.27011444,978.83141575)(630.25012695,978.88141876)
\curveto(630.23011448,978.94141564)(630.22511448,978.99641559)(630.23512695,979.04641876)
\curveto(630.25511445,979.10641548)(630.25511445,979.16641542)(630.23512695,979.22641876)
\curveto(630.22511448,979.25641533)(630.22011449,979.29141529)(630.22012695,979.33141876)
\curveto(630.22011449,979.37141521)(630.21511449,979.41141517)(630.20512695,979.45141876)
\curveto(630.18511452,979.53141505)(630.16511454,979.60641498)(630.14512695,979.67641876)
\curveto(630.13511457,979.75641483)(630.12011459,979.83641475)(630.10012695,979.91641876)
\curveto(630.07011464,979.97641461)(630.04511466,980.03641455)(630.02512695,980.09641876)
\curveto(630.0051147,980.15641443)(629.97511473,980.21641437)(629.93512695,980.27641876)
\curveto(629.83511487,980.44641414)(629.705115,980.581414)(629.54512695,980.68141876)
\curveto(629.46511524,980.73141385)(629.37011534,980.76641382)(629.26012695,980.78641876)
\curveto(629.15011556,980.80641378)(629.02511568,980.81641377)(628.88512695,980.81641876)
\curveto(628.86511584,980.80641378)(628.84011587,980.80141378)(628.81012695,980.80141876)
\curveto(628.78011593,980.81141377)(628.75011596,980.81141377)(628.72012695,980.80141876)
\lineto(628.57012695,980.74141876)
\curveto(628.52011619,980.73141385)(628.47511623,980.71641387)(628.43512695,980.69641876)
\curveto(628.24511646,980.586414)(628.10011661,980.44141414)(628.00012695,980.26141876)
\curveto(627.9101168,980.0814145)(627.83011688,979.87641471)(627.76012695,979.64641876)
\curveto(627.72011699,979.51641507)(627.70011701,979.3814152)(627.70012695,979.24141876)
\curveto(627.70011701,979.11141547)(627.69011702,978.96641562)(627.67012695,978.80641876)
\curveto(627.66011705,978.75641583)(627.65011706,978.69641589)(627.64012695,978.62641876)
\curveto(627.64011707,978.55641603)(627.65011706,978.49641609)(627.67012695,978.44641876)
\lineto(627.67012695,978.28141876)
\lineto(627.67012695,978.10141876)
\curveto(627.68011703,978.05141653)(627.69011702,977.99641659)(627.70012695,977.93641876)
\curveto(627.710117,977.8864167)(627.71511699,977.83141675)(627.71512695,977.77141876)
\curveto(627.72511698,977.71141687)(627.74011697,977.65641693)(627.76012695,977.60641876)
\curveto(627.8101169,977.41641717)(627.87011684,977.24141734)(627.94012695,977.08141876)
\curveto(628.0101167,976.92141766)(628.11511659,976.79141779)(628.25512695,976.69141876)
\curveto(628.38511632,976.59141799)(628.52511618,976.52141806)(628.67512695,976.48141876)
\curveto(628.705116,976.47141811)(628.73011598,976.46641812)(628.75012695,976.46641876)
\curveto(628.78011593,976.47641811)(628.8101159,976.47641811)(628.84012695,976.46641876)
\curveto(628.86011585,976.46641812)(628.89011582,976.46141812)(628.93012695,976.45141876)
\curveto(628.97011574,976.45141813)(629.0051157,976.45641813)(629.03512695,976.46641876)
\curveto(629.07511563,976.47641811)(629.11511559,976.4814181)(629.15512695,976.48141876)
\curveto(629.19511551,976.4814181)(629.23511547,976.49141809)(629.27512695,976.51141876)
\curveto(629.51511519,976.59141799)(629.710115,976.72641786)(629.86012695,976.91641876)
\curveto(629.98011473,977.09641749)(630.07011464,977.30141728)(630.13012695,977.53141876)
\curveto(630.15011456,977.60141698)(630.16511454,977.67141691)(630.17512695,977.74141876)
\curveto(630.18511452,977.82141676)(630.20011451,977.90141668)(630.22012695,977.98141876)
\curveto(630.22011449,978.04141654)(630.22511448,978.0864165)(630.23512695,978.11641876)
\curveto(630.23511447,978.13641645)(630.23511447,978.16141642)(630.23512695,978.19141876)
\curveto(630.23511447,978.23141635)(630.24011447,978.26141632)(630.25012695,978.28141876)
\lineto(630.25012695,978.43141876)
}
}
{
\newrgbcolor{curcolor}{0 0 0}
\pscustom[linestyle=none,fillstyle=solid,fillcolor=curcolor]
{
\newpath
\moveto(586.74863525,736.05426392)
\curveto(586.75862753,736.01426087)(586.75862753,735.96426092)(586.74863525,735.90426392)
\curveto(586.74862754,735.84426104)(586.74362755,735.79426109)(586.73363525,735.75426392)
\curveto(586.73362756,735.71426117)(586.72862756,735.67426121)(586.71863525,735.63426392)
\lineto(586.71863525,735.52926392)
\curveto(586.69862759,735.44926143)(586.68362761,735.36926151)(586.67363525,735.28926392)
\curveto(586.66362763,735.20926167)(586.64362765,735.13426175)(586.61363525,735.06426392)
\curveto(586.5936277,734.9842619)(586.57362772,734.90926197)(586.55363525,734.83926392)
\curveto(586.53362776,734.76926211)(586.50362779,734.69426219)(586.46363525,734.61426392)
\curveto(586.28362801,734.19426269)(586.02862826,733.85426303)(585.69863525,733.59426392)
\curveto(585.36862892,733.33426355)(584.97862931,733.12926375)(584.52863525,732.97926392)
\curveto(584.40862988,732.93926394)(584.28363001,732.91426397)(584.15363525,732.90426392)
\curveto(584.03363026,732.884264)(583.90863038,732.85926402)(583.77863525,732.82926392)
\curveto(583.71863057,732.81926406)(583.65363064,732.81426407)(583.58363525,732.81426392)
\curveto(583.52363077,732.81426407)(583.45863083,732.80926407)(583.38863525,732.79926392)
\lineto(583.26863525,732.79926392)
\lineto(583.07363525,732.79926392)
\curveto(583.01363128,732.78926409)(582.95863133,732.79426409)(582.90863525,732.81426392)
\curveto(582.83863145,732.83426405)(582.77363152,732.83926404)(582.71363525,732.82926392)
\curveto(582.65363164,732.81926406)(582.5936317,732.82426406)(582.53363525,732.84426392)
\curveto(582.48363181,732.85426403)(582.43863185,732.85926402)(582.39863525,732.85926392)
\curveto(582.35863193,732.85926402)(582.31363198,732.86926401)(582.26363525,732.88926392)
\curveto(582.18363211,732.90926397)(582.10863218,732.92926395)(582.03863525,732.94926392)
\curveto(581.96863232,732.95926392)(581.89863239,732.97426391)(581.82863525,732.99426392)
\curveto(581.34863294,733.16426372)(580.94863334,733.37426351)(580.62863525,733.62426392)
\curveto(580.31863397,733.884263)(580.06863422,734.23926264)(579.87863525,734.68926392)
\curveto(579.84863444,734.74926213)(579.82363447,734.80926207)(579.80363525,734.86926392)
\curveto(579.7936345,734.93926194)(579.77863451,735.01426187)(579.75863525,735.09426392)
\curveto(579.73863455,735.15426173)(579.72363457,735.21926166)(579.71363525,735.28926392)
\curveto(579.70363459,735.35926152)(579.6886346,735.42926145)(579.66863525,735.49926392)
\curveto(579.65863463,735.54926133)(579.65363464,735.58926129)(579.65363525,735.61926392)
\lineto(579.65363525,735.73926392)
\curveto(579.64363465,735.7792611)(579.63363466,735.82926105)(579.62363525,735.88926392)
\curveto(579.62363467,735.94926093)(579.62863466,735.99926088)(579.63863525,736.03926392)
\lineto(579.63863525,736.17426392)
\curveto(579.64863464,736.22426066)(579.65363464,736.27426061)(579.65363525,736.32426392)
\curveto(579.67363462,736.42426046)(579.6886346,736.51926036)(579.69863525,736.60926392)
\curveto(579.70863458,736.70926017)(579.72863456,736.80426008)(579.75863525,736.89426392)
\curveto(579.80863448,737.04425984)(579.86363443,737.1842597)(579.92363525,737.31426392)
\curveto(579.98363431,737.44425944)(580.05363424,737.56425932)(580.13363525,737.67426392)
\curveto(580.16363413,737.72425916)(580.1936341,737.76425912)(580.22363525,737.79426392)
\curveto(580.26363403,737.82425906)(580.29863399,737.85925902)(580.32863525,737.89926392)
\curveto(580.3886339,737.9792589)(580.45863383,738.04925883)(580.53863525,738.10926392)
\curveto(580.59863369,738.15925872)(580.65863363,738.20425868)(580.71863525,738.24426392)
\lineto(580.92863525,738.39426392)
\curveto(580.97863331,738.43425845)(581.02863326,738.46925841)(581.07863525,738.49926392)
\curveto(581.12863316,738.53925834)(581.16363313,738.59425829)(581.18363525,738.66426392)
\curveto(581.18363311,738.69425819)(581.17363312,738.71925816)(581.15363525,738.73926392)
\curveto(581.14363315,738.76925811)(581.13363316,738.79425809)(581.12363525,738.81426392)
\curveto(581.08363321,738.86425802)(581.03363326,738.90925797)(580.97363525,738.94926392)
\curveto(580.92363337,738.99925788)(580.87363342,739.04425784)(580.82363525,739.08426392)
\curveto(580.78363351,739.11425777)(580.73363356,739.16925771)(580.67363525,739.24926392)
\curveto(580.65363364,739.2792576)(580.62363367,739.30425758)(580.58363525,739.32426392)
\curveto(580.55363374,739.35425753)(580.52863376,739.38925749)(580.50863525,739.42926392)
\curveto(580.33863395,739.63925724)(580.20863408,739.884257)(580.11863525,740.16426392)
\curveto(580.09863419,740.24425664)(580.08363421,740.32425656)(580.07363525,740.40426392)
\curveto(580.06363423,740.4842564)(580.04863424,740.56425632)(580.02863525,740.64426392)
\curveto(580.00863428,740.69425619)(579.99863429,740.75925612)(579.99863525,740.83926392)
\curveto(579.99863429,740.92925595)(580.00863428,740.99925588)(580.02863525,741.04926392)
\curveto(580.02863426,741.14925573)(580.03363426,741.21925566)(580.04363525,741.25926392)
\curveto(580.06363423,741.33925554)(580.07863421,741.40925547)(580.08863525,741.46926392)
\curveto(580.09863419,741.53925534)(580.11363418,741.60925527)(580.13363525,741.67926392)
\curveto(580.28363401,742.10925477)(580.49863379,742.45425443)(580.77863525,742.71426392)
\curveto(581.06863322,742.97425391)(581.41863287,743.18925369)(581.82863525,743.35926392)
\curveto(581.93863235,743.40925347)(582.05363224,743.43925344)(582.17363525,743.44926392)
\curveto(582.30363199,743.46925341)(582.43363186,743.49925338)(582.56363525,743.53926392)
\curveto(582.64363165,743.53925334)(582.71363158,743.53925334)(582.77363525,743.53926392)
\curveto(582.84363145,743.54925333)(582.91863137,743.55925332)(582.99863525,743.56926392)
\curveto(583.7886305,743.58925329)(584.44362985,743.45925342)(584.96363525,743.17926392)
\curveto(585.4936288,742.89925398)(585.87362842,742.48925439)(586.10363525,741.94926392)
\curveto(586.21362808,741.71925516)(586.28362801,741.43425545)(586.31363525,741.09426392)
\curveto(586.35362794,740.76425612)(586.32362797,740.45925642)(586.22363525,740.17926392)
\curveto(586.18362811,740.04925683)(586.13362816,739.92925695)(586.07363525,739.81926392)
\curveto(586.02362827,739.70925717)(585.96362833,739.60425728)(585.89363525,739.50426392)
\curveto(585.87362842,739.46425742)(585.84362845,739.42925745)(585.80363525,739.39926392)
\lineto(585.71363525,739.30926392)
\curveto(585.66362863,739.21925766)(585.60362869,739.15425773)(585.53363525,739.11426392)
\curveto(585.48362881,739.06425782)(585.42862886,739.01425787)(585.36863525,738.96426392)
\curveto(585.31862897,738.92425796)(585.27362902,738.879258)(585.23363525,738.82926392)
\curveto(585.21362908,738.80925807)(585.1936291,738.7842581)(585.17363525,738.75426392)
\curveto(585.16362913,738.73425815)(585.16362913,738.70925817)(585.17363525,738.67926392)
\curveto(585.18362911,738.62925825)(585.21362908,738.5792583)(585.26363525,738.52926392)
\curveto(585.31362898,738.48925839)(585.36862892,738.44925843)(585.42863525,738.40926392)
\lineto(585.60863525,738.28926392)
\curveto(585.66862862,738.25925862)(585.71862857,738.22925865)(585.75863525,738.19926392)
\curveto(586.0886282,737.95925892)(586.33862795,737.64925923)(586.50863525,737.26926392)
\curveto(586.54862774,737.18925969)(586.57862771,737.10425978)(586.59863525,737.01426392)
\curveto(586.62862766,736.92425996)(586.65362764,736.83426005)(586.67363525,736.74426392)
\curveto(586.68362761,736.69426019)(586.6936276,736.63926024)(586.70363525,736.57926392)
\lineto(586.73363525,736.42926392)
\curveto(586.74362755,736.36926051)(586.74362755,736.30426058)(586.73363525,736.23426392)
\curveto(586.72362757,736.17426071)(586.72862756,736.11426077)(586.74863525,736.05426392)
\moveto(581.36363525,741.09426392)
\curveto(581.33363296,740.9842559)(581.32863296,740.84425604)(581.34863525,740.67426392)
\curveto(581.36863292,740.51425637)(581.3936329,740.38925649)(581.42363525,740.29926392)
\curveto(581.53363276,739.9792569)(581.68363261,739.73425715)(581.87363525,739.56426392)
\curveto(582.06363223,739.40425748)(582.32863196,739.27425761)(582.66863525,739.17426392)
\curveto(582.79863149,739.14425774)(582.96363133,739.11925776)(583.16363525,739.09926392)
\curveto(583.36363093,739.08925779)(583.53363076,739.10425778)(583.67363525,739.14426392)
\curveto(583.96363033,739.22425766)(584.20363009,739.33425755)(584.39363525,739.47426392)
\curveto(584.5936297,739.62425726)(584.74862954,739.82425706)(584.85863525,740.07426392)
\curveto(584.87862941,740.12425676)(584.8886294,740.16925671)(584.88863525,740.20926392)
\curveto(584.89862939,740.24925663)(584.91362938,740.29425659)(584.93363525,740.34426392)
\curveto(584.96362933,740.45425643)(584.98362931,740.59425629)(584.99363525,740.76426392)
\curveto(585.00362929,740.93425595)(584.9936293,741.0792558)(584.96363525,741.19926392)
\curveto(584.94362935,741.28925559)(584.91862937,741.37425551)(584.88863525,741.45426392)
\curveto(584.86862942,741.53425535)(584.83362946,741.61425527)(584.78363525,741.69426392)
\curveto(584.61362968,741.96425492)(584.3886299,742.15925472)(584.10863525,742.27926392)
\curveto(583.83863045,742.39925448)(583.47863081,742.45925442)(583.02863525,742.45926392)
\curveto(583.00863128,742.43925444)(582.97863131,742.43425445)(582.93863525,742.44426392)
\curveto(582.89863139,742.45425443)(582.86363143,742.45425443)(582.83363525,742.44426392)
\curveto(582.78363151,742.42425446)(582.72863156,742.40925447)(582.66863525,742.39926392)
\curveto(582.61863167,742.39925448)(582.56863172,742.38925449)(582.51863525,742.36926392)
\curveto(582.27863201,742.2792546)(582.06863222,742.16425472)(581.88863525,742.02426392)
\curveto(581.70863258,741.89425499)(581.56863272,741.71425517)(581.46863525,741.48426392)
\curveto(581.44863284,741.42425546)(581.42863286,741.35925552)(581.40863525,741.28926392)
\curveto(581.39863289,741.22925565)(581.38363291,741.16425572)(581.36363525,741.09426392)
\moveto(585.38363525,735.55926392)
\curveto(585.43362886,735.74926113)(585.43862885,735.95426093)(585.39863525,736.17426392)
\curveto(585.36862892,736.39426049)(585.32362897,736.57426031)(585.26363525,736.71426392)
\curveto(585.0936292,737.0842598)(584.83362946,737.38925949)(584.48363525,737.62926392)
\curveto(584.14363015,737.86925901)(583.70863058,737.98925889)(583.17863525,737.98926392)
\curveto(583.14863114,737.96925891)(583.10863118,737.96425892)(583.05863525,737.97426392)
\curveto(583.00863128,737.99425889)(582.96863132,737.99925888)(582.93863525,737.98926392)
\lineto(582.66863525,737.92926392)
\curveto(582.5886317,737.91925896)(582.50863178,737.90425898)(582.42863525,737.88426392)
\curveto(582.12863216,737.77425911)(581.86363243,737.62925925)(581.63363525,737.44926392)
\curveto(581.41363288,737.26925961)(581.24363305,737.03925984)(581.12363525,736.75926392)
\curveto(581.0936332,736.6792602)(581.06863322,736.59926028)(581.04863525,736.51926392)
\curveto(581.02863326,736.43926044)(581.00863328,736.35426053)(580.98863525,736.26426392)
\curveto(580.95863333,736.14426074)(580.94863334,735.99426089)(580.95863525,735.81426392)
\curveto(580.97863331,735.63426125)(581.00363329,735.49426139)(581.03363525,735.39426392)
\curveto(581.05363324,735.34426154)(581.06363323,735.29926158)(581.06363525,735.25926392)
\curveto(581.07363322,735.22926165)(581.0886332,735.18926169)(581.10863525,735.13926392)
\curveto(581.20863308,734.91926196)(581.33863295,734.71926216)(581.49863525,734.53926392)
\curveto(581.66863262,734.35926252)(581.86363243,734.22426266)(582.08363525,734.13426392)
\curveto(582.15363214,734.09426279)(582.24863204,734.05926282)(582.36863525,734.02926392)
\curveto(582.5886317,733.93926294)(582.84363145,733.89426299)(583.13363525,733.89426392)
\lineto(583.41863525,733.89426392)
\curveto(583.51863077,733.91426297)(583.61363068,733.92926295)(583.70363525,733.93926392)
\curveto(583.7936305,733.94926293)(583.88363041,733.96926291)(583.97363525,733.99926392)
\curveto(584.23363006,734.0792628)(584.47362982,734.20926267)(584.69363525,734.38926392)
\curveto(584.92362937,734.5792623)(585.0936292,734.79426209)(585.20363525,735.03426392)
\curveto(585.24362905,735.11426177)(585.27362902,735.19426169)(585.29363525,735.27426392)
\curveto(585.32362897,735.36426152)(585.35362894,735.45926142)(585.38363525,735.55926392)
}
}
{
\newrgbcolor{curcolor}{0 0 0}
\pscustom[linestyle=none,fillstyle=solid,fillcolor=curcolor]
{
\newpath
\moveto(592.14324463,743.58426392)
\curveto(592.24323977,743.5842533)(592.33823968,743.57425331)(592.42824463,743.55426392)
\curveto(592.5182395,743.54425334)(592.58323943,743.51425337)(592.62324463,743.46426392)
\curveto(592.68323933,743.3842535)(592.7132393,743.2792536)(592.71324463,743.14926392)
\lineto(592.71324463,742.75926392)
\lineto(592.71324463,741.25926392)
\lineto(592.71324463,734.86926392)
\lineto(592.71324463,733.69926392)
\lineto(592.71324463,733.38426392)
\curveto(592.72323929,733.2842636)(592.70823931,733.20426368)(592.66824463,733.14426392)
\curveto(592.6182394,733.06426382)(592.54323947,733.01426387)(592.44324463,732.99426392)
\curveto(592.35323966,732.9842639)(592.24323977,732.9792639)(592.11324463,732.97926392)
\lineto(591.88824463,732.97926392)
\curveto(591.80824021,732.99926388)(591.73824028,733.01426387)(591.67824463,733.02426392)
\curveto(591.6182404,733.04426384)(591.56824045,733.0842638)(591.52824463,733.14426392)
\curveto(591.48824053,733.20426368)(591.46824055,733.2792636)(591.46824463,733.36926392)
\lineto(591.46824463,733.66926392)
\lineto(591.46824463,734.76426392)
\lineto(591.46824463,740.10426392)
\curveto(591.44824057,740.19425669)(591.43324058,740.26925661)(591.42324463,740.32926392)
\curveto(591.42324059,740.39925648)(591.39324062,740.45925642)(591.33324463,740.50926392)
\curveto(591.26324075,740.55925632)(591.17324084,740.5842563)(591.06324463,740.58426392)
\curveto(590.96324105,740.59425629)(590.85324116,740.59925628)(590.73324463,740.59926392)
\lineto(589.59324463,740.59926392)
\lineto(589.09824463,740.59926392)
\curveto(588.93824308,740.60925627)(588.82824319,740.66925621)(588.76824463,740.77926392)
\curveto(588.74824327,740.80925607)(588.73824328,740.83925604)(588.73824463,740.86926392)
\curveto(588.73824328,740.90925597)(588.73324328,740.95425593)(588.72324463,741.00426392)
\curveto(588.70324331,741.12425576)(588.70824331,741.23425565)(588.73824463,741.33426392)
\curveto(588.77824324,741.43425545)(588.83324318,741.50425538)(588.90324463,741.54426392)
\curveto(588.98324303,741.59425529)(589.10324291,741.61925526)(589.26324463,741.61926392)
\curveto(589.42324259,741.61925526)(589.55824246,741.63425525)(589.66824463,741.66426392)
\curveto(589.7182423,741.67425521)(589.77324224,741.6792552)(589.83324463,741.67926392)
\curveto(589.89324212,741.68925519)(589.95324206,741.70425518)(590.01324463,741.72426392)
\curveto(590.16324185,741.77425511)(590.30824171,741.82425506)(590.44824463,741.87426392)
\curveto(590.58824143,741.93425495)(590.72324129,742.00425488)(590.85324463,742.08426392)
\curveto(590.99324102,742.17425471)(591.1132409,742.2792546)(591.21324463,742.39926392)
\curveto(591.3132407,742.51925436)(591.40824061,742.64925423)(591.49824463,742.78926392)
\curveto(591.55824046,742.88925399)(591.60324041,742.99925388)(591.63324463,743.11926392)
\curveto(591.67324034,743.23925364)(591.72324029,743.34425354)(591.78324463,743.43426392)
\curveto(591.83324018,743.49425339)(591.90324011,743.53425335)(591.99324463,743.55426392)
\curveto(592.01324,743.56425332)(592.03823998,743.56925331)(592.06824463,743.56926392)
\curveto(592.09823992,743.56925331)(592.12323989,743.57425331)(592.14324463,743.58426392)
}
}
{
\newrgbcolor{curcolor}{0 0 0}
\pscustom[linestyle=none,fillstyle=solid,fillcolor=curcolor]
{
\newpath
\moveto(597.387854,734.61426392)
\lineto(597.687854,734.61426392)
\curveto(597.79785194,734.62426226)(597.90285184,734.62426226)(598.002854,734.61426392)
\curveto(598.11285163,734.61426227)(598.21285153,734.60426228)(598.302854,734.58426392)
\curveto(598.39285135,734.57426231)(598.46285128,734.54926233)(598.512854,734.50926392)
\curveto(598.53285121,734.48926239)(598.54785119,734.45926242)(598.557854,734.41926392)
\curveto(598.57785116,734.3792625)(598.59785114,734.33426255)(598.617854,734.28426392)
\lineto(598.617854,734.20926392)
\curveto(598.62785111,734.15926272)(598.62785111,734.10426278)(598.617854,734.04426392)
\lineto(598.617854,733.89426392)
\lineto(598.617854,733.41426392)
\curveto(598.61785112,733.24426364)(598.57785116,733.12426376)(598.497854,733.05426392)
\curveto(598.42785131,733.00426388)(598.3378514,732.9792639)(598.227854,732.97926392)
\lineto(597.897854,732.97926392)
\lineto(597.447854,732.97926392)
\curveto(597.29785244,732.9792639)(597.18285256,733.00926387)(597.102854,733.06926392)
\curveto(597.06285268,733.09926378)(597.03285271,733.14926373)(597.012854,733.21926392)
\curveto(596.99285275,733.29926358)(596.97785276,733.3842635)(596.967854,733.47426392)
\lineto(596.967854,733.75926392)
\curveto(596.97785276,733.85926302)(596.98285276,733.94426294)(596.982854,734.01426392)
\lineto(596.982854,734.20926392)
\curveto(596.98285276,734.26926261)(596.99285275,734.32426256)(597.012854,734.37426392)
\curveto(597.05285269,734.4842624)(597.12285262,734.55426233)(597.222854,734.58426392)
\curveto(597.25285249,734.5842623)(597.30785243,734.59426229)(597.387854,734.61426392)
}
}
{
\newrgbcolor{curcolor}{0 0 0}
\pscustom[linestyle=none,fillstyle=solid,fillcolor=curcolor]
{
\newpath
\moveto(601.05301025,743.38926392)
\lineto(605.85301025,743.38926392)
\lineto(606.85801025,743.38926392)
\curveto(606.99800315,743.38925349)(607.11800303,743.3792535)(607.21801025,743.35926392)
\curveto(607.32800282,743.34925353)(607.40800274,743.30425358)(607.45801025,743.22426392)
\curveto(607.47800267,743.1842537)(607.48800266,743.13425375)(607.48801025,743.07426392)
\curveto(607.49800265,743.01425387)(607.50300265,742.94925393)(607.50301025,742.87926392)
\lineto(607.50301025,742.60926392)
\curveto(607.50300265,742.51925436)(607.49300266,742.43925444)(607.47301025,742.36926392)
\curveto(607.43300272,742.28925459)(607.38800276,742.21925466)(607.33801025,742.15926392)
\lineto(607.18801025,741.97926392)
\curveto(607.15800299,741.92925495)(607.12300303,741.88925499)(607.08301025,741.85926392)
\curveto(607.04300311,741.82925505)(607.00300315,741.78925509)(606.96301025,741.73926392)
\curveto(606.88300327,741.62925525)(606.79800335,741.51925536)(606.70801025,741.40926392)
\curveto(606.61800353,741.30925557)(606.53300362,741.20425568)(606.45301025,741.09426392)
\curveto(606.31300384,740.89425599)(606.17300398,740.6842562)(606.03301025,740.46426392)
\curveto(605.89300426,740.25425663)(605.7530044,740.03925684)(605.61301025,739.81926392)
\curveto(605.56300459,739.72925715)(605.51300464,739.63425725)(605.46301025,739.53426392)
\curveto(605.41300474,739.43425745)(605.35800479,739.33925754)(605.29801025,739.24926392)
\curveto(605.27800487,739.22925765)(605.26800488,739.20425768)(605.26801025,739.17426392)
\curveto(605.26800488,739.14425774)(605.25800489,739.11925776)(605.23801025,739.09926392)
\curveto(605.16800498,738.99925788)(605.10300505,738.884258)(605.04301025,738.75426392)
\curveto(604.98300517,738.63425825)(604.92800522,738.51925836)(604.87801025,738.40926392)
\curveto(604.77800537,738.1792587)(604.68300547,737.94425894)(604.59301025,737.70426392)
\curveto(604.50300565,737.46425942)(604.40300575,737.22425966)(604.29301025,736.98426392)
\curveto(604.27300588,736.93425995)(604.25800589,736.88925999)(604.24801025,736.84926392)
\curveto(604.2480059,736.80926007)(604.23800591,736.76426012)(604.21801025,736.71426392)
\curveto(604.16800598,736.59426029)(604.12300603,736.46926041)(604.08301025,736.33926392)
\curveto(604.0530061,736.21926066)(604.01800613,736.09926078)(603.97801025,735.97926392)
\curveto(603.89800625,735.74926113)(603.83300632,735.50926137)(603.78301025,735.25926392)
\curveto(603.74300641,735.01926186)(603.69300646,734.7792621)(603.63301025,734.53926392)
\curveto(603.59300656,734.38926249)(603.56800658,734.23926264)(603.55801025,734.08926392)
\curveto(603.5480066,733.93926294)(603.52800662,733.78926309)(603.49801025,733.63926392)
\curveto(603.48800666,733.59926328)(603.48300667,733.53926334)(603.48301025,733.45926392)
\curveto(603.4530067,733.33926354)(603.42300673,733.23926364)(603.39301025,733.15926392)
\curveto(603.36300679,733.0792638)(603.29300686,733.02426386)(603.18301025,732.99426392)
\curveto(603.13300702,732.97426391)(603.07800707,732.96426392)(603.01801025,732.96426392)
\lineto(602.82301025,732.96426392)
\curveto(602.68300747,732.96426392)(602.54300761,732.96926391)(602.40301025,732.97926392)
\curveto(602.27300788,732.98926389)(602.17800797,733.03426385)(602.11801025,733.11426392)
\curveto(602.07800807,733.17426371)(602.05800809,733.25926362)(602.05801025,733.36926392)
\curveto(602.06800808,733.4792634)(602.08300807,733.57426331)(602.10301025,733.65426392)
\lineto(602.10301025,733.72926392)
\curveto(602.11300804,733.75926312)(602.11800803,733.78926309)(602.11801025,733.81926392)
\curveto(602.13800801,733.89926298)(602.148008,733.97426291)(602.14801025,734.04426392)
\curveto(602.148008,734.11426277)(602.15800799,734.1842627)(602.17801025,734.25426392)
\curveto(602.22800792,734.44426244)(602.26800788,734.62926225)(602.29801025,734.80926392)
\curveto(602.32800782,734.99926188)(602.36800778,735.1792617)(602.41801025,735.34926392)
\curveto(602.43800771,735.39926148)(602.4480077,735.43926144)(602.44801025,735.46926392)
\curveto(602.4480077,735.49926138)(602.4530077,735.53426135)(602.46301025,735.57426392)
\curveto(602.56300759,735.87426101)(602.6530075,736.16926071)(602.73301025,736.45926392)
\curveto(602.82300733,736.74926013)(602.92800722,737.02925985)(603.04801025,737.29926392)
\curveto(603.30800684,737.879259)(603.57800657,738.42925845)(603.85801025,738.94926392)
\curveto(604.13800601,739.4792574)(604.4480057,739.9842569)(604.78801025,740.46426392)
\curveto(604.92800522,740.66425622)(605.07800507,740.85425603)(605.23801025,741.03426392)
\curveto(605.39800475,741.22425566)(605.5480046,741.41425547)(605.68801025,741.60426392)
\curveto(605.72800442,741.65425523)(605.76300439,741.69925518)(605.79301025,741.73926392)
\curveto(605.83300432,741.78925509)(605.86800428,741.83925504)(605.89801025,741.88926392)
\curveto(605.90800424,741.90925497)(605.91800423,741.93425495)(605.92801025,741.96426392)
\curveto(605.9480042,741.99425489)(605.9480042,742.02425486)(605.92801025,742.05426392)
\curveto(605.90800424,742.11425477)(605.87300428,742.14925473)(605.82301025,742.15926392)
\curveto(605.77300438,742.1792547)(605.72300443,742.19925468)(605.67301025,742.21926392)
\lineto(605.56801025,742.21926392)
\curveto(605.52800462,742.22925465)(605.47800467,742.22925465)(605.41801025,742.21926392)
\lineto(605.26801025,742.21926392)
\lineto(604.66801025,742.21926392)
\lineto(602.02801025,742.21926392)
\lineto(601.29301025,742.21926392)
\lineto(601.05301025,742.21926392)
\curveto(600.98300917,742.22925465)(600.92300923,742.24425464)(600.87301025,742.26426392)
\curveto(600.78300937,742.30425458)(600.72300943,742.36425452)(600.69301025,742.44426392)
\curveto(600.64300951,742.54425434)(600.62800952,742.68925419)(600.64801025,742.87926392)
\curveto(600.66800948,743.0792538)(600.70300945,743.21425367)(600.75301025,743.28426392)
\curveto(600.77300938,743.30425358)(600.79800935,743.31925356)(600.82801025,743.32926392)
\lineto(600.94801025,743.38926392)
\curveto(600.96800918,743.38925349)(600.98300917,743.3842535)(600.99301025,743.37426392)
\curveto(601.01300914,743.37425351)(601.03300912,743.3792535)(601.05301025,743.38926392)
}
}
{
\newrgbcolor{curcolor}{0 0 0}
\pscustom[linestyle=none,fillstyle=solid,fillcolor=curcolor]
{
\newpath
\moveto(618.74761963,741.49926392)
\curveto(618.54760933,741.20925567)(618.33760954,740.92425596)(618.11761963,740.64426392)
\curveto(617.90760997,740.36425652)(617.70261017,740.0792568)(617.50261963,739.78926392)
\curveto(616.90261097,738.93925794)(616.29761158,738.09925878)(615.68761963,737.26926392)
\curveto(615.0776128,736.44926043)(614.4726134,735.61426127)(613.87261963,734.76426392)
\lineto(613.36261963,734.04426392)
\lineto(612.85261963,733.35426392)
\curveto(612.7726151,733.24426364)(612.69261518,733.12926375)(612.61261963,733.00926392)
\curveto(612.53261534,732.88926399)(612.43761544,732.79426409)(612.32761963,732.72426392)
\curveto(612.28761559,732.70426418)(612.22261565,732.68926419)(612.13261963,732.67926392)
\curveto(612.05261582,732.65926422)(611.96261591,732.64926423)(611.86261963,732.64926392)
\curveto(611.76261611,732.64926423)(611.66761621,732.65426423)(611.57761963,732.66426392)
\curveto(611.49761638,732.67426421)(611.43761644,732.69426419)(611.39761963,732.72426392)
\curveto(611.36761651,732.74426414)(611.34261653,732.7792641)(611.32261963,732.82926392)
\curveto(611.31261656,732.86926401)(611.31761656,732.91426397)(611.33761963,732.96426392)
\curveto(611.3776165,733.04426384)(611.42261645,733.11926376)(611.47261963,733.18926392)
\curveto(611.53261634,733.26926361)(611.58761629,733.34926353)(611.63761963,733.42926392)
\curveto(611.877616,733.76926311)(612.12261575,734.10426278)(612.37261963,734.43426392)
\curveto(612.62261525,734.76426212)(612.86261501,735.09926178)(613.09261963,735.43926392)
\curveto(613.25261462,735.65926122)(613.41261446,735.87426101)(613.57261963,736.08426392)
\curveto(613.73261414,736.29426059)(613.89261398,736.50926037)(614.05261963,736.72926392)
\curveto(614.41261346,737.24925963)(614.7776131,737.75925912)(615.14761963,738.25926392)
\curveto(615.51761236,738.75925812)(615.88761199,739.26925761)(616.25761963,739.78926392)
\curveto(616.39761148,739.98925689)(616.53761134,740.1842567)(616.67761963,740.37426392)
\curveto(616.82761105,740.56425632)(616.9726109,740.75925612)(617.11261963,740.95926392)
\curveto(617.32261055,741.25925562)(617.53761034,741.55925532)(617.75761963,741.85926392)
\lineto(618.41761963,742.75926392)
\lineto(618.59761963,743.02926392)
\lineto(618.80761963,743.29926392)
\lineto(618.92761963,743.47926392)
\curveto(618.9776089,743.53925334)(619.02760885,743.59425329)(619.07761963,743.64426392)
\curveto(619.14760873,743.69425319)(619.22260865,743.72925315)(619.30261963,743.74926392)
\curveto(619.32260855,743.75925312)(619.34760853,743.75925312)(619.37761963,743.74926392)
\curveto(619.41760846,743.74925313)(619.44760843,743.75925312)(619.46761963,743.77926392)
\curveto(619.58760829,743.7792531)(619.72260815,743.77425311)(619.87261963,743.76426392)
\curveto(620.02260785,743.76425312)(620.11260776,743.71925316)(620.14261963,743.62926392)
\curveto(620.16260771,743.59925328)(620.16760771,743.56425332)(620.15761963,743.52426392)
\curveto(620.14760773,743.4842534)(620.13260774,743.45425343)(620.11261963,743.43426392)
\curveto(620.0726078,743.35425353)(620.03260784,743.2842536)(619.99261963,743.22426392)
\curveto(619.95260792,743.16425372)(619.90760797,743.10425378)(619.85761963,743.04426392)
\lineto(619.28761963,742.26426392)
\curveto(619.10760877,742.01425487)(618.92760895,741.75925512)(618.74761963,741.49926392)
\moveto(611.89261963,737.59926392)
\curveto(611.84261603,737.61925926)(611.79261608,737.62425926)(611.74261963,737.61426392)
\curveto(611.69261618,737.60425928)(611.64261623,737.60925927)(611.59261963,737.62926392)
\curveto(611.48261639,737.64925923)(611.3776165,737.66925921)(611.27761963,737.68926392)
\curveto(611.18761669,737.71925916)(611.09261678,737.75925912)(610.99261963,737.80926392)
\curveto(610.66261721,737.94925893)(610.40761747,738.14425874)(610.22761963,738.39426392)
\curveto(610.04761783,738.65425823)(609.90261797,738.96425792)(609.79261963,739.32426392)
\curveto(609.76261811,739.40425748)(609.74261813,739.4842574)(609.73261963,739.56426392)
\curveto(609.72261815,739.65425723)(609.70761817,739.73925714)(609.68761963,739.81926392)
\curveto(609.6776182,739.86925701)(609.6726182,739.93425695)(609.67261963,740.01426392)
\curveto(609.66261821,740.04425684)(609.65761822,740.07425681)(609.65761963,740.10426392)
\curveto(609.65761822,740.14425674)(609.65261822,740.1792567)(609.64261963,740.20926392)
\lineto(609.64261963,740.35926392)
\curveto(609.63261824,740.40925647)(609.62761825,740.46925641)(609.62761963,740.53926392)
\curveto(609.62761825,740.61925626)(609.63261824,740.6842562)(609.64261963,740.73426392)
\lineto(609.64261963,740.89926392)
\curveto(609.66261821,740.94925593)(609.66761821,740.99425589)(609.65761963,741.03426392)
\curveto(609.65761822,741.0842558)(609.66261821,741.12925575)(609.67261963,741.16926392)
\curveto(609.68261819,741.20925567)(609.68761819,741.24425564)(609.68761963,741.27426392)
\curveto(609.68761819,741.31425557)(609.69261818,741.35425553)(609.70261963,741.39426392)
\curveto(609.73261814,741.50425538)(609.75261812,741.61425527)(609.76261963,741.72426392)
\curveto(609.78261809,741.84425504)(609.81761806,741.95925492)(609.86761963,742.06926392)
\curveto(610.00761787,742.40925447)(610.16761771,742.6842542)(610.34761963,742.89426392)
\curveto(610.53761734,743.11425377)(610.80761707,743.29425359)(611.15761963,743.43426392)
\curveto(611.23761664,743.46425342)(611.32261655,743.4842534)(611.41261963,743.49426392)
\curveto(611.50261637,743.51425337)(611.59761628,743.53425335)(611.69761963,743.55426392)
\curveto(611.72761615,743.56425332)(611.78261609,743.56425332)(611.86261963,743.55426392)
\curveto(611.94261593,743.55425333)(611.99261588,743.56425332)(612.01261963,743.58426392)
\curveto(612.5726153,743.59425329)(613.02261485,743.4842534)(613.36261963,743.25426392)
\curveto(613.71261416,743.02425386)(613.9726139,742.71925416)(614.14261963,742.33926392)
\curveto(614.18261369,742.24925463)(614.21761366,742.15425473)(614.24761963,742.05426392)
\curveto(614.2776136,741.95425493)(614.30261357,741.85425503)(614.32261963,741.75426392)
\curveto(614.34261353,741.72425516)(614.34761353,741.69425519)(614.33761963,741.66426392)
\curveto(614.33761354,741.63425525)(614.34261353,741.60425528)(614.35261963,741.57426392)
\curveto(614.38261349,741.46425542)(614.40261347,741.33925554)(614.41261963,741.19926392)
\curveto(614.42261345,741.06925581)(614.43261344,740.93425595)(614.44261963,740.79426392)
\lineto(614.44261963,740.62926392)
\curveto(614.45261342,740.56925631)(614.45261342,740.51425637)(614.44261963,740.46426392)
\curveto(614.43261344,740.41425647)(614.42761345,740.36425652)(614.42761963,740.31426392)
\lineto(614.42761963,740.17926392)
\curveto(614.41761346,740.13925674)(614.41261346,740.09925678)(614.41261963,740.05926392)
\curveto(614.42261345,740.01925686)(614.41761346,739.97425691)(614.39761963,739.92426392)
\curveto(614.3776135,739.81425707)(614.35761352,739.70925717)(614.33761963,739.60926392)
\curveto(614.32761355,739.50925737)(614.30761357,739.40925747)(614.27761963,739.30926392)
\curveto(614.14761373,738.94925793)(613.98261389,738.63425825)(613.78261963,738.36426392)
\curveto(613.58261429,738.09425879)(613.30761457,737.88925899)(612.95761963,737.74926392)
\curveto(612.877615,737.71925916)(612.79261508,737.69425919)(612.70261963,737.67426392)
\lineto(612.43261963,737.61426392)
\curveto(612.38261549,737.60425928)(612.33761554,737.59925928)(612.29761963,737.59926392)
\curveto(612.25761562,737.60925927)(612.21761566,737.60925927)(612.17761963,737.59926392)
\curveto(612.0776158,737.5792593)(611.98261589,737.5792593)(611.89261963,737.59926392)
\moveto(611.05261963,738.99426392)
\curveto(611.09261678,738.92425796)(611.13261674,738.85925802)(611.17261963,738.79926392)
\curveto(611.21261666,738.74925813)(611.26261661,738.69925818)(611.32261963,738.64926392)
\lineto(611.47261963,738.52926392)
\curveto(611.53261634,738.49925838)(611.59761628,738.47425841)(611.66761963,738.45426392)
\curveto(611.70761617,738.43425845)(611.74261613,738.42425846)(611.77261963,738.42426392)
\curveto(611.81261606,738.43425845)(611.85261602,738.42925845)(611.89261963,738.40926392)
\curveto(611.92261595,738.40925847)(611.96261591,738.40425848)(612.01261963,738.39426392)
\curveto(612.06261581,738.39425849)(612.10261577,738.39925848)(612.13261963,738.40926392)
\lineto(612.35761963,738.45426392)
\curveto(612.60761527,738.53425835)(612.79261508,738.65925822)(612.91261963,738.82926392)
\curveto(612.99261488,738.92925795)(613.06261481,739.05925782)(613.12261963,739.21926392)
\curveto(613.20261467,739.39925748)(613.26261461,739.62425726)(613.30261963,739.89426392)
\curveto(613.34261453,740.17425671)(613.35761452,740.45425643)(613.34761963,740.73426392)
\curveto(613.33761454,741.02425586)(613.30761457,741.29925558)(613.25761963,741.55926392)
\curveto(613.20761467,741.81925506)(613.13261474,742.02925485)(613.03261963,742.18926392)
\curveto(612.91261496,742.38925449)(612.76261511,742.53925434)(612.58261963,742.63926392)
\curveto(612.50261537,742.68925419)(612.41261546,742.71925416)(612.31261963,742.72926392)
\curveto(612.21261566,742.74925413)(612.10761577,742.75925412)(611.99761963,742.75926392)
\curveto(611.9776159,742.74925413)(611.95261592,742.74425414)(611.92261963,742.74426392)
\curveto(611.90261597,742.75425413)(611.88261599,742.75425413)(611.86261963,742.74426392)
\curveto(611.81261606,742.73425415)(611.76761611,742.72425416)(611.72761963,742.71426392)
\curveto(611.68761619,742.71425417)(611.64761623,742.70425418)(611.60761963,742.68426392)
\curveto(611.42761645,742.60425428)(611.2776166,742.4842544)(611.15761963,742.32426392)
\curveto(611.04761683,742.16425472)(610.95761692,741.9842549)(610.88761963,741.78426392)
\curveto(610.82761705,741.59425529)(610.78261709,741.36925551)(610.75261963,741.10926392)
\curveto(610.73261714,740.84925603)(610.72761715,740.5842563)(610.73761963,740.31426392)
\curveto(610.74761713,740.05425683)(610.7776171,739.80425708)(610.82761963,739.56426392)
\curveto(610.88761699,739.33425755)(610.96261691,739.14425774)(611.05261963,738.99426392)
\moveto(621.85261963,736.00926392)
\curveto(621.86260601,735.95926092)(621.86760601,735.86926101)(621.86761963,735.73926392)
\curveto(621.86760601,735.60926127)(621.85760602,735.51926136)(621.83761963,735.46926392)
\curveto(621.81760606,735.41926146)(621.81260606,735.36426152)(621.82261963,735.30426392)
\curveto(621.83260604,735.25426163)(621.83260604,735.20426168)(621.82261963,735.15426392)
\curveto(621.78260609,735.01426187)(621.75260612,734.879262)(621.73261963,734.74926392)
\curveto(621.72260615,734.61926226)(621.69260618,734.49926238)(621.64261963,734.38926392)
\curveto(621.50260637,734.03926284)(621.33760654,733.74426314)(621.14761963,733.50426392)
\curveto(620.95760692,733.27426361)(620.68760719,733.08926379)(620.33761963,732.94926392)
\curveto(620.25760762,732.91926396)(620.1726077,732.89926398)(620.08261963,732.88926392)
\curveto(619.99260788,732.86926401)(619.90760797,732.84926403)(619.82761963,732.82926392)
\curveto(619.7776081,732.81926406)(619.72760815,732.81426407)(619.67761963,732.81426392)
\curveto(619.62760825,732.81426407)(619.5776083,732.80926407)(619.52761963,732.79926392)
\curveto(619.49760838,732.78926409)(619.44760843,732.78926409)(619.37761963,732.79926392)
\curveto(619.30760857,732.79926408)(619.25760862,732.80426408)(619.22761963,732.81426392)
\curveto(619.16760871,732.83426405)(619.10760877,732.84426404)(619.04761963,732.84426392)
\curveto(618.99760888,732.83426405)(618.94760893,732.83926404)(618.89761963,732.85926392)
\curveto(618.80760907,732.879264)(618.71760916,732.90426398)(618.62761963,732.93426392)
\curveto(618.54760933,732.95426393)(618.46760941,732.9842639)(618.38761963,733.02426392)
\curveto(618.06760981,733.16426372)(617.81761006,733.35926352)(617.63761963,733.60926392)
\curveto(617.45761042,733.86926301)(617.30761057,734.17426271)(617.18761963,734.52426392)
\curveto(617.16761071,734.60426228)(617.15261072,734.68926219)(617.14261963,734.77926392)
\curveto(617.13261074,734.86926201)(617.11761076,734.95426193)(617.09761963,735.03426392)
\curveto(617.08761079,735.06426182)(617.08261079,735.09426179)(617.08261963,735.12426392)
\lineto(617.08261963,735.22926392)
\curveto(617.06261081,735.30926157)(617.05261082,735.38926149)(617.05261963,735.46926392)
\lineto(617.05261963,735.60426392)
\curveto(617.03261084,735.70426118)(617.03261084,735.80426108)(617.05261963,735.90426392)
\lineto(617.05261963,736.08426392)
\curveto(617.06261081,736.13426075)(617.06761081,736.1792607)(617.06761963,736.21926392)
\curveto(617.06761081,736.26926061)(617.0726108,736.31426057)(617.08261963,736.35426392)
\curveto(617.09261078,736.39426049)(617.09761078,736.42926045)(617.09761963,736.45926392)
\curveto(617.09761078,736.49926038)(617.10261077,736.53926034)(617.11261963,736.57926392)
\lineto(617.17261963,736.90926392)
\curveto(617.19261068,737.02925985)(617.22261065,737.13925974)(617.26261963,737.23926392)
\curveto(617.40261047,737.56925931)(617.56261031,737.84425904)(617.74261963,738.06426392)
\curveto(617.93260994,738.29425859)(618.19260968,738.4792584)(618.52261963,738.61926392)
\curveto(618.60260927,738.65925822)(618.68760919,738.6842582)(618.77761963,738.69426392)
\lineto(619.07761963,738.75426392)
\lineto(619.21261963,738.75426392)
\curveto(619.26260861,738.76425812)(619.31260856,738.76925811)(619.36261963,738.76926392)
\curveto(619.93260794,738.78925809)(620.39260748,738.6842582)(620.74261963,738.45426392)
\curveto(621.10260677,738.23425865)(621.36760651,737.93425895)(621.53761963,737.55426392)
\curveto(621.58760629,737.45425943)(621.62760625,737.35425953)(621.65761963,737.25426392)
\curveto(621.68760619,737.15425973)(621.71760616,737.04925983)(621.74761963,736.93926392)
\curveto(621.75760612,736.89925998)(621.76260611,736.86426002)(621.76261963,736.83426392)
\curveto(621.76260611,736.81426007)(621.76760611,736.7842601)(621.77761963,736.74426392)
\curveto(621.79760608,736.67426021)(621.80760607,736.59926028)(621.80761963,736.51926392)
\curveto(621.80760607,736.43926044)(621.81760606,736.35926052)(621.83761963,736.27926392)
\curveto(621.83760604,736.22926065)(621.83760604,736.1842607)(621.83761963,736.14426392)
\curveto(621.83760604,736.10426078)(621.84260603,736.05926082)(621.85261963,736.00926392)
\moveto(620.74261963,735.57426392)
\curveto(620.75260712,735.62426126)(620.75760712,735.69926118)(620.75761963,735.79926392)
\curveto(620.76760711,735.89926098)(620.76260711,735.97426091)(620.74261963,736.02426392)
\curveto(620.72260715,736.0842608)(620.71760716,736.13926074)(620.72761963,736.18926392)
\curveto(620.74760713,736.24926063)(620.74760713,736.30926057)(620.72761963,736.36926392)
\curveto(620.71760716,736.39926048)(620.71260716,736.43426045)(620.71261963,736.47426392)
\curveto(620.71260716,736.51426037)(620.70760717,736.55426033)(620.69761963,736.59426392)
\curveto(620.6776072,736.67426021)(620.65760722,736.74926013)(620.63761963,736.81926392)
\curveto(620.62760725,736.89925998)(620.61260726,736.9792599)(620.59261963,737.05926392)
\curveto(620.56260731,737.11925976)(620.53760734,737.1792597)(620.51761963,737.23926392)
\curveto(620.49760738,737.29925958)(620.46760741,737.35925952)(620.42761963,737.41926392)
\curveto(620.32760755,737.58925929)(620.19760768,737.72425916)(620.03761963,737.82426392)
\curveto(619.95760792,737.87425901)(619.86260801,737.90925897)(619.75261963,737.92926392)
\curveto(619.64260823,737.94925893)(619.51760836,737.95925892)(619.37761963,737.95926392)
\curveto(619.35760852,737.94925893)(619.33260854,737.94425894)(619.30261963,737.94426392)
\curveto(619.2726086,737.95425893)(619.24260863,737.95425893)(619.21261963,737.94426392)
\lineto(619.06261963,737.88426392)
\curveto(619.01260886,737.87425901)(618.96760891,737.85925902)(618.92761963,737.83926392)
\curveto(618.73760914,737.72925915)(618.59260928,737.5842593)(618.49261963,737.40426392)
\curveto(618.40260947,737.22425966)(618.32260955,737.01925986)(618.25261963,736.78926392)
\curveto(618.21260966,736.65926022)(618.19260968,736.52426036)(618.19261963,736.38426392)
\curveto(618.19260968,736.25426063)(618.18260969,736.10926077)(618.16261963,735.94926392)
\curveto(618.15260972,735.89926098)(618.14260973,735.83926104)(618.13261963,735.76926392)
\curveto(618.13260974,735.69926118)(618.14260973,735.63926124)(618.16261963,735.58926392)
\lineto(618.16261963,735.42426392)
\lineto(618.16261963,735.24426392)
\curveto(618.1726097,735.19426169)(618.18260969,735.13926174)(618.19261963,735.07926392)
\curveto(618.20260967,735.02926185)(618.20760967,734.97426191)(618.20761963,734.91426392)
\curveto(618.21760966,734.85426203)(618.23260964,734.79926208)(618.25261963,734.74926392)
\curveto(618.30260957,734.55926232)(618.36260951,734.3842625)(618.43261963,734.22426392)
\curveto(618.50260937,734.06426282)(618.60760927,733.93426295)(618.74761963,733.83426392)
\curveto(618.877609,733.73426315)(619.01760886,733.66426322)(619.16761963,733.62426392)
\curveto(619.19760868,733.61426327)(619.22260865,733.60926327)(619.24261963,733.60926392)
\curveto(619.2726086,733.61926326)(619.30260857,733.61926326)(619.33261963,733.60926392)
\curveto(619.35260852,733.60926327)(619.38260849,733.60426328)(619.42261963,733.59426392)
\curveto(619.46260841,733.59426329)(619.49760838,733.59926328)(619.52761963,733.60926392)
\curveto(619.56760831,733.61926326)(619.60760827,733.62426326)(619.64761963,733.62426392)
\curveto(619.68760819,733.62426326)(619.72760815,733.63426325)(619.76761963,733.65426392)
\curveto(620.00760787,733.73426315)(620.20260767,733.86926301)(620.35261963,734.05926392)
\curveto(620.4726074,734.23926264)(620.56260731,734.44426244)(620.62261963,734.67426392)
\curveto(620.64260723,734.74426214)(620.65760722,734.81426207)(620.66761963,734.88426392)
\curveto(620.6776072,734.96426192)(620.69260718,735.04426184)(620.71261963,735.12426392)
\curveto(620.71260716,735.1842617)(620.71760716,735.22926165)(620.72761963,735.25926392)
\curveto(620.72760715,735.2792616)(620.72760715,735.30426158)(620.72761963,735.33426392)
\curveto(620.72760715,735.37426151)(620.73260714,735.40426148)(620.74261963,735.42426392)
\lineto(620.74261963,735.57426392)
}
}
{
\newrgbcolor{curcolor}{0 0 0}
\pscustom[linestyle=none,fillstyle=solid,fillcolor=curcolor]
{
\newpath
\moveto(253.07932556,622.15570557)
\curveto(253.17932071,622.15569495)(253.27432061,622.14569496)(253.36432556,622.12570557)
\curveto(253.45432043,622.11569499)(253.51932037,622.08569502)(253.55932556,622.03570557)
\curveto(253.61932027,621.95569515)(253.64932024,621.85069525)(253.64932556,621.72070557)
\lineto(253.64932556,621.33070557)
\lineto(253.64932556,619.83070557)
\lineto(253.64932556,613.44070557)
\lineto(253.64932556,612.27070557)
\lineto(253.64932556,611.95570557)
\curveto(253.65932023,611.85570525)(253.64432024,611.77570533)(253.60432556,611.71570557)
\curveto(253.55432033,611.63570547)(253.47932041,611.58570552)(253.37932556,611.56570557)
\curveto(253.2893206,611.55570555)(253.17932071,611.55070555)(253.04932556,611.55070557)
\lineto(252.82432556,611.55070557)
\curveto(252.74432114,611.57070553)(252.67432121,611.58570552)(252.61432556,611.59570557)
\curveto(252.55432133,611.61570549)(252.50432138,611.65570545)(252.46432556,611.71570557)
\curveto(252.42432146,611.77570533)(252.40432148,611.85070525)(252.40432556,611.94070557)
\lineto(252.40432556,612.24070557)
\lineto(252.40432556,613.33570557)
\lineto(252.40432556,618.67570557)
\curveto(252.3843215,618.76569834)(252.36932152,618.84069826)(252.35932556,618.90070557)
\curveto(252.35932153,618.97069813)(252.32932156,619.03069807)(252.26932556,619.08070557)
\curveto(252.19932169,619.13069797)(252.10932178,619.15569795)(251.99932556,619.15570557)
\curveto(251.89932199,619.16569794)(251.7893221,619.17069793)(251.66932556,619.17070557)
\lineto(250.52932556,619.17070557)
\lineto(250.03432556,619.17070557)
\curveto(249.87432401,619.18069792)(249.76432412,619.24069786)(249.70432556,619.35070557)
\curveto(249.6843242,619.38069772)(249.67432421,619.41069769)(249.67432556,619.44070557)
\curveto(249.67432421,619.48069762)(249.66932422,619.52569758)(249.65932556,619.57570557)
\curveto(249.63932425,619.69569741)(249.64432424,619.8056973)(249.67432556,619.90570557)
\curveto(249.71432417,620.0056971)(249.76932412,620.07569703)(249.83932556,620.11570557)
\curveto(249.91932397,620.16569694)(250.03932385,620.19069691)(250.19932556,620.19070557)
\curveto(250.35932353,620.19069691)(250.49432339,620.2056969)(250.60432556,620.23570557)
\curveto(250.65432323,620.24569686)(250.70932318,620.25069685)(250.76932556,620.25070557)
\curveto(250.82932306,620.26069684)(250.889323,620.27569683)(250.94932556,620.29570557)
\curveto(251.09932279,620.34569676)(251.24432264,620.39569671)(251.38432556,620.44570557)
\curveto(251.52432236,620.5056966)(251.65932223,620.57569653)(251.78932556,620.65570557)
\curveto(251.92932196,620.74569636)(252.04932184,620.85069625)(252.14932556,620.97070557)
\curveto(252.24932164,621.09069601)(252.34432154,621.22069588)(252.43432556,621.36070557)
\curveto(252.49432139,621.46069564)(252.53932135,621.57069553)(252.56932556,621.69070557)
\curveto(252.60932128,621.81069529)(252.65932123,621.91569519)(252.71932556,622.00570557)
\curveto(252.76932112,622.06569504)(252.83932105,622.105695)(252.92932556,622.12570557)
\curveto(252.94932094,622.13569497)(252.97432091,622.14069496)(253.00432556,622.14070557)
\curveto(253.03432085,622.14069496)(253.05932083,622.14569496)(253.07932556,622.15570557)
}
}
{
\newrgbcolor{curcolor}{0 0 0}
\pscustom[linestyle=none,fillstyle=solid,fillcolor=curcolor]
{
\newpath
\moveto(260.48393494,622.15570557)
\curveto(262.1139295,622.18569492)(263.16392845,621.63069547)(263.63393494,620.49070557)
\curveto(263.73392788,620.26069684)(263.79892781,619.97069713)(263.82893494,619.62070557)
\curveto(263.86892774,619.28069782)(263.84392777,618.97069813)(263.75393494,618.69070557)
\curveto(263.66392795,618.43069867)(263.54392807,618.2056989)(263.39393494,618.01570557)
\curveto(263.37392824,617.97569913)(263.34892826,617.94069916)(263.31893494,617.91070557)
\curveto(263.28892832,617.89069921)(263.26392835,617.86569924)(263.24393494,617.83570557)
\lineto(263.15393494,617.71570557)
\curveto(263.12392849,617.68569942)(263.08892852,617.66069944)(263.04893494,617.64070557)
\curveto(262.99892861,617.59069951)(262.94392867,617.54569956)(262.88393494,617.50570557)
\curveto(262.83392878,617.46569964)(262.78892882,617.41569969)(262.74893494,617.35570557)
\curveto(262.7089289,617.31569979)(262.69392892,617.26569984)(262.70393494,617.20570557)
\curveto(262.7139289,617.15569995)(262.74392887,617.11069999)(262.79393494,617.07070557)
\curveto(262.84392877,617.03070007)(262.89892871,616.99070011)(262.95893494,616.95070557)
\curveto(263.02892858,616.92070018)(263.09392852,616.89070021)(263.15393494,616.86070557)
\curveto(263.2139284,616.83070027)(263.26392835,616.8007003)(263.30393494,616.77070557)
\curveto(263.62392799,616.55070055)(263.87892773,616.24070086)(264.06893494,615.84070557)
\curveto(264.1089275,615.75070135)(264.13892747,615.65570145)(264.15893494,615.55570557)
\curveto(264.18892742,615.46570164)(264.2139274,615.37570173)(264.23393494,615.28570557)
\curveto(264.24392737,615.23570187)(264.24892736,615.18570192)(264.24893494,615.13570557)
\curveto(264.25892735,615.09570201)(264.26892734,615.05070205)(264.27893494,615.00070557)
\curveto(264.28892732,614.95070215)(264.28892732,614.9007022)(264.27893494,614.85070557)
\curveto(264.26892734,614.8007023)(264.27392734,614.75070235)(264.29393494,614.70070557)
\curveto(264.30392731,614.65070245)(264.3089273,614.59070251)(264.30893494,614.52070557)
\curveto(264.3089273,614.45070265)(264.29892731,614.39070271)(264.27893494,614.34070557)
\lineto(264.27893494,614.11570557)
\lineto(264.21893494,613.87570557)
\curveto(264.2089274,613.8057033)(264.19392742,613.73570337)(264.17393494,613.66570557)
\curveto(264.14392747,613.57570353)(264.1139275,613.49070361)(264.08393494,613.41070557)
\curveto(264.06392755,613.33070377)(264.03392758,613.25070385)(263.99393494,613.17070557)
\curveto(263.97392764,613.11070399)(263.94392767,613.05070405)(263.90393494,612.99070557)
\curveto(263.87392774,612.94070416)(263.83892777,612.89070421)(263.79893494,612.84070557)
\curveto(263.59892801,612.53070457)(263.34892826,612.27070483)(263.04893494,612.06070557)
\curveto(262.74892886,611.86070524)(262.40392921,611.69570541)(262.01393494,611.56570557)
\curveto(261.89392972,611.52570558)(261.76392985,611.5007056)(261.62393494,611.49070557)
\curveto(261.49393012,611.47070563)(261.35893025,611.44570566)(261.21893494,611.41570557)
\curveto(261.14893046,611.4057057)(261.07893053,611.4007057)(261.00893494,611.40070557)
\curveto(260.94893066,611.4007057)(260.88393073,611.39570571)(260.81393494,611.38570557)
\curveto(260.77393084,611.37570573)(260.7139309,611.37070573)(260.63393494,611.37070557)
\curveto(260.56393105,611.37070573)(260.5139311,611.37570573)(260.48393494,611.38570557)
\curveto(260.43393118,611.39570571)(260.38893122,611.4007057)(260.34893494,611.40070557)
\lineto(260.22893494,611.40070557)
\curveto(260.12893148,611.42070568)(260.02893158,611.43570567)(259.92893494,611.44570557)
\curveto(259.82893178,611.45570565)(259.73393188,611.47070563)(259.64393494,611.49070557)
\curveto(259.53393208,611.52070558)(259.42393219,611.54570556)(259.31393494,611.56570557)
\curveto(259.2139324,611.59570551)(259.1089325,611.63570547)(258.99893494,611.68570557)
\curveto(258.62893298,611.84570526)(258.3139333,612.04570506)(258.05393494,612.28570557)
\curveto(257.79393382,612.53570457)(257.58393403,612.84570426)(257.42393494,613.21570557)
\curveto(257.38393423,613.3057038)(257.34893426,613.4007037)(257.31893494,613.50070557)
\curveto(257.28893432,613.6007035)(257.25893435,613.7057034)(257.22893494,613.81570557)
\curveto(257.2089344,613.86570324)(257.19893441,613.91570319)(257.19893494,613.96570557)
\curveto(257.19893441,614.02570308)(257.18893442,614.08570302)(257.16893494,614.14570557)
\curveto(257.14893446,614.2057029)(257.13893447,614.28570282)(257.13893494,614.38570557)
\curveto(257.13893447,614.48570262)(257.15393446,614.56070254)(257.18393494,614.61070557)
\curveto(257.19393442,614.64070246)(257.2089344,614.66570244)(257.22893494,614.68570557)
\lineto(257.28893494,614.74570557)
\curveto(257.32893428,614.76570234)(257.38893422,614.78070232)(257.46893494,614.79070557)
\curveto(257.55893405,614.8007023)(257.64893396,614.8057023)(257.73893494,614.80570557)
\curveto(257.82893378,614.8057023)(257.9139337,614.8007023)(257.99393494,614.79070557)
\curveto(258.08393353,614.78070232)(258.14893346,614.77070233)(258.18893494,614.76070557)
\curveto(258.2089334,614.74070236)(258.22893338,614.72570238)(258.24893494,614.71570557)
\curveto(258.26893334,614.71570239)(258.28893332,614.7057024)(258.30893494,614.68570557)
\curveto(258.37893323,614.59570251)(258.41893319,614.48070262)(258.42893494,614.34070557)
\curveto(258.44893316,614.2007029)(258.47893313,614.07570303)(258.51893494,613.96570557)
\lineto(258.66893494,613.60570557)
\curveto(258.71893289,613.49570361)(258.78393283,613.39070371)(258.86393494,613.29070557)
\curveto(258.88393273,613.26070384)(258.90393271,613.23570387)(258.92393494,613.21570557)
\curveto(258.95393266,613.19570391)(258.97893263,613.17070393)(258.99893494,613.14070557)
\curveto(259.03893257,613.08070402)(259.07393254,613.03570407)(259.10393494,613.00570557)
\curveto(259.14393247,612.97570413)(259.17893243,612.94570416)(259.20893494,612.91570557)
\curveto(259.24893236,612.88570422)(259.29393232,612.85570425)(259.34393494,612.82570557)
\curveto(259.43393218,612.76570434)(259.52893208,612.71570439)(259.62893494,612.67570557)
\lineto(259.95893494,612.55570557)
\curveto(260.1089315,612.5057046)(260.3089313,612.47570463)(260.55893494,612.46570557)
\curveto(260.8089308,612.45570465)(261.01893059,612.47570463)(261.18893494,612.52570557)
\curveto(261.26893034,612.54570456)(261.33893027,612.56070454)(261.39893494,612.57070557)
\lineto(261.60893494,612.63070557)
\curveto(261.88892972,612.75070435)(262.12892948,612.9007042)(262.32893494,613.08070557)
\curveto(262.53892907,613.26070384)(262.70392891,613.49070361)(262.82393494,613.77070557)
\curveto(262.85392876,613.84070326)(262.87392874,613.91070319)(262.88393494,613.98070557)
\lineto(262.94393494,614.22070557)
\curveto(262.98392863,614.36070274)(262.99392862,614.52070258)(262.97393494,614.70070557)
\curveto(262.95392866,614.89070221)(262.92392869,615.04070206)(262.88393494,615.15070557)
\curveto(262.75392886,615.53070157)(262.56892904,615.82070128)(262.32893494,616.02070557)
\curveto(262.09892951,616.22070088)(261.78892982,616.38070072)(261.39893494,616.50070557)
\curveto(261.28893032,616.53070057)(261.16893044,616.55070055)(261.03893494,616.56070557)
\curveto(260.91893069,616.57070053)(260.79393082,616.57570053)(260.66393494,616.57570557)
\curveto(260.50393111,616.57570053)(260.36393125,616.58070052)(260.24393494,616.59070557)
\curveto(260.12393149,616.6007005)(260.03893157,616.66070044)(259.98893494,616.77070557)
\curveto(259.96893164,616.8007003)(259.95893165,616.83570027)(259.95893494,616.87570557)
\lineto(259.95893494,617.01070557)
\curveto(259.94893166,617.11069999)(259.94893166,617.2056999)(259.95893494,617.29570557)
\curveto(259.97893163,617.38569972)(260.01893159,617.45069965)(260.07893494,617.49070557)
\curveto(260.11893149,617.52069958)(260.15893145,617.54069956)(260.19893494,617.55070557)
\curveto(260.24893136,617.56069954)(260.30393131,617.57069953)(260.36393494,617.58070557)
\curveto(260.38393123,617.59069951)(260.4089312,617.59069951)(260.43893494,617.58070557)
\curveto(260.46893114,617.58069952)(260.49393112,617.58569952)(260.51393494,617.59570557)
\lineto(260.64893494,617.59570557)
\curveto(260.75893085,617.61569949)(260.85893075,617.62569948)(260.94893494,617.62570557)
\curveto(261.04893056,617.63569947)(261.14393047,617.65569945)(261.23393494,617.68570557)
\curveto(261.55393006,617.79569931)(261.8089298,617.94069916)(261.99893494,618.12070557)
\curveto(262.18892942,618.3006988)(262.33892927,618.55069855)(262.44893494,618.87070557)
\curveto(262.47892913,618.97069813)(262.49892911,619.09569801)(262.50893494,619.24570557)
\curveto(262.52892908,619.4056977)(262.52392909,619.55069755)(262.49393494,619.68070557)
\curveto(262.47392914,619.75069735)(262.45392916,619.81569729)(262.43393494,619.87570557)
\curveto(262.42392919,619.94569716)(262.40392921,620.01069709)(262.37393494,620.07070557)
\curveto(262.27392934,620.31069679)(262.12892948,620.5006966)(261.93893494,620.64070557)
\curveto(261.74892986,620.78069632)(261.52393009,620.89069621)(261.26393494,620.97070557)
\curveto(261.20393041,620.99069611)(261.14393047,621.0006961)(261.08393494,621.00070557)
\curveto(261.02393059,621.0006961)(260.95893065,621.01069609)(260.88893494,621.03070557)
\curveto(260.8089308,621.05069605)(260.7139309,621.06069604)(260.60393494,621.06070557)
\curveto(260.49393112,621.06069604)(260.39893121,621.05069605)(260.31893494,621.03070557)
\curveto(260.26893134,621.01069609)(260.21893139,621.0006961)(260.16893494,621.00070557)
\curveto(260.12893148,621.0006961)(260.08393153,620.99069611)(260.03393494,620.97070557)
\curveto(259.85393176,620.92069618)(259.68393193,620.84569626)(259.52393494,620.74570557)
\curveto(259.37393224,620.65569645)(259.24393237,620.54069656)(259.13393494,620.40070557)
\curveto(259.04393257,620.28069682)(258.96393265,620.15069695)(258.89393494,620.01070557)
\curveto(258.82393279,619.87069723)(258.75893285,619.71569739)(258.69893494,619.54570557)
\curveto(258.66893294,619.43569767)(258.64893296,619.31569779)(258.63893494,619.18570557)
\curveto(258.62893298,619.06569804)(258.59393302,618.96569814)(258.53393494,618.88570557)
\curveto(258.5139331,618.84569826)(258.45393316,618.8056983)(258.35393494,618.76570557)
\curveto(258.3139333,618.75569835)(258.25393336,618.74569836)(258.17393494,618.73570557)
\lineto(257.91893494,618.73570557)
\curveto(257.82893378,618.74569836)(257.74393387,618.75569835)(257.66393494,618.76570557)
\curveto(257.59393402,618.77569833)(257.54393407,618.79069831)(257.51393494,618.81070557)
\curveto(257.47393414,618.84069826)(257.43893417,618.89569821)(257.40893494,618.97570557)
\curveto(257.37893423,619.05569805)(257.37393424,619.14069796)(257.39393494,619.23070557)
\curveto(257.40393421,619.28069782)(257.4089342,619.33069777)(257.40893494,619.38070557)
\lineto(257.43893494,619.56070557)
\curveto(257.46893414,619.66069744)(257.49393412,619.76069734)(257.51393494,619.86070557)
\curveto(257.54393407,619.96069714)(257.57893403,620.05069705)(257.61893494,620.13070557)
\curveto(257.66893394,620.24069686)(257.7139339,620.34569676)(257.75393494,620.44570557)
\curveto(257.79393382,620.55569655)(257.84393377,620.66069644)(257.90393494,620.76070557)
\curveto(258.23393338,621.3006958)(258.70393291,621.69569541)(259.31393494,621.94570557)
\curveto(259.43393218,621.99569511)(259.55893205,622.03069507)(259.68893494,622.05070557)
\curveto(259.82893178,622.07069503)(259.96893164,622.09569501)(260.10893494,622.12570557)
\curveto(260.16893144,622.13569497)(260.22893138,622.14069496)(260.28893494,622.14070557)
\curveto(260.35893125,622.14069496)(260.42393119,622.14569496)(260.48393494,622.15570557)
}
}
{
\newrgbcolor{curcolor}{0 0 0}
\pscustom[linestyle=none,fillstyle=solid,fillcolor=curcolor]
{
\newpath
\moveto(266.67354431,613.18570557)
\lineto(266.97354431,613.18570557)
\curveto(267.08354225,613.19570391)(267.18854215,613.19570391)(267.28854431,613.18570557)
\curveto(267.39854194,613.18570392)(267.49854184,613.17570393)(267.58854431,613.15570557)
\curveto(267.67854166,613.14570396)(267.74854159,613.12070398)(267.79854431,613.08070557)
\curveto(267.81854152,613.06070404)(267.8335415,613.03070407)(267.84354431,612.99070557)
\curveto(267.86354147,612.95070415)(267.88354145,612.9057042)(267.90354431,612.85570557)
\lineto(267.90354431,612.78070557)
\curveto(267.91354142,612.73070437)(267.91354142,612.67570443)(267.90354431,612.61570557)
\lineto(267.90354431,612.46570557)
\lineto(267.90354431,611.98570557)
\curveto(267.90354143,611.81570529)(267.86354147,611.69570541)(267.78354431,611.62570557)
\curveto(267.71354162,611.57570553)(267.62354171,611.55070555)(267.51354431,611.55070557)
\lineto(267.18354431,611.55070557)
\lineto(266.73354431,611.55070557)
\curveto(266.58354275,611.55070555)(266.46854287,611.58070552)(266.38854431,611.64070557)
\curveto(266.34854299,611.67070543)(266.31854302,611.72070538)(266.29854431,611.79070557)
\curveto(266.27854306,611.87070523)(266.26354307,611.95570515)(266.25354431,612.04570557)
\lineto(266.25354431,612.33070557)
\curveto(266.26354307,612.43070467)(266.26854307,612.51570459)(266.26854431,612.58570557)
\lineto(266.26854431,612.78070557)
\curveto(266.26854307,612.84070426)(266.27854306,612.89570421)(266.29854431,612.94570557)
\curveto(266.338543,613.05570405)(266.40854293,613.12570398)(266.50854431,613.15570557)
\curveto(266.5385428,613.15570395)(266.59354274,613.16570394)(266.67354431,613.18570557)
}
}
{
\newrgbcolor{curcolor}{0 0 0}
\pscustom[linestyle=none,fillstyle=solid,fillcolor=curcolor]
{
\newpath
\moveto(276.81870056,617.14570557)
\curveto(276.81869293,617.06570004)(276.82369292,616.98570012)(276.83370056,616.90570557)
\curveto(276.8436929,616.82570028)(276.83869291,616.75070035)(276.81870056,616.68070557)
\curveto(276.79869295,616.64070046)(276.79369295,616.59570051)(276.80370056,616.54570557)
\curveto(276.81369293,616.5057006)(276.81369293,616.46570064)(276.80370056,616.42570557)
\lineto(276.80370056,616.27570557)
\curveto(276.79369295,616.18570092)(276.78869296,616.09570101)(276.78870056,616.00570557)
\curveto(276.78869296,615.92570118)(276.78369296,615.84570126)(276.77370056,615.76570557)
\lineto(276.74370056,615.52570557)
\curveto(276.73369301,615.45570165)(276.72369302,615.38070172)(276.71370056,615.30070557)
\curveto(276.70369304,615.26070184)(276.69869305,615.22070188)(276.69870056,615.18070557)
\curveto(276.69869305,615.14070196)(276.69369305,615.09570201)(276.68370056,615.04570557)
\curveto(276.6436931,614.9057022)(276.61369313,614.76570234)(276.59370056,614.62570557)
\curveto(276.58369316,614.48570262)(276.55369319,614.35070275)(276.50370056,614.22070557)
\curveto(276.45369329,614.05070305)(276.39869335,613.88570322)(276.33870056,613.72570557)
\curveto(276.28869346,613.56570354)(276.22869352,613.41070369)(276.15870056,613.26070557)
\curveto(276.13869361,613.2007039)(276.10869364,613.14070396)(276.06870056,613.08070557)
\lineto(275.97870056,612.93070557)
\curveto(275.77869397,612.61070449)(275.56369418,612.34570476)(275.33370056,612.13570557)
\curveto(275.10369464,611.92570518)(274.80869494,611.74570536)(274.44870056,611.59570557)
\curveto(274.32869542,611.54570556)(274.19869555,611.51070559)(274.05870056,611.49070557)
\curveto(273.92869582,611.47070563)(273.79369595,611.44570566)(273.65370056,611.41570557)
\curveto(273.59369615,611.4057057)(273.53369621,611.4007057)(273.47370056,611.40070557)
\curveto(273.41369633,611.4007057)(273.3486964,611.39570571)(273.27870056,611.38570557)
\curveto(273.2486965,611.37570573)(273.19869655,611.37570573)(273.12870056,611.38570557)
\lineto(272.97870056,611.38570557)
\lineto(272.82870056,611.38570557)
\curveto(272.748697,611.4057057)(272.66369708,611.42070568)(272.57370056,611.43070557)
\curveto(272.49369725,611.43070567)(272.41869733,611.44070566)(272.34870056,611.46070557)
\curveto(272.30869744,611.47070563)(272.27369747,611.47570563)(272.24370056,611.47570557)
\curveto(272.22369752,611.46570564)(272.19869755,611.47070563)(272.16870056,611.49070557)
\lineto(271.89870056,611.55070557)
\curveto(271.80869794,611.58070552)(271.72369802,611.61070549)(271.64370056,611.64070557)
\curveto(271.06369868,611.88070522)(270.62869912,612.25070485)(270.33870056,612.75070557)
\curveto(270.25869949,612.88070422)(270.19369955,613.01570409)(270.14370056,613.15570557)
\curveto(270.10369964,613.29570381)(270.05869969,613.44570366)(270.00870056,613.60570557)
\curveto(269.98869976,613.68570342)(269.98369976,613.76570334)(269.99370056,613.84570557)
\curveto(270.01369973,613.92570318)(270.0486997,613.98070312)(270.09870056,614.01070557)
\curveto(270.12869962,614.03070307)(270.18369956,614.04570306)(270.26370056,614.05570557)
\curveto(270.3436994,614.07570303)(270.42869932,614.08570302)(270.51870056,614.08570557)
\curveto(270.60869914,614.09570301)(270.69369905,614.09570301)(270.77370056,614.08570557)
\curveto(270.86369888,614.07570303)(270.93369881,614.06570304)(270.98370056,614.05570557)
\curveto(271.00369874,614.04570306)(271.02869872,614.03070307)(271.05870056,614.01070557)
\curveto(271.09869865,613.99070311)(271.12869862,613.97070313)(271.14870056,613.95070557)
\curveto(271.20869854,613.87070323)(271.25369849,613.77570333)(271.28370056,613.66570557)
\curveto(271.32369842,613.55570355)(271.36869838,613.45570365)(271.41870056,613.36570557)
\curveto(271.66869808,612.97570413)(272.03869771,612.7057044)(272.52870056,612.55570557)
\curveto(272.59869715,612.53570457)(272.66869708,612.52070458)(272.73870056,612.51070557)
\curveto(272.81869693,612.51070459)(272.89869685,612.5007046)(272.97870056,612.48070557)
\curveto(273.01869673,612.47070463)(273.07369667,612.46570464)(273.14370056,612.46570557)
\curveto(273.22369652,612.46570464)(273.27869647,612.47070463)(273.30870056,612.48070557)
\curveto(273.33869641,612.49070461)(273.36869638,612.49570461)(273.39870056,612.49570557)
\lineto(273.50370056,612.49570557)
\curveto(273.58369616,612.51570459)(273.65869609,612.53570457)(273.72870056,612.55570557)
\curveto(273.80869594,612.57570453)(273.88369586,612.6007045)(273.95370056,612.63070557)
\curveto(274.30369544,612.78070432)(274.57369517,612.99570411)(274.76370056,613.27570557)
\curveto(274.95369479,613.55570355)(275.10869464,613.88070322)(275.22870056,614.25070557)
\curveto(275.25869449,614.33070277)(275.27869447,614.4057027)(275.28870056,614.47570557)
\curveto(275.30869444,614.54570256)(275.32869442,614.62070248)(275.34870056,614.70070557)
\curveto(275.36869438,614.79070231)(275.38369436,614.88570222)(275.39370056,614.98570557)
\curveto(275.41369433,615.09570201)(275.43369431,615.2007019)(275.45370056,615.30070557)
\curveto(275.46369428,615.35070175)(275.46869428,615.4007017)(275.46870056,615.45070557)
\curveto(275.47869427,615.51070159)(275.48369426,615.56570154)(275.48370056,615.61570557)
\curveto(275.50369424,615.67570143)(275.51369423,615.75070135)(275.51370056,615.84070557)
\curveto(275.51369423,615.94070116)(275.50369424,616.02070108)(275.48370056,616.08070557)
\curveto(275.45369429,616.17070093)(275.40369434,616.21070089)(275.33370056,616.20070557)
\curveto(275.27369447,616.19070091)(275.21869453,616.16070094)(275.16870056,616.11070557)
\curveto(275.08869466,616.06070104)(275.01869473,616.0007011)(274.95870056,615.93070557)
\curveto(274.90869484,615.86070124)(274.8436949,615.8007013)(274.76370056,615.75070557)
\curveto(274.60369514,615.64070146)(274.43869531,615.54070156)(274.26870056,615.45070557)
\curveto(274.09869565,615.37070173)(273.90369584,615.3007018)(273.68370056,615.24070557)
\curveto(273.58369616,615.21070189)(273.48369626,615.19570191)(273.38370056,615.19570557)
\curveto(273.29369645,615.19570191)(273.19369655,615.18570192)(273.08370056,615.16570557)
\lineto(272.93370056,615.16570557)
\curveto(272.88369686,615.18570192)(272.83369691,615.19070191)(272.78370056,615.18070557)
\curveto(272.743697,615.17070193)(272.70369704,615.17070193)(272.66370056,615.18070557)
\curveto(272.63369711,615.19070191)(272.58869716,615.19570191)(272.52870056,615.19570557)
\curveto(272.46869728,615.2057019)(272.40369734,615.21570189)(272.33370056,615.22570557)
\lineto(272.15370056,615.25570557)
\curveto(271.70369804,615.37570173)(271.32369842,615.54070156)(271.01370056,615.75070557)
\curveto(270.743699,615.94070116)(270.51369923,616.17070093)(270.32370056,616.44070557)
\curveto(270.1436996,616.72070038)(269.99869975,617.03570007)(269.88870056,617.38570557)
\lineto(269.82870056,617.59570557)
\curveto(269.81869993,617.67569943)(269.80369994,617.75569935)(269.78370056,617.83570557)
\curveto(269.77369997,617.86569924)(269.76869998,617.89569921)(269.76870056,617.92570557)
\curveto(269.76869998,617.95569915)(269.76369998,617.98569912)(269.75370056,618.01570557)
\curveto(269.7437,618.07569903)(269.73870001,618.13569897)(269.73870056,618.19570557)
\curveto(269.73870001,618.26569884)(269.72870002,618.32569878)(269.70870056,618.37570557)
\lineto(269.70870056,618.55570557)
\curveto(269.69870005,618.6056985)(269.69370005,618.67569843)(269.69370056,618.76570557)
\curveto(269.69370005,618.85569825)(269.70370004,618.92569818)(269.72370056,618.97570557)
\lineto(269.72370056,619.14070557)
\curveto(269.7437,619.22069788)(269.75369999,619.29569781)(269.75370056,619.36570557)
\curveto(269.76369998,619.43569767)(269.77869997,619.5056976)(269.79870056,619.57570557)
\curveto(269.85869989,619.77569733)(269.91869983,619.96569714)(269.97870056,620.14570557)
\curveto(270.0486997,620.32569678)(270.13869961,620.49569661)(270.24870056,620.65570557)
\curveto(270.28869946,620.72569638)(270.32869942,620.79069631)(270.36870056,620.85070557)
\lineto(270.51870056,621.03070557)
\curveto(270.53869921,621.04069606)(270.55869919,621.05569605)(270.57870056,621.07570557)
\curveto(270.66869908,621.2056959)(270.77869897,621.31569579)(270.90870056,621.40570557)
\curveto(271.16869858,621.6056955)(271.43369831,621.76069534)(271.70370056,621.87070557)
\curveto(271.78369796,621.91069519)(271.86369788,621.94069516)(271.94370056,621.96070557)
\curveto(272.03369771,621.99069511)(272.12369762,622.01569509)(272.21370056,622.03570557)
\curveto(272.31369743,622.06569504)(272.41369733,622.08569502)(272.51370056,622.09570557)
\curveto(272.61369713,622.105695)(272.71869703,622.12069498)(272.82870056,622.14070557)
\curveto(272.85869689,622.15069495)(272.89869685,622.15069495)(272.94870056,622.14070557)
\curveto(273.00869674,622.13069497)(273.0486967,622.13569497)(273.06870056,622.15570557)
\curveto(273.78869596,622.17569493)(274.38869536,622.06069504)(274.86870056,621.81070557)
\curveto(275.3486944,621.56069554)(275.72369402,621.22069588)(275.99370056,620.79070557)
\curveto(276.08369366,620.65069645)(276.16369358,620.5056966)(276.23370056,620.35570557)
\curveto(276.30369344,620.2056969)(276.37369337,620.04569706)(276.44370056,619.87570557)
\curveto(276.49369325,619.73569737)(276.53369321,619.58569752)(276.56370056,619.42570557)
\curveto(276.59369315,619.26569784)(276.62869312,619.105698)(276.66870056,618.94570557)
\curveto(276.68869306,618.89569821)(276.69869305,618.84069826)(276.69870056,618.78070557)
\curveto(276.69869305,618.73069837)(276.70369304,618.68069842)(276.71370056,618.63070557)
\curveto(276.73369301,618.57069853)(276.743693,618.5056986)(276.74370056,618.43570557)
\curveto(276.743693,618.37569873)(276.75369299,618.32069878)(276.77370056,618.27070557)
\lineto(276.77370056,618.10570557)
\curveto(276.79369295,618.05569905)(276.79869295,618.0056991)(276.78870056,617.95570557)
\curveto(276.77869297,617.9056992)(276.78369296,617.85569925)(276.80370056,617.80570557)
\curveto(276.80369294,617.78569932)(276.79869295,617.76069934)(276.78870056,617.73070557)
\curveto(276.78869296,617.7006994)(276.79369295,617.67569943)(276.80370056,617.65570557)
\curveto(276.81369293,617.62569948)(276.81369293,617.59069951)(276.80370056,617.55070557)
\curveto(276.80369294,617.51069959)(276.80869294,617.47069963)(276.81870056,617.43070557)
\curveto(276.82869292,617.39069971)(276.82869292,617.34569976)(276.81870056,617.29570557)
\lineto(276.81870056,617.14570557)
\moveto(275.31870056,618.45070557)
\curveto(275.32869442,618.5006986)(275.33369441,618.56069854)(275.33370056,618.63070557)
\curveto(275.33369441,618.7006984)(275.32869442,618.76069834)(275.31870056,618.81070557)
\curveto(275.30869444,618.86069824)(275.30369444,618.93569817)(275.30370056,619.03570557)
\curveto(275.28369446,619.11569799)(275.26369448,619.19069791)(275.24370056,619.26070557)
\curveto(275.23369451,619.33069777)(275.21869453,619.4006977)(275.19870056,619.47070557)
\curveto(275.05869469,619.9006972)(274.86369488,620.23569687)(274.61370056,620.47570557)
\curveto(274.37369537,620.71569639)(274.02869572,620.89569621)(273.57870056,621.01570557)
\curveto(273.48869626,621.03569607)(273.38869636,621.04569606)(273.27870056,621.04570557)
\lineto(272.94870056,621.04570557)
\curveto(272.92869682,621.02569608)(272.89369685,621.01569609)(272.84370056,621.01570557)
\curveto(272.79369695,621.02569608)(272.748697,621.02569608)(272.70870056,621.01570557)
\curveto(272.62869712,620.99569611)(272.55369719,620.97569613)(272.48370056,620.95570557)
\lineto(272.27370056,620.89570557)
\curveto(271.98369776,620.76569634)(271.75369799,620.58569652)(271.58370056,620.35570557)
\curveto(271.41369833,620.13569697)(271.27869847,619.87569723)(271.17870056,619.57570557)
\curveto(271.1486986,619.48569762)(271.12369862,619.39069771)(271.10370056,619.29070557)
\curveto(271.09369865,619.2006979)(271.07869867,619.105698)(271.05870056,619.00570557)
\lineto(271.05870056,618.87070557)
\curveto(271.02869872,618.76069834)(271.01869873,618.62069848)(271.02870056,618.45070557)
\curveto(271.0486987,618.29069881)(271.06869868,618.16069894)(271.08870056,618.06070557)
\curveto(271.10869864,618.0006991)(271.12369862,617.94069916)(271.13370056,617.88070557)
\curveto(271.1436986,617.83069927)(271.15869859,617.78069932)(271.17870056,617.73070557)
\curveto(271.25869849,617.53069957)(271.35369839,617.34069976)(271.46370056,617.16070557)
\curveto(271.58369816,616.98070012)(271.72369802,616.83570027)(271.88370056,616.72570557)
\curveto(271.93369781,616.67570043)(271.98869776,616.63570047)(272.04870056,616.60570557)
\curveto(272.10869764,616.57570053)(272.16869758,616.54070056)(272.22870056,616.50070557)
\curveto(272.37869737,616.42070068)(272.56369718,616.35570075)(272.78370056,616.30570557)
\curveto(272.83369691,616.28570082)(272.87369687,616.28070082)(272.90370056,616.29070557)
\curveto(272.9436968,616.3007008)(272.98869676,616.29570081)(273.03870056,616.27570557)
\curveto(273.07869667,616.26570084)(273.13369661,616.26070084)(273.20370056,616.26070557)
\curveto(273.27369647,616.26070084)(273.33369641,616.26570084)(273.38370056,616.27570557)
\curveto(273.48369626,616.29570081)(273.57869617,616.31070079)(273.66870056,616.32070557)
\curveto(273.75869599,616.34070076)(273.8486959,616.37070073)(273.93870056,616.41070557)
\curveto(274.47869527,616.63070047)(274.87369487,617.02570008)(275.12370056,617.59570557)
\curveto(275.17369457,617.69569941)(275.20869454,617.79569931)(275.22870056,617.89570557)
\curveto(275.2486945,618.0056991)(275.27369447,618.11569899)(275.30370056,618.22570557)
\curveto(275.30369444,618.32569878)(275.30869444,618.4006987)(275.31870056,618.45070557)
}
}
{
\newrgbcolor{curcolor}{0 0 0}
\pscustom[linestyle=none,fillstyle=solid,fillcolor=curcolor]
{
\newpath
\moveto(288.03330994,620.07070557)
\curveto(287.83329964,619.78069732)(287.62329985,619.49569761)(287.40330994,619.21570557)
\curveto(287.19330028,618.93569817)(286.98830048,618.65069845)(286.78830994,618.36070557)
\curveto(286.18830128,617.51069959)(285.58330189,616.67070043)(284.97330994,615.84070557)
\curveto(284.36330311,615.02070208)(283.75830371,614.18570292)(283.15830994,613.33570557)
\lineto(282.64830994,612.61570557)
\lineto(282.13830994,611.92570557)
\curveto(282.05830541,611.81570529)(281.97830549,611.7007054)(281.89830994,611.58070557)
\curveto(281.81830565,611.46070564)(281.72330575,611.36570574)(281.61330994,611.29570557)
\curveto(281.5733059,611.27570583)(281.50830596,611.26070584)(281.41830994,611.25070557)
\curveto(281.33830613,611.23070587)(281.24830622,611.22070588)(281.14830994,611.22070557)
\curveto(281.04830642,611.22070588)(280.95330652,611.22570588)(280.86330994,611.23570557)
\curveto(280.78330669,611.24570586)(280.72330675,611.26570584)(280.68330994,611.29570557)
\curveto(280.65330682,611.31570579)(280.62830684,611.35070575)(280.60830994,611.40070557)
\curveto(280.59830687,611.44070566)(280.60330687,611.48570562)(280.62330994,611.53570557)
\curveto(280.66330681,611.61570549)(280.70830676,611.69070541)(280.75830994,611.76070557)
\curveto(280.81830665,611.84070526)(280.8733066,611.92070518)(280.92330994,612.00070557)
\curveto(281.16330631,612.34070476)(281.40830606,612.67570443)(281.65830994,613.00570557)
\curveto(281.90830556,613.33570377)(282.14830532,613.67070343)(282.37830994,614.01070557)
\curveto(282.53830493,614.23070287)(282.69830477,614.44570266)(282.85830994,614.65570557)
\curveto(283.01830445,614.86570224)(283.17830429,615.08070202)(283.33830994,615.30070557)
\curveto(283.69830377,615.82070128)(284.06330341,616.33070077)(284.43330994,616.83070557)
\curveto(284.80330267,617.33069977)(285.1733023,617.84069926)(285.54330994,618.36070557)
\curveto(285.68330179,618.56069854)(285.82330165,618.75569835)(285.96330994,618.94570557)
\curveto(286.11330136,619.13569797)(286.25830121,619.33069777)(286.39830994,619.53070557)
\curveto(286.60830086,619.83069727)(286.82330065,620.13069697)(287.04330994,620.43070557)
\lineto(287.70330994,621.33070557)
\lineto(287.88330994,621.60070557)
\lineto(288.09330994,621.87070557)
\lineto(288.21330994,622.05070557)
\curveto(288.26329921,622.11069499)(288.31329916,622.16569494)(288.36330994,622.21570557)
\curveto(288.43329904,622.26569484)(288.50829896,622.3006948)(288.58830994,622.32070557)
\curveto(288.60829886,622.33069477)(288.63329884,622.33069477)(288.66330994,622.32070557)
\curveto(288.70329877,622.32069478)(288.73329874,622.33069477)(288.75330994,622.35070557)
\curveto(288.8732986,622.35069475)(289.00829846,622.34569476)(289.15830994,622.33570557)
\curveto(289.30829816,622.33569477)(289.39829807,622.29069481)(289.42830994,622.20070557)
\curveto(289.44829802,622.17069493)(289.45329802,622.13569497)(289.44330994,622.09570557)
\curveto(289.43329804,622.05569505)(289.41829805,622.02569508)(289.39830994,622.00570557)
\curveto(289.35829811,621.92569518)(289.31829815,621.85569525)(289.27830994,621.79570557)
\curveto(289.23829823,621.73569537)(289.19329828,621.67569543)(289.14330994,621.61570557)
\lineto(288.57330994,620.83570557)
\curveto(288.39329908,620.58569652)(288.21329926,620.33069677)(288.03330994,620.07070557)
\moveto(281.17830994,616.17070557)
\curveto(281.12830634,616.19070091)(281.07830639,616.19570091)(281.02830994,616.18570557)
\curveto(280.97830649,616.17570093)(280.92830654,616.18070092)(280.87830994,616.20070557)
\curveto(280.7683067,616.22070088)(280.66330681,616.24070086)(280.56330994,616.26070557)
\curveto(280.473307,616.29070081)(280.37830709,616.33070077)(280.27830994,616.38070557)
\curveto(279.94830752,616.52070058)(279.69330778,616.71570039)(279.51330994,616.96570557)
\curveto(279.33330814,617.22569988)(279.18830828,617.53569957)(279.07830994,617.89570557)
\curveto(279.04830842,617.97569913)(279.02830844,618.05569905)(279.01830994,618.13570557)
\curveto(279.00830846,618.22569888)(278.99330848,618.31069879)(278.97330994,618.39070557)
\curveto(278.96330851,618.44069866)(278.95830851,618.5056986)(278.95830994,618.58570557)
\curveto(278.94830852,618.61569849)(278.94330853,618.64569846)(278.94330994,618.67570557)
\curveto(278.94330853,618.71569839)(278.93830853,618.75069835)(278.92830994,618.78070557)
\lineto(278.92830994,618.93070557)
\curveto(278.91830855,618.98069812)(278.91330856,619.04069806)(278.91330994,619.11070557)
\curveto(278.91330856,619.19069791)(278.91830855,619.25569785)(278.92830994,619.30570557)
\lineto(278.92830994,619.47070557)
\curveto(278.94830852,619.52069758)(278.95330852,619.56569754)(278.94330994,619.60570557)
\curveto(278.94330853,619.65569745)(278.94830852,619.7006974)(278.95830994,619.74070557)
\curveto(278.9683085,619.78069732)(278.9733085,619.81569729)(278.97330994,619.84570557)
\curveto(278.9733085,619.88569722)(278.97830849,619.92569718)(278.98830994,619.96570557)
\curveto(279.01830845,620.07569703)(279.03830843,620.18569692)(279.04830994,620.29570557)
\curveto(279.0683084,620.41569669)(279.10330837,620.53069657)(279.15330994,620.64070557)
\curveto(279.29330818,620.98069612)(279.45330802,621.25569585)(279.63330994,621.46570557)
\curveto(279.82330765,621.68569542)(280.09330738,621.86569524)(280.44330994,622.00570557)
\curveto(280.52330695,622.03569507)(280.60830686,622.05569505)(280.69830994,622.06570557)
\curveto(280.78830668,622.08569502)(280.88330659,622.105695)(280.98330994,622.12570557)
\curveto(281.01330646,622.13569497)(281.0683064,622.13569497)(281.14830994,622.12570557)
\curveto(281.22830624,622.12569498)(281.27830619,622.13569497)(281.29830994,622.15570557)
\curveto(281.85830561,622.16569494)(282.30830516,622.05569505)(282.64830994,621.82570557)
\curveto(282.99830447,621.59569551)(283.25830421,621.29069581)(283.42830994,620.91070557)
\curveto(283.468304,620.82069628)(283.50330397,620.72569638)(283.53330994,620.62570557)
\curveto(283.56330391,620.52569658)(283.58830388,620.42569668)(283.60830994,620.32570557)
\curveto(283.62830384,620.29569681)(283.63330384,620.26569684)(283.62330994,620.23570557)
\curveto(283.62330385,620.2056969)(283.62830384,620.17569693)(283.63830994,620.14570557)
\curveto(283.6683038,620.03569707)(283.68830378,619.91069719)(283.69830994,619.77070557)
\curveto(283.70830376,619.64069746)(283.71830375,619.5056976)(283.72830994,619.36570557)
\lineto(283.72830994,619.20070557)
\curveto(283.73830373,619.14069796)(283.73830373,619.08569802)(283.72830994,619.03570557)
\curveto(283.71830375,618.98569812)(283.71330376,618.93569817)(283.71330994,618.88570557)
\lineto(283.71330994,618.75070557)
\curveto(283.70330377,618.71069839)(283.69830377,618.67069843)(283.69830994,618.63070557)
\curveto(283.70830376,618.59069851)(283.70330377,618.54569856)(283.68330994,618.49570557)
\curveto(283.66330381,618.38569872)(283.64330383,618.28069882)(283.62330994,618.18070557)
\curveto(283.61330386,618.08069902)(283.59330388,617.98069912)(283.56330994,617.88070557)
\curveto(283.43330404,617.52069958)(283.2683042,617.2056999)(283.06830994,616.93570557)
\curveto(282.8683046,616.66570044)(282.59330488,616.46070064)(282.24330994,616.32070557)
\curveto(282.16330531,616.29070081)(282.07830539,616.26570084)(281.98830994,616.24570557)
\lineto(281.71830994,616.18570557)
\curveto(281.6683058,616.17570093)(281.62330585,616.17070093)(281.58330994,616.17070557)
\curveto(281.54330593,616.18070092)(281.50330597,616.18070092)(281.46330994,616.17070557)
\curveto(281.36330611,616.15070095)(281.2683062,616.15070095)(281.17830994,616.17070557)
\moveto(280.33830994,617.56570557)
\curveto(280.37830709,617.49569961)(280.41830705,617.43069967)(280.45830994,617.37070557)
\curveto(280.49830697,617.32069978)(280.54830692,617.27069983)(280.60830994,617.22070557)
\lineto(280.75830994,617.10070557)
\curveto(280.81830665,617.07070003)(280.88330659,617.04570006)(280.95330994,617.02570557)
\curveto(280.99330648,617.0057001)(281.02830644,616.99570011)(281.05830994,616.99570557)
\curveto(281.09830637,617.0057001)(281.13830633,617.0007001)(281.17830994,616.98070557)
\curveto(281.20830626,616.98070012)(281.24830622,616.97570013)(281.29830994,616.96570557)
\curveto(281.34830612,616.96570014)(281.38830608,616.97070013)(281.41830994,616.98070557)
\lineto(281.64330994,617.02570557)
\curveto(281.89330558,617.1057)(282.07830539,617.23069987)(282.19830994,617.40070557)
\curveto(282.27830519,617.5006996)(282.34830512,617.63069947)(282.40830994,617.79070557)
\curveto(282.48830498,617.97069913)(282.54830492,618.19569891)(282.58830994,618.46570557)
\curveto(282.62830484,618.74569836)(282.64330483,619.02569808)(282.63330994,619.30570557)
\curveto(282.62330485,619.59569751)(282.59330488,619.87069723)(282.54330994,620.13070557)
\curveto(282.49330498,620.39069671)(282.41830505,620.6006965)(282.31830994,620.76070557)
\curveto(282.19830527,620.96069614)(282.04830542,621.11069599)(281.86830994,621.21070557)
\curveto(281.78830568,621.26069584)(281.69830577,621.29069581)(281.59830994,621.30070557)
\curveto(281.49830597,621.32069578)(281.39330608,621.33069577)(281.28330994,621.33070557)
\curveto(281.26330621,621.32069578)(281.23830623,621.31569579)(281.20830994,621.31570557)
\curveto(281.18830628,621.32569578)(281.1683063,621.32569578)(281.14830994,621.31570557)
\curveto(281.09830637,621.3056958)(281.05330642,621.29569581)(281.01330994,621.28570557)
\curveto(280.9733065,621.28569582)(280.93330654,621.27569583)(280.89330994,621.25570557)
\curveto(280.71330676,621.17569593)(280.56330691,621.05569605)(280.44330994,620.89570557)
\curveto(280.33330714,620.73569637)(280.24330723,620.55569655)(280.17330994,620.35570557)
\curveto(280.11330736,620.16569694)(280.0683074,619.94069716)(280.03830994,619.68070557)
\curveto(280.01830745,619.42069768)(280.01330746,619.15569795)(280.02330994,618.88570557)
\curveto(280.03330744,618.62569848)(280.06330741,618.37569873)(280.11330994,618.13570557)
\curveto(280.1733073,617.9056992)(280.24830722,617.71569939)(280.33830994,617.56570557)
\moveto(291.13830994,614.58070557)
\curveto(291.14829632,614.53070257)(291.15329632,614.44070266)(291.15330994,614.31070557)
\curveto(291.15329632,614.18070292)(291.14329633,614.09070301)(291.12330994,614.04070557)
\curveto(291.10329637,613.99070311)(291.09829637,613.93570317)(291.10830994,613.87570557)
\curveto(291.11829635,613.82570328)(291.11829635,613.77570333)(291.10830994,613.72570557)
\curveto(291.0682964,613.58570352)(291.03829643,613.45070365)(291.01830994,613.32070557)
\curveto(291.00829646,613.19070391)(290.97829649,613.07070403)(290.92830994,612.96070557)
\curveto(290.78829668,612.61070449)(290.62329685,612.31570479)(290.43330994,612.07570557)
\curveto(290.24329723,611.84570526)(289.9732975,611.66070544)(289.62330994,611.52070557)
\curveto(289.54329793,611.49070561)(289.45829801,611.47070563)(289.36830994,611.46070557)
\curveto(289.27829819,611.44070566)(289.19329828,611.42070568)(289.11330994,611.40070557)
\curveto(289.06329841,611.39070571)(289.01329846,611.38570572)(288.96330994,611.38570557)
\curveto(288.91329856,611.38570572)(288.86329861,611.38070572)(288.81330994,611.37070557)
\curveto(288.78329869,611.36070574)(288.73329874,611.36070574)(288.66330994,611.37070557)
\curveto(288.59329888,611.37070573)(288.54329893,611.37570573)(288.51330994,611.38570557)
\curveto(288.45329902,611.4057057)(288.39329908,611.41570569)(288.33330994,611.41570557)
\curveto(288.28329919,611.4057057)(288.23329924,611.41070569)(288.18330994,611.43070557)
\curveto(288.09329938,611.45070565)(288.00329947,611.47570563)(287.91330994,611.50570557)
\curveto(287.83329964,611.52570558)(287.75329972,611.55570555)(287.67330994,611.59570557)
\curveto(287.35330012,611.73570537)(287.10330037,611.93070517)(286.92330994,612.18070557)
\curveto(286.74330073,612.44070466)(286.59330088,612.74570436)(286.47330994,613.09570557)
\curveto(286.45330102,613.17570393)(286.43830103,613.26070384)(286.42830994,613.35070557)
\curveto(286.41830105,613.44070366)(286.40330107,613.52570358)(286.38330994,613.60570557)
\curveto(286.3733011,613.63570347)(286.3683011,613.66570344)(286.36830994,613.69570557)
\lineto(286.36830994,613.80070557)
\curveto(286.34830112,613.88070322)(286.33830113,613.96070314)(286.33830994,614.04070557)
\lineto(286.33830994,614.17570557)
\curveto(286.31830115,614.27570283)(286.31830115,614.37570273)(286.33830994,614.47570557)
\lineto(286.33830994,614.65570557)
\curveto(286.34830112,614.7057024)(286.35330112,614.75070235)(286.35330994,614.79070557)
\curveto(286.35330112,614.84070226)(286.35830111,614.88570222)(286.36830994,614.92570557)
\curveto(286.37830109,614.96570214)(286.38330109,615.0007021)(286.38330994,615.03070557)
\curveto(286.38330109,615.07070203)(286.38830108,615.11070199)(286.39830994,615.15070557)
\lineto(286.45830994,615.48070557)
\curveto(286.47830099,615.6007015)(286.50830096,615.71070139)(286.54830994,615.81070557)
\curveto(286.68830078,616.14070096)(286.84830062,616.41570069)(287.02830994,616.63570557)
\curveto(287.21830025,616.86570024)(287.47829999,617.05070005)(287.80830994,617.19070557)
\curveto(287.88829958,617.23069987)(287.9732995,617.25569985)(288.06330994,617.26570557)
\lineto(288.36330994,617.32570557)
\lineto(288.49830994,617.32570557)
\curveto(288.54829892,617.33569977)(288.59829887,617.34069976)(288.64830994,617.34070557)
\curveto(289.21829825,617.36069974)(289.67829779,617.25569985)(290.02830994,617.02570557)
\curveto(290.38829708,616.8057003)(290.65329682,616.5057006)(290.82330994,616.12570557)
\curveto(290.8732966,616.02570108)(290.91329656,615.92570118)(290.94330994,615.82570557)
\curveto(290.9732965,615.72570138)(291.00329647,615.62070148)(291.03330994,615.51070557)
\curveto(291.04329643,615.47070163)(291.04829642,615.43570167)(291.04830994,615.40570557)
\curveto(291.04829642,615.38570172)(291.05329642,615.35570175)(291.06330994,615.31570557)
\curveto(291.08329639,615.24570186)(291.09329638,615.17070193)(291.09330994,615.09070557)
\curveto(291.09329638,615.01070209)(291.10329637,614.93070217)(291.12330994,614.85070557)
\curveto(291.12329635,614.8007023)(291.12329635,614.75570235)(291.12330994,614.71570557)
\curveto(291.12329635,614.67570243)(291.12829634,614.63070247)(291.13830994,614.58070557)
\moveto(290.02830994,614.14570557)
\curveto(290.03829743,614.19570291)(290.04329743,614.27070283)(290.04330994,614.37070557)
\curveto(290.05329742,614.47070263)(290.04829742,614.54570256)(290.02830994,614.59570557)
\curveto(290.00829746,614.65570245)(290.00329747,614.71070239)(290.01330994,614.76070557)
\curveto(290.03329744,614.82070228)(290.03329744,614.88070222)(290.01330994,614.94070557)
\curveto(290.00329747,614.97070213)(289.99829747,615.0057021)(289.99830994,615.04570557)
\curveto(289.99829747,615.08570202)(289.99329748,615.12570198)(289.98330994,615.16570557)
\curveto(289.96329751,615.24570186)(289.94329753,615.32070178)(289.92330994,615.39070557)
\curveto(289.91329756,615.47070163)(289.89829757,615.55070155)(289.87830994,615.63070557)
\curveto(289.84829762,615.69070141)(289.82329765,615.75070135)(289.80330994,615.81070557)
\curveto(289.78329769,615.87070123)(289.75329772,615.93070117)(289.71330994,615.99070557)
\curveto(289.61329786,616.16070094)(289.48329799,616.29570081)(289.32330994,616.39570557)
\curveto(289.24329823,616.44570066)(289.14829832,616.48070062)(289.03830994,616.50070557)
\curveto(288.92829854,616.52070058)(288.80329867,616.53070057)(288.66330994,616.53070557)
\curveto(288.64329883,616.52070058)(288.61829885,616.51570059)(288.58830994,616.51570557)
\curveto(288.55829891,616.52570058)(288.52829894,616.52570058)(288.49830994,616.51570557)
\lineto(288.34830994,616.45570557)
\curveto(288.29829917,616.44570066)(288.25329922,616.43070067)(288.21330994,616.41070557)
\curveto(288.02329945,616.3007008)(287.87829959,616.15570095)(287.77830994,615.97570557)
\curveto(287.68829978,615.79570131)(287.60829986,615.59070151)(287.53830994,615.36070557)
\curveto(287.49829997,615.23070187)(287.47829999,615.09570201)(287.47830994,614.95570557)
\curveto(287.47829999,614.82570228)(287.4683,614.68070242)(287.44830994,614.52070557)
\curveto(287.43830003,614.47070263)(287.42830004,614.41070269)(287.41830994,614.34070557)
\curveto(287.41830005,614.27070283)(287.42830004,614.21070289)(287.44830994,614.16070557)
\lineto(287.44830994,613.99570557)
\lineto(287.44830994,613.81570557)
\curveto(287.45830001,613.76570334)(287.4683,613.71070339)(287.47830994,613.65070557)
\curveto(287.48829998,613.6007035)(287.49329998,613.54570356)(287.49330994,613.48570557)
\curveto(287.50329997,613.42570368)(287.51829995,613.37070373)(287.53830994,613.32070557)
\curveto(287.58829988,613.13070397)(287.64829982,612.95570415)(287.71830994,612.79570557)
\curveto(287.78829968,612.63570447)(287.89329958,612.5057046)(288.03330994,612.40570557)
\curveto(288.16329931,612.3057048)(288.30329917,612.23570487)(288.45330994,612.19570557)
\curveto(288.48329899,612.18570492)(288.50829896,612.18070492)(288.52830994,612.18070557)
\curveto(288.55829891,612.19070491)(288.58829888,612.19070491)(288.61830994,612.18070557)
\curveto(288.63829883,612.18070492)(288.6682988,612.17570493)(288.70830994,612.16570557)
\curveto(288.74829872,612.16570494)(288.78329869,612.17070493)(288.81330994,612.18070557)
\curveto(288.85329862,612.19070491)(288.89329858,612.19570491)(288.93330994,612.19570557)
\curveto(288.9732985,612.19570491)(289.01329846,612.2057049)(289.05330994,612.22570557)
\curveto(289.29329818,612.3057048)(289.48829798,612.44070466)(289.63830994,612.63070557)
\curveto(289.75829771,612.81070429)(289.84829762,613.01570409)(289.90830994,613.24570557)
\curveto(289.92829754,613.31570379)(289.94329753,613.38570372)(289.95330994,613.45570557)
\curveto(289.96329751,613.53570357)(289.97829749,613.61570349)(289.99830994,613.69570557)
\curveto(289.99829747,613.75570335)(290.00329747,613.8007033)(290.01330994,613.83070557)
\curveto(290.01329746,613.85070325)(290.01329746,613.87570323)(290.01330994,613.90570557)
\curveto(290.01329746,613.94570316)(290.01829745,613.97570313)(290.02830994,613.99570557)
\lineto(290.02830994,614.14570557)
}
}
{
\newrgbcolor{curcolor}{0 0 0}
\pscustom[linestyle=none,fillstyle=solid,fillcolor=curcolor]
{
\newpath
\moveto(524.60575195,617.4828833)
\curveto(524.61574423,617.44288025)(524.61574423,617.3928803)(524.60575195,617.3328833)
\curveto(524.60574424,617.27288042)(524.60074425,617.22288047)(524.59075195,617.1828833)
\curveto(524.59074426,617.14288055)(524.58574426,617.10288059)(524.57575195,617.0628833)
\lineto(524.57575195,616.9578833)
\curveto(524.55574429,616.87788082)(524.54074431,616.7978809)(524.53075195,616.7178833)
\curveto(524.52074433,616.63788106)(524.50074435,616.56288113)(524.47075195,616.4928833)
\curveto(524.4507444,616.41288128)(524.43074442,616.33788136)(524.41075195,616.2678833)
\curveto(524.39074446,616.1978815)(524.36074449,616.12288157)(524.32075195,616.0428833)
\curveto(524.14074471,615.62288207)(523.88574496,615.28288241)(523.55575195,615.0228833)
\curveto(523.22574562,614.76288293)(522.83574601,614.55788314)(522.38575195,614.4078833)
\curveto(522.26574658,614.36788333)(522.14074671,614.34288335)(522.01075195,614.3328833)
\curveto(521.89074696,614.31288338)(521.76574708,614.28788341)(521.63575195,614.2578833)
\curveto(521.57574727,614.24788345)(521.51074734,614.24288345)(521.44075195,614.2428833)
\curveto(521.38074747,614.24288345)(521.31574753,614.23788346)(521.24575195,614.2278833)
\lineto(521.12575195,614.2278833)
\lineto(520.93075195,614.2278833)
\curveto(520.87074798,614.21788348)(520.81574803,614.22288347)(520.76575195,614.2428833)
\curveto(520.69574815,614.26288343)(520.63074822,614.26788343)(520.57075195,614.2578833)
\curveto(520.51074834,614.24788345)(520.4507484,614.25288344)(520.39075195,614.2728833)
\curveto(520.34074851,614.28288341)(520.29574855,614.28788341)(520.25575195,614.2878833)
\curveto(520.21574863,614.28788341)(520.17074868,614.2978834)(520.12075195,614.3178833)
\curveto(520.04074881,614.33788336)(519.96574888,614.35788334)(519.89575195,614.3778833)
\curveto(519.82574902,614.38788331)(519.75574909,614.40288329)(519.68575195,614.4228833)
\curveto(519.20574964,614.5928831)(518.80575004,614.80288289)(518.48575195,615.0528833)
\curveto(518.17575067,615.31288238)(517.92575092,615.66788203)(517.73575195,616.1178833)
\curveto(517.70575114,616.17788152)(517.68075117,616.23788146)(517.66075195,616.2978833)
\curveto(517.6507512,616.36788133)(517.63575121,616.44288125)(517.61575195,616.5228833)
\curveto(517.59575125,616.58288111)(517.58075127,616.64788105)(517.57075195,616.7178833)
\curveto(517.56075129,616.78788091)(517.5457513,616.85788084)(517.52575195,616.9278833)
\curveto(517.51575133,616.97788072)(517.51075134,617.01788068)(517.51075195,617.0478833)
\lineto(517.51075195,617.1678833)
\curveto(517.50075135,617.20788049)(517.49075136,617.25788044)(517.48075195,617.3178833)
\curveto(517.48075137,617.37788032)(517.48575136,617.42788027)(517.49575195,617.4678833)
\lineto(517.49575195,617.6028833)
\curveto(517.50575134,617.65288004)(517.51075134,617.70287999)(517.51075195,617.7528833)
\curveto(517.53075132,617.85287984)(517.5457513,617.94787975)(517.55575195,618.0378833)
\curveto(517.56575128,618.13787956)(517.58575126,618.23287946)(517.61575195,618.3228833)
\curveto(517.66575118,618.47287922)(517.72075113,618.61287908)(517.78075195,618.7428833)
\curveto(517.84075101,618.87287882)(517.91075094,618.9928787)(517.99075195,619.1028833)
\curveto(518.02075083,619.15287854)(518.0507508,619.1928785)(518.08075195,619.2228833)
\curveto(518.12075073,619.25287844)(518.15575069,619.28787841)(518.18575195,619.3278833)
\curveto(518.2457506,619.40787829)(518.31575053,619.47787822)(518.39575195,619.5378833)
\curveto(518.45575039,619.58787811)(518.51575033,619.63287806)(518.57575195,619.6728833)
\lineto(518.78575195,619.8228833)
\curveto(518.83575001,619.86287783)(518.88574996,619.8978778)(518.93575195,619.9278833)
\curveto(518.98574986,619.96787773)(519.02074983,620.02287767)(519.04075195,620.0928833)
\curveto(519.04074981,620.12287757)(519.03074982,620.14787755)(519.01075195,620.1678833)
\curveto(519.00074985,620.1978775)(518.99074986,620.22287747)(518.98075195,620.2428833)
\curveto(518.94074991,620.2928774)(518.89074996,620.33787736)(518.83075195,620.3778833)
\curveto(518.78075007,620.42787727)(518.73075012,620.47287722)(518.68075195,620.5128833)
\curveto(518.64075021,620.54287715)(518.59075026,620.5978771)(518.53075195,620.6778833)
\curveto(518.51075034,620.70787699)(518.48075037,620.73287696)(518.44075195,620.7528833)
\curveto(518.41075044,620.78287691)(518.38575046,620.81787688)(518.36575195,620.8578833)
\curveto(518.19575065,621.06787663)(518.06575078,621.31287638)(517.97575195,621.5928833)
\curveto(517.95575089,621.67287602)(517.94075091,621.75287594)(517.93075195,621.8328833)
\curveto(517.92075093,621.91287578)(517.90575094,621.9928757)(517.88575195,622.0728833)
\curveto(517.86575098,622.12287557)(517.85575099,622.18787551)(517.85575195,622.2678833)
\curveto(517.85575099,622.35787534)(517.86575098,622.42787527)(517.88575195,622.4778833)
\curveto(517.88575096,622.57787512)(517.89075096,622.64787505)(517.90075195,622.6878833)
\curveto(517.92075093,622.76787493)(517.93575091,622.83787486)(517.94575195,622.8978833)
\curveto(517.95575089,622.96787473)(517.97075088,623.03787466)(517.99075195,623.1078833)
\curveto(518.14075071,623.53787416)(518.35575049,623.88287381)(518.63575195,624.1428833)
\curveto(518.92574992,624.40287329)(519.27574957,624.61787308)(519.68575195,624.7878833)
\curveto(519.79574905,624.83787286)(519.91074894,624.86787283)(520.03075195,624.8778833)
\curveto(520.16074869,624.8978728)(520.29074856,624.92787277)(520.42075195,624.9678833)
\curveto(520.50074835,624.96787273)(520.57074828,624.96787273)(520.63075195,624.9678833)
\curveto(520.70074815,624.97787272)(520.77574807,624.98787271)(520.85575195,624.9978833)
\curveto(521.6457472,625.01787268)(522.30074655,624.88787281)(522.82075195,624.6078833)
\curveto(523.3507455,624.32787337)(523.73074512,623.91787378)(523.96075195,623.3778833)
\curveto(524.07074478,623.14787455)(524.14074471,622.86287483)(524.17075195,622.5228833)
\curveto(524.21074464,622.1928755)(524.18074467,621.88787581)(524.08075195,621.6078833)
\curveto(524.04074481,621.47787622)(523.99074486,621.35787634)(523.93075195,621.2478833)
\curveto(523.88074497,621.13787656)(523.82074503,621.03287666)(523.75075195,620.9328833)
\curveto(523.73074512,620.8928768)(523.70074515,620.85787684)(523.66075195,620.8278833)
\lineto(523.57075195,620.7378833)
\curveto(523.52074533,620.64787705)(523.46074539,620.58287711)(523.39075195,620.5428833)
\curveto(523.34074551,620.4928772)(523.28574556,620.44287725)(523.22575195,620.3928833)
\curveto(523.17574567,620.35287734)(523.13074572,620.30787739)(523.09075195,620.2578833)
\curveto(523.07074578,620.23787746)(523.0507458,620.21287748)(523.03075195,620.1828833)
\curveto(523.02074583,620.16287753)(523.02074583,620.13787756)(523.03075195,620.1078833)
\curveto(523.04074581,620.05787764)(523.07074578,620.00787769)(523.12075195,619.9578833)
\curveto(523.17074568,619.91787778)(523.22574562,619.87787782)(523.28575195,619.8378833)
\lineto(523.46575195,619.7178833)
\curveto(523.52574532,619.68787801)(523.57574527,619.65787804)(523.61575195,619.6278833)
\curveto(523.9457449,619.38787831)(524.19574465,619.07787862)(524.36575195,618.6978833)
\curveto(524.40574444,618.61787908)(524.43574441,618.53287916)(524.45575195,618.4428833)
\curveto(524.48574436,618.35287934)(524.51074434,618.26287943)(524.53075195,618.1728833)
\curveto(524.54074431,618.12287957)(524.5507443,618.06787963)(524.56075195,618.0078833)
\lineto(524.59075195,617.8578833)
\curveto(524.60074425,617.7978799)(524.60074425,617.73287996)(524.59075195,617.6628833)
\curveto(524.58074427,617.60288009)(524.58574426,617.54288015)(524.60575195,617.4828833)
\moveto(519.22075195,622.5228833)
\curveto(519.19074966,622.41287528)(519.18574966,622.27287542)(519.20575195,622.1028833)
\curveto(519.22574962,621.94287575)(519.2507496,621.81787588)(519.28075195,621.7278833)
\curveto(519.39074946,621.40787629)(519.54074931,621.16287653)(519.73075195,620.9928833)
\curveto(519.92074893,620.83287686)(520.18574866,620.70287699)(520.52575195,620.6028833)
\curveto(520.65574819,620.57287712)(520.82074803,620.54787715)(521.02075195,620.5278833)
\curveto(521.22074763,620.51787718)(521.39074746,620.53287716)(521.53075195,620.5728833)
\curveto(521.82074703,620.65287704)(522.06074679,620.76287693)(522.25075195,620.9028833)
\curveto(522.4507464,621.05287664)(522.60574624,621.25287644)(522.71575195,621.5028833)
\curveto(522.73574611,621.55287614)(522.7457461,621.5978761)(522.74575195,621.6378833)
\curveto(522.75574609,621.67787602)(522.77074608,621.72287597)(522.79075195,621.7728833)
\curveto(522.82074603,621.88287581)(522.84074601,622.02287567)(522.85075195,622.1928833)
\curveto(522.86074599,622.36287533)(522.850746,622.50787519)(522.82075195,622.6278833)
\curveto(522.80074605,622.71787498)(522.77574607,622.80287489)(522.74575195,622.8828833)
\curveto(522.72574612,622.96287473)(522.69074616,623.04287465)(522.64075195,623.1228833)
\curveto(522.47074638,623.3928743)(522.2457466,623.58787411)(521.96575195,623.7078833)
\curveto(521.69574715,623.82787387)(521.33574751,623.88787381)(520.88575195,623.8878833)
\curveto(520.86574798,623.86787383)(520.83574801,623.86287383)(520.79575195,623.8728833)
\curveto(520.75574809,623.88287381)(520.72074813,623.88287381)(520.69075195,623.8728833)
\curveto(520.64074821,623.85287384)(520.58574826,623.83787386)(520.52575195,623.8278833)
\curveto(520.47574837,623.82787387)(520.42574842,623.81787388)(520.37575195,623.7978833)
\curveto(520.13574871,623.70787399)(519.92574892,623.5928741)(519.74575195,623.4528833)
\curveto(519.56574928,623.32287437)(519.42574942,623.14287455)(519.32575195,622.9128833)
\curveto(519.30574954,622.85287484)(519.28574956,622.78787491)(519.26575195,622.7178833)
\curveto(519.25574959,622.65787504)(519.24074961,622.5928751)(519.22075195,622.5228833)
\moveto(523.24075195,616.9878833)
\curveto(523.29074556,617.17788052)(523.29574555,617.38288031)(523.25575195,617.6028833)
\curveto(523.22574562,617.82287987)(523.18074567,618.00287969)(523.12075195,618.1428833)
\curveto(522.9507459,618.51287918)(522.69074616,618.81787888)(522.34075195,619.0578833)
\curveto(522.00074685,619.2978784)(521.56574728,619.41787828)(521.03575195,619.4178833)
\curveto(521.00574784,619.3978783)(520.96574788,619.3928783)(520.91575195,619.4028833)
\curveto(520.86574798,619.42287827)(520.82574802,619.42787827)(520.79575195,619.4178833)
\lineto(520.52575195,619.3578833)
\curveto(520.4457484,619.34787835)(520.36574848,619.33287836)(520.28575195,619.3128833)
\curveto(519.98574886,619.20287849)(519.72074913,619.05787864)(519.49075195,618.8778833)
\curveto(519.27074958,618.697879)(519.10074975,618.46787923)(518.98075195,618.1878833)
\curveto(518.9507499,618.10787959)(518.92574992,618.02787967)(518.90575195,617.9478833)
\curveto(518.88574996,617.86787983)(518.86574998,617.78287991)(518.84575195,617.6928833)
\curveto(518.81575003,617.57288012)(518.80575004,617.42288027)(518.81575195,617.2428833)
\curveto(518.83575001,617.06288063)(518.86074999,616.92288077)(518.89075195,616.8228833)
\curveto(518.91074994,616.77288092)(518.92074993,616.72788097)(518.92075195,616.6878833)
\curveto(518.93074992,616.65788104)(518.9457499,616.61788108)(518.96575195,616.5678833)
\curveto(519.06574978,616.34788135)(519.19574965,616.14788155)(519.35575195,615.9678833)
\curveto(519.52574932,615.78788191)(519.72074913,615.65288204)(519.94075195,615.5628833)
\curveto(520.01074884,615.52288217)(520.10574874,615.48788221)(520.22575195,615.4578833)
\curveto(520.4457484,615.36788233)(520.70074815,615.32288237)(520.99075195,615.3228833)
\lineto(521.27575195,615.3228833)
\curveto(521.37574747,615.34288235)(521.47074738,615.35788234)(521.56075195,615.3678833)
\curveto(521.6507472,615.37788232)(521.74074711,615.3978823)(521.83075195,615.4278833)
\curveto(522.09074676,615.50788219)(522.33074652,615.63788206)(522.55075195,615.8178833)
\curveto(522.78074607,616.00788169)(522.9507459,616.22288147)(523.06075195,616.4628833)
\curveto(523.10074575,616.54288115)(523.13074572,616.62288107)(523.15075195,616.7028833)
\curveto(523.18074567,616.7928809)(523.21074564,616.88788081)(523.24075195,616.9878833)
}
}
{
\newrgbcolor{curcolor}{0 0 0}
\pscustom[linestyle=none,fillstyle=solid,fillcolor=curcolor]
{
\newpath
\moveto(526.89536133,616.0428833)
\lineto(527.19536133,616.0428833)
\curveto(527.30535927,616.05288164)(527.41035916,616.05288164)(527.51036133,616.0428833)
\curveto(527.62035895,616.04288165)(527.72035885,616.03288166)(527.81036133,616.0128833)
\curveto(527.90035867,616.00288169)(527.9703586,615.97788172)(528.02036133,615.9378833)
\curveto(528.04035853,615.91788178)(528.05535852,615.88788181)(528.06536133,615.8478833)
\curveto(528.08535849,615.80788189)(528.10535847,615.76288193)(528.12536133,615.7128833)
\lineto(528.12536133,615.6378833)
\curveto(528.13535844,615.58788211)(528.13535844,615.53288216)(528.12536133,615.4728833)
\lineto(528.12536133,615.3228833)
\lineto(528.12536133,614.8428833)
\curveto(528.12535845,614.67288302)(528.08535849,614.55288314)(528.00536133,614.4828833)
\curveto(527.93535864,614.43288326)(527.84535873,614.40788329)(527.73536133,614.4078833)
\lineto(527.40536133,614.4078833)
\lineto(526.95536133,614.4078833)
\curveto(526.80535977,614.40788329)(526.69035988,614.43788326)(526.61036133,614.4978833)
\curveto(526.57036,614.52788317)(526.54036003,614.57788312)(526.52036133,614.6478833)
\curveto(526.50036007,614.72788297)(526.48536009,614.81288288)(526.47536133,614.9028833)
\lineto(526.47536133,615.1878833)
\curveto(526.48536009,615.28788241)(526.49036008,615.37288232)(526.49036133,615.4428833)
\lineto(526.49036133,615.6378833)
\curveto(526.49036008,615.697882)(526.50036007,615.75288194)(526.52036133,615.8028833)
\curveto(526.56036001,615.91288178)(526.63035994,615.98288171)(526.73036133,616.0128833)
\curveto(526.76035981,616.01288168)(526.81535976,616.02288167)(526.89536133,616.0428833)
}
}
{
\newrgbcolor{curcolor}{0 0 0}
\pscustom[linestyle=none,fillstyle=solid,fillcolor=curcolor]
{
\newpath
\moveto(534.16051758,625.0128833)
\curveto(534.26051272,625.01287268)(534.35551263,625.00287269)(534.44551758,624.9828833)
\curveto(534.53551245,624.97287272)(534.60051238,624.94287275)(534.64051758,624.8928833)
\curveto(534.70051228,624.81287288)(534.73051225,624.70787299)(534.73051758,624.5778833)
\lineto(534.73051758,624.1878833)
\lineto(534.73051758,622.6878833)
\lineto(534.73051758,616.2978833)
\lineto(534.73051758,615.1278833)
\lineto(534.73051758,614.8128833)
\curveto(534.74051224,614.71288298)(534.72551226,614.63288306)(534.68551758,614.5728833)
\curveto(534.63551235,614.4928832)(534.56051242,614.44288325)(534.46051758,614.4228833)
\curveto(534.37051261,614.41288328)(534.26051272,614.40788329)(534.13051758,614.4078833)
\lineto(533.90551758,614.4078833)
\curveto(533.82551316,614.42788327)(533.75551323,614.44288325)(533.69551758,614.4528833)
\curveto(533.63551335,614.47288322)(533.5855134,614.51288318)(533.54551758,614.5728833)
\curveto(533.50551348,614.63288306)(533.4855135,614.70788299)(533.48551758,614.7978833)
\lineto(533.48551758,615.0978833)
\lineto(533.48551758,616.1928833)
\lineto(533.48551758,621.5328833)
\curveto(533.46551352,621.62287607)(533.45051353,621.697876)(533.44051758,621.7578833)
\curveto(533.44051354,621.82787587)(533.41051357,621.88787581)(533.35051758,621.9378833)
\curveto(533.2805137,621.98787571)(533.19051379,622.01287568)(533.08051758,622.0128833)
\curveto(532.980514,622.02287567)(532.87051411,622.02787567)(532.75051758,622.0278833)
\lineto(531.61051758,622.0278833)
\lineto(531.11551758,622.0278833)
\curveto(530.95551603,622.03787566)(530.84551614,622.0978756)(530.78551758,622.2078833)
\curveto(530.76551622,622.23787546)(530.75551623,622.26787543)(530.75551758,622.2978833)
\curveto(530.75551623,622.33787536)(530.75051623,622.38287531)(530.74051758,622.4328833)
\curveto(530.72051626,622.55287514)(530.72551626,622.66287503)(530.75551758,622.7628833)
\curveto(530.79551619,622.86287483)(530.85051613,622.93287476)(530.92051758,622.9728833)
\curveto(531.00051598,623.02287467)(531.12051586,623.04787465)(531.28051758,623.0478833)
\curveto(531.44051554,623.04787465)(531.57551541,623.06287463)(531.68551758,623.0928833)
\curveto(531.73551525,623.10287459)(531.79051519,623.10787459)(531.85051758,623.1078833)
\curveto(531.91051507,623.11787458)(531.97051501,623.13287456)(532.03051758,623.1528833)
\curveto(532.1805148,623.20287449)(532.32551466,623.25287444)(532.46551758,623.3028833)
\curveto(532.60551438,623.36287433)(532.74051424,623.43287426)(532.87051758,623.5128833)
\curveto(533.01051397,623.60287409)(533.13051385,623.70787399)(533.23051758,623.8278833)
\curveto(533.33051365,623.94787375)(533.42551356,624.07787362)(533.51551758,624.2178833)
\curveto(533.57551341,624.31787338)(533.62051336,624.42787327)(533.65051758,624.5478833)
\curveto(533.69051329,624.66787303)(533.74051324,624.77287292)(533.80051758,624.8628833)
\curveto(533.85051313,624.92287277)(533.92051306,624.96287273)(534.01051758,624.9828833)
\curveto(534.03051295,624.9928727)(534.05551293,624.9978727)(534.08551758,624.9978833)
\curveto(534.11551287,624.9978727)(534.14051284,625.00287269)(534.16051758,625.0128833)
}
}
{
\newrgbcolor{curcolor}{0 0 0}
\pscustom[linestyle=none,fillstyle=solid,fillcolor=curcolor]
{
\newpath
\moveto(548.25512695,622.9278833)
\curveto(548.05511665,622.63787506)(547.84511686,622.35287534)(547.62512695,622.0728833)
\curveto(547.41511729,621.7928759)(547.2101175,621.50787619)(547.01012695,621.2178833)
\curveto(546.4101183,620.36787733)(545.8051189,619.52787817)(545.19512695,618.6978833)
\curveto(544.58512012,617.87787982)(543.98012073,617.04288065)(543.38012695,616.1928833)
\lineto(542.87012695,615.4728833)
\lineto(542.36012695,614.7828833)
\curveto(542.28012243,614.67288302)(542.20012251,614.55788314)(542.12012695,614.4378833)
\curveto(542.04012267,614.31788338)(541.94512276,614.22288347)(541.83512695,614.1528833)
\curveto(541.79512291,614.13288356)(541.73012298,614.11788358)(541.64012695,614.1078833)
\curveto(541.56012315,614.08788361)(541.47012324,614.07788362)(541.37012695,614.0778833)
\curveto(541.27012344,614.07788362)(541.17512353,614.08288361)(541.08512695,614.0928833)
\curveto(541.0051237,614.10288359)(540.94512376,614.12288357)(540.90512695,614.1528833)
\curveto(540.87512383,614.17288352)(540.85012386,614.20788349)(540.83012695,614.2578833)
\curveto(540.82012389,614.2978834)(540.82512388,614.34288335)(540.84512695,614.3928833)
\curveto(540.88512382,614.47288322)(540.93012378,614.54788315)(540.98012695,614.6178833)
\curveto(541.04012367,614.697883)(541.09512361,614.77788292)(541.14512695,614.8578833)
\curveto(541.38512332,615.1978825)(541.63012308,615.53288216)(541.88012695,615.8628833)
\curveto(542.13012258,616.1928815)(542.37012234,616.52788117)(542.60012695,616.8678833)
\curveto(542.76012195,617.08788061)(542.92012179,617.30288039)(543.08012695,617.5128833)
\curveto(543.24012147,617.72287997)(543.40012131,617.93787976)(543.56012695,618.1578833)
\curveto(543.92012079,618.67787902)(544.28512042,619.18787851)(544.65512695,619.6878833)
\curveto(545.02511968,620.18787751)(545.39511931,620.697877)(545.76512695,621.2178833)
\curveto(545.9051188,621.41787628)(546.04511866,621.61287608)(546.18512695,621.8028833)
\curveto(546.33511837,621.9928757)(546.48011823,622.18787551)(546.62012695,622.3878833)
\curveto(546.83011788,622.68787501)(547.04511766,622.98787471)(547.26512695,623.2878833)
\lineto(547.92512695,624.1878833)
\lineto(548.10512695,624.4578833)
\lineto(548.31512695,624.7278833)
\lineto(548.43512695,624.9078833)
\curveto(548.48511622,624.96787273)(548.53511617,625.02287267)(548.58512695,625.0728833)
\curveto(548.65511605,625.12287257)(548.73011598,625.15787254)(548.81012695,625.1778833)
\curveto(548.83011588,625.18787251)(548.85511585,625.18787251)(548.88512695,625.1778833)
\curveto(548.92511578,625.17787252)(548.95511575,625.18787251)(548.97512695,625.2078833)
\curveto(549.09511561,625.20787249)(549.23011548,625.20287249)(549.38012695,625.1928833)
\curveto(549.53011518,625.1928725)(549.62011509,625.14787255)(549.65012695,625.0578833)
\curveto(549.67011504,625.02787267)(549.67511503,624.9928727)(549.66512695,624.9528833)
\curveto(549.65511505,624.91287278)(549.64011507,624.88287281)(549.62012695,624.8628833)
\curveto(549.58011513,624.78287291)(549.54011517,624.71287298)(549.50012695,624.6528833)
\curveto(549.46011525,624.5928731)(549.41511529,624.53287316)(549.36512695,624.4728833)
\lineto(548.79512695,623.6928833)
\curveto(548.61511609,623.44287425)(548.43511627,623.18787451)(548.25512695,622.9278833)
\moveto(541.40012695,619.0278833)
\curveto(541.35012336,619.04787865)(541.30012341,619.05287864)(541.25012695,619.0428833)
\curveto(541.20012351,619.03287866)(541.15012356,619.03787866)(541.10012695,619.0578833)
\curveto(540.99012372,619.07787862)(540.88512382,619.0978786)(540.78512695,619.1178833)
\curveto(540.69512401,619.14787855)(540.60012411,619.18787851)(540.50012695,619.2378833)
\curveto(540.17012454,619.37787832)(539.91512479,619.57287812)(539.73512695,619.8228833)
\curveto(539.55512515,620.08287761)(539.4101253,620.3928773)(539.30012695,620.7528833)
\curveto(539.27012544,620.83287686)(539.25012546,620.91287678)(539.24012695,620.9928833)
\curveto(539.23012548,621.08287661)(539.21512549,621.16787653)(539.19512695,621.2478833)
\curveto(539.18512552,621.2978764)(539.18012553,621.36287633)(539.18012695,621.4428833)
\curveto(539.17012554,621.47287622)(539.16512554,621.50287619)(539.16512695,621.5328833)
\curveto(539.16512554,621.57287612)(539.16012555,621.60787609)(539.15012695,621.6378833)
\lineto(539.15012695,621.7878833)
\curveto(539.14012557,621.83787586)(539.13512557,621.8978758)(539.13512695,621.9678833)
\curveto(539.13512557,622.04787565)(539.14012557,622.11287558)(539.15012695,622.1628833)
\lineto(539.15012695,622.3278833)
\curveto(539.17012554,622.37787532)(539.17512553,622.42287527)(539.16512695,622.4628833)
\curveto(539.16512554,622.51287518)(539.17012554,622.55787514)(539.18012695,622.5978833)
\curveto(539.19012552,622.63787506)(539.19512551,622.67287502)(539.19512695,622.7028833)
\curveto(539.19512551,622.74287495)(539.20012551,622.78287491)(539.21012695,622.8228833)
\curveto(539.24012547,622.93287476)(539.26012545,623.04287465)(539.27012695,623.1528833)
\curveto(539.29012542,623.27287442)(539.32512538,623.38787431)(539.37512695,623.4978833)
\curveto(539.51512519,623.83787386)(539.67512503,624.11287358)(539.85512695,624.3228833)
\curveto(540.04512466,624.54287315)(540.31512439,624.72287297)(540.66512695,624.8628833)
\curveto(540.74512396,624.8928728)(540.83012388,624.91287278)(540.92012695,624.9228833)
\curveto(541.0101237,624.94287275)(541.1051236,624.96287273)(541.20512695,624.9828833)
\curveto(541.23512347,624.9928727)(541.29012342,624.9928727)(541.37012695,624.9828833)
\curveto(541.45012326,624.98287271)(541.50012321,624.9928727)(541.52012695,625.0128833)
\curveto(542.08012263,625.02287267)(542.53012218,624.91287278)(542.87012695,624.6828833)
\curveto(543.22012149,624.45287324)(543.48012123,624.14787355)(543.65012695,623.7678833)
\curveto(543.69012102,623.67787402)(543.72512098,623.58287411)(543.75512695,623.4828833)
\curveto(543.78512092,623.38287431)(543.8101209,623.28287441)(543.83012695,623.1828833)
\curveto(543.85012086,623.15287454)(543.85512085,623.12287457)(543.84512695,623.0928833)
\curveto(543.84512086,623.06287463)(543.85012086,623.03287466)(543.86012695,623.0028833)
\curveto(543.89012082,622.8928748)(543.9101208,622.76787493)(543.92012695,622.6278833)
\curveto(543.93012078,622.4978752)(543.94012077,622.36287533)(543.95012695,622.2228833)
\lineto(543.95012695,622.0578833)
\curveto(543.96012075,621.9978757)(543.96012075,621.94287575)(543.95012695,621.8928833)
\curveto(543.94012077,621.84287585)(543.93512077,621.7928759)(543.93512695,621.7428833)
\lineto(543.93512695,621.6078833)
\curveto(543.92512078,621.56787613)(543.92012079,621.52787617)(543.92012695,621.4878833)
\curveto(543.93012078,621.44787625)(543.92512078,621.40287629)(543.90512695,621.3528833)
\curveto(543.88512082,621.24287645)(543.86512084,621.13787656)(543.84512695,621.0378833)
\curveto(543.83512087,620.93787676)(543.81512089,620.83787686)(543.78512695,620.7378833)
\curveto(543.65512105,620.37787732)(543.49012122,620.06287763)(543.29012695,619.7928833)
\curveto(543.09012162,619.52287817)(542.81512189,619.31787838)(542.46512695,619.1778833)
\curveto(542.38512232,619.14787855)(542.30012241,619.12287857)(542.21012695,619.1028833)
\lineto(541.94012695,619.0428833)
\curveto(541.89012282,619.03287866)(541.84512286,619.02787867)(541.80512695,619.0278833)
\curveto(541.76512294,619.03787866)(541.72512298,619.03787866)(541.68512695,619.0278833)
\curveto(541.58512312,619.00787869)(541.49012322,619.00787869)(541.40012695,619.0278833)
\moveto(540.56012695,620.4228833)
\curveto(540.60012411,620.35287734)(540.64012407,620.28787741)(540.68012695,620.2278833)
\curveto(540.72012399,620.17787752)(540.77012394,620.12787757)(540.83012695,620.0778833)
\lineto(540.98012695,619.9578833)
\curveto(541.04012367,619.92787777)(541.1051236,619.90287779)(541.17512695,619.8828833)
\curveto(541.21512349,619.86287783)(541.25012346,619.85287784)(541.28012695,619.8528833)
\curveto(541.32012339,619.86287783)(541.36012335,619.85787784)(541.40012695,619.8378833)
\curveto(541.43012328,619.83787786)(541.47012324,619.83287786)(541.52012695,619.8228833)
\curveto(541.57012314,619.82287787)(541.6101231,619.82787787)(541.64012695,619.8378833)
\lineto(541.86512695,619.8828833)
\curveto(542.11512259,619.96287773)(542.30012241,620.08787761)(542.42012695,620.2578833)
\curveto(542.50012221,620.35787734)(542.57012214,620.48787721)(542.63012695,620.6478833)
\curveto(542.710122,620.82787687)(542.77012194,621.05287664)(542.81012695,621.3228833)
\curveto(542.85012186,621.60287609)(542.86512184,621.88287581)(542.85512695,622.1628833)
\curveto(542.84512186,622.45287524)(542.81512189,622.72787497)(542.76512695,622.9878833)
\curveto(542.71512199,623.24787445)(542.64012207,623.45787424)(542.54012695,623.6178833)
\curveto(542.42012229,623.81787388)(542.27012244,623.96787373)(542.09012695,624.0678833)
\curveto(542.0101227,624.11787358)(541.92012279,624.14787355)(541.82012695,624.1578833)
\curveto(541.72012299,624.17787352)(541.61512309,624.18787351)(541.50512695,624.1878833)
\curveto(541.48512322,624.17787352)(541.46012325,624.17287352)(541.43012695,624.1728833)
\curveto(541.4101233,624.18287351)(541.39012332,624.18287351)(541.37012695,624.1728833)
\curveto(541.32012339,624.16287353)(541.27512343,624.15287354)(541.23512695,624.1428833)
\curveto(541.19512351,624.14287355)(541.15512355,624.13287356)(541.11512695,624.1128833)
\curveto(540.93512377,624.03287366)(540.78512392,623.91287378)(540.66512695,623.7528833)
\curveto(540.55512415,623.5928741)(540.46512424,623.41287428)(540.39512695,623.2128833)
\curveto(540.33512437,623.02287467)(540.29012442,622.7978749)(540.26012695,622.5378833)
\curveto(540.24012447,622.27787542)(540.23512447,622.01287568)(540.24512695,621.7428833)
\curveto(540.25512445,621.48287621)(540.28512442,621.23287646)(540.33512695,620.9928833)
\curveto(540.39512431,620.76287693)(540.47012424,620.57287712)(540.56012695,620.4228833)
\moveto(551.36012695,617.4378833)
\curveto(551.37011334,617.38788031)(551.37511333,617.2978804)(551.37512695,617.1678833)
\curveto(551.37511333,617.03788066)(551.36511334,616.94788075)(551.34512695,616.8978833)
\curveto(551.32511338,616.84788085)(551.32011339,616.7928809)(551.33012695,616.7328833)
\curveto(551.34011337,616.68288101)(551.34011337,616.63288106)(551.33012695,616.5828833)
\curveto(551.29011342,616.44288125)(551.26011345,616.30788139)(551.24012695,616.1778833)
\curveto(551.23011348,616.04788165)(551.20011351,615.92788177)(551.15012695,615.8178833)
\curveto(551.0101137,615.46788223)(550.84511386,615.17288252)(550.65512695,614.9328833)
\curveto(550.46511424,614.70288299)(550.19511451,614.51788318)(549.84512695,614.3778833)
\curveto(549.76511494,614.34788335)(549.68011503,614.32788337)(549.59012695,614.3178833)
\curveto(549.50011521,614.2978834)(549.41511529,614.27788342)(549.33512695,614.2578833)
\curveto(549.28511542,614.24788345)(549.23511547,614.24288345)(549.18512695,614.2428833)
\curveto(549.13511557,614.24288345)(549.08511562,614.23788346)(549.03512695,614.2278833)
\curveto(549.0051157,614.21788348)(548.95511575,614.21788348)(548.88512695,614.2278833)
\curveto(548.81511589,614.22788347)(548.76511594,614.23288346)(548.73512695,614.2428833)
\curveto(548.67511603,614.26288343)(548.61511609,614.27288342)(548.55512695,614.2728833)
\curveto(548.5051162,614.26288343)(548.45511625,614.26788343)(548.40512695,614.2878833)
\curveto(548.31511639,614.30788339)(548.22511648,614.33288336)(548.13512695,614.3628833)
\curveto(548.05511665,614.38288331)(547.97511673,614.41288328)(547.89512695,614.4528833)
\curveto(547.57511713,614.5928831)(547.32511738,614.78788291)(547.14512695,615.0378833)
\curveto(546.96511774,615.2978824)(546.81511789,615.60288209)(546.69512695,615.9528833)
\curveto(546.67511803,616.03288166)(546.66011805,616.11788158)(546.65012695,616.2078833)
\curveto(546.64011807,616.2978814)(546.62511808,616.38288131)(546.60512695,616.4628833)
\curveto(546.59511811,616.4928812)(546.59011812,616.52288117)(546.59012695,616.5528833)
\lineto(546.59012695,616.6578833)
\curveto(546.57011814,616.73788096)(546.56011815,616.81788088)(546.56012695,616.8978833)
\lineto(546.56012695,617.0328833)
\curveto(546.54011817,617.13288056)(546.54011817,617.23288046)(546.56012695,617.3328833)
\lineto(546.56012695,617.5128833)
\curveto(546.57011814,617.56288013)(546.57511813,617.60788009)(546.57512695,617.6478833)
\curveto(546.57511813,617.69788)(546.58011813,617.74287995)(546.59012695,617.7828833)
\curveto(546.60011811,617.82287987)(546.6051181,617.85787984)(546.60512695,617.8878833)
\curveto(546.6051181,617.92787977)(546.6101181,617.96787973)(546.62012695,618.0078833)
\lineto(546.68012695,618.3378833)
\curveto(546.70011801,618.45787924)(546.73011798,618.56787913)(546.77012695,618.6678833)
\curveto(546.9101178,618.9978787)(547.07011764,619.27287842)(547.25012695,619.4928833)
\curveto(547.44011727,619.72287797)(547.70011701,619.90787779)(548.03012695,620.0478833)
\curveto(548.1101166,620.08787761)(548.19511651,620.11287758)(548.28512695,620.1228833)
\lineto(548.58512695,620.1828833)
\lineto(548.72012695,620.1828833)
\curveto(548.77011594,620.1928775)(548.82011589,620.1978775)(548.87012695,620.1978833)
\curveto(549.44011527,620.21787748)(549.90011481,620.11287758)(550.25012695,619.8828833)
\curveto(550.6101141,619.66287803)(550.87511383,619.36287833)(551.04512695,618.9828833)
\curveto(551.09511361,618.88287881)(551.13511357,618.78287891)(551.16512695,618.6828833)
\curveto(551.19511351,618.58287911)(551.22511348,618.47787922)(551.25512695,618.3678833)
\curveto(551.26511344,618.32787937)(551.27011344,618.2928794)(551.27012695,618.2628833)
\curveto(551.27011344,618.24287945)(551.27511343,618.21287948)(551.28512695,618.1728833)
\curveto(551.3051134,618.10287959)(551.31511339,618.02787967)(551.31512695,617.9478833)
\curveto(551.31511339,617.86787983)(551.32511338,617.78787991)(551.34512695,617.7078833)
\curveto(551.34511336,617.65788004)(551.34511336,617.61288008)(551.34512695,617.5728833)
\curveto(551.34511336,617.53288016)(551.35011336,617.48788021)(551.36012695,617.4378833)
\moveto(550.25012695,617.0028833)
\curveto(550.26011445,617.05288064)(550.26511444,617.12788057)(550.26512695,617.2278833)
\curveto(550.27511443,617.32788037)(550.27011444,617.40288029)(550.25012695,617.4528833)
\curveto(550.23011448,617.51288018)(550.22511448,617.56788013)(550.23512695,617.6178833)
\curveto(550.25511445,617.67788002)(550.25511445,617.73787996)(550.23512695,617.7978833)
\curveto(550.22511448,617.82787987)(550.22011449,617.86287983)(550.22012695,617.9028833)
\curveto(550.22011449,617.94287975)(550.21511449,617.98287971)(550.20512695,618.0228833)
\curveto(550.18511452,618.10287959)(550.16511454,618.17787952)(550.14512695,618.2478833)
\curveto(550.13511457,618.32787937)(550.12011459,618.40787929)(550.10012695,618.4878833)
\curveto(550.07011464,618.54787915)(550.04511466,618.60787909)(550.02512695,618.6678833)
\curveto(550.0051147,618.72787897)(549.97511473,618.78787891)(549.93512695,618.8478833)
\curveto(549.83511487,619.01787868)(549.705115,619.15287854)(549.54512695,619.2528833)
\curveto(549.46511524,619.30287839)(549.37011534,619.33787836)(549.26012695,619.3578833)
\curveto(549.15011556,619.37787832)(549.02511568,619.38787831)(548.88512695,619.3878833)
\curveto(548.86511584,619.37787832)(548.84011587,619.37287832)(548.81012695,619.3728833)
\curveto(548.78011593,619.38287831)(548.75011596,619.38287831)(548.72012695,619.3728833)
\lineto(548.57012695,619.3128833)
\curveto(548.52011619,619.30287839)(548.47511623,619.28787841)(548.43512695,619.2678833)
\curveto(548.24511646,619.15787854)(548.10011661,619.01287868)(548.00012695,618.8328833)
\curveto(547.9101168,618.65287904)(547.83011688,618.44787925)(547.76012695,618.2178833)
\curveto(547.72011699,618.08787961)(547.70011701,617.95287974)(547.70012695,617.8128833)
\curveto(547.70011701,617.68288001)(547.69011702,617.53788016)(547.67012695,617.3778833)
\curveto(547.66011705,617.32788037)(547.65011706,617.26788043)(547.64012695,617.1978833)
\curveto(547.64011707,617.12788057)(547.65011706,617.06788063)(547.67012695,617.0178833)
\lineto(547.67012695,616.8528833)
\lineto(547.67012695,616.6728833)
\curveto(547.68011703,616.62288107)(547.69011702,616.56788113)(547.70012695,616.5078833)
\curveto(547.710117,616.45788124)(547.71511699,616.40288129)(547.71512695,616.3428833)
\curveto(547.72511698,616.28288141)(547.74011697,616.22788147)(547.76012695,616.1778833)
\curveto(547.8101169,615.98788171)(547.87011684,615.81288188)(547.94012695,615.6528833)
\curveto(548.0101167,615.4928822)(548.11511659,615.36288233)(548.25512695,615.2628833)
\curveto(548.38511632,615.16288253)(548.52511618,615.0928826)(548.67512695,615.0528833)
\curveto(548.705116,615.04288265)(548.73011598,615.03788266)(548.75012695,615.0378833)
\curveto(548.78011593,615.04788265)(548.8101159,615.04788265)(548.84012695,615.0378833)
\curveto(548.86011585,615.03788266)(548.89011582,615.03288266)(548.93012695,615.0228833)
\curveto(548.97011574,615.02288267)(549.0051157,615.02788267)(549.03512695,615.0378833)
\curveto(549.07511563,615.04788265)(549.11511559,615.05288264)(549.15512695,615.0528833)
\curveto(549.19511551,615.05288264)(549.23511547,615.06288263)(549.27512695,615.0828833)
\curveto(549.51511519,615.16288253)(549.710115,615.2978824)(549.86012695,615.4878833)
\curveto(549.98011473,615.66788203)(550.07011464,615.87288182)(550.13012695,616.1028833)
\curveto(550.15011456,616.17288152)(550.16511454,616.24288145)(550.17512695,616.3128833)
\curveto(550.18511452,616.3928813)(550.20011451,616.47288122)(550.22012695,616.5528833)
\curveto(550.22011449,616.61288108)(550.22511448,616.65788104)(550.23512695,616.6878833)
\curveto(550.23511447,616.70788099)(550.23511447,616.73288096)(550.23512695,616.7628833)
\curveto(550.23511447,616.80288089)(550.24011447,616.83288086)(550.25012695,616.8528833)
\lineto(550.25012695,617.0028833)
}
}
{
\newrgbcolor{curcolor}{0 0 0}
\pscustom[linestyle=none,fillstyle=solid,fillcolor=curcolor]
{
\newpath
\moveto(225.02456482,293.03217773)
\lineto(228.62456482,293.03217773)
\lineto(229.26956482,293.03217773)
\curveto(229.34955829,293.03216731)(229.42455821,293.02716731)(229.49456482,293.01717773)
\curveto(229.56455807,293.01716732)(229.62455801,293.00716733)(229.67456482,292.98717773)
\curveto(229.74455789,292.95716738)(229.79955784,292.89716744)(229.83956482,292.80717773)
\curveto(229.85955778,292.77716756)(229.86955777,292.7371676)(229.86956482,292.68717773)
\lineto(229.86956482,292.55217773)
\curveto(229.87955776,292.4421679)(229.87455776,292.337168)(229.85456482,292.23717773)
\curveto(229.84455779,292.1371682)(229.80955783,292.06716827)(229.74956482,292.02717773)
\curveto(229.65955798,291.95716838)(229.52455811,291.92216842)(229.34456482,291.92217773)
\curveto(229.16455847,291.93216841)(228.99955864,291.9371684)(228.84956482,291.93717773)
\lineto(226.85456482,291.93717773)
\lineto(226.35956482,291.93717773)
\lineto(226.22456482,291.93717773)
\curveto(226.18456145,291.9371684)(226.14456149,291.93216841)(226.10456482,291.92217773)
\lineto(225.89456482,291.92217773)
\curveto(225.78456185,291.89216845)(225.70456193,291.85216849)(225.65456482,291.80217773)
\curveto(225.60456203,291.76216858)(225.56956207,291.70716863)(225.54956482,291.63717773)
\curveto(225.52956211,291.57716876)(225.51456212,291.50716883)(225.50456482,291.42717773)
\curveto(225.49456214,291.34716899)(225.47456216,291.25716908)(225.44456482,291.15717773)
\curveto(225.39456224,290.95716938)(225.35456228,290.75216959)(225.32456482,290.54217773)
\curveto(225.29456234,290.33217001)(225.25456238,290.12717021)(225.20456482,289.92717773)
\curveto(225.18456245,289.85717048)(225.17456246,289.78717055)(225.17456482,289.71717773)
\curveto(225.17456246,289.65717068)(225.16456247,289.59217075)(225.14456482,289.52217773)
\curveto(225.1345625,289.49217085)(225.12456251,289.45217089)(225.11456482,289.40217773)
\curveto(225.11456252,289.36217098)(225.11956252,289.32217102)(225.12956482,289.28217773)
\curveto(225.14956249,289.23217111)(225.17456246,289.18717115)(225.20456482,289.14717773)
\curveto(225.24456239,289.11717122)(225.30456233,289.11217123)(225.38456482,289.13217773)
\curveto(225.44456219,289.15217119)(225.50456213,289.17717116)(225.56456482,289.20717773)
\curveto(225.62456201,289.24717109)(225.68456195,289.28217106)(225.74456482,289.31217773)
\curveto(225.80456183,289.33217101)(225.85456178,289.34717099)(225.89456482,289.35717773)
\curveto(226.08456155,289.4371709)(226.28956135,289.49217085)(226.50956482,289.52217773)
\curveto(226.7395609,289.55217079)(226.96956067,289.56217078)(227.19956482,289.55217773)
\curveto(227.4395602,289.55217079)(227.66955997,289.52717081)(227.88956482,289.47717773)
\curveto(228.10955953,289.4371709)(228.30955933,289.37717096)(228.48956482,289.29717773)
\curveto(228.5395591,289.27717106)(228.58455905,289.25717108)(228.62456482,289.23717773)
\curveto(228.67455896,289.21717112)(228.72455891,289.19217115)(228.77456482,289.16217773)
\curveto(229.12455851,288.95217139)(229.40455823,288.72217162)(229.61456482,288.47217773)
\curveto(229.8345578,288.22217212)(230.02955761,287.89717244)(230.19956482,287.49717773)
\curveto(230.24955739,287.38717295)(230.28455735,287.27717306)(230.30456482,287.16717773)
\curveto(230.32455731,287.05717328)(230.34955729,286.9421734)(230.37956482,286.82217773)
\curveto(230.38955725,286.79217355)(230.39455724,286.74717359)(230.39456482,286.68717773)
\curveto(230.41455722,286.62717371)(230.42455721,286.55717378)(230.42456482,286.47717773)
\curveto(230.42455721,286.40717393)(230.4345572,286.342174)(230.45456482,286.28217773)
\lineto(230.45456482,286.11717773)
\curveto(230.46455717,286.06717427)(230.46955717,285.99717434)(230.46956482,285.90717773)
\curveto(230.46955717,285.81717452)(230.45955718,285.74717459)(230.43956482,285.69717773)
\curveto(230.41955722,285.6371747)(230.41455722,285.57717476)(230.42456482,285.51717773)
\curveto(230.4345572,285.46717487)(230.42955721,285.41717492)(230.40956482,285.36717773)
\curveto(230.36955727,285.20717513)(230.3345573,285.05717528)(230.30456482,284.91717773)
\curveto(230.27455736,284.77717556)(230.22955741,284.6421757)(230.16956482,284.51217773)
\curveto(230.00955763,284.1421762)(229.78955785,283.80717653)(229.50956482,283.50717773)
\curveto(229.22955841,283.20717713)(228.90955873,282.97717736)(228.54956482,282.81717773)
\curveto(228.37955926,282.7371776)(228.17955946,282.66217768)(227.94956482,282.59217773)
\curveto(227.8395598,282.55217779)(227.72455991,282.52717781)(227.60456482,282.51717773)
\curveto(227.48456015,282.50717783)(227.36456027,282.48717785)(227.24456482,282.45717773)
\curveto(227.19456044,282.4371779)(227.1395605,282.4371779)(227.07956482,282.45717773)
\curveto(227.01956062,282.46717787)(226.95956068,282.46217788)(226.89956482,282.44217773)
\curveto(226.79956084,282.42217792)(226.69956094,282.42217792)(226.59956482,282.44217773)
\lineto(226.46456482,282.44217773)
\curveto(226.41456122,282.46217788)(226.35456128,282.47217787)(226.28456482,282.47217773)
\curveto(226.22456141,282.46217788)(226.16956147,282.46717787)(226.11956482,282.48717773)
\curveto(226.07956156,282.49717784)(226.04456159,282.50217784)(226.01456482,282.50217773)
\curveto(225.98456165,282.50217784)(225.94956169,282.50717783)(225.90956482,282.51717773)
\lineto(225.63956482,282.57717773)
\curveto(225.54956209,282.59717774)(225.46456217,282.62717771)(225.38456482,282.66717773)
\curveto(225.04456259,282.80717753)(224.75456288,282.96217738)(224.51456482,283.13217773)
\curveto(224.27456336,283.31217703)(224.05456358,283.5421768)(223.85456482,283.82217773)
\curveto(223.70456393,284.05217629)(223.58956405,284.29217605)(223.50956482,284.54217773)
\curveto(223.48956415,284.59217575)(223.47956416,284.6371757)(223.47956482,284.67717773)
\curveto(223.47956416,284.72717561)(223.46956417,284.77717556)(223.44956482,284.82717773)
\curveto(223.42956421,284.88717545)(223.41456422,284.96717537)(223.40456482,285.06717773)
\curveto(223.40456423,285.16717517)(223.42456421,285.2421751)(223.46456482,285.29217773)
\curveto(223.51456412,285.37217497)(223.59456404,285.41717492)(223.70456482,285.42717773)
\curveto(223.81456382,285.4371749)(223.92956371,285.4421749)(224.04956482,285.44217773)
\lineto(224.21456482,285.44217773)
\curveto(224.27456336,285.4421749)(224.32956331,285.43217491)(224.37956482,285.41217773)
\curveto(224.46956317,285.39217495)(224.5395631,285.35217499)(224.58956482,285.29217773)
\curveto(224.65956298,285.20217514)(224.70456293,285.09217525)(224.72456482,284.96217773)
\curveto(224.75456288,284.8421755)(224.79956284,284.7371756)(224.85956482,284.64717773)
\curveto(225.04956259,284.30717603)(225.30956233,284.0371763)(225.63956482,283.83717773)
\curveto(225.7395619,283.77717656)(225.84456179,283.72717661)(225.95456482,283.68717773)
\curveto(226.07456156,283.65717668)(226.19456144,283.62217672)(226.31456482,283.58217773)
\curveto(226.48456115,283.53217681)(226.68956095,283.51217683)(226.92956482,283.52217773)
\curveto(227.17956046,283.5421768)(227.37956026,283.57717676)(227.52956482,283.62717773)
\curveto(227.89955974,283.74717659)(228.18955945,283.90717643)(228.39956482,284.10717773)
\curveto(228.61955902,284.31717602)(228.79955884,284.59717574)(228.93956482,284.94717773)
\curveto(228.98955865,285.04717529)(229.01955862,285.15217519)(229.02956482,285.26217773)
\curveto(229.04955859,285.37217497)(229.07455856,285.48717485)(229.10456482,285.60717773)
\lineto(229.10456482,285.71217773)
\curveto(229.11455852,285.75217459)(229.11955852,285.79217455)(229.11956482,285.83217773)
\curveto(229.12955851,285.86217448)(229.12955851,285.89717444)(229.11956482,285.93717773)
\lineto(229.11956482,286.05717773)
\curveto(229.11955852,286.31717402)(229.08955855,286.56217378)(229.02956482,286.79217773)
\curveto(228.91955872,287.1421732)(228.76455887,287.4371729)(228.56456482,287.67717773)
\curveto(228.36455927,287.92717241)(228.10455953,288.12217222)(227.78456482,288.26217773)
\lineto(227.60456482,288.32217773)
\curveto(227.55456008,288.342172)(227.49456014,288.36217198)(227.42456482,288.38217773)
\curveto(227.37456026,288.40217194)(227.31456032,288.41217193)(227.24456482,288.41217773)
\curveto(227.18456045,288.42217192)(227.11956052,288.4371719)(227.04956482,288.45717773)
\lineto(226.89956482,288.45717773)
\curveto(226.85956078,288.47717186)(226.80456083,288.48717185)(226.73456482,288.48717773)
\curveto(226.67456096,288.48717185)(226.61956102,288.47717186)(226.56956482,288.45717773)
\lineto(226.46456482,288.45717773)
\curveto(226.4345612,288.45717188)(226.39956124,288.45217189)(226.35956482,288.44217773)
\lineto(226.11956482,288.38217773)
\curveto(226.0395616,288.37217197)(225.95956168,288.35217199)(225.87956482,288.32217773)
\curveto(225.639562,288.22217212)(225.40956223,288.08717225)(225.18956482,287.91717773)
\curveto(225.09956254,287.84717249)(225.01456262,287.77217257)(224.93456482,287.69217773)
\curveto(224.85456278,287.62217272)(224.75456288,287.56717277)(224.63456482,287.52717773)
\curveto(224.54456309,287.49717284)(224.40456323,287.48717285)(224.21456482,287.49717773)
\curveto(224.0345636,287.50717283)(223.91456372,287.53217281)(223.85456482,287.57217773)
\curveto(223.80456383,287.61217273)(223.76456387,287.67217267)(223.73456482,287.75217773)
\curveto(223.71456392,287.83217251)(223.71456392,287.91717242)(223.73456482,288.00717773)
\curveto(223.76456387,288.12717221)(223.78456385,288.24717209)(223.79456482,288.36717773)
\curveto(223.81456382,288.49717184)(223.8395638,288.62217172)(223.86956482,288.74217773)
\curveto(223.88956375,288.78217156)(223.89456374,288.81717152)(223.88456482,288.84717773)
\curveto(223.88456375,288.88717145)(223.89456374,288.93217141)(223.91456482,288.98217773)
\curveto(223.9345637,289.07217127)(223.94956369,289.16217118)(223.95956482,289.25217773)
\curveto(223.96956367,289.35217099)(223.98956365,289.44717089)(224.01956482,289.53717773)
\curveto(224.02956361,289.59717074)(224.0345636,289.65717068)(224.03456482,289.71717773)
\curveto(224.04456359,289.77717056)(224.05956358,289.8371705)(224.07956482,289.89717773)
\curveto(224.12956351,290.09717024)(224.16456347,290.30217004)(224.18456482,290.51217773)
\curveto(224.21456342,290.73216961)(224.25456338,290.9421694)(224.30456482,291.14217773)
\curveto(224.3345633,291.2421691)(224.35456328,291.342169)(224.36456482,291.44217773)
\curveto(224.37456326,291.5421688)(224.38956325,291.6421687)(224.40956482,291.74217773)
\curveto(224.41956322,291.77216857)(224.42456321,291.81216853)(224.42456482,291.86217773)
\curveto(224.45456318,291.97216837)(224.47456316,292.07716826)(224.48456482,292.17717773)
\curveto(224.50456313,292.28716805)(224.52956311,292.39716794)(224.55956482,292.50717773)
\curveto(224.57956306,292.58716775)(224.59456304,292.65716768)(224.60456482,292.71717773)
\curveto(224.61456302,292.78716755)(224.639563,292.84716749)(224.67956482,292.89717773)
\curveto(224.69956294,292.92716741)(224.72956291,292.94716739)(224.76956482,292.95717773)
\curveto(224.80956283,292.97716736)(224.85456278,292.99716734)(224.90456482,293.01717773)
\curveto(224.96456267,293.01716732)(225.00456263,293.02216732)(225.02456482,293.03217773)
}
}
{
\newrgbcolor{curcolor}{0 0 0}
\pscustom[linestyle=none,fillstyle=solid,fillcolor=curcolor]
{
\newpath
\moveto(232.81917419,284.25717773)
\lineto(233.11917419,284.25717773)
\curveto(233.22917213,284.26717607)(233.33417203,284.26717607)(233.43417419,284.25717773)
\curveto(233.54417182,284.25717608)(233.64417172,284.24717609)(233.73417419,284.22717773)
\curveto(233.82417154,284.21717612)(233.89417147,284.19217615)(233.94417419,284.15217773)
\curveto(233.9641714,284.13217621)(233.97917138,284.10217624)(233.98917419,284.06217773)
\curveto(234.00917135,284.02217632)(234.02917133,283.97717636)(234.04917419,283.92717773)
\lineto(234.04917419,283.85217773)
\curveto(234.0591713,283.80217654)(234.0591713,283.74717659)(234.04917419,283.68717773)
\lineto(234.04917419,283.53717773)
\lineto(234.04917419,283.05717773)
\curveto(234.04917131,282.88717745)(234.00917135,282.76717757)(233.92917419,282.69717773)
\curveto(233.8591715,282.64717769)(233.76917159,282.62217772)(233.65917419,282.62217773)
\lineto(233.32917419,282.62217773)
\lineto(232.87917419,282.62217773)
\curveto(232.72917263,282.62217772)(232.61417275,282.65217769)(232.53417419,282.71217773)
\curveto(232.49417287,282.7421776)(232.4641729,282.79217755)(232.44417419,282.86217773)
\curveto(232.42417294,282.9421774)(232.40917295,283.02717731)(232.39917419,283.11717773)
\lineto(232.39917419,283.40217773)
\curveto(232.40917295,283.50217684)(232.41417295,283.58717675)(232.41417419,283.65717773)
\lineto(232.41417419,283.85217773)
\curveto(232.41417295,283.91217643)(232.42417294,283.96717637)(232.44417419,284.01717773)
\curveto(232.48417288,284.12717621)(232.55417281,284.19717614)(232.65417419,284.22717773)
\curveto(232.68417268,284.22717611)(232.73917262,284.2371761)(232.81917419,284.25717773)
}
}
{
\newrgbcolor{curcolor}{0 0 0}
\pscustom[linestyle=none,fillstyle=solid,fillcolor=curcolor]
{
\newpath
\moveto(239.13933044,293.22717773)
\curveto(240.769325,293.25716708)(241.81932395,292.70216764)(242.28933044,291.56217773)
\curveto(242.38932338,291.33216901)(242.45432332,291.0421693)(242.48433044,290.69217773)
\curveto(242.52432325,290.35216999)(242.49932327,290.0421703)(242.40933044,289.76217773)
\curveto(242.31932345,289.50217084)(242.19932357,289.27717106)(242.04933044,289.08717773)
\curveto(242.02932374,289.04717129)(242.00432377,289.01217133)(241.97433044,288.98217773)
\curveto(241.94432383,288.96217138)(241.91932385,288.9371714)(241.89933044,288.90717773)
\lineto(241.80933044,288.78717773)
\curveto(241.77932399,288.75717158)(241.74432403,288.73217161)(241.70433044,288.71217773)
\curveto(241.65432412,288.66217168)(241.59932417,288.61717172)(241.53933044,288.57717773)
\curveto(241.48932428,288.5371718)(241.44432433,288.48717185)(241.40433044,288.42717773)
\curveto(241.36432441,288.38717195)(241.34932442,288.337172)(241.35933044,288.27717773)
\curveto(241.3693244,288.22717211)(241.39932437,288.18217216)(241.44933044,288.14217773)
\curveto(241.49932427,288.10217224)(241.55432422,288.06217228)(241.61433044,288.02217773)
\curveto(241.68432409,287.99217235)(241.74932402,287.96217238)(241.80933044,287.93217773)
\curveto(241.8693239,287.90217244)(241.91932385,287.87217247)(241.95933044,287.84217773)
\curveto(242.27932349,287.62217272)(242.53432324,287.31217303)(242.72433044,286.91217773)
\curveto(242.76432301,286.82217352)(242.79432298,286.72717361)(242.81433044,286.62717773)
\curveto(242.84432293,286.5371738)(242.8693229,286.44717389)(242.88933044,286.35717773)
\curveto(242.89932287,286.30717403)(242.90432287,286.25717408)(242.90433044,286.20717773)
\curveto(242.91432286,286.16717417)(242.92432285,286.12217422)(242.93433044,286.07217773)
\curveto(242.94432283,286.02217432)(242.94432283,285.97217437)(242.93433044,285.92217773)
\curveto(242.92432285,285.87217447)(242.92932284,285.82217452)(242.94933044,285.77217773)
\curveto(242.95932281,285.72217462)(242.96432281,285.66217468)(242.96433044,285.59217773)
\curveto(242.96432281,285.52217482)(242.95432282,285.46217488)(242.93433044,285.41217773)
\lineto(242.93433044,285.18717773)
\lineto(242.87433044,284.94717773)
\curveto(242.86432291,284.87717546)(242.84932292,284.80717553)(242.82933044,284.73717773)
\curveto(242.79932297,284.64717569)(242.769323,284.56217578)(242.73933044,284.48217773)
\curveto(242.71932305,284.40217594)(242.68932308,284.32217602)(242.64933044,284.24217773)
\curveto(242.62932314,284.18217616)(242.59932317,284.12217622)(242.55933044,284.06217773)
\curveto(242.52932324,284.01217633)(242.49432328,283.96217638)(242.45433044,283.91217773)
\curveto(242.25432352,283.60217674)(242.00432377,283.342177)(241.70433044,283.13217773)
\curveto(241.40432437,282.93217741)(241.05932471,282.76717757)(240.66933044,282.63717773)
\curveto(240.54932522,282.59717774)(240.41932535,282.57217777)(240.27933044,282.56217773)
\curveto(240.14932562,282.5421778)(240.01432576,282.51717782)(239.87433044,282.48717773)
\curveto(239.80432597,282.47717786)(239.73432604,282.47217787)(239.66433044,282.47217773)
\curveto(239.60432617,282.47217787)(239.53932623,282.46717787)(239.46933044,282.45717773)
\curveto(239.42932634,282.44717789)(239.3693264,282.4421779)(239.28933044,282.44217773)
\curveto(239.21932655,282.4421779)(239.1693266,282.44717789)(239.13933044,282.45717773)
\curveto(239.08932668,282.46717787)(239.04432673,282.47217787)(239.00433044,282.47217773)
\lineto(238.88433044,282.47217773)
\curveto(238.78432699,282.49217785)(238.68432709,282.50717783)(238.58433044,282.51717773)
\curveto(238.48432729,282.52717781)(238.38932738,282.5421778)(238.29933044,282.56217773)
\curveto(238.18932758,282.59217775)(238.07932769,282.61717772)(237.96933044,282.63717773)
\curveto(237.8693279,282.66717767)(237.76432801,282.70717763)(237.65433044,282.75717773)
\curveto(237.28432849,282.91717742)(236.9693288,283.11717722)(236.70933044,283.35717773)
\curveto(236.44932932,283.60717673)(236.23932953,283.91717642)(236.07933044,284.28717773)
\curveto(236.03932973,284.37717596)(236.00432977,284.47217587)(235.97433044,284.57217773)
\curveto(235.94432983,284.67217567)(235.91432986,284.77717556)(235.88433044,284.88717773)
\curveto(235.86432991,284.9371754)(235.85432992,284.98717535)(235.85433044,285.03717773)
\curveto(235.85432992,285.09717524)(235.84432993,285.15717518)(235.82433044,285.21717773)
\curveto(235.80432997,285.27717506)(235.79432998,285.35717498)(235.79433044,285.45717773)
\curveto(235.79432998,285.55717478)(235.80932996,285.63217471)(235.83933044,285.68217773)
\curveto(235.84932992,285.71217463)(235.86432991,285.7371746)(235.88433044,285.75717773)
\lineto(235.94433044,285.81717773)
\curveto(235.98432979,285.8371745)(236.04432973,285.85217449)(236.12433044,285.86217773)
\curveto(236.21432956,285.87217447)(236.30432947,285.87717446)(236.39433044,285.87717773)
\curveto(236.48432929,285.87717446)(236.5693292,285.87217447)(236.64933044,285.86217773)
\curveto(236.73932903,285.85217449)(236.80432897,285.8421745)(236.84433044,285.83217773)
\curveto(236.86432891,285.81217453)(236.88432889,285.79717454)(236.90433044,285.78717773)
\curveto(236.92432885,285.78717455)(236.94432883,285.77717456)(236.96433044,285.75717773)
\curveto(237.03432874,285.66717467)(237.0743287,285.55217479)(237.08433044,285.41217773)
\curveto(237.10432867,285.27217507)(237.13432864,285.14717519)(237.17433044,285.03717773)
\lineto(237.32433044,284.67717773)
\curveto(237.3743284,284.56717577)(237.43932833,284.46217588)(237.51933044,284.36217773)
\curveto(237.53932823,284.33217601)(237.55932821,284.30717603)(237.57933044,284.28717773)
\curveto(237.60932816,284.26717607)(237.63432814,284.2421761)(237.65433044,284.21217773)
\curveto(237.69432808,284.15217619)(237.72932804,284.10717623)(237.75933044,284.07717773)
\curveto(237.79932797,284.04717629)(237.83432794,284.01717632)(237.86433044,283.98717773)
\curveto(237.90432787,283.95717638)(237.94932782,283.92717641)(237.99933044,283.89717773)
\curveto(238.08932768,283.8371765)(238.18432759,283.78717655)(238.28433044,283.74717773)
\lineto(238.61433044,283.62717773)
\curveto(238.76432701,283.57717676)(238.96432681,283.54717679)(239.21433044,283.53717773)
\curveto(239.46432631,283.52717681)(239.6743261,283.54717679)(239.84433044,283.59717773)
\curveto(239.92432585,283.61717672)(239.99432578,283.63217671)(240.05433044,283.64217773)
\lineto(240.26433044,283.70217773)
\curveto(240.54432523,283.82217652)(240.78432499,283.97217637)(240.98433044,284.15217773)
\curveto(241.19432458,284.33217601)(241.35932441,284.56217578)(241.47933044,284.84217773)
\curveto(241.50932426,284.91217543)(241.52932424,284.98217536)(241.53933044,285.05217773)
\lineto(241.59933044,285.29217773)
\curveto(241.63932413,285.43217491)(241.64932412,285.59217475)(241.62933044,285.77217773)
\curveto(241.60932416,285.96217438)(241.57932419,286.11217423)(241.53933044,286.22217773)
\curveto(241.40932436,286.60217374)(241.22432455,286.89217345)(240.98433044,287.09217773)
\curveto(240.75432502,287.29217305)(240.44432533,287.45217289)(240.05433044,287.57217773)
\curveto(239.94432583,287.60217274)(239.82432595,287.62217272)(239.69433044,287.63217773)
\curveto(239.5743262,287.6421727)(239.44932632,287.64717269)(239.31933044,287.64717773)
\curveto(239.15932661,287.64717269)(239.01932675,287.65217269)(238.89933044,287.66217773)
\curveto(238.77932699,287.67217267)(238.69432708,287.73217261)(238.64433044,287.84217773)
\curveto(238.62432715,287.87217247)(238.61432716,287.90717243)(238.61433044,287.94717773)
\lineto(238.61433044,288.08217773)
\curveto(238.60432717,288.18217216)(238.60432717,288.27717206)(238.61433044,288.36717773)
\curveto(238.63432714,288.45717188)(238.6743271,288.52217182)(238.73433044,288.56217773)
\curveto(238.774327,288.59217175)(238.81432696,288.61217173)(238.85433044,288.62217773)
\curveto(238.90432687,288.63217171)(238.95932681,288.6421717)(239.01933044,288.65217773)
\curveto(239.03932673,288.66217168)(239.06432671,288.66217168)(239.09433044,288.65217773)
\curveto(239.12432665,288.65217169)(239.14932662,288.65717168)(239.16933044,288.66717773)
\lineto(239.30433044,288.66717773)
\curveto(239.41432636,288.68717165)(239.51432626,288.69717164)(239.60433044,288.69717773)
\curveto(239.70432607,288.70717163)(239.79932597,288.72717161)(239.88933044,288.75717773)
\curveto(240.20932556,288.86717147)(240.46432531,289.01217133)(240.65433044,289.19217773)
\curveto(240.84432493,289.37217097)(240.99432478,289.62217072)(241.10433044,289.94217773)
\curveto(241.13432464,290.0421703)(241.15432462,290.16717017)(241.16433044,290.31717773)
\curveto(241.18432459,290.47716986)(241.17932459,290.62216972)(241.14933044,290.75217773)
\curveto(241.12932464,290.82216952)(241.10932466,290.88716945)(241.08933044,290.94717773)
\curveto(241.07932469,291.01716932)(241.05932471,291.08216926)(241.02933044,291.14217773)
\curveto(240.92932484,291.38216896)(240.78432499,291.57216877)(240.59433044,291.71217773)
\curveto(240.40432537,291.85216849)(240.17932559,291.96216838)(239.91933044,292.04217773)
\curveto(239.85932591,292.06216828)(239.79932597,292.07216827)(239.73933044,292.07217773)
\curveto(239.67932609,292.07216827)(239.61432616,292.08216826)(239.54433044,292.10217773)
\curveto(239.46432631,292.12216822)(239.3693264,292.13216821)(239.25933044,292.13217773)
\curveto(239.14932662,292.13216821)(239.05432672,292.12216822)(238.97433044,292.10217773)
\curveto(238.92432685,292.08216826)(238.8743269,292.07216827)(238.82433044,292.07217773)
\curveto(238.78432699,292.07216827)(238.73932703,292.06216828)(238.68933044,292.04217773)
\curveto(238.50932726,291.99216835)(238.33932743,291.91716842)(238.17933044,291.81717773)
\curveto(238.02932774,291.72716861)(237.89932787,291.61216873)(237.78933044,291.47217773)
\curveto(237.69932807,291.35216899)(237.61932815,291.22216912)(237.54933044,291.08217773)
\curveto(237.47932829,290.9421694)(237.41432836,290.78716955)(237.35433044,290.61717773)
\curveto(237.32432845,290.50716983)(237.30432847,290.38716995)(237.29433044,290.25717773)
\curveto(237.28432849,290.1371702)(237.24932852,290.0371703)(237.18933044,289.95717773)
\curveto(237.1693286,289.91717042)(237.10932866,289.87717046)(237.00933044,289.83717773)
\curveto(236.9693288,289.82717051)(236.90932886,289.81717052)(236.82933044,289.80717773)
\lineto(236.57433044,289.80717773)
\curveto(236.48432929,289.81717052)(236.39932937,289.82717051)(236.31933044,289.83717773)
\curveto(236.24932952,289.84717049)(236.19932957,289.86217048)(236.16933044,289.88217773)
\curveto(236.12932964,289.91217043)(236.09432968,289.96717037)(236.06433044,290.04717773)
\curveto(236.03432974,290.12717021)(236.02932974,290.21217013)(236.04933044,290.30217773)
\curveto(236.05932971,290.35216999)(236.06432971,290.40216994)(236.06433044,290.45217773)
\lineto(236.09433044,290.63217773)
\curveto(236.12432965,290.73216961)(236.14932962,290.83216951)(236.16933044,290.93217773)
\curveto(236.19932957,291.03216931)(236.23432954,291.12216922)(236.27433044,291.20217773)
\curveto(236.32432945,291.31216903)(236.3693294,291.41716892)(236.40933044,291.51717773)
\curveto(236.44932932,291.62716871)(236.49932927,291.73216861)(236.55933044,291.83217773)
\curveto(236.88932888,292.37216797)(237.35932841,292.76716757)(237.96933044,293.01717773)
\curveto(238.08932768,293.06716727)(238.21432756,293.10216724)(238.34433044,293.12217773)
\curveto(238.48432729,293.1421672)(238.62432715,293.16716717)(238.76433044,293.19717773)
\curveto(238.82432695,293.20716713)(238.88432689,293.21216713)(238.94433044,293.21217773)
\curveto(239.01432676,293.21216713)(239.07932669,293.21716712)(239.13933044,293.22717773)
}
}
{
\newrgbcolor{curcolor}{0 0 0}
\pscustom[linestyle=none,fillstyle=solid,fillcolor=curcolor]
{
\newpath
\moveto(254.17893982,291.14217773)
\curveto(253.97892952,290.85216949)(253.76892973,290.56716977)(253.54893982,290.28717773)
\curveto(253.33893016,290.00717033)(253.13393036,289.72217062)(252.93393982,289.43217773)
\curveto(252.33393116,288.58217176)(251.72893177,287.7421726)(251.11893982,286.91217773)
\curveto(250.50893299,286.09217425)(249.90393359,285.25717508)(249.30393982,284.40717773)
\lineto(248.79393982,283.68717773)
\lineto(248.28393982,282.99717773)
\curveto(248.20393529,282.88717745)(248.12393537,282.77217757)(248.04393982,282.65217773)
\curveto(247.96393553,282.53217781)(247.86893563,282.4371779)(247.75893982,282.36717773)
\curveto(247.71893578,282.34717799)(247.65393584,282.33217801)(247.56393982,282.32217773)
\curveto(247.48393601,282.30217804)(247.3939361,282.29217805)(247.29393982,282.29217773)
\curveto(247.1939363,282.29217805)(247.0989364,282.29717804)(247.00893982,282.30717773)
\curveto(246.92893657,282.31717802)(246.86893663,282.337178)(246.82893982,282.36717773)
\curveto(246.7989367,282.38717795)(246.77393672,282.42217792)(246.75393982,282.47217773)
\curveto(246.74393675,282.51217783)(246.74893675,282.55717778)(246.76893982,282.60717773)
\curveto(246.80893669,282.68717765)(246.85393664,282.76217758)(246.90393982,282.83217773)
\curveto(246.96393653,282.91217743)(247.01893648,282.99217735)(247.06893982,283.07217773)
\curveto(247.30893619,283.41217693)(247.55393594,283.74717659)(247.80393982,284.07717773)
\curveto(248.05393544,284.40717593)(248.2939352,284.7421756)(248.52393982,285.08217773)
\curveto(248.68393481,285.30217504)(248.84393465,285.51717482)(249.00393982,285.72717773)
\curveto(249.16393433,285.9371744)(249.32393417,286.15217419)(249.48393982,286.37217773)
\curveto(249.84393365,286.89217345)(250.20893329,287.40217294)(250.57893982,287.90217773)
\curveto(250.94893255,288.40217194)(251.31893218,288.91217143)(251.68893982,289.43217773)
\curveto(251.82893167,289.63217071)(251.96893153,289.82717051)(252.10893982,290.01717773)
\curveto(252.25893124,290.20717013)(252.40393109,290.40216994)(252.54393982,290.60217773)
\curveto(252.75393074,290.90216944)(252.96893053,291.20216914)(253.18893982,291.50217773)
\lineto(253.84893982,292.40217773)
\lineto(254.02893982,292.67217773)
\lineto(254.23893982,292.94217773)
\lineto(254.35893982,293.12217773)
\curveto(254.40892909,293.18216716)(254.45892904,293.2371671)(254.50893982,293.28717773)
\curveto(254.57892892,293.337167)(254.65392884,293.37216697)(254.73393982,293.39217773)
\curveto(254.75392874,293.40216694)(254.77892872,293.40216694)(254.80893982,293.39217773)
\curveto(254.84892865,293.39216695)(254.87892862,293.40216694)(254.89893982,293.42217773)
\curveto(255.01892848,293.42216692)(255.15392834,293.41716692)(255.30393982,293.40717773)
\curveto(255.45392804,293.40716693)(255.54392795,293.36216698)(255.57393982,293.27217773)
\curveto(255.5939279,293.2421671)(255.5989279,293.20716713)(255.58893982,293.16717773)
\curveto(255.57892792,293.12716721)(255.56392793,293.09716724)(255.54393982,293.07717773)
\curveto(255.50392799,292.99716734)(255.46392803,292.92716741)(255.42393982,292.86717773)
\curveto(255.38392811,292.80716753)(255.33892816,292.74716759)(255.28893982,292.68717773)
\lineto(254.71893982,291.90717773)
\curveto(254.53892896,291.65716868)(254.35892914,291.40216894)(254.17893982,291.14217773)
\moveto(247.32393982,287.24217773)
\curveto(247.27393622,287.26217308)(247.22393627,287.26717307)(247.17393982,287.25717773)
\curveto(247.12393637,287.24717309)(247.07393642,287.25217309)(247.02393982,287.27217773)
\curveto(246.91393658,287.29217305)(246.80893669,287.31217303)(246.70893982,287.33217773)
\curveto(246.61893688,287.36217298)(246.52393697,287.40217294)(246.42393982,287.45217773)
\curveto(246.0939374,287.59217275)(245.83893766,287.78717255)(245.65893982,288.03717773)
\curveto(245.47893802,288.29717204)(245.33393816,288.60717173)(245.22393982,288.96717773)
\curveto(245.1939383,289.04717129)(245.17393832,289.12717121)(245.16393982,289.20717773)
\curveto(245.15393834,289.29717104)(245.13893836,289.38217096)(245.11893982,289.46217773)
\curveto(245.10893839,289.51217083)(245.10393839,289.57717076)(245.10393982,289.65717773)
\curveto(245.0939384,289.68717065)(245.08893841,289.71717062)(245.08893982,289.74717773)
\curveto(245.08893841,289.78717055)(245.08393841,289.82217052)(245.07393982,289.85217773)
\lineto(245.07393982,290.00217773)
\curveto(245.06393843,290.05217029)(245.05893844,290.11217023)(245.05893982,290.18217773)
\curveto(245.05893844,290.26217008)(245.06393843,290.32717001)(245.07393982,290.37717773)
\lineto(245.07393982,290.54217773)
\curveto(245.0939384,290.59216975)(245.0989384,290.6371697)(245.08893982,290.67717773)
\curveto(245.08893841,290.72716961)(245.0939384,290.77216957)(245.10393982,290.81217773)
\curveto(245.11393838,290.85216949)(245.11893838,290.88716945)(245.11893982,290.91717773)
\curveto(245.11893838,290.95716938)(245.12393837,290.99716934)(245.13393982,291.03717773)
\curveto(245.16393833,291.14716919)(245.18393831,291.25716908)(245.19393982,291.36717773)
\curveto(245.21393828,291.48716885)(245.24893825,291.60216874)(245.29893982,291.71217773)
\curveto(245.43893806,292.05216829)(245.5989379,292.32716801)(245.77893982,292.53717773)
\curveto(245.96893753,292.75716758)(246.23893726,292.9371674)(246.58893982,293.07717773)
\curveto(246.66893683,293.10716723)(246.75393674,293.12716721)(246.84393982,293.13717773)
\curveto(246.93393656,293.15716718)(247.02893647,293.17716716)(247.12893982,293.19717773)
\curveto(247.15893634,293.20716713)(247.21393628,293.20716713)(247.29393982,293.19717773)
\curveto(247.37393612,293.19716714)(247.42393607,293.20716713)(247.44393982,293.22717773)
\curveto(248.00393549,293.2371671)(248.45393504,293.12716721)(248.79393982,292.89717773)
\curveto(249.14393435,292.66716767)(249.40393409,292.36216798)(249.57393982,291.98217773)
\curveto(249.61393388,291.89216845)(249.64893385,291.79716854)(249.67893982,291.69717773)
\curveto(249.70893379,291.59716874)(249.73393376,291.49716884)(249.75393982,291.39717773)
\curveto(249.77393372,291.36716897)(249.77893372,291.337169)(249.76893982,291.30717773)
\curveto(249.76893373,291.27716906)(249.77393372,291.24716909)(249.78393982,291.21717773)
\curveto(249.81393368,291.10716923)(249.83393366,290.98216936)(249.84393982,290.84217773)
\curveto(249.85393364,290.71216963)(249.86393363,290.57716976)(249.87393982,290.43717773)
\lineto(249.87393982,290.27217773)
\curveto(249.88393361,290.21217013)(249.88393361,290.15717018)(249.87393982,290.10717773)
\curveto(249.86393363,290.05717028)(249.85893364,290.00717033)(249.85893982,289.95717773)
\lineto(249.85893982,289.82217773)
\curveto(249.84893365,289.78217056)(249.84393365,289.7421706)(249.84393982,289.70217773)
\curveto(249.85393364,289.66217068)(249.84893365,289.61717072)(249.82893982,289.56717773)
\curveto(249.80893369,289.45717088)(249.78893371,289.35217099)(249.76893982,289.25217773)
\curveto(249.75893374,289.15217119)(249.73893376,289.05217129)(249.70893982,288.95217773)
\curveto(249.57893392,288.59217175)(249.41393408,288.27717206)(249.21393982,288.00717773)
\curveto(249.01393448,287.7371726)(248.73893476,287.53217281)(248.38893982,287.39217773)
\curveto(248.30893519,287.36217298)(248.22393527,287.337173)(248.13393982,287.31717773)
\lineto(247.86393982,287.25717773)
\curveto(247.81393568,287.24717309)(247.76893573,287.2421731)(247.72893982,287.24217773)
\curveto(247.68893581,287.25217309)(247.64893585,287.25217309)(247.60893982,287.24217773)
\curveto(247.50893599,287.22217312)(247.41393608,287.22217312)(247.32393982,287.24217773)
\moveto(246.48393982,288.63717773)
\curveto(246.52393697,288.56717177)(246.56393693,288.50217184)(246.60393982,288.44217773)
\curveto(246.64393685,288.39217195)(246.6939368,288.342172)(246.75393982,288.29217773)
\lineto(246.90393982,288.17217773)
\curveto(246.96393653,288.1421722)(247.02893647,288.11717222)(247.09893982,288.09717773)
\curveto(247.13893636,288.07717226)(247.17393632,288.06717227)(247.20393982,288.06717773)
\curveto(247.24393625,288.07717226)(247.28393621,288.07217227)(247.32393982,288.05217773)
\curveto(247.35393614,288.05217229)(247.3939361,288.04717229)(247.44393982,288.03717773)
\curveto(247.493936,288.0371723)(247.53393596,288.0421723)(247.56393982,288.05217773)
\lineto(247.78893982,288.09717773)
\curveto(248.03893546,288.17717216)(248.22393527,288.30217204)(248.34393982,288.47217773)
\curveto(248.42393507,288.57217177)(248.493935,288.70217164)(248.55393982,288.86217773)
\curveto(248.63393486,289.0421713)(248.6939348,289.26717107)(248.73393982,289.53717773)
\curveto(248.77393472,289.81717052)(248.78893471,290.09717024)(248.77893982,290.37717773)
\curveto(248.76893473,290.66716967)(248.73893476,290.9421694)(248.68893982,291.20217773)
\curveto(248.63893486,291.46216888)(248.56393493,291.67216867)(248.46393982,291.83217773)
\curveto(248.34393515,292.03216831)(248.1939353,292.18216816)(248.01393982,292.28217773)
\curveto(247.93393556,292.33216801)(247.84393565,292.36216798)(247.74393982,292.37217773)
\curveto(247.64393585,292.39216795)(247.53893596,292.40216794)(247.42893982,292.40217773)
\curveto(247.40893609,292.39216795)(247.38393611,292.38716795)(247.35393982,292.38717773)
\curveto(247.33393616,292.39716794)(247.31393618,292.39716794)(247.29393982,292.38717773)
\curveto(247.24393625,292.37716796)(247.1989363,292.36716797)(247.15893982,292.35717773)
\curveto(247.11893638,292.35716798)(247.07893642,292.34716799)(247.03893982,292.32717773)
\curveto(246.85893664,292.24716809)(246.70893679,292.12716821)(246.58893982,291.96717773)
\curveto(246.47893702,291.80716853)(246.38893711,291.62716871)(246.31893982,291.42717773)
\curveto(246.25893724,291.2371691)(246.21393728,291.01216933)(246.18393982,290.75217773)
\curveto(246.16393733,290.49216985)(246.15893734,290.22717011)(246.16893982,289.95717773)
\curveto(246.17893732,289.69717064)(246.20893729,289.44717089)(246.25893982,289.20717773)
\curveto(246.31893718,288.97717136)(246.3939371,288.78717155)(246.48393982,288.63717773)
\moveto(257.28393982,285.65217773)
\curveto(257.2939262,285.60217474)(257.2989262,285.51217483)(257.29893982,285.38217773)
\curveto(257.2989262,285.25217509)(257.28892621,285.16217518)(257.26893982,285.11217773)
\curveto(257.24892625,285.06217528)(257.24392625,285.00717533)(257.25393982,284.94717773)
\curveto(257.26392623,284.89717544)(257.26392623,284.84717549)(257.25393982,284.79717773)
\curveto(257.21392628,284.65717568)(257.18392631,284.52217582)(257.16393982,284.39217773)
\curveto(257.15392634,284.26217608)(257.12392637,284.1421762)(257.07393982,284.03217773)
\curveto(256.93392656,283.68217666)(256.76892673,283.38717695)(256.57893982,283.14717773)
\curveto(256.38892711,282.91717742)(256.11892738,282.73217761)(255.76893982,282.59217773)
\curveto(255.68892781,282.56217778)(255.60392789,282.5421778)(255.51393982,282.53217773)
\curveto(255.42392807,282.51217783)(255.33892816,282.49217785)(255.25893982,282.47217773)
\curveto(255.20892829,282.46217788)(255.15892834,282.45717788)(255.10893982,282.45717773)
\curveto(255.05892844,282.45717788)(255.00892849,282.45217789)(254.95893982,282.44217773)
\curveto(254.92892857,282.43217791)(254.87892862,282.43217791)(254.80893982,282.44217773)
\curveto(254.73892876,282.4421779)(254.68892881,282.44717789)(254.65893982,282.45717773)
\curveto(254.5989289,282.47717786)(254.53892896,282.48717785)(254.47893982,282.48717773)
\curveto(254.42892907,282.47717786)(254.37892912,282.48217786)(254.32893982,282.50217773)
\curveto(254.23892926,282.52217782)(254.14892935,282.54717779)(254.05893982,282.57717773)
\curveto(253.97892952,282.59717774)(253.8989296,282.62717771)(253.81893982,282.66717773)
\curveto(253.49893,282.80717753)(253.24893025,283.00217734)(253.06893982,283.25217773)
\curveto(252.88893061,283.51217683)(252.73893076,283.81717652)(252.61893982,284.16717773)
\curveto(252.5989309,284.24717609)(252.58393091,284.33217601)(252.57393982,284.42217773)
\curveto(252.56393093,284.51217583)(252.54893095,284.59717574)(252.52893982,284.67717773)
\curveto(252.51893098,284.70717563)(252.51393098,284.7371756)(252.51393982,284.76717773)
\lineto(252.51393982,284.87217773)
\curveto(252.493931,284.95217539)(252.48393101,285.03217531)(252.48393982,285.11217773)
\lineto(252.48393982,285.24717773)
\curveto(252.46393103,285.34717499)(252.46393103,285.44717489)(252.48393982,285.54717773)
\lineto(252.48393982,285.72717773)
\curveto(252.493931,285.77717456)(252.498931,285.82217452)(252.49893982,285.86217773)
\curveto(252.498931,285.91217443)(252.50393099,285.95717438)(252.51393982,285.99717773)
\curveto(252.52393097,286.0371743)(252.52893097,286.07217427)(252.52893982,286.10217773)
\curveto(252.52893097,286.1421742)(252.53393096,286.18217416)(252.54393982,286.22217773)
\lineto(252.60393982,286.55217773)
\curveto(252.62393087,286.67217367)(252.65393084,286.78217356)(252.69393982,286.88217773)
\curveto(252.83393066,287.21217313)(252.9939305,287.48717285)(253.17393982,287.70717773)
\curveto(253.36393013,287.9371724)(253.62392987,288.12217222)(253.95393982,288.26217773)
\curveto(254.03392946,288.30217204)(254.11892938,288.32717201)(254.20893982,288.33717773)
\lineto(254.50893982,288.39717773)
\lineto(254.64393982,288.39717773)
\curveto(254.6939288,288.40717193)(254.74392875,288.41217193)(254.79393982,288.41217773)
\curveto(255.36392813,288.43217191)(255.82392767,288.32717201)(256.17393982,288.09717773)
\curveto(256.53392696,287.87717246)(256.7989267,287.57717276)(256.96893982,287.19717773)
\curveto(257.01892648,287.09717324)(257.05892644,286.99717334)(257.08893982,286.89717773)
\curveto(257.11892638,286.79717354)(257.14892635,286.69217365)(257.17893982,286.58217773)
\curveto(257.18892631,286.5421738)(257.1939263,286.50717383)(257.19393982,286.47717773)
\curveto(257.1939263,286.45717388)(257.1989263,286.42717391)(257.20893982,286.38717773)
\curveto(257.22892627,286.31717402)(257.23892626,286.2421741)(257.23893982,286.16217773)
\curveto(257.23892626,286.08217426)(257.24892625,286.00217434)(257.26893982,285.92217773)
\curveto(257.26892623,285.87217447)(257.26892623,285.82717451)(257.26893982,285.78717773)
\curveto(257.26892623,285.74717459)(257.27392622,285.70217464)(257.28393982,285.65217773)
\moveto(256.17393982,285.21717773)
\curveto(256.18392731,285.26717507)(256.18892731,285.342175)(256.18893982,285.44217773)
\curveto(256.1989273,285.5421748)(256.1939273,285.61717472)(256.17393982,285.66717773)
\curveto(256.15392734,285.72717461)(256.14892735,285.78217456)(256.15893982,285.83217773)
\curveto(256.17892732,285.89217445)(256.17892732,285.95217439)(256.15893982,286.01217773)
\curveto(256.14892735,286.0421743)(256.14392735,286.07717426)(256.14393982,286.11717773)
\curveto(256.14392735,286.15717418)(256.13892736,286.19717414)(256.12893982,286.23717773)
\curveto(256.10892739,286.31717402)(256.08892741,286.39217395)(256.06893982,286.46217773)
\curveto(256.05892744,286.5421738)(256.04392745,286.62217372)(256.02393982,286.70217773)
\curveto(255.9939275,286.76217358)(255.96892753,286.82217352)(255.94893982,286.88217773)
\curveto(255.92892757,286.9421734)(255.8989276,287.00217334)(255.85893982,287.06217773)
\curveto(255.75892774,287.23217311)(255.62892787,287.36717297)(255.46893982,287.46717773)
\curveto(255.38892811,287.51717282)(255.2939282,287.55217279)(255.18393982,287.57217773)
\curveto(255.07392842,287.59217275)(254.94892855,287.60217274)(254.80893982,287.60217773)
\curveto(254.78892871,287.59217275)(254.76392873,287.58717275)(254.73393982,287.58717773)
\curveto(254.70392879,287.59717274)(254.67392882,287.59717274)(254.64393982,287.58717773)
\lineto(254.49393982,287.52717773)
\curveto(254.44392905,287.51717282)(254.3989291,287.50217284)(254.35893982,287.48217773)
\curveto(254.16892933,287.37217297)(254.02392947,287.22717311)(253.92393982,287.04717773)
\curveto(253.83392966,286.86717347)(253.75392974,286.66217368)(253.68393982,286.43217773)
\curveto(253.64392985,286.30217404)(253.62392987,286.16717417)(253.62393982,286.02717773)
\curveto(253.62392987,285.89717444)(253.61392988,285.75217459)(253.59393982,285.59217773)
\curveto(253.58392991,285.5421748)(253.57392992,285.48217486)(253.56393982,285.41217773)
\curveto(253.56392993,285.342175)(253.57392992,285.28217506)(253.59393982,285.23217773)
\lineto(253.59393982,285.06717773)
\lineto(253.59393982,284.88717773)
\curveto(253.60392989,284.8371755)(253.61392988,284.78217556)(253.62393982,284.72217773)
\curveto(253.63392986,284.67217567)(253.63892986,284.61717572)(253.63893982,284.55717773)
\curveto(253.64892985,284.49717584)(253.66392983,284.4421759)(253.68393982,284.39217773)
\curveto(253.73392976,284.20217614)(253.7939297,284.02717631)(253.86393982,283.86717773)
\curveto(253.93392956,283.70717663)(254.03892946,283.57717676)(254.17893982,283.47717773)
\curveto(254.30892919,283.37717696)(254.44892905,283.30717703)(254.59893982,283.26717773)
\curveto(254.62892887,283.25717708)(254.65392884,283.25217709)(254.67393982,283.25217773)
\curveto(254.70392879,283.26217708)(254.73392876,283.26217708)(254.76393982,283.25217773)
\curveto(254.78392871,283.25217709)(254.81392868,283.24717709)(254.85393982,283.23717773)
\curveto(254.8939286,283.2371771)(254.92892857,283.2421771)(254.95893982,283.25217773)
\curveto(254.9989285,283.26217708)(255.03892846,283.26717707)(255.07893982,283.26717773)
\curveto(255.11892838,283.26717707)(255.15892834,283.27717706)(255.19893982,283.29717773)
\curveto(255.43892806,283.37717696)(255.63392786,283.51217683)(255.78393982,283.70217773)
\curveto(255.90392759,283.88217646)(255.9939275,284.08717625)(256.05393982,284.31717773)
\curveto(256.07392742,284.38717595)(256.08892741,284.45717588)(256.09893982,284.52717773)
\curveto(256.10892739,284.60717573)(256.12392737,284.68717565)(256.14393982,284.76717773)
\curveto(256.14392735,284.82717551)(256.14892735,284.87217547)(256.15893982,284.90217773)
\curveto(256.15892734,284.92217542)(256.15892734,284.94717539)(256.15893982,284.97717773)
\curveto(256.15892734,285.01717532)(256.16392733,285.04717529)(256.17393982,285.06717773)
\lineto(256.17393982,285.21717773)
}
}
{
\newrgbcolor{curcolor}{0 0 0}
\pscustom[linestyle=none,fillstyle=solid,fillcolor=curcolor]
{
\newpath
\moveto(182.28994568,292.82288818)
\lineto(185.88994568,292.82288818)
\lineto(186.53494568,292.82288818)
\curveto(186.61493915,292.82287776)(186.68993907,292.81787776)(186.75994568,292.80788818)
\curveto(186.82993893,292.80787777)(186.88993887,292.79787778)(186.93994568,292.77788818)
\curveto(187.00993875,292.74787783)(187.0649387,292.68787789)(187.10494568,292.59788818)
\curveto(187.12493864,292.56787801)(187.13493863,292.52787805)(187.13494568,292.47788818)
\lineto(187.13494568,292.34288818)
\curveto(187.14493862,292.23287835)(187.13993862,292.12787845)(187.11994568,292.02788818)
\curveto(187.10993865,291.92787865)(187.07493869,291.85787872)(187.01494568,291.81788818)
\curveto(186.92493884,291.74787883)(186.78993897,291.71287887)(186.60994568,291.71288818)
\curveto(186.42993933,291.72287886)(186.2649395,291.72787885)(186.11494568,291.72788818)
\lineto(184.11994568,291.72788818)
\lineto(183.62494568,291.72788818)
\lineto(183.48994568,291.72788818)
\curveto(183.44994231,291.72787885)(183.40994235,291.72287886)(183.36994568,291.71288818)
\lineto(183.15994568,291.71288818)
\curveto(183.04994271,291.6828789)(182.96994279,291.64287894)(182.91994568,291.59288818)
\curveto(182.86994289,291.55287903)(182.83494293,291.49787908)(182.81494568,291.42788818)
\curveto(182.79494297,291.36787921)(182.77994298,291.29787928)(182.76994568,291.21788818)
\curveto(182.759943,291.13787944)(182.73994302,291.04787953)(182.70994568,290.94788818)
\curveto(182.6599431,290.74787983)(182.61994314,290.54288004)(182.58994568,290.33288818)
\curveto(182.5599432,290.12288046)(182.51994324,289.91788066)(182.46994568,289.71788818)
\curveto(182.44994331,289.64788093)(182.43994332,289.577881)(182.43994568,289.50788818)
\curveto(182.43994332,289.44788113)(182.42994333,289.3828812)(182.40994568,289.31288818)
\curveto(182.39994336,289.2828813)(182.38994337,289.24288134)(182.37994568,289.19288818)
\curveto(182.37994338,289.15288143)(182.38494338,289.11288147)(182.39494568,289.07288818)
\curveto(182.41494335,289.02288156)(182.43994332,288.9778816)(182.46994568,288.93788818)
\curveto(182.50994325,288.90788167)(182.56994319,288.90288168)(182.64994568,288.92288818)
\curveto(182.70994305,288.94288164)(182.76994299,288.96788161)(182.82994568,288.99788818)
\curveto(182.88994287,289.03788154)(182.94994281,289.07288151)(183.00994568,289.10288818)
\curveto(183.06994269,289.12288146)(183.11994264,289.13788144)(183.15994568,289.14788818)
\curveto(183.34994241,289.22788135)(183.55494221,289.2828813)(183.77494568,289.31288818)
\curveto(184.00494176,289.34288124)(184.23494153,289.35288123)(184.46494568,289.34288818)
\curveto(184.70494106,289.34288124)(184.93494083,289.31788126)(185.15494568,289.26788818)
\curveto(185.37494039,289.22788135)(185.57494019,289.16788141)(185.75494568,289.08788818)
\curveto(185.80493996,289.06788151)(185.84993991,289.04788153)(185.88994568,289.02788818)
\curveto(185.93993982,289.00788157)(185.98993977,288.9828816)(186.03994568,288.95288818)
\curveto(186.38993937,288.74288184)(186.66993909,288.51288207)(186.87994568,288.26288818)
\curveto(187.09993866,288.01288257)(187.29493847,287.68788289)(187.46494568,287.28788818)
\curveto(187.51493825,287.1778834)(187.54993821,287.06788351)(187.56994568,286.95788818)
\curveto(187.58993817,286.84788373)(187.61493815,286.73288385)(187.64494568,286.61288818)
\curveto(187.65493811,286.582884)(187.6599381,286.53788404)(187.65994568,286.47788818)
\curveto(187.67993808,286.41788416)(187.68993807,286.34788423)(187.68994568,286.26788818)
\curveto(187.68993807,286.19788438)(187.69993806,286.13288445)(187.71994568,286.07288818)
\lineto(187.71994568,285.90788818)
\curveto(187.72993803,285.85788472)(187.73493803,285.78788479)(187.73494568,285.69788818)
\curveto(187.73493803,285.60788497)(187.72493804,285.53788504)(187.70494568,285.48788818)
\curveto(187.68493808,285.42788515)(187.67993808,285.36788521)(187.68994568,285.30788818)
\curveto(187.69993806,285.25788532)(187.69493807,285.20788537)(187.67494568,285.15788818)
\curveto(187.63493813,284.99788558)(187.59993816,284.84788573)(187.56994568,284.70788818)
\curveto(187.53993822,284.56788601)(187.49493827,284.43288615)(187.43494568,284.30288818)
\curveto(187.27493849,283.93288665)(187.05493871,283.59788698)(186.77494568,283.29788818)
\curveto(186.49493927,282.99788758)(186.17493959,282.76788781)(185.81494568,282.60788818)
\curveto(185.64494012,282.52788805)(185.44494032,282.45288813)(185.21494568,282.38288818)
\curveto(185.10494066,282.34288824)(184.98994077,282.31788826)(184.86994568,282.30788818)
\curveto(184.74994101,282.29788828)(184.62994113,282.2778883)(184.50994568,282.24788818)
\curveto(184.4599413,282.22788835)(184.40494136,282.22788835)(184.34494568,282.24788818)
\curveto(184.28494148,282.25788832)(184.22494154,282.25288833)(184.16494568,282.23288818)
\curveto(184.0649417,282.21288837)(183.9649418,282.21288837)(183.86494568,282.23288818)
\lineto(183.72994568,282.23288818)
\curveto(183.67994208,282.25288833)(183.61994214,282.26288832)(183.54994568,282.26288818)
\curveto(183.48994227,282.25288833)(183.43494233,282.25788832)(183.38494568,282.27788818)
\curveto(183.34494242,282.28788829)(183.30994245,282.29288829)(183.27994568,282.29288818)
\curveto(183.24994251,282.29288829)(183.21494255,282.29788828)(183.17494568,282.30788818)
\lineto(182.90494568,282.36788818)
\curveto(182.81494295,282.38788819)(182.72994303,282.41788816)(182.64994568,282.45788818)
\curveto(182.30994345,282.59788798)(182.01994374,282.75288783)(181.77994568,282.92288818)
\curveto(181.53994422,283.10288748)(181.31994444,283.33288725)(181.11994568,283.61288818)
\curveto(180.96994479,283.84288674)(180.85494491,284.0828865)(180.77494568,284.33288818)
\curveto(180.75494501,284.3828862)(180.74494502,284.42788615)(180.74494568,284.46788818)
\curveto(180.74494502,284.51788606)(180.73494503,284.56788601)(180.71494568,284.61788818)
\curveto(180.69494507,284.6778859)(180.67994508,284.75788582)(180.66994568,284.85788818)
\curveto(180.66994509,284.95788562)(180.68994507,285.03288555)(180.72994568,285.08288818)
\curveto(180.77994498,285.16288542)(180.8599449,285.20788537)(180.96994568,285.21788818)
\curveto(181.07994468,285.22788535)(181.19494457,285.23288535)(181.31494568,285.23288818)
\lineto(181.47994568,285.23288818)
\curveto(181.53994422,285.23288535)(181.59494417,285.22288536)(181.64494568,285.20288818)
\curveto(181.73494403,285.1828854)(181.80494396,285.14288544)(181.85494568,285.08288818)
\curveto(181.92494384,284.99288559)(181.96994379,284.8828857)(181.98994568,284.75288818)
\curveto(182.01994374,284.63288595)(182.0649437,284.52788605)(182.12494568,284.43788818)
\curveto(182.31494345,284.09788648)(182.57494319,283.82788675)(182.90494568,283.62788818)
\curveto(183.00494276,283.56788701)(183.10994265,283.51788706)(183.21994568,283.47788818)
\curveto(183.33994242,283.44788713)(183.4599423,283.41288717)(183.57994568,283.37288818)
\curveto(183.74994201,283.32288726)(183.95494181,283.30288728)(184.19494568,283.31288818)
\curveto(184.44494132,283.33288725)(184.64494112,283.36788721)(184.79494568,283.41788818)
\curveto(185.1649406,283.53788704)(185.45494031,283.69788688)(185.66494568,283.89788818)
\curveto(185.88493988,284.10788647)(186.0649397,284.38788619)(186.20494568,284.73788818)
\curveto(186.25493951,284.83788574)(186.28493948,284.94288564)(186.29494568,285.05288818)
\curveto(186.31493945,285.16288542)(186.33993942,285.2778853)(186.36994568,285.39788818)
\lineto(186.36994568,285.50288818)
\curveto(186.37993938,285.54288504)(186.38493938,285.582885)(186.38494568,285.62288818)
\curveto(186.39493937,285.65288493)(186.39493937,285.68788489)(186.38494568,285.72788818)
\lineto(186.38494568,285.84788818)
\curveto(186.38493938,286.10788447)(186.35493941,286.35288423)(186.29494568,286.58288818)
\curveto(186.18493958,286.93288365)(186.02993973,287.22788335)(185.82994568,287.46788818)
\curveto(185.62994013,287.71788286)(185.36994039,287.91288267)(185.04994568,288.05288818)
\lineto(184.86994568,288.11288818)
\curveto(184.81994094,288.13288245)(184.759941,288.15288243)(184.68994568,288.17288818)
\curveto(184.63994112,288.19288239)(184.57994118,288.20288238)(184.50994568,288.20288818)
\curveto(184.44994131,288.21288237)(184.38494138,288.22788235)(184.31494568,288.24788818)
\lineto(184.16494568,288.24788818)
\curveto(184.12494164,288.26788231)(184.06994169,288.2778823)(183.99994568,288.27788818)
\curveto(183.93994182,288.2778823)(183.88494188,288.26788231)(183.83494568,288.24788818)
\lineto(183.72994568,288.24788818)
\curveto(183.69994206,288.24788233)(183.6649421,288.24288234)(183.62494568,288.23288818)
\lineto(183.38494568,288.17288818)
\curveto(183.30494246,288.16288242)(183.22494254,288.14288244)(183.14494568,288.11288818)
\curveto(182.90494286,288.01288257)(182.67494309,287.8778827)(182.45494568,287.70788818)
\curveto(182.3649434,287.63788294)(182.27994348,287.56288302)(182.19994568,287.48288818)
\curveto(182.11994364,287.41288317)(182.01994374,287.35788322)(181.89994568,287.31788818)
\curveto(181.80994395,287.28788329)(181.66994409,287.2778833)(181.47994568,287.28788818)
\curveto(181.29994446,287.29788328)(181.17994458,287.32288326)(181.11994568,287.36288818)
\curveto(181.06994469,287.40288318)(181.02994473,287.46288312)(180.99994568,287.54288818)
\curveto(180.97994478,287.62288296)(180.97994478,287.70788287)(180.99994568,287.79788818)
\curveto(181.02994473,287.91788266)(181.04994471,288.03788254)(181.05994568,288.15788818)
\curveto(181.07994468,288.28788229)(181.10494466,288.41288217)(181.13494568,288.53288818)
\curveto(181.15494461,288.57288201)(181.1599446,288.60788197)(181.14994568,288.63788818)
\curveto(181.14994461,288.6778819)(181.1599446,288.72288186)(181.17994568,288.77288818)
\curveto(181.19994456,288.86288172)(181.21494455,288.95288163)(181.22494568,289.04288818)
\curveto(181.23494453,289.14288144)(181.25494451,289.23788134)(181.28494568,289.32788818)
\curveto(181.29494447,289.38788119)(181.29994446,289.44788113)(181.29994568,289.50788818)
\curveto(181.30994445,289.56788101)(181.32494444,289.62788095)(181.34494568,289.68788818)
\curveto(181.39494437,289.88788069)(181.42994433,290.09288049)(181.44994568,290.30288818)
\curveto(181.47994428,290.52288006)(181.51994424,290.73287985)(181.56994568,290.93288818)
\curveto(181.59994416,291.03287955)(181.61994414,291.13287945)(181.62994568,291.23288818)
\curveto(181.63994412,291.33287925)(181.65494411,291.43287915)(181.67494568,291.53288818)
\curveto(181.68494408,291.56287902)(181.68994407,291.60287898)(181.68994568,291.65288818)
\curveto(181.71994404,291.76287882)(181.73994402,291.86787871)(181.74994568,291.96788818)
\curveto(181.76994399,292.0778785)(181.79494397,292.18787839)(181.82494568,292.29788818)
\curveto(181.84494392,292.3778782)(181.8599439,292.44787813)(181.86994568,292.50788818)
\curveto(181.87994388,292.577878)(181.90494386,292.63787794)(181.94494568,292.68788818)
\curveto(181.9649438,292.71787786)(181.99494377,292.73787784)(182.03494568,292.74788818)
\curveto(182.07494369,292.76787781)(182.11994364,292.78787779)(182.16994568,292.80788818)
\curveto(182.22994353,292.80787777)(182.26994349,292.81287777)(182.28994568,292.82288818)
}
}
{
\newrgbcolor{curcolor}{0 0 0}
\pscustom[linestyle=none,fillstyle=solid,fillcolor=curcolor]
{
\newpath
\moveto(190.08455505,284.04788818)
\lineto(190.38455505,284.04788818)
\curveto(190.49455299,284.05788652)(190.59955289,284.05788652)(190.69955505,284.04788818)
\curveto(190.80955268,284.04788653)(190.90955258,284.03788654)(190.99955505,284.01788818)
\curveto(191.0895524,284.00788657)(191.15955233,283.9828866)(191.20955505,283.94288818)
\curveto(191.22955226,283.92288666)(191.24455224,283.89288669)(191.25455505,283.85288818)
\curveto(191.27455221,283.81288677)(191.29455219,283.76788681)(191.31455505,283.71788818)
\lineto(191.31455505,283.64288818)
\curveto(191.32455216,283.59288699)(191.32455216,283.53788704)(191.31455505,283.47788818)
\lineto(191.31455505,283.32788818)
\lineto(191.31455505,282.84788818)
\curveto(191.31455217,282.6778879)(191.27455221,282.55788802)(191.19455505,282.48788818)
\curveto(191.12455236,282.43788814)(191.03455245,282.41288817)(190.92455505,282.41288818)
\lineto(190.59455505,282.41288818)
\lineto(190.14455505,282.41288818)
\curveto(189.99455349,282.41288817)(189.87955361,282.44288814)(189.79955505,282.50288818)
\curveto(189.75955373,282.53288805)(189.72955376,282.582888)(189.70955505,282.65288818)
\curveto(189.6895538,282.73288785)(189.67455381,282.81788776)(189.66455505,282.90788818)
\lineto(189.66455505,283.19288818)
\curveto(189.67455381,283.29288729)(189.67955381,283.3778872)(189.67955505,283.44788818)
\lineto(189.67955505,283.64288818)
\curveto(189.67955381,283.70288688)(189.6895538,283.75788682)(189.70955505,283.80788818)
\curveto(189.74955374,283.91788666)(189.81955367,283.98788659)(189.91955505,284.01788818)
\curveto(189.94955354,284.01788656)(190.00455348,284.02788655)(190.08455505,284.04788818)
}
}
{
\newrgbcolor{curcolor}{0 0 0}
\pscustom[linestyle=none,fillstyle=solid,fillcolor=curcolor]
{
\newpath
\moveto(196.4047113,293.01788818)
\curveto(198.03470586,293.04787753)(199.08470481,292.49287809)(199.5547113,291.35288818)
\curveto(199.65470424,291.12287946)(199.71970418,290.83287975)(199.7497113,290.48288818)
\curveto(199.78970411,290.14288044)(199.76470413,289.83288075)(199.6747113,289.55288818)
\curveto(199.58470431,289.29288129)(199.46470443,289.06788151)(199.3147113,288.87788818)
\curveto(199.2947046,288.83788174)(199.26970463,288.80288178)(199.2397113,288.77288818)
\curveto(199.20970469,288.75288183)(199.18470471,288.72788185)(199.1647113,288.69788818)
\lineto(199.0747113,288.57788818)
\curveto(199.04470485,288.54788203)(199.00970489,288.52288206)(198.9697113,288.50288818)
\curveto(198.91970498,288.45288213)(198.86470503,288.40788217)(198.8047113,288.36788818)
\curveto(198.75470514,288.32788225)(198.70970519,288.2778823)(198.6697113,288.21788818)
\curveto(198.62970527,288.1778824)(198.61470528,288.12788245)(198.6247113,288.06788818)
\curveto(198.63470526,288.01788256)(198.66470523,287.97288261)(198.7147113,287.93288818)
\curveto(198.76470513,287.89288269)(198.81970508,287.85288273)(198.8797113,287.81288818)
\curveto(198.94970495,287.7828828)(199.01470488,287.75288283)(199.0747113,287.72288818)
\curveto(199.13470476,287.69288289)(199.18470471,287.66288292)(199.2247113,287.63288818)
\curveto(199.54470435,287.41288317)(199.7997041,287.10288348)(199.9897113,286.70288818)
\curveto(200.02970387,286.61288397)(200.05970384,286.51788406)(200.0797113,286.41788818)
\curveto(200.10970379,286.32788425)(200.13470376,286.23788434)(200.1547113,286.14788818)
\curveto(200.16470373,286.09788448)(200.16970373,286.04788453)(200.1697113,285.99788818)
\curveto(200.17970372,285.95788462)(200.18970371,285.91288467)(200.1997113,285.86288818)
\curveto(200.20970369,285.81288477)(200.20970369,285.76288482)(200.1997113,285.71288818)
\curveto(200.18970371,285.66288492)(200.1947037,285.61288497)(200.2147113,285.56288818)
\curveto(200.22470367,285.51288507)(200.22970367,285.45288513)(200.2297113,285.38288818)
\curveto(200.22970367,285.31288527)(200.21970368,285.25288533)(200.1997113,285.20288818)
\lineto(200.1997113,284.97788818)
\lineto(200.1397113,284.73788818)
\curveto(200.12970377,284.66788591)(200.11470378,284.59788598)(200.0947113,284.52788818)
\curveto(200.06470383,284.43788614)(200.03470386,284.35288623)(200.0047113,284.27288818)
\curveto(199.98470391,284.19288639)(199.95470394,284.11288647)(199.9147113,284.03288818)
\curveto(199.894704,283.97288661)(199.86470403,283.91288667)(199.8247113,283.85288818)
\curveto(199.7947041,283.80288678)(199.75970414,283.75288683)(199.7197113,283.70288818)
\curveto(199.51970438,283.39288719)(199.26970463,283.13288745)(198.9697113,282.92288818)
\curveto(198.66970523,282.72288786)(198.32470557,282.55788802)(197.9347113,282.42788818)
\curveto(197.81470608,282.38788819)(197.68470621,282.36288822)(197.5447113,282.35288818)
\curveto(197.41470648,282.33288825)(197.27970662,282.30788827)(197.1397113,282.27788818)
\curveto(197.06970683,282.26788831)(196.9997069,282.26288832)(196.9297113,282.26288818)
\curveto(196.86970703,282.26288832)(196.80470709,282.25788832)(196.7347113,282.24788818)
\curveto(196.6947072,282.23788834)(196.63470726,282.23288835)(196.5547113,282.23288818)
\curveto(196.48470741,282.23288835)(196.43470746,282.23788834)(196.4047113,282.24788818)
\curveto(196.35470754,282.25788832)(196.30970759,282.26288832)(196.2697113,282.26288818)
\lineto(196.1497113,282.26288818)
\curveto(196.04970785,282.2828883)(195.94970795,282.29788828)(195.8497113,282.30788818)
\curveto(195.74970815,282.31788826)(195.65470824,282.33288825)(195.5647113,282.35288818)
\curveto(195.45470844,282.3828882)(195.34470855,282.40788817)(195.2347113,282.42788818)
\curveto(195.13470876,282.45788812)(195.02970887,282.49788808)(194.9197113,282.54788818)
\curveto(194.54970935,282.70788787)(194.23470966,282.90788767)(193.9747113,283.14788818)
\curveto(193.71471018,283.39788718)(193.50471039,283.70788687)(193.3447113,284.07788818)
\curveto(193.30471059,284.16788641)(193.26971063,284.26288632)(193.2397113,284.36288818)
\curveto(193.20971069,284.46288612)(193.17971072,284.56788601)(193.1497113,284.67788818)
\curveto(193.12971077,284.72788585)(193.11971078,284.7778858)(193.1197113,284.82788818)
\curveto(193.11971078,284.88788569)(193.10971079,284.94788563)(193.0897113,285.00788818)
\curveto(193.06971083,285.06788551)(193.05971084,285.14788543)(193.0597113,285.24788818)
\curveto(193.05971084,285.34788523)(193.07471082,285.42288516)(193.1047113,285.47288818)
\curveto(193.11471078,285.50288508)(193.12971077,285.52788505)(193.1497113,285.54788818)
\lineto(193.2097113,285.60788818)
\curveto(193.24971065,285.62788495)(193.30971059,285.64288494)(193.3897113,285.65288818)
\curveto(193.47971042,285.66288492)(193.56971033,285.66788491)(193.6597113,285.66788818)
\curveto(193.74971015,285.66788491)(193.83471006,285.66288492)(193.9147113,285.65288818)
\curveto(194.00470989,285.64288494)(194.06970983,285.63288495)(194.1097113,285.62288818)
\curveto(194.12970977,285.60288498)(194.14970975,285.58788499)(194.1697113,285.57788818)
\curveto(194.18970971,285.577885)(194.20970969,285.56788501)(194.2297113,285.54788818)
\curveto(194.2997096,285.45788512)(194.33970956,285.34288524)(194.3497113,285.20288818)
\curveto(194.36970953,285.06288552)(194.3997095,284.93788564)(194.4397113,284.82788818)
\lineto(194.5897113,284.46788818)
\curveto(194.63970926,284.35788622)(194.70470919,284.25288633)(194.7847113,284.15288818)
\curveto(194.80470909,284.12288646)(194.82470907,284.09788648)(194.8447113,284.07788818)
\curveto(194.87470902,284.05788652)(194.899709,284.03288655)(194.9197113,284.00288818)
\curveto(194.95970894,283.94288664)(194.9947089,283.89788668)(195.0247113,283.86788818)
\curveto(195.06470883,283.83788674)(195.0997088,283.80788677)(195.1297113,283.77788818)
\curveto(195.16970873,283.74788683)(195.21470868,283.71788686)(195.2647113,283.68788818)
\curveto(195.35470854,283.62788695)(195.44970845,283.577887)(195.5497113,283.53788818)
\lineto(195.8797113,283.41788818)
\curveto(196.02970787,283.36788721)(196.22970767,283.33788724)(196.4797113,283.32788818)
\curveto(196.72970717,283.31788726)(196.93970696,283.33788724)(197.1097113,283.38788818)
\curveto(197.18970671,283.40788717)(197.25970664,283.42288716)(197.3197113,283.43288818)
\lineto(197.5297113,283.49288818)
\curveto(197.80970609,283.61288697)(198.04970585,283.76288682)(198.2497113,283.94288818)
\curveto(198.45970544,284.12288646)(198.62470527,284.35288623)(198.7447113,284.63288818)
\curveto(198.77470512,284.70288588)(198.7947051,284.77288581)(198.8047113,284.84288818)
\lineto(198.8647113,285.08288818)
\curveto(198.90470499,285.22288536)(198.91470498,285.3828852)(198.8947113,285.56288818)
\curveto(198.87470502,285.75288483)(198.84470505,285.90288468)(198.8047113,286.01288818)
\curveto(198.67470522,286.39288419)(198.48970541,286.6828839)(198.2497113,286.88288818)
\curveto(198.01970588,287.0828835)(197.70970619,287.24288334)(197.3197113,287.36288818)
\curveto(197.20970669,287.39288319)(197.08970681,287.41288317)(196.9597113,287.42288818)
\curveto(196.83970706,287.43288315)(196.71470718,287.43788314)(196.5847113,287.43788818)
\curveto(196.42470747,287.43788314)(196.28470761,287.44288314)(196.1647113,287.45288818)
\curveto(196.04470785,287.46288312)(195.95970794,287.52288306)(195.9097113,287.63288818)
\curveto(195.88970801,287.66288292)(195.87970802,287.69788288)(195.8797113,287.73788818)
\lineto(195.8797113,287.87288818)
\curveto(195.86970803,287.97288261)(195.86970803,288.06788251)(195.8797113,288.15788818)
\curveto(195.899708,288.24788233)(195.93970796,288.31288227)(195.9997113,288.35288818)
\curveto(196.03970786,288.3828822)(196.07970782,288.40288218)(196.1197113,288.41288818)
\curveto(196.16970773,288.42288216)(196.22470767,288.43288215)(196.2847113,288.44288818)
\curveto(196.30470759,288.45288213)(196.32970757,288.45288213)(196.3597113,288.44288818)
\curveto(196.38970751,288.44288214)(196.41470748,288.44788213)(196.4347113,288.45788818)
\lineto(196.5697113,288.45788818)
\curveto(196.67970722,288.4778821)(196.77970712,288.48788209)(196.8697113,288.48788818)
\curveto(196.96970693,288.49788208)(197.06470683,288.51788206)(197.1547113,288.54788818)
\curveto(197.47470642,288.65788192)(197.72970617,288.80288178)(197.9197113,288.98288818)
\curveto(198.10970579,289.16288142)(198.25970564,289.41288117)(198.3697113,289.73288818)
\curveto(198.3997055,289.83288075)(198.41970548,289.95788062)(198.4297113,290.10788818)
\curveto(198.44970545,290.26788031)(198.44470545,290.41288017)(198.4147113,290.54288818)
\curveto(198.3947055,290.61287997)(198.37470552,290.6778799)(198.3547113,290.73788818)
\curveto(198.34470555,290.80787977)(198.32470557,290.87287971)(198.2947113,290.93288818)
\curveto(198.1947057,291.17287941)(198.04970585,291.36287922)(197.8597113,291.50288818)
\curveto(197.66970623,291.64287894)(197.44470645,291.75287883)(197.1847113,291.83288818)
\curveto(197.12470677,291.85287873)(197.06470683,291.86287872)(197.0047113,291.86288818)
\curveto(196.94470695,291.86287872)(196.87970702,291.87287871)(196.8097113,291.89288818)
\curveto(196.72970717,291.91287867)(196.63470726,291.92287866)(196.5247113,291.92288818)
\curveto(196.41470748,291.92287866)(196.31970758,291.91287867)(196.2397113,291.89288818)
\curveto(196.18970771,291.87287871)(196.13970776,291.86287872)(196.0897113,291.86288818)
\curveto(196.04970785,291.86287872)(196.00470789,291.85287873)(195.9547113,291.83288818)
\curveto(195.77470812,291.7828788)(195.60470829,291.70787887)(195.4447113,291.60788818)
\curveto(195.2947086,291.51787906)(195.16470873,291.40287918)(195.0547113,291.26288818)
\curveto(194.96470893,291.14287944)(194.88470901,291.01287957)(194.8147113,290.87288818)
\curveto(194.74470915,290.73287985)(194.67970922,290.57788)(194.6197113,290.40788818)
\curveto(194.58970931,290.29788028)(194.56970933,290.1778804)(194.5597113,290.04788818)
\curveto(194.54970935,289.92788065)(194.51470938,289.82788075)(194.4547113,289.74788818)
\curveto(194.43470946,289.70788087)(194.37470952,289.66788091)(194.2747113,289.62788818)
\curveto(194.23470966,289.61788096)(194.17470972,289.60788097)(194.0947113,289.59788818)
\lineto(193.8397113,289.59788818)
\curveto(193.74971015,289.60788097)(193.66471023,289.61788096)(193.5847113,289.62788818)
\curveto(193.51471038,289.63788094)(193.46471043,289.65288093)(193.4347113,289.67288818)
\curveto(193.3947105,289.70288088)(193.35971054,289.75788082)(193.3297113,289.83788818)
\curveto(193.2997106,289.91788066)(193.2947106,290.00288058)(193.3147113,290.09288818)
\curveto(193.32471057,290.14288044)(193.32971057,290.19288039)(193.3297113,290.24288818)
\lineto(193.3597113,290.42288818)
\curveto(193.38971051,290.52288006)(193.41471048,290.62287996)(193.4347113,290.72288818)
\curveto(193.46471043,290.82287976)(193.4997104,290.91287967)(193.5397113,290.99288818)
\curveto(193.58971031,291.10287948)(193.63471026,291.20787937)(193.6747113,291.30788818)
\curveto(193.71471018,291.41787916)(193.76471013,291.52287906)(193.8247113,291.62288818)
\curveto(194.15470974,292.16287842)(194.62470927,292.55787802)(195.2347113,292.80788818)
\curveto(195.35470854,292.85787772)(195.47970842,292.89287769)(195.6097113,292.91288818)
\curveto(195.74970815,292.93287765)(195.88970801,292.95787762)(196.0297113,292.98788818)
\curveto(196.08970781,292.99787758)(196.14970775,293.00287758)(196.2097113,293.00288818)
\curveto(196.27970762,293.00287758)(196.34470755,293.00787757)(196.4047113,293.01788818)
}
}
{
\newrgbcolor{curcolor}{0 0 0}
\pscustom[linestyle=none,fillstyle=solid,fillcolor=curcolor]
{
\newpath
\moveto(211.44432068,290.93288818)
\curveto(211.24431038,290.64287994)(211.03431059,290.35788022)(210.81432068,290.07788818)
\curveto(210.60431102,289.79788078)(210.39931122,289.51288107)(210.19932068,289.22288818)
\curveto(209.59931202,288.37288221)(208.99431263,287.53288305)(208.38432068,286.70288818)
\curveto(207.77431385,285.8828847)(207.16931445,285.04788553)(206.56932068,284.19788818)
\lineto(206.05932068,283.47788818)
\lineto(205.54932068,282.78788818)
\curveto(205.46931615,282.6778879)(205.38931623,282.56288802)(205.30932068,282.44288818)
\curveto(205.22931639,282.32288826)(205.13431649,282.22788835)(205.02432068,282.15788818)
\curveto(204.98431664,282.13788844)(204.9193167,282.12288846)(204.82932068,282.11288818)
\curveto(204.74931687,282.09288849)(204.65931696,282.0828885)(204.55932068,282.08288818)
\curveto(204.45931716,282.0828885)(204.36431726,282.08788849)(204.27432068,282.09788818)
\curveto(204.19431743,282.10788847)(204.13431749,282.12788845)(204.09432068,282.15788818)
\curveto(204.06431756,282.1778884)(204.03931758,282.21288837)(204.01932068,282.26288818)
\curveto(204.00931761,282.30288828)(204.01431761,282.34788823)(204.03432068,282.39788818)
\curveto(204.07431755,282.4778881)(204.1193175,282.55288803)(204.16932068,282.62288818)
\curveto(204.22931739,282.70288788)(204.28431734,282.7828878)(204.33432068,282.86288818)
\curveto(204.57431705,283.20288738)(204.8193168,283.53788704)(205.06932068,283.86788818)
\curveto(205.3193163,284.19788638)(205.55931606,284.53288605)(205.78932068,284.87288818)
\curveto(205.94931567,285.09288549)(206.10931551,285.30788527)(206.26932068,285.51788818)
\curveto(206.42931519,285.72788485)(206.58931503,285.94288464)(206.74932068,286.16288818)
\curveto(207.10931451,286.6828839)(207.47431415,287.19288339)(207.84432068,287.69288818)
\curveto(208.21431341,288.19288239)(208.58431304,288.70288188)(208.95432068,289.22288818)
\curveto(209.09431253,289.42288116)(209.23431239,289.61788096)(209.37432068,289.80788818)
\curveto(209.5243121,289.99788058)(209.66931195,290.19288039)(209.80932068,290.39288818)
\curveto(210.0193116,290.69287989)(210.23431139,290.99287959)(210.45432068,291.29288818)
\lineto(211.11432068,292.19288818)
\lineto(211.29432068,292.46288818)
\lineto(211.50432068,292.73288818)
\lineto(211.62432068,292.91288818)
\curveto(211.67430995,292.97287761)(211.7243099,293.02787755)(211.77432068,293.07788818)
\curveto(211.84430978,293.12787745)(211.9193097,293.16287742)(211.99932068,293.18288818)
\curveto(212.0193096,293.19287739)(212.04430958,293.19287739)(212.07432068,293.18288818)
\curveto(212.11430951,293.1828774)(212.14430948,293.19287739)(212.16432068,293.21288818)
\curveto(212.28430934,293.21287737)(212.4193092,293.20787737)(212.56932068,293.19788818)
\curveto(212.7193089,293.19787738)(212.80930881,293.15287743)(212.83932068,293.06288818)
\curveto(212.85930876,293.03287755)(212.86430876,292.99787758)(212.85432068,292.95788818)
\curveto(212.84430878,292.91787766)(212.82930879,292.88787769)(212.80932068,292.86788818)
\curveto(212.76930885,292.78787779)(212.72930889,292.71787786)(212.68932068,292.65788818)
\curveto(212.64930897,292.59787798)(212.60430902,292.53787804)(212.55432068,292.47788818)
\lineto(211.98432068,291.69788818)
\curveto(211.80430982,291.44787913)(211.62431,291.19287939)(211.44432068,290.93288818)
\moveto(204.58932068,287.03288818)
\curveto(204.53931708,287.05288353)(204.48931713,287.05788352)(204.43932068,287.04788818)
\curveto(204.38931723,287.03788354)(204.33931728,287.04288354)(204.28932068,287.06288818)
\curveto(204.17931744,287.0828835)(204.07431755,287.10288348)(203.97432068,287.12288818)
\curveto(203.88431774,287.15288343)(203.78931783,287.19288339)(203.68932068,287.24288818)
\curveto(203.35931826,287.3828832)(203.10431852,287.577883)(202.92432068,287.82788818)
\curveto(202.74431888,288.08788249)(202.59931902,288.39788218)(202.48932068,288.75788818)
\curveto(202.45931916,288.83788174)(202.43931918,288.91788166)(202.42932068,288.99788818)
\curveto(202.4193192,289.08788149)(202.40431922,289.17288141)(202.38432068,289.25288818)
\curveto(202.37431925,289.30288128)(202.36931925,289.36788121)(202.36932068,289.44788818)
\curveto(202.35931926,289.4778811)(202.35431927,289.50788107)(202.35432068,289.53788818)
\curveto(202.35431927,289.577881)(202.34931927,289.61288097)(202.33932068,289.64288818)
\lineto(202.33932068,289.79288818)
\curveto(202.32931929,289.84288074)(202.3243193,289.90288068)(202.32432068,289.97288818)
\curveto(202.3243193,290.05288053)(202.32931929,290.11788046)(202.33932068,290.16788818)
\lineto(202.33932068,290.33288818)
\curveto(202.35931926,290.3828802)(202.36431926,290.42788015)(202.35432068,290.46788818)
\curveto(202.35431927,290.51788006)(202.35931926,290.56288002)(202.36932068,290.60288818)
\curveto(202.37931924,290.64287994)(202.38431924,290.6778799)(202.38432068,290.70788818)
\curveto(202.38431924,290.74787983)(202.38931923,290.78787979)(202.39932068,290.82788818)
\curveto(202.42931919,290.93787964)(202.44931917,291.04787953)(202.45932068,291.15788818)
\curveto(202.47931914,291.2778793)(202.51431911,291.39287919)(202.56432068,291.50288818)
\curveto(202.70431892,291.84287874)(202.86431876,292.11787846)(203.04432068,292.32788818)
\curveto(203.23431839,292.54787803)(203.50431812,292.72787785)(203.85432068,292.86788818)
\curveto(203.93431769,292.89787768)(204.0193176,292.91787766)(204.10932068,292.92788818)
\curveto(204.19931742,292.94787763)(204.29431733,292.96787761)(204.39432068,292.98788818)
\curveto(204.4243172,292.99787758)(204.47931714,292.99787758)(204.55932068,292.98788818)
\curveto(204.63931698,292.98787759)(204.68931693,292.99787758)(204.70932068,293.01788818)
\curveto(205.26931635,293.02787755)(205.7193159,292.91787766)(206.05932068,292.68788818)
\curveto(206.40931521,292.45787812)(206.66931495,292.15287843)(206.83932068,291.77288818)
\curveto(206.87931474,291.6828789)(206.91431471,291.58787899)(206.94432068,291.48788818)
\curveto(206.97431465,291.38787919)(206.99931462,291.28787929)(207.01932068,291.18788818)
\curveto(207.03931458,291.15787942)(207.04431458,291.12787945)(207.03432068,291.09788818)
\curveto(207.03431459,291.06787951)(207.03931458,291.03787954)(207.04932068,291.00788818)
\curveto(207.07931454,290.89787968)(207.09931452,290.77287981)(207.10932068,290.63288818)
\curveto(207.1193145,290.50288008)(207.12931449,290.36788021)(207.13932068,290.22788818)
\lineto(207.13932068,290.06288818)
\curveto(207.14931447,290.00288058)(207.14931447,289.94788063)(207.13932068,289.89788818)
\curveto(207.12931449,289.84788073)(207.1243145,289.79788078)(207.12432068,289.74788818)
\lineto(207.12432068,289.61288818)
\curveto(207.11431451,289.57288101)(207.10931451,289.53288105)(207.10932068,289.49288818)
\curveto(207.1193145,289.45288113)(207.11431451,289.40788117)(207.09432068,289.35788818)
\curveto(207.07431455,289.24788133)(207.05431457,289.14288144)(207.03432068,289.04288818)
\curveto(207.0243146,288.94288164)(207.00431462,288.84288174)(206.97432068,288.74288818)
\curveto(206.84431478,288.3828822)(206.67931494,288.06788251)(206.47932068,287.79788818)
\curveto(206.27931534,287.52788305)(206.00431562,287.32288326)(205.65432068,287.18288818)
\curveto(205.57431605,287.15288343)(205.48931613,287.12788345)(205.39932068,287.10788818)
\lineto(205.12932068,287.04788818)
\curveto(205.07931654,287.03788354)(205.03431659,287.03288355)(204.99432068,287.03288818)
\curveto(204.95431667,287.04288354)(204.91431671,287.04288354)(204.87432068,287.03288818)
\curveto(204.77431685,287.01288357)(204.67931694,287.01288357)(204.58932068,287.03288818)
\moveto(203.74932068,288.42788818)
\curveto(203.78931783,288.35788222)(203.82931779,288.29288229)(203.86932068,288.23288818)
\curveto(203.90931771,288.1828824)(203.95931766,288.13288245)(204.01932068,288.08288818)
\lineto(204.16932068,287.96288818)
\curveto(204.22931739,287.93288265)(204.29431733,287.90788267)(204.36432068,287.88788818)
\curveto(204.40431722,287.86788271)(204.43931718,287.85788272)(204.46932068,287.85788818)
\curveto(204.50931711,287.86788271)(204.54931707,287.86288272)(204.58932068,287.84288818)
\curveto(204.619317,287.84288274)(204.65931696,287.83788274)(204.70932068,287.82788818)
\curveto(204.75931686,287.82788275)(204.79931682,287.83288275)(204.82932068,287.84288818)
\lineto(205.05432068,287.88788818)
\curveto(205.30431632,287.96788261)(205.48931613,288.09288249)(205.60932068,288.26288818)
\curveto(205.68931593,288.36288222)(205.75931586,288.49288209)(205.81932068,288.65288818)
\curveto(205.89931572,288.83288175)(205.95931566,289.05788152)(205.99932068,289.32788818)
\curveto(206.03931558,289.60788097)(206.05431557,289.88788069)(206.04432068,290.16788818)
\curveto(206.03431559,290.45788012)(206.00431562,290.73287985)(205.95432068,290.99288818)
\curveto(205.90431572,291.25287933)(205.82931579,291.46287912)(205.72932068,291.62288818)
\curveto(205.60931601,291.82287876)(205.45931616,291.97287861)(205.27932068,292.07288818)
\curveto(205.19931642,292.12287846)(205.10931651,292.15287843)(205.00932068,292.16288818)
\curveto(204.90931671,292.1828784)(204.80431682,292.19287839)(204.69432068,292.19288818)
\curveto(204.67431695,292.1828784)(204.64931697,292.1778784)(204.61932068,292.17788818)
\curveto(204.59931702,292.18787839)(204.57931704,292.18787839)(204.55932068,292.17788818)
\curveto(204.50931711,292.16787841)(204.46431716,292.15787842)(204.42432068,292.14788818)
\curveto(204.38431724,292.14787843)(204.34431728,292.13787844)(204.30432068,292.11788818)
\curveto(204.1243175,292.03787854)(203.97431765,291.91787866)(203.85432068,291.75788818)
\curveto(203.74431788,291.59787898)(203.65431797,291.41787916)(203.58432068,291.21788818)
\curveto(203.5243181,291.02787955)(203.47931814,290.80287978)(203.44932068,290.54288818)
\curveto(203.42931819,290.2828803)(203.4243182,290.01788056)(203.43432068,289.74788818)
\curveto(203.44431818,289.48788109)(203.47431815,289.23788134)(203.52432068,288.99788818)
\curveto(203.58431804,288.76788181)(203.65931796,288.577882)(203.74932068,288.42788818)
\moveto(214.54932068,285.44288818)
\curveto(214.55930706,285.39288519)(214.56430706,285.30288528)(214.56432068,285.17288818)
\curveto(214.56430706,285.04288554)(214.55430707,284.95288563)(214.53432068,284.90288818)
\curveto(214.51430711,284.85288573)(214.50930711,284.79788578)(214.51932068,284.73788818)
\curveto(214.52930709,284.68788589)(214.52930709,284.63788594)(214.51932068,284.58788818)
\curveto(214.47930714,284.44788613)(214.44930717,284.31288627)(214.42932068,284.18288818)
\curveto(214.4193072,284.05288653)(214.38930723,283.93288665)(214.33932068,283.82288818)
\curveto(214.19930742,283.47288711)(214.03430759,283.1778874)(213.84432068,282.93788818)
\curveto(213.65430797,282.70788787)(213.38430824,282.52288806)(213.03432068,282.38288818)
\curveto(212.95430867,282.35288823)(212.86930875,282.33288825)(212.77932068,282.32288818)
\curveto(212.68930893,282.30288828)(212.60430902,282.2828883)(212.52432068,282.26288818)
\curveto(212.47430915,282.25288833)(212.4243092,282.24788833)(212.37432068,282.24788818)
\curveto(212.3243093,282.24788833)(212.27430935,282.24288834)(212.22432068,282.23288818)
\curveto(212.19430943,282.22288836)(212.14430948,282.22288836)(212.07432068,282.23288818)
\curveto(212.00430962,282.23288835)(211.95430967,282.23788834)(211.92432068,282.24788818)
\curveto(211.86430976,282.26788831)(211.80430982,282.2778883)(211.74432068,282.27788818)
\curveto(211.69430993,282.26788831)(211.64430998,282.27288831)(211.59432068,282.29288818)
\curveto(211.50431012,282.31288827)(211.41431021,282.33788824)(211.32432068,282.36788818)
\curveto(211.24431038,282.38788819)(211.16431046,282.41788816)(211.08432068,282.45788818)
\curveto(210.76431086,282.59788798)(210.51431111,282.79288779)(210.33432068,283.04288818)
\curveto(210.15431147,283.30288728)(210.00431162,283.60788697)(209.88432068,283.95788818)
\curveto(209.86431176,284.03788654)(209.84931177,284.12288646)(209.83932068,284.21288818)
\curveto(209.82931179,284.30288628)(209.81431181,284.38788619)(209.79432068,284.46788818)
\curveto(209.78431184,284.49788608)(209.77931184,284.52788605)(209.77932068,284.55788818)
\lineto(209.77932068,284.66288818)
\curveto(209.75931186,284.74288584)(209.74931187,284.82288576)(209.74932068,284.90288818)
\lineto(209.74932068,285.03788818)
\curveto(209.72931189,285.13788544)(209.72931189,285.23788534)(209.74932068,285.33788818)
\lineto(209.74932068,285.51788818)
\curveto(209.75931186,285.56788501)(209.76431186,285.61288497)(209.76432068,285.65288818)
\curveto(209.76431186,285.70288488)(209.76931185,285.74788483)(209.77932068,285.78788818)
\curveto(209.78931183,285.82788475)(209.79431183,285.86288472)(209.79432068,285.89288818)
\curveto(209.79431183,285.93288465)(209.79931182,285.97288461)(209.80932068,286.01288818)
\lineto(209.86932068,286.34288818)
\curveto(209.88931173,286.46288412)(209.9193117,286.57288401)(209.95932068,286.67288818)
\curveto(210.09931152,287.00288358)(210.25931136,287.2778833)(210.43932068,287.49788818)
\curveto(210.62931099,287.72788285)(210.88931073,287.91288267)(211.21932068,288.05288818)
\curveto(211.29931032,288.09288249)(211.38431024,288.11788246)(211.47432068,288.12788818)
\lineto(211.77432068,288.18788818)
\lineto(211.90932068,288.18788818)
\curveto(211.95930966,288.19788238)(212.00930961,288.20288238)(212.05932068,288.20288818)
\curveto(212.62930899,288.22288236)(213.08930853,288.11788246)(213.43932068,287.88788818)
\curveto(213.79930782,287.66788291)(214.06430756,287.36788321)(214.23432068,286.98788818)
\curveto(214.28430734,286.88788369)(214.3243073,286.78788379)(214.35432068,286.68788818)
\curveto(214.38430724,286.58788399)(214.41430721,286.4828841)(214.44432068,286.37288818)
\curveto(214.45430717,286.33288425)(214.45930716,286.29788428)(214.45932068,286.26788818)
\curveto(214.45930716,286.24788433)(214.46430716,286.21788436)(214.47432068,286.17788818)
\curveto(214.49430713,286.10788447)(214.50430712,286.03288455)(214.50432068,285.95288818)
\curveto(214.50430712,285.87288471)(214.51430711,285.79288479)(214.53432068,285.71288818)
\curveto(214.53430709,285.66288492)(214.53430709,285.61788496)(214.53432068,285.57788818)
\curveto(214.53430709,285.53788504)(214.53930708,285.49288509)(214.54932068,285.44288818)
\moveto(213.43932068,285.00788818)
\curveto(213.44930817,285.05788552)(213.45430817,285.13288545)(213.45432068,285.23288818)
\curveto(213.46430816,285.33288525)(213.45930816,285.40788517)(213.43932068,285.45788818)
\curveto(213.4193082,285.51788506)(213.41430821,285.57288501)(213.42432068,285.62288818)
\curveto(213.44430818,285.6828849)(213.44430818,285.74288484)(213.42432068,285.80288818)
\curveto(213.41430821,285.83288475)(213.40930821,285.86788471)(213.40932068,285.90788818)
\curveto(213.40930821,285.94788463)(213.40430822,285.98788459)(213.39432068,286.02788818)
\curveto(213.37430825,286.10788447)(213.35430827,286.1828844)(213.33432068,286.25288818)
\curveto(213.3243083,286.33288425)(213.30930831,286.41288417)(213.28932068,286.49288818)
\curveto(213.25930836,286.55288403)(213.23430839,286.61288397)(213.21432068,286.67288818)
\curveto(213.19430843,286.73288385)(213.16430846,286.79288379)(213.12432068,286.85288818)
\curveto(213.0243086,287.02288356)(212.89430873,287.15788342)(212.73432068,287.25788818)
\curveto(212.65430897,287.30788327)(212.55930906,287.34288324)(212.44932068,287.36288818)
\curveto(212.33930928,287.3828832)(212.21430941,287.39288319)(212.07432068,287.39288818)
\curveto(212.05430957,287.3828832)(212.02930959,287.3778832)(211.99932068,287.37788818)
\curveto(211.96930965,287.38788319)(211.93930968,287.38788319)(211.90932068,287.37788818)
\lineto(211.75932068,287.31788818)
\curveto(211.70930991,287.30788327)(211.66430996,287.29288329)(211.62432068,287.27288818)
\curveto(211.43431019,287.16288342)(211.28931033,287.01788356)(211.18932068,286.83788818)
\curveto(211.09931052,286.65788392)(211.0193106,286.45288413)(210.94932068,286.22288818)
\curveto(210.90931071,286.09288449)(210.88931073,285.95788462)(210.88932068,285.81788818)
\curveto(210.88931073,285.68788489)(210.87931074,285.54288504)(210.85932068,285.38288818)
\curveto(210.84931077,285.33288525)(210.83931078,285.27288531)(210.82932068,285.20288818)
\curveto(210.82931079,285.13288545)(210.83931078,285.07288551)(210.85932068,285.02288818)
\lineto(210.85932068,284.85788818)
\lineto(210.85932068,284.67788818)
\curveto(210.86931075,284.62788595)(210.87931074,284.57288601)(210.88932068,284.51288818)
\curveto(210.89931072,284.46288612)(210.90431072,284.40788617)(210.90432068,284.34788818)
\curveto(210.91431071,284.28788629)(210.92931069,284.23288635)(210.94932068,284.18288818)
\curveto(210.99931062,283.99288659)(211.05931056,283.81788676)(211.12932068,283.65788818)
\curveto(211.19931042,283.49788708)(211.30431032,283.36788721)(211.44432068,283.26788818)
\curveto(211.57431005,283.16788741)(211.71430991,283.09788748)(211.86432068,283.05788818)
\curveto(211.89430973,283.04788753)(211.9193097,283.04288754)(211.93932068,283.04288818)
\curveto(211.96930965,283.05288753)(211.99930962,283.05288753)(212.02932068,283.04288818)
\curveto(212.04930957,283.04288754)(212.07930954,283.03788754)(212.11932068,283.02788818)
\curveto(212.15930946,283.02788755)(212.19430943,283.03288755)(212.22432068,283.04288818)
\curveto(212.26430936,283.05288753)(212.30430932,283.05788752)(212.34432068,283.05788818)
\curveto(212.38430924,283.05788752)(212.4243092,283.06788751)(212.46432068,283.08788818)
\curveto(212.70430892,283.16788741)(212.89930872,283.30288728)(213.04932068,283.49288818)
\curveto(213.16930845,283.67288691)(213.25930836,283.8778867)(213.31932068,284.10788818)
\curveto(213.33930828,284.1778864)(213.35430827,284.24788633)(213.36432068,284.31788818)
\curveto(213.37430825,284.39788618)(213.38930823,284.4778861)(213.40932068,284.55788818)
\curveto(213.40930821,284.61788596)(213.41430821,284.66288592)(213.42432068,284.69288818)
\curveto(213.4243082,284.71288587)(213.4243082,284.73788584)(213.42432068,284.76788818)
\curveto(213.4243082,284.80788577)(213.42930819,284.83788574)(213.43932068,284.85788818)
\lineto(213.43932068,285.00788818)
}
}
{
\newrgbcolor{curcolor}{0 0 0}
\pscustom[linestyle=none,fillstyle=solid,fillcolor=curcolor]
{
\newpath
\moveto(130.97265137,262.51282227)
\curveto(131.07264651,262.51281165)(131.16764642,262.50281166)(131.25765137,262.48282227)
\curveto(131.34764624,262.47281169)(131.41264617,262.44281172)(131.45265137,262.39282227)
\curveto(131.51264607,262.31281185)(131.54264604,262.20781195)(131.54265137,262.07782227)
\lineto(131.54265137,261.68782227)
\lineto(131.54265137,260.18782227)
\lineto(131.54265137,253.79782227)
\lineto(131.54265137,252.62782227)
\lineto(131.54265137,252.31282227)
\curveto(131.55264603,252.21282195)(131.53764605,252.13282203)(131.49765137,252.07282227)
\curveto(131.44764614,251.99282217)(131.37264621,251.94282222)(131.27265137,251.92282227)
\curveto(131.1826464,251.91282225)(131.07264651,251.90782225)(130.94265137,251.90782227)
\lineto(130.71765137,251.90782227)
\curveto(130.63764695,251.92782223)(130.56764702,251.94282222)(130.50765137,251.95282227)
\curveto(130.44764714,251.97282219)(130.39764719,252.01282215)(130.35765137,252.07282227)
\curveto(130.31764727,252.13282203)(130.29764729,252.20782195)(130.29765137,252.29782227)
\lineto(130.29765137,252.59782227)
\lineto(130.29765137,253.69282227)
\lineto(130.29765137,259.03282227)
\curveto(130.27764731,259.12281504)(130.26264732,259.19781496)(130.25265137,259.25782227)
\curveto(130.25264733,259.32781483)(130.22264736,259.38781477)(130.16265137,259.43782227)
\curveto(130.09264749,259.48781467)(130.00264758,259.51281465)(129.89265137,259.51282227)
\curveto(129.79264779,259.52281464)(129.6826479,259.52781463)(129.56265137,259.52782227)
\lineto(128.42265137,259.52782227)
\lineto(127.92765137,259.52782227)
\curveto(127.76764982,259.53781462)(127.65764993,259.59781456)(127.59765137,259.70782227)
\curveto(127.57765001,259.73781442)(127.56765002,259.76781439)(127.56765137,259.79782227)
\curveto(127.56765002,259.83781432)(127.56265002,259.88281428)(127.55265137,259.93282227)
\curveto(127.53265005,260.05281411)(127.53765005,260.162814)(127.56765137,260.26282227)
\curveto(127.60764998,260.3628138)(127.66264992,260.43281373)(127.73265137,260.47282227)
\curveto(127.81264977,260.52281364)(127.93264965,260.54781361)(128.09265137,260.54782227)
\curveto(128.25264933,260.54781361)(128.3876492,260.5628136)(128.49765137,260.59282227)
\curveto(128.54764904,260.60281356)(128.60264898,260.60781355)(128.66265137,260.60782227)
\curveto(128.72264886,260.61781354)(128.7826488,260.63281353)(128.84265137,260.65282227)
\curveto(128.99264859,260.70281346)(129.13764845,260.75281341)(129.27765137,260.80282227)
\curveto(129.41764817,260.8628133)(129.55264803,260.93281323)(129.68265137,261.01282227)
\curveto(129.82264776,261.10281306)(129.94264764,261.20781295)(130.04265137,261.32782227)
\curveto(130.14264744,261.44781271)(130.23764735,261.57781258)(130.32765137,261.71782227)
\curveto(130.3876472,261.81781234)(130.43264715,261.92781223)(130.46265137,262.04782227)
\curveto(130.50264708,262.16781199)(130.55264703,262.27281189)(130.61265137,262.36282227)
\curveto(130.66264692,262.42281174)(130.73264685,262.4628117)(130.82265137,262.48282227)
\curveto(130.84264674,262.49281167)(130.86764672,262.49781166)(130.89765137,262.49782227)
\curveto(130.92764666,262.49781166)(130.95264663,262.50281166)(130.97265137,262.51282227)
}
}
{
\newrgbcolor{curcolor}{0 0 0}
\pscustom[linestyle=none,fillstyle=solid,fillcolor=curcolor]
{
\newpath
\moveto(142.26226074,256.99282227)
\lineto(142.26226074,256.73782227)
\curveto(142.27225304,256.6578175)(142.26725304,256.58281758)(142.24726074,256.51282227)
\lineto(142.24726074,256.27282227)
\lineto(142.24726074,256.10782227)
\curveto(142.22725308,256.00781815)(142.21725309,255.90281826)(142.21726074,255.79282227)
\curveto(142.21725309,255.69281847)(142.2072531,255.59281857)(142.18726074,255.49282227)
\lineto(142.18726074,255.34282227)
\curveto(142.15725315,255.20281896)(142.13725317,255.0628191)(142.12726074,254.92282227)
\curveto(142.11725319,254.79281937)(142.09225322,254.6628195)(142.05226074,254.53282227)
\curveto(142.03225328,254.45281971)(142.0122533,254.36781979)(141.99226074,254.27782227)
\lineto(141.93226074,254.03782227)
\lineto(141.81226074,253.73782227)
\curveto(141.78225353,253.64782051)(141.74725356,253.5578206)(141.70726074,253.46782227)
\curveto(141.6072537,253.24782091)(141.47225384,253.03282113)(141.30226074,252.82282227)
\curveto(141.14225417,252.61282155)(140.96725434,252.44282172)(140.77726074,252.31282227)
\curveto(140.72725458,252.27282189)(140.66725464,252.23282193)(140.59726074,252.19282227)
\curveto(140.53725477,252.162822)(140.47725483,252.12782203)(140.41726074,252.08782227)
\curveto(140.33725497,252.03782212)(140.24225507,251.99782216)(140.13226074,251.96782227)
\curveto(140.02225529,251.93782222)(139.91725539,251.90782225)(139.81726074,251.87782227)
\curveto(139.7072556,251.83782232)(139.59725571,251.81282235)(139.48726074,251.80282227)
\curveto(139.37725593,251.79282237)(139.26225605,251.77782238)(139.14226074,251.75782227)
\curveto(139.10225621,251.74782241)(139.05725625,251.74782241)(139.00726074,251.75782227)
\curveto(138.96725634,251.7578224)(138.92725638,251.75282241)(138.88726074,251.74282227)
\curveto(138.84725646,251.73282243)(138.79225652,251.72782243)(138.72226074,251.72782227)
\curveto(138.65225666,251.72782243)(138.60225671,251.73282243)(138.57226074,251.74282227)
\curveto(138.52225679,251.7628224)(138.47725683,251.76782239)(138.43726074,251.75782227)
\curveto(138.39725691,251.74782241)(138.36225695,251.74782241)(138.33226074,251.75782227)
\lineto(138.24226074,251.75782227)
\curveto(138.18225713,251.77782238)(138.11725719,251.79282237)(138.04726074,251.80282227)
\curveto(137.98725732,251.80282236)(137.92225739,251.80782235)(137.85226074,251.81782227)
\curveto(137.68225763,251.86782229)(137.52225779,251.91782224)(137.37226074,251.96782227)
\curveto(137.22225809,252.01782214)(137.07725823,252.08282208)(136.93726074,252.16282227)
\curveto(136.88725842,252.20282196)(136.83225848,252.23282193)(136.77226074,252.25282227)
\curveto(136.72225859,252.28282188)(136.67225864,252.31782184)(136.62226074,252.35782227)
\curveto(136.38225893,252.53782162)(136.18225913,252.7578214)(136.02226074,253.01782227)
\curveto(135.86225945,253.27782088)(135.72225959,253.5628206)(135.60226074,253.87282227)
\curveto(135.54225977,254.01282015)(135.49725981,254.15282001)(135.46726074,254.29282227)
\curveto(135.43725987,254.44281972)(135.40225991,254.59781956)(135.36226074,254.75782227)
\curveto(135.34225997,254.86781929)(135.32725998,254.97781918)(135.31726074,255.08782227)
\curveto(135.30726,255.19781896)(135.29226002,255.30781885)(135.27226074,255.41782227)
\curveto(135.26226005,255.4578187)(135.25726005,255.49781866)(135.25726074,255.53782227)
\curveto(135.26726004,255.57781858)(135.26726004,255.61781854)(135.25726074,255.65782227)
\curveto(135.24726006,255.70781845)(135.24226007,255.7578184)(135.24226074,255.80782227)
\lineto(135.24226074,255.97282227)
\curveto(135.22226009,256.02281814)(135.21726009,256.07281809)(135.22726074,256.12282227)
\curveto(135.23726007,256.18281798)(135.23726007,256.23781792)(135.22726074,256.28782227)
\curveto(135.21726009,256.32781783)(135.21726009,256.37281779)(135.22726074,256.42282227)
\curveto(135.23726007,256.47281769)(135.23226008,256.52281764)(135.21226074,256.57282227)
\curveto(135.19226012,256.64281752)(135.18726012,256.71781744)(135.19726074,256.79782227)
\curveto(135.2072601,256.88781727)(135.2122601,256.97281719)(135.21226074,257.05282227)
\curveto(135.2122601,257.14281702)(135.2072601,257.24281692)(135.19726074,257.35282227)
\curveto(135.18726012,257.47281669)(135.19226012,257.57281659)(135.21226074,257.65282227)
\lineto(135.21226074,257.93782227)
\lineto(135.25726074,258.56782227)
\curveto(135.26726004,258.66781549)(135.27726003,258.7628154)(135.28726074,258.85282227)
\lineto(135.31726074,259.15282227)
\curveto(135.33725997,259.20281496)(135.34225997,259.25281491)(135.33226074,259.30282227)
\curveto(135.33225998,259.3628148)(135.34225997,259.41781474)(135.36226074,259.46782227)
\curveto(135.4122599,259.63781452)(135.45225986,259.80281436)(135.48226074,259.96282227)
\curveto(135.5122598,260.13281403)(135.56225975,260.29281387)(135.63226074,260.44282227)
\curveto(135.82225949,260.90281326)(136.04225927,261.27781288)(136.29226074,261.56782227)
\curveto(136.55225876,261.8578123)(136.9122584,262.10281206)(137.37226074,262.30282227)
\curveto(137.50225781,262.35281181)(137.63225768,262.38781177)(137.76226074,262.40782227)
\curveto(137.90225741,262.42781173)(138.04225727,262.45281171)(138.18226074,262.48282227)
\curveto(138.25225706,262.49281167)(138.31725699,262.49781166)(138.37726074,262.49782227)
\curveto(138.43725687,262.49781166)(138.50225681,262.50281166)(138.57226074,262.51282227)
\curveto(139.40225591,262.53281163)(140.07225524,262.38281178)(140.58226074,262.06282227)
\curveto(141.09225422,261.75281241)(141.47225384,261.31281285)(141.72226074,260.74282227)
\curveto(141.77225354,260.62281354)(141.81725349,260.49781366)(141.85726074,260.36782227)
\curveto(141.89725341,260.23781392)(141.94225337,260.10281406)(141.99226074,259.96282227)
\curveto(142.0122533,259.88281428)(142.02725328,259.79781436)(142.03726074,259.70782227)
\lineto(142.09726074,259.46782227)
\curveto(142.12725318,259.3578148)(142.14225317,259.24781491)(142.14226074,259.13782227)
\curveto(142.15225316,259.02781513)(142.16725314,258.91781524)(142.18726074,258.80782227)
\curveto(142.2072531,258.7578154)(142.2122531,258.71281545)(142.20226074,258.67282227)
\curveto(142.20225311,258.63281553)(142.2072531,258.59281557)(142.21726074,258.55282227)
\curveto(142.22725308,258.50281566)(142.22725308,258.44781571)(142.21726074,258.38782227)
\curveto(142.21725309,258.33781582)(142.22225309,258.28781587)(142.23226074,258.23782227)
\lineto(142.23226074,258.10282227)
\curveto(142.25225306,258.04281612)(142.25225306,257.97281619)(142.23226074,257.89282227)
\curveto(142.22225309,257.82281634)(142.22725308,257.7578164)(142.24726074,257.69782227)
\curveto(142.25725305,257.66781649)(142.26225305,257.62781653)(142.26226074,257.57782227)
\lineto(142.26226074,257.45782227)
\lineto(142.26226074,256.99282227)
\moveto(140.71726074,254.66782227)
\curveto(140.81725449,254.98781917)(140.87725443,255.35281881)(140.89726074,255.76282227)
\curveto(140.91725439,256.17281799)(140.92725438,256.58281758)(140.92726074,256.99282227)
\curveto(140.92725438,257.42281674)(140.91725439,257.84281632)(140.89726074,258.25282227)
\curveto(140.87725443,258.6628155)(140.83225448,259.04781511)(140.76226074,259.40782227)
\curveto(140.69225462,259.76781439)(140.58225473,260.08781407)(140.43226074,260.36782227)
\curveto(140.29225502,260.6578135)(140.09725521,260.89281327)(139.84726074,261.07282227)
\curveto(139.68725562,261.18281298)(139.5072558,261.2628129)(139.30726074,261.31282227)
\curveto(139.1072562,261.37281279)(138.86225645,261.40281276)(138.57226074,261.40282227)
\curveto(138.55225676,261.38281278)(138.51725679,261.37281279)(138.46726074,261.37282227)
\curveto(138.41725689,261.38281278)(138.37725693,261.38281278)(138.34726074,261.37282227)
\curveto(138.26725704,261.35281281)(138.19225712,261.33281283)(138.12226074,261.31282227)
\curveto(138.06225725,261.30281286)(137.99725731,261.28281288)(137.92726074,261.25282227)
\curveto(137.65725765,261.13281303)(137.43725787,260.9628132)(137.26726074,260.74282227)
\curveto(137.1072582,260.53281363)(136.97225834,260.28781387)(136.86226074,260.00782227)
\curveto(136.8122585,259.89781426)(136.77225854,259.77781438)(136.74226074,259.64782227)
\curveto(136.72225859,259.52781463)(136.69725861,259.40281476)(136.66726074,259.27282227)
\curveto(136.64725866,259.22281494)(136.63725867,259.16781499)(136.63726074,259.10782227)
\curveto(136.63725867,259.0578151)(136.63225868,259.00781515)(136.62226074,258.95782227)
\curveto(136.6122587,258.86781529)(136.60225871,258.77281539)(136.59226074,258.67282227)
\curveto(136.58225873,258.58281558)(136.57225874,258.48781567)(136.56226074,258.38782227)
\curveto(136.56225875,258.30781585)(136.55725875,258.22281594)(136.54726074,258.13282227)
\lineto(136.54726074,257.89282227)
\lineto(136.54726074,257.71282227)
\curveto(136.53725877,257.68281648)(136.53225878,257.64781651)(136.53226074,257.60782227)
\lineto(136.53226074,257.47282227)
\lineto(136.53226074,257.02282227)
\curveto(136.53225878,256.94281722)(136.52725878,256.8578173)(136.51726074,256.76782227)
\curveto(136.51725879,256.68781747)(136.52725878,256.61281755)(136.54726074,256.54282227)
\lineto(136.54726074,256.27282227)
\curveto(136.54725876,256.25281791)(136.54225877,256.22281794)(136.53226074,256.18282227)
\curveto(136.53225878,256.15281801)(136.53725877,256.12781803)(136.54726074,256.10782227)
\curveto(136.55725875,256.00781815)(136.56225875,255.90781825)(136.56226074,255.80782227)
\curveto(136.57225874,255.71781844)(136.58225873,255.61781854)(136.59226074,255.50782227)
\curveto(136.62225869,255.38781877)(136.63725867,255.2628189)(136.63726074,255.13282227)
\curveto(136.64725866,255.01281915)(136.67225864,254.89781926)(136.71226074,254.78782227)
\curveto(136.79225852,254.48781967)(136.87725843,254.22281994)(136.96726074,253.99282227)
\curveto(137.06725824,253.7628204)(137.2122581,253.54782061)(137.40226074,253.34782227)
\curveto(137.6122577,253.14782101)(137.87725743,252.99782116)(138.19726074,252.89782227)
\curveto(138.23725707,252.87782128)(138.27225704,252.86782129)(138.30226074,252.86782227)
\curveto(138.34225697,252.87782128)(138.38725692,252.87282129)(138.43726074,252.85282227)
\curveto(138.47725683,252.84282132)(138.54725676,252.83282133)(138.64726074,252.82282227)
\curveto(138.75725655,252.81282135)(138.84225647,252.81782134)(138.90226074,252.83782227)
\curveto(138.97225634,252.8578213)(139.04225627,252.86782129)(139.11226074,252.86782227)
\curveto(139.18225613,252.87782128)(139.24725606,252.89282127)(139.30726074,252.91282227)
\curveto(139.5072558,252.97282119)(139.68725562,253.0578211)(139.84726074,253.16782227)
\curveto(139.87725543,253.18782097)(139.90225541,253.20782095)(139.92226074,253.22782227)
\lineto(139.98226074,253.28782227)
\curveto(140.02225529,253.30782085)(140.07225524,253.34782081)(140.13226074,253.40782227)
\curveto(140.23225508,253.54782061)(140.31725499,253.67782048)(140.38726074,253.79782227)
\curveto(140.45725485,253.91782024)(140.52725478,254.0628201)(140.59726074,254.23282227)
\curveto(140.62725468,254.30281986)(140.64725466,254.37281979)(140.65726074,254.44282227)
\curveto(140.67725463,254.51281965)(140.69725461,254.58781957)(140.71726074,254.66782227)
}
}
{
\newrgbcolor{curcolor}{0 0 0}
\pscustom[linestyle=none,fillstyle=solid,fillcolor=curcolor]
{
\newpath
\moveto(144.56687012,253.54282227)
\lineto(144.86687012,253.54282227)
\curveto(144.97686806,253.55282061)(145.08186795,253.55282061)(145.18187012,253.54282227)
\curveto(145.29186774,253.54282062)(145.39186764,253.53282063)(145.48187012,253.51282227)
\curveto(145.57186746,253.50282066)(145.64186739,253.47782068)(145.69187012,253.43782227)
\curveto(145.71186732,253.41782074)(145.72686731,253.38782077)(145.73687012,253.34782227)
\curveto(145.75686728,253.30782085)(145.77686726,253.2628209)(145.79687012,253.21282227)
\lineto(145.79687012,253.13782227)
\curveto(145.80686723,253.08782107)(145.80686723,253.03282113)(145.79687012,252.97282227)
\lineto(145.79687012,252.82282227)
\lineto(145.79687012,252.34282227)
\curveto(145.79686724,252.17282199)(145.75686728,252.05282211)(145.67687012,251.98282227)
\curveto(145.60686743,251.93282223)(145.51686752,251.90782225)(145.40687012,251.90782227)
\lineto(145.07687012,251.90782227)
\lineto(144.62687012,251.90782227)
\curveto(144.47686856,251.90782225)(144.36186867,251.93782222)(144.28187012,251.99782227)
\curveto(144.24186879,252.02782213)(144.21186882,252.07782208)(144.19187012,252.14782227)
\curveto(144.17186886,252.22782193)(144.15686888,252.31282185)(144.14687012,252.40282227)
\lineto(144.14687012,252.68782227)
\curveto(144.15686888,252.78782137)(144.16186887,252.87282129)(144.16187012,252.94282227)
\lineto(144.16187012,253.13782227)
\curveto(144.16186887,253.19782096)(144.17186886,253.25282091)(144.19187012,253.30282227)
\curveto(144.2318688,253.41282075)(144.30186873,253.48282068)(144.40187012,253.51282227)
\curveto(144.4318686,253.51282065)(144.48686855,253.52282064)(144.56687012,253.54282227)
}
}
{
\newrgbcolor{curcolor}{0 0 0}
\pscustom[linestyle=none,fillstyle=solid,fillcolor=curcolor]
{
\newpath
\moveto(149.28202637,262.31782227)
\lineto(152.88202637,262.31782227)
\lineto(153.52702637,262.31782227)
\curveto(153.60701984,262.31781184)(153.68201976,262.31281185)(153.75202637,262.30282227)
\curveto(153.82201962,262.30281186)(153.88201956,262.29281187)(153.93202637,262.27282227)
\curveto(154.00201944,262.24281192)(154.05701939,262.18281198)(154.09702637,262.09282227)
\curveto(154.11701933,262.0628121)(154.12701932,262.02281214)(154.12702637,261.97282227)
\lineto(154.12702637,261.83782227)
\curveto(154.13701931,261.72781243)(154.13201931,261.62281254)(154.11202637,261.52282227)
\curveto(154.10201934,261.42281274)(154.06701938,261.35281281)(154.00702637,261.31282227)
\curveto(153.91701953,261.24281292)(153.78201966,261.20781295)(153.60202637,261.20782227)
\curveto(153.42202002,261.21781294)(153.25702019,261.22281294)(153.10702637,261.22282227)
\lineto(151.11202637,261.22282227)
\lineto(150.61702637,261.22282227)
\lineto(150.48202637,261.22282227)
\curveto(150.442023,261.22281294)(150.40202304,261.21781294)(150.36202637,261.20782227)
\lineto(150.15202637,261.20782227)
\curveto(150.0420234,261.17781298)(149.96202348,261.13781302)(149.91202637,261.08782227)
\curveto(149.86202358,261.04781311)(149.82702362,260.99281317)(149.80702637,260.92282227)
\curveto(149.78702366,260.8628133)(149.77202367,260.79281337)(149.76202637,260.71282227)
\curveto(149.75202369,260.63281353)(149.73202371,260.54281362)(149.70202637,260.44282227)
\curveto(149.65202379,260.24281392)(149.61202383,260.03781412)(149.58202637,259.82782227)
\curveto(149.55202389,259.61781454)(149.51202393,259.41281475)(149.46202637,259.21282227)
\curveto(149.442024,259.14281502)(149.43202401,259.07281509)(149.43202637,259.00282227)
\curveto(149.43202401,258.94281522)(149.42202402,258.87781528)(149.40202637,258.80782227)
\curveto(149.39202405,258.77781538)(149.38202406,258.73781542)(149.37202637,258.68782227)
\curveto(149.37202407,258.64781551)(149.37702407,258.60781555)(149.38702637,258.56782227)
\curveto(149.40702404,258.51781564)(149.43202401,258.47281569)(149.46202637,258.43282227)
\curveto(149.50202394,258.40281576)(149.56202388,258.39781576)(149.64202637,258.41782227)
\curveto(149.70202374,258.43781572)(149.76202368,258.4628157)(149.82202637,258.49282227)
\curveto(149.88202356,258.53281563)(149.9420235,258.56781559)(150.00202637,258.59782227)
\curveto(150.06202338,258.61781554)(150.11202333,258.63281553)(150.15202637,258.64282227)
\curveto(150.3420231,258.72281544)(150.5470229,258.77781538)(150.76702637,258.80782227)
\curveto(150.99702245,258.83781532)(151.22702222,258.84781531)(151.45702637,258.83782227)
\curveto(151.69702175,258.83781532)(151.92702152,258.81281535)(152.14702637,258.76282227)
\curveto(152.36702108,258.72281544)(152.56702088,258.6628155)(152.74702637,258.58282227)
\curveto(152.79702065,258.5628156)(152.8420206,258.54281562)(152.88202637,258.52282227)
\curveto(152.93202051,258.50281566)(152.98202046,258.47781568)(153.03202637,258.44782227)
\curveto(153.38202006,258.23781592)(153.66201978,258.00781615)(153.87202637,257.75782227)
\curveto(154.09201935,257.50781665)(154.28701916,257.18281698)(154.45702637,256.78282227)
\curveto(154.50701894,256.67281749)(154.5420189,256.5628176)(154.56202637,256.45282227)
\curveto(154.58201886,256.34281782)(154.60701884,256.22781793)(154.63702637,256.10782227)
\curveto(154.6470188,256.07781808)(154.65201879,256.03281813)(154.65202637,255.97282227)
\curveto(154.67201877,255.91281825)(154.68201876,255.84281832)(154.68202637,255.76282227)
\curveto(154.68201876,255.69281847)(154.69201875,255.62781853)(154.71202637,255.56782227)
\lineto(154.71202637,255.40282227)
\curveto(154.72201872,255.35281881)(154.72701872,255.28281888)(154.72702637,255.19282227)
\curveto(154.72701872,255.10281906)(154.71701873,255.03281913)(154.69702637,254.98282227)
\curveto(154.67701877,254.92281924)(154.67201877,254.8628193)(154.68202637,254.80282227)
\curveto(154.69201875,254.75281941)(154.68701876,254.70281946)(154.66702637,254.65282227)
\curveto(154.62701882,254.49281967)(154.59201885,254.34281982)(154.56202637,254.20282227)
\curveto(154.53201891,254.0628201)(154.48701896,253.92782023)(154.42702637,253.79782227)
\curveto(154.26701918,253.42782073)(154.0470194,253.09282107)(153.76702637,252.79282227)
\curveto(153.48701996,252.49282167)(153.16702028,252.2628219)(152.80702637,252.10282227)
\curveto(152.63702081,252.02282214)(152.43702101,251.94782221)(152.20702637,251.87782227)
\curveto(152.09702135,251.83782232)(151.98202146,251.81282235)(151.86202637,251.80282227)
\curveto(151.7420217,251.79282237)(151.62202182,251.77282239)(151.50202637,251.74282227)
\curveto(151.45202199,251.72282244)(151.39702205,251.72282244)(151.33702637,251.74282227)
\curveto(151.27702217,251.75282241)(151.21702223,251.74782241)(151.15702637,251.72782227)
\curveto(151.05702239,251.70782245)(150.95702249,251.70782245)(150.85702637,251.72782227)
\lineto(150.72202637,251.72782227)
\curveto(150.67202277,251.74782241)(150.61202283,251.7578224)(150.54202637,251.75782227)
\curveto(150.48202296,251.74782241)(150.42702302,251.75282241)(150.37702637,251.77282227)
\curveto(150.33702311,251.78282238)(150.30202314,251.78782237)(150.27202637,251.78782227)
\curveto(150.2420232,251.78782237)(150.20702324,251.79282237)(150.16702637,251.80282227)
\lineto(149.89702637,251.86282227)
\curveto(149.80702364,251.88282228)(149.72202372,251.91282225)(149.64202637,251.95282227)
\curveto(149.30202414,252.09282207)(149.01202443,252.24782191)(148.77202637,252.41782227)
\curveto(148.53202491,252.59782156)(148.31202513,252.82782133)(148.11202637,253.10782227)
\curveto(147.96202548,253.33782082)(147.8470256,253.57782058)(147.76702637,253.82782227)
\curveto(147.7470257,253.87782028)(147.73702571,253.92282024)(147.73702637,253.96282227)
\curveto(147.73702571,254.01282015)(147.72702572,254.0628201)(147.70702637,254.11282227)
\curveto(147.68702576,254.17281999)(147.67202577,254.25281991)(147.66202637,254.35282227)
\curveto(147.66202578,254.45281971)(147.68202576,254.52781963)(147.72202637,254.57782227)
\curveto(147.77202567,254.6578195)(147.85202559,254.70281946)(147.96202637,254.71282227)
\curveto(148.07202537,254.72281944)(148.18702526,254.72781943)(148.30702637,254.72782227)
\lineto(148.47202637,254.72782227)
\curveto(148.53202491,254.72781943)(148.58702486,254.71781944)(148.63702637,254.69782227)
\curveto(148.72702472,254.67781948)(148.79702465,254.63781952)(148.84702637,254.57782227)
\curveto(148.91702453,254.48781967)(148.96202448,254.37781978)(148.98202637,254.24782227)
\curveto(149.01202443,254.12782003)(149.05702439,254.02282014)(149.11702637,253.93282227)
\curveto(149.30702414,253.59282057)(149.56702388,253.32282084)(149.89702637,253.12282227)
\curveto(149.99702345,253.0628211)(150.10202334,253.01282115)(150.21202637,252.97282227)
\curveto(150.33202311,252.94282122)(150.45202299,252.90782125)(150.57202637,252.86782227)
\curveto(150.7420227,252.81782134)(150.9470225,252.79782136)(151.18702637,252.80782227)
\curveto(151.43702201,252.82782133)(151.63702181,252.8628213)(151.78702637,252.91282227)
\curveto(152.15702129,253.03282113)(152.447021,253.19282097)(152.65702637,253.39282227)
\curveto(152.87702057,253.60282056)(153.05702039,253.88282028)(153.19702637,254.23282227)
\curveto(153.2470202,254.33281983)(153.27702017,254.43781972)(153.28702637,254.54782227)
\curveto(153.30702014,254.6578195)(153.33202011,254.77281939)(153.36202637,254.89282227)
\lineto(153.36202637,254.99782227)
\curveto(153.37202007,255.03781912)(153.37702007,255.07781908)(153.37702637,255.11782227)
\curveto(153.38702006,255.14781901)(153.38702006,255.18281898)(153.37702637,255.22282227)
\lineto(153.37702637,255.34282227)
\curveto(153.37702007,255.60281856)(153.3470201,255.84781831)(153.28702637,256.07782227)
\curveto(153.17702027,256.42781773)(153.02202042,256.72281744)(152.82202637,256.96282227)
\curveto(152.62202082,257.21281695)(152.36202108,257.40781675)(152.04202637,257.54782227)
\lineto(151.86202637,257.60782227)
\curveto(151.81202163,257.62781653)(151.75202169,257.64781651)(151.68202637,257.66782227)
\curveto(151.63202181,257.68781647)(151.57202187,257.69781646)(151.50202637,257.69782227)
\curveto(151.442022,257.70781645)(151.37702207,257.72281644)(151.30702637,257.74282227)
\lineto(151.15702637,257.74282227)
\curveto(151.11702233,257.7628164)(151.06202238,257.77281639)(150.99202637,257.77282227)
\curveto(150.93202251,257.77281639)(150.87702257,257.7628164)(150.82702637,257.74282227)
\lineto(150.72202637,257.74282227)
\curveto(150.69202275,257.74281642)(150.65702279,257.73781642)(150.61702637,257.72782227)
\lineto(150.37702637,257.66782227)
\curveto(150.29702315,257.6578165)(150.21702323,257.63781652)(150.13702637,257.60782227)
\curveto(149.89702355,257.50781665)(149.66702378,257.37281679)(149.44702637,257.20282227)
\curveto(149.35702409,257.13281703)(149.27202417,257.0578171)(149.19202637,256.97782227)
\curveto(149.11202433,256.90781725)(149.01202443,256.85281731)(148.89202637,256.81282227)
\curveto(148.80202464,256.78281738)(148.66202478,256.77281739)(148.47202637,256.78282227)
\curveto(148.29202515,256.79281737)(148.17202527,256.81781734)(148.11202637,256.85782227)
\curveto(148.06202538,256.89781726)(148.02202542,256.9578172)(147.99202637,257.03782227)
\curveto(147.97202547,257.11781704)(147.97202547,257.20281696)(147.99202637,257.29282227)
\curveto(148.02202542,257.41281675)(148.0420254,257.53281663)(148.05202637,257.65282227)
\curveto(148.07202537,257.78281638)(148.09702535,257.90781625)(148.12702637,258.02782227)
\curveto(148.1470253,258.06781609)(148.15202529,258.10281606)(148.14202637,258.13282227)
\curveto(148.1420253,258.17281599)(148.15202529,258.21781594)(148.17202637,258.26782227)
\curveto(148.19202525,258.3578158)(148.20702524,258.44781571)(148.21702637,258.53782227)
\curveto(148.22702522,258.63781552)(148.2470252,258.73281543)(148.27702637,258.82282227)
\curveto(148.28702516,258.88281528)(148.29202515,258.94281522)(148.29202637,259.00282227)
\curveto(148.30202514,259.0628151)(148.31702513,259.12281504)(148.33702637,259.18282227)
\curveto(148.38702506,259.38281478)(148.42202502,259.58781457)(148.44202637,259.79782227)
\curveto(148.47202497,260.01781414)(148.51202493,260.22781393)(148.56202637,260.42782227)
\curveto(148.59202485,260.52781363)(148.61202483,260.62781353)(148.62202637,260.72782227)
\curveto(148.63202481,260.82781333)(148.6470248,260.92781323)(148.66702637,261.02782227)
\curveto(148.67702477,261.0578131)(148.68202476,261.09781306)(148.68202637,261.14782227)
\curveto(148.71202473,261.2578129)(148.73202471,261.3628128)(148.74202637,261.46282227)
\curveto(148.76202468,261.57281259)(148.78702466,261.68281248)(148.81702637,261.79282227)
\curveto(148.83702461,261.87281229)(148.85202459,261.94281222)(148.86202637,262.00282227)
\curveto(148.87202457,262.07281209)(148.89702455,262.13281203)(148.93702637,262.18282227)
\curveto(148.95702449,262.21281195)(148.98702446,262.23281193)(149.02702637,262.24282227)
\curveto(149.06702438,262.2628119)(149.11202433,262.28281188)(149.16202637,262.30282227)
\curveto(149.22202422,262.30281186)(149.26202418,262.30781185)(149.28202637,262.31782227)
}
}
{
\newrgbcolor{curcolor}{0 0 0}
\pscustom[linestyle=none,fillstyle=solid,fillcolor=curcolor]
{
\newpath
\moveto(165.92663574,260.42782227)
\curveto(165.72662544,260.13781402)(165.51662565,259.85281431)(165.29663574,259.57282227)
\curveto(165.08662608,259.29281487)(164.88162629,259.00781515)(164.68163574,258.71782227)
\curveto(164.08162709,257.86781629)(163.47662769,257.02781713)(162.86663574,256.19782227)
\curveto(162.25662891,255.37781878)(161.65162952,254.54281962)(161.05163574,253.69282227)
\lineto(160.54163574,252.97282227)
\lineto(160.03163574,252.28282227)
\curveto(159.95163122,252.17282199)(159.8716313,252.0578221)(159.79163574,251.93782227)
\curveto(159.71163146,251.81782234)(159.61663155,251.72282244)(159.50663574,251.65282227)
\curveto(159.4666317,251.63282253)(159.40163177,251.61782254)(159.31163574,251.60782227)
\curveto(159.23163194,251.58782257)(159.14163203,251.57782258)(159.04163574,251.57782227)
\curveto(158.94163223,251.57782258)(158.84663232,251.58282258)(158.75663574,251.59282227)
\curveto(158.67663249,251.60282256)(158.61663255,251.62282254)(158.57663574,251.65282227)
\curveto(158.54663262,251.67282249)(158.52163265,251.70782245)(158.50163574,251.75782227)
\curveto(158.49163268,251.79782236)(158.49663267,251.84282232)(158.51663574,251.89282227)
\curveto(158.55663261,251.97282219)(158.60163257,252.04782211)(158.65163574,252.11782227)
\curveto(158.71163246,252.19782196)(158.7666324,252.27782188)(158.81663574,252.35782227)
\curveto(159.05663211,252.69782146)(159.30163187,253.03282113)(159.55163574,253.36282227)
\curveto(159.80163137,253.69282047)(160.04163113,254.02782013)(160.27163574,254.36782227)
\curveto(160.43163074,254.58781957)(160.59163058,254.80281936)(160.75163574,255.01282227)
\curveto(160.91163026,255.22281894)(161.0716301,255.43781872)(161.23163574,255.65782227)
\curveto(161.59162958,256.17781798)(161.95662921,256.68781747)(162.32663574,257.18782227)
\curveto(162.69662847,257.68781647)(163.0666281,258.19781596)(163.43663574,258.71782227)
\curveto(163.57662759,258.91781524)(163.71662745,259.11281505)(163.85663574,259.30282227)
\curveto(164.00662716,259.49281467)(164.15162702,259.68781447)(164.29163574,259.88782227)
\curveto(164.50162667,260.18781397)(164.71662645,260.48781367)(164.93663574,260.78782227)
\lineto(165.59663574,261.68782227)
\lineto(165.77663574,261.95782227)
\lineto(165.98663574,262.22782227)
\lineto(166.10663574,262.40782227)
\curveto(166.15662501,262.46781169)(166.20662496,262.52281164)(166.25663574,262.57282227)
\curveto(166.32662484,262.62281154)(166.40162477,262.6578115)(166.48163574,262.67782227)
\curveto(166.50162467,262.68781147)(166.52662464,262.68781147)(166.55663574,262.67782227)
\curveto(166.59662457,262.67781148)(166.62662454,262.68781147)(166.64663574,262.70782227)
\curveto(166.7666244,262.70781145)(166.90162427,262.70281146)(167.05163574,262.69282227)
\curveto(167.20162397,262.69281147)(167.29162388,262.64781151)(167.32163574,262.55782227)
\curveto(167.34162383,262.52781163)(167.34662382,262.49281167)(167.33663574,262.45282227)
\curveto(167.32662384,262.41281175)(167.31162386,262.38281178)(167.29163574,262.36282227)
\curveto(167.25162392,262.28281188)(167.21162396,262.21281195)(167.17163574,262.15282227)
\curveto(167.13162404,262.09281207)(167.08662408,262.03281213)(167.03663574,261.97282227)
\lineto(166.46663574,261.19282227)
\curveto(166.28662488,260.94281322)(166.10662506,260.68781347)(165.92663574,260.42782227)
\moveto(159.07163574,256.52782227)
\curveto(159.02163215,256.54781761)(158.9716322,256.55281761)(158.92163574,256.54282227)
\curveto(158.8716323,256.53281763)(158.82163235,256.53781762)(158.77163574,256.55782227)
\curveto(158.66163251,256.57781758)(158.55663261,256.59781756)(158.45663574,256.61782227)
\curveto(158.3666328,256.64781751)(158.2716329,256.68781747)(158.17163574,256.73782227)
\curveto(157.84163333,256.87781728)(157.58663358,257.07281709)(157.40663574,257.32282227)
\curveto(157.22663394,257.58281658)(157.08163409,257.89281627)(156.97163574,258.25282227)
\curveto(156.94163423,258.33281583)(156.92163425,258.41281575)(156.91163574,258.49282227)
\curveto(156.90163427,258.58281558)(156.88663428,258.66781549)(156.86663574,258.74782227)
\curveto(156.85663431,258.79781536)(156.85163432,258.8628153)(156.85163574,258.94282227)
\curveto(156.84163433,258.97281519)(156.83663433,259.00281516)(156.83663574,259.03282227)
\curveto(156.83663433,259.07281509)(156.83163434,259.10781505)(156.82163574,259.13782227)
\lineto(156.82163574,259.28782227)
\curveto(156.81163436,259.33781482)(156.80663436,259.39781476)(156.80663574,259.46782227)
\curveto(156.80663436,259.54781461)(156.81163436,259.61281455)(156.82163574,259.66282227)
\lineto(156.82163574,259.82782227)
\curveto(156.84163433,259.87781428)(156.84663432,259.92281424)(156.83663574,259.96282227)
\curveto(156.83663433,260.01281415)(156.84163433,260.0578141)(156.85163574,260.09782227)
\curveto(156.86163431,260.13781402)(156.8666343,260.17281399)(156.86663574,260.20282227)
\curveto(156.8666343,260.24281392)(156.8716343,260.28281388)(156.88163574,260.32282227)
\curveto(156.91163426,260.43281373)(156.93163424,260.54281362)(156.94163574,260.65282227)
\curveto(156.96163421,260.77281339)(156.99663417,260.88781327)(157.04663574,260.99782227)
\curveto(157.18663398,261.33781282)(157.34663382,261.61281255)(157.52663574,261.82282227)
\curveto(157.71663345,262.04281212)(157.98663318,262.22281194)(158.33663574,262.36282227)
\curveto(158.41663275,262.39281177)(158.50163267,262.41281175)(158.59163574,262.42282227)
\curveto(158.68163249,262.44281172)(158.77663239,262.4628117)(158.87663574,262.48282227)
\curveto(158.90663226,262.49281167)(158.96163221,262.49281167)(159.04163574,262.48282227)
\curveto(159.12163205,262.48281168)(159.171632,262.49281167)(159.19163574,262.51282227)
\curveto(159.75163142,262.52281164)(160.20163097,262.41281175)(160.54163574,262.18282227)
\curveto(160.89163028,261.95281221)(161.15163002,261.64781251)(161.32163574,261.26782227)
\curveto(161.36162981,261.17781298)(161.39662977,261.08281308)(161.42663574,260.98282227)
\curveto(161.45662971,260.88281328)(161.48162969,260.78281338)(161.50163574,260.68282227)
\curveto(161.52162965,260.65281351)(161.52662964,260.62281354)(161.51663574,260.59282227)
\curveto(161.51662965,260.5628136)(161.52162965,260.53281363)(161.53163574,260.50282227)
\curveto(161.56162961,260.39281377)(161.58162959,260.26781389)(161.59163574,260.12782227)
\curveto(161.60162957,259.99781416)(161.61162956,259.8628143)(161.62163574,259.72282227)
\lineto(161.62163574,259.55782227)
\curveto(161.63162954,259.49781466)(161.63162954,259.44281472)(161.62163574,259.39282227)
\curveto(161.61162956,259.34281482)(161.60662956,259.29281487)(161.60663574,259.24282227)
\lineto(161.60663574,259.10782227)
\curveto(161.59662957,259.06781509)(161.59162958,259.02781513)(161.59163574,258.98782227)
\curveto(161.60162957,258.94781521)(161.59662957,258.90281526)(161.57663574,258.85282227)
\curveto(161.55662961,258.74281542)(161.53662963,258.63781552)(161.51663574,258.53782227)
\curveto(161.50662966,258.43781572)(161.48662968,258.33781582)(161.45663574,258.23782227)
\curveto(161.32662984,257.87781628)(161.16163001,257.5628166)(160.96163574,257.29282227)
\curveto(160.76163041,257.02281714)(160.48663068,256.81781734)(160.13663574,256.67782227)
\curveto(160.05663111,256.64781751)(159.9716312,256.62281754)(159.88163574,256.60282227)
\lineto(159.61163574,256.54282227)
\curveto(159.56163161,256.53281763)(159.51663165,256.52781763)(159.47663574,256.52782227)
\curveto(159.43663173,256.53781762)(159.39663177,256.53781762)(159.35663574,256.52782227)
\curveto(159.25663191,256.50781765)(159.16163201,256.50781765)(159.07163574,256.52782227)
\moveto(158.23163574,257.92282227)
\curveto(158.2716329,257.85281631)(158.31163286,257.78781637)(158.35163574,257.72782227)
\curveto(158.39163278,257.67781648)(158.44163273,257.62781653)(158.50163574,257.57782227)
\lineto(158.65163574,257.45782227)
\curveto(158.71163246,257.42781673)(158.77663239,257.40281676)(158.84663574,257.38282227)
\curveto(158.88663228,257.3628168)(158.92163225,257.35281681)(158.95163574,257.35282227)
\curveto(158.99163218,257.3628168)(159.03163214,257.3578168)(159.07163574,257.33782227)
\curveto(159.10163207,257.33781682)(159.14163203,257.33281683)(159.19163574,257.32282227)
\curveto(159.24163193,257.32281684)(159.28163189,257.32781683)(159.31163574,257.33782227)
\lineto(159.53663574,257.38282227)
\curveto(159.78663138,257.4628167)(159.9716312,257.58781657)(160.09163574,257.75782227)
\curveto(160.171631,257.8578163)(160.24163093,257.98781617)(160.30163574,258.14782227)
\curveto(160.38163079,258.32781583)(160.44163073,258.55281561)(160.48163574,258.82282227)
\curveto(160.52163065,259.10281506)(160.53663063,259.38281478)(160.52663574,259.66282227)
\curveto(160.51663065,259.95281421)(160.48663068,260.22781393)(160.43663574,260.48782227)
\curveto(160.38663078,260.74781341)(160.31163086,260.9578132)(160.21163574,261.11782227)
\curveto(160.09163108,261.31781284)(159.94163123,261.46781269)(159.76163574,261.56782227)
\curveto(159.68163149,261.61781254)(159.59163158,261.64781251)(159.49163574,261.65782227)
\curveto(159.39163178,261.67781248)(159.28663188,261.68781247)(159.17663574,261.68782227)
\curveto(159.15663201,261.67781248)(159.13163204,261.67281249)(159.10163574,261.67282227)
\curveto(159.08163209,261.68281248)(159.06163211,261.68281248)(159.04163574,261.67282227)
\curveto(158.99163218,261.6628125)(158.94663222,261.65281251)(158.90663574,261.64282227)
\curveto(158.8666323,261.64281252)(158.82663234,261.63281253)(158.78663574,261.61282227)
\curveto(158.60663256,261.53281263)(158.45663271,261.41281275)(158.33663574,261.25282227)
\curveto(158.22663294,261.09281307)(158.13663303,260.91281325)(158.06663574,260.71282227)
\curveto(158.00663316,260.52281364)(157.96163321,260.29781386)(157.93163574,260.03782227)
\curveto(157.91163326,259.77781438)(157.90663326,259.51281465)(157.91663574,259.24282227)
\curveto(157.92663324,258.98281518)(157.95663321,258.73281543)(158.00663574,258.49282227)
\curveto(158.0666331,258.2628159)(158.14163303,258.07281609)(158.23163574,257.92282227)
\moveto(169.03163574,254.93782227)
\curveto(169.04162213,254.88781927)(169.04662212,254.79781936)(169.04663574,254.66782227)
\curveto(169.04662212,254.53781962)(169.03662213,254.44781971)(169.01663574,254.39782227)
\curveto(168.99662217,254.34781981)(168.99162218,254.29281987)(169.00163574,254.23282227)
\curveto(169.01162216,254.18281998)(169.01162216,254.13282003)(169.00163574,254.08282227)
\curveto(168.96162221,253.94282022)(168.93162224,253.80782035)(168.91163574,253.67782227)
\curveto(168.90162227,253.54782061)(168.8716223,253.42782073)(168.82163574,253.31782227)
\curveto(168.68162249,252.96782119)(168.51662265,252.67282149)(168.32663574,252.43282227)
\curveto(168.13662303,252.20282196)(167.8666233,252.01782214)(167.51663574,251.87782227)
\curveto(167.43662373,251.84782231)(167.35162382,251.82782233)(167.26163574,251.81782227)
\curveto(167.171624,251.79782236)(167.08662408,251.77782238)(167.00663574,251.75782227)
\curveto(166.95662421,251.74782241)(166.90662426,251.74282242)(166.85663574,251.74282227)
\curveto(166.80662436,251.74282242)(166.75662441,251.73782242)(166.70663574,251.72782227)
\curveto(166.67662449,251.71782244)(166.62662454,251.71782244)(166.55663574,251.72782227)
\curveto(166.48662468,251.72782243)(166.43662473,251.73282243)(166.40663574,251.74282227)
\curveto(166.34662482,251.7628224)(166.28662488,251.77282239)(166.22663574,251.77282227)
\curveto(166.17662499,251.7628224)(166.12662504,251.76782239)(166.07663574,251.78782227)
\curveto(165.98662518,251.80782235)(165.89662527,251.83282233)(165.80663574,251.86282227)
\curveto(165.72662544,251.88282228)(165.64662552,251.91282225)(165.56663574,251.95282227)
\curveto(165.24662592,252.09282207)(164.99662617,252.28782187)(164.81663574,252.53782227)
\curveto(164.63662653,252.79782136)(164.48662668,253.10282106)(164.36663574,253.45282227)
\curveto(164.34662682,253.53282063)(164.33162684,253.61782054)(164.32163574,253.70782227)
\curveto(164.31162686,253.79782036)(164.29662687,253.88282028)(164.27663574,253.96282227)
\curveto(164.2666269,253.99282017)(164.26162691,254.02282014)(164.26163574,254.05282227)
\lineto(164.26163574,254.15782227)
\curveto(164.24162693,254.23781992)(164.23162694,254.31781984)(164.23163574,254.39782227)
\lineto(164.23163574,254.53282227)
\curveto(164.21162696,254.63281953)(164.21162696,254.73281943)(164.23163574,254.83282227)
\lineto(164.23163574,255.01282227)
\curveto(164.24162693,255.0628191)(164.24662692,255.10781905)(164.24663574,255.14782227)
\curveto(164.24662692,255.19781896)(164.25162692,255.24281892)(164.26163574,255.28282227)
\curveto(164.2716269,255.32281884)(164.27662689,255.3578188)(164.27663574,255.38782227)
\curveto(164.27662689,255.42781873)(164.28162689,255.46781869)(164.29163574,255.50782227)
\lineto(164.35163574,255.83782227)
\curveto(164.3716268,255.9578182)(164.40162677,256.06781809)(164.44163574,256.16782227)
\curveto(164.58162659,256.49781766)(164.74162643,256.77281739)(164.92163574,256.99282227)
\curveto(165.11162606,257.22281694)(165.3716258,257.40781675)(165.70163574,257.54782227)
\curveto(165.78162539,257.58781657)(165.8666253,257.61281655)(165.95663574,257.62282227)
\lineto(166.25663574,257.68282227)
\lineto(166.39163574,257.68282227)
\curveto(166.44162473,257.69281647)(166.49162468,257.69781646)(166.54163574,257.69782227)
\curveto(167.11162406,257.71781644)(167.5716236,257.61281655)(167.92163574,257.38282227)
\curveto(168.28162289,257.162817)(168.54662262,256.8628173)(168.71663574,256.48282227)
\curveto(168.7666224,256.38281778)(168.80662236,256.28281788)(168.83663574,256.18282227)
\curveto(168.8666223,256.08281808)(168.89662227,255.97781818)(168.92663574,255.86782227)
\curveto(168.93662223,255.82781833)(168.94162223,255.79281837)(168.94163574,255.76282227)
\curveto(168.94162223,255.74281842)(168.94662222,255.71281845)(168.95663574,255.67282227)
\curveto(168.97662219,255.60281856)(168.98662218,255.52781863)(168.98663574,255.44782227)
\curveto(168.98662218,255.36781879)(168.99662217,255.28781887)(169.01663574,255.20782227)
\curveto(169.01662215,255.157819)(169.01662215,255.11281905)(169.01663574,255.07282227)
\curveto(169.01662215,255.03281913)(169.02162215,254.98781917)(169.03163574,254.93782227)
\moveto(167.92163574,254.50282227)
\curveto(167.93162324,254.55281961)(167.93662323,254.62781953)(167.93663574,254.72782227)
\curveto(167.94662322,254.82781933)(167.94162323,254.90281926)(167.92163574,254.95282227)
\curveto(167.90162327,255.01281915)(167.89662327,255.06781909)(167.90663574,255.11782227)
\curveto(167.92662324,255.17781898)(167.92662324,255.23781892)(167.90663574,255.29782227)
\curveto(167.89662327,255.32781883)(167.89162328,255.3628188)(167.89163574,255.40282227)
\curveto(167.89162328,255.44281872)(167.88662328,255.48281868)(167.87663574,255.52282227)
\curveto(167.85662331,255.60281856)(167.83662333,255.67781848)(167.81663574,255.74782227)
\curveto(167.80662336,255.82781833)(167.79162338,255.90781825)(167.77163574,255.98782227)
\curveto(167.74162343,256.04781811)(167.71662345,256.10781805)(167.69663574,256.16782227)
\curveto(167.67662349,256.22781793)(167.64662352,256.28781787)(167.60663574,256.34782227)
\curveto(167.50662366,256.51781764)(167.37662379,256.65281751)(167.21663574,256.75282227)
\curveto(167.13662403,256.80281736)(167.04162413,256.83781732)(166.93163574,256.85782227)
\curveto(166.82162435,256.87781728)(166.69662447,256.88781727)(166.55663574,256.88782227)
\curveto(166.53662463,256.87781728)(166.51162466,256.87281729)(166.48163574,256.87282227)
\curveto(166.45162472,256.88281728)(166.42162475,256.88281728)(166.39163574,256.87282227)
\lineto(166.24163574,256.81282227)
\curveto(166.19162498,256.80281736)(166.14662502,256.78781737)(166.10663574,256.76782227)
\curveto(165.91662525,256.6578175)(165.7716254,256.51281765)(165.67163574,256.33282227)
\curveto(165.58162559,256.15281801)(165.50162567,255.94781821)(165.43163574,255.71782227)
\curveto(165.39162578,255.58781857)(165.3716258,255.45281871)(165.37163574,255.31282227)
\curveto(165.3716258,255.18281898)(165.36162581,255.03781912)(165.34163574,254.87782227)
\curveto(165.33162584,254.82781933)(165.32162585,254.76781939)(165.31163574,254.69782227)
\curveto(165.31162586,254.62781953)(165.32162585,254.56781959)(165.34163574,254.51782227)
\lineto(165.34163574,254.35282227)
\lineto(165.34163574,254.17282227)
\curveto(165.35162582,254.12282004)(165.36162581,254.06782009)(165.37163574,254.00782227)
\curveto(165.38162579,253.9578202)(165.38662578,253.90282026)(165.38663574,253.84282227)
\curveto(165.39662577,253.78282038)(165.41162576,253.72782043)(165.43163574,253.67782227)
\curveto(165.48162569,253.48782067)(165.54162563,253.31282085)(165.61163574,253.15282227)
\curveto(165.68162549,252.99282117)(165.78662538,252.8628213)(165.92663574,252.76282227)
\curveto(166.05662511,252.6628215)(166.19662497,252.59282157)(166.34663574,252.55282227)
\curveto(166.37662479,252.54282162)(166.40162477,252.53782162)(166.42163574,252.53782227)
\curveto(166.45162472,252.54782161)(166.48162469,252.54782161)(166.51163574,252.53782227)
\curveto(166.53162464,252.53782162)(166.56162461,252.53282163)(166.60163574,252.52282227)
\curveto(166.64162453,252.52282164)(166.67662449,252.52782163)(166.70663574,252.53782227)
\curveto(166.74662442,252.54782161)(166.78662438,252.55282161)(166.82663574,252.55282227)
\curveto(166.8666243,252.55282161)(166.90662426,252.5628216)(166.94663574,252.58282227)
\curveto(167.18662398,252.6628215)(167.38162379,252.79782136)(167.53163574,252.98782227)
\curveto(167.65162352,253.16782099)(167.74162343,253.37282079)(167.80163574,253.60282227)
\curveto(167.82162335,253.67282049)(167.83662333,253.74282042)(167.84663574,253.81282227)
\curveto(167.85662331,253.89282027)(167.8716233,253.97282019)(167.89163574,254.05282227)
\curveto(167.89162328,254.11282005)(167.89662327,254.15782)(167.90663574,254.18782227)
\curveto(167.90662326,254.20781995)(167.90662326,254.23281993)(167.90663574,254.26282227)
\curveto(167.90662326,254.30281986)(167.91162326,254.33281983)(167.92163574,254.35282227)
\lineto(167.92163574,254.50282227)
}
}
{
\newrgbcolor{curcolor}{0 0 0}
\pscustom[linestyle=none,fillstyle=solid,fillcolor=curcolor]
{
\newpath
\moveto(96.99407776,217.3178833)
\lineto(100.59407776,217.3178833)
\lineto(101.23907776,217.3178833)
\curveto(101.31907123,217.31787288)(101.39407115,217.31287288)(101.46407776,217.3028833)
\curveto(101.53407101,217.30287289)(101.59407095,217.2928729)(101.64407776,217.2728833)
\curveto(101.71407083,217.24287295)(101.76907078,217.18287301)(101.80907776,217.0928833)
\curveto(101.82907072,217.06287313)(101.83907071,217.02287317)(101.83907776,216.9728833)
\lineto(101.83907776,216.8378833)
\curveto(101.8490707,216.72787347)(101.8440707,216.62287357)(101.82407776,216.5228833)
\curveto(101.81407073,216.42287377)(101.77907077,216.35287384)(101.71907776,216.3128833)
\curveto(101.62907092,216.24287395)(101.49407105,216.20787399)(101.31407776,216.2078833)
\curveto(101.13407141,216.21787398)(100.96907158,216.22287397)(100.81907776,216.2228833)
\lineto(98.82407776,216.2228833)
\lineto(98.32907776,216.2228833)
\lineto(98.19407776,216.2228833)
\curveto(98.15407439,216.22287397)(98.11407443,216.21787398)(98.07407776,216.2078833)
\lineto(97.86407776,216.2078833)
\curveto(97.75407479,216.17787402)(97.67407487,216.13787406)(97.62407776,216.0878833)
\curveto(97.57407497,216.04787415)(97.53907501,215.9928742)(97.51907776,215.9228833)
\curveto(97.49907505,215.86287433)(97.48407506,215.7928744)(97.47407776,215.7128833)
\curveto(97.46407508,215.63287456)(97.4440751,215.54287465)(97.41407776,215.4428833)
\curveto(97.36407518,215.24287495)(97.32407522,215.03787516)(97.29407776,214.8278833)
\curveto(97.26407528,214.61787558)(97.22407532,214.41287578)(97.17407776,214.2128833)
\curveto(97.15407539,214.14287605)(97.1440754,214.07287612)(97.14407776,214.0028833)
\curveto(97.1440754,213.94287625)(97.13407541,213.87787632)(97.11407776,213.8078833)
\curveto(97.10407544,213.77787642)(97.09407545,213.73787646)(97.08407776,213.6878833)
\curveto(97.08407546,213.64787655)(97.08907546,213.60787659)(97.09907776,213.5678833)
\curveto(97.11907543,213.51787668)(97.1440754,213.47287672)(97.17407776,213.4328833)
\curveto(97.21407533,213.40287679)(97.27407527,213.3978768)(97.35407776,213.4178833)
\curveto(97.41407513,213.43787676)(97.47407507,213.46287673)(97.53407776,213.4928833)
\curveto(97.59407495,213.53287666)(97.65407489,213.56787663)(97.71407776,213.5978833)
\curveto(97.77407477,213.61787658)(97.82407472,213.63287656)(97.86407776,213.6428833)
\curveto(98.05407449,213.72287647)(98.25907429,213.77787642)(98.47907776,213.8078833)
\curveto(98.70907384,213.83787636)(98.93907361,213.84787635)(99.16907776,213.8378833)
\curveto(99.40907314,213.83787636)(99.63907291,213.81287638)(99.85907776,213.7628833)
\curveto(100.07907247,213.72287647)(100.27907227,213.66287653)(100.45907776,213.5828833)
\curveto(100.50907204,213.56287663)(100.55407199,213.54287665)(100.59407776,213.5228833)
\curveto(100.6440719,213.50287669)(100.69407185,213.47787672)(100.74407776,213.4478833)
\curveto(101.09407145,213.23787696)(101.37407117,213.00787719)(101.58407776,212.7578833)
\curveto(101.80407074,212.50787769)(101.99907055,212.18287801)(102.16907776,211.7828833)
\curveto(102.21907033,211.67287852)(102.25407029,211.56287863)(102.27407776,211.4528833)
\curveto(102.29407025,211.34287885)(102.31907023,211.22787897)(102.34907776,211.1078833)
\curveto(102.35907019,211.07787912)(102.36407018,211.03287916)(102.36407776,210.9728833)
\curveto(102.38407016,210.91287928)(102.39407015,210.84287935)(102.39407776,210.7628833)
\curveto(102.39407015,210.6928795)(102.40407014,210.62787957)(102.42407776,210.5678833)
\lineto(102.42407776,210.4028833)
\curveto(102.43407011,210.35287984)(102.43907011,210.28287991)(102.43907776,210.1928833)
\curveto(102.43907011,210.10288009)(102.42907012,210.03288016)(102.40907776,209.9828833)
\curveto(102.38907016,209.92288027)(102.38407016,209.86288033)(102.39407776,209.8028833)
\curveto(102.40407014,209.75288044)(102.39907015,209.70288049)(102.37907776,209.6528833)
\curveto(102.33907021,209.4928807)(102.30407024,209.34288085)(102.27407776,209.2028833)
\curveto(102.2440703,209.06288113)(102.19907035,208.92788127)(102.13907776,208.7978833)
\curveto(101.97907057,208.42788177)(101.75907079,208.0928821)(101.47907776,207.7928833)
\curveto(101.19907135,207.4928827)(100.87907167,207.26288293)(100.51907776,207.1028833)
\curveto(100.3490722,207.02288317)(100.1490724,206.94788325)(99.91907776,206.8778833)
\curveto(99.80907274,206.83788336)(99.69407285,206.81288338)(99.57407776,206.8028833)
\curveto(99.45407309,206.7928834)(99.33407321,206.77288342)(99.21407776,206.7428833)
\curveto(99.16407338,206.72288347)(99.10907344,206.72288347)(99.04907776,206.7428833)
\curveto(98.98907356,206.75288344)(98.92907362,206.74788345)(98.86907776,206.7278833)
\curveto(98.76907378,206.70788349)(98.66907388,206.70788349)(98.56907776,206.7278833)
\lineto(98.43407776,206.7278833)
\curveto(98.38407416,206.74788345)(98.32407422,206.75788344)(98.25407776,206.7578833)
\curveto(98.19407435,206.74788345)(98.13907441,206.75288344)(98.08907776,206.7728833)
\curveto(98.0490745,206.78288341)(98.01407453,206.78788341)(97.98407776,206.7878833)
\curveto(97.95407459,206.78788341)(97.91907463,206.7928834)(97.87907776,206.8028833)
\lineto(97.60907776,206.8628833)
\curveto(97.51907503,206.88288331)(97.43407511,206.91288328)(97.35407776,206.9528833)
\curveto(97.01407553,207.0928831)(96.72407582,207.24788295)(96.48407776,207.4178833)
\curveto(96.2440763,207.5978826)(96.02407652,207.82788237)(95.82407776,208.1078833)
\curveto(95.67407687,208.33788186)(95.55907699,208.57788162)(95.47907776,208.8278833)
\curveto(95.45907709,208.87788132)(95.4490771,208.92288127)(95.44907776,208.9628833)
\curveto(95.4490771,209.01288118)(95.43907711,209.06288113)(95.41907776,209.1128833)
\curveto(95.39907715,209.17288102)(95.38407716,209.25288094)(95.37407776,209.3528833)
\curveto(95.37407717,209.45288074)(95.39407715,209.52788067)(95.43407776,209.5778833)
\curveto(95.48407706,209.65788054)(95.56407698,209.70288049)(95.67407776,209.7128833)
\curveto(95.78407676,209.72288047)(95.89907665,209.72788047)(96.01907776,209.7278833)
\lineto(96.18407776,209.7278833)
\curveto(96.2440763,209.72788047)(96.29907625,209.71788048)(96.34907776,209.6978833)
\curveto(96.43907611,209.67788052)(96.50907604,209.63788056)(96.55907776,209.5778833)
\curveto(96.62907592,209.48788071)(96.67407587,209.37788082)(96.69407776,209.2478833)
\curveto(96.72407582,209.12788107)(96.76907578,209.02288117)(96.82907776,208.9328833)
\curveto(97.01907553,208.5928816)(97.27907527,208.32288187)(97.60907776,208.1228833)
\curveto(97.70907484,208.06288213)(97.81407473,208.01288218)(97.92407776,207.9728833)
\curveto(98.0440745,207.94288225)(98.16407438,207.90788229)(98.28407776,207.8678833)
\curveto(98.45407409,207.81788238)(98.65907389,207.7978824)(98.89907776,207.8078833)
\curveto(99.1490734,207.82788237)(99.3490732,207.86288233)(99.49907776,207.9128833)
\curveto(99.86907268,208.03288216)(100.15907239,208.192882)(100.36907776,208.3928833)
\curveto(100.58907196,208.60288159)(100.76907178,208.88288131)(100.90907776,209.2328833)
\curveto(100.95907159,209.33288086)(100.98907156,209.43788076)(100.99907776,209.5478833)
\curveto(101.01907153,209.65788054)(101.0440715,209.77288042)(101.07407776,209.8928833)
\lineto(101.07407776,209.9978833)
\curveto(101.08407146,210.03788016)(101.08907146,210.07788012)(101.08907776,210.1178833)
\curveto(101.09907145,210.14788005)(101.09907145,210.18288001)(101.08907776,210.2228833)
\lineto(101.08907776,210.3428833)
\curveto(101.08907146,210.60287959)(101.05907149,210.84787935)(100.99907776,211.0778833)
\curveto(100.88907166,211.42787877)(100.73407181,211.72287847)(100.53407776,211.9628833)
\curveto(100.33407221,212.21287798)(100.07407247,212.40787779)(99.75407776,212.5478833)
\lineto(99.57407776,212.6078833)
\curveto(99.52407302,212.62787757)(99.46407308,212.64787755)(99.39407776,212.6678833)
\curveto(99.3440732,212.68787751)(99.28407326,212.6978775)(99.21407776,212.6978833)
\curveto(99.15407339,212.70787749)(99.08907346,212.72287747)(99.01907776,212.7428833)
\lineto(98.86907776,212.7428833)
\curveto(98.82907372,212.76287743)(98.77407377,212.77287742)(98.70407776,212.7728833)
\curveto(98.6440739,212.77287742)(98.58907396,212.76287743)(98.53907776,212.7428833)
\lineto(98.43407776,212.7428833)
\curveto(98.40407414,212.74287745)(98.36907418,212.73787746)(98.32907776,212.7278833)
\lineto(98.08907776,212.6678833)
\curveto(98.00907454,212.65787754)(97.92907462,212.63787756)(97.84907776,212.6078833)
\curveto(97.60907494,212.50787769)(97.37907517,212.37287782)(97.15907776,212.2028833)
\curveto(97.06907548,212.13287806)(96.98407556,212.05787814)(96.90407776,211.9778833)
\curveto(96.82407572,211.90787829)(96.72407582,211.85287834)(96.60407776,211.8128833)
\curveto(96.51407603,211.78287841)(96.37407617,211.77287842)(96.18407776,211.7828833)
\curveto(96.00407654,211.7928784)(95.88407666,211.81787838)(95.82407776,211.8578833)
\curveto(95.77407677,211.8978783)(95.73407681,211.95787824)(95.70407776,212.0378833)
\curveto(95.68407686,212.11787808)(95.68407686,212.20287799)(95.70407776,212.2928833)
\curveto(95.73407681,212.41287778)(95.75407679,212.53287766)(95.76407776,212.6528833)
\curveto(95.78407676,212.78287741)(95.80907674,212.90787729)(95.83907776,213.0278833)
\curveto(95.85907669,213.06787713)(95.86407668,213.10287709)(95.85407776,213.1328833)
\curveto(95.85407669,213.17287702)(95.86407668,213.21787698)(95.88407776,213.2678833)
\curveto(95.90407664,213.35787684)(95.91907663,213.44787675)(95.92907776,213.5378833)
\curveto(95.93907661,213.63787656)(95.95907659,213.73287646)(95.98907776,213.8228833)
\curveto(95.99907655,213.88287631)(96.00407654,213.94287625)(96.00407776,214.0028833)
\curveto(96.01407653,214.06287613)(96.02907652,214.12287607)(96.04907776,214.1828833)
\curveto(96.09907645,214.38287581)(96.13407641,214.58787561)(96.15407776,214.7978833)
\curveto(96.18407636,215.01787518)(96.22407632,215.22787497)(96.27407776,215.4278833)
\curveto(96.30407624,215.52787467)(96.32407622,215.62787457)(96.33407776,215.7278833)
\curveto(96.3440762,215.82787437)(96.35907619,215.92787427)(96.37907776,216.0278833)
\curveto(96.38907616,216.05787414)(96.39407615,216.0978741)(96.39407776,216.1478833)
\curveto(96.42407612,216.25787394)(96.4440761,216.36287383)(96.45407776,216.4628833)
\curveto(96.47407607,216.57287362)(96.49907605,216.68287351)(96.52907776,216.7928833)
\curveto(96.549076,216.87287332)(96.56407598,216.94287325)(96.57407776,217.0028833)
\curveto(96.58407596,217.07287312)(96.60907594,217.13287306)(96.64907776,217.1828833)
\curveto(96.66907588,217.21287298)(96.69907585,217.23287296)(96.73907776,217.2428833)
\curveto(96.77907577,217.26287293)(96.82407572,217.28287291)(96.87407776,217.3028833)
\curveto(96.93407561,217.30287289)(96.97407557,217.30787289)(96.99407776,217.3178833)
}
}
{
\newrgbcolor{curcolor}{0 0 0}
\pscustom[linestyle=none,fillstyle=solid,fillcolor=curcolor]
{
\newpath
\moveto(104.78868713,208.5428833)
\lineto(105.08868713,208.5428833)
\curveto(105.19868507,208.55288164)(105.30368497,208.55288164)(105.40368713,208.5428833)
\curveto(105.51368476,208.54288165)(105.61368466,208.53288166)(105.70368713,208.5128833)
\curveto(105.79368448,208.50288169)(105.86368441,208.47788172)(105.91368713,208.4378833)
\curveto(105.93368434,208.41788178)(105.94868432,208.38788181)(105.95868713,208.3478833)
\curveto(105.97868429,208.30788189)(105.99868427,208.26288193)(106.01868713,208.2128833)
\lineto(106.01868713,208.1378833)
\curveto(106.02868424,208.08788211)(106.02868424,208.03288216)(106.01868713,207.9728833)
\lineto(106.01868713,207.8228833)
\lineto(106.01868713,207.3428833)
\curveto(106.01868425,207.17288302)(105.97868429,207.05288314)(105.89868713,206.9828833)
\curveto(105.82868444,206.93288326)(105.73868453,206.90788329)(105.62868713,206.9078833)
\lineto(105.29868713,206.9078833)
\lineto(104.84868713,206.9078833)
\curveto(104.69868557,206.90788329)(104.58368569,206.93788326)(104.50368713,206.9978833)
\curveto(104.46368581,207.02788317)(104.43368584,207.07788312)(104.41368713,207.1478833)
\curveto(104.39368588,207.22788297)(104.37868589,207.31288288)(104.36868713,207.4028833)
\lineto(104.36868713,207.6878833)
\curveto(104.37868589,207.78788241)(104.38368589,207.87288232)(104.38368713,207.9428833)
\lineto(104.38368713,208.1378833)
\curveto(104.38368589,208.197882)(104.39368588,208.25288194)(104.41368713,208.3028833)
\curveto(104.45368582,208.41288178)(104.52368575,208.48288171)(104.62368713,208.5128833)
\curveto(104.65368562,208.51288168)(104.70868556,208.52288167)(104.78868713,208.5428833)
}
}
{
\newrgbcolor{curcolor}{0 0 0}
\pscustom[linestyle=none,fillstyle=solid,fillcolor=curcolor]
{
\newpath
\moveto(111.10884338,217.5128833)
\curveto(112.73883794,217.54287265)(113.78883689,216.98787321)(114.25884338,215.8478833)
\curveto(114.35883632,215.61787458)(114.42383626,215.32787487)(114.45384338,214.9778833)
\curveto(114.49383619,214.63787556)(114.46883621,214.32787587)(114.37884338,214.0478833)
\curveto(114.28883639,213.78787641)(114.16883651,213.56287663)(114.01884338,213.3728833)
\curveto(113.99883668,213.33287686)(113.97383671,213.2978769)(113.94384338,213.2678833)
\curveto(113.91383677,213.24787695)(113.88883679,213.22287697)(113.86884338,213.1928833)
\lineto(113.77884338,213.0728833)
\curveto(113.74883693,213.04287715)(113.71383697,213.01787718)(113.67384338,212.9978833)
\curveto(113.62383706,212.94787725)(113.56883711,212.90287729)(113.50884338,212.8628833)
\curveto(113.45883722,212.82287737)(113.41383727,212.77287742)(113.37384338,212.7128833)
\curveto(113.33383735,212.67287752)(113.31883736,212.62287757)(113.32884338,212.5628833)
\curveto(113.33883734,212.51287768)(113.36883731,212.46787773)(113.41884338,212.4278833)
\curveto(113.46883721,212.38787781)(113.52383716,212.34787785)(113.58384338,212.3078833)
\curveto(113.65383703,212.27787792)(113.71883696,212.24787795)(113.77884338,212.2178833)
\curveto(113.83883684,212.18787801)(113.88883679,212.15787804)(113.92884338,212.1278833)
\curveto(114.24883643,211.90787829)(114.50383618,211.5978786)(114.69384338,211.1978833)
\curveto(114.73383595,211.10787909)(114.76383592,211.01287918)(114.78384338,210.9128833)
\curveto(114.81383587,210.82287937)(114.83883584,210.73287946)(114.85884338,210.6428833)
\curveto(114.86883581,210.5928796)(114.87383581,210.54287965)(114.87384338,210.4928833)
\curveto(114.8838358,210.45287974)(114.89383579,210.40787979)(114.90384338,210.3578833)
\curveto(114.91383577,210.30787989)(114.91383577,210.25787994)(114.90384338,210.2078833)
\curveto(114.89383579,210.15788004)(114.89883578,210.10788009)(114.91884338,210.0578833)
\curveto(114.92883575,210.00788019)(114.93383575,209.94788025)(114.93384338,209.8778833)
\curveto(114.93383575,209.80788039)(114.92383576,209.74788045)(114.90384338,209.6978833)
\lineto(114.90384338,209.4728833)
\lineto(114.84384338,209.2328833)
\curveto(114.83383585,209.16288103)(114.81883586,209.0928811)(114.79884338,209.0228833)
\curveto(114.76883591,208.93288126)(114.73883594,208.84788135)(114.70884338,208.7678833)
\curveto(114.68883599,208.68788151)(114.65883602,208.60788159)(114.61884338,208.5278833)
\curveto(114.59883608,208.46788173)(114.56883611,208.40788179)(114.52884338,208.3478833)
\curveto(114.49883618,208.2978819)(114.46383622,208.24788195)(114.42384338,208.1978833)
\curveto(114.22383646,207.88788231)(113.97383671,207.62788257)(113.67384338,207.4178833)
\curveto(113.37383731,207.21788298)(113.02883765,207.05288314)(112.63884338,206.9228833)
\curveto(112.51883816,206.88288331)(112.38883829,206.85788334)(112.24884338,206.8478833)
\curveto(112.11883856,206.82788337)(111.9838387,206.80288339)(111.84384338,206.7728833)
\curveto(111.77383891,206.76288343)(111.70383898,206.75788344)(111.63384338,206.7578833)
\curveto(111.57383911,206.75788344)(111.50883917,206.75288344)(111.43884338,206.7428833)
\curveto(111.39883928,206.73288346)(111.33883934,206.72788347)(111.25884338,206.7278833)
\curveto(111.18883949,206.72788347)(111.13883954,206.73288346)(111.10884338,206.7428833)
\curveto(111.05883962,206.75288344)(111.01383967,206.75788344)(110.97384338,206.7578833)
\lineto(110.85384338,206.7578833)
\curveto(110.75383993,206.77788342)(110.65384003,206.7928834)(110.55384338,206.8028833)
\curveto(110.45384023,206.81288338)(110.35884032,206.82788337)(110.26884338,206.8478833)
\curveto(110.15884052,206.87788332)(110.04884063,206.90288329)(109.93884338,206.9228833)
\curveto(109.83884084,206.95288324)(109.73384095,206.9928832)(109.62384338,207.0428833)
\curveto(109.25384143,207.20288299)(108.93884174,207.40288279)(108.67884338,207.6428833)
\curveto(108.41884226,207.8928823)(108.20884247,208.20288199)(108.04884338,208.5728833)
\curveto(108.00884267,208.66288153)(107.97384271,208.75788144)(107.94384338,208.8578833)
\curveto(107.91384277,208.95788124)(107.8838428,209.06288113)(107.85384338,209.1728833)
\curveto(107.83384285,209.22288097)(107.82384286,209.27288092)(107.82384338,209.3228833)
\curveto(107.82384286,209.38288081)(107.81384287,209.44288075)(107.79384338,209.5028833)
\curveto(107.77384291,209.56288063)(107.76384292,209.64288055)(107.76384338,209.7428833)
\curveto(107.76384292,209.84288035)(107.7788429,209.91788028)(107.80884338,209.9678833)
\curveto(107.81884286,209.9978802)(107.83384285,210.02288017)(107.85384338,210.0428833)
\lineto(107.91384338,210.1028833)
\curveto(107.95384273,210.12288007)(108.01384267,210.13788006)(108.09384338,210.1478833)
\curveto(108.1838425,210.15788004)(108.27384241,210.16288003)(108.36384338,210.1628833)
\curveto(108.45384223,210.16288003)(108.53884214,210.15788004)(108.61884338,210.1478833)
\curveto(108.70884197,210.13788006)(108.77384191,210.12788007)(108.81384338,210.1178833)
\curveto(108.83384185,210.0978801)(108.85384183,210.08288011)(108.87384338,210.0728833)
\curveto(108.89384179,210.07288012)(108.91384177,210.06288013)(108.93384338,210.0428833)
\curveto(109.00384168,209.95288024)(109.04384164,209.83788036)(109.05384338,209.6978833)
\curveto(109.07384161,209.55788064)(109.10384158,209.43288076)(109.14384338,209.3228833)
\lineto(109.29384338,208.9628833)
\curveto(109.34384134,208.85288134)(109.40884127,208.74788145)(109.48884338,208.6478833)
\curveto(109.50884117,208.61788158)(109.52884115,208.5928816)(109.54884338,208.5728833)
\curveto(109.5788411,208.55288164)(109.60384108,208.52788167)(109.62384338,208.4978833)
\curveto(109.66384102,208.43788176)(109.69884098,208.3928818)(109.72884338,208.3628833)
\curveto(109.76884091,208.33288186)(109.80384088,208.30288189)(109.83384338,208.2728833)
\curveto(109.87384081,208.24288195)(109.91884076,208.21288198)(109.96884338,208.1828833)
\curveto(110.05884062,208.12288207)(110.15384053,208.07288212)(110.25384338,208.0328833)
\lineto(110.58384338,207.9128833)
\curveto(110.73383995,207.86288233)(110.93383975,207.83288236)(111.18384338,207.8228833)
\curveto(111.43383925,207.81288238)(111.64383904,207.83288236)(111.81384338,207.8828833)
\curveto(111.89383879,207.90288229)(111.96383872,207.91788228)(112.02384338,207.9278833)
\lineto(112.23384338,207.9878833)
\curveto(112.51383817,208.10788209)(112.75383793,208.25788194)(112.95384338,208.4378833)
\curveto(113.16383752,208.61788158)(113.32883735,208.84788135)(113.44884338,209.1278833)
\curveto(113.4788372,209.197881)(113.49883718,209.26788093)(113.50884338,209.3378833)
\lineto(113.56884338,209.5778833)
\curveto(113.60883707,209.71788048)(113.61883706,209.87788032)(113.59884338,210.0578833)
\curveto(113.5788371,210.24787995)(113.54883713,210.3978798)(113.50884338,210.5078833)
\curveto(113.3788373,210.88787931)(113.19383749,211.17787902)(112.95384338,211.3778833)
\curveto(112.72383796,211.57787862)(112.41383827,211.73787846)(112.02384338,211.8578833)
\curveto(111.91383877,211.88787831)(111.79383889,211.90787829)(111.66384338,211.9178833)
\curveto(111.54383914,211.92787827)(111.41883926,211.93287826)(111.28884338,211.9328833)
\curveto(111.12883955,211.93287826)(110.98883969,211.93787826)(110.86884338,211.9478833)
\curveto(110.74883993,211.95787824)(110.66384002,212.01787818)(110.61384338,212.1278833)
\curveto(110.59384009,212.15787804)(110.5838401,212.192878)(110.58384338,212.2328833)
\lineto(110.58384338,212.3678833)
\curveto(110.57384011,212.46787773)(110.57384011,212.56287763)(110.58384338,212.6528833)
\curveto(110.60384008,212.74287745)(110.64384004,212.80787739)(110.70384338,212.8478833)
\curveto(110.74383994,212.87787732)(110.7838399,212.8978773)(110.82384338,212.9078833)
\curveto(110.87383981,212.91787728)(110.92883975,212.92787727)(110.98884338,212.9378833)
\curveto(111.00883967,212.94787725)(111.03383965,212.94787725)(111.06384338,212.9378833)
\curveto(111.09383959,212.93787726)(111.11883956,212.94287725)(111.13884338,212.9528833)
\lineto(111.27384338,212.9528833)
\curveto(111.3838393,212.97287722)(111.4838392,212.98287721)(111.57384338,212.9828833)
\curveto(111.67383901,212.9928772)(111.76883891,213.01287718)(111.85884338,213.0428833)
\curveto(112.1788385,213.15287704)(112.43383825,213.2978769)(112.62384338,213.4778833)
\curveto(112.81383787,213.65787654)(112.96383772,213.90787629)(113.07384338,214.2278833)
\curveto(113.10383758,214.32787587)(113.12383756,214.45287574)(113.13384338,214.6028833)
\curveto(113.15383753,214.76287543)(113.14883753,214.90787529)(113.11884338,215.0378833)
\curveto(113.09883758,215.10787509)(113.0788376,215.17287502)(113.05884338,215.2328833)
\curveto(113.04883763,215.30287489)(113.02883765,215.36787483)(112.99884338,215.4278833)
\curveto(112.89883778,215.66787453)(112.75383793,215.85787434)(112.56384338,215.9978833)
\curveto(112.37383831,216.13787406)(112.14883853,216.24787395)(111.88884338,216.3278833)
\curveto(111.82883885,216.34787385)(111.76883891,216.35787384)(111.70884338,216.3578833)
\curveto(111.64883903,216.35787384)(111.5838391,216.36787383)(111.51384338,216.3878833)
\curveto(111.43383925,216.40787379)(111.33883934,216.41787378)(111.22884338,216.4178833)
\curveto(111.11883956,216.41787378)(111.02383966,216.40787379)(110.94384338,216.3878833)
\curveto(110.89383979,216.36787383)(110.84383984,216.35787384)(110.79384338,216.3578833)
\curveto(110.75383993,216.35787384)(110.70883997,216.34787385)(110.65884338,216.3278833)
\curveto(110.4788402,216.27787392)(110.30884037,216.20287399)(110.14884338,216.1028833)
\curveto(109.99884068,216.01287418)(109.86884081,215.8978743)(109.75884338,215.7578833)
\curveto(109.66884101,215.63787456)(109.58884109,215.50787469)(109.51884338,215.3678833)
\curveto(109.44884123,215.22787497)(109.3838413,215.07287512)(109.32384338,214.9028833)
\curveto(109.29384139,214.7928754)(109.27384141,214.67287552)(109.26384338,214.5428833)
\curveto(109.25384143,214.42287577)(109.21884146,214.32287587)(109.15884338,214.2428833)
\curveto(109.13884154,214.20287599)(109.0788416,214.16287603)(108.97884338,214.1228833)
\curveto(108.93884174,214.11287608)(108.8788418,214.10287609)(108.79884338,214.0928833)
\lineto(108.54384338,214.0928833)
\curveto(108.45384223,214.10287609)(108.36884231,214.11287608)(108.28884338,214.1228833)
\curveto(108.21884246,214.13287606)(108.16884251,214.14787605)(108.13884338,214.1678833)
\curveto(108.09884258,214.197876)(108.06384262,214.25287594)(108.03384338,214.3328833)
\curveto(108.00384268,214.41287578)(107.99884268,214.4978757)(108.01884338,214.5878833)
\curveto(108.02884265,214.63787556)(108.03384265,214.68787551)(108.03384338,214.7378833)
\lineto(108.06384338,214.9178833)
\curveto(108.09384259,215.01787518)(108.11884256,215.11787508)(108.13884338,215.2178833)
\curveto(108.16884251,215.31787488)(108.20384248,215.40787479)(108.24384338,215.4878833)
\curveto(108.29384239,215.5978746)(108.33884234,215.70287449)(108.37884338,215.8028833)
\curveto(108.41884226,215.91287428)(108.46884221,216.01787418)(108.52884338,216.1178833)
\curveto(108.85884182,216.65787354)(109.32884135,217.05287314)(109.93884338,217.3028833)
\curveto(110.05884062,217.35287284)(110.1838405,217.38787281)(110.31384338,217.4078833)
\curveto(110.45384023,217.42787277)(110.59384009,217.45287274)(110.73384338,217.4828833)
\curveto(110.79383989,217.4928727)(110.85383983,217.4978727)(110.91384338,217.4978833)
\curveto(110.9838397,217.4978727)(111.04883963,217.50287269)(111.10884338,217.5128833)
}
}
{
\newrgbcolor{curcolor}{0 0 0}
\pscustom[linestyle=none,fillstyle=solid,fillcolor=curcolor]
{
\newpath
\moveto(126.14845276,215.4278833)
\curveto(125.94844246,215.13787506)(125.73844267,214.85287534)(125.51845276,214.5728833)
\curveto(125.3084431,214.2928759)(125.1034433,214.00787619)(124.90345276,213.7178833)
\curveto(124.3034441,212.86787733)(123.69844471,212.02787817)(123.08845276,211.1978833)
\curveto(122.47844593,210.37787982)(121.87344653,209.54288065)(121.27345276,208.6928833)
\lineto(120.76345276,207.9728833)
\lineto(120.25345276,207.2828833)
\curveto(120.17344823,207.17288302)(120.09344831,207.05788314)(120.01345276,206.9378833)
\curveto(119.93344847,206.81788338)(119.83844857,206.72288347)(119.72845276,206.6528833)
\curveto(119.68844872,206.63288356)(119.62344878,206.61788358)(119.53345276,206.6078833)
\curveto(119.45344895,206.58788361)(119.36344904,206.57788362)(119.26345276,206.5778833)
\curveto(119.16344924,206.57788362)(119.06844934,206.58288361)(118.97845276,206.5928833)
\curveto(118.89844951,206.60288359)(118.83844957,206.62288357)(118.79845276,206.6528833)
\curveto(118.76844964,206.67288352)(118.74344966,206.70788349)(118.72345276,206.7578833)
\curveto(118.71344969,206.7978834)(118.71844969,206.84288335)(118.73845276,206.8928833)
\curveto(118.77844963,206.97288322)(118.82344958,207.04788315)(118.87345276,207.1178833)
\curveto(118.93344947,207.197883)(118.98844942,207.27788292)(119.03845276,207.3578833)
\curveto(119.27844913,207.6978825)(119.52344888,208.03288216)(119.77345276,208.3628833)
\curveto(120.02344838,208.6928815)(120.26344814,209.02788117)(120.49345276,209.3678833)
\curveto(120.65344775,209.58788061)(120.81344759,209.80288039)(120.97345276,210.0128833)
\curveto(121.13344727,210.22287997)(121.29344711,210.43787976)(121.45345276,210.6578833)
\curveto(121.81344659,211.17787902)(122.17844623,211.68787851)(122.54845276,212.1878833)
\curveto(122.91844549,212.68787751)(123.28844512,213.197877)(123.65845276,213.7178833)
\curveto(123.79844461,213.91787628)(123.93844447,214.11287608)(124.07845276,214.3028833)
\curveto(124.22844418,214.4928757)(124.37344403,214.68787551)(124.51345276,214.8878833)
\curveto(124.72344368,215.18787501)(124.93844347,215.48787471)(125.15845276,215.7878833)
\lineto(125.81845276,216.6878833)
\lineto(125.99845276,216.9578833)
\lineto(126.20845276,217.2278833)
\lineto(126.32845276,217.4078833)
\curveto(126.37844203,217.46787273)(126.42844198,217.52287267)(126.47845276,217.5728833)
\curveto(126.54844186,217.62287257)(126.62344178,217.65787254)(126.70345276,217.6778833)
\curveto(126.72344168,217.68787251)(126.74844166,217.68787251)(126.77845276,217.6778833)
\curveto(126.81844159,217.67787252)(126.84844156,217.68787251)(126.86845276,217.7078833)
\curveto(126.98844142,217.70787249)(127.12344128,217.70287249)(127.27345276,217.6928833)
\curveto(127.42344098,217.6928725)(127.51344089,217.64787255)(127.54345276,217.5578833)
\curveto(127.56344084,217.52787267)(127.56844084,217.4928727)(127.55845276,217.4528833)
\curveto(127.54844086,217.41287278)(127.53344087,217.38287281)(127.51345276,217.3628833)
\curveto(127.47344093,217.28287291)(127.43344097,217.21287298)(127.39345276,217.1528833)
\curveto(127.35344105,217.0928731)(127.3084411,217.03287316)(127.25845276,216.9728833)
\lineto(126.68845276,216.1928833)
\curveto(126.5084419,215.94287425)(126.32844208,215.68787451)(126.14845276,215.4278833)
\moveto(119.29345276,211.5278833)
\curveto(119.24344916,211.54787865)(119.19344921,211.55287864)(119.14345276,211.5428833)
\curveto(119.09344931,211.53287866)(119.04344936,211.53787866)(118.99345276,211.5578833)
\curveto(118.88344952,211.57787862)(118.77844963,211.5978786)(118.67845276,211.6178833)
\curveto(118.58844982,211.64787855)(118.49344991,211.68787851)(118.39345276,211.7378833)
\curveto(118.06345034,211.87787832)(117.8084506,212.07287812)(117.62845276,212.3228833)
\curveto(117.44845096,212.58287761)(117.3034511,212.8928773)(117.19345276,213.2528833)
\curveto(117.16345124,213.33287686)(117.14345126,213.41287678)(117.13345276,213.4928833)
\curveto(117.12345128,213.58287661)(117.1084513,213.66787653)(117.08845276,213.7478833)
\curveto(117.07845133,213.7978764)(117.07345133,213.86287633)(117.07345276,213.9428833)
\curveto(117.06345134,213.97287622)(117.05845135,214.00287619)(117.05845276,214.0328833)
\curveto(117.05845135,214.07287612)(117.05345135,214.10787609)(117.04345276,214.1378833)
\lineto(117.04345276,214.2878833)
\curveto(117.03345137,214.33787586)(117.02845138,214.3978758)(117.02845276,214.4678833)
\curveto(117.02845138,214.54787565)(117.03345137,214.61287558)(117.04345276,214.6628833)
\lineto(117.04345276,214.8278833)
\curveto(117.06345134,214.87787532)(117.06845134,214.92287527)(117.05845276,214.9628833)
\curveto(117.05845135,215.01287518)(117.06345134,215.05787514)(117.07345276,215.0978833)
\curveto(117.08345132,215.13787506)(117.08845132,215.17287502)(117.08845276,215.2028833)
\curveto(117.08845132,215.24287495)(117.09345131,215.28287491)(117.10345276,215.3228833)
\curveto(117.13345127,215.43287476)(117.15345125,215.54287465)(117.16345276,215.6528833)
\curveto(117.18345122,215.77287442)(117.21845119,215.88787431)(117.26845276,215.9978833)
\curveto(117.408451,216.33787386)(117.56845084,216.61287358)(117.74845276,216.8228833)
\curveto(117.93845047,217.04287315)(118.2084502,217.22287297)(118.55845276,217.3628833)
\curveto(118.63844977,217.3928728)(118.72344968,217.41287278)(118.81345276,217.4228833)
\curveto(118.9034495,217.44287275)(118.99844941,217.46287273)(119.09845276,217.4828833)
\curveto(119.12844928,217.4928727)(119.18344922,217.4928727)(119.26345276,217.4828833)
\curveto(119.34344906,217.48287271)(119.39344901,217.4928727)(119.41345276,217.5128833)
\curveto(119.97344843,217.52287267)(120.42344798,217.41287278)(120.76345276,217.1828833)
\curveto(121.11344729,216.95287324)(121.37344703,216.64787355)(121.54345276,216.2678833)
\curveto(121.58344682,216.17787402)(121.61844679,216.08287411)(121.64845276,215.9828833)
\curveto(121.67844673,215.88287431)(121.7034467,215.78287441)(121.72345276,215.6828833)
\curveto(121.74344666,215.65287454)(121.74844666,215.62287457)(121.73845276,215.5928833)
\curveto(121.73844667,215.56287463)(121.74344666,215.53287466)(121.75345276,215.5028833)
\curveto(121.78344662,215.3928748)(121.8034466,215.26787493)(121.81345276,215.1278833)
\curveto(121.82344658,214.9978752)(121.83344657,214.86287533)(121.84345276,214.7228833)
\lineto(121.84345276,214.5578833)
\curveto(121.85344655,214.4978757)(121.85344655,214.44287575)(121.84345276,214.3928833)
\curveto(121.83344657,214.34287585)(121.82844658,214.2928759)(121.82845276,214.2428833)
\lineto(121.82845276,214.1078833)
\curveto(121.81844659,214.06787613)(121.81344659,214.02787617)(121.81345276,213.9878833)
\curveto(121.82344658,213.94787625)(121.81844659,213.90287629)(121.79845276,213.8528833)
\curveto(121.77844663,213.74287645)(121.75844665,213.63787656)(121.73845276,213.5378833)
\curveto(121.72844668,213.43787676)(121.7084467,213.33787686)(121.67845276,213.2378833)
\curveto(121.54844686,212.87787732)(121.38344702,212.56287763)(121.18345276,212.2928833)
\curveto(120.98344742,212.02287817)(120.7084477,211.81787838)(120.35845276,211.6778833)
\curveto(120.27844813,211.64787855)(120.19344821,211.62287857)(120.10345276,211.6028833)
\lineto(119.83345276,211.5428833)
\curveto(119.78344862,211.53287866)(119.73844867,211.52787867)(119.69845276,211.5278833)
\curveto(119.65844875,211.53787866)(119.61844879,211.53787866)(119.57845276,211.5278833)
\curveto(119.47844893,211.50787869)(119.38344902,211.50787869)(119.29345276,211.5278833)
\moveto(118.45345276,212.9228833)
\curveto(118.49344991,212.85287734)(118.53344987,212.78787741)(118.57345276,212.7278833)
\curveto(118.61344979,212.67787752)(118.66344974,212.62787757)(118.72345276,212.5778833)
\lineto(118.87345276,212.4578833)
\curveto(118.93344947,212.42787777)(118.99844941,212.40287779)(119.06845276,212.3828833)
\curveto(119.1084493,212.36287783)(119.14344926,212.35287784)(119.17345276,212.3528833)
\curveto(119.21344919,212.36287783)(119.25344915,212.35787784)(119.29345276,212.3378833)
\curveto(119.32344908,212.33787786)(119.36344904,212.33287786)(119.41345276,212.3228833)
\curveto(119.46344894,212.32287787)(119.5034489,212.32787787)(119.53345276,212.3378833)
\lineto(119.75845276,212.3828833)
\curveto(120.0084484,212.46287773)(120.19344821,212.58787761)(120.31345276,212.7578833)
\curveto(120.39344801,212.85787734)(120.46344794,212.98787721)(120.52345276,213.1478833)
\curveto(120.6034478,213.32787687)(120.66344774,213.55287664)(120.70345276,213.8228833)
\curveto(120.74344766,214.10287609)(120.75844765,214.38287581)(120.74845276,214.6628833)
\curveto(120.73844767,214.95287524)(120.7084477,215.22787497)(120.65845276,215.4878833)
\curveto(120.6084478,215.74787445)(120.53344787,215.95787424)(120.43345276,216.1178833)
\curveto(120.31344809,216.31787388)(120.16344824,216.46787373)(119.98345276,216.5678833)
\curveto(119.9034485,216.61787358)(119.81344859,216.64787355)(119.71345276,216.6578833)
\curveto(119.61344879,216.67787352)(119.5084489,216.68787351)(119.39845276,216.6878833)
\curveto(119.37844903,216.67787352)(119.35344905,216.67287352)(119.32345276,216.6728833)
\curveto(119.3034491,216.68287351)(119.28344912,216.68287351)(119.26345276,216.6728833)
\curveto(119.21344919,216.66287353)(119.16844924,216.65287354)(119.12845276,216.6428833)
\curveto(119.08844932,216.64287355)(119.04844936,216.63287356)(119.00845276,216.6128833)
\curveto(118.82844958,216.53287366)(118.67844973,216.41287378)(118.55845276,216.2528833)
\curveto(118.44844996,216.0928741)(118.35845005,215.91287428)(118.28845276,215.7128833)
\curveto(118.22845018,215.52287467)(118.18345022,215.2978749)(118.15345276,215.0378833)
\curveto(118.13345027,214.77787542)(118.12845028,214.51287568)(118.13845276,214.2428833)
\curveto(118.14845026,213.98287621)(118.17845023,213.73287646)(118.22845276,213.4928833)
\curveto(118.28845012,213.26287693)(118.36345004,213.07287712)(118.45345276,212.9228833)
\moveto(129.25345276,209.9378833)
\curveto(129.26343914,209.88788031)(129.26843914,209.7978804)(129.26845276,209.6678833)
\curveto(129.26843914,209.53788066)(129.25843915,209.44788075)(129.23845276,209.3978833)
\curveto(129.21843919,209.34788085)(129.21343919,209.2928809)(129.22345276,209.2328833)
\curveto(129.23343917,209.18288101)(129.23343917,209.13288106)(129.22345276,209.0828833)
\curveto(129.18343922,208.94288125)(129.15343925,208.80788139)(129.13345276,208.6778833)
\curveto(129.12343928,208.54788165)(129.09343931,208.42788177)(129.04345276,208.3178833)
\curveto(128.9034395,207.96788223)(128.73843967,207.67288252)(128.54845276,207.4328833)
\curveto(128.35844005,207.20288299)(128.08844032,207.01788318)(127.73845276,206.8778833)
\curveto(127.65844075,206.84788335)(127.57344083,206.82788337)(127.48345276,206.8178833)
\curveto(127.39344101,206.7978834)(127.3084411,206.77788342)(127.22845276,206.7578833)
\curveto(127.17844123,206.74788345)(127.12844128,206.74288345)(127.07845276,206.7428833)
\curveto(127.02844138,206.74288345)(126.97844143,206.73788346)(126.92845276,206.7278833)
\curveto(126.89844151,206.71788348)(126.84844156,206.71788348)(126.77845276,206.7278833)
\curveto(126.7084417,206.72788347)(126.65844175,206.73288346)(126.62845276,206.7428833)
\curveto(126.56844184,206.76288343)(126.5084419,206.77288342)(126.44845276,206.7728833)
\curveto(126.39844201,206.76288343)(126.34844206,206.76788343)(126.29845276,206.7878833)
\curveto(126.2084422,206.80788339)(126.11844229,206.83288336)(126.02845276,206.8628833)
\curveto(125.94844246,206.88288331)(125.86844254,206.91288328)(125.78845276,206.9528833)
\curveto(125.46844294,207.0928831)(125.21844319,207.28788291)(125.03845276,207.5378833)
\curveto(124.85844355,207.7978824)(124.7084437,208.10288209)(124.58845276,208.4528833)
\curveto(124.56844384,208.53288166)(124.55344385,208.61788158)(124.54345276,208.7078833)
\curveto(124.53344387,208.7978814)(124.51844389,208.88288131)(124.49845276,208.9628833)
\curveto(124.48844392,208.9928812)(124.48344392,209.02288117)(124.48345276,209.0528833)
\lineto(124.48345276,209.1578833)
\curveto(124.46344394,209.23788096)(124.45344395,209.31788088)(124.45345276,209.3978833)
\lineto(124.45345276,209.5328833)
\curveto(124.43344397,209.63288056)(124.43344397,209.73288046)(124.45345276,209.8328833)
\lineto(124.45345276,210.0128833)
\curveto(124.46344394,210.06288013)(124.46844394,210.10788009)(124.46845276,210.1478833)
\curveto(124.46844394,210.19788)(124.47344393,210.24287995)(124.48345276,210.2828833)
\curveto(124.49344391,210.32287987)(124.49844391,210.35787984)(124.49845276,210.3878833)
\curveto(124.49844391,210.42787977)(124.5034439,210.46787973)(124.51345276,210.5078833)
\lineto(124.57345276,210.8378833)
\curveto(124.59344381,210.95787924)(124.62344378,211.06787913)(124.66345276,211.1678833)
\curveto(124.8034436,211.4978787)(124.96344344,211.77287842)(125.14345276,211.9928833)
\curveto(125.33344307,212.22287797)(125.59344281,212.40787779)(125.92345276,212.5478833)
\curveto(126.0034424,212.58787761)(126.08844232,212.61287758)(126.17845276,212.6228833)
\lineto(126.47845276,212.6828833)
\lineto(126.61345276,212.6828833)
\curveto(126.66344174,212.6928775)(126.71344169,212.6978775)(126.76345276,212.6978833)
\curveto(127.33344107,212.71787748)(127.79344061,212.61287758)(128.14345276,212.3828833)
\curveto(128.5034399,212.16287803)(128.76843964,211.86287833)(128.93845276,211.4828833)
\curveto(128.98843942,211.38287881)(129.02843938,211.28287891)(129.05845276,211.1828833)
\curveto(129.08843932,211.08287911)(129.11843929,210.97787922)(129.14845276,210.8678833)
\curveto(129.15843925,210.82787937)(129.16343924,210.7928794)(129.16345276,210.7628833)
\curveto(129.16343924,210.74287945)(129.16843924,210.71287948)(129.17845276,210.6728833)
\curveto(129.19843921,210.60287959)(129.2084392,210.52787967)(129.20845276,210.4478833)
\curveto(129.2084392,210.36787983)(129.21843919,210.28787991)(129.23845276,210.2078833)
\curveto(129.23843917,210.15788004)(129.23843917,210.11288008)(129.23845276,210.0728833)
\curveto(129.23843917,210.03288016)(129.24343916,209.98788021)(129.25345276,209.9378833)
\moveto(128.14345276,209.5028833)
\curveto(128.15344025,209.55288064)(128.15844025,209.62788057)(128.15845276,209.7278833)
\curveto(128.16844024,209.82788037)(128.16344024,209.90288029)(128.14345276,209.9528833)
\curveto(128.12344028,210.01288018)(128.11844029,210.06788013)(128.12845276,210.1178833)
\curveto(128.14844026,210.17788002)(128.14844026,210.23787996)(128.12845276,210.2978833)
\curveto(128.11844029,210.32787987)(128.11344029,210.36287983)(128.11345276,210.4028833)
\curveto(128.11344029,210.44287975)(128.1084403,210.48287971)(128.09845276,210.5228833)
\curveto(128.07844033,210.60287959)(128.05844035,210.67787952)(128.03845276,210.7478833)
\curveto(128.02844038,210.82787937)(128.01344039,210.90787929)(127.99345276,210.9878833)
\curveto(127.96344044,211.04787915)(127.93844047,211.10787909)(127.91845276,211.1678833)
\curveto(127.89844051,211.22787897)(127.86844054,211.28787891)(127.82845276,211.3478833)
\curveto(127.72844068,211.51787868)(127.59844081,211.65287854)(127.43845276,211.7528833)
\curveto(127.35844105,211.80287839)(127.26344114,211.83787836)(127.15345276,211.8578833)
\curveto(127.04344136,211.87787832)(126.91844149,211.88787831)(126.77845276,211.8878833)
\curveto(126.75844165,211.87787832)(126.73344167,211.87287832)(126.70345276,211.8728833)
\curveto(126.67344173,211.88287831)(126.64344176,211.88287831)(126.61345276,211.8728833)
\lineto(126.46345276,211.8128833)
\curveto(126.41344199,211.80287839)(126.36844204,211.78787841)(126.32845276,211.7678833)
\curveto(126.13844227,211.65787854)(125.99344241,211.51287868)(125.89345276,211.3328833)
\curveto(125.8034426,211.15287904)(125.72344268,210.94787925)(125.65345276,210.7178833)
\curveto(125.61344279,210.58787961)(125.59344281,210.45287974)(125.59345276,210.3128833)
\curveto(125.59344281,210.18288001)(125.58344282,210.03788016)(125.56345276,209.8778833)
\curveto(125.55344285,209.82788037)(125.54344286,209.76788043)(125.53345276,209.6978833)
\curveto(125.53344287,209.62788057)(125.54344286,209.56788063)(125.56345276,209.5178833)
\lineto(125.56345276,209.3528833)
\lineto(125.56345276,209.1728833)
\curveto(125.57344283,209.12288107)(125.58344282,209.06788113)(125.59345276,209.0078833)
\curveto(125.6034428,208.95788124)(125.6084428,208.90288129)(125.60845276,208.8428833)
\curveto(125.61844279,208.78288141)(125.63344277,208.72788147)(125.65345276,208.6778833)
\curveto(125.7034427,208.48788171)(125.76344264,208.31288188)(125.83345276,208.1528833)
\curveto(125.9034425,207.9928822)(126.0084424,207.86288233)(126.14845276,207.7628833)
\curveto(126.27844213,207.66288253)(126.41844199,207.5928826)(126.56845276,207.5528833)
\curveto(126.59844181,207.54288265)(126.62344178,207.53788266)(126.64345276,207.5378833)
\curveto(126.67344173,207.54788265)(126.7034417,207.54788265)(126.73345276,207.5378833)
\curveto(126.75344165,207.53788266)(126.78344162,207.53288266)(126.82345276,207.5228833)
\curveto(126.86344154,207.52288267)(126.89844151,207.52788267)(126.92845276,207.5378833)
\curveto(126.96844144,207.54788265)(127.0084414,207.55288264)(127.04845276,207.5528833)
\curveto(127.08844132,207.55288264)(127.12844128,207.56288263)(127.16845276,207.5828833)
\curveto(127.408441,207.66288253)(127.6034408,207.7978824)(127.75345276,207.9878833)
\curveto(127.87344053,208.16788203)(127.96344044,208.37288182)(128.02345276,208.6028833)
\curveto(128.04344036,208.67288152)(128.05844035,208.74288145)(128.06845276,208.8128833)
\curveto(128.07844033,208.8928813)(128.09344031,208.97288122)(128.11345276,209.0528833)
\curveto(128.11344029,209.11288108)(128.11844029,209.15788104)(128.12845276,209.1878833)
\curveto(128.12844028,209.20788099)(128.12844028,209.23288096)(128.12845276,209.2628833)
\curveto(128.12844028,209.30288089)(128.13344027,209.33288086)(128.14345276,209.3528833)
\lineto(128.14345276,209.5028833)
}
}
{
\newrgbcolor{curcolor}{0 0 0}
\pscustom[linestyle=none,fillstyle=solid,fillcolor=curcolor]
{
\newpath
\moveto(92.40122498,142.5128833)
\curveto(92.50122012,142.51287268)(92.59622003,142.50287269)(92.68622498,142.4828833)
\curveto(92.77621985,142.47287272)(92.84121978,142.44287275)(92.88122498,142.3928833)
\curveto(92.94121968,142.31287288)(92.97121965,142.20787299)(92.97122498,142.0778833)
\lineto(92.97122498,141.6878833)
\lineto(92.97122498,140.1878833)
\lineto(92.97122498,133.7978833)
\lineto(92.97122498,132.6278833)
\lineto(92.97122498,132.3128833)
\curveto(92.98121964,132.21288298)(92.96621966,132.13288306)(92.92622498,132.0728833)
\curveto(92.87621975,131.9928832)(92.80121982,131.94288325)(92.70122498,131.9228833)
\curveto(92.61122001,131.91288328)(92.50122012,131.90788329)(92.37122498,131.9078833)
\lineto(92.14622498,131.9078833)
\curveto(92.06622056,131.92788327)(91.99622063,131.94288325)(91.93622498,131.9528833)
\curveto(91.87622075,131.97288322)(91.8262208,132.01288318)(91.78622498,132.0728833)
\curveto(91.74622088,132.13288306)(91.7262209,132.20788299)(91.72622498,132.2978833)
\lineto(91.72622498,132.5978833)
\lineto(91.72622498,133.6928833)
\lineto(91.72622498,139.0328833)
\curveto(91.70622092,139.12287607)(91.69122093,139.197876)(91.68122498,139.2578833)
\curveto(91.68122094,139.32787587)(91.65122097,139.38787581)(91.59122498,139.4378833)
\curveto(91.5212211,139.48787571)(91.43122119,139.51287568)(91.32122498,139.5128833)
\curveto(91.2212214,139.52287567)(91.11122151,139.52787567)(90.99122498,139.5278833)
\lineto(89.85122498,139.5278833)
\lineto(89.35622498,139.5278833)
\curveto(89.19622343,139.53787566)(89.08622354,139.5978756)(89.02622498,139.7078833)
\curveto(89.00622362,139.73787546)(88.99622363,139.76787543)(88.99622498,139.7978833)
\curveto(88.99622363,139.83787536)(88.99122363,139.88287531)(88.98122498,139.9328833)
\curveto(88.96122366,140.05287514)(88.96622366,140.16287503)(88.99622498,140.2628833)
\curveto(89.03622359,140.36287483)(89.09122353,140.43287476)(89.16122498,140.4728833)
\curveto(89.24122338,140.52287467)(89.36122326,140.54787465)(89.52122498,140.5478833)
\curveto(89.68122294,140.54787465)(89.81622281,140.56287463)(89.92622498,140.5928833)
\curveto(89.97622265,140.60287459)(90.03122259,140.60787459)(90.09122498,140.6078833)
\curveto(90.15122247,140.61787458)(90.21122241,140.63287456)(90.27122498,140.6528833)
\curveto(90.4212222,140.70287449)(90.56622206,140.75287444)(90.70622498,140.8028833)
\curveto(90.84622178,140.86287433)(90.98122164,140.93287426)(91.11122498,141.0128833)
\curveto(91.25122137,141.10287409)(91.37122125,141.20787399)(91.47122498,141.3278833)
\curveto(91.57122105,141.44787375)(91.66622096,141.57787362)(91.75622498,141.7178833)
\curveto(91.81622081,141.81787338)(91.86122076,141.92787327)(91.89122498,142.0478833)
\curveto(91.93122069,142.16787303)(91.98122064,142.27287292)(92.04122498,142.3628833)
\curveto(92.09122053,142.42287277)(92.16122046,142.46287273)(92.25122498,142.4828833)
\curveto(92.27122035,142.4928727)(92.29622033,142.4978727)(92.32622498,142.4978833)
\curveto(92.35622027,142.4978727)(92.38122024,142.50287269)(92.40122498,142.5128833)
}
}
{
\newrgbcolor{curcolor}{0 0 0}
\pscustom[linestyle=none,fillstyle=solid,fillcolor=curcolor]
{
\newpath
\moveto(98.20083435,142.3178833)
\lineto(101.80083435,142.3178833)
\lineto(102.44583435,142.3178833)
\curveto(102.52582782,142.31787288)(102.60082775,142.31287288)(102.67083435,142.3028833)
\curveto(102.74082761,142.30287289)(102.80082755,142.2928729)(102.85083435,142.2728833)
\curveto(102.92082743,142.24287295)(102.97582737,142.18287301)(103.01583435,142.0928833)
\curveto(103.03582731,142.06287313)(103.0458273,142.02287317)(103.04583435,141.9728833)
\lineto(103.04583435,141.8378833)
\curveto(103.05582729,141.72787347)(103.0508273,141.62287357)(103.03083435,141.5228833)
\curveto(103.02082733,141.42287377)(102.98582736,141.35287384)(102.92583435,141.3128833)
\curveto(102.83582751,141.24287395)(102.70082765,141.20787399)(102.52083435,141.2078833)
\curveto(102.34082801,141.21787398)(102.17582817,141.22287397)(102.02583435,141.2228833)
\lineto(100.03083435,141.2228833)
\lineto(99.53583435,141.2228833)
\lineto(99.40083435,141.2228833)
\curveto(99.36083099,141.22287397)(99.32083103,141.21787398)(99.28083435,141.2078833)
\lineto(99.07083435,141.2078833)
\curveto(98.96083139,141.17787402)(98.88083147,141.13787406)(98.83083435,141.0878833)
\curveto(98.78083157,141.04787415)(98.7458316,140.9928742)(98.72583435,140.9228833)
\curveto(98.70583164,140.86287433)(98.69083166,140.7928744)(98.68083435,140.7128833)
\curveto(98.67083168,140.63287456)(98.6508317,140.54287465)(98.62083435,140.4428833)
\curveto(98.57083178,140.24287495)(98.53083182,140.03787516)(98.50083435,139.8278833)
\curveto(98.47083188,139.61787558)(98.43083192,139.41287578)(98.38083435,139.2128833)
\curveto(98.36083199,139.14287605)(98.350832,139.07287612)(98.35083435,139.0028833)
\curveto(98.350832,138.94287625)(98.34083201,138.87787632)(98.32083435,138.8078833)
\curveto(98.31083204,138.77787642)(98.30083205,138.73787646)(98.29083435,138.6878833)
\curveto(98.29083206,138.64787655)(98.29583205,138.60787659)(98.30583435,138.5678833)
\curveto(98.32583202,138.51787668)(98.350832,138.47287672)(98.38083435,138.4328833)
\curveto(98.42083193,138.40287679)(98.48083187,138.3978768)(98.56083435,138.4178833)
\curveto(98.62083173,138.43787676)(98.68083167,138.46287673)(98.74083435,138.4928833)
\curveto(98.80083155,138.53287666)(98.86083149,138.56787663)(98.92083435,138.5978833)
\curveto(98.98083137,138.61787658)(99.03083132,138.63287656)(99.07083435,138.6428833)
\curveto(99.26083109,138.72287647)(99.46583088,138.77787642)(99.68583435,138.8078833)
\curveto(99.91583043,138.83787636)(100.1458302,138.84787635)(100.37583435,138.8378833)
\curveto(100.61582973,138.83787636)(100.8458295,138.81287638)(101.06583435,138.7628833)
\curveto(101.28582906,138.72287647)(101.48582886,138.66287653)(101.66583435,138.5828833)
\curveto(101.71582863,138.56287663)(101.76082859,138.54287665)(101.80083435,138.5228833)
\curveto(101.8508285,138.50287669)(101.90082845,138.47787672)(101.95083435,138.4478833)
\curveto(102.30082805,138.23787696)(102.58082777,138.00787719)(102.79083435,137.7578833)
\curveto(103.01082734,137.50787769)(103.20582714,137.18287801)(103.37583435,136.7828833)
\curveto(103.42582692,136.67287852)(103.46082689,136.56287863)(103.48083435,136.4528833)
\curveto(103.50082685,136.34287885)(103.52582682,136.22787897)(103.55583435,136.1078833)
\curveto(103.56582678,136.07787912)(103.57082678,136.03287916)(103.57083435,135.9728833)
\curveto(103.59082676,135.91287928)(103.60082675,135.84287935)(103.60083435,135.7628833)
\curveto(103.60082675,135.6928795)(103.61082674,135.62787957)(103.63083435,135.5678833)
\lineto(103.63083435,135.4028833)
\curveto(103.64082671,135.35287984)(103.6458267,135.28287991)(103.64583435,135.1928833)
\curveto(103.6458267,135.10288009)(103.63582671,135.03288016)(103.61583435,134.9828833)
\curveto(103.59582675,134.92288027)(103.59082676,134.86288033)(103.60083435,134.8028833)
\curveto(103.61082674,134.75288044)(103.60582674,134.70288049)(103.58583435,134.6528833)
\curveto(103.5458268,134.4928807)(103.51082684,134.34288085)(103.48083435,134.2028833)
\curveto(103.4508269,134.06288113)(103.40582694,133.92788127)(103.34583435,133.7978833)
\curveto(103.18582716,133.42788177)(102.96582738,133.0928821)(102.68583435,132.7928833)
\curveto(102.40582794,132.4928827)(102.08582826,132.26288293)(101.72583435,132.1028833)
\curveto(101.55582879,132.02288317)(101.35582899,131.94788325)(101.12583435,131.8778833)
\curveto(101.01582933,131.83788336)(100.90082945,131.81288338)(100.78083435,131.8028833)
\curveto(100.66082969,131.7928834)(100.54082981,131.77288342)(100.42083435,131.7428833)
\curveto(100.37082998,131.72288347)(100.31583003,131.72288347)(100.25583435,131.7428833)
\curveto(100.19583015,131.75288344)(100.13583021,131.74788345)(100.07583435,131.7278833)
\curveto(99.97583037,131.70788349)(99.87583047,131.70788349)(99.77583435,131.7278833)
\lineto(99.64083435,131.7278833)
\curveto(99.59083076,131.74788345)(99.53083082,131.75788344)(99.46083435,131.7578833)
\curveto(99.40083095,131.74788345)(99.345831,131.75288344)(99.29583435,131.7728833)
\curveto(99.25583109,131.78288341)(99.22083113,131.78788341)(99.19083435,131.7878833)
\curveto(99.16083119,131.78788341)(99.12583122,131.7928834)(99.08583435,131.8028833)
\lineto(98.81583435,131.8628833)
\curveto(98.72583162,131.88288331)(98.64083171,131.91288328)(98.56083435,131.9528833)
\curveto(98.22083213,132.0928831)(97.93083242,132.24788295)(97.69083435,132.4178833)
\curveto(97.4508329,132.5978826)(97.23083312,132.82788237)(97.03083435,133.1078833)
\curveto(96.88083347,133.33788186)(96.76583358,133.57788162)(96.68583435,133.8278833)
\curveto(96.66583368,133.87788132)(96.65583369,133.92288127)(96.65583435,133.9628833)
\curveto(96.65583369,134.01288118)(96.6458337,134.06288113)(96.62583435,134.1128833)
\curveto(96.60583374,134.17288102)(96.59083376,134.25288094)(96.58083435,134.3528833)
\curveto(96.58083377,134.45288074)(96.60083375,134.52788067)(96.64083435,134.5778833)
\curveto(96.69083366,134.65788054)(96.77083358,134.70288049)(96.88083435,134.7128833)
\curveto(96.99083336,134.72288047)(97.10583324,134.72788047)(97.22583435,134.7278833)
\lineto(97.39083435,134.7278833)
\curveto(97.4508329,134.72788047)(97.50583284,134.71788048)(97.55583435,134.6978833)
\curveto(97.6458327,134.67788052)(97.71583263,134.63788056)(97.76583435,134.5778833)
\curveto(97.83583251,134.48788071)(97.88083247,134.37788082)(97.90083435,134.2478833)
\curveto(97.93083242,134.12788107)(97.97583237,134.02288117)(98.03583435,133.9328833)
\curveto(98.22583212,133.5928816)(98.48583186,133.32288187)(98.81583435,133.1228833)
\curveto(98.91583143,133.06288213)(99.02083133,133.01288218)(99.13083435,132.9728833)
\curveto(99.2508311,132.94288225)(99.37083098,132.90788229)(99.49083435,132.8678833)
\curveto(99.66083069,132.81788238)(99.86583048,132.7978824)(100.10583435,132.8078833)
\curveto(100.35582999,132.82788237)(100.55582979,132.86288233)(100.70583435,132.9128833)
\curveto(101.07582927,133.03288216)(101.36582898,133.192882)(101.57583435,133.3928833)
\curveto(101.79582855,133.60288159)(101.97582837,133.88288131)(102.11583435,134.2328833)
\curveto(102.16582818,134.33288086)(102.19582815,134.43788076)(102.20583435,134.5478833)
\curveto(102.22582812,134.65788054)(102.2508281,134.77288042)(102.28083435,134.8928833)
\lineto(102.28083435,134.9978833)
\curveto(102.29082806,135.03788016)(102.29582805,135.07788012)(102.29583435,135.1178833)
\curveto(102.30582804,135.14788005)(102.30582804,135.18288001)(102.29583435,135.2228833)
\lineto(102.29583435,135.3428833)
\curveto(102.29582805,135.60287959)(102.26582808,135.84787935)(102.20583435,136.0778833)
\curveto(102.09582825,136.42787877)(101.94082841,136.72287847)(101.74083435,136.9628833)
\curveto(101.54082881,137.21287798)(101.28082907,137.40787779)(100.96083435,137.5478833)
\lineto(100.78083435,137.6078833)
\curveto(100.73082962,137.62787757)(100.67082968,137.64787755)(100.60083435,137.6678833)
\curveto(100.5508298,137.68787751)(100.49082986,137.6978775)(100.42083435,137.6978833)
\curveto(100.36082999,137.70787749)(100.29583005,137.72287747)(100.22583435,137.7428833)
\lineto(100.07583435,137.7428833)
\curveto(100.03583031,137.76287743)(99.98083037,137.77287742)(99.91083435,137.7728833)
\curveto(99.8508305,137.77287742)(99.79583055,137.76287743)(99.74583435,137.7428833)
\lineto(99.64083435,137.7428833)
\curveto(99.61083074,137.74287745)(99.57583077,137.73787746)(99.53583435,137.7278833)
\lineto(99.29583435,137.6678833)
\curveto(99.21583113,137.65787754)(99.13583121,137.63787756)(99.05583435,137.6078833)
\curveto(98.81583153,137.50787769)(98.58583176,137.37287782)(98.36583435,137.2028833)
\curveto(98.27583207,137.13287806)(98.19083216,137.05787814)(98.11083435,136.9778833)
\curveto(98.03083232,136.90787829)(97.93083242,136.85287834)(97.81083435,136.8128833)
\curveto(97.72083263,136.78287841)(97.58083277,136.77287842)(97.39083435,136.7828833)
\curveto(97.21083314,136.7928784)(97.09083326,136.81787838)(97.03083435,136.8578833)
\curveto(96.98083337,136.8978783)(96.94083341,136.95787824)(96.91083435,137.0378833)
\curveto(96.89083346,137.11787808)(96.89083346,137.20287799)(96.91083435,137.2928833)
\curveto(96.94083341,137.41287778)(96.96083339,137.53287766)(96.97083435,137.6528833)
\curveto(96.99083336,137.78287741)(97.01583333,137.90787729)(97.04583435,138.0278833)
\curveto(97.06583328,138.06787713)(97.07083328,138.10287709)(97.06083435,138.1328833)
\curveto(97.06083329,138.17287702)(97.07083328,138.21787698)(97.09083435,138.2678833)
\curveto(97.11083324,138.35787684)(97.12583322,138.44787675)(97.13583435,138.5378833)
\curveto(97.1458332,138.63787656)(97.16583318,138.73287646)(97.19583435,138.8228833)
\curveto(97.20583314,138.88287631)(97.21083314,138.94287625)(97.21083435,139.0028833)
\curveto(97.22083313,139.06287613)(97.23583311,139.12287607)(97.25583435,139.1828833)
\curveto(97.30583304,139.38287581)(97.34083301,139.58787561)(97.36083435,139.7978833)
\curveto(97.39083296,140.01787518)(97.43083292,140.22787497)(97.48083435,140.4278833)
\curveto(97.51083284,140.52787467)(97.53083282,140.62787457)(97.54083435,140.7278833)
\curveto(97.5508328,140.82787437)(97.56583278,140.92787427)(97.58583435,141.0278833)
\curveto(97.59583275,141.05787414)(97.60083275,141.0978741)(97.60083435,141.1478833)
\curveto(97.63083272,141.25787394)(97.6508327,141.36287383)(97.66083435,141.4628833)
\curveto(97.68083267,141.57287362)(97.70583264,141.68287351)(97.73583435,141.7928833)
\curveto(97.75583259,141.87287332)(97.77083258,141.94287325)(97.78083435,142.0028833)
\curveto(97.79083256,142.07287312)(97.81583253,142.13287306)(97.85583435,142.1828833)
\curveto(97.87583247,142.21287298)(97.90583244,142.23287296)(97.94583435,142.2428833)
\curveto(97.98583236,142.26287293)(98.03083232,142.28287291)(98.08083435,142.3028833)
\curveto(98.14083221,142.30287289)(98.18083217,142.30787289)(98.20083435,142.3178833)
}
}
{
\newrgbcolor{curcolor}{0 0 0}
\pscustom[linestyle=none,fillstyle=solid,fillcolor=curcolor]
{
\newpath
\moveto(105.99544373,133.5428833)
\lineto(106.29544373,133.5428833)
\curveto(106.40544167,133.55288164)(106.51044156,133.55288164)(106.61044373,133.5428833)
\curveto(106.72044135,133.54288165)(106.82044125,133.53288166)(106.91044373,133.5128833)
\curveto(107.00044107,133.50288169)(107.070441,133.47788172)(107.12044373,133.4378833)
\curveto(107.14044093,133.41788178)(107.15544092,133.38788181)(107.16544373,133.3478833)
\curveto(107.18544089,133.30788189)(107.20544087,133.26288193)(107.22544373,133.2128833)
\lineto(107.22544373,133.1378833)
\curveto(107.23544084,133.08788211)(107.23544084,133.03288216)(107.22544373,132.9728833)
\lineto(107.22544373,132.8228833)
\lineto(107.22544373,132.3428833)
\curveto(107.22544085,132.17288302)(107.18544089,132.05288314)(107.10544373,131.9828833)
\curveto(107.03544104,131.93288326)(106.94544113,131.90788329)(106.83544373,131.9078833)
\lineto(106.50544373,131.9078833)
\lineto(106.05544373,131.9078833)
\curveto(105.90544217,131.90788329)(105.79044228,131.93788326)(105.71044373,131.9978833)
\curveto(105.6704424,132.02788317)(105.64044243,132.07788312)(105.62044373,132.1478833)
\curveto(105.60044247,132.22788297)(105.58544249,132.31288288)(105.57544373,132.4028833)
\lineto(105.57544373,132.6878833)
\curveto(105.58544249,132.78788241)(105.59044248,132.87288232)(105.59044373,132.9428833)
\lineto(105.59044373,133.1378833)
\curveto(105.59044248,133.197882)(105.60044247,133.25288194)(105.62044373,133.3028833)
\curveto(105.66044241,133.41288178)(105.73044234,133.48288171)(105.83044373,133.5128833)
\curveto(105.86044221,133.51288168)(105.91544216,133.52288167)(105.99544373,133.5428833)
}
}
{
\newrgbcolor{curcolor}{0 0 0}
\pscustom[linestyle=none,fillstyle=solid,fillcolor=curcolor]
{
\newpath
\moveto(116.21559998,134.9828833)
\curveto(116.22559226,134.94288025)(116.22559226,134.8928803)(116.21559998,134.8328833)
\curveto(116.21559227,134.77288042)(116.21059227,134.72288047)(116.20059998,134.6828833)
\curveto(116.20059228,134.64288055)(116.19559229,134.60288059)(116.18559998,134.5628833)
\lineto(116.18559998,134.4578833)
\curveto(116.16559232,134.37788082)(116.15059233,134.2978809)(116.14059998,134.2178833)
\curveto(116.13059235,134.13788106)(116.11059237,134.06288113)(116.08059998,133.9928833)
\curveto(116.06059242,133.91288128)(116.04059244,133.83788136)(116.02059998,133.7678833)
\curveto(116.00059248,133.6978815)(115.97059251,133.62288157)(115.93059998,133.5428833)
\curveto(115.75059273,133.12288207)(115.49559299,132.78288241)(115.16559998,132.5228833)
\curveto(114.83559365,132.26288293)(114.44559404,132.05788314)(113.99559998,131.9078833)
\curveto(113.87559461,131.86788333)(113.75059473,131.84288335)(113.62059998,131.8328833)
\curveto(113.50059498,131.81288338)(113.37559511,131.78788341)(113.24559998,131.7578833)
\curveto(113.1855953,131.74788345)(113.12059536,131.74288345)(113.05059998,131.7428833)
\curveto(112.99059549,131.74288345)(112.92559556,131.73788346)(112.85559998,131.7278833)
\lineto(112.73559998,131.7278833)
\lineto(112.54059998,131.7278833)
\curveto(112.480596,131.71788348)(112.42559606,131.72288347)(112.37559998,131.7428833)
\curveto(112.30559618,131.76288343)(112.24059624,131.76788343)(112.18059998,131.7578833)
\curveto(112.12059636,131.74788345)(112.06059642,131.75288344)(112.00059998,131.7728833)
\curveto(111.95059653,131.78288341)(111.90559658,131.78788341)(111.86559998,131.7878833)
\curveto(111.82559666,131.78788341)(111.7805967,131.7978834)(111.73059998,131.8178833)
\curveto(111.65059683,131.83788336)(111.57559691,131.85788334)(111.50559998,131.8778833)
\curveto(111.43559705,131.88788331)(111.36559712,131.90288329)(111.29559998,131.9228833)
\curveto(110.81559767,132.0928831)(110.41559807,132.30288289)(110.09559998,132.5528833)
\curveto(109.7855987,132.81288238)(109.53559895,133.16788203)(109.34559998,133.6178833)
\curveto(109.31559917,133.67788152)(109.29059919,133.73788146)(109.27059998,133.7978833)
\curveto(109.26059922,133.86788133)(109.24559924,133.94288125)(109.22559998,134.0228833)
\curveto(109.20559928,134.08288111)(109.19059929,134.14788105)(109.18059998,134.2178833)
\curveto(109.17059931,134.28788091)(109.15559933,134.35788084)(109.13559998,134.4278833)
\curveto(109.12559936,134.47788072)(109.12059936,134.51788068)(109.12059998,134.5478833)
\lineto(109.12059998,134.6678833)
\curveto(109.11059937,134.70788049)(109.10059938,134.75788044)(109.09059998,134.8178833)
\curveto(109.09059939,134.87788032)(109.09559939,134.92788027)(109.10559998,134.9678833)
\lineto(109.10559998,135.1028833)
\curveto(109.11559937,135.15288004)(109.12059936,135.20287999)(109.12059998,135.2528833)
\curveto(109.14059934,135.35287984)(109.15559933,135.44787975)(109.16559998,135.5378833)
\curveto(109.17559931,135.63787956)(109.19559929,135.73287946)(109.22559998,135.8228833)
\curveto(109.27559921,135.97287922)(109.33059915,136.11287908)(109.39059998,136.2428833)
\curveto(109.45059903,136.37287882)(109.52059896,136.4928787)(109.60059998,136.6028833)
\curveto(109.63059885,136.65287854)(109.66059882,136.6928785)(109.69059998,136.7228833)
\curveto(109.73059875,136.75287844)(109.76559872,136.78787841)(109.79559998,136.8278833)
\curveto(109.85559863,136.90787829)(109.92559856,136.97787822)(110.00559998,137.0378833)
\curveto(110.06559842,137.08787811)(110.12559836,137.13287806)(110.18559998,137.1728833)
\lineto(110.39559998,137.3228833)
\curveto(110.44559804,137.36287783)(110.49559799,137.3978778)(110.54559998,137.4278833)
\curveto(110.59559789,137.46787773)(110.63059785,137.52287767)(110.65059998,137.5928833)
\curveto(110.65059783,137.62287757)(110.64059784,137.64787755)(110.62059998,137.6678833)
\curveto(110.61059787,137.6978775)(110.60059788,137.72287747)(110.59059998,137.7428833)
\curveto(110.55059793,137.7928774)(110.50059798,137.83787736)(110.44059998,137.8778833)
\curveto(110.39059809,137.92787727)(110.34059814,137.97287722)(110.29059998,138.0128833)
\curveto(110.25059823,138.04287715)(110.20059828,138.0978771)(110.14059998,138.1778833)
\curveto(110.12059836,138.20787699)(110.09059839,138.23287696)(110.05059998,138.2528833)
\curveto(110.02059846,138.28287691)(109.99559849,138.31787688)(109.97559998,138.3578833)
\curveto(109.80559868,138.56787663)(109.67559881,138.81287638)(109.58559998,139.0928833)
\curveto(109.56559892,139.17287602)(109.55059893,139.25287594)(109.54059998,139.3328833)
\curveto(109.53059895,139.41287578)(109.51559897,139.4928757)(109.49559998,139.5728833)
\curveto(109.47559901,139.62287557)(109.46559902,139.68787551)(109.46559998,139.7678833)
\curveto(109.46559902,139.85787534)(109.47559901,139.92787527)(109.49559998,139.9778833)
\curveto(109.49559899,140.07787512)(109.50059898,140.14787505)(109.51059998,140.1878833)
\curveto(109.53059895,140.26787493)(109.54559894,140.33787486)(109.55559998,140.3978833)
\curveto(109.56559892,140.46787473)(109.5805989,140.53787466)(109.60059998,140.6078833)
\curveto(109.75059873,141.03787416)(109.96559852,141.38287381)(110.24559998,141.6428833)
\curveto(110.53559795,141.90287329)(110.8855976,142.11787308)(111.29559998,142.2878833)
\curveto(111.40559708,142.33787286)(111.52059696,142.36787283)(111.64059998,142.3778833)
\curveto(111.77059671,142.3978728)(111.90059658,142.42787277)(112.03059998,142.4678833)
\curveto(112.11059637,142.46787273)(112.1805963,142.46787273)(112.24059998,142.4678833)
\curveto(112.31059617,142.47787272)(112.3855961,142.48787271)(112.46559998,142.4978833)
\curveto(113.25559523,142.51787268)(113.91059457,142.38787281)(114.43059998,142.1078833)
\curveto(114.96059352,141.82787337)(115.34059314,141.41787378)(115.57059998,140.8778833)
\curveto(115.6805928,140.64787455)(115.75059273,140.36287483)(115.78059998,140.0228833)
\curveto(115.82059266,139.6928755)(115.79059269,139.38787581)(115.69059998,139.1078833)
\curveto(115.65059283,138.97787622)(115.60059288,138.85787634)(115.54059998,138.7478833)
\curveto(115.49059299,138.63787656)(115.43059305,138.53287666)(115.36059998,138.4328833)
\curveto(115.34059314,138.3928768)(115.31059317,138.35787684)(115.27059998,138.3278833)
\lineto(115.18059998,138.2378833)
\curveto(115.13059335,138.14787705)(115.07059341,138.08287711)(115.00059998,138.0428833)
\curveto(114.95059353,137.9928772)(114.89559359,137.94287725)(114.83559998,137.8928833)
\curveto(114.7855937,137.85287734)(114.74059374,137.80787739)(114.70059998,137.7578833)
\curveto(114.6805938,137.73787746)(114.66059382,137.71287748)(114.64059998,137.6828833)
\curveto(114.63059385,137.66287753)(114.63059385,137.63787756)(114.64059998,137.6078833)
\curveto(114.65059383,137.55787764)(114.6805938,137.50787769)(114.73059998,137.4578833)
\curveto(114.7805937,137.41787778)(114.83559365,137.37787782)(114.89559998,137.3378833)
\lineto(115.07559998,137.2178833)
\curveto(115.13559335,137.18787801)(115.1855933,137.15787804)(115.22559998,137.1278833)
\curveto(115.55559293,136.88787831)(115.80559268,136.57787862)(115.97559998,136.1978833)
\curveto(116.01559247,136.11787908)(116.04559244,136.03287916)(116.06559998,135.9428833)
\curveto(116.09559239,135.85287934)(116.12059236,135.76287943)(116.14059998,135.6728833)
\curveto(116.15059233,135.62287957)(116.16059232,135.56787963)(116.17059998,135.5078833)
\lineto(116.20059998,135.3578833)
\curveto(116.21059227,135.2978799)(116.21059227,135.23287996)(116.20059998,135.1628833)
\curveto(116.19059229,135.10288009)(116.19559229,135.04288015)(116.21559998,134.9828833)
\moveto(110.83059998,140.0228833)
\curveto(110.80059768,139.91287528)(110.79559769,139.77287542)(110.81559998,139.6028833)
\curveto(110.83559765,139.44287575)(110.86059762,139.31787588)(110.89059998,139.2278833)
\curveto(111.00059748,138.90787629)(111.15059733,138.66287653)(111.34059998,138.4928833)
\curveto(111.53059695,138.33287686)(111.79559669,138.20287699)(112.13559998,138.1028833)
\curveto(112.26559622,138.07287712)(112.43059605,138.04787715)(112.63059998,138.0278833)
\curveto(112.83059565,138.01787718)(113.00059548,138.03287716)(113.14059998,138.0728833)
\curveto(113.43059505,138.15287704)(113.67059481,138.26287693)(113.86059998,138.4028833)
\curveto(114.06059442,138.55287664)(114.21559427,138.75287644)(114.32559998,139.0028833)
\curveto(114.34559414,139.05287614)(114.35559413,139.0978761)(114.35559998,139.1378833)
\curveto(114.36559412,139.17787602)(114.3805941,139.22287597)(114.40059998,139.2728833)
\curveto(114.43059405,139.38287581)(114.45059403,139.52287567)(114.46059998,139.6928833)
\curveto(114.47059401,139.86287533)(114.46059402,140.00787519)(114.43059998,140.1278833)
\curveto(114.41059407,140.21787498)(114.3855941,140.30287489)(114.35559998,140.3828833)
\curveto(114.33559415,140.46287473)(114.30059418,140.54287465)(114.25059998,140.6228833)
\curveto(114.0805944,140.8928743)(113.85559463,141.08787411)(113.57559998,141.2078833)
\curveto(113.30559518,141.32787387)(112.94559554,141.38787381)(112.49559998,141.3878833)
\curveto(112.47559601,141.36787383)(112.44559604,141.36287383)(112.40559998,141.3728833)
\curveto(112.36559612,141.38287381)(112.33059615,141.38287381)(112.30059998,141.3728833)
\curveto(112.25059623,141.35287384)(112.19559629,141.33787386)(112.13559998,141.3278833)
\curveto(112.0855964,141.32787387)(112.03559645,141.31787388)(111.98559998,141.2978833)
\curveto(111.74559674,141.20787399)(111.53559695,141.0928741)(111.35559998,140.9528833)
\curveto(111.17559731,140.82287437)(111.03559745,140.64287455)(110.93559998,140.4128833)
\curveto(110.91559757,140.35287484)(110.89559759,140.28787491)(110.87559998,140.2178833)
\curveto(110.86559762,140.15787504)(110.85059763,140.0928751)(110.83059998,140.0228833)
\moveto(114.85059998,134.4878833)
\curveto(114.90059358,134.67788052)(114.90559358,134.88288031)(114.86559998,135.1028833)
\curveto(114.83559365,135.32287987)(114.79059369,135.50287969)(114.73059998,135.6428833)
\curveto(114.56059392,136.01287918)(114.30059418,136.31787888)(113.95059998,136.5578833)
\curveto(113.61059487,136.7978784)(113.17559531,136.91787828)(112.64559998,136.9178833)
\curveto(112.61559587,136.8978783)(112.57559591,136.8928783)(112.52559998,136.9028833)
\curveto(112.47559601,136.92287827)(112.43559605,136.92787827)(112.40559998,136.9178833)
\lineto(112.13559998,136.8578833)
\curveto(112.05559643,136.84787835)(111.97559651,136.83287836)(111.89559998,136.8128833)
\curveto(111.59559689,136.70287849)(111.33059715,136.55787864)(111.10059998,136.3778833)
\curveto(110.8805976,136.197879)(110.71059777,135.96787923)(110.59059998,135.6878833)
\curveto(110.56059792,135.60787959)(110.53559795,135.52787967)(110.51559998,135.4478833)
\curveto(110.49559799,135.36787983)(110.47559801,135.28287991)(110.45559998,135.1928833)
\curveto(110.42559806,135.07288012)(110.41559807,134.92288027)(110.42559998,134.7428833)
\curveto(110.44559804,134.56288063)(110.47059801,134.42288077)(110.50059998,134.3228833)
\curveto(110.52059796,134.27288092)(110.53059795,134.22788097)(110.53059998,134.1878833)
\curveto(110.54059794,134.15788104)(110.55559793,134.11788108)(110.57559998,134.0678833)
\curveto(110.67559781,133.84788135)(110.80559768,133.64788155)(110.96559998,133.4678833)
\curveto(111.13559735,133.28788191)(111.33059715,133.15288204)(111.55059998,133.0628833)
\curveto(111.62059686,133.02288217)(111.71559677,132.98788221)(111.83559998,132.9578833)
\curveto(112.05559643,132.86788233)(112.31059617,132.82288237)(112.60059998,132.8228833)
\lineto(112.88559998,132.8228833)
\curveto(112.9855955,132.84288235)(113.0805954,132.85788234)(113.17059998,132.8678833)
\curveto(113.26059522,132.87788232)(113.35059513,132.8978823)(113.44059998,132.9278833)
\curveto(113.70059478,133.00788219)(113.94059454,133.13788206)(114.16059998,133.3178833)
\curveto(114.39059409,133.50788169)(114.56059392,133.72288147)(114.67059998,133.9628833)
\curveto(114.71059377,134.04288115)(114.74059374,134.12288107)(114.76059998,134.2028833)
\curveto(114.79059369,134.2928809)(114.82059366,134.38788081)(114.85059998,134.4878833)
}
}
{
\newrgbcolor{curcolor}{0 0 0}
\pscustom[linestyle=none,fillstyle=solid,fillcolor=curcolor]
{
\newpath
\moveto(127.35520935,140.4278833)
\curveto(127.15519905,140.13787506)(126.94519926,139.85287534)(126.72520935,139.5728833)
\curveto(126.51519969,139.2928759)(126.3101999,139.00787619)(126.11020935,138.7178833)
\curveto(125.5102007,137.86787733)(124.9052013,137.02787817)(124.29520935,136.1978833)
\curveto(123.68520252,135.37787982)(123.08020313,134.54288065)(122.48020935,133.6928833)
\lineto(121.97020935,132.9728833)
\lineto(121.46020935,132.2828833)
\curveto(121.38020483,132.17288302)(121.30020491,132.05788314)(121.22020935,131.9378833)
\curveto(121.14020507,131.81788338)(121.04520516,131.72288347)(120.93520935,131.6528833)
\curveto(120.89520531,131.63288356)(120.83020538,131.61788358)(120.74020935,131.6078833)
\curveto(120.66020555,131.58788361)(120.57020564,131.57788362)(120.47020935,131.5778833)
\curveto(120.37020584,131.57788362)(120.27520593,131.58288361)(120.18520935,131.5928833)
\curveto(120.1052061,131.60288359)(120.04520616,131.62288357)(120.00520935,131.6528833)
\curveto(119.97520623,131.67288352)(119.95020626,131.70788349)(119.93020935,131.7578833)
\curveto(119.92020629,131.7978834)(119.92520628,131.84288335)(119.94520935,131.8928833)
\curveto(119.98520622,131.97288322)(120.03020618,132.04788315)(120.08020935,132.1178833)
\curveto(120.14020607,132.197883)(120.19520601,132.27788292)(120.24520935,132.3578833)
\curveto(120.48520572,132.6978825)(120.73020548,133.03288216)(120.98020935,133.3628833)
\curveto(121.23020498,133.6928815)(121.47020474,134.02788117)(121.70020935,134.3678833)
\curveto(121.86020435,134.58788061)(122.02020419,134.80288039)(122.18020935,135.0128833)
\curveto(122.34020387,135.22287997)(122.50020371,135.43787976)(122.66020935,135.6578833)
\curveto(123.02020319,136.17787902)(123.38520282,136.68787851)(123.75520935,137.1878833)
\curveto(124.12520208,137.68787751)(124.49520171,138.197877)(124.86520935,138.7178833)
\curveto(125.0052012,138.91787628)(125.14520106,139.11287608)(125.28520935,139.3028833)
\curveto(125.43520077,139.4928757)(125.58020063,139.68787551)(125.72020935,139.8878833)
\curveto(125.93020028,140.18787501)(126.14520006,140.48787471)(126.36520935,140.7878833)
\lineto(127.02520935,141.6878833)
\lineto(127.20520935,141.9578833)
\lineto(127.41520935,142.2278833)
\lineto(127.53520935,142.4078833)
\curveto(127.58519862,142.46787273)(127.63519857,142.52287267)(127.68520935,142.5728833)
\curveto(127.75519845,142.62287257)(127.83019838,142.65787254)(127.91020935,142.6778833)
\curveto(127.93019828,142.68787251)(127.95519825,142.68787251)(127.98520935,142.6778833)
\curveto(128.02519818,142.67787252)(128.05519815,142.68787251)(128.07520935,142.7078833)
\curveto(128.19519801,142.70787249)(128.33019788,142.70287249)(128.48020935,142.6928833)
\curveto(128.63019758,142.6928725)(128.72019749,142.64787255)(128.75020935,142.5578833)
\curveto(128.77019744,142.52787267)(128.77519743,142.4928727)(128.76520935,142.4528833)
\curveto(128.75519745,142.41287278)(128.74019747,142.38287281)(128.72020935,142.3628833)
\curveto(128.68019753,142.28287291)(128.64019757,142.21287298)(128.60020935,142.1528833)
\curveto(128.56019765,142.0928731)(128.51519769,142.03287316)(128.46520935,141.9728833)
\lineto(127.89520935,141.1928833)
\curveto(127.71519849,140.94287425)(127.53519867,140.68787451)(127.35520935,140.4278833)
\moveto(120.50020935,136.5278833)
\curveto(120.45020576,136.54787865)(120.40020581,136.55287864)(120.35020935,136.5428833)
\curveto(120.30020591,136.53287866)(120.25020596,136.53787866)(120.20020935,136.5578833)
\curveto(120.09020612,136.57787862)(119.98520622,136.5978786)(119.88520935,136.6178833)
\curveto(119.79520641,136.64787855)(119.70020651,136.68787851)(119.60020935,136.7378833)
\curveto(119.27020694,136.87787832)(119.01520719,137.07287812)(118.83520935,137.3228833)
\curveto(118.65520755,137.58287761)(118.5102077,137.8928773)(118.40020935,138.2528833)
\curveto(118.37020784,138.33287686)(118.35020786,138.41287678)(118.34020935,138.4928833)
\curveto(118.33020788,138.58287661)(118.31520789,138.66787653)(118.29520935,138.7478833)
\curveto(118.28520792,138.7978764)(118.28020793,138.86287633)(118.28020935,138.9428833)
\curveto(118.27020794,138.97287622)(118.26520794,139.00287619)(118.26520935,139.0328833)
\curveto(118.26520794,139.07287612)(118.26020795,139.10787609)(118.25020935,139.1378833)
\lineto(118.25020935,139.2878833)
\curveto(118.24020797,139.33787586)(118.23520797,139.3978758)(118.23520935,139.4678833)
\curveto(118.23520797,139.54787565)(118.24020797,139.61287558)(118.25020935,139.6628833)
\lineto(118.25020935,139.8278833)
\curveto(118.27020794,139.87787532)(118.27520793,139.92287527)(118.26520935,139.9628833)
\curveto(118.26520794,140.01287518)(118.27020794,140.05787514)(118.28020935,140.0978833)
\curveto(118.29020792,140.13787506)(118.29520791,140.17287502)(118.29520935,140.2028833)
\curveto(118.29520791,140.24287495)(118.30020791,140.28287491)(118.31020935,140.3228833)
\curveto(118.34020787,140.43287476)(118.36020785,140.54287465)(118.37020935,140.6528833)
\curveto(118.39020782,140.77287442)(118.42520778,140.88787431)(118.47520935,140.9978833)
\curveto(118.61520759,141.33787386)(118.77520743,141.61287358)(118.95520935,141.8228833)
\curveto(119.14520706,142.04287315)(119.41520679,142.22287297)(119.76520935,142.3628833)
\curveto(119.84520636,142.3928728)(119.93020628,142.41287278)(120.02020935,142.4228833)
\curveto(120.1102061,142.44287275)(120.205206,142.46287273)(120.30520935,142.4828833)
\curveto(120.33520587,142.4928727)(120.39020582,142.4928727)(120.47020935,142.4828833)
\curveto(120.55020566,142.48287271)(120.60020561,142.4928727)(120.62020935,142.5128833)
\curveto(121.18020503,142.52287267)(121.63020458,142.41287278)(121.97020935,142.1828833)
\curveto(122.32020389,141.95287324)(122.58020363,141.64787355)(122.75020935,141.2678833)
\curveto(122.79020342,141.17787402)(122.82520338,141.08287411)(122.85520935,140.9828833)
\curveto(122.88520332,140.88287431)(122.9102033,140.78287441)(122.93020935,140.6828833)
\curveto(122.95020326,140.65287454)(122.95520325,140.62287457)(122.94520935,140.5928833)
\curveto(122.94520326,140.56287463)(122.95020326,140.53287466)(122.96020935,140.5028833)
\curveto(122.99020322,140.3928748)(123.0102032,140.26787493)(123.02020935,140.1278833)
\curveto(123.03020318,139.9978752)(123.04020317,139.86287533)(123.05020935,139.7228833)
\lineto(123.05020935,139.5578833)
\curveto(123.06020315,139.4978757)(123.06020315,139.44287575)(123.05020935,139.3928833)
\curveto(123.04020317,139.34287585)(123.03520317,139.2928759)(123.03520935,139.2428833)
\lineto(123.03520935,139.1078833)
\curveto(123.02520318,139.06787613)(123.02020319,139.02787617)(123.02020935,138.9878833)
\curveto(123.03020318,138.94787625)(123.02520318,138.90287629)(123.00520935,138.8528833)
\curveto(122.98520322,138.74287645)(122.96520324,138.63787656)(122.94520935,138.5378833)
\curveto(122.93520327,138.43787676)(122.91520329,138.33787686)(122.88520935,138.2378833)
\curveto(122.75520345,137.87787732)(122.59020362,137.56287763)(122.39020935,137.2928833)
\curveto(122.19020402,137.02287817)(121.91520429,136.81787838)(121.56520935,136.6778833)
\curveto(121.48520472,136.64787855)(121.40020481,136.62287857)(121.31020935,136.6028833)
\lineto(121.04020935,136.5428833)
\curveto(120.99020522,136.53287866)(120.94520526,136.52787867)(120.90520935,136.5278833)
\curveto(120.86520534,136.53787866)(120.82520538,136.53787866)(120.78520935,136.5278833)
\curveto(120.68520552,136.50787869)(120.59020562,136.50787869)(120.50020935,136.5278833)
\moveto(119.66020935,137.9228833)
\curveto(119.70020651,137.85287734)(119.74020647,137.78787741)(119.78020935,137.7278833)
\curveto(119.82020639,137.67787752)(119.87020634,137.62787757)(119.93020935,137.5778833)
\lineto(120.08020935,137.4578833)
\curveto(120.14020607,137.42787777)(120.205206,137.40287779)(120.27520935,137.3828833)
\curveto(120.31520589,137.36287783)(120.35020586,137.35287784)(120.38020935,137.3528833)
\curveto(120.42020579,137.36287783)(120.46020575,137.35787784)(120.50020935,137.3378833)
\curveto(120.53020568,137.33787786)(120.57020564,137.33287786)(120.62020935,137.3228833)
\curveto(120.67020554,137.32287787)(120.7102055,137.32787787)(120.74020935,137.3378833)
\lineto(120.96520935,137.3828833)
\curveto(121.21520499,137.46287773)(121.40020481,137.58787761)(121.52020935,137.7578833)
\curveto(121.60020461,137.85787734)(121.67020454,137.98787721)(121.73020935,138.1478833)
\curveto(121.8102044,138.32787687)(121.87020434,138.55287664)(121.91020935,138.8228833)
\curveto(121.95020426,139.10287609)(121.96520424,139.38287581)(121.95520935,139.6628833)
\curveto(121.94520426,139.95287524)(121.91520429,140.22787497)(121.86520935,140.4878833)
\curveto(121.81520439,140.74787445)(121.74020447,140.95787424)(121.64020935,141.1178833)
\curveto(121.52020469,141.31787388)(121.37020484,141.46787373)(121.19020935,141.5678833)
\curveto(121.1102051,141.61787358)(121.02020519,141.64787355)(120.92020935,141.6578833)
\curveto(120.82020539,141.67787352)(120.71520549,141.68787351)(120.60520935,141.6878833)
\curveto(120.58520562,141.67787352)(120.56020565,141.67287352)(120.53020935,141.6728833)
\curveto(120.5102057,141.68287351)(120.49020572,141.68287351)(120.47020935,141.6728833)
\curveto(120.42020579,141.66287353)(120.37520583,141.65287354)(120.33520935,141.6428833)
\curveto(120.29520591,141.64287355)(120.25520595,141.63287356)(120.21520935,141.6128833)
\curveto(120.03520617,141.53287366)(119.88520632,141.41287378)(119.76520935,141.2528833)
\curveto(119.65520655,141.0928741)(119.56520664,140.91287428)(119.49520935,140.7128833)
\curveto(119.43520677,140.52287467)(119.39020682,140.2978749)(119.36020935,140.0378833)
\curveto(119.34020687,139.77787542)(119.33520687,139.51287568)(119.34520935,139.2428833)
\curveto(119.35520685,138.98287621)(119.38520682,138.73287646)(119.43520935,138.4928833)
\curveto(119.49520671,138.26287693)(119.57020664,138.07287712)(119.66020935,137.9228833)
\moveto(130.46020935,134.9378833)
\curveto(130.47019574,134.88788031)(130.47519573,134.7978804)(130.47520935,134.6678833)
\curveto(130.47519573,134.53788066)(130.46519574,134.44788075)(130.44520935,134.3978833)
\curveto(130.42519578,134.34788085)(130.42019579,134.2928809)(130.43020935,134.2328833)
\curveto(130.44019577,134.18288101)(130.44019577,134.13288106)(130.43020935,134.0828833)
\curveto(130.39019582,133.94288125)(130.36019585,133.80788139)(130.34020935,133.6778833)
\curveto(130.33019588,133.54788165)(130.30019591,133.42788177)(130.25020935,133.3178833)
\curveto(130.1101961,132.96788223)(129.94519626,132.67288252)(129.75520935,132.4328833)
\curveto(129.56519664,132.20288299)(129.29519691,132.01788318)(128.94520935,131.8778833)
\curveto(128.86519734,131.84788335)(128.78019743,131.82788337)(128.69020935,131.8178833)
\curveto(128.60019761,131.7978834)(128.51519769,131.77788342)(128.43520935,131.7578833)
\curveto(128.38519782,131.74788345)(128.33519787,131.74288345)(128.28520935,131.7428833)
\curveto(128.23519797,131.74288345)(128.18519802,131.73788346)(128.13520935,131.7278833)
\curveto(128.1051981,131.71788348)(128.05519815,131.71788348)(127.98520935,131.7278833)
\curveto(127.91519829,131.72788347)(127.86519834,131.73288346)(127.83520935,131.7428833)
\curveto(127.77519843,131.76288343)(127.71519849,131.77288342)(127.65520935,131.7728833)
\curveto(127.6051986,131.76288343)(127.55519865,131.76788343)(127.50520935,131.7878833)
\curveto(127.41519879,131.80788339)(127.32519888,131.83288336)(127.23520935,131.8628833)
\curveto(127.15519905,131.88288331)(127.07519913,131.91288328)(126.99520935,131.9528833)
\curveto(126.67519953,132.0928831)(126.42519978,132.28788291)(126.24520935,132.5378833)
\curveto(126.06520014,132.7978824)(125.91520029,133.10288209)(125.79520935,133.4528833)
\curveto(125.77520043,133.53288166)(125.76020045,133.61788158)(125.75020935,133.7078833)
\curveto(125.74020047,133.7978814)(125.72520048,133.88288131)(125.70520935,133.9628833)
\curveto(125.69520051,133.9928812)(125.69020052,134.02288117)(125.69020935,134.0528833)
\lineto(125.69020935,134.1578833)
\curveto(125.67020054,134.23788096)(125.66020055,134.31788088)(125.66020935,134.3978833)
\lineto(125.66020935,134.5328833)
\curveto(125.64020057,134.63288056)(125.64020057,134.73288046)(125.66020935,134.8328833)
\lineto(125.66020935,135.0128833)
\curveto(125.67020054,135.06288013)(125.67520053,135.10788009)(125.67520935,135.1478833)
\curveto(125.67520053,135.19788)(125.68020053,135.24287995)(125.69020935,135.2828833)
\curveto(125.70020051,135.32287987)(125.7052005,135.35787984)(125.70520935,135.3878833)
\curveto(125.7052005,135.42787977)(125.7102005,135.46787973)(125.72020935,135.5078833)
\lineto(125.78020935,135.8378833)
\curveto(125.80020041,135.95787924)(125.83020038,136.06787913)(125.87020935,136.1678833)
\curveto(126.0102002,136.4978787)(126.17020004,136.77287842)(126.35020935,136.9928833)
\curveto(126.54019967,137.22287797)(126.80019941,137.40787779)(127.13020935,137.5478833)
\curveto(127.210199,137.58787761)(127.29519891,137.61287758)(127.38520935,137.6228833)
\lineto(127.68520935,137.6828833)
\lineto(127.82020935,137.6828833)
\curveto(127.87019834,137.6928775)(127.92019829,137.6978775)(127.97020935,137.6978833)
\curveto(128.54019767,137.71787748)(129.00019721,137.61287758)(129.35020935,137.3828833)
\curveto(129.7101965,137.16287803)(129.97519623,136.86287833)(130.14520935,136.4828833)
\curveto(130.19519601,136.38287881)(130.23519597,136.28287891)(130.26520935,136.1828833)
\curveto(130.29519591,136.08287911)(130.32519588,135.97787922)(130.35520935,135.8678833)
\curveto(130.36519584,135.82787937)(130.37019584,135.7928794)(130.37020935,135.7628833)
\curveto(130.37019584,135.74287945)(130.37519583,135.71287948)(130.38520935,135.6728833)
\curveto(130.4051958,135.60287959)(130.41519579,135.52787967)(130.41520935,135.4478833)
\curveto(130.41519579,135.36787983)(130.42519578,135.28787991)(130.44520935,135.2078833)
\curveto(130.44519576,135.15788004)(130.44519576,135.11288008)(130.44520935,135.0728833)
\curveto(130.44519576,135.03288016)(130.45019576,134.98788021)(130.46020935,134.9378833)
\moveto(129.35020935,134.5028833)
\curveto(129.36019685,134.55288064)(129.36519684,134.62788057)(129.36520935,134.7278833)
\curveto(129.37519683,134.82788037)(129.37019684,134.90288029)(129.35020935,134.9528833)
\curveto(129.33019688,135.01288018)(129.32519688,135.06788013)(129.33520935,135.1178833)
\curveto(129.35519685,135.17788002)(129.35519685,135.23787996)(129.33520935,135.2978833)
\curveto(129.32519688,135.32787987)(129.32019689,135.36287983)(129.32020935,135.4028833)
\curveto(129.32019689,135.44287975)(129.31519689,135.48287971)(129.30520935,135.5228833)
\curveto(129.28519692,135.60287959)(129.26519694,135.67787952)(129.24520935,135.7478833)
\curveto(129.23519697,135.82787937)(129.22019699,135.90787929)(129.20020935,135.9878833)
\curveto(129.17019704,136.04787915)(129.14519706,136.10787909)(129.12520935,136.1678833)
\curveto(129.1051971,136.22787897)(129.07519713,136.28787891)(129.03520935,136.3478833)
\curveto(128.93519727,136.51787868)(128.8051974,136.65287854)(128.64520935,136.7528833)
\curveto(128.56519764,136.80287839)(128.47019774,136.83787836)(128.36020935,136.8578833)
\curveto(128.25019796,136.87787832)(128.12519808,136.88787831)(127.98520935,136.8878833)
\curveto(127.96519824,136.87787832)(127.94019827,136.87287832)(127.91020935,136.8728833)
\curveto(127.88019833,136.88287831)(127.85019836,136.88287831)(127.82020935,136.8728833)
\lineto(127.67020935,136.8128833)
\curveto(127.62019859,136.80287839)(127.57519863,136.78787841)(127.53520935,136.7678833)
\curveto(127.34519886,136.65787854)(127.20019901,136.51287868)(127.10020935,136.3328833)
\curveto(127.0101992,136.15287904)(126.93019928,135.94787925)(126.86020935,135.7178833)
\curveto(126.82019939,135.58787961)(126.80019941,135.45287974)(126.80020935,135.3128833)
\curveto(126.80019941,135.18288001)(126.79019942,135.03788016)(126.77020935,134.8778833)
\curveto(126.76019945,134.82788037)(126.75019946,134.76788043)(126.74020935,134.6978833)
\curveto(126.74019947,134.62788057)(126.75019946,134.56788063)(126.77020935,134.5178833)
\lineto(126.77020935,134.3528833)
\lineto(126.77020935,134.1728833)
\curveto(126.78019943,134.12288107)(126.79019942,134.06788113)(126.80020935,134.0078833)
\curveto(126.8101994,133.95788124)(126.81519939,133.90288129)(126.81520935,133.8428833)
\curveto(126.82519938,133.78288141)(126.84019937,133.72788147)(126.86020935,133.6778833)
\curveto(126.9101993,133.48788171)(126.97019924,133.31288188)(127.04020935,133.1528833)
\curveto(127.1101991,132.9928822)(127.21519899,132.86288233)(127.35520935,132.7628833)
\curveto(127.48519872,132.66288253)(127.62519858,132.5928826)(127.77520935,132.5528833)
\curveto(127.8051984,132.54288265)(127.83019838,132.53788266)(127.85020935,132.5378833)
\curveto(127.88019833,132.54788265)(127.9101983,132.54788265)(127.94020935,132.5378833)
\curveto(127.96019825,132.53788266)(127.99019822,132.53288266)(128.03020935,132.5228833)
\curveto(128.07019814,132.52288267)(128.1051981,132.52788267)(128.13520935,132.5378833)
\curveto(128.17519803,132.54788265)(128.21519799,132.55288264)(128.25520935,132.5528833)
\curveto(128.29519791,132.55288264)(128.33519787,132.56288263)(128.37520935,132.5828833)
\curveto(128.61519759,132.66288253)(128.8101974,132.7978824)(128.96020935,132.9878833)
\curveto(129.08019713,133.16788203)(129.17019704,133.37288182)(129.23020935,133.6028833)
\curveto(129.25019696,133.67288152)(129.26519694,133.74288145)(129.27520935,133.8128833)
\curveto(129.28519692,133.8928813)(129.30019691,133.97288122)(129.32020935,134.0528833)
\curveto(129.32019689,134.11288108)(129.32519688,134.15788104)(129.33520935,134.1878833)
\curveto(129.33519687,134.20788099)(129.33519687,134.23288096)(129.33520935,134.2628833)
\curveto(129.33519687,134.30288089)(129.34019687,134.33288086)(129.35020935,134.3528833)
\lineto(129.35020935,134.5028833)
}
}
{
\newrgbcolor{curcolor}{0 0 0}
\pscustom[linestyle=none,fillstyle=solid,fillcolor=curcolor]
{
\newpath
\moveto(292.70835693,109.46070557)
\lineto(296.30835693,109.46070557)
\lineto(296.95335693,109.46070557)
\curveto(297.0333504,109.46069514)(297.10835033,109.45569515)(297.17835693,109.44570557)
\curveto(297.24835019,109.44569516)(297.30835013,109.43569517)(297.35835693,109.41570557)
\curveto(297.42835001,109.38569522)(297.48334995,109.32569528)(297.52335693,109.23570557)
\curveto(297.54334989,109.2056954)(297.55334988,109.16569544)(297.55335693,109.11570557)
\lineto(297.55335693,108.98070557)
\curveto(297.56334987,108.87069573)(297.55834988,108.76569584)(297.53835693,108.66570557)
\curveto(297.52834991,108.56569604)(297.49334994,108.49569611)(297.43335693,108.45570557)
\curveto(297.34335009,108.38569622)(297.20835023,108.35069625)(297.02835693,108.35070557)
\curveto(296.84835059,108.36069624)(296.68335075,108.36569624)(296.53335693,108.36570557)
\lineto(294.53835693,108.36570557)
\lineto(294.04335693,108.36570557)
\lineto(293.90835693,108.36570557)
\curveto(293.86835357,108.36569624)(293.82835361,108.36069624)(293.78835693,108.35070557)
\lineto(293.57835693,108.35070557)
\curveto(293.46835397,108.32069628)(293.38835405,108.28069632)(293.33835693,108.23070557)
\curveto(293.28835415,108.19069641)(293.25335418,108.13569647)(293.23335693,108.06570557)
\curveto(293.21335422,108.0056966)(293.19835424,107.93569667)(293.18835693,107.85570557)
\curveto(293.17835426,107.77569683)(293.15835428,107.68569692)(293.12835693,107.58570557)
\curveto(293.07835436,107.38569722)(293.0383544,107.18069742)(293.00835693,106.97070557)
\curveto(292.97835446,106.76069784)(292.9383545,106.55569805)(292.88835693,106.35570557)
\curveto(292.86835457,106.28569832)(292.85835458,106.21569839)(292.85835693,106.14570557)
\curveto(292.85835458,106.08569852)(292.84835459,106.02069858)(292.82835693,105.95070557)
\curveto(292.81835462,105.92069868)(292.80835463,105.88069872)(292.79835693,105.83070557)
\curveto(292.79835464,105.79069881)(292.80335463,105.75069885)(292.81335693,105.71070557)
\curveto(292.8333546,105.66069894)(292.85835458,105.61569899)(292.88835693,105.57570557)
\curveto(292.92835451,105.54569906)(292.98835445,105.54069906)(293.06835693,105.56070557)
\curveto(293.12835431,105.58069902)(293.18835425,105.605699)(293.24835693,105.63570557)
\curveto(293.30835413,105.67569893)(293.36835407,105.71069889)(293.42835693,105.74070557)
\curveto(293.48835395,105.76069884)(293.5383539,105.77569883)(293.57835693,105.78570557)
\curveto(293.76835367,105.86569874)(293.97335346,105.92069868)(294.19335693,105.95070557)
\curveto(294.42335301,105.98069862)(294.65335278,105.99069861)(294.88335693,105.98070557)
\curveto(295.12335231,105.98069862)(295.35335208,105.95569865)(295.57335693,105.90570557)
\curveto(295.79335164,105.86569874)(295.99335144,105.8056988)(296.17335693,105.72570557)
\curveto(296.22335121,105.7056989)(296.26835117,105.68569892)(296.30835693,105.66570557)
\curveto(296.35835108,105.64569896)(296.40835103,105.62069898)(296.45835693,105.59070557)
\curveto(296.80835063,105.38069922)(297.08835035,105.15069945)(297.29835693,104.90070557)
\curveto(297.51834992,104.65069995)(297.71334972,104.32570028)(297.88335693,103.92570557)
\curveto(297.9333495,103.81570079)(297.96834947,103.7057009)(297.98835693,103.59570557)
\curveto(298.00834943,103.48570112)(298.0333494,103.37070123)(298.06335693,103.25070557)
\curveto(298.07334936,103.22070138)(298.07834936,103.17570143)(298.07835693,103.11570557)
\curveto(298.09834934,103.05570155)(298.10834933,102.98570162)(298.10835693,102.90570557)
\curveto(298.10834933,102.83570177)(298.11834932,102.77070183)(298.13835693,102.71070557)
\lineto(298.13835693,102.54570557)
\curveto(298.14834929,102.49570211)(298.15334928,102.42570218)(298.15335693,102.33570557)
\curveto(298.15334928,102.24570236)(298.14334929,102.17570243)(298.12335693,102.12570557)
\curveto(298.10334933,102.06570254)(298.09834934,102.0057026)(298.10835693,101.94570557)
\curveto(298.11834932,101.89570271)(298.11334932,101.84570276)(298.09335693,101.79570557)
\curveto(298.05334938,101.63570297)(298.01834942,101.48570312)(297.98835693,101.34570557)
\curveto(297.95834948,101.2057034)(297.91334952,101.07070353)(297.85335693,100.94070557)
\curveto(297.69334974,100.57070403)(297.47334996,100.23570437)(297.19335693,99.93570557)
\curveto(296.91335052,99.63570497)(296.59335084,99.4057052)(296.23335693,99.24570557)
\curveto(296.06335137,99.16570544)(295.86335157,99.09070551)(295.63335693,99.02070557)
\curveto(295.52335191,98.98070562)(295.40835203,98.95570565)(295.28835693,98.94570557)
\curveto(295.16835227,98.93570567)(295.04835239,98.91570569)(294.92835693,98.88570557)
\curveto(294.87835256,98.86570574)(294.82335261,98.86570574)(294.76335693,98.88570557)
\curveto(294.70335273,98.89570571)(294.64335279,98.89070571)(294.58335693,98.87070557)
\curveto(294.48335295,98.85070575)(294.38335305,98.85070575)(294.28335693,98.87070557)
\lineto(294.14835693,98.87070557)
\curveto(294.09835334,98.89070571)(294.0383534,98.9007057)(293.96835693,98.90070557)
\curveto(293.90835353,98.89070571)(293.85335358,98.89570571)(293.80335693,98.91570557)
\curveto(293.76335367,98.92570568)(293.72835371,98.93070567)(293.69835693,98.93070557)
\curveto(293.66835377,98.93070567)(293.6333538,98.93570567)(293.59335693,98.94570557)
\lineto(293.32335693,99.00570557)
\curveto(293.2333542,99.02570558)(293.14835429,99.05570555)(293.06835693,99.09570557)
\curveto(292.72835471,99.23570537)(292.438355,99.39070521)(292.19835693,99.56070557)
\curveto(291.95835548,99.74070486)(291.7383557,99.97070463)(291.53835693,100.25070557)
\curveto(291.38835605,100.48070412)(291.27335616,100.72070388)(291.19335693,100.97070557)
\curveto(291.17335626,101.02070358)(291.16335627,101.06570354)(291.16335693,101.10570557)
\curveto(291.16335627,101.15570345)(291.15335628,101.2057034)(291.13335693,101.25570557)
\curveto(291.11335632,101.31570329)(291.09835634,101.39570321)(291.08835693,101.49570557)
\curveto(291.08835635,101.59570301)(291.10835633,101.67070293)(291.14835693,101.72070557)
\curveto(291.19835624,101.8007028)(291.27835616,101.84570276)(291.38835693,101.85570557)
\curveto(291.49835594,101.86570274)(291.61335582,101.87070273)(291.73335693,101.87070557)
\lineto(291.89835693,101.87070557)
\curveto(291.95835548,101.87070273)(292.01335542,101.86070274)(292.06335693,101.84070557)
\curveto(292.15335528,101.82070278)(292.22335521,101.78070282)(292.27335693,101.72070557)
\curveto(292.34335509,101.63070297)(292.38835505,101.52070308)(292.40835693,101.39070557)
\curveto(292.438355,101.27070333)(292.48335495,101.16570344)(292.54335693,101.07570557)
\curveto(292.7333547,100.73570387)(292.99335444,100.46570414)(293.32335693,100.26570557)
\curveto(293.42335401,100.2057044)(293.52835391,100.15570445)(293.63835693,100.11570557)
\curveto(293.75835368,100.08570452)(293.87835356,100.05070455)(293.99835693,100.01070557)
\curveto(294.16835327,99.96070464)(294.37335306,99.94070466)(294.61335693,99.95070557)
\curveto(294.86335257,99.97070463)(295.06335237,100.0057046)(295.21335693,100.05570557)
\curveto(295.58335185,100.17570443)(295.87335156,100.33570427)(296.08335693,100.53570557)
\curveto(296.30335113,100.74570386)(296.48335095,101.02570358)(296.62335693,101.37570557)
\curveto(296.67335076,101.47570313)(296.70335073,101.58070302)(296.71335693,101.69070557)
\curveto(296.7333507,101.8007028)(296.75835068,101.91570269)(296.78835693,102.03570557)
\lineto(296.78835693,102.14070557)
\curveto(296.79835064,102.18070242)(296.80335063,102.22070238)(296.80335693,102.26070557)
\curveto(296.81335062,102.29070231)(296.81335062,102.32570228)(296.80335693,102.36570557)
\lineto(296.80335693,102.48570557)
\curveto(296.80335063,102.74570186)(296.77335066,102.99070161)(296.71335693,103.22070557)
\curveto(296.60335083,103.57070103)(296.44835099,103.86570074)(296.24835693,104.10570557)
\curveto(296.04835139,104.35570025)(295.78835165,104.55070005)(295.46835693,104.69070557)
\lineto(295.28835693,104.75070557)
\curveto(295.2383522,104.77069983)(295.17835226,104.79069981)(295.10835693,104.81070557)
\curveto(295.05835238,104.83069977)(294.99835244,104.84069976)(294.92835693,104.84070557)
\curveto(294.86835257,104.85069975)(294.80335263,104.86569974)(294.73335693,104.88570557)
\lineto(294.58335693,104.88570557)
\curveto(294.54335289,104.9056997)(294.48835295,104.91569969)(294.41835693,104.91570557)
\curveto(294.35835308,104.91569969)(294.30335313,104.9056997)(294.25335693,104.88570557)
\lineto(294.14835693,104.88570557)
\curveto(294.11835332,104.88569972)(294.08335335,104.88069972)(294.04335693,104.87070557)
\lineto(293.80335693,104.81070557)
\curveto(293.72335371,104.8006998)(293.64335379,104.78069982)(293.56335693,104.75070557)
\curveto(293.32335411,104.65069995)(293.09335434,104.51570009)(292.87335693,104.34570557)
\curveto(292.78335465,104.27570033)(292.69835474,104.2007004)(292.61835693,104.12070557)
\curveto(292.5383549,104.05070055)(292.438355,103.99570061)(292.31835693,103.95570557)
\curveto(292.22835521,103.92570068)(292.08835535,103.91570069)(291.89835693,103.92570557)
\curveto(291.71835572,103.93570067)(291.59835584,103.96070064)(291.53835693,104.00070557)
\curveto(291.48835595,104.04070056)(291.44835599,104.1007005)(291.41835693,104.18070557)
\curveto(291.39835604,104.26070034)(291.39835604,104.34570026)(291.41835693,104.43570557)
\curveto(291.44835599,104.55570005)(291.46835597,104.67569993)(291.47835693,104.79570557)
\curveto(291.49835594,104.92569968)(291.52335591,105.05069955)(291.55335693,105.17070557)
\curveto(291.57335586,105.21069939)(291.57835586,105.24569936)(291.56835693,105.27570557)
\curveto(291.56835587,105.31569929)(291.57835586,105.36069924)(291.59835693,105.41070557)
\curveto(291.61835582,105.5006991)(291.6333558,105.59069901)(291.64335693,105.68070557)
\curveto(291.65335578,105.78069882)(291.67335576,105.87569873)(291.70335693,105.96570557)
\curveto(291.71335572,106.02569858)(291.71835572,106.08569852)(291.71835693,106.14570557)
\curveto(291.72835571,106.2056984)(291.74335569,106.26569834)(291.76335693,106.32570557)
\curveto(291.81335562,106.52569808)(291.84835559,106.73069787)(291.86835693,106.94070557)
\curveto(291.89835554,107.16069744)(291.9383555,107.37069723)(291.98835693,107.57070557)
\curveto(292.01835542,107.67069693)(292.0383554,107.77069683)(292.04835693,107.87070557)
\curveto(292.05835538,107.97069663)(292.07335536,108.07069653)(292.09335693,108.17070557)
\curveto(292.10335533,108.2006964)(292.10835533,108.24069636)(292.10835693,108.29070557)
\curveto(292.1383553,108.4006962)(292.15835528,108.5056961)(292.16835693,108.60570557)
\curveto(292.18835525,108.71569589)(292.21335522,108.82569578)(292.24335693,108.93570557)
\curveto(292.26335517,109.01569559)(292.27835516,109.08569552)(292.28835693,109.14570557)
\curveto(292.29835514,109.21569539)(292.32335511,109.27569533)(292.36335693,109.32570557)
\curveto(292.38335505,109.35569525)(292.41335502,109.37569523)(292.45335693,109.38570557)
\curveto(292.49335494,109.4056952)(292.5383549,109.42569518)(292.58835693,109.44570557)
\curveto(292.64835479,109.44569516)(292.68835475,109.45069515)(292.70835693,109.46070557)
}
}
{
\newrgbcolor{curcolor}{0 0 0}
\pscustom[linestyle=none,fillstyle=solid,fillcolor=curcolor]
{
\newpath
\moveto(300.00796631,109.46070557)
\lineto(304.80796631,109.46070557)
\lineto(305.81296631,109.46070557)
\curveto(305.95295921,109.46069514)(306.07295909,109.45069515)(306.17296631,109.43070557)
\curveto(306.28295888,109.42069518)(306.3629588,109.37569523)(306.41296631,109.29570557)
\curveto(306.43295873,109.25569535)(306.44295872,109.2056954)(306.44296631,109.14570557)
\curveto(306.45295871,109.08569552)(306.4579587,109.02069558)(306.45796631,108.95070557)
\lineto(306.45796631,108.68070557)
\curveto(306.4579587,108.59069601)(306.44795871,108.51069609)(306.42796631,108.44070557)
\curveto(306.38795877,108.36069624)(306.34295882,108.29069631)(306.29296631,108.23070557)
\lineto(306.14296631,108.05070557)
\curveto(306.11295905,108.0006966)(306.07795908,107.96069664)(306.03796631,107.93070557)
\curveto(305.99795916,107.9006967)(305.9579592,107.86069674)(305.91796631,107.81070557)
\curveto(305.83795932,107.7006969)(305.75295941,107.59069701)(305.66296631,107.48070557)
\curveto(305.57295959,107.38069722)(305.48795967,107.27569733)(305.40796631,107.16570557)
\curveto(305.26795989,106.96569764)(305.12796003,106.75569785)(304.98796631,106.53570557)
\curveto(304.84796031,106.32569828)(304.70796045,106.11069849)(304.56796631,105.89070557)
\curveto(304.51796064,105.8006988)(304.46796069,105.7056989)(304.41796631,105.60570557)
\curveto(304.36796079,105.5056991)(304.31296085,105.41069919)(304.25296631,105.32070557)
\curveto(304.23296093,105.3006993)(304.22296094,105.27569933)(304.22296631,105.24570557)
\curveto(304.22296094,105.21569939)(304.21296095,105.19069941)(304.19296631,105.17070557)
\curveto(304.12296104,105.07069953)(304.0579611,104.95569965)(303.99796631,104.82570557)
\curveto(303.93796122,104.7056999)(303.88296128,104.59070001)(303.83296631,104.48070557)
\curveto(303.73296143,104.25070035)(303.63796152,104.01570059)(303.54796631,103.77570557)
\curveto(303.4579617,103.53570107)(303.3579618,103.29570131)(303.24796631,103.05570557)
\curveto(303.22796193,103.0057016)(303.21296195,102.96070164)(303.20296631,102.92070557)
\curveto(303.20296196,102.88070172)(303.19296197,102.83570177)(303.17296631,102.78570557)
\curveto(303.12296204,102.66570194)(303.07796208,102.54070206)(303.03796631,102.41070557)
\curveto(303.00796215,102.29070231)(302.97296219,102.17070243)(302.93296631,102.05070557)
\curveto(302.85296231,101.82070278)(302.78796237,101.58070302)(302.73796631,101.33070557)
\curveto(302.69796246,101.09070351)(302.64796251,100.85070375)(302.58796631,100.61070557)
\curveto(302.54796261,100.46070414)(302.52296264,100.31070429)(302.51296631,100.16070557)
\curveto(302.50296266,100.01070459)(302.48296268,99.86070474)(302.45296631,99.71070557)
\curveto(302.44296272,99.67070493)(302.43796272,99.61070499)(302.43796631,99.53070557)
\curveto(302.40796275,99.41070519)(302.37796278,99.31070529)(302.34796631,99.23070557)
\curveto(302.31796284,99.15070545)(302.24796291,99.09570551)(302.13796631,99.06570557)
\curveto(302.08796307,99.04570556)(302.03296313,99.03570557)(301.97296631,99.03570557)
\lineto(301.77796631,99.03570557)
\curveto(301.63796352,99.03570557)(301.49796366,99.04070556)(301.35796631,99.05070557)
\curveto(301.22796393,99.06070554)(301.13296403,99.1057055)(301.07296631,99.18570557)
\curveto(301.03296413,99.24570536)(301.01296415,99.33070527)(301.01296631,99.44070557)
\curveto(301.02296414,99.55070505)(301.03796412,99.64570496)(301.05796631,99.72570557)
\lineto(301.05796631,99.80070557)
\curveto(301.06796409,99.83070477)(301.07296409,99.86070474)(301.07296631,99.89070557)
\curveto(301.09296407,99.97070463)(301.10296406,100.04570456)(301.10296631,100.11570557)
\curveto(301.10296406,100.18570442)(301.11296405,100.25570435)(301.13296631,100.32570557)
\curveto(301.18296398,100.51570409)(301.22296394,100.7007039)(301.25296631,100.88070557)
\curveto(301.28296388,101.07070353)(301.32296384,101.25070335)(301.37296631,101.42070557)
\curveto(301.39296377,101.47070313)(301.40296376,101.51070309)(301.40296631,101.54070557)
\curveto(301.40296376,101.57070303)(301.40796375,101.605703)(301.41796631,101.64570557)
\curveto(301.51796364,101.94570266)(301.60796355,102.24070236)(301.68796631,102.53070557)
\curveto(301.77796338,102.82070178)(301.88296328,103.1007015)(302.00296631,103.37070557)
\curveto(302.2629629,103.95070065)(302.53296263,104.5007001)(302.81296631,105.02070557)
\curveto(303.09296207,105.55069905)(303.40296176,106.05569855)(303.74296631,106.53570557)
\curveto(303.88296128,106.73569787)(304.03296113,106.92569768)(304.19296631,107.10570557)
\curveto(304.35296081,107.29569731)(304.50296066,107.48569712)(304.64296631,107.67570557)
\curveto(304.68296048,107.72569688)(304.71796044,107.77069683)(304.74796631,107.81070557)
\curveto(304.78796037,107.86069674)(304.82296034,107.91069669)(304.85296631,107.96070557)
\curveto(304.8629603,107.98069662)(304.87296029,108.0056966)(304.88296631,108.03570557)
\curveto(304.90296026,108.06569654)(304.90296026,108.09569651)(304.88296631,108.12570557)
\curveto(304.8629603,108.18569642)(304.82796033,108.22069638)(304.77796631,108.23070557)
\curveto(304.72796043,108.25069635)(304.67796048,108.27069633)(304.62796631,108.29070557)
\lineto(304.52296631,108.29070557)
\curveto(304.48296068,108.3006963)(304.43296073,108.3006963)(304.37296631,108.29070557)
\lineto(304.22296631,108.29070557)
\lineto(303.62296631,108.29070557)
\lineto(300.98296631,108.29070557)
\lineto(300.24796631,108.29070557)
\lineto(300.00796631,108.29070557)
\curveto(299.93796522,108.3006963)(299.87796528,108.31569629)(299.82796631,108.33570557)
\curveto(299.73796542,108.37569623)(299.67796548,108.43569617)(299.64796631,108.51570557)
\curveto(299.59796556,108.61569599)(299.58296558,108.76069584)(299.60296631,108.95070557)
\curveto(299.62296554,109.15069545)(299.6579655,109.28569532)(299.70796631,109.35570557)
\curveto(299.72796543,109.37569523)(299.75296541,109.39069521)(299.78296631,109.40070557)
\lineto(299.90296631,109.46070557)
\curveto(299.92296524,109.46069514)(299.93796522,109.45569515)(299.94796631,109.44570557)
\curveto(299.96796519,109.44569516)(299.98796517,109.45069515)(300.00796631,109.46070557)
}
}
{
\newrgbcolor{curcolor}{0 0 0}
\pscustom[linestyle=none,fillstyle=solid,fillcolor=curcolor]
{
\newpath
\moveto(308.85257568,100.68570557)
\lineto(309.15257568,100.68570557)
\curveto(309.26257362,100.69570391)(309.36757352,100.69570391)(309.46757568,100.68570557)
\curveto(309.57757331,100.68570392)(309.67757321,100.67570393)(309.76757568,100.65570557)
\curveto(309.85757303,100.64570396)(309.92757296,100.62070398)(309.97757568,100.58070557)
\curveto(309.99757289,100.56070404)(310.01257287,100.53070407)(310.02257568,100.49070557)
\curveto(310.04257284,100.45070415)(310.06257282,100.4057042)(310.08257568,100.35570557)
\lineto(310.08257568,100.28070557)
\curveto(310.09257279,100.23070437)(310.09257279,100.17570443)(310.08257568,100.11570557)
\lineto(310.08257568,99.96570557)
\lineto(310.08257568,99.48570557)
\curveto(310.0825728,99.31570529)(310.04257284,99.19570541)(309.96257568,99.12570557)
\curveto(309.89257299,99.07570553)(309.80257308,99.05070555)(309.69257568,99.05070557)
\lineto(309.36257568,99.05070557)
\lineto(308.91257568,99.05070557)
\curveto(308.76257412,99.05070555)(308.64757424,99.08070552)(308.56757568,99.14070557)
\curveto(308.52757436,99.17070543)(308.49757439,99.22070538)(308.47757568,99.29070557)
\curveto(308.45757443,99.37070523)(308.44257444,99.45570515)(308.43257568,99.54570557)
\lineto(308.43257568,99.83070557)
\curveto(308.44257444,99.93070467)(308.44757444,100.01570459)(308.44757568,100.08570557)
\lineto(308.44757568,100.28070557)
\curveto(308.44757444,100.34070426)(308.45757443,100.39570421)(308.47757568,100.44570557)
\curveto(308.51757437,100.55570405)(308.5875743,100.62570398)(308.68757568,100.65570557)
\curveto(308.71757417,100.65570395)(308.77257411,100.66570394)(308.85257568,100.68570557)
}
}
{
\newrgbcolor{curcolor}{0 0 0}
\pscustom[linestyle=none,fillstyle=solid,fillcolor=curcolor]
{
\newpath
\moveto(318.99773193,104.64570557)
\curveto(318.9977243,104.56570004)(319.00272429,104.48570012)(319.01273193,104.40570557)
\curveto(319.02272427,104.32570028)(319.01772428,104.25070035)(318.99773193,104.18070557)
\curveto(318.97772432,104.14070046)(318.97272432,104.09570051)(318.98273193,104.04570557)
\curveto(318.9927243,104.0057006)(318.9927243,103.96570064)(318.98273193,103.92570557)
\lineto(318.98273193,103.77570557)
\curveto(318.97272432,103.68570092)(318.96772433,103.59570101)(318.96773193,103.50570557)
\curveto(318.96772433,103.42570118)(318.96272433,103.34570126)(318.95273193,103.26570557)
\lineto(318.92273193,103.02570557)
\curveto(318.91272438,102.95570165)(318.90272439,102.88070172)(318.89273193,102.80070557)
\curveto(318.88272441,102.76070184)(318.87772442,102.72070188)(318.87773193,102.68070557)
\curveto(318.87772442,102.64070196)(318.87272442,102.59570201)(318.86273193,102.54570557)
\curveto(318.82272447,102.4057022)(318.7927245,102.26570234)(318.77273193,102.12570557)
\curveto(318.76272453,101.98570262)(318.73272456,101.85070275)(318.68273193,101.72070557)
\curveto(318.63272466,101.55070305)(318.57772472,101.38570322)(318.51773193,101.22570557)
\curveto(318.46772483,101.06570354)(318.40772489,100.91070369)(318.33773193,100.76070557)
\curveto(318.31772498,100.7007039)(318.28772501,100.64070396)(318.24773193,100.58070557)
\lineto(318.15773193,100.43070557)
\curveto(317.95772534,100.11070449)(317.74272555,99.84570476)(317.51273193,99.63570557)
\curveto(317.28272601,99.42570518)(316.98772631,99.24570536)(316.62773193,99.09570557)
\curveto(316.50772679,99.04570556)(316.37772692,99.01070559)(316.23773193,98.99070557)
\curveto(316.10772719,98.97070563)(315.97272732,98.94570566)(315.83273193,98.91570557)
\curveto(315.77272752,98.9057057)(315.71272758,98.9007057)(315.65273193,98.90070557)
\curveto(315.5927277,98.9007057)(315.52772777,98.89570571)(315.45773193,98.88570557)
\curveto(315.42772787,98.87570573)(315.37772792,98.87570573)(315.30773193,98.88570557)
\lineto(315.15773193,98.88570557)
\lineto(315.00773193,98.88570557)
\curveto(314.92772837,98.9057057)(314.84272845,98.92070568)(314.75273193,98.93070557)
\curveto(314.67272862,98.93070567)(314.5977287,98.94070566)(314.52773193,98.96070557)
\curveto(314.48772881,98.97070563)(314.45272884,98.97570563)(314.42273193,98.97570557)
\curveto(314.40272889,98.96570564)(314.37772892,98.97070563)(314.34773193,98.99070557)
\lineto(314.07773193,99.05070557)
\curveto(313.98772931,99.08070552)(313.90272939,99.11070549)(313.82273193,99.14070557)
\curveto(313.24273005,99.38070522)(312.80773049,99.75070485)(312.51773193,100.25070557)
\curveto(312.43773086,100.38070422)(312.37273092,100.51570409)(312.32273193,100.65570557)
\curveto(312.28273101,100.79570381)(312.23773106,100.94570366)(312.18773193,101.10570557)
\curveto(312.16773113,101.18570342)(312.16273113,101.26570334)(312.17273193,101.34570557)
\curveto(312.1927311,101.42570318)(312.22773107,101.48070312)(312.27773193,101.51070557)
\curveto(312.30773099,101.53070307)(312.36273093,101.54570306)(312.44273193,101.55570557)
\curveto(312.52273077,101.57570303)(312.60773069,101.58570302)(312.69773193,101.58570557)
\curveto(312.78773051,101.59570301)(312.87273042,101.59570301)(312.95273193,101.58570557)
\curveto(313.04273025,101.57570303)(313.11273018,101.56570304)(313.16273193,101.55570557)
\curveto(313.18273011,101.54570306)(313.20773009,101.53070307)(313.23773193,101.51070557)
\curveto(313.27773002,101.49070311)(313.30772999,101.47070313)(313.32773193,101.45070557)
\curveto(313.38772991,101.37070323)(313.43272986,101.27570333)(313.46273193,101.16570557)
\curveto(313.50272979,101.05570355)(313.54772975,100.95570365)(313.59773193,100.86570557)
\curveto(313.84772945,100.47570413)(314.21772908,100.2057044)(314.70773193,100.05570557)
\curveto(314.77772852,100.03570457)(314.84772845,100.02070458)(314.91773193,100.01070557)
\curveto(314.9977283,100.01070459)(315.07772822,100.0007046)(315.15773193,99.98070557)
\curveto(315.1977281,99.97070463)(315.25272804,99.96570464)(315.32273193,99.96570557)
\curveto(315.40272789,99.96570464)(315.45772784,99.97070463)(315.48773193,99.98070557)
\curveto(315.51772778,99.99070461)(315.54772775,99.99570461)(315.57773193,99.99570557)
\lineto(315.68273193,99.99570557)
\curveto(315.76272753,100.01570459)(315.83772746,100.03570457)(315.90773193,100.05570557)
\curveto(315.98772731,100.07570453)(316.06272723,100.1007045)(316.13273193,100.13070557)
\curveto(316.48272681,100.28070432)(316.75272654,100.49570411)(316.94273193,100.77570557)
\curveto(317.13272616,101.05570355)(317.28772601,101.38070322)(317.40773193,101.75070557)
\curveto(317.43772586,101.83070277)(317.45772584,101.9057027)(317.46773193,101.97570557)
\curveto(317.48772581,102.04570256)(317.50772579,102.12070248)(317.52773193,102.20070557)
\curveto(317.54772575,102.29070231)(317.56272573,102.38570222)(317.57273193,102.48570557)
\curveto(317.5927257,102.59570201)(317.61272568,102.7007019)(317.63273193,102.80070557)
\curveto(317.64272565,102.85070175)(317.64772565,102.9007017)(317.64773193,102.95070557)
\curveto(317.65772564,103.01070159)(317.66272563,103.06570154)(317.66273193,103.11570557)
\curveto(317.68272561,103.17570143)(317.6927256,103.25070135)(317.69273193,103.34070557)
\curveto(317.6927256,103.44070116)(317.68272561,103.52070108)(317.66273193,103.58070557)
\curveto(317.63272566,103.67070093)(317.58272571,103.71070089)(317.51273193,103.70070557)
\curveto(317.45272584,103.69070091)(317.3977259,103.66070094)(317.34773193,103.61070557)
\curveto(317.26772603,103.56070104)(317.1977261,103.5007011)(317.13773193,103.43070557)
\curveto(317.08772621,103.36070124)(317.02272627,103.3007013)(316.94273193,103.25070557)
\curveto(316.78272651,103.14070146)(316.61772668,103.04070156)(316.44773193,102.95070557)
\curveto(316.27772702,102.87070173)(316.08272721,102.8007018)(315.86273193,102.74070557)
\curveto(315.76272753,102.71070189)(315.66272763,102.69570191)(315.56273193,102.69570557)
\curveto(315.47272782,102.69570191)(315.37272792,102.68570192)(315.26273193,102.66570557)
\lineto(315.11273193,102.66570557)
\curveto(315.06272823,102.68570192)(315.01272828,102.69070191)(314.96273193,102.68070557)
\curveto(314.92272837,102.67070193)(314.88272841,102.67070193)(314.84273193,102.68070557)
\curveto(314.81272848,102.69070191)(314.76772853,102.69570191)(314.70773193,102.69570557)
\curveto(314.64772865,102.7057019)(314.58272871,102.71570189)(314.51273193,102.72570557)
\lineto(314.33273193,102.75570557)
\curveto(313.88272941,102.87570173)(313.50272979,103.04070156)(313.19273193,103.25070557)
\curveto(312.92273037,103.44070116)(312.6927306,103.67070093)(312.50273193,103.94070557)
\curveto(312.32273097,104.22070038)(312.17773112,104.53570007)(312.06773193,104.88570557)
\lineto(312.00773193,105.09570557)
\curveto(311.9977313,105.17569943)(311.98273131,105.25569935)(311.96273193,105.33570557)
\curveto(311.95273134,105.36569924)(311.94773135,105.39569921)(311.94773193,105.42570557)
\curveto(311.94773135,105.45569915)(311.94273135,105.48569912)(311.93273193,105.51570557)
\curveto(311.92273137,105.57569903)(311.91773138,105.63569897)(311.91773193,105.69570557)
\curveto(311.91773138,105.76569884)(311.90773139,105.82569878)(311.88773193,105.87570557)
\lineto(311.88773193,106.05570557)
\curveto(311.87773142,106.1056985)(311.87273142,106.17569843)(311.87273193,106.26570557)
\curveto(311.87273142,106.35569825)(311.88273141,106.42569818)(311.90273193,106.47570557)
\lineto(311.90273193,106.64070557)
\curveto(311.92273137,106.72069788)(311.93273136,106.79569781)(311.93273193,106.86570557)
\curveto(311.94273135,106.93569767)(311.95773134,107.0056976)(311.97773193,107.07570557)
\curveto(312.03773126,107.27569733)(312.0977312,107.46569714)(312.15773193,107.64570557)
\curveto(312.22773107,107.82569678)(312.31773098,107.99569661)(312.42773193,108.15570557)
\curveto(312.46773083,108.22569638)(312.50773079,108.29069631)(312.54773193,108.35070557)
\lineto(312.69773193,108.53070557)
\curveto(312.71773058,108.54069606)(312.73773056,108.55569605)(312.75773193,108.57570557)
\curveto(312.84773045,108.7056959)(312.95773034,108.81569579)(313.08773193,108.90570557)
\curveto(313.34772995,109.1056955)(313.61272968,109.26069534)(313.88273193,109.37070557)
\curveto(313.96272933,109.41069519)(314.04272925,109.44069516)(314.12273193,109.46070557)
\curveto(314.21272908,109.49069511)(314.30272899,109.51569509)(314.39273193,109.53570557)
\curveto(314.4927288,109.56569504)(314.5927287,109.58569502)(314.69273193,109.59570557)
\curveto(314.7927285,109.605695)(314.8977284,109.62069498)(315.00773193,109.64070557)
\curveto(315.03772826,109.65069495)(315.07772822,109.65069495)(315.12773193,109.64070557)
\curveto(315.18772811,109.63069497)(315.22772807,109.63569497)(315.24773193,109.65570557)
\curveto(315.96772733,109.67569493)(316.56772673,109.56069504)(317.04773193,109.31070557)
\curveto(317.52772577,109.06069554)(317.90272539,108.72069588)(318.17273193,108.29070557)
\curveto(318.26272503,108.15069645)(318.34272495,108.0056966)(318.41273193,107.85570557)
\curveto(318.48272481,107.7056969)(318.55272474,107.54569706)(318.62273193,107.37570557)
\curveto(318.67272462,107.23569737)(318.71272458,107.08569752)(318.74273193,106.92570557)
\curveto(318.77272452,106.76569784)(318.80772449,106.605698)(318.84773193,106.44570557)
\curveto(318.86772443,106.39569821)(318.87772442,106.34069826)(318.87773193,106.28070557)
\curveto(318.87772442,106.23069837)(318.88272441,106.18069842)(318.89273193,106.13070557)
\curveto(318.91272438,106.07069853)(318.92272437,106.0056986)(318.92273193,105.93570557)
\curveto(318.92272437,105.87569873)(318.93272436,105.82069878)(318.95273193,105.77070557)
\lineto(318.95273193,105.60570557)
\curveto(318.97272432,105.55569905)(318.97772432,105.5056991)(318.96773193,105.45570557)
\curveto(318.95772434,105.4056992)(318.96272433,105.35569925)(318.98273193,105.30570557)
\curveto(318.98272431,105.28569932)(318.97772432,105.26069934)(318.96773193,105.23070557)
\curveto(318.96772433,105.2006994)(318.97272432,105.17569943)(318.98273193,105.15570557)
\curveto(318.9927243,105.12569948)(318.9927243,105.09069951)(318.98273193,105.05070557)
\curveto(318.98272431,105.01069959)(318.98772431,104.97069963)(318.99773193,104.93070557)
\curveto(319.00772429,104.89069971)(319.00772429,104.84569976)(318.99773193,104.79570557)
\lineto(318.99773193,104.64570557)
\moveto(317.49773193,105.95070557)
\curveto(317.50772579,106.0006986)(317.51272578,106.06069854)(317.51273193,106.13070557)
\curveto(317.51272578,106.2006984)(317.50772579,106.26069834)(317.49773193,106.31070557)
\curveto(317.48772581,106.36069824)(317.48272581,106.43569817)(317.48273193,106.53570557)
\curveto(317.46272583,106.61569799)(317.44272585,106.69069791)(317.42273193,106.76070557)
\curveto(317.41272588,106.83069777)(317.3977259,106.9006977)(317.37773193,106.97070557)
\curveto(317.23772606,107.4006972)(317.04272625,107.73569687)(316.79273193,107.97570557)
\curveto(316.55272674,108.21569639)(316.20772709,108.39569621)(315.75773193,108.51570557)
\curveto(315.66772763,108.53569607)(315.56772773,108.54569606)(315.45773193,108.54570557)
\lineto(315.12773193,108.54570557)
\curveto(315.10772819,108.52569608)(315.07272822,108.51569609)(315.02273193,108.51570557)
\curveto(314.97272832,108.52569608)(314.92772837,108.52569608)(314.88773193,108.51570557)
\curveto(314.80772849,108.49569611)(314.73272856,108.47569613)(314.66273193,108.45570557)
\lineto(314.45273193,108.39570557)
\curveto(314.16272913,108.26569634)(313.93272936,108.08569652)(313.76273193,107.85570557)
\curveto(313.5927297,107.63569697)(313.45772984,107.37569723)(313.35773193,107.07570557)
\curveto(313.32772997,106.98569762)(313.30272999,106.89069771)(313.28273193,106.79070557)
\curveto(313.27273002,106.7006979)(313.25773004,106.605698)(313.23773193,106.50570557)
\lineto(313.23773193,106.37070557)
\curveto(313.20773009,106.26069834)(313.1977301,106.12069848)(313.20773193,105.95070557)
\curveto(313.22773007,105.79069881)(313.24773005,105.66069894)(313.26773193,105.56070557)
\curveto(313.28773001,105.5006991)(313.30272999,105.44069916)(313.31273193,105.38070557)
\curveto(313.32272997,105.33069927)(313.33772996,105.28069932)(313.35773193,105.23070557)
\curveto(313.43772986,105.03069957)(313.53272976,104.84069976)(313.64273193,104.66070557)
\curveto(313.76272953,104.48070012)(313.90272939,104.33570027)(314.06273193,104.22570557)
\curveto(314.11272918,104.17570043)(314.16772913,104.13570047)(314.22773193,104.10570557)
\curveto(314.28772901,104.07570053)(314.34772895,104.04070056)(314.40773193,104.00070557)
\curveto(314.55772874,103.92070068)(314.74272855,103.85570075)(314.96273193,103.80570557)
\curveto(315.01272828,103.78570082)(315.05272824,103.78070082)(315.08273193,103.79070557)
\curveto(315.12272817,103.8007008)(315.16772813,103.79570081)(315.21773193,103.77570557)
\curveto(315.25772804,103.76570084)(315.31272798,103.76070084)(315.38273193,103.76070557)
\curveto(315.45272784,103.76070084)(315.51272778,103.76570084)(315.56273193,103.77570557)
\curveto(315.66272763,103.79570081)(315.75772754,103.81070079)(315.84773193,103.82070557)
\curveto(315.93772736,103.84070076)(316.02772727,103.87070073)(316.11773193,103.91070557)
\curveto(316.65772664,104.13070047)(317.05272624,104.52570008)(317.30273193,105.09570557)
\curveto(317.35272594,105.19569941)(317.38772591,105.29569931)(317.40773193,105.39570557)
\curveto(317.42772587,105.5056991)(317.45272584,105.61569899)(317.48273193,105.72570557)
\curveto(317.48272581,105.82569878)(317.48772581,105.9006987)(317.49773193,105.95070557)
}
}
{
\newrgbcolor{curcolor}{0 0 0}
\pscustom[linestyle=none,fillstyle=solid,fillcolor=curcolor]
{
\newpath
\moveto(330.21234131,107.57070557)
\curveto(330.01233101,107.28069732)(329.80233122,106.99569761)(329.58234131,106.71570557)
\curveto(329.37233165,106.43569817)(329.16733185,106.15069845)(328.96734131,105.86070557)
\curveto(328.36733265,105.01069959)(327.76233326,104.17070043)(327.15234131,103.34070557)
\curveto(326.54233448,102.52070208)(325.93733508,101.68570292)(325.33734131,100.83570557)
\lineto(324.82734131,100.11570557)
\lineto(324.31734131,99.42570557)
\curveto(324.23733678,99.31570529)(324.15733686,99.2007054)(324.07734131,99.08070557)
\curveto(323.99733702,98.96070564)(323.90233712,98.86570574)(323.79234131,98.79570557)
\curveto(323.75233727,98.77570583)(323.68733733,98.76070584)(323.59734131,98.75070557)
\curveto(323.5173375,98.73070587)(323.42733759,98.72070588)(323.32734131,98.72070557)
\curveto(323.22733779,98.72070588)(323.13233789,98.72570588)(323.04234131,98.73570557)
\curveto(322.96233806,98.74570586)(322.90233812,98.76570584)(322.86234131,98.79570557)
\curveto(322.83233819,98.81570579)(322.80733821,98.85070575)(322.78734131,98.90070557)
\curveto(322.77733824,98.94070566)(322.78233824,98.98570562)(322.80234131,99.03570557)
\curveto(322.84233818,99.11570549)(322.88733813,99.19070541)(322.93734131,99.26070557)
\curveto(322.99733802,99.34070526)(323.05233797,99.42070518)(323.10234131,99.50070557)
\curveto(323.34233768,99.84070476)(323.58733743,100.17570443)(323.83734131,100.50570557)
\curveto(324.08733693,100.83570377)(324.32733669,101.17070343)(324.55734131,101.51070557)
\curveto(324.7173363,101.73070287)(324.87733614,101.94570266)(325.03734131,102.15570557)
\curveto(325.19733582,102.36570224)(325.35733566,102.58070202)(325.51734131,102.80070557)
\curveto(325.87733514,103.32070128)(326.24233478,103.83070077)(326.61234131,104.33070557)
\curveto(326.98233404,104.83069977)(327.35233367,105.34069926)(327.72234131,105.86070557)
\curveto(327.86233316,106.06069854)(328.00233302,106.25569835)(328.14234131,106.44570557)
\curveto(328.29233273,106.63569797)(328.43733258,106.83069777)(328.57734131,107.03070557)
\curveto(328.78733223,107.33069727)(329.00233202,107.63069697)(329.22234131,107.93070557)
\lineto(329.88234131,108.83070557)
\lineto(330.06234131,109.10070557)
\lineto(330.27234131,109.37070557)
\lineto(330.39234131,109.55070557)
\curveto(330.44233058,109.61069499)(330.49233053,109.66569494)(330.54234131,109.71570557)
\curveto(330.61233041,109.76569484)(330.68733033,109.8006948)(330.76734131,109.82070557)
\curveto(330.78733023,109.83069477)(330.81233021,109.83069477)(330.84234131,109.82070557)
\curveto(330.88233014,109.82069478)(330.91233011,109.83069477)(330.93234131,109.85070557)
\curveto(331.05232997,109.85069475)(331.18732983,109.84569476)(331.33734131,109.83570557)
\curveto(331.48732953,109.83569477)(331.57732944,109.79069481)(331.60734131,109.70070557)
\curveto(331.62732939,109.67069493)(331.63232939,109.63569497)(331.62234131,109.59570557)
\curveto(331.61232941,109.55569505)(331.59732942,109.52569508)(331.57734131,109.50570557)
\curveto(331.53732948,109.42569518)(331.49732952,109.35569525)(331.45734131,109.29570557)
\curveto(331.4173296,109.23569537)(331.37232965,109.17569543)(331.32234131,109.11570557)
\lineto(330.75234131,108.33570557)
\curveto(330.57233045,108.08569652)(330.39233063,107.83069677)(330.21234131,107.57070557)
\moveto(323.35734131,103.67070557)
\curveto(323.30733771,103.69070091)(323.25733776,103.69570091)(323.20734131,103.68570557)
\curveto(323.15733786,103.67570093)(323.10733791,103.68070092)(323.05734131,103.70070557)
\curveto(322.94733807,103.72070088)(322.84233818,103.74070086)(322.74234131,103.76070557)
\curveto(322.65233837,103.79070081)(322.55733846,103.83070077)(322.45734131,103.88070557)
\curveto(322.12733889,104.02070058)(321.87233915,104.21570039)(321.69234131,104.46570557)
\curveto(321.51233951,104.72569988)(321.36733965,105.03569957)(321.25734131,105.39570557)
\curveto(321.22733979,105.47569913)(321.20733981,105.55569905)(321.19734131,105.63570557)
\curveto(321.18733983,105.72569888)(321.17233985,105.81069879)(321.15234131,105.89070557)
\curveto(321.14233988,105.94069866)(321.13733988,106.0056986)(321.13734131,106.08570557)
\curveto(321.12733989,106.11569849)(321.1223399,106.14569846)(321.12234131,106.17570557)
\curveto(321.1223399,106.21569839)(321.1173399,106.25069835)(321.10734131,106.28070557)
\lineto(321.10734131,106.43070557)
\curveto(321.09733992,106.48069812)(321.09233993,106.54069806)(321.09234131,106.61070557)
\curveto(321.09233993,106.69069791)(321.09733992,106.75569785)(321.10734131,106.80570557)
\lineto(321.10734131,106.97070557)
\curveto(321.12733989,107.02069758)(321.13233989,107.06569754)(321.12234131,107.10570557)
\curveto(321.1223399,107.15569745)(321.12733989,107.2006974)(321.13734131,107.24070557)
\curveto(321.14733987,107.28069732)(321.15233987,107.31569729)(321.15234131,107.34570557)
\curveto(321.15233987,107.38569722)(321.15733986,107.42569718)(321.16734131,107.46570557)
\curveto(321.19733982,107.57569703)(321.2173398,107.68569692)(321.22734131,107.79570557)
\curveto(321.24733977,107.91569669)(321.28233974,108.03069657)(321.33234131,108.14070557)
\curveto(321.47233955,108.48069612)(321.63233939,108.75569585)(321.81234131,108.96570557)
\curveto(322.00233902,109.18569542)(322.27233875,109.36569524)(322.62234131,109.50570557)
\curveto(322.70233832,109.53569507)(322.78733823,109.55569505)(322.87734131,109.56570557)
\curveto(322.96733805,109.58569502)(323.06233796,109.605695)(323.16234131,109.62570557)
\curveto(323.19233783,109.63569497)(323.24733777,109.63569497)(323.32734131,109.62570557)
\curveto(323.40733761,109.62569498)(323.45733756,109.63569497)(323.47734131,109.65570557)
\curveto(324.03733698,109.66569494)(324.48733653,109.55569505)(324.82734131,109.32570557)
\curveto(325.17733584,109.09569551)(325.43733558,108.79069581)(325.60734131,108.41070557)
\curveto(325.64733537,108.32069628)(325.68233534,108.22569638)(325.71234131,108.12570557)
\curveto(325.74233528,108.02569658)(325.76733525,107.92569668)(325.78734131,107.82570557)
\curveto(325.80733521,107.79569681)(325.81233521,107.76569684)(325.80234131,107.73570557)
\curveto(325.80233522,107.7056969)(325.80733521,107.67569693)(325.81734131,107.64570557)
\curveto(325.84733517,107.53569707)(325.86733515,107.41069719)(325.87734131,107.27070557)
\curveto(325.88733513,107.14069746)(325.89733512,107.0056976)(325.90734131,106.86570557)
\lineto(325.90734131,106.70070557)
\curveto(325.9173351,106.64069796)(325.9173351,106.58569802)(325.90734131,106.53570557)
\curveto(325.89733512,106.48569812)(325.89233513,106.43569817)(325.89234131,106.38570557)
\lineto(325.89234131,106.25070557)
\curveto(325.88233514,106.21069839)(325.87733514,106.17069843)(325.87734131,106.13070557)
\curveto(325.88733513,106.09069851)(325.88233514,106.04569856)(325.86234131,105.99570557)
\curveto(325.84233518,105.88569872)(325.8223352,105.78069882)(325.80234131,105.68070557)
\curveto(325.79233523,105.58069902)(325.77233525,105.48069912)(325.74234131,105.38070557)
\curveto(325.61233541,105.02069958)(325.44733557,104.7056999)(325.24734131,104.43570557)
\curveto(325.04733597,104.16570044)(324.77233625,103.96070064)(324.42234131,103.82070557)
\curveto(324.34233668,103.79070081)(324.25733676,103.76570084)(324.16734131,103.74570557)
\lineto(323.89734131,103.68570557)
\curveto(323.84733717,103.67570093)(323.80233722,103.67070093)(323.76234131,103.67070557)
\curveto(323.7223373,103.68070092)(323.68233734,103.68070092)(323.64234131,103.67070557)
\curveto(323.54233748,103.65070095)(323.44733757,103.65070095)(323.35734131,103.67070557)
\moveto(322.51734131,105.06570557)
\curveto(322.55733846,104.99569961)(322.59733842,104.93069967)(322.63734131,104.87070557)
\curveto(322.67733834,104.82069978)(322.72733829,104.77069983)(322.78734131,104.72070557)
\lineto(322.93734131,104.60070557)
\curveto(322.99733802,104.57070003)(323.06233796,104.54570006)(323.13234131,104.52570557)
\curveto(323.17233785,104.5057001)(323.20733781,104.49570011)(323.23734131,104.49570557)
\curveto(323.27733774,104.5057001)(323.3173377,104.5007001)(323.35734131,104.48070557)
\curveto(323.38733763,104.48070012)(323.42733759,104.47570013)(323.47734131,104.46570557)
\curveto(323.52733749,104.46570014)(323.56733745,104.47070013)(323.59734131,104.48070557)
\lineto(323.82234131,104.52570557)
\curveto(324.07233695,104.6057)(324.25733676,104.73069987)(324.37734131,104.90070557)
\curveto(324.45733656,105.0006996)(324.52733649,105.13069947)(324.58734131,105.29070557)
\curveto(324.66733635,105.47069913)(324.72733629,105.69569891)(324.76734131,105.96570557)
\curveto(324.80733621,106.24569836)(324.8223362,106.52569808)(324.81234131,106.80570557)
\curveto(324.80233622,107.09569751)(324.77233625,107.37069723)(324.72234131,107.63070557)
\curveto(324.67233635,107.89069671)(324.59733642,108.1006965)(324.49734131,108.26070557)
\curveto(324.37733664,108.46069614)(324.22733679,108.61069599)(324.04734131,108.71070557)
\curveto(323.96733705,108.76069584)(323.87733714,108.79069581)(323.77734131,108.80070557)
\curveto(323.67733734,108.82069578)(323.57233745,108.83069577)(323.46234131,108.83070557)
\curveto(323.44233758,108.82069578)(323.4173376,108.81569579)(323.38734131,108.81570557)
\curveto(323.36733765,108.82569578)(323.34733767,108.82569578)(323.32734131,108.81570557)
\curveto(323.27733774,108.8056958)(323.23233779,108.79569581)(323.19234131,108.78570557)
\curveto(323.15233787,108.78569582)(323.11233791,108.77569583)(323.07234131,108.75570557)
\curveto(322.89233813,108.67569593)(322.74233828,108.55569605)(322.62234131,108.39570557)
\curveto(322.51233851,108.23569637)(322.4223386,108.05569655)(322.35234131,107.85570557)
\curveto(322.29233873,107.66569694)(322.24733877,107.44069716)(322.21734131,107.18070557)
\curveto(322.19733882,106.92069768)(322.19233883,106.65569795)(322.20234131,106.38570557)
\curveto(322.21233881,106.12569848)(322.24233878,105.87569873)(322.29234131,105.63570557)
\curveto(322.35233867,105.4056992)(322.42733859,105.21569939)(322.51734131,105.06570557)
\moveto(333.31734131,102.08070557)
\curveto(333.32732769,102.03070257)(333.33232769,101.94070266)(333.33234131,101.81070557)
\curveto(333.33232769,101.68070292)(333.3223277,101.59070301)(333.30234131,101.54070557)
\curveto(333.28232774,101.49070311)(333.27732774,101.43570317)(333.28734131,101.37570557)
\curveto(333.29732772,101.32570328)(333.29732772,101.27570333)(333.28734131,101.22570557)
\curveto(333.24732777,101.08570352)(333.2173278,100.95070365)(333.19734131,100.82070557)
\curveto(333.18732783,100.69070391)(333.15732786,100.57070403)(333.10734131,100.46070557)
\curveto(332.96732805,100.11070449)(332.80232822,99.81570479)(332.61234131,99.57570557)
\curveto(332.4223286,99.34570526)(332.15232887,99.16070544)(331.80234131,99.02070557)
\curveto(331.7223293,98.99070561)(331.63732938,98.97070563)(331.54734131,98.96070557)
\curveto(331.45732956,98.94070566)(331.37232965,98.92070568)(331.29234131,98.90070557)
\curveto(331.24232978,98.89070571)(331.19232983,98.88570572)(331.14234131,98.88570557)
\curveto(331.09232993,98.88570572)(331.04232998,98.88070572)(330.99234131,98.87070557)
\curveto(330.96233006,98.86070574)(330.91233011,98.86070574)(330.84234131,98.87070557)
\curveto(330.77233025,98.87070573)(330.7223303,98.87570573)(330.69234131,98.88570557)
\curveto(330.63233039,98.9057057)(330.57233045,98.91570569)(330.51234131,98.91570557)
\curveto(330.46233056,98.9057057)(330.41233061,98.91070569)(330.36234131,98.93070557)
\curveto(330.27233075,98.95070565)(330.18233084,98.97570563)(330.09234131,99.00570557)
\curveto(330.01233101,99.02570558)(329.93233109,99.05570555)(329.85234131,99.09570557)
\curveto(329.53233149,99.23570537)(329.28233174,99.43070517)(329.10234131,99.68070557)
\curveto(328.9223321,99.94070466)(328.77233225,100.24570436)(328.65234131,100.59570557)
\curveto(328.63233239,100.67570393)(328.6173324,100.76070384)(328.60734131,100.85070557)
\curveto(328.59733242,100.94070366)(328.58233244,101.02570358)(328.56234131,101.10570557)
\curveto(328.55233247,101.13570347)(328.54733247,101.16570344)(328.54734131,101.19570557)
\lineto(328.54734131,101.30070557)
\curveto(328.52733249,101.38070322)(328.5173325,101.46070314)(328.51734131,101.54070557)
\lineto(328.51734131,101.67570557)
\curveto(328.49733252,101.77570283)(328.49733252,101.87570273)(328.51734131,101.97570557)
\lineto(328.51734131,102.15570557)
\curveto(328.52733249,102.2057024)(328.53233249,102.25070235)(328.53234131,102.29070557)
\curveto(328.53233249,102.34070226)(328.53733248,102.38570222)(328.54734131,102.42570557)
\curveto(328.55733246,102.46570214)(328.56233246,102.5007021)(328.56234131,102.53070557)
\curveto(328.56233246,102.57070203)(328.56733245,102.61070199)(328.57734131,102.65070557)
\lineto(328.63734131,102.98070557)
\curveto(328.65733236,103.1007015)(328.68733233,103.21070139)(328.72734131,103.31070557)
\curveto(328.86733215,103.64070096)(329.02733199,103.91570069)(329.20734131,104.13570557)
\curveto(329.39733162,104.36570024)(329.65733136,104.55070005)(329.98734131,104.69070557)
\curveto(330.06733095,104.73069987)(330.15233087,104.75569985)(330.24234131,104.76570557)
\lineto(330.54234131,104.82570557)
\lineto(330.67734131,104.82570557)
\curveto(330.72733029,104.83569977)(330.77733024,104.84069976)(330.82734131,104.84070557)
\curveto(331.39732962,104.86069974)(331.85732916,104.75569985)(332.20734131,104.52570557)
\curveto(332.56732845,104.3057003)(332.83232819,104.0057006)(333.00234131,103.62570557)
\curveto(333.05232797,103.52570108)(333.09232793,103.42570118)(333.12234131,103.32570557)
\curveto(333.15232787,103.22570138)(333.18232784,103.12070148)(333.21234131,103.01070557)
\curveto(333.2223278,102.97070163)(333.22732779,102.93570167)(333.22734131,102.90570557)
\curveto(333.22732779,102.88570172)(333.23232779,102.85570175)(333.24234131,102.81570557)
\curveto(333.26232776,102.74570186)(333.27232775,102.67070193)(333.27234131,102.59070557)
\curveto(333.27232775,102.51070209)(333.28232774,102.43070217)(333.30234131,102.35070557)
\curveto(333.30232772,102.3007023)(333.30232772,102.25570235)(333.30234131,102.21570557)
\curveto(333.30232772,102.17570243)(333.30732771,102.13070247)(333.31734131,102.08070557)
\moveto(332.20734131,101.64570557)
\curveto(332.2173288,101.69570291)(332.2223288,101.77070283)(332.22234131,101.87070557)
\curveto(332.23232879,101.97070263)(332.22732879,102.04570256)(332.20734131,102.09570557)
\curveto(332.18732883,102.15570245)(332.18232884,102.21070239)(332.19234131,102.26070557)
\curveto(332.21232881,102.32070228)(332.21232881,102.38070222)(332.19234131,102.44070557)
\curveto(332.18232884,102.47070213)(332.17732884,102.5057021)(332.17734131,102.54570557)
\curveto(332.17732884,102.58570202)(332.17232885,102.62570198)(332.16234131,102.66570557)
\curveto(332.14232888,102.74570186)(332.1223289,102.82070178)(332.10234131,102.89070557)
\curveto(332.09232893,102.97070163)(332.07732894,103.05070155)(332.05734131,103.13070557)
\curveto(332.02732899,103.19070141)(332.00232902,103.25070135)(331.98234131,103.31070557)
\curveto(331.96232906,103.37070123)(331.93232909,103.43070117)(331.89234131,103.49070557)
\curveto(331.79232923,103.66070094)(331.66232936,103.79570081)(331.50234131,103.89570557)
\curveto(331.4223296,103.94570066)(331.32732969,103.98070062)(331.21734131,104.00070557)
\curveto(331.10732991,104.02070058)(330.98233004,104.03070057)(330.84234131,104.03070557)
\curveto(330.8223302,104.02070058)(330.79733022,104.01570059)(330.76734131,104.01570557)
\curveto(330.73733028,104.02570058)(330.70733031,104.02570058)(330.67734131,104.01570557)
\lineto(330.52734131,103.95570557)
\curveto(330.47733054,103.94570066)(330.43233059,103.93070067)(330.39234131,103.91070557)
\curveto(330.20233082,103.8007008)(330.05733096,103.65570095)(329.95734131,103.47570557)
\curveto(329.86733115,103.29570131)(329.78733123,103.09070151)(329.71734131,102.86070557)
\curveto(329.67733134,102.73070187)(329.65733136,102.59570201)(329.65734131,102.45570557)
\curveto(329.65733136,102.32570228)(329.64733137,102.18070242)(329.62734131,102.02070557)
\curveto(329.6173314,101.97070263)(329.60733141,101.91070269)(329.59734131,101.84070557)
\curveto(329.59733142,101.77070283)(329.60733141,101.71070289)(329.62734131,101.66070557)
\lineto(329.62734131,101.49570557)
\lineto(329.62734131,101.31570557)
\curveto(329.63733138,101.26570334)(329.64733137,101.21070339)(329.65734131,101.15070557)
\curveto(329.66733135,101.1007035)(329.67233135,101.04570356)(329.67234131,100.98570557)
\curveto(329.68233134,100.92570368)(329.69733132,100.87070373)(329.71734131,100.82070557)
\curveto(329.76733125,100.63070397)(329.82733119,100.45570415)(329.89734131,100.29570557)
\curveto(329.96733105,100.13570447)(330.07233095,100.0057046)(330.21234131,99.90570557)
\curveto(330.34233068,99.8057048)(330.48233054,99.73570487)(330.63234131,99.69570557)
\curveto(330.66233036,99.68570492)(330.68733033,99.68070492)(330.70734131,99.68070557)
\curveto(330.73733028,99.69070491)(330.76733025,99.69070491)(330.79734131,99.68070557)
\curveto(330.8173302,99.68070492)(330.84733017,99.67570493)(330.88734131,99.66570557)
\curveto(330.92733009,99.66570494)(330.96233006,99.67070493)(330.99234131,99.68070557)
\curveto(331.03232999,99.69070491)(331.07232995,99.69570491)(331.11234131,99.69570557)
\curveto(331.15232987,99.69570491)(331.19232983,99.7057049)(331.23234131,99.72570557)
\curveto(331.47232955,99.8057048)(331.66732935,99.94070466)(331.81734131,100.13070557)
\curveto(331.93732908,100.31070429)(332.02732899,100.51570409)(332.08734131,100.74570557)
\curveto(332.10732891,100.81570379)(332.1223289,100.88570372)(332.13234131,100.95570557)
\curveto(332.14232888,101.03570357)(332.15732886,101.11570349)(332.17734131,101.19570557)
\curveto(332.17732884,101.25570335)(332.18232884,101.3007033)(332.19234131,101.33070557)
\curveto(332.19232883,101.35070325)(332.19232883,101.37570323)(332.19234131,101.40570557)
\curveto(332.19232883,101.44570316)(332.19732882,101.47570313)(332.20734131,101.49570557)
\lineto(332.20734131,101.64570557)
}
}
\end{pspicture}

\caption{Porcentajes de intención de los usuarios clasificados por rol}
\label{usuarios_pie_1}
\end{figure}

\subsection{Actividad}
Respecto a la actividad de los usuarios sobre el sistema, puede verse en el
cuadro \ref{usuarios_tabla_2} la escasisima actividad, participación y
popularidad en todos los roles, exceptuando el de los desarrolladores.

\begin{table}
\centering
\begin{tabular}{l|c c c c}
$Rol$ & $Actividad$ & $Participacion$ & $Sociabilidad$ & $Popularidad$ \\
\hline
$Invitado     $ & $ 2$ & $ 0$ & $25$ & $ 1$ \\
$Estudiante   $ & $ 3$ & $10$ & $58$ & $ 0$ \\
$Auxiliar     $ & $ 7$ & $ 0$ & $12$ & $ 0$ \\
$Docente      $ & $ 4$ & $ 0$ & $ 6$ & $ 0$ \\
$Moderador    $ & $27$ & $ 0$ & $ 5$ & $ 3$ \\
$Desarrollador$ & $61$ & $54$ & $96$ & $19$ \\
$Administrador$ & $ 0$ & $ 2$ & $10$ & $ 0$ \\
\end{tabular}
\caption{Actividad de los usuarios clasificados por rol}
\label{usuarios_tabla_2}
\end{table}

Puede verse también en la figura \ref{usuarios_bars_2} el prometedor indicador
de sociabilidad, que como puede verse en la figura \ref{usuarios_pie_2} es el
mas homogéneo, lo que augura una conectividad mas que deseable para los
usuarios.

\begin{figure}
\centering
%LaTeX with PSTricks extensions
%%Creator: inkscape 0.48.5
%%Please note this file requires PSTricks extensions
\psset{xunit=.5pt,yunit=.5pt,runit=.5pt}
\begin{pspicture}(960,480)
{
\newrgbcolor{curcolor}{0 0 0}
\pscustom[linestyle=none,fillstyle=solid,fillcolor=curcolor]
{
\newpath
\moveto(438.45475342,18.82530273)
\curveto(438.48474374,18.69530247)(438.44474378,18.59530257)(438.33475342,18.52530273)
\curveto(438.28474394,18.49530267)(438.21974401,18.47530269)(438.13975342,18.46530273)
\lineto(437.89975342,18.46530273)
\lineto(437.41975342,18.46530273)
\curveto(437.25974497,18.4653027)(437.14474508,18.50030267)(437.07475342,18.57030273)
\curveto(437.00474522,18.62030255)(436.96474526,18.69530247)(436.95475342,18.79530273)
\lineto(436.95475342,19.12530273)
\lineto(436.95475342,19.23030273)
\curveto(436.96474526,19.2703019)(436.97474525,19.30530186)(436.98475342,19.33530273)
\curveto(436.97474525,19.38530178)(436.97974525,19.43030174)(436.99975342,19.47030273)
\curveto(437.01974521,19.51030166)(437.0247452,19.55030162)(437.01475342,19.59030273)
\lineto(437.04475342,19.77030273)
\lineto(437.07475342,19.95030273)
\lineto(437.16475342,20.62530273)
\curveto(437.16474506,20.69530047)(437.16974506,20.7653004)(437.17975342,20.83530273)
\curveto(437.18974504,20.90530026)(437.19474503,20.98030019)(437.19475342,21.06030273)
\curveto(437.18474504,21.24029993)(437.18474504,21.42029975)(437.19475342,21.60030273)
\curveto(437.20474502,21.78029939)(437.18974504,21.95029922)(437.14975342,22.11030273)
\curveto(437.03974519,22.53029864)(436.77974545,22.81029836)(436.36975342,22.95030273)
\curveto(436.24974598,23.00029817)(436.10974612,23.02529814)(435.94975342,23.02530273)
\curveto(435.79974643,23.03529813)(435.63974659,23.04029813)(435.46975342,23.04030273)
\lineto(432.70975342,23.04030273)
\curveto(432.63974959,23.02029815)(432.57474965,23.00029817)(432.51475342,22.98030273)
\curveto(432.45474977,22.9702982)(432.39974983,22.94029823)(432.34975342,22.89030273)
\curveto(432.25974997,22.79029838)(432.19475003,22.62529854)(432.15475342,22.39530273)
\curveto(432.11475011,22.17529899)(432.07975015,21.98029919)(432.04975342,21.81030273)
\lineto(431.61475342,19.63530273)
\curveto(431.58475064,19.49530167)(431.55475067,19.32030185)(431.52475342,19.11030273)
\curveto(431.49475073,18.91030226)(431.44475078,18.76030241)(431.37475342,18.66030273)
\curveto(431.34475088,18.59030258)(431.29475093,18.54530262)(431.22475342,18.52530273)
\curveto(431.18475104,18.50530266)(431.14475108,18.49530267)(431.10475342,18.49530273)
\curveto(431.07475115,18.49530267)(431.0297512,18.48530268)(430.96975342,18.46530273)
\curveto(430.9297513,18.45530271)(430.88475134,18.45030272)(430.83475342,18.45030273)
\curveto(430.78475144,18.46030271)(430.73475149,18.4653027)(430.68475342,18.46530273)
\lineto(430.36975342,18.46530273)
\curveto(430.26975196,18.47530269)(430.18975204,18.50530266)(430.12975342,18.55530273)
\curveto(430.05975217,18.60530256)(430.03475219,18.69530247)(430.05475342,18.82530273)
\curveto(430.08475214,18.9653022)(430.11475211,19.10030207)(430.14475342,19.23030273)
\lineto(431.95975342,28.35030273)
\curveto(431.97975025,28.46029271)(431.99975023,28.57529259)(432.01975342,28.69530273)
\curveto(432.03975019,28.81529235)(432.08475014,28.91029226)(432.15475342,28.98030273)
\curveto(432.20475002,29.04029213)(432.28974994,29.09029208)(432.40975342,29.13030273)
\curveto(432.4297498,29.14029203)(432.44974978,29.14029203)(432.46975342,29.13030273)
\curveto(432.48974974,29.13029204)(432.50974972,29.13529203)(432.52975342,29.14530273)
\lineto(436.87975342,29.14530273)
\curveto(436.94974528,29.14529202)(437.0247452,29.14529202)(437.10475342,29.14530273)
\curveto(437.18474504,29.15529201)(437.25474497,29.15529201)(437.31475342,29.14530273)
\lineto(437.47975342,29.14530273)
\curveto(437.53974469,29.13529203)(437.59474463,29.12529204)(437.64475342,29.11530273)
\curveto(437.70474452,29.11529205)(437.76974446,29.11029206)(437.83975342,29.10030273)
\curveto(437.91974431,29.08029209)(437.99974423,29.0652921)(438.07975342,29.05530273)
\curveto(438.15974407,29.04529212)(438.23974399,29.03029214)(438.31975342,29.01030273)
\curveto(438.49974373,28.95029222)(438.65974357,28.88529228)(438.79975342,28.81530273)
\curveto(438.94974328,28.74529242)(439.08474314,28.66029251)(439.20475342,28.56030273)
\curveto(439.4247428,28.39029278)(439.58474264,28.18029299)(439.68475342,27.93030273)
\curveto(439.79474243,27.69029348)(439.86474236,27.40529376)(439.89475342,27.07530273)
\curveto(439.90474232,26.99529417)(439.89974233,26.91029426)(439.87975342,26.82030273)
\curveto(439.86974236,26.74029443)(439.86974236,26.66029451)(439.87975342,26.58030273)
\lineto(439.84975342,26.43030273)
\curveto(439.84974238,26.38029479)(439.83974239,26.32029485)(439.81975342,26.25030273)
\curveto(439.79974243,26.19029498)(439.77974245,26.13529503)(439.75975342,26.08530273)
\lineto(439.72975342,25.92030273)
\curveto(439.68974254,25.84029533)(439.65974257,25.7652954)(439.63975342,25.69530273)
\curveto(439.61974261,25.62529554)(439.59474263,25.55529561)(439.56475342,25.48530273)
\curveto(439.48474274,25.33529583)(439.40974282,25.19029598)(439.33975342,25.05030273)
\curveto(439.26974296,24.92029625)(439.17974305,24.79529637)(439.06975342,24.67530273)
\curveto(439.0297432,24.62529654)(438.98974324,24.58029659)(438.94975342,24.54030273)
\curveto(438.90974332,24.50029667)(438.86974336,24.45529671)(438.82975342,24.40530273)
\curveto(438.80974342,24.39529677)(438.79474343,24.38529678)(438.78475342,24.37530273)
\curveto(438.77474345,24.37529679)(438.76474346,24.3702968)(438.75475342,24.36030273)
\curveto(438.73474349,24.34029683)(438.70474352,24.31529685)(438.66475342,24.28530273)
\lineto(438.58975342,24.21030273)
\curveto(438.49974373,24.15029702)(438.41474381,24.09029708)(438.33475342,24.03030273)
\curveto(438.25474397,23.98029719)(438.16974406,23.93029724)(438.07975342,23.88030273)
\curveto(438.01974421,23.85029732)(437.96474426,23.81529735)(437.91475342,23.77530273)
\curveto(437.87474435,23.74529742)(437.83974439,23.70029747)(437.80975342,23.64030273)
\curveto(437.78974444,23.58029759)(437.80474442,23.53029764)(437.85475342,23.49030273)
\curveto(437.90474432,23.45029772)(437.94474428,23.42029775)(437.97475342,23.40030273)
\curveto(438.07474415,23.33029784)(438.16474406,23.25529791)(438.24475342,23.17530273)
\curveto(438.3247439,23.09529807)(438.38474384,23.00029817)(438.42475342,22.89030273)
\curveto(438.51474371,22.75029842)(438.56474366,22.59029858)(438.57475342,22.41030273)
\curveto(438.59474363,22.24029893)(438.60974362,22.05529911)(438.61975342,21.85530273)
\lineto(438.58975342,21.61530273)
\curveto(438.58974364,21.54529962)(438.58474364,21.4702997)(438.57475342,21.39030273)
\curveto(438.58474364,21.32029985)(438.57474365,21.25029992)(438.54475342,21.18030273)
\curveto(438.5247437,21.11030006)(438.51974371,21.04030013)(438.52975342,20.97030273)
\lineto(438.49975342,20.83530273)
\curveto(438.50974372,20.7653004)(438.49974373,20.69030048)(438.46975342,20.61030273)
\curveto(438.43974379,20.53030064)(438.4297438,20.45030072)(438.43975342,20.37030273)
\curveto(438.43974379,20.33030084)(438.43474379,20.29030088)(438.42475342,20.25030273)
\curveto(438.41474381,20.22030095)(438.40974382,20.18030099)(438.40975342,20.13030273)
\curveto(438.40974382,20.03030114)(438.39974383,19.92530124)(438.37975342,19.81530273)
\curveto(438.36974386,19.71530145)(438.37474385,19.62030155)(438.39475342,19.53030273)
\curveto(438.39474383,19.4703017)(438.38974384,19.41030176)(438.37975342,19.35030273)
\curveto(438.37974385,19.30030187)(438.38474384,19.24530192)(438.39475342,19.18530273)
\lineto(438.45475342,18.82530273)
\moveto(437.88475342,25.05030273)
\curveto(437.97474425,25.16029601)(438.04974418,25.27529589)(438.10975342,25.39530273)
\curveto(438.16974406,25.51529565)(438.229744,25.64529552)(438.28975342,25.78530273)
\lineto(438.31975342,25.92030273)
\curveto(438.37974385,26.06029511)(438.41474381,26.21029496)(438.42475342,26.37030273)
\curveto(438.43474379,26.54029463)(438.4297438,26.68029449)(438.40975342,26.79030273)
\curveto(438.34974388,27.29029388)(438.10474412,27.63529353)(437.67475342,27.82530273)
\curveto(437.49474473,27.90529326)(437.26974496,27.95029322)(436.99975342,27.96030273)
\curveto(436.73974549,27.9702932)(436.46974576,27.97529319)(436.18975342,27.97530273)
\lineto(433.71475342,27.97530273)
\curveto(433.69474853,27.9652932)(433.66974856,27.96029321)(433.63975342,27.96030273)
\curveto(433.61974861,27.96029321)(433.59474863,27.95529321)(433.56475342,27.94530273)
\curveto(433.43474879,27.91529325)(433.33974889,27.85029332)(433.27975342,27.75030273)
\curveto(433.21974901,27.66029351)(433.17474905,27.53529363)(433.14475342,27.37530273)
\curveto(433.1247491,27.21529395)(433.09974913,27.0702941)(433.06975342,26.94030273)
\lineto(432.72475342,25.21530273)
\curveto(432.69474953,25.0652961)(432.65974957,24.90529626)(432.61975342,24.73530273)
\curveto(432.58974964,24.57529659)(432.59974963,24.45029672)(432.64975342,24.36030273)
\curveto(432.68974954,24.29029688)(432.75474947,24.24529692)(432.84475342,24.22530273)
\curveto(432.94474928,24.21529695)(433.05474917,24.21029696)(433.17475342,24.21030273)
\lineto(434.10475342,24.21030273)
\curveto(434.49474773,24.21029696)(434.87474735,24.20529696)(435.24475342,24.19530273)
\curveto(435.61474661,24.19529697)(435.95974627,24.21529695)(436.27975342,24.25530273)
\curveto(436.60974562,24.30529686)(436.90974532,24.39029678)(437.17975342,24.51030273)
\curveto(437.44974478,24.63029654)(437.68474454,24.81029636)(437.88475342,25.05030273)
}
}
{
\newrgbcolor{curcolor}{0 0 0}
\pscustom[linestyle=none,fillstyle=solid,fillcolor=curcolor]
{
\newpath
\moveto(447.97295654,22.65030273)
\curveto(447.98294765,22.59029858)(447.97294766,22.49529867)(447.94295654,22.36530273)
\curveto(447.92294771,22.24529892)(447.90294773,22.16029901)(447.88295654,22.11030273)
\lineto(447.85295654,21.96030273)
\curveto(447.82294781,21.88029929)(447.79794784,21.80529936)(447.77795654,21.73530273)
\curveto(447.76794787,21.67529949)(447.74794789,21.60529956)(447.71795654,21.52530273)
\curveto(447.68794795,21.4652997)(447.66294797,21.40529976)(447.64295654,21.34530273)
\curveto(447.632948,21.28529988)(447.60794803,21.22529994)(447.56795654,21.16530273)
\lineto(447.38795654,20.77530273)
\curveto(447.3379483,20.64530052)(447.27294836,20.52530064)(447.19295654,20.41530273)
\curveto(446.89294874,19.93530123)(446.5329491,19.53030164)(446.11295654,19.20030273)
\curveto(445.70294993,18.88030229)(445.22295041,18.63530253)(444.67295654,18.46530273)
\curveto(444.56295107,18.42530274)(444.44295119,18.39530277)(444.31295654,18.37530273)
\curveto(444.18295145,18.35530281)(444.04795159,18.33530283)(443.90795654,18.31530273)
\curveto(443.84795179,18.30530286)(443.78295185,18.30030287)(443.71295654,18.30030273)
\curveto(443.65295198,18.29030288)(443.59295204,18.28530288)(443.53295654,18.28530273)
\curveto(443.49295214,18.27530289)(443.4329522,18.2703029)(443.35295654,18.27030273)
\curveto(443.28295235,18.2703029)(443.2329524,18.27530289)(443.20295654,18.28530273)
\curveto(443.16295247,18.29530287)(443.12295251,18.30030287)(443.08295654,18.30030273)
\curveto(443.04295259,18.29030288)(443.00795263,18.29030288)(442.97795654,18.30030273)
\lineto(442.88795654,18.30030273)
\lineto(442.54295654,18.34530273)
\lineto(442.15295654,18.46530273)
\curveto(442.0329536,18.50530266)(441.91795372,18.55030262)(441.80795654,18.60030273)
\curveto(441.39795424,18.80030237)(441.07795456,19.06030211)(440.84795654,19.38030273)
\curveto(440.62795501,19.70030147)(440.46795517,20.09030108)(440.36795654,20.55030273)
\curveto(440.3379553,20.65030052)(440.31795532,20.75030042)(440.30795654,20.85030273)
\lineto(440.30795654,21.16530273)
\curveto(440.29795534,21.20529996)(440.29795534,21.23529993)(440.30795654,21.25530273)
\curveto(440.31795532,21.28529988)(440.32295531,21.32029985)(440.32295654,21.36030273)
\curveto(440.32295531,21.44029973)(440.32795531,21.52029965)(440.33795654,21.60030273)
\curveto(440.34795529,21.69029948)(440.35295528,21.77529939)(440.35295654,21.85530273)
\curveto(440.36295527,21.90529926)(440.36795527,21.94529922)(440.36795654,21.97530273)
\curveto(440.37795526,22.01529915)(440.38295525,22.06029911)(440.38295654,22.11030273)
\curveto(440.38295525,22.16029901)(440.39295524,22.24529892)(440.41295654,22.36530273)
\curveto(440.44295519,22.49529867)(440.47295516,22.59029858)(440.50295654,22.65030273)
\curveto(440.54295509,22.72029845)(440.56295507,22.79029838)(440.56295654,22.86030273)
\curveto(440.56295507,22.93029824)(440.58295505,23.00029817)(440.62295654,23.07030273)
\curveto(440.64295499,23.12029805)(440.65795498,23.16029801)(440.66795654,23.19030273)
\curveto(440.67795496,23.23029794)(440.69295494,23.27529789)(440.71295654,23.32530273)
\curveto(440.77295486,23.44529772)(440.82295481,23.5652976)(440.86295654,23.68530273)
\curveto(440.91295472,23.80529736)(440.97795466,23.92029725)(441.05795654,24.03030273)
\curveto(441.27795436,24.40029677)(441.52295411,24.73029644)(441.79295654,25.02030273)
\curveto(442.07295356,25.32029585)(442.38795325,25.5702956)(442.73795654,25.77030273)
\curveto(442.86795277,25.85029532)(443.00295263,25.91529525)(443.14295654,25.96530273)
\lineto(443.59295654,26.14530273)
\curveto(443.72295191,26.19529497)(443.85795178,26.22529494)(443.99795654,26.23530273)
\curveto(444.1379515,26.25529491)(444.28295135,26.28529488)(444.43295654,26.32530273)
\lineto(444.62795654,26.32530273)
\lineto(444.83795654,26.35530273)
\curveto(445.72794991,26.3652948)(446.42794921,26.18029499)(446.93795654,25.80030273)
\curveto(447.45794818,25.42029575)(447.78294785,24.92529624)(447.91295654,24.31530273)
\curveto(447.94294769,24.21529695)(447.96294767,24.11529705)(447.97295654,24.01530273)
\curveto(447.98294765,23.91529725)(447.99794764,23.81029736)(448.01795654,23.70030273)
\curveto(448.02794761,23.59029758)(448.02794761,23.4702977)(448.01795654,23.34030273)
\lineto(448.01795654,22.96530273)
\curveto(448.01794762,22.91529825)(448.00794763,22.86029831)(447.98795654,22.80030273)
\curveto(447.97794766,22.75029842)(447.97294766,22.70029847)(447.97295654,22.65030273)
\moveto(446.47295654,21.79530273)
\curveto(446.50294913,21.8652993)(446.52294911,21.94529922)(446.53295654,22.03530273)
\curveto(446.55294908,22.12529904)(446.56794907,22.21029896)(446.57795654,22.29030273)
\curveto(446.65794898,22.68029849)(446.69294894,23.01029816)(446.68295654,23.28030273)
\curveto(446.66294897,23.36029781)(446.64794899,23.44029773)(446.63795654,23.52030273)
\curveto(446.637949,23.60029757)(446.632949,23.67529749)(446.62295654,23.74530273)
\curveto(446.47294916,24.39529677)(446.11794952,24.84529632)(445.55795654,25.09530273)
\curveto(445.48795015,25.12529604)(445.41295022,25.14529602)(445.33295654,25.15530273)
\curveto(445.26295037,25.17529599)(445.18795045,25.19529597)(445.10795654,25.21530273)
\curveto(445.0379506,25.23529593)(444.95795068,25.24529592)(444.86795654,25.24530273)
\lineto(444.59795654,25.24530273)
\lineto(444.31295654,25.20030273)
\curveto(444.21295142,25.18029599)(444.11795152,25.15529601)(444.02795654,25.12530273)
\curveto(443.9379517,25.10529606)(443.84795179,25.07529609)(443.75795654,25.03530273)
\curveto(443.68795195,25.01529615)(443.61795202,24.98529618)(443.54795654,24.94530273)
\curveto(443.47795216,24.90529626)(443.41295222,24.8652963)(443.35295654,24.82530273)
\curveto(443.08295255,24.65529651)(442.84795279,24.45029672)(442.64795654,24.21030273)
\curveto(442.44795319,23.9702972)(442.26295337,23.69029748)(442.09295654,23.37030273)
\curveto(442.04295359,23.2702979)(442.00295363,23.165298)(441.97295654,23.05530273)
\curveto(441.94295369,22.95529821)(441.90295373,22.85029832)(441.85295654,22.74030273)
\curveto(441.84295379,22.70029847)(441.82795381,22.63529853)(441.80795654,22.54530273)
\curveto(441.78795385,22.51529865)(441.77795386,22.48029869)(441.77795654,22.44030273)
\curveto(441.77795386,22.40029877)(441.77295386,22.35529881)(441.76295654,22.30530273)
\lineto(441.70295654,22.00530273)
\curveto(441.68295395,21.90529926)(441.67295396,21.81529935)(441.67295654,21.73530273)
\lineto(441.67295654,21.55530273)
\curveto(441.67295396,21.45529971)(441.66795397,21.35529981)(441.65795654,21.25530273)
\curveto(441.65795398,21.1653)(441.66795397,21.08030009)(441.68795654,21.00030273)
\curveto(441.7379539,20.76030041)(441.80795383,20.53530063)(441.89795654,20.32530273)
\curveto(441.99795364,20.11530105)(442.1329535,19.94030123)(442.30295654,19.80030273)
\curveto(442.35295328,19.7703014)(442.39295324,19.74530142)(442.42295654,19.72530273)
\curveto(442.46295317,19.70530146)(442.50295313,19.68030149)(442.54295654,19.65030273)
\curveto(442.61295302,19.60030157)(442.69295294,19.55530161)(442.78295654,19.51530273)
\curveto(442.87295276,19.48530168)(442.96795267,19.45530171)(443.06795654,19.42530273)
\curveto(443.11795252,19.40530176)(443.16295247,19.39530177)(443.20295654,19.39530273)
\curveto(443.25295238,19.40530176)(443.30295233,19.40530176)(443.35295654,19.39530273)
\curveto(443.38295225,19.38530178)(443.44295219,19.37530179)(443.53295654,19.36530273)
\curveto(443.62295201,19.35530181)(443.69795194,19.36030181)(443.75795654,19.38030273)
\curveto(443.79795184,19.39030178)(443.8379518,19.39030178)(443.87795654,19.38030273)
\curveto(443.91795172,19.38030179)(443.95795168,19.39030178)(443.99795654,19.41030273)
\curveto(444.07795156,19.43030174)(444.15795148,19.44530172)(444.23795654,19.45530273)
\curveto(444.32795131,19.47530169)(444.41295122,19.50030167)(444.49295654,19.53030273)
\curveto(444.85295078,19.6703015)(445.16295047,19.8653013)(445.42295654,20.11530273)
\curveto(445.68294995,20.3653008)(445.91794972,20.66030051)(446.12795654,21.00030273)
\curveto(446.20794943,21.12030005)(446.26794937,21.24529992)(446.30795654,21.37530273)
\curveto(446.34794929,21.51529965)(446.40294923,21.65529951)(446.47295654,21.79530273)
}
}
{
\newrgbcolor{curcolor}{0 0 0}
\pscustom[linestyle=none,fillstyle=solid,fillcolor=curcolor]
{
\newpath
\moveto(451.33623779,29.13030273)
\curveto(451.46623403,29.13029204)(451.6012339,29.13029204)(451.74123779,29.13030273)
\curveto(451.89123361,29.13029204)(451.99123351,29.09529207)(452.04123779,29.02530273)
\curveto(452.08123342,28.95529221)(452.09123341,28.86029231)(452.07123779,28.74030273)
\curveto(452.05123345,28.63029254)(452.03123347,28.51529265)(452.01123779,28.39530273)
\lineto(451.74123779,27.06030273)
\lineto(450.52623779,20.98530273)
\lineto(450.19623779,19.30530273)
\curveto(450.16623533,19.18530198)(450.13623536,19.05530211)(450.10623779,18.91530273)
\curveto(450.08623541,18.77530239)(450.04123546,18.6653025)(449.97123779,18.58530273)
\curveto(449.93123557,18.53530263)(449.88123562,18.50530266)(449.82123779,18.49530273)
\curveto(449.77123573,18.48530268)(449.7012358,18.4703027)(449.61123779,18.45030273)
\lineto(449.40123779,18.45030273)
\lineto(449.08623779,18.45030273)
\curveto(448.98623651,18.46030271)(448.92123658,18.49530267)(448.89123779,18.55530273)
\curveto(448.85123665,18.63530253)(448.84123666,18.73530243)(448.86123779,18.85530273)
\curveto(448.88123662,18.97530219)(448.90623659,19.10030207)(448.93623779,19.23030273)
\lineto(449.20623779,20.61030273)
\lineto(450.45123779,26.85030273)
\lineto(450.75123779,28.32030273)
\curveto(450.77123473,28.43029274)(450.79123471,28.54529262)(450.81123779,28.66530273)
\curveto(450.83123467,28.79529237)(450.87123463,28.89529227)(450.93123779,28.96530273)
\curveto(450.99123451,29.02529214)(451.07623442,29.07529209)(451.18623779,29.11530273)
\curveto(451.21623428,29.12529204)(451.24123426,29.12529204)(451.26123779,29.11530273)
\curveto(451.28123422,29.11529205)(451.30623419,29.12029205)(451.33623779,29.13030273)
}
}
{
\newrgbcolor{curcolor}{0 0 0}
\pscustom[linestyle=none,fillstyle=solid,fillcolor=curcolor]
{
\newpath
\moveto(459.55108154,22.62030273)
\curveto(459.55107304,22.52029865)(459.53107306,22.40529876)(459.49108154,22.27530273)
\curveto(459.45107314,22.15529901)(459.40107319,22.0702991)(459.34108154,22.02030273)
\curveto(459.28107331,21.98029919)(459.20107339,21.95029922)(459.10108154,21.93030273)
\curveto(459.00107359,21.92029925)(458.8910737,21.91529925)(458.77108154,21.91530273)
\lineto(458.41108154,21.91530273)
\curveto(458.30107429,21.92529924)(458.20107439,21.93029924)(458.11108154,21.93030273)
\lineto(454.27108154,21.93030273)
\curveto(454.1910784,21.93029924)(454.10607848,21.92529924)(454.01608154,21.91530273)
\curveto(453.93607865,21.91529925)(453.87107872,21.90029927)(453.82108154,21.87030273)
\curveto(453.77107882,21.85029932)(453.72107887,21.81029936)(453.67108154,21.75030273)
\lineto(453.58108154,21.61530273)
\curveto(453.55107904,21.5652996)(453.54107905,21.51529965)(453.55108154,21.46530273)
\curveto(453.55107904,21.41529975)(453.54607904,21.3702998)(453.53608154,21.33030273)
\lineto(453.53608154,21.21030273)
\lineto(453.53608154,20.95530273)
\curveto(453.54607904,20.87530029)(453.56107903,20.79530037)(453.58108154,20.71530273)
\curveto(453.71107888,20.17530099)(454.01607857,19.79030138)(454.49608154,19.56030273)
\curveto(454.54607804,19.53030164)(454.60607798,19.50530166)(454.67608154,19.48530273)
\curveto(454.74607784,19.4653017)(454.81107778,19.44530172)(454.87108154,19.42530273)
\curveto(454.90107769,19.41530175)(454.95107764,19.41030176)(455.02108154,19.41030273)
\curveto(455.15107744,19.3703018)(455.33107726,19.35030182)(455.56108154,19.35030273)
\curveto(455.7910768,19.35030182)(455.98107661,19.3703018)(456.13108154,19.41030273)
\curveto(456.28107631,19.45030172)(456.41607617,19.49030168)(456.53608154,19.53030273)
\curveto(456.66607592,19.58030159)(456.7860758,19.64030153)(456.89608154,19.71030273)
\curveto(457.01607557,19.78030139)(457.12607546,19.86030131)(457.22608154,19.95030273)
\curveto(457.32607526,20.05030112)(457.41607517,20.15530101)(457.49608154,20.26530273)
\curveto(457.57607501,20.3653008)(457.65107494,20.4703007)(457.72108154,20.58030273)
\curveto(457.7910748,20.69030048)(457.8860747,20.7703004)(458.00608154,20.82030273)
\curveto(458.04607454,20.84030033)(458.11107448,20.85530031)(458.20108154,20.86530273)
\curveto(458.30107429,20.87530029)(458.3910742,20.87530029)(458.47108154,20.86530273)
\curveto(458.56107403,20.8653003)(458.64607394,20.86030031)(458.72608154,20.85030273)
\curveto(458.80607378,20.84030033)(458.85607373,20.82030035)(458.87608154,20.79030273)
\curveto(458.96607362,20.72030045)(458.97107362,20.60530056)(458.89108154,20.44530273)
\curveto(458.75107384,20.17530099)(458.59607399,19.93530123)(458.42608154,19.72530273)
\curveto(458.16607442,19.40530176)(457.8860747,19.14030203)(457.58608154,18.93030273)
\curveto(457.29607529,18.73030244)(456.94107565,18.5653026)(456.52108154,18.43530273)
\curveto(456.41107618,18.39530277)(456.30607628,18.3703028)(456.20608154,18.36030273)
\curveto(456.10607648,18.34030283)(455.99607659,18.32030285)(455.87608154,18.30030273)
\curveto(455.82607676,18.29030288)(455.77607681,18.28530288)(455.72608154,18.28530273)
\curveto(455.6860769,18.28530288)(455.64107695,18.28030289)(455.59108154,18.27030273)
\lineto(455.44108154,18.27030273)
\curveto(455.3910772,18.26030291)(455.33107726,18.25530291)(455.26108154,18.25530273)
\curveto(455.20107739,18.25530291)(455.15107744,18.26030291)(455.11108154,18.27030273)
\lineto(454.97608154,18.27030273)
\curveto(454.92607766,18.28030289)(454.88107771,18.28530288)(454.84108154,18.28530273)
\curveto(454.80107779,18.28530288)(454.76107783,18.29030288)(454.72108154,18.30030273)
\curveto(454.67107792,18.31030286)(454.61607797,18.32030285)(454.55608154,18.33030273)
\curveto(454.50607808,18.33030284)(454.45607813,18.33530283)(454.40608154,18.34530273)
\curveto(454.31607827,18.3653028)(454.22607836,18.39030278)(454.13608154,18.42030273)
\curveto(454.05607853,18.44030273)(453.98107861,18.4653027)(453.91108154,18.49530273)
\curveto(453.87107872,18.51530265)(453.83607875,18.52530264)(453.80608154,18.52530273)
\curveto(453.77607881,18.53530263)(453.74607884,18.55030262)(453.71608154,18.57030273)
\curveto(453.57607901,18.64030253)(453.43107916,18.72530244)(453.28108154,18.82530273)
\curveto(453.03107956,19.01530215)(452.83107976,19.24530192)(452.68108154,19.51530273)
\curveto(452.53108006,19.79530137)(452.42108017,20.10530106)(452.35108154,20.44530273)
\curveto(452.32108027,20.55530061)(452.30608028,20.6703005)(452.30608154,20.79030273)
\curveto(452.30608028,20.91030026)(452.29608029,21.03030014)(452.27608154,21.15030273)
\lineto(452.27608154,21.25530273)
\curveto(452.2860803,21.28529988)(452.2910803,21.32529984)(452.29108154,21.37530273)
\lineto(452.29108154,21.63030273)
\curveto(452.30108029,21.72029945)(452.30608028,21.81029936)(452.30608154,21.90030273)
\lineto(452.35108154,22.11030273)
\curveto(452.35108024,22.15029902)(452.35608023,22.20529896)(452.36608154,22.27530273)
\curveto(452.37608021,22.35529881)(452.3910802,22.42029875)(452.41108154,22.47030273)
\lineto(452.44108154,22.63530273)
\curveto(452.47108012,22.68529848)(452.4860801,22.73529843)(452.48608154,22.78530273)
\curveto(452.49608009,22.84529832)(452.51108008,22.90029827)(452.53108154,22.95030273)
\curveto(452.60107999,23.11029806)(452.66607992,23.2702979)(452.72608154,23.43030273)
\curveto(452.7860798,23.59029758)(452.86107973,23.74029743)(452.95108154,23.88030273)
\curveto(453.02107957,23.99029718)(453.0860795,24.10029707)(453.14608154,24.21030273)
\curveto(453.21607937,24.33029684)(453.29607929,24.44529672)(453.38608154,24.55530273)
\curveto(453.67607891,24.90529626)(453.9860786,25.20529596)(454.31608154,25.45530273)
\curveto(454.64607794,25.71529545)(455.03107756,25.93029524)(455.47108154,26.10030273)
\curveto(455.60107699,26.15029502)(455.73107686,26.18529498)(455.86108154,26.20530273)
\curveto(455.9910766,26.23529493)(456.13107646,26.2652949)(456.28108154,26.29530273)
\curveto(456.33107626,26.30529486)(456.37607621,26.31029486)(456.41608154,26.31030273)
\curveto(456.45607613,26.32029485)(456.50107609,26.32529484)(456.55108154,26.32530273)
\curveto(456.57107602,26.33529483)(456.59607599,26.33529483)(456.62608154,26.32530273)
\curveto(456.65607593,26.31529485)(456.68107591,26.32029485)(456.70108154,26.34030273)
\curveto(457.13107546,26.35029482)(457.4910751,26.30529486)(457.78108154,26.20530273)
\curveto(458.07107452,26.11529505)(458.32607426,25.99029518)(458.54608154,25.83030273)
\curveto(458.586074,25.81029536)(458.61607397,25.78029539)(458.63608154,25.74030273)
\curveto(458.66607392,25.71029546)(458.69607389,25.68529548)(458.72608154,25.66530273)
\curveto(458.79607379,25.60529556)(458.86607372,25.53529563)(458.93608154,25.45530273)
\curveto(459.00607358,25.37529579)(459.06107353,25.29529587)(459.10108154,25.21530273)
\curveto(459.22107337,25.00529616)(459.31607327,24.80529636)(459.38608154,24.61530273)
\curveto(459.43607315,24.50529666)(459.46607312,24.38529678)(459.47608154,24.25530273)
\lineto(459.53608154,23.86530273)
\curveto(459.56607302,23.73529743)(459.57607301,23.60029757)(459.56608154,23.46030273)
\curveto(459.56607302,23.32029785)(459.57107302,23.18029799)(459.58108154,23.04030273)
\curveto(459.58107301,22.9702982)(459.57607301,22.90029827)(459.56608154,22.83030273)
\curveto(459.55607303,22.76029841)(459.55107304,22.69029848)(459.55108154,22.62030273)
\moveto(458.20108154,23.13030273)
\curveto(458.23107436,23.170298)(458.26107433,23.22029795)(458.29108154,23.28030273)
\curveto(458.33107426,23.35029782)(458.34607424,23.42029775)(458.33608154,23.49030273)
\curveto(458.32607426,23.71029746)(458.2860743,23.91529725)(458.21608154,24.10530273)
\curveto(458.11607447,24.33529683)(457.99607459,24.53029664)(457.85608154,24.69030273)
\curveto(457.72607486,24.85029632)(457.53607505,24.98529618)(457.28608154,25.09530273)
\curveto(457.21607537,25.11529605)(457.14607544,25.13029604)(457.07608154,25.14030273)
\curveto(457.01607557,25.16029601)(456.94607564,25.18029599)(456.86608154,25.20030273)
\curveto(456.79607579,25.22029595)(456.71607587,25.23029594)(456.62608154,25.23030273)
\lineto(456.37108154,25.23030273)
\curveto(456.33107626,25.21029596)(456.2910763,25.20029597)(456.25108154,25.20030273)
\curveto(456.21107638,25.21029596)(456.17607641,25.21029596)(456.14608154,25.20030273)
\lineto(455.90608154,25.14030273)
\curveto(455.82607676,25.13029604)(455.75107684,25.11529605)(455.68108154,25.09530273)
\curveto(455.36107723,24.97529619)(455.09607749,24.82529634)(454.88608154,24.64530273)
\curveto(454.67607791,24.4652967)(454.47607811,24.24029693)(454.28608154,23.97030273)
\curveto(454.24607834,23.92029725)(454.20107839,23.85529731)(454.15108154,23.77530273)
\curveto(454.11107848,23.70529746)(454.07107852,23.62529754)(454.03108154,23.53530273)
\curveto(453.9910786,23.44529772)(453.96607862,23.36029781)(453.95608154,23.28030273)
\curveto(453.95607863,23.20029797)(453.98107861,23.14029803)(454.03108154,23.10030273)
\curveto(454.10107849,23.04029813)(454.23107836,23.01029816)(454.42108154,23.01030273)
\curveto(454.62107797,23.02029815)(454.7910778,23.02529814)(454.93108154,23.02530273)
\lineto(457.21108154,23.02530273)
\curveto(457.36107523,23.02529814)(457.54107505,23.02029815)(457.75108154,23.01030273)
\curveto(457.96107463,23.01029816)(458.11107448,23.05029812)(458.20108154,23.13030273)
}
}
{
\newrgbcolor{curcolor}{0 0 0}
\pscustom[linestyle=none,fillstyle=solid,fillcolor=curcolor]
{
\newpath
\moveto(464.03772217,26.35530273)
\curveto(464.75771651,26.3652948)(465.34271593,26.28029489)(465.79272217,26.10030273)
\curveto(466.25271502,25.93029524)(466.5727147,25.62529554)(466.75272217,25.18530273)
\curveto(466.80271447,25.07529609)(466.83271444,24.96029621)(466.84272217,24.84030273)
\curveto(466.86271441,24.73029644)(466.87771439,24.60529656)(466.88772217,24.46530273)
\curveto(466.89771437,24.39529677)(466.88771438,24.32029685)(466.85772217,24.24030273)
\curveto(466.83771443,24.170297)(466.81271446,24.11529705)(466.78272217,24.07530273)
\curveto(466.76271451,24.05529711)(466.73271454,24.03529713)(466.69272217,24.01530273)
\curveto(466.66271461,24.00529716)(466.63771463,23.99029718)(466.61772217,23.97030273)
\curveto(466.55771471,23.95029722)(466.50271477,23.94529722)(466.45272217,23.95530273)
\curveto(466.41271486,23.9652972)(466.3677149,23.9652972)(466.31772217,23.95530273)
\curveto(466.22771504,23.93529723)(466.11771515,23.93029724)(465.98772217,23.94030273)
\curveto(465.8677154,23.96029721)(465.78271549,23.98529718)(465.73272217,24.01530273)
\curveto(465.66271561,24.0652971)(465.62271565,24.13029704)(465.61272217,24.21030273)
\curveto(465.61271566,24.30029687)(465.59271568,24.38529678)(465.55272217,24.46530273)
\curveto(465.50271577,24.62529654)(465.40771586,24.7702964)(465.26772217,24.90030273)
\curveto(465.17771609,24.98029619)(465.0677162,25.04029613)(464.93772217,25.08030273)
\curveto(464.81771645,25.12029605)(464.68771658,25.16029601)(464.54772217,25.20030273)
\curveto(464.50771676,25.22029595)(464.45771681,25.22529594)(464.39772217,25.21530273)
\curveto(464.34771692,25.21529595)(464.30271697,25.22029595)(464.26272217,25.23030273)
\curveto(464.20271707,25.25029592)(464.12771714,25.26029591)(464.03772217,25.26030273)
\curveto(463.94771732,25.26029591)(463.8727174,25.25029592)(463.81272217,25.23030273)
\lineto(463.72272217,25.23030273)
\curveto(463.66271761,25.22029595)(463.60771766,25.21029596)(463.55772217,25.20030273)
\curveto(463.50771776,25.20029597)(463.45771781,25.19529597)(463.40772217,25.18530273)
\curveto(463.13771813,25.12529604)(462.90271837,25.04029613)(462.70272217,24.93030273)
\curveto(462.51271876,24.82029635)(462.36271891,24.63529653)(462.25272217,24.37530273)
\curveto(462.22271905,24.30529686)(462.20771906,24.23529693)(462.20772217,24.16530273)
\curveto(462.20771906,24.09529707)(462.21271906,24.03529713)(462.22272217,23.98530273)
\curveto(462.25271902,23.83529733)(462.30271897,23.72529744)(462.37272217,23.65530273)
\curveto(462.44271883,23.59529757)(462.53771873,23.52529764)(462.65772217,23.44530273)
\curveto(462.79771847,23.34529782)(462.96271831,23.2702979)(463.15272217,23.22030273)
\curveto(463.34271793,23.18029799)(463.53271774,23.13029804)(463.72272217,23.07030273)
\curveto(463.84271743,23.03029814)(463.96271731,23.00029817)(464.08272217,22.98030273)
\curveto(464.21271706,22.96029821)(464.33771693,22.93029824)(464.45772217,22.89030273)
\curveto(464.65771661,22.83029834)(464.85271642,22.7702984)(465.04272217,22.71030273)
\curveto(465.23271604,22.66029851)(465.41771585,22.59529857)(465.59772217,22.51530273)
\curveto(465.64771562,22.49529867)(465.69271558,22.47529869)(465.73272217,22.45530273)
\curveto(465.78271549,22.43529873)(465.83271544,22.41029876)(465.88272217,22.38030273)
\curveto(466.05271522,22.26029891)(466.19771507,22.12529904)(466.31772217,21.97530273)
\curveto(466.43771483,21.82529934)(466.52771474,21.63529953)(466.58772217,21.40530273)
\lineto(466.58772217,21.12030273)
\curveto(466.58771468,21.05030012)(466.58271469,20.97530019)(466.57272217,20.89530273)
\curveto(466.56271471,20.82530034)(466.55271472,20.74530042)(466.54272217,20.65530273)
\lineto(466.51272217,20.50530273)
\curveto(466.4727148,20.43530073)(466.44271483,20.3653008)(466.42272217,20.29530273)
\curveto(466.41271486,20.22530094)(466.39271488,20.15530101)(466.36272217,20.08530273)
\curveto(466.31271496,19.97530119)(466.25771501,19.8703013)(466.19772217,19.77030273)
\curveto(466.13771513,19.6703015)(466.0727152,19.58030159)(466.00272217,19.50030273)
\curveto(465.79271548,19.24030193)(465.54771572,19.03030214)(465.26772217,18.87030273)
\curveto(464.98771628,18.72030245)(464.68271659,18.59030258)(464.35272217,18.48030273)
\curveto(464.25271702,18.45030272)(464.15271712,18.43030274)(464.05272217,18.42030273)
\curveto(463.95271732,18.40030277)(463.85771741,18.37530279)(463.76772217,18.34530273)
\curveto(463.65771761,18.32530284)(463.55271772,18.31530285)(463.45272217,18.31530273)
\curveto(463.35271792,18.31530285)(463.25271802,18.30530286)(463.15272217,18.28530273)
\lineto(463.00272217,18.28530273)
\curveto(462.95271832,18.27530289)(462.88271839,18.2703029)(462.79272217,18.27030273)
\curveto(462.70271857,18.2703029)(462.63271864,18.27530289)(462.58272217,18.28530273)
\lineto(462.41772217,18.28530273)
\curveto(462.35771891,18.30530286)(462.29271898,18.31530285)(462.22272217,18.31530273)
\curveto(462.15271912,18.30530286)(462.09771917,18.31030286)(462.05772217,18.33030273)
\curveto(462.00771926,18.34030283)(461.94271933,18.34530282)(461.86272217,18.34530273)
\curveto(461.78271949,18.3653028)(461.70771956,18.38530278)(461.63772217,18.40530273)
\curveto(461.5677197,18.41530275)(461.49271978,18.43530273)(461.41272217,18.46530273)
\curveto(461.12272015,18.5653026)(460.87772039,18.69030248)(460.67772217,18.84030273)
\curveto(460.47772079,18.99030218)(460.31772095,19.18530198)(460.19772217,19.42530273)
\curveto(460.13772113,19.55530161)(460.08772118,19.69030148)(460.04772217,19.83030273)
\curveto(460.01772125,19.9703012)(459.99772127,20.12530104)(459.98772217,20.29530273)
\curveto(459.97772129,20.35530081)(459.98272129,20.42530074)(460.00272217,20.50530273)
\curveto(460.02272125,20.59530057)(460.04772122,20.6653005)(460.07772217,20.71530273)
\curveto(460.11772115,20.75530041)(460.17772109,20.79530037)(460.25772217,20.83530273)
\curveto(460.30772096,20.85530031)(460.37772089,20.8653003)(460.46772217,20.86530273)
\curveto(460.5677207,20.87530029)(460.65772061,20.87530029)(460.73772217,20.86530273)
\curveto(460.82772044,20.85530031)(460.91272036,20.84030033)(460.99272217,20.82030273)
\curveto(461.08272019,20.81030036)(461.13772013,20.79530037)(461.15772217,20.77530273)
\curveto(461.21772005,20.72530044)(461.24772002,20.65030052)(461.24772217,20.55030273)
\curveto(461.25772001,20.46030071)(461.27771999,20.37530079)(461.30772217,20.29530273)
\curveto(461.35771991,20.07530109)(461.45771981,19.90530126)(461.60772217,19.78530273)
\curveto(461.70771956,19.69530147)(461.82771944,19.62530154)(461.96772217,19.57530273)
\curveto(462.10771916,19.52530164)(462.25771901,19.47530169)(462.41772217,19.42530273)
\lineto(462.73272217,19.38030273)
\lineto(462.82272217,19.38030273)
\curveto(462.88271839,19.36030181)(462.9677183,19.35030182)(463.07772217,19.35030273)
\curveto(463.19771807,19.35030182)(463.30271797,19.36030181)(463.39272217,19.38030273)
\curveto(463.46271781,19.38030179)(463.51771775,19.38530178)(463.55772217,19.39530273)
\curveto(463.61771765,19.40530176)(463.67771759,19.41030176)(463.73772217,19.41030273)
\curveto(463.79771747,19.42030175)(463.85271742,19.43030174)(463.90272217,19.44030273)
\curveto(464.21271706,19.52030165)(464.46271681,19.62530154)(464.65272217,19.75530273)
\curveto(464.85271642,19.88530128)(465.01771625,20.10530106)(465.14772217,20.41530273)
\curveto(465.17771609,20.4653007)(465.19271608,20.52030065)(465.19272217,20.58030273)
\curveto(465.20271607,20.64030053)(465.20271607,20.68530048)(465.19272217,20.71530273)
\curveto(465.18271609,20.90530026)(465.14271613,21.04530012)(465.07272217,21.13530273)
\curveto(465.00271627,21.23529993)(464.90771636,21.32529984)(464.78772217,21.40530273)
\curveto(464.70771656,21.4652997)(464.61271666,21.51529965)(464.50272217,21.55530273)
\lineto(464.20272217,21.67530273)
\curveto(464.1727171,21.68529948)(464.14271713,21.69029948)(464.11272217,21.69030273)
\curveto(464.09271718,21.69029948)(464.0727172,21.70029947)(464.05272217,21.72030273)
\curveto(463.73271754,21.83029934)(463.39271788,21.91029926)(463.03272217,21.96030273)
\curveto(462.68271859,22.02029915)(462.36271891,22.11529905)(462.07272217,22.24530273)
\curveto(461.98271929,22.28529888)(461.89271938,22.32029885)(461.80272217,22.35030273)
\curveto(461.72271955,22.38029879)(461.64771962,22.42029875)(461.57772217,22.47030273)
\curveto(461.40771986,22.58029859)(461.25772001,22.70529846)(461.12772217,22.84530273)
\curveto(460.99772027,22.98529818)(460.90772036,23.16029801)(460.85772217,23.37030273)
\curveto(460.83772043,23.44029773)(460.82772044,23.51029766)(460.82772217,23.58030273)
\lineto(460.82772217,23.80530273)
\curveto(460.81772045,23.92529724)(460.83272044,24.06029711)(460.87272217,24.21030273)
\curveto(460.91272036,24.3702968)(460.95272032,24.50529666)(460.99272217,24.61530273)
\curveto(461.02272025,24.6652965)(461.04272023,24.70529646)(461.05272217,24.73530273)
\curveto(461.0727202,24.77529639)(461.09772017,24.81529635)(461.12772217,24.85530273)
\curveto(461.25772001,25.08529608)(461.41771985,25.28529588)(461.60772217,25.45530273)
\curveto(461.79771947,25.62529554)(462.00771926,25.77529539)(462.23772217,25.90530273)
\curveto(462.39771887,25.99529517)(462.5727187,26.0652951)(462.76272217,26.11530273)
\curveto(462.96271831,26.17529499)(463.1677181,26.23029494)(463.37772217,26.28030273)
\curveto(463.44771782,26.29029488)(463.51271776,26.30029487)(463.57272217,26.31030273)
\curveto(463.64271763,26.32029485)(463.71771755,26.33029484)(463.79772217,26.34030273)
\curveto(463.83771743,26.35029482)(463.87771739,26.35029482)(463.91772217,26.34030273)
\curveto(463.9677173,26.33029484)(464.00771726,26.33529483)(464.03772217,26.35530273)
}
}
{
\newrgbcolor{curcolor}{0 0 0}
\pscustom[linestyle=none,fillstyle=solid,fillcolor=curcolor]
{
}
}
{
\newrgbcolor{curcolor}{0 0 0}
\pscustom[linestyle=none,fillstyle=solid,fillcolor=curcolor]
{
\newpath
\moveto(478.80287842,19.26030273)
\lineto(478.71287842,18.87030273)
\curveto(478.69287049,18.75030242)(478.65287053,18.65030252)(478.59287842,18.57030273)
\curveto(478.52287066,18.50030267)(478.42787075,18.46030271)(478.30787842,18.45030273)
\lineto(477.96287842,18.45030273)
\curveto(477.90287128,18.45030272)(477.84287134,18.44530272)(477.78287842,18.43530273)
\curveto(477.73287145,18.43530273)(477.68787149,18.44530272)(477.64787842,18.46530273)
\curveto(477.56787161,18.48530268)(477.51787166,18.52530264)(477.49787842,18.58530273)
\curveto(477.46787171,18.63530253)(477.45787172,18.69530247)(477.46787842,18.76530273)
\curveto(477.4778717,18.83530233)(477.47287171,18.90530226)(477.45287842,18.97530273)
\curveto(477.45287173,18.99530217)(477.44287174,19.01030216)(477.42287842,19.02030273)
\lineto(477.39287842,19.08030273)
\curveto(477.29287189,19.09030208)(477.20787197,19.0703021)(477.13787842,19.02030273)
\curveto(477.0778721,18.9703022)(477.01287217,18.92030225)(476.94287842,18.87030273)
\curveto(476.71287247,18.72030245)(476.48787269,18.60530256)(476.26787842,18.52530273)
\curveto(476.0778731,18.44530272)(475.85787332,18.38530278)(475.60787842,18.34530273)
\curveto(475.36787381,18.30530286)(475.12287406,18.28530288)(474.87287842,18.28530273)
\curveto(474.63287455,18.27530289)(474.39287479,18.29030288)(474.15287842,18.33030273)
\curveto(473.92287526,18.36030281)(473.72787545,18.41530275)(473.56787842,18.49530273)
\curveto(473.08787609,18.71530245)(472.72287646,19.01030216)(472.47287842,19.38030273)
\curveto(472.23287695,19.76030141)(472.0778771,20.23030094)(472.00787842,20.79030273)
\curveto(471.98787719,20.88030029)(471.9778772,20.9703002)(471.97787842,21.06030273)
\curveto(471.98787719,21.16030001)(471.98787719,21.26029991)(471.97787842,21.36030273)
\curveto(471.9778772,21.41029976)(471.9828772,21.46029971)(471.99287842,21.51030273)
\curveto(472.00287718,21.56029961)(472.00787717,21.61029956)(472.00787842,21.66030273)
\curveto(471.99787718,21.71029946)(471.99787718,21.76029941)(472.00787842,21.81030273)
\curveto(472.02787715,21.8702993)(472.03787714,21.92529924)(472.03787842,21.97530273)
\lineto(472.06787842,22.12530273)
\curveto(472.05787712,22.17529899)(472.05787712,22.24029893)(472.06787842,22.32030273)
\curveto(472.08787709,22.40029877)(472.11287707,22.4652987)(472.14287842,22.51530273)
\lineto(472.18787842,22.68030273)
\curveto(472.21787696,22.75029842)(472.23787694,22.82029835)(472.24787842,22.89030273)
\curveto(472.25787692,22.9702982)(472.2778769,23.04529812)(472.30787842,23.11530273)
\curveto(472.32787685,23.165298)(472.34287684,23.21029796)(472.35287842,23.25030273)
\curveto(472.36287682,23.29029788)(472.3778768,23.33529783)(472.39787842,23.38530273)
\curveto(472.44787673,23.48529768)(472.49287669,23.58029759)(472.53287842,23.67030273)
\curveto(472.57287661,23.7702974)(472.61787656,23.8652973)(472.66787842,23.95530273)
\curveto(472.86787631,24.33529683)(473.09787608,24.67529649)(473.35787842,24.97530273)
\curveto(473.62787555,25.28529588)(473.92787525,25.54029563)(474.25787842,25.74030273)
\curveto(474.45787472,25.86029531)(474.65787452,25.96029521)(474.85787842,26.04030273)
\curveto(475.05787412,26.12029505)(475.27287391,26.19029498)(475.50287842,26.25030273)
\lineto(475.71287842,26.28030273)
\curveto(475.7828734,26.29029488)(475.85287333,26.30529486)(475.92287842,26.32530273)
\lineto(476.07287842,26.32530273)
\curveto(476.16287302,26.34529482)(476.2828729,26.35529481)(476.43287842,26.35530273)
\curveto(476.59287259,26.35529481)(476.70787247,26.34529482)(476.77787842,26.32530273)
\curveto(476.81787236,26.31529485)(476.87287231,26.31029486)(476.94287842,26.31030273)
\curveto(477.04287214,26.28029489)(477.14787203,26.25529491)(477.25787842,26.23530273)
\curveto(477.36787181,26.22529494)(477.46787171,26.19529497)(477.55787842,26.14530273)
\curveto(477.69787148,26.08529508)(477.82787135,26.02029515)(477.94787842,25.95030273)
\curveto(478.06787111,25.88029529)(478.177871,25.80029537)(478.27787842,25.71030273)
\curveto(478.32787085,25.66029551)(478.3778708,25.60529556)(478.42787842,25.54530273)
\curveto(478.48787069,25.49529567)(478.57287061,25.48029569)(478.68287842,25.50030273)
\lineto(478.75787842,25.57530273)
\curveto(478.7778704,25.59529557)(478.79287039,25.62529554)(478.80287842,25.66530273)
\curveto(478.85287033,25.75529541)(478.88787029,25.8702953)(478.90787842,26.01030273)
\curveto(478.93787024,26.15029502)(478.96287022,26.27529489)(478.98287842,26.38530273)
\lineto(479.32787842,28.11030273)
\curveto(479.35786982,28.25029292)(479.38786979,28.40529276)(479.41787842,28.57530273)
\curveto(479.45786972,28.75529241)(479.50786967,28.88529228)(479.56787842,28.96530273)
\curveto(479.62786955,29.03529213)(479.69786948,29.08029209)(479.77787842,29.10030273)
\curveto(479.79786938,29.10029207)(479.82286936,29.10029207)(479.85287842,29.10030273)
\curveto(479.8828693,29.11029206)(479.90786927,29.11529205)(479.92787842,29.11530273)
\curveto(480.0778691,29.12529204)(480.22786895,29.12529204)(480.37787842,29.11530273)
\curveto(480.52786865,29.11529205)(480.62786855,29.07529209)(480.67787842,28.99530273)
\curveto(480.70786847,28.91529225)(480.70786847,28.81529235)(480.67787842,28.69530273)
\curveto(480.65786852,28.57529259)(480.63786854,28.45029272)(480.61787842,28.32030273)
\lineto(478.80287842,19.26030273)
\moveto(478.15787842,22.09530273)
\curveto(478.18787099,22.14529902)(478.20787097,22.21029896)(478.21787842,22.29030273)
\curveto(478.23787094,22.38029879)(478.24287094,22.45029872)(478.23287842,22.50030273)
\lineto(478.27787842,22.72530273)
\curveto(478.2778709,22.81529835)(478.2828709,22.90529826)(478.29287842,22.99530273)
\curveto(478.30287088,23.09529807)(478.29787088,23.18529798)(478.27787842,23.26530273)
\lineto(478.27787842,23.49030273)
\curveto(478.2778709,23.56029761)(478.26787091,23.63029754)(478.24787842,23.70030273)
\curveto(478.18787099,24.00029717)(478.0828711,24.2652969)(477.93287842,24.49530273)
\curveto(477.79287139,24.72529644)(477.59287159,24.90529626)(477.33287842,25.03530273)
\curveto(477.24287194,25.08529608)(477.14787203,25.12029605)(477.04787842,25.14030273)
\curveto(476.94787223,25.170296)(476.83787234,25.19529597)(476.71787842,25.21530273)
\curveto(476.64787253,25.23529593)(476.56287262,25.24529592)(476.46287842,25.24530273)
\lineto(476.19287842,25.24530273)
\lineto(476.04287842,25.21530273)
\lineto(475.90787842,25.21530273)
\curveto(475.82787335,25.19529597)(475.74287344,25.17529599)(475.65287842,25.15530273)
\curveto(475.56287362,25.13529603)(475.4778737,25.11029606)(475.39787842,25.08030273)
\curveto(475.04787413,24.94029623)(474.74787443,24.73529643)(474.49787842,24.46530273)
\curveto(474.24787493,24.20529696)(474.02787515,23.90029727)(473.83787842,23.55030273)
\curveto(473.7778754,23.44029773)(473.72787545,23.32529784)(473.68787842,23.20530273)
\lineto(473.56787842,22.87530273)
\lineto(473.53787842,22.75530273)
\curveto(473.52787565,22.72529844)(473.51787566,22.69029848)(473.50787842,22.65030273)
\curveto(473.4778757,22.60029857)(473.45787572,22.54529862)(473.44787842,22.48530273)
\curveto(473.44787573,22.42529874)(473.44287574,22.3702988)(473.43287842,22.32030273)
\curveto(473.41287577,22.21029896)(473.38787579,22.10029907)(473.35787842,21.99030273)
\curveto(473.33787584,21.89029928)(473.33287585,21.79529937)(473.34287842,21.70530273)
\curveto(473.34287584,21.67529949)(473.33787584,21.62529954)(473.32787842,21.55530273)
\lineto(473.32787842,21.34530273)
\curveto(473.32787585,21.27529989)(473.33287585,21.20529996)(473.34287842,21.13530273)
\curveto(473.3828758,20.78530038)(473.47287571,20.48530068)(473.61287842,20.23530273)
\curveto(473.75287543,19.98530118)(473.95287523,19.78030139)(474.21287842,19.62030273)
\curveto(474.29287489,19.5703016)(474.37287481,19.53030164)(474.45287842,19.50030273)
\curveto(474.54287464,19.4703017)(474.63787454,19.44030173)(474.73787842,19.41030273)
\curveto(474.78787439,19.39030178)(474.83787434,19.38530178)(474.88787842,19.39530273)
\curveto(474.94787423,19.40530176)(475.00287418,19.40030177)(475.05287842,19.38030273)
\curveto(475.0828741,19.3703018)(475.11787406,19.3653018)(475.15787842,19.36530273)
\lineto(475.29287842,19.36530273)
\lineto(475.42787842,19.36530273)
\curveto(475.46787371,19.37530179)(475.52287366,19.38030179)(475.59287842,19.38030273)
\curveto(475.67287351,19.40030177)(475.75287343,19.41530175)(475.83287842,19.42530273)
\curveto(475.92287326,19.44530172)(476.00287318,19.4703017)(476.07287842,19.50030273)
\curveto(476.43287275,19.64030153)(476.73787244,19.81530135)(476.98787842,20.02530273)
\curveto(477.23787194,20.24530092)(477.46287172,20.52030065)(477.66287842,20.85030273)
\curveto(477.73287145,20.96030021)(477.78787139,21.0703001)(477.82787842,21.18030273)
\lineto(477.97787842,21.51030273)
\curveto(478.00787117,21.55029962)(478.02287116,21.58529958)(478.02287842,21.61530273)
\curveto(478.03287115,21.65529951)(478.04787113,21.69529947)(478.06787842,21.73530273)
\curveto(478.08787109,21.79529937)(478.10287108,21.85529931)(478.11287842,21.91530273)
\curveto(478.12287106,21.97529919)(478.13787104,22.03529913)(478.15787842,22.09530273)
}
}
{
\newrgbcolor{curcolor}{0 0 0}
\pscustom[linestyle=none,fillstyle=solid,fillcolor=curcolor]
{
\newpath
\moveto(488.17412842,22.62030273)
\curveto(488.17411991,22.52029865)(488.15411993,22.40529876)(488.11412842,22.27530273)
\curveto(488.07412001,22.15529901)(488.02412006,22.0702991)(487.96412842,22.02030273)
\curveto(487.90412018,21.98029919)(487.82412026,21.95029922)(487.72412842,21.93030273)
\curveto(487.62412046,21.92029925)(487.51412057,21.91529925)(487.39412842,21.91530273)
\lineto(487.03412842,21.91530273)
\curveto(486.92412116,21.92529924)(486.82412126,21.93029924)(486.73412842,21.93030273)
\lineto(482.89412842,21.93030273)
\curveto(482.81412527,21.93029924)(482.72912536,21.92529924)(482.63912842,21.91530273)
\curveto(482.55912553,21.91529925)(482.49412559,21.90029927)(482.44412842,21.87030273)
\curveto(482.39412569,21.85029932)(482.34412574,21.81029936)(482.29412842,21.75030273)
\lineto(482.20412842,21.61530273)
\curveto(482.17412591,21.5652996)(482.16412592,21.51529965)(482.17412842,21.46530273)
\curveto(482.17412591,21.41529975)(482.16912592,21.3702998)(482.15912842,21.33030273)
\lineto(482.15912842,21.21030273)
\lineto(482.15912842,20.95530273)
\curveto(482.16912592,20.87530029)(482.1841259,20.79530037)(482.20412842,20.71530273)
\curveto(482.33412575,20.17530099)(482.63912545,19.79030138)(483.11912842,19.56030273)
\curveto(483.16912492,19.53030164)(483.22912486,19.50530166)(483.29912842,19.48530273)
\curveto(483.36912472,19.4653017)(483.43412465,19.44530172)(483.49412842,19.42530273)
\curveto(483.52412456,19.41530175)(483.57412451,19.41030176)(483.64412842,19.41030273)
\curveto(483.77412431,19.3703018)(483.95412413,19.35030182)(484.18412842,19.35030273)
\curveto(484.41412367,19.35030182)(484.60412348,19.3703018)(484.75412842,19.41030273)
\curveto(484.90412318,19.45030172)(485.03912305,19.49030168)(485.15912842,19.53030273)
\curveto(485.2891228,19.58030159)(485.40912268,19.64030153)(485.51912842,19.71030273)
\curveto(485.63912245,19.78030139)(485.74912234,19.86030131)(485.84912842,19.95030273)
\curveto(485.94912214,20.05030112)(486.03912205,20.15530101)(486.11912842,20.26530273)
\curveto(486.19912189,20.3653008)(486.27412181,20.4703007)(486.34412842,20.58030273)
\curveto(486.41412167,20.69030048)(486.50912158,20.7703004)(486.62912842,20.82030273)
\curveto(486.66912142,20.84030033)(486.73412135,20.85530031)(486.82412842,20.86530273)
\curveto(486.92412116,20.87530029)(487.01412107,20.87530029)(487.09412842,20.86530273)
\curveto(487.1841209,20.8653003)(487.26912082,20.86030031)(487.34912842,20.85030273)
\curveto(487.42912066,20.84030033)(487.47912061,20.82030035)(487.49912842,20.79030273)
\curveto(487.5891205,20.72030045)(487.59412049,20.60530056)(487.51412842,20.44530273)
\curveto(487.37412071,20.17530099)(487.21912087,19.93530123)(487.04912842,19.72530273)
\curveto(486.7891213,19.40530176)(486.50912158,19.14030203)(486.20912842,18.93030273)
\curveto(485.91912217,18.73030244)(485.56412252,18.5653026)(485.14412842,18.43530273)
\curveto(485.03412305,18.39530277)(484.92912316,18.3703028)(484.82912842,18.36030273)
\curveto(484.72912336,18.34030283)(484.61912347,18.32030285)(484.49912842,18.30030273)
\curveto(484.44912364,18.29030288)(484.39912369,18.28530288)(484.34912842,18.28530273)
\curveto(484.30912378,18.28530288)(484.26412382,18.28030289)(484.21412842,18.27030273)
\lineto(484.06412842,18.27030273)
\curveto(484.01412407,18.26030291)(483.95412413,18.25530291)(483.88412842,18.25530273)
\curveto(483.82412426,18.25530291)(483.77412431,18.26030291)(483.73412842,18.27030273)
\lineto(483.59912842,18.27030273)
\curveto(483.54912454,18.28030289)(483.50412458,18.28530288)(483.46412842,18.28530273)
\curveto(483.42412466,18.28530288)(483.3841247,18.29030288)(483.34412842,18.30030273)
\curveto(483.29412479,18.31030286)(483.23912485,18.32030285)(483.17912842,18.33030273)
\curveto(483.12912496,18.33030284)(483.07912501,18.33530283)(483.02912842,18.34530273)
\curveto(482.93912515,18.3653028)(482.84912524,18.39030278)(482.75912842,18.42030273)
\curveto(482.67912541,18.44030273)(482.60412548,18.4653027)(482.53412842,18.49530273)
\curveto(482.49412559,18.51530265)(482.45912563,18.52530264)(482.42912842,18.52530273)
\curveto(482.39912569,18.53530263)(482.36912572,18.55030262)(482.33912842,18.57030273)
\curveto(482.19912589,18.64030253)(482.05412603,18.72530244)(481.90412842,18.82530273)
\curveto(481.65412643,19.01530215)(481.45412663,19.24530192)(481.30412842,19.51530273)
\curveto(481.15412693,19.79530137)(481.04412704,20.10530106)(480.97412842,20.44530273)
\curveto(480.94412714,20.55530061)(480.92912716,20.6703005)(480.92912842,20.79030273)
\curveto(480.92912716,20.91030026)(480.91912717,21.03030014)(480.89912842,21.15030273)
\lineto(480.89912842,21.25530273)
\curveto(480.90912718,21.28529988)(480.91412717,21.32529984)(480.91412842,21.37530273)
\lineto(480.91412842,21.63030273)
\curveto(480.92412716,21.72029945)(480.92912716,21.81029936)(480.92912842,21.90030273)
\lineto(480.97412842,22.11030273)
\curveto(480.97412711,22.15029902)(480.97912711,22.20529896)(480.98912842,22.27530273)
\curveto(480.99912709,22.35529881)(481.01412707,22.42029875)(481.03412842,22.47030273)
\lineto(481.06412842,22.63530273)
\curveto(481.09412699,22.68529848)(481.10912698,22.73529843)(481.10912842,22.78530273)
\curveto(481.11912697,22.84529832)(481.13412695,22.90029827)(481.15412842,22.95030273)
\curveto(481.22412686,23.11029806)(481.2891268,23.2702979)(481.34912842,23.43030273)
\curveto(481.40912668,23.59029758)(481.4841266,23.74029743)(481.57412842,23.88030273)
\curveto(481.64412644,23.99029718)(481.70912638,24.10029707)(481.76912842,24.21030273)
\curveto(481.83912625,24.33029684)(481.91912617,24.44529672)(482.00912842,24.55530273)
\curveto(482.29912579,24.90529626)(482.60912548,25.20529596)(482.93912842,25.45530273)
\curveto(483.26912482,25.71529545)(483.65412443,25.93029524)(484.09412842,26.10030273)
\curveto(484.22412386,26.15029502)(484.35412373,26.18529498)(484.48412842,26.20530273)
\curveto(484.61412347,26.23529493)(484.75412333,26.2652949)(484.90412842,26.29530273)
\curveto(484.95412313,26.30529486)(484.99912309,26.31029486)(485.03912842,26.31030273)
\curveto(485.07912301,26.32029485)(485.12412296,26.32529484)(485.17412842,26.32530273)
\curveto(485.19412289,26.33529483)(485.21912287,26.33529483)(485.24912842,26.32530273)
\curveto(485.27912281,26.31529485)(485.30412278,26.32029485)(485.32412842,26.34030273)
\curveto(485.75412233,26.35029482)(486.11412197,26.30529486)(486.40412842,26.20530273)
\curveto(486.69412139,26.11529505)(486.94912114,25.99029518)(487.16912842,25.83030273)
\curveto(487.20912088,25.81029536)(487.23912085,25.78029539)(487.25912842,25.74030273)
\curveto(487.2891208,25.71029546)(487.31912077,25.68529548)(487.34912842,25.66530273)
\curveto(487.41912067,25.60529556)(487.4891206,25.53529563)(487.55912842,25.45530273)
\curveto(487.62912046,25.37529579)(487.6841204,25.29529587)(487.72412842,25.21530273)
\curveto(487.84412024,25.00529616)(487.93912015,24.80529636)(488.00912842,24.61530273)
\curveto(488.05912003,24.50529666)(488.08912,24.38529678)(488.09912842,24.25530273)
\lineto(488.15912842,23.86530273)
\curveto(488.1891199,23.73529743)(488.19911989,23.60029757)(488.18912842,23.46030273)
\curveto(488.1891199,23.32029785)(488.19411989,23.18029799)(488.20412842,23.04030273)
\curveto(488.20411988,22.9702982)(488.19911989,22.90029827)(488.18912842,22.83030273)
\curveto(488.17911991,22.76029841)(488.17411991,22.69029848)(488.17412842,22.62030273)
\moveto(486.82412842,23.13030273)
\curveto(486.85412123,23.170298)(486.8841212,23.22029795)(486.91412842,23.28030273)
\curveto(486.95412113,23.35029782)(486.96912112,23.42029775)(486.95912842,23.49030273)
\curveto(486.94912114,23.71029746)(486.90912118,23.91529725)(486.83912842,24.10530273)
\curveto(486.73912135,24.33529683)(486.61912147,24.53029664)(486.47912842,24.69030273)
\curveto(486.34912174,24.85029632)(486.15912193,24.98529618)(485.90912842,25.09530273)
\curveto(485.83912225,25.11529605)(485.76912232,25.13029604)(485.69912842,25.14030273)
\curveto(485.63912245,25.16029601)(485.56912252,25.18029599)(485.48912842,25.20030273)
\curveto(485.41912267,25.22029595)(485.33912275,25.23029594)(485.24912842,25.23030273)
\lineto(484.99412842,25.23030273)
\curveto(484.95412313,25.21029596)(484.91412317,25.20029597)(484.87412842,25.20030273)
\curveto(484.83412325,25.21029596)(484.79912329,25.21029596)(484.76912842,25.20030273)
\lineto(484.52912842,25.14030273)
\curveto(484.44912364,25.13029604)(484.37412371,25.11529605)(484.30412842,25.09530273)
\curveto(483.9841241,24.97529619)(483.71912437,24.82529634)(483.50912842,24.64530273)
\curveto(483.29912479,24.4652967)(483.09912499,24.24029693)(482.90912842,23.97030273)
\curveto(482.86912522,23.92029725)(482.82412526,23.85529731)(482.77412842,23.77530273)
\curveto(482.73412535,23.70529746)(482.69412539,23.62529754)(482.65412842,23.53530273)
\curveto(482.61412547,23.44529772)(482.5891255,23.36029781)(482.57912842,23.28030273)
\curveto(482.57912551,23.20029797)(482.60412548,23.14029803)(482.65412842,23.10030273)
\curveto(482.72412536,23.04029813)(482.85412523,23.01029816)(483.04412842,23.01030273)
\curveto(483.24412484,23.02029815)(483.41412467,23.02529814)(483.55412842,23.02530273)
\lineto(485.83412842,23.02530273)
\curveto(485.9841221,23.02529814)(486.16412192,23.02029815)(486.37412842,23.01030273)
\curveto(486.5841215,23.01029816)(486.73412135,23.05029812)(486.82412842,23.13030273)
}
}
{
\newrgbcolor{curcolor}{0 0 0}
\pscustom[linestyle=none,fillstyle=solid,fillcolor=curcolor]
{
\newpath
\moveto(493.24576904,29.25030273)
\curveto(493.42576334,29.26029191)(493.61576315,29.26029191)(493.81576904,29.25030273)
\curveto(494.01576275,29.24029193)(494.14576262,29.18029199)(494.20576904,29.07030273)
\curveto(494.23576253,29.01029216)(494.24576252,28.93529223)(494.23576904,28.84530273)
\curveto(494.22576254,28.7652924)(494.21076256,28.67529249)(494.19076904,28.57530273)
\curveto(494.1707626,28.44529272)(494.12576264,28.34029283)(494.05576904,28.26030273)
\curveto(494.00576276,28.21029296)(493.94076283,28.17529299)(493.86076904,28.15530273)
\curveto(493.78076299,28.14529302)(493.69576307,28.14029303)(493.60576904,28.14030273)
\lineto(493.33576904,28.14030273)
\curveto(493.24576352,28.15029302)(493.16076361,28.15029302)(493.08076904,28.14030273)
\curveto(492.79076398,28.06029311)(492.58576418,27.93029324)(492.46576904,27.75030273)
\curveto(492.34576442,27.58029359)(492.25076452,27.32029385)(492.18076904,26.97030273)
\curveto(492.16076461,26.90029427)(492.13576463,26.82529434)(492.10576904,26.74530273)
\curveto(492.08576468,26.67529449)(492.08076469,26.61029456)(492.09076904,26.55030273)
\curveto(492.09076468,26.40029477)(492.13576463,26.29529487)(492.22576904,26.23530273)
\curveto(492.29576447,26.20529496)(492.39076438,26.19029498)(492.51076904,26.19030273)
\lineto(492.87076904,26.19030273)
\lineto(493.09576904,26.19030273)
\curveto(493.12576364,26.170295)(493.15576361,26.165295)(493.18576904,26.17530273)
\curveto(493.21576355,26.18529498)(493.24576352,26.18029499)(493.27576904,26.16030273)
\curveto(493.3657634,26.13029504)(493.41576335,26.0702951)(493.42576904,25.98030273)
\curveto(493.44576332,25.90029527)(493.44076333,25.79529537)(493.41076904,25.66530273)
\lineto(493.38076904,25.54530273)
\lineto(493.35076904,25.42530273)
\curveto(493.29076348,25.27529589)(493.20576356,25.17529599)(493.09576904,25.12530273)
\curveto(492.95576381,25.07529609)(492.78576398,25.06029611)(492.58576904,25.08030273)
\curveto(492.38576438,25.11029606)(492.21076456,25.10529606)(492.06076904,25.06530273)
\curveto(491.98076479,25.04529612)(491.91576485,25.00529616)(491.86576904,24.94530273)
\curveto(491.81576495,24.89529627)(491.770765,24.82529634)(491.73076904,24.73530273)
\curveto(491.70076507,24.6652965)(491.68076509,24.58529658)(491.67076904,24.49530273)
\curveto(491.66076511,24.40529676)(491.64576512,24.32029685)(491.62576904,24.24030273)
\lineto(491.43076904,23.25030273)
\lineto(490.80076904,20.07030273)
\lineto(490.65076904,19.32030273)
\curveto(490.64076613,19.26030191)(490.63076614,19.19530197)(490.62076904,19.12530273)
\curveto(490.61076616,19.05530211)(490.59076618,18.99530217)(490.56076904,18.94530273)
\lineto(490.53076904,18.82530273)
\lineto(490.47076904,18.70530273)
\curveto(490.46076631,18.6653025)(490.44076633,18.63030254)(490.41076904,18.60030273)
\curveto(490.35076642,18.53030264)(490.2657665,18.49030268)(490.15576904,18.48030273)
\curveto(490.05576671,18.4703027)(489.94576682,18.4653027)(489.82576904,18.46530273)
\lineto(489.54076904,18.46530273)
\curveto(489.50076727,18.48530268)(489.45576731,18.50030267)(489.40576904,18.51030273)
\curveto(489.3657674,18.53030264)(489.33576743,18.5653026)(489.31576904,18.61530273)
\curveto(489.30576746,18.64530252)(489.30076747,18.71030246)(489.30076904,18.81030273)
\lineto(489.31576904,18.91530273)
\curveto(489.30576746,18.9653022)(489.31076746,19.01530215)(489.33076904,19.06530273)
\curveto(489.35076742,19.12530204)(489.3657674,19.18030199)(489.37576904,19.23030273)
\lineto(489.49576904,19.83030273)
\lineto(490.30576904,23.92530273)
\curveto(490.32576644,24.03529713)(490.35076642,24.15029702)(490.38076904,24.27030273)
\curveto(490.41076636,24.39029678)(490.43076634,24.50029667)(490.44076904,24.60030273)
\curveto(490.46076631,24.71029646)(490.46076631,24.80529636)(490.44076904,24.88530273)
\curveto(490.43076634,24.9652962)(490.38576638,25.02029615)(490.30576904,25.05030273)
\curveto(490.25576651,25.08029609)(490.19076658,25.09529607)(490.11076904,25.09530273)
\lineto(489.88576904,25.09530273)
\lineto(489.64576904,25.09530273)
\curveto(489.57576719,25.09529607)(489.51076726,25.10529606)(489.45076904,25.12530273)
\curveto(489.3707674,25.165296)(489.32576744,25.25029592)(489.31576904,25.38030273)
\lineto(489.31576904,25.51530273)
\curveto(489.32576744,25.55529561)(489.33576743,25.60029557)(489.34576904,25.65030273)
\curveto(489.37576739,25.79029538)(489.41076736,25.90029527)(489.45076904,25.98030273)
\curveto(489.50076727,26.0702951)(489.58076719,26.13029504)(489.69076904,26.16030273)
\curveto(489.770767,26.19029498)(489.85576691,26.20029497)(489.94576904,26.19030273)
\lineto(490.21576904,26.19030273)
\curveto(490.31576645,26.19029498)(490.40576636,26.20029497)(490.48576904,26.22030273)
\curveto(490.5657662,26.24029493)(490.63576613,26.28029489)(490.69576904,26.34030273)
\curveto(490.78576598,26.42029475)(490.84576592,26.54529462)(490.87576904,26.71530273)
\curveto(490.90576586,26.88529428)(490.93576583,27.04529412)(490.96576904,27.19530273)
\curveto(491.00576576,27.39529377)(491.05576571,27.58029359)(491.11576904,27.75030273)
\curveto(491.17576559,27.93029324)(491.25076552,28.09029308)(491.34076904,28.23030273)
\curveto(491.49076528,28.4702927)(491.6707651,28.6652925)(491.88076904,28.81530273)
\curveto(492.10076467,28.9652922)(492.35076442,29.08029209)(492.63076904,29.16030273)
\curveto(492.69076408,29.18029199)(492.75576401,29.19029198)(492.82576904,29.19030273)
\curveto(492.89576387,29.20029197)(492.9657638,29.21529195)(493.03576904,29.23530273)
\curveto(493.05576371,29.24529192)(493.09076368,29.24529192)(493.14076904,29.23530273)
\curveto(493.19076358,29.23529193)(493.22576354,29.24029193)(493.24576904,29.25030273)
\moveto(495.19576904,27.67530273)
\curveto(495.25576151,27.62529354)(495.33576143,27.60029357)(495.43576904,27.60030273)
\lineto(495.75076904,27.60030273)
\lineto(495.91576904,27.60030273)
\curveto(495.97576079,27.60029357)(496.03576073,27.61029356)(496.09576904,27.63030273)
\curveto(496.23576053,27.68029349)(496.32076045,27.78529338)(496.35076904,27.94530273)
\curveto(496.39076038,28.10529306)(496.43076034,28.27529289)(496.47076904,28.45530273)
\curveto(496.48076029,28.54529262)(496.49576027,28.63029254)(496.51576904,28.71030273)
\curveto(496.53576023,28.80029237)(496.53576023,28.87529229)(496.51576904,28.93530273)
\curveto(496.48576028,29.04529212)(496.39576037,29.10529206)(496.24576904,29.11530273)
\curveto(496.10576066,29.12529204)(495.95076082,29.13029204)(495.78076904,29.13030273)
\curveto(495.75076102,29.12029205)(495.72576104,29.11529205)(495.70576904,29.11530273)
\curveto(495.68576108,29.12529204)(495.66076111,29.12529204)(495.63076904,29.11530273)
\curveto(495.51076126,29.07529209)(495.42076135,29.01529215)(495.36076904,28.93530273)
\curveto(495.32076145,28.87529229)(495.29076148,28.80029237)(495.27076904,28.71030273)
\curveto(495.25076152,28.62029255)(495.23576153,28.53529263)(495.22576904,28.45530273)
\curveto(495.19576157,28.30529286)(495.1657616,28.15029302)(495.13576904,27.99030273)
\curveto(495.10576166,27.84029333)(495.12576164,27.73529343)(495.19576904,27.67530273)
\moveto(495.87076904,25.51530273)
\curveto(495.89076088,25.61529555)(495.91076086,25.71029546)(495.93076904,25.80030273)
\curveto(495.95076082,25.90029527)(495.94076083,25.98029519)(495.90076904,26.04030273)
\curveto(495.8707609,26.12029505)(495.78576098,26.16029501)(495.64576904,26.16030273)
\curveto(495.51576125,26.170295)(495.38576138,26.17529499)(495.25576904,26.17530273)
\curveto(495.23576153,26.165295)(495.21076156,26.16029501)(495.18076904,26.16030273)
\curveto(495.16076161,26.170295)(495.14076163,26.17529499)(495.12076904,26.17530273)
\curveto(495.06076171,26.15529501)(495.00076177,26.14029503)(494.94076904,26.13030273)
\curveto(494.89076188,26.12029505)(494.84576192,26.09029508)(494.80576904,26.04030273)
\curveto(494.74576202,25.98029519)(494.70576206,25.89529527)(494.68576904,25.78530273)
\curveto(494.6657621,25.68529548)(494.64576212,25.58029559)(494.62576904,25.47030273)
\lineto(493.35076904,19.12530273)
\curveto(493.33076344,19.03530213)(493.31076346,18.94030223)(493.29076904,18.84030273)
\curveto(493.28076349,18.75030242)(493.28576348,18.67530249)(493.30576904,18.61530273)
\curveto(493.34576342,18.53530263)(493.41076336,18.48530268)(493.50076904,18.46530273)
\curveto(493.59076318,18.45530271)(493.70076307,18.45030272)(493.83076904,18.45030273)
\lineto(494.05576904,18.45030273)
\curveto(494.14576262,18.4703027)(494.22076255,18.48530268)(494.28076904,18.49530273)
\curveto(494.34076243,18.51530265)(494.39076238,18.55530261)(494.43076904,18.61530273)
\curveto(494.50076227,18.67530249)(494.54076223,18.75530241)(494.55076904,18.85530273)
\curveto(494.5707622,18.9653022)(494.59076218,19.0703021)(494.61076904,19.17030273)
\lineto(495.87076904,25.51530273)
}
}
{
\newrgbcolor{curcolor}{0 0 0}
\pscustom[linestyle=none,fillstyle=solid,fillcolor=curcolor]
{
\newpath
\moveto(501.66944092,26.32530273)
\curveto(502.3094341,26.34529482)(502.79943361,26.26029491)(503.13944092,26.07030273)
\curveto(503.47943293,25.88029529)(503.72443268,25.59529557)(503.87444092,25.21530273)
\curveto(503.91443249,25.11529605)(503.93943247,25.00529616)(503.94944092,24.88530273)
\curveto(503.96943244,24.77529639)(503.97943243,24.66029651)(503.97944092,24.54030273)
\curveto(503.99943241,24.35029682)(503.98943242,24.14529702)(503.94944092,23.92530273)
\curveto(503.91943249,23.70529746)(503.87943253,23.48029769)(503.82944092,23.25030273)
\lineto(503.51444092,21.64530273)
\lineto(503.04944092,19.30530273)
\lineto(502.92944092,18.79530273)
\curveto(502.88943352,18.62530254)(502.79943361,18.51530265)(502.65944092,18.46530273)
\curveto(502.6094338,18.44530272)(502.55443385,18.43530273)(502.49444092,18.43530273)
\curveto(502.44443396,18.42530274)(502.38943402,18.42030275)(502.32944092,18.42030273)
\curveto(502.19943421,18.42030275)(502.07443433,18.42530274)(501.95444092,18.43530273)
\curveto(501.83443457,18.43530273)(501.75943465,18.47530269)(501.72944092,18.55530273)
\curveto(501.68943472,18.62530254)(501.67943473,18.71530245)(501.69944092,18.82530273)
\curveto(501.71943469,18.93530223)(501.74443466,19.04530212)(501.77444092,19.15530273)
\lineto(502.02944092,20.44530273)
\lineto(502.50944092,22.89030273)
\curveto(502.56943384,23.16029801)(502.61943379,23.42529774)(502.65944092,23.68530273)
\curveto(502.69943371,23.95529721)(502.69943371,24.18529698)(502.65944092,24.37530273)
\curveto(502.61943379,24.57529659)(502.52943388,24.73529643)(502.38944092,24.85530273)
\curveto(502.25943415,24.98529618)(502.09943431,25.08529608)(501.90944092,25.15530273)
\curveto(501.84943456,25.17529599)(501.78443462,25.18529598)(501.71444092,25.18530273)
\curveto(501.65443475,25.19529597)(501.59943481,25.21029596)(501.54944092,25.23030273)
\curveto(501.49943491,25.24029593)(501.41943499,25.24029593)(501.30944092,25.23030273)
\curveto(501.2094352,25.23029594)(501.13443527,25.22529594)(501.08444092,25.21530273)
\curveto(501.04443536,25.19529597)(501.0094354,25.18529598)(500.97944092,25.18530273)
\curveto(500.94943546,25.19529597)(500.91443549,25.19529597)(500.87444092,25.18530273)
\curveto(500.73443567,25.15529601)(500.6044358,25.12029605)(500.48444092,25.08030273)
\curveto(500.36443604,25.05029612)(500.24943616,25.00529616)(500.13944092,24.94530273)
\curveto(500.08943632,24.92529624)(500.04943636,24.90529626)(500.01944092,24.88530273)
\curveto(499.98943642,24.8652963)(499.94943646,24.84529632)(499.89944092,24.82530273)
\curveto(499.49943691,24.57529659)(499.16943724,24.20029697)(498.90944092,23.70030273)
\curveto(498.86943754,23.62029755)(498.83443757,23.53529763)(498.80444092,23.44530273)
\lineto(498.71444092,23.20530273)
\curveto(498.68443772,23.15529801)(498.66943774,23.10529806)(498.66944092,23.05530273)
\curveto(498.66943774,23.01529815)(498.65443775,22.97529819)(498.62444092,22.93530273)
\lineto(498.56444092,22.62030273)
\curveto(498.54443786,22.59029858)(498.53443787,22.55529861)(498.53444092,22.51530273)
\curveto(498.53443787,22.47529869)(498.52943788,22.43029874)(498.51944092,22.38030273)
\lineto(498.42944092,21.93030273)
\lineto(498.12944092,20.49030273)
\lineto(497.87444092,19.17030273)
\curveto(497.85443855,19.06030211)(497.82943858,18.94530222)(497.79944092,18.82530273)
\curveto(497.77943863,18.71530245)(497.73943867,18.62530254)(497.67944092,18.55530273)
\curveto(497.6094388,18.47530269)(497.5094389,18.43530273)(497.37944092,18.43530273)
\curveto(497.25943915,18.42530274)(497.13443927,18.42030275)(497.00444092,18.42030273)
\curveto(496.92443948,18.42030275)(496.84943956,18.42530274)(496.77944092,18.43530273)
\curveto(496.7094397,18.44530272)(496.65443975,18.4703027)(496.61444092,18.51030273)
\curveto(496.54443986,18.56030261)(496.52443988,18.65530251)(496.55444092,18.79530273)
\curveto(496.58443982,18.93530223)(496.6094398,19.0703021)(496.62944092,19.20030273)
\lineto(496.98944092,20.97030273)
\lineto(497.70944092,24.60030273)
\lineto(497.88944092,25.51530273)
\lineto(497.94944092,25.78530273)
\curveto(497.96943844,25.87529529)(498.0044384,25.94529522)(498.05444092,25.99530273)
\curveto(498.09443831,26.05529511)(498.14943826,26.09529507)(498.21944092,26.11530273)
\curveto(498.26943814,26.12529504)(498.32943808,26.13529503)(498.39944092,26.14530273)
\curveto(498.47943793,26.15529501)(498.55943785,26.16029501)(498.63944092,26.16030273)
\curveto(498.71943769,26.16029501)(498.79443761,26.15529501)(498.86444092,26.14530273)
\curveto(498.94443746,26.13529503)(498.99443741,26.12029505)(499.01444092,26.10030273)
\curveto(499.11443729,26.03029514)(499.14943726,25.94029523)(499.11944092,25.83030273)
\curveto(499.08943732,25.73029544)(499.07943733,25.61529555)(499.08944092,25.48530273)
\curveto(499.09943731,25.42529574)(499.12943728,25.37529579)(499.17944092,25.33530273)
\curveto(499.29943711,25.32529584)(499.404437,25.3702958)(499.49444092,25.47030273)
\curveto(499.59443681,25.5702956)(499.68943672,25.65029552)(499.77944092,25.71030273)
\curveto(499.93943647,25.81029536)(500.09943631,25.90029527)(500.25944092,25.98030273)
\curveto(500.41943599,26.0702951)(500.6044358,26.14529502)(500.81444092,26.20530273)
\curveto(500.89443551,26.23529493)(500.98443542,26.25529491)(501.08444092,26.26530273)
\curveto(501.18443522,26.27529489)(501.27943513,26.29029488)(501.36944092,26.31030273)
\curveto(501.41943499,26.32029485)(501.46943494,26.32529484)(501.51944092,26.32530273)
\lineto(501.66944092,26.32530273)
}
}
{
\newrgbcolor{curcolor}{0 0 0}
\pscustom[linestyle=none,fillstyle=solid,fillcolor=curcolor]
{
\newpath
\moveto(506.88405029,27.67530273)
\curveto(506.81404732,27.73529343)(506.79404734,27.84029333)(506.82405029,27.99030273)
\curveto(506.85404728,28.15029302)(506.88404725,28.30529286)(506.91405029,28.45530273)
\curveto(506.92404721,28.53529263)(506.93904719,28.62029255)(506.95905029,28.71030273)
\curveto(506.97904715,28.80029237)(507.00904712,28.87529229)(507.04905029,28.93530273)
\curveto(507.10904702,29.01529215)(507.19904693,29.07529209)(507.31905029,29.11530273)
\curveto(507.34904678,29.12529204)(507.37404676,29.12529204)(507.39405029,29.11530273)
\curveto(507.41404672,29.11529205)(507.43904669,29.12029205)(507.46905029,29.13030273)
\curveto(507.63904649,29.13029204)(507.79404634,29.12529204)(507.93405029,29.11530273)
\curveto(508.08404605,29.10529206)(508.17404596,29.04529212)(508.20405029,28.93530273)
\curveto(508.22404591,28.87529229)(508.22404591,28.80029237)(508.20405029,28.71030273)
\curveto(508.18404595,28.63029254)(508.16904596,28.54529262)(508.15905029,28.45530273)
\curveto(508.11904601,28.27529289)(508.07904605,28.10529306)(508.03905029,27.94530273)
\curveto(508.00904612,27.78529338)(507.92404621,27.68029349)(507.78405029,27.63030273)
\curveto(507.72404641,27.61029356)(507.66404647,27.60029357)(507.60405029,27.60030273)
\lineto(507.43905029,27.60030273)
\lineto(507.12405029,27.60030273)
\curveto(507.02404711,27.60029357)(506.94404719,27.62529354)(506.88405029,27.67530273)
\moveto(506.29905029,19.17030273)
\curveto(506.27904785,19.0703021)(506.25904787,18.9653022)(506.23905029,18.85530273)
\curveto(506.2290479,18.75530241)(506.18904794,18.67530249)(506.11905029,18.61530273)
\curveto(506.07904805,18.55530261)(506.0290481,18.51530265)(505.96905029,18.49530273)
\curveto(505.90904822,18.48530268)(505.8340483,18.4703027)(505.74405029,18.45030273)
\lineto(505.51905029,18.45030273)
\curveto(505.38904874,18.45030272)(505.27904885,18.45530271)(505.18905029,18.46530273)
\curveto(505.09904903,18.48530268)(505.0340491,18.53530263)(504.99405029,18.61530273)
\curveto(504.97404916,18.67530249)(504.96904916,18.75030242)(504.97905029,18.84030273)
\curveto(504.99904913,18.94030223)(505.01904911,19.03530213)(505.03905029,19.12530273)
\lineto(506.31405029,25.47030273)
\curveto(506.3340478,25.58029559)(506.35404778,25.68529548)(506.37405029,25.78530273)
\curveto(506.39404774,25.89529527)(506.4340477,25.98029519)(506.49405029,26.04030273)
\curveto(506.5340476,26.09029508)(506.57904755,26.12029505)(506.62905029,26.13030273)
\curveto(506.68904744,26.14029503)(506.74904738,26.15529501)(506.80905029,26.17530273)
\curveto(506.8290473,26.17529499)(506.84904728,26.170295)(506.86905029,26.16030273)
\curveto(506.89904723,26.16029501)(506.92404721,26.165295)(506.94405029,26.17530273)
\curveto(507.07404706,26.17529499)(507.20404693,26.170295)(507.33405029,26.16030273)
\curveto(507.47404666,26.16029501)(507.55904657,26.12029505)(507.58905029,26.04030273)
\curveto(507.6290465,25.98029519)(507.63904649,25.90029527)(507.61905029,25.80030273)
\curveto(507.59904653,25.71029546)(507.57904655,25.61529555)(507.55905029,25.51530273)
\lineto(506.29905029,19.17030273)
}
}
{
\newrgbcolor{curcolor}{0 0 0}
\pscustom[linestyle=none,fillstyle=solid,fillcolor=curcolor]
{
\newpath
\moveto(515.21889404,19.26030273)
\lineto(515.12889404,18.87030273)
\curveto(515.10888611,18.75030242)(515.06888615,18.65030252)(515.00889404,18.57030273)
\curveto(514.93888628,18.50030267)(514.84388638,18.46030271)(514.72389404,18.45030273)
\lineto(514.37889404,18.45030273)
\curveto(514.3188869,18.45030272)(514.25888696,18.44530272)(514.19889404,18.43530273)
\curveto(514.14888707,18.43530273)(514.10388712,18.44530272)(514.06389404,18.46530273)
\curveto(513.98388724,18.48530268)(513.93388729,18.52530264)(513.91389404,18.58530273)
\curveto(513.88388734,18.63530253)(513.87388735,18.69530247)(513.88389404,18.76530273)
\curveto(513.89388733,18.83530233)(513.88888733,18.90530226)(513.86889404,18.97530273)
\curveto(513.86888735,18.99530217)(513.85888736,19.01030216)(513.83889404,19.02030273)
\lineto(513.80889404,19.08030273)
\curveto(513.70888751,19.09030208)(513.6238876,19.0703021)(513.55389404,19.02030273)
\curveto(513.49388773,18.9703022)(513.42888779,18.92030225)(513.35889404,18.87030273)
\curveto(513.12888809,18.72030245)(512.90388832,18.60530256)(512.68389404,18.52530273)
\curveto(512.49388873,18.44530272)(512.27388895,18.38530278)(512.02389404,18.34530273)
\curveto(511.78388944,18.30530286)(511.53888968,18.28530288)(511.28889404,18.28530273)
\curveto(511.04889017,18.27530289)(510.80889041,18.29030288)(510.56889404,18.33030273)
\curveto(510.33889088,18.36030281)(510.14389108,18.41530275)(509.98389404,18.49530273)
\curveto(509.50389172,18.71530245)(509.13889208,19.01030216)(508.88889404,19.38030273)
\curveto(508.64889257,19.76030141)(508.49389273,20.23030094)(508.42389404,20.79030273)
\curveto(508.40389282,20.88030029)(508.39389283,20.9703002)(508.39389404,21.06030273)
\curveto(508.40389282,21.16030001)(508.40389282,21.26029991)(508.39389404,21.36030273)
\curveto(508.39389283,21.41029976)(508.39889282,21.46029971)(508.40889404,21.51030273)
\curveto(508.4188928,21.56029961)(508.4238928,21.61029956)(508.42389404,21.66030273)
\curveto(508.41389281,21.71029946)(508.41389281,21.76029941)(508.42389404,21.81030273)
\curveto(508.44389278,21.8702993)(508.45389277,21.92529924)(508.45389404,21.97530273)
\lineto(508.48389404,22.12530273)
\curveto(508.47389275,22.17529899)(508.47389275,22.24029893)(508.48389404,22.32030273)
\curveto(508.50389272,22.40029877)(508.52889269,22.4652987)(508.55889404,22.51530273)
\lineto(508.60389404,22.68030273)
\curveto(508.63389259,22.75029842)(508.65389257,22.82029835)(508.66389404,22.89030273)
\curveto(508.67389255,22.9702982)(508.69389253,23.04529812)(508.72389404,23.11530273)
\curveto(508.74389248,23.165298)(508.75889246,23.21029796)(508.76889404,23.25030273)
\curveto(508.77889244,23.29029788)(508.79389243,23.33529783)(508.81389404,23.38530273)
\curveto(508.86389236,23.48529768)(508.90889231,23.58029759)(508.94889404,23.67030273)
\curveto(508.98889223,23.7702974)(509.03389219,23.8652973)(509.08389404,23.95530273)
\curveto(509.28389194,24.33529683)(509.51389171,24.67529649)(509.77389404,24.97530273)
\curveto(510.04389118,25.28529588)(510.34389088,25.54029563)(510.67389404,25.74030273)
\curveto(510.87389035,25.86029531)(511.07389015,25.96029521)(511.27389404,26.04030273)
\curveto(511.47388975,26.12029505)(511.68888953,26.19029498)(511.91889404,26.25030273)
\lineto(512.12889404,26.28030273)
\curveto(512.19888902,26.29029488)(512.26888895,26.30529486)(512.33889404,26.32530273)
\lineto(512.48889404,26.32530273)
\curveto(512.57888864,26.34529482)(512.69888852,26.35529481)(512.84889404,26.35530273)
\curveto(513.00888821,26.35529481)(513.1238881,26.34529482)(513.19389404,26.32530273)
\curveto(513.23388799,26.31529485)(513.28888793,26.31029486)(513.35889404,26.31030273)
\curveto(513.45888776,26.28029489)(513.56388766,26.25529491)(513.67389404,26.23530273)
\curveto(513.78388744,26.22529494)(513.88388734,26.19529497)(513.97389404,26.14530273)
\curveto(514.11388711,26.08529508)(514.24388698,26.02029515)(514.36389404,25.95030273)
\curveto(514.48388674,25.88029529)(514.59388663,25.80029537)(514.69389404,25.71030273)
\curveto(514.74388648,25.66029551)(514.79388643,25.60529556)(514.84389404,25.54530273)
\curveto(514.90388632,25.49529567)(514.98888623,25.48029569)(515.09889404,25.50030273)
\lineto(515.17389404,25.57530273)
\curveto(515.19388603,25.59529557)(515.20888601,25.62529554)(515.21889404,25.66530273)
\curveto(515.26888595,25.75529541)(515.30388592,25.8702953)(515.32389404,26.01030273)
\curveto(515.35388587,26.15029502)(515.37888584,26.27529489)(515.39889404,26.38530273)
\lineto(515.74389404,28.11030273)
\curveto(515.77388545,28.25029292)(515.80388542,28.40529276)(515.83389404,28.57530273)
\curveto(515.87388535,28.75529241)(515.9238853,28.88529228)(515.98389404,28.96530273)
\curveto(516.04388518,29.03529213)(516.11388511,29.08029209)(516.19389404,29.10030273)
\curveto(516.21388501,29.10029207)(516.23888498,29.10029207)(516.26889404,29.10030273)
\curveto(516.29888492,29.11029206)(516.3238849,29.11529205)(516.34389404,29.11530273)
\curveto(516.49388473,29.12529204)(516.64388458,29.12529204)(516.79389404,29.11530273)
\curveto(516.94388428,29.11529205)(517.04388418,29.07529209)(517.09389404,28.99530273)
\curveto(517.1238841,28.91529225)(517.1238841,28.81529235)(517.09389404,28.69530273)
\curveto(517.07388415,28.57529259)(517.05388417,28.45029272)(517.03389404,28.32030273)
\lineto(515.21889404,19.26030273)
\moveto(514.57389404,22.09530273)
\curveto(514.60388662,22.14529902)(514.6238866,22.21029896)(514.63389404,22.29030273)
\curveto(514.65388657,22.38029879)(514.65888656,22.45029872)(514.64889404,22.50030273)
\lineto(514.69389404,22.72530273)
\curveto(514.69388653,22.81529835)(514.69888652,22.90529826)(514.70889404,22.99530273)
\curveto(514.7188865,23.09529807)(514.71388651,23.18529798)(514.69389404,23.26530273)
\lineto(514.69389404,23.49030273)
\curveto(514.69388653,23.56029761)(514.68388654,23.63029754)(514.66389404,23.70030273)
\curveto(514.60388662,24.00029717)(514.49888672,24.2652969)(514.34889404,24.49530273)
\curveto(514.20888701,24.72529644)(514.00888721,24.90529626)(513.74889404,25.03530273)
\curveto(513.65888756,25.08529608)(513.56388766,25.12029605)(513.46389404,25.14030273)
\curveto(513.36388786,25.170296)(513.25388797,25.19529597)(513.13389404,25.21530273)
\curveto(513.06388816,25.23529593)(512.97888824,25.24529592)(512.87889404,25.24530273)
\lineto(512.60889404,25.24530273)
\lineto(512.45889404,25.21530273)
\lineto(512.32389404,25.21530273)
\curveto(512.24388898,25.19529597)(512.15888906,25.17529599)(512.06889404,25.15530273)
\curveto(511.97888924,25.13529603)(511.89388933,25.11029606)(511.81389404,25.08030273)
\curveto(511.46388976,24.94029623)(511.16389006,24.73529643)(510.91389404,24.46530273)
\curveto(510.66389056,24.20529696)(510.44389078,23.90029727)(510.25389404,23.55030273)
\curveto(510.19389103,23.44029773)(510.14389108,23.32529784)(510.10389404,23.20530273)
\lineto(509.98389404,22.87530273)
\lineto(509.95389404,22.75530273)
\curveto(509.94389128,22.72529844)(509.93389129,22.69029848)(509.92389404,22.65030273)
\curveto(509.89389133,22.60029857)(509.87389135,22.54529862)(509.86389404,22.48530273)
\curveto(509.86389136,22.42529874)(509.85889136,22.3702988)(509.84889404,22.32030273)
\curveto(509.82889139,22.21029896)(509.80389142,22.10029907)(509.77389404,21.99030273)
\curveto(509.75389147,21.89029928)(509.74889147,21.79529937)(509.75889404,21.70530273)
\curveto(509.75889146,21.67529949)(509.75389147,21.62529954)(509.74389404,21.55530273)
\lineto(509.74389404,21.34530273)
\curveto(509.74389148,21.27529989)(509.74889147,21.20529996)(509.75889404,21.13530273)
\curveto(509.79889142,20.78530038)(509.88889133,20.48530068)(510.02889404,20.23530273)
\curveto(510.16889105,19.98530118)(510.36889085,19.78030139)(510.62889404,19.62030273)
\curveto(510.70889051,19.5703016)(510.78889043,19.53030164)(510.86889404,19.50030273)
\curveto(510.95889026,19.4703017)(511.05389017,19.44030173)(511.15389404,19.41030273)
\curveto(511.20389002,19.39030178)(511.25388997,19.38530178)(511.30389404,19.39530273)
\curveto(511.36388986,19.40530176)(511.4188898,19.40030177)(511.46889404,19.38030273)
\curveto(511.49888972,19.3703018)(511.53388969,19.3653018)(511.57389404,19.36530273)
\lineto(511.70889404,19.36530273)
\lineto(511.84389404,19.36530273)
\curveto(511.88388934,19.37530179)(511.93888928,19.38030179)(512.00889404,19.38030273)
\curveto(512.08888913,19.40030177)(512.16888905,19.41530175)(512.24889404,19.42530273)
\curveto(512.33888888,19.44530172)(512.4188888,19.4703017)(512.48889404,19.50030273)
\curveto(512.84888837,19.64030153)(513.15388807,19.81530135)(513.40389404,20.02530273)
\curveto(513.65388757,20.24530092)(513.87888734,20.52030065)(514.07889404,20.85030273)
\curveto(514.14888707,20.96030021)(514.20388702,21.0703001)(514.24389404,21.18030273)
\lineto(514.39389404,21.51030273)
\curveto(514.4238868,21.55029962)(514.43888678,21.58529958)(514.43889404,21.61530273)
\curveto(514.44888677,21.65529951)(514.46388676,21.69529947)(514.48389404,21.73530273)
\curveto(514.50388672,21.79529937)(514.5188867,21.85529931)(514.52889404,21.91530273)
\curveto(514.53888668,21.97529919)(514.55388667,22.03529913)(514.57389404,22.09530273)
}
}
{
\newrgbcolor{curcolor}{0 0 0}
\pscustom[linestyle=none,fillstyle=solid,fillcolor=curcolor]
{
\newpath
\moveto(524.96514404,22.65030273)
\curveto(524.97513515,22.59029858)(524.96513516,22.49529867)(524.93514404,22.36530273)
\curveto(524.91513521,22.24529892)(524.89513523,22.16029901)(524.87514404,22.11030273)
\lineto(524.84514404,21.96030273)
\curveto(524.81513531,21.88029929)(524.79013534,21.80529936)(524.77014404,21.73530273)
\curveto(524.76013537,21.67529949)(524.74013539,21.60529956)(524.71014404,21.52530273)
\curveto(524.68013545,21.4652997)(524.65513547,21.40529976)(524.63514404,21.34530273)
\curveto(524.6251355,21.28529988)(524.60013553,21.22529994)(524.56014404,21.16530273)
\lineto(524.38014404,20.77530273)
\curveto(524.3301358,20.64530052)(524.26513586,20.52530064)(524.18514404,20.41530273)
\curveto(523.88513624,19.93530123)(523.5251366,19.53030164)(523.10514404,19.20030273)
\curveto(522.69513743,18.88030229)(522.21513791,18.63530253)(521.66514404,18.46530273)
\curveto(521.55513857,18.42530274)(521.43513869,18.39530277)(521.30514404,18.37530273)
\curveto(521.17513895,18.35530281)(521.04013909,18.33530283)(520.90014404,18.31530273)
\curveto(520.84013929,18.30530286)(520.77513935,18.30030287)(520.70514404,18.30030273)
\curveto(520.64513948,18.29030288)(520.58513954,18.28530288)(520.52514404,18.28530273)
\curveto(520.48513964,18.27530289)(520.4251397,18.2703029)(520.34514404,18.27030273)
\curveto(520.27513985,18.2703029)(520.2251399,18.27530289)(520.19514404,18.28530273)
\curveto(520.15513997,18.29530287)(520.11514001,18.30030287)(520.07514404,18.30030273)
\curveto(520.03514009,18.29030288)(520.00014013,18.29030288)(519.97014404,18.30030273)
\lineto(519.88014404,18.30030273)
\lineto(519.53514404,18.34530273)
\lineto(519.14514404,18.46530273)
\curveto(519.0251411,18.50530266)(518.91014122,18.55030262)(518.80014404,18.60030273)
\curveto(518.39014174,18.80030237)(518.07014206,19.06030211)(517.84014404,19.38030273)
\curveto(517.62014251,19.70030147)(517.46014267,20.09030108)(517.36014404,20.55030273)
\curveto(517.3301428,20.65030052)(517.31014282,20.75030042)(517.30014404,20.85030273)
\lineto(517.30014404,21.16530273)
\curveto(517.29014284,21.20529996)(517.29014284,21.23529993)(517.30014404,21.25530273)
\curveto(517.31014282,21.28529988)(517.31514281,21.32029985)(517.31514404,21.36030273)
\curveto(517.31514281,21.44029973)(517.32014281,21.52029965)(517.33014404,21.60030273)
\curveto(517.34014279,21.69029948)(517.34514278,21.77529939)(517.34514404,21.85530273)
\curveto(517.35514277,21.90529926)(517.36014277,21.94529922)(517.36014404,21.97530273)
\curveto(517.37014276,22.01529915)(517.37514275,22.06029911)(517.37514404,22.11030273)
\curveto(517.37514275,22.16029901)(517.38514274,22.24529892)(517.40514404,22.36530273)
\curveto(517.43514269,22.49529867)(517.46514266,22.59029858)(517.49514404,22.65030273)
\curveto(517.53514259,22.72029845)(517.55514257,22.79029838)(517.55514404,22.86030273)
\curveto(517.55514257,22.93029824)(517.57514255,23.00029817)(517.61514404,23.07030273)
\curveto(517.63514249,23.12029805)(517.65014248,23.16029801)(517.66014404,23.19030273)
\curveto(517.67014246,23.23029794)(517.68514244,23.27529789)(517.70514404,23.32530273)
\curveto(517.76514236,23.44529772)(517.81514231,23.5652976)(517.85514404,23.68530273)
\curveto(517.90514222,23.80529736)(517.97014216,23.92029725)(518.05014404,24.03030273)
\curveto(518.27014186,24.40029677)(518.51514161,24.73029644)(518.78514404,25.02030273)
\curveto(519.06514106,25.32029585)(519.38014075,25.5702956)(519.73014404,25.77030273)
\curveto(519.86014027,25.85029532)(519.99514013,25.91529525)(520.13514404,25.96530273)
\lineto(520.58514404,26.14530273)
\curveto(520.71513941,26.19529497)(520.85013928,26.22529494)(520.99014404,26.23530273)
\curveto(521.130139,26.25529491)(521.27513885,26.28529488)(521.42514404,26.32530273)
\lineto(521.62014404,26.32530273)
\lineto(521.83014404,26.35530273)
\curveto(522.72013741,26.3652948)(523.42013671,26.18029499)(523.93014404,25.80030273)
\curveto(524.45013568,25.42029575)(524.77513535,24.92529624)(524.90514404,24.31530273)
\curveto(524.93513519,24.21529695)(524.95513517,24.11529705)(524.96514404,24.01530273)
\curveto(524.97513515,23.91529725)(524.99013514,23.81029736)(525.01014404,23.70030273)
\curveto(525.02013511,23.59029758)(525.02013511,23.4702977)(525.01014404,23.34030273)
\lineto(525.01014404,22.96530273)
\curveto(525.01013512,22.91529825)(525.00013513,22.86029831)(524.98014404,22.80030273)
\curveto(524.97013516,22.75029842)(524.96513516,22.70029847)(524.96514404,22.65030273)
\moveto(523.46514404,21.79530273)
\curveto(523.49513663,21.8652993)(523.51513661,21.94529922)(523.52514404,22.03530273)
\curveto(523.54513658,22.12529904)(523.56013657,22.21029896)(523.57014404,22.29030273)
\curveto(523.65013648,22.68029849)(523.68513644,23.01029816)(523.67514404,23.28030273)
\curveto(523.65513647,23.36029781)(523.64013649,23.44029773)(523.63014404,23.52030273)
\curveto(523.6301365,23.60029757)(523.6251365,23.67529749)(523.61514404,23.74530273)
\curveto(523.46513666,24.39529677)(523.11013702,24.84529632)(522.55014404,25.09530273)
\curveto(522.48013765,25.12529604)(522.40513772,25.14529602)(522.32514404,25.15530273)
\curveto(522.25513787,25.17529599)(522.18013795,25.19529597)(522.10014404,25.21530273)
\curveto(522.0301381,25.23529593)(521.95013818,25.24529592)(521.86014404,25.24530273)
\lineto(521.59014404,25.24530273)
\lineto(521.30514404,25.20030273)
\curveto(521.20513892,25.18029599)(521.11013902,25.15529601)(521.02014404,25.12530273)
\curveto(520.9301392,25.10529606)(520.84013929,25.07529609)(520.75014404,25.03530273)
\curveto(520.68013945,25.01529615)(520.61013952,24.98529618)(520.54014404,24.94530273)
\curveto(520.47013966,24.90529626)(520.40513972,24.8652963)(520.34514404,24.82530273)
\curveto(520.07514005,24.65529651)(519.84014029,24.45029672)(519.64014404,24.21030273)
\curveto(519.44014069,23.9702972)(519.25514087,23.69029748)(519.08514404,23.37030273)
\curveto(519.03514109,23.2702979)(518.99514113,23.165298)(518.96514404,23.05530273)
\curveto(518.93514119,22.95529821)(518.89514123,22.85029832)(518.84514404,22.74030273)
\curveto(518.83514129,22.70029847)(518.82014131,22.63529853)(518.80014404,22.54530273)
\curveto(518.78014135,22.51529865)(518.77014136,22.48029869)(518.77014404,22.44030273)
\curveto(518.77014136,22.40029877)(518.76514136,22.35529881)(518.75514404,22.30530273)
\lineto(518.69514404,22.00530273)
\curveto(518.67514145,21.90529926)(518.66514146,21.81529935)(518.66514404,21.73530273)
\lineto(518.66514404,21.55530273)
\curveto(518.66514146,21.45529971)(518.66014147,21.35529981)(518.65014404,21.25530273)
\curveto(518.65014148,21.1653)(518.66014147,21.08030009)(518.68014404,21.00030273)
\curveto(518.7301414,20.76030041)(518.80014133,20.53530063)(518.89014404,20.32530273)
\curveto(518.99014114,20.11530105)(519.125141,19.94030123)(519.29514404,19.80030273)
\curveto(519.34514078,19.7703014)(519.38514074,19.74530142)(519.41514404,19.72530273)
\curveto(519.45514067,19.70530146)(519.49514063,19.68030149)(519.53514404,19.65030273)
\curveto(519.60514052,19.60030157)(519.68514044,19.55530161)(519.77514404,19.51530273)
\curveto(519.86514026,19.48530168)(519.96014017,19.45530171)(520.06014404,19.42530273)
\curveto(520.11014002,19.40530176)(520.15513997,19.39530177)(520.19514404,19.39530273)
\curveto(520.24513988,19.40530176)(520.29513983,19.40530176)(520.34514404,19.39530273)
\curveto(520.37513975,19.38530178)(520.43513969,19.37530179)(520.52514404,19.36530273)
\curveto(520.61513951,19.35530181)(520.69013944,19.36030181)(520.75014404,19.38030273)
\curveto(520.79013934,19.39030178)(520.8301393,19.39030178)(520.87014404,19.38030273)
\curveto(520.91013922,19.38030179)(520.95013918,19.39030178)(520.99014404,19.41030273)
\curveto(521.07013906,19.43030174)(521.15013898,19.44530172)(521.23014404,19.45530273)
\curveto(521.32013881,19.47530169)(521.40513872,19.50030167)(521.48514404,19.53030273)
\curveto(521.84513828,19.6703015)(522.15513797,19.8653013)(522.41514404,20.11530273)
\curveto(522.67513745,20.3653008)(522.91013722,20.66030051)(523.12014404,21.00030273)
\curveto(523.20013693,21.12030005)(523.26013687,21.24529992)(523.30014404,21.37530273)
\curveto(523.34013679,21.51529965)(523.39513673,21.65529951)(523.46514404,21.79530273)
}
}
{
\newrgbcolor{curcolor}{0 0 0}
\pscustom[linestyle=none,fillstyle=solid,fillcolor=curcolor]
{
\newpath
\moveto(529.63342529,26.35530273)
\curveto(530.35341964,26.3652948)(530.93841905,26.28029489)(531.38842529,26.10030273)
\curveto(531.84841814,25.93029524)(532.16841782,25.62529554)(532.34842529,25.18530273)
\curveto(532.39841759,25.07529609)(532.42841756,24.96029621)(532.43842529,24.84030273)
\curveto(532.45841753,24.73029644)(532.47341752,24.60529656)(532.48342529,24.46530273)
\curveto(532.4934175,24.39529677)(532.48341751,24.32029685)(532.45342529,24.24030273)
\curveto(532.43341756,24.170297)(532.40841758,24.11529705)(532.37842529,24.07530273)
\curveto(532.35841763,24.05529711)(532.32841766,24.03529713)(532.28842529,24.01530273)
\curveto(532.25841773,24.00529716)(532.23341776,23.99029718)(532.21342529,23.97030273)
\curveto(532.15341784,23.95029722)(532.09841789,23.94529722)(532.04842529,23.95530273)
\curveto(532.00841798,23.9652972)(531.96341803,23.9652972)(531.91342529,23.95530273)
\curveto(531.82341817,23.93529723)(531.71341828,23.93029724)(531.58342529,23.94030273)
\curveto(531.46341853,23.96029721)(531.37841861,23.98529718)(531.32842529,24.01530273)
\curveto(531.25841873,24.0652971)(531.21841877,24.13029704)(531.20842529,24.21030273)
\curveto(531.20841878,24.30029687)(531.1884188,24.38529678)(531.14842529,24.46530273)
\curveto(531.09841889,24.62529654)(531.00341899,24.7702964)(530.86342529,24.90030273)
\curveto(530.77341922,24.98029619)(530.66341933,25.04029613)(530.53342529,25.08030273)
\curveto(530.41341958,25.12029605)(530.28341971,25.16029601)(530.14342529,25.20030273)
\curveto(530.10341989,25.22029595)(530.05341994,25.22529594)(529.99342529,25.21530273)
\curveto(529.94342005,25.21529595)(529.89842009,25.22029595)(529.85842529,25.23030273)
\curveto(529.79842019,25.25029592)(529.72342027,25.26029591)(529.63342529,25.26030273)
\curveto(529.54342045,25.26029591)(529.46842052,25.25029592)(529.40842529,25.23030273)
\lineto(529.31842529,25.23030273)
\curveto(529.25842073,25.22029595)(529.20342079,25.21029596)(529.15342529,25.20030273)
\curveto(529.10342089,25.20029597)(529.05342094,25.19529597)(529.00342529,25.18530273)
\curveto(528.73342126,25.12529604)(528.49842149,25.04029613)(528.29842529,24.93030273)
\curveto(528.10842188,24.82029635)(527.95842203,24.63529653)(527.84842529,24.37530273)
\curveto(527.81842217,24.30529686)(527.80342219,24.23529693)(527.80342529,24.16530273)
\curveto(527.80342219,24.09529707)(527.80842218,24.03529713)(527.81842529,23.98530273)
\curveto(527.84842214,23.83529733)(527.89842209,23.72529744)(527.96842529,23.65530273)
\curveto(528.03842195,23.59529757)(528.13342186,23.52529764)(528.25342529,23.44530273)
\curveto(528.3934216,23.34529782)(528.55842143,23.2702979)(528.74842529,23.22030273)
\curveto(528.93842105,23.18029799)(529.12842086,23.13029804)(529.31842529,23.07030273)
\curveto(529.43842055,23.03029814)(529.55842043,23.00029817)(529.67842529,22.98030273)
\curveto(529.80842018,22.96029821)(529.93342006,22.93029824)(530.05342529,22.89030273)
\curveto(530.25341974,22.83029834)(530.44841954,22.7702984)(530.63842529,22.71030273)
\curveto(530.82841916,22.66029851)(531.01341898,22.59529857)(531.19342529,22.51530273)
\curveto(531.24341875,22.49529867)(531.2884187,22.47529869)(531.32842529,22.45530273)
\curveto(531.37841861,22.43529873)(531.42841856,22.41029876)(531.47842529,22.38030273)
\curveto(531.64841834,22.26029891)(531.7934182,22.12529904)(531.91342529,21.97530273)
\curveto(532.03341796,21.82529934)(532.12341787,21.63529953)(532.18342529,21.40530273)
\lineto(532.18342529,21.12030273)
\curveto(532.18341781,21.05030012)(532.17841781,20.97530019)(532.16842529,20.89530273)
\curveto(532.15841783,20.82530034)(532.14841784,20.74530042)(532.13842529,20.65530273)
\lineto(532.10842529,20.50530273)
\curveto(532.06841792,20.43530073)(532.03841795,20.3653008)(532.01842529,20.29530273)
\curveto(532.00841798,20.22530094)(531.988418,20.15530101)(531.95842529,20.08530273)
\curveto(531.90841808,19.97530119)(531.85341814,19.8703013)(531.79342529,19.77030273)
\curveto(531.73341826,19.6703015)(531.66841832,19.58030159)(531.59842529,19.50030273)
\curveto(531.3884186,19.24030193)(531.14341885,19.03030214)(530.86342529,18.87030273)
\curveto(530.58341941,18.72030245)(530.27841971,18.59030258)(529.94842529,18.48030273)
\curveto(529.84842014,18.45030272)(529.74842024,18.43030274)(529.64842529,18.42030273)
\curveto(529.54842044,18.40030277)(529.45342054,18.37530279)(529.36342529,18.34530273)
\curveto(529.25342074,18.32530284)(529.14842084,18.31530285)(529.04842529,18.31530273)
\curveto(528.94842104,18.31530285)(528.84842114,18.30530286)(528.74842529,18.28530273)
\lineto(528.59842529,18.28530273)
\curveto(528.54842144,18.27530289)(528.47842151,18.2703029)(528.38842529,18.27030273)
\curveto(528.29842169,18.2703029)(528.22842176,18.27530289)(528.17842529,18.28530273)
\lineto(528.01342529,18.28530273)
\curveto(527.95342204,18.30530286)(527.8884221,18.31530285)(527.81842529,18.31530273)
\curveto(527.74842224,18.30530286)(527.6934223,18.31030286)(527.65342529,18.33030273)
\curveto(527.60342239,18.34030283)(527.53842245,18.34530282)(527.45842529,18.34530273)
\curveto(527.37842261,18.3653028)(527.30342269,18.38530278)(527.23342529,18.40530273)
\curveto(527.16342283,18.41530275)(527.0884229,18.43530273)(527.00842529,18.46530273)
\curveto(526.71842327,18.5653026)(526.47342352,18.69030248)(526.27342529,18.84030273)
\curveto(526.07342392,18.99030218)(525.91342408,19.18530198)(525.79342529,19.42530273)
\curveto(525.73342426,19.55530161)(525.68342431,19.69030148)(525.64342529,19.83030273)
\curveto(525.61342438,19.9703012)(525.5934244,20.12530104)(525.58342529,20.29530273)
\curveto(525.57342442,20.35530081)(525.57842441,20.42530074)(525.59842529,20.50530273)
\curveto(525.61842437,20.59530057)(525.64342435,20.6653005)(525.67342529,20.71530273)
\curveto(525.71342428,20.75530041)(525.77342422,20.79530037)(525.85342529,20.83530273)
\curveto(525.90342409,20.85530031)(525.97342402,20.8653003)(526.06342529,20.86530273)
\curveto(526.16342383,20.87530029)(526.25342374,20.87530029)(526.33342529,20.86530273)
\curveto(526.42342357,20.85530031)(526.50842348,20.84030033)(526.58842529,20.82030273)
\curveto(526.67842331,20.81030036)(526.73342326,20.79530037)(526.75342529,20.77530273)
\curveto(526.81342318,20.72530044)(526.84342315,20.65030052)(526.84342529,20.55030273)
\curveto(526.85342314,20.46030071)(526.87342312,20.37530079)(526.90342529,20.29530273)
\curveto(526.95342304,20.07530109)(527.05342294,19.90530126)(527.20342529,19.78530273)
\curveto(527.30342269,19.69530147)(527.42342257,19.62530154)(527.56342529,19.57530273)
\curveto(527.70342229,19.52530164)(527.85342214,19.47530169)(528.01342529,19.42530273)
\lineto(528.32842529,19.38030273)
\lineto(528.41842529,19.38030273)
\curveto(528.47842151,19.36030181)(528.56342143,19.35030182)(528.67342529,19.35030273)
\curveto(528.7934212,19.35030182)(528.89842109,19.36030181)(528.98842529,19.38030273)
\curveto(529.05842093,19.38030179)(529.11342088,19.38530178)(529.15342529,19.39530273)
\curveto(529.21342078,19.40530176)(529.27342072,19.41030176)(529.33342529,19.41030273)
\curveto(529.3934206,19.42030175)(529.44842054,19.43030174)(529.49842529,19.44030273)
\curveto(529.80842018,19.52030165)(530.05841993,19.62530154)(530.24842529,19.75530273)
\curveto(530.44841954,19.88530128)(530.61341938,20.10530106)(530.74342529,20.41530273)
\curveto(530.77341922,20.4653007)(530.7884192,20.52030065)(530.78842529,20.58030273)
\curveto(530.79841919,20.64030053)(530.79841919,20.68530048)(530.78842529,20.71530273)
\curveto(530.77841921,20.90530026)(530.73841925,21.04530012)(530.66842529,21.13530273)
\curveto(530.59841939,21.23529993)(530.50341949,21.32529984)(530.38342529,21.40530273)
\curveto(530.30341969,21.4652997)(530.20841978,21.51529965)(530.09842529,21.55530273)
\lineto(529.79842529,21.67530273)
\curveto(529.76842022,21.68529948)(529.73842025,21.69029948)(529.70842529,21.69030273)
\curveto(529.6884203,21.69029948)(529.66842032,21.70029947)(529.64842529,21.72030273)
\curveto(529.32842066,21.83029934)(528.988421,21.91029926)(528.62842529,21.96030273)
\curveto(528.27842171,22.02029915)(527.95842203,22.11529905)(527.66842529,22.24530273)
\curveto(527.57842241,22.28529888)(527.4884225,22.32029885)(527.39842529,22.35030273)
\curveto(527.31842267,22.38029879)(527.24342275,22.42029875)(527.17342529,22.47030273)
\curveto(527.00342299,22.58029859)(526.85342314,22.70529846)(526.72342529,22.84530273)
\curveto(526.5934234,22.98529818)(526.50342349,23.16029801)(526.45342529,23.37030273)
\curveto(526.43342356,23.44029773)(526.42342357,23.51029766)(526.42342529,23.58030273)
\lineto(526.42342529,23.80530273)
\curveto(526.41342358,23.92529724)(526.42842356,24.06029711)(526.46842529,24.21030273)
\curveto(526.50842348,24.3702968)(526.54842344,24.50529666)(526.58842529,24.61530273)
\curveto(526.61842337,24.6652965)(526.63842335,24.70529646)(526.64842529,24.73530273)
\curveto(526.66842332,24.77529639)(526.6934233,24.81529635)(526.72342529,24.85530273)
\curveto(526.85342314,25.08529608)(527.01342298,25.28529588)(527.20342529,25.45530273)
\curveto(527.3934226,25.62529554)(527.60342239,25.77529539)(527.83342529,25.90530273)
\curveto(527.993422,25.99529517)(528.16842182,26.0652951)(528.35842529,26.11530273)
\curveto(528.55842143,26.17529499)(528.76342123,26.23029494)(528.97342529,26.28030273)
\curveto(529.04342095,26.29029488)(529.10842088,26.30029487)(529.16842529,26.31030273)
\curveto(529.23842075,26.32029485)(529.31342068,26.33029484)(529.39342529,26.34030273)
\curveto(529.43342056,26.35029482)(529.47342052,26.35029482)(529.51342529,26.34030273)
\curveto(529.56342043,26.33029484)(529.60342039,26.33529483)(529.63342529,26.35530273)
}
}
{
\newrgbcolor{curcolor}{0 0 0}
\pscustom[linewidth=1,linecolor=curcolor]
{
\newpath
\moveto(145.93638,88.52252)
\lineto(783.91628,88.52252)
}
}
{
\newrgbcolor{curcolor}{0 0 0}
\pscustom[linewidth=1,linecolor=curcolor]
{
\newpath
\moveto(159.01786,162.51623)
\lineto(796.99776,162.51623)
}
}
{
\newrgbcolor{curcolor}{0 0 0}
\pscustom[linewidth=1,linecolor=curcolor]
{
\newpath
\moveto(159.01786,236.55822)
\lineto(796.99776,236.55822)
}
}
{
\newrgbcolor{curcolor}{0 0 0}
\pscustom[linewidth=1,linecolor=curcolor]
{
\newpath
\moveto(159.01786,311.62335)
\lineto(796.99776,311.62335)
}
}
{
\newrgbcolor{curcolor}{0 0 0}
\pscustom[linewidth=1,linecolor=curcolor]
{
\newpath
\moveto(159.01786,385.589017)
\lineto(796.99776,385.589017)
}
}
{
\newrgbcolor{curcolor}{0 0 0}
\pscustom[linestyle=none,fillstyle=solid,fillcolor=curcolor]
{
\newpath
\moveto(160.44285278,421.81210297)
\lineto(161.73285278,421.81210297)
\curveto(161.84284996,421.81209229)(161.94784985,421.80709229)(162.04785278,421.79710297)
\curveto(162.14784965,421.7970923)(162.22284958,421.76209234)(162.27285278,421.69210297)
\curveto(162.32284948,421.62209248)(162.34784945,421.53209257)(162.34785278,421.42210297)
\curveto(162.35784944,421.31209279)(162.36284944,421.19209291)(162.36285278,421.06210297)
\lineto(162.36285278,419.75710297)
\lineto(162.36285278,414.55210297)
\lineto(162.36285278,412.09210297)
\lineto(162.36285278,411.65710297)
\curveto(162.37284943,411.4971026)(162.35284945,411.37710272)(162.30285278,411.29710297)
\curveto(162.26284954,411.22710287)(162.17284963,411.17210293)(162.03285278,411.13210297)
\curveto(161.96284984,411.11210299)(161.88784991,411.10710299)(161.80785278,411.11710297)
\curveto(161.72785007,411.12710297)(161.64785015,411.13210297)(161.56785278,411.13210297)
\lineto(160.68285278,411.13210297)
\curveto(160.57285123,411.13210297)(160.46785133,411.13710296)(160.36785278,411.14710297)
\curveto(160.27785152,411.15710294)(160.2028516,411.18710291)(160.14285278,411.23710297)
\curveto(160.09285171,411.28710281)(160.06285174,411.36210274)(160.05285278,411.46210297)
\curveto(160.04285176,411.56210254)(160.03785176,411.66710243)(160.03785278,411.77710297)
\lineto(160.03785278,413.08210297)
\lineto(160.03785278,418.55710297)
\lineto(160.03785278,420.74710297)
\curveto(160.03785176,420.88709321)(160.03285177,421.05209305)(160.02285278,421.24210297)
\curveto(160.02285178,421.43209267)(160.04785175,421.56709253)(160.09785278,421.64710297)
\curveto(160.13785166,421.70709239)(160.2028516,421.75709234)(160.29285278,421.79710297)
\curveto(160.32285148,421.7970923)(160.34785145,421.7970923)(160.36785278,421.79710297)
\curveto(160.3978514,421.80709229)(160.42285138,421.81209229)(160.44285278,421.81210297)
}
}
{
\newrgbcolor{curcolor}{0 0 0}
\pscustom[linestyle=none,fillstyle=solid,fillcolor=curcolor]
{
\newpath
\moveto(168.64668091,419.06710297)
\curveto(169.2466751,419.08709501)(169.7466746,419.0020951)(170.14668091,418.81210297)
\curveto(170.5466738,418.62209548)(170.86167349,418.34209576)(171.09168091,417.97210297)
\curveto(171.16167319,417.86209624)(171.21667313,417.74209636)(171.25668091,417.61210297)
\curveto(171.29667305,417.49209661)(171.33667301,417.36709673)(171.37668091,417.23710297)
\curveto(171.39667295,417.15709694)(171.40667294,417.08209702)(171.40668091,417.01210297)
\curveto(171.41667293,416.94209716)(171.43167292,416.87209723)(171.45168091,416.80210297)
\curveto(171.4516729,416.74209736)(171.45667289,416.7020974)(171.46668091,416.68210297)
\curveto(171.48667286,416.54209756)(171.49667285,416.3970977)(171.49668091,416.24710297)
\lineto(171.49668091,415.81210297)
\lineto(171.49668091,414.47710297)
\lineto(171.49668091,412.04710297)
\curveto(171.49667285,411.85710224)(171.49167286,411.67210243)(171.48168091,411.49210297)
\curveto(171.48167287,411.32210278)(171.41167294,411.21210289)(171.27168091,411.16210297)
\curveto(171.21167314,411.14210296)(171.14167321,411.13210297)(171.06168091,411.13210297)
\lineto(170.82168091,411.13210297)
\lineto(170.01168091,411.13210297)
\curveto(169.89167446,411.13210297)(169.78167457,411.13710296)(169.68168091,411.14710297)
\curveto(169.59167476,411.16710293)(169.52167483,411.21210289)(169.47168091,411.28210297)
\curveto(169.43167492,411.34210276)(169.40667494,411.41710268)(169.39668091,411.50710297)
\lineto(169.39668091,411.82210297)
\lineto(169.39668091,412.87210297)
\lineto(169.39668091,415.10710297)
\curveto(169.39667495,415.47709862)(169.38167497,415.81709828)(169.35168091,416.12710297)
\curveto(169.32167503,416.44709765)(169.23167512,416.71709738)(169.08168091,416.93710297)
\curveto(168.94167541,417.13709696)(168.73667561,417.27709682)(168.46668091,417.35710297)
\curveto(168.41667593,417.37709672)(168.36167599,417.38709671)(168.30168091,417.38710297)
\curveto(168.2516761,417.38709671)(168.19667615,417.3970967)(168.13668091,417.41710297)
\curveto(168.08667626,417.42709667)(168.02167633,417.42709667)(167.94168091,417.41710297)
\curveto(167.87167648,417.41709668)(167.81667653,417.41209669)(167.77668091,417.40210297)
\curveto(167.73667661,417.39209671)(167.70167665,417.38709671)(167.67168091,417.38710297)
\curveto(167.64167671,417.38709671)(167.61167674,417.38209672)(167.58168091,417.37210297)
\curveto(167.351677,417.31209679)(167.16667718,417.23209687)(167.02668091,417.13210297)
\curveto(166.70667764,416.9020972)(166.51667783,416.56709753)(166.45668091,416.12710297)
\curveto(166.39667795,415.68709841)(166.36667798,415.19209891)(166.36668091,414.64210297)
\lineto(166.36668091,412.76710297)
\lineto(166.36668091,411.85210297)
\lineto(166.36668091,411.58210297)
\curveto(166.36667798,411.49210261)(166.351678,411.41710268)(166.32168091,411.35710297)
\curveto(166.27167808,411.24710285)(166.19167816,411.18210292)(166.08168091,411.16210297)
\curveto(165.97167838,411.14210296)(165.83667851,411.13210297)(165.67668091,411.13210297)
\lineto(164.92668091,411.13210297)
\curveto(164.81667953,411.13210297)(164.70667964,411.13710296)(164.59668091,411.14710297)
\curveto(164.48667986,411.15710294)(164.40667994,411.19210291)(164.35668091,411.25210297)
\curveto(164.28668006,411.34210276)(164.2516801,411.47210263)(164.25168091,411.64210297)
\curveto(164.26168009,411.81210229)(164.26668008,411.97210213)(164.26668091,412.12210297)
\lineto(164.26668091,414.16210297)
\lineto(164.26668091,417.46210297)
\lineto(164.26668091,418.22710297)
\lineto(164.26668091,418.52710297)
\curveto(164.27668007,418.61709548)(164.30668004,418.69209541)(164.35668091,418.75210297)
\curveto(164.37667997,418.78209532)(164.40667994,418.8020953)(164.44668091,418.81210297)
\curveto(164.49667985,418.83209527)(164.5466798,418.84709525)(164.59668091,418.85710297)
\lineto(164.67168091,418.85710297)
\curveto(164.72167963,418.86709523)(164.77167958,418.87209523)(164.82168091,418.87210297)
\lineto(164.98668091,418.87210297)
\lineto(165.61668091,418.87210297)
\curveto(165.69667865,418.87209523)(165.77167858,418.86709523)(165.84168091,418.85710297)
\curveto(165.92167843,418.85709524)(165.99167836,418.84709525)(166.05168091,418.82710297)
\curveto(166.12167823,418.7970953)(166.16667818,418.75209535)(166.18668091,418.69210297)
\curveto(166.21667813,418.63209547)(166.24167811,418.56209554)(166.26168091,418.48210297)
\curveto(166.27167808,418.44209566)(166.27167808,418.40709569)(166.26168091,418.37710297)
\curveto(166.26167809,418.34709575)(166.27167808,418.31709578)(166.29168091,418.28710297)
\curveto(166.31167804,418.23709586)(166.32667802,418.20709589)(166.33668091,418.19710297)
\curveto(166.35667799,418.18709591)(166.38167797,418.17209593)(166.41168091,418.15210297)
\curveto(166.52167783,418.14209596)(166.61167774,418.17709592)(166.68168091,418.25710297)
\curveto(166.7516776,418.34709575)(166.82667752,418.41709568)(166.90668091,418.46710297)
\curveto(167.17667717,418.66709543)(167.47667687,418.82709527)(167.80668091,418.94710297)
\curveto(167.89667645,418.97709512)(167.98667636,418.9970951)(168.07668091,419.00710297)
\curveto(168.17667617,419.01709508)(168.28167607,419.03209507)(168.39168091,419.05210297)
\curveto(168.42167593,419.06209504)(168.46667588,419.06209504)(168.52668091,419.05210297)
\curveto(168.58667576,419.05209505)(168.62667572,419.05709504)(168.64668091,419.06710297)
}
}
{
\newrgbcolor{curcolor}{0 0 0}
\pscustom[linestyle=none,fillstyle=solid,fillcolor=curcolor]
{
\newpath
\moveto(180.71793091,411.98710297)
\lineto(180.71793091,411.56710297)
\curveto(180.71792254,411.43710266)(180.68792257,411.33210277)(180.62793091,411.25210297)
\curveto(180.57792268,411.2021029)(180.51292274,411.16710293)(180.43293091,411.14710297)
\curveto(180.3529229,411.13710296)(180.26292299,411.13210297)(180.16293091,411.13210297)
\lineto(179.33793091,411.13210297)
\lineto(179.05293091,411.13210297)
\curveto(178.97292428,411.14210296)(178.90792435,411.16710293)(178.85793091,411.20710297)
\curveto(178.78792447,411.25710284)(178.74792451,411.32210278)(178.73793091,411.40210297)
\curveto(178.72792453,411.48210262)(178.70792455,411.56210254)(178.67793091,411.64210297)
\curveto(178.6579246,411.66210244)(178.63792462,411.67710242)(178.61793091,411.68710297)
\curveto(178.60792465,411.70710239)(178.59292466,411.72710237)(178.57293091,411.74710297)
\curveto(178.46292479,411.74710235)(178.38292487,411.72210238)(178.33293091,411.67210297)
\lineto(178.18293091,411.52210297)
\curveto(178.11292514,411.47210263)(178.04792521,411.42710267)(177.98793091,411.38710297)
\curveto(177.92792533,411.35710274)(177.86292539,411.31710278)(177.79293091,411.26710297)
\curveto(177.7529255,411.24710285)(177.70792555,411.22710287)(177.65793091,411.20710297)
\curveto(177.61792564,411.18710291)(177.57292568,411.16710293)(177.52293091,411.14710297)
\curveto(177.38292587,411.097103)(177.23292602,411.05210305)(177.07293091,411.01210297)
\curveto(177.02292623,410.99210311)(176.97792628,410.98210312)(176.93793091,410.98210297)
\curveto(176.89792636,410.98210312)(176.8579264,410.97710312)(176.81793091,410.96710297)
\lineto(176.68293091,410.96710297)
\curveto(176.6529266,410.95710314)(176.61292664,410.95210315)(176.56293091,410.95210297)
\lineto(176.42793091,410.95210297)
\curveto(176.36792689,410.93210317)(176.27792698,410.92710317)(176.15793091,410.93710297)
\curveto(176.03792722,410.93710316)(175.9529273,410.94710315)(175.90293091,410.96710297)
\curveto(175.83292742,410.98710311)(175.76792749,410.9971031)(175.70793091,410.99710297)
\curveto(175.6579276,410.98710311)(175.60292765,410.99210311)(175.54293091,411.01210297)
\lineto(175.18293091,411.13210297)
\curveto(175.07292818,411.16210294)(174.96292829,411.2021029)(174.85293091,411.25210297)
\curveto(174.50292875,411.4021027)(174.18792907,411.63210247)(173.90793091,411.94210297)
\curveto(173.63792962,412.26210184)(173.42292983,412.5971015)(173.26293091,412.94710297)
\curveto(173.21293004,413.05710104)(173.17293008,413.16210094)(173.14293091,413.26210297)
\curveto(173.11293014,413.37210073)(173.07793018,413.48210062)(173.03793091,413.59210297)
\curveto(173.02793023,413.63210047)(173.02293023,413.66710043)(173.02293091,413.69710297)
\curveto(173.02293023,413.73710036)(173.01293024,413.78210032)(172.99293091,413.83210297)
\curveto(172.97293028,413.91210019)(172.9529303,413.9971001)(172.93293091,414.08710297)
\curveto(172.92293033,414.18709991)(172.90793035,414.28709981)(172.88793091,414.38710297)
\curveto(172.87793038,414.41709968)(172.87293038,414.45209965)(172.87293091,414.49210297)
\curveto(172.88293037,414.53209957)(172.88293037,414.56709953)(172.87293091,414.59710297)
\lineto(172.87293091,414.73210297)
\curveto(172.87293038,414.78209932)(172.86793039,414.83209927)(172.85793091,414.88210297)
\curveto(172.84793041,414.93209917)(172.84293041,414.98709911)(172.84293091,415.04710297)
\curveto(172.84293041,415.11709898)(172.84793041,415.17209893)(172.85793091,415.21210297)
\curveto(172.86793039,415.26209884)(172.87293038,415.30709879)(172.87293091,415.34710297)
\lineto(172.87293091,415.49710297)
\curveto(172.88293037,415.54709855)(172.88293037,415.59209851)(172.87293091,415.63210297)
\curveto(172.87293038,415.68209842)(172.88293037,415.73209837)(172.90293091,415.78210297)
\curveto(172.92293033,415.89209821)(172.93793032,415.9970981)(172.94793091,416.09710297)
\curveto(172.96793029,416.1970979)(172.99293026,416.2970978)(173.02293091,416.39710297)
\curveto(173.06293019,416.51709758)(173.09793016,416.63209747)(173.12793091,416.74210297)
\curveto(173.1579301,416.85209725)(173.19793006,416.96209714)(173.24793091,417.07210297)
\curveto(173.38792987,417.37209673)(173.56292969,417.65709644)(173.77293091,417.92710297)
\curveto(173.79292946,417.95709614)(173.81792944,417.98209612)(173.84793091,418.00210297)
\curveto(173.88792937,418.03209607)(173.91792934,418.06209604)(173.93793091,418.09210297)
\curveto(173.97792928,418.14209596)(174.01792924,418.18709591)(174.05793091,418.22710297)
\curveto(174.09792916,418.26709583)(174.14292911,418.30709579)(174.19293091,418.34710297)
\curveto(174.23292902,418.36709573)(174.26792899,418.39209571)(174.29793091,418.42210297)
\curveto(174.32792893,418.46209564)(174.36292889,418.49209561)(174.40293091,418.51210297)
\curveto(174.6529286,418.68209542)(174.94292831,418.82209528)(175.27293091,418.93210297)
\curveto(175.34292791,418.95209515)(175.41292784,418.96709513)(175.48293091,418.97710297)
\curveto(175.56292769,418.98709511)(175.64292761,419.0020951)(175.72293091,419.02210297)
\curveto(175.79292746,419.04209506)(175.88292737,419.05209505)(175.99293091,419.05210297)
\curveto(176.10292715,419.06209504)(176.21292704,419.06709503)(176.32293091,419.06710297)
\curveto(176.43292682,419.06709503)(176.53792672,419.06209504)(176.63793091,419.05210297)
\curveto(176.74792651,419.04209506)(176.83792642,419.02709507)(176.90793091,419.00710297)
\curveto(177.0579262,418.95709514)(177.20292605,418.91209519)(177.34293091,418.87210297)
\curveto(177.48292577,418.83209527)(177.61292564,418.77709532)(177.73293091,418.70710297)
\curveto(177.80292545,418.65709544)(177.86792539,418.60709549)(177.92793091,418.55710297)
\curveto(177.98792527,418.51709558)(178.0529252,418.47209563)(178.12293091,418.42210297)
\curveto(178.16292509,418.39209571)(178.21792504,418.35209575)(178.28793091,418.30210297)
\curveto(178.36792489,418.25209585)(178.44292481,418.25209585)(178.51293091,418.30210297)
\curveto(178.5529247,418.32209578)(178.57292468,418.35709574)(178.57293091,418.40710297)
\curveto(178.57292468,418.45709564)(178.58292467,418.50709559)(178.60293091,418.55710297)
\lineto(178.60293091,418.70710297)
\curveto(178.61292464,418.73709536)(178.61792464,418.77209533)(178.61793091,418.81210297)
\lineto(178.61793091,418.93210297)
\lineto(178.61793091,420.97210297)
\curveto(178.61792464,421.08209302)(178.61292464,421.2020929)(178.60293091,421.33210297)
\curveto(178.60292465,421.47209263)(178.62792463,421.57709252)(178.67793091,421.64710297)
\curveto(178.71792454,421.72709237)(178.79292446,421.77709232)(178.90293091,421.79710297)
\curveto(178.92292433,421.80709229)(178.94292431,421.80709229)(178.96293091,421.79710297)
\curveto(178.98292427,421.7970923)(179.00292425,421.8020923)(179.02293091,421.81210297)
\lineto(180.08793091,421.81210297)
\curveto(180.20792305,421.81209229)(180.31792294,421.80709229)(180.41793091,421.79710297)
\curveto(180.51792274,421.78709231)(180.59292266,421.74709235)(180.64293091,421.67710297)
\curveto(180.69292256,421.5970925)(180.71792254,421.49209261)(180.71793091,421.36210297)
\lineto(180.71793091,421.00210297)
\lineto(180.71793091,411.98710297)
\moveto(178.67793091,414.92710297)
\curveto(178.68792457,414.96709913)(178.68792457,415.00709909)(178.67793091,415.04710297)
\lineto(178.67793091,415.18210297)
\curveto(178.67792458,415.28209882)(178.67292458,415.38209872)(178.66293091,415.48210297)
\curveto(178.6529246,415.58209852)(178.63792462,415.67209843)(178.61793091,415.75210297)
\curveto(178.59792466,415.86209824)(178.57792468,415.96209814)(178.55793091,416.05210297)
\curveto(178.54792471,416.14209796)(178.52292473,416.22709787)(178.48293091,416.30710297)
\curveto(178.34292491,416.66709743)(178.13792512,416.95209715)(177.86793091,417.16210297)
\curveto(177.60792565,417.37209673)(177.22792603,417.47709662)(176.72793091,417.47710297)
\curveto(176.66792659,417.47709662)(176.58792667,417.46709663)(176.48793091,417.44710297)
\curveto(176.40792685,417.42709667)(176.33292692,417.40709669)(176.26293091,417.38710297)
\curveto(176.20292705,417.37709672)(176.14292711,417.35709674)(176.08293091,417.32710297)
\curveto(175.81292744,417.21709688)(175.60292765,417.04709705)(175.45293091,416.81710297)
\curveto(175.30292795,416.58709751)(175.18292807,416.32709777)(175.09293091,416.03710297)
\curveto(175.06292819,415.93709816)(175.04292821,415.83709826)(175.03293091,415.73710297)
\curveto(175.02292823,415.63709846)(175.00292825,415.53209857)(174.97293091,415.42210297)
\lineto(174.97293091,415.21210297)
\curveto(174.9529283,415.12209898)(174.94792831,414.9970991)(174.95793091,414.83710297)
\curveto(174.96792829,414.68709941)(174.98292827,414.57709952)(175.00293091,414.50710297)
\lineto(175.00293091,414.41710297)
\curveto(175.01292824,414.3970997)(175.01792824,414.37709972)(175.01793091,414.35710297)
\curveto(175.03792822,414.27709982)(175.0529282,414.2020999)(175.06293091,414.13210297)
\curveto(175.08292817,414.06210004)(175.10292815,413.98710011)(175.12293091,413.90710297)
\curveto(175.29292796,413.38710071)(175.58292767,413.0021011)(175.99293091,412.75210297)
\curveto(176.12292713,412.66210144)(176.30292695,412.59210151)(176.53293091,412.54210297)
\curveto(176.57292668,412.53210157)(176.63292662,412.52710157)(176.71293091,412.52710297)
\curveto(176.74292651,412.51710158)(176.78792647,412.50710159)(176.84793091,412.49710297)
\curveto(176.91792634,412.4971016)(176.97292628,412.5021016)(177.01293091,412.51210297)
\curveto(177.09292616,412.53210157)(177.17292608,412.54710155)(177.25293091,412.55710297)
\curveto(177.33292592,412.56710153)(177.41292584,412.58710151)(177.49293091,412.61710297)
\curveto(177.74292551,412.72710137)(177.94292531,412.86710123)(178.09293091,413.03710297)
\curveto(178.24292501,413.20710089)(178.37292488,413.42210068)(178.48293091,413.68210297)
\curveto(178.52292473,413.77210033)(178.5529247,413.86210024)(178.57293091,413.95210297)
\curveto(178.59292466,414.05210005)(178.61292464,414.15709994)(178.63293091,414.26710297)
\curveto(178.64292461,414.31709978)(178.64292461,414.36209974)(178.63293091,414.40210297)
\curveto(178.63292462,414.45209965)(178.64292461,414.5020996)(178.66293091,414.55210297)
\curveto(178.67292458,414.58209952)(178.67792458,414.61709948)(178.67793091,414.65710297)
\lineto(178.67793091,414.79210297)
\lineto(178.67793091,414.92710297)
}
}
{
\newrgbcolor{curcolor}{0 0 0}
\pscustom[linestyle=none,fillstyle=solid,fillcolor=curcolor]
{
\newpath
\moveto(184.39785278,421.72210297)
\curveto(184.46784983,421.64209246)(184.5028498,421.52209258)(184.50285278,421.36210297)
\lineto(184.50285278,420.89710297)
\lineto(184.50285278,420.49210297)
\curveto(184.5028498,420.35209375)(184.46784983,420.25709384)(184.39785278,420.20710297)
\curveto(184.33784996,420.15709394)(184.25785004,420.12709397)(184.15785278,420.11710297)
\curveto(184.06785023,420.10709399)(183.96785033,420.102094)(183.85785278,420.10210297)
\lineto(183.01785278,420.10210297)
\curveto(182.90785139,420.102094)(182.80785149,420.10709399)(182.71785278,420.11710297)
\curveto(182.63785166,420.12709397)(182.56785173,420.15709394)(182.50785278,420.20710297)
\curveto(182.46785183,420.23709386)(182.43785186,420.29209381)(182.41785278,420.37210297)
\curveto(182.40785189,420.46209364)(182.3978519,420.55709354)(182.38785278,420.65710297)
\lineto(182.38785278,420.98710297)
\curveto(182.3978519,421.097093)(182.4028519,421.19209291)(182.40285278,421.27210297)
\lineto(182.40285278,421.48210297)
\curveto(182.41285189,421.55209255)(182.43285187,421.61209249)(182.46285278,421.66210297)
\curveto(182.48285182,421.7020924)(182.50785179,421.73209237)(182.53785278,421.75210297)
\lineto(182.65785278,421.81210297)
\curveto(182.67785162,421.81209229)(182.7028516,421.81209229)(182.73285278,421.81210297)
\curveto(182.76285154,421.82209228)(182.78785151,421.82709227)(182.80785278,421.82710297)
\lineto(183.90285278,421.82710297)
\curveto(184.0028503,421.82709227)(184.0978502,421.82209228)(184.18785278,421.81210297)
\curveto(184.27785002,421.8020923)(184.34784995,421.77209233)(184.39785278,421.72210297)
\moveto(184.50285278,411.95710297)
\curveto(184.5028498,411.75710234)(184.4978498,411.58710251)(184.48785278,411.44710297)
\curveto(184.47784982,411.30710279)(184.38784991,411.21210289)(184.21785278,411.16210297)
\curveto(184.15785014,411.14210296)(184.09285021,411.13210297)(184.02285278,411.13210297)
\curveto(183.95285035,411.14210296)(183.87785042,411.14710295)(183.79785278,411.14710297)
\lineto(182.95785278,411.14710297)
\curveto(182.86785143,411.14710295)(182.77785152,411.15210295)(182.68785278,411.16210297)
\curveto(182.60785169,411.17210293)(182.54785175,411.2021029)(182.50785278,411.25210297)
\curveto(182.44785185,411.32210278)(182.41285189,411.40710269)(182.40285278,411.50710297)
\lineto(182.40285278,411.85210297)
\lineto(182.40285278,418.18210297)
\lineto(182.40285278,418.48210297)
\curveto(182.4028519,418.58209552)(182.42285188,418.66209544)(182.46285278,418.72210297)
\curveto(182.52285178,418.79209531)(182.60785169,418.83709526)(182.71785278,418.85710297)
\curveto(182.73785156,418.86709523)(182.76285154,418.86709523)(182.79285278,418.85710297)
\curveto(182.83285147,418.85709524)(182.86285144,418.86209524)(182.88285278,418.87210297)
\lineto(183.63285278,418.87210297)
\lineto(183.82785278,418.87210297)
\curveto(183.90785039,418.88209522)(183.97285033,418.88209522)(184.02285278,418.87210297)
\lineto(184.14285278,418.87210297)
\curveto(184.2028501,418.85209525)(184.25785004,418.83709526)(184.30785278,418.82710297)
\curveto(184.35784994,418.81709528)(184.3978499,418.78709531)(184.42785278,418.73710297)
\curveto(184.46784983,418.68709541)(184.48784981,418.61709548)(184.48785278,418.52710297)
\curveto(184.4978498,418.43709566)(184.5028498,418.34209576)(184.50285278,418.24210297)
\lineto(184.50285278,411.95710297)
}
}
{
\newrgbcolor{curcolor}{0 0 0}
\pscustom[linestyle=none,fillstyle=solid,fillcolor=curcolor]
{
\newpath
\moveto(189.73504028,419.08210297)
\curveto(190.54503512,419.102095)(191.22003445,418.98209512)(191.76004028,418.72210297)
\curveto(192.31003336,418.46209564)(192.74503292,418.09209601)(193.06504028,417.61210297)
\curveto(193.22503244,417.37209673)(193.34503232,417.097097)(193.42504028,416.78710297)
\curveto(193.44503222,416.73709736)(193.46003221,416.67209743)(193.47004028,416.59210297)
\curveto(193.49003218,416.51209759)(193.49003218,416.44209766)(193.47004028,416.38210297)
\curveto(193.43003224,416.27209783)(193.36003231,416.20709789)(193.26004028,416.18710297)
\curveto(193.16003251,416.17709792)(193.04003263,416.17209793)(192.90004028,416.17210297)
\lineto(192.12004028,416.17210297)
\lineto(191.83504028,416.17210297)
\curveto(191.74503392,416.17209793)(191.670034,416.19209791)(191.61004028,416.23210297)
\curveto(191.53003414,416.27209783)(191.47503419,416.33209777)(191.44504028,416.41210297)
\curveto(191.41503425,416.5020976)(191.37503429,416.59209751)(191.32504028,416.68210297)
\curveto(191.2650344,416.79209731)(191.20003447,416.89209721)(191.13004028,416.98210297)
\curveto(191.06003461,417.07209703)(190.98003469,417.15209695)(190.89004028,417.22210297)
\curveto(190.75003492,417.31209679)(190.59503507,417.38209672)(190.42504028,417.43210297)
\curveto(190.3650353,417.45209665)(190.30503536,417.46209664)(190.24504028,417.46210297)
\curveto(190.18503548,417.46209664)(190.13003554,417.47209663)(190.08004028,417.49210297)
\lineto(189.93004028,417.49210297)
\curveto(189.73003594,417.49209661)(189.5700361,417.47209663)(189.45004028,417.43210297)
\curveto(189.16003651,417.34209676)(188.92503674,417.2020969)(188.74504028,417.01210297)
\curveto(188.5650371,416.83209727)(188.42003725,416.61209749)(188.31004028,416.35210297)
\curveto(188.26003741,416.24209786)(188.22003745,416.12209798)(188.19004028,415.99210297)
\curveto(188.1700375,415.87209823)(188.14503752,415.74209836)(188.11504028,415.60210297)
\curveto(188.10503756,415.56209854)(188.10003757,415.52209858)(188.10004028,415.48210297)
\curveto(188.10003757,415.44209866)(188.09503757,415.4020987)(188.08504028,415.36210297)
\curveto(188.0650376,415.26209884)(188.05503761,415.12209898)(188.05504028,414.94210297)
\curveto(188.0650376,414.76209934)(188.08003759,414.62209948)(188.10004028,414.52210297)
\curveto(188.10003757,414.44209966)(188.10503756,414.38709971)(188.11504028,414.35710297)
\curveto(188.13503753,414.28709981)(188.14503752,414.21709988)(188.14504028,414.14710297)
\curveto(188.15503751,414.07710002)(188.1700375,414.00710009)(188.19004028,413.93710297)
\curveto(188.2700374,413.70710039)(188.3650373,413.4971006)(188.47504028,413.30710297)
\curveto(188.58503708,413.11710098)(188.72503694,412.95710114)(188.89504028,412.82710297)
\curveto(188.93503673,412.7971013)(188.99503667,412.76210134)(189.07504028,412.72210297)
\curveto(189.18503648,412.65210145)(189.29503637,412.60710149)(189.40504028,412.58710297)
\curveto(189.52503614,412.56710153)(189.670036,412.54710155)(189.84004028,412.52710297)
\lineto(189.93004028,412.52710297)
\curveto(189.9700357,412.52710157)(190.00003567,412.53210157)(190.02004028,412.54210297)
\lineto(190.15504028,412.54210297)
\curveto(190.22503544,412.56210154)(190.29003538,412.57710152)(190.35004028,412.58710297)
\curveto(190.42003525,412.60710149)(190.48503518,412.62710147)(190.54504028,412.64710297)
\curveto(190.84503482,412.77710132)(191.07503459,412.96710113)(191.23504028,413.21710297)
\curveto(191.27503439,413.26710083)(191.31003436,413.32210078)(191.34004028,413.38210297)
\curveto(191.3700343,413.45210065)(191.39503427,413.51210059)(191.41504028,413.56210297)
\curveto(191.45503421,413.67210043)(191.49003418,413.76710033)(191.52004028,413.84710297)
\curveto(191.55003412,413.93710016)(191.62003405,414.00710009)(191.73004028,414.05710297)
\curveto(191.82003385,414.0971)(191.9650337,414.11209999)(192.16504028,414.10210297)
\lineto(192.66004028,414.10210297)
\lineto(192.87004028,414.10210297)
\curveto(192.95003272,414.11209999)(193.01503265,414.10709999)(193.06504028,414.08710297)
\lineto(193.18504028,414.08710297)
\lineto(193.30504028,414.05710297)
\curveto(193.34503232,414.05710004)(193.37503229,414.04710005)(193.39504028,414.02710297)
\curveto(193.44503222,413.98710011)(193.47503219,413.92710017)(193.48504028,413.84710297)
\curveto(193.50503216,413.77710032)(193.50503216,413.7021004)(193.48504028,413.62210297)
\curveto(193.39503227,413.29210081)(193.28503238,412.9971011)(193.15504028,412.73710297)
\curveto(192.74503292,411.96710213)(192.09003358,411.43210267)(191.19004028,411.13210297)
\curveto(191.09003458,411.102103)(190.98503468,411.08210302)(190.87504028,411.07210297)
\curveto(190.7650349,411.05210305)(190.65503501,411.02710307)(190.54504028,410.99710297)
\curveto(190.48503518,410.98710311)(190.42503524,410.98210312)(190.36504028,410.98210297)
\curveto(190.30503536,410.98210312)(190.24503542,410.97710312)(190.18504028,410.96710297)
\lineto(190.02004028,410.96710297)
\curveto(189.9700357,410.94710315)(189.89503577,410.94210316)(189.79504028,410.95210297)
\curveto(189.69503597,410.95210315)(189.62003605,410.95710314)(189.57004028,410.96710297)
\curveto(189.49003618,410.98710311)(189.41503625,410.9971031)(189.34504028,410.99710297)
\curveto(189.28503638,410.98710311)(189.22003645,410.99210311)(189.15004028,411.01210297)
\lineto(189.00004028,411.04210297)
\curveto(188.95003672,411.04210306)(188.90003677,411.04710305)(188.85004028,411.05710297)
\curveto(188.74003693,411.08710301)(188.63503703,411.11710298)(188.53504028,411.14710297)
\curveto(188.43503723,411.17710292)(188.34003733,411.21210289)(188.25004028,411.25210297)
\curveto(187.78003789,411.45210265)(187.38503828,411.70710239)(187.06504028,412.01710297)
\curveto(186.74503892,412.33710176)(186.48503918,412.73210137)(186.28504028,413.20210297)
\curveto(186.23503943,413.29210081)(186.19503947,413.38710071)(186.16504028,413.48710297)
\lineto(186.07504028,413.81710297)
\curveto(186.0650396,413.85710024)(186.06003961,413.89210021)(186.06004028,413.92210297)
\curveto(186.06003961,413.96210014)(186.05003962,414.00710009)(186.03004028,414.05710297)
\curveto(186.01003966,414.12709997)(186.00003967,414.1970999)(186.00004028,414.26710297)
\curveto(186.00003967,414.34709975)(185.99003968,414.42209968)(185.97004028,414.49210297)
\lineto(185.97004028,414.74710297)
\curveto(185.95003972,414.7970993)(185.94003973,414.85209925)(185.94004028,414.91210297)
\curveto(185.94003973,414.98209912)(185.95003972,415.04209906)(185.97004028,415.09210297)
\curveto(185.98003969,415.14209896)(185.98003969,415.18709891)(185.97004028,415.22710297)
\curveto(185.96003971,415.26709883)(185.96003971,415.30709879)(185.97004028,415.34710297)
\curveto(185.99003968,415.41709868)(185.99503967,415.48209862)(185.98504028,415.54210297)
\curveto(185.98503968,415.6020985)(185.99503967,415.66209844)(186.01504028,415.72210297)
\curveto(186.0650396,415.9020982)(186.10503956,416.07209803)(186.13504028,416.23210297)
\curveto(186.1650395,416.4020977)(186.21003946,416.56709753)(186.27004028,416.72710297)
\curveto(186.49003918,417.23709686)(186.7650389,417.66209644)(187.09504028,418.00210297)
\curveto(187.43503823,418.34209576)(187.8650378,418.61709548)(188.38504028,418.82710297)
\curveto(188.52503714,418.88709521)(188.670037,418.92709517)(188.82004028,418.94710297)
\curveto(188.9700367,418.97709512)(189.12503654,419.01209509)(189.28504028,419.05210297)
\curveto(189.3650363,419.06209504)(189.44003623,419.06709503)(189.51004028,419.06710297)
\curveto(189.58003609,419.06709503)(189.65503601,419.07209503)(189.73504028,419.08210297)
}
}
{
\newrgbcolor{curcolor}{0 0 0}
\pscustom[linestyle=none,fillstyle=solid,fillcolor=curcolor]
{
\newpath
\moveto(201.82832153,411.73210297)
\curveto(201.84831368,411.62210248)(201.85831367,411.51210259)(201.85832153,411.40210297)
\curveto(201.86831366,411.29210281)(201.81831371,411.21710288)(201.70832153,411.17710297)
\curveto(201.64831388,411.14710295)(201.57831395,411.13210297)(201.49832153,411.13210297)
\lineto(201.25832153,411.13210297)
\lineto(200.44832153,411.13210297)
\lineto(200.17832153,411.13210297)
\curveto(200.09831543,411.14210296)(200.0333155,411.16710293)(199.98332153,411.20710297)
\curveto(199.91331562,411.24710285)(199.85831567,411.3021028)(199.81832153,411.37210297)
\curveto(199.78831574,411.45210265)(199.74331579,411.51710258)(199.68332153,411.56710297)
\curveto(199.66331587,411.58710251)(199.63831589,411.6021025)(199.60832153,411.61210297)
\curveto(199.57831595,411.63210247)(199.53831599,411.63710246)(199.48832153,411.62710297)
\curveto(199.43831609,411.60710249)(199.38831614,411.58210252)(199.33832153,411.55210297)
\curveto(199.29831623,411.52210258)(199.25331628,411.4971026)(199.20332153,411.47710297)
\curveto(199.15331638,411.43710266)(199.09831643,411.4021027)(199.03832153,411.37210297)
\lineto(198.85832153,411.28210297)
\curveto(198.7283168,411.22210288)(198.59331694,411.17210293)(198.45332153,411.13210297)
\curveto(198.31331722,411.102103)(198.16831736,411.06710303)(198.01832153,411.02710297)
\curveto(197.94831758,411.00710309)(197.87831765,410.9971031)(197.80832153,410.99710297)
\curveto(197.74831778,410.98710311)(197.68331785,410.97710312)(197.61332153,410.96710297)
\lineto(197.52332153,410.96710297)
\curveto(197.49331804,410.95710314)(197.46331807,410.95210315)(197.43332153,410.95210297)
\lineto(197.26832153,410.95210297)
\curveto(197.16831836,410.93210317)(197.06831846,410.93210317)(196.96832153,410.95210297)
\lineto(196.83332153,410.95210297)
\curveto(196.76331877,410.97210313)(196.69331884,410.98210312)(196.62332153,410.98210297)
\curveto(196.56331897,410.97210313)(196.50331903,410.97710312)(196.44332153,410.99710297)
\curveto(196.34331919,411.01710308)(196.24831928,411.03710306)(196.15832153,411.05710297)
\curveto(196.06831946,411.06710303)(195.98331955,411.09210301)(195.90332153,411.13210297)
\curveto(195.61331992,411.24210286)(195.36332017,411.38210272)(195.15332153,411.55210297)
\curveto(194.95332058,411.73210237)(194.79332074,411.96710213)(194.67332153,412.25710297)
\curveto(194.64332089,412.32710177)(194.61332092,412.4021017)(194.58332153,412.48210297)
\curveto(194.56332097,412.56210154)(194.54332099,412.64710145)(194.52332153,412.73710297)
\curveto(194.50332103,412.78710131)(194.49332104,412.83710126)(194.49332153,412.88710297)
\curveto(194.50332103,412.93710116)(194.50332103,412.98710111)(194.49332153,413.03710297)
\curveto(194.48332105,413.06710103)(194.47332106,413.12710097)(194.46332153,413.21710297)
\curveto(194.46332107,413.31710078)(194.46832106,413.38710071)(194.47832153,413.42710297)
\curveto(194.49832103,413.52710057)(194.50832102,413.61210049)(194.50832153,413.68210297)
\lineto(194.59832153,414.01210297)
\curveto(194.6283209,414.13209997)(194.66832086,414.23709986)(194.71832153,414.32710297)
\curveto(194.88832064,414.61709948)(195.08332045,414.83709926)(195.30332153,414.98710297)
\curveto(195.52332001,415.13709896)(195.80331973,415.26709883)(196.14332153,415.37710297)
\curveto(196.27331926,415.42709867)(196.40831912,415.46209864)(196.54832153,415.48210297)
\curveto(196.68831884,415.5020986)(196.8283187,415.52709857)(196.96832153,415.55710297)
\curveto(197.04831848,415.57709852)(197.1333184,415.58709851)(197.22332153,415.58710297)
\curveto(197.31331822,415.5970985)(197.40331813,415.61209849)(197.49332153,415.63210297)
\curveto(197.56331797,415.65209845)(197.6333179,415.65709844)(197.70332153,415.64710297)
\curveto(197.77331776,415.64709845)(197.84831768,415.65709844)(197.92832153,415.67710297)
\curveto(197.99831753,415.6970984)(198.06831746,415.70709839)(198.13832153,415.70710297)
\curveto(198.20831732,415.70709839)(198.28331725,415.71709838)(198.36332153,415.73710297)
\curveto(198.57331696,415.78709831)(198.76331677,415.82709827)(198.93332153,415.85710297)
\curveto(199.11331642,415.8970982)(199.27331626,415.98709811)(199.41332153,416.12710297)
\curveto(199.50331603,416.21709788)(199.56331597,416.31709778)(199.59332153,416.42710297)
\curveto(199.60331593,416.45709764)(199.60331593,416.48209762)(199.59332153,416.50210297)
\curveto(199.59331594,416.52209758)(199.59831593,416.54209756)(199.60832153,416.56210297)
\curveto(199.61831591,416.58209752)(199.62331591,416.61209749)(199.62332153,416.65210297)
\lineto(199.62332153,416.74210297)
\lineto(199.59332153,416.86210297)
\curveto(199.59331594,416.9020972)(199.58831594,416.93709716)(199.57832153,416.96710297)
\curveto(199.47831605,417.26709683)(199.26831626,417.47209663)(198.94832153,417.58210297)
\curveto(198.85831667,417.61209649)(198.74831678,417.63209647)(198.61832153,417.64210297)
\curveto(198.49831703,417.66209644)(198.37331716,417.66709643)(198.24332153,417.65710297)
\curveto(198.11331742,417.65709644)(197.98831754,417.64709645)(197.86832153,417.62710297)
\curveto(197.74831778,417.60709649)(197.64331789,417.58209652)(197.55332153,417.55210297)
\curveto(197.49331804,417.53209657)(197.4333181,417.5020966)(197.37332153,417.46210297)
\curveto(197.32331821,417.43209667)(197.27331826,417.3970967)(197.22332153,417.35710297)
\curveto(197.17331836,417.31709678)(197.11831841,417.26209684)(197.05832153,417.19210297)
\curveto(197.00831852,417.12209698)(196.97331856,417.05709704)(196.95332153,416.99710297)
\curveto(196.90331863,416.8970972)(196.85831867,416.8020973)(196.81832153,416.71210297)
\curveto(196.78831874,416.62209748)(196.71831881,416.56209754)(196.60832153,416.53210297)
\curveto(196.528319,416.51209759)(196.44331909,416.5020976)(196.35332153,416.50210297)
\lineto(196.08332153,416.50210297)
\lineto(195.51332153,416.50210297)
\curveto(195.46332007,416.5020976)(195.41332012,416.4970976)(195.36332153,416.48710297)
\curveto(195.31332022,416.48709761)(195.26832026,416.49209761)(195.22832153,416.50210297)
\lineto(195.09332153,416.50210297)
\curveto(195.07332046,416.51209759)(195.04832048,416.51709758)(195.01832153,416.51710297)
\curveto(194.98832054,416.51709758)(194.96332057,416.52709757)(194.94332153,416.54710297)
\curveto(194.86332067,416.56709753)(194.80832072,416.63209747)(194.77832153,416.74210297)
\curveto(194.76832076,416.79209731)(194.76832076,416.84209726)(194.77832153,416.89210297)
\curveto(194.78832074,416.94209716)(194.79832073,416.98709711)(194.80832153,417.02710297)
\curveto(194.83832069,417.13709696)(194.86832066,417.23709686)(194.89832153,417.32710297)
\curveto(194.93832059,417.42709667)(194.98332055,417.51709658)(195.03332153,417.59710297)
\lineto(195.12332153,417.74710297)
\lineto(195.21332153,417.89710297)
\curveto(195.29332024,418.00709609)(195.39332014,418.11209599)(195.51332153,418.21210297)
\curveto(195.53332,418.22209588)(195.56331997,418.24709585)(195.60332153,418.28710297)
\curveto(195.65331988,418.32709577)(195.69831983,418.36209574)(195.73832153,418.39210297)
\curveto(195.77831975,418.42209568)(195.82331971,418.45209565)(195.87332153,418.48210297)
\curveto(196.04331949,418.59209551)(196.22331931,418.67709542)(196.41332153,418.73710297)
\curveto(196.60331893,418.80709529)(196.79831873,418.87209523)(196.99832153,418.93210297)
\curveto(197.11831841,418.96209514)(197.24331829,418.98209512)(197.37332153,418.99210297)
\curveto(197.50331803,419.0020951)(197.6333179,419.02209508)(197.76332153,419.05210297)
\curveto(197.80331773,419.06209504)(197.86331767,419.06209504)(197.94332153,419.05210297)
\curveto(198.0333175,419.04209506)(198.08831744,419.04709505)(198.10832153,419.06710297)
\curveto(198.51831701,419.07709502)(198.90831662,419.06209504)(199.27832153,419.02210297)
\curveto(199.65831587,418.98209512)(199.99831553,418.90709519)(200.29832153,418.79710297)
\curveto(200.60831492,418.68709541)(200.87331466,418.53709556)(201.09332153,418.34710297)
\curveto(201.31331422,418.16709593)(201.48331405,417.93209617)(201.60332153,417.64210297)
\curveto(201.67331386,417.47209663)(201.71331382,417.27709682)(201.72332153,417.05710297)
\curveto(201.7333138,416.83709726)(201.73831379,416.61209749)(201.73832153,416.38210297)
\lineto(201.73832153,413.03710297)
\lineto(201.73832153,412.45210297)
\curveto(201.73831379,412.26210184)(201.75831377,412.08710201)(201.79832153,411.92710297)
\curveto(201.80831372,411.8971022)(201.81331372,411.86210224)(201.81332153,411.82210297)
\curveto(201.81331372,411.79210231)(201.81831371,411.76210234)(201.82832153,411.73210297)
\moveto(199.62332153,414.04210297)
\curveto(199.6333159,414.09210001)(199.63831589,414.14709995)(199.63832153,414.20710297)
\curveto(199.63831589,414.27709982)(199.6333159,414.33709976)(199.62332153,414.38710297)
\curveto(199.60331593,414.44709965)(199.59331594,414.5020996)(199.59332153,414.55210297)
\curveto(199.59331594,414.6020995)(199.57331596,414.64209946)(199.53332153,414.67210297)
\curveto(199.48331605,414.71209939)(199.40831612,414.73209937)(199.30832153,414.73210297)
\curveto(199.26831626,414.72209938)(199.2333163,414.71209939)(199.20332153,414.70210297)
\curveto(199.17331636,414.7020994)(199.13831639,414.6970994)(199.09832153,414.68710297)
\curveto(199.0283165,414.66709943)(198.95331658,414.65209945)(198.87332153,414.64210297)
\curveto(198.79331674,414.63209947)(198.71331682,414.61709948)(198.63332153,414.59710297)
\curveto(198.60331693,414.58709951)(198.55831697,414.58209952)(198.49832153,414.58210297)
\curveto(198.36831716,414.55209955)(198.23831729,414.53209957)(198.10832153,414.52210297)
\curveto(197.97831755,414.51209959)(197.85331768,414.48709961)(197.73332153,414.44710297)
\curveto(197.65331788,414.42709967)(197.57831795,414.40709969)(197.50832153,414.38710297)
\curveto(197.43831809,414.37709972)(197.36831816,414.35709974)(197.29832153,414.32710297)
\curveto(197.08831844,414.23709986)(196.90831862,414.1021)(196.75832153,413.92210297)
\curveto(196.61831891,413.74210036)(196.56831896,413.49210061)(196.60832153,413.17210297)
\curveto(196.6283189,413.0021011)(196.68331885,412.86210124)(196.77332153,412.75210297)
\curveto(196.84331869,412.64210146)(196.94831858,412.55210155)(197.08832153,412.48210297)
\curveto(197.2283183,412.42210168)(197.37831815,412.37710172)(197.53832153,412.34710297)
\curveto(197.70831782,412.31710178)(197.88331765,412.30710179)(198.06332153,412.31710297)
\curveto(198.25331728,412.33710176)(198.4283171,412.37210173)(198.58832153,412.42210297)
\curveto(198.84831668,412.5021016)(199.05331648,412.62710147)(199.20332153,412.79710297)
\curveto(199.35331618,412.97710112)(199.46831606,413.1971009)(199.54832153,413.45710297)
\curveto(199.56831596,413.52710057)(199.57831595,413.5971005)(199.57832153,413.66710297)
\curveto(199.58831594,413.74710035)(199.60331593,413.82710027)(199.62332153,413.90710297)
\lineto(199.62332153,414.04210297)
}
}
{
\newrgbcolor{curcolor}{0 0 0}
\pscustom[linestyle=none,fillstyle=solid,fillcolor=curcolor]
{
\newpath
\moveto(210.98160278,411.98710297)
\lineto(210.98160278,411.56710297)
\curveto(210.98159441,411.43710266)(210.95159444,411.33210277)(210.89160278,411.25210297)
\curveto(210.84159455,411.2021029)(210.77659462,411.16710293)(210.69660278,411.14710297)
\curveto(210.61659478,411.13710296)(210.52659487,411.13210297)(210.42660278,411.13210297)
\lineto(209.60160278,411.13210297)
\lineto(209.31660278,411.13210297)
\curveto(209.23659616,411.14210296)(209.17159622,411.16710293)(209.12160278,411.20710297)
\curveto(209.05159634,411.25710284)(209.01159638,411.32210278)(209.00160278,411.40210297)
\curveto(208.9915964,411.48210262)(208.97159642,411.56210254)(208.94160278,411.64210297)
\curveto(208.92159647,411.66210244)(208.90159649,411.67710242)(208.88160278,411.68710297)
\curveto(208.87159652,411.70710239)(208.85659654,411.72710237)(208.83660278,411.74710297)
\curveto(208.72659667,411.74710235)(208.64659675,411.72210238)(208.59660278,411.67210297)
\lineto(208.44660278,411.52210297)
\curveto(208.37659702,411.47210263)(208.31159708,411.42710267)(208.25160278,411.38710297)
\curveto(208.1915972,411.35710274)(208.12659727,411.31710278)(208.05660278,411.26710297)
\curveto(208.01659738,411.24710285)(207.97159742,411.22710287)(207.92160278,411.20710297)
\curveto(207.88159751,411.18710291)(207.83659756,411.16710293)(207.78660278,411.14710297)
\curveto(207.64659775,411.097103)(207.4965979,411.05210305)(207.33660278,411.01210297)
\curveto(207.28659811,410.99210311)(207.24159815,410.98210312)(207.20160278,410.98210297)
\curveto(207.16159823,410.98210312)(207.12159827,410.97710312)(207.08160278,410.96710297)
\lineto(206.94660278,410.96710297)
\curveto(206.91659848,410.95710314)(206.87659852,410.95210315)(206.82660278,410.95210297)
\lineto(206.69160278,410.95210297)
\curveto(206.63159876,410.93210317)(206.54159885,410.92710317)(206.42160278,410.93710297)
\curveto(206.30159909,410.93710316)(206.21659918,410.94710315)(206.16660278,410.96710297)
\curveto(206.0965993,410.98710311)(206.03159936,410.9971031)(205.97160278,410.99710297)
\curveto(205.92159947,410.98710311)(205.86659953,410.99210311)(205.80660278,411.01210297)
\lineto(205.44660278,411.13210297)
\curveto(205.33660006,411.16210294)(205.22660017,411.2021029)(205.11660278,411.25210297)
\curveto(204.76660063,411.4021027)(204.45160094,411.63210247)(204.17160278,411.94210297)
\curveto(203.90160149,412.26210184)(203.68660171,412.5971015)(203.52660278,412.94710297)
\curveto(203.47660192,413.05710104)(203.43660196,413.16210094)(203.40660278,413.26210297)
\curveto(203.37660202,413.37210073)(203.34160205,413.48210062)(203.30160278,413.59210297)
\curveto(203.2916021,413.63210047)(203.28660211,413.66710043)(203.28660278,413.69710297)
\curveto(203.28660211,413.73710036)(203.27660212,413.78210032)(203.25660278,413.83210297)
\curveto(203.23660216,413.91210019)(203.21660218,413.9971001)(203.19660278,414.08710297)
\curveto(203.18660221,414.18709991)(203.17160222,414.28709981)(203.15160278,414.38710297)
\curveto(203.14160225,414.41709968)(203.13660226,414.45209965)(203.13660278,414.49210297)
\curveto(203.14660225,414.53209957)(203.14660225,414.56709953)(203.13660278,414.59710297)
\lineto(203.13660278,414.73210297)
\curveto(203.13660226,414.78209932)(203.13160226,414.83209927)(203.12160278,414.88210297)
\curveto(203.11160228,414.93209917)(203.10660229,414.98709911)(203.10660278,415.04710297)
\curveto(203.10660229,415.11709898)(203.11160228,415.17209893)(203.12160278,415.21210297)
\curveto(203.13160226,415.26209884)(203.13660226,415.30709879)(203.13660278,415.34710297)
\lineto(203.13660278,415.49710297)
\curveto(203.14660225,415.54709855)(203.14660225,415.59209851)(203.13660278,415.63210297)
\curveto(203.13660226,415.68209842)(203.14660225,415.73209837)(203.16660278,415.78210297)
\curveto(203.18660221,415.89209821)(203.20160219,415.9970981)(203.21160278,416.09710297)
\curveto(203.23160216,416.1970979)(203.25660214,416.2970978)(203.28660278,416.39710297)
\curveto(203.32660207,416.51709758)(203.36160203,416.63209747)(203.39160278,416.74210297)
\curveto(203.42160197,416.85209725)(203.46160193,416.96209714)(203.51160278,417.07210297)
\curveto(203.65160174,417.37209673)(203.82660157,417.65709644)(204.03660278,417.92710297)
\curveto(204.05660134,417.95709614)(204.08160131,417.98209612)(204.11160278,418.00210297)
\curveto(204.15160124,418.03209607)(204.18160121,418.06209604)(204.20160278,418.09210297)
\curveto(204.24160115,418.14209596)(204.28160111,418.18709591)(204.32160278,418.22710297)
\curveto(204.36160103,418.26709583)(204.40660099,418.30709579)(204.45660278,418.34710297)
\curveto(204.4966009,418.36709573)(204.53160086,418.39209571)(204.56160278,418.42210297)
\curveto(204.5916008,418.46209564)(204.62660077,418.49209561)(204.66660278,418.51210297)
\curveto(204.91660048,418.68209542)(205.20660019,418.82209528)(205.53660278,418.93210297)
\curveto(205.60659979,418.95209515)(205.67659972,418.96709513)(205.74660278,418.97710297)
\curveto(205.82659957,418.98709511)(205.90659949,419.0020951)(205.98660278,419.02210297)
\curveto(206.05659934,419.04209506)(206.14659925,419.05209505)(206.25660278,419.05210297)
\curveto(206.36659903,419.06209504)(206.47659892,419.06709503)(206.58660278,419.06710297)
\curveto(206.6965987,419.06709503)(206.80159859,419.06209504)(206.90160278,419.05210297)
\curveto(207.01159838,419.04209506)(207.10159829,419.02709507)(207.17160278,419.00710297)
\curveto(207.32159807,418.95709514)(207.46659793,418.91209519)(207.60660278,418.87210297)
\curveto(207.74659765,418.83209527)(207.87659752,418.77709532)(207.99660278,418.70710297)
\curveto(208.06659733,418.65709544)(208.13159726,418.60709549)(208.19160278,418.55710297)
\curveto(208.25159714,418.51709558)(208.31659708,418.47209563)(208.38660278,418.42210297)
\curveto(208.42659697,418.39209571)(208.48159691,418.35209575)(208.55160278,418.30210297)
\curveto(208.63159676,418.25209585)(208.70659669,418.25209585)(208.77660278,418.30210297)
\curveto(208.81659658,418.32209578)(208.83659656,418.35709574)(208.83660278,418.40710297)
\curveto(208.83659656,418.45709564)(208.84659655,418.50709559)(208.86660278,418.55710297)
\lineto(208.86660278,418.70710297)
\curveto(208.87659652,418.73709536)(208.88159651,418.77209533)(208.88160278,418.81210297)
\lineto(208.88160278,418.93210297)
\lineto(208.88160278,420.97210297)
\curveto(208.88159651,421.08209302)(208.87659652,421.2020929)(208.86660278,421.33210297)
\curveto(208.86659653,421.47209263)(208.8915965,421.57709252)(208.94160278,421.64710297)
\curveto(208.98159641,421.72709237)(209.05659634,421.77709232)(209.16660278,421.79710297)
\curveto(209.18659621,421.80709229)(209.20659619,421.80709229)(209.22660278,421.79710297)
\curveto(209.24659615,421.7970923)(209.26659613,421.8020923)(209.28660278,421.81210297)
\lineto(210.35160278,421.81210297)
\curveto(210.47159492,421.81209229)(210.58159481,421.80709229)(210.68160278,421.79710297)
\curveto(210.78159461,421.78709231)(210.85659454,421.74709235)(210.90660278,421.67710297)
\curveto(210.95659444,421.5970925)(210.98159441,421.49209261)(210.98160278,421.36210297)
\lineto(210.98160278,421.00210297)
\lineto(210.98160278,411.98710297)
\moveto(208.94160278,414.92710297)
\curveto(208.95159644,414.96709913)(208.95159644,415.00709909)(208.94160278,415.04710297)
\lineto(208.94160278,415.18210297)
\curveto(208.94159645,415.28209882)(208.93659646,415.38209872)(208.92660278,415.48210297)
\curveto(208.91659648,415.58209852)(208.90159649,415.67209843)(208.88160278,415.75210297)
\curveto(208.86159653,415.86209824)(208.84159655,415.96209814)(208.82160278,416.05210297)
\curveto(208.81159658,416.14209796)(208.78659661,416.22709787)(208.74660278,416.30710297)
\curveto(208.60659679,416.66709743)(208.40159699,416.95209715)(208.13160278,417.16210297)
\curveto(207.87159752,417.37209673)(207.4915979,417.47709662)(206.99160278,417.47710297)
\curveto(206.93159846,417.47709662)(206.85159854,417.46709663)(206.75160278,417.44710297)
\curveto(206.67159872,417.42709667)(206.5965988,417.40709669)(206.52660278,417.38710297)
\curveto(206.46659893,417.37709672)(206.40659899,417.35709674)(206.34660278,417.32710297)
\curveto(206.07659932,417.21709688)(205.86659953,417.04709705)(205.71660278,416.81710297)
\curveto(205.56659983,416.58709751)(205.44659995,416.32709777)(205.35660278,416.03710297)
\curveto(205.32660007,415.93709816)(205.30660009,415.83709826)(205.29660278,415.73710297)
\curveto(205.28660011,415.63709846)(205.26660013,415.53209857)(205.23660278,415.42210297)
\lineto(205.23660278,415.21210297)
\curveto(205.21660018,415.12209898)(205.21160018,414.9970991)(205.22160278,414.83710297)
\curveto(205.23160016,414.68709941)(205.24660015,414.57709952)(205.26660278,414.50710297)
\lineto(205.26660278,414.41710297)
\curveto(205.27660012,414.3970997)(205.28160011,414.37709972)(205.28160278,414.35710297)
\curveto(205.30160009,414.27709982)(205.31660008,414.2020999)(205.32660278,414.13210297)
\curveto(205.34660005,414.06210004)(205.36660003,413.98710011)(205.38660278,413.90710297)
\curveto(205.55659984,413.38710071)(205.84659955,413.0021011)(206.25660278,412.75210297)
\curveto(206.38659901,412.66210144)(206.56659883,412.59210151)(206.79660278,412.54210297)
\curveto(206.83659856,412.53210157)(206.8965985,412.52710157)(206.97660278,412.52710297)
\curveto(207.00659839,412.51710158)(207.05159834,412.50710159)(207.11160278,412.49710297)
\curveto(207.18159821,412.4971016)(207.23659816,412.5021016)(207.27660278,412.51210297)
\curveto(207.35659804,412.53210157)(207.43659796,412.54710155)(207.51660278,412.55710297)
\curveto(207.5965978,412.56710153)(207.67659772,412.58710151)(207.75660278,412.61710297)
\curveto(208.00659739,412.72710137)(208.20659719,412.86710123)(208.35660278,413.03710297)
\curveto(208.50659689,413.20710089)(208.63659676,413.42210068)(208.74660278,413.68210297)
\curveto(208.78659661,413.77210033)(208.81659658,413.86210024)(208.83660278,413.95210297)
\curveto(208.85659654,414.05210005)(208.87659652,414.15709994)(208.89660278,414.26710297)
\curveto(208.90659649,414.31709978)(208.90659649,414.36209974)(208.89660278,414.40210297)
\curveto(208.8965965,414.45209965)(208.90659649,414.5020996)(208.92660278,414.55210297)
\curveto(208.93659646,414.58209952)(208.94159645,414.61709948)(208.94160278,414.65710297)
\lineto(208.94160278,414.79210297)
\lineto(208.94160278,414.92710297)
}
}
{
\newrgbcolor{curcolor}{0 0 0}
\pscustom[linestyle=none,fillstyle=solid,fillcolor=curcolor]
{
\newpath
\moveto(220.33152466,415.31710297)
\curveto(220.35151609,415.25709884)(220.36151608,415.17209893)(220.36152466,415.06210297)
\curveto(220.36151608,414.95209915)(220.35151609,414.86709923)(220.33152466,414.80710297)
\lineto(220.33152466,414.65710297)
\curveto(220.31151613,414.57709952)(220.30151614,414.4970996)(220.30152466,414.41710297)
\curveto(220.31151613,414.33709976)(220.30651613,414.25709984)(220.28652466,414.17710297)
\curveto(220.26651617,414.10709999)(220.25151619,414.04210006)(220.24152466,413.98210297)
\curveto(220.23151621,413.92210018)(220.22151622,413.85710024)(220.21152466,413.78710297)
\curveto(220.17151627,413.67710042)(220.1365163,413.56210054)(220.10652466,413.44210297)
\curveto(220.07651636,413.33210077)(220.0365164,413.22710087)(219.98652466,413.12710297)
\curveto(219.77651666,412.64710145)(219.50151694,412.25710184)(219.16152466,411.95710297)
\curveto(218.82151762,411.65710244)(218.41151803,411.40710269)(217.93152466,411.20710297)
\curveto(217.81151863,411.15710294)(217.68651875,411.12210298)(217.55652466,411.10210297)
\curveto(217.436519,411.07210303)(217.31151913,411.04210306)(217.18152466,411.01210297)
\curveto(217.13151931,410.99210311)(217.07651936,410.98210312)(217.01652466,410.98210297)
\curveto(216.95651948,410.98210312)(216.90151954,410.97710312)(216.85152466,410.96710297)
\lineto(216.74652466,410.96710297)
\curveto(216.71651972,410.95710314)(216.68651975,410.95210315)(216.65652466,410.95210297)
\curveto(216.60651983,410.94210316)(216.52651991,410.93710316)(216.41652466,410.93710297)
\curveto(216.30652013,410.92710317)(216.22152022,410.93210317)(216.16152466,410.95210297)
\lineto(216.01152466,410.95210297)
\curveto(215.96152048,410.96210314)(215.90652053,410.96710313)(215.84652466,410.96710297)
\curveto(215.79652064,410.95710314)(215.74652069,410.96210314)(215.69652466,410.98210297)
\curveto(215.65652078,410.99210311)(215.61652082,410.9971031)(215.57652466,410.99710297)
\curveto(215.54652089,410.9971031)(215.50652093,411.0021031)(215.45652466,411.01210297)
\curveto(215.35652108,411.04210306)(215.25652118,411.06710303)(215.15652466,411.08710297)
\curveto(215.05652138,411.10710299)(214.96152148,411.13710296)(214.87152466,411.17710297)
\curveto(214.75152169,411.21710288)(214.6365218,411.25710284)(214.52652466,411.29710297)
\curveto(214.42652201,411.33710276)(214.32152212,411.38710271)(214.21152466,411.44710297)
\curveto(213.86152258,411.65710244)(213.56152288,411.9021022)(213.31152466,412.18210297)
\curveto(213.06152338,412.46210164)(212.85152359,412.7971013)(212.68152466,413.18710297)
\curveto(212.63152381,413.27710082)(212.59152385,413.37210073)(212.56152466,413.47210297)
\curveto(212.5415239,413.57210053)(212.51652392,413.67710042)(212.48652466,413.78710297)
\curveto(212.46652397,413.83710026)(212.45652398,413.88210022)(212.45652466,413.92210297)
\curveto(212.45652398,413.96210014)(212.44652399,414.00710009)(212.42652466,414.05710297)
\curveto(212.40652403,414.13709996)(212.39652404,414.21709988)(212.39652466,414.29710297)
\curveto(212.39652404,414.38709971)(212.38652405,414.47209963)(212.36652466,414.55210297)
\curveto(212.35652408,414.6020995)(212.35152409,414.64709945)(212.35152466,414.68710297)
\lineto(212.35152466,414.82210297)
\curveto(212.33152411,414.88209922)(212.32152412,414.96709913)(212.32152466,415.07710297)
\curveto(212.33152411,415.18709891)(212.34652409,415.27209883)(212.36652466,415.33210297)
\lineto(212.36652466,415.43710297)
\curveto(212.37652406,415.48709861)(212.37652406,415.53709856)(212.36652466,415.58710297)
\curveto(212.36652407,415.64709845)(212.37652406,415.7020984)(212.39652466,415.75210297)
\curveto(212.40652403,415.8020983)(212.41152403,415.84709825)(212.41152466,415.88710297)
\curveto(212.41152403,415.93709816)(212.42152402,415.98709811)(212.44152466,416.03710297)
\curveto(212.48152396,416.16709793)(212.51652392,416.29209781)(212.54652466,416.41210297)
\curveto(212.57652386,416.54209756)(212.61652382,416.66709743)(212.66652466,416.78710297)
\curveto(212.84652359,417.1970969)(213.06152338,417.53709656)(213.31152466,417.80710297)
\curveto(213.56152288,418.08709601)(213.86652257,418.34209576)(214.22652466,418.57210297)
\curveto(214.32652211,418.62209548)(214.43152201,418.66709543)(214.54152466,418.70710297)
\curveto(214.65152179,418.74709535)(214.76152168,418.79209531)(214.87152466,418.84210297)
\curveto(215.00152144,418.89209521)(215.1365213,418.92709517)(215.27652466,418.94710297)
\curveto(215.41652102,418.96709513)(215.56152088,418.9970951)(215.71152466,419.03710297)
\curveto(215.79152065,419.04709505)(215.86652057,419.05209505)(215.93652466,419.05210297)
\curveto(216.00652043,419.05209505)(216.07652036,419.05709504)(216.14652466,419.06710297)
\curveto(216.72651971,419.07709502)(217.22651921,419.01709508)(217.64652466,418.88710297)
\curveto(218.07651836,418.75709534)(218.45651798,418.57709552)(218.78652466,418.34710297)
\curveto(218.89651754,418.26709583)(219.00651743,418.17709592)(219.11652466,418.07710297)
\curveto(219.2365172,417.98709611)(219.3365171,417.88709621)(219.41652466,417.77710297)
\curveto(219.49651694,417.67709642)(219.56651687,417.57709652)(219.62652466,417.47710297)
\curveto(219.69651674,417.37709672)(219.76651667,417.27209683)(219.83652466,417.16210297)
\curveto(219.90651653,417.05209705)(219.96151648,416.93209717)(220.00152466,416.80210297)
\curveto(220.0415164,416.68209742)(220.08651635,416.55209755)(220.13652466,416.41210297)
\curveto(220.16651627,416.33209777)(220.19151625,416.24709785)(220.21152466,416.15710297)
\lineto(220.27152466,415.88710297)
\curveto(220.28151616,415.84709825)(220.28651615,415.80709829)(220.28652466,415.76710297)
\curveto(220.28651615,415.72709837)(220.29151615,415.68709841)(220.30152466,415.64710297)
\curveto(220.32151612,415.5970985)(220.32651611,415.54209856)(220.31652466,415.48210297)
\curveto(220.30651613,415.42209868)(220.31151613,415.36709873)(220.33152466,415.31710297)
\moveto(218.23152466,414.77710297)
\curveto(218.2415182,414.82709927)(218.24651819,414.8970992)(218.24652466,414.98710297)
\curveto(218.24651819,415.08709901)(218.2415182,415.16209894)(218.23152466,415.21210297)
\lineto(218.23152466,415.33210297)
\curveto(218.21151823,415.38209872)(218.20151824,415.43709866)(218.20152466,415.49710297)
\curveto(218.20151824,415.55709854)(218.19651824,415.61209849)(218.18652466,415.66210297)
\curveto(218.18651825,415.7020984)(218.18151826,415.73209837)(218.17152466,415.75210297)
\lineto(218.11152466,415.99210297)
\curveto(218.10151834,416.08209802)(218.08151836,416.16709793)(218.05152466,416.24710297)
\curveto(217.9415185,416.50709759)(217.81151863,416.72709737)(217.66152466,416.90710297)
\curveto(217.51151893,417.097097)(217.31151913,417.24709685)(217.06152466,417.35710297)
\curveto(217.00151944,417.37709672)(216.9415195,417.39209671)(216.88152466,417.40210297)
\curveto(216.82151962,417.42209668)(216.75651968,417.44209666)(216.68652466,417.46210297)
\curveto(216.60651983,417.48209662)(216.52151992,417.48709661)(216.43152466,417.47710297)
\lineto(216.16152466,417.47710297)
\curveto(216.13152031,417.45709664)(216.09652034,417.44709665)(216.05652466,417.44710297)
\curveto(216.01652042,417.45709664)(215.98152046,417.45709664)(215.95152466,417.44710297)
\lineto(215.74152466,417.38710297)
\curveto(215.68152076,417.37709672)(215.62652081,417.35709674)(215.57652466,417.32710297)
\curveto(215.32652111,417.21709688)(215.12152132,417.05709704)(214.96152466,416.84710297)
\curveto(214.81152163,416.64709745)(214.69152175,416.41209769)(214.60152466,416.14210297)
\curveto(214.57152187,416.04209806)(214.54652189,415.93709816)(214.52652466,415.82710297)
\curveto(214.51652192,415.71709838)(214.50152194,415.60709849)(214.48152466,415.49710297)
\curveto(214.47152197,415.44709865)(214.46652197,415.3970987)(214.46652466,415.34710297)
\lineto(214.46652466,415.19710297)
\curveto(214.44652199,415.12709897)(214.436522,415.02209908)(214.43652466,414.88210297)
\curveto(214.44652199,414.74209936)(214.46152198,414.63709946)(214.48152466,414.56710297)
\lineto(214.48152466,414.43210297)
\curveto(214.50152194,414.35209975)(214.51652192,414.27209983)(214.52652466,414.19210297)
\curveto(214.5365219,414.12209998)(214.55152189,414.04710005)(214.57152466,413.96710297)
\curveto(214.67152177,413.66710043)(214.77652166,413.42210068)(214.88652466,413.23210297)
\curveto(215.00652143,413.05210105)(215.19152125,412.88710121)(215.44152466,412.73710297)
\curveto(215.51152093,412.68710141)(215.58652085,412.64710145)(215.66652466,412.61710297)
\curveto(215.75652068,412.58710151)(215.84652059,412.56210154)(215.93652466,412.54210297)
\curveto(215.97652046,412.53210157)(216.01152043,412.52710157)(216.04152466,412.52710297)
\curveto(216.07152037,412.53710156)(216.10652033,412.53710156)(216.14652466,412.52710297)
\lineto(216.26652466,412.49710297)
\curveto(216.31652012,412.4971016)(216.36152008,412.5021016)(216.40152466,412.51210297)
\lineto(216.52152466,412.51210297)
\curveto(216.60151984,412.53210157)(216.68151976,412.54710155)(216.76152466,412.55710297)
\curveto(216.8415196,412.56710153)(216.91651952,412.58710151)(216.98652466,412.61710297)
\curveto(217.24651919,412.71710138)(217.45651898,412.85210125)(217.61652466,413.02210297)
\curveto(217.77651866,413.19210091)(217.91151853,413.4021007)(218.02152466,413.65210297)
\curveto(218.06151838,413.75210035)(218.09151835,413.85210025)(218.11152466,413.95210297)
\curveto(218.13151831,414.05210005)(218.15651828,414.15709994)(218.18652466,414.26710297)
\curveto(218.19651824,414.30709979)(218.20151824,414.34209976)(218.20152466,414.37210297)
\curveto(218.20151824,414.41209969)(218.20651823,414.45209965)(218.21652466,414.49210297)
\lineto(218.21652466,414.62710297)
\curveto(218.21651822,414.67709942)(218.22151822,414.72709937)(218.23152466,414.77710297)
}
}
{
\newrgbcolor{curcolor}{0 0 0}
\pscustom[linestyle=none,fillstyle=solid,fillcolor=curcolor]
{
\newpath
\moveto(226.15644653,419.06710297)
\curveto(226.26644122,419.06709503)(226.36144112,419.05709504)(226.44144653,419.03710297)
\curveto(226.53144095,419.01709508)(226.60144088,418.97209513)(226.65144653,418.90210297)
\curveto(226.71144077,418.82209528)(226.74144074,418.68209542)(226.74144653,418.48210297)
\lineto(226.74144653,417.97210297)
\lineto(226.74144653,417.59710297)
\curveto(226.75144073,417.45709664)(226.73644075,417.34709675)(226.69644653,417.26710297)
\curveto(226.65644083,417.1970969)(226.59644089,417.15209695)(226.51644653,417.13210297)
\curveto(226.44644104,417.11209699)(226.36144112,417.102097)(226.26144653,417.10210297)
\curveto(226.17144131,417.102097)(226.07144141,417.10709699)(225.96144653,417.11710297)
\curveto(225.86144162,417.12709697)(225.76644172,417.12209698)(225.67644653,417.10210297)
\curveto(225.60644188,417.08209702)(225.53644195,417.06709703)(225.46644653,417.05710297)
\curveto(225.39644209,417.05709704)(225.33144215,417.04709705)(225.27144653,417.02710297)
\curveto(225.11144237,416.97709712)(224.95144253,416.9020972)(224.79144653,416.80210297)
\curveto(224.63144285,416.71209739)(224.50644298,416.60709749)(224.41644653,416.48710297)
\curveto(224.36644312,416.40709769)(224.31144317,416.32209778)(224.25144653,416.23210297)
\curveto(224.20144328,416.15209795)(224.15144333,416.06709803)(224.10144653,415.97710297)
\curveto(224.07144341,415.8970982)(224.04144344,415.81209829)(224.01144653,415.72210297)
\lineto(223.95144653,415.48210297)
\curveto(223.93144355,415.41209869)(223.92144356,415.33709876)(223.92144653,415.25710297)
\curveto(223.92144356,415.18709891)(223.91144357,415.11709898)(223.89144653,415.04710297)
\curveto(223.8814436,415.00709909)(223.87644361,414.96709913)(223.87644653,414.92710297)
\curveto(223.8864436,414.8970992)(223.8864436,414.86709923)(223.87644653,414.83710297)
\lineto(223.87644653,414.59710297)
\curveto(223.85644363,414.52709957)(223.85144363,414.44709965)(223.86144653,414.35710297)
\curveto(223.87144361,414.27709982)(223.87644361,414.1970999)(223.87644653,414.11710297)
\lineto(223.87644653,413.15710297)
\lineto(223.87644653,411.88210297)
\curveto(223.87644361,411.75210235)(223.87144361,411.63210247)(223.86144653,411.52210297)
\curveto(223.85144363,411.41210269)(223.82144366,411.32210278)(223.77144653,411.25210297)
\curveto(223.75144373,411.22210288)(223.71644377,411.1971029)(223.66644653,411.17710297)
\curveto(223.62644386,411.16710293)(223.5814439,411.15710294)(223.53144653,411.14710297)
\lineto(223.45644653,411.14710297)
\curveto(223.40644408,411.13710296)(223.35144413,411.13210297)(223.29144653,411.13210297)
\lineto(223.12644653,411.13210297)
\lineto(222.48144653,411.13210297)
\curveto(222.42144506,411.14210296)(222.35644513,411.14710295)(222.28644653,411.14710297)
\lineto(222.09144653,411.14710297)
\curveto(222.04144544,411.16710293)(221.99144549,411.18210292)(221.94144653,411.19210297)
\curveto(221.89144559,411.21210289)(221.85644563,411.24710285)(221.83644653,411.29710297)
\curveto(221.79644569,411.34710275)(221.77144571,411.41710268)(221.76144653,411.50710297)
\lineto(221.76144653,411.80710297)
\lineto(221.76144653,412.82710297)
\lineto(221.76144653,417.05710297)
\lineto(221.76144653,418.16710297)
\lineto(221.76144653,418.45210297)
\curveto(221.76144572,418.55209555)(221.7814457,418.63209547)(221.82144653,418.69210297)
\curveto(221.87144561,418.77209533)(221.94644554,418.82209528)(222.04644653,418.84210297)
\curveto(222.14644534,418.86209524)(222.26644522,418.87209523)(222.40644653,418.87210297)
\lineto(223.17144653,418.87210297)
\curveto(223.29144419,418.87209523)(223.39644409,418.86209524)(223.48644653,418.84210297)
\curveto(223.57644391,418.83209527)(223.64644384,418.78709531)(223.69644653,418.70710297)
\curveto(223.72644376,418.65709544)(223.74144374,418.58709551)(223.74144653,418.49710297)
\lineto(223.77144653,418.22710297)
\curveto(223.7814437,418.14709595)(223.79644369,418.07209603)(223.81644653,418.00210297)
\curveto(223.84644364,417.93209617)(223.89644359,417.8970962)(223.96644653,417.89710297)
\curveto(223.9864435,417.91709618)(224.00644348,417.92709617)(224.02644653,417.92710297)
\curveto(224.04644344,417.92709617)(224.06644342,417.93709616)(224.08644653,417.95710297)
\curveto(224.14644334,418.00709609)(224.19644329,418.06209604)(224.23644653,418.12210297)
\curveto(224.2864432,418.19209591)(224.34644314,418.25209585)(224.41644653,418.30210297)
\curveto(224.45644303,418.33209577)(224.49144299,418.36209574)(224.52144653,418.39210297)
\curveto(224.55144293,418.43209567)(224.5864429,418.46709563)(224.62644653,418.49710297)
\lineto(224.89644653,418.67710297)
\curveto(224.99644249,418.73709536)(225.09644239,418.79209531)(225.19644653,418.84210297)
\curveto(225.29644219,418.88209522)(225.39644209,418.91709518)(225.49644653,418.94710297)
\lineto(225.82644653,419.03710297)
\curveto(225.85644163,419.04709505)(225.91144157,419.04709505)(225.99144653,419.03710297)
\curveto(226.0814414,419.03709506)(226.13644135,419.04709505)(226.15644653,419.06710297)
}
}
{
\newrgbcolor{curcolor}{0 0 0}
\pscustom[linestyle=none,fillstyle=solid,fillcolor=curcolor]
{
\newpath
\moveto(234.66285278,415.07710297)
\curveto(234.68284462,414.9970991)(234.68284462,414.90709919)(234.66285278,414.80710297)
\curveto(234.64284466,414.70709939)(234.60784469,414.64209946)(234.55785278,414.61210297)
\curveto(234.50784479,414.57209953)(234.43284487,414.54209956)(234.33285278,414.52210297)
\curveto(234.24284506,414.51209959)(234.13784516,414.5020996)(234.01785278,414.49210297)
\lineto(233.67285278,414.49210297)
\curveto(233.56284574,414.5020996)(233.46284584,414.50709959)(233.37285278,414.50710297)
\lineto(229.71285278,414.50710297)
\lineto(229.50285278,414.50710297)
\curveto(229.44284986,414.50709959)(229.38784991,414.4970996)(229.33785278,414.47710297)
\curveto(229.25785004,414.43709966)(229.20785009,414.3970997)(229.18785278,414.35710297)
\curveto(229.16785013,414.33709976)(229.14785015,414.2970998)(229.12785278,414.23710297)
\curveto(229.10785019,414.18709991)(229.1028502,414.13709996)(229.11285278,414.08710297)
\curveto(229.13285017,414.02710007)(229.14285016,413.96710013)(229.14285278,413.90710297)
\curveto(229.15285015,413.85710024)(229.16785013,413.8021003)(229.18785278,413.74210297)
\curveto(229.26785003,413.5021006)(229.36284994,413.3021008)(229.47285278,413.14210297)
\curveto(229.59284971,412.99210111)(229.75284955,412.85710124)(229.95285278,412.73710297)
\curveto(230.03284927,412.68710141)(230.11284919,412.65210145)(230.19285278,412.63210297)
\curveto(230.28284902,412.62210148)(230.37284893,412.6021015)(230.46285278,412.57210297)
\curveto(230.54284876,412.55210155)(230.65284865,412.53710156)(230.79285278,412.52710297)
\curveto(230.93284837,412.51710158)(231.05284825,412.52210158)(231.15285278,412.54210297)
\lineto(231.28785278,412.54210297)
\curveto(231.38784791,412.56210154)(231.47784782,412.58210152)(231.55785278,412.60210297)
\curveto(231.64784765,412.63210147)(231.73284757,412.66210144)(231.81285278,412.69210297)
\curveto(231.91284739,412.74210136)(232.02284728,412.80710129)(232.14285278,412.88710297)
\curveto(232.27284703,412.96710113)(232.36784693,413.04710105)(232.42785278,413.12710297)
\curveto(232.47784682,413.1971009)(232.52784677,413.26210084)(232.57785278,413.32210297)
\curveto(232.63784666,413.39210071)(232.70784659,413.44210066)(232.78785278,413.47210297)
\curveto(232.88784641,413.52210058)(233.01284629,413.54210056)(233.16285278,413.53210297)
\lineto(233.59785278,413.53210297)
\lineto(233.77785278,413.53210297)
\curveto(233.84784545,413.54210056)(233.90784539,413.53710056)(233.95785278,413.51710297)
\lineto(234.10785278,413.51710297)
\curveto(234.20784509,413.4971006)(234.27784502,413.47210063)(234.31785278,413.44210297)
\curveto(234.35784494,413.42210068)(234.37784492,413.37710072)(234.37785278,413.30710297)
\curveto(234.38784491,413.23710086)(234.38284492,413.17710092)(234.36285278,413.12710297)
\curveto(234.31284499,412.98710111)(234.25784504,412.86210124)(234.19785278,412.75210297)
\curveto(234.13784516,412.64210146)(234.06784523,412.53210157)(233.98785278,412.42210297)
\curveto(233.76784553,412.09210201)(233.51784578,411.82710227)(233.23785278,411.62710297)
\curveto(232.95784634,411.42710267)(232.60784669,411.25710284)(232.18785278,411.11710297)
\curveto(232.07784722,411.07710302)(231.96784733,411.05210305)(231.85785278,411.04210297)
\curveto(231.74784755,411.03210307)(231.63284767,411.01210309)(231.51285278,410.98210297)
\curveto(231.47284783,410.97210313)(231.42784787,410.97210313)(231.37785278,410.98210297)
\curveto(231.33784796,410.98210312)(231.297848,410.97710312)(231.25785278,410.96710297)
\lineto(231.09285278,410.96710297)
\curveto(231.04284826,410.94710315)(230.98284832,410.94210316)(230.91285278,410.95210297)
\curveto(230.85284845,410.95210315)(230.7978485,410.95710314)(230.74785278,410.96710297)
\curveto(230.66784863,410.97710312)(230.5978487,410.97710312)(230.53785278,410.96710297)
\curveto(230.47784882,410.95710314)(230.41284889,410.96210314)(230.34285278,410.98210297)
\curveto(230.29284901,411.0021031)(230.23784906,411.01210309)(230.17785278,411.01210297)
\curveto(230.11784918,411.01210309)(230.06284924,411.02210308)(230.01285278,411.04210297)
\curveto(229.9028494,411.06210304)(229.79284951,411.08710301)(229.68285278,411.11710297)
\curveto(229.57284973,411.13710296)(229.47284983,411.17210293)(229.38285278,411.22210297)
\curveto(229.27285003,411.26210284)(229.16785013,411.2971028)(229.06785278,411.32710297)
\curveto(228.97785032,411.36710273)(228.89285041,411.41210269)(228.81285278,411.46210297)
\curveto(228.49285081,411.66210244)(228.20785109,411.89210221)(227.95785278,412.15210297)
\curveto(227.70785159,412.42210168)(227.5028518,412.73210137)(227.34285278,413.08210297)
\curveto(227.29285201,413.19210091)(227.25285205,413.3021008)(227.22285278,413.41210297)
\curveto(227.19285211,413.53210057)(227.15285215,413.65210045)(227.10285278,413.77210297)
\curveto(227.09285221,413.81210029)(227.08785221,413.84710025)(227.08785278,413.87710297)
\curveto(227.08785221,413.91710018)(227.08285222,413.95710014)(227.07285278,413.99710297)
\curveto(227.03285227,414.11709998)(227.00785229,414.24709985)(226.99785278,414.38710297)
\lineto(226.96785278,414.80710297)
\curveto(226.96785233,414.85709924)(226.96285234,414.91209919)(226.95285278,414.97210297)
\curveto(226.95285235,415.03209907)(226.95785234,415.08709901)(226.96785278,415.13710297)
\lineto(226.96785278,415.31710297)
\lineto(227.01285278,415.67710297)
\curveto(227.05285225,415.84709825)(227.08785221,416.01209809)(227.11785278,416.17210297)
\curveto(227.14785215,416.33209777)(227.19285211,416.48209762)(227.25285278,416.62210297)
\curveto(227.68285162,417.66209644)(228.41285089,418.3970957)(229.44285278,418.82710297)
\curveto(229.58284972,418.88709521)(229.72284958,418.92709517)(229.86285278,418.94710297)
\curveto(230.01284929,418.97709512)(230.16784913,419.01209509)(230.32785278,419.05210297)
\curveto(230.40784889,419.06209504)(230.48284882,419.06709503)(230.55285278,419.06710297)
\curveto(230.62284868,419.06709503)(230.6978486,419.07209503)(230.77785278,419.08210297)
\curveto(231.28784801,419.09209501)(231.72284758,419.03209507)(232.08285278,418.90210297)
\curveto(232.45284685,418.78209532)(232.78284652,418.62209548)(233.07285278,418.42210297)
\curveto(233.16284614,418.36209574)(233.25284605,418.29209581)(233.34285278,418.21210297)
\curveto(233.43284587,418.14209596)(233.51284579,418.06709603)(233.58285278,417.98710297)
\curveto(233.61284569,417.93709616)(233.65284565,417.8970962)(233.70285278,417.86710297)
\curveto(233.78284552,417.75709634)(233.85784544,417.64209646)(233.92785278,417.52210297)
\curveto(233.9978453,417.41209669)(234.07284523,417.2970968)(234.15285278,417.17710297)
\curveto(234.2028451,417.08709701)(234.24284506,416.99209711)(234.27285278,416.89210297)
\curveto(234.31284499,416.8020973)(234.35284495,416.7020974)(234.39285278,416.59210297)
\curveto(234.44284486,416.46209764)(234.48284482,416.32709777)(234.51285278,416.18710297)
\curveto(234.54284476,416.04709805)(234.57784472,415.90709819)(234.61785278,415.76710297)
\curveto(234.63784466,415.68709841)(234.64284466,415.5970985)(234.63285278,415.49710297)
\curveto(234.63284467,415.40709869)(234.64284466,415.32209878)(234.66285278,415.24210297)
\lineto(234.66285278,415.07710297)
\moveto(232.41285278,415.96210297)
\curveto(232.48284682,416.06209804)(232.48784681,416.18209792)(232.42785278,416.32210297)
\curveto(232.37784692,416.47209763)(232.33784696,416.58209752)(232.30785278,416.65210297)
\curveto(232.16784713,416.92209718)(231.98284732,417.12709697)(231.75285278,417.26710297)
\curveto(231.52284778,417.41709668)(231.2028481,417.4970966)(230.79285278,417.50710297)
\curveto(230.76284854,417.48709661)(230.72784857,417.48209662)(230.68785278,417.49210297)
\curveto(230.64784865,417.5020966)(230.61284869,417.5020966)(230.58285278,417.49210297)
\curveto(230.53284877,417.47209663)(230.47784882,417.45709664)(230.41785278,417.44710297)
\curveto(230.35784894,417.44709665)(230.302849,417.43709666)(230.25285278,417.41710297)
\curveto(229.81284949,417.27709682)(229.48784981,417.0020971)(229.27785278,416.59210297)
\curveto(229.25785004,416.55209755)(229.23285007,416.4970976)(229.20285278,416.42710297)
\curveto(229.18285012,416.36709773)(229.16785013,416.3020978)(229.15785278,416.23210297)
\curveto(229.14785015,416.17209793)(229.14785015,416.11209799)(229.15785278,416.05210297)
\curveto(229.17785012,415.99209811)(229.21285009,415.94209816)(229.26285278,415.90210297)
\curveto(229.34284996,415.85209825)(229.45284985,415.82709827)(229.59285278,415.82710297)
\lineto(229.99785278,415.82710297)
\lineto(231.66285278,415.82710297)
\lineto(232.09785278,415.82710297)
\curveto(232.25784704,415.83709826)(232.36284694,415.88209822)(232.41285278,415.96210297)
}
}
{
\newrgbcolor{curcolor}{0 0 0}
\pscustom[linestyle=none,fillstyle=solid,fillcolor=curcolor]
{
\newpath
\moveto(238.88113403,419.08210297)
\curveto(239.63112953,419.102095)(240.28112888,419.01709508)(240.83113403,418.82710297)
\curveto(241.39112777,418.64709545)(241.81612735,418.33209577)(242.10613403,417.88210297)
\curveto(242.17612699,417.77209633)(242.23612693,417.65709644)(242.28613403,417.53710297)
\curveto(242.34612682,417.42709667)(242.39612677,417.3020968)(242.43613403,417.16210297)
\curveto(242.45612671,417.102097)(242.4661267,417.03709706)(242.46613403,416.96710297)
\curveto(242.4661267,416.8970972)(242.45612671,416.83709726)(242.43613403,416.78710297)
\curveto(242.39612677,416.72709737)(242.34112682,416.68709741)(242.27113403,416.66710297)
\curveto(242.22112694,416.64709745)(242.161127,416.63709746)(242.09113403,416.63710297)
\lineto(241.88113403,416.63710297)
\lineto(241.22113403,416.63710297)
\curveto(241.15112801,416.63709746)(241.08112808,416.63209747)(241.01113403,416.62210297)
\curveto(240.94112822,416.62209748)(240.87612829,416.63209747)(240.81613403,416.65210297)
\curveto(240.71612845,416.67209743)(240.64112852,416.71209739)(240.59113403,416.77210297)
\curveto(240.54112862,416.83209727)(240.49612867,416.89209721)(240.45613403,416.95210297)
\lineto(240.33613403,417.16210297)
\curveto(240.30612886,417.24209686)(240.25612891,417.30709679)(240.18613403,417.35710297)
\curveto(240.08612908,417.43709666)(239.98612918,417.4970966)(239.88613403,417.53710297)
\curveto(239.79612937,417.57709652)(239.68112948,417.61209649)(239.54113403,417.64210297)
\curveto(239.47112969,417.66209644)(239.3661298,417.67709642)(239.22613403,417.68710297)
\curveto(239.09613007,417.6970964)(238.99613017,417.69209641)(238.92613403,417.67210297)
\lineto(238.82113403,417.67210297)
\lineto(238.67113403,417.64210297)
\curveto(238.63113053,417.64209646)(238.58613058,417.63709646)(238.53613403,417.62710297)
\curveto(238.3661308,417.57709652)(238.22613094,417.50709659)(238.11613403,417.41710297)
\curveto(238.01613115,417.33709676)(237.94613122,417.21209689)(237.90613403,417.04210297)
\curveto(237.88613128,416.97209713)(237.88613128,416.90709719)(237.90613403,416.84710297)
\curveto(237.92613124,416.78709731)(237.94613122,416.73709736)(237.96613403,416.69710297)
\curveto(238.03613113,416.57709752)(238.11613105,416.48209762)(238.20613403,416.41210297)
\curveto(238.30613086,416.34209776)(238.42113074,416.28209782)(238.55113403,416.23210297)
\curveto(238.74113042,416.15209795)(238.94613022,416.08209802)(239.16613403,416.02210297)
\lineto(239.85613403,415.87210297)
\curveto(240.09612907,415.83209827)(240.32612884,415.78209832)(240.54613403,415.72210297)
\curveto(240.77612839,415.67209843)(240.99112817,415.60709849)(241.19113403,415.52710297)
\curveto(241.28112788,415.48709861)(241.3661278,415.45209865)(241.44613403,415.42210297)
\curveto(241.53612763,415.4020987)(241.62112754,415.36709873)(241.70113403,415.31710297)
\curveto(241.89112727,415.1970989)(242.0611271,415.06709903)(242.21113403,414.92710297)
\curveto(242.37112679,414.78709931)(242.49612667,414.61209949)(242.58613403,414.40210297)
\curveto(242.61612655,414.33209977)(242.64112652,414.26209984)(242.66113403,414.19210297)
\curveto(242.68112648,414.12209998)(242.70112646,414.04710005)(242.72113403,413.96710297)
\curveto(242.73112643,413.90710019)(242.73612643,413.81210029)(242.73613403,413.68210297)
\curveto(242.74612642,413.56210054)(242.74612642,413.46710063)(242.73613403,413.39710297)
\lineto(242.73613403,413.32210297)
\curveto(242.71612645,413.26210084)(242.70112646,413.2021009)(242.69113403,413.14210297)
\curveto(242.69112647,413.09210101)(242.68612648,413.04210106)(242.67613403,412.99210297)
\curveto(242.60612656,412.69210141)(242.49612667,412.42710167)(242.34613403,412.19710297)
\curveto(242.18612698,411.95710214)(241.99112717,411.76210234)(241.76113403,411.61210297)
\curveto(241.53112763,411.46210264)(241.27112789,411.33210277)(240.98113403,411.22210297)
\curveto(240.87112829,411.17210293)(240.75112841,411.13710296)(240.62113403,411.11710297)
\curveto(240.50112866,411.097103)(240.38112878,411.07210303)(240.26113403,411.04210297)
\curveto(240.17112899,411.02210308)(240.07612909,411.01210309)(239.97613403,411.01210297)
\curveto(239.88612928,411.0021031)(239.79612937,410.98710311)(239.70613403,410.96710297)
\lineto(239.43613403,410.96710297)
\curveto(239.37612979,410.94710315)(239.27112989,410.93710316)(239.12113403,410.93710297)
\curveto(238.98113018,410.93710316)(238.88113028,410.94710315)(238.82113403,410.96710297)
\curveto(238.79113037,410.96710313)(238.75613041,410.97210313)(238.71613403,410.98210297)
\lineto(238.61113403,410.98210297)
\curveto(238.49113067,411.0021031)(238.37113079,411.01710308)(238.25113403,411.02710297)
\curveto(238.13113103,411.03710306)(238.01613115,411.05710304)(237.90613403,411.08710297)
\curveto(237.51613165,411.1971029)(237.17113199,411.32210278)(236.87113403,411.46210297)
\curveto(236.57113259,411.61210249)(236.31613285,411.83210227)(236.10613403,412.12210297)
\curveto(235.9661332,412.31210179)(235.84613332,412.53210157)(235.74613403,412.78210297)
\curveto(235.72613344,412.84210126)(235.70613346,412.92210118)(235.68613403,413.02210297)
\curveto(235.6661335,413.07210103)(235.65113351,413.14210096)(235.64113403,413.23210297)
\curveto(235.63113353,413.32210078)(235.63613353,413.3971007)(235.65613403,413.45710297)
\curveto(235.68613348,413.52710057)(235.73613343,413.57710052)(235.80613403,413.60710297)
\curveto(235.85613331,413.62710047)(235.91613325,413.63710046)(235.98613403,413.63710297)
\lineto(236.21113403,413.63710297)
\lineto(236.91613403,413.63710297)
\lineto(237.15613403,413.63710297)
\curveto(237.23613193,413.63710046)(237.30613186,413.62710047)(237.36613403,413.60710297)
\curveto(237.47613169,413.56710053)(237.54613162,413.5021006)(237.57613403,413.41210297)
\curveto(237.61613155,413.32210078)(237.6611315,413.22710087)(237.71113403,413.12710297)
\curveto(237.73113143,413.07710102)(237.7661314,413.01210109)(237.81613403,412.93210297)
\curveto(237.87613129,412.85210125)(237.92613124,412.8021013)(237.96613403,412.78210297)
\curveto(238.08613108,412.68210142)(238.20113096,412.6021015)(238.31113403,412.54210297)
\curveto(238.42113074,412.49210161)(238.5611306,412.44210166)(238.73113403,412.39210297)
\curveto(238.78113038,412.37210173)(238.83113033,412.36210174)(238.88113403,412.36210297)
\curveto(238.93113023,412.37210173)(238.98113018,412.37210173)(239.03113403,412.36210297)
\curveto(239.11113005,412.34210176)(239.19612997,412.33210177)(239.28613403,412.33210297)
\curveto(239.38612978,412.34210176)(239.47112969,412.35710174)(239.54113403,412.37710297)
\curveto(239.59112957,412.38710171)(239.63612953,412.39210171)(239.67613403,412.39210297)
\curveto(239.72612944,412.39210171)(239.77612939,412.4021017)(239.82613403,412.42210297)
\curveto(239.9661292,412.47210163)(240.09112907,412.53210157)(240.20113403,412.60210297)
\curveto(240.32112884,412.67210143)(240.41612875,412.76210134)(240.48613403,412.87210297)
\curveto(240.53612863,412.95210115)(240.57612859,413.07710102)(240.60613403,413.24710297)
\curveto(240.62612854,413.31710078)(240.62612854,413.38210072)(240.60613403,413.44210297)
\curveto(240.58612858,413.5021006)(240.5661286,413.55210055)(240.54613403,413.59210297)
\curveto(240.47612869,413.73210037)(240.38612878,413.83710026)(240.27613403,413.90710297)
\curveto(240.17612899,413.97710012)(240.05612911,414.04210006)(239.91613403,414.10210297)
\curveto(239.72612944,414.18209992)(239.52612964,414.24709985)(239.31613403,414.29710297)
\curveto(239.10613006,414.34709975)(238.89613027,414.4020997)(238.68613403,414.46210297)
\curveto(238.60613056,414.48209962)(238.52113064,414.4970996)(238.43113403,414.50710297)
\curveto(238.35113081,414.51709958)(238.27113089,414.53209957)(238.19113403,414.55210297)
\curveto(237.87113129,414.64209946)(237.5661316,414.72709937)(237.27613403,414.80710297)
\curveto(236.98613218,414.8970992)(236.72113244,415.02709907)(236.48113403,415.19710297)
\curveto(236.20113296,415.3970987)(235.99613317,415.66709843)(235.86613403,416.00710297)
\curveto(235.84613332,416.07709802)(235.82613334,416.17209793)(235.80613403,416.29210297)
\curveto(235.78613338,416.36209774)(235.77113339,416.44709765)(235.76113403,416.54710297)
\curveto(235.75113341,416.64709745)(235.75613341,416.73709736)(235.77613403,416.81710297)
\curveto(235.79613337,416.86709723)(235.80113336,416.90709719)(235.79113403,416.93710297)
\curveto(235.78113338,416.97709712)(235.78613338,417.02209708)(235.80613403,417.07210297)
\curveto(235.82613334,417.18209692)(235.84613332,417.28209682)(235.86613403,417.37210297)
\curveto(235.89613327,417.47209663)(235.93113323,417.56709653)(235.97113403,417.65710297)
\curveto(236.10113306,417.94709615)(236.28113288,418.18209592)(236.51113403,418.36210297)
\curveto(236.74113242,418.54209556)(237.00113216,418.68709541)(237.29113403,418.79710297)
\curveto(237.40113176,418.84709525)(237.51613165,418.88209522)(237.63613403,418.90210297)
\curveto(237.75613141,418.93209517)(237.88113128,418.96209514)(238.01113403,418.99210297)
\curveto(238.07113109,419.01209509)(238.13113103,419.02209508)(238.19113403,419.02210297)
\lineto(238.37113403,419.05210297)
\curveto(238.45113071,419.06209504)(238.53613063,419.06709503)(238.62613403,419.06710297)
\curveto(238.71613045,419.06709503)(238.80113036,419.07209503)(238.88113403,419.08210297)
}
}
{
\newrgbcolor{curcolor}{0 0 0}
\pscustom[linestyle=none,fillstyle=solid,fillcolor=curcolor]
{
}
}
{
\newrgbcolor{curcolor}{0 0 0}
\pscustom[linestyle=none,fillstyle=solid,fillcolor=curcolor]
{
\newpath
\moveto(255.71793091,411.98710297)
\lineto(255.71793091,411.56710297)
\curveto(255.71792254,411.43710266)(255.68792257,411.33210277)(255.62793091,411.25210297)
\curveto(255.57792268,411.2021029)(255.51292274,411.16710293)(255.43293091,411.14710297)
\curveto(255.3529229,411.13710296)(255.26292299,411.13210297)(255.16293091,411.13210297)
\lineto(254.33793091,411.13210297)
\lineto(254.05293091,411.13210297)
\curveto(253.97292428,411.14210296)(253.90792435,411.16710293)(253.85793091,411.20710297)
\curveto(253.78792447,411.25710284)(253.74792451,411.32210278)(253.73793091,411.40210297)
\curveto(253.72792453,411.48210262)(253.70792455,411.56210254)(253.67793091,411.64210297)
\curveto(253.6579246,411.66210244)(253.63792462,411.67710242)(253.61793091,411.68710297)
\curveto(253.60792465,411.70710239)(253.59292466,411.72710237)(253.57293091,411.74710297)
\curveto(253.46292479,411.74710235)(253.38292487,411.72210238)(253.33293091,411.67210297)
\lineto(253.18293091,411.52210297)
\curveto(253.11292514,411.47210263)(253.04792521,411.42710267)(252.98793091,411.38710297)
\curveto(252.92792533,411.35710274)(252.86292539,411.31710278)(252.79293091,411.26710297)
\curveto(252.7529255,411.24710285)(252.70792555,411.22710287)(252.65793091,411.20710297)
\curveto(252.61792564,411.18710291)(252.57292568,411.16710293)(252.52293091,411.14710297)
\curveto(252.38292587,411.097103)(252.23292602,411.05210305)(252.07293091,411.01210297)
\curveto(252.02292623,410.99210311)(251.97792628,410.98210312)(251.93793091,410.98210297)
\curveto(251.89792636,410.98210312)(251.8579264,410.97710312)(251.81793091,410.96710297)
\lineto(251.68293091,410.96710297)
\curveto(251.6529266,410.95710314)(251.61292664,410.95210315)(251.56293091,410.95210297)
\lineto(251.42793091,410.95210297)
\curveto(251.36792689,410.93210317)(251.27792698,410.92710317)(251.15793091,410.93710297)
\curveto(251.03792722,410.93710316)(250.9529273,410.94710315)(250.90293091,410.96710297)
\curveto(250.83292742,410.98710311)(250.76792749,410.9971031)(250.70793091,410.99710297)
\curveto(250.6579276,410.98710311)(250.60292765,410.99210311)(250.54293091,411.01210297)
\lineto(250.18293091,411.13210297)
\curveto(250.07292818,411.16210294)(249.96292829,411.2021029)(249.85293091,411.25210297)
\curveto(249.50292875,411.4021027)(249.18792907,411.63210247)(248.90793091,411.94210297)
\curveto(248.63792962,412.26210184)(248.42292983,412.5971015)(248.26293091,412.94710297)
\curveto(248.21293004,413.05710104)(248.17293008,413.16210094)(248.14293091,413.26210297)
\curveto(248.11293014,413.37210073)(248.07793018,413.48210062)(248.03793091,413.59210297)
\curveto(248.02793023,413.63210047)(248.02293023,413.66710043)(248.02293091,413.69710297)
\curveto(248.02293023,413.73710036)(248.01293024,413.78210032)(247.99293091,413.83210297)
\curveto(247.97293028,413.91210019)(247.9529303,413.9971001)(247.93293091,414.08710297)
\curveto(247.92293033,414.18709991)(247.90793035,414.28709981)(247.88793091,414.38710297)
\curveto(247.87793038,414.41709968)(247.87293038,414.45209965)(247.87293091,414.49210297)
\curveto(247.88293037,414.53209957)(247.88293037,414.56709953)(247.87293091,414.59710297)
\lineto(247.87293091,414.73210297)
\curveto(247.87293038,414.78209932)(247.86793039,414.83209927)(247.85793091,414.88210297)
\curveto(247.84793041,414.93209917)(247.84293041,414.98709911)(247.84293091,415.04710297)
\curveto(247.84293041,415.11709898)(247.84793041,415.17209893)(247.85793091,415.21210297)
\curveto(247.86793039,415.26209884)(247.87293038,415.30709879)(247.87293091,415.34710297)
\lineto(247.87293091,415.49710297)
\curveto(247.88293037,415.54709855)(247.88293037,415.59209851)(247.87293091,415.63210297)
\curveto(247.87293038,415.68209842)(247.88293037,415.73209837)(247.90293091,415.78210297)
\curveto(247.92293033,415.89209821)(247.93793032,415.9970981)(247.94793091,416.09710297)
\curveto(247.96793029,416.1970979)(247.99293026,416.2970978)(248.02293091,416.39710297)
\curveto(248.06293019,416.51709758)(248.09793016,416.63209747)(248.12793091,416.74210297)
\curveto(248.1579301,416.85209725)(248.19793006,416.96209714)(248.24793091,417.07210297)
\curveto(248.38792987,417.37209673)(248.56292969,417.65709644)(248.77293091,417.92710297)
\curveto(248.79292946,417.95709614)(248.81792944,417.98209612)(248.84793091,418.00210297)
\curveto(248.88792937,418.03209607)(248.91792934,418.06209604)(248.93793091,418.09210297)
\curveto(248.97792928,418.14209596)(249.01792924,418.18709591)(249.05793091,418.22710297)
\curveto(249.09792916,418.26709583)(249.14292911,418.30709579)(249.19293091,418.34710297)
\curveto(249.23292902,418.36709573)(249.26792899,418.39209571)(249.29793091,418.42210297)
\curveto(249.32792893,418.46209564)(249.36292889,418.49209561)(249.40293091,418.51210297)
\curveto(249.6529286,418.68209542)(249.94292831,418.82209528)(250.27293091,418.93210297)
\curveto(250.34292791,418.95209515)(250.41292784,418.96709513)(250.48293091,418.97710297)
\curveto(250.56292769,418.98709511)(250.64292761,419.0020951)(250.72293091,419.02210297)
\curveto(250.79292746,419.04209506)(250.88292737,419.05209505)(250.99293091,419.05210297)
\curveto(251.10292715,419.06209504)(251.21292704,419.06709503)(251.32293091,419.06710297)
\curveto(251.43292682,419.06709503)(251.53792672,419.06209504)(251.63793091,419.05210297)
\curveto(251.74792651,419.04209506)(251.83792642,419.02709507)(251.90793091,419.00710297)
\curveto(252.0579262,418.95709514)(252.20292605,418.91209519)(252.34293091,418.87210297)
\curveto(252.48292577,418.83209527)(252.61292564,418.77709532)(252.73293091,418.70710297)
\curveto(252.80292545,418.65709544)(252.86792539,418.60709549)(252.92793091,418.55710297)
\curveto(252.98792527,418.51709558)(253.0529252,418.47209563)(253.12293091,418.42210297)
\curveto(253.16292509,418.39209571)(253.21792504,418.35209575)(253.28793091,418.30210297)
\curveto(253.36792489,418.25209585)(253.44292481,418.25209585)(253.51293091,418.30210297)
\curveto(253.5529247,418.32209578)(253.57292468,418.35709574)(253.57293091,418.40710297)
\curveto(253.57292468,418.45709564)(253.58292467,418.50709559)(253.60293091,418.55710297)
\lineto(253.60293091,418.70710297)
\curveto(253.61292464,418.73709536)(253.61792464,418.77209533)(253.61793091,418.81210297)
\lineto(253.61793091,418.93210297)
\lineto(253.61793091,420.97210297)
\curveto(253.61792464,421.08209302)(253.61292464,421.2020929)(253.60293091,421.33210297)
\curveto(253.60292465,421.47209263)(253.62792463,421.57709252)(253.67793091,421.64710297)
\curveto(253.71792454,421.72709237)(253.79292446,421.77709232)(253.90293091,421.79710297)
\curveto(253.92292433,421.80709229)(253.94292431,421.80709229)(253.96293091,421.79710297)
\curveto(253.98292427,421.7970923)(254.00292425,421.8020923)(254.02293091,421.81210297)
\lineto(255.08793091,421.81210297)
\curveto(255.20792305,421.81209229)(255.31792294,421.80709229)(255.41793091,421.79710297)
\curveto(255.51792274,421.78709231)(255.59292266,421.74709235)(255.64293091,421.67710297)
\curveto(255.69292256,421.5970925)(255.71792254,421.49209261)(255.71793091,421.36210297)
\lineto(255.71793091,421.00210297)
\lineto(255.71793091,411.98710297)
\moveto(253.67793091,414.92710297)
\curveto(253.68792457,414.96709913)(253.68792457,415.00709909)(253.67793091,415.04710297)
\lineto(253.67793091,415.18210297)
\curveto(253.67792458,415.28209882)(253.67292458,415.38209872)(253.66293091,415.48210297)
\curveto(253.6529246,415.58209852)(253.63792462,415.67209843)(253.61793091,415.75210297)
\curveto(253.59792466,415.86209824)(253.57792468,415.96209814)(253.55793091,416.05210297)
\curveto(253.54792471,416.14209796)(253.52292473,416.22709787)(253.48293091,416.30710297)
\curveto(253.34292491,416.66709743)(253.13792512,416.95209715)(252.86793091,417.16210297)
\curveto(252.60792565,417.37209673)(252.22792603,417.47709662)(251.72793091,417.47710297)
\curveto(251.66792659,417.47709662)(251.58792667,417.46709663)(251.48793091,417.44710297)
\curveto(251.40792685,417.42709667)(251.33292692,417.40709669)(251.26293091,417.38710297)
\curveto(251.20292705,417.37709672)(251.14292711,417.35709674)(251.08293091,417.32710297)
\curveto(250.81292744,417.21709688)(250.60292765,417.04709705)(250.45293091,416.81710297)
\curveto(250.30292795,416.58709751)(250.18292807,416.32709777)(250.09293091,416.03710297)
\curveto(250.06292819,415.93709816)(250.04292821,415.83709826)(250.03293091,415.73710297)
\curveto(250.02292823,415.63709846)(250.00292825,415.53209857)(249.97293091,415.42210297)
\lineto(249.97293091,415.21210297)
\curveto(249.9529283,415.12209898)(249.94792831,414.9970991)(249.95793091,414.83710297)
\curveto(249.96792829,414.68709941)(249.98292827,414.57709952)(250.00293091,414.50710297)
\lineto(250.00293091,414.41710297)
\curveto(250.01292824,414.3970997)(250.01792824,414.37709972)(250.01793091,414.35710297)
\curveto(250.03792822,414.27709982)(250.0529282,414.2020999)(250.06293091,414.13210297)
\curveto(250.08292817,414.06210004)(250.10292815,413.98710011)(250.12293091,413.90710297)
\curveto(250.29292796,413.38710071)(250.58292767,413.0021011)(250.99293091,412.75210297)
\curveto(251.12292713,412.66210144)(251.30292695,412.59210151)(251.53293091,412.54210297)
\curveto(251.57292668,412.53210157)(251.63292662,412.52710157)(251.71293091,412.52710297)
\curveto(251.74292651,412.51710158)(251.78792647,412.50710159)(251.84793091,412.49710297)
\curveto(251.91792634,412.4971016)(251.97292628,412.5021016)(252.01293091,412.51210297)
\curveto(252.09292616,412.53210157)(252.17292608,412.54710155)(252.25293091,412.55710297)
\curveto(252.33292592,412.56710153)(252.41292584,412.58710151)(252.49293091,412.61710297)
\curveto(252.74292551,412.72710137)(252.94292531,412.86710123)(253.09293091,413.03710297)
\curveto(253.24292501,413.20710089)(253.37292488,413.42210068)(253.48293091,413.68210297)
\curveto(253.52292473,413.77210033)(253.5529247,413.86210024)(253.57293091,413.95210297)
\curveto(253.59292466,414.05210005)(253.61292464,414.15709994)(253.63293091,414.26710297)
\curveto(253.64292461,414.31709978)(253.64292461,414.36209974)(253.63293091,414.40210297)
\curveto(253.63292462,414.45209965)(253.64292461,414.5020996)(253.66293091,414.55210297)
\curveto(253.67292458,414.58209952)(253.67792458,414.61709948)(253.67793091,414.65710297)
\lineto(253.67793091,414.79210297)
\lineto(253.67793091,414.92710297)
}
}
{
\newrgbcolor{curcolor}{0 0 0}
\pscustom[linestyle=none,fillstyle=solid,fillcolor=curcolor]
{
\newpath
\moveto(264.66285278,415.07710297)
\curveto(264.68284462,414.9970991)(264.68284462,414.90709919)(264.66285278,414.80710297)
\curveto(264.64284466,414.70709939)(264.60784469,414.64209946)(264.55785278,414.61210297)
\curveto(264.50784479,414.57209953)(264.43284487,414.54209956)(264.33285278,414.52210297)
\curveto(264.24284506,414.51209959)(264.13784516,414.5020996)(264.01785278,414.49210297)
\lineto(263.67285278,414.49210297)
\curveto(263.56284574,414.5020996)(263.46284584,414.50709959)(263.37285278,414.50710297)
\lineto(259.71285278,414.50710297)
\lineto(259.50285278,414.50710297)
\curveto(259.44284986,414.50709959)(259.38784991,414.4970996)(259.33785278,414.47710297)
\curveto(259.25785004,414.43709966)(259.20785009,414.3970997)(259.18785278,414.35710297)
\curveto(259.16785013,414.33709976)(259.14785015,414.2970998)(259.12785278,414.23710297)
\curveto(259.10785019,414.18709991)(259.1028502,414.13709996)(259.11285278,414.08710297)
\curveto(259.13285017,414.02710007)(259.14285016,413.96710013)(259.14285278,413.90710297)
\curveto(259.15285015,413.85710024)(259.16785013,413.8021003)(259.18785278,413.74210297)
\curveto(259.26785003,413.5021006)(259.36284994,413.3021008)(259.47285278,413.14210297)
\curveto(259.59284971,412.99210111)(259.75284955,412.85710124)(259.95285278,412.73710297)
\curveto(260.03284927,412.68710141)(260.11284919,412.65210145)(260.19285278,412.63210297)
\curveto(260.28284902,412.62210148)(260.37284893,412.6021015)(260.46285278,412.57210297)
\curveto(260.54284876,412.55210155)(260.65284865,412.53710156)(260.79285278,412.52710297)
\curveto(260.93284837,412.51710158)(261.05284825,412.52210158)(261.15285278,412.54210297)
\lineto(261.28785278,412.54210297)
\curveto(261.38784791,412.56210154)(261.47784782,412.58210152)(261.55785278,412.60210297)
\curveto(261.64784765,412.63210147)(261.73284757,412.66210144)(261.81285278,412.69210297)
\curveto(261.91284739,412.74210136)(262.02284728,412.80710129)(262.14285278,412.88710297)
\curveto(262.27284703,412.96710113)(262.36784693,413.04710105)(262.42785278,413.12710297)
\curveto(262.47784682,413.1971009)(262.52784677,413.26210084)(262.57785278,413.32210297)
\curveto(262.63784666,413.39210071)(262.70784659,413.44210066)(262.78785278,413.47210297)
\curveto(262.88784641,413.52210058)(263.01284629,413.54210056)(263.16285278,413.53210297)
\lineto(263.59785278,413.53210297)
\lineto(263.77785278,413.53210297)
\curveto(263.84784545,413.54210056)(263.90784539,413.53710056)(263.95785278,413.51710297)
\lineto(264.10785278,413.51710297)
\curveto(264.20784509,413.4971006)(264.27784502,413.47210063)(264.31785278,413.44210297)
\curveto(264.35784494,413.42210068)(264.37784492,413.37710072)(264.37785278,413.30710297)
\curveto(264.38784491,413.23710086)(264.38284492,413.17710092)(264.36285278,413.12710297)
\curveto(264.31284499,412.98710111)(264.25784504,412.86210124)(264.19785278,412.75210297)
\curveto(264.13784516,412.64210146)(264.06784523,412.53210157)(263.98785278,412.42210297)
\curveto(263.76784553,412.09210201)(263.51784578,411.82710227)(263.23785278,411.62710297)
\curveto(262.95784634,411.42710267)(262.60784669,411.25710284)(262.18785278,411.11710297)
\curveto(262.07784722,411.07710302)(261.96784733,411.05210305)(261.85785278,411.04210297)
\curveto(261.74784755,411.03210307)(261.63284767,411.01210309)(261.51285278,410.98210297)
\curveto(261.47284783,410.97210313)(261.42784787,410.97210313)(261.37785278,410.98210297)
\curveto(261.33784796,410.98210312)(261.297848,410.97710312)(261.25785278,410.96710297)
\lineto(261.09285278,410.96710297)
\curveto(261.04284826,410.94710315)(260.98284832,410.94210316)(260.91285278,410.95210297)
\curveto(260.85284845,410.95210315)(260.7978485,410.95710314)(260.74785278,410.96710297)
\curveto(260.66784863,410.97710312)(260.5978487,410.97710312)(260.53785278,410.96710297)
\curveto(260.47784882,410.95710314)(260.41284889,410.96210314)(260.34285278,410.98210297)
\curveto(260.29284901,411.0021031)(260.23784906,411.01210309)(260.17785278,411.01210297)
\curveto(260.11784918,411.01210309)(260.06284924,411.02210308)(260.01285278,411.04210297)
\curveto(259.9028494,411.06210304)(259.79284951,411.08710301)(259.68285278,411.11710297)
\curveto(259.57284973,411.13710296)(259.47284983,411.17210293)(259.38285278,411.22210297)
\curveto(259.27285003,411.26210284)(259.16785013,411.2971028)(259.06785278,411.32710297)
\curveto(258.97785032,411.36710273)(258.89285041,411.41210269)(258.81285278,411.46210297)
\curveto(258.49285081,411.66210244)(258.20785109,411.89210221)(257.95785278,412.15210297)
\curveto(257.70785159,412.42210168)(257.5028518,412.73210137)(257.34285278,413.08210297)
\curveto(257.29285201,413.19210091)(257.25285205,413.3021008)(257.22285278,413.41210297)
\curveto(257.19285211,413.53210057)(257.15285215,413.65210045)(257.10285278,413.77210297)
\curveto(257.09285221,413.81210029)(257.08785221,413.84710025)(257.08785278,413.87710297)
\curveto(257.08785221,413.91710018)(257.08285222,413.95710014)(257.07285278,413.99710297)
\curveto(257.03285227,414.11709998)(257.00785229,414.24709985)(256.99785278,414.38710297)
\lineto(256.96785278,414.80710297)
\curveto(256.96785233,414.85709924)(256.96285234,414.91209919)(256.95285278,414.97210297)
\curveto(256.95285235,415.03209907)(256.95785234,415.08709901)(256.96785278,415.13710297)
\lineto(256.96785278,415.31710297)
\lineto(257.01285278,415.67710297)
\curveto(257.05285225,415.84709825)(257.08785221,416.01209809)(257.11785278,416.17210297)
\curveto(257.14785215,416.33209777)(257.19285211,416.48209762)(257.25285278,416.62210297)
\curveto(257.68285162,417.66209644)(258.41285089,418.3970957)(259.44285278,418.82710297)
\curveto(259.58284972,418.88709521)(259.72284958,418.92709517)(259.86285278,418.94710297)
\curveto(260.01284929,418.97709512)(260.16784913,419.01209509)(260.32785278,419.05210297)
\curveto(260.40784889,419.06209504)(260.48284882,419.06709503)(260.55285278,419.06710297)
\curveto(260.62284868,419.06709503)(260.6978486,419.07209503)(260.77785278,419.08210297)
\curveto(261.28784801,419.09209501)(261.72284758,419.03209507)(262.08285278,418.90210297)
\curveto(262.45284685,418.78209532)(262.78284652,418.62209548)(263.07285278,418.42210297)
\curveto(263.16284614,418.36209574)(263.25284605,418.29209581)(263.34285278,418.21210297)
\curveto(263.43284587,418.14209596)(263.51284579,418.06709603)(263.58285278,417.98710297)
\curveto(263.61284569,417.93709616)(263.65284565,417.8970962)(263.70285278,417.86710297)
\curveto(263.78284552,417.75709634)(263.85784544,417.64209646)(263.92785278,417.52210297)
\curveto(263.9978453,417.41209669)(264.07284523,417.2970968)(264.15285278,417.17710297)
\curveto(264.2028451,417.08709701)(264.24284506,416.99209711)(264.27285278,416.89210297)
\curveto(264.31284499,416.8020973)(264.35284495,416.7020974)(264.39285278,416.59210297)
\curveto(264.44284486,416.46209764)(264.48284482,416.32709777)(264.51285278,416.18710297)
\curveto(264.54284476,416.04709805)(264.57784472,415.90709819)(264.61785278,415.76710297)
\curveto(264.63784466,415.68709841)(264.64284466,415.5970985)(264.63285278,415.49710297)
\curveto(264.63284467,415.40709869)(264.64284466,415.32209878)(264.66285278,415.24210297)
\lineto(264.66285278,415.07710297)
\moveto(262.41285278,415.96210297)
\curveto(262.48284682,416.06209804)(262.48784681,416.18209792)(262.42785278,416.32210297)
\curveto(262.37784692,416.47209763)(262.33784696,416.58209752)(262.30785278,416.65210297)
\curveto(262.16784713,416.92209718)(261.98284732,417.12709697)(261.75285278,417.26710297)
\curveto(261.52284778,417.41709668)(261.2028481,417.4970966)(260.79285278,417.50710297)
\curveto(260.76284854,417.48709661)(260.72784857,417.48209662)(260.68785278,417.49210297)
\curveto(260.64784865,417.5020966)(260.61284869,417.5020966)(260.58285278,417.49210297)
\curveto(260.53284877,417.47209663)(260.47784882,417.45709664)(260.41785278,417.44710297)
\curveto(260.35784894,417.44709665)(260.302849,417.43709666)(260.25285278,417.41710297)
\curveto(259.81284949,417.27709682)(259.48784981,417.0020971)(259.27785278,416.59210297)
\curveto(259.25785004,416.55209755)(259.23285007,416.4970976)(259.20285278,416.42710297)
\curveto(259.18285012,416.36709773)(259.16785013,416.3020978)(259.15785278,416.23210297)
\curveto(259.14785015,416.17209793)(259.14785015,416.11209799)(259.15785278,416.05210297)
\curveto(259.17785012,415.99209811)(259.21285009,415.94209816)(259.26285278,415.90210297)
\curveto(259.34284996,415.85209825)(259.45284985,415.82709827)(259.59285278,415.82710297)
\lineto(259.99785278,415.82710297)
\lineto(261.66285278,415.82710297)
\lineto(262.09785278,415.82710297)
\curveto(262.25784704,415.83709826)(262.36284694,415.88209822)(262.41285278,415.96210297)
}
}
{
\newrgbcolor{curcolor}{0 0 0}
\pscustom[linestyle=none,fillstyle=solid,fillcolor=curcolor]
{
}
}
{
\newrgbcolor{curcolor}{0 0 0}
\pscustom[linestyle=none,fillstyle=solid,fillcolor=curcolor]
{
\newpath
\moveto(272.17129028,421.72210297)
\curveto(272.24128733,421.64209246)(272.2762873,421.52209258)(272.27629028,421.36210297)
\lineto(272.27629028,420.89710297)
\lineto(272.27629028,420.49210297)
\curveto(272.2762873,420.35209375)(272.24128733,420.25709384)(272.17129028,420.20710297)
\curveto(272.11128746,420.15709394)(272.03128754,420.12709397)(271.93129028,420.11710297)
\curveto(271.84128773,420.10709399)(271.74128783,420.102094)(271.63129028,420.10210297)
\lineto(270.79129028,420.10210297)
\curveto(270.68128889,420.102094)(270.58128899,420.10709399)(270.49129028,420.11710297)
\curveto(270.41128916,420.12709397)(270.34128923,420.15709394)(270.28129028,420.20710297)
\curveto(270.24128933,420.23709386)(270.21128936,420.29209381)(270.19129028,420.37210297)
\curveto(270.18128939,420.46209364)(270.1712894,420.55709354)(270.16129028,420.65710297)
\lineto(270.16129028,420.98710297)
\curveto(270.1712894,421.097093)(270.1762894,421.19209291)(270.17629028,421.27210297)
\lineto(270.17629028,421.48210297)
\curveto(270.18628939,421.55209255)(270.20628937,421.61209249)(270.23629028,421.66210297)
\curveto(270.25628932,421.7020924)(270.28128929,421.73209237)(270.31129028,421.75210297)
\lineto(270.43129028,421.81210297)
\curveto(270.45128912,421.81209229)(270.4762891,421.81209229)(270.50629028,421.81210297)
\curveto(270.53628904,421.82209228)(270.56128901,421.82709227)(270.58129028,421.82710297)
\lineto(271.67629028,421.82710297)
\curveto(271.7762878,421.82709227)(271.8712877,421.82209228)(271.96129028,421.81210297)
\curveto(272.05128752,421.8020923)(272.12128745,421.77209233)(272.17129028,421.72210297)
\moveto(272.27629028,411.95710297)
\curveto(272.2762873,411.75710234)(272.2712873,411.58710251)(272.26129028,411.44710297)
\curveto(272.25128732,411.30710279)(272.16128741,411.21210289)(271.99129028,411.16210297)
\curveto(271.93128764,411.14210296)(271.86628771,411.13210297)(271.79629028,411.13210297)
\curveto(271.72628785,411.14210296)(271.65128792,411.14710295)(271.57129028,411.14710297)
\lineto(270.73129028,411.14710297)
\curveto(270.64128893,411.14710295)(270.55128902,411.15210295)(270.46129028,411.16210297)
\curveto(270.38128919,411.17210293)(270.32128925,411.2021029)(270.28129028,411.25210297)
\curveto(270.22128935,411.32210278)(270.18628939,411.40710269)(270.17629028,411.50710297)
\lineto(270.17629028,411.85210297)
\lineto(270.17629028,418.18210297)
\lineto(270.17629028,418.48210297)
\curveto(270.1762894,418.58209552)(270.19628938,418.66209544)(270.23629028,418.72210297)
\curveto(270.29628928,418.79209531)(270.38128919,418.83709526)(270.49129028,418.85710297)
\curveto(270.51128906,418.86709523)(270.53628904,418.86709523)(270.56629028,418.85710297)
\curveto(270.60628897,418.85709524)(270.63628894,418.86209524)(270.65629028,418.87210297)
\lineto(271.40629028,418.87210297)
\lineto(271.60129028,418.87210297)
\curveto(271.68128789,418.88209522)(271.74628783,418.88209522)(271.79629028,418.87210297)
\lineto(271.91629028,418.87210297)
\curveto(271.9762876,418.85209525)(272.03128754,418.83709526)(272.08129028,418.82710297)
\curveto(272.13128744,418.81709528)(272.1712874,418.78709531)(272.20129028,418.73710297)
\curveto(272.24128733,418.68709541)(272.26128731,418.61709548)(272.26129028,418.52710297)
\curveto(272.2712873,418.43709566)(272.2762873,418.34209576)(272.27629028,418.24210297)
\lineto(272.27629028,411.95710297)
}
}
{
\newrgbcolor{curcolor}{0 0 0}
\pscustom[linestyle=none,fillstyle=solid,fillcolor=curcolor]
{
\newpath
\moveto(278.36347778,419.06710297)
\curveto(278.96347198,419.08709501)(279.46347148,419.0020951)(279.86347778,418.81210297)
\curveto(280.26347068,418.62209548)(280.57847036,418.34209576)(280.80847778,417.97210297)
\curveto(280.87847006,417.86209624)(280.93347001,417.74209636)(280.97347778,417.61210297)
\curveto(281.01346993,417.49209661)(281.05346989,417.36709673)(281.09347778,417.23710297)
\curveto(281.11346983,417.15709694)(281.12346982,417.08209702)(281.12347778,417.01210297)
\curveto(281.13346981,416.94209716)(281.14846979,416.87209723)(281.16847778,416.80210297)
\curveto(281.16846977,416.74209736)(281.17346977,416.7020974)(281.18347778,416.68210297)
\curveto(281.20346974,416.54209756)(281.21346973,416.3970977)(281.21347778,416.24710297)
\lineto(281.21347778,415.81210297)
\lineto(281.21347778,414.47710297)
\lineto(281.21347778,412.04710297)
\curveto(281.21346973,411.85710224)(281.20846973,411.67210243)(281.19847778,411.49210297)
\curveto(281.19846974,411.32210278)(281.12846981,411.21210289)(280.98847778,411.16210297)
\curveto(280.92847001,411.14210296)(280.85847008,411.13210297)(280.77847778,411.13210297)
\lineto(280.53847778,411.13210297)
\lineto(279.72847778,411.13210297)
\curveto(279.60847133,411.13210297)(279.49847144,411.13710296)(279.39847778,411.14710297)
\curveto(279.30847163,411.16710293)(279.2384717,411.21210289)(279.18847778,411.28210297)
\curveto(279.14847179,411.34210276)(279.12347182,411.41710268)(279.11347778,411.50710297)
\lineto(279.11347778,411.82210297)
\lineto(279.11347778,412.87210297)
\lineto(279.11347778,415.10710297)
\curveto(279.11347183,415.47709862)(279.09847184,415.81709828)(279.06847778,416.12710297)
\curveto(279.0384719,416.44709765)(278.94847199,416.71709738)(278.79847778,416.93710297)
\curveto(278.65847228,417.13709696)(278.45347249,417.27709682)(278.18347778,417.35710297)
\curveto(278.13347281,417.37709672)(278.07847286,417.38709671)(278.01847778,417.38710297)
\curveto(277.96847297,417.38709671)(277.91347303,417.3970967)(277.85347778,417.41710297)
\curveto(277.80347314,417.42709667)(277.7384732,417.42709667)(277.65847778,417.41710297)
\curveto(277.58847335,417.41709668)(277.53347341,417.41209669)(277.49347778,417.40210297)
\curveto(277.45347349,417.39209671)(277.41847352,417.38709671)(277.38847778,417.38710297)
\curveto(277.35847358,417.38709671)(277.32847361,417.38209672)(277.29847778,417.37210297)
\curveto(277.06847387,417.31209679)(276.88347406,417.23209687)(276.74347778,417.13210297)
\curveto(276.42347452,416.9020972)(276.23347471,416.56709753)(276.17347778,416.12710297)
\curveto(276.11347483,415.68709841)(276.08347486,415.19209891)(276.08347778,414.64210297)
\lineto(276.08347778,412.76710297)
\lineto(276.08347778,411.85210297)
\lineto(276.08347778,411.58210297)
\curveto(276.08347486,411.49210261)(276.06847487,411.41710268)(276.03847778,411.35710297)
\curveto(275.98847495,411.24710285)(275.90847503,411.18210292)(275.79847778,411.16210297)
\curveto(275.68847525,411.14210296)(275.55347539,411.13210297)(275.39347778,411.13210297)
\lineto(274.64347778,411.13210297)
\curveto(274.53347641,411.13210297)(274.42347652,411.13710296)(274.31347778,411.14710297)
\curveto(274.20347674,411.15710294)(274.12347682,411.19210291)(274.07347778,411.25210297)
\curveto(274.00347694,411.34210276)(273.96847697,411.47210263)(273.96847778,411.64210297)
\curveto(273.97847696,411.81210229)(273.98347696,411.97210213)(273.98347778,412.12210297)
\lineto(273.98347778,414.16210297)
\lineto(273.98347778,417.46210297)
\lineto(273.98347778,418.22710297)
\lineto(273.98347778,418.52710297)
\curveto(273.99347695,418.61709548)(274.02347692,418.69209541)(274.07347778,418.75210297)
\curveto(274.09347685,418.78209532)(274.12347682,418.8020953)(274.16347778,418.81210297)
\curveto(274.21347673,418.83209527)(274.26347668,418.84709525)(274.31347778,418.85710297)
\lineto(274.38847778,418.85710297)
\curveto(274.4384765,418.86709523)(274.48847645,418.87209523)(274.53847778,418.87210297)
\lineto(274.70347778,418.87210297)
\lineto(275.33347778,418.87210297)
\curveto(275.41347553,418.87209523)(275.48847545,418.86709523)(275.55847778,418.85710297)
\curveto(275.6384753,418.85709524)(275.70847523,418.84709525)(275.76847778,418.82710297)
\curveto(275.8384751,418.7970953)(275.88347506,418.75209535)(275.90347778,418.69210297)
\curveto(275.93347501,418.63209547)(275.95847498,418.56209554)(275.97847778,418.48210297)
\curveto(275.98847495,418.44209566)(275.98847495,418.40709569)(275.97847778,418.37710297)
\curveto(275.97847496,418.34709575)(275.98847495,418.31709578)(276.00847778,418.28710297)
\curveto(276.02847491,418.23709586)(276.0434749,418.20709589)(276.05347778,418.19710297)
\curveto(276.07347487,418.18709591)(276.09847484,418.17209593)(276.12847778,418.15210297)
\curveto(276.2384747,418.14209596)(276.32847461,418.17709592)(276.39847778,418.25710297)
\curveto(276.46847447,418.34709575)(276.5434744,418.41709568)(276.62347778,418.46710297)
\curveto(276.89347405,418.66709543)(277.19347375,418.82709527)(277.52347778,418.94710297)
\curveto(277.61347333,418.97709512)(277.70347324,418.9970951)(277.79347778,419.00710297)
\curveto(277.89347305,419.01709508)(277.99847294,419.03209507)(278.10847778,419.05210297)
\curveto(278.1384728,419.06209504)(278.18347276,419.06209504)(278.24347778,419.05210297)
\curveto(278.30347264,419.05209505)(278.3434726,419.05709504)(278.36347778,419.06710297)
}
}
{
\newrgbcolor{curcolor}{0 0 0}
\pscustom[linestyle=none,fillstyle=solid,fillcolor=curcolor]
{
\newpath
\moveto(283.89472778,421.18210297)
\lineto(284.89972778,421.18210297)
\curveto(285.0497248,421.18209292)(285.17972467,421.17209293)(285.28972778,421.15210297)
\curveto(285.40972444,421.14209296)(285.49472435,421.08209302)(285.54472778,420.97210297)
\curveto(285.56472428,420.92209318)(285.57472427,420.86209324)(285.57472778,420.79210297)
\lineto(285.57472778,420.58210297)
\lineto(285.57472778,419.90710297)
\curveto(285.57472427,419.85709424)(285.56972428,419.7970943)(285.55972778,419.72710297)
\curveto(285.55972429,419.66709443)(285.56472428,419.61209449)(285.57472778,419.56210297)
\lineto(285.57472778,419.39710297)
\curveto(285.57472427,419.31709478)(285.57972427,419.24209486)(285.58972778,419.17210297)
\curveto(285.59972425,419.11209499)(285.62472422,419.05709504)(285.66472778,419.00710297)
\curveto(285.73472411,418.91709518)(285.85972399,418.86709523)(286.03972778,418.85710297)
\lineto(286.57972778,418.85710297)
\lineto(286.75972778,418.85710297)
\curveto(286.81972303,418.85709524)(286.87472297,418.84709525)(286.92472778,418.82710297)
\curveto(287.03472281,418.77709532)(287.09472275,418.68709541)(287.10472778,418.55710297)
\curveto(287.12472272,418.42709567)(287.13472271,418.28209582)(287.13472778,418.12210297)
\lineto(287.13472778,417.91210297)
\curveto(287.1447227,417.84209626)(287.13972271,417.78209632)(287.11972778,417.73210297)
\curveto(287.06972278,417.57209653)(286.96472288,417.48709661)(286.80472778,417.47710297)
\curveto(286.6447232,417.46709663)(286.46472338,417.46209664)(286.26472778,417.46210297)
\lineto(286.12972778,417.46210297)
\curveto(286.08972376,417.47209663)(286.05472379,417.47209663)(286.02472778,417.46210297)
\curveto(285.98472386,417.45209665)(285.9497239,417.44709665)(285.91972778,417.44710297)
\curveto(285.88972396,417.45709664)(285.85972399,417.45209665)(285.82972778,417.43210297)
\curveto(285.7497241,417.41209669)(285.68972416,417.36709673)(285.64972778,417.29710297)
\curveto(285.61972423,417.23709686)(285.59472425,417.16209694)(285.57472778,417.07210297)
\curveto(285.56472428,417.02209708)(285.56472428,416.96709713)(285.57472778,416.90710297)
\curveto(285.58472426,416.84709725)(285.58472426,416.79209731)(285.57472778,416.74210297)
\lineto(285.57472778,415.81210297)
\lineto(285.57472778,414.05710297)
\curveto(285.57472427,413.80710029)(285.57972427,413.58710051)(285.58972778,413.39710297)
\curveto(285.60972424,413.21710088)(285.67472417,413.05710104)(285.78472778,412.91710297)
\curveto(285.83472401,412.85710124)(285.89972395,412.81210129)(285.97972778,412.78210297)
\lineto(286.24972778,412.72210297)
\curveto(286.27972357,412.71210139)(286.30972354,412.70710139)(286.33972778,412.70710297)
\curveto(286.37972347,412.71710138)(286.40972344,412.71710138)(286.42972778,412.70710297)
\lineto(286.59472778,412.70710297)
\curveto(286.70472314,412.70710139)(286.79972305,412.7021014)(286.87972778,412.69210297)
\curveto(286.95972289,412.68210142)(287.02472282,412.64210146)(287.07472778,412.57210297)
\curveto(287.11472273,412.51210159)(287.13472271,412.43210167)(287.13472778,412.33210297)
\lineto(287.13472778,412.04710297)
\curveto(287.13472271,411.83710226)(287.12972272,411.64210246)(287.11972778,411.46210297)
\curveto(287.11972273,411.29210281)(287.03972281,411.17710292)(286.87972778,411.11710297)
\curveto(286.82972302,411.097103)(286.78472306,411.09210301)(286.74472778,411.10210297)
\curveto(286.70472314,411.102103)(286.65972319,411.09210301)(286.60972778,411.07210297)
\lineto(286.45972778,411.07210297)
\curveto(286.43972341,411.07210303)(286.40972344,411.07710302)(286.36972778,411.08710297)
\curveto(286.32972352,411.08710301)(286.29472355,411.08210302)(286.26472778,411.07210297)
\curveto(286.21472363,411.06210304)(286.15972369,411.06210304)(286.09972778,411.07210297)
\lineto(285.94972778,411.07210297)
\lineto(285.79972778,411.07210297)
\curveto(285.7497241,411.06210304)(285.70472414,411.06210304)(285.66472778,411.07210297)
\lineto(285.49972778,411.07210297)
\curveto(285.4497244,411.08210302)(285.39472445,411.08710301)(285.33472778,411.08710297)
\curveto(285.27472457,411.08710301)(285.21972463,411.09210301)(285.16972778,411.10210297)
\curveto(285.09972475,411.11210299)(285.03472481,411.12210298)(284.97472778,411.13210297)
\lineto(284.79472778,411.16210297)
\curveto(284.68472516,411.19210291)(284.57972527,411.22710287)(284.47972778,411.26710297)
\curveto(284.37972547,411.30710279)(284.28472556,411.35210275)(284.19472778,411.40210297)
\lineto(284.10472778,411.46210297)
\curveto(284.07472577,411.49210261)(284.03972581,411.52210258)(283.99972778,411.55210297)
\curveto(283.97972587,411.57210253)(283.95472589,411.59210251)(283.92472778,411.61210297)
\lineto(283.84972778,411.68710297)
\curveto(283.70972614,411.87710222)(283.60472624,412.08710201)(283.53472778,412.31710297)
\curveto(283.51472633,412.35710174)(283.50472634,412.39210171)(283.50472778,412.42210297)
\curveto(283.51472633,412.46210164)(283.51472633,412.50710159)(283.50472778,412.55710297)
\curveto(283.49472635,412.57710152)(283.48972636,412.6021015)(283.48972778,412.63210297)
\curveto(283.48972636,412.66210144)(283.48472636,412.68710141)(283.47472778,412.70710297)
\lineto(283.47472778,412.85710297)
\curveto(283.46472638,412.8971012)(283.45972639,412.94210116)(283.45972778,412.99210297)
\curveto(283.46972638,413.04210106)(283.47472637,413.09210101)(283.47472778,413.14210297)
\lineto(283.47472778,413.71210297)
\lineto(283.47472778,415.94710297)
\lineto(283.47472778,416.74210297)
\lineto(283.47472778,416.95210297)
\curveto(283.48472636,417.02209708)(283.47972637,417.08709701)(283.45972778,417.14710297)
\curveto(283.41972643,417.28709681)(283.3497265,417.37709672)(283.24972778,417.41710297)
\curveto(283.13972671,417.46709663)(282.99972685,417.48209662)(282.82972778,417.46210297)
\curveto(282.65972719,417.44209666)(282.51472733,417.45709664)(282.39472778,417.50710297)
\curveto(282.31472753,417.53709656)(282.26472758,417.58209652)(282.24472778,417.64210297)
\curveto(282.22472762,417.7020964)(282.20472764,417.77709632)(282.18472778,417.86710297)
\lineto(282.18472778,418.18210297)
\curveto(282.18472766,418.36209574)(282.19472765,418.50709559)(282.21472778,418.61710297)
\curveto(282.23472761,418.72709537)(282.31972753,418.8020953)(282.46972778,418.84210297)
\curveto(282.50972734,418.86209524)(282.5497273,418.86709523)(282.58972778,418.85710297)
\lineto(282.72472778,418.85710297)
\curveto(282.87472697,418.85709524)(283.01472683,418.86209524)(283.14472778,418.87210297)
\curveto(283.27472657,418.89209521)(283.36472648,418.95209515)(283.41472778,419.05210297)
\curveto(283.4447264,419.12209498)(283.45972639,419.2020949)(283.45972778,419.29210297)
\curveto(283.46972638,419.38209472)(283.47472637,419.47209463)(283.47472778,419.56210297)
\lineto(283.47472778,420.49210297)
\lineto(283.47472778,420.74710297)
\curveto(283.47472637,420.83709326)(283.48472636,420.91209319)(283.50472778,420.97210297)
\curveto(283.55472629,421.07209303)(283.62972622,421.13709296)(283.72972778,421.16710297)
\curveto(283.7497261,421.17709292)(283.77472607,421.17709292)(283.80472778,421.16710297)
\curveto(283.844726,421.16709293)(283.87472597,421.17209293)(283.89472778,421.18210297)
}
}
{
\newrgbcolor{curcolor}{0 0 0}
\pscustom[linestyle=none,fillstyle=solid,fillcolor=curcolor]
{
\newpath
\moveto(295.48316528,415.07710297)
\curveto(295.50315712,414.9970991)(295.50315712,414.90709919)(295.48316528,414.80710297)
\curveto(295.46315716,414.70709939)(295.42815719,414.64209946)(295.37816528,414.61210297)
\curveto(295.32815729,414.57209953)(295.25315737,414.54209956)(295.15316528,414.52210297)
\curveto(295.06315756,414.51209959)(294.95815766,414.5020996)(294.83816528,414.49210297)
\lineto(294.49316528,414.49210297)
\curveto(294.38315824,414.5020996)(294.28315834,414.50709959)(294.19316528,414.50710297)
\lineto(290.53316528,414.50710297)
\lineto(290.32316528,414.50710297)
\curveto(290.26316236,414.50709959)(290.20816241,414.4970996)(290.15816528,414.47710297)
\curveto(290.07816254,414.43709966)(290.02816259,414.3970997)(290.00816528,414.35710297)
\curveto(289.98816263,414.33709976)(289.96816265,414.2970998)(289.94816528,414.23710297)
\curveto(289.92816269,414.18709991)(289.9231627,414.13709996)(289.93316528,414.08710297)
\curveto(289.95316267,414.02710007)(289.96316266,413.96710013)(289.96316528,413.90710297)
\curveto(289.97316265,413.85710024)(289.98816263,413.8021003)(290.00816528,413.74210297)
\curveto(290.08816253,413.5021006)(290.18316244,413.3021008)(290.29316528,413.14210297)
\curveto(290.41316221,412.99210111)(290.57316205,412.85710124)(290.77316528,412.73710297)
\curveto(290.85316177,412.68710141)(290.93316169,412.65210145)(291.01316528,412.63210297)
\curveto(291.10316152,412.62210148)(291.19316143,412.6021015)(291.28316528,412.57210297)
\curveto(291.36316126,412.55210155)(291.47316115,412.53710156)(291.61316528,412.52710297)
\curveto(291.75316087,412.51710158)(291.87316075,412.52210158)(291.97316528,412.54210297)
\lineto(292.10816528,412.54210297)
\curveto(292.20816041,412.56210154)(292.29816032,412.58210152)(292.37816528,412.60210297)
\curveto(292.46816015,412.63210147)(292.55316007,412.66210144)(292.63316528,412.69210297)
\curveto(292.73315989,412.74210136)(292.84315978,412.80710129)(292.96316528,412.88710297)
\curveto(293.09315953,412.96710113)(293.18815943,413.04710105)(293.24816528,413.12710297)
\curveto(293.29815932,413.1971009)(293.34815927,413.26210084)(293.39816528,413.32210297)
\curveto(293.45815916,413.39210071)(293.52815909,413.44210066)(293.60816528,413.47210297)
\curveto(293.70815891,413.52210058)(293.83315879,413.54210056)(293.98316528,413.53210297)
\lineto(294.41816528,413.53210297)
\lineto(294.59816528,413.53210297)
\curveto(294.66815795,413.54210056)(294.72815789,413.53710056)(294.77816528,413.51710297)
\lineto(294.92816528,413.51710297)
\curveto(295.02815759,413.4971006)(295.09815752,413.47210063)(295.13816528,413.44210297)
\curveto(295.17815744,413.42210068)(295.19815742,413.37710072)(295.19816528,413.30710297)
\curveto(295.20815741,413.23710086)(295.20315742,413.17710092)(295.18316528,413.12710297)
\curveto(295.13315749,412.98710111)(295.07815754,412.86210124)(295.01816528,412.75210297)
\curveto(294.95815766,412.64210146)(294.88815773,412.53210157)(294.80816528,412.42210297)
\curveto(294.58815803,412.09210201)(294.33815828,411.82710227)(294.05816528,411.62710297)
\curveto(293.77815884,411.42710267)(293.42815919,411.25710284)(293.00816528,411.11710297)
\curveto(292.89815972,411.07710302)(292.78815983,411.05210305)(292.67816528,411.04210297)
\curveto(292.56816005,411.03210307)(292.45316017,411.01210309)(292.33316528,410.98210297)
\curveto(292.29316033,410.97210313)(292.24816037,410.97210313)(292.19816528,410.98210297)
\curveto(292.15816046,410.98210312)(292.1181605,410.97710312)(292.07816528,410.96710297)
\lineto(291.91316528,410.96710297)
\curveto(291.86316076,410.94710315)(291.80316082,410.94210316)(291.73316528,410.95210297)
\curveto(291.67316095,410.95210315)(291.618161,410.95710314)(291.56816528,410.96710297)
\curveto(291.48816113,410.97710312)(291.4181612,410.97710312)(291.35816528,410.96710297)
\curveto(291.29816132,410.95710314)(291.23316139,410.96210314)(291.16316528,410.98210297)
\curveto(291.11316151,411.0021031)(291.05816156,411.01210309)(290.99816528,411.01210297)
\curveto(290.93816168,411.01210309)(290.88316174,411.02210308)(290.83316528,411.04210297)
\curveto(290.7231619,411.06210304)(290.61316201,411.08710301)(290.50316528,411.11710297)
\curveto(290.39316223,411.13710296)(290.29316233,411.17210293)(290.20316528,411.22210297)
\curveto(290.09316253,411.26210284)(289.98816263,411.2971028)(289.88816528,411.32710297)
\curveto(289.79816282,411.36710273)(289.71316291,411.41210269)(289.63316528,411.46210297)
\curveto(289.31316331,411.66210244)(289.02816359,411.89210221)(288.77816528,412.15210297)
\curveto(288.52816409,412.42210168)(288.3231643,412.73210137)(288.16316528,413.08210297)
\curveto(288.11316451,413.19210091)(288.07316455,413.3021008)(288.04316528,413.41210297)
\curveto(288.01316461,413.53210057)(287.97316465,413.65210045)(287.92316528,413.77210297)
\curveto(287.91316471,413.81210029)(287.90816471,413.84710025)(287.90816528,413.87710297)
\curveto(287.90816471,413.91710018)(287.90316472,413.95710014)(287.89316528,413.99710297)
\curveto(287.85316477,414.11709998)(287.82816479,414.24709985)(287.81816528,414.38710297)
\lineto(287.78816528,414.80710297)
\curveto(287.78816483,414.85709924)(287.78316484,414.91209919)(287.77316528,414.97210297)
\curveto(287.77316485,415.03209907)(287.77816484,415.08709901)(287.78816528,415.13710297)
\lineto(287.78816528,415.31710297)
\lineto(287.83316528,415.67710297)
\curveto(287.87316475,415.84709825)(287.90816471,416.01209809)(287.93816528,416.17210297)
\curveto(287.96816465,416.33209777)(288.01316461,416.48209762)(288.07316528,416.62210297)
\curveto(288.50316412,417.66209644)(289.23316339,418.3970957)(290.26316528,418.82710297)
\curveto(290.40316222,418.88709521)(290.54316208,418.92709517)(290.68316528,418.94710297)
\curveto(290.83316179,418.97709512)(290.98816163,419.01209509)(291.14816528,419.05210297)
\curveto(291.22816139,419.06209504)(291.30316132,419.06709503)(291.37316528,419.06710297)
\curveto(291.44316118,419.06709503)(291.5181611,419.07209503)(291.59816528,419.08210297)
\curveto(292.10816051,419.09209501)(292.54316008,419.03209507)(292.90316528,418.90210297)
\curveto(293.27315935,418.78209532)(293.60315902,418.62209548)(293.89316528,418.42210297)
\curveto(293.98315864,418.36209574)(294.07315855,418.29209581)(294.16316528,418.21210297)
\curveto(294.25315837,418.14209596)(294.33315829,418.06709603)(294.40316528,417.98710297)
\curveto(294.43315819,417.93709616)(294.47315815,417.8970962)(294.52316528,417.86710297)
\curveto(294.60315802,417.75709634)(294.67815794,417.64209646)(294.74816528,417.52210297)
\curveto(294.8181578,417.41209669)(294.89315773,417.2970968)(294.97316528,417.17710297)
\curveto(295.0231576,417.08709701)(295.06315756,416.99209711)(295.09316528,416.89210297)
\curveto(295.13315749,416.8020973)(295.17315745,416.7020974)(295.21316528,416.59210297)
\curveto(295.26315736,416.46209764)(295.30315732,416.32709777)(295.33316528,416.18710297)
\curveto(295.36315726,416.04709805)(295.39815722,415.90709819)(295.43816528,415.76710297)
\curveto(295.45815716,415.68709841)(295.46315716,415.5970985)(295.45316528,415.49710297)
\curveto(295.45315717,415.40709869)(295.46315716,415.32209878)(295.48316528,415.24210297)
\lineto(295.48316528,415.07710297)
\moveto(293.23316528,415.96210297)
\curveto(293.30315932,416.06209804)(293.30815931,416.18209792)(293.24816528,416.32210297)
\curveto(293.19815942,416.47209763)(293.15815946,416.58209752)(293.12816528,416.65210297)
\curveto(292.98815963,416.92209718)(292.80315982,417.12709697)(292.57316528,417.26710297)
\curveto(292.34316028,417.41709668)(292.0231606,417.4970966)(291.61316528,417.50710297)
\curveto(291.58316104,417.48709661)(291.54816107,417.48209662)(291.50816528,417.49210297)
\curveto(291.46816115,417.5020966)(291.43316119,417.5020966)(291.40316528,417.49210297)
\curveto(291.35316127,417.47209663)(291.29816132,417.45709664)(291.23816528,417.44710297)
\curveto(291.17816144,417.44709665)(291.1231615,417.43709666)(291.07316528,417.41710297)
\curveto(290.63316199,417.27709682)(290.30816231,417.0020971)(290.09816528,416.59210297)
\curveto(290.07816254,416.55209755)(290.05316257,416.4970976)(290.02316528,416.42710297)
\curveto(290.00316262,416.36709773)(289.98816263,416.3020978)(289.97816528,416.23210297)
\curveto(289.96816265,416.17209793)(289.96816265,416.11209799)(289.97816528,416.05210297)
\curveto(289.99816262,415.99209811)(290.03316259,415.94209816)(290.08316528,415.90210297)
\curveto(290.16316246,415.85209825)(290.27316235,415.82709827)(290.41316528,415.82710297)
\lineto(290.81816528,415.82710297)
\lineto(292.48316528,415.82710297)
\lineto(292.91816528,415.82710297)
\curveto(293.07815954,415.83709826)(293.18315944,415.88209822)(293.23316528,415.96210297)
}
}
{
\newrgbcolor{curcolor}{0 0 0}
\pscustom[linestyle=none,fillstyle=solid,fillcolor=curcolor]
{
\newpath
\moveto(301.15644653,419.06710297)
\curveto(301.26644122,419.06709503)(301.36144112,419.05709504)(301.44144653,419.03710297)
\curveto(301.53144095,419.01709508)(301.60144088,418.97209513)(301.65144653,418.90210297)
\curveto(301.71144077,418.82209528)(301.74144074,418.68209542)(301.74144653,418.48210297)
\lineto(301.74144653,417.97210297)
\lineto(301.74144653,417.59710297)
\curveto(301.75144073,417.45709664)(301.73644075,417.34709675)(301.69644653,417.26710297)
\curveto(301.65644083,417.1970969)(301.59644089,417.15209695)(301.51644653,417.13210297)
\curveto(301.44644104,417.11209699)(301.36144112,417.102097)(301.26144653,417.10210297)
\curveto(301.17144131,417.102097)(301.07144141,417.10709699)(300.96144653,417.11710297)
\curveto(300.86144162,417.12709697)(300.76644172,417.12209698)(300.67644653,417.10210297)
\curveto(300.60644188,417.08209702)(300.53644195,417.06709703)(300.46644653,417.05710297)
\curveto(300.39644209,417.05709704)(300.33144215,417.04709705)(300.27144653,417.02710297)
\curveto(300.11144237,416.97709712)(299.95144253,416.9020972)(299.79144653,416.80210297)
\curveto(299.63144285,416.71209739)(299.50644298,416.60709749)(299.41644653,416.48710297)
\curveto(299.36644312,416.40709769)(299.31144317,416.32209778)(299.25144653,416.23210297)
\curveto(299.20144328,416.15209795)(299.15144333,416.06709803)(299.10144653,415.97710297)
\curveto(299.07144341,415.8970982)(299.04144344,415.81209829)(299.01144653,415.72210297)
\lineto(298.95144653,415.48210297)
\curveto(298.93144355,415.41209869)(298.92144356,415.33709876)(298.92144653,415.25710297)
\curveto(298.92144356,415.18709891)(298.91144357,415.11709898)(298.89144653,415.04710297)
\curveto(298.8814436,415.00709909)(298.87644361,414.96709913)(298.87644653,414.92710297)
\curveto(298.8864436,414.8970992)(298.8864436,414.86709923)(298.87644653,414.83710297)
\lineto(298.87644653,414.59710297)
\curveto(298.85644363,414.52709957)(298.85144363,414.44709965)(298.86144653,414.35710297)
\curveto(298.87144361,414.27709982)(298.87644361,414.1970999)(298.87644653,414.11710297)
\lineto(298.87644653,413.15710297)
\lineto(298.87644653,411.88210297)
\curveto(298.87644361,411.75210235)(298.87144361,411.63210247)(298.86144653,411.52210297)
\curveto(298.85144363,411.41210269)(298.82144366,411.32210278)(298.77144653,411.25210297)
\curveto(298.75144373,411.22210288)(298.71644377,411.1971029)(298.66644653,411.17710297)
\curveto(298.62644386,411.16710293)(298.5814439,411.15710294)(298.53144653,411.14710297)
\lineto(298.45644653,411.14710297)
\curveto(298.40644408,411.13710296)(298.35144413,411.13210297)(298.29144653,411.13210297)
\lineto(298.12644653,411.13210297)
\lineto(297.48144653,411.13210297)
\curveto(297.42144506,411.14210296)(297.35644513,411.14710295)(297.28644653,411.14710297)
\lineto(297.09144653,411.14710297)
\curveto(297.04144544,411.16710293)(296.99144549,411.18210292)(296.94144653,411.19210297)
\curveto(296.89144559,411.21210289)(296.85644563,411.24710285)(296.83644653,411.29710297)
\curveto(296.79644569,411.34710275)(296.77144571,411.41710268)(296.76144653,411.50710297)
\lineto(296.76144653,411.80710297)
\lineto(296.76144653,412.82710297)
\lineto(296.76144653,417.05710297)
\lineto(296.76144653,418.16710297)
\lineto(296.76144653,418.45210297)
\curveto(296.76144572,418.55209555)(296.7814457,418.63209547)(296.82144653,418.69210297)
\curveto(296.87144561,418.77209533)(296.94644554,418.82209528)(297.04644653,418.84210297)
\curveto(297.14644534,418.86209524)(297.26644522,418.87209523)(297.40644653,418.87210297)
\lineto(298.17144653,418.87210297)
\curveto(298.29144419,418.87209523)(298.39644409,418.86209524)(298.48644653,418.84210297)
\curveto(298.57644391,418.83209527)(298.64644384,418.78709531)(298.69644653,418.70710297)
\curveto(298.72644376,418.65709544)(298.74144374,418.58709551)(298.74144653,418.49710297)
\lineto(298.77144653,418.22710297)
\curveto(298.7814437,418.14709595)(298.79644369,418.07209603)(298.81644653,418.00210297)
\curveto(298.84644364,417.93209617)(298.89644359,417.8970962)(298.96644653,417.89710297)
\curveto(298.9864435,417.91709618)(299.00644348,417.92709617)(299.02644653,417.92710297)
\curveto(299.04644344,417.92709617)(299.06644342,417.93709616)(299.08644653,417.95710297)
\curveto(299.14644334,418.00709609)(299.19644329,418.06209604)(299.23644653,418.12210297)
\curveto(299.2864432,418.19209591)(299.34644314,418.25209585)(299.41644653,418.30210297)
\curveto(299.45644303,418.33209577)(299.49144299,418.36209574)(299.52144653,418.39210297)
\curveto(299.55144293,418.43209567)(299.5864429,418.46709563)(299.62644653,418.49710297)
\lineto(299.89644653,418.67710297)
\curveto(299.99644249,418.73709536)(300.09644239,418.79209531)(300.19644653,418.84210297)
\curveto(300.29644219,418.88209522)(300.39644209,418.91709518)(300.49644653,418.94710297)
\lineto(300.82644653,419.03710297)
\curveto(300.85644163,419.04709505)(300.91144157,419.04709505)(300.99144653,419.03710297)
\curveto(301.0814414,419.03709506)(301.13644135,419.04709505)(301.15644653,419.06710297)
}
}
{
\newrgbcolor{curcolor}{0 0 0}
\pscustom[linestyle=none,fillstyle=solid,fillcolor=curcolor]
{
\newpath
\moveto(309.61152466,411.73210297)
\curveto(309.63151681,411.62210248)(309.6415168,411.51210259)(309.64152466,411.40210297)
\curveto(309.65151679,411.29210281)(309.60151684,411.21710288)(309.49152466,411.17710297)
\curveto(309.43151701,411.14710295)(309.36151708,411.13210297)(309.28152466,411.13210297)
\lineto(309.04152466,411.13210297)
\lineto(308.23152466,411.13210297)
\lineto(307.96152466,411.13210297)
\curveto(307.88151856,411.14210296)(307.81651862,411.16710293)(307.76652466,411.20710297)
\curveto(307.69651874,411.24710285)(307.6415188,411.3021028)(307.60152466,411.37210297)
\curveto(307.57151887,411.45210265)(307.52651891,411.51710258)(307.46652466,411.56710297)
\curveto(307.44651899,411.58710251)(307.42151902,411.6021025)(307.39152466,411.61210297)
\curveto(307.36151908,411.63210247)(307.32151912,411.63710246)(307.27152466,411.62710297)
\curveto(307.22151922,411.60710249)(307.17151927,411.58210252)(307.12152466,411.55210297)
\curveto(307.08151936,411.52210258)(307.0365194,411.4971026)(306.98652466,411.47710297)
\curveto(306.9365195,411.43710266)(306.88151956,411.4021027)(306.82152466,411.37210297)
\lineto(306.64152466,411.28210297)
\curveto(306.51151993,411.22210288)(306.37652006,411.17210293)(306.23652466,411.13210297)
\curveto(306.09652034,411.102103)(305.95152049,411.06710303)(305.80152466,411.02710297)
\curveto(305.73152071,411.00710309)(305.66152078,410.9971031)(305.59152466,410.99710297)
\curveto(305.53152091,410.98710311)(305.46652097,410.97710312)(305.39652466,410.96710297)
\lineto(305.30652466,410.96710297)
\curveto(305.27652116,410.95710314)(305.24652119,410.95210315)(305.21652466,410.95210297)
\lineto(305.05152466,410.95210297)
\curveto(304.95152149,410.93210317)(304.85152159,410.93210317)(304.75152466,410.95210297)
\lineto(304.61652466,410.95210297)
\curveto(304.54652189,410.97210313)(304.47652196,410.98210312)(304.40652466,410.98210297)
\curveto(304.34652209,410.97210313)(304.28652215,410.97710312)(304.22652466,410.99710297)
\curveto(304.12652231,411.01710308)(304.03152241,411.03710306)(303.94152466,411.05710297)
\curveto(303.85152259,411.06710303)(303.76652267,411.09210301)(303.68652466,411.13210297)
\curveto(303.39652304,411.24210286)(303.14652329,411.38210272)(302.93652466,411.55210297)
\curveto(302.7365237,411.73210237)(302.57652386,411.96710213)(302.45652466,412.25710297)
\curveto(302.42652401,412.32710177)(302.39652404,412.4021017)(302.36652466,412.48210297)
\curveto(302.34652409,412.56210154)(302.32652411,412.64710145)(302.30652466,412.73710297)
\curveto(302.28652415,412.78710131)(302.27652416,412.83710126)(302.27652466,412.88710297)
\curveto(302.28652415,412.93710116)(302.28652415,412.98710111)(302.27652466,413.03710297)
\curveto(302.26652417,413.06710103)(302.25652418,413.12710097)(302.24652466,413.21710297)
\curveto(302.24652419,413.31710078)(302.25152419,413.38710071)(302.26152466,413.42710297)
\curveto(302.28152416,413.52710057)(302.29152415,413.61210049)(302.29152466,413.68210297)
\lineto(302.38152466,414.01210297)
\curveto(302.41152403,414.13209997)(302.45152399,414.23709986)(302.50152466,414.32710297)
\curveto(302.67152377,414.61709948)(302.86652357,414.83709926)(303.08652466,414.98710297)
\curveto(303.30652313,415.13709896)(303.58652285,415.26709883)(303.92652466,415.37710297)
\curveto(304.05652238,415.42709867)(304.19152225,415.46209864)(304.33152466,415.48210297)
\curveto(304.47152197,415.5020986)(304.61152183,415.52709857)(304.75152466,415.55710297)
\curveto(304.83152161,415.57709852)(304.91652152,415.58709851)(305.00652466,415.58710297)
\curveto(305.09652134,415.5970985)(305.18652125,415.61209849)(305.27652466,415.63210297)
\curveto(305.34652109,415.65209845)(305.41652102,415.65709844)(305.48652466,415.64710297)
\curveto(305.55652088,415.64709845)(305.63152081,415.65709844)(305.71152466,415.67710297)
\curveto(305.78152066,415.6970984)(305.85152059,415.70709839)(305.92152466,415.70710297)
\curveto(305.99152045,415.70709839)(306.06652037,415.71709838)(306.14652466,415.73710297)
\curveto(306.35652008,415.78709831)(306.54651989,415.82709827)(306.71652466,415.85710297)
\curveto(306.89651954,415.8970982)(307.05651938,415.98709811)(307.19652466,416.12710297)
\curveto(307.28651915,416.21709788)(307.34651909,416.31709778)(307.37652466,416.42710297)
\curveto(307.38651905,416.45709764)(307.38651905,416.48209762)(307.37652466,416.50210297)
\curveto(307.37651906,416.52209758)(307.38151906,416.54209756)(307.39152466,416.56210297)
\curveto(307.40151904,416.58209752)(307.40651903,416.61209749)(307.40652466,416.65210297)
\lineto(307.40652466,416.74210297)
\lineto(307.37652466,416.86210297)
\curveto(307.37651906,416.9020972)(307.37151907,416.93709716)(307.36152466,416.96710297)
\curveto(307.26151918,417.26709683)(307.05151939,417.47209663)(306.73152466,417.58210297)
\curveto(306.6415198,417.61209649)(306.53151991,417.63209647)(306.40152466,417.64210297)
\curveto(306.28152016,417.66209644)(306.15652028,417.66709643)(306.02652466,417.65710297)
\curveto(305.89652054,417.65709644)(305.77152067,417.64709645)(305.65152466,417.62710297)
\curveto(305.53152091,417.60709649)(305.42652101,417.58209652)(305.33652466,417.55210297)
\curveto(305.27652116,417.53209657)(305.21652122,417.5020966)(305.15652466,417.46210297)
\curveto(305.10652133,417.43209667)(305.05652138,417.3970967)(305.00652466,417.35710297)
\curveto(304.95652148,417.31709678)(304.90152154,417.26209684)(304.84152466,417.19210297)
\curveto(304.79152165,417.12209698)(304.75652168,417.05709704)(304.73652466,416.99710297)
\curveto(304.68652175,416.8970972)(304.6415218,416.8020973)(304.60152466,416.71210297)
\curveto(304.57152187,416.62209748)(304.50152194,416.56209754)(304.39152466,416.53210297)
\curveto(304.31152213,416.51209759)(304.22652221,416.5020976)(304.13652466,416.50210297)
\lineto(303.86652466,416.50210297)
\lineto(303.29652466,416.50210297)
\curveto(303.24652319,416.5020976)(303.19652324,416.4970976)(303.14652466,416.48710297)
\curveto(303.09652334,416.48709761)(303.05152339,416.49209761)(303.01152466,416.50210297)
\lineto(302.87652466,416.50210297)
\curveto(302.85652358,416.51209759)(302.83152361,416.51709758)(302.80152466,416.51710297)
\curveto(302.77152367,416.51709758)(302.74652369,416.52709757)(302.72652466,416.54710297)
\curveto(302.64652379,416.56709753)(302.59152385,416.63209747)(302.56152466,416.74210297)
\curveto(302.55152389,416.79209731)(302.55152389,416.84209726)(302.56152466,416.89210297)
\curveto(302.57152387,416.94209716)(302.58152386,416.98709711)(302.59152466,417.02710297)
\curveto(302.62152382,417.13709696)(302.65152379,417.23709686)(302.68152466,417.32710297)
\curveto(302.72152372,417.42709667)(302.76652367,417.51709658)(302.81652466,417.59710297)
\lineto(302.90652466,417.74710297)
\lineto(302.99652466,417.89710297)
\curveto(303.07652336,418.00709609)(303.17652326,418.11209599)(303.29652466,418.21210297)
\curveto(303.31652312,418.22209588)(303.34652309,418.24709585)(303.38652466,418.28710297)
\curveto(303.436523,418.32709577)(303.48152296,418.36209574)(303.52152466,418.39210297)
\curveto(303.56152288,418.42209568)(303.60652283,418.45209565)(303.65652466,418.48210297)
\curveto(303.82652261,418.59209551)(304.00652243,418.67709542)(304.19652466,418.73710297)
\curveto(304.38652205,418.80709529)(304.58152186,418.87209523)(304.78152466,418.93210297)
\curveto(304.90152154,418.96209514)(305.02652141,418.98209512)(305.15652466,418.99210297)
\curveto(305.28652115,419.0020951)(305.41652102,419.02209508)(305.54652466,419.05210297)
\curveto(305.58652085,419.06209504)(305.64652079,419.06209504)(305.72652466,419.05210297)
\curveto(305.81652062,419.04209506)(305.87152057,419.04709505)(305.89152466,419.06710297)
\curveto(306.30152014,419.07709502)(306.69151975,419.06209504)(307.06152466,419.02210297)
\curveto(307.441519,418.98209512)(307.78151866,418.90709519)(308.08152466,418.79710297)
\curveto(308.39151805,418.68709541)(308.65651778,418.53709556)(308.87652466,418.34710297)
\curveto(309.09651734,418.16709593)(309.26651717,417.93209617)(309.38652466,417.64210297)
\curveto(309.45651698,417.47209663)(309.49651694,417.27709682)(309.50652466,417.05710297)
\curveto(309.51651692,416.83709726)(309.52151692,416.61209749)(309.52152466,416.38210297)
\lineto(309.52152466,413.03710297)
\lineto(309.52152466,412.45210297)
\curveto(309.52151692,412.26210184)(309.5415169,412.08710201)(309.58152466,411.92710297)
\curveto(309.59151685,411.8971022)(309.59651684,411.86210224)(309.59652466,411.82210297)
\curveto(309.59651684,411.79210231)(309.60151684,411.76210234)(309.61152466,411.73210297)
\moveto(307.40652466,414.04210297)
\curveto(307.41651902,414.09210001)(307.42151902,414.14709995)(307.42152466,414.20710297)
\curveto(307.42151902,414.27709982)(307.41651902,414.33709976)(307.40652466,414.38710297)
\curveto(307.38651905,414.44709965)(307.37651906,414.5020996)(307.37652466,414.55210297)
\curveto(307.37651906,414.6020995)(307.35651908,414.64209946)(307.31652466,414.67210297)
\curveto(307.26651917,414.71209939)(307.19151925,414.73209937)(307.09152466,414.73210297)
\curveto(307.05151939,414.72209938)(307.01651942,414.71209939)(306.98652466,414.70210297)
\curveto(306.95651948,414.7020994)(306.92151952,414.6970994)(306.88152466,414.68710297)
\curveto(306.81151963,414.66709943)(306.7365197,414.65209945)(306.65652466,414.64210297)
\curveto(306.57651986,414.63209947)(306.49651994,414.61709948)(306.41652466,414.59710297)
\curveto(306.38652005,414.58709951)(306.3415201,414.58209952)(306.28152466,414.58210297)
\curveto(306.15152029,414.55209955)(306.02152042,414.53209957)(305.89152466,414.52210297)
\curveto(305.76152068,414.51209959)(305.6365208,414.48709961)(305.51652466,414.44710297)
\curveto(305.436521,414.42709967)(305.36152108,414.40709969)(305.29152466,414.38710297)
\curveto(305.22152122,414.37709972)(305.15152129,414.35709974)(305.08152466,414.32710297)
\curveto(304.87152157,414.23709986)(304.69152175,414.1021)(304.54152466,413.92210297)
\curveto(304.40152204,413.74210036)(304.35152209,413.49210061)(304.39152466,413.17210297)
\curveto(304.41152203,413.0021011)(304.46652197,412.86210124)(304.55652466,412.75210297)
\curveto(304.62652181,412.64210146)(304.73152171,412.55210155)(304.87152466,412.48210297)
\curveto(305.01152143,412.42210168)(305.16152128,412.37710172)(305.32152466,412.34710297)
\curveto(305.49152095,412.31710178)(305.66652077,412.30710179)(305.84652466,412.31710297)
\curveto(306.0365204,412.33710176)(306.21152023,412.37210173)(306.37152466,412.42210297)
\curveto(306.63151981,412.5021016)(306.8365196,412.62710147)(306.98652466,412.79710297)
\curveto(307.1365193,412.97710112)(307.25151919,413.1971009)(307.33152466,413.45710297)
\curveto(307.35151909,413.52710057)(307.36151908,413.5971005)(307.36152466,413.66710297)
\curveto(307.37151907,413.74710035)(307.38651905,413.82710027)(307.40652466,413.90710297)
\lineto(307.40652466,414.04210297)
}
}
{
\newrgbcolor{curcolor}{0 0 0}
\pscustom[linestyle=none,fillstyle=solid,fillcolor=curcolor]
{
\newpath
\moveto(314.74480591,419.08210297)
\curveto(315.55480075,419.102095)(316.22980007,418.98209512)(316.76980591,418.72210297)
\curveto(317.31979898,418.46209564)(317.75479855,418.09209601)(318.07480591,417.61210297)
\curveto(318.23479807,417.37209673)(318.35479795,417.097097)(318.43480591,416.78710297)
\curveto(318.45479785,416.73709736)(318.46979783,416.67209743)(318.47980591,416.59210297)
\curveto(318.4997978,416.51209759)(318.4997978,416.44209766)(318.47980591,416.38210297)
\curveto(318.43979786,416.27209783)(318.36979793,416.20709789)(318.26980591,416.18710297)
\curveto(318.16979813,416.17709792)(318.04979825,416.17209793)(317.90980591,416.17210297)
\lineto(317.12980591,416.17210297)
\lineto(316.84480591,416.17210297)
\curveto(316.75479955,416.17209793)(316.67979962,416.19209791)(316.61980591,416.23210297)
\curveto(316.53979976,416.27209783)(316.48479982,416.33209777)(316.45480591,416.41210297)
\curveto(316.42479988,416.5020976)(316.38479992,416.59209751)(316.33480591,416.68210297)
\curveto(316.27480003,416.79209731)(316.20980009,416.89209721)(316.13980591,416.98210297)
\curveto(316.06980023,417.07209703)(315.98980031,417.15209695)(315.89980591,417.22210297)
\curveto(315.75980054,417.31209679)(315.6048007,417.38209672)(315.43480591,417.43210297)
\curveto(315.37480093,417.45209665)(315.31480099,417.46209664)(315.25480591,417.46210297)
\curveto(315.19480111,417.46209664)(315.13980116,417.47209663)(315.08980591,417.49210297)
\lineto(314.93980591,417.49210297)
\curveto(314.73980156,417.49209661)(314.57980172,417.47209663)(314.45980591,417.43210297)
\curveto(314.16980213,417.34209676)(313.93480237,417.2020969)(313.75480591,417.01210297)
\curveto(313.57480273,416.83209727)(313.42980287,416.61209749)(313.31980591,416.35210297)
\curveto(313.26980303,416.24209786)(313.22980307,416.12209798)(313.19980591,415.99210297)
\curveto(313.17980312,415.87209823)(313.15480315,415.74209836)(313.12480591,415.60210297)
\curveto(313.11480319,415.56209854)(313.10980319,415.52209858)(313.10980591,415.48210297)
\curveto(313.10980319,415.44209866)(313.1048032,415.4020987)(313.09480591,415.36210297)
\curveto(313.07480323,415.26209884)(313.06480324,415.12209898)(313.06480591,414.94210297)
\curveto(313.07480323,414.76209934)(313.08980321,414.62209948)(313.10980591,414.52210297)
\curveto(313.10980319,414.44209966)(313.11480319,414.38709971)(313.12480591,414.35710297)
\curveto(313.14480316,414.28709981)(313.15480315,414.21709988)(313.15480591,414.14710297)
\curveto(313.16480314,414.07710002)(313.17980312,414.00710009)(313.19980591,413.93710297)
\curveto(313.27980302,413.70710039)(313.37480293,413.4971006)(313.48480591,413.30710297)
\curveto(313.59480271,413.11710098)(313.73480257,412.95710114)(313.90480591,412.82710297)
\curveto(313.94480236,412.7971013)(314.0048023,412.76210134)(314.08480591,412.72210297)
\curveto(314.19480211,412.65210145)(314.304802,412.60710149)(314.41480591,412.58710297)
\curveto(314.53480177,412.56710153)(314.67980162,412.54710155)(314.84980591,412.52710297)
\lineto(314.93980591,412.52710297)
\curveto(314.97980132,412.52710157)(315.00980129,412.53210157)(315.02980591,412.54210297)
\lineto(315.16480591,412.54210297)
\curveto(315.23480107,412.56210154)(315.299801,412.57710152)(315.35980591,412.58710297)
\curveto(315.42980087,412.60710149)(315.49480081,412.62710147)(315.55480591,412.64710297)
\curveto(315.85480045,412.77710132)(316.08480022,412.96710113)(316.24480591,413.21710297)
\curveto(316.28480002,413.26710083)(316.31979998,413.32210078)(316.34980591,413.38210297)
\curveto(316.37979992,413.45210065)(316.4047999,413.51210059)(316.42480591,413.56210297)
\curveto(316.46479984,413.67210043)(316.4997998,413.76710033)(316.52980591,413.84710297)
\curveto(316.55979974,413.93710016)(316.62979967,414.00710009)(316.73980591,414.05710297)
\curveto(316.82979947,414.0971)(316.97479933,414.11209999)(317.17480591,414.10210297)
\lineto(317.66980591,414.10210297)
\lineto(317.87980591,414.10210297)
\curveto(317.95979834,414.11209999)(318.02479828,414.10709999)(318.07480591,414.08710297)
\lineto(318.19480591,414.08710297)
\lineto(318.31480591,414.05710297)
\curveto(318.35479795,414.05710004)(318.38479792,414.04710005)(318.40480591,414.02710297)
\curveto(318.45479785,413.98710011)(318.48479782,413.92710017)(318.49480591,413.84710297)
\curveto(318.51479779,413.77710032)(318.51479779,413.7021004)(318.49480591,413.62210297)
\curveto(318.4047979,413.29210081)(318.29479801,412.9971011)(318.16480591,412.73710297)
\curveto(317.75479855,411.96710213)(317.0997992,411.43210267)(316.19980591,411.13210297)
\curveto(316.0998002,411.102103)(315.99480031,411.08210302)(315.88480591,411.07210297)
\curveto(315.77480053,411.05210305)(315.66480064,411.02710307)(315.55480591,410.99710297)
\curveto(315.49480081,410.98710311)(315.43480087,410.98210312)(315.37480591,410.98210297)
\curveto(315.31480099,410.98210312)(315.25480105,410.97710312)(315.19480591,410.96710297)
\lineto(315.02980591,410.96710297)
\curveto(314.97980132,410.94710315)(314.9048014,410.94210316)(314.80480591,410.95210297)
\curveto(314.7048016,410.95210315)(314.62980167,410.95710314)(314.57980591,410.96710297)
\curveto(314.4998018,410.98710311)(314.42480188,410.9971031)(314.35480591,410.99710297)
\curveto(314.29480201,410.98710311)(314.22980207,410.99210311)(314.15980591,411.01210297)
\lineto(314.00980591,411.04210297)
\curveto(313.95980234,411.04210306)(313.90980239,411.04710305)(313.85980591,411.05710297)
\curveto(313.74980255,411.08710301)(313.64480266,411.11710298)(313.54480591,411.14710297)
\curveto(313.44480286,411.17710292)(313.34980295,411.21210289)(313.25980591,411.25210297)
\curveto(312.78980351,411.45210265)(312.39480391,411.70710239)(312.07480591,412.01710297)
\curveto(311.75480455,412.33710176)(311.49480481,412.73210137)(311.29480591,413.20210297)
\curveto(311.24480506,413.29210081)(311.2048051,413.38710071)(311.17480591,413.48710297)
\lineto(311.08480591,413.81710297)
\curveto(311.07480523,413.85710024)(311.06980523,413.89210021)(311.06980591,413.92210297)
\curveto(311.06980523,413.96210014)(311.05980524,414.00710009)(311.03980591,414.05710297)
\curveto(311.01980528,414.12709997)(311.00980529,414.1970999)(311.00980591,414.26710297)
\curveto(311.00980529,414.34709975)(310.9998053,414.42209968)(310.97980591,414.49210297)
\lineto(310.97980591,414.74710297)
\curveto(310.95980534,414.7970993)(310.94980535,414.85209925)(310.94980591,414.91210297)
\curveto(310.94980535,414.98209912)(310.95980534,415.04209906)(310.97980591,415.09210297)
\curveto(310.98980531,415.14209896)(310.98980531,415.18709891)(310.97980591,415.22710297)
\curveto(310.96980533,415.26709883)(310.96980533,415.30709879)(310.97980591,415.34710297)
\curveto(310.9998053,415.41709868)(311.0048053,415.48209862)(310.99480591,415.54210297)
\curveto(310.99480531,415.6020985)(311.0048053,415.66209844)(311.02480591,415.72210297)
\curveto(311.07480523,415.9020982)(311.11480519,416.07209803)(311.14480591,416.23210297)
\curveto(311.17480513,416.4020977)(311.21980508,416.56709753)(311.27980591,416.72710297)
\curveto(311.4998048,417.23709686)(311.77480453,417.66209644)(312.10480591,418.00210297)
\curveto(312.44480386,418.34209576)(312.87480343,418.61709548)(313.39480591,418.82710297)
\curveto(313.53480277,418.88709521)(313.67980262,418.92709517)(313.82980591,418.94710297)
\curveto(313.97980232,418.97709512)(314.13480217,419.01209509)(314.29480591,419.05210297)
\curveto(314.37480193,419.06209504)(314.44980185,419.06709503)(314.51980591,419.06710297)
\curveto(314.58980171,419.06709503)(314.66480164,419.07209503)(314.74480591,419.08210297)
}
}
{
\newrgbcolor{curcolor}{0 0 0}
\pscustom[linestyle=none,fillstyle=solid,fillcolor=curcolor]
{
\newpath
\moveto(323.35808716,419.08210297)
\curveto(324.168082,419.102095)(324.84308132,418.98209512)(325.38308716,418.72210297)
\curveto(325.93308023,418.46209564)(326.3680798,418.09209601)(326.68808716,417.61210297)
\curveto(326.84807932,417.37209673)(326.9680792,417.097097)(327.04808716,416.78710297)
\curveto(327.0680791,416.73709736)(327.08307908,416.67209743)(327.09308716,416.59210297)
\curveto(327.11307905,416.51209759)(327.11307905,416.44209766)(327.09308716,416.38210297)
\curveto(327.05307911,416.27209783)(326.98307918,416.20709789)(326.88308716,416.18710297)
\curveto(326.78307938,416.17709792)(326.6630795,416.17209793)(326.52308716,416.17210297)
\lineto(325.74308716,416.17210297)
\lineto(325.45808716,416.17210297)
\curveto(325.3680808,416.17209793)(325.29308087,416.19209791)(325.23308716,416.23210297)
\curveto(325.15308101,416.27209783)(325.09808107,416.33209777)(325.06808716,416.41210297)
\curveto(325.03808113,416.5020976)(324.99808117,416.59209751)(324.94808716,416.68210297)
\curveto(324.88808128,416.79209731)(324.82308134,416.89209721)(324.75308716,416.98210297)
\curveto(324.68308148,417.07209703)(324.60308156,417.15209695)(324.51308716,417.22210297)
\curveto(324.37308179,417.31209679)(324.21808195,417.38209672)(324.04808716,417.43210297)
\curveto(323.98808218,417.45209665)(323.92808224,417.46209664)(323.86808716,417.46210297)
\curveto(323.80808236,417.46209664)(323.75308241,417.47209663)(323.70308716,417.49210297)
\lineto(323.55308716,417.49210297)
\curveto(323.35308281,417.49209661)(323.19308297,417.47209663)(323.07308716,417.43210297)
\curveto(322.78308338,417.34209676)(322.54808362,417.2020969)(322.36808716,417.01210297)
\curveto(322.18808398,416.83209727)(322.04308412,416.61209749)(321.93308716,416.35210297)
\curveto(321.88308428,416.24209786)(321.84308432,416.12209798)(321.81308716,415.99210297)
\curveto(321.79308437,415.87209823)(321.7680844,415.74209836)(321.73808716,415.60210297)
\curveto(321.72808444,415.56209854)(321.72308444,415.52209858)(321.72308716,415.48210297)
\curveto(321.72308444,415.44209866)(321.71808445,415.4020987)(321.70808716,415.36210297)
\curveto(321.68808448,415.26209884)(321.67808449,415.12209898)(321.67808716,414.94210297)
\curveto(321.68808448,414.76209934)(321.70308446,414.62209948)(321.72308716,414.52210297)
\curveto(321.72308444,414.44209966)(321.72808444,414.38709971)(321.73808716,414.35710297)
\curveto(321.75808441,414.28709981)(321.7680844,414.21709988)(321.76808716,414.14710297)
\curveto(321.77808439,414.07710002)(321.79308437,414.00710009)(321.81308716,413.93710297)
\curveto(321.89308427,413.70710039)(321.98808418,413.4971006)(322.09808716,413.30710297)
\curveto(322.20808396,413.11710098)(322.34808382,412.95710114)(322.51808716,412.82710297)
\curveto(322.55808361,412.7971013)(322.61808355,412.76210134)(322.69808716,412.72210297)
\curveto(322.80808336,412.65210145)(322.91808325,412.60710149)(323.02808716,412.58710297)
\curveto(323.14808302,412.56710153)(323.29308287,412.54710155)(323.46308716,412.52710297)
\lineto(323.55308716,412.52710297)
\curveto(323.59308257,412.52710157)(323.62308254,412.53210157)(323.64308716,412.54210297)
\lineto(323.77808716,412.54210297)
\curveto(323.84808232,412.56210154)(323.91308225,412.57710152)(323.97308716,412.58710297)
\curveto(324.04308212,412.60710149)(324.10808206,412.62710147)(324.16808716,412.64710297)
\curveto(324.4680817,412.77710132)(324.69808147,412.96710113)(324.85808716,413.21710297)
\curveto(324.89808127,413.26710083)(324.93308123,413.32210078)(324.96308716,413.38210297)
\curveto(324.99308117,413.45210065)(325.01808115,413.51210059)(325.03808716,413.56210297)
\curveto(325.07808109,413.67210043)(325.11308105,413.76710033)(325.14308716,413.84710297)
\curveto(325.17308099,413.93710016)(325.24308092,414.00710009)(325.35308716,414.05710297)
\curveto(325.44308072,414.0971)(325.58808058,414.11209999)(325.78808716,414.10210297)
\lineto(326.28308716,414.10210297)
\lineto(326.49308716,414.10210297)
\curveto(326.57307959,414.11209999)(326.63807953,414.10709999)(326.68808716,414.08710297)
\lineto(326.80808716,414.08710297)
\lineto(326.92808716,414.05710297)
\curveto(326.9680792,414.05710004)(326.99807917,414.04710005)(327.01808716,414.02710297)
\curveto(327.0680791,413.98710011)(327.09807907,413.92710017)(327.10808716,413.84710297)
\curveto(327.12807904,413.77710032)(327.12807904,413.7021004)(327.10808716,413.62210297)
\curveto(327.01807915,413.29210081)(326.90807926,412.9971011)(326.77808716,412.73710297)
\curveto(326.3680798,411.96710213)(325.71308045,411.43210267)(324.81308716,411.13210297)
\curveto(324.71308145,411.102103)(324.60808156,411.08210302)(324.49808716,411.07210297)
\curveto(324.38808178,411.05210305)(324.27808189,411.02710307)(324.16808716,410.99710297)
\curveto(324.10808206,410.98710311)(324.04808212,410.98210312)(323.98808716,410.98210297)
\curveto(323.92808224,410.98210312)(323.8680823,410.97710312)(323.80808716,410.96710297)
\lineto(323.64308716,410.96710297)
\curveto(323.59308257,410.94710315)(323.51808265,410.94210316)(323.41808716,410.95210297)
\curveto(323.31808285,410.95210315)(323.24308292,410.95710314)(323.19308716,410.96710297)
\curveto(323.11308305,410.98710311)(323.03808313,410.9971031)(322.96808716,410.99710297)
\curveto(322.90808326,410.98710311)(322.84308332,410.99210311)(322.77308716,411.01210297)
\lineto(322.62308716,411.04210297)
\curveto(322.57308359,411.04210306)(322.52308364,411.04710305)(322.47308716,411.05710297)
\curveto(322.3630838,411.08710301)(322.25808391,411.11710298)(322.15808716,411.14710297)
\curveto(322.05808411,411.17710292)(321.9630842,411.21210289)(321.87308716,411.25210297)
\curveto(321.40308476,411.45210265)(321.00808516,411.70710239)(320.68808716,412.01710297)
\curveto(320.3680858,412.33710176)(320.10808606,412.73210137)(319.90808716,413.20210297)
\curveto(319.85808631,413.29210081)(319.81808635,413.38710071)(319.78808716,413.48710297)
\lineto(319.69808716,413.81710297)
\curveto(319.68808648,413.85710024)(319.68308648,413.89210021)(319.68308716,413.92210297)
\curveto(319.68308648,413.96210014)(319.67308649,414.00710009)(319.65308716,414.05710297)
\curveto(319.63308653,414.12709997)(319.62308654,414.1970999)(319.62308716,414.26710297)
\curveto(319.62308654,414.34709975)(319.61308655,414.42209968)(319.59308716,414.49210297)
\lineto(319.59308716,414.74710297)
\curveto(319.57308659,414.7970993)(319.5630866,414.85209925)(319.56308716,414.91210297)
\curveto(319.5630866,414.98209912)(319.57308659,415.04209906)(319.59308716,415.09210297)
\curveto(319.60308656,415.14209896)(319.60308656,415.18709891)(319.59308716,415.22710297)
\curveto(319.58308658,415.26709883)(319.58308658,415.30709879)(319.59308716,415.34710297)
\curveto(319.61308655,415.41709868)(319.61808655,415.48209862)(319.60808716,415.54210297)
\curveto(319.60808656,415.6020985)(319.61808655,415.66209844)(319.63808716,415.72210297)
\curveto(319.68808648,415.9020982)(319.72808644,416.07209803)(319.75808716,416.23210297)
\curveto(319.78808638,416.4020977)(319.83308633,416.56709753)(319.89308716,416.72710297)
\curveto(320.11308605,417.23709686)(320.38808578,417.66209644)(320.71808716,418.00210297)
\curveto(321.05808511,418.34209576)(321.48808468,418.61709548)(322.00808716,418.82710297)
\curveto(322.14808402,418.88709521)(322.29308387,418.92709517)(322.44308716,418.94710297)
\curveto(322.59308357,418.97709512)(322.74808342,419.01209509)(322.90808716,419.05210297)
\curveto(322.98808318,419.06209504)(323.0630831,419.06709503)(323.13308716,419.06710297)
\curveto(323.20308296,419.06709503)(323.27808289,419.07209503)(323.35808716,419.08210297)
}
}
{
\newrgbcolor{curcolor}{0 0 0}
\pscustom[linestyle=none,fillstyle=solid,fillcolor=curcolor]
{
\newpath
\moveto(330.50136841,421.72210297)
\curveto(330.57136546,421.64209246)(330.60636542,421.52209258)(330.60636841,421.36210297)
\lineto(330.60636841,420.89710297)
\lineto(330.60636841,420.49210297)
\curveto(330.60636542,420.35209375)(330.57136546,420.25709384)(330.50136841,420.20710297)
\curveto(330.44136559,420.15709394)(330.36136567,420.12709397)(330.26136841,420.11710297)
\curveto(330.17136586,420.10709399)(330.07136596,420.102094)(329.96136841,420.10210297)
\lineto(329.12136841,420.10210297)
\curveto(329.01136702,420.102094)(328.91136712,420.10709399)(328.82136841,420.11710297)
\curveto(328.74136729,420.12709397)(328.67136736,420.15709394)(328.61136841,420.20710297)
\curveto(328.57136746,420.23709386)(328.54136749,420.29209381)(328.52136841,420.37210297)
\curveto(328.51136752,420.46209364)(328.50136753,420.55709354)(328.49136841,420.65710297)
\lineto(328.49136841,420.98710297)
\curveto(328.50136753,421.097093)(328.50636752,421.19209291)(328.50636841,421.27210297)
\lineto(328.50636841,421.48210297)
\curveto(328.51636751,421.55209255)(328.53636749,421.61209249)(328.56636841,421.66210297)
\curveto(328.58636744,421.7020924)(328.61136742,421.73209237)(328.64136841,421.75210297)
\lineto(328.76136841,421.81210297)
\curveto(328.78136725,421.81209229)(328.80636722,421.81209229)(328.83636841,421.81210297)
\curveto(328.86636716,421.82209228)(328.89136714,421.82709227)(328.91136841,421.82710297)
\lineto(330.00636841,421.82710297)
\curveto(330.10636592,421.82709227)(330.20136583,421.82209228)(330.29136841,421.81210297)
\curveto(330.38136565,421.8020923)(330.45136558,421.77209233)(330.50136841,421.72210297)
\moveto(330.60636841,411.95710297)
\curveto(330.60636542,411.75710234)(330.60136543,411.58710251)(330.59136841,411.44710297)
\curveto(330.58136545,411.30710279)(330.49136554,411.21210289)(330.32136841,411.16210297)
\curveto(330.26136577,411.14210296)(330.19636583,411.13210297)(330.12636841,411.13210297)
\curveto(330.05636597,411.14210296)(329.98136605,411.14710295)(329.90136841,411.14710297)
\lineto(329.06136841,411.14710297)
\curveto(328.97136706,411.14710295)(328.88136715,411.15210295)(328.79136841,411.16210297)
\curveto(328.71136732,411.17210293)(328.65136738,411.2021029)(328.61136841,411.25210297)
\curveto(328.55136748,411.32210278)(328.51636751,411.40710269)(328.50636841,411.50710297)
\lineto(328.50636841,411.85210297)
\lineto(328.50636841,418.18210297)
\lineto(328.50636841,418.48210297)
\curveto(328.50636752,418.58209552)(328.5263675,418.66209544)(328.56636841,418.72210297)
\curveto(328.6263674,418.79209531)(328.71136732,418.83709526)(328.82136841,418.85710297)
\curveto(328.84136719,418.86709523)(328.86636716,418.86709523)(328.89636841,418.85710297)
\curveto(328.93636709,418.85709524)(328.96636706,418.86209524)(328.98636841,418.87210297)
\lineto(329.73636841,418.87210297)
\lineto(329.93136841,418.87210297)
\curveto(330.01136602,418.88209522)(330.07636595,418.88209522)(330.12636841,418.87210297)
\lineto(330.24636841,418.87210297)
\curveto(330.30636572,418.85209525)(330.36136567,418.83709526)(330.41136841,418.82710297)
\curveto(330.46136557,418.81709528)(330.50136553,418.78709531)(330.53136841,418.73710297)
\curveto(330.57136546,418.68709541)(330.59136544,418.61709548)(330.59136841,418.52710297)
\curveto(330.60136543,418.43709566)(330.60636542,418.34209576)(330.60636841,418.24210297)
\lineto(330.60636841,411.95710297)
}
}
{
\newrgbcolor{curcolor}{0 0 0}
\pscustom[linestyle=none,fillstyle=solid,fillcolor=curcolor]
{
\newpath
\moveto(340.03855591,415.31710297)
\curveto(340.01854738,415.36709873)(340.01354738,415.42209868)(340.02355591,415.48210297)
\curveto(340.03354736,415.54209856)(340.02854737,415.5970985)(340.00855591,415.64710297)
\curveto(339.9985474,415.68709841)(339.9935474,415.72709837)(339.99355591,415.76710297)
\curveto(339.9935474,415.80709829)(339.98854741,415.84709825)(339.97855591,415.88710297)
\lineto(339.91855591,416.15710297)
\curveto(339.8985475,416.24709785)(339.87354752,416.33209777)(339.84355591,416.41210297)
\curveto(339.7935476,416.55209755)(339.74854765,416.68209742)(339.70855591,416.80210297)
\curveto(339.66854773,416.93209717)(339.61354778,417.05209705)(339.54355591,417.16210297)
\curveto(339.47354792,417.27209683)(339.40354799,417.37709672)(339.33355591,417.47710297)
\curveto(339.27354812,417.57709652)(339.20354819,417.67709642)(339.12355591,417.77710297)
\curveto(339.04354835,417.88709621)(338.94354845,417.98709611)(338.82355591,418.07710297)
\curveto(338.71354868,418.17709592)(338.60354879,418.26709583)(338.49355591,418.34710297)
\curveto(338.16354923,418.57709552)(337.78354961,418.75709534)(337.35355591,418.88710297)
\curveto(336.93355046,419.01709508)(336.43355096,419.07709502)(335.85355591,419.06710297)
\curveto(335.78355161,419.05709504)(335.71355168,419.05209505)(335.64355591,419.05210297)
\curveto(335.57355182,419.05209505)(335.4985519,419.04709505)(335.41855591,419.03710297)
\curveto(335.26855213,418.9970951)(335.12355227,418.96709513)(334.98355591,418.94710297)
\curveto(334.84355255,418.92709517)(334.70855269,418.89209521)(334.57855591,418.84210297)
\curveto(334.46855293,418.79209531)(334.35855304,418.74709535)(334.24855591,418.70710297)
\curveto(334.13855326,418.66709543)(334.03355336,418.62209548)(333.93355591,418.57210297)
\curveto(333.57355382,418.34209576)(333.26855413,418.08709601)(333.01855591,417.80710297)
\curveto(332.76855463,417.53709656)(332.55355484,417.1970969)(332.37355591,416.78710297)
\curveto(332.32355507,416.66709743)(332.28355511,416.54209756)(332.25355591,416.41210297)
\curveto(332.22355517,416.29209781)(332.18855521,416.16709793)(332.14855591,416.03710297)
\curveto(332.12855527,415.98709811)(332.11855528,415.93709816)(332.11855591,415.88710297)
\curveto(332.11855528,415.84709825)(332.11355528,415.8020983)(332.10355591,415.75210297)
\curveto(332.08355531,415.7020984)(332.07355532,415.64709845)(332.07355591,415.58710297)
\curveto(332.08355531,415.53709856)(332.08355531,415.48709861)(332.07355591,415.43710297)
\lineto(332.07355591,415.33210297)
\curveto(332.05355534,415.27209883)(332.03855536,415.18709891)(332.02855591,415.07710297)
\curveto(332.02855537,414.96709913)(332.03855536,414.88209922)(332.05855591,414.82210297)
\lineto(332.05855591,414.68710297)
\curveto(332.05855534,414.64709945)(332.06355533,414.6020995)(332.07355591,414.55210297)
\curveto(332.0935553,414.47209963)(332.10355529,414.38709971)(332.10355591,414.29710297)
\curveto(332.10355529,414.21709988)(332.11355528,414.13709996)(332.13355591,414.05710297)
\curveto(332.15355524,414.00710009)(332.16355523,413.96210014)(332.16355591,413.92210297)
\curveto(332.16355523,413.88210022)(332.17355522,413.83710026)(332.19355591,413.78710297)
\curveto(332.22355517,413.67710042)(332.24855515,413.57210053)(332.26855591,413.47210297)
\curveto(332.2985551,413.37210073)(332.33855506,413.27710082)(332.38855591,413.18710297)
\curveto(332.55855484,412.7971013)(332.76855463,412.46210164)(333.01855591,412.18210297)
\curveto(333.26855413,411.9021022)(333.56855383,411.65710244)(333.91855591,411.44710297)
\curveto(334.02855337,411.38710271)(334.13355326,411.33710276)(334.23355591,411.29710297)
\curveto(334.34355305,411.25710284)(334.45855294,411.21710288)(334.57855591,411.17710297)
\curveto(334.66855273,411.13710296)(334.76355263,411.10710299)(334.86355591,411.08710297)
\curveto(334.96355243,411.06710303)(335.06355233,411.04210306)(335.16355591,411.01210297)
\curveto(335.21355218,411.0021031)(335.25355214,410.9971031)(335.28355591,410.99710297)
\curveto(335.32355207,410.9971031)(335.36355203,410.99210311)(335.40355591,410.98210297)
\curveto(335.45355194,410.96210314)(335.50355189,410.95710314)(335.55355591,410.96710297)
\curveto(335.61355178,410.96710313)(335.66855173,410.96210314)(335.71855591,410.95210297)
\lineto(335.86855591,410.95210297)
\curveto(335.92855147,410.93210317)(336.01355138,410.92710317)(336.12355591,410.93710297)
\curveto(336.23355116,410.93710316)(336.31355108,410.94210316)(336.36355591,410.95210297)
\curveto(336.393551,410.95210315)(336.42355097,410.95710314)(336.45355591,410.96710297)
\lineto(336.55855591,410.96710297)
\curveto(336.60855079,410.97710312)(336.66355073,410.98210312)(336.72355591,410.98210297)
\curveto(336.78355061,410.98210312)(336.83855056,410.99210311)(336.88855591,411.01210297)
\curveto(337.01855038,411.04210306)(337.14355025,411.07210303)(337.26355591,411.10210297)
\curveto(337.39355,411.12210298)(337.51854988,411.15710294)(337.63855591,411.20710297)
\curveto(338.11854928,411.40710269)(338.52854887,411.65710244)(338.86855591,411.95710297)
\curveto(339.20854819,412.25710184)(339.48354791,412.64710145)(339.69355591,413.12710297)
\curveto(339.74354765,413.22710087)(339.78354761,413.33210077)(339.81355591,413.44210297)
\curveto(339.84354755,413.56210054)(339.87854752,413.67710042)(339.91855591,413.78710297)
\curveto(339.92854747,413.85710024)(339.93854746,413.92210018)(339.94855591,413.98210297)
\curveto(339.95854744,414.04210006)(339.97354742,414.10709999)(339.99355591,414.17710297)
\curveto(340.01354738,414.25709984)(340.01854738,414.33709976)(340.00855591,414.41710297)
\curveto(340.00854739,414.4970996)(340.01854738,414.57709952)(340.03855591,414.65710297)
\lineto(340.03855591,414.80710297)
\curveto(340.05854734,414.86709923)(340.06854733,414.95209915)(340.06855591,415.06210297)
\curveto(340.06854733,415.17209893)(340.05854734,415.25709884)(340.03855591,415.31710297)
\moveto(337.93855591,414.77710297)
\curveto(337.92854947,414.72709937)(337.92354947,414.67709942)(337.92355591,414.62710297)
\lineto(337.92355591,414.49210297)
\curveto(337.91354948,414.45209965)(337.90854949,414.41209969)(337.90855591,414.37210297)
\curveto(337.90854949,414.34209976)(337.90354949,414.30709979)(337.89355591,414.26710297)
\curveto(337.86354953,414.15709994)(337.83854956,414.05210005)(337.81855591,413.95210297)
\curveto(337.7985496,413.85210025)(337.76854963,413.75210035)(337.72855591,413.65210297)
\curveto(337.61854978,413.4021007)(337.48354991,413.19210091)(337.32355591,413.02210297)
\curveto(337.16355023,412.85210125)(336.95355044,412.71710138)(336.69355591,412.61710297)
\curveto(336.62355077,412.58710151)(336.54855085,412.56710153)(336.46855591,412.55710297)
\curveto(336.38855101,412.54710155)(336.30855109,412.53210157)(336.22855591,412.51210297)
\lineto(336.10855591,412.51210297)
\curveto(336.06855133,412.5021016)(336.02355137,412.4971016)(335.97355591,412.49710297)
\lineto(335.85355591,412.52710297)
\curveto(335.81355158,412.53710156)(335.77855162,412.53710156)(335.74855591,412.52710297)
\curveto(335.71855168,412.52710157)(335.68355171,412.53210157)(335.64355591,412.54210297)
\curveto(335.55355184,412.56210154)(335.46355193,412.58710151)(335.37355591,412.61710297)
\curveto(335.2935521,412.64710145)(335.21855218,412.68710141)(335.14855591,412.73710297)
\curveto(334.8985525,412.88710121)(334.71355268,413.05210105)(334.59355591,413.23210297)
\curveto(334.48355291,413.42210068)(334.37855302,413.66710043)(334.27855591,413.96710297)
\curveto(334.25855314,414.04710005)(334.24355315,414.12209998)(334.23355591,414.19210297)
\curveto(334.22355317,414.27209983)(334.20855319,414.35209975)(334.18855591,414.43210297)
\lineto(334.18855591,414.56710297)
\curveto(334.16855323,414.63709946)(334.15355324,414.74209936)(334.14355591,414.88210297)
\curveto(334.14355325,415.02209908)(334.15355324,415.12709897)(334.17355591,415.19710297)
\lineto(334.17355591,415.34710297)
\curveto(334.17355322,415.3970987)(334.17855322,415.44709865)(334.18855591,415.49710297)
\curveto(334.20855319,415.60709849)(334.22355317,415.71709838)(334.23355591,415.82710297)
\curveto(334.25355314,415.93709816)(334.27855312,416.04209806)(334.30855591,416.14210297)
\curveto(334.398553,416.41209769)(334.51855288,416.64709745)(334.66855591,416.84710297)
\curveto(334.82855257,417.05709704)(335.03355236,417.21709688)(335.28355591,417.32710297)
\curveto(335.33355206,417.35709674)(335.38855201,417.37709672)(335.44855591,417.38710297)
\lineto(335.65855591,417.44710297)
\curveto(335.68855171,417.45709664)(335.72355167,417.45709664)(335.76355591,417.44710297)
\curveto(335.80355159,417.44709665)(335.83855156,417.45709664)(335.86855591,417.47710297)
\lineto(336.13855591,417.47710297)
\curveto(336.22855117,417.48709661)(336.31355108,417.48209662)(336.39355591,417.46210297)
\curveto(336.46355093,417.44209666)(336.52855087,417.42209668)(336.58855591,417.40210297)
\curveto(336.64855075,417.39209671)(336.70855069,417.37709672)(336.76855591,417.35710297)
\curveto(337.01855038,417.24709685)(337.21855018,417.097097)(337.36855591,416.90710297)
\curveto(337.51854988,416.72709737)(337.64854975,416.50709759)(337.75855591,416.24710297)
\curveto(337.78854961,416.16709793)(337.80854959,416.08209802)(337.81855591,415.99210297)
\lineto(337.87855591,415.75210297)
\curveto(337.88854951,415.73209837)(337.8935495,415.7020984)(337.89355591,415.66210297)
\curveto(337.90354949,415.61209849)(337.90854949,415.55709854)(337.90855591,415.49710297)
\curveto(337.90854949,415.43709866)(337.91854948,415.38209872)(337.93855591,415.33210297)
\lineto(337.93855591,415.21210297)
\curveto(337.94854945,415.16209894)(337.95354944,415.08709901)(337.95355591,414.98710297)
\curveto(337.95354944,414.8970992)(337.94854945,414.82709927)(337.93855591,414.77710297)
\moveto(336.70855591,421.94710297)
\lineto(337.77355591,421.94710297)
\curveto(337.85354954,421.94709215)(337.94854945,421.94709215)(338.05855591,421.94710297)
\curveto(338.16854923,421.94709215)(338.24854915,421.93209217)(338.29855591,421.90210297)
\curveto(338.31854908,421.89209221)(338.32854907,421.87709222)(338.32855591,421.85710297)
\curveto(338.33854906,421.84709225)(338.35354904,421.83709226)(338.37355591,421.82710297)
\curveto(338.38354901,421.70709239)(338.33354906,421.6020925)(338.22355591,421.51210297)
\curveto(338.12354927,421.42209268)(338.03854936,421.34209276)(337.96855591,421.27210297)
\curveto(337.88854951,421.2020929)(337.80854959,421.12709297)(337.72855591,421.04710297)
\curveto(337.65854974,420.97709312)(337.58354981,420.91209319)(337.50355591,420.85210297)
\curveto(337.46354993,420.82209328)(337.42854997,420.78709331)(337.39855591,420.74710297)
\curveto(337.37855002,420.71709338)(337.34855005,420.69209341)(337.30855591,420.67210297)
\curveto(337.28855011,420.64209346)(337.26355013,420.61709348)(337.23355591,420.59710297)
\lineto(337.08355591,420.44710297)
\lineto(336.93355591,420.32710297)
\lineto(336.88855591,420.28210297)
\curveto(336.88855051,420.27209383)(336.87855052,420.25709384)(336.85855591,420.23710297)
\curveto(336.77855062,420.17709392)(336.6985507,420.11209399)(336.61855591,420.04210297)
\curveto(336.54855085,419.97209413)(336.45855094,419.91709418)(336.34855591,419.87710297)
\curveto(336.30855109,419.86709423)(336.26855113,419.86209424)(336.22855591,419.86210297)
\curveto(336.1985512,419.86209424)(336.15855124,419.85709424)(336.10855591,419.84710297)
\curveto(336.07855132,419.83709426)(336.03855136,419.83209427)(335.98855591,419.83210297)
\curveto(335.93855146,419.84209426)(335.8935515,419.84709425)(335.85355591,419.84710297)
\lineto(335.50855591,419.84710297)
\curveto(335.38855201,419.84709425)(335.2985521,419.87209423)(335.23855591,419.92210297)
\curveto(335.17855222,419.96209414)(335.16355223,420.03209407)(335.19355591,420.13210297)
\curveto(335.21355218,420.21209389)(335.24855215,420.28209382)(335.29855591,420.34210297)
\curveto(335.34855205,420.41209369)(335.393552,420.48209362)(335.43355591,420.55210297)
\curveto(335.53355186,420.69209341)(335.62855177,420.82709327)(335.71855591,420.95710297)
\curveto(335.80855159,421.08709301)(335.8985515,421.22209288)(335.98855591,421.36210297)
\curveto(336.03855136,421.44209266)(336.08855131,421.52709257)(336.13855591,421.61710297)
\curveto(336.1985512,421.70709239)(336.26355113,421.77709232)(336.33355591,421.82710297)
\curveto(336.37355102,421.85709224)(336.44355095,421.89209221)(336.54355591,421.93210297)
\curveto(336.56355083,421.94209216)(336.58855081,421.94209216)(336.61855591,421.93210297)
\curveto(336.65855074,421.93209217)(336.68855071,421.93709216)(336.70855591,421.94710297)
}
}
{
\newrgbcolor{curcolor}{0 0 0}
\pscustom[linestyle=none,fillstyle=solid,fillcolor=curcolor]
{
\newpath
\moveto(345.86347778,419.06710297)
\curveto(346.46347198,419.08709501)(346.96347148,419.0020951)(347.36347778,418.81210297)
\curveto(347.76347068,418.62209548)(348.07847036,418.34209576)(348.30847778,417.97210297)
\curveto(348.37847006,417.86209624)(348.43347001,417.74209636)(348.47347778,417.61210297)
\curveto(348.51346993,417.49209661)(348.55346989,417.36709673)(348.59347778,417.23710297)
\curveto(348.61346983,417.15709694)(348.62346982,417.08209702)(348.62347778,417.01210297)
\curveto(348.63346981,416.94209716)(348.64846979,416.87209723)(348.66847778,416.80210297)
\curveto(348.66846977,416.74209736)(348.67346977,416.7020974)(348.68347778,416.68210297)
\curveto(348.70346974,416.54209756)(348.71346973,416.3970977)(348.71347778,416.24710297)
\lineto(348.71347778,415.81210297)
\lineto(348.71347778,414.47710297)
\lineto(348.71347778,412.04710297)
\curveto(348.71346973,411.85710224)(348.70846973,411.67210243)(348.69847778,411.49210297)
\curveto(348.69846974,411.32210278)(348.62846981,411.21210289)(348.48847778,411.16210297)
\curveto(348.42847001,411.14210296)(348.35847008,411.13210297)(348.27847778,411.13210297)
\lineto(348.03847778,411.13210297)
\lineto(347.22847778,411.13210297)
\curveto(347.10847133,411.13210297)(346.99847144,411.13710296)(346.89847778,411.14710297)
\curveto(346.80847163,411.16710293)(346.7384717,411.21210289)(346.68847778,411.28210297)
\curveto(346.64847179,411.34210276)(346.62347182,411.41710268)(346.61347778,411.50710297)
\lineto(346.61347778,411.82210297)
\lineto(346.61347778,412.87210297)
\lineto(346.61347778,415.10710297)
\curveto(346.61347183,415.47709862)(346.59847184,415.81709828)(346.56847778,416.12710297)
\curveto(346.5384719,416.44709765)(346.44847199,416.71709738)(346.29847778,416.93710297)
\curveto(346.15847228,417.13709696)(345.95347249,417.27709682)(345.68347778,417.35710297)
\curveto(345.63347281,417.37709672)(345.57847286,417.38709671)(345.51847778,417.38710297)
\curveto(345.46847297,417.38709671)(345.41347303,417.3970967)(345.35347778,417.41710297)
\curveto(345.30347314,417.42709667)(345.2384732,417.42709667)(345.15847778,417.41710297)
\curveto(345.08847335,417.41709668)(345.03347341,417.41209669)(344.99347778,417.40210297)
\curveto(344.95347349,417.39209671)(344.91847352,417.38709671)(344.88847778,417.38710297)
\curveto(344.85847358,417.38709671)(344.82847361,417.38209672)(344.79847778,417.37210297)
\curveto(344.56847387,417.31209679)(344.38347406,417.23209687)(344.24347778,417.13210297)
\curveto(343.92347452,416.9020972)(343.73347471,416.56709753)(343.67347778,416.12710297)
\curveto(343.61347483,415.68709841)(343.58347486,415.19209891)(343.58347778,414.64210297)
\lineto(343.58347778,412.76710297)
\lineto(343.58347778,411.85210297)
\lineto(343.58347778,411.58210297)
\curveto(343.58347486,411.49210261)(343.56847487,411.41710268)(343.53847778,411.35710297)
\curveto(343.48847495,411.24710285)(343.40847503,411.18210292)(343.29847778,411.16210297)
\curveto(343.18847525,411.14210296)(343.05347539,411.13210297)(342.89347778,411.13210297)
\lineto(342.14347778,411.13210297)
\curveto(342.03347641,411.13210297)(341.92347652,411.13710296)(341.81347778,411.14710297)
\curveto(341.70347674,411.15710294)(341.62347682,411.19210291)(341.57347778,411.25210297)
\curveto(341.50347694,411.34210276)(341.46847697,411.47210263)(341.46847778,411.64210297)
\curveto(341.47847696,411.81210229)(341.48347696,411.97210213)(341.48347778,412.12210297)
\lineto(341.48347778,414.16210297)
\lineto(341.48347778,417.46210297)
\lineto(341.48347778,418.22710297)
\lineto(341.48347778,418.52710297)
\curveto(341.49347695,418.61709548)(341.52347692,418.69209541)(341.57347778,418.75210297)
\curveto(341.59347685,418.78209532)(341.62347682,418.8020953)(341.66347778,418.81210297)
\curveto(341.71347673,418.83209527)(341.76347668,418.84709525)(341.81347778,418.85710297)
\lineto(341.88847778,418.85710297)
\curveto(341.9384765,418.86709523)(341.98847645,418.87209523)(342.03847778,418.87210297)
\lineto(342.20347778,418.87210297)
\lineto(342.83347778,418.87210297)
\curveto(342.91347553,418.87209523)(342.98847545,418.86709523)(343.05847778,418.85710297)
\curveto(343.1384753,418.85709524)(343.20847523,418.84709525)(343.26847778,418.82710297)
\curveto(343.3384751,418.7970953)(343.38347506,418.75209535)(343.40347778,418.69210297)
\curveto(343.43347501,418.63209547)(343.45847498,418.56209554)(343.47847778,418.48210297)
\curveto(343.48847495,418.44209566)(343.48847495,418.40709569)(343.47847778,418.37710297)
\curveto(343.47847496,418.34709575)(343.48847495,418.31709578)(343.50847778,418.28710297)
\curveto(343.52847491,418.23709586)(343.5434749,418.20709589)(343.55347778,418.19710297)
\curveto(343.57347487,418.18709591)(343.59847484,418.17209593)(343.62847778,418.15210297)
\curveto(343.7384747,418.14209596)(343.82847461,418.17709592)(343.89847778,418.25710297)
\curveto(343.96847447,418.34709575)(344.0434744,418.41709568)(344.12347778,418.46710297)
\curveto(344.39347405,418.66709543)(344.69347375,418.82709527)(345.02347778,418.94710297)
\curveto(345.11347333,418.97709512)(345.20347324,418.9970951)(345.29347778,419.00710297)
\curveto(345.39347305,419.01709508)(345.49847294,419.03209507)(345.60847778,419.05210297)
\curveto(345.6384728,419.06209504)(345.68347276,419.06209504)(345.74347778,419.05210297)
\curveto(345.80347264,419.05209505)(345.8434726,419.05709504)(345.86347778,419.06710297)
}
}
{
\newrgbcolor{curcolor}{0 0 0}
\pscustom[linestyle=none,fillstyle=solid,fillcolor=curcolor]
{
}
}
{
\newrgbcolor{curcolor}{0 0 0}
\pscustom[linestyle=none,fillstyle=solid,fillcolor=curcolor]
{
\newpath
\moveto(361.86988403,415.07710297)
\curveto(361.88987587,414.9970991)(361.88987587,414.90709919)(361.86988403,414.80710297)
\curveto(361.84987591,414.70709939)(361.81487594,414.64209946)(361.76488403,414.61210297)
\curveto(361.71487604,414.57209953)(361.63987612,414.54209956)(361.53988403,414.52210297)
\curveto(361.44987631,414.51209959)(361.34487641,414.5020996)(361.22488403,414.49210297)
\lineto(360.87988403,414.49210297)
\curveto(360.76987699,414.5020996)(360.66987709,414.50709959)(360.57988403,414.50710297)
\lineto(356.91988403,414.50710297)
\lineto(356.70988403,414.50710297)
\curveto(356.64988111,414.50709959)(356.59488116,414.4970996)(356.54488403,414.47710297)
\curveto(356.46488129,414.43709966)(356.41488134,414.3970997)(356.39488403,414.35710297)
\curveto(356.37488138,414.33709976)(356.3548814,414.2970998)(356.33488403,414.23710297)
\curveto(356.31488144,414.18709991)(356.30988145,414.13709996)(356.31988403,414.08710297)
\curveto(356.33988142,414.02710007)(356.34988141,413.96710013)(356.34988403,413.90710297)
\curveto(356.3598814,413.85710024)(356.37488138,413.8021003)(356.39488403,413.74210297)
\curveto(356.47488128,413.5021006)(356.56988119,413.3021008)(356.67988403,413.14210297)
\curveto(356.79988096,412.99210111)(356.9598808,412.85710124)(357.15988403,412.73710297)
\curveto(357.23988052,412.68710141)(357.31988044,412.65210145)(357.39988403,412.63210297)
\curveto(357.48988027,412.62210148)(357.57988018,412.6021015)(357.66988403,412.57210297)
\curveto(357.74988001,412.55210155)(357.8598799,412.53710156)(357.99988403,412.52710297)
\curveto(358.13987962,412.51710158)(358.2598795,412.52210158)(358.35988403,412.54210297)
\lineto(358.49488403,412.54210297)
\curveto(358.59487916,412.56210154)(358.68487907,412.58210152)(358.76488403,412.60210297)
\curveto(358.8548789,412.63210147)(358.93987882,412.66210144)(359.01988403,412.69210297)
\curveto(359.11987864,412.74210136)(359.22987853,412.80710129)(359.34988403,412.88710297)
\curveto(359.47987828,412.96710113)(359.57487818,413.04710105)(359.63488403,413.12710297)
\curveto(359.68487807,413.1971009)(359.73487802,413.26210084)(359.78488403,413.32210297)
\curveto(359.84487791,413.39210071)(359.91487784,413.44210066)(359.99488403,413.47210297)
\curveto(360.09487766,413.52210058)(360.21987754,413.54210056)(360.36988403,413.53210297)
\lineto(360.80488403,413.53210297)
\lineto(360.98488403,413.53210297)
\curveto(361.0548767,413.54210056)(361.11487664,413.53710056)(361.16488403,413.51710297)
\lineto(361.31488403,413.51710297)
\curveto(361.41487634,413.4971006)(361.48487627,413.47210063)(361.52488403,413.44210297)
\curveto(361.56487619,413.42210068)(361.58487617,413.37710072)(361.58488403,413.30710297)
\curveto(361.59487616,413.23710086)(361.58987617,413.17710092)(361.56988403,413.12710297)
\curveto(361.51987624,412.98710111)(361.46487629,412.86210124)(361.40488403,412.75210297)
\curveto(361.34487641,412.64210146)(361.27487648,412.53210157)(361.19488403,412.42210297)
\curveto(360.97487678,412.09210201)(360.72487703,411.82710227)(360.44488403,411.62710297)
\curveto(360.16487759,411.42710267)(359.81487794,411.25710284)(359.39488403,411.11710297)
\curveto(359.28487847,411.07710302)(359.17487858,411.05210305)(359.06488403,411.04210297)
\curveto(358.9548788,411.03210307)(358.83987892,411.01210309)(358.71988403,410.98210297)
\curveto(358.67987908,410.97210313)(358.63487912,410.97210313)(358.58488403,410.98210297)
\curveto(358.54487921,410.98210312)(358.50487925,410.97710312)(358.46488403,410.96710297)
\lineto(358.29988403,410.96710297)
\curveto(358.24987951,410.94710315)(358.18987957,410.94210316)(358.11988403,410.95210297)
\curveto(358.0598797,410.95210315)(358.00487975,410.95710314)(357.95488403,410.96710297)
\curveto(357.87487988,410.97710312)(357.80487995,410.97710312)(357.74488403,410.96710297)
\curveto(357.68488007,410.95710314)(357.61988014,410.96210314)(357.54988403,410.98210297)
\curveto(357.49988026,411.0021031)(357.44488031,411.01210309)(357.38488403,411.01210297)
\curveto(357.32488043,411.01210309)(357.26988049,411.02210308)(357.21988403,411.04210297)
\curveto(357.10988065,411.06210304)(356.99988076,411.08710301)(356.88988403,411.11710297)
\curveto(356.77988098,411.13710296)(356.67988108,411.17210293)(356.58988403,411.22210297)
\curveto(356.47988128,411.26210284)(356.37488138,411.2971028)(356.27488403,411.32710297)
\curveto(356.18488157,411.36710273)(356.09988166,411.41210269)(356.01988403,411.46210297)
\curveto(355.69988206,411.66210244)(355.41488234,411.89210221)(355.16488403,412.15210297)
\curveto(354.91488284,412.42210168)(354.70988305,412.73210137)(354.54988403,413.08210297)
\curveto(354.49988326,413.19210091)(354.4598833,413.3021008)(354.42988403,413.41210297)
\curveto(354.39988336,413.53210057)(354.3598834,413.65210045)(354.30988403,413.77210297)
\curveto(354.29988346,413.81210029)(354.29488346,413.84710025)(354.29488403,413.87710297)
\curveto(354.29488346,413.91710018)(354.28988347,413.95710014)(354.27988403,413.99710297)
\curveto(354.23988352,414.11709998)(354.21488354,414.24709985)(354.20488403,414.38710297)
\lineto(354.17488403,414.80710297)
\curveto(354.17488358,414.85709924)(354.16988359,414.91209919)(354.15988403,414.97210297)
\curveto(354.1598836,415.03209907)(354.16488359,415.08709901)(354.17488403,415.13710297)
\lineto(354.17488403,415.31710297)
\lineto(354.21988403,415.67710297)
\curveto(354.2598835,415.84709825)(354.29488346,416.01209809)(354.32488403,416.17210297)
\curveto(354.3548834,416.33209777)(354.39988336,416.48209762)(354.45988403,416.62210297)
\curveto(354.88988287,417.66209644)(355.61988214,418.3970957)(356.64988403,418.82710297)
\curveto(356.78988097,418.88709521)(356.92988083,418.92709517)(357.06988403,418.94710297)
\curveto(357.21988054,418.97709512)(357.37488038,419.01209509)(357.53488403,419.05210297)
\curveto(357.61488014,419.06209504)(357.68988007,419.06709503)(357.75988403,419.06710297)
\curveto(357.82987993,419.06709503)(357.90487985,419.07209503)(357.98488403,419.08210297)
\curveto(358.49487926,419.09209501)(358.92987883,419.03209507)(359.28988403,418.90210297)
\curveto(359.6598781,418.78209532)(359.98987777,418.62209548)(360.27988403,418.42210297)
\curveto(360.36987739,418.36209574)(360.4598773,418.29209581)(360.54988403,418.21210297)
\curveto(360.63987712,418.14209596)(360.71987704,418.06709603)(360.78988403,417.98710297)
\curveto(360.81987694,417.93709616)(360.8598769,417.8970962)(360.90988403,417.86710297)
\curveto(360.98987677,417.75709634)(361.06487669,417.64209646)(361.13488403,417.52210297)
\curveto(361.20487655,417.41209669)(361.27987648,417.2970968)(361.35988403,417.17710297)
\curveto(361.40987635,417.08709701)(361.44987631,416.99209711)(361.47988403,416.89210297)
\curveto(361.51987624,416.8020973)(361.5598762,416.7020974)(361.59988403,416.59210297)
\curveto(361.64987611,416.46209764)(361.68987607,416.32709777)(361.71988403,416.18710297)
\curveto(361.74987601,416.04709805)(361.78487597,415.90709819)(361.82488403,415.76710297)
\curveto(361.84487591,415.68709841)(361.84987591,415.5970985)(361.83988403,415.49710297)
\curveto(361.83987592,415.40709869)(361.84987591,415.32209878)(361.86988403,415.24210297)
\lineto(361.86988403,415.07710297)
\moveto(359.61988403,415.96210297)
\curveto(359.68987807,416.06209804)(359.69487806,416.18209792)(359.63488403,416.32210297)
\curveto(359.58487817,416.47209763)(359.54487821,416.58209752)(359.51488403,416.65210297)
\curveto(359.37487838,416.92209718)(359.18987857,417.12709697)(358.95988403,417.26710297)
\curveto(358.72987903,417.41709668)(358.40987935,417.4970966)(357.99988403,417.50710297)
\curveto(357.96987979,417.48709661)(357.93487982,417.48209662)(357.89488403,417.49210297)
\curveto(357.8548799,417.5020966)(357.81987994,417.5020966)(357.78988403,417.49210297)
\curveto(357.73988002,417.47209663)(357.68488007,417.45709664)(357.62488403,417.44710297)
\curveto(357.56488019,417.44709665)(357.50988025,417.43709666)(357.45988403,417.41710297)
\curveto(357.01988074,417.27709682)(356.69488106,417.0020971)(356.48488403,416.59210297)
\curveto(356.46488129,416.55209755)(356.43988132,416.4970976)(356.40988403,416.42710297)
\curveto(356.38988137,416.36709773)(356.37488138,416.3020978)(356.36488403,416.23210297)
\curveto(356.3548814,416.17209793)(356.3548814,416.11209799)(356.36488403,416.05210297)
\curveto(356.38488137,415.99209811)(356.41988134,415.94209816)(356.46988403,415.90210297)
\curveto(356.54988121,415.85209825)(356.6598811,415.82709827)(356.79988403,415.82710297)
\lineto(357.20488403,415.82710297)
\lineto(358.86988403,415.82710297)
\lineto(359.30488403,415.82710297)
\curveto(359.46487829,415.83709826)(359.56987819,415.88209822)(359.61988403,415.96210297)
}
}
{
\newrgbcolor{curcolor}{0 0 0}
\pscustom[linestyle=none,fillstyle=solid,fillcolor=curcolor]
{
\newpath
\moveto(367.54316528,419.06710297)
\curveto(368.14315948,419.08709501)(368.64315898,419.0020951)(369.04316528,418.81210297)
\curveto(369.44315818,418.62209548)(369.75815786,418.34209576)(369.98816528,417.97210297)
\curveto(370.05815756,417.86209624)(370.11315751,417.74209636)(370.15316528,417.61210297)
\curveto(370.19315743,417.49209661)(370.23315739,417.36709673)(370.27316528,417.23710297)
\curveto(370.29315733,417.15709694)(370.30315732,417.08209702)(370.30316528,417.01210297)
\curveto(370.31315731,416.94209716)(370.32815729,416.87209723)(370.34816528,416.80210297)
\curveto(370.34815727,416.74209736)(370.35315727,416.7020974)(370.36316528,416.68210297)
\curveto(370.38315724,416.54209756)(370.39315723,416.3970977)(370.39316528,416.24710297)
\lineto(370.39316528,415.81210297)
\lineto(370.39316528,414.47710297)
\lineto(370.39316528,412.04710297)
\curveto(370.39315723,411.85710224)(370.38815723,411.67210243)(370.37816528,411.49210297)
\curveto(370.37815724,411.32210278)(370.30815731,411.21210289)(370.16816528,411.16210297)
\curveto(370.10815751,411.14210296)(370.03815758,411.13210297)(369.95816528,411.13210297)
\lineto(369.71816528,411.13210297)
\lineto(368.90816528,411.13210297)
\curveto(368.78815883,411.13210297)(368.67815894,411.13710296)(368.57816528,411.14710297)
\curveto(368.48815913,411.16710293)(368.4181592,411.21210289)(368.36816528,411.28210297)
\curveto(368.32815929,411.34210276)(368.30315932,411.41710268)(368.29316528,411.50710297)
\lineto(368.29316528,411.82210297)
\lineto(368.29316528,412.87210297)
\lineto(368.29316528,415.10710297)
\curveto(368.29315933,415.47709862)(368.27815934,415.81709828)(368.24816528,416.12710297)
\curveto(368.2181594,416.44709765)(368.12815949,416.71709738)(367.97816528,416.93710297)
\curveto(367.83815978,417.13709696)(367.63315999,417.27709682)(367.36316528,417.35710297)
\curveto(367.31316031,417.37709672)(367.25816036,417.38709671)(367.19816528,417.38710297)
\curveto(367.14816047,417.38709671)(367.09316053,417.3970967)(367.03316528,417.41710297)
\curveto(366.98316064,417.42709667)(366.9181607,417.42709667)(366.83816528,417.41710297)
\curveto(366.76816085,417.41709668)(366.71316091,417.41209669)(366.67316528,417.40210297)
\curveto(366.63316099,417.39209671)(366.59816102,417.38709671)(366.56816528,417.38710297)
\curveto(366.53816108,417.38709671)(366.50816111,417.38209672)(366.47816528,417.37210297)
\curveto(366.24816137,417.31209679)(366.06316156,417.23209687)(365.92316528,417.13210297)
\curveto(365.60316202,416.9020972)(365.41316221,416.56709753)(365.35316528,416.12710297)
\curveto(365.29316233,415.68709841)(365.26316236,415.19209891)(365.26316528,414.64210297)
\lineto(365.26316528,412.76710297)
\lineto(365.26316528,411.85210297)
\lineto(365.26316528,411.58210297)
\curveto(365.26316236,411.49210261)(365.24816237,411.41710268)(365.21816528,411.35710297)
\curveto(365.16816245,411.24710285)(365.08816253,411.18210292)(364.97816528,411.16210297)
\curveto(364.86816275,411.14210296)(364.73316289,411.13210297)(364.57316528,411.13210297)
\lineto(363.82316528,411.13210297)
\curveto(363.71316391,411.13210297)(363.60316402,411.13710296)(363.49316528,411.14710297)
\curveto(363.38316424,411.15710294)(363.30316432,411.19210291)(363.25316528,411.25210297)
\curveto(363.18316444,411.34210276)(363.14816447,411.47210263)(363.14816528,411.64210297)
\curveto(363.15816446,411.81210229)(363.16316446,411.97210213)(363.16316528,412.12210297)
\lineto(363.16316528,414.16210297)
\lineto(363.16316528,417.46210297)
\lineto(363.16316528,418.22710297)
\lineto(363.16316528,418.52710297)
\curveto(363.17316445,418.61709548)(363.20316442,418.69209541)(363.25316528,418.75210297)
\curveto(363.27316435,418.78209532)(363.30316432,418.8020953)(363.34316528,418.81210297)
\curveto(363.39316423,418.83209527)(363.44316418,418.84709525)(363.49316528,418.85710297)
\lineto(363.56816528,418.85710297)
\curveto(363.618164,418.86709523)(363.66816395,418.87209523)(363.71816528,418.87210297)
\lineto(363.88316528,418.87210297)
\lineto(364.51316528,418.87210297)
\curveto(364.59316303,418.87209523)(364.66816295,418.86709523)(364.73816528,418.85710297)
\curveto(364.8181628,418.85709524)(364.88816273,418.84709525)(364.94816528,418.82710297)
\curveto(365.0181626,418.7970953)(365.06316256,418.75209535)(365.08316528,418.69210297)
\curveto(365.11316251,418.63209547)(365.13816248,418.56209554)(365.15816528,418.48210297)
\curveto(365.16816245,418.44209566)(365.16816245,418.40709569)(365.15816528,418.37710297)
\curveto(365.15816246,418.34709575)(365.16816245,418.31709578)(365.18816528,418.28710297)
\curveto(365.20816241,418.23709586)(365.2231624,418.20709589)(365.23316528,418.19710297)
\curveto(365.25316237,418.18709591)(365.27816234,418.17209593)(365.30816528,418.15210297)
\curveto(365.4181622,418.14209596)(365.50816211,418.17709592)(365.57816528,418.25710297)
\curveto(365.64816197,418.34709575)(365.7231619,418.41709568)(365.80316528,418.46710297)
\curveto(366.07316155,418.66709543)(366.37316125,418.82709527)(366.70316528,418.94710297)
\curveto(366.79316083,418.97709512)(366.88316074,418.9970951)(366.97316528,419.00710297)
\curveto(367.07316055,419.01709508)(367.17816044,419.03209507)(367.28816528,419.05210297)
\curveto(367.3181603,419.06209504)(367.36316026,419.06209504)(367.42316528,419.05210297)
\curveto(367.48316014,419.05209505)(367.5231601,419.05709504)(367.54316528,419.06710297)
}
}
{
\newrgbcolor{curcolor}{0 0 0}
\pscustom[linestyle=none,fillstyle=solid,fillcolor=curcolor]
{
}
}
{
\newrgbcolor{curcolor}{0 0 0}
\pscustom[linestyle=none,fillstyle=solid,fillcolor=curcolor]
{
\newpath
\moveto(383.54957153,415.07710297)
\curveto(383.56956337,414.9970991)(383.56956337,414.90709919)(383.54957153,414.80710297)
\curveto(383.52956341,414.70709939)(383.49456344,414.64209946)(383.44457153,414.61210297)
\curveto(383.39456354,414.57209953)(383.31956362,414.54209956)(383.21957153,414.52210297)
\curveto(383.12956381,414.51209959)(383.02456391,414.5020996)(382.90457153,414.49210297)
\lineto(382.55957153,414.49210297)
\curveto(382.44956449,414.5020996)(382.34956459,414.50709959)(382.25957153,414.50710297)
\lineto(378.59957153,414.50710297)
\lineto(378.38957153,414.50710297)
\curveto(378.32956861,414.50709959)(378.27456866,414.4970996)(378.22457153,414.47710297)
\curveto(378.14456879,414.43709966)(378.09456884,414.3970997)(378.07457153,414.35710297)
\curveto(378.05456888,414.33709976)(378.0345689,414.2970998)(378.01457153,414.23710297)
\curveto(377.99456894,414.18709991)(377.98956895,414.13709996)(377.99957153,414.08710297)
\curveto(378.01956892,414.02710007)(378.02956891,413.96710013)(378.02957153,413.90710297)
\curveto(378.0395689,413.85710024)(378.05456888,413.8021003)(378.07457153,413.74210297)
\curveto(378.15456878,413.5021006)(378.24956869,413.3021008)(378.35957153,413.14210297)
\curveto(378.47956846,412.99210111)(378.6395683,412.85710124)(378.83957153,412.73710297)
\curveto(378.91956802,412.68710141)(378.99956794,412.65210145)(379.07957153,412.63210297)
\curveto(379.16956777,412.62210148)(379.25956768,412.6021015)(379.34957153,412.57210297)
\curveto(379.42956751,412.55210155)(379.5395674,412.53710156)(379.67957153,412.52710297)
\curveto(379.81956712,412.51710158)(379.939567,412.52210158)(380.03957153,412.54210297)
\lineto(380.17457153,412.54210297)
\curveto(380.27456666,412.56210154)(380.36456657,412.58210152)(380.44457153,412.60210297)
\curveto(380.5345664,412.63210147)(380.61956632,412.66210144)(380.69957153,412.69210297)
\curveto(380.79956614,412.74210136)(380.90956603,412.80710129)(381.02957153,412.88710297)
\curveto(381.15956578,412.96710113)(381.25456568,413.04710105)(381.31457153,413.12710297)
\curveto(381.36456557,413.1971009)(381.41456552,413.26210084)(381.46457153,413.32210297)
\curveto(381.52456541,413.39210071)(381.59456534,413.44210066)(381.67457153,413.47210297)
\curveto(381.77456516,413.52210058)(381.89956504,413.54210056)(382.04957153,413.53210297)
\lineto(382.48457153,413.53210297)
\lineto(382.66457153,413.53210297)
\curveto(382.7345642,413.54210056)(382.79456414,413.53710056)(382.84457153,413.51710297)
\lineto(382.99457153,413.51710297)
\curveto(383.09456384,413.4971006)(383.16456377,413.47210063)(383.20457153,413.44210297)
\curveto(383.24456369,413.42210068)(383.26456367,413.37710072)(383.26457153,413.30710297)
\curveto(383.27456366,413.23710086)(383.26956367,413.17710092)(383.24957153,413.12710297)
\curveto(383.19956374,412.98710111)(383.14456379,412.86210124)(383.08457153,412.75210297)
\curveto(383.02456391,412.64210146)(382.95456398,412.53210157)(382.87457153,412.42210297)
\curveto(382.65456428,412.09210201)(382.40456453,411.82710227)(382.12457153,411.62710297)
\curveto(381.84456509,411.42710267)(381.49456544,411.25710284)(381.07457153,411.11710297)
\curveto(380.96456597,411.07710302)(380.85456608,411.05210305)(380.74457153,411.04210297)
\curveto(380.6345663,411.03210307)(380.51956642,411.01210309)(380.39957153,410.98210297)
\curveto(380.35956658,410.97210313)(380.31456662,410.97210313)(380.26457153,410.98210297)
\curveto(380.22456671,410.98210312)(380.18456675,410.97710312)(380.14457153,410.96710297)
\lineto(379.97957153,410.96710297)
\curveto(379.92956701,410.94710315)(379.86956707,410.94210316)(379.79957153,410.95210297)
\curveto(379.7395672,410.95210315)(379.68456725,410.95710314)(379.63457153,410.96710297)
\curveto(379.55456738,410.97710312)(379.48456745,410.97710312)(379.42457153,410.96710297)
\curveto(379.36456757,410.95710314)(379.29956764,410.96210314)(379.22957153,410.98210297)
\curveto(379.17956776,411.0021031)(379.12456781,411.01210309)(379.06457153,411.01210297)
\curveto(379.00456793,411.01210309)(378.94956799,411.02210308)(378.89957153,411.04210297)
\curveto(378.78956815,411.06210304)(378.67956826,411.08710301)(378.56957153,411.11710297)
\curveto(378.45956848,411.13710296)(378.35956858,411.17210293)(378.26957153,411.22210297)
\curveto(378.15956878,411.26210284)(378.05456888,411.2971028)(377.95457153,411.32710297)
\curveto(377.86456907,411.36710273)(377.77956916,411.41210269)(377.69957153,411.46210297)
\curveto(377.37956956,411.66210244)(377.09456984,411.89210221)(376.84457153,412.15210297)
\curveto(376.59457034,412.42210168)(376.38957055,412.73210137)(376.22957153,413.08210297)
\curveto(376.17957076,413.19210091)(376.1395708,413.3021008)(376.10957153,413.41210297)
\curveto(376.07957086,413.53210057)(376.0395709,413.65210045)(375.98957153,413.77210297)
\curveto(375.97957096,413.81210029)(375.97457096,413.84710025)(375.97457153,413.87710297)
\curveto(375.97457096,413.91710018)(375.96957097,413.95710014)(375.95957153,413.99710297)
\curveto(375.91957102,414.11709998)(375.89457104,414.24709985)(375.88457153,414.38710297)
\lineto(375.85457153,414.80710297)
\curveto(375.85457108,414.85709924)(375.84957109,414.91209919)(375.83957153,414.97210297)
\curveto(375.8395711,415.03209907)(375.84457109,415.08709901)(375.85457153,415.13710297)
\lineto(375.85457153,415.31710297)
\lineto(375.89957153,415.67710297)
\curveto(375.939571,415.84709825)(375.97457096,416.01209809)(376.00457153,416.17210297)
\curveto(376.0345709,416.33209777)(376.07957086,416.48209762)(376.13957153,416.62210297)
\curveto(376.56957037,417.66209644)(377.29956964,418.3970957)(378.32957153,418.82710297)
\curveto(378.46956847,418.88709521)(378.60956833,418.92709517)(378.74957153,418.94710297)
\curveto(378.89956804,418.97709512)(379.05456788,419.01209509)(379.21457153,419.05210297)
\curveto(379.29456764,419.06209504)(379.36956757,419.06709503)(379.43957153,419.06710297)
\curveto(379.50956743,419.06709503)(379.58456735,419.07209503)(379.66457153,419.08210297)
\curveto(380.17456676,419.09209501)(380.60956633,419.03209507)(380.96957153,418.90210297)
\curveto(381.3395656,418.78209532)(381.66956527,418.62209548)(381.95957153,418.42210297)
\curveto(382.04956489,418.36209574)(382.1395648,418.29209581)(382.22957153,418.21210297)
\curveto(382.31956462,418.14209596)(382.39956454,418.06709603)(382.46957153,417.98710297)
\curveto(382.49956444,417.93709616)(382.5395644,417.8970962)(382.58957153,417.86710297)
\curveto(382.66956427,417.75709634)(382.74456419,417.64209646)(382.81457153,417.52210297)
\curveto(382.88456405,417.41209669)(382.95956398,417.2970968)(383.03957153,417.17710297)
\curveto(383.08956385,417.08709701)(383.12956381,416.99209711)(383.15957153,416.89210297)
\curveto(383.19956374,416.8020973)(383.2395637,416.7020974)(383.27957153,416.59210297)
\curveto(383.32956361,416.46209764)(383.36956357,416.32709777)(383.39957153,416.18710297)
\curveto(383.42956351,416.04709805)(383.46456347,415.90709819)(383.50457153,415.76710297)
\curveto(383.52456341,415.68709841)(383.52956341,415.5970985)(383.51957153,415.49710297)
\curveto(383.51956342,415.40709869)(383.52956341,415.32209878)(383.54957153,415.24210297)
\lineto(383.54957153,415.07710297)
\moveto(381.29957153,415.96210297)
\curveto(381.36956557,416.06209804)(381.37456556,416.18209792)(381.31457153,416.32210297)
\curveto(381.26456567,416.47209763)(381.22456571,416.58209752)(381.19457153,416.65210297)
\curveto(381.05456588,416.92209718)(380.86956607,417.12709697)(380.63957153,417.26710297)
\curveto(380.40956653,417.41709668)(380.08956685,417.4970966)(379.67957153,417.50710297)
\curveto(379.64956729,417.48709661)(379.61456732,417.48209662)(379.57457153,417.49210297)
\curveto(379.5345674,417.5020966)(379.49956744,417.5020966)(379.46957153,417.49210297)
\curveto(379.41956752,417.47209663)(379.36456757,417.45709664)(379.30457153,417.44710297)
\curveto(379.24456769,417.44709665)(379.18956775,417.43709666)(379.13957153,417.41710297)
\curveto(378.69956824,417.27709682)(378.37456856,417.0020971)(378.16457153,416.59210297)
\curveto(378.14456879,416.55209755)(378.11956882,416.4970976)(378.08957153,416.42710297)
\curveto(378.06956887,416.36709773)(378.05456888,416.3020978)(378.04457153,416.23210297)
\curveto(378.0345689,416.17209793)(378.0345689,416.11209799)(378.04457153,416.05210297)
\curveto(378.06456887,415.99209811)(378.09956884,415.94209816)(378.14957153,415.90210297)
\curveto(378.22956871,415.85209825)(378.3395686,415.82709827)(378.47957153,415.82710297)
\lineto(378.88457153,415.82710297)
\lineto(380.54957153,415.82710297)
\lineto(380.98457153,415.82710297)
\curveto(381.14456579,415.83709826)(381.24956569,415.88209822)(381.29957153,415.96210297)
}
}
{
\newrgbcolor{curcolor}{0 0 0}
\pscustom[linestyle=none,fillstyle=solid,fillcolor=curcolor]
{
\newpath
\moveto(385.30785278,421.82710297)
\lineto(386.40285278,421.82710297)
\curveto(386.5028503,421.82709227)(386.5978502,421.82209228)(386.68785278,421.81210297)
\curveto(386.77785002,421.8020923)(386.84784995,421.77209233)(386.89785278,421.72210297)
\curveto(386.95784984,421.65209245)(386.98784981,421.55709254)(386.98785278,421.43710297)
\curveto(386.9978498,421.32709277)(387.0028498,421.21209289)(387.00285278,421.09210297)
\lineto(387.00285278,419.75710297)
\lineto(387.00285278,414.37210297)
\lineto(387.00285278,412.07710297)
\lineto(387.00285278,411.65710297)
\curveto(387.01284979,411.50710259)(386.99284981,411.39210271)(386.94285278,411.31210297)
\curveto(386.89284991,411.23210287)(386.80285,411.17710292)(386.67285278,411.14710297)
\curveto(386.61285019,411.12710297)(386.54285026,411.12210298)(386.46285278,411.13210297)
\curveto(386.39285041,411.14210296)(386.32285048,411.14710295)(386.25285278,411.14710297)
\lineto(385.53285278,411.14710297)
\curveto(385.42285138,411.14710295)(385.32285148,411.15210295)(385.23285278,411.16210297)
\curveto(385.14285166,411.17210293)(385.06785173,411.2021029)(385.00785278,411.25210297)
\curveto(384.94785185,411.3021028)(384.91285189,411.37710272)(384.90285278,411.47710297)
\lineto(384.90285278,411.80710297)
\lineto(384.90285278,413.14210297)
\lineto(384.90285278,418.76710297)
\lineto(384.90285278,420.80710297)
\curveto(384.9028519,420.93709316)(384.8978519,421.09209301)(384.88785278,421.27210297)
\curveto(384.88785191,421.45209265)(384.91285189,421.58209252)(384.96285278,421.66210297)
\curveto(384.98285182,421.7020924)(385.00785179,421.73209237)(385.03785278,421.75210297)
\lineto(385.15785278,421.81210297)
\curveto(385.17785162,421.81209229)(385.2028516,421.81209229)(385.23285278,421.81210297)
\curveto(385.26285154,421.82209228)(385.28785151,421.82709227)(385.30785278,421.82710297)
}
}
{
\newrgbcolor{curcolor}{0 0 0}
\pscustom[linestyle=none,fillstyle=solid,fillcolor=curcolor]
{
}
}
{
\newrgbcolor{curcolor}{0 0 0}
\pscustom[linestyle=none,fillstyle=solid,fillcolor=curcolor]
{
\newpath
\moveto(395.79519653,419.08210297)
\curveto(396.54519203,419.102095)(397.19519138,419.01709508)(397.74519653,418.82710297)
\curveto(398.30519027,418.64709545)(398.73018985,418.33209577)(399.02019653,417.88210297)
\curveto(399.09018949,417.77209633)(399.15018943,417.65709644)(399.20019653,417.53710297)
\curveto(399.26018932,417.42709667)(399.31018927,417.3020968)(399.35019653,417.16210297)
\curveto(399.37018921,417.102097)(399.3801892,417.03709706)(399.38019653,416.96710297)
\curveto(399.3801892,416.8970972)(399.37018921,416.83709726)(399.35019653,416.78710297)
\curveto(399.31018927,416.72709737)(399.25518932,416.68709741)(399.18519653,416.66710297)
\curveto(399.13518944,416.64709745)(399.0751895,416.63709746)(399.00519653,416.63710297)
\lineto(398.79519653,416.63710297)
\lineto(398.13519653,416.63710297)
\curveto(398.06519051,416.63709746)(397.99519058,416.63209747)(397.92519653,416.62210297)
\curveto(397.85519072,416.62209748)(397.79019079,416.63209747)(397.73019653,416.65210297)
\curveto(397.63019095,416.67209743)(397.55519102,416.71209739)(397.50519653,416.77210297)
\curveto(397.45519112,416.83209727)(397.41019117,416.89209721)(397.37019653,416.95210297)
\lineto(397.25019653,417.16210297)
\curveto(397.22019136,417.24209686)(397.17019141,417.30709679)(397.10019653,417.35710297)
\curveto(397.00019158,417.43709666)(396.90019168,417.4970966)(396.80019653,417.53710297)
\curveto(396.71019187,417.57709652)(396.59519198,417.61209649)(396.45519653,417.64210297)
\curveto(396.38519219,417.66209644)(396.2801923,417.67709642)(396.14019653,417.68710297)
\curveto(396.01019257,417.6970964)(395.91019267,417.69209641)(395.84019653,417.67210297)
\lineto(395.73519653,417.67210297)
\lineto(395.58519653,417.64210297)
\curveto(395.54519303,417.64209646)(395.50019308,417.63709646)(395.45019653,417.62710297)
\curveto(395.2801933,417.57709652)(395.14019344,417.50709659)(395.03019653,417.41710297)
\curveto(394.93019365,417.33709676)(394.86019372,417.21209689)(394.82019653,417.04210297)
\curveto(394.80019378,416.97209713)(394.80019378,416.90709719)(394.82019653,416.84710297)
\curveto(394.84019374,416.78709731)(394.86019372,416.73709736)(394.88019653,416.69710297)
\curveto(394.95019363,416.57709752)(395.03019355,416.48209762)(395.12019653,416.41210297)
\curveto(395.22019336,416.34209776)(395.33519324,416.28209782)(395.46519653,416.23210297)
\curveto(395.65519292,416.15209795)(395.86019272,416.08209802)(396.08019653,416.02210297)
\lineto(396.77019653,415.87210297)
\curveto(397.01019157,415.83209827)(397.24019134,415.78209832)(397.46019653,415.72210297)
\curveto(397.69019089,415.67209843)(397.90519067,415.60709849)(398.10519653,415.52710297)
\curveto(398.19519038,415.48709861)(398.2801903,415.45209865)(398.36019653,415.42210297)
\curveto(398.45019013,415.4020987)(398.53519004,415.36709873)(398.61519653,415.31710297)
\curveto(398.80518977,415.1970989)(398.9751896,415.06709903)(399.12519653,414.92710297)
\curveto(399.28518929,414.78709931)(399.41018917,414.61209949)(399.50019653,414.40210297)
\curveto(399.53018905,414.33209977)(399.55518902,414.26209984)(399.57519653,414.19210297)
\curveto(399.59518898,414.12209998)(399.61518896,414.04710005)(399.63519653,413.96710297)
\curveto(399.64518893,413.90710019)(399.65018893,413.81210029)(399.65019653,413.68210297)
\curveto(399.66018892,413.56210054)(399.66018892,413.46710063)(399.65019653,413.39710297)
\lineto(399.65019653,413.32210297)
\curveto(399.63018895,413.26210084)(399.61518896,413.2021009)(399.60519653,413.14210297)
\curveto(399.60518897,413.09210101)(399.60018898,413.04210106)(399.59019653,412.99210297)
\curveto(399.52018906,412.69210141)(399.41018917,412.42710167)(399.26019653,412.19710297)
\curveto(399.10018948,411.95710214)(398.90518967,411.76210234)(398.67519653,411.61210297)
\curveto(398.44519013,411.46210264)(398.18519039,411.33210277)(397.89519653,411.22210297)
\curveto(397.78519079,411.17210293)(397.66519091,411.13710296)(397.53519653,411.11710297)
\curveto(397.41519116,411.097103)(397.29519128,411.07210303)(397.17519653,411.04210297)
\curveto(397.08519149,411.02210308)(396.99019159,411.01210309)(396.89019653,411.01210297)
\curveto(396.80019178,411.0021031)(396.71019187,410.98710311)(396.62019653,410.96710297)
\lineto(396.35019653,410.96710297)
\curveto(396.29019229,410.94710315)(396.18519239,410.93710316)(396.03519653,410.93710297)
\curveto(395.89519268,410.93710316)(395.79519278,410.94710315)(395.73519653,410.96710297)
\curveto(395.70519287,410.96710313)(395.67019291,410.97210313)(395.63019653,410.98210297)
\lineto(395.52519653,410.98210297)
\curveto(395.40519317,411.0021031)(395.28519329,411.01710308)(395.16519653,411.02710297)
\curveto(395.04519353,411.03710306)(394.93019365,411.05710304)(394.82019653,411.08710297)
\curveto(394.43019415,411.1971029)(394.08519449,411.32210278)(393.78519653,411.46210297)
\curveto(393.48519509,411.61210249)(393.23019535,411.83210227)(393.02019653,412.12210297)
\curveto(392.8801957,412.31210179)(392.76019582,412.53210157)(392.66019653,412.78210297)
\curveto(392.64019594,412.84210126)(392.62019596,412.92210118)(392.60019653,413.02210297)
\curveto(392.580196,413.07210103)(392.56519601,413.14210096)(392.55519653,413.23210297)
\curveto(392.54519603,413.32210078)(392.55019603,413.3971007)(392.57019653,413.45710297)
\curveto(392.60019598,413.52710057)(392.65019593,413.57710052)(392.72019653,413.60710297)
\curveto(392.77019581,413.62710047)(392.83019575,413.63710046)(392.90019653,413.63710297)
\lineto(393.12519653,413.63710297)
\lineto(393.83019653,413.63710297)
\lineto(394.07019653,413.63710297)
\curveto(394.15019443,413.63710046)(394.22019436,413.62710047)(394.28019653,413.60710297)
\curveto(394.39019419,413.56710053)(394.46019412,413.5021006)(394.49019653,413.41210297)
\curveto(394.53019405,413.32210078)(394.575194,413.22710087)(394.62519653,413.12710297)
\curveto(394.64519393,413.07710102)(394.6801939,413.01210109)(394.73019653,412.93210297)
\curveto(394.79019379,412.85210125)(394.84019374,412.8021013)(394.88019653,412.78210297)
\curveto(395.00019358,412.68210142)(395.11519346,412.6021015)(395.22519653,412.54210297)
\curveto(395.33519324,412.49210161)(395.4751931,412.44210166)(395.64519653,412.39210297)
\curveto(395.69519288,412.37210173)(395.74519283,412.36210174)(395.79519653,412.36210297)
\curveto(395.84519273,412.37210173)(395.89519268,412.37210173)(395.94519653,412.36210297)
\curveto(396.02519255,412.34210176)(396.11019247,412.33210177)(396.20019653,412.33210297)
\curveto(396.30019228,412.34210176)(396.38519219,412.35710174)(396.45519653,412.37710297)
\curveto(396.50519207,412.38710171)(396.55019203,412.39210171)(396.59019653,412.39210297)
\curveto(396.64019194,412.39210171)(396.69019189,412.4021017)(396.74019653,412.42210297)
\curveto(396.8801917,412.47210163)(397.00519157,412.53210157)(397.11519653,412.60210297)
\curveto(397.23519134,412.67210143)(397.33019125,412.76210134)(397.40019653,412.87210297)
\curveto(397.45019113,412.95210115)(397.49019109,413.07710102)(397.52019653,413.24710297)
\curveto(397.54019104,413.31710078)(397.54019104,413.38210072)(397.52019653,413.44210297)
\curveto(397.50019108,413.5021006)(397.4801911,413.55210055)(397.46019653,413.59210297)
\curveto(397.39019119,413.73210037)(397.30019128,413.83710026)(397.19019653,413.90710297)
\curveto(397.09019149,413.97710012)(396.97019161,414.04210006)(396.83019653,414.10210297)
\curveto(396.64019194,414.18209992)(396.44019214,414.24709985)(396.23019653,414.29710297)
\curveto(396.02019256,414.34709975)(395.81019277,414.4020997)(395.60019653,414.46210297)
\curveto(395.52019306,414.48209962)(395.43519314,414.4970996)(395.34519653,414.50710297)
\curveto(395.26519331,414.51709958)(395.18519339,414.53209957)(395.10519653,414.55210297)
\curveto(394.78519379,414.64209946)(394.4801941,414.72709937)(394.19019653,414.80710297)
\curveto(393.90019468,414.8970992)(393.63519494,415.02709907)(393.39519653,415.19710297)
\curveto(393.11519546,415.3970987)(392.91019567,415.66709843)(392.78019653,416.00710297)
\curveto(392.76019582,416.07709802)(392.74019584,416.17209793)(392.72019653,416.29210297)
\curveto(392.70019588,416.36209774)(392.68519589,416.44709765)(392.67519653,416.54710297)
\curveto(392.66519591,416.64709745)(392.67019591,416.73709736)(392.69019653,416.81710297)
\curveto(392.71019587,416.86709723)(392.71519586,416.90709719)(392.70519653,416.93710297)
\curveto(392.69519588,416.97709712)(392.70019588,417.02209708)(392.72019653,417.07210297)
\curveto(392.74019584,417.18209692)(392.76019582,417.28209682)(392.78019653,417.37210297)
\curveto(392.81019577,417.47209663)(392.84519573,417.56709653)(392.88519653,417.65710297)
\curveto(393.01519556,417.94709615)(393.19519538,418.18209592)(393.42519653,418.36210297)
\curveto(393.65519492,418.54209556)(393.91519466,418.68709541)(394.20519653,418.79710297)
\curveto(394.31519426,418.84709525)(394.43019415,418.88209522)(394.55019653,418.90210297)
\curveto(394.67019391,418.93209517)(394.79519378,418.96209514)(394.92519653,418.99210297)
\curveto(394.98519359,419.01209509)(395.04519353,419.02209508)(395.10519653,419.02210297)
\lineto(395.28519653,419.05210297)
\curveto(395.36519321,419.06209504)(395.45019313,419.06709503)(395.54019653,419.06710297)
\curveto(395.63019295,419.06709503)(395.71519286,419.07209503)(395.79519653,419.08210297)
}
}
{
\newrgbcolor{curcolor}{0 0 0}
\pscustom[linestyle=none,fillstyle=solid,fillcolor=curcolor]
{
\newpath
\moveto(402.98183716,421.72210297)
\curveto(403.05183421,421.64209246)(403.08683417,421.52209258)(403.08683716,421.36210297)
\lineto(403.08683716,420.89710297)
\lineto(403.08683716,420.49210297)
\curveto(403.08683417,420.35209375)(403.05183421,420.25709384)(402.98183716,420.20710297)
\curveto(402.92183434,420.15709394)(402.84183442,420.12709397)(402.74183716,420.11710297)
\curveto(402.65183461,420.10709399)(402.55183471,420.102094)(402.44183716,420.10210297)
\lineto(401.60183716,420.10210297)
\curveto(401.49183577,420.102094)(401.39183587,420.10709399)(401.30183716,420.11710297)
\curveto(401.22183604,420.12709397)(401.15183611,420.15709394)(401.09183716,420.20710297)
\curveto(401.05183621,420.23709386)(401.02183624,420.29209381)(401.00183716,420.37210297)
\curveto(400.99183627,420.46209364)(400.98183628,420.55709354)(400.97183716,420.65710297)
\lineto(400.97183716,420.98710297)
\curveto(400.98183628,421.097093)(400.98683627,421.19209291)(400.98683716,421.27210297)
\lineto(400.98683716,421.48210297)
\curveto(400.99683626,421.55209255)(401.01683624,421.61209249)(401.04683716,421.66210297)
\curveto(401.06683619,421.7020924)(401.09183617,421.73209237)(401.12183716,421.75210297)
\lineto(401.24183716,421.81210297)
\curveto(401.261836,421.81209229)(401.28683597,421.81209229)(401.31683716,421.81210297)
\curveto(401.34683591,421.82209228)(401.37183589,421.82709227)(401.39183716,421.82710297)
\lineto(402.48683716,421.82710297)
\curveto(402.58683467,421.82709227)(402.68183458,421.82209228)(402.77183716,421.81210297)
\curveto(402.8618344,421.8020923)(402.93183433,421.77209233)(402.98183716,421.72210297)
\moveto(403.08683716,411.95710297)
\curveto(403.08683417,411.75710234)(403.08183418,411.58710251)(403.07183716,411.44710297)
\curveto(403.0618342,411.30710279)(402.97183429,411.21210289)(402.80183716,411.16210297)
\curveto(402.74183452,411.14210296)(402.67683458,411.13210297)(402.60683716,411.13210297)
\curveto(402.53683472,411.14210296)(402.4618348,411.14710295)(402.38183716,411.14710297)
\lineto(401.54183716,411.14710297)
\curveto(401.45183581,411.14710295)(401.3618359,411.15210295)(401.27183716,411.16210297)
\curveto(401.19183607,411.17210293)(401.13183613,411.2021029)(401.09183716,411.25210297)
\curveto(401.03183623,411.32210278)(400.99683626,411.40710269)(400.98683716,411.50710297)
\lineto(400.98683716,411.85210297)
\lineto(400.98683716,418.18210297)
\lineto(400.98683716,418.48210297)
\curveto(400.98683627,418.58209552)(401.00683625,418.66209544)(401.04683716,418.72210297)
\curveto(401.10683615,418.79209531)(401.19183607,418.83709526)(401.30183716,418.85710297)
\curveto(401.32183594,418.86709523)(401.34683591,418.86709523)(401.37683716,418.85710297)
\curveto(401.41683584,418.85709524)(401.44683581,418.86209524)(401.46683716,418.87210297)
\lineto(402.21683716,418.87210297)
\lineto(402.41183716,418.87210297)
\curveto(402.49183477,418.88209522)(402.5568347,418.88209522)(402.60683716,418.87210297)
\lineto(402.72683716,418.87210297)
\curveto(402.78683447,418.85209525)(402.84183442,418.83709526)(402.89183716,418.82710297)
\curveto(402.94183432,418.81709528)(402.98183428,418.78709531)(403.01183716,418.73710297)
\curveto(403.05183421,418.68709541)(403.07183419,418.61709548)(403.07183716,418.52710297)
\curveto(403.08183418,418.43709566)(403.08683417,418.34209576)(403.08683716,418.24210297)
\lineto(403.08683716,411.95710297)
}
}
{
\newrgbcolor{curcolor}{0 0 0}
\pscustom[linestyle=none,fillstyle=solid,fillcolor=curcolor]
{
\newpath
\moveto(407.71902466,419.08210297)
\curveto(408.46902016,419.102095)(409.11901951,419.01709508)(409.66902466,418.82710297)
\curveto(410.2290184,418.64709545)(410.65401797,418.33209577)(410.94402466,417.88210297)
\curveto(411.01401761,417.77209633)(411.07401755,417.65709644)(411.12402466,417.53710297)
\curveto(411.18401744,417.42709667)(411.23401739,417.3020968)(411.27402466,417.16210297)
\curveto(411.29401733,417.102097)(411.30401732,417.03709706)(411.30402466,416.96710297)
\curveto(411.30401732,416.8970972)(411.29401733,416.83709726)(411.27402466,416.78710297)
\curveto(411.23401739,416.72709737)(411.17901745,416.68709741)(411.10902466,416.66710297)
\curveto(411.05901757,416.64709745)(410.99901763,416.63709746)(410.92902466,416.63710297)
\lineto(410.71902466,416.63710297)
\lineto(410.05902466,416.63710297)
\curveto(409.98901864,416.63709746)(409.91901871,416.63209747)(409.84902466,416.62210297)
\curveto(409.77901885,416.62209748)(409.71401891,416.63209747)(409.65402466,416.65210297)
\curveto(409.55401907,416.67209743)(409.47901915,416.71209739)(409.42902466,416.77210297)
\curveto(409.37901925,416.83209727)(409.33401929,416.89209721)(409.29402466,416.95210297)
\lineto(409.17402466,417.16210297)
\curveto(409.14401948,417.24209686)(409.09401953,417.30709679)(409.02402466,417.35710297)
\curveto(408.9240197,417.43709666)(408.8240198,417.4970966)(408.72402466,417.53710297)
\curveto(408.63401999,417.57709652)(408.51902011,417.61209649)(408.37902466,417.64210297)
\curveto(408.30902032,417.66209644)(408.20402042,417.67709642)(408.06402466,417.68710297)
\curveto(407.93402069,417.6970964)(407.83402079,417.69209641)(407.76402466,417.67210297)
\lineto(407.65902466,417.67210297)
\lineto(407.50902466,417.64210297)
\curveto(407.46902116,417.64209646)(407.4240212,417.63709646)(407.37402466,417.62710297)
\curveto(407.20402142,417.57709652)(407.06402156,417.50709659)(406.95402466,417.41710297)
\curveto(406.85402177,417.33709676)(406.78402184,417.21209689)(406.74402466,417.04210297)
\curveto(406.7240219,416.97209713)(406.7240219,416.90709719)(406.74402466,416.84710297)
\curveto(406.76402186,416.78709731)(406.78402184,416.73709736)(406.80402466,416.69710297)
\curveto(406.87402175,416.57709752)(406.95402167,416.48209762)(407.04402466,416.41210297)
\curveto(407.14402148,416.34209776)(407.25902137,416.28209782)(407.38902466,416.23210297)
\curveto(407.57902105,416.15209795)(407.78402084,416.08209802)(408.00402466,416.02210297)
\lineto(408.69402466,415.87210297)
\curveto(408.93401969,415.83209827)(409.16401946,415.78209832)(409.38402466,415.72210297)
\curveto(409.61401901,415.67209843)(409.8290188,415.60709849)(410.02902466,415.52710297)
\curveto(410.11901851,415.48709861)(410.20401842,415.45209865)(410.28402466,415.42210297)
\curveto(410.37401825,415.4020987)(410.45901817,415.36709873)(410.53902466,415.31710297)
\curveto(410.7290179,415.1970989)(410.89901773,415.06709903)(411.04902466,414.92710297)
\curveto(411.20901742,414.78709931)(411.33401729,414.61209949)(411.42402466,414.40210297)
\curveto(411.45401717,414.33209977)(411.47901715,414.26209984)(411.49902466,414.19210297)
\curveto(411.51901711,414.12209998)(411.53901709,414.04710005)(411.55902466,413.96710297)
\curveto(411.56901706,413.90710019)(411.57401705,413.81210029)(411.57402466,413.68210297)
\curveto(411.58401704,413.56210054)(411.58401704,413.46710063)(411.57402466,413.39710297)
\lineto(411.57402466,413.32210297)
\curveto(411.55401707,413.26210084)(411.53901709,413.2021009)(411.52902466,413.14210297)
\curveto(411.5290171,413.09210101)(411.5240171,413.04210106)(411.51402466,412.99210297)
\curveto(411.44401718,412.69210141)(411.33401729,412.42710167)(411.18402466,412.19710297)
\curveto(411.0240176,411.95710214)(410.8290178,411.76210234)(410.59902466,411.61210297)
\curveto(410.36901826,411.46210264)(410.10901852,411.33210277)(409.81902466,411.22210297)
\curveto(409.70901892,411.17210293)(409.58901904,411.13710296)(409.45902466,411.11710297)
\curveto(409.33901929,411.097103)(409.21901941,411.07210303)(409.09902466,411.04210297)
\curveto(409.00901962,411.02210308)(408.91401971,411.01210309)(408.81402466,411.01210297)
\curveto(408.7240199,411.0021031)(408.63401999,410.98710311)(408.54402466,410.96710297)
\lineto(408.27402466,410.96710297)
\curveto(408.21402041,410.94710315)(408.10902052,410.93710316)(407.95902466,410.93710297)
\curveto(407.81902081,410.93710316)(407.71902091,410.94710315)(407.65902466,410.96710297)
\curveto(407.629021,410.96710313)(407.59402103,410.97210313)(407.55402466,410.98210297)
\lineto(407.44902466,410.98210297)
\curveto(407.3290213,411.0021031)(407.20902142,411.01710308)(407.08902466,411.02710297)
\curveto(406.96902166,411.03710306)(406.85402177,411.05710304)(406.74402466,411.08710297)
\curveto(406.35402227,411.1971029)(406.00902262,411.32210278)(405.70902466,411.46210297)
\curveto(405.40902322,411.61210249)(405.15402347,411.83210227)(404.94402466,412.12210297)
\curveto(404.80402382,412.31210179)(404.68402394,412.53210157)(404.58402466,412.78210297)
\curveto(404.56402406,412.84210126)(404.54402408,412.92210118)(404.52402466,413.02210297)
\curveto(404.50402412,413.07210103)(404.48902414,413.14210096)(404.47902466,413.23210297)
\curveto(404.46902416,413.32210078)(404.47402415,413.3971007)(404.49402466,413.45710297)
\curveto(404.5240241,413.52710057)(404.57402405,413.57710052)(404.64402466,413.60710297)
\curveto(404.69402393,413.62710047)(404.75402387,413.63710046)(404.82402466,413.63710297)
\lineto(405.04902466,413.63710297)
\lineto(405.75402466,413.63710297)
\lineto(405.99402466,413.63710297)
\curveto(406.07402255,413.63710046)(406.14402248,413.62710047)(406.20402466,413.60710297)
\curveto(406.31402231,413.56710053)(406.38402224,413.5021006)(406.41402466,413.41210297)
\curveto(406.45402217,413.32210078)(406.49902213,413.22710087)(406.54902466,413.12710297)
\curveto(406.56902206,413.07710102)(406.60402202,413.01210109)(406.65402466,412.93210297)
\curveto(406.71402191,412.85210125)(406.76402186,412.8021013)(406.80402466,412.78210297)
\curveto(406.9240217,412.68210142)(407.03902159,412.6021015)(407.14902466,412.54210297)
\curveto(407.25902137,412.49210161)(407.39902123,412.44210166)(407.56902466,412.39210297)
\curveto(407.61902101,412.37210173)(407.66902096,412.36210174)(407.71902466,412.36210297)
\curveto(407.76902086,412.37210173)(407.81902081,412.37210173)(407.86902466,412.36210297)
\curveto(407.94902068,412.34210176)(408.03402059,412.33210177)(408.12402466,412.33210297)
\curveto(408.2240204,412.34210176)(408.30902032,412.35710174)(408.37902466,412.37710297)
\curveto(408.4290202,412.38710171)(408.47402015,412.39210171)(408.51402466,412.39210297)
\curveto(408.56402006,412.39210171)(408.61402001,412.4021017)(408.66402466,412.42210297)
\curveto(408.80401982,412.47210163)(408.9290197,412.53210157)(409.03902466,412.60210297)
\curveto(409.15901947,412.67210143)(409.25401937,412.76210134)(409.32402466,412.87210297)
\curveto(409.37401925,412.95210115)(409.41401921,413.07710102)(409.44402466,413.24710297)
\curveto(409.46401916,413.31710078)(409.46401916,413.38210072)(409.44402466,413.44210297)
\curveto(409.4240192,413.5021006)(409.40401922,413.55210055)(409.38402466,413.59210297)
\curveto(409.31401931,413.73210037)(409.2240194,413.83710026)(409.11402466,413.90710297)
\curveto(409.01401961,413.97710012)(408.89401973,414.04210006)(408.75402466,414.10210297)
\curveto(408.56402006,414.18209992)(408.36402026,414.24709985)(408.15402466,414.29710297)
\curveto(407.94402068,414.34709975)(407.73402089,414.4020997)(407.52402466,414.46210297)
\curveto(407.44402118,414.48209962)(407.35902127,414.4970996)(407.26902466,414.50710297)
\curveto(407.18902144,414.51709958)(407.10902152,414.53209957)(407.02902466,414.55210297)
\curveto(406.70902192,414.64209946)(406.40402222,414.72709937)(406.11402466,414.80710297)
\curveto(405.8240228,414.8970992)(405.55902307,415.02709907)(405.31902466,415.19710297)
\curveto(405.03902359,415.3970987)(404.83402379,415.66709843)(404.70402466,416.00710297)
\curveto(404.68402394,416.07709802)(404.66402396,416.17209793)(404.64402466,416.29210297)
\curveto(404.624024,416.36209774)(404.60902402,416.44709765)(404.59902466,416.54710297)
\curveto(404.58902404,416.64709745)(404.59402403,416.73709736)(404.61402466,416.81710297)
\curveto(404.63402399,416.86709723)(404.63902399,416.90709719)(404.62902466,416.93710297)
\curveto(404.61902401,416.97709712)(404.624024,417.02209708)(404.64402466,417.07210297)
\curveto(404.66402396,417.18209692)(404.68402394,417.28209682)(404.70402466,417.37210297)
\curveto(404.73402389,417.47209663)(404.76902386,417.56709653)(404.80902466,417.65710297)
\curveto(404.93902369,417.94709615)(405.11902351,418.18209592)(405.34902466,418.36210297)
\curveto(405.57902305,418.54209556)(405.83902279,418.68709541)(406.12902466,418.79710297)
\curveto(406.23902239,418.84709525)(406.35402227,418.88209522)(406.47402466,418.90210297)
\curveto(406.59402203,418.93209517)(406.71902191,418.96209514)(406.84902466,418.99210297)
\curveto(406.90902172,419.01209509)(406.96902166,419.02209508)(407.02902466,419.02210297)
\lineto(407.20902466,419.05210297)
\curveto(407.28902134,419.06209504)(407.37402125,419.06709503)(407.46402466,419.06710297)
\curveto(407.55402107,419.06709503)(407.63902099,419.07209503)(407.71902466,419.08210297)
}
}
{
\newrgbcolor{curcolor}{0 0 0}
\pscustom[linestyle=none,fillstyle=solid,fillcolor=curcolor]
{
\newpath
\moveto(413.85566528,421.18210297)
\lineto(414.86066528,421.18210297)
\curveto(415.0106623,421.18209292)(415.14066217,421.17209293)(415.25066528,421.15210297)
\curveto(415.37066194,421.14209296)(415.45566185,421.08209302)(415.50566528,420.97210297)
\curveto(415.52566178,420.92209318)(415.53566177,420.86209324)(415.53566528,420.79210297)
\lineto(415.53566528,420.58210297)
\lineto(415.53566528,419.90710297)
\curveto(415.53566177,419.85709424)(415.53066178,419.7970943)(415.52066528,419.72710297)
\curveto(415.52066179,419.66709443)(415.52566178,419.61209449)(415.53566528,419.56210297)
\lineto(415.53566528,419.39710297)
\curveto(415.53566177,419.31709478)(415.54066177,419.24209486)(415.55066528,419.17210297)
\curveto(415.56066175,419.11209499)(415.58566172,419.05709504)(415.62566528,419.00710297)
\curveto(415.69566161,418.91709518)(415.82066149,418.86709523)(416.00066528,418.85710297)
\lineto(416.54066528,418.85710297)
\lineto(416.72066528,418.85710297)
\curveto(416.78066053,418.85709524)(416.83566047,418.84709525)(416.88566528,418.82710297)
\curveto(416.99566031,418.77709532)(417.05566025,418.68709541)(417.06566528,418.55710297)
\curveto(417.08566022,418.42709567)(417.09566021,418.28209582)(417.09566528,418.12210297)
\lineto(417.09566528,417.91210297)
\curveto(417.1056602,417.84209626)(417.10066021,417.78209632)(417.08066528,417.73210297)
\curveto(417.03066028,417.57209653)(416.92566038,417.48709661)(416.76566528,417.47710297)
\curveto(416.6056607,417.46709663)(416.42566088,417.46209664)(416.22566528,417.46210297)
\lineto(416.09066528,417.46210297)
\curveto(416.05066126,417.47209663)(416.01566129,417.47209663)(415.98566528,417.46210297)
\curveto(415.94566136,417.45209665)(415.9106614,417.44709665)(415.88066528,417.44710297)
\curveto(415.85066146,417.45709664)(415.82066149,417.45209665)(415.79066528,417.43210297)
\curveto(415.7106616,417.41209669)(415.65066166,417.36709673)(415.61066528,417.29710297)
\curveto(415.58066173,417.23709686)(415.55566175,417.16209694)(415.53566528,417.07210297)
\curveto(415.52566178,417.02209708)(415.52566178,416.96709713)(415.53566528,416.90710297)
\curveto(415.54566176,416.84709725)(415.54566176,416.79209731)(415.53566528,416.74210297)
\lineto(415.53566528,415.81210297)
\lineto(415.53566528,414.05710297)
\curveto(415.53566177,413.80710029)(415.54066177,413.58710051)(415.55066528,413.39710297)
\curveto(415.57066174,413.21710088)(415.63566167,413.05710104)(415.74566528,412.91710297)
\curveto(415.79566151,412.85710124)(415.86066145,412.81210129)(415.94066528,412.78210297)
\lineto(416.21066528,412.72210297)
\curveto(416.24066107,412.71210139)(416.27066104,412.70710139)(416.30066528,412.70710297)
\curveto(416.34066097,412.71710138)(416.37066094,412.71710138)(416.39066528,412.70710297)
\lineto(416.55566528,412.70710297)
\curveto(416.66566064,412.70710139)(416.76066055,412.7021014)(416.84066528,412.69210297)
\curveto(416.92066039,412.68210142)(416.98566032,412.64210146)(417.03566528,412.57210297)
\curveto(417.07566023,412.51210159)(417.09566021,412.43210167)(417.09566528,412.33210297)
\lineto(417.09566528,412.04710297)
\curveto(417.09566021,411.83710226)(417.09066022,411.64210246)(417.08066528,411.46210297)
\curveto(417.08066023,411.29210281)(417.00066031,411.17710292)(416.84066528,411.11710297)
\curveto(416.79066052,411.097103)(416.74566056,411.09210301)(416.70566528,411.10210297)
\curveto(416.66566064,411.102103)(416.62066069,411.09210301)(416.57066528,411.07210297)
\lineto(416.42066528,411.07210297)
\curveto(416.40066091,411.07210303)(416.37066094,411.07710302)(416.33066528,411.08710297)
\curveto(416.29066102,411.08710301)(416.25566105,411.08210302)(416.22566528,411.07210297)
\curveto(416.17566113,411.06210304)(416.12066119,411.06210304)(416.06066528,411.07210297)
\lineto(415.91066528,411.07210297)
\lineto(415.76066528,411.07210297)
\curveto(415.7106616,411.06210304)(415.66566164,411.06210304)(415.62566528,411.07210297)
\lineto(415.46066528,411.07210297)
\curveto(415.4106619,411.08210302)(415.35566195,411.08710301)(415.29566528,411.08710297)
\curveto(415.23566207,411.08710301)(415.18066213,411.09210301)(415.13066528,411.10210297)
\curveto(415.06066225,411.11210299)(414.99566231,411.12210298)(414.93566528,411.13210297)
\lineto(414.75566528,411.16210297)
\curveto(414.64566266,411.19210291)(414.54066277,411.22710287)(414.44066528,411.26710297)
\curveto(414.34066297,411.30710279)(414.24566306,411.35210275)(414.15566528,411.40210297)
\lineto(414.06566528,411.46210297)
\curveto(414.03566327,411.49210261)(414.00066331,411.52210258)(413.96066528,411.55210297)
\curveto(413.94066337,411.57210253)(413.91566339,411.59210251)(413.88566528,411.61210297)
\lineto(413.81066528,411.68710297)
\curveto(413.67066364,411.87710222)(413.56566374,412.08710201)(413.49566528,412.31710297)
\curveto(413.47566383,412.35710174)(413.46566384,412.39210171)(413.46566528,412.42210297)
\curveto(413.47566383,412.46210164)(413.47566383,412.50710159)(413.46566528,412.55710297)
\curveto(413.45566385,412.57710152)(413.45066386,412.6021015)(413.45066528,412.63210297)
\curveto(413.45066386,412.66210144)(413.44566386,412.68710141)(413.43566528,412.70710297)
\lineto(413.43566528,412.85710297)
\curveto(413.42566388,412.8971012)(413.42066389,412.94210116)(413.42066528,412.99210297)
\curveto(413.43066388,413.04210106)(413.43566387,413.09210101)(413.43566528,413.14210297)
\lineto(413.43566528,413.71210297)
\lineto(413.43566528,415.94710297)
\lineto(413.43566528,416.74210297)
\lineto(413.43566528,416.95210297)
\curveto(413.44566386,417.02209708)(413.44066387,417.08709701)(413.42066528,417.14710297)
\curveto(413.38066393,417.28709681)(413.310664,417.37709672)(413.21066528,417.41710297)
\curveto(413.10066421,417.46709663)(412.96066435,417.48209662)(412.79066528,417.46210297)
\curveto(412.62066469,417.44209666)(412.47566483,417.45709664)(412.35566528,417.50710297)
\curveto(412.27566503,417.53709656)(412.22566508,417.58209652)(412.20566528,417.64210297)
\curveto(412.18566512,417.7020964)(412.16566514,417.77709632)(412.14566528,417.86710297)
\lineto(412.14566528,418.18210297)
\curveto(412.14566516,418.36209574)(412.15566515,418.50709559)(412.17566528,418.61710297)
\curveto(412.19566511,418.72709537)(412.28066503,418.8020953)(412.43066528,418.84210297)
\curveto(412.47066484,418.86209524)(412.5106648,418.86709523)(412.55066528,418.85710297)
\lineto(412.68566528,418.85710297)
\curveto(412.83566447,418.85709524)(412.97566433,418.86209524)(413.10566528,418.87210297)
\curveto(413.23566407,418.89209521)(413.32566398,418.95209515)(413.37566528,419.05210297)
\curveto(413.4056639,419.12209498)(413.42066389,419.2020949)(413.42066528,419.29210297)
\curveto(413.43066388,419.38209472)(413.43566387,419.47209463)(413.43566528,419.56210297)
\lineto(413.43566528,420.49210297)
\lineto(413.43566528,420.74710297)
\curveto(413.43566387,420.83709326)(413.44566386,420.91209319)(413.46566528,420.97210297)
\curveto(413.51566379,421.07209303)(413.59066372,421.13709296)(413.69066528,421.16710297)
\curveto(413.7106636,421.17709292)(413.73566357,421.17709292)(413.76566528,421.16710297)
\curveto(413.8056635,421.16709293)(413.83566347,421.17209293)(413.85566528,421.18210297)
}
}
{
\newrgbcolor{curcolor}{0 0 0}
\pscustom[linestyle=none,fillstyle=solid,fillcolor=curcolor]
{
\newpath
\moveto(425.44410278,415.07710297)
\curveto(425.46409462,414.9970991)(425.46409462,414.90709919)(425.44410278,414.80710297)
\curveto(425.42409466,414.70709939)(425.38909469,414.64209946)(425.33910278,414.61210297)
\curveto(425.28909479,414.57209953)(425.21409487,414.54209956)(425.11410278,414.52210297)
\curveto(425.02409506,414.51209959)(424.91909516,414.5020996)(424.79910278,414.49210297)
\lineto(424.45410278,414.49210297)
\curveto(424.34409574,414.5020996)(424.24409584,414.50709959)(424.15410278,414.50710297)
\lineto(420.49410278,414.50710297)
\lineto(420.28410278,414.50710297)
\curveto(420.22409986,414.50709959)(420.16909991,414.4970996)(420.11910278,414.47710297)
\curveto(420.03910004,414.43709966)(419.98910009,414.3970997)(419.96910278,414.35710297)
\curveto(419.94910013,414.33709976)(419.92910015,414.2970998)(419.90910278,414.23710297)
\curveto(419.88910019,414.18709991)(419.8841002,414.13709996)(419.89410278,414.08710297)
\curveto(419.91410017,414.02710007)(419.92410016,413.96710013)(419.92410278,413.90710297)
\curveto(419.93410015,413.85710024)(419.94910013,413.8021003)(419.96910278,413.74210297)
\curveto(420.04910003,413.5021006)(420.14409994,413.3021008)(420.25410278,413.14210297)
\curveto(420.37409971,412.99210111)(420.53409955,412.85710124)(420.73410278,412.73710297)
\curveto(420.81409927,412.68710141)(420.89409919,412.65210145)(420.97410278,412.63210297)
\curveto(421.06409902,412.62210148)(421.15409893,412.6021015)(421.24410278,412.57210297)
\curveto(421.32409876,412.55210155)(421.43409865,412.53710156)(421.57410278,412.52710297)
\curveto(421.71409837,412.51710158)(421.83409825,412.52210158)(421.93410278,412.54210297)
\lineto(422.06910278,412.54210297)
\curveto(422.16909791,412.56210154)(422.25909782,412.58210152)(422.33910278,412.60210297)
\curveto(422.42909765,412.63210147)(422.51409757,412.66210144)(422.59410278,412.69210297)
\curveto(422.69409739,412.74210136)(422.80409728,412.80710129)(422.92410278,412.88710297)
\curveto(423.05409703,412.96710113)(423.14909693,413.04710105)(423.20910278,413.12710297)
\curveto(423.25909682,413.1971009)(423.30909677,413.26210084)(423.35910278,413.32210297)
\curveto(423.41909666,413.39210071)(423.48909659,413.44210066)(423.56910278,413.47210297)
\curveto(423.66909641,413.52210058)(423.79409629,413.54210056)(423.94410278,413.53210297)
\lineto(424.37910278,413.53210297)
\lineto(424.55910278,413.53210297)
\curveto(424.62909545,413.54210056)(424.68909539,413.53710056)(424.73910278,413.51710297)
\lineto(424.88910278,413.51710297)
\curveto(424.98909509,413.4971006)(425.05909502,413.47210063)(425.09910278,413.44210297)
\curveto(425.13909494,413.42210068)(425.15909492,413.37710072)(425.15910278,413.30710297)
\curveto(425.16909491,413.23710086)(425.16409492,413.17710092)(425.14410278,413.12710297)
\curveto(425.09409499,412.98710111)(425.03909504,412.86210124)(424.97910278,412.75210297)
\curveto(424.91909516,412.64210146)(424.84909523,412.53210157)(424.76910278,412.42210297)
\curveto(424.54909553,412.09210201)(424.29909578,411.82710227)(424.01910278,411.62710297)
\curveto(423.73909634,411.42710267)(423.38909669,411.25710284)(422.96910278,411.11710297)
\curveto(422.85909722,411.07710302)(422.74909733,411.05210305)(422.63910278,411.04210297)
\curveto(422.52909755,411.03210307)(422.41409767,411.01210309)(422.29410278,410.98210297)
\curveto(422.25409783,410.97210313)(422.20909787,410.97210313)(422.15910278,410.98210297)
\curveto(422.11909796,410.98210312)(422.079098,410.97710312)(422.03910278,410.96710297)
\lineto(421.87410278,410.96710297)
\curveto(421.82409826,410.94710315)(421.76409832,410.94210316)(421.69410278,410.95210297)
\curveto(421.63409845,410.95210315)(421.5790985,410.95710314)(421.52910278,410.96710297)
\curveto(421.44909863,410.97710312)(421.3790987,410.97710312)(421.31910278,410.96710297)
\curveto(421.25909882,410.95710314)(421.19409889,410.96210314)(421.12410278,410.98210297)
\curveto(421.07409901,411.0021031)(421.01909906,411.01210309)(420.95910278,411.01210297)
\curveto(420.89909918,411.01210309)(420.84409924,411.02210308)(420.79410278,411.04210297)
\curveto(420.6840994,411.06210304)(420.57409951,411.08710301)(420.46410278,411.11710297)
\curveto(420.35409973,411.13710296)(420.25409983,411.17210293)(420.16410278,411.22210297)
\curveto(420.05410003,411.26210284)(419.94910013,411.2971028)(419.84910278,411.32710297)
\curveto(419.75910032,411.36710273)(419.67410041,411.41210269)(419.59410278,411.46210297)
\curveto(419.27410081,411.66210244)(418.98910109,411.89210221)(418.73910278,412.15210297)
\curveto(418.48910159,412.42210168)(418.2841018,412.73210137)(418.12410278,413.08210297)
\curveto(418.07410201,413.19210091)(418.03410205,413.3021008)(418.00410278,413.41210297)
\curveto(417.97410211,413.53210057)(417.93410215,413.65210045)(417.88410278,413.77210297)
\curveto(417.87410221,413.81210029)(417.86910221,413.84710025)(417.86910278,413.87710297)
\curveto(417.86910221,413.91710018)(417.86410222,413.95710014)(417.85410278,413.99710297)
\curveto(417.81410227,414.11709998)(417.78910229,414.24709985)(417.77910278,414.38710297)
\lineto(417.74910278,414.80710297)
\curveto(417.74910233,414.85709924)(417.74410234,414.91209919)(417.73410278,414.97210297)
\curveto(417.73410235,415.03209907)(417.73910234,415.08709901)(417.74910278,415.13710297)
\lineto(417.74910278,415.31710297)
\lineto(417.79410278,415.67710297)
\curveto(417.83410225,415.84709825)(417.86910221,416.01209809)(417.89910278,416.17210297)
\curveto(417.92910215,416.33209777)(417.97410211,416.48209762)(418.03410278,416.62210297)
\curveto(418.46410162,417.66209644)(419.19410089,418.3970957)(420.22410278,418.82710297)
\curveto(420.36409972,418.88709521)(420.50409958,418.92709517)(420.64410278,418.94710297)
\curveto(420.79409929,418.97709512)(420.94909913,419.01209509)(421.10910278,419.05210297)
\curveto(421.18909889,419.06209504)(421.26409882,419.06709503)(421.33410278,419.06710297)
\curveto(421.40409868,419.06709503)(421.4790986,419.07209503)(421.55910278,419.08210297)
\curveto(422.06909801,419.09209501)(422.50409758,419.03209507)(422.86410278,418.90210297)
\curveto(423.23409685,418.78209532)(423.56409652,418.62209548)(423.85410278,418.42210297)
\curveto(423.94409614,418.36209574)(424.03409605,418.29209581)(424.12410278,418.21210297)
\curveto(424.21409587,418.14209596)(424.29409579,418.06709603)(424.36410278,417.98710297)
\curveto(424.39409569,417.93709616)(424.43409565,417.8970962)(424.48410278,417.86710297)
\curveto(424.56409552,417.75709634)(424.63909544,417.64209646)(424.70910278,417.52210297)
\curveto(424.7790953,417.41209669)(424.85409523,417.2970968)(424.93410278,417.17710297)
\curveto(424.9840951,417.08709701)(425.02409506,416.99209711)(425.05410278,416.89210297)
\curveto(425.09409499,416.8020973)(425.13409495,416.7020974)(425.17410278,416.59210297)
\curveto(425.22409486,416.46209764)(425.26409482,416.32709777)(425.29410278,416.18710297)
\curveto(425.32409476,416.04709805)(425.35909472,415.90709819)(425.39910278,415.76710297)
\curveto(425.41909466,415.68709841)(425.42409466,415.5970985)(425.41410278,415.49710297)
\curveto(425.41409467,415.40709869)(425.42409466,415.32209878)(425.44410278,415.24210297)
\lineto(425.44410278,415.07710297)
\moveto(423.19410278,415.96210297)
\curveto(423.26409682,416.06209804)(423.26909681,416.18209792)(423.20910278,416.32210297)
\curveto(423.15909692,416.47209763)(423.11909696,416.58209752)(423.08910278,416.65210297)
\curveto(422.94909713,416.92209718)(422.76409732,417.12709697)(422.53410278,417.26710297)
\curveto(422.30409778,417.41709668)(421.9840981,417.4970966)(421.57410278,417.50710297)
\curveto(421.54409854,417.48709661)(421.50909857,417.48209662)(421.46910278,417.49210297)
\curveto(421.42909865,417.5020966)(421.39409869,417.5020966)(421.36410278,417.49210297)
\curveto(421.31409877,417.47209663)(421.25909882,417.45709664)(421.19910278,417.44710297)
\curveto(421.13909894,417.44709665)(421.084099,417.43709666)(421.03410278,417.41710297)
\curveto(420.59409949,417.27709682)(420.26909981,417.0020971)(420.05910278,416.59210297)
\curveto(420.03910004,416.55209755)(420.01410007,416.4970976)(419.98410278,416.42710297)
\curveto(419.96410012,416.36709773)(419.94910013,416.3020978)(419.93910278,416.23210297)
\curveto(419.92910015,416.17209793)(419.92910015,416.11209799)(419.93910278,416.05210297)
\curveto(419.95910012,415.99209811)(419.99410009,415.94209816)(420.04410278,415.90210297)
\curveto(420.12409996,415.85209825)(420.23409985,415.82709827)(420.37410278,415.82710297)
\lineto(420.77910278,415.82710297)
\lineto(422.44410278,415.82710297)
\lineto(422.87910278,415.82710297)
\curveto(423.03909704,415.83709826)(423.14409694,415.88209822)(423.19410278,415.96210297)
}
}
{
\newrgbcolor{curcolor}{0 0 0}
\pscustom[linestyle=none,fillstyle=solid,fillcolor=curcolor]
{
\newpath
\moveto(431.14738403,419.06710297)
\curveto(431.51737843,419.07709502)(431.8423781,419.03709506)(432.12238403,418.94710297)
\curveto(432.40237754,418.85709524)(432.6473773,418.73209537)(432.85738403,418.57210297)
\curveto(432.93737701,418.51209559)(433.00737694,418.44209566)(433.06738403,418.36210297)
\curveto(433.13737681,418.28209582)(433.21237673,418.2020959)(433.29238403,418.12210297)
\curveto(433.31237663,418.102096)(433.3423766,418.07209603)(433.38238403,418.03210297)
\curveto(433.43237651,418.0020961)(433.48237646,417.9970961)(433.53238403,418.01710297)
\curveto(433.6423763,418.04709605)(433.7473762,418.11709598)(433.84738403,418.22710297)
\curveto(433.947376,418.34709575)(434.0423759,418.43709566)(434.13238403,418.49710297)
\curveto(434.27237567,418.60709549)(434.42237552,418.6970954)(434.58238403,418.76710297)
\curveto(434.7423752,418.84709525)(434.92237502,418.92209518)(435.12238403,418.99210297)
\curveto(435.20237474,419.01209509)(435.29737465,419.02709507)(435.40738403,419.03710297)
\curveto(435.52737442,419.05709504)(435.6473743,419.06709503)(435.76738403,419.06710297)
\curveto(435.89737405,419.07709502)(436.01737393,419.07709502)(436.12738403,419.06710297)
\curveto(436.2473737,419.05709504)(436.35237359,419.04209506)(436.44238403,419.02210297)
\curveto(436.49237345,419.01209509)(436.53737341,419.00709509)(436.57738403,419.00710297)
\curveto(436.61737333,419.00709509)(436.66237328,418.9970951)(436.71238403,418.97710297)
\curveto(436.85237309,418.93709516)(436.98737296,418.8970952)(437.11738403,418.85710297)
\curveto(437.2473727,418.81709528)(437.36737258,418.76209534)(437.47738403,418.69210297)
\curveto(437.89737205,418.43209567)(438.21237173,418.05209605)(438.42238403,417.55210297)
\curveto(438.46237148,417.46209664)(438.49237145,417.36709673)(438.51238403,417.26710297)
\curveto(438.53237141,417.17709692)(438.55237139,417.08709701)(438.57238403,416.99710297)
\curveto(438.58237136,416.92709717)(438.58737136,416.86209724)(438.58738403,416.80210297)
\curveto(438.59737135,416.74209736)(438.60737134,416.68209742)(438.61738403,416.62210297)
\lineto(438.61738403,416.47210297)
\curveto(438.62737132,416.41209769)(438.62737132,416.34209776)(438.61738403,416.26210297)
\curveto(438.61737133,416.18209792)(438.61737133,416.10709799)(438.61738403,416.03710297)
\lineto(438.61738403,415.16710297)
\lineto(438.61738403,412.24210297)
\curveto(438.61737133,412.16210194)(438.61737133,412.06710203)(438.61738403,411.95710297)
\curveto(438.62737132,411.85710224)(438.62737132,411.75710234)(438.61738403,411.65710297)
\curveto(438.61737133,411.56710253)(438.60737134,411.47710262)(438.58738403,411.38710297)
\curveto(438.56737138,411.30710279)(438.53737141,411.25210285)(438.49738403,411.22210297)
\curveto(438.43737151,411.17210293)(438.35737159,411.14210296)(438.25738403,411.13210297)
\lineto(437.95738403,411.13210297)
\lineto(437.16238403,411.13210297)
\curveto(437.02237292,411.13210297)(436.89737305,411.14210296)(436.78738403,411.16210297)
\curveto(436.67737327,411.18210292)(436.60237334,411.23710286)(436.56238403,411.32710297)
\curveto(436.53237341,411.3971027)(436.51737343,411.47210263)(436.51738403,411.55210297)
\curveto(436.51737343,411.64210246)(436.51737343,411.72710237)(436.51738403,411.80710297)
\lineto(436.51738403,412.64710297)
\lineto(436.51738403,414.67210297)
\lineto(436.51738403,415.30210297)
\curveto(436.51737343,415.35209875)(436.51737343,415.40709869)(436.51738403,415.46710297)
\curveto(436.52737342,415.52709857)(436.52237342,415.58209852)(436.50238403,415.63210297)
\lineto(436.50238403,415.75210297)
\curveto(436.50237344,415.81209829)(436.50237344,415.87209823)(436.50238403,415.93210297)
\curveto(436.50237344,415.99209811)(436.49737345,416.05209805)(436.48738403,416.11210297)
\curveto(436.47737347,416.15209795)(436.47237347,416.19209791)(436.47238403,416.23210297)
\curveto(436.47237347,416.28209782)(436.46737348,416.32709777)(436.45738403,416.36710297)
\curveto(436.41737353,416.51709758)(436.37237357,416.64709745)(436.32238403,416.75710297)
\curveto(436.28237366,416.87709722)(436.21737373,416.98209712)(436.12738403,417.07210297)
\curveto(435.98737396,417.21209689)(435.81737413,417.31209679)(435.61738403,417.37210297)
\curveto(435.57737437,417.38209672)(435.5423744,417.38209672)(435.51238403,417.37210297)
\curveto(435.48237446,417.37209673)(435.4473745,417.38209672)(435.40738403,417.40210297)
\curveto(435.36737458,417.41209669)(435.31737463,417.41709668)(435.25738403,417.41710297)
\curveto(435.20737474,417.42709667)(435.15737479,417.42709667)(435.10738403,417.41710297)
\curveto(435.0473749,417.3970967)(434.98737496,417.38709671)(434.92738403,417.38710297)
\curveto(434.86737508,417.38709671)(434.80737514,417.37709672)(434.74738403,417.35710297)
\curveto(434.45737549,417.25709684)(434.2473757,417.10709699)(434.11738403,416.90710297)
\curveto(433.947376,416.67709742)(433.8423761,416.38709771)(433.80238403,416.03710297)
\curveto(433.77237617,415.6970984)(433.75737619,415.32209878)(433.75738403,414.91210297)
\lineto(433.75738403,412.93210297)
\lineto(433.75738403,411.82210297)
\lineto(433.75738403,411.52210297)
\curveto(433.75737619,411.42210268)(433.73237621,411.34210276)(433.68238403,411.28210297)
\curveto(433.63237631,411.21210289)(433.55737639,411.16710293)(433.45738403,411.14710297)
\curveto(433.36737658,411.13710296)(433.26237668,411.13210297)(433.14238403,411.13210297)
\lineto(432.33238403,411.13210297)
\lineto(432.06238403,411.13210297)
\curveto(431.98237796,411.14210296)(431.91237803,411.15710294)(431.85238403,411.17710297)
\curveto(431.75237819,411.22710287)(431.69237825,411.30710279)(431.67238403,411.41710297)
\curveto(431.66237828,411.52710257)(431.65737829,411.65210245)(431.65738403,411.79210297)
\lineto(431.65738403,413.06710297)
\lineto(431.65738403,415.42210297)
\curveto(431.65737829,415.71209839)(431.6473783,415.98709811)(431.62738403,416.24710297)
\curveto(431.60737834,416.50709759)(431.5423784,416.72209738)(431.43238403,416.89210297)
\curveto(431.35237859,417.03209707)(431.2473787,417.13709696)(431.11738403,417.20710297)
\curveto(430.99737895,417.27709682)(430.8473791,417.33709676)(430.66738403,417.38710297)
\curveto(430.62737932,417.3970967)(430.58737936,417.3970967)(430.54738403,417.38710297)
\curveto(430.50737944,417.38709671)(430.46237948,417.39209671)(430.41238403,417.40210297)
\curveto(430.30237964,417.42209668)(430.19737975,417.41209669)(430.09738403,417.37210297)
\curveto(430.07737987,417.37209673)(430.05737989,417.36709673)(430.03738403,417.35710297)
\lineto(429.97738403,417.35710297)
\curveto(429.81738013,417.30709679)(429.66238028,417.22209688)(429.51238403,417.10210297)
\curveto(429.35238059,416.98209712)(429.22738072,416.84209726)(429.13738403,416.68210297)
\curveto(429.05738089,416.53209757)(428.99738095,416.35709774)(428.95738403,416.15710297)
\curveto(428.92738102,415.96709813)(428.90738104,415.75709834)(428.89738403,415.52710297)
\lineto(428.89738403,414.77710297)
\lineto(428.89738403,412.75210297)
\lineto(428.89738403,411.83710297)
\lineto(428.89738403,411.56710297)
\curveto(428.89738105,411.47710262)(428.88238106,411.3971027)(428.85238403,411.32710297)
\curveto(428.81238113,411.23710286)(428.73738121,411.18210292)(428.62738403,411.16210297)
\curveto(428.51738143,411.14210296)(428.39238155,411.13210297)(428.25238403,411.13210297)
\lineto(427.47238403,411.13210297)
\lineto(427.17238403,411.13210297)
\curveto(427.08238286,411.14210296)(427.00738294,411.16710293)(426.94738403,411.20710297)
\curveto(426.85738309,411.25710284)(426.80738314,411.34710275)(426.79738403,411.47710297)
\lineto(426.79738403,411.91210297)
\lineto(426.79738403,413.66710297)
\lineto(426.79738403,417.32710297)
\lineto(426.79738403,418.22710297)
\lineto(426.79738403,418.51210297)
\curveto(426.80738314,418.6020955)(426.83238311,418.67709542)(426.87238403,418.73710297)
\curveto(426.92238302,418.7970953)(427.00238294,418.83709526)(427.11238403,418.85710297)
\lineto(427.20238403,418.85710297)
\curveto(427.25238269,418.86709523)(427.30238264,418.87209523)(427.35238403,418.87210297)
\lineto(427.51738403,418.87210297)
\lineto(428.13238403,418.87210297)
\curveto(428.21238173,418.87209523)(428.28738166,418.86709523)(428.35738403,418.85710297)
\curveto(428.43738151,418.85709524)(428.50738144,418.84709525)(428.56738403,418.82710297)
\curveto(428.6473813,418.7970953)(428.69738125,418.74709535)(428.71738403,418.67710297)
\curveto(428.7473812,418.60709549)(428.77238117,418.52709557)(428.79238403,418.43710297)
\curveto(428.80238114,418.40709569)(428.80238114,418.37709572)(428.79238403,418.34710297)
\curveto(428.79238115,418.32709577)(428.80238114,418.30709579)(428.82238403,418.28710297)
\curveto(428.83238111,418.25709584)(428.8423811,418.23209587)(428.85238403,418.21210297)
\curveto(428.87238107,418.2020959)(428.89238105,418.18709591)(428.91238403,418.16710297)
\curveto(429.03238091,418.15709594)(429.13238081,418.19209591)(429.21238403,418.27210297)
\curveto(429.29238065,418.36209574)(429.36738058,418.43209567)(429.43738403,418.48210297)
\curveto(429.57738037,418.58209552)(429.71738023,418.67209543)(429.85738403,418.75210297)
\curveto(430.00737994,418.83209527)(430.16737978,418.8970952)(430.33738403,418.94710297)
\curveto(430.42737952,418.97709512)(430.51737943,418.9970951)(430.60738403,419.00710297)
\curveto(430.69737925,419.01709508)(430.79237915,419.03209507)(430.89238403,419.05210297)
\curveto(430.92237902,419.06209504)(430.96737898,419.06209504)(431.02738403,419.05210297)
\curveto(431.08737886,419.05209505)(431.12737882,419.05709504)(431.14738403,419.06710297)
}
}
{
\newrgbcolor{curcolor}{0 0 0}
\pscustom[linestyle=none,fillstyle=solid,fillcolor=curcolor]
{
\newpath
\moveto(447.33613403,411.73210297)
\curveto(447.35612618,411.62210248)(447.36612617,411.51210259)(447.36613403,411.40210297)
\curveto(447.37612616,411.29210281)(447.32612621,411.21710288)(447.21613403,411.17710297)
\curveto(447.15612638,411.14710295)(447.08612645,411.13210297)(447.00613403,411.13210297)
\lineto(446.76613403,411.13210297)
\lineto(445.95613403,411.13210297)
\lineto(445.68613403,411.13210297)
\curveto(445.60612793,411.14210296)(445.541128,411.16710293)(445.49113403,411.20710297)
\curveto(445.42112812,411.24710285)(445.36612817,411.3021028)(445.32613403,411.37210297)
\curveto(445.29612824,411.45210265)(445.25112829,411.51710258)(445.19113403,411.56710297)
\curveto(445.17112837,411.58710251)(445.14612839,411.6021025)(445.11613403,411.61210297)
\curveto(445.08612845,411.63210247)(445.04612849,411.63710246)(444.99613403,411.62710297)
\curveto(444.94612859,411.60710249)(444.89612864,411.58210252)(444.84613403,411.55210297)
\curveto(444.80612873,411.52210258)(444.76112878,411.4971026)(444.71113403,411.47710297)
\curveto(444.66112888,411.43710266)(444.60612893,411.4021027)(444.54613403,411.37210297)
\lineto(444.36613403,411.28210297)
\curveto(444.2361293,411.22210288)(444.10112944,411.17210293)(443.96113403,411.13210297)
\curveto(443.82112972,411.102103)(443.67612986,411.06710303)(443.52613403,411.02710297)
\curveto(443.45613008,411.00710309)(443.38613015,410.9971031)(443.31613403,410.99710297)
\curveto(443.25613028,410.98710311)(443.19113035,410.97710312)(443.12113403,410.96710297)
\lineto(443.03113403,410.96710297)
\curveto(443.00113054,410.95710314)(442.97113057,410.95210315)(442.94113403,410.95210297)
\lineto(442.77613403,410.95210297)
\curveto(442.67613086,410.93210317)(442.57613096,410.93210317)(442.47613403,410.95210297)
\lineto(442.34113403,410.95210297)
\curveto(442.27113127,410.97210313)(442.20113134,410.98210312)(442.13113403,410.98210297)
\curveto(442.07113147,410.97210313)(442.01113153,410.97710312)(441.95113403,410.99710297)
\curveto(441.85113169,411.01710308)(441.75613178,411.03710306)(441.66613403,411.05710297)
\curveto(441.57613196,411.06710303)(441.49113205,411.09210301)(441.41113403,411.13210297)
\curveto(441.12113242,411.24210286)(440.87113267,411.38210272)(440.66113403,411.55210297)
\curveto(440.46113308,411.73210237)(440.30113324,411.96710213)(440.18113403,412.25710297)
\curveto(440.15113339,412.32710177)(440.12113342,412.4021017)(440.09113403,412.48210297)
\curveto(440.07113347,412.56210154)(440.05113349,412.64710145)(440.03113403,412.73710297)
\curveto(440.01113353,412.78710131)(440.00113354,412.83710126)(440.00113403,412.88710297)
\curveto(440.01113353,412.93710116)(440.01113353,412.98710111)(440.00113403,413.03710297)
\curveto(439.99113355,413.06710103)(439.98113356,413.12710097)(439.97113403,413.21710297)
\curveto(439.97113357,413.31710078)(439.97613356,413.38710071)(439.98613403,413.42710297)
\curveto(440.00613353,413.52710057)(440.01613352,413.61210049)(440.01613403,413.68210297)
\lineto(440.10613403,414.01210297)
\curveto(440.1361334,414.13209997)(440.17613336,414.23709986)(440.22613403,414.32710297)
\curveto(440.39613314,414.61709948)(440.59113295,414.83709926)(440.81113403,414.98710297)
\curveto(441.03113251,415.13709896)(441.31113223,415.26709883)(441.65113403,415.37710297)
\curveto(441.78113176,415.42709867)(441.91613162,415.46209864)(442.05613403,415.48210297)
\curveto(442.19613134,415.5020986)(442.3361312,415.52709857)(442.47613403,415.55710297)
\curveto(442.55613098,415.57709852)(442.6411309,415.58709851)(442.73113403,415.58710297)
\curveto(442.82113072,415.5970985)(442.91113063,415.61209849)(443.00113403,415.63210297)
\curveto(443.07113047,415.65209845)(443.1411304,415.65709844)(443.21113403,415.64710297)
\curveto(443.28113026,415.64709845)(443.35613018,415.65709844)(443.43613403,415.67710297)
\curveto(443.50613003,415.6970984)(443.57612996,415.70709839)(443.64613403,415.70710297)
\curveto(443.71612982,415.70709839)(443.79112975,415.71709838)(443.87113403,415.73710297)
\curveto(444.08112946,415.78709831)(444.27112927,415.82709827)(444.44113403,415.85710297)
\curveto(444.62112892,415.8970982)(444.78112876,415.98709811)(444.92113403,416.12710297)
\curveto(445.01112853,416.21709788)(445.07112847,416.31709778)(445.10113403,416.42710297)
\curveto(445.11112843,416.45709764)(445.11112843,416.48209762)(445.10113403,416.50210297)
\curveto(445.10112844,416.52209758)(445.10612843,416.54209756)(445.11613403,416.56210297)
\curveto(445.12612841,416.58209752)(445.13112841,416.61209749)(445.13113403,416.65210297)
\lineto(445.13113403,416.74210297)
\lineto(445.10113403,416.86210297)
\curveto(445.10112844,416.9020972)(445.09612844,416.93709716)(445.08613403,416.96710297)
\curveto(444.98612855,417.26709683)(444.77612876,417.47209663)(444.45613403,417.58210297)
\curveto(444.36612917,417.61209649)(444.25612928,417.63209647)(444.12613403,417.64210297)
\curveto(444.00612953,417.66209644)(443.88112966,417.66709643)(443.75113403,417.65710297)
\curveto(443.62112992,417.65709644)(443.49613004,417.64709645)(443.37613403,417.62710297)
\curveto(443.25613028,417.60709649)(443.15113039,417.58209652)(443.06113403,417.55210297)
\curveto(443.00113054,417.53209657)(442.9411306,417.5020966)(442.88113403,417.46210297)
\curveto(442.83113071,417.43209667)(442.78113076,417.3970967)(442.73113403,417.35710297)
\curveto(442.68113086,417.31709678)(442.62613091,417.26209684)(442.56613403,417.19210297)
\curveto(442.51613102,417.12209698)(442.48113106,417.05709704)(442.46113403,416.99710297)
\curveto(442.41113113,416.8970972)(442.36613117,416.8020973)(442.32613403,416.71210297)
\curveto(442.29613124,416.62209748)(442.22613131,416.56209754)(442.11613403,416.53210297)
\curveto(442.0361315,416.51209759)(441.95113159,416.5020976)(441.86113403,416.50210297)
\lineto(441.59113403,416.50210297)
\lineto(441.02113403,416.50210297)
\curveto(440.97113257,416.5020976)(440.92113262,416.4970976)(440.87113403,416.48710297)
\curveto(440.82113272,416.48709761)(440.77613276,416.49209761)(440.73613403,416.50210297)
\lineto(440.60113403,416.50210297)
\curveto(440.58113296,416.51209759)(440.55613298,416.51709758)(440.52613403,416.51710297)
\curveto(440.49613304,416.51709758)(440.47113307,416.52709757)(440.45113403,416.54710297)
\curveto(440.37113317,416.56709753)(440.31613322,416.63209747)(440.28613403,416.74210297)
\curveto(440.27613326,416.79209731)(440.27613326,416.84209726)(440.28613403,416.89210297)
\curveto(440.29613324,416.94209716)(440.30613323,416.98709711)(440.31613403,417.02710297)
\curveto(440.34613319,417.13709696)(440.37613316,417.23709686)(440.40613403,417.32710297)
\curveto(440.44613309,417.42709667)(440.49113305,417.51709658)(440.54113403,417.59710297)
\lineto(440.63113403,417.74710297)
\lineto(440.72113403,417.89710297)
\curveto(440.80113274,418.00709609)(440.90113264,418.11209599)(441.02113403,418.21210297)
\curveto(441.0411325,418.22209588)(441.07113247,418.24709585)(441.11113403,418.28710297)
\curveto(441.16113238,418.32709577)(441.20613233,418.36209574)(441.24613403,418.39210297)
\curveto(441.28613225,418.42209568)(441.33113221,418.45209565)(441.38113403,418.48210297)
\curveto(441.55113199,418.59209551)(441.73113181,418.67709542)(441.92113403,418.73710297)
\curveto(442.11113143,418.80709529)(442.30613123,418.87209523)(442.50613403,418.93210297)
\curveto(442.62613091,418.96209514)(442.75113079,418.98209512)(442.88113403,418.99210297)
\curveto(443.01113053,419.0020951)(443.1411304,419.02209508)(443.27113403,419.05210297)
\curveto(443.31113023,419.06209504)(443.37113017,419.06209504)(443.45113403,419.05210297)
\curveto(443.54113,419.04209506)(443.59612994,419.04709505)(443.61613403,419.06710297)
\curveto(444.02612951,419.07709502)(444.41612912,419.06209504)(444.78613403,419.02210297)
\curveto(445.16612837,418.98209512)(445.50612803,418.90709519)(445.80613403,418.79710297)
\curveto(446.11612742,418.68709541)(446.38112716,418.53709556)(446.60113403,418.34710297)
\curveto(446.82112672,418.16709593)(446.99112655,417.93209617)(447.11113403,417.64210297)
\curveto(447.18112636,417.47209663)(447.22112632,417.27709682)(447.23113403,417.05710297)
\curveto(447.2411263,416.83709726)(447.24612629,416.61209749)(447.24613403,416.38210297)
\lineto(447.24613403,413.03710297)
\lineto(447.24613403,412.45210297)
\curveto(447.24612629,412.26210184)(447.26612627,412.08710201)(447.30613403,411.92710297)
\curveto(447.31612622,411.8971022)(447.32112622,411.86210224)(447.32113403,411.82210297)
\curveto(447.32112622,411.79210231)(447.32612621,411.76210234)(447.33613403,411.73210297)
\moveto(445.13113403,414.04210297)
\curveto(445.1411284,414.09210001)(445.14612839,414.14709995)(445.14613403,414.20710297)
\curveto(445.14612839,414.27709982)(445.1411284,414.33709976)(445.13113403,414.38710297)
\curveto(445.11112843,414.44709965)(445.10112844,414.5020996)(445.10113403,414.55210297)
\curveto(445.10112844,414.6020995)(445.08112846,414.64209946)(445.04113403,414.67210297)
\curveto(444.99112855,414.71209939)(444.91612862,414.73209937)(444.81613403,414.73210297)
\curveto(444.77612876,414.72209938)(444.7411288,414.71209939)(444.71113403,414.70210297)
\curveto(444.68112886,414.7020994)(444.64612889,414.6970994)(444.60613403,414.68710297)
\curveto(444.536129,414.66709943)(444.46112908,414.65209945)(444.38113403,414.64210297)
\curveto(444.30112924,414.63209947)(444.22112932,414.61709948)(444.14113403,414.59710297)
\curveto(444.11112943,414.58709951)(444.06612947,414.58209952)(444.00613403,414.58210297)
\curveto(443.87612966,414.55209955)(443.74612979,414.53209957)(443.61613403,414.52210297)
\curveto(443.48613005,414.51209959)(443.36113018,414.48709961)(443.24113403,414.44710297)
\curveto(443.16113038,414.42709967)(443.08613045,414.40709969)(443.01613403,414.38710297)
\curveto(442.94613059,414.37709972)(442.87613066,414.35709974)(442.80613403,414.32710297)
\curveto(442.59613094,414.23709986)(442.41613112,414.1021)(442.26613403,413.92210297)
\curveto(442.12613141,413.74210036)(442.07613146,413.49210061)(442.11613403,413.17210297)
\curveto(442.1361314,413.0021011)(442.19113135,412.86210124)(442.28113403,412.75210297)
\curveto(442.35113119,412.64210146)(442.45613108,412.55210155)(442.59613403,412.48210297)
\curveto(442.7361308,412.42210168)(442.88613065,412.37710172)(443.04613403,412.34710297)
\curveto(443.21613032,412.31710178)(443.39113015,412.30710179)(443.57113403,412.31710297)
\curveto(443.76112978,412.33710176)(443.9361296,412.37210173)(444.09613403,412.42210297)
\curveto(444.35612918,412.5021016)(444.56112898,412.62710147)(444.71113403,412.79710297)
\curveto(444.86112868,412.97710112)(444.97612856,413.1971009)(445.05613403,413.45710297)
\curveto(445.07612846,413.52710057)(445.08612845,413.5971005)(445.08613403,413.66710297)
\curveto(445.09612844,413.74710035)(445.11112843,413.82710027)(445.13113403,413.90710297)
\lineto(445.13113403,414.04210297)
}
}
{
\newrgbcolor{curcolor}{0 0 0}
\pscustom[linestyle=none,fillstyle=solid,fillcolor=curcolor]
{
\newpath
\moveto(55.91582169,179.85068481)
\curveto(55.99582947,179.85067413)(56.10582936,179.83567415)(56.24582169,179.80568481)
\curveto(56.37582909,179.7856742)(56.47582899,179.75567423)(56.54582169,179.71568481)
\curveto(56.61582885,179.6856743)(56.68082878,179.67067431)(56.74082169,179.67068481)
\curveto(56.80082866,179.67067431)(56.8658286,179.65067433)(56.93582169,179.61068481)
\curveto(56.99582847,179.5806744)(57.0608284,179.55067443)(57.13082169,179.52068481)
\curveto(57.19082827,179.50067448)(57.25082821,179.47567451)(57.31082169,179.44568481)
\curveto(57.39082807,179.41567457)(57.465828,179.3806746)(57.53582169,179.34068481)
\curveto(57.60582786,179.30067468)(57.67582779,179.25567473)(57.74582169,179.20568481)
\curveto(57.77582769,179.1856748)(57.80582766,179.17067481)(57.83582169,179.16068481)
\curveto(57.85582761,179.15067483)(57.87582759,179.13067485)(57.89582169,179.10068481)
\curveto(58.09582737,178.95067503)(58.27582719,178.79567519)(58.43582169,178.63568481)
\curveto(58.47582699,178.60567538)(58.51582695,178.56567542)(58.55582169,178.51568481)
\curveto(58.59582687,178.46567552)(58.62582684,178.42067556)(58.64582169,178.38068481)
\curveto(58.6658268,178.34067564)(58.69582677,178.30067568)(58.73582169,178.26068481)
\curveto(58.7658267,178.23067575)(58.79082667,178.19567579)(58.81082169,178.15568481)
\lineto(58.99082169,177.81068481)
\curveto(59.07082639,177.6806763)(59.13082633,177.53567645)(59.17082169,177.37568481)
\curveto(59.21082625,177.22567676)(59.25082621,177.07067691)(59.29082169,176.91068481)
\curveto(59.31082615,176.82067716)(59.32582614,176.73567725)(59.33582169,176.65568481)
\curveto(59.33582613,176.57567741)(59.34582612,176.49567749)(59.36582169,176.41568481)
\curveto(59.37582609,176.37567761)(59.38082608,176.33567765)(59.38082169,176.29568481)
\curveto(59.37082609,176.26567772)(59.37082609,176.23567775)(59.38082169,176.20568481)
\curveto(59.39082607,176.14567784)(59.39082607,176.09567789)(59.38082169,176.05568481)
\curveto(59.37082609,176.01567797)(59.37082609,175.97067801)(59.38082169,175.92068481)
\lineto(59.38082169,173.65568481)
\lineto(59.38082169,173.16068481)
\curveto(59.37082609,172.99068099)(59.40082606,172.85068113)(59.47082169,172.74068481)
\curveto(59.55082591,172.62068136)(59.69582577,172.54068144)(59.90582169,172.50068481)
\curveto(60.11582535,172.46068152)(60.31082515,172.42568156)(60.49082169,172.39568481)
\lineto(62.69582169,171.94568481)
\curveto(62.82582264,171.92568206)(62.99082247,171.89568209)(63.19082169,171.85568481)
\curveto(63.38082208,171.82568216)(63.51582195,171.77568221)(63.59582169,171.70568481)
\curveto(63.65582181,171.65568233)(63.69582177,171.60068238)(63.71582169,171.54068481)
\curveto(63.72582174,171.49068249)(63.74082172,171.42068256)(63.76082169,171.33068481)
\lineto(63.76082169,171.06068481)
\curveto(63.7608217,170.91068307)(63.75582171,170.77568321)(63.74582169,170.65568481)
\curveto(63.73582173,170.54568344)(63.68582178,170.47068351)(63.59582169,170.43068481)
\curveto(63.53582193,170.41068357)(63.45582201,170.41068357)(63.35582169,170.43068481)
\curveto(63.24582222,170.45068353)(63.14082232,170.47068351)(63.04082169,170.49068481)
\lineto(53.93582169,172.30568481)
\curveto(53.82583164,172.32568166)(53.70583176,172.34568164)(53.57582169,172.36568481)
\curveto(53.43583203,172.39568159)(53.32583214,172.44068154)(53.24582169,172.50068481)
\curveto(53.18583228,172.56068142)(53.13583233,172.64568134)(53.09582169,172.75568481)
\curveto(53.08583238,172.7856812)(53.08583238,172.80568118)(53.09582169,172.81568481)
\curveto(53.09583237,172.83568115)(53.09083237,172.86068112)(53.08082169,172.89068481)
\lineto(53.08082169,176.29568481)
\curveto(53.08083238,176.67567731)(53.08583238,177.04567694)(53.09582169,177.40568481)
\curveto(53.09583237,177.76567622)(53.14083232,178.0856759)(53.23082169,178.36568481)
\curveto(53.38083208,178.7856752)(53.57583189,179.11067487)(53.81582169,179.34068481)
\curveto(54.05583141,179.57067441)(54.38583108,179.73567425)(54.80582169,179.83568481)
\curveto(54.91583055,179.85567413)(55.03083043,179.86567412)(55.15082169,179.86568481)
\curveto(55.27083019,179.87567411)(55.39583007,179.8856741)(55.52582169,179.89568481)
\curveto(55.59582987,179.90567408)(55.6608298,179.89567409)(55.72082169,179.86568481)
\curveto(55.78082968,179.84567414)(55.84582962,179.84067414)(55.91582169,179.85068481)
\moveto(56.45582169,178.33568481)
\curveto(56.31582915,178.39567559)(56.15582931,178.43067555)(55.97582169,178.44068481)
\curveto(55.78582968,178.45067553)(55.63582983,178.45067553)(55.52582169,178.44068481)
\curveto(55.24583022,178.40067558)(55.02583044,178.31067567)(54.86582169,178.17068481)
\curveto(54.69583077,178.04067594)(54.55583091,177.85567613)(54.44582169,177.61568481)
\curveto(54.35583111,177.41567657)(54.30083116,177.17567681)(54.28082169,176.89568481)
\curveto(54.2608312,176.61567737)(54.25083121,176.32067766)(54.25082169,176.01068481)
\lineto(54.25082169,174.07568481)
\curveto(54.2608312,174.05567993)(54.2658312,174.03067995)(54.26582169,174.00068481)
\curveto(54.2658312,173.98068)(54.27083119,173.95568003)(54.28082169,173.92568481)
\curveto(54.31083115,173.79568019)(54.37583109,173.70068028)(54.47582169,173.64068481)
\curveto(54.57583089,173.5806804)(54.71083075,173.53568045)(54.88082169,173.50568481)
\curveto(55.04083042,173.47568051)(55.19083027,173.45068053)(55.33082169,173.43068481)
\lineto(57.13082169,173.07068481)
\curveto(57.28082818,173.04068094)(57.44582802,173.00068098)(57.62582169,172.95068481)
\curveto(57.80582766,172.91068107)(57.94582752,172.91068107)(58.04582169,172.95068481)
\curveto(58.12582734,172.99068099)(58.17582729,173.06068092)(58.19582169,173.16068481)
\curveto(58.20582726,173.26068072)(58.21082725,173.37568061)(58.21082169,173.50568481)
\lineto(58.21082169,175.53068481)
\lineto(58.21082169,175.99568481)
\curveto(58.20082726,176.15567783)(58.18082728,176.30067768)(58.15082169,176.43068481)
\curveto(58.08082738,176.71067727)(58.00082746,176.96067702)(57.91082169,177.18068481)
\curveto(57.81082765,177.41067657)(57.6608278,177.61567637)(57.46082169,177.79568481)
\curveto(57.34082812,177.90567608)(57.21582825,177.99567599)(57.08582169,178.06568481)
\curveto(56.95582851,178.13567585)(56.81082865,178.20567578)(56.65082169,178.27568481)
\curveto(56.61082885,178.29567569)(56.54582892,178.31567567)(56.45582169,178.33568481)
}
}
{
\newrgbcolor{curcolor}{0 0 0}
\pscustom[linestyle=none,fillstyle=solid,fillcolor=curcolor]
{
\newpath
\moveto(56.03582169,181.82224731)
\lineto(56.03582169,182.25724731)
\curveto(56.03582943,182.4072438)(56.07582939,182.5072437)(56.15582169,182.55724731)
\curveto(56.23582923,182.58724362)(56.33582913,182.59224362)(56.45582169,182.57224731)
\lineto(56.81582169,182.51224731)
\lineto(58.24082169,182.22724731)
\lineto(60.50582169,181.77724731)
\curveto(60.72582474,181.72724448)(60.95582451,181.67724453)(61.19582169,181.62724731)
\curveto(61.42582404,181.58724462)(61.62582384,181.57224464)(61.79582169,181.58224731)
\curveto(62.24582322,181.65224456)(62.5608229,181.89224432)(62.74082169,182.30224731)
\curveto(62.83082263,182.50224371)(62.8658226,182.75724345)(62.84582169,183.06724731)
\curveto(62.81582265,183.38724282)(62.7608227,183.65224256)(62.68082169,183.86224731)
\curveto(62.54082292,184.212242)(62.3658231,184.5072417)(62.15582169,184.74724731)
\curveto(61.93582353,184.98724122)(61.65082381,185.19724101)(61.30082169,185.37724731)
\curveto(61.22082424,185.42724078)(61.14082432,185.46224075)(61.06082169,185.48224731)
\curveto(60.98082448,185.5122407)(60.89582457,185.54724066)(60.80582169,185.58724731)
\curveto(60.75582471,185.6072406)(60.71082475,185.61724059)(60.67082169,185.61724731)
\curveto(60.63082483,185.61724059)(60.58582488,185.63224058)(60.53582169,185.66224731)
\lineto(60.22082169,185.72224731)
\curveto(60.14082532,185.76224045)(60.05082541,185.78724042)(59.95082169,185.79724731)
\curveto(59.84082562,185.8072404)(59.74082572,185.82224039)(59.65082169,185.84224731)
\lineto(58.48082169,186.08224731)
\lineto(56.89082169,186.39724731)
\curveto(56.77082869,186.41723979)(56.64582882,186.43723977)(56.51582169,186.45724731)
\curveto(56.37582909,186.48723972)(56.2658292,186.53223968)(56.18582169,186.59224731)
\curveto(56.13582933,186.64223957)(56.10582936,186.69723951)(56.09582169,186.75724731)
\curveto(56.07582939,186.81723939)(56.05582941,186.88723932)(56.03582169,186.96724731)
\lineto(56.03582169,187.19224731)
\curveto(56.03582943,187.3122389)(56.04082942,187.41723879)(56.05082169,187.50724731)
\curveto(56.0608294,187.6072386)(56.10582936,187.67223854)(56.18582169,187.70224731)
\curveto(56.23582923,187.74223847)(56.31082915,187.75223846)(56.41082169,187.73224731)
\curveto(56.50082896,187.7122385)(56.59582887,187.69223852)(56.69582169,187.67224731)
\lineto(57.71582169,187.47724731)
\lineto(61.75082169,186.66724731)
\lineto(63.10082169,186.39724731)
\curveto(63.22082224,186.37723983)(63.33582213,186.35223986)(63.44582169,186.32224731)
\curveto(63.54582192,186.29223992)(63.62082184,186.23723997)(63.67082169,186.15724731)
\curveto(63.70082176,186.11724009)(63.72582174,186.05224016)(63.74582169,185.96224731)
\curveto(63.75582171,185.88224033)(63.7658217,185.79224042)(63.77582169,185.69224731)
\curveto(63.77582169,185.60224061)(63.77082169,185.5122407)(63.76082169,185.42224731)
\curveto(63.75082171,185.34224087)(63.73082173,185.28724092)(63.70082169,185.25724731)
\curveto(63.6608218,185.21724099)(63.59582187,185.18724102)(63.50582169,185.16724731)
\curveto(63.465822,185.15724105)(63.41082205,185.15724105)(63.34082169,185.16724731)
\curveto(63.27082219,185.17724103)(63.20582226,185.18224103)(63.14582169,185.18224731)
\curveto(63.07582239,185.19224102)(63.02082244,185.18224103)(62.98082169,185.15224731)
\curveto(62.94082252,185.13224108)(62.92582254,185.09224112)(62.93582169,185.03224731)
\curveto(62.95582251,184.95224126)(63.01582245,184.86224135)(63.11582169,184.76224731)
\curveto(63.20582226,184.66224155)(63.27582219,184.57224164)(63.32582169,184.49224731)
\curveto(63.48582198,184.24224197)(63.62582184,183.96224225)(63.74582169,183.65224731)
\curveto(63.79582167,183.53224268)(63.82582164,183.4122428)(63.83582169,183.29224731)
\curveto(63.85582161,183.18224303)(63.88082158,183.06224315)(63.91082169,182.93224731)
\curveto(63.92082154,182.88224333)(63.92082154,182.82724338)(63.91082169,182.76724731)
\curveto(63.90082156,182.71724349)(63.90582156,182.66724354)(63.92582169,182.61724731)
\curveto(63.94582152,182.51724369)(63.94582152,182.42724378)(63.92582169,182.34724731)
\lineto(63.92582169,182.19724731)
\curveto(63.90582156,182.14724406)(63.89582157,182.08724412)(63.89582169,182.01724731)
\curveto(63.89582157,181.95724425)(63.89082157,181.9072443)(63.88082169,181.86724731)
\curveto(63.8608216,181.82724438)(63.85082161,181.78724442)(63.85082169,181.74724731)
\curveto(63.8608216,181.71724449)(63.85582161,181.67724453)(63.83582169,181.62724731)
\curveto(63.81582165,181.55724465)(63.79582167,181.48224473)(63.77582169,181.40224731)
\curveto(63.75582171,181.33224488)(63.72582174,181.26224495)(63.68582169,181.19224731)
\curveto(63.57582189,180.95224526)(63.43082203,180.76224545)(63.25082169,180.62224731)
\curveto(63.0608224,180.49224572)(62.83582263,180.39724581)(62.57582169,180.33724731)
\curveto(62.48582298,180.31724589)(62.39582307,180.3072459)(62.30582169,180.30724731)
\lineto(62.00582169,180.30724731)
\curveto(61.94582352,180.29724591)(61.89082357,180.29724591)(61.84082169,180.30724731)
\curveto(61.78082368,180.32724588)(61.71582375,180.33224588)(61.64582169,180.32224731)
\lineto(61.57082169,180.32224731)
\curveto(61.53082393,180.33224588)(61.49582397,180.33724587)(61.46582169,180.33724731)
\lineto(61.31582169,180.36724731)
\curveto(61.27582419,180.36724584)(61.23082423,180.37224584)(61.18082169,180.38224731)
\curveto(61.12082434,180.40224581)(61.0658244,180.41724579)(61.01582169,180.42724731)
\lineto(60.41582169,180.54724731)
\lineto(57.65582169,181.10224731)
\lineto(56.69582169,181.28224731)
\lineto(56.42582169,181.34224731)
\curveto(56.33582913,181.36224485)(56.2608292,181.39724481)(56.20082169,181.44724731)
\curveto(56.13082933,181.49724471)(56.08082938,181.58224463)(56.05082169,181.70224731)
\curveto(56.04082942,181.72224449)(56.04082942,181.74224447)(56.05082169,181.76224731)
\curveto(56.05082941,181.78224443)(56.04582942,181.80224441)(56.03582169,181.82224731)
}
}
{
\newrgbcolor{curcolor}{0 0 0}
\pscustom[linestyle=none,fillstyle=solid,fillcolor=curcolor]
{
\newpath
\moveto(55.88582169,193.42685669)
\curveto(55.8658296,194.06684987)(55.95082951,194.55684938)(56.14082169,194.89685669)
\curveto(56.33082913,195.2368487)(56.61582885,195.48184845)(56.99582169,195.63185669)
\curveto(57.09582837,195.67184826)(57.20582826,195.69684824)(57.32582169,195.70685669)
\curveto(57.43582803,195.72684821)(57.55082791,195.7368482)(57.67082169,195.73685669)
\curveto(57.8608276,195.75684818)(58.0658274,195.74684819)(58.28582169,195.70685669)
\curveto(58.50582696,195.67684826)(58.73082673,195.6368483)(58.96082169,195.58685669)
\lineto(60.56582169,195.27185669)
\lineto(62.90582169,194.80685669)
\lineto(63.41582169,194.68685669)
\curveto(63.58582188,194.64684929)(63.69582177,194.55684938)(63.74582169,194.41685669)
\curveto(63.7658217,194.36684957)(63.77582169,194.31184962)(63.77582169,194.25185669)
\curveto(63.78582168,194.20184973)(63.79082167,194.14684979)(63.79082169,194.08685669)
\curveto(63.79082167,193.95684998)(63.78582168,193.8318501)(63.77582169,193.71185669)
\curveto(63.77582169,193.59185034)(63.73582173,193.51685042)(63.65582169,193.48685669)
\curveto(63.58582188,193.44685049)(63.49582197,193.4368505)(63.38582169,193.45685669)
\curveto(63.27582219,193.47685046)(63.1658223,193.50185043)(63.05582169,193.53185669)
\lineto(61.76582169,193.78685669)
\lineto(59.32082169,194.26685669)
\curveto(59.05082641,194.32684961)(58.78582668,194.37684956)(58.52582169,194.41685669)
\curveto(58.25582721,194.45684948)(58.02582744,194.45684948)(57.83582169,194.41685669)
\curveto(57.63582783,194.37684956)(57.47582799,194.28684965)(57.35582169,194.14685669)
\curveto(57.22582824,194.01684992)(57.12582834,193.85685008)(57.05582169,193.66685669)
\curveto(57.03582843,193.60685033)(57.02582844,193.54185039)(57.02582169,193.47185669)
\curveto(57.01582845,193.41185052)(57.00082846,193.35685058)(56.98082169,193.30685669)
\curveto(56.97082849,193.25685068)(56.97082849,193.17685076)(56.98082169,193.06685669)
\curveto(56.98082848,192.96685097)(56.98582848,192.89185104)(56.99582169,192.84185669)
\curveto(57.01582845,192.80185113)(57.02582844,192.76685117)(57.02582169,192.73685669)
\curveto(57.01582845,192.70685123)(57.01582845,192.67185126)(57.02582169,192.63185669)
\curveto(57.05582841,192.49185144)(57.09082837,192.36185157)(57.13082169,192.24185669)
\curveto(57.1608283,192.12185181)(57.20582826,192.00685193)(57.26582169,191.89685669)
\curveto(57.28582818,191.84685209)(57.30582816,191.80685213)(57.32582169,191.77685669)
\curveto(57.34582812,191.74685219)(57.3658281,191.70685223)(57.38582169,191.65685669)
\curveto(57.63582783,191.25685268)(58.01082745,190.92685301)(58.51082169,190.66685669)
\curveto(58.59082687,190.62685331)(58.67582679,190.59185334)(58.76582169,190.56185669)
\lineto(59.00582169,190.47185669)
\curveto(59.05582641,190.44185349)(59.10582636,190.42685351)(59.15582169,190.42685669)
\curveto(59.19582627,190.42685351)(59.23582623,190.41185352)(59.27582169,190.38185669)
\lineto(59.59082169,190.32185669)
\curveto(59.62082584,190.30185363)(59.65582581,190.29185364)(59.69582169,190.29185669)
\curveto(59.73582573,190.29185364)(59.78082568,190.28685365)(59.83082169,190.27685669)
\lineto(60.28082169,190.18685669)
\lineto(61.72082169,189.88685669)
\lineto(63.04082169,189.63185669)
\curveto(63.15082231,189.61185432)(63.2658222,189.58685435)(63.38582169,189.55685669)
\curveto(63.49582197,189.5368544)(63.58582188,189.49685444)(63.65582169,189.43685669)
\curveto(63.73582173,189.36685457)(63.77582169,189.26685467)(63.77582169,189.13685669)
\curveto(63.78582168,189.01685492)(63.79082167,188.89185504)(63.79082169,188.76185669)
\curveto(63.79082167,188.68185525)(63.78582168,188.60685533)(63.77582169,188.53685669)
\curveto(63.7658217,188.46685547)(63.74082172,188.41185552)(63.70082169,188.37185669)
\curveto(63.65082181,188.30185563)(63.55582191,188.28185565)(63.41582169,188.31185669)
\curveto(63.27582219,188.34185559)(63.14082232,188.36685557)(63.01082169,188.38685669)
\lineto(61.24082169,188.74685669)
\lineto(57.61082169,189.46685669)
\lineto(56.69582169,189.64685669)
\lineto(56.42582169,189.70685669)
\curveto(56.33582913,189.72685421)(56.2658292,189.76185417)(56.21582169,189.81185669)
\curveto(56.15582931,189.85185408)(56.11582935,189.90685403)(56.09582169,189.97685669)
\curveto(56.08582938,190.02685391)(56.07582939,190.08685385)(56.06582169,190.15685669)
\curveto(56.05582941,190.2368537)(56.05082941,190.31685362)(56.05082169,190.39685669)
\curveto(56.05082941,190.47685346)(56.05582941,190.55185338)(56.06582169,190.62185669)
\curveto(56.07582939,190.70185323)(56.09082937,190.75185318)(56.11082169,190.77185669)
\curveto(56.18082928,190.87185306)(56.27082919,190.90685303)(56.38082169,190.87685669)
\curveto(56.48082898,190.84685309)(56.59582887,190.8368531)(56.72582169,190.84685669)
\curveto(56.78582868,190.85685308)(56.83582863,190.88685305)(56.87582169,190.93685669)
\curveto(56.88582858,191.05685288)(56.84082862,191.16185277)(56.74082169,191.25185669)
\curveto(56.64082882,191.35185258)(56.5608289,191.44685249)(56.50082169,191.53685669)
\curveto(56.40082906,191.69685224)(56.31082915,191.85685208)(56.23082169,192.01685669)
\curveto(56.14082932,192.17685176)(56.0658294,192.36185157)(56.00582169,192.57185669)
\curveto(55.97582949,192.65185128)(55.95582951,192.74185119)(55.94582169,192.84185669)
\curveto(55.93582953,192.94185099)(55.92082954,193.0368509)(55.90082169,193.12685669)
\curveto(55.89082957,193.17685076)(55.88582958,193.22685071)(55.88582169,193.27685669)
\lineto(55.88582169,193.42685669)
}
}
{
\newrgbcolor{curcolor}{0 0 0}
\pscustom[linestyle=none,fillstyle=solid,fillcolor=curcolor]
{
\newpath
\moveto(53.69582169,199.46646606)
\curveto(53.69583177,199.61646204)(53.70083176,199.76646189)(53.71082169,199.91646606)
\curveto(53.71083175,200.06646159)(53.75083171,200.16646149)(53.83082169,200.21646606)
\curveto(53.89083157,200.24646141)(53.97583149,200.25146141)(54.08582169,200.23146606)
\curveto(54.18583128,200.22146144)(54.29083117,200.20646145)(54.40082169,200.18646606)
\lineto(55.27082169,200.00646606)
\curveto(55.35083011,199.99646166)(55.43583003,199.97646168)(55.52582169,199.94646606)
\curveto(55.60582986,199.92646173)(55.67582979,199.92146174)(55.73582169,199.93146606)
\curveto(55.87582959,199.94146172)(55.9658295,200.01646164)(56.00582169,200.15646606)
\curveto(56.01582945,200.19646146)(56.02082944,200.23646142)(56.02082169,200.27646606)
\lineto(56.02082169,200.42646606)
\lineto(56.02082169,200.83146606)
\curveto(56.01082945,201.00146066)(56.02082944,201.11646054)(56.05082169,201.17646606)
\curveto(56.11082935,201.2564604)(56.17082929,201.30646035)(56.23082169,201.32646606)
\curveto(56.27082919,201.33646032)(56.31582915,201.33646032)(56.36582169,201.32646606)
\lineto(56.51582169,201.29646606)
\curveto(56.62582884,201.27646038)(56.73082873,201.25146041)(56.83082169,201.22146606)
\curveto(56.92082854,201.19146047)(56.99082847,201.14146052)(57.04082169,201.07146606)
\curveto(57.09082837,201.00146066)(57.12082834,200.91146075)(57.13082169,200.80146606)
\lineto(57.13082169,200.47146606)
\curveto(57.12082834,200.3614613)(57.11582835,200.25146141)(57.11582169,200.14146606)
\curveto(57.11582835,200.03146163)(57.13082833,199.93146173)(57.16082169,199.84146606)
\curveto(57.19082827,199.77146189)(57.24082822,199.71146195)(57.31082169,199.66146606)
\curveto(57.38082808,199.62146204)(57.465828,199.58646207)(57.56582169,199.55646606)
\curveto(57.65582781,199.52646213)(57.75582771,199.50146216)(57.86582169,199.48146606)
\curveto(57.9658275,199.47146219)(58.0658274,199.4564622)(58.16582169,199.43646606)
\lineto(61.13582169,198.83646606)
\curveto(61.35582411,198.79646286)(61.59082387,198.74646291)(61.84082169,198.68646606)
\curveto(62.08082338,198.63646302)(62.2658232,198.64146302)(62.39582169,198.70146606)
\curveto(62.47582299,198.74146292)(62.53082293,198.79646286)(62.56082169,198.86646606)
\curveto(62.59082287,198.94646271)(62.61582285,199.03646262)(62.63582169,199.13646606)
\curveto(62.64582282,199.16646249)(62.65082281,199.19646246)(62.65082169,199.22646606)
\curveto(62.64082282,199.26646239)(62.64082282,199.30146236)(62.65082169,199.33146606)
\lineto(62.65082169,199.52646606)
\curveto(62.65082281,199.62646203)(62.6608228,199.71646194)(62.68082169,199.79646606)
\curveto(62.69082277,199.87646178)(62.72582274,199.93146173)(62.78582169,199.96146606)
\curveto(62.81582265,199.98146168)(62.87082259,199.99146167)(62.95082169,199.99146606)
\curveto(63.02082244,199.99146167)(63.09582237,199.98146168)(63.17582169,199.96146606)
\curveto(63.25582221,199.95146171)(63.33582213,199.93146173)(63.41582169,199.90146606)
\curveto(63.48582198,199.88146178)(63.54082192,199.8564618)(63.58082169,199.82646606)
\curveto(63.65082181,199.76646189)(63.70082176,199.68146198)(63.73082169,199.57146606)
\curveto(63.75082171,199.48146218)(63.7608217,199.38646227)(63.76082169,199.28646606)
\curveto(63.75082171,199.18646247)(63.74582172,199.09646256)(63.74582169,199.01646606)
\curveto(63.74582172,198.9564627)(63.75082171,198.89646276)(63.76082169,198.83646606)
\curveto(63.7608217,198.77646288)(63.75582171,198.72146294)(63.74582169,198.67146606)
\lineto(63.74582169,198.49146606)
\curveto(63.73582173,198.45146321)(63.73082173,198.40646325)(63.73082169,198.35646606)
\curveto(63.72082174,198.31646334)(63.71582175,198.27146339)(63.71582169,198.22146606)
\curveto(63.6658218,198.03146363)(63.61082185,197.86646379)(63.55082169,197.72646606)
\curveto(63.49082197,197.59646406)(63.38582208,197.49646416)(63.23582169,197.42646606)
\curveto(63.03582243,197.32646433)(62.78582268,197.29646436)(62.48582169,197.33646606)
\curveto(62.17582329,197.37646428)(61.84582362,197.43146423)(61.49582169,197.50146606)
\lineto(57.56582169,198.29646606)
\curveto(57.43582803,198.28646337)(57.34082812,198.27646338)(57.28082169,198.26646606)
\curveto(57.22082824,198.2564634)(57.17082829,198.19646346)(57.13082169,198.08646606)
\curveto(57.12082834,198.04646361)(57.12082834,198.00146366)(57.13082169,197.95146606)
\curveto(57.14082832,197.91146375)(57.13582833,197.87646378)(57.11582169,197.84646606)
\lineto(57.11582169,197.60646606)
\curveto(57.11582835,197.47646418)(57.10582836,197.37146429)(57.08582169,197.29146606)
\curveto(57.05582841,197.21146445)(56.99582847,197.16646449)(56.90582169,197.15646606)
\curveto(56.8658286,197.13646452)(56.82082864,197.12646453)(56.77082169,197.12646606)
\lineto(56.62082169,197.15646606)
\curveto(56.48082898,197.18646447)(56.3658291,197.22146444)(56.27582169,197.26146606)
\curveto(56.17582929,197.30146436)(56.10082936,197.37646428)(56.05082169,197.48646606)
\curveto(56.01082945,197.60646405)(56.00082946,197.75146391)(56.02082169,197.92146606)
\curveto(56.04082942,198.09146357)(56.03082943,198.24146342)(55.99082169,198.37146606)
\curveto(55.94082952,198.47146319)(55.87082959,198.5564631)(55.78082169,198.62646606)
\curveto(55.72082974,198.656463)(55.64582982,198.67646298)(55.55582169,198.68646606)
\curveto(55.46583,198.70646295)(55.38083008,198.72646293)(55.30082169,198.74646606)
\lineto(54.37082169,198.92646606)
\curveto(54.29083117,198.94646271)(54.21083125,198.9614627)(54.13082169,198.97146606)
\curveto(54.04083142,198.99146267)(53.9658315,199.02146264)(53.90582169,199.06146606)
\curveto(53.82583164,199.11146255)(53.7608317,199.19646246)(53.71082169,199.31646606)
\curveto(53.71083175,199.34646231)(53.71083175,199.37146229)(53.71082169,199.39146606)
\curveto(53.70083176,199.42146224)(53.69583177,199.44646221)(53.69582169,199.46646606)
}
}
{
\newrgbcolor{curcolor}{0 0 0}
\pscustom[linestyle=none,fillstyle=solid,fillcolor=curcolor]
{
\newpath
\moveto(63.20582169,208.23326294)
\curveto(63.3658221,208.22325503)(63.50082196,208.17825507)(63.61082169,208.09826294)
\curveto(63.71082175,208.01825523)(63.78582168,207.92325533)(63.83582169,207.81326294)
\curveto(63.85582161,207.76325549)(63.8658216,207.70825554)(63.86582169,207.64826294)
\curveto(63.8658216,207.59825565)(63.87582159,207.53825571)(63.89582169,207.46826294)
\curveto(63.94582152,207.23825601)(63.93082153,207.02325623)(63.85082169,206.82326294)
\curveto(63.78082168,206.62325663)(63.69082177,206.49825675)(63.58082169,206.44826294)
\curveto(63.51082195,206.40825684)(63.43082203,206.37825687)(63.34082169,206.35826294)
\curveto(63.24082222,206.33825691)(63.1608223,206.30325695)(63.10082169,206.25326294)
\lineto(63.04082169,206.19326294)
\curveto(63.02082244,206.17325708)(63.01582245,206.14325711)(63.02582169,206.10326294)
\curveto(63.05582241,205.98325727)(63.11082235,205.86825738)(63.19082169,205.75826294)
\curveto(63.27082219,205.6482576)(63.34082212,205.54325771)(63.40082169,205.44326294)
\curveto(63.48082198,205.29325796)(63.55582191,205.13825811)(63.62582169,204.97826294)
\curveto(63.68582178,204.81825843)(63.74082172,204.6482586)(63.79082169,204.46826294)
\curveto(63.82082164,204.35825889)(63.84082162,204.24325901)(63.85082169,204.12326294)
\curveto(63.8608216,204.01325924)(63.87582159,203.89825935)(63.89582169,203.77826294)
\curveto(63.90582156,203.72825952)(63.91082155,203.68325957)(63.91082169,203.64326294)
\lineto(63.91082169,203.53826294)
\curveto(63.93082153,203.42825982)(63.93082153,203.32325993)(63.91082169,203.22326294)
\lineto(63.91082169,203.08826294)
\curveto(63.90082156,203.03826021)(63.89582157,202.98826026)(63.89582169,202.93826294)
\curveto(63.89582157,202.88826036)(63.88582158,202.8482604)(63.86582169,202.81826294)
\curveto(63.85582161,202.77826047)(63.85082161,202.74326051)(63.85082169,202.71326294)
\curveto(63.8608216,202.69326056)(63.8608216,202.66826058)(63.85082169,202.63826294)
\lineto(63.79082169,202.39826294)
\curveto(63.78082168,202.32826092)(63.7608217,202.26326099)(63.73082169,202.20326294)
\curveto(63.60082186,201.92326133)(63.45582201,201.70826154)(63.29582169,201.55826294)
\curveto(63.12582234,201.40826184)(62.89082257,201.30326195)(62.59082169,201.24326294)
\curveto(62.37082309,201.19326206)(62.10582336,201.19826205)(61.79582169,201.25826294)
\lineto(61.48082169,201.33326294)
\curveto(61.43082403,201.3532619)(61.38082408,201.36826188)(61.33082169,201.37826294)
\lineto(61.15082169,201.43826294)
\lineto(60.82082169,201.61826294)
\curveto(60.71082475,201.68826156)(60.61082485,201.75826149)(60.52082169,201.82826294)
\curveto(60.23082523,202.06826118)(60.01582545,202.35826089)(59.87582169,202.69826294)
\curveto(59.73582573,203.03826021)(59.61082585,203.40325985)(59.50082169,203.79326294)
\curveto(59.460826,203.94325931)(59.43082603,204.09325916)(59.41082169,204.24326294)
\curveto(59.39082607,204.40325885)(59.3658261,204.55825869)(59.33582169,204.70826294)
\curveto(59.31582615,204.78825846)(59.30582616,204.85825839)(59.30582169,204.91826294)
\curveto(59.30582616,204.98825826)(59.29582617,205.06325819)(59.27582169,205.14326294)
\curveto(59.25582621,205.21325804)(59.24582622,205.28325797)(59.24582169,205.35326294)
\curveto(59.23582623,205.43325782)(59.22082624,205.51325774)(59.20082169,205.59326294)
\curveto(59.14082632,205.8532574)(59.09082637,206.09825715)(59.05082169,206.32826294)
\curveto(59.00082646,206.55825669)(58.88582658,206.75825649)(58.70582169,206.92826294)
\curveto(58.62582684,206.99825625)(58.52582694,207.06325619)(58.40582169,207.12326294)
\curveto(58.27582719,207.19325606)(58.13582733,207.22325603)(57.98582169,207.21326294)
\curveto(57.74582772,207.20325605)(57.55582791,207.1532561)(57.41582169,207.06326294)
\curveto(57.27582819,206.98325627)(57.1658283,206.84325641)(57.08582169,206.64326294)
\curveto(57.03582843,206.53325672)(57.00082846,206.39825685)(56.98082169,206.23826294)
\curveto(56.9608285,206.07825717)(56.95082851,205.90825734)(56.95082169,205.72826294)
\curveto(56.95082851,205.5482577)(56.9608285,205.36825788)(56.98082169,205.18826294)
\curveto(57.00082846,205.01825823)(57.03082843,204.86825838)(57.07082169,204.73826294)
\curveto(57.13082833,204.55825869)(57.21582825,204.37825887)(57.32582169,204.19826294)
\curveto(57.38582808,204.10825914)(57.465828,204.01825923)(57.56582169,203.92826294)
\curveto(57.65582781,203.8482594)(57.75582771,203.77325948)(57.86582169,203.70326294)
\curveto(57.94582752,203.6532596)(58.03082743,203.60825964)(58.12082169,203.56826294)
\curveto(58.21082725,203.52825972)(58.28082718,203.46825978)(58.33082169,203.38826294)
\curveto(58.3608271,203.33825991)(58.38582708,203.26325999)(58.40582169,203.16326294)
\curveto(58.41582705,203.06326019)(58.42082704,202.96326029)(58.42082169,202.86326294)
\curveto(58.42082704,202.76326049)(58.41582705,202.66826058)(58.40582169,202.57826294)
\curveto(58.38582708,202.48826076)(58.3608271,202.42826082)(58.33082169,202.39826294)
\curveto(58.30082716,202.35826089)(58.25082721,202.33326092)(58.18082169,202.32326294)
\curveto(58.11082735,202.32326093)(58.03582743,202.34326091)(57.95582169,202.38326294)
\curveto(57.82582764,202.43326082)(57.70582776,202.48826076)(57.59582169,202.54826294)
\curveto(57.47582799,202.60826064)(57.3608281,202.67326058)(57.25082169,202.74326294)
\curveto(56.90082856,203.00326025)(56.63082883,203.29825995)(56.44082169,203.62826294)
\curveto(56.24082922,203.95825929)(56.08082938,204.3482589)(55.96082169,204.79826294)
\curveto(55.94082952,204.90825834)(55.92582954,205.01325824)(55.91582169,205.11326294)
\curveto(55.90582956,205.22325803)(55.89082957,205.33325792)(55.87082169,205.44326294)
\curveto(55.8608296,205.49325776)(55.8608296,205.55825769)(55.87082169,205.63826294)
\curveto(55.87082959,205.72825752)(55.8608296,205.78825746)(55.84082169,205.81826294)
\curveto(55.83082963,206.51825673)(55.91082955,207.10825614)(56.08082169,207.58826294)
\curveto(56.25082921,208.07825517)(56.57582889,208.38325487)(57.05582169,208.50326294)
\curveto(57.25582821,208.5532547)(57.49082797,208.55825469)(57.76082169,208.51826294)
\curveto(58.02082744,208.47825477)(58.29582717,208.42825482)(58.58582169,208.36826294)
\lineto(61.90082169,207.70826294)
\curveto(62.04082342,207.67825557)(62.17582329,207.6532556)(62.30582169,207.63326294)
\curveto(62.43582303,207.62325563)(62.54082292,207.63325562)(62.62082169,207.66326294)
\curveto(62.69082277,207.70325555)(62.74082272,207.75825549)(62.77082169,207.82826294)
\curveto(62.81082265,207.91825533)(62.84082262,207.99825525)(62.86082169,208.06826294)
\curveto(62.87082259,208.1482551)(62.91582255,208.19825505)(62.99582169,208.21826294)
\curveto(63.02582244,208.23825501)(63.05582241,208.24325501)(63.08582169,208.23326294)
\lineto(63.20582169,208.23326294)
\moveto(61.54082169,206.41826294)
\curveto(61.40082406,206.50825674)(61.24082422,206.57325668)(61.06082169,206.61326294)
\curveto(60.87082459,206.6532566)(60.67582479,206.69325656)(60.47582169,206.73326294)
\curveto(60.3658251,206.7532565)(60.2658252,206.76825648)(60.17582169,206.77826294)
\curveto(60.08582538,206.78825646)(60.01582545,206.76325649)(59.96582169,206.70326294)
\curveto(59.94582552,206.67325658)(59.93582553,206.60325665)(59.93582169,206.49326294)
\curveto(59.95582551,206.47325678)(59.9658255,206.43825681)(59.96582169,206.38826294)
\curveto(59.9658255,206.33825691)(59.97582549,206.28825696)(59.99582169,206.23826294)
\curveto(60.01582545,206.15825709)(60.03582543,206.06325719)(60.05582169,205.95326294)
\lineto(60.11582169,205.65326294)
\curveto(60.11582535,205.62325763)(60.12082534,205.58825766)(60.13082169,205.54826294)
\lineto(60.13082169,205.44326294)
\curveto(60.17082529,205.28325797)(60.19582527,205.11325814)(60.20582169,204.93326294)
\curveto(60.20582526,204.76325849)(60.22582524,204.59825865)(60.26582169,204.43826294)
\curveto(60.28582518,204.3482589)(60.30582516,204.26825898)(60.32582169,204.19826294)
\curveto(60.33582513,204.13825911)(60.35082511,204.06325919)(60.37082169,203.97326294)
\curveto(60.42082504,203.80325945)(60.48582498,203.63825961)(60.56582169,203.47826294)
\curveto(60.63582483,203.32825992)(60.72582474,203.19326006)(60.83582169,203.07326294)
\curveto(60.94582452,202.9532603)(61.08082438,202.8532604)(61.24082169,202.77326294)
\curveto(61.39082407,202.69326056)(61.57582389,202.63326062)(61.79582169,202.59326294)
\curveto(61.89582357,202.57326068)(61.99082347,202.57326068)(62.08082169,202.59326294)
\curveto(62.1608233,202.61326064)(62.23582323,202.64326061)(62.30582169,202.68326294)
\curveto(62.41582305,202.73326052)(62.51082295,202.81326044)(62.59082169,202.92326294)
\curveto(62.6608228,203.04326021)(62.72082274,203.17326008)(62.77082169,203.31326294)
\curveto(62.78082268,203.36325989)(62.78582268,203.41325984)(62.78582169,203.46326294)
\curveto(62.78582268,203.51325974)(62.79082267,203.56325969)(62.80082169,203.61326294)
\curveto(62.82082264,203.68325957)(62.83582263,203.76825948)(62.84582169,203.86826294)
\curveto(62.84582262,203.96825928)(62.83582263,204.05825919)(62.81582169,204.13826294)
\curveto(62.79582267,204.19825905)(62.79082267,204.25825899)(62.80082169,204.31826294)
\curveto(62.80082266,204.37825887)(62.79082267,204.43825881)(62.77082169,204.49826294)
\curveto(62.75082271,204.58825866)(62.73582273,204.66825858)(62.72582169,204.73826294)
\curveto(62.71582275,204.81825843)(62.69582277,204.89825835)(62.66582169,204.97826294)
\curveto(62.54582292,205.28825796)(62.40082306,205.56325769)(62.23082169,205.80326294)
\curveto(62.0608234,206.04325721)(61.83082363,206.248257)(61.54082169,206.41826294)
}
}
{
\newrgbcolor{curcolor}{0 0 0}
\pscustom[linestyle=none,fillstyle=solid,fillcolor=curcolor]
{
\newpath
\moveto(54.50582169,211.39990356)
\curveto(54.45583101,211.3299006)(54.35583111,211.30990062)(54.20582169,211.33990356)
\curveto(54.04583142,211.36990056)(53.90083156,211.39490054)(53.77082169,211.41490356)
\curveto(53.68083178,211.4349005)(53.59583187,211.45490048)(53.51582169,211.47490356)
\curveto(53.42583204,211.49490044)(53.34583212,211.52490041)(53.27582169,211.56490356)
\curveto(53.22583224,211.59490034)(53.18583228,211.6349003)(53.15582169,211.68490356)
\curveto(53.12583234,211.7349002)(53.10083236,211.79490014)(53.08082169,211.86490356)
\curveto(53.08083238,211.88490005)(53.08583238,211.89990003)(53.09582169,211.90990356)
\curveto(53.09583237,211.9299)(53.08583238,211.95489998)(53.06582169,211.98490356)
\curveto(53.0658324,212.15489978)(53.07083239,212.31489962)(53.08082169,212.46490356)
\curveto(53.09083237,212.61489932)(53.15083231,212.70489923)(53.26082169,212.73490356)
\curveto(53.33083213,212.74489919)(53.41083205,212.73989919)(53.50082169,212.71990356)
\curveto(53.58083188,212.70989922)(53.6658318,212.69989923)(53.75582169,212.68990356)
\curveto(53.93583153,212.64989928)(54.10583136,212.60989932)(54.26582169,212.56990356)
\curveto(54.41583105,212.5298994)(54.51583095,212.43989949)(54.56582169,212.29990356)
\curveto(54.58583088,212.23989969)(54.60083086,212.17989975)(54.61082169,212.11990356)
\lineto(54.61082169,211.96990356)
\curveto(54.61083085,211.84990008)(54.60583086,211.73990019)(54.59582169,211.63990356)
\curveto(54.58583088,211.53990039)(54.55583091,211.45990047)(54.50582169,211.39990356)
\moveto(64.42082169,210.54490356)
\curveto(64.49082097,210.5349014)(64.5658209,210.52490141)(64.64582169,210.51490356)
\curveto(64.72582074,210.50490143)(64.79582067,210.48490145)(64.85582169,210.45490356)
\lineto(64.97582169,210.42490356)
\curveto(65.03582043,210.40490153)(65.09082037,210.38490155)(65.14082169,210.36490356)
\curveto(65.20082026,210.35490158)(65.2608202,210.3349016)(65.32082169,210.30490356)
\curveto(65.49081997,210.2349017)(65.64081982,210.15990177)(65.77082169,210.07990356)
\curveto(65.90081956,209.99990193)(66.02081944,209.90490203)(66.13082169,209.79490356)
\curveto(66.25081921,209.68490225)(66.35081911,209.54990238)(66.43082169,209.38990356)
\curveto(66.51081895,209.23990269)(66.57581889,209.08490285)(66.62582169,208.92490356)
\curveto(66.64581882,208.85490308)(66.65581881,208.78990314)(66.65582169,208.72990356)
\curveto(66.65581881,208.66990326)(66.6658188,208.59990333)(66.68582169,208.51990356)
\curveto(66.70581876,208.45990347)(66.71581875,208.36990356)(66.71582169,208.24990356)
\curveto(66.72581874,208.1299038)(66.72081874,208.03990389)(66.70082169,207.97990356)
\curveto(66.68081878,207.91990401)(66.6608188,207.86990406)(66.64082169,207.82990356)
\curveto(66.63081883,207.77990415)(66.60581886,207.74490419)(66.56582169,207.72490356)
\curveto(66.52581894,207.71490422)(66.47081899,207.70990422)(66.40082169,207.70990356)
\curveto(66.33081913,207.69990423)(66.25581921,207.69990423)(66.17582169,207.70990356)
\curveto(66.10581936,207.71990421)(66.03081943,207.73990419)(65.95082169,207.76990356)
\curveto(65.88081958,207.78990414)(65.82581964,207.81490412)(65.78582169,207.84490356)
\curveto(65.70581976,207.90490403)(65.6608198,207.96990396)(65.65082169,208.03990356)
\curveto(65.64081982,208.10990382)(65.62581984,208.19990373)(65.60582169,208.30990356)
\curveto(65.60581986,208.3299036)(65.60081986,208.34990358)(65.59082169,208.36990356)
\lineto(65.59082169,208.42990356)
\curveto(65.57081989,208.54990338)(65.53581993,208.65490328)(65.48582169,208.74490356)
\curveto(65.40582006,208.90490303)(65.25582021,209.0299029)(65.03582169,209.11990356)
\curveto(64.99582047,209.13990279)(64.95582051,209.15490278)(64.91582169,209.16490356)
\curveto(64.87582059,209.18490275)(64.83082063,209.20490273)(64.78082169,209.22490356)
\curveto(64.72082074,209.24490269)(64.65582081,209.25490268)(64.58582169,209.25490356)
\curveto(64.51582095,209.26490267)(64.45582101,209.27990265)(64.40582169,209.29990356)
\lineto(56.83082169,210.81490356)
\curveto(56.72082874,210.8349011)(56.60582886,210.85490108)(56.48582169,210.87490356)
\curveto(56.35582911,210.90490103)(56.25082921,210.94990098)(56.17082169,211.00990356)
\curveto(56.12082934,211.05990087)(56.07582939,211.14490079)(56.03582169,211.26490356)
\curveto(56.03582943,211.28490065)(56.04082942,211.30490063)(56.05082169,211.32490356)
\curveto(56.05082941,211.35490058)(56.04082942,211.37990055)(56.02082169,211.39990356)
\curveto(56.02082944,211.54990038)(56.02582944,211.69490024)(56.03582169,211.83490356)
\curveto(56.03582943,211.98489995)(56.08082938,212.07989985)(56.17082169,212.11990356)
\curveto(56.25082921,212.15989977)(56.37082909,212.16489977)(56.53082169,212.13490356)
\curveto(56.68082878,212.10489983)(56.82582864,212.07489986)(56.96582169,212.04490356)
\lineto(64.42082169,210.54490356)
}
}
{
\newrgbcolor{curcolor}{0 0 0}
\pscustom[linestyle=none,fillstyle=solid,fillcolor=curcolor]
{
\newpath
\moveto(59.59082169,220.21474731)
\curveto(59.69082577,220.21473881)(59.80582566,220.19473883)(59.93582169,220.15474731)
\curveto(60.05582541,220.11473891)(60.14082532,220.06473896)(60.19082169,220.00474731)
\curveto(60.23082523,219.94473908)(60.2608252,219.86473916)(60.28082169,219.76474731)
\curveto(60.29082517,219.66473936)(60.29582517,219.55473947)(60.29582169,219.43474731)
\lineto(60.29582169,219.07474731)
\curveto(60.28582518,218.96474006)(60.28082518,218.86474016)(60.28082169,218.77474731)
\lineto(60.28082169,214.93474731)
\curveto(60.28082518,214.85474417)(60.28582518,214.76974425)(60.29582169,214.67974731)
\curveto(60.29582517,214.59974442)(60.31082515,214.53474449)(60.34082169,214.48474731)
\curveto(60.3608251,214.43474459)(60.40082506,214.38474464)(60.46082169,214.33474731)
\lineto(60.59582169,214.24474731)
\curveto(60.64582482,214.21474481)(60.69582477,214.20474482)(60.74582169,214.21474731)
\curveto(60.79582467,214.21474481)(60.84082462,214.20974481)(60.88082169,214.19974731)
\lineto(61.00082169,214.19974731)
\lineto(61.25582169,214.19974731)
\curveto(61.33582413,214.20974481)(61.41582405,214.2247448)(61.49582169,214.24474731)
\curveto(62.03582343,214.37474465)(62.42082304,214.67974434)(62.65082169,215.15974731)
\curveto(62.68082278,215.20974381)(62.70582276,215.26974375)(62.72582169,215.33974731)
\curveto(62.74582272,215.40974361)(62.7658227,215.47474355)(62.78582169,215.53474731)
\curveto(62.79582267,215.56474346)(62.80082266,215.61474341)(62.80082169,215.68474731)
\curveto(62.84082262,215.81474321)(62.8608226,215.99474303)(62.86082169,216.22474731)
\curveto(62.8608226,216.45474257)(62.84082262,216.64474238)(62.80082169,216.79474731)
\curveto(62.7608227,216.94474208)(62.72082274,217.07974194)(62.68082169,217.19974731)
\curveto(62.63082283,217.32974169)(62.57082289,217.44974157)(62.50082169,217.55974731)
\curveto(62.43082303,217.67974134)(62.35082311,217.78974123)(62.26082169,217.88974731)
\curveto(62.1608233,217.98974103)(62.05582341,218.07974094)(61.94582169,218.15974731)
\curveto(61.84582362,218.23974078)(61.74082372,218.31474071)(61.63082169,218.38474731)
\curveto(61.52082394,218.45474057)(61.44082402,218.54974047)(61.39082169,218.66974731)
\curveto(61.37082409,218.70974031)(61.35582411,218.77474025)(61.34582169,218.86474731)
\curveto(61.33582413,218.96474006)(61.33582413,219.05473997)(61.34582169,219.13474731)
\curveto(61.34582412,219.2247398)(61.35082411,219.30973971)(61.36082169,219.38974731)
\curveto(61.37082409,219.46973955)(61.39082407,219.5197395)(61.42082169,219.53974731)
\curveto(61.49082397,219.62973939)(61.60582386,219.63473939)(61.76582169,219.55474731)
\curveto(62.03582343,219.41473961)(62.27582319,219.25973976)(62.48582169,219.08974731)
\curveto(62.80582266,218.82974019)(63.07082239,218.54974047)(63.28082169,218.24974731)
\curveto(63.48082198,217.95974106)(63.64582182,217.60474142)(63.77582169,217.18474731)
\curveto(63.81582165,217.07474195)(63.84082162,216.96974205)(63.85082169,216.86974731)
\curveto(63.87082159,216.76974225)(63.89082157,216.65974236)(63.91082169,216.53974731)
\curveto(63.92082154,216.48974253)(63.92582154,216.43974258)(63.92582169,216.38974731)
\curveto(63.92582154,216.34974267)(63.93082153,216.30474272)(63.94082169,216.25474731)
\lineto(63.94082169,216.10474731)
\curveto(63.95082151,216.05474297)(63.95582151,215.99474303)(63.95582169,215.92474731)
\curveto(63.95582151,215.86474316)(63.95082151,215.81474321)(63.94082169,215.77474731)
\lineto(63.94082169,215.63974731)
\curveto(63.93082153,215.58974343)(63.92582154,215.54474348)(63.92582169,215.50474731)
\curveto(63.92582154,215.46474356)(63.92082154,215.4247436)(63.91082169,215.38474731)
\curveto(63.90082156,215.33474369)(63.89082157,215.27974374)(63.88082169,215.21974731)
\curveto(63.88082158,215.16974385)(63.87582159,215.1197439)(63.86582169,215.06974731)
\curveto(63.84582162,214.97974404)(63.82082164,214.88974413)(63.79082169,214.79974731)
\curveto(63.77082169,214.7197443)(63.74582172,214.64474438)(63.71582169,214.57474731)
\curveto(63.69582177,214.53474449)(63.68582178,214.49974452)(63.68582169,214.46974731)
\curveto(63.67582179,214.43974458)(63.6608218,214.40974461)(63.64082169,214.37974731)
\curveto(63.57082189,214.23974478)(63.48582198,214.09474493)(63.38582169,213.94474731)
\curveto(63.19582227,213.69474533)(62.9658225,213.49474553)(62.69582169,213.34474731)
\curveto(62.41582305,213.19474583)(62.10582336,213.08474594)(61.76582169,213.01474731)
\curveto(61.65582381,212.98474604)(61.54082392,212.96974605)(61.42082169,212.96974731)
\curveto(61.30082416,212.96974605)(61.18082428,212.95974606)(61.06082169,212.93974731)
\lineto(60.95582169,212.93974731)
\curveto(60.92582454,212.94974607)(60.88582458,212.95474607)(60.83582169,212.95474731)
\lineto(60.58082169,212.95474731)
\curveto(60.49082497,212.96474606)(60.40082506,212.96974605)(60.31082169,212.96974731)
\lineto(60.10082169,213.01474731)
\curveto(60.0608254,213.01474601)(60.00582546,213.019746)(59.93582169,213.02974731)
\curveto(59.85582561,213.03974598)(59.79082567,213.05474597)(59.74082169,213.07474731)
\lineto(59.57582169,213.10474731)
\curveto(59.52582594,213.13474589)(59.47582599,213.14974587)(59.42582169,213.14974731)
\curveto(59.3658261,213.15974586)(59.31082615,213.17474585)(59.26082169,213.19474731)
\curveto(59.10082636,213.26474576)(58.94082652,213.32974569)(58.78082169,213.38974731)
\curveto(58.62082684,213.44974557)(58.47082699,213.5247455)(58.33082169,213.61474731)
\curveto(58.22082724,213.68474534)(58.11082735,213.74974527)(58.00082169,213.80974731)
\curveto(57.88082758,213.87974514)(57.7658277,213.95974506)(57.65582169,214.04974731)
\curveto(57.30582816,214.33974468)(57.00582846,214.64974437)(56.75582169,214.97974731)
\curveto(56.49582897,215.30974371)(56.28082918,215.69474333)(56.11082169,216.13474731)
\curveto(56.0608294,216.26474276)(56.02582944,216.39474263)(56.00582169,216.52474731)
\curveto(55.97582949,216.65474237)(55.94582952,216.79474223)(55.91582169,216.94474731)
\curveto(55.90582956,216.99474203)(55.90082956,217.03974198)(55.90082169,217.07974731)
\curveto(55.89082957,217.1197419)(55.88582958,217.16474186)(55.88582169,217.21474731)
\curveto(55.87582959,217.23474179)(55.87582959,217.25974176)(55.88582169,217.28974731)
\curveto(55.89582957,217.3197417)(55.89082957,217.34474168)(55.87082169,217.36474731)
\curveto(55.8608296,217.79474123)(55.90582956,218.15474087)(56.00582169,218.44474731)
\curveto(56.09582937,218.73474029)(56.22082924,218.98974003)(56.38082169,219.20974731)
\curveto(56.40082906,219.24973977)(56.43082903,219.27973974)(56.47082169,219.29974731)
\curveto(56.50082896,219.32973969)(56.52582894,219.35973966)(56.54582169,219.38974731)
\curveto(56.60582886,219.45973956)(56.67582879,219.52973949)(56.75582169,219.59974731)
\curveto(56.83582863,219.66973935)(56.91582855,219.7247393)(56.99582169,219.76474731)
\curveto(57.20582826,219.88473914)(57.40582806,219.97973904)(57.59582169,220.04974731)
\curveto(57.70582776,220.09973892)(57.82582764,220.12973889)(57.95582169,220.13974731)
\lineto(58.34582169,220.19974731)
\curveto(58.47582699,220.22973879)(58.61082685,220.23973878)(58.75082169,220.22974731)
\curveto(58.89082657,220.22973879)(59.03082643,220.23473879)(59.17082169,220.24474731)
\curveto(59.24082622,220.24473878)(59.31082615,220.23973878)(59.38082169,220.22974731)
\curveto(59.45082601,220.2197388)(59.52082594,220.21473881)(59.59082169,220.21474731)
\moveto(59.08082169,218.86474731)
\curveto(59.04082642,218.89474013)(58.99082647,218.9247401)(58.93082169,218.95474731)
\curveto(58.8608266,218.99474003)(58.79082667,219.00974001)(58.72082169,218.99974731)
\curveto(58.50082696,218.98974003)(58.29582717,218.94974007)(58.10582169,218.87974731)
\curveto(57.87582759,218.77974024)(57.68082778,218.65974036)(57.52082169,218.51974731)
\curveto(57.3608281,218.38974063)(57.22582824,218.19974082)(57.11582169,217.94974731)
\curveto(57.09582837,217.87974114)(57.08082838,217.80974121)(57.07082169,217.73974731)
\curveto(57.05082841,217.67974134)(57.03082843,217.60974141)(57.01082169,217.52974731)
\curveto(56.99082847,217.45974156)(56.98082848,217.37974164)(56.98082169,217.28974731)
\lineto(56.98082169,217.03474731)
\curveto(57.00082846,216.99474203)(57.01082845,216.95474207)(57.01082169,216.91474731)
\curveto(57.00082846,216.87474215)(57.00082846,216.83974218)(57.01082169,216.80974731)
\lineto(57.07082169,216.56974731)
\curveto(57.08082838,216.48974253)(57.09582837,216.41474261)(57.11582169,216.34474731)
\curveto(57.23582823,216.024743)(57.38582808,215.75974326)(57.56582169,215.54974731)
\curveto(57.74582772,215.33974368)(57.97082749,215.13974388)(58.24082169,214.94974731)
\curveto(58.29082717,214.90974411)(58.35582711,214.86474416)(58.43582169,214.81474731)
\curveto(58.50582696,214.77474425)(58.58582688,214.73474429)(58.67582169,214.69474731)
\curveto(58.7658267,214.65474437)(58.85082661,214.62974439)(58.93082169,214.61974731)
\curveto(59.01082645,214.6197444)(59.07082639,214.64474438)(59.11082169,214.69474731)
\curveto(59.17082629,214.76474426)(59.20082626,214.89474413)(59.20082169,215.08474731)
\curveto(59.19082627,215.28474374)(59.18582628,215.45474357)(59.18582169,215.59474731)
\lineto(59.18582169,217.87474731)
\curveto(59.18582628,218.024741)(59.19082627,218.20474082)(59.20082169,218.41474731)
\curveto(59.20082626,218.6247404)(59.1608263,218.77474025)(59.08082169,218.86474731)
}
}
{
\newrgbcolor{curcolor}{0 0 0}
\pscustom[linestyle=none,fillstyle=solid,fillcolor=curcolor]
{
}
}
{
\newrgbcolor{curcolor}{0 0 0}
\pscustom[linestyle=none,fillstyle=solid,fillcolor=curcolor]
{
\newpath
\moveto(63.20582169,231.84654419)
\curveto(63.3658221,231.83653628)(63.50082196,231.79153632)(63.61082169,231.71154419)
\curveto(63.71082175,231.63153648)(63.78582168,231.53653658)(63.83582169,231.42654419)
\curveto(63.85582161,231.37653674)(63.8658216,231.32153679)(63.86582169,231.26154419)
\curveto(63.8658216,231.2115369)(63.87582159,231.15153696)(63.89582169,231.08154419)
\curveto(63.94582152,230.85153726)(63.93082153,230.63653748)(63.85082169,230.43654419)
\curveto(63.78082168,230.23653788)(63.69082177,230.111538)(63.58082169,230.06154419)
\curveto(63.51082195,230.02153809)(63.43082203,229.99153812)(63.34082169,229.97154419)
\curveto(63.24082222,229.95153816)(63.1608223,229.9165382)(63.10082169,229.86654419)
\lineto(63.04082169,229.80654419)
\curveto(63.02082244,229.78653833)(63.01582245,229.75653836)(63.02582169,229.71654419)
\curveto(63.05582241,229.59653852)(63.11082235,229.48153863)(63.19082169,229.37154419)
\curveto(63.27082219,229.26153885)(63.34082212,229.15653896)(63.40082169,229.05654419)
\curveto(63.48082198,228.90653921)(63.55582191,228.75153936)(63.62582169,228.59154419)
\curveto(63.68582178,228.43153968)(63.74082172,228.26153985)(63.79082169,228.08154419)
\curveto(63.82082164,227.97154014)(63.84082162,227.85654026)(63.85082169,227.73654419)
\curveto(63.8608216,227.62654049)(63.87582159,227.5115406)(63.89582169,227.39154419)
\curveto(63.90582156,227.34154077)(63.91082155,227.29654082)(63.91082169,227.25654419)
\lineto(63.91082169,227.15154419)
\curveto(63.93082153,227.04154107)(63.93082153,226.93654118)(63.91082169,226.83654419)
\lineto(63.91082169,226.70154419)
\curveto(63.90082156,226.65154146)(63.89582157,226.60154151)(63.89582169,226.55154419)
\curveto(63.89582157,226.50154161)(63.88582158,226.46154165)(63.86582169,226.43154419)
\curveto(63.85582161,226.39154172)(63.85082161,226.35654176)(63.85082169,226.32654419)
\curveto(63.8608216,226.30654181)(63.8608216,226.28154183)(63.85082169,226.25154419)
\lineto(63.79082169,226.01154419)
\curveto(63.78082168,225.94154217)(63.7608217,225.87654224)(63.73082169,225.81654419)
\curveto(63.60082186,225.53654258)(63.45582201,225.32154279)(63.29582169,225.17154419)
\curveto(63.12582234,225.02154309)(62.89082257,224.9165432)(62.59082169,224.85654419)
\curveto(62.37082309,224.80654331)(62.10582336,224.8115433)(61.79582169,224.87154419)
\lineto(61.48082169,224.94654419)
\curveto(61.43082403,224.96654315)(61.38082408,224.98154313)(61.33082169,224.99154419)
\lineto(61.15082169,225.05154419)
\lineto(60.82082169,225.23154419)
\curveto(60.71082475,225.30154281)(60.61082485,225.37154274)(60.52082169,225.44154419)
\curveto(60.23082523,225.68154243)(60.01582545,225.97154214)(59.87582169,226.31154419)
\curveto(59.73582573,226.65154146)(59.61082585,227.0165411)(59.50082169,227.40654419)
\curveto(59.460826,227.55654056)(59.43082603,227.70654041)(59.41082169,227.85654419)
\curveto(59.39082607,228.0165401)(59.3658261,228.17153994)(59.33582169,228.32154419)
\curveto(59.31582615,228.40153971)(59.30582616,228.47153964)(59.30582169,228.53154419)
\curveto(59.30582616,228.60153951)(59.29582617,228.67653944)(59.27582169,228.75654419)
\curveto(59.25582621,228.82653929)(59.24582622,228.89653922)(59.24582169,228.96654419)
\curveto(59.23582623,229.04653907)(59.22082624,229.12653899)(59.20082169,229.20654419)
\curveto(59.14082632,229.46653865)(59.09082637,229.7115384)(59.05082169,229.94154419)
\curveto(59.00082646,230.17153794)(58.88582658,230.37153774)(58.70582169,230.54154419)
\curveto(58.62582684,230.6115375)(58.52582694,230.67653744)(58.40582169,230.73654419)
\curveto(58.27582719,230.80653731)(58.13582733,230.83653728)(57.98582169,230.82654419)
\curveto(57.74582772,230.8165373)(57.55582791,230.76653735)(57.41582169,230.67654419)
\curveto(57.27582819,230.59653752)(57.1658283,230.45653766)(57.08582169,230.25654419)
\curveto(57.03582843,230.14653797)(57.00082846,230.0115381)(56.98082169,229.85154419)
\curveto(56.9608285,229.69153842)(56.95082851,229.52153859)(56.95082169,229.34154419)
\curveto(56.95082851,229.16153895)(56.9608285,228.98153913)(56.98082169,228.80154419)
\curveto(57.00082846,228.63153948)(57.03082843,228.48153963)(57.07082169,228.35154419)
\curveto(57.13082833,228.17153994)(57.21582825,227.99154012)(57.32582169,227.81154419)
\curveto(57.38582808,227.72154039)(57.465828,227.63154048)(57.56582169,227.54154419)
\curveto(57.65582781,227.46154065)(57.75582771,227.38654073)(57.86582169,227.31654419)
\curveto(57.94582752,227.26654085)(58.03082743,227.22154089)(58.12082169,227.18154419)
\curveto(58.21082725,227.14154097)(58.28082718,227.08154103)(58.33082169,227.00154419)
\curveto(58.3608271,226.95154116)(58.38582708,226.87654124)(58.40582169,226.77654419)
\curveto(58.41582705,226.67654144)(58.42082704,226.57654154)(58.42082169,226.47654419)
\curveto(58.42082704,226.37654174)(58.41582705,226.28154183)(58.40582169,226.19154419)
\curveto(58.38582708,226.10154201)(58.3608271,226.04154207)(58.33082169,226.01154419)
\curveto(58.30082716,225.97154214)(58.25082721,225.94654217)(58.18082169,225.93654419)
\curveto(58.11082735,225.93654218)(58.03582743,225.95654216)(57.95582169,225.99654419)
\curveto(57.82582764,226.04654207)(57.70582776,226.10154201)(57.59582169,226.16154419)
\curveto(57.47582799,226.22154189)(57.3608281,226.28654183)(57.25082169,226.35654419)
\curveto(56.90082856,226.6165415)(56.63082883,226.9115412)(56.44082169,227.24154419)
\curveto(56.24082922,227.57154054)(56.08082938,227.96154015)(55.96082169,228.41154419)
\curveto(55.94082952,228.52153959)(55.92582954,228.62653949)(55.91582169,228.72654419)
\curveto(55.90582956,228.83653928)(55.89082957,228.94653917)(55.87082169,229.05654419)
\curveto(55.8608296,229.10653901)(55.8608296,229.17153894)(55.87082169,229.25154419)
\curveto(55.87082959,229.34153877)(55.8608296,229.40153871)(55.84082169,229.43154419)
\curveto(55.83082963,230.13153798)(55.91082955,230.72153739)(56.08082169,231.20154419)
\curveto(56.25082921,231.69153642)(56.57582889,231.99653612)(57.05582169,232.11654419)
\curveto(57.25582821,232.16653595)(57.49082797,232.17153594)(57.76082169,232.13154419)
\curveto(58.02082744,232.09153602)(58.29582717,232.04153607)(58.58582169,231.98154419)
\lineto(61.90082169,231.32154419)
\curveto(62.04082342,231.29153682)(62.17582329,231.26653685)(62.30582169,231.24654419)
\curveto(62.43582303,231.23653688)(62.54082292,231.24653687)(62.62082169,231.27654419)
\curveto(62.69082277,231.3165368)(62.74082272,231.37153674)(62.77082169,231.44154419)
\curveto(62.81082265,231.53153658)(62.84082262,231.6115365)(62.86082169,231.68154419)
\curveto(62.87082259,231.76153635)(62.91582255,231.8115363)(62.99582169,231.83154419)
\curveto(63.02582244,231.85153626)(63.05582241,231.85653626)(63.08582169,231.84654419)
\lineto(63.20582169,231.84654419)
\moveto(61.54082169,230.03154419)
\curveto(61.40082406,230.12153799)(61.24082422,230.18653793)(61.06082169,230.22654419)
\curveto(60.87082459,230.26653785)(60.67582479,230.30653781)(60.47582169,230.34654419)
\curveto(60.3658251,230.36653775)(60.2658252,230.38153773)(60.17582169,230.39154419)
\curveto(60.08582538,230.40153771)(60.01582545,230.37653774)(59.96582169,230.31654419)
\curveto(59.94582552,230.28653783)(59.93582553,230.2165379)(59.93582169,230.10654419)
\curveto(59.95582551,230.08653803)(59.9658255,230.05153806)(59.96582169,230.00154419)
\curveto(59.9658255,229.95153816)(59.97582549,229.90153821)(59.99582169,229.85154419)
\curveto(60.01582545,229.77153834)(60.03582543,229.67653844)(60.05582169,229.56654419)
\lineto(60.11582169,229.26654419)
\curveto(60.11582535,229.23653888)(60.12082534,229.20153891)(60.13082169,229.16154419)
\lineto(60.13082169,229.05654419)
\curveto(60.17082529,228.89653922)(60.19582527,228.72653939)(60.20582169,228.54654419)
\curveto(60.20582526,228.37653974)(60.22582524,228.2115399)(60.26582169,228.05154419)
\curveto(60.28582518,227.96154015)(60.30582516,227.88154023)(60.32582169,227.81154419)
\curveto(60.33582513,227.75154036)(60.35082511,227.67654044)(60.37082169,227.58654419)
\curveto(60.42082504,227.4165407)(60.48582498,227.25154086)(60.56582169,227.09154419)
\curveto(60.63582483,226.94154117)(60.72582474,226.80654131)(60.83582169,226.68654419)
\curveto(60.94582452,226.56654155)(61.08082438,226.46654165)(61.24082169,226.38654419)
\curveto(61.39082407,226.30654181)(61.57582389,226.24654187)(61.79582169,226.20654419)
\curveto(61.89582357,226.18654193)(61.99082347,226.18654193)(62.08082169,226.20654419)
\curveto(62.1608233,226.22654189)(62.23582323,226.25654186)(62.30582169,226.29654419)
\curveto(62.41582305,226.34654177)(62.51082295,226.42654169)(62.59082169,226.53654419)
\curveto(62.6608228,226.65654146)(62.72082274,226.78654133)(62.77082169,226.92654419)
\curveto(62.78082268,226.97654114)(62.78582268,227.02654109)(62.78582169,227.07654419)
\curveto(62.78582268,227.12654099)(62.79082267,227.17654094)(62.80082169,227.22654419)
\curveto(62.82082264,227.29654082)(62.83582263,227.38154073)(62.84582169,227.48154419)
\curveto(62.84582262,227.58154053)(62.83582263,227.67154044)(62.81582169,227.75154419)
\curveto(62.79582267,227.8115403)(62.79082267,227.87154024)(62.80082169,227.93154419)
\curveto(62.80082266,227.99154012)(62.79082267,228.05154006)(62.77082169,228.11154419)
\curveto(62.75082271,228.20153991)(62.73582273,228.28153983)(62.72582169,228.35154419)
\curveto(62.71582275,228.43153968)(62.69582277,228.5115396)(62.66582169,228.59154419)
\curveto(62.54582292,228.90153921)(62.40082306,229.17653894)(62.23082169,229.41654419)
\curveto(62.0608234,229.65653846)(61.83082363,229.86153825)(61.54082169,230.03154419)
}
}
{
\newrgbcolor{curcolor}{0 0 0}
\pscustom[linestyle=none,fillstyle=solid,fillcolor=curcolor]
{
\newpath
\moveto(55.85582169,237.62318481)
\curveto(55.84582962,238.36317843)(55.95582951,238.95817784)(56.18582169,239.40818481)
\curveto(56.40582906,239.85817694)(56.74082872,240.18317661)(57.19082169,240.38318481)
\curveto(57.39082807,240.47317632)(57.63582783,240.53317626)(57.92582169,240.56318481)
\curveto(57.97582749,240.57317622)(58.04082742,240.57317622)(58.12082169,240.56318481)
\curveto(58.20082726,240.56317623)(58.27082719,240.54817625)(58.33082169,240.51818481)
\curveto(58.38082708,240.47817632)(58.42582704,240.41817638)(58.46582169,240.33818481)
\curveto(58.48582698,240.2981765)(58.49582697,240.26317653)(58.49582169,240.23318481)
\curveto(58.48582698,240.21317658)(58.48582698,240.17817662)(58.49582169,240.12818481)
\curveto(58.50582696,240.08817671)(58.51082695,240.04817675)(58.51082169,240.00818481)
\curveto(58.50082696,239.96817683)(58.49582697,239.92817687)(58.49582169,239.88818481)
\lineto(58.49582169,239.57318481)
\curveto(58.48582698,239.48317731)(58.45582701,239.40817739)(58.40582169,239.34818481)
\curveto(58.34582712,239.27817752)(58.2608272,239.23817756)(58.15082169,239.22818481)
\curveto(58.04082742,239.21817758)(57.94582752,239.1981776)(57.86582169,239.16818481)
\curveto(57.60582786,239.06817773)(57.40082806,238.91317788)(57.25082169,238.70318481)
\curveto(57.20082826,238.63317816)(57.1608283,238.55317824)(57.13082169,238.46318481)
\curveto(57.09082837,238.38317841)(57.05582841,238.2981785)(57.02582169,238.20818481)
\curveto(56.98582848,238.07817872)(56.9658285,237.8981789)(56.96582169,237.66818481)
\curveto(56.95582851,237.43817936)(56.97582849,237.24317955)(57.02582169,237.08318481)
\curveto(57.04582842,237.01317978)(57.0608284,236.94317985)(57.07082169,236.87318481)
\curveto(57.08082838,236.81317998)(57.09582837,236.74818005)(57.11582169,236.67818481)
\curveto(57.22582824,236.3981804)(57.37582809,236.13818066)(57.56582169,235.89818481)
\curveto(57.75582771,235.65818114)(57.98082748,235.45818134)(58.24082169,235.29818481)
\curveto(58.33082713,235.23818156)(58.42582704,235.18318161)(58.52582169,235.13318481)
\curveto(58.61582685,235.08318171)(58.71582675,235.03318176)(58.82582169,234.98318481)
\lineto(59.23082169,234.81818481)
\curveto(59.28082618,234.798182)(59.33582613,234.78318201)(59.39582169,234.77318481)
\curveto(59.45582601,234.76318203)(59.51082595,234.74318205)(59.56082169,234.71318481)
\lineto(59.68082169,234.69818481)
\curveto(59.72082574,234.67818212)(59.78582568,234.65318214)(59.87582169,234.62318481)
\curveto(59.9658255,234.60318219)(60.03082543,234.5981822)(60.07082169,234.60818481)
\curveto(60.12082534,234.60818219)(60.17082529,234.5981822)(60.22082169,234.57818481)
\curveto(60.27082519,234.55818224)(60.32082514,234.54818225)(60.37082169,234.54818481)
\curveto(60.41082505,234.55818224)(60.48082498,234.55318224)(60.58082169,234.53318481)
\curveto(60.6608248,234.53318226)(60.74582472,234.52818227)(60.83582169,234.51818481)
\curveto(60.92582454,234.51818228)(61.01082445,234.52318227)(61.09082169,234.53318481)
\curveto(61.41082405,234.57318222)(61.69082377,234.64318215)(61.93082169,234.74318481)
\curveto(62.1608233,234.84318195)(62.3608231,235.00818179)(62.53082169,235.23818481)
\curveto(62.58082288,235.31818148)(62.62582284,235.3981814)(62.66582169,235.47818481)
\curveto(62.70582276,235.56818123)(62.74582272,235.66318113)(62.78582169,235.76318481)
\curveto(62.79582267,235.81318098)(62.80082266,235.85318094)(62.80082169,235.88318481)
\curveto(62.80082266,235.91318088)(62.80582266,235.95318084)(62.81582169,236.00318481)
\curveto(62.82582264,236.03318076)(62.83082263,236.08318071)(62.83082169,236.15318481)
\lineto(62.83082169,236.31818481)
\curveto(62.84082262,236.30818049)(62.84582262,236.32318047)(62.84582169,236.36318481)
\curveto(62.83582263,236.3931804)(62.83582263,236.41818038)(62.84582169,236.43818481)
\curveto(62.84582262,236.46818033)(62.84082262,236.50318029)(62.83082169,236.54318481)
\curveto(62.81082265,236.61318018)(62.80582266,236.67818012)(62.81582169,236.73818481)
\curveto(62.81582265,236.80817999)(62.80582266,236.87817992)(62.78582169,236.94818481)
\curveto(62.70582276,237.22817957)(62.60582286,237.47317932)(62.48582169,237.68318481)
\curveto(62.35582311,237.90317889)(62.19082327,238.0981787)(61.99082169,238.26818481)
\curveto(61.91082355,238.32817847)(61.82582364,238.38817841)(61.73582169,238.44818481)
\lineto(61.46582169,238.62818481)
\curveto(61.38582408,238.65817814)(61.31082415,238.6931781)(61.24082169,238.73318481)
\curveto(61.1608243,238.77317802)(61.09582437,238.83317796)(61.04582169,238.91318481)
\curveto(61.01582445,238.95317784)(60.99582447,239.01817778)(60.98582169,239.10818481)
\curveto(60.9658245,239.20817759)(60.95582451,239.30817749)(60.95582169,239.40818481)
\curveto(60.94582452,239.51817728)(60.94582452,239.61817718)(60.95582169,239.70818481)
\curveto(60.9658245,239.798177)(60.98582448,239.86317693)(61.01582169,239.90318481)
\curveto(61.05582441,239.95317684)(61.11582435,239.97817682)(61.19582169,239.97818481)
\curveto(61.27582419,239.97817682)(61.3608241,239.95817684)(61.45082169,239.91818481)
\curveto(61.60082386,239.83817696)(61.74582372,239.76317703)(61.88582169,239.69318481)
\curveto(62.01582345,239.62317717)(62.14582332,239.53817726)(62.27582169,239.43818481)
\curveto(62.57582289,239.22817757)(62.84082262,238.98817781)(63.07082169,238.71818481)
\curveto(63.30082216,238.44817835)(63.48582198,238.13817866)(63.62582169,237.78818481)
\curveto(63.6658218,237.6981791)(63.70082176,237.60317919)(63.73082169,237.50318481)
\curveto(63.75082171,237.41317938)(63.77582169,237.31817948)(63.80582169,237.21818481)
\curveto(63.84582162,237.0981797)(63.8658216,236.98317981)(63.86582169,236.87318481)
\curveto(63.87582159,236.76318003)(63.89082157,236.64818015)(63.91082169,236.52818481)
\curveto(63.93082153,236.48818031)(63.93582153,236.44818035)(63.92582169,236.40818481)
\curveto(63.91582155,236.36818043)(63.91582155,236.32818047)(63.92582169,236.28818481)
\lineto(63.92582169,236.15318481)
\lineto(63.92582169,235.91318481)
\curveto(63.93582153,235.84318095)(63.93082153,235.77818102)(63.91082169,235.71818481)
\lineto(63.91082169,235.64318481)
\lineto(63.86582169,235.29818481)
\curveto(63.82582164,235.17818162)(63.79082167,235.05818174)(63.76082169,234.93818481)
\curveto(63.73082173,234.82818197)(63.69082177,234.72318207)(63.64082169,234.62318481)
\curveto(63.48082198,234.2931825)(63.29082217,234.03318276)(63.07082169,233.84318481)
\curveto(62.85082261,233.65318314)(62.58082288,233.48818331)(62.26082169,233.34818481)
\curveto(62.18082328,233.31818348)(62.09082337,233.2931835)(61.99082169,233.27318481)
\lineto(61.69082169,233.21318481)
\curveto(61.58082388,233.18318361)(61.465824,233.16818363)(61.34582169,233.16818481)
\curveto(61.22582424,233.17818362)(61.10582436,233.17818362)(60.98582169,233.16818481)
\curveto(60.94582452,233.16818363)(60.90582456,233.17318362)(60.86582169,233.18318481)
\curveto(60.82582464,233.1931836)(60.78582468,233.1931836)(60.74582169,233.18318481)
\curveto(60.68582478,233.18318361)(60.62082484,233.18818361)(60.55082169,233.19818481)
\curveto(60.48082498,233.21818358)(60.41582505,233.22818357)(60.35582169,233.22818481)
\lineto(60.20582169,233.25818481)
\curveto(60.15582531,233.25818354)(60.08582538,233.26318353)(59.99582169,233.27318481)
\curveto(59.90582556,233.2931835)(59.83582563,233.31318348)(59.78582169,233.33318481)
\curveto(59.73582573,233.35318344)(59.69082577,233.36318343)(59.65082169,233.36318481)
\curveto(59.61082585,233.37318342)(59.57082589,233.38818341)(59.53082169,233.40818481)
\curveto(59.460826,233.43818336)(59.39082607,233.45818334)(59.32082169,233.46818481)
\curveto(59.25082621,233.47818332)(59.18582628,233.4981833)(59.12582169,233.52818481)
\curveto(58.95582651,233.5981832)(58.78582668,233.66318313)(58.61582169,233.72318481)
\curveto(58.44582702,233.793183)(58.28582718,233.87318292)(58.13582169,233.96318481)
\curveto(57.61582785,234.28318251)(57.19582827,234.62318217)(56.87582169,234.98318481)
\curveto(56.55582891,235.34318145)(56.29082917,235.80818099)(56.08082169,236.37818481)
\curveto(56.03082943,236.4981803)(55.99582947,236.62318017)(55.97582169,236.75318481)
\curveto(55.95582951,236.88317991)(55.93082953,237.02317977)(55.90082169,237.17318481)
\curveto(55.89082957,237.24317955)(55.88582958,237.31317948)(55.88582169,237.38318481)
\curveto(55.87582959,237.45317934)(55.8658296,237.53317926)(55.85582169,237.62318481)
}
}
{
\newrgbcolor{curcolor}{0 0 0}
\pscustom[linestyle=none,fillstyle=solid,fillcolor=curcolor]
{
\newpath
\moveto(56.03582169,242.96482544)
\lineto(56.03582169,243.39982544)
\curveto(56.03582943,243.54982193)(56.07582939,243.64982183)(56.15582169,243.69982544)
\curveto(56.23582923,243.72982175)(56.33582913,243.73482174)(56.45582169,243.71482544)
\lineto(56.81582169,243.65482544)
\lineto(58.24082169,243.36982544)
\lineto(60.50582169,242.91982544)
\curveto(60.72582474,242.86982261)(60.95582451,242.81982266)(61.19582169,242.76982544)
\curveto(61.42582404,242.72982275)(61.62582384,242.71482276)(61.79582169,242.72482544)
\curveto(62.24582322,242.79482268)(62.5608229,243.03482244)(62.74082169,243.44482544)
\curveto(62.83082263,243.64482183)(62.8658226,243.89982158)(62.84582169,244.20982544)
\curveto(62.81582265,244.52982095)(62.7608227,244.79482068)(62.68082169,245.00482544)
\curveto(62.54082292,245.35482012)(62.3658231,245.64981983)(62.15582169,245.88982544)
\curveto(61.93582353,246.12981935)(61.65082381,246.33981914)(61.30082169,246.51982544)
\curveto(61.22082424,246.56981891)(61.14082432,246.60481887)(61.06082169,246.62482544)
\curveto(60.98082448,246.65481882)(60.89582457,246.68981879)(60.80582169,246.72982544)
\curveto(60.75582471,246.74981873)(60.71082475,246.75981872)(60.67082169,246.75982544)
\curveto(60.63082483,246.75981872)(60.58582488,246.7748187)(60.53582169,246.80482544)
\lineto(60.22082169,246.86482544)
\curveto(60.14082532,246.90481857)(60.05082541,246.92981855)(59.95082169,246.93982544)
\curveto(59.84082562,246.94981853)(59.74082572,246.96481851)(59.65082169,246.98482544)
\lineto(58.48082169,247.22482544)
\lineto(56.89082169,247.53982544)
\curveto(56.77082869,247.55981792)(56.64582882,247.5798179)(56.51582169,247.59982544)
\curveto(56.37582909,247.62981785)(56.2658292,247.6748178)(56.18582169,247.73482544)
\curveto(56.13582933,247.78481769)(56.10582936,247.83981764)(56.09582169,247.89982544)
\curveto(56.07582939,247.95981752)(56.05582941,248.02981745)(56.03582169,248.10982544)
\lineto(56.03582169,248.33482544)
\curveto(56.03582943,248.45481702)(56.04082942,248.55981692)(56.05082169,248.64982544)
\curveto(56.0608294,248.74981673)(56.10582936,248.81481666)(56.18582169,248.84482544)
\curveto(56.23582923,248.88481659)(56.31082915,248.89481658)(56.41082169,248.87482544)
\curveto(56.50082896,248.85481662)(56.59582887,248.83481664)(56.69582169,248.81482544)
\lineto(57.71582169,248.61982544)
\lineto(61.75082169,247.80982544)
\lineto(63.10082169,247.53982544)
\curveto(63.22082224,247.51981796)(63.33582213,247.49481798)(63.44582169,247.46482544)
\curveto(63.54582192,247.43481804)(63.62082184,247.3798181)(63.67082169,247.29982544)
\curveto(63.70082176,247.25981822)(63.72582174,247.19481828)(63.74582169,247.10482544)
\curveto(63.75582171,247.02481845)(63.7658217,246.93481854)(63.77582169,246.83482544)
\curveto(63.77582169,246.74481873)(63.77082169,246.65481882)(63.76082169,246.56482544)
\curveto(63.75082171,246.48481899)(63.73082173,246.42981905)(63.70082169,246.39982544)
\curveto(63.6608218,246.35981912)(63.59582187,246.32981915)(63.50582169,246.30982544)
\curveto(63.465822,246.29981918)(63.41082205,246.29981918)(63.34082169,246.30982544)
\curveto(63.27082219,246.31981916)(63.20582226,246.32481915)(63.14582169,246.32482544)
\curveto(63.07582239,246.33481914)(63.02082244,246.32481915)(62.98082169,246.29482544)
\curveto(62.94082252,246.2748192)(62.92582254,246.23481924)(62.93582169,246.17482544)
\curveto(62.95582251,246.09481938)(63.01582245,246.00481947)(63.11582169,245.90482544)
\curveto(63.20582226,245.80481967)(63.27582219,245.71481976)(63.32582169,245.63482544)
\curveto(63.48582198,245.38482009)(63.62582184,245.10482037)(63.74582169,244.79482544)
\curveto(63.79582167,244.6748208)(63.82582164,244.55482092)(63.83582169,244.43482544)
\curveto(63.85582161,244.32482115)(63.88082158,244.20482127)(63.91082169,244.07482544)
\curveto(63.92082154,244.02482145)(63.92082154,243.96982151)(63.91082169,243.90982544)
\curveto(63.90082156,243.85982162)(63.90582156,243.80982167)(63.92582169,243.75982544)
\curveto(63.94582152,243.65982182)(63.94582152,243.56982191)(63.92582169,243.48982544)
\lineto(63.92582169,243.33982544)
\curveto(63.90582156,243.28982219)(63.89582157,243.22982225)(63.89582169,243.15982544)
\curveto(63.89582157,243.09982238)(63.89082157,243.04982243)(63.88082169,243.00982544)
\curveto(63.8608216,242.96982251)(63.85082161,242.92982255)(63.85082169,242.88982544)
\curveto(63.8608216,242.85982262)(63.85582161,242.81982266)(63.83582169,242.76982544)
\curveto(63.81582165,242.69982278)(63.79582167,242.62482285)(63.77582169,242.54482544)
\curveto(63.75582171,242.474823)(63.72582174,242.40482307)(63.68582169,242.33482544)
\curveto(63.57582189,242.09482338)(63.43082203,241.90482357)(63.25082169,241.76482544)
\curveto(63.0608224,241.63482384)(62.83582263,241.53982394)(62.57582169,241.47982544)
\curveto(62.48582298,241.45982402)(62.39582307,241.44982403)(62.30582169,241.44982544)
\lineto(62.00582169,241.44982544)
\curveto(61.94582352,241.43982404)(61.89082357,241.43982404)(61.84082169,241.44982544)
\curveto(61.78082368,241.46982401)(61.71582375,241.474824)(61.64582169,241.46482544)
\lineto(61.57082169,241.46482544)
\curveto(61.53082393,241.474824)(61.49582397,241.479824)(61.46582169,241.47982544)
\lineto(61.31582169,241.50982544)
\curveto(61.27582419,241.50982397)(61.23082423,241.51482396)(61.18082169,241.52482544)
\curveto(61.12082434,241.54482393)(61.0658244,241.55982392)(61.01582169,241.56982544)
\lineto(60.41582169,241.68982544)
\lineto(57.65582169,242.24482544)
\lineto(56.69582169,242.42482544)
\lineto(56.42582169,242.48482544)
\curveto(56.33582913,242.50482297)(56.2608292,242.53982294)(56.20082169,242.58982544)
\curveto(56.13082933,242.63982284)(56.08082938,242.72482275)(56.05082169,242.84482544)
\curveto(56.04082942,242.86482261)(56.04082942,242.88482259)(56.05082169,242.90482544)
\curveto(56.05082941,242.92482255)(56.04582942,242.94482253)(56.03582169,242.96482544)
}
}
{
\newrgbcolor{curcolor}{0 0 0}
\pscustom[linestyle=none,fillstyle=solid,fillcolor=curcolor]
{
\newpath
\moveto(55.85582169,254.58443481)
\curveto(55.84582962,254.96442824)(55.88582958,255.27442793)(55.97582169,255.51443481)
\curveto(56.0658294,255.76442744)(56.19582927,255.98442722)(56.36582169,256.17443481)
\curveto(56.41582905,256.24442696)(56.48582898,256.29442691)(56.57582169,256.32443481)
\curveto(56.65582881,256.36442684)(56.73082873,256.41442679)(56.80082169,256.47443481)
\curveto(56.82082864,256.49442671)(56.84582862,256.51942668)(56.87582169,256.54943481)
\curveto(56.90582856,256.57942662)(56.91582855,256.62942657)(56.90582169,256.69943481)
\curveto(56.87582859,256.7994264)(56.81582865,256.89442631)(56.72582169,256.98443481)
\curveto(56.62582884,257.08442612)(56.55082891,257.17942602)(56.50082169,257.26943481)
\curveto(56.39082907,257.42942577)(56.29582917,257.59442561)(56.21582169,257.76443481)
\curveto(56.12582934,257.93442527)(56.05082941,258.11442509)(55.99082169,258.30443481)
\curveto(55.9608295,258.38442482)(55.94082952,258.47442473)(55.93082169,258.57443481)
\curveto(55.91082955,258.67442453)(55.89082957,258.76942443)(55.87082169,258.85943481)
\curveto(55.8608296,258.91942428)(55.85582961,258.96942423)(55.85582169,259.00943481)
\lineto(55.85582169,259.15943481)
\curveto(55.83582963,259.20942399)(55.83082963,259.27942392)(55.84082169,259.36943481)
\curveto(55.84082962,259.45942374)(55.84582962,259.52442368)(55.85582169,259.56443481)
\curveto(55.8658296,259.6044236)(55.87082959,259.67442353)(55.87082169,259.77443481)
\curveto(55.89082957,259.86442334)(55.91082955,259.94942325)(55.93082169,260.02943481)
\curveto(55.94082952,260.10942309)(55.9608295,260.18942301)(55.99082169,260.26943481)
\curveto(56.01082945,260.30942289)(56.02582944,260.34942285)(56.03582169,260.38943481)
\curveto(56.03582943,260.43942276)(56.04582942,260.48942271)(56.06582169,260.53943481)
\curveto(56.28582918,260.98942221)(56.62582884,261.26442194)(57.08582169,261.36443481)
\curveto(57.1658283,261.39442181)(57.25582821,261.40942179)(57.35582169,261.40943481)
\curveto(57.44582802,261.40942179)(57.54582792,261.40942179)(57.65582169,261.40943481)
\curveto(57.68582778,261.41942178)(57.72082774,261.41942178)(57.76082169,261.40943481)
\curveto(57.79082767,261.3994218)(57.82082764,261.39442181)(57.85082169,261.39443481)
\lineto(57.98582169,261.36443481)
\curveto(58.02582744,261.36442184)(58.07082739,261.35442185)(58.12082169,261.33443481)
\curveto(58.17082729,261.32442188)(58.22082724,261.31442189)(58.27082169,261.30443481)
\lineto(58.85582169,261.19943481)
\lineto(59.81582169,261.00443481)
\lineto(62.66582169,260.43443481)
\curveto(62.82582264,260.4044228)(63.01582245,260.36942283)(63.23582169,260.32943481)
\curveto(63.45582201,260.28942291)(63.60082186,260.21442299)(63.67082169,260.10443481)
\curveto(63.70082176,260.06442314)(63.72582174,259.99442321)(63.74582169,259.89443481)
\curveto(63.75582171,259.8044234)(63.7608217,259.70942349)(63.76082169,259.60943481)
\curveto(63.7608217,259.50942369)(63.75582171,259.41442379)(63.74582169,259.32443481)
\curveto(63.73582173,259.23442397)(63.71582175,259.16942403)(63.68582169,259.12943481)
\curveto(63.65582181,259.07942412)(63.59582187,259.05442415)(63.50582169,259.05443481)
\curveto(63.44582202,259.04442416)(63.38582208,259.04942415)(63.32582169,259.06943481)
\curveto(63.25582221,259.08942411)(63.19082227,259.0994241)(63.13082169,259.09943481)
\curveto(63.08082238,259.0994241)(63.02582244,259.1044241)(62.96582169,259.11443481)
\curveto(62.89582257,259.13442407)(62.83082263,259.14942405)(62.77082169,259.15943481)
\lineto(62.09582169,259.29443481)
\lineto(59.23082169,259.87943481)
\curveto(58.90082656,259.93942326)(58.59082687,259.98942321)(58.30082169,260.02943481)
\curveto(58.00082746,260.07942312)(57.75082771,260.05942314)(57.55082169,259.96943481)
\curveto(57.31082815,259.86942333)(57.13582833,259.67442353)(57.02582169,259.38443481)
\curveto(57.00582846,259.32442388)(56.99582847,259.26442394)(56.99582169,259.20443481)
\curveto(56.98582848,259.14442406)(56.97082849,259.07942412)(56.95082169,259.00943481)
\curveto(56.94082852,258.96942423)(56.94082852,258.9044243)(56.95082169,258.81443481)
\curveto(56.95082851,258.72442448)(56.95582851,258.66442454)(56.96582169,258.63443481)
\curveto(56.97582849,258.58442462)(56.98082848,258.53442467)(56.98082169,258.48443481)
\curveto(56.97082849,258.43442477)(56.97582849,258.37942482)(56.99582169,258.31943481)
\curveto(57.02582844,258.17942502)(57.0658284,258.03942516)(57.11582169,257.89943481)
\curveto(57.33582813,257.32942587)(57.72082774,256.8994263)(58.27082169,256.60943481)
\curveto(58.44082702,256.52942667)(58.63582683,256.46442674)(58.85582169,256.41443481)
\curveto(59.07582639,256.36442684)(59.30082616,256.31942688)(59.53082169,256.27943481)
\lineto(61.49582169,255.87443481)
\lineto(62.95082169,255.58943481)
\curveto(63.07082239,255.56942763)(63.19582227,255.54442766)(63.32582169,255.51443481)
\curveto(63.44582202,255.49442771)(63.54082192,255.44942775)(63.61082169,255.37943481)
\curveto(63.69082177,255.31942788)(63.73582173,255.22442798)(63.74582169,255.09443481)
\curveto(63.75582171,254.97442823)(63.7608217,254.84942835)(63.76082169,254.71943481)
\curveto(63.7608217,254.54942865)(63.74082172,254.42942877)(63.70082169,254.35943481)
\curveto(63.65082181,254.27942892)(63.57082189,254.24442896)(63.46082169,254.25443481)
\curveto(63.34082212,254.26442894)(63.21082225,254.27942892)(63.07082169,254.29943481)
\lineto(61.64582169,254.58443481)
\lineto(59.17082169,255.07943481)
\curveto(58.85082661,255.14942805)(58.55582691,255.199428)(58.28582169,255.22943481)
\curveto(58.00582746,255.26942793)(57.7658277,255.24942795)(57.56582169,255.16943481)
\curveto(57.38582808,255.08942811)(57.25582821,254.98942821)(57.17582169,254.86943481)
\curveto(57.08582838,254.74942845)(57.01582845,254.57442863)(56.96582169,254.34443481)
\curveto(56.95582851,254.3044289)(56.95082851,254.25942894)(56.95082169,254.20943481)
\lineto(56.95082169,254.07443481)
\curveto(56.95082851,253.85442935)(56.97582849,253.65442955)(57.02582169,253.47443481)
\curveto(57.07582839,253.3044299)(57.14082832,253.14443006)(57.22082169,252.99443481)
\curveto(57.53082793,252.42443078)(57.99582747,251.99443121)(58.61582169,251.70443481)
\curveto(58.74582672,251.63443157)(58.89582657,251.58443162)(59.06582169,251.55443481)
\curveto(59.22582624,251.53443167)(59.38582608,251.5044317)(59.54582169,251.46443481)
\lineto(61.24082169,251.13443481)
\lineto(62.89082169,250.80443481)
\curveto(63.02082244,250.77443243)(63.15582231,250.74443246)(63.29582169,250.71443481)
\curveto(63.43582203,250.69443251)(63.54582192,250.64443256)(63.62582169,250.56443481)
\curveto(63.69582177,250.5044327)(63.73582173,250.41943278)(63.74582169,250.30943481)
\curveto(63.75582171,250.20943299)(63.7608217,250.1044331)(63.76082169,249.99443481)
\lineto(63.76082169,249.76943481)
\curveto(63.74082172,249.71943348)(63.72582174,249.66443354)(63.71582169,249.60443481)
\curveto(63.70582176,249.55443365)(63.67582179,249.51443369)(63.62582169,249.48443481)
\curveto(63.5658219,249.44443376)(63.49082197,249.43443377)(63.40082169,249.45443481)
\curveto(63.30082216,249.47443373)(63.19582227,249.49443371)(63.08582169,249.51443481)
\lineto(62.11082169,249.70943481)
\lineto(57.82082169,250.56443481)
\lineto(56.71082169,250.78943481)
\curveto(56.61082885,250.80943239)(56.51582895,250.82943237)(56.42582169,250.84943481)
\curveto(56.32582914,250.86943233)(56.24582922,250.9044323)(56.18582169,250.95443481)
\curveto(56.12582934,250.99443221)(56.08582938,251.04943215)(56.06582169,251.11943481)
\curveto(56.03582943,251.199432)(56.02082944,251.32443188)(56.02082169,251.49443481)
\curveto(56.02082944,251.67443153)(56.03582943,251.8044314)(56.06582169,251.88443481)
\curveto(56.10582936,251.95443125)(56.15582931,251.9994312)(56.21582169,252.01943481)
\curveto(56.2658292,252.02943117)(56.32582914,252.02943117)(56.39582169,252.01943481)
\curveto(56.465829,252.00943119)(56.53082893,252.0044312)(56.59082169,252.00443481)
\curveto(56.65082881,252.0044312)(56.70082876,252.01443119)(56.74082169,252.03443481)
\curveto(56.78082868,252.05443115)(56.80082866,252.0994311)(56.80082169,252.16943481)
\curveto(56.78082868,252.18943101)(56.77082869,252.20943099)(56.77082169,252.22943481)
\curveto(56.77082869,252.25943094)(56.7608287,252.28943091)(56.74082169,252.31943481)
\curveto(56.69082877,252.38943081)(56.64082882,252.45443075)(56.59082169,252.51443481)
\lineto(56.44082169,252.72443481)
\curveto(56.28082918,252.97443023)(56.14082932,253.24942995)(56.02082169,253.54943481)
\curveto(55.98082948,253.65942954)(55.95582951,253.76442944)(55.94582169,253.86443481)
\curveto(55.92582954,253.97442923)(55.90082956,254.08442912)(55.87082169,254.19443481)
\lineto(55.87082169,254.38943481)
\curveto(55.8608296,254.45942874)(55.85582961,254.52442868)(55.85582169,254.58443481)
}
}
{
\newrgbcolor{curcolor}{0 0 0}
\pscustom[linestyle=none,fillstyle=solid,fillcolor=curcolor]
{
\newpath
\moveto(56.03582169,264.11716919)
\lineto(56.03582169,264.55216919)
\curveto(56.03582943,264.70216568)(56.07582939,264.80216558)(56.15582169,264.85216919)
\curveto(56.23582923,264.8821655)(56.33582913,264.88716549)(56.45582169,264.86716919)
\lineto(56.81582169,264.80716919)
\lineto(58.24082169,264.52216919)
\lineto(60.50582169,264.07216919)
\curveto(60.72582474,264.02216636)(60.95582451,263.97216641)(61.19582169,263.92216919)
\curveto(61.42582404,263.8821665)(61.62582384,263.86716651)(61.79582169,263.87716919)
\curveto(62.24582322,263.94716643)(62.5608229,264.18716619)(62.74082169,264.59716919)
\curveto(62.83082263,264.79716558)(62.8658226,265.05216533)(62.84582169,265.36216919)
\curveto(62.81582265,265.6821647)(62.7608227,265.94716443)(62.68082169,266.15716919)
\curveto(62.54082292,266.50716387)(62.3658231,266.80216358)(62.15582169,267.04216919)
\curveto(61.93582353,267.2821631)(61.65082381,267.49216289)(61.30082169,267.67216919)
\curveto(61.22082424,267.72216266)(61.14082432,267.75716262)(61.06082169,267.77716919)
\curveto(60.98082448,267.80716257)(60.89582457,267.84216254)(60.80582169,267.88216919)
\curveto(60.75582471,267.90216248)(60.71082475,267.91216247)(60.67082169,267.91216919)
\curveto(60.63082483,267.91216247)(60.58582488,267.92716245)(60.53582169,267.95716919)
\lineto(60.22082169,268.01716919)
\curveto(60.14082532,268.05716232)(60.05082541,268.0821623)(59.95082169,268.09216919)
\curveto(59.84082562,268.10216228)(59.74082572,268.11716226)(59.65082169,268.13716919)
\lineto(58.48082169,268.37716919)
\lineto(56.89082169,268.69216919)
\curveto(56.77082869,268.71216167)(56.64582882,268.73216165)(56.51582169,268.75216919)
\curveto(56.37582909,268.7821616)(56.2658292,268.82716155)(56.18582169,268.88716919)
\curveto(56.13582933,268.93716144)(56.10582936,268.99216139)(56.09582169,269.05216919)
\curveto(56.07582939,269.11216127)(56.05582941,269.1821612)(56.03582169,269.26216919)
\lineto(56.03582169,269.48716919)
\curveto(56.03582943,269.60716077)(56.04082942,269.71216067)(56.05082169,269.80216919)
\curveto(56.0608294,269.90216048)(56.10582936,269.96716041)(56.18582169,269.99716919)
\curveto(56.23582923,270.03716034)(56.31082915,270.04716033)(56.41082169,270.02716919)
\curveto(56.50082896,270.00716037)(56.59582887,269.98716039)(56.69582169,269.96716919)
\lineto(57.71582169,269.77216919)
\lineto(61.75082169,268.96216919)
\lineto(63.10082169,268.69216919)
\curveto(63.22082224,268.67216171)(63.33582213,268.64716173)(63.44582169,268.61716919)
\curveto(63.54582192,268.58716179)(63.62082184,268.53216185)(63.67082169,268.45216919)
\curveto(63.70082176,268.41216197)(63.72582174,268.34716203)(63.74582169,268.25716919)
\curveto(63.75582171,268.1771622)(63.7658217,268.08716229)(63.77582169,267.98716919)
\curveto(63.77582169,267.89716248)(63.77082169,267.80716257)(63.76082169,267.71716919)
\curveto(63.75082171,267.63716274)(63.73082173,267.5821628)(63.70082169,267.55216919)
\curveto(63.6608218,267.51216287)(63.59582187,267.4821629)(63.50582169,267.46216919)
\curveto(63.465822,267.45216293)(63.41082205,267.45216293)(63.34082169,267.46216919)
\curveto(63.27082219,267.47216291)(63.20582226,267.4771629)(63.14582169,267.47716919)
\curveto(63.07582239,267.48716289)(63.02082244,267.4771629)(62.98082169,267.44716919)
\curveto(62.94082252,267.42716295)(62.92582254,267.38716299)(62.93582169,267.32716919)
\curveto(62.95582251,267.24716313)(63.01582245,267.15716322)(63.11582169,267.05716919)
\curveto(63.20582226,266.95716342)(63.27582219,266.86716351)(63.32582169,266.78716919)
\curveto(63.48582198,266.53716384)(63.62582184,266.25716412)(63.74582169,265.94716919)
\curveto(63.79582167,265.82716455)(63.82582164,265.70716467)(63.83582169,265.58716919)
\curveto(63.85582161,265.4771649)(63.88082158,265.35716502)(63.91082169,265.22716919)
\curveto(63.92082154,265.1771652)(63.92082154,265.12216526)(63.91082169,265.06216919)
\curveto(63.90082156,265.01216537)(63.90582156,264.96216542)(63.92582169,264.91216919)
\curveto(63.94582152,264.81216557)(63.94582152,264.72216566)(63.92582169,264.64216919)
\lineto(63.92582169,264.49216919)
\curveto(63.90582156,264.44216594)(63.89582157,264.382166)(63.89582169,264.31216919)
\curveto(63.89582157,264.25216613)(63.89082157,264.20216618)(63.88082169,264.16216919)
\curveto(63.8608216,264.12216626)(63.85082161,264.0821663)(63.85082169,264.04216919)
\curveto(63.8608216,264.01216637)(63.85582161,263.97216641)(63.83582169,263.92216919)
\curveto(63.81582165,263.85216653)(63.79582167,263.7771666)(63.77582169,263.69716919)
\curveto(63.75582171,263.62716675)(63.72582174,263.55716682)(63.68582169,263.48716919)
\curveto(63.57582189,263.24716713)(63.43082203,263.05716732)(63.25082169,262.91716919)
\curveto(63.0608224,262.78716759)(62.83582263,262.69216769)(62.57582169,262.63216919)
\curveto(62.48582298,262.61216777)(62.39582307,262.60216778)(62.30582169,262.60216919)
\lineto(62.00582169,262.60216919)
\curveto(61.94582352,262.59216779)(61.89082357,262.59216779)(61.84082169,262.60216919)
\curveto(61.78082368,262.62216776)(61.71582375,262.62716775)(61.64582169,262.61716919)
\lineto(61.57082169,262.61716919)
\curveto(61.53082393,262.62716775)(61.49582397,262.63216775)(61.46582169,262.63216919)
\lineto(61.31582169,262.66216919)
\curveto(61.27582419,262.66216772)(61.23082423,262.66716771)(61.18082169,262.67716919)
\curveto(61.12082434,262.69716768)(61.0658244,262.71216767)(61.01582169,262.72216919)
\lineto(60.41582169,262.84216919)
\lineto(57.65582169,263.39716919)
\lineto(56.69582169,263.57716919)
\lineto(56.42582169,263.63716919)
\curveto(56.33582913,263.65716672)(56.2608292,263.69216669)(56.20082169,263.74216919)
\curveto(56.13082933,263.79216659)(56.08082938,263.8771665)(56.05082169,263.99716919)
\curveto(56.04082942,264.01716636)(56.04082942,264.03716634)(56.05082169,264.05716919)
\curveto(56.05082941,264.0771663)(56.04582942,264.09716628)(56.03582169,264.11716919)
}
}
{
\newrgbcolor{curcolor}{0 0 0}
\pscustom[linestyle=none,fillstyle=solid,fillcolor=curcolor]
{
\newpath
\moveto(53.08082169,273.17177856)
\curveto(53.08083238,273.3017748)(53.08083238,273.43677467)(53.08082169,273.57677856)
\curveto(53.08083238,273.72677438)(53.11583235,273.82677428)(53.18582169,273.87677856)
\curveto(53.25583221,273.91677419)(53.35083211,273.92677418)(53.47082169,273.90677856)
\curveto(53.58083188,273.88677422)(53.69583177,273.86677424)(53.81582169,273.84677856)
\lineto(55.15082169,273.57677856)
\lineto(61.22582169,272.36177856)
\lineto(62.90582169,272.03177856)
\curveto(63.02582244,272.0017761)(63.15582231,271.97177613)(63.29582169,271.94177856)
\curveto(63.43582203,271.92177618)(63.54582192,271.87677623)(63.62582169,271.80677856)
\curveto(63.67582179,271.76677634)(63.70582176,271.71677639)(63.71582169,271.65677856)
\curveto(63.72582174,271.6067765)(63.74082172,271.53677657)(63.76082169,271.44677856)
\lineto(63.76082169,271.23677856)
\lineto(63.76082169,270.92177856)
\curveto(63.75082171,270.82177728)(63.71582175,270.75677735)(63.65582169,270.72677856)
\curveto(63.57582189,270.68677742)(63.47582199,270.67677743)(63.35582169,270.69677856)
\curveto(63.23582223,270.71677739)(63.11082235,270.74177736)(62.98082169,270.77177856)
\lineto(61.60082169,271.04177856)
\lineto(55.36082169,272.28677856)
\lineto(53.89082169,272.58677856)
\curveto(53.78083168,272.6067755)(53.6658318,272.62677548)(53.54582169,272.64677856)
\curveto(53.41583205,272.66677544)(53.31583215,272.7067754)(53.24582169,272.76677856)
\curveto(53.18583228,272.82677528)(53.13583233,272.91177519)(53.09582169,273.02177856)
\curveto(53.08583238,273.05177505)(53.08583238,273.07677503)(53.09582169,273.09677856)
\curveto(53.09583237,273.11677499)(53.09083237,273.14177496)(53.08082169,273.17177856)
}
}
{
\newrgbcolor{curcolor}{0 0 0}
\pscustom[linestyle=none,fillstyle=solid,fillcolor=curcolor]
{
\newpath
\moveto(63.20582169,280.80162231)
\curveto(63.3658221,280.7916144)(63.50082196,280.74661445)(63.61082169,280.66662231)
\curveto(63.71082175,280.58661461)(63.78582168,280.4916147)(63.83582169,280.38162231)
\curveto(63.85582161,280.33161486)(63.8658216,280.27661492)(63.86582169,280.21662231)
\curveto(63.8658216,280.16661503)(63.87582159,280.10661509)(63.89582169,280.03662231)
\curveto(63.94582152,279.80661539)(63.93082153,279.5916156)(63.85082169,279.39162231)
\curveto(63.78082168,279.191616)(63.69082177,279.06661613)(63.58082169,279.01662231)
\curveto(63.51082195,278.97661622)(63.43082203,278.94661625)(63.34082169,278.92662231)
\curveto(63.24082222,278.90661629)(63.1608223,278.87161632)(63.10082169,278.82162231)
\lineto(63.04082169,278.76162231)
\curveto(63.02082244,278.74161645)(63.01582245,278.71161648)(63.02582169,278.67162231)
\curveto(63.05582241,278.55161664)(63.11082235,278.43661676)(63.19082169,278.32662231)
\curveto(63.27082219,278.21661698)(63.34082212,278.11161708)(63.40082169,278.01162231)
\curveto(63.48082198,277.86161733)(63.55582191,277.70661749)(63.62582169,277.54662231)
\curveto(63.68582178,277.38661781)(63.74082172,277.21661798)(63.79082169,277.03662231)
\curveto(63.82082164,276.92661827)(63.84082162,276.81161838)(63.85082169,276.69162231)
\curveto(63.8608216,276.58161861)(63.87582159,276.46661873)(63.89582169,276.34662231)
\curveto(63.90582156,276.2966189)(63.91082155,276.25161894)(63.91082169,276.21162231)
\lineto(63.91082169,276.10662231)
\curveto(63.93082153,275.9966192)(63.93082153,275.8916193)(63.91082169,275.79162231)
\lineto(63.91082169,275.65662231)
\curveto(63.90082156,275.60661959)(63.89582157,275.55661964)(63.89582169,275.50662231)
\curveto(63.89582157,275.45661974)(63.88582158,275.41661978)(63.86582169,275.38662231)
\curveto(63.85582161,275.34661985)(63.85082161,275.31161988)(63.85082169,275.28162231)
\curveto(63.8608216,275.26161993)(63.8608216,275.23661996)(63.85082169,275.20662231)
\lineto(63.79082169,274.96662231)
\curveto(63.78082168,274.8966203)(63.7608217,274.83162036)(63.73082169,274.77162231)
\curveto(63.60082186,274.4916207)(63.45582201,274.27662092)(63.29582169,274.12662231)
\curveto(63.12582234,273.97662122)(62.89082257,273.87162132)(62.59082169,273.81162231)
\curveto(62.37082309,273.76162143)(62.10582336,273.76662143)(61.79582169,273.82662231)
\lineto(61.48082169,273.90162231)
\curveto(61.43082403,273.92162127)(61.38082408,273.93662126)(61.33082169,273.94662231)
\lineto(61.15082169,274.00662231)
\lineto(60.82082169,274.18662231)
\curveto(60.71082475,274.25662094)(60.61082485,274.32662087)(60.52082169,274.39662231)
\curveto(60.23082523,274.63662056)(60.01582545,274.92662027)(59.87582169,275.26662231)
\curveto(59.73582573,275.60661959)(59.61082585,275.97161922)(59.50082169,276.36162231)
\curveto(59.460826,276.51161868)(59.43082603,276.66161853)(59.41082169,276.81162231)
\curveto(59.39082607,276.97161822)(59.3658261,277.12661807)(59.33582169,277.27662231)
\curveto(59.31582615,277.35661784)(59.30582616,277.42661777)(59.30582169,277.48662231)
\curveto(59.30582616,277.55661764)(59.29582617,277.63161756)(59.27582169,277.71162231)
\curveto(59.25582621,277.78161741)(59.24582622,277.85161734)(59.24582169,277.92162231)
\curveto(59.23582623,278.00161719)(59.22082624,278.08161711)(59.20082169,278.16162231)
\curveto(59.14082632,278.42161677)(59.09082637,278.66661653)(59.05082169,278.89662231)
\curveto(59.00082646,279.12661607)(58.88582658,279.32661587)(58.70582169,279.49662231)
\curveto(58.62582684,279.56661563)(58.52582694,279.63161556)(58.40582169,279.69162231)
\curveto(58.27582719,279.76161543)(58.13582733,279.7916154)(57.98582169,279.78162231)
\curveto(57.74582772,279.77161542)(57.55582791,279.72161547)(57.41582169,279.63162231)
\curveto(57.27582819,279.55161564)(57.1658283,279.41161578)(57.08582169,279.21162231)
\curveto(57.03582843,279.10161609)(57.00082846,278.96661623)(56.98082169,278.80662231)
\curveto(56.9608285,278.64661655)(56.95082851,278.47661672)(56.95082169,278.29662231)
\curveto(56.95082851,278.11661708)(56.9608285,277.93661726)(56.98082169,277.75662231)
\curveto(57.00082846,277.58661761)(57.03082843,277.43661776)(57.07082169,277.30662231)
\curveto(57.13082833,277.12661807)(57.21582825,276.94661825)(57.32582169,276.76662231)
\curveto(57.38582808,276.67661852)(57.465828,276.58661861)(57.56582169,276.49662231)
\curveto(57.65582781,276.41661878)(57.75582771,276.34161885)(57.86582169,276.27162231)
\curveto(57.94582752,276.22161897)(58.03082743,276.17661902)(58.12082169,276.13662231)
\curveto(58.21082725,276.0966191)(58.28082718,276.03661916)(58.33082169,275.95662231)
\curveto(58.3608271,275.90661929)(58.38582708,275.83161936)(58.40582169,275.73162231)
\curveto(58.41582705,275.63161956)(58.42082704,275.53161966)(58.42082169,275.43162231)
\curveto(58.42082704,275.33161986)(58.41582705,275.23661996)(58.40582169,275.14662231)
\curveto(58.38582708,275.05662014)(58.3608271,274.9966202)(58.33082169,274.96662231)
\curveto(58.30082716,274.92662027)(58.25082721,274.90162029)(58.18082169,274.89162231)
\curveto(58.11082735,274.8916203)(58.03582743,274.91162028)(57.95582169,274.95162231)
\curveto(57.82582764,275.00162019)(57.70582776,275.05662014)(57.59582169,275.11662231)
\curveto(57.47582799,275.17662002)(57.3608281,275.24161995)(57.25082169,275.31162231)
\curveto(56.90082856,275.57161962)(56.63082883,275.86661933)(56.44082169,276.19662231)
\curveto(56.24082922,276.52661867)(56.08082938,276.91661828)(55.96082169,277.36662231)
\curveto(55.94082952,277.47661772)(55.92582954,277.58161761)(55.91582169,277.68162231)
\curveto(55.90582956,277.7916174)(55.89082957,277.90161729)(55.87082169,278.01162231)
\curveto(55.8608296,278.06161713)(55.8608296,278.12661707)(55.87082169,278.20662231)
\curveto(55.87082959,278.2966169)(55.8608296,278.35661684)(55.84082169,278.38662231)
\curveto(55.83082963,279.08661611)(55.91082955,279.67661552)(56.08082169,280.15662231)
\curveto(56.25082921,280.64661455)(56.57582889,280.95161424)(57.05582169,281.07162231)
\curveto(57.25582821,281.12161407)(57.49082797,281.12661407)(57.76082169,281.08662231)
\curveto(58.02082744,281.04661415)(58.29582717,280.9966142)(58.58582169,280.93662231)
\lineto(61.90082169,280.27662231)
\curveto(62.04082342,280.24661495)(62.17582329,280.22161497)(62.30582169,280.20162231)
\curveto(62.43582303,280.191615)(62.54082292,280.20161499)(62.62082169,280.23162231)
\curveto(62.69082277,280.27161492)(62.74082272,280.32661487)(62.77082169,280.39662231)
\curveto(62.81082265,280.48661471)(62.84082262,280.56661463)(62.86082169,280.63662231)
\curveto(62.87082259,280.71661448)(62.91582255,280.76661443)(62.99582169,280.78662231)
\curveto(63.02582244,280.80661439)(63.05582241,280.81161438)(63.08582169,280.80162231)
\lineto(63.20582169,280.80162231)
\moveto(61.54082169,278.98662231)
\curveto(61.40082406,279.07661612)(61.24082422,279.14161605)(61.06082169,279.18162231)
\curveto(60.87082459,279.22161597)(60.67582479,279.26161593)(60.47582169,279.30162231)
\curveto(60.3658251,279.32161587)(60.2658252,279.33661586)(60.17582169,279.34662231)
\curveto(60.08582538,279.35661584)(60.01582545,279.33161586)(59.96582169,279.27162231)
\curveto(59.94582552,279.24161595)(59.93582553,279.17161602)(59.93582169,279.06162231)
\curveto(59.95582551,279.04161615)(59.9658255,279.00661619)(59.96582169,278.95662231)
\curveto(59.9658255,278.90661629)(59.97582549,278.85661634)(59.99582169,278.80662231)
\curveto(60.01582545,278.72661647)(60.03582543,278.63161656)(60.05582169,278.52162231)
\lineto(60.11582169,278.22162231)
\curveto(60.11582535,278.191617)(60.12082534,278.15661704)(60.13082169,278.11662231)
\lineto(60.13082169,278.01162231)
\curveto(60.17082529,277.85161734)(60.19582527,277.68161751)(60.20582169,277.50162231)
\curveto(60.20582526,277.33161786)(60.22582524,277.16661803)(60.26582169,277.00662231)
\curveto(60.28582518,276.91661828)(60.30582516,276.83661836)(60.32582169,276.76662231)
\curveto(60.33582513,276.70661849)(60.35082511,276.63161856)(60.37082169,276.54162231)
\curveto(60.42082504,276.37161882)(60.48582498,276.20661899)(60.56582169,276.04662231)
\curveto(60.63582483,275.8966193)(60.72582474,275.76161943)(60.83582169,275.64162231)
\curveto(60.94582452,275.52161967)(61.08082438,275.42161977)(61.24082169,275.34162231)
\curveto(61.39082407,275.26161993)(61.57582389,275.20161999)(61.79582169,275.16162231)
\curveto(61.89582357,275.14162005)(61.99082347,275.14162005)(62.08082169,275.16162231)
\curveto(62.1608233,275.18162001)(62.23582323,275.21161998)(62.30582169,275.25162231)
\curveto(62.41582305,275.30161989)(62.51082295,275.38161981)(62.59082169,275.49162231)
\curveto(62.6608228,275.61161958)(62.72082274,275.74161945)(62.77082169,275.88162231)
\curveto(62.78082268,275.93161926)(62.78582268,275.98161921)(62.78582169,276.03162231)
\curveto(62.78582268,276.08161911)(62.79082267,276.13161906)(62.80082169,276.18162231)
\curveto(62.82082264,276.25161894)(62.83582263,276.33661886)(62.84582169,276.43662231)
\curveto(62.84582262,276.53661866)(62.83582263,276.62661857)(62.81582169,276.70662231)
\curveto(62.79582267,276.76661843)(62.79082267,276.82661837)(62.80082169,276.88662231)
\curveto(62.80082266,276.94661825)(62.79082267,277.00661819)(62.77082169,277.06662231)
\curveto(62.75082271,277.15661804)(62.73582273,277.23661796)(62.72582169,277.30662231)
\curveto(62.71582275,277.38661781)(62.69582277,277.46661773)(62.66582169,277.54662231)
\curveto(62.54582292,277.85661734)(62.40082306,278.13161706)(62.23082169,278.37162231)
\curveto(62.0608234,278.61161658)(61.83082363,278.81661638)(61.54082169,278.98662231)
}
}
{
\newrgbcolor{curcolor}{0 0 0}
\pscustom[linestyle=none,fillstyle=solid,fillcolor=curcolor]
{
\newpath
\moveto(62.95082169,288.97826294)
\lineto(63.34082169,288.88826294)
\curveto(63.460822,288.86825501)(63.5608219,288.82825505)(63.64082169,288.76826294)
\curveto(63.71082175,288.69825518)(63.75082171,288.60325527)(63.76082169,288.48326294)
\lineto(63.76082169,288.13826294)
\curveto(63.7608217,288.0782558)(63.7658217,288.01825586)(63.77582169,287.95826294)
\curveto(63.77582169,287.90825597)(63.7658217,287.86325601)(63.74582169,287.82326294)
\curveto(63.72582174,287.74325613)(63.68582178,287.69325618)(63.62582169,287.67326294)
\curveto(63.57582189,287.64325623)(63.51582195,287.63325624)(63.44582169,287.64326294)
\curveto(63.37582209,287.65325622)(63.30582216,287.64825623)(63.23582169,287.62826294)
\curveto(63.21582225,287.62825625)(63.20082226,287.61825626)(63.19082169,287.59826294)
\lineto(63.13082169,287.56826294)
\curveto(63.12082234,287.46825641)(63.14082232,287.38325649)(63.19082169,287.31326294)
\curveto(63.24082222,287.25325662)(63.29082217,287.18825669)(63.34082169,287.11826294)
\curveto(63.49082197,286.88825699)(63.60582186,286.66325721)(63.68582169,286.44326294)
\curveto(63.7658217,286.25325762)(63.82582164,286.03325784)(63.86582169,285.78326294)
\curveto(63.90582156,285.54325833)(63.92582154,285.29825858)(63.92582169,285.04826294)
\curveto(63.93582153,284.80825907)(63.92082154,284.56825931)(63.88082169,284.32826294)
\curveto(63.85082161,284.09825978)(63.79582167,283.90325997)(63.71582169,283.74326294)
\curveto(63.49582197,283.26326061)(63.20082226,282.89826098)(62.83082169,282.64826294)
\curveto(62.45082301,282.40826147)(61.98082348,282.25326162)(61.42082169,282.18326294)
\curveto(61.33082413,282.16326171)(61.24082422,282.15326172)(61.15082169,282.15326294)
\curveto(61.05082441,282.16326171)(60.95082451,282.16326171)(60.85082169,282.15326294)
\curveto(60.80082466,282.15326172)(60.75082471,282.15826172)(60.70082169,282.16826294)
\curveto(60.65082481,282.1782617)(60.60082486,282.18326169)(60.55082169,282.18326294)
\curveto(60.50082496,282.1732617)(60.45082501,282.1732617)(60.40082169,282.18326294)
\curveto(60.34082512,282.20326167)(60.28582518,282.21326166)(60.23582169,282.21326294)
\lineto(60.08582169,282.24326294)
\curveto(60.03582543,282.23326164)(59.97082549,282.23326164)(59.89082169,282.24326294)
\curveto(59.81082565,282.26326161)(59.74582572,282.28826159)(59.69582169,282.31826294)
\lineto(59.53082169,282.36326294)
\curveto(59.460826,282.39326148)(59.39082607,282.41326146)(59.32082169,282.42326294)
\curveto(59.24082622,282.43326144)(59.1658263,282.45326142)(59.09582169,282.48326294)
\curveto(59.04582642,282.50326137)(59.00082646,282.51826136)(58.96082169,282.52826294)
\curveto(58.92082654,282.53826134)(58.87582659,282.55326132)(58.82582169,282.57326294)
\curveto(58.72582674,282.62326125)(58.63082683,282.66826121)(58.54082169,282.70826294)
\curveto(58.44082702,282.74826113)(58.34582712,282.79326108)(58.25582169,282.84326294)
\curveto(57.87582759,283.04326083)(57.53582793,283.2732606)(57.23582169,283.53326294)
\curveto(56.92582854,283.80326007)(56.67082879,284.10325977)(56.47082169,284.43326294)
\curveto(56.35082911,284.63325924)(56.25082921,284.83325904)(56.17082169,285.03326294)
\curveto(56.09082937,285.23325864)(56.02082944,285.44825843)(55.96082169,285.67826294)
\lineto(55.93082169,285.88826294)
\curveto(55.92082954,285.95825792)(55.90582956,286.02825785)(55.88582169,286.09826294)
\lineto(55.88582169,286.24826294)
\curveto(55.8658296,286.33825754)(55.85582961,286.45825742)(55.85582169,286.60826294)
\curveto(55.85582961,286.76825711)(55.8658296,286.88325699)(55.88582169,286.95326294)
\curveto(55.89582957,286.99325688)(55.90082956,287.04825683)(55.90082169,287.11826294)
\curveto(55.93082953,287.21825666)(55.95582951,287.32325655)(55.97582169,287.43326294)
\curveto(55.98582948,287.54325633)(56.01582945,287.64325623)(56.06582169,287.73326294)
\curveto(56.12582934,287.873256)(56.19082927,288.00325587)(56.26082169,288.12326294)
\curveto(56.33082913,288.24325563)(56.41082905,288.35325552)(56.50082169,288.45326294)
\curveto(56.55082891,288.50325537)(56.60582886,288.55325532)(56.66582169,288.60326294)
\curveto(56.71582875,288.66325521)(56.73082873,288.74825513)(56.71082169,288.85826294)
\lineto(56.63582169,288.93326294)
\curveto(56.61582885,288.95325492)(56.58582888,288.96825491)(56.54582169,288.97826294)
\curveto(56.45582901,289.02825485)(56.34082912,289.06325481)(56.20082169,289.08326294)
\curveto(56.0608294,289.11325476)(55.93582953,289.13825474)(55.82582169,289.15826294)
\lineto(54.10082169,289.50326294)
\curveto(53.9608315,289.53325434)(53.80583166,289.56325431)(53.63582169,289.59326294)
\curveto(53.45583201,289.63325424)(53.32583214,289.68325419)(53.24582169,289.74326294)
\curveto(53.17583229,289.80325407)(53.13083233,289.873254)(53.11082169,289.95326294)
\curveto(53.11083235,289.9732539)(53.11083235,289.99825388)(53.11082169,290.02826294)
\curveto(53.10083236,290.05825382)(53.09583237,290.08325379)(53.09582169,290.10326294)
\curveto(53.08583238,290.25325362)(53.08583238,290.40325347)(53.09582169,290.55326294)
\curveto(53.09583237,290.70325317)(53.13583233,290.80325307)(53.21582169,290.85326294)
\curveto(53.29583217,290.88325299)(53.39583207,290.88325299)(53.51582169,290.85326294)
\curveto(53.63583183,290.83325304)(53.7608317,290.81325306)(53.89082169,290.79326294)
\lineto(62.95082169,288.97826294)
\moveto(60.11582169,288.33326294)
\curveto(60.0658254,288.36325551)(60.00082546,288.38325549)(59.92082169,288.39326294)
\curveto(59.83082563,288.41325546)(59.7608257,288.41825546)(59.71082169,288.40826294)
\lineto(59.48582169,288.45326294)
\curveto(59.39582607,288.45325542)(59.30582616,288.45825542)(59.21582169,288.46826294)
\curveto(59.11582635,288.4782554)(59.02582644,288.4732554)(58.94582169,288.45326294)
\lineto(58.72082169,288.45326294)
\curveto(58.65082681,288.45325542)(58.58082688,288.44325543)(58.51082169,288.42326294)
\curveto(58.21082725,288.36325551)(57.94582752,288.25825562)(57.71582169,288.10826294)
\curveto(57.48582798,287.96825591)(57.30582816,287.76825611)(57.17582169,287.50826294)
\curveto(57.12582834,287.41825646)(57.09082837,287.32325655)(57.07082169,287.22326294)
\curveto(57.04082842,287.12325675)(57.01582845,287.01325686)(56.99582169,286.89326294)
\curveto(56.97582849,286.82325705)(56.9658285,286.73825714)(56.96582169,286.63826294)
\lineto(56.96582169,286.36826294)
\lineto(56.99582169,286.21826294)
\lineto(56.99582169,286.08326294)
\curveto(57.01582845,286.00325787)(57.03582843,285.91825796)(57.05582169,285.82826294)
\curveto(57.07582839,285.73825814)(57.10082836,285.65325822)(57.13082169,285.57326294)
\curveto(57.27082819,285.22325865)(57.47582799,284.92325895)(57.74582169,284.67326294)
\curveto(58.00582746,284.42325945)(58.31082715,284.20325967)(58.66082169,284.01326294)
\curveto(58.77082669,283.95325992)(58.88582658,283.90325997)(59.00582169,283.86326294)
\lineto(59.33582169,283.74326294)
\lineto(59.45582169,283.71326294)
\curveto(59.48582598,283.70326017)(59.52082594,283.69326018)(59.56082169,283.68326294)
\curveto(59.61082585,283.65326022)(59.6658258,283.63326024)(59.72582169,283.62326294)
\curveto(59.78582568,283.62326025)(59.84082562,283.61826026)(59.89082169,283.60826294)
\curveto(60.00082546,283.58826029)(60.11082535,283.56326031)(60.22082169,283.53326294)
\curveto(60.32082514,283.51326036)(60.41582505,283.50826037)(60.50582169,283.51826294)
\curveto(60.53582493,283.51826036)(60.58582488,283.51326036)(60.65582169,283.50326294)
\lineto(60.86582169,283.50326294)
\curveto(60.93582453,283.50326037)(61.00582446,283.50826037)(61.07582169,283.51826294)
\curveto(61.42582404,283.55826032)(61.72582374,283.64826023)(61.97582169,283.78826294)
\curveto(62.22582324,283.92825995)(62.43082303,284.12825975)(62.59082169,284.38826294)
\curveto(62.64082282,284.46825941)(62.68082278,284.54825933)(62.71082169,284.62826294)
\curveto(62.74082272,284.71825916)(62.77082269,284.81325906)(62.80082169,284.91326294)
\curveto(62.82082264,284.96325891)(62.82582264,285.01325886)(62.81582169,285.06326294)
\curveto(62.80582266,285.12325875)(62.81082265,285.1782587)(62.83082169,285.22826294)
\curveto(62.84082262,285.25825862)(62.84582262,285.29325858)(62.84582169,285.33326294)
\lineto(62.84582169,285.46826294)
\lineto(62.84582169,285.60326294)
\curveto(62.83582263,285.64325823)(62.83082263,285.69825818)(62.83082169,285.76826294)
\curveto(62.81082265,285.84825803)(62.79582267,285.92825795)(62.78582169,286.00826294)
\curveto(62.7658227,286.09825778)(62.74082272,286.1782577)(62.71082169,286.24826294)
\curveto(62.57082289,286.60825727)(62.39582307,286.91325696)(62.18582169,287.16326294)
\curveto(61.9658235,287.41325646)(61.69082377,287.63825624)(61.36082169,287.83826294)
\curveto(61.25082421,287.90825597)(61.14082432,287.96325591)(61.03082169,288.00326294)
\lineto(60.70082169,288.15326294)
\curveto(60.6608248,288.18325569)(60.62582484,288.19825568)(60.59582169,288.19826294)
\curveto(60.55582491,288.20825567)(60.51582495,288.22325565)(60.47582169,288.24326294)
\curveto(60.41582505,288.26325561)(60.35582511,288.2782556)(60.29582169,288.28826294)
\curveto(60.23582523,288.29825558)(60.17582529,288.31325556)(60.11582169,288.33326294)
}
}
{
\newrgbcolor{curcolor}{0 0 0}
\pscustom[linestyle=none,fillstyle=solid,fillcolor=curcolor]
{
\newpath
\moveto(59.56082169,298.72451294)
\curveto(59.62082584,298.73450405)(59.71582575,298.72450406)(59.84582169,298.69451294)
\curveto(59.9658255,298.67450411)(60.05082541,298.65450413)(60.10082169,298.63451294)
\lineto(60.25082169,298.60451294)
\curveto(60.33082513,298.57450421)(60.40582506,298.54950423)(60.47582169,298.52951294)
\curveto(60.53582493,298.51950426)(60.60582486,298.49950428)(60.68582169,298.46951294)
\curveto(60.74582472,298.43950434)(60.80582466,298.41450437)(60.86582169,298.39451294)
\curveto(60.92582454,298.3845044)(60.98582448,298.35950442)(61.04582169,298.31951294)
\lineto(61.43582169,298.13951294)
\curveto(61.5658239,298.08950469)(61.68582378,298.02450476)(61.79582169,297.94451294)
\curveto(62.27582319,297.64450514)(62.68082278,297.2845055)(63.01082169,296.86451294)
\curveto(63.33082213,296.45450633)(63.57582189,295.97450681)(63.74582169,295.42451294)
\curveto(63.78582168,295.31450747)(63.81582165,295.19450759)(63.83582169,295.06451294)
\curveto(63.85582161,294.93450785)(63.87582159,294.79950798)(63.89582169,294.65951294)
\curveto(63.90582156,294.59950818)(63.91082155,294.53450825)(63.91082169,294.46451294)
\curveto(63.92082154,294.40450838)(63.92582154,294.34450844)(63.92582169,294.28451294)
\curveto(63.93582153,294.24450854)(63.94082152,294.1845086)(63.94082169,294.10451294)
\curveto(63.94082152,294.03450875)(63.93582153,293.9845088)(63.92582169,293.95451294)
\curveto(63.91582155,293.91450887)(63.91082155,293.87450891)(63.91082169,293.83451294)
\curveto(63.92082154,293.79450899)(63.92082154,293.75950902)(63.91082169,293.72951294)
\lineto(63.91082169,293.63951294)
\lineto(63.86582169,293.29451294)
\lineto(63.74582169,292.90451294)
\curveto(63.70582176,292.78451)(63.6608218,292.66951011)(63.61082169,292.55951294)
\curveto(63.41082205,292.14951063)(63.15082231,291.82951095)(62.83082169,291.59951294)
\curveto(62.51082295,291.3795114)(62.12082334,291.21951156)(61.66082169,291.11951294)
\curveto(61.5608239,291.08951169)(61.460824,291.06951171)(61.36082169,291.05951294)
\lineto(61.04582169,291.05951294)
\curveto(61.00582446,291.04951173)(60.97582449,291.04951173)(60.95582169,291.05951294)
\curveto(60.92582454,291.06951171)(60.89082457,291.07451171)(60.85082169,291.07451294)
\curveto(60.77082469,291.07451171)(60.69082477,291.0795117)(60.61082169,291.08951294)
\curveto(60.52082494,291.09951168)(60.43582503,291.10451168)(60.35582169,291.10451294)
\curveto(60.30582516,291.11451167)(60.2658252,291.11951166)(60.23582169,291.11951294)
\curveto(60.19582527,291.12951165)(60.15082531,291.13451165)(60.10082169,291.13451294)
\curveto(60.05082541,291.13451165)(59.9658255,291.14451164)(59.84582169,291.16451294)
\curveto(59.71582575,291.19451159)(59.62082584,291.22451156)(59.56082169,291.25451294)
\curveto(59.49082597,291.29451149)(59.42082604,291.31451147)(59.35082169,291.31451294)
\curveto(59.28082618,291.31451147)(59.21082625,291.33451145)(59.14082169,291.37451294)
\curveto(59.09082637,291.39451139)(59.05082641,291.40951137)(59.02082169,291.41951294)
\curveto(58.98082648,291.42951135)(58.93582653,291.44451134)(58.88582169,291.46451294)
\curveto(58.7658267,291.52451126)(58.64582682,291.57451121)(58.52582169,291.61451294)
\curveto(58.40582706,291.66451112)(58.29082717,291.72951105)(58.18082169,291.80951294)
\curveto(57.81082765,292.02951075)(57.48082798,292.27451051)(57.19082169,292.54451294)
\curveto(56.89082857,292.82450996)(56.64082882,293.13950964)(56.44082169,293.48951294)
\curveto(56.3608291,293.61950916)(56.29582917,293.75450903)(56.24582169,293.89451294)
\lineto(56.06582169,294.34451294)
\curveto(56.01582945,294.47450831)(55.98582948,294.60950817)(55.97582169,294.74951294)
\curveto(55.95582951,294.88950789)(55.92582954,295.03450775)(55.88582169,295.18451294)
\lineto(55.88582169,295.37951294)
\lineto(55.85582169,295.58951294)
\curveto(55.84582962,296.4795063)(56.03082943,297.1795056)(56.41082169,297.68951294)
\curveto(56.79082867,298.20950457)(57.28582818,298.53450425)(57.89582169,298.66451294)
\curveto(57.99582747,298.69450409)(58.09582737,298.71450407)(58.19582169,298.72451294)
\curveto(58.29582717,298.73450405)(58.40082706,298.74950403)(58.51082169,298.76951294)
\curveto(58.62082684,298.779504)(58.74082672,298.779504)(58.87082169,298.76951294)
\lineto(59.24582169,298.76951294)
\curveto(59.29582617,298.76950401)(59.35082611,298.75950402)(59.41082169,298.73951294)
\curveto(59.460826,298.72950405)(59.51082595,298.72450406)(59.56082169,298.72451294)
\moveto(60.41582169,297.22451294)
\curveto(60.34582512,297.25450553)(60.2658252,297.27450551)(60.17582169,297.28451294)
\curveto(60.08582538,297.30450548)(60.00082546,297.31950546)(59.92082169,297.32951294)
\curveto(59.53082593,297.40950537)(59.20082626,297.44450534)(58.93082169,297.43451294)
\curveto(58.85082661,297.41450537)(58.77082669,297.39950538)(58.69082169,297.38951294)
\curveto(58.61082685,297.38950539)(58.53582693,297.3845054)(58.46582169,297.37451294)
\curveto(57.81582765,297.22450556)(57.3658281,296.86950591)(57.11582169,296.30951294)
\curveto(57.08582838,296.23950654)(57.0658284,296.16450662)(57.05582169,296.08451294)
\curveto(57.03582843,296.01450677)(57.01582845,295.93950684)(56.99582169,295.85951294)
\curveto(56.97582849,295.78950699)(56.9658285,295.70950707)(56.96582169,295.61951294)
\lineto(56.96582169,295.34951294)
\lineto(57.01082169,295.06451294)
\curveto(57.03082843,294.96450782)(57.05582841,294.86950791)(57.08582169,294.77951294)
\curveto(57.10582836,294.68950809)(57.13582833,294.59950818)(57.17582169,294.50951294)
\curveto(57.19582827,294.43950834)(57.22582824,294.36950841)(57.26582169,294.29951294)
\curveto(57.30582816,294.22950855)(57.34582812,294.16450862)(57.38582169,294.10451294)
\curveto(57.55582791,293.83450895)(57.7608277,293.59950918)(58.00082169,293.39951294)
\curveto(58.24082722,293.19950958)(58.52082694,293.01450977)(58.84082169,292.84451294)
\curveto(58.94082652,292.79450999)(59.04582642,292.75451003)(59.15582169,292.72451294)
\curveto(59.25582621,292.69451009)(59.3608261,292.65451013)(59.47082169,292.60451294)
\curveto(59.51082595,292.59451019)(59.57582589,292.5795102)(59.66582169,292.55951294)
\curveto(59.69582577,292.53951024)(59.73082573,292.52951025)(59.77082169,292.52951294)
\curveto(59.81082565,292.52951025)(59.85582561,292.52451026)(59.90582169,292.51451294)
\lineto(60.20582169,292.45451294)
\curveto(60.30582516,292.43451035)(60.39582507,292.42451036)(60.47582169,292.42451294)
\lineto(60.65582169,292.42451294)
\curveto(60.75582471,292.42451036)(60.85582461,292.41951036)(60.95582169,292.40951294)
\curveto(61.04582442,292.40951037)(61.13082433,292.41951036)(61.21082169,292.43951294)
\curveto(61.45082401,292.48951029)(61.67582379,292.55951022)(61.88582169,292.64951294)
\curveto(62.09582337,292.74951003)(62.27082319,292.8845099)(62.41082169,293.05451294)
\curveto(62.44082302,293.10450968)(62.465823,293.14450964)(62.48582169,293.17451294)
\curveto(62.50582296,293.21450957)(62.53082293,293.25450953)(62.56082169,293.29451294)
\curveto(62.61082285,293.36450942)(62.65582281,293.44450934)(62.69582169,293.53451294)
\curveto(62.72582274,293.62450916)(62.75582271,293.71950906)(62.78582169,293.81951294)
\curveto(62.80582266,293.86950891)(62.81582265,293.91450887)(62.81582169,293.95451294)
\curveto(62.80582266,294.00450878)(62.80582266,294.05450873)(62.81582169,294.10451294)
\curveto(62.82582264,294.13450865)(62.83582263,294.19450859)(62.84582169,294.28451294)
\curveto(62.85582261,294.37450841)(62.85082261,294.44950833)(62.83082169,294.50951294)
\curveto(62.82082264,294.54950823)(62.82082264,294.58950819)(62.83082169,294.62951294)
\curveto(62.83082263,294.66950811)(62.82082264,294.70950807)(62.80082169,294.74951294)
\curveto(62.78082268,294.82950795)(62.7658227,294.90950787)(62.75582169,294.98951294)
\curveto(62.73582273,295.0795077)(62.71082275,295.16450762)(62.68082169,295.24451294)
\curveto(62.54082292,295.60450718)(62.34582312,295.91450687)(62.09582169,296.17451294)
\curveto(61.84582362,296.43450635)(61.55082391,296.66950611)(61.21082169,296.87951294)
\curveto(61.09082437,296.95950582)(60.9658245,297.01950576)(60.83582169,297.05951294)
\curveto(60.69582477,297.09950568)(60.55582491,297.15450563)(60.41582169,297.22451294)
}
}
{
\newrgbcolor{curcolor}{0 0 0}
\pscustom[linestyle=none,fillstyle=solid,fillcolor=curcolor]
{
\newpath
\moveto(181.45952515,78.65651611)
\curveto(181.52952341,78.65650545)(181.60952333,78.65650545)(181.69952515,78.65651611)
\curveto(181.78952315,78.66650544)(181.87452307,78.66650544)(181.95452515,78.65651611)
\curveto(182.0445229,78.65650545)(182.12452282,78.64650546)(182.19452515,78.62651611)
\curveto(182.26452268,78.6065055)(182.31452263,78.57650553)(182.34452515,78.53651611)
\curveto(182.40452254,78.46650564)(182.43452251,78.36650574)(182.43452515,78.23651611)
\curveto(182.4445225,78.11650599)(182.44952249,77.99150611)(182.44952515,77.86151611)
\lineto(182.44952515,76.40651611)
\lineto(182.44952515,70.61651611)
\lineto(182.44952515,68.86151611)
\lineto(182.44952515,68.44151611)
\curveto(182.44952249,68.3015158)(182.42452252,68.19151591)(182.37452515,68.11151611)
\curveto(182.33452261,68.06151604)(182.28452266,68.03151607)(182.22452515,68.02151611)
\curveto(182.17452277,68.01151609)(182.10952283,67.99651611)(182.02952515,67.97651611)
\lineto(181.74452515,67.97651611)
\curveto(181.60452334,67.97651613)(181.47452347,67.98151612)(181.35452515,67.99151611)
\curveto(181.23452371,68.0015161)(181.14952379,68.05151605)(181.09952515,68.14151611)
\curveto(181.05952388,68.2015159)(181.0395239,68.28151582)(181.03952515,68.38151611)
\lineto(181.03952515,68.71151611)
\lineto(181.03952515,69.91151611)
\lineto(181.03952515,76.18151611)
\lineto(181.03952515,77.80151611)
\curveto(181.0395239,77.91150619)(181.03452391,78.03150607)(181.02452515,78.16151611)
\curveto(181.02452392,78.3015058)(181.04952389,78.41150569)(181.09952515,78.49151611)
\curveto(181.1395238,78.56150554)(181.21952372,78.61150549)(181.33952515,78.64151611)
\curveto(181.35952358,78.65150545)(181.37952356,78.65150545)(181.39952515,78.64151611)
\curveto(181.41952352,78.64150546)(181.4395235,78.64650546)(181.45952515,78.65651611)
}
}
{
\newrgbcolor{curcolor}{0 0 0}
\pscustom[linestyle=none,fillstyle=solid,fillcolor=curcolor]
{
\newpath
\moveto(188.29600952,75.88151611)
\curveto(188.92600429,75.9015082)(189.43100378,75.81650829)(189.81100952,75.62651611)
\curveto(190.19100302,75.43650867)(190.49600272,75.15150895)(190.72600952,74.77151611)
\curveto(190.78600243,74.67150943)(190.83100238,74.56150954)(190.86100952,74.44151611)
\curveto(190.90100231,74.33150977)(190.93600228,74.21650989)(190.96600952,74.09651611)
\curveto(191.0160022,73.9065102)(191.04600217,73.7015104)(191.05600952,73.48151611)
\curveto(191.06600215,73.26151084)(191.07100214,73.03651107)(191.07100952,72.80651611)
\lineto(191.07100952,71.20151611)
\lineto(191.07100952,68.86151611)
\curveto(191.07100214,68.69151541)(191.06600215,68.52151558)(191.05600952,68.35151611)
\curveto(191.05600216,68.18151592)(190.99100222,68.07151603)(190.86100952,68.02151611)
\curveto(190.8110024,68.0015161)(190.75600246,67.99151611)(190.69600952,67.99151611)
\curveto(190.64600257,67.98151612)(190.59100262,67.97651613)(190.53100952,67.97651611)
\curveto(190.40100281,67.97651613)(190.27600294,67.98151612)(190.15600952,67.99151611)
\curveto(190.03600318,67.99151611)(189.95100326,68.03151607)(189.90100952,68.11151611)
\curveto(189.85100336,68.18151592)(189.82600339,68.27151583)(189.82600952,68.38151611)
\lineto(189.82600952,68.71151611)
\lineto(189.82600952,70.00151611)
\lineto(189.82600952,72.44651611)
\curveto(189.82600339,72.71651139)(189.82100339,72.98151112)(189.81100952,73.24151611)
\curveto(189.80100341,73.51151059)(189.75600346,73.74151036)(189.67600952,73.93151611)
\curveto(189.59600362,74.13150997)(189.47600374,74.29150981)(189.31600952,74.41151611)
\curveto(189.15600406,74.54150956)(188.97100424,74.64150946)(188.76100952,74.71151611)
\curveto(188.70100451,74.73150937)(188.63600458,74.74150936)(188.56600952,74.74151611)
\curveto(188.50600471,74.75150935)(188.44600477,74.76650934)(188.38600952,74.78651611)
\curveto(188.33600488,74.79650931)(188.25600496,74.79650931)(188.14600952,74.78651611)
\curveto(188.04600517,74.78650932)(187.97600524,74.78150932)(187.93600952,74.77151611)
\curveto(187.89600532,74.75150935)(187.86100535,74.74150936)(187.83100952,74.74151611)
\curveto(187.80100541,74.75150935)(187.76600545,74.75150935)(187.72600952,74.74151611)
\curveto(187.59600562,74.71150939)(187.47100574,74.67650943)(187.35100952,74.63651611)
\curveto(187.24100597,74.6065095)(187.13600608,74.56150954)(187.03600952,74.50151611)
\curveto(186.99600622,74.48150962)(186.96100625,74.46150964)(186.93100952,74.44151611)
\curveto(186.90100631,74.42150968)(186.86600635,74.4015097)(186.82600952,74.38151611)
\curveto(186.47600674,74.13150997)(186.22100699,73.75651035)(186.06100952,73.25651611)
\curveto(186.03100718,73.17651093)(186.0110072,73.09151101)(186.00100952,73.00151611)
\curveto(185.99100722,72.92151118)(185.97600724,72.84151126)(185.95600952,72.76151611)
\curveto(185.93600728,72.71151139)(185.93100728,72.66151144)(185.94100952,72.61151611)
\curveto(185.95100726,72.57151153)(185.94600727,72.53151157)(185.92600952,72.49151611)
\lineto(185.92600952,72.17651611)
\curveto(185.9160073,72.14651196)(185.9110073,72.11151199)(185.91100952,72.07151611)
\curveto(185.92100729,72.03151207)(185.92600729,71.98651212)(185.92600952,71.93651611)
\lineto(185.92600952,71.48651611)
\lineto(185.92600952,70.04651611)
\lineto(185.92600952,68.72651611)
\lineto(185.92600952,68.38151611)
\curveto(185.92600729,68.27151583)(185.90100731,68.18151592)(185.85100952,68.11151611)
\curveto(185.80100741,68.03151607)(185.7110075,67.99151611)(185.58100952,67.99151611)
\curveto(185.46100775,67.98151612)(185.33600788,67.97651613)(185.20600952,67.97651611)
\curveto(185.12600809,67.97651613)(185.05100816,67.98151612)(184.98100952,67.99151611)
\curveto(184.9110083,68.0015161)(184.85100836,68.02651608)(184.80100952,68.06651611)
\curveto(184.72100849,68.11651599)(184.68100853,68.21151589)(184.68100952,68.35151611)
\lineto(184.68100952,68.75651611)
\lineto(184.68100952,70.52651611)
\lineto(184.68100952,74.15651611)
\lineto(184.68100952,75.07151611)
\lineto(184.68100952,75.34151611)
\curveto(184.68100853,75.43150867)(184.70100851,75.5015086)(184.74100952,75.55151611)
\curveto(184.77100844,75.61150849)(184.82100839,75.65150845)(184.89100952,75.67151611)
\curveto(184.93100828,75.68150842)(184.98600823,75.69150841)(185.05600952,75.70151611)
\curveto(185.13600808,75.71150839)(185.216008,75.71650839)(185.29600952,75.71651611)
\curveto(185.37600784,75.71650839)(185.45100776,75.71150839)(185.52100952,75.70151611)
\curveto(185.60100761,75.69150841)(185.65600756,75.67650843)(185.68600952,75.65651611)
\curveto(185.79600742,75.58650852)(185.84600737,75.49650861)(185.83600952,75.38651611)
\curveto(185.82600739,75.28650882)(185.84100737,75.17150893)(185.88100952,75.04151611)
\curveto(185.90100731,74.98150912)(185.94100727,74.93150917)(186.00100952,74.89151611)
\curveto(186.12100709,74.88150922)(186.216007,74.92650918)(186.28600952,75.02651611)
\curveto(186.36600685,75.12650898)(186.44600677,75.2065089)(186.52600952,75.26651611)
\curveto(186.66600655,75.36650874)(186.80600641,75.45650865)(186.94600952,75.53651611)
\curveto(187.09600612,75.62650848)(187.26600595,75.7015084)(187.45600952,75.76151611)
\curveto(187.53600568,75.79150831)(187.62100559,75.81150829)(187.71100952,75.82151611)
\curveto(187.8110054,75.83150827)(187.90600531,75.84650826)(187.99600952,75.86651611)
\curveto(188.04600517,75.87650823)(188.09600512,75.88150822)(188.14600952,75.88151611)
\lineto(188.29600952,75.88151611)
}
}
{
\newrgbcolor{curcolor}{0 0 0}
\pscustom[linestyle=none,fillstyle=solid,fillcolor=curcolor]
{
\newpath
\moveto(192.7306189,75.73151611)
\lineto(193.2106189,75.73151611)
\curveto(193.38061756,75.73150837)(193.51061743,75.7015084)(193.6006189,75.64151611)
\curveto(193.67061727,75.59150851)(193.71561722,75.52650858)(193.7356189,75.44651611)
\curveto(193.76561717,75.37650873)(193.79561714,75.3015088)(193.8256189,75.22151611)
\curveto(193.88561705,75.08150902)(193.935617,74.94150916)(193.9756189,74.80151611)
\curveto(194.01561692,74.66150944)(194.06061688,74.52150958)(194.1106189,74.38151611)
\curveto(194.31061663,73.84151026)(194.49561644,73.29651081)(194.6656189,72.74651611)
\curveto(194.8356161,72.2065119)(195.02061592,71.66651244)(195.2206189,71.12651611)
\curveto(195.29061565,70.94651316)(195.35061559,70.76151334)(195.4006189,70.57151611)
\curveto(195.45061549,70.39151371)(195.51561542,70.21151389)(195.5956189,70.03151611)
\curveto(195.61561532,69.96151414)(195.6406153,69.88651422)(195.6706189,69.80651611)
\curveto(195.70061524,69.72651438)(195.75061519,69.67651443)(195.8206189,69.65651611)
\curveto(195.90061504,69.63651447)(195.96061498,69.67151443)(196.0006189,69.76151611)
\curveto(196.05061489,69.85151425)(196.08561485,69.92151418)(196.1056189,69.97151611)
\curveto(196.18561475,70.16151394)(196.25061469,70.35151375)(196.3006189,70.54151611)
\curveto(196.36061458,70.74151336)(196.42561451,70.94151316)(196.4956189,71.14151611)
\curveto(196.62561431,71.52151258)(196.75061419,71.89651221)(196.8706189,72.26651611)
\curveto(196.99061395,72.64651146)(197.11561382,73.02651108)(197.2456189,73.40651611)
\curveto(197.29561364,73.57651053)(197.34561359,73.74151036)(197.3956189,73.90151611)
\curveto(197.44561349,74.07151003)(197.50561343,74.23650987)(197.5756189,74.39651611)
\curveto(197.62561331,74.53650957)(197.67061327,74.67650943)(197.7106189,74.81651611)
\curveto(197.75061319,74.95650915)(197.79561314,75.09650901)(197.8456189,75.23651611)
\curveto(197.86561307,75.3065088)(197.89061305,75.37650873)(197.9206189,75.44651611)
\curveto(197.95061299,75.51650859)(197.99061295,75.57650853)(198.0406189,75.62651611)
\curveto(198.12061282,75.67650843)(198.21061273,75.7065084)(198.3106189,75.71651611)
\curveto(198.41061253,75.72650838)(198.53061241,75.73150837)(198.6706189,75.73151611)
\curveto(198.7406122,75.73150837)(198.80561213,75.72650838)(198.8656189,75.71651611)
\curveto(198.92561201,75.71650839)(198.98061196,75.7065084)(199.0306189,75.68651611)
\curveto(199.12061182,75.64650846)(199.16561177,75.58150852)(199.1656189,75.49151611)
\curveto(199.17561176,75.4015087)(199.16061178,75.31150879)(199.1206189,75.22151611)
\curveto(199.06061188,75.05150905)(199.00061194,74.87650923)(198.9406189,74.69651611)
\curveto(198.88061206,74.51650959)(198.81061213,74.34150976)(198.7306189,74.17151611)
\curveto(198.71061223,74.12150998)(198.69561224,74.07151003)(198.6856189,74.02151611)
\curveto(198.67561226,73.98151012)(198.66061228,73.93651017)(198.6406189,73.88651611)
\curveto(198.56061238,73.71651039)(198.49561244,73.54151056)(198.4456189,73.36151611)
\curveto(198.39561254,73.18151092)(198.33061261,73.0015111)(198.2506189,72.82151611)
\curveto(198.20061274,72.69151141)(198.15061279,72.55651155)(198.1006189,72.41651611)
\curveto(198.06061288,72.28651182)(198.01061293,72.15651195)(197.9506189,72.02651611)
\curveto(197.78061316,71.61651249)(197.62561331,71.2015129)(197.4856189,70.78151611)
\curveto(197.35561358,70.36151374)(197.20561373,69.94651416)(197.0356189,69.53651611)
\curveto(196.97561396,69.37651473)(196.92061402,69.21651489)(196.8706189,69.05651611)
\curveto(196.82061412,68.89651521)(196.76061418,68.73651537)(196.6906189,68.57651611)
\curveto(196.6406143,68.46651564)(196.59561434,68.36151574)(196.5556189,68.26151611)
\curveto(196.52561441,68.17151593)(196.45561448,68.101516)(196.3456189,68.05151611)
\curveto(196.28561465,68.02151608)(196.21561472,68.0065161)(196.1356189,68.00651611)
\lineto(195.9106189,68.00651611)
\lineto(195.4456189,68.00651611)
\curveto(195.29561564,68.01651609)(195.18561575,68.06651604)(195.1156189,68.15651611)
\curveto(195.04561589,68.23651587)(194.99561594,68.33151577)(194.9656189,68.44151611)
\curveto(194.935616,68.56151554)(194.89561604,68.67651543)(194.8456189,68.78651611)
\curveto(194.78561615,68.92651518)(194.72561621,69.07151503)(194.6656189,69.22151611)
\curveto(194.61561632,69.38151472)(194.56561637,69.53151457)(194.5156189,69.67151611)
\curveto(194.49561644,69.72151438)(194.48061646,69.76151434)(194.4706189,69.79151611)
\curveto(194.46061648,69.83151427)(194.44561649,69.87651423)(194.4256189,69.92651611)
\curveto(194.22561671,70.4065137)(194.0406169,70.89151321)(193.8706189,71.38151611)
\curveto(193.71061723,71.87151223)(193.53061741,72.35651175)(193.3306189,72.83651611)
\curveto(193.27061767,72.99651111)(193.21061773,73.15151095)(193.1506189,73.30151611)
\curveto(193.10061784,73.46151064)(193.04561789,73.62151048)(192.9856189,73.78151611)
\lineto(192.9256189,73.93151611)
\curveto(192.91561802,73.99151011)(192.90061804,74.04651006)(192.8806189,74.09651611)
\curveto(192.80061814,74.26650984)(192.73061821,74.43650967)(192.6706189,74.60651611)
\curveto(192.62061832,74.77650933)(192.56061838,74.94650916)(192.4906189,75.11651611)
\curveto(192.47061847,75.17650893)(192.44561849,75.25650885)(192.4156189,75.35651611)
\curveto(192.38561855,75.45650865)(192.39061855,75.54150856)(192.4306189,75.61151611)
\curveto(192.48061846,75.66150844)(192.5406184,75.69650841)(192.6106189,75.71651611)
\curveto(192.68061826,75.71650839)(192.72061822,75.72150838)(192.7306189,75.73151611)
}
}
{
\newrgbcolor{curcolor}{0 0 0}
\pscustom[linestyle=none,fillstyle=solid,fillcolor=curcolor]
{
\newpath
\moveto(200.7406189,77.23151611)
\curveto(200.66061778,77.29150681)(200.61561782,77.39650671)(200.6056189,77.54651611)
\lineto(200.6056189,78.01151611)
\lineto(200.6056189,78.26651611)
\curveto(200.60561783,78.35650575)(200.62061782,78.43150567)(200.6506189,78.49151611)
\curveto(200.69061775,78.57150553)(200.77061767,78.63150547)(200.8906189,78.67151611)
\curveto(200.91061753,78.68150542)(200.93061751,78.68150542)(200.9506189,78.67151611)
\curveto(200.98061746,78.67150543)(201.00561743,78.67650543)(201.0256189,78.68651611)
\curveto(201.19561724,78.68650542)(201.35561708,78.68150542)(201.5056189,78.67151611)
\curveto(201.65561678,78.66150544)(201.75561668,78.6015055)(201.8056189,78.49151611)
\curveto(201.8356166,78.43150567)(201.85061659,78.35650575)(201.8506189,78.26651611)
\lineto(201.8506189,78.01151611)
\curveto(201.85061659,77.83150627)(201.84561659,77.66150644)(201.8356189,77.50151611)
\curveto(201.8356166,77.34150676)(201.77061667,77.23650687)(201.6406189,77.18651611)
\curveto(201.59061685,77.16650694)(201.5356169,77.15650695)(201.4756189,77.15651611)
\lineto(201.3106189,77.15651611)
\lineto(200.9956189,77.15651611)
\curveto(200.89561754,77.15650695)(200.81061763,77.18150692)(200.7406189,77.23151611)
\moveto(201.8506189,68.72651611)
\lineto(201.8506189,68.41151611)
\curveto(201.86061658,68.31151579)(201.8406166,68.23151587)(201.7906189,68.17151611)
\curveto(201.76061668,68.11151599)(201.71561672,68.07151603)(201.6556189,68.05151611)
\curveto(201.59561684,68.04151606)(201.52561691,68.02651608)(201.4456189,68.00651611)
\lineto(201.2206189,68.00651611)
\curveto(201.09061735,68.0065161)(200.97561746,68.01151609)(200.8756189,68.02151611)
\curveto(200.78561765,68.04151606)(200.71561772,68.09151601)(200.6656189,68.17151611)
\curveto(200.62561781,68.23151587)(200.60561783,68.3065158)(200.6056189,68.39651611)
\lineto(200.6056189,68.68151611)
\lineto(200.6056189,75.02651611)
\lineto(200.6056189,75.34151611)
\curveto(200.60561783,75.45150865)(200.63061781,75.53650857)(200.6806189,75.59651611)
\curveto(200.71061773,75.64650846)(200.75061769,75.67650843)(200.8006189,75.68651611)
\curveto(200.85061759,75.69650841)(200.90561753,75.71150839)(200.9656189,75.73151611)
\curveto(200.98561745,75.73150837)(201.00561743,75.72650838)(201.0256189,75.71651611)
\curveto(201.05561738,75.71650839)(201.08061736,75.72150838)(201.1006189,75.73151611)
\curveto(201.23061721,75.73150837)(201.36061708,75.72650838)(201.4906189,75.71651611)
\curveto(201.63061681,75.71650839)(201.72561671,75.67650843)(201.7756189,75.59651611)
\curveto(201.82561661,75.53650857)(201.85061659,75.45650865)(201.8506189,75.35651611)
\lineto(201.8506189,75.07151611)
\lineto(201.8506189,68.72651611)
}
}
{
\newrgbcolor{curcolor}{0 0 0}
\pscustom[linestyle=none,fillstyle=solid,fillcolor=curcolor]
{
\newpath
\moveto(204.74046265,78.07151611)
\curveto(204.89046064,78.07150603)(205.04046049,78.06650604)(205.19046265,78.05651611)
\curveto(205.34046019,78.05650605)(205.44546008,78.01650609)(205.50546265,77.93651611)
\curveto(205.55545997,77.87650623)(205.58045995,77.79150631)(205.58046265,77.68151611)
\curveto(205.59045994,77.58150652)(205.59545993,77.47650663)(205.59546265,77.36651611)
\lineto(205.59546265,76.49651611)
\curveto(205.59545993,76.41650769)(205.59045994,76.33150777)(205.58046265,76.24151611)
\curveto(205.58045995,76.16150794)(205.59045994,76.09150801)(205.61046265,76.03151611)
\curveto(205.65045988,75.89150821)(205.74045979,75.8015083)(205.88046265,75.76151611)
\curveto(205.9304596,75.75150835)(205.97545955,75.74650836)(206.01546265,75.74651611)
\lineto(206.16546265,75.74651611)
\lineto(206.57046265,75.74651611)
\curveto(206.7304588,75.75650835)(206.84545868,75.74650836)(206.91546265,75.71651611)
\curveto(207.00545852,75.65650845)(207.06545846,75.59650851)(207.09546265,75.53651611)
\curveto(207.11545841,75.49650861)(207.1254584,75.45150865)(207.12546265,75.40151611)
\lineto(207.12546265,75.25151611)
\curveto(207.1254584,75.14150896)(207.12045841,75.03650907)(207.11046265,74.93651611)
\curveto(207.10045843,74.84650926)(207.06545846,74.77650933)(207.00546265,74.72651611)
\curveto(206.94545858,74.67650943)(206.86045867,74.64650946)(206.75046265,74.63651611)
\lineto(206.42046265,74.63651611)
\curveto(206.31045922,74.64650946)(206.20045933,74.65150945)(206.09046265,74.65151611)
\curveto(205.98045955,74.65150945)(205.88545964,74.63650947)(205.80546265,74.60651611)
\curveto(205.73545979,74.57650953)(205.68545984,74.52650958)(205.65546265,74.45651611)
\curveto(205.6254599,74.38650972)(205.60545992,74.3015098)(205.59546265,74.20151611)
\curveto(205.58545994,74.11150999)(205.58045995,74.01151009)(205.58046265,73.90151611)
\curveto(205.59045994,73.8015103)(205.59545993,73.7015104)(205.59546265,73.60151611)
\lineto(205.59546265,70.63151611)
\curveto(205.59545993,70.41151369)(205.59045994,70.17651393)(205.58046265,69.92651611)
\curveto(205.58045995,69.68651442)(205.6254599,69.5015146)(205.71546265,69.37151611)
\curveto(205.76545976,69.29151481)(205.8304597,69.23651487)(205.91046265,69.20651611)
\curveto(205.99045954,69.17651493)(206.08545944,69.15151495)(206.19546265,69.13151611)
\curveto(206.2254593,69.12151498)(206.25545927,69.11651499)(206.28546265,69.11651611)
\curveto(206.3254592,69.12651498)(206.36045917,69.12651498)(206.39046265,69.11651611)
\lineto(206.58546265,69.11651611)
\curveto(206.68545884,69.11651499)(206.77545875,69.106515)(206.85546265,69.08651611)
\curveto(206.94545858,69.07651503)(207.01045852,69.04151506)(207.05046265,68.98151611)
\curveto(207.07045846,68.95151515)(207.08545844,68.89651521)(207.09546265,68.81651611)
\curveto(207.11545841,68.74651536)(207.1254584,68.67151543)(207.12546265,68.59151611)
\curveto(207.13545839,68.51151559)(207.13545839,68.43151567)(207.12546265,68.35151611)
\curveto(207.11545841,68.28151582)(207.09545843,68.22651588)(207.06546265,68.18651611)
\curveto(207.0254585,68.11651599)(206.95045858,68.06651604)(206.84046265,68.03651611)
\curveto(206.76045877,68.01651609)(206.67045886,68.0065161)(206.57046265,68.00651611)
\curveto(206.47045906,68.01651609)(206.38045915,68.02151608)(206.30046265,68.02151611)
\curveto(206.24045929,68.02151608)(206.18045935,68.01651609)(206.12046265,68.00651611)
\curveto(206.06045947,68.0065161)(206.00545952,68.01151609)(205.95546265,68.02151611)
\lineto(205.77546265,68.02151611)
\curveto(205.7254598,68.03151607)(205.67545985,68.03651607)(205.62546265,68.03651611)
\curveto(205.58545994,68.04651606)(205.54045999,68.05151605)(205.49046265,68.05151611)
\curveto(205.29046024,68.101516)(205.11546041,68.15651595)(204.96546265,68.21651611)
\curveto(204.8254607,68.27651583)(204.70546082,68.38151572)(204.60546265,68.53151611)
\curveto(204.46546106,68.73151537)(204.38546114,68.98151512)(204.36546265,69.28151611)
\curveto(204.34546118,69.59151451)(204.33546119,69.92151418)(204.33546265,70.27151611)
\lineto(204.33546265,74.20151611)
\curveto(204.30546122,74.33150977)(204.27546125,74.42650968)(204.24546265,74.48651611)
\curveto(204.2254613,74.54650956)(204.15546137,74.59650951)(204.03546265,74.63651611)
\curveto(203.99546153,74.64650946)(203.95546157,74.64650946)(203.91546265,74.63651611)
\curveto(203.87546165,74.62650948)(203.83546169,74.63150947)(203.79546265,74.65151611)
\lineto(203.55546265,74.65151611)
\curveto(203.4254621,74.65150945)(203.31546221,74.66150944)(203.22546265,74.68151611)
\curveto(203.14546238,74.71150939)(203.09046244,74.77150933)(203.06046265,74.86151611)
\curveto(203.04046249,74.9015092)(203.0254625,74.94650916)(203.01546265,74.99651611)
\lineto(203.01546265,75.14651611)
\curveto(203.01546251,75.28650882)(203.0254625,75.4015087)(203.04546265,75.49151611)
\curveto(203.06546246,75.59150851)(203.1254624,75.66650844)(203.22546265,75.71651611)
\curveto(203.33546219,75.75650835)(203.47546205,75.76650834)(203.64546265,75.74651611)
\curveto(203.8254617,75.72650838)(203.97546155,75.73650837)(204.09546265,75.77651611)
\curveto(204.18546134,75.82650828)(204.25546127,75.89650821)(204.30546265,75.98651611)
\curveto(204.3254612,76.04650806)(204.33546119,76.12150798)(204.33546265,76.21151611)
\lineto(204.33546265,76.46651611)
\lineto(204.33546265,77.39651611)
\lineto(204.33546265,77.63651611)
\curveto(204.33546119,77.72650638)(204.34546118,77.8015063)(204.36546265,77.86151611)
\curveto(204.40546112,77.94150616)(204.48046105,78.0065061)(204.59046265,78.05651611)
\curveto(204.62046091,78.05650605)(204.64546088,78.05650605)(204.66546265,78.05651611)
\curveto(204.69546083,78.06650604)(204.72046081,78.07150603)(204.74046265,78.07151611)
}
}
{
\newrgbcolor{curcolor}{0 0 0}
\pscustom[linestyle=none,fillstyle=solid,fillcolor=curcolor]
{
\newpath
\moveto(215.39725952,68.56151611)
\curveto(215.42725169,68.4015157)(215.41225171,68.26651584)(215.35225952,68.15651611)
\curveto(215.29225183,68.05651605)(215.21225191,67.98151612)(215.11225952,67.93151611)
\curveto(215.06225206,67.91151619)(215.00725211,67.9015162)(214.94725952,67.90151611)
\curveto(214.89725222,67.9015162)(214.84225228,67.89151621)(214.78225952,67.87151611)
\curveto(214.56225256,67.82151628)(214.34225278,67.83651627)(214.12225952,67.91651611)
\curveto(213.91225321,67.98651612)(213.76725335,68.07651603)(213.68725952,68.18651611)
\curveto(213.63725348,68.25651585)(213.59225353,68.33651577)(213.55225952,68.42651611)
\curveto(213.51225361,68.52651558)(213.46225366,68.6065155)(213.40225952,68.66651611)
\curveto(213.38225374,68.68651542)(213.35725376,68.7065154)(213.32725952,68.72651611)
\curveto(213.30725381,68.74651536)(213.27725384,68.75151535)(213.23725952,68.74151611)
\curveto(213.12725399,68.71151539)(213.0222541,68.65651545)(212.92225952,68.57651611)
\curveto(212.83225429,68.49651561)(212.74225438,68.42651568)(212.65225952,68.36651611)
\curveto(212.5222546,68.28651582)(212.38225474,68.21151589)(212.23225952,68.14151611)
\curveto(212.08225504,68.08151602)(211.9222552,68.02651608)(211.75225952,67.97651611)
\curveto(211.65225547,67.94651616)(211.54225558,67.92651618)(211.42225952,67.91651611)
\curveto(211.31225581,67.9065162)(211.20225592,67.89151621)(211.09225952,67.87151611)
\curveto(211.04225608,67.86151624)(210.99725612,67.85651625)(210.95725952,67.85651611)
\lineto(210.85225952,67.85651611)
\curveto(210.74225638,67.83651627)(210.63725648,67.83651627)(210.53725952,67.85651611)
\lineto(210.40225952,67.85651611)
\curveto(210.35225677,67.86651624)(210.30225682,67.87151623)(210.25225952,67.87151611)
\curveto(210.20225692,67.87151623)(210.15725696,67.88151622)(210.11725952,67.90151611)
\curveto(210.07725704,67.91151619)(210.04225708,67.91651619)(210.01225952,67.91651611)
\curveto(209.99225713,67.9065162)(209.96725715,67.9065162)(209.93725952,67.91651611)
\lineto(209.69725952,67.97651611)
\curveto(209.6172575,67.98651612)(209.54225758,68.0065161)(209.47225952,68.03651611)
\curveto(209.17225795,68.16651594)(208.92725819,68.31151579)(208.73725952,68.47151611)
\curveto(208.55725856,68.64151546)(208.40725871,68.87651523)(208.28725952,69.17651611)
\curveto(208.19725892,69.39651471)(208.15225897,69.66151444)(208.15225952,69.97151611)
\lineto(208.15225952,70.28651611)
\curveto(208.16225896,70.33651377)(208.16725895,70.38651372)(208.16725952,70.43651611)
\lineto(208.19725952,70.61651611)
\lineto(208.31725952,70.94651611)
\curveto(208.35725876,71.05651305)(208.40725871,71.15651295)(208.46725952,71.24651611)
\curveto(208.64725847,71.53651257)(208.89225823,71.75151235)(209.20225952,71.89151611)
\curveto(209.51225761,72.03151207)(209.85225727,72.15651195)(210.22225952,72.26651611)
\curveto(210.36225676,72.3065118)(210.50725661,72.33651177)(210.65725952,72.35651611)
\curveto(210.80725631,72.37651173)(210.95725616,72.4015117)(211.10725952,72.43151611)
\curveto(211.17725594,72.45151165)(211.24225588,72.46151164)(211.30225952,72.46151611)
\curveto(211.37225575,72.46151164)(211.44725567,72.47151163)(211.52725952,72.49151611)
\curveto(211.59725552,72.51151159)(211.66725545,72.52151158)(211.73725952,72.52151611)
\curveto(211.80725531,72.53151157)(211.88225524,72.54651156)(211.96225952,72.56651611)
\curveto(212.21225491,72.62651148)(212.44725467,72.67651143)(212.66725952,72.71651611)
\curveto(212.88725423,72.76651134)(213.06225406,72.88151122)(213.19225952,73.06151611)
\curveto(213.25225387,73.14151096)(213.30225382,73.24151086)(213.34225952,73.36151611)
\curveto(213.38225374,73.49151061)(213.38225374,73.63151047)(213.34225952,73.78151611)
\curveto(213.28225384,74.02151008)(213.19225393,74.21150989)(213.07225952,74.35151611)
\curveto(212.96225416,74.49150961)(212.80225432,74.6015095)(212.59225952,74.68151611)
\curveto(212.47225465,74.73150937)(212.32725479,74.76650934)(212.15725952,74.78651611)
\curveto(211.99725512,74.8065093)(211.82725529,74.81650929)(211.64725952,74.81651611)
\curveto(211.46725565,74.81650929)(211.29225583,74.8065093)(211.12225952,74.78651611)
\curveto(210.95225617,74.76650934)(210.80725631,74.73650937)(210.68725952,74.69651611)
\curveto(210.5172566,74.63650947)(210.35225677,74.55150955)(210.19225952,74.44151611)
\curveto(210.11225701,74.38150972)(210.03725708,74.3015098)(209.96725952,74.20151611)
\curveto(209.90725721,74.11150999)(209.85225727,74.01151009)(209.80225952,73.90151611)
\curveto(209.77225735,73.82151028)(209.74225738,73.73651037)(209.71225952,73.64651611)
\curveto(209.69225743,73.55651055)(209.64725747,73.48651062)(209.57725952,73.43651611)
\curveto(209.53725758,73.4065107)(209.46725765,73.38151072)(209.36725952,73.36151611)
\curveto(209.27725784,73.35151075)(209.18225794,73.34651076)(209.08225952,73.34651611)
\curveto(208.98225814,73.34651076)(208.88225824,73.35151075)(208.78225952,73.36151611)
\curveto(208.69225843,73.38151072)(208.62725849,73.4065107)(208.58725952,73.43651611)
\curveto(208.54725857,73.46651064)(208.5172586,73.51651059)(208.49725952,73.58651611)
\curveto(208.47725864,73.65651045)(208.47725864,73.73151037)(208.49725952,73.81151611)
\curveto(208.52725859,73.94151016)(208.55725856,74.06151004)(208.58725952,74.17151611)
\curveto(208.62725849,74.29150981)(208.67225845,74.4065097)(208.72225952,74.51651611)
\curveto(208.91225821,74.86650924)(209.15225797,75.13650897)(209.44225952,75.32651611)
\curveto(209.73225739,75.52650858)(210.09225703,75.68650842)(210.52225952,75.80651611)
\curveto(210.6222565,75.82650828)(210.7222564,75.84150826)(210.82225952,75.85151611)
\curveto(210.93225619,75.86150824)(211.04225608,75.87650823)(211.15225952,75.89651611)
\curveto(211.19225593,75.9065082)(211.25725586,75.9065082)(211.34725952,75.89651611)
\curveto(211.43725568,75.89650821)(211.49225563,75.9065082)(211.51225952,75.92651611)
\curveto(212.21225491,75.93650817)(212.8222543,75.85650825)(213.34225952,75.68651611)
\curveto(213.86225326,75.51650859)(214.22725289,75.19150891)(214.43725952,74.71151611)
\curveto(214.52725259,74.51150959)(214.57725254,74.27650983)(214.58725952,74.00651611)
\curveto(214.60725251,73.74651036)(214.6172525,73.47151063)(214.61725952,73.18151611)
\lineto(214.61725952,69.86651611)
\curveto(214.6172525,69.72651438)(214.6222525,69.59151451)(214.63225952,69.46151611)
\curveto(214.64225248,69.33151477)(214.67225245,69.22651488)(214.72225952,69.14651611)
\curveto(214.77225235,69.07651503)(214.83725228,69.02651508)(214.91725952,68.99651611)
\curveto(215.00725211,68.95651515)(215.09225203,68.92651518)(215.17225952,68.90651611)
\curveto(215.25225187,68.89651521)(215.31225181,68.85151525)(215.35225952,68.77151611)
\curveto(215.37225175,68.74151536)(215.38225174,68.71151539)(215.38225952,68.68151611)
\curveto(215.38225174,68.65151545)(215.38725173,68.61151549)(215.39725952,68.56151611)
\moveto(213.25225952,70.22651611)
\curveto(213.31225381,70.36651374)(213.34225378,70.52651358)(213.34225952,70.70651611)
\curveto(213.35225377,70.89651321)(213.35725376,71.09151301)(213.35725952,71.29151611)
\curveto(213.35725376,71.4015127)(213.35225377,71.5015126)(213.34225952,71.59151611)
\curveto(213.33225379,71.68151242)(213.29225383,71.75151235)(213.22225952,71.80151611)
\curveto(213.19225393,71.82151228)(213.122254,71.83151227)(213.01225952,71.83151611)
\curveto(212.99225413,71.81151229)(212.95725416,71.8015123)(212.90725952,71.80151611)
\curveto(212.85725426,71.8015123)(212.81225431,71.79151231)(212.77225952,71.77151611)
\curveto(212.69225443,71.75151235)(212.60225452,71.73151237)(212.50225952,71.71151611)
\lineto(212.20225952,71.65151611)
\curveto(212.17225495,71.65151245)(212.13725498,71.64651246)(212.09725952,71.63651611)
\lineto(211.99225952,71.63651611)
\curveto(211.84225528,71.59651251)(211.67725544,71.57151253)(211.49725952,71.56151611)
\curveto(211.32725579,71.56151254)(211.16725595,71.54151256)(211.01725952,71.50151611)
\curveto(210.93725618,71.48151262)(210.86225626,71.46151264)(210.79225952,71.44151611)
\curveto(210.73225639,71.43151267)(210.66225646,71.41651269)(210.58225952,71.39651611)
\curveto(210.4222567,71.34651276)(210.27225685,71.28151282)(210.13225952,71.20151611)
\curveto(209.99225713,71.13151297)(209.87225725,71.04151306)(209.77225952,70.93151611)
\curveto(209.67225745,70.82151328)(209.59725752,70.68651342)(209.54725952,70.52651611)
\curveto(209.49725762,70.37651373)(209.47725764,70.19151391)(209.48725952,69.97151611)
\curveto(209.48725763,69.87151423)(209.50225762,69.77651433)(209.53225952,69.68651611)
\curveto(209.57225755,69.6065145)(209.6172575,69.53151457)(209.66725952,69.46151611)
\curveto(209.74725737,69.35151475)(209.85225727,69.25651485)(209.98225952,69.17651611)
\curveto(210.11225701,69.106515)(210.25225687,69.04651506)(210.40225952,68.99651611)
\curveto(210.45225667,68.98651512)(210.50225662,68.98151512)(210.55225952,68.98151611)
\curveto(210.60225652,68.98151512)(210.65225647,68.97651513)(210.70225952,68.96651611)
\curveto(210.77225635,68.94651516)(210.85725626,68.93151517)(210.95725952,68.92151611)
\curveto(211.06725605,68.92151518)(211.15725596,68.93151517)(211.22725952,68.95151611)
\curveto(211.28725583,68.97151513)(211.34725577,68.97651513)(211.40725952,68.96651611)
\curveto(211.46725565,68.96651514)(211.52725559,68.97651513)(211.58725952,68.99651611)
\curveto(211.66725545,69.01651509)(211.74225538,69.03151507)(211.81225952,69.04151611)
\curveto(211.89225523,69.05151505)(211.96725515,69.07151503)(212.03725952,69.10151611)
\curveto(212.32725479,69.22151488)(212.57225455,69.36651474)(212.77225952,69.53651611)
\curveto(212.98225414,69.7065144)(213.14225398,69.93651417)(213.25225952,70.22651611)
}
}
{
\newrgbcolor{curcolor}{0 0 0}
\pscustom[linestyle=none,fillstyle=solid,fillcolor=curcolor]
{
\newpath
\moveto(223.52890015,68.81651611)
\lineto(223.52890015,68.42651611)
\curveto(223.52889227,68.3065158)(223.5038923,68.2065159)(223.45390015,68.12651611)
\curveto(223.4038924,68.05651605)(223.31889248,68.01651609)(223.19890015,68.00651611)
\lineto(222.85390015,68.00651611)
\curveto(222.79389301,68.0065161)(222.73389307,68.0015161)(222.67390015,67.99151611)
\curveto(222.62389318,67.99151611)(222.57889322,68.0015161)(222.53890015,68.02151611)
\curveto(222.44889335,68.04151606)(222.38889341,68.08151602)(222.35890015,68.14151611)
\curveto(222.31889348,68.19151591)(222.29389351,68.25151585)(222.28390015,68.32151611)
\curveto(222.28389352,68.39151571)(222.26889353,68.46151564)(222.23890015,68.53151611)
\curveto(222.22889357,68.55151555)(222.21389359,68.56651554)(222.19390015,68.57651611)
\curveto(222.18389362,68.59651551)(222.16889363,68.61651549)(222.14890015,68.63651611)
\curveto(222.04889375,68.64651546)(221.96889383,68.62651548)(221.90890015,68.57651611)
\curveto(221.85889394,68.52651558)(221.803894,68.47651563)(221.74390015,68.42651611)
\curveto(221.54389426,68.27651583)(221.34389446,68.16151594)(221.14390015,68.08151611)
\curveto(220.96389484,68.0015161)(220.75389505,67.94151616)(220.51390015,67.90151611)
\curveto(220.28389552,67.86151624)(220.04389576,67.84151626)(219.79390015,67.84151611)
\curveto(219.55389625,67.83151627)(219.31389649,67.84651626)(219.07390015,67.88651611)
\curveto(218.83389697,67.91651619)(218.62389718,67.97151613)(218.44390015,68.05151611)
\curveto(217.92389788,68.27151583)(217.5038983,68.56651554)(217.18390015,68.93651611)
\curveto(216.86389894,69.31651479)(216.61389919,69.78651432)(216.43390015,70.34651611)
\curveto(216.39389941,70.43651367)(216.36389944,70.52651358)(216.34390015,70.61651611)
\curveto(216.33389947,70.71651339)(216.31389949,70.81651329)(216.28390015,70.91651611)
\curveto(216.27389953,70.96651314)(216.26889953,71.01651309)(216.26890015,71.06651611)
\curveto(216.26889953,71.11651299)(216.26389954,71.16651294)(216.25390015,71.21651611)
\curveto(216.23389957,71.26651284)(216.22389958,71.31651279)(216.22390015,71.36651611)
\curveto(216.23389957,71.42651268)(216.23389957,71.48151262)(216.22390015,71.53151611)
\lineto(216.22390015,71.68151611)
\curveto(216.2038996,71.73151237)(216.19389961,71.79651231)(216.19390015,71.87651611)
\curveto(216.19389961,71.95651215)(216.2038996,72.02151208)(216.22390015,72.07151611)
\lineto(216.22390015,72.23651611)
\curveto(216.24389956,72.3065118)(216.24889955,72.37651173)(216.23890015,72.44651611)
\curveto(216.23889956,72.52651158)(216.24889955,72.6015115)(216.26890015,72.67151611)
\curveto(216.27889952,72.72151138)(216.28389952,72.76651134)(216.28390015,72.80651611)
\curveto(216.28389952,72.84651126)(216.28889951,72.89151121)(216.29890015,72.94151611)
\curveto(216.32889947,73.04151106)(216.35389945,73.13651097)(216.37390015,73.22651611)
\curveto(216.39389941,73.32651078)(216.41889938,73.42151068)(216.44890015,73.51151611)
\curveto(216.57889922,73.89151021)(216.74389906,74.23150987)(216.94390015,74.53151611)
\curveto(217.15389865,74.84150926)(217.4038984,75.09650901)(217.69390015,75.29651611)
\curveto(217.86389794,75.41650869)(218.03889776,75.51650859)(218.21890015,75.59651611)
\curveto(218.40889739,75.67650843)(218.61389719,75.74650836)(218.83390015,75.80651611)
\curveto(218.9038969,75.81650829)(218.96889683,75.82650828)(219.02890015,75.83651611)
\curveto(219.0988967,75.84650826)(219.16889663,75.86150824)(219.23890015,75.88151611)
\lineto(219.38890015,75.88151611)
\curveto(219.46889633,75.9015082)(219.58389622,75.91150819)(219.73390015,75.91151611)
\curveto(219.89389591,75.91150819)(220.01389579,75.9015082)(220.09390015,75.88151611)
\curveto(220.13389567,75.87150823)(220.18889561,75.86650824)(220.25890015,75.86651611)
\curveto(220.36889543,75.83650827)(220.47889532,75.81150829)(220.58890015,75.79151611)
\curveto(220.6988951,75.78150832)(220.803895,75.75150835)(220.90390015,75.70151611)
\curveto(221.05389475,75.64150846)(221.19389461,75.57650853)(221.32390015,75.50651611)
\curveto(221.46389434,75.43650867)(221.59389421,75.35650875)(221.71390015,75.26651611)
\curveto(221.77389403,75.21650889)(221.83389397,75.16150894)(221.89390015,75.10151611)
\curveto(221.96389384,75.05150905)(222.05389375,75.03650907)(222.16390015,75.05651611)
\curveto(222.18389362,75.08650902)(222.1988936,75.11150899)(222.20890015,75.13151611)
\curveto(222.22889357,75.15150895)(222.24389356,75.18150892)(222.25390015,75.22151611)
\curveto(222.28389352,75.31150879)(222.29389351,75.42650868)(222.28390015,75.56651611)
\lineto(222.28390015,75.94151611)
\lineto(222.28390015,77.66651611)
\lineto(222.28390015,78.13151611)
\curveto(222.28389352,78.31150579)(222.30889349,78.44150566)(222.35890015,78.52151611)
\curveto(222.3988934,78.59150551)(222.45889334,78.63650547)(222.53890015,78.65651611)
\curveto(222.55889324,78.65650545)(222.58389322,78.65650545)(222.61390015,78.65651611)
\curveto(222.64389316,78.66650544)(222.66889313,78.67150543)(222.68890015,78.67151611)
\curveto(222.82889297,78.68150542)(222.97389283,78.68150542)(223.12390015,78.67151611)
\curveto(223.28389252,78.67150543)(223.39389241,78.63150547)(223.45390015,78.55151611)
\curveto(223.5038923,78.47150563)(223.52889227,78.37150573)(223.52890015,78.25151611)
\lineto(223.52890015,77.87651611)
\lineto(223.52890015,68.81651611)
\moveto(222.31390015,71.65151611)
\curveto(222.33389347,71.7015124)(222.34389346,71.76651234)(222.34390015,71.84651611)
\curveto(222.34389346,71.93651217)(222.33389347,72.0065121)(222.31390015,72.05651611)
\lineto(222.31390015,72.28151611)
\curveto(222.29389351,72.37151173)(222.27889352,72.46151164)(222.26890015,72.55151611)
\curveto(222.25889354,72.65151145)(222.23889356,72.74151136)(222.20890015,72.82151611)
\curveto(222.18889361,72.9015112)(222.16889363,72.97651113)(222.14890015,73.04651611)
\curveto(222.13889366,73.11651099)(222.11889368,73.18651092)(222.08890015,73.25651611)
\curveto(221.96889383,73.55651055)(221.81389399,73.82151028)(221.62390015,74.05151611)
\curveto(221.43389437,74.28150982)(221.19389461,74.46150964)(220.90390015,74.59151611)
\curveto(220.803895,74.64150946)(220.6988951,74.67650943)(220.58890015,74.69651611)
\curveto(220.48889531,74.72650938)(220.37889542,74.75150935)(220.25890015,74.77151611)
\curveto(220.17889562,74.79150931)(220.08889571,74.8015093)(219.98890015,74.80151611)
\lineto(219.71890015,74.80151611)
\curveto(219.66889613,74.79150931)(219.62389618,74.78150932)(219.58390015,74.77151611)
\lineto(219.44890015,74.77151611)
\curveto(219.36889643,74.75150935)(219.28389652,74.73150937)(219.19390015,74.71151611)
\curveto(219.11389669,74.69150941)(219.03389677,74.66650944)(218.95390015,74.63651611)
\curveto(218.63389717,74.49650961)(218.37389743,74.29150981)(218.17390015,74.02151611)
\curveto(217.98389782,73.76151034)(217.82889797,73.45651065)(217.70890015,73.10651611)
\curveto(217.66889813,72.99651111)(217.63889816,72.88151122)(217.61890015,72.76151611)
\curveto(217.60889819,72.65151145)(217.59389821,72.54151156)(217.57390015,72.43151611)
\curveto(217.57389823,72.39151171)(217.56889823,72.35151175)(217.55890015,72.31151611)
\lineto(217.55890015,72.20651611)
\curveto(217.53889826,72.15651195)(217.52889827,72.101512)(217.52890015,72.04151611)
\curveto(217.53889826,71.98151212)(217.54389826,71.92651218)(217.54390015,71.87651611)
\lineto(217.54390015,71.54651611)
\curveto(217.54389826,71.44651266)(217.55389825,71.35151275)(217.57390015,71.26151611)
\curveto(217.58389822,71.23151287)(217.58889821,71.18151292)(217.58890015,71.11151611)
\curveto(217.60889819,71.04151306)(217.62389818,70.97151313)(217.63390015,70.90151611)
\lineto(217.69390015,70.69151611)
\curveto(217.803898,70.34151376)(217.95389785,70.04151406)(218.14390015,69.79151611)
\curveto(218.33389747,69.54151456)(218.57389723,69.33651477)(218.86390015,69.17651611)
\curveto(218.95389685,69.12651498)(219.04389676,69.08651502)(219.13390015,69.05651611)
\curveto(219.22389658,69.02651508)(219.32389648,68.99651511)(219.43390015,68.96651611)
\curveto(219.48389632,68.94651516)(219.53389627,68.94151516)(219.58390015,68.95151611)
\curveto(219.64389616,68.96151514)(219.6988961,68.95651515)(219.74890015,68.93651611)
\curveto(219.78889601,68.92651518)(219.82889597,68.92151518)(219.86890015,68.92151611)
\lineto(220.00390015,68.92151611)
\lineto(220.13890015,68.92151611)
\curveto(220.16889563,68.93151517)(220.21889558,68.93651517)(220.28890015,68.93651611)
\curveto(220.36889543,68.95651515)(220.44889535,68.97151513)(220.52890015,68.98151611)
\curveto(220.60889519,69.0015151)(220.68389512,69.02651508)(220.75390015,69.05651611)
\curveto(221.08389472,69.19651491)(221.34889445,69.37151473)(221.54890015,69.58151611)
\curveto(221.75889404,69.8015143)(221.93389387,70.07651403)(222.07390015,70.40651611)
\curveto(222.12389368,70.51651359)(222.15889364,70.62651348)(222.17890015,70.73651611)
\curveto(222.1988936,70.84651326)(222.22389358,70.95651315)(222.25390015,71.06651611)
\curveto(222.27389353,71.106513)(222.28389352,71.14151296)(222.28390015,71.17151611)
\curveto(222.28389352,71.21151289)(222.28889351,71.25151285)(222.29890015,71.29151611)
\curveto(222.30889349,71.35151275)(222.30889349,71.41151269)(222.29890015,71.47151611)
\curveto(222.2988935,71.53151257)(222.3038935,71.59151251)(222.31390015,71.65151611)
}
}
{
\newrgbcolor{curcolor}{0 0 0}
\pscustom[linestyle=none,fillstyle=solid,fillcolor=curcolor]
{
\newpath
\moveto(232.60015015,72.20651611)
\curveto(232.62014209,72.14651196)(232.63014208,72.05151205)(232.63015015,71.92151611)
\curveto(232.63014208,71.8015123)(232.62514208,71.71651239)(232.61515015,71.66651611)
\lineto(232.61515015,71.51651611)
\curveto(232.6051421,71.43651267)(232.59514211,71.36151274)(232.58515015,71.29151611)
\curveto(232.58514212,71.23151287)(232.58014213,71.16151294)(232.57015015,71.08151611)
\curveto(232.55014216,71.02151308)(232.53514217,70.96151314)(232.52515015,70.90151611)
\curveto(232.52514218,70.84151326)(232.51514219,70.78151332)(232.49515015,70.72151611)
\curveto(232.45514225,70.59151351)(232.42014229,70.46151364)(232.39015015,70.33151611)
\curveto(232.36014235,70.2015139)(232.32014239,70.08151402)(232.27015015,69.97151611)
\curveto(232.06014265,69.49151461)(231.78014293,69.08651502)(231.43015015,68.75651611)
\curveto(231.08014363,68.43651567)(230.65014406,68.19151591)(230.14015015,68.02151611)
\curveto(230.03014468,67.98151612)(229.9101448,67.95151615)(229.78015015,67.93151611)
\curveto(229.66014505,67.91151619)(229.53514517,67.89151621)(229.40515015,67.87151611)
\curveto(229.34514536,67.86151624)(229.28014543,67.85651625)(229.21015015,67.85651611)
\curveto(229.15014556,67.84651626)(229.09014562,67.84151626)(229.03015015,67.84151611)
\curveto(228.99014572,67.83151627)(228.93014578,67.82651628)(228.85015015,67.82651611)
\curveto(228.78014593,67.82651628)(228.73014598,67.83151627)(228.70015015,67.84151611)
\curveto(228.66014605,67.85151625)(228.62014609,67.85651625)(228.58015015,67.85651611)
\curveto(228.54014617,67.84651626)(228.5051462,67.84651626)(228.47515015,67.85651611)
\lineto(228.38515015,67.85651611)
\lineto(228.02515015,67.90151611)
\curveto(227.88514682,67.94151616)(227.75014696,67.98151612)(227.62015015,68.02151611)
\curveto(227.49014722,68.06151604)(227.36514734,68.106516)(227.24515015,68.15651611)
\curveto(226.79514791,68.35651575)(226.42514828,68.61651549)(226.13515015,68.93651611)
\curveto(225.84514886,69.25651485)(225.6051491,69.64651446)(225.41515015,70.10651611)
\curveto(225.36514934,70.2065139)(225.32514938,70.3065138)(225.29515015,70.40651611)
\curveto(225.27514943,70.5065136)(225.25514945,70.61151349)(225.23515015,70.72151611)
\curveto(225.21514949,70.76151334)(225.2051495,70.79151331)(225.20515015,70.81151611)
\curveto(225.21514949,70.84151326)(225.21514949,70.87651323)(225.20515015,70.91651611)
\curveto(225.18514952,70.99651311)(225.17014954,71.07651303)(225.16015015,71.15651611)
\curveto(225.16014955,71.24651286)(225.15014956,71.33151277)(225.13015015,71.41151611)
\lineto(225.13015015,71.53151611)
\curveto(225.13014958,71.57151253)(225.12514958,71.61651249)(225.11515015,71.66651611)
\curveto(225.1051496,71.71651239)(225.10014961,71.8015123)(225.10015015,71.92151611)
\curveto(225.10014961,72.05151205)(225.1101496,72.14651196)(225.13015015,72.20651611)
\curveto(225.15014956,72.27651183)(225.15514955,72.34651176)(225.14515015,72.41651611)
\curveto(225.13514957,72.48651162)(225.14014957,72.55651155)(225.16015015,72.62651611)
\curveto(225.17014954,72.67651143)(225.17514953,72.71651139)(225.17515015,72.74651611)
\curveto(225.18514952,72.78651132)(225.19514951,72.83151127)(225.20515015,72.88151611)
\curveto(225.23514947,73.0015111)(225.26014945,73.12151098)(225.28015015,73.24151611)
\curveto(225.3101494,73.36151074)(225.35014936,73.47651063)(225.40015015,73.58651611)
\curveto(225.55014916,73.95651015)(225.73014898,74.28650982)(225.94015015,74.57651611)
\curveto(226.16014855,74.87650923)(226.42514828,75.12650898)(226.73515015,75.32651611)
\curveto(226.85514785,75.4065087)(226.98014773,75.47150863)(227.11015015,75.52151611)
\curveto(227.24014747,75.58150852)(227.37514733,75.64150846)(227.51515015,75.70151611)
\curveto(227.63514707,75.75150835)(227.76514694,75.78150832)(227.90515015,75.79151611)
\curveto(228.04514666,75.81150829)(228.18514652,75.84150826)(228.32515015,75.88151611)
\lineto(228.52015015,75.88151611)
\curveto(228.59014612,75.89150821)(228.65514605,75.9015082)(228.71515015,75.91151611)
\curveto(229.6051451,75.92150818)(230.34514436,75.73650837)(230.93515015,75.35651611)
\curveto(231.52514318,74.97650913)(231.95014276,74.48150962)(232.21015015,73.87151611)
\curveto(232.26014245,73.77151033)(232.30014241,73.67151043)(232.33015015,73.57151611)
\curveto(232.36014235,73.47151063)(232.39514231,73.36651074)(232.43515015,73.25651611)
\curveto(232.46514224,73.14651096)(232.49014222,73.02651108)(232.51015015,72.89651611)
\curveto(232.53014218,72.77651133)(232.55514215,72.65151145)(232.58515015,72.52151611)
\curveto(232.59514211,72.47151163)(232.59514211,72.41651169)(232.58515015,72.35651611)
\curveto(232.58514212,72.3065118)(232.59014212,72.25651185)(232.60015015,72.20651611)
\moveto(231.26515015,71.35151611)
\curveto(231.28514342,71.42151268)(231.29014342,71.5015126)(231.28015015,71.59151611)
\lineto(231.28015015,71.84651611)
\curveto(231.28014343,72.23651187)(231.24514346,72.56651154)(231.17515015,72.83651611)
\curveto(231.14514356,72.91651119)(231.12014359,72.99651111)(231.10015015,73.07651611)
\curveto(231.08014363,73.15651095)(231.05514365,73.23151087)(231.02515015,73.30151611)
\curveto(230.74514396,73.95151015)(230.30014441,74.4015097)(229.69015015,74.65151611)
\curveto(229.62014509,74.68150942)(229.54514516,74.7015094)(229.46515015,74.71151611)
\lineto(229.22515015,74.77151611)
\curveto(229.14514556,74.79150931)(229.06014565,74.8015093)(228.97015015,74.80151611)
\lineto(228.70015015,74.80151611)
\lineto(228.43015015,74.75651611)
\curveto(228.33014638,74.73650937)(228.23514647,74.71150939)(228.14515015,74.68151611)
\curveto(228.06514664,74.66150944)(227.98514672,74.63150947)(227.90515015,74.59151611)
\curveto(227.83514687,74.57150953)(227.77014694,74.54150956)(227.71015015,74.50151611)
\curveto(227.65014706,74.46150964)(227.59514711,74.42150968)(227.54515015,74.38151611)
\curveto(227.3051474,74.21150989)(227.1101476,74.0065101)(226.96015015,73.76651611)
\curveto(226.8101479,73.52651058)(226.68014803,73.24651086)(226.57015015,72.92651611)
\curveto(226.54014817,72.82651128)(226.52014819,72.72151138)(226.51015015,72.61151611)
\curveto(226.50014821,72.51151159)(226.48514822,72.4065117)(226.46515015,72.29651611)
\curveto(226.45514825,72.25651185)(226.45014826,72.19151191)(226.45015015,72.10151611)
\curveto(226.44014827,72.07151203)(226.43514827,72.03651207)(226.43515015,71.99651611)
\curveto(226.44514826,71.95651215)(226.45014826,71.91151219)(226.45015015,71.86151611)
\lineto(226.45015015,71.56151611)
\curveto(226.45014826,71.46151264)(226.46014825,71.37151273)(226.48015015,71.29151611)
\lineto(226.51015015,71.11151611)
\curveto(226.53014818,71.01151309)(226.54514816,70.91151319)(226.55515015,70.81151611)
\curveto(226.57514813,70.72151338)(226.6051481,70.63651347)(226.64515015,70.55651611)
\curveto(226.74514796,70.31651379)(226.86014785,70.09151401)(226.99015015,69.88151611)
\curveto(227.13014758,69.67151443)(227.30014741,69.49651461)(227.50015015,69.35651611)
\curveto(227.55014716,69.32651478)(227.59514711,69.3015148)(227.63515015,69.28151611)
\curveto(227.67514703,69.26151484)(227.72014699,69.23651487)(227.77015015,69.20651611)
\curveto(227.85014686,69.15651495)(227.93514677,69.11151499)(228.02515015,69.07151611)
\curveto(228.12514658,69.04151506)(228.23014648,69.01151509)(228.34015015,68.98151611)
\curveto(228.39014632,68.96151514)(228.43514627,68.95151515)(228.47515015,68.95151611)
\curveto(228.52514618,68.96151514)(228.57514613,68.96151514)(228.62515015,68.95151611)
\curveto(228.65514605,68.94151516)(228.71514599,68.93151517)(228.80515015,68.92151611)
\curveto(228.9051458,68.91151519)(228.98014573,68.91651519)(229.03015015,68.93651611)
\curveto(229.07014564,68.94651516)(229.1101456,68.94651516)(229.15015015,68.93651611)
\curveto(229.19014552,68.93651517)(229.23014548,68.94651516)(229.27015015,68.96651611)
\curveto(229.35014536,68.98651512)(229.43014528,69.0015151)(229.51015015,69.01151611)
\curveto(229.59014512,69.03151507)(229.66514504,69.05651505)(229.73515015,69.08651611)
\curveto(230.07514463,69.22651488)(230.35014436,69.42151468)(230.56015015,69.67151611)
\curveto(230.77014394,69.92151418)(230.94514376,70.21651389)(231.08515015,70.55651611)
\curveto(231.13514357,70.67651343)(231.16514354,70.8015133)(231.17515015,70.93151611)
\curveto(231.19514351,71.07151303)(231.22514348,71.21151289)(231.26515015,71.35151611)
}
}
{
\newrgbcolor{curcolor}{0 0 0}
\pscustom[linestyle=none,fillstyle=solid,fillcolor=curcolor]
{
\newpath
\moveto(373.29141846,68.78651611)
\curveto(373.31140891,68.73651537)(373.33640889,68.67651543)(373.36641846,68.60651611)
\curveto(373.39640883,68.53651557)(373.41640881,68.46151564)(373.42641846,68.38151611)
\curveto(373.44640878,68.31151579)(373.44640878,68.24151586)(373.42641846,68.17151611)
\curveto(373.41640881,68.11151599)(373.37640885,68.06651604)(373.30641846,68.03651611)
\curveto(373.25640897,68.01651609)(373.19640903,68.0065161)(373.12641846,68.00651611)
\lineto(372.91641846,68.00651611)
\lineto(372.46641846,68.00651611)
\curveto(372.31640991,68.0065161)(372.19641003,68.03151607)(372.10641846,68.08151611)
\curveto(372.00641022,68.14151596)(371.93141029,68.24651586)(371.88141846,68.39651611)
\curveto(371.84141038,68.54651556)(371.79641043,68.68151542)(371.74641846,68.80151611)
\curveto(371.63641059,69.06151504)(371.53641069,69.33151477)(371.44641846,69.61151611)
\curveto(371.35641087,69.89151421)(371.25641097,70.16651394)(371.14641846,70.43651611)
\curveto(371.11641111,70.52651358)(371.08641114,70.61151349)(371.05641846,70.69151611)
\curveto(371.03641119,70.77151333)(371.00641122,70.84651326)(370.96641846,70.91651611)
\curveto(370.93641129,70.98651312)(370.89141133,71.04651306)(370.83141846,71.09651611)
\curveto(370.77141145,71.14651296)(370.69141153,71.18651292)(370.59141846,71.21651611)
\curveto(370.54141168,71.23651287)(370.48141174,71.24151286)(370.41141846,71.23151611)
\lineto(370.21641846,71.23151611)
\lineto(367.38141846,71.23151611)
\lineto(367.08141846,71.23151611)
\curveto(366.97141525,71.24151286)(366.86641536,71.24151286)(366.76641846,71.23151611)
\curveto(366.66641556,71.22151288)(366.57141565,71.2065129)(366.48141846,71.18651611)
\curveto(366.40141582,71.16651294)(366.34141588,71.12651298)(366.30141846,71.06651611)
\curveto(366.221416,70.96651314)(366.16141606,70.85151325)(366.12141846,70.72151611)
\curveto(366.09141613,70.6015135)(366.05141617,70.47651363)(366.00141846,70.34651611)
\curveto(365.90141632,70.11651399)(365.80641642,69.87651423)(365.71641846,69.62651611)
\curveto(365.63641659,69.37651473)(365.54641668,69.13651497)(365.44641846,68.90651611)
\curveto(365.4264168,68.84651526)(365.40141682,68.77651533)(365.37141846,68.69651611)
\curveto(365.35141687,68.62651548)(365.3264169,68.55151555)(365.29641846,68.47151611)
\curveto(365.26641696,68.39151571)(365.23141699,68.31651579)(365.19141846,68.24651611)
\curveto(365.16141706,68.18651592)(365.1264171,68.14151596)(365.08641846,68.11151611)
\curveto(365.00641722,68.05151605)(364.89641733,68.01651609)(364.75641846,68.00651611)
\lineto(364.33641846,68.00651611)
\lineto(364.09641846,68.00651611)
\curveto(364.0264182,68.01651609)(363.96641826,68.04151606)(363.91641846,68.08151611)
\curveto(363.86641836,68.11151599)(363.83641839,68.15651595)(363.82641846,68.21651611)
\curveto(363.8264184,68.27651583)(363.83141839,68.33651577)(363.84141846,68.39651611)
\curveto(363.86141836,68.46651564)(363.88141834,68.53151557)(363.90141846,68.59151611)
\curveto(363.93141829,68.66151544)(363.95641827,68.71151539)(363.97641846,68.74151611)
\curveto(364.11641811,69.06151504)(364.24141798,69.37651473)(364.35141846,69.68651611)
\curveto(364.46141776,70.0065141)(364.58141764,70.32651378)(364.71141846,70.64651611)
\curveto(364.80141742,70.86651324)(364.88641734,71.08151302)(364.96641846,71.29151611)
\curveto(365.04641718,71.51151259)(365.13141709,71.73151237)(365.22141846,71.95151611)
\curveto(365.5214167,72.67151143)(365.80641642,73.39651071)(366.07641846,74.12651611)
\curveto(366.34641588,74.86650924)(366.63141559,75.6015085)(366.93141846,76.33151611)
\curveto(367.04141518,76.59150751)(367.14141508,76.85650725)(367.23141846,77.12651611)
\curveto(367.33141489,77.39650671)(367.43641479,77.66150644)(367.54641846,77.92151611)
\curveto(367.59641463,78.03150607)(367.64141458,78.15150595)(367.68141846,78.28151611)
\curveto(367.73141449,78.42150568)(367.80141442,78.52150558)(367.89141846,78.58151611)
\curveto(367.93141429,78.62150548)(367.99641423,78.65150545)(368.08641846,78.67151611)
\curveto(368.10641412,78.68150542)(368.1264141,78.68150542)(368.14641846,78.67151611)
\curveto(368.17641405,78.67150543)(368.20141402,78.67650543)(368.22141846,78.68651611)
\curveto(368.40141382,78.68650542)(368.61141361,78.68650542)(368.85141846,78.68651611)
\curveto(369.09141313,78.69650541)(369.26641296,78.66150544)(369.37641846,78.58151611)
\curveto(369.45641277,78.52150558)(369.51641271,78.42150568)(369.55641846,78.28151611)
\curveto(369.60641262,78.15150595)(369.65641257,78.03150607)(369.70641846,77.92151611)
\curveto(369.80641242,77.69150641)(369.89641233,77.46150664)(369.97641846,77.23151611)
\curveto(370.05641217,77.0015071)(370.14641208,76.77150733)(370.24641846,76.54151611)
\curveto(370.3264119,76.34150776)(370.40141182,76.13650797)(370.47141846,75.92651611)
\curveto(370.55141167,75.71650839)(370.63641159,75.51150859)(370.72641846,75.31151611)
\curveto(371.0264112,74.58150952)(371.31141091,73.84151026)(371.58141846,73.09151611)
\curveto(371.86141036,72.35151175)(372.15641007,71.61651249)(372.46641846,70.88651611)
\curveto(372.50640972,70.79651331)(372.53640969,70.71151339)(372.55641846,70.63151611)
\curveto(372.58640964,70.55151355)(372.61640961,70.46651364)(372.64641846,70.37651611)
\curveto(372.75640947,70.11651399)(372.86140936,69.85151425)(372.96141846,69.58151611)
\curveto(373.07140915,69.31151479)(373.18140904,69.04651506)(373.29141846,68.78651611)
\moveto(370.08141846,72.43151611)
\curveto(370.17141205,72.46151164)(370.226412,72.51151159)(370.24641846,72.58151611)
\curveto(370.27641195,72.65151145)(370.28141194,72.72651138)(370.26141846,72.80651611)
\curveto(370.25141197,72.89651121)(370.226412,72.98151112)(370.18641846,73.06151611)
\curveto(370.15641207,73.15151095)(370.1264121,73.22651088)(370.09641846,73.28651611)
\curveto(370.07641215,73.32651078)(370.06641216,73.36151074)(370.06641846,73.39151611)
\curveto(370.06641216,73.42151068)(370.05641217,73.45651065)(370.03641846,73.49651611)
\lineto(369.94641846,73.73651611)
\curveto(369.9264123,73.82651028)(369.89641233,73.91651019)(369.85641846,74.00651611)
\curveto(369.70641252,74.36650974)(369.57141265,74.73150937)(369.45141846,75.10151611)
\curveto(369.34141288,75.48150862)(369.21141301,75.85150825)(369.06141846,76.21151611)
\curveto(369.01141321,76.32150778)(368.96641326,76.43150767)(368.92641846,76.54151611)
\curveto(368.89641333,76.65150745)(368.85641337,76.75650735)(368.80641846,76.85651611)
\curveto(368.78641344,76.9065072)(368.76141346,76.95150715)(368.73141846,76.99151611)
\curveto(368.71141351,77.04150706)(368.66141356,77.06650704)(368.58141846,77.06651611)
\curveto(368.56141366,77.04650706)(368.54141368,77.03150707)(368.52141846,77.02151611)
\curveto(368.50141372,77.01150709)(368.48141374,76.99650711)(368.46141846,76.97651611)
\curveto(368.4214138,76.92650718)(368.39141383,76.87150723)(368.37141846,76.81151611)
\curveto(368.35141387,76.76150734)(368.33141389,76.7065074)(368.31141846,76.64651611)
\curveto(368.26141396,76.53650757)(368.221414,76.42650768)(368.19141846,76.31651611)
\curveto(368.16141406,76.2065079)(368.1214141,76.09650801)(368.07141846,75.98651611)
\curveto(367.90141432,75.59650851)(367.75141447,75.2015089)(367.62141846,74.80151611)
\curveto(367.50141472,74.4015097)(367.36141486,74.01151009)(367.20141846,73.63151611)
\lineto(367.14141846,73.48151611)
\curveto(367.13141509,73.43151067)(367.11641511,73.38151072)(367.09641846,73.33151611)
\lineto(367.00641846,73.09151611)
\curveto(366.97641525,73.01151109)(366.95141527,72.93151117)(366.93141846,72.85151611)
\curveto(366.91141531,72.8015113)(366.90141532,72.74651136)(366.90141846,72.68651611)
\curveto(366.91141531,72.62651148)(366.9264153,72.57651153)(366.94641846,72.53651611)
\curveto(366.99641523,72.45651165)(367.10141512,72.41151169)(367.26141846,72.40151611)
\lineto(367.71141846,72.40151611)
\lineto(369.31641846,72.40151611)
\curveto(369.4264128,72.4015117)(369.56141266,72.39651171)(369.72141846,72.38651611)
\curveto(369.88141234,72.38651172)(370.00141222,72.4015117)(370.08141846,72.43151611)
}
}
{
\newrgbcolor{curcolor}{0 0 0}
\pscustom[linestyle=none,fillstyle=solid,fillcolor=curcolor]
{
\newpath
\moveto(374.87298096,75.73151611)
\lineto(375.30798096,75.73151611)
\curveto(375.45797899,75.73150837)(375.56297889,75.69150841)(375.62298096,75.61151611)
\curveto(375.67297878,75.53150857)(375.69797875,75.43150867)(375.69798096,75.31151611)
\curveto(375.70797874,75.19150891)(375.71297874,75.07150903)(375.71298096,74.95151611)
\lineto(375.71298096,73.52651611)
\lineto(375.71298096,71.26151611)
\lineto(375.71298096,70.57151611)
\curveto(375.71297874,70.34151376)(375.73797871,70.14151396)(375.78798096,69.97151611)
\curveto(375.9479785,69.52151458)(376.2479782,69.2065149)(376.68798096,69.02651611)
\curveto(376.90797754,68.93651517)(377.17297728,68.9015152)(377.48298096,68.92151611)
\curveto(377.79297666,68.95151515)(378.04297641,69.0065151)(378.23298096,69.08651611)
\curveto(378.56297589,69.22651488)(378.82297563,69.4015147)(379.01298096,69.61151611)
\curveto(379.21297524,69.83151427)(379.36797508,70.11651399)(379.47798096,70.46651611)
\curveto(379.50797494,70.54651356)(379.52797492,70.62651348)(379.53798096,70.70651611)
\curveto(379.5479749,70.78651332)(379.56297489,70.87151323)(379.58298096,70.96151611)
\curveto(379.59297486,71.01151309)(379.59297486,71.05651305)(379.58298096,71.09651611)
\curveto(379.58297487,71.13651297)(379.59297486,71.18151292)(379.61298096,71.23151611)
\lineto(379.61298096,71.54651611)
\curveto(379.63297482,71.62651248)(379.63797481,71.71651239)(379.62798096,71.81651611)
\curveto(379.61797483,71.92651218)(379.61297484,72.02651208)(379.61298096,72.11651611)
\lineto(379.61298096,73.28651611)
\lineto(379.61298096,74.87651611)
\curveto(379.61297484,74.99650911)(379.60797484,75.12150898)(379.59798096,75.25151611)
\curveto(379.59797485,75.39150871)(379.62297483,75.5015086)(379.67298096,75.58151611)
\curveto(379.71297474,75.63150847)(379.75797469,75.66150844)(379.80798096,75.67151611)
\curveto(379.86797458,75.69150841)(379.93797451,75.71150839)(380.01798096,75.73151611)
\lineto(380.24298096,75.73151611)
\curveto(380.36297409,75.73150837)(380.46797398,75.72650838)(380.55798096,75.71651611)
\curveto(380.65797379,75.7065084)(380.73297372,75.66150844)(380.78298096,75.58151611)
\curveto(380.83297362,75.53150857)(380.85797359,75.45650865)(380.85798096,75.35651611)
\lineto(380.85798096,75.07151611)
\lineto(380.85798096,74.05151611)
\lineto(380.85798096,70.01651611)
\lineto(380.85798096,68.66651611)
\curveto(380.85797359,68.54651556)(380.8529736,68.43151567)(380.84298096,68.32151611)
\curveto(380.84297361,68.22151588)(380.80797364,68.14651596)(380.73798096,68.09651611)
\curveto(380.69797375,68.06651604)(380.63797381,68.04151606)(380.55798096,68.02151611)
\curveto(380.47797397,68.01151609)(380.38797406,68.0015161)(380.28798096,67.99151611)
\curveto(380.19797425,67.99151611)(380.10797434,67.99651611)(380.01798096,68.00651611)
\curveto(379.93797451,68.01651609)(379.87797457,68.03651607)(379.83798096,68.06651611)
\curveto(379.78797466,68.106516)(379.74297471,68.17151593)(379.70298096,68.26151611)
\curveto(379.69297476,68.3015158)(379.68297477,68.35651575)(379.67298096,68.42651611)
\curveto(379.67297478,68.49651561)(379.66797478,68.56151554)(379.65798096,68.62151611)
\curveto(379.6479748,68.69151541)(379.62797482,68.74651536)(379.59798096,68.78651611)
\curveto(379.56797488,68.82651528)(379.52297493,68.84151526)(379.46298096,68.83151611)
\curveto(379.38297507,68.81151529)(379.30297515,68.75151535)(379.22298096,68.65151611)
\curveto(379.14297531,68.56151554)(379.06797538,68.49151561)(378.99798096,68.44151611)
\curveto(378.77797567,68.28151582)(378.52797592,68.14151596)(378.24798096,68.02151611)
\curveto(378.13797631,67.97151613)(378.02297643,67.94151616)(377.90298096,67.93151611)
\curveto(377.79297666,67.91151619)(377.67797677,67.88651622)(377.55798096,67.85651611)
\curveto(377.50797694,67.84651626)(377.452977,67.84651626)(377.39298096,67.85651611)
\curveto(377.34297711,67.86651624)(377.29297716,67.86151624)(377.24298096,67.84151611)
\curveto(377.14297731,67.82151628)(377.0529774,67.82151628)(376.97298096,67.84151611)
\lineto(376.82298096,67.84151611)
\curveto(376.77297768,67.86151624)(376.71297774,67.87151623)(376.64298096,67.87151611)
\curveto(376.58297787,67.87151623)(376.52797792,67.87651623)(376.47798096,67.88651611)
\curveto(376.43797801,67.9065162)(376.39797805,67.91651619)(376.35798096,67.91651611)
\curveto(376.32797812,67.9065162)(376.28797816,67.91151619)(376.23798096,67.93151611)
\lineto(375.99798096,67.99151611)
\curveto(375.92797852,68.01151609)(375.8529786,68.04151606)(375.77298096,68.08151611)
\curveto(375.51297894,68.19151591)(375.29297916,68.33651577)(375.11298096,68.51651611)
\curveto(374.94297951,68.7065154)(374.80297965,68.93151517)(374.69298096,69.19151611)
\curveto(374.6529798,69.28151482)(374.62297983,69.37151473)(374.60298096,69.46151611)
\lineto(374.54298096,69.76151611)
\curveto(374.52297993,69.82151428)(374.51297994,69.87651423)(374.51298096,69.92651611)
\curveto(374.52297993,69.98651412)(374.51797993,70.05151405)(374.49798096,70.12151611)
\curveto(374.48797996,70.14151396)(374.48297997,70.16651394)(374.48298096,70.19651611)
\curveto(374.48297997,70.23651387)(374.47797997,70.27151383)(374.46798096,70.30151611)
\lineto(374.46798096,70.45151611)
\curveto(374.45797999,70.49151361)(374.45298,70.53651357)(374.45298096,70.58651611)
\curveto(374.46297999,70.64651346)(374.46797998,70.7015134)(374.46798096,70.75151611)
\lineto(374.46798096,71.35151611)
\lineto(374.46798096,74.11151611)
\lineto(374.46798096,75.07151611)
\lineto(374.46798096,75.34151611)
\curveto(374.46797998,75.43150867)(374.48797996,75.5065086)(374.52798096,75.56651611)
\curveto(374.56797988,75.63650847)(374.64297981,75.68650842)(374.75298096,75.71651611)
\curveto(374.77297968,75.72650838)(374.79297966,75.72650838)(374.81298096,75.71651611)
\curveto(374.83297962,75.71650839)(374.8529796,75.72150838)(374.87298096,75.73151611)
}
}
{
\newrgbcolor{curcolor}{0 0 0}
\pscustom[linestyle=none,fillstyle=solid,fillcolor=curcolor]
{
\newpath
\moveto(382.77259033,75.73151611)
\lineto(383.29759033,75.73151611)
\curveto(383.49758868,75.74150836)(383.64758853,75.72150838)(383.74759033,75.67151611)
\curveto(383.86758831,75.62150848)(383.96258821,75.54150856)(384.03259033,75.43151611)
\curveto(384.11258806,75.32150878)(384.18758799,75.21150889)(384.25759033,75.10151611)
\curveto(384.38758779,74.9015092)(384.51758766,74.7065094)(384.64759033,74.51651611)
\curveto(384.7775874,74.33650977)(384.91258726,74.14650996)(385.05259033,73.94651611)
\curveto(385.10258707,73.86651024)(385.15258702,73.79151031)(385.20259033,73.72151611)
\curveto(385.26258691,73.65151045)(385.31758686,73.58151052)(385.36759033,73.51151611)
\curveto(385.40758677,73.45151065)(385.44758673,73.39651071)(385.48759033,73.34651611)
\curveto(385.52758665,73.29651081)(385.58758659,73.26151084)(385.66759033,73.24151611)
\curveto(385.71758646,73.22151088)(385.75758642,73.22151088)(385.78759033,73.24151611)
\curveto(385.82758635,73.27151083)(385.85758632,73.29651081)(385.87759033,73.31651611)
\curveto(385.95758622,73.36651074)(386.02258615,73.43651067)(386.07259033,73.52651611)
\curveto(386.13258604,73.61651049)(386.18758599,73.7015104)(386.23759033,73.78151611)
\curveto(386.38758579,73.98151012)(386.53758564,74.18650992)(386.68759033,74.39651611)
\lineto(387.13759033,75.02651611)
\curveto(387.21758496,75.13650897)(387.29758488,75.25150885)(387.37759033,75.37151611)
\curveto(387.45758472,75.49150861)(387.55258462,75.58650852)(387.66259033,75.65651611)
\curveto(387.74258443,75.7065084)(387.83758434,75.73150837)(387.94759033,75.73151611)
\lineto(388.29259033,75.73151611)
\lineto(388.42759033,75.73151611)
\curveto(388.4775837,75.73150837)(388.52758365,75.72650838)(388.57759033,75.71651611)
\lineto(388.65259033,75.71651611)
\curveto(388.7725834,75.69650841)(388.85258332,75.65650845)(388.89259033,75.59651611)
\curveto(388.91258326,75.54650856)(388.90758327,75.49150861)(388.87759033,75.43151611)
\curveto(388.85758332,75.38150872)(388.83758334,75.34150876)(388.81759033,75.31151611)
\lineto(388.60759033,75.01151611)
\curveto(388.53758364,74.92150918)(388.46258371,74.82650928)(388.38259033,74.72651611)
\curveto(388.15258402,74.4065097)(387.91758426,74.09151001)(387.67759033,73.78151611)
\curveto(387.44758473,73.48151062)(387.21758496,73.17151093)(386.98759033,72.85151611)
\curveto(386.93758524,72.77151133)(386.88258529,72.69151141)(386.82259033,72.61151611)
\curveto(386.76258541,72.54151156)(386.70758547,72.46151164)(386.65759033,72.37151611)
\curveto(386.63758554,72.34151176)(386.61758556,72.3015118)(386.59759033,72.25151611)
\curveto(386.5775856,72.21151189)(386.5775856,72.16151194)(386.59759033,72.10151611)
\curveto(386.61758556,72.01151209)(386.64758553,71.93651217)(386.68759033,71.87651611)
\curveto(386.73758544,71.81651229)(386.78758539,71.75151235)(386.83759033,71.68151611)
\lineto(387.01759033,71.41151611)
\curveto(387.08758509,71.32151278)(387.15258502,71.23151287)(387.21259033,71.14151611)
\lineto(387.90259033,70.18151611)
\lineto(388.59259033,69.22151611)
\curveto(388.6725835,69.11151499)(388.75258342,68.99651511)(388.83259033,68.87651611)
\lineto(389.07259033,68.54651611)
\curveto(389.12258305,68.47651563)(389.16258301,68.41151569)(389.19259033,68.35151611)
\curveto(389.23258294,68.3015158)(389.24258293,68.22151588)(389.22259033,68.11151611)
\curveto(389.20258297,68.101516)(389.18258299,68.08651602)(389.16259033,68.06651611)
\curveto(389.15258302,68.05651605)(389.13758304,68.04651606)(389.11759033,68.03651611)
\curveto(389.06758311,68.01651609)(389.00258317,68.0065161)(388.92259033,68.00651611)
\lineto(388.68259033,68.00651611)
\lineto(388.17259033,68.00651611)
\curveto(388.03258414,68.01651609)(387.90758427,68.06151604)(387.79759033,68.14151611)
\curveto(387.74758443,68.17151593)(387.70758447,68.2065159)(387.67759033,68.24651611)
\curveto(387.65758452,68.29651581)(387.63258454,68.34651576)(387.60259033,68.39651611)
\lineto(387.45259033,68.60651611)
\curveto(387.40258477,68.67651543)(387.35258482,68.75151535)(387.30259033,68.83151611)
\lineto(386.35759033,70.22651611)
\curveto(386.30758587,70.3065138)(386.25758592,70.38151372)(386.20759033,70.45151611)
\curveto(386.15758602,70.52151358)(386.10758607,70.59651351)(386.05759033,70.67651611)
\curveto(386.00758617,70.74651336)(385.95758622,70.8065133)(385.90759033,70.85651611)
\curveto(385.86758631,70.91651319)(385.80758637,70.95651315)(385.72759033,70.97651611)
\curveto(385.6775865,70.99651311)(385.62758655,70.98651312)(385.57759033,70.94651611)
\curveto(385.53758664,70.91651319)(385.50758667,70.89151321)(385.48759033,70.87151611)
\curveto(385.40758677,70.79151331)(385.33758684,70.7015134)(385.27759033,70.60151611)
\curveto(385.21758696,70.5015136)(385.15758702,70.4065137)(385.09759033,70.31651611)
\curveto(384.92758725,70.05651405)(384.75258742,69.79651431)(384.57259033,69.53651611)
\curveto(384.40258777,69.28651482)(384.22758795,69.03651507)(384.04759033,68.78651611)
\curveto(383.99758818,68.7065154)(383.94258823,68.62651548)(383.88259033,68.54651611)
\lineto(383.73259033,68.30651611)
\curveto(383.71258846,68.27651583)(383.68758849,68.24151586)(383.65759033,68.20151611)
\curveto(383.63758854,68.17151593)(383.61258856,68.14651596)(383.58259033,68.12651611)
\curveto(383.48258869,68.05651605)(383.36258881,68.01651609)(383.22259033,68.00651611)
\lineto(382.77259033,68.00651611)
\lineto(382.54759033,68.00651611)
\curveto(382.4775897,68.0065161)(382.41758976,68.01651609)(382.36759033,68.03651611)
\curveto(382.33758984,68.05651605)(382.31258986,68.07151603)(382.29259033,68.08151611)
\curveto(382.28258989,68.101516)(382.26758991,68.12151598)(382.24759033,68.14151611)
\curveto(382.23758994,68.25151585)(382.25258992,68.33651577)(382.29259033,68.39651611)
\curveto(382.34258983,68.45651565)(382.39258978,68.52151558)(382.44259033,68.59151611)
\curveto(382.52258965,68.7015154)(382.59758958,68.8015153)(382.66759033,68.89151611)
\curveto(382.73758944,68.99151511)(382.80758937,69.09651501)(382.87759033,69.20651611)
\curveto(383.09758908,69.5065146)(383.31258886,69.8065143)(383.52259033,70.10651611)
\lineto(384.15259033,71.00651611)
\curveto(384.22258795,71.09651301)(384.28758789,71.18651292)(384.34759033,71.27651611)
\curveto(384.41758776,71.36651274)(384.48258769,71.46151264)(384.54259033,71.56151611)
\curveto(384.59258758,71.63151247)(384.64258753,71.69651241)(384.69259033,71.75651611)
\curveto(384.74258743,71.82651228)(384.7775874,71.91651219)(384.79759033,72.02651611)
\curveto(384.81758736,72.07651203)(384.81258736,72.12651198)(384.78259033,72.17651611)
\curveto(384.76258741,72.22651188)(384.74258743,72.26651184)(384.72259033,72.29651611)
\curveto(384.6725875,72.38651172)(384.61758756,72.47151163)(384.55759033,72.55151611)
\lineto(384.37759033,72.79151611)
\curveto(384.14758803,73.11151099)(383.91258826,73.43151067)(383.67259033,73.75151611)
\lineto(382.98259033,74.71151611)
\curveto(382.90258927,74.82150928)(382.82258935,74.92150918)(382.74259033,75.01151611)
\curveto(382.6725895,75.101509)(382.60258957,75.2015089)(382.53259033,75.31151611)
\curveto(382.51258966,75.34150876)(382.49258968,75.38150872)(382.47259033,75.43151611)
\curveto(382.45258972,75.49150861)(382.45258972,75.54150856)(382.47259033,75.58151611)
\curveto(382.49258968,75.63150847)(382.52258965,75.66150844)(382.56259033,75.67151611)
\curveto(382.60258957,75.69150841)(382.64758953,75.7065084)(382.69759033,75.71651611)
\curveto(382.71758946,75.72650838)(382.73258944,75.72650838)(382.74259033,75.71651611)
\curveto(382.75258942,75.71650839)(382.76258941,75.72150838)(382.77259033,75.73151611)
}
}
{
\newrgbcolor{curcolor}{0 0 0}
\pscustom[linestyle=none,fillstyle=solid,fillcolor=curcolor]
{
\newpath
\moveto(390.80626221,77.23151611)
\curveto(390.72626109,77.29150681)(390.68126113,77.39650671)(390.67126221,77.54651611)
\lineto(390.67126221,78.01151611)
\lineto(390.67126221,78.26651611)
\curveto(390.67126114,78.35650575)(390.68626113,78.43150567)(390.71626221,78.49151611)
\curveto(390.75626106,78.57150553)(390.83626098,78.63150547)(390.95626221,78.67151611)
\curveto(390.97626084,78.68150542)(390.99626082,78.68150542)(391.01626221,78.67151611)
\curveto(391.04626077,78.67150543)(391.07126074,78.67650543)(391.09126221,78.68651611)
\curveto(391.26126055,78.68650542)(391.42126039,78.68150542)(391.57126221,78.67151611)
\curveto(391.72126009,78.66150544)(391.82125999,78.6015055)(391.87126221,78.49151611)
\curveto(391.90125991,78.43150567)(391.9162599,78.35650575)(391.91626221,78.26651611)
\lineto(391.91626221,78.01151611)
\curveto(391.9162599,77.83150627)(391.9112599,77.66150644)(391.90126221,77.50151611)
\curveto(391.90125991,77.34150676)(391.83625998,77.23650687)(391.70626221,77.18651611)
\curveto(391.65626016,77.16650694)(391.60126021,77.15650695)(391.54126221,77.15651611)
\lineto(391.37626221,77.15651611)
\lineto(391.06126221,77.15651611)
\curveto(390.96126085,77.15650695)(390.87626094,77.18150692)(390.80626221,77.23151611)
\moveto(391.91626221,68.72651611)
\lineto(391.91626221,68.41151611)
\curveto(391.92625989,68.31151579)(391.90625991,68.23151587)(391.85626221,68.17151611)
\curveto(391.82625999,68.11151599)(391.78126003,68.07151603)(391.72126221,68.05151611)
\curveto(391.66126015,68.04151606)(391.59126022,68.02651608)(391.51126221,68.00651611)
\lineto(391.28626221,68.00651611)
\curveto(391.15626066,68.0065161)(391.04126077,68.01151609)(390.94126221,68.02151611)
\curveto(390.85126096,68.04151606)(390.78126103,68.09151601)(390.73126221,68.17151611)
\curveto(390.69126112,68.23151587)(390.67126114,68.3065158)(390.67126221,68.39651611)
\lineto(390.67126221,68.68151611)
\lineto(390.67126221,75.02651611)
\lineto(390.67126221,75.34151611)
\curveto(390.67126114,75.45150865)(390.69626112,75.53650857)(390.74626221,75.59651611)
\curveto(390.77626104,75.64650846)(390.816261,75.67650843)(390.86626221,75.68651611)
\curveto(390.9162609,75.69650841)(390.97126084,75.71150839)(391.03126221,75.73151611)
\curveto(391.05126076,75.73150837)(391.07126074,75.72650838)(391.09126221,75.71651611)
\curveto(391.12126069,75.71650839)(391.14626067,75.72150838)(391.16626221,75.73151611)
\curveto(391.29626052,75.73150837)(391.42626039,75.72650838)(391.55626221,75.71651611)
\curveto(391.69626012,75.71650839)(391.79126002,75.67650843)(391.84126221,75.59651611)
\curveto(391.89125992,75.53650857)(391.9162599,75.45650865)(391.91626221,75.35651611)
\lineto(391.91626221,75.07151611)
\lineto(391.91626221,68.72651611)
}
}
{
\newrgbcolor{curcolor}{0 0 0}
\pscustom[linestyle=none,fillstyle=solid,fillcolor=curcolor]
{
\newpath
\moveto(394.43110596,78.68651611)
\curveto(394.56110434,78.68650542)(394.69610421,78.68650542)(394.83610596,78.68651611)
\curveto(394.98610392,78.68650542)(395.09610381,78.65150545)(395.16610596,78.58151611)
\curveto(395.21610369,78.51150559)(395.24110366,78.41650569)(395.24110596,78.29651611)
\curveto(395.25110365,78.18650592)(395.25610365,78.07150603)(395.25610596,77.95151611)
\lineto(395.25610596,76.61651611)
\lineto(395.25610596,70.54151611)
\lineto(395.25610596,68.86151611)
\lineto(395.25610596,68.47151611)
\curveto(395.25610365,68.33151577)(395.23110367,68.22151588)(395.18110596,68.14151611)
\curveto(395.15110375,68.09151601)(395.1061038,68.06151604)(395.04610596,68.05151611)
\curveto(394.99610391,68.04151606)(394.93110397,68.02651608)(394.85110596,68.00651611)
\lineto(394.64110596,68.00651611)
\lineto(394.32610596,68.00651611)
\curveto(394.22610468,68.01651609)(394.15110475,68.05151605)(394.10110596,68.11151611)
\curveto(394.05110485,68.19151591)(394.02110488,68.29151581)(394.01110596,68.41151611)
\lineto(394.01110596,68.78651611)
\lineto(394.01110596,70.16651611)
\lineto(394.01110596,76.40651611)
\lineto(394.01110596,77.87651611)
\curveto(394.01110489,77.98650612)(394.0061049,78.101506)(393.99610596,78.22151611)
\curveto(393.99610491,78.35150575)(394.02110488,78.45150565)(394.07110596,78.52151611)
\curveto(394.11110479,78.58150552)(394.18610472,78.63150547)(394.29610596,78.67151611)
\curveto(394.31610459,78.68150542)(394.33610457,78.68150542)(394.35610596,78.67151611)
\curveto(394.38610452,78.67150543)(394.41110449,78.67650543)(394.43110596,78.68651611)
}
}
{
\newrgbcolor{curcolor}{0 0 0}
\pscustom[linestyle=none,fillstyle=solid,fillcolor=curcolor]
{
\newpath
\moveto(397.48594971,77.23151611)
\curveto(397.40594859,77.29150681)(397.36094863,77.39650671)(397.35094971,77.54651611)
\lineto(397.35094971,78.01151611)
\lineto(397.35094971,78.26651611)
\curveto(397.35094864,78.35650575)(397.36594863,78.43150567)(397.39594971,78.49151611)
\curveto(397.43594856,78.57150553)(397.51594848,78.63150547)(397.63594971,78.67151611)
\curveto(397.65594834,78.68150542)(397.67594832,78.68150542)(397.69594971,78.67151611)
\curveto(397.72594827,78.67150543)(397.75094824,78.67650543)(397.77094971,78.68651611)
\curveto(397.94094805,78.68650542)(398.10094789,78.68150542)(398.25094971,78.67151611)
\curveto(398.40094759,78.66150544)(398.50094749,78.6015055)(398.55094971,78.49151611)
\curveto(398.58094741,78.43150567)(398.5959474,78.35650575)(398.59594971,78.26651611)
\lineto(398.59594971,78.01151611)
\curveto(398.5959474,77.83150627)(398.5909474,77.66150644)(398.58094971,77.50151611)
\curveto(398.58094741,77.34150676)(398.51594748,77.23650687)(398.38594971,77.18651611)
\curveto(398.33594766,77.16650694)(398.28094771,77.15650695)(398.22094971,77.15651611)
\lineto(398.05594971,77.15651611)
\lineto(397.74094971,77.15651611)
\curveto(397.64094835,77.15650695)(397.55594844,77.18150692)(397.48594971,77.23151611)
\moveto(398.59594971,68.72651611)
\lineto(398.59594971,68.41151611)
\curveto(398.60594739,68.31151579)(398.58594741,68.23151587)(398.53594971,68.17151611)
\curveto(398.50594749,68.11151599)(398.46094753,68.07151603)(398.40094971,68.05151611)
\curveto(398.34094765,68.04151606)(398.27094772,68.02651608)(398.19094971,68.00651611)
\lineto(397.96594971,68.00651611)
\curveto(397.83594816,68.0065161)(397.72094827,68.01151609)(397.62094971,68.02151611)
\curveto(397.53094846,68.04151606)(397.46094853,68.09151601)(397.41094971,68.17151611)
\curveto(397.37094862,68.23151587)(397.35094864,68.3065158)(397.35094971,68.39651611)
\lineto(397.35094971,68.68151611)
\lineto(397.35094971,75.02651611)
\lineto(397.35094971,75.34151611)
\curveto(397.35094864,75.45150865)(397.37594862,75.53650857)(397.42594971,75.59651611)
\curveto(397.45594854,75.64650846)(397.4959485,75.67650843)(397.54594971,75.68651611)
\curveto(397.5959484,75.69650841)(397.65094834,75.71150839)(397.71094971,75.73151611)
\curveto(397.73094826,75.73150837)(397.75094824,75.72650838)(397.77094971,75.71651611)
\curveto(397.80094819,75.71650839)(397.82594817,75.72150838)(397.84594971,75.73151611)
\curveto(397.97594802,75.73150837)(398.10594789,75.72650838)(398.23594971,75.71651611)
\curveto(398.37594762,75.71650839)(398.47094752,75.67650843)(398.52094971,75.59651611)
\curveto(398.57094742,75.53650857)(398.5959474,75.45650865)(398.59594971,75.35651611)
\lineto(398.59594971,75.07151611)
\lineto(398.59594971,68.72651611)
}
}
{
\newrgbcolor{curcolor}{0 0 0}
\pscustom[linestyle=none,fillstyle=solid,fillcolor=curcolor]
{
\newpath
\moveto(407.42579346,68.56151611)
\curveto(407.45578563,68.4015157)(407.44078564,68.26651584)(407.38079346,68.15651611)
\curveto(407.32078576,68.05651605)(407.24078584,67.98151612)(407.14079346,67.93151611)
\curveto(407.09078599,67.91151619)(407.03578605,67.9015162)(406.97579346,67.90151611)
\curveto(406.92578616,67.9015162)(406.87078621,67.89151621)(406.81079346,67.87151611)
\curveto(406.59078649,67.82151628)(406.37078671,67.83651627)(406.15079346,67.91651611)
\curveto(405.94078714,67.98651612)(405.79578729,68.07651603)(405.71579346,68.18651611)
\curveto(405.66578742,68.25651585)(405.62078746,68.33651577)(405.58079346,68.42651611)
\curveto(405.54078754,68.52651558)(405.49078759,68.6065155)(405.43079346,68.66651611)
\curveto(405.41078767,68.68651542)(405.3857877,68.7065154)(405.35579346,68.72651611)
\curveto(405.33578775,68.74651536)(405.30578778,68.75151535)(405.26579346,68.74151611)
\curveto(405.15578793,68.71151539)(405.05078803,68.65651545)(404.95079346,68.57651611)
\curveto(404.86078822,68.49651561)(404.77078831,68.42651568)(404.68079346,68.36651611)
\curveto(404.55078853,68.28651582)(404.41078867,68.21151589)(404.26079346,68.14151611)
\curveto(404.11078897,68.08151602)(403.95078913,68.02651608)(403.78079346,67.97651611)
\curveto(403.6807894,67.94651616)(403.57078951,67.92651618)(403.45079346,67.91651611)
\curveto(403.34078974,67.9065162)(403.23078985,67.89151621)(403.12079346,67.87151611)
\curveto(403.07079001,67.86151624)(403.02579006,67.85651625)(402.98579346,67.85651611)
\lineto(402.88079346,67.85651611)
\curveto(402.77079031,67.83651627)(402.66579042,67.83651627)(402.56579346,67.85651611)
\lineto(402.43079346,67.85651611)
\curveto(402.3807907,67.86651624)(402.33079075,67.87151623)(402.28079346,67.87151611)
\curveto(402.23079085,67.87151623)(402.1857909,67.88151622)(402.14579346,67.90151611)
\curveto(402.10579098,67.91151619)(402.07079101,67.91651619)(402.04079346,67.91651611)
\curveto(402.02079106,67.9065162)(401.99579109,67.9065162)(401.96579346,67.91651611)
\lineto(401.72579346,67.97651611)
\curveto(401.64579144,67.98651612)(401.57079151,68.0065161)(401.50079346,68.03651611)
\curveto(401.20079188,68.16651594)(400.95579213,68.31151579)(400.76579346,68.47151611)
\curveto(400.5857925,68.64151546)(400.43579265,68.87651523)(400.31579346,69.17651611)
\curveto(400.22579286,69.39651471)(400.1807929,69.66151444)(400.18079346,69.97151611)
\lineto(400.18079346,70.28651611)
\curveto(400.19079289,70.33651377)(400.19579289,70.38651372)(400.19579346,70.43651611)
\lineto(400.22579346,70.61651611)
\lineto(400.34579346,70.94651611)
\curveto(400.3857927,71.05651305)(400.43579265,71.15651295)(400.49579346,71.24651611)
\curveto(400.67579241,71.53651257)(400.92079216,71.75151235)(401.23079346,71.89151611)
\curveto(401.54079154,72.03151207)(401.8807912,72.15651195)(402.25079346,72.26651611)
\curveto(402.39079069,72.3065118)(402.53579055,72.33651177)(402.68579346,72.35651611)
\curveto(402.83579025,72.37651173)(402.9857901,72.4015117)(403.13579346,72.43151611)
\curveto(403.20578988,72.45151165)(403.27078981,72.46151164)(403.33079346,72.46151611)
\curveto(403.40078968,72.46151164)(403.47578961,72.47151163)(403.55579346,72.49151611)
\curveto(403.62578946,72.51151159)(403.69578939,72.52151158)(403.76579346,72.52151611)
\curveto(403.83578925,72.53151157)(403.91078917,72.54651156)(403.99079346,72.56651611)
\curveto(404.24078884,72.62651148)(404.47578861,72.67651143)(404.69579346,72.71651611)
\curveto(404.91578817,72.76651134)(405.09078799,72.88151122)(405.22079346,73.06151611)
\curveto(405.2807878,73.14151096)(405.33078775,73.24151086)(405.37079346,73.36151611)
\curveto(405.41078767,73.49151061)(405.41078767,73.63151047)(405.37079346,73.78151611)
\curveto(405.31078777,74.02151008)(405.22078786,74.21150989)(405.10079346,74.35151611)
\curveto(404.99078809,74.49150961)(404.83078825,74.6015095)(404.62079346,74.68151611)
\curveto(404.50078858,74.73150937)(404.35578873,74.76650934)(404.18579346,74.78651611)
\curveto(404.02578906,74.8065093)(403.85578923,74.81650929)(403.67579346,74.81651611)
\curveto(403.49578959,74.81650929)(403.32078976,74.8065093)(403.15079346,74.78651611)
\curveto(402.9807901,74.76650934)(402.83579025,74.73650937)(402.71579346,74.69651611)
\curveto(402.54579054,74.63650947)(402.3807907,74.55150955)(402.22079346,74.44151611)
\curveto(402.14079094,74.38150972)(402.06579102,74.3015098)(401.99579346,74.20151611)
\curveto(401.93579115,74.11150999)(401.8807912,74.01151009)(401.83079346,73.90151611)
\curveto(401.80079128,73.82151028)(401.77079131,73.73651037)(401.74079346,73.64651611)
\curveto(401.72079136,73.55651055)(401.67579141,73.48651062)(401.60579346,73.43651611)
\curveto(401.56579152,73.4065107)(401.49579159,73.38151072)(401.39579346,73.36151611)
\curveto(401.30579178,73.35151075)(401.21079187,73.34651076)(401.11079346,73.34651611)
\curveto(401.01079207,73.34651076)(400.91079217,73.35151075)(400.81079346,73.36151611)
\curveto(400.72079236,73.38151072)(400.65579243,73.4065107)(400.61579346,73.43651611)
\curveto(400.57579251,73.46651064)(400.54579254,73.51651059)(400.52579346,73.58651611)
\curveto(400.50579258,73.65651045)(400.50579258,73.73151037)(400.52579346,73.81151611)
\curveto(400.55579253,73.94151016)(400.5857925,74.06151004)(400.61579346,74.17151611)
\curveto(400.65579243,74.29150981)(400.70079238,74.4065097)(400.75079346,74.51651611)
\curveto(400.94079214,74.86650924)(401.1807919,75.13650897)(401.47079346,75.32651611)
\curveto(401.76079132,75.52650858)(402.12079096,75.68650842)(402.55079346,75.80651611)
\curveto(402.65079043,75.82650828)(402.75079033,75.84150826)(402.85079346,75.85151611)
\curveto(402.96079012,75.86150824)(403.07079001,75.87650823)(403.18079346,75.89651611)
\curveto(403.22078986,75.9065082)(403.2857898,75.9065082)(403.37579346,75.89651611)
\curveto(403.46578962,75.89650821)(403.52078956,75.9065082)(403.54079346,75.92651611)
\curveto(404.24078884,75.93650817)(404.85078823,75.85650825)(405.37079346,75.68651611)
\curveto(405.89078719,75.51650859)(406.25578683,75.19150891)(406.46579346,74.71151611)
\curveto(406.55578653,74.51150959)(406.60578648,74.27650983)(406.61579346,74.00651611)
\curveto(406.63578645,73.74651036)(406.64578644,73.47151063)(406.64579346,73.18151611)
\lineto(406.64579346,69.86651611)
\curveto(406.64578644,69.72651438)(406.65078643,69.59151451)(406.66079346,69.46151611)
\curveto(406.67078641,69.33151477)(406.70078638,69.22651488)(406.75079346,69.14651611)
\curveto(406.80078628,69.07651503)(406.86578622,69.02651508)(406.94579346,68.99651611)
\curveto(407.03578605,68.95651515)(407.12078596,68.92651518)(407.20079346,68.90651611)
\curveto(407.2807858,68.89651521)(407.34078574,68.85151525)(407.38079346,68.77151611)
\curveto(407.40078568,68.74151536)(407.41078567,68.71151539)(407.41079346,68.68151611)
\curveto(407.41078567,68.65151545)(407.41578567,68.61151549)(407.42579346,68.56151611)
\moveto(405.28079346,70.22651611)
\curveto(405.34078774,70.36651374)(405.37078771,70.52651358)(405.37079346,70.70651611)
\curveto(405.3807877,70.89651321)(405.3857877,71.09151301)(405.38579346,71.29151611)
\curveto(405.3857877,71.4015127)(405.3807877,71.5015126)(405.37079346,71.59151611)
\curveto(405.36078772,71.68151242)(405.32078776,71.75151235)(405.25079346,71.80151611)
\curveto(405.22078786,71.82151228)(405.15078793,71.83151227)(405.04079346,71.83151611)
\curveto(405.02078806,71.81151229)(404.9857881,71.8015123)(404.93579346,71.80151611)
\curveto(404.8857882,71.8015123)(404.84078824,71.79151231)(404.80079346,71.77151611)
\curveto(404.72078836,71.75151235)(404.63078845,71.73151237)(404.53079346,71.71151611)
\lineto(404.23079346,71.65151611)
\curveto(404.20078888,71.65151245)(404.16578892,71.64651246)(404.12579346,71.63651611)
\lineto(404.02079346,71.63651611)
\curveto(403.87078921,71.59651251)(403.70578938,71.57151253)(403.52579346,71.56151611)
\curveto(403.35578973,71.56151254)(403.19578989,71.54151256)(403.04579346,71.50151611)
\curveto(402.96579012,71.48151262)(402.89079019,71.46151264)(402.82079346,71.44151611)
\curveto(402.76079032,71.43151267)(402.69079039,71.41651269)(402.61079346,71.39651611)
\curveto(402.45079063,71.34651276)(402.30079078,71.28151282)(402.16079346,71.20151611)
\curveto(402.02079106,71.13151297)(401.90079118,71.04151306)(401.80079346,70.93151611)
\curveto(401.70079138,70.82151328)(401.62579146,70.68651342)(401.57579346,70.52651611)
\curveto(401.52579156,70.37651373)(401.50579158,70.19151391)(401.51579346,69.97151611)
\curveto(401.51579157,69.87151423)(401.53079155,69.77651433)(401.56079346,69.68651611)
\curveto(401.60079148,69.6065145)(401.64579144,69.53151457)(401.69579346,69.46151611)
\curveto(401.77579131,69.35151475)(401.8807912,69.25651485)(402.01079346,69.17651611)
\curveto(402.14079094,69.106515)(402.2807908,69.04651506)(402.43079346,68.99651611)
\curveto(402.4807906,68.98651512)(402.53079055,68.98151512)(402.58079346,68.98151611)
\curveto(402.63079045,68.98151512)(402.6807904,68.97651513)(402.73079346,68.96651611)
\curveto(402.80079028,68.94651516)(402.8857902,68.93151517)(402.98579346,68.92151611)
\curveto(403.09578999,68.92151518)(403.1857899,68.93151517)(403.25579346,68.95151611)
\curveto(403.31578977,68.97151513)(403.37578971,68.97651513)(403.43579346,68.96651611)
\curveto(403.49578959,68.96651514)(403.55578953,68.97651513)(403.61579346,68.99651611)
\curveto(403.69578939,69.01651509)(403.77078931,69.03151507)(403.84079346,69.04151611)
\curveto(403.92078916,69.05151505)(403.99578909,69.07151503)(404.06579346,69.10151611)
\curveto(404.35578873,69.22151488)(404.60078848,69.36651474)(404.80079346,69.53651611)
\curveto(405.01078807,69.7065144)(405.17078791,69.93651417)(405.28079346,70.22651611)
}
}
{
\newrgbcolor{curcolor}{0 0 0}
\pscustom[linestyle=none,fillstyle=solid,fillcolor=curcolor]
{
\newpath
\moveto(412.24243408,75.91151611)
\curveto(412.47242929,75.91150819)(412.60242916,75.85150825)(412.63243408,75.73151611)
\curveto(412.6624291,75.62150848)(412.67742909,75.45650865)(412.67743408,75.23651611)
\lineto(412.67743408,74.95151611)
\curveto(412.67742909,74.86150924)(412.65242911,74.78650932)(412.60243408,74.72651611)
\curveto(412.54242922,74.64650946)(412.45742931,74.6015095)(412.34743408,74.59151611)
\curveto(412.23742953,74.59150951)(412.12742964,74.57650953)(412.01743408,74.54651611)
\curveto(411.87742989,74.51650959)(411.74243002,74.48650962)(411.61243408,74.45651611)
\curveto(411.49243027,74.42650968)(411.37743039,74.38650972)(411.26743408,74.33651611)
\curveto(410.97743079,74.2065099)(410.74243102,74.02651008)(410.56243408,73.79651611)
\curveto(410.38243138,73.57651053)(410.22743154,73.32151078)(410.09743408,73.03151611)
\curveto(410.05743171,72.92151118)(410.02743174,72.8065113)(410.00743408,72.68651611)
\curveto(409.98743178,72.57651153)(409.9624318,72.46151164)(409.93243408,72.34151611)
\curveto(409.92243184,72.29151181)(409.91743185,72.24151186)(409.91743408,72.19151611)
\curveto(409.92743184,72.14151196)(409.92743184,72.09151201)(409.91743408,72.04151611)
\curveto(409.88743188,71.92151218)(409.87243189,71.78151232)(409.87243408,71.62151611)
\curveto(409.88243188,71.47151263)(409.88743188,71.32651278)(409.88743408,71.18651611)
\lineto(409.88743408,69.34151611)
\lineto(409.88743408,68.99651611)
\curveto(409.88743188,68.87651523)(409.88243188,68.76151534)(409.87243408,68.65151611)
\curveto(409.8624319,68.54151556)(409.85743191,68.44651566)(409.85743408,68.36651611)
\curveto(409.8674319,68.28651582)(409.84743192,68.21651589)(409.79743408,68.15651611)
\curveto(409.74743202,68.08651602)(409.6674321,68.04651606)(409.55743408,68.03651611)
\curveto(409.45743231,68.02651608)(409.34743242,68.02151608)(409.22743408,68.02151611)
\lineto(408.95743408,68.02151611)
\curveto(408.90743286,68.04151606)(408.85743291,68.05651605)(408.80743408,68.06651611)
\curveto(408.767433,68.08651602)(408.73743303,68.11151599)(408.71743408,68.14151611)
\curveto(408.6674331,68.21151589)(408.63743313,68.29651581)(408.62743408,68.39651611)
\lineto(408.62743408,68.72651611)
\lineto(408.62743408,69.88151611)
\lineto(408.62743408,74.03651611)
\lineto(408.62743408,75.07151611)
\lineto(408.62743408,75.37151611)
\curveto(408.63743313,75.47150863)(408.6674331,75.55650855)(408.71743408,75.62651611)
\curveto(408.74743302,75.66650844)(408.79743297,75.69650841)(408.86743408,75.71651611)
\curveto(408.94743282,75.73650837)(409.03243273,75.74650836)(409.12243408,75.74651611)
\curveto(409.21243255,75.75650835)(409.30243246,75.75650835)(409.39243408,75.74651611)
\curveto(409.48243228,75.73650837)(409.55243221,75.72150838)(409.60243408,75.70151611)
\curveto(409.68243208,75.67150843)(409.73243203,75.61150849)(409.75243408,75.52151611)
\curveto(409.78243198,75.44150866)(409.79743197,75.35150875)(409.79743408,75.25151611)
\lineto(409.79743408,74.95151611)
\curveto(409.79743197,74.85150925)(409.81743195,74.76150934)(409.85743408,74.68151611)
\curveto(409.8674319,74.66150944)(409.87743189,74.64650946)(409.88743408,74.63651611)
\lineto(409.93243408,74.59151611)
\curveto(410.04243172,74.59150951)(410.13243163,74.63650947)(410.20243408,74.72651611)
\curveto(410.27243149,74.82650928)(410.33243143,74.9065092)(410.38243408,74.96651611)
\lineto(410.47243408,75.05651611)
\curveto(410.5624312,75.16650894)(410.68743108,75.28150882)(410.84743408,75.40151611)
\curveto(411.00743076,75.52150858)(411.15743061,75.61150849)(411.29743408,75.67151611)
\curveto(411.38743038,75.72150838)(411.48243028,75.75650835)(411.58243408,75.77651611)
\curveto(411.68243008,75.8065083)(411.78742998,75.83650827)(411.89743408,75.86651611)
\curveto(411.95742981,75.87650823)(412.01742975,75.88150822)(412.07743408,75.88151611)
\curveto(412.13742963,75.89150821)(412.19242957,75.9015082)(412.24243408,75.91151611)
}
}
{
\newrgbcolor{curcolor}{0 0 0}
\pscustom[linestyle=none,fillstyle=solid,fillcolor=curcolor]
{
\newpath
\moveto(536.28160645,78.68651611)
\lineto(537.19660645,78.68651611)
\curveto(537.2966038,78.68650542)(537.3916037,78.68650542)(537.48160645,78.68651611)
\curveto(537.57160352,78.68650542)(537.64660345,78.66650544)(537.70660645,78.62651611)
\curveto(537.7966033,78.56650554)(537.85660324,78.48650562)(537.88660645,78.38651611)
\curveto(537.92660317,78.28650582)(537.97160312,78.18150592)(538.02160645,78.07151611)
\curveto(538.10160299,77.88150622)(538.17160292,77.69150641)(538.23160645,77.50151611)
\curveto(538.30160279,77.31150679)(538.37660272,77.12150698)(538.45660645,76.93151611)
\curveto(538.52660257,76.75150735)(538.5916025,76.56650754)(538.65160645,76.37651611)
\curveto(538.71160238,76.19650791)(538.78160231,76.01650809)(538.86160645,75.83651611)
\curveto(538.92160217,75.69650841)(538.97660212,75.55150855)(539.02660645,75.40151611)
\curveto(539.07660202,75.25150885)(539.13160196,75.106509)(539.19160645,74.96651611)
\curveto(539.37160172,74.51650959)(539.54160155,74.06151004)(539.70160645,73.60151611)
\curveto(539.86160123,73.15151095)(540.03160106,72.7015114)(540.21160645,72.25151611)
\curveto(540.23160086,72.2015119)(540.24660085,72.15151195)(540.25660645,72.10151611)
\lineto(540.31660645,71.95151611)
\curveto(540.40660069,71.73151237)(540.4916006,71.5065126)(540.57160645,71.27651611)
\curveto(540.65160044,71.05651305)(540.73660036,70.83651327)(540.82660645,70.61651611)
\curveto(540.86660023,70.52651358)(540.90660019,70.41651369)(540.94660645,70.28651611)
\curveto(540.98660011,70.16651394)(541.05160004,70.09651401)(541.14160645,70.07651611)
\curveto(541.18159991,70.06651404)(541.21159988,70.06651404)(541.23160645,70.07651611)
\lineto(541.29160645,70.13651611)
\curveto(541.34159975,70.18651392)(541.37659972,70.24151386)(541.39660645,70.30151611)
\curveto(541.42659967,70.36151374)(541.45659964,70.42651368)(541.48660645,70.49651611)
\lineto(541.72660645,71.12651611)
\curveto(541.80659929,71.34651276)(541.88659921,71.56151254)(541.96660645,71.77151611)
\lineto(542.02660645,71.92151611)
\lineto(542.08660645,72.10151611)
\curveto(542.16659893,72.29151181)(542.23659886,72.48151162)(542.29660645,72.67151611)
\curveto(542.36659873,72.87151123)(542.44159865,73.07151103)(542.52160645,73.27151611)
\curveto(542.76159833,73.85151025)(542.98159811,74.43650967)(543.18160645,75.02651611)
\curveto(543.3915977,75.61650849)(543.61659748,76.2015079)(543.85660645,76.78151611)
\curveto(543.93659716,76.98150712)(544.01159708,77.18650692)(544.08160645,77.39651611)
\curveto(544.16159693,77.6065065)(544.24159685,77.81150629)(544.32160645,78.01151611)
\curveto(544.36159673,78.09150601)(544.3965967,78.19150591)(544.42660645,78.31151611)
\curveto(544.46659663,78.43150567)(544.52159657,78.51650559)(544.59160645,78.56651611)
\curveto(544.65159644,78.6065055)(544.72659637,78.63650547)(544.81660645,78.65651611)
\curveto(544.91659618,78.67650543)(545.02659607,78.68650542)(545.14660645,78.68651611)
\curveto(545.26659583,78.69650541)(545.38659571,78.69650541)(545.50660645,78.68651611)
\curveto(545.62659547,78.68650542)(545.73659536,78.68650542)(545.83660645,78.68651611)
\curveto(545.92659517,78.68650542)(546.01659508,78.68650542)(546.10660645,78.68651611)
\curveto(546.20659489,78.68650542)(546.28159481,78.66650544)(546.33160645,78.62651611)
\curveto(546.42159467,78.57650553)(546.47159462,78.48650562)(546.48160645,78.35651611)
\curveto(546.4915946,78.22650588)(546.4965946,78.08650602)(546.49660645,77.93651611)
\lineto(546.49660645,76.28651611)
\lineto(546.49660645,70.01651611)
\lineto(546.49660645,68.75651611)
\curveto(546.4965946,68.64651546)(546.4965946,68.53651557)(546.49660645,68.42651611)
\curveto(546.50659459,68.31651579)(546.48659461,68.23151587)(546.43660645,68.17151611)
\curveto(546.40659469,68.11151599)(546.36159473,68.07151603)(546.30160645,68.05151611)
\curveto(546.24159485,68.04151606)(546.17159492,68.02651608)(546.09160645,68.00651611)
\lineto(545.85160645,68.00651611)
\lineto(545.49160645,68.00651611)
\curveto(545.38159571,68.01651609)(545.30159579,68.06151604)(545.25160645,68.14151611)
\curveto(545.23159586,68.17151593)(545.21659588,68.2015159)(545.20660645,68.23151611)
\curveto(545.20659589,68.27151583)(545.1965959,68.31651579)(545.17660645,68.36651611)
\lineto(545.17660645,68.53151611)
\curveto(545.16659593,68.59151551)(545.16159593,68.66151544)(545.16160645,68.74151611)
\curveto(545.17159592,68.82151528)(545.17659592,68.89651521)(545.17660645,68.96651611)
\lineto(545.17660645,69.80651611)
\lineto(545.17660645,74.23151611)
\curveto(545.17659592,74.48150962)(545.17659592,74.73150937)(545.17660645,74.98151611)
\curveto(545.17659592,75.24150886)(545.17159592,75.49150861)(545.16160645,75.73151611)
\curveto(545.16159593,75.83150827)(545.15659594,75.94150816)(545.14660645,76.06151611)
\curveto(545.13659596,76.18150792)(545.08159601,76.24150786)(544.98160645,76.24151611)
\lineto(544.98160645,76.22651611)
\curveto(544.91159618,76.2065079)(544.85159624,76.14150796)(544.80160645,76.03151611)
\curveto(544.76159633,75.92150818)(544.72659637,75.82650828)(544.69660645,75.74651611)
\curveto(544.62659647,75.57650853)(544.56159653,75.4015087)(544.50160645,75.22151611)
\curveto(544.44159665,75.05150905)(544.37159672,74.88150922)(544.29160645,74.71151611)
\curveto(544.27159682,74.66150944)(544.25659684,74.61650949)(544.24660645,74.57651611)
\curveto(544.23659686,74.53650957)(544.22159687,74.49150961)(544.20160645,74.44151611)
\curveto(544.12159697,74.26150984)(544.05159704,74.07651003)(543.99160645,73.88651611)
\curveto(543.94159715,73.7065104)(543.87659722,73.52651058)(543.79660645,73.34651611)
\curveto(543.72659737,73.19651091)(543.66659743,73.04151106)(543.61660645,72.88151611)
\curveto(543.56659753,72.73151137)(543.51159758,72.58151152)(543.45160645,72.43151611)
\curveto(543.25159784,71.96151214)(543.07159802,71.48651262)(542.91160645,71.00651611)
\curveto(542.75159834,70.53651357)(542.57659852,70.07151403)(542.38660645,69.61151611)
\curveto(542.30659879,69.43151467)(542.23659886,69.25151485)(542.17660645,69.07151611)
\curveto(542.11659898,68.89151521)(542.05159904,68.71151539)(541.98160645,68.53151611)
\curveto(541.93159916,68.42151568)(541.88159921,68.31651579)(541.83160645,68.21651611)
\curveto(541.7915993,68.12651598)(541.70659939,68.06151604)(541.57660645,68.02151611)
\curveto(541.55659954,68.01151609)(541.53159956,68.0065161)(541.50160645,68.00651611)
\curveto(541.48159961,68.01651609)(541.45659964,68.01651609)(541.42660645,68.00651611)
\curveto(541.3965997,67.99651611)(541.36159973,67.99151611)(541.32160645,67.99151611)
\curveto(541.28159981,68.0015161)(541.24159985,68.0065161)(541.20160645,68.00651611)
\lineto(540.90160645,68.00651611)
\curveto(540.80160029,68.0065161)(540.72160037,68.03151607)(540.66160645,68.08151611)
\curveto(540.58160051,68.13151597)(540.52160057,68.2015159)(540.48160645,68.29151611)
\curveto(540.45160064,68.39151571)(540.41160068,68.49151561)(540.36160645,68.59151611)
\curveto(540.28160081,68.79151531)(540.20160089,68.99651511)(540.12160645,69.20651611)
\curveto(540.05160104,69.42651468)(539.97660112,69.63651447)(539.89660645,69.83651611)
\curveto(539.81660128,70.01651409)(539.74660135,70.19651391)(539.68660645,70.37651611)
\curveto(539.63660146,70.56651354)(539.57160152,70.75151335)(539.49160645,70.93151611)
\curveto(539.26160183,71.49151261)(539.04660205,72.05651205)(538.84660645,72.62651611)
\curveto(538.64660245,73.19651091)(538.43160266,73.76151034)(538.20160645,74.32151611)
\lineto(537.96160645,74.95151611)
\curveto(537.8916032,75.17150893)(537.81660328,75.38150872)(537.73660645,75.58151611)
\curveto(537.68660341,75.69150841)(537.64160345,75.79650831)(537.60160645,75.89651611)
\curveto(537.57160352,76.0065081)(537.52160357,76.101508)(537.45160645,76.18151611)
\curveto(537.44160365,76.2015079)(537.43160366,76.21150789)(537.42160645,76.21151611)
\lineto(537.39160645,76.24151611)
\lineto(537.31660645,76.24151611)
\lineto(537.28660645,76.21151611)
\curveto(537.27660382,76.21150789)(537.26660383,76.2065079)(537.25660645,76.19651611)
\curveto(537.23660386,76.14650796)(537.22660387,76.09150801)(537.22660645,76.03151611)
\curveto(537.22660387,75.97150813)(537.21660388,75.91150819)(537.19660645,75.85151611)
\lineto(537.19660645,75.68651611)
\curveto(537.17660392,75.62650848)(537.17160392,75.56150854)(537.18160645,75.49151611)
\curveto(537.1916039,75.42150868)(537.1966039,75.35150875)(537.19660645,75.28151611)
\lineto(537.19660645,74.47151611)
\lineto(537.19660645,69.91151611)
\lineto(537.19660645,68.72651611)
\curveto(537.1966039,68.61651549)(537.1916039,68.5065156)(537.18160645,68.39651611)
\curveto(537.18160391,68.28651582)(537.15660394,68.2015159)(537.10660645,68.14151611)
\curveto(537.05660404,68.06151604)(536.96660413,68.01651609)(536.83660645,68.00651611)
\lineto(536.44660645,68.00651611)
\lineto(536.25160645,68.00651611)
\curveto(536.20160489,68.0065161)(536.15160494,68.01651609)(536.10160645,68.03651611)
\curveto(535.97160512,68.07651603)(535.8966052,68.16151594)(535.87660645,68.29151611)
\curveto(535.86660523,68.42151568)(535.86160523,68.57151553)(535.86160645,68.74151611)
\lineto(535.86160645,70.48151611)
\lineto(535.86160645,76.48151611)
\lineto(535.86160645,77.89151611)
\curveto(535.86160523,78.0015061)(535.85660524,78.11650599)(535.84660645,78.23651611)
\curveto(535.84660525,78.35650575)(535.87160522,78.45150565)(535.92160645,78.52151611)
\curveto(535.96160513,78.58150552)(536.03660506,78.63150547)(536.14660645,78.67151611)
\curveto(536.16660493,78.68150542)(536.18660491,78.68150542)(536.20660645,78.67151611)
\curveto(536.23660486,78.67150543)(536.26160483,78.67650543)(536.28160645,78.68651611)
}
}
{
\newrgbcolor{curcolor}{0 0 0}
\pscustom[linestyle=none,fillstyle=solid,fillcolor=curcolor]
{
\newpath
\moveto(555.72371582,72.20651611)
\curveto(555.74370776,72.14651196)(555.75370775,72.05151205)(555.75371582,71.92151611)
\curveto(555.75370775,71.8015123)(555.74870776,71.71651239)(555.73871582,71.66651611)
\lineto(555.73871582,71.51651611)
\curveto(555.72870778,71.43651267)(555.71870779,71.36151274)(555.70871582,71.29151611)
\curveto(555.7087078,71.23151287)(555.7037078,71.16151294)(555.69371582,71.08151611)
\curveto(555.67370783,71.02151308)(555.65870785,70.96151314)(555.64871582,70.90151611)
\curveto(555.64870786,70.84151326)(555.63870787,70.78151332)(555.61871582,70.72151611)
\curveto(555.57870793,70.59151351)(555.54370796,70.46151364)(555.51371582,70.33151611)
\curveto(555.48370802,70.2015139)(555.44370806,70.08151402)(555.39371582,69.97151611)
\curveto(555.18370832,69.49151461)(554.9037086,69.08651502)(554.55371582,68.75651611)
\curveto(554.2037093,68.43651567)(553.77370973,68.19151591)(553.26371582,68.02151611)
\curveto(553.15371035,67.98151612)(553.03371047,67.95151615)(552.90371582,67.93151611)
\curveto(552.78371072,67.91151619)(552.65871085,67.89151621)(552.52871582,67.87151611)
\curveto(552.46871104,67.86151624)(552.4037111,67.85651625)(552.33371582,67.85651611)
\curveto(552.27371123,67.84651626)(552.21371129,67.84151626)(552.15371582,67.84151611)
\curveto(552.11371139,67.83151627)(552.05371145,67.82651628)(551.97371582,67.82651611)
\curveto(551.9037116,67.82651628)(551.85371165,67.83151627)(551.82371582,67.84151611)
\curveto(551.78371172,67.85151625)(551.74371176,67.85651625)(551.70371582,67.85651611)
\curveto(551.66371184,67.84651626)(551.62871188,67.84651626)(551.59871582,67.85651611)
\lineto(551.50871582,67.85651611)
\lineto(551.14871582,67.90151611)
\curveto(551.0087125,67.94151616)(550.87371263,67.98151612)(550.74371582,68.02151611)
\curveto(550.61371289,68.06151604)(550.48871302,68.106516)(550.36871582,68.15651611)
\curveto(549.91871359,68.35651575)(549.54871396,68.61651549)(549.25871582,68.93651611)
\curveto(548.96871454,69.25651485)(548.72871478,69.64651446)(548.53871582,70.10651611)
\curveto(548.48871502,70.2065139)(548.44871506,70.3065138)(548.41871582,70.40651611)
\curveto(548.39871511,70.5065136)(548.37871513,70.61151349)(548.35871582,70.72151611)
\curveto(548.33871517,70.76151334)(548.32871518,70.79151331)(548.32871582,70.81151611)
\curveto(548.33871517,70.84151326)(548.33871517,70.87651323)(548.32871582,70.91651611)
\curveto(548.3087152,70.99651311)(548.29371521,71.07651303)(548.28371582,71.15651611)
\curveto(548.28371522,71.24651286)(548.27371523,71.33151277)(548.25371582,71.41151611)
\lineto(548.25371582,71.53151611)
\curveto(548.25371525,71.57151253)(548.24871526,71.61651249)(548.23871582,71.66651611)
\curveto(548.22871528,71.71651239)(548.22371528,71.8015123)(548.22371582,71.92151611)
\curveto(548.22371528,72.05151205)(548.23371527,72.14651196)(548.25371582,72.20651611)
\curveto(548.27371523,72.27651183)(548.27871523,72.34651176)(548.26871582,72.41651611)
\curveto(548.25871525,72.48651162)(548.26371524,72.55651155)(548.28371582,72.62651611)
\curveto(548.29371521,72.67651143)(548.29871521,72.71651139)(548.29871582,72.74651611)
\curveto(548.3087152,72.78651132)(548.31871519,72.83151127)(548.32871582,72.88151611)
\curveto(548.35871515,73.0015111)(548.38371512,73.12151098)(548.40371582,73.24151611)
\curveto(548.43371507,73.36151074)(548.47371503,73.47651063)(548.52371582,73.58651611)
\curveto(548.67371483,73.95651015)(548.85371465,74.28650982)(549.06371582,74.57651611)
\curveto(549.28371422,74.87650923)(549.54871396,75.12650898)(549.85871582,75.32651611)
\curveto(549.97871353,75.4065087)(550.1037134,75.47150863)(550.23371582,75.52151611)
\curveto(550.36371314,75.58150852)(550.49871301,75.64150846)(550.63871582,75.70151611)
\curveto(550.75871275,75.75150835)(550.88871262,75.78150832)(551.02871582,75.79151611)
\curveto(551.16871234,75.81150829)(551.3087122,75.84150826)(551.44871582,75.88151611)
\lineto(551.64371582,75.88151611)
\curveto(551.71371179,75.89150821)(551.77871173,75.9015082)(551.83871582,75.91151611)
\curveto(552.72871078,75.92150818)(553.46871004,75.73650837)(554.05871582,75.35651611)
\curveto(554.64870886,74.97650913)(555.07370843,74.48150962)(555.33371582,73.87151611)
\curveto(555.38370812,73.77151033)(555.42370808,73.67151043)(555.45371582,73.57151611)
\curveto(555.48370802,73.47151063)(555.51870799,73.36651074)(555.55871582,73.25651611)
\curveto(555.58870792,73.14651096)(555.61370789,73.02651108)(555.63371582,72.89651611)
\curveto(555.65370785,72.77651133)(555.67870783,72.65151145)(555.70871582,72.52151611)
\curveto(555.71870779,72.47151163)(555.71870779,72.41651169)(555.70871582,72.35651611)
\curveto(555.7087078,72.3065118)(555.71370779,72.25651185)(555.72371582,72.20651611)
\moveto(554.38871582,71.35151611)
\curveto(554.4087091,71.42151268)(554.41370909,71.5015126)(554.40371582,71.59151611)
\lineto(554.40371582,71.84651611)
\curveto(554.4037091,72.23651187)(554.36870914,72.56651154)(554.29871582,72.83651611)
\curveto(554.26870924,72.91651119)(554.24370926,72.99651111)(554.22371582,73.07651611)
\curveto(554.2037093,73.15651095)(554.17870933,73.23151087)(554.14871582,73.30151611)
\curveto(553.86870964,73.95151015)(553.42371008,74.4015097)(552.81371582,74.65151611)
\curveto(552.74371076,74.68150942)(552.66871084,74.7015094)(552.58871582,74.71151611)
\lineto(552.34871582,74.77151611)
\curveto(552.26871124,74.79150931)(552.18371132,74.8015093)(552.09371582,74.80151611)
\lineto(551.82371582,74.80151611)
\lineto(551.55371582,74.75651611)
\curveto(551.45371205,74.73650937)(551.35871215,74.71150939)(551.26871582,74.68151611)
\curveto(551.18871232,74.66150944)(551.1087124,74.63150947)(551.02871582,74.59151611)
\curveto(550.95871255,74.57150953)(550.89371261,74.54150956)(550.83371582,74.50151611)
\curveto(550.77371273,74.46150964)(550.71871279,74.42150968)(550.66871582,74.38151611)
\curveto(550.42871308,74.21150989)(550.23371327,74.0065101)(550.08371582,73.76651611)
\curveto(549.93371357,73.52651058)(549.8037137,73.24651086)(549.69371582,72.92651611)
\curveto(549.66371384,72.82651128)(549.64371386,72.72151138)(549.63371582,72.61151611)
\curveto(549.62371388,72.51151159)(549.6087139,72.4065117)(549.58871582,72.29651611)
\curveto(549.57871393,72.25651185)(549.57371393,72.19151191)(549.57371582,72.10151611)
\curveto(549.56371394,72.07151203)(549.55871395,72.03651207)(549.55871582,71.99651611)
\curveto(549.56871394,71.95651215)(549.57371393,71.91151219)(549.57371582,71.86151611)
\lineto(549.57371582,71.56151611)
\curveto(549.57371393,71.46151264)(549.58371392,71.37151273)(549.60371582,71.29151611)
\lineto(549.63371582,71.11151611)
\curveto(549.65371385,71.01151309)(549.66871384,70.91151319)(549.67871582,70.81151611)
\curveto(549.69871381,70.72151338)(549.72871378,70.63651347)(549.76871582,70.55651611)
\curveto(549.86871364,70.31651379)(549.98371352,70.09151401)(550.11371582,69.88151611)
\curveto(550.25371325,69.67151443)(550.42371308,69.49651461)(550.62371582,69.35651611)
\curveto(550.67371283,69.32651478)(550.71871279,69.3015148)(550.75871582,69.28151611)
\curveto(550.79871271,69.26151484)(550.84371266,69.23651487)(550.89371582,69.20651611)
\curveto(550.97371253,69.15651495)(551.05871245,69.11151499)(551.14871582,69.07151611)
\curveto(551.24871226,69.04151506)(551.35371215,69.01151509)(551.46371582,68.98151611)
\curveto(551.51371199,68.96151514)(551.55871195,68.95151515)(551.59871582,68.95151611)
\curveto(551.64871186,68.96151514)(551.69871181,68.96151514)(551.74871582,68.95151611)
\curveto(551.77871173,68.94151516)(551.83871167,68.93151517)(551.92871582,68.92151611)
\curveto(552.02871148,68.91151519)(552.1037114,68.91651519)(552.15371582,68.93651611)
\curveto(552.19371131,68.94651516)(552.23371127,68.94651516)(552.27371582,68.93651611)
\curveto(552.31371119,68.93651517)(552.35371115,68.94651516)(552.39371582,68.96651611)
\curveto(552.47371103,68.98651512)(552.55371095,69.0015151)(552.63371582,69.01151611)
\curveto(552.71371079,69.03151507)(552.78871072,69.05651505)(552.85871582,69.08651611)
\curveto(553.19871031,69.22651488)(553.47371003,69.42151468)(553.68371582,69.67151611)
\curveto(553.89370961,69.92151418)(554.06870944,70.21651389)(554.20871582,70.55651611)
\curveto(554.25870925,70.67651343)(554.28870922,70.8015133)(554.29871582,70.93151611)
\curveto(554.31870919,71.07151303)(554.34870916,71.21151289)(554.38871582,71.35151611)
}
}
{
\newrgbcolor{curcolor}{0 0 0}
\pscustom[linestyle=none,fillstyle=solid,fillcolor=curcolor]
{
\newpath
\moveto(564.17199707,68.81651611)
\lineto(564.17199707,68.42651611)
\curveto(564.1719892,68.3065158)(564.14698922,68.2065159)(564.09699707,68.12651611)
\curveto(564.04698932,68.05651605)(563.96198941,68.01651609)(563.84199707,68.00651611)
\lineto(563.49699707,68.00651611)
\curveto(563.43698993,68.0065161)(563.37698999,68.0015161)(563.31699707,67.99151611)
\curveto(563.2669901,67.99151611)(563.22199015,68.0015161)(563.18199707,68.02151611)
\curveto(563.09199028,68.04151606)(563.03199034,68.08151602)(563.00199707,68.14151611)
\curveto(562.96199041,68.19151591)(562.93699043,68.25151585)(562.92699707,68.32151611)
\curveto(562.92699044,68.39151571)(562.91199046,68.46151564)(562.88199707,68.53151611)
\curveto(562.8719905,68.55151555)(562.85699051,68.56651554)(562.83699707,68.57651611)
\curveto(562.82699054,68.59651551)(562.81199056,68.61651549)(562.79199707,68.63651611)
\curveto(562.69199068,68.64651546)(562.61199076,68.62651548)(562.55199707,68.57651611)
\curveto(562.50199087,68.52651558)(562.44699092,68.47651563)(562.38699707,68.42651611)
\curveto(562.18699118,68.27651583)(561.98699138,68.16151594)(561.78699707,68.08151611)
\curveto(561.60699176,68.0015161)(561.39699197,67.94151616)(561.15699707,67.90151611)
\curveto(560.92699244,67.86151624)(560.68699268,67.84151626)(560.43699707,67.84151611)
\curveto(560.19699317,67.83151627)(559.95699341,67.84651626)(559.71699707,67.88651611)
\curveto(559.47699389,67.91651619)(559.2669941,67.97151613)(559.08699707,68.05151611)
\curveto(558.5669948,68.27151583)(558.14699522,68.56651554)(557.82699707,68.93651611)
\curveto(557.50699586,69.31651479)(557.25699611,69.78651432)(557.07699707,70.34651611)
\curveto(557.03699633,70.43651367)(557.00699636,70.52651358)(556.98699707,70.61651611)
\curveto(556.97699639,70.71651339)(556.95699641,70.81651329)(556.92699707,70.91651611)
\curveto(556.91699645,70.96651314)(556.91199646,71.01651309)(556.91199707,71.06651611)
\curveto(556.91199646,71.11651299)(556.90699646,71.16651294)(556.89699707,71.21651611)
\curveto(556.87699649,71.26651284)(556.8669965,71.31651279)(556.86699707,71.36651611)
\curveto(556.87699649,71.42651268)(556.87699649,71.48151262)(556.86699707,71.53151611)
\lineto(556.86699707,71.68151611)
\curveto(556.84699652,71.73151237)(556.83699653,71.79651231)(556.83699707,71.87651611)
\curveto(556.83699653,71.95651215)(556.84699652,72.02151208)(556.86699707,72.07151611)
\lineto(556.86699707,72.23651611)
\curveto(556.88699648,72.3065118)(556.89199648,72.37651173)(556.88199707,72.44651611)
\curveto(556.88199649,72.52651158)(556.89199648,72.6015115)(556.91199707,72.67151611)
\curveto(556.92199645,72.72151138)(556.92699644,72.76651134)(556.92699707,72.80651611)
\curveto(556.92699644,72.84651126)(556.93199644,72.89151121)(556.94199707,72.94151611)
\curveto(556.9719964,73.04151106)(556.99699637,73.13651097)(557.01699707,73.22651611)
\curveto(557.03699633,73.32651078)(557.06199631,73.42151068)(557.09199707,73.51151611)
\curveto(557.22199615,73.89151021)(557.38699598,74.23150987)(557.58699707,74.53151611)
\curveto(557.79699557,74.84150926)(558.04699532,75.09650901)(558.33699707,75.29651611)
\curveto(558.50699486,75.41650869)(558.68199469,75.51650859)(558.86199707,75.59651611)
\curveto(559.05199432,75.67650843)(559.25699411,75.74650836)(559.47699707,75.80651611)
\curveto(559.54699382,75.81650829)(559.61199376,75.82650828)(559.67199707,75.83651611)
\curveto(559.74199363,75.84650826)(559.81199356,75.86150824)(559.88199707,75.88151611)
\lineto(560.03199707,75.88151611)
\curveto(560.11199326,75.9015082)(560.22699314,75.91150819)(560.37699707,75.91151611)
\curveto(560.53699283,75.91150819)(560.65699271,75.9015082)(560.73699707,75.88151611)
\curveto(560.77699259,75.87150823)(560.83199254,75.86650824)(560.90199707,75.86651611)
\curveto(561.01199236,75.83650827)(561.12199225,75.81150829)(561.23199707,75.79151611)
\curveto(561.34199203,75.78150832)(561.44699192,75.75150835)(561.54699707,75.70151611)
\curveto(561.69699167,75.64150846)(561.83699153,75.57650853)(561.96699707,75.50651611)
\curveto(562.10699126,75.43650867)(562.23699113,75.35650875)(562.35699707,75.26651611)
\curveto(562.41699095,75.21650889)(562.47699089,75.16150894)(562.53699707,75.10151611)
\curveto(562.60699076,75.05150905)(562.69699067,75.03650907)(562.80699707,75.05651611)
\curveto(562.82699054,75.08650902)(562.84199053,75.11150899)(562.85199707,75.13151611)
\curveto(562.8719905,75.15150895)(562.88699048,75.18150892)(562.89699707,75.22151611)
\curveto(562.92699044,75.31150879)(562.93699043,75.42650868)(562.92699707,75.56651611)
\lineto(562.92699707,75.94151611)
\lineto(562.92699707,77.66651611)
\lineto(562.92699707,78.13151611)
\curveto(562.92699044,78.31150579)(562.95199042,78.44150566)(563.00199707,78.52151611)
\curveto(563.04199033,78.59150551)(563.10199027,78.63650547)(563.18199707,78.65651611)
\curveto(563.20199017,78.65650545)(563.22699014,78.65650545)(563.25699707,78.65651611)
\curveto(563.28699008,78.66650544)(563.31199006,78.67150543)(563.33199707,78.67151611)
\curveto(563.4719899,78.68150542)(563.61698975,78.68150542)(563.76699707,78.67151611)
\curveto(563.92698944,78.67150543)(564.03698933,78.63150547)(564.09699707,78.55151611)
\curveto(564.14698922,78.47150563)(564.1719892,78.37150573)(564.17199707,78.25151611)
\lineto(564.17199707,77.87651611)
\lineto(564.17199707,68.81651611)
\moveto(562.95699707,71.65151611)
\curveto(562.97699039,71.7015124)(562.98699038,71.76651234)(562.98699707,71.84651611)
\curveto(562.98699038,71.93651217)(562.97699039,72.0065121)(562.95699707,72.05651611)
\lineto(562.95699707,72.28151611)
\curveto(562.93699043,72.37151173)(562.92199045,72.46151164)(562.91199707,72.55151611)
\curveto(562.90199047,72.65151145)(562.88199049,72.74151136)(562.85199707,72.82151611)
\curveto(562.83199054,72.9015112)(562.81199056,72.97651113)(562.79199707,73.04651611)
\curveto(562.78199059,73.11651099)(562.76199061,73.18651092)(562.73199707,73.25651611)
\curveto(562.61199076,73.55651055)(562.45699091,73.82151028)(562.26699707,74.05151611)
\curveto(562.07699129,74.28150982)(561.83699153,74.46150964)(561.54699707,74.59151611)
\curveto(561.44699192,74.64150946)(561.34199203,74.67650943)(561.23199707,74.69651611)
\curveto(561.13199224,74.72650938)(561.02199235,74.75150935)(560.90199707,74.77151611)
\curveto(560.82199255,74.79150931)(560.73199264,74.8015093)(560.63199707,74.80151611)
\lineto(560.36199707,74.80151611)
\curveto(560.31199306,74.79150931)(560.2669931,74.78150932)(560.22699707,74.77151611)
\lineto(560.09199707,74.77151611)
\curveto(560.01199336,74.75150935)(559.92699344,74.73150937)(559.83699707,74.71151611)
\curveto(559.75699361,74.69150941)(559.67699369,74.66650944)(559.59699707,74.63651611)
\curveto(559.27699409,74.49650961)(559.01699435,74.29150981)(558.81699707,74.02151611)
\curveto(558.62699474,73.76151034)(558.4719949,73.45651065)(558.35199707,73.10651611)
\curveto(558.31199506,72.99651111)(558.28199509,72.88151122)(558.26199707,72.76151611)
\curveto(558.25199512,72.65151145)(558.23699513,72.54151156)(558.21699707,72.43151611)
\curveto(558.21699515,72.39151171)(558.21199516,72.35151175)(558.20199707,72.31151611)
\lineto(558.20199707,72.20651611)
\curveto(558.18199519,72.15651195)(558.1719952,72.101512)(558.17199707,72.04151611)
\curveto(558.18199519,71.98151212)(558.18699518,71.92651218)(558.18699707,71.87651611)
\lineto(558.18699707,71.54651611)
\curveto(558.18699518,71.44651266)(558.19699517,71.35151275)(558.21699707,71.26151611)
\curveto(558.22699514,71.23151287)(558.23199514,71.18151292)(558.23199707,71.11151611)
\curveto(558.25199512,71.04151306)(558.2669951,70.97151313)(558.27699707,70.90151611)
\lineto(558.33699707,70.69151611)
\curveto(558.44699492,70.34151376)(558.59699477,70.04151406)(558.78699707,69.79151611)
\curveto(558.97699439,69.54151456)(559.21699415,69.33651477)(559.50699707,69.17651611)
\curveto(559.59699377,69.12651498)(559.68699368,69.08651502)(559.77699707,69.05651611)
\curveto(559.8669935,69.02651508)(559.9669934,68.99651511)(560.07699707,68.96651611)
\curveto(560.12699324,68.94651516)(560.17699319,68.94151516)(560.22699707,68.95151611)
\curveto(560.28699308,68.96151514)(560.34199303,68.95651515)(560.39199707,68.93651611)
\curveto(560.43199294,68.92651518)(560.4719929,68.92151518)(560.51199707,68.92151611)
\lineto(560.64699707,68.92151611)
\lineto(560.78199707,68.92151611)
\curveto(560.81199256,68.93151517)(560.86199251,68.93651517)(560.93199707,68.93651611)
\curveto(561.01199236,68.95651515)(561.09199228,68.97151513)(561.17199707,68.98151611)
\curveto(561.25199212,69.0015151)(561.32699204,69.02651508)(561.39699707,69.05651611)
\curveto(561.72699164,69.19651491)(561.99199138,69.37151473)(562.19199707,69.58151611)
\curveto(562.40199097,69.8015143)(562.57699079,70.07651403)(562.71699707,70.40651611)
\curveto(562.7669906,70.51651359)(562.80199057,70.62651348)(562.82199707,70.73651611)
\curveto(562.84199053,70.84651326)(562.8669905,70.95651315)(562.89699707,71.06651611)
\curveto(562.91699045,71.106513)(562.92699044,71.14151296)(562.92699707,71.17151611)
\curveto(562.92699044,71.21151289)(562.93199044,71.25151285)(562.94199707,71.29151611)
\curveto(562.95199042,71.35151275)(562.95199042,71.41151269)(562.94199707,71.47151611)
\curveto(562.94199043,71.53151257)(562.94699042,71.59151251)(562.95699707,71.65151611)
}
}
{
\newrgbcolor{curcolor}{0 0 0}
\pscustom[linestyle=none,fillstyle=solid,fillcolor=curcolor]
{
\newpath
\moveto(572.86824707,72.17651611)
\curveto(572.88823939,72.07651203)(572.88823939,71.96151214)(572.86824707,71.83151611)
\curveto(572.85823942,71.71151239)(572.82823945,71.62651248)(572.77824707,71.57651611)
\curveto(572.72823955,71.53651257)(572.65323962,71.5065126)(572.55324707,71.48651611)
\curveto(572.46323981,71.47651263)(572.35823992,71.47151263)(572.23824707,71.47151611)
\lineto(571.87824707,71.47151611)
\curveto(571.75824052,71.48151262)(571.65324062,71.48651262)(571.56324707,71.48651611)
\lineto(567.72324707,71.48651611)
\curveto(567.64324463,71.48651262)(567.56324471,71.48151262)(567.48324707,71.47151611)
\curveto(567.40324487,71.47151263)(567.33824494,71.45651265)(567.28824707,71.42651611)
\curveto(567.24824503,71.4065127)(567.20824507,71.36651274)(567.16824707,71.30651611)
\curveto(567.14824513,71.27651283)(567.12824515,71.23151287)(567.10824707,71.17151611)
\curveto(567.08824519,71.12151298)(567.08824519,71.07151303)(567.10824707,71.02151611)
\curveto(567.11824516,70.97151313)(567.12324515,70.92651318)(567.12324707,70.88651611)
\curveto(567.12324515,70.84651326)(567.12824515,70.8065133)(567.13824707,70.76651611)
\curveto(567.15824512,70.68651342)(567.1782451,70.6015135)(567.19824707,70.51151611)
\curveto(567.21824506,70.43151367)(567.24824503,70.35151375)(567.28824707,70.27151611)
\curveto(567.51824476,69.73151437)(567.89824438,69.34651476)(568.42824707,69.11651611)
\curveto(568.48824379,69.08651502)(568.55324372,69.06151504)(568.62324707,69.04151611)
\lineto(568.83324707,68.98151611)
\curveto(568.86324341,68.97151513)(568.91324336,68.96651514)(568.98324707,68.96651611)
\curveto(569.12324315,68.92651518)(569.30824297,68.9065152)(569.53824707,68.90651611)
\curveto(569.76824251,68.9065152)(569.95324232,68.92651518)(570.09324707,68.96651611)
\curveto(570.23324204,69.0065151)(570.35824192,69.04651506)(570.46824707,69.08651611)
\curveto(570.58824169,69.13651497)(570.69824158,69.19651491)(570.79824707,69.26651611)
\curveto(570.90824137,69.33651477)(571.00324127,69.41651469)(571.08324707,69.50651611)
\curveto(571.16324111,69.6065145)(571.23324104,69.71151439)(571.29324707,69.82151611)
\curveto(571.35324092,69.92151418)(571.40324087,70.02651408)(571.44324707,70.13651611)
\curveto(571.49324078,70.24651386)(571.5732407,70.32651378)(571.68324707,70.37651611)
\curveto(571.72324055,70.39651371)(571.78824049,70.41151369)(571.87824707,70.42151611)
\curveto(571.96824031,70.43151367)(572.05824022,70.43151367)(572.14824707,70.42151611)
\curveto(572.23824004,70.42151368)(572.32323995,70.41651369)(572.40324707,70.40651611)
\curveto(572.48323979,70.39651371)(572.53823974,70.37651373)(572.56824707,70.34651611)
\curveto(572.66823961,70.27651383)(572.69323958,70.16151394)(572.64324707,70.00151611)
\curveto(572.56323971,69.73151437)(572.45823982,69.49151461)(572.32824707,69.28151611)
\curveto(572.12824015,68.96151514)(571.89824038,68.69651541)(571.63824707,68.48651611)
\curveto(571.38824089,68.28651582)(571.06824121,68.12151598)(570.67824707,67.99151611)
\curveto(570.5782417,67.95151615)(570.4782418,67.92651618)(570.37824707,67.91651611)
\curveto(570.278242,67.89651621)(570.1732421,67.87651623)(570.06324707,67.85651611)
\curveto(570.01324226,67.84651626)(569.96324231,67.84151626)(569.91324707,67.84151611)
\curveto(569.8732424,67.84151626)(569.82824245,67.83651627)(569.77824707,67.82651611)
\lineto(569.62824707,67.82651611)
\curveto(569.5782427,67.81651629)(569.51824276,67.81151629)(569.44824707,67.81151611)
\curveto(569.38824289,67.81151629)(569.33824294,67.81651629)(569.29824707,67.82651611)
\lineto(569.16324707,67.82651611)
\curveto(569.11324316,67.83651627)(569.06824321,67.84151626)(569.02824707,67.84151611)
\curveto(568.98824329,67.84151626)(568.94824333,67.84651626)(568.90824707,67.85651611)
\curveto(568.85824342,67.86651624)(568.80324347,67.87651623)(568.74324707,67.88651611)
\curveto(568.68324359,67.88651622)(568.62824365,67.89151621)(568.57824707,67.90151611)
\curveto(568.48824379,67.92151618)(568.39824388,67.94651616)(568.30824707,67.97651611)
\curveto(568.21824406,67.99651611)(568.13324414,68.02151608)(568.05324707,68.05151611)
\curveto(568.01324426,68.07151603)(567.9782443,68.08151602)(567.94824707,68.08151611)
\curveto(567.91824436,68.09151601)(567.88324439,68.106516)(567.84324707,68.12651611)
\curveto(567.69324458,68.19651591)(567.53324474,68.28151582)(567.36324707,68.38151611)
\curveto(567.0732452,68.57151553)(566.82324545,68.8015153)(566.61324707,69.07151611)
\curveto(566.41324586,69.35151475)(566.24324603,69.66151444)(566.10324707,70.00151611)
\curveto(566.05324622,70.11151399)(566.01324626,70.22651388)(565.98324707,70.34651611)
\curveto(565.96324631,70.46651364)(565.93324634,70.58651352)(565.89324707,70.70651611)
\curveto(565.88324639,70.74651336)(565.8782464,70.78151332)(565.87824707,70.81151611)
\curveto(565.8782464,70.84151326)(565.8732464,70.88151322)(565.86324707,70.93151611)
\curveto(565.84324643,71.01151309)(565.82824645,71.09651301)(565.81824707,71.18651611)
\curveto(565.80824647,71.27651283)(565.79324648,71.36651274)(565.77324707,71.45651611)
\lineto(565.77324707,71.66651611)
\curveto(565.76324651,71.7065124)(565.75324652,71.76151234)(565.74324707,71.83151611)
\curveto(565.74324653,71.91151219)(565.74824653,71.97651213)(565.75824707,72.02651611)
\lineto(565.75824707,72.19151611)
\curveto(565.7782465,72.24151186)(565.78324649,72.29151181)(565.77324707,72.34151611)
\curveto(565.7732465,72.4015117)(565.7782465,72.45651165)(565.78824707,72.50651611)
\curveto(565.82824645,72.66651144)(565.85824642,72.82651128)(565.87824707,72.98651611)
\curveto(565.90824637,73.14651096)(565.95324632,73.29651081)(566.01324707,73.43651611)
\curveto(566.06324621,73.54651056)(566.10824617,73.65651045)(566.14824707,73.76651611)
\curveto(566.19824608,73.88651022)(566.25324602,74.0015101)(566.31324707,74.11151611)
\curveto(566.53324574,74.46150964)(566.78324549,74.76150934)(567.06324707,75.01151611)
\curveto(567.34324493,75.27150883)(567.68824459,75.48650862)(568.09824707,75.65651611)
\curveto(568.21824406,75.7065084)(568.33824394,75.74150836)(568.45824707,75.76151611)
\curveto(568.58824369,75.79150831)(568.72324355,75.82150828)(568.86324707,75.85151611)
\curveto(568.91324336,75.86150824)(568.95824332,75.86650824)(568.99824707,75.86651611)
\curveto(569.03824324,75.87650823)(569.08324319,75.88150822)(569.13324707,75.88151611)
\curveto(569.15324312,75.89150821)(569.1782431,75.89150821)(569.20824707,75.88151611)
\curveto(569.23824304,75.87150823)(569.26324301,75.87650823)(569.28324707,75.89651611)
\curveto(569.70324257,75.9065082)(570.06824221,75.86150824)(570.37824707,75.76151611)
\curveto(570.68824159,75.67150843)(570.96824131,75.54650856)(571.21824707,75.38651611)
\curveto(571.26824101,75.36650874)(571.30824097,75.33650877)(571.33824707,75.29651611)
\curveto(571.36824091,75.26650884)(571.40324087,75.24150886)(571.44324707,75.22151611)
\curveto(571.52324075,75.16150894)(571.60324067,75.09150901)(571.68324707,75.01151611)
\curveto(571.7732405,74.93150917)(571.84824043,74.85150925)(571.90824707,74.77151611)
\curveto(572.06824021,74.56150954)(572.20324007,74.36150974)(572.31324707,74.17151611)
\curveto(572.38323989,74.06151004)(572.43823984,73.94151016)(572.47824707,73.81151611)
\curveto(572.51823976,73.68151042)(572.56323971,73.55151055)(572.61324707,73.42151611)
\curveto(572.66323961,73.29151081)(572.69823958,73.15651095)(572.71824707,73.01651611)
\curveto(572.74823953,72.87651123)(572.78323949,72.73651137)(572.82324707,72.59651611)
\curveto(572.83323944,72.52651158)(572.83823944,72.45651165)(572.83824707,72.38651611)
\lineto(572.86824707,72.17651611)
\moveto(571.41324707,72.68651611)
\curveto(571.44324083,72.72651138)(571.46824081,72.77651133)(571.48824707,72.83651611)
\curveto(571.50824077,72.9065112)(571.50824077,72.97651113)(571.48824707,73.04651611)
\curveto(571.42824085,73.26651084)(571.34324093,73.47151063)(571.23324707,73.66151611)
\curveto(571.09324118,73.89151021)(570.93824134,74.08651002)(570.76824707,74.24651611)
\curveto(570.59824168,74.4065097)(570.3782419,74.54150956)(570.10824707,74.65151611)
\curveto(570.03824224,74.67150943)(569.96824231,74.68650942)(569.89824707,74.69651611)
\curveto(569.82824245,74.71650939)(569.75324252,74.73650937)(569.67324707,74.75651611)
\curveto(569.59324268,74.77650933)(569.50824277,74.78650932)(569.41824707,74.78651611)
\lineto(569.16324707,74.78651611)
\curveto(569.13324314,74.76650934)(569.09824318,74.75650935)(569.05824707,74.75651611)
\curveto(569.01824326,74.76650934)(568.98324329,74.76650934)(568.95324707,74.75651611)
\lineto(568.71324707,74.69651611)
\curveto(568.64324363,74.68650942)(568.5732437,74.67150943)(568.50324707,74.65151611)
\curveto(568.21324406,74.53150957)(567.9782443,74.38150972)(567.79824707,74.20151611)
\curveto(567.62824465,74.02151008)(567.4732448,73.79651031)(567.33324707,73.52651611)
\curveto(567.30324497,73.47651063)(567.273245,73.41151069)(567.24324707,73.33151611)
\curveto(567.21324506,73.26151084)(567.18824509,73.18151092)(567.16824707,73.09151611)
\curveto(567.14824513,73.0015111)(567.14324513,72.91651119)(567.15324707,72.83651611)
\curveto(567.16324511,72.75651135)(567.19824508,72.69651141)(567.25824707,72.65651611)
\curveto(567.33824494,72.59651151)(567.4732448,72.56651154)(567.66324707,72.56651611)
\curveto(567.86324441,72.57651153)(568.03324424,72.58151152)(568.17324707,72.58151611)
\lineto(570.45324707,72.58151611)
\curveto(570.60324167,72.58151152)(570.78324149,72.57651153)(570.99324707,72.56651611)
\curveto(571.20324107,72.56651154)(571.34324093,72.6065115)(571.41324707,72.68651611)
}
}
{
\newrgbcolor{curcolor}{0 0 0}
\pscustom[linestyle=none,fillstyle=solid,fillcolor=curcolor]
{
\newpath
\moveto(577.8198877,75.91151611)
\curveto(578.04988291,75.91150819)(578.17988278,75.85150825)(578.2098877,75.73151611)
\curveto(578.23988272,75.62150848)(578.2548827,75.45650865)(578.2548877,75.23651611)
\lineto(578.2548877,74.95151611)
\curveto(578.2548827,74.86150924)(578.22988273,74.78650932)(578.1798877,74.72651611)
\curveto(578.11988284,74.64650946)(578.03488292,74.6015095)(577.9248877,74.59151611)
\curveto(577.81488314,74.59150951)(577.70488325,74.57650953)(577.5948877,74.54651611)
\curveto(577.4548835,74.51650959)(577.31988364,74.48650962)(577.1898877,74.45651611)
\curveto(577.06988389,74.42650968)(576.954884,74.38650972)(576.8448877,74.33651611)
\curveto(576.5548844,74.2065099)(576.31988464,74.02651008)(576.1398877,73.79651611)
\curveto(575.959885,73.57651053)(575.80488515,73.32151078)(575.6748877,73.03151611)
\curveto(575.63488532,72.92151118)(575.60488535,72.8065113)(575.5848877,72.68651611)
\curveto(575.56488539,72.57651153)(575.53988542,72.46151164)(575.5098877,72.34151611)
\curveto(575.49988546,72.29151181)(575.49488546,72.24151186)(575.4948877,72.19151611)
\curveto(575.50488545,72.14151196)(575.50488545,72.09151201)(575.4948877,72.04151611)
\curveto(575.46488549,71.92151218)(575.44988551,71.78151232)(575.4498877,71.62151611)
\curveto(575.4598855,71.47151263)(575.46488549,71.32651278)(575.4648877,71.18651611)
\lineto(575.4648877,69.34151611)
\lineto(575.4648877,68.99651611)
\curveto(575.46488549,68.87651523)(575.4598855,68.76151534)(575.4498877,68.65151611)
\curveto(575.43988552,68.54151556)(575.43488552,68.44651566)(575.4348877,68.36651611)
\curveto(575.44488551,68.28651582)(575.42488553,68.21651589)(575.3748877,68.15651611)
\curveto(575.32488563,68.08651602)(575.24488571,68.04651606)(575.1348877,68.03651611)
\curveto(575.03488592,68.02651608)(574.92488603,68.02151608)(574.8048877,68.02151611)
\lineto(574.5348877,68.02151611)
\curveto(574.48488647,68.04151606)(574.43488652,68.05651605)(574.3848877,68.06651611)
\curveto(574.34488661,68.08651602)(574.31488664,68.11151599)(574.2948877,68.14151611)
\curveto(574.24488671,68.21151589)(574.21488674,68.29651581)(574.2048877,68.39651611)
\lineto(574.2048877,68.72651611)
\lineto(574.2048877,69.88151611)
\lineto(574.2048877,74.03651611)
\lineto(574.2048877,75.07151611)
\lineto(574.2048877,75.37151611)
\curveto(574.21488674,75.47150863)(574.24488671,75.55650855)(574.2948877,75.62651611)
\curveto(574.32488663,75.66650844)(574.37488658,75.69650841)(574.4448877,75.71651611)
\curveto(574.52488643,75.73650837)(574.60988635,75.74650836)(574.6998877,75.74651611)
\curveto(574.78988617,75.75650835)(574.87988608,75.75650835)(574.9698877,75.74651611)
\curveto(575.0598859,75.73650837)(575.12988583,75.72150838)(575.1798877,75.70151611)
\curveto(575.2598857,75.67150843)(575.30988565,75.61150849)(575.3298877,75.52151611)
\curveto(575.3598856,75.44150866)(575.37488558,75.35150875)(575.3748877,75.25151611)
\lineto(575.3748877,74.95151611)
\curveto(575.37488558,74.85150925)(575.39488556,74.76150934)(575.4348877,74.68151611)
\curveto(575.44488551,74.66150944)(575.4548855,74.64650946)(575.4648877,74.63651611)
\lineto(575.5098877,74.59151611)
\curveto(575.61988534,74.59150951)(575.70988525,74.63650947)(575.7798877,74.72651611)
\curveto(575.84988511,74.82650928)(575.90988505,74.9065092)(575.9598877,74.96651611)
\lineto(576.0498877,75.05651611)
\curveto(576.13988482,75.16650894)(576.26488469,75.28150882)(576.4248877,75.40151611)
\curveto(576.58488437,75.52150858)(576.73488422,75.61150849)(576.8748877,75.67151611)
\curveto(576.96488399,75.72150838)(577.0598839,75.75650835)(577.1598877,75.77651611)
\curveto(577.2598837,75.8065083)(577.36488359,75.83650827)(577.4748877,75.86651611)
\curveto(577.53488342,75.87650823)(577.59488336,75.88150822)(577.6548877,75.88151611)
\curveto(577.71488324,75.89150821)(577.76988319,75.9015082)(577.8198877,75.91151611)
}
}
{
\newrgbcolor{curcolor}{0 0 0}
\pscustom[linestyle=none,fillstyle=solid,fillcolor=curcolor]
{
\newpath
\moveto(586.06965332,68.56151611)
\curveto(586.09964549,68.4015157)(586.08464551,68.26651584)(586.02465332,68.15651611)
\curveto(585.96464563,68.05651605)(585.88464571,67.98151612)(585.78465332,67.93151611)
\curveto(585.73464586,67.91151619)(585.67964591,67.9015162)(585.61965332,67.90151611)
\curveto(585.56964602,67.9015162)(585.51464608,67.89151621)(585.45465332,67.87151611)
\curveto(585.23464636,67.82151628)(585.01464658,67.83651627)(584.79465332,67.91651611)
\curveto(584.58464701,67.98651612)(584.43964715,68.07651603)(584.35965332,68.18651611)
\curveto(584.30964728,68.25651585)(584.26464733,68.33651577)(584.22465332,68.42651611)
\curveto(584.18464741,68.52651558)(584.13464746,68.6065155)(584.07465332,68.66651611)
\curveto(584.05464754,68.68651542)(584.02964756,68.7065154)(583.99965332,68.72651611)
\curveto(583.97964761,68.74651536)(583.94964764,68.75151535)(583.90965332,68.74151611)
\curveto(583.79964779,68.71151539)(583.6946479,68.65651545)(583.59465332,68.57651611)
\curveto(583.50464809,68.49651561)(583.41464818,68.42651568)(583.32465332,68.36651611)
\curveto(583.1946484,68.28651582)(583.05464854,68.21151589)(582.90465332,68.14151611)
\curveto(582.75464884,68.08151602)(582.594649,68.02651608)(582.42465332,67.97651611)
\curveto(582.32464927,67.94651616)(582.21464938,67.92651618)(582.09465332,67.91651611)
\curveto(581.98464961,67.9065162)(581.87464972,67.89151621)(581.76465332,67.87151611)
\curveto(581.71464988,67.86151624)(581.66964992,67.85651625)(581.62965332,67.85651611)
\lineto(581.52465332,67.85651611)
\curveto(581.41465018,67.83651627)(581.30965028,67.83651627)(581.20965332,67.85651611)
\lineto(581.07465332,67.85651611)
\curveto(581.02465057,67.86651624)(580.97465062,67.87151623)(580.92465332,67.87151611)
\curveto(580.87465072,67.87151623)(580.82965076,67.88151622)(580.78965332,67.90151611)
\curveto(580.74965084,67.91151619)(580.71465088,67.91651619)(580.68465332,67.91651611)
\curveto(580.66465093,67.9065162)(580.63965095,67.9065162)(580.60965332,67.91651611)
\lineto(580.36965332,67.97651611)
\curveto(580.2896513,67.98651612)(580.21465138,68.0065161)(580.14465332,68.03651611)
\curveto(579.84465175,68.16651594)(579.59965199,68.31151579)(579.40965332,68.47151611)
\curveto(579.22965236,68.64151546)(579.07965251,68.87651523)(578.95965332,69.17651611)
\curveto(578.86965272,69.39651471)(578.82465277,69.66151444)(578.82465332,69.97151611)
\lineto(578.82465332,70.28651611)
\curveto(578.83465276,70.33651377)(578.83965275,70.38651372)(578.83965332,70.43651611)
\lineto(578.86965332,70.61651611)
\lineto(578.98965332,70.94651611)
\curveto(579.02965256,71.05651305)(579.07965251,71.15651295)(579.13965332,71.24651611)
\curveto(579.31965227,71.53651257)(579.56465203,71.75151235)(579.87465332,71.89151611)
\curveto(580.18465141,72.03151207)(580.52465107,72.15651195)(580.89465332,72.26651611)
\curveto(581.03465056,72.3065118)(581.17965041,72.33651177)(581.32965332,72.35651611)
\curveto(581.47965011,72.37651173)(581.62964996,72.4015117)(581.77965332,72.43151611)
\curveto(581.84964974,72.45151165)(581.91464968,72.46151164)(581.97465332,72.46151611)
\curveto(582.04464955,72.46151164)(582.11964947,72.47151163)(582.19965332,72.49151611)
\curveto(582.26964932,72.51151159)(582.33964925,72.52151158)(582.40965332,72.52151611)
\curveto(582.47964911,72.53151157)(582.55464904,72.54651156)(582.63465332,72.56651611)
\curveto(582.88464871,72.62651148)(583.11964847,72.67651143)(583.33965332,72.71651611)
\curveto(583.55964803,72.76651134)(583.73464786,72.88151122)(583.86465332,73.06151611)
\curveto(583.92464767,73.14151096)(583.97464762,73.24151086)(584.01465332,73.36151611)
\curveto(584.05464754,73.49151061)(584.05464754,73.63151047)(584.01465332,73.78151611)
\curveto(583.95464764,74.02151008)(583.86464773,74.21150989)(583.74465332,74.35151611)
\curveto(583.63464796,74.49150961)(583.47464812,74.6015095)(583.26465332,74.68151611)
\curveto(583.14464845,74.73150937)(582.99964859,74.76650934)(582.82965332,74.78651611)
\curveto(582.66964892,74.8065093)(582.49964909,74.81650929)(582.31965332,74.81651611)
\curveto(582.13964945,74.81650929)(581.96464963,74.8065093)(581.79465332,74.78651611)
\curveto(581.62464997,74.76650934)(581.47965011,74.73650937)(581.35965332,74.69651611)
\curveto(581.1896504,74.63650947)(581.02465057,74.55150955)(580.86465332,74.44151611)
\curveto(580.78465081,74.38150972)(580.70965088,74.3015098)(580.63965332,74.20151611)
\curveto(580.57965101,74.11150999)(580.52465107,74.01151009)(580.47465332,73.90151611)
\curveto(580.44465115,73.82151028)(580.41465118,73.73651037)(580.38465332,73.64651611)
\curveto(580.36465123,73.55651055)(580.31965127,73.48651062)(580.24965332,73.43651611)
\curveto(580.20965138,73.4065107)(580.13965145,73.38151072)(580.03965332,73.36151611)
\curveto(579.94965164,73.35151075)(579.85465174,73.34651076)(579.75465332,73.34651611)
\curveto(579.65465194,73.34651076)(579.55465204,73.35151075)(579.45465332,73.36151611)
\curveto(579.36465223,73.38151072)(579.29965229,73.4065107)(579.25965332,73.43651611)
\curveto(579.21965237,73.46651064)(579.1896524,73.51651059)(579.16965332,73.58651611)
\curveto(579.14965244,73.65651045)(579.14965244,73.73151037)(579.16965332,73.81151611)
\curveto(579.19965239,73.94151016)(579.22965236,74.06151004)(579.25965332,74.17151611)
\curveto(579.29965229,74.29150981)(579.34465225,74.4065097)(579.39465332,74.51651611)
\curveto(579.58465201,74.86650924)(579.82465177,75.13650897)(580.11465332,75.32651611)
\curveto(580.40465119,75.52650858)(580.76465083,75.68650842)(581.19465332,75.80651611)
\curveto(581.2946503,75.82650828)(581.3946502,75.84150826)(581.49465332,75.85151611)
\curveto(581.60464999,75.86150824)(581.71464988,75.87650823)(581.82465332,75.89651611)
\curveto(581.86464973,75.9065082)(581.92964966,75.9065082)(582.01965332,75.89651611)
\curveto(582.10964948,75.89650821)(582.16464943,75.9065082)(582.18465332,75.92651611)
\curveto(582.88464871,75.93650817)(583.4946481,75.85650825)(584.01465332,75.68651611)
\curveto(584.53464706,75.51650859)(584.89964669,75.19150891)(585.10965332,74.71151611)
\curveto(585.19964639,74.51150959)(585.24964634,74.27650983)(585.25965332,74.00651611)
\curveto(585.27964631,73.74651036)(585.2896463,73.47151063)(585.28965332,73.18151611)
\lineto(585.28965332,69.86651611)
\curveto(585.2896463,69.72651438)(585.2946463,69.59151451)(585.30465332,69.46151611)
\curveto(585.31464628,69.33151477)(585.34464625,69.22651488)(585.39465332,69.14651611)
\curveto(585.44464615,69.07651503)(585.50964608,69.02651508)(585.58965332,68.99651611)
\curveto(585.67964591,68.95651515)(585.76464583,68.92651518)(585.84465332,68.90651611)
\curveto(585.92464567,68.89651521)(585.98464561,68.85151525)(586.02465332,68.77151611)
\curveto(586.04464555,68.74151536)(586.05464554,68.71151539)(586.05465332,68.68151611)
\curveto(586.05464554,68.65151545)(586.05964553,68.61151549)(586.06965332,68.56151611)
\moveto(583.92465332,70.22651611)
\curveto(583.98464761,70.36651374)(584.01464758,70.52651358)(584.01465332,70.70651611)
\curveto(584.02464757,70.89651321)(584.02964756,71.09151301)(584.02965332,71.29151611)
\curveto(584.02964756,71.4015127)(584.02464757,71.5015126)(584.01465332,71.59151611)
\curveto(584.00464759,71.68151242)(583.96464763,71.75151235)(583.89465332,71.80151611)
\curveto(583.86464773,71.82151228)(583.7946478,71.83151227)(583.68465332,71.83151611)
\curveto(583.66464793,71.81151229)(583.62964796,71.8015123)(583.57965332,71.80151611)
\curveto(583.52964806,71.8015123)(583.48464811,71.79151231)(583.44465332,71.77151611)
\curveto(583.36464823,71.75151235)(583.27464832,71.73151237)(583.17465332,71.71151611)
\lineto(582.87465332,71.65151611)
\curveto(582.84464875,71.65151245)(582.80964878,71.64651246)(582.76965332,71.63651611)
\lineto(582.66465332,71.63651611)
\curveto(582.51464908,71.59651251)(582.34964924,71.57151253)(582.16965332,71.56151611)
\curveto(581.99964959,71.56151254)(581.83964975,71.54151256)(581.68965332,71.50151611)
\curveto(581.60964998,71.48151262)(581.53465006,71.46151264)(581.46465332,71.44151611)
\curveto(581.40465019,71.43151267)(581.33465026,71.41651269)(581.25465332,71.39651611)
\curveto(581.0946505,71.34651276)(580.94465065,71.28151282)(580.80465332,71.20151611)
\curveto(580.66465093,71.13151297)(580.54465105,71.04151306)(580.44465332,70.93151611)
\curveto(580.34465125,70.82151328)(580.26965132,70.68651342)(580.21965332,70.52651611)
\curveto(580.16965142,70.37651373)(580.14965144,70.19151391)(580.15965332,69.97151611)
\curveto(580.15965143,69.87151423)(580.17465142,69.77651433)(580.20465332,69.68651611)
\curveto(580.24465135,69.6065145)(580.2896513,69.53151457)(580.33965332,69.46151611)
\curveto(580.41965117,69.35151475)(580.52465107,69.25651485)(580.65465332,69.17651611)
\curveto(580.78465081,69.106515)(580.92465067,69.04651506)(581.07465332,68.99651611)
\curveto(581.12465047,68.98651512)(581.17465042,68.98151512)(581.22465332,68.98151611)
\curveto(581.27465032,68.98151512)(581.32465027,68.97651513)(581.37465332,68.96651611)
\curveto(581.44465015,68.94651516)(581.52965006,68.93151517)(581.62965332,68.92151611)
\curveto(581.73964985,68.92151518)(581.82964976,68.93151517)(581.89965332,68.95151611)
\curveto(581.95964963,68.97151513)(582.01964957,68.97651513)(582.07965332,68.96651611)
\curveto(582.13964945,68.96651514)(582.19964939,68.97651513)(582.25965332,68.99651611)
\curveto(582.33964925,69.01651509)(582.41464918,69.03151507)(582.48465332,69.04151611)
\curveto(582.56464903,69.05151505)(582.63964895,69.07151503)(582.70965332,69.10151611)
\curveto(582.99964859,69.22151488)(583.24464835,69.36651474)(583.44465332,69.53651611)
\curveto(583.65464794,69.7065144)(583.81464778,69.93651417)(583.92465332,70.22651611)
}
}
{
\newrgbcolor{curcolor}{0 0 0}
\pscustom[linestyle=none,fillstyle=solid,fillcolor=curcolor]
{
\newpath
\moveto(594.20129395,68.81651611)
\lineto(594.20129395,68.42651611)
\curveto(594.20128607,68.3065158)(594.1762861,68.2065159)(594.12629395,68.12651611)
\curveto(594.0762862,68.05651605)(593.99128628,68.01651609)(593.87129395,68.00651611)
\lineto(593.52629395,68.00651611)
\curveto(593.46628681,68.0065161)(593.40628687,68.0015161)(593.34629395,67.99151611)
\curveto(593.29628698,67.99151611)(593.25128702,68.0015161)(593.21129395,68.02151611)
\curveto(593.12128715,68.04151606)(593.06128721,68.08151602)(593.03129395,68.14151611)
\curveto(592.99128728,68.19151591)(592.96628731,68.25151585)(592.95629395,68.32151611)
\curveto(592.95628732,68.39151571)(592.94128733,68.46151564)(592.91129395,68.53151611)
\curveto(592.90128737,68.55151555)(592.88628739,68.56651554)(592.86629395,68.57651611)
\curveto(592.85628742,68.59651551)(592.84128743,68.61651549)(592.82129395,68.63651611)
\curveto(592.72128755,68.64651546)(592.64128763,68.62651548)(592.58129395,68.57651611)
\curveto(592.53128774,68.52651558)(592.4762878,68.47651563)(592.41629395,68.42651611)
\curveto(592.21628806,68.27651583)(592.01628826,68.16151594)(591.81629395,68.08151611)
\curveto(591.63628864,68.0015161)(591.42628885,67.94151616)(591.18629395,67.90151611)
\curveto(590.95628932,67.86151624)(590.71628956,67.84151626)(590.46629395,67.84151611)
\curveto(590.22629005,67.83151627)(589.98629029,67.84651626)(589.74629395,67.88651611)
\curveto(589.50629077,67.91651619)(589.29629098,67.97151613)(589.11629395,68.05151611)
\curveto(588.59629168,68.27151583)(588.1762921,68.56651554)(587.85629395,68.93651611)
\curveto(587.53629274,69.31651479)(587.28629299,69.78651432)(587.10629395,70.34651611)
\curveto(587.06629321,70.43651367)(587.03629324,70.52651358)(587.01629395,70.61651611)
\curveto(587.00629327,70.71651339)(586.98629329,70.81651329)(586.95629395,70.91651611)
\curveto(586.94629333,70.96651314)(586.94129333,71.01651309)(586.94129395,71.06651611)
\curveto(586.94129333,71.11651299)(586.93629334,71.16651294)(586.92629395,71.21651611)
\curveto(586.90629337,71.26651284)(586.89629338,71.31651279)(586.89629395,71.36651611)
\curveto(586.90629337,71.42651268)(586.90629337,71.48151262)(586.89629395,71.53151611)
\lineto(586.89629395,71.68151611)
\curveto(586.8762934,71.73151237)(586.86629341,71.79651231)(586.86629395,71.87651611)
\curveto(586.86629341,71.95651215)(586.8762934,72.02151208)(586.89629395,72.07151611)
\lineto(586.89629395,72.23651611)
\curveto(586.91629336,72.3065118)(586.92129335,72.37651173)(586.91129395,72.44651611)
\curveto(586.91129336,72.52651158)(586.92129335,72.6015115)(586.94129395,72.67151611)
\curveto(586.95129332,72.72151138)(586.95629332,72.76651134)(586.95629395,72.80651611)
\curveto(586.95629332,72.84651126)(586.96129331,72.89151121)(586.97129395,72.94151611)
\curveto(587.00129327,73.04151106)(587.02629325,73.13651097)(587.04629395,73.22651611)
\curveto(587.06629321,73.32651078)(587.09129318,73.42151068)(587.12129395,73.51151611)
\curveto(587.25129302,73.89151021)(587.41629286,74.23150987)(587.61629395,74.53151611)
\curveto(587.82629245,74.84150926)(588.0762922,75.09650901)(588.36629395,75.29651611)
\curveto(588.53629174,75.41650869)(588.71129156,75.51650859)(588.89129395,75.59651611)
\curveto(589.08129119,75.67650843)(589.28629099,75.74650836)(589.50629395,75.80651611)
\curveto(589.5762907,75.81650829)(589.64129063,75.82650828)(589.70129395,75.83651611)
\curveto(589.7712905,75.84650826)(589.84129043,75.86150824)(589.91129395,75.88151611)
\lineto(590.06129395,75.88151611)
\curveto(590.14129013,75.9015082)(590.25629002,75.91150819)(590.40629395,75.91151611)
\curveto(590.56628971,75.91150819)(590.68628959,75.9015082)(590.76629395,75.88151611)
\curveto(590.80628947,75.87150823)(590.86128941,75.86650824)(590.93129395,75.86651611)
\curveto(591.04128923,75.83650827)(591.15128912,75.81150829)(591.26129395,75.79151611)
\curveto(591.3712889,75.78150832)(591.4762888,75.75150835)(591.57629395,75.70151611)
\curveto(591.72628855,75.64150846)(591.86628841,75.57650853)(591.99629395,75.50651611)
\curveto(592.13628814,75.43650867)(592.26628801,75.35650875)(592.38629395,75.26651611)
\curveto(592.44628783,75.21650889)(592.50628777,75.16150894)(592.56629395,75.10151611)
\curveto(592.63628764,75.05150905)(592.72628755,75.03650907)(592.83629395,75.05651611)
\curveto(592.85628742,75.08650902)(592.8712874,75.11150899)(592.88129395,75.13151611)
\curveto(592.90128737,75.15150895)(592.91628736,75.18150892)(592.92629395,75.22151611)
\curveto(592.95628732,75.31150879)(592.96628731,75.42650868)(592.95629395,75.56651611)
\lineto(592.95629395,75.94151611)
\lineto(592.95629395,77.66651611)
\lineto(592.95629395,78.13151611)
\curveto(592.95628732,78.31150579)(592.98128729,78.44150566)(593.03129395,78.52151611)
\curveto(593.0712872,78.59150551)(593.13128714,78.63650547)(593.21129395,78.65651611)
\curveto(593.23128704,78.65650545)(593.25628702,78.65650545)(593.28629395,78.65651611)
\curveto(593.31628696,78.66650544)(593.34128693,78.67150543)(593.36129395,78.67151611)
\curveto(593.50128677,78.68150542)(593.64628663,78.68150542)(593.79629395,78.67151611)
\curveto(593.95628632,78.67150543)(594.06628621,78.63150547)(594.12629395,78.55151611)
\curveto(594.1762861,78.47150563)(594.20128607,78.37150573)(594.20129395,78.25151611)
\lineto(594.20129395,77.87651611)
\lineto(594.20129395,68.81651611)
\moveto(592.98629395,71.65151611)
\curveto(593.00628727,71.7015124)(593.01628726,71.76651234)(593.01629395,71.84651611)
\curveto(593.01628726,71.93651217)(593.00628727,72.0065121)(592.98629395,72.05651611)
\lineto(592.98629395,72.28151611)
\curveto(592.96628731,72.37151173)(592.95128732,72.46151164)(592.94129395,72.55151611)
\curveto(592.93128734,72.65151145)(592.91128736,72.74151136)(592.88129395,72.82151611)
\curveto(592.86128741,72.9015112)(592.84128743,72.97651113)(592.82129395,73.04651611)
\curveto(592.81128746,73.11651099)(592.79128748,73.18651092)(592.76129395,73.25651611)
\curveto(592.64128763,73.55651055)(592.48628779,73.82151028)(592.29629395,74.05151611)
\curveto(592.10628817,74.28150982)(591.86628841,74.46150964)(591.57629395,74.59151611)
\curveto(591.4762888,74.64150946)(591.3712889,74.67650943)(591.26129395,74.69651611)
\curveto(591.16128911,74.72650938)(591.05128922,74.75150935)(590.93129395,74.77151611)
\curveto(590.85128942,74.79150931)(590.76128951,74.8015093)(590.66129395,74.80151611)
\lineto(590.39129395,74.80151611)
\curveto(590.34128993,74.79150931)(590.29628998,74.78150932)(590.25629395,74.77151611)
\lineto(590.12129395,74.77151611)
\curveto(590.04129023,74.75150935)(589.95629032,74.73150937)(589.86629395,74.71151611)
\curveto(589.78629049,74.69150941)(589.70629057,74.66650944)(589.62629395,74.63651611)
\curveto(589.30629097,74.49650961)(589.04629123,74.29150981)(588.84629395,74.02151611)
\curveto(588.65629162,73.76151034)(588.50129177,73.45651065)(588.38129395,73.10651611)
\curveto(588.34129193,72.99651111)(588.31129196,72.88151122)(588.29129395,72.76151611)
\curveto(588.28129199,72.65151145)(588.26629201,72.54151156)(588.24629395,72.43151611)
\curveto(588.24629203,72.39151171)(588.24129203,72.35151175)(588.23129395,72.31151611)
\lineto(588.23129395,72.20651611)
\curveto(588.21129206,72.15651195)(588.20129207,72.101512)(588.20129395,72.04151611)
\curveto(588.21129206,71.98151212)(588.21629206,71.92651218)(588.21629395,71.87651611)
\lineto(588.21629395,71.54651611)
\curveto(588.21629206,71.44651266)(588.22629205,71.35151275)(588.24629395,71.26151611)
\curveto(588.25629202,71.23151287)(588.26129201,71.18151292)(588.26129395,71.11151611)
\curveto(588.28129199,71.04151306)(588.29629198,70.97151313)(588.30629395,70.90151611)
\lineto(588.36629395,70.69151611)
\curveto(588.4762918,70.34151376)(588.62629165,70.04151406)(588.81629395,69.79151611)
\curveto(589.00629127,69.54151456)(589.24629103,69.33651477)(589.53629395,69.17651611)
\curveto(589.62629065,69.12651498)(589.71629056,69.08651502)(589.80629395,69.05651611)
\curveto(589.89629038,69.02651508)(589.99629028,68.99651511)(590.10629395,68.96651611)
\curveto(590.15629012,68.94651516)(590.20629007,68.94151516)(590.25629395,68.95151611)
\curveto(590.31628996,68.96151514)(590.3712899,68.95651515)(590.42129395,68.93651611)
\curveto(590.46128981,68.92651518)(590.50128977,68.92151518)(590.54129395,68.92151611)
\lineto(590.67629395,68.92151611)
\lineto(590.81129395,68.92151611)
\curveto(590.84128943,68.93151517)(590.89128938,68.93651517)(590.96129395,68.93651611)
\curveto(591.04128923,68.95651515)(591.12128915,68.97151513)(591.20129395,68.98151611)
\curveto(591.28128899,69.0015151)(591.35628892,69.02651508)(591.42629395,69.05651611)
\curveto(591.75628852,69.19651491)(592.02128825,69.37151473)(592.22129395,69.58151611)
\curveto(592.43128784,69.8015143)(592.60628767,70.07651403)(592.74629395,70.40651611)
\curveto(592.79628748,70.51651359)(592.83128744,70.62651348)(592.85129395,70.73651611)
\curveto(592.8712874,70.84651326)(592.89628738,70.95651315)(592.92629395,71.06651611)
\curveto(592.94628733,71.106513)(592.95628732,71.14151296)(592.95629395,71.17151611)
\curveto(592.95628732,71.21151289)(592.96128731,71.25151285)(592.97129395,71.29151611)
\curveto(592.98128729,71.35151275)(592.98128729,71.41151269)(592.97129395,71.47151611)
\curveto(592.9712873,71.53151257)(592.9762873,71.59151251)(592.98629395,71.65151611)
}
}
{
\newrgbcolor{curcolor}{0 0 0}
\pscustom[linestyle=none,fillstyle=solid,fillcolor=curcolor]
{
\newpath
\moveto(603.27254395,72.20651611)
\curveto(603.29253589,72.14651196)(603.30253588,72.05151205)(603.30254395,71.92151611)
\curveto(603.30253588,71.8015123)(603.29753588,71.71651239)(603.28754395,71.66651611)
\lineto(603.28754395,71.51651611)
\curveto(603.2775359,71.43651267)(603.26753591,71.36151274)(603.25754395,71.29151611)
\curveto(603.25753592,71.23151287)(603.25253593,71.16151294)(603.24254395,71.08151611)
\curveto(603.22253596,71.02151308)(603.20753597,70.96151314)(603.19754395,70.90151611)
\curveto(603.19753598,70.84151326)(603.18753599,70.78151332)(603.16754395,70.72151611)
\curveto(603.12753605,70.59151351)(603.09253609,70.46151364)(603.06254395,70.33151611)
\curveto(603.03253615,70.2015139)(602.99253619,70.08151402)(602.94254395,69.97151611)
\curveto(602.73253645,69.49151461)(602.45253673,69.08651502)(602.10254395,68.75651611)
\curveto(601.75253743,68.43651567)(601.32253786,68.19151591)(600.81254395,68.02151611)
\curveto(600.70253848,67.98151612)(600.5825386,67.95151615)(600.45254395,67.93151611)
\curveto(600.33253885,67.91151619)(600.20753897,67.89151621)(600.07754395,67.87151611)
\curveto(600.01753916,67.86151624)(599.95253923,67.85651625)(599.88254395,67.85651611)
\curveto(599.82253936,67.84651626)(599.76253942,67.84151626)(599.70254395,67.84151611)
\curveto(599.66253952,67.83151627)(599.60253958,67.82651628)(599.52254395,67.82651611)
\curveto(599.45253973,67.82651628)(599.40253978,67.83151627)(599.37254395,67.84151611)
\curveto(599.33253985,67.85151625)(599.29253989,67.85651625)(599.25254395,67.85651611)
\curveto(599.21253997,67.84651626)(599.17754,67.84651626)(599.14754395,67.85651611)
\lineto(599.05754395,67.85651611)
\lineto(598.69754395,67.90151611)
\curveto(598.55754062,67.94151616)(598.42254076,67.98151612)(598.29254395,68.02151611)
\curveto(598.16254102,68.06151604)(598.03754114,68.106516)(597.91754395,68.15651611)
\curveto(597.46754171,68.35651575)(597.09754208,68.61651549)(596.80754395,68.93651611)
\curveto(596.51754266,69.25651485)(596.2775429,69.64651446)(596.08754395,70.10651611)
\curveto(596.03754314,70.2065139)(595.99754318,70.3065138)(595.96754395,70.40651611)
\curveto(595.94754323,70.5065136)(595.92754325,70.61151349)(595.90754395,70.72151611)
\curveto(595.88754329,70.76151334)(595.8775433,70.79151331)(595.87754395,70.81151611)
\curveto(595.88754329,70.84151326)(595.88754329,70.87651323)(595.87754395,70.91651611)
\curveto(595.85754332,70.99651311)(595.84254334,71.07651303)(595.83254395,71.15651611)
\curveto(595.83254335,71.24651286)(595.82254336,71.33151277)(595.80254395,71.41151611)
\lineto(595.80254395,71.53151611)
\curveto(595.80254338,71.57151253)(595.79754338,71.61651249)(595.78754395,71.66651611)
\curveto(595.7775434,71.71651239)(595.77254341,71.8015123)(595.77254395,71.92151611)
\curveto(595.77254341,72.05151205)(595.7825434,72.14651196)(595.80254395,72.20651611)
\curveto(595.82254336,72.27651183)(595.82754335,72.34651176)(595.81754395,72.41651611)
\curveto(595.80754337,72.48651162)(595.81254337,72.55651155)(595.83254395,72.62651611)
\curveto(595.84254334,72.67651143)(595.84754333,72.71651139)(595.84754395,72.74651611)
\curveto(595.85754332,72.78651132)(595.86754331,72.83151127)(595.87754395,72.88151611)
\curveto(595.90754327,73.0015111)(595.93254325,73.12151098)(595.95254395,73.24151611)
\curveto(595.9825432,73.36151074)(596.02254316,73.47651063)(596.07254395,73.58651611)
\curveto(596.22254296,73.95651015)(596.40254278,74.28650982)(596.61254395,74.57651611)
\curveto(596.83254235,74.87650923)(597.09754208,75.12650898)(597.40754395,75.32651611)
\curveto(597.52754165,75.4065087)(597.65254153,75.47150863)(597.78254395,75.52151611)
\curveto(597.91254127,75.58150852)(598.04754113,75.64150846)(598.18754395,75.70151611)
\curveto(598.30754087,75.75150835)(598.43754074,75.78150832)(598.57754395,75.79151611)
\curveto(598.71754046,75.81150829)(598.85754032,75.84150826)(598.99754395,75.88151611)
\lineto(599.19254395,75.88151611)
\curveto(599.26253992,75.89150821)(599.32753985,75.9015082)(599.38754395,75.91151611)
\curveto(600.2775389,75.92150818)(601.01753816,75.73650837)(601.60754395,75.35651611)
\curveto(602.19753698,74.97650913)(602.62253656,74.48150962)(602.88254395,73.87151611)
\curveto(602.93253625,73.77151033)(602.97253621,73.67151043)(603.00254395,73.57151611)
\curveto(603.03253615,73.47151063)(603.06753611,73.36651074)(603.10754395,73.25651611)
\curveto(603.13753604,73.14651096)(603.16253602,73.02651108)(603.18254395,72.89651611)
\curveto(603.20253598,72.77651133)(603.22753595,72.65151145)(603.25754395,72.52151611)
\curveto(603.26753591,72.47151163)(603.26753591,72.41651169)(603.25754395,72.35651611)
\curveto(603.25753592,72.3065118)(603.26253592,72.25651185)(603.27254395,72.20651611)
\moveto(601.93754395,71.35151611)
\curveto(601.95753722,71.42151268)(601.96253722,71.5015126)(601.95254395,71.59151611)
\lineto(601.95254395,71.84651611)
\curveto(601.95253723,72.23651187)(601.91753726,72.56651154)(601.84754395,72.83651611)
\curveto(601.81753736,72.91651119)(601.79253739,72.99651111)(601.77254395,73.07651611)
\curveto(601.75253743,73.15651095)(601.72753745,73.23151087)(601.69754395,73.30151611)
\curveto(601.41753776,73.95151015)(600.97253821,74.4015097)(600.36254395,74.65151611)
\curveto(600.29253889,74.68150942)(600.21753896,74.7015094)(600.13754395,74.71151611)
\lineto(599.89754395,74.77151611)
\curveto(599.81753936,74.79150931)(599.73253945,74.8015093)(599.64254395,74.80151611)
\lineto(599.37254395,74.80151611)
\lineto(599.10254395,74.75651611)
\curveto(599.00254018,74.73650937)(598.90754027,74.71150939)(598.81754395,74.68151611)
\curveto(598.73754044,74.66150944)(598.65754052,74.63150947)(598.57754395,74.59151611)
\curveto(598.50754067,74.57150953)(598.44254074,74.54150956)(598.38254395,74.50151611)
\curveto(598.32254086,74.46150964)(598.26754091,74.42150968)(598.21754395,74.38151611)
\curveto(597.9775412,74.21150989)(597.7825414,74.0065101)(597.63254395,73.76651611)
\curveto(597.4825417,73.52651058)(597.35254183,73.24651086)(597.24254395,72.92651611)
\curveto(597.21254197,72.82651128)(597.19254199,72.72151138)(597.18254395,72.61151611)
\curveto(597.17254201,72.51151159)(597.15754202,72.4065117)(597.13754395,72.29651611)
\curveto(597.12754205,72.25651185)(597.12254206,72.19151191)(597.12254395,72.10151611)
\curveto(597.11254207,72.07151203)(597.10754207,72.03651207)(597.10754395,71.99651611)
\curveto(597.11754206,71.95651215)(597.12254206,71.91151219)(597.12254395,71.86151611)
\lineto(597.12254395,71.56151611)
\curveto(597.12254206,71.46151264)(597.13254205,71.37151273)(597.15254395,71.29151611)
\lineto(597.18254395,71.11151611)
\curveto(597.20254198,71.01151309)(597.21754196,70.91151319)(597.22754395,70.81151611)
\curveto(597.24754193,70.72151338)(597.2775419,70.63651347)(597.31754395,70.55651611)
\curveto(597.41754176,70.31651379)(597.53254165,70.09151401)(597.66254395,69.88151611)
\curveto(597.80254138,69.67151443)(597.97254121,69.49651461)(598.17254395,69.35651611)
\curveto(598.22254096,69.32651478)(598.26754091,69.3015148)(598.30754395,69.28151611)
\curveto(598.34754083,69.26151484)(598.39254079,69.23651487)(598.44254395,69.20651611)
\curveto(598.52254066,69.15651495)(598.60754057,69.11151499)(598.69754395,69.07151611)
\curveto(598.79754038,69.04151506)(598.90254028,69.01151509)(599.01254395,68.98151611)
\curveto(599.06254012,68.96151514)(599.10754007,68.95151515)(599.14754395,68.95151611)
\curveto(599.19753998,68.96151514)(599.24753993,68.96151514)(599.29754395,68.95151611)
\curveto(599.32753985,68.94151516)(599.38753979,68.93151517)(599.47754395,68.92151611)
\curveto(599.5775396,68.91151519)(599.65253953,68.91651519)(599.70254395,68.93651611)
\curveto(599.74253944,68.94651516)(599.7825394,68.94651516)(599.82254395,68.93651611)
\curveto(599.86253932,68.93651517)(599.90253928,68.94651516)(599.94254395,68.96651611)
\curveto(600.02253916,68.98651512)(600.10253908,69.0015151)(600.18254395,69.01151611)
\curveto(600.26253892,69.03151507)(600.33753884,69.05651505)(600.40754395,69.08651611)
\curveto(600.74753843,69.22651488)(601.02253816,69.42151468)(601.23254395,69.67151611)
\curveto(601.44253774,69.92151418)(601.61753756,70.21651389)(601.75754395,70.55651611)
\curveto(601.80753737,70.67651343)(601.83753734,70.8015133)(601.84754395,70.93151611)
\curveto(601.86753731,71.07151303)(601.89753728,71.21151289)(601.93754395,71.35151611)
}
}
{
\newrgbcolor{curcolor}{0 0 0}
\pscustom[linestyle=none,fillstyle=solid,fillcolor=curcolor]
{
\newpath
\moveto(608.4058252,75.91151611)
\curveto(608.63582041,75.91150819)(608.76582028,75.85150825)(608.7958252,75.73151611)
\curveto(608.82582022,75.62150848)(608.8408202,75.45650865)(608.8408252,75.23651611)
\lineto(608.8408252,74.95151611)
\curveto(608.8408202,74.86150924)(608.81582023,74.78650932)(608.7658252,74.72651611)
\curveto(608.70582034,74.64650946)(608.62082042,74.6015095)(608.5108252,74.59151611)
\curveto(608.40082064,74.59150951)(608.29082075,74.57650953)(608.1808252,74.54651611)
\curveto(608.040821,74.51650959)(607.90582114,74.48650962)(607.7758252,74.45651611)
\curveto(607.65582139,74.42650968)(607.5408215,74.38650972)(607.4308252,74.33651611)
\curveto(607.1408219,74.2065099)(606.90582214,74.02651008)(606.7258252,73.79651611)
\curveto(606.5458225,73.57651053)(606.39082265,73.32151078)(606.2608252,73.03151611)
\curveto(606.22082282,72.92151118)(606.19082285,72.8065113)(606.1708252,72.68651611)
\curveto(606.15082289,72.57651153)(606.12582292,72.46151164)(606.0958252,72.34151611)
\curveto(606.08582296,72.29151181)(606.08082296,72.24151186)(606.0808252,72.19151611)
\curveto(606.09082295,72.14151196)(606.09082295,72.09151201)(606.0808252,72.04151611)
\curveto(606.05082299,71.92151218)(606.03582301,71.78151232)(606.0358252,71.62151611)
\curveto(606.045823,71.47151263)(606.05082299,71.32651278)(606.0508252,71.18651611)
\lineto(606.0508252,69.34151611)
\lineto(606.0508252,68.99651611)
\curveto(606.05082299,68.87651523)(606.045823,68.76151534)(606.0358252,68.65151611)
\curveto(606.02582302,68.54151556)(606.02082302,68.44651566)(606.0208252,68.36651611)
\curveto(606.03082301,68.28651582)(606.01082303,68.21651589)(605.9608252,68.15651611)
\curveto(605.91082313,68.08651602)(605.83082321,68.04651606)(605.7208252,68.03651611)
\curveto(605.62082342,68.02651608)(605.51082353,68.02151608)(605.3908252,68.02151611)
\lineto(605.1208252,68.02151611)
\curveto(605.07082397,68.04151606)(605.02082402,68.05651605)(604.9708252,68.06651611)
\curveto(604.93082411,68.08651602)(604.90082414,68.11151599)(604.8808252,68.14151611)
\curveto(604.83082421,68.21151589)(604.80082424,68.29651581)(604.7908252,68.39651611)
\lineto(604.7908252,68.72651611)
\lineto(604.7908252,69.88151611)
\lineto(604.7908252,74.03651611)
\lineto(604.7908252,75.07151611)
\lineto(604.7908252,75.37151611)
\curveto(604.80082424,75.47150863)(604.83082421,75.55650855)(604.8808252,75.62651611)
\curveto(604.91082413,75.66650844)(604.96082408,75.69650841)(605.0308252,75.71651611)
\curveto(605.11082393,75.73650837)(605.19582385,75.74650836)(605.2858252,75.74651611)
\curveto(605.37582367,75.75650835)(605.46582358,75.75650835)(605.5558252,75.74651611)
\curveto(605.6458234,75.73650837)(605.71582333,75.72150838)(605.7658252,75.70151611)
\curveto(605.8458232,75.67150843)(605.89582315,75.61150849)(605.9158252,75.52151611)
\curveto(605.9458231,75.44150866)(605.96082308,75.35150875)(605.9608252,75.25151611)
\lineto(605.9608252,74.95151611)
\curveto(605.96082308,74.85150925)(605.98082306,74.76150934)(606.0208252,74.68151611)
\curveto(606.03082301,74.66150944)(606.040823,74.64650946)(606.0508252,74.63651611)
\lineto(606.0958252,74.59151611)
\curveto(606.20582284,74.59150951)(606.29582275,74.63650947)(606.3658252,74.72651611)
\curveto(606.43582261,74.82650928)(606.49582255,74.9065092)(606.5458252,74.96651611)
\lineto(606.6358252,75.05651611)
\curveto(606.72582232,75.16650894)(606.85082219,75.28150882)(607.0108252,75.40151611)
\curveto(607.17082187,75.52150858)(607.32082172,75.61150849)(607.4608252,75.67151611)
\curveto(607.55082149,75.72150838)(607.6458214,75.75650835)(607.7458252,75.77651611)
\curveto(607.8458212,75.8065083)(607.95082109,75.83650827)(608.0608252,75.86651611)
\curveto(608.12082092,75.87650823)(608.18082086,75.88150822)(608.2408252,75.88151611)
\curveto(608.30082074,75.89150821)(608.35582069,75.9015082)(608.4058252,75.91151611)
}
}
{
\newrgbcolor{curcolor}{0 0 0}
\pscustom[linestyle=none,fillstyle=solid,fillcolor=curcolor]
{
\newpath
\moveto(718.2480835,68.78651611)
\curveto(718.26807395,68.73651537)(718.29307393,68.67651543)(718.3230835,68.60651611)
\curveto(718.35307387,68.53651557)(718.37307385,68.46151564)(718.3830835,68.38151611)
\curveto(718.40307382,68.31151579)(718.40307382,68.24151586)(718.3830835,68.17151611)
\curveto(718.37307385,68.11151599)(718.33307389,68.06651604)(718.2630835,68.03651611)
\curveto(718.21307401,68.01651609)(718.15307407,68.0065161)(718.0830835,68.00651611)
\lineto(717.8730835,68.00651611)
\lineto(717.4230835,68.00651611)
\curveto(717.27307495,68.0065161)(717.15307507,68.03151607)(717.0630835,68.08151611)
\curveto(716.96307526,68.14151596)(716.88807533,68.24651586)(716.8380835,68.39651611)
\curveto(716.79807542,68.54651556)(716.75307547,68.68151542)(716.7030835,68.80151611)
\curveto(716.59307563,69.06151504)(716.49307573,69.33151477)(716.4030835,69.61151611)
\curveto(716.31307591,69.89151421)(716.21307601,70.16651394)(716.1030835,70.43651611)
\curveto(716.07307615,70.52651358)(716.04307618,70.61151349)(716.0130835,70.69151611)
\curveto(715.99307623,70.77151333)(715.96307626,70.84651326)(715.9230835,70.91651611)
\curveto(715.89307633,70.98651312)(715.84807637,71.04651306)(715.7880835,71.09651611)
\curveto(715.72807649,71.14651296)(715.64807657,71.18651292)(715.5480835,71.21651611)
\curveto(715.49807672,71.23651287)(715.43807678,71.24151286)(715.3680835,71.23151611)
\lineto(715.1730835,71.23151611)
\lineto(712.3380835,71.23151611)
\lineto(712.0380835,71.23151611)
\curveto(711.92808029,71.24151286)(711.8230804,71.24151286)(711.7230835,71.23151611)
\curveto(711.6230806,71.22151288)(711.52808069,71.2065129)(711.4380835,71.18651611)
\curveto(711.35808086,71.16651294)(711.29808092,71.12651298)(711.2580835,71.06651611)
\curveto(711.17808104,70.96651314)(711.1180811,70.85151325)(711.0780835,70.72151611)
\curveto(711.04808117,70.6015135)(711.00808121,70.47651363)(710.9580835,70.34651611)
\curveto(710.85808136,70.11651399)(710.76308146,69.87651423)(710.6730835,69.62651611)
\curveto(710.59308163,69.37651473)(710.50308172,69.13651497)(710.4030835,68.90651611)
\curveto(710.38308184,68.84651526)(710.35808186,68.77651533)(710.3280835,68.69651611)
\curveto(710.30808191,68.62651548)(710.28308194,68.55151555)(710.2530835,68.47151611)
\curveto(710.223082,68.39151571)(710.18808203,68.31651579)(710.1480835,68.24651611)
\curveto(710.1180821,68.18651592)(710.08308214,68.14151596)(710.0430835,68.11151611)
\curveto(709.96308226,68.05151605)(709.85308237,68.01651609)(709.7130835,68.00651611)
\lineto(709.2930835,68.00651611)
\lineto(709.0530835,68.00651611)
\curveto(708.98308324,68.01651609)(708.9230833,68.04151606)(708.8730835,68.08151611)
\curveto(708.8230834,68.11151599)(708.79308343,68.15651595)(708.7830835,68.21651611)
\curveto(708.78308344,68.27651583)(708.78808343,68.33651577)(708.7980835,68.39651611)
\curveto(708.8180834,68.46651564)(708.83808338,68.53151557)(708.8580835,68.59151611)
\curveto(708.88808333,68.66151544)(708.91308331,68.71151539)(708.9330835,68.74151611)
\curveto(709.07308315,69.06151504)(709.19808302,69.37651473)(709.3080835,69.68651611)
\curveto(709.4180828,70.0065141)(709.53808268,70.32651378)(709.6680835,70.64651611)
\curveto(709.75808246,70.86651324)(709.84308238,71.08151302)(709.9230835,71.29151611)
\curveto(710.00308222,71.51151259)(710.08808213,71.73151237)(710.1780835,71.95151611)
\curveto(710.47808174,72.67151143)(710.76308146,73.39651071)(711.0330835,74.12651611)
\curveto(711.30308092,74.86650924)(711.58808063,75.6015085)(711.8880835,76.33151611)
\curveto(711.99808022,76.59150751)(712.09808012,76.85650725)(712.1880835,77.12651611)
\curveto(712.28807993,77.39650671)(712.39307983,77.66150644)(712.5030835,77.92151611)
\curveto(712.55307967,78.03150607)(712.59807962,78.15150595)(712.6380835,78.28151611)
\curveto(712.68807953,78.42150568)(712.75807946,78.52150558)(712.8480835,78.58151611)
\curveto(712.88807933,78.62150548)(712.95307927,78.65150545)(713.0430835,78.67151611)
\curveto(713.06307916,78.68150542)(713.08307914,78.68150542)(713.1030835,78.67151611)
\curveto(713.13307909,78.67150543)(713.15807906,78.67650543)(713.1780835,78.68651611)
\curveto(713.35807886,78.68650542)(713.56807865,78.68650542)(713.8080835,78.68651611)
\curveto(714.04807817,78.69650541)(714.223078,78.66150544)(714.3330835,78.58151611)
\curveto(714.41307781,78.52150558)(714.47307775,78.42150568)(714.5130835,78.28151611)
\curveto(714.56307766,78.15150595)(714.61307761,78.03150607)(714.6630835,77.92151611)
\curveto(714.76307746,77.69150641)(714.85307737,77.46150664)(714.9330835,77.23151611)
\curveto(715.01307721,77.0015071)(715.10307712,76.77150733)(715.2030835,76.54151611)
\curveto(715.28307694,76.34150776)(715.35807686,76.13650797)(715.4280835,75.92651611)
\curveto(715.50807671,75.71650839)(715.59307663,75.51150859)(715.6830835,75.31151611)
\curveto(715.98307624,74.58150952)(716.26807595,73.84151026)(716.5380835,73.09151611)
\curveto(716.8180754,72.35151175)(717.11307511,71.61651249)(717.4230835,70.88651611)
\curveto(717.46307476,70.79651331)(717.49307473,70.71151339)(717.5130835,70.63151611)
\curveto(717.54307468,70.55151355)(717.57307465,70.46651364)(717.6030835,70.37651611)
\curveto(717.71307451,70.11651399)(717.8180744,69.85151425)(717.9180835,69.58151611)
\curveto(718.02807419,69.31151479)(718.13807408,69.04651506)(718.2480835,68.78651611)
\moveto(715.0380835,72.43151611)
\curveto(715.12807709,72.46151164)(715.18307704,72.51151159)(715.2030835,72.58151611)
\curveto(715.23307699,72.65151145)(715.23807698,72.72651138)(715.2180835,72.80651611)
\curveto(715.20807701,72.89651121)(715.18307704,72.98151112)(715.1430835,73.06151611)
\curveto(715.11307711,73.15151095)(715.08307714,73.22651088)(715.0530835,73.28651611)
\curveto(715.03307719,73.32651078)(715.0230772,73.36151074)(715.0230835,73.39151611)
\curveto(715.0230772,73.42151068)(715.01307721,73.45651065)(714.9930835,73.49651611)
\lineto(714.9030835,73.73651611)
\curveto(714.88307734,73.82651028)(714.85307737,73.91651019)(714.8130835,74.00651611)
\curveto(714.66307756,74.36650974)(714.52807769,74.73150937)(714.4080835,75.10151611)
\curveto(714.29807792,75.48150862)(714.16807805,75.85150825)(714.0180835,76.21151611)
\curveto(713.96807825,76.32150778)(713.9230783,76.43150767)(713.8830835,76.54151611)
\curveto(713.85307837,76.65150745)(713.81307841,76.75650735)(713.7630835,76.85651611)
\curveto(713.74307848,76.9065072)(713.7180785,76.95150715)(713.6880835,76.99151611)
\curveto(713.66807855,77.04150706)(713.6180786,77.06650704)(713.5380835,77.06651611)
\curveto(713.5180787,77.04650706)(713.49807872,77.03150707)(713.4780835,77.02151611)
\curveto(713.45807876,77.01150709)(713.43807878,76.99650711)(713.4180835,76.97651611)
\curveto(713.37807884,76.92650718)(713.34807887,76.87150723)(713.3280835,76.81151611)
\curveto(713.30807891,76.76150734)(713.28807893,76.7065074)(713.2680835,76.64651611)
\curveto(713.218079,76.53650757)(713.17807904,76.42650768)(713.1480835,76.31651611)
\curveto(713.1180791,76.2065079)(713.07807914,76.09650801)(713.0280835,75.98651611)
\curveto(712.85807936,75.59650851)(712.70807951,75.2015089)(712.5780835,74.80151611)
\curveto(712.45807976,74.4015097)(712.3180799,74.01151009)(712.1580835,73.63151611)
\lineto(712.0980835,73.48151611)
\curveto(712.08808013,73.43151067)(712.07308015,73.38151072)(712.0530835,73.33151611)
\lineto(711.9630835,73.09151611)
\curveto(711.93308029,73.01151109)(711.90808031,72.93151117)(711.8880835,72.85151611)
\curveto(711.86808035,72.8015113)(711.85808036,72.74651136)(711.8580835,72.68651611)
\curveto(711.86808035,72.62651148)(711.88308034,72.57651153)(711.9030835,72.53651611)
\curveto(711.95308027,72.45651165)(712.05808016,72.41151169)(712.2180835,72.40151611)
\lineto(712.6680835,72.40151611)
\lineto(714.2730835,72.40151611)
\curveto(714.38307784,72.4015117)(714.5180777,72.39651171)(714.6780835,72.38651611)
\curveto(714.83807738,72.38651172)(714.95807726,72.4015117)(715.0380835,72.43151611)
}
}
{
\newrgbcolor{curcolor}{0 0 0}
\pscustom[linestyle=none,fillstyle=solid,fillcolor=curcolor]
{
\newpath
\moveto(726.324646,68.81651611)
\lineto(726.324646,68.42651611)
\curveto(726.32463812,68.3065158)(726.29963815,68.2065159)(726.249646,68.12651611)
\curveto(726.19963825,68.05651605)(726.11463833,68.01651609)(725.994646,68.00651611)
\lineto(725.649646,68.00651611)
\curveto(725.58963886,68.0065161)(725.52963892,68.0015161)(725.469646,67.99151611)
\curveto(725.41963903,67.99151611)(725.37463907,68.0015161)(725.334646,68.02151611)
\curveto(725.2446392,68.04151606)(725.18463926,68.08151602)(725.154646,68.14151611)
\curveto(725.11463933,68.19151591)(725.08963936,68.25151585)(725.079646,68.32151611)
\curveto(725.07963937,68.39151571)(725.06463938,68.46151564)(725.034646,68.53151611)
\curveto(725.02463942,68.55151555)(725.00963944,68.56651554)(724.989646,68.57651611)
\curveto(724.97963947,68.59651551)(724.96463948,68.61651549)(724.944646,68.63651611)
\curveto(724.8446396,68.64651546)(724.76463968,68.62651548)(724.704646,68.57651611)
\curveto(724.65463979,68.52651558)(724.59963985,68.47651563)(724.539646,68.42651611)
\curveto(724.33964011,68.27651583)(724.13964031,68.16151594)(723.939646,68.08151611)
\curveto(723.75964069,68.0015161)(723.5496409,67.94151616)(723.309646,67.90151611)
\curveto(723.07964137,67.86151624)(722.83964161,67.84151626)(722.589646,67.84151611)
\curveto(722.3496421,67.83151627)(722.10964234,67.84651626)(721.869646,67.88651611)
\curveto(721.62964282,67.91651619)(721.41964303,67.97151613)(721.239646,68.05151611)
\curveto(720.71964373,68.27151583)(720.29964415,68.56651554)(719.979646,68.93651611)
\curveto(719.65964479,69.31651479)(719.40964504,69.78651432)(719.229646,70.34651611)
\curveto(719.18964526,70.43651367)(719.15964529,70.52651358)(719.139646,70.61651611)
\curveto(719.12964532,70.71651339)(719.10964534,70.81651329)(719.079646,70.91651611)
\curveto(719.06964538,70.96651314)(719.06464538,71.01651309)(719.064646,71.06651611)
\curveto(719.06464538,71.11651299)(719.05964539,71.16651294)(719.049646,71.21651611)
\curveto(719.02964542,71.26651284)(719.01964543,71.31651279)(719.019646,71.36651611)
\curveto(719.02964542,71.42651268)(719.02964542,71.48151262)(719.019646,71.53151611)
\lineto(719.019646,71.68151611)
\curveto(718.99964545,71.73151237)(718.98964546,71.79651231)(718.989646,71.87651611)
\curveto(718.98964546,71.95651215)(718.99964545,72.02151208)(719.019646,72.07151611)
\lineto(719.019646,72.23651611)
\curveto(719.03964541,72.3065118)(719.0446454,72.37651173)(719.034646,72.44651611)
\curveto(719.03464541,72.52651158)(719.0446454,72.6015115)(719.064646,72.67151611)
\curveto(719.07464537,72.72151138)(719.07964537,72.76651134)(719.079646,72.80651611)
\curveto(719.07964537,72.84651126)(719.08464536,72.89151121)(719.094646,72.94151611)
\curveto(719.12464532,73.04151106)(719.1496453,73.13651097)(719.169646,73.22651611)
\curveto(719.18964526,73.32651078)(719.21464523,73.42151068)(719.244646,73.51151611)
\curveto(719.37464507,73.89151021)(719.53964491,74.23150987)(719.739646,74.53151611)
\curveto(719.9496445,74.84150926)(720.19964425,75.09650901)(720.489646,75.29651611)
\curveto(720.65964379,75.41650869)(720.83464361,75.51650859)(721.014646,75.59651611)
\curveto(721.20464324,75.67650843)(721.40964304,75.74650836)(721.629646,75.80651611)
\curveto(721.69964275,75.81650829)(721.76464268,75.82650828)(721.824646,75.83651611)
\curveto(721.89464255,75.84650826)(721.96464248,75.86150824)(722.034646,75.88151611)
\lineto(722.184646,75.88151611)
\curveto(722.26464218,75.9015082)(722.37964207,75.91150819)(722.529646,75.91151611)
\curveto(722.68964176,75.91150819)(722.80964164,75.9015082)(722.889646,75.88151611)
\curveto(722.92964152,75.87150823)(722.98464146,75.86650824)(723.054646,75.86651611)
\curveto(723.16464128,75.83650827)(723.27464117,75.81150829)(723.384646,75.79151611)
\curveto(723.49464095,75.78150832)(723.59964085,75.75150835)(723.699646,75.70151611)
\curveto(723.8496406,75.64150846)(723.98964046,75.57650853)(724.119646,75.50651611)
\curveto(724.25964019,75.43650867)(724.38964006,75.35650875)(724.509646,75.26651611)
\curveto(724.56963988,75.21650889)(724.62963982,75.16150894)(724.689646,75.10151611)
\curveto(724.75963969,75.05150905)(724.8496396,75.03650907)(724.959646,75.05651611)
\curveto(724.97963947,75.08650902)(724.99463945,75.11150899)(725.004646,75.13151611)
\curveto(725.02463942,75.15150895)(725.03963941,75.18150892)(725.049646,75.22151611)
\curveto(725.07963937,75.31150879)(725.08963936,75.42650868)(725.079646,75.56651611)
\lineto(725.079646,75.94151611)
\lineto(725.079646,77.66651611)
\lineto(725.079646,78.13151611)
\curveto(725.07963937,78.31150579)(725.10463934,78.44150566)(725.154646,78.52151611)
\curveto(725.19463925,78.59150551)(725.25463919,78.63650547)(725.334646,78.65651611)
\curveto(725.35463909,78.65650545)(725.37963907,78.65650545)(725.409646,78.65651611)
\curveto(725.43963901,78.66650544)(725.46463898,78.67150543)(725.484646,78.67151611)
\curveto(725.62463882,78.68150542)(725.76963868,78.68150542)(725.919646,78.67151611)
\curveto(726.07963837,78.67150543)(726.18963826,78.63150547)(726.249646,78.55151611)
\curveto(726.29963815,78.47150563)(726.32463812,78.37150573)(726.324646,78.25151611)
\lineto(726.324646,77.87651611)
\lineto(726.324646,68.81651611)
\moveto(725.109646,71.65151611)
\curveto(725.12963932,71.7015124)(725.13963931,71.76651234)(725.139646,71.84651611)
\curveto(725.13963931,71.93651217)(725.12963932,72.0065121)(725.109646,72.05651611)
\lineto(725.109646,72.28151611)
\curveto(725.08963936,72.37151173)(725.07463937,72.46151164)(725.064646,72.55151611)
\curveto(725.05463939,72.65151145)(725.03463941,72.74151136)(725.004646,72.82151611)
\curveto(724.98463946,72.9015112)(724.96463948,72.97651113)(724.944646,73.04651611)
\curveto(724.93463951,73.11651099)(724.91463953,73.18651092)(724.884646,73.25651611)
\curveto(724.76463968,73.55651055)(724.60963984,73.82151028)(724.419646,74.05151611)
\curveto(724.22964022,74.28150982)(723.98964046,74.46150964)(723.699646,74.59151611)
\curveto(723.59964085,74.64150946)(723.49464095,74.67650943)(723.384646,74.69651611)
\curveto(723.28464116,74.72650938)(723.17464127,74.75150935)(723.054646,74.77151611)
\curveto(722.97464147,74.79150931)(722.88464156,74.8015093)(722.784646,74.80151611)
\lineto(722.514646,74.80151611)
\curveto(722.46464198,74.79150931)(722.41964203,74.78150932)(722.379646,74.77151611)
\lineto(722.244646,74.77151611)
\curveto(722.16464228,74.75150935)(722.07964237,74.73150937)(721.989646,74.71151611)
\curveto(721.90964254,74.69150941)(721.82964262,74.66650944)(721.749646,74.63651611)
\curveto(721.42964302,74.49650961)(721.16964328,74.29150981)(720.969646,74.02151611)
\curveto(720.77964367,73.76151034)(720.62464382,73.45651065)(720.504646,73.10651611)
\curveto(720.46464398,72.99651111)(720.43464401,72.88151122)(720.414646,72.76151611)
\curveto(720.40464404,72.65151145)(720.38964406,72.54151156)(720.369646,72.43151611)
\curveto(720.36964408,72.39151171)(720.36464408,72.35151175)(720.354646,72.31151611)
\lineto(720.354646,72.20651611)
\curveto(720.33464411,72.15651195)(720.32464412,72.101512)(720.324646,72.04151611)
\curveto(720.33464411,71.98151212)(720.33964411,71.92651218)(720.339646,71.87651611)
\lineto(720.339646,71.54651611)
\curveto(720.33964411,71.44651266)(720.3496441,71.35151275)(720.369646,71.26151611)
\curveto(720.37964407,71.23151287)(720.38464406,71.18151292)(720.384646,71.11151611)
\curveto(720.40464404,71.04151306)(720.41964403,70.97151313)(720.429646,70.90151611)
\lineto(720.489646,70.69151611)
\curveto(720.59964385,70.34151376)(720.7496437,70.04151406)(720.939646,69.79151611)
\curveto(721.12964332,69.54151456)(721.36964308,69.33651477)(721.659646,69.17651611)
\curveto(721.7496427,69.12651498)(721.83964261,69.08651502)(721.929646,69.05651611)
\curveto(722.01964243,69.02651508)(722.11964233,68.99651511)(722.229646,68.96651611)
\curveto(722.27964217,68.94651516)(722.32964212,68.94151516)(722.379646,68.95151611)
\curveto(722.43964201,68.96151514)(722.49464195,68.95651515)(722.544646,68.93651611)
\curveto(722.58464186,68.92651518)(722.62464182,68.92151518)(722.664646,68.92151611)
\lineto(722.799646,68.92151611)
\lineto(722.934646,68.92151611)
\curveto(722.96464148,68.93151517)(723.01464143,68.93651517)(723.084646,68.93651611)
\curveto(723.16464128,68.95651515)(723.2446412,68.97151513)(723.324646,68.98151611)
\curveto(723.40464104,69.0015151)(723.47964097,69.02651508)(723.549646,69.05651611)
\curveto(723.87964057,69.19651491)(724.1446403,69.37151473)(724.344646,69.58151611)
\curveto(724.55463989,69.8015143)(724.72963972,70.07651403)(724.869646,70.40651611)
\curveto(724.91963953,70.51651359)(724.95463949,70.62651348)(724.974646,70.73651611)
\curveto(724.99463945,70.84651326)(725.01963943,70.95651315)(725.049646,71.06651611)
\curveto(725.06963938,71.106513)(725.07963937,71.14151296)(725.079646,71.17151611)
\curveto(725.07963937,71.21151289)(725.08463936,71.25151285)(725.094646,71.29151611)
\curveto(725.10463934,71.35151275)(725.10463934,71.41151269)(725.094646,71.47151611)
\curveto(725.09463935,71.53151257)(725.09963935,71.59151251)(725.109646,71.65151611)
}
}
{
\newrgbcolor{curcolor}{0 0 0}
\pscustom[linestyle=none,fillstyle=solid,fillcolor=curcolor]
{
\newpath
\moveto(731.960896,75.91151611)
\curveto(732.34089101,75.92150818)(732.66089069,75.88150822)(732.920896,75.79151611)
\curveto(733.19089016,75.7015084)(733.43588992,75.57150853)(733.655896,75.40151611)
\curveto(733.73588962,75.35150875)(733.80088955,75.28150882)(733.850896,75.19151611)
\curveto(733.91088944,75.11150899)(733.97588938,75.03650907)(734.045896,74.96651611)
\curveto(734.06588929,74.94650916)(734.09588926,74.92150918)(734.135896,74.89151611)
\curveto(734.17588918,74.86150924)(734.22588913,74.85150925)(734.285896,74.86151611)
\curveto(734.38588897,74.89150921)(734.47088888,74.95150915)(734.540896,75.04151611)
\curveto(734.62088873,75.14150896)(734.70088865,75.21650889)(734.780896,75.26651611)
\curveto(734.92088843,75.37650873)(735.06588829,75.47150863)(735.215896,75.55151611)
\curveto(735.36588799,75.64150846)(735.53088782,75.71650839)(735.710896,75.77651611)
\curveto(735.79088756,75.8065083)(735.87588748,75.82650828)(735.965896,75.83651611)
\curveto(736.06588729,75.85650825)(736.16088719,75.87650823)(736.250896,75.89651611)
\curveto(736.30088705,75.9065082)(736.34588701,75.91150819)(736.385896,75.91151611)
\lineto(736.535896,75.91151611)
\curveto(736.58588677,75.93150817)(736.6558867,75.93650817)(736.745896,75.92651611)
\curveto(736.83588652,75.92650818)(736.90088645,75.92150818)(736.940896,75.91151611)
\curveto(736.99088636,75.9015082)(737.06588629,75.89650821)(737.165896,75.89651611)
\curveto(737.2558861,75.87650823)(737.34088601,75.85650825)(737.420896,75.83651611)
\curveto(737.51088584,75.82650828)(737.59588576,75.8065083)(737.675896,75.77651611)
\curveto(737.72588563,75.75650835)(737.77088558,75.74150836)(737.810896,75.73151611)
\curveto(737.86088549,75.73150837)(737.91088544,75.72150838)(737.960896,75.70151611)
\curveto(738.46088489,75.48150862)(738.80588455,75.14150896)(738.995896,74.68151611)
\curveto(739.03588432,74.6015095)(739.06588429,74.51150959)(739.085896,74.41151611)
\curveto(739.10588425,74.32150978)(739.12588423,74.22150988)(739.145896,74.11151611)
\curveto(739.16588419,74.08151002)(739.17088418,74.04651006)(739.160896,74.00651611)
\curveto(739.16088419,73.97651013)(739.16588419,73.94651016)(739.175896,73.91651611)
\lineto(739.175896,73.78151611)
\curveto(739.18588417,73.74151036)(739.18588417,73.69651041)(739.175896,73.64651611)
\curveto(739.17588418,73.59651051)(739.17588418,73.54651056)(739.175896,73.49651611)
\lineto(739.175896,72.91151611)
\lineto(739.175896,71.95151611)
\lineto(739.175896,69.10151611)
\curveto(739.17588418,68.94151516)(739.17588418,68.75151535)(739.175896,68.53151611)
\curveto(739.18588417,68.31151579)(739.14588421,68.16651594)(739.055896,68.09651611)
\curveto(739.01588434,68.06651604)(738.9508844,68.04151606)(738.860896,68.02151611)
\curveto(738.77088458,68.01151609)(738.67588468,68.0065161)(738.575896,68.00651611)
\curveto(738.47588488,68.0065161)(738.37588498,68.01151609)(738.275896,68.02151611)
\curveto(738.18588517,68.03151607)(738.12088523,68.05151605)(738.080896,68.08151611)
\curveto(738.02088533,68.11151599)(737.98088537,68.17151593)(737.960896,68.26151611)
\curveto(737.94088541,68.32151578)(737.93588542,68.38151572)(737.945896,68.44151611)
\curveto(737.9558854,68.51151559)(737.9508854,68.57651553)(737.930896,68.63651611)
\curveto(737.92088543,68.68651542)(737.91588544,68.74151536)(737.915896,68.80151611)
\curveto(737.92588543,68.87151523)(737.93088542,68.93651517)(737.930896,68.99651611)
\lineto(737.930896,69.67151611)
\lineto(737.930896,72.53651611)
\curveto(737.93088542,72.86651124)(737.92088543,73.17651093)(737.900896,73.46651611)
\curveto(737.89088546,73.76651034)(737.82088553,74.01651009)(737.690896,74.21651611)
\curveto(737.54088581,74.45650965)(737.31088604,74.63150947)(737.000896,74.74151611)
\curveto(736.94088641,74.76150934)(736.87588648,74.77150933)(736.805896,74.77151611)
\curveto(736.74588661,74.78150932)(736.68088667,74.79650931)(736.610896,74.81651611)
\curveto(736.57088678,74.82650928)(736.50588685,74.82650928)(736.415896,74.81651611)
\curveto(736.32588703,74.81650929)(736.26588709,74.81150929)(736.235896,74.80151611)
\curveto(736.18588717,74.79150931)(736.13588722,74.78650932)(736.085896,74.78651611)
\curveto(736.03588732,74.79650931)(735.98588737,74.79150931)(735.935896,74.77151611)
\curveto(735.79588756,74.74150936)(735.66088769,74.7015094)(735.530896,74.65151611)
\curveto(735.01088834,74.43150967)(734.66088869,74.04651006)(734.480896,73.49651611)
\curveto(734.43088892,73.32651078)(734.40088895,73.13151097)(734.390896,72.91151611)
\lineto(734.390896,72.23651611)
\lineto(734.390896,70.27151611)
\lineto(734.390896,68.81651611)
\lineto(734.390896,68.44151611)
\curveto(734.39088896,68.32151578)(734.36588899,68.22651588)(734.315896,68.15651611)
\curveto(734.26588909,68.07651603)(734.18088917,68.03151607)(734.060896,68.02151611)
\curveto(733.94088941,68.01151609)(733.81588954,68.0065161)(733.685896,68.00651611)
\curveto(733.51588984,68.0065161)(733.39088996,68.02651608)(733.310896,68.06651611)
\curveto(733.22089013,68.11651599)(733.16589019,68.19651591)(733.145896,68.30651611)
\curveto(733.13589022,68.42651568)(733.13089022,68.55651555)(733.130896,68.69651611)
\lineto(733.130896,70.12151611)
\lineto(733.130896,72.59651611)
\curveto(733.13089022,72.91651119)(733.12089023,73.21151089)(733.100896,73.48151611)
\curveto(733.08089027,73.76151034)(733.01089034,74.0015101)(732.890896,74.20151611)
\curveto(732.78089057,74.38150972)(732.6558907,74.51150959)(732.515896,74.59151611)
\curveto(732.37589098,74.68150942)(732.18589117,74.75150935)(731.945896,74.80151611)
\curveto(731.90589145,74.81150929)(731.86089149,74.81650929)(731.810896,74.81651611)
\lineto(731.675896,74.81651611)
\curveto(731.4558919,74.81650929)(731.26089209,74.79150931)(731.090896,74.74151611)
\curveto(730.93089242,74.69150941)(730.78589257,74.62650948)(730.655896,74.54651611)
\curveto(730.14589321,74.23650987)(729.80589355,73.77151033)(729.635896,73.15151611)
\curveto(729.59589376,73.02151108)(729.57589378,72.87151123)(729.575896,72.70151611)
\curveto(729.58589377,72.54151156)(729.59089376,72.38151172)(729.590896,72.22151611)
\lineto(729.590896,70.52651611)
\lineto(729.590896,68.87651611)
\lineto(729.590896,68.47151611)
\curveto(729.59089376,68.33151577)(729.56089379,68.22151588)(729.500896,68.14151611)
\curveto(729.4508939,68.07151603)(729.37589398,68.03151607)(729.275896,68.02151611)
\curveto(729.17589418,68.01151609)(729.07089428,68.0065161)(728.960896,68.00651611)
\lineto(728.735896,68.00651611)
\curveto(728.67589468,68.02651608)(728.61589474,68.04151606)(728.555896,68.05151611)
\curveto(728.50589485,68.06151604)(728.46089489,68.09151601)(728.420896,68.14151611)
\curveto(728.37089498,68.2015159)(728.34589501,68.27651583)(728.345896,68.36651611)
\lineto(728.345896,68.68151611)
\lineto(728.345896,69.65651611)
\lineto(728.345896,73.94651611)
\lineto(728.345896,75.05651611)
\lineto(728.345896,75.34151611)
\curveto(728.34589501,75.44150866)(728.36589499,75.52150858)(728.405896,75.58151611)
\curveto(728.43589492,75.64150846)(728.48089487,75.68150842)(728.540896,75.70151611)
\curveto(728.62089473,75.73150837)(728.74589461,75.74650836)(728.915896,75.74651611)
\curveto(729.09589426,75.74650836)(729.22589413,75.73150837)(729.305896,75.70151611)
\curveto(729.38589397,75.66150844)(729.44089391,75.61150849)(729.470896,75.55151611)
\curveto(729.49089386,75.5015086)(729.50089385,75.44150866)(729.500896,75.37151611)
\curveto(729.51089384,75.3015088)(729.52089383,75.23650887)(729.530896,75.17651611)
\curveto(729.54089381,75.11650899)(729.56089379,75.06650904)(729.590896,75.02651611)
\curveto(729.62089373,74.98650912)(729.67089368,74.96650914)(729.740896,74.96651611)
\curveto(729.76089359,74.98650912)(729.78089357,74.99650911)(729.800896,74.99651611)
\curveto(729.83089352,74.99650911)(729.8558935,75.0065091)(729.875896,75.02651611)
\curveto(729.93589342,75.07650903)(729.99089336,75.12650898)(730.040896,75.17651611)
\lineto(730.220896,75.32651611)
\curveto(730.44089291,75.48650862)(730.69089266,75.62650848)(730.970896,75.74651611)
\curveto(731.07089228,75.78650832)(731.17089218,75.81150829)(731.270896,75.82151611)
\curveto(731.37089198,75.84150826)(731.47589188,75.86650824)(731.585896,75.89651611)
\lineto(731.765896,75.89651611)
\curveto(731.83589152,75.9065082)(731.90089145,75.91150819)(731.960896,75.91151611)
}
}
{
\newrgbcolor{curcolor}{0 0 0}
\pscustom[linestyle=none,fillstyle=solid,fillcolor=curcolor]
{
\newpath
\moveto(741.35863037,77.23151611)
\curveto(741.27862925,77.29150681)(741.2336293,77.39650671)(741.22363037,77.54651611)
\lineto(741.22363037,78.01151611)
\lineto(741.22363037,78.26651611)
\curveto(741.22362931,78.35650575)(741.23862929,78.43150567)(741.26863037,78.49151611)
\curveto(741.30862922,78.57150553)(741.38862914,78.63150547)(741.50863037,78.67151611)
\curveto(741.528629,78.68150542)(741.54862898,78.68150542)(741.56863037,78.67151611)
\curveto(741.59862893,78.67150543)(741.62362891,78.67650543)(741.64363037,78.68651611)
\curveto(741.81362872,78.68650542)(741.97362856,78.68150542)(742.12363037,78.67151611)
\curveto(742.27362826,78.66150544)(742.37362816,78.6015055)(742.42363037,78.49151611)
\curveto(742.45362808,78.43150567)(742.46862806,78.35650575)(742.46863037,78.26651611)
\lineto(742.46863037,78.01151611)
\curveto(742.46862806,77.83150627)(742.46362807,77.66150644)(742.45363037,77.50151611)
\curveto(742.45362808,77.34150676)(742.38862814,77.23650687)(742.25863037,77.18651611)
\curveto(742.20862832,77.16650694)(742.15362838,77.15650695)(742.09363037,77.15651611)
\lineto(741.92863037,77.15651611)
\lineto(741.61363037,77.15651611)
\curveto(741.51362902,77.15650695)(741.4286291,77.18150692)(741.35863037,77.23151611)
\moveto(742.46863037,68.72651611)
\lineto(742.46863037,68.41151611)
\curveto(742.47862805,68.31151579)(742.45862807,68.23151587)(742.40863037,68.17151611)
\curveto(742.37862815,68.11151599)(742.3336282,68.07151603)(742.27363037,68.05151611)
\curveto(742.21362832,68.04151606)(742.14362839,68.02651608)(742.06363037,68.00651611)
\lineto(741.83863037,68.00651611)
\curveto(741.70862882,68.0065161)(741.59362894,68.01151609)(741.49363037,68.02151611)
\curveto(741.40362913,68.04151606)(741.3336292,68.09151601)(741.28363037,68.17151611)
\curveto(741.24362929,68.23151587)(741.22362931,68.3065158)(741.22363037,68.39651611)
\lineto(741.22363037,68.68151611)
\lineto(741.22363037,75.02651611)
\lineto(741.22363037,75.34151611)
\curveto(741.22362931,75.45150865)(741.24862928,75.53650857)(741.29863037,75.59651611)
\curveto(741.3286292,75.64650846)(741.36862916,75.67650843)(741.41863037,75.68651611)
\curveto(741.46862906,75.69650841)(741.52362901,75.71150839)(741.58363037,75.73151611)
\curveto(741.60362893,75.73150837)(741.62362891,75.72650838)(741.64363037,75.71651611)
\curveto(741.67362886,75.71650839)(741.69862883,75.72150838)(741.71863037,75.73151611)
\curveto(741.84862868,75.73150837)(741.97862855,75.72650838)(742.10863037,75.71651611)
\curveto(742.24862828,75.71650839)(742.34362819,75.67650843)(742.39363037,75.59651611)
\curveto(742.44362809,75.53650857)(742.46862806,75.45650865)(742.46863037,75.35651611)
\lineto(742.46863037,75.07151611)
\lineto(742.46863037,68.72651611)
}
}
{
\newrgbcolor{curcolor}{0 0 0}
\pscustom[linestyle=none,fillstyle=solid,fillcolor=curcolor]
{
\newpath
\moveto(748.10347412,75.88151611)
\curveto(748.73346889,75.9015082)(749.23846838,75.81650829)(749.61847412,75.62651611)
\curveto(749.99846762,75.43650867)(750.30346732,75.15150895)(750.53347412,74.77151611)
\curveto(750.59346703,74.67150943)(750.63846698,74.56150954)(750.66847412,74.44151611)
\curveto(750.70846691,74.33150977)(750.74346688,74.21650989)(750.77347412,74.09651611)
\curveto(750.8234668,73.9065102)(750.85346677,73.7015104)(750.86347412,73.48151611)
\curveto(750.87346675,73.26151084)(750.87846674,73.03651107)(750.87847412,72.80651611)
\lineto(750.87847412,71.20151611)
\lineto(750.87847412,68.86151611)
\curveto(750.87846674,68.69151541)(750.87346675,68.52151558)(750.86347412,68.35151611)
\curveto(750.86346676,68.18151592)(750.79846682,68.07151603)(750.66847412,68.02151611)
\curveto(750.618467,68.0015161)(750.56346706,67.99151611)(750.50347412,67.99151611)
\curveto(750.45346717,67.98151612)(750.39846722,67.97651613)(750.33847412,67.97651611)
\curveto(750.20846741,67.97651613)(750.08346754,67.98151612)(749.96347412,67.99151611)
\curveto(749.84346778,67.99151611)(749.75846786,68.03151607)(749.70847412,68.11151611)
\curveto(749.65846796,68.18151592)(749.63346799,68.27151583)(749.63347412,68.38151611)
\lineto(749.63347412,68.71151611)
\lineto(749.63347412,70.00151611)
\lineto(749.63347412,72.44651611)
\curveto(749.63346799,72.71651139)(749.62846799,72.98151112)(749.61847412,73.24151611)
\curveto(749.60846801,73.51151059)(749.56346806,73.74151036)(749.48347412,73.93151611)
\curveto(749.40346822,74.13150997)(749.28346834,74.29150981)(749.12347412,74.41151611)
\curveto(748.96346866,74.54150956)(748.77846884,74.64150946)(748.56847412,74.71151611)
\curveto(748.50846911,74.73150937)(748.44346918,74.74150936)(748.37347412,74.74151611)
\curveto(748.31346931,74.75150935)(748.25346937,74.76650934)(748.19347412,74.78651611)
\curveto(748.14346948,74.79650931)(748.06346956,74.79650931)(747.95347412,74.78651611)
\curveto(747.85346977,74.78650932)(747.78346984,74.78150932)(747.74347412,74.77151611)
\curveto(747.70346992,74.75150935)(747.66846995,74.74150936)(747.63847412,74.74151611)
\curveto(747.60847001,74.75150935)(747.57347005,74.75150935)(747.53347412,74.74151611)
\curveto(747.40347022,74.71150939)(747.27847034,74.67650943)(747.15847412,74.63651611)
\curveto(747.04847057,74.6065095)(746.94347068,74.56150954)(746.84347412,74.50151611)
\curveto(746.80347082,74.48150962)(746.76847085,74.46150964)(746.73847412,74.44151611)
\curveto(746.70847091,74.42150968)(746.67347095,74.4015097)(746.63347412,74.38151611)
\curveto(746.28347134,74.13150997)(746.02847159,73.75651035)(745.86847412,73.25651611)
\curveto(745.83847178,73.17651093)(745.8184718,73.09151101)(745.80847412,73.00151611)
\curveto(745.79847182,72.92151118)(745.78347184,72.84151126)(745.76347412,72.76151611)
\curveto(745.74347188,72.71151139)(745.73847188,72.66151144)(745.74847412,72.61151611)
\curveto(745.75847186,72.57151153)(745.75347187,72.53151157)(745.73347412,72.49151611)
\lineto(745.73347412,72.17651611)
\curveto(745.7234719,72.14651196)(745.7184719,72.11151199)(745.71847412,72.07151611)
\curveto(745.72847189,72.03151207)(745.73347189,71.98651212)(745.73347412,71.93651611)
\lineto(745.73347412,71.48651611)
\lineto(745.73347412,70.04651611)
\lineto(745.73347412,68.72651611)
\lineto(745.73347412,68.38151611)
\curveto(745.73347189,68.27151583)(745.70847191,68.18151592)(745.65847412,68.11151611)
\curveto(745.60847201,68.03151607)(745.5184721,67.99151611)(745.38847412,67.99151611)
\curveto(745.26847235,67.98151612)(745.14347248,67.97651613)(745.01347412,67.97651611)
\curveto(744.93347269,67.97651613)(744.85847276,67.98151612)(744.78847412,67.99151611)
\curveto(744.7184729,68.0015161)(744.65847296,68.02651608)(744.60847412,68.06651611)
\curveto(744.52847309,68.11651599)(744.48847313,68.21151589)(744.48847412,68.35151611)
\lineto(744.48847412,68.75651611)
\lineto(744.48847412,70.52651611)
\lineto(744.48847412,74.15651611)
\lineto(744.48847412,75.07151611)
\lineto(744.48847412,75.34151611)
\curveto(744.48847313,75.43150867)(744.50847311,75.5015086)(744.54847412,75.55151611)
\curveto(744.57847304,75.61150849)(744.62847299,75.65150845)(744.69847412,75.67151611)
\curveto(744.73847288,75.68150842)(744.79347283,75.69150841)(744.86347412,75.70151611)
\curveto(744.94347268,75.71150839)(745.0234726,75.71650839)(745.10347412,75.71651611)
\curveto(745.18347244,75.71650839)(745.25847236,75.71150839)(745.32847412,75.70151611)
\curveto(745.40847221,75.69150841)(745.46347216,75.67650843)(745.49347412,75.65651611)
\curveto(745.60347202,75.58650852)(745.65347197,75.49650861)(745.64347412,75.38651611)
\curveto(745.63347199,75.28650882)(745.64847197,75.17150893)(745.68847412,75.04151611)
\curveto(745.70847191,74.98150912)(745.74847187,74.93150917)(745.80847412,74.89151611)
\curveto(745.92847169,74.88150922)(746.0234716,74.92650918)(746.09347412,75.02651611)
\curveto(746.17347145,75.12650898)(746.25347137,75.2065089)(746.33347412,75.26651611)
\curveto(746.47347115,75.36650874)(746.61347101,75.45650865)(746.75347412,75.53651611)
\curveto(746.90347072,75.62650848)(747.07347055,75.7015084)(747.26347412,75.76151611)
\curveto(747.34347028,75.79150831)(747.42847019,75.81150829)(747.51847412,75.82151611)
\curveto(747.61847,75.83150827)(747.71346991,75.84650826)(747.80347412,75.86651611)
\curveto(747.85346977,75.87650823)(747.90346972,75.88150822)(747.95347412,75.88151611)
\lineto(748.10347412,75.88151611)
}
}
{
\newrgbcolor{curcolor}{0 0 0}
\pscustom[linestyle=none,fillstyle=solid,fillcolor=curcolor]
{
\newpath
\moveto(753.0480835,77.23151611)
\curveto(752.96808238,77.29150681)(752.92308242,77.39650671)(752.9130835,77.54651611)
\lineto(752.9130835,78.01151611)
\lineto(752.9130835,78.26651611)
\curveto(752.91308243,78.35650575)(752.92808242,78.43150567)(752.9580835,78.49151611)
\curveto(752.99808235,78.57150553)(753.07808227,78.63150547)(753.1980835,78.67151611)
\curveto(753.21808213,78.68150542)(753.23808211,78.68150542)(753.2580835,78.67151611)
\curveto(753.28808206,78.67150543)(753.31308203,78.67650543)(753.3330835,78.68651611)
\curveto(753.50308184,78.68650542)(753.66308168,78.68150542)(753.8130835,78.67151611)
\curveto(753.96308138,78.66150544)(754.06308128,78.6015055)(754.1130835,78.49151611)
\curveto(754.1430812,78.43150567)(754.15808119,78.35650575)(754.1580835,78.26651611)
\lineto(754.1580835,78.01151611)
\curveto(754.15808119,77.83150627)(754.15308119,77.66150644)(754.1430835,77.50151611)
\curveto(754.1430812,77.34150676)(754.07808127,77.23650687)(753.9480835,77.18651611)
\curveto(753.89808145,77.16650694)(753.8430815,77.15650695)(753.7830835,77.15651611)
\lineto(753.6180835,77.15651611)
\lineto(753.3030835,77.15651611)
\curveto(753.20308214,77.15650695)(753.11808223,77.18150692)(753.0480835,77.23151611)
\moveto(754.1580835,68.72651611)
\lineto(754.1580835,68.41151611)
\curveto(754.16808118,68.31151579)(754.1480812,68.23151587)(754.0980835,68.17151611)
\curveto(754.06808128,68.11151599)(754.02308132,68.07151603)(753.9630835,68.05151611)
\curveto(753.90308144,68.04151606)(753.83308151,68.02651608)(753.7530835,68.00651611)
\lineto(753.5280835,68.00651611)
\curveto(753.39808195,68.0065161)(753.28308206,68.01151609)(753.1830835,68.02151611)
\curveto(753.09308225,68.04151606)(753.02308232,68.09151601)(752.9730835,68.17151611)
\curveto(752.93308241,68.23151587)(752.91308243,68.3065158)(752.9130835,68.39651611)
\lineto(752.9130835,68.68151611)
\lineto(752.9130835,75.02651611)
\lineto(752.9130835,75.34151611)
\curveto(752.91308243,75.45150865)(752.93808241,75.53650857)(752.9880835,75.59651611)
\curveto(753.01808233,75.64650846)(753.05808229,75.67650843)(753.1080835,75.68651611)
\curveto(753.15808219,75.69650841)(753.21308213,75.71150839)(753.2730835,75.73151611)
\curveto(753.29308205,75.73150837)(753.31308203,75.72650838)(753.3330835,75.71651611)
\curveto(753.36308198,75.71650839)(753.38808196,75.72150838)(753.4080835,75.73151611)
\curveto(753.53808181,75.73150837)(753.66808168,75.72650838)(753.7980835,75.71651611)
\curveto(753.93808141,75.71650839)(754.03308131,75.67650843)(754.0830835,75.59651611)
\curveto(754.13308121,75.53650857)(754.15808119,75.45650865)(754.1580835,75.35651611)
\lineto(754.1580835,75.07151611)
\lineto(754.1580835,68.72651611)
}
}
{
\newrgbcolor{curcolor}{0 0 0}
\pscustom[linestyle=none,fillstyle=solid,fillcolor=curcolor]
{
\newpath
\moveto(758.53292725,75.91151611)
\curveto(759.25292318,75.92150818)(759.85792258,75.83650827)(760.34792725,75.65651611)
\curveto(760.8379216,75.48650862)(761.21792122,75.18150892)(761.48792725,74.74151611)
\curveto(761.55792088,74.63150947)(761.61292082,74.51650959)(761.65292725,74.39651611)
\curveto(761.69292074,74.28650982)(761.7329207,74.16150994)(761.77292725,74.02151611)
\curveto(761.79292064,73.95151015)(761.79792064,73.87651023)(761.78792725,73.79651611)
\curveto(761.77792066,73.72651038)(761.76292067,73.67151043)(761.74292725,73.63151611)
\curveto(761.72292071,73.61151049)(761.69792074,73.59151051)(761.66792725,73.57151611)
\curveto(761.6379208,73.56151054)(761.61292082,73.54651056)(761.59292725,73.52651611)
\curveto(761.54292089,73.5065106)(761.49292094,73.5015106)(761.44292725,73.51151611)
\curveto(761.39292104,73.52151058)(761.34292109,73.52151058)(761.29292725,73.51151611)
\curveto(761.21292122,73.49151061)(761.10792133,73.48651062)(760.97792725,73.49651611)
\curveto(760.84792159,73.51651059)(760.75792168,73.54151056)(760.70792725,73.57151611)
\curveto(760.62792181,73.62151048)(760.57292186,73.68651042)(760.54292725,73.76651611)
\curveto(760.52292191,73.85651025)(760.48792195,73.94151016)(760.43792725,74.02151611)
\curveto(760.34792209,74.18150992)(760.22292221,74.32650978)(760.06292725,74.45651611)
\curveto(759.95292248,74.53650957)(759.8329226,74.59650951)(759.70292725,74.63651611)
\curveto(759.57292286,74.67650943)(759.432923,74.71650939)(759.28292725,74.75651611)
\curveto(759.2329232,74.77650933)(759.18292325,74.78150932)(759.13292725,74.77151611)
\curveto(759.08292335,74.77150933)(759.0329234,74.77650933)(758.98292725,74.78651611)
\curveto(758.92292351,74.8065093)(758.84792359,74.81650929)(758.75792725,74.81651611)
\curveto(758.66792377,74.81650929)(758.59292384,74.8065093)(758.53292725,74.78651611)
\lineto(758.44292725,74.78651611)
\lineto(758.29292725,74.75651611)
\curveto(758.24292419,74.75650935)(758.19292424,74.75150935)(758.14292725,74.74151611)
\curveto(757.88292455,74.68150942)(757.66792477,74.59650951)(757.49792725,74.48651611)
\curveto(757.32792511,74.37650973)(757.21292522,74.19150991)(757.15292725,73.93151611)
\curveto(757.1329253,73.86151024)(757.12792531,73.79151031)(757.13792725,73.72151611)
\curveto(757.15792528,73.65151045)(757.17792526,73.59151051)(757.19792725,73.54151611)
\curveto(757.25792518,73.39151071)(757.32792511,73.28151082)(757.40792725,73.21151611)
\curveto(757.49792494,73.15151095)(757.60792483,73.08151102)(757.73792725,73.00151611)
\curveto(757.89792454,72.9015112)(758.07792436,72.82651128)(758.27792725,72.77651611)
\curveto(758.47792396,72.73651137)(758.67792376,72.68651142)(758.87792725,72.62651611)
\curveto(759.00792343,72.58651152)(759.1379233,72.55651155)(759.26792725,72.53651611)
\curveto(759.39792304,72.51651159)(759.52792291,72.48651162)(759.65792725,72.44651611)
\curveto(759.86792257,72.38651172)(760.07292236,72.32651178)(760.27292725,72.26651611)
\curveto(760.47292196,72.21651189)(760.67292176,72.15151195)(760.87292725,72.07151611)
\lineto(761.02292725,72.01151611)
\curveto(761.07292136,71.99151211)(761.12292131,71.96651214)(761.17292725,71.93651611)
\curveto(761.37292106,71.81651229)(761.54792089,71.68151242)(761.69792725,71.53151611)
\curveto(761.84792059,71.38151272)(761.97292046,71.19151291)(762.07292725,70.96151611)
\curveto(762.09292034,70.89151321)(762.11292032,70.79651331)(762.13292725,70.67651611)
\curveto(762.15292028,70.6065135)(762.16292027,70.53151357)(762.16292725,70.45151611)
\curveto(762.17292026,70.38151372)(762.17792026,70.3015138)(762.17792725,70.21151611)
\lineto(762.17792725,70.06151611)
\curveto(762.15792028,69.99151411)(762.14792029,69.92151418)(762.14792725,69.85151611)
\curveto(762.14792029,69.78151432)(762.1379203,69.71151439)(762.11792725,69.64151611)
\curveto(762.08792035,69.53151457)(762.05292038,69.42651468)(762.01292725,69.32651611)
\curveto(761.97292046,69.22651488)(761.92792051,69.13651497)(761.87792725,69.05651611)
\curveto(761.71792072,68.79651531)(761.51292092,68.58651552)(761.26292725,68.42651611)
\curveto(761.01292142,68.27651583)(760.7329217,68.14651596)(760.42292725,68.03651611)
\curveto(760.3329221,68.0065161)(760.2379222,67.98651612)(760.13792725,67.97651611)
\curveto(760.04792239,67.95651615)(759.95792248,67.93151617)(759.86792725,67.90151611)
\curveto(759.76792267,67.88151622)(759.66792277,67.87151623)(759.56792725,67.87151611)
\curveto(759.46792297,67.87151623)(759.36792307,67.86151624)(759.26792725,67.84151611)
\lineto(759.11792725,67.84151611)
\curveto(759.06792337,67.83151627)(758.99792344,67.82651628)(758.90792725,67.82651611)
\curveto(758.81792362,67.82651628)(758.74792369,67.83151627)(758.69792725,67.84151611)
\lineto(758.53292725,67.84151611)
\curveto(758.47292396,67.86151624)(758.40792403,67.87151623)(758.33792725,67.87151611)
\curveto(758.26792417,67.86151624)(758.20792423,67.86651624)(758.15792725,67.88651611)
\curveto(758.10792433,67.89651621)(758.04292439,67.9015162)(757.96292725,67.90151611)
\lineto(757.72292725,67.96151611)
\curveto(757.65292478,67.97151613)(757.57792486,67.99151611)(757.49792725,68.02151611)
\curveto(757.18792525,68.12151598)(756.91792552,68.24651586)(756.68792725,68.39651611)
\curveto(756.45792598,68.54651556)(756.25792618,68.74151536)(756.08792725,68.98151611)
\curveto(755.99792644,69.11151499)(755.92292651,69.24651486)(755.86292725,69.38651611)
\curveto(755.80292663,69.52651458)(755.74792669,69.68151442)(755.69792725,69.85151611)
\curveto(755.67792676,69.91151419)(755.66792677,69.98151412)(755.66792725,70.06151611)
\curveto(755.67792676,70.15151395)(755.69292674,70.22151388)(755.71292725,70.27151611)
\curveto(755.74292669,70.31151379)(755.79292664,70.35151375)(755.86292725,70.39151611)
\curveto(755.91292652,70.41151369)(755.98292645,70.42151368)(756.07292725,70.42151611)
\curveto(756.16292627,70.43151367)(756.25292618,70.43151367)(756.34292725,70.42151611)
\curveto(756.432926,70.41151369)(756.51792592,70.39651371)(756.59792725,70.37651611)
\curveto(756.68792575,70.36651374)(756.74792569,70.35151375)(756.77792725,70.33151611)
\curveto(756.84792559,70.28151382)(756.89292554,70.2065139)(756.91292725,70.10651611)
\curveto(756.94292549,70.01651409)(756.97792546,69.93151417)(757.01792725,69.85151611)
\curveto(757.11792532,69.63151447)(757.25292518,69.46151464)(757.42292725,69.34151611)
\curveto(757.54292489,69.25151485)(757.67792476,69.18151492)(757.82792725,69.13151611)
\curveto(757.97792446,69.08151502)(758.1379243,69.03151507)(758.30792725,68.98151611)
\lineto(758.62292725,68.93651611)
\lineto(758.71292725,68.93651611)
\curveto(758.78292365,68.91651519)(758.87292356,68.9065152)(758.98292725,68.90651611)
\curveto(759.10292333,68.9065152)(759.20292323,68.91651519)(759.28292725,68.93651611)
\curveto(759.35292308,68.93651517)(759.40792303,68.94151516)(759.44792725,68.95151611)
\curveto(759.50792293,68.96151514)(759.56792287,68.96651514)(759.62792725,68.96651611)
\curveto(759.68792275,68.97651513)(759.74292269,68.98651512)(759.79292725,68.99651611)
\curveto(760.08292235,69.07651503)(760.31292212,69.18151492)(760.48292725,69.31151611)
\curveto(760.65292178,69.44151466)(760.77292166,69.66151444)(760.84292725,69.97151611)
\curveto(760.86292157,70.02151408)(760.86792157,70.07651403)(760.85792725,70.13651611)
\curveto(760.84792159,70.19651391)(760.8379216,70.24151386)(760.82792725,70.27151611)
\curveto(760.77792166,70.46151364)(760.70792173,70.6015135)(760.61792725,70.69151611)
\curveto(760.52792191,70.79151331)(760.41292202,70.88151322)(760.27292725,70.96151611)
\curveto(760.18292225,71.02151308)(760.08292235,71.07151303)(759.97292725,71.11151611)
\lineto(759.64292725,71.23151611)
\curveto(759.61292282,71.24151286)(759.58292285,71.24651286)(759.55292725,71.24651611)
\curveto(759.5329229,71.24651286)(759.50792293,71.25651285)(759.47792725,71.27651611)
\curveto(759.1379233,71.38651272)(758.78292365,71.46651264)(758.41292725,71.51651611)
\curveto(758.05292438,71.57651253)(757.71292472,71.67151243)(757.39292725,71.80151611)
\curveto(757.29292514,71.84151226)(757.19792524,71.87651223)(757.10792725,71.90651611)
\curveto(757.01792542,71.93651217)(756.9329255,71.97651213)(756.85292725,72.02651611)
\curveto(756.66292577,72.13651197)(756.48792595,72.26151184)(756.32792725,72.40151611)
\curveto(756.16792627,72.54151156)(756.04292639,72.71651139)(755.95292725,72.92651611)
\curveto(755.92292651,72.99651111)(755.89792654,73.06651104)(755.87792725,73.13651611)
\curveto(755.86792657,73.2065109)(755.85292658,73.28151082)(755.83292725,73.36151611)
\curveto(755.80292663,73.48151062)(755.79292664,73.61651049)(755.80292725,73.76651611)
\curveto(755.81292662,73.92651018)(755.82792661,74.06151004)(755.84792725,74.17151611)
\curveto(755.86792657,74.22150988)(755.87792656,74.26150984)(755.87792725,74.29151611)
\curveto(755.88792655,74.33150977)(755.90292653,74.37150973)(755.92292725,74.41151611)
\curveto(756.01292642,74.64150946)(756.1329263,74.84150926)(756.28292725,75.01151611)
\curveto(756.44292599,75.18150892)(756.62292581,75.33150877)(756.82292725,75.46151611)
\curveto(756.97292546,75.55150855)(757.1379253,75.62150848)(757.31792725,75.67151611)
\curveto(757.49792494,75.73150837)(757.68792475,75.78650832)(757.88792725,75.83651611)
\curveto(757.95792448,75.84650826)(758.02292441,75.85650825)(758.08292725,75.86651611)
\curveto(758.15292428,75.87650823)(758.22792421,75.88650822)(758.30792725,75.89651611)
\curveto(758.3379241,75.9065082)(758.37792406,75.9065082)(758.42792725,75.89651611)
\curveto(758.47792396,75.88650822)(758.51292392,75.89150821)(758.53292725,75.91151611)
}
}
{
\newrgbcolor{curcolor}{0 0 0}
\pscustom[linestyle=none,fillstyle=solid,fillcolor=curcolor]
{
\newpath
\moveto(764.54792725,78.07151611)
\curveto(764.69792524,78.07150603)(764.84792509,78.06650604)(764.99792725,78.05651611)
\curveto(765.14792479,78.05650605)(765.25292468,78.01650609)(765.31292725,77.93651611)
\curveto(765.36292457,77.87650623)(765.38792455,77.79150631)(765.38792725,77.68151611)
\curveto(765.39792454,77.58150652)(765.40292453,77.47650663)(765.40292725,77.36651611)
\lineto(765.40292725,76.49651611)
\curveto(765.40292453,76.41650769)(765.39792454,76.33150777)(765.38792725,76.24151611)
\curveto(765.38792455,76.16150794)(765.39792454,76.09150801)(765.41792725,76.03151611)
\curveto(765.45792448,75.89150821)(765.54792439,75.8015083)(765.68792725,75.76151611)
\curveto(765.7379242,75.75150835)(765.78292415,75.74650836)(765.82292725,75.74651611)
\lineto(765.97292725,75.74651611)
\lineto(766.37792725,75.74651611)
\curveto(766.5379234,75.75650835)(766.65292328,75.74650836)(766.72292725,75.71651611)
\curveto(766.81292312,75.65650845)(766.87292306,75.59650851)(766.90292725,75.53651611)
\curveto(766.92292301,75.49650861)(766.932923,75.45150865)(766.93292725,75.40151611)
\lineto(766.93292725,75.25151611)
\curveto(766.932923,75.14150896)(766.92792301,75.03650907)(766.91792725,74.93651611)
\curveto(766.90792303,74.84650926)(766.87292306,74.77650933)(766.81292725,74.72651611)
\curveto(766.75292318,74.67650943)(766.66792327,74.64650946)(766.55792725,74.63651611)
\lineto(766.22792725,74.63651611)
\curveto(766.11792382,74.64650946)(766.00792393,74.65150945)(765.89792725,74.65151611)
\curveto(765.78792415,74.65150945)(765.69292424,74.63650947)(765.61292725,74.60651611)
\curveto(765.54292439,74.57650953)(765.49292444,74.52650958)(765.46292725,74.45651611)
\curveto(765.4329245,74.38650972)(765.41292452,74.3015098)(765.40292725,74.20151611)
\curveto(765.39292454,74.11150999)(765.38792455,74.01151009)(765.38792725,73.90151611)
\curveto(765.39792454,73.8015103)(765.40292453,73.7015104)(765.40292725,73.60151611)
\lineto(765.40292725,70.63151611)
\curveto(765.40292453,70.41151369)(765.39792454,70.17651393)(765.38792725,69.92651611)
\curveto(765.38792455,69.68651442)(765.4329245,69.5015146)(765.52292725,69.37151611)
\curveto(765.57292436,69.29151481)(765.6379243,69.23651487)(765.71792725,69.20651611)
\curveto(765.79792414,69.17651493)(765.89292404,69.15151495)(766.00292725,69.13151611)
\curveto(766.0329239,69.12151498)(766.06292387,69.11651499)(766.09292725,69.11651611)
\curveto(766.1329238,69.12651498)(766.16792377,69.12651498)(766.19792725,69.11651611)
\lineto(766.39292725,69.11651611)
\curveto(766.49292344,69.11651499)(766.58292335,69.106515)(766.66292725,69.08651611)
\curveto(766.75292318,69.07651503)(766.81792312,69.04151506)(766.85792725,68.98151611)
\curveto(766.87792306,68.95151515)(766.89292304,68.89651521)(766.90292725,68.81651611)
\curveto(766.92292301,68.74651536)(766.932923,68.67151543)(766.93292725,68.59151611)
\curveto(766.94292299,68.51151559)(766.94292299,68.43151567)(766.93292725,68.35151611)
\curveto(766.92292301,68.28151582)(766.90292303,68.22651588)(766.87292725,68.18651611)
\curveto(766.8329231,68.11651599)(766.75792318,68.06651604)(766.64792725,68.03651611)
\curveto(766.56792337,68.01651609)(766.47792346,68.0065161)(766.37792725,68.00651611)
\curveto(766.27792366,68.01651609)(766.18792375,68.02151608)(766.10792725,68.02151611)
\curveto(766.04792389,68.02151608)(765.98792395,68.01651609)(765.92792725,68.00651611)
\curveto(765.86792407,68.0065161)(765.81292412,68.01151609)(765.76292725,68.02151611)
\lineto(765.58292725,68.02151611)
\curveto(765.5329244,68.03151607)(765.48292445,68.03651607)(765.43292725,68.03651611)
\curveto(765.39292454,68.04651606)(765.34792459,68.05151605)(765.29792725,68.05151611)
\curveto(765.09792484,68.101516)(764.92292501,68.15651595)(764.77292725,68.21651611)
\curveto(764.6329253,68.27651583)(764.51292542,68.38151572)(764.41292725,68.53151611)
\curveto(764.27292566,68.73151537)(764.19292574,68.98151512)(764.17292725,69.28151611)
\curveto(764.15292578,69.59151451)(764.14292579,69.92151418)(764.14292725,70.27151611)
\lineto(764.14292725,74.20151611)
\curveto(764.11292582,74.33150977)(764.08292585,74.42650968)(764.05292725,74.48651611)
\curveto(764.0329259,74.54650956)(763.96292597,74.59650951)(763.84292725,74.63651611)
\curveto(763.80292613,74.64650946)(763.76292617,74.64650946)(763.72292725,74.63651611)
\curveto(763.68292625,74.62650948)(763.64292629,74.63150947)(763.60292725,74.65151611)
\lineto(763.36292725,74.65151611)
\curveto(763.2329267,74.65150945)(763.12292681,74.66150944)(763.03292725,74.68151611)
\curveto(762.95292698,74.71150939)(762.89792704,74.77150933)(762.86792725,74.86151611)
\curveto(762.84792709,74.9015092)(762.8329271,74.94650916)(762.82292725,74.99651611)
\lineto(762.82292725,75.14651611)
\curveto(762.82292711,75.28650882)(762.8329271,75.4015087)(762.85292725,75.49151611)
\curveto(762.87292706,75.59150851)(762.932927,75.66650844)(763.03292725,75.71651611)
\curveto(763.14292679,75.75650835)(763.28292665,75.76650834)(763.45292725,75.74651611)
\curveto(763.6329263,75.72650838)(763.78292615,75.73650837)(763.90292725,75.77651611)
\curveto(763.99292594,75.82650828)(764.06292587,75.89650821)(764.11292725,75.98651611)
\curveto(764.1329258,76.04650806)(764.14292579,76.12150798)(764.14292725,76.21151611)
\lineto(764.14292725,76.46651611)
\lineto(764.14292725,77.39651611)
\lineto(764.14292725,77.63651611)
\curveto(764.14292579,77.72650638)(764.15292578,77.8015063)(764.17292725,77.86151611)
\curveto(764.21292572,77.94150616)(764.28792565,78.0065061)(764.39792725,78.05651611)
\curveto(764.42792551,78.05650605)(764.45292548,78.05650605)(764.47292725,78.05651611)
\curveto(764.50292543,78.06650604)(764.52792541,78.07150603)(764.54792725,78.07151611)
}
}
{
\newrgbcolor{curcolor}{0 0 0}
\pscustom[linestyle=none,fillstyle=solid,fillcolor=curcolor]
{
\newpath
\moveto(771.96472412,75.91151611)
\curveto(772.19471933,75.91150819)(772.3247192,75.85150825)(772.35472412,75.73151611)
\curveto(772.38471914,75.62150848)(772.39971913,75.45650865)(772.39972412,75.23651611)
\lineto(772.39972412,74.95151611)
\curveto(772.39971913,74.86150924)(772.37471915,74.78650932)(772.32472412,74.72651611)
\curveto(772.26471926,74.64650946)(772.17971935,74.6015095)(772.06972412,74.59151611)
\curveto(771.95971957,74.59150951)(771.84971968,74.57650953)(771.73972412,74.54651611)
\curveto(771.59971993,74.51650959)(771.46472006,74.48650962)(771.33472412,74.45651611)
\curveto(771.21472031,74.42650968)(771.09972043,74.38650972)(770.98972412,74.33651611)
\curveto(770.69972083,74.2065099)(770.46472106,74.02651008)(770.28472412,73.79651611)
\curveto(770.10472142,73.57651053)(769.94972158,73.32151078)(769.81972412,73.03151611)
\curveto(769.77972175,72.92151118)(769.74972178,72.8065113)(769.72972412,72.68651611)
\curveto(769.70972182,72.57651153)(769.68472184,72.46151164)(769.65472412,72.34151611)
\curveto(769.64472188,72.29151181)(769.63972189,72.24151186)(769.63972412,72.19151611)
\curveto(769.64972188,72.14151196)(769.64972188,72.09151201)(769.63972412,72.04151611)
\curveto(769.60972192,71.92151218)(769.59472193,71.78151232)(769.59472412,71.62151611)
\curveto(769.60472192,71.47151263)(769.60972192,71.32651278)(769.60972412,71.18651611)
\lineto(769.60972412,69.34151611)
\lineto(769.60972412,68.99651611)
\curveto(769.60972192,68.87651523)(769.60472192,68.76151534)(769.59472412,68.65151611)
\curveto(769.58472194,68.54151556)(769.57972195,68.44651566)(769.57972412,68.36651611)
\curveto(769.58972194,68.28651582)(769.56972196,68.21651589)(769.51972412,68.15651611)
\curveto(769.46972206,68.08651602)(769.38972214,68.04651606)(769.27972412,68.03651611)
\curveto(769.17972235,68.02651608)(769.06972246,68.02151608)(768.94972412,68.02151611)
\lineto(768.67972412,68.02151611)
\curveto(768.6297229,68.04151606)(768.57972295,68.05651605)(768.52972412,68.06651611)
\curveto(768.48972304,68.08651602)(768.45972307,68.11151599)(768.43972412,68.14151611)
\curveto(768.38972314,68.21151589)(768.35972317,68.29651581)(768.34972412,68.39651611)
\lineto(768.34972412,68.72651611)
\lineto(768.34972412,69.88151611)
\lineto(768.34972412,74.03651611)
\lineto(768.34972412,75.07151611)
\lineto(768.34972412,75.37151611)
\curveto(768.35972317,75.47150863)(768.38972314,75.55650855)(768.43972412,75.62651611)
\curveto(768.46972306,75.66650844)(768.51972301,75.69650841)(768.58972412,75.71651611)
\curveto(768.66972286,75.73650837)(768.75472277,75.74650836)(768.84472412,75.74651611)
\curveto(768.93472259,75.75650835)(769.0247225,75.75650835)(769.11472412,75.74651611)
\curveto(769.20472232,75.73650837)(769.27472225,75.72150838)(769.32472412,75.70151611)
\curveto(769.40472212,75.67150843)(769.45472207,75.61150849)(769.47472412,75.52151611)
\curveto(769.50472202,75.44150866)(769.51972201,75.35150875)(769.51972412,75.25151611)
\lineto(769.51972412,74.95151611)
\curveto(769.51972201,74.85150925)(769.53972199,74.76150934)(769.57972412,74.68151611)
\curveto(769.58972194,74.66150944)(769.59972193,74.64650946)(769.60972412,74.63651611)
\lineto(769.65472412,74.59151611)
\curveto(769.76472176,74.59150951)(769.85472167,74.63650947)(769.92472412,74.72651611)
\curveto(769.99472153,74.82650928)(770.05472147,74.9065092)(770.10472412,74.96651611)
\lineto(770.19472412,75.05651611)
\curveto(770.28472124,75.16650894)(770.40972112,75.28150882)(770.56972412,75.40151611)
\curveto(770.7297208,75.52150858)(770.87972065,75.61150849)(771.01972412,75.67151611)
\curveto(771.10972042,75.72150838)(771.20472032,75.75650835)(771.30472412,75.77651611)
\curveto(771.40472012,75.8065083)(771.50972002,75.83650827)(771.61972412,75.86651611)
\curveto(771.67971985,75.87650823)(771.73971979,75.88150822)(771.79972412,75.88151611)
\curveto(771.85971967,75.89150821)(771.91471961,75.9015082)(771.96472412,75.91151611)
}
}
{
\newrgbcolor{curcolor}{0 0 0}
\pscustom[linestyle=none,fillstyle=solid,fillcolor=curcolor]
{
\newpath
\moveto(780.21448975,68.56151611)
\curveto(780.24448192,68.4015157)(780.22948193,68.26651584)(780.16948975,68.15651611)
\curveto(780.10948205,68.05651605)(780.02948213,67.98151612)(779.92948975,67.93151611)
\curveto(779.87948228,67.91151619)(779.82448234,67.9015162)(779.76448975,67.90151611)
\curveto(779.71448245,67.9015162)(779.6594825,67.89151621)(779.59948975,67.87151611)
\curveto(779.37948278,67.82151628)(779.159483,67.83651627)(778.93948975,67.91651611)
\curveto(778.72948343,67.98651612)(778.58448358,68.07651603)(778.50448975,68.18651611)
\curveto(778.45448371,68.25651585)(778.40948375,68.33651577)(778.36948975,68.42651611)
\curveto(778.32948383,68.52651558)(778.27948388,68.6065155)(778.21948975,68.66651611)
\curveto(778.19948396,68.68651542)(778.17448399,68.7065154)(778.14448975,68.72651611)
\curveto(778.12448404,68.74651536)(778.09448407,68.75151535)(778.05448975,68.74151611)
\curveto(777.94448422,68.71151539)(777.83948432,68.65651545)(777.73948975,68.57651611)
\curveto(777.64948451,68.49651561)(777.5594846,68.42651568)(777.46948975,68.36651611)
\curveto(777.33948482,68.28651582)(777.19948496,68.21151589)(777.04948975,68.14151611)
\curveto(776.89948526,68.08151602)(776.73948542,68.02651608)(776.56948975,67.97651611)
\curveto(776.46948569,67.94651616)(776.3594858,67.92651618)(776.23948975,67.91651611)
\curveto(776.12948603,67.9065162)(776.01948614,67.89151621)(775.90948975,67.87151611)
\curveto(775.8594863,67.86151624)(775.81448635,67.85651625)(775.77448975,67.85651611)
\lineto(775.66948975,67.85651611)
\curveto(775.5594866,67.83651627)(775.45448671,67.83651627)(775.35448975,67.85651611)
\lineto(775.21948975,67.85651611)
\curveto(775.16948699,67.86651624)(775.11948704,67.87151623)(775.06948975,67.87151611)
\curveto(775.01948714,67.87151623)(774.97448719,67.88151622)(774.93448975,67.90151611)
\curveto(774.89448727,67.91151619)(774.8594873,67.91651619)(774.82948975,67.91651611)
\curveto(774.80948735,67.9065162)(774.78448738,67.9065162)(774.75448975,67.91651611)
\lineto(774.51448975,67.97651611)
\curveto(774.43448773,67.98651612)(774.3594878,68.0065161)(774.28948975,68.03651611)
\curveto(773.98948817,68.16651594)(773.74448842,68.31151579)(773.55448975,68.47151611)
\curveto(773.37448879,68.64151546)(773.22448894,68.87651523)(773.10448975,69.17651611)
\curveto(773.01448915,69.39651471)(772.96948919,69.66151444)(772.96948975,69.97151611)
\lineto(772.96948975,70.28651611)
\curveto(772.97948918,70.33651377)(772.98448918,70.38651372)(772.98448975,70.43651611)
\lineto(773.01448975,70.61651611)
\lineto(773.13448975,70.94651611)
\curveto(773.17448899,71.05651305)(773.22448894,71.15651295)(773.28448975,71.24651611)
\curveto(773.4644887,71.53651257)(773.70948845,71.75151235)(774.01948975,71.89151611)
\curveto(774.32948783,72.03151207)(774.66948749,72.15651195)(775.03948975,72.26651611)
\curveto(775.17948698,72.3065118)(775.32448684,72.33651177)(775.47448975,72.35651611)
\curveto(775.62448654,72.37651173)(775.77448639,72.4015117)(775.92448975,72.43151611)
\curveto(775.99448617,72.45151165)(776.0594861,72.46151164)(776.11948975,72.46151611)
\curveto(776.18948597,72.46151164)(776.2644859,72.47151163)(776.34448975,72.49151611)
\curveto(776.41448575,72.51151159)(776.48448568,72.52151158)(776.55448975,72.52151611)
\curveto(776.62448554,72.53151157)(776.69948546,72.54651156)(776.77948975,72.56651611)
\curveto(777.02948513,72.62651148)(777.2644849,72.67651143)(777.48448975,72.71651611)
\curveto(777.70448446,72.76651134)(777.87948428,72.88151122)(778.00948975,73.06151611)
\curveto(778.06948409,73.14151096)(778.11948404,73.24151086)(778.15948975,73.36151611)
\curveto(778.19948396,73.49151061)(778.19948396,73.63151047)(778.15948975,73.78151611)
\curveto(778.09948406,74.02151008)(778.00948415,74.21150989)(777.88948975,74.35151611)
\curveto(777.77948438,74.49150961)(777.61948454,74.6015095)(777.40948975,74.68151611)
\curveto(777.28948487,74.73150937)(777.14448502,74.76650934)(776.97448975,74.78651611)
\curveto(776.81448535,74.8065093)(776.64448552,74.81650929)(776.46448975,74.81651611)
\curveto(776.28448588,74.81650929)(776.10948605,74.8065093)(775.93948975,74.78651611)
\curveto(775.76948639,74.76650934)(775.62448654,74.73650937)(775.50448975,74.69651611)
\curveto(775.33448683,74.63650947)(775.16948699,74.55150955)(775.00948975,74.44151611)
\curveto(774.92948723,74.38150972)(774.85448731,74.3015098)(774.78448975,74.20151611)
\curveto(774.72448744,74.11150999)(774.66948749,74.01151009)(774.61948975,73.90151611)
\curveto(774.58948757,73.82151028)(774.5594876,73.73651037)(774.52948975,73.64651611)
\curveto(774.50948765,73.55651055)(774.4644877,73.48651062)(774.39448975,73.43651611)
\curveto(774.35448781,73.4065107)(774.28448788,73.38151072)(774.18448975,73.36151611)
\curveto(774.09448807,73.35151075)(773.99948816,73.34651076)(773.89948975,73.34651611)
\curveto(773.79948836,73.34651076)(773.69948846,73.35151075)(773.59948975,73.36151611)
\curveto(773.50948865,73.38151072)(773.44448872,73.4065107)(773.40448975,73.43651611)
\curveto(773.3644888,73.46651064)(773.33448883,73.51651059)(773.31448975,73.58651611)
\curveto(773.29448887,73.65651045)(773.29448887,73.73151037)(773.31448975,73.81151611)
\curveto(773.34448882,73.94151016)(773.37448879,74.06151004)(773.40448975,74.17151611)
\curveto(773.44448872,74.29150981)(773.48948867,74.4065097)(773.53948975,74.51651611)
\curveto(773.72948843,74.86650924)(773.96948819,75.13650897)(774.25948975,75.32651611)
\curveto(774.54948761,75.52650858)(774.90948725,75.68650842)(775.33948975,75.80651611)
\curveto(775.43948672,75.82650828)(775.53948662,75.84150826)(775.63948975,75.85151611)
\curveto(775.74948641,75.86150824)(775.8594863,75.87650823)(775.96948975,75.89651611)
\curveto(776.00948615,75.9065082)(776.07448609,75.9065082)(776.16448975,75.89651611)
\curveto(776.25448591,75.89650821)(776.30948585,75.9065082)(776.32948975,75.92651611)
\curveto(777.02948513,75.93650817)(777.63948452,75.85650825)(778.15948975,75.68651611)
\curveto(778.67948348,75.51650859)(779.04448312,75.19150891)(779.25448975,74.71151611)
\curveto(779.34448282,74.51150959)(779.39448277,74.27650983)(779.40448975,74.00651611)
\curveto(779.42448274,73.74651036)(779.43448273,73.47151063)(779.43448975,73.18151611)
\lineto(779.43448975,69.86651611)
\curveto(779.43448273,69.72651438)(779.43948272,69.59151451)(779.44948975,69.46151611)
\curveto(779.4594827,69.33151477)(779.48948267,69.22651488)(779.53948975,69.14651611)
\curveto(779.58948257,69.07651503)(779.65448251,69.02651508)(779.73448975,68.99651611)
\curveto(779.82448234,68.95651515)(779.90948225,68.92651518)(779.98948975,68.90651611)
\curveto(780.06948209,68.89651521)(780.12948203,68.85151525)(780.16948975,68.77151611)
\curveto(780.18948197,68.74151536)(780.19948196,68.71151539)(780.19948975,68.68151611)
\curveto(780.19948196,68.65151545)(780.20448196,68.61151549)(780.21448975,68.56151611)
\moveto(778.06948975,70.22651611)
\curveto(778.12948403,70.36651374)(778.159484,70.52651358)(778.15948975,70.70651611)
\curveto(778.16948399,70.89651321)(778.17448399,71.09151301)(778.17448975,71.29151611)
\curveto(778.17448399,71.4015127)(778.16948399,71.5015126)(778.15948975,71.59151611)
\curveto(778.14948401,71.68151242)(778.10948405,71.75151235)(778.03948975,71.80151611)
\curveto(778.00948415,71.82151228)(777.93948422,71.83151227)(777.82948975,71.83151611)
\curveto(777.80948435,71.81151229)(777.77448439,71.8015123)(777.72448975,71.80151611)
\curveto(777.67448449,71.8015123)(777.62948453,71.79151231)(777.58948975,71.77151611)
\curveto(777.50948465,71.75151235)(777.41948474,71.73151237)(777.31948975,71.71151611)
\lineto(777.01948975,71.65151611)
\curveto(776.98948517,71.65151245)(776.95448521,71.64651246)(776.91448975,71.63651611)
\lineto(776.80948975,71.63651611)
\curveto(776.6594855,71.59651251)(776.49448567,71.57151253)(776.31448975,71.56151611)
\curveto(776.14448602,71.56151254)(775.98448618,71.54151256)(775.83448975,71.50151611)
\curveto(775.75448641,71.48151262)(775.67948648,71.46151264)(775.60948975,71.44151611)
\curveto(775.54948661,71.43151267)(775.47948668,71.41651269)(775.39948975,71.39651611)
\curveto(775.23948692,71.34651276)(775.08948707,71.28151282)(774.94948975,71.20151611)
\curveto(774.80948735,71.13151297)(774.68948747,71.04151306)(774.58948975,70.93151611)
\curveto(774.48948767,70.82151328)(774.41448775,70.68651342)(774.36448975,70.52651611)
\curveto(774.31448785,70.37651373)(774.29448787,70.19151391)(774.30448975,69.97151611)
\curveto(774.30448786,69.87151423)(774.31948784,69.77651433)(774.34948975,69.68651611)
\curveto(774.38948777,69.6065145)(774.43448773,69.53151457)(774.48448975,69.46151611)
\curveto(774.5644876,69.35151475)(774.66948749,69.25651485)(774.79948975,69.17651611)
\curveto(774.92948723,69.106515)(775.06948709,69.04651506)(775.21948975,68.99651611)
\curveto(775.26948689,68.98651512)(775.31948684,68.98151512)(775.36948975,68.98151611)
\curveto(775.41948674,68.98151512)(775.46948669,68.97651513)(775.51948975,68.96651611)
\curveto(775.58948657,68.94651516)(775.67448649,68.93151517)(775.77448975,68.92151611)
\curveto(775.88448628,68.92151518)(775.97448619,68.93151517)(776.04448975,68.95151611)
\curveto(776.10448606,68.97151513)(776.164486,68.97651513)(776.22448975,68.96651611)
\curveto(776.28448588,68.96651514)(776.34448582,68.97651513)(776.40448975,68.99651611)
\curveto(776.48448568,69.01651509)(776.5594856,69.03151507)(776.62948975,69.04151611)
\curveto(776.70948545,69.05151505)(776.78448538,69.07151503)(776.85448975,69.10151611)
\curveto(777.14448502,69.22151488)(777.38948477,69.36651474)(777.58948975,69.53651611)
\curveto(777.79948436,69.7065144)(777.9594842,69.93651417)(778.06948975,70.22651611)
}
}
{
\newrgbcolor{curcolor}{0 0 0}
\pscustom[linestyle=none,fillstyle=solid,fillcolor=curcolor]
{
\newpath
\moveto(788.34613037,68.81651611)
\lineto(788.34613037,68.42651611)
\curveto(788.3461225,68.3065158)(788.32112252,68.2065159)(788.27113037,68.12651611)
\curveto(788.22112262,68.05651605)(788.13612271,68.01651609)(788.01613037,68.00651611)
\lineto(787.67113037,68.00651611)
\curveto(787.61112323,68.0065161)(787.55112329,68.0015161)(787.49113037,67.99151611)
\curveto(787.4411234,67.99151611)(787.39612345,68.0015161)(787.35613037,68.02151611)
\curveto(787.26612358,68.04151606)(787.20612364,68.08151602)(787.17613037,68.14151611)
\curveto(787.13612371,68.19151591)(787.11112373,68.25151585)(787.10113037,68.32151611)
\curveto(787.10112374,68.39151571)(787.08612376,68.46151564)(787.05613037,68.53151611)
\curveto(787.0461238,68.55151555)(787.03112381,68.56651554)(787.01113037,68.57651611)
\curveto(787.00112384,68.59651551)(786.98612386,68.61651549)(786.96613037,68.63651611)
\curveto(786.86612398,68.64651546)(786.78612406,68.62651548)(786.72613037,68.57651611)
\curveto(786.67612417,68.52651558)(786.62112422,68.47651563)(786.56113037,68.42651611)
\curveto(786.36112448,68.27651583)(786.16112468,68.16151594)(785.96113037,68.08151611)
\curveto(785.78112506,68.0015161)(785.57112527,67.94151616)(785.33113037,67.90151611)
\curveto(785.10112574,67.86151624)(784.86112598,67.84151626)(784.61113037,67.84151611)
\curveto(784.37112647,67.83151627)(784.13112671,67.84651626)(783.89113037,67.88651611)
\curveto(783.65112719,67.91651619)(783.4411274,67.97151613)(783.26113037,68.05151611)
\curveto(782.7411281,68.27151583)(782.32112852,68.56651554)(782.00113037,68.93651611)
\curveto(781.68112916,69.31651479)(781.43112941,69.78651432)(781.25113037,70.34651611)
\curveto(781.21112963,70.43651367)(781.18112966,70.52651358)(781.16113037,70.61651611)
\curveto(781.15112969,70.71651339)(781.13112971,70.81651329)(781.10113037,70.91651611)
\curveto(781.09112975,70.96651314)(781.08612976,71.01651309)(781.08613037,71.06651611)
\curveto(781.08612976,71.11651299)(781.08112976,71.16651294)(781.07113037,71.21651611)
\curveto(781.05112979,71.26651284)(781.0411298,71.31651279)(781.04113037,71.36651611)
\curveto(781.05112979,71.42651268)(781.05112979,71.48151262)(781.04113037,71.53151611)
\lineto(781.04113037,71.68151611)
\curveto(781.02112982,71.73151237)(781.01112983,71.79651231)(781.01113037,71.87651611)
\curveto(781.01112983,71.95651215)(781.02112982,72.02151208)(781.04113037,72.07151611)
\lineto(781.04113037,72.23651611)
\curveto(781.06112978,72.3065118)(781.06612978,72.37651173)(781.05613037,72.44651611)
\curveto(781.05612979,72.52651158)(781.06612978,72.6015115)(781.08613037,72.67151611)
\curveto(781.09612975,72.72151138)(781.10112974,72.76651134)(781.10113037,72.80651611)
\curveto(781.10112974,72.84651126)(781.10612974,72.89151121)(781.11613037,72.94151611)
\curveto(781.1461297,73.04151106)(781.17112967,73.13651097)(781.19113037,73.22651611)
\curveto(781.21112963,73.32651078)(781.23612961,73.42151068)(781.26613037,73.51151611)
\curveto(781.39612945,73.89151021)(781.56112928,74.23150987)(781.76113037,74.53151611)
\curveto(781.97112887,74.84150926)(782.22112862,75.09650901)(782.51113037,75.29651611)
\curveto(782.68112816,75.41650869)(782.85612799,75.51650859)(783.03613037,75.59651611)
\curveto(783.22612762,75.67650843)(783.43112741,75.74650836)(783.65113037,75.80651611)
\curveto(783.72112712,75.81650829)(783.78612706,75.82650828)(783.84613037,75.83651611)
\curveto(783.91612693,75.84650826)(783.98612686,75.86150824)(784.05613037,75.88151611)
\lineto(784.20613037,75.88151611)
\curveto(784.28612656,75.9015082)(784.40112644,75.91150819)(784.55113037,75.91151611)
\curveto(784.71112613,75.91150819)(784.83112601,75.9015082)(784.91113037,75.88151611)
\curveto(784.95112589,75.87150823)(785.00612584,75.86650824)(785.07613037,75.86651611)
\curveto(785.18612566,75.83650827)(785.29612555,75.81150829)(785.40613037,75.79151611)
\curveto(785.51612533,75.78150832)(785.62112522,75.75150835)(785.72113037,75.70151611)
\curveto(785.87112497,75.64150846)(786.01112483,75.57650853)(786.14113037,75.50651611)
\curveto(786.28112456,75.43650867)(786.41112443,75.35650875)(786.53113037,75.26651611)
\curveto(786.59112425,75.21650889)(786.65112419,75.16150894)(786.71113037,75.10151611)
\curveto(786.78112406,75.05150905)(786.87112397,75.03650907)(786.98113037,75.05651611)
\curveto(787.00112384,75.08650902)(787.01612383,75.11150899)(787.02613037,75.13151611)
\curveto(787.0461238,75.15150895)(787.06112378,75.18150892)(787.07113037,75.22151611)
\curveto(787.10112374,75.31150879)(787.11112373,75.42650868)(787.10113037,75.56651611)
\lineto(787.10113037,75.94151611)
\lineto(787.10113037,77.66651611)
\lineto(787.10113037,78.13151611)
\curveto(787.10112374,78.31150579)(787.12612372,78.44150566)(787.17613037,78.52151611)
\curveto(787.21612363,78.59150551)(787.27612357,78.63650547)(787.35613037,78.65651611)
\curveto(787.37612347,78.65650545)(787.40112344,78.65650545)(787.43113037,78.65651611)
\curveto(787.46112338,78.66650544)(787.48612336,78.67150543)(787.50613037,78.67151611)
\curveto(787.6461232,78.68150542)(787.79112305,78.68150542)(787.94113037,78.67151611)
\curveto(788.10112274,78.67150543)(788.21112263,78.63150547)(788.27113037,78.55151611)
\curveto(788.32112252,78.47150563)(788.3461225,78.37150573)(788.34613037,78.25151611)
\lineto(788.34613037,77.87651611)
\lineto(788.34613037,68.81651611)
\moveto(787.13113037,71.65151611)
\curveto(787.15112369,71.7015124)(787.16112368,71.76651234)(787.16113037,71.84651611)
\curveto(787.16112368,71.93651217)(787.15112369,72.0065121)(787.13113037,72.05651611)
\lineto(787.13113037,72.28151611)
\curveto(787.11112373,72.37151173)(787.09612375,72.46151164)(787.08613037,72.55151611)
\curveto(787.07612377,72.65151145)(787.05612379,72.74151136)(787.02613037,72.82151611)
\curveto(787.00612384,72.9015112)(786.98612386,72.97651113)(786.96613037,73.04651611)
\curveto(786.95612389,73.11651099)(786.93612391,73.18651092)(786.90613037,73.25651611)
\curveto(786.78612406,73.55651055)(786.63112421,73.82151028)(786.44113037,74.05151611)
\curveto(786.25112459,74.28150982)(786.01112483,74.46150964)(785.72113037,74.59151611)
\curveto(785.62112522,74.64150946)(785.51612533,74.67650943)(785.40613037,74.69651611)
\curveto(785.30612554,74.72650938)(785.19612565,74.75150935)(785.07613037,74.77151611)
\curveto(784.99612585,74.79150931)(784.90612594,74.8015093)(784.80613037,74.80151611)
\lineto(784.53613037,74.80151611)
\curveto(784.48612636,74.79150931)(784.4411264,74.78150932)(784.40113037,74.77151611)
\lineto(784.26613037,74.77151611)
\curveto(784.18612666,74.75150935)(784.10112674,74.73150937)(784.01113037,74.71151611)
\curveto(783.93112691,74.69150941)(783.85112699,74.66650944)(783.77113037,74.63651611)
\curveto(783.45112739,74.49650961)(783.19112765,74.29150981)(782.99113037,74.02151611)
\curveto(782.80112804,73.76151034)(782.6461282,73.45651065)(782.52613037,73.10651611)
\curveto(782.48612836,72.99651111)(782.45612839,72.88151122)(782.43613037,72.76151611)
\curveto(782.42612842,72.65151145)(782.41112843,72.54151156)(782.39113037,72.43151611)
\curveto(782.39112845,72.39151171)(782.38612846,72.35151175)(782.37613037,72.31151611)
\lineto(782.37613037,72.20651611)
\curveto(782.35612849,72.15651195)(782.3461285,72.101512)(782.34613037,72.04151611)
\curveto(782.35612849,71.98151212)(782.36112848,71.92651218)(782.36113037,71.87651611)
\lineto(782.36113037,71.54651611)
\curveto(782.36112848,71.44651266)(782.37112847,71.35151275)(782.39113037,71.26151611)
\curveto(782.40112844,71.23151287)(782.40612844,71.18151292)(782.40613037,71.11151611)
\curveto(782.42612842,71.04151306)(782.4411284,70.97151313)(782.45113037,70.90151611)
\lineto(782.51113037,70.69151611)
\curveto(782.62112822,70.34151376)(782.77112807,70.04151406)(782.96113037,69.79151611)
\curveto(783.15112769,69.54151456)(783.39112745,69.33651477)(783.68113037,69.17651611)
\curveto(783.77112707,69.12651498)(783.86112698,69.08651502)(783.95113037,69.05651611)
\curveto(784.0411268,69.02651508)(784.1411267,68.99651511)(784.25113037,68.96651611)
\curveto(784.30112654,68.94651516)(784.35112649,68.94151516)(784.40113037,68.95151611)
\curveto(784.46112638,68.96151514)(784.51612633,68.95651515)(784.56613037,68.93651611)
\curveto(784.60612624,68.92651518)(784.6461262,68.92151518)(784.68613037,68.92151611)
\lineto(784.82113037,68.92151611)
\lineto(784.95613037,68.92151611)
\curveto(784.98612586,68.93151517)(785.03612581,68.93651517)(785.10613037,68.93651611)
\curveto(785.18612566,68.95651515)(785.26612558,68.97151513)(785.34613037,68.98151611)
\curveto(785.42612542,69.0015151)(785.50112534,69.02651508)(785.57113037,69.05651611)
\curveto(785.90112494,69.19651491)(786.16612468,69.37151473)(786.36613037,69.58151611)
\curveto(786.57612427,69.8015143)(786.75112409,70.07651403)(786.89113037,70.40651611)
\curveto(786.9411239,70.51651359)(786.97612387,70.62651348)(786.99613037,70.73651611)
\curveto(787.01612383,70.84651326)(787.0411238,70.95651315)(787.07113037,71.06651611)
\curveto(787.09112375,71.106513)(787.10112374,71.14151296)(787.10113037,71.17151611)
\curveto(787.10112374,71.21151289)(787.10612374,71.25151285)(787.11613037,71.29151611)
\curveto(787.12612372,71.35151275)(787.12612372,71.41151269)(787.11613037,71.47151611)
\curveto(787.11612373,71.53151257)(787.12112372,71.59151251)(787.13113037,71.65151611)
}
}
{
\newrgbcolor{curcolor}{0 0 0}
\pscustom[linestyle=none,fillstyle=solid,fillcolor=curcolor]
{
\newpath
\moveto(797.41738037,72.20651611)
\curveto(797.43737231,72.14651196)(797.4473723,72.05151205)(797.44738037,71.92151611)
\curveto(797.4473723,71.8015123)(797.44237231,71.71651239)(797.43238037,71.66651611)
\lineto(797.43238037,71.51651611)
\curveto(797.42237233,71.43651267)(797.41237234,71.36151274)(797.40238037,71.29151611)
\curveto(797.40237235,71.23151287)(797.39737235,71.16151294)(797.38738037,71.08151611)
\curveto(797.36737238,71.02151308)(797.3523724,70.96151314)(797.34238037,70.90151611)
\curveto(797.34237241,70.84151326)(797.33237242,70.78151332)(797.31238037,70.72151611)
\curveto(797.27237248,70.59151351)(797.23737251,70.46151364)(797.20738037,70.33151611)
\curveto(797.17737257,70.2015139)(797.13737261,70.08151402)(797.08738037,69.97151611)
\curveto(796.87737287,69.49151461)(796.59737315,69.08651502)(796.24738037,68.75651611)
\curveto(795.89737385,68.43651567)(795.46737428,68.19151591)(794.95738037,68.02151611)
\curveto(794.8473749,67.98151612)(794.72737502,67.95151615)(794.59738037,67.93151611)
\curveto(794.47737527,67.91151619)(794.3523754,67.89151621)(794.22238037,67.87151611)
\curveto(794.16237559,67.86151624)(794.09737565,67.85651625)(794.02738037,67.85651611)
\curveto(793.96737578,67.84651626)(793.90737584,67.84151626)(793.84738037,67.84151611)
\curveto(793.80737594,67.83151627)(793.747376,67.82651628)(793.66738037,67.82651611)
\curveto(793.59737615,67.82651628)(793.5473762,67.83151627)(793.51738037,67.84151611)
\curveto(793.47737627,67.85151625)(793.43737631,67.85651625)(793.39738037,67.85651611)
\curveto(793.35737639,67.84651626)(793.32237643,67.84651626)(793.29238037,67.85651611)
\lineto(793.20238037,67.85651611)
\lineto(792.84238037,67.90151611)
\curveto(792.70237705,67.94151616)(792.56737718,67.98151612)(792.43738037,68.02151611)
\curveto(792.30737744,68.06151604)(792.18237757,68.106516)(792.06238037,68.15651611)
\curveto(791.61237814,68.35651575)(791.24237851,68.61651549)(790.95238037,68.93651611)
\curveto(790.66237909,69.25651485)(790.42237933,69.64651446)(790.23238037,70.10651611)
\curveto(790.18237957,70.2065139)(790.14237961,70.3065138)(790.11238037,70.40651611)
\curveto(790.09237966,70.5065136)(790.07237968,70.61151349)(790.05238037,70.72151611)
\curveto(790.03237972,70.76151334)(790.02237973,70.79151331)(790.02238037,70.81151611)
\curveto(790.03237972,70.84151326)(790.03237972,70.87651323)(790.02238037,70.91651611)
\curveto(790.00237975,70.99651311)(789.98737976,71.07651303)(789.97738037,71.15651611)
\curveto(789.97737977,71.24651286)(789.96737978,71.33151277)(789.94738037,71.41151611)
\lineto(789.94738037,71.53151611)
\curveto(789.9473798,71.57151253)(789.94237981,71.61651249)(789.93238037,71.66651611)
\curveto(789.92237983,71.71651239)(789.91737983,71.8015123)(789.91738037,71.92151611)
\curveto(789.91737983,72.05151205)(789.92737982,72.14651196)(789.94738037,72.20651611)
\curveto(789.96737978,72.27651183)(789.97237978,72.34651176)(789.96238037,72.41651611)
\curveto(789.9523798,72.48651162)(789.95737979,72.55651155)(789.97738037,72.62651611)
\curveto(789.98737976,72.67651143)(789.99237976,72.71651139)(789.99238037,72.74651611)
\curveto(790.00237975,72.78651132)(790.01237974,72.83151127)(790.02238037,72.88151611)
\curveto(790.0523797,73.0015111)(790.07737967,73.12151098)(790.09738037,73.24151611)
\curveto(790.12737962,73.36151074)(790.16737958,73.47651063)(790.21738037,73.58651611)
\curveto(790.36737938,73.95651015)(790.5473792,74.28650982)(790.75738037,74.57651611)
\curveto(790.97737877,74.87650923)(791.24237851,75.12650898)(791.55238037,75.32651611)
\curveto(791.67237808,75.4065087)(791.79737795,75.47150863)(791.92738037,75.52151611)
\curveto(792.05737769,75.58150852)(792.19237756,75.64150846)(792.33238037,75.70151611)
\curveto(792.4523773,75.75150835)(792.58237717,75.78150832)(792.72238037,75.79151611)
\curveto(792.86237689,75.81150829)(793.00237675,75.84150826)(793.14238037,75.88151611)
\lineto(793.33738037,75.88151611)
\curveto(793.40737634,75.89150821)(793.47237628,75.9015082)(793.53238037,75.91151611)
\curveto(794.42237533,75.92150818)(795.16237459,75.73650837)(795.75238037,75.35651611)
\curveto(796.34237341,74.97650913)(796.76737298,74.48150962)(797.02738037,73.87151611)
\curveto(797.07737267,73.77151033)(797.11737263,73.67151043)(797.14738037,73.57151611)
\curveto(797.17737257,73.47151063)(797.21237254,73.36651074)(797.25238037,73.25651611)
\curveto(797.28237247,73.14651096)(797.30737244,73.02651108)(797.32738037,72.89651611)
\curveto(797.3473724,72.77651133)(797.37237238,72.65151145)(797.40238037,72.52151611)
\curveto(797.41237234,72.47151163)(797.41237234,72.41651169)(797.40238037,72.35651611)
\curveto(797.40237235,72.3065118)(797.40737234,72.25651185)(797.41738037,72.20651611)
\moveto(796.08238037,71.35151611)
\curveto(796.10237365,71.42151268)(796.10737364,71.5015126)(796.09738037,71.59151611)
\lineto(796.09738037,71.84651611)
\curveto(796.09737365,72.23651187)(796.06237369,72.56651154)(795.99238037,72.83651611)
\curveto(795.96237379,72.91651119)(795.93737381,72.99651111)(795.91738037,73.07651611)
\curveto(795.89737385,73.15651095)(795.87237388,73.23151087)(795.84238037,73.30151611)
\curveto(795.56237419,73.95151015)(795.11737463,74.4015097)(794.50738037,74.65151611)
\curveto(794.43737531,74.68150942)(794.36237539,74.7015094)(794.28238037,74.71151611)
\lineto(794.04238037,74.77151611)
\curveto(793.96237579,74.79150931)(793.87737587,74.8015093)(793.78738037,74.80151611)
\lineto(793.51738037,74.80151611)
\lineto(793.24738037,74.75651611)
\curveto(793.1473766,74.73650937)(793.0523767,74.71150939)(792.96238037,74.68151611)
\curveto(792.88237687,74.66150944)(792.80237695,74.63150947)(792.72238037,74.59151611)
\curveto(792.6523771,74.57150953)(792.58737716,74.54150956)(792.52738037,74.50151611)
\curveto(792.46737728,74.46150964)(792.41237734,74.42150968)(792.36238037,74.38151611)
\curveto(792.12237763,74.21150989)(791.92737782,74.0065101)(791.77738037,73.76651611)
\curveto(791.62737812,73.52651058)(791.49737825,73.24651086)(791.38738037,72.92651611)
\curveto(791.35737839,72.82651128)(791.33737841,72.72151138)(791.32738037,72.61151611)
\curveto(791.31737843,72.51151159)(791.30237845,72.4065117)(791.28238037,72.29651611)
\curveto(791.27237848,72.25651185)(791.26737848,72.19151191)(791.26738037,72.10151611)
\curveto(791.25737849,72.07151203)(791.2523785,72.03651207)(791.25238037,71.99651611)
\curveto(791.26237849,71.95651215)(791.26737848,71.91151219)(791.26738037,71.86151611)
\lineto(791.26738037,71.56151611)
\curveto(791.26737848,71.46151264)(791.27737847,71.37151273)(791.29738037,71.29151611)
\lineto(791.32738037,71.11151611)
\curveto(791.3473784,71.01151309)(791.36237839,70.91151319)(791.37238037,70.81151611)
\curveto(791.39237836,70.72151338)(791.42237833,70.63651347)(791.46238037,70.55651611)
\curveto(791.56237819,70.31651379)(791.67737807,70.09151401)(791.80738037,69.88151611)
\curveto(791.9473778,69.67151443)(792.11737763,69.49651461)(792.31738037,69.35651611)
\curveto(792.36737738,69.32651478)(792.41237734,69.3015148)(792.45238037,69.28151611)
\curveto(792.49237726,69.26151484)(792.53737721,69.23651487)(792.58738037,69.20651611)
\curveto(792.66737708,69.15651495)(792.752377,69.11151499)(792.84238037,69.07151611)
\curveto(792.94237681,69.04151506)(793.0473767,69.01151509)(793.15738037,68.98151611)
\curveto(793.20737654,68.96151514)(793.2523765,68.95151515)(793.29238037,68.95151611)
\curveto(793.34237641,68.96151514)(793.39237636,68.96151514)(793.44238037,68.95151611)
\curveto(793.47237628,68.94151516)(793.53237622,68.93151517)(793.62238037,68.92151611)
\curveto(793.72237603,68.91151519)(793.79737595,68.91651519)(793.84738037,68.93651611)
\curveto(793.88737586,68.94651516)(793.92737582,68.94651516)(793.96738037,68.93651611)
\curveto(794.00737574,68.93651517)(794.0473757,68.94651516)(794.08738037,68.96651611)
\curveto(794.16737558,68.98651512)(794.2473755,69.0015151)(794.32738037,69.01151611)
\curveto(794.40737534,69.03151507)(794.48237527,69.05651505)(794.55238037,69.08651611)
\curveto(794.89237486,69.22651488)(795.16737458,69.42151468)(795.37738037,69.67151611)
\curveto(795.58737416,69.92151418)(795.76237399,70.21651389)(795.90238037,70.55651611)
\curveto(795.9523738,70.67651343)(795.98237377,70.8015133)(795.99238037,70.93151611)
\curveto(796.01237374,71.07151303)(796.04237371,71.21151289)(796.08238037,71.35151611)
}
}
{
\newrgbcolor{curcolor}{0 0 0}
\pscustom[linestyle=none,fillstyle=solid,fillcolor=curcolor]
{
\newpath
\moveto(802.55066162,75.91151611)
\curveto(802.78065683,75.91150819)(802.9106567,75.85150825)(802.94066162,75.73151611)
\curveto(802.97065664,75.62150848)(802.98565663,75.45650865)(802.98566162,75.23651611)
\lineto(802.98566162,74.95151611)
\curveto(802.98565663,74.86150924)(802.96065665,74.78650932)(802.91066162,74.72651611)
\curveto(802.85065676,74.64650946)(802.76565685,74.6015095)(802.65566162,74.59151611)
\curveto(802.54565707,74.59150951)(802.43565718,74.57650953)(802.32566162,74.54651611)
\curveto(802.18565743,74.51650959)(802.05065756,74.48650962)(801.92066162,74.45651611)
\curveto(801.80065781,74.42650968)(801.68565793,74.38650972)(801.57566162,74.33651611)
\curveto(801.28565833,74.2065099)(801.05065856,74.02651008)(800.87066162,73.79651611)
\curveto(800.69065892,73.57651053)(800.53565908,73.32151078)(800.40566162,73.03151611)
\curveto(800.36565925,72.92151118)(800.33565928,72.8065113)(800.31566162,72.68651611)
\curveto(800.29565932,72.57651153)(800.27065934,72.46151164)(800.24066162,72.34151611)
\curveto(800.23065938,72.29151181)(800.22565939,72.24151186)(800.22566162,72.19151611)
\curveto(800.23565938,72.14151196)(800.23565938,72.09151201)(800.22566162,72.04151611)
\curveto(800.19565942,71.92151218)(800.18065943,71.78151232)(800.18066162,71.62151611)
\curveto(800.19065942,71.47151263)(800.19565942,71.32651278)(800.19566162,71.18651611)
\lineto(800.19566162,69.34151611)
\lineto(800.19566162,68.99651611)
\curveto(800.19565942,68.87651523)(800.19065942,68.76151534)(800.18066162,68.65151611)
\curveto(800.17065944,68.54151556)(800.16565945,68.44651566)(800.16566162,68.36651611)
\curveto(800.17565944,68.28651582)(800.15565946,68.21651589)(800.10566162,68.15651611)
\curveto(800.05565956,68.08651602)(799.97565964,68.04651606)(799.86566162,68.03651611)
\curveto(799.76565985,68.02651608)(799.65565996,68.02151608)(799.53566162,68.02151611)
\lineto(799.26566162,68.02151611)
\curveto(799.2156604,68.04151606)(799.16566045,68.05651605)(799.11566162,68.06651611)
\curveto(799.07566054,68.08651602)(799.04566057,68.11151599)(799.02566162,68.14151611)
\curveto(798.97566064,68.21151589)(798.94566067,68.29651581)(798.93566162,68.39651611)
\lineto(798.93566162,68.72651611)
\lineto(798.93566162,69.88151611)
\lineto(798.93566162,74.03651611)
\lineto(798.93566162,75.07151611)
\lineto(798.93566162,75.37151611)
\curveto(798.94566067,75.47150863)(798.97566064,75.55650855)(799.02566162,75.62651611)
\curveto(799.05566056,75.66650844)(799.10566051,75.69650841)(799.17566162,75.71651611)
\curveto(799.25566036,75.73650837)(799.34066027,75.74650836)(799.43066162,75.74651611)
\curveto(799.52066009,75.75650835)(799.61066,75.75650835)(799.70066162,75.74651611)
\curveto(799.79065982,75.73650837)(799.86065975,75.72150838)(799.91066162,75.70151611)
\curveto(799.99065962,75.67150843)(800.04065957,75.61150849)(800.06066162,75.52151611)
\curveto(800.09065952,75.44150866)(800.10565951,75.35150875)(800.10566162,75.25151611)
\lineto(800.10566162,74.95151611)
\curveto(800.10565951,74.85150925)(800.12565949,74.76150934)(800.16566162,74.68151611)
\curveto(800.17565944,74.66150944)(800.18565943,74.64650946)(800.19566162,74.63651611)
\lineto(800.24066162,74.59151611)
\curveto(800.35065926,74.59150951)(800.44065917,74.63650947)(800.51066162,74.72651611)
\curveto(800.58065903,74.82650928)(800.64065897,74.9065092)(800.69066162,74.96651611)
\lineto(800.78066162,75.05651611)
\curveto(800.87065874,75.16650894)(800.99565862,75.28150882)(801.15566162,75.40151611)
\curveto(801.3156583,75.52150858)(801.46565815,75.61150849)(801.60566162,75.67151611)
\curveto(801.69565792,75.72150838)(801.79065782,75.75650835)(801.89066162,75.77651611)
\curveto(801.99065762,75.8065083)(802.09565752,75.83650827)(802.20566162,75.86651611)
\curveto(802.26565735,75.87650823)(802.32565729,75.88150822)(802.38566162,75.88151611)
\curveto(802.44565717,75.89150821)(802.50065711,75.9015082)(802.55066162,75.91151611)
}
}
{
\newrgbcolor{curcolor}{0 0 0}
\pscustom[linestyle=none,fillstyle=solid,fillcolor=curcolor]
{
\newpath
\moveto(264.47920532,60.82939941)
\lineto(269.38420532,60.82939941)
\lineto(270.67420532,60.82939941)
\curveto(270.78419744,60.82938872)(270.89419733,60.82938872)(271.00420532,60.82939941)
\curveto(271.11419711,60.83938871)(271.20419702,60.81938873)(271.27420532,60.76939941)
\curveto(271.30419692,60.7493888)(271.3291969,60.72438882)(271.34920532,60.69439941)
\curveto(271.36919686,60.66438888)(271.38919684,60.63438891)(271.40920532,60.60439941)
\curveto(271.4291968,60.53438901)(271.43919679,60.41938913)(271.43920532,60.25939941)
\curveto(271.43919679,60.10938944)(271.4291968,59.99438955)(271.40920532,59.91439941)
\curveto(271.36919686,59.77438977)(271.28419694,59.69438985)(271.15420532,59.67439941)
\curveto(271.0241972,59.66438988)(270.86919736,59.65938989)(270.68920532,59.65939941)
\lineto(269.18920532,59.65939941)
\lineto(266.66920532,59.65939941)
\lineto(266.09920532,59.65939941)
\curveto(265.88920234,59.66938988)(265.73420249,59.6443899)(265.63420532,59.58439941)
\curveto(265.53420269,59.52439002)(265.47920275,59.41939013)(265.46920532,59.26939941)
\lineto(265.46920532,58.80439941)
\lineto(265.46920532,57.27439941)
\curveto(265.46920276,57.16439238)(265.46420276,57.03439251)(265.45420532,56.88439941)
\curveto(265.45420277,56.73439281)(265.46420276,56.61439293)(265.48420532,56.52439941)
\curveto(265.51420271,56.40439314)(265.57420265,56.32439322)(265.66420532,56.28439941)
\curveto(265.70420252,56.26439328)(265.77420245,56.2443933)(265.87420532,56.22439941)
\lineto(266.02420532,56.22439941)
\curveto(266.06420216,56.21439333)(266.10420212,56.20939334)(266.14420532,56.20939941)
\curveto(266.19420203,56.21939333)(266.24420198,56.22439332)(266.29420532,56.22439941)
\lineto(266.80420532,56.22439941)
\lineto(269.74420532,56.22439941)
\lineto(270.04420532,56.22439941)
\curveto(270.15419807,56.23439331)(270.26419796,56.23439331)(270.37420532,56.22439941)
\curveto(270.49419773,56.22439332)(270.59919763,56.21439333)(270.68920532,56.19439941)
\curveto(270.78919744,56.18439336)(270.86419736,56.16439338)(270.91420532,56.13439941)
\curveto(270.94419728,56.11439343)(270.96919726,56.06939348)(270.98920532,55.99939941)
\curveto(271.00919722,55.92939362)(271.0241972,55.85439369)(271.03420532,55.77439941)
\curveto(271.04419718,55.69439385)(271.04419718,55.60939394)(271.03420532,55.51939941)
\curveto(271.03419719,55.43939411)(271.0241972,55.36939418)(271.00420532,55.30939941)
\curveto(270.98419724,55.21939433)(270.93919729,55.15439439)(270.86920532,55.11439941)
\curveto(270.84919738,55.09439445)(270.81919741,55.07939447)(270.77920532,55.06939941)
\curveto(270.74919748,55.06939448)(270.71919751,55.06439448)(270.68920532,55.05439941)
\lineto(270.59920532,55.05439941)
\curveto(270.54919768,55.0443945)(270.49919773,55.03939451)(270.44920532,55.03939941)
\curveto(270.39919783,55.0493945)(270.34919788,55.05439449)(270.29920532,55.05439941)
\lineto(269.74420532,55.05439941)
\lineto(266.57920532,55.05439941)
\lineto(266.21920532,55.05439941)
\curveto(266.10920212,55.06439448)(266.00420222,55.05939449)(265.90420532,55.03939941)
\curveto(265.80420242,55.02939452)(265.71420251,55.00439454)(265.63420532,54.96439941)
\curveto(265.56420266,54.92439462)(265.51420271,54.85439469)(265.48420532,54.75439941)
\curveto(265.46420276,54.69439485)(265.45420277,54.62439492)(265.45420532,54.54439941)
\curveto(265.46420276,54.46439508)(265.46920276,54.38439516)(265.46920532,54.30439941)
\lineto(265.46920532,53.46439941)
\lineto(265.46920532,52.03939941)
\curveto(265.46920276,51.89939765)(265.47420275,51.76939778)(265.48420532,51.64939941)
\curveto(265.49420273,51.53939801)(265.53420269,51.45939809)(265.60420532,51.40939941)
\curveto(265.67420255,51.35939819)(265.75420247,51.32939822)(265.84420532,51.31939941)
\lineto(266.14420532,51.31939941)
\lineto(267.10420532,51.31939941)
\lineto(269.87920532,51.31939941)
\lineto(270.73420532,51.31939941)
\lineto(270.97420532,51.31939941)
\curveto(271.05419717,51.32939822)(271.1241971,51.32439822)(271.18420532,51.30439941)
\curveto(271.30419692,51.26439828)(271.38419684,51.20939834)(271.42420532,51.13939941)
\curveto(271.44419678,51.10939844)(271.45919677,51.05939849)(271.46920532,50.98939941)
\curveto(271.47919675,50.91939863)(271.48419674,50.8443987)(271.48420532,50.76439941)
\curveto(271.49419673,50.69439885)(271.49419673,50.61939893)(271.48420532,50.53939941)
\curveto(271.47419675,50.46939908)(271.46419676,50.41439913)(271.45420532,50.37439941)
\curveto(271.41419681,50.29439925)(271.36919686,50.23939931)(271.31920532,50.20939941)
\curveto(271.25919697,50.16939938)(271.17919705,50.1493994)(271.07920532,50.14939941)
\lineto(270.80920532,50.14939941)
\lineto(269.75920532,50.14939941)
\lineto(265.76920532,50.14939941)
\lineto(264.71920532,50.14939941)
\curveto(264.57920365,50.1493994)(264.45920377,50.15439939)(264.35920532,50.16439941)
\curveto(264.25920397,50.18439936)(264.18420404,50.23439931)(264.13420532,50.31439941)
\curveto(264.09420413,50.37439917)(264.07420415,50.4493991)(264.07420532,50.53939941)
\lineto(264.07420532,50.82439941)
\lineto(264.07420532,51.87439941)
\lineto(264.07420532,55.89439941)
\lineto(264.07420532,59.25439941)
\lineto(264.07420532,60.18439941)
\lineto(264.07420532,60.45439941)
\curveto(264.07420415,60.544389)(264.09420413,60.61438893)(264.13420532,60.66439941)
\curveto(264.17420405,60.73438881)(264.24920398,60.78438876)(264.35920532,60.81439941)
\curveto(264.37920385,60.82438872)(264.39920383,60.82438872)(264.41920532,60.81439941)
\curveto(264.43920379,60.81438873)(264.45920377,60.81938873)(264.47920532,60.82939941)
}
}
{
\newrgbcolor{curcolor}{0 0 0}
\pscustom[linestyle=none,fillstyle=solid,fillcolor=curcolor]
{
\newpath
\moveto(275.4191272,58.05439941)
\curveto(276.13912313,58.06439148)(276.74412253,57.97939157)(277.2341272,57.79939941)
\curveto(277.72412155,57.62939192)(278.10412117,57.32439222)(278.3741272,56.88439941)
\curveto(278.44412083,56.77439277)(278.49912077,56.65939289)(278.5391272,56.53939941)
\curveto(278.57912069,56.42939312)(278.61912065,56.30439324)(278.6591272,56.16439941)
\curveto(278.67912059,56.09439345)(278.68412059,56.01939353)(278.6741272,55.93939941)
\curveto(278.66412061,55.86939368)(278.64912062,55.81439373)(278.6291272,55.77439941)
\curveto(278.60912066,55.75439379)(278.58412069,55.73439381)(278.5541272,55.71439941)
\curveto(278.52412075,55.70439384)(278.49912077,55.68939386)(278.4791272,55.66939941)
\curveto(278.42912084,55.6493939)(278.37912089,55.6443939)(278.3291272,55.65439941)
\curveto(278.27912099,55.66439388)(278.22912104,55.66439388)(278.1791272,55.65439941)
\curveto(278.09912117,55.63439391)(277.99412128,55.62939392)(277.8641272,55.63939941)
\curveto(277.73412154,55.65939389)(277.64412163,55.68439386)(277.5941272,55.71439941)
\curveto(277.51412176,55.76439378)(277.45912181,55.82939372)(277.4291272,55.90939941)
\curveto(277.40912186,55.99939355)(277.3741219,56.08439346)(277.3241272,56.16439941)
\curveto(277.23412204,56.32439322)(277.10912216,56.46939308)(276.9491272,56.59939941)
\curveto(276.83912243,56.67939287)(276.71912255,56.73939281)(276.5891272,56.77939941)
\curveto(276.45912281,56.81939273)(276.31912295,56.85939269)(276.1691272,56.89939941)
\curveto(276.11912315,56.91939263)(276.0691232,56.92439262)(276.0191272,56.91439941)
\curveto(275.9691233,56.91439263)(275.91912335,56.91939263)(275.8691272,56.92939941)
\curveto(275.80912346,56.9493926)(275.73412354,56.95939259)(275.6441272,56.95939941)
\curveto(275.55412372,56.95939259)(275.47912379,56.9493926)(275.4191272,56.92939941)
\lineto(275.3291272,56.92939941)
\lineto(275.1791272,56.89939941)
\curveto(275.12912414,56.89939265)(275.07912419,56.89439265)(275.0291272,56.88439941)
\curveto(274.7691245,56.82439272)(274.55412472,56.73939281)(274.3841272,56.62939941)
\curveto(274.21412506,56.51939303)(274.09912517,56.33439321)(274.0391272,56.07439941)
\curveto(274.01912525,56.00439354)(274.01412526,55.93439361)(274.0241272,55.86439941)
\curveto(274.04412523,55.79439375)(274.06412521,55.73439381)(274.0841272,55.68439941)
\curveto(274.14412513,55.53439401)(274.21412506,55.42439412)(274.2941272,55.35439941)
\curveto(274.38412489,55.29439425)(274.49412478,55.22439432)(274.6241272,55.14439941)
\curveto(274.78412449,55.0443945)(274.96412431,54.96939458)(275.1641272,54.91939941)
\curveto(275.36412391,54.87939467)(275.56412371,54.82939472)(275.7641272,54.76939941)
\curveto(275.89412338,54.72939482)(276.02412325,54.69939485)(276.1541272,54.67939941)
\curveto(276.28412299,54.65939489)(276.41412286,54.62939492)(276.5441272,54.58939941)
\curveto(276.75412252,54.52939502)(276.95912231,54.46939508)(277.1591272,54.40939941)
\curveto(277.35912191,54.35939519)(277.55912171,54.29439525)(277.7591272,54.21439941)
\lineto(277.9091272,54.15439941)
\curveto(277.95912131,54.13439541)(278.00912126,54.10939544)(278.0591272,54.07939941)
\curveto(278.25912101,53.95939559)(278.43412084,53.82439572)(278.5841272,53.67439941)
\curveto(278.73412054,53.52439602)(278.85912041,53.33439621)(278.9591272,53.10439941)
\curveto(278.97912029,53.03439651)(278.99912027,52.93939661)(279.0191272,52.81939941)
\curveto(279.03912023,52.7493968)(279.04912022,52.67439687)(279.0491272,52.59439941)
\curveto(279.05912021,52.52439702)(279.06412021,52.4443971)(279.0641272,52.35439941)
\lineto(279.0641272,52.20439941)
\curveto(279.04412023,52.13439741)(279.03412024,52.06439748)(279.0341272,51.99439941)
\curveto(279.03412024,51.92439762)(279.02412025,51.85439769)(279.0041272,51.78439941)
\curveto(278.9741203,51.67439787)(278.93912033,51.56939798)(278.8991272,51.46939941)
\curveto(278.85912041,51.36939818)(278.81412046,51.27939827)(278.7641272,51.19939941)
\curveto(278.60412067,50.93939861)(278.39912087,50.72939882)(278.1491272,50.56939941)
\curveto(277.89912137,50.41939913)(277.61912165,50.28939926)(277.3091272,50.17939941)
\curveto(277.21912205,50.1493994)(277.12412215,50.12939942)(277.0241272,50.11939941)
\curveto(276.93412234,50.09939945)(276.84412243,50.07439947)(276.7541272,50.04439941)
\curveto(276.65412262,50.02439952)(276.55412272,50.01439953)(276.4541272,50.01439941)
\curveto(276.35412292,50.01439953)(276.25412302,50.00439954)(276.1541272,49.98439941)
\lineto(276.0041272,49.98439941)
\curveto(275.95412332,49.97439957)(275.88412339,49.96939958)(275.7941272,49.96939941)
\curveto(275.70412357,49.96939958)(275.63412364,49.97439957)(275.5841272,49.98439941)
\lineto(275.4191272,49.98439941)
\curveto(275.35912391,50.00439954)(275.29412398,50.01439953)(275.2241272,50.01439941)
\curveto(275.15412412,50.00439954)(275.09412418,50.00939954)(275.0441272,50.02939941)
\curveto(274.99412428,50.03939951)(274.92912434,50.0443995)(274.8491272,50.04439941)
\lineto(274.6091272,50.10439941)
\curveto(274.53912473,50.11439943)(274.46412481,50.13439941)(274.3841272,50.16439941)
\curveto(274.0741252,50.26439928)(273.80412547,50.38939916)(273.5741272,50.53939941)
\curveto(273.34412593,50.68939886)(273.14412613,50.88439866)(272.9741272,51.12439941)
\curveto(272.88412639,51.25439829)(272.80912646,51.38939816)(272.7491272,51.52939941)
\curveto(272.68912658,51.66939788)(272.63412664,51.82439772)(272.5841272,51.99439941)
\curveto(272.56412671,52.05439749)(272.55412672,52.12439742)(272.5541272,52.20439941)
\curveto(272.56412671,52.29439725)(272.57912669,52.36439718)(272.5991272,52.41439941)
\curveto(272.62912664,52.45439709)(272.67912659,52.49439705)(272.7491272,52.53439941)
\curveto(272.79912647,52.55439699)(272.8691264,52.56439698)(272.9591272,52.56439941)
\curveto(273.04912622,52.57439697)(273.13912613,52.57439697)(273.2291272,52.56439941)
\curveto(273.31912595,52.55439699)(273.40412587,52.53939701)(273.4841272,52.51939941)
\curveto(273.5741257,52.50939704)(273.63412564,52.49439705)(273.6641272,52.47439941)
\curveto(273.73412554,52.42439712)(273.77912549,52.3493972)(273.7991272,52.24939941)
\curveto(273.82912544,52.15939739)(273.86412541,52.07439747)(273.9041272,51.99439941)
\curveto(274.00412527,51.77439777)(274.13912513,51.60439794)(274.3091272,51.48439941)
\curveto(274.42912484,51.39439815)(274.56412471,51.32439822)(274.7141272,51.27439941)
\curveto(274.86412441,51.22439832)(275.02412425,51.17439837)(275.1941272,51.12439941)
\lineto(275.5091272,51.07939941)
\lineto(275.5991272,51.07939941)
\curveto(275.6691236,51.05939849)(275.75912351,51.0493985)(275.8691272,51.04939941)
\curveto(275.98912328,51.0493985)(276.08912318,51.05939849)(276.1691272,51.07939941)
\curveto(276.23912303,51.07939847)(276.29412298,51.08439846)(276.3341272,51.09439941)
\curveto(276.39412288,51.10439844)(276.45412282,51.10939844)(276.5141272,51.10939941)
\curveto(276.5741227,51.11939843)(276.62912264,51.12939842)(276.6791272,51.13939941)
\curveto(276.9691223,51.21939833)(277.19912207,51.32439822)(277.3691272,51.45439941)
\curveto(277.53912173,51.58439796)(277.65912161,51.80439774)(277.7291272,52.11439941)
\curveto(277.74912152,52.16439738)(277.75412152,52.21939733)(277.7441272,52.27939941)
\curveto(277.73412154,52.33939721)(277.72412155,52.38439716)(277.7141272,52.41439941)
\curveto(277.66412161,52.60439694)(277.59412168,52.7443968)(277.5041272,52.83439941)
\curveto(277.41412186,52.93439661)(277.29912197,53.02439652)(277.1591272,53.10439941)
\curveto(277.0691222,53.16439638)(276.9691223,53.21439633)(276.8591272,53.25439941)
\lineto(276.5291272,53.37439941)
\curveto(276.49912277,53.38439616)(276.4691228,53.38939616)(276.4391272,53.38939941)
\curveto(276.41912285,53.38939616)(276.39412288,53.39939615)(276.3641272,53.41939941)
\curveto(276.02412325,53.52939602)(275.6691236,53.60939594)(275.2991272,53.65939941)
\curveto(274.93912433,53.71939583)(274.59912467,53.81439573)(274.2791272,53.94439941)
\curveto(274.17912509,53.98439556)(274.08412519,54.01939553)(273.9941272,54.04939941)
\curveto(273.90412537,54.07939547)(273.81912545,54.11939543)(273.7391272,54.16939941)
\curveto(273.54912572,54.27939527)(273.3741259,54.40439514)(273.2141272,54.54439941)
\curveto(273.05412622,54.68439486)(272.92912634,54.85939469)(272.8391272,55.06939941)
\curveto(272.80912646,55.13939441)(272.78412649,55.20939434)(272.7641272,55.27939941)
\curveto(272.75412652,55.3493942)(272.73912653,55.42439412)(272.7191272,55.50439941)
\curveto(272.68912658,55.62439392)(272.67912659,55.75939379)(272.6891272,55.90939941)
\curveto(272.69912657,56.06939348)(272.71412656,56.20439334)(272.7341272,56.31439941)
\curveto(272.75412652,56.36439318)(272.76412651,56.40439314)(272.7641272,56.43439941)
\curveto(272.7741265,56.47439307)(272.78912648,56.51439303)(272.8091272,56.55439941)
\curveto(272.89912637,56.78439276)(273.01912625,56.98439256)(273.1691272,57.15439941)
\curveto(273.32912594,57.32439222)(273.50912576,57.47439207)(273.7091272,57.60439941)
\curveto(273.85912541,57.69439185)(274.02412525,57.76439178)(274.2041272,57.81439941)
\curveto(274.38412489,57.87439167)(274.5741247,57.92939162)(274.7741272,57.97939941)
\curveto(274.84412443,57.98939156)(274.90912436,57.99939155)(274.9691272,58.00939941)
\curveto(275.03912423,58.01939153)(275.11412416,58.02939152)(275.1941272,58.03939941)
\curveto(275.22412405,58.0493915)(275.26412401,58.0493915)(275.3141272,58.03939941)
\curveto(275.36412391,58.02939152)(275.39912387,58.03439151)(275.4191272,58.05439941)
}
}
{
\newrgbcolor{curcolor}{0 0 0}
\pscustom[linestyle=none,fillstyle=solid,fillcolor=curcolor]
{
\newpath
\moveto(281.4341272,60.21439941)
\curveto(281.58412519,60.21438933)(281.73412504,60.20938934)(281.8841272,60.19939941)
\curveto(282.03412474,60.19938935)(282.13912463,60.15938939)(282.1991272,60.07939941)
\curveto(282.24912452,60.01938953)(282.2741245,59.93438961)(282.2741272,59.82439941)
\curveto(282.28412449,59.72438982)(282.28912448,59.61938993)(282.2891272,59.50939941)
\lineto(282.2891272,58.63939941)
\curveto(282.28912448,58.55939099)(282.28412449,58.47439107)(282.2741272,58.38439941)
\curveto(282.2741245,58.30439124)(282.28412449,58.23439131)(282.3041272,58.17439941)
\curveto(282.34412443,58.03439151)(282.43412434,57.9443916)(282.5741272,57.90439941)
\curveto(282.62412415,57.89439165)(282.6691241,57.88939166)(282.7091272,57.88939941)
\lineto(282.8591272,57.88939941)
\lineto(283.2641272,57.88939941)
\curveto(283.42412335,57.89939165)(283.53912323,57.88939166)(283.6091272,57.85939941)
\curveto(283.69912307,57.79939175)(283.75912301,57.73939181)(283.7891272,57.67939941)
\curveto(283.80912296,57.63939191)(283.81912295,57.59439195)(283.8191272,57.54439941)
\lineto(283.8191272,57.39439941)
\curveto(283.81912295,57.28439226)(283.81412296,57.17939237)(283.8041272,57.07939941)
\curveto(283.79412298,56.98939256)(283.75912301,56.91939263)(283.6991272,56.86939941)
\curveto(283.63912313,56.81939273)(283.55412322,56.78939276)(283.4441272,56.77939941)
\lineto(283.1141272,56.77939941)
\curveto(283.00412377,56.78939276)(282.89412388,56.79439275)(282.7841272,56.79439941)
\curveto(282.6741241,56.79439275)(282.57912419,56.77939277)(282.4991272,56.74939941)
\curveto(282.42912434,56.71939283)(282.37912439,56.66939288)(282.3491272,56.59939941)
\curveto(282.31912445,56.52939302)(282.29912447,56.4443931)(282.2891272,56.34439941)
\curveto(282.27912449,56.25439329)(282.2741245,56.15439339)(282.2741272,56.04439941)
\curveto(282.28412449,55.9443936)(282.28912448,55.8443937)(282.2891272,55.74439941)
\lineto(282.2891272,52.77439941)
\curveto(282.28912448,52.55439699)(282.28412449,52.31939723)(282.2741272,52.06939941)
\curveto(282.2741245,51.82939772)(282.31912445,51.6443979)(282.4091272,51.51439941)
\curveto(282.45912431,51.43439811)(282.52412425,51.37939817)(282.6041272,51.34939941)
\curveto(282.68412409,51.31939823)(282.77912399,51.29439825)(282.8891272,51.27439941)
\curveto(282.91912385,51.26439828)(282.94912382,51.25939829)(282.9791272,51.25939941)
\curveto(283.01912375,51.26939828)(283.05412372,51.26939828)(283.0841272,51.25939941)
\lineto(283.2791272,51.25939941)
\curveto(283.37912339,51.25939829)(283.4691233,51.2493983)(283.5491272,51.22939941)
\curveto(283.63912313,51.21939833)(283.70412307,51.18439836)(283.7441272,51.12439941)
\curveto(283.76412301,51.09439845)(283.77912299,51.03939851)(283.7891272,50.95939941)
\curveto(283.80912296,50.88939866)(283.81912295,50.81439873)(283.8191272,50.73439941)
\curveto(283.82912294,50.65439889)(283.82912294,50.57439897)(283.8191272,50.49439941)
\curveto(283.80912296,50.42439912)(283.78912298,50.36939918)(283.7591272,50.32939941)
\curveto(283.71912305,50.25939929)(283.64412313,50.20939934)(283.5341272,50.17939941)
\curveto(283.45412332,50.15939939)(283.36412341,50.1493994)(283.2641272,50.14939941)
\curveto(283.16412361,50.15939939)(283.0741237,50.16439938)(282.9941272,50.16439941)
\curveto(282.93412384,50.16439938)(282.8741239,50.15939939)(282.8141272,50.14939941)
\curveto(282.75412402,50.1493994)(282.69912407,50.15439939)(282.6491272,50.16439941)
\lineto(282.4691272,50.16439941)
\curveto(282.41912435,50.17439937)(282.3691244,50.17939937)(282.3191272,50.17939941)
\curveto(282.27912449,50.18939936)(282.23412454,50.19439935)(282.1841272,50.19439941)
\curveto(281.98412479,50.2443993)(281.80912496,50.29939925)(281.6591272,50.35939941)
\curveto(281.51912525,50.41939913)(281.39912537,50.52439902)(281.2991272,50.67439941)
\curveto(281.15912561,50.87439867)(281.07912569,51.12439842)(281.0591272,51.42439941)
\curveto(281.03912573,51.73439781)(281.02912574,52.06439748)(281.0291272,52.41439941)
\lineto(281.0291272,56.34439941)
\curveto(280.99912577,56.47439307)(280.9691258,56.56939298)(280.9391272,56.62939941)
\curveto(280.91912585,56.68939286)(280.84912592,56.73939281)(280.7291272,56.77939941)
\curveto(280.68912608,56.78939276)(280.64912612,56.78939276)(280.6091272,56.77939941)
\curveto(280.5691262,56.76939278)(280.52912624,56.77439277)(280.4891272,56.79439941)
\lineto(280.2491272,56.79439941)
\curveto(280.11912665,56.79439275)(280.00912676,56.80439274)(279.9191272,56.82439941)
\curveto(279.83912693,56.85439269)(279.78412699,56.91439263)(279.7541272,57.00439941)
\curveto(279.73412704,57.0443925)(279.71912705,57.08939246)(279.7091272,57.13939941)
\lineto(279.7091272,57.28939941)
\curveto(279.70912706,57.42939212)(279.71912705,57.544392)(279.7391272,57.63439941)
\curveto(279.75912701,57.73439181)(279.81912695,57.80939174)(279.9191272,57.85939941)
\curveto(280.02912674,57.89939165)(280.1691266,57.90939164)(280.3391272,57.88939941)
\curveto(280.51912625,57.86939168)(280.6691261,57.87939167)(280.7891272,57.91939941)
\curveto(280.87912589,57.96939158)(280.94912582,58.03939151)(280.9991272,58.12939941)
\curveto(281.01912575,58.18939136)(281.02912574,58.26439128)(281.0291272,58.35439941)
\lineto(281.0291272,58.60939941)
\lineto(281.0291272,59.53939941)
\lineto(281.0291272,59.77939941)
\curveto(281.02912574,59.86938968)(281.03912573,59.9443896)(281.0591272,60.00439941)
\curveto(281.09912567,60.08438946)(281.1741256,60.1493894)(281.2841272,60.19939941)
\curveto(281.31412546,60.19938935)(281.33912543,60.19938935)(281.3591272,60.19939941)
\curveto(281.38912538,60.20938934)(281.41412536,60.21438933)(281.4341272,60.21439941)
}
}
{
\newrgbcolor{curcolor}{0 0 0}
\pscustom[linestyle=none,fillstyle=solid,fillcolor=curcolor]
{
\newpath
\moveto(285.67092407,57.87439941)
\lineto(286.10592407,57.87439941)
\curveto(286.25592211,57.87439167)(286.360922,57.83439171)(286.42092407,57.75439941)
\curveto(286.47092189,57.67439187)(286.49592187,57.57439197)(286.49592407,57.45439941)
\curveto(286.50592186,57.33439221)(286.51092185,57.21439233)(286.51092407,57.09439941)
\lineto(286.51092407,55.66939941)
\lineto(286.51092407,53.40439941)
\lineto(286.51092407,52.71439941)
\curveto(286.51092185,52.48439706)(286.53592183,52.28439726)(286.58592407,52.11439941)
\curveto(286.74592162,51.66439788)(287.04592132,51.3493982)(287.48592407,51.16939941)
\curveto(287.70592066,51.07939847)(287.97092039,51.0443985)(288.28092407,51.06439941)
\curveto(288.59091977,51.09439845)(288.84091952,51.1493984)(289.03092407,51.22939941)
\curveto(289.360919,51.36939818)(289.62091874,51.544398)(289.81092407,51.75439941)
\curveto(290.01091835,51.97439757)(290.1659182,52.25939729)(290.27592407,52.60939941)
\curveto(290.30591806,52.68939686)(290.32591804,52.76939678)(290.33592407,52.84939941)
\curveto(290.34591802,52.92939662)(290.360918,53.01439653)(290.38092407,53.10439941)
\curveto(290.39091797,53.15439639)(290.39091797,53.19939635)(290.38092407,53.23939941)
\curveto(290.38091798,53.27939627)(290.39091797,53.32439622)(290.41092407,53.37439941)
\lineto(290.41092407,53.68939941)
\curveto(290.43091793,53.76939578)(290.43591793,53.85939569)(290.42592407,53.95939941)
\curveto(290.41591795,54.06939548)(290.41091795,54.16939538)(290.41092407,54.25939941)
\lineto(290.41092407,55.42939941)
\lineto(290.41092407,57.01939941)
\curveto(290.41091795,57.13939241)(290.40591796,57.26439228)(290.39592407,57.39439941)
\curveto(290.39591797,57.53439201)(290.42091794,57.6443919)(290.47092407,57.72439941)
\curveto(290.51091785,57.77439177)(290.55591781,57.80439174)(290.60592407,57.81439941)
\curveto(290.6659177,57.83439171)(290.73591763,57.85439169)(290.81592407,57.87439941)
\lineto(291.04092407,57.87439941)
\curveto(291.1609172,57.87439167)(291.2659171,57.86939168)(291.35592407,57.85939941)
\curveto(291.45591691,57.8493917)(291.53091683,57.80439174)(291.58092407,57.72439941)
\curveto(291.63091673,57.67439187)(291.65591671,57.59939195)(291.65592407,57.49939941)
\lineto(291.65592407,57.21439941)
\lineto(291.65592407,56.19439941)
\lineto(291.65592407,52.15939941)
\lineto(291.65592407,50.80939941)
\curveto(291.65591671,50.68939886)(291.65091671,50.57439897)(291.64092407,50.46439941)
\curveto(291.64091672,50.36439918)(291.60591676,50.28939926)(291.53592407,50.23939941)
\curveto(291.49591687,50.20939934)(291.43591693,50.18439936)(291.35592407,50.16439941)
\curveto(291.27591709,50.15439939)(291.18591718,50.1443994)(291.08592407,50.13439941)
\curveto(290.99591737,50.13439941)(290.90591746,50.13939941)(290.81592407,50.14939941)
\curveto(290.73591763,50.15939939)(290.67591769,50.17939937)(290.63592407,50.20939941)
\curveto(290.58591778,50.2493993)(290.54091782,50.31439923)(290.50092407,50.40439941)
\curveto(290.49091787,50.4443991)(290.48091788,50.49939905)(290.47092407,50.56939941)
\curveto(290.47091789,50.63939891)(290.4659179,50.70439884)(290.45592407,50.76439941)
\curveto(290.44591792,50.83439871)(290.42591794,50.88939866)(290.39592407,50.92939941)
\curveto(290.365918,50.96939858)(290.32091804,50.98439856)(290.26092407,50.97439941)
\curveto(290.18091818,50.95439859)(290.10091826,50.89439865)(290.02092407,50.79439941)
\curveto(289.94091842,50.70439884)(289.8659185,50.63439891)(289.79592407,50.58439941)
\curveto(289.57591879,50.42439912)(289.32591904,50.28439926)(289.04592407,50.16439941)
\curveto(288.93591943,50.11439943)(288.82091954,50.08439946)(288.70092407,50.07439941)
\curveto(288.59091977,50.05439949)(288.47591989,50.02939952)(288.35592407,49.99939941)
\curveto(288.30592006,49.98939956)(288.25092011,49.98939956)(288.19092407,49.99939941)
\curveto(288.14092022,50.00939954)(288.09092027,50.00439954)(288.04092407,49.98439941)
\curveto(287.94092042,49.96439958)(287.85092051,49.96439958)(287.77092407,49.98439941)
\lineto(287.62092407,49.98439941)
\curveto(287.57092079,50.00439954)(287.51092085,50.01439953)(287.44092407,50.01439941)
\curveto(287.38092098,50.01439953)(287.32592104,50.01939953)(287.27592407,50.02939941)
\curveto(287.23592113,50.0493995)(287.19592117,50.05939949)(287.15592407,50.05939941)
\curveto(287.12592124,50.0493995)(287.08592128,50.05439949)(287.03592407,50.07439941)
\lineto(286.79592407,50.13439941)
\curveto(286.72592164,50.15439939)(286.65092171,50.18439936)(286.57092407,50.22439941)
\curveto(286.31092205,50.33439921)(286.09092227,50.47939907)(285.91092407,50.65939941)
\curveto(285.74092262,50.8493987)(285.60092276,51.07439847)(285.49092407,51.33439941)
\curveto(285.45092291,51.42439812)(285.42092294,51.51439803)(285.40092407,51.60439941)
\lineto(285.34092407,51.90439941)
\curveto(285.32092304,51.96439758)(285.31092305,52.01939753)(285.31092407,52.06939941)
\curveto(285.32092304,52.12939742)(285.31592305,52.19439735)(285.29592407,52.26439941)
\curveto(285.28592308,52.28439726)(285.28092308,52.30939724)(285.28092407,52.33939941)
\curveto(285.28092308,52.37939717)(285.27592309,52.41439713)(285.26592407,52.44439941)
\lineto(285.26592407,52.59439941)
\curveto(285.25592311,52.63439691)(285.25092311,52.67939687)(285.25092407,52.72939941)
\curveto(285.2609231,52.78939676)(285.2659231,52.8443967)(285.26592407,52.89439941)
\lineto(285.26592407,53.49439941)
\lineto(285.26592407,56.25439941)
\lineto(285.26592407,57.21439941)
\lineto(285.26592407,57.48439941)
\curveto(285.2659231,57.57439197)(285.28592308,57.6493919)(285.32592407,57.70939941)
\curveto(285.365923,57.77939177)(285.44092292,57.82939172)(285.55092407,57.85939941)
\curveto(285.57092279,57.86939168)(285.59092277,57.86939168)(285.61092407,57.85939941)
\curveto(285.63092273,57.85939169)(285.65092271,57.86439168)(285.67092407,57.87439941)
}
}
{
\newrgbcolor{curcolor}{0 0 0}
\pscustom[linestyle=none,fillstyle=solid,fillcolor=curcolor]
{
\newpath
\moveto(300.51553345,50.95939941)
\lineto(300.51553345,50.56939941)
\curveto(300.51552557,50.4493991)(300.4905256,50.3493992)(300.44053345,50.26939941)
\curveto(300.3905257,50.19939935)(300.30552578,50.15939939)(300.18553345,50.14939941)
\lineto(299.84053345,50.14939941)
\curveto(299.78052631,50.1493994)(299.72052637,50.1443994)(299.66053345,50.13439941)
\curveto(299.61052648,50.13439941)(299.56552652,50.1443994)(299.52553345,50.16439941)
\curveto(299.43552665,50.18439936)(299.37552671,50.22439932)(299.34553345,50.28439941)
\curveto(299.30552678,50.33439921)(299.28052681,50.39439915)(299.27053345,50.46439941)
\curveto(299.27052682,50.53439901)(299.25552683,50.60439894)(299.22553345,50.67439941)
\curveto(299.21552687,50.69439885)(299.20052689,50.70939884)(299.18053345,50.71939941)
\curveto(299.17052692,50.73939881)(299.15552693,50.75939879)(299.13553345,50.77939941)
\curveto(299.03552705,50.78939876)(298.95552713,50.76939878)(298.89553345,50.71939941)
\curveto(298.84552724,50.66939888)(298.7905273,50.61939893)(298.73053345,50.56939941)
\curveto(298.53052756,50.41939913)(298.33052776,50.30439924)(298.13053345,50.22439941)
\curveto(297.95052814,50.1443994)(297.74052835,50.08439946)(297.50053345,50.04439941)
\curveto(297.27052882,50.00439954)(297.03052906,49.98439956)(296.78053345,49.98439941)
\curveto(296.54052955,49.97439957)(296.30052979,49.98939956)(296.06053345,50.02939941)
\curveto(295.82053027,50.05939949)(295.61053048,50.11439943)(295.43053345,50.19439941)
\curveto(294.91053118,50.41439913)(294.4905316,50.70939884)(294.17053345,51.07939941)
\curveto(293.85053224,51.45939809)(293.60053249,51.92939762)(293.42053345,52.48939941)
\curveto(293.38053271,52.57939697)(293.35053274,52.66939688)(293.33053345,52.75939941)
\curveto(293.32053277,52.85939669)(293.30053279,52.95939659)(293.27053345,53.05939941)
\curveto(293.26053283,53.10939644)(293.25553283,53.15939639)(293.25553345,53.20939941)
\curveto(293.25553283,53.25939629)(293.25053284,53.30939624)(293.24053345,53.35939941)
\curveto(293.22053287,53.40939614)(293.21053288,53.45939609)(293.21053345,53.50939941)
\curveto(293.22053287,53.56939598)(293.22053287,53.62439592)(293.21053345,53.67439941)
\lineto(293.21053345,53.82439941)
\curveto(293.1905329,53.87439567)(293.18053291,53.93939561)(293.18053345,54.01939941)
\curveto(293.18053291,54.09939545)(293.1905329,54.16439538)(293.21053345,54.21439941)
\lineto(293.21053345,54.37939941)
\curveto(293.23053286,54.4493951)(293.23553285,54.51939503)(293.22553345,54.58939941)
\curveto(293.22553286,54.66939488)(293.23553285,54.7443948)(293.25553345,54.81439941)
\curveto(293.26553282,54.86439468)(293.27053282,54.90939464)(293.27053345,54.94939941)
\curveto(293.27053282,54.98939456)(293.27553281,55.03439451)(293.28553345,55.08439941)
\curveto(293.31553277,55.18439436)(293.34053275,55.27939427)(293.36053345,55.36939941)
\curveto(293.38053271,55.46939408)(293.40553268,55.56439398)(293.43553345,55.65439941)
\curveto(293.56553252,56.03439351)(293.73053236,56.37439317)(293.93053345,56.67439941)
\curveto(294.14053195,56.98439256)(294.3905317,57.23939231)(294.68053345,57.43939941)
\curveto(294.85053124,57.55939199)(295.02553106,57.65939189)(295.20553345,57.73939941)
\curveto(295.39553069,57.81939173)(295.60053049,57.88939166)(295.82053345,57.94939941)
\curveto(295.8905302,57.95939159)(295.95553013,57.96939158)(296.01553345,57.97939941)
\curveto(296.08553,57.98939156)(296.15552993,58.00439154)(296.22553345,58.02439941)
\lineto(296.37553345,58.02439941)
\curveto(296.45552963,58.0443915)(296.57052952,58.05439149)(296.72053345,58.05439941)
\curveto(296.88052921,58.05439149)(297.00052909,58.0443915)(297.08053345,58.02439941)
\curveto(297.12052897,58.01439153)(297.17552891,58.00939154)(297.24553345,58.00939941)
\curveto(297.35552873,57.97939157)(297.46552862,57.95439159)(297.57553345,57.93439941)
\curveto(297.6855284,57.92439162)(297.7905283,57.89439165)(297.89053345,57.84439941)
\curveto(298.04052805,57.78439176)(298.18052791,57.71939183)(298.31053345,57.64939941)
\curveto(298.45052764,57.57939197)(298.58052751,57.49939205)(298.70053345,57.40939941)
\curveto(298.76052733,57.35939219)(298.82052727,57.30439224)(298.88053345,57.24439941)
\curveto(298.95052714,57.19439235)(299.04052705,57.17939237)(299.15053345,57.19939941)
\curveto(299.17052692,57.22939232)(299.1855269,57.25439229)(299.19553345,57.27439941)
\curveto(299.21552687,57.29439225)(299.23052686,57.32439222)(299.24053345,57.36439941)
\curveto(299.27052682,57.45439209)(299.28052681,57.56939198)(299.27053345,57.70939941)
\lineto(299.27053345,58.08439941)
\lineto(299.27053345,59.80939941)
\lineto(299.27053345,60.27439941)
\curveto(299.27052682,60.45438909)(299.29552679,60.58438896)(299.34553345,60.66439941)
\curveto(299.3855267,60.73438881)(299.44552664,60.77938877)(299.52553345,60.79939941)
\curveto(299.54552654,60.79938875)(299.57052652,60.79938875)(299.60053345,60.79939941)
\curveto(299.63052646,60.80938874)(299.65552643,60.81438873)(299.67553345,60.81439941)
\curveto(299.81552627,60.82438872)(299.96052613,60.82438872)(300.11053345,60.81439941)
\curveto(300.27052582,60.81438873)(300.38052571,60.77438877)(300.44053345,60.69439941)
\curveto(300.4905256,60.61438893)(300.51552557,60.51438903)(300.51553345,60.39439941)
\lineto(300.51553345,60.01939941)
\lineto(300.51553345,50.95939941)
\moveto(299.30053345,53.79439941)
\curveto(299.32052677,53.8443957)(299.33052676,53.90939564)(299.33053345,53.98939941)
\curveto(299.33052676,54.07939547)(299.32052677,54.1493954)(299.30053345,54.19939941)
\lineto(299.30053345,54.42439941)
\curveto(299.28052681,54.51439503)(299.26552682,54.60439494)(299.25553345,54.69439941)
\curveto(299.24552684,54.79439475)(299.22552686,54.88439466)(299.19553345,54.96439941)
\curveto(299.17552691,55.0443945)(299.15552693,55.11939443)(299.13553345,55.18939941)
\curveto(299.12552696,55.25939429)(299.10552698,55.32939422)(299.07553345,55.39939941)
\curveto(298.95552713,55.69939385)(298.80052729,55.96439358)(298.61053345,56.19439941)
\curveto(298.42052767,56.42439312)(298.18052791,56.60439294)(297.89053345,56.73439941)
\curveto(297.7905283,56.78439276)(297.6855284,56.81939273)(297.57553345,56.83939941)
\curveto(297.47552861,56.86939268)(297.36552872,56.89439265)(297.24553345,56.91439941)
\curveto(297.16552892,56.93439261)(297.07552901,56.9443926)(296.97553345,56.94439941)
\lineto(296.70553345,56.94439941)
\curveto(296.65552943,56.93439261)(296.61052948,56.92439262)(296.57053345,56.91439941)
\lineto(296.43553345,56.91439941)
\curveto(296.35552973,56.89439265)(296.27052982,56.87439267)(296.18053345,56.85439941)
\curveto(296.10052999,56.83439271)(296.02053007,56.80939274)(295.94053345,56.77939941)
\curveto(295.62053047,56.63939291)(295.36053073,56.43439311)(295.16053345,56.16439941)
\curveto(294.97053112,55.90439364)(294.81553127,55.59939395)(294.69553345,55.24939941)
\curveto(294.65553143,55.13939441)(294.62553146,55.02439452)(294.60553345,54.90439941)
\curveto(294.59553149,54.79439475)(294.58053151,54.68439486)(294.56053345,54.57439941)
\curveto(294.56053153,54.53439501)(294.55553153,54.49439505)(294.54553345,54.45439941)
\lineto(294.54553345,54.34939941)
\curveto(294.52553156,54.29939525)(294.51553157,54.2443953)(294.51553345,54.18439941)
\curveto(294.52553156,54.12439542)(294.53053156,54.06939548)(294.53053345,54.01939941)
\lineto(294.53053345,53.68939941)
\curveto(294.53053156,53.58939596)(294.54053155,53.49439605)(294.56053345,53.40439941)
\curveto(294.57053152,53.37439617)(294.57553151,53.32439622)(294.57553345,53.25439941)
\curveto(294.59553149,53.18439636)(294.61053148,53.11439643)(294.62053345,53.04439941)
\lineto(294.68053345,52.83439941)
\curveto(294.7905313,52.48439706)(294.94053115,52.18439736)(295.13053345,51.93439941)
\curveto(295.32053077,51.68439786)(295.56053053,51.47939807)(295.85053345,51.31939941)
\curveto(295.94053015,51.26939828)(296.03053006,51.22939832)(296.12053345,51.19939941)
\curveto(296.21052988,51.16939838)(296.31052978,51.13939841)(296.42053345,51.10939941)
\curveto(296.47052962,51.08939846)(296.52052957,51.08439846)(296.57053345,51.09439941)
\curveto(296.63052946,51.10439844)(296.6855294,51.09939845)(296.73553345,51.07939941)
\curveto(296.77552931,51.06939848)(296.81552927,51.06439848)(296.85553345,51.06439941)
\lineto(296.99053345,51.06439941)
\lineto(297.12553345,51.06439941)
\curveto(297.15552893,51.07439847)(297.20552888,51.07939847)(297.27553345,51.07939941)
\curveto(297.35552873,51.09939845)(297.43552865,51.11439843)(297.51553345,51.12439941)
\curveto(297.59552849,51.1443984)(297.67052842,51.16939838)(297.74053345,51.19939941)
\curveto(298.07052802,51.33939821)(298.33552775,51.51439803)(298.53553345,51.72439941)
\curveto(298.74552734,51.9443976)(298.92052717,52.21939733)(299.06053345,52.54939941)
\curveto(299.11052698,52.65939689)(299.14552694,52.76939678)(299.16553345,52.87939941)
\curveto(299.1855269,52.98939656)(299.21052688,53.09939645)(299.24053345,53.20939941)
\curveto(299.26052683,53.2493963)(299.27052682,53.28439626)(299.27053345,53.31439941)
\curveto(299.27052682,53.35439619)(299.27552681,53.39439615)(299.28553345,53.43439941)
\curveto(299.29552679,53.49439605)(299.29552679,53.55439599)(299.28553345,53.61439941)
\curveto(299.2855268,53.67439587)(299.2905268,53.73439581)(299.30053345,53.79439941)
}
}
{
\newrgbcolor{curcolor}{0 0 0}
\pscustom[linestyle=none,fillstyle=solid,fillcolor=curcolor]
{
\newpath
\moveto(302.74678345,59.37439941)
\curveto(302.66678233,59.43439011)(302.62178237,59.53939001)(302.61178345,59.68939941)
\lineto(302.61178345,60.15439941)
\lineto(302.61178345,60.40939941)
\curveto(302.61178238,60.49938905)(302.62678237,60.57438897)(302.65678345,60.63439941)
\curveto(302.6967823,60.71438883)(302.77678222,60.77438877)(302.89678345,60.81439941)
\curveto(302.91678208,60.82438872)(302.93678206,60.82438872)(302.95678345,60.81439941)
\curveto(302.98678201,60.81438873)(303.01178198,60.81938873)(303.03178345,60.82939941)
\curveto(303.20178179,60.82938872)(303.36178163,60.82438872)(303.51178345,60.81439941)
\curveto(303.66178133,60.80438874)(303.76178123,60.7443888)(303.81178345,60.63439941)
\curveto(303.84178115,60.57438897)(303.85678114,60.49938905)(303.85678345,60.40939941)
\lineto(303.85678345,60.15439941)
\curveto(303.85678114,59.97438957)(303.85178114,59.80438974)(303.84178345,59.64439941)
\curveto(303.84178115,59.48439006)(303.77678122,59.37939017)(303.64678345,59.32939941)
\curveto(303.5967814,59.30939024)(303.54178145,59.29939025)(303.48178345,59.29939941)
\lineto(303.31678345,59.29939941)
\lineto(303.00178345,59.29939941)
\curveto(302.90178209,59.29939025)(302.81678218,59.32439022)(302.74678345,59.37439941)
\moveto(303.85678345,50.86939941)
\lineto(303.85678345,50.55439941)
\curveto(303.86678113,50.45439909)(303.84678115,50.37439917)(303.79678345,50.31439941)
\curveto(303.76678123,50.25439929)(303.72178127,50.21439933)(303.66178345,50.19439941)
\curveto(303.60178139,50.18439936)(303.53178146,50.16939938)(303.45178345,50.14939941)
\lineto(303.22678345,50.14939941)
\curveto(303.0967819,50.1493994)(302.98178201,50.15439939)(302.88178345,50.16439941)
\curveto(302.7917822,50.18439936)(302.72178227,50.23439931)(302.67178345,50.31439941)
\curveto(302.63178236,50.37439917)(302.61178238,50.4493991)(302.61178345,50.53939941)
\lineto(302.61178345,50.82439941)
\lineto(302.61178345,57.16939941)
\lineto(302.61178345,57.48439941)
\curveto(302.61178238,57.59439195)(302.63678236,57.67939187)(302.68678345,57.73939941)
\curveto(302.71678228,57.78939176)(302.75678224,57.81939173)(302.80678345,57.82939941)
\curveto(302.85678214,57.83939171)(302.91178208,57.85439169)(302.97178345,57.87439941)
\curveto(302.991782,57.87439167)(303.01178198,57.86939168)(303.03178345,57.85939941)
\curveto(303.06178193,57.85939169)(303.08678191,57.86439168)(303.10678345,57.87439941)
\curveto(303.23678176,57.87439167)(303.36678163,57.86939168)(303.49678345,57.85939941)
\curveto(303.63678136,57.85939169)(303.73178126,57.81939173)(303.78178345,57.73939941)
\curveto(303.83178116,57.67939187)(303.85678114,57.59939195)(303.85678345,57.49939941)
\lineto(303.85678345,57.21439941)
\lineto(303.85678345,50.86939941)
}
}
{
\newrgbcolor{curcolor}{0 0 0}
\pscustom[linestyle=none,fillstyle=solid,fillcolor=curcolor]
{
\newpath
\moveto(312.6866272,50.70439941)
\curveto(312.71661937,50.544399)(312.70161938,50.40939914)(312.6416272,50.29939941)
\curveto(312.5816195,50.19939935)(312.50161958,50.12439942)(312.4016272,50.07439941)
\curveto(312.35161973,50.05439949)(312.29661979,50.0443995)(312.2366272,50.04439941)
\curveto(312.1866199,50.0443995)(312.13161995,50.03439951)(312.0716272,50.01439941)
\curveto(311.85162023,49.96439958)(311.63162045,49.97939957)(311.4116272,50.05939941)
\curveto(311.20162088,50.12939942)(311.05662103,50.21939933)(310.9766272,50.32939941)
\curveto(310.92662116,50.39939915)(310.8816212,50.47939907)(310.8416272,50.56939941)
\curveto(310.80162128,50.66939888)(310.75162133,50.7493988)(310.6916272,50.80939941)
\curveto(310.67162141,50.82939872)(310.64662144,50.8493987)(310.6166272,50.86939941)
\curveto(310.59662149,50.88939866)(310.56662152,50.89439865)(310.5266272,50.88439941)
\curveto(310.41662167,50.85439869)(310.31162177,50.79939875)(310.2116272,50.71939941)
\curveto(310.12162196,50.63939891)(310.03162205,50.56939898)(309.9416272,50.50939941)
\curveto(309.81162227,50.42939912)(309.67162241,50.35439919)(309.5216272,50.28439941)
\curveto(309.37162271,50.22439932)(309.21162287,50.16939938)(309.0416272,50.11939941)
\curveto(308.94162314,50.08939946)(308.83162325,50.06939948)(308.7116272,50.05939941)
\curveto(308.60162348,50.0493995)(308.49162359,50.03439951)(308.3816272,50.01439941)
\curveto(308.33162375,50.00439954)(308.2866238,49.99939955)(308.2466272,49.99939941)
\lineto(308.1416272,49.99939941)
\curveto(308.03162405,49.97939957)(307.92662416,49.97939957)(307.8266272,49.99939941)
\lineto(307.6916272,49.99939941)
\curveto(307.64162444,50.00939954)(307.59162449,50.01439953)(307.5416272,50.01439941)
\curveto(307.49162459,50.01439953)(307.44662464,50.02439952)(307.4066272,50.04439941)
\curveto(307.36662472,50.05439949)(307.33162475,50.05939949)(307.3016272,50.05939941)
\curveto(307.2816248,50.0493995)(307.25662483,50.0493995)(307.2266272,50.05939941)
\lineto(306.9866272,50.11939941)
\curveto(306.90662518,50.12939942)(306.83162525,50.1493994)(306.7616272,50.17939941)
\curveto(306.46162562,50.30939924)(306.21662587,50.45439909)(306.0266272,50.61439941)
\curveto(305.84662624,50.78439876)(305.69662639,51.01939853)(305.5766272,51.31939941)
\curveto(305.4866266,51.53939801)(305.44162664,51.80439774)(305.4416272,52.11439941)
\lineto(305.4416272,52.42939941)
\curveto(305.45162663,52.47939707)(305.45662663,52.52939702)(305.4566272,52.57939941)
\lineto(305.4866272,52.75939941)
\lineto(305.6066272,53.08939941)
\curveto(305.64662644,53.19939635)(305.69662639,53.29939625)(305.7566272,53.38939941)
\curveto(305.93662615,53.67939587)(306.1816259,53.89439565)(306.4916272,54.03439941)
\curveto(306.80162528,54.17439537)(307.14162494,54.29939525)(307.5116272,54.40939941)
\curveto(307.65162443,54.4493951)(307.79662429,54.47939507)(307.9466272,54.49939941)
\curveto(308.09662399,54.51939503)(308.24662384,54.544395)(308.3966272,54.57439941)
\curveto(308.46662362,54.59439495)(308.53162355,54.60439494)(308.5916272,54.60439941)
\curveto(308.66162342,54.60439494)(308.73662335,54.61439493)(308.8166272,54.63439941)
\curveto(308.8866232,54.65439489)(308.95662313,54.66439488)(309.0266272,54.66439941)
\curveto(309.09662299,54.67439487)(309.17162291,54.68939486)(309.2516272,54.70939941)
\curveto(309.50162258,54.76939478)(309.73662235,54.81939473)(309.9566272,54.85939941)
\curveto(310.17662191,54.90939464)(310.35162173,55.02439452)(310.4816272,55.20439941)
\curveto(310.54162154,55.28439426)(310.59162149,55.38439416)(310.6316272,55.50439941)
\curveto(310.67162141,55.63439391)(310.67162141,55.77439377)(310.6316272,55.92439941)
\curveto(310.57162151,56.16439338)(310.4816216,56.35439319)(310.3616272,56.49439941)
\curveto(310.25162183,56.63439291)(310.09162199,56.7443928)(309.8816272,56.82439941)
\curveto(309.76162232,56.87439267)(309.61662247,56.90939264)(309.4466272,56.92939941)
\curveto(309.2866228,56.9493926)(309.11662297,56.95939259)(308.9366272,56.95939941)
\curveto(308.75662333,56.95939259)(308.5816235,56.9493926)(308.4116272,56.92939941)
\curveto(308.24162384,56.90939264)(308.09662399,56.87939267)(307.9766272,56.83939941)
\curveto(307.80662428,56.77939277)(307.64162444,56.69439285)(307.4816272,56.58439941)
\curveto(307.40162468,56.52439302)(307.32662476,56.4443931)(307.2566272,56.34439941)
\curveto(307.19662489,56.25439329)(307.14162494,56.15439339)(307.0916272,56.04439941)
\curveto(307.06162502,55.96439358)(307.03162505,55.87939367)(307.0016272,55.78939941)
\curveto(306.9816251,55.69939385)(306.93662515,55.62939392)(306.8666272,55.57939941)
\curveto(306.82662526,55.549394)(306.75662533,55.52439402)(306.6566272,55.50439941)
\curveto(306.56662552,55.49439405)(306.47162561,55.48939406)(306.3716272,55.48939941)
\curveto(306.27162581,55.48939406)(306.17162591,55.49439405)(306.0716272,55.50439941)
\curveto(305.9816261,55.52439402)(305.91662617,55.549394)(305.8766272,55.57939941)
\curveto(305.83662625,55.60939394)(305.80662628,55.65939389)(305.7866272,55.72939941)
\curveto(305.76662632,55.79939375)(305.76662632,55.87439367)(305.7866272,55.95439941)
\curveto(305.81662627,56.08439346)(305.84662624,56.20439334)(305.8766272,56.31439941)
\curveto(305.91662617,56.43439311)(305.96162612,56.549393)(306.0116272,56.65939941)
\curveto(306.20162588,57.00939254)(306.44162564,57.27939227)(306.7316272,57.46939941)
\curveto(307.02162506,57.66939188)(307.3816247,57.82939172)(307.8116272,57.94939941)
\curveto(307.91162417,57.96939158)(308.01162407,57.98439156)(308.1116272,57.99439941)
\curveto(308.22162386,58.00439154)(308.33162375,58.01939153)(308.4416272,58.03939941)
\curveto(308.4816236,58.0493915)(308.54662354,58.0493915)(308.6366272,58.03939941)
\curveto(308.72662336,58.03939151)(308.7816233,58.0493915)(308.8016272,58.06939941)
\curveto(309.50162258,58.07939147)(310.11162197,57.99939155)(310.6316272,57.82939941)
\curveto(311.15162093,57.65939189)(311.51662057,57.33439221)(311.7266272,56.85439941)
\curveto(311.81662027,56.65439289)(311.86662022,56.41939313)(311.8766272,56.14939941)
\curveto(311.89662019,55.88939366)(311.90662018,55.61439393)(311.9066272,55.32439941)
\lineto(311.9066272,52.00939941)
\curveto(311.90662018,51.86939768)(311.91162017,51.73439781)(311.9216272,51.60439941)
\curveto(311.93162015,51.47439807)(311.96162012,51.36939818)(312.0116272,51.28939941)
\curveto(312.06162002,51.21939833)(312.12661996,51.16939838)(312.2066272,51.13939941)
\curveto(312.29661979,51.09939845)(312.3816197,51.06939848)(312.4616272,51.04939941)
\curveto(312.54161954,51.03939851)(312.60161948,50.99439855)(312.6416272,50.91439941)
\curveto(312.66161942,50.88439866)(312.67161941,50.85439869)(312.6716272,50.82439941)
\curveto(312.67161941,50.79439875)(312.67661941,50.75439879)(312.6866272,50.70439941)
\moveto(310.5416272,52.36939941)
\curveto(310.60162148,52.50939704)(310.63162145,52.66939688)(310.6316272,52.84939941)
\curveto(310.64162144,53.03939651)(310.64662144,53.23439631)(310.6466272,53.43439941)
\curveto(310.64662144,53.544396)(310.64162144,53.6443959)(310.6316272,53.73439941)
\curveto(310.62162146,53.82439572)(310.5816215,53.89439565)(310.5116272,53.94439941)
\curveto(310.4816216,53.96439558)(310.41162167,53.97439557)(310.3016272,53.97439941)
\curveto(310.2816218,53.95439559)(310.24662184,53.9443956)(310.1966272,53.94439941)
\curveto(310.14662194,53.9443956)(310.10162198,53.93439561)(310.0616272,53.91439941)
\curveto(309.9816221,53.89439565)(309.89162219,53.87439567)(309.7916272,53.85439941)
\lineto(309.4916272,53.79439941)
\curveto(309.46162262,53.79439575)(309.42662266,53.78939576)(309.3866272,53.77939941)
\lineto(309.2816272,53.77939941)
\curveto(309.13162295,53.73939581)(308.96662312,53.71439583)(308.7866272,53.70439941)
\curveto(308.61662347,53.70439584)(308.45662363,53.68439586)(308.3066272,53.64439941)
\curveto(308.22662386,53.62439592)(308.15162393,53.60439594)(308.0816272,53.58439941)
\curveto(308.02162406,53.57439597)(307.95162413,53.55939599)(307.8716272,53.53939941)
\curveto(307.71162437,53.48939606)(307.56162452,53.42439612)(307.4216272,53.34439941)
\curveto(307.2816248,53.27439627)(307.16162492,53.18439636)(307.0616272,53.07439941)
\curveto(306.96162512,52.96439658)(306.8866252,52.82939672)(306.8366272,52.66939941)
\curveto(306.7866253,52.51939703)(306.76662532,52.33439721)(306.7766272,52.11439941)
\curveto(306.77662531,52.01439753)(306.79162529,51.91939763)(306.8216272,51.82939941)
\curveto(306.86162522,51.7493978)(306.90662518,51.67439787)(306.9566272,51.60439941)
\curveto(307.03662505,51.49439805)(307.14162494,51.39939815)(307.2716272,51.31939941)
\curveto(307.40162468,51.2493983)(307.54162454,51.18939836)(307.6916272,51.13939941)
\curveto(307.74162434,51.12939842)(307.79162429,51.12439842)(307.8416272,51.12439941)
\curveto(307.89162419,51.12439842)(307.94162414,51.11939843)(307.9916272,51.10939941)
\curveto(308.06162402,51.08939846)(308.14662394,51.07439847)(308.2466272,51.06439941)
\curveto(308.35662373,51.06439848)(308.44662364,51.07439847)(308.5166272,51.09439941)
\curveto(308.57662351,51.11439843)(308.63662345,51.11939843)(308.6966272,51.10939941)
\curveto(308.75662333,51.10939844)(308.81662327,51.11939843)(308.8766272,51.13939941)
\curveto(308.95662313,51.15939839)(309.03162305,51.17439837)(309.1016272,51.18439941)
\curveto(309.1816229,51.19439835)(309.25662283,51.21439833)(309.3266272,51.24439941)
\curveto(309.61662247,51.36439818)(309.86162222,51.50939804)(310.0616272,51.67939941)
\curveto(310.27162181,51.8493977)(310.43162165,52.07939747)(310.5416272,52.36939941)
}
}
{
\newrgbcolor{curcolor}{0 0 0}
\pscustom[linestyle=none,fillstyle=solid,fillcolor=curcolor]
{
\newpath
\moveto(317.54826782,58.02439941)
\curveto(318.17826259,58.0443915)(318.68326208,57.95939159)(319.06326782,57.76939941)
\curveto(319.44326132,57.57939197)(319.74826102,57.29439225)(319.97826782,56.91439941)
\curveto(320.03826073,56.81439273)(320.08326068,56.70439284)(320.11326782,56.58439941)
\curveto(320.15326061,56.47439307)(320.18826058,56.35939319)(320.21826782,56.23939941)
\curveto(320.2682605,56.0493935)(320.29826047,55.8443937)(320.30826782,55.62439941)
\curveto(320.31826045,55.40439414)(320.32326044,55.17939437)(320.32326782,54.94939941)
\lineto(320.32326782,53.34439941)
\lineto(320.32326782,51.00439941)
\curveto(320.32326044,50.83439871)(320.31826045,50.66439888)(320.30826782,50.49439941)
\curveto(320.30826046,50.32439922)(320.24326052,50.21439933)(320.11326782,50.16439941)
\curveto(320.0632607,50.1443994)(320.00826076,50.13439941)(319.94826782,50.13439941)
\curveto(319.89826087,50.12439942)(319.84326092,50.11939943)(319.78326782,50.11939941)
\curveto(319.65326111,50.11939943)(319.52826124,50.12439942)(319.40826782,50.13439941)
\curveto(319.28826148,50.13439941)(319.20326156,50.17439937)(319.15326782,50.25439941)
\curveto(319.10326166,50.32439922)(319.07826169,50.41439913)(319.07826782,50.52439941)
\lineto(319.07826782,50.85439941)
\lineto(319.07826782,52.14439941)
\lineto(319.07826782,54.58939941)
\curveto(319.07826169,54.85939469)(319.07326169,55.12439442)(319.06326782,55.38439941)
\curveto(319.05326171,55.65439389)(319.00826176,55.88439366)(318.92826782,56.07439941)
\curveto(318.84826192,56.27439327)(318.72826204,56.43439311)(318.56826782,56.55439941)
\curveto(318.40826236,56.68439286)(318.22326254,56.78439276)(318.01326782,56.85439941)
\curveto(317.95326281,56.87439267)(317.88826288,56.88439266)(317.81826782,56.88439941)
\curveto(317.75826301,56.89439265)(317.69826307,56.90939264)(317.63826782,56.92939941)
\curveto(317.58826318,56.93939261)(317.50826326,56.93939261)(317.39826782,56.92939941)
\curveto(317.29826347,56.92939262)(317.22826354,56.92439262)(317.18826782,56.91439941)
\curveto(317.14826362,56.89439265)(317.11326365,56.88439266)(317.08326782,56.88439941)
\curveto(317.05326371,56.89439265)(317.01826375,56.89439265)(316.97826782,56.88439941)
\curveto(316.84826392,56.85439269)(316.72326404,56.81939273)(316.60326782,56.77939941)
\curveto(316.49326427,56.7493928)(316.38826438,56.70439284)(316.28826782,56.64439941)
\curveto(316.24826452,56.62439292)(316.21326455,56.60439294)(316.18326782,56.58439941)
\curveto(316.15326461,56.56439298)(316.11826465,56.544393)(316.07826782,56.52439941)
\curveto(315.72826504,56.27439327)(315.47326529,55.89939365)(315.31326782,55.39939941)
\curveto(315.28326548,55.31939423)(315.2632655,55.23439431)(315.25326782,55.14439941)
\curveto(315.24326552,55.06439448)(315.22826554,54.98439456)(315.20826782,54.90439941)
\curveto(315.18826558,54.85439469)(315.18326558,54.80439474)(315.19326782,54.75439941)
\curveto(315.20326556,54.71439483)(315.19826557,54.67439487)(315.17826782,54.63439941)
\lineto(315.17826782,54.31939941)
\curveto(315.1682656,54.28939526)(315.1632656,54.25439529)(315.16326782,54.21439941)
\curveto(315.17326559,54.17439537)(315.17826559,54.12939542)(315.17826782,54.07939941)
\lineto(315.17826782,53.62939941)
\lineto(315.17826782,52.18939941)
\lineto(315.17826782,50.86939941)
\lineto(315.17826782,50.52439941)
\curveto(315.17826559,50.41439913)(315.15326561,50.32439922)(315.10326782,50.25439941)
\curveto(315.05326571,50.17439937)(314.9632658,50.13439941)(314.83326782,50.13439941)
\curveto(314.71326605,50.12439942)(314.58826618,50.11939943)(314.45826782,50.11939941)
\curveto(314.37826639,50.11939943)(314.30326646,50.12439942)(314.23326782,50.13439941)
\curveto(314.1632666,50.1443994)(314.10326666,50.16939938)(314.05326782,50.20939941)
\curveto(313.97326679,50.25939929)(313.93326683,50.35439919)(313.93326782,50.49439941)
\lineto(313.93326782,50.89939941)
\lineto(313.93326782,52.66939941)
\lineto(313.93326782,56.29939941)
\lineto(313.93326782,57.21439941)
\lineto(313.93326782,57.48439941)
\curveto(313.93326683,57.57439197)(313.95326681,57.6443919)(313.99326782,57.69439941)
\curveto(314.02326674,57.75439179)(314.07326669,57.79439175)(314.14326782,57.81439941)
\curveto(314.18326658,57.82439172)(314.23826653,57.83439171)(314.30826782,57.84439941)
\curveto(314.38826638,57.85439169)(314.4682663,57.85939169)(314.54826782,57.85939941)
\curveto(314.62826614,57.85939169)(314.70326606,57.85439169)(314.77326782,57.84439941)
\curveto(314.85326591,57.83439171)(314.90826586,57.81939173)(314.93826782,57.79939941)
\curveto(315.04826572,57.72939182)(315.09826567,57.63939191)(315.08826782,57.52939941)
\curveto(315.07826569,57.42939212)(315.09326567,57.31439223)(315.13326782,57.18439941)
\curveto(315.15326561,57.12439242)(315.19326557,57.07439247)(315.25326782,57.03439941)
\curveto(315.37326539,57.02439252)(315.4682653,57.06939248)(315.53826782,57.16939941)
\curveto(315.61826515,57.26939228)(315.69826507,57.3493922)(315.77826782,57.40939941)
\curveto(315.91826485,57.50939204)(316.05826471,57.59939195)(316.19826782,57.67939941)
\curveto(316.34826442,57.76939178)(316.51826425,57.8443917)(316.70826782,57.90439941)
\curveto(316.78826398,57.93439161)(316.87326389,57.95439159)(316.96326782,57.96439941)
\curveto(317.0632637,57.97439157)(317.15826361,57.98939156)(317.24826782,58.00939941)
\curveto(317.29826347,58.01939153)(317.34826342,58.02439152)(317.39826782,58.02439941)
\lineto(317.54826782,58.02439941)
}
}
{
\newrgbcolor{curcolor}{0 0 0}
\pscustom[linestyle=none,fillstyle=solid,fillcolor=curcolor]
{
\newpath
\moveto(323.1528772,60.21439941)
\curveto(323.30287519,60.21438933)(323.45287504,60.20938934)(323.6028772,60.19939941)
\curveto(323.75287474,60.19938935)(323.85787463,60.15938939)(323.9178772,60.07939941)
\curveto(323.96787452,60.01938953)(323.9928745,59.93438961)(323.9928772,59.82439941)
\curveto(324.00287449,59.72438982)(324.00787448,59.61938993)(324.0078772,59.50939941)
\lineto(324.0078772,58.63939941)
\curveto(324.00787448,58.55939099)(324.00287449,58.47439107)(323.9928772,58.38439941)
\curveto(323.9928745,58.30439124)(324.00287449,58.23439131)(324.0228772,58.17439941)
\curveto(324.06287443,58.03439151)(324.15287434,57.9443916)(324.2928772,57.90439941)
\curveto(324.34287415,57.89439165)(324.3878741,57.88939166)(324.4278772,57.88939941)
\lineto(324.5778772,57.88939941)
\lineto(324.9828772,57.88939941)
\curveto(325.14287335,57.89939165)(325.25787323,57.88939166)(325.3278772,57.85939941)
\curveto(325.41787307,57.79939175)(325.47787301,57.73939181)(325.5078772,57.67939941)
\curveto(325.52787296,57.63939191)(325.53787295,57.59439195)(325.5378772,57.54439941)
\lineto(325.5378772,57.39439941)
\curveto(325.53787295,57.28439226)(325.53287296,57.17939237)(325.5228772,57.07939941)
\curveto(325.51287298,56.98939256)(325.47787301,56.91939263)(325.4178772,56.86939941)
\curveto(325.35787313,56.81939273)(325.27287322,56.78939276)(325.1628772,56.77939941)
\lineto(324.8328772,56.77939941)
\curveto(324.72287377,56.78939276)(324.61287388,56.79439275)(324.5028772,56.79439941)
\curveto(324.3928741,56.79439275)(324.29787419,56.77939277)(324.2178772,56.74939941)
\curveto(324.14787434,56.71939283)(324.09787439,56.66939288)(324.0678772,56.59939941)
\curveto(324.03787445,56.52939302)(324.01787447,56.4443931)(324.0078772,56.34439941)
\curveto(323.99787449,56.25439329)(323.9928745,56.15439339)(323.9928772,56.04439941)
\curveto(324.00287449,55.9443936)(324.00787448,55.8443937)(324.0078772,55.74439941)
\lineto(324.0078772,52.77439941)
\curveto(324.00787448,52.55439699)(324.00287449,52.31939723)(323.9928772,52.06939941)
\curveto(323.9928745,51.82939772)(324.03787445,51.6443979)(324.1278772,51.51439941)
\curveto(324.17787431,51.43439811)(324.24287425,51.37939817)(324.3228772,51.34939941)
\curveto(324.40287409,51.31939823)(324.49787399,51.29439825)(324.6078772,51.27439941)
\curveto(324.63787385,51.26439828)(324.66787382,51.25939829)(324.6978772,51.25939941)
\curveto(324.73787375,51.26939828)(324.77287372,51.26939828)(324.8028772,51.25939941)
\lineto(324.9978772,51.25939941)
\curveto(325.09787339,51.25939829)(325.1878733,51.2493983)(325.2678772,51.22939941)
\curveto(325.35787313,51.21939833)(325.42287307,51.18439836)(325.4628772,51.12439941)
\curveto(325.48287301,51.09439845)(325.49787299,51.03939851)(325.5078772,50.95939941)
\curveto(325.52787296,50.88939866)(325.53787295,50.81439873)(325.5378772,50.73439941)
\curveto(325.54787294,50.65439889)(325.54787294,50.57439897)(325.5378772,50.49439941)
\curveto(325.52787296,50.42439912)(325.50787298,50.36939918)(325.4778772,50.32939941)
\curveto(325.43787305,50.25939929)(325.36287313,50.20939934)(325.2528772,50.17939941)
\curveto(325.17287332,50.15939939)(325.08287341,50.1493994)(324.9828772,50.14939941)
\curveto(324.88287361,50.15939939)(324.7928737,50.16439938)(324.7128772,50.16439941)
\curveto(324.65287384,50.16439938)(324.5928739,50.15939939)(324.5328772,50.14939941)
\curveto(324.47287402,50.1493994)(324.41787407,50.15439939)(324.3678772,50.16439941)
\lineto(324.1878772,50.16439941)
\curveto(324.13787435,50.17439937)(324.0878744,50.17939937)(324.0378772,50.17939941)
\curveto(323.99787449,50.18939936)(323.95287454,50.19439935)(323.9028772,50.19439941)
\curveto(323.70287479,50.2443993)(323.52787496,50.29939925)(323.3778772,50.35939941)
\curveto(323.23787525,50.41939913)(323.11787537,50.52439902)(323.0178772,50.67439941)
\curveto(322.87787561,50.87439867)(322.79787569,51.12439842)(322.7778772,51.42439941)
\curveto(322.75787573,51.73439781)(322.74787574,52.06439748)(322.7478772,52.41439941)
\lineto(322.7478772,56.34439941)
\curveto(322.71787577,56.47439307)(322.6878758,56.56939298)(322.6578772,56.62939941)
\curveto(322.63787585,56.68939286)(322.56787592,56.73939281)(322.4478772,56.77939941)
\curveto(322.40787608,56.78939276)(322.36787612,56.78939276)(322.3278772,56.77939941)
\curveto(322.2878762,56.76939278)(322.24787624,56.77439277)(322.2078772,56.79439941)
\lineto(321.9678772,56.79439941)
\curveto(321.83787665,56.79439275)(321.72787676,56.80439274)(321.6378772,56.82439941)
\curveto(321.55787693,56.85439269)(321.50287699,56.91439263)(321.4728772,57.00439941)
\curveto(321.45287704,57.0443925)(321.43787705,57.08939246)(321.4278772,57.13939941)
\lineto(321.4278772,57.28939941)
\curveto(321.42787706,57.42939212)(321.43787705,57.544392)(321.4578772,57.63439941)
\curveto(321.47787701,57.73439181)(321.53787695,57.80939174)(321.6378772,57.85939941)
\curveto(321.74787674,57.89939165)(321.8878766,57.90939164)(322.0578772,57.88939941)
\curveto(322.23787625,57.86939168)(322.3878761,57.87939167)(322.5078772,57.91939941)
\curveto(322.59787589,57.96939158)(322.66787582,58.03939151)(322.7178772,58.12939941)
\curveto(322.73787575,58.18939136)(322.74787574,58.26439128)(322.7478772,58.35439941)
\lineto(322.7478772,58.60939941)
\lineto(322.7478772,59.53939941)
\lineto(322.7478772,59.77939941)
\curveto(322.74787574,59.86938968)(322.75787573,59.9443896)(322.7778772,60.00439941)
\curveto(322.81787567,60.08438946)(322.8928756,60.1493894)(323.0028772,60.19939941)
\curveto(323.03287546,60.19938935)(323.05787543,60.19938935)(323.0778772,60.19939941)
\curveto(323.10787538,60.20938934)(323.13287536,60.21438933)(323.1528772,60.21439941)
}
}
{
\newrgbcolor{curcolor}{0 0 0}
\pscustom[linestyle=none,fillstyle=solid,fillcolor=curcolor]
{
\newpath
\moveto(333.67467407,54.31939941)
\curveto(333.69466639,54.21939533)(333.69466639,54.10439544)(333.67467407,53.97439941)
\curveto(333.66466642,53.85439569)(333.63466645,53.76939578)(333.58467407,53.71939941)
\curveto(333.53466655,53.67939587)(333.45966662,53.6493959)(333.35967407,53.62939941)
\curveto(333.26966681,53.61939593)(333.16466692,53.61439593)(333.04467407,53.61439941)
\lineto(332.68467407,53.61439941)
\curveto(332.56466752,53.62439592)(332.45966762,53.62939592)(332.36967407,53.62939941)
\lineto(328.52967407,53.62939941)
\curveto(328.44967163,53.62939592)(328.36967171,53.62439592)(328.28967407,53.61439941)
\curveto(328.20967187,53.61439593)(328.14467194,53.59939595)(328.09467407,53.56939941)
\curveto(328.05467203,53.549396)(328.01467207,53.50939604)(327.97467407,53.44939941)
\curveto(327.95467213,53.41939613)(327.93467215,53.37439617)(327.91467407,53.31439941)
\curveto(327.89467219,53.26439628)(327.89467219,53.21439633)(327.91467407,53.16439941)
\curveto(327.92467216,53.11439643)(327.92967215,53.06939648)(327.92967407,53.02939941)
\curveto(327.92967215,52.98939656)(327.93467215,52.9493966)(327.94467407,52.90939941)
\curveto(327.96467212,52.82939672)(327.9846721,52.7443968)(328.00467407,52.65439941)
\curveto(328.02467206,52.57439697)(328.05467203,52.49439705)(328.09467407,52.41439941)
\curveto(328.32467176,51.87439767)(328.70467138,51.48939806)(329.23467407,51.25939941)
\curveto(329.29467079,51.22939832)(329.35967072,51.20439834)(329.42967407,51.18439941)
\lineto(329.63967407,51.12439941)
\curveto(329.66967041,51.11439843)(329.71967036,51.10939844)(329.78967407,51.10939941)
\curveto(329.92967015,51.06939848)(330.11466997,51.0493985)(330.34467407,51.04939941)
\curveto(330.57466951,51.0493985)(330.75966932,51.06939848)(330.89967407,51.10939941)
\curveto(331.03966904,51.1493984)(331.16466892,51.18939836)(331.27467407,51.22939941)
\curveto(331.39466869,51.27939827)(331.50466858,51.33939821)(331.60467407,51.40939941)
\curveto(331.71466837,51.47939807)(331.80966827,51.55939799)(331.88967407,51.64939941)
\curveto(331.96966811,51.7493978)(332.03966804,51.85439769)(332.09967407,51.96439941)
\curveto(332.15966792,52.06439748)(332.20966787,52.16939738)(332.24967407,52.27939941)
\curveto(332.29966778,52.38939716)(332.3796677,52.46939708)(332.48967407,52.51939941)
\curveto(332.52966755,52.53939701)(332.59466749,52.55439699)(332.68467407,52.56439941)
\curveto(332.77466731,52.57439697)(332.86466722,52.57439697)(332.95467407,52.56439941)
\curveto(333.04466704,52.56439698)(333.12966695,52.55939699)(333.20967407,52.54939941)
\curveto(333.28966679,52.53939701)(333.34466674,52.51939703)(333.37467407,52.48939941)
\curveto(333.47466661,52.41939713)(333.49966658,52.30439724)(333.44967407,52.14439941)
\curveto(333.36966671,51.87439767)(333.26466682,51.63439791)(333.13467407,51.42439941)
\curveto(332.93466715,51.10439844)(332.70466738,50.83939871)(332.44467407,50.62939941)
\curveto(332.19466789,50.42939912)(331.87466821,50.26439928)(331.48467407,50.13439941)
\curveto(331.3846687,50.09439945)(331.2846688,50.06939948)(331.18467407,50.05939941)
\curveto(331.084669,50.03939951)(330.9796691,50.01939953)(330.86967407,49.99939941)
\curveto(330.81966926,49.98939956)(330.76966931,49.98439956)(330.71967407,49.98439941)
\curveto(330.6796694,49.98439956)(330.63466945,49.97939957)(330.58467407,49.96939941)
\lineto(330.43467407,49.96939941)
\curveto(330.3846697,49.95939959)(330.32466976,49.95439959)(330.25467407,49.95439941)
\curveto(330.19466989,49.95439959)(330.14466994,49.95939959)(330.10467407,49.96939941)
\lineto(329.96967407,49.96939941)
\curveto(329.91967016,49.97939957)(329.87467021,49.98439956)(329.83467407,49.98439941)
\curveto(329.79467029,49.98439956)(329.75467033,49.98939956)(329.71467407,49.99939941)
\curveto(329.66467042,50.00939954)(329.60967047,50.01939953)(329.54967407,50.02939941)
\curveto(329.48967059,50.02939952)(329.43467065,50.03439951)(329.38467407,50.04439941)
\curveto(329.29467079,50.06439948)(329.20467088,50.08939946)(329.11467407,50.11939941)
\curveto(329.02467106,50.13939941)(328.93967114,50.16439938)(328.85967407,50.19439941)
\curveto(328.81967126,50.21439933)(328.7846713,50.22439932)(328.75467407,50.22439941)
\curveto(328.72467136,50.23439931)(328.68967139,50.2493993)(328.64967407,50.26939941)
\curveto(328.49967158,50.33939921)(328.33967174,50.42439912)(328.16967407,50.52439941)
\curveto(327.8796722,50.71439883)(327.62967245,50.9443986)(327.41967407,51.21439941)
\curveto(327.21967286,51.49439805)(327.04967303,51.80439774)(326.90967407,52.14439941)
\curveto(326.85967322,52.25439729)(326.81967326,52.36939718)(326.78967407,52.48939941)
\curveto(326.76967331,52.60939694)(326.73967334,52.72939682)(326.69967407,52.84939941)
\curveto(326.68967339,52.88939666)(326.6846734,52.92439662)(326.68467407,52.95439941)
\curveto(326.6846734,52.98439656)(326.6796734,53.02439652)(326.66967407,53.07439941)
\curveto(326.64967343,53.15439639)(326.63467345,53.23939631)(326.62467407,53.32939941)
\curveto(326.61467347,53.41939613)(326.59967348,53.50939604)(326.57967407,53.59939941)
\lineto(326.57967407,53.80939941)
\curveto(326.56967351,53.8493957)(326.55967352,53.90439564)(326.54967407,53.97439941)
\curveto(326.54967353,54.05439549)(326.55467353,54.11939543)(326.56467407,54.16939941)
\lineto(326.56467407,54.33439941)
\curveto(326.5846735,54.38439516)(326.58967349,54.43439511)(326.57967407,54.48439941)
\curveto(326.5796735,54.544395)(326.5846735,54.59939495)(326.59467407,54.64939941)
\curveto(326.63467345,54.80939474)(326.66467342,54.96939458)(326.68467407,55.12939941)
\curveto(326.71467337,55.28939426)(326.75967332,55.43939411)(326.81967407,55.57939941)
\curveto(326.86967321,55.68939386)(326.91467317,55.79939375)(326.95467407,55.90939941)
\curveto(327.00467308,56.02939352)(327.05967302,56.1443934)(327.11967407,56.25439941)
\curveto(327.33967274,56.60439294)(327.58967249,56.90439264)(327.86967407,57.15439941)
\curveto(328.14967193,57.41439213)(328.49467159,57.62939192)(328.90467407,57.79939941)
\curveto(329.02467106,57.8493917)(329.14467094,57.88439166)(329.26467407,57.90439941)
\curveto(329.39467069,57.93439161)(329.52967055,57.96439158)(329.66967407,57.99439941)
\curveto(329.71967036,58.00439154)(329.76467032,58.00939154)(329.80467407,58.00939941)
\curveto(329.84467024,58.01939153)(329.88967019,58.02439152)(329.93967407,58.02439941)
\curveto(329.95967012,58.03439151)(329.9846701,58.03439151)(330.01467407,58.02439941)
\curveto(330.04467004,58.01439153)(330.06967001,58.01939153)(330.08967407,58.03939941)
\curveto(330.50966957,58.0493915)(330.87466921,58.00439154)(331.18467407,57.90439941)
\curveto(331.49466859,57.81439173)(331.77466831,57.68939186)(332.02467407,57.52939941)
\curveto(332.07466801,57.50939204)(332.11466797,57.47939207)(332.14467407,57.43939941)
\curveto(332.17466791,57.40939214)(332.20966787,57.38439216)(332.24967407,57.36439941)
\curveto(332.32966775,57.30439224)(332.40966767,57.23439231)(332.48967407,57.15439941)
\curveto(332.5796675,57.07439247)(332.65466743,56.99439255)(332.71467407,56.91439941)
\curveto(332.87466721,56.70439284)(333.00966707,56.50439304)(333.11967407,56.31439941)
\curveto(333.18966689,56.20439334)(333.24466684,56.08439346)(333.28467407,55.95439941)
\curveto(333.32466676,55.82439372)(333.36966671,55.69439385)(333.41967407,55.56439941)
\curveto(333.46966661,55.43439411)(333.50466658,55.29939425)(333.52467407,55.15939941)
\curveto(333.55466653,55.01939453)(333.58966649,54.87939467)(333.62967407,54.73939941)
\curveto(333.63966644,54.66939488)(333.64466644,54.59939495)(333.64467407,54.52939941)
\lineto(333.67467407,54.31939941)
\moveto(332.21967407,54.82939941)
\curveto(332.24966783,54.86939468)(332.27466781,54.91939463)(332.29467407,54.97939941)
\curveto(332.31466777,55.0493945)(332.31466777,55.11939443)(332.29467407,55.18939941)
\curveto(332.23466785,55.40939414)(332.14966793,55.61439393)(332.03967407,55.80439941)
\curveto(331.89966818,56.03439351)(331.74466834,56.22939332)(331.57467407,56.38939941)
\curveto(331.40466868,56.549393)(331.1846689,56.68439286)(330.91467407,56.79439941)
\curveto(330.84466924,56.81439273)(330.77466931,56.82939272)(330.70467407,56.83939941)
\curveto(330.63466945,56.85939269)(330.55966952,56.87939267)(330.47967407,56.89939941)
\curveto(330.39966968,56.91939263)(330.31466977,56.92939262)(330.22467407,56.92939941)
\lineto(329.96967407,56.92939941)
\curveto(329.93967014,56.90939264)(329.90467018,56.89939265)(329.86467407,56.89939941)
\curveto(329.82467026,56.90939264)(329.78967029,56.90939264)(329.75967407,56.89939941)
\lineto(329.51967407,56.83939941)
\curveto(329.44967063,56.82939272)(329.3796707,56.81439273)(329.30967407,56.79439941)
\curveto(329.01967106,56.67439287)(328.7846713,56.52439302)(328.60467407,56.34439941)
\curveto(328.43467165,56.16439338)(328.2796718,55.93939361)(328.13967407,55.66939941)
\curveto(328.10967197,55.61939393)(328.079672,55.55439399)(328.04967407,55.47439941)
\curveto(328.01967206,55.40439414)(327.99467209,55.32439422)(327.97467407,55.23439941)
\curveto(327.95467213,55.1443944)(327.94967213,55.05939449)(327.95967407,54.97939941)
\curveto(327.96967211,54.89939465)(328.00467208,54.83939471)(328.06467407,54.79939941)
\curveto(328.14467194,54.73939481)(328.2796718,54.70939484)(328.46967407,54.70939941)
\curveto(328.66967141,54.71939483)(328.83967124,54.72439482)(328.97967407,54.72439941)
\lineto(331.25967407,54.72439941)
\curveto(331.40966867,54.72439482)(331.58966849,54.71939483)(331.79967407,54.70939941)
\curveto(332.00966807,54.70939484)(332.14966793,54.7493948)(332.21967407,54.82939941)
}
}
{
\newrgbcolor{curcolor}{0 0 0}
\pscustom[linestyle=none,fillstyle=solid,fillcolor=curcolor]
{
\newpath
\moveto(461.67456909,55.72938477)
\lineto(461.67456909,55.45938477)
\curveto(461.68455912,55.36937952)(461.67955913,55.2893796)(461.65956909,55.21938477)
\lineto(461.65956909,55.06938477)
\curveto(461.64955916,55.03937985)(461.64455916,55.00437988)(461.64456909,54.96438477)
\curveto(461.65455915,54.92437996)(461.65455915,54.89437999)(461.64456909,54.87438477)
\curveto(461.63455917,54.82438006)(461.62955918,54.76938012)(461.62956909,54.70938477)
\curveto(461.62955918,54.65938023)(461.62455918,54.60938028)(461.61456909,54.55938477)
\curveto(461.58455922,54.41938047)(461.56455924,54.26938062)(461.55456909,54.10938477)
\curveto(461.54455926,53.95938093)(461.51455929,53.81438107)(461.46456909,53.67438477)
\curveto(461.43455937,53.55438133)(461.39955941,53.42938146)(461.35956909,53.29938477)
\curveto(461.32955948,53.17938171)(461.28955952,53.05938183)(461.23956909,52.93938477)
\curveto(461.06955974,52.50938238)(460.85455995,52.11938277)(460.59456909,51.76938477)
\curveto(460.34456046,51.42938346)(460.02956078,51.13938375)(459.64956909,50.89938477)
\curveto(459.45956135,50.77938411)(459.25456155,50.67438421)(459.03456909,50.58438477)
\curveto(458.82456198,50.50438438)(458.59456221,50.42438446)(458.34456909,50.34438477)
\curveto(458.23456257,50.30438458)(458.11456269,50.27438461)(457.98456909,50.25438477)
\curveto(457.86456294,50.24438464)(457.74456306,50.22438466)(457.62456909,50.19438477)
\curveto(457.51456329,50.17438471)(457.4045634,50.16438472)(457.29456909,50.16438477)
\curveto(457.19456361,50.16438472)(457.09456371,50.15438473)(456.99456909,50.13438477)
\lineto(456.78456909,50.13438477)
\curveto(456.75456405,50.12438476)(456.71956409,50.11938477)(456.67956909,50.11938477)
\curveto(456.63956417,50.12938476)(456.59956421,50.13438475)(456.55956909,50.13438477)
\lineto(453.55956909,50.13438477)
\curveto(453.4095674,50.13438475)(453.27456753,50.13938475)(453.15456909,50.14938477)
\curveto(453.04456776,50.16938472)(452.96956784,50.23438465)(452.92956909,50.34438477)
\curveto(452.88956792,50.42438446)(452.86956794,50.53938435)(452.86956909,50.68938477)
\curveto(452.87956793,50.83938405)(452.88456792,50.97438391)(452.88456909,51.09438477)
\lineto(452.88456909,59.95938477)
\curveto(452.88456792,60.07937481)(452.87956793,60.20437468)(452.86956909,60.33438477)
\curveto(452.86956794,60.47437441)(452.89456791,60.5843743)(452.94456909,60.66438477)
\curveto(452.98456782,60.73437415)(453.05956775,60.77937411)(453.16956909,60.79938477)
\curveto(453.18956762,60.80937408)(453.2095676,60.80937408)(453.22956909,60.79938477)
\curveto(453.24956756,60.79937409)(453.26956754,60.80437408)(453.28956909,60.81438477)
\lineto(456.54456909,60.81438477)
\curveto(456.59456421,60.81437407)(456.63956417,60.81437407)(456.67956909,60.81438477)
\curveto(456.72956408,60.82437406)(456.77456403,60.82437406)(456.81456909,60.81438477)
\curveto(456.86456394,60.79437409)(456.91456389,60.7893741)(456.96456909,60.79938477)
\curveto(457.02456378,60.80937408)(457.07956373,60.80937408)(457.12956909,60.79938477)
\curveto(457.17956363,60.7893741)(457.23456357,60.7843741)(457.29456909,60.78438477)
\curveto(457.35456345,60.7843741)(457.4095634,60.77937411)(457.45956909,60.76938477)
\curveto(457.5095633,60.75937413)(457.55456325,60.75437413)(457.59456909,60.75438477)
\curveto(457.64456316,60.75437413)(457.69456311,60.74937414)(457.74456909,60.73938477)
\curveto(457.85456295,60.71937417)(457.95956285,60.69937419)(458.05956909,60.67938477)
\curveto(458.15956265,60.66937422)(458.25956255,60.64937424)(458.35956909,60.61938477)
\curveto(458.57956223,60.54937434)(458.78956202,60.47937441)(458.98956909,60.40938477)
\curveto(459.18956162,60.34937454)(459.37456143,60.26437462)(459.54456909,60.15438477)
\curveto(459.68456112,60.07437481)(459.809561,59.99437489)(459.91956909,59.91438477)
\curveto(459.94956086,59.89437499)(459.97956083,59.86937502)(460.00956909,59.83938477)
\curveto(460.03956077,59.81937507)(460.06956074,59.79937509)(460.09956909,59.77938477)
\curveto(460.15956065,59.72937516)(460.21456059,59.67937521)(460.26456909,59.62938477)
\curveto(460.31456049,59.57937531)(460.36456044,59.52937536)(460.41456909,59.47938477)
\curveto(460.46456034,59.42937546)(460.5045603,59.39437549)(460.53456909,59.37438477)
\curveto(460.57456023,59.31437557)(460.61456019,59.25937563)(460.65456909,59.20938477)
\curveto(460.7045601,59.15937573)(460.74956006,59.10437578)(460.78956909,59.04438477)
\curveto(460.83955997,58.9843759)(460.87955993,58.91937597)(460.90956909,58.84938477)
\curveto(460.94955986,58.7893761)(460.99455981,58.72437616)(461.04456909,58.65438477)
\curveto(461.06455974,58.61437627)(461.07955973,58.57937631)(461.08956909,58.54938477)
\curveto(461.09955971,58.51937637)(461.11455969,58.4843764)(461.13456909,58.44438477)
\curveto(461.17455963,58.36437652)(461.2095596,58.2843766)(461.23956909,58.20438477)
\curveto(461.26955954,58.13437675)(461.3045595,58.05937683)(461.34456909,57.97938477)
\curveto(461.38455942,57.86937702)(461.41455939,57.75437713)(461.43456909,57.63438477)
\curveto(461.46455934,57.52437736)(461.49455931,57.41437747)(461.52456909,57.30438477)
\curveto(461.54455926,57.24437764)(461.55455925,57.1843777)(461.55456909,57.12438477)
\curveto(461.55455925,57.07437781)(461.56455924,57.01937787)(461.58456909,56.95938477)
\curveto(461.63455917,56.77937811)(461.65955915,56.57937831)(461.65956909,56.35938477)
\curveto(461.66955914,56.14937874)(461.67455913,55.93937895)(461.67456909,55.72938477)
\moveto(460.24956909,54.94938477)
\curveto(460.26956054,55.04937984)(460.27956053,55.15437973)(460.27956909,55.26438477)
\lineto(460.27956909,55.60938477)
\lineto(460.27956909,55.83438477)
\curveto(460.28956052,55.91437897)(460.28456052,55.9893789)(460.26456909,56.05938477)
\curveto(460.26456054,56.0893788)(460.25956055,56.11937877)(460.24956909,56.14938477)
\lineto(460.24956909,56.25438477)
\curveto(460.22956058,56.36437852)(460.21456059,56.47437841)(460.20456909,56.58438477)
\curveto(460.2045606,56.69437819)(460.18956062,56.80437808)(460.15956909,56.91438477)
\curveto(460.13956067,56.99437789)(460.11956069,57.06937782)(460.09956909,57.13938477)
\curveto(460.08956072,57.21937767)(460.07456073,57.29937759)(460.05456909,57.37938477)
\curveto(459.94456086,57.73937715)(459.804561,58.05437683)(459.63456909,58.32438477)
\curveto(459.35456145,58.77437611)(458.93956187,59.11437577)(458.38956909,59.34438477)
\curveto(458.29956251,59.39437549)(458.2045626,59.42937546)(458.10456909,59.44938477)
\curveto(458.0045628,59.47937541)(457.89956291,59.50937538)(457.78956909,59.53938477)
\curveto(457.67956313,59.56937532)(457.56456324,59.5843753)(457.44456909,59.58438477)
\curveto(457.33456347,59.59437529)(457.22456358,59.60937528)(457.11456909,59.62938477)
\lineto(456.79956909,59.62938477)
\curveto(456.76956404,59.63937525)(456.73456407,59.64437524)(456.69456909,59.64438477)
\lineto(456.57456909,59.64438477)
\lineto(454.74456909,59.64438477)
\curveto(454.72456608,59.63437525)(454.69956611,59.62937526)(454.66956909,59.62938477)
\curveto(454.63956617,59.63937525)(454.61456619,59.63937525)(454.59456909,59.62938477)
\lineto(454.44456909,59.56938477)
\curveto(454.4045664,59.54937534)(454.37456643,59.51937537)(454.35456909,59.47938477)
\curveto(454.33456647,59.43937545)(454.31456649,59.36937552)(454.29456909,59.26938477)
\lineto(454.29456909,59.14938477)
\curveto(454.28456652,59.10937578)(454.27956653,59.06437582)(454.27956909,59.01438477)
\lineto(454.27956909,58.87938477)
\lineto(454.27956909,52.06938477)
\lineto(454.27956909,51.91938477)
\curveto(454.27956653,51.87938301)(454.28456652,51.83938305)(454.29456909,51.79938477)
\lineto(454.29456909,51.67938477)
\curveto(454.31456649,51.57938331)(454.33456647,51.50938338)(454.35456909,51.46938477)
\curveto(454.43456637,51.34938354)(454.58456622,51.2893836)(454.80456909,51.28938477)
\curveto(455.02456578,51.29938359)(455.23456557,51.30438358)(455.43456909,51.30438477)
\lineto(456.30456909,51.30438477)
\curveto(456.37456443,51.30438358)(456.44956436,51.29938359)(456.52956909,51.28938477)
\curveto(456.6095642,51.2893836)(456.67956413,51.29938359)(456.73956909,51.31938477)
\lineto(456.90456909,51.31938477)
\curveto(456.95456385,51.32938356)(457.0095638,51.32938356)(457.06956909,51.31938477)
\curveto(457.12956368,51.31938357)(457.18956362,51.32438356)(457.24956909,51.33438477)
\curveto(457.3095635,51.35438353)(457.36956344,51.36438352)(457.42956909,51.36438477)
\curveto(457.48956332,51.37438351)(457.55456325,51.3893835)(457.62456909,51.40938477)
\curveto(457.73456307,51.43938345)(457.83956297,51.46938342)(457.93956909,51.49938477)
\curveto(458.04956276,51.52938336)(458.15956265,51.56938332)(458.26956909,51.61938477)
\curveto(458.63956217,51.77938311)(458.95456185,51.99438289)(459.21456909,52.26438477)
\curveto(459.48456132,52.54438234)(459.7045611,52.87438201)(459.87456909,53.25438477)
\curveto(459.92456088,53.36438152)(459.96456084,53.47938141)(459.99456909,53.59938477)
\lineto(460.11456909,53.98938477)
\curveto(460.14456066,54.09938079)(460.16456064,54.21438067)(460.17456909,54.33438477)
\curveto(460.19456061,54.46438042)(460.21456059,54.5893803)(460.23456909,54.70938477)
\curveto(460.24456056,54.75938013)(460.24956056,54.79938009)(460.24956909,54.82938477)
\lineto(460.24956909,54.94938477)
}
}
{
\newrgbcolor{curcolor}{0 0 0}
\pscustom[linestyle=none,fillstyle=solid,fillcolor=curcolor]
{
\newpath
\moveto(470.30144409,54.33438477)
\curveto(470.32143603,54.27438061)(470.33143602,54.17938071)(470.33144409,54.04938477)
\curveto(470.33143602,53.92938096)(470.32643603,53.84438104)(470.31644409,53.79438477)
\lineto(470.31644409,53.64438477)
\curveto(470.30643605,53.56438132)(470.29643606,53.4893814)(470.28644409,53.41938477)
\curveto(470.28643607,53.35938153)(470.28143607,53.2893816)(470.27144409,53.20938477)
\curveto(470.2514361,53.14938174)(470.23643612,53.0893818)(470.22644409,53.02938477)
\curveto(470.22643613,52.96938192)(470.21643614,52.90938198)(470.19644409,52.84938477)
\curveto(470.1564362,52.71938217)(470.12143623,52.5893823)(470.09144409,52.45938477)
\curveto(470.06143629,52.32938256)(470.02143633,52.20938268)(469.97144409,52.09938477)
\curveto(469.76143659,51.61938327)(469.48143687,51.21438367)(469.13144409,50.88438477)
\curveto(468.78143757,50.56438432)(468.351438,50.31938457)(467.84144409,50.14938477)
\curveto(467.73143862,50.10938478)(467.61143874,50.07938481)(467.48144409,50.05938477)
\curveto(467.36143899,50.03938485)(467.23643912,50.01938487)(467.10644409,49.99938477)
\curveto(467.04643931,49.9893849)(466.98143937,49.9843849)(466.91144409,49.98438477)
\curveto(466.8514395,49.97438491)(466.79143956,49.96938492)(466.73144409,49.96938477)
\curveto(466.69143966,49.95938493)(466.63143972,49.95438493)(466.55144409,49.95438477)
\curveto(466.48143987,49.95438493)(466.43143992,49.95938493)(466.40144409,49.96938477)
\curveto(466.36143999,49.97938491)(466.32144003,49.9843849)(466.28144409,49.98438477)
\curveto(466.24144011,49.97438491)(466.20644015,49.97438491)(466.17644409,49.98438477)
\lineto(466.08644409,49.98438477)
\lineto(465.72644409,50.02938477)
\curveto(465.58644077,50.06938482)(465.4514409,50.10938478)(465.32144409,50.14938477)
\curveto(465.19144116,50.1893847)(465.06644129,50.23438465)(464.94644409,50.28438477)
\curveto(464.49644186,50.4843844)(464.12644223,50.74438414)(463.83644409,51.06438477)
\curveto(463.54644281,51.3843835)(463.30644305,51.77438311)(463.11644409,52.23438477)
\curveto(463.06644329,52.33438255)(463.02644333,52.43438245)(462.99644409,52.53438477)
\curveto(462.97644338,52.63438225)(462.9564434,52.73938215)(462.93644409,52.84938477)
\curveto(462.91644344,52.889382)(462.90644345,52.91938197)(462.90644409,52.93938477)
\curveto(462.91644344,52.96938192)(462.91644344,53.00438188)(462.90644409,53.04438477)
\curveto(462.88644347,53.12438176)(462.87144348,53.20438168)(462.86144409,53.28438477)
\curveto(462.86144349,53.37438151)(462.8514435,53.45938143)(462.83144409,53.53938477)
\lineto(462.83144409,53.65938477)
\curveto(462.83144352,53.69938119)(462.82644353,53.74438114)(462.81644409,53.79438477)
\curveto(462.80644355,53.84438104)(462.80144355,53.92938096)(462.80144409,54.04938477)
\curveto(462.80144355,54.17938071)(462.81144354,54.27438061)(462.83144409,54.33438477)
\curveto(462.8514435,54.40438048)(462.8564435,54.47438041)(462.84644409,54.54438477)
\curveto(462.83644352,54.61438027)(462.84144351,54.6843802)(462.86144409,54.75438477)
\curveto(462.87144348,54.80438008)(462.87644348,54.84438004)(462.87644409,54.87438477)
\curveto(462.88644347,54.91437997)(462.89644346,54.95937993)(462.90644409,55.00938477)
\curveto(462.93644342,55.12937976)(462.96144339,55.24937964)(462.98144409,55.36938477)
\curveto(463.01144334,55.4893794)(463.0514433,55.60437928)(463.10144409,55.71438477)
\curveto(463.2514431,56.0843788)(463.43144292,56.41437847)(463.64144409,56.70438477)
\curveto(463.86144249,57.00437788)(464.12644223,57.25437763)(464.43644409,57.45438477)
\curveto(464.5564418,57.53437735)(464.68144167,57.59937729)(464.81144409,57.64938477)
\curveto(464.94144141,57.70937718)(465.07644128,57.76937712)(465.21644409,57.82938477)
\curveto(465.33644102,57.87937701)(465.46644089,57.90937698)(465.60644409,57.91938477)
\curveto(465.74644061,57.93937695)(465.88644047,57.96937692)(466.02644409,58.00938477)
\lineto(466.22144409,58.00938477)
\curveto(466.29144006,58.01937687)(466.35644,58.02937686)(466.41644409,58.03938477)
\curveto(467.30643905,58.04937684)(468.04643831,57.86437702)(468.63644409,57.48438477)
\curveto(469.22643713,57.10437778)(469.6514367,56.60937828)(469.91144409,55.99938477)
\curveto(469.96143639,55.89937899)(470.00143635,55.79937909)(470.03144409,55.69938477)
\curveto(470.06143629,55.59937929)(470.09643626,55.49437939)(470.13644409,55.38438477)
\curveto(470.16643619,55.27437961)(470.19143616,55.15437973)(470.21144409,55.02438477)
\curveto(470.23143612,54.90437998)(470.2564361,54.77938011)(470.28644409,54.64938477)
\curveto(470.29643606,54.59938029)(470.29643606,54.54438034)(470.28644409,54.48438477)
\curveto(470.28643607,54.43438045)(470.29143606,54.3843805)(470.30144409,54.33438477)
\moveto(468.96644409,53.47938477)
\curveto(468.98643737,53.54938134)(468.99143736,53.62938126)(468.98144409,53.71938477)
\lineto(468.98144409,53.97438477)
\curveto(468.98143737,54.36438052)(468.94643741,54.69438019)(468.87644409,54.96438477)
\curveto(468.84643751,55.04437984)(468.82143753,55.12437976)(468.80144409,55.20438477)
\curveto(468.78143757,55.2843796)(468.7564376,55.35937953)(468.72644409,55.42938477)
\curveto(468.44643791,56.07937881)(468.00143835,56.52937836)(467.39144409,56.77938477)
\curveto(467.32143903,56.80937808)(467.24643911,56.82937806)(467.16644409,56.83938477)
\lineto(466.92644409,56.89938477)
\curveto(466.84643951,56.91937797)(466.76143959,56.92937796)(466.67144409,56.92938477)
\lineto(466.40144409,56.92938477)
\lineto(466.13144409,56.88438477)
\curveto(466.03144032,56.86437802)(465.93644042,56.83937805)(465.84644409,56.80938477)
\curveto(465.76644059,56.7893781)(465.68644067,56.75937813)(465.60644409,56.71938477)
\curveto(465.53644082,56.69937819)(465.47144088,56.66937822)(465.41144409,56.62938477)
\curveto(465.351441,56.5893783)(465.29644106,56.54937834)(465.24644409,56.50938477)
\curveto(465.00644135,56.33937855)(464.81144154,56.13437875)(464.66144409,55.89438477)
\curveto(464.51144184,55.65437923)(464.38144197,55.37437951)(464.27144409,55.05438477)
\curveto(464.24144211,54.95437993)(464.22144213,54.84938004)(464.21144409,54.73938477)
\curveto(464.20144215,54.63938025)(464.18644217,54.53438035)(464.16644409,54.42438477)
\curveto(464.1564422,54.3843805)(464.1514422,54.31938057)(464.15144409,54.22938477)
\curveto(464.14144221,54.19938069)(464.13644222,54.16438072)(464.13644409,54.12438477)
\curveto(464.14644221,54.0843808)(464.1514422,54.03938085)(464.15144409,53.98938477)
\lineto(464.15144409,53.68938477)
\curveto(464.1514422,53.5893813)(464.16144219,53.49938139)(464.18144409,53.41938477)
\lineto(464.21144409,53.23938477)
\curveto(464.23144212,53.13938175)(464.24644211,53.03938185)(464.25644409,52.93938477)
\curveto(464.27644208,52.84938204)(464.30644205,52.76438212)(464.34644409,52.68438477)
\curveto(464.44644191,52.44438244)(464.56144179,52.21938267)(464.69144409,52.00938477)
\curveto(464.83144152,51.79938309)(465.00144135,51.62438326)(465.20144409,51.48438477)
\curveto(465.2514411,51.45438343)(465.29644106,51.42938346)(465.33644409,51.40938477)
\curveto(465.37644098,51.3893835)(465.42144093,51.36438352)(465.47144409,51.33438477)
\curveto(465.5514408,51.2843836)(465.63644072,51.23938365)(465.72644409,51.19938477)
\curveto(465.82644053,51.16938372)(465.93144042,51.13938375)(466.04144409,51.10938477)
\curveto(466.09144026,51.0893838)(466.13644022,51.07938381)(466.17644409,51.07938477)
\curveto(466.22644013,51.0893838)(466.27644008,51.0893838)(466.32644409,51.07938477)
\curveto(466.35644,51.06938382)(466.41643994,51.05938383)(466.50644409,51.04938477)
\curveto(466.60643975,51.03938385)(466.68143967,51.04438384)(466.73144409,51.06438477)
\curveto(466.77143958,51.07438381)(466.81143954,51.07438381)(466.85144409,51.06438477)
\curveto(466.89143946,51.06438382)(466.93143942,51.07438381)(466.97144409,51.09438477)
\curveto(467.0514393,51.11438377)(467.13143922,51.12938376)(467.21144409,51.13938477)
\curveto(467.29143906,51.15938373)(467.36643899,51.1843837)(467.43644409,51.21438477)
\curveto(467.77643858,51.35438353)(468.0514383,51.54938334)(468.26144409,51.79938477)
\curveto(468.47143788,52.04938284)(468.64643771,52.34438254)(468.78644409,52.68438477)
\curveto(468.83643752,52.80438208)(468.86643749,52.92938196)(468.87644409,53.05938477)
\curveto(468.89643746,53.19938169)(468.92643743,53.33938155)(468.96644409,53.47938477)
}
}
{
\newrgbcolor{curcolor}{0 0 0}
\pscustom[linestyle=none,fillstyle=solid,fillcolor=curcolor]
{
\newpath
\moveto(474.92472534,58.03938477)
\curveto(475.66472055,58.04937684)(476.27971994,57.93937695)(476.76972534,57.70938477)
\curveto(477.26971895,57.4893774)(477.66471855,57.15437773)(477.95472534,56.70438477)
\curveto(478.08471813,56.50437838)(478.19471802,56.25937863)(478.28472534,55.96938477)
\curveto(478.30471791,55.91937897)(478.3197179,55.85437903)(478.32972534,55.77438477)
\curveto(478.33971788,55.69437919)(478.33471788,55.62437926)(478.31472534,55.56438477)
\curveto(478.28471793,55.51437937)(478.23471798,55.46937942)(478.16472534,55.42938477)
\curveto(478.13471808,55.40937948)(478.10471811,55.39937949)(478.07472534,55.39938477)
\curveto(478.04471817,55.40937948)(478.00971821,55.40937948)(477.96972534,55.39938477)
\curveto(477.92971829,55.3893795)(477.88971833,55.3843795)(477.84972534,55.38438477)
\curveto(477.80971841,55.39437949)(477.76971845,55.39937949)(477.72972534,55.39938477)
\lineto(477.41472534,55.39938477)
\curveto(477.3147189,55.40937948)(477.22971899,55.43937945)(477.15972534,55.48938477)
\curveto(477.07971914,55.54937934)(477.02471919,55.63437925)(476.99472534,55.74438477)
\curveto(476.96471925,55.85437903)(476.92471929,55.94937894)(476.87472534,56.02938477)
\curveto(476.72471949,56.2893786)(476.52971969,56.49437839)(476.28972534,56.64438477)
\curveto(476.20972001,56.69437819)(476.12472009,56.73437815)(476.03472534,56.76438477)
\curveto(475.94472027,56.80437808)(475.84972037,56.83937805)(475.74972534,56.86938477)
\curveto(475.60972061,56.90937798)(475.42472079,56.92937796)(475.19472534,56.92938477)
\curveto(474.96472125,56.93937795)(474.77472144,56.91937797)(474.62472534,56.86938477)
\curveto(474.55472166,56.84937804)(474.48972173,56.83437805)(474.42972534,56.82438477)
\curveto(474.36972185,56.81437807)(474.30472191,56.79937809)(474.23472534,56.77938477)
\curveto(473.97472224,56.66937822)(473.74472247,56.51937837)(473.54472534,56.32938477)
\curveto(473.34472287,56.13937875)(473.18972303,55.91437897)(473.07972534,55.65438477)
\curveto(473.03972318,55.56437932)(473.00472321,55.46937942)(472.97472534,55.36938477)
\curveto(472.94472327,55.27937961)(472.9147233,55.17937971)(472.88472534,55.06938477)
\lineto(472.79472534,54.66438477)
\curveto(472.78472343,54.61438027)(472.77972344,54.55938033)(472.77972534,54.49938477)
\curveto(472.78972343,54.43938045)(472.78472343,54.3843805)(472.76472534,54.33438477)
\lineto(472.76472534,54.21438477)
\curveto(472.75472346,54.17438071)(472.74472347,54.10938078)(472.73472534,54.01938477)
\curveto(472.73472348,53.92938096)(472.74472347,53.86438102)(472.76472534,53.82438477)
\curveto(472.77472344,53.77438111)(472.77472344,53.72438116)(472.76472534,53.67438477)
\curveto(472.75472346,53.62438126)(472.75472346,53.57438131)(472.76472534,53.52438477)
\curveto(472.77472344,53.4843814)(472.77972344,53.41438147)(472.77972534,53.31438477)
\curveto(472.79972342,53.23438165)(472.8147234,53.14938174)(472.82472534,53.05938477)
\curveto(472.84472337,52.96938192)(472.86472335,52.884382)(472.88472534,52.80438477)
\curveto(472.99472322,52.4843824)(473.1197231,52.20438268)(473.25972534,51.96438477)
\curveto(473.40972281,51.73438315)(473.6147226,51.53438335)(473.87472534,51.36438477)
\curveto(473.96472225,51.31438357)(474.05472216,51.26938362)(474.14472534,51.22938477)
\curveto(474.24472197,51.1893837)(474.34972187,51.14938374)(474.45972534,51.10938477)
\curveto(474.50972171,51.09938379)(474.54972167,51.09438379)(474.57972534,51.09438477)
\curveto(474.60972161,51.09438379)(474.64972157,51.0893838)(474.69972534,51.07938477)
\curveto(474.72972149,51.06938382)(474.77972144,51.06438382)(474.84972534,51.06438477)
\lineto(475.01472534,51.06438477)
\curveto(475.0147212,51.05438383)(475.03472118,51.04938384)(475.07472534,51.04938477)
\curveto(475.09472112,51.05938383)(475.1197211,51.05938383)(475.14972534,51.04938477)
\curveto(475.17972104,51.04938384)(475.20972101,51.05438383)(475.23972534,51.06438477)
\curveto(475.30972091,51.0843838)(475.37472084,51.0893838)(475.43472534,51.07938477)
\curveto(475.50472071,51.07938381)(475.57472064,51.0893838)(475.64472534,51.10938477)
\curveto(475.90472031,51.1893837)(476.12972009,51.2893836)(476.31972534,51.40938477)
\curveto(476.50971971,51.53938335)(476.66971955,51.70438318)(476.79972534,51.90438477)
\curveto(476.84971937,51.9843829)(476.89471932,52.06938282)(476.93472534,52.15938477)
\lineto(477.05472534,52.42938477)
\curveto(477.07471914,52.50938238)(477.09471912,52.5843823)(477.11472534,52.65438477)
\curveto(477.14471907,52.73438215)(477.19471902,52.79938209)(477.26472534,52.84938477)
\curveto(477.29471892,52.87938201)(477.35471886,52.89938199)(477.44472534,52.90938477)
\curveto(477.53471868,52.92938196)(477.62971859,52.93938195)(477.72972534,52.93938477)
\curveto(477.83971838,52.94938194)(477.93971828,52.94938194)(478.02972534,52.93938477)
\curveto(478.12971809,52.92938196)(478.19971802,52.90938198)(478.23972534,52.87938477)
\curveto(478.29971792,52.83938205)(478.33471788,52.77938211)(478.34472534,52.69938477)
\curveto(478.36471785,52.61938227)(478.36471785,52.53438235)(478.34472534,52.44438477)
\curveto(478.29471792,52.29438259)(478.24471797,52.14938274)(478.19472534,52.00938477)
\curveto(478.15471806,51.87938301)(478.09971812,51.74938314)(478.02972534,51.61938477)
\curveto(477.87971834,51.31938357)(477.68971853,51.05438383)(477.45972534,50.82438477)
\curveto(477.23971898,50.59438429)(476.96971925,50.40938448)(476.64972534,50.26938477)
\curveto(476.56971965,50.22938466)(476.48471973,50.19438469)(476.39472534,50.16438477)
\curveto(476.30471991,50.14438474)(476.20972001,50.11938477)(476.10972534,50.08938477)
\curveto(475.99972022,50.04938484)(475.88972033,50.02938486)(475.77972534,50.02938477)
\curveto(475.66972055,50.01938487)(475.55972066,50.00438488)(475.44972534,49.98438477)
\curveto(475.40972081,49.96438492)(475.36972085,49.95938493)(475.32972534,49.96938477)
\curveto(475.28972093,49.97938491)(475.24972097,49.97938491)(475.20972534,49.96938477)
\lineto(475.07472534,49.96938477)
\lineto(474.83472534,49.96938477)
\curveto(474.76472145,49.95938493)(474.69972152,49.96438492)(474.63972534,49.98438477)
\lineto(474.56472534,49.98438477)
\lineto(474.20472534,50.02938477)
\curveto(474.07472214,50.06938482)(473.94972227,50.10438478)(473.82972534,50.13438477)
\curveto(473.70972251,50.16438472)(473.59472262,50.20438468)(473.48472534,50.25438477)
\curveto(473.12472309,50.41438447)(472.82472339,50.60438428)(472.58472534,50.82438477)
\curveto(472.35472386,51.04438384)(472.13972408,51.31438357)(471.93972534,51.63438477)
\curveto(471.88972433,51.71438317)(471.84472437,51.80438308)(471.80472534,51.90438477)
\lineto(471.68472534,52.20438477)
\curveto(471.63472458,52.31438257)(471.59972462,52.42938246)(471.57972534,52.54938477)
\curveto(471.55972466,52.66938222)(471.53472468,52.7893821)(471.50472534,52.90938477)
\curveto(471.49472472,52.94938194)(471.48972473,52.9893819)(471.48972534,53.02938477)
\curveto(471.48972473,53.06938182)(471.48472473,53.10938178)(471.47472534,53.14938477)
\curveto(471.45472476,53.20938168)(471.44472477,53.27438161)(471.44472534,53.34438477)
\curveto(471.45472476,53.41438147)(471.44972477,53.47938141)(471.42972534,53.53938477)
\lineto(471.42972534,53.68938477)
\curveto(471.4197248,53.73938115)(471.4147248,53.80938108)(471.41472534,53.89938477)
\curveto(471.4147248,53.9893809)(471.4197248,54.05938083)(471.42972534,54.10938477)
\curveto(471.43972478,54.15938073)(471.43972478,54.20438068)(471.42972534,54.24438477)
\curveto(471.42972479,54.2843806)(471.43472478,54.32438056)(471.44472534,54.36438477)
\curveto(471.46472475,54.43438045)(471.46972475,54.50438038)(471.45972534,54.57438477)
\curveto(471.45972476,54.64438024)(471.46972475,54.70938018)(471.48972534,54.76938477)
\curveto(471.52972469,54.93937995)(471.56472465,55.10937978)(471.59472534,55.27938477)
\curveto(471.62472459,55.44937944)(471.66972455,55.60937928)(471.72972534,55.75938477)
\curveto(471.93972428,56.27937861)(472.19472402,56.69937819)(472.49472534,57.01938477)
\curveto(472.79472342,57.33937755)(473.20472301,57.60437728)(473.72472534,57.81438477)
\curveto(473.83472238,57.86437702)(473.95472226,57.89937699)(474.08472534,57.91938477)
\curveto(474.214722,57.93937695)(474.34972187,57.96437692)(474.48972534,57.99438477)
\curveto(474.55972166,58.00437688)(474.62972159,58.00937688)(474.69972534,58.00938477)
\curveto(474.76972145,58.01937687)(474.84472137,58.02937686)(474.92472534,58.03938477)
}
}
{
\newrgbcolor{curcolor}{0 0 0}
\pscustom[linestyle=none,fillstyle=solid,fillcolor=curcolor]
{
\newpath
\moveto(486.59636597,54.30438477)
\curveto(486.61635828,54.20438068)(486.61635828,54.0893808)(486.59636597,53.95938477)
\curveto(486.58635831,53.83938105)(486.55635834,53.75438113)(486.50636597,53.70438477)
\curveto(486.45635844,53.66438122)(486.38135852,53.63438125)(486.28136597,53.61438477)
\curveto(486.19135871,53.60438128)(486.08635881,53.59938129)(485.96636597,53.59938477)
\lineto(485.60636597,53.59938477)
\curveto(485.48635941,53.60938128)(485.38135952,53.61438127)(485.29136597,53.61438477)
\lineto(481.45136597,53.61438477)
\curveto(481.37136353,53.61438127)(481.29136361,53.60938128)(481.21136597,53.59938477)
\curveto(481.13136377,53.59938129)(481.06636383,53.5843813)(481.01636597,53.55438477)
\curveto(480.97636392,53.53438135)(480.93636396,53.49438139)(480.89636597,53.43438477)
\curveto(480.87636402,53.40438148)(480.85636404,53.35938153)(480.83636597,53.29938477)
\curveto(480.81636408,53.24938164)(480.81636408,53.19938169)(480.83636597,53.14938477)
\curveto(480.84636405,53.09938179)(480.85136405,53.05438183)(480.85136597,53.01438477)
\curveto(480.85136405,52.97438191)(480.85636404,52.93438195)(480.86636597,52.89438477)
\curveto(480.88636401,52.81438207)(480.90636399,52.72938216)(480.92636597,52.63938477)
\curveto(480.94636395,52.55938233)(480.97636392,52.47938241)(481.01636597,52.39938477)
\curveto(481.24636365,51.85938303)(481.62636327,51.47438341)(482.15636597,51.24438477)
\curveto(482.21636268,51.21438367)(482.28136262,51.1893837)(482.35136597,51.16938477)
\lineto(482.56136597,51.10938477)
\curveto(482.59136231,51.09938379)(482.64136226,51.09438379)(482.71136597,51.09438477)
\curveto(482.85136205,51.05438383)(483.03636186,51.03438385)(483.26636597,51.03438477)
\curveto(483.4963614,51.03438385)(483.68136122,51.05438383)(483.82136597,51.09438477)
\curveto(483.96136094,51.13438375)(484.08636081,51.17438371)(484.19636597,51.21438477)
\curveto(484.31636058,51.26438362)(484.42636047,51.32438356)(484.52636597,51.39438477)
\curveto(484.63636026,51.46438342)(484.73136017,51.54438334)(484.81136597,51.63438477)
\curveto(484.89136001,51.73438315)(484.96135994,51.83938305)(485.02136597,51.94938477)
\curveto(485.08135982,52.04938284)(485.13135977,52.15438273)(485.17136597,52.26438477)
\curveto(485.22135968,52.37438251)(485.3013596,52.45438243)(485.41136597,52.50438477)
\curveto(485.45135945,52.52438236)(485.51635938,52.53938235)(485.60636597,52.54938477)
\curveto(485.6963592,52.55938233)(485.78635911,52.55938233)(485.87636597,52.54938477)
\curveto(485.96635893,52.54938234)(486.05135885,52.54438234)(486.13136597,52.53438477)
\curveto(486.21135869,52.52438236)(486.26635863,52.50438238)(486.29636597,52.47438477)
\curveto(486.3963585,52.40438248)(486.42135848,52.2893826)(486.37136597,52.12938477)
\curveto(486.29135861,51.85938303)(486.18635871,51.61938327)(486.05636597,51.40938477)
\curveto(485.85635904,51.0893838)(485.62635927,50.82438406)(485.36636597,50.61438477)
\curveto(485.11635978,50.41438447)(484.7963601,50.24938464)(484.40636597,50.11938477)
\curveto(484.30636059,50.07938481)(484.20636069,50.05438483)(484.10636597,50.04438477)
\curveto(484.00636089,50.02438486)(483.901361,50.00438488)(483.79136597,49.98438477)
\curveto(483.74136116,49.97438491)(483.69136121,49.96938492)(483.64136597,49.96938477)
\curveto(483.6013613,49.96938492)(483.55636134,49.96438492)(483.50636597,49.95438477)
\lineto(483.35636597,49.95438477)
\curveto(483.30636159,49.94438494)(483.24636165,49.93938495)(483.17636597,49.93938477)
\curveto(483.11636178,49.93938495)(483.06636183,49.94438494)(483.02636597,49.95438477)
\lineto(482.89136597,49.95438477)
\curveto(482.84136206,49.96438492)(482.7963621,49.96938492)(482.75636597,49.96938477)
\curveto(482.71636218,49.96938492)(482.67636222,49.97438491)(482.63636597,49.98438477)
\curveto(482.58636231,49.99438489)(482.53136237,50.00438488)(482.47136597,50.01438477)
\curveto(482.41136249,50.01438487)(482.35636254,50.01938487)(482.30636597,50.02938477)
\curveto(482.21636268,50.04938484)(482.12636277,50.07438481)(482.03636597,50.10438477)
\curveto(481.94636295,50.12438476)(481.86136304,50.14938474)(481.78136597,50.17938477)
\curveto(481.74136316,50.19938469)(481.70636319,50.20938468)(481.67636597,50.20938477)
\curveto(481.64636325,50.21938467)(481.61136329,50.23438465)(481.57136597,50.25438477)
\curveto(481.42136348,50.32438456)(481.26136364,50.40938448)(481.09136597,50.50938477)
\curveto(480.8013641,50.69938419)(480.55136435,50.92938396)(480.34136597,51.19938477)
\curveto(480.14136476,51.47938341)(479.97136493,51.7893831)(479.83136597,52.12938477)
\curveto(479.78136512,52.23938265)(479.74136516,52.35438253)(479.71136597,52.47438477)
\curveto(479.69136521,52.59438229)(479.66136524,52.71438217)(479.62136597,52.83438477)
\curveto(479.61136529,52.87438201)(479.60636529,52.90938198)(479.60636597,52.93938477)
\curveto(479.60636529,52.96938192)(479.6013653,53.00938188)(479.59136597,53.05938477)
\curveto(479.57136533,53.13938175)(479.55636534,53.22438166)(479.54636597,53.31438477)
\curveto(479.53636536,53.40438148)(479.52136538,53.49438139)(479.50136597,53.58438477)
\lineto(479.50136597,53.79438477)
\curveto(479.49136541,53.83438105)(479.48136542,53.889381)(479.47136597,53.95938477)
\curveto(479.47136543,54.03938085)(479.47636542,54.10438078)(479.48636597,54.15438477)
\lineto(479.48636597,54.31938477)
\curveto(479.50636539,54.36938052)(479.51136539,54.41938047)(479.50136597,54.46938477)
\curveto(479.5013654,54.52938036)(479.50636539,54.5843803)(479.51636597,54.63438477)
\curveto(479.55636534,54.79438009)(479.58636531,54.95437993)(479.60636597,55.11438477)
\curveto(479.63636526,55.27437961)(479.68136522,55.42437946)(479.74136597,55.56438477)
\curveto(479.79136511,55.67437921)(479.83636506,55.7843791)(479.87636597,55.89438477)
\curveto(479.92636497,56.01437887)(479.98136492,56.12937876)(480.04136597,56.23938477)
\curveto(480.26136464,56.5893783)(480.51136439,56.889378)(480.79136597,57.13938477)
\curveto(481.07136383,57.39937749)(481.41636348,57.61437727)(481.82636597,57.78438477)
\curveto(481.94636295,57.83437705)(482.06636283,57.86937702)(482.18636597,57.88938477)
\curveto(482.31636258,57.91937697)(482.45136245,57.94937694)(482.59136597,57.97938477)
\curveto(482.64136226,57.9893769)(482.68636221,57.99437689)(482.72636597,57.99438477)
\curveto(482.76636213,58.00437688)(482.81136209,58.00937688)(482.86136597,58.00938477)
\curveto(482.88136202,58.01937687)(482.90636199,58.01937687)(482.93636597,58.00938477)
\curveto(482.96636193,57.99937689)(482.99136191,58.00437688)(483.01136597,58.02438477)
\curveto(483.43136147,58.03437685)(483.7963611,57.9893769)(484.10636597,57.88938477)
\curveto(484.41636048,57.79937709)(484.6963602,57.67437721)(484.94636597,57.51438477)
\curveto(484.9963599,57.49437739)(485.03635986,57.46437742)(485.06636597,57.42438477)
\curveto(485.0963598,57.39437749)(485.13135977,57.36937752)(485.17136597,57.34938477)
\curveto(485.25135965,57.2893776)(485.33135957,57.21937767)(485.41136597,57.13938477)
\curveto(485.5013594,57.05937783)(485.57635932,56.97937791)(485.63636597,56.89938477)
\curveto(485.7963591,56.6893782)(485.93135897,56.4893784)(486.04136597,56.29938477)
\curveto(486.11135879,56.1893787)(486.16635873,56.06937882)(486.20636597,55.93938477)
\curveto(486.24635865,55.80937908)(486.29135861,55.67937921)(486.34136597,55.54938477)
\curveto(486.39135851,55.41937947)(486.42635847,55.2843796)(486.44636597,55.14438477)
\curveto(486.47635842,55.00437988)(486.51135839,54.86438002)(486.55136597,54.72438477)
\curveto(486.56135834,54.65438023)(486.56635833,54.5843803)(486.56636597,54.51438477)
\lineto(486.59636597,54.30438477)
\moveto(485.14136597,54.81438477)
\curveto(485.17135973,54.85438003)(485.1963597,54.90437998)(485.21636597,54.96438477)
\curveto(485.23635966,55.03437985)(485.23635966,55.10437978)(485.21636597,55.17438477)
\curveto(485.15635974,55.39437949)(485.07135983,55.59937929)(484.96136597,55.78938477)
\curveto(484.82136008,56.01937887)(484.66636023,56.21437867)(484.49636597,56.37438477)
\curveto(484.32636057,56.53437835)(484.10636079,56.66937822)(483.83636597,56.77938477)
\curveto(483.76636113,56.79937809)(483.6963612,56.81437807)(483.62636597,56.82438477)
\curveto(483.55636134,56.84437804)(483.48136142,56.86437802)(483.40136597,56.88438477)
\curveto(483.32136158,56.90437798)(483.23636166,56.91437797)(483.14636597,56.91438477)
\lineto(482.89136597,56.91438477)
\curveto(482.86136204,56.89437799)(482.82636207,56.884378)(482.78636597,56.88438477)
\curveto(482.74636215,56.89437799)(482.71136219,56.89437799)(482.68136597,56.88438477)
\lineto(482.44136597,56.82438477)
\curveto(482.37136253,56.81437807)(482.3013626,56.79937809)(482.23136597,56.77938477)
\curveto(481.94136296,56.65937823)(481.70636319,56.50937838)(481.52636597,56.32938477)
\curveto(481.35636354,56.14937874)(481.2013637,55.92437896)(481.06136597,55.65438477)
\curveto(481.03136387,55.60437928)(481.0013639,55.53937935)(480.97136597,55.45938477)
\curveto(480.94136396,55.3893795)(480.91636398,55.30937958)(480.89636597,55.21938477)
\curveto(480.87636402,55.12937976)(480.87136403,55.04437984)(480.88136597,54.96438477)
\curveto(480.89136401,54.88438)(480.92636397,54.82438006)(480.98636597,54.78438477)
\curveto(481.06636383,54.72438016)(481.2013637,54.69438019)(481.39136597,54.69438477)
\curveto(481.59136331,54.70438018)(481.76136314,54.70938018)(481.90136597,54.70938477)
\lineto(484.18136597,54.70938477)
\curveto(484.33136057,54.70938018)(484.51136039,54.70438018)(484.72136597,54.69438477)
\curveto(484.93135997,54.69438019)(485.07135983,54.73438015)(485.14136597,54.81438477)
}
}
{
\newrgbcolor{curcolor}{0 0 0}
\pscustom[linestyle=none,fillstyle=solid,fillcolor=curcolor]
{
\newpath
\moveto(491.59300659,58.00938477)
\curveto(492.22300136,58.02937686)(492.72800085,57.94437694)(493.10800659,57.75438477)
\curveto(493.48800009,57.56437732)(493.79299979,57.27937761)(494.02300659,56.89938477)
\curveto(494.0829995,56.79937809)(494.12799945,56.6893782)(494.15800659,56.56938477)
\curveto(494.19799938,56.45937843)(494.23299935,56.34437854)(494.26300659,56.22438477)
\curveto(494.31299927,56.03437885)(494.34299924,55.82937906)(494.35300659,55.60938477)
\curveto(494.36299922,55.3893795)(494.36799921,55.16437972)(494.36800659,54.93438477)
\lineto(494.36800659,53.32938477)
\lineto(494.36800659,50.98938477)
\curveto(494.36799921,50.81938407)(494.36299922,50.64938424)(494.35300659,50.47938477)
\curveto(494.35299923,50.30938458)(494.28799929,50.19938469)(494.15800659,50.14938477)
\curveto(494.10799947,50.12938476)(494.05299953,50.11938477)(493.99300659,50.11938477)
\curveto(493.94299964,50.10938478)(493.88799969,50.10438478)(493.82800659,50.10438477)
\curveto(493.69799988,50.10438478)(493.57300001,50.10938478)(493.45300659,50.11938477)
\curveto(493.33300025,50.11938477)(493.24800033,50.15938473)(493.19800659,50.23938477)
\curveto(493.14800043,50.30938458)(493.12300046,50.39938449)(493.12300659,50.50938477)
\lineto(493.12300659,50.83938477)
\lineto(493.12300659,52.12938477)
\lineto(493.12300659,54.57438477)
\curveto(493.12300046,54.84438004)(493.11800046,55.10937978)(493.10800659,55.36938477)
\curveto(493.09800048,55.63937925)(493.05300053,55.86937902)(492.97300659,56.05938477)
\curveto(492.89300069,56.25937863)(492.77300081,56.41937847)(492.61300659,56.53938477)
\curveto(492.45300113,56.66937822)(492.26800131,56.76937812)(492.05800659,56.83938477)
\curveto(491.99800158,56.85937803)(491.93300165,56.86937802)(491.86300659,56.86938477)
\curveto(491.80300178,56.87937801)(491.74300184,56.89437799)(491.68300659,56.91438477)
\curveto(491.63300195,56.92437796)(491.55300203,56.92437796)(491.44300659,56.91438477)
\curveto(491.34300224,56.91437797)(491.27300231,56.90937798)(491.23300659,56.89938477)
\curveto(491.19300239,56.87937801)(491.15800242,56.86937802)(491.12800659,56.86938477)
\curveto(491.09800248,56.87937801)(491.06300252,56.87937801)(491.02300659,56.86938477)
\curveto(490.89300269,56.83937805)(490.76800281,56.80437808)(490.64800659,56.76438477)
\curveto(490.53800304,56.73437815)(490.43300315,56.6893782)(490.33300659,56.62938477)
\curveto(490.29300329,56.60937828)(490.25800332,56.5893783)(490.22800659,56.56938477)
\curveto(490.19800338,56.54937834)(490.16300342,56.52937836)(490.12300659,56.50938477)
\curveto(489.77300381,56.25937863)(489.51800406,55.884379)(489.35800659,55.38438477)
\curveto(489.32800425,55.30437958)(489.30800427,55.21937967)(489.29800659,55.12938477)
\curveto(489.28800429,55.04937984)(489.27300431,54.96937992)(489.25300659,54.88938477)
\curveto(489.23300435,54.83938005)(489.22800435,54.7893801)(489.23800659,54.73938477)
\curveto(489.24800433,54.69938019)(489.24300434,54.65938023)(489.22300659,54.61938477)
\lineto(489.22300659,54.30438477)
\curveto(489.21300437,54.27438061)(489.20800437,54.23938065)(489.20800659,54.19938477)
\curveto(489.21800436,54.15938073)(489.22300436,54.11438077)(489.22300659,54.06438477)
\lineto(489.22300659,53.61438477)
\lineto(489.22300659,52.17438477)
\lineto(489.22300659,50.85438477)
\lineto(489.22300659,50.50938477)
\curveto(489.22300436,50.39938449)(489.19800438,50.30938458)(489.14800659,50.23938477)
\curveto(489.09800448,50.15938473)(489.00800457,50.11938477)(488.87800659,50.11938477)
\curveto(488.75800482,50.10938478)(488.63300495,50.10438478)(488.50300659,50.10438477)
\curveto(488.42300516,50.10438478)(488.34800523,50.10938478)(488.27800659,50.11938477)
\curveto(488.20800537,50.12938476)(488.14800543,50.15438473)(488.09800659,50.19438477)
\curveto(488.01800556,50.24438464)(487.9780056,50.33938455)(487.97800659,50.47938477)
\lineto(487.97800659,50.88438477)
\lineto(487.97800659,52.65438477)
\lineto(487.97800659,56.28438477)
\lineto(487.97800659,57.19938477)
\lineto(487.97800659,57.46938477)
\curveto(487.9780056,57.55937733)(487.99800558,57.62937726)(488.03800659,57.67938477)
\curveto(488.06800551,57.73937715)(488.11800546,57.77937711)(488.18800659,57.79938477)
\curveto(488.22800535,57.80937708)(488.2830053,57.81937707)(488.35300659,57.82938477)
\curveto(488.43300515,57.83937705)(488.51300507,57.84437704)(488.59300659,57.84438477)
\curveto(488.67300491,57.84437704)(488.74800483,57.83937705)(488.81800659,57.82938477)
\curveto(488.89800468,57.81937707)(488.95300463,57.80437708)(488.98300659,57.78438477)
\curveto(489.09300449,57.71437717)(489.14300444,57.62437726)(489.13300659,57.51438477)
\curveto(489.12300446,57.41437747)(489.13800444,57.29937759)(489.17800659,57.16938477)
\curveto(489.19800438,57.10937778)(489.23800434,57.05937783)(489.29800659,57.01938477)
\curveto(489.41800416,57.00937788)(489.51300407,57.05437783)(489.58300659,57.15438477)
\curveto(489.66300392,57.25437763)(489.74300384,57.33437755)(489.82300659,57.39438477)
\curveto(489.96300362,57.49437739)(490.10300348,57.5843773)(490.24300659,57.66438477)
\curveto(490.39300319,57.75437713)(490.56300302,57.82937706)(490.75300659,57.88938477)
\curveto(490.83300275,57.91937697)(490.91800266,57.93937695)(491.00800659,57.94938477)
\curveto(491.10800247,57.95937693)(491.20300238,57.97437691)(491.29300659,57.99438477)
\curveto(491.34300224,58.00437688)(491.39300219,58.00937688)(491.44300659,58.00938477)
\lineto(491.59300659,58.00938477)
}
}
{
\newrgbcolor{curcolor}{0 0 0}
\pscustom[linestyle=none,fillstyle=solid,fillcolor=curcolor]
{
\newpath
\moveto(497.19761597,60.19938477)
\curveto(497.34761396,60.19937469)(497.49761381,60.19437469)(497.64761597,60.18438477)
\curveto(497.79761351,60.1843747)(497.9026134,60.14437474)(497.96261597,60.06438477)
\curveto(498.01261329,60.00437488)(498.03761327,59.91937497)(498.03761597,59.80938477)
\curveto(498.04761326,59.70937518)(498.05261325,59.60437528)(498.05261597,59.49438477)
\lineto(498.05261597,58.62438477)
\curveto(498.05261325,58.54437634)(498.04761326,58.45937643)(498.03761597,58.36938477)
\curveto(498.03761327,58.2893766)(498.04761326,58.21937667)(498.06761597,58.15938477)
\curveto(498.1076132,58.01937687)(498.19761311,57.92937696)(498.33761597,57.88938477)
\curveto(498.38761292,57.87937701)(498.43261287,57.87437701)(498.47261597,57.87438477)
\lineto(498.62261597,57.87438477)
\lineto(499.02761597,57.87438477)
\curveto(499.18761212,57.884377)(499.302612,57.87437701)(499.37261597,57.84438477)
\curveto(499.46261184,57.7843771)(499.52261178,57.72437716)(499.55261597,57.66438477)
\curveto(499.57261173,57.62437726)(499.58261172,57.57937731)(499.58261597,57.52938477)
\lineto(499.58261597,57.37938477)
\curveto(499.58261172,57.26937762)(499.57761173,57.16437772)(499.56761597,57.06438477)
\curveto(499.55761175,56.97437791)(499.52261178,56.90437798)(499.46261597,56.85438477)
\curveto(499.4026119,56.80437808)(499.31761199,56.77437811)(499.20761597,56.76438477)
\lineto(498.87761597,56.76438477)
\curveto(498.76761254,56.77437811)(498.65761265,56.77937811)(498.54761597,56.77938477)
\curveto(498.43761287,56.77937811)(498.34261296,56.76437812)(498.26261597,56.73438477)
\curveto(498.19261311,56.70437818)(498.14261316,56.65437823)(498.11261597,56.58438477)
\curveto(498.08261322,56.51437837)(498.06261324,56.42937846)(498.05261597,56.32938477)
\curveto(498.04261326,56.23937865)(498.03761327,56.13937875)(498.03761597,56.02938477)
\curveto(498.04761326,55.92937896)(498.05261325,55.82937906)(498.05261597,55.72938477)
\lineto(498.05261597,52.75938477)
\curveto(498.05261325,52.53938235)(498.04761326,52.30438258)(498.03761597,52.05438477)
\curveto(498.03761327,51.81438307)(498.08261322,51.62938326)(498.17261597,51.49938477)
\curveto(498.22261308,51.41938347)(498.28761302,51.36438352)(498.36761597,51.33438477)
\curveto(498.44761286,51.30438358)(498.54261276,51.27938361)(498.65261597,51.25938477)
\curveto(498.68261262,51.24938364)(498.71261259,51.24438364)(498.74261597,51.24438477)
\curveto(498.78261252,51.25438363)(498.81761249,51.25438363)(498.84761597,51.24438477)
\lineto(499.04261597,51.24438477)
\curveto(499.14261216,51.24438364)(499.23261207,51.23438365)(499.31261597,51.21438477)
\curveto(499.4026119,51.20438368)(499.46761184,51.16938372)(499.50761597,51.10938477)
\curveto(499.52761178,51.07938381)(499.54261176,51.02438386)(499.55261597,50.94438477)
\curveto(499.57261173,50.87438401)(499.58261172,50.79938409)(499.58261597,50.71938477)
\curveto(499.59261171,50.63938425)(499.59261171,50.55938433)(499.58261597,50.47938477)
\curveto(499.57261173,50.40938448)(499.55261175,50.35438453)(499.52261597,50.31438477)
\curveto(499.48261182,50.24438464)(499.4076119,50.19438469)(499.29761597,50.16438477)
\curveto(499.21761209,50.14438474)(499.12761218,50.13438475)(499.02761597,50.13438477)
\curveto(498.92761238,50.14438474)(498.83761247,50.14938474)(498.75761597,50.14938477)
\curveto(498.69761261,50.14938474)(498.63761267,50.14438474)(498.57761597,50.13438477)
\curveto(498.51761279,50.13438475)(498.46261284,50.13938475)(498.41261597,50.14938477)
\lineto(498.23261597,50.14938477)
\curveto(498.18261312,50.15938473)(498.13261317,50.16438472)(498.08261597,50.16438477)
\curveto(498.04261326,50.17438471)(497.99761331,50.17938471)(497.94761597,50.17938477)
\curveto(497.74761356,50.22938466)(497.57261373,50.2843846)(497.42261597,50.34438477)
\curveto(497.28261402,50.40438448)(497.16261414,50.50938438)(497.06261597,50.65938477)
\curveto(496.92261438,50.85938403)(496.84261446,51.10938378)(496.82261597,51.40938477)
\curveto(496.8026145,51.71938317)(496.79261451,52.04938284)(496.79261597,52.39938477)
\lineto(496.79261597,56.32938477)
\curveto(496.76261454,56.45937843)(496.73261457,56.55437833)(496.70261597,56.61438477)
\curveto(496.68261462,56.67437821)(496.61261469,56.72437816)(496.49261597,56.76438477)
\curveto(496.45261485,56.77437811)(496.41261489,56.77437811)(496.37261597,56.76438477)
\curveto(496.33261497,56.75437813)(496.29261501,56.75937813)(496.25261597,56.77938477)
\lineto(496.01261597,56.77938477)
\curveto(495.88261542,56.77937811)(495.77261553,56.7893781)(495.68261597,56.80938477)
\curveto(495.6026157,56.83937805)(495.54761576,56.89937799)(495.51761597,56.98938477)
\curveto(495.49761581,57.02937786)(495.48261582,57.07437781)(495.47261597,57.12438477)
\lineto(495.47261597,57.27438477)
\curveto(495.47261583,57.41437747)(495.48261582,57.52937736)(495.50261597,57.61938477)
\curveto(495.52261578,57.71937717)(495.58261572,57.79437709)(495.68261597,57.84438477)
\curveto(495.79261551,57.884377)(495.93261537,57.89437699)(496.10261597,57.87438477)
\curveto(496.28261502,57.85437703)(496.43261487,57.86437702)(496.55261597,57.90438477)
\curveto(496.64261466,57.95437693)(496.71261459,58.02437686)(496.76261597,58.11438477)
\curveto(496.78261452,58.17437671)(496.79261451,58.24937664)(496.79261597,58.33938477)
\lineto(496.79261597,58.59438477)
\lineto(496.79261597,59.52438477)
\lineto(496.79261597,59.76438477)
\curveto(496.79261451,59.85437503)(496.8026145,59.92937496)(496.82261597,59.98938477)
\curveto(496.86261444,60.06937482)(496.93761437,60.13437475)(497.04761597,60.18438477)
\curveto(497.07761423,60.1843747)(497.1026142,60.1843747)(497.12261597,60.18438477)
\curveto(497.15261415,60.19437469)(497.17761413,60.19937469)(497.19761597,60.19938477)
}
}
{
\newrgbcolor{curcolor}{0 0 0}
\pscustom[linestyle=none,fillstyle=solid,fillcolor=curcolor]
{
\newpath
\moveto(507.71941284,54.30438477)
\curveto(507.73940516,54.20438068)(507.73940516,54.0893808)(507.71941284,53.95938477)
\curveto(507.70940519,53.83938105)(507.67940522,53.75438113)(507.62941284,53.70438477)
\curveto(507.57940532,53.66438122)(507.50440539,53.63438125)(507.40441284,53.61438477)
\curveto(507.31440558,53.60438128)(507.20940569,53.59938129)(507.08941284,53.59938477)
\lineto(506.72941284,53.59938477)
\curveto(506.60940629,53.60938128)(506.50440639,53.61438127)(506.41441284,53.61438477)
\lineto(502.57441284,53.61438477)
\curveto(502.4944104,53.61438127)(502.41441048,53.60938128)(502.33441284,53.59938477)
\curveto(502.25441064,53.59938129)(502.18941071,53.5843813)(502.13941284,53.55438477)
\curveto(502.0994108,53.53438135)(502.05941084,53.49438139)(502.01941284,53.43438477)
\curveto(501.9994109,53.40438148)(501.97941092,53.35938153)(501.95941284,53.29938477)
\curveto(501.93941096,53.24938164)(501.93941096,53.19938169)(501.95941284,53.14938477)
\curveto(501.96941093,53.09938179)(501.97441092,53.05438183)(501.97441284,53.01438477)
\curveto(501.97441092,52.97438191)(501.97941092,52.93438195)(501.98941284,52.89438477)
\curveto(502.00941089,52.81438207)(502.02941087,52.72938216)(502.04941284,52.63938477)
\curveto(502.06941083,52.55938233)(502.0994108,52.47938241)(502.13941284,52.39938477)
\curveto(502.36941053,51.85938303)(502.74941015,51.47438341)(503.27941284,51.24438477)
\curveto(503.33940956,51.21438367)(503.40440949,51.1893837)(503.47441284,51.16938477)
\lineto(503.68441284,51.10938477)
\curveto(503.71440918,51.09938379)(503.76440913,51.09438379)(503.83441284,51.09438477)
\curveto(503.97440892,51.05438383)(504.15940874,51.03438385)(504.38941284,51.03438477)
\curveto(504.61940828,51.03438385)(504.80440809,51.05438383)(504.94441284,51.09438477)
\curveto(505.08440781,51.13438375)(505.20940769,51.17438371)(505.31941284,51.21438477)
\curveto(505.43940746,51.26438362)(505.54940735,51.32438356)(505.64941284,51.39438477)
\curveto(505.75940714,51.46438342)(505.85440704,51.54438334)(505.93441284,51.63438477)
\curveto(506.01440688,51.73438315)(506.08440681,51.83938305)(506.14441284,51.94938477)
\curveto(506.20440669,52.04938284)(506.25440664,52.15438273)(506.29441284,52.26438477)
\curveto(506.34440655,52.37438251)(506.42440647,52.45438243)(506.53441284,52.50438477)
\curveto(506.57440632,52.52438236)(506.63940626,52.53938235)(506.72941284,52.54938477)
\curveto(506.81940608,52.55938233)(506.90940599,52.55938233)(506.99941284,52.54938477)
\curveto(507.08940581,52.54938234)(507.17440572,52.54438234)(507.25441284,52.53438477)
\curveto(507.33440556,52.52438236)(507.38940551,52.50438238)(507.41941284,52.47438477)
\curveto(507.51940538,52.40438248)(507.54440535,52.2893826)(507.49441284,52.12938477)
\curveto(507.41440548,51.85938303)(507.30940559,51.61938327)(507.17941284,51.40938477)
\curveto(506.97940592,51.0893838)(506.74940615,50.82438406)(506.48941284,50.61438477)
\curveto(506.23940666,50.41438447)(505.91940698,50.24938464)(505.52941284,50.11938477)
\curveto(505.42940747,50.07938481)(505.32940757,50.05438483)(505.22941284,50.04438477)
\curveto(505.12940777,50.02438486)(505.02440787,50.00438488)(504.91441284,49.98438477)
\curveto(504.86440803,49.97438491)(504.81440808,49.96938492)(504.76441284,49.96938477)
\curveto(504.72440817,49.96938492)(504.67940822,49.96438492)(504.62941284,49.95438477)
\lineto(504.47941284,49.95438477)
\curveto(504.42940847,49.94438494)(504.36940853,49.93938495)(504.29941284,49.93938477)
\curveto(504.23940866,49.93938495)(504.18940871,49.94438494)(504.14941284,49.95438477)
\lineto(504.01441284,49.95438477)
\curveto(503.96440893,49.96438492)(503.91940898,49.96938492)(503.87941284,49.96938477)
\curveto(503.83940906,49.96938492)(503.7994091,49.97438491)(503.75941284,49.98438477)
\curveto(503.70940919,49.99438489)(503.65440924,50.00438488)(503.59441284,50.01438477)
\curveto(503.53440936,50.01438487)(503.47940942,50.01938487)(503.42941284,50.02938477)
\curveto(503.33940956,50.04938484)(503.24940965,50.07438481)(503.15941284,50.10438477)
\curveto(503.06940983,50.12438476)(502.98440991,50.14938474)(502.90441284,50.17938477)
\curveto(502.86441003,50.19938469)(502.82941007,50.20938468)(502.79941284,50.20938477)
\curveto(502.76941013,50.21938467)(502.73441016,50.23438465)(502.69441284,50.25438477)
\curveto(502.54441035,50.32438456)(502.38441051,50.40938448)(502.21441284,50.50938477)
\curveto(501.92441097,50.69938419)(501.67441122,50.92938396)(501.46441284,51.19938477)
\curveto(501.26441163,51.47938341)(501.0944118,51.7893831)(500.95441284,52.12938477)
\curveto(500.90441199,52.23938265)(500.86441203,52.35438253)(500.83441284,52.47438477)
\curveto(500.81441208,52.59438229)(500.78441211,52.71438217)(500.74441284,52.83438477)
\curveto(500.73441216,52.87438201)(500.72941217,52.90938198)(500.72941284,52.93938477)
\curveto(500.72941217,52.96938192)(500.72441217,53.00938188)(500.71441284,53.05938477)
\curveto(500.6944122,53.13938175)(500.67941222,53.22438166)(500.66941284,53.31438477)
\curveto(500.65941224,53.40438148)(500.64441225,53.49438139)(500.62441284,53.58438477)
\lineto(500.62441284,53.79438477)
\curveto(500.61441228,53.83438105)(500.60441229,53.889381)(500.59441284,53.95938477)
\curveto(500.5944123,54.03938085)(500.5994123,54.10438078)(500.60941284,54.15438477)
\lineto(500.60941284,54.31938477)
\curveto(500.62941227,54.36938052)(500.63441226,54.41938047)(500.62441284,54.46938477)
\curveto(500.62441227,54.52938036)(500.62941227,54.5843803)(500.63941284,54.63438477)
\curveto(500.67941222,54.79438009)(500.70941219,54.95437993)(500.72941284,55.11438477)
\curveto(500.75941214,55.27437961)(500.80441209,55.42437946)(500.86441284,55.56438477)
\curveto(500.91441198,55.67437921)(500.95941194,55.7843791)(500.99941284,55.89438477)
\curveto(501.04941185,56.01437887)(501.10441179,56.12937876)(501.16441284,56.23938477)
\curveto(501.38441151,56.5893783)(501.63441126,56.889378)(501.91441284,57.13938477)
\curveto(502.1944107,57.39937749)(502.53941036,57.61437727)(502.94941284,57.78438477)
\curveto(503.06940983,57.83437705)(503.18940971,57.86937702)(503.30941284,57.88938477)
\curveto(503.43940946,57.91937697)(503.57440932,57.94937694)(503.71441284,57.97938477)
\curveto(503.76440913,57.9893769)(503.80940909,57.99437689)(503.84941284,57.99438477)
\curveto(503.88940901,58.00437688)(503.93440896,58.00937688)(503.98441284,58.00938477)
\curveto(504.00440889,58.01937687)(504.02940887,58.01937687)(504.05941284,58.00938477)
\curveto(504.08940881,57.99937689)(504.11440878,58.00437688)(504.13441284,58.02438477)
\curveto(504.55440834,58.03437685)(504.91940798,57.9893769)(505.22941284,57.88938477)
\curveto(505.53940736,57.79937709)(505.81940708,57.67437721)(506.06941284,57.51438477)
\curveto(506.11940678,57.49437739)(506.15940674,57.46437742)(506.18941284,57.42438477)
\curveto(506.21940668,57.39437749)(506.25440664,57.36937752)(506.29441284,57.34938477)
\curveto(506.37440652,57.2893776)(506.45440644,57.21937767)(506.53441284,57.13938477)
\curveto(506.62440627,57.05937783)(506.6994062,56.97937791)(506.75941284,56.89938477)
\curveto(506.91940598,56.6893782)(507.05440584,56.4893784)(507.16441284,56.29938477)
\curveto(507.23440566,56.1893787)(507.28940561,56.06937882)(507.32941284,55.93938477)
\curveto(507.36940553,55.80937908)(507.41440548,55.67937921)(507.46441284,55.54938477)
\curveto(507.51440538,55.41937947)(507.54940535,55.2843796)(507.56941284,55.14438477)
\curveto(507.5994053,55.00437988)(507.63440526,54.86438002)(507.67441284,54.72438477)
\curveto(507.68440521,54.65438023)(507.68940521,54.5843803)(507.68941284,54.51438477)
\lineto(507.71941284,54.30438477)
\moveto(506.26441284,54.81438477)
\curveto(506.2944066,54.85438003)(506.31940658,54.90437998)(506.33941284,54.96438477)
\curveto(506.35940654,55.03437985)(506.35940654,55.10437978)(506.33941284,55.17438477)
\curveto(506.27940662,55.39437949)(506.1944067,55.59937929)(506.08441284,55.78938477)
\curveto(505.94440695,56.01937887)(505.78940711,56.21437867)(505.61941284,56.37438477)
\curveto(505.44940745,56.53437835)(505.22940767,56.66937822)(504.95941284,56.77938477)
\curveto(504.88940801,56.79937809)(504.81940808,56.81437807)(504.74941284,56.82438477)
\curveto(504.67940822,56.84437804)(504.60440829,56.86437802)(504.52441284,56.88438477)
\curveto(504.44440845,56.90437798)(504.35940854,56.91437797)(504.26941284,56.91438477)
\lineto(504.01441284,56.91438477)
\curveto(503.98440891,56.89437799)(503.94940895,56.884378)(503.90941284,56.88438477)
\curveto(503.86940903,56.89437799)(503.83440906,56.89437799)(503.80441284,56.88438477)
\lineto(503.56441284,56.82438477)
\curveto(503.4944094,56.81437807)(503.42440947,56.79937809)(503.35441284,56.77938477)
\curveto(503.06440983,56.65937823)(502.82941007,56.50937838)(502.64941284,56.32938477)
\curveto(502.47941042,56.14937874)(502.32441057,55.92437896)(502.18441284,55.65438477)
\curveto(502.15441074,55.60437928)(502.12441077,55.53937935)(502.09441284,55.45938477)
\curveto(502.06441083,55.3893795)(502.03941086,55.30937958)(502.01941284,55.21938477)
\curveto(501.9994109,55.12937976)(501.9944109,55.04437984)(502.00441284,54.96438477)
\curveto(502.01441088,54.88438)(502.04941085,54.82438006)(502.10941284,54.78438477)
\curveto(502.18941071,54.72438016)(502.32441057,54.69438019)(502.51441284,54.69438477)
\curveto(502.71441018,54.70438018)(502.88441001,54.70938018)(503.02441284,54.70938477)
\lineto(505.30441284,54.70938477)
\curveto(505.45440744,54.70938018)(505.63440726,54.70438018)(505.84441284,54.69438477)
\curveto(506.05440684,54.69438019)(506.1944067,54.73438015)(506.26441284,54.81438477)
}
}
{
\newrgbcolor{curcolor}{0 0 0}
\pscustom[linestyle=none,fillstyle=solid,fillcolor=curcolor]
{
\newpath
\moveto(628.93253418,55.74439941)
\lineto(628.93253418,55.47439941)
\curveto(628.94252421,55.38439416)(628.93752421,55.30439424)(628.91753418,55.23439941)
\lineto(628.91753418,55.08439941)
\curveto(628.90752424,55.05439449)(628.90252425,55.01939453)(628.90253418,54.97939941)
\curveto(628.91252424,54.93939461)(628.91252424,54.90939464)(628.90253418,54.88939941)
\curveto(628.89252426,54.83939471)(628.88752426,54.78439476)(628.88753418,54.72439941)
\curveto(628.88752426,54.67439487)(628.88252427,54.62439492)(628.87253418,54.57439941)
\curveto(628.84252431,54.43439511)(628.82252433,54.28439526)(628.81253418,54.12439941)
\curveto(628.80252435,53.97439557)(628.77252438,53.82939572)(628.72253418,53.68939941)
\curveto(628.69252446,53.56939598)(628.65752449,53.4443961)(628.61753418,53.31439941)
\curveto(628.58752456,53.19439635)(628.5475246,53.07439647)(628.49753418,52.95439941)
\curveto(628.32752482,52.52439702)(628.11252504,52.13439741)(627.85253418,51.78439941)
\curveto(627.60252555,51.4443981)(627.28752586,51.15439839)(626.90753418,50.91439941)
\curveto(626.71752643,50.79439875)(626.51252664,50.68939886)(626.29253418,50.59939941)
\curveto(626.08252707,50.51939903)(625.8525273,50.43939911)(625.60253418,50.35939941)
\curveto(625.49252766,50.31939923)(625.37252778,50.28939926)(625.24253418,50.26939941)
\curveto(625.12252803,50.25939929)(625.00252815,50.23939931)(624.88253418,50.20939941)
\curveto(624.77252838,50.18939936)(624.66252849,50.17939937)(624.55253418,50.17939941)
\curveto(624.4525287,50.17939937)(624.3525288,50.16939938)(624.25253418,50.14939941)
\lineto(624.04253418,50.14939941)
\curveto(624.01252914,50.13939941)(623.97752917,50.13439941)(623.93753418,50.13439941)
\curveto(623.89752925,50.1443994)(623.85752929,50.1493994)(623.81753418,50.14939941)
\lineto(620.81753418,50.14939941)
\curveto(620.66753248,50.1493994)(620.53253262,50.15439939)(620.41253418,50.16439941)
\curveto(620.30253285,50.18439936)(620.22753292,50.2493993)(620.18753418,50.35939941)
\curveto(620.147533,50.43939911)(620.12753302,50.55439899)(620.12753418,50.70439941)
\curveto(620.13753301,50.85439869)(620.14253301,50.98939856)(620.14253418,51.10939941)
\lineto(620.14253418,59.97439941)
\curveto(620.14253301,60.09438945)(620.13753301,60.21938933)(620.12753418,60.34939941)
\curveto(620.12753302,60.48938906)(620.152533,60.59938895)(620.20253418,60.67939941)
\curveto(620.24253291,60.7493888)(620.31753283,60.79438875)(620.42753418,60.81439941)
\curveto(620.4475327,60.82438872)(620.46753268,60.82438872)(620.48753418,60.81439941)
\curveto(620.50753264,60.81438873)(620.52753262,60.81938873)(620.54753418,60.82939941)
\lineto(623.80253418,60.82939941)
\curveto(623.8525293,60.82938872)(623.89752925,60.82938872)(623.93753418,60.82939941)
\curveto(623.98752916,60.83938871)(624.03252912,60.83938871)(624.07253418,60.82939941)
\curveto(624.12252903,60.80938874)(624.17252898,60.80438874)(624.22253418,60.81439941)
\curveto(624.28252887,60.82438872)(624.33752881,60.82438872)(624.38753418,60.81439941)
\curveto(624.43752871,60.80438874)(624.49252866,60.79938875)(624.55253418,60.79939941)
\curveto(624.61252854,60.79938875)(624.66752848,60.79438875)(624.71753418,60.78439941)
\curveto(624.76752838,60.77438877)(624.81252834,60.76938878)(624.85253418,60.76939941)
\curveto(624.90252825,60.76938878)(624.9525282,60.76438878)(625.00253418,60.75439941)
\curveto(625.11252804,60.73438881)(625.21752793,60.71438883)(625.31753418,60.69439941)
\curveto(625.41752773,60.68438886)(625.51752763,60.66438888)(625.61753418,60.63439941)
\curveto(625.83752731,60.56438898)(626.0475271,60.49438905)(626.24753418,60.42439941)
\curveto(626.4475267,60.36438918)(626.63252652,60.27938927)(626.80253418,60.16939941)
\curveto(626.94252621,60.08938946)(627.06752608,60.00938954)(627.17753418,59.92939941)
\curveto(627.20752594,59.90938964)(627.23752591,59.88438966)(627.26753418,59.85439941)
\curveto(627.29752585,59.83438971)(627.32752582,59.81438973)(627.35753418,59.79439941)
\curveto(627.41752573,59.7443898)(627.47252568,59.69438985)(627.52253418,59.64439941)
\curveto(627.57252558,59.59438995)(627.62252553,59.54439)(627.67253418,59.49439941)
\curveto(627.72252543,59.4443901)(627.76252539,59.40939014)(627.79253418,59.38939941)
\curveto(627.83252532,59.32939022)(627.87252528,59.27439027)(627.91253418,59.22439941)
\curveto(627.96252519,59.17439037)(628.00752514,59.11939043)(628.04753418,59.05939941)
\curveto(628.09752505,58.99939055)(628.13752501,58.93439061)(628.16753418,58.86439941)
\curveto(628.20752494,58.80439074)(628.2525249,58.73939081)(628.30253418,58.66939941)
\curveto(628.32252483,58.62939092)(628.33752481,58.59439095)(628.34753418,58.56439941)
\curveto(628.35752479,58.53439101)(628.37252478,58.49939105)(628.39253418,58.45939941)
\curveto(628.43252472,58.37939117)(628.46752468,58.29939125)(628.49753418,58.21939941)
\curveto(628.52752462,58.1493914)(628.56252459,58.07439147)(628.60253418,57.99439941)
\curveto(628.64252451,57.88439166)(628.67252448,57.76939178)(628.69253418,57.64939941)
\curveto(628.72252443,57.53939201)(628.7525244,57.42939212)(628.78253418,57.31939941)
\curveto(628.80252435,57.25939229)(628.81252434,57.19939235)(628.81253418,57.13939941)
\curveto(628.81252434,57.08939246)(628.82252433,57.03439251)(628.84253418,56.97439941)
\curveto(628.89252426,56.79439275)(628.91752423,56.59439295)(628.91753418,56.37439941)
\curveto(628.92752422,56.16439338)(628.93252422,55.95439359)(628.93253418,55.74439941)
\moveto(627.50753418,54.96439941)
\curveto(627.52752562,55.06439448)(627.53752561,55.16939438)(627.53753418,55.27939941)
\lineto(627.53753418,55.62439941)
\lineto(627.53753418,55.84939941)
\curveto(627.5475256,55.92939362)(627.54252561,56.00439354)(627.52253418,56.07439941)
\curveto(627.52252563,56.10439344)(627.51752563,56.13439341)(627.50753418,56.16439941)
\lineto(627.50753418,56.26939941)
\curveto(627.48752566,56.37939317)(627.47252568,56.48939306)(627.46253418,56.59939941)
\curveto(627.46252569,56.70939284)(627.4475257,56.81939273)(627.41753418,56.92939941)
\curveto(627.39752575,57.00939254)(627.37752577,57.08439246)(627.35753418,57.15439941)
\curveto(627.3475258,57.23439231)(627.33252582,57.31439223)(627.31253418,57.39439941)
\curveto(627.20252595,57.75439179)(627.06252609,58.06939148)(626.89253418,58.33939941)
\curveto(626.61252654,58.78939076)(626.19752695,59.12939042)(625.64753418,59.35939941)
\curveto(625.55752759,59.40939014)(625.46252769,59.4443901)(625.36253418,59.46439941)
\curveto(625.26252789,59.49439005)(625.15752799,59.52439002)(625.04753418,59.55439941)
\curveto(624.93752821,59.58438996)(624.82252833,59.59938995)(624.70253418,59.59939941)
\curveto(624.59252856,59.60938994)(624.48252867,59.62438992)(624.37253418,59.64439941)
\lineto(624.05753418,59.64439941)
\curveto(624.02752912,59.65438989)(623.99252916,59.65938989)(623.95253418,59.65939941)
\lineto(623.83253418,59.65939941)
\lineto(622.00253418,59.65939941)
\curveto(621.98253117,59.6493899)(621.95753119,59.6443899)(621.92753418,59.64439941)
\curveto(621.89753125,59.65438989)(621.87253128,59.65438989)(621.85253418,59.64439941)
\lineto(621.70253418,59.58439941)
\curveto(621.66253149,59.56438998)(621.63253152,59.53439001)(621.61253418,59.49439941)
\curveto(621.59253156,59.45439009)(621.57253158,59.38439016)(621.55253418,59.28439941)
\lineto(621.55253418,59.16439941)
\curveto(621.54253161,59.12439042)(621.53753161,59.07939047)(621.53753418,59.02939941)
\lineto(621.53753418,58.89439941)
\lineto(621.53753418,52.08439941)
\lineto(621.53753418,51.93439941)
\curveto(621.53753161,51.89439765)(621.54253161,51.85439769)(621.55253418,51.81439941)
\lineto(621.55253418,51.69439941)
\curveto(621.57253158,51.59439795)(621.59253156,51.52439802)(621.61253418,51.48439941)
\curveto(621.69253146,51.36439818)(621.84253131,51.30439824)(622.06253418,51.30439941)
\curveto(622.28253087,51.31439823)(622.49253066,51.31939823)(622.69253418,51.31939941)
\lineto(623.56253418,51.31939941)
\curveto(623.63252952,51.31939823)(623.70752944,51.31439823)(623.78753418,51.30439941)
\curveto(623.86752928,51.30439824)(623.93752921,51.31439823)(623.99753418,51.33439941)
\lineto(624.16253418,51.33439941)
\curveto(624.21252894,51.3443982)(624.26752888,51.3443982)(624.32753418,51.33439941)
\curveto(624.38752876,51.33439821)(624.4475287,51.33939821)(624.50753418,51.34939941)
\curveto(624.56752858,51.36939818)(624.62752852,51.37939817)(624.68753418,51.37939941)
\curveto(624.7475284,51.38939816)(624.81252834,51.40439814)(624.88253418,51.42439941)
\curveto(624.99252816,51.45439809)(625.09752805,51.48439806)(625.19753418,51.51439941)
\curveto(625.30752784,51.544398)(625.41752773,51.58439796)(625.52753418,51.63439941)
\curveto(625.89752725,51.79439775)(626.21252694,52.00939754)(626.47253418,52.27939941)
\curveto(626.74252641,52.55939699)(626.96252619,52.88939666)(627.13253418,53.26939941)
\curveto(627.18252597,53.37939617)(627.22252593,53.49439605)(627.25253418,53.61439941)
\lineto(627.37253418,54.00439941)
\curveto(627.40252575,54.11439543)(627.42252573,54.22939532)(627.43253418,54.34939941)
\curveto(627.4525257,54.47939507)(627.47252568,54.60439494)(627.49253418,54.72439941)
\curveto(627.50252565,54.77439477)(627.50752564,54.81439473)(627.50753418,54.84439941)
\lineto(627.50753418,54.96439941)
}
}
{
\newrgbcolor{curcolor}{0 0 0}
\pscustom[linestyle=none,fillstyle=solid,fillcolor=curcolor]
{
\newpath
\moveto(637.18440918,54.31939941)
\curveto(637.20440149,54.21939533)(637.20440149,54.10439544)(637.18440918,53.97439941)
\curveto(637.17440152,53.85439569)(637.14440155,53.76939578)(637.09440918,53.71939941)
\curveto(637.04440165,53.67939587)(636.96940173,53.6493959)(636.86940918,53.62939941)
\curveto(636.77940192,53.61939593)(636.67440202,53.61439593)(636.55440918,53.61439941)
\lineto(636.19440918,53.61439941)
\curveto(636.07440262,53.62439592)(635.96940273,53.62939592)(635.87940918,53.62939941)
\lineto(632.03940918,53.62939941)
\curveto(631.95940674,53.62939592)(631.87940682,53.62439592)(631.79940918,53.61439941)
\curveto(631.71940698,53.61439593)(631.65440704,53.59939595)(631.60440918,53.56939941)
\curveto(631.56440713,53.549396)(631.52440717,53.50939604)(631.48440918,53.44939941)
\curveto(631.46440723,53.41939613)(631.44440725,53.37439617)(631.42440918,53.31439941)
\curveto(631.40440729,53.26439628)(631.40440729,53.21439633)(631.42440918,53.16439941)
\curveto(631.43440726,53.11439643)(631.43940726,53.06939648)(631.43940918,53.02939941)
\curveto(631.43940726,52.98939656)(631.44440725,52.9493966)(631.45440918,52.90939941)
\curveto(631.47440722,52.82939672)(631.4944072,52.7443968)(631.51440918,52.65439941)
\curveto(631.53440716,52.57439697)(631.56440713,52.49439705)(631.60440918,52.41439941)
\curveto(631.83440686,51.87439767)(632.21440648,51.48939806)(632.74440918,51.25939941)
\curveto(632.80440589,51.22939832)(632.86940583,51.20439834)(632.93940918,51.18439941)
\lineto(633.14940918,51.12439941)
\curveto(633.17940552,51.11439843)(633.22940547,51.10939844)(633.29940918,51.10939941)
\curveto(633.43940526,51.06939848)(633.62440507,51.0493985)(633.85440918,51.04939941)
\curveto(634.08440461,51.0493985)(634.26940443,51.06939848)(634.40940918,51.10939941)
\curveto(634.54940415,51.1493984)(634.67440402,51.18939836)(634.78440918,51.22939941)
\curveto(634.90440379,51.27939827)(635.01440368,51.33939821)(635.11440918,51.40939941)
\curveto(635.22440347,51.47939807)(635.31940338,51.55939799)(635.39940918,51.64939941)
\curveto(635.47940322,51.7493978)(635.54940315,51.85439769)(635.60940918,51.96439941)
\curveto(635.66940303,52.06439748)(635.71940298,52.16939738)(635.75940918,52.27939941)
\curveto(635.80940289,52.38939716)(635.88940281,52.46939708)(635.99940918,52.51939941)
\curveto(636.03940266,52.53939701)(636.10440259,52.55439699)(636.19440918,52.56439941)
\curveto(636.28440241,52.57439697)(636.37440232,52.57439697)(636.46440918,52.56439941)
\curveto(636.55440214,52.56439698)(636.63940206,52.55939699)(636.71940918,52.54939941)
\curveto(636.7994019,52.53939701)(636.85440184,52.51939703)(636.88440918,52.48939941)
\curveto(636.98440171,52.41939713)(637.00940169,52.30439724)(636.95940918,52.14439941)
\curveto(636.87940182,51.87439767)(636.77440192,51.63439791)(636.64440918,51.42439941)
\curveto(636.44440225,51.10439844)(636.21440248,50.83939871)(635.95440918,50.62939941)
\curveto(635.70440299,50.42939912)(635.38440331,50.26439928)(634.99440918,50.13439941)
\curveto(634.8944038,50.09439945)(634.7944039,50.06939948)(634.69440918,50.05939941)
\curveto(634.5944041,50.03939951)(634.48940421,50.01939953)(634.37940918,49.99939941)
\curveto(634.32940437,49.98939956)(634.27940442,49.98439956)(634.22940918,49.98439941)
\curveto(634.18940451,49.98439956)(634.14440455,49.97939957)(634.09440918,49.96939941)
\lineto(633.94440918,49.96939941)
\curveto(633.8944048,49.95939959)(633.83440486,49.95439959)(633.76440918,49.95439941)
\curveto(633.70440499,49.95439959)(633.65440504,49.95939959)(633.61440918,49.96939941)
\lineto(633.47940918,49.96939941)
\curveto(633.42940527,49.97939957)(633.38440531,49.98439956)(633.34440918,49.98439941)
\curveto(633.30440539,49.98439956)(633.26440543,49.98939956)(633.22440918,49.99939941)
\curveto(633.17440552,50.00939954)(633.11940558,50.01939953)(633.05940918,50.02939941)
\curveto(632.9994057,50.02939952)(632.94440575,50.03439951)(632.89440918,50.04439941)
\curveto(632.80440589,50.06439948)(632.71440598,50.08939946)(632.62440918,50.11939941)
\curveto(632.53440616,50.13939941)(632.44940625,50.16439938)(632.36940918,50.19439941)
\curveto(632.32940637,50.21439933)(632.2944064,50.22439932)(632.26440918,50.22439941)
\curveto(632.23440646,50.23439931)(632.1994065,50.2493993)(632.15940918,50.26939941)
\curveto(632.00940669,50.33939921)(631.84940685,50.42439912)(631.67940918,50.52439941)
\curveto(631.38940731,50.71439883)(631.13940756,50.9443986)(630.92940918,51.21439941)
\curveto(630.72940797,51.49439805)(630.55940814,51.80439774)(630.41940918,52.14439941)
\curveto(630.36940833,52.25439729)(630.32940837,52.36939718)(630.29940918,52.48939941)
\curveto(630.27940842,52.60939694)(630.24940845,52.72939682)(630.20940918,52.84939941)
\curveto(630.1994085,52.88939666)(630.1944085,52.92439662)(630.19440918,52.95439941)
\curveto(630.1944085,52.98439656)(630.18940851,53.02439652)(630.17940918,53.07439941)
\curveto(630.15940854,53.15439639)(630.14440855,53.23939631)(630.13440918,53.32939941)
\curveto(630.12440857,53.41939613)(630.10940859,53.50939604)(630.08940918,53.59939941)
\lineto(630.08940918,53.80939941)
\curveto(630.07940862,53.8493957)(630.06940863,53.90439564)(630.05940918,53.97439941)
\curveto(630.05940864,54.05439549)(630.06440863,54.11939543)(630.07440918,54.16939941)
\lineto(630.07440918,54.33439941)
\curveto(630.0944086,54.38439516)(630.0994086,54.43439511)(630.08940918,54.48439941)
\curveto(630.08940861,54.544395)(630.0944086,54.59939495)(630.10440918,54.64939941)
\curveto(630.14440855,54.80939474)(630.17440852,54.96939458)(630.19440918,55.12939941)
\curveto(630.22440847,55.28939426)(630.26940843,55.43939411)(630.32940918,55.57939941)
\curveto(630.37940832,55.68939386)(630.42440827,55.79939375)(630.46440918,55.90939941)
\curveto(630.51440818,56.02939352)(630.56940813,56.1443934)(630.62940918,56.25439941)
\curveto(630.84940785,56.60439294)(631.0994076,56.90439264)(631.37940918,57.15439941)
\curveto(631.65940704,57.41439213)(632.00440669,57.62939192)(632.41440918,57.79939941)
\curveto(632.53440616,57.8493917)(632.65440604,57.88439166)(632.77440918,57.90439941)
\curveto(632.90440579,57.93439161)(633.03940566,57.96439158)(633.17940918,57.99439941)
\curveto(633.22940547,58.00439154)(633.27440542,58.00939154)(633.31440918,58.00939941)
\curveto(633.35440534,58.01939153)(633.3994053,58.02439152)(633.44940918,58.02439941)
\curveto(633.46940523,58.03439151)(633.4944052,58.03439151)(633.52440918,58.02439941)
\curveto(633.55440514,58.01439153)(633.57940512,58.01939153)(633.59940918,58.03939941)
\curveto(634.01940468,58.0493915)(634.38440431,58.00439154)(634.69440918,57.90439941)
\curveto(635.00440369,57.81439173)(635.28440341,57.68939186)(635.53440918,57.52939941)
\curveto(635.58440311,57.50939204)(635.62440307,57.47939207)(635.65440918,57.43939941)
\curveto(635.68440301,57.40939214)(635.71940298,57.38439216)(635.75940918,57.36439941)
\curveto(635.83940286,57.30439224)(635.91940278,57.23439231)(635.99940918,57.15439941)
\curveto(636.08940261,57.07439247)(636.16440253,56.99439255)(636.22440918,56.91439941)
\curveto(636.38440231,56.70439284)(636.51940218,56.50439304)(636.62940918,56.31439941)
\curveto(636.699402,56.20439334)(636.75440194,56.08439346)(636.79440918,55.95439941)
\curveto(636.83440186,55.82439372)(636.87940182,55.69439385)(636.92940918,55.56439941)
\curveto(636.97940172,55.43439411)(637.01440168,55.29939425)(637.03440918,55.15939941)
\curveto(637.06440163,55.01939453)(637.0994016,54.87939467)(637.13940918,54.73939941)
\curveto(637.14940155,54.66939488)(637.15440154,54.59939495)(637.15440918,54.52939941)
\lineto(637.18440918,54.31939941)
\moveto(635.72940918,54.82939941)
\curveto(635.75940294,54.86939468)(635.78440291,54.91939463)(635.80440918,54.97939941)
\curveto(635.82440287,55.0493945)(635.82440287,55.11939443)(635.80440918,55.18939941)
\curveto(635.74440295,55.40939414)(635.65940304,55.61439393)(635.54940918,55.80439941)
\curveto(635.40940329,56.03439351)(635.25440344,56.22939332)(635.08440918,56.38939941)
\curveto(634.91440378,56.549393)(634.694404,56.68439286)(634.42440918,56.79439941)
\curveto(634.35440434,56.81439273)(634.28440441,56.82939272)(634.21440918,56.83939941)
\curveto(634.14440455,56.85939269)(634.06940463,56.87939267)(633.98940918,56.89939941)
\curveto(633.90940479,56.91939263)(633.82440487,56.92939262)(633.73440918,56.92939941)
\lineto(633.47940918,56.92939941)
\curveto(633.44940525,56.90939264)(633.41440528,56.89939265)(633.37440918,56.89939941)
\curveto(633.33440536,56.90939264)(633.2994054,56.90939264)(633.26940918,56.89939941)
\lineto(633.02940918,56.83939941)
\curveto(632.95940574,56.82939272)(632.88940581,56.81439273)(632.81940918,56.79439941)
\curveto(632.52940617,56.67439287)(632.2944064,56.52439302)(632.11440918,56.34439941)
\curveto(631.94440675,56.16439338)(631.78940691,55.93939361)(631.64940918,55.66939941)
\curveto(631.61940708,55.61939393)(631.58940711,55.55439399)(631.55940918,55.47439941)
\curveto(631.52940717,55.40439414)(631.50440719,55.32439422)(631.48440918,55.23439941)
\curveto(631.46440723,55.1443944)(631.45940724,55.05939449)(631.46940918,54.97939941)
\curveto(631.47940722,54.89939465)(631.51440718,54.83939471)(631.57440918,54.79939941)
\curveto(631.65440704,54.73939481)(631.78940691,54.70939484)(631.97940918,54.70939941)
\curveto(632.17940652,54.71939483)(632.34940635,54.72439482)(632.48940918,54.72439941)
\lineto(634.76940918,54.72439941)
\curveto(634.91940378,54.72439482)(635.0994036,54.71939483)(635.30940918,54.70939941)
\curveto(635.51940318,54.70939484)(635.65940304,54.7493948)(635.72940918,54.82939941)
}
}
{
\newrgbcolor{curcolor}{0 0 0}
\pscustom[linestyle=none,fillstyle=solid,fillcolor=curcolor]
{
\newpath
\moveto(640.9210498,58.05439941)
\curveto(641.64104574,58.06439148)(642.24604513,57.97939157)(642.7360498,57.79939941)
\curveto(643.22604415,57.62939192)(643.60604377,57.32439222)(643.8760498,56.88439941)
\curveto(643.94604343,56.77439277)(644.00104338,56.65939289)(644.0410498,56.53939941)
\curveto(644.0810433,56.42939312)(644.12104326,56.30439324)(644.1610498,56.16439941)
\curveto(644.1810432,56.09439345)(644.18604319,56.01939353)(644.1760498,55.93939941)
\curveto(644.16604321,55.86939368)(644.15104323,55.81439373)(644.1310498,55.77439941)
\curveto(644.11104327,55.75439379)(644.08604329,55.73439381)(644.0560498,55.71439941)
\curveto(644.02604335,55.70439384)(644.00104338,55.68939386)(643.9810498,55.66939941)
\curveto(643.93104345,55.6493939)(643.8810435,55.6443939)(643.8310498,55.65439941)
\curveto(643.7810436,55.66439388)(643.73104365,55.66439388)(643.6810498,55.65439941)
\curveto(643.60104378,55.63439391)(643.49604388,55.62939392)(643.3660498,55.63939941)
\curveto(643.23604414,55.65939389)(643.14604423,55.68439386)(643.0960498,55.71439941)
\curveto(643.01604436,55.76439378)(642.96104442,55.82939372)(642.9310498,55.90939941)
\curveto(642.91104447,55.99939355)(642.8760445,56.08439346)(642.8260498,56.16439941)
\curveto(642.73604464,56.32439322)(642.61104477,56.46939308)(642.4510498,56.59939941)
\curveto(642.34104504,56.67939287)(642.22104516,56.73939281)(642.0910498,56.77939941)
\curveto(641.96104542,56.81939273)(641.82104556,56.85939269)(641.6710498,56.89939941)
\curveto(641.62104576,56.91939263)(641.57104581,56.92439262)(641.5210498,56.91439941)
\curveto(641.47104591,56.91439263)(641.42104596,56.91939263)(641.3710498,56.92939941)
\curveto(641.31104607,56.9493926)(641.23604614,56.95939259)(641.1460498,56.95939941)
\curveto(641.05604632,56.95939259)(640.9810464,56.9493926)(640.9210498,56.92939941)
\lineto(640.8310498,56.92939941)
\lineto(640.6810498,56.89939941)
\curveto(640.63104675,56.89939265)(640.5810468,56.89439265)(640.5310498,56.88439941)
\curveto(640.27104711,56.82439272)(640.05604732,56.73939281)(639.8860498,56.62939941)
\curveto(639.71604766,56.51939303)(639.60104778,56.33439321)(639.5410498,56.07439941)
\curveto(639.52104786,56.00439354)(639.51604786,55.93439361)(639.5260498,55.86439941)
\curveto(639.54604783,55.79439375)(639.56604781,55.73439381)(639.5860498,55.68439941)
\curveto(639.64604773,55.53439401)(639.71604766,55.42439412)(639.7960498,55.35439941)
\curveto(639.88604749,55.29439425)(639.99604738,55.22439432)(640.1260498,55.14439941)
\curveto(640.28604709,55.0443945)(640.46604691,54.96939458)(640.6660498,54.91939941)
\curveto(640.86604651,54.87939467)(641.06604631,54.82939472)(641.2660498,54.76939941)
\curveto(641.39604598,54.72939482)(641.52604585,54.69939485)(641.6560498,54.67939941)
\curveto(641.78604559,54.65939489)(641.91604546,54.62939492)(642.0460498,54.58939941)
\curveto(642.25604512,54.52939502)(642.46104492,54.46939508)(642.6610498,54.40939941)
\curveto(642.86104452,54.35939519)(643.06104432,54.29439525)(643.2610498,54.21439941)
\lineto(643.4110498,54.15439941)
\curveto(643.46104392,54.13439541)(643.51104387,54.10939544)(643.5610498,54.07939941)
\curveto(643.76104362,53.95939559)(643.93604344,53.82439572)(644.0860498,53.67439941)
\curveto(644.23604314,53.52439602)(644.36104302,53.33439621)(644.4610498,53.10439941)
\curveto(644.4810429,53.03439651)(644.50104288,52.93939661)(644.5210498,52.81939941)
\curveto(644.54104284,52.7493968)(644.55104283,52.67439687)(644.5510498,52.59439941)
\curveto(644.56104282,52.52439702)(644.56604281,52.4443971)(644.5660498,52.35439941)
\lineto(644.5660498,52.20439941)
\curveto(644.54604283,52.13439741)(644.53604284,52.06439748)(644.5360498,51.99439941)
\curveto(644.53604284,51.92439762)(644.52604285,51.85439769)(644.5060498,51.78439941)
\curveto(644.4760429,51.67439787)(644.44104294,51.56939798)(644.4010498,51.46939941)
\curveto(644.36104302,51.36939818)(644.31604306,51.27939827)(644.2660498,51.19939941)
\curveto(644.10604327,50.93939861)(643.90104348,50.72939882)(643.6510498,50.56939941)
\curveto(643.40104398,50.41939913)(643.12104426,50.28939926)(642.8110498,50.17939941)
\curveto(642.72104466,50.1493994)(642.62604475,50.12939942)(642.5260498,50.11939941)
\curveto(642.43604494,50.09939945)(642.34604503,50.07439947)(642.2560498,50.04439941)
\curveto(642.15604522,50.02439952)(642.05604532,50.01439953)(641.9560498,50.01439941)
\curveto(641.85604552,50.01439953)(641.75604562,50.00439954)(641.6560498,49.98439941)
\lineto(641.5060498,49.98439941)
\curveto(641.45604592,49.97439957)(641.38604599,49.96939958)(641.2960498,49.96939941)
\curveto(641.20604617,49.96939958)(641.13604624,49.97439957)(641.0860498,49.98439941)
\lineto(640.9210498,49.98439941)
\curveto(640.86104652,50.00439954)(640.79604658,50.01439953)(640.7260498,50.01439941)
\curveto(640.65604672,50.00439954)(640.59604678,50.00939954)(640.5460498,50.02939941)
\curveto(640.49604688,50.03939951)(640.43104695,50.0443995)(640.3510498,50.04439941)
\lineto(640.1110498,50.10439941)
\curveto(640.04104734,50.11439943)(639.96604741,50.13439941)(639.8860498,50.16439941)
\curveto(639.5760478,50.26439928)(639.30604807,50.38939916)(639.0760498,50.53939941)
\curveto(638.84604853,50.68939886)(638.64604873,50.88439866)(638.4760498,51.12439941)
\curveto(638.38604899,51.25439829)(638.31104907,51.38939816)(638.2510498,51.52939941)
\curveto(638.19104919,51.66939788)(638.13604924,51.82439772)(638.0860498,51.99439941)
\curveto(638.06604931,52.05439749)(638.05604932,52.12439742)(638.0560498,52.20439941)
\curveto(638.06604931,52.29439725)(638.0810493,52.36439718)(638.1010498,52.41439941)
\curveto(638.13104925,52.45439709)(638.1810492,52.49439705)(638.2510498,52.53439941)
\curveto(638.30104908,52.55439699)(638.37104901,52.56439698)(638.4610498,52.56439941)
\curveto(638.55104883,52.57439697)(638.64104874,52.57439697)(638.7310498,52.56439941)
\curveto(638.82104856,52.55439699)(638.90604847,52.53939701)(638.9860498,52.51939941)
\curveto(639.0760483,52.50939704)(639.13604824,52.49439705)(639.1660498,52.47439941)
\curveto(639.23604814,52.42439712)(639.2810481,52.3493972)(639.3010498,52.24939941)
\curveto(639.33104805,52.15939739)(639.36604801,52.07439747)(639.4060498,51.99439941)
\curveto(639.50604787,51.77439777)(639.64104774,51.60439794)(639.8110498,51.48439941)
\curveto(639.93104745,51.39439815)(640.06604731,51.32439822)(640.2160498,51.27439941)
\curveto(640.36604701,51.22439832)(640.52604685,51.17439837)(640.6960498,51.12439941)
\lineto(641.0110498,51.07939941)
\lineto(641.1010498,51.07939941)
\curveto(641.17104621,51.05939849)(641.26104612,51.0493985)(641.3710498,51.04939941)
\curveto(641.49104589,51.0493985)(641.59104579,51.05939849)(641.6710498,51.07939941)
\curveto(641.74104564,51.07939847)(641.79604558,51.08439846)(641.8360498,51.09439941)
\curveto(641.89604548,51.10439844)(641.95604542,51.10939844)(642.0160498,51.10939941)
\curveto(642.0760453,51.11939843)(642.13104525,51.12939842)(642.1810498,51.13939941)
\curveto(642.47104491,51.21939833)(642.70104468,51.32439822)(642.8710498,51.45439941)
\curveto(643.04104434,51.58439796)(643.16104422,51.80439774)(643.2310498,52.11439941)
\curveto(643.25104413,52.16439738)(643.25604412,52.21939733)(643.2460498,52.27939941)
\curveto(643.23604414,52.33939721)(643.22604415,52.38439716)(643.2160498,52.41439941)
\curveto(643.16604421,52.60439694)(643.09604428,52.7443968)(643.0060498,52.83439941)
\curveto(642.91604446,52.93439661)(642.80104458,53.02439652)(642.6610498,53.10439941)
\curveto(642.57104481,53.16439638)(642.47104491,53.21439633)(642.3610498,53.25439941)
\lineto(642.0310498,53.37439941)
\curveto(642.00104538,53.38439616)(641.97104541,53.38939616)(641.9410498,53.38939941)
\curveto(641.92104546,53.38939616)(641.89604548,53.39939615)(641.8660498,53.41939941)
\curveto(641.52604585,53.52939602)(641.17104621,53.60939594)(640.8010498,53.65939941)
\curveto(640.44104694,53.71939583)(640.10104728,53.81439573)(639.7810498,53.94439941)
\curveto(639.6810477,53.98439556)(639.58604779,54.01939553)(639.4960498,54.04939941)
\curveto(639.40604797,54.07939547)(639.32104806,54.11939543)(639.2410498,54.16939941)
\curveto(639.05104833,54.27939527)(638.8760485,54.40439514)(638.7160498,54.54439941)
\curveto(638.55604882,54.68439486)(638.43104895,54.85939469)(638.3410498,55.06939941)
\curveto(638.31104907,55.13939441)(638.28604909,55.20939434)(638.2660498,55.27939941)
\curveto(638.25604912,55.3493942)(638.24104914,55.42439412)(638.2210498,55.50439941)
\curveto(638.19104919,55.62439392)(638.1810492,55.75939379)(638.1910498,55.90939941)
\curveto(638.20104918,56.06939348)(638.21604916,56.20439334)(638.2360498,56.31439941)
\curveto(638.25604912,56.36439318)(638.26604911,56.40439314)(638.2660498,56.43439941)
\curveto(638.2760491,56.47439307)(638.29104909,56.51439303)(638.3110498,56.55439941)
\curveto(638.40104898,56.78439276)(638.52104886,56.98439256)(638.6710498,57.15439941)
\curveto(638.83104855,57.32439222)(639.01104837,57.47439207)(639.2110498,57.60439941)
\curveto(639.36104802,57.69439185)(639.52604785,57.76439178)(639.7060498,57.81439941)
\curveto(639.88604749,57.87439167)(640.0760473,57.92939162)(640.2760498,57.97939941)
\curveto(640.34604703,57.98939156)(640.41104697,57.99939155)(640.4710498,58.00939941)
\curveto(640.54104684,58.01939153)(640.61604676,58.02939152)(640.6960498,58.03939941)
\curveto(640.72604665,58.0493915)(640.76604661,58.0493915)(640.8160498,58.03939941)
\curveto(640.86604651,58.02939152)(640.90104648,58.03439151)(640.9210498,58.05439941)
}
}
{
\newrgbcolor{curcolor}{0 0 0}
\pscustom[linestyle=none,fillstyle=solid,fillcolor=curcolor]
{
\newpath
\moveto(652.8760498,50.70439941)
\curveto(652.90604197,50.544399)(652.89104199,50.40939914)(652.8310498,50.29939941)
\curveto(652.77104211,50.19939935)(652.69104219,50.12439942)(652.5910498,50.07439941)
\curveto(652.54104234,50.05439949)(652.48604239,50.0443995)(652.4260498,50.04439941)
\curveto(652.3760425,50.0443995)(652.32104256,50.03439951)(652.2610498,50.01439941)
\curveto(652.04104284,49.96439958)(651.82104306,49.97939957)(651.6010498,50.05939941)
\curveto(651.39104349,50.12939942)(651.24604363,50.21939933)(651.1660498,50.32939941)
\curveto(651.11604376,50.39939915)(651.07104381,50.47939907)(651.0310498,50.56939941)
\curveto(650.99104389,50.66939888)(650.94104394,50.7493988)(650.8810498,50.80939941)
\curveto(650.86104402,50.82939872)(650.83604404,50.8493987)(650.8060498,50.86939941)
\curveto(650.78604409,50.88939866)(650.75604412,50.89439865)(650.7160498,50.88439941)
\curveto(650.60604427,50.85439869)(650.50104438,50.79939875)(650.4010498,50.71939941)
\curveto(650.31104457,50.63939891)(650.22104466,50.56939898)(650.1310498,50.50939941)
\curveto(650.00104488,50.42939912)(649.86104502,50.35439919)(649.7110498,50.28439941)
\curveto(649.56104532,50.22439932)(649.40104548,50.16939938)(649.2310498,50.11939941)
\curveto(649.13104575,50.08939946)(649.02104586,50.06939948)(648.9010498,50.05939941)
\curveto(648.79104609,50.0493995)(648.6810462,50.03439951)(648.5710498,50.01439941)
\curveto(648.52104636,50.00439954)(648.4760464,49.99939955)(648.4360498,49.99939941)
\lineto(648.3310498,49.99939941)
\curveto(648.22104666,49.97939957)(648.11604676,49.97939957)(648.0160498,49.99939941)
\lineto(647.8810498,49.99939941)
\curveto(647.83104705,50.00939954)(647.7810471,50.01439953)(647.7310498,50.01439941)
\curveto(647.6810472,50.01439953)(647.63604724,50.02439952)(647.5960498,50.04439941)
\curveto(647.55604732,50.05439949)(647.52104736,50.05939949)(647.4910498,50.05939941)
\curveto(647.47104741,50.0493995)(647.44604743,50.0493995)(647.4160498,50.05939941)
\lineto(647.1760498,50.11939941)
\curveto(647.09604778,50.12939942)(647.02104786,50.1493994)(646.9510498,50.17939941)
\curveto(646.65104823,50.30939924)(646.40604847,50.45439909)(646.2160498,50.61439941)
\curveto(646.03604884,50.78439876)(645.88604899,51.01939853)(645.7660498,51.31939941)
\curveto(645.6760492,51.53939801)(645.63104925,51.80439774)(645.6310498,52.11439941)
\lineto(645.6310498,52.42939941)
\curveto(645.64104924,52.47939707)(645.64604923,52.52939702)(645.6460498,52.57939941)
\lineto(645.6760498,52.75939941)
\lineto(645.7960498,53.08939941)
\curveto(645.83604904,53.19939635)(645.88604899,53.29939625)(645.9460498,53.38939941)
\curveto(646.12604875,53.67939587)(646.37104851,53.89439565)(646.6810498,54.03439941)
\curveto(646.99104789,54.17439537)(647.33104755,54.29939525)(647.7010498,54.40939941)
\curveto(647.84104704,54.4493951)(647.98604689,54.47939507)(648.1360498,54.49939941)
\curveto(648.28604659,54.51939503)(648.43604644,54.544395)(648.5860498,54.57439941)
\curveto(648.65604622,54.59439495)(648.72104616,54.60439494)(648.7810498,54.60439941)
\curveto(648.85104603,54.60439494)(648.92604595,54.61439493)(649.0060498,54.63439941)
\curveto(649.0760458,54.65439489)(649.14604573,54.66439488)(649.2160498,54.66439941)
\curveto(649.28604559,54.67439487)(649.36104552,54.68939486)(649.4410498,54.70939941)
\curveto(649.69104519,54.76939478)(649.92604495,54.81939473)(650.1460498,54.85939941)
\curveto(650.36604451,54.90939464)(650.54104434,55.02439452)(650.6710498,55.20439941)
\curveto(650.73104415,55.28439426)(650.7810441,55.38439416)(650.8210498,55.50439941)
\curveto(650.86104402,55.63439391)(650.86104402,55.77439377)(650.8210498,55.92439941)
\curveto(650.76104412,56.16439338)(650.67104421,56.35439319)(650.5510498,56.49439941)
\curveto(650.44104444,56.63439291)(650.2810446,56.7443928)(650.0710498,56.82439941)
\curveto(649.95104493,56.87439267)(649.80604507,56.90939264)(649.6360498,56.92939941)
\curveto(649.4760454,56.9493926)(649.30604557,56.95939259)(649.1260498,56.95939941)
\curveto(648.94604593,56.95939259)(648.77104611,56.9493926)(648.6010498,56.92939941)
\curveto(648.43104645,56.90939264)(648.28604659,56.87939267)(648.1660498,56.83939941)
\curveto(647.99604688,56.77939277)(647.83104705,56.69439285)(647.6710498,56.58439941)
\curveto(647.59104729,56.52439302)(647.51604736,56.4443931)(647.4460498,56.34439941)
\curveto(647.38604749,56.25439329)(647.33104755,56.15439339)(647.2810498,56.04439941)
\curveto(647.25104763,55.96439358)(647.22104766,55.87939367)(647.1910498,55.78939941)
\curveto(647.17104771,55.69939385)(647.12604775,55.62939392)(647.0560498,55.57939941)
\curveto(647.01604786,55.549394)(646.94604793,55.52439402)(646.8460498,55.50439941)
\curveto(646.75604812,55.49439405)(646.66104822,55.48939406)(646.5610498,55.48939941)
\curveto(646.46104842,55.48939406)(646.36104852,55.49439405)(646.2610498,55.50439941)
\curveto(646.17104871,55.52439402)(646.10604877,55.549394)(646.0660498,55.57939941)
\curveto(646.02604885,55.60939394)(645.99604888,55.65939389)(645.9760498,55.72939941)
\curveto(645.95604892,55.79939375)(645.95604892,55.87439367)(645.9760498,55.95439941)
\curveto(646.00604887,56.08439346)(646.03604884,56.20439334)(646.0660498,56.31439941)
\curveto(646.10604877,56.43439311)(646.15104873,56.549393)(646.2010498,56.65939941)
\curveto(646.39104849,57.00939254)(646.63104825,57.27939227)(646.9210498,57.46939941)
\curveto(647.21104767,57.66939188)(647.57104731,57.82939172)(648.0010498,57.94939941)
\curveto(648.10104678,57.96939158)(648.20104668,57.98439156)(648.3010498,57.99439941)
\curveto(648.41104647,58.00439154)(648.52104636,58.01939153)(648.6310498,58.03939941)
\curveto(648.67104621,58.0493915)(648.73604614,58.0493915)(648.8260498,58.03939941)
\curveto(648.91604596,58.03939151)(648.97104591,58.0493915)(648.9910498,58.06939941)
\curveto(649.69104519,58.07939147)(650.30104458,57.99939155)(650.8210498,57.82939941)
\curveto(651.34104354,57.65939189)(651.70604317,57.33439221)(651.9160498,56.85439941)
\curveto(652.00604287,56.65439289)(652.05604282,56.41939313)(652.0660498,56.14939941)
\curveto(652.08604279,55.88939366)(652.09604278,55.61439393)(652.0960498,55.32439941)
\lineto(652.0960498,52.00939941)
\curveto(652.09604278,51.86939768)(652.10104278,51.73439781)(652.1110498,51.60439941)
\curveto(652.12104276,51.47439807)(652.15104273,51.36939818)(652.2010498,51.28939941)
\curveto(652.25104263,51.21939833)(652.31604256,51.16939838)(652.3960498,51.13939941)
\curveto(652.48604239,51.09939845)(652.57104231,51.06939848)(652.6510498,51.04939941)
\curveto(652.73104215,51.03939851)(652.79104209,50.99439855)(652.8310498,50.91439941)
\curveto(652.85104203,50.88439866)(652.86104202,50.85439869)(652.8610498,50.82439941)
\curveto(652.86104202,50.79439875)(652.86604201,50.75439879)(652.8760498,50.70439941)
\moveto(650.7310498,52.36939941)
\curveto(650.79104409,52.50939704)(650.82104406,52.66939688)(650.8210498,52.84939941)
\curveto(650.83104405,53.03939651)(650.83604404,53.23439631)(650.8360498,53.43439941)
\curveto(650.83604404,53.544396)(650.83104405,53.6443959)(650.8210498,53.73439941)
\curveto(650.81104407,53.82439572)(650.77104411,53.89439565)(650.7010498,53.94439941)
\curveto(650.67104421,53.96439558)(650.60104428,53.97439557)(650.4910498,53.97439941)
\curveto(650.47104441,53.95439559)(650.43604444,53.9443956)(650.3860498,53.94439941)
\curveto(650.33604454,53.9443956)(650.29104459,53.93439561)(650.2510498,53.91439941)
\curveto(650.17104471,53.89439565)(650.0810448,53.87439567)(649.9810498,53.85439941)
\lineto(649.6810498,53.79439941)
\curveto(649.65104523,53.79439575)(649.61604526,53.78939576)(649.5760498,53.77939941)
\lineto(649.4710498,53.77939941)
\curveto(649.32104556,53.73939581)(649.15604572,53.71439583)(648.9760498,53.70439941)
\curveto(648.80604607,53.70439584)(648.64604623,53.68439586)(648.4960498,53.64439941)
\curveto(648.41604646,53.62439592)(648.34104654,53.60439594)(648.2710498,53.58439941)
\curveto(648.21104667,53.57439597)(648.14104674,53.55939599)(648.0610498,53.53939941)
\curveto(647.90104698,53.48939606)(647.75104713,53.42439612)(647.6110498,53.34439941)
\curveto(647.47104741,53.27439627)(647.35104753,53.18439636)(647.2510498,53.07439941)
\curveto(647.15104773,52.96439658)(647.0760478,52.82939672)(647.0260498,52.66939941)
\curveto(646.9760479,52.51939703)(646.95604792,52.33439721)(646.9660498,52.11439941)
\curveto(646.96604791,52.01439753)(646.9810479,51.91939763)(647.0110498,51.82939941)
\curveto(647.05104783,51.7493978)(647.09604778,51.67439787)(647.1460498,51.60439941)
\curveto(647.22604765,51.49439805)(647.33104755,51.39939815)(647.4610498,51.31939941)
\curveto(647.59104729,51.2493983)(647.73104715,51.18939836)(647.8810498,51.13939941)
\curveto(647.93104695,51.12939842)(647.9810469,51.12439842)(648.0310498,51.12439941)
\curveto(648.0810468,51.12439842)(648.13104675,51.11939843)(648.1810498,51.10939941)
\curveto(648.25104663,51.08939846)(648.33604654,51.07439847)(648.4360498,51.06439941)
\curveto(648.54604633,51.06439848)(648.63604624,51.07439847)(648.7060498,51.09439941)
\curveto(648.76604611,51.11439843)(648.82604605,51.11939843)(648.8860498,51.10939941)
\curveto(648.94604593,51.10939844)(649.00604587,51.11939843)(649.0660498,51.13939941)
\curveto(649.14604573,51.15939839)(649.22104566,51.17439837)(649.2910498,51.18439941)
\curveto(649.37104551,51.19439835)(649.44604543,51.21439833)(649.5160498,51.24439941)
\curveto(649.80604507,51.36439818)(650.05104483,51.50939804)(650.2510498,51.67939941)
\curveto(650.46104442,51.8493977)(650.62104426,52.07939747)(650.7310498,52.36939941)
}
}
{
\newrgbcolor{curcolor}{0 0 0}
\pscustom[linestyle=none,fillstyle=solid,fillcolor=curcolor]
{
\newpath
\moveto(657.69269043,58.05439941)
\curveto(657.92268564,58.05439149)(658.05268551,57.99439155)(658.08269043,57.87439941)
\curveto(658.11268545,57.76439178)(658.12768543,57.59939195)(658.12769043,57.37939941)
\lineto(658.12769043,57.09439941)
\curveto(658.12768543,57.00439254)(658.10268546,56.92939262)(658.05269043,56.86939941)
\curveto(657.99268557,56.78939276)(657.90768565,56.7443928)(657.79769043,56.73439941)
\curveto(657.68768587,56.73439281)(657.57768598,56.71939283)(657.46769043,56.68939941)
\curveto(657.32768623,56.65939289)(657.19268637,56.62939292)(657.06269043,56.59939941)
\curveto(656.94268662,56.56939298)(656.82768673,56.52939302)(656.71769043,56.47939941)
\curveto(656.42768713,56.3493932)(656.19268737,56.16939338)(656.01269043,55.93939941)
\curveto(655.83268773,55.71939383)(655.67768788,55.46439408)(655.54769043,55.17439941)
\curveto(655.50768805,55.06439448)(655.47768808,54.9493946)(655.45769043,54.82939941)
\curveto(655.43768812,54.71939483)(655.41268815,54.60439494)(655.38269043,54.48439941)
\curveto(655.37268819,54.43439511)(655.36768819,54.38439516)(655.36769043,54.33439941)
\curveto(655.37768818,54.28439526)(655.37768818,54.23439531)(655.36769043,54.18439941)
\curveto(655.33768822,54.06439548)(655.32268824,53.92439562)(655.32269043,53.76439941)
\curveto(655.33268823,53.61439593)(655.33768822,53.46939608)(655.33769043,53.32939941)
\lineto(655.33769043,51.48439941)
\lineto(655.33769043,51.13939941)
\curveto(655.33768822,51.01939853)(655.33268823,50.90439864)(655.32269043,50.79439941)
\curveto(655.31268825,50.68439886)(655.30768825,50.58939896)(655.30769043,50.50939941)
\curveto(655.31768824,50.42939912)(655.29768826,50.35939919)(655.24769043,50.29939941)
\curveto(655.19768836,50.22939932)(655.11768844,50.18939936)(655.00769043,50.17939941)
\curveto(654.90768865,50.16939938)(654.79768876,50.16439938)(654.67769043,50.16439941)
\lineto(654.40769043,50.16439941)
\curveto(654.3576892,50.18439936)(654.30768925,50.19939935)(654.25769043,50.20939941)
\curveto(654.21768934,50.22939932)(654.18768937,50.25439929)(654.16769043,50.28439941)
\curveto(654.11768944,50.35439919)(654.08768947,50.43939911)(654.07769043,50.53939941)
\lineto(654.07769043,50.86939941)
\lineto(654.07769043,52.02439941)
\lineto(654.07769043,56.17939941)
\lineto(654.07769043,57.21439941)
\lineto(654.07769043,57.51439941)
\curveto(654.08768947,57.61439193)(654.11768944,57.69939185)(654.16769043,57.76939941)
\curveto(654.19768936,57.80939174)(654.24768931,57.83939171)(654.31769043,57.85939941)
\curveto(654.39768916,57.87939167)(654.48268908,57.88939166)(654.57269043,57.88939941)
\curveto(654.6626889,57.89939165)(654.75268881,57.89939165)(654.84269043,57.88939941)
\curveto(654.93268863,57.87939167)(655.00268856,57.86439168)(655.05269043,57.84439941)
\curveto(655.13268843,57.81439173)(655.18268838,57.75439179)(655.20269043,57.66439941)
\curveto(655.23268833,57.58439196)(655.24768831,57.49439205)(655.24769043,57.39439941)
\lineto(655.24769043,57.09439941)
\curveto(655.24768831,56.99439255)(655.26768829,56.90439264)(655.30769043,56.82439941)
\curveto(655.31768824,56.80439274)(655.32768823,56.78939276)(655.33769043,56.77939941)
\lineto(655.38269043,56.73439941)
\curveto(655.49268807,56.73439281)(655.58268798,56.77939277)(655.65269043,56.86939941)
\curveto(655.72268784,56.96939258)(655.78268778,57.0493925)(655.83269043,57.10939941)
\lineto(655.92269043,57.19939941)
\curveto(656.01268755,57.30939224)(656.13768742,57.42439212)(656.29769043,57.54439941)
\curveto(656.4576871,57.66439188)(656.60768695,57.75439179)(656.74769043,57.81439941)
\curveto(656.83768672,57.86439168)(656.93268663,57.89939165)(657.03269043,57.91939941)
\curveto(657.13268643,57.9493916)(657.23768632,57.97939157)(657.34769043,58.00939941)
\curveto(657.40768615,58.01939153)(657.46768609,58.02439152)(657.52769043,58.02439941)
\curveto(657.58768597,58.03439151)(657.64268592,58.0443915)(657.69269043,58.05439941)
}
}
{
\newrgbcolor{curcolor}{0 0 0}
\pscustom[linestyle=none,fillstyle=solid,fillcolor=curcolor]
{
\newpath
\moveto(662.70245605,58.05439941)
\curveto(662.93245126,58.05439149)(663.06245113,57.99439155)(663.09245605,57.87439941)
\curveto(663.12245107,57.76439178)(663.13745106,57.59939195)(663.13745605,57.37939941)
\lineto(663.13745605,57.09439941)
\curveto(663.13745106,57.00439254)(663.11245108,56.92939262)(663.06245605,56.86939941)
\curveto(663.00245119,56.78939276)(662.91745128,56.7443928)(662.80745605,56.73439941)
\curveto(662.6974515,56.73439281)(662.58745161,56.71939283)(662.47745605,56.68939941)
\curveto(662.33745186,56.65939289)(662.20245199,56.62939292)(662.07245605,56.59939941)
\curveto(661.95245224,56.56939298)(661.83745236,56.52939302)(661.72745605,56.47939941)
\curveto(661.43745276,56.3493932)(661.20245299,56.16939338)(661.02245605,55.93939941)
\curveto(660.84245335,55.71939383)(660.68745351,55.46439408)(660.55745605,55.17439941)
\curveto(660.51745368,55.06439448)(660.48745371,54.9493946)(660.46745605,54.82939941)
\curveto(660.44745375,54.71939483)(660.42245377,54.60439494)(660.39245605,54.48439941)
\curveto(660.38245381,54.43439511)(660.37745382,54.38439516)(660.37745605,54.33439941)
\curveto(660.38745381,54.28439526)(660.38745381,54.23439531)(660.37745605,54.18439941)
\curveto(660.34745385,54.06439548)(660.33245386,53.92439562)(660.33245605,53.76439941)
\curveto(660.34245385,53.61439593)(660.34745385,53.46939608)(660.34745605,53.32939941)
\lineto(660.34745605,51.48439941)
\lineto(660.34745605,51.13939941)
\curveto(660.34745385,51.01939853)(660.34245385,50.90439864)(660.33245605,50.79439941)
\curveto(660.32245387,50.68439886)(660.31745388,50.58939896)(660.31745605,50.50939941)
\curveto(660.32745387,50.42939912)(660.30745389,50.35939919)(660.25745605,50.29939941)
\curveto(660.20745399,50.22939932)(660.12745407,50.18939936)(660.01745605,50.17939941)
\curveto(659.91745428,50.16939938)(659.80745439,50.16439938)(659.68745605,50.16439941)
\lineto(659.41745605,50.16439941)
\curveto(659.36745483,50.18439936)(659.31745488,50.19939935)(659.26745605,50.20939941)
\curveto(659.22745497,50.22939932)(659.197455,50.25439929)(659.17745605,50.28439941)
\curveto(659.12745507,50.35439919)(659.0974551,50.43939911)(659.08745605,50.53939941)
\lineto(659.08745605,50.86939941)
\lineto(659.08745605,52.02439941)
\lineto(659.08745605,56.17939941)
\lineto(659.08745605,57.21439941)
\lineto(659.08745605,57.51439941)
\curveto(659.0974551,57.61439193)(659.12745507,57.69939185)(659.17745605,57.76939941)
\curveto(659.20745499,57.80939174)(659.25745494,57.83939171)(659.32745605,57.85939941)
\curveto(659.40745479,57.87939167)(659.4924547,57.88939166)(659.58245605,57.88939941)
\curveto(659.67245452,57.89939165)(659.76245443,57.89939165)(659.85245605,57.88939941)
\curveto(659.94245425,57.87939167)(660.01245418,57.86439168)(660.06245605,57.84439941)
\curveto(660.14245405,57.81439173)(660.192454,57.75439179)(660.21245605,57.66439941)
\curveto(660.24245395,57.58439196)(660.25745394,57.49439205)(660.25745605,57.39439941)
\lineto(660.25745605,57.09439941)
\curveto(660.25745394,56.99439255)(660.27745392,56.90439264)(660.31745605,56.82439941)
\curveto(660.32745387,56.80439274)(660.33745386,56.78939276)(660.34745605,56.77939941)
\lineto(660.39245605,56.73439941)
\curveto(660.50245369,56.73439281)(660.5924536,56.77939277)(660.66245605,56.86939941)
\curveto(660.73245346,56.96939258)(660.7924534,57.0493925)(660.84245605,57.10939941)
\lineto(660.93245605,57.19939941)
\curveto(661.02245317,57.30939224)(661.14745305,57.42439212)(661.30745605,57.54439941)
\curveto(661.46745273,57.66439188)(661.61745258,57.75439179)(661.75745605,57.81439941)
\curveto(661.84745235,57.86439168)(661.94245225,57.89939165)(662.04245605,57.91939941)
\curveto(662.14245205,57.9493916)(662.24745195,57.97939157)(662.35745605,58.00939941)
\curveto(662.41745178,58.01939153)(662.47745172,58.02439152)(662.53745605,58.02439941)
\curveto(662.5974516,58.03439151)(662.65245154,58.0443915)(662.70245605,58.05439941)
}
}
{
\newrgbcolor{curcolor}{0 0 0}
\pscustom[linestyle=none,fillstyle=solid,fillcolor=curcolor]
{
\newpath
\moveto(671.19222168,54.34939941)
\curveto(671.21221362,54.28939526)(671.22221361,54.19439535)(671.22222168,54.06439941)
\curveto(671.22221361,53.9443956)(671.21721361,53.85939569)(671.20722168,53.80939941)
\lineto(671.20722168,53.65939941)
\curveto(671.19721363,53.57939597)(671.18721364,53.50439604)(671.17722168,53.43439941)
\curveto(671.17721365,53.37439617)(671.17221366,53.30439624)(671.16222168,53.22439941)
\curveto(671.14221369,53.16439638)(671.1272137,53.10439644)(671.11722168,53.04439941)
\curveto(671.11721371,52.98439656)(671.10721372,52.92439662)(671.08722168,52.86439941)
\curveto(671.04721378,52.73439681)(671.01221382,52.60439694)(670.98222168,52.47439941)
\curveto(670.95221388,52.3443972)(670.91221392,52.22439732)(670.86222168,52.11439941)
\curveto(670.65221418,51.63439791)(670.37221446,51.22939832)(670.02222168,50.89939941)
\curveto(669.67221516,50.57939897)(669.24221559,50.33439921)(668.73222168,50.16439941)
\curveto(668.62221621,50.12439942)(668.50221633,50.09439945)(668.37222168,50.07439941)
\curveto(668.25221658,50.05439949)(668.1272167,50.03439951)(667.99722168,50.01439941)
\curveto(667.93721689,50.00439954)(667.87221696,49.99939955)(667.80222168,49.99939941)
\curveto(667.74221709,49.98939956)(667.68221715,49.98439956)(667.62222168,49.98439941)
\curveto(667.58221725,49.97439957)(667.52221731,49.96939958)(667.44222168,49.96939941)
\curveto(667.37221746,49.96939958)(667.32221751,49.97439957)(667.29222168,49.98439941)
\curveto(667.25221758,49.99439955)(667.21221762,49.99939955)(667.17222168,49.99939941)
\curveto(667.1322177,49.98939956)(667.09721773,49.98939956)(667.06722168,49.99939941)
\lineto(666.97722168,49.99939941)
\lineto(666.61722168,50.04439941)
\curveto(666.47721835,50.08439946)(666.34221849,50.12439942)(666.21222168,50.16439941)
\curveto(666.08221875,50.20439934)(665.95721887,50.2493993)(665.83722168,50.29939941)
\curveto(665.38721944,50.49939905)(665.01721981,50.75939879)(664.72722168,51.07939941)
\curveto(664.43722039,51.39939815)(664.19722063,51.78939776)(664.00722168,52.24939941)
\curveto(663.95722087,52.3493972)(663.91722091,52.4493971)(663.88722168,52.54939941)
\curveto(663.86722096,52.6493969)(663.84722098,52.75439679)(663.82722168,52.86439941)
\curveto(663.80722102,52.90439664)(663.79722103,52.93439661)(663.79722168,52.95439941)
\curveto(663.80722102,52.98439656)(663.80722102,53.01939653)(663.79722168,53.05939941)
\curveto(663.77722105,53.13939641)(663.76222107,53.21939633)(663.75222168,53.29939941)
\curveto(663.75222108,53.38939616)(663.74222109,53.47439607)(663.72222168,53.55439941)
\lineto(663.72222168,53.67439941)
\curveto(663.72222111,53.71439583)(663.71722111,53.75939579)(663.70722168,53.80939941)
\curveto(663.69722113,53.85939569)(663.69222114,53.9443956)(663.69222168,54.06439941)
\curveto(663.69222114,54.19439535)(663.70222113,54.28939526)(663.72222168,54.34939941)
\curveto(663.74222109,54.41939513)(663.74722108,54.48939506)(663.73722168,54.55939941)
\curveto(663.7272211,54.62939492)(663.7322211,54.69939485)(663.75222168,54.76939941)
\curveto(663.76222107,54.81939473)(663.76722106,54.85939469)(663.76722168,54.88939941)
\curveto(663.77722105,54.92939462)(663.78722104,54.97439457)(663.79722168,55.02439941)
\curveto(663.827221,55.1443944)(663.85222098,55.26439428)(663.87222168,55.38439941)
\curveto(663.90222093,55.50439404)(663.94222089,55.61939393)(663.99222168,55.72939941)
\curveto(664.14222069,56.09939345)(664.32222051,56.42939312)(664.53222168,56.71939941)
\curveto(664.75222008,57.01939253)(665.01721981,57.26939228)(665.32722168,57.46939941)
\curveto(665.44721938,57.549392)(665.57221926,57.61439193)(665.70222168,57.66439941)
\curveto(665.832219,57.72439182)(665.96721886,57.78439176)(666.10722168,57.84439941)
\curveto(666.2272186,57.89439165)(666.35721847,57.92439162)(666.49722168,57.93439941)
\curveto(666.63721819,57.95439159)(666.77721805,57.98439156)(666.91722168,58.02439941)
\lineto(667.11222168,58.02439941)
\curveto(667.18221765,58.03439151)(667.24721758,58.0443915)(667.30722168,58.05439941)
\curveto(668.19721663,58.06439148)(668.93721589,57.87939167)(669.52722168,57.49939941)
\curveto(670.11721471,57.11939243)(670.54221429,56.62439292)(670.80222168,56.01439941)
\curveto(670.85221398,55.91439363)(670.89221394,55.81439373)(670.92222168,55.71439941)
\curveto(670.95221388,55.61439393)(670.98721384,55.50939404)(671.02722168,55.39939941)
\curveto(671.05721377,55.28939426)(671.08221375,55.16939438)(671.10222168,55.03939941)
\curveto(671.12221371,54.91939463)(671.14721368,54.79439475)(671.17722168,54.66439941)
\curveto(671.18721364,54.61439493)(671.18721364,54.55939499)(671.17722168,54.49939941)
\curveto(671.17721365,54.4493951)(671.18221365,54.39939515)(671.19222168,54.34939941)
\moveto(669.85722168,53.49439941)
\curveto(669.87721495,53.56439598)(669.88221495,53.6443959)(669.87222168,53.73439941)
\lineto(669.87222168,53.98939941)
\curveto(669.87221496,54.37939517)(669.83721499,54.70939484)(669.76722168,54.97939941)
\curveto(669.73721509,55.05939449)(669.71221512,55.13939441)(669.69222168,55.21939941)
\curveto(669.67221516,55.29939425)(669.64721518,55.37439417)(669.61722168,55.44439941)
\curveto(669.33721549,56.09439345)(668.89221594,56.544393)(668.28222168,56.79439941)
\curveto(668.21221662,56.82439272)(668.13721669,56.8443927)(668.05722168,56.85439941)
\lineto(667.81722168,56.91439941)
\curveto(667.73721709,56.93439261)(667.65221718,56.9443926)(667.56222168,56.94439941)
\lineto(667.29222168,56.94439941)
\lineto(667.02222168,56.89939941)
\curveto(666.92221791,56.87939267)(666.827218,56.85439269)(666.73722168,56.82439941)
\curveto(666.65721817,56.80439274)(666.57721825,56.77439277)(666.49722168,56.73439941)
\curveto(666.4272184,56.71439283)(666.36221847,56.68439286)(666.30222168,56.64439941)
\curveto(666.24221859,56.60439294)(666.18721864,56.56439298)(666.13722168,56.52439941)
\curveto(665.89721893,56.35439319)(665.70221913,56.1493934)(665.55222168,55.90939941)
\curveto(665.40221943,55.66939388)(665.27221956,55.38939416)(665.16222168,55.06939941)
\curveto(665.1322197,54.96939458)(665.11221972,54.86439468)(665.10222168,54.75439941)
\curveto(665.09221974,54.65439489)(665.07721975,54.549395)(665.05722168,54.43939941)
\curveto(665.04721978,54.39939515)(665.04221979,54.33439521)(665.04222168,54.24439941)
\curveto(665.0322198,54.21439533)(665.0272198,54.17939537)(665.02722168,54.13939941)
\curveto(665.03721979,54.09939545)(665.04221979,54.05439549)(665.04222168,54.00439941)
\lineto(665.04222168,53.70439941)
\curveto(665.04221979,53.60439594)(665.05221978,53.51439603)(665.07222168,53.43439941)
\lineto(665.10222168,53.25439941)
\curveto(665.12221971,53.15439639)(665.13721969,53.05439649)(665.14722168,52.95439941)
\curveto(665.16721966,52.86439668)(665.19721963,52.77939677)(665.23722168,52.69939941)
\curveto(665.33721949,52.45939709)(665.45221938,52.23439731)(665.58222168,52.02439941)
\curveto(665.72221911,51.81439773)(665.89221894,51.63939791)(666.09222168,51.49939941)
\curveto(666.14221869,51.46939808)(666.18721864,51.4443981)(666.22722168,51.42439941)
\curveto(666.26721856,51.40439814)(666.31221852,51.37939817)(666.36222168,51.34939941)
\curveto(666.44221839,51.29939825)(666.5272183,51.25439829)(666.61722168,51.21439941)
\curveto(666.71721811,51.18439836)(666.82221801,51.15439839)(666.93222168,51.12439941)
\curveto(666.98221785,51.10439844)(667.0272178,51.09439845)(667.06722168,51.09439941)
\curveto(667.11721771,51.10439844)(667.16721766,51.10439844)(667.21722168,51.09439941)
\curveto(667.24721758,51.08439846)(667.30721752,51.07439847)(667.39722168,51.06439941)
\curveto(667.49721733,51.05439849)(667.57221726,51.05939849)(667.62222168,51.07939941)
\curveto(667.66221717,51.08939846)(667.70221713,51.08939846)(667.74222168,51.07939941)
\curveto(667.78221705,51.07939847)(667.82221701,51.08939846)(667.86222168,51.10939941)
\curveto(667.94221689,51.12939842)(668.02221681,51.1443984)(668.10222168,51.15439941)
\curveto(668.18221665,51.17439837)(668.25721657,51.19939835)(668.32722168,51.22939941)
\curveto(668.66721616,51.36939818)(668.94221589,51.56439798)(669.15222168,51.81439941)
\curveto(669.36221547,52.06439748)(669.53721529,52.35939719)(669.67722168,52.69939941)
\curveto(669.7272151,52.81939673)(669.75721507,52.9443966)(669.76722168,53.07439941)
\curveto(669.78721504,53.21439633)(669.81721501,53.35439619)(669.85722168,53.49439941)
}
}
{
\newrgbcolor{curcolor}{0 0 0}
\pscustom[linestyle=none,fillstyle=solid,fillcolor=curcolor]
{
\newpath
\moveto(673.25050293,60.82939941)
\curveto(673.38050131,60.82938872)(673.51550118,60.82938872)(673.65550293,60.82939941)
\curveto(673.80550089,60.82938872)(673.91550078,60.79438875)(673.98550293,60.72439941)
\curveto(674.03550066,60.65438889)(674.06050063,60.55938899)(674.06050293,60.43939941)
\curveto(674.07050062,60.32938922)(674.07550062,60.21438933)(674.07550293,60.09439941)
\lineto(674.07550293,58.75939941)
\lineto(674.07550293,52.68439941)
\lineto(674.07550293,51.00439941)
\lineto(674.07550293,50.61439941)
\curveto(674.07550062,50.47439907)(674.05050064,50.36439918)(674.00050293,50.28439941)
\curveto(673.97050072,50.23439931)(673.92550077,50.20439934)(673.86550293,50.19439941)
\curveto(673.81550088,50.18439936)(673.75050094,50.16939938)(673.67050293,50.14939941)
\lineto(673.46050293,50.14939941)
\lineto(673.14550293,50.14939941)
\curveto(673.04550165,50.15939939)(672.97050172,50.19439935)(672.92050293,50.25439941)
\curveto(672.87050182,50.33439921)(672.84050185,50.43439911)(672.83050293,50.55439941)
\lineto(672.83050293,50.92939941)
\lineto(672.83050293,52.30939941)
\lineto(672.83050293,58.54939941)
\lineto(672.83050293,60.01939941)
\curveto(672.83050186,60.12938942)(672.82550187,60.2443893)(672.81550293,60.36439941)
\curveto(672.81550188,60.49438905)(672.84050185,60.59438895)(672.89050293,60.66439941)
\curveto(672.93050176,60.72438882)(673.00550169,60.77438877)(673.11550293,60.81439941)
\curveto(673.13550156,60.82438872)(673.15550154,60.82438872)(673.17550293,60.81439941)
\curveto(673.20550149,60.81438873)(673.23050146,60.81938873)(673.25050293,60.82939941)
}
}
{
\newrgbcolor{curcolor}{0 0 0}
\pscustom[linestyle=none,fillstyle=solid,fillcolor=curcolor]
{
\newpath
\moveto(676.59034668,60.82939941)
\curveto(676.72034506,60.82938872)(676.85534493,60.82938872)(676.99534668,60.82939941)
\curveto(677.14534464,60.82938872)(677.25534453,60.79438875)(677.32534668,60.72439941)
\curveto(677.37534441,60.65438889)(677.40034438,60.55938899)(677.40034668,60.43939941)
\curveto(677.41034437,60.32938922)(677.41534437,60.21438933)(677.41534668,60.09439941)
\lineto(677.41534668,58.75939941)
\lineto(677.41534668,52.68439941)
\lineto(677.41534668,51.00439941)
\lineto(677.41534668,50.61439941)
\curveto(677.41534437,50.47439907)(677.39034439,50.36439918)(677.34034668,50.28439941)
\curveto(677.31034447,50.23439931)(677.26534452,50.20439934)(677.20534668,50.19439941)
\curveto(677.15534463,50.18439936)(677.09034469,50.16939938)(677.01034668,50.14939941)
\lineto(676.80034668,50.14939941)
\lineto(676.48534668,50.14939941)
\curveto(676.3853454,50.15939939)(676.31034547,50.19439935)(676.26034668,50.25439941)
\curveto(676.21034557,50.33439921)(676.1803456,50.43439911)(676.17034668,50.55439941)
\lineto(676.17034668,50.92939941)
\lineto(676.17034668,52.30939941)
\lineto(676.17034668,58.54939941)
\lineto(676.17034668,60.01939941)
\curveto(676.17034561,60.12938942)(676.16534562,60.2443893)(676.15534668,60.36439941)
\curveto(676.15534563,60.49438905)(676.1803456,60.59438895)(676.23034668,60.66439941)
\curveto(676.27034551,60.72438882)(676.34534544,60.77438877)(676.45534668,60.81439941)
\curveto(676.47534531,60.82438872)(676.49534529,60.82438872)(676.51534668,60.81439941)
\curveto(676.54534524,60.81438873)(676.57034521,60.81938873)(676.59034668,60.82939941)
}
}
{
\newrgbcolor{curcolor}{0 0 0}
\pscustom[linestyle=none,fillstyle=solid,fillcolor=curcolor]
{
\newpath
\moveto(686.24519043,50.70439941)
\curveto(686.2751826,50.544399)(686.26018261,50.40939914)(686.20019043,50.29939941)
\curveto(686.14018273,50.19939935)(686.06018281,50.12439942)(685.96019043,50.07439941)
\curveto(685.91018296,50.05439949)(685.85518302,50.0443995)(685.79519043,50.04439941)
\curveto(685.74518313,50.0443995)(685.69018318,50.03439951)(685.63019043,50.01439941)
\curveto(685.41018346,49.96439958)(685.19018368,49.97939957)(684.97019043,50.05939941)
\curveto(684.76018411,50.12939942)(684.61518426,50.21939933)(684.53519043,50.32939941)
\curveto(684.48518439,50.39939915)(684.44018443,50.47939907)(684.40019043,50.56939941)
\curveto(684.36018451,50.66939888)(684.31018456,50.7493988)(684.25019043,50.80939941)
\curveto(684.23018464,50.82939872)(684.20518467,50.8493987)(684.17519043,50.86939941)
\curveto(684.15518472,50.88939866)(684.12518475,50.89439865)(684.08519043,50.88439941)
\curveto(683.9751849,50.85439869)(683.870185,50.79939875)(683.77019043,50.71939941)
\curveto(683.68018519,50.63939891)(683.59018528,50.56939898)(683.50019043,50.50939941)
\curveto(683.3701855,50.42939912)(683.23018564,50.35439919)(683.08019043,50.28439941)
\curveto(682.93018594,50.22439932)(682.7701861,50.16939938)(682.60019043,50.11939941)
\curveto(682.50018637,50.08939946)(682.39018648,50.06939948)(682.27019043,50.05939941)
\curveto(682.16018671,50.0493995)(682.05018682,50.03439951)(681.94019043,50.01439941)
\curveto(681.89018698,50.00439954)(681.84518703,49.99939955)(681.80519043,49.99939941)
\lineto(681.70019043,49.99939941)
\curveto(681.59018728,49.97939957)(681.48518739,49.97939957)(681.38519043,49.99939941)
\lineto(681.25019043,49.99939941)
\curveto(681.20018767,50.00939954)(681.15018772,50.01439953)(681.10019043,50.01439941)
\curveto(681.05018782,50.01439953)(681.00518787,50.02439952)(680.96519043,50.04439941)
\curveto(680.92518795,50.05439949)(680.89018798,50.05939949)(680.86019043,50.05939941)
\curveto(680.84018803,50.0493995)(680.81518806,50.0493995)(680.78519043,50.05939941)
\lineto(680.54519043,50.11939941)
\curveto(680.46518841,50.12939942)(680.39018848,50.1493994)(680.32019043,50.17939941)
\curveto(680.02018885,50.30939924)(679.7751891,50.45439909)(679.58519043,50.61439941)
\curveto(679.40518947,50.78439876)(679.25518962,51.01939853)(679.13519043,51.31939941)
\curveto(679.04518983,51.53939801)(679.00018987,51.80439774)(679.00019043,52.11439941)
\lineto(679.00019043,52.42939941)
\curveto(679.01018986,52.47939707)(679.01518986,52.52939702)(679.01519043,52.57939941)
\lineto(679.04519043,52.75939941)
\lineto(679.16519043,53.08939941)
\curveto(679.20518967,53.19939635)(679.25518962,53.29939625)(679.31519043,53.38939941)
\curveto(679.49518938,53.67939587)(679.74018913,53.89439565)(680.05019043,54.03439941)
\curveto(680.36018851,54.17439537)(680.70018817,54.29939525)(681.07019043,54.40939941)
\curveto(681.21018766,54.4493951)(681.35518752,54.47939507)(681.50519043,54.49939941)
\curveto(681.65518722,54.51939503)(681.80518707,54.544395)(681.95519043,54.57439941)
\curveto(682.02518685,54.59439495)(682.09018678,54.60439494)(682.15019043,54.60439941)
\curveto(682.22018665,54.60439494)(682.29518658,54.61439493)(682.37519043,54.63439941)
\curveto(682.44518643,54.65439489)(682.51518636,54.66439488)(682.58519043,54.66439941)
\curveto(682.65518622,54.67439487)(682.73018614,54.68939486)(682.81019043,54.70939941)
\curveto(683.06018581,54.76939478)(683.29518558,54.81939473)(683.51519043,54.85939941)
\curveto(683.73518514,54.90939464)(683.91018496,55.02439452)(684.04019043,55.20439941)
\curveto(684.10018477,55.28439426)(684.15018472,55.38439416)(684.19019043,55.50439941)
\curveto(684.23018464,55.63439391)(684.23018464,55.77439377)(684.19019043,55.92439941)
\curveto(684.13018474,56.16439338)(684.04018483,56.35439319)(683.92019043,56.49439941)
\curveto(683.81018506,56.63439291)(683.65018522,56.7443928)(683.44019043,56.82439941)
\curveto(683.32018555,56.87439267)(683.1751857,56.90939264)(683.00519043,56.92939941)
\curveto(682.84518603,56.9493926)(682.6751862,56.95939259)(682.49519043,56.95939941)
\curveto(682.31518656,56.95939259)(682.14018673,56.9493926)(681.97019043,56.92939941)
\curveto(681.80018707,56.90939264)(681.65518722,56.87939267)(681.53519043,56.83939941)
\curveto(681.36518751,56.77939277)(681.20018767,56.69439285)(681.04019043,56.58439941)
\curveto(680.96018791,56.52439302)(680.88518799,56.4443931)(680.81519043,56.34439941)
\curveto(680.75518812,56.25439329)(680.70018817,56.15439339)(680.65019043,56.04439941)
\curveto(680.62018825,55.96439358)(680.59018828,55.87939367)(680.56019043,55.78939941)
\curveto(680.54018833,55.69939385)(680.49518838,55.62939392)(680.42519043,55.57939941)
\curveto(680.38518849,55.549394)(680.31518856,55.52439402)(680.21519043,55.50439941)
\curveto(680.12518875,55.49439405)(680.03018884,55.48939406)(679.93019043,55.48939941)
\curveto(679.83018904,55.48939406)(679.73018914,55.49439405)(679.63019043,55.50439941)
\curveto(679.54018933,55.52439402)(679.4751894,55.549394)(679.43519043,55.57939941)
\curveto(679.39518948,55.60939394)(679.36518951,55.65939389)(679.34519043,55.72939941)
\curveto(679.32518955,55.79939375)(679.32518955,55.87439367)(679.34519043,55.95439941)
\curveto(679.3751895,56.08439346)(679.40518947,56.20439334)(679.43519043,56.31439941)
\curveto(679.4751894,56.43439311)(679.52018935,56.549393)(679.57019043,56.65939941)
\curveto(679.76018911,57.00939254)(680.00018887,57.27939227)(680.29019043,57.46939941)
\curveto(680.58018829,57.66939188)(680.94018793,57.82939172)(681.37019043,57.94939941)
\curveto(681.4701874,57.96939158)(681.5701873,57.98439156)(681.67019043,57.99439941)
\curveto(681.78018709,58.00439154)(681.89018698,58.01939153)(682.00019043,58.03939941)
\curveto(682.04018683,58.0493915)(682.10518677,58.0493915)(682.19519043,58.03939941)
\curveto(682.28518659,58.03939151)(682.34018653,58.0493915)(682.36019043,58.06939941)
\curveto(683.06018581,58.07939147)(683.6701852,57.99939155)(684.19019043,57.82939941)
\curveto(684.71018416,57.65939189)(685.0751838,57.33439221)(685.28519043,56.85439941)
\curveto(685.3751835,56.65439289)(685.42518345,56.41939313)(685.43519043,56.14939941)
\curveto(685.45518342,55.88939366)(685.46518341,55.61439393)(685.46519043,55.32439941)
\lineto(685.46519043,52.00939941)
\curveto(685.46518341,51.86939768)(685.4701834,51.73439781)(685.48019043,51.60439941)
\curveto(685.49018338,51.47439807)(685.52018335,51.36939818)(685.57019043,51.28939941)
\curveto(685.62018325,51.21939833)(685.68518319,51.16939838)(685.76519043,51.13939941)
\curveto(685.85518302,51.09939845)(685.94018293,51.06939848)(686.02019043,51.04939941)
\curveto(686.10018277,51.03939851)(686.16018271,50.99439855)(686.20019043,50.91439941)
\curveto(686.22018265,50.88439866)(686.23018264,50.85439869)(686.23019043,50.82439941)
\curveto(686.23018264,50.79439875)(686.23518264,50.75439879)(686.24519043,50.70439941)
\moveto(684.10019043,52.36939941)
\curveto(684.16018471,52.50939704)(684.19018468,52.66939688)(684.19019043,52.84939941)
\curveto(684.20018467,53.03939651)(684.20518467,53.23439631)(684.20519043,53.43439941)
\curveto(684.20518467,53.544396)(684.20018467,53.6443959)(684.19019043,53.73439941)
\curveto(684.18018469,53.82439572)(684.14018473,53.89439565)(684.07019043,53.94439941)
\curveto(684.04018483,53.96439558)(683.9701849,53.97439557)(683.86019043,53.97439941)
\curveto(683.84018503,53.95439559)(683.80518507,53.9443956)(683.75519043,53.94439941)
\curveto(683.70518517,53.9443956)(683.66018521,53.93439561)(683.62019043,53.91439941)
\curveto(683.54018533,53.89439565)(683.45018542,53.87439567)(683.35019043,53.85439941)
\lineto(683.05019043,53.79439941)
\curveto(683.02018585,53.79439575)(682.98518589,53.78939576)(682.94519043,53.77939941)
\lineto(682.84019043,53.77939941)
\curveto(682.69018618,53.73939581)(682.52518635,53.71439583)(682.34519043,53.70439941)
\curveto(682.1751867,53.70439584)(682.01518686,53.68439586)(681.86519043,53.64439941)
\curveto(681.78518709,53.62439592)(681.71018716,53.60439594)(681.64019043,53.58439941)
\curveto(681.58018729,53.57439597)(681.51018736,53.55939599)(681.43019043,53.53939941)
\curveto(681.2701876,53.48939606)(681.12018775,53.42439612)(680.98019043,53.34439941)
\curveto(680.84018803,53.27439627)(680.72018815,53.18439636)(680.62019043,53.07439941)
\curveto(680.52018835,52.96439658)(680.44518843,52.82939672)(680.39519043,52.66939941)
\curveto(680.34518853,52.51939703)(680.32518855,52.33439721)(680.33519043,52.11439941)
\curveto(680.33518854,52.01439753)(680.35018852,51.91939763)(680.38019043,51.82939941)
\curveto(680.42018845,51.7493978)(680.46518841,51.67439787)(680.51519043,51.60439941)
\curveto(680.59518828,51.49439805)(680.70018817,51.39939815)(680.83019043,51.31939941)
\curveto(680.96018791,51.2493983)(681.10018777,51.18939836)(681.25019043,51.13939941)
\curveto(681.30018757,51.12939842)(681.35018752,51.12439842)(681.40019043,51.12439941)
\curveto(681.45018742,51.12439842)(681.50018737,51.11939843)(681.55019043,51.10939941)
\curveto(681.62018725,51.08939846)(681.70518717,51.07439847)(681.80519043,51.06439941)
\curveto(681.91518696,51.06439848)(682.00518687,51.07439847)(682.07519043,51.09439941)
\curveto(682.13518674,51.11439843)(682.19518668,51.11939843)(682.25519043,51.10939941)
\curveto(682.31518656,51.10939844)(682.3751865,51.11939843)(682.43519043,51.13939941)
\curveto(682.51518636,51.15939839)(682.59018628,51.17439837)(682.66019043,51.18439941)
\curveto(682.74018613,51.19439835)(682.81518606,51.21439833)(682.88519043,51.24439941)
\curveto(683.1751857,51.36439818)(683.42018545,51.50939804)(683.62019043,51.67939941)
\curveto(683.83018504,51.8493977)(683.99018488,52.07939747)(684.10019043,52.36939941)
}
}
{
\newrgbcolor{curcolor}{0 0 0}
\pscustom[linestyle=none,fillstyle=solid,fillcolor=curcolor]
{
\newpath
\moveto(694.37683105,50.95939941)
\lineto(694.37683105,50.56939941)
\curveto(694.37682318,50.4493991)(694.3518232,50.3493992)(694.30183105,50.26939941)
\curveto(694.2518233,50.19939935)(694.16682339,50.15939939)(694.04683105,50.14939941)
\lineto(693.70183105,50.14939941)
\curveto(693.64182391,50.1493994)(693.58182397,50.1443994)(693.52183105,50.13439941)
\curveto(693.47182408,50.13439941)(693.42682413,50.1443994)(693.38683105,50.16439941)
\curveto(693.29682426,50.18439936)(693.23682432,50.22439932)(693.20683105,50.28439941)
\curveto(693.16682439,50.33439921)(693.14182441,50.39439915)(693.13183105,50.46439941)
\curveto(693.13182442,50.53439901)(693.11682444,50.60439894)(693.08683105,50.67439941)
\curveto(693.07682448,50.69439885)(693.06182449,50.70939884)(693.04183105,50.71939941)
\curveto(693.03182452,50.73939881)(693.01682454,50.75939879)(692.99683105,50.77939941)
\curveto(692.89682466,50.78939876)(692.81682474,50.76939878)(692.75683105,50.71939941)
\curveto(692.70682485,50.66939888)(692.6518249,50.61939893)(692.59183105,50.56939941)
\curveto(692.39182516,50.41939913)(692.19182536,50.30439924)(691.99183105,50.22439941)
\curveto(691.81182574,50.1443994)(691.60182595,50.08439946)(691.36183105,50.04439941)
\curveto(691.13182642,50.00439954)(690.89182666,49.98439956)(690.64183105,49.98439941)
\curveto(690.40182715,49.97439957)(690.16182739,49.98939956)(689.92183105,50.02939941)
\curveto(689.68182787,50.05939949)(689.47182808,50.11439943)(689.29183105,50.19439941)
\curveto(688.77182878,50.41439913)(688.3518292,50.70939884)(688.03183105,51.07939941)
\curveto(687.71182984,51.45939809)(687.46183009,51.92939762)(687.28183105,52.48939941)
\curveto(687.24183031,52.57939697)(687.21183034,52.66939688)(687.19183105,52.75939941)
\curveto(687.18183037,52.85939669)(687.16183039,52.95939659)(687.13183105,53.05939941)
\curveto(687.12183043,53.10939644)(687.11683044,53.15939639)(687.11683105,53.20939941)
\curveto(687.11683044,53.25939629)(687.11183044,53.30939624)(687.10183105,53.35939941)
\curveto(687.08183047,53.40939614)(687.07183048,53.45939609)(687.07183105,53.50939941)
\curveto(687.08183047,53.56939598)(687.08183047,53.62439592)(687.07183105,53.67439941)
\lineto(687.07183105,53.82439941)
\curveto(687.0518305,53.87439567)(687.04183051,53.93939561)(687.04183105,54.01939941)
\curveto(687.04183051,54.09939545)(687.0518305,54.16439538)(687.07183105,54.21439941)
\lineto(687.07183105,54.37939941)
\curveto(687.09183046,54.4493951)(687.09683046,54.51939503)(687.08683105,54.58939941)
\curveto(687.08683047,54.66939488)(687.09683046,54.7443948)(687.11683105,54.81439941)
\curveto(687.12683043,54.86439468)(687.13183042,54.90939464)(687.13183105,54.94939941)
\curveto(687.13183042,54.98939456)(687.13683042,55.03439451)(687.14683105,55.08439941)
\curveto(687.17683038,55.18439436)(687.20183035,55.27939427)(687.22183105,55.36939941)
\curveto(687.24183031,55.46939408)(687.26683029,55.56439398)(687.29683105,55.65439941)
\curveto(687.42683013,56.03439351)(687.59182996,56.37439317)(687.79183105,56.67439941)
\curveto(688.00182955,56.98439256)(688.2518293,57.23939231)(688.54183105,57.43939941)
\curveto(688.71182884,57.55939199)(688.88682867,57.65939189)(689.06683105,57.73939941)
\curveto(689.2568283,57.81939173)(689.46182809,57.88939166)(689.68183105,57.94939941)
\curveto(689.7518278,57.95939159)(689.81682774,57.96939158)(689.87683105,57.97939941)
\curveto(689.94682761,57.98939156)(690.01682754,58.00439154)(690.08683105,58.02439941)
\lineto(690.23683105,58.02439941)
\curveto(690.31682724,58.0443915)(690.43182712,58.05439149)(690.58183105,58.05439941)
\curveto(690.74182681,58.05439149)(690.86182669,58.0443915)(690.94183105,58.02439941)
\curveto(690.98182657,58.01439153)(691.03682652,58.00939154)(691.10683105,58.00939941)
\curveto(691.21682634,57.97939157)(691.32682623,57.95439159)(691.43683105,57.93439941)
\curveto(691.54682601,57.92439162)(691.6518259,57.89439165)(691.75183105,57.84439941)
\curveto(691.90182565,57.78439176)(692.04182551,57.71939183)(692.17183105,57.64939941)
\curveto(692.31182524,57.57939197)(692.44182511,57.49939205)(692.56183105,57.40939941)
\curveto(692.62182493,57.35939219)(692.68182487,57.30439224)(692.74183105,57.24439941)
\curveto(692.81182474,57.19439235)(692.90182465,57.17939237)(693.01183105,57.19939941)
\curveto(693.03182452,57.22939232)(693.04682451,57.25439229)(693.05683105,57.27439941)
\curveto(693.07682448,57.29439225)(693.09182446,57.32439222)(693.10183105,57.36439941)
\curveto(693.13182442,57.45439209)(693.14182441,57.56939198)(693.13183105,57.70939941)
\lineto(693.13183105,58.08439941)
\lineto(693.13183105,59.80939941)
\lineto(693.13183105,60.27439941)
\curveto(693.13182442,60.45438909)(693.1568244,60.58438896)(693.20683105,60.66439941)
\curveto(693.24682431,60.73438881)(693.30682425,60.77938877)(693.38683105,60.79939941)
\curveto(693.40682415,60.79938875)(693.43182412,60.79938875)(693.46183105,60.79939941)
\curveto(693.49182406,60.80938874)(693.51682404,60.81438873)(693.53683105,60.81439941)
\curveto(693.67682388,60.82438872)(693.82182373,60.82438872)(693.97183105,60.81439941)
\curveto(694.13182342,60.81438873)(694.24182331,60.77438877)(694.30183105,60.69439941)
\curveto(694.3518232,60.61438893)(694.37682318,60.51438903)(694.37683105,60.39439941)
\lineto(694.37683105,60.01939941)
\lineto(694.37683105,50.95939941)
\moveto(693.16183105,53.79439941)
\curveto(693.18182437,53.8443957)(693.19182436,53.90939564)(693.19183105,53.98939941)
\curveto(693.19182436,54.07939547)(693.18182437,54.1493954)(693.16183105,54.19939941)
\lineto(693.16183105,54.42439941)
\curveto(693.14182441,54.51439503)(693.12682443,54.60439494)(693.11683105,54.69439941)
\curveto(693.10682445,54.79439475)(693.08682447,54.88439466)(693.05683105,54.96439941)
\curveto(693.03682452,55.0443945)(693.01682454,55.11939443)(692.99683105,55.18939941)
\curveto(692.98682457,55.25939429)(692.96682459,55.32939422)(692.93683105,55.39939941)
\curveto(692.81682474,55.69939385)(692.66182489,55.96439358)(692.47183105,56.19439941)
\curveto(692.28182527,56.42439312)(692.04182551,56.60439294)(691.75183105,56.73439941)
\curveto(691.6518259,56.78439276)(691.54682601,56.81939273)(691.43683105,56.83939941)
\curveto(691.33682622,56.86939268)(691.22682633,56.89439265)(691.10683105,56.91439941)
\curveto(691.02682653,56.93439261)(690.93682662,56.9443926)(690.83683105,56.94439941)
\lineto(690.56683105,56.94439941)
\curveto(690.51682704,56.93439261)(690.47182708,56.92439262)(690.43183105,56.91439941)
\lineto(690.29683105,56.91439941)
\curveto(690.21682734,56.89439265)(690.13182742,56.87439267)(690.04183105,56.85439941)
\curveto(689.96182759,56.83439271)(689.88182767,56.80939274)(689.80183105,56.77939941)
\curveto(689.48182807,56.63939291)(689.22182833,56.43439311)(689.02183105,56.16439941)
\curveto(688.83182872,55.90439364)(688.67682888,55.59939395)(688.55683105,55.24939941)
\curveto(688.51682904,55.13939441)(688.48682907,55.02439452)(688.46683105,54.90439941)
\curveto(688.4568291,54.79439475)(688.44182911,54.68439486)(688.42183105,54.57439941)
\curveto(688.42182913,54.53439501)(688.41682914,54.49439505)(688.40683105,54.45439941)
\lineto(688.40683105,54.34939941)
\curveto(688.38682917,54.29939525)(688.37682918,54.2443953)(688.37683105,54.18439941)
\curveto(688.38682917,54.12439542)(688.39182916,54.06939548)(688.39183105,54.01939941)
\lineto(688.39183105,53.68939941)
\curveto(688.39182916,53.58939596)(688.40182915,53.49439605)(688.42183105,53.40439941)
\curveto(688.43182912,53.37439617)(688.43682912,53.32439622)(688.43683105,53.25439941)
\curveto(688.4568291,53.18439636)(688.47182908,53.11439643)(688.48183105,53.04439941)
\lineto(688.54183105,52.83439941)
\curveto(688.6518289,52.48439706)(688.80182875,52.18439736)(688.99183105,51.93439941)
\curveto(689.18182837,51.68439786)(689.42182813,51.47939807)(689.71183105,51.31939941)
\curveto(689.80182775,51.26939828)(689.89182766,51.22939832)(689.98183105,51.19939941)
\curveto(690.07182748,51.16939838)(690.17182738,51.13939841)(690.28183105,51.10939941)
\curveto(690.33182722,51.08939846)(690.38182717,51.08439846)(690.43183105,51.09439941)
\curveto(690.49182706,51.10439844)(690.54682701,51.09939845)(690.59683105,51.07939941)
\curveto(690.63682692,51.06939848)(690.67682688,51.06439848)(690.71683105,51.06439941)
\lineto(690.85183105,51.06439941)
\lineto(690.98683105,51.06439941)
\curveto(691.01682654,51.07439847)(691.06682649,51.07939847)(691.13683105,51.07939941)
\curveto(691.21682634,51.09939845)(691.29682626,51.11439843)(691.37683105,51.12439941)
\curveto(691.4568261,51.1443984)(691.53182602,51.16939838)(691.60183105,51.19939941)
\curveto(691.93182562,51.33939821)(692.19682536,51.51439803)(692.39683105,51.72439941)
\curveto(692.60682495,51.9443976)(692.78182477,52.21939733)(692.92183105,52.54939941)
\curveto(692.97182458,52.65939689)(693.00682455,52.76939678)(693.02683105,52.87939941)
\curveto(693.04682451,52.98939656)(693.07182448,53.09939645)(693.10183105,53.20939941)
\curveto(693.12182443,53.2493963)(693.13182442,53.28439626)(693.13183105,53.31439941)
\curveto(693.13182442,53.35439619)(693.13682442,53.39439615)(693.14683105,53.43439941)
\curveto(693.1568244,53.49439605)(693.1568244,53.55439599)(693.14683105,53.61439941)
\curveto(693.14682441,53.67439587)(693.1518244,53.73439581)(693.16183105,53.79439941)
}
}
{
\newrgbcolor{curcolor}{0 0 0}
\pscustom[linestyle=none,fillstyle=solid,fillcolor=curcolor]
{
\newpath
\moveto(703.44808105,54.34939941)
\curveto(703.46807299,54.28939526)(703.47807298,54.19439535)(703.47808105,54.06439941)
\curveto(703.47807298,53.9443956)(703.47307299,53.85939569)(703.46308105,53.80939941)
\lineto(703.46308105,53.65939941)
\curveto(703.45307301,53.57939597)(703.44307302,53.50439604)(703.43308105,53.43439941)
\curveto(703.43307303,53.37439617)(703.42807303,53.30439624)(703.41808105,53.22439941)
\curveto(703.39807306,53.16439638)(703.38307308,53.10439644)(703.37308105,53.04439941)
\curveto(703.37307309,52.98439656)(703.3630731,52.92439662)(703.34308105,52.86439941)
\curveto(703.30307316,52.73439681)(703.26807319,52.60439694)(703.23808105,52.47439941)
\curveto(703.20807325,52.3443972)(703.16807329,52.22439732)(703.11808105,52.11439941)
\curveto(702.90807355,51.63439791)(702.62807383,51.22939832)(702.27808105,50.89939941)
\curveto(701.92807453,50.57939897)(701.49807496,50.33439921)(700.98808105,50.16439941)
\curveto(700.87807558,50.12439942)(700.7580757,50.09439945)(700.62808105,50.07439941)
\curveto(700.50807595,50.05439949)(700.38307608,50.03439951)(700.25308105,50.01439941)
\curveto(700.19307627,50.00439954)(700.12807633,49.99939955)(700.05808105,49.99939941)
\curveto(699.99807646,49.98939956)(699.93807652,49.98439956)(699.87808105,49.98439941)
\curveto(699.83807662,49.97439957)(699.77807668,49.96939958)(699.69808105,49.96939941)
\curveto(699.62807683,49.96939958)(699.57807688,49.97439957)(699.54808105,49.98439941)
\curveto(699.50807695,49.99439955)(699.46807699,49.99939955)(699.42808105,49.99939941)
\curveto(699.38807707,49.98939956)(699.35307711,49.98939956)(699.32308105,49.99939941)
\lineto(699.23308105,49.99939941)
\lineto(698.87308105,50.04439941)
\curveto(698.73307773,50.08439946)(698.59807786,50.12439942)(698.46808105,50.16439941)
\curveto(698.33807812,50.20439934)(698.21307825,50.2493993)(698.09308105,50.29939941)
\curveto(697.64307882,50.49939905)(697.27307919,50.75939879)(696.98308105,51.07939941)
\curveto(696.69307977,51.39939815)(696.45308001,51.78939776)(696.26308105,52.24939941)
\curveto(696.21308025,52.3493972)(696.17308029,52.4493971)(696.14308105,52.54939941)
\curveto(696.12308034,52.6493969)(696.10308036,52.75439679)(696.08308105,52.86439941)
\curveto(696.0630804,52.90439664)(696.05308041,52.93439661)(696.05308105,52.95439941)
\curveto(696.0630804,52.98439656)(696.0630804,53.01939653)(696.05308105,53.05939941)
\curveto(696.03308043,53.13939641)(696.01808044,53.21939633)(696.00808105,53.29939941)
\curveto(696.00808045,53.38939616)(695.99808046,53.47439607)(695.97808105,53.55439941)
\lineto(695.97808105,53.67439941)
\curveto(695.97808048,53.71439583)(695.97308049,53.75939579)(695.96308105,53.80939941)
\curveto(695.95308051,53.85939569)(695.94808051,53.9443956)(695.94808105,54.06439941)
\curveto(695.94808051,54.19439535)(695.9580805,54.28939526)(695.97808105,54.34939941)
\curveto(695.99808046,54.41939513)(696.00308046,54.48939506)(695.99308105,54.55939941)
\curveto(695.98308048,54.62939492)(695.98808047,54.69939485)(696.00808105,54.76939941)
\curveto(696.01808044,54.81939473)(696.02308044,54.85939469)(696.02308105,54.88939941)
\curveto(696.03308043,54.92939462)(696.04308042,54.97439457)(696.05308105,55.02439941)
\curveto(696.08308038,55.1443944)(696.10808035,55.26439428)(696.12808105,55.38439941)
\curveto(696.1580803,55.50439404)(696.19808026,55.61939393)(696.24808105,55.72939941)
\curveto(696.39808006,56.09939345)(696.57807988,56.42939312)(696.78808105,56.71939941)
\curveto(697.00807945,57.01939253)(697.27307919,57.26939228)(697.58308105,57.46939941)
\curveto(697.70307876,57.549392)(697.82807863,57.61439193)(697.95808105,57.66439941)
\curveto(698.08807837,57.72439182)(698.22307824,57.78439176)(698.36308105,57.84439941)
\curveto(698.48307798,57.89439165)(698.61307785,57.92439162)(698.75308105,57.93439941)
\curveto(698.89307757,57.95439159)(699.03307743,57.98439156)(699.17308105,58.02439941)
\lineto(699.36808105,58.02439941)
\curveto(699.43807702,58.03439151)(699.50307696,58.0443915)(699.56308105,58.05439941)
\curveto(700.45307601,58.06439148)(701.19307527,57.87939167)(701.78308105,57.49939941)
\curveto(702.37307409,57.11939243)(702.79807366,56.62439292)(703.05808105,56.01439941)
\curveto(703.10807335,55.91439363)(703.14807331,55.81439373)(703.17808105,55.71439941)
\curveto(703.20807325,55.61439393)(703.24307322,55.50939404)(703.28308105,55.39939941)
\curveto(703.31307315,55.28939426)(703.33807312,55.16939438)(703.35808105,55.03939941)
\curveto(703.37807308,54.91939463)(703.40307306,54.79439475)(703.43308105,54.66439941)
\curveto(703.44307302,54.61439493)(703.44307302,54.55939499)(703.43308105,54.49939941)
\curveto(703.43307303,54.4493951)(703.43807302,54.39939515)(703.44808105,54.34939941)
\moveto(702.11308105,53.49439941)
\curveto(702.13307433,53.56439598)(702.13807432,53.6443959)(702.12808105,53.73439941)
\lineto(702.12808105,53.98939941)
\curveto(702.12807433,54.37939517)(702.09307437,54.70939484)(702.02308105,54.97939941)
\curveto(701.99307447,55.05939449)(701.96807449,55.13939441)(701.94808105,55.21939941)
\curveto(701.92807453,55.29939425)(701.90307456,55.37439417)(701.87308105,55.44439941)
\curveto(701.59307487,56.09439345)(701.14807531,56.544393)(700.53808105,56.79439941)
\curveto(700.46807599,56.82439272)(700.39307607,56.8443927)(700.31308105,56.85439941)
\lineto(700.07308105,56.91439941)
\curveto(699.99307647,56.93439261)(699.90807655,56.9443926)(699.81808105,56.94439941)
\lineto(699.54808105,56.94439941)
\lineto(699.27808105,56.89939941)
\curveto(699.17807728,56.87939267)(699.08307738,56.85439269)(698.99308105,56.82439941)
\curveto(698.91307755,56.80439274)(698.83307763,56.77439277)(698.75308105,56.73439941)
\curveto(698.68307778,56.71439283)(698.61807784,56.68439286)(698.55808105,56.64439941)
\curveto(698.49807796,56.60439294)(698.44307802,56.56439298)(698.39308105,56.52439941)
\curveto(698.15307831,56.35439319)(697.9580785,56.1493934)(697.80808105,55.90939941)
\curveto(697.6580788,55.66939388)(697.52807893,55.38939416)(697.41808105,55.06939941)
\curveto(697.38807907,54.96939458)(697.36807909,54.86439468)(697.35808105,54.75439941)
\curveto(697.34807911,54.65439489)(697.33307913,54.549395)(697.31308105,54.43939941)
\curveto(697.30307916,54.39939515)(697.29807916,54.33439521)(697.29808105,54.24439941)
\curveto(697.28807917,54.21439533)(697.28307918,54.17939537)(697.28308105,54.13939941)
\curveto(697.29307917,54.09939545)(697.29807916,54.05439549)(697.29808105,54.00439941)
\lineto(697.29808105,53.70439941)
\curveto(697.29807916,53.60439594)(697.30807915,53.51439603)(697.32808105,53.43439941)
\lineto(697.35808105,53.25439941)
\curveto(697.37807908,53.15439639)(697.39307907,53.05439649)(697.40308105,52.95439941)
\curveto(697.42307904,52.86439668)(697.45307901,52.77939677)(697.49308105,52.69939941)
\curveto(697.59307887,52.45939709)(697.70807875,52.23439731)(697.83808105,52.02439941)
\curveto(697.97807848,51.81439773)(698.14807831,51.63939791)(698.34808105,51.49939941)
\curveto(698.39807806,51.46939808)(698.44307802,51.4443981)(698.48308105,51.42439941)
\curveto(698.52307794,51.40439814)(698.56807789,51.37939817)(698.61808105,51.34939941)
\curveto(698.69807776,51.29939825)(698.78307768,51.25439829)(698.87308105,51.21439941)
\curveto(698.97307749,51.18439836)(699.07807738,51.15439839)(699.18808105,51.12439941)
\curveto(699.23807722,51.10439844)(699.28307718,51.09439845)(699.32308105,51.09439941)
\curveto(699.37307709,51.10439844)(699.42307704,51.10439844)(699.47308105,51.09439941)
\curveto(699.50307696,51.08439846)(699.5630769,51.07439847)(699.65308105,51.06439941)
\curveto(699.75307671,51.05439849)(699.82807663,51.05939849)(699.87808105,51.07939941)
\curveto(699.91807654,51.08939846)(699.9580765,51.08939846)(699.99808105,51.07939941)
\curveto(700.03807642,51.07939847)(700.07807638,51.08939846)(700.11808105,51.10939941)
\curveto(700.19807626,51.12939842)(700.27807618,51.1443984)(700.35808105,51.15439941)
\curveto(700.43807602,51.17439837)(700.51307595,51.19939835)(700.58308105,51.22939941)
\curveto(700.92307554,51.36939818)(701.19807526,51.56439798)(701.40808105,51.81439941)
\curveto(701.61807484,52.06439748)(701.79307467,52.35939719)(701.93308105,52.69939941)
\curveto(701.98307448,52.81939673)(702.01307445,52.9443966)(702.02308105,53.07439941)
\curveto(702.04307442,53.21439633)(702.07307439,53.35439619)(702.11308105,53.49439941)
}
}
{
\newrgbcolor{curcolor}{0 0 0}
\pscustom[linestyle=none,fillstyle=solid,fillcolor=curcolor]
{
\newpath
\moveto(708.5813623,58.05439941)
\curveto(708.81135751,58.05439149)(708.94135738,57.99439155)(708.9713623,57.87439941)
\curveto(709.00135732,57.76439178)(709.01635731,57.59939195)(709.0163623,57.37939941)
\lineto(709.0163623,57.09439941)
\curveto(709.01635731,57.00439254)(708.99135733,56.92939262)(708.9413623,56.86939941)
\curveto(708.88135744,56.78939276)(708.79635753,56.7443928)(708.6863623,56.73439941)
\curveto(708.57635775,56.73439281)(708.46635786,56.71939283)(708.3563623,56.68939941)
\curveto(708.21635811,56.65939289)(708.08135824,56.62939292)(707.9513623,56.59939941)
\curveto(707.83135849,56.56939298)(707.71635861,56.52939302)(707.6063623,56.47939941)
\curveto(707.31635901,56.3493932)(707.08135924,56.16939338)(706.9013623,55.93939941)
\curveto(706.7213596,55.71939383)(706.56635976,55.46439408)(706.4363623,55.17439941)
\curveto(706.39635993,55.06439448)(706.36635996,54.9493946)(706.3463623,54.82939941)
\curveto(706.32636,54.71939483)(706.30136002,54.60439494)(706.2713623,54.48439941)
\curveto(706.26136006,54.43439511)(706.25636007,54.38439516)(706.2563623,54.33439941)
\curveto(706.26636006,54.28439526)(706.26636006,54.23439531)(706.2563623,54.18439941)
\curveto(706.2263601,54.06439548)(706.21136011,53.92439562)(706.2113623,53.76439941)
\curveto(706.2213601,53.61439593)(706.2263601,53.46939608)(706.2263623,53.32939941)
\lineto(706.2263623,51.48439941)
\lineto(706.2263623,51.13939941)
\curveto(706.2263601,51.01939853)(706.2213601,50.90439864)(706.2113623,50.79439941)
\curveto(706.20136012,50.68439886)(706.19636013,50.58939896)(706.1963623,50.50939941)
\curveto(706.20636012,50.42939912)(706.18636014,50.35939919)(706.1363623,50.29939941)
\curveto(706.08636024,50.22939932)(706.00636032,50.18939936)(705.8963623,50.17939941)
\curveto(705.79636053,50.16939938)(705.68636064,50.16439938)(705.5663623,50.16439941)
\lineto(705.2963623,50.16439941)
\curveto(705.24636108,50.18439936)(705.19636113,50.19939935)(705.1463623,50.20939941)
\curveto(705.10636122,50.22939932)(705.07636125,50.25439929)(705.0563623,50.28439941)
\curveto(705.00636132,50.35439919)(704.97636135,50.43939911)(704.9663623,50.53939941)
\lineto(704.9663623,50.86939941)
\lineto(704.9663623,52.02439941)
\lineto(704.9663623,56.17939941)
\lineto(704.9663623,57.21439941)
\lineto(704.9663623,57.51439941)
\curveto(704.97636135,57.61439193)(705.00636132,57.69939185)(705.0563623,57.76939941)
\curveto(705.08636124,57.80939174)(705.13636119,57.83939171)(705.2063623,57.85939941)
\curveto(705.28636104,57.87939167)(705.37136095,57.88939166)(705.4613623,57.88939941)
\curveto(705.55136077,57.89939165)(705.64136068,57.89939165)(705.7313623,57.88939941)
\curveto(705.8213605,57.87939167)(705.89136043,57.86439168)(705.9413623,57.84439941)
\curveto(706.0213603,57.81439173)(706.07136025,57.75439179)(706.0913623,57.66439941)
\curveto(706.1213602,57.58439196)(706.13636019,57.49439205)(706.1363623,57.39439941)
\lineto(706.1363623,57.09439941)
\curveto(706.13636019,56.99439255)(706.15636017,56.90439264)(706.1963623,56.82439941)
\curveto(706.20636012,56.80439274)(706.21636011,56.78939276)(706.2263623,56.77939941)
\lineto(706.2713623,56.73439941)
\curveto(706.38135994,56.73439281)(706.47135985,56.77939277)(706.5413623,56.86939941)
\curveto(706.61135971,56.96939258)(706.67135965,57.0493925)(706.7213623,57.10939941)
\lineto(706.8113623,57.19939941)
\curveto(706.90135942,57.30939224)(707.0263593,57.42439212)(707.1863623,57.54439941)
\curveto(707.34635898,57.66439188)(707.49635883,57.75439179)(707.6363623,57.81439941)
\curveto(707.7263586,57.86439168)(707.8213585,57.89939165)(707.9213623,57.91939941)
\curveto(708.0213583,57.9493916)(708.1263582,57.97939157)(708.2363623,58.00939941)
\curveto(708.29635803,58.01939153)(708.35635797,58.02439152)(708.4163623,58.02439941)
\curveto(708.47635785,58.03439151)(708.53135779,58.0443915)(708.5813623,58.05439941)
}
}
{
\newrgbcolor{curcolor}{0 0 0}
\pscustom[linestyle=none,fillstyle=solid,fillcolor=curcolor]
{
\newpath
\moveto(143.59249146,89.16295776)
\lineto(143.59249146,88.90795776)
\curveto(143.60248375,88.827953)(143.59748376,88.75295307)(143.57749146,88.68295776)
\lineto(143.57749146,88.44295776)
\lineto(143.57749146,88.27795776)
\curveto(143.5574838,88.17795365)(143.54748381,88.07295375)(143.54749146,87.96295776)
\curveto(143.54748381,87.86295396)(143.53748382,87.76295406)(143.51749146,87.66295776)
\lineto(143.51749146,87.51295776)
\curveto(143.48748387,87.37295445)(143.46748389,87.23295459)(143.45749146,87.09295776)
\curveto(143.44748391,86.96295486)(143.42248393,86.83295499)(143.38249146,86.70295776)
\curveto(143.36248399,86.6229552)(143.34248401,86.53795529)(143.32249146,86.44795776)
\lineto(143.26249146,86.20795776)
\lineto(143.14249146,85.90795776)
\curveto(143.11248424,85.81795601)(143.07748428,85.7279561)(143.03749146,85.63795776)
\curveto(142.93748442,85.41795641)(142.80248455,85.20295662)(142.63249146,84.99295776)
\curveto(142.47248488,84.78295704)(142.29748506,84.61295721)(142.10749146,84.48295776)
\curveto(142.0574853,84.44295738)(141.99748536,84.40295742)(141.92749146,84.36295776)
\curveto(141.86748549,84.33295749)(141.80748555,84.29795753)(141.74749146,84.25795776)
\curveto(141.66748569,84.20795762)(141.57248578,84.16795766)(141.46249146,84.13795776)
\curveto(141.352486,84.10795772)(141.24748611,84.07795775)(141.14749146,84.04795776)
\curveto(141.03748632,84.00795782)(140.92748643,83.98295784)(140.81749146,83.97295776)
\curveto(140.70748665,83.96295786)(140.59248676,83.94795788)(140.47249146,83.92795776)
\curveto(140.43248692,83.91795791)(140.38748697,83.91795791)(140.33749146,83.92795776)
\curveto(140.29748706,83.9279579)(140.2574871,83.9229579)(140.21749146,83.91295776)
\curveto(140.17748718,83.90295792)(140.12248723,83.89795793)(140.05249146,83.89795776)
\curveto(139.98248737,83.89795793)(139.93248742,83.90295792)(139.90249146,83.91295776)
\curveto(139.8524875,83.93295789)(139.80748755,83.93795789)(139.76749146,83.92795776)
\curveto(139.72748763,83.91795791)(139.69248766,83.91795791)(139.66249146,83.92795776)
\lineto(139.57249146,83.92795776)
\curveto(139.51248784,83.94795788)(139.44748791,83.96295786)(139.37749146,83.97295776)
\curveto(139.31748804,83.97295785)(139.2524881,83.97795785)(139.18249146,83.98795776)
\curveto(139.01248834,84.03795779)(138.8524885,84.08795774)(138.70249146,84.13795776)
\curveto(138.5524888,84.18795764)(138.40748895,84.25295757)(138.26749146,84.33295776)
\curveto(138.21748914,84.37295745)(138.16248919,84.40295742)(138.10249146,84.42295776)
\curveto(138.0524893,84.45295737)(138.00248935,84.48795734)(137.95249146,84.52795776)
\curveto(137.71248964,84.70795712)(137.51248984,84.9279569)(137.35249146,85.18795776)
\curveto(137.19249016,85.44795638)(137.0524903,85.73295609)(136.93249146,86.04295776)
\curveto(136.87249048,86.18295564)(136.82749053,86.3229555)(136.79749146,86.46295776)
\curveto(136.76749059,86.61295521)(136.73249062,86.76795506)(136.69249146,86.92795776)
\curveto(136.67249068,87.03795479)(136.6574907,87.14795468)(136.64749146,87.25795776)
\curveto(136.63749072,87.36795446)(136.62249073,87.47795435)(136.60249146,87.58795776)
\curveto(136.59249076,87.6279542)(136.58749077,87.66795416)(136.58749146,87.70795776)
\curveto(136.59749076,87.74795408)(136.59749076,87.78795404)(136.58749146,87.82795776)
\curveto(136.57749078,87.87795395)(136.57249078,87.9279539)(136.57249146,87.97795776)
\lineto(136.57249146,88.14295776)
\curveto(136.5524908,88.19295363)(136.54749081,88.24295358)(136.55749146,88.29295776)
\curveto(136.56749079,88.35295347)(136.56749079,88.40795342)(136.55749146,88.45795776)
\curveto(136.54749081,88.49795333)(136.54749081,88.54295328)(136.55749146,88.59295776)
\curveto(136.56749079,88.64295318)(136.56249079,88.69295313)(136.54249146,88.74295776)
\curveto(136.52249083,88.81295301)(136.51749084,88.88795294)(136.52749146,88.96795776)
\curveto(136.53749082,89.05795277)(136.54249081,89.14295268)(136.54249146,89.22295776)
\curveto(136.54249081,89.31295251)(136.53749082,89.41295241)(136.52749146,89.52295776)
\curveto(136.51749084,89.64295218)(136.52249083,89.74295208)(136.54249146,89.82295776)
\lineto(136.54249146,90.10795776)
\lineto(136.58749146,90.73795776)
\curveto(136.59749076,90.83795099)(136.60749075,90.93295089)(136.61749146,91.02295776)
\lineto(136.64749146,91.32295776)
\curveto(136.66749069,91.37295045)(136.67249068,91.4229504)(136.66249146,91.47295776)
\curveto(136.66249069,91.53295029)(136.67249068,91.58795024)(136.69249146,91.63795776)
\curveto(136.74249061,91.80795002)(136.78249057,91.97294985)(136.81249146,92.13295776)
\curveto(136.84249051,92.30294952)(136.89249046,92.46294936)(136.96249146,92.61295776)
\curveto(137.1524902,93.07294875)(137.37248998,93.44794838)(137.62249146,93.73795776)
\curveto(137.88248947,94.0279478)(138.24248911,94.27294755)(138.70249146,94.47295776)
\curveto(138.83248852,94.5229473)(138.96248839,94.55794727)(139.09249146,94.57795776)
\curveto(139.23248812,94.59794723)(139.37248798,94.6229472)(139.51249146,94.65295776)
\curveto(139.58248777,94.66294716)(139.64748771,94.66794716)(139.70749146,94.66795776)
\curveto(139.76748759,94.66794716)(139.83248752,94.67294715)(139.90249146,94.68295776)
\curveto(140.73248662,94.70294712)(141.40248595,94.55294727)(141.91249146,94.23295776)
\curveto(142.42248493,93.9229479)(142.80248455,93.48294834)(143.05249146,92.91295776)
\curveto(143.10248425,92.79294903)(143.14748421,92.66794916)(143.18749146,92.53795776)
\curveto(143.22748413,92.40794942)(143.27248408,92.27294955)(143.32249146,92.13295776)
\curveto(143.34248401,92.05294977)(143.357484,91.96794986)(143.36749146,91.87795776)
\lineto(143.42749146,91.63795776)
\curveto(143.4574839,91.5279503)(143.47248388,91.41795041)(143.47249146,91.30795776)
\curveto(143.48248387,91.19795063)(143.49748386,91.08795074)(143.51749146,90.97795776)
\curveto(143.53748382,90.9279509)(143.54248381,90.88295094)(143.53249146,90.84295776)
\curveto(143.53248382,90.80295102)(143.53748382,90.76295106)(143.54749146,90.72295776)
\curveto(143.5574838,90.67295115)(143.5574838,90.61795121)(143.54749146,90.55795776)
\curveto(143.54748381,90.50795132)(143.5524838,90.45795137)(143.56249146,90.40795776)
\lineto(143.56249146,90.27295776)
\curveto(143.58248377,90.21295161)(143.58248377,90.14295168)(143.56249146,90.06295776)
\curveto(143.5524838,89.99295183)(143.5574838,89.9279519)(143.57749146,89.86795776)
\curveto(143.58748377,89.83795199)(143.59248376,89.79795203)(143.59249146,89.74795776)
\lineto(143.59249146,89.62795776)
\lineto(143.59249146,89.16295776)
\moveto(142.04749146,86.83795776)
\curveto(142.14748521,87.15795467)(142.20748515,87.5229543)(142.22749146,87.93295776)
\curveto(142.24748511,88.34295348)(142.2574851,88.75295307)(142.25749146,89.16295776)
\curveto(142.2574851,89.59295223)(142.24748511,90.01295181)(142.22749146,90.42295776)
\curveto(142.20748515,90.83295099)(142.16248519,91.21795061)(142.09249146,91.57795776)
\curveto(142.02248533,91.93794989)(141.91248544,92.25794957)(141.76249146,92.53795776)
\curveto(141.62248573,92.827949)(141.42748593,93.06294876)(141.17749146,93.24295776)
\curveto(141.01748634,93.35294847)(140.83748652,93.43294839)(140.63749146,93.48295776)
\curveto(140.43748692,93.54294828)(140.19248716,93.57294825)(139.90249146,93.57295776)
\curveto(139.88248747,93.55294827)(139.84748751,93.54294828)(139.79749146,93.54295776)
\curveto(139.74748761,93.55294827)(139.70748765,93.55294827)(139.67749146,93.54295776)
\curveto(139.59748776,93.5229483)(139.52248783,93.50294832)(139.45249146,93.48295776)
\curveto(139.39248796,93.47294835)(139.32748803,93.45294837)(139.25749146,93.42295776)
\curveto(138.98748837,93.30294852)(138.76748859,93.13294869)(138.59749146,92.91295776)
\curveto(138.43748892,92.70294912)(138.30248905,92.45794937)(138.19249146,92.17795776)
\curveto(138.14248921,92.06794976)(138.10248925,91.94794988)(138.07249146,91.81795776)
\curveto(138.0524893,91.69795013)(138.02748933,91.57295025)(137.99749146,91.44295776)
\curveto(137.97748938,91.39295043)(137.96748939,91.33795049)(137.96749146,91.27795776)
\curveto(137.96748939,91.2279506)(137.96248939,91.17795065)(137.95249146,91.12795776)
\curveto(137.94248941,91.03795079)(137.93248942,90.94295088)(137.92249146,90.84295776)
\curveto(137.91248944,90.75295107)(137.90248945,90.65795117)(137.89249146,90.55795776)
\curveto(137.89248946,90.47795135)(137.88748947,90.39295143)(137.87749146,90.30295776)
\lineto(137.87749146,90.06295776)
\lineto(137.87749146,89.88295776)
\curveto(137.86748949,89.85295197)(137.86248949,89.81795201)(137.86249146,89.77795776)
\lineto(137.86249146,89.64295776)
\lineto(137.86249146,89.19295776)
\curveto(137.86248949,89.11295271)(137.8574895,89.0279528)(137.84749146,88.93795776)
\curveto(137.84748951,88.85795297)(137.8574895,88.78295304)(137.87749146,88.71295776)
\lineto(137.87749146,88.44295776)
\curveto(137.87748948,88.4229534)(137.87248948,88.39295343)(137.86249146,88.35295776)
\curveto(137.86248949,88.3229535)(137.86748949,88.29795353)(137.87749146,88.27795776)
\curveto(137.88748947,88.17795365)(137.89248946,88.07795375)(137.89249146,87.97795776)
\curveto(137.90248945,87.88795394)(137.91248944,87.78795404)(137.92249146,87.67795776)
\curveto(137.9524894,87.55795427)(137.96748939,87.43295439)(137.96749146,87.30295776)
\curveto(137.97748938,87.18295464)(138.00248935,87.06795476)(138.04249146,86.95795776)
\curveto(138.12248923,86.65795517)(138.20748915,86.39295543)(138.29749146,86.16295776)
\curveto(138.39748896,85.93295589)(138.54248881,85.71795611)(138.73249146,85.51795776)
\curveto(138.94248841,85.31795651)(139.20748815,85.16795666)(139.52749146,85.06795776)
\curveto(139.56748779,85.04795678)(139.60248775,85.03795679)(139.63249146,85.03795776)
\curveto(139.67248768,85.04795678)(139.71748764,85.04295678)(139.76749146,85.02295776)
\curveto(139.80748755,85.01295681)(139.87748748,85.00295682)(139.97749146,84.99295776)
\curveto(140.08748727,84.98295684)(140.17248718,84.98795684)(140.23249146,85.00795776)
\curveto(140.30248705,85.0279568)(140.37248698,85.03795679)(140.44249146,85.03795776)
\curveto(140.51248684,85.04795678)(140.57748678,85.06295676)(140.63749146,85.08295776)
\curveto(140.83748652,85.14295668)(141.01748634,85.2279566)(141.17749146,85.33795776)
\curveto(141.20748615,85.35795647)(141.23248612,85.37795645)(141.25249146,85.39795776)
\lineto(141.31249146,85.45795776)
\curveto(141.352486,85.47795635)(141.40248595,85.51795631)(141.46249146,85.57795776)
\curveto(141.56248579,85.71795611)(141.64748571,85.84795598)(141.71749146,85.96795776)
\curveto(141.78748557,86.08795574)(141.8574855,86.23295559)(141.92749146,86.40295776)
\curveto(141.9574854,86.47295535)(141.97748538,86.54295528)(141.98749146,86.61295776)
\curveto(142.00748535,86.68295514)(142.02748533,86.75795507)(142.04749146,86.83795776)
}
}
{
\newrgbcolor{curcolor}{0 0 0}
\pscustom[linestyle=none,fillstyle=solid,fillcolor=curcolor]
{
\newpath
\moveto(131.55287964,168.96866333)
\curveto(132.242875,168.9786527)(132.8428744,168.85865282)(133.35287964,168.60866333)
\curveto(133.87287337,168.35865332)(134.26787298,168.02365366)(134.53787964,167.60366333)
\curveto(134.58787266,167.52365416)(134.63287261,167.43365425)(134.67287964,167.33366333)
\curveto(134.71287253,167.24365444)(134.75787249,167.14865453)(134.80787964,167.04866333)
\curveto(134.8478724,166.94865473)(134.87787237,166.84865483)(134.89787964,166.74866333)
\curveto(134.91787233,166.64865503)(134.93787231,166.54365514)(134.95787964,166.43366333)
\curveto(134.97787227,166.3836553)(134.98287226,166.33865534)(134.97287964,166.29866333)
\curveto(134.96287228,166.25865542)(134.96787228,166.21365547)(134.98787964,166.16366333)
\curveto(134.99787225,166.11365557)(135.00287224,166.02865565)(135.00287964,165.90866333)
\curveto(135.00287224,165.79865588)(134.99787225,165.71365597)(134.98787964,165.65366333)
\curveto(134.96787228,165.59365609)(134.95787229,165.53365615)(134.95787964,165.47366333)
\curveto(134.96787228,165.41365627)(134.96287228,165.35365633)(134.94287964,165.29366333)
\curveto(134.90287234,165.15365653)(134.86787238,165.01865666)(134.83787964,164.88866333)
\curveto(134.80787244,164.75865692)(134.76787248,164.63365705)(134.71787964,164.51366333)
\curveto(134.65787259,164.37365731)(134.58787266,164.24865743)(134.50787964,164.13866333)
\curveto(134.43787281,164.02865765)(134.36287288,163.91865776)(134.28287964,163.80866333)
\lineto(134.22287964,163.74866333)
\curveto(134.21287303,163.72865795)(134.19787305,163.70865797)(134.17787964,163.68866333)
\curveto(134.05787319,163.52865815)(133.92287332,163.3836583)(133.77287964,163.25366333)
\curveto(133.62287362,163.12365856)(133.46287378,162.99865868)(133.29287964,162.87866333)
\curveto(132.98287426,162.65865902)(132.68787456,162.45365923)(132.40787964,162.26366333)
\curveto(132.17787507,162.12365956)(131.9478753,161.98865969)(131.71787964,161.85866333)
\curveto(131.49787575,161.72865995)(131.27787597,161.59366009)(131.05787964,161.45366333)
\curveto(130.80787644,161.2836604)(130.56787668,161.10366058)(130.33787964,160.91366333)
\curveto(130.11787713,160.72366096)(129.92787732,160.49866118)(129.76787964,160.23866333)
\curveto(129.72787752,160.1786615)(129.69287755,160.11866156)(129.66287964,160.05866333)
\curveto(129.63287761,160.00866167)(129.60287764,159.94366174)(129.57287964,159.86366333)
\curveto(129.55287769,159.79366189)(129.5478777,159.73366195)(129.55787964,159.68366333)
\curveto(129.57787767,159.61366207)(129.61287763,159.55866212)(129.66287964,159.51866333)
\curveto(129.71287753,159.48866219)(129.77287747,159.46866221)(129.84287964,159.45866333)
\lineto(130.08287964,159.45866333)
\lineto(130.83287964,159.45866333)
\lineto(133.63787964,159.45866333)
\lineto(134.29787964,159.45866333)
\curveto(134.38787286,159.45866222)(134.47287277,159.45366223)(134.55287964,159.44366333)
\curveto(134.63287261,159.44366224)(134.69787255,159.42366226)(134.74787964,159.38366333)
\curveto(134.79787245,159.34366234)(134.83787241,159.26866241)(134.86787964,159.15866333)
\curveto(134.90787234,159.05866262)(134.91787233,158.95866272)(134.89787964,158.85866333)
\lineto(134.89787964,158.72366333)
\curveto(134.87787237,158.65366303)(134.85787239,158.59366309)(134.83787964,158.54366333)
\curveto(134.81787243,158.49366319)(134.78287246,158.45366323)(134.73287964,158.42366333)
\curveto(134.68287256,158.3836633)(134.61287263,158.36366332)(134.52287964,158.36366333)
\lineto(134.25287964,158.36366333)
\lineto(133.35287964,158.36366333)
\lineto(129.84287964,158.36366333)
\lineto(128.77787964,158.36366333)
\curveto(128.69787855,158.36366332)(128.60787864,158.35866332)(128.50787964,158.34866333)
\curveto(128.40787884,158.34866333)(128.32287892,158.35866332)(128.25287964,158.37866333)
\curveto(128.0428792,158.44866323)(127.97787927,158.62866305)(128.05787964,158.91866333)
\curveto(128.06787918,158.95866272)(128.06787918,158.99366269)(128.05787964,159.02366333)
\curveto(128.05787919,159.06366262)(128.06787918,159.10866257)(128.08787964,159.15866333)
\curveto(128.10787914,159.23866244)(128.12787912,159.32366236)(128.14787964,159.41366333)
\curveto(128.16787908,159.50366218)(128.19287905,159.58866209)(128.22287964,159.66866333)
\curveto(128.38287886,160.15866152)(128.58287866,160.57366111)(128.82287964,160.91366333)
\curveto(129.00287824,161.16366052)(129.20787804,161.38866029)(129.43787964,161.58866333)
\curveto(129.66787758,161.79865988)(129.90787734,161.99365969)(130.15787964,162.17366333)
\curveto(130.41787683,162.35365933)(130.68287656,162.52365916)(130.95287964,162.68366333)
\curveto(131.23287601,162.85365883)(131.50287574,163.02865865)(131.76287964,163.20866333)
\curveto(131.87287537,163.28865839)(131.97787527,163.36365832)(132.07787964,163.43366333)
\curveto(132.18787506,163.50365818)(132.29787495,163.5786581)(132.40787964,163.65866333)
\curveto(132.4478748,163.68865799)(132.48287476,163.71865796)(132.51287964,163.74866333)
\curveto(132.55287469,163.78865789)(132.59287465,163.81865786)(132.63287964,163.83866333)
\curveto(132.77287447,163.94865773)(132.89787435,164.07365761)(133.00787964,164.21366333)
\curveto(133.02787422,164.24365744)(133.05287419,164.26865741)(133.08287964,164.28866333)
\curveto(133.11287413,164.31865736)(133.13787411,164.34865733)(133.15787964,164.37866333)
\curveto(133.23787401,164.4786572)(133.30287394,164.5786571)(133.35287964,164.67866333)
\curveto(133.41287383,164.7786569)(133.46787378,164.88865679)(133.51787964,165.00866333)
\curveto(133.5478737,165.0786566)(133.56787368,165.15365653)(133.57787964,165.23366333)
\lineto(133.63787964,165.47366333)
\lineto(133.63787964,165.56366333)
\curveto(133.6478736,165.59365609)(133.65287359,165.62365606)(133.65287964,165.65366333)
\curveto(133.67287357,165.72365596)(133.67787357,165.81865586)(133.66787964,165.93866333)
\curveto(133.66787358,166.06865561)(133.65787359,166.16865551)(133.63787964,166.23866333)
\curveto(133.61787363,166.31865536)(133.59787365,166.39365529)(133.57787964,166.46366333)
\curveto(133.56787368,166.54365514)(133.5478737,166.62365506)(133.51787964,166.70366333)
\curveto(133.40787384,166.94365474)(133.25787399,167.14365454)(133.06787964,167.30366333)
\curveto(132.88787436,167.47365421)(132.66787458,167.61365407)(132.40787964,167.72366333)
\curveto(132.33787491,167.74365394)(132.26787498,167.75865392)(132.19787964,167.76866333)
\curveto(132.12787512,167.78865389)(132.05287519,167.80865387)(131.97287964,167.82866333)
\curveto(131.89287535,167.84865383)(131.78287546,167.85865382)(131.64287964,167.85866333)
\curveto(131.51287573,167.85865382)(131.40787584,167.84865383)(131.32787964,167.82866333)
\curveto(131.26787598,167.81865386)(131.21287603,167.81365387)(131.16287964,167.81366333)
\curveto(131.11287613,167.81365387)(131.06287618,167.80365388)(131.01287964,167.78366333)
\curveto(130.91287633,167.74365394)(130.81787643,167.70365398)(130.72787964,167.66366333)
\curveto(130.6478766,167.62365406)(130.56787668,167.5786541)(130.48787964,167.52866333)
\curveto(130.45787679,167.50865417)(130.42787682,167.4836542)(130.39787964,167.45366333)
\curveto(130.37787687,167.42365426)(130.35287689,167.39865428)(130.32287964,167.37866333)
\lineto(130.24787964,167.30366333)
\curveto(130.21787703,167.2836544)(130.19287705,167.26365442)(130.17287964,167.24366333)
\lineto(130.02287964,167.03366333)
\curveto(129.98287726,166.97365471)(129.93787731,166.90865477)(129.88787964,166.83866333)
\curveto(129.82787742,166.74865493)(129.77787747,166.64365504)(129.73787964,166.52366333)
\curveto(129.70787754,166.41365527)(129.67287757,166.30365538)(129.63287964,166.19366333)
\curveto(129.59287765,166.0836556)(129.56787768,165.93865574)(129.55787964,165.75866333)
\curveto(129.5478777,165.58865609)(129.51787773,165.46365622)(129.46787964,165.38366333)
\curveto(129.41787783,165.30365638)(129.3428779,165.25865642)(129.24287964,165.24866333)
\curveto(129.1428781,165.23865644)(129.03287821,165.23365645)(128.91287964,165.23366333)
\curveto(128.87287837,165.23365645)(128.83287841,165.22865645)(128.79287964,165.21866333)
\curveto(128.75287849,165.21865646)(128.71787853,165.22365646)(128.68787964,165.23366333)
\curveto(128.63787861,165.25365643)(128.58787866,165.26365642)(128.53787964,165.26366333)
\curveto(128.49787875,165.26365642)(128.45787879,165.27365641)(128.41787964,165.29366333)
\curveto(128.32787892,165.35365633)(128.28287896,165.48865619)(128.28287964,165.69866333)
\lineto(128.28287964,165.81866333)
\curveto(128.29287895,165.8786558)(128.29787895,165.93865574)(128.29787964,165.99866333)
\curveto(128.30787894,166.06865561)(128.31787893,166.13365555)(128.32787964,166.19366333)
\curveto(128.3478789,166.30365538)(128.36787888,166.40365528)(128.38787964,166.49366333)
\curveto(128.40787884,166.59365509)(128.43787881,166.68865499)(128.47787964,166.77866333)
\curveto(128.49787875,166.84865483)(128.51787873,166.90865477)(128.53787964,166.95866333)
\lineto(128.59787964,167.13866333)
\curveto(128.71787853,167.39865428)(128.87287837,167.64365404)(129.06287964,167.87366333)
\curveto(129.26287798,168.10365358)(129.47787777,168.28865339)(129.70787964,168.42866333)
\curveto(129.81787743,168.50865317)(129.93287731,168.57365311)(130.05287964,168.62366333)
\lineto(130.44287964,168.77366333)
\curveto(130.55287669,168.82365286)(130.66787658,168.85365283)(130.78787964,168.86366333)
\curveto(130.90787634,168.8836528)(131.03287621,168.90865277)(131.16287964,168.93866333)
\curveto(131.23287601,168.93865274)(131.29787595,168.93865274)(131.35787964,168.93866333)
\curveto(131.41787583,168.94865273)(131.48287576,168.95865272)(131.55287964,168.96866333)
}
}
{
\newrgbcolor{curcolor}{0 0 0}
\pscustom[linestyle=none,fillstyle=solid,fillcolor=curcolor]
{
\newpath
\moveto(138.16248901,168.77366333)
\lineto(141.76248901,168.77366333)
\lineto(142.40748901,168.77366333)
\curveto(142.48748248,168.77365291)(142.56248241,168.76865291)(142.63248901,168.75866333)
\curveto(142.70248227,168.75865292)(142.76248221,168.74865293)(142.81248901,168.72866333)
\curveto(142.88248209,168.69865298)(142.93748203,168.63865304)(142.97748901,168.54866333)
\curveto(142.99748197,168.51865316)(143.00748196,168.4786532)(143.00748901,168.42866333)
\lineto(143.00748901,168.29366333)
\curveto(143.01748195,168.1836535)(143.01248196,168.0786536)(142.99248901,167.97866333)
\curveto(142.98248199,167.8786538)(142.94748202,167.80865387)(142.88748901,167.76866333)
\curveto(142.79748217,167.69865398)(142.66248231,167.66365402)(142.48248901,167.66366333)
\curveto(142.30248267,167.67365401)(142.13748283,167.678654)(141.98748901,167.67866333)
\lineto(139.99248901,167.67866333)
\lineto(139.49748901,167.67866333)
\lineto(139.36248901,167.67866333)
\curveto(139.32248565,167.678654)(139.28248569,167.67365401)(139.24248901,167.66366333)
\lineto(139.03248901,167.66366333)
\curveto(138.92248605,167.63365405)(138.84248613,167.59365409)(138.79248901,167.54366333)
\curveto(138.74248623,167.50365418)(138.70748626,167.44865423)(138.68748901,167.37866333)
\curveto(138.6674863,167.31865436)(138.65248632,167.24865443)(138.64248901,167.16866333)
\curveto(138.63248634,167.08865459)(138.61248636,166.99865468)(138.58248901,166.89866333)
\curveto(138.53248644,166.69865498)(138.49248648,166.49365519)(138.46248901,166.28366333)
\curveto(138.43248654,166.07365561)(138.39248658,165.86865581)(138.34248901,165.66866333)
\curveto(138.32248665,165.59865608)(138.31248666,165.52865615)(138.31248901,165.45866333)
\curveto(138.31248666,165.39865628)(138.30248667,165.33365635)(138.28248901,165.26366333)
\curveto(138.2724867,165.23365645)(138.26248671,165.19365649)(138.25248901,165.14366333)
\curveto(138.25248672,165.10365658)(138.25748671,165.06365662)(138.26748901,165.02366333)
\curveto(138.28748668,164.97365671)(138.31248666,164.92865675)(138.34248901,164.88866333)
\curveto(138.38248659,164.85865682)(138.44248653,164.85365683)(138.52248901,164.87366333)
\curveto(138.58248639,164.89365679)(138.64248633,164.91865676)(138.70248901,164.94866333)
\curveto(138.76248621,164.98865669)(138.82248615,165.02365666)(138.88248901,165.05366333)
\curveto(138.94248603,165.07365661)(138.99248598,165.08865659)(139.03248901,165.09866333)
\curveto(139.22248575,165.1786565)(139.42748554,165.23365645)(139.64748901,165.26366333)
\curveto(139.87748509,165.29365639)(140.10748486,165.30365638)(140.33748901,165.29366333)
\curveto(140.57748439,165.29365639)(140.80748416,165.26865641)(141.02748901,165.21866333)
\curveto(141.24748372,165.1786565)(141.44748352,165.11865656)(141.62748901,165.03866333)
\curveto(141.67748329,165.01865666)(141.72248325,164.99865668)(141.76248901,164.97866333)
\curveto(141.81248316,164.95865672)(141.86248311,164.93365675)(141.91248901,164.90366333)
\curveto(142.26248271,164.69365699)(142.54248243,164.46365722)(142.75248901,164.21366333)
\curveto(142.972482,163.96365772)(143.1674818,163.63865804)(143.33748901,163.23866333)
\curveto(143.38748158,163.12865855)(143.42248155,163.01865866)(143.44248901,162.90866333)
\curveto(143.46248151,162.79865888)(143.48748148,162.683659)(143.51748901,162.56366333)
\curveto(143.52748144,162.53365915)(143.53248144,162.48865919)(143.53248901,162.42866333)
\curveto(143.55248142,162.36865931)(143.56248141,162.29865938)(143.56248901,162.21866333)
\curveto(143.56248141,162.14865953)(143.5724814,162.0836596)(143.59248901,162.02366333)
\lineto(143.59248901,161.85866333)
\curveto(143.60248137,161.80865987)(143.60748136,161.73865994)(143.60748901,161.64866333)
\curveto(143.60748136,161.55866012)(143.59748137,161.48866019)(143.57748901,161.43866333)
\curveto(143.55748141,161.3786603)(143.55248142,161.31866036)(143.56248901,161.25866333)
\curveto(143.5724814,161.20866047)(143.5674814,161.15866052)(143.54748901,161.10866333)
\curveto(143.50748146,160.94866073)(143.4724815,160.79866088)(143.44248901,160.65866333)
\curveto(143.41248156,160.51866116)(143.3674816,160.3836613)(143.30748901,160.25366333)
\curveto(143.14748182,159.8836618)(142.92748204,159.54866213)(142.64748901,159.24866333)
\curveto(142.3674826,158.94866273)(142.04748292,158.71866296)(141.68748901,158.55866333)
\curveto(141.51748345,158.4786632)(141.31748365,158.40366328)(141.08748901,158.33366333)
\curveto(140.97748399,158.29366339)(140.86248411,158.26866341)(140.74248901,158.25866333)
\curveto(140.62248435,158.24866343)(140.50248447,158.22866345)(140.38248901,158.19866333)
\curveto(140.33248464,158.1786635)(140.27748469,158.1786635)(140.21748901,158.19866333)
\curveto(140.15748481,158.20866347)(140.09748487,158.20366348)(140.03748901,158.18366333)
\curveto(139.93748503,158.16366352)(139.83748513,158.16366352)(139.73748901,158.18366333)
\lineto(139.60248901,158.18366333)
\curveto(139.55248542,158.20366348)(139.49248548,158.21366347)(139.42248901,158.21366333)
\curveto(139.36248561,158.20366348)(139.30748566,158.20866347)(139.25748901,158.22866333)
\curveto(139.21748575,158.23866344)(139.18248579,158.24366344)(139.15248901,158.24366333)
\curveto(139.12248585,158.24366344)(139.08748588,158.24866343)(139.04748901,158.25866333)
\lineto(138.77748901,158.31866333)
\curveto(138.68748628,158.33866334)(138.60248637,158.36866331)(138.52248901,158.40866333)
\curveto(138.18248679,158.54866313)(137.89248708,158.70366298)(137.65248901,158.87366333)
\curveto(137.41248756,159.05366263)(137.19248778,159.2836624)(136.99248901,159.56366333)
\curveto(136.84248813,159.79366189)(136.72748824,160.03366165)(136.64748901,160.28366333)
\curveto(136.62748834,160.33366135)(136.61748835,160.3786613)(136.61748901,160.41866333)
\curveto(136.61748835,160.46866121)(136.60748836,160.51866116)(136.58748901,160.56866333)
\curveto(136.5674884,160.62866105)(136.55248842,160.70866097)(136.54248901,160.80866333)
\curveto(136.54248843,160.90866077)(136.56248841,160.9836607)(136.60248901,161.03366333)
\curveto(136.65248832,161.11366057)(136.73248824,161.15866052)(136.84248901,161.16866333)
\curveto(136.95248802,161.1786605)(137.0674879,161.1836605)(137.18748901,161.18366333)
\lineto(137.35248901,161.18366333)
\curveto(137.41248756,161.1836605)(137.4674875,161.17366051)(137.51748901,161.15366333)
\curveto(137.60748736,161.13366055)(137.67748729,161.09366059)(137.72748901,161.03366333)
\curveto(137.79748717,160.94366074)(137.84248713,160.83366085)(137.86248901,160.70366333)
\curveto(137.89248708,160.5836611)(137.93748703,160.4786612)(137.99748901,160.38866333)
\curveto(138.18748678,160.04866163)(138.44748652,159.7786619)(138.77748901,159.57866333)
\curveto(138.87748609,159.51866216)(138.98248599,159.46866221)(139.09248901,159.42866333)
\curveto(139.21248576,159.39866228)(139.33248564,159.36366232)(139.45248901,159.32366333)
\curveto(139.62248535,159.27366241)(139.82748514,159.25366243)(140.06748901,159.26366333)
\curveto(140.31748465,159.2836624)(140.51748445,159.31866236)(140.66748901,159.36866333)
\curveto(141.03748393,159.48866219)(141.32748364,159.64866203)(141.53748901,159.84866333)
\curveto(141.75748321,160.05866162)(141.93748303,160.33866134)(142.07748901,160.68866333)
\curveto(142.12748284,160.78866089)(142.15748281,160.89366079)(142.16748901,161.00366333)
\curveto(142.18748278,161.11366057)(142.21248276,161.22866045)(142.24248901,161.34866333)
\lineto(142.24248901,161.45366333)
\curveto(142.25248272,161.49366019)(142.25748271,161.53366015)(142.25748901,161.57366333)
\curveto(142.2674827,161.60366008)(142.2674827,161.63866004)(142.25748901,161.67866333)
\lineto(142.25748901,161.79866333)
\curveto(142.25748271,162.05865962)(142.22748274,162.30365938)(142.16748901,162.53366333)
\curveto(142.05748291,162.8836588)(141.90248307,163.1786585)(141.70248901,163.41866333)
\curveto(141.50248347,163.66865801)(141.24248373,163.86365782)(140.92248901,164.00366333)
\lineto(140.74248901,164.06366333)
\curveto(140.69248428,164.0836576)(140.63248434,164.10365758)(140.56248901,164.12366333)
\curveto(140.51248446,164.14365754)(140.45248452,164.15365753)(140.38248901,164.15366333)
\curveto(140.32248465,164.16365752)(140.25748471,164.1786575)(140.18748901,164.19866333)
\lineto(140.03748901,164.19866333)
\curveto(139.99748497,164.21865746)(139.94248503,164.22865745)(139.87248901,164.22866333)
\curveto(139.81248516,164.22865745)(139.75748521,164.21865746)(139.70748901,164.19866333)
\lineto(139.60248901,164.19866333)
\curveto(139.5724854,164.19865748)(139.53748543,164.19365749)(139.49748901,164.18366333)
\lineto(139.25748901,164.12366333)
\curveto(139.17748579,164.11365757)(139.09748587,164.09365759)(139.01748901,164.06366333)
\curveto(138.77748619,163.96365772)(138.54748642,163.82865785)(138.32748901,163.65866333)
\curveto(138.23748673,163.58865809)(138.15248682,163.51365817)(138.07248901,163.43366333)
\curveto(137.99248698,163.36365832)(137.89248708,163.30865837)(137.77248901,163.26866333)
\curveto(137.68248729,163.23865844)(137.54248743,163.22865845)(137.35248901,163.23866333)
\curveto(137.1724878,163.24865843)(137.05248792,163.27365841)(136.99248901,163.31366333)
\curveto(136.94248803,163.35365833)(136.90248807,163.41365827)(136.87248901,163.49366333)
\curveto(136.85248812,163.57365811)(136.85248812,163.65865802)(136.87248901,163.74866333)
\curveto(136.90248807,163.86865781)(136.92248805,163.98865769)(136.93248901,164.10866333)
\curveto(136.95248802,164.23865744)(136.97748799,164.36365732)(137.00748901,164.48366333)
\curveto(137.02748794,164.52365716)(137.03248794,164.55865712)(137.02248901,164.58866333)
\curveto(137.02248795,164.62865705)(137.03248794,164.67365701)(137.05248901,164.72366333)
\curveto(137.0724879,164.81365687)(137.08748788,164.90365678)(137.09748901,164.99366333)
\curveto(137.10748786,165.09365659)(137.12748784,165.18865649)(137.15748901,165.27866333)
\curveto(137.1674878,165.33865634)(137.1724878,165.39865628)(137.17248901,165.45866333)
\curveto(137.18248779,165.51865616)(137.19748777,165.5786561)(137.21748901,165.63866333)
\curveto(137.2674877,165.83865584)(137.30248767,166.04365564)(137.32248901,166.25366333)
\curveto(137.35248762,166.47365521)(137.39248758,166.683655)(137.44248901,166.88366333)
\curveto(137.4724875,166.9836547)(137.49248748,167.0836546)(137.50248901,167.18366333)
\curveto(137.51248746,167.2836544)(137.52748744,167.3836543)(137.54748901,167.48366333)
\curveto(137.55748741,167.51365417)(137.56248741,167.55365413)(137.56248901,167.60366333)
\curveto(137.59248738,167.71365397)(137.61248736,167.81865386)(137.62248901,167.91866333)
\curveto(137.64248733,168.02865365)(137.6674873,168.13865354)(137.69748901,168.24866333)
\curveto(137.71748725,168.32865335)(137.73248724,168.39865328)(137.74248901,168.45866333)
\curveto(137.75248722,168.52865315)(137.77748719,168.58865309)(137.81748901,168.63866333)
\curveto(137.83748713,168.66865301)(137.8674871,168.68865299)(137.90748901,168.69866333)
\curveto(137.94748702,168.71865296)(137.99248698,168.73865294)(138.04248901,168.75866333)
\curveto(138.10248687,168.75865292)(138.14248683,168.76365292)(138.16248901,168.77366333)
}
}
{
\newrgbcolor{curcolor}{0 0 0}
\pscustom[linestyle=none,fillstyle=solid,fillcolor=curcolor]
{
\newpath
\moveto(129.75288208,242.70223694)
\lineto(133.35288208,242.70223694)
\lineto(133.99788208,242.70223694)
\curveto(134.07787555,242.70222651)(134.15287548,242.69722652)(134.22288208,242.68723694)
\curveto(134.29287534,242.68722653)(134.35287528,242.67722654)(134.40288208,242.65723694)
\curveto(134.47287516,242.62722659)(134.5278751,242.56722665)(134.56788208,242.47723694)
\curveto(134.58787504,242.44722677)(134.59787503,242.40722681)(134.59788208,242.35723694)
\lineto(134.59788208,242.22223694)
\curveto(134.60787502,242.1122271)(134.60287503,242.00722721)(134.58288208,241.90723694)
\curveto(134.57287506,241.80722741)(134.53787509,241.73722748)(134.47788208,241.69723694)
\curveto(134.38787524,241.62722759)(134.25287538,241.59222762)(134.07288208,241.59223694)
\curveto(133.89287574,241.60222761)(133.7278759,241.60722761)(133.57788208,241.60723694)
\lineto(131.58288208,241.60723694)
\lineto(131.08788208,241.60723694)
\lineto(130.95288208,241.60723694)
\curveto(130.91287872,241.60722761)(130.87287876,241.60222761)(130.83288208,241.59223694)
\lineto(130.62288208,241.59223694)
\curveto(130.51287912,241.56222765)(130.4328792,241.52222769)(130.38288208,241.47223694)
\curveto(130.3328793,241.43222778)(130.29787933,241.37722784)(130.27788208,241.30723694)
\curveto(130.25787937,241.24722797)(130.24287939,241.17722804)(130.23288208,241.09723694)
\curveto(130.22287941,241.0172282)(130.20287943,240.92722829)(130.17288208,240.82723694)
\curveto(130.12287951,240.62722859)(130.08287955,240.42222879)(130.05288208,240.21223694)
\curveto(130.02287961,240.00222921)(129.98287965,239.79722942)(129.93288208,239.59723694)
\curveto(129.91287972,239.52722969)(129.90287973,239.45722976)(129.90288208,239.38723694)
\curveto(129.90287973,239.32722989)(129.89287974,239.26222995)(129.87288208,239.19223694)
\curveto(129.86287977,239.16223005)(129.85287978,239.12223009)(129.84288208,239.07223694)
\curveto(129.84287979,239.03223018)(129.84787978,238.99223022)(129.85788208,238.95223694)
\curveto(129.87787975,238.90223031)(129.90287973,238.85723036)(129.93288208,238.81723694)
\curveto(129.97287966,238.78723043)(130.0328796,238.78223043)(130.11288208,238.80223694)
\curveto(130.17287946,238.82223039)(130.2328794,238.84723037)(130.29288208,238.87723694)
\curveto(130.35287928,238.9172303)(130.41287922,238.95223026)(130.47288208,238.98223694)
\curveto(130.5328791,239.00223021)(130.58287905,239.0172302)(130.62288208,239.02723694)
\curveto(130.81287882,239.10723011)(131.01787861,239.16223005)(131.23788208,239.19223694)
\curveto(131.46787816,239.22222999)(131.69787793,239.23222998)(131.92788208,239.22223694)
\curveto(132.16787746,239.22222999)(132.39787723,239.19723002)(132.61788208,239.14723694)
\curveto(132.83787679,239.10723011)(133.03787659,239.04723017)(133.21788208,238.96723694)
\curveto(133.26787636,238.94723027)(133.31287632,238.92723029)(133.35288208,238.90723694)
\curveto(133.40287623,238.88723033)(133.45287618,238.86223035)(133.50288208,238.83223694)
\curveto(133.85287578,238.62223059)(134.1328755,238.39223082)(134.34288208,238.14223694)
\curveto(134.56287507,237.89223132)(134.75787487,237.56723165)(134.92788208,237.16723694)
\curveto(134.97787465,237.05723216)(135.01287462,236.94723227)(135.03288208,236.83723694)
\curveto(135.05287458,236.72723249)(135.07787455,236.6122326)(135.10788208,236.49223694)
\curveto(135.11787451,236.46223275)(135.12287451,236.4172328)(135.12288208,236.35723694)
\curveto(135.14287449,236.29723292)(135.15287448,236.22723299)(135.15288208,236.14723694)
\curveto(135.15287448,236.07723314)(135.16287447,236.0122332)(135.18288208,235.95223694)
\lineto(135.18288208,235.78723694)
\curveto(135.19287444,235.73723348)(135.19787443,235.66723355)(135.19788208,235.57723694)
\curveto(135.19787443,235.48723373)(135.18787444,235.4172338)(135.16788208,235.36723694)
\curveto(135.14787448,235.30723391)(135.14287449,235.24723397)(135.15288208,235.18723694)
\curveto(135.16287447,235.13723408)(135.15787447,235.08723413)(135.13788208,235.03723694)
\curveto(135.09787453,234.87723434)(135.06287457,234.72723449)(135.03288208,234.58723694)
\curveto(135.00287463,234.44723477)(134.95787467,234.3122349)(134.89788208,234.18223694)
\curveto(134.73787489,233.8122354)(134.51787511,233.47723574)(134.23788208,233.17723694)
\curveto(133.95787567,232.87723634)(133.63787599,232.64723657)(133.27788208,232.48723694)
\curveto(133.10787652,232.40723681)(132.90787672,232.33223688)(132.67788208,232.26223694)
\curveto(132.56787706,232.22223699)(132.45287718,232.19723702)(132.33288208,232.18723694)
\curveto(132.21287742,232.17723704)(132.09287754,232.15723706)(131.97288208,232.12723694)
\curveto(131.92287771,232.10723711)(131.86787776,232.10723711)(131.80788208,232.12723694)
\curveto(131.74787788,232.13723708)(131.68787794,232.13223708)(131.62788208,232.11223694)
\curveto(131.5278781,232.09223712)(131.4278782,232.09223712)(131.32788208,232.11223694)
\lineto(131.19288208,232.11223694)
\curveto(131.14287849,232.13223708)(131.08287855,232.14223707)(131.01288208,232.14223694)
\curveto(130.95287868,232.13223708)(130.89787873,232.13723708)(130.84788208,232.15723694)
\curveto(130.80787882,232.16723705)(130.77287886,232.17223704)(130.74288208,232.17223694)
\curveto(130.71287892,232.17223704)(130.67787895,232.17723704)(130.63788208,232.18723694)
\lineto(130.36788208,232.24723694)
\curveto(130.27787935,232.26723695)(130.19287944,232.29723692)(130.11288208,232.33723694)
\curveto(129.77287986,232.47723674)(129.48288015,232.63223658)(129.24288208,232.80223694)
\curveto(129.00288063,232.98223623)(128.78288085,233.212236)(128.58288208,233.49223694)
\curveto(128.4328812,233.72223549)(128.31788131,233.96223525)(128.23788208,234.21223694)
\curveto(128.21788141,234.26223495)(128.20788142,234.30723491)(128.20788208,234.34723694)
\curveto(128.20788142,234.39723482)(128.19788143,234.44723477)(128.17788208,234.49723694)
\curveto(128.15788147,234.55723466)(128.14288149,234.63723458)(128.13288208,234.73723694)
\curveto(128.1328815,234.83723438)(128.15288148,234.9122343)(128.19288208,234.96223694)
\curveto(128.24288139,235.04223417)(128.32288131,235.08723413)(128.43288208,235.09723694)
\curveto(128.54288109,235.10723411)(128.65788097,235.1122341)(128.77788208,235.11223694)
\lineto(128.94288208,235.11223694)
\curveto(129.00288063,235.1122341)(129.05788057,235.10223411)(129.10788208,235.08223694)
\curveto(129.19788043,235.06223415)(129.26788036,235.02223419)(129.31788208,234.96223694)
\curveto(129.38788024,234.87223434)(129.4328802,234.76223445)(129.45288208,234.63223694)
\curveto(129.48288015,234.5122347)(129.5278801,234.40723481)(129.58788208,234.31723694)
\curveto(129.77787985,233.97723524)(130.03787959,233.70723551)(130.36788208,233.50723694)
\curveto(130.46787916,233.44723577)(130.57287906,233.39723582)(130.68288208,233.35723694)
\curveto(130.80287883,233.32723589)(130.92287871,233.29223592)(131.04288208,233.25223694)
\curveto(131.21287842,233.20223601)(131.41787821,233.18223603)(131.65788208,233.19223694)
\curveto(131.90787772,233.212236)(132.10787752,233.24723597)(132.25788208,233.29723694)
\curveto(132.627877,233.4172358)(132.91787671,233.57723564)(133.12788208,233.77723694)
\curveto(133.34787628,233.98723523)(133.5278761,234.26723495)(133.66788208,234.61723694)
\curveto(133.71787591,234.7172345)(133.74787588,234.82223439)(133.75788208,234.93223694)
\curveto(133.77787585,235.04223417)(133.80287583,235.15723406)(133.83288208,235.27723694)
\lineto(133.83288208,235.38223694)
\curveto(133.84287579,235.42223379)(133.84787578,235.46223375)(133.84788208,235.50223694)
\curveto(133.85787577,235.53223368)(133.85787577,235.56723365)(133.84788208,235.60723694)
\lineto(133.84788208,235.72723694)
\curveto(133.84787578,235.98723323)(133.81787581,236.23223298)(133.75788208,236.46223694)
\curveto(133.64787598,236.8122324)(133.49287614,237.10723211)(133.29288208,237.34723694)
\curveto(133.09287654,237.59723162)(132.8328768,237.79223142)(132.51288208,237.93223694)
\lineto(132.33288208,237.99223694)
\curveto(132.28287735,238.0122312)(132.22287741,238.03223118)(132.15288208,238.05223694)
\curveto(132.10287753,238.07223114)(132.04287759,238.08223113)(131.97288208,238.08223694)
\curveto(131.91287772,238.09223112)(131.84787778,238.10723111)(131.77788208,238.12723694)
\lineto(131.62788208,238.12723694)
\curveto(131.58787804,238.14723107)(131.5328781,238.15723106)(131.46288208,238.15723694)
\curveto(131.40287823,238.15723106)(131.34787828,238.14723107)(131.29788208,238.12723694)
\lineto(131.19288208,238.12723694)
\curveto(131.16287847,238.12723109)(131.1278785,238.12223109)(131.08788208,238.11223694)
\lineto(130.84788208,238.05223694)
\curveto(130.76787886,238.04223117)(130.68787894,238.02223119)(130.60788208,237.99223694)
\curveto(130.36787926,237.89223132)(130.13787949,237.75723146)(129.91788208,237.58723694)
\curveto(129.8278798,237.5172317)(129.74287989,237.44223177)(129.66288208,237.36223694)
\curveto(129.58288005,237.29223192)(129.48288015,237.23723198)(129.36288208,237.19723694)
\curveto(129.27288036,237.16723205)(129.1328805,237.15723206)(128.94288208,237.16723694)
\curveto(128.76288087,237.17723204)(128.64288099,237.20223201)(128.58288208,237.24223694)
\curveto(128.5328811,237.28223193)(128.49288114,237.34223187)(128.46288208,237.42223694)
\curveto(128.44288119,237.50223171)(128.44288119,237.58723163)(128.46288208,237.67723694)
\curveto(128.49288114,237.79723142)(128.51288112,237.9172313)(128.52288208,238.03723694)
\curveto(128.54288109,238.16723105)(128.56788106,238.29223092)(128.59788208,238.41223694)
\curveto(128.61788101,238.45223076)(128.62288101,238.48723073)(128.61288208,238.51723694)
\curveto(128.61288102,238.55723066)(128.62288101,238.60223061)(128.64288208,238.65223694)
\curveto(128.66288097,238.74223047)(128.67788095,238.83223038)(128.68788208,238.92223694)
\curveto(128.69788093,239.02223019)(128.71788091,239.1172301)(128.74788208,239.20723694)
\curveto(128.75788087,239.26722995)(128.76288087,239.32722989)(128.76288208,239.38723694)
\curveto(128.77288086,239.44722977)(128.78788084,239.50722971)(128.80788208,239.56723694)
\curveto(128.85788077,239.76722945)(128.89288074,239.97222924)(128.91288208,240.18223694)
\curveto(128.94288069,240.40222881)(128.98288065,240.6122286)(129.03288208,240.81223694)
\curveto(129.06288057,240.9122283)(129.08288055,241.0122282)(129.09288208,241.11223694)
\curveto(129.10288053,241.212228)(129.11788051,241.3122279)(129.13788208,241.41223694)
\curveto(129.14788048,241.44222777)(129.15288048,241.48222773)(129.15288208,241.53223694)
\curveto(129.18288045,241.64222757)(129.20288043,241.74722747)(129.21288208,241.84723694)
\curveto(129.2328804,241.95722726)(129.25788037,242.06722715)(129.28788208,242.17723694)
\curveto(129.30788032,242.25722696)(129.32288031,242.32722689)(129.33288208,242.38723694)
\curveto(129.34288029,242.45722676)(129.36788026,242.5172267)(129.40788208,242.56723694)
\curveto(129.4278802,242.59722662)(129.45788017,242.6172266)(129.49788208,242.62723694)
\curveto(129.53788009,242.64722657)(129.58288005,242.66722655)(129.63288208,242.68723694)
\curveto(129.69287994,242.68722653)(129.7328799,242.69222652)(129.75288208,242.70223694)
}
}
{
\newrgbcolor{curcolor}{0 0 0}
\pscustom[linestyle=none,fillstyle=solid,fillcolor=curcolor]
{
\newpath
\moveto(143.59249146,237.37723694)
\lineto(143.59249146,237.12223694)
\curveto(143.60248375,237.04223217)(143.59748376,236.96723225)(143.57749146,236.89723694)
\lineto(143.57749146,236.65723694)
\lineto(143.57749146,236.49223694)
\curveto(143.5574838,236.39223282)(143.54748381,236.28723293)(143.54749146,236.17723694)
\curveto(143.54748381,236.07723314)(143.53748382,235.97723324)(143.51749146,235.87723694)
\lineto(143.51749146,235.72723694)
\curveto(143.48748387,235.58723363)(143.46748389,235.44723377)(143.45749146,235.30723694)
\curveto(143.44748391,235.17723404)(143.42248393,235.04723417)(143.38249146,234.91723694)
\curveto(143.36248399,234.83723438)(143.34248401,234.75223446)(143.32249146,234.66223694)
\lineto(143.26249146,234.42223694)
\lineto(143.14249146,234.12223694)
\curveto(143.11248424,234.03223518)(143.07748428,233.94223527)(143.03749146,233.85223694)
\curveto(142.93748442,233.63223558)(142.80248455,233.4172358)(142.63249146,233.20723694)
\curveto(142.47248488,232.99723622)(142.29748506,232.82723639)(142.10749146,232.69723694)
\curveto(142.0574853,232.65723656)(141.99748536,232.6172366)(141.92749146,232.57723694)
\curveto(141.86748549,232.54723667)(141.80748555,232.5122367)(141.74749146,232.47223694)
\curveto(141.66748569,232.42223679)(141.57248578,232.38223683)(141.46249146,232.35223694)
\curveto(141.352486,232.32223689)(141.24748611,232.29223692)(141.14749146,232.26223694)
\curveto(141.03748632,232.22223699)(140.92748643,232.19723702)(140.81749146,232.18723694)
\curveto(140.70748665,232.17723704)(140.59248676,232.16223705)(140.47249146,232.14223694)
\curveto(140.43248692,232.13223708)(140.38748697,232.13223708)(140.33749146,232.14223694)
\curveto(140.29748706,232.14223707)(140.2574871,232.13723708)(140.21749146,232.12723694)
\curveto(140.17748718,232.1172371)(140.12248723,232.1122371)(140.05249146,232.11223694)
\curveto(139.98248737,232.1122371)(139.93248742,232.1172371)(139.90249146,232.12723694)
\curveto(139.8524875,232.14723707)(139.80748755,232.15223706)(139.76749146,232.14223694)
\curveto(139.72748763,232.13223708)(139.69248766,232.13223708)(139.66249146,232.14223694)
\lineto(139.57249146,232.14223694)
\curveto(139.51248784,232.16223705)(139.44748791,232.17723704)(139.37749146,232.18723694)
\curveto(139.31748804,232.18723703)(139.2524881,232.19223702)(139.18249146,232.20223694)
\curveto(139.01248834,232.25223696)(138.8524885,232.30223691)(138.70249146,232.35223694)
\curveto(138.5524888,232.40223681)(138.40748895,232.46723675)(138.26749146,232.54723694)
\curveto(138.21748914,232.58723663)(138.16248919,232.6172366)(138.10249146,232.63723694)
\curveto(138.0524893,232.66723655)(138.00248935,232.70223651)(137.95249146,232.74223694)
\curveto(137.71248964,232.92223629)(137.51248984,233.14223607)(137.35249146,233.40223694)
\curveto(137.19249016,233.66223555)(137.0524903,233.94723527)(136.93249146,234.25723694)
\curveto(136.87249048,234.39723482)(136.82749053,234.53723468)(136.79749146,234.67723694)
\curveto(136.76749059,234.82723439)(136.73249062,234.98223423)(136.69249146,235.14223694)
\curveto(136.67249068,235.25223396)(136.6574907,235.36223385)(136.64749146,235.47223694)
\curveto(136.63749072,235.58223363)(136.62249073,235.69223352)(136.60249146,235.80223694)
\curveto(136.59249076,235.84223337)(136.58749077,235.88223333)(136.58749146,235.92223694)
\curveto(136.59749076,235.96223325)(136.59749076,236.00223321)(136.58749146,236.04223694)
\curveto(136.57749078,236.09223312)(136.57249078,236.14223307)(136.57249146,236.19223694)
\lineto(136.57249146,236.35723694)
\curveto(136.5524908,236.40723281)(136.54749081,236.45723276)(136.55749146,236.50723694)
\curveto(136.56749079,236.56723265)(136.56749079,236.62223259)(136.55749146,236.67223694)
\curveto(136.54749081,236.7122325)(136.54749081,236.75723246)(136.55749146,236.80723694)
\curveto(136.56749079,236.85723236)(136.56249079,236.90723231)(136.54249146,236.95723694)
\curveto(136.52249083,237.02723219)(136.51749084,237.10223211)(136.52749146,237.18223694)
\curveto(136.53749082,237.27223194)(136.54249081,237.35723186)(136.54249146,237.43723694)
\curveto(136.54249081,237.52723169)(136.53749082,237.62723159)(136.52749146,237.73723694)
\curveto(136.51749084,237.85723136)(136.52249083,237.95723126)(136.54249146,238.03723694)
\lineto(136.54249146,238.32223694)
\lineto(136.58749146,238.95223694)
\curveto(136.59749076,239.05223016)(136.60749075,239.14723007)(136.61749146,239.23723694)
\lineto(136.64749146,239.53723694)
\curveto(136.66749069,239.58722963)(136.67249068,239.63722958)(136.66249146,239.68723694)
\curveto(136.66249069,239.74722947)(136.67249068,239.80222941)(136.69249146,239.85223694)
\curveto(136.74249061,240.02222919)(136.78249057,240.18722903)(136.81249146,240.34723694)
\curveto(136.84249051,240.5172287)(136.89249046,240.67722854)(136.96249146,240.82723694)
\curveto(137.1524902,241.28722793)(137.37248998,241.66222755)(137.62249146,241.95223694)
\curveto(137.88248947,242.24222697)(138.24248911,242.48722673)(138.70249146,242.68723694)
\curveto(138.83248852,242.73722648)(138.96248839,242.77222644)(139.09249146,242.79223694)
\curveto(139.23248812,242.8122264)(139.37248798,242.83722638)(139.51249146,242.86723694)
\curveto(139.58248777,242.87722634)(139.64748771,242.88222633)(139.70749146,242.88223694)
\curveto(139.76748759,242.88222633)(139.83248752,242.88722633)(139.90249146,242.89723694)
\curveto(140.73248662,242.9172263)(141.40248595,242.76722645)(141.91249146,242.44723694)
\curveto(142.42248493,242.13722708)(142.80248455,241.69722752)(143.05249146,241.12723694)
\curveto(143.10248425,241.00722821)(143.14748421,240.88222833)(143.18749146,240.75223694)
\curveto(143.22748413,240.62222859)(143.27248408,240.48722873)(143.32249146,240.34723694)
\curveto(143.34248401,240.26722895)(143.357484,240.18222903)(143.36749146,240.09223694)
\lineto(143.42749146,239.85223694)
\curveto(143.4574839,239.74222947)(143.47248388,239.63222958)(143.47249146,239.52223694)
\curveto(143.48248387,239.4122298)(143.49748386,239.30222991)(143.51749146,239.19223694)
\curveto(143.53748382,239.14223007)(143.54248381,239.09723012)(143.53249146,239.05723694)
\curveto(143.53248382,239.0172302)(143.53748382,238.97723024)(143.54749146,238.93723694)
\curveto(143.5574838,238.88723033)(143.5574838,238.83223038)(143.54749146,238.77223694)
\curveto(143.54748381,238.72223049)(143.5524838,238.67223054)(143.56249146,238.62223694)
\lineto(143.56249146,238.48723694)
\curveto(143.58248377,238.42723079)(143.58248377,238.35723086)(143.56249146,238.27723694)
\curveto(143.5524838,238.20723101)(143.5574838,238.14223107)(143.57749146,238.08223694)
\curveto(143.58748377,238.05223116)(143.59248376,238.0122312)(143.59249146,237.96223694)
\lineto(143.59249146,237.84223694)
\lineto(143.59249146,237.37723694)
\moveto(142.04749146,235.05223694)
\curveto(142.14748521,235.37223384)(142.20748515,235.73723348)(142.22749146,236.14723694)
\curveto(142.24748511,236.55723266)(142.2574851,236.96723225)(142.25749146,237.37723694)
\curveto(142.2574851,237.80723141)(142.24748511,238.22723099)(142.22749146,238.63723694)
\curveto(142.20748515,239.04723017)(142.16248519,239.43222978)(142.09249146,239.79223694)
\curveto(142.02248533,240.15222906)(141.91248544,240.47222874)(141.76249146,240.75223694)
\curveto(141.62248573,241.04222817)(141.42748593,241.27722794)(141.17749146,241.45723694)
\curveto(141.01748634,241.56722765)(140.83748652,241.64722757)(140.63749146,241.69723694)
\curveto(140.43748692,241.75722746)(140.19248716,241.78722743)(139.90249146,241.78723694)
\curveto(139.88248747,241.76722745)(139.84748751,241.75722746)(139.79749146,241.75723694)
\curveto(139.74748761,241.76722745)(139.70748765,241.76722745)(139.67749146,241.75723694)
\curveto(139.59748776,241.73722748)(139.52248783,241.7172275)(139.45249146,241.69723694)
\curveto(139.39248796,241.68722753)(139.32748803,241.66722755)(139.25749146,241.63723694)
\curveto(138.98748837,241.5172277)(138.76748859,241.34722787)(138.59749146,241.12723694)
\curveto(138.43748892,240.9172283)(138.30248905,240.67222854)(138.19249146,240.39223694)
\curveto(138.14248921,240.28222893)(138.10248925,240.16222905)(138.07249146,240.03223694)
\curveto(138.0524893,239.9122293)(138.02748933,239.78722943)(137.99749146,239.65723694)
\curveto(137.97748938,239.60722961)(137.96748939,239.55222966)(137.96749146,239.49223694)
\curveto(137.96748939,239.44222977)(137.96248939,239.39222982)(137.95249146,239.34223694)
\curveto(137.94248941,239.25222996)(137.93248942,239.15723006)(137.92249146,239.05723694)
\curveto(137.91248944,238.96723025)(137.90248945,238.87223034)(137.89249146,238.77223694)
\curveto(137.89248946,238.69223052)(137.88748947,238.60723061)(137.87749146,238.51723694)
\lineto(137.87749146,238.27723694)
\lineto(137.87749146,238.09723694)
\curveto(137.86748949,238.06723115)(137.86248949,238.03223118)(137.86249146,237.99223694)
\lineto(137.86249146,237.85723694)
\lineto(137.86249146,237.40723694)
\curveto(137.86248949,237.32723189)(137.8574895,237.24223197)(137.84749146,237.15223694)
\curveto(137.84748951,237.07223214)(137.8574895,236.99723222)(137.87749146,236.92723694)
\lineto(137.87749146,236.65723694)
\curveto(137.87748948,236.63723258)(137.87248948,236.60723261)(137.86249146,236.56723694)
\curveto(137.86248949,236.53723268)(137.86748949,236.5122327)(137.87749146,236.49223694)
\curveto(137.88748947,236.39223282)(137.89248946,236.29223292)(137.89249146,236.19223694)
\curveto(137.90248945,236.10223311)(137.91248944,236.00223321)(137.92249146,235.89223694)
\curveto(137.9524894,235.77223344)(137.96748939,235.64723357)(137.96749146,235.51723694)
\curveto(137.97748938,235.39723382)(138.00248935,235.28223393)(138.04249146,235.17223694)
\curveto(138.12248923,234.87223434)(138.20748915,234.60723461)(138.29749146,234.37723694)
\curveto(138.39748896,234.14723507)(138.54248881,233.93223528)(138.73249146,233.73223694)
\curveto(138.94248841,233.53223568)(139.20748815,233.38223583)(139.52749146,233.28223694)
\curveto(139.56748779,233.26223595)(139.60248775,233.25223596)(139.63249146,233.25223694)
\curveto(139.67248768,233.26223595)(139.71748764,233.25723596)(139.76749146,233.23723694)
\curveto(139.80748755,233.22723599)(139.87748748,233.217236)(139.97749146,233.20723694)
\curveto(140.08748727,233.19723602)(140.17248718,233.20223601)(140.23249146,233.22223694)
\curveto(140.30248705,233.24223597)(140.37248698,233.25223596)(140.44249146,233.25223694)
\curveto(140.51248684,233.26223595)(140.57748678,233.27723594)(140.63749146,233.29723694)
\curveto(140.83748652,233.35723586)(141.01748634,233.44223577)(141.17749146,233.55223694)
\curveto(141.20748615,233.57223564)(141.23248612,233.59223562)(141.25249146,233.61223694)
\lineto(141.31249146,233.67223694)
\curveto(141.352486,233.69223552)(141.40248595,233.73223548)(141.46249146,233.79223694)
\curveto(141.56248579,233.93223528)(141.64748571,234.06223515)(141.71749146,234.18223694)
\curveto(141.78748557,234.30223491)(141.8574855,234.44723477)(141.92749146,234.61723694)
\curveto(141.9574854,234.68723453)(141.97748538,234.75723446)(141.98749146,234.82723694)
\curveto(142.00748535,234.89723432)(142.02748533,234.97223424)(142.04749146,235.05223694)
}
}
{
\newrgbcolor{curcolor}{0 0 0}
\pscustom[linestyle=none,fillstyle=solid,fillcolor=curcolor]
{
\newpath
\moveto(128.76287964,317.70223694)
\lineto(133.56287964,317.70223694)
\lineto(134.56787964,317.70223694)
\curveto(134.70787254,317.70222651)(134.82787242,317.69222652)(134.92787964,317.67223694)
\curveto(135.03787221,317.66222655)(135.11787213,317.6172266)(135.16787964,317.53723694)
\curveto(135.18787206,317.49722672)(135.19787205,317.44722677)(135.19787964,317.38723694)
\curveto(135.20787204,317.32722689)(135.21287203,317.26222695)(135.21287964,317.19223694)
\lineto(135.21287964,316.92223694)
\curveto(135.21287203,316.83222738)(135.20287204,316.75222746)(135.18287964,316.68223694)
\curveto(135.1428721,316.60222761)(135.09787215,316.53222768)(135.04787964,316.47223694)
\lineto(134.89787964,316.29223694)
\curveto(134.86787238,316.24222797)(134.83287241,316.20222801)(134.79287964,316.17223694)
\curveto(134.75287249,316.14222807)(134.71287253,316.10222811)(134.67287964,316.05223694)
\curveto(134.59287265,315.94222827)(134.50787274,315.83222838)(134.41787964,315.72223694)
\curveto(134.32787292,315.62222859)(134.242873,315.5172287)(134.16287964,315.40723694)
\curveto(134.02287322,315.20722901)(133.88287336,314.99722922)(133.74287964,314.77723694)
\curveto(133.60287364,314.56722965)(133.46287378,314.35222986)(133.32287964,314.13223694)
\curveto(133.27287397,314.04223017)(133.22287402,313.94723027)(133.17287964,313.84723694)
\curveto(133.12287412,313.74723047)(133.06787418,313.65223056)(133.00787964,313.56223694)
\curveto(132.98787426,313.54223067)(132.97787427,313.5172307)(132.97787964,313.48723694)
\curveto(132.97787427,313.45723076)(132.96787428,313.43223078)(132.94787964,313.41223694)
\curveto(132.87787437,313.3122309)(132.81287443,313.19723102)(132.75287964,313.06723694)
\curveto(132.69287455,312.94723127)(132.63787461,312.83223138)(132.58787964,312.72223694)
\curveto(132.48787476,312.49223172)(132.39287485,312.25723196)(132.30287964,312.01723694)
\curveto(132.21287503,311.77723244)(132.11287513,311.53723268)(132.00287964,311.29723694)
\curveto(131.98287526,311.24723297)(131.96787528,311.20223301)(131.95787964,311.16223694)
\curveto(131.95787529,311.12223309)(131.9478753,311.07723314)(131.92787964,311.02723694)
\curveto(131.87787537,310.90723331)(131.83287541,310.78223343)(131.79287964,310.65223694)
\curveto(131.76287548,310.53223368)(131.72787552,310.4122338)(131.68787964,310.29223694)
\curveto(131.60787564,310.06223415)(131.5428757,309.82223439)(131.49287964,309.57223694)
\curveto(131.45287579,309.33223488)(131.40287584,309.09223512)(131.34287964,308.85223694)
\curveto(131.30287594,308.70223551)(131.27787597,308.55223566)(131.26787964,308.40223694)
\curveto(131.25787599,308.25223596)(131.23787601,308.10223611)(131.20787964,307.95223694)
\curveto(131.19787605,307.9122363)(131.19287605,307.85223636)(131.19287964,307.77223694)
\curveto(131.16287608,307.65223656)(131.13287611,307.55223666)(131.10287964,307.47223694)
\curveto(131.07287617,307.39223682)(131.00287624,307.33723688)(130.89287964,307.30723694)
\curveto(130.8428764,307.28723693)(130.78787646,307.27723694)(130.72787964,307.27723694)
\lineto(130.53287964,307.27723694)
\curveto(130.39287685,307.27723694)(130.25287699,307.28223693)(130.11287964,307.29223694)
\curveto(129.98287726,307.30223691)(129.88787736,307.34723687)(129.82787964,307.42723694)
\curveto(129.78787746,307.48723673)(129.76787748,307.57223664)(129.76787964,307.68223694)
\curveto(129.77787747,307.79223642)(129.79287745,307.88723633)(129.81287964,307.96723694)
\lineto(129.81287964,308.04223694)
\curveto(129.82287742,308.07223614)(129.82787742,308.10223611)(129.82787964,308.13223694)
\curveto(129.8478774,308.212236)(129.85787739,308.28723593)(129.85787964,308.35723694)
\curveto(129.85787739,308.42723579)(129.86787738,308.49723572)(129.88787964,308.56723694)
\curveto(129.93787731,308.75723546)(129.97787727,308.94223527)(130.00787964,309.12223694)
\curveto(130.03787721,309.3122349)(130.07787717,309.49223472)(130.12787964,309.66223694)
\curveto(130.1478771,309.7122345)(130.15787709,309.75223446)(130.15787964,309.78223694)
\curveto(130.15787709,309.8122344)(130.16287708,309.84723437)(130.17287964,309.88723694)
\curveto(130.27287697,310.18723403)(130.36287688,310.48223373)(130.44287964,310.77223694)
\curveto(130.53287671,311.06223315)(130.63787661,311.34223287)(130.75787964,311.61223694)
\curveto(131.01787623,312.19223202)(131.28787596,312.74223147)(131.56787964,313.26223694)
\curveto(131.8478754,313.79223042)(132.15787509,314.29722992)(132.49787964,314.77723694)
\curveto(132.63787461,314.97722924)(132.78787446,315.16722905)(132.94787964,315.34723694)
\curveto(133.10787414,315.53722868)(133.25787399,315.72722849)(133.39787964,315.91723694)
\curveto(133.43787381,315.96722825)(133.47287377,316.0122282)(133.50287964,316.05223694)
\curveto(133.5428737,316.10222811)(133.57787367,316.15222806)(133.60787964,316.20223694)
\curveto(133.61787363,316.22222799)(133.62787362,316.24722797)(133.63787964,316.27723694)
\curveto(133.65787359,316.30722791)(133.65787359,316.33722788)(133.63787964,316.36723694)
\curveto(133.61787363,316.42722779)(133.58287366,316.46222775)(133.53287964,316.47223694)
\curveto(133.48287376,316.49222772)(133.43287381,316.5122277)(133.38287964,316.53223694)
\lineto(133.27787964,316.53223694)
\curveto(133.23787401,316.54222767)(133.18787406,316.54222767)(133.12787964,316.53223694)
\lineto(132.97787964,316.53223694)
\lineto(132.37787964,316.53223694)
\lineto(129.73787964,316.53223694)
\lineto(129.00287964,316.53223694)
\lineto(128.76287964,316.53223694)
\curveto(128.69287855,316.54222767)(128.63287861,316.55722766)(128.58287964,316.57723694)
\curveto(128.49287875,316.6172276)(128.43287881,316.67722754)(128.40287964,316.75723694)
\curveto(128.35287889,316.85722736)(128.33787891,317.00222721)(128.35787964,317.19223694)
\curveto(128.37787887,317.39222682)(128.41287883,317.52722669)(128.46287964,317.59723694)
\curveto(128.48287876,317.6172266)(128.50787874,317.63222658)(128.53787964,317.64223694)
\lineto(128.65787964,317.70223694)
\curveto(128.67787857,317.70222651)(128.69287855,317.69722652)(128.70287964,317.68723694)
\curveto(128.72287852,317.68722653)(128.7428785,317.69222652)(128.76287964,317.70223694)
}
}
{
\newrgbcolor{curcolor}{0 0 0}
\pscustom[linestyle=none,fillstyle=solid,fillcolor=curcolor]
{
\newpath
\moveto(138.16248901,317.70223694)
\lineto(141.76248901,317.70223694)
\lineto(142.40748901,317.70223694)
\curveto(142.48748248,317.70222651)(142.56248241,317.69722652)(142.63248901,317.68723694)
\curveto(142.70248227,317.68722653)(142.76248221,317.67722654)(142.81248901,317.65723694)
\curveto(142.88248209,317.62722659)(142.93748203,317.56722665)(142.97748901,317.47723694)
\curveto(142.99748197,317.44722677)(143.00748196,317.40722681)(143.00748901,317.35723694)
\lineto(143.00748901,317.22223694)
\curveto(143.01748195,317.1122271)(143.01248196,317.00722721)(142.99248901,316.90723694)
\curveto(142.98248199,316.80722741)(142.94748202,316.73722748)(142.88748901,316.69723694)
\curveto(142.79748217,316.62722759)(142.66248231,316.59222762)(142.48248901,316.59223694)
\curveto(142.30248267,316.60222761)(142.13748283,316.60722761)(141.98748901,316.60723694)
\lineto(139.99248901,316.60723694)
\lineto(139.49748901,316.60723694)
\lineto(139.36248901,316.60723694)
\curveto(139.32248565,316.60722761)(139.28248569,316.60222761)(139.24248901,316.59223694)
\lineto(139.03248901,316.59223694)
\curveto(138.92248605,316.56222765)(138.84248613,316.52222769)(138.79248901,316.47223694)
\curveto(138.74248623,316.43222778)(138.70748626,316.37722784)(138.68748901,316.30723694)
\curveto(138.6674863,316.24722797)(138.65248632,316.17722804)(138.64248901,316.09723694)
\curveto(138.63248634,316.0172282)(138.61248636,315.92722829)(138.58248901,315.82723694)
\curveto(138.53248644,315.62722859)(138.49248648,315.42222879)(138.46248901,315.21223694)
\curveto(138.43248654,315.00222921)(138.39248658,314.79722942)(138.34248901,314.59723694)
\curveto(138.32248665,314.52722969)(138.31248666,314.45722976)(138.31248901,314.38723694)
\curveto(138.31248666,314.32722989)(138.30248667,314.26222995)(138.28248901,314.19223694)
\curveto(138.2724867,314.16223005)(138.26248671,314.12223009)(138.25248901,314.07223694)
\curveto(138.25248672,314.03223018)(138.25748671,313.99223022)(138.26748901,313.95223694)
\curveto(138.28748668,313.90223031)(138.31248666,313.85723036)(138.34248901,313.81723694)
\curveto(138.38248659,313.78723043)(138.44248653,313.78223043)(138.52248901,313.80223694)
\curveto(138.58248639,313.82223039)(138.64248633,313.84723037)(138.70248901,313.87723694)
\curveto(138.76248621,313.9172303)(138.82248615,313.95223026)(138.88248901,313.98223694)
\curveto(138.94248603,314.00223021)(138.99248598,314.0172302)(139.03248901,314.02723694)
\curveto(139.22248575,314.10723011)(139.42748554,314.16223005)(139.64748901,314.19223694)
\curveto(139.87748509,314.22222999)(140.10748486,314.23222998)(140.33748901,314.22223694)
\curveto(140.57748439,314.22222999)(140.80748416,314.19723002)(141.02748901,314.14723694)
\curveto(141.24748372,314.10723011)(141.44748352,314.04723017)(141.62748901,313.96723694)
\curveto(141.67748329,313.94723027)(141.72248325,313.92723029)(141.76248901,313.90723694)
\curveto(141.81248316,313.88723033)(141.86248311,313.86223035)(141.91248901,313.83223694)
\curveto(142.26248271,313.62223059)(142.54248243,313.39223082)(142.75248901,313.14223694)
\curveto(142.972482,312.89223132)(143.1674818,312.56723165)(143.33748901,312.16723694)
\curveto(143.38748158,312.05723216)(143.42248155,311.94723227)(143.44248901,311.83723694)
\curveto(143.46248151,311.72723249)(143.48748148,311.6122326)(143.51748901,311.49223694)
\curveto(143.52748144,311.46223275)(143.53248144,311.4172328)(143.53248901,311.35723694)
\curveto(143.55248142,311.29723292)(143.56248141,311.22723299)(143.56248901,311.14723694)
\curveto(143.56248141,311.07723314)(143.5724814,311.0122332)(143.59248901,310.95223694)
\lineto(143.59248901,310.78723694)
\curveto(143.60248137,310.73723348)(143.60748136,310.66723355)(143.60748901,310.57723694)
\curveto(143.60748136,310.48723373)(143.59748137,310.4172338)(143.57748901,310.36723694)
\curveto(143.55748141,310.30723391)(143.55248142,310.24723397)(143.56248901,310.18723694)
\curveto(143.5724814,310.13723408)(143.5674814,310.08723413)(143.54748901,310.03723694)
\curveto(143.50748146,309.87723434)(143.4724815,309.72723449)(143.44248901,309.58723694)
\curveto(143.41248156,309.44723477)(143.3674816,309.3122349)(143.30748901,309.18223694)
\curveto(143.14748182,308.8122354)(142.92748204,308.47723574)(142.64748901,308.17723694)
\curveto(142.3674826,307.87723634)(142.04748292,307.64723657)(141.68748901,307.48723694)
\curveto(141.51748345,307.40723681)(141.31748365,307.33223688)(141.08748901,307.26223694)
\curveto(140.97748399,307.22223699)(140.86248411,307.19723702)(140.74248901,307.18723694)
\curveto(140.62248435,307.17723704)(140.50248447,307.15723706)(140.38248901,307.12723694)
\curveto(140.33248464,307.10723711)(140.27748469,307.10723711)(140.21748901,307.12723694)
\curveto(140.15748481,307.13723708)(140.09748487,307.13223708)(140.03748901,307.11223694)
\curveto(139.93748503,307.09223712)(139.83748513,307.09223712)(139.73748901,307.11223694)
\lineto(139.60248901,307.11223694)
\curveto(139.55248542,307.13223708)(139.49248548,307.14223707)(139.42248901,307.14223694)
\curveto(139.36248561,307.13223708)(139.30748566,307.13723708)(139.25748901,307.15723694)
\curveto(139.21748575,307.16723705)(139.18248579,307.17223704)(139.15248901,307.17223694)
\curveto(139.12248585,307.17223704)(139.08748588,307.17723704)(139.04748901,307.18723694)
\lineto(138.77748901,307.24723694)
\curveto(138.68748628,307.26723695)(138.60248637,307.29723692)(138.52248901,307.33723694)
\curveto(138.18248679,307.47723674)(137.89248708,307.63223658)(137.65248901,307.80223694)
\curveto(137.41248756,307.98223623)(137.19248778,308.212236)(136.99248901,308.49223694)
\curveto(136.84248813,308.72223549)(136.72748824,308.96223525)(136.64748901,309.21223694)
\curveto(136.62748834,309.26223495)(136.61748835,309.30723491)(136.61748901,309.34723694)
\curveto(136.61748835,309.39723482)(136.60748836,309.44723477)(136.58748901,309.49723694)
\curveto(136.5674884,309.55723466)(136.55248842,309.63723458)(136.54248901,309.73723694)
\curveto(136.54248843,309.83723438)(136.56248841,309.9122343)(136.60248901,309.96223694)
\curveto(136.65248832,310.04223417)(136.73248824,310.08723413)(136.84248901,310.09723694)
\curveto(136.95248802,310.10723411)(137.0674879,310.1122341)(137.18748901,310.11223694)
\lineto(137.35248901,310.11223694)
\curveto(137.41248756,310.1122341)(137.4674875,310.10223411)(137.51748901,310.08223694)
\curveto(137.60748736,310.06223415)(137.67748729,310.02223419)(137.72748901,309.96223694)
\curveto(137.79748717,309.87223434)(137.84248713,309.76223445)(137.86248901,309.63223694)
\curveto(137.89248708,309.5122347)(137.93748703,309.40723481)(137.99748901,309.31723694)
\curveto(138.18748678,308.97723524)(138.44748652,308.70723551)(138.77748901,308.50723694)
\curveto(138.87748609,308.44723577)(138.98248599,308.39723582)(139.09248901,308.35723694)
\curveto(139.21248576,308.32723589)(139.33248564,308.29223592)(139.45248901,308.25223694)
\curveto(139.62248535,308.20223601)(139.82748514,308.18223603)(140.06748901,308.19223694)
\curveto(140.31748465,308.212236)(140.51748445,308.24723597)(140.66748901,308.29723694)
\curveto(141.03748393,308.4172358)(141.32748364,308.57723564)(141.53748901,308.77723694)
\curveto(141.75748321,308.98723523)(141.93748303,309.26723495)(142.07748901,309.61723694)
\curveto(142.12748284,309.7172345)(142.15748281,309.82223439)(142.16748901,309.93223694)
\curveto(142.18748278,310.04223417)(142.21248276,310.15723406)(142.24248901,310.27723694)
\lineto(142.24248901,310.38223694)
\curveto(142.25248272,310.42223379)(142.25748271,310.46223375)(142.25748901,310.50223694)
\curveto(142.2674827,310.53223368)(142.2674827,310.56723365)(142.25748901,310.60723694)
\lineto(142.25748901,310.72723694)
\curveto(142.25748271,310.98723323)(142.22748274,311.23223298)(142.16748901,311.46223694)
\curveto(142.05748291,311.8122324)(141.90248307,312.10723211)(141.70248901,312.34723694)
\curveto(141.50248347,312.59723162)(141.24248373,312.79223142)(140.92248901,312.93223694)
\lineto(140.74248901,312.99223694)
\curveto(140.69248428,313.0122312)(140.63248434,313.03223118)(140.56248901,313.05223694)
\curveto(140.51248446,313.07223114)(140.45248452,313.08223113)(140.38248901,313.08223694)
\curveto(140.32248465,313.09223112)(140.25748471,313.10723111)(140.18748901,313.12723694)
\lineto(140.03748901,313.12723694)
\curveto(139.99748497,313.14723107)(139.94248503,313.15723106)(139.87248901,313.15723694)
\curveto(139.81248516,313.15723106)(139.75748521,313.14723107)(139.70748901,313.12723694)
\lineto(139.60248901,313.12723694)
\curveto(139.5724854,313.12723109)(139.53748543,313.12223109)(139.49748901,313.11223694)
\lineto(139.25748901,313.05223694)
\curveto(139.17748579,313.04223117)(139.09748587,313.02223119)(139.01748901,312.99223694)
\curveto(138.77748619,312.89223132)(138.54748642,312.75723146)(138.32748901,312.58723694)
\curveto(138.23748673,312.5172317)(138.15248682,312.44223177)(138.07248901,312.36223694)
\curveto(137.99248698,312.29223192)(137.89248708,312.23723198)(137.77248901,312.19723694)
\curveto(137.68248729,312.16723205)(137.54248743,312.15723206)(137.35248901,312.16723694)
\curveto(137.1724878,312.17723204)(137.05248792,312.20223201)(136.99248901,312.24223694)
\curveto(136.94248803,312.28223193)(136.90248807,312.34223187)(136.87248901,312.42223694)
\curveto(136.85248812,312.50223171)(136.85248812,312.58723163)(136.87248901,312.67723694)
\curveto(136.90248807,312.79723142)(136.92248805,312.9172313)(136.93248901,313.03723694)
\curveto(136.95248802,313.16723105)(136.97748799,313.29223092)(137.00748901,313.41223694)
\curveto(137.02748794,313.45223076)(137.03248794,313.48723073)(137.02248901,313.51723694)
\curveto(137.02248795,313.55723066)(137.03248794,313.60223061)(137.05248901,313.65223694)
\curveto(137.0724879,313.74223047)(137.08748788,313.83223038)(137.09748901,313.92223694)
\curveto(137.10748786,314.02223019)(137.12748784,314.1172301)(137.15748901,314.20723694)
\curveto(137.1674878,314.26722995)(137.1724878,314.32722989)(137.17248901,314.38723694)
\curveto(137.18248779,314.44722977)(137.19748777,314.50722971)(137.21748901,314.56723694)
\curveto(137.2674877,314.76722945)(137.30248767,314.97222924)(137.32248901,315.18223694)
\curveto(137.35248762,315.40222881)(137.39248758,315.6122286)(137.44248901,315.81223694)
\curveto(137.4724875,315.9122283)(137.49248748,316.0122282)(137.50248901,316.11223694)
\curveto(137.51248746,316.212228)(137.52748744,316.3122279)(137.54748901,316.41223694)
\curveto(137.55748741,316.44222777)(137.56248741,316.48222773)(137.56248901,316.53223694)
\curveto(137.59248738,316.64222757)(137.61248736,316.74722747)(137.62248901,316.84723694)
\curveto(137.64248733,316.95722726)(137.6674873,317.06722715)(137.69748901,317.17723694)
\curveto(137.71748725,317.25722696)(137.73248724,317.32722689)(137.74248901,317.38723694)
\curveto(137.75248722,317.45722676)(137.77748719,317.5172267)(137.81748901,317.56723694)
\curveto(137.83748713,317.59722662)(137.8674871,317.6172266)(137.90748901,317.62723694)
\curveto(137.94748702,317.64722657)(137.99248698,317.66722655)(138.04248901,317.68723694)
\curveto(138.10248687,317.68722653)(138.14248683,317.69222652)(138.16248901,317.70223694)
}
}
{
\newrgbcolor{curcolor}{0 0 0}
\pscustom[linestyle=none,fillstyle=solid,fillcolor=curcolor]
{
\newpath
\moveto(123.95327271,392.18295013)
\curveto(124.05326785,392.18293951)(124.14826776,392.17293952)(124.23827271,392.15295013)
\curveto(124.32826758,392.14293955)(124.39326751,392.11293958)(124.43327271,392.06295013)
\curveto(124.49326741,391.98293971)(124.52326738,391.87793982)(124.52327271,391.74795013)
\lineto(124.52327271,391.35795013)
\lineto(124.52327271,389.85795013)
\lineto(124.52327271,383.46795013)
\lineto(124.52327271,382.29795013)
\lineto(124.52327271,381.98295013)
\curveto(124.53326737,381.88294981)(124.51826739,381.80294989)(124.47827271,381.74295013)
\curveto(124.42826748,381.66295003)(124.35326755,381.61295008)(124.25327271,381.59295013)
\curveto(124.16326774,381.58295011)(124.05326785,381.57795012)(123.92327271,381.57795013)
\lineto(123.69827271,381.57795013)
\curveto(123.61826829,381.5979501)(123.54826836,381.61295008)(123.48827271,381.62295013)
\curveto(123.42826848,381.64295005)(123.37826853,381.68295001)(123.33827271,381.74295013)
\curveto(123.29826861,381.80294989)(123.27826863,381.87794982)(123.27827271,381.96795013)
\lineto(123.27827271,382.26795013)
\lineto(123.27827271,383.36295013)
\lineto(123.27827271,388.70295013)
\curveto(123.25826865,388.7929429)(123.24326866,388.86794283)(123.23327271,388.92795013)
\curveto(123.23326867,388.9979427)(123.2032687,389.05794264)(123.14327271,389.10795013)
\curveto(123.07326883,389.15794254)(122.98326892,389.18294251)(122.87327271,389.18295013)
\curveto(122.77326913,389.1929425)(122.66326924,389.1979425)(122.54327271,389.19795013)
\lineto(121.40327271,389.19795013)
\lineto(120.90827271,389.19795013)
\curveto(120.74827116,389.20794249)(120.63827127,389.26794243)(120.57827271,389.37795013)
\curveto(120.55827135,389.40794229)(120.54827136,389.43794226)(120.54827271,389.46795013)
\curveto(120.54827136,389.50794219)(120.54327136,389.55294214)(120.53327271,389.60295013)
\curveto(120.51327139,389.72294197)(120.51827139,389.83294186)(120.54827271,389.93295013)
\curveto(120.58827132,390.03294166)(120.64327126,390.10294159)(120.71327271,390.14295013)
\curveto(120.79327111,390.1929415)(120.91327099,390.21794148)(121.07327271,390.21795013)
\curveto(121.23327067,390.21794148)(121.36827054,390.23294146)(121.47827271,390.26295013)
\curveto(121.52827038,390.27294142)(121.58327032,390.27794142)(121.64327271,390.27795013)
\curveto(121.7032702,390.28794141)(121.76327014,390.30294139)(121.82327271,390.32295013)
\curveto(121.97326993,390.37294132)(122.11826979,390.42294127)(122.25827271,390.47295013)
\curveto(122.39826951,390.53294116)(122.53326937,390.60294109)(122.66327271,390.68295013)
\curveto(122.8032691,390.77294092)(122.92326898,390.87794082)(123.02327271,390.99795013)
\curveto(123.12326878,391.11794058)(123.21826869,391.24794045)(123.30827271,391.38795013)
\curveto(123.36826854,391.48794021)(123.41326849,391.5979401)(123.44327271,391.71795013)
\curveto(123.48326842,391.83793986)(123.53326837,391.94293975)(123.59327271,392.03295013)
\curveto(123.64326826,392.0929396)(123.71326819,392.13293956)(123.80327271,392.15295013)
\curveto(123.82326808,392.16293953)(123.84826806,392.16793953)(123.87827271,392.16795013)
\curveto(123.908268,392.16793953)(123.93326797,392.17293952)(123.95327271,392.18295013)
}
}
{
\newrgbcolor{curcolor}{0 0 0}
\pscustom[linestyle=none,fillstyle=solid,fillcolor=curcolor]
{
\newpath
\moveto(135.24288208,386.66295013)
\lineto(135.24288208,386.40795013)
\curveto(135.25287438,386.32794537)(135.24787438,386.25294544)(135.22788208,386.18295013)
\lineto(135.22788208,385.94295013)
\lineto(135.22788208,385.77795013)
\curveto(135.20787442,385.67794602)(135.19787443,385.57294612)(135.19788208,385.46295013)
\curveto(135.19787443,385.36294633)(135.18787444,385.26294643)(135.16788208,385.16295013)
\lineto(135.16788208,385.01295013)
\curveto(135.13787449,384.87294682)(135.11787451,384.73294696)(135.10788208,384.59295013)
\curveto(135.09787453,384.46294723)(135.07287456,384.33294736)(135.03288208,384.20295013)
\curveto(135.01287462,384.12294757)(134.99287464,384.03794766)(134.97288208,383.94795013)
\lineto(134.91288208,383.70795013)
\lineto(134.79288208,383.40795013)
\curveto(134.76287487,383.31794838)(134.7278749,383.22794847)(134.68788208,383.13795013)
\curveto(134.58787504,382.91794878)(134.45287518,382.70294899)(134.28288208,382.49295013)
\curveto(134.12287551,382.28294941)(133.94787568,382.11294958)(133.75788208,381.98295013)
\curveto(133.70787592,381.94294975)(133.64787598,381.90294979)(133.57788208,381.86295013)
\curveto(133.51787611,381.83294986)(133.45787617,381.7979499)(133.39788208,381.75795013)
\curveto(133.31787631,381.70794999)(133.22287641,381.66795003)(133.11288208,381.63795013)
\curveto(133.00287663,381.60795009)(132.89787673,381.57795012)(132.79788208,381.54795013)
\curveto(132.68787694,381.50795019)(132.57787705,381.48295021)(132.46788208,381.47295013)
\curveto(132.35787727,381.46295023)(132.24287739,381.44795025)(132.12288208,381.42795013)
\curveto(132.08287755,381.41795028)(132.03787759,381.41795028)(131.98788208,381.42795013)
\curveto(131.94787768,381.42795027)(131.90787772,381.42295027)(131.86788208,381.41295013)
\curveto(131.8278778,381.40295029)(131.77287786,381.3979503)(131.70288208,381.39795013)
\curveto(131.632878,381.3979503)(131.58287805,381.40295029)(131.55288208,381.41295013)
\curveto(131.50287813,381.43295026)(131.45787817,381.43795026)(131.41788208,381.42795013)
\curveto(131.37787825,381.41795028)(131.34287829,381.41795028)(131.31288208,381.42795013)
\lineto(131.22288208,381.42795013)
\curveto(131.16287847,381.44795025)(131.09787853,381.46295023)(131.02788208,381.47295013)
\curveto(130.96787866,381.47295022)(130.90287873,381.47795022)(130.83288208,381.48795013)
\curveto(130.66287897,381.53795016)(130.50287913,381.58795011)(130.35288208,381.63795013)
\curveto(130.20287943,381.68795001)(130.05787957,381.75294994)(129.91788208,381.83295013)
\curveto(129.86787976,381.87294982)(129.81287982,381.90294979)(129.75288208,381.92295013)
\curveto(129.70287993,381.95294974)(129.65287998,381.98794971)(129.60288208,382.02795013)
\curveto(129.36288027,382.20794949)(129.16288047,382.42794927)(129.00288208,382.68795013)
\curveto(128.84288079,382.94794875)(128.70288093,383.23294846)(128.58288208,383.54295013)
\curveto(128.52288111,383.68294801)(128.47788115,383.82294787)(128.44788208,383.96295013)
\curveto(128.41788121,384.11294758)(128.38288125,384.26794743)(128.34288208,384.42795013)
\curveto(128.32288131,384.53794716)(128.30788132,384.64794705)(128.29788208,384.75795013)
\curveto(128.28788134,384.86794683)(128.27288136,384.97794672)(128.25288208,385.08795013)
\curveto(128.24288139,385.12794657)(128.23788139,385.16794653)(128.23788208,385.20795013)
\curveto(128.24788138,385.24794645)(128.24788138,385.28794641)(128.23788208,385.32795013)
\curveto(128.2278814,385.37794632)(128.22288141,385.42794627)(128.22288208,385.47795013)
\lineto(128.22288208,385.64295013)
\curveto(128.20288143,385.692946)(128.19788143,385.74294595)(128.20788208,385.79295013)
\curveto(128.21788141,385.85294584)(128.21788141,385.90794579)(128.20788208,385.95795013)
\curveto(128.19788143,385.9979457)(128.19788143,386.04294565)(128.20788208,386.09295013)
\curveto(128.21788141,386.14294555)(128.21288142,386.1929455)(128.19288208,386.24295013)
\curveto(128.17288146,386.31294538)(128.16788146,386.38794531)(128.17788208,386.46795013)
\curveto(128.18788144,386.55794514)(128.19288144,386.64294505)(128.19288208,386.72295013)
\curveto(128.19288144,386.81294488)(128.18788144,386.91294478)(128.17788208,387.02295013)
\curveto(128.16788146,387.14294455)(128.17288146,387.24294445)(128.19288208,387.32295013)
\lineto(128.19288208,387.60795013)
\lineto(128.23788208,388.23795013)
\curveto(128.24788138,388.33794336)(128.25788137,388.43294326)(128.26788208,388.52295013)
\lineto(128.29788208,388.82295013)
\curveto(128.31788131,388.87294282)(128.32288131,388.92294277)(128.31288208,388.97295013)
\curveto(128.31288132,389.03294266)(128.32288131,389.08794261)(128.34288208,389.13795013)
\curveto(128.39288124,389.30794239)(128.4328812,389.47294222)(128.46288208,389.63295013)
\curveto(128.49288114,389.80294189)(128.54288109,389.96294173)(128.61288208,390.11295013)
\curveto(128.80288083,390.57294112)(129.02288061,390.94794075)(129.27288208,391.23795013)
\curveto(129.5328801,391.52794017)(129.89287974,391.77293992)(130.35288208,391.97295013)
\curveto(130.48287915,392.02293967)(130.61287902,392.05793964)(130.74288208,392.07795013)
\curveto(130.88287875,392.0979396)(131.02287861,392.12293957)(131.16288208,392.15295013)
\curveto(131.2328784,392.16293953)(131.29787833,392.16793953)(131.35788208,392.16795013)
\curveto(131.41787821,392.16793953)(131.48287815,392.17293952)(131.55288208,392.18295013)
\curveto(132.38287725,392.20293949)(133.05287658,392.05293964)(133.56288208,391.73295013)
\curveto(134.07287556,391.42294027)(134.45287518,390.98294071)(134.70288208,390.41295013)
\curveto(134.75287488,390.2929414)(134.79787483,390.16794153)(134.83788208,390.03795013)
\curveto(134.87787475,389.90794179)(134.92287471,389.77294192)(134.97288208,389.63295013)
\curveto(134.99287464,389.55294214)(135.00787462,389.46794223)(135.01788208,389.37795013)
\lineto(135.07788208,389.13795013)
\curveto(135.10787452,389.02794267)(135.12287451,388.91794278)(135.12288208,388.80795013)
\curveto(135.1328745,388.697943)(135.14787448,388.58794311)(135.16788208,388.47795013)
\curveto(135.18787444,388.42794327)(135.19287444,388.38294331)(135.18288208,388.34295013)
\curveto(135.18287445,388.30294339)(135.18787444,388.26294343)(135.19788208,388.22295013)
\curveto(135.20787442,388.17294352)(135.20787442,388.11794358)(135.19788208,388.05795013)
\curveto(135.19787443,388.00794369)(135.20287443,387.95794374)(135.21288208,387.90795013)
\lineto(135.21288208,387.77295013)
\curveto(135.2328744,387.71294398)(135.2328744,387.64294405)(135.21288208,387.56295013)
\curveto(135.20287443,387.4929442)(135.20787442,387.42794427)(135.22788208,387.36795013)
\curveto(135.23787439,387.33794436)(135.24287439,387.2979444)(135.24288208,387.24795013)
\lineto(135.24288208,387.12795013)
\lineto(135.24288208,386.66295013)
\moveto(133.69788208,384.33795013)
\curveto(133.79787583,384.65794704)(133.85787577,385.02294667)(133.87788208,385.43295013)
\curveto(133.89787573,385.84294585)(133.90787572,386.25294544)(133.90788208,386.66295013)
\curveto(133.90787572,387.0929446)(133.89787573,387.51294418)(133.87788208,387.92295013)
\curveto(133.85787577,388.33294336)(133.81287582,388.71794298)(133.74288208,389.07795013)
\curveto(133.67287596,389.43794226)(133.56287607,389.75794194)(133.41288208,390.03795013)
\curveto(133.27287636,390.32794137)(133.07787655,390.56294113)(132.82788208,390.74295013)
\curveto(132.66787696,390.85294084)(132.48787714,390.93294076)(132.28788208,390.98295013)
\curveto(132.08787754,391.04294065)(131.84287779,391.07294062)(131.55288208,391.07295013)
\curveto(131.5328781,391.05294064)(131.49787813,391.04294065)(131.44788208,391.04295013)
\curveto(131.39787823,391.05294064)(131.35787827,391.05294064)(131.32788208,391.04295013)
\curveto(131.24787838,391.02294067)(131.17287846,391.00294069)(131.10288208,390.98295013)
\curveto(131.04287859,390.97294072)(130.97787865,390.95294074)(130.90788208,390.92295013)
\curveto(130.63787899,390.80294089)(130.41787921,390.63294106)(130.24788208,390.41295013)
\curveto(130.08787954,390.20294149)(129.95287968,389.95794174)(129.84288208,389.67795013)
\curveto(129.79287984,389.56794213)(129.75287988,389.44794225)(129.72288208,389.31795013)
\curveto(129.70287993,389.1979425)(129.67787995,389.07294262)(129.64788208,388.94295013)
\curveto(129.62788,388.8929428)(129.61788001,388.83794286)(129.61788208,388.77795013)
\curveto(129.61788001,388.72794297)(129.61288002,388.67794302)(129.60288208,388.62795013)
\curveto(129.59288004,388.53794316)(129.58288005,388.44294325)(129.57288208,388.34295013)
\curveto(129.56288007,388.25294344)(129.55288008,388.15794354)(129.54288208,388.05795013)
\curveto(129.54288009,387.97794372)(129.53788009,387.8929438)(129.52788208,387.80295013)
\lineto(129.52788208,387.56295013)
\lineto(129.52788208,387.38295013)
\curveto(129.51788011,387.35294434)(129.51288012,387.31794438)(129.51288208,387.27795013)
\lineto(129.51288208,387.14295013)
\lineto(129.51288208,386.69295013)
\curveto(129.51288012,386.61294508)(129.50788012,386.52794517)(129.49788208,386.43795013)
\curveto(129.49788013,386.35794534)(129.50788012,386.28294541)(129.52788208,386.21295013)
\lineto(129.52788208,385.94295013)
\curveto(129.5278801,385.92294577)(129.52288011,385.8929458)(129.51288208,385.85295013)
\curveto(129.51288012,385.82294587)(129.51788011,385.7979459)(129.52788208,385.77795013)
\curveto(129.53788009,385.67794602)(129.54288009,385.57794612)(129.54288208,385.47795013)
\curveto(129.55288008,385.38794631)(129.56288007,385.28794641)(129.57288208,385.17795013)
\curveto(129.60288003,385.05794664)(129.61788001,384.93294676)(129.61788208,384.80295013)
\curveto(129.62788,384.68294701)(129.65287998,384.56794713)(129.69288208,384.45795013)
\curveto(129.77287986,384.15794754)(129.85787977,383.8929478)(129.94788208,383.66295013)
\curveto(130.04787958,383.43294826)(130.19287944,383.21794848)(130.38288208,383.01795013)
\curveto(130.59287904,382.81794888)(130.85787877,382.66794903)(131.17788208,382.56795013)
\curveto(131.21787841,382.54794915)(131.25287838,382.53794916)(131.28288208,382.53795013)
\curveto(131.32287831,382.54794915)(131.36787826,382.54294915)(131.41788208,382.52295013)
\curveto(131.45787817,382.51294918)(131.5278781,382.50294919)(131.62788208,382.49295013)
\curveto(131.73787789,382.48294921)(131.82287781,382.48794921)(131.88288208,382.50795013)
\curveto(131.95287768,382.52794917)(132.02287761,382.53794916)(132.09288208,382.53795013)
\curveto(132.16287747,382.54794915)(132.2278774,382.56294913)(132.28788208,382.58295013)
\curveto(132.48787714,382.64294905)(132.66787696,382.72794897)(132.82788208,382.83795013)
\curveto(132.85787677,382.85794884)(132.88287675,382.87794882)(132.90288208,382.89795013)
\lineto(132.96288208,382.95795013)
\curveto(133.00287663,382.97794872)(133.05287658,383.01794868)(133.11288208,383.07795013)
\curveto(133.21287642,383.21794848)(133.29787633,383.34794835)(133.36788208,383.46795013)
\curveto(133.43787619,383.58794811)(133.50787612,383.73294796)(133.57788208,383.90295013)
\curveto(133.60787602,383.97294772)(133.627876,384.04294765)(133.63788208,384.11295013)
\curveto(133.65787597,384.18294751)(133.67787595,384.25794744)(133.69788208,384.33795013)
}
}
{
\newrgbcolor{curcolor}{0 0 0}
\pscustom[linestyle=none,fillstyle=solid,fillcolor=curcolor]
{
\newpath
\moveto(143.59249146,386.66295013)
\lineto(143.59249146,386.40795013)
\curveto(143.60248375,386.32794537)(143.59748376,386.25294544)(143.57749146,386.18295013)
\lineto(143.57749146,385.94295013)
\lineto(143.57749146,385.77795013)
\curveto(143.5574838,385.67794602)(143.54748381,385.57294612)(143.54749146,385.46295013)
\curveto(143.54748381,385.36294633)(143.53748382,385.26294643)(143.51749146,385.16295013)
\lineto(143.51749146,385.01295013)
\curveto(143.48748387,384.87294682)(143.46748389,384.73294696)(143.45749146,384.59295013)
\curveto(143.44748391,384.46294723)(143.42248393,384.33294736)(143.38249146,384.20295013)
\curveto(143.36248399,384.12294757)(143.34248401,384.03794766)(143.32249146,383.94795013)
\lineto(143.26249146,383.70795013)
\lineto(143.14249146,383.40795013)
\curveto(143.11248424,383.31794838)(143.07748428,383.22794847)(143.03749146,383.13795013)
\curveto(142.93748442,382.91794878)(142.80248455,382.70294899)(142.63249146,382.49295013)
\curveto(142.47248488,382.28294941)(142.29748506,382.11294958)(142.10749146,381.98295013)
\curveto(142.0574853,381.94294975)(141.99748536,381.90294979)(141.92749146,381.86295013)
\curveto(141.86748549,381.83294986)(141.80748555,381.7979499)(141.74749146,381.75795013)
\curveto(141.66748569,381.70794999)(141.57248578,381.66795003)(141.46249146,381.63795013)
\curveto(141.352486,381.60795009)(141.24748611,381.57795012)(141.14749146,381.54795013)
\curveto(141.03748632,381.50795019)(140.92748643,381.48295021)(140.81749146,381.47295013)
\curveto(140.70748665,381.46295023)(140.59248676,381.44795025)(140.47249146,381.42795013)
\curveto(140.43248692,381.41795028)(140.38748697,381.41795028)(140.33749146,381.42795013)
\curveto(140.29748706,381.42795027)(140.2574871,381.42295027)(140.21749146,381.41295013)
\curveto(140.17748718,381.40295029)(140.12248723,381.3979503)(140.05249146,381.39795013)
\curveto(139.98248737,381.3979503)(139.93248742,381.40295029)(139.90249146,381.41295013)
\curveto(139.8524875,381.43295026)(139.80748755,381.43795026)(139.76749146,381.42795013)
\curveto(139.72748763,381.41795028)(139.69248766,381.41795028)(139.66249146,381.42795013)
\lineto(139.57249146,381.42795013)
\curveto(139.51248784,381.44795025)(139.44748791,381.46295023)(139.37749146,381.47295013)
\curveto(139.31748804,381.47295022)(139.2524881,381.47795022)(139.18249146,381.48795013)
\curveto(139.01248834,381.53795016)(138.8524885,381.58795011)(138.70249146,381.63795013)
\curveto(138.5524888,381.68795001)(138.40748895,381.75294994)(138.26749146,381.83295013)
\curveto(138.21748914,381.87294982)(138.16248919,381.90294979)(138.10249146,381.92295013)
\curveto(138.0524893,381.95294974)(138.00248935,381.98794971)(137.95249146,382.02795013)
\curveto(137.71248964,382.20794949)(137.51248984,382.42794927)(137.35249146,382.68795013)
\curveto(137.19249016,382.94794875)(137.0524903,383.23294846)(136.93249146,383.54295013)
\curveto(136.87249048,383.68294801)(136.82749053,383.82294787)(136.79749146,383.96295013)
\curveto(136.76749059,384.11294758)(136.73249062,384.26794743)(136.69249146,384.42795013)
\curveto(136.67249068,384.53794716)(136.6574907,384.64794705)(136.64749146,384.75795013)
\curveto(136.63749072,384.86794683)(136.62249073,384.97794672)(136.60249146,385.08795013)
\curveto(136.59249076,385.12794657)(136.58749077,385.16794653)(136.58749146,385.20795013)
\curveto(136.59749076,385.24794645)(136.59749076,385.28794641)(136.58749146,385.32795013)
\curveto(136.57749078,385.37794632)(136.57249078,385.42794627)(136.57249146,385.47795013)
\lineto(136.57249146,385.64295013)
\curveto(136.5524908,385.692946)(136.54749081,385.74294595)(136.55749146,385.79295013)
\curveto(136.56749079,385.85294584)(136.56749079,385.90794579)(136.55749146,385.95795013)
\curveto(136.54749081,385.9979457)(136.54749081,386.04294565)(136.55749146,386.09295013)
\curveto(136.56749079,386.14294555)(136.56249079,386.1929455)(136.54249146,386.24295013)
\curveto(136.52249083,386.31294538)(136.51749084,386.38794531)(136.52749146,386.46795013)
\curveto(136.53749082,386.55794514)(136.54249081,386.64294505)(136.54249146,386.72295013)
\curveto(136.54249081,386.81294488)(136.53749082,386.91294478)(136.52749146,387.02295013)
\curveto(136.51749084,387.14294455)(136.52249083,387.24294445)(136.54249146,387.32295013)
\lineto(136.54249146,387.60795013)
\lineto(136.58749146,388.23795013)
\curveto(136.59749076,388.33794336)(136.60749075,388.43294326)(136.61749146,388.52295013)
\lineto(136.64749146,388.82295013)
\curveto(136.66749069,388.87294282)(136.67249068,388.92294277)(136.66249146,388.97295013)
\curveto(136.66249069,389.03294266)(136.67249068,389.08794261)(136.69249146,389.13795013)
\curveto(136.74249061,389.30794239)(136.78249057,389.47294222)(136.81249146,389.63295013)
\curveto(136.84249051,389.80294189)(136.89249046,389.96294173)(136.96249146,390.11295013)
\curveto(137.1524902,390.57294112)(137.37248998,390.94794075)(137.62249146,391.23795013)
\curveto(137.88248947,391.52794017)(138.24248911,391.77293992)(138.70249146,391.97295013)
\curveto(138.83248852,392.02293967)(138.96248839,392.05793964)(139.09249146,392.07795013)
\curveto(139.23248812,392.0979396)(139.37248798,392.12293957)(139.51249146,392.15295013)
\curveto(139.58248777,392.16293953)(139.64748771,392.16793953)(139.70749146,392.16795013)
\curveto(139.76748759,392.16793953)(139.83248752,392.17293952)(139.90249146,392.18295013)
\curveto(140.73248662,392.20293949)(141.40248595,392.05293964)(141.91249146,391.73295013)
\curveto(142.42248493,391.42294027)(142.80248455,390.98294071)(143.05249146,390.41295013)
\curveto(143.10248425,390.2929414)(143.14748421,390.16794153)(143.18749146,390.03795013)
\curveto(143.22748413,389.90794179)(143.27248408,389.77294192)(143.32249146,389.63295013)
\curveto(143.34248401,389.55294214)(143.357484,389.46794223)(143.36749146,389.37795013)
\lineto(143.42749146,389.13795013)
\curveto(143.4574839,389.02794267)(143.47248388,388.91794278)(143.47249146,388.80795013)
\curveto(143.48248387,388.697943)(143.49748386,388.58794311)(143.51749146,388.47795013)
\curveto(143.53748382,388.42794327)(143.54248381,388.38294331)(143.53249146,388.34295013)
\curveto(143.53248382,388.30294339)(143.53748382,388.26294343)(143.54749146,388.22295013)
\curveto(143.5574838,388.17294352)(143.5574838,388.11794358)(143.54749146,388.05795013)
\curveto(143.54748381,388.00794369)(143.5524838,387.95794374)(143.56249146,387.90795013)
\lineto(143.56249146,387.77295013)
\curveto(143.58248377,387.71294398)(143.58248377,387.64294405)(143.56249146,387.56295013)
\curveto(143.5524838,387.4929442)(143.5574838,387.42794427)(143.57749146,387.36795013)
\curveto(143.58748377,387.33794436)(143.59248376,387.2979444)(143.59249146,387.24795013)
\lineto(143.59249146,387.12795013)
\lineto(143.59249146,386.66295013)
\moveto(142.04749146,384.33795013)
\curveto(142.14748521,384.65794704)(142.20748515,385.02294667)(142.22749146,385.43295013)
\curveto(142.24748511,385.84294585)(142.2574851,386.25294544)(142.25749146,386.66295013)
\curveto(142.2574851,387.0929446)(142.24748511,387.51294418)(142.22749146,387.92295013)
\curveto(142.20748515,388.33294336)(142.16248519,388.71794298)(142.09249146,389.07795013)
\curveto(142.02248533,389.43794226)(141.91248544,389.75794194)(141.76249146,390.03795013)
\curveto(141.62248573,390.32794137)(141.42748593,390.56294113)(141.17749146,390.74295013)
\curveto(141.01748634,390.85294084)(140.83748652,390.93294076)(140.63749146,390.98295013)
\curveto(140.43748692,391.04294065)(140.19248716,391.07294062)(139.90249146,391.07295013)
\curveto(139.88248747,391.05294064)(139.84748751,391.04294065)(139.79749146,391.04295013)
\curveto(139.74748761,391.05294064)(139.70748765,391.05294064)(139.67749146,391.04295013)
\curveto(139.59748776,391.02294067)(139.52248783,391.00294069)(139.45249146,390.98295013)
\curveto(139.39248796,390.97294072)(139.32748803,390.95294074)(139.25749146,390.92295013)
\curveto(138.98748837,390.80294089)(138.76748859,390.63294106)(138.59749146,390.41295013)
\curveto(138.43748892,390.20294149)(138.30248905,389.95794174)(138.19249146,389.67795013)
\curveto(138.14248921,389.56794213)(138.10248925,389.44794225)(138.07249146,389.31795013)
\curveto(138.0524893,389.1979425)(138.02748933,389.07294262)(137.99749146,388.94295013)
\curveto(137.97748938,388.8929428)(137.96748939,388.83794286)(137.96749146,388.77795013)
\curveto(137.96748939,388.72794297)(137.96248939,388.67794302)(137.95249146,388.62795013)
\curveto(137.94248941,388.53794316)(137.93248942,388.44294325)(137.92249146,388.34295013)
\curveto(137.91248944,388.25294344)(137.90248945,388.15794354)(137.89249146,388.05795013)
\curveto(137.89248946,387.97794372)(137.88748947,387.8929438)(137.87749146,387.80295013)
\lineto(137.87749146,387.56295013)
\lineto(137.87749146,387.38295013)
\curveto(137.86748949,387.35294434)(137.86248949,387.31794438)(137.86249146,387.27795013)
\lineto(137.86249146,387.14295013)
\lineto(137.86249146,386.69295013)
\curveto(137.86248949,386.61294508)(137.8574895,386.52794517)(137.84749146,386.43795013)
\curveto(137.84748951,386.35794534)(137.8574895,386.28294541)(137.87749146,386.21295013)
\lineto(137.87749146,385.94295013)
\curveto(137.87748948,385.92294577)(137.87248948,385.8929458)(137.86249146,385.85295013)
\curveto(137.86248949,385.82294587)(137.86748949,385.7979459)(137.87749146,385.77795013)
\curveto(137.88748947,385.67794602)(137.89248946,385.57794612)(137.89249146,385.47795013)
\curveto(137.90248945,385.38794631)(137.91248944,385.28794641)(137.92249146,385.17795013)
\curveto(137.9524894,385.05794664)(137.96748939,384.93294676)(137.96749146,384.80295013)
\curveto(137.97748938,384.68294701)(138.00248935,384.56794713)(138.04249146,384.45795013)
\curveto(138.12248923,384.15794754)(138.20748915,383.8929478)(138.29749146,383.66295013)
\curveto(138.39748896,383.43294826)(138.54248881,383.21794848)(138.73249146,383.01795013)
\curveto(138.94248841,382.81794888)(139.20748815,382.66794903)(139.52749146,382.56795013)
\curveto(139.56748779,382.54794915)(139.60248775,382.53794916)(139.63249146,382.53795013)
\curveto(139.67248768,382.54794915)(139.71748764,382.54294915)(139.76749146,382.52295013)
\curveto(139.80748755,382.51294918)(139.87748748,382.50294919)(139.97749146,382.49295013)
\curveto(140.08748727,382.48294921)(140.17248718,382.48794921)(140.23249146,382.50795013)
\curveto(140.30248705,382.52794917)(140.37248698,382.53794916)(140.44249146,382.53795013)
\curveto(140.51248684,382.54794915)(140.57748678,382.56294913)(140.63749146,382.58295013)
\curveto(140.83748652,382.64294905)(141.01748634,382.72794897)(141.17749146,382.83795013)
\curveto(141.20748615,382.85794884)(141.23248612,382.87794882)(141.25249146,382.89795013)
\lineto(141.31249146,382.95795013)
\curveto(141.352486,382.97794872)(141.40248595,383.01794868)(141.46249146,383.07795013)
\curveto(141.56248579,383.21794848)(141.64748571,383.34794835)(141.71749146,383.46795013)
\curveto(141.78748557,383.58794811)(141.8574855,383.73294796)(141.92749146,383.90295013)
\curveto(141.9574854,383.97294772)(141.97748538,384.04294765)(141.98749146,384.11295013)
\curveto(142.00748535,384.18294751)(142.02748533,384.25794744)(142.04749146,384.33795013)
}
}
{
\newrgbcolor{curcolor}{0 0 0}
\pscustom[linestyle=none,fillstyle=solid,fillcolor=curcolor]
{
\newpath
\moveto(839.91147461,375.09397308)
\curveto(839.93146506,375.04397234)(839.95646504,374.9839724)(839.98647461,374.91397308)
\curveto(840.01646498,374.84397254)(840.03646496,374.76897261)(840.04647461,374.68897308)
\curveto(840.06646493,374.61897276)(840.06646493,374.54897283)(840.04647461,374.47897308)
\curveto(840.03646496,374.41897296)(839.996465,374.37397301)(839.92647461,374.34397308)
\curveto(839.87646512,374.32397306)(839.81646518,374.31397307)(839.74647461,374.31397308)
\lineto(839.53647461,374.31397308)
\lineto(839.08647461,374.31397308)
\curveto(838.93646606,374.31397307)(838.81646618,374.33897304)(838.72647461,374.38897308)
\curveto(838.62646637,374.44897293)(838.55146644,374.55397283)(838.50147461,374.70397308)
\curveto(838.46146653,374.85397253)(838.41646658,374.98897239)(838.36647461,375.10897308)
\curveto(838.25646674,375.36897201)(838.15646684,375.63897174)(838.06647461,375.91897308)
\curveto(837.97646702,376.19897118)(837.87646712,376.47397091)(837.76647461,376.74397308)
\curveto(837.73646726,376.83397055)(837.70646729,376.91897046)(837.67647461,376.99897308)
\curveto(837.65646734,377.0789703)(837.62646737,377.15397023)(837.58647461,377.22397308)
\curveto(837.55646744,377.29397009)(837.51146748,377.35397003)(837.45147461,377.40397308)
\curveto(837.3914676,377.45396993)(837.31146768,377.49396989)(837.21147461,377.52397308)
\curveto(837.16146783,377.54396984)(837.10146789,377.54896983)(837.03147461,377.53897308)
\lineto(836.83647461,377.53897308)
\lineto(834.00147461,377.53897308)
\lineto(833.70147461,377.53897308)
\curveto(833.5914714,377.54896983)(833.48647151,377.54896983)(833.38647461,377.53897308)
\curveto(833.28647171,377.52896985)(833.1914718,377.51396987)(833.10147461,377.49397308)
\curveto(833.02147197,377.47396991)(832.96147203,377.43396995)(832.92147461,377.37397308)
\curveto(832.84147215,377.27397011)(832.78147221,377.15897022)(832.74147461,377.02897308)
\curveto(832.71147228,376.90897047)(832.67147232,376.7839706)(832.62147461,376.65397308)
\curveto(832.52147247,376.42397096)(832.42647257,376.1839712)(832.33647461,375.93397308)
\curveto(832.25647274,375.6839717)(832.16647283,375.44397194)(832.06647461,375.21397308)
\curveto(832.04647295,375.15397223)(832.02147297,375.0839723)(831.99147461,375.00397308)
\curveto(831.97147302,374.93397245)(831.94647305,374.85897252)(831.91647461,374.77897308)
\curveto(831.88647311,374.69897268)(831.85147314,374.62397276)(831.81147461,374.55397308)
\curveto(831.78147321,374.49397289)(831.74647325,374.44897293)(831.70647461,374.41897308)
\curveto(831.62647337,374.35897302)(831.51647348,374.32397306)(831.37647461,374.31397308)
\lineto(830.95647461,374.31397308)
\lineto(830.71647461,374.31397308)
\curveto(830.64647435,374.32397306)(830.58647441,374.34897303)(830.53647461,374.38897308)
\curveto(830.48647451,374.41897296)(830.45647454,374.46397292)(830.44647461,374.52397308)
\curveto(830.44647455,374.5839728)(830.45147454,374.64397274)(830.46147461,374.70397308)
\curveto(830.48147451,374.77397261)(830.50147449,374.83897254)(830.52147461,374.89897308)
\curveto(830.55147444,374.96897241)(830.57647442,375.01897236)(830.59647461,375.04897308)
\curveto(830.73647426,375.36897201)(830.86147413,375.6839717)(830.97147461,375.99397308)
\curveto(831.08147391,376.31397107)(831.20147379,376.63397075)(831.33147461,376.95397308)
\curveto(831.42147357,377.17397021)(831.50647349,377.38896999)(831.58647461,377.59897308)
\curveto(831.66647333,377.81896956)(831.75147324,378.03896934)(831.84147461,378.25897308)
\curveto(832.14147285,378.9789684)(832.42647257,379.70396768)(832.69647461,380.43397308)
\curveto(832.96647203,381.17396621)(833.25147174,381.90896547)(833.55147461,382.63897308)
\curveto(833.66147133,382.89896448)(833.76147123,383.16396422)(833.85147461,383.43397308)
\curveto(833.95147104,383.70396368)(834.05647094,383.96896341)(834.16647461,384.22897308)
\curveto(834.21647078,384.33896304)(834.26147073,384.45896292)(834.30147461,384.58897308)
\curveto(834.35147064,384.72896265)(834.42147057,384.82896255)(834.51147461,384.88897308)
\curveto(834.55147044,384.92896245)(834.61647038,384.95896242)(834.70647461,384.97897308)
\curveto(834.72647027,384.98896239)(834.74647025,384.98896239)(834.76647461,384.97897308)
\curveto(834.7964702,384.9789624)(834.82147017,384.9839624)(834.84147461,384.99397308)
\curveto(835.02146997,384.99396239)(835.23146976,384.99396239)(835.47147461,384.99397308)
\curveto(835.71146928,385.00396238)(835.88646911,384.96896241)(835.99647461,384.88897308)
\curveto(836.07646892,384.82896255)(836.13646886,384.72896265)(836.17647461,384.58897308)
\curveto(836.22646877,384.45896292)(836.27646872,384.33896304)(836.32647461,384.22897308)
\curveto(836.42646857,383.99896338)(836.51646848,383.76896361)(836.59647461,383.53897308)
\curveto(836.67646832,383.30896407)(836.76646823,383.0789643)(836.86647461,382.84897308)
\curveto(836.94646805,382.64896473)(837.02146797,382.44396494)(837.09147461,382.23397308)
\curveto(837.17146782,382.02396536)(837.25646774,381.81896556)(837.34647461,381.61897308)
\curveto(837.64646735,380.88896649)(837.93146706,380.14896723)(838.20147461,379.39897308)
\curveto(838.48146651,378.65896872)(838.77646622,377.92396946)(839.08647461,377.19397308)
\curveto(839.12646587,377.10397028)(839.15646584,377.01897036)(839.17647461,376.93897308)
\curveto(839.20646579,376.85897052)(839.23646576,376.77397061)(839.26647461,376.68397308)
\curveto(839.37646562,376.42397096)(839.48146551,376.15897122)(839.58147461,375.88897308)
\curveto(839.6914653,375.61897176)(839.80146519,375.35397203)(839.91147461,375.09397308)
\moveto(836.70147461,378.73897308)
\curveto(836.7914682,378.76896861)(836.84646815,378.81896856)(836.86647461,378.88897308)
\curveto(836.8964681,378.95896842)(836.90146809,379.03396835)(836.88147461,379.11397308)
\curveto(836.87146812,379.20396818)(836.84646815,379.28896809)(836.80647461,379.36897308)
\curveto(836.77646822,379.45896792)(836.74646825,379.53396785)(836.71647461,379.59397308)
\curveto(836.6964683,379.63396775)(836.68646831,379.66896771)(836.68647461,379.69897308)
\curveto(836.68646831,379.72896765)(836.67646832,379.76396762)(836.65647461,379.80397308)
\lineto(836.56647461,380.04397308)
\curveto(836.54646845,380.13396725)(836.51646848,380.22396716)(836.47647461,380.31397308)
\curveto(836.32646867,380.67396671)(836.1914688,381.03896634)(836.07147461,381.40897308)
\curveto(835.96146903,381.78896559)(835.83146916,382.15896522)(835.68147461,382.51897308)
\curveto(835.63146936,382.62896475)(835.58646941,382.73896464)(835.54647461,382.84897308)
\curveto(835.51646948,382.95896442)(835.47646952,383.06396432)(835.42647461,383.16397308)
\curveto(835.40646959,383.21396417)(835.38146961,383.25896412)(835.35147461,383.29897308)
\curveto(835.33146966,383.34896403)(835.28146971,383.37396401)(835.20147461,383.37397308)
\curveto(835.18146981,383.35396403)(835.16146983,383.33896404)(835.14147461,383.32897308)
\curveto(835.12146987,383.31896406)(835.10146989,383.30396408)(835.08147461,383.28397308)
\curveto(835.04146995,383.23396415)(835.01146998,383.1789642)(834.99147461,383.11897308)
\curveto(834.97147002,383.06896431)(834.95147004,383.01396437)(834.93147461,382.95397308)
\curveto(834.88147011,382.84396454)(834.84147015,382.73396465)(834.81147461,382.62397308)
\curveto(834.78147021,382.51396487)(834.74147025,382.40396498)(834.69147461,382.29397308)
\curveto(834.52147047,381.90396548)(834.37147062,381.50896587)(834.24147461,381.10897308)
\curveto(834.12147087,380.70896667)(833.98147101,380.31896706)(833.82147461,379.93897308)
\lineto(833.76147461,379.78897308)
\curveto(833.75147124,379.73896764)(833.73647126,379.68896769)(833.71647461,379.63897308)
\lineto(833.62647461,379.39897308)
\curveto(833.5964714,379.31896806)(833.57147142,379.23896814)(833.55147461,379.15897308)
\curveto(833.53147146,379.10896827)(833.52147147,379.05396833)(833.52147461,378.99397308)
\curveto(833.53147146,378.93396845)(833.54647145,378.8839685)(833.56647461,378.84397308)
\curveto(833.61647138,378.76396862)(833.72147127,378.71896866)(833.88147461,378.70897308)
\lineto(834.33147461,378.70897308)
\lineto(835.93647461,378.70897308)
\curveto(836.04646895,378.70896867)(836.18146881,378.70396868)(836.34147461,378.69397308)
\curveto(836.50146849,378.69396869)(836.62146837,378.70896867)(836.70147461,378.73897308)
}
}
{
\newrgbcolor{curcolor}{0 0 0}
\pscustom[linestyle=none,fillstyle=solid,fillcolor=curcolor]
{
\newpath
\moveto(844.16303711,382.21897308)
\curveto(844.90303232,382.22896515)(845.5180317,382.11896526)(846.00803711,381.88897308)
\curveto(846.50803071,381.66896571)(846.90303032,381.33396605)(847.19303711,380.88397308)
\curveto(847.3230299,380.6839667)(847.43302979,380.43896694)(847.52303711,380.14897308)
\curveto(847.54302968,380.09896728)(847.55802966,380.03396735)(847.56803711,379.95397308)
\curveto(847.57802964,379.87396751)(847.57302965,379.80396758)(847.55303711,379.74397308)
\curveto(847.5230297,379.69396769)(847.47302975,379.64896773)(847.40303711,379.60897308)
\curveto(847.37302985,379.58896779)(847.34302988,379.5789678)(847.31303711,379.57897308)
\curveto(847.28302994,379.58896779)(847.24802997,379.58896779)(847.20803711,379.57897308)
\curveto(847.16803005,379.56896781)(847.12803009,379.56396782)(847.08803711,379.56397308)
\curveto(847.04803017,379.57396781)(847.00803021,379.5789678)(846.96803711,379.57897308)
\lineto(846.65303711,379.57897308)
\curveto(846.55303067,379.58896779)(846.46803075,379.61896776)(846.39803711,379.66897308)
\curveto(846.3180309,379.72896765)(846.26303096,379.81396757)(846.23303711,379.92397308)
\curveto(846.20303102,380.03396735)(846.16303106,380.12896725)(846.11303711,380.20897308)
\curveto(845.96303126,380.46896691)(845.76803145,380.67396671)(845.52803711,380.82397308)
\curveto(845.44803177,380.87396651)(845.36303186,380.91396647)(845.27303711,380.94397308)
\curveto(845.18303204,380.9839664)(845.08803213,381.01896636)(844.98803711,381.04897308)
\curveto(844.84803237,381.08896629)(844.66303256,381.10896627)(844.43303711,381.10897308)
\curveto(844.20303302,381.11896626)(844.01303321,381.09896628)(843.86303711,381.04897308)
\curveto(843.79303343,381.02896635)(843.72803349,381.01396637)(843.66803711,381.00397308)
\curveto(843.60803361,380.99396639)(843.54303368,380.9789664)(843.47303711,380.95897308)
\curveto(843.21303401,380.84896653)(842.98303424,380.69896668)(842.78303711,380.50897308)
\curveto(842.58303464,380.31896706)(842.42803479,380.09396729)(842.31803711,379.83397308)
\curveto(842.27803494,379.74396764)(842.24303498,379.64896773)(842.21303711,379.54897308)
\curveto(842.18303504,379.45896792)(842.15303507,379.35896802)(842.12303711,379.24897308)
\lineto(842.03303711,378.84397308)
\curveto(842.0230352,378.79396859)(842.0180352,378.73896864)(842.01803711,378.67897308)
\curveto(842.02803519,378.61896876)(842.0230352,378.56396882)(842.00303711,378.51397308)
\lineto(842.00303711,378.39397308)
\curveto(841.99303523,378.35396903)(841.98303524,378.28896909)(841.97303711,378.19897308)
\curveto(841.97303525,378.10896927)(841.98303524,378.04396934)(842.00303711,378.00397308)
\curveto(842.01303521,377.95396943)(842.01303521,377.90396948)(842.00303711,377.85397308)
\curveto(841.99303523,377.80396958)(841.99303523,377.75396963)(842.00303711,377.70397308)
\curveto(842.01303521,377.66396972)(842.0180352,377.59396979)(842.01803711,377.49397308)
\curveto(842.03803518,377.41396997)(842.05303517,377.32897005)(842.06303711,377.23897308)
\curveto(842.08303514,377.14897023)(842.10303512,377.06397032)(842.12303711,376.98397308)
\curveto(842.23303499,376.66397072)(842.35803486,376.383971)(842.49803711,376.14397308)
\curveto(842.64803457,375.91397147)(842.85303437,375.71397167)(843.11303711,375.54397308)
\curveto(843.20303402,375.49397189)(843.29303393,375.44897193)(843.38303711,375.40897308)
\curveto(843.48303374,375.36897201)(843.58803363,375.32897205)(843.69803711,375.28897308)
\curveto(843.74803347,375.2789721)(843.78803343,375.27397211)(843.81803711,375.27397308)
\curveto(843.84803337,375.27397211)(843.88803333,375.26897211)(843.93803711,375.25897308)
\curveto(843.96803325,375.24897213)(844.0180332,375.24397214)(844.08803711,375.24397308)
\lineto(844.25303711,375.24397308)
\curveto(844.25303297,375.23397215)(844.27303295,375.22897215)(844.31303711,375.22897308)
\curveto(844.33303289,375.23897214)(844.35803286,375.23897214)(844.38803711,375.22897308)
\curveto(844.4180328,375.22897215)(844.44803277,375.23397215)(844.47803711,375.24397308)
\curveto(844.54803267,375.26397212)(844.61303261,375.26897211)(844.67303711,375.25897308)
\curveto(844.74303248,375.25897212)(844.81303241,375.26897211)(844.88303711,375.28897308)
\curveto(845.14303208,375.36897201)(845.36803185,375.46897191)(845.55803711,375.58897308)
\curveto(845.74803147,375.71897166)(845.90803131,375.8839715)(846.03803711,376.08397308)
\curveto(846.08803113,376.16397122)(846.13303109,376.24897113)(846.17303711,376.33897308)
\lineto(846.29303711,376.60897308)
\curveto(846.31303091,376.68897069)(846.33303089,376.76397062)(846.35303711,376.83397308)
\curveto(846.38303084,376.91397047)(846.43303079,376.9789704)(846.50303711,377.02897308)
\curveto(846.53303069,377.05897032)(846.59303063,377.0789703)(846.68303711,377.08897308)
\curveto(846.77303045,377.10897027)(846.86803035,377.11897026)(846.96803711,377.11897308)
\curveto(847.07803014,377.12897025)(847.17803004,377.12897025)(847.26803711,377.11897308)
\curveto(847.36802985,377.10897027)(847.43802978,377.08897029)(847.47803711,377.05897308)
\curveto(847.53802968,377.01897036)(847.57302965,376.95897042)(847.58303711,376.87897308)
\curveto(847.60302962,376.79897058)(847.60302962,376.71397067)(847.58303711,376.62397308)
\curveto(847.53302969,376.47397091)(847.48302974,376.32897105)(847.43303711,376.18897308)
\curveto(847.39302983,376.05897132)(847.33802988,375.92897145)(847.26803711,375.79897308)
\curveto(847.1180301,375.49897188)(846.92803029,375.23397215)(846.69803711,375.00397308)
\curveto(846.47803074,374.77397261)(846.20803101,374.58897279)(845.88803711,374.44897308)
\curveto(845.80803141,374.40897297)(845.7230315,374.37397301)(845.63303711,374.34397308)
\curveto(845.54303168,374.32397306)(845.44803177,374.29897308)(845.34803711,374.26897308)
\curveto(845.23803198,374.22897315)(845.12803209,374.20897317)(845.01803711,374.20897308)
\curveto(844.90803231,374.19897318)(844.79803242,374.1839732)(844.68803711,374.16397308)
\curveto(844.64803257,374.14397324)(844.60803261,374.13897324)(844.56803711,374.14897308)
\curveto(844.52803269,374.15897322)(844.48803273,374.15897322)(844.44803711,374.14897308)
\lineto(844.31303711,374.14897308)
\lineto(844.07303711,374.14897308)
\curveto(844.00303322,374.13897324)(843.93803328,374.14397324)(843.87803711,374.16397308)
\lineto(843.80303711,374.16397308)
\lineto(843.44303711,374.20897308)
\curveto(843.31303391,374.24897313)(843.18803403,374.2839731)(843.06803711,374.31397308)
\curveto(842.94803427,374.34397304)(842.83303439,374.383973)(842.72303711,374.43397308)
\curveto(842.36303486,374.59397279)(842.06303516,374.7839726)(841.82303711,375.00397308)
\curveto(841.59303563,375.22397216)(841.37803584,375.49397189)(841.17803711,375.81397308)
\curveto(841.12803609,375.89397149)(841.08303614,375.9839714)(841.04303711,376.08397308)
\lineto(840.92303711,376.38397308)
\curveto(840.87303635,376.49397089)(840.83803638,376.60897077)(840.81803711,376.72897308)
\curveto(840.79803642,376.84897053)(840.77303645,376.96897041)(840.74303711,377.08897308)
\curveto(840.73303649,377.12897025)(840.72803649,377.16897021)(840.72803711,377.20897308)
\curveto(840.72803649,377.24897013)(840.7230365,377.28897009)(840.71303711,377.32897308)
\curveto(840.69303653,377.38896999)(840.68303654,377.45396993)(840.68303711,377.52397308)
\curveto(840.69303653,377.59396979)(840.68803653,377.65896972)(840.66803711,377.71897308)
\lineto(840.66803711,377.86897308)
\curveto(840.65803656,377.91896946)(840.65303657,377.98896939)(840.65303711,378.07897308)
\curveto(840.65303657,378.16896921)(840.65803656,378.23896914)(840.66803711,378.28897308)
\curveto(840.67803654,378.33896904)(840.67803654,378.383969)(840.66803711,378.42397308)
\curveto(840.66803655,378.46396892)(840.67303655,378.50396888)(840.68303711,378.54397308)
\curveto(840.70303652,378.61396877)(840.70803651,378.6839687)(840.69803711,378.75397308)
\curveto(840.69803652,378.82396856)(840.70803651,378.88896849)(840.72803711,378.94897308)
\curveto(840.76803645,379.11896826)(840.80303642,379.28896809)(840.83303711,379.45897308)
\curveto(840.86303636,379.62896775)(840.90803631,379.78896759)(840.96803711,379.93897308)
\curveto(841.17803604,380.45896692)(841.43303579,380.8789665)(841.73303711,381.19897308)
\curveto(842.03303519,381.51896586)(842.44303478,381.7839656)(842.96303711,381.99397308)
\curveto(843.07303415,382.04396534)(843.19303403,382.0789653)(843.32303711,382.09897308)
\curveto(843.45303377,382.11896526)(843.58803363,382.14396524)(843.72803711,382.17397308)
\curveto(843.79803342,382.1839652)(843.86803335,382.18896519)(843.93803711,382.18897308)
\curveto(844.00803321,382.19896518)(844.08303314,382.20896517)(844.16303711,382.21897308)
}
}
{
\newrgbcolor{curcolor}{0 0 0}
\pscustom[linestyle=none,fillstyle=solid,fillcolor=curcolor]
{
\newpath
\moveto(850.02967773,384.37897308)
\curveto(850.17967572,384.378963)(850.32967557,384.37396301)(850.47967773,384.36397308)
\curveto(850.62967527,384.36396302)(850.73467517,384.32396306)(850.79467773,384.24397308)
\curveto(850.84467506,384.1839632)(850.86967503,384.09896328)(850.86967773,383.98897308)
\curveto(850.87967502,383.88896349)(850.88467502,383.7839636)(850.88467773,383.67397308)
\lineto(850.88467773,382.80397308)
\curveto(850.88467502,382.72396466)(850.87967502,382.63896474)(850.86967773,382.54897308)
\curveto(850.86967503,382.46896491)(850.87967502,382.39896498)(850.89967773,382.33897308)
\curveto(850.93967496,382.19896518)(851.02967487,382.10896527)(851.16967773,382.06897308)
\curveto(851.21967468,382.05896532)(851.26467464,382.05396533)(851.30467773,382.05397308)
\lineto(851.45467773,382.05397308)
\lineto(851.85967773,382.05397308)
\curveto(852.01967388,382.06396532)(852.13467377,382.05396533)(852.20467773,382.02397308)
\curveto(852.29467361,381.96396542)(852.35467355,381.90396548)(852.38467773,381.84397308)
\curveto(852.4046735,381.80396558)(852.41467349,381.75896562)(852.41467773,381.70897308)
\lineto(852.41467773,381.55897308)
\curveto(852.41467349,381.44896593)(852.40967349,381.34396604)(852.39967773,381.24397308)
\curveto(852.38967351,381.15396623)(852.35467355,381.0839663)(852.29467773,381.03397308)
\curveto(852.23467367,380.9839664)(852.14967375,380.95396643)(852.03967773,380.94397308)
\lineto(851.70967773,380.94397308)
\curveto(851.5996743,380.95396643)(851.48967441,380.95896642)(851.37967773,380.95897308)
\curveto(851.26967463,380.95896642)(851.17467473,380.94396644)(851.09467773,380.91397308)
\curveto(851.02467488,380.8839665)(850.97467493,380.83396655)(850.94467773,380.76397308)
\curveto(850.91467499,380.69396669)(850.89467501,380.60896677)(850.88467773,380.50897308)
\curveto(850.87467503,380.41896696)(850.86967503,380.31896706)(850.86967773,380.20897308)
\curveto(850.87967502,380.10896727)(850.88467502,380.00896737)(850.88467773,379.90897308)
\lineto(850.88467773,376.93897308)
\curveto(850.88467502,376.71897066)(850.87967502,376.4839709)(850.86967773,376.23397308)
\curveto(850.86967503,375.99397139)(850.91467499,375.80897157)(851.00467773,375.67897308)
\curveto(851.05467485,375.59897178)(851.11967478,375.54397184)(851.19967773,375.51397308)
\curveto(851.27967462,375.4839719)(851.37467453,375.45897192)(851.48467773,375.43897308)
\curveto(851.51467439,375.42897195)(851.54467436,375.42397196)(851.57467773,375.42397308)
\curveto(851.61467429,375.43397195)(851.64967425,375.43397195)(851.67967773,375.42397308)
\lineto(851.87467773,375.42397308)
\curveto(851.97467393,375.42397196)(852.06467384,375.41397197)(852.14467773,375.39397308)
\curveto(852.23467367,375.383972)(852.2996736,375.34897203)(852.33967773,375.28897308)
\curveto(852.35967354,375.25897212)(852.37467353,375.20397218)(852.38467773,375.12397308)
\curveto(852.4046735,375.05397233)(852.41467349,374.9789724)(852.41467773,374.89897308)
\curveto(852.42467348,374.81897256)(852.42467348,374.73897264)(852.41467773,374.65897308)
\curveto(852.4046735,374.58897279)(852.38467352,374.53397285)(852.35467773,374.49397308)
\curveto(852.31467359,374.42397296)(852.23967366,374.37397301)(852.12967773,374.34397308)
\curveto(852.04967385,374.32397306)(851.95967394,374.31397307)(851.85967773,374.31397308)
\curveto(851.75967414,374.32397306)(851.66967423,374.32897305)(851.58967773,374.32897308)
\curveto(851.52967437,374.32897305)(851.46967443,374.32397306)(851.40967773,374.31397308)
\curveto(851.34967455,374.31397307)(851.29467461,374.31897306)(851.24467773,374.32897308)
\lineto(851.06467773,374.32897308)
\curveto(851.01467489,374.33897304)(850.96467494,374.34397304)(850.91467773,374.34397308)
\curveto(850.87467503,374.35397303)(850.82967507,374.35897302)(850.77967773,374.35897308)
\curveto(850.57967532,374.40897297)(850.4046755,374.46397292)(850.25467773,374.52397308)
\curveto(850.11467579,374.5839728)(849.99467591,374.68897269)(849.89467773,374.83897308)
\curveto(849.75467615,375.03897234)(849.67467623,375.28897209)(849.65467773,375.58897308)
\curveto(849.63467627,375.89897148)(849.62467628,376.22897115)(849.62467773,376.57897308)
\lineto(849.62467773,380.50897308)
\curveto(849.59467631,380.63896674)(849.56467634,380.73396665)(849.53467773,380.79397308)
\curveto(849.51467639,380.85396653)(849.44467646,380.90396648)(849.32467773,380.94397308)
\curveto(849.28467662,380.95396643)(849.24467666,380.95396643)(849.20467773,380.94397308)
\curveto(849.16467674,380.93396645)(849.12467678,380.93896644)(849.08467773,380.95897308)
\lineto(848.84467773,380.95897308)
\curveto(848.71467719,380.95896642)(848.6046773,380.96896641)(848.51467773,380.98897308)
\curveto(848.43467747,381.01896636)(848.37967752,381.0789663)(848.34967773,381.16897308)
\curveto(848.32967757,381.20896617)(848.31467759,381.25396613)(848.30467773,381.30397308)
\lineto(848.30467773,381.45397308)
\curveto(848.3046776,381.59396579)(848.31467759,381.70896567)(848.33467773,381.79897308)
\curveto(848.35467755,381.89896548)(848.41467749,381.97396541)(848.51467773,382.02397308)
\curveto(848.62467728,382.06396532)(848.76467714,382.07396531)(848.93467773,382.05397308)
\curveto(849.11467679,382.03396535)(849.26467664,382.04396534)(849.38467773,382.08397308)
\curveto(849.47467643,382.13396525)(849.54467636,382.20396518)(849.59467773,382.29397308)
\curveto(849.61467629,382.35396503)(849.62467628,382.42896495)(849.62467773,382.51897308)
\lineto(849.62467773,382.77397308)
\lineto(849.62467773,383.70397308)
\lineto(849.62467773,383.94397308)
\curveto(849.62467628,384.03396335)(849.63467627,384.10896327)(849.65467773,384.16897308)
\curveto(849.69467621,384.24896313)(849.76967613,384.31396307)(849.87967773,384.36397308)
\curveto(849.90967599,384.36396302)(849.93467597,384.36396302)(849.95467773,384.36397308)
\curveto(849.98467592,384.37396301)(850.00967589,384.378963)(850.02967773,384.37897308)
}
}
{
\newrgbcolor{curcolor}{0 0 0}
\pscustom[linestyle=none,fillstyle=solid,fillcolor=curcolor]
{
\newpath
\moveto(854.08647461,383.53897308)
\curveto(854.00647349,383.59896378)(853.96147353,383.70396368)(853.95147461,383.85397308)
\lineto(853.95147461,384.31897308)
\lineto(853.95147461,384.57397308)
\curveto(853.95147354,384.66396272)(853.96647353,384.73896264)(853.99647461,384.79897308)
\curveto(854.03647346,384.8789625)(854.11647338,384.93896244)(854.23647461,384.97897308)
\curveto(854.25647324,384.98896239)(854.27647322,384.98896239)(854.29647461,384.97897308)
\curveto(854.32647317,384.9789624)(854.35147314,384.9839624)(854.37147461,384.99397308)
\curveto(854.54147295,384.99396239)(854.70147279,384.98896239)(854.85147461,384.97897308)
\curveto(855.00147249,384.96896241)(855.10147239,384.90896247)(855.15147461,384.79897308)
\curveto(855.18147231,384.73896264)(855.1964723,384.66396272)(855.19647461,384.57397308)
\lineto(855.19647461,384.31897308)
\curveto(855.1964723,384.13896324)(855.1914723,383.96896341)(855.18147461,383.80897308)
\curveto(855.18147231,383.64896373)(855.11647238,383.54396384)(854.98647461,383.49397308)
\curveto(854.93647256,383.47396391)(854.88147261,383.46396392)(854.82147461,383.46397308)
\lineto(854.65647461,383.46397308)
\lineto(854.34147461,383.46397308)
\curveto(854.24147325,383.46396392)(854.15647334,383.48896389)(854.08647461,383.53897308)
\moveto(855.19647461,375.03397308)
\lineto(855.19647461,374.71897308)
\curveto(855.20647229,374.61897276)(855.18647231,374.53897284)(855.13647461,374.47897308)
\curveto(855.10647239,374.41897296)(855.06147243,374.378973)(855.00147461,374.35897308)
\curveto(854.94147255,374.34897303)(854.87147262,374.33397305)(854.79147461,374.31397308)
\lineto(854.56647461,374.31397308)
\curveto(854.43647306,374.31397307)(854.32147317,374.31897306)(854.22147461,374.32897308)
\curveto(854.13147336,374.34897303)(854.06147343,374.39897298)(854.01147461,374.47897308)
\curveto(853.97147352,374.53897284)(853.95147354,374.61397277)(853.95147461,374.70397308)
\lineto(853.95147461,374.98897308)
\lineto(853.95147461,381.33397308)
\lineto(853.95147461,381.64897308)
\curveto(853.95147354,381.75896562)(853.97647352,381.84396554)(854.02647461,381.90397308)
\curveto(854.05647344,381.95396543)(854.0964734,381.9839654)(854.14647461,381.99397308)
\curveto(854.1964733,382.00396538)(854.25147324,382.01896536)(854.31147461,382.03897308)
\curveto(854.33147316,382.03896534)(854.35147314,382.03396535)(854.37147461,382.02397308)
\curveto(854.40147309,382.02396536)(854.42647307,382.02896535)(854.44647461,382.03897308)
\curveto(854.57647292,382.03896534)(854.70647279,382.03396535)(854.83647461,382.02397308)
\curveto(854.97647252,382.02396536)(855.07147242,381.9839654)(855.12147461,381.90397308)
\curveto(855.17147232,381.84396554)(855.1964723,381.76396562)(855.19647461,381.66397308)
\lineto(855.19647461,381.37897308)
\lineto(855.19647461,375.03397308)
}
}
{
\newrgbcolor{curcolor}{0 0 0}
\pscustom[linestyle=none,fillstyle=solid,fillcolor=curcolor]
{
\newpath
\moveto(856.91631836,382.03897308)
\lineto(857.39631836,382.03897308)
\curveto(857.56631702,382.03896534)(857.69631689,382.00896537)(857.78631836,381.94897308)
\curveto(857.85631673,381.89896548)(857.90131668,381.83396555)(857.92131836,381.75397308)
\curveto(857.95131663,381.6839657)(857.9813166,381.60896577)(858.01131836,381.52897308)
\curveto(858.07131651,381.38896599)(858.12131646,381.24896613)(858.16131836,381.10897308)
\curveto(858.20131638,380.96896641)(858.24631634,380.82896655)(858.29631836,380.68897308)
\curveto(858.49631609,380.14896723)(858.6813159,379.60396778)(858.85131836,379.05397308)
\curveto(859.02131556,378.51396887)(859.20631538,377.97396941)(859.40631836,377.43397308)
\curveto(859.47631511,377.25397013)(859.53631505,377.06897031)(859.58631836,376.87897308)
\curveto(859.63631495,376.69897068)(859.70131488,376.51897086)(859.78131836,376.33897308)
\curveto(859.80131478,376.26897111)(859.82631476,376.19397119)(859.85631836,376.11397308)
\curveto(859.8863147,376.03397135)(859.93631465,375.9839714)(860.00631836,375.96397308)
\curveto(860.0863145,375.94397144)(860.14631444,375.9789714)(860.18631836,376.06897308)
\curveto(860.23631435,376.15897122)(860.27131431,376.22897115)(860.29131836,376.27897308)
\curveto(860.37131421,376.46897091)(860.43631415,376.65897072)(860.48631836,376.84897308)
\curveto(860.54631404,377.04897033)(860.61131397,377.24897013)(860.68131836,377.44897308)
\curveto(860.81131377,377.82896955)(860.93631365,378.20396918)(861.05631836,378.57397308)
\curveto(861.17631341,378.95396843)(861.30131328,379.33396805)(861.43131836,379.71397308)
\curveto(861.4813131,379.8839675)(861.53131305,380.04896733)(861.58131836,380.20897308)
\curveto(861.63131295,380.378967)(861.69131289,380.54396684)(861.76131836,380.70397308)
\curveto(861.81131277,380.84396654)(861.85631273,380.9839664)(861.89631836,381.12397308)
\curveto(861.93631265,381.26396612)(861.9813126,381.40396598)(862.03131836,381.54397308)
\curveto(862.05131253,381.61396577)(862.07631251,381.6839657)(862.10631836,381.75397308)
\curveto(862.13631245,381.82396556)(862.17631241,381.8839655)(862.22631836,381.93397308)
\curveto(862.30631228,381.9839654)(862.39631219,382.01396537)(862.49631836,382.02397308)
\curveto(862.59631199,382.03396535)(862.71631187,382.03896534)(862.85631836,382.03897308)
\curveto(862.92631166,382.03896534)(862.99131159,382.03396535)(863.05131836,382.02397308)
\curveto(863.11131147,382.02396536)(863.16631142,382.01396537)(863.21631836,381.99397308)
\curveto(863.30631128,381.95396543)(863.35131123,381.88896549)(863.35131836,381.79897308)
\curveto(863.36131122,381.70896567)(863.34631124,381.61896576)(863.30631836,381.52897308)
\curveto(863.24631134,381.35896602)(863.1863114,381.1839662)(863.12631836,381.00397308)
\curveto(863.06631152,380.82396656)(862.99631159,380.64896673)(862.91631836,380.47897308)
\curveto(862.89631169,380.42896695)(862.8813117,380.378967)(862.87131836,380.32897308)
\curveto(862.86131172,380.28896709)(862.84631174,380.24396714)(862.82631836,380.19397308)
\curveto(862.74631184,380.02396736)(862.6813119,379.84896753)(862.63131836,379.66897308)
\curveto(862.581312,379.48896789)(862.51631207,379.30896807)(862.43631836,379.12897308)
\curveto(862.3863122,378.99896838)(862.33631225,378.86396852)(862.28631836,378.72397308)
\curveto(862.24631234,378.59396879)(862.19631239,378.46396892)(862.13631836,378.33397308)
\curveto(861.96631262,377.92396946)(861.81131277,377.50896987)(861.67131836,377.08897308)
\curveto(861.54131304,376.66897071)(861.39131319,376.25397113)(861.22131836,375.84397308)
\curveto(861.16131342,375.6839717)(861.10631348,375.52397186)(861.05631836,375.36397308)
\curveto(861.00631358,375.20397218)(860.94631364,375.04397234)(860.87631836,374.88397308)
\curveto(860.82631376,374.77397261)(860.7813138,374.66897271)(860.74131836,374.56897308)
\curveto(860.71131387,374.4789729)(860.64131394,374.40897297)(860.53131836,374.35897308)
\curveto(860.47131411,374.32897305)(860.40131418,374.31397307)(860.32131836,374.31397308)
\lineto(860.09631836,374.31397308)
\lineto(859.63131836,374.31397308)
\curveto(859.4813151,374.32397306)(859.37131521,374.37397301)(859.30131836,374.46397308)
\curveto(859.23131535,374.54397284)(859.1813154,374.63897274)(859.15131836,374.74897308)
\curveto(859.12131546,374.86897251)(859.0813155,374.9839724)(859.03131836,375.09397308)
\curveto(858.97131561,375.23397215)(858.91131567,375.378972)(858.85131836,375.52897308)
\curveto(858.80131578,375.68897169)(858.75131583,375.83897154)(858.70131836,375.97897308)
\curveto(858.6813159,376.02897135)(858.66631592,376.06897131)(858.65631836,376.09897308)
\curveto(858.64631594,376.13897124)(858.63131595,376.1839712)(858.61131836,376.23397308)
\curveto(858.41131617,376.71397067)(858.22631636,377.19897018)(858.05631836,377.68897308)
\curveto(857.89631669,378.1789692)(857.71631687,378.66396872)(857.51631836,379.14397308)
\curveto(857.45631713,379.30396808)(857.39631719,379.45896792)(857.33631836,379.60897308)
\curveto(857.2863173,379.76896761)(857.23131735,379.92896745)(857.17131836,380.08897308)
\lineto(857.11131836,380.23897308)
\curveto(857.10131748,380.29896708)(857.0863175,380.35396703)(857.06631836,380.40397308)
\curveto(856.9863176,380.57396681)(856.91631767,380.74396664)(856.85631836,380.91397308)
\curveto(856.80631778,381.0839663)(856.74631784,381.25396613)(856.67631836,381.42397308)
\curveto(856.65631793,381.4839659)(856.63131795,381.56396582)(856.60131836,381.66397308)
\curveto(856.57131801,381.76396562)(856.57631801,381.84896553)(856.61631836,381.91897308)
\curveto(856.66631792,381.96896541)(856.72631786,382.00396538)(856.79631836,382.02397308)
\curveto(856.86631772,382.02396536)(856.90631768,382.02896535)(856.91631836,382.03897308)
}
}
{
\newrgbcolor{curcolor}{0 0 0}
\pscustom[linestyle=none,fillstyle=solid,fillcolor=curcolor]
{
\newpath
\moveto(864.92631836,383.53897308)
\curveto(864.84631724,383.59896378)(864.80131728,383.70396368)(864.79131836,383.85397308)
\lineto(864.79131836,384.31897308)
\lineto(864.79131836,384.57397308)
\curveto(864.79131729,384.66396272)(864.80631728,384.73896264)(864.83631836,384.79897308)
\curveto(864.87631721,384.8789625)(864.95631713,384.93896244)(865.07631836,384.97897308)
\curveto(865.09631699,384.98896239)(865.11631697,384.98896239)(865.13631836,384.97897308)
\curveto(865.16631692,384.9789624)(865.19131689,384.9839624)(865.21131836,384.99397308)
\curveto(865.3813167,384.99396239)(865.54131654,384.98896239)(865.69131836,384.97897308)
\curveto(865.84131624,384.96896241)(865.94131614,384.90896247)(865.99131836,384.79897308)
\curveto(866.02131606,384.73896264)(866.03631605,384.66396272)(866.03631836,384.57397308)
\lineto(866.03631836,384.31897308)
\curveto(866.03631605,384.13896324)(866.03131605,383.96896341)(866.02131836,383.80897308)
\curveto(866.02131606,383.64896373)(865.95631613,383.54396384)(865.82631836,383.49397308)
\curveto(865.77631631,383.47396391)(865.72131636,383.46396392)(865.66131836,383.46397308)
\lineto(865.49631836,383.46397308)
\lineto(865.18131836,383.46397308)
\curveto(865.081317,383.46396392)(864.99631709,383.48896389)(864.92631836,383.53897308)
\moveto(866.03631836,375.03397308)
\lineto(866.03631836,374.71897308)
\curveto(866.04631604,374.61897276)(866.02631606,374.53897284)(865.97631836,374.47897308)
\curveto(865.94631614,374.41897296)(865.90131618,374.378973)(865.84131836,374.35897308)
\curveto(865.7813163,374.34897303)(865.71131637,374.33397305)(865.63131836,374.31397308)
\lineto(865.40631836,374.31397308)
\curveto(865.27631681,374.31397307)(865.16131692,374.31897306)(865.06131836,374.32897308)
\curveto(864.97131711,374.34897303)(864.90131718,374.39897298)(864.85131836,374.47897308)
\curveto(864.81131727,374.53897284)(864.79131729,374.61397277)(864.79131836,374.70397308)
\lineto(864.79131836,374.98897308)
\lineto(864.79131836,381.33397308)
\lineto(864.79131836,381.64897308)
\curveto(864.79131729,381.75896562)(864.81631727,381.84396554)(864.86631836,381.90397308)
\curveto(864.89631719,381.95396543)(864.93631715,381.9839654)(864.98631836,381.99397308)
\curveto(865.03631705,382.00396538)(865.09131699,382.01896536)(865.15131836,382.03897308)
\curveto(865.17131691,382.03896534)(865.19131689,382.03396535)(865.21131836,382.02397308)
\curveto(865.24131684,382.02396536)(865.26631682,382.02896535)(865.28631836,382.03897308)
\curveto(865.41631667,382.03896534)(865.54631654,382.03396535)(865.67631836,382.02397308)
\curveto(865.81631627,382.02396536)(865.91131617,381.9839654)(865.96131836,381.90397308)
\curveto(866.01131607,381.84396554)(866.03631605,381.76396562)(866.03631836,381.66397308)
\lineto(866.03631836,381.37897308)
\lineto(866.03631836,375.03397308)
}
}
{
\newrgbcolor{curcolor}{0 0 0}
\pscustom[linestyle=none,fillstyle=solid,fillcolor=curcolor]
{
\newpath
\moveto(874.94116211,375.12397308)
\lineto(874.94116211,374.73397308)
\curveto(874.94115423,374.61397277)(874.91615426,374.51397287)(874.86616211,374.43397308)
\curveto(874.81615436,374.36397302)(874.73115444,374.32397306)(874.61116211,374.31397308)
\lineto(874.26616211,374.31397308)
\curveto(874.20615497,374.31397307)(874.14615503,374.30897307)(874.08616211,374.29897308)
\curveto(874.03615514,374.29897308)(873.99115518,374.30897307)(873.95116211,374.32897308)
\curveto(873.86115531,374.34897303)(873.80115537,374.38897299)(873.77116211,374.44897308)
\curveto(873.73115544,374.49897288)(873.70615547,374.55897282)(873.69616211,374.62897308)
\curveto(873.69615548,374.69897268)(873.68115549,374.76897261)(873.65116211,374.83897308)
\curveto(873.64115553,374.85897252)(873.62615555,374.87397251)(873.60616211,374.88397308)
\curveto(873.59615558,374.90397248)(873.58115559,374.92397246)(873.56116211,374.94397308)
\curveto(873.46115571,374.95397243)(873.38115579,374.93397245)(873.32116211,374.88397308)
\curveto(873.2711559,374.83397255)(873.21615596,374.7839726)(873.15616211,374.73397308)
\curveto(872.95615622,374.5839728)(872.75615642,374.46897291)(872.55616211,374.38897308)
\curveto(872.3761568,374.30897307)(872.16615701,374.24897313)(871.92616211,374.20897308)
\curveto(871.69615748,374.16897321)(871.45615772,374.14897323)(871.20616211,374.14897308)
\curveto(870.96615821,374.13897324)(870.72615845,374.15397323)(870.48616211,374.19397308)
\curveto(870.24615893,374.22397316)(870.03615914,374.2789731)(869.85616211,374.35897308)
\curveto(869.33615984,374.5789728)(868.91616026,374.87397251)(868.59616211,375.24397308)
\curveto(868.2761609,375.62397176)(868.02616115,376.09397129)(867.84616211,376.65397308)
\curveto(867.80616137,376.74397064)(867.7761614,376.83397055)(867.75616211,376.92397308)
\curveto(867.74616143,377.02397036)(867.72616145,377.12397026)(867.69616211,377.22397308)
\curveto(867.68616149,377.27397011)(867.68116149,377.32397006)(867.68116211,377.37397308)
\curveto(867.68116149,377.42396996)(867.6761615,377.47396991)(867.66616211,377.52397308)
\curveto(867.64616153,377.57396981)(867.63616154,377.62396976)(867.63616211,377.67397308)
\curveto(867.64616153,377.73396965)(867.64616153,377.78896959)(867.63616211,377.83897308)
\lineto(867.63616211,377.98897308)
\curveto(867.61616156,378.03896934)(867.60616157,378.10396928)(867.60616211,378.18397308)
\curveto(867.60616157,378.26396912)(867.61616156,378.32896905)(867.63616211,378.37897308)
\lineto(867.63616211,378.54397308)
\curveto(867.65616152,378.61396877)(867.66116151,378.6839687)(867.65116211,378.75397308)
\curveto(867.65116152,378.83396855)(867.66116151,378.90896847)(867.68116211,378.97897308)
\curveto(867.69116148,379.02896835)(867.69616148,379.07396831)(867.69616211,379.11397308)
\curveto(867.69616148,379.15396823)(867.70116147,379.19896818)(867.71116211,379.24897308)
\curveto(867.74116143,379.34896803)(867.76616141,379.44396794)(867.78616211,379.53397308)
\curveto(867.80616137,379.63396775)(867.83116134,379.72896765)(867.86116211,379.81897308)
\curveto(867.99116118,380.19896718)(868.15616102,380.53896684)(868.35616211,380.83897308)
\curveto(868.56616061,381.14896623)(868.81616036,381.40396598)(869.10616211,381.60397308)
\curveto(869.2761599,381.72396566)(869.45115972,381.82396556)(869.63116211,381.90397308)
\curveto(869.82115935,381.9839654)(870.02615915,382.05396533)(870.24616211,382.11397308)
\curveto(870.31615886,382.12396526)(870.38115879,382.13396525)(870.44116211,382.14397308)
\curveto(870.51115866,382.15396523)(870.58115859,382.16896521)(870.65116211,382.18897308)
\lineto(870.80116211,382.18897308)
\curveto(870.88115829,382.20896517)(870.99615818,382.21896516)(871.14616211,382.21897308)
\curveto(871.30615787,382.21896516)(871.42615775,382.20896517)(871.50616211,382.18897308)
\curveto(871.54615763,382.1789652)(871.60115757,382.17396521)(871.67116211,382.17397308)
\curveto(871.78115739,382.14396524)(871.89115728,382.11896526)(872.00116211,382.09897308)
\curveto(872.11115706,382.08896529)(872.21615696,382.05896532)(872.31616211,382.00897308)
\curveto(872.46615671,381.94896543)(872.60615657,381.8839655)(872.73616211,381.81397308)
\curveto(872.8761563,381.74396564)(873.00615617,381.66396572)(873.12616211,381.57397308)
\curveto(873.18615599,381.52396586)(873.24615593,381.46896591)(873.30616211,381.40897308)
\curveto(873.3761558,381.35896602)(873.46615571,381.34396604)(873.57616211,381.36397308)
\curveto(873.59615558,381.39396599)(873.61115556,381.41896596)(873.62116211,381.43897308)
\curveto(873.64115553,381.45896592)(873.65615552,381.48896589)(873.66616211,381.52897308)
\curveto(873.69615548,381.61896576)(873.70615547,381.73396565)(873.69616211,381.87397308)
\lineto(873.69616211,382.24897308)
\lineto(873.69616211,383.97397308)
\lineto(873.69616211,384.43897308)
\curveto(873.69615548,384.61896276)(873.72115545,384.74896263)(873.77116211,384.82897308)
\curveto(873.81115536,384.89896248)(873.8711553,384.94396244)(873.95116211,384.96397308)
\curveto(873.9711552,384.96396242)(873.99615518,384.96396242)(874.02616211,384.96397308)
\curveto(874.05615512,384.97396241)(874.08115509,384.9789624)(874.10116211,384.97897308)
\curveto(874.24115493,384.98896239)(874.38615479,384.98896239)(874.53616211,384.97897308)
\curveto(874.69615448,384.9789624)(874.80615437,384.93896244)(874.86616211,384.85897308)
\curveto(874.91615426,384.7789626)(874.94115423,384.6789627)(874.94116211,384.55897308)
\lineto(874.94116211,384.18397308)
\lineto(874.94116211,375.12397308)
\moveto(873.72616211,377.95897308)
\curveto(873.74615543,378.00896937)(873.75615542,378.07396931)(873.75616211,378.15397308)
\curveto(873.75615542,378.24396914)(873.74615543,378.31396907)(873.72616211,378.36397308)
\lineto(873.72616211,378.58897308)
\curveto(873.70615547,378.6789687)(873.69115548,378.76896861)(873.68116211,378.85897308)
\curveto(873.6711555,378.95896842)(873.65115552,379.04896833)(873.62116211,379.12897308)
\curveto(873.60115557,379.20896817)(873.58115559,379.2839681)(873.56116211,379.35397308)
\curveto(873.55115562,379.42396796)(873.53115564,379.49396789)(873.50116211,379.56397308)
\curveto(873.38115579,379.86396752)(873.22615595,380.12896725)(873.03616211,380.35897308)
\curveto(872.84615633,380.58896679)(872.60615657,380.76896661)(872.31616211,380.89897308)
\curveto(872.21615696,380.94896643)(872.11115706,380.9839664)(872.00116211,381.00397308)
\curveto(871.90115727,381.03396635)(871.79115738,381.05896632)(871.67116211,381.07897308)
\curveto(871.59115758,381.09896628)(871.50115767,381.10896627)(871.40116211,381.10897308)
\lineto(871.13116211,381.10897308)
\curveto(871.08115809,381.09896628)(871.03615814,381.08896629)(870.99616211,381.07897308)
\lineto(870.86116211,381.07897308)
\curveto(870.78115839,381.05896632)(870.69615848,381.03896634)(870.60616211,381.01897308)
\curveto(870.52615865,380.99896638)(870.44615873,380.97396641)(870.36616211,380.94397308)
\curveto(870.04615913,380.80396658)(869.78615939,380.59896678)(869.58616211,380.32897308)
\curveto(869.39615978,380.06896731)(869.24115993,379.76396762)(869.12116211,379.41397308)
\curveto(869.08116009,379.30396808)(869.05116012,379.18896819)(869.03116211,379.06897308)
\curveto(869.02116015,378.95896842)(869.00616017,378.84896853)(868.98616211,378.73897308)
\curveto(868.98616019,378.69896868)(868.98116019,378.65896872)(868.97116211,378.61897308)
\lineto(868.97116211,378.51397308)
\curveto(868.95116022,378.46396892)(868.94116023,378.40896897)(868.94116211,378.34897308)
\curveto(868.95116022,378.28896909)(868.95616022,378.23396915)(868.95616211,378.18397308)
\lineto(868.95616211,377.85397308)
\curveto(868.95616022,377.75396963)(868.96616021,377.65896972)(868.98616211,377.56897308)
\curveto(868.99616018,377.53896984)(869.00116017,377.48896989)(869.00116211,377.41897308)
\curveto(869.02116015,377.34897003)(869.03616014,377.2789701)(869.04616211,377.20897308)
\lineto(869.10616211,376.99897308)
\curveto(869.21615996,376.64897073)(869.36615981,376.34897103)(869.55616211,376.09897308)
\curveto(869.74615943,375.84897153)(869.98615919,375.64397174)(870.27616211,375.48397308)
\curveto(870.36615881,375.43397195)(870.45615872,375.39397199)(870.54616211,375.36397308)
\curveto(870.63615854,375.33397205)(870.73615844,375.30397208)(870.84616211,375.27397308)
\curveto(870.89615828,375.25397213)(870.94615823,375.24897213)(870.99616211,375.25897308)
\curveto(871.05615812,375.26897211)(871.11115806,375.26397212)(871.16116211,375.24397308)
\curveto(871.20115797,375.23397215)(871.24115793,375.22897215)(871.28116211,375.22897308)
\lineto(871.41616211,375.22897308)
\lineto(871.55116211,375.22897308)
\curveto(871.58115759,375.23897214)(871.63115754,375.24397214)(871.70116211,375.24397308)
\curveto(871.78115739,375.26397212)(871.86115731,375.2789721)(871.94116211,375.28897308)
\curveto(872.02115715,375.30897207)(872.09615708,375.33397205)(872.16616211,375.36397308)
\curveto(872.49615668,375.50397188)(872.76115641,375.6789717)(872.96116211,375.88897308)
\curveto(873.171156,376.10897127)(873.34615583,376.383971)(873.48616211,376.71397308)
\curveto(873.53615564,376.82397056)(873.5711556,376.93397045)(873.59116211,377.04397308)
\curveto(873.61115556,377.15397023)(873.63615554,377.26397012)(873.66616211,377.37397308)
\curveto(873.68615549,377.41396997)(873.69615548,377.44896993)(873.69616211,377.47897308)
\curveto(873.69615548,377.51896986)(873.70115547,377.55896982)(873.71116211,377.59897308)
\curveto(873.72115545,377.65896972)(873.72115545,377.71896966)(873.71116211,377.77897308)
\curveto(873.71115546,377.83896954)(873.71615546,377.89896948)(873.72616211,377.95897308)
}
}
{
\newrgbcolor{curcolor}{0 0 0}
\pscustom[linestyle=none,fillstyle=solid,fillcolor=curcolor]
{
\newpath
\moveto(883.77241211,374.86897308)
\curveto(883.80240428,374.70897267)(883.78740429,374.57397281)(883.72741211,374.46397308)
\curveto(883.66740441,374.36397302)(883.58740449,374.28897309)(883.48741211,374.23897308)
\curveto(883.43740464,374.21897316)(883.3824047,374.20897317)(883.32241211,374.20897308)
\curveto(883.27240481,374.20897317)(883.21740486,374.19897318)(883.15741211,374.17897308)
\curveto(882.93740514,374.12897325)(882.71740536,374.14397324)(882.49741211,374.22397308)
\curveto(882.28740579,374.29397309)(882.14240594,374.383973)(882.06241211,374.49397308)
\curveto(882.01240607,374.56397282)(881.96740611,374.64397274)(881.92741211,374.73397308)
\curveto(881.88740619,374.83397255)(881.83740624,374.91397247)(881.77741211,374.97397308)
\curveto(881.75740632,374.99397239)(881.73240635,375.01397237)(881.70241211,375.03397308)
\curveto(881.6824064,375.05397233)(881.65240643,375.05897232)(881.61241211,375.04897308)
\curveto(881.50240658,375.01897236)(881.39740668,374.96397242)(881.29741211,374.88397308)
\curveto(881.20740687,374.80397258)(881.11740696,374.73397265)(881.02741211,374.67397308)
\curveto(880.89740718,374.59397279)(880.75740732,374.51897286)(880.60741211,374.44897308)
\curveto(880.45740762,374.38897299)(880.29740778,374.33397305)(880.12741211,374.28397308)
\curveto(880.02740805,374.25397313)(879.91740816,374.23397315)(879.79741211,374.22397308)
\curveto(879.68740839,374.21397317)(879.5774085,374.19897318)(879.46741211,374.17897308)
\curveto(879.41740866,374.16897321)(879.37240871,374.16397322)(879.33241211,374.16397308)
\lineto(879.22741211,374.16397308)
\curveto(879.11740896,374.14397324)(879.01240907,374.14397324)(878.91241211,374.16397308)
\lineto(878.77741211,374.16397308)
\curveto(878.72740935,374.17397321)(878.6774094,374.1789732)(878.62741211,374.17897308)
\curveto(878.5774095,374.1789732)(878.53240955,374.18897319)(878.49241211,374.20897308)
\curveto(878.45240963,374.21897316)(878.41740966,374.22397316)(878.38741211,374.22397308)
\curveto(878.36740971,374.21397317)(878.34240974,374.21397317)(878.31241211,374.22397308)
\lineto(878.07241211,374.28397308)
\curveto(877.99241009,374.29397309)(877.91741016,374.31397307)(877.84741211,374.34397308)
\curveto(877.54741053,374.47397291)(877.30241078,374.61897276)(877.11241211,374.77897308)
\curveto(876.93241115,374.94897243)(876.7824113,375.1839722)(876.66241211,375.48397308)
\curveto(876.57241151,375.70397168)(876.52741155,375.96897141)(876.52741211,376.27897308)
\lineto(876.52741211,376.59397308)
\curveto(876.53741154,376.64397074)(876.54241154,376.69397069)(876.54241211,376.74397308)
\lineto(876.57241211,376.92397308)
\lineto(876.69241211,377.25397308)
\curveto(876.73241135,377.36397002)(876.7824113,377.46396992)(876.84241211,377.55397308)
\curveto(877.02241106,377.84396954)(877.26741081,378.05896932)(877.57741211,378.19897308)
\curveto(877.88741019,378.33896904)(878.22740985,378.46396892)(878.59741211,378.57397308)
\curveto(878.73740934,378.61396877)(878.8824092,378.64396874)(879.03241211,378.66397308)
\curveto(879.1824089,378.6839687)(879.33240875,378.70896867)(879.48241211,378.73897308)
\curveto(879.55240853,378.75896862)(879.61740846,378.76896861)(879.67741211,378.76897308)
\curveto(879.74740833,378.76896861)(879.82240826,378.7789686)(879.90241211,378.79897308)
\curveto(879.97240811,378.81896856)(880.04240804,378.82896855)(880.11241211,378.82897308)
\curveto(880.1824079,378.83896854)(880.25740782,378.85396853)(880.33741211,378.87397308)
\curveto(880.58740749,378.93396845)(880.82240726,378.9839684)(881.04241211,379.02397308)
\curveto(881.26240682,379.07396831)(881.43740664,379.18896819)(881.56741211,379.36897308)
\curveto(881.62740645,379.44896793)(881.6774064,379.54896783)(881.71741211,379.66897308)
\curveto(881.75740632,379.79896758)(881.75740632,379.93896744)(881.71741211,380.08897308)
\curveto(881.65740642,380.32896705)(881.56740651,380.51896686)(881.44741211,380.65897308)
\curveto(881.33740674,380.79896658)(881.1774069,380.90896647)(880.96741211,380.98897308)
\curveto(880.84740723,381.03896634)(880.70240738,381.07396631)(880.53241211,381.09397308)
\curveto(880.37240771,381.11396627)(880.20240788,381.12396626)(880.02241211,381.12397308)
\curveto(879.84240824,381.12396626)(879.66740841,381.11396627)(879.49741211,381.09397308)
\curveto(879.32740875,381.07396631)(879.1824089,381.04396634)(879.06241211,381.00397308)
\curveto(878.89240919,380.94396644)(878.72740935,380.85896652)(878.56741211,380.74897308)
\curveto(878.48740959,380.68896669)(878.41240967,380.60896677)(878.34241211,380.50897308)
\curveto(878.2824098,380.41896696)(878.22740985,380.31896706)(878.17741211,380.20897308)
\curveto(878.14740993,380.12896725)(878.11740996,380.04396734)(878.08741211,379.95397308)
\curveto(878.06741001,379.86396752)(878.02241006,379.79396759)(877.95241211,379.74397308)
\curveto(877.91241017,379.71396767)(877.84241024,379.68896769)(877.74241211,379.66897308)
\curveto(877.65241043,379.65896772)(877.55741052,379.65396773)(877.45741211,379.65397308)
\curveto(877.35741072,379.65396773)(877.25741082,379.65896772)(877.15741211,379.66897308)
\curveto(877.06741101,379.68896769)(877.00241108,379.71396767)(876.96241211,379.74397308)
\curveto(876.92241116,379.77396761)(876.89241119,379.82396756)(876.87241211,379.89397308)
\curveto(876.85241123,379.96396742)(876.85241123,380.03896734)(876.87241211,380.11897308)
\curveto(876.90241118,380.24896713)(876.93241115,380.36896701)(876.96241211,380.47897308)
\curveto(877.00241108,380.59896678)(877.04741103,380.71396667)(877.09741211,380.82397308)
\curveto(877.28741079,381.17396621)(877.52741055,381.44396594)(877.81741211,381.63397308)
\curveto(878.10740997,381.83396555)(878.46740961,381.99396539)(878.89741211,382.11397308)
\curveto(878.99740908,382.13396525)(879.09740898,382.14896523)(879.19741211,382.15897308)
\curveto(879.30740877,382.16896521)(879.41740866,382.1839652)(879.52741211,382.20397308)
\curveto(879.56740851,382.21396517)(879.63240845,382.21396517)(879.72241211,382.20397308)
\curveto(879.81240827,382.20396518)(879.86740821,382.21396517)(879.88741211,382.23397308)
\curveto(880.58740749,382.24396514)(881.19740688,382.16396522)(881.71741211,381.99397308)
\curveto(882.23740584,381.82396556)(882.60240548,381.49896588)(882.81241211,381.01897308)
\curveto(882.90240518,380.81896656)(882.95240513,380.5839668)(882.96241211,380.31397308)
\curveto(882.9824051,380.05396733)(882.99240509,379.7789676)(882.99241211,379.48897308)
\lineto(882.99241211,376.17397308)
\curveto(882.99240509,376.03397135)(882.99740508,375.89897148)(883.00741211,375.76897308)
\curveto(883.01740506,375.63897174)(883.04740503,375.53397185)(883.09741211,375.45397308)
\curveto(883.14740493,375.383972)(883.21240487,375.33397205)(883.29241211,375.30397308)
\curveto(883.3824047,375.26397212)(883.46740461,375.23397215)(883.54741211,375.21397308)
\curveto(883.62740445,375.20397218)(883.68740439,375.15897222)(883.72741211,375.07897308)
\curveto(883.74740433,375.04897233)(883.75740432,375.01897236)(883.75741211,374.98897308)
\curveto(883.75740432,374.95897242)(883.76240432,374.91897246)(883.77241211,374.86897308)
\moveto(881.62741211,376.53397308)
\curveto(881.68740639,376.67397071)(881.71740636,376.83397055)(881.71741211,377.01397308)
\curveto(881.72740635,377.20397018)(881.73240635,377.39896998)(881.73241211,377.59897308)
\curveto(881.73240635,377.70896967)(881.72740635,377.80896957)(881.71741211,377.89897308)
\curveto(881.70740637,377.98896939)(881.66740641,378.05896932)(881.59741211,378.10897308)
\curveto(881.56740651,378.12896925)(881.49740658,378.13896924)(881.38741211,378.13897308)
\curveto(881.36740671,378.11896926)(881.33240675,378.10896927)(881.28241211,378.10897308)
\curveto(881.23240685,378.10896927)(881.18740689,378.09896928)(881.14741211,378.07897308)
\curveto(881.06740701,378.05896932)(880.9774071,378.03896934)(880.87741211,378.01897308)
\lineto(880.57741211,377.95897308)
\curveto(880.54740753,377.95896942)(880.51240757,377.95396943)(880.47241211,377.94397308)
\lineto(880.36741211,377.94397308)
\curveto(880.21740786,377.90396948)(880.05240803,377.8789695)(879.87241211,377.86897308)
\curveto(879.70240838,377.86896951)(879.54240854,377.84896953)(879.39241211,377.80897308)
\curveto(879.31240877,377.78896959)(879.23740884,377.76896961)(879.16741211,377.74897308)
\curveto(879.10740897,377.73896964)(879.03740904,377.72396966)(878.95741211,377.70397308)
\curveto(878.79740928,377.65396973)(878.64740943,377.58896979)(878.50741211,377.50897308)
\curveto(878.36740971,377.43896994)(878.24740983,377.34897003)(878.14741211,377.23897308)
\curveto(878.04741003,377.12897025)(877.97241011,376.99397039)(877.92241211,376.83397308)
\curveto(877.87241021,376.6839707)(877.85241023,376.49897088)(877.86241211,376.27897308)
\curveto(877.86241022,376.1789712)(877.8774102,376.0839713)(877.90741211,375.99397308)
\curveto(877.94741013,375.91397147)(877.99241009,375.83897154)(878.04241211,375.76897308)
\curveto(878.12240996,375.65897172)(878.22740985,375.56397182)(878.35741211,375.48397308)
\curveto(878.48740959,375.41397197)(878.62740945,375.35397203)(878.77741211,375.30397308)
\curveto(878.82740925,375.29397209)(878.8774092,375.28897209)(878.92741211,375.28897308)
\curveto(878.9774091,375.28897209)(879.02740905,375.2839721)(879.07741211,375.27397308)
\curveto(879.14740893,375.25397213)(879.23240885,375.23897214)(879.33241211,375.22897308)
\curveto(879.44240864,375.22897215)(879.53240855,375.23897214)(879.60241211,375.25897308)
\curveto(879.66240842,375.2789721)(879.72240836,375.2839721)(879.78241211,375.27397308)
\curveto(879.84240824,375.27397211)(879.90240818,375.2839721)(879.96241211,375.30397308)
\curveto(880.04240804,375.32397206)(880.11740796,375.33897204)(880.18741211,375.34897308)
\curveto(880.26740781,375.35897202)(880.34240774,375.378972)(880.41241211,375.40897308)
\curveto(880.70240738,375.52897185)(880.94740713,375.67397171)(881.14741211,375.84397308)
\curveto(881.35740672,376.01397137)(881.51740656,376.24397114)(881.62741211,376.53397308)
}
}
{
\newrgbcolor{curcolor}{0 0 0}
\pscustom[linestyle=none,fillstyle=solid,fillcolor=curcolor]
{
\newpath
\moveto(891.90405273,375.12397308)
\lineto(891.90405273,374.73397308)
\curveto(891.90404486,374.61397277)(891.87904488,374.51397287)(891.82905273,374.43397308)
\curveto(891.77904498,374.36397302)(891.69404507,374.32397306)(891.57405273,374.31397308)
\lineto(891.22905273,374.31397308)
\curveto(891.16904559,374.31397307)(891.10904565,374.30897307)(891.04905273,374.29897308)
\curveto(890.99904576,374.29897308)(890.95404581,374.30897307)(890.91405273,374.32897308)
\curveto(890.82404594,374.34897303)(890.764046,374.38897299)(890.73405273,374.44897308)
\curveto(890.69404607,374.49897288)(890.66904609,374.55897282)(890.65905273,374.62897308)
\curveto(890.6590461,374.69897268)(890.64404612,374.76897261)(890.61405273,374.83897308)
\curveto(890.60404616,374.85897252)(890.58904617,374.87397251)(890.56905273,374.88397308)
\curveto(890.5590462,374.90397248)(890.54404622,374.92397246)(890.52405273,374.94397308)
\curveto(890.42404634,374.95397243)(890.34404642,374.93397245)(890.28405273,374.88397308)
\curveto(890.23404653,374.83397255)(890.17904658,374.7839726)(890.11905273,374.73397308)
\curveto(889.91904684,374.5839728)(889.71904704,374.46897291)(889.51905273,374.38897308)
\curveto(889.33904742,374.30897307)(889.12904763,374.24897313)(888.88905273,374.20897308)
\curveto(888.6590481,374.16897321)(888.41904834,374.14897323)(888.16905273,374.14897308)
\curveto(887.92904883,374.13897324)(887.68904907,374.15397323)(887.44905273,374.19397308)
\curveto(887.20904955,374.22397316)(886.99904976,374.2789731)(886.81905273,374.35897308)
\curveto(886.29905046,374.5789728)(885.87905088,374.87397251)(885.55905273,375.24397308)
\curveto(885.23905152,375.62397176)(884.98905177,376.09397129)(884.80905273,376.65397308)
\curveto(884.76905199,376.74397064)(884.73905202,376.83397055)(884.71905273,376.92397308)
\curveto(884.70905205,377.02397036)(884.68905207,377.12397026)(884.65905273,377.22397308)
\curveto(884.64905211,377.27397011)(884.64405212,377.32397006)(884.64405273,377.37397308)
\curveto(884.64405212,377.42396996)(884.63905212,377.47396991)(884.62905273,377.52397308)
\curveto(884.60905215,377.57396981)(884.59905216,377.62396976)(884.59905273,377.67397308)
\curveto(884.60905215,377.73396965)(884.60905215,377.78896959)(884.59905273,377.83897308)
\lineto(884.59905273,377.98897308)
\curveto(884.57905218,378.03896934)(884.56905219,378.10396928)(884.56905273,378.18397308)
\curveto(884.56905219,378.26396912)(884.57905218,378.32896905)(884.59905273,378.37897308)
\lineto(884.59905273,378.54397308)
\curveto(884.61905214,378.61396877)(884.62405214,378.6839687)(884.61405273,378.75397308)
\curveto(884.61405215,378.83396855)(884.62405214,378.90896847)(884.64405273,378.97897308)
\curveto(884.65405211,379.02896835)(884.6590521,379.07396831)(884.65905273,379.11397308)
\curveto(884.6590521,379.15396823)(884.6640521,379.19896818)(884.67405273,379.24897308)
\curveto(884.70405206,379.34896803)(884.72905203,379.44396794)(884.74905273,379.53397308)
\curveto(884.76905199,379.63396775)(884.79405197,379.72896765)(884.82405273,379.81897308)
\curveto(884.95405181,380.19896718)(885.11905164,380.53896684)(885.31905273,380.83897308)
\curveto(885.52905123,381.14896623)(885.77905098,381.40396598)(886.06905273,381.60397308)
\curveto(886.23905052,381.72396566)(886.41405035,381.82396556)(886.59405273,381.90397308)
\curveto(886.78404998,381.9839654)(886.98904977,382.05396533)(887.20905273,382.11397308)
\curveto(887.27904948,382.12396526)(887.34404942,382.13396525)(887.40405273,382.14397308)
\curveto(887.47404929,382.15396523)(887.54404922,382.16896521)(887.61405273,382.18897308)
\lineto(887.76405273,382.18897308)
\curveto(887.84404892,382.20896517)(887.9590488,382.21896516)(888.10905273,382.21897308)
\curveto(888.26904849,382.21896516)(888.38904837,382.20896517)(888.46905273,382.18897308)
\curveto(888.50904825,382.1789652)(888.5640482,382.17396521)(888.63405273,382.17397308)
\curveto(888.74404802,382.14396524)(888.85404791,382.11896526)(888.96405273,382.09897308)
\curveto(889.07404769,382.08896529)(889.17904758,382.05896532)(889.27905273,382.00897308)
\curveto(889.42904733,381.94896543)(889.56904719,381.8839655)(889.69905273,381.81397308)
\curveto(889.83904692,381.74396564)(889.96904679,381.66396572)(890.08905273,381.57397308)
\curveto(890.14904661,381.52396586)(890.20904655,381.46896591)(890.26905273,381.40897308)
\curveto(890.33904642,381.35896602)(890.42904633,381.34396604)(890.53905273,381.36397308)
\curveto(890.5590462,381.39396599)(890.57404619,381.41896596)(890.58405273,381.43897308)
\curveto(890.60404616,381.45896592)(890.61904614,381.48896589)(890.62905273,381.52897308)
\curveto(890.6590461,381.61896576)(890.66904609,381.73396565)(890.65905273,381.87397308)
\lineto(890.65905273,382.24897308)
\lineto(890.65905273,383.97397308)
\lineto(890.65905273,384.43897308)
\curveto(890.6590461,384.61896276)(890.68404608,384.74896263)(890.73405273,384.82897308)
\curveto(890.77404599,384.89896248)(890.83404593,384.94396244)(890.91405273,384.96397308)
\curveto(890.93404583,384.96396242)(890.9590458,384.96396242)(890.98905273,384.96397308)
\curveto(891.01904574,384.97396241)(891.04404572,384.9789624)(891.06405273,384.97897308)
\curveto(891.20404556,384.98896239)(891.34904541,384.98896239)(891.49905273,384.97897308)
\curveto(891.6590451,384.9789624)(891.76904499,384.93896244)(891.82905273,384.85897308)
\curveto(891.87904488,384.7789626)(891.90404486,384.6789627)(891.90405273,384.55897308)
\lineto(891.90405273,384.18397308)
\lineto(891.90405273,375.12397308)
\moveto(890.68905273,377.95897308)
\curveto(890.70904605,378.00896937)(890.71904604,378.07396931)(890.71905273,378.15397308)
\curveto(890.71904604,378.24396914)(890.70904605,378.31396907)(890.68905273,378.36397308)
\lineto(890.68905273,378.58897308)
\curveto(890.66904609,378.6789687)(890.65404611,378.76896861)(890.64405273,378.85897308)
\curveto(890.63404613,378.95896842)(890.61404615,379.04896833)(890.58405273,379.12897308)
\curveto(890.5640462,379.20896817)(890.54404622,379.2839681)(890.52405273,379.35397308)
\curveto(890.51404625,379.42396796)(890.49404627,379.49396789)(890.46405273,379.56397308)
\curveto(890.34404642,379.86396752)(890.18904657,380.12896725)(889.99905273,380.35897308)
\curveto(889.80904695,380.58896679)(889.56904719,380.76896661)(889.27905273,380.89897308)
\curveto(889.17904758,380.94896643)(889.07404769,380.9839664)(888.96405273,381.00397308)
\curveto(888.8640479,381.03396635)(888.75404801,381.05896632)(888.63405273,381.07897308)
\curveto(888.55404821,381.09896628)(888.4640483,381.10896627)(888.36405273,381.10897308)
\lineto(888.09405273,381.10897308)
\curveto(888.04404872,381.09896628)(887.99904876,381.08896629)(887.95905273,381.07897308)
\lineto(887.82405273,381.07897308)
\curveto(887.74404902,381.05896632)(887.6590491,381.03896634)(887.56905273,381.01897308)
\curveto(887.48904927,380.99896638)(887.40904935,380.97396641)(887.32905273,380.94397308)
\curveto(887.00904975,380.80396658)(886.74905001,380.59896678)(886.54905273,380.32897308)
\curveto(886.3590504,380.06896731)(886.20405056,379.76396762)(886.08405273,379.41397308)
\curveto(886.04405072,379.30396808)(886.01405075,379.18896819)(885.99405273,379.06897308)
\curveto(885.98405078,378.95896842)(885.96905079,378.84896853)(885.94905273,378.73897308)
\curveto(885.94905081,378.69896868)(885.94405082,378.65896872)(885.93405273,378.61897308)
\lineto(885.93405273,378.51397308)
\curveto(885.91405085,378.46396892)(885.90405086,378.40896897)(885.90405273,378.34897308)
\curveto(885.91405085,378.28896909)(885.91905084,378.23396915)(885.91905273,378.18397308)
\lineto(885.91905273,377.85397308)
\curveto(885.91905084,377.75396963)(885.92905083,377.65896972)(885.94905273,377.56897308)
\curveto(885.9590508,377.53896984)(885.9640508,377.48896989)(885.96405273,377.41897308)
\curveto(885.98405078,377.34897003)(885.99905076,377.2789701)(886.00905273,377.20897308)
\lineto(886.06905273,376.99897308)
\curveto(886.17905058,376.64897073)(886.32905043,376.34897103)(886.51905273,376.09897308)
\curveto(886.70905005,375.84897153)(886.94904981,375.64397174)(887.23905273,375.48397308)
\curveto(887.32904943,375.43397195)(887.41904934,375.39397199)(887.50905273,375.36397308)
\curveto(887.59904916,375.33397205)(887.69904906,375.30397208)(887.80905273,375.27397308)
\curveto(887.8590489,375.25397213)(887.90904885,375.24897213)(887.95905273,375.25897308)
\curveto(888.01904874,375.26897211)(888.07404869,375.26397212)(888.12405273,375.24397308)
\curveto(888.1640486,375.23397215)(888.20404856,375.22897215)(888.24405273,375.22897308)
\lineto(888.37905273,375.22897308)
\lineto(888.51405273,375.22897308)
\curveto(888.54404822,375.23897214)(888.59404817,375.24397214)(888.66405273,375.24397308)
\curveto(888.74404802,375.26397212)(888.82404794,375.2789721)(888.90405273,375.28897308)
\curveto(888.98404778,375.30897207)(889.0590477,375.33397205)(889.12905273,375.36397308)
\curveto(889.4590473,375.50397188)(889.72404704,375.6789717)(889.92405273,375.88897308)
\curveto(890.13404663,376.10897127)(890.30904645,376.383971)(890.44905273,376.71397308)
\curveto(890.49904626,376.82397056)(890.53404623,376.93397045)(890.55405273,377.04397308)
\curveto(890.57404619,377.15397023)(890.59904616,377.26397012)(890.62905273,377.37397308)
\curveto(890.64904611,377.41396997)(890.6590461,377.44896993)(890.65905273,377.47897308)
\curveto(890.6590461,377.51896986)(890.6640461,377.55896982)(890.67405273,377.59897308)
\curveto(890.68404608,377.65896972)(890.68404608,377.71896966)(890.67405273,377.77897308)
\curveto(890.67404609,377.83896954)(890.67904608,377.89896948)(890.68905273,377.95897308)
}
}
{
\newrgbcolor{curcolor}{0 0 0}
\pscustom[linestyle=none,fillstyle=solid,fillcolor=curcolor]
{
\newpath
\moveto(839.46144531,358.84265076)
\curveto(839.48143619,358.76264298)(839.49143618,358.65264309)(839.49144531,358.51265076)
\curveto(839.49143618,358.38264336)(839.48143619,358.28264346)(839.46144531,358.21265076)
\curveto(839.44143623,358.1426436)(839.43643623,358.07764366)(839.44644531,358.01765076)
\curveto(839.45643621,357.95764378)(839.45143622,357.89264385)(839.43144531,357.82265076)
\curveto(839.41143626,357.76264398)(839.39643627,357.69764404)(839.38644531,357.62765076)
\curveto(839.37643629,357.56764417)(839.36143631,357.50764423)(839.34144531,357.44765076)
\curveto(839.32143635,357.36764437)(839.29643637,357.29264445)(839.26644531,357.22265076)
\curveto(839.24643642,357.15264459)(839.22143645,357.08264466)(839.19144531,357.01265076)
\curveto(839.1714365,356.98264476)(839.15643651,356.95264479)(839.14644531,356.92265076)
\curveto(839.14643652,356.90264484)(839.13643653,356.88264486)(839.11644531,356.86265076)
\curveto(839.00643666,356.66264508)(838.88643678,356.48264526)(838.75644531,356.32265076)
\curveto(838.73643693,356.28264546)(838.70143697,356.2426455)(838.65144531,356.20265076)
\curveto(838.61143706,356.16264558)(838.57643709,356.13264561)(838.54644531,356.11265076)
\curveto(838.50643716,356.09264565)(838.4714372,356.06264568)(838.44144531,356.02265076)
\curveto(838.41143726,355.99264575)(838.38143729,355.96764577)(838.35144531,355.94765076)
\lineto(838.03644531,355.76765076)
\curveto(837.92643774,355.68764605)(837.79643787,355.62764611)(837.64644531,355.58765076)
\lineto(837.19644531,355.46765076)
\curveto(837.11643855,355.44764629)(837.03643863,355.43264631)(836.95644531,355.42265076)
\curveto(836.87643879,355.42264632)(836.79643887,355.41264633)(836.71644531,355.39265076)
\curveto(836.67643899,355.38264636)(836.63643903,355.37764636)(836.59644531,355.37765076)
\curveto(836.5664391,355.38764635)(836.53643913,355.38764635)(836.50644531,355.37765076)
\curveto(836.45643921,355.36764637)(836.40643926,355.36764637)(836.35644531,355.37765076)
\curveto(836.31643935,355.38764635)(836.2714394,355.38764635)(836.22144531,355.37765076)
\lineto(833.95644531,355.37765076)
\lineto(833.46144531,355.37765076)
\curveto(833.29144238,355.38764635)(833.16144251,355.35764638)(833.07144531,355.28765076)
\curveto(832.96144271,355.20764653)(832.90644276,355.06264668)(832.90644531,354.85265076)
\curveto(832.91644275,354.6426471)(832.92144275,354.44764729)(832.92144531,354.26765076)
\lineto(832.92144531,352.06265076)
\lineto(832.92144531,351.56765076)
\curveto(832.93144274,351.37765036)(832.91144276,351.2426505)(832.86144531,351.16265076)
\curveto(832.82144285,351.10265064)(832.7714429,351.06265068)(832.71144531,351.04265076)
\curveto(832.66144301,351.03265071)(832.59644307,351.01765072)(832.51644531,350.99765076)
\lineto(832.24644531,350.99765076)
\curveto(832.09644357,350.99765074)(831.96144371,351.00265074)(831.84144531,351.01265076)
\curveto(831.72144395,351.02265072)(831.63644403,351.07265067)(831.58644531,351.16265076)
\curveto(831.54644412,351.22265052)(831.52644414,351.30265044)(831.52644531,351.40265076)
\lineto(831.52644531,351.71765076)
\lineto(831.52644531,360.82265076)
\curveto(831.52644414,360.93264081)(831.52144415,361.05264069)(831.51144531,361.18265076)
\curveto(831.51144416,361.32264042)(831.53644413,361.43264031)(831.58644531,361.51265076)
\curveto(831.62644404,361.57264017)(831.70144397,361.62264012)(831.81144531,361.66265076)
\curveto(831.83144384,361.67264007)(831.85144382,361.67264007)(831.87144531,361.66265076)
\curveto(831.89144378,361.66264008)(831.91144376,361.66764007)(831.93144531,361.67765076)
\lineto(835.33644531,361.67765076)
\curveto(835.71643995,361.67764006)(836.08643958,361.67264007)(836.44644531,361.66265076)
\curveto(836.81643885,361.66264008)(837.14643852,361.61764012)(837.43644531,361.52765076)
\curveto(837.88643778,361.37764036)(838.25143742,361.18264056)(838.53144531,360.94265076)
\curveto(838.81143686,360.70264104)(839.04143663,360.37264137)(839.22144531,359.95265076)
\curveto(839.2714364,359.8426419)(839.30643636,359.72764201)(839.32644531,359.60765076)
\curveto(839.35643631,359.48764225)(839.39143628,359.36264238)(839.43144531,359.23265076)
\curveto(839.45143622,359.16264258)(839.45643621,359.09764264)(839.44644531,359.03765076)
\curveto(839.43643623,358.97764276)(839.44143623,358.91264283)(839.46144531,358.84265076)
\moveto(838.05144531,358.30265076)
\curveto(838.09143758,358.4426433)(838.09643757,358.60264314)(838.06644531,358.78265076)
\curveto(838.03643763,358.97264277)(838.00643766,359.12264262)(837.97644531,359.23265076)
\curveto(837.87643779,359.51264223)(837.74143793,359.73264201)(837.57144531,359.89265076)
\curveto(837.41143826,360.06264168)(837.20143847,360.20264154)(836.94144531,360.31265076)
\curveto(836.72143895,360.40264134)(836.4664392,360.45764128)(836.17644531,360.47765076)
\curveto(835.89643977,360.49764124)(835.60144007,360.50764123)(835.29144531,360.50765076)
\lineto(833.35644531,360.50765076)
\curveto(833.33644233,360.49764124)(833.31144236,360.49264125)(833.28144531,360.49265076)
\curveto(833.26144241,360.49264125)(833.23644243,360.48764125)(833.20644531,360.47765076)
\curveto(833.08644258,360.44764129)(833.00644266,360.38264136)(832.96644531,360.28265076)
\curveto(832.92644274,360.18264156)(832.90644276,360.04764169)(832.90644531,359.87765076)
\curveto(832.91644275,359.71764202)(832.92144275,359.56764217)(832.92144531,359.42765076)
\lineto(832.92144531,357.62765076)
\curveto(832.92144275,357.47764426)(832.91644275,357.31264443)(832.90644531,357.13265076)
\curveto(832.90644276,356.95264479)(832.93644273,356.81264493)(832.99644531,356.71265076)
\curveto(833.04644262,356.63264511)(833.12144255,356.58264516)(833.22144531,356.56265076)
\curveto(833.33144234,356.55264519)(833.45144222,356.54764519)(833.58144531,356.54765076)
\lineto(835.60644531,356.54765076)
\lineto(836.07144531,356.54765076)
\curveto(836.23143944,356.55764518)(836.3714393,356.57764516)(836.49144531,356.60765076)
\curveto(836.76143891,356.67764506)(836.99643867,356.75764498)(837.19644531,356.84765076)
\curveto(837.40643826,356.94764479)(837.58143809,357.09764464)(837.72144531,357.29765076)
\curveto(837.80143787,357.41764432)(837.86143781,357.5426442)(837.90144531,357.67265076)
\curveto(837.95143772,357.80264394)(837.99643767,357.94764379)(838.03644531,358.10765076)
\curveto(838.04643762,358.14764359)(838.05143762,358.21264353)(838.05144531,358.30265076)
}
}
{
\newrgbcolor{curcolor}{0 0 0}
\pscustom[linestyle=none,fillstyle=solid,fillcolor=curcolor]
{
\newpath
\moveto(847.88300781,351.55265076)
\curveto(847.91299998,351.39265035)(847.898,351.25765048)(847.83800781,351.14765076)
\curveto(847.77800012,351.04765069)(847.6980002,350.97265077)(847.59800781,350.92265076)
\curveto(847.54800035,350.90265084)(847.4930004,350.89265085)(847.43300781,350.89265076)
\curveto(847.38300051,350.89265085)(847.32800057,350.88265086)(847.26800781,350.86265076)
\curveto(847.04800085,350.81265093)(846.82800107,350.82765091)(846.60800781,350.90765076)
\curveto(846.3980015,350.97765076)(846.25300164,351.06765067)(846.17300781,351.17765076)
\curveto(846.12300177,351.24765049)(846.07800182,351.32765041)(846.03800781,351.41765076)
\curveto(845.9980019,351.51765022)(845.94800195,351.59765014)(845.88800781,351.65765076)
\curveto(845.86800203,351.67765006)(845.84300205,351.69765004)(845.81300781,351.71765076)
\curveto(845.7930021,351.73765)(845.76300213,351.74265)(845.72300781,351.73265076)
\curveto(845.61300228,351.70265004)(845.50800239,351.64765009)(845.40800781,351.56765076)
\curveto(845.31800258,351.48765025)(845.22800267,351.41765032)(845.13800781,351.35765076)
\curveto(845.00800289,351.27765046)(844.86800303,351.20265054)(844.71800781,351.13265076)
\curveto(844.56800333,351.07265067)(844.40800349,351.01765072)(844.23800781,350.96765076)
\curveto(844.13800376,350.9376508)(844.02800387,350.91765082)(843.90800781,350.90765076)
\curveto(843.7980041,350.89765084)(843.68800421,350.88265086)(843.57800781,350.86265076)
\curveto(843.52800437,350.85265089)(843.48300441,350.84765089)(843.44300781,350.84765076)
\lineto(843.33800781,350.84765076)
\curveto(843.22800467,350.82765091)(843.12300477,350.82765091)(843.02300781,350.84765076)
\lineto(842.88800781,350.84765076)
\curveto(842.83800506,350.85765088)(842.78800511,350.86265088)(842.73800781,350.86265076)
\curveto(842.68800521,350.86265088)(842.64300525,350.87265087)(842.60300781,350.89265076)
\curveto(842.56300533,350.90265084)(842.52800537,350.90765083)(842.49800781,350.90765076)
\curveto(842.47800542,350.89765084)(842.45300544,350.89765084)(842.42300781,350.90765076)
\lineto(842.18300781,350.96765076)
\curveto(842.10300579,350.97765076)(842.02800587,350.99765074)(841.95800781,351.02765076)
\curveto(841.65800624,351.15765058)(841.41300648,351.30265044)(841.22300781,351.46265076)
\curveto(841.04300685,351.63265011)(840.893007,351.86764987)(840.77300781,352.16765076)
\curveto(840.68300721,352.38764935)(840.63800726,352.65264909)(840.63800781,352.96265076)
\lineto(840.63800781,353.27765076)
\curveto(840.64800725,353.32764841)(840.65300724,353.37764836)(840.65300781,353.42765076)
\lineto(840.68300781,353.60765076)
\lineto(840.80300781,353.93765076)
\curveto(840.84300705,354.04764769)(840.893007,354.14764759)(840.95300781,354.23765076)
\curveto(841.13300676,354.52764721)(841.37800652,354.742647)(841.68800781,354.88265076)
\curveto(841.9980059,355.02264672)(842.33800556,355.14764659)(842.70800781,355.25765076)
\curveto(842.84800505,355.29764644)(842.9930049,355.32764641)(843.14300781,355.34765076)
\curveto(843.2930046,355.36764637)(843.44300445,355.39264635)(843.59300781,355.42265076)
\curveto(843.66300423,355.4426463)(843.72800417,355.45264629)(843.78800781,355.45265076)
\curveto(843.85800404,355.45264629)(843.93300396,355.46264628)(844.01300781,355.48265076)
\curveto(844.08300381,355.50264624)(844.15300374,355.51264623)(844.22300781,355.51265076)
\curveto(844.2930036,355.52264622)(844.36800353,355.5376462)(844.44800781,355.55765076)
\curveto(844.6980032,355.61764612)(844.93300296,355.66764607)(845.15300781,355.70765076)
\curveto(845.37300252,355.75764598)(845.54800235,355.87264587)(845.67800781,356.05265076)
\curveto(845.73800216,356.13264561)(845.78800211,356.23264551)(845.82800781,356.35265076)
\curveto(845.86800203,356.48264526)(845.86800203,356.62264512)(845.82800781,356.77265076)
\curveto(845.76800213,357.01264473)(845.67800222,357.20264454)(845.55800781,357.34265076)
\curveto(845.44800245,357.48264426)(845.28800261,357.59264415)(845.07800781,357.67265076)
\curveto(844.95800294,357.72264402)(844.81300308,357.75764398)(844.64300781,357.77765076)
\curveto(844.48300341,357.79764394)(844.31300358,357.80764393)(844.13300781,357.80765076)
\curveto(843.95300394,357.80764393)(843.77800412,357.79764394)(843.60800781,357.77765076)
\curveto(843.43800446,357.75764398)(843.2930046,357.72764401)(843.17300781,357.68765076)
\curveto(843.00300489,357.62764411)(842.83800506,357.5426442)(842.67800781,357.43265076)
\curveto(842.5980053,357.37264437)(842.52300537,357.29264445)(842.45300781,357.19265076)
\curveto(842.3930055,357.10264464)(842.33800556,357.00264474)(842.28800781,356.89265076)
\curveto(842.25800564,356.81264493)(842.22800567,356.72764501)(842.19800781,356.63765076)
\curveto(842.17800572,356.54764519)(842.13300576,356.47764526)(842.06300781,356.42765076)
\curveto(842.02300587,356.39764534)(841.95300594,356.37264537)(841.85300781,356.35265076)
\curveto(841.76300613,356.3426454)(841.66800623,356.3376454)(841.56800781,356.33765076)
\curveto(841.46800643,356.3376454)(841.36800653,356.3426454)(841.26800781,356.35265076)
\curveto(841.17800672,356.37264537)(841.11300678,356.39764534)(841.07300781,356.42765076)
\curveto(841.03300686,356.45764528)(841.00300689,356.50764523)(840.98300781,356.57765076)
\curveto(840.96300693,356.64764509)(840.96300693,356.72264502)(840.98300781,356.80265076)
\curveto(841.01300688,356.93264481)(841.04300685,357.05264469)(841.07300781,357.16265076)
\curveto(841.11300678,357.28264446)(841.15800674,357.39764434)(841.20800781,357.50765076)
\curveto(841.3980065,357.85764388)(841.63800626,358.12764361)(841.92800781,358.31765076)
\curveto(842.21800568,358.51764322)(842.57800532,358.67764306)(843.00800781,358.79765076)
\curveto(843.10800479,358.81764292)(843.20800469,358.83264291)(843.30800781,358.84265076)
\curveto(843.41800448,358.85264289)(843.52800437,358.86764287)(843.63800781,358.88765076)
\curveto(843.67800422,358.89764284)(843.74300415,358.89764284)(843.83300781,358.88765076)
\curveto(843.92300397,358.88764285)(843.97800392,358.89764284)(843.99800781,358.91765076)
\curveto(844.6980032,358.92764281)(845.30800259,358.84764289)(845.82800781,358.67765076)
\curveto(846.34800155,358.50764323)(846.71300118,358.18264356)(846.92300781,357.70265076)
\curveto(847.01300088,357.50264424)(847.06300083,357.26764447)(847.07300781,356.99765076)
\curveto(847.0930008,356.737645)(847.10300079,356.46264528)(847.10300781,356.17265076)
\lineto(847.10300781,352.85765076)
\curveto(847.10300079,352.71764902)(847.10800079,352.58264916)(847.11800781,352.45265076)
\curveto(847.12800077,352.32264942)(847.15800074,352.21764952)(847.20800781,352.13765076)
\curveto(847.25800064,352.06764967)(847.32300057,352.01764972)(847.40300781,351.98765076)
\curveto(847.4930004,351.94764979)(847.57800032,351.91764982)(847.65800781,351.89765076)
\curveto(847.73800016,351.88764985)(847.7980001,351.8426499)(847.83800781,351.76265076)
\curveto(847.85800004,351.73265001)(847.86800003,351.70265004)(847.86800781,351.67265076)
\curveto(847.86800003,351.6426501)(847.87300002,351.60265014)(847.88300781,351.55265076)
\moveto(845.73800781,353.21765076)
\curveto(845.7980021,353.35764838)(845.82800207,353.51764822)(845.82800781,353.69765076)
\curveto(845.83800206,353.88764785)(845.84300205,354.08264766)(845.84300781,354.28265076)
\curveto(845.84300205,354.39264735)(845.83800206,354.49264725)(845.82800781,354.58265076)
\curveto(845.81800208,354.67264707)(845.77800212,354.742647)(845.70800781,354.79265076)
\curveto(845.67800222,354.81264693)(845.60800229,354.82264692)(845.49800781,354.82265076)
\curveto(845.47800242,354.80264694)(845.44300245,354.79264695)(845.39300781,354.79265076)
\curveto(845.34300255,354.79264695)(845.2980026,354.78264696)(845.25800781,354.76265076)
\curveto(845.17800272,354.742647)(845.08800281,354.72264702)(844.98800781,354.70265076)
\lineto(844.68800781,354.64265076)
\curveto(844.65800324,354.6426471)(844.62300327,354.6376471)(844.58300781,354.62765076)
\lineto(844.47800781,354.62765076)
\curveto(844.32800357,354.58764715)(844.16300373,354.56264718)(843.98300781,354.55265076)
\curveto(843.81300408,354.55264719)(843.65300424,354.53264721)(843.50300781,354.49265076)
\curveto(843.42300447,354.47264727)(843.34800455,354.45264729)(843.27800781,354.43265076)
\curveto(843.21800468,354.42264732)(843.14800475,354.40764733)(843.06800781,354.38765076)
\curveto(842.90800499,354.3376474)(842.75800514,354.27264747)(842.61800781,354.19265076)
\curveto(842.47800542,354.12264762)(842.35800554,354.03264771)(842.25800781,353.92265076)
\curveto(842.15800574,353.81264793)(842.08300581,353.67764806)(842.03300781,353.51765076)
\curveto(841.98300591,353.36764837)(841.96300593,353.18264856)(841.97300781,352.96265076)
\curveto(841.97300592,352.86264888)(841.98800591,352.76764897)(842.01800781,352.67765076)
\curveto(842.05800584,352.59764914)(842.10300579,352.52264922)(842.15300781,352.45265076)
\curveto(842.23300566,352.3426494)(842.33800556,352.24764949)(842.46800781,352.16765076)
\curveto(842.5980053,352.09764964)(842.73800516,352.0376497)(842.88800781,351.98765076)
\curveto(842.93800496,351.97764976)(842.98800491,351.97264977)(843.03800781,351.97265076)
\curveto(843.08800481,351.97264977)(843.13800476,351.96764977)(843.18800781,351.95765076)
\curveto(843.25800464,351.9376498)(843.34300455,351.92264982)(843.44300781,351.91265076)
\curveto(843.55300434,351.91264983)(843.64300425,351.92264982)(843.71300781,351.94265076)
\curveto(843.77300412,351.96264978)(843.83300406,351.96764977)(843.89300781,351.95765076)
\curveto(843.95300394,351.95764978)(844.01300388,351.96764977)(844.07300781,351.98765076)
\curveto(844.15300374,352.00764973)(844.22800367,352.02264972)(844.29800781,352.03265076)
\curveto(844.37800352,352.0426497)(844.45300344,352.06264968)(844.52300781,352.09265076)
\curveto(844.81300308,352.21264953)(845.05800284,352.35764938)(845.25800781,352.52765076)
\curveto(845.46800243,352.69764904)(845.62800227,352.92764881)(845.73800781,353.21765076)
}
}
{
\newrgbcolor{curcolor}{0 0 0}
\pscustom[linestyle=none,fillstyle=solid,fillcolor=curcolor]
{
\newpath
\moveto(852.69964844,358.90265076)
\curveto(852.92964365,358.90264284)(853.05964352,358.8426429)(853.08964844,358.72265076)
\curveto(853.11964346,358.61264313)(853.13464344,358.44764329)(853.13464844,358.22765076)
\lineto(853.13464844,357.94265076)
\curveto(853.13464344,357.85264389)(853.10964347,357.77764396)(853.05964844,357.71765076)
\curveto(852.99964358,357.6376441)(852.91464366,357.59264415)(852.80464844,357.58265076)
\curveto(852.69464388,357.58264416)(852.58464399,357.56764417)(852.47464844,357.53765076)
\curveto(852.33464424,357.50764423)(852.19964438,357.47764426)(852.06964844,357.44765076)
\curveto(851.94964463,357.41764432)(851.83464474,357.37764436)(851.72464844,357.32765076)
\curveto(851.43464514,357.19764454)(851.19964538,357.01764472)(851.01964844,356.78765076)
\curveto(850.83964574,356.56764517)(850.68464589,356.31264543)(850.55464844,356.02265076)
\curveto(850.51464606,355.91264583)(850.48464609,355.79764594)(850.46464844,355.67765076)
\curveto(850.44464613,355.56764617)(850.41964616,355.45264629)(850.38964844,355.33265076)
\curveto(850.3796462,355.28264646)(850.3746462,355.23264651)(850.37464844,355.18265076)
\curveto(850.38464619,355.13264661)(850.38464619,355.08264666)(850.37464844,355.03265076)
\curveto(850.34464623,354.91264683)(850.32964625,354.77264697)(850.32964844,354.61265076)
\curveto(850.33964624,354.46264728)(850.34464623,354.31764742)(850.34464844,354.17765076)
\lineto(850.34464844,352.33265076)
\lineto(850.34464844,351.98765076)
\curveto(850.34464623,351.86764987)(850.33964624,351.75264999)(850.32964844,351.64265076)
\curveto(850.31964626,351.53265021)(850.31464626,351.4376503)(850.31464844,351.35765076)
\curveto(850.32464625,351.27765046)(850.30464627,351.20765053)(850.25464844,351.14765076)
\curveto(850.20464637,351.07765066)(850.12464645,351.0376507)(850.01464844,351.02765076)
\curveto(849.91464666,351.01765072)(849.80464677,351.01265073)(849.68464844,351.01265076)
\lineto(849.41464844,351.01265076)
\curveto(849.36464721,351.03265071)(849.31464726,351.04765069)(849.26464844,351.05765076)
\curveto(849.22464735,351.07765066)(849.19464738,351.10265064)(849.17464844,351.13265076)
\curveto(849.12464745,351.20265054)(849.09464748,351.28765045)(849.08464844,351.38765076)
\lineto(849.08464844,351.71765076)
\lineto(849.08464844,352.87265076)
\lineto(849.08464844,357.02765076)
\lineto(849.08464844,358.06265076)
\lineto(849.08464844,358.36265076)
\curveto(849.09464748,358.46264328)(849.12464745,358.54764319)(849.17464844,358.61765076)
\curveto(849.20464737,358.65764308)(849.25464732,358.68764305)(849.32464844,358.70765076)
\curveto(849.40464717,358.72764301)(849.48964709,358.737643)(849.57964844,358.73765076)
\curveto(849.66964691,358.74764299)(849.75964682,358.74764299)(849.84964844,358.73765076)
\curveto(849.93964664,358.72764301)(850.00964657,358.71264303)(850.05964844,358.69265076)
\curveto(850.13964644,358.66264308)(850.18964639,358.60264314)(850.20964844,358.51265076)
\curveto(850.23964634,358.43264331)(850.25464632,358.3426434)(850.25464844,358.24265076)
\lineto(850.25464844,357.94265076)
\curveto(850.25464632,357.8426439)(850.2746463,357.75264399)(850.31464844,357.67265076)
\curveto(850.32464625,357.65264409)(850.33464624,357.6376441)(850.34464844,357.62765076)
\lineto(850.38964844,357.58265076)
\curveto(850.49964608,357.58264416)(850.58964599,357.62764411)(850.65964844,357.71765076)
\curveto(850.72964585,357.81764392)(850.78964579,357.89764384)(850.83964844,357.95765076)
\lineto(850.92964844,358.04765076)
\curveto(851.01964556,358.15764358)(851.14464543,358.27264347)(851.30464844,358.39265076)
\curveto(851.46464511,358.51264323)(851.61464496,358.60264314)(851.75464844,358.66265076)
\curveto(851.84464473,358.71264303)(851.93964464,358.74764299)(852.03964844,358.76765076)
\curveto(852.13964444,358.79764294)(852.24464433,358.82764291)(852.35464844,358.85765076)
\curveto(852.41464416,358.86764287)(852.4746441,358.87264287)(852.53464844,358.87265076)
\curveto(852.59464398,358.88264286)(852.64964393,358.89264285)(852.69964844,358.90265076)
}
}
{
\newrgbcolor{curcolor}{0 0 0}
\pscustom[linestyle=none,fillstyle=solid,fillcolor=curcolor]
{
\newpath
\moveto(855.00941406,361.06265076)
\curveto(855.15941205,361.06264068)(855.3094119,361.05764068)(855.45941406,361.04765076)
\curveto(855.6094116,361.04764069)(855.7144115,361.00764073)(855.77441406,360.92765076)
\curveto(855.82441139,360.86764087)(855.84941136,360.78264096)(855.84941406,360.67265076)
\curveto(855.85941135,360.57264117)(855.86441135,360.46764127)(855.86441406,360.35765076)
\lineto(855.86441406,359.48765076)
\curveto(855.86441135,359.40764233)(855.85941135,359.32264242)(855.84941406,359.23265076)
\curveto(855.84941136,359.15264259)(855.85941135,359.08264266)(855.87941406,359.02265076)
\curveto(855.91941129,358.88264286)(856.0094112,358.79264295)(856.14941406,358.75265076)
\curveto(856.19941101,358.742643)(856.24441097,358.737643)(856.28441406,358.73765076)
\lineto(856.43441406,358.73765076)
\lineto(856.83941406,358.73765076)
\curveto(856.99941021,358.74764299)(857.1144101,358.737643)(857.18441406,358.70765076)
\curveto(857.27440994,358.64764309)(857.33440988,358.58764315)(857.36441406,358.52765076)
\curveto(857.38440983,358.48764325)(857.39440982,358.4426433)(857.39441406,358.39265076)
\lineto(857.39441406,358.24265076)
\curveto(857.39440982,358.13264361)(857.38940982,358.02764371)(857.37941406,357.92765076)
\curveto(857.36940984,357.8376439)(857.33440988,357.76764397)(857.27441406,357.71765076)
\curveto(857.21441,357.66764407)(857.12941008,357.6376441)(857.01941406,357.62765076)
\lineto(856.68941406,357.62765076)
\curveto(856.57941063,357.6376441)(856.46941074,357.6426441)(856.35941406,357.64265076)
\curveto(856.24941096,357.6426441)(856.15441106,357.62764411)(856.07441406,357.59765076)
\curveto(856.00441121,357.56764417)(855.95441126,357.51764422)(855.92441406,357.44765076)
\curveto(855.89441132,357.37764436)(855.87441134,357.29264445)(855.86441406,357.19265076)
\curveto(855.85441136,357.10264464)(855.84941136,357.00264474)(855.84941406,356.89265076)
\curveto(855.85941135,356.79264495)(855.86441135,356.69264505)(855.86441406,356.59265076)
\lineto(855.86441406,353.62265076)
\curveto(855.86441135,353.40264834)(855.85941135,353.16764857)(855.84941406,352.91765076)
\curveto(855.84941136,352.67764906)(855.89441132,352.49264925)(855.98441406,352.36265076)
\curveto(856.03441118,352.28264946)(856.09941111,352.22764951)(856.17941406,352.19765076)
\curveto(856.25941095,352.16764957)(856.35441086,352.1426496)(856.46441406,352.12265076)
\curveto(856.49441072,352.11264963)(856.52441069,352.10764963)(856.55441406,352.10765076)
\curveto(856.59441062,352.11764962)(856.62941058,352.11764962)(856.65941406,352.10765076)
\lineto(856.85441406,352.10765076)
\curveto(856.95441026,352.10764963)(857.04441017,352.09764964)(857.12441406,352.07765076)
\curveto(857.21441,352.06764967)(857.27940993,352.03264971)(857.31941406,351.97265076)
\curveto(857.33940987,351.9426498)(857.35440986,351.88764985)(857.36441406,351.80765076)
\curveto(857.38440983,351.73765)(857.39440982,351.66265008)(857.39441406,351.58265076)
\curveto(857.40440981,351.50265024)(857.40440981,351.42265032)(857.39441406,351.34265076)
\curveto(857.38440983,351.27265047)(857.36440985,351.21765052)(857.33441406,351.17765076)
\curveto(857.29440992,351.10765063)(857.21940999,351.05765068)(857.10941406,351.02765076)
\curveto(857.02941018,351.00765073)(856.93941027,350.99765074)(856.83941406,350.99765076)
\curveto(856.73941047,351.00765073)(856.64941056,351.01265073)(856.56941406,351.01265076)
\curveto(856.5094107,351.01265073)(856.44941076,351.00765073)(856.38941406,350.99765076)
\curveto(856.32941088,350.99765074)(856.27441094,351.00265074)(856.22441406,351.01265076)
\lineto(856.04441406,351.01265076)
\curveto(855.99441122,351.02265072)(855.94441127,351.02765071)(855.89441406,351.02765076)
\curveto(855.85441136,351.0376507)(855.8094114,351.0426507)(855.75941406,351.04265076)
\curveto(855.55941165,351.09265065)(855.38441183,351.14765059)(855.23441406,351.20765076)
\curveto(855.09441212,351.26765047)(854.97441224,351.37265037)(854.87441406,351.52265076)
\curveto(854.73441248,351.72265002)(854.65441256,351.97264977)(854.63441406,352.27265076)
\curveto(854.6144126,352.58264916)(854.60441261,352.91264883)(854.60441406,353.26265076)
\lineto(854.60441406,357.19265076)
\curveto(854.57441264,357.32264442)(854.54441267,357.41764432)(854.51441406,357.47765076)
\curveto(854.49441272,357.5376442)(854.42441279,357.58764415)(854.30441406,357.62765076)
\curveto(854.26441295,357.6376441)(854.22441299,357.6376441)(854.18441406,357.62765076)
\curveto(854.14441307,357.61764412)(854.10441311,357.62264412)(854.06441406,357.64265076)
\lineto(853.82441406,357.64265076)
\curveto(853.69441352,357.6426441)(853.58441363,357.65264409)(853.49441406,357.67265076)
\curveto(853.4144138,357.70264404)(853.35941385,357.76264398)(853.32941406,357.85265076)
\curveto(853.3094139,357.89264385)(853.29441392,357.9376438)(853.28441406,357.98765076)
\lineto(853.28441406,358.13765076)
\curveto(853.28441393,358.27764346)(853.29441392,358.39264335)(853.31441406,358.48265076)
\curveto(853.33441388,358.58264316)(853.39441382,358.65764308)(853.49441406,358.70765076)
\curveto(853.60441361,358.74764299)(853.74441347,358.75764298)(853.91441406,358.73765076)
\curveto(854.09441312,358.71764302)(854.24441297,358.72764301)(854.36441406,358.76765076)
\curveto(854.45441276,358.81764292)(854.52441269,358.88764285)(854.57441406,358.97765076)
\curveto(854.59441262,359.0376427)(854.60441261,359.11264263)(854.60441406,359.20265076)
\lineto(854.60441406,359.45765076)
\lineto(854.60441406,360.38765076)
\lineto(854.60441406,360.62765076)
\curveto(854.60441261,360.71764102)(854.6144126,360.79264095)(854.63441406,360.85265076)
\curveto(854.67441254,360.93264081)(854.74941246,360.99764074)(854.85941406,361.04765076)
\curveto(854.88941232,361.04764069)(854.9144123,361.04764069)(854.93441406,361.04765076)
\curveto(854.96441225,361.05764068)(854.98941222,361.06264068)(855.00941406,361.06265076)
}
}
{
\newrgbcolor{curcolor}{0 0 0}
\pscustom[linestyle=none,fillstyle=solid,fillcolor=curcolor]
{
\newpath
\moveto(859.06621094,360.22265076)
\curveto(858.98620982,360.28264146)(858.94120986,360.38764135)(858.93121094,360.53765076)
\lineto(858.93121094,361.00265076)
\lineto(858.93121094,361.25765076)
\curveto(858.93120987,361.34764039)(858.94620986,361.42264032)(858.97621094,361.48265076)
\curveto(859.01620979,361.56264018)(859.09620971,361.62264012)(859.21621094,361.66265076)
\curveto(859.23620957,361.67264007)(859.25620955,361.67264007)(859.27621094,361.66265076)
\curveto(859.3062095,361.66264008)(859.33120947,361.66764007)(859.35121094,361.67765076)
\curveto(859.52120928,361.67764006)(859.68120912,361.67264007)(859.83121094,361.66265076)
\curveto(859.98120882,361.65264009)(860.08120872,361.59264015)(860.13121094,361.48265076)
\curveto(860.16120864,361.42264032)(860.17620863,361.34764039)(860.17621094,361.25765076)
\lineto(860.17621094,361.00265076)
\curveto(860.17620863,360.82264092)(860.17120863,360.65264109)(860.16121094,360.49265076)
\curveto(860.16120864,360.33264141)(860.09620871,360.22764151)(859.96621094,360.17765076)
\curveto(859.91620889,360.15764158)(859.86120894,360.14764159)(859.80121094,360.14765076)
\lineto(859.63621094,360.14765076)
\lineto(859.32121094,360.14765076)
\curveto(859.22120958,360.14764159)(859.13620967,360.17264157)(859.06621094,360.22265076)
\moveto(860.17621094,351.71765076)
\lineto(860.17621094,351.40265076)
\curveto(860.18620862,351.30265044)(860.16620864,351.22265052)(860.11621094,351.16265076)
\curveto(860.08620872,351.10265064)(860.04120876,351.06265068)(859.98121094,351.04265076)
\curveto(859.92120888,351.03265071)(859.85120895,351.01765072)(859.77121094,350.99765076)
\lineto(859.54621094,350.99765076)
\curveto(859.41620939,350.99765074)(859.3012095,351.00265074)(859.20121094,351.01265076)
\curveto(859.11120969,351.03265071)(859.04120976,351.08265066)(858.99121094,351.16265076)
\curveto(858.95120985,351.22265052)(858.93120987,351.29765044)(858.93121094,351.38765076)
\lineto(858.93121094,351.67265076)
\lineto(858.93121094,358.01765076)
\lineto(858.93121094,358.33265076)
\curveto(858.93120987,358.4426433)(858.95620985,358.52764321)(859.00621094,358.58765076)
\curveto(859.03620977,358.6376431)(859.07620973,358.66764307)(859.12621094,358.67765076)
\curveto(859.17620963,358.68764305)(859.23120957,358.70264304)(859.29121094,358.72265076)
\curveto(859.31120949,358.72264302)(859.33120947,358.71764302)(859.35121094,358.70765076)
\curveto(859.38120942,358.70764303)(859.4062094,358.71264303)(859.42621094,358.72265076)
\curveto(859.55620925,358.72264302)(859.68620912,358.71764302)(859.81621094,358.70765076)
\curveto(859.95620885,358.70764303)(860.05120875,358.66764307)(860.10121094,358.58765076)
\curveto(860.15120865,358.52764321)(860.17620863,358.44764329)(860.17621094,358.34765076)
\lineto(860.17621094,358.06265076)
\lineto(860.17621094,351.71765076)
}
}
{
\newrgbcolor{curcolor}{0 0 0}
\pscustom[linestyle=none,fillstyle=solid,fillcolor=curcolor]
{
\newpath
\moveto(865.25605469,358.90265076)
\curveto(865.9960499,358.91264283)(866.61104928,358.80264294)(867.10105469,358.57265076)
\curveto(867.60104829,358.35264339)(867.9960479,358.01764372)(868.28605469,357.56765076)
\curveto(868.41604748,357.36764437)(868.52604737,357.12264462)(868.61605469,356.83265076)
\curveto(868.63604726,356.78264496)(868.65104724,356.71764502)(868.66105469,356.63765076)
\curveto(868.67104722,356.55764518)(868.66604723,356.48764525)(868.64605469,356.42765076)
\curveto(868.61604728,356.37764536)(868.56604733,356.33264541)(868.49605469,356.29265076)
\curveto(868.46604743,356.27264547)(868.43604746,356.26264548)(868.40605469,356.26265076)
\curveto(868.37604752,356.27264547)(868.34104755,356.27264547)(868.30105469,356.26265076)
\curveto(868.26104763,356.25264549)(868.22104767,356.24764549)(868.18105469,356.24765076)
\curveto(868.14104775,356.25764548)(868.10104779,356.26264548)(868.06105469,356.26265076)
\lineto(867.74605469,356.26265076)
\curveto(867.64604825,356.27264547)(867.56104833,356.30264544)(867.49105469,356.35265076)
\curveto(867.41104848,356.41264533)(867.35604854,356.49764524)(867.32605469,356.60765076)
\curveto(867.2960486,356.71764502)(867.25604864,356.81264493)(867.20605469,356.89265076)
\curveto(867.05604884,357.15264459)(866.86104903,357.35764438)(866.62105469,357.50765076)
\curveto(866.54104935,357.55764418)(866.45604944,357.59764414)(866.36605469,357.62765076)
\curveto(866.27604962,357.66764407)(866.18104971,357.70264404)(866.08105469,357.73265076)
\curveto(865.94104995,357.77264397)(865.75605014,357.79264395)(865.52605469,357.79265076)
\curveto(865.2960506,357.80264394)(865.10605079,357.78264396)(864.95605469,357.73265076)
\curveto(864.88605101,357.71264403)(864.82105107,357.69764404)(864.76105469,357.68765076)
\curveto(864.70105119,357.67764406)(864.63605126,357.66264408)(864.56605469,357.64265076)
\curveto(864.30605159,357.53264421)(864.07605182,357.38264436)(863.87605469,357.19265076)
\curveto(863.67605222,357.00264474)(863.52105237,356.77764496)(863.41105469,356.51765076)
\curveto(863.37105252,356.42764531)(863.33605256,356.33264541)(863.30605469,356.23265076)
\curveto(863.27605262,356.1426456)(863.24605265,356.0426457)(863.21605469,355.93265076)
\lineto(863.12605469,355.52765076)
\curveto(863.11605278,355.47764626)(863.11105278,355.42264632)(863.11105469,355.36265076)
\curveto(863.12105277,355.30264644)(863.11605278,355.24764649)(863.09605469,355.19765076)
\lineto(863.09605469,355.07765076)
\curveto(863.08605281,355.0376467)(863.07605282,354.97264677)(863.06605469,354.88265076)
\curveto(863.06605283,354.79264695)(863.07605282,354.72764701)(863.09605469,354.68765076)
\curveto(863.10605279,354.6376471)(863.10605279,354.58764715)(863.09605469,354.53765076)
\curveto(863.08605281,354.48764725)(863.08605281,354.4376473)(863.09605469,354.38765076)
\curveto(863.10605279,354.34764739)(863.11105278,354.27764746)(863.11105469,354.17765076)
\curveto(863.13105276,354.09764764)(863.14605275,354.01264773)(863.15605469,353.92265076)
\curveto(863.17605272,353.83264791)(863.1960527,353.74764799)(863.21605469,353.66765076)
\curveto(863.32605257,353.34764839)(863.45105244,353.06764867)(863.59105469,352.82765076)
\curveto(863.74105215,352.59764914)(863.94605195,352.39764934)(864.20605469,352.22765076)
\curveto(864.2960516,352.17764956)(864.38605151,352.13264961)(864.47605469,352.09265076)
\curveto(864.57605132,352.05264969)(864.68105121,352.01264973)(864.79105469,351.97265076)
\curveto(864.84105105,351.96264978)(864.88105101,351.95764978)(864.91105469,351.95765076)
\curveto(864.94105095,351.95764978)(864.98105091,351.95264979)(865.03105469,351.94265076)
\curveto(865.06105083,351.93264981)(865.11105078,351.92764981)(865.18105469,351.92765076)
\lineto(865.34605469,351.92765076)
\curveto(865.34605055,351.91764982)(865.36605053,351.91264983)(865.40605469,351.91265076)
\curveto(865.42605047,351.92264982)(865.45105044,351.92264982)(865.48105469,351.91265076)
\curveto(865.51105038,351.91264983)(865.54105035,351.91764982)(865.57105469,351.92765076)
\curveto(865.64105025,351.94764979)(865.70605019,351.95264979)(865.76605469,351.94265076)
\curveto(865.83605006,351.9426498)(865.90604999,351.95264979)(865.97605469,351.97265076)
\curveto(866.23604966,352.05264969)(866.46104943,352.15264959)(866.65105469,352.27265076)
\curveto(866.84104905,352.40264934)(867.00104889,352.56764917)(867.13105469,352.76765076)
\curveto(867.18104871,352.84764889)(867.22604867,352.93264881)(867.26605469,353.02265076)
\lineto(867.38605469,353.29265076)
\curveto(867.40604849,353.37264837)(867.42604847,353.44764829)(867.44605469,353.51765076)
\curveto(867.47604842,353.59764814)(867.52604837,353.66264808)(867.59605469,353.71265076)
\curveto(867.62604827,353.742648)(867.68604821,353.76264798)(867.77605469,353.77265076)
\curveto(867.86604803,353.79264795)(867.96104793,353.80264794)(868.06105469,353.80265076)
\curveto(868.17104772,353.81264793)(868.27104762,353.81264793)(868.36105469,353.80265076)
\curveto(868.46104743,353.79264795)(868.53104736,353.77264797)(868.57105469,353.74265076)
\curveto(868.63104726,353.70264804)(868.66604723,353.6426481)(868.67605469,353.56265076)
\curveto(868.6960472,353.48264826)(868.6960472,353.39764834)(868.67605469,353.30765076)
\curveto(868.62604727,353.15764858)(868.57604732,353.01264873)(868.52605469,352.87265076)
\curveto(868.48604741,352.742649)(868.43104746,352.61264913)(868.36105469,352.48265076)
\curveto(868.21104768,352.18264956)(868.02104787,351.91764982)(867.79105469,351.68765076)
\curveto(867.57104832,351.45765028)(867.30104859,351.27265047)(866.98105469,351.13265076)
\curveto(866.90104899,351.09265065)(866.81604908,351.05765068)(866.72605469,351.02765076)
\curveto(866.63604926,351.00765073)(866.54104935,350.98265076)(866.44105469,350.95265076)
\curveto(866.33104956,350.91265083)(866.22104967,350.89265085)(866.11105469,350.89265076)
\curveto(866.00104989,350.88265086)(865.89105,350.86765087)(865.78105469,350.84765076)
\curveto(865.74105015,350.82765091)(865.70105019,350.82265092)(865.66105469,350.83265076)
\curveto(865.62105027,350.8426509)(865.58105031,350.8426509)(865.54105469,350.83265076)
\lineto(865.40605469,350.83265076)
\lineto(865.16605469,350.83265076)
\curveto(865.0960508,350.82265092)(865.03105086,350.82765091)(864.97105469,350.84765076)
\lineto(864.89605469,350.84765076)
\lineto(864.53605469,350.89265076)
\curveto(864.40605149,350.93265081)(864.28105161,350.96765077)(864.16105469,350.99765076)
\curveto(864.04105185,351.02765071)(863.92605197,351.06765067)(863.81605469,351.11765076)
\curveto(863.45605244,351.27765046)(863.15605274,351.46765027)(862.91605469,351.68765076)
\curveto(862.68605321,351.90764983)(862.47105342,352.17764956)(862.27105469,352.49765076)
\curveto(862.22105367,352.57764916)(862.17605372,352.66764907)(862.13605469,352.76765076)
\lineto(862.01605469,353.06765076)
\curveto(861.96605393,353.17764856)(861.93105396,353.29264845)(861.91105469,353.41265076)
\curveto(861.891054,353.53264821)(861.86605403,353.65264809)(861.83605469,353.77265076)
\curveto(861.82605407,353.81264793)(861.82105407,353.85264789)(861.82105469,353.89265076)
\curveto(861.82105407,353.93264781)(861.81605408,353.97264777)(861.80605469,354.01265076)
\curveto(861.78605411,354.07264767)(861.77605412,354.1376476)(861.77605469,354.20765076)
\curveto(861.78605411,354.27764746)(861.78105411,354.3426474)(861.76105469,354.40265076)
\lineto(861.76105469,354.55265076)
\curveto(861.75105414,354.60264714)(861.74605415,354.67264707)(861.74605469,354.76265076)
\curveto(861.74605415,354.85264689)(861.75105414,354.92264682)(861.76105469,354.97265076)
\curveto(861.77105412,355.02264672)(861.77105412,355.06764667)(861.76105469,355.10765076)
\curveto(861.76105413,355.14764659)(861.76605413,355.18764655)(861.77605469,355.22765076)
\curveto(861.7960541,355.29764644)(861.80105409,355.36764637)(861.79105469,355.43765076)
\curveto(861.7910541,355.50764623)(861.80105409,355.57264617)(861.82105469,355.63265076)
\curveto(861.86105403,355.80264594)(861.896054,355.97264577)(861.92605469,356.14265076)
\curveto(861.95605394,356.31264543)(862.00105389,356.47264527)(862.06105469,356.62265076)
\curveto(862.27105362,357.1426446)(862.52605337,357.56264418)(862.82605469,357.88265076)
\curveto(863.12605277,358.20264354)(863.53605236,358.46764327)(864.05605469,358.67765076)
\curveto(864.16605173,358.72764301)(864.28605161,358.76264298)(864.41605469,358.78265076)
\curveto(864.54605135,358.80264294)(864.68105121,358.82764291)(864.82105469,358.85765076)
\curveto(864.891051,358.86764287)(864.96105093,358.87264287)(865.03105469,358.87265076)
\curveto(865.10105079,358.88264286)(865.17605072,358.89264285)(865.25605469,358.90265076)
}
}
{
\newrgbcolor{curcolor}{0 0 0}
\pscustom[linestyle=none,fillstyle=solid,fillcolor=curcolor]
{
\newpath
\moveto(870.46269531,360.22265076)
\curveto(870.38269419,360.28264146)(870.33769424,360.38764135)(870.32769531,360.53765076)
\lineto(870.32769531,361.00265076)
\lineto(870.32769531,361.25765076)
\curveto(870.32769425,361.34764039)(870.34269423,361.42264032)(870.37269531,361.48265076)
\curveto(870.41269416,361.56264018)(870.49269408,361.62264012)(870.61269531,361.66265076)
\curveto(870.63269394,361.67264007)(870.65269392,361.67264007)(870.67269531,361.66265076)
\curveto(870.70269387,361.66264008)(870.72769385,361.66764007)(870.74769531,361.67765076)
\curveto(870.91769366,361.67764006)(871.0776935,361.67264007)(871.22769531,361.66265076)
\curveto(871.3776932,361.65264009)(871.4776931,361.59264015)(871.52769531,361.48265076)
\curveto(871.55769302,361.42264032)(871.572693,361.34764039)(871.57269531,361.25765076)
\lineto(871.57269531,361.00265076)
\curveto(871.572693,360.82264092)(871.56769301,360.65264109)(871.55769531,360.49265076)
\curveto(871.55769302,360.33264141)(871.49269308,360.22764151)(871.36269531,360.17765076)
\curveto(871.31269326,360.15764158)(871.25769332,360.14764159)(871.19769531,360.14765076)
\lineto(871.03269531,360.14765076)
\lineto(870.71769531,360.14765076)
\curveto(870.61769396,360.14764159)(870.53269404,360.17264157)(870.46269531,360.22265076)
\moveto(871.57269531,351.71765076)
\lineto(871.57269531,351.40265076)
\curveto(871.58269299,351.30265044)(871.56269301,351.22265052)(871.51269531,351.16265076)
\curveto(871.48269309,351.10265064)(871.43769314,351.06265068)(871.37769531,351.04265076)
\curveto(871.31769326,351.03265071)(871.24769333,351.01765072)(871.16769531,350.99765076)
\lineto(870.94269531,350.99765076)
\curveto(870.81269376,350.99765074)(870.69769388,351.00265074)(870.59769531,351.01265076)
\curveto(870.50769407,351.03265071)(870.43769414,351.08265066)(870.38769531,351.16265076)
\curveto(870.34769423,351.22265052)(870.32769425,351.29765044)(870.32769531,351.38765076)
\lineto(870.32769531,351.67265076)
\lineto(870.32769531,358.01765076)
\lineto(870.32769531,358.33265076)
\curveto(870.32769425,358.4426433)(870.35269422,358.52764321)(870.40269531,358.58765076)
\curveto(870.43269414,358.6376431)(870.4726941,358.66764307)(870.52269531,358.67765076)
\curveto(870.572694,358.68764305)(870.62769395,358.70264304)(870.68769531,358.72265076)
\curveto(870.70769387,358.72264302)(870.72769385,358.71764302)(870.74769531,358.70765076)
\curveto(870.7776938,358.70764303)(870.80269377,358.71264303)(870.82269531,358.72265076)
\curveto(870.95269362,358.72264302)(871.08269349,358.71764302)(871.21269531,358.70765076)
\curveto(871.35269322,358.70764303)(871.44769313,358.66764307)(871.49769531,358.58765076)
\curveto(871.54769303,358.52764321)(871.572693,358.44764329)(871.57269531,358.34765076)
\lineto(871.57269531,358.06265076)
\lineto(871.57269531,351.71765076)
}
}
{
\newrgbcolor{curcolor}{0 0 0}
\pscustom[linestyle=none,fillstyle=solid,fillcolor=curcolor]
{
\newpath
\moveto(880.94253906,355.06265076)
\curveto(880.95253071,355.01264673)(880.95753071,354.94764679)(880.95753906,354.86765076)
\curveto(880.95753071,354.78764695)(880.95253071,354.72264702)(880.94253906,354.67265076)
\curveto(880.92253074,354.62264712)(880.91753075,354.57264717)(880.92753906,354.52265076)
\curveto(880.93753073,354.48264726)(880.93753073,354.4426473)(880.92753906,354.40265076)
\curveto(880.92753074,354.33264741)(880.92253074,354.27764746)(880.91253906,354.23765076)
\curveto(880.89253077,354.14764759)(880.87753079,354.05764768)(880.86753906,353.96765076)
\curveto(880.8675308,353.87764786)(880.85753081,353.78764795)(880.83753906,353.69765076)
\lineto(880.77753906,353.45765076)
\curveto(880.75753091,353.38764835)(880.73253093,353.31264843)(880.70253906,353.23265076)
\curveto(880.58253108,352.86264888)(880.41753125,352.52764921)(880.20753906,352.22765076)
\curveto(880.14753152,352.1376496)(880.08253158,352.04764969)(880.01253906,351.95765076)
\curveto(879.94253172,351.87764986)(879.8675318,351.80264994)(879.78753906,351.73265076)
\lineto(879.71253906,351.65765076)
\curveto(879.64253202,351.60765013)(879.57753209,351.55765018)(879.51753906,351.50765076)
\curveto(879.45753221,351.45765028)(879.38753228,351.40765033)(879.30753906,351.35765076)
\curveto(879.19753247,351.27765046)(879.07253259,351.20765053)(878.93253906,351.14765076)
\curveto(878.80253286,351.09765064)(878.667533,351.04765069)(878.52753906,350.99765076)
\curveto(878.44753322,350.97765076)(878.3675333,350.96265078)(878.28753906,350.95265076)
\curveto(878.21753345,350.9426508)(878.14253352,350.92765081)(878.06253906,350.90765076)
\lineto(878.00253906,350.90765076)
\curveto(877.99253367,350.89765084)(877.97753369,350.89265085)(877.95753906,350.89265076)
\curveto(877.8675338,350.87265087)(877.73253393,350.86265088)(877.55253906,350.86265076)
\curveto(877.38253428,350.85265089)(877.24753442,350.85765088)(877.14753906,350.87765076)
\lineto(877.07253906,350.87765076)
\curveto(877.00253466,350.88765085)(876.93753473,350.89765084)(876.87753906,350.90765076)
\curveto(876.81753485,350.90765083)(876.75753491,350.91765082)(876.69753906,350.93765076)
\curveto(876.52753514,350.98765075)(876.3675353,351.03265071)(876.21753906,351.07265076)
\curveto(876.0675356,351.11265063)(875.92753574,351.17265057)(875.79753906,351.25265076)
\curveto(875.63753603,351.3426504)(875.49753617,351.4376503)(875.37753906,351.53765076)
\curveto(875.33753633,351.56765017)(875.27753639,351.60765013)(875.19753906,351.65765076)
\curveto(875.11753655,351.71765002)(875.04253662,351.72265002)(874.97253906,351.67265076)
\curveto(874.93253673,351.6426501)(874.91253675,351.60265014)(874.91253906,351.55265076)
\curveto(874.91253675,351.50265024)(874.90253676,351.44765029)(874.88253906,351.38765076)
\curveto(874.87253679,351.35765038)(874.87253679,351.32265042)(874.88253906,351.28265076)
\curveto(874.89253677,351.25265049)(874.89253677,351.21765052)(874.88253906,351.17765076)
\curveto(874.8625368,351.11765062)(874.85253681,351.05265069)(874.85253906,350.98265076)
\curveto(874.8625368,350.90265084)(874.8675368,350.83265091)(874.86753906,350.77265076)
\lineto(874.86753906,348.97265076)
\lineto(874.86753906,348.53765076)
\curveto(874.8675368,348.38765335)(874.83753683,348.27265347)(874.77753906,348.19265076)
\curveto(874.72753694,348.12265362)(874.64753702,348.08765365)(874.53753906,348.08765076)
\curveto(874.42753724,348.07765366)(874.31753735,348.07265367)(874.20753906,348.07265076)
\lineto(873.96753906,348.07265076)
\curveto(873.89753777,348.09265365)(873.83753783,348.11265363)(873.78753906,348.13265076)
\curveto(873.74753792,348.15265359)(873.71253795,348.18765355)(873.68253906,348.23765076)
\curveto(873.63253803,348.30765343)(873.60753806,348.41765332)(873.60753906,348.56765076)
\curveto(873.61753805,348.71765302)(873.62253804,348.84765289)(873.62253906,348.95765076)
\lineto(873.62253906,357.95765076)
\lineto(873.62253906,358.31765076)
\curveto(873.63253803,358.44764329)(873.662538,358.55264319)(873.71253906,358.63265076)
\curveto(873.74253792,358.67264307)(873.80753786,358.70264304)(873.90753906,358.72265076)
\curveto(874.01753765,358.75264299)(874.13253753,358.76264298)(874.25253906,358.75265076)
\curveto(874.37253729,358.75264299)(874.48253718,358.737643)(874.58253906,358.70765076)
\curveto(874.69253697,358.68764305)(874.7625369,358.65764308)(874.79253906,358.61765076)
\curveto(874.83253683,358.56764317)(874.85253681,358.50764323)(874.85253906,358.43765076)
\curveto(874.8625368,358.36764337)(874.88253678,358.29764344)(874.91253906,358.22765076)
\curveto(874.93253673,358.19764354)(874.94753672,358.17264357)(874.95753906,358.15265076)
\curveto(874.97753669,358.1426436)(874.99753667,358.12764361)(875.01753906,358.10765076)
\curveto(875.12753654,358.09764364)(875.21753645,358.13264361)(875.28753906,358.21265076)
\curveto(875.3675363,358.29264345)(875.44253622,358.35764338)(875.51253906,358.40765076)
\curveto(875.77253589,358.58764315)(876.08253558,358.72764301)(876.44253906,358.82765076)
\curveto(876.53253513,358.84764289)(876.62253504,358.86264288)(876.71253906,358.87265076)
\curveto(876.81253485,358.88264286)(876.91253475,358.89764284)(877.01253906,358.91765076)
\curveto(877.05253461,358.92764281)(877.10253456,358.92764281)(877.16253906,358.91765076)
\curveto(877.22253444,358.90764283)(877.2625344,358.91264283)(877.28253906,358.93265076)
\curveto(877.71253395,358.9426428)(878.09253357,358.89764284)(878.42253906,358.79765076)
\curveto(878.75253291,358.70764303)(879.04753262,358.57764316)(879.30753906,358.40765076)
\lineto(879.45753906,358.28765076)
\curveto(879.50753216,358.25764348)(879.55753211,358.22264352)(879.60753906,358.18265076)
\curveto(879.62753204,358.16264358)(879.64253202,358.1426436)(879.65253906,358.12265076)
\curveto(879.67253199,358.11264363)(879.69253197,358.09764364)(879.71253906,358.07765076)
\curveto(879.7625319,358.02764371)(879.81753185,357.97264377)(879.87753906,357.91265076)
\curveto(879.93753173,357.85264389)(879.99253167,357.79264395)(880.04253906,357.73265076)
\curveto(880.1625315,357.56264418)(880.28753138,357.37764436)(880.41753906,357.17765076)
\curveto(880.49753117,357.04764469)(880.5625311,356.90264484)(880.61253906,356.74265076)
\curveto(880.67253099,356.58264516)(880.72753094,356.42264532)(880.77753906,356.26265076)
\curveto(880.79753087,356.18264556)(880.81253085,356.09764564)(880.82253906,356.00765076)
\curveto(880.84253082,355.91764582)(880.8625308,355.83264591)(880.88253906,355.75265076)
\lineto(880.88253906,355.63265076)
\curveto(880.89253077,355.60264614)(880.89753077,355.57264617)(880.89753906,355.54265076)
\curveto(880.91753075,355.49264625)(880.92253074,355.4376463)(880.91253906,355.37765076)
\curveto(880.91253075,355.31764642)(880.92253074,355.26264648)(880.94253906,355.21265076)
\lineto(880.94253906,355.06265076)
\moveto(879.60753906,354.65765076)
\curveto(879.62753204,354.70764703)(879.63253203,354.76764697)(879.62253906,354.83765076)
\curveto(879.61253205,354.91764682)(879.60753206,354.98764675)(879.60753906,355.04765076)
\curveto(879.60753206,355.21764652)(879.59753207,355.37764636)(879.57753906,355.52765076)
\curveto(879.5675321,355.67764606)(879.53753213,355.82264592)(879.48753906,355.96265076)
\lineto(879.42753906,356.14265076)
\curveto(879.41753225,356.21264553)(879.39753227,356.27764546)(879.36753906,356.33765076)
\curveto(879.25753241,356.60764513)(879.08253258,356.86764487)(878.84253906,357.11765076)
\curveto(878.61253305,357.36764437)(878.39253327,357.5376442)(878.18253906,357.62765076)
\curveto(878.10253356,357.66764407)(878.01753365,357.69764404)(877.92753906,357.71765076)
\curveto(877.84753382,357.737644)(877.7625339,357.76264398)(877.67253906,357.79265076)
\curveto(877.58253408,357.81264393)(877.47753419,357.82264392)(877.35753906,357.82265076)
\lineto(877.02753906,357.82265076)
\curveto(877.00753466,357.80264394)(876.9675347,357.79264395)(876.90753906,357.79265076)
\curveto(876.85753481,357.80264394)(876.81253485,357.80264394)(876.77253906,357.79265076)
\lineto(876.50253906,357.73265076)
\curveto(876.42253524,357.71264403)(876.34253532,357.68264406)(876.26253906,357.64265076)
\curveto(875.94253572,357.50264424)(875.67753599,357.29764444)(875.46753906,357.02765076)
\curveto(875.2675364,356.76764497)(875.11253655,356.46264528)(875.00253906,356.11265076)
\curveto(874.9625367,356.00264574)(874.93253673,355.89264585)(874.91253906,355.78265076)
\curveto(874.90253676,355.67264607)(874.88753678,355.56264618)(874.86753906,355.45265076)
\curveto(874.85753681,355.41264633)(874.85253681,355.37264637)(874.85253906,355.33265076)
\curveto(874.85253681,355.30264644)(874.84753682,355.26764647)(874.83753906,355.22765076)
\lineto(874.83753906,355.10765076)
\curveto(874.82753684,355.05764668)(874.82253684,354.98264676)(874.82253906,354.88265076)
\curveto(874.82253684,354.79264695)(874.82753684,354.72264702)(874.83753906,354.67265076)
\lineto(874.83753906,354.55265076)
\curveto(874.84753682,354.51264723)(874.85253681,354.47264727)(874.85253906,354.43265076)
\curveto(874.85253681,354.39264735)(874.85753681,354.35764738)(874.86753906,354.32765076)
\curveto(874.87753679,354.29764744)(874.88253678,354.26764747)(874.88253906,354.23765076)
\curveto(874.88253678,354.20764753)(874.88753678,354.17264757)(874.89753906,354.13265076)
\curveto(874.91753675,354.05264769)(874.93253673,353.97264777)(874.94253906,353.89265076)
\lineto(875.00253906,353.65265076)
\curveto(875.11253655,353.31264843)(875.2625364,353.01264873)(875.45253906,352.75265076)
\curveto(875.65253601,352.50264924)(875.91253575,352.30764943)(876.23253906,352.16765076)
\curveto(876.42253524,352.08764965)(876.61753505,352.02764971)(876.81753906,351.98765076)
\curveto(876.85753481,351.96764977)(876.89753477,351.95764978)(876.93753906,351.95765076)
\curveto(876.97753469,351.96764977)(877.01753465,351.96764977)(877.05753906,351.95765076)
\lineto(877.17753906,351.95765076)
\curveto(877.24753442,351.9376498)(877.31753435,351.9376498)(877.38753906,351.95765076)
\lineto(877.50753906,351.95765076)
\curveto(877.61753405,351.97764976)(877.72253394,351.99264975)(877.82253906,352.00265076)
\curveto(877.92253374,352.01264973)(878.02253364,352.0376497)(878.12253906,352.07765076)
\curveto(878.43253323,352.20764953)(878.68253298,352.37764936)(878.87253906,352.58765076)
\curveto(879.07253259,352.80764893)(879.23753243,353.07264867)(879.36753906,353.38265076)
\curveto(879.41753225,353.52264822)(879.45253221,353.66264808)(879.47253906,353.80265076)
\curveto(879.50253216,353.95264779)(879.53753213,354.10764763)(879.57753906,354.26765076)
\curveto(879.58753208,354.31764742)(879.59253207,354.36264738)(879.59253906,354.40265076)
\curveto(879.59253207,354.4426473)(879.59753207,354.48764725)(879.60753906,354.53765076)
\lineto(879.60753906,354.65765076)
}
}
{
\newrgbcolor{curcolor}{0 0 0}
\pscustom[linestyle=none,fillstyle=solid,fillcolor=curcolor]
{
\newpath
\moveto(889.30878906,351.55265076)
\curveto(889.33878123,351.39265035)(889.32378125,351.25765048)(889.26378906,351.14765076)
\curveto(889.20378137,351.04765069)(889.12378145,350.97265077)(889.02378906,350.92265076)
\curveto(888.9737816,350.90265084)(888.91878165,350.89265085)(888.85878906,350.89265076)
\curveto(888.80878176,350.89265085)(888.75378182,350.88265086)(888.69378906,350.86265076)
\curveto(888.4737821,350.81265093)(888.25378232,350.82765091)(888.03378906,350.90765076)
\curveto(887.82378275,350.97765076)(887.67878289,351.06765067)(887.59878906,351.17765076)
\curveto(887.54878302,351.24765049)(887.50378307,351.32765041)(887.46378906,351.41765076)
\curveto(887.42378315,351.51765022)(887.3737832,351.59765014)(887.31378906,351.65765076)
\curveto(887.29378328,351.67765006)(887.2687833,351.69765004)(887.23878906,351.71765076)
\curveto(887.21878335,351.73765)(887.18878338,351.74265)(887.14878906,351.73265076)
\curveto(887.03878353,351.70265004)(886.93378364,351.64765009)(886.83378906,351.56765076)
\curveto(886.74378383,351.48765025)(886.65378392,351.41765032)(886.56378906,351.35765076)
\curveto(886.43378414,351.27765046)(886.29378428,351.20265054)(886.14378906,351.13265076)
\curveto(885.99378458,351.07265067)(885.83378474,351.01765072)(885.66378906,350.96765076)
\curveto(885.56378501,350.9376508)(885.45378512,350.91765082)(885.33378906,350.90765076)
\curveto(885.22378535,350.89765084)(885.11378546,350.88265086)(885.00378906,350.86265076)
\curveto(884.95378562,350.85265089)(884.90878566,350.84765089)(884.86878906,350.84765076)
\lineto(884.76378906,350.84765076)
\curveto(884.65378592,350.82765091)(884.54878602,350.82765091)(884.44878906,350.84765076)
\lineto(884.31378906,350.84765076)
\curveto(884.26378631,350.85765088)(884.21378636,350.86265088)(884.16378906,350.86265076)
\curveto(884.11378646,350.86265088)(884.0687865,350.87265087)(884.02878906,350.89265076)
\curveto(883.98878658,350.90265084)(883.95378662,350.90765083)(883.92378906,350.90765076)
\curveto(883.90378667,350.89765084)(883.87878669,350.89765084)(883.84878906,350.90765076)
\lineto(883.60878906,350.96765076)
\curveto(883.52878704,350.97765076)(883.45378712,350.99765074)(883.38378906,351.02765076)
\curveto(883.08378749,351.15765058)(882.83878773,351.30265044)(882.64878906,351.46265076)
\curveto(882.4687881,351.63265011)(882.31878825,351.86764987)(882.19878906,352.16765076)
\curveto(882.10878846,352.38764935)(882.06378851,352.65264909)(882.06378906,352.96265076)
\lineto(882.06378906,353.27765076)
\curveto(882.0737885,353.32764841)(882.07878849,353.37764836)(882.07878906,353.42765076)
\lineto(882.10878906,353.60765076)
\lineto(882.22878906,353.93765076)
\curveto(882.2687883,354.04764769)(882.31878825,354.14764759)(882.37878906,354.23765076)
\curveto(882.55878801,354.52764721)(882.80378777,354.742647)(883.11378906,354.88265076)
\curveto(883.42378715,355.02264672)(883.76378681,355.14764659)(884.13378906,355.25765076)
\curveto(884.2737863,355.29764644)(884.41878615,355.32764641)(884.56878906,355.34765076)
\curveto(884.71878585,355.36764637)(884.8687857,355.39264635)(885.01878906,355.42265076)
\curveto(885.08878548,355.4426463)(885.15378542,355.45264629)(885.21378906,355.45265076)
\curveto(885.28378529,355.45264629)(885.35878521,355.46264628)(885.43878906,355.48265076)
\curveto(885.50878506,355.50264624)(885.57878499,355.51264623)(885.64878906,355.51265076)
\curveto(885.71878485,355.52264622)(885.79378478,355.5376462)(885.87378906,355.55765076)
\curveto(886.12378445,355.61764612)(886.35878421,355.66764607)(886.57878906,355.70765076)
\curveto(886.79878377,355.75764598)(886.9737836,355.87264587)(887.10378906,356.05265076)
\curveto(887.16378341,356.13264561)(887.21378336,356.23264551)(887.25378906,356.35265076)
\curveto(887.29378328,356.48264526)(887.29378328,356.62264512)(887.25378906,356.77265076)
\curveto(887.19378338,357.01264473)(887.10378347,357.20264454)(886.98378906,357.34265076)
\curveto(886.8737837,357.48264426)(886.71378386,357.59264415)(886.50378906,357.67265076)
\curveto(886.38378419,357.72264402)(886.23878433,357.75764398)(886.06878906,357.77765076)
\curveto(885.90878466,357.79764394)(885.73878483,357.80764393)(885.55878906,357.80765076)
\curveto(885.37878519,357.80764393)(885.20378537,357.79764394)(885.03378906,357.77765076)
\curveto(884.86378571,357.75764398)(884.71878585,357.72764401)(884.59878906,357.68765076)
\curveto(884.42878614,357.62764411)(884.26378631,357.5426442)(884.10378906,357.43265076)
\curveto(884.02378655,357.37264437)(883.94878662,357.29264445)(883.87878906,357.19265076)
\curveto(883.81878675,357.10264464)(883.76378681,357.00264474)(883.71378906,356.89265076)
\curveto(883.68378689,356.81264493)(883.65378692,356.72764501)(883.62378906,356.63765076)
\curveto(883.60378697,356.54764519)(883.55878701,356.47764526)(883.48878906,356.42765076)
\curveto(883.44878712,356.39764534)(883.37878719,356.37264537)(883.27878906,356.35265076)
\curveto(883.18878738,356.3426454)(883.09378748,356.3376454)(882.99378906,356.33765076)
\curveto(882.89378768,356.3376454)(882.79378778,356.3426454)(882.69378906,356.35265076)
\curveto(882.60378797,356.37264537)(882.53878803,356.39764534)(882.49878906,356.42765076)
\curveto(882.45878811,356.45764528)(882.42878814,356.50764523)(882.40878906,356.57765076)
\curveto(882.38878818,356.64764509)(882.38878818,356.72264502)(882.40878906,356.80265076)
\curveto(882.43878813,356.93264481)(882.4687881,357.05264469)(882.49878906,357.16265076)
\curveto(882.53878803,357.28264446)(882.58378799,357.39764434)(882.63378906,357.50765076)
\curveto(882.82378775,357.85764388)(883.06378751,358.12764361)(883.35378906,358.31765076)
\curveto(883.64378693,358.51764322)(884.00378657,358.67764306)(884.43378906,358.79765076)
\curveto(884.53378604,358.81764292)(884.63378594,358.83264291)(884.73378906,358.84265076)
\curveto(884.84378573,358.85264289)(884.95378562,358.86764287)(885.06378906,358.88765076)
\curveto(885.10378547,358.89764284)(885.1687854,358.89764284)(885.25878906,358.88765076)
\curveto(885.34878522,358.88764285)(885.40378517,358.89764284)(885.42378906,358.91765076)
\curveto(886.12378445,358.92764281)(886.73378384,358.84764289)(887.25378906,358.67765076)
\curveto(887.7737828,358.50764323)(888.13878243,358.18264356)(888.34878906,357.70265076)
\curveto(888.43878213,357.50264424)(888.48878208,357.26764447)(888.49878906,356.99765076)
\curveto(888.51878205,356.737645)(888.52878204,356.46264528)(888.52878906,356.17265076)
\lineto(888.52878906,352.85765076)
\curveto(888.52878204,352.71764902)(888.53378204,352.58264916)(888.54378906,352.45265076)
\curveto(888.55378202,352.32264942)(888.58378199,352.21764952)(888.63378906,352.13765076)
\curveto(888.68378189,352.06764967)(888.74878182,352.01764972)(888.82878906,351.98765076)
\curveto(888.91878165,351.94764979)(889.00378157,351.91764982)(889.08378906,351.89765076)
\curveto(889.16378141,351.88764985)(889.22378135,351.8426499)(889.26378906,351.76265076)
\curveto(889.28378129,351.73265001)(889.29378128,351.70265004)(889.29378906,351.67265076)
\curveto(889.29378128,351.6426501)(889.29878127,351.60265014)(889.30878906,351.55265076)
\moveto(887.16378906,353.21765076)
\curveto(887.22378335,353.35764838)(887.25378332,353.51764822)(887.25378906,353.69765076)
\curveto(887.26378331,353.88764785)(887.2687833,354.08264766)(887.26878906,354.28265076)
\curveto(887.2687833,354.39264735)(887.26378331,354.49264725)(887.25378906,354.58265076)
\curveto(887.24378333,354.67264707)(887.20378337,354.742647)(887.13378906,354.79265076)
\curveto(887.10378347,354.81264693)(887.03378354,354.82264692)(886.92378906,354.82265076)
\curveto(886.90378367,354.80264694)(886.8687837,354.79264695)(886.81878906,354.79265076)
\curveto(886.7687838,354.79264695)(886.72378385,354.78264696)(886.68378906,354.76265076)
\curveto(886.60378397,354.742647)(886.51378406,354.72264702)(886.41378906,354.70265076)
\lineto(886.11378906,354.64265076)
\curveto(886.08378449,354.6426471)(886.04878452,354.6376471)(886.00878906,354.62765076)
\lineto(885.90378906,354.62765076)
\curveto(885.75378482,354.58764715)(885.58878498,354.56264718)(885.40878906,354.55265076)
\curveto(885.23878533,354.55264719)(885.07878549,354.53264721)(884.92878906,354.49265076)
\curveto(884.84878572,354.47264727)(884.7737858,354.45264729)(884.70378906,354.43265076)
\curveto(884.64378593,354.42264732)(884.573786,354.40764733)(884.49378906,354.38765076)
\curveto(884.33378624,354.3376474)(884.18378639,354.27264747)(884.04378906,354.19265076)
\curveto(883.90378667,354.12264762)(883.78378679,354.03264771)(883.68378906,353.92265076)
\curveto(883.58378699,353.81264793)(883.50878706,353.67764806)(883.45878906,353.51765076)
\curveto(883.40878716,353.36764837)(883.38878718,353.18264856)(883.39878906,352.96265076)
\curveto(883.39878717,352.86264888)(883.41378716,352.76764897)(883.44378906,352.67765076)
\curveto(883.48378709,352.59764914)(883.52878704,352.52264922)(883.57878906,352.45265076)
\curveto(883.65878691,352.3426494)(883.76378681,352.24764949)(883.89378906,352.16765076)
\curveto(884.02378655,352.09764964)(884.16378641,352.0376497)(884.31378906,351.98765076)
\curveto(884.36378621,351.97764976)(884.41378616,351.97264977)(884.46378906,351.97265076)
\curveto(884.51378606,351.97264977)(884.56378601,351.96764977)(884.61378906,351.95765076)
\curveto(884.68378589,351.9376498)(884.7687858,351.92264982)(884.86878906,351.91265076)
\curveto(884.97878559,351.91264983)(885.0687855,351.92264982)(885.13878906,351.94265076)
\curveto(885.19878537,351.96264978)(885.25878531,351.96764977)(885.31878906,351.95765076)
\curveto(885.37878519,351.95764978)(885.43878513,351.96764977)(885.49878906,351.98765076)
\curveto(885.57878499,352.00764973)(885.65378492,352.02264972)(885.72378906,352.03265076)
\curveto(885.80378477,352.0426497)(885.87878469,352.06264968)(885.94878906,352.09265076)
\curveto(886.23878433,352.21264953)(886.48378409,352.35764938)(886.68378906,352.52765076)
\curveto(886.89378368,352.69764904)(887.05378352,352.92764881)(887.16378906,353.21765076)
}
}
{
\newrgbcolor{curcolor}{0 0 0}
\pscustom[linestyle=none,fillstyle=solid,fillcolor=curcolor]
{
\newpath
\moveto(893.61542969,358.90265076)
\curveto(894.3554249,358.91264283)(894.97042428,358.80264294)(895.46042969,358.57265076)
\curveto(895.96042329,358.35264339)(896.3554229,358.01764372)(896.64542969,357.56765076)
\curveto(896.77542248,357.36764437)(896.88542237,357.12264462)(896.97542969,356.83265076)
\curveto(896.99542226,356.78264496)(897.01042224,356.71764502)(897.02042969,356.63765076)
\curveto(897.03042222,356.55764518)(897.02542223,356.48764525)(897.00542969,356.42765076)
\curveto(896.97542228,356.37764536)(896.92542233,356.33264541)(896.85542969,356.29265076)
\curveto(896.82542243,356.27264547)(896.79542246,356.26264548)(896.76542969,356.26265076)
\curveto(896.73542252,356.27264547)(896.70042255,356.27264547)(896.66042969,356.26265076)
\curveto(896.62042263,356.25264549)(896.58042267,356.24764549)(896.54042969,356.24765076)
\curveto(896.50042275,356.25764548)(896.46042279,356.26264548)(896.42042969,356.26265076)
\lineto(896.10542969,356.26265076)
\curveto(896.00542325,356.27264547)(895.92042333,356.30264544)(895.85042969,356.35265076)
\curveto(895.77042348,356.41264533)(895.71542354,356.49764524)(895.68542969,356.60765076)
\curveto(895.6554236,356.71764502)(895.61542364,356.81264493)(895.56542969,356.89265076)
\curveto(895.41542384,357.15264459)(895.22042403,357.35764438)(894.98042969,357.50765076)
\curveto(894.90042435,357.55764418)(894.81542444,357.59764414)(894.72542969,357.62765076)
\curveto(894.63542462,357.66764407)(894.54042471,357.70264404)(894.44042969,357.73265076)
\curveto(894.30042495,357.77264397)(894.11542514,357.79264395)(893.88542969,357.79265076)
\curveto(893.6554256,357.80264394)(893.46542579,357.78264396)(893.31542969,357.73265076)
\curveto(893.24542601,357.71264403)(893.18042607,357.69764404)(893.12042969,357.68765076)
\curveto(893.06042619,357.67764406)(892.99542626,357.66264408)(892.92542969,357.64265076)
\curveto(892.66542659,357.53264421)(892.43542682,357.38264436)(892.23542969,357.19265076)
\curveto(892.03542722,357.00264474)(891.88042737,356.77764496)(891.77042969,356.51765076)
\curveto(891.73042752,356.42764531)(891.69542756,356.33264541)(891.66542969,356.23265076)
\curveto(891.63542762,356.1426456)(891.60542765,356.0426457)(891.57542969,355.93265076)
\lineto(891.48542969,355.52765076)
\curveto(891.47542778,355.47764626)(891.47042778,355.42264632)(891.47042969,355.36265076)
\curveto(891.48042777,355.30264644)(891.47542778,355.24764649)(891.45542969,355.19765076)
\lineto(891.45542969,355.07765076)
\curveto(891.44542781,355.0376467)(891.43542782,354.97264677)(891.42542969,354.88265076)
\curveto(891.42542783,354.79264695)(891.43542782,354.72764701)(891.45542969,354.68765076)
\curveto(891.46542779,354.6376471)(891.46542779,354.58764715)(891.45542969,354.53765076)
\curveto(891.44542781,354.48764725)(891.44542781,354.4376473)(891.45542969,354.38765076)
\curveto(891.46542779,354.34764739)(891.47042778,354.27764746)(891.47042969,354.17765076)
\curveto(891.49042776,354.09764764)(891.50542775,354.01264773)(891.51542969,353.92265076)
\curveto(891.53542772,353.83264791)(891.5554277,353.74764799)(891.57542969,353.66765076)
\curveto(891.68542757,353.34764839)(891.81042744,353.06764867)(891.95042969,352.82765076)
\curveto(892.10042715,352.59764914)(892.30542695,352.39764934)(892.56542969,352.22765076)
\curveto(892.6554266,352.17764956)(892.74542651,352.13264961)(892.83542969,352.09265076)
\curveto(892.93542632,352.05264969)(893.04042621,352.01264973)(893.15042969,351.97265076)
\curveto(893.20042605,351.96264978)(893.24042601,351.95764978)(893.27042969,351.95765076)
\curveto(893.30042595,351.95764978)(893.34042591,351.95264979)(893.39042969,351.94265076)
\curveto(893.42042583,351.93264981)(893.47042578,351.92764981)(893.54042969,351.92765076)
\lineto(893.70542969,351.92765076)
\curveto(893.70542555,351.91764982)(893.72542553,351.91264983)(893.76542969,351.91265076)
\curveto(893.78542547,351.92264982)(893.81042544,351.92264982)(893.84042969,351.91265076)
\curveto(893.87042538,351.91264983)(893.90042535,351.91764982)(893.93042969,351.92765076)
\curveto(894.00042525,351.94764979)(894.06542519,351.95264979)(894.12542969,351.94265076)
\curveto(894.19542506,351.9426498)(894.26542499,351.95264979)(894.33542969,351.97265076)
\curveto(894.59542466,352.05264969)(894.82042443,352.15264959)(895.01042969,352.27265076)
\curveto(895.20042405,352.40264934)(895.36042389,352.56764917)(895.49042969,352.76765076)
\curveto(895.54042371,352.84764889)(895.58542367,352.93264881)(895.62542969,353.02265076)
\lineto(895.74542969,353.29265076)
\curveto(895.76542349,353.37264837)(895.78542347,353.44764829)(895.80542969,353.51765076)
\curveto(895.83542342,353.59764814)(895.88542337,353.66264808)(895.95542969,353.71265076)
\curveto(895.98542327,353.742648)(896.04542321,353.76264798)(896.13542969,353.77265076)
\curveto(896.22542303,353.79264795)(896.32042293,353.80264794)(896.42042969,353.80265076)
\curveto(896.53042272,353.81264793)(896.63042262,353.81264793)(896.72042969,353.80265076)
\curveto(896.82042243,353.79264795)(896.89042236,353.77264797)(896.93042969,353.74265076)
\curveto(896.99042226,353.70264804)(897.02542223,353.6426481)(897.03542969,353.56265076)
\curveto(897.0554222,353.48264826)(897.0554222,353.39764834)(897.03542969,353.30765076)
\curveto(896.98542227,353.15764858)(896.93542232,353.01264873)(896.88542969,352.87265076)
\curveto(896.84542241,352.742649)(896.79042246,352.61264913)(896.72042969,352.48265076)
\curveto(896.57042268,352.18264956)(896.38042287,351.91764982)(896.15042969,351.68765076)
\curveto(895.93042332,351.45765028)(895.66042359,351.27265047)(895.34042969,351.13265076)
\curveto(895.26042399,351.09265065)(895.17542408,351.05765068)(895.08542969,351.02765076)
\curveto(894.99542426,351.00765073)(894.90042435,350.98265076)(894.80042969,350.95265076)
\curveto(894.69042456,350.91265083)(894.58042467,350.89265085)(894.47042969,350.89265076)
\curveto(894.36042489,350.88265086)(894.250425,350.86765087)(894.14042969,350.84765076)
\curveto(894.10042515,350.82765091)(894.06042519,350.82265092)(894.02042969,350.83265076)
\curveto(893.98042527,350.8426509)(893.94042531,350.8426509)(893.90042969,350.83265076)
\lineto(893.76542969,350.83265076)
\lineto(893.52542969,350.83265076)
\curveto(893.4554258,350.82265092)(893.39042586,350.82765091)(893.33042969,350.84765076)
\lineto(893.25542969,350.84765076)
\lineto(892.89542969,350.89265076)
\curveto(892.76542649,350.93265081)(892.64042661,350.96765077)(892.52042969,350.99765076)
\curveto(892.40042685,351.02765071)(892.28542697,351.06765067)(892.17542969,351.11765076)
\curveto(891.81542744,351.27765046)(891.51542774,351.46765027)(891.27542969,351.68765076)
\curveto(891.04542821,351.90764983)(890.83042842,352.17764956)(890.63042969,352.49765076)
\curveto(890.58042867,352.57764916)(890.53542872,352.66764907)(890.49542969,352.76765076)
\lineto(890.37542969,353.06765076)
\curveto(890.32542893,353.17764856)(890.29042896,353.29264845)(890.27042969,353.41265076)
\curveto(890.250429,353.53264821)(890.22542903,353.65264809)(890.19542969,353.77265076)
\curveto(890.18542907,353.81264793)(890.18042907,353.85264789)(890.18042969,353.89265076)
\curveto(890.18042907,353.93264781)(890.17542908,353.97264777)(890.16542969,354.01265076)
\curveto(890.14542911,354.07264767)(890.13542912,354.1376476)(890.13542969,354.20765076)
\curveto(890.14542911,354.27764746)(890.14042911,354.3426474)(890.12042969,354.40265076)
\lineto(890.12042969,354.55265076)
\curveto(890.11042914,354.60264714)(890.10542915,354.67264707)(890.10542969,354.76265076)
\curveto(890.10542915,354.85264689)(890.11042914,354.92264682)(890.12042969,354.97265076)
\curveto(890.13042912,355.02264672)(890.13042912,355.06764667)(890.12042969,355.10765076)
\curveto(890.12042913,355.14764659)(890.12542913,355.18764655)(890.13542969,355.22765076)
\curveto(890.1554291,355.29764644)(890.16042909,355.36764637)(890.15042969,355.43765076)
\curveto(890.1504291,355.50764623)(890.16042909,355.57264617)(890.18042969,355.63265076)
\curveto(890.22042903,355.80264594)(890.255429,355.97264577)(890.28542969,356.14265076)
\curveto(890.31542894,356.31264543)(890.36042889,356.47264527)(890.42042969,356.62265076)
\curveto(890.63042862,357.1426446)(890.88542837,357.56264418)(891.18542969,357.88265076)
\curveto(891.48542777,358.20264354)(891.89542736,358.46764327)(892.41542969,358.67765076)
\curveto(892.52542673,358.72764301)(892.64542661,358.76264298)(892.77542969,358.78265076)
\curveto(892.90542635,358.80264294)(893.04042621,358.82764291)(893.18042969,358.85765076)
\curveto(893.250426,358.86764287)(893.32042593,358.87264287)(893.39042969,358.87265076)
\curveto(893.46042579,358.88264286)(893.53542572,358.89264285)(893.61542969,358.90265076)
}
}
{
\newrgbcolor{curcolor}{0 0 0}
\pscustom[linestyle=none,fillstyle=solid,fillcolor=curcolor]
{
\newpath
\moveto(898.82207031,360.22265076)
\curveto(898.74206919,360.28264146)(898.69706924,360.38764135)(898.68707031,360.53765076)
\lineto(898.68707031,361.00265076)
\lineto(898.68707031,361.25765076)
\curveto(898.68706925,361.34764039)(898.70206923,361.42264032)(898.73207031,361.48265076)
\curveto(898.77206916,361.56264018)(898.85206908,361.62264012)(898.97207031,361.66265076)
\curveto(898.99206894,361.67264007)(899.01206892,361.67264007)(899.03207031,361.66265076)
\curveto(899.06206887,361.66264008)(899.08706885,361.66764007)(899.10707031,361.67765076)
\curveto(899.27706866,361.67764006)(899.4370685,361.67264007)(899.58707031,361.66265076)
\curveto(899.7370682,361.65264009)(899.8370681,361.59264015)(899.88707031,361.48265076)
\curveto(899.91706802,361.42264032)(899.932068,361.34764039)(899.93207031,361.25765076)
\lineto(899.93207031,361.00265076)
\curveto(899.932068,360.82264092)(899.92706801,360.65264109)(899.91707031,360.49265076)
\curveto(899.91706802,360.33264141)(899.85206808,360.22764151)(899.72207031,360.17765076)
\curveto(899.67206826,360.15764158)(899.61706832,360.14764159)(899.55707031,360.14765076)
\lineto(899.39207031,360.14765076)
\lineto(899.07707031,360.14765076)
\curveto(898.97706896,360.14764159)(898.89206904,360.17264157)(898.82207031,360.22265076)
\moveto(899.93207031,351.71765076)
\lineto(899.93207031,351.40265076)
\curveto(899.94206799,351.30265044)(899.92206801,351.22265052)(899.87207031,351.16265076)
\curveto(899.84206809,351.10265064)(899.79706814,351.06265068)(899.73707031,351.04265076)
\curveto(899.67706826,351.03265071)(899.60706833,351.01765072)(899.52707031,350.99765076)
\lineto(899.30207031,350.99765076)
\curveto(899.17206876,350.99765074)(899.05706888,351.00265074)(898.95707031,351.01265076)
\curveto(898.86706907,351.03265071)(898.79706914,351.08265066)(898.74707031,351.16265076)
\curveto(898.70706923,351.22265052)(898.68706925,351.29765044)(898.68707031,351.38765076)
\lineto(898.68707031,351.67265076)
\lineto(898.68707031,358.01765076)
\lineto(898.68707031,358.33265076)
\curveto(898.68706925,358.4426433)(898.71206922,358.52764321)(898.76207031,358.58765076)
\curveto(898.79206914,358.6376431)(898.8320691,358.66764307)(898.88207031,358.67765076)
\curveto(898.932069,358.68764305)(898.98706895,358.70264304)(899.04707031,358.72265076)
\curveto(899.06706887,358.72264302)(899.08706885,358.71764302)(899.10707031,358.70765076)
\curveto(899.1370688,358.70764303)(899.16206877,358.71264303)(899.18207031,358.72265076)
\curveto(899.31206862,358.72264302)(899.44206849,358.71764302)(899.57207031,358.70765076)
\curveto(899.71206822,358.70764303)(899.80706813,358.66764307)(899.85707031,358.58765076)
\curveto(899.90706803,358.52764321)(899.932068,358.44764329)(899.93207031,358.34765076)
\lineto(899.93207031,358.06265076)
\lineto(899.93207031,351.71765076)
}
}
{
\newrgbcolor{curcolor}{0 0 0}
\pscustom[linestyle=none,fillstyle=solid,fillcolor=curcolor]
{
\newpath
\moveto(905.95691406,361.91765076)
\curveto(906.026909,361.91763982)(906.11190891,361.91763982)(906.21191406,361.91765076)
\curveto(906.3219087,361.92763981)(906.4219086,361.92763981)(906.51191406,361.91765076)
\curveto(906.61190841,361.91763982)(906.70190832,361.90763983)(906.78191406,361.88765076)
\curveto(906.86190816,361.86763987)(906.91690811,361.8376399)(906.94691406,361.79765076)
\curveto(906.95690807,361.75763998)(906.95190807,361.70264004)(906.93191406,361.63265076)
\curveto(906.91190811,361.57264017)(906.87190815,361.51264023)(906.81191406,361.45265076)
\lineto(906.64691406,361.28765076)
\curveto(906.59690843,361.2376405)(906.54690848,361.18264056)(906.49691406,361.12265076)
\curveto(906.45690857,361.07264067)(906.41190861,361.01764072)(906.36191406,360.95765076)
\curveto(906.33190869,360.90764083)(906.29190873,360.86264088)(906.24191406,360.82265076)
\curveto(906.20190882,360.79264095)(906.16190886,360.75264099)(906.12191406,360.70265076)
\lineto(906.07691406,360.65765076)
\curveto(906.07690895,360.64764109)(906.06690896,360.6376411)(906.04691406,360.62765076)
\curveto(906.00690902,360.57764116)(905.96690906,360.53264121)(905.92691406,360.49265076)
\curveto(905.88690914,360.46264128)(905.84690918,360.42264132)(905.80691406,360.37265076)
\curveto(905.78690924,360.33264141)(905.75690927,360.29764144)(905.71691406,360.26765076)
\lineto(905.62691406,360.17765076)
\curveto(905.58690944,360.12764161)(905.54190948,360.07764166)(905.49191406,360.02765076)
\curveto(905.45190957,359.97764176)(905.40690962,359.9376418)(905.35691406,359.90765076)
\curveto(905.28690974,359.86764187)(905.17190985,359.83264191)(905.01191406,359.80265076)
\curveto(904.86191016,359.78264196)(904.74191028,359.79764194)(904.65191406,359.84765076)
\curveto(904.6219104,359.86764187)(904.59191043,359.89764184)(904.56191406,359.93765076)
\curveto(904.54191048,359.98764175)(904.54191048,360.0426417)(904.56191406,360.10265076)
\curveto(904.58191044,360.18264156)(904.61191041,360.25264149)(904.65191406,360.31265076)
\curveto(904.69191033,360.38264136)(904.73691029,360.44764129)(904.78691406,360.50765076)
\curveto(904.86691016,360.64764109)(904.95191007,360.79264095)(905.04191406,360.94265076)
\curveto(905.13190989,361.09264065)(905.2219098,361.2376405)(905.31191406,361.37765076)
\lineto(905.43191406,361.58765076)
\curveto(905.47190955,361.66764007)(905.5269095,361.73264001)(905.59691406,361.78265076)
\curveto(905.66690936,361.83263991)(905.73690929,361.87263987)(905.80691406,361.90265076)
\curveto(905.83690919,361.90263984)(905.86190916,361.90263984)(905.88191406,361.90265076)
\curveto(905.91190911,361.91263983)(905.93690909,361.91763982)(905.95691406,361.91765076)
\moveto(909.00191406,355.19765076)
\curveto(908.99190603,355.24764649)(908.98690604,355.29764644)(908.98691406,355.34765076)
\curveto(908.99690603,355.40764633)(908.99690603,355.46264628)(908.98691406,355.51265076)
\curveto(908.95690607,355.6426461)(908.93190609,355.76764597)(908.91191406,355.88765076)
\curveto(908.89190613,356.01764572)(908.86690616,356.1376456)(908.83691406,356.24765076)
\curveto(908.79690623,356.35764538)(908.76190626,356.46264528)(908.73191406,356.56265076)
\curveto(908.70190632,356.66264508)(908.66190636,356.76264498)(908.61191406,356.86265076)
\curveto(908.35190667,357.47264427)(907.9269071,357.96764377)(907.33691406,358.34765076)
\curveto(906.74690828,358.72764301)(906.00690902,358.91264283)(905.11691406,358.90265076)
\curveto(905.05690997,358.89264285)(904.99191003,358.88264286)(904.92191406,358.87265076)
\lineto(904.72691406,358.87265076)
\curveto(904.58691044,358.83264291)(904.44691058,358.80264294)(904.30691406,358.78265076)
\curveto(904.16691086,358.77264297)(904.03691099,358.742643)(903.91691406,358.69265076)
\curveto(903.77691125,358.63264311)(903.64191138,358.57264317)(903.51191406,358.51265076)
\curveto(903.38191164,358.46264328)(903.25691177,358.39764334)(903.13691406,358.31765076)
\curveto(902.8269122,358.11764362)(902.56191246,357.86764387)(902.34191406,357.56765076)
\curveto(902.13191289,357.27764446)(901.95191307,356.94764479)(901.80191406,356.57765076)
\curveto(901.75191327,356.46764527)(901.71191331,356.35264539)(901.68191406,356.23265076)
\curveto(901.66191336,356.11264563)(901.63691339,355.99264575)(901.60691406,355.87265076)
\curveto(901.59691343,355.82264592)(901.58691344,355.77764596)(901.57691406,355.73765076)
\curveto(901.57691345,355.70764603)(901.57191345,355.66764607)(901.56191406,355.61765076)
\curveto(901.54191348,355.54764619)(901.53691349,355.47764626)(901.54691406,355.40765076)
\curveto(901.55691347,355.3376464)(901.55191347,355.26764647)(901.53191406,355.19765076)
\curveto(901.51191351,355.1376466)(901.50191352,355.0426467)(901.50191406,354.91265076)
\curveto(901.50191352,354.79264695)(901.50691352,354.70764703)(901.51691406,354.65765076)
\curveto(901.5269135,354.60764713)(901.53191349,354.56264718)(901.53191406,354.52265076)
\lineto(901.53191406,354.40265076)
\curveto(901.55191347,354.32264742)(901.56191346,354.2376475)(901.56191406,354.14765076)
\curveto(901.57191345,354.06764767)(901.58691344,353.98764775)(901.60691406,353.90765076)
\curveto(901.61691341,353.86764787)(901.61691341,353.83264791)(901.60691406,353.80265076)
\curveto(901.60691342,353.78264796)(901.61691341,353.75264799)(901.63691406,353.71265076)
\curveto(901.65691337,353.60264814)(901.67691335,353.49764824)(901.69691406,353.39765076)
\curveto(901.7269133,353.29764844)(901.76691326,353.19764854)(901.81691406,353.09765076)
\curveto(902.00691302,352.6376491)(902.24691278,352.24764949)(902.53691406,351.92765076)
\curveto(902.8269122,351.60765013)(903.19691183,351.34765039)(903.64691406,351.14765076)
\curveto(903.76691126,351.09765064)(903.89191113,351.05265069)(904.02191406,351.01265076)
\curveto(904.15191087,350.97265077)(904.28691074,350.93265081)(904.42691406,350.89265076)
\lineto(904.78691406,350.84765076)
\lineto(904.87691406,350.84765076)
\curveto(904.90691012,350.8376509)(904.94191008,350.8376509)(904.98191406,350.84765076)
\curveto(905.02191,350.84765089)(905.06190996,350.8426509)(905.10191406,350.83265076)
\curveto(905.13190989,350.82265092)(905.18190984,350.81765092)(905.25191406,350.81765076)
\curveto(905.33190969,350.81765092)(905.39190963,350.82265092)(905.43191406,350.83265076)
\curveto(905.49190953,350.83265091)(905.55190947,350.8376509)(905.61191406,350.84765076)
\curveto(905.68190934,350.84765089)(905.74690928,350.85265089)(905.80691406,350.86265076)
\curveto(905.93690909,350.88265086)(906.06190896,350.90265084)(906.18191406,350.92265076)
\curveto(906.31190871,350.9426508)(906.43190859,350.97265077)(906.54191406,351.01265076)
\curveto(907.05190797,351.18265056)(907.48190754,351.42765031)(907.83191406,351.74765076)
\curveto(908.18190684,352.07764966)(908.46190656,352.48264926)(908.67191406,352.96265076)
\curveto(908.7219063,353.07264867)(908.76190626,353.19264855)(908.79191406,353.32265076)
\curveto(908.8219062,353.45264829)(908.85690617,353.58264816)(908.89691406,353.71265076)
\curveto(908.91690611,353.77264797)(908.9269061,353.83264791)(908.92691406,353.89265076)
\curveto(908.93690609,353.95264779)(908.95190607,354.01264773)(908.97191406,354.07265076)
\curveto(908.98190604,354.15264759)(908.98690604,354.22264752)(908.98691406,354.28265076)
\curveto(908.99690603,354.35264739)(909.00690602,354.42764731)(909.01691406,354.50765076)
\lineto(909.01691406,354.65765076)
\curveto(909.026906,354.70764703)(909.03190599,354.79264695)(909.03191406,354.91265076)
\curveto(909.03190599,355.0426467)(909.021906,355.1376466)(909.00191406,355.19765076)
\moveto(907.66691406,354.34265076)
\curveto(907.6269074,354.20264754)(907.59690743,354.06264768)(907.57691406,353.92265076)
\curveto(907.56690746,353.79264795)(907.53690749,353.66764807)(907.48691406,353.54765076)
\curveto(907.34690768,353.20764853)(907.17190785,352.91264883)(906.96191406,352.66265076)
\curveto(906.75190827,352.41264933)(906.47690855,352.21764952)(906.13691406,352.07765076)
\curveto(906.06690896,352.04764969)(905.99190903,352.02264972)(905.91191406,352.00265076)
\curveto(905.83190919,351.99264975)(905.75190927,351.97764976)(905.67191406,351.95765076)
\curveto(905.63190939,351.9376498)(905.59190943,351.92764981)(905.55191406,351.92765076)
\curveto(905.51190951,351.9376498)(905.47190955,351.9376498)(905.43191406,351.92765076)
\curveto(905.38190964,351.90764983)(905.30690972,351.90264984)(905.20691406,351.91265076)
\curveto(905.11690991,351.92264982)(905.05690997,351.93264981)(905.02691406,351.94265076)
\curveto(904.97691005,351.95264979)(904.9269101,351.95264979)(904.87691406,351.94265076)
\curveto(904.83691019,351.9426498)(904.79191023,351.95264979)(904.74191406,351.97265076)
\curveto(904.63191039,352.00264974)(904.5269105,352.03264971)(904.42691406,352.06265076)
\curveto(904.33691069,352.10264964)(904.25191077,352.14764959)(904.17191406,352.19765076)
\curveto(904.1219109,352.22764951)(904.07691095,352.25264949)(904.03691406,352.27265076)
\curveto(903.99691103,352.29264945)(903.95191107,352.31764942)(903.90191406,352.34765076)
\curveto(903.70191132,352.48764925)(903.53191149,352.66264908)(903.39191406,352.87265076)
\curveto(903.26191176,353.08264866)(903.14691188,353.30764843)(903.04691406,353.54765076)
\curveto(903.00691202,353.62764811)(902.97691205,353.71264803)(902.95691406,353.80265076)
\curveto(902.94691208,353.90264784)(902.93191209,354.00264774)(902.91191406,354.10265076)
\lineto(902.88191406,354.28265076)
\curveto(902.86191216,354.36264738)(902.85191217,354.45264729)(902.85191406,354.55265076)
\lineto(902.85191406,354.85265076)
\curveto(902.85191217,354.90264684)(902.84691218,354.94764679)(902.83691406,354.98765076)
\curveto(902.83691219,355.02764671)(902.84191218,355.06264668)(902.85191406,355.09265076)
\curveto(902.85191217,355.18264656)(902.85691217,355.24764649)(902.86691406,355.28765076)
\curveto(902.88691214,355.39764634)(902.90191212,355.50264624)(902.91191406,355.60265076)
\curveto(902.9219121,355.71264603)(902.94191208,355.81764592)(902.97191406,355.91765076)
\curveto(903.08191194,356.2376455)(903.21191181,356.51764522)(903.36191406,356.75765076)
\curveto(903.51191151,356.99764474)(903.70691132,357.20264454)(903.94691406,357.37265076)
\curveto(903.99691103,357.41264433)(904.05191097,357.45264429)(904.11191406,357.49265076)
\curveto(904.17191085,357.53264421)(904.23691079,357.56264418)(904.30691406,357.58265076)
\curveto(904.38691064,357.62264412)(904.46691056,357.65264409)(904.54691406,357.67265076)
\curveto(904.63691039,357.70264404)(904.73191029,357.72764401)(904.83191406,357.74765076)
\lineto(905.10191406,357.79265076)
\lineto(905.37191406,357.79265076)
\curveto(905.46190956,357.79264395)(905.54690948,357.78264396)(905.62691406,357.76265076)
\lineto(905.86691406,357.70265076)
\curveto(905.94690908,357.69264405)(906.021909,357.67264407)(906.09191406,357.64265076)
\curveto(906.70190832,357.39264435)(907.14690788,356.9426448)(907.42691406,356.29265076)
\curveto(907.45690757,356.22264552)(907.48190754,356.14764559)(907.50191406,356.06765076)
\curveto(907.5219075,355.98764575)(907.54690748,355.90764583)(907.57691406,355.82765076)
\curveto(907.64690738,355.55764618)(907.68190734,355.22764651)(907.68191406,354.83765076)
\lineto(907.68191406,354.58265076)
\curveto(907.69190733,354.49264725)(907.68690734,354.41264733)(907.66691406,354.34265076)
}
}
{
\newrgbcolor{curcolor}{0 0 0}
\pscustom[linestyle=none,fillstyle=solid,fillcolor=curcolor]
{
\newpath
\moveto(914.18019531,358.87265076)
\curveto(914.81019008,358.89264285)(915.31518957,358.80764293)(915.69519531,358.61765076)
\curveto(916.07518881,358.42764331)(916.38018851,358.1426436)(916.61019531,357.76265076)
\curveto(916.67018822,357.66264408)(916.71518817,357.55264419)(916.74519531,357.43265076)
\curveto(916.7851881,357.32264442)(916.82018807,357.20764453)(916.85019531,357.08765076)
\curveto(916.90018799,356.89764484)(916.93018796,356.69264505)(916.94019531,356.47265076)
\curveto(916.95018794,356.25264549)(916.95518793,356.02764571)(916.95519531,355.79765076)
\lineto(916.95519531,354.19265076)
\lineto(916.95519531,351.85265076)
\curveto(916.95518793,351.68265006)(916.95018794,351.51265023)(916.94019531,351.34265076)
\curveto(916.94018795,351.17265057)(916.87518801,351.06265068)(916.74519531,351.01265076)
\curveto(916.69518819,350.99265075)(916.64018825,350.98265076)(916.58019531,350.98265076)
\curveto(916.53018836,350.97265077)(916.47518841,350.96765077)(916.41519531,350.96765076)
\curveto(916.2851886,350.96765077)(916.16018873,350.97265077)(916.04019531,350.98265076)
\curveto(915.92018897,350.98265076)(915.83518905,351.02265072)(915.78519531,351.10265076)
\curveto(915.73518915,351.17265057)(915.71018918,351.26265048)(915.71019531,351.37265076)
\lineto(915.71019531,351.70265076)
\lineto(915.71019531,352.99265076)
\lineto(915.71019531,355.43765076)
\curveto(915.71018918,355.70764603)(915.70518918,355.97264577)(915.69519531,356.23265076)
\curveto(915.6851892,356.50264524)(915.64018925,356.73264501)(915.56019531,356.92265076)
\curveto(915.48018941,357.12264462)(915.36018953,357.28264446)(915.20019531,357.40265076)
\curveto(915.04018985,357.53264421)(914.85519003,357.63264411)(914.64519531,357.70265076)
\curveto(914.5851903,357.72264402)(914.52019037,357.73264401)(914.45019531,357.73265076)
\curveto(914.3901905,357.742644)(914.33019056,357.75764398)(914.27019531,357.77765076)
\curveto(914.22019067,357.78764395)(914.14019075,357.78764395)(914.03019531,357.77765076)
\curveto(913.93019096,357.77764396)(913.86019103,357.77264397)(913.82019531,357.76265076)
\curveto(913.78019111,357.742644)(913.74519114,357.73264401)(913.71519531,357.73265076)
\curveto(913.6851912,357.742644)(913.65019124,357.742644)(913.61019531,357.73265076)
\curveto(913.48019141,357.70264404)(913.35519153,357.66764407)(913.23519531,357.62765076)
\curveto(913.12519176,357.59764414)(913.02019187,357.55264419)(912.92019531,357.49265076)
\curveto(912.88019201,357.47264427)(912.84519204,357.45264429)(912.81519531,357.43265076)
\curveto(912.7851921,357.41264433)(912.75019214,357.39264435)(912.71019531,357.37265076)
\curveto(912.36019253,357.12264462)(912.10519278,356.74764499)(911.94519531,356.24765076)
\curveto(911.91519297,356.16764557)(911.89519299,356.08264566)(911.88519531,355.99265076)
\curveto(911.87519301,355.91264583)(911.86019303,355.83264591)(911.84019531,355.75265076)
\curveto(911.82019307,355.70264604)(911.81519307,355.65264609)(911.82519531,355.60265076)
\curveto(911.83519305,355.56264618)(911.83019306,355.52264622)(911.81019531,355.48265076)
\lineto(911.81019531,355.16765076)
\curveto(911.80019309,355.1376466)(911.79519309,355.10264664)(911.79519531,355.06265076)
\curveto(911.80519308,355.02264672)(911.81019308,354.97764676)(911.81019531,354.92765076)
\lineto(911.81019531,354.47765076)
\lineto(911.81019531,353.03765076)
\lineto(911.81019531,351.71765076)
\lineto(911.81019531,351.37265076)
\curveto(911.81019308,351.26265048)(911.7851931,351.17265057)(911.73519531,351.10265076)
\curveto(911.6851932,351.02265072)(911.59519329,350.98265076)(911.46519531,350.98265076)
\curveto(911.34519354,350.97265077)(911.22019367,350.96765077)(911.09019531,350.96765076)
\curveto(911.01019388,350.96765077)(910.93519395,350.97265077)(910.86519531,350.98265076)
\curveto(910.79519409,350.99265075)(910.73519415,351.01765072)(910.68519531,351.05765076)
\curveto(910.60519428,351.10765063)(910.56519432,351.20265054)(910.56519531,351.34265076)
\lineto(910.56519531,351.74765076)
\lineto(910.56519531,353.51765076)
\lineto(910.56519531,357.14765076)
\lineto(910.56519531,358.06265076)
\lineto(910.56519531,358.33265076)
\curveto(910.56519432,358.42264332)(910.5851943,358.49264325)(910.62519531,358.54265076)
\curveto(910.65519423,358.60264314)(910.70519418,358.6426431)(910.77519531,358.66265076)
\curveto(910.81519407,358.67264307)(910.87019402,358.68264306)(910.94019531,358.69265076)
\curveto(911.02019387,358.70264304)(911.10019379,358.70764303)(911.18019531,358.70765076)
\curveto(911.26019363,358.70764303)(911.33519355,358.70264304)(911.40519531,358.69265076)
\curveto(911.4851934,358.68264306)(911.54019335,358.66764307)(911.57019531,358.64765076)
\curveto(911.68019321,358.57764316)(911.73019316,358.48764325)(911.72019531,358.37765076)
\curveto(911.71019318,358.27764346)(911.72519316,358.16264358)(911.76519531,358.03265076)
\curveto(911.7851931,357.97264377)(911.82519306,357.92264382)(911.88519531,357.88265076)
\curveto(912.00519288,357.87264387)(912.10019279,357.91764382)(912.17019531,358.01765076)
\curveto(912.25019264,358.11764362)(912.33019256,358.19764354)(912.41019531,358.25765076)
\curveto(912.55019234,358.35764338)(912.6901922,358.44764329)(912.83019531,358.52765076)
\curveto(912.98019191,358.61764312)(913.15019174,358.69264305)(913.34019531,358.75265076)
\curveto(913.42019147,358.78264296)(913.50519138,358.80264294)(913.59519531,358.81265076)
\curveto(913.69519119,358.82264292)(913.7901911,358.8376429)(913.88019531,358.85765076)
\curveto(913.93019096,358.86764287)(913.98019091,358.87264287)(914.03019531,358.87265076)
\lineto(914.18019531,358.87265076)
}
}
{
\newrgbcolor{curcolor}{0 0 0}
\pscustom[linestyle=none,fillstyle=solid,fillcolor=curcolor]
{
\newpath
\moveto(835.48642334,338.33134644)
\curveto(836.05641818,338.3513355)(836.56141767,338.30633554)(837.00142334,338.19634644)
\curveto(837.44141679,338.08633576)(837.8314164,337.92133593)(838.17142334,337.70134644)
\curveto(838.231416,337.66133619)(838.28641595,337.62633622)(838.33642334,337.59634644)
\curveto(838.39641584,337.56633628)(838.45141578,337.52633632)(838.50142334,337.47634644)
\curveto(838.52141571,337.45633639)(838.54141569,337.43633641)(838.56142334,337.41634644)
\curveto(838.58141565,337.40633644)(838.60141563,337.39133646)(838.62142334,337.37134644)
\curveto(838.64141559,337.3513365)(838.66641557,337.32633652)(838.69642334,337.29634644)
\curveto(838.72641551,337.27633657)(838.75141548,337.2513366)(838.77142334,337.22134644)
\curveto(838.81141542,337.17133668)(838.85141538,337.12133673)(838.89142334,337.07134644)
\curveto(838.9314153,337.02133683)(838.97141526,336.97133688)(839.01142334,336.92134644)
\curveto(839.05141518,336.87133698)(839.08141515,336.81633703)(839.10142334,336.75634644)
\curveto(839.1314151,336.70633714)(839.16141507,336.65633719)(839.19142334,336.60634644)
\curveto(839.28141495,336.46633738)(839.35141488,336.30633754)(839.40142334,336.12634644)
\curveto(839.45141478,335.95633789)(839.50141473,335.77633807)(839.55142334,335.58634644)
\curveto(839.57141466,335.50633834)(839.57641466,335.41633843)(839.56642334,335.31634644)
\curveto(839.56641467,335.22633862)(839.55141468,335.15633869)(839.52142334,335.10634644)
\curveto(839.47141476,335.03633881)(839.39141484,334.99633885)(839.28142334,334.98634644)
\curveto(839.17141506,334.97633887)(839.05641518,334.97133888)(838.93642334,334.97134644)
\lineto(838.80142334,334.97134644)
\curveto(838.76141547,334.97133888)(838.72141551,334.97633887)(838.68142334,334.98634644)
\curveto(838.6314156,334.99633885)(838.58641565,335.00133885)(838.54642334,335.00134644)
\curveto(838.51641572,335.00133885)(838.48141575,335.01133884)(838.44142334,335.03134644)
\curveto(838.42141581,335.04133881)(838.40141583,335.0513388)(838.38142334,335.06134644)
\curveto(838.36141587,335.08133877)(838.34141589,335.10133875)(838.32142334,335.12134644)
\curveto(838.26141597,335.21133864)(838.21641602,335.32133853)(838.18642334,335.45134644)
\curveto(838.16641607,335.58133827)(838.1314161,335.69633815)(838.08142334,335.79634644)
\curveto(837.91141632,336.19633765)(837.67141656,336.49633735)(837.36142334,336.69634644)
\curveto(837.1314171,336.85633699)(836.85641738,336.97633687)(836.53642334,337.05634644)
\curveto(836.47641776,337.07633677)(836.41641782,337.08633676)(836.35642334,337.08634644)
\curveto(836.29641794,337.09633675)(836.236418,337.11133674)(836.17642334,337.13134644)
\lineto(836.05642334,337.13134644)
\curveto(835.97641826,337.1513367)(835.89641834,337.16133669)(835.81642334,337.16134644)
\lineto(835.57642334,337.16134644)
\curveto(835.05641918,337.16133669)(834.62141961,337.08133677)(834.27142334,336.92134644)
\curveto(834.07142016,336.84133701)(833.89142034,336.73133712)(833.73142334,336.59134644)
\curveto(833.57142066,336.46133739)(833.45142078,336.29633755)(833.37142334,336.09634644)
\curveto(833.3314209,336.01633783)(833.30142093,335.93133792)(833.28142334,335.84134644)
\curveto(833.27142096,335.76133809)(833.25642098,335.67133818)(833.23642334,335.57134644)
\curveto(833.21642102,335.49133836)(833.20642103,335.40133845)(833.20642334,335.30134644)
\curveto(833.21642102,335.20133865)(833.231421,335.11633873)(833.25142334,335.04634644)
\curveto(833.32142091,334.81633903)(833.40642083,334.64133921)(833.50642334,334.52134644)
\curveto(833.61642062,334.40133945)(833.76642047,334.28633956)(833.95642334,334.17634644)
\curveto(834.17642006,334.03633981)(834.42141981,333.93133992)(834.69142334,333.86134644)
\curveto(834.96141927,333.79134006)(835.24141899,333.71634013)(835.53142334,333.63634644)
\curveto(835.62141861,333.61634023)(835.71141852,333.60134025)(835.80142334,333.59134644)
\curveto(835.89141834,333.58134027)(835.98141825,333.56134029)(836.07142334,333.53134644)
\curveto(836.21141802,333.49134036)(836.35641788,333.45634039)(836.50642334,333.42634644)
\curveto(836.66641757,333.40634044)(836.81641742,333.37134048)(836.95642334,333.32134644)
\curveto(836.99641724,333.31134054)(837.0314172,333.30134055)(837.06142334,333.29134644)
\curveto(837.09141714,333.29134056)(837.12641711,333.28634056)(837.16642334,333.27634644)
\curveto(837.25641698,333.25634059)(837.34641689,333.23134062)(837.43642334,333.20134644)
\curveto(837.52641671,333.18134067)(837.61641662,333.15634069)(837.70642334,333.12634644)
\curveto(837.90641633,333.05634079)(838.10141613,332.98134087)(838.29142334,332.90134644)
\curveto(838.48141575,332.82134103)(838.66141557,332.72634112)(838.83142334,332.61634644)
\curveto(839.05141518,332.47634137)(839.24641499,332.31134154)(839.41642334,332.12134644)
\curveto(839.59641464,331.94134191)(839.74141449,331.72634212)(839.85142334,331.47634644)
\curveto(839.89141434,331.38634246)(839.92141431,331.29634255)(839.94142334,331.20634644)
\curveto(839.96141427,331.11634273)(839.98141425,331.02134283)(840.00142334,330.92134644)
\curveto(840.02141421,330.87134298)(840.0314142,330.81134304)(840.03142334,330.74134644)
\curveto(840.0314142,330.68134317)(840.04141419,330.62134323)(840.06142334,330.56134644)
\lineto(840.06142334,330.39634644)
\lineto(840.06142334,330.26134644)
\curveto(840.05141418,330.22134363)(840.04641419,330.18134367)(840.04642334,330.14134644)
\lineto(840.04642334,330.02134644)
\curveto(840.02641421,329.94134391)(840.01141422,329.85634399)(840.00142334,329.76634644)
\curveto(840.00141423,329.68634416)(839.98641425,329.60634424)(839.95642334,329.52634644)
\curveto(839.90641433,329.37634447)(839.85141438,329.23134462)(839.79142334,329.09134644)
\curveto(839.7314145,328.9513449)(839.65641458,328.82134503)(839.56642334,328.70134644)
\curveto(839.39641484,328.44134541)(839.18141505,328.22134563)(838.92142334,328.04134644)
\curveto(838.66141557,327.86134599)(838.38141585,327.70134615)(838.08142334,327.56134644)
\curveto(837.98141625,327.52134633)(837.88141635,327.48634636)(837.78142334,327.45634644)
\curveto(837.69141654,327.42634642)(837.59141664,327.39634645)(837.48142334,327.36634644)
\curveto(837.37141686,327.32634652)(837.25641698,327.30134655)(837.13642334,327.29134644)
\curveto(837.01641722,327.27134658)(836.89641734,327.2463466)(836.77642334,327.21634644)
\curveto(836.72641751,327.20634664)(836.67141756,327.20134665)(836.61142334,327.20134644)
\curveto(836.56141767,327.20134665)(836.51141772,327.19634665)(836.46142334,327.18634644)
\lineto(836.28142334,327.18634644)
\curveto(836.22141801,327.17634667)(836.16141807,327.17134668)(836.10142334,327.17134644)
\curveto(836.05141818,327.16134669)(835.97641826,327.15634669)(835.87642334,327.15634644)
\curveto(835.78641845,327.15634669)(835.71641852,327.16134669)(835.66642334,327.17134644)
\lineto(835.50142334,327.17134644)
\curveto(835.40141883,327.19134666)(835.30141893,327.20134665)(835.20142334,327.20134644)
\curveto(835.10141913,327.20134665)(835.00641923,327.21134664)(834.91642334,327.23134644)
\lineto(834.79642334,327.26134644)
\curveto(834.75641948,327.26134659)(834.71141952,327.26634658)(834.66142334,327.27634644)
\curveto(834.55141968,327.29634655)(834.44641979,327.31634653)(834.34642334,327.33634644)
\curveto(834.25641998,327.35634649)(834.16142007,327.38634646)(834.06142334,327.42634644)
\curveto(833.57142066,327.58634626)(833.15642108,327.79134606)(832.81642334,328.04134644)
\curveto(832.48642175,328.29134556)(832.20142203,328.61634523)(831.96142334,329.01634644)
\curveto(831.88142235,329.1463447)(831.81142242,329.28134457)(831.75142334,329.42134644)
\curveto(831.70142253,329.56134429)(831.65142258,329.71134414)(831.60142334,329.87134644)
\curveto(831.58142265,329.93134392)(831.56642267,329.99134386)(831.55642334,330.05134644)
\curveto(831.55642268,330.11134374)(831.54642269,330.16634368)(831.52642334,330.21634644)
\curveto(831.50642273,330.30634354)(831.49642274,330.40634344)(831.49642334,330.51634644)
\curveto(831.49642274,330.63634321)(831.52142271,330.72134313)(831.57142334,330.77134644)
\curveto(831.62142261,330.84134301)(831.70142253,330.88134297)(831.81142334,330.89134644)
\curveto(831.9314223,330.90134295)(832.04642219,330.90634294)(832.15642334,330.90634644)
\lineto(832.27642334,330.90634644)
\curveto(832.32642191,330.90634294)(832.36642187,330.90134295)(832.39642334,330.89134644)
\curveto(832.44642179,330.88134297)(832.48642175,330.87634297)(832.51642334,330.87634644)
\curveto(832.54642169,330.88634296)(832.58142165,330.88134297)(832.62142334,330.86134644)
\curveto(832.69142154,330.82134303)(832.7364215,330.78134307)(832.75642334,330.74134644)
\curveto(832.78642145,330.69134316)(832.80142143,330.63634321)(832.80142334,330.57634644)
\curveto(832.81142142,330.52634332)(832.82642141,330.47134338)(832.84642334,330.41134644)
\curveto(832.86642137,330.34134351)(832.88142135,330.27134358)(832.89142334,330.20134644)
\curveto(832.91142132,330.13134372)(832.9364213,330.06134379)(832.96642334,329.99134644)
\curveto(833.1364211,329.56134429)(833.39142084,329.22134463)(833.73142334,328.97134644)
\curveto(833.90142033,328.84134501)(834.08642015,328.73634511)(834.28642334,328.65634644)
\curveto(834.48641975,328.57634527)(834.70141953,328.50134535)(834.93142334,328.43134644)
\curveto(835.01141922,328.41134544)(835.09141914,328.39634545)(835.17142334,328.38634644)
\lineto(835.41142334,328.35634644)
\curveto(835.4314188,328.35634549)(835.45641878,328.3513455)(835.48642334,328.34134644)
\lineto(835.57642334,328.34134644)
\curveto(835.61641862,328.33134552)(835.66141857,328.33134552)(835.71142334,328.34134644)
\curveto(835.77141846,328.3513455)(835.82641841,328.3463455)(835.87642334,328.32634644)
\curveto(835.91641832,328.31634553)(835.97141826,328.31134554)(836.04142334,328.31134644)
\curveto(836.12141811,328.31134554)(836.18641805,328.32134553)(836.23642334,328.34134644)
\lineto(836.34142334,328.34134644)
\curveto(836.39141784,328.3513455)(836.44141779,328.35634549)(836.49142334,328.35634644)
\curveto(836.54141769,328.35634549)(836.59141764,328.36134549)(836.64142334,328.37134644)
\curveto(836.67141756,328.38134547)(836.69641754,328.38134547)(836.71642334,328.37134644)
\curveto(836.74641749,328.37134548)(836.78141745,328.38134547)(836.82142334,328.40134644)
\curveto(836.90141733,328.42134543)(836.98641725,328.43634541)(837.07642334,328.44634644)
\curveto(837.16641707,328.46634538)(837.25141698,328.49134536)(837.33142334,328.52134644)
\curveto(837.59141664,328.63134522)(837.82141641,328.75634509)(838.02142334,328.89634644)
\curveto(838.231416,329.0463448)(838.39141584,329.2463446)(838.50142334,329.49634644)
\curveto(838.5314157,329.57634427)(838.55641568,329.65634419)(838.57642334,329.73634644)
\lineto(838.63642334,329.97634644)
\curveto(838.65641558,330.05634379)(838.66641557,330.1513437)(838.66642334,330.26134644)
\curveto(838.66641557,330.37134348)(838.65641558,330.46134339)(838.63642334,330.53134644)
\curveto(838.61641562,330.58134327)(838.60641563,330.62134323)(838.60642334,330.65134644)
\curveto(838.60641563,330.69134316)(838.59641564,330.73134312)(838.57642334,330.77134644)
\curveto(838.51641572,330.94134291)(838.4364158,331.08134277)(838.33642334,331.19134644)
\curveto(838.236416,331.31134254)(838.11641612,331.42134243)(837.97642334,331.52134644)
\curveto(837.76641647,331.67134218)(837.5314167,331.79134206)(837.27142334,331.88134644)
\curveto(837.01141722,331.97134188)(836.7364175,332.0463418)(836.44642334,332.10634644)
\curveto(836.16641807,332.17634167)(835.87641836,332.24134161)(835.57642334,332.30134644)
\curveto(835.27641896,332.36134149)(834.98641925,332.43134142)(834.70642334,332.51134644)
\curveto(834.54641969,332.56134129)(834.38641985,332.60134125)(834.22642334,332.63134644)
\curveto(834.06642017,332.67134118)(833.91642032,332.72134113)(833.77642334,332.78134644)
\curveto(833.75642048,332.79134106)(833.74142049,332.79634105)(833.73142334,332.79634644)
\curveto(833.72142051,332.79634105)(833.70642053,332.80134105)(833.68642334,332.81134644)
\lineto(833.38642334,332.93134644)
\curveto(833.28642095,332.97134088)(833.19142104,333.02134083)(833.10142334,333.08134644)
\curveto(832.87142136,333.22134063)(832.66142157,333.37134048)(832.47142334,333.53134644)
\curveto(832.28142195,333.70134015)(832.1314221,333.91633993)(832.02142334,334.17634644)
\curveto(831.97142226,334.26633958)(831.9364223,334.36133949)(831.91642334,334.46134644)
\curveto(831.89642234,334.57133928)(831.87142236,334.68133917)(831.84142334,334.79134644)
\curveto(831.84142239,334.83133902)(831.8364224,334.86633898)(831.82642334,334.89634644)
\lineto(831.82642334,334.98634644)
\curveto(831.79642244,335.10633874)(831.79142244,335.25633859)(831.81142334,335.43634644)
\curveto(831.8314224,335.62633822)(831.85642238,335.77133808)(831.88642334,335.87134644)
\curveto(831.91642232,335.98133787)(831.94142229,336.08633776)(831.96142334,336.18634644)
\curveto(831.99142224,336.28633756)(832.02642221,336.38133747)(832.06642334,336.47134644)
\curveto(832.20642203,336.77133708)(832.38642185,337.02133683)(832.60642334,337.22134644)
\curveto(832.82642141,337.43133642)(833.07142116,337.62133623)(833.34142334,337.79134644)
\curveto(833.36142087,337.81133604)(833.38642085,337.82133603)(833.41642334,337.82134644)
\curveto(833.44642079,337.83133602)(833.47142076,337.846336)(833.49142334,337.86634644)
\curveto(833.71142052,337.97633587)(833.96642027,338.07133578)(834.25642334,338.15134644)
\curveto(834.3364199,338.17133568)(834.41141982,338.18633566)(834.48142334,338.19634644)
\curveto(834.55141968,338.21633563)(834.62641961,338.23633561)(834.70642334,338.25634644)
\curveto(834.78641945,338.27633557)(834.87141936,338.28633556)(834.96142334,338.28634644)
\curveto(835.06141917,338.28633556)(835.15141908,338.29633555)(835.23142334,338.31634644)
\curveto(835.26141897,338.32633552)(835.30641893,338.32633552)(835.36642334,338.31634644)
\curveto(835.42641881,338.30633554)(835.46641877,338.31133554)(835.48642334,338.33134644)
}
}
{
\newrgbcolor{curcolor}{0 0 0}
\pscustom[linestyle=none,fillstyle=solid,fillcolor=curcolor]
{
\newpath
\moveto(848.66298584,331.62634644)
\curveto(848.68297778,331.56634228)(848.69297777,331.47134238)(848.69298584,331.34134644)
\curveto(848.69297777,331.22134263)(848.68797777,331.13634271)(848.67798584,331.08634644)
\lineto(848.67798584,330.93634644)
\curveto(848.66797779,330.85634299)(848.6579778,330.78134307)(848.64798584,330.71134644)
\curveto(848.64797781,330.6513432)(848.64297782,330.58134327)(848.63298584,330.50134644)
\curveto(848.61297785,330.44134341)(848.59797786,330.38134347)(848.58798584,330.32134644)
\curveto(848.58797787,330.26134359)(848.57797788,330.20134365)(848.55798584,330.14134644)
\curveto(848.51797794,330.01134384)(848.48297798,329.88134397)(848.45298584,329.75134644)
\curveto(848.42297804,329.62134423)(848.38297808,329.50134435)(848.33298584,329.39134644)
\curveto(848.12297834,328.91134494)(847.84297862,328.50634534)(847.49298584,328.17634644)
\curveto(847.14297932,327.85634599)(846.71297975,327.61134624)(846.20298584,327.44134644)
\curveto(846.09298037,327.40134645)(845.97298049,327.37134648)(845.84298584,327.35134644)
\curveto(845.72298074,327.33134652)(845.59798086,327.31134654)(845.46798584,327.29134644)
\curveto(845.40798105,327.28134657)(845.34298112,327.27634657)(845.27298584,327.27634644)
\curveto(845.21298125,327.26634658)(845.15298131,327.26134659)(845.09298584,327.26134644)
\curveto(845.05298141,327.2513466)(844.99298147,327.2463466)(844.91298584,327.24634644)
\curveto(844.84298162,327.2463466)(844.79298167,327.2513466)(844.76298584,327.26134644)
\curveto(844.72298174,327.27134658)(844.68298178,327.27634657)(844.64298584,327.27634644)
\curveto(844.60298186,327.26634658)(844.56798189,327.26634658)(844.53798584,327.27634644)
\lineto(844.44798584,327.27634644)
\lineto(844.08798584,327.32134644)
\curveto(843.94798251,327.36134649)(843.81298265,327.40134645)(843.68298584,327.44134644)
\curveto(843.55298291,327.48134637)(843.42798303,327.52634632)(843.30798584,327.57634644)
\curveto(842.8579836,327.77634607)(842.48798397,328.03634581)(842.19798584,328.35634644)
\curveto(841.90798455,328.67634517)(841.66798479,329.06634478)(841.47798584,329.52634644)
\curveto(841.42798503,329.62634422)(841.38798507,329.72634412)(841.35798584,329.82634644)
\curveto(841.33798512,329.92634392)(841.31798514,330.03134382)(841.29798584,330.14134644)
\curveto(841.27798518,330.18134367)(841.26798519,330.21134364)(841.26798584,330.23134644)
\curveto(841.27798518,330.26134359)(841.27798518,330.29634355)(841.26798584,330.33634644)
\curveto(841.24798521,330.41634343)(841.23298523,330.49634335)(841.22298584,330.57634644)
\curveto(841.22298524,330.66634318)(841.21298525,330.7513431)(841.19298584,330.83134644)
\lineto(841.19298584,330.95134644)
\curveto(841.19298527,330.99134286)(841.18798527,331.03634281)(841.17798584,331.08634644)
\curveto(841.16798529,331.13634271)(841.1629853,331.22134263)(841.16298584,331.34134644)
\curveto(841.1629853,331.47134238)(841.17298529,331.56634228)(841.19298584,331.62634644)
\curveto(841.21298525,331.69634215)(841.21798524,331.76634208)(841.20798584,331.83634644)
\curveto(841.19798526,331.90634194)(841.20298526,331.97634187)(841.22298584,332.04634644)
\curveto(841.23298523,332.09634175)(841.23798522,332.13634171)(841.23798584,332.16634644)
\curveto(841.24798521,332.20634164)(841.2579852,332.2513416)(841.26798584,332.30134644)
\curveto(841.29798516,332.42134143)(841.32298514,332.54134131)(841.34298584,332.66134644)
\curveto(841.37298509,332.78134107)(841.41298505,332.89634095)(841.46298584,333.00634644)
\curveto(841.61298485,333.37634047)(841.79298467,333.70634014)(842.00298584,333.99634644)
\curveto(842.22298424,334.29633955)(842.48798397,334.5463393)(842.79798584,334.74634644)
\curveto(842.91798354,334.82633902)(843.04298342,334.89133896)(843.17298584,334.94134644)
\curveto(843.30298316,335.00133885)(843.43798302,335.06133879)(843.57798584,335.12134644)
\curveto(843.69798276,335.17133868)(843.82798263,335.20133865)(843.96798584,335.21134644)
\curveto(844.10798235,335.23133862)(844.24798221,335.26133859)(844.38798584,335.30134644)
\lineto(844.58298584,335.30134644)
\curveto(844.65298181,335.31133854)(844.71798174,335.32133853)(844.77798584,335.33134644)
\curveto(845.66798079,335.34133851)(846.40798005,335.15633869)(846.99798584,334.77634644)
\curveto(847.58797887,334.39633945)(848.01297845,333.90133995)(848.27298584,333.29134644)
\curveto(848.32297814,333.19134066)(848.3629781,333.09134076)(848.39298584,332.99134644)
\curveto(848.42297804,332.89134096)(848.457978,332.78634106)(848.49798584,332.67634644)
\curveto(848.52797793,332.56634128)(848.55297791,332.4463414)(848.57298584,332.31634644)
\curveto(848.59297787,332.19634165)(848.61797784,332.07134178)(848.64798584,331.94134644)
\curveto(848.6579778,331.89134196)(848.6579778,331.83634201)(848.64798584,331.77634644)
\curveto(848.64797781,331.72634212)(848.65297781,331.67634217)(848.66298584,331.62634644)
\moveto(847.32798584,330.77134644)
\curveto(847.34797911,330.84134301)(847.35297911,330.92134293)(847.34298584,331.01134644)
\lineto(847.34298584,331.26634644)
\curveto(847.34297912,331.65634219)(847.30797915,331.98634186)(847.23798584,332.25634644)
\curveto(847.20797925,332.33634151)(847.18297928,332.41634143)(847.16298584,332.49634644)
\curveto(847.14297932,332.57634127)(847.11797934,332.6513412)(847.08798584,332.72134644)
\curveto(846.80797965,333.37134048)(846.3629801,333.82134003)(845.75298584,334.07134644)
\curveto(845.68298078,334.10133975)(845.60798085,334.12133973)(845.52798584,334.13134644)
\lineto(845.28798584,334.19134644)
\curveto(845.20798125,334.21133964)(845.12298134,334.22133963)(845.03298584,334.22134644)
\lineto(844.76298584,334.22134644)
\lineto(844.49298584,334.17634644)
\curveto(844.39298207,334.15633969)(844.29798216,334.13133972)(844.20798584,334.10134644)
\curveto(844.12798233,334.08133977)(844.04798241,334.0513398)(843.96798584,334.01134644)
\curveto(843.89798256,333.99133986)(843.83298263,333.96133989)(843.77298584,333.92134644)
\curveto(843.71298275,333.88133997)(843.6579828,333.84134001)(843.60798584,333.80134644)
\curveto(843.36798309,333.63134022)(843.17298329,333.42634042)(843.02298584,333.18634644)
\curveto(842.87298359,332.9463409)(842.74298372,332.66634118)(842.63298584,332.34634644)
\curveto(842.60298386,332.2463416)(842.58298388,332.14134171)(842.57298584,332.03134644)
\curveto(842.5629839,331.93134192)(842.54798391,331.82634202)(842.52798584,331.71634644)
\curveto(842.51798394,331.67634217)(842.51298395,331.61134224)(842.51298584,331.52134644)
\curveto(842.50298396,331.49134236)(842.49798396,331.45634239)(842.49798584,331.41634644)
\curveto(842.50798395,331.37634247)(842.51298395,331.33134252)(842.51298584,331.28134644)
\lineto(842.51298584,330.98134644)
\curveto(842.51298395,330.88134297)(842.52298394,330.79134306)(842.54298584,330.71134644)
\lineto(842.57298584,330.53134644)
\curveto(842.59298387,330.43134342)(842.60798385,330.33134352)(842.61798584,330.23134644)
\curveto(842.63798382,330.14134371)(842.66798379,330.05634379)(842.70798584,329.97634644)
\curveto(842.80798365,329.73634411)(842.92298354,329.51134434)(843.05298584,329.30134644)
\curveto(843.19298327,329.09134476)(843.3629831,328.91634493)(843.56298584,328.77634644)
\curveto(843.61298285,328.7463451)(843.6579828,328.72134513)(843.69798584,328.70134644)
\curveto(843.73798272,328.68134517)(843.78298268,328.65634519)(843.83298584,328.62634644)
\curveto(843.91298255,328.57634527)(843.99798246,328.53134532)(844.08798584,328.49134644)
\curveto(844.18798227,328.46134539)(844.29298217,328.43134542)(844.40298584,328.40134644)
\curveto(844.45298201,328.38134547)(844.49798196,328.37134548)(844.53798584,328.37134644)
\curveto(844.58798187,328.38134547)(844.63798182,328.38134547)(844.68798584,328.37134644)
\curveto(844.71798174,328.36134549)(844.77798168,328.3513455)(844.86798584,328.34134644)
\curveto(844.96798149,328.33134552)(845.04298142,328.33634551)(845.09298584,328.35634644)
\curveto(845.13298133,328.36634548)(845.17298129,328.36634548)(845.21298584,328.35634644)
\curveto(845.25298121,328.35634549)(845.29298117,328.36634548)(845.33298584,328.38634644)
\curveto(845.41298105,328.40634544)(845.49298097,328.42134543)(845.57298584,328.43134644)
\curveto(845.65298081,328.4513454)(845.72798073,328.47634537)(845.79798584,328.50634644)
\curveto(846.13798032,328.6463452)(846.41298005,328.84134501)(846.62298584,329.09134644)
\curveto(846.83297963,329.34134451)(847.00797945,329.63634421)(847.14798584,329.97634644)
\curveto(847.19797926,330.09634375)(847.22797923,330.22134363)(847.23798584,330.35134644)
\curveto(847.2579792,330.49134336)(847.28797917,330.63134322)(847.32798584,330.77134644)
}
}
{
\newrgbcolor{curcolor}{0 0 0}
\pscustom[linestyle=none,fillstyle=solid,fillcolor=curcolor]
{
\newpath
\moveto(853.28626709,335.33134644)
\curveto(854.0262623,335.34133851)(854.64126168,335.23133862)(855.13126709,335.00134644)
\curveto(855.63126069,334.78133907)(856.0262603,334.4463394)(856.31626709,333.99634644)
\curveto(856.44625988,333.79634005)(856.55625977,333.5513403)(856.64626709,333.26134644)
\curveto(856.66625966,333.21134064)(856.68125964,333.1463407)(856.69126709,333.06634644)
\curveto(856.70125962,332.98634086)(856.69625963,332.91634093)(856.67626709,332.85634644)
\curveto(856.64625968,332.80634104)(856.59625973,332.76134109)(856.52626709,332.72134644)
\curveto(856.49625983,332.70134115)(856.46625986,332.69134116)(856.43626709,332.69134644)
\curveto(856.40625992,332.70134115)(856.37125995,332.70134115)(856.33126709,332.69134644)
\curveto(856.29126003,332.68134117)(856.25126007,332.67634117)(856.21126709,332.67634644)
\curveto(856.17126015,332.68634116)(856.13126019,332.69134116)(856.09126709,332.69134644)
\lineto(855.77626709,332.69134644)
\curveto(855.67626065,332.70134115)(855.59126073,332.73134112)(855.52126709,332.78134644)
\curveto(855.44126088,332.84134101)(855.38626094,332.92634092)(855.35626709,333.03634644)
\curveto(855.326261,333.1463407)(855.28626104,333.24134061)(855.23626709,333.32134644)
\curveto(855.08626124,333.58134027)(854.89126143,333.78634006)(854.65126709,333.93634644)
\curveto(854.57126175,333.98633986)(854.48626184,334.02633982)(854.39626709,334.05634644)
\curveto(854.30626202,334.09633975)(854.21126211,334.13133972)(854.11126709,334.16134644)
\curveto(853.97126235,334.20133965)(853.78626254,334.22133963)(853.55626709,334.22134644)
\curveto(853.326263,334.23133962)(853.13626319,334.21133964)(852.98626709,334.16134644)
\curveto(852.91626341,334.14133971)(852.85126347,334.12633972)(852.79126709,334.11634644)
\curveto(852.73126359,334.10633974)(852.66626366,334.09133976)(852.59626709,334.07134644)
\curveto(852.33626399,333.96133989)(852.10626422,333.81134004)(851.90626709,333.62134644)
\curveto(851.70626462,333.43134042)(851.55126477,333.20634064)(851.44126709,332.94634644)
\curveto(851.40126492,332.85634099)(851.36626496,332.76134109)(851.33626709,332.66134644)
\curveto(851.30626502,332.57134128)(851.27626505,332.47134138)(851.24626709,332.36134644)
\lineto(851.15626709,331.95634644)
\curveto(851.14626518,331.90634194)(851.14126518,331.851342)(851.14126709,331.79134644)
\curveto(851.15126517,331.73134212)(851.14626518,331.67634217)(851.12626709,331.62634644)
\lineto(851.12626709,331.50634644)
\curveto(851.11626521,331.46634238)(851.10626522,331.40134245)(851.09626709,331.31134644)
\curveto(851.09626523,331.22134263)(851.10626522,331.15634269)(851.12626709,331.11634644)
\curveto(851.13626519,331.06634278)(851.13626519,331.01634283)(851.12626709,330.96634644)
\curveto(851.11626521,330.91634293)(851.11626521,330.86634298)(851.12626709,330.81634644)
\curveto(851.13626519,330.77634307)(851.14126518,330.70634314)(851.14126709,330.60634644)
\curveto(851.16126516,330.52634332)(851.17626515,330.44134341)(851.18626709,330.35134644)
\curveto(851.20626512,330.26134359)(851.2262651,330.17634367)(851.24626709,330.09634644)
\curveto(851.35626497,329.77634407)(851.48126484,329.49634435)(851.62126709,329.25634644)
\curveto(851.77126455,329.02634482)(851.97626435,328.82634502)(852.23626709,328.65634644)
\curveto(852.326264,328.60634524)(852.41626391,328.56134529)(852.50626709,328.52134644)
\curveto(852.60626372,328.48134537)(852.71126361,328.44134541)(852.82126709,328.40134644)
\curveto(852.87126345,328.39134546)(852.91126341,328.38634546)(852.94126709,328.38634644)
\curveto(852.97126335,328.38634546)(853.01126331,328.38134547)(853.06126709,328.37134644)
\curveto(853.09126323,328.36134549)(853.14126318,328.35634549)(853.21126709,328.35634644)
\lineto(853.37626709,328.35634644)
\curveto(853.37626295,328.3463455)(853.39626293,328.34134551)(853.43626709,328.34134644)
\curveto(853.45626287,328.3513455)(853.48126284,328.3513455)(853.51126709,328.34134644)
\curveto(853.54126278,328.34134551)(853.57126275,328.3463455)(853.60126709,328.35634644)
\curveto(853.67126265,328.37634547)(853.73626259,328.38134547)(853.79626709,328.37134644)
\curveto(853.86626246,328.37134548)(853.93626239,328.38134547)(854.00626709,328.40134644)
\curveto(854.26626206,328.48134537)(854.49126183,328.58134527)(854.68126709,328.70134644)
\curveto(854.87126145,328.83134502)(855.03126129,328.99634485)(855.16126709,329.19634644)
\curveto(855.21126111,329.27634457)(855.25626107,329.36134449)(855.29626709,329.45134644)
\lineto(855.41626709,329.72134644)
\curveto(855.43626089,329.80134405)(855.45626087,329.87634397)(855.47626709,329.94634644)
\curveto(855.50626082,330.02634382)(855.55626077,330.09134376)(855.62626709,330.14134644)
\curveto(855.65626067,330.17134368)(855.71626061,330.19134366)(855.80626709,330.20134644)
\curveto(855.89626043,330.22134363)(855.99126033,330.23134362)(856.09126709,330.23134644)
\curveto(856.20126012,330.24134361)(856.30126002,330.24134361)(856.39126709,330.23134644)
\curveto(856.49125983,330.22134363)(856.56125976,330.20134365)(856.60126709,330.17134644)
\curveto(856.66125966,330.13134372)(856.69625963,330.07134378)(856.70626709,329.99134644)
\curveto(856.7262596,329.91134394)(856.7262596,329.82634402)(856.70626709,329.73634644)
\curveto(856.65625967,329.58634426)(856.60625972,329.44134441)(856.55626709,329.30134644)
\curveto(856.51625981,329.17134468)(856.46125986,329.04134481)(856.39126709,328.91134644)
\curveto(856.24126008,328.61134524)(856.05126027,328.3463455)(855.82126709,328.11634644)
\curveto(855.60126072,327.88634596)(855.33126099,327.70134615)(855.01126709,327.56134644)
\curveto(854.93126139,327.52134633)(854.84626148,327.48634636)(854.75626709,327.45634644)
\curveto(854.66626166,327.43634641)(854.57126175,327.41134644)(854.47126709,327.38134644)
\curveto(854.36126196,327.34134651)(854.25126207,327.32134653)(854.14126709,327.32134644)
\curveto(854.03126229,327.31134654)(853.9212624,327.29634655)(853.81126709,327.27634644)
\curveto(853.77126255,327.25634659)(853.73126259,327.2513466)(853.69126709,327.26134644)
\curveto(853.65126267,327.27134658)(853.61126271,327.27134658)(853.57126709,327.26134644)
\lineto(853.43626709,327.26134644)
\lineto(853.19626709,327.26134644)
\curveto(853.1262632,327.2513466)(853.06126326,327.25634659)(853.00126709,327.27634644)
\lineto(852.92626709,327.27634644)
\lineto(852.56626709,327.32134644)
\curveto(852.43626389,327.36134649)(852.31126401,327.39634645)(852.19126709,327.42634644)
\curveto(852.07126425,327.45634639)(851.95626437,327.49634635)(851.84626709,327.54634644)
\curveto(851.48626484,327.70634614)(851.18626514,327.89634595)(850.94626709,328.11634644)
\curveto(850.71626561,328.33634551)(850.50126582,328.60634524)(850.30126709,328.92634644)
\curveto(850.25126607,329.00634484)(850.20626612,329.09634475)(850.16626709,329.19634644)
\lineto(850.04626709,329.49634644)
\curveto(849.99626633,329.60634424)(849.96126636,329.72134413)(849.94126709,329.84134644)
\curveto(849.9212664,329.96134389)(849.89626643,330.08134377)(849.86626709,330.20134644)
\curveto(849.85626647,330.24134361)(849.85126647,330.28134357)(849.85126709,330.32134644)
\curveto(849.85126647,330.36134349)(849.84626648,330.40134345)(849.83626709,330.44134644)
\curveto(849.81626651,330.50134335)(849.80626652,330.56634328)(849.80626709,330.63634644)
\curveto(849.81626651,330.70634314)(849.81126651,330.77134308)(849.79126709,330.83134644)
\lineto(849.79126709,330.98134644)
\curveto(849.78126654,331.03134282)(849.77626655,331.10134275)(849.77626709,331.19134644)
\curveto(849.77626655,331.28134257)(849.78126654,331.3513425)(849.79126709,331.40134644)
\curveto(849.80126652,331.4513424)(849.80126652,331.49634235)(849.79126709,331.53634644)
\curveto(849.79126653,331.57634227)(849.79626653,331.61634223)(849.80626709,331.65634644)
\curveto(849.8262665,331.72634212)(849.83126649,331.79634205)(849.82126709,331.86634644)
\curveto(849.8212665,331.93634191)(849.83126649,332.00134185)(849.85126709,332.06134644)
\curveto(849.89126643,332.23134162)(849.9262664,332.40134145)(849.95626709,332.57134644)
\curveto(849.98626634,332.74134111)(850.03126629,332.90134095)(850.09126709,333.05134644)
\curveto(850.30126602,333.57134028)(850.55626577,333.99133986)(850.85626709,334.31134644)
\curveto(851.15626517,334.63133922)(851.56626476,334.89633895)(852.08626709,335.10634644)
\curveto(852.19626413,335.15633869)(852.31626401,335.19133866)(852.44626709,335.21134644)
\curveto(852.57626375,335.23133862)(852.71126361,335.25633859)(852.85126709,335.28634644)
\curveto(852.9212634,335.29633855)(852.99126333,335.30133855)(853.06126709,335.30134644)
\curveto(853.13126319,335.31133854)(853.20626312,335.32133853)(853.28626709,335.33134644)
}
}
{
\newrgbcolor{curcolor}{0 0 0}
\pscustom[linestyle=none,fillstyle=solid,fillcolor=curcolor]
{
\newpath
\moveto(858.49290771,336.65134644)
\curveto(858.41290659,336.71133714)(858.36790664,336.81633703)(858.35790771,336.96634644)
\lineto(858.35790771,337.43134644)
\lineto(858.35790771,337.68634644)
\curveto(858.35790665,337.77633607)(858.37290663,337.851336)(858.40290771,337.91134644)
\curveto(858.44290656,337.99133586)(858.52290648,338.0513358)(858.64290771,338.09134644)
\curveto(858.66290634,338.10133575)(858.68290632,338.10133575)(858.70290771,338.09134644)
\curveto(858.73290627,338.09133576)(858.75790625,338.09633575)(858.77790771,338.10634644)
\curveto(858.94790606,338.10633574)(859.1079059,338.10133575)(859.25790771,338.09134644)
\curveto(859.4079056,338.08133577)(859.5079055,338.02133583)(859.55790771,337.91134644)
\curveto(859.58790542,337.851336)(859.6029054,337.77633607)(859.60290771,337.68634644)
\lineto(859.60290771,337.43134644)
\curveto(859.6029054,337.2513366)(859.59790541,337.08133677)(859.58790771,336.92134644)
\curveto(859.58790542,336.76133709)(859.52290548,336.65633719)(859.39290771,336.60634644)
\curveto(859.34290566,336.58633726)(859.28790572,336.57633727)(859.22790771,336.57634644)
\lineto(859.06290771,336.57634644)
\lineto(858.74790771,336.57634644)
\curveto(858.64790636,336.57633727)(858.56290644,336.60133725)(858.49290771,336.65134644)
\moveto(859.60290771,328.14634644)
\lineto(859.60290771,327.83134644)
\curveto(859.61290539,327.73134612)(859.59290541,327.6513462)(859.54290771,327.59134644)
\curveto(859.51290549,327.53134632)(859.46790554,327.49134636)(859.40790771,327.47134644)
\curveto(859.34790566,327.46134639)(859.27790573,327.4463464)(859.19790771,327.42634644)
\lineto(858.97290771,327.42634644)
\curveto(858.84290616,327.42634642)(858.72790628,327.43134642)(858.62790771,327.44134644)
\curveto(858.53790647,327.46134639)(858.46790654,327.51134634)(858.41790771,327.59134644)
\curveto(858.37790663,327.6513462)(858.35790665,327.72634612)(858.35790771,327.81634644)
\lineto(858.35790771,328.10134644)
\lineto(858.35790771,334.44634644)
\lineto(858.35790771,334.76134644)
\curveto(858.35790665,334.87133898)(858.38290662,334.95633889)(858.43290771,335.01634644)
\curveto(858.46290654,335.06633878)(858.5029065,335.09633875)(858.55290771,335.10634644)
\curveto(858.6029064,335.11633873)(858.65790635,335.13133872)(858.71790771,335.15134644)
\curveto(858.73790627,335.1513387)(858.75790625,335.1463387)(858.77790771,335.13634644)
\curveto(858.8079062,335.13633871)(858.83290617,335.14133871)(858.85290771,335.15134644)
\curveto(858.98290602,335.1513387)(859.11290589,335.1463387)(859.24290771,335.13634644)
\curveto(859.38290562,335.13633871)(859.47790553,335.09633875)(859.52790771,335.01634644)
\curveto(859.57790543,334.95633889)(859.6029054,334.87633897)(859.60290771,334.77634644)
\lineto(859.60290771,334.49134644)
\lineto(859.60290771,328.14634644)
}
}
{
\newrgbcolor{curcolor}{0 0 0}
\pscustom[linestyle=none,fillstyle=solid,fillcolor=curcolor]
{
\newpath
\moveto(868.43275146,327.98134644)
\curveto(868.46274363,327.82134603)(868.44774365,327.68634616)(868.38775146,327.57634644)
\curveto(868.32774377,327.47634637)(868.24774385,327.40134645)(868.14775146,327.35134644)
\curveto(868.097744,327.33134652)(868.04274405,327.32134653)(867.98275146,327.32134644)
\curveto(867.93274416,327.32134653)(867.87774422,327.31134654)(867.81775146,327.29134644)
\curveto(867.5977445,327.24134661)(867.37774472,327.25634659)(867.15775146,327.33634644)
\curveto(866.94774515,327.40634644)(866.80274529,327.49634635)(866.72275146,327.60634644)
\curveto(866.67274542,327.67634617)(866.62774547,327.75634609)(866.58775146,327.84634644)
\curveto(866.54774555,327.9463459)(866.4977456,328.02634582)(866.43775146,328.08634644)
\curveto(866.41774568,328.10634574)(866.3927457,328.12634572)(866.36275146,328.14634644)
\curveto(866.34274575,328.16634568)(866.31274578,328.17134568)(866.27275146,328.16134644)
\curveto(866.16274593,328.13134572)(866.05774604,328.07634577)(865.95775146,327.99634644)
\curveto(865.86774623,327.91634593)(865.77774632,327.846346)(865.68775146,327.78634644)
\curveto(865.55774654,327.70634614)(865.41774668,327.63134622)(865.26775146,327.56134644)
\curveto(865.11774698,327.50134635)(864.95774714,327.4463464)(864.78775146,327.39634644)
\curveto(864.68774741,327.36634648)(864.57774752,327.3463465)(864.45775146,327.33634644)
\curveto(864.34774775,327.32634652)(864.23774786,327.31134654)(864.12775146,327.29134644)
\curveto(864.07774802,327.28134657)(864.03274806,327.27634657)(863.99275146,327.27634644)
\lineto(863.88775146,327.27634644)
\curveto(863.77774832,327.25634659)(863.67274842,327.25634659)(863.57275146,327.27634644)
\lineto(863.43775146,327.27634644)
\curveto(863.38774871,327.28634656)(863.33774876,327.29134656)(863.28775146,327.29134644)
\curveto(863.23774886,327.29134656)(863.1927489,327.30134655)(863.15275146,327.32134644)
\curveto(863.11274898,327.33134652)(863.07774902,327.33634651)(863.04775146,327.33634644)
\curveto(863.02774907,327.32634652)(863.00274909,327.32634652)(862.97275146,327.33634644)
\lineto(862.73275146,327.39634644)
\curveto(862.65274944,327.40634644)(862.57774952,327.42634642)(862.50775146,327.45634644)
\curveto(862.20774989,327.58634626)(861.96275013,327.73134612)(861.77275146,327.89134644)
\curveto(861.5927505,328.06134579)(861.44275065,328.29634555)(861.32275146,328.59634644)
\curveto(861.23275086,328.81634503)(861.18775091,329.08134477)(861.18775146,329.39134644)
\lineto(861.18775146,329.70634644)
\curveto(861.1977509,329.75634409)(861.20275089,329.80634404)(861.20275146,329.85634644)
\lineto(861.23275146,330.03634644)
\lineto(861.35275146,330.36634644)
\curveto(861.3927507,330.47634337)(861.44275065,330.57634327)(861.50275146,330.66634644)
\curveto(861.68275041,330.95634289)(861.92775017,331.17134268)(862.23775146,331.31134644)
\curveto(862.54774955,331.4513424)(862.88774921,331.57634227)(863.25775146,331.68634644)
\curveto(863.3977487,331.72634212)(863.54274855,331.75634209)(863.69275146,331.77634644)
\curveto(863.84274825,331.79634205)(863.9927481,331.82134203)(864.14275146,331.85134644)
\curveto(864.21274788,331.87134198)(864.27774782,331.88134197)(864.33775146,331.88134644)
\curveto(864.40774769,331.88134197)(864.48274761,331.89134196)(864.56275146,331.91134644)
\curveto(864.63274746,331.93134192)(864.70274739,331.94134191)(864.77275146,331.94134644)
\curveto(864.84274725,331.9513419)(864.91774718,331.96634188)(864.99775146,331.98634644)
\curveto(865.24774685,332.0463418)(865.48274661,332.09634175)(865.70275146,332.13634644)
\curveto(865.92274617,332.18634166)(866.097746,332.30134155)(866.22775146,332.48134644)
\curveto(866.28774581,332.56134129)(866.33774576,332.66134119)(866.37775146,332.78134644)
\curveto(866.41774568,332.91134094)(866.41774568,333.0513408)(866.37775146,333.20134644)
\curveto(866.31774578,333.44134041)(866.22774587,333.63134022)(866.10775146,333.77134644)
\curveto(865.9977461,333.91133994)(865.83774626,334.02133983)(865.62775146,334.10134644)
\curveto(865.50774659,334.1513397)(865.36274673,334.18633966)(865.19275146,334.20634644)
\curveto(865.03274706,334.22633962)(864.86274723,334.23633961)(864.68275146,334.23634644)
\curveto(864.50274759,334.23633961)(864.32774777,334.22633962)(864.15775146,334.20634644)
\curveto(863.98774811,334.18633966)(863.84274825,334.15633969)(863.72275146,334.11634644)
\curveto(863.55274854,334.05633979)(863.38774871,333.97133988)(863.22775146,333.86134644)
\curveto(863.14774895,333.80134005)(863.07274902,333.72134013)(863.00275146,333.62134644)
\curveto(862.94274915,333.53134032)(862.88774921,333.43134042)(862.83775146,333.32134644)
\curveto(862.80774929,333.24134061)(862.77774932,333.15634069)(862.74775146,333.06634644)
\curveto(862.72774937,332.97634087)(862.68274941,332.90634094)(862.61275146,332.85634644)
\curveto(862.57274952,332.82634102)(862.50274959,332.80134105)(862.40275146,332.78134644)
\curveto(862.31274978,332.77134108)(862.21774988,332.76634108)(862.11775146,332.76634644)
\curveto(862.01775008,332.76634108)(861.91775018,332.77134108)(861.81775146,332.78134644)
\curveto(861.72775037,332.80134105)(861.66275043,332.82634102)(861.62275146,332.85634644)
\curveto(861.58275051,332.88634096)(861.55275054,332.93634091)(861.53275146,333.00634644)
\curveto(861.51275058,333.07634077)(861.51275058,333.1513407)(861.53275146,333.23134644)
\curveto(861.56275053,333.36134049)(861.5927505,333.48134037)(861.62275146,333.59134644)
\curveto(861.66275043,333.71134014)(861.70775039,333.82634002)(861.75775146,333.93634644)
\curveto(861.94775015,334.28633956)(862.18774991,334.55633929)(862.47775146,334.74634644)
\curveto(862.76774933,334.9463389)(863.12774897,335.10633874)(863.55775146,335.22634644)
\curveto(863.65774844,335.2463386)(863.75774834,335.26133859)(863.85775146,335.27134644)
\curveto(863.96774813,335.28133857)(864.07774802,335.29633855)(864.18775146,335.31634644)
\curveto(864.22774787,335.32633852)(864.2927478,335.32633852)(864.38275146,335.31634644)
\curveto(864.47274762,335.31633853)(864.52774757,335.32633852)(864.54775146,335.34634644)
\curveto(865.24774685,335.35633849)(865.85774624,335.27633857)(866.37775146,335.10634644)
\curveto(866.8977452,334.93633891)(867.26274483,334.61133924)(867.47275146,334.13134644)
\curveto(867.56274453,333.93133992)(867.61274448,333.69634015)(867.62275146,333.42634644)
\curveto(867.64274445,333.16634068)(867.65274444,332.89134096)(867.65275146,332.60134644)
\lineto(867.65275146,329.28634644)
\curveto(867.65274444,329.1463447)(867.65774444,329.01134484)(867.66775146,328.88134644)
\curveto(867.67774442,328.7513451)(867.70774439,328.6463452)(867.75775146,328.56634644)
\curveto(867.80774429,328.49634535)(867.87274422,328.4463454)(867.95275146,328.41634644)
\curveto(868.04274405,328.37634547)(868.12774397,328.3463455)(868.20775146,328.32634644)
\curveto(868.28774381,328.31634553)(868.34774375,328.27134558)(868.38775146,328.19134644)
\curveto(868.40774369,328.16134569)(868.41774368,328.13134572)(868.41775146,328.10134644)
\curveto(868.41774368,328.07134578)(868.42274367,328.03134582)(868.43275146,327.98134644)
\moveto(866.28775146,329.64634644)
\curveto(866.34774575,329.78634406)(866.37774572,329.9463439)(866.37775146,330.12634644)
\curveto(866.38774571,330.31634353)(866.3927457,330.51134334)(866.39275146,330.71134644)
\curveto(866.3927457,330.82134303)(866.38774571,330.92134293)(866.37775146,331.01134644)
\curveto(866.36774573,331.10134275)(866.32774577,331.17134268)(866.25775146,331.22134644)
\curveto(866.22774587,331.24134261)(866.15774594,331.2513426)(866.04775146,331.25134644)
\curveto(866.02774607,331.23134262)(865.9927461,331.22134263)(865.94275146,331.22134644)
\curveto(865.8927462,331.22134263)(865.84774625,331.21134264)(865.80775146,331.19134644)
\curveto(865.72774637,331.17134268)(865.63774646,331.1513427)(865.53775146,331.13134644)
\lineto(865.23775146,331.07134644)
\curveto(865.20774689,331.07134278)(865.17274692,331.06634278)(865.13275146,331.05634644)
\lineto(865.02775146,331.05634644)
\curveto(864.87774722,331.01634283)(864.71274738,330.99134286)(864.53275146,330.98134644)
\curveto(864.36274773,330.98134287)(864.20274789,330.96134289)(864.05275146,330.92134644)
\curveto(863.97274812,330.90134295)(863.8977482,330.88134297)(863.82775146,330.86134644)
\curveto(863.76774833,330.851343)(863.6977484,330.83634301)(863.61775146,330.81634644)
\curveto(863.45774864,330.76634308)(863.30774879,330.70134315)(863.16775146,330.62134644)
\curveto(863.02774907,330.5513433)(862.90774919,330.46134339)(862.80775146,330.35134644)
\curveto(862.70774939,330.24134361)(862.63274946,330.10634374)(862.58275146,329.94634644)
\curveto(862.53274956,329.79634405)(862.51274958,329.61134424)(862.52275146,329.39134644)
\curveto(862.52274957,329.29134456)(862.53774956,329.19634465)(862.56775146,329.10634644)
\curveto(862.60774949,329.02634482)(862.65274944,328.9513449)(862.70275146,328.88134644)
\curveto(862.78274931,328.77134508)(862.88774921,328.67634517)(863.01775146,328.59634644)
\curveto(863.14774895,328.52634532)(863.28774881,328.46634538)(863.43775146,328.41634644)
\curveto(863.48774861,328.40634544)(863.53774856,328.40134545)(863.58775146,328.40134644)
\curveto(863.63774846,328.40134545)(863.68774841,328.39634545)(863.73775146,328.38634644)
\curveto(863.80774829,328.36634548)(863.8927482,328.3513455)(863.99275146,328.34134644)
\curveto(864.10274799,328.34134551)(864.1927479,328.3513455)(864.26275146,328.37134644)
\curveto(864.32274777,328.39134546)(864.38274771,328.39634545)(864.44275146,328.38634644)
\curveto(864.50274759,328.38634546)(864.56274753,328.39634545)(864.62275146,328.41634644)
\curveto(864.70274739,328.43634541)(864.77774732,328.4513454)(864.84775146,328.46134644)
\curveto(864.92774717,328.47134538)(865.00274709,328.49134536)(865.07275146,328.52134644)
\curveto(865.36274673,328.64134521)(865.60774649,328.78634506)(865.80775146,328.95634644)
\curveto(866.01774608,329.12634472)(866.17774592,329.35634449)(866.28775146,329.64634644)
}
}
{
\newrgbcolor{curcolor}{0 0 0}
\pscustom[linestyle=none,fillstyle=solid,fillcolor=curcolor]
{
\newpath
\moveto(877.02939209,331.46134644)
\curveto(877.03938374,331.41134244)(877.04438373,331.34134251)(877.04439209,331.25134644)
\curveto(877.04438373,331.17134268)(877.03938374,331.10634274)(877.02939209,331.05634644)
\curveto(877.02938375,331.01634283)(877.02438375,330.97634287)(877.01439209,330.93634644)
\lineto(877.01439209,330.81634644)
\curveto(876.99438378,330.73634311)(876.98438379,330.65634319)(876.98439209,330.57634644)
\curveto(876.98438379,330.49634335)(876.9743838,330.41634343)(876.95439209,330.33634644)
\curveto(876.94438383,330.29634355)(876.93938384,330.25634359)(876.93939209,330.21634644)
\curveto(876.93938384,330.18634366)(876.93438384,330.1513437)(876.92439209,330.11134644)
\curveto(876.89438388,330.00134385)(876.86438391,329.89634395)(876.83439209,329.79634644)
\curveto(876.81438396,329.69634415)(876.78438399,329.59634425)(876.74439209,329.49634644)
\curveto(876.60438417,329.1463447)(876.43438434,328.83134502)(876.23439209,328.55134644)
\curveto(876.03438474,328.27134558)(875.78438499,328.03134582)(875.48439209,327.83134644)
\curveto(875.33438544,327.73134612)(875.18938559,327.6463462)(875.04939209,327.57634644)
\curveto(874.93938584,327.52634632)(874.82938595,327.48634636)(874.71939209,327.45634644)
\curveto(874.61938616,327.42634642)(874.51438626,327.39634645)(874.40439209,327.36634644)
\curveto(874.33438644,327.3463465)(874.26938651,327.33634651)(874.20939209,327.33634644)
\curveto(874.14938663,327.32634652)(874.08938669,327.31134654)(874.02939209,327.29134644)
\lineto(873.87939209,327.29134644)
\curveto(873.82938695,327.27134658)(873.75438702,327.26134659)(873.65439209,327.26134644)
\curveto(873.55438722,327.2513466)(873.4743873,327.25634659)(873.41439209,327.27634644)
\lineto(873.26439209,327.27634644)
\curveto(873.22438755,327.28634656)(873.1793876,327.29134656)(873.12939209,327.29134644)
\curveto(873.08938769,327.29134656)(873.04438773,327.29634655)(872.99439209,327.30634644)
\curveto(872.84438793,327.3463465)(872.69438808,327.38134647)(872.54439209,327.41134644)
\curveto(872.40438837,327.44134641)(872.26438851,327.48634636)(872.12439209,327.54634644)
\curveto(871.92438885,327.62634622)(871.74438903,327.72634612)(871.58439209,327.84634644)
\lineto(871.40439209,327.99634644)
\curveto(871.34438943,328.05634579)(871.2743895,328.09634575)(871.19439209,328.11634644)
\curveto(871.13438964,328.12634572)(871.08438969,328.11134574)(871.04439209,328.07134644)
\curveto(871.01438976,328.04134581)(870.98938979,327.99634585)(870.96939209,327.93634644)
\curveto(870.95938982,327.87634597)(870.94938983,327.81134604)(870.93939209,327.74134644)
\curveto(870.93938984,327.68134617)(870.92938985,327.63634621)(870.90939209,327.60634644)
\curveto(870.86938991,327.55634629)(870.82438995,327.51134634)(870.77439209,327.47134644)
\curveto(870.72439005,327.4513464)(870.65439012,327.43634641)(870.56439209,327.42634644)
\lineto(870.29439209,327.42634644)
\curveto(870.20439057,327.42634642)(870.11939066,327.43134642)(870.03939209,327.44134644)
\curveto(869.95939082,327.46134639)(869.89939088,327.48134637)(869.85939209,327.50134644)
\curveto(869.83939094,327.52134633)(869.81939096,327.5463463)(869.79939209,327.57634644)
\lineto(869.73939209,327.66634644)
\curveto(869.70939107,327.7463461)(869.69439108,327.86634598)(869.69439209,328.02634644)
\curveto(869.70439107,328.18634566)(869.70939107,328.32134553)(869.70939209,328.43134644)
\lineto(869.70939209,337.23634644)
\curveto(869.70939107,337.35633649)(869.70439107,337.48133637)(869.69439209,337.61134644)
\curveto(869.69439108,337.7513361)(869.71939106,337.86133599)(869.76939209,337.94134644)
\curveto(869.80939097,338.00133585)(869.8743909,338.0513358)(869.96439209,338.09134644)
\curveto(869.98439079,338.09133576)(870.00939077,338.09133576)(870.03939209,338.09134644)
\curveto(870.06939071,338.10133575)(870.09439068,338.10633574)(870.11439209,338.10634644)
\curveto(870.25439052,338.11633573)(870.39939038,338.11633573)(870.54939209,338.10634644)
\curveto(870.70939007,338.10633574)(870.81938996,338.06633578)(870.87939209,337.98634644)
\curveto(870.92938985,337.90633594)(870.95438982,337.79133606)(870.95439209,337.64134644)
\lineto(870.95439209,337.23634644)
\lineto(870.95439209,335.48134644)
\lineto(870.95439209,335.22634644)
\lineto(870.95439209,334.94134644)
\curveto(870.96438981,334.851339)(870.9743898,334.76633908)(870.98439209,334.68634644)
\curveto(871.00438977,334.61633923)(871.03438974,334.56633928)(871.07439209,334.53634644)
\curveto(871.11438966,334.50633934)(871.15938962,334.50133935)(871.20939209,334.52134644)
\curveto(871.25938952,334.54133931)(871.29938948,334.56133929)(871.32939209,334.58134644)
\curveto(871.3793894,334.62133923)(871.42438935,334.66133919)(871.46439209,334.70134644)
\lineto(871.61439209,334.82134644)
\curveto(871.68438909,334.87133898)(871.75438902,334.91633893)(871.82439209,334.95634644)
\lineto(872.06439209,335.07634644)
\curveto(872.24438853,335.16633868)(872.45938832,335.23133862)(872.70939209,335.27134644)
\curveto(872.95938782,335.31133854)(873.21438756,335.33133852)(873.47439209,335.33134644)
\curveto(873.73438704,335.33133852)(873.98938679,335.30633854)(874.23939209,335.25634644)
\curveto(874.48938629,335.21633863)(874.70938607,335.15633869)(874.89939209,335.07634644)
\curveto(875.29938548,334.90633894)(875.64438513,334.67133918)(875.93439209,334.37134644)
\curveto(876.22438455,334.07133978)(876.45438432,333.72134013)(876.62439209,333.32134644)
\curveto(876.6743841,333.21134064)(876.71438406,333.10134075)(876.74439209,332.99134644)
\curveto(876.78438399,332.89134096)(876.82438395,332.78634106)(876.86439209,332.67634644)
\curveto(876.89438388,332.56634128)(876.91438386,332.4513414)(876.92439209,332.33134644)
\lineto(876.98439209,332.00134644)
\curveto(876.99438378,331.97134188)(876.99938378,331.93634191)(876.99939209,331.89634644)
\curveto(876.99938378,331.86634198)(877.00438377,331.83634201)(877.01439209,331.80634644)
\curveto(877.03438374,331.7463421)(877.03438374,331.68634216)(877.01439209,331.62634644)
\curveto(877.00438377,331.57634227)(877.00938377,331.52134233)(877.02939209,331.46134644)
\moveto(875.69439209,331.07134644)
\curveto(875.71438506,331.12134273)(875.71938506,331.18134267)(875.70939209,331.25134644)
\curveto(875.69938508,331.32134253)(875.69438508,331.38634246)(875.69439209,331.44634644)
\curveto(875.69438508,331.61634223)(875.68438509,331.77634207)(875.66439209,331.92634644)
\curveto(875.65438512,332.07634177)(875.62438515,332.21134164)(875.57439209,332.33134644)
\curveto(875.54438523,332.43134142)(875.51938526,332.52134133)(875.49939209,332.60134644)
\curveto(875.4793853,332.68134117)(875.44938533,332.76134109)(875.40939209,332.84134644)
\curveto(875.29938548,333.09134076)(875.14938563,333.32134053)(874.95939209,333.53134644)
\curveto(874.76938601,333.7513401)(874.54938623,333.91633993)(874.29939209,334.02634644)
\curveto(874.21938656,334.05633979)(874.13938664,334.08133977)(874.05939209,334.10134644)
\curveto(873.98938679,334.13133972)(873.91438686,334.15633969)(873.83439209,334.17634644)
\curveto(873.72438705,334.20633964)(873.61438716,334.22133963)(873.50439209,334.22134644)
\curveto(873.39438738,334.23133962)(873.2743875,334.23633961)(873.14439209,334.23634644)
\curveto(873.09438768,334.22633962)(873.04938773,334.21633963)(873.00939209,334.20634644)
\lineto(872.87439209,334.20634644)
\lineto(872.60439209,334.14634644)
\curveto(872.52438825,334.12633972)(872.44438833,334.09633975)(872.36439209,334.05634644)
\curveto(872.02438875,333.91633993)(871.75438902,333.70634014)(871.55439209,333.42634644)
\curveto(871.35438942,333.15634069)(871.19438958,332.83634101)(871.07439209,332.46634644)
\curveto(871.03438974,332.35634149)(871.00938977,332.2463416)(870.99939209,332.13634644)
\curveto(870.98938979,332.02634182)(870.96938981,331.91134194)(870.93939209,331.79134644)
\curveto(870.92938985,331.74134211)(870.92938985,331.69634215)(870.93939209,331.65634644)
\curveto(870.94938983,331.61634223)(870.94438983,331.57134228)(870.92439209,331.52134644)
\curveto(870.91438986,331.47134238)(870.90938987,331.39634245)(870.90939209,331.29634644)
\curveto(870.90938987,331.20634264)(870.91438986,331.13634271)(870.92439209,331.08634644)
\lineto(870.92439209,330.96634644)
\curveto(870.93438984,330.92634292)(870.93938984,330.88634296)(870.93939209,330.84634644)
\curveto(870.93938984,330.80634304)(870.94438983,330.77134308)(870.95439209,330.74134644)
\curveto(870.96438981,330.71134314)(870.96938981,330.67634317)(870.96939209,330.63634644)
\curveto(870.96938981,330.60634324)(870.9743898,330.57634327)(870.98439209,330.54634644)
\curveto(871.00438977,330.46634338)(871.01938976,330.38634346)(871.02939209,330.30634644)
\lineto(871.08939209,330.06634644)
\curveto(871.19938958,329.72634412)(871.34938943,329.43634441)(871.53939209,329.19634644)
\curveto(871.73938904,328.95634489)(871.98438879,328.75634509)(872.27439209,328.59634644)
\curveto(872.36438841,328.5463453)(872.45938832,328.50634534)(872.55939209,328.47634644)
\curveto(872.65938812,328.45634539)(872.76438801,328.43134542)(872.87439209,328.40134644)
\curveto(872.92438785,328.38134547)(872.96938781,328.37134548)(873.00939209,328.37134644)
\curveto(873.05938772,328.38134547)(873.10938767,328.38134547)(873.15939209,328.37134644)
\curveto(873.19938758,328.36134549)(873.24438753,328.35634549)(873.29439209,328.35634644)
\lineto(873.42939209,328.35634644)
\lineto(873.56439209,328.35634644)
\curveto(873.60438717,328.36634548)(873.63938714,328.37134548)(873.66939209,328.37134644)
\curveto(873.69938708,328.37134548)(873.73438704,328.37634547)(873.77439209,328.38634644)
\curveto(873.85438692,328.40634544)(873.92938685,328.42134543)(873.99939209,328.43134644)
\curveto(874.06938671,328.4513454)(874.14438663,328.47634537)(874.22439209,328.50634644)
\curveto(874.53438624,328.63634521)(874.78438599,328.80634504)(874.97439209,329.01634644)
\curveto(875.16438561,329.23634461)(875.32438545,329.50134435)(875.45439209,329.81134644)
\curveto(875.50438527,329.9513439)(875.53938524,330.09134376)(875.55939209,330.23134644)
\curveto(875.58938519,330.38134347)(875.62438515,330.53134332)(875.66439209,330.68134644)
\curveto(875.68438509,330.73134312)(875.68938509,330.77634307)(875.67939209,330.81634644)
\curveto(875.6793851,330.86634298)(875.68438509,330.91634293)(875.69439209,330.96634644)
\lineto(875.69439209,331.07134644)
}
}
{
\newrgbcolor{curcolor}{0 0 0}
\pscustom[linestyle=none,fillstyle=solid,fillcolor=curcolor]
{
\newpath
\moveto(878.79564209,336.65134644)
\curveto(878.71564097,336.71133714)(878.67064101,336.81633703)(878.66064209,336.96634644)
\lineto(878.66064209,337.43134644)
\lineto(878.66064209,337.68634644)
\curveto(878.66064102,337.77633607)(878.67564101,337.851336)(878.70564209,337.91134644)
\curveto(878.74564094,337.99133586)(878.82564086,338.0513358)(878.94564209,338.09134644)
\curveto(878.96564072,338.10133575)(878.9856407,338.10133575)(879.00564209,338.09134644)
\curveto(879.03564065,338.09133576)(879.06064062,338.09633575)(879.08064209,338.10634644)
\curveto(879.25064043,338.10633574)(879.41064027,338.10133575)(879.56064209,338.09134644)
\curveto(879.71063997,338.08133577)(879.81063987,338.02133583)(879.86064209,337.91134644)
\curveto(879.89063979,337.851336)(879.90563978,337.77633607)(879.90564209,337.68634644)
\lineto(879.90564209,337.43134644)
\curveto(879.90563978,337.2513366)(879.90063978,337.08133677)(879.89064209,336.92134644)
\curveto(879.89063979,336.76133709)(879.82563986,336.65633719)(879.69564209,336.60634644)
\curveto(879.64564004,336.58633726)(879.59064009,336.57633727)(879.53064209,336.57634644)
\lineto(879.36564209,336.57634644)
\lineto(879.05064209,336.57634644)
\curveto(878.95064073,336.57633727)(878.86564082,336.60133725)(878.79564209,336.65134644)
\moveto(879.90564209,328.14634644)
\lineto(879.90564209,327.83134644)
\curveto(879.91563977,327.73134612)(879.89563979,327.6513462)(879.84564209,327.59134644)
\curveto(879.81563987,327.53134632)(879.77063991,327.49134636)(879.71064209,327.47134644)
\curveto(879.65064003,327.46134639)(879.5806401,327.4463464)(879.50064209,327.42634644)
\lineto(879.27564209,327.42634644)
\curveto(879.14564054,327.42634642)(879.03064065,327.43134642)(878.93064209,327.44134644)
\curveto(878.84064084,327.46134639)(878.77064091,327.51134634)(878.72064209,327.59134644)
\curveto(878.680641,327.6513462)(878.66064102,327.72634612)(878.66064209,327.81634644)
\lineto(878.66064209,328.10134644)
\lineto(878.66064209,334.44634644)
\lineto(878.66064209,334.76134644)
\curveto(878.66064102,334.87133898)(878.685641,334.95633889)(878.73564209,335.01634644)
\curveto(878.76564092,335.06633878)(878.80564088,335.09633875)(878.85564209,335.10634644)
\curveto(878.90564078,335.11633873)(878.96064072,335.13133872)(879.02064209,335.15134644)
\curveto(879.04064064,335.1513387)(879.06064062,335.1463387)(879.08064209,335.13634644)
\curveto(879.11064057,335.13633871)(879.13564055,335.14133871)(879.15564209,335.15134644)
\curveto(879.2856404,335.1513387)(879.41564027,335.1463387)(879.54564209,335.13634644)
\curveto(879.68564,335.13633871)(879.7806399,335.09633875)(879.83064209,335.01634644)
\curveto(879.8806398,334.95633889)(879.90563978,334.87633897)(879.90564209,334.77634644)
\lineto(879.90564209,334.49134644)
\lineto(879.90564209,328.14634644)
}
}
{
\newrgbcolor{curcolor}{0 0 0}
\pscustom[linestyle=none,fillstyle=solid,fillcolor=curcolor]
{
\newpath
\moveto(882.42048584,338.10634644)
\curveto(882.55048422,338.10633574)(882.68548409,338.10633574)(882.82548584,338.10634644)
\curveto(882.9754838,338.10633574)(883.08548369,338.07133578)(883.15548584,338.00134644)
\curveto(883.20548357,337.93133592)(883.23048354,337.83633601)(883.23048584,337.71634644)
\curveto(883.24048353,337.60633624)(883.24548353,337.49133636)(883.24548584,337.37134644)
\lineto(883.24548584,336.03634644)
\lineto(883.24548584,329.96134644)
\lineto(883.24548584,328.28134644)
\lineto(883.24548584,327.89134644)
\curveto(883.24548353,327.7513461)(883.22048355,327.64134621)(883.17048584,327.56134644)
\curveto(883.14048363,327.51134634)(883.09548368,327.48134637)(883.03548584,327.47134644)
\curveto(882.98548379,327.46134639)(882.92048385,327.4463464)(882.84048584,327.42634644)
\lineto(882.63048584,327.42634644)
\lineto(882.31548584,327.42634644)
\curveto(882.21548456,327.43634641)(882.14048463,327.47134638)(882.09048584,327.53134644)
\curveto(882.04048473,327.61134624)(882.01048476,327.71134614)(882.00048584,327.83134644)
\lineto(882.00048584,328.20634644)
\lineto(882.00048584,329.58634644)
\lineto(882.00048584,335.82634644)
\lineto(882.00048584,337.29634644)
\curveto(882.00048477,337.40633644)(881.99548478,337.52133633)(881.98548584,337.64134644)
\curveto(881.98548479,337.77133608)(882.01048476,337.87133598)(882.06048584,337.94134644)
\curveto(882.10048467,338.00133585)(882.1754846,338.0513358)(882.28548584,338.09134644)
\curveto(882.30548447,338.10133575)(882.32548445,338.10133575)(882.34548584,338.09134644)
\curveto(882.3754844,338.09133576)(882.40048437,338.09633575)(882.42048584,338.10634644)
}
}
{
\newrgbcolor{curcolor}{0 0 0}
\pscustom[linestyle=none,fillstyle=solid,fillcolor=curcolor]
{
\newpath
\moveto(885.47532959,336.65134644)
\curveto(885.39532847,336.71133714)(885.35032851,336.81633703)(885.34032959,336.96634644)
\lineto(885.34032959,337.43134644)
\lineto(885.34032959,337.68634644)
\curveto(885.34032852,337.77633607)(885.35532851,337.851336)(885.38532959,337.91134644)
\curveto(885.42532844,337.99133586)(885.50532836,338.0513358)(885.62532959,338.09134644)
\curveto(885.64532822,338.10133575)(885.6653282,338.10133575)(885.68532959,338.09134644)
\curveto(885.71532815,338.09133576)(885.74032812,338.09633575)(885.76032959,338.10634644)
\curveto(885.93032793,338.10633574)(886.09032777,338.10133575)(886.24032959,338.09134644)
\curveto(886.39032747,338.08133577)(886.49032737,338.02133583)(886.54032959,337.91134644)
\curveto(886.57032729,337.851336)(886.58532728,337.77633607)(886.58532959,337.68634644)
\lineto(886.58532959,337.43134644)
\curveto(886.58532728,337.2513366)(886.58032728,337.08133677)(886.57032959,336.92134644)
\curveto(886.57032729,336.76133709)(886.50532736,336.65633719)(886.37532959,336.60634644)
\curveto(886.32532754,336.58633726)(886.27032759,336.57633727)(886.21032959,336.57634644)
\lineto(886.04532959,336.57634644)
\lineto(885.73032959,336.57634644)
\curveto(885.63032823,336.57633727)(885.54532832,336.60133725)(885.47532959,336.65134644)
\moveto(886.58532959,328.14634644)
\lineto(886.58532959,327.83134644)
\curveto(886.59532727,327.73134612)(886.57532729,327.6513462)(886.52532959,327.59134644)
\curveto(886.49532737,327.53134632)(886.45032741,327.49134636)(886.39032959,327.47134644)
\curveto(886.33032753,327.46134639)(886.2603276,327.4463464)(886.18032959,327.42634644)
\lineto(885.95532959,327.42634644)
\curveto(885.82532804,327.42634642)(885.71032815,327.43134642)(885.61032959,327.44134644)
\curveto(885.52032834,327.46134639)(885.45032841,327.51134634)(885.40032959,327.59134644)
\curveto(885.3603285,327.6513462)(885.34032852,327.72634612)(885.34032959,327.81634644)
\lineto(885.34032959,328.10134644)
\lineto(885.34032959,334.44634644)
\lineto(885.34032959,334.76134644)
\curveto(885.34032852,334.87133898)(885.3653285,334.95633889)(885.41532959,335.01634644)
\curveto(885.44532842,335.06633878)(885.48532838,335.09633875)(885.53532959,335.10634644)
\curveto(885.58532828,335.11633873)(885.64032822,335.13133872)(885.70032959,335.15134644)
\curveto(885.72032814,335.1513387)(885.74032812,335.1463387)(885.76032959,335.13634644)
\curveto(885.79032807,335.13633871)(885.81532805,335.14133871)(885.83532959,335.15134644)
\curveto(885.9653279,335.1513387)(886.09532777,335.1463387)(886.22532959,335.13634644)
\curveto(886.3653275,335.13633871)(886.4603274,335.09633875)(886.51032959,335.01634644)
\curveto(886.5603273,334.95633889)(886.58532728,334.87633897)(886.58532959,334.77634644)
\lineto(886.58532959,334.49134644)
\lineto(886.58532959,328.14634644)
}
}
{
\newrgbcolor{curcolor}{0 0 0}
\pscustom[linestyle=none,fillstyle=solid,fillcolor=curcolor]
{
\newpath
\moveto(895.49017334,328.23634644)
\lineto(895.49017334,327.84634644)
\curveto(895.49016546,327.72634612)(895.46516549,327.62634622)(895.41517334,327.54634644)
\curveto(895.36516559,327.47634637)(895.28016567,327.43634641)(895.16017334,327.42634644)
\lineto(894.81517334,327.42634644)
\curveto(894.7551662,327.42634642)(894.69516626,327.42134643)(894.63517334,327.41134644)
\curveto(894.58516637,327.41134644)(894.54016641,327.42134643)(894.50017334,327.44134644)
\curveto(894.41016654,327.46134639)(894.3501666,327.50134635)(894.32017334,327.56134644)
\curveto(894.28016667,327.61134624)(894.2551667,327.67134618)(894.24517334,327.74134644)
\curveto(894.24516671,327.81134604)(894.23016672,327.88134597)(894.20017334,327.95134644)
\curveto(894.19016676,327.97134588)(894.17516678,327.98634586)(894.15517334,327.99634644)
\curveto(894.14516681,328.01634583)(894.13016682,328.03634581)(894.11017334,328.05634644)
\curveto(894.01016694,328.06634578)(893.93016702,328.0463458)(893.87017334,327.99634644)
\curveto(893.82016713,327.9463459)(893.76516719,327.89634595)(893.70517334,327.84634644)
\curveto(893.50516745,327.69634615)(893.30516765,327.58134627)(893.10517334,327.50134644)
\curveto(892.92516803,327.42134643)(892.71516824,327.36134649)(892.47517334,327.32134644)
\curveto(892.24516871,327.28134657)(892.00516895,327.26134659)(891.75517334,327.26134644)
\curveto(891.51516944,327.2513466)(891.27516968,327.26634658)(891.03517334,327.30634644)
\curveto(890.79517016,327.33634651)(890.58517037,327.39134646)(890.40517334,327.47134644)
\curveto(889.88517107,327.69134616)(889.46517149,327.98634586)(889.14517334,328.35634644)
\curveto(888.82517213,328.73634511)(888.57517238,329.20634464)(888.39517334,329.76634644)
\curveto(888.3551726,329.85634399)(888.32517263,329.9463439)(888.30517334,330.03634644)
\curveto(888.29517266,330.13634371)(888.27517268,330.23634361)(888.24517334,330.33634644)
\curveto(888.23517272,330.38634346)(888.23017272,330.43634341)(888.23017334,330.48634644)
\curveto(888.23017272,330.53634331)(888.22517273,330.58634326)(888.21517334,330.63634644)
\curveto(888.19517276,330.68634316)(888.18517277,330.73634311)(888.18517334,330.78634644)
\curveto(888.19517276,330.846343)(888.19517276,330.90134295)(888.18517334,330.95134644)
\lineto(888.18517334,331.10134644)
\curveto(888.16517279,331.1513427)(888.1551728,331.21634263)(888.15517334,331.29634644)
\curveto(888.1551728,331.37634247)(888.16517279,331.44134241)(888.18517334,331.49134644)
\lineto(888.18517334,331.65634644)
\curveto(888.20517275,331.72634212)(888.21017274,331.79634205)(888.20017334,331.86634644)
\curveto(888.20017275,331.9463419)(888.21017274,332.02134183)(888.23017334,332.09134644)
\curveto(888.24017271,332.14134171)(888.24517271,332.18634166)(888.24517334,332.22634644)
\curveto(888.24517271,332.26634158)(888.2501727,332.31134154)(888.26017334,332.36134644)
\curveto(888.29017266,332.46134139)(888.31517264,332.55634129)(888.33517334,332.64634644)
\curveto(888.3551726,332.7463411)(888.38017257,332.84134101)(888.41017334,332.93134644)
\curveto(888.54017241,333.31134054)(888.70517225,333.6513402)(888.90517334,333.95134644)
\curveto(889.11517184,334.26133959)(889.36517159,334.51633933)(889.65517334,334.71634644)
\curveto(889.82517113,334.83633901)(890.00017095,334.93633891)(890.18017334,335.01634644)
\curveto(890.37017058,335.09633875)(890.57517038,335.16633868)(890.79517334,335.22634644)
\curveto(890.86517009,335.23633861)(890.93017002,335.2463386)(890.99017334,335.25634644)
\curveto(891.06016989,335.26633858)(891.13016982,335.28133857)(891.20017334,335.30134644)
\lineto(891.35017334,335.30134644)
\curveto(891.43016952,335.32133853)(891.54516941,335.33133852)(891.69517334,335.33134644)
\curveto(891.8551691,335.33133852)(891.97516898,335.32133853)(892.05517334,335.30134644)
\curveto(892.09516886,335.29133856)(892.1501688,335.28633856)(892.22017334,335.28634644)
\curveto(892.33016862,335.25633859)(892.44016851,335.23133862)(892.55017334,335.21134644)
\curveto(892.66016829,335.20133865)(892.76516819,335.17133868)(892.86517334,335.12134644)
\curveto(893.01516794,335.06133879)(893.1551678,334.99633885)(893.28517334,334.92634644)
\curveto(893.42516753,334.85633899)(893.5551674,334.77633907)(893.67517334,334.68634644)
\curveto(893.73516722,334.63633921)(893.79516716,334.58133927)(893.85517334,334.52134644)
\curveto(893.92516703,334.47133938)(894.01516694,334.45633939)(894.12517334,334.47634644)
\curveto(894.14516681,334.50633934)(894.16016679,334.53133932)(894.17017334,334.55134644)
\curveto(894.19016676,334.57133928)(894.20516675,334.60133925)(894.21517334,334.64134644)
\curveto(894.24516671,334.73133912)(894.2551667,334.846339)(894.24517334,334.98634644)
\lineto(894.24517334,335.36134644)
\lineto(894.24517334,337.08634644)
\lineto(894.24517334,337.55134644)
\curveto(894.24516671,337.73133612)(894.27016668,337.86133599)(894.32017334,337.94134644)
\curveto(894.36016659,338.01133584)(894.42016653,338.05633579)(894.50017334,338.07634644)
\curveto(894.52016643,338.07633577)(894.54516641,338.07633577)(894.57517334,338.07634644)
\curveto(894.60516635,338.08633576)(894.63016632,338.09133576)(894.65017334,338.09134644)
\curveto(894.79016616,338.10133575)(894.93516602,338.10133575)(895.08517334,338.09134644)
\curveto(895.24516571,338.09133576)(895.3551656,338.0513358)(895.41517334,337.97134644)
\curveto(895.46516549,337.89133596)(895.49016546,337.79133606)(895.49017334,337.67134644)
\lineto(895.49017334,337.29634644)
\lineto(895.49017334,328.23634644)
\moveto(894.27517334,331.07134644)
\curveto(894.29516666,331.12134273)(894.30516665,331.18634266)(894.30517334,331.26634644)
\curveto(894.30516665,331.35634249)(894.29516666,331.42634242)(894.27517334,331.47634644)
\lineto(894.27517334,331.70134644)
\curveto(894.2551667,331.79134206)(894.24016671,331.88134197)(894.23017334,331.97134644)
\curveto(894.22016673,332.07134178)(894.20016675,332.16134169)(894.17017334,332.24134644)
\curveto(894.1501668,332.32134153)(894.13016682,332.39634145)(894.11017334,332.46634644)
\curveto(894.10016685,332.53634131)(894.08016687,332.60634124)(894.05017334,332.67634644)
\curveto(893.93016702,332.97634087)(893.77516718,333.24134061)(893.58517334,333.47134644)
\curveto(893.39516756,333.70134015)(893.1551678,333.88133997)(892.86517334,334.01134644)
\curveto(892.76516819,334.06133979)(892.66016829,334.09633975)(892.55017334,334.11634644)
\curveto(892.4501685,334.1463397)(892.34016861,334.17133968)(892.22017334,334.19134644)
\curveto(892.14016881,334.21133964)(892.0501689,334.22133963)(891.95017334,334.22134644)
\lineto(891.68017334,334.22134644)
\curveto(891.63016932,334.21133964)(891.58516937,334.20133965)(891.54517334,334.19134644)
\lineto(891.41017334,334.19134644)
\curveto(891.33016962,334.17133968)(891.24516971,334.1513397)(891.15517334,334.13134644)
\curveto(891.07516988,334.11133974)(890.99516996,334.08633976)(890.91517334,334.05634644)
\curveto(890.59517036,333.91633993)(890.33517062,333.71134014)(890.13517334,333.44134644)
\curveto(889.94517101,333.18134067)(889.79017116,332.87634097)(889.67017334,332.52634644)
\curveto(889.63017132,332.41634143)(889.60017135,332.30134155)(889.58017334,332.18134644)
\curveto(889.57017138,332.07134178)(889.5551714,331.96134189)(889.53517334,331.85134644)
\curveto(889.53517142,331.81134204)(889.53017142,331.77134208)(889.52017334,331.73134644)
\lineto(889.52017334,331.62634644)
\curveto(889.50017145,331.57634227)(889.49017146,331.52134233)(889.49017334,331.46134644)
\curveto(889.50017145,331.40134245)(889.50517145,331.3463425)(889.50517334,331.29634644)
\lineto(889.50517334,330.96634644)
\curveto(889.50517145,330.86634298)(889.51517144,330.77134308)(889.53517334,330.68134644)
\curveto(889.54517141,330.6513432)(889.5501714,330.60134325)(889.55017334,330.53134644)
\curveto(889.57017138,330.46134339)(889.58517137,330.39134346)(889.59517334,330.32134644)
\lineto(889.65517334,330.11134644)
\curveto(889.76517119,329.76134409)(889.91517104,329.46134439)(890.10517334,329.21134644)
\curveto(890.29517066,328.96134489)(890.53517042,328.75634509)(890.82517334,328.59634644)
\curveto(890.91517004,328.5463453)(891.00516995,328.50634534)(891.09517334,328.47634644)
\curveto(891.18516977,328.4463454)(891.28516967,328.41634543)(891.39517334,328.38634644)
\curveto(891.44516951,328.36634548)(891.49516946,328.36134549)(891.54517334,328.37134644)
\curveto(891.60516935,328.38134547)(891.66016929,328.37634547)(891.71017334,328.35634644)
\curveto(891.7501692,328.3463455)(891.79016916,328.34134551)(891.83017334,328.34134644)
\lineto(891.96517334,328.34134644)
\lineto(892.10017334,328.34134644)
\curveto(892.13016882,328.3513455)(892.18016877,328.35634549)(892.25017334,328.35634644)
\curveto(892.33016862,328.37634547)(892.41016854,328.39134546)(892.49017334,328.40134644)
\curveto(892.57016838,328.42134543)(892.64516831,328.4463454)(892.71517334,328.47634644)
\curveto(893.04516791,328.61634523)(893.31016764,328.79134506)(893.51017334,329.00134644)
\curveto(893.72016723,329.22134463)(893.89516706,329.49634435)(894.03517334,329.82634644)
\curveto(894.08516687,329.93634391)(894.12016683,330.0463438)(894.14017334,330.15634644)
\curveto(894.16016679,330.26634358)(894.18516677,330.37634347)(894.21517334,330.48634644)
\curveto(894.23516672,330.52634332)(894.24516671,330.56134329)(894.24517334,330.59134644)
\curveto(894.24516671,330.63134322)(894.2501667,330.67134318)(894.26017334,330.71134644)
\curveto(894.27016668,330.77134308)(894.27016668,330.83134302)(894.26017334,330.89134644)
\curveto(894.26016669,330.9513429)(894.26516669,331.01134284)(894.27517334,331.07134644)
}
}
{
\newrgbcolor{curcolor}{0 0 0}
\pscustom[linestyle=none,fillstyle=solid,fillcolor=curcolor]
{
\newpath
\moveto(904.32142334,327.98134644)
\curveto(904.35141551,327.82134603)(904.33641552,327.68634616)(904.27642334,327.57634644)
\curveto(904.21641564,327.47634637)(904.13641572,327.40134645)(904.03642334,327.35134644)
\curveto(903.98641587,327.33134652)(903.93141593,327.32134653)(903.87142334,327.32134644)
\curveto(903.82141604,327.32134653)(903.76641609,327.31134654)(903.70642334,327.29134644)
\curveto(903.48641637,327.24134661)(903.26641659,327.25634659)(903.04642334,327.33634644)
\curveto(902.83641702,327.40634644)(902.69141717,327.49634635)(902.61142334,327.60634644)
\curveto(902.5614173,327.67634617)(902.51641734,327.75634609)(902.47642334,327.84634644)
\curveto(902.43641742,327.9463459)(902.38641747,328.02634582)(902.32642334,328.08634644)
\curveto(902.30641755,328.10634574)(902.28141758,328.12634572)(902.25142334,328.14634644)
\curveto(902.23141763,328.16634568)(902.20141766,328.17134568)(902.16142334,328.16134644)
\curveto(902.05141781,328.13134572)(901.94641791,328.07634577)(901.84642334,327.99634644)
\curveto(901.7564181,327.91634593)(901.66641819,327.846346)(901.57642334,327.78634644)
\curveto(901.44641841,327.70634614)(901.30641855,327.63134622)(901.15642334,327.56134644)
\curveto(901.00641885,327.50134635)(900.84641901,327.4463464)(900.67642334,327.39634644)
\curveto(900.57641928,327.36634648)(900.46641939,327.3463465)(900.34642334,327.33634644)
\curveto(900.23641962,327.32634652)(900.12641973,327.31134654)(900.01642334,327.29134644)
\curveto(899.96641989,327.28134657)(899.92141994,327.27634657)(899.88142334,327.27634644)
\lineto(899.77642334,327.27634644)
\curveto(899.66642019,327.25634659)(899.5614203,327.25634659)(899.46142334,327.27634644)
\lineto(899.32642334,327.27634644)
\curveto(899.27642058,327.28634656)(899.22642063,327.29134656)(899.17642334,327.29134644)
\curveto(899.12642073,327.29134656)(899.08142078,327.30134655)(899.04142334,327.32134644)
\curveto(899.00142086,327.33134652)(898.96642089,327.33634651)(898.93642334,327.33634644)
\curveto(898.91642094,327.32634652)(898.89142097,327.32634652)(898.86142334,327.33634644)
\lineto(898.62142334,327.39634644)
\curveto(898.54142132,327.40634644)(898.46642139,327.42634642)(898.39642334,327.45634644)
\curveto(898.09642176,327.58634626)(897.85142201,327.73134612)(897.66142334,327.89134644)
\curveto(897.48142238,328.06134579)(897.33142253,328.29634555)(897.21142334,328.59634644)
\curveto(897.12142274,328.81634503)(897.07642278,329.08134477)(897.07642334,329.39134644)
\lineto(897.07642334,329.70634644)
\curveto(897.08642277,329.75634409)(897.09142277,329.80634404)(897.09142334,329.85634644)
\lineto(897.12142334,330.03634644)
\lineto(897.24142334,330.36634644)
\curveto(897.28142258,330.47634337)(897.33142253,330.57634327)(897.39142334,330.66634644)
\curveto(897.57142229,330.95634289)(897.81642204,331.17134268)(898.12642334,331.31134644)
\curveto(898.43642142,331.4513424)(898.77642108,331.57634227)(899.14642334,331.68634644)
\curveto(899.28642057,331.72634212)(899.43142043,331.75634209)(899.58142334,331.77634644)
\curveto(899.73142013,331.79634205)(899.88141998,331.82134203)(900.03142334,331.85134644)
\curveto(900.10141976,331.87134198)(900.16641969,331.88134197)(900.22642334,331.88134644)
\curveto(900.29641956,331.88134197)(900.37141949,331.89134196)(900.45142334,331.91134644)
\curveto(900.52141934,331.93134192)(900.59141927,331.94134191)(900.66142334,331.94134644)
\curveto(900.73141913,331.9513419)(900.80641905,331.96634188)(900.88642334,331.98634644)
\curveto(901.13641872,332.0463418)(901.37141849,332.09634175)(901.59142334,332.13634644)
\curveto(901.81141805,332.18634166)(901.98641787,332.30134155)(902.11642334,332.48134644)
\curveto(902.17641768,332.56134129)(902.22641763,332.66134119)(902.26642334,332.78134644)
\curveto(902.30641755,332.91134094)(902.30641755,333.0513408)(902.26642334,333.20134644)
\curveto(902.20641765,333.44134041)(902.11641774,333.63134022)(901.99642334,333.77134644)
\curveto(901.88641797,333.91133994)(901.72641813,334.02133983)(901.51642334,334.10134644)
\curveto(901.39641846,334.1513397)(901.25141861,334.18633966)(901.08142334,334.20634644)
\curveto(900.92141894,334.22633962)(900.75141911,334.23633961)(900.57142334,334.23634644)
\curveto(900.39141947,334.23633961)(900.21641964,334.22633962)(900.04642334,334.20634644)
\curveto(899.87641998,334.18633966)(899.73142013,334.15633969)(899.61142334,334.11634644)
\curveto(899.44142042,334.05633979)(899.27642058,333.97133988)(899.11642334,333.86134644)
\curveto(899.03642082,333.80134005)(898.9614209,333.72134013)(898.89142334,333.62134644)
\curveto(898.83142103,333.53134032)(898.77642108,333.43134042)(898.72642334,333.32134644)
\curveto(898.69642116,333.24134061)(898.66642119,333.15634069)(898.63642334,333.06634644)
\curveto(898.61642124,332.97634087)(898.57142129,332.90634094)(898.50142334,332.85634644)
\curveto(898.4614214,332.82634102)(898.39142147,332.80134105)(898.29142334,332.78134644)
\curveto(898.20142166,332.77134108)(898.10642175,332.76634108)(898.00642334,332.76634644)
\curveto(897.90642195,332.76634108)(897.80642205,332.77134108)(897.70642334,332.78134644)
\curveto(897.61642224,332.80134105)(897.55142231,332.82634102)(897.51142334,332.85634644)
\curveto(897.47142239,332.88634096)(897.44142242,332.93634091)(897.42142334,333.00634644)
\curveto(897.40142246,333.07634077)(897.40142246,333.1513407)(897.42142334,333.23134644)
\curveto(897.45142241,333.36134049)(897.48142238,333.48134037)(897.51142334,333.59134644)
\curveto(897.55142231,333.71134014)(897.59642226,333.82634002)(897.64642334,333.93634644)
\curveto(897.83642202,334.28633956)(898.07642178,334.55633929)(898.36642334,334.74634644)
\curveto(898.6564212,334.9463389)(899.01642084,335.10633874)(899.44642334,335.22634644)
\curveto(899.54642031,335.2463386)(899.64642021,335.26133859)(899.74642334,335.27134644)
\curveto(899.85642,335.28133857)(899.96641989,335.29633855)(900.07642334,335.31634644)
\curveto(900.11641974,335.32633852)(900.18141968,335.32633852)(900.27142334,335.31634644)
\curveto(900.3614195,335.31633853)(900.41641944,335.32633852)(900.43642334,335.34634644)
\curveto(901.13641872,335.35633849)(901.74641811,335.27633857)(902.26642334,335.10634644)
\curveto(902.78641707,334.93633891)(903.15141671,334.61133924)(903.36142334,334.13134644)
\curveto(903.45141641,333.93133992)(903.50141636,333.69634015)(903.51142334,333.42634644)
\curveto(903.53141633,333.16634068)(903.54141632,332.89134096)(903.54142334,332.60134644)
\lineto(903.54142334,329.28634644)
\curveto(903.54141632,329.1463447)(903.54641631,329.01134484)(903.55642334,328.88134644)
\curveto(903.56641629,328.7513451)(903.59641626,328.6463452)(903.64642334,328.56634644)
\curveto(903.69641616,328.49634535)(903.7614161,328.4463454)(903.84142334,328.41634644)
\curveto(903.93141593,328.37634547)(904.01641584,328.3463455)(904.09642334,328.32634644)
\curveto(904.17641568,328.31634553)(904.23641562,328.27134558)(904.27642334,328.19134644)
\curveto(904.29641556,328.16134569)(904.30641555,328.13134572)(904.30642334,328.10134644)
\curveto(904.30641555,328.07134578)(904.31141555,328.03134582)(904.32142334,327.98134644)
\moveto(902.17642334,329.64634644)
\curveto(902.23641762,329.78634406)(902.26641759,329.9463439)(902.26642334,330.12634644)
\curveto(902.27641758,330.31634353)(902.28141758,330.51134334)(902.28142334,330.71134644)
\curveto(902.28141758,330.82134303)(902.27641758,330.92134293)(902.26642334,331.01134644)
\curveto(902.2564176,331.10134275)(902.21641764,331.17134268)(902.14642334,331.22134644)
\curveto(902.11641774,331.24134261)(902.04641781,331.2513426)(901.93642334,331.25134644)
\curveto(901.91641794,331.23134262)(901.88141798,331.22134263)(901.83142334,331.22134644)
\curveto(901.78141808,331.22134263)(901.73641812,331.21134264)(901.69642334,331.19134644)
\curveto(901.61641824,331.17134268)(901.52641833,331.1513427)(901.42642334,331.13134644)
\lineto(901.12642334,331.07134644)
\curveto(901.09641876,331.07134278)(901.0614188,331.06634278)(901.02142334,331.05634644)
\lineto(900.91642334,331.05634644)
\curveto(900.76641909,331.01634283)(900.60141926,330.99134286)(900.42142334,330.98134644)
\curveto(900.25141961,330.98134287)(900.09141977,330.96134289)(899.94142334,330.92134644)
\curveto(899.86142,330.90134295)(899.78642007,330.88134297)(899.71642334,330.86134644)
\curveto(899.6564202,330.851343)(899.58642027,330.83634301)(899.50642334,330.81634644)
\curveto(899.34642051,330.76634308)(899.19642066,330.70134315)(899.05642334,330.62134644)
\curveto(898.91642094,330.5513433)(898.79642106,330.46134339)(898.69642334,330.35134644)
\curveto(898.59642126,330.24134361)(898.52142134,330.10634374)(898.47142334,329.94634644)
\curveto(898.42142144,329.79634405)(898.40142146,329.61134424)(898.41142334,329.39134644)
\curveto(898.41142145,329.29134456)(898.42642143,329.19634465)(898.45642334,329.10634644)
\curveto(898.49642136,329.02634482)(898.54142132,328.9513449)(898.59142334,328.88134644)
\curveto(898.67142119,328.77134508)(898.77642108,328.67634517)(898.90642334,328.59634644)
\curveto(899.03642082,328.52634532)(899.17642068,328.46634538)(899.32642334,328.41634644)
\curveto(899.37642048,328.40634544)(899.42642043,328.40134545)(899.47642334,328.40134644)
\curveto(899.52642033,328.40134545)(899.57642028,328.39634545)(899.62642334,328.38634644)
\curveto(899.69642016,328.36634548)(899.78142008,328.3513455)(899.88142334,328.34134644)
\curveto(899.99141987,328.34134551)(900.08141978,328.3513455)(900.15142334,328.37134644)
\curveto(900.21141965,328.39134546)(900.27141959,328.39634545)(900.33142334,328.38634644)
\curveto(900.39141947,328.38634546)(900.45141941,328.39634545)(900.51142334,328.41634644)
\curveto(900.59141927,328.43634541)(900.66641919,328.4513454)(900.73642334,328.46134644)
\curveto(900.81641904,328.47134538)(900.89141897,328.49134536)(900.96142334,328.52134644)
\curveto(901.25141861,328.64134521)(901.49641836,328.78634506)(901.69642334,328.95634644)
\curveto(901.90641795,329.12634472)(902.06641779,329.35634449)(902.17642334,329.64634644)
}
}
{
\newrgbcolor{curcolor}{0 0 0}
\pscustom[linestyle=none,fillstyle=solid,fillcolor=curcolor]
{
\newpath
\moveto(912.45306396,328.23634644)
\lineto(912.45306396,327.84634644)
\curveto(912.45305609,327.72634612)(912.42805611,327.62634622)(912.37806396,327.54634644)
\curveto(912.32805621,327.47634637)(912.2430563,327.43634641)(912.12306396,327.42634644)
\lineto(911.77806396,327.42634644)
\curveto(911.71805682,327.42634642)(911.65805688,327.42134643)(911.59806396,327.41134644)
\curveto(911.54805699,327.41134644)(911.50305704,327.42134643)(911.46306396,327.44134644)
\curveto(911.37305717,327.46134639)(911.31305723,327.50134635)(911.28306396,327.56134644)
\curveto(911.2430573,327.61134624)(911.21805732,327.67134618)(911.20806396,327.74134644)
\curveto(911.20805733,327.81134604)(911.19305735,327.88134597)(911.16306396,327.95134644)
\curveto(911.15305739,327.97134588)(911.1380574,327.98634586)(911.11806396,327.99634644)
\curveto(911.10805743,328.01634583)(911.09305745,328.03634581)(911.07306396,328.05634644)
\curveto(910.97305757,328.06634578)(910.89305765,328.0463458)(910.83306396,327.99634644)
\curveto(910.78305776,327.9463459)(910.72805781,327.89634595)(910.66806396,327.84634644)
\curveto(910.46805807,327.69634615)(910.26805827,327.58134627)(910.06806396,327.50134644)
\curveto(909.88805865,327.42134643)(909.67805886,327.36134649)(909.43806396,327.32134644)
\curveto(909.20805933,327.28134657)(908.96805957,327.26134659)(908.71806396,327.26134644)
\curveto(908.47806006,327.2513466)(908.2380603,327.26634658)(907.99806396,327.30634644)
\curveto(907.75806078,327.33634651)(907.54806099,327.39134646)(907.36806396,327.47134644)
\curveto(906.84806169,327.69134616)(906.42806211,327.98634586)(906.10806396,328.35634644)
\curveto(905.78806275,328.73634511)(905.538063,329.20634464)(905.35806396,329.76634644)
\curveto(905.31806322,329.85634399)(905.28806325,329.9463439)(905.26806396,330.03634644)
\curveto(905.25806328,330.13634371)(905.2380633,330.23634361)(905.20806396,330.33634644)
\curveto(905.19806334,330.38634346)(905.19306335,330.43634341)(905.19306396,330.48634644)
\curveto(905.19306335,330.53634331)(905.18806335,330.58634326)(905.17806396,330.63634644)
\curveto(905.15806338,330.68634316)(905.14806339,330.73634311)(905.14806396,330.78634644)
\curveto(905.15806338,330.846343)(905.15806338,330.90134295)(905.14806396,330.95134644)
\lineto(905.14806396,331.10134644)
\curveto(905.12806341,331.1513427)(905.11806342,331.21634263)(905.11806396,331.29634644)
\curveto(905.11806342,331.37634247)(905.12806341,331.44134241)(905.14806396,331.49134644)
\lineto(905.14806396,331.65634644)
\curveto(905.16806337,331.72634212)(905.17306337,331.79634205)(905.16306396,331.86634644)
\curveto(905.16306338,331.9463419)(905.17306337,332.02134183)(905.19306396,332.09134644)
\curveto(905.20306334,332.14134171)(905.20806333,332.18634166)(905.20806396,332.22634644)
\curveto(905.20806333,332.26634158)(905.21306333,332.31134154)(905.22306396,332.36134644)
\curveto(905.25306329,332.46134139)(905.27806326,332.55634129)(905.29806396,332.64634644)
\curveto(905.31806322,332.7463411)(905.3430632,332.84134101)(905.37306396,332.93134644)
\curveto(905.50306304,333.31134054)(905.66806287,333.6513402)(905.86806396,333.95134644)
\curveto(906.07806246,334.26133959)(906.32806221,334.51633933)(906.61806396,334.71634644)
\curveto(906.78806175,334.83633901)(906.96306158,334.93633891)(907.14306396,335.01634644)
\curveto(907.33306121,335.09633875)(907.538061,335.16633868)(907.75806396,335.22634644)
\curveto(907.82806071,335.23633861)(907.89306065,335.2463386)(907.95306396,335.25634644)
\curveto(908.02306052,335.26633858)(908.09306045,335.28133857)(908.16306396,335.30134644)
\lineto(908.31306396,335.30134644)
\curveto(908.39306015,335.32133853)(908.50806003,335.33133852)(908.65806396,335.33134644)
\curveto(908.81805972,335.33133852)(908.9380596,335.32133853)(909.01806396,335.30134644)
\curveto(909.05805948,335.29133856)(909.11305943,335.28633856)(909.18306396,335.28634644)
\curveto(909.29305925,335.25633859)(909.40305914,335.23133862)(909.51306396,335.21134644)
\curveto(909.62305892,335.20133865)(909.72805881,335.17133868)(909.82806396,335.12134644)
\curveto(909.97805856,335.06133879)(910.11805842,334.99633885)(910.24806396,334.92634644)
\curveto(910.38805815,334.85633899)(910.51805802,334.77633907)(910.63806396,334.68634644)
\curveto(910.69805784,334.63633921)(910.75805778,334.58133927)(910.81806396,334.52134644)
\curveto(910.88805765,334.47133938)(910.97805756,334.45633939)(911.08806396,334.47634644)
\curveto(911.10805743,334.50633934)(911.12305742,334.53133932)(911.13306396,334.55134644)
\curveto(911.15305739,334.57133928)(911.16805737,334.60133925)(911.17806396,334.64134644)
\curveto(911.20805733,334.73133912)(911.21805732,334.846339)(911.20806396,334.98634644)
\lineto(911.20806396,335.36134644)
\lineto(911.20806396,337.08634644)
\lineto(911.20806396,337.55134644)
\curveto(911.20805733,337.73133612)(911.23305731,337.86133599)(911.28306396,337.94134644)
\curveto(911.32305722,338.01133584)(911.38305716,338.05633579)(911.46306396,338.07634644)
\curveto(911.48305706,338.07633577)(911.50805703,338.07633577)(911.53806396,338.07634644)
\curveto(911.56805697,338.08633576)(911.59305695,338.09133576)(911.61306396,338.09134644)
\curveto(911.75305679,338.10133575)(911.89805664,338.10133575)(912.04806396,338.09134644)
\curveto(912.20805633,338.09133576)(912.31805622,338.0513358)(912.37806396,337.97134644)
\curveto(912.42805611,337.89133596)(912.45305609,337.79133606)(912.45306396,337.67134644)
\lineto(912.45306396,337.29634644)
\lineto(912.45306396,328.23634644)
\moveto(911.23806396,331.07134644)
\curveto(911.25805728,331.12134273)(911.26805727,331.18634266)(911.26806396,331.26634644)
\curveto(911.26805727,331.35634249)(911.25805728,331.42634242)(911.23806396,331.47634644)
\lineto(911.23806396,331.70134644)
\curveto(911.21805732,331.79134206)(911.20305734,331.88134197)(911.19306396,331.97134644)
\curveto(911.18305736,332.07134178)(911.16305738,332.16134169)(911.13306396,332.24134644)
\curveto(911.11305743,332.32134153)(911.09305745,332.39634145)(911.07306396,332.46634644)
\curveto(911.06305748,332.53634131)(911.0430575,332.60634124)(911.01306396,332.67634644)
\curveto(910.89305765,332.97634087)(910.7380578,333.24134061)(910.54806396,333.47134644)
\curveto(910.35805818,333.70134015)(910.11805842,333.88133997)(909.82806396,334.01134644)
\curveto(909.72805881,334.06133979)(909.62305892,334.09633975)(909.51306396,334.11634644)
\curveto(909.41305913,334.1463397)(909.30305924,334.17133968)(909.18306396,334.19134644)
\curveto(909.10305944,334.21133964)(909.01305953,334.22133963)(908.91306396,334.22134644)
\lineto(908.64306396,334.22134644)
\curveto(908.59305995,334.21133964)(908.54805999,334.20133965)(908.50806396,334.19134644)
\lineto(908.37306396,334.19134644)
\curveto(908.29306025,334.17133968)(908.20806033,334.1513397)(908.11806396,334.13134644)
\curveto(908.0380605,334.11133974)(907.95806058,334.08633976)(907.87806396,334.05634644)
\curveto(907.55806098,333.91633993)(907.29806124,333.71134014)(907.09806396,333.44134644)
\curveto(906.90806163,333.18134067)(906.75306179,332.87634097)(906.63306396,332.52634644)
\curveto(906.59306195,332.41634143)(906.56306198,332.30134155)(906.54306396,332.18134644)
\curveto(906.53306201,332.07134178)(906.51806202,331.96134189)(906.49806396,331.85134644)
\curveto(906.49806204,331.81134204)(906.49306205,331.77134208)(906.48306396,331.73134644)
\lineto(906.48306396,331.62634644)
\curveto(906.46306208,331.57634227)(906.45306209,331.52134233)(906.45306396,331.46134644)
\curveto(906.46306208,331.40134245)(906.46806207,331.3463425)(906.46806396,331.29634644)
\lineto(906.46806396,330.96634644)
\curveto(906.46806207,330.86634298)(906.47806206,330.77134308)(906.49806396,330.68134644)
\curveto(906.50806203,330.6513432)(906.51306203,330.60134325)(906.51306396,330.53134644)
\curveto(906.53306201,330.46134339)(906.54806199,330.39134346)(906.55806396,330.32134644)
\lineto(906.61806396,330.11134644)
\curveto(906.72806181,329.76134409)(906.87806166,329.46134439)(907.06806396,329.21134644)
\curveto(907.25806128,328.96134489)(907.49806104,328.75634509)(907.78806396,328.59634644)
\curveto(907.87806066,328.5463453)(907.96806057,328.50634534)(908.05806396,328.47634644)
\curveto(908.14806039,328.4463454)(908.24806029,328.41634543)(908.35806396,328.38634644)
\curveto(908.40806013,328.36634548)(908.45806008,328.36134549)(908.50806396,328.37134644)
\curveto(908.56805997,328.38134547)(908.62305992,328.37634547)(908.67306396,328.35634644)
\curveto(908.71305983,328.3463455)(908.75305979,328.34134551)(908.79306396,328.34134644)
\lineto(908.92806396,328.34134644)
\lineto(909.06306396,328.34134644)
\curveto(909.09305945,328.3513455)(909.1430594,328.35634549)(909.21306396,328.35634644)
\curveto(909.29305925,328.37634547)(909.37305917,328.39134546)(909.45306396,328.40134644)
\curveto(909.53305901,328.42134543)(909.60805893,328.4463454)(909.67806396,328.47634644)
\curveto(910.00805853,328.61634523)(910.27305827,328.79134506)(910.47306396,329.00134644)
\curveto(910.68305786,329.22134463)(910.85805768,329.49634435)(910.99806396,329.82634644)
\curveto(911.04805749,329.93634391)(911.08305746,330.0463438)(911.10306396,330.15634644)
\curveto(911.12305742,330.26634358)(911.14805739,330.37634347)(911.17806396,330.48634644)
\curveto(911.19805734,330.52634332)(911.20805733,330.56134329)(911.20806396,330.59134644)
\curveto(911.20805733,330.63134322)(911.21305733,330.67134318)(911.22306396,330.71134644)
\curveto(911.23305731,330.77134308)(911.23305731,330.83134302)(911.22306396,330.89134644)
\curveto(911.22305732,330.9513429)(911.22805731,331.01134284)(911.23806396,331.07134644)
}
}
{
\newrgbcolor{curcolor}{0 0 0}
\pscustom[linestyle=none,fillstyle=solid,fillcolor=curcolor]
{
\newpath
\moveto(839.47645996,312.9330896)
\curveto(839.49645084,312.85308182)(839.50645083,312.74308193)(839.50645996,312.6030896)
\curveto(839.50645083,312.4730822)(839.49645084,312.3730823)(839.47645996,312.3030896)
\curveto(839.45645088,312.23308244)(839.45145088,312.1680825)(839.46145996,312.1080896)
\curveto(839.47145086,312.04808262)(839.46645087,311.98308269)(839.44645996,311.9130896)
\curveto(839.42645091,311.85308282)(839.41145092,311.78808288)(839.40145996,311.7180896)
\curveto(839.39145094,311.65808301)(839.37645096,311.59808307)(839.35645996,311.5380896)
\curveto(839.336451,311.45808321)(839.31145102,311.38308329)(839.28145996,311.3130896)
\curveto(839.26145107,311.24308343)(839.2364511,311.1730835)(839.20645996,311.1030896)
\curveto(839.18645115,311.0730836)(839.17145116,311.04308363)(839.16145996,311.0130896)
\curveto(839.16145117,310.99308368)(839.15145118,310.9730837)(839.13145996,310.9530896)
\curveto(839.02145131,310.75308392)(838.90145143,310.5730841)(838.77145996,310.4130896)
\curveto(838.75145158,310.3730843)(838.71645162,310.33308434)(838.66645996,310.2930896)
\curveto(838.62645171,310.25308442)(838.59145174,310.22308445)(838.56145996,310.2030896)
\curveto(838.52145181,310.18308449)(838.48645185,310.15308452)(838.45645996,310.1130896)
\curveto(838.42645191,310.08308459)(838.39645194,310.05808461)(838.36645996,310.0380896)
\lineto(838.05145996,309.8580896)
\curveto(837.94145239,309.77808489)(837.81145252,309.71808495)(837.66145996,309.6780896)
\lineto(837.21145996,309.5580896)
\curveto(837.1314532,309.53808513)(837.05145328,309.52308515)(836.97145996,309.5130896)
\curveto(836.89145344,309.51308516)(836.81145352,309.50308517)(836.73145996,309.4830896)
\curveto(836.69145364,309.4730852)(836.65145368,309.4680852)(836.61145996,309.4680896)
\curveto(836.58145375,309.47808519)(836.55145378,309.47808519)(836.52145996,309.4680896)
\curveto(836.47145386,309.45808521)(836.42145391,309.45808521)(836.37145996,309.4680896)
\curveto(836.331454,309.47808519)(836.28645405,309.47808519)(836.23645996,309.4680896)
\lineto(833.97145996,309.4680896)
\lineto(833.47645996,309.4680896)
\curveto(833.30645703,309.47808519)(833.17645716,309.44808522)(833.08645996,309.3780896)
\curveto(832.97645736,309.29808537)(832.92145741,309.15308552)(832.92145996,308.9430896)
\curveto(832.9314574,308.73308594)(832.9364574,308.53808613)(832.93645996,308.3580896)
\lineto(832.93645996,306.1530896)
\lineto(832.93645996,305.6580896)
\curveto(832.94645739,305.4680892)(832.92645741,305.33308934)(832.87645996,305.2530896)
\curveto(832.8364575,305.19308948)(832.78645755,305.15308952)(832.72645996,305.1330896)
\curveto(832.67645766,305.12308955)(832.61145772,305.10808956)(832.53145996,305.0880896)
\lineto(832.26145996,305.0880896)
\curveto(832.11145822,305.08808958)(831.97645836,305.09308958)(831.85645996,305.1030896)
\curveto(831.7364586,305.11308956)(831.65145868,305.16308951)(831.60145996,305.2530896)
\curveto(831.56145877,305.31308936)(831.54145879,305.39308928)(831.54145996,305.4930896)
\lineto(831.54145996,305.8080896)
\lineto(831.54145996,314.9130896)
\curveto(831.54145879,315.02307965)(831.5364588,315.14307953)(831.52645996,315.2730896)
\curveto(831.52645881,315.41307926)(831.55145878,315.52307915)(831.60145996,315.6030896)
\curveto(831.64145869,315.66307901)(831.71645862,315.71307896)(831.82645996,315.7530896)
\curveto(831.84645849,315.76307891)(831.86645847,315.76307891)(831.88645996,315.7530896)
\curveto(831.90645843,315.75307892)(831.92645841,315.75807891)(831.94645996,315.7680896)
\lineto(835.35145996,315.7680896)
\curveto(835.7314546,315.7680789)(836.10145423,315.76307891)(836.46145996,315.7530896)
\curveto(836.8314535,315.75307892)(837.16145317,315.70807896)(837.45145996,315.6180896)
\curveto(837.90145243,315.4680792)(838.26645207,315.2730794)(838.54645996,315.0330896)
\curveto(838.82645151,314.79307988)(839.05645128,314.46308021)(839.23645996,314.0430896)
\curveto(839.28645105,313.93308074)(839.32145101,313.81808085)(839.34145996,313.6980896)
\curveto(839.37145096,313.57808109)(839.40645093,313.45308122)(839.44645996,313.3230896)
\curveto(839.46645087,313.25308142)(839.47145086,313.18808148)(839.46145996,313.1280896)
\curveto(839.45145088,313.0680816)(839.45645088,313.00308167)(839.47645996,312.9330896)
\moveto(838.06645996,312.3930896)
\curveto(838.10645223,312.53308214)(838.11145222,312.69308198)(838.08145996,312.8730896)
\curveto(838.05145228,313.06308161)(838.02145231,313.21308146)(837.99145996,313.3230896)
\curveto(837.89145244,313.60308107)(837.75645258,313.82308085)(837.58645996,313.9830896)
\curveto(837.42645291,314.15308052)(837.21645312,314.29308038)(836.95645996,314.4030896)
\curveto(836.7364536,314.49308018)(836.48145385,314.54808012)(836.19145996,314.5680896)
\curveto(835.91145442,314.58808008)(835.61645472,314.59808007)(835.30645996,314.5980896)
\lineto(833.37145996,314.5980896)
\curveto(833.35145698,314.58808008)(833.32645701,314.58308009)(833.29645996,314.5830896)
\curveto(833.27645706,314.58308009)(833.25145708,314.57808009)(833.22145996,314.5680896)
\curveto(833.10145723,314.53808013)(833.02145731,314.4730802)(832.98145996,314.3730896)
\curveto(832.94145739,314.2730804)(832.92145741,314.13808053)(832.92145996,313.9680896)
\curveto(832.9314574,313.80808086)(832.9364574,313.65808101)(832.93645996,313.5180896)
\lineto(832.93645996,311.7180896)
\curveto(832.9364574,311.5680831)(832.9314574,311.40308327)(832.92145996,311.2230896)
\curveto(832.92145741,311.04308363)(832.95145738,310.90308377)(833.01145996,310.8030896)
\curveto(833.06145727,310.72308395)(833.1364572,310.673084)(833.23645996,310.6530896)
\curveto(833.34645699,310.64308403)(833.46645687,310.63808403)(833.59645996,310.6380896)
\lineto(835.62145996,310.6380896)
\lineto(836.08645996,310.6380896)
\curveto(836.24645409,310.64808402)(836.38645395,310.668084)(836.50645996,310.6980896)
\curveto(836.77645356,310.7680839)(837.01145332,310.84808382)(837.21145996,310.9380896)
\curveto(837.42145291,311.03808363)(837.59645274,311.18808348)(837.73645996,311.3880896)
\curveto(837.81645252,311.50808316)(837.87645246,311.63308304)(837.91645996,311.7630896)
\curveto(837.96645237,311.89308278)(838.01145232,312.03808263)(838.05145996,312.1980896)
\curveto(838.06145227,312.23808243)(838.06645227,312.30308237)(838.06645996,312.3930896)
}
}
{
\newrgbcolor{curcolor}{0 0 0}
\pscustom[linestyle=none,fillstyle=solid,fillcolor=curcolor]
{
\newpath
\moveto(848.13802246,309.2880896)
\curveto(848.1580144,309.22808544)(848.16801439,309.13308554)(848.16802246,309.0030896)
\curveto(848.16801439,308.88308579)(848.1630144,308.79808587)(848.15302246,308.7480896)
\lineto(848.15302246,308.5980896)
\curveto(848.14301442,308.51808615)(848.13301443,308.44308623)(848.12302246,308.3730896)
\curveto(848.12301444,308.31308636)(848.11801444,308.24308643)(848.10802246,308.1630896)
\curveto(848.08801447,308.10308657)(848.07301449,308.04308663)(848.06302246,307.9830896)
\curveto(848.0630145,307.92308675)(848.05301451,307.86308681)(848.03302246,307.8030896)
\curveto(847.99301457,307.673087)(847.9580146,307.54308713)(847.92802246,307.4130896)
\curveto(847.89801466,307.28308739)(847.8580147,307.16308751)(847.80802246,307.0530896)
\curveto(847.59801496,306.5730881)(847.31801524,306.1680885)(846.96802246,305.8380896)
\curveto(846.61801594,305.51808915)(846.18801637,305.2730894)(845.67802246,305.1030896)
\curveto(845.56801699,305.06308961)(845.44801711,305.03308964)(845.31802246,305.0130896)
\curveto(845.19801736,304.99308968)(845.07301749,304.9730897)(844.94302246,304.9530896)
\curveto(844.88301768,304.94308973)(844.81801774,304.93808973)(844.74802246,304.9380896)
\curveto(844.68801787,304.92808974)(844.62801793,304.92308975)(844.56802246,304.9230896)
\curveto(844.52801803,304.91308976)(844.46801809,304.90808976)(844.38802246,304.9080896)
\curveto(844.31801824,304.90808976)(844.26801829,304.91308976)(844.23802246,304.9230896)
\curveto(844.19801836,304.93308974)(844.1580184,304.93808973)(844.11802246,304.9380896)
\curveto(844.07801848,304.92808974)(844.04301852,304.92808974)(844.01302246,304.9380896)
\lineto(843.92302246,304.9380896)
\lineto(843.56302246,304.9830896)
\curveto(843.42301914,305.02308965)(843.28801927,305.06308961)(843.15802246,305.1030896)
\curveto(843.02801953,305.14308953)(842.90301966,305.18808948)(842.78302246,305.2380896)
\curveto(842.33302023,305.43808923)(841.9630206,305.69808897)(841.67302246,306.0180896)
\curveto(841.38302118,306.33808833)(841.14302142,306.72808794)(840.95302246,307.1880896)
\curveto(840.90302166,307.28808738)(840.8630217,307.38808728)(840.83302246,307.4880896)
\curveto(840.81302175,307.58808708)(840.79302177,307.69308698)(840.77302246,307.8030896)
\curveto(840.75302181,307.84308683)(840.74302182,307.8730868)(840.74302246,307.8930896)
\curveto(840.75302181,307.92308675)(840.75302181,307.95808671)(840.74302246,307.9980896)
\curveto(840.72302184,308.07808659)(840.70802185,308.15808651)(840.69802246,308.2380896)
\curveto(840.69802186,308.32808634)(840.68802187,308.41308626)(840.66802246,308.4930896)
\lineto(840.66802246,308.6130896)
\curveto(840.66802189,308.65308602)(840.6630219,308.69808597)(840.65302246,308.7480896)
\curveto(840.64302192,308.79808587)(840.63802192,308.88308579)(840.63802246,309.0030896)
\curveto(840.63802192,309.13308554)(840.64802191,309.22808544)(840.66802246,309.2880896)
\curveto(840.68802187,309.35808531)(840.69302187,309.42808524)(840.68302246,309.4980896)
\curveto(840.67302189,309.5680851)(840.67802188,309.63808503)(840.69802246,309.7080896)
\curveto(840.70802185,309.75808491)(840.71302185,309.79808487)(840.71302246,309.8280896)
\curveto(840.72302184,309.8680848)(840.73302183,309.91308476)(840.74302246,309.9630896)
\curveto(840.77302179,310.08308459)(840.79802176,310.20308447)(840.81802246,310.3230896)
\curveto(840.84802171,310.44308423)(840.88802167,310.55808411)(840.93802246,310.6680896)
\curveto(841.08802147,311.03808363)(841.26802129,311.3680833)(841.47802246,311.6580896)
\curveto(841.69802086,311.95808271)(841.9630206,312.20808246)(842.27302246,312.4080896)
\curveto(842.39302017,312.48808218)(842.51802004,312.55308212)(842.64802246,312.6030896)
\curveto(842.77801978,312.66308201)(842.91301965,312.72308195)(843.05302246,312.7830896)
\curveto(843.17301939,312.83308184)(843.30301926,312.86308181)(843.44302246,312.8730896)
\curveto(843.58301898,312.89308178)(843.72301884,312.92308175)(843.86302246,312.9630896)
\lineto(844.05802246,312.9630896)
\curveto(844.12801843,312.9730817)(844.19301837,312.98308169)(844.25302246,312.9930896)
\curveto(845.14301742,313.00308167)(845.88301668,312.81808185)(846.47302246,312.4380896)
\curveto(847.0630155,312.05808261)(847.48801507,311.56308311)(847.74802246,310.9530896)
\curveto(847.79801476,310.85308382)(847.83801472,310.75308392)(847.86802246,310.6530896)
\curveto(847.89801466,310.55308412)(847.93301463,310.44808422)(847.97302246,310.3380896)
\curveto(848.00301456,310.22808444)(848.02801453,310.10808456)(848.04802246,309.9780896)
\curveto(848.06801449,309.85808481)(848.09301447,309.73308494)(848.12302246,309.6030896)
\curveto(848.13301443,309.55308512)(848.13301443,309.49808517)(848.12302246,309.4380896)
\curveto(848.12301444,309.38808528)(848.12801443,309.33808533)(848.13802246,309.2880896)
\moveto(846.80302246,308.4330896)
\curveto(846.82301574,308.50308617)(846.82801573,308.58308609)(846.81802246,308.6730896)
\lineto(846.81802246,308.9280896)
\curveto(846.81801574,309.31808535)(846.78301578,309.64808502)(846.71302246,309.9180896)
\curveto(846.68301588,309.99808467)(846.6580159,310.07808459)(846.63802246,310.1580896)
\curveto(846.61801594,310.23808443)(846.59301597,310.31308436)(846.56302246,310.3830896)
\curveto(846.28301628,311.03308364)(845.83801672,311.48308319)(845.22802246,311.7330896)
\curveto(845.1580174,311.76308291)(845.08301748,311.78308289)(845.00302246,311.7930896)
\lineto(844.76302246,311.8530896)
\curveto(844.68301788,311.8730828)(844.59801796,311.88308279)(844.50802246,311.8830896)
\lineto(844.23802246,311.8830896)
\lineto(843.96802246,311.8380896)
\curveto(843.86801869,311.81808285)(843.77301879,311.79308288)(843.68302246,311.7630896)
\curveto(843.60301896,311.74308293)(843.52301904,311.71308296)(843.44302246,311.6730896)
\curveto(843.37301919,311.65308302)(843.30801925,311.62308305)(843.24802246,311.5830896)
\curveto(843.18801937,311.54308313)(843.13301943,311.50308317)(843.08302246,311.4630896)
\curveto(842.84301972,311.29308338)(842.64801991,311.08808358)(842.49802246,310.8480896)
\curveto(842.34802021,310.60808406)(842.21802034,310.32808434)(842.10802246,310.0080896)
\curveto(842.07802048,309.90808476)(842.0580205,309.80308487)(842.04802246,309.6930896)
\curveto(842.03802052,309.59308508)(842.02302054,309.48808518)(842.00302246,309.3780896)
\curveto(841.99302057,309.33808533)(841.98802057,309.2730854)(841.98802246,309.1830896)
\curveto(841.97802058,309.15308552)(841.97302059,309.11808555)(841.97302246,309.0780896)
\curveto(841.98302058,309.03808563)(841.98802057,308.99308568)(841.98802246,308.9430896)
\lineto(841.98802246,308.6430896)
\curveto(841.98802057,308.54308613)(841.99802056,308.45308622)(842.01802246,308.3730896)
\lineto(842.04802246,308.1930896)
\curveto(842.06802049,308.09308658)(842.08302048,307.99308668)(842.09302246,307.8930896)
\curveto(842.11302045,307.80308687)(842.14302042,307.71808695)(842.18302246,307.6380896)
\curveto(842.28302028,307.39808727)(842.39802016,307.1730875)(842.52802246,306.9630896)
\curveto(842.66801989,306.75308792)(842.83801972,306.57808809)(843.03802246,306.4380896)
\curveto(843.08801947,306.40808826)(843.13301943,306.38308829)(843.17302246,306.3630896)
\curveto(843.21301935,306.34308833)(843.2580193,306.31808835)(843.30802246,306.2880896)
\curveto(843.38801917,306.23808843)(843.47301909,306.19308848)(843.56302246,306.1530896)
\curveto(843.6630189,306.12308855)(843.76801879,306.09308858)(843.87802246,306.0630896)
\curveto(843.92801863,306.04308863)(843.97301859,306.03308864)(844.01302246,306.0330896)
\curveto(844.0630185,306.04308863)(844.11301845,306.04308863)(844.16302246,306.0330896)
\curveto(844.19301837,306.02308865)(844.25301831,306.01308866)(844.34302246,306.0030896)
\curveto(844.44301812,305.99308868)(844.51801804,305.99808867)(844.56802246,306.0180896)
\curveto(844.60801795,306.02808864)(844.64801791,306.02808864)(844.68802246,306.0180896)
\curveto(844.72801783,306.01808865)(844.76801779,306.02808864)(844.80802246,306.0480896)
\curveto(844.88801767,306.0680886)(844.96801759,306.08308859)(845.04802246,306.0930896)
\curveto(845.12801743,306.11308856)(845.20301736,306.13808853)(845.27302246,306.1680896)
\curveto(845.61301695,306.30808836)(845.88801667,306.50308817)(846.09802246,306.7530896)
\curveto(846.30801625,307.00308767)(846.48301608,307.29808737)(846.62302246,307.6380896)
\curveto(846.67301589,307.75808691)(846.70301586,307.88308679)(846.71302246,308.0130896)
\curveto(846.73301583,308.15308652)(846.7630158,308.29308638)(846.80302246,308.4330896)
}
}
{
\newrgbcolor{curcolor}{0 0 0}
\pscustom[linestyle=none,fillstyle=solid,fillcolor=curcolor]
{
\newpath
\moveto(857.05130371,309.1530896)
\curveto(857.06129536,309.10308557)(857.06629536,309.03808563)(857.06630371,308.9580896)
\curveto(857.06629536,308.87808579)(857.06129536,308.81308586)(857.05130371,308.7630896)
\curveto(857.03129539,308.71308596)(857.0262954,308.66308601)(857.03630371,308.6130896)
\curveto(857.04629538,308.5730861)(857.04629538,308.53308614)(857.03630371,308.4930896)
\curveto(857.03629539,308.42308625)(857.03129539,308.3680863)(857.02130371,308.3280896)
\curveto(857.00129542,308.23808643)(856.98629544,308.14808652)(856.97630371,308.0580896)
\curveto(856.97629545,307.9680867)(856.96629546,307.87808679)(856.94630371,307.7880896)
\lineto(856.88630371,307.5480896)
\curveto(856.86629556,307.47808719)(856.84129558,307.40308727)(856.81130371,307.3230896)
\curveto(856.69129573,306.95308772)(856.5262959,306.61808805)(856.31630371,306.3180896)
\curveto(856.25629617,306.22808844)(856.19129623,306.13808853)(856.12130371,306.0480896)
\curveto(856.05129637,305.9680887)(855.97629645,305.89308878)(855.89630371,305.8230896)
\lineto(855.82130371,305.7480896)
\curveto(855.75129667,305.69808897)(855.68629674,305.64808902)(855.62630371,305.5980896)
\curveto(855.56629686,305.54808912)(855.49629693,305.49808917)(855.41630371,305.4480896)
\curveto(855.30629712,305.3680893)(855.18129724,305.29808937)(855.04130371,305.2380896)
\curveto(854.91129751,305.18808948)(854.77629765,305.13808953)(854.63630371,305.0880896)
\curveto(854.55629787,305.0680896)(854.47629795,305.05308962)(854.39630371,305.0430896)
\curveto(854.3262981,305.03308964)(854.25129817,305.01808965)(854.17130371,304.9980896)
\lineto(854.11130371,304.9980896)
\curveto(854.10129832,304.98808968)(854.08629834,304.98308969)(854.06630371,304.9830896)
\curveto(853.97629845,304.96308971)(853.84129858,304.95308972)(853.66130371,304.9530896)
\curveto(853.49129893,304.94308973)(853.35629907,304.94808972)(853.25630371,304.9680896)
\lineto(853.18130371,304.9680896)
\curveto(853.11129931,304.97808969)(853.04629938,304.98808968)(852.98630371,304.9980896)
\curveto(852.9262995,304.99808967)(852.86629956,305.00808966)(852.80630371,305.0280896)
\curveto(852.63629979,305.07808959)(852.47629995,305.12308955)(852.32630371,305.1630896)
\curveto(852.17630025,305.20308947)(852.03630039,305.26308941)(851.90630371,305.3430896)
\curveto(851.74630068,305.43308924)(851.60630082,305.52808914)(851.48630371,305.6280896)
\curveto(851.44630098,305.65808901)(851.38630104,305.69808897)(851.30630371,305.7480896)
\curveto(851.2263012,305.80808886)(851.15130127,305.81308886)(851.08130371,305.7630896)
\curveto(851.04130138,305.73308894)(851.0213014,305.69308898)(851.02130371,305.6430896)
\curveto(851.0213014,305.59308908)(851.01130141,305.53808913)(850.99130371,305.4780896)
\curveto(850.98130144,305.44808922)(850.98130144,305.41308926)(850.99130371,305.3730896)
\curveto(851.00130142,305.34308933)(851.00130142,305.30808936)(850.99130371,305.2680896)
\curveto(850.97130145,305.20808946)(850.96130146,305.14308953)(850.96130371,305.0730896)
\curveto(850.97130145,304.99308968)(850.97630145,304.92308975)(850.97630371,304.8630896)
\lineto(850.97630371,303.0630896)
\lineto(850.97630371,302.6280896)
\curveto(850.97630145,302.47809219)(850.94630148,302.36309231)(850.88630371,302.2830896)
\curveto(850.83630159,302.21309246)(850.75630167,302.17809249)(850.64630371,302.1780896)
\curveto(850.53630189,302.1680925)(850.426302,302.16309251)(850.31630371,302.1630896)
\lineto(850.07630371,302.1630896)
\curveto(850.00630242,302.18309249)(849.94630248,302.20309247)(849.89630371,302.2230896)
\curveto(849.85630257,302.24309243)(849.8213026,302.27809239)(849.79130371,302.3280896)
\curveto(849.74130268,302.39809227)(849.71630271,302.50809216)(849.71630371,302.6580896)
\curveto(849.7263027,302.80809186)(849.73130269,302.93809173)(849.73130371,303.0480896)
\lineto(849.73130371,312.0480896)
\lineto(849.73130371,312.4080896)
\curveto(849.74130268,312.53808213)(849.77130265,312.64308203)(849.82130371,312.7230896)
\curveto(849.85130257,312.76308191)(849.91630251,312.79308188)(850.01630371,312.8130896)
\curveto(850.1263023,312.84308183)(850.24130218,312.85308182)(850.36130371,312.8430896)
\curveto(850.48130194,312.84308183)(850.59130183,312.82808184)(850.69130371,312.7980896)
\curveto(850.80130162,312.77808189)(850.87130155,312.74808192)(850.90130371,312.7080896)
\curveto(850.94130148,312.65808201)(850.96130146,312.59808207)(850.96130371,312.5280896)
\curveto(850.97130145,312.45808221)(850.99130143,312.38808228)(851.02130371,312.3180896)
\curveto(851.04130138,312.28808238)(851.05630137,312.26308241)(851.06630371,312.2430896)
\curveto(851.08630134,312.23308244)(851.10630132,312.21808245)(851.12630371,312.1980896)
\curveto(851.23630119,312.18808248)(851.3263011,312.22308245)(851.39630371,312.3030896)
\curveto(851.47630095,312.38308229)(851.55130087,312.44808222)(851.62130371,312.4980896)
\curveto(851.88130054,312.67808199)(852.19130023,312.81808185)(852.55130371,312.9180896)
\curveto(852.64129978,312.93808173)(852.73129969,312.95308172)(852.82130371,312.9630896)
\curveto(852.9212995,312.9730817)(853.0212994,312.98808168)(853.12130371,313.0080896)
\curveto(853.16129926,313.01808165)(853.21129921,313.01808165)(853.27130371,313.0080896)
\curveto(853.33129909,312.99808167)(853.37129905,313.00308167)(853.39130371,313.0230896)
\curveto(853.8212986,313.03308164)(854.20129822,312.98808168)(854.53130371,312.8880896)
\curveto(854.86129756,312.79808187)(855.15629727,312.668082)(855.41630371,312.4980896)
\lineto(855.56630371,312.3780896)
\curveto(855.61629681,312.34808232)(855.66629676,312.31308236)(855.71630371,312.2730896)
\curveto(855.73629669,312.25308242)(855.75129667,312.23308244)(855.76130371,312.2130896)
\curveto(855.78129664,312.20308247)(855.80129662,312.18808248)(855.82130371,312.1680896)
\curveto(855.87129655,312.11808255)(855.9262965,312.06308261)(855.98630371,312.0030896)
\curveto(856.04629638,311.94308273)(856.10129632,311.88308279)(856.15130371,311.8230896)
\curveto(856.27129615,311.65308302)(856.39629603,311.4680832)(856.52630371,311.2680896)
\curveto(856.60629582,311.13808353)(856.67129575,310.99308368)(856.72130371,310.8330896)
\curveto(856.78129564,310.673084)(856.83629559,310.51308416)(856.88630371,310.3530896)
\curveto(856.90629552,310.2730844)(856.9212955,310.18808448)(856.93130371,310.0980896)
\curveto(856.95129547,310.00808466)(856.97129545,309.92308475)(856.99130371,309.8430896)
\lineto(856.99130371,309.7230896)
\curveto(857.00129542,309.69308498)(857.00629542,309.66308501)(857.00630371,309.6330896)
\curveto(857.0262954,309.58308509)(857.03129539,309.52808514)(857.02130371,309.4680896)
\curveto(857.0212954,309.40808526)(857.03129539,309.35308532)(857.05130371,309.3030896)
\lineto(857.05130371,309.1530896)
\moveto(855.71630371,308.7480896)
\curveto(855.73629669,308.79808587)(855.74129668,308.85808581)(855.73130371,308.9280896)
\curveto(855.7212967,309.00808566)(855.71629671,309.07808559)(855.71630371,309.1380896)
\curveto(855.71629671,309.30808536)(855.70629672,309.4680852)(855.68630371,309.6180896)
\curveto(855.67629675,309.7680849)(855.64629678,309.91308476)(855.59630371,310.0530896)
\lineto(855.53630371,310.2330896)
\curveto(855.5262969,310.30308437)(855.50629692,310.3680843)(855.47630371,310.4280896)
\curveto(855.36629706,310.69808397)(855.19129723,310.95808371)(854.95130371,311.2080896)
\curveto(854.7212977,311.45808321)(854.50129792,311.62808304)(854.29130371,311.7180896)
\curveto(854.21129821,311.75808291)(854.1262983,311.78808288)(854.03630371,311.8080896)
\curveto(853.95629847,311.82808284)(853.87129855,311.85308282)(853.78130371,311.8830896)
\curveto(853.69129873,311.90308277)(853.58629884,311.91308276)(853.46630371,311.9130896)
\lineto(853.13630371,311.9130896)
\curveto(853.11629931,311.89308278)(853.07629935,311.88308279)(853.01630371,311.8830896)
\curveto(852.96629946,311.89308278)(852.9212995,311.89308278)(852.88130371,311.8830896)
\lineto(852.61130371,311.8230896)
\curveto(852.53129989,311.80308287)(852.45129997,311.7730829)(852.37130371,311.7330896)
\curveto(852.05130037,311.59308308)(851.78630064,311.38808328)(851.57630371,311.1180896)
\curveto(851.37630105,310.85808381)(851.2213012,310.55308412)(851.11130371,310.2030896)
\curveto(851.07130135,310.09308458)(851.04130138,309.98308469)(851.02130371,309.8730896)
\curveto(851.01130141,309.76308491)(850.99630143,309.65308502)(850.97630371,309.5430896)
\curveto(850.96630146,309.50308517)(850.96130146,309.46308521)(850.96130371,309.4230896)
\curveto(850.96130146,309.39308528)(850.95630147,309.35808531)(850.94630371,309.3180896)
\lineto(850.94630371,309.1980896)
\curveto(850.93630149,309.14808552)(850.93130149,309.0730856)(850.93130371,308.9730896)
\curveto(850.93130149,308.88308579)(850.93630149,308.81308586)(850.94630371,308.7630896)
\lineto(850.94630371,308.6430896)
\curveto(850.95630147,308.60308607)(850.96130146,308.56308611)(850.96130371,308.5230896)
\curveto(850.96130146,308.48308619)(850.96630146,308.44808622)(850.97630371,308.4180896)
\curveto(850.98630144,308.38808628)(850.99130143,308.35808631)(850.99130371,308.3280896)
\curveto(850.99130143,308.29808637)(850.99630143,308.26308641)(851.00630371,308.2230896)
\curveto(851.0263014,308.14308653)(851.04130138,308.06308661)(851.05130371,307.9830896)
\lineto(851.11130371,307.7430896)
\curveto(851.2213012,307.40308727)(851.37130105,307.10308757)(851.56130371,306.8430896)
\curveto(851.76130066,306.59308808)(852.0213004,306.39808827)(852.34130371,306.2580896)
\curveto(852.53129989,306.17808849)(852.7262997,306.11808855)(852.92630371,306.0780896)
\curveto(852.96629946,306.05808861)(853.00629942,306.04808862)(853.04630371,306.0480896)
\curveto(853.08629934,306.05808861)(853.1262993,306.05808861)(853.16630371,306.0480896)
\lineto(853.28630371,306.0480896)
\curveto(853.35629907,306.02808864)(853.426299,306.02808864)(853.49630371,306.0480896)
\lineto(853.61630371,306.0480896)
\curveto(853.7262987,306.0680886)(853.83129859,306.08308859)(853.93130371,306.0930896)
\curveto(854.03129839,306.10308857)(854.13129829,306.12808854)(854.23130371,306.1680896)
\curveto(854.54129788,306.29808837)(854.79129763,306.4680882)(854.98130371,306.6780896)
\curveto(855.18129724,306.89808777)(855.34629708,307.16308751)(855.47630371,307.4730896)
\curveto(855.5262969,307.61308706)(855.56129686,307.75308692)(855.58130371,307.8930896)
\curveto(855.61129681,308.04308663)(855.64629678,308.19808647)(855.68630371,308.3580896)
\curveto(855.69629673,308.40808626)(855.70129672,308.45308622)(855.70130371,308.4930896)
\curveto(855.70129672,308.53308614)(855.70629672,308.57808609)(855.71630371,308.6280896)
\lineto(855.71630371,308.7480896)
}
}
{
\newrgbcolor{curcolor}{0 0 0}
\pscustom[linestyle=none,fillstyle=solid,fillcolor=curcolor]
{
\newpath
\moveto(858.99755371,312.8130896)
\lineto(859.43255371,312.8130896)
\curveto(859.58255175,312.81308186)(859.68755164,312.7730819)(859.74755371,312.6930896)
\curveto(859.79755153,312.61308206)(859.82255151,312.51308216)(859.82255371,312.3930896)
\curveto(859.8325515,312.2730824)(859.83755149,312.15308252)(859.83755371,312.0330896)
\lineto(859.83755371,310.6080896)
\lineto(859.83755371,308.3430896)
\lineto(859.83755371,307.6530896)
\curveto(859.83755149,307.42308725)(859.86255147,307.22308745)(859.91255371,307.0530896)
\curveto(860.07255126,306.60308807)(860.37255096,306.28808838)(860.81255371,306.1080896)
\curveto(861.0325503,306.01808865)(861.29755003,305.98308869)(861.60755371,306.0030896)
\curveto(861.91754941,306.03308864)(862.16754916,306.08808858)(862.35755371,306.1680896)
\curveto(862.68754864,306.30808836)(862.94754838,306.48308819)(863.13755371,306.6930896)
\curveto(863.33754799,306.91308776)(863.49254784,307.19808747)(863.60255371,307.5480896)
\curveto(863.6325477,307.62808704)(863.65254768,307.70808696)(863.66255371,307.7880896)
\curveto(863.67254766,307.8680868)(863.68754764,307.95308672)(863.70755371,308.0430896)
\curveto(863.71754761,308.09308658)(863.71754761,308.13808653)(863.70755371,308.1780896)
\curveto(863.70754762,308.21808645)(863.71754761,308.26308641)(863.73755371,308.3130896)
\lineto(863.73755371,308.6280896)
\curveto(863.75754757,308.70808596)(863.76254757,308.79808587)(863.75255371,308.8980896)
\curveto(863.74254759,309.00808566)(863.73754759,309.10808556)(863.73755371,309.1980896)
\lineto(863.73755371,310.3680896)
\lineto(863.73755371,311.9580896)
\curveto(863.73754759,312.07808259)(863.7325476,312.20308247)(863.72255371,312.3330896)
\curveto(863.72254761,312.4730822)(863.74754758,312.58308209)(863.79755371,312.6630896)
\curveto(863.83754749,312.71308196)(863.88254745,312.74308193)(863.93255371,312.7530896)
\curveto(863.99254734,312.7730819)(864.06254727,312.79308188)(864.14255371,312.8130896)
\lineto(864.36755371,312.8130896)
\curveto(864.48754684,312.81308186)(864.59254674,312.80808186)(864.68255371,312.7980896)
\curveto(864.78254655,312.78808188)(864.85754647,312.74308193)(864.90755371,312.6630896)
\curveto(864.95754637,312.61308206)(864.98254635,312.53808213)(864.98255371,312.4380896)
\lineto(864.98255371,312.1530896)
\lineto(864.98255371,311.1330896)
\lineto(864.98255371,307.0980896)
\lineto(864.98255371,305.7480896)
\curveto(864.98254635,305.62808904)(864.97754635,305.51308916)(864.96755371,305.4030896)
\curveto(864.96754636,305.30308937)(864.9325464,305.22808944)(864.86255371,305.1780896)
\curveto(864.82254651,305.14808952)(864.76254657,305.12308955)(864.68255371,305.1030896)
\curveto(864.60254673,305.09308958)(864.51254682,305.08308959)(864.41255371,305.0730896)
\curveto(864.32254701,305.0730896)(864.2325471,305.07808959)(864.14255371,305.0880896)
\curveto(864.06254727,305.09808957)(864.00254733,305.11808955)(863.96255371,305.1480896)
\curveto(863.91254742,305.18808948)(863.86754746,305.25308942)(863.82755371,305.3430896)
\curveto(863.81754751,305.38308929)(863.80754752,305.43808923)(863.79755371,305.5080896)
\curveto(863.79754753,305.57808909)(863.79254754,305.64308903)(863.78255371,305.7030896)
\curveto(863.77254756,305.7730889)(863.75254758,305.82808884)(863.72255371,305.8680896)
\curveto(863.69254764,305.90808876)(863.64754768,305.92308875)(863.58755371,305.9130896)
\curveto(863.50754782,305.89308878)(863.4275479,305.83308884)(863.34755371,305.7330896)
\curveto(863.26754806,305.64308903)(863.19254814,305.5730891)(863.12255371,305.5230896)
\curveto(862.90254843,305.36308931)(862.65254868,305.22308945)(862.37255371,305.1030896)
\curveto(862.26254907,305.05308962)(862.14754918,305.02308965)(862.02755371,305.0130896)
\curveto(861.91754941,304.99308968)(861.80254953,304.9680897)(861.68255371,304.9380896)
\curveto(861.6325497,304.92808974)(861.57754975,304.92808974)(861.51755371,304.9380896)
\curveto(861.46754986,304.94808972)(861.41754991,304.94308973)(861.36755371,304.9230896)
\curveto(861.26755006,304.90308977)(861.17755015,304.90308977)(861.09755371,304.9230896)
\lineto(860.94755371,304.9230896)
\curveto(860.89755043,304.94308973)(860.83755049,304.95308972)(860.76755371,304.9530896)
\curveto(860.70755062,304.95308972)(860.65255068,304.95808971)(860.60255371,304.9680896)
\curveto(860.56255077,304.98808968)(860.52255081,304.99808967)(860.48255371,304.9980896)
\curveto(860.45255088,304.98808968)(860.41255092,304.99308968)(860.36255371,305.0130896)
\lineto(860.12255371,305.0730896)
\curveto(860.05255128,305.09308958)(859.97755135,305.12308955)(859.89755371,305.1630896)
\curveto(859.63755169,305.2730894)(859.41755191,305.41808925)(859.23755371,305.5980896)
\curveto(859.06755226,305.78808888)(858.9275524,306.01308866)(858.81755371,306.2730896)
\curveto(858.77755255,306.36308831)(858.74755258,306.45308822)(858.72755371,306.5430896)
\lineto(858.66755371,306.8430896)
\curveto(858.64755268,306.90308777)(858.63755269,306.95808771)(858.63755371,307.0080896)
\curveto(858.64755268,307.0680876)(858.64255269,307.13308754)(858.62255371,307.2030896)
\curveto(858.61255272,307.22308745)(858.60755272,307.24808742)(858.60755371,307.2780896)
\curveto(858.60755272,307.31808735)(858.60255273,307.35308732)(858.59255371,307.3830896)
\lineto(858.59255371,307.5330896)
\curveto(858.58255275,307.5730871)(858.57755275,307.61808705)(858.57755371,307.6680896)
\curveto(858.58755274,307.72808694)(858.59255274,307.78308689)(858.59255371,307.8330896)
\lineto(858.59255371,308.4330896)
\lineto(858.59255371,311.1930896)
\lineto(858.59255371,312.1530896)
\lineto(858.59255371,312.4230896)
\curveto(858.59255274,312.51308216)(858.61255272,312.58808208)(858.65255371,312.6480896)
\curveto(858.69255264,312.71808195)(858.76755256,312.7680819)(858.87755371,312.7980896)
\curveto(858.89755243,312.80808186)(858.91755241,312.80808186)(858.93755371,312.7980896)
\curveto(858.95755237,312.79808187)(858.97755235,312.80308187)(858.99755371,312.8130896)
}
}
{
\newrgbcolor{curcolor}{0 0 0}
\pscustom[linestyle=none,fillstyle=solid,fillcolor=curcolor]
{
\newpath
\moveto(867.45216309,315.7680896)
\curveto(867.58216147,315.7680789)(867.71716134,315.7680789)(867.85716309,315.7680896)
\curveto(868.00716105,315.7680789)(868.11716094,315.73307894)(868.18716309,315.6630896)
\curveto(868.23716082,315.59307908)(868.26216079,315.49807917)(868.26216309,315.3780896)
\curveto(868.27216078,315.2680794)(868.27716078,315.15307952)(868.27716309,315.0330896)
\lineto(868.27716309,313.6980896)
\lineto(868.27716309,307.6230896)
\lineto(868.27716309,305.9430896)
\lineto(868.27716309,305.5530896)
\curveto(868.27716078,305.41308926)(868.2521608,305.30308937)(868.20216309,305.2230896)
\curveto(868.17216088,305.1730895)(868.12716093,305.14308953)(868.06716309,305.1330896)
\curveto(868.01716104,305.12308955)(867.9521611,305.10808956)(867.87216309,305.0880896)
\lineto(867.66216309,305.0880896)
\lineto(867.34716309,305.0880896)
\curveto(867.24716181,305.09808957)(867.17216188,305.13308954)(867.12216309,305.1930896)
\curveto(867.07216198,305.2730894)(867.04216201,305.3730893)(867.03216309,305.4930896)
\lineto(867.03216309,305.8680896)
\lineto(867.03216309,307.2480896)
\lineto(867.03216309,313.4880896)
\lineto(867.03216309,314.9580896)
\curveto(867.03216202,315.0680796)(867.02716203,315.18307949)(867.01716309,315.3030896)
\curveto(867.01716204,315.43307924)(867.04216201,315.53307914)(867.09216309,315.6030896)
\curveto(867.13216192,315.66307901)(867.20716185,315.71307896)(867.31716309,315.7530896)
\curveto(867.33716172,315.76307891)(867.3571617,315.76307891)(867.37716309,315.7530896)
\curveto(867.40716165,315.75307892)(867.43216162,315.75807891)(867.45216309,315.7680896)
}
}
{
\newrgbcolor{curcolor}{0 0 0}
\pscustom[linestyle=none,fillstyle=solid,fillcolor=curcolor]
{
\newpath
\moveto(877.10700684,305.6430896)
\curveto(877.13699901,305.48308919)(877.12199902,305.34808932)(877.06200684,305.2380896)
\curveto(877.00199914,305.13808953)(876.92199922,305.06308961)(876.82200684,305.0130896)
\curveto(876.77199937,304.99308968)(876.71699943,304.98308969)(876.65700684,304.9830896)
\curveto(876.60699954,304.98308969)(876.55199959,304.9730897)(876.49200684,304.9530896)
\curveto(876.27199987,304.90308977)(876.05200009,304.91808975)(875.83200684,304.9980896)
\curveto(875.62200052,305.0680896)(875.47700067,305.15808951)(875.39700684,305.2680896)
\curveto(875.3470008,305.33808933)(875.30200084,305.41808925)(875.26200684,305.5080896)
\curveto(875.22200092,305.60808906)(875.17200097,305.68808898)(875.11200684,305.7480896)
\curveto(875.09200105,305.7680889)(875.06700108,305.78808888)(875.03700684,305.8080896)
\curveto(875.01700113,305.82808884)(874.98700116,305.83308884)(874.94700684,305.8230896)
\curveto(874.83700131,305.79308888)(874.73200141,305.73808893)(874.63200684,305.6580896)
\curveto(874.5420016,305.57808909)(874.45200169,305.50808916)(874.36200684,305.4480896)
\curveto(874.23200191,305.3680893)(874.09200205,305.29308938)(873.94200684,305.2230896)
\curveto(873.79200235,305.16308951)(873.63200251,305.10808956)(873.46200684,305.0580896)
\curveto(873.36200278,305.02808964)(873.25200289,305.00808966)(873.13200684,304.9980896)
\curveto(873.02200312,304.98808968)(872.91200323,304.9730897)(872.80200684,304.9530896)
\curveto(872.75200339,304.94308973)(872.70700344,304.93808973)(872.66700684,304.9380896)
\lineto(872.56200684,304.9380896)
\curveto(872.45200369,304.91808975)(872.3470038,304.91808975)(872.24700684,304.9380896)
\lineto(872.11200684,304.9380896)
\curveto(872.06200408,304.94808972)(872.01200413,304.95308972)(871.96200684,304.9530896)
\curveto(871.91200423,304.95308972)(871.86700428,304.96308971)(871.82700684,304.9830896)
\curveto(871.78700436,304.99308968)(871.75200439,304.99808967)(871.72200684,304.9980896)
\curveto(871.70200444,304.98808968)(871.67700447,304.98808968)(871.64700684,304.9980896)
\lineto(871.40700684,305.0580896)
\curveto(871.32700482,305.0680896)(871.25200489,305.08808958)(871.18200684,305.1180896)
\curveto(870.88200526,305.24808942)(870.63700551,305.39308928)(870.44700684,305.5530896)
\curveto(870.26700588,305.72308895)(870.11700603,305.95808871)(869.99700684,306.2580896)
\curveto(869.90700624,306.47808819)(869.86200628,306.74308793)(869.86200684,307.0530896)
\lineto(869.86200684,307.3680896)
\curveto(869.87200627,307.41808725)(869.87700627,307.4680872)(869.87700684,307.5180896)
\lineto(869.90700684,307.6980896)
\lineto(870.02700684,308.0280896)
\curveto(870.06700608,308.13808653)(870.11700603,308.23808643)(870.17700684,308.3280896)
\curveto(870.35700579,308.61808605)(870.60200554,308.83308584)(870.91200684,308.9730896)
\curveto(871.22200492,309.11308556)(871.56200458,309.23808543)(871.93200684,309.3480896)
\curveto(872.07200407,309.38808528)(872.21700393,309.41808525)(872.36700684,309.4380896)
\curveto(872.51700363,309.45808521)(872.66700348,309.48308519)(872.81700684,309.5130896)
\curveto(872.88700326,309.53308514)(872.95200319,309.54308513)(873.01200684,309.5430896)
\curveto(873.08200306,309.54308513)(873.15700299,309.55308512)(873.23700684,309.5730896)
\curveto(873.30700284,309.59308508)(873.37700277,309.60308507)(873.44700684,309.6030896)
\curveto(873.51700263,309.61308506)(873.59200255,309.62808504)(873.67200684,309.6480896)
\curveto(873.92200222,309.70808496)(874.15700199,309.75808491)(874.37700684,309.7980896)
\curveto(874.59700155,309.84808482)(874.77200137,309.96308471)(874.90200684,310.1430896)
\curveto(874.96200118,310.22308445)(875.01200113,310.32308435)(875.05200684,310.4430896)
\curveto(875.09200105,310.5730841)(875.09200105,310.71308396)(875.05200684,310.8630896)
\curveto(874.99200115,311.10308357)(874.90200124,311.29308338)(874.78200684,311.4330896)
\curveto(874.67200147,311.5730831)(874.51200163,311.68308299)(874.30200684,311.7630896)
\curveto(874.18200196,311.81308286)(874.03700211,311.84808282)(873.86700684,311.8680896)
\curveto(873.70700244,311.88808278)(873.53700261,311.89808277)(873.35700684,311.8980896)
\curveto(873.17700297,311.89808277)(873.00200314,311.88808278)(872.83200684,311.8680896)
\curveto(872.66200348,311.84808282)(872.51700363,311.81808285)(872.39700684,311.7780896)
\curveto(872.22700392,311.71808295)(872.06200408,311.63308304)(871.90200684,311.5230896)
\curveto(871.82200432,311.46308321)(871.7470044,311.38308329)(871.67700684,311.2830896)
\curveto(871.61700453,311.19308348)(871.56200458,311.09308358)(871.51200684,310.9830896)
\curveto(871.48200466,310.90308377)(871.45200469,310.81808385)(871.42200684,310.7280896)
\curveto(871.40200474,310.63808403)(871.35700479,310.5680841)(871.28700684,310.5180896)
\curveto(871.2470049,310.48808418)(871.17700497,310.46308421)(871.07700684,310.4430896)
\curveto(870.98700516,310.43308424)(870.89200525,310.42808424)(870.79200684,310.4280896)
\curveto(870.69200545,310.42808424)(870.59200555,310.43308424)(870.49200684,310.4430896)
\curveto(870.40200574,310.46308421)(870.33700581,310.48808418)(870.29700684,310.5180896)
\curveto(870.25700589,310.54808412)(870.22700592,310.59808407)(870.20700684,310.6680896)
\curveto(870.18700596,310.73808393)(870.18700596,310.81308386)(870.20700684,310.8930896)
\curveto(870.23700591,311.02308365)(870.26700588,311.14308353)(870.29700684,311.2530896)
\curveto(870.33700581,311.3730833)(870.38200576,311.48808318)(870.43200684,311.5980896)
\curveto(870.62200552,311.94808272)(870.86200528,312.21808245)(871.15200684,312.4080896)
\curveto(871.4420047,312.60808206)(871.80200434,312.7680819)(872.23200684,312.8880896)
\curveto(872.33200381,312.90808176)(872.43200371,312.92308175)(872.53200684,312.9330896)
\curveto(872.6420035,312.94308173)(872.75200339,312.95808171)(872.86200684,312.9780896)
\curveto(872.90200324,312.98808168)(872.96700318,312.98808168)(873.05700684,312.9780896)
\curveto(873.147003,312.97808169)(873.20200294,312.98808168)(873.22200684,313.0080896)
\curveto(873.92200222,313.01808165)(874.53200161,312.93808173)(875.05200684,312.7680896)
\curveto(875.57200057,312.59808207)(875.93700021,312.2730824)(876.14700684,311.7930896)
\curveto(876.23699991,311.59308308)(876.28699986,311.35808331)(876.29700684,311.0880896)
\curveto(876.31699983,310.82808384)(876.32699982,310.55308412)(876.32700684,310.2630896)
\lineto(876.32700684,306.9480896)
\curveto(876.32699982,306.80808786)(876.33199981,306.673088)(876.34200684,306.5430896)
\curveto(876.35199979,306.41308826)(876.38199976,306.30808836)(876.43200684,306.2280896)
\curveto(876.48199966,306.15808851)(876.5469996,306.10808856)(876.62700684,306.0780896)
\curveto(876.71699943,306.03808863)(876.80199934,306.00808866)(876.88200684,305.9880896)
\curveto(876.96199918,305.97808869)(877.02199912,305.93308874)(877.06200684,305.8530896)
\curveto(877.08199906,305.82308885)(877.09199905,305.79308888)(877.09200684,305.7630896)
\curveto(877.09199905,305.73308894)(877.09699905,305.69308898)(877.10700684,305.6430896)
\moveto(874.96200684,307.3080896)
\curveto(875.02200112,307.44808722)(875.05200109,307.60808706)(875.05200684,307.7880896)
\curveto(875.06200108,307.97808669)(875.06700108,308.1730865)(875.06700684,308.3730896)
\curveto(875.06700108,308.48308619)(875.06200108,308.58308609)(875.05200684,308.6730896)
\curveto(875.0420011,308.76308591)(875.00200114,308.83308584)(874.93200684,308.8830896)
\curveto(874.90200124,308.90308577)(874.83200131,308.91308576)(874.72200684,308.9130896)
\curveto(874.70200144,308.89308578)(874.66700148,308.88308579)(874.61700684,308.8830896)
\curveto(874.56700158,308.88308579)(874.52200162,308.8730858)(874.48200684,308.8530896)
\curveto(874.40200174,308.83308584)(874.31200183,308.81308586)(874.21200684,308.7930896)
\lineto(873.91200684,308.7330896)
\curveto(873.88200226,308.73308594)(873.8470023,308.72808594)(873.80700684,308.7180896)
\lineto(873.70200684,308.7180896)
\curveto(873.55200259,308.67808599)(873.38700276,308.65308602)(873.20700684,308.6430896)
\curveto(873.03700311,308.64308603)(872.87700327,308.62308605)(872.72700684,308.5830896)
\curveto(872.6470035,308.56308611)(872.57200357,308.54308613)(872.50200684,308.5230896)
\curveto(872.4420037,308.51308616)(872.37200377,308.49808617)(872.29200684,308.4780896)
\curveto(872.13200401,308.42808624)(871.98200416,308.36308631)(871.84200684,308.2830896)
\curveto(871.70200444,308.21308646)(871.58200456,308.12308655)(871.48200684,308.0130896)
\curveto(871.38200476,307.90308677)(871.30700484,307.7680869)(871.25700684,307.6080896)
\curveto(871.20700494,307.45808721)(871.18700496,307.2730874)(871.19700684,307.0530896)
\curveto(871.19700495,306.95308772)(871.21200493,306.85808781)(871.24200684,306.7680896)
\curveto(871.28200486,306.68808798)(871.32700482,306.61308806)(871.37700684,306.5430896)
\curveto(871.45700469,306.43308824)(871.56200458,306.33808833)(871.69200684,306.2580896)
\curveto(871.82200432,306.18808848)(871.96200418,306.12808854)(872.11200684,306.0780896)
\curveto(872.16200398,306.0680886)(872.21200393,306.06308861)(872.26200684,306.0630896)
\curveto(872.31200383,306.06308861)(872.36200378,306.05808861)(872.41200684,306.0480896)
\curveto(872.48200366,306.02808864)(872.56700358,306.01308866)(872.66700684,306.0030896)
\curveto(872.77700337,306.00308867)(872.86700328,306.01308866)(872.93700684,306.0330896)
\curveto(872.99700315,306.05308862)(873.05700309,306.05808861)(873.11700684,306.0480896)
\curveto(873.17700297,306.04808862)(873.23700291,306.05808861)(873.29700684,306.0780896)
\curveto(873.37700277,306.09808857)(873.45200269,306.11308856)(873.52200684,306.1230896)
\curveto(873.60200254,306.13308854)(873.67700247,306.15308852)(873.74700684,306.1830896)
\curveto(874.03700211,306.30308837)(874.28200186,306.44808822)(874.48200684,306.6180896)
\curveto(874.69200145,306.78808788)(874.85200129,307.01808765)(874.96200684,307.3080896)
}
}
{
\newrgbcolor{curcolor}{0 0 0}
\pscustom[linestyle=none,fillstyle=solid,fillcolor=curcolor]
{
\newpath
\moveto(881.92364746,312.9930896)
\curveto(882.15364267,312.99308168)(882.28364254,312.93308174)(882.31364746,312.8130896)
\curveto(882.34364248,312.70308197)(882.35864247,312.53808213)(882.35864746,312.3180896)
\lineto(882.35864746,312.0330896)
\curveto(882.35864247,311.94308273)(882.33364249,311.8680828)(882.28364746,311.8080896)
\curveto(882.2236426,311.72808294)(882.13864269,311.68308299)(882.02864746,311.6730896)
\curveto(881.91864291,311.673083)(881.80864302,311.65808301)(881.69864746,311.6280896)
\curveto(881.55864327,311.59808307)(881.4236434,311.5680831)(881.29364746,311.5380896)
\curveto(881.17364365,311.50808316)(881.05864377,311.4680832)(880.94864746,311.4180896)
\curveto(880.65864417,311.28808338)(880.4236444,311.10808356)(880.24364746,310.8780896)
\curveto(880.06364476,310.65808401)(879.90864492,310.40308427)(879.77864746,310.1130896)
\curveto(879.73864509,310.00308467)(879.70864512,309.88808478)(879.68864746,309.7680896)
\curveto(879.66864516,309.65808501)(879.64364518,309.54308513)(879.61364746,309.4230896)
\curveto(879.60364522,309.3730853)(879.59864523,309.32308535)(879.59864746,309.2730896)
\curveto(879.60864522,309.22308545)(879.60864522,309.1730855)(879.59864746,309.1230896)
\curveto(879.56864526,309.00308567)(879.55364527,308.86308581)(879.55364746,308.7030896)
\curveto(879.56364526,308.55308612)(879.56864526,308.40808626)(879.56864746,308.2680896)
\lineto(879.56864746,306.4230896)
\lineto(879.56864746,306.0780896)
\curveto(879.56864526,305.95808871)(879.56364526,305.84308883)(879.55364746,305.7330896)
\curveto(879.54364528,305.62308905)(879.53864529,305.52808914)(879.53864746,305.4480896)
\curveto(879.54864528,305.3680893)(879.5286453,305.29808937)(879.47864746,305.2380896)
\curveto(879.4286454,305.1680895)(879.34864548,305.12808954)(879.23864746,305.1180896)
\curveto(879.13864569,305.10808956)(879.0286458,305.10308957)(878.90864746,305.1030896)
\lineto(878.63864746,305.1030896)
\curveto(878.58864624,305.12308955)(878.53864629,305.13808953)(878.48864746,305.1480896)
\curveto(878.44864638,305.1680895)(878.41864641,305.19308948)(878.39864746,305.2230896)
\curveto(878.34864648,305.29308938)(878.31864651,305.37808929)(878.30864746,305.4780896)
\lineto(878.30864746,305.8080896)
\lineto(878.30864746,306.9630896)
\lineto(878.30864746,311.1180896)
\lineto(878.30864746,312.1530896)
\lineto(878.30864746,312.4530896)
\curveto(878.31864651,312.55308212)(878.34864648,312.63808203)(878.39864746,312.7080896)
\curveto(878.4286464,312.74808192)(878.47864635,312.77808189)(878.54864746,312.7980896)
\curveto(878.6286462,312.81808185)(878.71364611,312.82808184)(878.80364746,312.8280896)
\curveto(878.89364593,312.83808183)(878.98364584,312.83808183)(879.07364746,312.8280896)
\curveto(879.16364566,312.81808185)(879.23364559,312.80308187)(879.28364746,312.7830896)
\curveto(879.36364546,312.75308192)(879.41364541,312.69308198)(879.43364746,312.6030896)
\curveto(879.46364536,312.52308215)(879.47864535,312.43308224)(879.47864746,312.3330896)
\lineto(879.47864746,312.0330896)
\curveto(879.47864535,311.93308274)(879.49864533,311.84308283)(879.53864746,311.7630896)
\curveto(879.54864528,311.74308293)(879.55864527,311.72808294)(879.56864746,311.7180896)
\lineto(879.61364746,311.6730896)
\curveto(879.7236451,311.673083)(879.81364501,311.71808295)(879.88364746,311.8080896)
\curveto(879.95364487,311.90808276)(880.01364481,311.98808268)(880.06364746,312.0480896)
\lineto(880.15364746,312.1380896)
\curveto(880.24364458,312.24808242)(880.36864446,312.36308231)(880.52864746,312.4830896)
\curveto(880.68864414,312.60308207)(880.83864399,312.69308198)(880.97864746,312.7530896)
\curveto(881.06864376,312.80308187)(881.16364366,312.83808183)(881.26364746,312.8580896)
\curveto(881.36364346,312.88808178)(881.46864336,312.91808175)(881.57864746,312.9480896)
\curveto(881.63864319,312.95808171)(881.69864313,312.96308171)(881.75864746,312.9630896)
\curveto(881.81864301,312.9730817)(881.87364295,312.98308169)(881.92364746,312.9930896)
}
}
{
\newrgbcolor{curcolor}{0 0 0}
\pscustom[linestyle=none,fillstyle=solid,fillcolor=curcolor]
{
\newpath
\moveto(883.57341309,314.3130896)
\curveto(883.49341197,314.3730803)(883.44841201,314.47808019)(883.43841309,314.6280896)
\lineto(883.43841309,315.0930896)
\lineto(883.43841309,315.3480896)
\curveto(883.43841202,315.43807923)(883.45341201,315.51307916)(883.48341309,315.5730896)
\curveto(883.52341194,315.65307902)(883.60341186,315.71307896)(883.72341309,315.7530896)
\curveto(883.74341172,315.76307891)(883.7634117,315.76307891)(883.78341309,315.7530896)
\curveto(883.81341165,315.75307892)(883.83841162,315.75807891)(883.85841309,315.7680896)
\curveto(884.02841143,315.7680789)(884.18841127,315.76307891)(884.33841309,315.7530896)
\curveto(884.48841097,315.74307893)(884.58841087,315.68307899)(884.63841309,315.5730896)
\curveto(884.66841079,315.51307916)(884.68341078,315.43807923)(884.68341309,315.3480896)
\lineto(884.68341309,315.0930896)
\curveto(884.68341078,314.91307976)(884.67841078,314.74307993)(884.66841309,314.5830896)
\curveto(884.66841079,314.42308025)(884.60341086,314.31808035)(884.47341309,314.2680896)
\curveto(884.42341104,314.24808042)(884.36841109,314.23808043)(884.30841309,314.2380896)
\lineto(884.14341309,314.2380896)
\lineto(883.82841309,314.2380896)
\curveto(883.72841173,314.23808043)(883.64341182,314.26308041)(883.57341309,314.3130896)
\moveto(884.68341309,305.8080896)
\lineto(884.68341309,305.4930896)
\curveto(884.69341077,305.39308928)(884.67341079,305.31308936)(884.62341309,305.2530896)
\curveto(884.59341087,305.19308948)(884.54841091,305.15308952)(884.48841309,305.1330896)
\curveto(884.42841103,305.12308955)(884.3584111,305.10808956)(884.27841309,305.0880896)
\lineto(884.05341309,305.0880896)
\curveto(883.92341154,305.08808958)(883.80841165,305.09308958)(883.70841309,305.1030896)
\curveto(883.61841184,305.12308955)(883.54841191,305.1730895)(883.49841309,305.2530896)
\curveto(883.458412,305.31308936)(883.43841202,305.38808928)(883.43841309,305.4780896)
\lineto(883.43841309,305.7630896)
\lineto(883.43841309,312.1080896)
\lineto(883.43841309,312.4230896)
\curveto(883.43841202,312.53308214)(883.463412,312.61808205)(883.51341309,312.6780896)
\curveto(883.54341192,312.72808194)(883.58341188,312.75808191)(883.63341309,312.7680896)
\curveto(883.68341178,312.77808189)(883.73841172,312.79308188)(883.79841309,312.8130896)
\curveto(883.81841164,312.81308186)(883.83841162,312.80808186)(883.85841309,312.7980896)
\curveto(883.88841157,312.79808187)(883.91341155,312.80308187)(883.93341309,312.8130896)
\curveto(884.0634114,312.81308186)(884.19341127,312.80808186)(884.32341309,312.7980896)
\curveto(884.463411,312.79808187)(884.5584109,312.75808191)(884.60841309,312.6780896)
\curveto(884.6584108,312.61808205)(884.68341078,312.53808213)(884.68341309,312.4380896)
\lineto(884.68341309,312.1530896)
\lineto(884.68341309,305.8080896)
}
}
{
\newrgbcolor{curcolor}{0 0 0}
\pscustom[linestyle=none,fillstyle=solid,fillcolor=curcolor]
{
\newpath
\moveto(893.58825684,305.8980896)
\lineto(893.58825684,305.5080896)
\curveto(893.58824896,305.38808928)(893.56324899,305.28808938)(893.51325684,305.2080896)
\curveto(893.46324909,305.13808953)(893.37824917,305.09808957)(893.25825684,305.0880896)
\lineto(892.91325684,305.0880896)
\curveto(892.8532497,305.08808958)(892.79324976,305.08308959)(892.73325684,305.0730896)
\curveto(892.68324987,305.0730896)(892.63824991,305.08308959)(892.59825684,305.1030896)
\curveto(892.50825004,305.12308955)(892.4482501,305.16308951)(892.41825684,305.2230896)
\curveto(892.37825017,305.2730894)(892.3532502,305.33308934)(892.34325684,305.4030896)
\curveto(892.34325021,305.4730892)(892.32825022,305.54308913)(892.29825684,305.6130896)
\curveto(892.28825026,305.63308904)(892.27325028,305.64808902)(892.25325684,305.6580896)
\curveto(892.24325031,305.67808899)(892.22825032,305.69808897)(892.20825684,305.7180896)
\curveto(892.10825044,305.72808894)(892.02825052,305.70808896)(891.96825684,305.6580896)
\curveto(891.91825063,305.60808906)(891.86325069,305.55808911)(891.80325684,305.5080896)
\curveto(891.60325095,305.35808931)(891.40325115,305.24308943)(891.20325684,305.1630896)
\curveto(891.02325153,305.08308959)(890.81325174,305.02308965)(890.57325684,304.9830896)
\curveto(890.34325221,304.94308973)(890.10325245,304.92308975)(889.85325684,304.9230896)
\curveto(889.61325294,304.91308976)(889.37325318,304.92808974)(889.13325684,304.9680896)
\curveto(888.89325366,304.99808967)(888.68325387,305.05308962)(888.50325684,305.1330896)
\curveto(887.98325457,305.35308932)(887.56325499,305.64808902)(887.24325684,306.0180896)
\curveto(886.92325563,306.39808827)(886.67325588,306.8680878)(886.49325684,307.4280896)
\curveto(886.4532561,307.51808715)(886.42325613,307.60808706)(886.40325684,307.6980896)
\curveto(886.39325616,307.79808687)(886.37325618,307.89808677)(886.34325684,307.9980896)
\curveto(886.33325622,308.04808662)(886.32825622,308.09808657)(886.32825684,308.1480896)
\curveto(886.32825622,308.19808647)(886.32325623,308.24808642)(886.31325684,308.2980896)
\curveto(886.29325626,308.34808632)(886.28325627,308.39808627)(886.28325684,308.4480896)
\curveto(886.29325626,308.50808616)(886.29325626,308.56308611)(886.28325684,308.6130896)
\lineto(886.28325684,308.7630896)
\curveto(886.26325629,308.81308586)(886.2532563,308.87808579)(886.25325684,308.9580896)
\curveto(886.2532563,309.03808563)(886.26325629,309.10308557)(886.28325684,309.1530896)
\lineto(886.28325684,309.3180896)
\curveto(886.30325625,309.38808528)(886.30825624,309.45808521)(886.29825684,309.5280896)
\curveto(886.29825625,309.60808506)(886.30825624,309.68308499)(886.32825684,309.7530896)
\curveto(886.33825621,309.80308487)(886.34325621,309.84808482)(886.34325684,309.8880896)
\curveto(886.34325621,309.92808474)(886.3482562,309.9730847)(886.35825684,310.0230896)
\curveto(886.38825616,310.12308455)(886.41325614,310.21808445)(886.43325684,310.3080896)
\curveto(886.4532561,310.40808426)(886.47825607,310.50308417)(886.50825684,310.5930896)
\curveto(886.63825591,310.9730837)(886.80325575,311.31308336)(887.00325684,311.6130896)
\curveto(887.21325534,311.92308275)(887.46325509,312.17808249)(887.75325684,312.3780896)
\curveto(887.92325463,312.49808217)(888.09825445,312.59808207)(888.27825684,312.6780896)
\curveto(888.46825408,312.75808191)(888.67325388,312.82808184)(888.89325684,312.8880896)
\curveto(888.96325359,312.89808177)(889.02825352,312.90808176)(889.08825684,312.9180896)
\curveto(889.15825339,312.92808174)(889.22825332,312.94308173)(889.29825684,312.9630896)
\lineto(889.44825684,312.9630896)
\curveto(889.52825302,312.98308169)(889.64325291,312.99308168)(889.79325684,312.9930896)
\curveto(889.9532526,312.99308168)(890.07325248,312.98308169)(890.15325684,312.9630896)
\curveto(890.19325236,312.95308172)(890.2482523,312.94808172)(890.31825684,312.9480896)
\curveto(890.42825212,312.91808175)(890.53825201,312.89308178)(890.64825684,312.8730896)
\curveto(890.75825179,312.86308181)(890.86325169,312.83308184)(890.96325684,312.7830896)
\curveto(891.11325144,312.72308195)(891.2532513,312.65808201)(891.38325684,312.5880896)
\curveto(891.52325103,312.51808215)(891.6532509,312.43808223)(891.77325684,312.3480896)
\curveto(891.83325072,312.29808237)(891.89325066,312.24308243)(891.95325684,312.1830896)
\curveto(892.02325053,312.13308254)(892.11325044,312.11808255)(892.22325684,312.1380896)
\curveto(892.24325031,312.1680825)(892.25825029,312.19308248)(892.26825684,312.2130896)
\curveto(892.28825026,312.23308244)(892.30325025,312.26308241)(892.31325684,312.3030896)
\curveto(892.34325021,312.39308228)(892.3532502,312.50808216)(892.34325684,312.6480896)
\lineto(892.34325684,313.0230896)
\lineto(892.34325684,314.7480896)
\lineto(892.34325684,315.2130896)
\curveto(892.34325021,315.39307928)(892.36825018,315.52307915)(892.41825684,315.6030896)
\curveto(892.45825009,315.673079)(892.51825003,315.71807895)(892.59825684,315.7380896)
\curveto(892.61824993,315.73807893)(892.64324991,315.73807893)(892.67325684,315.7380896)
\curveto(892.70324985,315.74807892)(892.72824982,315.75307892)(892.74825684,315.7530896)
\curveto(892.88824966,315.76307891)(893.03324952,315.76307891)(893.18325684,315.7530896)
\curveto(893.34324921,315.75307892)(893.4532491,315.71307896)(893.51325684,315.6330896)
\curveto(893.56324899,315.55307912)(893.58824896,315.45307922)(893.58825684,315.3330896)
\lineto(893.58825684,314.9580896)
\lineto(893.58825684,305.8980896)
\moveto(892.37325684,308.7330896)
\curveto(892.39325016,308.78308589)(892.40325015,308.84808582)(892.40325684,308.9280896)
\curveto(892.40325015,309.01808565)(892.39325016,309.08808558)(892.37325684,309.1380896)
\lineto(892.37325684,309.3630896)
\curveto(892.3532502,309.45308522)(892.33825021,309.54308513)(892.32825684,309.6330896)
\curveto(892.31825023,309.73308494)(892.29825025,309.82308485)(892.26825684,309.9030896)
\curveto(892.2482503,309.98308469)(892.22825032,310.05808461)(892.20825684,310.1280896)
\curveto(892.19825035,310.19808447)(892.17825037,310.2680844)(892.14825684,310.3380896)
\curveto(892.02825052,310.63808403)(891.87325068,310.90308377)(891.68325684,311.1330896)
\curveto(891.49325106,311.36308331)(891.2532513,311.54308313)(890.96325684,311.6730896)
\curveto(890.86325169,311.72308295)(890.75825179,311.75808291)(890.64825684,311.7780896)
\curveto(890.548252,311.80808286)(890.43825211,311.83308284)(890.31825684,311.8530896)
\curveto(890.23825231,311.8730828)(890.1482524,311.88308279)(890.04825684,311.8830896)
\lineto(889.77825684,311.8830896)
\curveto(889.72825282,311.8730828)(889.68325287,311.86308281)(889.64325684,311.8530896)
\lineto(889.50825684,311.8530896)
\curveto(889.42825312,311.83308284)(889.34325321,311.81308286)(889.25325684,311.7930896)
\curveto(889.17325338,311.7730829)(889.09325346,311.74808292)(889.01325684,311.7180896)
\curveto(888.69325386,311.57808309)(888.43325412,311.3730833)(888.23325684,311.1030896)
\curveto(888.04325451,310.84308383)(887.88825466,310.53808413)(887.76825684,310.1880896)
\curveto(887.72825482,310.07808459)(887.69825485,309.96308471)(887.67825684,309.8430896)
\curveto(887.66825488,309.73308494)(887.6532549,309.62308505)(887.63325684,309.5130896)
\curveto(887.63325492,309.4730852)(887.62825492,309.43308524)(887.61825684,309.3930896)
\lineto(887.61825684,309.2880896)
\curveto(887.59825495,309.23808543)(887.58825496,309.18308549)(887.58825684,309.1230896)
\curveto(887.59825495,309.06308561)(887.60325495,309.00808566)(887.60325684,308.9580896)
\lineto(887.60325684,308.6280896)
\curveto(887.60325495,308.52808614)(887.61325494,308.43308624)(887.63325684,308.3430896)
\curveto(887.64325491,308.31308636)(887.6482549,308.26308641)(887.64825684,308.1930896)
\curveto(887.66825488,308.12308655)(887.68325487,308.05308662)(887.69325684,307.9830896)
\lineto(887.75325684,307.7730896)
\curveto(887.86325469,307.42308725)(888.01325454,307.12308755)(888.20325684,306.8730896)
\curveto(888.39325416,306.62308805)(888.63325392,306.41808825)(888.92325684,306.2580896)
\curveto(889.01325354,306.20808846)(889.10325345,306.1680885)(889.19325684,306.1380896)
\curveto(889.28325327,306.10808856)(889.38325317,306.07808859)(889.49325684,306.0480896)
\curveto(889.54325301,306.02808864)(889.59325296,306.02308865)(889.64325684,306.0330896)
\curveto(889.70325285,306.04308863)(889.75825279,306.03808863)(889.80825684,306.0180896)
\curveto(889.8482527,306.00808866)(889.88825266,306.00308867)(889.92825684,306.0030896)
\lineto(890.06325684,306.0030896)
\lineto(890.19825684,306.0030896)
\curveto(890.22825232,306.01308866)(890.27825227,306.01808865)(890.34825684,306.0180896)
\curveto(890.42825212,306.03808863)(890.50825204,306.05308862)(890.58825684,306.0630896)
\curveto(890.66825188,306.08308859)(890.74325181,306.10808856)(890.81325684,306.1380896)
\curveto(891.14325141,306.27808839)(891.40825114,306.45308822)(891.60825684,306.6630896)
\curveto(891.81825073,306.88308779)(891.99325056,307.15808751)(892.13325684,307.4880896)
\curveto(892.18325037,307.59808707)(892.21825033,307.70808696)(892.23825684,307.8180896)
\curveto(892.25825029,307.92808674)(892.28325027,308.03808663)(892.31325684,308.1480896)
\curveto(892.33325022,308.18808648)(892.34325021,308.22308645)(892.34325684,308.2530896)
\curveto(892.34325021,308.29308638)(892.3482502,308.33308634)(892.35825684,308.3730896)
\curveto(892.36825018,308.43308624)(892.36825018,308.49308618)(892.35825684,308.5530896)
\curveto(892.35825019,308.61308606)(892.36325019,308.673086)(892.37325684,308.7330896)
}
}
{
\newrgbcolor{curcolor}{0 0 0}
\pscustom[linestyle=none,fillstyle=solid,fillcolor=curcolor]
{
\newpath
\moveto(902.41950684,305.6430896)
\curveto(902.44949901,305.48308919)(902.43449902,305.34808932)(902.37450684,305.2380896)
\curveto(902.31449914,305.13808953)(902.23449922,305.06308961)(902.13450684,305.0130896)
\curveto(902.08449937,304.99308968)(902.02949943,304.98308969)(901.96950684,304.9830896)
\curveto(901.91949954,304.98308969)(901.86449959,304.9730897)(901.80450684,304.9530896)
\curveto(901.58449987,304.90308977)(901.36450009,304.91808975)(901.14450684,304.9980896)
\curveto(900.93450052,305.0680896)(900.78950067,305.15808951)(900.70950684,305.2680896)
\curveto(900.6595008,305.33808933)(900.61450084,305.41808925)(900.57450684,305.5080896)
\curveto(900.53450092,305.60808906)(900.48450097,305.68808898)(900.42450684,305.7480896)
\curveto(900.40450105,305.7680889)(900.37950108,305.78808888)(900.34950684,305.8080896)
\curveto(900.32950113,305.82808884)(900.29950116,305.83308884)(900.25950684,305.8230896)
\curveto(900.14950131,305.79308888)(900.04450141,305.73808893)(899.94450684,305.6580896)
\curveto(899.8545016,305.57808909)(899.76450169,305.50808916)(899.67450684,305.4480896)
\curveto(899.54450191,305.3680893)(899.40450205,305.29308938)(899.25450684,305.2230896)
\curveto(899.10450235,305.16308951)(898.94450251,305.10808956)(898.77450684,305.0580896)
\curveto(898.67450278,305.02808964)(898.56450289,305.00808966)(898.44450684,304.9980896)
\curveto(898.33450312,304.98808968)(898.22450323,304.9730897)(898.11450684,304.9530896)
\curveto(898.06450339,304.94308973)(898.01950344,304.93808973)(897.97950684,304.9380896)
\lineto(897.87450684,304.9380896)
\curveto(897.76450369,304.91808975)(897.6595038,304.91808975)(897.55950684,304.9380896)
\lineto(897.42450684,304.9380896)
\curveto(897.37450408,304.94808972)(897.32450413,304.95308972)(897.27450684,304.9530896)
\curveto(897.22450423,304.95308972)(897.17950428,304.96308971)(897.13950684,304.9830896)
\curveto(897.09950436,304.99308968)(897.06450439,304.99808967)(897.03450684,304.9980896)
\curveto(897.01450444,304.98808968)(896.98950447,304.98808968)(896.95950684,304.9980896)
\lineto(896.71950684,305.0580896)
\curveto(896.63950482,305.0680896)(896.56450489,305.08808958)(896.49450684,305.1180896)
\curveto(896.19450526,305.24808942)(895.94950551,305.39308928)(895.75950684,305.5530896)
\curveto(895.57950588,305.72308895)(895.42950603,305.95808871)(895.30950684,306.2580896)
\curveto(895.21950624,306.47808819)(895.17450628,306.74308793)(895.17450684,307.0530896)
\lineto(895.17450684,307.3680896)
\curveto(895.18450627,307.41808725)(895.18950627,307.4680872)(895.18950684,307.5180896)
\lineto(895.21950684,307.6980896)
\lineto(895.33950684,308.0280896)
\curveto(895.37950608,308.13808653)(895.42950603,308.23808643)(895.48950684,308.3280896)
\curveto(895.66950579,308.61808605)(895.91450554,308.83308584)(896.22450684,308.9730896)
\curveto(896.53450492,309.11308556)(896.87450458,309.23808543)(897.24450684,309.3480896)
\curveto(897.38450407,309.38808528)(897.52950393,309.41808525)(897.67950684,309.4380896)
\curveto(897.82950363,309.45808521)(897.97950348,309.48308519)(898.12950684,309.5130896)
\curveto(898.19950326,309.53308514)(898.26450319,309.54308513)(898.32450684,309.5430896)
\curveto(898.39450306,309.54308513)(898.46950299,309.55308512)(898.54950684,309.5730896)
\curveto(898.61950284,309.59308508)(898.68950277,309.60308507)(898.75950684,309.6030896)
\curveto(898.82950263,309.61308506)(898.90450255,309.62808504)(898.98450684,309.6480896)
\curveto(899.23450222,309.70808496)(899.46950199,309.75808491)(899.68950684,309.7980896)
\curveto(899.90950155,309.84808482)(900.08450137,309.96308471)(900.21450684,310.1430896)
\curveto(900.27450118,310.22308445)(900.32450113,310.32308435)(900.36450684,310.4430896)
\curveto(900.40450105,310.5730841)(900.40450105,310.71308396)(900.36450684,310.8630896)
\curveto(900.30450115,311.10308357)(900.21450124,311.29308338)(900.09450684,311.4330896)
\curveto(899.98450147,311.5730831)(899.82450163,311.68308299)(899.61450684,311.7630896)
\curveto(899.49450196,311.81308286)(899.34950211,311.84808282)(899.17950684,311.8680896)
\curveto(899.01950244,311.88808278)(898.84950261,311.89808277)(898.66950684,311.8980896)
\curveto(898.48950297,311.89808277)(898.31450314,311.88808278)(898.14450684,311.8680896)
\curveto(897.97450348,311.84808282)(897.82950363,311.81808285)(897.70950684,311.7780896)
\curveto(897.53950392,311.71808295)(897.37450408,311.63308304)(897.21450684,311.5230896)
\curveto(897.13450432,311.46308321)(897.0595044,311.38308329)(896.98950684,311.2830896)
\curveto(896.92950453,311.19308348)(896.87450458,311.09308358)(896.82450684,310.9830896)
\curveto(896.79450466,310.90308377)(896.76450469,310.81808385)(896.73450684,310.7280896)
\curveto(896.71450474,310.63808403)(896.66950479,310.5680841)(896.59950684,310.5180896)
\curveto(896.5595049,310.48808418)(896.48950497,310.46308421)(896.38950684,310.4430896)
\curveto(896.29950516,310.43308424)(896.20450525,310.42808424)(896.10450684,310.4280896)
\curveto(896.00450545,310.42808424)(895.90450555,310.43308424)(895.80450684,310.4430896)
\curveto(895.71450574,310.46308421)(895.64950581,310.48808418)(895.60950684,310.5180896)
\curveto(895.56950589,310.54808412)(895.53950592,310.59808407)(895.51950684,310.6680896)
\curveto(895.49950596,310.73808393)(895.49950596,310.81308386)(895.51950684,310.8930896)
\curveto(895.54950591,311.02308365)(895.57950588,311.14308353)(895.60950684,311.2530896)
\curveto(895.64950581,311.3730833)(895.69450576,311.48808318)(895.74450684,311.5980896)
\curveto(895.93450552,311.94808272)(896.17450528,312.21808245)(896.46450684,312.4080896)
\curveto(896.7545047,312.60808206)(897.11450434,312.7680819)(897.54450684,312.8880896)
\curveto(897.64450381,312.90808176)(897.74450371,312.92308175)(897.84450684,312.9330896)
\curveto(897.9545035,312.94308173)(898.06450339,312.95808171)(898.17450684,312.9780896)
\curveto(898.21450324,312.98808168)(898.27950318,312.98808168)(898.36950684,312.9780896)
\curveto(898.459503,312.97808169)(898.51450294,312.98808168)(898.53450684,313.0080896)
\curveto(899.23450222,313.01808165)(899.84450161,312.93808173)(900.36450684,312.7680896)
\curveto(900.88450057,312.59808207)(901.24950021,312.2730824)(901.45950684,311.7930896)
\curveto(901.54949991,311.59308308)(901.59949986,311.35808331)(901.60950684,311.0880896)
\curveto(901.62949983,310.82808384)(901.63949982,310.55308412)(901.63950684,310.2630896)
\lineto(901.63950684,306.9480896)
\curveto(901.63949982,306.80808786)(901.64449981,306.673088)(901.65450684,306.5430896)
\curveto(901.66449979,306.41308826)(901.69449976,306.30808836)(901.74450684,306.2280896)
\curveto(901.79449966,306.15808851)(901.8594996,306.10808856)(901.93950684,306.0780896)
\curveto(902.02949943,306.03808863)(902.11449934,306.00808866)(902.19450684,305.9880896)
\curveto(902.27449918,305.97808869)(902.33449912,305.93308874)(902.37450684,305.8530896)
\curveto(902.39449906,305.82308885)(902.40449905,305.79308888)(902.40450684,305.7630896)
\curveto(902.40449905,305.73308894)(902.40949905,305.69308898)(902.41950684,305.6430896)
\moveto(900.27450684,307.3080896)
\curveto(900.33450112,307.44808722)(900.36450109,307.60808706)(900.36450684,307.7880896)
\curveto(900.37450108,307.97808669)(900.37950108,308.1730865)(900.37950684,308.3730896)
\curveto(900.37950108,308.48308619)(900.37450108,308.58308609)(900.36450684,308.6730896)
\curveto(900.3545011,308.76308591)(900.31450114,308.83308584)(900.24450684,308.8830896)
\curveto(900.21450124,308.90308577)(900.14450131,308.91308576)(900.03450684,308.9130896)
\curveto(900.01450144,308.89308578)(899.97950148,308.88308579)(899.92950684,308.8830896)
\curveto(899.87950158,308.88308579)(899.83450162,308.8730858)(899.79450684,308.8530896)
\curveto(899.71450174,308.83308584)(899.62450183,308.81308586)(899.52450684,308.7930896)
\lineto(899.22450684,308.7330896)
\curveto(899.19450226,308.73308594)(899.1595023,308.72808594)(899.11950684,308.7180896)
\lineto(899.01450684,308.7180896)
\curveto(898.86450259,308.67808599)(898.69950276,308.65308602)(898.51950684,308.6430896)
\curveto(898.34950311,308.64308603)(898.18950327,308.62308605)(898.03950684,308.5830896)
\curveto(897.9595035,308.56308611)(897.88450357,308.54308613)(897.81450684,308.5230896)
\curveto(897.7545037,308.51308616)(897.68450377,308.49808617)(897.60450684,308.4780896)
\curveto(897.44450401,308.42808624)(897.29450416,308.36308631)(897.15450684,308.2830896)
\curveto(897.01450444,308.21308646)(896.89450456,308.12308655)(896.79450684,308.0130896)
\curveto(896.69450476,307.90308677)(896.61950484,307.7680869)(896.56950684,307.6080896)
\curveto(896.51950494,307.45808721)(896.49950496,307.2730874)(896.50950684,307.0530896)
\curveto(896.50950495,306.95308772)(896.52450493,306.85808781)(896.55450684,306.7680896)
\curveto(896.59450486,306.68808798)(896.63950482,306.61308806)(896.68950684,306.5430896)
\curveto(896.76950469,306.43308824)(896.87450458,306.33808833)(897.00450684,306.2580896)
\curveto(897.13450432,306.18808848)(897.27450418,306.12808854)(897.42450684,306.0780896)
\curveto(897.47450398,306.0680886)(897.52450393,306.06308861)(897.57450684,306.0630896)
\curveto(897.62450383,306.06308861)(897.67450378,306.05808861)(897.72450684,306.0480896)
\curveto(897.79450366,306.02808864)(897.87950358,306.01308866)(897.97950684,306.0030896)
\curveto(898.08950337,306.00308867)(898.17950328,306.01308866)(898.24950684,306.0330896)
\curveto(898.30950315,306.05308862)(898.36950309,306.05808861)(898.42950684,306.0480896)
\curveto(898.48950297,306.04808862)(898.54950291,306.05808861)(898.60950684,306.0780896)
\curveto(898.68950277,306.09808857)(898.76450269,306.11308856)(898.83450684,306.1230896)
\curveto(898.91450254,306.13308854)(898.98950247,306.15308852)(899.05950684,306.1830896)
\curveto(899.34950211,306.30308837)(899.59450186,306.44808822)(899.79450684,306.6180896)
\curveto(900.00450145,306.78808788)(900.16450129,307.01808765)(900.27450684,307.3080896)
}
}
{
\newrgbcolor{curcolor}{0 0 0}
\pscustom[linestyle=none,fillstyle=solid,fillcolor=curcolor]
{
\newpath
\moveto(910.55114746,305.8980896)
\lineto(910.55114746,305.5080896)
\curveto(910.55113959,305.38808928)(910.52613961,305.28808938)(910.47614746,305.2080896)
\curveto(910.42613971,305.13808953)(910.3411398,305.09808957)(910.22114746,305.0880896)
\lineto(909.87614746,305.0880896)
\curveto(909.81614032,305.08808958)(909.75614038,305.08308959)(909.69614746,305.0730896)
\curveto(909.64614049,305.0730896)(909.60114054,305.08308959)(909.56114746,305.1030896)
\curveto(909.47114067,305.12308955)(909.41114073,305.16308951)(909.38114746,305.2230896)
\curveto(909.3411408,305.2730894)(909.31614082,305.33308934)(909.30614746,305.4030896)
\curveto(909.30614083,305.4730892)(909.29114085,305.54308913)(909.26114746,305.6130896)
\curveto(909.25114089,305.63308904)(909.2361409,305.64808902)(909.21614746,305.6580896)
\curveto(909.20614093,305.67808899)(909.19114095,305.69808897)(909.17114746,305.7180896)
\curveto(909.07114107,305.72808894)(908.99114115,305.70808896)(908.93114746,305.6580896)
\curveto(908.88114126,305.60808906)(908.82614131,305.55808911)(908.76614746,305.5080896)
\curveto(908.56614157,305.35808931)(908.36614177,305.24308943)(908.16614746,305.1630896)
\curveto(907.98614215,305.08308959)(907.77614236,305.02308965)(907.53614746,304.9830896)
\curveto(907.30614283,304.94308973)(907.06614307,304.92308975)(906.81614746,304.9230896)
\curveto(906.57614356,304.91308976)(906.3361438,304.92808974)(906.09614746,304.9680896)
\curveto(905.85614428,304.99808967)(905.64614449,305.05308962)(905.46614746,305.1330896)
\curveto(904.94614519,305.35308932)(904.52614561,305.64808902)(904.20614746,306.0180896)
\curveto(903.88614625,306.39808827)(903.6361465,306.8680878)(903.45614746,307.4280896)
\curveto(903.41614672,307.51808715)(903.38614675,307.60808706)(903.36614746,307.6980896)
\curveto(903.35614678,307.79808687)(903.3361468,307.89808677)(903.30614746,307.9980896)
\curveto(903.29614684,308.04808662)(903.29114685,308.09808657)(903.29114746,308.1480896)
\curveto(903.29114685,308.19808647)(903.28614685,308.24808642)(903.27614746,308.2980896)
\curveto(903.25614688,308.34808632)(903.24614689,308.39808627)(903.24614746,308.4480896)
\curveto(903.25614688,308.50808616)(903.25614688,308.56308611)(903.24614746,308.6130896)
\lineto(903.24614746,308.7630896)
\curveto(903.22614691,308.81308586)(903.21614692,308.87808579)(903.21614746,308.9580896)
\curveto(903.21614692,309.03808563)(903.22614691,309.10308557)(903.24614746,309.1530896)
\lineto(903.24614746,309.3180896)
\curveto(903.26614687,309.38808528)(903.27114687,309.45808521)(903.26114746,309.5280896)
\curveto(903.26114688,309.60808506)(903.27114687,309.68308499)(903.29114746,309.7530896)
\curveto(903.30114684,309.80308487)(903.30614683,309.84808482)(903.30614746,309.8880896)
\curveto(903.30614683,309.92808474)(903.31114683,309.9730847)(903.32114746,310.0230896)
\curveto(903.35114679,310.12308455)(903.37614676,310.21808445)(903.39614746,310.3080896)
\curveto(903.41614672,310.40808426)(903.4411467,310.50308417)(903.47114746,310.5930896)
\curveto(903.60114654,310.9730837)(903.76614637,311.31308336)(903.96614746,311.6130896)
\curveto(904.17614596,311.92308275)(904.42614571,312.17808249)(904.71614746,312.3780896)
\curveto(904.88614525,312.49808217)(905.06114508,312.59808207)(905.24114746,312.6780896)
\curveto(905.43114471,312.75808191)(905.6361445,312.82808184)(905.85614746,312.8880896)
\curveto(905.92614421,312.89808177)(905.99114415,312.90808176)(906.05114746,312.9180896)
\curveto(906.12114402,312.92808174)(906.19114395,312.94308173)(906.26114746,312.9630896)
\lineto(906.41114746,312.9630896)
\curveto(906.49114365,312.98308169)(906.60614353,312.99308168)(906.75614746,312.9930896)
\curveto(906.91614322,312.99308168)(907.0361431,312.98308169)(907.11614746,312.9630896)
\curveto(907.15614298,312.95308172)(907.21114293,312.94808172)(907.28114746,312.9480896)
\curveto(907.39114275,312.91808175)(907.50114264,312.89308178)(907.61114746,312.8730896)
\curveto(907.72114242,312.86308181)(907.82614231,312.83308184)(907.92614746,312.7830896)
\curveto(908.07614206,312.72308195)(908.21614192,312.65808201)(908.34614746,312.5880896)
\curveto(908.48614165,312.51808215)(908.61614152,312.43808223)(908.73614746,312.3480896)
\curveto(908.79614134,312.29808237)(908.85614128,312.24308243)(908.91614746,312.1830896)
\curveto(908.98614115,312.13308254)(909.07614106,312.11808255)(909.18614746,312.1380896)
\curveto(909.20614093,312.1680825)(909.22114092,312.19308248)(909.23114746,312.2130896)
\curveto(909.25114089,312.23308244)(909.26614087,312.26308241)(909.27614746,312.3030896)
\curveto(909.30614083,312.39308228)(909.31614082,312.50808216)(909.30614746,312.6480896)
\lineto(909.30614746,313.0230896)
\lineto(909.30614746,314.7480896)
\lineto(909.30614746,315.2130896)
\curveto(909.30614083,315.39307928)(909.33114081,315.52307915)(909.38114746,315.6030896)
\curveto(909.42114072,315.673079)(909.48114066,315.71807895)(909.56114746,315.7380896)
\curveto(909.58114056,315.73807893)(909.60614053,315.73807893)(909.63614746,315.7380896)
\curveto(909.66614047,315.74807892)(909.69114045,315.75307892)(909.71114746,315.7530896)
\curveto(909.85114029,315.76307891)(909.99614014,315.76307891)(910.14614746,315.7530896)
\curveto(910.30613983,315.75307892)(910.41613972,315.71307896)(910.47614746,315.6330896)
\curveto(910.52613961,315.55307912)(910.55113959,315.45307922)(910.55114746,315.3330896)
\lineto(910.55114746,314.9580896)
\lineto(910.55114746,305.8980896)
\moveto(909.33614746,308.7330896)
\curveto(909.35614078,308.78308589)(909.36614077,308.84808582)(909.36614746,308.9280896)
\curveto(909.36614077,309.01808565)(909.35614078,309.08808558)(909.33614746,309.1380896)
\lineto(909.33614746,309.3630896)
\curveto(909.31614082,309.45308522)(909.30114084,309.54308513)(909.29114746,309.6330896)
\curveto(909.28114086,309.73308494)(909.26114088,309.82308485)(909.23114746,309.9030896)
\curveto(909.21114093,309.98308469)(909.19114095,310.05808461)(909.17114746,310.1280896)
\curveto(909.16114098,310.19808447)(909.141141,310.2680844)(909.11114746,310.3380896)
\curveto(908.99114115,310.63808403)(908.8361413,310.90308377)(908.64614746,311.1330896)
\curveto(908.45614168,311.36308331)(908.21614192,311.54308313)(907.92614746,311.6730896)
\curveto(907.82614231,311.72308295)(907.72114242,311.75808291)(907.61114746,311.7780896)
\curveto(907.51114263,311.80808286)(907.40114274,311.83308284)(907.28114746,311.8530896)
\curveto(907.20114294,311.8730828)(907.11114303,311.88308279)(907.01114746,311.8830896)
\lineto(906.74114746,311.8830896)
\curveto(906.69114345,311.8730828)(906.64614349,311.86308281)(906.60614746,311.8530896)
\lineto(906.47114746,311.8530896)
\curveto(906.39114375,311.83308284)(906.30614383,311.81308286)(906.21614746,311.7930896)
\curveto(906.136144,311.7730829)(906.05614408,311.74808292)(905.97614746,311.7180896)
\curveto(905.65614448,311.57808309)(905.39614474,311.3730833)(905.19614746,311.1030896)
\curveto(905.00614513,310.84308383)(904.85114529,310.53808413)(904.73114746,310.1880896)
\curveto(904.69114545,310.07808459)(904.66114548,309.96308471)(904.64114746,309.8430896)
\curveto(904.63114551,309.73308494)(904.61614552,309.62308505)(904.59614746,309.5130896)
\curveto(904.59614554,309.4730852)(904.59114555,309.43308524)(904.58114746,309.3930896)
\lineto(904.58114746,309.2880896)
\curveto(904.56114558,309.23808543)(904.55114559,309.18308549)(904.55114746,309.1230896)
\curveto(904.56114558,309.06308561)(904.56614557,309.00808566)(904.56614746,308.9580896)
\lineto(904.56614746,308.6280896)
\curveto(904.56614557,308.52808614)(904.57614556,308.43308624)(904.59614746,308.3430896)
\curveto(904.60614553,308.31308636)(904.61114553,308.26308641)(904.61114746,308.1930896)
\curveto(904.63114551,308.12308655)(904.64614549,308.05308662)(904.65614746,307.9830896)
\lineto(904.71614746,307.7730896)
\curveto(904.82614531,307.42308725)(904.97614516,307.12308755)(905.16614746,306.8730896)
\curveto(905.35614478,306.62308805)(905.59614454,306.41808825)(905.88614746,306.2580896)
\curveto(905.97614416,306.20808846)(906.06614407,306.1680885)(906.15614746,306.1380896)
\curveto(906.24614389,306.10808856)(906.34614379,306.07808859)(906.45614746,306.0480896)
\curveto(906.50614363,306.02808864)(906.55614358,306.02308865)(906.60614746,306.0330896)
\curveto(906.66614347,306.04308863)(906.72114342,306.03808863)(906.77114746,306.0180896)
\curveto(906.81114333,306.00808866)(906.85114329,306.00308867)(906.89114746,306.0030896)
\lineto(907.02614746,306.0030896)
\lineto(907.16114746,306.0030896)
\curveto(907.19114295,306.01308866)(907.2411429,306.01808865)(907.31114746,306.0180896)
\curveto(907.39114275,306.03808863)(907.47114267,306.05308862)(907.55114746,306.0630896)
\curveto(907.63114251,306.08308859)(907.70614243,306.10808856)(907.77614746,306.1380896)
\curveto(908.10614203,306.27808839)(908.37114177,306.45308822)(908.57114746,306.6630896)
\curveto(908.78114136,306.88308779)(908.95614118,307.15808751)(909.09614746,307.4880896)
\curveto(909.14614099,307.59808707)(909.18114096,307.70808696)(909.20114746,307.8180896)
\curveto(909.22114092,307.92808674)(909.24614089,308.03808663)(909.27614746,308.1480896)
\curveto(909.29614084,308.18808648)(909.30614083,308.22308645)(909.30614746,308.2530896)
\curveto(909.30614083,308.29308638)(909.31114083,308.33308634)(909.32114746,308.3730896)
\curveto(909.33114081,308.43308624)(909.33114081,308.49308618)(909.32114746,308.5530896)
\curveto(909.32114082,308.61308606)(909.32614081,308.673086)(909.33614746,308.7330896)
}
}
{
\newrgbcolor{curcolor}{0.80000001 0.80000001 0.80000001}
\pscustom[linestyle=none,fillstyle=solid,fillcolor=curcolor]
{
\newpath
\moveto(811.9732666,385.02397156)
\lineto(826.9732666,385.02397156)
\lineto(826.9732666,370.02397156)
\lineto(811.9732666,370.02397156)
\closepath
}
}
{
\newrgbcolor{curcolor}{0.7019608 0.7019608 0.7019608}
\pscustom[linestyle=none,fillstyle=solid,fillcolor=curcolor]
{
\newpath
\moveto(811.9732666,361.69264984)
\lineto(826.9732666,361.69264984)
\lineto(826.9732666,346.69264984)
\lineto(811.9732666,346.69264984)
\closepath
}
}
{
\newrgbcolor{curcolor}{0.60000002 0.60000002 0.60000002}
\pscustom[linestyle=none,fillstyle=solid,fillcolor=curcolor]
{
\newpath
\moveto(811.9732666,338.37634277)
\lineto(826.9732666,338.37634277)
\lineto(826.9732666,323.37634277)
\lineto(811.9732666,323.37634277)
\closepath
}
}
{
\newrgbcolor{curcolor}{0.50196081 0.50196081 0.50196081}
\pscustom[linestyle=none,fillstyle=solid,fillcolor=curcolor]
{
\newpath
\moveto(811.9732666,315.78309631)
\lineto(826.9732666,315.78309631)
\lineto(826.9732666,300.78309631)
\lineto(811.9732666,300.78309631)
\closepath
}
}
{
\newrgbcolor{curcolor}{0.80000001 0.80000001 0.80000001}
\pscustom[linestyle=none,fillstyle=solid,fillcolor=curcolor]
{
\newpath
\moveto(632.94793701,269.31411743)
\lineto(646.96645641,269.31411743)
\lineto(646.96645641,89.08279419)
\lineto(632.94793701,89.08279419)
\closepath
}
}
{
\newrgbcolor{curcolor}{0.7019608 0.7019608 0.7019608}
\pscustom[linestyle=none,fillstyle=solid,fillcolor=curcolor]
{
\newpath
\moveto(646.94274902,248.45448303)
\lineto(660.96126842,248.45448303)
\lineto(660.96126842,89.08280945)
\lineto(646.94274902,89.08280945)
\closepath
}
}
{
\newrgbcolor{curcolor}{0.60000002 0.60000002 0.60000002}
\pscustom[linestyle=none,fillstyle=solid,fillcolor=curcolor]
{
\newpath
\moveto(660.93756104,373.08204651)
\lineto(674.95608044,373.08204651)
\lineto(674.95608044,89.08280945)
\lineto(660.93756104,89.08280945)
\closepath
}
}
{
\newrgbcolor{curcolor}{0.50196081 0.50196081 0.50196081}
\pscustom[linestyle=none,fillstyle=solid,fillcolor=curcolor]
{
\newpath
\moveto(674.93237305,144.50976562)
\lineto(688.95089245,144.50976562)
\lineto(688.95089245,89.08277893)
\lineto(674.93237305,89.08277893)
\closepath
}
}
{
\newrgbcolor{curcolor}{0.7019608 0.7019608 0.7019608}
\pscustom[linestyle=none,fillstyle=solid,fillcolor=curcolor]
{
\newpath
\moveto(737.96801758,94.08184814)
\lineto(751.98653698,94.08184814)
\lineto(751.98653698,89.08279419)
\lineto(737.96801758,89.08279419)
\closepath
}
}
{
\newrgbcolor{curcolor}{0.60000002 0.60000002 0.60000002}
\pscustom[linestyle=none,fillstyle=solid,fillcolor=curcolor]
{
\newpath
\moveto(751.96282959,118.08184814)
\lineto(765.98134899,118.08184814)
\lineto(765.98134899,89.08279419)
\lineto(751.96282959,89.08279419)
\closepath
}
}
{
\newrgbcolor{curcolor}{0.80000001 0.80000001 0.80000001}
\pscustom[linestyle=none,fillstyle=solid,fillcolor=curcolor]
{
\newpath
\moveto(540.95080566,168.08184814)
\lineto(554.96932507,168.08184814)
\lineto(554.96932507,89.08279419)
\lineto(540.95080566,89.08279419)
\closepath
}
}
{
\newrgbcolor{curcolor}{0.60000002 0.60000002 0.60000002}
\pscustom[linestyle=none,fillstyle=solid,fillcolor=curcolor]
{
\newpath
\moveto(568.94042969,102.95684814)
\lineto(582.95894909,102.95684814)
\lineto(582.95894909,89.08279419)
\lineto(568.94042969,89.08279419)
\closepath
}
}
{
\newrgbcolor{curcolor}{0.50196081 0.50196081 0.50196081}
\pscustom[linestyle=none,fillstyle=solid,fillcolor=curcolor]
{
\newpath
\moveto(582.9352417,96.95684814)
\lineto(596.9537611,96.95684814)
\lineto(596.9537611,89.08279419)
\lineto(582.9352417,89.08279419)
\closepath
}
}
{
\newrgbcolor{curcolor}{0.80000001 0.80000001 0.80000001}
\pscustom[linestyle=none,fillstyle=solid,fillcolor=curcolor]
{
\newpath
\moveto(449.91079712,100.0062561)
\lineto(463.92931652,100.0062561)
\lineto(463.92931652,89.08280373)
\lineto(449.91079712,89.08280373)
\closepath
}
}
{
\newrgbcolor{curcolor}{0.60000002 0.60000002 0.60000002}
\pscustom[linestyle=none,fillstyle=solid,fillcolor=curcolor]
{
\newpath
\moveto(477.90042114,106.06085205)
\lineto(491.91894054,106.06085205)
\lineto(491.91894054,89.082798)
\lineto(477.90042114,89.082798)
\closepath
}
}
{
\newrgbcolor{curcolor}{0.80000001 0.80000001 0.80000001}
\pscustom[linestyle=none,fillstyle=solid,fillcolor=curcolor]
{
\newpath
\moveto(358.94403076,109.06604004)
\lineto(372.96255016,109.06604004)
\lineto(372.96255016,89.08278275)
\lineto(358.94403076,89.08278275)
\closepath
}
}
{
\newrgbcolor{curcolor}{0.60000002 0.60000002 0.60000002}
\pscustom[linestyle=none,fillstyle=solid,fillcolor=curcolor]
{
\newpath
\moveto(386.93365479,124.00369263)
\lineto(400.95217419,124.00369263)
\lineto(400.95217419,89.08280563)
\lineto(386.93365479,89.08280563)
\closepath
}
}
{
\newrgbcolor{curcolor}{0.80000001 0.80000001 0.80000001}
\pscustom[linestyle=none,fillstyle=solid,fillcolor=curcolor]
{
\newpath
\moveto(266.84335327,96.95684814)
\lineto(280.86187267,96.95684814)
\lineto(280.86187267,89.08279419)
\lineto(266.84335327,89.08279419)
\closepath
}
}
{
\newrgbcolor{curcolor}{0.7019608 0.7019608 0.7019608}
\pscustom[linestyle=none,fillstyle=solid,fillcolor=curcolor]
{
\newpath
\moveto(280.83816528,118.08166504)
\lineto(294.85668468,118.08166504)
\lineto(294.85668468,89.0827961)
\lineto(280.83816528,89.0827961)
\closepath
}
}
{
\newrgbcolor{curcolor}{0.60000002 0.60000002 0.60000002}
\pscustom[linestyle=none,fillstyle=solid,fillcolor=curcolor]
{
\newpath
\moveto(294.83297729,259.94497681)
\lineto(308.8514967,259.94497681)
\lineto(308.8514967,89.08280945)
\lineto(294.83297729,89.08280945)
\closepath
}
}
{
\newrgbcolor{curcolor}{0.80000001 0.80000001 0.80000001}
\pscustom[linestyle=none,fillstyle=solid,fillcolor=curcolor]
{
\newpath
\moveto(175.9801178,94.01934814)
\lineto(189.9986372,94.01934814)
\lineto(189.9986372,89.08279419)
\lineto(175.9801178,89.08279419)
\closepath
}
}
{
\newrgbcolor{curcolor}{0.60000002 0.60000002 0.60000002}
\pscustom[linestyle=none,fillstyle=solid,fillcolor=curcolor]
{
\newpath
\moveto(203.96974182,162.01934814)
\lineto(217.98826122,162.01934814)
\lineto(217.98826122,89.08279419)
\lineto(203.96974182,89.08279419)
\closepath
}
}
{
\newrgbcolor{curcolor}{0.50196081 0.50196081 0.50196081}
\pscustom[linestyle=none,fillstyle=solid,fillcolor=curcolor]
{
\newpath
\moveto(217.96455383,91.08184814)
\lineto(231.98307323,91.08184814)
\lineto(231.98307323,89.08279419)
\lineto(217.96455383,89.08279419)
\closepath
}
}
\end{pspicture}

\caption{Diagrama de barras de la actividad de los usuarios clasificados por
rol}
\label{usuarios_bars_2}
\end{figure}

\begin{figure}
\centering
%LaTeX with PSTricks extensions
%%Creator: inkscape 0.48.5
%%Please note this file requires PSTricks extensions
\psset{xunit=.5pt,yunit=.5pt,runit=.5pt}
\begin{pspicture}(943,728)
{
\newrgbcolor{curcolor}{0 0 0}
\pscustom[linestyle=none,fillstyle=solid,fillcolor=curcolor]
{
\newpath
\moveto(19.26111654,709.52799154)
\lineto(20.55111654,709.52799154)
\curveto(20.66111372,709.52798086)(20.76611361,709.52298087)(20.86611654,709.51299154)
\curveto(20.96611341,709.51298088)(21.04111334,709.47798091)(21.09111654,709.40799154)
\curveto(21.14111324,709.33798105)(21.16611321,709.24798114)(21.16611654,709.13799154)
\curveto(21.1761132,709.02798136)(21.1811132,708.90798148)(21.18111654,708.77799154)
\lineto(21.18111654,707.47299154)
\lineto(21.18111654,702.26799154)
\lineto(21.18111654,699.80799154)
\lineto(21.18111654,699.37299154)
\curveto(21.19111319,699.21299118)(21.17111321,699.0929913)(21.12111654,699.01299154)
\curveto(21.0811133,698.94299145)(20.99111339,698.8879915)(20.85111654,698.84799154)
\curveto(20.7811136,698.82799156)(20.70611367,698.82299157)(20.62611654,698.83299154)
\curveto(20.54611383,698.84299155)(20.46611391,698.84799154)(20.38611654,698.84799154)
\lineto(19.50111654,698.84799154)
\curveto(19.39111499,698.84799154)(19.28611509,698.85299154)(19.18611654,698.86299154)
\curveto(19.09611528,698.87299152)(19.02111536,698.90299149)(18.96111654,698.95299154)
\curveto(18.91111547,699.00299139)(18.8811155,699.07799131)(18.87111654,699.17799154)
\curveto(18.86111552,699.27799111)(18.85611552,699.38299101)(18.85611654,699.49299154)
\lineto(18.85611654,700.79799154)
\lineto(18.85611654,706.27299154)
\lineto(18.85611654,708.46299154)
\curveto(18.85611552,708.60298179)(18.85111553,708.76798162)(18.84111654,708.95799154)
\curveto(18.84111554,709.14798124)(18.86611551,709.28298111)(18.91611654,709.36299154)
\curveto(18.95611542,709.42298097)(19.02111536,709.47298092)(19.11111654,709.51299154)
\curveto(19.14111524,709.51298088)(19.16611521,709.51298088)(19.18611654,709.51299154)
\curveto(19.21611516,709.52298087)(19.24111514,709.52798086)(19.26111654,709.52799154)
}
}
{
\newrgbcolor{curcolor}{0 0 0}
\pscustom[linestyle=none,fillstyle=solid,fillcolor=curcolor]
{
\newpath
\moveto(27.46494467,706.78299154)
\curveto(28.06493886,706.80298359)(28.56493836,706.71798367)(28.96494467,706.52799154)
\curveto(29.36493756,706.33798405)(29.67993725,706.05798433)(29.90994467,705.68799154)
\curveto(29.97993695,705.57798481)(30.03493689,705.45798493)(30.07494467,705.32799154)
\curveto(30.11493681,705.20798518)(30.15493677,705.08298531)(30.19494467,704.95299154)
\curveto(30.21493671,704.87298552)(30.2249367,704.79798559)(30.22494467,704.72799154)
\curveto(30.23493669,704.65798573)(30.24993668,704.5879858)(30.26994467,704.51799154)
\curveto(30.26993666,704.45798593)(30.27493665,704.41798597)(30.28494467,704.39799154)
\curveto(30.30493662,704.25798613)(30.31493661,704.11298628)(30.31494467,703.96299154)
\lineto(30.31494467,703.52799154)
\lineto(30.31494467,702.19299154)
\lineto(30.31494467,699.76299154)
\curveto(30.31493661,699.57299082)(30.30993662,699.387991)(30.29994467,699.20799154)
\curveto(30.29993663,699.03799135)(30.2299367,698.92799146)(30.08994467,698.87799154)
\curveto(30.0299369,698.85799153)(29.95993697,698.84799154)(29.87994467,698.84799154)
\lineto(29.63994467,698.84799154)
\lineto(28.82994467,698.84799154)
\curveto(28.70993822,698.84799154)(28.59993833,698.85299154)(28.49994467,698.86299154)
\curveto(28.40993852,698.88299151)(28.33993859,698.92799146)(28.28994467,698.99799154)
\curveto(28.24993868,699.05799133)(28.2249387,699.13299126)(28.21494467,699.22299154)
\lineto(28.21494467,699.53799154)
\lineto(28.21494467,700.58799154)
\lineto(28.21494467,702.82299154)
\curveto(28.21493871,703.1929872)(28.19993873,703.53298686)(28.16994467,703.84299154)
\curveto(28.13993879,704.16298623)(28.04993888,704.43298596)(27.89994467,704.65299154)
\curveto(27.75993917,704.85298554)(27.55493937,704.9929854)(27.28494467,705.07299154)
\curveto(27.23493969,705.0929853)(27.17993975,705.10298529)(27.11994467,705.10299154)
\curveto(27.06993986,705.10298529)(27.01493991,705.11298528)(26.95494467,705.13299154)
\curveto(26.90494002,705.14298525)(26.83994009,705.14298525)(26.75994467,705.13299154)
\curveto(26.68994024,705.13298526)(26.63494029,705.12798526)(26.59494467,705.11799154)
\curveto(26.55494037,705.10798528)(26.51994041,705.10298529)(26.48994467,705.10299154)
\curveto(26.45994047,705.10298529)(26.4299405,705.09798529)(26.39994467,705.08799154)
\curveto(26.16994076,705.02798536)(25.98494094,704.94798544)(25.84494467,704.84799154)
\curveto(25.5249414,704.61798577)(25.33494159,704.28298611)(25.27494467,703.84299154)
\curveto(25.21494171,703.40298699)(25.18494174,702.90798748)(25.18494467,702.35799154)
\lineto(25.18494467,700.48299154)
\lineto(25.18494467,699.56799154)
\lineto(25.18494467,699.29799154)
\curveto(25.18494174,699.20799118)(25.16994176,699.13299126)(25.13994467,699.07299154)
\curveto(25.08994184,698.96299143)(25.00994192,698.89799149)(24.89994467,698.87799154)
\curveto(24.78994214,698.85799153)(24.65494227,698.84799154)(24.49494467,698.84799154)
\lineto(23.74494467,698.84799154)
\curveto(23.63494329,698.84799154)(23.5249434,698.85299154)(23.41494467,698.86299154)
\curveto(23.30494362,698.87299152)(23.2249437,698.90799148)(23.17494467,698.96799154)
\curveto(23.10494382,699.05799133)(23.06994386,699.1879912)(23.06994467,699.35799154)
\curveto(23.07994385,699.52799086)(23.08494384,699.6879907)(23.08494467,699.83799154)
\lineto(23.08494467,701.87799154)
\lineto(23.08494467,705.17799154)
\lineto(23.08494467,705.94299154)
\lineto(23.08494467,706.24299154)
\curveto(23.09494383,706.33298406)(23.1249438,706.40798398)(23.17494467,706.46799154)
\curveto(23.19494373,706.49798389)(23.2249437,706.51798387)(23.26494467,706.52799154)
\curveto(23.31494361,706.54798384)(23.36494356,706.56298383)(23.41494467,706.57299154)
\lineto(23.48994467,706.57299154)
\curveto(23.53994339,706.58298381)(23.58994334,706.5879838)(23.63994467,706.58799154)
\lineto(23.80494467,706.58799154)
\lineto(24.43494467,706.58799154)
\curveto(24.51494241,706.5879838)(24.58994234,706.58298381)(24.65994467,706.57299154)
\curveto(24.73994219,706.57298382)(24.80994212,706.56298383)(24.86994467,706.54299154)
\curveto(24.93994199,706.51298388)(24.98494194,706.46798392)(25.00494467,706.40799154)
\curveto(25.03494189,706.34798404)(25.05994187,706.27798411)(25.07994467,706.19799154)
\curveto(25.08994184,706.15798423)(25.08994184,706.12298427)(25.07994467,706.09299154)
\curveto(25.07994185,706.06298433)(25.08994184,706.03298436)(25.10994467,706.00299154)
\curveto(25.1299418,705.95298444)(25.14494178,705.92298447)(25.15494467,705.91299154)
\curveto(25.17494175,705.90298449)(25.19994173,705.8879845)(25.22994467,705.86799154)
\curveto(25.33994159,705.85798453)(25.4299415,705.8929845)(25.49994467,705.97299154)
\curveto(25.56994136,706.06298433)(25.64494128,706.13298426)(25.72494467,706.18299154)
\curveto(25.99494093,706.38298401)(26.29494063,706.54298385)(26.62494467,706.66299154)
\curveto(26.71494021,706.6929837)(26.80494012,706.71298368)(26.89494467,706.72299154)
\curveto(26.99493993,706.73298366)(27.09993983,706.74798364)(27.20994467,706.76799154)
\curveto(27.23993969,706.77798361)(27.28493964,706.77798361)(27.34494467,706.76799154)
\curveto(27.40493952,706.76798362)(27.44493948,706.77298362)(27.46494467,706.78299154)
}
}
{
\newrgbcolor{curcolor}{0 0 0}
\pscustom[linestyle=none,fillstyle=solid,fillcolor=curcolor]
{
\newpath
\moveto(39.53619467,699.70299154)
\lineto(39.53619467,699.28299154)
\curveto(39.5361863,699.15299124)(39.50618633,699.04799134)(39.44619467,698.96799154)
\curveto(39.39618644,698.91799147)(39.3311865,698.88299151)(39.25119467,698.86299154)
\curveto(39.17118666,698.85299154)(39.08118675,698.84799154)(38.98119467,698.84799154)
\lineto(38.15619467,698.84799154)
\lineto(37.87119467,698.84799154)
\curveto(37.79118804,698.85799153)(37.72618811,698.88299151)(37.67619467,698.92299154)
\curveto(37.60618823,698.97299142)(37.56618827,699.03799135)(37.55619467,699.11799154)
\curveto(37.54618829,699.19799119)(37.52618831,699.27799111)(37.49619467,699.35799154)
\curveto(37.47618836,699.37799101)(37.45618838,699.392991)(37.43619467,699.40299154)
\curveto(37.42618841,699.42299097)(37.41118842,699.44299095)(37.39119467,699.46299154)
\curveto(37.28118855,699.46299093)(37.20118863,699.43799095)(37.15119467,699.38799154)
\lineto(37.00119467,699.23799154)
\curveto(36.9311889,699.1879912)(36.86618897,699.14299125)(36.80619467,699.10299154)
\curveto(36.74618909,699.07299132)(36.68118915,699.03299136)(36.61119467,698.98299154)
\curveto(36.57118926,698.96299143)(36.52618931,698.94299145)(36.47619467,698.92299154)
\curveto(36.4361894,698.90299149)(36.39118944,698.88299151)(36.34119467,698.86299154)
\curveto(36.20118963,698.81299158)(36.05118978,698.76799162)(35.89119467,698.72799154)
\curveto(35.84118999,698.70799168)(35.79619004,698.69799169)(35.75619467,698.69799154)
\curveto(35.71619012,698.69799169)(35.67619016,698.6929917)(35.63619467,698.68299154)
\lineto(35.50119467,698.68299154)
\curveto(35.47119036,698.67299172)(35.4311904,698.66799172)(35.38119467,698.66799154)
\lineto(35.24619467,698.66799154)
\curveto(35.18619065,698.64799174)(35.09619074,698.64299175)(34.97619467,698.65299154)
\curveto(34.85619098,698.65299174)(34.77119106,698.66299173)(34.72119467,698.68299154)
\curveto(34.65119118,698.70299169)(34.58619125,698.71299168)(34.52619467,698.71299154)
\curveto(34.47619136,698.70299169)(34.42119141,698.70799168)(34.36119467,698.72799154)
\lineto(34.00119467,698.84799154)
\curveto(33.89119194,698.87799151)(33.78119205,698.91799147)(33.67119467,698.96799154)
\curveto(33.32119251,699.11799127)(33.00619283,699.34799104)(32.72619467,699.65799154)
\curveto(32.45619338,699.97799041)(32.24119359,700.31299008)(32.08119467,700.66299154)
\curveto(32.0311938,700.77298962)(31.99119384,700.87798951)(31.96119467,700.97799154)
\curveto(31.9311939,701.0879893)(31.89619394,701.19798919)(31.85619467,701.30799154)
\curveto(31.84619399,701.34798904)(31.84119399,701.38298901)(31.84119467,701.41299154)
\curveto(31.84119399,701.45298894)(31.831194,701.49798889)(31.81119467,701.54799154)
\curveto(31.79119404,701.62798876)(31.77119406,701.71298868)(31.75119467,701.80299154)
\curveto(31.74119409,701.90298849)(31.72619411,702.00298839)(31.70619467,702.10299154)
\curveto(31.69619414,702.13298826)(31.69119414,702.16798822)(31.69119467,702.20799154)
\curveto(31.70119413,702.24798814)(31.70119413,702.28298811)(31.69119467,702.31299154)
\lineto(31.69119467,702.44799154)
\curveto(31.69119414,702.49798789)(31.68619415,702.54798784)(31.67619467,702.59799154)
\curveto(31.66619417,702.64798774)(31.66119417,702.70298769)(31.66119467,702.76299154)
\curveto(31.66119417,702.83298756)(31.66619417,702.8879875)(31.67619467,702.92799154)
\curveto(31.68619415,702.97798741)(31.69119414,703.02298737)(31.69119467,703.06299154)
\lineto(31.69119467,703.21299154)
\curveto(31.70119413,703.26298713)(31.70119413,703.30798708)(31.69119467,703.34799154)
\curveto(31.69119414,703.39798699)(31.70119413,703.44798694)(31.72119467,703.49799154)
\curveto(31.74119409,703.60798678)(31.75619408,703.71298668)(31.76619467,703.81299154)
\curveto(31.78619405,703.91298648)(31.81119402,704.01298638)(31.84119467,704.11299154)
\curveto(31.88119395,704.23298616)(31.91619392,704.34798604)(31.94619467,704.45799154)
\curveto(31.97619386,704.56798582)(32.01619382,704.67798571)(32.06619467,704.78799154)
\curveto(32.20619363,705.0879853)(32.38119345,705.37298502)(32.59119467,705.64299154)
\curveto(32.61119322,705.67298472)(32.6361932,705.69798469)(32.66619467,705.71799154)
\curveto(32.70619313,705.74798464)(32.7361931,705.77798461)(32.75619467,705.80799154)
\curveto(32.79619304,705.85798453)(32.836193,705.90298449)(32.87619467,705.94299154)
\curveto(32.91619292,705.98298441)(32.96119287,706.02298437)(33.01119467,706.06299154)
\curveto(33.05119278,706.08298431)(33.08619275,706.10798428)(33.11619467,706.13799154)
\curveto(33.14619269,706.17798421)(33.18119265,706.20798418)(33.22119467,706.22799154)
\curveto(33.47119236,706.39798399)(33.76119207,706.53798385)(34.09119467,706.64799154)
\curveto(34.16119167,706.66798372)(34.2311916,706.68298371)(34.30119467,706.69299154)
\curveto(34.38119145,706.70298369)(34.46119137,706.71798367)(34.54119467,706.73799154)
\curveto(34.61119122,706.75798363)(34.70119113,706.76798362)(34.81119467,706.76799154)
\curveto(34.92119091,706.77798361)(35.0311908,706.78298361)(35.14119467,706.78299154)
\curveto(35.25119058,706.78298361)(35.35619048,706.77798361)(35.45619467,706.76799154)
\curveto(35.56619027,706.75798363)(35.65619018,706.74298365)(35.72619467,706.72299154)
\curveto(35.87618996,706.67298372)(36.02118981,706.62798376)(36.16119467,706.58799154)
\curveto(36.30118953,706.54798384)(36.4311894,706.4929839)(36.55119467,706.42299154)
\curveto(36.62118921,706.37298402)(36.68618915,706.32298407)(36.74619467,706.27299154)
\curveto(36.80618903,706.23298416)(36.87118896,706.1879842)(36.94119467,706.13799154)
\curveto(36.98118885,706.10798428)(37.0361888,706.06798432)(37.10619467,706.01799154)
\curveto(37.18618865,705.96798442)(37.26118857,705.96798442)(37.33119467,706.01799154)
\curveto(37.37118846,706.03798435)(37.39118844,706.07298432)(37.39119467,706.12299154)
\curveto(37.39118844,706.17298422)(37.40118843,706.22298417)(37.42119467,706.27299154)
\lineto(37.42119467,706.42299154)
\curveto(37.4311884,706.45298394)(37.4361884,706.4879839)(37.43619467,706.52799154)
\lineto(37.43619467,706.64799154)
\lineto(37.43619467,708.68799154)
\curveto(37.4361884,708.79798159)(37.4311884,708.91798147)(37.42119467,709.04799154)
\curveto(37.42118841,709.1879812)(37.44618839,709.2929811)(37.49619467,709.36299154)
\curveto(37.5361883,709.44298095)(37.61118822,709.4929809)(37.72119467,709.51299154)
\curveto(37.74118809,709.52298087)(37.76118807,709.52298087)(37.78119467,709.51299154)
\curveto(37.80118803,709.51298088)(37.82118801,709.51798087)(37.84119467,709.52799154)
\lineto(38.90619467,709.52799154)
\curveto(39.02618681,709.52798086)(39.1361867,709.52298087)(39.23619467,709.51299154)
\curveto(39.3361865,709.50298089)(39.41118642,709.46298093)(39.46119467,709.39299154)
\curveto(39.51118632,709.31298108)(39.5361863,709.20798118)(39.53619467,709.07799154)
\lineto(39.53619467,708.71799154)
\lineto(39.53619467,699.70299154)
\moveto(37.49619467,702.64299154)
\curveto(37.50618833,702.68298771)(37.50618833,702.72298767)(37.49619467,702.76299154)
\lineto(37.49619467,702.89799154)
\curveto(37.49618834,702.99798739)(37.49118834,703.09798729)(37.48119467,703.19799154)
\curveto(37.47118836,703.29798709)(37.45618838,703.387987)(37.43619467,703.46799154)
\curveto(37.41618842,703.57798681)(37.39618844,703.67798671)(37.37619467,703.76799154)
\curveto(37.36618847,703.85798653)(37.34118849,703.94298645)(37.30119467,704.02299154)
\curveto(37.16118867,704.38298601)(36.95618888,704.66798572)(36.68619467,704.87799154)
\curveto(36.42618941,705.0879853)(36.04618979,705.1929852)(35.54619467,705.19299154)
\curveto(35.48619035,705.1929852)(35.40619043,705.18298521)(35.30619467,705.16299154)
\curveto(35.22619061,705.14298525)(35.15119068,705.12298527)(35.08119467,705.10299154)
\curveto(35.02119081,705.0929853)(34.96119087,705.07298532)(34.90119467,705.04299154)
\curveto(34.6311912,704.93298546)(34.42119141,704.76298563)(34.27119467,704.53299154)
\curveto(34.12119171,704.30298609)(34.00119183,704.04298635)(33.91119467,703.75299154)
\curveto(33.88119195,703.65298674)(33.86119197,703.55298684)(33.85119467,703.45299154)
\curveto(33.84119199,703.35298704)(33.82119201,703.24798714)(33.79119467,703.13799154)
\lineto(33.79119467,702.92799154)
\curveto(33.77119206,702.83798755)(33.76619207,702.71298768)(33.77619467,702.55299154)
\curveto(33.78619205,702.40298799)(33.80119203,702.2929881)(33.82119467,702.22299154)
\lineto(33.82119467,702.13299154)
\curveto(33.831192,702.11298828)(33.836192,702.0929883)(33.83619467,702.07299154)
\curveto(33.85619198,701.9929884)(33.87119196,701.91798847)(33.88119467,701.84799154)
\curveto(33.90119193,701.77798861)(33.92119191,701.70298869)(33.94119467,701.62299154)
\curveto(34.11119172,701.10298929)(34.40119143,700.71798967)(34.81119467,700.46799154)
\curveto(34.94119089,700.37799001)(35.12119071,700.30799008)(35.35119467,700.25799154)
\curveto(35.39119044,700.24799014)(35.45119038,700.24299015)(35.53119467,700.24299154)
\curveto(35.56119027,700.23299016)(35.60619023,700.22299017)(35.66619467,700.21299154)
\curveto(35.7361901,700.21299018)(35.79119004,700.21799017)(35.83119467,700.22799154)
\curveto(35.91118992,700.24799014)(35.99118984,700.26299013)(36.07119467,700.27299154)
\curveto(36.15118968,700.28299011)(36.2311896,700.30299009)(36.31119467,700.33299154)
\curveto(36.56118927,700.44298995)(36.76118907,700.58298981)(36.91119467,700.75299154)
\curveto(37.06118877,700.92298947)(37.19118864,701.13798925)(37.30119467,701.39799154)
\curveto(37.34118849,701.4879889)(37.37118846,701.57798881)(37.39119467,701.66799154)
\curveto(37.41118842,701.76798862)(37.4311884,701.87298852)(37.45119467,701.98299154)
\curveto(37.46118837,702.03298836)(37.46118837,702.07798831)(37.45119467,702.11799154)
\curveto(37.45118838,702.16798822)(37.46118837,702.21798817)(37.48119467,702.26799154)
\curveto(37.49118834,702.29798809)(37.49618834,702.33298806)(37.49619467,702.37299154)
\lineto(37.49619467,702.50799154)
\lineto(37.49619467,702.64299154)
}
}
{
\newrgbcolor{curcolor}{0 0 0}
\pscustom[linestyle=none,fillstyle=solid,fillcolor=curcolor]
{
\newpath
\moveto(43.21611654,709.43799154)
\curveto(43.28611359,709.35798103)(43.32111356,709.23798115)(43.32111654,709.07799154)
\lineto(43.32111654,708.61299154)
\lineto(43.32111654,708.20799154)
\curveto(43.32111356,708.06798232)(43.28611359,707.97298242)(43.21611654,707.92299154)
\curveto(43.15611372,707.87298252)(43.0761138,707.84298255)(42.97611654,707.83299154)
\curveto(42.88611399,707.82298257)(42.78611409,707.81798257)(42.67611654,707.81799154)
\lineto(41.83611654,707.81799154)
\curveto(41.72611515,707.81798257)(41.62611525,707.82298257)(41.53611654,707.83299154)
\curveto(41.45611542,707.84298255)(41.38611549,707.87298252)(41.32611654,707.92299154)
\curveto(41.28611559,707.95298244)(41.25611562,708.00798238)(41.23611654,708.08799154)
\curveto(41.22611565,708.17798221)(41.21611566,708.27298212)(41.20611654,708.37299154)
\lineto(41.20611654,708.70299154)
\curveto(41.21611566,708.81298158)(41.22111566,708.90798148)(41.22111654,708.98799154)
\lineto(41.22111654,709.19799154)
\curveto(41.23111565,709.26798112)(41.25111563,709.32798106)(41.28111654,709.37799154)
\curveto(41.30111558,709.41798097)(41.32611555,709.44798094)(41.35611654,709.46799154)
\lineto(41.47611654,709.52799154)
\curveto(41.49611538,709.52798086)(41.52111536,709.52798086)(41.55111654,709.52799154)
\curveto(41.5811153,709.53798085)(41.60611527,709.54298085)(41.62611654,709.54299154)
\lineto(42.72111654,709.54299154)
\curveto(42.82111406,709.54298085)(42.91611396,709.53798085)(43.00611654,709.52799154)
\curveto(43.09611378,709.51798087)(43.16611371,709.4879809)(43.21611654,709.43799154)
\moveto(43.32111654,699.67299154)
\curveto(43.32111356,699.47299092)(43.31611356,699.30299109)(43.30611654,699.16299154)
\curveto(43.29611358,699.02299137)(43.20611367,698.92799146)(43.03611654,698.87799154)
\curveto(42.9761139,698.85799153)(42.91111397,698.84799154)(42.84111654,698.84799154)
\curveto(42.77111411,698.85799153)(42.69611418,698.86299153)(42.61611654,698.86299154)
\lineto(41.77611654,698.86299154)
\curveto(41.68611519,698.86299153)(41.59611528,698.86799152)(41.50611654,698.87799154)
\curveto(41.42611545,698.8879915)(41.36611551,698.91799147)(41.32611654,698.96799154)
\curveto(41.26611561,699.03799135)(41.23111565,699.12299127)(41.22111654,699.22299154)
\lineto(41.22111654,699.56799154)
\lineto(41.22111654,705.89799154)
\lineto(41.22111654,706.19799154)
\curveto(41.22111566,706.29798409)(41.24111564,706.37798401)(41.28111654,706.43799154)
\curveto(41.34111554,706.50798388)(41.42611545,706.55298384)(41.53611654,706.57299154)
\curveto(41.55611532,706.58298381)(41.5811153,706.58298381)(41.61111654,706.57299154)
\curveto(41.65111523,706.57298382)(41.6811152,706.57798381)(41.70111654,706.58799154)
\lineto(42.45111654,706.58799154)
\lineto(42.64611654,706.58799154)
\curveto(42.72611415,706.59798379)(42.79111409,706.59798379)(42.84111654,706.58799154)
\lineto(42.96111654,706.58799154)
\curveto(43.02111386,706.56798382)(43.0761138,706.55298384)(43.12611654,706.54299154)
\curveto(43.1761137,706.53298386)(43.21611366,706.50298389)(43.24611654,706.45299154)
\curveto(43.28611359,706.40298399)(43.30611357,706.33298406)(43.30611654,706.24299154)
\curveto(43.31611356,706.15298424)(43.32111356,706.05798433)(43.32111654,705.95799154)
\lineto(43.32111654,699.67299154)
}
}
{
\newrgbcolor{curcolor}{0 0 0}
\pscustom[linestyle=none,fillstyle=solid,fillcolor=curcolor]
{
\newpath
\moveto(48.55330404,706.79799154)
\curveto(49.36329888,706.81798357)(50.03829821,706.69798369)(50.57830404,706.43799154)
\curveto(51.12829712,706.17798421)(51.56329668,705.80798458)(51.88330404,705.32799154)
\curveto(52.0432962,705.0879853)(52.16329608,704.81298558)(52.24330404,704.50299154)
\curveto(52.26329598,704.45298594)(52.27829597,704.387986)(52.28830404,704.30799154)
\curveto(52.30829594,704.22798616)(52.30829594,704.15798623)(52.28830404,704.09799154)
\curveto(52.248296,703.9879864)(52.17829607,703.92298647)(52.07830404,703.90299154)
\curveto(51.97829627,703.8929865)(51.85829639,703.8879865)(51.71830404,703.88799154)
\lineto(50.93830404,703.88799154)
\lineto(50.65330404,703.88799154)
\curveto(50.56329768,703.8879865)(50.48829776,703.90798648)(50.42830404,703.94799154)
\curveto(50.3482979,703.9879864)(50.29329795,704.04798634)(50.26330404,704.12799154)
\curveto(50.23329801,704.21798617)(50.19329805,704.30798608)(50.14330404,704.39799154)
\curveto(50.08329816,704.50798588)(50.01829823,704.60798578)(49.94830404,704.69799154)
\curveto(49.87829837,704.7879856)(49.79829845,704.86798552)(49.70830404,704.93799154)
\curveto(49.56829868,705.02798536)(49.41329883,705.09798529)(49.24330404,705.14799154)
\curveto(49.18329906,705.16798522)(49.12329912,705.17798521)(49.06330404,705.17799154)
\curveto(49.00329924,705.17798521)(48.9482993,705.1879852)(48.89830404,705.20799154)
\lineto(48.74830404,705.20799154)
\curveto(48.5482997,705.20798518)(48.38829986,705.1879852)(48.26830404,705.14799154)
\curveto(47.97830027,705.05798533)(47.7433005,704.91798547)(47.56330404,704.72799154)
\curveto(47.38330086,704.54798584)(47.23830101,704.32798606)(47.12830404,704.06799154)
\curveto(47.07830117,703.95798643)(47.03830121,703.83798655)(47.00830404,703.70799154)
\curveto(46.98830126,703.5879868)(46.96330128,703.45798693)(46.93330404,703.31799154)
\curveto(46.92330132,703.27798711)(46.91830133,703.23798715)(46.91830404,703.19799154)
\curveto(46.91830133,703.15798723)(46.91330133,703.11798727)(46.90330404,703.07799154)
\curveto(46.88330136,702.97798741)(46.87330137,702.83798755)(46.87330404,702.65799154)
\curveto(46.88330136,702.47798791)(46.89830135,702.33798805)(46.91830404,702.23799154)
\curveto(46.91830133,702.15798823)(46.92330132,702.10298829)(46.93330404,702.07299154)
\curveto(46.95330129,702.00298839)(46.96330128,701.93298846)(46.96330404,701.86299154)
\curveto(46.97330127,701.7929886)(46.98830126,701.72298867)(47.00830404,701.65299154)
\curveto(47.08830116,701.42298897)(47.18330106,701.21298918)(47.29330404,701.02299154)
\curveto(47.40330084,700.83298956)(47.5433007,700.67298972)(47.71330404,700.54299154)
\curveto(47.75330049,700.51298988)(47.81330043,700.47798991)(47.89330404,700.43799154)
\curveto(48.00330024,700.36799002)(48.11330013,700.32299007)(48.22330404,700.30299154)
\curveto(48.3432999,700.28299011)(48.48829976,700.26299013)(48.65830404,700.24299154)
\lineto(48.74830404,700.24299154)
\curveto(48.78829946,700.24299015)(48.81829943,700.24799014)(48.83830404,700.25799154)
\lineto(48.97330404,700.25799154)
\curveto(49.0432992,700.27799011)(49.10829914,700.2929901)(49.16830404,700.30299154)
\curveto(49.23829901,700.32299007)(49.30329894,700.34299005)(49.36330404,700.36299154)
\curveto(49.66329858,700.4929899)(49.89329835,700.68298971)(50.05330404,700.93299154)
\curveto(50.09329815,700.98298941)(50.12829812,701.03798935)(50.15830404,701.09799154)
\curveto(50.18829806,701.16798922)(50.21329803,701.22798916)(50.23330404,701.27799154)
\curveto(50.27329797,701.387989)(50.30829794,701.48298891)(50.33830404,701.56299154)
\curveto(50.36829788,701.65298874)(50.43829781,701.72298867)(50.54830404,701.77299154)
\curveto(50.63829761,701.81298858)(50.78329746,701.82798856)(50.98330404,701.81799154)
\lineto(51.47830404,701.81799154)
\lineto(51.68830404,701.81799154)
\curveto(51.76829648,701.82798856)(51.83329641,701.82298857)(51.88330404,701.80299154)
\lineto(52.00330404,701.80299154)
\lineto(52.12330404,701.77299154)
\curveto(52.16329608,701.77298862)(52.19329605,701.76298863)(52.21330404,701.74299154)
\curveto(52.26329598,701.70298869)(52.29329595,701.64298875)(52.30330404,701.56299154)
\curveto(52.32329592,701.4929889)(52.32329592,701.41798897)(52.30330404,701.33799154)
\curveto(52.21329603,701.00798938)(52.10329614,700.71298968)(51.97330404,700.45299154)
\curveto(51.56329668,699.68299071)(50.90829734,699.14799124)(50.00830404,698.84799154)
\curveto(49.90829834,698.81799157)(49.80329844,698.79799159)(49.69330404,698.78799154)
\curveto(49.58329866,698.76799162)(49.47329877,698.74299165)(49.36330404,698.71299154)
\curveto(49.30329894,698.70299169)(49.243299,698.69799169)(49.18330404,698.69799154)
\curveto(49.12329912,698.69799169)(49.06329918,698.6929917)(49.00330404,698.68299154)
\lineto(48.83830404,698.68299154)
\curveto(48.78829946,698.66299173)(48.71329953,698.65799173)(48.61330404,698.66799154)
\curveto(48.51329973,698.66799172)(48.43829981,698.67299172)(48.38830404,698.68299154)
\curveto(48.30829994,698.70299169)(48.23330001,698.71299168)(48.16330404,698.71299154)
\curveto(48.10330014,698.70299169)(48.03830021,698.70799168)(47.96830404,698.72799154)
\lineto(47.81830404,698.75799154)
\curveto(47.76830048,698.75799163)(47.71830053,698.76299163)(47.66830404,698.77299154)
\curveto(47.55830069,698.80299159)(47.45330079,698.83299156)(47.35330404,698.86299154)
\curveto(47.25330099,698.8929915)(47.15830109,698.92799146)(47.06830404,698.96799154)
\curveto(46.59830165,699.16799122)(46.20330204,699.42299097)(45.88330404,699.73299154)
\curveto(45.56330268,700.05299034)(45.30330294,700.44798994)(45.10330404,700.91799154)
\curveto(45.05330319,701.00798938)(45.01330323,701.10298929)(44.98330404,701.20299154)
\lineto(44.89330404,701.53299154)
\curveto(44.88330336,701.57298882)(44.87830337,701.60798878)(44.87830404,701.63799154)
\curveto(44.87830337,701.67798871)(44.86830338,701.72298867)(44.84830404,701.77299154)
\curveto(44.82830342,701.84298855)(44.81830343,701.91298848)(44.81830404,701.98299154)
\curveto(44.81830343,702.06298833)(44.80830344,702.13798825)(44.78830404,702.20799154)
\lineto(44.78830404,702.46299154)
\curveto(44.76830348,702.51298788)(44.75830349,702.56798782)(44.75830404,702.62799154)
\curveto(44.75830349,702.69798769)(44.76830348,702.75798763)(44.78830404,702.80799154)
\curveto(44.79830345,702.85798753)(44.79830345,702.90298749)(44.78830404,702.94299154)
\curveto(44.77830347,702.98298741)(44.77830347,703.02298737)(44.78830404,703.06299154)
\curveto(44.80830344,703.13298726)(44.81330343,703.19798719)(44.80330404,703.25799154)
\curveto(44.80330344,703.31798707)(44.81330343,703.37798701)(44.83330404,703.43799154)
\curveto(44.88330336,703.61798677)(44.92330332,703.7879866)(44.95330404,703.94799154)
\curveto(44.98330326,704.11798627)(45.02830322,704.28298611)(45.08830404,704.44299154)
\curveto(45.30830294,704.95298544)(45.58330266,705.37798501)(45.91330404,705.71799154)
\curveto(46.25330199,706.05798433)(46.68330156,706.33298406)(47.20330404,706.54299154)
\curveto(47.3433009,706.60298379)(47.48830076,706.64298375)(47.63830404,706.66299154)
\curveto(47.78830046,706.6929837)(47.9433003,706.72798366)(48.10330404,706.76799154)
\curveto(48.18330006,706.77798361)(48.25829999,706.78298361)(48.32830404,706.78299154)
\curveto(48.39829985,706.78298361)(48.47329977,706.7879836)(48.55330404,706.79799154)
}
}
{
\newrgbcolor{curcolor}{0 0 0}
\pscustom[linestyle=none,fillstyle=solid,fillcolor=curcolor]
{
\newpath
\moveto(60.64658529,699.44799154)
\curveto(60.66657744,699.33799105)(60.67657743,699.22799116)(60.67658529,699.11799154)
\curveto(60.68657742,699.00799138)(60.63657747,698.93299146)(60.52658529,698.89299154)
\curveto(60.46657764,698.86299153)(60.39657771,698.84799154)(60.31658529,698.84799154)
\lineto(60.07658529,698.84799154)
\lineto(59.26658529,698.84799154)
\lineto(58.99658529,698.84799154)
\curveto(58.91657919,698.85799153)(58.85157926,698.88299151)(58.80158529,698.92299154)
\curveto(58.73157938,698.96299143)(58.67657943,699.01799137)(58.63658529,699.08799154)
\curveto(58.6065795,699.16799122)(58.56157955,699.23299116)(58.50158529,699.28299154)
\curveto(58.48157963,699.30299109)(58.45657965,699.31799107)(58.42658529,699.32799154)
\curveto(58.39657971,699.34799104)(58.35657975,699.35299104)(58.30658529,699.34299154)
\curveto(58.25657985,699.32299107)(58.2065799,699.29799109)(58.15658529,699.26799154)
\curveto(58.11657999,699.23799115)(58.07158004,699.21299118)(58.02158529,699.19299154)
\curveto(57.97158014,699.15299124)(57.91658019,699.11799127)(57.85658529,699.08799154)
\lineto(57.67658529,698.99799154)
\curveto(57.54658056,698.93799145)(57.4115807,698.8879915)(57.27158529,698.84799154)
\curveto(57.13158098,698.81799157)(56.98658112,698.78299161)(56.83658529,698.74299154)
\curveto(56.76658134,698.72299167)(56.69658141,698.71299168)(56.62658529,698.71299154)
\curveto(56.56658154,698.70299169)(56.50158161,698.6929917)(56.43158529,698.68299154)
\lineto(56.34158529,698.68299154)
\curveto(56.3115818,698.67299172)(56.28158183,698.66799172)(56.25158529,698.66799154)
\lineto(56.08658529,698.66799154)
\curveto(55.98658212,698.64799174)(55.88658222,698.64799174)(55.78658529,698.66799154)
\lineto(55.65158529,698.66799154)
\curveto(55.58158253,698.6879917)(55.5115826,698.69799169)(55.44158529,698.69799154)
\curveto(55.38158273,698.6879917)(55.32158279,698.6929917)(55.26158529,698.71299154)
\curveto(55.16158295,698.73299166)(55.06658304,698.75299164)(54.97658529,698.77299154)
\curveto(54.88658322,698.78299161)(54.80158331,698.80799158)(54.72158529,698.84799154)
\curveto(54.43158368,698.95799143)(54.18158393,699.09799129)(53.97158529,699.26799154)
\curveto(53.77158434,699.44799094)(53.6115845,699.68299071)(53.49158529,699.97299154)
\curveto(53.46158465,700.04299035)(53.43158468,700.11799027)(53.40158529,700.19799154)
\curveto(53.38158473,700.27799011)(53.36158475,700.36299003)(53.34158529,700.45299154)
\curveto(53.32158479,700.50298989)(53.3115848,700.55298984)(53.31158529,700.60299154)
\curveto(53.32158479,700.65298974)(53.32158479,700.70298969)(53.31158529,700.75299154)
\curveto(53.30158481,700.78298961)(53.29158482,700.84298955)(53.28158529,700.93299154)
\curveto(53.28158483,701.03298936)(53.28658482,701.10298929)(53.29658529,701.14299154)
\curveto(53.31658479,701.24298915)(53.32658478,701.32798906)(53.32658529,701.39799154)
\lineto(53.41658529,701.72799154)
\curveto(53.44658466,701.84798854)(53.48658462,701.95298844)(53.53658529,702.04299154)
\curveto(53.7065844,702.33298806)(53.90158421,702.55298784)(54.12158529,702.70299154)
\curveto(54.34158377,702.85298754)(54.62158349,702.98298741)(54.96158529,703.09299154)
\curveto(55.09158302,703.14298725)(55.22658288,703.17798721)(55.36658529,703.19799154)
\curveto(55.5065826,703.21798717)(55.64658246,703.24298715)(55.78658529,703.27299154)
\curveto(55.86658224,703.2929871)(55.95158216,703.30298709)(56.04158529,703.30299154)
\curveto(56.13158198,703.31298708)(56.22158189,703.32798706)(56.31158529,703.34799154)
\curveto(56.38158173,703.36798702)(56.45158166,703.37298702)(56.52158529,703.36299154)
\curveto(56.59158152,703.36298703)(56.66658144,703.37298702)(56.74658529,703.39299154)
\curveto(56.81658129,703.41298698)(56.88658122,703.42298697)(56.95658529,703.42299154)
\curveto(57.02658108,703.42298697)(57.10158101,703.43298696)(57.18158529,703.45299154)
\curveto(57.39158072,703.50298689)(57.58158053,703.54298685)(57.75158529,703.57299154)
\curveto(57.93158018,703.61298678)(58.09158002,703.70298669)(58.23158529,703.84299154)
\curveto(58.32157979,703.93298646)(58.38157973,704.03298636)(58.41158529,704.14299154)
\curveto(58.42157969,704.17298622)(58.42157969,704.19798619)(58.41158529,704.21799154)
\curveto(58.4115797,704.23798615)(58.41657969,704.25798613)(58.42658529,704.27799154)
\curveto(58.43657967,704.29798609)(58.44157967,704.32798606)(58.44158529,704.36799154)
\lineto(58.44158529,704.45799154)
\lineto(58.41158529,704.57799154)
\curveto(58.4115797,704.61798577)(58.4065797,704.65298574)(58.39658529,704.68299154)
\curveto(58.29657981,704.98298541)(58.08658002,705.1879852)(57.76658529,705.29799154)
\curveto(57.67658043,705.32798506)(57.56658054,705.34798504)(57.43658529,705.35799154)
\curveto(57.31658079,705.37798501)(57.19158092,705.38298501)(57.06158529,705.37299154)
\curveto(56.93158118,705.37298502)(56.8065813,705.36298503)(56.68658529,705.34299154)
\curveto(56.56658154,705.32298507)(56.46158165,705.29798509)(56.37158529,705.26799154)
\curveto(56.3115818,705.24798514)(56.25158186,705.21798517)(56.19158529,705.17799154)
\curveto(56.14158197,705.14798524)(56.09158202,705.11298528)(56.04158529,705.07299154)
\curveto(55.99158212,705.03298536)(55.93658217,704.97798541)(55.87658529,704.90799154)
\curveto(55.82658228,704.83798555)(55.79158232,704.77298562)(55.77158529,704.71299154)
\curveto(55.72158239,704.61298578)(55.67658243,704.51798587)(55.63658529,704.42799154)
\curveto(55.6065825,704.33798605)(55.53658257,704.27798611)(55.42658529,704.24799154)
\curveto(55.34658276,704.22798616)(55.26158285,704.21798617)(55.17158529,704.21799154)
\lineto(54.90158529,704.21799154)
\lineto(54.33158529,704.21799154)
\curveto(54.28158383,704.21798617)(54.23158388,704.21298618)(54.18158529,704.20299154)
\curveto(54.13158398,704.20298619)(54.08658402,704.20798618)(54.04658529,704.21799154)
\lineto(53.91158529,704.21799154)
\curveto(53.89158422,704.22798616)(53.86658424,704.23298616)(53.83658529,704.23299154)
\curveto(53.8065843,704.23298616)(53.78158433,704.24298615)(53.76158529,704.26299154)
\curveto(53.68158443,704.28298611)(53.62658448,704.34798604)(53.59658529,704.45799154)
\curveto(53.58658452,704.50798588)(53.58658452,704.55798583)(53.59658529,704.60799154)
\curveto(53.6065845,704.65798573)(53.61658449,704.70298569)(53.62658529,704.74299154)
\curveto(53.65658445,704.85298554)(53.68658442,704.95298544)(53.71658529,705.04299154)
\curveto(53.75658435,705.14298525)(53.80158431,705.23298516)(53.85158529,705.31299154)
\lineto(53.94158529,705.46299154)
\lineto(54.03158529,705.61299154)
\curveto(54.111584,705.72298467)(54.2115839,705.82798456)(54.33158529,705.92799154)
\curveto(54.35158376,705.93798445)(54.38158373,705.96298443)(54.42158529,706.00299154)
\curveto(54.47158364,706.04298435)(54.51658359,706.07798431)(54.55658529,706.10799154)
\curveto(54.59658351,706.13798425)(54.64158347,706.16798422)(54.69158529,706.19799154)
\curveto(54.86158325,706.30798408)(55.04158307,706.392984)(55.23158529,706.45299154)
\curveto(55.42158269,706.52298387)(55.61658249,706.5879838)(55.81658529,706.64799154)
\curveto(55.93658217,706.67798371)(56.06158205,706.69798369)(56.19158529,706.70799154)
\curveto(56.32158179,706.71798367)(56.45158166,706.73798365)(56.58158529,706.76799154)
\curveto(56.62158149,706.77798361)(56.68158143,706.77798361)(56.76158529,706.76799154)
\curveto(56.85158126,706.75798363)(56.9065812,706.76298363)(56.92658529,706.78299154)
\curveto(57.33658077,706.7929836)(57.72658038,706.77798361)(58.09658529,706.73799154)
\curveto(58.47657963,706.69798369)(58.81657929,706.62298377)(59.11658529,706.51299154)
\curveto(59.42657868,706.40298399)(59.69157842,706.25298414)(59.91158529,706.06299154)
\curveto(60.13157798,705.88298451)(60.30157781,705.64798474)(60.42158529,705.35799154)
\curveto(60.49157762,705.1879852)(60.53157758,704.9929854)(60.54158529,704.77299154)
\curveto(60.55157756,704.55298584)(60.55657755,704.32798606)(60.55658529,704.09799154)
\lineto(60.55658529,700.75299154)
\lineto(60.55658529,700.16799154)
\curveto(60.55657755,699.97799041)(60.57657753,699.80299059)(60.61658529,699.64299154)
\curveto(60.62657748,699.61299078)(60.63157748,699.57799081)(60.63158529,699.53799154)
\curveto(60.63157748,699.50799088)(60.63657747,699.47799091)(60.64658529,699.44799154)
\moveto(58.44158529,701.75799154)
\curveto(58.45157966,701.80798858)(58.45657965,701.86298853)(58.45658529,701.92299154)
\curveto(58.45657965,701.9929884)(58.45157966,702.05298834)(58.44158529,702.10299154)
\curveto(58.42157969,702.16298823)(58.4115797,702.21798817)(58.41158529,702.26799154)
\curveto(58.4115797,702.31798807)(58.39157972,702.35798803)(58.35158529,702.38799154)
\curveto(58.30157981,702.42798796)(58.22657988,702.44798794)(58.12658529,702.44799154)
\curveto(58.08658002,702.43798795)(58.05158006,702.42798796)(58.02158529,702.41799154)
\curveto(57.99158012,702.41798797)(57.95658015,702.41298798)(57.91658529,702.40299154)
\curveto(57.84658026,702.38298801)(57.77158034,702.36798802)(57.69158529,702.35799154)
\curveto(57.6115805,702.34798804)(57.53158058,702.33298806)(57.45158529,702.31299154)
\curveto(57.42158069,702.30298809)(57.37658073,702.29798809)(57.31658529,702.29799154)
\curveto(57.18658092,702.26798812)(57.05658105,702.24798814)(56.92658529,702.23799154)
\curveto(56.79658131,702.22798816)(56.67158144,702.20298819)(56.55158529,702.16299154)
\curveto(56.47158164,702.14298825)(56.39658171,702.12298827)(56.32658529,702.10299154)
\curveto(56.25658185,702.0929883)(56.18658192,702.07298832)(56.11658529,702.04299154)
\curveto(55.9065822,701.95298844)(55.72658238,701.81798857)(55.57658529,701.63799154)
\curveto(55.43658267,701.45798893)(55.38658272,701.20798918)(55.42658529,700.88799154)
\curveto(55.44658266,700.71798967)(55.50158261,700.57798981)(55.59158529,700.46799154)
\curveto(55.66158245,700.35799003)(55.76658234,700.26799012)(55.90658529,700.19799154)
\curveto(56.04658206,700.13799025)(56.19658191,700.0929903)(56.35658529,700.06299154)
\curveto(56.52658158,700.03299036)(56.70158141,700.02299037)(56.88158529,700.03299154)
\curveto(57.07158104,700.05299034)(57.24658086,700.0879903)(57.40658529,700.13799154)
\curveto(57.66658044,700.21799017)(57.87158024,700.34299005)(58.02158529,700.51299154)
\curveto(58.17157994,700.6929897)(58.28657982,700.91298948)(58.36658529,701.17299154)
\curveto(58.38657972,701.24298915)(58.39657971,701.31298908)(58.39658529,701.38299154)
\curveto(58.4065797,701.46298893)(58.42157969,701.54298885)(58.44158529,701.62299154)
\lineto(58.44158529,701.75799154)
}
}
{
\newrgbcolor{curcolor}{0 0 0}
\pscustom[linestyle=none,fillstyle=solid,fillcolor=curcolor]
{
\newpath
\moveto(69.79986654,699.70299154)
\lineto(69.79986654,699.28299154)
\curveto(69.79985817,699.15299124)(69.7698582,699.04799134)(69.70986654,698.96799154)
\curveto(69.65985831,698.91799147)(69.59485838,698.88299151)(69.51486654,698.86299154)
\curveto(69.43485854,698.85299154)(69.34485863,698.84799154)(69.24486654,698.84799154)
\lineto(68.41986654,698.84799154)
\lineto(68.13486654,698.84799154)
\curveto(68.05485992,698.85799153)(67.98985998,698.88299151)(67.93986654,698.92299154)
\curveto(67.8698601,698.97299142)(67.82986014,699.03799135)(67.81986654,699.11799154)
\curveto(67.80986016,699.19799119)(67.78986018,699.27799111)(67.75986654,699.35799154)
\curveto(67.73986023,699.37799101)(67.71986025,699.392991)(67.69986654,699.40299154)
\curveto(67.68986028,699.42299097)(67.6748603,699.44299095)(67.65486654,699.46299154)
\curveto(67.54486043,699.46299093)(67.46486051,699.43799095)(67.41486654,699.38799154)
\lineto(67.26486654,699.23799154)
\curveto(67.19486078,699.1879912)(67.12986084,699.14299125)(67.06986654,699.10299154)
\curveto(67.00986096,699.07299132)(66.94486103,699.03299136)(66.87486654,698.98299154)
\curveto(66.83486114,698.96299143)(66.78986118,698.94299145)(66.73986654,698.92299154)
\curveto(66.69986127,698.90299149)(66.65486132,698.88299151)(66.60486654,698.86299154)
\curveto(66.46486151,698.81299158)(66.31486166,698.76799162)(66.15486654,698.72799154)
\curveto(66.10486187,698.70799168)(66.05986191,698.69799169)(66.01986654,698.69799154)
\curveto(65.97986199,698.69799169)(65.93986203,698.6929917)(65.89986654,698.68299154)
\lineto(65.76486654,698.68299154)
\curveto(65.73486224,698.67299172)(65.69486228,698.66799172)(65.64486654,698.66799154)
\lineto(65.50986654,698.66799154)
\curveto(65.44986252,698.64799174)(65.35986261,698.64299175)(65.23986654,698.65299154)
\curveto(65.11986285,698.65299174)(65.03486294,698.66299173)(64.98486654,698.68299154)
\curveto(64.91486306,698.70299169)(64.84986312,698.71299168)(64.78986654,698.71299154)
\curveto(64.73986323,698.70299169)(64.68486329,698.70799168)(64.62486654,698.72799154)
\lineto(64.26486654,698.84799154)
\curveto(64.15486382,698.87799151)(64.04486393,698.91799147)(63.93486654,698.96799154)
\curveto(63.58486439,699.11799127)(63.2698647,699.34799104)(62.98986654,699.65799154)
\curveto(62.71986525,699.97799041)(62.50486547,700.31299008)(62.34486654,700.66299154)
\curveto(62.29486568,700.77298962)(62.25486572,700.87798951)(62.22486654,700.97799154)
\curveto(62.19486578,701.0879893)(62.15986581,701.19798919)(62.11986654,701.30799154)
\curveto(62.10986586,701.34798904)(62.10486587,701.38298901)(62.10486654,701.41299154)
\curveto(62.10486587,701.45298894)(62.09486588,701.49798889)(62.07486654,701.54799154)
\curveto(62.05486592,701.62798876)(62.03486594,701.71298868)(62.01486654,701.80299154)
\curveto(62.00486597,701.90298849)(61.98986598,702.00298839)(61.96986654,702.10299154)
\curveto(61.95986601,702.13298826)(61.95486602,702.16798822)(61.95486654,702.20799154)
\curveto(61.96486601,702.24798814)(61.96486601,702.28298811)(61.95486654,702.31299154)
\lineto(61.95486654,702.44799154)
\curveto(61.95486602,702.49798789)(61.94986602,702.54798784)(61.93986654,702.59799154)
\curveto(61.92986604,702.64798774)(61.92486605,702.70298769)(61.92486654,702.76299154)
\curveto(61.92486605,702.83298756)(61.92986604,702.8879875)(61.93986654,702.92799154)
\curveto(61.94986602,702.97798741)(61.95486602,703.02298737)(61.95486654,703.06299154)
\lineto(61.95486654,703.21299154)
\curveto(61.96486601,703.26298713)(61.96486601,703.30798708)(61.95486654,703.34799154)
\curveto(61.95486602,703.39798699)(61.96486601,703.44798694)(61.98486654,703.49799154)
\curveto(62.00486597,703.60798678)(62.01986595,703.71298668)(62.02986654,703.81299154)
\curveto(62.04986592,703.91298648)(62.0748659,704.01298638)(62.10486654,704.11299154)
\curveto(62.14486583,704.23298616)(62.17986579,704.34798604)(62.20986654,704.45799154)
\curveto(62.23986573,704.56798582)(62.27986569,704.67798571)(62.32986654,704.78799154)
\curveto(62.4698655,705.0879853)(62.64486533,705.37298502)(62.85486654,705.64299154)
\curveto(62.8748651,705.67298472)(62.89986507,705.69798469)(62.92986654,705.71799154)
\curveto(62.969865,705.74798464)(62.99986497,705.77798461)(63.01986654,705.80799154)
\curveto(63.05986491,705.85798453)(63.09986487,705.90298449)(63.13986654,705.94299154)
\curveto(63.17986479,705.98298441)(63.22486475,706.02298437)(63.27486654,706.06299154)
\curveto(63.31486466,706.08298431)(63.34986462,706.10798428)(63.37986654,706.13799154)
\curveto(63.40986456,706.17798421)(63.44486453,706.20798418)(63.48486654,706.22799154)
\curveto(63.73486424,706.39798399)(64.02486395,706.53798385)(64.35486654,706.64799154)
\curveto(64.42486355,706.66798372)(64.49486348,706.68298371)(64.56486654,706.69299154)
\curveto(64.64486333,706.70298369)(64.72486325,706.71798367)(64.80486654,706.73799154)
\curveto(64.8748631,706.75798363)(64.96486301,706.76798362)(65.07486654,706.76799154)
\curveto(65.18486279,706.77798361)(65.29486268,706.78298361)(65.40486654,706.78299154)
\curveto(65.51486246,706.78298361)(65.61986235,706.77798361)(65.71986654,706.76799154)
\curveto(65.82986214,706.75798363)(65.91986205,706.74298365)(65.98986654,706.72299154)
\curveto(66.13986183,706.67298372)(66.28486169,706.62798376)(66.42486654,706.58799154)
\curveto(66.56486141,706.54798384)(66.69486128,706.4929839)(66.81486654,706.42299154)
\curveto(66.88486109,706.37298402)(66.94986102,706.32298407)(67.00986654,706.27299154)
\curveto(67.0698609,706.23298416)(67.13486084,706.1879842)(67.20486654,706.13799154)
\curveto(67.24486073,706.10798428)(67.29986067,706.06798432)(67.36986654,706.01799154)
\curveto(67.44986052,705.96798442)(67.52486045,705.96798442)(67.59486654,706.01799154)
\curveto(67.63486034,706.03798435)(67.65486032,706.07298432)(67.65486654,706.12299154)
\curveto(67.65486032,706.17298422)(67.66486031,706.22298417)(67.68486654,706.27299154)
\lineto(67.68486654,706.42299154)
\curveto(67.69486028,706.45298394)(67.69986027,706.4879839)(67.69986654,706.52799154)
\lineto(67.69986654,706.64799154)
\lineto(67.69986654,708.68799154)
\curveto(67.69986027,708.79798159)(67.69486028,708.91798147)(67.68486654,709.04799154)
\curveto(67.68486029,709.1879812)(67.70986026,709.2929811)(67.75986654,709.36299154)
\curveto(67.79986017,709.44298095)(67.8748601,709.4929809)(67.98486654,709.51299154)
\curveto(68.00485997,709.52298087)(68.02485995,709.52298087)(68.04486654,709.51299154)
\curveto(68.06485991,709.51298088)(68.08485989,709.51798087)(68.10486654,709.52799154)
\lineto(69.16986654,709.52799154)
\curveto(69.28985868,709.52798086)(69.39985857,709.52298087)(69.49986654,709.51299154)
\curveto(69.59985837,709.50298089)(69.6748583,709.46298093)(69.72486654,709.39299154)
\curveto(69.7748582,709.31298108)(69.79985817,709.20798118)(69.79986654,709.07799154)
\lineto(69.79986654,708.71799154)
\lineto(69.79986654,699.70299154)
\moveto(67.75986654,702.64299154)
\curveto(67.7698602,702.68298771)(67.7698602,702.72298767)(67.75986654,702.76299154)
\lineto(67.75986654,702.89799154)
\curveto(67.75986021,702.99798739)(67.75486022,703.09798729)(67.74486654,703.19799154)
\curveto(67.73486024,703.29798709)(67.71986025,703.387987)(67.69986654,703.46799154)
\curveto(67.67986029,703.57798681)(67.65986031,703.67798671)(67.63986654,703.76799154)
\curveto(67.62986034,703.85798653)(67.60486037,703.94298645)(67.56486654,704.02299154)
\curveto(67.42486055,704.38298601)(67.21986075,704.66798572)(66.94986654,704.87799154)
\curveto(66.68986128,705.0879853)(66.30986166,705.1929852)(65.80986654,705.19299154)
\curveto(65.74986222,705.1929852)(65.6698623,705.18298521)(65.56986654,705.16299154)
\curveto(65.48986248,705.14298525)(65.41486256,705.12298527)(65.34486654,705.10299154)
\curveto(65.28486269,705.0929853)(65.22486275,705.07298532)(65.16486654,705.04299154)
\curveto(64.89486308,704.93298546)(64.68486329,704.76298563)(64.53486654,704.53299154)
\curveto(64.38486359,704.30298609)(64.26486371,704.04298635)(64.17486654,703.75299154)
\curveto(64.14486383,703.65298674)(64.12486385,703.55298684)(64.11486654,703.45299154)
\curveto(64.10486387,703.35298704)(64.08486389,703.24798714)(64.05486654,703.13799154)
\lineto(64.05486654,702.92799154)
\curveto(64.03486394,702.83798755)(64.02986394,702.71298768)(64.03986654,702.55299154)
\curveto(64.04986392,702.40298799)(64.06486391,702.2929881)(64.08486654,702.22299154)
\lineto(64.08486654,702.13299154)
\curveto(64.09486388,702.11298828)(64.09986387,702.0929883)(64.09986654,702.07299154)
\curveto(64.11986385,701.9929884)(64.13486384,701.91798847)(64.14486654,701.84799154)
\curveto(64.16486381,701.77798861)(64.18486379,701.70298869)(64.20486654,701.62299154)
\curveto(64.3748636,701.10298929)(64.66486331,700.71798967)(65.07486654,700.46799154)
\curveto(65.20486277,700.37799001)(65.38486259,700.30799008)(65.61486654,700.25799154)
\curveto(65.65486232,700.24799014)(65.71486226,700.24299015)(65.79486654,700.24299154)
\curveto(65.82486215,700.23299016)(65.8698621,700.22299017)(65.92986654,700.21299154)
\curveto(65.99986197,700.21299018)(66.05486192,700.21799017)(66.09486654,700.22799154)
\curveto(66.1748618,700.24799014)(66.25486172,700.26299013)(66.33486654,700.27299154)
\curveto(66.41486156,700.28299011)(66.49486148,700.30299009)(66.57486654,700.33299154)
\curveto(66.82486115,700.44298995)(67.02486095,700.58298981)(67.17486654,700.75299154)
\curveto(67.32486065,700.92298947)(67.45486052,701.13798925)(67.56486654,701.39799154)
\curveto(67.60486037,701.4879889)(67.63486034,701.57798881)(67.65486654,701.66799154)
\curveto(67.6748603,701.76798862)(67.69486028,701.87298852)(67.71486654,701.98299154)
\curveto(67.72486025,702.03298836)(67.72486025,702.07798831)(67.71486654,702.11799154)
\curveto(67.71486026,702.16798822)(67.72486025,702.21798817)(67.74486654,702.26799154)
\curveto(67.75486022,702.29798809)(67.75986021,702.33298806)(67.75986654,702.37299154)
\lineto(67.75986654,702.50799154)
\lineto(67.75986654,702.64299154)
}
}
{
\newrgbcolor{curcolor}{0 0 0}
\pscustom[linestyle=none,fillstyle=solid,fillcolor=curcolor]
{
\newpath
\moveto(79.14978842,703.03299154)
\curveto(79.16977985,702.97298742)(79.17977984,702.8879875)(79.17978842,702.77799154)
\curveto(79.17977984,702.66798772)(79.16977985,702.58298781)(79.14978842,702.52299154)
\lineto(79.14978842,702.37299154)
\curveto(79.12977989,702.2929881)(79.1197799,702.21298818)(79.11978842,702.13299154)
\curveto(79.12977989,702.05298834)(79.12477989,701.97298842)(79.10478842,701.89299154)
\curveto(79.08477993,701.82298857)(79.06977995,701.75798863)(79.05978842,701.69799154)
\curveto(79.04977997,701.63798875)(79.03977998,701.57298882)(79.02978842,701.50299154)
\curveto(78.98978003,701.392989)(78.95478006,701.27798911)(78.92478842,701.15799154)
\curveto(78.89478012,701.04798934)(78.85478016,700.94298945)(78.80478842,700.84299154)
\curveto(78.59478042,700.36299003)(78.3197807,699.97299042)(77.97978842,699.67299154)
\curveto(77.63978138,699.37299102)(77.22978179,699.12299127)(76.74978842,698.92299154)
\curveto(76.62978239,698.87299152)(76.50478251,698.83799155)(76.37478842,698.81799154)
\curveto(76.25478276,698.7879916)(76.12978289,698.75799163)(75.99978842,698.72799154)
\curveto(75.94978307,698.70799168)(75.89478312,698.69799169)(75.83478842,698.69799154)
\curveto(75.77478324,698.69799169)(75.7197833,698.6929917)(75.66978842,698.68299154)
\lineto(75.56478842,698.68299154)
\curveto(75.53478348,698.67299172)(75.50478351,698.66799172)(75.47478842,698.66799154)
\curveto(75.42478359,698.65799173)(75.34478367,698.65299174)(75.23478842,698.65299154)
\curveto(75.12478389,698.64299175)(75.03978398,698.64799174)(74.97978842,698.66799154)
\lineto(74.82978842,698.66799154)
\curveto(74.77978424,698.67799171)(74.72478429,698.68299171)(74.66478842,698.68299154)
\curveto(74.6147844,698.67299172)(74.56478445,698.67799171)(74.51478842,698.69799154)
\curveto(74.47478454,698.70799168)(74.43478458,698.71299168)(74.39478842,698.71299154)
\curveto(74.36478465,698.71299168)(74.32478469,698.71799167)(74.27478842,698.72799154)
\curveto(74.17478484,698.75799163)(74.07478494,698.78299161)(73.97478842,698.80299154)
\curveto(73.87478514,698.82299157)(73.77978524,698.85299154)(73.68978842,698.89299154)
\curveto(73.56978545,698.93299146)(73.45478556,698.97299142)(73.34478842,699.01299154)
\curveto(73.24478577,699.05299134)(73.13978588,699.10299129)(73.02978842,699.16299154)
\curveto(72.67978634,699.37299102)(72.37978664,699.61799077)(72.12978842,699.89799154)
\curveto(71.87978714,700.17799021)(71.66978735,700.51298988)(71.49978842,700.90299154)
\curveto(71.44978757,700.9929894)(71.40978761,701.0879893)(71.37978842,701.18799154)
\curveto(71.35978766,701.2879891)(71.33478768,701.392989)(71.30478842,701.50299154)
\curveto(71.28478773,701.55298884)(71.27478774,701.59798879)(71.27478842,701.63799154)
\curveto(71.27478774,701.67798871)(71.26478775,701.72298867)(71.24478842,701.77299154)
\curveto(71.22478779,701.85298854)(71.2147878,701.93298846)(71.21478842,702.01299154)
\curveto(71.2147878,702.10298829)(71.20478781,702.1879882)(71.18478842,702.26799154)
\curveto(71.17478784,702.31798807)(71.16978785,702.36298803)(71.16978842,702.40299154)
\lineto(71.16978842,702.53799154)
\curveto(71.14978787,702.59798779)(71.13978788,702.68298771)(71.13978842,702.79299154)
\curveto(71.14978787,702.90298749)(71.16478785,702.9879874)(71.18478842,703.04799154)
\lineto(71.18478842,703.15299154)
\curveto(71.19478782,703.20298719)(71.19478782,703.25298714)(71.18478842,703.30299154)
\curveto(71.18478783,703.36298703)(71.19478782,703.41798697)(71.21478842,703.46799154)
\curveto(71.22478779,703.51798687)(71.22978779,703.56298683)(71.22978842,703.60299154)
\curveto(71.22978779,703.65298674)(71.23978778,703.70298669)(71.25978842,703.75299154)
\curveto(71.29978772,703.88298651)(71.33478768,704.00798638)(71.36478842,704.12799154)
\curveto(71.39478762,704.25798613)(71.43478758,704.38298601)(71.48478842,704.50299154)
\curveto(71.66478735,704.91298548)(71.87978714,705.25298514)(72.12978842,705.52299154)
\curveto(72.37978664,705.80298459)(72.68478633,706.05798433)(73.04478842,706.28799154)
\curveto(73.14478587,706.33798405)(73.24978577,706.38298401)(73.35978842,706.42299154)
\curveto(73.46978555,706.46298393)(73.57978544,706.50798388)(73.68978842,706.55799154)
\curveto(73.8197852,706.60798378)(73.95478506,706.64298375)(74.09478842,706.66299154)
\curveto(74.23478478,706.68298371)(74.37978464,706.71298368)(74.52978842,706.75299154)
\curveto(74.60978441,706.76298363)(74.68478433,706.76798362)(74.75478842,706.76799154)
\curveto(74.82478419,706.76798362)(74.89478412,706.77298362)(74.96478842,706.78299154)
\curveto(75.54478347,706.7929836)(76.04478297,706.73298366)(76.46478842,706.60299154)
\curveto(76.89478212,706.47298392)(77.27478174,706.2929841)(77.60478842,706.06299154)
\curveto(77.7147813,705.98298441)(77.82478119,705.8929845)(77.93478842,705.79299154)
\curveto(78.05478096,705.70298469)(78.15478086,705.60298479)(78.23478842,705.49299154)
\curveto(78.3147807,705.392985)(78.38478063,705.2929851)(78.44478842,705.19299154)
\curveto(78.5147805,705.0929853)(78.58478043,704.9879854)(78.65478842,704.87799154)
\curveto(78.72478029,704.76798562)(78.77978024,704.64798574)(78.81978842,704.51799154)
\curveto(78.85978016,704.39798599)(78.90478011,704.26798612)(78.95478842,704.12799154)
\curveto(78.98478003,704.04798634)(79.00978001,703.96298643)(79.02978842,703.87299154)
\lineto(79.08978842,703.60299154)
\curveto(79.09977992,703.56298683)(79.10477991,703.52298687)(79.10478842,703.48299154)
\curveto(79.10477991,703.44298695)(79.10977991,703.40298699)(79.11978842,703.36299154)
\curveto(79.13977988,703.31298708)(79.14477987,703.25798713)(79.13478842,703.19799154)
\curveto(79.12477989,703.13798725)(79.12977989,703.08298731)(79.14978842,703.03299154)
\moveto(77.04978842,702.49299154)
\curveto(77.05978196,702.54298785)(77.06478195,702.61298778)(77.06478842,702.70299154)
\curveto(77.06478195,702.80298759)(77.05978196,702.87798751)(77.04978842,702.92799154)
\lineto(77.04978842,703.04799154)
\curveto(77.02978199,703.09798729)(77.019782,703.15298724)(77.01978842,703.21299154)
\curveto(77.019782,703.27298712)(77.014782,703.32798706)(77.00478842,703.37799154)
\curveto(77.00478201,703.41798697)(76.99978202,703.44798694)(76.98978842,703.46799154)
\lineto(76.92978842,703.70799154)
\curveto(76.9197821,703.79798659)(76.89978212,703.88298651)(76.86978842,703.96299154)
\curveto(76.75978226,704.22298617)(76.62978239,704.44298595)(76.47978842,704.62299154)
\curveto(76.32978269,704.81298558)(76.12978289,704.96298543)(75.87978842,705.07299154)
\curveto(75.8197832,705.0929853)(75.75978326,705.10798528)(75.69978842,705.11799154)
\curveto(75.63978338,705.13798525)(75.57478344,705.15798523)(75.50478842,705.17799154)
\curveto(75.42478359,705.19798519)(75.33978368,705.20298519)(75.24978842,705.19299154)
\lineto(74.97978842,705.19299154)
\curveto(74.94978407,705.17298522)(74.9147841,705.16298523)(74.87478842,705.16299154)
\curveto(74.83478418,705.17298522)(74.79978422,705.17298522)(74.76978842,705.16299154)
\lineto(74.55978842,705.10299154)
\curveto(74.49978452,705.0929853)(74.44478457,705.07298532)(74.39478842,705.04299154)
\curveto(74.14478487,704.93298546)(73.93978508,704.77298562)(73.77978842,704.56299154)
\curveto(73.62978539,704.36298603)(73.50978551,704.12798626)(73.41978842,703.85799154)
\curveto(73.38978563,703.75798663)(73.36478565,703.65298674)(73.34478842,703.54299154)
\curveto(73.33478568,703.43298696)(73.3197857,703.32298707)(73.29978842,703.21299154)
\curveto(73.28978573,703.16298723)(73.28478573,703.11298728)(73.28478842,703.06299154)
\lineto(73.28478842,702.91299154)
\curveto(73.26478575,702.84298755)(73.25478576,702.73798765)(73.25478842,702.59799154)
\curveto(73.26478575,702.45798793)(73.27978574,702.35298804)(73.29978842,702.28299154)
\lineto(73.29978842,702.14799154)
\curveto(73.3197857,702.06798832)(73.33478568,701.9879884)(73.34478842,701.90799154)
\curveto(73.35478566,701.83798855)(73.36978565,701.76298863)(73.38978842,701.68299154)
\curveto(73.48978553,701.38298901)(73.59478542,701.13798925)(73.70478842,700.94799154)
\curveto(73.82478519,700.76798962)(74.00978501,700.60298979)(74.25978842,700.45299154)
\curveto(74.32978469,700.40298999)(74.40478461,700.36299003)(74.48478842,700.33299154)
\curveto(74.57478444,700.30299009)(74.66478435,700.27799011)(74.75478842,700.25799154)
\curveto(74.79478422,700.24799014)(74.82978419,700.24299015)(74.85978842,700.24299154)
\curveto(74.88978413,700.25299014)(74.92478409,700.25299014)(74.96478842,700.24299154)
\lineto(75.08478842,700.21299154)
\curveto(75.13478388,700.21299018)(75.17978384,700.21799017)(75.21978842,700.22799154)
\lineto(75.33978842,700.22799154)
\curveto(75.4197836,700.24799014)(75.49978352,700.26299013)(75.57978842,700.27299154)
\curveto(75.65978336,700.28299011)(75.73478328,700.30299009)(75.80478842,700.33299154)
\curveto(76.06478295,700.43298996)(76.27478274,700.56798982)(76.43478842,700.73799154)
\curveto(76.59478242,700.90798948)(76.72978229,701.11798927)(76.83978842,701.36799154)
\curveto(76.87978214,701.46798892)(76.90978211,701.56798882)(76.92978842,701.66799154)
\curveto(76.94978207,701.76798862)(76.97478204,701.87298852)(77.00478842,701.98299154)
\curveto(77.014782,702.02298837)(77.019782,702.05798833)(77.01978842,702.08799154)
\curveto(77.019782,702.12798826)(77.02478199,702.16798822)(77.03478842,702.20799154)
\lineto(77.03478842,702.34299154)
\curveto(77.03478198,702.392988)(77.03978198,702.44298795)(77.04978842,702.49299154)
}
}
{
\newrgbcolor{curcolor}{0 0 0}
\pscustom[linestyle=none,fillstyle=solid,fillcolor=curcolor]
{
\newpath
\moveto(84.97471029,706.78299154)
\curveto(85.08470498,706.78298361)(85.17970488,706.77298362)(85.25971029,706.75299154)
\curveto(85.34970471,706.73298366)(85.41970464,706.6879837)(85.46971029,706.61799154)
\curveto(85.52970453,706.53798385)(85.5597045,706.39798399)(85.55971029,706.19799154)
\lineto(85.55971029,705.68799154)
\lineto(85.55971029,705.31299154)
\curveto(85.56970449,705.17298522)(85.55470451,705.06298533)(85.51471029,704.98299154)
\curveto(85.47470459,704.91298548)(85.41470465,704.86798552)(85.33471029,704.84799154)
\curveto(85.2647048,704.82798556)(85.17970488,704.81798557)(85.07971029,704.81799154)
\curveto(84.98970507,704.81798557)(84.88970517,704.82298557)(84.77971029,704.83299154)
\curveto(84.67970538,704.84298555)(84.58470548,704.83798555)(84.49471029,704.81799154)
\curveto(84.42470564,704.79798559)(84.35470571,704.78298561)(84.28471029,704.77299154)
\curveto(84.21470585,704.77298562)(84.14970591,704.76298563)(84.08971029,704.74299154)
\curveto(83.92970613,704.6929857)(83.76970629,704.61798577)(83.60971029,704.51799154)
\curveto(83.44970661,704.42798596)(83.32470674,704.32298607)(83.23471029,704.20299154)
\curveto(83.18470688,704.12298627)(83.12970693,704.03798635)(83.06971029,703.94799154)
\curveto(83.01970704,703.86798652)(82.96970709,703.78298661)(82.91971029,703.69299154)
\curveto(82.88970717,703.61298678)(82.8597072,703.52798686)(82.82971029,703.43799154)
\lineto(82.76971029,703.19799154)
\curveto(82.74970731,703.12798726)(82.73970732,703.05298734)(82.73971029,702.97299154)
\curveto(82.73970732,702.90298749)(82.72970733,702.83298756)(82.70971029,702.76299154)
\curveto(82.69970736,702.72298767)(82.69470737,702.68298771)(82.69471029,702.64299154)
\curveto(82.70470736,702.61298778)(82.70470736,702.58298781)(82.69471029,702.55299154)
\lineto(82.69471029,702.31299154)
\curveto(82.67470739,702.24298815)(82.66970739,702.16298823)(82.67971029,702.07299154)
\curveto(82.68970737,701.9929884)(82.69470737,701.91298848)(82.69471029,701.83299154)
\lineto(82.69471029,700.87299154)
\lineto(82.69471029,699.59799154)
\curveto(82.69470737,699.46799092)(82.68970737,699.34799104)(82.67971029,699.23799154)
\curveto(82.66970739,699.12799126)(82.63970742,699.03799135)(82.58971029,698.96799154)
\curveto(82.56970749,698.93799145)(82.53470753,698.91299148)(82.48471029,698.89299154)
\curveto(82.44470762,698.88299151)(82.39970766,698.87299152)(82.34971029,698.86299154)
\lineto(82.27471029,698.86299154)
\curveto(82.22470784,698.85299154)(82.16970789,698.84799154)(82.10971029,698.84799154)
\lineto(81.94471029,698.84799154)
\lineto(81.29971029,698.84799154)
\curveto(81.23970882,698.85799153)(81.17470889,698.86299153)(81.10471029,698.86299154)
\lineto(80.90971029,698.86299154)
\curveto(80.8597092,698.88299151)(80.80970925,698.89799149)(80.75971029,698.90799154)
\curveto(80.70970935,698.92799146)(80.67470939,698.96299143)(80.65471029,699.01299154)
\curveto(80.61470945,699.06299133)(80.58970947,699.13299126)(80.57971029,699.22299154)
\lineto(80.57971029,699.52299154)
\lineto(80.57971029,700.54299154)
\lineto(80.57971029,704.77299154)
\lineto(80.57971029,705.88299154)
\lineto(80.57971029,706.16799154)
\curveto(80.57970948,706.26798412)(80.59970946,706.34798404)(80.63971029,706.40799154)
\curveto(80.68970937,706.4879839)(80.7647093,706.53798385)(80.86471029,706.55799154)
\curveto(80.9647091,706.57798381)(81.08470898,706.5879838)(81.22471029,706.58799154)
\lineto(81.98971029,706.58799154)
\curveto(82.10970795,706.5879838)(82.21470785,706.57798381)(82.30471029,706.55799154)
\curveto(82.39470767,706.54798384)(82.4647076,706.50298389)(82.51471029,706.42299154)
\curveto(82.54470752,706.37298402)(82.5597075,706.30298409)(82.55971029,706.21299154)
\lineto(82.58971029,705.94299154)
\curveto(82.59970746,705.86298453)(82.61470745,705.7879846)(82.63471029,705.71799154)
\curveto(82.6647074,705.64798474)(82.71470735,705.61298478)(82.78471029,705.61299154)
\curveto(82.80470726,705.63298476)(82.82470724,705.64298475)(82.84471029,705.64299154)
\curveto(82.8647072,705.64298475)(82.88470718,705.65298474)(82.90471029,705.67299154)
\curveto(82.9647071,705.72298467)(83.01470705,705.77798461)(83.05471029,705.83799154)
\curveto(83.10470696,705.90798448)(83.1647069,705.96798442)(83.23471029,706.01799154)
\curveto(83.27470679,706.04798434)(83.30970675,706.07798431)(83.33971029,706.10799154)
\curveto(83.36970669,706.14798424)(83.40470666,706.18298421)(83.44471029,706.21299154)
\lineto(83.71471029,706.39299154)
\curveto(83.81470625,706.45298394)(83.91470615,706.50798388)(84.01471029,706.55799154)
\curveto(84.11470595,706.59798379)(84.21470585,706.63298376)(84.31471029,706.66299154)
\lineto(84.64471029,706.75299154)
\curveto(84.67470539,706.76298363)(84.72970533,706.76298363)(84.80971029,706.75299154)
\curveto(84.89970516,706.75298364)(84.95470511,706.76298363)(84.97471029,706.78299154)
}
}
{
\newrgbcolor{curcolor}{0 0 0}
\pscustom[linestyle=none,fillstyle=solid,fillcolor=curcolor]
{
\newpath
\moveto(93.48111654,702.79299154)
\curveto(93.50110838,702.71298768)(93.50110838,702.62298777)(93.48111654,702.52299154)
\curveto(93.46110842,702.42298797)(93.42610845,702.35798803)(93.37611654,702.32799154)
\curveto(93.32610855,702.2879881)(93.25110863,702.25798813)(93.15111654,702.23799154)
\curveto(93.06110882,702.22798816)(92.95610892,702.21798817)(92.83611654,702.20799154)
\lineto(92.49111654,702.20799154)
\curveto(92.3811095,702.21798817)(92.2811096,702.22298817)(92.19111654,702.22299154)
\lineto(88.53111654,702.22299154)
\lineto(88.32111654,702.22299154)
\curveto(88.26111362,702.22298817)(88.20611367,702.21298818)(88.15611654,702.19299154)
\curveto(88.0761138,702.15298824)(88.02611385,702.11298828)(88.00611654,702.07299154)
\curveto(87.98611389,702.05298834)(87.96611391,702.01298838)(87.94611654,701.95299154)
\curveto(87.92611395,701.90298849)(87.92111396,701.85298854)(87.93111654,701.80299154)
\curveto(87.95111393,701.74298865)(87.96111392,701.68298871)(87.96111654,701.62299154)
\curveto(87.97111391,701.57298882)(87.98611389,701.51798887)(88.00611654,701.45799154)
\curveto(88.08611379,701.21798917)(88.1811137,701.01798937)(88.29111654,700.85799154)
\curveto(88.41111347,700.70798968)(88.57111331,700.57298982)(88.77111654,700.45299154)
\curveto(88.85111303,700.40298999)(88.93111295,700.36799002)(89.01111654,700.34799154)
\curveto(89.10111278,700.33799005)(89.19111269,700.31799007)(89.28111654,700.28799154)
\curveto(89.36111252,700.26799012)(89.47111241,700.25299014)(89.61111654,700.24299154)
\curveto(89.75111213,700.23299016)(89.87111201,700.23799015)(89.97111654,700.25799154)
\lineto(90.10611654,700.25799154)
\curveto(90.20611167,700.27799011)(90.29611158,700.29799009)(90.37611654,700.31799154)
\curveto(90.46611141,700.34799004)(90.55111133,700.37799001)(90.63111654,700.40799154)
\curveto(90.73111115,700.45798993)(90.84111104,700.52298987)(90.96111654,700.60299154)
\curveto(91.09111079,700.68298971)(91.18611069,700.76298963)(91.24611654,700.84299154)
\curveto(91.29611058,700.91298948)(91.34611053,700.97798941)(91.39611654,701.03799154)
\curveto(91.45611042,701.10798928)(91.52611035,701.15798923)(91.60611654,701.18799154)
\curveto(91.70611017,701.23798915)(91.83111005,701.25798913)(91.98111654,701.24799154)
\lineto(92.41611654,701.24799154)
\lineto(92.59611654,701.24799154)
\curveto(92.66610921,701.25798913)(92.72610915,701.25298914)(92.77611654,701.23299154)
\lineto(92.92611654,701.23299154)
\curveto(93.02610885,701.21298918)(93.09610878,701.1879892)(93.13611654,701.15799154)
\curveto(93.1761087,701.13798925)(93.19610868,701.0929893)(93.19611654,701.02299154)
\curveto(93.20610867,700.95298944)(93.20110868,700.8929895)(93.18111654,700.84299154)
\curveto(93.13110875,700.70298969)(93.0761088,700.57798981)(93.01611654,700.46799154)
\curveto(92.95610892,700.35799003)(92.88610899,700.24799014)(92.80611654,700.13799154)
\curveto(92.58610929,699.80799058)(92.33610954,699.54299085)(92.05611654,699.34299154)
\curveto(91.7761101,699.14299125)(91.42611045,698.97299142)(91.00611654,698.83299154)
\curveto(90.89611098,698.7929916)(90.78611109,698.76799162)(90.67611654,698.75799154)
\curveto(90.56611131,698.74799164)(90.45111143,698.72799166)(90.33111654,698.69799154)
\curveto(90.29111159,698.6879917)(90.24611163,698.6879917)(90.19611654,698.69799154)
\curveto(90.15611172,698.69799169)(90.11611176,698.6929917)(90.07611654,698.68299154)
\lineto(89.91111654,698.68299154)
\curveto(89.86111202,698.66299173)(89.80111208,698.65799173)(89.73111654,698.66799154)
\curveto(89.67111221,698.66799172)(89.61611226,698.67299172)(89.56611654,698.68299154)
\curveto(89.48611239,698.6929917)(89.41611246,698.6929917)(89.35611654,698.68299154)
\curveto(89.29611258,698.67299172)(89.23111265,698.67799171)(89.16111654,698.69799154)
\curveto(89.11111277,698.71799167)(89.05611282,698.72799166)(88.99611654,698.72799154)
\curveto(88.93611294,698.72799166)(88.881113,698.73799165)(88.83111654,698.75799154)
\curveto(88.72111316,698.77799161)(88.61111327,698.80299159)(88.50111654,698.83299154)
\curveto(88.39111349,698.85299154)(88.29111359,698.8879915)(88.20111654,698.93799154)
\curveto(88.09111379,698.97799141)(87.98611389,699.01299138)(87.88611654,699.04299154)
\curveto(87.79611408,699.08299131)(87.71111417,699.12799126)(87.63111654,699.17799154)
\curveto(87.31111457,699.37799101)(87.02611485,699.60799078)(86.77611654,699.86799154)
\curveto(86.52611535,700.13799025)(86.32111556,700.44798994)(86.16111654,700.79799154)
\curveto(86.11111577,700.90798948)(86.07111581,701.01798937)(86.04111654,701.12799154)
\curveto(86.01111587,701.24798914)(85.97111591,701.36798902)(85.92111654,701.48799154)
\curveto(85.91111597,701.52798886)(85.90611597,701.56298883)(85.90611654,701.59299154)
\curveto(85.90611597,701.63298876)(85.90111598,701.67298872)(85.89111654,701.71299154)
\curveto(85.85111603,701.83298856)(85.82611605,701.96298843)(85.81611654,702.10299154)
\lineto(85.78611654,702.52299154)
\curveto(85.78611609,702.57298782)(85.7811161,702.62798776)(85.77111654,702.68799154)
\curveto(85.77111611,702.74798764)(85.7761161,702.80298759)(85.78611654,702.85299154)
\lineto(85.78611654,703.03299154)
\lineto(85.83111654,703.39299154)
\curveto(85.87111601,703.56298683)(85.90611597,703.72798666)(85.93611654,703.88799154)
\curveto(85.96611591,704.04798634)(86.01111587,704.19798619)(86.07111654,704.33799154)
\curveto(86.50111538,705.37798501)(87.23111465,706.11298428)(88.26111654,706.54299154)
\curveto(88.40111348,706.60298379)(88.54111334,706.64298375)(88.68111654,706.66299154)
\curveto(88.83111305,706.6929837)(88.98611289,706.72798366)(89.14611654,706.76799154)
\curveto(89.22611265,706.77798361)(89.30111258,706.78298361)(89.37111654,706.78299154)
\curveto(89.44111244,706.78298361)(89.51611236,706.7879836)(89.59611654,706.79799154)
\curveto(90.10611177,706.80798358)(90.54111134,706.74798364)(90.90111654,706.61799154)
\curveto(91.27111061,706.49798389)(91.60111028,706.33798405)(91.89111654,706.13799154)
\curveto(91.9811099,706.07798431)(92.07110981,706.00798438)(92.16111654,705.92799154)
\curveto(92.25110963,705.85798453)(92.33110955,705.78298461)(92.40111654,705.70299154)
\curveto(92.43110945,705.65298474)(92.47110941,705.61298478)(92.52111654,705.58299154)
\curveto(92.60110928,705.47298492)(92.6761092,705.35798503)(92.74611654,705.23799154)
\curveto(92.81610906,705.12798526)(92.89110899,705.01298538)(92.97111654,704.89299154)
\curveto(93.02110886,704.80298559)(93.06110882,704.70798568)(93.09111654,704.60799154)
\curveto(93.13110875,704.51798587)(93.17110871,704.41798597)(93.21111654,704.30799154)
\curveto(93.26110862,704.17798621)(93.30110858,704.04298635)(93.33111654,703.90299154)
\curveto(93.36110852,703.76298663)(93.39610848,703.62298677)(93.43611654,703.48299154)
\curveto(93.45610842,703.40298699)(93.46110842,703.31298708)(93.45111654,703.21299154)
\curveto(93.45110843,703.12298727)(93.46110842,703.03798735)(93.48111654,702.95799154)
\lineto(93.48111654,702.79299154)
\moveto(91.23111654,703.67799154)
\curveto(91.30111058,703.77798661)(91.30611057,703.89798649)(91.24611654,704.03799154)
\curveto(91.19611068,704.1879862)(91.15611072,704.29798609)(91.12611654,704.36799154)
\curveto(90.98611089,704.63798575)(90.80111108,704.84298555)(90.57111654,704.98299154)
\curveto(90.34111154,705.13298526)(90.02111186,705.21298518)(89.61111654,705.22299154)
\curveto(89.5811123,705.20298519)(89.54611233,705.19798519)(89.50611654,705.20799154)
\curveto(89.46611241,705.21798517)(89.43111245,705.21798517)(89.40111654,705.20799154)
\curveto(89.35111253,705.1879852)(89.29611258,705.17298522)(89.23611654,705.16299154)
\curveto(89.1761127,705.16298523)(89.12111276,705.15298524)(89.07111654,705.13299154)
\curveto(88.63111325,704.9929854)(88.30611357,704.71798567)(88.09611654,704.30799154)
\curveto(88.0761138,704.26798612)(88.05111383,704.21298618)(88.02111654,704.14299154)
\curveto(88.00111388,704.08298631)(87.98611389,704.01798637)(87.97611654,703.94799154)
\curveto(87.96611391,703.8879865)(87.96611391,703.82798656)(87.97611654,703.76799154)
\curveto(87.99611388,703.70798668)(88.03111385,703.65798673)(88.08111654,703.61799154)
\curveto(88.16111372,703.56798682)(88.27111361,703.54298685)(88.41111654,703.54299154)
\lineto(88.81611654,703.54299154)
\lineto(90.48111654,703.54299154)
\lineto(90.91611654,703.54299154)
\curveto(91.0761108,703.55298684)(91.1811107,703.59798679)(91.23111654,703.67799154)
}
}
{
\newrgbcolor{curcolor}{0 0 0}
\pscustom[linestyle=none,fillstyle=solid,fillcolor=curcolor]
{
\newpath
\moveto(97.69939779,706.79799154)
\curveto(98.44939329,706.81798357)(99.09939264,706.73298366)(99.64939779,706.54299154)
\curveto(100.20939153,706.36298403)(100.63439111,706.04798434)(100.92439779,705.59799154)
\curveto(100.99439075,705.4879849)(101.05439069,705.37298502)(101.10439779,705.25299154)
\curveto(101.16439058,705.14298525)(101.21439053,705.01798537)(101.25439779,704.87799154)
\curveto(101.27439047,704.81798557)(101.28439046,704.75298564)(101.28439779,704.68299154)
\curveto(101.28439046,704.61298578)(101.27439047,704.55298584)(101.25439779,704.50299154)
\curveto(101.21439053,704.44298595)(101.15939058,704.40298599)(101.08939779,704.38299154)
\curveto(101.0393907,704.36298603)(100.97939076,704.35298604)(100.90939779,704.35299154)
\lineto(100.69939779,704.35299154)
\lineto(100.03939779,704.35299154)
\curveto(99.96939177,704.35298604)(99.89939184,704.34798604)(99.82939779,704.33799154)
\curveto(99.75939198,704.33798605)(99.69439205,704.34798604)(99.63439779,704.36799154)
\curveto(99.53439221,704.387986)(99.45939228,704.42798596)(99.40939779,704.48799154)
\curveto(99.35939238,704.54798584)(99.31439243,704.60798578)(99.27439779,704.66799154)
\lineto(99.15439779,704.87799154)
\curveto(99.12439262,704.95798543)(99.07439267,705.02298537)(99.00439779,705.07299154)
\curveto(98.90439284,705.15298524)(98.80439294,705.21298518)(98.70439779,705.25299154)
\curveto(98.61439313,705.2929851)(98.49939324,705.32798506)(98.35939779,705.35799154)
\curveto(98.28939345,705.37798501)(98.18439356,705.392985)(98.04439779,705.40299154)
\curveto(97.91439383,705.41298498)(97.81439393,705.40798498)(97.74439779,705.38799154)
\lineto(97.63939779,705.38799154)
\lineto(97.48939779,705.35799154)
\curveto(97.44939429,705.35798503)(97.40439434,705.35298504)(97.35439779,705.34299154)
\curveto(97.18439456,705.2929851)(97.0443947,705.22298517)(96.93439779,705.13299154)
\curveto(96.83439491,705.05298534)(96.76439498,704.92798546)(96.72439779,704.75799154)
\curveto(96.70439504,704.6879857)(96.70439504,704.62298577)(96.72439779,704.56299154)
\curveto(96.744395,704.50298589)(96.76439498,704.45298594)(96.78439779,704.41299154)
\curveto(96.85439489,704.2929861)(96.93439481,704.19798619)(97.02439779,704.12799154)
\curveto(97.12439462,704.05798633)(97.2393945,703.99798639)(97.36939779,703.94799154)
\curveto(97.55939418,703.86798652)(97.76439398,703.79798659)(97.98439779,703.73799154)
\lineto(98.67439779,703.58799154)
\curveto(98.91439283,703.54798684)(99.1443926,703.49798689)(99.36439779,703.43799154)
\curveto(99.59439215,703.387987)(99.80939193,703.32298707)(100.00939779,703.24299154)
\curveto(100.09939164,703.20298719)(100.18439156,703.16798722)(100.26439779,703.13799154)
\curveto(100.35439139,703.11798727)(100.4393913,703.08298731)(100.51939779,703.03299154)
\curveto(100.70939103,702.91298748)(100.87939086,702.78298761)(101.02939779,702.64299154)
\curveto(101.18939055,702.50298789)(101.31439043,702.32798806)(101.40439779,702.11799154)
\curveto(101.43439031,702.04798834)(101.45939028,701.97798841)(101.47939779,701.90799154)
\curveto(101.49939024,701.83798855)(101.51939022,701.76298863)(101.53939779,701.68299154)
\curveto(101.54939019,701.62298877)(101.55439019,701.52798886)(101.55439779,701.39799154)
\curveto(101.56439018,701.27798911)(101.56439018,701.18298921)(101.55439779,701.11299154)
\lineto(101.55439779,701.03799154)
\curveto(101.53439021,700.97798941)(101.51939022,700.91798947)(101.50939779,700.85799154)
\curveto(101.50939023,700.80798958)(101.50439024,700.75798963)(101.49439779,700.70799154)
\curveto(101.42439032,700.40798998)(101.31439043,700.14299025)(101.16439779,699.91299154)
\curveto(101.00439074,699.67299072)(100.80939093,699.47799091)(100.57939779,699.32799154)
\curveto(100.34939139,699.17799121)(100.08939165,699.04799134)(99.79939779,698.93799154)
\curveto(99.68939205,698.8879915)(99.56939217,698.85299154)(99.43939779,698.83299154)
\curveto(99.31939242,698.81299158)(99.19939254,698.7879916)(99.07939779,698.75799154)
\curveto(98.98939275,698.73799165)(98.89439285,698.72799166)(98.79439779,698.72799154)
\curveto(98.70439304,698.71799167)(98.61439313,698.70299169)(98.52439779,698.68299154)
\lineto(98.25439779,698.68299154)
\curveto(98.19439355,698.66299173)(98.08939365,698.65299174)(97.93939779,698.65299154)
\curveto(97.79939394,698.65299174)(97.69939404,698.66299173)(97.63939779,698.68299154)
\curveto(97.60939413,698.68299171)(97.57439417,698.6879917)(97.53439779,698.69799154)
\lineto(97.42939779,698.69799154)
\curveto(97.30939443,698.71799167)(97.18939455,698.73299166)(97.06939779,698.74299154)
\curveto(96.94939479,698.75299164)(96.83439491,698.77299162)(96.72439779,698.80299154)
\curveto(96.33439541,698.91299148)(95.98939575,699.03799135)(95.68939779,699.17799154)
\curveto(95.38939635,699.32799106)(95.13439661,699.54799084)(94.92439779,699.83799154)
\curveto(94.78439696,700.02799036)(94.66439708,700.24799014)(94.56439779,700.49799154)
\curveto(94.5443972,700.55798983)(94.52439722,700.63798975)(94.50439779,700.73799154)
\curveto(94.48439726,700.7879896)(94.46939727,700.85798953)(94.45939779,700.94799154)
\curveto(94.44939729,701.03798935)(94.45439729,701.11298928)(94.47439779,701.17299154)
\curveto(94.50439724,701.24298915)(94.55439719,701.2929891)(94.62439779,701.32299154)
\curveto(94.67439707,701.34298905)(94.73439701,701.35298904)(94.80439779,701.35299154)
\lineto(95.02939779,701.35299154)
\lineto(95.73439779,701.35299154)
\lineto(95.97439779,701.35299154)
\curveto(96.05439569,701.35298904)(96.12439562,701.34298905)(96.18439779,701.32299154)
\curveto(96.29439545,701.28298911)(96.36439538,701.21798917)(96.39439779,701.12799154)
\curveto(96.43439531,701.03798935)(96.47939526,700.94298945)(96.52939779,700.84299154)
\curveto(96.54939519,700.7929896)(96.58439516,700.72798966)(96.63439779,700.64799154)
\curveto(96.69439505,700.56798982)(96.744395,700.51798987)(96.78439779,700.49799154)
\curveto(96.90439484,700.39798999)(97.01939472,700.31799007)(97.12939779,700.25799154)
\curveto(97.2393945,700.20799018)(97.37939436,700.15799023)(97.54939779,700.10799154)
\curveto(97.59939414,700.0879903)(97.64939409,700.07799031)(97.69939779,700.07799154)
\curveto(97.74939399,700.0879903)(97.79939394,700.0879903)(97.84939779,700.07799154)
\curveto(97.92939381,700.05799033)(98.01439373,700.04799034)(98.10439779,700.04799154)
\curveto(98.20439354,700.05799033)(98.28939345,700.07299032)(98.35939779,700.09299154)
\curveto(98.40939333,700.10299029)(98.45439329,700.10799028)(98.49439779,700.10799154)
\curveto(98.5443932,700.10799028)(98.59439315,700.11799027)(98.64439779,700.13799154)
\curveto(98.78439296,700.1879902)(98.90939283,700.24799014)(99.01939779,700.31799154)
\curveto(99.1393926,700.38799)(99.23439251,700.47798991)(99.30439779,700.58799154)
\curveto(99.35439239,700.66798972)(99.39439235,700.7929896)(99.42439779,700.96299154)
\curveto(99.4443923,701.03298936)(99.4443923,701.09798929)(99.42439779,701.15799154)
\curveto(99.40439234,701.21798917)(99.38439236,701.26798912)(99.36439779,701.30799154)
\curveto(99.29439245,701.44798894)(99.20439254,701.55298884)(99.09439779,701.62299154)
\curveto(98.99439275,701.6929887)(98.87439287,701.75798863)(98.73439779,701.81799154)
\curveto(98.5443932,701.89798849)(98.3443934,701.96298843)(98.13439779,702.01299154)
\curveto(97.92439382,702.06298833)(97.71439403,702.11798827)(97.50439779,702.17799154)
\curveto(97.42439432,702.19798819)(97.3393944,702.21298818)(97.24939779,702.22299154)
\curveto(97.16939457,702.23298816)(97.08939465,702.24798814)(97.00939779,702.26799154)
\curveto(96.68939505,702.35798803)(96.38439536,702.44298795)(96.09439779,702.52299154)
\curveto(95.80439594,702.61298778)(95.5393962,702.74298765)(95.29939779,702.91299154)
\curveto(95.01939672,703.11298728)(94.81439693,703.38298701)(94.68439779,703.72299154)
\curveto(94.66439708,703.7929866)(94.6443971,703.8879865)(94.62439779,704.00799154)
\curveto(94.60439714,704.07798631)(94.58939715,704.16298623)(94.57939779,704.26299154)
\curveto(94.56939717,704.36298603)(94.57439717,704.45298594)(94.59439779,704.53299154)
\curveto(94.61439713,704.58298581)(94.61939712,704.62298577)(94.60939779,704.65299154)
\curveto(94.59939714,704.6929857)(94.60439714,704.73798565)(94.62439779,704.78799154)
\curveto(94.6443971,704.89798549)(94.66439708,704.99798539)(94.68439779,705.08799154)
\curveto(94.71439703,705.1879852)(94.74939699,705.28298511)(94.78939779,705.37299154)
\curveto(94.91939682,705.66298473)(95.09939664,705.89798449)(95.32939779,706.07799154)
\curveto(95.55939618,706.25798413)(95.81939592,706.40298399)(96.10939779,706.51299154)
\curveto(96.21939552,706.56298383)(96.33439541,706.59798379)(96.45439779,706.61799154)
\curveto(96.57439517,706.64798374)(96.69939504,706.67798371)(96.82939779,706.70799154)
\curveto(96.88939485,706.72798366)(96.94939479,706.73798365)(97.00939779,706.73799154)
\lineto(97.18939779,706.76799154)
\curveto(97.26939447,706.77798361)(97.35439439,706.78298361)(97.44439779,706.78299154)
\curveto(97.53439421,706.78298361)(97.61939412,706.7879836)(97.69939779,706.79799154)
}
}
{
\newrgbcolor{curcolor}{0 0 0}
\pscustom[linestyle=none,fillstyle=solid,fillcolor=curcolor]
{
}
}
{
\newrgbcolor{curcolor}{0 0 0}
\pscustom[linestyle=none,fillstyle=solid,fillcolor=curcolor]
{
\newpath
\moveto(114.53619467,699.70299154)
\lineto(114.53619467,699.28299154)
\curveto(114.5361863,699.15299124)(114.50618633,699.04799134)(114.44619467,698.96799154)
\curveto(114.39618644,698.91799147)(114.3311865,698.88299151)(114.25119467,698.86299154)
\curveto(114.17118666,698.85299154)(114.08118675,698.84799154)(113.98119467,698.84799154)
\lineto(113.15619467,698.84799154)
\lineto(112.87119467,698.84799154)
\curveto(112.79118804,698.85799153)(112.72618811,698.88299151)(112.67619467,698.92299154)
\curveto(112.60618823,698.97299142)(112.56618827,699.03799135)(112.55619467,699.11799154)
\curveto(112.54618829,699.19799119)(112.52618831,699.27799111)(112.49619467,699.35799154)
\curveto(112.47618836,699.37799101)(112.45618838,699.392991)(112.43619467,699.40299154)
\curveto(112.42618841,699.42299097)(112.41118842,699.44299095)(112.39119467,699.46299154)
\curveto(112.28118855,699.46299093)(112.20118863,699.43799095)(112.15119467,699.38799154)
\lineto(112.00119467,699.23799154)
\curveto(111.9311889,699.1879912)(111.86618897,699.14299125)(111.80619467,699.10299154)
\curveto(111.74618909,699.07299132)(111.68118915,699.03299136)(111.61119467,698.98299154)
\curveto(111.57118926,698.96299143)(111.52618931,698.94299145)(111.47619467,698.92299154)
\curveto(111.4361894,698.90299149)(111.39118944,698.88299151)(111.34119467,698.86299154)
\curveto(111.20118963,698.81299158)(111.05118978,698.76799162)(110.89119467,698.72799154)
\curveto(110.84118999,698.70799168)(110.79619004,698.69799169)(110.75619467,698.69799154)
\curveto(110.71619012,698.69799169)(110.67619016,698.6929917)(110.63619467,698.68299154)
\lineto(110.50119467,698.68299154)
\curveto(110.47119036,698.67299172)(110.4311904,698.66799172)(110.38119467,698.66799154)
\lineto(110.24619467,698.66799154)
\curveto(110.18619065,698.64799174)(110.09619074,698.64299175)(109.97619467,698.65299154)
\curveto(109.85619098,698.65299174)(109.77119106,698.66299173)(109.72119467,698.68299154)
\curveto(109.65119118,698.70299169)(109.58619125,698.71299168)(109.52619467,698.71299154)
\curveto(109.47619136,698.70299169)(109.42119141,698.70799168)(109.36119467,698.72799154)
\lineto(109.00119467,698.84799154)
\curveto(108.89119194,698.87799151)(108.78119205,698.91799147)(108.67119467,698.96799154)
\curveto(108.32119251,699.11799127)(108.00619283,699.34799104)(107.72619467,699.65799154)
\curveto(107.45619338,699.97799041)(107.24119359,700.31299008)(107.08119467,700.66299154)
\curveto(107.0311938,700.77298962)(106.99119384,700.87798951)(106.96119467,700.97799154)
\curveto(106.9311939,701.0879893)(106.89619394,701.19798919)(106.85619467,701.30799154)
\curveto(106.84619399,701.34798904)(106.84119399,701.38298901)(106.84119467,701.41299154)
\curveto(106.84119399,701.45298894)(106.831194,701.49798889)(106.81119467,701.54799154)
\curveto(106.79119404,701.62798876)(106.77119406,701.71298868)(106.75119467,701.80299154)
\curveto(106.74119409,701.90298849)(106.72619411,702.00298839)(106.70619467,702.10299154)
\curveto(106.69619414,702.13298826)(106.69119414,702.16798822)(106.69119467,702.20799154)
\curveto(106.70119413,702.24798814)(106.70119413,702.28298811)(106.69119467,702.31299154)
\lineto(106.69119467,702.44799154)
\curveto(106.69119414,702.49798789)(106.68619415,702.54798784)(106.67619467,702.59799154)
\curveto(106.66619417,702.64798774)(106.66119417,702.70298769)(106.66119467,702.76299154)
\curveto(106.66119417,702.83298756)(106.66619417,702.8879875)(106.67619467,702.92799154)
\curveto(106.68619415,702.97798741)(106.69119414,703.02298737)(106.69119467,703.06299154)
\lineto(106.69119467,703.21299154)
\curveto(106.70119413,703.26298713)(106.70119413,703.30798708)(106.69119467,703.34799154)
\curveto(106.69119414,703.39798699)(106.70119413,703.44798694)(106.72119467,703.49799154)
\curveto(106.74119409,703.60798678)(106.75619408,703.71298668)(106.76619467,703.81299154)
\curveto(106.78619405,703.91298648)(106.81119402,704.01298638)(106.84119467,704.11299154)
\curveto(106.88119395,704.23298616)(106.91619392,704.34798604)(106.94619467,704.45799154)
\curveto(106.97619386,704.56798582)(107.01619382,704.67798571)(107.06619467,704.78799154)
\curveto(107.20619363,705.0879853)(107.38119345,705.37298502)(107.59119467,705.64299154)
\curveto(107.61119322,705.67298472)(107.6361932,705.69798469)(107.66619467,705.71799154)
\curveto(107.70619313,705.74798464)(107.7361931,705.77798461)(107.75619467,705.80799154)
\curveto(107.79619304,705.85798453)(107.836193,705.90298449)(107.87619467,705.94299154)
\curveto(107.91619292,705.98298441)(107.96119287,706.02298437)(108.01119467,706.06299154)
\curveto(108.05119278,706.08298431)(108.08619275,706.10798428)(108.11619467,706.13799154)
\curveto(108.14619269,706.17798421)(108.18119265,706.20798418)(108.22119467,706.22799154)
\curveto(108.47119236,706.39798399)(108.76119207,706.53798385)(109.09119467,706.64799154)
\curveto(109.16119167,706.66798372)(109.2311916,706.68298371)(109.30119467,706.69299154)
\curveto(109.38119145,706.70298369)(109.46119137,706.71798367)(109.54119467,706.73799154)
\curveto(109.61119122,706.75798363)(109.70119113,706.76798362)(109.81119467,706.76799154)
\curveto(109.92119091,706.77798361)(110.0311908,706.78298361)(110.14119467,706.78299154)
\curveto(110.25119058,706.78298361)(110.35619048,706.77798361)(110.45619467,706.76799154)
\curveto(110.56619027,706.75798363)(110.65619018,706.74298365)(110.72619467,706.72299154)
\curveto(110.87618996,706.67298372)(111.02118981,706.62798376)(111.16119467,706.58799154)
\curveto(111.30118953,706.54798384)(111.4311894,706.4929839)(111.55119467,706.42299154)
\curveto(111.62118921,706.37298402)(111.68618915,706.32298407)(111.74619467,706.27299154)
\curveto(111.80618903,706.23298416)(111.87118896,706.1879842)(111.94119467,706.13799154)
\curveto(111.98118885,706.10798428)(112.0361888,706.06798432)(112.10619467,706.01799154)
\curveto(112.18618865,705.96798442)(112.26118857,705.96798442)(112.33119467,706.01799154)
\curveto(112.37118846,706.03798435)(112.39118844,706.07298432)(112.39119467,706.12299154)
\curveto(112.39118844,706.17298422)(112.40118843,706.22298417)(112.42119467,706.27299154)
\lineto(112.42119467,706.42299154)
\curveto(112.4311884,706.45298394)(112.4361884,706.4879839)(112.43619467,706.52799154)
\lineto(112.43619467,706.64799154)
\lineto(112.43619467,708.68799154)
\curveto(112.4361884,708.79798159)(112.4311884,708.91798147)(112.42119467,709.04799154)
\curveto(112.42118841,709.1879812)(112.44618839,709.2929811)(112.49619467,709.36299154)
\curveto(112.5361883,709.44298095)(112.61118822,709.4929809)(112.72119467,709.51299154)
\curveto(112.74118809,709.52298087)(112.76118807,709.52298087)(112.78119467,709.51299154)
\curveto(112.80118803,709.51298088)(112.82118801,709.51798087)(112.84119467,709.52799154)
\lineto(113.90619467,709.52799154)
\curveto(114.02618681,709.52798086)(114.1361867,709.52298087)(114.23619467,709.51299154)
\curveto(114.3361865,709.50298089)(114.41118642,709.46298093)(114.46119467,709.39299154)
\curveto(114.51118632,709.31298108)(114.5361863,709.20798118)(114.53619467,709.07799154)
\lineto(114.53619467,708.71799154)
\lineto(114.53619467,699.70299154)
\moveto(112.49619467,702.64299154)
\curveto(112.50618833,702.68298771)(112.50618833,702.72298767)(112.49619467,702.76299154)
\lineto(112.49619467,702.89799154)
\curveto(112.49618834,702.99798739)(112.49118834,703.09798729)(112.48119467,703.19799154)
\curveto(112.47118836,703.29798709)(112.45618838,703.387987)(112.43619467,703.46799154)
\curveto(112.41618842,703.57798681)(112.39618844,703.67798671)(112.37619467,703.76799154)
\curveto(112.36618847,703.85798653)(112.34118849,703.94298645)(112.30119467,704.02299154)
\curveto(112.16118867,704.38298601)(111.95618888,704.66798572)(111.68619467,704.87799154)
\curveto(111.42618941,705.0879853)(111.04618979,705.1929852)(110.54619467,705.19299154)
\curveto(110.48619035,705.1929852)(110.40619043,705.18298521)(110.30619467,705.16299154)
\curveto(110.22619061,705.14298525)(110.15119068,705.12298527)(110.08119467,705.10299154)
\curveto(110.02119081,705.0929853)(109.96119087,705.07298532)(109.90119467,705.04299154)
\curveto(109.6311912,704.93298546)(109.42119141,704.76298563)(109.27119467,704.53299154)
\curveto(109.12119171,704.30298609)(109.00119183,704.04298635)(108.91119467,703.75299154)
\curveto(108.88119195,703.65298674)(108.86119197,703.55298684)(108.85119467,703.45299154)
\curveto(108.84119199,703.35298704)(108.82119201,703.24798714)(108.79119467,703.13799154)
\lineto(108.79119467,702.92799154)
\curveto(108.77119206,702.83798755)(108.76619207,702.71298768)(108.77619467,702.55299154)
\curveto(108.78619205,702.40298799)(108.80119203,702.2929881)(108.82119467,702.22299154)
\lineto(108.82119467,702.13299154)
\curveto(108.831192,702.11298828)(108.836192,702.0929883)(108.83619467,702.07299154)
\curveto(108.85619198,701.9929884)(108.87119196,701.91798847)(108.88119467,701.84799154)
\curveto(108.90119193,701.77798861)(108.92119191,701.70298869)(108.94119467,701.62299154)
\curveto(109.11119172,701.10298929)(109.40119143,700.71798967)(109.81119467,700.46799154)
\curveto(109.94119089,700.37799001)(110.12119071,700.30799008)(110.35119467,700.25799154)
\curveto(110.39119044,700.24799014)(110.45119038,700.24299015)(110.53119467,700.24299154)
\curveto(110.56119027,700.23299016)(110.60619023,700.22299017)(110.66619467,700.21299154)
\curveto(110.7361901,700.21299018)(110.79119004,700.21799017)(110.83119467,700.22799154)
\curveto(110.91118992,700.24799014)(110.99118984,700.26299013)(111.07119467,700.27299154)
\curveto(111.15118968,700.28299011)(111.2311896,700.30299009)(111.31119467,700.33299154)
\curveto(111.56118927,700.44298995)(111.76118907,700.58298981)(111.91119467,700.75299154)
\curveto(112.06118877,700.92298947)(112.19118864,701.13798925)(112.30119467,701.39799154)
\curveto(112.34118849,701.4879889)(112.37118846,701.57798881)(112.39119467,701.66799154)
\curveto(112.41118842,701.76798862)(112.4311884,701.87298852)(112.45119467,701.98299154)
\curveto(112.46118837,702.03298836)(112.46118837,702.07798831)(112.45119467,702.11799154)
\curveto(112.45118838,702.16798822)(112.46118837,702.21798817)(112.48119467,702.26799154)
\curveto(112.49118834,702.29798809)(112.49618834,702.33298806)(112.49619467,702.37299154)
\lineto(112.49619467,702.50799154)
\lineto(112.49619467,702.64299154)
}
}
{
\newrgbcolor{curcolor}{0 0 0}
\pscustom[linestyle=none,fillstyle=solid,fillcolor=curcolor]
{
\newpath
\moveto(123.48111654,702.79299154)
\curveto(123.50110838,702.71298768)(123.50110838,702.62298777)(123.48111654,702.52299154)
\curveto(123.46110842,702.42298797)(123.42610845,702.35798803)(123.37611654,702.32799154)
\curveto(123.32610855,702.2879881)(123.25110863,702.25798813)(123.15111654,702.23799154)
\curveto(123.06110882,702.22798816)(122.95610892,702.21798817)(122.83611654,702.20799154)
\lineto(122.49111654,702.20799154)
\curveto(122.3811095,702.21798817)(122.2811096,702.22298817)(122.19111654,702.22299154)
\lineto(118.53111654,702.22299154)
\lineto(118.32111654,702.22299154)
\curveto(118.26111362,702.22298817)(118.20611367,702.21298818)(118.15611654,702.19299154)
\curveto(118.0761138,702.15298824)(118.02611385,702.11298828)(118.00611654,702.07299154)
\curveto(117.98611389,702.05298834)(117.96611391,702.01298838)(117.94611654,701.95299154)
\curveto(117.92611395,701.90298849)(117.92111396,701.85298854)(117.93111654,701.80299154)
\curveto(117.95111393,701.74298865)(117.96111392,701.68298871)(117.96111654,701.62299154)
\curveto(117.97111391,701.57298882)(117.98611389,701.51798887)(118.00611654,701.45799154)
\curveto(118.08611379,701.21798917)(118.1811137,701.01798937)(118.29111654,700.85799154)
\curveto(118.41111347,700.70798968)(118.57111331,700.57298982)(118.77111654,700.45299154)
\curveto(118.85111303,700.40298999)(118.93111295,700.36799002)(119.01111654,700.34799154)
\curveto(119.10111278,700.33799005)(119.19111269,700.31799007)(119.28111654,700.28799154)
\curveto(119.36111252,700.26799012)(119.47111241,700.25299014)(119.61111654,700.24299154)
\curveto(119.75111213,700.23299016)(119.87111201,700.23799015)(119.97111654,700.25799154)
\lineto(120.10611654,700.25799154)
\curveto(120.20611167,700.27799011)(120.29611158,700.29799009)(120.37611654,700.31799154)
\curveto(120.46611141,700.34799004)(120.55111133,700.37799001)(120.63111654,700.40799154)
\curveto(120.73111115,700.45798993)(120.84111104,700.52298987)(120.96111654,700.60299154)
\curveto(121.09111079,700.68298971)(121.18611069,700.76298963)(121.24611654,700.84299154)
\curveto(121.29611058,700.91298948)(121.34611053,700.97798941)(121.39611654,701.03799154)
\curveto(121.45611042,701.10798928)(121.52611035,701.15798923)(121.60611654,701.18799154)
\curveto(121.70611017,701.23798915)(121.83111005,701.25798913)(121.98111654,701.24799154)
\lineto(122.41611654,701.24799154)
\lineto(122.59611654,701.24799154)
\curveto(122.66610921,701.25798913)(122.72610915,701.25298914)(122.77611654,701.23299154)
\lineto(122.92611654,701.23299154)
\curveto(123.02610885,701.21298918)(123.09610878,701.1879892)(123.13611654,701.15799154)
\curveto(123.1761087,701.13798925)(123.19610868,701.0929893)(123.19611654,701.02299154)
\curveto(123.20610867,700.95298944)(123.20110868,700.8929895)(123.18111654,700.84299154)
\curveto(123.13110875,700.70298969)(123.0761088,700.57798981)(123.01611654,700.46799154)
\curveto(122.95610892,700.35799003)(122.88610899,700.24799014)(122.80611654,700.13799154)
\curveto(122.58610929,699.80799058)(122.33610954,699.54299085)(122.05611654,699.34299154)
\curveto(121.7761101,699.14299125)(121.42611045,698.97299142)(121.00611654,698.83299154)
\curveto(120.89611098,698.7929916)(120.78611109,698.76799162)(120.67611654,698.75799154)
\curveto(120.56611131,698.74799164)(120.45111143,698.72799166)(120.33111654,698.69799154)
\curveto(120.29111159,698.6879917)(120.24611163,698.6879917)(120.19611654,698.69799154)
\curveto(120.15611172,698.69799169)(120.11611176,698.6929917)(120.07611654,698.68299154)
\lineto(119.91111654,698.68299154)
\curveto(119.86111202,698.66299173)(119.80111208,698.65799173)(119.73111654,698.66799154)
\curveto(119.67111221,698.66799172)(119.61611226,698.67299172)(119.56611654,698.68299154)
\curveto(119.48611239,698.6929917)(119.41611246,698.6929917)(119.35611654,698.68299154)
\curveto(119.29611258,698.67299172)(119.23111265,698.67799171)(119.16111654,698.69799154)
\curveto(119.11111277,698.71799167)(119.05611282,698.72799166)(118.99611654,698.72799154)
\curveto(118.93611294,698.72799166)(118.881113,698.73799165)(118.83111654,698.75799154)
\curveto(118.72111316,698.77799161)(118.61111327,698.80299159)(118.50111654,698.83299154)
\curveto(118.39111349,698.85299154)(118.29111359,698.8879915)(118.20111654,698.93799154)
\curveto(118.09111379,698.97799141)(117.98611389,699.01299138)(117.88611654,699.04299154)
\curveto(117.79611408,699.08299131)(117.71111417,699.12799126)(117.63111654,699.17799154)
\curveto(117.31111457,699.37799101)(117.02611485,699.60799078)(116.77611654,699.86799154)
\curveto(116.52611535,700.13799025)(116.32111556,700.44798994)(116.16111654,700.79799154)
\curveto(116.11111577,700.90798948)(116.07111581,701.01798937)(116.04111654,701.12799154)
\curveto(116.01111587,701.24798914)(115.97111591,701.36798902)(115.92111654,701.48799154)
\curveto(115.91111597,701.52798886)(115.90611597,701.56298883)(115.90611654,701.59299154)
\curveto(115.90611597,701.63298876)(115.90111598,701.67298872)(115.89111654,701.71299154)
\curveto(115.85111603,701.83298856)(115.82611605,701.96298843)(115.81611654,702.10299154)
\lineto(115.78611654,702.52299154)
\curveto(115.78611609,702.57298782)(115.7811161,702.62798776)(115.77111654,702.68799154)
\curveto(115.77111611,702.74798764)(115.7761161,702.80298759)(115.78611654,702.85299154)
\lineto(115.78611654,703.03299154)
\lineto(115.83111654,703.39299154)
\curveto(115.87111601,703.56298683)(115.90611597,703.72798666)(115.93611654,703.88799154)
\curveto(115.96611591,704.04798634)(116.01111587,704.19798619)(116.07111654,704.33799154)
\curveto(116.50111538,705.37798501)(117.23111465,706.11298428)(118.26111654,706.54299154)
\curveto(118.40111348,706.60298379)(118.54111334,706.64298375)(118.68111654,706.66299154)
\curveto(118.83111305,706.6929837)(118.98611289,706.72798366)(119.14611654,706.76799154)
\curveto(119.22611265,706.77798361)(119.30111258,706.78298361)(119.37111654,706.78299154)
\curveto(119.44111244,706.78298361)(119.51611236,706.7879836)(119.59611654,706.79799154)
\curveto(120.10611177,706.80798358)(120.54111134,706.74798364)(120.90111654,706.61799154)
\curveto(121.27111061,706.49798389)(121.60111028,706.33798405)(121.89111654,706.13799154)
\curveto(121.9811099,706.07798431)(122.07110981,706.00798438)(122.16111654,705.92799154)
\curveto(122.25110963,705.85798453)(122.33110955,705.78298461)(122.40111654,705.70299154)
\curveto(122.43110945,705.65298474)(122.47110941,705.61298478)(122.52111654,705.58299154)
\curveto(122.60110928,705.47298492)(122.6761092,705.35798503)(122.74611654,705.23799154)
\curveto(122.81610906,705.12798526)(122.89110899,705.01298538)(122.97111654,704.89299154)
\curveto(123.02110886,704.80298559)(123.06110882,704.70798568)(123.09111654,704.60799154)
\curveto(123.13110875,704.51798587)(123.17110871,704.41798597)(123.21111654,704.30799154)
\curveto(123.26110862,704.17798621)(123.30110858,704.04298635)(123.33111654,703.90299154)
\curveto(123.36110852,703.76298663)(123.39610848,703.62298677)(123.43611654,703.48299154)
\curveto(123.45610842,703.40298699)(123.46110842,703.31298708)(123.45111654,703.21299154)
\curveto(123.45110843,703.12298727)(123.46110842,703.03798735)(123.48111654,702.95799154)
\lineto(123.48111654,702.79299154)
\moveto(121.23111654,703.67799154)
\curveto(121.30111058,703.77798661)(121.30611057,703.89798649)(121.24611654,704.03799154)
\curveto(121.19611068,704.1879862)(121.15611072,704.29798609)(121.12611654,704.36799154)
\curveto(120.98611089,704.63798575)(120.80111108,704.84298555)(120.57111654,704.98299154)
\curveto(120.34111154,705.13298526)(120.02111186,705.21298518)(119.61111654,705.22299154)
\curveto(119.5811123,705.20298519)(119.54611233,705.19798519)(119.50611654,705.20799154)
\curveto(119.46611241,705.21798517)(119.43111245,705.21798517)(119.40111654,705.20799154)
\curveto(119.35111253,705.1879852)(119.29611258,705.17298522)(119.23611654,705.16299154)
\curveto(119.1761127,705.16298523)(119.12111276,705.15298524)(119.07111654,705.13299154)
\curveto(118.63111325,704.9929854)(118.30611357,704.71798567)(118.09611654,704.30799154)
\curveto(118.0761138,704.26798612)(118.05111383,704.21298618)(118.02111654,704.14299154)
\curveto(118.00111388,704.08298631)(117.98611389,704.01798637)(117.97611654,703.94799154)
\curveto(117.96611391,703.8879865)(117.96611391,703.82798656)(117.97611654,703.76799154)
\curveto(117.99611388,703.70798668)(118.03111385,703.65798673)(118.08111654,703.61799154)
\curveto(118.16111372,703.56798682)(118.27111361,703.54298685)(118.41111654,703.54299154)
\lineto(118.81611654,703.54299154)
\lineto(120.48111654,703.54299154)
\lineto(120.91611654,703.54299154)
\curveto(121.0761108,703.55298684)(121.1811107,703.59798679)(121.23111654,703.67799154)
}
}
{
\newrgbcolor{curcolor}{0 0 0}
\pscustom[linestyle=none,fillstyle=solid,fillcolor=curcolor]
{
}
}
{
\newrgbcolor{curcolor}{0 0 0}
\pscustom[linestyle=none,fillstyle=solid,fillcolor=curcolor]
{
\newpath
\moveto(135.93955404,699.44799154)
\curveto(135.95954619,699.33799105)(135.96954618,699.22799116)(135.96955404,699.11799154)
\curveto(135.97954617,699.00799138)(135.92954622,698.93299146)(135.81955404,698.89299154)
\curveto(135.75954639,698.86299153)(135.68954646,698.84799154)(135.60955404,698.84799154)
\lineto(135.36955404,698.84799154)
\lineto(134.55955404,698.84799154)
\lineto(134.28955404,698.84799154)
\curveto(134.20954794,698.85799153)(134.14454801,698.88299151)(134.09455404,698.92299154)
\curveto(134.02454813,698.96299143)(133.96954818,699.01799137)(133.92955404,699.08799154)
\curveto(133.89954825,699.16799122)(133.8545483,699.23299116)(133.79455404,699.28299154)
\curveto(133.77454838,699.30299109)(133.7495484,699.31799107)(133.71955404,699.32799154)
\curveto(133.68954846,699.34799104)(133.6495485,699.35299104)(133.59955404,699.34299154)
\curveto(133.5495486,699.32299107)(133.49954865,699.29799109)(133.44955404,699.26799154)
\curveto(133.40954874,699.23799115)(133.36454879,699.21299118)(133.31455404,699.19299154)
\curveto(133.26454889,699.15299124)(133.20954894,699.11799127)(133.14955404,699.08799154)
\lineto(132.96955404,698.99799154)
\curveto(132.83954931,698.93799145)(132.70454945,698.8879915)(132.56455404,698.84799154)
\curveto(132.42454973,698.81799157)(132.27954987,698.78299161)(132.12955404,698.74299154)
\curveto(132.05955009,698.72299167)(131.98955016,698.71299168)(131.91955404,698.71299154)
\curveto(131.85955029,698.70299169)(131.79455036,698.6929917)(131.72455404,698.68299154)
\lineto(131.63455404,698.68299154)
\curveto(131.60455055,698.67299172)(131.57455058,698.66799172)(131.54455404,698.66799154)
\lineto(131.37955404,698.66799154)
\curveto(131.27955087,698.64799174)(131.17955097,698.64799174)(131.07955404,698.66799154)
\lineto(130.94455404,698.66799154)
\curveto(130.87455128,698.6879917)(130.80455135,698.69799169)(130.73455404,698.69799154)
\curveto(130.67455148,698.6879917)(130.61455154,698.6929917)(130.55455404,698.71299154)
\curveto(130.4545517,698.73299166)(130.35955179,698.75299164)(130.26955404,698.77299154)
\curveto(130.17955197,698.78299161)(130.09455206,698.80799158)(130.01455404,698.84799154)
\curveto(129.72455243,698.95799143)(129.47455268,699.09799129)(129.26455404,699.26799154)
\curveto(129.06455309,699.44799094)(128.90455325,699.68299071)(128.78455404,699.97299154)
\curveto(128.7545534,700.04299035)(128.72455343,700.11799027)(128.69455404,700.19799154)
\curveto(128.67455348,700.27799011)(128.6545535,700.36299003)(128.63455404,700.45299154)
\curveto(128.61455354,700.50298989)(128.60455355,700.55298984)(128.60455404,700.60299154)
\curveto(128.61455354,700.65298974)(128.61455354,700.70298969)(128.60455404,700.75299154)
\curveto(128.59455356,700.78298961)(128.58455357,700.84298955)(128.57455404,700.93299154)
\curveto(128.57455358,701.03298936)(128.57955357,701.10298929)(128.58955404,701.14299154)
\curveto(128.60955354,701.24298915)(128.61955353,701.32798906)(128.61955404,701.39799154)
\lineto(128.70955404,701.72799154)
\curveto(128.73955341,701.84798854)(128.77955337,701.95298844)(128.82955404,702.04299154)
\curveto(128.99955315,702.33298806)(129.19455296,702.55298784)(129.41455404,702.70299154)
\curveto(129.63455252,702.85298754)(129.91455224,702.98298741)(130.25455404,703.09299154)
\curveto(130.38455177,703.14298725)(130.51955163,703.17798721)(130.65955404,703.19799154)
\curveto(130.79955135,703.21798717)(130.93955121,703.24298715)(131.07955404,703.27299154)
\curveto(131.15955099,703.2929871)(131.24455091,703.30298709)(131.33455404,703.30299154)
\curveto(131.42455073,703.31298708)(131.51455064,703.32798706)(131.60455404,703.34799154)
\curveto(131.67455048,703.36798702)(131.74455041,703.37298702)(131.81455404,703.36299154)
\curveto(131.88455027,703.36298703)(131.95955019,703.37298702)(132.03955404,703.39299154)
\curveto(132.10955004,703.41298698)(132.17954997,703.42298697)(132.24955404,703.42299154)
\curveto(132.31954983,703.42298697)(132.39454976,703.43298696)(132.47455404,703.45299154)
\curveto(132.68454947,703.50298689)(132.87454928,703.54298685)(133.04455404,703.57299154)
\curveto(133.22454893,703.61298678)(133.38454877,703.70298669)(133.52455404,703.84299154)
\curveto(133.61454854,703.93298646)(133.67454848,704.03298636)(133.70455404,704.14299154)
\curveto(133.71454844,704.17298622)(133.71454844,704.19798619)(133.70455404,704.21799154)
\curveto(133.70454845,704.23798615)(133.70954844,704.25798613)(133.71955404,704.27799154)
\curveto(133.72954842,704.29798609)(133.73454842,704.32798606)(133.73455404,704.36799154)
\lineto(133.73455404,704.45799154)
\lineto(133.70455404,704.57799154)
\curveto(133.70454845,704.61798577)(133.69954845,704.65298574)(133.68955404,704.68299154)
\curveto(133.58954856,704.98298541)(133.37954877,705.1879852)(133.05955404,705.29799154)
\curveto(132.96954918,705.32798506)(132.85954929,705.34798504)(132.72955404,705.35799154)
\curveto(132.60954954,705.37798501)(132.48454967,705.38298501)(132.35455404,705.37299154)
\curveto(132.22454993,705.37298502)(132.09955005,705.36298503)(131.97955404,705.34299154)
\curveto(131.85955029,705.32298507)(131.7545504,705.29798509)(131.66455404,705.26799154)
\curveto(131.60455055,705.24798514)(131.54455061,705.21798517)(131.48455404,705.17799154)
\curveto(131.43455072,705.14798524)(131.38455077,705.11298528)(131.33455404,705.07299154)
\curveto(131.28455087,705.03298536)(131.22955092,704.97798541)(131.16955404,704.90799154)
\curveto(131.11955103,704.83798555)(131.08455107,704.77298562)(131.06455404,704.71299154)
\curveto(131.01455114,704.61298578)(130.96955118,704.51798587)(130.92955404,704.42799154)
\curveto(130.89955125,704.33798605)(130.82955132,704.27798611)(130.71955404,704.24799154)
\curveto(130.63955151,704.22798616)(130.5545516,704.21798617)(130.46455404,704.21799154)
\lineto(130.19455404,704.21799154)
\lineto(129.62455404,704.21799154)
\curveto(129.57455258,704.21798617)(129.52455263,704.21298618)(129.47455404,704.20299154)
\curveto(129.42455273,704.20298619)(129.37955277,704.20798618)(129.33955404,704.21799154)
\lineto(129.20455404,704.21799154)
\curveto(129.18455297,704.22798616)(129.15955299,704.23298616)(129.12955404,704.23299154)
\curveto(129.09955305,704.23298616)(129.07455308,704.24298615)(129.05455404,704.26299154)
\curveto(128.97455318,704.28298611)(128.91955323,704.34798604)(128.88955404,704.45799154)
\curveto(128.87955327,704.50798588)(128.87955327,704.55798583)(128.88955404,704.60799154)
\curveto(128.89955325,704.65798573)(128.90955324,704.70298569)(128.91955404,704.74299154)
\curveto(128.9495532,704.85298554)(128.97955317,704.95298544)(129.00955404,705.04299154)
\curveto(129.0495531,705.14298525)(129.09455306,705.23298516)(129.14455404,705.31299154)
\lineto(129.23455404,705.46299154)
\lineto(129.32455404,705.61299154)
\curveto(129.40455275,705.72298467)(129.50455265,705.82798456)(129.62455404,705.92799154)
\curveto(129.64455251,705.93798445)(129.67455248,705.96298443)(129.71455404,706.00299154)
\curveto(129.76455239,706.04298435)(129.80955234,706.07798431)(129.84955404,706.10799154)
\curveto(129.88955226,706.13798425)(129.93455222,706.16798422)(129.98455404,706.19799154)
\curveto(130.154552,706.30798408)(130.33455182,706.392984)(130.52455404,706.45299154)
\curveto(130.71455144,706.52298387)(130.90955124,706.5879838)(131.10955404,706.64799154)
\curveto(131.22955092,706.67798371)(131.3545508,706.69798369)(131.48455404,706.70799154)
\curveto(131.61455054,706.71798367)(131.74455041,706.73798365)(131.87455404,706.76799154)
\curveto(131.91455024,706.77798361)(131.97455018,706.77798361)(132.05455404,706.76799154)
\curveto(132.14455001,706.75798363)(132.19954995,706.76298363)(132.21955404,706.78299154)
\curveto(132.62954952,706.7929836)(133.01954913,706.77798361)(133.38955404,706.73799154)
\curveto(133.76954838,706.69798369)(134.10954804,706.62298377)(134.40955404,706.51299154)
\curveto(134.71954743,706.40298399)(134.98454717,706.25298414)(135.20455404,706.06299154)
\curveto(135.42454673,705.88298451)(135.59454656,705.64798474)(135.71455404,705.35799154)
\curveto(135.78454637,705.1879852)(135.82454633,704.9929854)(135.83455404,704.77299154)
\curveto(135.84454631,704.55298584)(135.8495463,704.32798606)(135.84955404,704.09799154)
\lineto(135.84955404,700.75299154)
\lineto(135.84955404,700.16799154)
\curveto(135.8495463,699.97799041)(135.86954628,699.80299059)(135.90955404,699.64299154)
\curveto(135.91954623,699.61299078)(135.92454623,699.57799081)(135.92455404,699.53799154)
\curveto(135.92454623,699.50799088)(135.92954622,699.47799091)(135.93955404,699.44799154)
\moveto(133.73455404,701.75799154)
\curveto(133.74454841,701.80798858)(133.7495484,701.86298853)(133.74955404,701.92299154)
\curveto(133.7495484,701.9929884)(133.74454841,702.05298834)(133.73455404,702.10299154)
\curveto(133.71454844,702.16298823)(133.70454845,702.21798817)(133.70455404,702.26799154)
\curveto(133.70454845,702.31798807)(133.68454847,702.35798803)(133.64455404,702.38799154)
\curveto(133.59454856,702.42798796)(133.51954863,702.44798794)(133.41955404,702.44799154)
\curveto(133.37954877,702.43798795)(133.34454881,702.42798796)(133.31455404,702.41799154)
\curveto(133.28454887,702.41798797)(133.2495489,702.41298798)(133.20955404,702.40299154)
\curveto(133.13954901,702.38298801)(133.06454909,702.36798802)(132.98455404,702.35799154)
\curveto(132.90454925,702.34798804)(132.82454933,702.33298806)(132.74455404,702.31299154)
\curveto(132.71454944,702.30298809)(132.66954948,702.29798809)(132.60955404,702.29799154)
\curveto(132.47954967,702.26798812)(132.3495498,702.24798814)(132.21955404,702.23799154)
\curveto(132.08955006,702.22798816)(131.96455019,702.20298819)(131.84455404,702.16299154)
\curveto(131.76455039,702.14298825)(131.68955046,702.12298827)(131.61955404,702.10299154)
\curveto(131.5495506,702.0929883)(131.47955067,702.07298832)(131.40955404,702.04299154)
\curveto(131.19955095,701.95298844)(131.01955113,701.81798857)(130.86955404,701.63799154)
\curveto(130.72955142,701.45798893)(130.67955147,701.20798918)(130.71955404,700.88799154)
\curveto(130.73955141,700.71798967)(130.79455136,700.57798981)(130.88455404,700.46799154)
\curveto(130.9545512,700.35799003)(131.05955109,700.26799012)(131.19955404,700.19799154)
\curveto(131.33955081,700.13799025)(131.48955066,700.0929903)(131.64955404,700.06299154)
\curveto(131.81955033,700.03299036)(131.99455016,700.02299037)(132.17455404,700.03299154)
\curveto(132.36454979,700.05299034)(132.53954961,700.0879903)(132.69955404,700.13799154)
\curveto(132.95954919,700.21799017)(133.16454899,700.34299005)(133.31455404,700.51299154)
\curveto(133.46454869,700.6929897)(133.57954857,700.91298948)(133.65955404,701.17299154)
\curveto(133.67954847,701.24298915)(133.68954846,701.31298908)(133.68955404,701.38299154)
\curveto(133.69954845,701.46298893)(133.71454844,701.54298885)(133.73455404,701.62299154)
\lineto(133.73455404,701.75799154)
}
}
{
\newrgbcolor{curcolor}{0 0 0}
\pscustom[linestyle=none,fillstyle=solid,fillcolor=curcolor]
{
\newpath
\moveto(141.07283529,706.79799154)
\curveto(141.88283013,706.81798357)(142.55782946,706.69798369)(143.09783529,706.43799154)
\curveto(143.64782837,706.17798421)(144.08282793,705.80798458)(144.40283529,705.32799154)
\curveto(144.56282745,705.0879853)(144.68282733,704.81298558)(144.76283529,704.50299154)
\curveto(144.78282723,704.45298594)(144.79782722,704.387986)(144.80783529,704.30799154)
\curveto(144.82782719,704.22798616)(144.82782719,704.15798623)(144.80783529,704.09799154)
\curveto(144.76782725,703.9879864)(144.69782732,703.92298647)(144.59783529,703.90299154)
\curveto(144.49782752,703.8929865)(144.37782764,703.8879865)(144.23783529,703.88799154)
\lineto(143.45783529,703.88799154)
\lineto(143.17283529,703.88799154)
\curveto(143.08282893,703.8879865)(143.00782901,703.90798648)(142.94783529,703.94799154)
\curveto(142.86782915,703.9879864)(142.8128292,704.04798634)(142.78283529,704.12799154)
\curveto(142.75282926,704.21798617)(142.7128293,704.30798608)(142.66283529,704.39799154)
\curveto(142.60282941,704.50798588)(142.53782948,704.60798578)(142.46783529,704.69799154)
\curveto(142.39782962,704.7879856)(142.3178297,704.86798552)(142.22783529,704.93799154)
\curveto(142.08782993,705.02798536)(141.93283008,705.09798529)(141.76283529,705.14799154)
\curveto(141.70283031,705.16798522)(141.64283037,705.17798521)(141.58283529,705.17799154)
\curveto(141.52283049,705.17798521)(141.46783055,705.1879852)(141.41783529,705.20799154)
\lineto(141.26783529,705.20799154)
\curveto(141.06783095,705.20798518)(140.90783111,705.1879852)(140.78783529,705.14799154)
\curveto(140.49783152,705.05798533)(140.26283175,704.91798547)(140.08283529,704.72799154)
\curveto(139.90283211,704.54798584)(139.75783226,704.32798606)(139.64783529,704.06799154)
\curveto(139.59783242,703.95798643)(139.55783246,703.83798655)(139.52783529,703.70799154)
\curveto(139.50783251,703.5879868)(139.48283253,703.45798693)(139.45283529,703.31799154)
\curveto(139.44283257,703.27798711)(139.43783258,703.23798715)(139.43783529,703.19799154)
\curveto(139.43783258,703.15798723)(139.43283258,703.11798727)(139.42283529,703.07799154)
\curveto(139.40283261,702.97798741)(139.39283262,702.83798755)(139.39283529,702.65799154)
\curveto(139.40283261,702.47798791)(139.4178326,702.33798805)(139.43783529,702.23799154)
\curveto(139.43783258,702.15798823)(139.44283257,702.10298829)(139.45283529,702.07299154)
\curveto(139.47283254,702.00298839)(139.48283253,701.93298846)(139.48283529,701.86299154)
\curveto(139.49283252,701.7929886)(139.50783251,701.72298867)(139.52783529,701.65299154)
\curveto(139.60783241,701.42298897)(139.70283231,701.21298918)(139.81283529,701.02299154)
\curveto(139.92283209,700.83298956)(140.06283195,700.67298972)(140.23283529,700.54299154)
\curveto(140.27283174,700.51298988)(140.33283168,700.47798991)(140.41283529,700.43799154)
\curveto(140.52283149,700.36799002)(140.63283138,700.32299007)(140.74283529,700.30299154)
\curveto(140.86283115,700.28299011)(141.00783101,700.26299013)(141.17783529,700.24299154)
\lineto(141.26783529,700.24299154)
\curveto(141.30783071,700.24299015)(141.33783068,700.24799014)(141.35783529,700.25799154)
\lineto(141.49283529,700.25799154)
\curveto(141.56283045,700.27799011)(141.62783039,700.2929901)(141.68783529,700.30299154)
\curveto(141.75783026,700.32299007)(141.82283019,700.34299005)(141.88283529,700.36299154)
\curveto(142.18282983,700.4929899)(142.4128296,700.68298971)(142.57283529,700.93299154)
\curveto(142.6128294,700.98298941)(142.64782937,701.03798935)(142.67783529,701.09799154)
\curveto(142.70782931,701.16798922)(142.73282928,701.22798916)(142.75283529,701.27799154)
\curveto(142.79282922,701.387989)(142.82782919,701.48298891)(142.85783529,701.56299154)
\curveto(142.88782913,701.65298874)(142.95782906,701.72298867)(143.06783529,701.77299154)
\curveto(143.15782886,701.81298858)(143.30282871,701.82798856)(143.50283529,701.81799154)
\lineto(143.99783529,701.81799154)
\lineto(144.20783529,701.81799154)
\curveto(144.28782773,701.82798856)(144.35282766,701.82298857)(144.40283529,701.80299154)
\lineto(144.52283529,701.80299154)
\lineto(144.64283529,701.77299154)
\curveto(144.68282733,701.77298862)(144.7128273,701.76298863)(144.73283529,701.74299154)
\curveto(144.78282723,701.70298869)(144.8128272,701.64298875)(144.82283529,701.56299154)
\curveto(144.84282717,701.4929889)(144.84282717,701.41798897)(144.82283529,701.33799154)
\curveto(144.73282728,701.00798938)(144.62282739,700.71298968)(144.49283529,700.45299154)
\curveto(144.08282793,699.68299071)(143.42782859,699.14799124)(142.52783529,698.84799154)
\curveto(142.42782959,698.81799157)(142.32282969,698.79799159)(142.21283529,698.78799154)
\curveto(142.10282991,698.76799162)(141.99283002,698.74299165)(141.88283529,698.71299154)
\curveto(141.82283019,698.70299169)(141.76283025,698.69799169)(141.70283529,698.69799154)
\curveto(141.64283037,698.69799169)(141.58283043,698.6929917)(141.52283529,698.68299154)
\lineto(141.35783529,698.68299154)
\curveto(141.30783071,698.66299173)(141.23283078,698.65799173)(141.13283529,698.66799154)
\curveto(141.03283098,698.66799172)(140.95783106,698.67299172)(140.90783529,698.68299154)
\curveto(140.82783119,698.70299169)(140.75283126,698.71299168)(140.68283529,698.71299154)
\curveto(140.62283139,698.70299169)(140.55783146,698.70799168)(140.48783529,698.72799154)
\lineto(140.33783529,698.75799154)
\curveto(140.28783173,698.75799163)(140.23783178,698.76299163)(140.18783529,698.77299154)
\curveto(140.07783194,698.80299159)(139.97283204,698.83299156)(139.87283529,698.86299154)
\curveto(139.77283224,698.8929915)(139.67783234,698.92799146)(139.58783529,698.96799154)
\curveto(139.1178329,699.16799122)(138.72283329,699.42299097)(138.40283529,699.73299154)
\curveto(138.08283393,700.05299034)(137.82283419,700.44798994)(137.62283529,700.91799154)
\curveto(137.57283444,701.00798938)(137.53283448,701.10298929)(137.50283529,701.20299154)
\lineto(137.41283529,701.53299154)
\curveto(137.40283461,701.57298882)(137.39783462,701.60798878)(137.39783529,701.63799154)
\curveto(137.39783462,701.67798871)(137.38783463,701.72298867)(137.36783529,701.77299154)
\curveto(137.34783467,701.84298855)(137.33783468,701.91298848)(137.33783529,701.98299154)
\curveto(137.33783468,702.06298833)(137.32783469,702.13798825)(137.30783529,702.20799154)
\lineto(137.30783529,702.46299154)
\curveto(137.28783473,702.51298788)(137.27783474,702.56798782)(137.27783529,702.62799154)
\curveto(137.27783474,702.69798769)(137.28783473,702.75798763)(137.30783529,702.80799154)
\curveto(137.3178347,702.85798753)(137.3178347,702.90298749)(137.30783529,702.94299154)
\curveto(137.29783472,702.98298741)(137.29783472,703.02298737)(137.30783529,703.06299154)
\curveto(137.32783469,703.13298726)(137.33283468,703.19798719)(137.32283529,703.25799154)
\curveto(137.32283469,703.31798707)(137.33283468,703.37798701)(137.35283529,703.43799154)
\curveto(137.40283461,703.61798677)(137.44283457,703.7879866)(137.47283529,703.94799154)
\curveto(137.50283451,704.11798627)(137.54783447,704.28298611)(137.60783529,704.44299154)
\curveto(137.82783419,704.95298544)(138.10283391,705.37798501)(138.43283529,705.71799154)
\curveto(138.77283324,706.05798433)(139.20283281,706.33298406)(139.72283529,706.54299154)
\curveto(139.86283215,706.60298379)(140.00783201,706.64298375)(140.15783529,706.66299154)
\curveto(140.30783171,706.6929837)(140.46283155,706.72798366)(140.62283529,706.76799154)
\curveto(140.70283131,706.77798361)(140.77783124,706.78298361)(140.84783529,706.78299154)
\curveto(140.9178311,706.78298361)(140.99283102,706.7879836)(141.07283529,706.79799154)
}
}
{
\newrgbcolor{curcolor}{0 0 0}
\pscustom[linestyle=none,fillstyle=solid,fillcolor=curcolor]
{
\newpath
\moveto(147.16611654,708.89799154)
\lineto(148.17111654,708.89799154)
\curveto(148.32111356,708.89798149)(148.45111343,708.8879815)(148.56111654,708.86799154)
\curveto(148.6811132,708.85798153)(148.76611311,708.79798159)(148.81611654,708.68799154)
\curveto(148.83611304,708.63798175)(148.84611303,708.57798181)(148.84611654,708.50799154)
\lineto(148.84611654,708.29799154)
\lineto(148.84611654,707.62299154)
\curveto(148.84611303,707.57298282)(148.84111304,707.51298288)(148.83111654,707.44299154)
\curveto(148.83111305,707.38298301)(148.83611304,707.32798306)(148.84611654,707.27799154)
\lineto(148.84611654,707.11299154)
\curveto(148.84611303,707.03298336)(148.85111303,706.95798343)(148.86111654,706.88799154)
\curveto(148.87111301,706.82798356)(148.89611298,706.77298362)(148.93611654,706.72299154)
\curveto(149.00611287,706.63298376)(149.13111275,706.58298381)(149.31111654,706.57299154)
\lineto(149.85111654,706.57299154)
\lineto(150.03111654,706.57299154)
\curveto(150.09111179,706.57298382)(150.14611173,706.56298383)(150.19611654,706.54299154)
\curveto(150.30611157,706.4929839)(150.36611151,706.40298399)(150.37611654,706.27299154)
\curveto(150.39611148,706.14298425)(150.40611147,705.99798439)(150.40611654,705.83799154)
\lineto(150.40611654,705.62799154)
\curveto(150.41611146,705.55798483)(150.41111147,705.49798489)(150.39111654,705.44799154)
\curveto(150.34111154,705.2879851)(150.23611164,705.20298519)(150.07611654,705.19299154)
\curveto(149.91611196,705.18298521)(149.73611214,705.17798521)(149.53611654,705.17799154)
\lineto(149.40111654,705.17799154)
\curveto(149.36111252,705.1879852)(149.32611255,705.1879852)(149.29611654,705.17799154)
\curveto(149.25611262,705.16798522)(149.22111266,705.16298523)(149.19111654,705.16299154)
\curveto(149.16111272,705.17298522)(149.13111275,705.16798522)(149.10111654,705.14799154)
\curveto(149.02111286,705.12798526)(148.96111292,705.08298531)(148.92111654,705.01299154)
\curveto(148.89111299,704.95298544)(148.86611301,704.87798551)(148.84611654,704.78799154)
\curveto(148.83611304,704.73798565)(148.83611304,704.68298571)(148.84611654,704.62299154)
\curveto(148.85611302,704.56298583)(148.85611302,704.50798588)(148.84611654,704.45799154)
\lineto(148.84611654,703.52799154)
\lineto(148.84611654,701.77299154)
\curveto(148.84611303,701.52298887)(148.85111303,701.30298909)(148.86111654,701.11299154)
\curveto(148.881113,700.93298946)(148.94611293,700.77298962)(149.05611654,700.63299154)
\curveto(149.10611277,700.57298982)(149.17111271,700.52798986)(149.25111654,700.49799154)
\lineto(149.52111654,700.43799154)
\curveto(149.55111233,700.42798996)(149.5811123,700.42298997)(149.61111654,700.42299154)
\curveto(149.65111223,700.43298996)(149.6811122,700.43298996)(149.70111654,700.42299154)
\lineto(149.86611654,700.42299154)
\curveto(149.9761119,700.42298997)(150.07111181,700.41798997)(150.15111654,700.40799154)
\curveto(150.23111165,700.39798999)(150.29611158,700.35799003)(150.34611654,700.28799154)
\curveto(150.38611149,700.22799016)(150.40611147,700.14799024)(150.40611654,700.04799154)
\lineto(150.40611654,699.76299154)
\curveto(150.40611147,699.55299084)(150.40111148,699.35799103)(150.39111654,699.17799154)
\curveto(150.39111149,699.00799138)(150.31111157,698.8929915)(150.15111654,698.83299154)
\curveto(150.10111178,698.81299158)(150.05611182,698.80799158)(150.01611654,698.81799154)
\curveto(149.9761119,698.81799157)(149.93111195,698.80799158)(149.88111654,698.78799154)
\lineto(149.73111654,698.78799154)
\curveto(149.71111217,698.7879916)(149.6811122,698.7929916)(149.64111654,698.80299154)
\curveto(149.60111228,698.80299159)(149.56611231,698.79799159)(149.53611654,698.78799154)
\curveto(149.48611239,698.77799161)(149.43111245,698.77799161)(149.37111654,698.78799154)
\lineto(149.22111654,698.78799154)
\lineto(149.07111654,698.78799154)
\curveto(149.02111286,698.77799161)(148.9761129,698.77799161)(148.93611654,698.78799154)
\lineto(148.77111654,698.78799154)
\curveto(148.72111316,698.79799159)(148.66611321,698.80299159)(148.60611654,698.80299154)
\curveto(148.54611333,698.80299159)(148.49111339,698.80799158)(148.44111654,698.81799154)
\curveto(148.37111351,698.82799156)(148.30611357,698.83799155)(148.24611654,698.84799154)
\lineto(148.06611654,698.87799154)
\curveto(147.95611392,698.90799148)(147.85111403,698.94299145)(147.75111654,698.98299154)
\curveto(147.65111423,699.02299137)(147.55611432,699.06799132)(147.46611654,699.11799154)
\lineto(147.37611654,699.17799154)
\curveto(147.34611453,699.20799118)(147.31111457,699.23799115)(147.27111654,699.26799154)
\curveto(147.25111463,699.2879911)(147.22611465,699.30799108)(147.19611654,699.32799154)
\lineto(147.12111654,699.40299154)
\curveto(146.9811149,699.5929908)(146.876115,699.80299059)(146.80611654,700.03299154)
\curveto(146.78611509,700.07299032)(146.7761151,700.10799028)(146.77611654,700.13799154)
\curveto(146.78611509,700.17799021)(146.78611509,700.22299017)(146.77611654,700.27299154)
\curveto(146.76611511,700.2929901)(146.76111512,700.31799007)(146.76111654,700.34799154)
\curveto(146.76111512,700.37799001)(146.75611512,700.40298999)(146.74611654,700.42299154)
\lineto(146.74611654,700.57299154)
\curveto(146.73611514,700.61298978)(146.73111515,700.65798973)(146.73111654,700.70799154)
\curveto(146.74111514,700.75798963)(146.74611513,700.80798958)(146.74611654,700.85799154)
\lineto(146.74611654,701.42799154)
\lineto(146.74611654,703.66299154)
\lineto(146.74611654,704.45799154)
\lineto(146.74611654,704.66799154)
\curveto(146.75611512,704.73798565)(146.75111513,704.80298559)(146.73111654,704.86299154)
\curveto(146.69111519,705.00298539)(146.62111526,705.0929853)(146.52111654,705.13299154)
\curveto(146.41111547,705.18298521)(146.27111561,705.19798519)(146.10111654,705.17799154)
\curveto(145.93111595,705.15798523)(145.78611609,705.17298522)(145.66611654,705.22299154)
\curveto(145.58611629,705.25298514)(145.53611634,705.29798509)(145.51611654,705.35799154)
\curveto(145.49611638,705.41798497)(145.4761164,705.4929849)(145.45611654,705.58299154)
\lineto(145.45611654,705.89799154)
\curveto(145.45611642,706.07798431)(145.46611641,706.22298417)(145.48611654,706.33299154)
\curveto(145.50611637,706.44298395)(145.59111629,706.51798387)(145.74111654,706.55799154)
\curveto(145.7811161,706.57798381)(145.82111606,706.58298381)(145.86111654,706.57299154)
\lineto(145.99611654,706.57299154)
\curveto(146.14611573,706.57298382)(146.28611559,706.57798381)(146.41611654,706.58799154)
\curveto(146.54611533,706.60798378)(146.63611524,706.66798372)(146.68611654,706.76799154)
\curveto(146.71611516,706.83798355)(146.73111515,706.91798347)(146.73111654,707.00799154)
\curveto(146.74111514,707.09798329)(146.74611513,707.1879832)(146.74611654,707.27799154)
\lineto(146.74611654,708.20799154)
\lineto(146.74611654,708.46299154)
\curveto(146.74611513,708.55298184)(146.75611512,708.62798176)(146.77611654,708.68799154)
\curveto(146.82611505,708.7879816)(146.90111498,708.85298154)(147.00111654,708.88299154)
\curveto(147.02111486,708.8929815)(147.04611483,708.8929815)(147.07611654,708.88299154)
\curveto(147.11611476,708.88298151)(147.14611473,708.8879815)(147.16611654,708.89799154)
}
}
{
\newrgbcolor{curcolor}{0 0 0}
\pscustom[linestyle=none,fillstyle=solid,fillcolor=curcolor]
{
\newpath
\moveto(153.48955404,709.43799154)
\curveto(153.55955109,709.35798103)(153.59455106,709.23798115)(153.59455404,709.07799154)
\lineto(153.59455404,708.61299154)
\lineto(153.59455404,708.20799154)
\curveto(153.59455106,708.06798232)(153.55955109,707.97298242)(153.48955404,707.92299154)
\curveto(153.42955122,707.87298252)(153.3495513,707.84298255)(153.24955404,707.83299154)
\curveto(153.15955149,707.82298257)(153.05955159,707.81798257)(152.94955404,707.81799154)
\lineto(152.10955404,707.81799154)
\curveto(151.99955265,707.81798257)(151.89955275,707.82298257)(151.80955404,707.83299154)
\curveto(151.72955292,707.84298255)(151.65955299,707.87298252)(151.59955404,707.92299154)
\curveto(151.55955309,707.95298244)(151.52955312,708.00798238)(151.50955404,708.08799154)
\curveto(151.49955315,708.17798221)(151.48955316,708.27298212)(151.47955404,708.37299154)
\lineto(151.47955404,708.70299154)
\curveto(151.48955316,708.81298158)(151.49455316,708.90798148)(151.49455404,708.98799154)
\lineto(151.49455404,709.19799154)
\curveto(151.50455315,709.26798112)(151.52455313,709.32798106)(151.55455404,709.37799154)
\curveto(151.57455308,709.41798097)(151.59955305,709.44798094)(151.62955404,709.46799154)
\lineto(151.74955404,709.52799154)
\curveto(151.76955288,709.52798086)(151.79455286,709.52798086)(151.82455404,709.52799154)
\curveto(151.8545528,709.53798085)(151.87955277,709.54298085)(151.89955404,709.54299154)
\lineto(152.99455404,709.54299154)
\curveto(153.09455156,709.54298085)(153.18955146,709.53798085)(153.27955404,709.52799154)
\curveto(153.36955128,709.51798087)(153.43955121,709.4879809)(153.48955404,709.43799154)
\moveto(153.59455404,699.67299154)
\curveto(153.59455106,699.47299092)(153.58955106,699.30299109)(153.57955404,699.16299154)
\curveto(153.56955108,699.02299137)(153.47955117,698.92799146)(153.30955404,698.87799154)
\curveto(153.2495514,698.85799153)(153.18455147,698.84799154)(153.11455404,698.84799154)
\curveto(153.04455161,698.85799153)(152.96955168,698.86299153)(152.88955404,698.86299154)
\lineto(152.04955404,698.86299154)
\curveto(151.95955269,698.86299153)(151.86955278,698.86799152)(151.77955404,698.87799154)
\curveto(151.69955295,698.8879915)(151.63955301,698.91799147)(151.59955404,698.96799154)
\curveto(151.53955311,699.03799135)(151.50455315,699.12299127)(151.49455404,699.22299154)
\lineto(151.49455404,699.56799154)
\lineto(151.49455404,705.89799154)
\lineto(151.49455404,706.19799154)
\curveto(151.49455316,706.29798409)(151.51455314,706.37798401)(151.55455404,706.43799154)
\curveto(151.61455304,706.50798388)(151.69955295,706.55298384)(151.80955404,706.57299154)
\curveto(151.82955282,706.58298381)(151.8545528,706.58298381)(151.88455404,706.57299154)
\curveto(151.92455273,706.57298382)(151.9545527,706.57798381)(151.97455404,706.58799154)
\lineto(152.72455404,706.58799154)
\lineto(152.91955404,706.58799154)
\curveto(152.99955165,706.59798379)(153.06455159,706.59798379)(153.11455404,706.58799154)
\lineto(153.23455404,706.58799154)
\curveto(153.29455136,706.56798382)(153.3495513,706.55298384)(153.39955404,706.54299154)
\curveto(153.4495512,706.53298386)(153.48955116,706.50298389)(153.51955404,706.45299154)
\curveto(153.55955109,706.40298399)(153.57955107,706.33298406)(153.57955404,706.24299154)
\curveto(153.58955106,706.15298424)(153.59455106,706.05798433)(153.59455404,705.95799154)
\lineto(153.59455404,699.67299154)
}
}
{
\newrgbcolor{curcolor}{0 0 0}
\pscustom[linestyle=none,fillstyle=solid,fillcolor=curcolor]
{
\newpath
\moveto(155.00174154,706.58799154)
\lineto(156.09674154,706.58799154)
\curveto(156.20673981,706.5879838)(156.31173971,706.58298381)(156.41174154,706.57299154)
\curveto(156.51173951,706.57298382)(156.59673942,706.55298384)(156.66674154,706.51299154)
\curveto(156.77673924,706.44298395)(156.85173917,706.31298408)(156.89174154,706.12299154)
\curveto(156.94173908,705.94298445)(156.99173903,705.78298461)(157.04174154,705.64299154)
\curveto(157.15173887,705.31298508)(157.25673876,704.97298542)(157.35674154,704.62299154)
\curveto(157.45673856,704.28298611)(157.56173846,703.94298645)(157.67174154,703.60299154)
\curveto(157.75173827,703.36298703)(157.82673819,703.11798727)(157.89674154,702.86799154)
\curveto(157.96673805,702.61798777)(158.04673797,702.37798801)(158.13674154,702.14799154)
\curveto(158.16673785,702.05798833)(158.19673782,701.96298843)(158.22674154,701.86299154)
\curveto(158.26673775,701.76298863)(158.34173768,701.71298868)(158.45174154,701.71299154)
\curveto(158.47173755,701.73298866)(158.48673753,701.74298865)(158.49674154,701.74299154)
\lineto(158.54174154,701.78799154)
\curveto(158.57173745,701.83798855)(158.59673742,701.8879885)(158.61674154,701.93799154)
\curveto(158.63673738,701.9879884)(158.65673736,702.04298835)(158.67674154,702.10299154)
\curveto(158.72673729,702.21298818)(158.76173726,702.32798806)(158.78174154,702.44799154)
\curveto(158.81173721,702.56798782)(158.84673717,702.68298771)(158.88674154,702.79299154)
\curveto(159.02673699,703.21298718)(159.15673686,703.63298676)(159.27674154,704.05299154)
\curveto(159.40673661,704.48298591)(159.54173648,704.90798548)(159.68174154,705.32799154)
\curveto(159.73173629,705.44798494)(159.77173625,705.56798482)(159.80174154,705.68799154)
\curveto(159.83173619,705.81798457)(159.86673615,705.93798445)(159.90674154,706.04799154)
\curveto(159.92673609,706.12798426)(159.95173607,706.20798418)(159.98174154,706.28799154)
\curveto(160.01173601,706.36798402)(160.05673596,706.43298396)(160.11674154,706.48299154)
\curveto(160.14673587,706.50298389)(160.21173581,706.53298386)(160.31174154,706.57299154)
\curveto(160.37173565,706.5929838)(160.43673558,706.59798379)(160.50674154,706.58799154)
\lineto(160.70174154,706.58799154)
\lineto(161.43674154,706.58799154)
\curveto(161.54673447,706.5879838)(161.64673437,706.58298381)(161.73674154,706.57299154)
\curveto(161.83673418,706.57298382)(161.91173411,706.54798384)(161.96174154,706.49799154)
\curveto(162.021734,706.44798394)(162.04173398,706.37298402)(162.02174154,706.27299154)
\curveto(162.01173401,706.18298421)(161.99673402,706.10798428)(161.97674154,706.04799154)
\curveto(161.92673409,705.90798448)(161.87673414,705.75798463)(161.82674154,705.59799154)
\curveto(161.77673424,705.44798494)(161.72673429,705.30298509)(161.67674154,705.16299154)
\curveto(161.64673437,705.0929853)(161.6217344,705.02298537)(161.60174154,704.95299154)
\curveto(161.59173443,704.8929855)(161.57673444,704.83298556)(161.55674154,704.77299154)
\curveto(161.50673451,704.66298573)(161.46173456,704.55298584)(161.42174154,704.44299154)
\curveto(161.39173463,704.33298606)(161.35673466,704.22298617)(161.31674154,704.11299154)
\curveto(161.30673471,704.08298631)(161.29673472,704.04798634)(161.28674154,704.00799154)
\curveto(161.28673473,703.97798641)(161.27673474,703.94798644)(161.25674154,703.91799154)
\curveto(161.19673482,703.76798662)(161.14173488,703.61298678)(161.09174154,703.45299154)
\curveto(161.05173497,703.30298709)(161.00673501,703.15298724)(160.95674154,703.00299154)
\curveto(160.84673517,702.70298769)(160.74173528,702.392988)(160.64174154,702.07299154)
\curveto(160.54173548,701.76298863)(160.43173559,701.45798893)(160.31174154,701.15799154)
\curveto(160.26173576,701.01798937)(160.2167358,700.87798951)(160.17674154,700.73799154)
\curveto(160.13673588,700.60798978)(160.09173593,700.47298992)(160.04174154,700.33299154)
\curveto(160.021736,700.28299011)(160.00673601,700.23799015)(159.99674154,700.19799154)
\curveto(159.98673603,700.16799022)(159.97173605,700.12799026)(159.95174154,700.07799154)
\curveto(159.91173611,699.96799042)(159.87173615,699.85299054)(159.83174154,699.73299154)
\curveto(159.80173622,699.62299077)(159.76673625,699.51299088)(159.72674154,699.40299154)
\curveto(159.67673634,699.2929911)(159.63173639,699.1879912)(159.59174154,699.08799154)
\curveto(159.55173647,698.99799139)(159.47673654,698.93299146)(159.36674154,698.89299154)
\curveto(159.28673673,698.86299153)(159.20173682,698.84799154)(159.11174154,698.84799154)
\curveto(159.021737,698.85799153)(158.93173709,698.86299153)(158.84174154,698.86299154)
\lineto(157.86674154,698.86299154)
\lineto(157.53674154,698.86299154)
\curveto(157.43673858,698.86299153)(157.34673867,698.88299151)(157.26674154,698.92299154)
\curveto(157.19673882,698.97299142)(157.14673887,699.03299136)(157.11674154,699.10299154)
\curveto(157.08673893,699.18299121)(157.05673896,699.26799112)(157.02674154,699.35799154)
\curveto(156.97673904,699.47799091)(156.93173909,699.60299079)(156.89174154,699.73299154)
\curveto(156.85173917,699.86299053)(156.80673921,699.9879904)(156.75674154,700.10799154)
\curveto(156.73673928,700.14799024)(156.7217393,700.18299021)(156.71174154,700.21299154)
\curveto(156.71173931,700.25299014)(156.70173932,700.29799009)(156.68174154,700.34799154)
\curveto(156.63173939,700.46798992)(156.58673943,700.5929898)(156.54674154,700.72299154)
\curveto(156.50673951,700.85298954)(156.46173956,700.98298941)(156.41174154,701.11299154)
\curveto(156.39173963,701.16298923)(156.37673964,701.20798918)(156.36674154,701.24799154)
\curveto(156.35673966,701.29798909)(156.34173968,701.34798904)(156.32174154,701.39799154)
\curveto(156.28173974,701.4879889)(156.24673977,701.57798881)(156.21674154,701.66799154)
\curveto(156.18673983,701.76798862)(156.15673986,701.86298853)(156.12674154,701.95299154)
\curveto(156.10673991,701.98298841)(156.09673992,702.01298838)(156.09674154,702.04299154)
\curveto(156.09673992,702.08298831)(156.08673993,702.11798827)(156.06674154,702.14799154)
\curveto(156.02673999,702.23798815)(155.99174003,702.32798806)(155.96174154,702.41799154)
\curveto(155.94174008,702.50798788)(155.91174011,702.60298779)(155.87174154,702.70299154)
\curveto(155.78174024,702.93298746)(155.69674032,703.16798722)(155.61674154,703.40799154)
\curveto(155.53674048,703.65798673)(155.45674056,703.90298649)(155.37674154,704.14299154)
\curveto(155.26674075,704.41298598)(155.17174085,704.68298571)(155.09174154,704.95299154)
\curveto(155.01174101,705.23298516)(154.9217411,705.50798488)(154.82174154,705.77799154)
\lineto(154.68674154,706.18299154)
\curveto(154.67674134,706.21298418)(154.66674135,706.24798414)(154.65674154,706.28799154)
\curveto(154.65674136,706.33798405)(154.66674135,706.38298401)(154.68674154,706.42299154)
\curveto(154.7167413,706.4929839)(154.78674123,706.54298385)(154.89674154,706.57299154)
\curveto(154.94674107,706.57298382)(154.98174104,706.57798381)(155.00174154,706.58799154)
}
}
{
\newrgbcolor{curcolor}{0 0 0}
\pscustom[linestyle=none,fillstyle=solid,fillcolor=curcolor]
{
\newpath
\moveto(165.14971029,709.43799154)
\curveto(165.21970734,709.35798103)(165.25470731,709.23798115)(165.25471029,709.07799154)
\lineto(165.25471029,708.61299154)
\lineto(165.25471029,708.20799154)
\curveto(165.25470731,708.06798232)(165.21970734,707.97298242)(165.14971029,707.92299154)
\curveto(165.08970747,707.87298252)(165.00970755,707.84298255)(164.90971029,707.83299154)
\curveto(164.81970774,707.82298257)(164.71970784,707.81798257)(164.60971029,707.81799154)
\lineto(163.76971029,707.81799154)
\curveto(163.6597089,707.81798257)(163.559709,707.82298257)(163.46971029,707.83299154)
\curveto(163.38970917,707.84298255)(163.31970924,707.87298252)(163.25971029,707.92299154)
\curveto(163.21970934,707.95298244)(163.18970937,708.00798238)(163.16971029,708.08799154)
\curveto(163.1597094,708.17798221)(163.14970941,708.27298212)(163.13971029,708.37299154)
\lineto(163.13971029,708.70299154)
\curveto(163.14970941,708.81298158)(163.15470941,708.90798148)(163.15471029,708.98799154)
\lineto(163.15471029,709.19799154)
\curveto(163.1647094,709.26798112)(163.18470938,709.32798106)(163.21471029,709.37799154)
\curveto(163.23470933,709.41798097)(163.2597093,709.44798094)(163.28971029,709.46799154)
\lineto(163.40971029,709.52799154)
\curveto(163.42970913,709.52798086)(163.45470911,709.52798086)(163.48471029,709.52799154)
\curveto(163.51470905,709.53798085)(163.53970902,709.54298085)(163.55971029,709.54299154)
\lineto(164.65471029,709.54299154)
\curveto(164.75470781,709.54298085)(164.84970771,709.53798085)(164.93971029,709.52799154)
\curveto(165.02970753,709.51798087)(165.09970746,709.4879809)(165.14971029,709.43799154)
\moveto(165.25471029,699.67299154)
\curveto(165.25470731,699.47299092)(165.24970731,699.30299109)(165.23971029,699.16299154)
\curveto(165.22970733,699.02299137)(165.13970742,698.92799146)(164.96971029,698.87799154)
\curveto(164.90970765,698.85799153)(164.84470772,698.84799154)(164.77471029,698.84799154)
\curveto(164.70470786,698.85799153)(164.62970793,698.86299153)(164.54971029,698.86299154)
\lineto(163.70971029,698.86299154)
\curveto(163.61970894,698.86299153)(163.52970903,698.86799152)(163.43971029,698.87799154)
\curveto(163.3597092,698.8879915)(163.29970926,698.91799147)(163.25971029,698.96799154)
\curveto(163.19970936,699.03799135)(163.1647094,699.12299127)(163.15471029,699.22299154)
\lineto(163.15471029,699.56799154)
\lineto(163.15471029,705.89799154)
\lineto(163.15471029,706.19799154)
\curveto(163.15470941,706.29798409)(163.17470939,706.37798401)(163.21471029,706.43799154)
\curveto(163.27470929,706.50798388)(163.3597092,706.55298384)(163.46971029,706.57299154)
\curveto(163.48970907,706.58298381)(163.51470905,706.58298381)(163.54471029,706.57299154)
\curveto(163.58470898,706.57298382)(163.61470895,706.57798381)(163.63471029,706.58799154)
\lineto(164.38471029,706.58799154)
\lineto(164.57971029,706.58799154)
\curveto(164.6597079,706.59798379)(164.72470784,706.59798379)(164.77471029,706.58799154)
\lineto(164.89471029,706.58799154)
\curveto(164.95470761,706.56798382)(165.00970755,706.55298384)(165.05971029,706.54299154)
\curveto(165.10970745,706.53298386)(165.14970741,706.50298389)(165.17971029,706.45299154)
\curveto(165.21970734,706.40298399)(165.23970732,706.33298406)(165.23971029,706.24299154)
\curveto(165.24970731,706.15298424)(165.25470731,706.05798433)(165.25471029,705.95799154)
\lineto(165.25471029,699.67299154)
}
}
{
\newrgbcolor{curcolor}{0 0 0}
\pscustom[linestyle=none,fillstyle=solid,fillcolor=curcolor]
{
\newpath
\moveto(174.50689779,699.70299154)
\lineto(174.50689779,699.28299154)
\curveto(174.50688942,699.15299124)(174.47688945,699.04799134)(174.41689779,698.96799154)
\curveto(174.36688956,698.91799147)(174.30188963,698.88299151)(174.22189779,698.86299154)
\curveto(174.14188979,698.85299154)(174.05188988,698.84799154)(173.95189779,698.84799154)
\lineto(173.12689779,698.84799154)
\lineto(172.84189779,698.84799154)
\curveto(172.76189117,698.85799153)(172.69689123,698.88299151)(172.64689779,698.92299154)
\curveto(172.57689135,698.97299142)(172.53689139,699.03799135)(172.52689779,699.11799154)
\curveto(172.51689141,699.19799119)(172.49689143,699.27799111)(172.46689779,699.35799154)
\curveto(172.44689148,699.37799101)(172.4268915,699.392991)(172.40689779,699.40299154)
\curveto(172.39689153,699.42299097)(172.38189155,699.44299095)(172.36189779,699.46299154)
\curveto(172.25189168,699.46299093)(172.17189176,699.43799095)(172.12189779,699.38799154)
\lineto(171.97189779,699.23799154)
\curveto(171.90189203,699.1879912)(171.83689209,699.14299125)(171.77689779,699.10299154)
\curveto(171.71689221,699.07299132)(171.65189228,699.03299136)(171.58189779,698.98299154)
\curveto(171.54189239,698.96299143)(171.49689243,698.94299145)(171.44689779,698.92299154)
\curveto(171.40689252,698.90299149)(171.36189257,698.88299151)(171.31189779,698.86299154)
\curveto(171.17189276,698.81299158)(171.02189291,698.76799162)(170.86189779,698.72799154)
\curveto(170.81189312,698.70799168)(170.76689316,698.69799169)(170.72689779,698.69799154)
\curveto(170.68689324,698.69799169)(170.64689328,698.6929917)(170.60689779,698.68299154)
\lineto(170.47189779,698.68299154)
\curveto(170.44189349,698.67299172)(170.40189353,698.66799172)(170.35189779,698.66799154)
\lineto(170.21689779,698.66799154)
\curveto(170.15689377,698.64799174)(170.06689386,698.64299175)(169.94689779,698.65299154)
\curveto(169.8268941,698.65299174)(169.74189419,698.66299173)(169.69189779,698.68299154)
\curveto(169.62189431,698.70299169)(169.55689437,698.71299168)(169.49689779,698.71299154)
\curveto(169.44689448,698.70299169)(169.39189454,698.70799168)(169.33189779,698.72799154)
\lineto(168.97189779,698.84799154)
\curveto(168.86189507,698.87799151)(168.75189518,698.91799147)(168.64189779,698.96799154)
\curveto(168.29189564,699.11799127)(167.97689595,699.34799104)(167.69689779,699.65799154)
\curveto(167.4268965,699.97799041)(167.21189672,700.31299008)(167.05189779,700.66299154)
\curveto(167.00189693,700.77298962)(166.96189697,700.87798951)(166.93189779,700.97799154)
\curveto(166.90189703,701.0879893)(166.86689706,701.19798919)(166.82689779,701.30799154)
\curveto(166.81689711,701.34798904)(166.81189712,701.38298901)(166.81189779,701.41299154)
\curveto(166.81189712,701.45298894)(166.80189713,701.49798889)(166.78189779,701.54799154)
\curveto(166.76189717,701.62798876)(166.74189719,701.71298868)(166.72189779,701.80299154)
\curveto(166.71189722,701.90298849)(166.69689723,702.00298839)(166.67689779,702.10299154)
\curveto(166.66689726,702.13298826)(166.66189727,702.16798822)(166.66189779,702.20799154)
\curveto(166.67189726,702.24798814)(166.67189726,702.28298811)(166.66189779,702.31299154)
\lineto(166.66189779,702.44799154)
\curveto(166.66189727,702.49798789)(166.65689727,702.54798784)(166.64689779,702.59799154)
\curveto(166.63689729,702.64798774)(166.6318973,702.70298769)(166.63189779,702.76299154)
\curveto(166.6318973,702.83298756)(166.63689729,702.8879875)(166.64689779,702.92799154)
\curveto(166.65689727,702.97798741)(166.66189727,703.02298737)(166.66189779,703.06299154)
\lineto(166.66189779,703.21299154)
\curveto(166.67189726,703.26298713)(166.67189726,703.30798708)(166.66189779,703.34799154)
\curveto(166.66189727,703.39798699)(166.67189726,703.44798694)(166.69189779,703.49799154)
\curveto(166.71189722,703.60798678)(166.7268972,703.71298668)(166.73689779,703.81299154)
\curveto(166.75689717,703.91298648)(166.78189715,704.01298638)(166.81189779,704.11299154)
\curveto(166.85189708,704.23298616)(166.88689704,704.34798604)(166.91689779,704.45799154)
\curveto(166.94689698,704.56798582)(166.98689694,704.67798571)(167.03689779,704.78799154)
\curveto(167.17689675,705.0879853)(167.35189658,705.37298502)(167.56189779,705.64299154)
\curveto(167.58189635,705.67298472)(167.60689632,705.69798469)(167.63689779,705.71799154)
\curveto(167.67689625,705.74798464)(167.70689622,705.77798461)(167.72689779,705.80799154)
\curveto(167.76689616,705.85798453)(167.80689612,705.90298449)(167.84689779,705.94299154)
\curveto(167.88689604,705.98298441)(167.931896,706.02298437)(167.98189779,706.06299154)
\curveto(168.02189591,706.08298431)(168.05689587,706.10798428)(168.08689779,706.13799154)
\curveto(168.11689581,706.17798421)(168.15189578,706.20798418)(168.19189779,706.22799154)
\curveto(168.44189549,706.39798399)(168.7318952,706.53798385)(169.06189779,706.64799154)
\curveto(169.1318948,706.66798372)(169.20189473,706.68298371)(169.27189779,706.69299154)
\curveto(169.35189458,706.70298369)(169.4318945,706.71798367)(169.51189779,706.73799154)
\curveto(169.58189435,706.75798363)(169.67189426,706.76798362)(169.78189779,706.76799154)
\curveto(169.89189404,706.77798361)(170.00189393,706.78298361)(170.11189779,706.78299154)
\curveto(170.22189371,706.78298361)(170.3268936,706.77798361)(170.42689779,706.76799154)
\curveto(170.53689339,706.75798363)(170.6268933,706.74298365)(170.69689779,706.72299154)
\curveto(170.84689308,706.67298372)(170.99189294,706.62798376)(171.13189779,706.58799154)
\curveto(171.27189266,706.54798384)(171.40189253,706.4929839)(171.52189779,706.42299154)
\curveto(171.59189234,706.37298402)(171.65689227,706.32298407)(171.71689779,706.27299154)
\curveto(171.77689215,706.23298416)(171.84189209,706.1879842)(171.91189779,706.13799154)
\curveto(171.95189198,706.10798428)(172.00689192,706.06798432)(172.07689779,706.01799154)
\curveto(172.15689177,705.96798442)(172.2318917,705.96798442)(172.30189779,706.01799154)
\curveto(172.34189159,706.03798435)(172.36189157,706.07298432)(172.36189779,706.12299154)
\curveto(172.36189157,706.17298422)(172.37189156,706.22298417)(172.39189779,706.27299154)
\lineto(172.39189779,706.42299154)
\curveto(172.40189153,706.45298394)(172.40689152,706.4879839)(172.40689779,706.52799154)
\lineto(172.40689779,706.64799154)
\lineto(172.40689779,708.68799154)
\curveto(172.40689152,708.79798159)(172.40189153,708.91798147)(172.39189779,709.04799154)
\curveto(172.39189154,709.1879812)(172.41689151,709.2929811)(172.46689779,709.36299154)
\curveto(172.50689142,709.44298095)(172.58189135,709.4929809)(172.69189779,709.51299154)
\curveto(172.71189122,709.52298087)(172.7318912,709.52298087)(172.75189779,709.51299154)
\curveto(172.77189116,709.51298088)(172.79189114,709.51798087)(172.81189779,709.52799154)
\lineto(173.87689779,709.52799154)
\curveto(173.99688993,709.52798086)(174.10688982,709.52298087)(174.20689779,709.51299154)
\curveto(174.30688962,709.50298089)(174.38188955,709.46298093)(174.43189779,709.39299154)
\curveto(174.48188945,709.31298108)(174.50688942,709.20798118)(174.50689779,709.07799154)
\lineto(174.50689779,708.71799154)
\lineto(174.50689779,699.70299154)
\moveto(172.46689779,702.64299154)
\curveto(172.47689145,702.68298771)(172.47689145,702.72298767)(172.46689779,702.76299154)
\lineto(172.46689779,702.89799154)
\curveto(172.46689146,702.99798739)(172.46189147,703.09798729)(172.45189779,703.19799154)
\curveto(172.44189149,703.29798709)(172.4268915,703.387987)(172.40689779,703.46799154)
\curveto(172.38689154,703.57798681)(172.36689156,703.67798671)(172.34689779,703.76799154)
\curveto(172.33689159,703.85798653)(172.31189162,703.94298645)(172.27189779,704.02299154)
\curveto(172.1318918,704.38298601)(171.926892,704.66798572)(171.65689779,704.87799154)
\curveto(171.39689253,705.0879853)(171.01689291,705.1929852)(170.51689779,705.19299154)
\curveto(170.45689347,705.1929852)(170.37689355,705.18298521)(170.27689779,705.16299154)
\curveto(170.19689373,705.14298525)(170.12189381,705.12298527)(170.05189779,705.10299154)
\curveto(169.99189394,705.0929853)(169.931894,705.07298532)(169.87189779,705.04299154)
\curveto(169.60189433,704.93298546)(169.39189454,704.76298563)(169.24189779,704.53299154)
\curveto(169.09189484,704.30298609)(168.97189496,704.04298635)(168.88189779,703.75299154)
\curveto(168.85189508,703.65298674)(168.8318951,703.55298684)(168.82189779,703.45299154)
\curveto(168.81189512,703.35298704)(168.79189514,703.24798714)(168.76189779,703.13799154)
\lineto(168.76189779,702.92799154)
\curveto(168.74189519,702.83798755)(168.73689519,702.71298768)(168.74689779,702.55299154)
\curveto(168.75689517,702.40298799)(168.77189516,702.2929881)(168.79189779,702.22299154)
\lineto(168.79189779,702.13299154)
\curveto(168.80189513,702.11298828)(168.80689512,702.0929883)(168.80689779,702.07299154)
\curveto(168.8268951,701.9929884)(168.84189509,701.91798847)(168.85189779,701.84799154)
\curveto(168.87189506,701.77798861)(168.89189504,701.70298869)(168.91189779,701.62299154)
\curveto(169.08189485,701.10298929)(169.37189456,700.71798967)(169.78189779,700.46799154)
\curveto(169.91189402,700.37799001)(170.09189384,700.30799008)(170.32189779,700.25799154)
\curveto(170.36189357,700.24799014)(170.42189351,700.24299015)(170.50189779,700.24299154)
\curveto(170.5318934,700.23299016)(170.57689335,700.22299017)(170.63689779,700.21299154)
\curveto(170.70689322,700.21299018)(170.76189317,700.21799017)(170.80189779,700.22799154)
\curveto(170.88189305,700.24799014)(170.96189297,700.26299013)(171.04189779,700.27299154)
\curveto(171.12189281,700.28299011)(171.20189273,700.30299009)(171.28189779,700.33299154)
\curveto(171.5318924,700.44298995)(171.7318922,700.58298981)(171.88189779,700.75299154)
\curveto(172.0318919,700.92298947)(172.16189177,701.13798925)(172.27189779,701.39799154)
\curveto(172.31189162,701.4879889)(172.34189159,701.57798881)(172.36189779,701.66799154)
\curveto(172.38189155,701.76798862)(172.40189153,701.87298852)(172.42189779,701.98299154)
\curveto(172.4318915,702.03298836)(172.4318915,702.07798831)(172.42189779,702.11799154)
\curveto(172.42189151,702.16798822)(172.4318915,702.21798817)(172.45189779,702.26799154)
\curveto(172.46189147,702.29798809)(172.46689146,702.33298806)(172.46689779,702.37299154)
\lineto(172.46689779,702.50799154)
\lineto(172.46689779,702.64299154)
}
}
{
\newrgbcolor{curcolor}{0 0 0}
\pscustom[linestyle=none,fillstyle=solid,fillcolor=curcolor]
{
\newpath
\moveto(183.13681967,699.44799154)
\curveto(183.15681182,699.33799105)(183.16681181,699.22799116)(183.16681967,699.11799154)
\curveto(183.1768118,699.00799138)(183.12681185,698.93299146)(183.01681967,698.89299154)
\curveto(182.95681202,698.86299153)(182.88681209,698.84799154)(182.80681967,698.84799154)
\lineto(182.56681967,698.84799154)
\lineto(181.75681967,698.84799154)
\lineto(181.48681967,698.84799154)
\curveto(181.40681357,698.85799153)(181.34181363,698.88299151)(181.29181967,698.92299154)
\curveto(181.22181375,698.96299143)(181.16681381,699.01799137)(181.12681967,699.08799154)
\curveto(181.09681388,699.16799122)(181.05181392,699.23299116)(180.99181967,699.28299154)
\curveto(180.971814,699.30299109)(180.94681403,699.31799107)(180.91681967,699.32799154)
\curveto(180.88681409,699.34799104)(180.84681413,699.35299104)(180.79681967,699.34299154)
\curveto(180.74681423,699.32299107)(180.69681428,699.29799109)(180.64681967,699.26799154)
\curveto(180.60681437,699.23799115)(180.56181441,699.21299118)(180.51181967,699.19299154)
\curveto(180.46181451,699.15299124)(180.40681457,699.11799127)(180.34681967,699.08799154)
\lineto(180.16681967,698.99799154)
\curveto(180.03681494,698.93799145)(179.90181507,698.8879915)(179.76181967,698.84799154)
\curveto(179.62181535,698.81799157)(179.4768155,698.78299161)(179.32681967,698.74299154)
\curveto(179.25681572,698.72299167)(179.18681579,698.71299168)(179.11681967,698.71299154)
\curveto(179.05681592,698.70299169)(178.99181598,698.6929917)(178.92181967,698.68299154)
\lineto(178.83181967,698.68299154)
\curveto(178.80181617,698.67299172)(178.7718162,698.66799172)(178.74181967,698.66799154)
\lineto(178.57681967,698.66799154)
\curveto(178.4768165,698.64799174)(178.3768166,698.64799174)(178.27681967,698.66799154)
\lineto(178.14181967,698.66799154)
\curveto(178.0718169,698.6879917)(178.00181697,698.69799169)(177.93181967,698.69799154)
\curveto(177.8718171,698.6879917)(177.81181716,698.6929917)(177.75181967,698.71299154)
\curveto(177.65181732,698.73299166)(177.55681742,698.75299164)(177.46681967,698.77299154)
\curveto(177.3768176,698.78299161)(177.29181768,698.80799158)(177.21181967,698.84799154)
\curveto(176.92181805,698.95799143)(176.6718183,699.09799129)(176.46181967,699.26799154)
\curveto(176.26181871,699.44799094)(176.10181887,699.68299071)(175.98181967,699.97299154)
\curveto(175.95181902,700.04299035)(175.92181905,700.11799027)(175.89181967,700.19799154)
\curveto(175.8718191,700.27799011)(175.85181912,700.36299003)(175.83181967,700.45299154)
\curveto(175.81181916,700.50298989)(175.80181917,700.55298984)(175.80181967,700.60299154)
\curveto(175.81181916,700.65298974)(175.81181916,700.70298969)(175.80181967,700.75299154)
\curveto(175.79181918,700.78298961)(175.78181919,700.84298955)(175.77181967,700.93299154)
\curveto(175.7718192,701.03298936)(175.7768192,701.10298929)(175.78681967,701.14299154)
\curveto(175.80681917,701.24298915)(175.81681916,701.32798906)(175.81681967,701.39799154)
\lineto(175.90681967,701.72799154)
\curveto(175.93681904,701.84798854)(175.976819,701.95298844)(176.02681967,702.04299154)
\curveto(176.19681878,702.33298806)(176.39181858,702.55298784)(176.61181967,702.70299154)
\curveto(176.83181814,702.85298754)(177.11181786,702.98298741)(177.45181967,703.09299154)
\curveto(177.58181739,703.14298725)(177.71681726,703.17798721)(177.85681967,703.19799154)
\curveto(177.99681698,703.21798717)(178.13681684,703.24298715)(178.27681967,703.27299154)
\curveto(178.35681662,703.2929871)(178.44181653,703.30298709)(178.53181967,703.30299154)
\curveto(178.62181635,703.31298708)(178.71181626,703.32798706)(178.80181967,703.34799154)
\curveto(178.8718161,703.36798702)(178.94181603,703.37298702)(179.01181967,703.36299154)
\curveto(179.08181589,703.36298703)(179.15681582,703.37298702)(179.23681967,703.39299154)
\curveto(179.30681567,703.41298698)(179.3768156,703.42298697)(179.44681967,703.42299154)
\curveto(179.51681546,703.42298697)(179.59181538,703.43298696)(179.67181967,703.45299154)
\curveto(179.88181509,703.50298689)(180.0718149,703.54298685)(180.24181967,703.57299154)
\curveto(180.42181455,703.61298678)(180.58181439,703.70298669)(180.72181967,703.84299154)
\curveto(180.81181416,703.93298646)(180.8718141,704.03298636)(180.90181967,704.14299154)
\curveto(180.91181406,704.17298622)(180.91181406,704.19798619)(180.90181967,704.21799154)
\curveto(180.90181407,704.23798615)(180.90681407,704.25798613)(180.91681967,704.27799154)
\curveto(180.92681405,704.29798609)(180.93181404,704.32798606)(180.93181967,704.36799154)
\lineto(180.93181967,704.45799154)
\lineto(180.90181967,704.57799154)
\curveto(180.90181407,704.61798577)(180.89681408,704.65298574)(180.88681967,704.68299154)
\curveto(180.78681419,704.98298541)(180.5768144,705.1879852)(180.25681967,705.29799154)
\curveto(180.16681481,705.32798506)(180.05681492,705.34798504)(179.92681967,705.35799154)
\curveto(179.80681517,705.37798501)(179.68181529,705.38298501)(179.55181967,705.37299154)
\curveto(179.42181555,705.37298502)(179.29681568,705.36298503)(179.17681967,705.34299154)
\curveto(179.05681592,705.32298507)(178.95181602,705.29798509)(178.86181967,705.26799154)
\curveto(178.80181617,705.24798514)(178.74181623,705.21798517)(178.68181967,705.17799154)
\curveto(178.63181634,705.14798524)(178.58181639,705.11298528)(178.53181967,705.07299154)
\curveto(178.48181649,705.03298536)(178.42681655,704.97798541)(178.36681967,704.90799154)
\curveto(178.31681666,704.83798555)(178.28181669,704.77298562)(178.26181967,704.71299154)
\curveto(178.21181676,704.61298578)(178.16681681,704.51798587)(178.12681967,704.42799154)
\curveto(178.09681688,704.33798605)(178.02681695,704.27798611)(177.91681967,704.24799154)
\curveto(177.83681714,704.22798616)(177.75181722,704.21798617)(177.66181967,704.21799154)
\lineto(177.39181967,704.21799154)
\lineto(176.82181967,704.21799154)
\curveto(176.7718182,704.21798617)(176.72181825,704.21298618)(176.67181967,704.20299154)
\curveto(176.62181835,704.20298619)(176.5768184,704.20798618)(176.53681967,704.21799154)
\lineto(176.40181967,704.21799154)
\curveto(176.38181859,704.22798616)(176.35681862,704.23298616)(176.32681967,704.23299154)
\curveto(176.29681868,704.23298616)(176.2718187,704.24298615)(176.25181967,704.26299154)
\curveto(176.1718188,704.28298611)(176.11681886,704.34798604)(176.08681967,704.45799154)
\curveto(176.0768189,704.50798588)(176.0768189,704.55798583)(176.08681967,704.60799154)
\curveto(176.09681888,704.65798573)(176.10681887,704.70298569)(176.11681967,704.74299154)
\curveto(176.14681883,704.85298554)(176.1768188,704.95298544)(176.20681967,705.04299154)
\curveto(176.24681873,705.14298525)(176.29181868,705.23298516)(176.34181967,705.31299154)
\lineto(176.43181967,705.46299154)
\lineto(176.52181967,705.61299154)
\curveto(176.60181837,705.72298467)(176.70181827,705.82798456)(176.82181967,705.92799154)
\curveto(176.84181813,705.93798445)(176.8718181,705.96298443)(176.91181967,706.00299154)
\curveto(176.96181801,706.04298435)(177.00681797,706.07798431)(177.04681967,706.10799154)
\curveto(177.08681789,706.13798425)(177.13181784,706.16798422)(177.18181967,706.19799154)
\curveto(177.35181762,706.30798408)(177.53181744,706.392984)(177.72181967,706.45299154)
\curveto(177.91181706,706.52298387)(178.10681687,706.5879838)(178.30681967,706.64799154)
\curveto(178.42681655,706.67798371)(178.55181642,706.69798369)(178.68181967,706.70799154)
\curveto(178.81181616,706.71798367)(178.94181603,706.73798365)(179.07181967,706.76799154)
\curveto(179.11181586,706.77798361)(179.1718158,706.77798361)(179.25181967,706.76799154)
\curveto(179.34181563,706.75798363)(179.39681558,706.76298363)(179.41681967,706.78299154)
\curveto(179.82681515,706.7929836)(180.21681476,706.77798361)(180.58681967,706.73799154)
\curveto(180.96681401,706.69798369)(181.30681367,706.62298377)(181.60681967,706.51299154)
\curveto(181.91681306,706.40298399)(182.18181279,706.25298414)(182.40181967,706.06299154)
\curveto(182.62181235,705.88298451)(182.79181218,705.64798474)(182.91181967,705.35799154)
\curveto(182.98181199,705.1879852)(183.02181195,704.9929854)(183.03181967,704.77299154)
\curveto(183.04181193,704.55298584)(183.04681193,704.32798606)(183.04681967,704.09799154)
\lineto(183.04681967,700.75299154)
\lineto(183.04681967,700.16799154)
\curveto(183.04681193,699.97799041)(183.06681191,699.80299059)(183.10681967,699.64299154)
\curveto(183.11681186,699.61299078)(183.12181185,699.57799081)(183.12181967,699.53799154)
\curveto(183.12181185,699.50799088)(183.12681185,699.47799091)(183.13681967,699.44799154)
\moveto(180.93181967,701.75799154)
\curveto(180.94181403,701.80798858)(180.94681403,701.86298853)(180.94681967,701.92299154)
\curveto(180.94681403,701.9929884)(180.94181403,702.05298834)(180.93181967,702.10299154)
\curveto(180.91181406,702.16298823)(180.90181407,702.21798817)(180.90181967,702.26799154)
\curveto(180.90181407,702.31798807)(180.88181409,702.35798803)(180.84181967,702.38799154)
\curveto(180.79181418,702.42798796)(180.71681426,702.44798794)(180.61681967,702.44799154)
\curveto(180.5768144,702.43798795)(180.54181443,702.42798796)(180.51181967,702.41799154)
\curveto(180.48181449,702.41798797)(180.44681453,702.41298798)(180.40681967,702.40299154)
\curveto(180.33681464,702.38298801)(180.26181471,702.36798802)(180.18181967,702.35799154)
\curveto(180.10181487,702.34798804)(180.02181495,702.33298806)(179.94181967,702.31299154)
\curveto(179.91181506,702.30298809)(179.86681511,702.29798809)(179.80681967,702.29799154)
\curveto(179.6768153,702.26798812)(179.54681543,702.24798814)(179.41681967,702.23799154)
\curveto(179.28681569,702.22798816)(179.16181581,702.20298819)(179.04181967,702.16299154)
\curveto(178.96181601,702.14298825)(178.88681609,702.12298827)(178.81681967,702.10299154)
\curveto(178.74681623,702.0929883)(178.6768163,702.07298832)(178.60681967,702.04299154)
\curveto(178.39681658,701.95298844)(178.21681676,701.81798857)(178.06681967,701.63799154)
\curveto(177.92681705,701.45798893)(177.8768171,701.20798918)(177.91681967,700.88799154)
\curveto(177.93681704,700.71798967)(177.99181698,700.57798981)(178.08181967,700.46799154)
\curveto(178.15181682,700.35799003)(178.25681672,700.26799012)(178.39681967,700.19799154)
\curveto(178.53681644,700.13799025)(178.68681629,700.0929903)(178.84681967,700.06299154)
\curveto(179.01681596,700.03299036)(179.19181578,700.02299037)(179.37181967,700.03299154)
\curveto(179.56181541,700.05299034)(179.73681524,700.0879903)(179.89681967,700.13799154)
\curveto(180.15681482,700.21799017)(180.36181461,700.34299005)(180.51181967,700.51299154)
\curveto(180.66181431,700.6929897)(180.7768142,700.91298948)(180.85681967,701.17299154)
\curveto(180.8768141,701.24298915)(180.88681409,701.31298908)(180.88681967,701.38299154)
\curveto(180.89681408,701.46298893)(180.91181406,701.54298885)(180.93181967,701.62299154)
\lineto(180.93181967,701.75799154)
}
}
{
\newrgbcolor{curcolor}{0 0 0}
\pscustom[linestyle=none,fillstyle=solid,fillcolor=curcolor]
{
\newpath
\moveto(192.29010092,699.70299154)
\lineto(192.29010092,699.28299154)
\curveto(192.29009255,699.15299124)(192.26009258,699.04799134)(192.20010092,698.96799154)
\curveto(192.15009269,698.91799147)(192.08509275,698.88299151)(192.00510092,698.86299154)
\curveto(191.92509291,698.85299154)(191.835093,698.84799154)(191.73510092,698.84799154)
\lineto(190.91010092,698.84799154)
\lineto(190.62510092,698.84799154)
\curveto(190.54509429,698.85799153)(190.48009436,698.88299151)(190.43010092,698.92299154)
\curveto(190.36009448,698.97299142)(190.32009452,699.03799135)(190.31010092,699.11799154)
\curveto(190.30009454,699.19799119)(190.28009456,699.27799111)(190.25010092,699.35799154)
\curveto(190.23009461,699.37799101)(190.21009463,699.392991)(190.19010092,699.40299154)
\curveto(190.18009466,699.42299097)(190.16509467,699.44299095)(190.14510092,699.46299154)
\curveto(190.0350948,699.46299093)(189.95509488,699.43799095)(189.90510092,699.38799154)
\lineto(189.75510092,699.23799154)
\curveto(189.68509515,699.1879912)(189.62009522,699.14299125)(189.56010092,699.10299154)
\curveto(189.50009534,699.07299132)(189.4350954,699.03299136)(189.36510092,698.98299154)
\curveto(189.32509551,698.96299143)(189.28009556,698.94299145)(189.23010092,698.92299154)
\curveto(189.19009565,698.90299149)(189.14509569,698.88299151)(189.09510092,698.86299154)
\curveto(188.95509588,698.81299158)(188.80509603,698.76799162)(188.64510092,698.72799154)
\curveto(188.59509624,698.70799168)(188.55009629,698.69799169)(188.51010092,698.69799154)
\curveto(188.47009637,698.69799169)(188.43009641,698.6929917)(188.39010092,698.68299154)
\lineto(188.25510092,698.68299154)
\curveto(188.22509661,698.67299172)(188.18509665,698.66799172)(188.13510092,698.66799154)
\lineto(188.00010092,698.66799154)
\curveto(187.9400969,698.64799174)(187.85009699,698.64299175)(187.73010092,698.65299154)
\curveto(187.61009723,698.65299174)(187.52509731,698.66299173)(187.47510092,698.68299154)
\curveto(187.40509743,698.70299169)(187.3400975,698.71299168)(187.28010092,698.71299154)
\curveto(187.23009761,698.70299169)(187.17509766,698.70799168)(187.11510092,698.72799154)
\lineto(186.75510092,698.84799154)
\curveto(186.64509819,698.87799151)(186.5350983,698.91799147)(186.42510092,698.96799154)
\curveto(186.07509876,699.11799127)(185.76009908,699.34799104)(185.48010092,699.65799154)
\curveto(185.21009963,699.97799041)(184.99509984,700.31299008)(184.83510092,700.66299154)
\curveto(184.78510005,700.77298962)(184.74510009,700.87798951)(184.71510092,700.97799154)
\curveto(184.68510015,701.0879893)(184.65010019,701.19798919)(184.61010092,701.30799154)
\curveto(184.60010024,701.34798904)(184.59510024,701.38298901)(184.59510092,701.41299154)
\curveto(184.59510024,701.45298894)(184.58510025,701.49798889)(184.56510092,701.54799154)
\curveto(184.54510029,701.62798876)(184.52510031,701.71298868)(184.50510092,701.80299154)
\curveto(184.49510034,701.90298849)(184.48010036,702.00298839)(184.46010092,702.10299154)
\curveto(184.45010039,702.13298826)(184.44510039,702.16798822)(184.44510092,702.20799154)
\curveto(184.45510038,702.24798814)(184.45510038,702.28298811)(184.44510092,702.31299154)
\lineto(184.44510092,702.44799154)
\curveto(184.44510039,702.49798789)(184.4401004,702.54798784)(184.43010092,702.59799154)
\curveto(184.42010042,702.64798774)(184.41510042,702.70298769)(184.41510092,702.76299154)
\curveto(184.41510042,702.83298756)(184.42010042,702.8879875)(184.43010092,702.92799154)
\curveto(184.4401004,702.97798741)(184.44510039,703.02298737)(184.44510092,703.06299154)
\lineto(184.44510092,703.21299154)
\curveto(184.45510038,703.26298713)(184.45510038,703.30798708)(184.44510092,703.34799154)
\curveto(184.44510039,703.39798699)(184.45510038,703.44798694)(184.47510092,703.49799154)
\curveto(184.49510034,703.60798678)(184.51010033,703.71298668)(184.52010092,703.81299154)
\curveto(184.5401003,703.91298648)(184.56510027,704.01298638)(184.59510092,704.11299154)
\curveto(184.6351002,704.23298616)(184.67010017,704.34798604)(184.70010092,704.45799154)
\curveto(184.73010011,704.56798582)(184.77010007,704.67798571)(184.82010092,704.78799154)
\curveto(184.96009988,705.0879853)(185.1350997,705.37298502)(185.34510092,705.64299154)
\curveto(185.36509947,705.67298472)(185.39009945,705.69798469)(185.42010092,705.71799154)
\curveto(185.46009938,705.74798464)(185.49009935,705.77798461)(185.51010092,705.80799154)
\curveto(185.55009929,705.85798453)(185.59009925,705.90298449)(185.63010092,705.94299154)
\curveto(185.67009917,705.98298441)(185.71509912,706.02298437)(185.76510092,706.06299154)
\curveto(185.80509903,706.08298431)(185.840099,706.10798428)(185.87010092,706.13799154)
\curveto(185.90009894,706.17798421)(185.9350989,706.20798418)(185.97510092,706.22799154)
\curveto(186.22509861,706.39798399)(186.51509832,706.53798385)(186.84510092,706.64799154)
\curveto(186.91509792,706.66798372)(186.98509785,706.68298371)(187.05510092,706.69299154)
\curveto(187.1350977,706.70298369)(187.21509762,706.71798367)(187.29510092,706.73799154)
\curveto(187.36509747,706.75798363)(187.45509738,706.76798362)(187.56510092,706.76799154)
\curveto(187.67509716,706.77798361)(187.78509705,706.78298361)(187.89510092,706.78299154)
\curveto(188.00509683,706.78298361)(188.11009673,706.77798361)(188.21010092,706.76799154)
\curveto(188.32009652,706.75798363)(188.41009643,706.74298365)(188.48010092,706.72299154)
\curveto(188.63009621,706.67298372)(188.77509606,706.62798376)(188.91510092,706.58799154)
\curveto(189.05509578,706.54798384)(189.18509565,706.4929839)(189.30510092,706.42299154)
\curveto(189.37509546,706.37298402)(189.4400954,706.32298407)(189.50010092,706.27299154)
\curveto(189.56009528,706.23298416)(189.62509521,706.1879842)(189.69510092,706.13799154)
\curveto(189.7350951,706.10798428)(189.79009505,706.06798432)(189.86010092,706.01799154)
\curveto(189.9400949,705.96798442)(190.01509482,705.96798442)(190.08510092,706.01799154)
\curveto(190.12509471,706.03798435)(190.14509469,706.07298432)(190.14510092,706.12299154)
\curveto(190.14509469,706.17298422)(190.15509468,706.22298417)(190.17510092,706.27299154)
\lineto(190.17510092,706.42299154)
\curveto(190.18509465,706.45298394)(190.19009465,706.4879839)(190.19010092,706.52799154)
\lineto(190.19010092,706.64799154)
\lineto(190.19010092,708.68799154)
\curveto(190.19009465,708.79798159)(190.18509465,708.91798147)(190.17510092,709.04799154)
\curveto(190.17509466,709.1879812)(190.20009464,709.2929811)(190.25010092,709.36299154)
\curveto(190.29009455,709.44298095)(190.36509447,709.4929809)(190.47510092,709.51299154)
\curveto(190.49509434,709.52298087)(190.51509432,709.52298087)(190.53510092,709.51299154)
\curveto(190.55509428,709.51298088)(190.57509426,709.51798087)(190.59510092,709.52799154)
\lineto(191.66010092,709.52799154)
\curveto(191.78009306,709.52798086)(191.89009295,709.52298087)(191.99010092,709.51299154)
\curveto(192.09009275,709.50298089)(192.16509267,709.46298093)(192.21510092,709.39299154)
\curveto(192.26509257,709.31298108)(192.29009255,709.20798118)(192.29010092,709.07799154)
\lineto(192.29010092,708.71799154)
\lineto(192.29010092,699.70299154)
\moveto(190.25010092,702.64299154)
\curveto(190.26009458,702.68298771)(190.26009458,702.72298767)(190.25010092,702.76299154)
\lineto(190.25010092,702.89799154)
\curveto(190.25009459,702.99798739)(190.24509459,703.09798729)(190.23510092,703.19799154)
\curveto(190.22509461,703.29798709)(190.21009463,703.387987)(190.19010092,703.46799154)
\curveto(190.17009467,703.57798681)(190.15009469,703.67798671)(190.13010092,703.76799154)
\curveto(190.12009472,703.85798653)(190.09509474,703.94298645)(190.05510092,704.02299154)
\curveto(189.91509492,704.38298601)(189.71009513,704.66798572)(189.44010092,704.87799154)
\curveto(189.18009566,705.0879853)(188.80009604,705.1929852)(188.30010092,705.19299154)
\curveto(188.2400966,705.1929852)(188.16009668,705.18298521)(188.06010092,705.16299154)
\curveto(187.98009686,705.14298525)(187.90509693,705.12298527)(187.83510092,705.10299154)
\curveto(187.77509706,705.0929853)(187.71509712,705.07298532)(187.65510092,705.04299154)
\curveto(187.38509745,704.93298546)(187.17509766,704.76298563)(187.02510092,704.53299154)
\curveto(186.87509796,704.30298609)(186.75509808,704.04298635)(186.66510092,703.75299154)
\curveto(186.6350982,703.65298674)(186.61509822,703.55298684)(186.60510092,703.45299154)
\curveto(186.59509824,703.35298704)(186.57509826,703.24798714)(186.54510092,703.13799154)
\lineto(186.54510092,702.92799154)
\curveto(186.52509831,702.83798755)(186.52009832,702.71298768)(186.53010092,702.55299154)
\curveto(186.5400983,702.40298799)(186.55509828,702.2929881)(186.57510092,702.22299154)
\lineto(186.57510092,702.13299154)
\curveto(186.58509825,702.11298828)(186.59009825,702.0929883)(186.59010092,702.07299154)
\curveto(186.61009823,701.9929884)(186.62509821,701.91798847)(186.63510092,701.84799154)
\curveto(186.65509818,701.77798861)(186.67509816,701.70298869)(186.69510092,701.62299154)
\curveto(186.86509797,701.10298929)(187.15509768,700.71798967)(187.56510092,700.46799154)
\curveto(187.69509714,700.37799001)(187.87509696,700.30799008)(188.10510092,700.25799154)
\curveto(188.14509669,700.24799014)(188.20509663,700.24299015)(188.28510092,700.24299154)
\curveto(188.31509652,700.23299016)(188.36009648,700.22299017)(188.42010092,700.21299154)
\curveto(188.49009635,700.21299018)(188.54509629,700.21799017)(188.58510092,700.22799154)
\curveto(188.66509617,700.24799014)(188.74509609,700.26299013)(188.82510092,700.27299154)
\curveto(188.90509593,700.28299011)(188.98509585,700.30299009)(189.06510092,700.33299154)
\curveto(189.31509552,700.44298995)(189.51509532,700.58298981)(189.66510092,700.75299154)
\curveto(189.81509502,700.92298947)(189.94509489,701.13798925)(190.05510092,701.39799154)
\curveto(190.09509474,701.4879889)(190.12509471,701.57798881)(190.14510092,701.66799154)
\curveto(190.16509467,701.76798862)(190.18509465,701.87298852)(190.20510092,701.98299154)
\curveto(190.21509462,702.03298836)(190.21509462,702.07798831)(190.20510092,702.11799154)
\curveto(190.20509463,702.16798822)(190.21509462,702.21798817)(190.23510092,702.26799154)
\curveto(190.24509459,702.29798809)(190.25009459,702.33298806)(190.25010092,702.37299154)
\lineto(190.25010092,702.50799154)
\lineto(190.25010092,702.64299154)
}
}
{
\newrgbcolor{curcolor}{0 0 0}
\pscustom[linestyle=none,fillstyle=solid,fillcolor=curcolor]
{
\newpath
\moveto(832.8825815,694.76149618)
\curveto(832.95257976,694.76148552)(833.03257968,694.76148552)(833.1225815,694.76149618)
\curveto(833.2125795,694.77148551)(833.29757942,694.77148551)(833.3775815,694.76149618)
\curveto(833.46757925,694.76148552)(833.54757917,694.75148553)(833.6175815,694.73149618)
\curveto(833.68757903,694.71148557)(833.73757898,694.6814856)(833.7675815,694.64149618)
\curveto(833.82757889,694.57148571)(833.85757886,694.47148581)(833.8575815,694.34149618)
\curveto(833.86757885,694.22148606)(833.87257884,694.09648618)(833.8725815,693.96649618)
\lineto(833.8725815,692.51149618)
\lineto(833.8725815,686.72149618)
\lineto(833.8725815,684.96649618)
\lineto(833.8725815,684.54649618)
\curveto(833.87257884,684.40649587)(833.84757887,684.29649598)(833.7975815,684.21649618)
\curveto(833.75757896,684.16649611)(833.70757901,684.13649614)(833.6475815,684.12649618)
\curveto(833.59757912,684.11649616)(833.53257918,684.10149618)(833.4525815,684.08149618)
\lineto(833.1675815,684.08149618)
\curveto(833.02757969,684.0814962)(832.89757982,684.08649619)(832.7775815,684.09649618)
\curveto(832.65758006,684.10649617)(832.57258014,684.15649612)(832.5225815,684.24649618)
\curveto(832.48258023,684.30649597)(832.46258025,684.38649589)(832.4625815,684.48649618)
\lineto(832.4625815,684.81649618)
\lineto(832.4625815,686.01649618)
\lineto(832.4625815,692.28649618)
\lineto(832.4625815,693.90649618)
\curveto(832.46258025,694.01648626)(832.45758026,694.13648614)(832.4475815,694.26649618)
\curveto(832.44758027,694.40648587)(832.47258024,694.51648576)(832.5225815,694.59649618)
\curveto(832.56258015,694.66648561)(832.64258007,694.71648556)(832.7625815,694.74649618)
\curveto(832.78257993,694.75648552)(832.80257991,694.75648552)(832.8225815,694.74649618)
\curveto(832.84257987,694.74648553)(832.86257985,694.75148553)(832.8825815,694.76149618)
}
}
{
\newrgbcolor{curcolor}{0 0 0}
\pscustom[linestyle=none,fillstyle=solid,fillcolor=curcolor]
{
\newpath
\moveto(839.71906587,691.98649618)
\curveto(840.34906064,692.00648827)(840.85406013,691.92148836)(841.23406587,691.73149618)
\curveto(841.61405937,691.54148874)(841.91905907,691.25648902)(842.14906587,690.87649618)
\curveto(842.20905878,690.7764895)(842.25405873,690.66648961)(842.28406587,690.54649618)
\curveto(842.32405866,690.43648984)(842.35905863,690.32148996)(842.38906587,690.20149618)
\curveto(842.43905855,690.01149027)(842.46905852,689.80649047)(842.47906587,689.58649618)
\curveto(842.4890585,689.36649091)(842.49405849,689.14149114)(842.49406587,688.91149618)
\lineto(842.49406587,687.30649618)
\lineto(842.49406587,684.96649618)
\curveto(842.49405849,684.79649548)(842.4890585,684.62649565)(842.47906587,684.45649618)
\curveto(842.47905851,684.28649599)(842.41405857,684.1764961)(842.28406587,684.12649618)
\curveto(842.23405875,684.10649617)(842.17905881,684.09649618)(842.11906587,684.09649618)
\curveto(842.06905892,684.08649619)(842.01405897,684.0814962)(841.95406587,684.08149618)
\curveto(841.82405916,684.0814962)(841.69905929,684.08649619)(841.57906587,684.09649618)
\curveto(841.45905953,684.09649618)(841.37405961,684.13649614)(841.32406587,684.21649618)
\curveto(841.27405971,684.28649599)(841.24905974,684.3764959)(841.24906587,684.48649618)
\lineto(841.24906587,684.81649618)
\lineto(841.24906587,686.10649618)
\lineto(841.24906587,688.55149618)
\curveto(841.24905974,688.82149146)(841.24405974,689.08649119)(841.23406587,689.34649618)
\curveto(841.22405976,689.61649066)(841.17905981,689.84649043)(841.09906587,690.03649618)
\curveto(841.01905997,690.23649004)(840.89906009,690.39648988)(840.73906587,690.51649618)
\curveto(840.57906041,690.64648963)(840.39406059,690.74648953)(840.18406587,690.81649618)
\curveto(840.12406086,690.83648944)(840.05906093,690.84648943)(839.98906587,690.84649618)
\curveto(839.92906106,690.85648942)(839.86906112,690.87148941)(839.80906587,690.89149618)
\curveto(839.75906123,690.90148938)(839.67906131,690.90148938)(839.56906587,690.89149618)
\curveto(839.46906152,690.89148939)(839.39906159,690.88648939)(839.35906587,690.87649618)
\curveto(839.31906167,690.85648942)(839.2840617,690.84648943)(839.25406587,690.84649618)
\curveto(839.22406176,690.85648942)(839.1890618,690.85648942)(839.14906587,690.84649618)
\curveto(839.01906197,690.81648946)(838.89406209,690.7814895)(838.77406587,690.74149618)
\curveto(838.66406232,690.71148957)(838.55906243,690.66648961)(838.45906587,690.60649618)
\curveto(838.41906257,690.58648969)(838.3840626,690.56648971)(838.35406587,690.54649618)
\curveto(838.32406266,690.52648975)(838.2890627,690.50648977)(838.24906587,690.48649618)
\curveto(837.89906309,690.23649004)(837.64406334,689.86149042)(837.48406587,689.36149618)
\curveto(837.45406353,689.281491)(837.43406355,689.19649108)(837.42406587,689.10649618)
\curveto(837.41406357,689.02649125)(837.39906359,688.94649133)(837.37906587,688.86649618)
\curveto(837.35906363,688.81649146)(837.35406363,688.76649151)(837.36406587,688.71649618)
\curveto(837.37406361,688.6764916)(837.36906362,688.63649164)(837.34906587,688.59649618)
\lineto(837.34906587,688.28149618)
\curveto(837.33906365,688.25149203)(837.33406365,688.21649206)(837.33406587,688.17649618)
\curveto(837.34406364,688.13649214)(837.34906364,688.09149219)(837.34906587,688.04149618)
\lineto(837.34906587,687.59149618)
\lineto(837.34906587,686.15149618)
\lineto(837.34906587,684.83149618)
\lineto(837.34906587,684.48649618)
\curveto(837.34906364,684.3764959)(837.32406366,684.28649599)(837.27406587,684.21649618)
\curveto(837.22406376,684.13649614)(837.13406385,684.09649618)(837.00406587,684.09649618)
\curveto(836.8840641,684.08649619)(836.75906423,684.0814962)(836.62906587,684.08149618)
\curveto(836.54906444,684.0814962)(836.47406451,684.08649619)(836.40406587,684.09649618)
\curveto(836.33406465,684.10649617)(836.27406471,684.13149615)(836.22406587,684.17149618)
\curveto(836.14406484,684.22149606)(836.10406488,684.31649596)(836.10406587,684.45649618)
\lineto(836.10406587,684.86149618)
\lineto(836.10406587,686.63149618)
\lineto(836.10406587,690.26149618)
\lineto(836.10406587,691.17649618)
\lineto(836.10406587,691.44649618)
\curveto(836.10406488,691.53648874)(836.12406486,691.60648867)(836.16406587,691.65649618)
\curveto(836.19406479,691.71648856)(836.24406474,691.75648852)(836.31406587,691.77649618)
\curveto(836.35406463,691.78648849)(836.40906458,691.79648848)(836.47906587,691.80649618)
\curveto(836.55906443,691.81648846)(836.63906435,691.82148846)(836.71906587,691.82149618)
\curveto(836.79906419,691.82148846)(836.87406411,691.81648846)(836.94406587,691.80649618)
\curveto(837.02406396,691.79648848)(837.07906391,691.7814885)(837.10906587,691.76149618)
\curveto(837.21906377,691.69148859)(837.26906372,691.60148868)(837.25906587,691.49149618)
\curveto(837.24906374,691.39148889)(837.26406372,691.276489)(837.30406587,691.14649618)
\curveto(837.32406366,691.08648919)(837.36406362,691.03648924)(837.42406587,690.99649618)
\curveto(837.54406344,690.98648929)(837.63906335,691.03148925)(837.70906587,691.13149618)
\curveto(837.7890632,691.23148905)(837.86906312,691.31148897)(837.94906587,691.37149618)
\curveto(838.0890629,691.47148881)(838.22906276,691.56148872)(838.36906587,691.64149618)
\curveto(838.51906247,691.73148855)(838.6890623,691.80648847)(838.87906587,691.86649618)
\curveto(838.95906203,691.89648838)(839.04406194,691.91648836)(839.13406587,691.92649618)
\curveto(839.23406175,691.93648834)(839.32906166,691.95148833)(839.41906587,691.97149618)
\curveto(839.46906152,691.9814883)(839.51906147,691.98648829)(839.56906587,691.98649618)
\lineto(839.71906587,691.98649618)
}
}
{
\newrgbcolor{curcolor}{0 0 0}
\pscustom[linestyle=none,fillstyle=solid,fillcolor=curcolor]
{
\newpath
\moveto(844.15367525,691.83649618)
\lineto(844.63367525,691.83649618)
\curveto(844.80367391,691.83648844)(844.93367378,691.80648847)(845.02367525,691.74649618)
\curveto(845.09367362,691.69648858)(845.13867357,691.63148865)(845.15867525,691.55149618)
\curveto(845.18867352,691.4814888)(845.21867349,691.40648887)(845.24867525,691.32649618)
\curveto(845.3086734,691.18648909)(845.35867335,691.04648923)(845.39867525,690.90649618)
\curveto(845.43867327,690.76648951)(845.48367323,690.62648965)(845.53367525,690.48649618)
\curveto(845.73367298,689.94649033)(845.91867279,689.40149088)(846.08867525,688.85149618)
\curveto(846.25867245,688.31149197)(846.44367227,687.77149251)(846.64367525,687.23149618)
\curveto(846.713672,687.05149323)(846.77367194,686.86649341)(846.82367525,686.67649618)
\curveto(846.87367184,686.49649378)(846.93867177,686.31649396)(847.01867525,686.13649618)
\curveto(847.03867167,686.06649421)(847.06367165,685.99149429)(847.09367525,685.91149618)
\curveto(847.12367159,685.83149445)(847.17367154,685.7814945)(847.24367525,685.76149618)
\curveto(847.32367139,685.74149454)(847.38367133,685.7764945)(847.42367525,685.86649618)
\curveto(847.47367124,685.95649432)(847.5086712,686.02649425)(847.52867525,686.07649618)
\curveto(847.6086711,686.26649401)(847.67367104,686.45649382)(847.72367525,686.64649618)
\curveto(847.78367093,686.84649343)(847.84867086,687.04649323)(847.91867525,687.24649618)
\curveto(848.04867066,687.62649265)(848.17367054,688.00149228)(848.29367525,688.37149618)
\curveto(848.4136703,688.75149153)(848.53867017,689.13149115)(848.66867525,689.51149618)
\curveto(848.71866999,689.6814906)(848.76866994,689.84649043)(848.81867525,690.00649618)
\curveto(848.86866984,690.1764901)(848.92866978,690.34148994)(848.99867525,690.50149618)
\curveto(849.04866966,690.64148964)(849.09366962,690.7814895)(849.13367525,690.92149618)
\curveto(849.17366954,691.06148922)(849.21866949,691.20148908)(849.26867525,691.34149618)
\curveto(849.28866942,691.41148887)(849.3136694,691.4814888)(849.34367525,691.55149618)
\curveto(849.37366934,691.62148866)(849.4136693,691.6814886)(849.46367525,691.73149618)
\curveto(849.54366917,691.7814885)(849.63366908,691.81148847)(849.73367525,691.82149618)
\curveto(849.83366888,691.83148845)(849.95366876,691.83648844)(850.09367525,691.83649618)
\curveto(850.16366855,691.83648844)(850.22866848,691.83148845)(850.28867525,691.82149618)
\curveto(850.34866836,691.82148846)(850.40366831,691.81148847)(850.45367525,691.79149618)
\curveto(850.54366817,691.75148853)(850.58866812,691.68648859)(850.58867525,691.59649618)
\curveto(850.59866811,691.50648877)(850.58366813,691.41648886)(850.54367525,691.32649618)
\curveto(850.48366823,691.15648912)(850.42366829,690.9814893)(850.36367525,690.80149618)
\curveto(850.30366841,690.62148966)(850.23366848,690.44648983)(850.15367525,690.27649618)
\curveto(850.13366858,690.22649005)(850.11866859,690.1764901)(850.10867525,690.12649618)
\curveto(850.09866861,690.08649019)(850.08366863,690.04149024)(850.06367525,689.99149618)
\curveto(849.98366873,689.82149046)(849.91866879,689.64649063)(849.86867525,689.46649618)
\curveto(849.81866889,689.28649099)(849.75366896,689.10649117)(849.67367525,688.92649618)
\curveto(849.62366909,688.79649148)(849.57366914,688.66149162)(849.52367525,688.52149618)
\curveto(849.48366923,688.39149189)(849.43366928,688.26149202)(849.37367525,688.13149618)
\curveto(849.20366951,687.72149256)(849.04866966,687.30649297)(848.90867525,686.88649618)
\curveto(848.77866993,686.46649381)(848.62867008,686.05149423)(848.45867525,685.64149618)
\curveto(848.39867031,685.4814948)(848.34367037,685.32149496)(848.29367525,685.16149618)
\curveto(848.24367047,685.00149528)(848.18367053,684.84149544)(848.11367525,684.68149618)
\curveto(848.06367065,684.57149571)(848.01867069,684.46649581)(847.97867525,684.36649618)
\curveto(847.94867076,684.276496)(847.87867083,684.20649607)(847.76867525,684.15649618)
\curveto(847.708671,684.12649615)(847.63867107,684.11149617)(847.55867525,684.11149618)
\lineto(847.33367525,684.11149618)
\lineto(846.86867525,684.11149618)
\curveto(846.71867199,684.12149616)(846.6086721,684.17149611)(846.53867525,684.26149618)
\curveto(846.46867224,684.34149594)(846.41867229,684.43649584)(846.38867525,684.54649618)
\curveto(846.35867235,684.66649561)(846.31867239,684.7814955)(846.26867525,684.89149618)
\curveto(846.2086725,685.03149525)(846.14867256,685.1764951)(846.08867525,685.32649618)
\curveto(846.03867267,685.48649479)(845.98867272,685.63649464)(845.93867525,685.77649618)
\curveto(845.91867279,685.82649445)(845.90367281,685.86649441)(845.89367525,685.89649618)
\curveto(845.88367283,685.93649434)(845.86867284,685.9814943)(845.84867525,686.03149618)
\curveto(845.64867306,686.51149377)(845.46367325,686.99649328)(845.29367525,687.48649618)
\curveto(845.13367358,687.9764923)(844.95367376,688.46149182)(844.75367525,688.94149618)
\curveto(844.69367402,689.10149118)(844.63367408,689.25649102)(844.57367525,689.40649618)
\curveto(844.52367419,689.56649071)(844.46867424,689.72649055)(844.40867525,689.88649618)
\lineto(844.34867525,690.03649618)
\curveto(844.33867437,690.09649018)(844.32367439,690.15149013)(844.30367525,690.20149618)
\curveto(844.22367449,690.37148991)(844.15367456,690.54148974)(844.09367525,690.71149618)
\curveto(844.04367467,690.8814894)(843.98367473,691.05148923)(843.91367525,691.22149618)
\curveto(843.89367482,691.281489)(843.86867484,691.36148892)(843.83867525,691.46149618)
\curveto(843.8086749,691.56148872)(843.8136749,691.64648863)(843.85367525,691.71649618)
\curveto(843.90367481,691.76648851)(843.96367475,691.80148848)(844.03367525,691.82149618)
\curveto(844.10367461,691.82148846)(844.14367457,691.82648845)(844.15367525,691.83649618)
}
}
{
\newrgbcolor{curcolor}{0 0 0}
\pscustom[linestyle=none,fillstyle=solid,fillcolor=curcolor]
{
\newpath
\moveto(852.16367525,693.33649618)
\curveto(852.08367413,693.39648688)(852.03867417,693.50148678)(852.02867525,693.65149618)
\lineto(852.02867525,694.11649618)
\lineto(852.02867525,694.37149618)
\curveto(852.02867418,694.46148582)(852.04367417,694.53648574)(852.07367525,694.59649618)
\curveto(852.1136741,694.6764856)(852.19367402,694.73648554)(852.31367525,694.77649618)
\curveto(852.33367388,694.78648549)(852.35367386,694.78648549)(852.37367525,694.77649618)
\curveto(852.40367381,694.7764855)(852.42867378,694.7814855)(852.44867525,694.79149618)
\curveto(852.61867359,694.79148549)(852.77867343,694.78648549)(852.92867525,694.77649618)
\curveto(853.07867313,694.76648551)(853.17867303,694.70648557)(853.22867525,694.59649618)
\curveto(853.25867295,694.53648574)(853.27367294,694.46148582)(853.27367525,694.37149618)
\lineto(853.27367525,694.11649618)
\curveto(853.27367294,693.93648634)(853.26867294,693.76648651)(853.25867525,693.60649618)
\curveto(853.25867295,693.44648683)(853.19367302,693.34148694)(853.06367525,693.29149618)
\curveto(853.0136732,693.27148701)(852.95867325,693.26148702)(852.89867525,693.26149618)
\lineto(852.73367525,693.26149618)
\lineto(852.41867525,693.26149618)
\curveto(852.31867389,693.26148702)(852.23367398,693.28648699)(852.16367525,693.33649618)
\moveto(853.27367525,684.83149618)
\lineto(853.27367525,684.51649618)
\curveto(853.28367293,684.41649586)(853.26367295,684.33649594)(853.21367525,684.27649618)
\curveto(853.18367303,684.21649606)(853.13867307,684.1764961)(853.07867525,684.15649618)
\curveto(853.01867319,684.14649613)(852.94867326,684.13149615)(852.86867525,684.11149618)
\lineto(852.64367525,684.11149618)
\curveto(852.5136737,684.11149617)(852.39867381,684.11649616)(852.29867525,684.12649618)
\curveto(852.208674,684.14649613)(852.13867407,684.19649608)(852.08867525,684.27649618)
\curveto(852.04867416,684.33649594)(852.02867418,684.41149587)(852.02867525,684.50149618)
\lineto(852.02867525,684.78649618)
\lineto(852.02867525,691.13149618)
\lineto(852.02867525,691.44649618)
\curveto(852.02867418,691.55648872)(852.05367416,691.64148864)(852.10367525,691.70149618)
\curveto(852.13367408,691.75148853)(852.17367404,691.7814885)(852.22367525,691.79149618)
\curveto(852.27367394,691.80148848)(852.32867388,691.81648846)(852.38867525,691.83649618)
\curveto(852.4086738,691.83648844)(852.42867378,691.83148845)(852.44867525,691.82149618)
\curveto(852.47867373,691.82148846)(852.50367371,691.82648845)(852.52367525,691.83649618)
\curveto(852.65367356,691.83648844)(852.78367343,691.83148845)(852.91367525,691.82149618)
\curveto(853.05367316,691.82148846)(853.14867306,691.7814885)(853.19867525,691.70149618)
\curveto(853.24867296,691.64148864)(853.27367294,691.56148872)(853.27367525,691.46149618)
\lineto(853.27367525,691.17649618)
\lineto(853.27367525,684.83149618)
}
}
{
\newrgbcolor{curcolor}{0 0 0}
\pscustom[linestyle=none,fillstyle=solid,fillcolor=curcolor]
{
\newpath
\moveto(856.163519,694.17649618)
\curveto(856.31351699,694.1764861)(856.46351684,694.17148611)(856.613519,694.16149618)
\curveto(856.76351654,694.16148612)(856.86851643,694.12148616)(856.928519,694.04149618)
\curveto(856.97851632,693.9814863)(857.0035163,693.89648638)(857.003519,693.78649618)
\curveto(857.01351629,693.68648659)(857.01851628,693.5814867)(857.018519,693.47149618)
\lineto(857.018519,692.60149618)
\curveto(857.01851628,692.52148776)(857.01351629,692.43648784)(857.003519,692.34649618)
\curveto(857.0035163,692.26648801)(857.01351629,692.19648808)(857.033519,692.13649618)
\curveto(857.07351623,691.99648828)(857.16351614,691.90648837)(857.303519,691.86649618)
\curveto(857.35351595,691.85648842)(857.3985159,691.85148843)(857.438519,691.85149618)
\lineto(857.588519,691.85149618)
\lineto(857.993519,691.85149618)
\curveto(858.15351515,691.86148842)(858.26851503,691.85148843)(858.338519,691.82149618)
\curveto(858.42851487,691.76148852)(858.48851481,691.70148858)(858.518519,691.64149618)
\curveto(858.53851476,691.60148868)(858.54851475,691.55648872)(858.548519,691.50649618)
\lineto(858.548519,691.35649618)
\curveto(858.54851475,691.24648903)(858.54351476,691.14148914)(858.533519,691.04149618)
\curveto(858.52351478,690.95148933)(858.48851481,690.8814894)(858.428519,690.83149618)
\curveto(858.36851493,690.7814895)(858.28351502,690.75148953)(858.173519,690.74149618)
\lineto(857.843519,690.74149618)
\curveto(857.73351557,690.75148953)(857.62351568,690.75648952)(857.513519,690.75649618)
\curveto(857.4035159,690.75648952)(857.30851599,690.74148954)(857.228519,690.71149618)
\curveto(857.15851614,690.6814896)(857.10851619,690.63148965)(857.078519,690.56149618)
\curveto(857.04851625,690.49148979)(857.02851627,690.40648987)(857.018519,690.30649618)
\curveto(857.00851629,690.21649006)(857.0035163,690.11649016)(857.003519,690.00649618)
\curveto(857.01351629,689.90649037)(857.01851628,689.80649047)(857.018519,689.70649618)
\lineto(857.018519,686.73649618)
\curveto(857.01851628,686.51649376)(857.01351629,686.281494)(857.003519,686.03149618)
\curveto(857.0035163,685.79149449)(857.04851625,685.60649467)(857.138519,685.47649618)
\curveto(857.18851611,685.39649488)(857.25351605,685.34149494)(857.333519,685.31149618)
\curveto(857.41351589,685.281495)(857.50851579,685.25649502)(857.618519,685.23649618)
\curveto(857.64851565,685.22649505)(857.67851562,685.22149506)(857.708519,685.22149618)
\curveto(857.74851555,685.23149505)(857.78351552,685.23149505)(857.813519,685.22149618)
\lineto(858.008519,685.22149618)
\curveto(858.10851519,685.22149506)(858.1985151,685.21149507)(858.278519,685.19149618)
\curveto(858.36851493,685.1814951)(858.43351487,685.14649513)(858.473519,685.08649618)
\curveto(858.49351481,685.05649522)(858.50851479,685.00149528)(858.518519,684.92149618)
\curveto(858.53851476,684.85149543)(858.54851475,684.7764955)(858.548519,684.69649618)
\curveto(858.55851474,684.61649566)(858.55851474,684.53649574)(858.548519,684.45649618)
\curveto(858.53851476,684.38649589)(858.51851478,684.33149595)(858.488519,684.29149618)
\curveto(858.44851485,684.22149606)(858.37351493,684.17149611)(858.263519,684.14149618)
\curveto(858.18351512,684.12149616)(858.09351521,684.11149617)(857.993519,684.11149618)
\curveto(857.89351541,684.12149616)(857.8035155,684.12649615)(857.723519,684.12649618)
\curveto(857.66351564,684.12649615)(857.6035157,684.12149616)(857.543519,684.11149618)
\curveto(857.48351582,684.11149617)(857.42851587,684.11649616)(857.378519,684.12649618)
\lineto(857.198519,684.12649618)
\curveto(857.14851615,684.13649614)(857.0985162,684.14149614)(857.048519,684.14149618)
\curveto(857.00851629,684.15149613)(856.96351634,684.15649612)(856.913519,684.15649618)
\curveto(856.71351659,684.20649607)(856.53851676,684.26149602)(856.388519,684.32149618)
\curveto(856.24851705,684.3814959)(856.12851717,684.48649579)(856.028519,684.63649618)
\curveto(855.88851741,684.83649544)(855.80851749,685.08649519)(855.788519,685.38649618)
\curveto(855.76851753,685.69649458)(855.75851754,686.02649425)(855.758519,686.37649618)
\lineto(855.758519,690.30649618)
\curveto(855.72851757,690.43648984)(855.6985176,690.53148975)(855.668519,690.59149618)
\curveto(855.64851765,690.65148963)(855.57851772,690.70148958)(855.458519,690.74149618)
\curveto(855.41851788,690.75148953)(855.37851792,690.75148953)(855.338519,690.74149618)
\curveto(855.298518,690.73148955)(855.25851804,690.73648954)(855.218519,690.75649618)
\lineto(854.978519,690.75649618)
\curveto(854.84851845,690.75648952)(854.73851856,690.76648951)(854.648519,690.78649618)
\curveto(854.56851873,690.81648946)(854.51351879,690.8764894)(854.483519,690.96649618)
\curveto(854.46351884,691.00648927)(854.44851885,691.05148923)(854.438519,691.10149618)
\lineto(854.438519,691.25149618)
\curveto(854.43851886,691.39148889)(854.44851885,691.50648877)(854.468519,691.59649618)
\curveto(854.48851881,691.69648858)(854.54851875,691.77148851)(854.648519,691.82149618)
\curveto(854.75851854,691.86148842)(854.8985184,691.87148841)(855.068519,691.85149618)
\curveto(855.24851805,691.83148845)(855.3985179,691.84148844)(855.518519,691.88149618)
\curveto(855.60851769,691.93148835)(855.67851762,692.00148828)(855.728519,692.09149618)
\curveto(855.74851755,692.15148813)(855.75851754,692.22648805)(855.758519,692.31649618)
\lineto(855.758519,692.57149618)
\lineto(855.758519,693.50149618)
\lineto(855.758519,693.74149618)
\curveto(855.75851754,693.83148645)(855.76851753,693.90648637)(855.788519,693.96649618)
\curveto(855.82851747,694.04648623)(855.9035174,694.11148617)(856.013519,694.16149618)
\curveto(856.04351726,694.16148612)(856.06851723,694.16148612)(856.088519,694.16149618)
\curveto(856.11851718,694.17148611)(856.14351716,694.1764861)(856.163519,694.17649618)
}
}
{
\newrgbcolor{curcolor}{0 0 0}
\pscustom[linestyle=none,fillstyle=solid,fillcolor=curcolor]
{
\newpath
\moveto(866.82031587,684.66649618)
\curveto(866.85030804,684.50649577)(866.83530806,684.37149591)(866.77531587,684.26149618)
\curveto(866.71530818,684.16149612)(866.63530826,684.08649619)(866.53531587,684.03649618)
\curveto(866.48530841,684.01649626)(866.43030846,684.00649627)(866.37031587,684.00649618)
\curveto(866.32030857,684.00649627)(866.26530863,683.99649628)(866.20531587,683.97649618)
\curveto(865.98530891,683.92649635)(865.76530913,683.94149634)(865.54531587,684.02149618)
\curveto(865.33530956,684.09149619)(865.1903097,684.1814961)(865.11031587,684.29149618)
\curveto(865.06030983,684.36149592)(865.01530988,684.44149584)(864.97531587,684.53149618)
\curveto(864.93530996,684.63149565)(864.88531001,684.71149557)(864.82531587,684.77149618)
\curveto(864.80531009,684.79149549)(864.78031011,684.81149547)(864.75031587,684.83149618)
\curveto(864.73031016,684.85149543)(864.70031019,684.85649542)(864.66031587,684.84649618)
\curveto(864.55031034,684.81649546)(864.44531045,684.76149552)(864.34531587,684.68149618)
\curveto(864.25531064,684.60149568)(864.16531073,684.53149575)(864.07531587,684.47149618)
\curveto(863.94531095,684.39149589)(863.80531109,684.31649596)(863.65531587,684.24649618)
\curveto(863.50531139,684.18649609)(863.34531155,684.13149615)(863.17531587,684.08149618)
\curveto(863.07531182,684.05149623)(862.96531193,684.03149625)(862.84531587,684.02149618)
\curveto(862.73531216,684.01149627)(862.62531227,683.99649628)(862.51531587,683.97649618)
\curveto(862.46531243,683.96649631)(862.42031247,683.96149632)(862.38031587,683.96149618)
\lineto(862.27531587,683.96149618)
\curveto(862.16531273,683.94149634)(862.06031283,683.94149634)(861.96031587,683.96149618)
\lineto(861.82531587,683.96149618)
\curveto(861.77531312,683.97149631)(861.72531317,683.9764963)(861.67531587,683.97649618)
\curveto(861.62531327,683.9764963)(861.58031331,683.98649629)(861.54031587,684.00649618)
\curveto(861.50031339,684.01649626)(861.46531343,684.02149626)(861.43531587,684.02149618)
\curveto(861.41531348,684.01149627)(861.3903135,684.01149627)(861.36031587,684.02149618)
\lineto(861.12031587,684.08149618)
\curveto(861.04031385,684.09149619)(860.96531393,684.11149617)(860.89531587,684.14149618)
\curveto(860.5953143,684.27149601)(860.35031454,684.41649586)(860.16031587,684.57649618)
\curveto(859.98031491,684.74649553)(859.83031506,684.9814953)(859.71031587,685.28149618)
\curveto(859.62031527,685.50149478)(859.57531532,685.76649451)(859.57531587,686.07649618)
\lineto(859.57531587,686.39149618)
\curveto(859.58531531,686.44149384)(859.5903153,686.49149379)(859.59031587,686.54149618)
\lineto(859.62031587,686.72149618)
\lineto(859.74031587,687.05149618)
\curveto(859.78031511,687.16149312)(859.83031506,687.26149302)(859.89031587,687.35149618)
\curveto(860.07031482,687.64149264)(860.31531458,687.85649242)(860.62531587,687.99649618)
\curveto(860.93531396,688.13649214)(861.27531362,688.26149202)(861.64531587,688.37149618)
\curveto(861.78531311,688.41149187)(861.93031296,688.44149184)(862.08031587,688.46149618)
\curveto(862.23031266,688.4814918)(862.38031251,688.50649177)(862.53031587,688.53649618)
\curveto(862.60031229,688.55649172)(862.66531223,688.56649171)(862.72531587,688.56649618)
\curveto(862.7953121,688.56649171)(862.87031202,688.5764917)(862.95031587,688.59649618)
\curveto(863.02031187,688.61649166)(863.0903118,688.62649165)(863.16031587,688.62649618)
\curveto(863.23031166,688.63649164)(863.30531159,688.65149163)(863.38531587,688.67149618)
\curveto(863.63531126,688.73149155)(863.87031102,688.7814915)(864.09031587,688.82149618)
\curveto(864.31031058,688.87149141)(864.48531041,688.98649129)(864.61531587,689.16649618)
\curveto(864.67531022,689.24649103)(864.72531017,689.34649093)(864.76531587,689.46649618)
\curveto(864.80531009,689.59649068)(864.80531009,689.73649054)(864.76531587,689.88649618)
\curveto(864.70531019,690.12649015)(864.61531028,690.31648996)(864.49531587,690.45649618)
\curveto(864.38531051,690.59648968)(864.22531067,690.70648957)(864.01531587,690.78649618)
\curveto(863.895311,690.83648944)(863.75031114,690.87148941)(863.58031587,690.89149618)
\curveto(863.42031147,690.91148937)(863.25031164,690.92148936)(863.07031587,690.92149618)
\curveto(862.890312,690.92148936)(862.71531218,690.91148937)(862.54531587,690.89149618)
\curveto(862.37531252,690.87148941)(862.23031266,690.84148944)(862.11031587,690.80149618)
\curveto(861.94031295,690.74148954)(861.77531312,690.65648962)(861.61531587,690.54649618)
\curveto(861.53531336,690.48648979)(861.46031343,690.40648987)(861.39031587,690.30649618)
\curveto(861.33031356,690.21649006)(861.27531362,690.11649016)(861.22531587,690.00649618)
\curveto(861.1953137,689.92649035)(861.16531373,689.84149044)(861.13531587,689.75149618)
\curveto(861.11531378,689.66149062)(861.07031382,689.59149069)(861.00031587,689.54149618)
\curveto(860.96031393,689.51149077)(860.890314,689.48649079)(860.79031587,689.46649618)
\curveto(860.70031419,689.45649082)(860.60531429,689.45149083)(860.50531587,689.45149618)
\curveto(860.40531449,689.45149083)(860.30531459,689.45649082)(860.20531587,689.46649618)
\curveto(860.11531478,689.48649079)(860.05031484,689.51149077)(860.01031587,689.54149618)
\curveto(859.97031492,689.57149071)(859.94031495,689.62149066)(859.92031587,689.69149618)
\curveto(859.90031499,689.76149052)(859.90031499,689.83649044)(859.92031587,689.91649618)
\curveto(859.95031494,690.04649023)(859.98031491,690.16649011)(860.01031587,690.27649618)
\curveto(860.05031484,690.39648988)(860.0953148,690.51148977)(860.14531587,690.62149618)
\curveto(860.33531456,690.97148931)(860.57531432,691.24148904)(860.86531587,691.43149618)
\curveto(861.15531374,691.63148865)(861.51531338,691.79148849)(861.94531587,691.91149618)
\curveto(862.04531285,691.93148835)(862.14531275,691.94648833)(862.24531587,691.95649618)
\curveto(862.35531254,691.96648831)(862.46531243,691.9814883)(862.57531587,692.00149618)
\curveto(862.61531228,692.01148827)(862.68031221,692.01148827)(862.77031587,692.00149618)
\curveto(862.86031203,692.00148828)(862.91531198,692.01148827)(862.93531587,692.03149618)
\curveto(863.63531126,692.04148824)(864.24531065,691.96148832)(864.76531587,691.79149618)
\curveto(865.28530961,691.62148866)(865.65030924,691.29648898)(865.86031587,690.81649618)
\curveto(865.95030894,690.61648966)(866.00030889,690.3814899)(866.01031587,690.11149618)
\curveto(866.03030886,689.85149043)(866.04030885,689.5764907)(866.04031587,689.28649618)
\lineto(866.04031587,685.97149618)
\curveto(866.04030885,685.83149445)(866.04530885,685.69649458)(866.05531587,685.56649618)
\curveto(866.06530883,685.43649484)(866.0953088,685.33149495)(866.14531587,685.25149618)
\curveto(866.1953087,685.1814951)(866.26030863,685.13149515)(866.34031587,685.10149618)
\curveto(866.43030846,685.06149522)(866.51530838,685.03149525)(866.59531587,685.01149618)
\curveto(866.67530822,685.00149528)(866.73530816,684.95649532)(866.77531587,684.87649618)
\curveto(866.7953081,684.84649543)(866.80530809,684.81649546)(866.80531587,684.78649618)
\curveto(866.80530809,684.75649552)(866.81030808,684.71649556)(866.82031587,684.66649618)
\moveto(864.67531587,686.33149618)
\curveto(864.73531016,686.47149381)(864.76531013,686.63149365)(864.76531587,686.81149618)
\curveto(864.77531012,687.00149328)(864.78031011,687.19649308)(864.78031587,687.39649618)
\curveto(864.78031011,687.50649277)(864.77531012,687.60649267)(864.76531587,687.69649618)
\curveto(864.75531014,687.78649249)(864.71531018,687.85649242)(864.64531587,687.90649618)
\curveto(864.61531028,687.92649235)(864.54531035,687.93649234)(864.43531587,687.93649618)
\curveto(864.41531048,687.91649236)(864.38031051,687.90649237)(864.33031587,687.90649618)
\curveto(864.28031061,687.90649237)(864.23531066,687.89649238)(864.19531587,687.87649618)
\curveto(864.11531078,687.85649242)(864.02531087,687.83649244)(863.92531587,687.81649618)
\lineto(863.62531587,687.75649618)
\curveto(863.5953113,687.75649252)(863.56031133,687.75149253)(863.52031587,687.74149618)
\lineto(863.41531587,687.74149618)
\curveto(863.26531163,687.70149258)(863.10031179,687.6764926)(862.92031587,687.66649618)
\curveto(862.75031214,687.66649261)(862.5903123,687.64649263)(862.44031587,687.60649618)
\curveto(862.36031253,687.58649269)(862.28531261,687.56649271)(862.21531587,687.54649618)
\curveto(862.15531274,687.53649274)(862.08531281,687.52149276)(862.00531587,687.50149618)
\curveto(861.84531305,687.45149283)(861.6953132,687.38649289)(861.55531587,687.30649618)
\curveto(861.41531348,687.23649304)(861.2953136,687.14649313)(861.19531587,687.03649618)
\curveto(861.0953138,686.92649335)(861.02031387,686.79149349)(860.97031587,686.63149618)
\curveto(860.92031397,686.4814938)(860.90031399,686.29649398)(860.91031587,686.07649618)
\curveto(860.91031398,685.9764943)(860.92531397,685.8814944)(860.95531587,685.79149618)
\curveto(860.9953139,685.71149457)(861.04031385,685.63649464)(861.09031587,685.56649618)
\curveto(861.17031372,685.45649482)(861.27531362,685.36149492)(861.40531587,685.28149618)
\curveto(861.53531336,685.21149507)(861.67531322,685.15149513)(861.82531587,685.10149618)
\curveto(861.87531302,685.09149519)(861.92531297,685.08649519)(861.97531587,685.08649618)
\curveto(862.02531287,685.08649519)(862.07531282,685.0814952)(862.12531587,685.07149618)
\curveto(862.1953127,685.05149523)(862.28031261,685.03649524)(862.38031587,685.02649618)
\curveto(862.4903124,685.02649525)(862.58031231,685.03649524)(862.65031587,685.05649618)
\curveto(862.71031218,685.0764952)(862.77031212,685.0814952)(862.83031587,685.07149618)
\curveto(862.890312,685.07149521)(862.95031194,685.0814952)(863.01031587,685.10149618)
\curveto(863.0903118,685.12149516)(863.16531173,685.13649514)(863.23531587,685.14649618)
\curveto(863.31531158,685.15649512)(863.3903115,685.1764951)(863.46031587,685.20649618)
\curveto(863.75031114,685.32649495)(863.9953109,685.47149481)(864.19531587,685.64149618)
\curveto(864.40531049,685.81149447)(864.56531033,686.04149424)(864.67531587,686.33149618)
}
}
{
\newrgbcolor{curcolor}{0 0 0}
\pscustom[linestyle=none,fillstyle=solid,fillcolor=curcolor]
{
\newpath
\moveto(874.9519565,684.92149618)
\lineto(874.9519565,684.53149618)
\curveto(874.95194862,684.41149587)(874.92694865,684.31149597)(874.8769565,684.23149618)
\curveto(874.82694875,684.16149612)(874.74194883,684.12149616)(874.6219565,684.11149618)
\lineto(874.2769565,684.11149618)
\curveto(874.21694936,684.11149617)(874.15694942,684.10649617)(874.0969565,684.09649618)
\curveto(874.04694953,684.09649618)(874.00194957,684.10649617)(873.9619565,684.12649618)
\curveto(873.8719497,684.14649613)(873.81194976,684.18649609)(873.7819565,684.24649618)
\curveto(873.74194983,684.29649598)(873.71694986,684.35649592)(873.7069565,684.42649618)
\curveto(873.70694987,684.49649578)(873.69194988,684.56649571)(873.6619565,684.63649618)
\curveto(873.65194992,684.65649562)(873.63694994,684.67149561)(873.6169565,684.68149618)
\curveto(873.60694997,684.70149558)(873.59194998,684.72149556)(873.5719565,684.74149618)
\curveto(873.4719501,684.75149553)(873.39195018,684.73149555)(873.3319565,684.68149618)
\curveto(873.28195029,684.63149565)(873.22695035,684.5814957)(873.1669565,684.53149618)
\curveto(872.96695061,684.3814959)(872.76695081,684.26649601)(872.5669565,684.18649618)
\curveto(872.38695119,684.10649617)(872.1769514,684.04649623)(871.9369565,684.00649618)
\curveto(871.70695187,683.96649631)(871.46695211,683.94649633)(871.2169565,683.94649618)
\curveto(870.9769526,683.93649634)(870.73695284,683.95149633)(870.4969565,683.99149618)
\curveto(870.25695332,684.02149626)(870.04695353,684.0764962)(869.8669565,684.15649618)
\curveto(869.34695423,684.3764959)(868.92695465,684.67149561)(868.6069565,685.04149618)
\curveto(868.28695529,685.42149486)(868.03695554,685.89149439)(867.8569565,686.45149618)
\curveto(867.81695576,686.54149374)(867.78695579,686.63149365)(867.7669565,686.72149618)
\curveto(867.75695582,686.82149346)(867.73695584,686.92149336)(867.7069565,687.02149618)
\curveto(867.69695588,687.07149321)(867.69195588,687.12149316)(867.6919565,687.17149618)
\curveto(867.69195588,687.22149306)(867.68695589,687.27149301)(867.6769565,687.32149618)
\curveto(867.65695592,687.37149291)(867.64695593,687.42149286)(867.6469565,687.47149618)
\curveto(867.65695592,687.53149275)(867.65695592,687.58649269)(867.6469565,687.63649618)
\lineto(867.6469565,687.78649618)
\curveto(867.62695595,687.83649244)(867.61695596,687.90149238)(867.6169565,687.98149618)
\curveto(867.61695596,688.06149222)(867.62695595,688.12649215)(867.6469565,688.17649618)
\lineto(867.6469565,688.34149618)
\curveto(867.66695591,688.41149187)(867.6719559,688.4814918)(867.6619565,688.55149618)
\curveto(867.66195591,688.63149165)(867.6719559,688.70649157)(867.6919565,688.77649618)
\curveto(867.70195587,688.82649145)(867.70695587,688.87149141)(867.7069565,688.91149618)
\curveto(867.70695587,688.95149133)(867.71195586,688.99649128)(867.7219565,689.04649618)
\curveto(867.75195582,689.14649113)(867.7769558,689.24149104)(867.7969565,689.33149618)
\curveto(867.81695576,689.43149085)(867.84195573,689.52649075)(867.8719565,689.61649618)
\curveto(868.00195557,689.99649028)(868.16695541,690.33648994)(868.3669565,690.63649618)
\curveto(868.576955,690.94648933)(868.82695475,691.20148908)(869.1169565,691.40149618)
\curveto(869.28695429,691.52148876)(869.46195411,691.62148866)(869.6419565,691.70149618)
\curveto(869.83195374,691.7814885)(870.03695354,691.85148843)(870.2569565,691.91149618)
\curveto(870.32695325,691.92148836)(870.39195318,691.93148835)(870.4519565,691.94149618)
\curveto(870.52195305,691.95148833)(870.59195298,691.96648831)(870.6619565,691.98649618)
\lineto(870.8119565,691.98649618)
\curveto(870.89195268,692.00648827)(871.00695257,692.01648826)(871.1569565,692.01649618)
\curveto(871.31695226,692.01648826)(871.43695214,692.00648827)(871.5169565,691.98649618)
\curveto(871.55695202,691.9764883)(871.61195196,691.97148831)(871.6819565,691.97149618)
\curveto(871.79195178,691.94148834)(871.90195167,691.91648836)(872.0119565,691.89649618)
\curveto(872.12195145,691.88648839)(872.22695135,691.85648842)(872.3269565,691.80649618)
\curveto(872.4769511,691.74648853)(872.61695096,691.6814886)(872.7469565,691.61149618)
\curveto(872.88695069,691.54148874)(873.01695056,691.46148882)(873.1369565,691.37149618)
\curveto(873.19695038,691.32148896)(873.25695032,691.26648901)(873.3169565,691.20649618)
\curveto(873.38695019,691.15648912)(873.4769501,691.14148914)(873.5869565,691.16149618)
\curveto(873.60694997,691.19148909)(873.62194995,691.21648906)(873.6319565,691.23649618)
\curveto(873.65194992,691.25648902)(873.66694991,691.28648899)(873.6769565,691.32649618)
\curveto(873.70694987,691.41648886)(873.71694986,691.53148875)(873.7069565,691.67149618)
\lineto(873.7069565,692.04649618)
\lineto(873.7069565,693.77149618)
\lineto(873.7069565,694.23649618)
\curveto(873.70694987,694.41648586)(873.73194984,694.54648573)(873.7819565,694.62649618)
\curveto(873.82194975,694.69648558)(873.88194969,694.74148554)(873.9619565,694.76149618)
\curveto(873.98194959,694.76148552)(874.00694957,694.76148552)(874.0369565,694.76149618)
\curveto(874.06694951,694.77148551)(874.09194948,694.7764855)(874.1119565,694.77649618)
\curveto(874.25194932,694.78648549)(874.39694918,694.78648549)(874.5469565,694.77649618)
\curveto(874.70694887,694.7764855)(874.81694876,694.73648554)(874.8769565,694.65649618)
\curveto(874.92694865,694.5764857)(874.95194862,694.4764858)(874.9519565,694.35649618)
\lineto(874.9519565,693.98149618)
\lineto(874.9519565,684.92149618)
\moveto(873.7369565,687.75649618)
\curveto(873.75694982,687.80649247)(873.76694981,687.87149241)(873.7669565,687.95149618)
\curveto(873.76694981,688.04149224)(873.75694982,688.11149217)(873.7369565,688.16149618)
\lineto(873.7369565,688.38649618)
\curveto(873.71694986,688.4764918)(873.70194987,688.56649171)(873.6919565,688.65649618)
\curveto(873.68194989,688.75649152)(873.66194991,688.84649143)(873.6319565,688.92649618)
\curveto(873.61194996,689.00649127)(873.59194998,689.0814912)(873.5719565,689.15149618)
\curveto(873.56195001,689.22149106)(873.54195003,689.29149099)(873.5119565,689.36149618)
\curveto(873.39195018,689.66149062)(873.23695034,689.92649035)(873.0469565,690.15649618)
\curveto(872.85695072,690.38648989)(872.61695096,690.56648971)(872.3269565,690.69649618)
\curveto(872.22695135,690.74648953)(872.12195145,690.7814895)(872.0119565,690.80149618)
\curveto(871.91195166,690.83148945)(871.80195177,690.85648942)(871.6819565,690.87649618)
\curveto(871.60195197,690.89648938)(871.51195206,690.90648937)(871.4119565,690.90649618)
\lineto(871.1419565,690.90649618)
\curveto(871.09195248,690.89648938)(871.04695253,690.88648939)(871.0069565,690.87649618)
\lineto(870.8719565,690.87649618)
\curveto(870.79195278,690.85648942)(870.70695287,690.83648944)(870.6169565,690.81649618)
\curveto(870.53695304,690.79648948)(870.45695312,690.77148951)(870.3769565,690.74149618)
\curveto(870.05695352,690.60148968)(869.79695378,690.39648988)(869.5969565,690.12649618)
\curveto(869.40695417,689.86649041)(869.25195432,689.56149072)(869.1319565,689.21149618)
\curveto(869.09195448,689.10149118)(869.06195451,688.98649129)(869.0419565,688.86649618)
\curveto(869.03195454,688.75649152)(869.01695456,688.64649163)(868.9969565,688.53649618)
\curveto(868.99695458,688.49649178)(868.99195458,688.45649182)(868.9819565,688.41649618)
\lineto(868.9819565,688.31149618)
\curveto(868.96195461,688.26149202)(868.95195462,688.20649207)(868.9519565,688.14649618)
\curveto(868.96195461,688.08649219)(868.96695461,688.03149225)(868.9669565,687.98149618)
\lineto(868.9669565,687.65149618)
\curveto(868.96695461,687.55149273)(868.9769546,687.45649282)(868.9969565,687.36649618)
\curveto(869.00695457,687.33649294)(869.01195456,687.28649299)(869.0119565,687.21649618)
\curveto(869.03195454,687.14649313)(869.04695453,687.0764932)(869.0569565,687.00649618)
\lineto(869.1169565,686.79649618)
\curveto(869.22695435,686.44649383)(869.3769542,686.14649413)(869.5669565,685.89649618)
\curveto(869.75695382,685.64649463)(869.99695358,685.44149484)(870.2869565,685.28149618)
\curveto(870.3769532,685.23149505)(870.46695311,685.19149509)(870.5569565,685.16149618)
\curveto(870.64695293,685.13149515)(870.74695283,685.10149518)(870.8569565,685.07149618)
\curveto(870.90695267,685.05149523)(870.95695262,685.04649523)(871.0069565,685.05649618)
\curveto(871.06695251,685.06649521)(871.12195245,685.06149522)(871.1719565,685.04149618)
\curveto(871.21195236,685.03149525)(871.25195232,685.02649525)(871.2919565,685.02649618)
\lineto(871.4269565,685.02649618)
\lineto(871.5619565,685.02649618)
\curveto(871.59195198,685.03649524)(871.64195193,685.04149524)(871.7119565,685.04149618)
\curveto(871.79195178,685.06149522)(871.8719517,685.0764952)(871.9519565,685.08649618)
\curveto(872.03195154,685.10649517)(872.10695147,685.13149515)(872.1769565,685.16149618)
\curveto(872.50695107,685.30149498)(872.7719508,685.4764948)(872.9719565,685.68649618)
\curveto(873.18195039,685.90649437)(873.35695022,686.1814941)(873.4969565,686.51149618)
\curveto(873.54695003,686.62149366)(873.58194999,686.73149355)(873.6019565,686.84149618)
\curveto(873.62194995,686.95149333)(873.64694993,687.06149322)(873.6769565,687.17149618)
\curveto(873.69694988,687.21149307)(873.70694987,687.24649303)(873.7069565,687.27649618)
\curveto(873.70694987,687.31649296)(873.71194986,687.35649292)(873.7219565,687.39649618)
\curveto(873.73194984,687.45649282)(873.73194984,687.51649276)(873.7219565,687.57649618)
\curveto(873.72194985,687.63649264)(873.72694985,687.69649258)(873.7369565,687.75649618)
}
}
{
\newrgbcolor{curcolor}{0 0 0}
\pscustom[linestyle=none,fillstyle=solid,fillcolor=curcolor]
{
\newpath
\moveto(884.0232065,688.31149618)
\curveto(884.04319844,688.25149203)(884.05319843,688.15649212)(884.0532065,688.02649618)
\curveto(884.05319843,687.90649237)(884.04819843,687.82149246)(884.0382065,687.77149618)
\lineto(884.0382065,687.62149618)
\curveto(884.02819845,687.54149274)(884.01819846,687.46649281)(884.0082065,687.39649618)
\curveto(884.00819847,687.33649294)(884.00319848,687.26649301)(883.9932065,687.18649618)
\curveto(883.97319851,687.12649315)(883.95819852,687.06649321)(883.9482065,687.00649618)
\curveto(883.94819853,686.94649333)(883.93819854,686.88649339)(883.9182065,686.82649618)
\curveto(883.8781986,686.69649358)(883.84319864,686.56649371)(883.8132065,686.43649618)
\curveto(883.7831987,686.30649397)(883.74319874,686.18649409)(883.6932065,686.07649618)
\curveto(883.483199,685.59649468)(883.20319928,685.19149509)(882.8532065,684.86149618)
\curveto(882.50319998,684.54149574)(882.07320041,684.29649598)(881.5632065,684.12649618)
\curveto(881.45320103,684.08649619)(881.33320115,684.05649622)(881.2032065,684.03649618)
\curveto(881.0832014,684.01649626)(880.95820152,683.99649628)(880.8282065,683.97649618)
\curveto(880.76820171,683.96649631)(880.70320178,683.96149632)(880.6332065,683.96149618)
\curveto(880.57320191,683.95149633)(880.51320197,683.94649633)(880.4532065,683.94649618)
\curveto(880.41320207,683.93649634)(880.35320213,683.93149635)(880.2732065,683.93149618)
\curveto(880.20320228,683.93149635)(880.15320233,683.93649634)(880.1232065,683.94649618)
\curveto(880.0832024,683.95649632)(880.04320244,683.96149632)(880.0032065,683.96149618)
\curveto(879.96320252,683.95149633)(879.92820255,683.95149633)(879.8982065,683.96149618)
\lineto(879.8082065,683.96149618)
\lineto(879.4482065,684.00649618)
\curveto(879.30820317,684.04649623)(879.17320331,684.08649619)(879.0432065,684.12649618)
\curveto(878.91320357,684.16649611)(878.78820369,684.21149607)(878.6682065,684.26149618)
\curveto(878.21820426,684.46149582)(877.84820463,684.72149556)(877.5582065,685.04149618)
\curveto(877.26820521,685.36149492)(877.02820545,685.75149453)(876.8382065,686.21149618)
\curveto(876.78820569,686.31149397)(876.74820573,686.41149387)(876.7182065,686.51149618)
\curveto(876.69820578,686.61149367)(876.6782058,686.71649356)(876.6582065,686.82649618)
\curveto(876.63820584,686.86649341)(876.62820585,686.89649338)(876.6282065,686.91649618)
\curveto(876.63820584,686.94649333)(876.63820584,686.9814933)(876.6282065,687.02149618)
\curveto(876.60820587,687.10149318)(876.59320589,687.1814931)(876.5832065,687.26149618)
\curveto(876.5832059,687.35149293)(876.57320591,687.43649284)(876.5532065,687.51649618)
\lineto(876.5532065,687.63649618)
\curveto(876.55320593,687.6764926)(876.54820593,687.72149256)(876.5382065,687.77149618)
\curveto(876.52820595,687.82149246)(876.52320596,687.90649237)(876.5232065,688.02649618)
\curveto(876.52320596,688.15649212)(876.53320595,688.25149203)(876.5532065,688.31149618)
\curveto(876.57320591,688.3814919)(876.5782059,688.45149183)(876.5682065,688.52149618)
\curveto(876.55820592,688.59149169)(876.56320592,688.66149162)(876.5832065,688.73149618)
\curveto(876.59320589,688.7814915)(876.59820588,688.82149146)(876.5982065,688.85149618)
\curveto(876.60820587,688.89149139)(876.61820586,688.93649134)(876.6282065,688.98649618)
\curveto(876.65820582,689.10649117)(876.6832058,689.22649105)(876.7032065,689.34649618)
\curveto(876.73320575,689.46649081)(876.77320571,689.5814907)(876.8232065,689.69149618)
\curveto(876.97320551,690.06149022)(877.15320533,690.39148989)(877.3632065,690.68149618)
\curveto(877.5832049,690.9814893)(877.84820463,691.23148905)(878.1582065,691.43149618)
\curveto(878.2782042,691.51148877)(878.40320408,691.5764887)(878.5332065,691.62649618)
\curveto(878.66320382,691.68648859)(878.79820368,691.74648853)(878.9382065,691.80649618)
\curveto(879.05820342,691.85648842)(879.18820329,691.88648839)(879.3282065,691.89649618)
\curveto(879.46820301,691.91648836)(879.60820287,691.94648833)(879.7482065,691.98649618)
\lineto(879.9432065,691.98649618)
\curveto(880.01320247,691.99648828)(880.0782024,692.00648827)(880.1382065,692.01649618)
\curveto(881.02820145,692.02648825)(881.76820071,691.84148844)(882.3582065,691.46149618)
\curveto(882.94819953,691.0814892)(883.37319911,690.58648969)(883.6332065,689.97649618)
\curveto(883.6831988,689.8764904)(883.72319876,689.7764905)(883.7532065,689.67649618)
\curveto(883.7831987,689.5764907)(883.81819866,689.47149081)(883.8582065,689.36149618)
\curveto(883.88819859,689.25149103)(883.91319857,689.13149115)(883.9332065,689.00149618)
\curveto(883.95319853,688.8814914)(883.9781985,688.75649152)(884.0082065,688.62649618)
\curveto(884.01819846,688.5764917)(884.01819846,688.52149176)(884.0082065,688.46149618)
\curveto(884.00819847,688.41149187)(884.01319847,688.36149192)(884.0232065,688.31149618)
\moveto(882.6882065,687.45649618)
\curveto(882.70819977,687.52649275)(882.71319977,687.60649267)(882.7032065,687.69649618)
\lineto(882.7032065,687.95149618)
\curveto(882.70319978,688.34149194)(882.66819981,688.67149161)(882.5982065,688.94149618)
\curveto(882.56819991,689.02149126)(882.54319994,689.10149118)(882.5232065,689.18149618)
\curveto(882.50319998,689.26149102)(882.4782,689.33649094)(882.4482065,689.40649618)
\curveto(882.16820031,690.05649022)(881.72320076,690.50648977)(881.1132065,690.75649618)
\curveto(881.04320144,690.78648949)(880.96820151,690.80648947)(880.8882065,690.81649618)
\lineto(880.6482065,690.87649618)
\curveto(880.56820191,690.89648938)(880.483202,690.90648937)(880.3932065,690.90649618)
\lineto(880.1232065,690.90649618)
\lineto(879.8532065,690.86149618)
\curveto(879.75320273,690.84148944)(879.65820282,690.81648946)(879.5682065,690.78649618)
\curveto(879.48820299,690.76648951)(879.40820307,690.73648954)(879.3282065,690.69649618)
\curveto(879.25820322,690.6764896)(879.19320329,690.64648963)(879.1332065,690.60649618)
\curveto(879.07320341,690.56648971)(879.01820346,690.52648975)(878.9682065,690.48649618)
\curveto(878.72820375,690.31648996)(878.53320395,690.11149017)(878.3832065,689.87149618)
\curveto(878.23320425,689.63149065)(878.10320438,689.35149093)(877.9932065,689.03149618)
\curveto(877.96320452,688.93149135)(877.94320454,688.82649145)(877.9332065,688.71649618)
\curveto(877.92320456,688.61649166)(877.90820457,688.51149177)(877.8882065,688.40149618)
\curveto(877.8782046,688.36149192)(877.87320461,688.29649198)(877.8732065,688.20649618)
\curveto(877.86320462,688.1764921)(877.85820462,688.14149214)(877.8582065,688.10149618)
\curveto(877.86820461,688.06149222)(877.87320461,688.01649226)(877.8732065,687.96649618)
\lineto(877.8732065,687.66649618)
\curveto(877.87320461,687.56649271)(877.8832046,687.4764928)(877.9032065,687.39649618)
\lineto(877.9332065,687.21649618)
\curveto(877.95320453,687.11649316)(877.96820451,687.01649326)(877.9782065,686.91649618)
\curveto(877.99820448,686.82649345)(878.02820445,686.74149354)(878.0682065,686.66149618)
\curveto(878.16820431,686.42149386)(878.2832042,686.19649408)(878.4132065,685.98649618)
\curveto(878.55320393,685.7764945)(878.72320376,685.60149468)(878.9232065,685.46149618)
\curveto(878.97320351,685.43149485)(879.01820346,685.40649487)(879.0582065,685.38649618)
\curveto(879.09820338,685.36649491)(879.14320334,685.34149494)(879.1932065,685.31149618)
\curveto(879.27320321,685.26149502)(879.35820312,685.21649506)(879.4482065,685.17649618)
\curveto(879.54820293,685.14649513)(879.65320283,685.11649516)(879.7632065,685.08649618)
\curveto(879.81320267,685.06649521)(879.85820262,685.05649522)(879.8982065,685.05649618)
\curveto(879.94820253,685.06649521)(879.99820248,685.06649521)(880.0482065,685.05649618)
\curveto(880.0782024,685.04649523)(880.13820234,685.03649524)(880.2282065,685.02649618)
\curveto(880.32820215,685.01649526)(880.40320208,685.02149526)(880.4532065,685.04149618)
\curveto(880.49320199,685.05149523)(880.53320195,685.05149523)(880.5732065,685.04149618)
\curveto(880.61320187,685.04149524)(880.65320183,685.05149523)(880.6932065,685.07149618)
\curveto(880.77320171,685.09149519)(880.85320163,685.10649517)(880.9332065,685.11649618)
\curveto(881.01320147,685.13649514)(881.08820139,685.16149512)(881.1582065,685.19149618)
\curveto(881.49820098,685.33149495)(881.77320071,685.52649475)(881.9832065,685.77649618)
\curveto(882.19320029,686.02649425)(882.36820011,686.32149396)(882.5082065,686.66149618)
\curveto(882.55819992,686.7814935)(882.58819989,686.90649337)(882.5982065,687.03649618)
\curveto(882.61819986,687.1764931)(882.64819983,687.31649296)(882.6882065,687.45649618)
}
}
{
\newrgbcolor{curcolor}{0.90196079 0.90196079 0.90196079}
\pscustom[linestyle=none,fillstyle=solid,fillcolor=curcolor]
{
\newpath
\moveto(812.80437349,694.8215328)
\lineto(827.80437349,694.8215328)
\lineto(827.80437349,679.8215328)
\lineto(812.80437349,679.8215328)
\closepath
}
}
{
\newrgbcolor{curcolor}{0 0 0}
\pscustom[linestyle=none,fillstyle=solid,fillcolor=curcolor]
{
\newpath
\moveto(832.7925815,671.75579062)
\lineto(837.6975815,671.75579062)
\lineto(838.9875815,671.75579062)
\curveto(839.09757362,671.75577992)(839.20757351,671.75577992)(839.3175815,671.75579062)
\curveto(839.42757329,671.76577991)(839.5175732,671.74577993)(839.5875815,671.69579062)
\curveto(839.6175731,671.67578)(839.64257307,671.65078003)(839.6625815,671.62079062)
\curveto(839.68257303,671.59078009)(839.70257301,671.56078012)(839.7225815,671.53079062)
\curveto(839.74257297,671.46078022)(839.75257296,671.34578033)(839.7525815,671.18579062)
\curveto(839.75257296,671.03578064)(839.74257297,670.92078076)(839.7225815,670.84079062)
\curveto(839.68257303,670.70078098)(839.59757312,670.62078106)(839.4675815,670.60079062)
\curveto(839.33757338,670.59078109)(839.18257353,670.58578109)(839.0025815,670.58579062)
\lineto(837.5025815,670.58579062)
\lineto(834.9825815,670.58579062)
\lineto(834.4125815,670.58579062)
\curveto(834.20257851,670.59578108)(834.04757867,670.57078111)(833.9475815,670.51079062)
\curveto(833.84757887,670.45078123)(833.79257892,670.34578133)(833.7825815,670.19579062)
\lineto(833.7825815,669.73079062)
\lineto(833.7825815,668.20079062)
\curveto(833.78257893,668.09078359)(833.77757894,667.96078372)(833.7675815,667.81079062)
\curveto(833.76757895,667.66078402)(833.77757894,667.54078414)(833.7975815,667.45079062)
\curveto(833.82757889,667.33078435)(833.88757883,667.25078443)(833.9775815,667.21079062)
\curveto(834.0175787,667.19078449)(834.08757863,667.17078451)(834.1875815,667.15079062)
\lineto(834.3375815,667.15079062)
\curveto(834.37757834,667.14078454)(834.4175783,667.13578454)(834.4575815,667.13579062)
\curveto(834.50757821,667.14578453)(834.55757816,667.15078453)(834.6075815,667.15079062)
\lineto(835.1175815,667.15079062)
\lineto(838.0575815,667.15079062)
\lineto(838.3575815,667.15079062)
\curveto(838.46757425,667.16078452)(838.57757414,667.16078452)(838.6875815,667.15079062)
\curveto(838.80757391,667.15078453)(838.9125738,667.14078454)(839.0025815,667.12079062)
\curveto(839.10257361,667.11078457)(839.17757354,667.09078459)(839.2275815,667.06079062)
\curveto(839.25757346,667.04078464)(839.28257343,666.99578468)(839.3025815,666.92579062)
\curveto(839.32257339,666.85578482)(839.33757338,666.7807849)(839.3475815,666.70079062)
\curveto(839.35757336,666.62078506)(839.35757336,666.53578514)(839.3475815,666.44579062)
\curveto(839.34757337,666.36578531)(839.33757338,666.29578538)(839.3175815,666.23579062)
\curveto(839.29757342,666.14578553)(839.25257346,666.0807856)(839.1825815,666.04079062)
\curveto(839.16257355,666.02078566)(839.13257358,666.00578567)(839.0925815,665.99579062)
\curveto(839.06257365,665.99578568)(839.03257368,665.99078569)(839.0025815,665.98079062)
\lineto(838.9125815,665.98079062)
\curveto(838.86257385,665.97078571)(838.8125739,665.96578571)(838.7625815,665.96579062)
\curveto(838.712574,665.9757857)(838.66257405,665.9807857)(838.6125815,665.98079062)
\lineto(838.0575815,665.98079062)
\lineto(834.8925815,665.98079062)
\lineto(834.5325815,665.98079062)
\curveto(834.42257829,665.99078569)(834.3175784,665.98578569)(834.2175815,665.96579062)
\curveto(834.1175786,665.95578572)(834.02757869,665.93078575)(833.9475815,665.89079062)
\curveto(833.87757884,665.85078583)(833.82757889,665.7807859)(833.7975815,665.68079062)
\curveto(833.77757894,665.62078606)(833.76757895,665.55078613)(833.7675815,665.47079062)
\curveto(833.77757894,665.39078629)(833.78257893,665.31078637)(833.7825815,665.23079062)
\lineto(833.7825815,664.39079062)
\lineto(833.7825815,662.96579062)
\curveto(833.78257893,662.82578885)(833.78757893,662.69578898)(833.7975815,662.57579062)
\curveto(833.80757891,662.46578921)(833.84757887,662.38578929)(833.9175815,662.33579062)
\curveto(833.98757873,662.28578939)(834.06757865,662.25578942)(834.1575815,662.24579062)
\lineto(834.4575815,662.24579062)
\lineto(835.4175815,662.24579062)
\lineto(838.1925815,662.24579062)
\lineto(839.0475815,662.24579062)
\lineto(839.2875815,662.24579062)
\curveto(839.36757335,662.25578942)(839.43757328,662.25078943)(839.4975815,662.23079062)
\curveto(839.6175731,662.19078949)(839.69757302,662.13578954)(839.7375815,662.06579062)
\curveto(839.75757296,662.03578964)(839.77257294,661.98578969)(839.7825815,661.91579062)
\curveto(839.79257292,661.84578983)(839.79757292,661.77078991)(839.7975815,661.69079062)
\curveto(839.80757291,661.62079006)(839.80757291,661.54579013)(839.7975815,661.46579062)
\curveto(839.78757293,661.39579028)(839.77757294,661.34079034)(839.7675815,661.30079062)
\curveto(839.72757299,661.22079046)(839.68257303,661.16579051)(839.6325815,661.13579062)
\curveto(839.57257314,661.09579058)(839.49257322,661.0757906)(839.3925815,661.07579062)
\lineto(839.1225815,661.07579062)
\lineto(838.0725815,661.07579062)
\lineto(834.0825815,661.07579062)
\lineto(833.0325815,661.07579062)
\curveto(832.89257982,661.0757906)(832.77257994,661.0807906)(832.6725815,661.09079062)
\curveto(832.57258014,661.11079057)(832.49758022,661.16079052)(832.4475815,661.24079062)
\curveto(832.40758031,661.30079038)(832.38758033,661.3757903)(832.3875815,661.46579062)
\lineto(832.3875815,661.75079062)
\lineto(832.3875815,662.80079062)
\lineto(832.3875815,666.82079062)
\lineto(832.3875815,670.18079062)
\lineto(832.3875815,671.11079062)
\lineto(832.3875815,671.38079062)
\curveto(832.38758033,671.47078021)(832.40758031,671.54078014)(832.4475815,671.59079062)
\curveto(832.48758023,671.66078002)(832.56258015,671.71077997)(832.6725815,671.74079062)
\curveto(832.69258002,671.75077993)(832.71258,671.75077993)(832.7325815,671.74079062)
\curveto(832.75257996,671.74077994)(832.77257994,671.74577993)(832.7925815,671.75579062)
}
}
{
\newrgbcolor{curcolor}{0 0 0}
\pscustom[linestyle=none,fillstyle=solid,fillcolor=curcolor]
{
\newpath
\moveto(843.73250337,668.98079062)
\curveto(844.45249931,668.99078269)(845.0574987,668.90578277)(845.54750337,668.72579062)
\curveto(846.03749772,668.55578312)(846.41749734,668.25078343)(846.68750337,667.81079062)
\curveto(846.757497,667.70078398)(846.81249695,667.58578409)(846.85250337,667.46579062)
\curveto(846.89249687,667.35578432)(846.93249683,667.23078445)(846.97250337,667.09079062)
\curveto(846.99249677,667.02078466)(846.99749676,666.94578473)(846.98750337,666.86579062)
\curveto(846.97749678,666.79578488)(846.9624968,666.74078494)(846.94250337,666.70079062)
\curveto(846.92249684,666.680785)(846.89749686,666.66078502)(846.86750337,666.64079062)
\curveto(846.83749692,666.63078505)(846.81249695,666.61578506)(846.79250337,666.59579062)
\curveto(846.74249702,666.5757851)(846.69249707,666.57078511)(846.64250337,666.58079062)
\curveto(846.59249717,666.59078509)(846.54249722,666.59078509)(846.49250337,666.58079062)
\curveto(846.41249735,666.56078512)(846.30749745,666.55578512)(846.17750337,666.56579062)
\curveto(846.04749771,666.58578509)(845.9574978,666.61078507)(845.90750337,666.64079062)
\curveto(845.82749793,666.69078499)(845.77249799,666.75578492)(845.74250337,666.83579062)
\curveto(845.72249804,666.92578475)(845.68749807,667.01078467)(845.63750337,667.09079062)
\curveto(845.54749821,667.25078443)(845.42249834,667.39578428)(845.26250337,667.52579062)
\curveto(845.15249861,667.60578407)(845.03249873,667.66578401)(844.90250337,667.70579062)
\curveto(844.77249899,667.74578393)(844.63249913,667.78578389)(844.48250337,667.82579062)
\curveto(844.43249933,667.84578383)(844.38249938,667.85078383)(844.33250337,667.84079062)
\curveto(844.28249948,667.84078384)(844.23249953,667.84578383)(844.18250337,667.85579062)
\curveto(844.12249964,667.8757838)(844.04749971,667.88578379)(843.95750337,667.88579062)
\curveto(843.86749989,667.88578379)(843.79249997,667.8757838)(843.73250337,667.85579062)
\lineto(843.64250337,667.85579062)
\lineto(843.49250337,667.82579062)
\curveto(843.44250032,667.82578385)(843.39250037,667.82078386)(843.34250337,667.81079062)
\curveto(843.08250068,667.75078393)(842.86750089,667.66578401)(842.69750337,667.55579062)
\curveto(842.52750123,667.44578423)(842.41250135,667.26078442)(842.35250337,667.00079062)
\curveto(842.33250143,666.93078475)(842.32750143,666.86078482)(842.33750337,666.79079062)
\curveto(842.3575014,666.72078496)(842.37750138,666.66078502)(842.39750337,666.61079062)
\curveto(842.4575013,666.46078522)(842.52750123,666.35078533)(842.60750337,666.28079062)
\curveto(842.69750106,666.22078546)(842.80750095,666.15078553)(842.93750337,666.07079062)
\curveto(843.09750066,665.97078571)(843.27750048,665.89578578)(843.47750337,665.84579062)
\curveto(843.67750008,665.80578587)(843.87749988,665.75578592)(844.07750337,665.69579062)
\curveto(844.20749955,665.65578602)(844.33749942,665.62578605)(844.46750337,665.60579062)
\curveto(844.59749916,665.58578609)(844.72749903,665.55578612)(844.85750337,665.51579062)
\curveto(845.06749869,665.45578622)(845.27249849,665.39578628)(845.47250337,665.33579062)
\curveto(845.67249809,665.28578639)(845.87249789,665.22078646)(846.07250337,665.14079062)
\lineto(846.22250337,665.08079062)
\curveto(846.27249749,665.06078662)(846.32249744,665.03578664)(846.37250337,665.00579062)
\curveto(846.57249719,664.88578679)(846.74749701,664.75078693)(846.89750337,664.60079062)
\curveto(847.04749671,664.45078723)(847.17249659,664.26078742)(847.27250337,664.03079062)
\curveto(847.29249647,663.96078772)(847.31249645,663.86578781)(847.33250337,663.74579062)
\curveto(847.35249641,663.675788)(847.3624964,663.60078808)(847.36250337,663.52079062)
\curveto(847.37249639,663.45078823)(847.37749638,663.37078831)(847.37750337,663.28079062)
\lineto(847.37750337,663.13079062)
\curveto(847.3574964,663.06078862)(847.34749641,662.99078869)(847.34750337,662.92079062)
\curveto(847.34749641,662.85078883)(847.33749642,662.7807889)(847.31750337,662.71079062)
\curveto(847.28749647,662.60078908)(847.25249651,662.49578918)(847.21250337,662.39579062)
\curveto(847.17249659,662.29578938)(847.12749663,662.20578947)(847.07750337,662.12579062)
\curveto(846.91749684,661.86578981)(846.71249705,661.65579002)(846.46250337,661.49579062)
\curveto(846.21249755,661.34579033)(845.93249783,661.21579046)(845.62250337,661.10579062)
\curveto(845.53249823,661.0757906)(845.43749832,661.05579062)(845.33750337,661.04579062)
\curveto(845.24749851,661.02579065)(845.1574986,661.00079068)(845.06750337,660.97079062)
\curveto(844.96749879,660.95079073)(844.86749889,660.94079074)(844.76750337,660.94079062)
\curveto(844.66749909,660.94079074)(844.56749919,660.93079075)(844.46750337,660.91079062)
\lineto(844.31750337,660.91079062)
\curveto(844.26749949,660.90079078)(844.19749956,660.89579078)(844.10750337,660.89579062)
\curveto(844.01749974,660.89579078)(843.94749981,660.90079078)(843.89750337,660.91079062)
\lineto(843.73250337,660.91079062)
\curveto(843.67250009,660.93079075)(843.60750015,660.94079074)(843.53750337,660.94079062)
\curveto(843.46750029,660.93079075)(843.40750035,660.93579074)(843.35750337,660.95579062)
\curveto(843.30750045,660.96579071)(843.24250052,660.97079071)(843.16250337,660.97079062)
\lineto(842.92250337,661.03079062)
\curveto(842.85250091,661.04079064)(842.77750098,661.06079062)(842.69750337,661.09079062)
\curveto(842.38750137,661.19079049)(842.11750164,661.31579036)(841.88750337,661.46579062)
\curveto(841.6575021,661.61579006)(841.4575023,661.81078987)(841.28750337,662.05079062)
\curveto(841.19750256,662.1807895)(841.12250264,662.31578936)(841.06250337,662.45579062)
\curveto(841.00250276,662.59578908)(840.94750281,662.75078893)(840.89750337,662.92079062)
\curveto(840.87750288,662.9807887)(840.86750289,663.05078863)(840.86750337,663.13079062)
\curveto(840.87750288,663.22078846)(840.89250287,663.29078839)(840.91250337,663.34079062)
\curveto(840.94250282,663.3807883)(840.99250277,663.42078826)(841.06250337,663.46079062)
\curveto(841.11250265,663.4807882)(841.18250258,663.49078819)(841.27250337,663.49079062)
\curveto(841.3625024,663.50078818)(841.45250231,663.50078818)(841.54250337,663.49079062)
\curveto(841.63250213,663.4807882)(841.71750204,663.46578821)(841.79750337,663.44579062)
\curveto(841.88750187,663.43578824)(841.94750181,663.42078826)(841.97750337,663.40079062)
\curveto(842.04750171,663.35078833)(842.09250167,663.2757884)(842.11250337,663.17579062)
\curveto(842.14250162,663.08578859)(842.17750158,663.00078868)(842.21750337,662.92079062)
\curveto(842.31750144,662.70078898)(842.45250131,662.53078915)(842.62250337,662.41079062)
\curveto(842.74250102,662.32078936)(842.87750088,662.25078943)(843.02750337,662.20079062)
\curveto(843.17750058,662.15078953)(843.33750042,662.10078958)(843.50750337,662.05079062)
\lineto(843.82250337,662.00579062)
\lineto(843.91250337,662.00579062)
\curveto(843.98249978,661.98578969)(844.07249969,661.9757897)(844.18250337,661.97579062)
\curveto(844.30249946,661.9757897)(844.40249936,661.98578969)(844.48250337,662.00579062)
\curveto(844.55249921,662.00578967)(844.60749915,662.01078967)(844.64750337,662.02079062)
\curveto(844.70749905,662.03078965)(844.76749899,662.03578964)(844.82750337,662.03579062)
\curveto(844.88749887,662.04578963)(844.94249882,662.05578962)(844.99250337,662.06579062)
\curveto(845.28249848,662.14578953)(845.51249825,662.25078943)(845.68250337,662.38079062)
\curveto(845.85249791,662.51078917)(845.97249779,662.73078895)(846.04250337,663.04079062)
\curveto(846.0624977,663.09078859)(846.06749769,663.14578853)(846.05750337,663.20579062)
\curveto(846.04749771,663.26578841)(846.03749772,663.31078837)(846.02750337,663.34079062)
\curveto(845.97749778,663.53078815)(845.90749785,663.67078801)(845.81750337,663.76079062)
\curveto(845.72749803,663.86078782)(845.61249815,663.95078773)(845.47250337,664.03079062)
\curveto(845.38249838,664.09078759)(845.28249848,664.14078754)(845.17250337,664.18079062)
\lineto(844.84250337,664.30079062)
\curveto(844.81249895,664.31078737)(844.78249898,664.31578736)(844.75250337,664.31579062)
\curveto(844.73249903,664.31578736)(844.70749905,664.32578735)(844.67750337,664.34579062)
\curveto(844.33749942,664.45578722)(843.98249978,664.53578714)(843.61250337,664.58579062)
\curveto(843.25250051,664.64578703)(842.91250085,664.74078694)(842.59250337,664.87079062)
\curveto(842.49250127,664.91078677)(842.39750136,664.94578673)(842.30750337,664.97579062)
\curveto(842.21750154,665.00578667)(842.13250163,665.04578663)(842.05250337,665.09579062)
\curveto(841.8625019,665.20578647)(841.68750207,665.33078635)(841.52750337,665.47079062)
\curveto(841.36750239,665.61078607)(841.24250252,665.78578589)(841.15250337,665.99579062)
\curveto(841.12250264,666.06578561)(841.09750266,666.13578554)(841.07750337,666.20579062)
\curveto(841.06750269,666.2757854)(841.05250271,666.35078533)(841.03250337,666.43079062)
\curveto(841.00250276,666.55078513)(840.99250277,666.68578499)(841.00250337,666.83579062)
\curveto(841.01250275,666.99578468)(841.02750273,667.13078455)(841.04750337,667.24079062)
\curveto(841.06750269,667.29078439)(841.07750268,667.33078435)(841.07750337,667.36079062)
\curveto(841.08750267,667.40078428)(841.10250266,667.44078424)(841.12250337,667.48079062)
\curveto(841.21250255,667.71078397)(841.33250243,667.91078377)(841.48250337,668.08079062)
\curveto(841.64250212,668.25078343)(841.82250194,668.40078328)(842.02250337,668.53079062)
\curveto(842.17250159,668.62078306)(842.33750142,668.69078299)(842.51750337,668.74079062)
\curveto(842.69750106,668.80078288)(842.88750087,668.85578282)(843.08750337,668.90579062)
\curveto(843.1575006,668.91578276)(843.22250054,668.92578275)(843.28250337,668.93579062)
\curveto(843.35250041,668.94578273)(843.42750033,668.95578272)(843.50750337,668.96579062)
\curveto(843.53750022,668.9757827)(843.57750018,668.9757827)(843.62750337,668.96579062)
\curveto(843.67750008,668.95578272)(843.71250005,668.96078272)(843.73250337,668.98079062)
}
}
{
\newrgbcolor{curcolor}{0 0 0}
\pscustom[linestyle=none,fillstyle=solid,fillcolor=curcolor]
{
\newpath
\moveto(849.74750337,671.14079062)
\curveto(849.89750136,671.14078054)(850.04750121,671.13578054)(850.19750337,671.12579062)
\curveto(850.34750091,671.12578055)(850.45250081,671.08578059)(850.51250337,671.00579062)
\curveto(850.5625007,670.94578073)(850.58750067,670.86078082)(850.58750337,670.75079062)
\curveto(850.59750066,670.65078103)(850.60250066,670.54578113)(850.60250337,670.43579062)
\lineto(850.60250337,669.56579062)
\curveto(850.60250066,669.48578219)(850.59750066,669.40078228)(850.58750337,669.31079062)
\curveto(850.58750067,669.23078245)(850.59750066,669.16078252)(850.61750337,669.10079062)
\curveto(850.6575006,668.96078272)(850.74750051,668.87078281)(850.88750337,668.83079062)
\curveto(850.93750032,668.82078286)(850.98250028,668.81578286)(851.02250337,668.81579062)
\lineto(851.17250337,668.81579062)
\lineto(851.57750337,668.81579062)
\curveto(851.73749952,668.82578285)(851.85249941,668.81578286)(851.92250337,668.78579062)
\curveto(852.01249925,668.72578295)(852.07249919,668.66578301)(852.10250337,668.60579062)
\curveto(852.12249914,668.56578311)(852.13249913,668.52078316)(852.13250337,668.47079062)
\lineto(852.13250337,668.32079062)
\curveto(852.13249913,668.21078347)(852.12749913,668.10578357)(852.11750337,668.00579062)
\curveto(852.10749915,667.91578376)(852.07249919,667.84578383)(852.01250337,667.79579062)
\curveto(851.95249931,667.74578393)(851.86749939,667.71578396)(851.75750337,667.70579062)
\lineto(851.42750337,667.70579062)
\curveto(851.31749994,667.71578396)(851.20750005,667.72078396)(851.09750337,667.72079062)
\curveto(850.98750027,667.72078396)(850.89250037,667.70578397)(850.81250337,667.67579062)
\curveto(850.74250052,667.64578403)(850.69250057,667.59578408)(850.66250337,667.52579062)
\curveto(850.63250063,667.45578422)(850.61250065,667.37078431)(850.60250337,667.27079062)
\curveto(850.59250067,667.1807845)(850.58750067,667.0807846)(850.58750337,666.97079062)
\curveto(850.59750066,666.87078481)(850.60250066,666.77078491)(850.60250337,666.67079062)
\lineto(850.60250337,663.70079062)
\curveto(850.60250066,663.4807882)(850.59750066,663.24578843)(850.58750337,662.99579062)
\curveto(850.58750067,662.75578892)(850.63250063,662.57078911)(850.72250337,662.44079062)
\curveto(850.77250049,662.36078932)(850.83750042,662.30578937)(850.91750337,662.27579062)
\curveto(850.99750026,662.24578943)(851.09250017,662.22078946)(851.20250337,662.20079062)
\curveto(851.23250003,662.19078949)(851.2625,662.18578949)(851.29250337,662.18579062)
\curveto(851.33249993,662.19578948)(851.36749989,662.19578948)(851.39750337,662.18579062)
\lineto(851.59250337,662.18579062)
\curveto(851.69249957,662.18578949)(851.78249948,662.1757895)(851.86250337,662.15579062)
\curveto(851.95249931,662.14578953)(852.01749924,662.11078957)(852.05750337,662.05079062)
\curveto(852.07749918,662.02078966)(852.09249917,661.96578971)(852.10250337,661.88579062)
\curveto(852.12249914,661.81578986)(852.13249913,661.74078994)(852.13250337,661.66079062)
\curveto(852.14249912,661.5807901)(852.14249912,661.50079018)(852.13250337,661.42079062)
\curveto(852.12249914,661.35079033)(852.10249916,661.29579038)(852.07250337,661.25579062)
\curveto(852.03249923,661.18579049)(851.9574993,661.13579054)(851.84750337,661.10579062)
\curveto(851.76749949,661.08579059)(851.67749958,661.0757906)(851.57750337,661.07579062)
\curveto(851.47749978,661.08579059)(851.38749987,661.09079059)(851.30750337,661.09079062)
\curveto(851.24750001,661.09079059)(851.18750007,661.08579059)(851.12750337,661.07579062)
\curveto(851.06750019,661.0757906)(851.01250025,661.0807906)(850.96250337,661.09079062)
\lineto(850.78250337,661.09079062)
\curveto(850.73250053,661.10079058)(850.68250058,661.10579057)(850.63250337,661.10579062)
\curveto(850.59250067,661.11579056)(850.54750071,661.12079056)(850.49750337,661.12079062)
\curveto(850.29750096,661.17079051)(850.12250114,661.22579045)(849.97250337,661.28579062)
\curveto(849.83250143,661.34579033)(849.71250155,661.45079023)(849.61250337,661.60079062)
\curveto(849.47250179,661.80078988)(849.39250187,662.05078963)(849.37250337,662.35079062)
\curveto(849.35250191,662.66078902)(849.34250192,662.99078869)(849.34250337,663.34079062)
\lineto(849.34250337,667.27079062)
\curveto(849.31250195,667.40078428)(849.28250198,667.49578418)(849.25250337,667.55579062)
\curveto(849.23250203,667.61578406)(849.1625021,667.66578401)(849.04250337,667.70579062)
\curveto(849.00250226,667.71578396)(848.9625023,667.71578396)(848.92250337,667.70579062)
\curveto(848.88250238,667.69578398)(848.84250242,667.70078398)(848.80250337,667.72079062)
\lineto(848.56250337,667.72079062)
\curveto(848.43250283,667.72078396)(848.32250294,667.73078395)(848.23250337,667.75079062)
\curveto(848.15250311,667.7807839)(848.09750316,667.84078384)(848.06750337,667.93079062)
\curveto(848.04750321,667.97078371)(848.03250323,668.01578366)(848.02250337,668.06579062)
\lineto(848.02250337,668.21579062)
\curveto(848.02250324,668.35578332)(848.03250323,668.47078321)(848.05250337,668.56079062)
\curveto(848.07250319,668.66078302)(848.13250313,668.73578294)(848.23250337,668.78579062)
\curveto(848.34250292,668.82578285)(848.48250278,668.83578284)(848.65250337,668.81579062)
\curveto(848.83250243,668.79578288)(848.98250228,668.80578287)(849.10250337,668.84579062)
\curveto(849.19250207,668.89578278)(849.262502,668.96578271)(849.31250337,669.05579062)
\curveto(849.33250193,669.11578256)(849.34250192,669.19078249)(849.34250337,669.28079062)
\lineto(849.34250337,669.53579062)
\lineto(849.34250337,670.46579062)
\lineto(849.34250337,670.70579062)
\curveto(849.34250192,670.79578088)(849.35250191,670.87078081)(849.37250337,670.93079062)
\curveto(849.41250185,671.01078067)(849.48750177,671.0757806)(849.59750337,671.12579062)
\curveto(849.62750163,671.12578055)(849.65250161,671.12578055)(849.67250337,671.12579062)
\curveto(849.70250156,671.13578054)(849.72750153,671.14078054)(849.74750337,671.14079062)
}
}
{
\newrgbcolor{curcolor}{0 0 0}
\pscustom[linestyle=none,fillstyle=solid,fillcolor=curcolor]
{
\newpath
\moveto(853.98430025,668.80079062)
\lineto(854.41930025,668.80079062)
\curveto(854.56929828,668.80078288)(854.67429818,668.76078292)(854.73430025,668.68079062)
\curveto(854.78429807,668.60078308)(854.80929804,668.50078318)(854.80930025,668.38079062)
\curveto(854.81929803,668.26078342)(854.82429803,668.14078354)(854.82430025,668.02079062)
\lineto(854.82430025,666.59579062)
\lineto(854.82430025,664.33079062)
\lineto(854.82430025,663.64079062)
\curveto(854.82429803,663.41078827)(854.849298,663.21078847)(854.89930025,663.04079062)
\curveto(855.05929779,662.59078909)(855.35929749,662.2757894)(855.79930025,662.09579062)
\curveto(856.01929683,662.00578967)(856.28429657,661.97078971)(856.59430025,661.99079062)
\curveto(856.90429595,662.02078966)(857.1542957,662.0757896)(857.34430025,662.15579062)
\curveto(857.67429518,662.29578938)(857.93429492,662.47078921)(858.12430025,662.68079062)
\curveto(858.32429453,662.90078878)(858.47929437,663.18578849)(858.58930025,663.53579062)
\curveto(858.61929423,663.61578806)(858.63929421,663.69578798)(858.64930025,663.77579062)
\curveto(858.65929419,663.85578782)(858.67429418,663.94078774)(858.69430025,664.03079062)
\curveto(858.70429415,664.0807876)(858.70429415,664.12578755)(858.69430025,664.16579062)
\curveto(858.69429416,664.20578747)(858.70429415,664.25078743)(858.72430025,664.30079062)
\lineto(858.72430025,664.61579062)
\curveto(858.74429411,664.69578698)(858.7492941,664.78578689)(858.73930025,664.88579062)
\curveto(858.72929412,664.99578668)(858.72429413,665.09578658)(858.72430025,665.18579062)
\lineto(858.72430025,666.35579062)
\lineto(858.72430025,667.94579062)
\curveto(858.72429413,668.06578361)(858.71929413,668.19078349)(858.70930025,668.32079062)
\curveto(858.70929414,668.46078322)(858.73429412,668.57078311)(858.78430025,668.65079062)
\curveto(858.82429403,668.70078298)(858.86929398,668.73078295)(858.91930025,668.74079062)
\curveto(858.97929387,668.76078292)(859.0492938,668.7807829)(859.12930025,668.80079062)
\lineto(859.35430025,668.80079062)
\curveto(859.47429338,668.80078288)(859.57929327,668.79578288)(859.66930025,668.78579062)
\curveto(859.76929308,668.7757829)(859.84429301,668.73078295)(859.89430025,668.65079062)
\curveto(859.94429291,668.60078308)(859.96929288,668.52578315)(859.96930025,668.42579062)
\lineto(859.96930025,668.14079062)
\lineto(859.96930025,667.12079062)
\lineto(859.96930025,663.08579062)
\lineto(859.96930025,661.73579062)
\curveto(859.96929288,661.61579006)(859.96429289,661.50079018)(859.95430025,661.39079062)
\curveto(859.9542929,661.29079039)(859.91929293,661.21579046)(859.84930025,661.16579062)
\curveto(859.80929304,661.13579054)(859.7492931,661.11079057)(859.66930025,661.09079062)
\curveto(859.58929326,661.0807906)(859.49929335,661.07079061)(859.39930025,661.06079062)
\curveto(859.30929354,661.06079062)(859.21929363,661.06579061)(859.12930025,661.07579062)
\curveto(859.0492938,661.08579059)(858.98929386,661.10579057)(858.94930025,661.13579062)
\curveto(858.89929395,661.1757905)(858.854294,661.24079044)(858.81430025,661.33079062)
\curveto(858.80429405,661.37079031)(858.79429406,661.42579025)(858.78430025,661.49579062)
\curveto(858.78429407,661.56579011)(858.77929407,661.63079005)(858.76930025,661.69079062)
\curveto(858.75929409,661.76078992)(858.73929411,661.81578986)(858.70930025,661.85579062)
\curveto(858.67929417,661.89578978)(858.63429422,661.91078977)(858.57430025,661.90079062)
\curveto(858.49429436,661.8807898)(858.41429444,661.82078986)(858.33430025,661.72079062)
\curveto(858.2542946,661.63079005)(858.17929467,661.56079012)(858.10930025,661.51079062)
\curveto(857.88929496,661.35079033)(857.63929521,661.21079047)(857.35930025,661.09079062)
\curveto(857.2492956,661.04079064)(857.13429572,661.01079067)(857.01430025,661.00079062)
\curveto(856.90429595,660.9807907)(856.78929606,660.95579072)(856.66930025,660.92579062)
\curveto(856.61929623,660.91579076)(856.56429629,660.91579076)(856.50430025,660.92579062)
\curveto(856.4542964,660.93579074)(856.40429645,660.93079075)(856.35430025,660.91079062)
\curveto(856.2542966,660.89079079)(856.16429669,660.89079079)(856.08430025,660.91079062)
\lineto(855.93430025,660.91079062)
\curveto(855.88429697,660.93079075)(855.82429703,660.94079074)(855.75430025,660.94079062)
\curveto(855.69429716,660.94079074)(855.63929721,660.94579073)(855.58930025,660.95579062)
\curveto(855.5492973,660.9757907)(855.50929734,660.98579069)(855.46930025,660.98579062)
\curveto(855.43929741,660.9757907)(855.39929745,660.9807907)(855.34930025,661.00079062)
\lineto(855.10930025,661.06079062)
\curveto(855.03929781,661.0807906)(854.96429789,661.11079057)(854.88430025,661.15079062)
\curveto(854.62429823,661.26079042)(854.40429845,661.40579027)(854.22430025,661.58579062)
\curveto(854.0542988,661.7757899)(853.91429894,662.00078968)(853.80430025,662.26079062)
\curveto(853.76429909,662.35078933)(853.73429912,662.44078924)(853.71430025,662.53079062)
\lineto(853.65430025,662.83079062)
\curveto(853.63429922,662.89078879)(853.62429923,662.94578873)(853.62430025,662.99579062)
\curveto(853.63429922,663.05578862)(853.62929922,663.12078856)(853.60930025,663.19079062)
\curveto(853.59929925,663.21078847)(853.59429926,663.23578844)(853.59430025,663.26579062)
\curveto(853.59429926,663.30578837)(853.58929926,663.34078834)(853.57930025,663.37079062)
\lineto(853.57930025,663.52079062)
\curveto(853.56929928,663.56078812)(853.56429929,663.60578807)(853.56430025,663.65579062)
\curveto(853.57429928,663.71578796)(853.57929927,663.77078791)(853.57930025,663.82079062)
\lineto(853.57930025,664.42079062)
\lineto(853.57930025,667.18079062)
\lineto(853.57930025,668.14079062)
\lineto(853.57930025,668.41079062)
\curveto(853.57929927,668.50078318)(853.59929925,668.5757831)(853.63930025,668.63579062)
\curveto(853.67929917,668.70578297)(853.7542991,668.75578292)(853.86430025,668.78579062)
\curveto(853.88429897,668.79578288)(853.90429895,668.79578288)(853.92430025,668.78579062)
\curveto(853.94429891,668.78578289)(853.96429889,668.79078289)(853.98430025,668.80079062)
}
}
{
\newrgbcolor{curcolor}{0 0 0}
\pscustom[linestyle=none,fillstyle=solid,fillcolor=curcolor]
{
\newpath
\moveto(868.82890962,661.88579062)
\lineto(868.82890962,661.49579062)
\curveto(868.82890175,661.3757903)(868.80390177,661.2757904)(868.75390962,661.19579062)
\curveto(868.70390187,661.12579055)(868.61890196,661.08579059)(868.49890962,661.07579062)
\lineto(868.15390962,661.07579062)
\curveto(868.09390248,661.0757906)(868.03390254,661.07079061)(867.97390962,661.06079062)
\curveto(867.92390265,661.06079062)(867.8789027,661.07079061)(867.83890962,661.09079062)
\curveto(867.74890283,661.11079057)(867.68890289,661.15079053)(867.65890962,661.21079062)
\curveto(867.61890296,661.26079042)(867.59390298,661.32079036)(867.58390962,661.39079062)
\curveto(867.58390299,661.46079022)(867.56890301,661.53079015)(867.53890962,661.60079062)
\curveto(867.52890305,661.62079006)(867.51390306,661.63579004)(867.49390962,661.64579062)
\curveto(867.48390309,661.66579001)(867.46890311,661.68578999)(867.44890962,661.70579062)
\curveto(867.34890323,661.71578996)(867.26890331,661.69578998)(867.20890962,661.64579062)
\curveto(867.15890342,661.59579008)(867.10390347,661.54579013)(867.04390962,661.49579062)
\curveto(866.84390373,661.34579033)(866.64390393,661.23079045)(866.44390962,661.15079062)
\curveto(866.26390431,661.07079061)(866.05390452,661.01079067)(865.81390962,660.97079062)
\curveto(865.58390499,660.93079075)(865.34390523,660.91079077)(865.09390962,660.91079062)
\curveto(864.85390572,660.90079078)(864.61390596,660.91579076)(864.37390962,660.95579062)
\curveto(864.13390644,660.98579069)(863.92390665,661.04079064)(863.74390962,661.12079062)
\curveto(863.22390735,661.34079034)(862.80390777,661.63579004)(862.48390962,662.00579062)
\curveto(862.16390841,662.38578929)(861.91390866,662.85578882)(861.73390962,663.41579062)
\curveto(861.69390888,663.50578817)(861.66390891,663.59578808)(861.64390962,663.68579062)
\curveto(861.63390894,663.78578789)(861.61390896,663.88578779)(861.58390962,663.98579062)
\curveto(861.573909,664.03578764)(861.56890901,664.08578759)(861.56890962,664.13579062)
\curveto(861.56890901,664.18578749)(861.56390901,664.23578744)(861.55390962,664.28579062)
\curveto(861.53390904,664.33578734)(861.52390905,664.38578729)(861.52390962,664.43579062)
\curveto(861.53390904,664.49578718)(861.53390904,664.55078713)(861.52390962,664.60079062)
\lineto(861.52390962,664.75079062)
\curveto(861.50390907,664.80078688)(861.49390908,664.86578681)(861.49390962,664.94579062)
\curveto(861.49390908,665.02578665)(861.50390907,665.09078659)(861.52390962,665.14079062)
\lineto(861.52390962,665.30579062)
\curveto(861.54390903,665.3757863)(861.54890903,665.44578623)(861.53890962,665.51579062)
\curveto(861.53890904,665.59578608)(861.54890903,665.67078601)(861.56890962,665.74079062)
\curveto(861.578909,665.79078589)(861.58390899,665.83578584)(861.58390962,665.87579062)
\curveto(861.58390899,665.91578576)(861.58890899,665.96078572)(861.59890962,666.01079062)
\curveto(861.62890895,666.11078557)(861.65390892,666.20578547)(861.67390962,666.29579062)
\curveto(861.69390888,666.39578528)(861.71890886,666.49078519)(861.74890962,666.58079062)
\curveto(861.8789087,666.96078472)(862.04390853,667.30078438)(862.24390962,667.60079062)
\curveto(862.45390812,667.91078377)(862.70390787,668.16578351)(862.99390962,668.36579062)
\curveto(863.16390741,668.48578319)(863.33890724,668.58578309)(863.51890962,668.66579062)
\curveto(863.70890687,668.74578293)(863.91390666,668.81578286)(864.13390962,668.87579062)
\curveto(864.20390637,668.88578279)(864.26890631,668.89578278)(864.32890962,668.90579062)
\curveto(864.39890618,668.91578276)(864.46890611,668.93078275)(864.53890962,668.95079062)
\lineto(864.68890962,668.95079062)
\curveto(864.76890581,668.97078271)(864.88390569,668.9807827)(865.03390962,668.98079062)
\curveto(865.19390538,668.9807827)(865.31390526,668.97078271)(865.39390962,668.95079062)
\curveto(865.43390514,668.94078274)(865.48890509,668.93578274)(865.55890962,668.93579062)
\curveto(865.66890491,668.90578277)(865.7789048,668.8807828)(865.88890962,668.86079062)
\curveto(865.99890458,668.85078283)(866.10390447,668.82078286)(866.20390962,668.77079062)
\curveto(866.35390422,668.71078297)(866.49390408,668.64578303)(866.62390962,668.57579062)
\curveto(866.76390381,668.50578317)(866.89390368,668.42578325)(867.01390962,668.33579062)
\curveto(867.0739035,668.28578339)(867.13390344,668.23078345)(867.19390962,668.17079062)
\curveto(867.26390331,668.12078356)(867.35390322,668.10578357)(867.46390962,668.12579062)
\curveto(867.48390309,668.15578352)(867.49890308,668.1807835)(867.50890962,668.20079062)
\curveto(867.52890305,668.22078346)(867.54390303,668.25078343)(867.55390962,668.29079062)
\curveto(867.58390299,668.3807833)(867.59390298,668.49578318)(867.58390962,668.63579062)
\lineto(867.58390962,669.01079062)
\lineto(867.58390962,670.73579062)
\lineto(867.58390962,671.20079062)
\curveto(867.58390299,671.3807803)(867.60890297,671.51078017)(867.65890962,671.59079062)
\curveto(867.69890288,671.66078002)(867.75890282,671.70577997)(867.83890962,671.72579062)
\curveto(867.85890272,671.72577995)(867.88390269,671.72577995)(867.91390962,671.72579062)
\curveto(867.94390263,671.73577994)(867.96890261,671.74077994)(867.98890962,671.74079062)
\curveto(868.12890245,671.75077993)(868.2739023,671.75077993)(868.42390962,671.74079062)
\curveto(868.58390199,671.74077994)(868.69390188,671.70077998)(868.75390962,671.62079062)
\curveto(868.80390177,671.54078014)(868.82890175,671.44078024)(868.82890962,671.32079062)
\lineto(868.82890962,670.94579062)
\lineto(868.82890962,661.88579062)
\moveto(867.61390962,664.72079062)
\curveto(867.63390294,664.77078691)(867.64390293,664.83578684)(867.64390962,664.91579062)
\curveto(867.64390293,665.00578667)(867.63390294,665.0757866)(867.61390962,665.12579062)
\lineto(867.61390962,665.35079062)
\curveto(867.59390298,665.44078624)(867.578903,665.53078615)(867.56890962,665.62079062)
\curveto(867.55890302,665.72078596)(867.53890304,665.81078587)(867.50890962,665.89079062)
\curveto(867.48890309,665.97078571)(867.46890311,666.04578563)(867.44890962,666.11579062)
\curveto(867.43890314,666.18578549)(867.41890316,666.25578542)(867.38890962,666.32579062)
\curveto(867.26890331,666.62578505)(867.11390346,666.89078479)(866.92390962,667.12079062)
\curveto(866.73390384,667.35078433)(866.49390408,667.53078415)(866.20390962,667.66079062)
\curveto(866.10390447,667.71078397)(865.99890458,667.74578393)(865.88890962,667.76579062)
\curveto(865.78890479,667.79578388)(865.6789049,667.82078386)(865.55890962,667.84079062)
\curveto(865.4789051,667.86078382)(865.38890519,667.87078381)(865.28890962,667.87079062)
\lineto(865.01890962,667.87079062)
\curveto(864.96890561,667.86078382)(864.92390565,667.85078383)(864.88390962,667.84079062)
\lineto(864.74890962,667.84079062)
\curveto(864.66890591,667.82078386)(864.58390599,667.80078388)(864.49390962,667.78079062)
\curveto(864.41390616,667.76078392)(864.33390624,667.73578394)(864.25390962,667.70579062)
\curveto(863.93390664,667.56578411)(863.6739069,667.36078432)(863.47390962,667.09079062)
\curveto(863.28390729,666.83078485)(863.12890745,666.52578515)(863.00890962,666.17579062)
\curveto(862.96890761,666.06578561)(862.93890764,665.95078573)(862.91890962,665.83079062)
\curveto(862.90890767,665.72078596)(862.89390768,665.61078607)(862.87390962,665.50079062)
\curveto(862.8739077,665.46078622)(862.86890771,665.42078626)(862.85890962,665.38079062)
\lineto(862.85890962,665.27579062)
\curveto(862.83890774,665.22578645)(862.82890775,665.17078651)(862.82890962,665.11079062)
\curveto(862.83890774,665.05078663)(862.84390773,664.99578668)(862.84390962,664.94579062)
\lineto(862.84390962,664.61579062)
\curveto(862.84390773,664.51578716)(862.85390772,664.42078726)(862.87390962,664.33079062)
\curveto(862.88390769,664.30078738)(862.88890769,664.25078743)(862.88890962,664.18079062)
\curveto(862.90890767,664.11078757)(862.92390765,664.04078764)(862.93390962,663.97079062)
\lineto(862.99390962,663.76079062)
\curveto(863.10390747,663.41078827)(863.25390732,663.11078857)(863.44390962,662.86079062)
\curveto(863.63390694,662.61078907)(863.8739067,662.40578927)(864.16390962,662.24579062)
\curveto(864.25390632,662.19578948)(864.34390623,662.15578952)(864.43390962,662.12579062)
\curveto(864.52390605,662.09578958)(864.62390595,662.06578961)(864.73390962,662.03579062)
\curveto(864.78390579,662.01578966)(864.83390574,662.01078967)(864.88390962,662.02079062)
\curveto(864.94390563,662.03078965)(864.99890558,662.02578965)(865.04890962,662.00579062)
\curveto(865.08890549,661.99578968)(865.12890545,661.99078969)(865.16890962,661.99079062)
\lineto(865.30390962,661.99079062)
\lineto(865.43890962,661.99079062)
\curveto(865.46890511,662.00078968)(865.51890506,662.00578967)(865.58890962,662.00579062)
\curveto(865.66890491,662.02578965)(865.74890483,662.04078964)(865.82890962,662.05079062)
\curveto(865.90890467,662.07078961)(865.98390459,662.09578958)(866.05390962,662.12579062)
\curveto(866.38390419,662.26578941)(866.64890393,662.44078924)(866.84890962,662.65079062)
\curveto(867.05890352,662.87078881)(867.23390334,663.14578853)(867.37390962,663.47579062)
\curveto(867.42390315,663.58578809)(867.45890312,663.69578798)(867.47890962,663.80579062)
\curveto(867.49890308,663.91578776)(867.52390305,664.02578765)(867.55390962,664.13579062)
\curveto(867.573903,664.1757875)(867.58390299,664.21078747)(867.58390962,664.24079062)
\curveto(867.58390299,664.2807874)(867.58890299,664.32078736)(867.59890962,664.36079062)
\curveto(867.60890297,664.42078726)(867.60890297,664.4807872)(867.59890962,664.54079062)
\curveto(867.59890298,664.60078708)(867.60390297,664.66078702)(867.61390962,664.72079062)
}
}
{
\newrgbcolor{curcolor}{0 0 0}
\pscustom[linestyle=none,fillstyle=solid,fillcolor=curcolor]
{
\newpath
\moveto(871.06015962,670.30079062)
\curveto(870.9801585,670.36078132)(870.93515855,670.46578121)(870.92515962,670.61579062)
\lineto(870.92515962,671.08079062)
\lineto(870.92515962,671.33579062)
\curveto(870.92515856,671.42578025)(870.94015854,671.50078018)(870.97015962,671.56079062)
\curveto(871.01015847,671.64078004)(871.09015839,671.70077998)(871.21015962,671.74079062)
\curveto(871.23015825,671.75077993)(871.25015823,671.75077993)(871.27015962,671.74079062)
\curveto(871.30015818,671.74077994)(871.32515816,671.74577993)(871.34515962,671.75579062)
\curveto(871.51515797,671.75577992)(871.67515781,671.75077993)(871.82515962,671.74079062)
\curveto(871.97515751,671.73077995)(872.07515741,671.67078001)(872.12515962,671.56079062)
\curveto(872.15515733,671.50078018)(872.17015731,671.42578025)(872.17015962,671.33579062)
\lineto(872.17015962,671.08079062)
\curveto(872.17015731,670.90078078)(872.16515732,670.73078095)(872.15515962,670.57079062)
\curveto(872.15515733,670.41078127)(872.09015739,670.30578137)(871.96015962,670.25579062)
\curveto(871.91015757,670.23578144)(871.85515763,670.22578145)(871.79515962,670.22579062)
\lineto(871.63015962,670.22579062)
\lineto(871.31515962,670.22579062)
\curveto(871.21515827,670.22578145)(871.13015835,670.25078143)(871.06015962,670.30079062)
\moveto(872.17015962,661.79579062)
\lineto(872.17015962,661.48079062)
\curveto(872.1801573,661.3807903)(872.16015732,661.30079038)(872.11015962,661.24079062)
\curveto(872.0801574,661.1807905)(872.03515745,661.14079054)(871.97515962,661.12079062)
\curveto(871.91515757,661.11079057)(871.84515764,661.09579058)(871.76515962,661.07579062)
\lineto(871.54015962,661.07579062)
\curveto(871.41015807,661.0757906)(871.29515819,661.0807906)(871.19515962,661.09079062)
\curveto(871.10515838,661.11079057)(871.03515845,661.16079052)(870.98515962,661.24079062)
\curveto(870.94515854,661.30079038)(870.92515856,661.3757903)(870.92515962,661.46579062)
\lineto(870.92515962,661.75079062)
\lineto(870.92515962,668.09579062)
\lineto(870.92515962,668.41079062)
\curveto(870.92515856,668.52078316)(870.95015853,668.60578307)(871.00015962,668.66579062)
\curveto(871.03015845,668.71578296)(871.07015841,668.74578293)(871.12015962,668.75579062)
\curveto(871.17015831,668.76578291)(871.22515826,668.7807829)(871.28515962,668.80079062)
\curveto(871.30515818,668.80078288)(871.32515816,668.79578288)(871.34515962,668.78579062)
\curveto(871.37515811,668.78578289)(871.40015808,668.79078289)(871.42015962,668.80079062)
\curveto(871.55015793,668.80078288)(871.6801578,668.79578288)(871.81015962,668.78579062)
\curveto(871.95015753,668.78578289)(872.04515744,668.74578293)(872.09515962,668.66579062)
\curveto(872.14515734,668.60578307)(872.17015731,668.52578315)(872.17015962,668.42579062)
\lineto(872.17015962,668.14079062)
\lineto(872.17015962,661.79579062)
}
}
{
\newrgbcolor{curcolor}{0 0 0}
\pscustom[linestyle=none,fillstyle=solid,fillcolor=curcolor]
{
\newpath
\moveto(881.00000337,661.63079062)
\curveto(881.02999554,661.47079021)(881.01499556,661.33579034)(880.95500337,661.22579062)
\curveto(880.89499568,661.12579055)(880.81499576,661.05079063)(880.71500337,661.00079062)
\curveto(880.66499591,660.9807907)(880.60999596,660.97079071)(880.55000337,660.97079062)
\curveto(880.49999607,660.97079071)(880.44499613,660.96079072)(880.38500337,660.94079062)
\curveto(880.16499641,660.89079079)(879.94499663,660.90579077)(879.72500337,660.98579062)
\curveto(879.51499706,661.05579062)(879.3699972,661.14579053)(879.29000337,661.25579062)
\curveto(879.23999733,661.32579035)(879.19499738,661.40579027)(879.15500337,661.49579062)
\curveto(879.11499746,661.59579008)(879.06499751,661.67579)(879.00500337,661.73579062)
\curveto(878.98499759,661.75578992)(878.95999761,661.7757899)(878.93000337,661.79579062)
\curveto(878.90999766,661.81578986)(878.87999769,661.82078986)(878.84000337,661.81079062)
\curveto(878.72999784,661.7807899)(878.62499795,661.72578995)(878.52500337,661.64579062)
\curveto(878.43499814,661.56579011)(878.34499823,661.49579018)(878.25500337,661.43579062)
\curveto(878.12499845,661.35579032)(877.98499859,661.2807904)(877.83500337,661.21079062)
\curveto(877.68499889,661.15079053)(877.52499905,661.09579058)(877.35500337,661.04579062)
\curveto(877.25499932,661.01579066)(877.14499943,660.99579068)(877.02500337,660.98579062)
\curveto(876.91499966,660.9757907)(876.80499977,660.96079072)(876.69500337,660.94079062)
\curveto(876.64499993,660.93079075)(876.59999997,660.92579075)(876.56000337,660.92579062)
\lineto(876.45500337,660.92579062)
\curveto(876.34500023,660.90579077)(876.24000033,660.90579077)(876.14000337,660.92579062)
\lineto(876.00500337,660.92579062)
\curveto(875.95500062,660.93579074)(875.90500067,660.94079074)(875.85500337,660.94079062)
\curveto(875.80500077,660.94079074)(875.76000081,660.95079073)(875.72000337,660.97079062)
\curveto(875.68000089,660.9807907)(875.64500093,660.98579069)(875.61500337,660.98579062)
\curveto(875.59500098,660.9757907)(875.570001,660.9757907)(875.54000337,660.98579062)
\lineto(875.30000337,661.04579062)
\curveto(875.22000135,661.05579062)(875.14500143,661.0757906)(875.07500337,661.10579062)
\curveto(874.7750018,661.23579044)(874.53000204,661.3807903)(874.34000337,661.54079062)
\curveto(874.16000241,661.71078997)(874.01000256,661.94578973)(873.89000337,662.24579062)
\curveto(873.80000277,662.46578921)(873.75500282,662.73078895)(873.75500337,663.04079062)
\lineto(873.75500337,663.35579062)
\curveto(873.76500281,663.40578827)(873.7700028,663.45578822)(873.77000337,663.50579062)
\lineto(873.80000337,663.68579062)
\lineto(873.92000337,664.01579062)
\curveto(873.96000261,664.12578755)(874.01000256,664.22578745)(874.07000337,664.31579062)
\curveto(874.25000232,664.60578707)(874.49500208,664.82078686)(874.80500337,664.96079062)
\curveto(875.11500146,665.10078658)(875.45500112,665.22578645)(875.82500337,665.33579062)
\curveto(875.96500061,665.3757863)(876.11000046,665.40578627)(876.26000337,665.42579062)
\curveto(876.41000016,665.44578623)(876.56000001,665.47078621)(876.71000337,665.50079062)
\curveto(876.77999979,665.52078616)(876.84499973,665.53078615)(876.90500337,665.53079062)
\curveto(876.9749996,665.53078615)(877.04999952,665.54078614)(877.13000337,665.56079062)
\curveto(877.19999937,665.5807861)(877.2699993,665.59078609)(877.34000337,665.59079062)
\curveto(877.40999916,665.60078608)(877.48499909,665.61578606)(877.56500337,665.63579062)
\curveto(877.81499876,665.69578598)(878.04999852,665.74578593)(878.27000337,665.78579062)
\curveto(878.48999808,665.83578584)(878.66499791,665.95078573)(878.79500337,666.13079062)
\curveto(878.85499772,666.21078547)(878.90499767,666.31078537)(878.94500337,666.43079062)
\curveto(878.98499759,666.56078512)(878.98499759,666.70078498)(878.94500337,666.85079062)
\curveto(878.88499769,667.09078459)(878.79499778,667.2807844)(878.67500337,667.42079062)
\curveto(878.56499801,667.56078412)(878.40499817,667.67078401)(878.19500337,667.75079062)
\curveto(878.0749985,667.80078388)(877.92999864,667.83578384)(877.76000337,667.85579062)
\curveto(877.59999897,667.8757838)(877.42999914,667.88578379)(877.25000337,667.88579062)
\curveto(877.0699995,667.88578379)(876.89499968,667.8757838)(876.72500337,667.85579062)
\curveto(876.55500002,667.83578384)(876.41000016,667.80578387)(876.29000337,667.76579062)
\curveto(876.12000045,667.70578397)(875.95500062,667.62078406)(875.79500337,667.51079062)
\curveto(875.71500086,667.45078423)(875.64000093,667.37078431)(875.57000337,667.27079062)
\curveto(875.51000106,667.1807845)(875.45500112,667.0807846)(875.40500337,666.97079062)
\curveto(875.3750012,666.89078479)(875.34500123,666.80578487)(875.31500337,666.71579062)
\curveto(875.29500128,666.62578505)(875.25000132,666.55578512)(875.18000337,666.50579062)
\curveto(875.14000143,666.4757852)(875.0700015,666.45078523)(874.97000337,666.43079062)
\curveto(874.88000169,666.42078526)(874.78500179,666.41578526)(874.68500337,666.41579062)
\curveto(874.58500199,666.41578526)(874.48500209,666.42078526)(874.38500337,666.43079062)
\curveto(874.29500228,666.45078523)(874.23000234,666.4757852)(874.19000337,666.50579062)
\curveto(874.15000242,666.53578514)(874.12000245,666.58578509)(874.10000337,666.65579062)
\curveto(874.08000249,666.72578495)(874.08000249,666.80078488)(874.10000337,666.88079062)
\curveto(874.13000244,667.01078467)(874.16000241,667.13078455)(874.19000337,667.24079062)
\curveto(874.23000234,667.36078432)(874.2750023,667.4757842)(874.32500337,667.58579062)
\curveto(874.51500206,667.93578374)(874.75500182,668.20578347)(875.04500337,668.39579062)
\curveto(875.33500124,668.59578308)(875.69500088,668.75578292)(876.12500337,668.87579062)
\curveto(876.22500035,668.89578278)(876.32500025,668.91078277)(876.42500337,668.92079062)
\curveto(876.53500004,668.93078275)(876.64499993,668.94578273)(876.75500337,668.96579062)
\curveto(876.79499978,668.9757827)(876.85999971,668.9757827)(876.95000337,668.96579062)
\curveto(877.03999953,668.96578271)(877.09499948,668.9757827)(877.11500337,668.99579062)
\curveto(877.81499876,669.00578267)(878.42499815,668.92578275)(878.94500337,668.75579062)
\curveto(879.46499711,668.58578309)(879.82999674,668.26078342)(880.04000337,667.78079062)
\curveto(880.12999644,667.5807841)(880.17999639,667.34578433)(880.19000337,667.07579062)
\curveto(880.20999636,666.81578486)(880.21999635,666.54078514)(880.22000337,666.25079062)
\lineto(880.22000337,662.93579062)
\curveto(880.21999635,662.79578888)(880.22499635,662.66078902)(880.23500337,662.53079062)
\curveto(880.24499633,662.40078928)(880.2749963,662.29578938)(880.32500337,662.21579062)
\curveto(880.3749962,662.14578953)(880.43999613,662.09578958)(880.52000337,662.06579062)
\curveto(880.60999596,662.02578965)(880.69499588,661.99578968)(880.77500337,661.97579062)
\curveto(880.85499572,661.96578971)(880.91499566,661.92078976)(880.95500337,661.84079062)
\curveto(880.9749956,661.81078987)(880.98499559,661.7807899)(880.98500337,661.75079062)
\curveto(880.98499559,661.72078996)(880.98999558,661.68079)(881.00000337,661.63079062)
\moveto(878.85500337,663.29579062)
\curveto(878.91499766,663.43578824)(878.94499763,663.59578808)(878.94500337,663.77579062)
\curveto(878.95499762,663.96578771)(878.95999761,664.16078752)(878.96000337,664.36079062)
\curveto(878.95999761,664.47078721)(878.95499762,664.57078711)(878.94500337,664.66079062)
\curveto(878.93499764,664.75078693)(878.89499768,664.82078686)(878.82500337,664.87079062)
\curveto(878.79499778,664.89078679)(878.72499785,664.90078678)(878.61500337,664.90079062)
\curveto(878.59499798,664.8807868)(878.55999801,664.87078681)(878.51000337,664.87079062)
\curveto(878.45999811,664.87078681)(878.41499816,664.86078682)(878.37500337,664.84079062)
\curveto(878.29499828,664.82078686)(878.20499837,664.80078688)(878.10500337,664.78079062)
\lineto(877.80500337,664.72079062)
\curveto(877.7749988,664.72078696)(877.73999883,664.71578696)(877.70000337,664.70579062)
\lineto(877.59500337,664.70579062)
\curveto(877.44499913,664.66578701)(877.27999929,664.64078704)(877.10000337,664.63079062)
\curveto(876.92999964,664.63078705)(876.7699998,664.61078707)(876.62000337,664.57079062)
\curveto(876.54000003,664.55078713)(876.46500011,664.53078715)(876.39500337,664.51079062)
\curveto(876.33500024,664.50078718)(876.26500031,664.48578719)(876.18500337,664.46579062)
\curveto(876.02500055,664.41578726)(875.8750007,664.35078733)(875.73500337,664.27079062)
\curveto(875.59500098,664.20078748)(875.4750011,664.11078757)(875.37500337,664.00079062)
\curveto(875.2750013,663.89078779)(875.20000137,663.75578792)(875.15000337,663.59579062)
\curveto(875.10000147,663.44578823)(875.08000149,663.26078842)(875.09000337,663.04079062)
\curveto(875.09000148,662.94078874)(875.10500147,662.84578883)(875.13500337,662.75579062)
\curveto(875.1750014,662.675789)(875.22000135,662.60078908)(875.27000337,662.53079062)
\curveto(875.35000122,662.42078926)(875.45500112,662.32578935)(875.58500337,662.24579062)
\curveto(875.71500086,662.1757895)(875.85500072,662.11578956)(876.00500337,662.06579062)
\curveto(876.05500052,662.05578962)(876.10500047,662.05078963)(876.15500337,662.05079062)
\curveto(876.20500037,662.05078963)(876.25500032,662.04578963)(876.30500337,662.03579062)
\curveto(876.3750002,662.01578966)(876.46000011,662.00078968)(876.56000337,661.99079062)
\curveto(876.6699999,661.99078969)(876.75999981,662.00078968)(876.83000337,662.02079062)
\curveto(876.88999968,662.04078964)(876.94999962,662.04578963)(877.01000337,662.03579062)
\curveto(877.0699995,662.03578964)(877.12999944,662.04578963)(877.19000337,662.06579062)
\curveto(877.2699993,662.08578959)(877.34499923,662.10078958)(877.41500337,662.11079062)
\curveto(877.49499908,662.12078956)(877.569999,662.14078954)(877.64000337,662.17079062)
\curveto(877.92999864,662.29078939)(878.1749984,662.43578924)(878.37500337,662.60579062)
\curveto(878.58499799,662.7757889)(878.74499783,663.00578867)(878.85500337,663.29579062)
}
}
{
\newrgbcolor{curcolor}{0 0 0}
\pscustom[linestyle=none,fillstyle=solid,fillcolor=curcolor]
{
\newpath
\moveto(885.861644,668.95079062)
\curveto(886.49163876,668.97078271)(886.99663826,668.88578279)(887.376644,668.69579062)
\curveto(887.7566375,668.50578317)(888.06163719,668.22078346)(888.291644,667.84079062)
\curveto(888.3516369,667.74078394)(888.39663686,667.63078405)(888.426644,667.51079062)
\curveto(888.46663679,667.40078428)(888.50163675,667.28578439)(888.531644,667.16579062)
\curveto(888.58163667,666.9757847)(888.61163664,666.77078491)(888.621644,666.55079062)
\curveto(888.63163662,666.33078535)(888.63663662,666.10578557)(888.636644,665.87579062)
\lineto(888.636644,664.27079062)
\lineto(888.636644,661.93079062)
\curveto(888.63663662,661.76078992)(888.63163662,661.59079009)(888.621644,661.42079062)
\curveto(888.62163663,661.25079043)(888.5566367,661.14079054)(888.426644,661.09079062)
\curveto(888.37663688,661.07079061)(888.32163693,661.06079062)(888.261644,661.06079062)
\curveto(888.21163704,661.05079063)(888.1566371,661.04579063)(888.096644,661.04579062)
\curveto(887.96663729,661.04579063)(887.84163741,661.05079063)(887.721644,661.06079062)
\curveto(887.60163765,661.06079062)(887.51663774,661.10079058)(887.466644,661.18079062)
\curveto(887.41663784,661.25079043)(887.39163786,661.34079034)(887.391644,661.45079062)
\lineto(887.391644,661.78079062)
\lineto(887.391644,663.07079062)
\lineto(887.391644,665.51579062)
\curveto(887.39163786,665.78578589)(887.38663787,666.05078563)(887.376644,666.31079062)
\curveto(887.36663789,666.5807851)(887.32163793,666.81078487)(887.241644,667.00079062)
\curveto(887.16163809,667.20078448)(887.04163821,667.36078432)(886.881644,667.48079062)
\curveto(886.72163853,667.61078407)(886.53663872,667.71078397)(886.326644,667.78079062)
\curveto(886.26663899,667.80078388)(886.20163905,667.81078387)(886.131644,667.81079062)
\curveto(886.07163918,667.82078386)(886.01163924,667.83578384)(885.951644,667.85579062)
\curveto(885.90163935,667.86578381)(885.82163943,667.86578381)(885.711644,667.85579062)
\curveto(885.61163964,667.85578382)(885.54163971,667.85078383)(885.501644,667.84079062)
\curveto(885.46163979,667.82078386)(885.42663983,667.81078387)(885.396644,667.81079062)
\curveto(885.36663989,667.82078386)(885.33163992,667.82078386)(885.291644,667.81079062)
\curveto(885.16164009,667.7807839)(885.03664022,667.74578393)(884.916644,667.70579062)
\curveto(884.80664045,667.675784)(884.70164055,667.63078405)(884.601644,667.57079062)
\curveto(884.56164069,667.55078413)(884.52664073,667.53078415)(884.496644,667.51079062)
\curveto(884.46664079,667.49078419)(884.43164082,667.47078421)(884.391644,667.45079062)
\curveto(884.04164121,667.20078448)(883.78664147,666.82578485)(883.626644,666.32579062)
\curveto(883.59664166,666.24578543)(883.57664168,666.16078552)(883.566644,666.07079062)
\curveto(883.5566417,665.99078569)(883.54164171,665.91078577)(883.521644,665.83079062)
\curveto(883.50164175,665.7807859)(883.49664176,665.73078595)(883.506644,665.68079062)
\curveto(883.51664174,665.64078604)(883.51164174,665.60078608)(883.491644,665.56079062)
\lineto(883.491644,665.24579062)
\curveto(883.48164177,665.21578646)(883.47664178,665.1807865)(883.476644,665.14079062)
\curveto(883.48664177,665.10078658)(883.49164176,665.05578662)(883.491644,665.00579062)
\lineto(883.491644,664.55579062)
\lineto(883.491644,663.11579062)
\lineto(883.491644,661.79579062)
\lineto(883.491644,661.45079062)
\curveto(883.49164176,661.34079034)(883.46664179,661.25079043)(883.416644,661.18079062)
\curveto(883.36664189,661.10079058)(883.27664198,661.06079062)(883.146644,661.06079062)
\curveto(883.02664223,661.05079063)(882.90164235,661.04579063)(882.771644,661.04579062)
\curveto(882.69164256,661.04579063)(882.61664264,661.05079063)(882.546644,661.06079062)
\curveto(882.47664278,661.07079061)(882.41664284,661.09579058)(882.366644,661.13579062)
\curveto(882.28664297,661.18579049)(882.24664301,661.2807904)(882.246644,661.42079062)
\lineto(882.246644,661.82579062)
\lineto(882.246644,663.59579062)
\lineto(882.246644,667.22579062)
\lineto(882.246644,668.14079062)
\lineto(882.246644,668.41079062)
\curveto(882.24664301,668.50078318)(882.26664299,668.57078311)(882.306644,668.62079062)
\curveto(882.33664292,668.680783)(882.38664287,668.72078296)(882.456644,668.74079062)
\curveto(882.49664276,668.75078293)(882.5516427,668.76078292)(882.621644,668.77079062)
\curveto(882.70164255,668.7807829)(882.78164247,668.78578289)(882.861644,668.78579062)
\curveto(882.94164231,668.78578289)(883.01664224,668.7807829)(883.086644,668.77079062)
\curveto(883.16664209,668.76078292)(883.22164203,668.74578293)(883.251644,668.72579062)
\curveto(883.36164189,668.65578302)(883.41164184,668.56578311)(883.401644,668.45579062)
\curveto(883.39164186,668.35578332)(883.40664185,668.24078344)(883.446644,668.11079062)
\curveto(883.46664179,668.05078363)(883.50664175,668.00078368)(883.566644,667.96079062)
\curveto(883.68664157,667.95078373)(883.78164147,667.99578368)(883.851644,668.09579062)
\curveto(883.93164132,668.19578348)(884.01164124,668.2757834)(884.091644,668.33579062)
\curveto(884.23164102,668.43578324)(884.37164088,668.52578315)(884.511644,668.60579062)
\curveto(884.66164059,668.69578298)(884.83164042,668.77078291)(885.021644,668.83079062)
\curveto(885.10164015,668.86078282)(885.18664007,668.8807828)(885.276644,668.89079062)
\curveto(885.37663988,668.90078278)(885.47163978,668.91578276)(885.561644,668.93579062)
\curveto(885.61163964,668.94578273)(885.66163959,668.95078273)(885.711644,668.95079062)
\lineto(885.861644,668.95079062)
}
}
{
\newrgbcolor{curcolor}{0 0 0}
\pscustom[linestyle=none,fillstyle=solid,fillcolor=curcolor]
{
\newpath
\moveto(891.46625337,671.14079062)
\curveto(891.61625136,671.14078054)(891.76625121,671.13578054)(891.91625337,671.12579062)
\curveto(892.06625091,671.12578055)(892.17125081,671.08578059)(892.23125337,671.00579062)
\curveto(892.2812507,670.94578073)(892.30625067,670.86078082)(892.30625337,670.75079062)
\curveto(892.31625066,670.65078103)(892.32125066,670.54578113)(892.32125337,670.43579062)
\lineto(892.32125337,669.56579062)
\curveto(892.32125066,669.48578219)(892.31625066,669.40078228)(892.30625337,669.31079062)
\curveto(892.30625067,669.23078245)(892.31625066,669.16078252)(892.33625337,669.10079062)
\curveto(892.3762506,668.96078272)(892.46625051,668.87078281)(892.60625337,668.83079062)
\curveto(892.65625032,668.82078286)(892.70125028,668.81578286)(892.74125337,668.81579062)
\lineto(892.89125337,668.81579062)
\lineto(893.29625337,668.81579062)
\curveto(893.45624952,668.82578285)(893.57124941,668.81578286)(893.64125337,668.78579062)
\curveto(893.73124925,668.72578295)(893.79124919,668.66578301)(893.82125337,668.60579062)
\curveto(893.84124914,668.56578311)(893.85124913,668.52078316)(893.85125337,668.47079062)
\lineto(893.85125337,668.32079062)
\curveto(893.85124913,668.21078347)(893.84624913,668.10578357)(893.83625337,668.00579062)
\curveto(893.82624915,667.91578376)(893.79124919,667.84578383)(893.73125337,667.79579062)
\curveto(893.67124931,667.74578393)(893.58624939,667.71578396)(893.47625337,667.70579062)
\lineto(893.14625337,667.70579062)
\curveto(893.03624994,667.71578396)(892.92625005,667.72078396)(892.81625337,667.72079062)
\curveto(892.70625027,667.72078396)(892.61125037,667.70578397)(892.53125337,667.67579062)
\curveto(892.46125052,667.64578403)(892.41125057,667.59578408)(892.38125337,667.52579062)
\curveto(892.35125063,667.45578422)(892.33125065,667.37078431)(892.32125337,667.27079062)
\curveto(892.31125067,667.1807845)(892.30625067,667.0807846)(892.30625337,666.97079062)
\curveto(892.31625066,666.87078481)(892.32125066,666.77078491)(892.32125337,666.67079062)
\lineto(892.32125337,663.70079062)
\curveto(892.32125066,663.4807882)(892.31625066,663.24578843)(892.30625337,662.99579062)
\curveto(892.30625067,662.75578892)(892.35125063,662.57078911)(892.44125337,662.44079062)
\curveto(892.49125049,662.36078932)(892.55625042,662.30578937)(892.63625337,662.27579062)
\curveto(892.71625026,662.24578943)(892.81125017,662.22078946)(892.92125337,662.20079062)
\curveto(892.95125003,662.19078949)(892.98125,662.18578949)(893.01125337,662.18579062)
\curveto(893.05124993,662.19578948)(893.08624989,662.19578948)(893.11625337,662.18579062)
\lineto(893.31125337,662.18579062)
\curveto(893.41124957,662.18578949)(893.50124948,662.1757895)(893.58125337,662.15579062)
\curveto(893.67124931,662.14578953)(893.73624924,662.11078957)(893.77625337,662.05079062)
\curveto(893.79624918,662.02078966)(893.81124917,661.96578971)(893.82125337,661.88579062)
\curveto(893.84124914,661.81578986)(893.85124913,661.74078994)(893.85125337,661.66079062)
\curveto(893.86124912,661.5807901)(893.86124912,661.50079018)(893.85125337,661.42079062)
\curveto(893.84124914,661.35079033)(893.82124916,661.29579038)(893.79125337,661.25579062)
\curveto(893.75124923,661.18579049)(893.6762493,661.13579054)(893.56625337,661.10579062)
\curveto(893.48624949,661.08579059)(893.39624958,661.0757906)(893.29625337,661.07579062)
\curveto(893.19624978,661.08579059)(893.10624987,661.09079059)(893.02625337,661.09079062)
\curveto(892.96625001,661.09079059)(892.90625007,661.08579059)(892.84625337,661.07579062)
\curveto(892.78625019,661.0757906)(892.73125025,661.0807906)(892.68125337,661.09079062)
\lineto(892.50125337,661.09079062)
\curveto(892.45125053,661.10079058)(892.40125058,661.10579057)(892.35125337,661.10579062)
\curveto(892.31125067,661.11579056)(892.26625071,661.12079056)(892.21625337,661.12079062)
\curveto(892.01625096,661.17079051)(891.84125114,661.22579045)(891.69125337,661.28579062)
\curveto(891.55125143,661.34579033)(891.43125155,661.45079023)(891.33125337,661.60079062)
\curveto(891.19125179,661.80078988)(891.11125187,662.05078963)(891.09125337,662.35079062)
\curveto(891.07125191,662.66078902)(891.06125192,662.99078869)(891.06125337,663.34079062)
\lineto(891.06125337,667.27079062)
\curveto(891.03125195,667.40078428)(891.00125198,667.49578418)(890.97125337,667.55579062)
\curveto(890.95125203,667.61578406)(890.8812521,667.66578401)(890.76125337,667.70579062)
\curveto(890.72125226,667.71578396)(890.6812523,667.71578396)(890.64125337,667.70579062)
\curveto(890.60125238,667.69578398)(890.56125242,667.70078398)(890.52125337,667.72079062)
\lineto(890.28125337,667.72079062)
\curveto(890.15125283,667.72078396)(890.04125294,667.73078395)(889.95125337,667.75079062)
\curveto(889.87125311,667.7807839)(889.81625316,667.84078384)(889.78625337,667.93079062)
\curveto(889.76625321,667.97078371)(889.75125323,668.01578366)(889.74125337,668.06579062)
\lineto(889.74125337,668.21579062)
\curveto(889.74125324,668.35578332)(889.75125323,668.47078321)(889.77125337,668.56079062)
\curveto(889.79125319,668.66078302)(889.85125313,668.73578294)(889.95125337,668.78579062)
\curveto(890.06125292,668.82578285)(890.20125278,668.83578284)(890.37125337,668.81579062)
\curveto(890.55125243,668.79578288)(890.70125228,668.80578287)(890.82125337,668.84579062)
\curveto(890.91125207,668.89578278)(890.981252,668.96578271)(891.03125337,669.05579062)
\curveto(891.05125193,669.11578256)(891.06125192,669.19078249)(891.06125337,669.28079062)
\lineto(891.06125337,669.53579062)
\lineto(891.06125337,670.46579062)
\lineto(891.06125337,670.70579062)
\curveto(891.06125192,670.79578088)(891.07125191,670.87078081)(891.09125337,670.93079062)
\curveto(891.13125185,671.01078067)(891.20625177,671.0757806)(891.31625337,671.12579062)
\curveto(891.34625163,671.12578055)(891.37125161,671.12578055)(891.39125337,671.12579062)
\curveto(891.42125156,671.13578054)(891.44625153,671.14078054)(891.46625337,671.14079062)
}
}
{
\newrgbcolor{curcolor}{0 0 0}
\pscustom[linestyle=none,fillstyle=solid,fillcolor=curcolor]
{
\newpath
\moveto(901.98805025,665.24579062)
\curveto(902.00804256,665.14578653)(902.00804256,665.03078665)(901.98805025,664.90079062)
\curveto(901.97804259,664.7807869)(901.94804262,664.69578698)(901.89805025,664.64579062)
\curveto(901.84804272,664.60578707)(901.7730428,664.5757871)(901.67305025,664.55579062)
\curveto(901.58304299,664.54578713)(901.47804309,664.54078714)(901.35805025,664.54079062)
\lineto(900.99805025,664.54079062)
\curveto(900.87804369,664.55078713)(900.7730438,664.55578712)(900.68305025,664.55579062)
\lineto(896.84305025,664.55579062)
\curveto(896.76304781,664.55578712)(896.68304789,664.55078713)(896.60305025,664.54079062)
\curveto(896.52304805,664.54078714)(896.45804811,664.52578715)(896.40805025,664.49579062)
\curveto(896.3680482,664.4757872)(896.32804824,664.43578724)(896.28805025,664.37579062)
\curveto(896.2680483,664.34578733)(896.24804832,664.30078738)(896.22805025,664.24079062)
\curveto(896.20804836,664.19078749)(896.20804836,664.14078754)(896.22805025,664.09079062)
\curveto(896.23804833,664.04078764)(896.24304833,663.99578768)(896.24305025,663.95579062)
\curveto(896.24304833,663.91578776)(896.24804832,663.8757878)(896.25805025,663.83579062)
\curveto(896.27804829,663.75578792)(896.29804827,663.67078801)(896.31805025,663.58079062)
\curveto(896.33804823,663.50078818)(896.3680482,663.42078826)(896.40805025,663.34079062)
\curveto(896.63804793,662.80078888)(897.01804755,662.41578926)(897.54805025,662.18579062)
\curveto(897.60804696,662.15578952)(897.6730469,662.13078955)(897.74305025,662.11079062)
\lineto(897.95305025,662.05079062)
\curveto(897.98304659,662.04078964)(898.03304654,662.03578964)(898.10305025,662.03579062)
\curveto(898.24304633,661.99578968)(898.42804614,661.9757897)(898.65805025,661.97579062)
\curveto(898.88804568,661.9757897)(899.0730455,661.99578968)(899.21305025,662.03579062)
\curveto(899.35304522,662.0757896)(899.47804509,662.11578956)(899.58805025,662.15579062)
\curveto(899.70804486,662.20578947)(899.81804475,662.26578941)(899.91805025,662.33579062)
\curveto(900.02804454,662.40578927)(900.12304445,662.48578919)(900.20305025,662.57579062)
\curveto(900.28304429,662.675789)(900.35304422,662.7807889)(900.41305025,662.89079062)
\curveto(900.4730441,662.99078869)(900.52304405,663.09578858)(900.56305025,663.20579062)
\curveto(900.61304396,663.31578836)(900.69304388,663.39578828)(900.80305025,663.44579062)
\curveto(900.84304373,663.46578821)(900.90804366,663.4807882)(900.99805025,663.49079062)
\curveto(901.08804348,663.50078818)(901.17804339,663.50078818)(901.26805025,663.49079062)
\curveto(901.35804321,663.49078819)(901.44304313,663.48578819)(901.52305025,663.47579062)
\curveto(901.60304297,663.46578821)(901.65804291,663.44578823)(901.68805025,663.41579062)
\curveto(901.78804278,663.34578833)(901.81304276,663.23078845)(901.76305025,663.07079062)
\curveto(901.68304289,662.80078888)(901.57804299,662.56078912)(901.44805025,662.35079062)
\curveto(901.24804332,662.03078965)(901.01804355,661.76578991)(900.75805025,661.55579062)
\curveto(900.50804406,661.35579032)(900.18804438,661.19079049)(899.79805025,661.06079062)
\curveto(899.69804487,661.02079066)(899.59804497,660.99579068)(899.49805025,660.98579062)
\curveto(899.39804517,660.96579071)(899.29304528,660.94579073)(899.18305025,660.92579062)
\curveto(899.13304544,660.91579076)(899.08304549,660.91079077)(899.03305025,660.91079062)
\curveto(898.99304558,660.91079077)(898.94804562,660.90579077)(898.89805025,660.89579062)
\lineto(898.74805025,660.89579062)
\curveto(898.69804587,660.88579079)(898.63804593,660.8807908)(898.56805025,660.88079062)
\curveto(898.50804606,660.8807908)(898.45804611,660.88579079)(898.41805025,660.89579062)
\lineto(898.28305025,660.89579062)
\curveto(898.23304634,660.90579077)(898.18804638,660.91079077)(898.14805025,660.91079062)
\curveto(898.10804646,660.91079077)(898.0680465,660.91579076)(898.02805025,660.92579062)
\curveto(897.97804659,660.93579074)(897.92304665,660.94579073)(897.86305025,660.95579062)
\curveto(897.80304677,660.95579072)(897.74804682,660.96079072)(897.69805025,660.97079062)
\curveto(897.60804696,660.99079069)(897.51804705,661.01579066)(897.42805025,661.04579062)
\curveto(897.33804723,661.06579061)(897.25304732,661.09079059)(897.17305025,661.12079062)
\curveto(897.13304744,661.14079054)(897.09804747,661.15079053)(897.06805025,661.15079062)
\curveto(897.03804753,661.16079052)(897.00304757,661.1757905)(896.96305025,661.19579062)
\curveto(896.81304776,661.26579041)(896.65304792,661.35079033)(896.48305025,661.45079062)
\curveto(896.19304838,661.64079004)(895.94304863,661.87078981)(895.73305025,662.14079062)
\curveto(895.53304904,662.42078926)(895.36304921,662.73078895)(895.22305025,663.07079062)
\curveto(895.1730494,663.1807885)(895.13304944,663.29578838)(895.10305025,663.41579062)
\curveto(895.08304949,663.53578814)(895.05304952,663.65578802)(895.01305025,663.77579062)
\curveto(895.00304957,663.81578786)(894.99804957,663.85078783)(894.99805025,663.88079062)
\curveto(894.99804957,663.91078777)(894.99304958,663.95078773)(894.98305025,664.00079062)
\curveto(894.96304961,664.0807876)(894.94804962,664.16578751)(894.93805025,664.25579062)
\curveto(894.92804964,664.34578733)(894.91304966,664.43578724)(894.89305025,664.52579062)
\lineto(894.89305025,664.73579062)
\curveto(894.88304969,664.7757869)(894.8730497,664.83078685)(894.86305025,664.90079062)
\curveto(894.86304971,664.9807867)(894.8680497,665.04578663)(894.87805025,665.09579062)
\lineto(894.87805025,665.26079062)
\curveto(894.89804967,665.31078637)(894.90304967,665.36078632)(894.89305025,665.41079062)
\curveto(894.89304968,665.47078621)(894.89804967,665.52578615)(894.90805025,665.57579062)
\curveto(894.94804962,665.73578594)(894.97804959,665.89578578)(894.99805025,666.05579062)
\curveto(895.02804954,666.21578546)(895.0730495,666.36578531)(895.13305025,666.50579062)
\curveto(895.18304939,666.61578506)(895.22804934,666.72578495)(895.26805025,666.83579062)
\curveto(895.31804925,666.95578472)(895.3730492,667.07078461)(895.43305025,667.18079062)
\curveto(895.65304892,667.53078415)(895.90304867,667.83078385)(896.18305025,668.08079062)
\curveto(896.46304811,668.34078334)(896.80804776,668.55578312)(897.21805025,668.72579062)
\curveto(897.33804723,668.7757829)(897.45804711,668.81078287)(897.57805025,668.83079062)
\curveto(897.70804686,668.86078282)(897.84304673,668.89078279)(897.98305025,668.92079062)
\curveto(898.03304654,668.93078275)(898.07804649,668.93578274)(898.11805025,668.93579062)
\curveto(898.15804641,668.94578273)(898.20304637,668.95078273)(898.25305025,668.95079062)
\curveto(898.2730463,668.96078272)(898.29804627,668.96078272)(898.32805025,668.95079062)
\curveto(898.35804621,668.94078274)(898.38304619,668.94578273)(898.40305025,668.96579062)
\curveto(898.82304575,668.9757827)(899.18804538,668.93078275)(899.49805025,668.83079062)
\curveto(899.80804476,668.74078294)(900.08804448,668.61578306)(900.33805025,668.45579062)
\curveto(900.38804418,668.43578324)(900.42804414,668.40578327)(900.45805025,668.36579062)
\curveto(900.48804408,668.33578334)(900.52304405,668.31078337)(900.56305025,668.29079062)
\curveto(900.64304393,668.23078345)(900.72304385,668.16078352)(900.80305025,668.08079062)
\curveto(900.89304368,668.00078368)(900.9680436,667.92078376)(901.02805025,667.84079062)
\curveto(901.18804338,667.63078405)(901.32304325,667.43078425)(901.43305025,667.24079062)
\curveto(901.50304307,667.13078455)(901.55804301,667.01078467)(901.59805025,666.88079062)
\curveto(901.63804293,666.75078493)(901.68304289,666.62078506)(901.73305025,666.49079062)
\curveto(901.78304279,666.36078532)(901.81804275,666.22578545)(901.83805025,666.08579062)
\curveto(901.8680427,665.94578573)(901.90304267,665.80578587)(901.94305025,665.66579062)
\curveto(901.95304262,665.59578608)(901.95804261,665.52578615)(901.95805025,665.45579062)
\lineto(901.98805025,665.24579062)
\moveto(900.53305025,665.75579062)
\curveto(900.56304401,665.79578588)(900.58804398,665.84578583)(900.60805025,665.90579062)
\curveto(900.62804394,665.9757857)(900.62804394,666.04578563)(900.60805025,666.11579062)
\curveto(900.54804402,666.33578534)(900.46304411,666.54078514)(900.35305025,666.73079062)
\curveto(900.21304436,666.96078472)(900.05804451,667.15578452)(899.88805025,667.31579062)
\curveto(899.71804485,667.4757842)(899.49804507,667.61078407)(899.22805025,667.72079062)
\curveto(899.15804541,667.74078394)(899.08804548,667.75578392)(899.01805025,667.76579062)
\curveto(898.94804562,667.78578389)(898.8730457,667.80578387)(898.79305025,667.82579062)
\curveto(898.71304586,667.84578383)(898.62804594,667.85578382)(898.53805025,667.85579062)
\lineto(898.28305025,667.85579062)
\curveto(898.25304632,667.83578384)(898.21804635,667.82578385)(898.17805025,667.82579062)
\curveto(898.13804643,667.83578384)(898.10304647,667.83578384)(898.07305025,667.82579062)
\lineto(897.83305025,667.76579062)
\curveto(897.76304681,667.75578392)(897.69304688,667.74078394)(897.62305025,667.72079062)
\curveto(897.33304724,667.60078408)(897.09804747,667.45078423)(896.91805025,667.27079062)
\curveto(896.74804782,667.09078459)(896.59304798,666.86578481)(896.45305025,666.59579062)
\curveto(896.42304815,666.54578513)(896.39304818,666.4807852)(896.36305025,666.40079062)
\curveto(896.33304824,666.33078535)(896.30804826,666.25078543)(896.28805025,666.16079062)
\curveto(896.2680483,666.07078561)(896.26304831,665.98578569)(896.27305025,665.90579062)
\curveto(896.28304829,665.82578585)(896.31804825,665.76578591)(896.37805025,665.72579062)
\curveto(896.45804811,665.66578601)(896.59304798,665.63578604)(896.78305025,665.63579062)
\curveto(896.98304759,665.64578603)(897.15304742,665.65078603)(897.29305025,665.65079062)
\lineto(899.57305025,665.65079062)
\curveto(899.72304485,665.65078603)(899.90304467,665.64578603)(900.11305025,665.63579062)
\curveto(900.32304425,665.63578604)(900.46304411,665.675786)(900.53305025,665.75579062)
}
}
{
\newrgbcolor{curcolor}{0.80000001 0.80000001 0.80000001}
\pscustom[linestyle=none,fillstyle=solid,fillcolor=curcolor]
{
\newpath
\moveto(812.80437349,671.78582724)
\lineto(827.80437349,671.78582724)
\lineto(827.80437349,656.78582724)
\lineto(812.80437349,656.78582724)
\closepath
}
}
{
\newrgbcolor{curcolor}{0 0 0}
\pscustom[linestyle=none,fillstyle=solid,fillcolor=curcolor]
{
\newpath
\moveto(840.7425815,638.99861288)
\curveto(840.76257195,638.94861214)(840.78757193,638.8886122)(840.8175815,638.81861288)
\curveto(840.84757187,638.74861234)(840.86757185,638.67361241)(840.8775815,638.59361288)
\curveto(840.89757182,638.52361256)(840.89757182,638.45361263)(840.8775815,638.38361288)
\curveto(840.86757185,638.32361276)(840.82757189,638.27861281)(840.7575815,638.24861288)
\curveto(840.70757201,638.22861286)(840.64757207,638.21861287)(840.5775815,638.21861288)
\lineto(840.3675815,638.21861288)
\lineto(839.9175815,638.21861288)
\curveto(839.76757295,638.21861287)(839.64757307,638.24361284)(839.5575815,638.29361288)
\curveto(839.45757326,638.35361273)(839.38257333,638.45861263)(839.3325815,638.60861288)
\curveto(839.29257342,638.75861233)(839.24757347,638.89361219)(839.1975815,639.01361288)
\curveto(839.08757363,639.27361181)(838.98757373,639.54361154)(838.8975815,639.82361288)
\curveto(838.80757391,640.10361098)(838.70757401,640.37861071)(838.5975815,640.64861288)
\curveto(838.56757415,640.73861035)(838.53757418,640.82361026)(838.5075815,640.90361288)
\curveto(838.48757423,640.9836101)(838.45757426,641.05861003)(838.4175815,641.12861288)
\curveto(838.38757433,641.19860989)(838.34257437,641.25860983)(838.2825815,641.30861288)
\curveto(838.22257449,641.35860973)(838.14257457,641.39860969)(838.0425815,641.42861288)
\curveto(837.99257472,641.44860964)(837.93257478,641.45360963)(837.8625815,641.44361288)
\lineto(837.6675815,641.44361288)
\lineto(834.8325815,641.44361288)
\lineto(834.5325815,641.44361288)
\curveto(834.42257829,641.45360963)(834.3175784,641.45360963)(834.2175815,641.44361288)
\curveto(834.1175786,641.43360965)(834.02257869,641.41860967)(833.9325815,641.39861288)
\curveto(833.85257886,641.37860971)(833.79257892,641.33860975)(833.7525815,641.27861288)
\curveto(833.67257904,641.17860991)(833.6125791,641.06361002)(833.5725815,640.93361288)
\curveto(833.54257917,640.81361027)(833.50257921,640.6886104)(833.4525815,640.55861288)
\curveto(833.35257936,640.32861076)(833.25757946,640.088611)(833.1675815,639.83861288)
\curveto(833.08757963,639.5886115)(832.99757972,639.34861174)(832.8975815,639.11861288)
\curveto(832.87757984,639.05861203)(832.85257986,638.9886121)(832.8225815,638.90861288)
\curveto(832.80257991,638.83861225)(832.77757994,638.76361232)(832.7475815,638.68361288)
\curveto(832.71758,638.60361248)(832.68258003,638.52861256)(832.6425815,638.45861288)
\curveto(832.6125801,638.39861269)(832.57758014,638.35361273)(832.5375815,638.32361288)
\curveto(832.45758026,638.26361282)(832.34758037,638.22861286)(832.2075815,638.21861288)
\lineto(831.7875815,638.21861288)
\lineto(831.5475815,638.21861288)
\curveto(831.47758124,638.22861286)(831.4175813,638.25361283)(831.3675815,638.29361288)
\curveto(831.3175814,638.32361276)(831.28758143,638.36861272)(831.2775815,638.42861288)
\curveto(831.27758144,638.4886126)(831.28258143,638.54861254)(831.2925815,638.60861288)
\curveto(831.3125814,638.67861241)(831.33258138,638.74361234)(831.3525815,638.80361288)
\curveto(831.38258133,638.87361221)(831.40758131,638.92361216)(831.4275815,638.95361288)
\curveto(831.56758115,639.27361181)(831.69258102,639.5886115)(831.8025815,639.89861288)
\curveto(831.9125808,640.21861087)(832.03258068,640.53861055)(832.1625815,640.85861288)
\curveto(832.25258046,641.07861001)(832.33758038,641.29360979)(832.4175815,641.50361288)
\curveto(832.49758022,641.72360936)(832.58258013,641.94360914)(832.6725815,642.16361288)
\curveto(832.97257974,642.8836082)(833.25757946,643.60860748)(833.5275815,644.33861288)
\curveto(833.79757892,645.07860601)(834.08257863,645.81360527)(834.3825815,646.54361288)
\curveto(834.49257822,646.80360428)(834.59257812,647.06860402)(834.6825815,647.33861288)
\curveto(834.78257793,647.60860348)(834.88757783,647.87360321)(834.9975815,648.13361288)
\curveto(835.04757767,648.24360284)(835.09257762,648.36360272)(835.1325815,648.49361288)
\curveto(835.18257753,648.63360245)(835.25257746,648.73360235)(835.3425815,648.79361288)
\curveto(835.38257733,648.83360225)(835.44757727,648.86360222)(835.5375815,648.88361288)
\curveto(835.55757716,648.89360219)(835.57757714,648.89360219)(835.5975815,648.88361288)
\curveto(835.62757709,648.8836022)(835.65257706,648.8886022)(835.6725815,648.89861288)
\curveto(835.85257686,648.89860219)(836.06257665,648.89860219)(836.3025815,648.89861288)
\curveto(836.54257617,648.90860218)(836.717576,648.87360221)(836.8275815,648.79361288)
\curveto(836.90757581,648.73360235)(836.96757575,648.63360245)(837.0075815,648.49361288)
\curveto(837.05757566,648.36360272)(837.10757561,648.24360284)(837.1575815,648.13361288)
\curveto(837.25757546,647.90360318)(837.34757537,647.67360341)(837.4275815,647.44361288)
\curveto(837.50757521,647.21360387)(837.59757512,646.9836041)(837.6975815,646.75361288)
\curveto(837.77757494,646.55360453)(837.85257486,646.34860474)(837.9225815,646.13861288)
\curveto(838.00257471,645.92860516)(838.08757463,645.72360536)(838.1775815,645.52361288)
\curveto(838.47757424,644.79360629)(838.76257395,644.05360703)(839.0325815,643.30361288)
\curveto(839.3125734,642.56360852)(839.60757311,641.82860926)(839.9175815,641.09861288)
\curveto(839.95757276,641.00861008)(839.98757273,640.92361016)(840.0075815,640.84361288)
\curveto(840.03757268,640.76361032)(840.06757265,640.67861041)(840.0975815,640.58861288)
\curveto(840.20757251,640.32861076)(840.3125724,640.06361102)(840.4125815,639.79361288)
\curveto(840.52257219,639.52361156)(840.63257208,639.25861183)(840.7425815,638.99861288)
\moveto(837.5325815,642.64361288)
\curveto(837.62257509,642.67360841)(837.67757504,642.72360836)(837.6975815,642.79361288)
\curveto(837.72757499,642.86360822)(837.73257498,642.93860815)(837.7125815,643.01861288)
\curveto(837.70257501,643.10860798)(837.67757504,643.19360789)(837.6375815,643.27361288)
\curveto(837.60757511,643.36360772)(837.57757514,643.43860765)(837.5475815,643.49861288)
\curveto(837.52757519,643.53860755)(837.5175752,643.57360751)(837.5175815,643.60361288)
\curveto(837.5175752,643.63360745)(837.50757521,643.66860742)(837.4875815,643.70861288)
\lineto(837.3975815,643.94861288)
\curveto(837.37757534,644.03860705)(837.34757537,644.12860696)(837.3075815,644.21861288)
\curveto(837.15757556,644.57860651)(837.02257569,644.94360614)(836.9025815,645.31361288)
\curveto(836.79257592,645.69360539)(836.66257605,646.06360502)(836.5125815,646.42361288)
\curveto(836.46257625,646.53360455)(836.4175763,646.64360444)(836.3775815,646.75361288)
\curveto(836.34757637,646.86360422)(836.30757641,646.96860412)(836.2575815,647.06861288)
\curveto(836.23757648,647.11860397)(836.2125765,647.16360392)(836.1825815,647.20361288)
\curveto(836.16257655,647.25360383)(836.1125766,647.27860381)(836.0325815,647.27861288)
\curveto(836.0125767,647.25860383)(835.99257672,647.24360384)(835.9725815,647.23361288)
\curveto(835.95257676,647.22360386)(835.93257678,647.20860388)(835.9125815,647.18861288)
\curveto(835.87257684,647.13860395)(835.84257687,647.083604)(835.8225815,647.02361288)
\curveto(835.80257691,646.97360411)(835.78257693,646.91860417)(835.7625815,646.85861288)
\curveto(835.712577,646.74860434)(835.67257704,646.63860445)(835.6425815,646.52861288)
\curveto(835.6125771,646.41860467)(835.57257714,646.30860478)(835.5225815,646.19861288)
\curveto(835.35257736,645.80860528)(835.20257751,645.41360567)(835.0725815,645.01361288)
\curveto(834.95257776,644.61360647)(834.8125779,644.22360686)(834.6525815,643.84361288)
\lineto(834.5925815,643.69361288)
\curveto(834.58257813,643.64360744)(834.56757815,643.59360749)(834.5475815,643.54361288)
\lineto(834.4575815,643.30361288)
\curveto(834.42757829,643.22360786)(834.40257831,643.14360794)(834.3825815,643.06361288)
\curveto(834.36257835,643.01360807)(834.35257836,642.95860813)(834.3525815,642.89861288)
\curveto(834.36257835,642.83860825)(834.37757834,642.7886083)(834.3975815,642.74861288)
\curveto(834.44757827,642.66860842)(834.55257816,642.62360846)(834.7125815,642.61361288)
\lineto(835.1625815,642.61361288)
\lineto(836.7675815,642.61361288)
\curveto(836.87757584,642.61360847)(837.0125757,642.60860848)(837.1725815,642.59861288)
\curveto(837.33257538,642.59860849)(837.45257526,642.61360847)(837.5325815,642.64361288)
}
}
{
\newrgbcolor{curcolor}{0 0 0}
\pscustom[linestyle=none,fillstyle=solid,fillcolor=curcolor]
{
\newpath
\moveto(842.324144,645.94361288)
\lineto(842.759144,645.94361288)
\curveto(842.90914203,645.94360514)(843.01414193,645.90360518)(843.074144,645.82361288)
\curveto(843.12414182,645.74360534)(843.14914179,645.64360544)(843.149144,645.52361288)
\curveto(843.15914178,645.40360568)(843.16414178,645.2836058)(843.164144,645.16361288)
\lineto(843.164144,643.73861288)
\lineto(843.164144,641.47361288)
\lineto(843.164144,640.78361288)
\curveto(843.16414178,640.55361053)(843.18914175,640.35361073)(843.239144,640.18361288)
\curveto(843.39914154,639.73361135)(843.69914124,639.41861167)(844.139144,639.23861288)
\curveto(844.35914058,639.14861194)(844.62414032,639.11361197)(844.934144,639.13361288)
\curveto(845.2441397,639.16361192)(845.49413945,639.21861187)(845.684144,639.29861288)
\curveto(846.01413893,639.43861165)(846.27413867,639.61361147)(846.464144,639.82361288)
\curveto(846.66413828,640.04361104)(846.81913812,640.32861076)(846.929144,640.67861288)
\curveto(846.95913798,640.75861033)(846.97913796,640.83861025)(846.989144,640.91861288)
\curveto(846.99913794,640.99861009)(847.01413793,641.08361)(847.034144,641.17361288)
\curveto(847.0441379,641.22360986)(847.0441379,641.26860982)(847.034144,641.30861288)
\curveto(847.03413791,641.34860974)(847.0441379,641.39360969)(847.064144,641.44361288)
\lineto(847.064144,641.75861288)
\curveto(847.08413786,641.83860925)(847.08913785,641.92860916)(847.079144,642.02861288)
\curveto(847.06913787,642.13860895)(847.06413788,642.23860885)(847.064144,642.32861288)
\lineto(847.064144,643.49861288)
\lineto(847.064144,645.08861288)
\curveto(847.06413788,645.20860588)(847.05913788,645.33360575)(847.049144,645.46361288)
\curveto(847.04913789,645.60360548)(847.07413787,645.71360537)(847.124144,645.79361288)
\curveto(847.16413778,645.84360524)(847.20913773,645.87360521)(847.259144,645.88361288)
\curveto(847.31913762,645.90360518)(847.38913755,645.92360516)(847.469144,645.94361288)
\lineto(847.694144,645.94361288)
\curveto(847.81413713,645.94360514)(847.91913702,645.93860515)(848.009144,645.92861288)
\curveto(848.10913683,645.91860517)(848.18413676,645.87360521)(848.234144,645.79361288)
\curveto(848.28413666,645.74360534)(848.30913663,645.66860542)(848.309144,645.56861288)
\lineto(848.309144,645.28361288)
\lineto(848.309144,644.26361288)
\lineto(848.309144,640.22861288)
\lineto(848.309144,638.87861288)
\curveto(848.30913663,638.75861233)(848.30413664,638.64361244)(848.294144,638.53361288)
\curveto(848.29413665,638.43361265)(848.25913668,638.35861273)(848.189144,638.30861288)
\curveto(848.14913679,638.27861281)(848.08913685,638.25361283)(848.009144,638.23361288)
\curveto(847.92913701,638.22361286)(847.8391371,638.21361287)(847.739144,638.20361288)
\curveto(847.64913729,638.20361288)(847.55913738,638.20861288)(847.469144,638.21861288)
\curveto(847.38913755,638.22861286)(847.32913761,638.24861284)(847.289144,638.27861288)
\curveto(847.2391377,638.31861277)(847.19413775,638.3836127)(847.154144,638.47361288)
\curveto(847.1441378,638.51361257)(847.13413781,638.56861252)(847.124144,638.63861288)
\curveto(847.12413782,638.70861238)(847.11913782,638.77361231)(847.109144,638.83361288)
\curveto(847.09913784,638.90361218)(847.07913786,638.95861213)(847.049144,638.99861288)
\curveto(847.01913792,639.03861205)(846.97413797,639.05361203)(846.914144,639.04361288)
\curveto(846.83413811,639.02361206)(846.75413819,638.96361212)(846.674144,638.86361288)
\curveto(846.59413835,638.77361231)(846.51913842,638.70361238)(846.449144,638.65361288)
\curveto(846.22913871,638.49361259)(845.97913896,638.35361273)(845.699144,638.23361288)
\curveto(845.58913935,638.1836129)(845.47413947,638.15361293)(845.354144,638.14361288)
\curveto(845.2441397,638.12361296)(845.12913981,638.09861299)(845.009144,638.06861288)
\curveto(844.95913998,638.05861303)(844.90414004,638.05861303)(844.844144,638.06861288)
\curveto(844.79414015,638.07861301)(844.7441402,638.07361301)(844.694144,638.05361288)
\curveto(844.59414035,638.03361305)(844.50414044,638.03361305)(844.424144,638.05361288)
\lineto(844.274144,638.05361288)
\curveto(844.22414072,638.07361301)(844.16414078,638.083613)(844.094144,638.08361288)
\curveto(844.03414091,638.083613)(843.97914096,638.088613)(843.929144,638.09861288)
\curveto(843.88914105,638.11861297)(843.84914109,638.12861296)(843.809144,638.12861288)
\curveto(843.77914116,638.11861297)(843.7391412,638.12361296)(843.689144,638.14361288)
\lineto(843.449144,638.20361288)
\curveto(843.37914156,638.22361286)(843.30414164,638.25361283)(843.224144,638.29361288)
\curveto(842.96414198,638.40361268)(842.7441422,638.54861254)(842.564144,638.72861288)
\curveto(842.39414255,638.91861217)(842.25414269,639.14361194)(842.144144,639.40361288)
\curveto(842.10414284,639.49361159)(842.07414287,639.5836115)(842.054144,639.67361288)
\lineto(841.994144,639.97361288)
\curveto(841.97414297,640.03361105)(841.96414298,640.088611)(841.964144,640.13861288)
\curveto(841.97414297,640.19861089)(841.96914297,640.26361082)(841.949144,640.33361288)
\curveto(841.939143,640.35361073)(841.93414301,640.37861071)(841.934144,640.40861288)
\curveto(841.93414301,640.44861064)(841.92914301,640.4836106)(841.919144,640.51361288)
\lineto(841.919144,640.66361288)
\curveto(841.90914303,640.70361038)(841.90414304,640.74861034)(841.904144,640.79861288)
\curveto(841.91414303,640.85861023)(841.91914302,640.91361017)(841.919144,640.96361288)
\lineto(841.919144,641.56361288)
\lineto(841.919144,644.32361288)
\lineto(841.919144,645.28361288)
\lineto(841.919144,645.55361288)
\curveto(841.91914302,645.64360544)(841.939143,645.71860537)(841.979144,645.77861288)
\curveto(842.01914292,645.84860524)(842.09414285,645.89860519)(842.204144,645.92861288)
\curveto(842.22414272,645.93860515)(842.2441427,645.93860515)(842.264144,645.92861288)
\curveto(842.28414266,645.92860516)(842.30414264,645.93360515)(842.324144,645.94361288)
}
}
{
\newrgbcolor{curcolor}{0 0 0}
\pscustom[linestyle=none,fillstyle=solid,fillcolor=curcolor]
{
\newpath
\moveto(850.22375337,645.94361288)
\lineto(850.74875337,645.94361288)
\curveto(850.94875172,645.95360513)(851.09875157,645.93360515)(851.19875337,645.88361288)
\curveto(851.31875135,645.83360525)(851.41375125,645.75360533)(851.48375337,645.64361288)
\curveto(851.5637511,645.53360555)(851.63875103,645.42360566)(851.70875337,645.31361288)
\curveto(851.83875083,645.11360597)(851.9687507,644.91860617)(852.09875337,644.72861288)
\curveto(852.22875044,644.54860654)(852.3637503,644.35860673)(852.50375337,644.15861288)
\curveto(852.55375011,644.07860701)(852.60375006,644.00360708)(852.65375337,643.93361288)
\curveto(852.71374995,643.86360722)(852.7687499,643.79360729)(852.81875337,643.72361288)
\curveto(852.85874981,643.66360742)(852.89874977,643.60860748)(852.93875337,643.55861288)
\curveto(852.97874969,643.50860758)(853.03874963,643.47360761)(853.11875337,643.45361288)
\curveto(853.1687495,643.43360765)(853.20874946,643.43360765)(853.23875337,643.45361288)
\curveto(853.27874939,643.4836076)(853.30874936,643.50860758)(853.32875337,643.52861288)
\curveto(853.40874926,643.57860751)(853.47374919,643.64860744)(853.52375337,643.73861288)
\curveto(853.58374908,643.82860726)(853.63874903,643.91360717)(853.68875337,643.99361288)
\curveto(853.83874883,644.19360689)(853.98874868,644.39860669)(854.13875337,644.60861288)
\lineto(854.58875337,645.23861288)
\curveto(854.668748,645.34860574)(854.74874792,645.46360562)(854.82875337,645.58361288)
\curveto(854.90874776,645.70360538)(855.00374766,645.79860529)(855.11375337,645.86861288)
\curveto(855.19374747,645.91860517)(855.28874738,645.94360514)(855.39875337,645.94361288)
\lineto(855.74375337,645.94361288)
\lineto(855.87875337,645.94361288)
\curveto(855.92874674,645.94360514)(855.97874669,645.93860515)(856.02875337,645.92861288)
\lineto(856.10375337,645.92861288)
\curveto(856.22374644,645.90860518)(856.30374636,645.86860522)(856.34375337,645.80861288)
\curveto(856.3637463,645.75860533)(856.35874631,645.70360538)(856.32875337,645.64361288)
\curveto(856.30874636,645.59360549)(856.28874638,645.55360553)(856.26875337,645.52361288)
\lineto(856.05875337,645.22361288)
\curveto(855.98874668,645.13360595)(855.91374675,645.03860605)(855.83375337,644.93861288)
\curveto(855.60374706,644.61860647)(855.3687473,644.30360678)(855.12875337,643.99361288)
\curveto(854.89874777,643.69360739)(854.668748,643.3836077)(854.43875337,643.06361288)
\curveto(854.38874828,642.9836081)(854.33374833,642.90360818)(854.27375337,642.82361288)
\curveto(854.21374845,642.75360833)(854.15874851,642.67360841)(854.10875337,642.58361288)
\curveto(854.08874858,642.55360853)(854.0687486,642.51360857)(854.04875337,642.46361288)
\curveto(854.02874864,642.42360866)(854.02874864,642.37360871)(854.04875337,642.31361288)
\curveto(854.0687486,642.22360886)(854.09874857,642.14860894)(854.13875337,642.08861288)
\curveto(854.18874848,642.02860906)(854.23874843,641.96360912)(854.28875337,641.89361288)
\lineto(854.46875337,641.62361288)
\curveto(854.53874813,641.53360955)(854.60374806,641.44360964)(854.66375337,641.35361288)
\lineto(855.35375337,640.39361288)
\lineto(856.04375337,639.43361288)
\curveto(856.12374654,639.32361176)(856.20374646,639.20861188)(856.28375337,639.08861288)
\lineto(856.52375337,638.75861288)
\curveto(856.57374609,638.6886124)(856.61374605,638.62361246)(856.64375337,638.56361288)
\curveto(856.68374598,638.51361257)(856.69374597,638.43361265)(856.67375337,638.32361288)
\curveto(856.65374601,638.31361277)(856.63374603,638.29861279)(856.61375337,638.27861288)
\curveto(856.60374606,638.26861282)(856.58874608,638.25861283)(856.56875337,638.24861288)
\curveto(856.51874615,638.22861286)(856.45374621,638.21861287)(856.37375337,638.21861288)
\lineto(856.13375337,638.21861288)
\lineto(855.62375337,638.21861288)
\curveto(855.48374718,638.22861286)(855.35874731,638.27361281)(855.24875337,638.35361288)
\curveto(855.19874747,638.3836127)(855.15874751,638.41861267)(855.12875337,638.45861288)
\curveto(855.10874756,638.50861258)(855.08374758,638.55861253)(855.05375337,638.60861288)
\lineto(854.90375337,638.81861288)
\curveto(854.85374781,638.8886122)(854.80374786,638.96361212)(854.75375337,639.04361288)
\lineto(853.80875337,640.43861288)
\curveto(853.75874891,640.51861057)(853.70874896,640.59361049)(853.65875337,640.66361288)
\curveto(853.60874906,640.73361035)(853.55874911,640.80861028)(853.50875337,640.88861288)
\curveto(853.45874921,640.95861013)(853.40874926,641.01861007)(853.35875337,641.06861288)
\curveto(853.31874935,641.12860996)(853.25874941,641.16860992)(853.17875337,641.18861288)
\curveto(853.12874954,641.20860988)(853.07874959,641.19860989)(853.02875337,641.15861288)
\curveto(852.98874968,641.12860996)(852.95874971,641.10360998)(852.93875337,641.08361288)
\curveto(852.85874981,641.00361008)(852.78874988,640.91361017)(852.72875337,640.81361288)
\curveto(852.66875,640.71361037)(852.60875006,640.61861047)(852.54875337,640.52861288)
\curveto(852.37875029,640.26861082)(852.20375046,640.00861108)(852.02375337,639.74861288)
\curveto(851.85375081,639.49861159)(851.67875099,639.24861184)(851.49875337,638.99861288)
\curveto(851.44875122,638.91861217)(851.39375127,638.83861225)(851.33375337,638.75861288)
\lineto(851.18375337,638.51861288)
\curveto(851.1637515,638.4886126)(851.13875153,638.45361263)(851.10875337,638.41361288)
\curveto(851.08875158,638.3836127)(851.0637516,638.35861273)(851.03375337,638.33861288)
\curveto(850.93375173,638.26861282)(850.81375185,638.22861286)(850.67375337,638.21861288)
\lineto(850.22375337,638.21861288)
\lineto(849.99875337,638.21861288)
\curveto(849.92875274,638.21861287)(849.8687528,638.22861286)(849.81875337,638.24861288)
\curveto(849.78875288,638.26861282)(849.7637529,638.2836128)(849.74375337,638.29361288)
\curveto(849.73375293,638.31361277)(849.71875295,638.33361275)(849.69875337,638.35361288)
\curveto(849.68875298,638.46361262)(849.70375296,638.54861254)(849.74375337,638.60861288)
\curveto(849.79375287,638.66861242)(849.84375282,638.73361235)(849.89375337,638.80361288)
\curveto(849.97375269,638.91361217)(850.04875262,639.01361207)(850.11875337,639.10361288)
\curveto(850.18875248,639.20361188)(850.25875241,639.30861178)(850.32875337,639.41861288)
\curveto(850.54875212,639.71861137)(850.7637519,640.01861107)(850.97375337,640.31861288)
\lineto(851.60375337,641.21861288)
\curveto(851.67375099,641.30860978)(851.73875093,641.39860969)(851.79875337,641.48861288)
\curveto(851.8687508,641.57860951)(851.93375073,641.67360941)(851.99375337,641.77361288)
\curveto(852.04375062,641.84360924)(852.09375057,641.90860918)(852.14375337,641.96861288)
\curveto(852.19375047,642.03860905)(852.22875044,642.12860896)(852.24875337,642.23861288)
\curveto(852.2687504,642.2886088)(852.2637504,642.33860875)(852.23375337,642.38861288)
\curveto(852.21375045,642.43860865)(852.19375047,642.47860861)(852.17375337,642.50861288)
\curveto(852.12375054,642.59860849)(852.0687506,642.6836084)(852.00875337,642.76361288)
\lineto(851.82875337,643.00361288)
\curveto(851.59875107,643.32360776)(851.3637513,643.64360744)(851.12375337,643.96361288)
\lineto(850.43375337,644.92361288)
\curveto(850.35375231,645.03360605)(850.27375239,645.13360595)(850.19375337,645.22361288)
\curveto(850.12375254,645.31360577)(850.05375261,645.41360567)(849.98375337,645.52361288)
\curveto(849.9637527,645.55360553)(849.94375272,645.59360549)(849.92375337,645.64361288)
\curveto(849.90375276,645.70360538)(849.90375276,645.75360533)(849.92375337,645.79361288)
\curveto(849.94375272,645.84360524)(849.97375269,645.87360521)(850.01375337,645.88361288)
\curveto(850.05375261,645.90360518)(850.09875257,645.91860517)(850.14875337,645.92861288)
\curveto(850.1687525,645.93860515)(850.18375248,645.93860515)(850.19375337,645.92861288)
\curveto(850.20375246,645.92860516)(850.21375245,645.93360515)(850.22375337,645.94361288)
}
}
{
\newrgbcolor{curcolor}{0 0 0}
\pscustom[linestyle=none,fillstyle=solid,fillcolor=curcolor]
{
\newpath
\moveto(858.25742525,647.44361288)
\curveto(858.17742413,647.50360358)(858.13242417,647.60860348)(858.12242525,647.75861288)
\lineto(858.12242525,648.22361288)
\lineto(858.12242525,648.47861288)
\curveto(858.12242418,648.56860252)(858.13742417,648.64360244)(858.16742525,648.70361288)
\curveto(858.2074241,648.7836023)(858.28742402,648.84360224)(858.40742525,648.88361288)
\curveto(858.42742388,648.89360219)(858.44742386,648.89360219)(858.46742525,648.88361288)
\curveto(858.49742381,648.8836022)(858.52242378,648.8886022)(858.54242525,648.89861288)
\curveto(858.71242359,648.89860219)(858.87242343,648.89360219)(859.02242525,648.88361288)
\curveto(859.17242313,648.87360221)(859.27242303,648.81360227)(859.32242525,648.70361288)
\curveto(859.35242295,648.64360244)(859.36742294,648.56860252)(859.36742525,648.47861288)
\lineto(859.36742525,648.22361288)
\curveto(859.36742294,648.04360304)(859.36242294,647.87360321)(859.35242525,647.71361288)
\curveto(859.35242295,647.55360353)(859.28742302,647.44860364)(859.15742525,647.39861288)
\curveto(859.1074232,647.37860371)(859.05242325,647.36860372)(858.99242525,647.36861288)
\lineto(858.82742525,647.36861288)
\lineto(858.51242525,647.36861288)
\curveto(858.41242389,647.36860372)(858.32742398,647.39360369)(858.25742525,647.44361288)
\moveto(859.36742525,638.93861288)
\lineto(859.36742525,638.62361288)
\curveto(859.37742293,638.52361256)(859.35742295,638.44361264)(859.30742525,638.38361288)
\curveto(859.27742303,638.32361276)(859.23242307,638.2836128)(859.17242525,638.26361288)
\curveto(859.11242319,638.25361283)(859.04242326,638.23861285)(858.96242525,638.21861288)
\lineto(858.73742525,638.21861288)
\curveto(858.6074237,638.21861287)(858.49242381,638.22361286)(858.39242525,638.23361288)
\curveto(858.302424,638.25361283)(858.23242407,638.30361278)(858.18242525,638.38361288)
\curveto(858.14242416,638.44361264)(858.12242418,638.51861257)(858.12242525,638.60861288)
\lineto(858.12242525,638.89361288)
\lineto(858.12242525,645.23861288)
\lineto(858.12242525,645.55361288)
\curveto(858.12242418,645.66360542)(858.14742416,645.74860534)(858.19742525,645.80861288)
\curveto(858.22742408,645.85860523)(858.26742404,645.8886052)(858.31742525,645.89861288)
\curveto(858.36742394,645.90860518)(858.42242388,645.92360516)(858.48242525,645.94361288)
\curveto(858.5024238,645.94360514)(858.52242378,645.93860515)(858.54242525,645.92861288)
\curveto(858.57242373,645.92860516)(858.59742371,645.93360515)(858.61742525,645.94361288)
\curveto(858.74742356,645.94360514)(858.87742343,645.93860515)(859.00742525,645.92861288)
\curveto(859.14742316,645.92860516)(859.24242306,645.8886052)(859.29242525,645.80861288)
\curveto(859.34242296,645.74860534)(859.36742294,645.66860542)(859.36742525,645.56861288)
\lineto(859.36742525,645.28361288)
\lineto(859.36742525,638.93861288)
}
}
{
\newrgbcolor{curcolor}{0 0 0}
\pscustom[linestyle=none,fillstyle=solid,fillcolor=curcolor]
{
\newpath
\moveto(861.882269,648.89861288)
\curveto(862.01226738,648.89860219)(862.14726725,648.89860219)(862.287269,648.89861288)
\curveto(862.43726696,648.89860219)(862.54726685,648.86360222)(862.617269,648.79361288)
\curveto(862.66726673,648.72360236)(862.6922667,648.62860246)(862.692269,648.50861288)
\curveto(862.70226669,648.39860269)(862.70726669,648.2836028)(862.707269,648.16361288)
\lineto(862.707269,646.82861288)
\lineto(862.707269,640.75361288)
\lineto(862.707269,639.07361288)
\lineto(862.707269,638.68361288)
\curveto(862.70726669,638.54361254)(862.68226671,638.43361265)(862.632269,638.35361288)
\curveto(862.60226679,638.30361278)(862.55726684,638.27361281)(862.497269,638.26361288)
\curveto(862.44726695,638.25361283)(862.38226701,638.23861285)(862.302269,638.21861288)
\lineto(862.092269,638.21861288)
\lineto(861.777269,638.21861288)
\curveto(861.67726772,638.22861286)(861.60226779,638.26361282)(861.552269,638.32361288)
\curveto(861.50226789,638.40361268)(861.47226792,638.50361258)(861.462269,638.62361288)
\lineto(861.462269,638.99861288)
\lineto(861.462269,640.37861288)
\lineto(861.462269,646.61861288)
\lineto(861.462269,648.08861288)
\curveto(861.46226793,648.19860289)(861.45726794,648.31360277)(861.447269,648.43361288)
\curveto(861.44726795,648.56360252)(861.47226792,648.66360242)(861.522269,648.73361288)
\curveto(861.56226783,648.79360229)(861.63726776,648.84360224)(861.747269,648.88361288)
\curveto(861.76726763,648.89360219)(861.78726761,648.89360219)(861.807269,648.88361288)
\curveto(861.83726756,648.8836022)(861.86226753,648.8886022)(861.882269,648.89861288)
}
}
{
\newrgbcolor{curcolor}{0 0 0}
\pscustom[linestyle=none,fillstyle=solid,fillcolor=curcolor]
{
\newpath
\moveto(864.93711275,647.44361288)
\curveto(864.85711163,647.50360358)(864.81211167,647.60860348)(864.80211275,647.75861288)
\lineto(864.80211275,648.22361288)
\lineto(864.80211275,648.47861288)
\curveto(864.80211168,648.56860252)(864.81711167,648.64360244)(864.84711275,648.70361288)
\curveto(864.8871116,648.7836023)(864.96711152,648.84360224)(865.08711275,648.88361288)
\curveto(865.10711138,648.89360219)(865.12711136,648.89360219)(865.14711275,648.88361288)
\curveto(865.17711131,648.8836022)(865.20211128,648.8886022)(865.22211275,648.89861288)
\curveto(865.39211109,648.89860219)(865.55211093,648.89360219)(865.70211275,648.88361288)
\curveto(865.85211063,648.87360221)(865.95211053,648.81360227)(866.00211275,648.70361288)
\curveto(866.03211045,648.64360244)(866.04711044,648.56860252)(866.04711275,648.47861288)
\lineto(866.04711275,648.22361288)
\curveto(866.04711044,648.04360304)(866.04211044,647.87360321)(866.03211275,647.71361288)
\curveto(866.03211045,647.55360353)(865.96711052,647.44860364)(865.83711275,647.39861288)
\curveto(865.7871107,647.37860371)(865.73211075,647.36860372)(865.67211275,647.36861288)
\lineto(865.50711275,647.36861288)
\lineto(865.19211275,647.36861288)
\curveto(865.09211139,647.36860372)(865.00711148,647.39360369)(864.93711275,647.44361288)
\moveto(866.04711275,638.93861288)
\lineto(866.04711275,638.62361288)
\curveto(866.05711043,638.52361256)(866.03711045,638.44361264)(865.98711275,638.38361288)
\curveto(865.95711053,638.32361276)(865.91211057,638.2836128)(865.85211275,638.26361288)
\curveto(865.79211069,638.25361283)(865.72211076,638.23861285)(865.64211275,638.21861288)
\lineto(865.41711275,638.21861288)
\curveto(865.2871112,638.21861287)(865.17211131,638.22361286)(865.07211275,638.23361288)
\curveto(864.9821115,638.25361283)(864.91211157,638.30361278)(864.86211275,638.38361288)
\curveto(864.82211166,638.44361264)(864.80211168,638.51861257)(864.80211275,638.60861288)
\lineto(864.80211275,638.89361288)
\lineto(864.80211275,645.23861288)
\lineto(864.80211275,645.55361288)
\curveto(864.80211168,645.66360542)(864.82711166,645.74860534)(864.87711275,645.80861288)
\curveto(864.90711158,645.85860523)(864.94711154,645.8886052)(864.99711275,645.89861288)
\curveto(865.04711144,645.90860518)(865.10211138,645.92360516)(865.16211275,645.94361288)
\curveto(865.1821113,645.94360514)(865.20211128,645.93860515)(865.22211275,645.92861288)
\curveto(865.25211123,645.92860516)(865.27711121,645.93360515)(865.29711275,645.94361288)
\curveto(865.42711106,645.94360514)(865.55711093,645.93860515)(865.68711275,645.92861288)
\curveto(865.82711066,645.92860516)(865.92211056,645.8886052)(865.97211275,645.80861288)
\curveto(866.02211046,645.74860534)(866.04711044,645.66860542)(866.04711275,645.56861288)
\lineto(866.04711275,645.28361288)
\lineto(866.04711275,638.93861288)
}
}
{
\newrgbcolor{curcolor}{0 0 0}
\pscustom[linestyle=none,fillstyle=solid,fillcolor=curcolor]
{
\newpath
\moveto(874.8769565,638.77361288)
\curveto(874.90694867,638.61361247)(874.89194868,638.47861261)(874.8319565,638.36861288)
\curveto(874.7719488,638.26861282)(874.69194888,638.19361289)(874.5919565,638.14361288)
\curveto(874.54194903,638.12361296)(874.48694909,638.11361297)(874.4269565,638.11361288)
\curveto(874.3769492,638.11361297)(874.32194925,638.10361298)(874.2619565,638.08361288)
\curveto(874.04194953,638.03361305)(873.82194975,638.04861304)(873.6019565,638.12861288)
\curveto(873.39195018,638.19861289)(873.24695033,638.2886128)(873.1669565,638.39861288)
\curveto(873.11695046,638.46861262)(873.0719505,638.54861254)(873.0319565,638.63861288)
\curveto(872.99195058,638.73861235)(872.94195063,638.81861227)(872.8819565,638.87861288)
\curveto(872.86195071,638.89861219)(872.83695074,638.91861217)(872.8069565,638.93861288)
\curveto(872.78695079,638.95861213)(872.75695082,638.96361212)(872.7169565,638.95361288)
\curveto(872.60695097,638.92361216)(872.50195107,638.86861222)(872.4019565,638.78861288)
\curveto(872.31195126,638.70861238)(872.22195135,638.63861245)(872.1319565,638.57861288)
\curveto(872.00195157,638.49861259)(871.86195171,638.42361266)(871.7119565,638.35361288)
\curveto(871.56195201,638.29361279)(871.40195217,638.23861285)(871.2319565,638.18861288)
\curveto(871.13195244,638.15861293)(871.02195255,638.13861295)(870.9019565,638.12861288)
\curveto(870.79195278,638.11861297)(870.68195289,638.10361298)(870.5719565,638.08361288)
\curveto(870.52195305,638.07361301)(870.4769531,638.06861302)(870.4369565,638.06861288)
\lineto(870.3319565,638.06861288)
\curveto(870.22195335,638.04861304)(870.11695346,638.04861304)(870.0169565,638.06861288)
\lineto(869.8819565,638.06861288)
\curveto(869.83195374,638.07861301)(869.78195379,638.083613)(869.7319565,638.08361288)
\curveto(869.68195389,638.083613)(869.63695394,638.09361299)(869.5969565,638.11361288)
\curveto(869.55695402,638.12361296)(869.52195405,638.12861296)(869.4919565,638.12861288)
\curveto(869.4719541,638.11861297)(869.44695413,638.11861297)(869.4169565,638.12861288)
\lineto(869.1769565,638.18861288)
\curveto(869.09695448,638.19861289)(869.02195455,638.21861287)(868.9519565,638.24861288)
\curveto(868.65195492,638.37861271)(868.40695517,638.52361256)(868.2169565,638.68361288)
\curveto(868.03695554,638.85361223)(867.88695569,639.088612)(867.7669565,639.38861288)
\curveto(867.6769559,639.60861148)(867.63195594,639.87361121)(867.6319565,640.18361288)
\lineto(867.6319565,640.49861288)
\curveto(867.64195593,640.54861054)(867.64695593,640.59861049)(867.6469565,640.64861288)
\lineto(867.6769565,640.82861288)
\lineto(867.7969565,641.15861288)
\curveto(867.83695574,641.26860982)(867.88695569,641.36860972)(867.9469565,641.45861288)
\curveto(868.12695545,641.74860934)(868.3719552,641.96360912)(868.6819565,642.10361288)
\curveto(868.99195458,642.24360884)(869.33195424,642.36860872)(869.7019565,642.47861288)
\curveto(869.84195373,642.51860857)(869.98695359,642.54860854)(870.1369565,642.56861288)
\curveto(870.28695329,642.5886085)(870.43695314,642.61360847)(870.5869565,642.64361288)
\curveto(870.65695292,642.66360842)(870.72195285,642.67360841)(870.7819565,642.67361288)
\curveto(870.85195272,642.67360841)(870.92695265,642.6836084)(871.0069565,642.70361288)
\curveto(871.0769525,642.72360836)(871.14695243,642.73360835)(871.2169565,642.73361288)
\curveto(871.28695229,642.74360834)(871.36195221,642.75860833)(871.4419565,642.77861288)
\curveto(871.69195188,642.83860825)(871.92695165,642.8886082)(872.1469565,642.92861288)
\curveto(872.36695121,642.97860811)(872.54195103,643.09360799)(872.6719565,643.27361288)
\curveto(872.73195084,643.35360773)(872.78195079,643.45360763)(872.8219565,643.57361288)
\curveto(872.86195071,643.70360738)(872.86195071,643.84360724)(872.8219565,643.99361288)
\curveto(872.76195081,644.23360685)(872.6719509,644.42360666)(872.5519565,644.56361288)
\curveto(872.44195113,644.70360638)(872.28195129,644.81360627)(872.0719565,644.89361288)
\curveto(871.95195162,644.94360614)(871.80695177,644.97860611)(871.6369565,644.99861288)
\curveto(871.4769521,645.01860607)(871.30695227,645.02860606)(871.1269565,645.02861288)
\curveto(870.94695263,645.02860606)(870.7719528,645.01860607)(870.6019565,644.99861288)
\curveto(870.43195314,644.97860611)(870.28695329,644.94860614)(870.1669565,644.90861288)
\curveto(869.99695358,644.84860624)(869.83195374,644.76360632)(869.6719565,644.65361288)
\curveto(869.59195398,644.59360649)(869.51695406,644.51360657)(869.4469565,644.41361288)
\curveto(869.38695419,644.32360676)(869.33195424,644.22360686)(869.2819565,644.11361288)
\curveto(869.25195432,644.03360705)(869.22195435,643.94860714)(869.1919565,643.85861288)
\curveto(869.1719544,643.76860732)(869.12695445,643.69860739)(869.0569565,643.64861288)
\curveto(869.01695456,643.61860747)(868.94695463,643.59360749)(868.8469565,643.57361288)
\curveto(868.75695482,643.56360752)(868.66195491,643.55860753)(868.5619565,643.55861288)
\curveto(868.46195511,643.55860753)(868.36195521,643.56360752)(868.2619565,643.57361288)
\curveto(868.1719554,643.59360749)(868.10695547,643.61860747)(868.0669565,643.64861288)
\curveto(868.02695555,643.67860741)(867.99695558,643.72860736)(867.9769565,643.79861288)
\curveto(867.95695562,643.86860722)(867.95695562,643.94360714)(867.9769565,644.02361288)
\curveto(868.00695557,644.15360693)(868.03695554,644.27360681)(868.0669565,644.38361288)
\curveto(868.10695547,644.50360658)(868.15195542,644.61860647)(868.2019565,644.72861288)
\curveto(868.39195518,645.07860601)(868.63195494,645.34860574)(868.9219565,645.53861288)
\curveto(869.21195436,645.73860535)(869.571954,645.89860519)(870.0019565,646.01861288)
\curveto(870.10195347,646.03860505)(870.20195337,646.05360503)(870.3019565,646.06361288)
\curveto(870.41195316,646.07360501)(870.52195305,646.088605)(870.6319565,646.10861288)
\curveto(870.6719529,646.11860497)(870.73695284,646.11860497)(870.8269565,646.10861288)
\curveto(870.91695266,646.10860498)(870.9719526,646.11860497)(870.9919565,646.13861288)
\curveto(871.69195188,646.14860494)(872.30195127,646.06860502)(872.8219565,645.89861288)
\curveto(873.34195023,645.72860536)(873.70694987,645.40360568)(873.9169565,644.92361288)
\curveto(874.00694957,644.72360636)(874.05694952,644.4886066)(874.0669565,644.21861288)
\curveto(874.08694949,643.95860713)(874.09694948,643.6836074)(874.0969565,643.39361288)
\lineto(874.0969565,640.07861288)
\curveto(874.09694948,639.93861115)(874.10194947,639.80361128)(874.1119565,639.67361288)
\curveto(874.12194945,639.54361154)(874.15194942,639.43861165)(874.2019565,639.35861288)
\curveto(874.25194932,639.2886118)(874.31694926,639.23861185)(874.3969565,639.20861288)
\curveto(874.48694909,639.16861192)(874.571949,639.13861195)(874.6519565,639.11861288)
\curveto(874.73194884,639.10861198)(874.79194878,639.06361202)(874.8319565,638.98361288)
\curveto(874.85194872,638.95361213)(874.86194871,638.92361216)(874.8619565,638.89361288)
\curveto(874.86194871,638.86361222)(874.86694871,638.82361226)(874.8769565,638.77361288)
\moveto(872.7319565,640.43861288)
\curveto(872.79195078,640.57861051)(872.82195075,640.73861035)(872.8219565,640.91861288)
\curveto(872.83195074,641.10860998)(872.83695074,641.30360978)(872.8369565,641.50361288)
\curveto(872.83695074,641.61360947)(872.83195074,641.71360937)(872.8219565,641.80361288)
\curveto(872.81195076,641.89360919)(872.7719508,641.96360912)(872.7019565,642.01361288)
\curveto(872.6719509,642.03360905)(872.60195097,642.04360904)(872.4919565,642.04361288)
\curveto(872.4719511,642.02360906)(872.43695114,642.01360907)(872.3869565,642.01361288)
\curveto(872.33695124,642.01360907)(872.29195128,642.00360908)(872.2519565,641.98361288)
\curveto(872.1719514,641.96360912)(872.08195149,641.94360914)(871.9819565,641.92361288)
\lineto(871.6819565,641.86361288)
\curveto(871.65195192,641.86360922)(871.61695196,641.85860923)(871.5769565,641.84861288)
\lineto(871.4719565,641.84861288)
\curveto(871.32195225,641.80860928)(871.15695242,641.7836093)(870.9769565,641.77361288)
\curveto(870.80695277,641.77360931)(870.64695293,641.75360933)(870.4969565,641.71361288)
\curveto(870.41695316,641.69360939)(870.34195323,641.67360941)(870.2719565,641.65361288)
\curveto(870.21195336,641.64360944)(870.14195343,641.62860946)(870.0619565,641.60861288)
\curveto(869.90195367,641.55860953)(869.75195382,641.49360959)(869.6119565,641.41361288)
\curveto(869.4719541,641.34360974)(869.35195422,641.25360983)(869.2519565,641.14361288)
\curveto(869.15195442,641.03361005)(869.0769545,640.89861019)(869.0269565,640.73861288)
\curveto(868.9769546,640.5886105)(868.95695462,640.40361068)(868.9669565,640.18361288)
\curveto(868.96695461,640.083611)(868.98195459,639.9886111)(869.0119565,639.89861288)
\curveto(869.05195452,639.81861127)(869.09695448,639.74361134)(869.1469565,639.67361288)
\curveto(869.22695435,639.56361152)(869.33195424,639.46861162)(869.4619565,639.38861288)
\curveto(869.59195398,639.31861177)(869.73195384,639.25861183)(869.8819565,639.20861288)
\curveto(869.93195364,639.19861189)(869.98195359,639.19361189)(870.0319565,639.19361288)
\curveto(870.08195349,639.19361189)(870.13195344,639.1886119)(870.1819565,639.17861288)
\curveto(870.25195332,639.15861193)(870.33695324,639.14361194)(870.4369565,639.13361288)
\curveto(870.54695303,639.13361195)(870.63695294,639.14361194)(870.7069565,639.16361288)
\curveto(870.76695281,639.1836119)(870.82695275,639.1886119)(870.8869565,639.17861288)
\curveto(870.94695263,639.17861191)(871.00695257,639.1886119)(871.0669565,639.20861288)
\curveto(871.14695243,639.22861186)(871.22195235,639.24361184)(871.2919565,639.25361288)
\curveto(871.3719522,639.26361182)(871.44695213,639.2836118)(871.5169565,639.31361288)
\curveto(871.80695177,639.43361165)(872.05195152,639.57861151)(872.2519565,639.74861288)
\curveto(872.46195111,639.91861117)(872.62195095,640.14861094)(872.7319565,640.43861288)
}
}
{
\newrgbcolor{curcolor}{0 0 0}
\pscustom[linestyle=none,fillstyle=solid,fillcolor=curcolor]
{
\newpath
\moveto(879.69359712,646.12361288)
\curveto(879.92359233,646.12360496)(880.0535922,646.06360502)(880.08359712,645.94361288)
\curveto(880.11359214,645.83360525)(880.12859213,645.66860542)(880.12859712,645.44861288)
\lineto(880.12859712,645.16361288)
\curveto(880.12859213,645.07360601)(880.10359215,644.99860609)(880.05359712,644.93861288)
\curveto(879.99359226,644.85860623)(879.90859235,644.81360627)(879.79859712,644.80361288)
\curveto(879.68859257,644.80360628)(879.57859268,644.7886063)(879.46859712,644.75861288)
\curveto(879.32859293,644.72860636)(879.19359306,644.69860639)(879.06359712,644.66861288)
\curveto(878.94359331,644.63860645)(878.82859343,644.59860649)(878.71859712,644.54861288)
\curveto(878.42859383,644.41860667)(878.19359406,644.23860685)(878.01359712,644.00861288)
\curveto(877.83359442,643.7886073)(877.67859458,643.53360755)(877.54859712,643.24361288)
\curveto(877.50859475,643.13360795)(877.47859478,643.01860807)(877.45859712,642.89861288)
\curveto(877.43859482,642.7886083)(877.41359484,642.67360841)(877.38359712,642.55361288)
\curveto(877.37359488,642.50360858)(877.36859489,642.45360863)(877.36859712,642.40361288)
\curveto(877.37859488,642.35360873)(877.37859488,642.30360878)(877.36859712,642.25361288)
\curveto(877.33859492,642.13360895)(877.32359493,641.99360909)(877.32359712,641.83361288)
\curveto(877.33359492,641.6836094)(877.33859492,641.53860955)(877.33859712,641.39861288)
\lineto(877.33859712,639.55361288)
\lineto(877.33859712,639.20861288)
\curveto(877.33859492,639.088612)(877.33359492,638.97361211)(877.32359712,638.86361288)
\curveto(877.31359494,638.75361233)(877.30859495,638.65861243)(877.30859712,638.57861288)
\curveto(877.31859494,638.49861259)(877.29859496,638.42861266)(877.24859712,638.36861288)
\curveto(877.19859506,638.29861279)(877.11859514,638.25861283)(877.00859712,638.24861288)
\curveto(876.90859535,638.23861285)(876.79859546,638.23361285)(876.67859712,638.23361288)
\lineto(876.40859712,638.23361288)
\curveto(876.3585959,638.25361283)(876.30859595,638.26861282)(876.25859712,638.27861288)
\curveto(876.21859604,638.29861279)(876.18859607,638.32361276)(876.16859712,638.35361288)
\curveto(876.11859614,638.42361266)(876.08859617,638.50861258)(876.07859712,638.60861288)
\lineto(876.07859712,638.93861288)
\lineto(876.07859712,640.09361288)
\lineto(876.07859712,644.24861288)
\lineto(876.07859712,645.28361288)
\lineto(876.07859712,645.58361288)
\curveto(876.08859617,645.6836054)(876.11859614,645.76860532)(876.16859712,645.83861288)
\curveto(876.19859606,645.87860521)(876.24859601,645.90860518)(876.31859712,645.92861288)
\curveto(876.39859586,645.94860514)(876.48359577,645.95860513)(876.57359712,645.95861288)
\curveto(876.66359559,645.96860512)(876.7535955,645.96860512)(876.84359712,645.95861288)
\curveto(876.93359532,645.94860514)(877.00359525,645.93360515)(877.05359712,645.91361288)
\curveto(877.13359512,645.8836052)(877.18359507,645.82360526)(877.20359712,645.73361288)
\curveto(877.23359502,645.65360543)(877.24859501,645.56360552)(877.24859712,645.46361288)
\lineto(877.24859712,645.16361288)
\curveto(877.24859501,645.06360602)(877.26859499,644.97360611)(877.30859712,644.89361288)
\curveto(877.31859494,644.87360621)(877.32859493,644.85860623)(877.33859712,644.84861288)
\lineto(877.38359712,644.80361288)
\curveto(877.49359476,644.80360628)(877.58359467,644.84860624)(877.65359712,644.93861288)
\curveto(877.72359453,645.03860605)(877.78359447,645.11860597)(877.83359712,645.17861288)
\lineto(877.92359712,645.26861288)
\curveto(878.01359424,645.37860571)(878.13859412,645.49360559)(878.29859712,645.61361288)
\curveto(878.4585938,645.73360535)(878.60859365,645.82360526)(878.74859712,645.88361288)
\curveto(878.83859342,645.93360515)(878.93359332,645.96860512)(879.03359712,645.98861288)
\curveto(879.13359312,646.01860507)(879.23859302,646.04860504)(879.34859712,646.07861288)
\curveto(879.40859285,646.088605)(879.46859279,646.09360499)(879.52859712,646.09361288)
\curveto(879.58859267,646.10360498)(879.64359261,646.11360497)(879.69359712,646.12361288)
}
}
{
\newrgbcolor{curcolor}{0.7019608 0.7019608 0.7019608}
\pscustom[linestyle=none,fillstyle=solid,fillcolor=curcolor]
{
\newpath
\moveto(812.80437349,648.9286495)
\lineto(827.80437349,648.9286495)
\lineto(827.80437349,633.9286495)
\lineto(812.80437349,633.9286495)
\closepath
}
}
{
\newrgbcolor{curcolor}{0 0 0}
\pscustom[linestyle=none,fillstyle=solid,fillcolor=curcolor]
{
\newpath
\moveto(841.1775815,620.77784628)
\lineto(841.1775815,620.50784628)
\curveto(841.18757153,620.41784103)(841.18257153,620.33784111)(841.1625815,620.26784628)
\lineto(841.1625815,620.11784628)
\curveto(841.15257156,620.08784136)(841.14757157,620.05284139)(841.1475815,620.01284628)
\curveto(841.15757156,619.97284147)(841.15757156,619.9428415)(841.1475815,619.92284628)
\curveto(841.13757158,619.87284157)(841.13257158,619.81784163)(841.1325815,619.75784628)
\curveto(841.13257158,619.70784174)(841.12757159,619.65784179)(841.1175815,619.60784628)
\curveto(841.08757163,619.46784198)(841.06757165,619.31784213)(841.0575815,619.15784628)
\curveto(841.04757167,619.00784244)(841.0175717,618.86284258)(840.9675815,618.72284628)
\curveto(840.93757178,618.60284284)(840.90257181,618.47784297)(840.8625815,618.34784628)
\curveto(840.83257188,618.22784322)(840.79257192,618.10784334)(840.7425815,617.98784628)
\curveto(840.57257214,617.55784389)(840.35757236,617.16784428)(840.0975815,616.81784628)
\curveto(839.84757287,616.47784497)(839.53257318,616.18784526)(839.1525815,615.94784628)
\curveto(838.96257375,615.82784562)(838.75757396,615.72284572)(838.5375815,615.63284628)
\curveto(838.32757439,615.55284589)(838.09757462,615.47284597)(837.8475815,615.39284628)
\curveto(837.73757498,615.35284609)(837.6175751,615.32284612)(837.4875815,615.30284628)
\curveto(837.36757535,615.29284615)(837.24757547,615.27284617)(837.1275815,615.24284628)
\curveto(837.0175757,615.22284622)(836.90757581,615.21284623)(836.7975815,615.21284628)
\curveto(836.69757602,615.21284623)(836.59757612,615.20284624)(836.4975815,615.18284628)
\lineto(836.2875815,615.18284628)
\curveto(836.25757646,615.17284627)(836.22257649,615.16784628)(836.1825815,615.16784628)
\curveto(836.14257657,615.17784627)(836.10257661,615.18284626)(836.0625815,615.18284628)
\lineto(833.0625815,615.18284628)
\curveto(832.9125798,615.18284626)(832.77757994,615.18784626)(832.6575815,615.19784628)
\curveto(832.54758017,615.21784623)(832.47258024,615.28284616)(832.4325815,615.39284628)
\curveto(832.39258032,615.47284597)(832.37258034,615.58784586)(832.3725815,615.73784628)
\curveto(832.38258033,615.88784556)(832.38758033,616.02284542)(832.3875815,616.14284628)
\lineto(832.3875815,625.00784628)
\curveto(832.38758033,625.12783632)(832.38258033,625.25283619)(832.3725815,625.38284628)
\curveto(832.37258034,625.52283592)(832.39758032,625.63283581)(832.4475815,625.71284628)
\curveto(832.48758023,625.78283566)(832.56258015,625.82783562)(832.6725815,625.84784628)
\curveto(832.69258002,625.85783559)(832.71258,625.85783559)(832.7325815,625.84784628)
\curveto(832.75257996,625.8478356)(832.77257994,625.85283559)(832.7925815,625.86284628)
\lineto(836.0475815,625.86284628)
\curveto(836.09757662,625.86283558)(836.14257657,625.86283558)(836.1825815,625.86284628)
\curveto(836.23257648,625.87283557)(836.27757644,625.87283557)(836.3175815,625.86284628)
\curveto(836.36757635,625.8428356)(836.4175763,625.83783561)(836.4675815,625.84784628)
\curveto(836.52757619,625.85783559)(836.58257613,625.85783559)(836.6325815,625.84784628)
\curveto(836.68257603,625.83783561)(836.73757598,625.83283561)(836.7975815,625.83284628)
\curveto(836.85757586,625.83283561)(836.9125758,625.82783562)(836.9625815,625.81784628)
\curveto(837.0125757,625.80783564)(837.05757566,625.80283564)(837.0975815,625.80284628)
\curveto(837.14757557,625.80283564)(837.19757552,625.79783565)(837.2475815,625.78784628)
\curveto(837.35757536,625.76783568)(837.46257525,625.7478357)(837.5625815,625.72784628)
\curveto(837.66257505,625.71783573)(837.76257495,625.69783575)(837.8625815,625.66784628)
\curveto(838.08257463,625.59783585)(838.29257442,625.52783592)(838.4925815,625.45784628)
\curveto(838.69257402,625.39783605)(838.87757384,625.31283613)(839.0475815,625.20284628)
\curveto(839.18757353,625.12283632)(839.3125734,625.0428364)(839.4225815,624.96284628)
\curveto(839.45257326,624.9428365)(839.48257323,624.91783653)(839.5125815,624.88784628)
\curveto(839.54257317,624.86783658)(839.57257314,624.8478366)(839.6025815,624.82784628)
\curveto(839.66257305,624.77783667)(839.717573,624.72783672)(839.7675815,624.67784628)
\curveto(839.8175729,624.62783682)(839.86757285,624.57783687)(839.9175815,624.52784628)
\curveto(839.96757275,624.47783697)(840.00757271,624.442837)(840.0375815,624.42284628)
\curveto(840.07757264,624.36283708)(840.1175726,624.30783714)(840.1575815,624.25784628)
\curveto(840.20757251,624.20783724)(840.25257246,624.15283729)(840.2925815,624.09284628)
\curveto(840.34257237,624.03283741)(840.38257233,623.96783748)(840.4125815,623.89784628)
\curveto(840.45257226,623.83783761)(840.49757222,623.77283767)(840.5475815,623.70284628)
\curveto(840.56757215,623.66283778)(840.58257213,623.62783782)(840.5925815,623.59784628)
\curveto(840.60257211,623.56783788)(840.6175721,623.53283791)(840.6375815,623.49284628)
\curveto(840.67757204,623.41283803)(840.712572,623.33283811)(840.7425815,623.25284628)
\curveto(840.77257194,623.18283826)(840.80757191,623.10783834)(840.8475815,623.02784628)
\curveto(840.88757183,622.91783853)(840.9175718,622.80283864)(840.9375815,622.68284628)
\curveto(840.96757175,622.57283887)(840.99757172,622.46283898)(841.0275815,622.35284628)
\curveto(841.04757167,622.29283915)(841.05757166,622.23283921)(841.0575815,622.17284628)
\curveto(841.05757166,622.12283932)(841.06757165,622.06783938)(841.0875815,622.00784628)
\curveto(841.13757158,621.82783962)(841.16257155,621.62783982)(841.1625815,621.40784628)
\curveto(841.17257154,621.19784025)(841.17757154,620.98784046)(841.1775815,620.77784628)
\moveto(839.7525815,619.99784628)
\curveto(839.77257294,620.09784135)(839.78257293,620.20284124)(839.7825815,620.31284628)
\lineto(839.7825815,620.65784628)
\lineto(839.7825815,620.88284628)
\curveto(839.79257292,620.96284048)(839.78757293,621.03784041)(839.7675815,621.10784628)
\curveto(839.76757295,621.13784031)(839.76257295,621.16784028)(839.7525815,621.19784628)
\lineto(839.7525815,621.30284628)
\curveto(839.73257298,621.41284003)(839.717573,621.52283992)(839.7075815,621.63284628)
\curveto(839.70757301,621.7428397)(839.69257302,621.85283959)(839.6625815,621.96284628)
\curveto(839.64257307,622.0428394)(839.62257309,622.11783933)(839.6025815,622.18784628)
\curveto(839.59257312,622.26783918)(839.57757314,622.3478391)(839.5575815,622.42784628)
\curveto(839.44757327,622.78783866)(839.30757341,623.10283834)(839.1375815,623.37284628)
\curveto(838.85757386,623.82283762)(838.44257427,624.16283728)(837.8925815,624.39284628)
\curveto(837.80257491,624.442837)(837.70757501,624.47783697)(837.6075815,624.49784628)
\curveto(837.50757521,624.52783692)(837.40257531,624.55783689)(837.2925815,624.58784628)
\curveto(837.18257553,624.61783683)(837.06757565,624.63283681)(836.9475815,624.63284628)
\curveto(836.83757588,624.6428368)(836.72757599,624.65783679)(836.6175815,624.67784628)
\lineto(836.3025815,624.67784628)
\curveto(836.27257644,624.68783676)(836.23757648,624.69283675)(836.1975815,624.69284628)
\lineto(836.0775815,624.69284628)
\lineto(834.2475815,624.69284628)
\curveto(834.22757849,624.68283676)(834.20257851,624.67783677)(834.1725815,624.67784628)
\curveto(834.14257857,624.68783676)(834.1175786,624.68783676)(834.0975815,624.67784628)
\lineto(833.9475815,624.61784628)
\curveto(833.90757881,624.59783685)(833.87757884,624.56783688)(833.8575815,624.52784628)
\curveto(833.83757888,624.48783696)(833.8175789,624.41783703)(833.7975815,624.31784628)
\lineto(833.7975815,624.19784628)
\curveto(833.78757893,624.15783729)(833.78257893,624.11283733)(833.7825815,624.06284628)
\lineto(833.7825815,623.92784628)
\lineto(833.7825815,617.11784628)
\lineto(833.7825815,616.96784628)
\curveto(833.78257893,616.92784452)(833.78757893,616.88784456)(833.7975815,616.84784628)
\lineto(833.7975815,616.72784628)
\curveto(833.8175789,616.62784482)(833.83757888,616.55784489)(833.8575815,616.51784628)
\curveto(833.93757878,616.39784505)(834.08757863,616.33784511)(834.3075815,616.33784628)
\curveto(834.52757819,616.3478451)(834.73757798,616.35284509)(834.9375815,616.35284628)
\lineto(835.8075815,616.35284628)
\curveto(835.87757684,616.35284509)(835.95257676,616.3478451)(836.0325815,616.33784628)
\curveto(836.1125766,616.33784511)(836.18257653,616.3478451)(836.2425815,616.36784628)
\lineto(836.4075815,616.36784628)
\curveto(836.45757626,616.37784507)(836.5125762,616.37784507)(836.5725815,616.36784628)
\curveto(836.63257608,616.36784508)(836.69257602,616.37284507)(836.7525815,616.38284628)
\curveto(836.8125759,616.40284504)(836.87257584,616.41284503)(836.9325815,616.41284628)
\curveto(836.99257572,616.42284502)(837.05757566,616.43784501)(837.1275815,616.45784628)
\curveto(837.23757548,616.48784496)(837.34257537,616.51784493)(837.4425815,616.54784628)
\curveto(837.55257516,616.57784487)(837.66257505,616.61784483)(837.7725815,616.66784628)
\curveto(838.14257457,616.82784462)(838.45757426,617.0428444)(838.7175815,617.31284628)
\curveto(838.98757373,617.59284385)(839.20757351,617.92284352)(839.3775815,618.30284628)
\curveto(839.42757329,618.41284303)(839.46757325,618.52784292)(839.4975815,618.64784628)
\lineto(839.6175815,619.03784628)
\curveto(839.64757307,619.1478423)(839.66757305,619.26284218)(839.6775815,619.38284628)
\curveto(839.69757302,619.51284193)(839.717573,619.63784181)(839.7375815,619.75784628)
\curveto(839.74757297,619.80784164)(839.75257296,619.8478416)(839.7525815,619.87784628)
\lineto(839.7525815,619.99784628)
}
}
{
\newrgbcolor{curcolor}{0 0 0}
\pscustom[linestyle=none,fillstyle=solid,fillcolor=curcolor]
{
\newpath
\moveto(849.8044565,619.38284628)
\curveto(849.82444844,619.32284212)(849.83444843,619.22784222)(849.8344565,619.09784628)
\curveto(849.83444843,618.97784247)(849.82944843,618.89284255)(849.8194565,618.84284628)
\lineto(849.8194565,618.69284628)
\curveto(849.80944845,618.61284283)(849.79944846,618.53784291)(849.7894565,618.46784628)
\curveto(849.78944847,618.40784304)(849.78444848,618.33784311)(849.7744565,618.25784628)
\curveto(849.75444851,618.19784325)(849.73944852,618.13784331)(849.7294565,618.07784628)
\curveto(849.72944853,618.01784343)(849.71944854,617.95784349)(849.6994565,617.89784628)
\curveto(849.6594486,617.76784368)(849.62444864,617.63784381)(849.5944565,617.50784628)
\curveto(849.5644487,617.37784407)(849.52444874,617.25784419)(849.4744565,617.14784628)
\curveto(849.264449,616.66784478)(848.98444928,616.26284518)(848.6344565,615.93284628)
\curveto(848.28444998,615.61284583)(847.85445041,615.36784608)(847.3444565,615.19784628)
\curveto(847.23445103,615.15784629)(847.11445115,615.12784632)(846.9844565,615.10784628)
\curveto(846.8644514,615.08784636)(846.73945152,615.06784638)(846.6094565,615.04784628)
\curveto(846.54945171,615.03784641)(846.48445178,615.03284641)(846.4144565,615.03284628)
\curveto(846.35445191,615.02284642)(846.29445197,615.01784643)(846.2344565,615.01784628)
\curveto(846.19445207,615.00784644)(846.13445213,615.00284644)(846.0544565,615.00284628)
\curveto(845.98445228,615.00284644)(845.93445233,615.00784644)(845.9044565,615.01784628)
\curveto(845.8644524,615.02784642)(845.82445244,615.03284641)(845.7844565,615.03284628)
\curveto(845.74445252,615.02284642)(845.70945255,615.02284642)(845.6794565,615.03284628)
\lineto(845.5894565,615.03284628)
\lineto(845.2294565,615.07784628)
\curveto(845.08945317,615.11784633)(844.95445331,615.15784629)(844.8244565,615.19784628)
\curveto(844.69445357,615.23784621)(844.56945369,615.28284616)(844.4494565,615.33284628)
\curveto(843.99945426,615.53284591)(843.62945463,615.79284565)(843.3394565,616.11284628)
\curveto(843.04945521,616.43284501)(842.80945545,616.82284462)(842.6194565,617.28284628)
\curveto(842.56945569,617.38284406)(842.52945573,617.48284396)(842.4994565,617.58284628)
\curveto(842.47945578,617.68284376)(842.4594558,617.78784366)(842.4394565,617.89784628)
\curveto(842.41945584,617.93784351)(842.40945585,617.96784348)(842.4094565,617.98784628)
\curveto(842.41945584,618.01784343)(842.41945584,618.05284339)(842.4094565,618.09284628)
\curveto(842.38945587,618.17284327)(842.37445589,618.25284319)(842.3644565,618.33284628)
\curveto(842.3644559,618.42284302)(842.35445591,618.50784294)(842.3344565,618.58784628)
\lineto(842.3344565,618.70784628)
\curveto(842.33445593,618.7478427)(842.32945593,618.79284265)(842.3194565,618.84284628)
\curveto(842.30945595,618.89284255)(842.30445596,618.97784247)(842.3044565,619.09784628)
\curveto(842.30445596,619.22784222)(842.31445595,619.32284212)(842.3344565,619.38284628)
\curveto(842.35445591,619.45284199)(842.3594559,619.52284192)(842.3494565,619.59284628)
\curveto(842.33945592,619.66284178)(842.34445592,619.73284171)(842.3644565,619.80284628)
\curveto(842.37445589,619.85284159)(842.37945588,619.89284155)(842.3794565,619.92284628)
\curveto(842.38945587,619.96284148)(842.39945586,620.00784144)(842.4094565,620.05784628)
\curveto(842.43945582,620.17784127)(842.4644558,620.29784115)(842.4844565,620.41784628)
\curveto(842.51445575,620.53784091)(842.55445571,620.65284079)(842.6044565,620.76284628)
\curveto(842.75445551,621.13284031)(842.93445533,621.46283998)(843.1444565,621.75284628)
\curveto(843.3644549,622.05283939)(843.62945463,622.30283914)(843.9394565,622.50284628)
\curveto(844.0594542,622.58283886)(844.18445408,622.6478388)(844.3144565,622.69784628)
\curveto(844.44445382,622.75783869)(844.57945368,622.81783863)(844.7194565,622.87784628)
\curveto(844.83945342,622.92783852)(844.96945329,622.95783849)(845.1094565,622.96784628)
\curveto(845.24945301,622.98783846)(845.38945287,623.01783843)(845.5294565,623.05784628)
\lineto(845.7244565,623.05784628)
\curveto(845.79445247,623.06783838)(845.8594524,623.07783837)(845.9194565,623.08784628)
\curveto(846.80945145,623.09783835)(847.54945071,622.91283853)(848.1394565,622.53284628)
\curveto(848.72944953,622.15283929)(849.15444911,621.65783979)(849.4144565,621.04784628)
\curveto(849.4644488,620.9478405)(849.50444876,620.8478406)(849.5344565,620.74784628)
\curveto(849.5644487,620.6478408)(849.59944866,620.5428409)(849.6394565,620.43284628)
\curveto(849.66944859,620.32284112)(849.69444857,620.20284124)(849.7144565,620.07284628)
\curveto(849.73444853,619.95284149)(849.7594485,619.82784162)(849.7894565,619.69784628)
\curveto(849.79944846,619.6478418)(849.79944846,619.59284185)(849.7894565,619.53284628)
\curveto(849.78944847,619.48284196)(849.79444847,619.43284201)(849.8044565,619.38284628)
\moveto(848.4694565,618.52784628)
\curveto(848.48944977,618.59784285)(848.49444977,618.67784277)(848.4844565,618.76784628)
\lineto(848.4844565,619.02284628)
\curveto(848.48444978,619.41284203)(848.44944981,619.7428417)(848.3794565,620.01284628)
\curveto(848.34944991,620.09284135)(848.32444994,620.17284127)(848.3044565,620.25284628)
\curveto(848.28444998,620.33284111)(848.25945,620.40784104)(848.2294565,620.47784628)
\curveto(847.94945031,621.12784032)(847.50445076,621.57783987)(846.8944565,621.82784628)
\curveto(846.82445144,621.85783959)(846.74945151,621.87783957)(846.6694565,621.88784628)
\lineto(846.4294565,621.94784628)
\curveto(846.34945191,621.96783948)(846.264452,621.97783947)(846.1744565,621.97784628)
\lineto(845.9044565,621.97784628)
\lineto(845.6344565,621.93284628)
\curveto(845.53445273,621.91283953)(845.43945282,621.88783956)(845.3494565,621.85784628)
\curveto(845.26945299,621.83783961)(845.18945307,621.80783964)(845.1094565,621.76784628)
\curveto(845.03945322,621.7478397)(844.97445329,621.71783973)(844.9144565,621.67784628)
\curveto(844.85445341,621.63783981)(844.79945346,621.59783985)(844.7494565,621.55784628)
\curveto(844.50945375,621.38784006)(844.31445395,621.18284026)(844.1644565,620.94284628)
\curveto(844.01445425,620.70284074)(843.88445438,620.42284102)(843.7744565,620.10284628)
\curveto(843.74445452,620.00284144)(843.72445454,619.89784155)(843.7144565,619.78784628)
\curveto(843.70445456,619.68784176)(843.68945457,619.58284186)(843.6694565,619.47284628)
\curveto(843.6594546,619.43284201)(843.65445461,619.36784208)(843.6544565,619.27784628)
\curveto(843.64445462,619.2478422)(843.63945462,619.21284223)(843.6394565,619.17284628)
\curveto(843.64945461,619.13284231)(843.65445461,619.08784236)(843.6544565,619.03784628)
\lineto(843.6544565,618.73784628)
\curveto(843.65445461,618.63784281)(843.6644546,618.5478429)(843.6844565,618.46784628)
\lineto(843.7144565,618.28784628)
\curveto(843.73445453,618.18784326)(843.74945451,618.08784336)(843.7594565,617.98784628)
\curveto(843.77945448,617.89784355)(843.80945445,617.81284363)(843.8494565,617.73284628)
\curveto(843.94945431,617.49284395)(844.0644542,617.26784418)(844.1944565,617.05784628)
\curveto(844.33445393,616.8478446)(844.50445376,616.67284477)(844.7044565,616.53284628)
\curveto(844.75445351,616.50284494)(844.79945346,616.47784497)(844.8394565,616.45784628)
\curveto(844.87945338,616.43784501)(844.92445334,616.41284503)(844.9744565,616.38284628)
\curveto(845.05445321,616.33284511)(845.13945312,616.28784516)(845.2294565,616.24784628)
\curveto(845.32945293,616.21784523)(845.43445283,616.18784526)(845.5444565,616.15784628)
\curveto(845.59445267,616.13784531)(845.63945262,616.12784532)(845.6794565,616.12784628)
\curveto(845.72945253,616.13784531)(845.77945248,616.13784531)(845.8294565,616.12784628)
\curveto(845.8594524,616.11784533)(845.91945234,616.10784534)(846.0094565,616.09784628)
\curveto(846.10945215,616.08784536)(846.18445208,616.09284535)(846.2344565,616.11284628)
\curveto(846.27445199,616.12284532)(846.31445195,616.12284532)(846.3544565,616.11284628)
\curveto(846.39445187,616.11284533)(846.43445183,616.12284532)(846.4744565,616.14284628)
\curveto(846.55445171,616.16284528)(846.63445163,616.17784527)(846.7144565,616.18784628)
\curveto(846.79445147,616.20784524)(846.86945139,616.23284521)(846.9394565,616.26284628)
\curveto(847.27945098,616.40284504)(847.55445071,616.59784485)(847.7644565,616.84784628)
\curveto(847.97445029,617.09784435)(848.14945011,617.39284405)(848.2894565,617.73284628)
\curveto(848.33944992,617.85284359)(848.36944989,617.97784347)(848.3794565,618.10784628)
\curveto(848.39944986,618.2478432)(848.42944983,618.38784306)(848.4694565,618.52784628)
}
}
{
\newrgbcolor{curcolor}{0 0 0}
\pscustom[linestyle=none,fillstyle=solid,fillcolor=curcolor]
{
\newpath
\moveto(854.42773775,623.08784628)
\curveto(855.16773296,623.09783835)(855.78273234,622.98783846)(856.27273775,622.75784628)
\curveto(856.77273135,622.53783891)(857.16773096,622.20283924)(857.45773775,621.75284628)
\curveto(857.58773054,621.55283989)(857.69773043,621.30784014)(857.78773775,621.01784628)
\curveto(857.80773032,620.96784048)(857.8227303,620.90284054)(857.83273775,620.82284628)
\curveto(857.84273028,620.7428407)(857.83773029,620.67284077)(857.81773775,620.61284628)
\curveto(857.78773034,620.56284088)(857.73773039,620.51784093)(857.66773775,620.47784628)
\curveto(857.63773049,620.45784099)(857.60773052,620.447841)(857.57773775,620.44784628)
\curveto(857.54773058,620.45784099)(857.51273061,620.45784099)(857.47273775,620.44784628)
\curveto(857.43273069,620.43784101)(857.39273073,620.43284101)(857.35273775,620.43284628)
\curveto(857.31273081,620.442841)(857.27273085,620.447841)(857.23273775,620.44784628)
\lineto(856.91773775,620.44784628)
\curveto(856.81773131,620.45784099)(856.73273139,620.48784096)(856.66273775,620.53784628)
\curveto(856.58273154,620.59784085)(856.5277316,620.68284076)(856.49773775,620.79284628)
\curveto(856.46773166,620.90284054)(856.4277317,620.99784045)(856.37773775,621.07784628)
\curveto(856.2277319,621.33784011)(856.03273209,621.5428399)(855.79273775,621.69284628)
\curveto(855.71273241,621.7428397)(855.6277325,621.78283966)(855.53773775,621.81284628)
\curveto(855.44773268,621.85283959)(855.35273277,621.88783956)(855.25273775,621.91784628)
\curveto(855.11273301,621.95783949)(854.9277332,621.97783947)(854.69773775,621.97784628)
\curveto(854.46773366,621.98783946)(854.27773385,621.96783948)(854.12773775,621.91784628)
\curveto(854.05773407,621.89783955)(853.99273413,621.88283956)(853.93273775,621.87284628)
\curveto(853.87273425,621.86283958)(853.80773432,621.8478396)(853.73773775,621.82784628)
\curveto(853.47773465,621.71783973)(853.24773488,621.56783988)(853.04773775,621.37784628)
\curveto(852.84773528,621.18784026)(852.69273543,620.96284048)(852.58273775,620.70284628)
\curveto(852.54273558,620.61284083)(852.50773562,620.51784093)(852.47773775,620.41784628)
\curveto(852.44773568,620.32784112)(852.41773571,620.22784122)(852.38773775,620.11784628)
\lineto(852.29773775,619.71284628)
\curveto(852.28773584,619.66284178)(852.28273584,619.60784184)(852.28273775,619.54784628)
\curveto(852.29273583,619.48784196)(852.28773584,619.43284201)(852.26773775,619.38284628)
\lineto(852.26773775,619.26284628)
\curveto(852.25773587,619.22284222)(852.24773588,619.15784229)(852.23773775,619.06784628)
\curveto(852.23773589,618.97784247)(852.24773588,618.91284253)(852.26773775,618.87284628)
\curveto(852.27773585,618.82284262)(852.27773585,618.77284267)(852.26773775,618.72284628)
\curveto(852.25773587,618.67284277)(852.25773587,618.62284282)(852.26773775,618.57284628)
\curveto(852.27773585,618.53284291)(852.28273584,618.46284298)(852.28273775,618.36284628)
\curveto(852.30273582,618.28284316)(852.31773581,618.19784325)(852.32773775,618.10784628)
\curveto(852.34773578,618.01784343)(852.36773576,617.93284351)(852.38773775,617.85284628)
\curveto(852.49773563,617.53284391)(852.6227355,617.25284419)(852.76273775,617.01284628)
\curveto(852.91273521,616.78284466)(853.11773501,616.58284486)(853.37773775,616.41284628)
\curveto(853.46773466,616.36284508)(853.55773457,616.31784513)(853.64773775,616.27784628)
\curveto(853.74773438,616.23784521)(853.85273427,616.19784525)(853.96273775,616.15784628)
\curveto(854.01273411,616.1478453)(854.05273407,616.1428453)(854.08273775,616.14284628)
\curveto(854.11273401,616.1428453)(854.15273397,616.13784531)(854.20273775,616.12784628)
\curveto(854.23273389,616.11784533)(854.28273384,616.11284533)(854.35273775,616.11284628)
\lineto(854.51773775,616.11284628)
\curveto(854.51773361,616.10284534)(854.53773359,616.09784535)(854.57773775,616.09784628)
\curveto(854.59773353,616.10784534)(854.6227335,616.10784534)(854.65273775,616.09784628)
\curveto(854.68273344,616.09784535)(854.71273341,616.10284534)(854.74273775,616.11284628)
\curveto(854.81273331,616.13284531)(854.87773325,616.13784531)(854.93773775,616.12784628)
\curveto(855.00773312,616.12784532)(855.07773305,616.13784531)(855.14773775,616.15784628)
\curveto(855.40773272,616.23784521)(855.63273249,616.33784511)(855.82273775,616.45784628)
\curveto(856.01273211,616.58784486)(856.17273195,616.75284469)(856.30273775,616.95284628)
\curveto(856.35273177,617.03284441)(856.39773173,617.11784433)(856.43773775,617.20784628)
\lineto(856.55773775,617.47784628)
\curveto(856.57773155,617.55784389)(856.59773153,617.63284381)(856.61773775,617.70284628)
\curveto(856.64773148,617.78284366)(856.69773143,617.8478436)(856.76773775,617.89784628)
\curveto(856.79773133,617.92784352)(856.85773127,617.9478435)(856.94773775,617.95784628)
\curveto(857.03773109,617.97784347)(857.13273099,617.98784346)(857.23273775,617.98784628)
\curveto(857.34273078,617.99784345)(857.44273068,617.99784345)(857.53273775,617.98784628)
\curveto(857.63273049,617.97784347)(857.70273042,617.95784349)(857.74273775,617.92784628)
\curveto(857.80273032,617.88784356)(857.83773029,617.82784362)(857.84773775,617.74784628)
\curveto(857.86773026,617.66784378)(857.86773026,617.58284386)(857.84773775,617.49284628)
\curveto(857.79773033,617.3428441)(857.74773038,617.19784425)(857.69773775,617.05784628)
\curveto(857.65773047,616.92784452)(857.60273052,616.79784465)(857.53273775,616.66784628)
\curveto(857.38273074,616.36784508)(857.19273093,616.10284534)(856.96273775,615.87284628)
\curveto(856.74273138,615.6428458)(856.47273165,615.45784599)(856.15273775,615.31784628)
\curveto(856.07273205,615.27784617)(855.98773214,615.2428462)(855.89773775,615.21284628)
\curveto(855.80773232,615.19284625)(855.71273241,615.16784628)(855.61273775,615.13784628)
\curveto(855.50273262,615.09784635)(855.39273273,615.07784637)(855.28273775,615.07784628)
\curveto(855.17273295,615.06784638)(855.06273306,615.05284639)(854.95273775,615.03284628)
\curveto(854.91273321,615.01284643)(854.87273325,615.00784644)(854.83273775,615.01784628)
\curveto(854.79273333,615.02784642)(854.75273337,615.02784642)(854.71273775,615.01784628)
\lineto(854.57773775,615.01784628)
\lineto(854.33773775,615.01784628)
\curveto(854.26773386,615.00784644)(854.20273392,615.01284643)(854.14273775,615.03284628)
\lineto(854.06773775,615.03284628)
\lineto(853.70773775,615.07784628)
\curveto(853.57773455,615.11784633)(853.45273467,615.15284629)(853.33273775,615.18284628)
\curveto(853.21273491,615.21284623)(853.09773503,615.25284619)(852.98773775,615.30284628)
\curveto(852.6277355,615.46284598)(852.3277358,615.65284579)(852.08773775,615.87284628)
\curveto(851.85773627,616.09284535)(851.64273648,616.36284508)(851.44273775,616.68284628)
\curveto(851.39273673,616.76284468)(851.34773678,616.85284459)(851.30773775,616.95284628)
\lineto(851.18773775,617.25284628)
\curveto(851.13773699,617.36284408)(851.10273702,617.47784397)(851.08273775,617.59784628)
\curveto(851.06273706,617.71784373)(851.03773709,617.83784361)(851.00773775,617.95784628)
\curveto(850.99773713,617.99784345)(850.99273713,618.03784341)(850.99273775,618.07784628)
\curveto(850.99273713,618.11784333)(850.98773714,618.15784329)(850.97773775,618.19784628)
\curveto(850.95773717,618.25784319)(850.94773718,618.32284312)(850.94773775,618.39284628)
\curveto(850.95773717,618.46284298)(850.95273717,618.52784292)(850.93273775,618.58784628)
\lineto(850.93273775,618.73784628)
\curveto(850.9227372,618.78784266)(850.91773721,618.85784259)(850.91773775,618.94784628)
\curveto(850.91773721,619.03784241)(850.9227372,619.10784234)(850.93273775,619.15784628)
\curveto(850.94273718,619.20784224)(850.94273718,619.25284219)(850.93273775,619.29284628)
\curveto(850.93273719,619.33284211)(850.93773719,619.37284207)(850.94773775,619.41284628)
\curveto(850.96773716,619.48284196)(850.97273715,619.55284189)(850.96273775,619.62284628)
\curveto(850.96273716,619.69284175)(850.97273715,619.75784169)(850.99273775,619.81784628)
\curveto(851.03273709,619.98784146)(851.06773706,620.15784129)(851.09773775,620.32784628)
\curveto(851.127737,620.49784095)(851.17273695,620.65784079)(851.23273775,620.80784628)
\curveto(851.44273668,621.32784012)(851.69773643,621.7478397)(851.99773775,622.06784628)
\curveto(852.29773583,622.38783906)(852.70773542,622.65283879)(853.22773775,622.86284628)
\curveto(853.33773479,622.91283853)(853.45773467,622.9478385)(853.58773775,622.96784628)
\curveto(853.71773441,622.98783846)(853.85273427,623.01283843)(853.99273775,623.04284628)
\curveto(854.06273406,623.05283839)(854.13273399,623.05783839)(854.20273775,623.05784628)
\curveto(854.27273385,623.06783838)(854.34773378,623.07783837)(854.42773775,623.08784628)
}
}
{
\newrgbcolor{curcolor}{0 0 0}
\pscustom[linestyle=none,fillstyle=solid,fillcolor=curcolor]
{
\newpath
\moveto(866.09937837,619.35284628)
\curveto(866.11937069,619.25284219)(866.11937069,619.13784231)(866.09937837,619.00784628)
\curveto(866.08937072,618.88784256)(866.05937075,618.80284264)(866.00937837,618.75284628)
\curveto(865.95937085,618.71284273)(865.88437092,618.68284276)(865.78437837,618.66284628)
\curveto(865.69437111,618.65284279)(865.58937122,618.6478428)(865.46937837,618.64784628)
\lineto(865.10937837,618.64784628)
\curveto(864.98937182,618.65784279)(864.88437192,618.66284278)(864.79437837,618.66284628)
\lineto(860.95437837,618.66284628)
\curveto(860.87437593,618.66284278)(860.79437601,618.65784279)(860.71437837,618.64784628)
\curveto(860.63437617,618.6478428)(860.56937624,618.63284281)(860.51937837,618.60284628)
\curveto(860.47937633,618.58284286)(860.43937637,618.5428429)(860.39937837,618.48284628)
\curveto(860.37937643,618.45284299)(860.35937645,618.40784304)(860.33937837,618.34784628)
\curveto(860.31937649,618.29784315)(860.31937649,618.2478432)(860.33937837,618.19784628)
\curveto(860.34937646,618.1478433)(860.35437645,618.10284334)(860.35437837,618.06284628)
\curveto(860.35437645,618.02284342)(860.35937645,617.98284346)(860.36937837,617.94284628)
\curveto(860.38937642,617.86284358)(860.4093764,617.77784367)(860.42937837,617.68784628)
\curveto(860.44937636,617.60784384)(860.47937633,617.52784392)(860.51937837,617.44784628)
\curveto(860.74937606,616.90784454)(861.12937568,616.52284492)(861.65937837,616.29284628)
\curveto(861.71937509,616.26284518)(861.78437502,616.23784521)(861.85437837,616.21784628)
\lineto(862.06437837,616.15784628)
\curveto(862.09437471,616.1478453)(862.14437466,616.1428453)(862.21437837,616.14284628)
\curveto(862.35437445,616.10284534)(862.53937427,616.08284536)(862.76937837,616.08284628)
\curveto(862.99937381,616.08284536)(863.18437362,616.10284534)(863.32437837,616.14284628)
\curveto(863.46437334,616.18284526)(863.58937322,616.22284522)(863.69937837,616.26284628)
\curveto(863.81937299,616.31284513)(863.92937288,616.37284507)(864.02937837,616.44284628)
\curveto(864.13937267,616.51284493)(864.23437257,616.59284485)(864.31437837,616.68284628)
\curveto(864.39437241,616.78284466)(864.46437234,616.88784456)(864.52437837,616.99784628)
\curveto(864.58437222,617.09784435)(864.63437217,617.20284424)(864.67437837,617.31284628)
\curveto(864.72437208,617.42284402)(864.804372,617.50284394)(864.91437837,617.55284628)
\curveto(864.95437185,617.57284387)(865.01937179,617.58784386)(865.10937837,617.59784628)
\curveto(865.19937161,617.60784384)(865.28937152,617.60784384)(865.37937837,617.59784628)
\curveto(865.46937134,617.59784385)(865.55437125,617.59284385)(865.63437837,617.58284628)
\curveto(865.71437109,617.57284387)(865.76937104,617.55284389)(865.79937837,617.52284628)
\curveto(865.89937091,617.45284399)(865.92437088,617.33784411)(865.87437837,617.17784628)
\curveto(865.79437101,616.90784454)(865.68937112,616.66784478)(865.55937837,616.45784628)
\curveto(865.35937145,616.13784531)(865.12937168,615.87284557)(864.86937837,615.66284628)
\curveto(864.61937219,615.46284598)(864.29937251,615.29784615)(863.90937837,615.16784628)
\curveto(863.809373,615.12784632)(863.7093731,615.10284634)(863.60937837,615.09284628)
\curveto(863.5093733,615.07284637)(863.4043734,615.05284639)(863.29437837,615.03284628)
\curveto(863.24437356,615.02284642)(863.19437361,615.01784643)(863.14437837,615.01784628)
\curveto(863.1043737,615.01784643)(863.05937375,615.01284643)(863.00937837,615.00284628)
\lineto(862.85937837,615.00284628)
\curveto(862.809374,614.99284645)(862.74937406,614.98784646)(862.67937837,614.98784628)
\curveto(862.61937419,614.98784646)(862.56937424,614.99284645)(862.52937837,615.00284628)
\lineto(862.39437837,615.00284628)
\curveto(862.34437446,615.01284643)(862.29937451,615.01784643)(862.25937837,615.01784628)
\curveto(862.21937459,615.01784643)(862.17937463,615.02284642)(862.13937837,615.03284628)
\curveto(862.08937472,615.0428464)(862.03437477,615.05284639)(861.97437837,615.06284628)
\curveto(861.91437489,615.06284638)(861.85937495,615.06784638)(861.80937837,615.07784628)
\curveto(861.71937509,615.09784635)(861.62937518,615.12284632)(861.53937837,615.15284628)
\curveto(861.44937536,615.17284627)(861.36437544,615.19784625)(861.28437837,615.22784628)
\curveto(861.24437556,615.2478462)(861.2093756,615.25784619)(861.17937837,615.25784628)
\curveto(861.14937566,615.26784618)(861.11437569,615.28284616)(861.07437837,615.30284628)
\curveto(860.92437588,615.37284607)(860.76437604,615.45784599)(860.59437837,615.55784628)
\curveto(860.3043765,615.7478457)(860.05437675,615.97784547)(859.84437837,616.24784628)
\curveto(859.64437716,616.52784492)(859.47437733,616.83784461)(859.33437837,617.17784628)
\curveto(859.28437752,617.28784416)(859.24437756,617.40284404)(859.21437837,617.52284628)
\curveto(859.19437761,617.6428438)(859.16437764,617.76284368)(859.12437837,617.88284628)
\curveto(859.11437769,617.92284352)(859.1093777,617.95784349)(859.10937837,617.98784628)
\curveto(859.1093777,618.01784343)(859.1043777,618.05784339)(859.09437837,618.10784628)
\curveto(859.07437773,618.18784326)(859.05937775,618.27284317)(859.04937837,618.36284628)
\curveto(859.03937777,618.45284299)(859.02437778,618.5428429)(859.00437837,618.63284628)
\lineto(859.00437837,618.84284628)
\curveto(858.99437781,618.88284256)(858.98437782,618.93784251)(858.97437837,619.00784628)
\curveto(858.97437783,619.08784236)(858.97937783,619.15284229)(858.98937837,619.20284628)
\lineto(858.98937837,619.36784628)
\curveto(859.0093778,619.41784203)(859.01437779,619.46784198)(859.00437837,619.51784628)
\curveto(859.0043778,619.57784187)(859.0093778,619.63284181)(859.01937837,619.68284628)
\curveto(859.05937775,619.8428416)(859.08937772,620.00284144)(859.10937837,620.16284628)
\curveto(859.13937767,620.32284112)(859.18437762,620.47284097)(859.24437837,620.61284628)
\curveto(859.29437751,620.72284072)(859.33937747,620.83284061)(859.37937837,620.94284628)
\curveto(859.42937738,621.06284038)(859.48437732,621.17784027)(859.54437837,621.28784628)
\curveto(859.76437704,621.63783981)(860.01437679,621.93783951)(860.29437837,622.18784628)
\curveto(860.57437623,622.447839)(860.91937589,622.66283878)(861.32937837,622.83284628)
\curveto(861.44937536,622.88283856)(861.56937524,622.91783853)(861.68937837,622.93784628)
\curveto(861.81937499,622.96783848)(861.95437485,622.99783845)(862.09437837,623.02784628)
\curveto(862.14437466,623.03783841)(862.18937462,623.0428384)(862.22937837,623.04284628)
\curveto(862.26937454,623.05283839)(862.31437449,623.05783839)(862.36437837,623.05784628)
\curveto(862.38437442,623.06783838)(862.4093744,623.06783838)(862.43937837,623.05784628)
\curveto(862.46937434,623.0478384)(862.49437431,623.05283839)(862.51437837,623.07284628)
\curveto(862.93437387,623.08283836)(863.29937351,623.03783841)(863.60937837,622.93784628)
\curveto(863.91937289,622.8478386)(864.19937261,622.72283872)(864.44937837,622.56284628)
\curveto(864.49937231,622.5428389)(864.53937227,622.51283893)(864.56937837,622.47284628)
\curveto(864.59937221,622.442839)(864.63437217,622.41783903)(864.67437837,622.39784628)
\curveto(864.75437205,622.33783911)(864.83437197,622.26783918)(864.91437837,622.18784628)
\curveto(865.0043718,622.10783934)(865.07937173,622.02783942)(865.13937837,621.94784628)
\curveto(865.29937151,621.73783971)(865.43437137,621.53783991)(865.54437837,621.34784628)
\curveto(865.61437119,621.23784021)(865.66937114,621.11784033)(865.70937837,620.98784628)
\curveto(865.74937106,620.85784059)(865.79437101,620.72784072)(865.84437837,620.59784628)
\curveto(865.89437091,620.46784098)(865.92937088,620.33284111)(865.94937837,620.19284628)
\curveto(865.97937083,620.05284139)(866.01437079,619.91284153)(866.05437837,619.77284628)
\curveto(866.06437074,619.70284174)(866.06937074,619.63284181)(866.06937837,619.56284628)
\lineto(866.09937837,619.35284628)
\moveto(864.64437837,619.86284628)
\curveto(864.67437213,619.90284154)(864.69937211,619.95284149)(864.71937837,620.01284628)
\curveto(864.73937207,620.08284136)(864.73937207,620.15284129)(864.71937837,620.22284628)
\curveto(864.65937215,620.442841)(864.57437223,620.6478408)(864.46437837,620.83784628)
\curveto(864.32437248,621.06784038)(864.16937264,621.26284018)(863.99937837,621.42284628)
\curveto(863.82937298,621.58283986)(863.6093732,621.71783973)(863.33937837,621.82784628)
\curveto(863.26937354,621.8478396)(863.19937361,621.86283958)(863.12937837,621.87284628)
\curveto(863.05937375,621.89283955)(862.98437382,621.91283953)(862.90437837,621.93284628)
\curveto(862.82437398,621.95283949)(862.73937407,621.96283948)(862.64937837,621.96284628)
\lineto(862.39437837,621.96284628)
\curveto(862.36437444,621.9428395)(862.32937448,621.93283951)(862.28937837,621.93284628)
\curveto(862.24937456,621.9428395)(862.21437459,621.9428395)(862.18437837,621.93284628)
\lineto(861.94437837,621.87284628)
\curveto(861.87437493,621.86283958)(861.804375,621.8478396)(861.73437837,621.82784628)
\curveto(861.44437536,621.70783974)(861.2093756,621.55783989)(861.02937837,621.37784628)
\curveto(860.85937595,621.19784025)(860.7043761,620.97284047)(860.56437837,620.70284628)
\curveto(860.53437627,620.65284079)(860.5043763,620.58784086)(860.47437837,620.50784628)
\curveto(860.44437636,620.43784101)(860.41937639,620.35784109)(860.39937837,620.26784628)
\curveto(860.37937643,620.17784127)(860.37437643,620.09284135)(860.38437837,620.01284628)
\curveto(860.39437641,619.93284151)(860.42937638,619.87284157)(860.48937837,619.83284628)
\curveto(860.56937624,619.77284167)(860.7043761,619.7428417)(860.89437837,619.74284628)
\curveto(861.09437571,619.75284169)(861.26437554,619.75784169)(861.40437837,619.75784628)
\lineto(863.68437837,619.75784628)
\curveto(863.83437297,619.75784169)(864.01437279,619.75284169)(864.22437837,619.74284628)
\curveto(864.43437237,619.7428417)(864.57437223,619.78284166)(864.64437837,619.86284628)
}
}
{
\newrgbcolor{curcolor}{0 0 0}
\pscustom[linestyle=none,fillstyle=solid,fillcolor=curcolor]
{
\newpath
\moveto(871.096019,623.05784628)
\curveto(871.72601376,623.07783837)(872.23101326,622.99283845)(872.611019,622.80284628)
\curveto(872.9910125,622.61283883)(873.29601219,622.32783912)(873.526019,621.94784628)
\curveto(873.5860119,621.8478396)(873.63101186,621.73783971)(873.661019,621.61784628)
\curveto(873.70101179,621.50783994)(873.73601175,621.39284005)(873.766019,621.27284628)
\curveto(873.81601167,621.08284036)(873.84601164,620.87784057)(873.856019,620.65784628)
\curveto(873.86601162,620.43784101)(873.87101162,620.21284123)(873.871019,619.98284628)
\lineto(873.871019,618.37784628)
\lineto(873.871019,616.03784628)
\curveto(873.87101162,615.86784558)(873.86601162,615.69784575)(873.856019,615.52784628)
\curveto(873.85601163,615.35784609)(873.7910117,615.2478462)(873.661019,615.19784628)
\curveto(873.61101188,615.17784627)(873.55601193,615.16784628)(873.496019,615.16784628)
\curveto(873.44601204,615.15784629)(873.3910121,615.15284629)(873.331019,615.15284628)
\curveto(873.20101229,615.15284629)(873.07601241,615.15784629)(872.956019,615.16784628)
\curveto(872.83601265,615.16784628)(872.75101274,615.20784624)(872.701019,615.28784628)
\curveto(872.65101284,615.35784609)(872.62601286,615.447846)(872.626019,615.55784628)
\lineto(872.626019,615.88784628)
\lineto(872.626019,617.17784628)
\lineto(872.626019,619.62284628)
\curveto(872.62601286,619.89284155)(872.62101287,620.15784129)(872.611019,620.41784628)
\curveto(872.60101289,620.68784076)(872.55601293,620.91784053)(872.476019,621.10784628)
\curveto(872.39601309,621.30784014)(872.27601321,621.46783998)(872.116019,621.58784628)
\curveto(871.95601353,621.71783973)(871.77101372,621.81783963)(871.561019,621.88784628)
\curveto(871.50101399,621.90783954)(871.43601405,621.91783953)(871.366019,621.91784628)
\curveto(871.30601418,621.92783952)(871.24601424,621.9428395)(871.186019,621.96284628)
\curveto(871.13601435,621.97283947)(871.05601443,621.97283947)(870.946019,621.96284628)
\curveto(870.84601464,621.96283948)(870.77601471,621.95783949)(870.736019,621.94784628)
\curveto(870.69601479,621.92783952)(870.66101483,621.91783953)(870.631019,621.91784628)
\curveto(870.60101489,621.92783952)(870.56601492,621.92783952)(870.526019,621.91784628)
\curveto(870.39601509,621.88783956)(870.27101522,621.85283959)(870.151019,621.81284628)
\curveto(870.04101545,621.78283966)(869.93601555,621.73783971)(869.836019,621.67784628)
\curveto(869.79601569,621.65783979)(869.76101573,621.63783981)(869.731019,621.61784628)
\curveto(869.70101579,621.59783985)(869.66601582,621.57783987)(869.626019,621.55784628)
\curveto(869.27601621,621.30784014)(869.02101647,620.93284051)(868.861019,620.43284628)
\curveto(868.83101666,620.35284109)(868.81101668,620.26784118)(868.801019,620.17784628)
\curveto(868.7910167,620.09784135)(868.77601671,620.01784143)(868.756019,619.93784628)
\curveto(868.73601675,619.88784156)(868.73101676,619.83784161)(868.741019,619.78784628)
\curveto(868.75101674,619.7478417)(868.74601674,619.70784174)(868.726019,619.66784628)
\lineto(868.726019,619.35284628)
\curveto(868.71601677,619.32284212)(868.71101678,619.28784216)(868.711019,619.24784628)
\curveto(868.72101677,619.20784224)(868.72601676,619.16284228)(868.726019,619.11284628)
\lineto(868.726019,618.66284628)
\lineto(868.726019,617.22284628)
\lineto(868.726019,615.90284628)
\lineto(868.726019,615.55784628)
\curveto(868.72601676,615.447846)(868.70101679,615.35784609)(868.651019,615.28784628)
\curveto(868.60101689,615.20784624)(868.51101698,615.16784628)(868.381019,615.16784628)
\curveto(868.26101723,615.15784629)(868.13601735,615.15284629)(868.006019,615.15284628)
\curveto(867.92601756,615.15284629)(867.85101764,615.15784629)(867.781019,615.16784628)
\curveto(867.71101778,615.17784627)(867.65101784,615.20284624)(867.601019,615.24284628)
\curveto(867.52101797,615.29284615)(867.48101801,615.38784606)(867.481019,615.52784628)
\lineto(867.481019,615.93284628)
\lineto(867.481019,617.70284628)
\lineto(867.481019,621.33284628)
\lineto(867.481019,622.24784628)
\lineto(867.481019,622.51784628)
\curveto(867.48101801,622.60783884)(867.50101799,622.67783877)(867.541019,622.72784628)
\curveto(867.57101792,622.78783866)(867.62101787,622.82783862)(867.691019,622.84784628)
\curveto(867.73101776,622.85783859)(867.7860177,622.86783858)(867.856019,622.87784628)
\curveto(867.93601755,622.88783856)(868.01601747,622.89283855)(868.096019,622.89284628)
\curveto(868.17601731,622.89283855)(868.25101724,622.88783856)(868.321019,622.87784628)
\curveto(868.40101709,622.86783858)(868.45601703,622.85283859)(868.486019,622.83284628)
\curveto(868.59601689,622.76283868)(868.64601684,622.67283877)(868.636019,622.56284628)
\curveto(868.62601686,622.46283898)(868.64101685,622.3478391)(868.681019,622.21784628)
\curveto(868.70101679,622.15783929)(868.74101675,622.10783934)(868.801019,622.06784628)
\curveto(868.92101657,622.05783939)(869.01601647,622.10283934)(869.086019,622.20284628)
\curveto(869.16601632,622.30283914)(869.24601624,622.38283906)(869.326019,622.44284628)
\curveto(869.46601602,622.5428389)(869.60601588,622.63283881)(869.746019,622.71284628)
\curveto(869.89601559,622.80283864)(870.06601542,622.87783857)(870.256019,622.93784628)
\curveto(870.33601515,622.96783848)(870.42101507,622.98783846)(870.511019,622.99784628)
\curveto(870.61101488,623.00783844)(870.70601478,623.02283842)(870.796019,623.04284628)
\curveto(870.84601464,623.05283839)(870.89601459,623.05783839)(870.946019,623.05784628)
\lineto(871.096019,623.05784628)
}
}
{
\newrgbcolor{curcolor}{0 0 0}
\pscustom[linestyle=none,fillstyle=solid,fillcolor=curcolor]
{
\newpath
\moveto(876.70062837,625.24784628)
\curveto(876.85062636,625.2478362)(877.00062621,625.2428362)(877.15062837,625.23284628)
\curveto(877.30062591,625.23283621)(877.40562581,625.19283625)(877.46562837,625.11284628)
\curveto(877.5156257,625.05283639)(877.54062567,624.96783648)(877.54062837,624.85784628)
\curveto(877.55062566,624.75783669)(877.55562566,624.65283679)(877.55562837,624.54284628)
\lineto(877.55562837,623.67284628)
\curveto(877.55562566,623.59283785)(877.55062566,623.50783794)(877.54062837,623.41784628)
\curveto(877.54062567,623.33783811)(877.55062566,623.26783818)(877.57062837,623.20784628)
\curveto(877.6106256,623.06783838)(877.70062551,622.97783847)(877.84062837,622.93784628)
\curveto(877.89062532,622.92783852)(877.93562528,622.92283852)(877.97562837,622.92284628)
\lineto(878.12562837,622.92284628)
\lineto(878.53062837,622.92284628)
\curveto(878.69062452,622.93283851)(878.80562441,622.92283852)(878.87562837,622.89284628)
\curveto(878.96562425,622.83283861)(879.02562419,622.77283867)(879.05562837,622.71284628)
\curveto(879.07562414,622.67283877)(879.08562413,622.62783882)(879.08562837,622.57784628)
\lineto(879.08562837,622.42784628)
\curveto(879.08562413,622.31783913)(879.08062413,622.21283923)(879.07062837,622.11284628)
\curveto(879.06062415,622.02283942)(879.02562419,621.95283949)(878.96562837,621.90284628)
\curveto(878.90562431,621.85283959)(878.82062439,621.82283962)(878.71062837,621.81284628)
\lineto(878.38062837,621.81284628)
\curveto(878.27062494,621.82283962)(878.16062505,621.82783962)(878.05062837,621.82784628)
\curveto(877.94062527,621.82783962)(877.84562537,621.81283963)(877.76562837,621.78284628)
\curveto(877.69562552,621.75283969)(877.64562557,621.70283974)(877.61562837,621.63284628)
\curveto(877.58562563,621.56283988)(877.56562565,621.47783997)(877.55562837,621.37784628)
\curveto(877.54562567,621.28784016)(877.54062567,621.18784026)(877.54062837,621.07784628)
\curveto(877.55062566,620.97784047)(877.55562566,620.87784057)(877.55562837,620.77784628)
\lineto(877.55562837,617.80784628)
\curveto(877.55562566,617.58784386)(877.55062566,617.35284409)(877.54062837,617.10284628)
\curveto(877.54062567,616.86284458)(877.58562563,616.67784477)(877.67562837,616.54784628)
\curveto(877.72562549,616.46784498)(877.79062542,616.41284503)(877.87062837,616.38284628)
\curveto(877.95062526,616.35284509)(878.04562517,616.32784512)(878.15562837,616.30784628)
\curveto(878.18562503,616.29784515)(878.215625,616.29284515)(878.24562837,616.29284628)
\curveto(878.28562493,616.30284514)(878.32062489,616.30284514)(878.35062837,616.29284628)
\lineto(878.54562837,616.29284628)
\curveto(878.64562457,616.29284515)(878.73562448,616.28284516)(878.81562837,616.26284628)
\curveto(878.90562431,616.25284519)(878.97062424,616.21784523)(879.01062837,616.15784628)
\curveto(879.03062418,616.12784532)(879.04562417,616.07284537)(879.05562837,615.99284628)
\curveto(879.07562414,615.92284552)(879.08562413,615.8478456)(879.08562837,615.76784628)
\curveto(879.09562412,615.68784576)(879.09562412,615.60784584)(879.08562837,615.52784628)
\curveto(879.07562414,615.45784599)(879.05562416,615.40284604)(879.02562837,615.36284628)
\curveto(878.98562423,615.29284615)(878.9106243,615.2428462)(878.80062837,615.21284628)
\curveto(878.72062449,615.19284625)(878.63062458,615.18284626)(878.53062837,615.18284628)
\curveto(878.43062478,615.19284625)(878.34062487,615.19784625)(878.26062837,615.19784628)
\curveto(878.20062501,615.19784625)(878.14062507,615.19284625)(878.08062837,615.18284628)
\curveto(878.02062519,615.18284626)(877.96562525,615.18784626)(877.91562837,615.19784628)
\lineto(877.73562837,615.19784628)
\curveto(877.68562553,615.20784624)(877.63562558,615.21284623)(877.58562837,615.21284628)
\curveto(877.54562567,615.22284622)(877.50062571,615.22784622)(877.45062837,615.22784628)
\curveto(877.25062596,615.27784617)(877.07562614,615.33284611)(876.92562837,615.39284628)
\curveto(876.78562643,615.45284599)(876.66562655,615.55784589)(876.56562837,615.70784628)
\curveto(876.42562679,615.90784554)(876.34562687,616.15784529)(876.32562837,616.45784628)
\curveto(876.30562691,616.76784468)(876.29562692,617.09784435)(876.29562837,617.44784628)
\lineto(876.29562837,621.37784628)
\curveto(876.26562695,621.50783994)(876.23562698,621.60283984)(876.20562837,621.66284628)
\curveto(876.18562703,621.72283972)(876.1156271,621.77283967)(875.99562837,621.81284628)
\curveto(875.95562726,621.82283962)(875.9156273,621.82283962)(875.87562837,621.81284628)
\curveto(875.83562738,621.80283964)(875.79562742,621.80783964)(875.75562837,621.82784628)
\lineto(875.51562837,621.82784628)
\curveto(875.38562783,621.82783962)(875.27562794,621.83783961)(875.18562837,621.85784628)
\curveto(875.10562811,621.88783956)(875.05062816,621.9478395)(875.02062837,622.03784628)
\curveto(875.00062821,622.07783937)(874.98562823,622.12283932)(874.97562837,622.17284628)
\lineto(874.97562837,622.32284628)
\curveto(874.97562824,622.46283898)(874.98562823,622.57783887)(875.00562837,622.66784628)
\curveto(875.02562819,622.76783868)(875.08562813,622.8428386)(875.18562837,622.89284628)
\curveto(875.29562792,622.93283851)(875.43562778,622.9428385)(875.60562837,622.92284628)
\curveto(875.78562743,622.90283854)(875.93562728,622.91283853)(876.05562837,622.95284628)
\curveto(876.14562707,623.00283844)(876.215627,623.07283837)(876.26562837,623.16284628)
\curveto(876.28562693,623.22283822)(876.29562692,623.29783815)(876.29562837,623.38784628)
\lineto(876.29562837,623.64284628)
\lineto(876.29562837,624.57284628)
\lineto(876.29562837,624.81284628)
\curveto(876.29562692,624.90283654)(876.30562691,624.97783647)(876.32562837,625.03784628)
\curveto(876.36562685,625.11783633)(876.44062677,625.18283626)(876.55062837,625.23284628)
\curveto(876.58062663,625.23283621)(876.60562661,625.23283621)(876.62562837,625.23284628)
\curveto(876.65562656,625.2428362)(876.68062653,625.2478362)(876.70062837,625.24784628)
}
}
{
\newrgbcolor{curcolor}{0 0 0}
\pscustom[linestyle=none,fillstyle=solid,fillcolor=curcolor]
{
\newpath
\moveto(887.22242525,619.35284628)
\curveto(887.24241756,619.25284219)(887.24241756,619.13784231)(887.22242525,619.00784628)
\curveto(887.21241759,618.88784256)(887.18241762,618.80284264)(887.13242525,618.75284628)
\curveto(887.08241772,618.71284273)(887.0074178,618.68284276)(886.90742525,618.66284628)
\curveto(886.81741799,618.65284279)(886.71241809,618.6478428)(886.59242525,618.64784628)
\lineto(886.23242525,618.64784628)
\curveto(886.11241869,618.65784279)(886.0074188,618.66284278)(885.91742525,618.66284628)
\lineto(882.07742525,618.66284628)
\curveto(881.99742281,618.66284278)(881.91742289,618.65784279)(881.83742525,618.64784628)
\curveto(881.75742305,618.6478428)(881.69242311,618.63284281)(881.64242525,618.60284628)
\curveto(881.6024232,618.58284286)(881.56242324,618.5428429)(881.52242525,618.48284628)
\curveto(881.5024233,618.45284299)(881.48242332,618.40784304)(881.46242525,618.34784628)
\curveto(881.44242336,618.29784315)(881.44242336,618.2478432)(881.46242525,618.19784628)
\curveto(881.47242333,618.1478433)(881.47742333,618.10284334)(881.47742525,618.06284628)
\curveto(881.47742333,618.02284342)(881.48242332,617.98284346)(881.49242525,617.94284628)
\curveto(881.51242329,617.86284358)(881.53242327,617.77784367)(881.55242525,617.68784628)
\curveto(881.57242323,617.60784384)(881.6024232,617.52784392)(881.64242525,617.44784628)
\curveto(881.87242293,616.90784454)(882.25242255,616.52284492)(882.78242525,616.29284628)
\curveto(882.84242196,616.26284518)(882.9074219,616.23784521)(882.97742525,616.21784628)
\lineto(883.18742525,616.15784628)
\curveto(883.21742159,616.1478453)(883.26742154,616.1428453)(883.33742525,616.14284628)
\curveto(883.47742133,616.10284534)(883.66242114,616.08284536)(883.89242525,616.08284628)
\curveto(884.12242068,616.08284536)(884.3074205,616.10284534)(884.44742525,616.14284628)
\curveto(884.58742022,616.18284526)(884.71242009,616.22284522)(884.82242525,616.26284628)
\curveto(884.94241986,616.31284513)(885.05241975,616.37284507)(885.15242525,616.44284628)
\curveto(885.26241954,616.51284493)(885.35741945,616.59284485)(885.43742525,616.68284628)
\curveto(885.51741929,616.78284466)(885.58741922,616.88784456)(885.64742525,616.99784628)
\curveto(885.7074191,617.09784435)(885.75741905,617.20284424)(885.79742525,617.31284628)
\curveto(885.84741896,617.42284402)(885.92741888,617.50284394)(886.03742525,617.55284628)
\curveto(886.07741873,617.57284387)(886.14241866,617.58784386)(886.23242525,617.59784628)
\curveto(886.32241848,617.60784384)(886.41241839,617.60784384)(886.50242525,617.59784628)
\curveto(886.59241821,617.59784385)(886.67741813,617.59284385)(886.75742525,617.58284628)
\curveto(886.83741797,617.57284387)(886.89241791,617.55284389)(886.92242525,617.52284628)
\curveto(887.02241778,617.45284399)(887.04741776,617.33784411)(886.99742525,617.17784628)
\curveto(886.91741789,616.90784454)(886.81241799,616.66784478)(886.68242525,616.45784628)
\curveto(886.48241832,616.13784531)(886.25241855,615.87284557)(885.99242525,615.66284628)
\curveto(885.74241906,615.46284598)(885.42241938,615.29784615)(885.03242525,615.16784628)
\curveto(884.93241987,615.12784632)(884.83241997,615.10284634)(884.73242525,615.09284628)
\curveto(884.63242017,615.07284637)(884.52742028,615.05284639)(884.41742525,615.03284628)
\curveto(884.36742044,615.02284642)(884.31742049,615.01784643)(884.26742525,615.01784628)
\curveto(884.22742058,615.01784643)(884.18242062,615.01284643)(884.13242525,615.00284628)
\lineto(883.98242525,615.00284628)
\curveto(883.93242087,614.99284645)(883.87242093,614.98784646)(883.80242525,614.98784628)
\curveto(883.74242106,614.98784646)(883.69242111,614.99284645)(883.65242525,615.00284628)
\lineto(883.51742525,615.00284628)
\curveto(883.46742134,615.01284643)(883.42242138,615.01784643)(883.38242525,615.01784628)
\curveto(883.34242146,615.01784643)(883.3024215,615.02284642)(883.26242525,615.03284628)
\curveto(883.21242159,615.0428464)(883.15742165,615.05284639)(883.09742525,615.06284628)
\curveto(883.03742177,615.06284638)(882.98242182,615.06784638)(882.93242525,615.07784628)
\curveto(882.84242196,615.09784635)(882.75242205,615.12284632)(882.66242525,615.15284628)
\curveto(882.57242223,615.17284627)(882.48742232,615.19784625)(882.40742525,615.22784628)
\curveto(882.36742244,615.2478462)(882.33242247,615.25784619)(882.30242525,615.25784628)
\curveto(882.27242253,615.26784618)(882.23742257,615.28284616)(882.19742525,615.30284628)
\curveto(882.04742276,615.37284607)(881.88742292,615.45784599)(881.71742525,615.55784628)
\curveto(881.42742338,615.7478457)(881.17742363,615.97784547)(880.96742525,616.24784628)
\curveto(880.76742404,616.52784492)(880.59742421,616.83784461)(880.45742525,617.17784628)
\curveto(880.4074244,617.28784416)(880.36742444,617.40284404)(880.33742525,617.52284628)
\curveto(880.31742449,617.6428438)(880.28742452,617.76284368)(880.24742525,617.88284628)
\curveto(880.23742457,617.92284352)(880.23242457,617.95784349)(880.23242525,617.98784628)
\curveto(880.23242457,618.01784343)(880.22742458,618.05784339)(880.21742525,618.10784628)
\curveto(880.19742461,618.18784326)(880.18242462,618.27284317)(880.17242525,618.36284628)
\curveto(880.16242464,618.45284299)(880.14742466,618.5428429)(880.12742525,618.63284628)
\lineto(880.12742525,618.84284628)
\curveto(880.11742469,618.88284256)(880.1074247,618.93784251)(880.09742525,619.00784628)
\curveto(880.09742471,619.08784236)(880.1024247,619.15284229)(880.11242525,619.20284628)
\lineto(880.11242525,619.36784628)
\curveto(880.13242467,619.41784203)(880.13742467,619.46784198)(880.12742525,619.51784628)
\curveto(880.12742468,619.57784187)(880.13242467,619.63284181)(880.14242525,619.68284628)
\curveto(880.18242462,619.8428416)(880.21242459,620.00284144)(880.23242525,620.16284628)
\curveto(880.26242454,620.32284112)(880.3074245,620.47284097)(880.36742525,620.61284628)
\curveto(880.41742439,620.72284072)(880.46242434,620.83284061)(880.50242525,620.94284628)
\curveto(880.55242425,621.06284038)(880.6074242,621.17784027)(880.66742525,621.28784628)
\curveto(880.88742392,621.63783981)(881.13742367,621.93783951)(881.41742525,622.18784628)
\curveto(881.69742311,622.447839)(882.04242276,622.66283878)(882.45242525,622.83284628)
\curveto(882.57242223,622.88283856)(882.69242211,622.91783853)(882.81242525,622.93784628)
\curveto(882.94242186,622.96783848)(883.07742173,622.99783845)(883.21742525,623.02784628)
\curveto(883.26742154,623.03783841)(883.31242149,623.0428384)(883.35242525,623.04284628)
\curveto(883.39242141,623.05283839)(883.43742137,623.05783839)(883.48742525,623.05784628)
\curveto(883.5074213,623.06783838)(883.53242127,623.06783838)(883.56242525,623.05784628)
\curveto(883.59242121,623.0478384)(883.61742119,623.05283839)(883.63742525,623.07284628)
\curveto(884.05742075,623.08283836)(884.42242038,623.03783841)(884.73242525,622.93784628)
\curveto(885.04241976,622.8478386)(885.32241948,622.72283872)(885.57242525,622.56284628)
\curveto(885.62241918,622.5428389)(885.66241914,622.51283893)(885.69242525,622.47284628)
\curveto(885.72241908,622.442839)(885.75741905,622.41783903)(885.79742525,622.39784628)
\curveto(885.87741893,622.33783911)(885.95741885,622.26783918)(886.03742525,622.18784628)
\curveto(886.12741868,622.10783934)(886.2024186,622.02783942)(886.26242525,621.94784628)
\curveto(886.42241838,621.73783971)(886.55741825,621.53783991)(886.66742525,621.34784628)
\curveto(886.73741807,621.23784021)(886.79241801,621.11784033)(886.83242525,620.98784628)
\curveto(886.87241793,620.85784059)(886.91741789,620.72784072)(886.96742525,620.59784628)
\curveto(887.01741779,620.46784098)(887.05241775,620.33284111)(887.07242525,620.19284628)
\curveto(887.1024177,620.05284139)(887.13741767,619.91284153)(887.17742525,619.77284628)
\curveto(887.18741762,619.70284174)(887.19241761,619.63284181)(887.19242525,619.56284628)
\lineto(887.22242525,619.35284628)
\moveto(885.76742525,619.86284628)
\curveto(885.79741901,619.90284154)(885.82241898,619.95284149)(885.84242525,620.01284628)
\curveto(885.86241894,620.08284136)(885.86241894,620.15284129)(885.84242525,620.22284628)
\curveto(885.78241902,620.442841)(885.69741911,620.6478408)(885.58742525,620.83784628)
\curveto(885.44741936,621.06784038)(885.29241951,621.26284018)(885.12242525,621.42284628)
\curveto(884.95241985,621.58283986)(884.73242007,621.71783973)(884.46242525,621.82784628)
\curveto(884.39242041,621.8478396)(884.32242048,621.86283958)(884.25242525,621.87284628)
\curveto(884.18242062,621.89283955)(884.1074207,621.91283953)(884.02742525,621.93284628)
\curveto(883.94742086,621.95283949)(883.86242094,621.96283948)(883.77242525,621.96284628)
\lineto(883.51742525,621.96284628)
\curveto(883.48742132,621.9428395)(883.45242135,621.93283951)(883.41242525,621.93284628)
\curveto(883.37242143,621.9428395)(883.33742147,621.9428395)(883.30742525,621.93284628)
\lineto(883.06742525,621.87284628)
\curveto(882.99742181,621.86283958)(882.92742188,621.8478396)(882.85742525,621.82784628)
\curveto(882.56742224,621.70783974)(882.33242247,621.55783989)(882.15242525,621.37784628)
\curveto(881.98242282,621.19784025)(881.82742298,620.97284047)(881.68742525,620.70284628)
\curveto(881.65742315,620.65284079)(881.62742318,620.58784086)(881.59742525,620.50784628)
\curveto(881.56742324,620.43784101)(881.54242326,620.35784109)(881.52242525,620.26784628)
\curveto(881.5024233,620.17784127)(881.49742331,620.09284135)(881.50742525,620.01284628)
\curveto(881.51742329,619.93284151)(881.55242325,619.87284157)(881.61242525,619.83284628)
\curveto(881.69242311,619.77284167)(881.82742298,619.7428417)(882.01742525,619.74284628)
\curveto(882.21742259,619.75284169)(882.38742242,619.75784169)(882.52742525,619.75784628)
\lineto(884.80742525,619.75784628)
\curveto(884.95741985,619.75784169)(885.13741967,619.75284169)(885.34742525,619.74284628)
\curveto(885.55741925,619.7428417)(885.69741911,619.78284166)(885.76742525,619.86284628)
}
}
{
\newrgbcolor{curcolor}{0.60000002 0.60000002 0.60000002}
\pscustom[linestyle=none,fillstyle=solid,fillcolor=curcolor]
{
\newpath
\moveto(812.80437349,625.8928829)
\lineto(827.80437349,625.8928829)
\lineto(827.80437349,610.8928829)
\lineto(812.80437349,610.8928829)
\closepath
}
}
{
\newrgbcolor{curcolor}{0 0 0}
\pscustom[linestyle=none,fillstyle=solid,fillcolor=curcolor]
{
\newpath
\moveto(832.8525815,602.82714071)
\lineto(833.7675815,602.82714071)
\curveto(833.86757885,602.82713002)(833.96257875,602.82713002)(834.0525815,602.82714071)
\curveto(834.14257857,602.82713002)(834.2175785,602.80713004)(834.2775815,602.76714071)
\curveto(834.36757835,602.70713014)(834.42757829,602.62713022)(834.4575815,602.52714071)
\curveto(834.49757822,602.42713042)(834.54257817,602.32213052)(834.5925815,602.21214071)
\curveto(834.67257804,602.02213082)(834.74257797,601.83213101)(834.8025815,601.64214071)
\curveto(834.87257784,601.45213139)(834.94757777,601.26213158)(835.0275815,601.07214071)
\curveto(835.09757762,600.89213195)(835.16257755,600.70713214)(835.2225815,600.51714071)
\curveto(835.28257743,600.33713251)(835.35257736,600.15713269)(835.4325815,599.97714071)
\curveto(835.49257722,599.83713301)(835.54757717,599.69213315)(835.5975815,599.54214071)
\curveto(835.64757707,599.39213345)(835.70257701,599.2471336)(835.7625815,599.10714071)
\curveto(835.94257677,598.65713419)(836.1125766,598.20213464)(836.2725815,597.74214071)
\curveto(836.43257628,597.29213555)(836.60257611,596.842136)(836.7825815,596.39214071)
\curveto(836.80257591,596.3421365)(836.8175759,596.29213655)(836.8275815,596.24214071)
\lineto(836.8875815,596.09214071)
\curveto(836.97757574,595.87213697)(837.06257565,595.6471372)(837.1425815,595.41714071)
\curveto(837.22257549,595.19713765)(837.30757541,594.97713787)(837.3975815,594.75714071)
\curveto(837.43757528,594.66713818)(837.47757524,594.55713829)(837.5175815,594.42714071)
\curveto(837.55757516,594.30713854)(837.62257509,594.23713861)(837.7125815,594.21714071)
\curveto(837.75257496,594.20713864)(837.78257493,594.20713864)(837.8025815,594.21714071)
\lineto(837.8625815,594.27714071)
\curveto(837.9125748,594.32713852)(837.94757477,594.38213846)(837.9675815,594.44214071)
\curveto(837.99757472,594.50213834)(838.02757469,594.56713828)(838.0575815,594.63714071)
\lineto(838.2975815,595.26714071)
\curveto(838.37757434,595.48713736)(838.45757426,595.70213714)(838.5375815,595.91214071)
\lineto(838.5975815,596.06214071)
\lineto(838.6575815,596.24214071)
\curveto(838.73757398,596.43213641)(838.80757391,596.62213622)(838.8675815,596.81214071)
\curveto(838.93757378,597.01213583)(839.0125737,597.21213563)(839.0925815,597.41214071)
\curveto(839.33257338,597.99213485)(839.55257316,598.57713427)(839.7525815,599.16714071)
\curveto(839.96257275,599.75713309)(840.18757253,600.3421325)(840.4275815,600.92214071)
\curveto(840.50757221,601.12213172)(840.58257213,601.32713152)(840.6525815,601.53714071)
\curveto(840.73257198,601.7471311)(840.8125719,601.95213089)(840.8925815,602.15214071)
\curveto(840.93257178,602.23213061)(840.96757175,602.33213051)(840.9975815,602.45214071)
\curveto(841.03757168,602.57213027)(841.09257162,602.65713019)(841.1625815,602.70714071)
\curveto(841.22257149,602.7471301)(841.29757142,602.77713007)(841.3875815,602.79714071)
\curveto(841.48757123,602.81713003)(841.59757112,602.82713002)(841.7175815,602.82714071)
\curveto(841.83757088,602.83713001)(841.95757076,602.83713001)(842.0775815,602.82714071)
\curveto(842.19757052,602.82713002)(842.30757041,602.82713002)(842.4075815,602.82714071)
\curveto(842.49757022,602.82713002)(842.58757013,602.82713002)(842.6775815,602.82714071)
\curveto(842.77756994,602.82713002)(842.85256986,602.80713004)(842.9025815,602.76714071)
\curveto(842.99256972,602.71713013)(843.04256967,602.62713022)(843.0525815,602.49714071)
\curveto(843.06256965,602.36713048)(843.06756965,602.22713062)(843.0675815,602.07714071)
\lineto(843.0675815,600.42714071)
\lineto(843.0675815,594.15714071)
\lineto(843.0675815,592.89714071)
\curveto(843.06756965,592.78714006)(843.06756965,592.67714017)(843.0675815,592.56714071)
\curveto(843.07756964,592.45714039)(843.05756966,592.37214047)(843.0075815,592.31214071)
\curveto(842.97756974,592.25214059)(842.93256978,592.21214063)(842.8725815,592.19214071)
\curveto(842.8125699,592.18214066)(842.74256997,592.16714068)(842.6625815,592.14714071)
\lineto(842.4225815,592.14714071)
\lineto(842.0625815,592.14714071)
\curveto(841.95257076,592.15714069)(841.87257084,592.20214064)(841.8225815,592.28214071)
\curveto(841.80257091,592.31214053)(841.78757093,592.3421405)(841.7775815,592.37214071)
\curveto(841.77757094,592.41214043)(841.76757095,592.45714039)(841.7475815,592.50714071)
\lineto(841.7475815,592.67214071)
\curveto(841.73757098,592.73214011)(841.73257098,592.80214004)(841.7325815,592.88214071)
\curveto(841.74257097,592.96213988)(841.74757097,593.03713981)(841.7475815,593.10714071)
\lineto(841.7475815,593.94714071)
\lineto(841.7475815,598.37214071)
\curveto(841.74757097,598.62213422)(841.74757097,598.87213397)(841.7475815,599.12214071)
\curveto(841.74757097,599.38213346)(841.74257097,599.63213321)(841.7325815,599.87214071)
\curveto(841.73257098,599.97213287)(841.72757099,600.08213276)(841.7175815,600.20214071)
\curveto(841.70757101,600.32213252)(841.65257106,600.38213246)(841.5525815,600.38214071)
\lineto(841.5525815,600.36714071)
\curveto(841.48257123,600.3471325)(841.42257129,600.28213256)(841.3725815,600.17214071)
\curveto(841.33257138,600.06213278)(841.29757142,599.96713288)(841.2675815,599.88714071)
\curveto(841.19757152,599.71713313)(841.13257158,599.5421333)(841.0725815,599.36214071)
\curveto(841.0125717,599.19213365)(840.94257177,599.02213382)(840.8625815,598.85214071)
\curveto(840.84257187,598.80213404)(840.82757189,598.75713409)(840.8175815,598.71714071)
\curveto(840.80757191,598.67713417)(840.79257192,598.63213421)(840.7725815,598.58214071)
\curveto(840.69257202,598.40213444)(840.62257209,598.21713463)(840.5625815,598.02714071)
\curveto(840.5125722,597.847135)(840.44757227,597.66713518)(840.3675815,597.48714071)
\curveto(840.29757242,597.33713551)(840.23757248,597.18213566)(840.1875815,597.02214071)
\curveto(840.13757258,596.87213597)(840.08257263,596.72213612)(840.0225815,596.57214071)
\curveto(839.82257289,596.10213674)(839.64257307,595.62713722)(839.4825815,595.14714071)
\curveto(839.32257339,594.67713817)(839.14757357,594.21213863)(838.9575815,593.75214071)
\curveto(838.87757384,593.57213927)(838.80757391,593.39213945)(838.7475815,593.21214071)
\curveto(838.68757403,593.03213981)(838.62257409,592.85213999)(838.5525815,592.67214071)
\curveto(838.50257421,592.56214028)(838.45257426,592.45714039)(838.4025815,592.35714071)
\curveto(838.36257435,592.26714058)(838.27757444,592.20214064)(838.1475815,592.16214071)
\curveto(838.12757459,592.15214069)(838.10257461,592.1471407)(838.0725815,592.14714071)
\curveto(838.05257466,592.15714069)(838.02757469,592.15714069)(837.9975815,592.14714071)
\curveto(837.96757475,592.13714071)(837.93257478,592.13214071)(837.8925815,592.13214071)
\curveto(837.85257486,592.1421407)(837.8125749,592.1471407)(837.7725815,592.14714071)
\lineto(837.4725815,592.14714071)
\curveto(837.37257534,592.1471407)(837.29257542,592.17214067)(837.2325815,592.22214071)
\curveto(837.15257556,592.27214057)(837.09257562,592.3421405)(837.0525815,592.43214071)
\curveto(837.02257569,592.53214031)(836.98257573,592.63214021)(836.9325815,592.73214071)
\curveto(836.85257586,592.93213991)(836.77257594,593.13713971)(836.6925815,593.34714071)
\curveto(836.62257609,593.56713928)(836.54757617,593.77713907)(836.4675815,593.97714071)
\curveto(836.38757633,594.15713869)(836.3175764,594.33713851)(836.2575815,594.51714071)
\curveto(836.20757651,594.70713814)(836.14257657,594.89213795)(836.0625815,595.07214071)
\curveto(835.83257688,595.63213721)(835.6175771,596.19713665)(835.4175815,596.76714071)
\curveto(835.2175775,597.33713551)(835.00257771,597.90213494)(834.7725815,598.46214071)
\lineto(834.5325815,599.09214071)
\curveto(834.46257825,599.31213353)(834.38757833,599.52213332)(834.3075815,599.72214071)
\curveto(834.25757846,599.83213301)(834.2125785,599.93713291)(834.1725815,600.03714071)
\curveto(834.14257857,600.1471327)(834.09257862,600.2421326)(834.0225815,600.32214071)
\curveto(834.0125787,600.3421325)(834.00257871,600.35213249)(833.9925815,600.35214071)
\lineto(833.9625815,600.38214071)
\lineto(833.8875815,600.38214071)
\lineto(833.8575815,600.35214071)
\curveto(833.84757887,600.35213249)(833.83757888,600.3471325)(833.8275815,600.33714071)
\curveto(833.80757891,600.28713256)(833.79757892,600.23213261)(833.7975815,600.17214071)
\curveto(833.79757892,600.11213273)(833.78757893,600.05213279)(833.7675815,599.99214071)
\lineto(833.7675815,599.82714071)
\curveto(833.74757897,599.76713308)(833.74257897,599.70213314)(833.7525815,599.63214071)
\curveto(833.76257895,599.56213328)(833.76757895,599.49213335)(833.7675815,599.42214071)
\lineto(833.7675815,598.61214071)
\lineto(833.7675815,594.05214071)
\lineto(833.7675815,592.86714071)
\curveto(833.76757895,592.75714009)(833.76257895,592.6471402)(833.7525815,592.53714071)
\curveto(833.75257896,592.42714042)(833.72757899,592.3421405)(833.6775815,592.28214071)
\curveto(833.62757909,592.20214064)(833.53757918,592.15714069)(833.4075815,592.14714071)
\lineto(833.0175815,592.14714071)
\lineto(832.8225815,592.14714071)
\curveto(832.77257994,592.1471407)(832.72257999,592.15714069)(832.6725815,592.17714071)
\curveto(832.54258017,592.21714063)(832.46758025,592.30214054)(832.4475815,592.43214071)
\curveto(832.43758028,592.56214028)(832.43258028,592.71214013)(832.4325815,592.88214071)
\lineto(832.4325815,594.62214071)
\lineto(832.4325815,600.62214071)
\lineto(832.4325815,602.03214071)
\curveto(832.43258028,602.1421307)(832.42758029,602.25713059)(832.4175815,602.37714071)
\curveto(832.4175803,602.49713035)(832.44258027,602.59213025)(832.4925815,602.66214071)
\curveto(832.53258018,602.72213012)(832.60758011,602.77213007)(832.7175815,602.81214071)
\curveto(832.73757998,602.82213002)(832.75757996,602.82213002)(832.7775815,602.81214071)
\curveto(832.80757991,602.81213003)(832.83257988,602.81713003)(832.8525815,602.82714071)
}
}
{
\newrgbcolor{curcolor}{0 0 0}
\pscustom[linestyle=none,fillstyle=solid,fillcolor=curcolor]
{
\newpath
\moveto(852.29469087,596.34714071)
\curveto(852.31468281,596.28713656)(852.3246828,596.19213665)(852.32469087,596.06214071)
\curveto(852.3246828,595.9421369)(852.31968281,595.85713699)(852.30969087,595.80714071)
\lineto(852.30969087,595.65714071)
\curveto(852.29968283,595.57713727)(852.28968284,595.50213734)(852.27969087,595.43214071)
\curveto(852.27968285,595.37213747)(852.27468285,595.30213754)(852.26469087,595.22214071)
\curveto(852.24468288,595.16213768)(852.2296829,595.10213774)(852.21969087,595.04214071)
\curveto(852.21968291,594.98213786)(852.20968292,594.92213792)(852.18969087,594.86214071)
\curveto(852.14968298,594.73213811)(852.11468301,594.60213824)(852.08469087,594.47214071)
\curveto(852.05468307,594.3421385)(852.01468311,594.22213862)(851.96469087,594.11214071)
\curveto(851.75468337,593.63213921)(851.47468365,593.22713962)(851.12469087,592.89714071)
\curveto(850.77468435,592.57714027)(850.34468478,592.33214051)(849.83469087,592.16214071)
\curveto(849.7246854,592.12214072)(849.60468552,592.09214075)(849.47469087,592.07214071)
\curveto(849.35468577,592.05214079)(849.2296859,592.03214081)(849.09969087,592.01214071)
\curveto(849.03968609,592.00214084)(848.97468615,591.99714085)(848.90469087,591.99714071)
\curveto(848.84468628,591.98714086)(848.78468634,591.98214086)(848.72469087,591.98214071)
\curveto(848.68468644,591.97214087)(848.6246865,591.96714088)(848.54469087,591.96714071)
\curveto(848.47468665,591.96714088)(848.4246867,591.97214087)(848.39469087,591.98214071)
\curveto(848.35468677,591.99214085)(848.31468681,591.99714085)(848.27469087,591.99714071)
\curveto(848.23468689,591.98714086)(848.19968693,591.98714086)(848.16969087,591.99714071)
\lineto(848.07969087,591.99714071)
\lineto(847.71969087,592.04214071)
\curveto(847.57968755,592.08214076)(847.44468768,592.12214072)(847.31469087,592.16214071)
\curveto(847.18468794,592.20214064)(847.05968807,592.2471406)(846.93969087,592.29714071)
\curveto(846.48968864,592.49714035)(846.11968901,592.75714009)(845.82969087,593.07714071)
\curveto(845.53968959,593.39713945)(845.29968983,593.78713906)(845.10969087,594.24714071)
\curveto(845.05969007,594.3471385)(845.01969011,594.4471384)(844.98969087,594.54714071)
\curveto(844.96969016,594.6471382)(844.94969018,594.75213809)(844.92969087,594.86214071)
\curveto(844.90969022,594.90213794)(844.89969023,594.93213791)(844.89969087,594.95214071)
\curveto(844.90969022,594.98213786)(844.90969022,595.01713783)(844.89969087,595.05714071)
\curveto(844.87969025,595.13713771)(844.86469026,595.21713763)(844.85469087,595.29714071)
\curveto(844.85469027,595.38713746)(844.84469028,595.47213737)(844.82469087,595.55214071)
\lineto(844.82469087,595.67214071)
\curveto(844.8246903,595.71213713)(844.81969031,595.75713709)(844.80969087,595.80714071)
\curveto(844.79969033,595.85713699)(844.79469033,595.9421369)(844.79469087,596.06214071)
\curveto(844.79469033,596.19213665)(844.80469032,596.28713656)(844.82469087,596.34714071)
\curveto(844.84469028,596.41713643)(844.84969028,596.48713636)(844.83969087,596.55714071)
\curveto(844.8296903,596.62713622)(844.83469029,596.69713615)(844.85469087,596.76714071)
\curveto(844.86469026,596.81713603)(844.86969026,596.85713599)(844.86969087,596.88714071)
\curveto(844.87969025,596.92713592)(844.88969024,596.97213587)(844.89969087,597.02214071)
\curveto(844.9296902,597.1421357)(844.95469017,597.26213558)(844.97469087,597.38214071)
\curveto(845.00469012,597.50213534)(845.04469008,597.61713523)(845.09469087,597.72714071)
\curveto(845.24468988,598.09713475)(845.4246897,598.42713442)(845.63469087,598.71714071)
\curveto(845.85468927,599.01713383)(846.11968901,599.26713358)(846.42969087,599.46714071)
\curveto(846.54968858,599.5471333)(846.67468845,599.61213323)(846.80469087,599.66214071)
\curveto(846.93468819,599.72213312)(847.06968806,599.78213306)(847.20969087,599.84214071)
\curveto(847.3296878,599.89213295)(847.45968767,599.92213292)(847.59969087,599.93214071)
\curveto(847.73968739,599.95213289)(847.87968725,599.98213286)(848.01969087,600.02214071)
\lineto(848.21469087,600.02214071)
\curveto(848.28468684,600.03213281)(848.34968678,600.0421328)(848.40969087,600.05214071)
\curveto(849.29968583,600.06213278)(850.03968509,599.87713297)(850.62969087,599.49714071)
\curveto(851.21968391,599.11713373)(851.64468348,598.62213422)(851.90469087,598.01214071)
\curveto(851.95468317,597.91213493)(851.99468313,597.81213503)(852.02469087,597.71214071)
\curveto(852.05468307,597.61213523)(852.08968304,597.50713534)(852.12969087,597.39714071)
\curveto(852.15968297,597.28713556)(852.18468294,597.16713568)(852.20469087,597.03714071)
\curveto(852.2246829,596.91713593)(852.24968288,596.79213605)(852.27969087,596.66214071)
\curveto(852.28968284,596.61213623)(852.28968284,596.55713629)(852.27969087,596.49714071)
\curveto(852.27968285,596.4471364)(852.28468284,596.39713645)(852.29469087,596.34714071)
\moveto(850.95969087,595.49214071)
\curveto(850.97968415,595.56213728)(850.98468414,595.6421372)(850.97469087,595.73214071)
\lineto(850.97469087,595.98714071)
\curveto(850.97468415,596.37713647)(850.93968419,596.70713614)(850.86969087,596.97714071)
\curveto(850.83968429,597.05713579)(850.81468431,597.13713571)(850.79469087,597.21714071)
\curveto(850.77468435,597.29713555)(850.74968438,597.37213547)(850.71969087,597.44214071)
\curveto(850.43968469,598.09213475)(849.99468513,598.5421343)(849.38469087,598.79214071)
\curveto(849.31468581,598.82213402)(849.23968589,598.842134)(849.15969087,598.85214071)
\lineto(848.91969087,598.91214071)
\curveto(848.83968629,598.93213391)(848.75468637,598.9421339)(848.66469087,598.94214071)
\lineto(848.39469087,598.94214071)
\lineto(848.12469087,598.89714071)
\curveto(848.0246871,598.87713397)(847.9296872,598.85213399)(847.83969087,598.82214071)
\curveto(847.75968737,598.80213404)(847.67968745,598.77213407)(847.59969087,598.73214071)
\curveto(847.5296876,598.71213413)(847.46468766,598.68213416)(847.40469087,598.64214071)
\curveto(847.34468778,598.60213424)(847.28968784,598.56213428)(847.23969087,598.52214071)
\curveto(846.99968813,598.35213449)(846.80468832,598.1471347)(846.65469087,597.90714071)
\curveto(846.50468862,597.66713518)(846.37468875,597.38713546)(846.26469087,597.06714071)
\curveto(846.23468889,596.96713588)(846.21468891,596.86213598)(846.20469087,596.75214071)
\curveto(846.19468893,596.65213619)(846.17968895,596.5471363)(846.15969087,596.43714071)
\curveto(846.14968898,596.39713645)(846.14468898,596.33213651)(846.14469087,596.24214071)
\curveto(846.13468899,596.21213663)(846.129689,596.17713667)(846.12969087,596.13714071)
\curveto(846.13968899,596.09713675)(846.14468898,596.05213679)(846.14469087,596.00214071)
\lineto(846.14469087,595.70214071)
\curveto(846.14468898,595.60213724)(846.15468897,595.51213733)(846.17469087,595.43214071)
\lineto(846.20469087,595.25214071)
\curveto(846.2246889,595.15213769)(846.23968889,595.05213779)(846.24969087,594.95214071)
\curveto(846.26968886,594.86213798)(846.29968883,594.77713807)(846.33969087,594.69714071)
\curveto(846.43968869,594.45713839)(846.55468857,594.23213861)(846.68469087,594.02214071)
\curveto(846.8246883,593.81213903)(846.99468813,593.63713921)(847.19469087,593.49714071)
\curveto(847.24468788,593.46713938)(847.28968784,593.4421394)(847.32969087,593.42214071)
\curveto(847.36968776,593.40213944)(847.41468771,593.37713947)(847.46469087,593.34714071)
\curveto(847.54468758,593.29713955)(847.6296875,593.25213959)(847.71969087,593.21214071)
\curveto(847.81968731,593.18213966)(847.9246872,593.15213969)(848.03469087,593.12214071)
\curveto(848.08468704,593.10213974)(848.129687,593.09213975)(848.16969087,593.09214071)
\curveto(848.21968691,593.10213974)(848.26968686,593.10213974)(848.31969087,593.09214071)
\curveto(848.34968678,593.08213976)(848.40968672,593.07213977)(848.49969087,593.06214071)
\curveto(848.59968653,593.05213979)(848.67468645,593.05713979)(848.72469087,593.07714071)
\curveto(848.76468636,593.08713976)(848.80468632,593.08713976)(848.84469087,593.07714071)
\curveto(848.88468624,593.07713977)(848.9246862,593.08713976)(848.96469087,593.10714071)
\curveto(849.04468608,593.12713972)(849.124686,593.1421397)(849.20469087,593.15214071)
\curveto(849.28468584,593.17213967)(849.35968577,593.19713965)(849.42969087,593.22714071)
\curveto(849.76968536,593.36713948)(850.04468508,593.56213928)(850.25469087,593.81214071)
\curveto(850.46468466,594.06213878)(850.63968449,594.35713849)(850.77969087,594.69714071)
\curveto(850.8296843,594.81713803)(850.85968427,594.9421379)(850.86969087,595.07214071)
\curveto(850.88968424,595.21213763)(850.91968421,595.35213749)(850.95969087,595.49214071)
}
}
{
\newrgbcolor{curcolor}{0 0 0}
\pscustom[linestyle=none,fillstyle=solid,fillcolor=curcolor]
{
\newpath
\moveto(860.74297212,592.95714071)
\lineto(860.74297212,592.56714071)
\curveto(860.74296425,592.4471404)(860.71796427,592.3471405)(860.66797212,592.26714071)
\curveto(860.61796437,592.19714065)(860.53296446,592.15714069)(860.41297212,592.14714071)
\lineto(860.06797212,592.14714071)
\curveto(860.00796498,592.1471407)(859.94796504,592.1421407)(859.88797212,592.13214071)
\curveto(859.83796515,592.13214071)(859.7929652,592.1421407)(859.75297212,592.16214071)
\curveto(859.66296533,592.18214066)(859.60296539,592.22214062)(859.57297212,592.28214071)
\curveto(859.53296546,592.33214051)(859.50796548,592.39214045)(859.49797212,592.46214071)
\curveto(859.49796549,592.53214031)(859.48296551,592.60214024)(859.45297212,592.67214071)
\curveto(859.44296555,592.69214015)(859.42796556,592.70714014)(859.40797212,592.71714071)
\curveto(859.39796559,592.73714011)(859.38296561,592.75714009)(859.36297212,592.77714071)
\curveto(859.26296573,592.78714006)(859.18296581,592.76714008)(859.12297212,592.71714071)
\curveto(859.07296592,592.66714018)(859.01796597,592.61714023)(858.95797212,592.56714071)
\curveto(858.75796623,592.41714043)(858.55796643,592.30214054)(858.35797212,592.22214071)
\curveto(858.17796681,592.1421407)(857.96796702,592.08214076)(857.72797212,592.04214071)
\curveto(857.49796749,592.00214084)(857.25796773,591.98214086)(857.00797212,591.98214071)
\curveto(856.76796822,591.97214087)(856.52796846,591.98714086)(856.28797212,592.02714071)
\curveto(856.04796894,592.05714079)(855.83796915,592.11214073)(855.65797212,592.19214071)
\curveto(855.13796985,592.41214043)(854.71797027,592.70714014)(854.39797212,593.07714071)
\curveto(854.07797091,593.45713939)(853.82797116,593.92713892)(853.64797212,594.48714071)
\curveto(853.60797138,594.57713827)(853.57797141,594.66713818)(853.55797212,594.75714071)
\curveto(853.54797144,594.85713799)(853.52797146,594.95713789)(853.49797212,595.05714071)
\curveto(853.4879715,595.10713774)(853.48297151,595.15713769)(853.48297212,595.20714071)
\curveto(853.48297151,595.25713759)(853.47797151,595.30713754)(853.46797212,595.35714071)
\curveto(853.44797154,595.40713744)(853.43797155,595.45713739)(853.43797212,595.50714071)
\curveto(853.44797154,595.56713728)(853.44797154,595.62213722)(853.43797212,595.67214071)
\lineto(853.43797212,595.82214071)
\curveto(853.41797157,595.87213697)(853.40797158,595.93713691)(853.40797212,596.01714071)
\curveto(853.40797158,596.09713675)(853.41797157,596.16213668)(853.43797212,596.21214071)
\lineto(853.43797212,596.37714071)
\curveto(853.45797153,596.4471364)(853.46297153,596.51713633)(853.45297212,596.58714071)
\curveto(853.45297154,596.66713618)(853.46297153,596.7421361)(853.48297212,596.81214071)
\curveto(853.4929715,596.86213598)(853.49797149,596.90713594)(853.49797212,596.94714071)
\curveto(853.49797149,596.98713586)(853.50297149,597.03213581)(853.51297212,597.08214071)
\curveto(853.54297145,597.18213566)(853.56797142,597.27713557)(853.58797212,597.36714071)
\curveto(853.60797138,597.46713538)(853.63297136,597.56213528)(853.66297212,597.65214071)
\curveto(853.7929712,598.03213481)(853.95797103,598.37213447)(854.15797212,598.67214071)
\curveto(854.36797062,598.98213386)(854.61797037,599.23713361)(854.90797212,599.43714071)
\curveto(855.07796991,599.55713329)(855.25296974,599.65713319)(855.43297212,599.73714071)
\curveto(855.62296937,599.81713303)(855.82796916,599.88713296)(856.04797212,599.94714071)
\curveto(856.11796887,599.95713289)(856.18296881,599.96713288)(856.24297212,599.97714071)
\curveto(856.31296868,599.98713286)(856.38296861,600.00213284)(856.45297212,600.02214071)
\lineto(856.60297212,600.02214071)
\curveto(856.68296831,600.0421328)(856.79796819,600.05213279)(856.94797212,600.05214071)
\curveto(857.10796788,600.05213279)(857.22796776,600.0421328)(857.30797212,600.02214071)
\curveto(857.34796764,600.01213283)(857.40296759,600.00713284)(857.47297212,600.00714071)
\curveto(857.58296741,599.97713287)(857.6929673,599.95213289)(857.80297212,599.93214071)
\curveto(857.91296708,599.92213292)(858.01796697,599.89213295)(858.11797212,599.84214071)
\curveto(858.26796672,599.78213306)(858.40796658,599.71713313)(858.53797212,599.64714071)
\curveto(858.67796631,599.57713327)(858.80796618,599.49713335)(858.92797212,599.40714071)
\curveto(858.987966,599.35713349)(859.04796594,599.30213354)(859.10797212,599.24214071)
\curveto(859.17796581,599.19213365)(859.26796572,599.17713367)(859.37797212,599.19714071)
\curveto(859.39796559,599.22713362)(859.41296558,599.25213359)(859.42297212,599.27214071)
\curveto(859.44296555,599.29213355)(859.45796553,599.32213352)(859.46797212,599.36214071)
\curveto(859.49796549,599.45213339)(859.50796548,599.56713328)(859.49797212,599.70714071)
\lineto(859.49797212,600.08214071)
\lineto(859.49797212,601.80714071)
\lineto(859.49797212,602.27214071)
\curveto(859.49796549,602.45213039)(859.52296547,602.58213026)(859.57297212,602.66214071)
\curveto(859.61296538,602.73213011)(859.67296532,602.77713007)(859.75297212,602.79714071)
\curveto(859.77296522,602.79713005)(859.79796519,602.79713005)(859.82797212,602.79714071)
\curveto(859.85796513,602.80713004)(859.88296511,602.81213003)(859.90297212,602.81214071)
\curveto(860.04296495,602.82213002)(860.1879648,602.82213002)(860.33797212,602.81214071)
\curveto(860.49796449,602.81213003)(860.60796438,602.77213007)(860.66797212,602.69214071)
\curveto(860.71796427,602.61213023)(860.74296425,602.51213033)(860.74297212,602.39214071)
\lineto(860.74297212,602.01714071)
\lineto(860.74297212,592.95714071)
\moveto(859.52797212,595.79214071)
\curveto(859.54796544,595.842137)(859.55796543,595.90713694)(859.55797212,595.98714071)
\curveto(859.55796543,596.07713677)(859.54796544,596.1471367)(859.52797212,596.19714071)
\lineto(859.52797212,596.42214071)
\curveto(859.50796548,596.51213633)(859.4929655,596.60213624)(859.48297212,596.69214071)
\curveto(859.47296552,596.79213605)(859.45296554,596.88213596)(859.42297212,596.96214071)
\curveto(859.40296559,597.0421358)(859.38296561,597.11713573)(859.36297212,597.18714071)
\curveto(859.35296564,597.25713559)(859.33296566,597.32713552)(859.30297212,597.39714071)
\curveto(859.18296581,597.69713515)(859.02796596,597.96213488)(858.83797212,598.19214071)
\curveto(858.64796634,598.42213442)(858.40796658,598.60213424)(858.11797212,598.73214071)
\curveto(858.01796697,598.78213406)(857.91296708,598.81713403)(857.80297212,598.83714071)
\curveto(857.70296729,598.86713398)(857.5929674,598.89213395)(857.47297212,598.91214071)
\curveto(857.3929676,598.93213391)(857.30296769,598.9421339)(857.20297212,598.94214071)
\lineto(856.93297212,598.94214071)
\curveto(856.88296811,598.93213391)(856.83796815,598.92213392)(856.79797212,598.91214071)
\lineto(856.66297212,598.91214071)
\curveto(856.58296841,598.89213395)(856.49796849,598.87213397)(856.40797212,598.85214071)
\curveto(856.32796866,598.83213401)(856.24796874,598.80713404)(856.16797212,598.77714071)
\curveto(855.84796914,598.63713421)(855.5879694,598.43213441)(855.38797212,598.16214071)
\curveto(855.19796979,597.90213494)(855.04296995,597.59713525)(854.92297212,597.24714071)
\curveto(854.88297011,597.13713571)(854.85297014,597.02213582)(854.83297212,596.90214071)
\curveto(854.82297017,596.79213605)(854.80797018,596.68213616)(854.78797212,596.57214071)
\curveto(854.7879702,596.53213631)(854.78297021,596.49213635)(854.77297212,596.45214071)
\lineto(854.77297212,596.34714071)
\curveto(854.75297024,596.29713655)(854.74297025,596.2421366)(854.74297212,596.18214071)
\curveto(854.75297024,596.12213672)(854.75797023,596.06713678)(854.75797212,596.01714071)
\lineto(854.75797212,595.68714071)
\curveto(854.75797023,595.58713726)(854.76797022,595.49213735)(854.78797212,595.40214071)
\curveto(854.79797019,595.37213747)(854.80297019,595.32213752)(854.80297212,595.25214071)
\curveto(854.82297017,595.18213766)(854.83797015,595.11213773)(854.84797212,595.04214071)
\lineto(854.90797212,594.83214071)
\curveto(855.01796997,594.48213836)(855.16796982,594.18213866)(855.35797212,593.93214071)
\curveto(855.54796944,593.68213916)(855.7879692,593.47713937)(856.07797212,593.31714071)
\curveto(856.16796882,593.26713958)(856.25796873,593.22713962)(856.34797212,593.19714071)
\curveto(856.43796855,593.16713968)(856.53796845,593.13713971)(856.64797212,593.10714071)
\curveto(856.69796829,593.08713976)(856.74796824,593.08213976)(856.79797212,593.09214071)
\curveto(856.85796813,593.10213974)(856.91296808,593.09713975)(856.96297212,593.07714071)
\curveto(857.00296799,593.06713978)(857.04296795,593.06213978)(857.08297212,593.06214071)
\lineto(857.21797212,593.06214071)
\lineto(857.35297212,593.06214071)
\curveto(857.38296761,593.07213977)(857.43296756,593.07713977)(857.50297212,593.07714071)
\curveto(857.58296741,593.09713975)(857.66296733,593.11213973)(857.74297212,593.12214071)
\curveto(857.82296717,593.1421397)(857.89796709,593.16713968)(857.96797212,593.19714071)
\curveto(858.29796669,593.33713951)(858.56296643,593.51213933)(858.76297212,593.72214071)
\curveto(858.97296602,593.9421389)(859.14796584,594.21713863)(859.28797212,594.54714071)
\curveto(859.33796565,594.65713819)(859.37296562,594.76713808)(859.39297212,594.87714071)
\curveto(859.41296558,594.98713786)(859.43796555,595.09713775)(859.46797212,595.20714071)
\curveto(859.4879655,595.2471376)(859.49796549,595.28213756)(859.49797212,595.31214071)
\curveto(859.49796549,595.35213749)(859.50296549,595.39213745)(859.51297212,595.43214071)
\curveto(859.52296547,595.49213735)(859.52296547,595.55213729)(859.51297212,595.61214071)
\curveto(859.51296548,595.67213717)(859.51796547,595.73213711)(859.52797212,595.79214071)
}
}
{
\newrgbcolor{curcolor}{0 0 0}
\pscustom[linestyle=none,fillstyle=solid,fillcolor=curcolor]
{
\newpath
\moveto(869.43922212,596.31714071)
\curveto(869.45921444,596.21713663)(869.45921444,596.10213674)(869.43922212,595.97214071)
\curveto(869.42921447,595.85213699)(869.3992145,595.76713708)(869.34922212,595.71714071)
\curveto(869.2992146,595.67713717)(869.22421467,595.6471372)(869.12422212,595.62714071)
\curveto(869.03421486,595.61713723)(868.92921497,595.61213723)(868.80922212,595.61214071)
\lineto(868.44922212,595.61214071)
\curveto(868.32921557,595.62213722)(868.22421567,595.62713722)(868.13422212,595.62714071)
\lineto(864.29422212,595.62714071)
\curveto(864.21421968,595.62713722)(864.13421976,595.62213722)(864.05422212,595.61214071)
\curveto(863.97421992,595.61213723)(863.90921999,595.59713725)(863.85922212,595.56714071)
\curveto(863.81922008,595.5471373)(863.77922012,595.50713734)(863.73922212,595.44714071)
\curveto(863.71922018,595.41713743)(863.6992202,595.37213747)(863.67922212,595.31214071)
\curveto(863.65922024,595.26213758)(863.65922024,595.21213763)(863.67922212,595.16214071)
\curveto(863.68922021,595.11213773)(863.6942202,595.06713778)(863.69422212,595.02714071)
\curveto(863.6942202,594.98713786)(863.6992202,594.9471379)(863.70922212,594.90714071)
\curveto(863.72922017,594.82713802)(863.74922015,594.7421381)(863.76922212,594.65214071)
\curveto(863.78922011,594.57213827)(863.81922008,594.49213835)(863.85922212,594.41214071)
\curveto(864.08921981,593.87213897)(864.46921943,593.48713936)(864.99922212,593.25714071)
\curveto(865.05921884,593.22713962)(865.12421877,593.20213964)(865.19422212,593.18214071)
\lineto(865.40422212,593.12214071)
\curveto(865.43421846,593.11213973)(865.48421841,593.10713974)(865.55422212,593.10714071)
\curveto(865.6942182,593.06713978)(865.87921802,593.0471398)(866.10922212,593.04714071)
\curveto(866.33921756,593.0471398)(866.52421737,593.06713978)(866.66422212,593.10714071)
\curveto(866.80421709,593.1471397)(866.92921697,593.18713966)(867.03922212,593.22714071)
\curveto(867.15921674,593.27713957)(867.26921663,593.33713951)(867.36922212,593.40714071)
\curveto(867.47921642,593.47713937)(867.57421632,593.55713929)(867.65422212,593.64714071)
\curveto(867.73421616,593.7471391)(867.80421609,593.85213899)(867.86422212,593.96214071)
\curveto(867.92421597,594.06213878)(867.97421592,594.16713868)(868.01422212,594.27714071)
\curveto(868.06421583,594.38713846)(868.14421575,594.46713838)(868.25422212,594.51714071)
\curveto(868.2942156,594.53713831)(868.35921554,594.55213829)(868.44922212,594.56214071)
\curveto(868.53921536,594.57213827)(868.62921527,594.57213827)(868.71922212,594.56214071)
\curveto(868.80921509,594.56213828)(868.894215,594.55713829)(868.97422212,594.54714071)
\curveto(869.05421484,594.53713831)(869.10921479,594.51713833)(869.13922212,594.48714071)
\curveto(869.23921466,594.41713843)(869.26421463,594.30213854)(869.21422212,594.14214071)
\curveto(869.13421476,593.87213897)(869.02921487,593.63213921)(868.89922212,593.42214071)
\curveto(868.6992152,593.10213974)(868.46921543,592.83714001)(868.20922212,592.62714071)
\curveto(867.95921594,592.42714042)(867.63921626,592.26214058)(867.24922212,592.13214071)
\curveto(867.14921675,592.09214075)(867.04921685,592.06714078)(866.94922212,592.05714071)
\curveto(866.84921705,592.03714081)(866.74421715,592.01714083)(866.63422212,591.99714071)
\curveto(866.58421731,591.98714086)(866.53421736,591.98214086)(866.48422212,591.98214071)
\curveto(866.44421745,591.98214086)(866.3992175,591.97714087)(866.34922212,591.96714071)
\lineto(866.19922212,591.96714071)
\curveto(866.14921775,591.95714089)(866.08921781,591.95214089)(866.01922212,591.95214071)
\curveto(865.95921794,591.95214089)(865.90921799,591.95714089)(865.86922212,591.96714071)
\lineto(865.73422212,591.96714071)
\curveto(865.68421821,591.97714087)(865.63921826,591.98214086)(865.59922212,591.98214071)
\curveto(865.55921834,591.98214086)(865.51921838,591.98714086)(865.47922212,591.99714071)
\curveto(865.42921847,592.00714084)(865.37421852,592.01714083)(865.31422212,592.02714071)
\curveto(865.25421864,592.02714082)(865.1992187,592.03214081)(865.14922212,592.04214071)
\curveto(865.05921884,592.06214078)(864.96921893,592.08714076)(864.87922212,592.11714071)
\curveto(864.78921911,592.13714071)(864.70421919,592.16214068)(864.62422212,592.19214071)
\curveto(864.58421931,592.21214063)(864.54921935,592.22214062)(864.51922212,592.22214071)
\curveto(864.48921941,592.23214061)(864.45421944,592.2471406)(864.41422212,592.26714071)
\curveto(864.26421963,592.33714051)(864.10421979,592.42214042)(863.93422212,592.52214071)
\curveto(863.64422025,592.71214013)(863.3942205,592.9421399)(863.18422212,593.21214071)
\curveto(862.98422091,593.49213935)(862.81422108,593.80213904)(862.67422212,594.14214071)
\curveto(862.62422127,594.25213859)(862.58422131,594.36713848)(862.55422212,594.48714071)
\curveto(862.53422136,594.60713824)(862.50422139,594.72713812)(862.46422212,594.84714071)
\curveto(862.45422144,594.88713796)(862.44922145,594.92213792)(862.44922212,594.95214071)
\curveto(862.44922145,594.98213786)(862.44422145,595.02213782)(862.43422212,595.07214071)
\curveto(862.41422148,595.15213769)(862.3992215,595.23713761)(862.38922212,595.32714071)
\curveto(862.37922152,595.41713743)(862.36422153,595.50713734)(862.34422212,595.59714071)
\lineto(862.34422212,595.80714071)
\curveto(862.33422156,595.847137)(862.32422157,595.90213694)(862.31422212,595.97214071)
\curveto(862.31422158,596.05213679)(862.31922158,596.11713673)(862.32922212,596.16714071)
\lineto(862.32922212,596.33214071)
\curveto(862.34922155,596.38213646)(862.35422154,596.43213641)(862.34422212,596.48214071)
\curveto(862.34422155,596.5421363)(862.34922155,596.59713625)(862.35922212,596.64714071)
\curveto(862.3992215,596.80713604)(862.42922147,596.96713588)(862.44922212,597.12714071)
\curveto(862.47922142,597.28713556)(862.52422137,597.43713541)(862.58422212,597.57714071)
\curveto(862.63422126,597.68713516)(862.67922122,597.79713505)(862.71922212,597.90714071)
\curveto(862.76922113,598.02713482)(862.82422107,598.1421347)(862.88422212,598.25214071)
\curveto(863.10422079,598.60213424)(863.35422054,598.90213394)(863.63422212,599.15214071)
\curveto(863.91421998,599.41213343)(864.25921964,599.62713322)(864.66922212,599.79714071)
\curveto(864.78921911,599.847133)(864.90921899,599.88213296)(865.02922212,599.90214071)
\curveto(865.15921874,599.93213291)(865.2942186,599.96213288)(865.43422212,599.99214071)
\curveto(865.48421841,600.00213284)(865.52921837,600.00713284)(865.56922212,600.00714071)
\curveto(865.60921829,600.01713283)(865.65421824,600.02213282)(865.70422212,600.02214071)
\curveto(865.72421817,600.03213281)(865.74921815,600.03213281)(865.77922212,600.02214071)
\curveto(865.80921809,600.01213283)(865.83421806,600.01713283)(865.85422212,600.03714071)
\curveto(866.27421762,600.0471328)(866.63921726,600.00213284)(866.94922212,599.90214071)
\curveto(867.25921664,599.81213303)(867.53921636,599.68713316)(867.78922212,599.52714071)
\curveto(867.83921606,599.50713334)(867.87921602,599.47713337)(867.90922212,599.43714071)
\curveto(867.93921596,599.40713344)(867.97421592,599.38213346)(868.01422212,599.36214071)
\curveto(868.0942158,599.30213354)(868.17421572,599.23213361)(868.25422212,599.15214071)
\curveto(868.34421555,599.07213377)(868.41921548,598.99213385)(868.47922212,598.91214071)
\curveto(868.63921526,598.70213414)(868.77421512,598.50213434)(868.88422212,598.31214071)
\curveto(868.95421494,598.20213464)(869.00921489,598.08213476)(869.04922212,597.95214071)
\curveto(869.08921481,597.82213502)(869.13421476,597.69213515)(869.18422212,597.56214071)
\curveto(869.23421466,597.43213541)(869.26921463,597.29713555)(869.28922212,597.15714071)
\curveto(869.31921458,597.01713583)(869.35421454,596.87713597)(869.39422212,596.73714071)
\curveto(869.40421449,596.66713618)(869.40921449,596.59713625)(869.40922212,596.52714071)
\lineto(869.43922212,596.31714071)
\moveto(867.98422212,596.82714071)
\curveto(868.01421588,596.86713598)(868.03921586,596.91713593)(868.05922212,596.97714071)
\curveto(868.07921582,597.0471358)(868.07921582,597.11713573)(868.05922212,597.18714071)
\curveto(867.9992159,597.40713544)(867.91421598,597.61213523)(867.80422212,597.80214071)
\curveto(867.66421623,598.03213481)(867.50921639,598.22713462)(867.33922212,598.38714071)
\curveto(867.16921673,598.5471343)(866.94921695,598.68213416)(866.67922212,598.79214071)
\curveto(866.60921729,598.81213403)(866.53921736,598.82713402)(866.46922212,598.83714071)
\curveto(866.3992175,598.85713399)(866.32421757,598.87713397)(866.24422212,598.89714071)
\curveto(866.16421773,598.91713393)(866.07921782,598.92713392)(865.98922212,598.92714071)
\lineto(865.73422212,598.92714071)
\curveto(865.70421819,598.90713394)(865.66921823,598.89713395)(865.62922212,598.89714071)
\curveto(865.58921831,598.90713394)(865.55421834,598.90713394)(865.52422212,598.89714071)
\lineto(865.28422212,598.83714071)
\curveto(865.21421868,598.82713402)(865.14421875,598.81213403)(865.07422212,598.79214071)
\curveto(864.78421911,598.67213417)(864.54921935,598.52213432)(864.36922212,598.34214071)
\curveto(864.1992197,598.16213468)(864.04421985,597.93713491)(863.90422212,597.66714071)
\curveto(863.87422002,597.61713523)(863.84422005,597.55213529)(863.81422212,597.47214071)
\curveto(863.78422011,597.40213544)(863.75922014,597.32213552)(863.73922212,597.23214071)
\curveto(863.71922018,597.1421357)(863.71422018,597.05713579)(863.72422212,596.97714071)
\curveto(863.73422016,596.89713595)(863.76922013,596.83713601)(863.82922212,596.79714071)
\curveto(863.90921999,596.73713611)(864.04421985,596.70713614)(864.23422212,596.70714071)
\curveto(864.43421946,596.71713613)(864.60421929,596.72213612)(864.74422212,596.72214071)
\lineto(867.02422212,596.72214071)
\curveto(867.17421672,596.72213612)(867.35421654,596.71713613)(867.56422212,596.70714071)
\curveto(867.77421612,596.70713614)(867.91421598,596.7471361)(867.98422212,596.82714071)
}
}
{
\newrgbcolor{curcolor}{0 0 0}
\pscustom[linestyle=none,fillstyle=solid,fillcolor=curcolor]
{
\newpath
\moveto(874.39086275,600.05214071)
\curveto(874.62085796,600.05213279)(874.75085783,599.99213285)(874.78086275,599.87214071)
\curveto(874.81085777,599.76213308)(874.82585775,599.59713325)(874.82586275,599.37714071)
\lineto(874.82586275,599.09214071)
\curveto(874.82585775,599.00213384)(874.80085778,598.92713392)(874.75086275,598.86714071)
\curveto(874.69085789,598.78713406)(874.60585797,598.7421341)(874.49586275,598.73214071)
\curveto(874.38585819,598.73213411)(874.2758583,598.71713413)(874.16586275,598.68714071)
\curveto(874.02585855,598.65713419)(873.89085869,598.62713422)(873.76086275,598.59714071)
\curveto(873.64085894,598.56713428)(873.52585905,598.52713432)(873.41586275,598.47714071)
\curveto(873.12585945,598.3471345)(872.89085969,598.16713468)(872.71086275,597.93714071)
\curveto(872.53086005,597.71713513)(872.3758602,597.46213538)(872.24586275,597.17214071)
\curveto(872.20586037,597.06213578)(872.1758604,596.9471359)(872.15586275,596.82714071)
\curveto(872.13586044,596.71713613)(872.11086047,596.60213624)(872.08086275,596.48214071)
\curveto(872.07086051,596.43213641)(872.06586051,596.38213646)(872.06586275,596.33214071)
\curveto(872.0758605,596.28213656)(872.0758605,596.23213661)(872.06586275,596.18214071)
\curveto(872.03586054,596.06213678)(872.02086056,595.92213692)(872.02086275,595.76214071)
\curveto(872.03086055,595.61213723)(872.03586054,595.46713738)(872.03586275,595.32714071)
\lineto(872.03586275,593.48214071)
\lineto(872.03586275,593.13714071)
\curveto(872.03586054,593.01713983)(872.03086055,592.90213994)(872.02086275,592.79214071)
\curveto(872.01086057,592.68214016)(872.00586057,592.58714026)(872.00586275,592.50714071)
\curveto(872.01586056,592.42714042)(871.99586058,592.35714049)(871.94586275,592.29714071)
\curveto(871.89586068,592.22714062)(871.81586076,592.18714066)(871.70586275,592.17714071)
\curveto(871.60586097,592.16714068)(871.49586108,592.16214068)(871.37586275,592.16214071)
\lineto(871.10586275,592.16214071)
\curveto(871.05586152,592.18214066)(871.00586157,592.19714065)(870.95586275,592.20714071)
\curveto(870.91586166,592.22714062)(870.88586169,592.25214059)(870.86586275,592.28214071)
\curveto(870.81586176,592.35214049)(870.78586179,592.43714041)(870.77586275,592.53714071)
\lineto(870.77586275,592.86714071)
\lineto(870.77586275,594.02214071)
\lineto(870.77586275,598.17714071)
\lineto(870.77586275,599.21214071)
\lineto(870.77586275,599.51214071)
\curveto(870.78586179,599.61213323)(870.81586176,599.69713315)(870.86586275,599.76714071)
\curveto(870.89586168,599.80713304)(870.94586163,599.83713301)(871.01586275,599.85714071)
\curveto(871.09586148,599.87713297)(871.1808614,599.88713296)(871.27086275,599.88714071)
\curveto(871.36086122,599.89713295)(871.45086113,599.89713295)(871.54086275,599.88714071)
\curveto(871.63086095,599.87713297)(871.70086088,599.86213298)(871.75086275,599.84214071)
\curveto(871.83086075,599.81213303)(871.8808607,599.75213309)(871.90086275,599.66214071)
\curveto(871.93086065,599.58213326)(871.94586063,599.49213335)(871.94586275,599.39214071)
\lineto(871.94586275,599.09214071)
\curveto(871.94586063,598.99213385)(871.96586061,598.90213394)(872.00586275,598.82214071)
\curveto(872.01586056,598.80213404)(872.02586055,598.78713406)(872.03586275,598.77714071)
\lineto(872.08086275,598.73214071)
\curveto(872.19086039,598.73213411)(872.2808603,598.77713407)(872.35086275,598.86714071)
\curveto(872.42086016,598.96713388)(872.4808601,599.0471338)(872.53086275,599.10714071)
\lineto(872.62086275,599.19714071)
\curveto(872.71085987,599.30713354)(872.83585974,599.42213342)(872.99586275,599.54214071)
\curveto(873.15585942,599.66213318)(873.30585927,599.75213309)(873.44586275,599.81214071)
\curveto(873.53585904,599.86213298)(873.63085895,599.89713295)(873.73086275,599.91714071)
\curveto(873.83085875,599.9471329)(873.93585864,599.97713287)(874.04586275,600.00714071)
\curveto(874.10585847,600.01713283)(874.16585841,600.02213282)(874.22586275,600.02214071)
\curveto(874.28585829,600.03213281)(874.34085824,600.0421328)(874.39086275,600.05214071)
}
}
{
\newrgbcolor{curcolor}{0 0 0}
\pscustom[linestyle=none,fillstyle=solid,fillcolor=curcolor]
{
\newpath
\moveto(882.64062837,592.70214071)
\curveto(882.67062054,592.5421403)(882.65562056,592.40714044)(882.59562837,592.29714071)
\curveto(882.53562068,592.19714065)(882.45562076,592.12214072)(882.35562837,592.07214071)
\curveto(882.30562091,592.05214079)(882.25062096,592.0421408)(882.19062837,592.04214071)
\curveto(882.14062107,592.0421408)(882.08562113,592.03214081)(882.02562837,592.01214071)
\curveto(881.80562141,591.96214088)(881.58562163,591.97714087)(881.36562837,592.05714071)
\curveto(881.15562206,592.12714072)(881.0106222,592.21714063)(880.93062837,592.32714071)
\curveto(880.88062233,592.39714045)(880.83562238,592.47714037)(880.79562837,592.56714071)
\curveto(880.75562246,592.66714018)(880.70562251,592.7471401)(880.64562837,592.80714071)
\curveto(880.62562259,592.82714002)(880.60062261,592.84714)(880.57062837,592.86714071)
\curveto(880.55062266,592.88713996)(880.52062269,592.89213995)(880.48062837,592.88214071)
\curveto(880.37062284,592.85213999)(880.26562295,592.79714005)(880.16562837,592.71714071)
\curveto(880.07562314,592.63714021)(879.98562323,592.56714028)(879.89562837,592.50714071)
\curveto(879.76562345,592.42714042)(879.62562359,592.35214049)(879.47562837,592.28214071)
\curveto(879.32562389,592.22214062)(879.16562405,592.16714068)(878.99562837,592.11714071)
\curveto(878.89562432,592.08714076)(878.78562443,592.06714078)(878.66562837,592.05714071)
\curveto(878.55562466,592.0471408)(878.44562477,592.03214081)(878.33562837,592.01214071)
\curveto(878.28562493,592.00214084)(878.24062497,591.99714085)(878.20062837,591.99714071)
\lineto(878.09562837,591.99714071)
\curveto(877.98562523,591.97714087)(877.88062533,591.97714087)(877.78062837,591.99714071)
\lineto(877.64562837,591.99714071)
\curveto(877.59562562,592.00714084)(877.54562567,592.01214083)(877.49562837,592.01214071)
\curveto(877.44562577,592.01214083)(877.40062581,592.02214082)(877.36062837,592.04214071)
\curveto(877.32062589,592.05214079)(877.28562593,592.05714079)(877.25562837,592.05714071)
\curveto(877.23562598,592.0471408)(877.210626,592.0471408)(877.18062837,592.05714071)
\lineto(876.94062837,592.11714071)
\curveto(876.86062635,592.12714072)(876.78562643,592.1471407)(876.71562837,592.17714071)
\curveto(876.4156268,592.30714054)(876.17062704,592.45214039)(875.98062837,592.61214071)
\curveto(875.80062741,592.78214006)(875.65062756,593.01713983)(875.53062837,593.31714071)
\curveto(875.44062777,593.53713931)(875.39562782,593.80213904)(875.39562837,594.11214071)
\lineto(875.39562837,594.42714071)
\curveto(875.40562781,594.47713837)(875.4106278,594.52713832)(875.41062837,594.57714071)
\lineto(875.44062837,594.75714071)
\lineto(875.56062837,595.08714071)
\curveto(875.60062761,595.19713765)(875.65062756,595.29713755)(875.71062837,595.38714071)
\curveto(875.89062732,595.67713717)(876.13562708,595.89213695)(876.44562837,596.03214071)
\curveto(876.75562646,596.17213667)(877.09562612,596.29713655)(877.46562837,596.40714071)
\curveto(877.60562561,596.4471364)(877.75062546,596.47713637)(877.90062837,596.49714071)
\curveto(878.05062516,596.51713633)(878.20062501,596.5421363)(878.35062837,596.57214071)
\curveto(878.42062479,596.59213625)(878.48562473,596.60213624)(878.54562837,596.60214071)
\curveto(878.6156246,596.60213624)(878.69062452,596.61213623)(878.77062837,596.63214071)
\curveto(878.84062437,596.65213619)(878.9106243,596.66213618)(878.98062837,596.66214071)
\curveto(879.05062416,596.67213617)(879.12562409,596.68713616)(879.20562837,596.70714071)
\curveto(879.45562376,596.76713608)(879.69062352,596.81713603)(879.91062837,596.85714071)
\curveto(880.13062308,596.90713594)(880.30562291,597.02213582)(880.43562837,597.20214071)
\curveto(880.49562272,597.28213556)(880.54562267,597.38213546)(880.58562837,597.50214071)
\curveto(880.62562259,597.63213521)(880.62562259,597.77213507)(880.58562837,597.92214071)
\curveto(880.52562269,598.16213468)(880.43562278,598.35213449)(880.31562837,598.49214071)
\curveto(880.20562301,598.63213421)(880.04562317,598.7421341)(879.83562837,598.82214071)
\curveto(879.7156235,598.87213397)(879.57062364,598.90713394)(879.40062837,598.92714071)
\curveto(879.24062397,598.9471339)(879.07062414,598.95713389)(878.89062837,598.95714071)
\curveto(878.7106245,598.95713389)(878.53562468,598.9471339)(878.36562837,598.92714071)
\curveto(878.19562502,598.90713394)(878.05062516,598.87713397)(877.93062837,598.83714071)
\curveto(877.76062545,598.77713407)(877.59562562,598.69213415)(877.43562837,598.58214071)
\curveto(877.35562586,598.52213432)(877.28062593,598.4421344)(877.21062837,598.34214071)
\curveto(877.15062606,598.25213459)(877.09562612,598.15213469)(877.04562837,598.04214071)
\curveto(877.0156262,597.96213488)(876.98562623,597.87713497)(876.95562837,597.78714071)
\curveto(876.93562628,597.69713515)(876.89062632,597.62713522)(876.82062837,597.57714071)
\curveto(876.78062643,597.5471353)(876.7106265,597.52213532)(876.61062837,597.50214071)
\curveto(876.52062669,597.49213535)(876.42562679,597.48713536)(876.32562837,597.48714071)
\curveto(876.22562699,597.48713536)(876.12562709,597.49213535)(876.02562837,597.50214071)
\curveto(875.93562728,597.52213532)(875.87062734,597.5471353)(875.83062837,597.57714071)
\curveto(875.79062742,597.60713524)(875.76062745,597.65713519)(875.74062837,597.72714071)
\curveto(875.72062749,597.79713505)(875.72062749,597.87213497)(875.74062837,597.95214071)
\curveto(875.77062744,598.08213476)(875.80062741,598.20213464)(875.83062837,598.31214071)
\curveto(875.87062734,598.43213441)(875.9156273,598.5471343)(875.96562837,598.65714071)
\curveto(876.15562706,599.00713384)(876.39562682,599.27713357)(876.68562837,599.46714071)
\curveto(876.97562624,599.66713318)(877.33562588,599.82713302)(877.76562837,599.94714071)
\curveto(877.86562535,599.96713288)(877.96562525,599.98213286)(878.06562837,599.99214071)
\curveto(878.17562504,600.00213284)(878.28562493,600.01713283)(878.39562837,600.03714071)
\curveto(878.43562478,600.0471328)(878.50062471,600.0471328)(878.59062837,600.03714071)
\curveto(878.68062453,600.03713281)(878.73562448,600.0471328)(878.75562837,600.06714071)
\curveto(879.45562376,600.07713277)(880.06562315,599.99713285)(880.58562837,599.82714071)
\curveto(881.10562211,599.65713319)(881.47062174,599.33213351)(881.68062837,598.85214071)
\curveto(881.77062144,598.65213419)(881.82062139,598.41713443)(881.83062837,598.14714071)
\curveto(881.85062136,597.88713496)(881.86062135,597.61213523)(881.86062837,597.32214071)
\lineto(881.86062837,594.00714071)
\curveto(881.86062135,593.86713898)(881.86562135,593.73213911)(881.87562837,593.60214071)
\curveto(881.88562133,593.47213937)(881.9156213,593.36713948)(881.96562837,593.28714071)
\curveto(882.0156212,593.21713963)(882.08062113,593.16713968)(882.16062837,593.13714071)
\curveto(882.25062096,593.09713975)(882.33562088,593.06713978)(882.41562837,593.04714071)
\curveto(882.49562072,593.03713981)(882.55562066,592.99213985)(882.59562837,592.91214071)
\curveto(882.6156206,592.88213996)(882.62562059,592.85213999)(882.62562837,592.82214071)
\curveto(882.62562059,592.79214005)(882.63062058,592.75214009)(882.64062837,592.70214071)
\moveto(880.49562837,594.36714071)
\curveto(880.55562266,594.50713834)(880.58562263,594.66713818)(880.58562837,594.84714071)
\curveto(880.59562262,595.03713781)(880.60062261,595.23213761)(880.60062837,595.43214071)
\curveto(880.60062261,595.5421373)(880.59562262,595.6421372)(880.58562837,595.73214071)
\curveto(880.57562264,595.82213702)(880.53562268,595.89213695)(880.46562837,595.94214071)
\curveto(880.43562278,595.96213688)(880.36562285,595.97213687)(880.25562837,595.97214071)
\curveto(880.23562298,595.95213689)(880.20062301,595.9421369)(880.15062837,595.94214071)
\curveto(880.10062311,595.9421369)(880.05562316,595.93213691)(880.01562837,595.91214071)
\curveto(879.93562328,595.89213695)(879.84562337,595.87213697)(879.74562837,595.85214071)
\lineto(879.44562837,595.79214071)
\curveto(879.4156238,595.79213705)(879.38062383,595.78713706)(879.34062837,595.77714071)
\lineto(879.23562837,595.77714071)
\curveto(879.08562413,595.73713711)(878.92062429,595.71213713)(878.74062837,595.70214071)
\curveto(878.57062464,595.70213714)(878.4106248,595.68213716)(878.26062837,595.64214071)
\curveto(878.18062503,595.62213722)(878.10562511,595.60213724)(878.03562837,595.58214071)
\curveto(877.97562524,595.57213727)(877.90562531,595.55713729)(877.82562837,595.53714071)
\curveto(877.66562555,595.48713736)(877.5156257,595.42213742)(877.37562837,595.34214071)
\curveto(877.23562598,595.27213757)(877.1156261,595.18213766)(877.01562837,595.07214071)
\curveto(876.9156263,594.96213788)(876.84062637,594.82713802)(876.79062837,594.66714071)
\curveto(876.74062647,594.51713833)(876.72062649,594.33213851)(876.73062837,594.11214071)
\curveto(876.73062648,594.01213883)(876.74562647,593.91713893)(876.77562837,593.82714071)
\curveto(876.8156264,593.7471391)(876.86062635,593.67213917)(876.91062837,593.60214071)
\curveto(876.99062622,593.49213935)(877.09562612,593.39713945)(877.22562837,593.31714071)
\curveto(877.35562586,593.2471396)(877.49562572,593.18713966)(877.64562837,593.13714071)
\curveto(877.69562552,593.12713972)(877.74562547,593.12213972)(877.79562837,593.12214071)
\curveto(877.84562537,593.12213972)(877.89562532,593.11713973)(877.94562837,593.10714071)
\curveto(878.0156252,593.08713976)(878.10062511,593.07213977)(878.20062837,593.06214071)
\curveto(878.3106249,593.06213978)(878.40062481,593.07213977)(878.47062837,593.09214071)
\curveto(878.53062468,593.11213973)(878.59062462,593.11713973)(878.65062837,593.10714071)
\curveto(878.7106245,593.10713974)(878.77062444,593.11713973)(878.83062837,593.13714071)
\curveto(878.9106243,593.15713969)(878.98562423,593.17213967)(879.05562837,593.18214071)
\curveto(879.13562408,593.19213965)(879.210624,593.21213963)(879.28062837,593.24214071)
\curveto(879.57062364,593.36213948)(879.8156234,593.50713934)(880.01562837,593.67714071)
\curveto(880.22562299,593.847139)(880.38562283,594.07713877)(880.49562837,594.36714071)
}
}
{
\newrgbcolor{curcolor}{0 0 0}
\pscustom[linestyle=none,fillstyle=solid,fillcolor=curcolor]
{
\newpath
\moveto(890.772269,592.95714071)
\lineto(890.772269,592.56714071)
\curveto(890.77226112,592.4471404)(890.74726115,592.3471405)(890.697269,592.26714071)
\curveto(890.64726125,592.19714065)(890.56226133,592.15714069)(890.442269,592.14714071)
\lineto(890.097269,592.14714071)
\curveto(890.03726186,592.1471407)(889.97726192,592.1421407)(889.917269,592.13214071)
\curveto(889.86726203,592.13214071)(889.82226207,592.1421407)(889.782269,592.16214071)
\curveto(889.6922622,592.18214066)(889.63226226,592.22214062)(889.602269,592.28214071)
\curveto(889.56226233,592.33214051)(889.53726236,592.39214045)(889.527269,592.46214071)
\curveto(889.52726237,592.53214031)(889.51226238,592.60214024)(889.482269,592.67214071)
\curveto(889.47226242,592.69214015)(889.45726244,592.70714014)(889.437269,592.71714071)
\curveto(889.42726247,592.73714011)(889.41226248,592.75714009)(889.392269,592.77714071)
\curveto(889.2922626,592.78714006)(889.21226268,592.76714008)(889.152269,592.71714071)
\curveto(889.10226279,592.66714018)(889.04726285,592.61714023)(888.987269,592.56714071)
\curveto(888.78726311,592.41714043)(888.58726331,592.30214054)(888.387269,592.22214071)
\curveto(888.20726369,592.1421407)(887.9972639,592.08214076)(887.757269,592.04214071)
\curveto(887.52726437,592.00214084)(887.28726461,591.98214086)(887.037269,591.98214071)
\curveto(886.7972651,591.97214087)(886.55726534,591.98714086)(886.317269,592.02714071)
\curveto(886.07726582,592.05714079)(885.86726603,592.11214073)(885.687269,592.19214071)
\curveto(885.16726673,592.41214043)(884.74726715,592.70714014)(884.427269,593.07714071)
\curveto(884.10726779,593.45713939)(883.85726804,593.92713892)(883.677269,594.48714071)
\curveto(883.63726826,594.57713827)(883.60726829,594.66713818)(883.587269,594.75714071)
\curveto(883.57726832,594.85713799)(883.55726834,594.95713789)(883.527269,595.05714071)
\curveto(883.51726838,595.10713774)(883.51226838,595.15713769)(883.512269,595.20714071)
\curveto(883.51226838,595.25713759)(883.50726839,595.30713754)(883.497269,595.35714071)
\curveto(883.47726842,595.40713744)(883.46726843,595.45713739)(883.467269,595.50714071)
\curveto(883.47726842,595.56713728)(883.47726842,595.62213722)(883.467269,595.67214071)
\lineto(883.467269,595.82214071)
\curveto(883.44726845,595.87213697)(883.43726846,595.93713691)(883.437269,596.01714071)
\curveto(883.43726846,596.09713675)(883.44726845,596.16213668)(883.467269,596.21214071)
\lineto(883.467269,596.37714071)
\curveto(883.48726841,596.4471364)(883.4922684,596.51713633)(883.482269,596.58714071)
\curveto(883.48226841,596.66713618)(883.4922684,596.7421361)(883.512269,596.81214071)
\curveto(883.52226837,596.86213598)(883.52726837,596.90713594)(883.527269,596.94714071)
\curveto(883.52726837,596.98713586)(883.53226836,597.03213581)(883.542269,597.08214071)
\curveto(883.57226832,597.18213566)(883.5972683,597.27713557)(883.617269,597.36714071)
\curveto(883.63726826,597.46713538)(883.66226823,597.56213528)(883.692269,597.65214071)
\curveto(883.82226807,598.03213481)(883.98726791,598.37213447)(884.187269,598.67214071)
\curveto(884.3972675,598.98213386)(884.64726725,599.23713361)(884.937269,599.43714071)
\curveto(885.10726679,599.55713329)(885.28226661,599.65713319)(885.462269,599.73714071)
\curveto(885.65226624,599.81713303)(885.85726604,599.88713296)(886.077269,599.94714071)
\curveto(886.14726575,599.95713289)(886.21226568,599.96713288)(886.272269,599.97714071)
\curveto(886.34226555,599.98713286)(886.41226548,600.00213284)(886.482269,600.02214071)
\lineto(886.632269,600.02214071)
\curveto(886.71226518,600.0421328)(886.82726507,600.05213279)(886.977269,600.05214071)
\curveto(887.13726476,600.05213279)(887.25726464,600.0421328)(887.337269,600.02214071)
\curveto(887.37726452,600.01213283)(887.43226446,600.00713284)(887.502269,600.00714071)
\curveto(887.61226428,599.97713287)(887.72226417,599.95213289)(887.832269,599.93214071)
\curveto(887.94226395,599.92213292)(888.04726385,599.89213295)(888.147269,599.84214071)
\curveto(888.2972636,599.78213306)(888.43726346,599.71713313)(888.567269,599.64714071)
\curveto(888.70726319,599.57713327)(888.83726306,599.49713335)(888.957269,599.40714071)
\curveto(889.01726288,599.35713349)(889.07726282,599.30213354)(889.137269,599.24214071)
\curveto(889.20726269,599.19213365)(889.2972626,599.17713367)(889.407269,599.19714071)
\curveto(889.42726247,599.22713362)(889.44226245,599.25213359)(889.452269,599.27214071)
\curveto(889.47226242,599.29213355)(889.48726241,599.32213352)(889.497269,599.36214071)
\curveto(889.52726237,599.45213339)(889.53726236,599.56713328)(889.527269,599.70714071)
\lineto(889.527269,600.08214071)
\lineto(889.527269,601.80714071)
\lineto(889.527269,602.27214071)
\curveto(889.52726237,602.45213039)(889.55226234,602.58213026)(889.602269,602.66214071)
\curveto(889.64226225,602.73213011)(889.70226219,602.77713007)(889.782269,602.79714071)
\curveto(889.80226209,602.79713005)(889.82726207,602.79713005)(889.857269,602.79714071)
\curveto(889.88726201,602.80713004)(889.91226198,602.81213003)(889.932269,602.81214071)
\curveto(890.07226182,602.82213002)(890.21726168,602.82213002)(890.367269,602.81214071)
\curveto(890.52726137,602.81213003)(890.63726126,602.77213007)(890.697269,602.69214071)
\curveto(890.74726115,602.61213023)(890.77226112,602.51213033)(890.772269,602.39214071)
\lineto(890.772269,602.01714071)
\lineto(890.772269,592.95714071)
\moveto(889.557269,595.79214071)
\curveto(889.57726232,595.842137)(889.58726231,595.90713694)(889.587269,595.98714071)
\curveto(889.58726231,596.07713677)(889.57726232,596.1471367)(889.557269,596.19714071)
\lineto(889.557269,596.42214071)
\curveto(889.53726236,596.51213633)(889.52226237,596.60213624)(889.512269,596.69214071)
\curveto(889.50226239,596.79213605)(889.48226241,596.88213596)(889.452269,596.96214071)
\curveto(889.43226246,597.0421358)(889.41226248,597.11713573)(889.392269,597.18714071)
\curveto(889.38226251,597.25713559)(889.36226253,597.32713552)(889.332269,597.39714071)
\curveto(889.21226268,597.69713515)(889.05726284,597.96213488)(888.867269,598.19214071)
\curveto(888.67726322,598.42213442)(888.43726346,598.60213424)(888.147269,598.73214071)
\curveto(888.04726385,598.78213406)(887.94226395,598.81713403)(887.832269,598.83714071)
\curveto(887.73226416,598.86713398)(887.62226427,598.89213395)(887.502269,598.91214071)
\curveto(887.42226447,598.93213391)(887.33226456,598.9421339)(887.232269,598.94214071)
\lineto(886.962269,598.94214071)
\curveto(886.91226498,598.93213391)(886.86726503,598.92213392)(886.827269,598.91214071)
\lineto(886.692269,598.91214071)
\curveto(886.61226528,598.89213395)(886.52726537,598.87213397)(886.437269,598.85214071)
\curveto(886.35726554,598.83213401)(886.27726562,598.80713404)(886.197269,598.77714071)
\curveto(885.87726602,598.63713421)(885.61726628,598.43213441)(885.417269,598.16214071)
\curveto(885.22726667,597.90213494)(885.07226682,597.59713525)(884.952269,597.24714071)
\curveto(884.91226698,597.13713571)(884.88226701,597.02213582)(884.862269,596.90214071)
\curveto(884.85226704,596.79213605)(884.83726706,596.68213616)(884.817269,596.57214071)
\curveto(884.81726708,596.53213631)(884.81226708,596.49213635)(884.802269,596.45214071)
\lineto(884.802269,596.34714071)
\curveto(884.78226711,596.29713655)(884.77226712,596.2421366)(884.772269,596.18214071)
\curveto(884.78226711,596.12213672)(884.78726711,596.06713678)(884.787269,596.01714071)
\lineto(884.787269,595.68714071)
\curveto(884.78726711,595.58713726)(884.7972671,595.49213735)(884.817269,595.40214071)
\curveto(884.82726707,595.37213747)(884.83226706,595.32213752)(884.832269,595.25214071)
\curveto(884.85226704,595.18213766)(884.86726703,595.11213773)(884.877269,595.04214071)
\lineto(884.937269,594.83214071)
\curveto(885.04726685,594.48213836)(885.1972667,594.18213866)(885.387269,593.93214071)
\curveto(885.57726632,593.68213916)(885.81726608,593.47713937)(886.107269,593.31714071)
\curveto(886.1972657,593.26713958)(886.28726561,593.22713962)(886.377269,593.19714071)
\curveto(886.46726543,593.16713968)(886.56726533,593.13713971)(886.677269,593.10714071)
\curveto(886.72726517,593.08713976)(886.77726512,593.08213976)(886.827269,593.09214071)
\curveto(886.88726501,593.10213974)(886.94226495,593.09713975)(886.992269,593.07714071)
\curveto(887.03226486,593.06713978)(887.07226482,593.06213978)(887.112269,593.06214071)
\lineto(887.247269,593.06214071)
\lineto(887.382269,593.06214071)
\curveto(887.41226448,593.07213977)(887.46226443,593.07713977)(887.532269,593.07714071)
\curveto(887.61226428,593.09713975)(887.6922642,593.11213973)(887.772269,593.12214071)
\curveto(887.85226404,593.1421397)(887.92726397,593.16713968)(887.997269,593.19714071)
\curveto(888.32726357,593.33713951)(888.5922633,593.51213933)(888.792269,593.72214071)
\curveto(889.00226289,593.9421389)(889.17726272,594.21713863)(889.317269,594.54714071)
\curveto(889.36726253,594.65713819)(889.40226249,594.76713808)(889.422269,594.87714071)
\curveto(889.44226245,594.98713786)(889.46726243,595.09713775)(889.497269,595.20714071)
\curveto(889.51726238,595.2471376)(889.52726237,595.28213756)(889.527269,595.31214071)
\curveto(889.52726237,595.35213749)(889.53226236,595.39213745)(889.542269,595.43214071)
\curveto(889.55226234,595.49213735)(889.55226234,595.55213729)(889.542269,595.61214071)
\curveto(889.54226235,595.67213717)(889.54726235,595.73213711)(889.557269,595.79214071)
}
}
{
\newrgbcolor{curcolor}{0 0 0}
\pscustom[linestyle=none,fillstyle=solid,fillcolor=curcolor]
{
\newpath
\moveto(899.843519,596.34714071)
\curveto(899.86351094,596.28713656)(899.87351093,596.19213665)(899.873519,596.06214071)
\curveto(899.87351093,595.9421369)(899.86851093,595.85713699)(899.858519,595.80714071)
\lineto(899.858519,595.65714071)
\curveto(899.84851095,595.57713727)(899.83851096,595.50213734)(899.828519,595.43214071)
\curveto(899.82851097,595.37213747)(899.82351098,595.30213754)(899.813519,595.22214071)
\curveto(899.79351101,595.16213768)(899.77851102,595.10213774)(899.768519,595.04214071)
\curveto(899.76851103,594.98213786)(899.75851104,594.92213792)(899.738519,594.86214071)
\curveto(899.6985111,594.73213811)(899.66351114,594.60213824)(899.633519,594.47214071)
\curveto(899.6035112,594.3421385)(899.56351124,594.22213862)(899.513519,594.11214071)
\curveto(899.3035115,593.63213921)(899.02351178,593.22713962)(898.673519,592.89714071)
\curveto(898.32351248,592.57714027)(897.89351291,592.33214051)(897.383519,592.16214071)
\curveto(897.27351353,592.12214072)(897.15351365,592.09214075)(897.023519,592.07214071)
\curveto(896.9035139,592.05214079)(896.77851402,592.03214081)(896.648519,592.01214071)
\curveto(896.58851421,592.00214084)(896.52351428,591.99714085)(896.453519,591.99714071)
\curveto(896.39351441,591.98714086)(896.33351447,591.98214086)(896.273519,591.98214071)
\curveto(896.23351457,591.97214087)(896.17351463,591.96714088)(896.093519,591.96714071)
\curveto(896.02351478,591.96714088)(895.97351483,591.97214087)(895.943519,591.98214071)
\curveto(895.9035149,591.99214085)(895.86351494,591.99714085)(895.823519,591.99714071)
\curveto(895.78351502,591.98714086)(895.74851505,591.98714086)(895.718519,591.99714071)
\lineto(895.628519,591.99714071)
\lineto(895.268519,592.04214071)
\curveto(895.12851567,592.08214076)(894.99351581,592.12214072)(894.863519,592.16214071)
\curveto(894.73351607,592.20214064)(894.60851619,592.2471406)(894.488519,592.29714071)
\curveto(894.03851676,592.49714035)(893.66851713,592.75714009)(893.378519,593.07714071)
\curveto(893.08851771,593.39713945)(892.84851795,593.78713906)(892.658519,594.24714071)
\curveto(892.60851819,594.3471385)(892.56851823,594.4471384)(892.538519,594.54714071)
\curveto(892.51851828,594.6471382)(892.4985183,594.75213809)(892.478519,594.86214071)
\curveto(892.45851834,594.90213794)(892.44851835,594.93213791)(892.448519,594.95214071)
\curveto(892.45851834,594.98213786)(892.45851834,595.01713783)(892.448519,595.05714071)
\curveto(892.42851837,595.13713771)(892.41351839,595.21713763)(892.403519,595.29714071)
\curveto(892.4035184,595.38713746)(892.39351841,595.47213737)(892.373519,595.55214071)
\lineto(892.373519,595.67214071)
\curveto(892.37351843,595.71213713)(892.36851843,595.75713709)(892.358519,595.80714071)
\curveto(892.34851845,595.85713699)(892.34351846,595.9421369)(892.343519,596.06214071)
\curveto(892.34351846,596.19213665)(892.35351845,596.28713656)(892.373519,596.34714071)
\curveto(892.39351841,596.41713643)(892.3985184,596.48713636)(892.388519,596.55714071)
\curveto(892.37851842,596.62713622)(892.38351842,596.69713615)(892.403519,596.76714071)
\curveto(892.41351839,596.81713603)(892.41851838,596.85713599)(892.418519,596.88714071)
\curveto(892.42851837,596.92713592)(892.43851836,596.97213587)(892.448519,597.02214071)
\curveto(892.47851832,597.1421357)(892.5035183,597.26213558)(892.523519,597.38214071)
\curveto(892.55351825,597.50213534)(892.59351821,597.61713523)(892.643519,597.72714071)
\curveto(892.79351801,598.09713475)(892.97351783,598.42713442)(893.183519,598.71714071)
\curveto(893.4035174,599.01713383)(893.66851713,599.26713358)(893.978519,599.46714071)
\curveto(894.0985167,599.5471333)(894.22351658,599.61213323)(894.353519,599.66214071)
\curveto(894.48351632,599.72213312)(894.61851618,599.78213306)(894.758519,599.84214071)
\curveto(894.87851592,599.89213295)(895.00851579,599.92213292)(895.148519,599.93214071)
\curveto(895.28851551,599.95213289)(895.42851537,599.98213286)(895.568519,600.02214071)
\lineto(895.763519,600.02214071)
\curveto(895.83351497,600.03213281)(895.8985149,600.0421328)(895.958519,600.05214071)
\curveto(896.84851395,600.06213278)(897.58851321,599.87713297)(898.178519,599.49714071)
\curveto(898.76851203,599.11713373)(899.19351161,598.62213422)(899.453519,598.01214071)
\curveto(899.5035113,597.91213493)(899.54351126,597.81213503)(899.573519,597.71214071)
\curveto(899.6035112,597.61213523)(899.63851116,597.50713534)(899.678519,597.39714071)
\curveto(899.70851109,597.28713556)(899.73351107,597.16713568)(899.753519,597.03714071)
\curveto(899.77351103,596.91713593)(899.798511,596.79213605)(899.828519,596.66214071)
\curveto(899.83851096,596.61213623)(899.83851096,596.55713629)(899.828519,596.49714071)
\curveto(899.82851097,596.4471364)(899.83351097,596.39713645)(899.843519,596.34714071)
\moveto(898.508519,595.49214071)
\curveto(898.52851227,595.56213728)(898.53351227,595.6421372)(898.523519,595.73214071)
\lineto(898.523519,595.98714071)
\curveto(898.52351228,596.37713647)(898.48851231,596.70713614)(898.418519,596.97714071)
\curveto(898.38851241,597.05713579)(898.36351244,597.13713571)(898.343519,597.21714071)
\curveto(898.32351248,597.29713555)(898.2985125,597.37213547)(898.268519,597.44214071)
\curveto(897.98851281,598.09213475)(897.54351326,598.5421343)(896.933519,598.79214071)
\curveto(896.86351394,598.82213402)(896.78851401,598.842134)(896.708519,598.85214071)
\lineto(896.468519,598.91214071)
\curveto(896.38851441,598.93213391)(896.3035145,598.9421339)(896.213519,598.94214071)
\lineto(895.943519,598.94214071)
\lineto(895.673519,598.89714071)
\curveto(895.57351523,598.87713397)(895.47851532,598.85213399)(895.388519,598.82214071)
\curveto(895.30851549,598.80213404)(895.22851557,598.77213407)(895.148519,598.73214071)
\curveto(895.07851572,598.71213413)(895.01351579,598.68213416)(894.953519,598.64214071)
\curveto(894.89351591,598.60213424)(894.83851596,598.56213428)(894.788519,598.52214071)
\curveto(894.54851625,598.35213449)(894.35351645,598.1471347)(894.203519,597.90714071)
\curveto(894.05351675,597.66713518)(893.92351688,597.38713546)(893.813519,597.06714071)
\curveto(893.78351702,596.96713588)(893.76351704,596.86213598)(893.753519,596.75214071)
\curveto(893.74351706,596.65213619)(893.72851707,596.5471363)(893.708519,596.43714071)
\curveto(893.6985171,596.39713645)(893.69351711,596.33213651)(893.693519,596.24214071)
\curveto(893.68351712,596.21213663)(893.67851712,596.17713667)(893.678519,596.13714071)
\curveto(893.68851711,596.09713675)(893.69351711,596.05213679)(893.693519,596.00214071)
\lineto(893.693519,595.70214071)
\curveto(893.69351711,595.60213724)(893.7035171,595.51213733)(893.723519,595.43214071)
\lineto(893.753519,595.25214071)
\curveto(893.77351703,595.15213769)(893.78851701,595.05213779)(893.798519,594.95214071)
\curveto(893.81851698,594.86213798)(893.84851695,594.77713807)(893.888519,594.69714071)
\curveto(893.98851681,594.45713839)(894.1035167,594.23213861)(894.233519,594.02214071)
\curveto(894.37351643,593.81213903)(894.54351626,593.63713921)(894.743519,593.49714071)
\curveto(894.79351601,593.46713938)(894.83851596,593.4421394)(894.878519,593.42214071)
\curveto(894.91851588,593.40213944)(894.96351584,593.37713947)(895.013519,593.34714071)
\curveto(895.09351571,593.29713955)(895.17851562,593.25213959)(895.268519,593.21214071)
\curveto(895.36851543,593.18213966)(895.47351533,593.15213969)(895.583519,593.12214071)
\curveto(895.63351517,593.10213974)(895.67851512,593.09213975)(895.718519,593.09214071)
\curveto(895.76851503,593.10213974)(895.81851498,593.10213974)(895.868519,593.09214071)
\curveto(895.8985149,593.08213976)(895.95851484,593.07213977)(896.048519,593.06214071)
\curveto(896.14851465,593.05213979)(896.22351458,593.05713979)(896.273519,593.07714071)
\curveto(896.31351449,593.08713976)(896.35351445,593.08713976)(896.393519,593.07714071)
\curveto(896.43351437,593.07713977)(896.47351433,593.08713976)(896.513519,593.10714071)
\curveto(896.59351421,593.12713972)(896.67351413,593.1421397)(896.753519,593.15214071)
\curveto(896.83351397,593.17213967)(896.90851389,593.19713965)(896.978519,593.22714071)
\curveto(897.31851348,593.36713948)(897.59351321,593.56213928)(897.803519,593.81214071)
\curveto(898.01351279,594.06213878)(898.18851261,594.35713849)(898.328519,594.69714071)
\curveto(898.37851242,594.81713803)(898.40851239,594.9421379)(898.418519,595.07214071)
\curveto(898.43851236,595.21213763)(898.46851233,595.35213749)(898.508519,595.49214071)
}
}
{
\newrgbcolor{curcolor}{0 0 0}
\pscustom[linestyle=none,fillstyle=solid,fillcolor=curcolor]
{
\newpath
\moveto(904.97680025,600.05214071)
\curveto(905.20679546,600.05213279)(905.33679533,599.99213285)(905.36680025,599.87214071)
\curveto(905.39679527,599.76213308)(905.41179525,599.59713325)(905.41180025,599.37714071)
\lineto(905.41180025,599.09214071)
\curveto(905.41179525,599.00213384)(905.38679528,598.92713392)(905.33680025,598.86714071)
\curveto(905.27679539,598.78713406)(905.19179547,598.7421341)(905.08180025,598.73214071)
\curveto(904.97179569,598.73213411)(904.8617958,598.71713413)(904.75180025,598.68714071)
\curveto(904.61179605,598.65713419)(904.47679619,598.62713422)(904.34680025,598.59714071)
\curveto(904.22679644,598.56713428)(904.11179655,598.52713432)(904.00180025,598.47714071)
\curveto(903.71179695,598.3471345)(903.47679719,598.16713468)(903.29680025,597.93714071)
\curveto(903.11679755,597.71713513)(902.9617977,597.46213538)(902.83180025,597.17214071)
\curveto(902.79179787,597.06213578)(902.7617979,596.9471359)(902.74180025,596.82714071)
\curveto(902.72179794,596.71713613)(902.69679797,596.60213624)(902.66680025,596.48214071)
\curveto(902.65679801,596.43213641)(902.65179801,596.38213646)(902.65180025,596.33214071)
\curveto(902.661798,596.28213656)(902.661798,596.23213661)(902.65180025,596.18214071)
\curveto(902.62179804,596.06213678)(902.60679806,595.92213692)(902.60680025,595.76214071)
\curveto(902.61679805,595.61213723)(902.62179804,595.46713738)(902.62180025,595.32714071)
\lineto(902.62180025,593.48214071)
\lineto(902.62180025,593.13714071)
\curveto(902.62179804,593.01713983)(902.61679805,592.90213994)(902.60680025,592.79214071)
\curveto(902.59679807,592.68214016)(902.59179807,592.58714026)(902.59180025,592.50714071)
\curveto(902.60179806,592.42714042)(902.58179808,592.35714049)(902.53180025,592.29714071)
\curveto(902.48179818,592.22714062)(902.40179826,592.18714066)(902.29180025,592.17714071)
\curveto(902.19179847,592.16714068)(902.08179858,592.16214068)(901.96180025,592.16214071)
\lineto(901.69180025,592.16214071)
\curveto(901.64179902,592.18214066)(901.59179907,592.19714065)(901.54180025,592.20714071)
\curveto(901.50179916,592.22714062)(901.47179919,592.25214059)(901.45180025,592.28214071)
\curveto(901.40179926,592.35214049)(901.37179929,592.43714041)(901.36180025,592.53714071)
\lineto(901.36180025,592.86714071)
\lineto(901.36180025,594.02214071)
\lineto(901.36180025,598.17714071)
\lineto(901.36180025,599.21214071)
\lineto(901.36180025,599.51214071)
\curveto(901.37179929,599.61213323)(901.40179926,599.69713315)(901.45180025,599.76714071)
\curveto(901.48179918,599.80713304)(901.53179913,599.83713301)(901.60180025,599.85714071)
\curveto(901.68179898,599.87713297)(901.7667989,599.88713296)(901.85680025,599.88714071)
\curveto(901.94679872,599.89713295)(902.03679863,599.89713295)(902.12680025,599.88714071)
\curveto(902.21679845,599.87713297)(902.28679838,599.86213298)(902.33680025,599.84214071)
\curveto(902.41679825,599.81213303)(902.4667982,599.75213309)(902.48680025,599.66214071)
\curveto(902.51679815,599.58213326)(902.53179813,599.49213335)(902.53180025,599.39214071)
\lineto(902.53180025,599.09214071)
\curveto(902.53179813,598.99213385)(902.55179811,598.90213394)(902.59180025,598.82214071)
\curveto(902.60179806,598.80213404)(902.61179805,598.78713406)(902.62180025,598.77714071)
\lineto(902.66680025,598.73214071)
\curveto(902.77679789,598.73213411)(902.8667978,598.77713407)(902.93680025,598.86714071)
\curveto(903.00679766,598.96713388)(903.0667976,599.0471338)(903.11680025,599.10714071)
\lineto(903.20680025,599.19714071)
\curveto(903.29679737,599.30713354)(903.42179724,599.42213342)(903.58180025,599.54214071)
\curveto(903.74179692,599.66213318)(903.89179677,599.75213309)(904.03180025,599.81214071)
\curveto(904.12179654,599.86213298)(904.21679645,599.89713295)(904.31680025,599.91714071)
\curveto(904.41679625,599.9471329)(904.52179614,599.97713287)(904.63180025,600.00714071)
\curveto(904.69179597,600.01713283)(904.75179591,600.02213282)(904.81180025,600.02214071)
\curveto(904.87179579,600.03213281)(904.92679574,600.0421328)(904.97680025,600.05214071)
}
}
{
\newrgbcolor{curcolor}{0.50196081 0.50196081 0.50196081}
\pscustom[linestyle=none,fillstyle=solid,fillcolor=curcolor]
{
\newpath
\moveto(812.80437349,602.85717733)
\lineto(827.80437349,602.85717733)
\lineto(827.80437349,587.85717733)
\lineto(812.80437349,587.85717733)
\closepath
}
}
{
\newrgbcolor{curcolor}{0 0 0}
\pscustom[linestyle=none,fillstyle=solid,fillcolor=curcolor]
{
\newpath
\moveto(841.1775815,574.88496298)
\lineto(841.1775815,574.61496298)
\curveto(841.18757153,574.52495773)(841.18257153,574.44495781)(841.1625815,574.37496298)
\lineto(841.1625815,574.22496298)
\curveto(841.15257156,574.19495806)(841.14757157,574.15995809)(841.1475815,574.11996298)
\curveto(841.15757156,574.07995817)(841.15757156,574.0499582)(841.1475815,574.02996298)
\curveto(841.13757158,573.97995827)(841.13257158,573.92495833)(841.1325815,573.86496298)
\curveto(841.13257158,573.81495844)(841.12757159,573.76495849)(841.1175815,573.71496298)
\curveto(841.08757163,573.57495868)(841.06757165,573.42495883)(841.0575815,573.26496298)
\curveto(841.04757167,573.11495914)(841.0175717,572.96995928)(840.9675815,572.82996298)
\curveto(840.93757178,572.70995954)(840.90257181,572.58495967)(840.8625815,572.45496298)
\curveto(840.83257188,572.33495992)(840.79257192,572.21496004)(840.7425815,572.09496298)
\curveto(840.57257214,571.66496059)(840.35757236,571.27496098)(840.0975815,570.92496298)
\curveto(839.84757287,570.58496167)(839.53257318,570.29496196)(839.1525815,570.05496298)
\curveto(838.96257375,569.93496232)(838.75757396,569.82996242)(838.5375815,569.73996298)
\curveto(838.32757439,569.65996259)(838.09757462,569.57996267)(837.8475815,569.49996298)
\curveto(837.73757498,569.45996279)(837.6175751,569.42996282)(837.4875815,569.40996298)
\curveto(837.36757535,569.39996285)(837.24757547,569.37996287)(837.1275815,569.34996298)
\curveto(837.0175757,569.32996292)(836.90757581,569.31996293)(836.7975815,569.31996298)
\curveto(836.69757602,569.31996293)(836.59757612,569.30996294)(836.4975815,569.28996298)
\lineto(836.2875815,569.28996298)
\curveto(836.25757646,569.27996297)(836.22257649,569.27496298)(836.1825815,569.27496298)
\curveto(836.14257657,569.28496297)(836.10257661,569.28996296)(836.0625815,569.28996298)
\lineto(833.0625815,569.28996298)
\curveto(832.9125798,569.28996296)(832.77757994,569.29496296)(832.6575815,569.30496298)
\curveto(832.54758017,569.32496293)(832.47258024,569.38996286)(832.4325815,569.49996298)
\curveto(832.39258032,569.57996267)(832.37258034,569.69496256)(832.3725815,569.84496298)
\curveto(832.38258033,569.99496226)(832.38758033,570.12996212)(832.3875815,570.24996298)
\lineto(832.3875815,579.11496298)
\curveto(832.38758033,579.23495302)(832.38258033,579.35995289)(832.3725815,579.48996298)
\curveto(832.37258034,579.62995262)(832.39758032,579.73995251)(832.4475815,579.81996298)
\curveto(832.48758023,579.88995236)(832.56258015,579.93495232)(832.6725815,579.95496298)
\curveto(832.69258002,579.96495229)(832.71258,579.96495229)(832.7325815,579.95496298)
\curveto(832.75257996,579.9549523)(832.77257994,579.95995229)(832.7925815,579.96996298)
\lineto(836.0475815,579.96996298)
\curveto(836.09757662,579.96995228)(836.14257657,579.96995228)(836.1825815,579.96996298)
\curveto(836.23257648,579.97995227)(836.27757644,579.97995227)(836.3175815,579.96996298)
\curveto(836.36757635,579.9499523)(836.4175763,579.94495231)(836.4675815,579.95496298)
\curveto(836.52757619,579.96495229)(836.58257613,579.96495229)(836.6325815,579.95496298)
\curveto(836.68257603,579.94495231)(836.73757598,579.93995231)(836.7975815,579.93996298)
\curveto(836.85757586,579.93995231)(836.9125758,579.93495232)(836.9625815,579.92496298)
\curveto(837.0125757,579.91495234)(837.05757566,579.90995234)(837.0975815,579.90996298)
\curveto(837.14757557,579.90995234)(837.19757552,579.90495235)(837.2475815,579.89496298)
\curveto(837.35757536,579.87495238)(837.46257525,579.8549524)(837.5625815,579.83496298)
\curveto(837.66257505,579.82495243)(837.76257495,579.80495245)(837.8625815,579.77496298)
\curveto(838.08257463,579.70495255)(838.29257442,579.63495262)(838.4925815,579.56496298)
\curveto(838.69257402,579.50495275)(838.87757384,579.41995283)(839.0475815,579.30996298)
\curveto(839.18757353,579.22995302)(839.3125734,579.1499531)(839.4225815,579.06996298)
\curveto(839.45257326,579.0499532)(839.48257323,579.02495323)(839.5125815,578.99496298)
\curveto(839.54257317,578.97495328)(839.57257314,578.9549533)(839.6025815,578.93496298)
\curveto(839.66257305,578.88495337)(839.717573,578.83495342)(839.7675815,578.78496298)
\curveto(839.8175729,578.73495352)(839.86757285,578.68495357)(839.9175815,578.63496298)
\curveto(839.96757275,578.58495367)(840.00757271,578.5499537)(840.0375815,578.52996298)
\curveto(840.07757264,578.46995378)(840.1175726,578.41495384)(840.1575815,578.36496298)
\curveto(840.20757251,578.31495394)(840.25257246,578.25995399)(840.2925815,578.19996298)
\curveto(840.34257237,578.13995411)(840.38257233,578.07495418)(840.4125815,578.00496298)
\curveto(840.45257226,577.94495431)(840.49757222,577.87995437)(840.5475815,577.80996298)
\curveto(840.56757215,577.76995448)(840.58257213,577.73495452)(840.5925815,577.70496298)
\curveto(840.60257211,577.67495458)(840.6175721,577.63995461)(840.6375815,577.59996298)
\curveto(840.67757204,577.51995473)(840.712572,577.43995481)(840.7425815,577.35996298)
\curveto(840.77257194,577.28995496)(840.80757191,577.21495504)(840.8475815,577.13496298)
\curveto(840.88757183,577.02495523)(840.9175718,576.90995534)(840.9375815,576.78996298)
\curveto(840.96757175,576.67995557)(840.99757172,576.56995568)(841.0275815,576.45996298)
\curveto(841.04757167,576.39995585)(841.05757166,576.33995591)(841.0575815,576.27996298)
\curveto(841.05757166,576.22995602)(841.06757165,576.17495608)(841.0875815,576.11496298)
\curveto(841.13757158,575.93495632)(841.16257155,575.73495652)(841.1625815,575.51496298)
\curveto(841.17257154,575.30495695)(841.17757154,575.09495716)(841.1775815,574.88496298)
\moveto(839.7525815,574.10496298)
\curveto(839.77257294,574.20495805)(839.78257293,574.30995794)(839.7825815,574.41996298)
\lineto(839.7825815,574.76496298)
\lineto(839.7825815,574.98996298)
\curveto(839.79257292,575.06995718)(839.78757293,575.14495711)(839.7675815,575.21496298)
\curveto(839.76757295,575.24495701)(839.76257295,575.27495698)(839.7525815,575.30496298)
\lineto(839.7525815,575.40996298)
\curveto(839.73257298,575.51995673)(839.717573,575.62995662)(839.7075815,575.73996298)
\curveto(839.70757301,575.8499564)(839.69257302,575.95995629)(839.6625815,576.06996298)
\curveto(839.64257307,576.1499561)(839.62257309,576.22495603)(839.6025815,576.29496298)
\curveto(839.59257312,576.37495588)(839.57757314,576.4549558)(839.5575815,576.53496298)
\curveto(839.44757327,576.89495536)(839.30757341,577.20995504)(839.1375815,577.47996298)
\curveto(838.85757386,577.92995432)(838.44257427,578.26995398)(837.8925815,578.49996298)
\curveto(837.80257491,578.5499537)(837.70757501,578.58495367)(837.6075815,578.60496298)
\curveto(837.50757521,578.63495362)(837.40257531,578.66495359)(837.2925815,578.69496298)
\curveto(837.18257553,578.72495353)(837.06757565,578.73995351)(836.9475815,578.73996298)
\curveto(836.83757588,578.7499535)(836.72757599,578.76495349)(836.6175815,578.78496298)
\lineto(836.3025815,578.78496298)
\curveto(836.27257644,578.79495346)(836.23757648,578.79995345)(836.1975815,578.79996298)
\lineto(836.0775815,578.79996298)
\lineto(834.2475815,578.79996298)
\curveto(834.22757849,578.78995346)(834.20257851,578.78495347)(834.1725815,578.78496298)
\curveto(834.14257857,578.79495346)(834.1175786,578.79495346)(834.0975815,578.78496298)
\lineto(833.9475815,578.72496298)
\curveto(833.90757881,578.70495355)(833.87757884,578.67495358)(833.8575815,578.63496298)
\curveto(833.83757888,578.59495366)(833.8175789,578.52495373)(833.7975815,578.42496298)
\lineto(833.7975815,578.30496298)
\curveto(833.78757893,578.26495399)(833.78257893,578.21995403)(833.7825815,578.16996298)
\lineto(833.7825815,578.03496298)
\lineto(833.7825815,571.22496298)
\lineto(833.7825815,571.07496298)
\curveto(833.78257893,571.03496122)(833.78757893,570.99496126)(833.7975815,570.95496298)
\lineto(833.7975815,570.83496298)
\curveto(833.8175789,570.73496152)(833.83757888,570.66496159)(833.8575815,570.62496298)
\curveto(833.93757878,570.50496175)(834.08757863,570.44496181)(834.3075815,570.44496298)
\curveto(834.52757819,570.4549618)(834.73757798,570.45996179)(834.9375815,570.45996298)
\lineto(835.8075815,570.45996298)
\curveto(835.87757684,570.45996179)(835.95257676,570.4549618)(836.0325815,570.44496298)
\curveto(836.1125766,570.44496181)(836.18257653,570.4549618)(836.2425815,570.47496298)
\lineto(836.4075815,570.47496298)
\curveto(836.45757626,570.48496177)(836.5125762,570.48496177)(836.5725815,570.47496298)
\curveto(836.63257608,570.47496178)(836.69257602,570.47996177)(836.7525815,570.48996298)
\curveto(836.8125759,570.50996174)(836.87257584,570.51996173)(836.9325815,570.51996298)
\curveto(836.99257572,570.52996172)(837.05757566,570.54496171)(837.1275815,570.56496298)
\curveto(837.23757548,570.59496166)(837.34257537,570.62496163)(837.4425815,570.65496298)
\curveto(837.55257516,570.68496157)(837.66257505,570.72496153)(837.7725815,570.77496298)
\curveto(838.14257457,570.93496132)(838.45757426,571.1499611)(838.7175815,571.41996298)
\curveto(838.98757373,571.69996055)(839.20757351,572.02996022)(839.3775815,572.40996298)
\curveto(839.42757329,572.51995973)(839.46757325,572.63495962)(839.4975815,572.75496298)
\lineto(839.6175815,573.14496298)
\curveto(839.64757307,573.254959)(839.66757305,573.36995888)(839.6775815,573.48996298)
\curveto(839.69757302,573.61995863)(839.717573,573.74495851)(839.7375815,573.86496298)
\curveto(839.74757297,573.91495834)(839.75257296,573.9549583)(839.7525815,573.98496298)
\lineto(839.7525815,574.10496298)
}
}
{
\newrgbcolor{curcolor}{0 0 0}
\pscustom[linestyle=none,fillstyle=solid,fillcolor=curcolor]
{
\newpath
\moveto(849.4294565,573.45996298)
\curveto(849.44944881,573.35995889)(849.44944881,573.24495901)(849.4294565,573.11496298)
\curveto(849.41944884,572.99495926)(849.38944887,572.90995934)(849.3394565,572.85996298)
\curveto(849.28944897,572.81995943)(849.21444905,572.78995946)(849.1144565,572.76996298)
\curveto(849.02444924,572.75995949)(848.91944934,572.7549595)(848.7994565,572.75496298)
\lineto(848.4394565,572.75496298)
\curveto(848.31944994,572.76495949)(848.21445005,572.76995948)(848.1244565,572.76996298)
\lineto(844.2844565,572.76996298)
\curveto(844.20445406,572.76995948)(844.12445414,572.76495949)(844.0444565,572.75496298)
\curveto(843.9644543,572.7549595)(843.89945436,572.73995951)(843.8494565,572.70996298)
\curveto(843.80945445,572.68995956)(843.76945449,572.6499596)(843.7294565,572.58996298)
\curveto(843.70945455,572.55995969)(843.68945457,572.51495974)(843.6694565,572.45496298)
\curveto(843.64945461,572.40495985)(843.64945461,572.3549599)(843.6694565,572.30496298)
\curveto(843.67945458,572.25496)(843.68445458,572.20996004)(843.6844565,572.16996298)
\curveto(843.68445458,572.12996012)(843.68945457,572.08996016)(843.6994565,572.04996298)
\curveto(843.71945454,571.96996028)(843.73945452,571.88496037)(843.7594565,571.79496298)
\curveto(843.77945448,571.71496054)(843.80945445,571.63496062)(843.8494565,571.55496298)
\curveto(844.07945418,571.01496124)(844.4594538,570.62996162)(844.9894565,570.39996298)
\curveto(845.04945321,570.36996188)(845.11445315,570.34496191)(845.1844565,570.32496298)
\lineto(845.3944565,570.26496298)
\curveto(845.42445284,570.254962)(845.47445279,570.249962)(845.5444565,570.24996298)
\curveto(845.68445258,570.20996204)(845.86945239,570.18996206)(846.0994565,570.18996298)
\curveto(846.32945193,570.18996206)(846.51445175,570.20996204)(846.6544565,570.24996298)
\curveto(846.79445147,570.28996196)(846.91945134,570.32996192)(847.0294565,570.36996298)
\curveto(847.14945111,570.41996183)(847.259451,570.47996177)(847.3594565,570.54996298)
\curveto(847.46945079,570.61996163)(847.5644507,570.69996155)(847.6444565,570.78996298)
\curveto(847.72445054,570.88996136)(847.79445047,570.99496126)(847.8544565,571.10496298)
\curveto(847.91445035,571.20496105)(847.9644503,571.30996094)(848.0044565,571.41996298)
\curveto(848.05445021,571.52996072)(848.13445013,571.60996064)(848.2444565,571.65996298)
\curveto(848.28444998,571.67996057)(848.34944991,571.69496056)(848.4394565,571.70496298)
\curveto(848.52944973,571.71496054)(848.61944964,571.71496054)(848.7094565,571.70496298)
\curveto(848.79944946,571.70496055)(848.88444938,571.69996055)(848.9644565,571.68996298)
\curveto(849.04444922,571.67996057)(849.09944916,571.65996059)(849.1294565,571.62996298)
\curveto(849.22944903,571.55996069)(849.25444901,571.44496081)(849.2044565,571.28496298)
\curveto(849.12444914,571.01496124)(849.01944924,570.77496148)(848.8894565,570.56496298)
\curveto(848.68944957,570.24496201)(848.4594498,569.97996227)(848.1994565,569.76996298)
\curveto(847.94945031,569.56996268)(847.62945063,569.40496285)(847.2394565,569.27496298)
\curveto(847.13945112,569.23496302)(847.03945122,569.20996304)(846.9394565,569.19996298)
\curveto(846.83945142,569.17996307)(846.73445153,569.15996309)(846.6244565,569.13996298)
\curveto(846.57445169,569.12996312)(846.52445174,569.12496313)(846.4744565,569.12496298)
\curveto(846.43445183,569.12496313)(846.38945187,569.11996313)(846.3394565,569.10996298)
\lineto(846.1894565,569.10996298)
\curveto(846.13945212,569.09996315)(846.07945218,569.09496316)(846.0094565,569.09496298)
\curveto(845.94945231,569.09496316)(845.89945236,569.09996315)(845.8594565,569.10996298)
\lineto(845.7244565,569.10996298)
\curveto(845.67445259,569.11996313)(845.62945263,569.12496313)(845.5894565,569.12496298)
\curveto(845.54945271,569.12496313)(845.50945275,569.12996312)(845.4694565,569.13996298)
\curveto(845.41945284,569.1499631)(845.3644529,569.15996309)(845.3044565,569.16996298)
\curveto(845.24445302,569.16996308)(845.18945307,569.17496308)(845.1394565,569.18496298)
\curveto(845.04945321,569.20496305)(844.9594533,569.22996302)(844.8694565,569.25996298)
\curveto(844.77945348,569.27996297)(844.69445357,569.30496295)(844.6144565,569.33496298)
\curveto(844.57445369,569.3549629)(844.53945372,569.36496289)(844.5094565,569.36496298)
\curveto(844.47945378,569.37496288)(844.44445382,569.38996286)(844.4044565,569.40996298)
\curveto(844.25445401,569.47996277)(844.09445417,569.56496269)(843.9244565,569.66496298)
\curveto(843.63445463,569.8549624)(843.38445488,570.08496217)(843.1744565,570.35496298)
\curveto(842.97445529,570.63496162)(842.80445546,570.94496131)(842.6644565,571.28496298)
\curveto(842.61445565,571.39496086)(842.57445569,571.50996074)(842.5444565,571.62996298)
\curveto(842.52445574,571.7499605)(842.49445577,571.86996038)(842.4544565,571.98996298)
\curveto(842.44445582,572.02996022)(842.43945582,572.06496019)(842.4394565,572.09496298)
\curveto(842.43945582,572.12496013)(842.43445583,572.16496009)(842.4244565,572.21496298)
\curveto(842.40445586,572.29495996)(842.38945587,572.37995987)(842.3794565,572.46996298)
\curveto(842.36945589,572.55995969)(842.35445591,572.6499596)(842.3344565,572.73996298)
\lineto(842.3344565,572.94996298)
\curveto(842.32445594,572.98995926)(842.31445595,573.04495921)(842.3044565,573.11496298)
\curveto(842.30445596,573.19495906)(842.30945595,573.25995899)(842.3194565,573.30996298)
\lineto(842.3194565,573.47496298)
\curveto(842.33945592,573.52495873)(842.34445592,573.57495868)(842.3344565,573.62496298)
\curveto(842.33445593,573.68495857)(842.33945592,573.73995851)(842.3494565,573.78996298)
\curveto(842.38945587,573.9499583)(842.41945584,574.10995814)(842.4394565,574.26996298)
\curveto(842.46945579,574.42995782)(842.51445575,574.57995767)(842.5744565,574.71996298)
\curveto(842.62445564,574.82995742)(842.66945559,574.93995731)(842.7094565,575.04996298)
\curveto(842.7594555,575.16995708)(842.81445545,575.28495697)(842.8744565,575.39496298)
\curveto(843.09445517,575.74495651)(843.34445492,576.04495621)(843.6244565,576.29496298)
\curveto(843.90445436,576.5549557)(844.24945401,576.76995548)(844.6594565,576.93996298)
\curveto(844.77945348,576.98995526)(844.89945336,577.02495523)(845.0194565,577.04496298)
\curveto(845.14945311,577.07495518)(845.28445298,577.10495515)(845.4244565,577.13496298)
\curveto(845.47445279,577.14495511)(845.51945274,577.1499551)(845.5594565,577.14996298)
\curveto(845.59945266,577.15995509)(845.64445262,577.16495509)(845.6944565,577.16496298)
\curveto(845.71445255,577.17495508)(845.73945252,577.17495508)(845.7694565,577.16496298)
\curveto(845.79945246,577.1549551)(845.82445244,577.15995509)(845.8444565,577.17996298)
\curveto(846.264452,577.18995506)(846.62945163,577.14495511)(846.9394565,577.04496298)
\curveto(847.24945101,576.9549553)(847.52945073,576.82995542)(847.7794565,576.66996298)
\curveto(847.82945043,576.6499556)(847.86945039,576.61995563)(847.8994565,576.57996298)
\curveto(847.92945033,576.5499557)(847.9644503,576.52495573)(848.0044565,576.50496298)
\curveto(848.08445018,576.44495581)(848.1644501,576.37495588)(848.2444565,576.29496298)
\curveto(848.33444993,576.21495604)(848.40944985,576.13495612)(848.4694565,576.05496298)
\curveto(848.62944963,575.84495641)(848.7644495,575.64495661)(848.8744565,575.45496298)
\curveto(848.94444932,575.34495691)(848.99944926,575.22495703)(849.0394565,575.09496298)
\curveto(849.07944918,574.96495729)(849.12444914,574.83495742)(849.1744565,574.70496298)
\curveto(849.22444904,574.57495768)(849.259449,574.43995781)(849.2794565,574.29996298)
\curveto(849.30944895,574.15995809)(849.34444892,574.01995823)(849.3844565,573.87996298)
\curveto(849.39444887,573.80995844)(849.39944886,573.73995851)(849.3994565,573.66996298)
\lineto(849.4294565,573.45996298)
\moveto(847.9744565,573.96996298)
\curveto(848.00445026,574.00995824)(848.02945023,574.05995819)(848.0494565,574.11996298)
\curveto(848.06945019,574.18995806)(848.06945019,574.25995799)(848.0494565,574.32996298)
\curveto(847.98945027,574.5499577)(847.90445036,574.7549575)(847.7944565,574.94496298)
\curveto(847.65445061,575.17495708)(847.49945076,575.36995688)(847.3294565,575.52996298)
\curveto(847.1594511,575.68995656)(846.93945132,575.82495643)(846.6694565,575.93496298)
\curveto(846.59945166,575.9549563)(846.52945173,575.96995628)(846.4594565,575.97996298)
\curveto(846.38945187,575.99995625)(846.31445195,576.01995623)(846.2344565,576.03996298)
\curveto(846.15445211,576.05995619)(846.06945219,576.06995618)(845.9794565,576.06996298)
\lineto(845.7244565,576.06996298)
\curveto(845.69445257,576.0499562)(845.6594526,576.03995621)(845.6194565,576.03996298)
\curveto(845.57945268,576.0499562)(845.54445272,576.0499562)(845.5144565,576.03996298)
\lineto(845.2744565,575.97996298)
\curveto(845.20445306,575.96995628)(845.13445313,575.9549563)(845.0644565,575.93496298)
\curveto(844.77445349,575.81495644)(844.53945372,575.66495659)(844.3594565,575.48496298)
\curveto(844.18945407,575.30495695)(844.03445423,575.07995717)(843.8944565,574.80996298)
\curveto(843.8644544,574.75995749)(843.83445443,574.69495756)(843.8044565,574.61496298)
\curveto(843.77445449,574.54495771)(843.74945451,574.46495779)(843.7294565,574.37496298)
\curveto(843.70945455,574.28495797)(843.70445456,574.19995805)(843.7144565,574.11996298)
\curveto(843.72445454,574.03995821)(843.7594545,573.97995827)(843.8194565,573.93996298)
\curveto(843.89945436,573.87995837)(844.03445423,573.8499584)(844.2244565,573.84996298)
\curveto(844.42445384,573.85995839)(844.59445367,573.86495839)(844.7344565,573.86496298)
\lineto(847.0144565,573.86496298)
\curveto(847.1644511,573.86495839)(847.34445092,573.85995839)(847.5544565,573.84996298)
\curveto(847.7644505,573.8499584)(847.90445036,573.88995836)(847.9744565,573.96996298)
}
}
{
\newrgbcolor{curcolor}{0 0 0}
\pscustom[linestyle=none,fillstyle=solid,fillcolor=curcolor]
{
\newpath
\moveto(853.16609712,577.19496298)
\curveto(853.88609306,577.20495505)(854.49109245,577.11995513)(854.98109712,576.93996298)
\curveto(855.47109147,576.76995548)(855.85109109,576.46495579)(856.12109712,576.02496298)
\curveto(856.19109075,575.91495634)(856.2460907,575.79995645)(856.28609712,575.67996298)
\curveto(856.32609062,575.56995668)(856.36609058,575.44495681)(856.40609712,575.30496298)
\curveto(856.42609052,575.23495702)(856.43109051,575.15995709)(856.42109712,575.07996298)
\curveto(856.41109053,575.00995724)(856.39609055,574.9549573)(856.37609712,574.91496298)
\curveto(856.35609059,574.89495736)(856.33109061,574.87495738)(856.30109712,574.85496298)
\curveto(856.27109067,574.84495741)(856.2460907,574.82995742)(856.22609712,574.80996298)
\curveto(856.17609077,574.78995746)(856.12609082,574.78495747)(856.07609712,574.79496298)
\curveto(856.02609092,574.80495745)(855.97609097,574.80495745)(855.92609712,574.79496298)
\curveto(855.8460911,574.77495748)(855.7410912,574.76995748)(855.61109712,574.77996298)
\curveto(855.48109146,574.79995745)(855.39109155,574.82495743)(855.34109712,574.85496298)
\curveto(855.26109168,574.90495735)(855.20609174,574.96995728)(855.17609712,575.04996298)
\curveto(855.15609179,575.13995711)(855.12109182,575.22495703)(855.07109712,575.30496298)
\curveto(854.98109196,575.46495679)(854.85609209,575.60995664)(854.69609712,575.73996298)
\curveto(854.58609236,575.81995643)(854.46609248,575.87995637)(854.33609712,575.91996298)
\curveto(854.20609274,575.95995629)(854.06609288,575.99995625)(853.91609712,576.03996298)
\curveto(853.86609308,576.05995619)(853.81609313,576.06495619)(853.76609712,576.05496298)
\curveto(853.71609323,576.0549562)(853.66609328,576.05995619)(853.61609712,576.06996298)
\curveto(853.55609339,576.08995616)(853.48109346,576.09995615)(853.39109712,576.09996298)
\curveto(853.30109364,576.09995615)(853.22609372,576.08995616)(853.16609712,576.06996298)
\lineto(853.07609712,576.06996298)
\lineto(852.92609712,576.03996298)
\curveto(852.87609407,576.03995621)(852.82609412,576.03495622)(852.77609712,576.02496298)
\curveto(852.51609443,575.96495629)(852.30109464,575.87995637)(852.13109712,575.76996298)
\curveto(851.96109498,575.65995659)(851.8460951,575.47495678)(851.78609712,575.21496298)
\curveto(851.76609518,575.14495711)(851.76109518,575.07495718)(851.77109712,575.00496298)
\curveto(851.79109515,574.93495732)(851.81109513,574.87495738)(851.83109712,574.82496298)
\curveto(851.89109505,574.67495758)(851.96109498,574.56495769)(852.04109712,574.49496298)
\curveto(852.13109481,574.43495782)(852.2410947,574.36495789)(852.37109712,574.28496298)
\curveto(852.53109441,574.18495807)(852.71109423,574.10995814)(852.91109712,574.05996298)
\curveto(853.11109383,574.01995823)(853.31109363,573.96995828)(853.51109712,573.90996298)
\curveto(853.6410933,573.86995838)(853.77109317,573.83995841)(853.90109712,573.81996298)
\curveto(854.03109291,573.79995845)(854.16109278,573.76995848)(854.29109712,573.72996298)
\curveto(854.50109244,573.66995858)(854.70609224,573.60995864)(854.90609712,573.54996298)
\curveto(855.10609184,573.49995875)(855.30609164,573.43495882)(855.50609712,573.35496298)
\lineto(855.65609712,573.29496298)
\curveto(855.70609124,573.27495898)(855.75609119,573.249959)(855.80609712,573.21996298)
\curveto(856.00609094,573.09995915)(856.18109076,572.96495929)(856.33109712,572.81496298)
\curveto(856.48109046,572.66495959)(856.60609034,572.47495978)(856.70609712,572.24496298)
\curveto(856.72609022,572.17496008)(856.7460902,572.07996017)(856.76609712,571.95996298)
\curveto(856.78609016,571.88996036)(856.79609015,571.81496044)(856.79609712,571.73496298)
\curveto(856.80609014,571.66496059)(856.81109013,571.58496067)(856.81109712,571.49496298)
\lineto(856.81109712,571.34496298)
\curveto(856.79109015,571.27496098)(856.78109016,571.20496105)(856.78109712,571.13496298)
\curveto(856.78109016,571.06496119)(856.77109017,570.99496126)(856.75109712,570.92496298)
\curveto(856.72109022,570.81496144)(856.68609026,570.70996154)(856.64609712,570.60996298)
\curveto(856.60609034,570.50996174)(856.56109038,570.41996183)(856.51109712,570.33996298)
\curveto(856.35109059,570.07996217)(856.1460908,569.86996238)(855.89609712,569.70996298)
\curveto(855.6460913,569.55996269)(855.36609158,569.42996282)(855.05609712,569.31996298)
\curveto(854.96609198,569.28996296)(854.87109207,569.26996298)(854.77109712,569.25996298)
\curveto(854.68109226,569.23996301)(854.59109235,569.21496304)(854.50109712,569.18496298)
\curveto(854.40109254,569.16496309)(854.30109264,569.1549631)(854.20109712,569.15496298)
\curveto(854.10109284,569.1549631)(854.00109294,569.14496311)(853.90109712,569.12496298)
\lineto(853.75109712,569.12496298)
\curveto(853.70109324,569.11496314)(853.63109331,569.10996314)(853.54109712,569.10996298)
\curveto(853.45109349,569.10996314)(853.38109356,569.11496314)(853.33109712,569.12496298)
\lineto(853.16609712,569.12496298)
\curveto(853.10609384,569.14496311)(853.0410939,569.1549631)(852.97109712,569.15496298)
\curveto(852.90109404,569.14496311)(852.8410941,569.1499631)(852.79109712,569.16996298)
\curveto(852.7410942,569.17996307)(852.67609427,569.18496307)(852.59609712,569.18496298)
\lineto(852.35609712,569.24496298)
\curveto(852.28609466,569.254963)(852.21109473,569.27496298)(852.13109712,569.30496298)
\curveto(851.82109512,569.40496285)(851.55109539,569.52996272)(851.32109712,569.67996298)
\curveto(851.09109585,569.82996242)(850.89109605,570.02496223)(850.72109712,570.26496298)
\curveto(850.63109631,570.39496186)(850.55609639,570.52996172)(850.49609712,570.66996298)
\curveto(850.43609651,570.80996144)(850.38109656,570.96496129)(850.33109712,571.13496298)
\curveto(850.31109663,571.19496106)(850.30109664,571.26496099)(850.30109712,571.34496298)
\curveto(850.31109663,571.43496082)(850.32609662,571.50496075)(850.34609712,571.55496298)
\curveto(850.37609657,571.59496066)(850.42609652,571.63496062)(850.49609712,571.67496298)
\curveto(850.5460964,571.69496056)(850.61609633,571.70496055)(850.70609712,571.70496298)
\curveto(850.79609615,571.71496054)(850.88609606,571.71496054)(850.97609712,571.70496298)
\curveto(851.06609588,571.69496056)(851.15109579,571.67996057)(851.23109712,571.65996298)
\curveto(851.32109562,571.6499606)(851.38109556,571.63496062)(851.41109712,571.61496298)
\curveto(851.48109546,571.56496069)(851.52609542,571.48996076)(851.54609712,571.38996298)
\curveto(851.57609537,571.29996095)(851.61109533,571.21496104)(851.65109712,571.13496298)
\curveto(851.75109519,570.91496134)(851.88609506,570.74496151)(852.05609712,570.62496298)
\curveto(852.17609477,570.53496172)(852.31109463,570.46496179)(852.46109712,570.41496298)
\curveto(852.61109433,570.36496189)(852.77109417,570.31496194)(852.94109712,570.26496298)
\lineto(853.25609712,570.21996298)
\lineto(853.34609712,570.21996298)
\curveto(853.41609353,570.19996205)(853.50609344,570.18996206)(853.61609712,570.18996298)
\curveto(853.73609321,570.18996206)(853.83609311,570.19996205)(853.91609712,570.21996298)
\curveto(853.98609296,570.21996203)(854.0410929,570.22496203)(854.08109712,570.23496298)
\curveto(854.1410928,570.24496201)(854.20109274,570.249962)(854.26109712,570.24996298)
\curveto(854.32109262,570.25996199)(854.37609257,570.26996198)(854.42609712,570.27996298)
\curveto(854.71609223,570.35996189)(854.946092,570.46496179)(855.11609712,570.59496298)
\curveto(855.28609166,570.72496153)(855.40609154,570.94496131)(855.47609712,571.25496298)
\curveto(855.49609145,571.30496095)(855.50109144,571.35996089)(855.49109712,571.41996298)
\curveto(855.48109146,571.47996077)(855.47109147,571.52496073)(855.46109712,571.55496298)
\curveto(855.41109153,571.74496051)(855.3410916,571.88496037)(855.25109712,571.97496298)
\curveto(855.16109178,572.07496018)(855.0460919,572.16496009)(854.90609712,572.24496298)
\curveto(854.81609213,572.30495995)(854.71609223,572.3549599)(854.60609712,572.39496298)
\lineto(854.27609712,572.51496298)
\curveto(854.2460927,572.52495973)(854.21609273,572.52995972)(854.18609712,572.52996298)
\curveto(854.16609278,572.52995972)(854.1410928,572.53995971)(854.11109712,572.55996298)
\curveto(853.77109317,572.66995958)(853.41609353,572.7499595)(853.04609712,572.79996298)
\curveto(852.68609426,572.85995939)(852.3460946,572.9549593)(852.02609712,573.08496298)
\curveto(851.92609502,573.12495913)(851.83109511,573.15995909)(851.74109712,573.18996298)
\curveto(851.65109529,573.21995903)(851.56609538,573.25995899)(851.48609712,573.30996298)
\curveto(851.29609565,573.41995883)(851.12109582,573.54495871)(850.96109712,573.68496298)
\curveto(850.80109614,573.82495843)(850.67609627,573.99995825)(850.58609712,574.20996298)
\curveto(850.55609639,574.27995797)(850.53109641,574.3499579)(850.51109712,574.41996298)
\curveto(850.50109644,574.48995776)(850.48609646,574.56495769)(850.46609712,574.64496298)
\curveto(850.43609651,574.76495749)(850.42609652,574.89995735)(850.43609712,575.04996298)
\curveto(850.4460965,575.20995704)(850.46109648,575.34495691)(850.48109712,575.45496298)
\curveto(850.50109644,575.50495675)(850.51109643,575.54495671)(850.51109712,575.57496298)
\curveto(850.52109642,575.61495664)(850.53609641,575.6549566)(850.55609712,575.69496298)
\curveto(850.6460963,575.92495633)(850.76609618,576.12495613)(850.91609712,576.29496298)
\curveto(851.07609587,576.46495579)(851.25609569,576.61495564)(851.45609712,576.74496298)
\curveto(851.60609534,576.83495542)(851.77109517,576.90495535)(851.95109712,576.95496298)
\curveto(852.13109481,577.01495524)(852.32109462,577.06995518)(852.52109712,577.11996298)
\curveto(852.59109435,577.12995512)(852.65609429,577.13995511)(852.71609712,577.14996298)
\curveto(852.78609416,577.15995509)(852.86109408,577.16995508)(852.94109712,577.17996298)
\curveto(852.97109397,577.18995506)(853.01109393,577.18995506)(853.06109712,577.17996298)
\curveto(853.11109383,577.16995508)(853.1460938,577.17495508)(853.16609712,577.19496298)
}
}
{
\newrgbcolor{curcolor}{0 0 0}
\pscustom[linestyle=none,fillstyle=solid,fillcolor=curcolor]
{
\newpath
\moveto(865.12109712,569.84496298)
\curveto(865.15108929,569.68496257)(865.13608931,569.5499627)(865.07609712,569.43996298)
\curveto(865.01608943,569.33996291)(864.93608951,569.26496299)(864.83609712,569.21496298)
\curveto(864.78608966,569.19496306)(864.73108971,569.18496307)(864.67109712,569.18496298)
\curveto(864.62108982,569.18496307)(864.56608988,569.17496308)(864.50609712,569.15496298)
\curveto(864.28609016,569.10496315)(864.06609038,569.11996313)(863.84609712,569.19996298)
\curveto(863.63609081,569.26996298)(863.49109095,569.35996289)(863.41109712,569.46996298)
\curveto(863.36109108,569.53996271)(863.31609113,569.61996263)(863.27609712,569.70996298)
\curveto(863.23609121,569.80996244)(863.18609126,569.88996236)(863.12609712,569.94996298)
\curveto(863.10609134,569.96996228)(863.08109136,569.98996226)(863.05109712,570.00996298)
\curveto(863.03109141,570.02996222)(863.00109144,570.03496222)(862.96109712,570.02496298)
\curveto(862.85109159,569.99496226)(862.7460917,569.93996231)(862.64609712,569.85996298)
\curveto(862.55609189,569.77996247)(862.46609198,569.70996254)(862.37609712,569.64996298)
\curveto(862.2460922,569.56996268)(862.10609234,569.49496276)(861.95609712,569.42496298)
\curveto(861.80609264,569.36496289)(861.6460928,569.30996294)(861.47609712,569.25996298)
\curveto(861.37609307,569.22996302)(861.26609318,569.20996304)(861.14609712,569.19996298)
\curveto(861.03609341,569.18996306)(860.92609352,569.17496308)(860.81609712,569.15496298)
\curveto(860.76609368,569.14496311)(860.72109372,569.13996311)(860.68109712,569.13996298)
\lineto(860.57609712,569.13996298)
\curveto(860.46609398,569.11996313)(860.36109408,569.11996313)(860.26109712,569.13996298)
\lineto(860.12609712,569.13996298)
\curveto(860.07609437,569.1499631)(860.02609442,569.1549631)(859.97609712,569.15496298)
\curveto(859.92609452,569.1549631)(859.88109456,569.16496309)(859.84109712,569.18496298)
\curveto(859.80109464,569.19496306)(859.76609468,569.19996305)(859.73609712,569.19996298)
\curveto(859.71609473,569.18996306)(859.69109475,569.18996306)(859.66109712,569.19996298)
\lineto(859.42109712,569.25996298)
\curveto(859.3410951,569.26996298)(859.26609518,569.28996296)(859.19609712,569.31996298)
\curveto(858.89609555,569.4499628)(858.65109579,569.59496266)(858.46109712,569.75496298)
\curveto(858.28109616,569.92496233)(858.13109631,570.15996209)(858.01109712,570.45996298)
\curveto(857.92109652,570.67996157)(857.87609657,570.94496131)(857.87609712,571.25496298)
\lineto(857.87609712,571.56996298)
\curveto(857.88609656,571.61996063)(857.89109655,571.66996058)(857.89109712,571.71996298)
\lineto(857.92109712,571.89996298)
\lineto(858.04109712,572.22996298)
\curveto(858.08109636,572.33995991)(858.13109631,572.43995981)(858.19109712,572.52996298)
\curveto(858.37109607,572.81995943)(858.61609583,573.03495922)(858.92609712,573.17496298)
\curveto(859.23609521,573.31495894)(859.57609487,573.43995881)(859.94609712,573.54996298)
\curveto(860.08609436,573.58995866)(860.23109421,573.61995863)(860.38109712,573.63996298)
\curveto(860.53109391,573.65995859)(860.68109376,573.68495857)(860.83109712,573.71496298)
\curveto(860.90109354,573.73495852)(860.96609348,573.74495851)(861.02609712,573.74496298)
\curveto(861.09609335,573.74495851)(861.17109327,573.7549585)(861.25109712,573.77496298)
\curveto(861.32109312,573.79495846)(861.39109305,573.80495845)(861.46109712,573.80496298)
\curveto(861.53109291,573.81495844)(861.60609284,573.82995842)(861.68609712,573.84996298)
\curveto(861.93609251,573.90995834)(862.17109227,573.95995829)(862.39109712,573.99996298)
\curveto(862.61109183,574.0499582)(862.78609166,574.16495809)(862.91609712,574.34496298)
\curveto(862.97609147,574.42495783)(863.02609142,574.52495773)(863.06609712,574.64496298)
\curveto(863.10609134,574.77495748)(863.10609134,574.91495734)(863.06609712,575.06496298)
\curveto(863.00609144,575.30495695)(862.91609153,575.49495676)(862.79609712,575.63496298)
\curveto(862.68609176,575.77495648)(862.52609192,575.88495637)(862.31609712,575.96496298)
\curveto(862.19609225,576.01495624)(862.05109239,576.0499562)(861.88109712,576.06996298)
\curveto(861.72109272,576.08995616)(861.55109289,576.09995615)(861.37109712,576.09996298)
\curveto(861.19109325,576.09995615)(861.01609343,576.08995616)(860.84609712,576.06996298)
\curveto(860.67609377,576.0499562)(860.53109391,576.01995623)(860.41109712,575.97996298)
\curveto(860.2410942,575.91995633)(860.07609437,575.83495642)(859.91609712,575.72496298)
\curveto(859.83609461,575.66495659)(859.76109468,575.58495667)(859.69109712,575.48496298)
\curveto(859.63109481,575.39495686)(859.57609487,575.29495696)(859.52609712,575.18496298)
\curveto(859.49609495,575.10495715)(859.46609498,575.01995723)(859.43609712,574.92996298)
\curveto(859.41609503,574.83995741)(859.37109507,574.76995748)(859.30109712,574.71996298)
\curveto(859.26109518,574.68995756)(859.19109525,574.66495759)(859.09109712,574.64496298)
\curveto(859.00109544,574.63495762)(858.90609554,574.62995762)(858.80609712,574.62996298)
\curveto(858.70609574,574.62995762)(858.60609584,574.63495762)(858.50609712,574.64496298)
\curveto(858.41609603,574.66495759)(858.35109609,574.68995756)(858.31109712,574.71996298)
\curveto(858.27109617,574.7499575)(858.2410962,574.79995745)(858.22109712,574.86996298)
\curveto(858.20109624,574.93995731)(858.20109624,575.01495724)(858.22109712,575.09496298)
\curveto(858.25109619,575.22495703)(858.28109616,575.34495691)(858.31109712,575.45496298)
\curveto(858.35109609,575.57495668)(858.39609605,575.68995656)(858.44609712,575.79996298)
\curveto(858.63609581,576.1499561)(858.87609557,576.41995583)(859.16609712,576.60996298)
\curveto(859.45609499,576.80995544)(859.81609463,576.96995528)(860.24609712,577.08996298)
\curveto(860.3460941,577.10995514)(860.446094,577.12495513)(860.54609712,577.13496298)
\curveto(860.65609379,577.14495511)(860.76609368,577.15995509)(860.87609712,577.17996298)
\curveto(860.91609353,577.18995506)(860.98109346,577.18995506)(861.07109712,577.17996298)
\curveto(861.16109328,577.17995507)(861.21609323,577.18995506)(861.23609712,577.20996298)
\curveto(861.93609251,577.21995503)(862.5460919,577.13995511)(863.06609712,576.96996298)
\curveto(863.58609086,576.79995545)(863.95109049,576.47495578)(864.16109712,575.99496298)
\curveto(864.25109019,575.79495646)(864.30109014,575.55995669)(864.31109712,575.28996298)
\curveto(864.33109011,575.02995722)(864.3410901,574.7549575)(864.34109712,574.46496298)
\lineto(864.34109712,571.14996298)
\curveto(864.3410901,571.00996124)(864.3460901,570.87496138)(864.35609712,570.74496298)
\curveto(864.36609008,570.61496164)(864.39609005,570.50996174)(864.44609712,570.42996298)
\curveto(864.49608995,570.35996189)(864.56108988,570.30996194)(864.64109712,570.27996298)
\curveto(864.73108971,570.23996201)(864.81608963,570.20996204)(864.89609712,570.18996298)
\curveto(864.97608947,570.17996207)(865.03608941,570.13496212)(865.07609712,570.05496298)
\curveto(865.09608935,570.02496223)(865.10608934,569.99496226)(865.10609712,569.96496298)
\curveto(865.10608934,569.93496232)(865.11108933,569.89496236)(865.12109712,569.84496298)
\moveto(862.97609712,571.50996298)
\curveto(863.03609141,571.6499606)(863.06609138,571.80996044)(863.06609712,571.98996298)
\curveto(863.07609137,572.17996007)(863.08109136,572.37495988)(863.08109712,572.57496298)
\curveto(863.08109136,572.68495957)(863.07609137,572.78495947)(863.06609712,572.87496298)
\curveto(863.05609139,572.96495929)(863.01609143,573.03495922)(862.94609712,573.08496298)
\curveto(862.91609153,573.10495915)(862.8460916,573.11495914)(862.73609712,573.11496298)
\curveto(862.71609173,573.09495916)(862.68109176,573.08495917)(862.63109712,573.08496298)
\curveto(862.58109186,573.08495917)(862.53609191,573.07495918)(862.49609712,573.05496298)
\curveto(862.41609203,573.03495922)(862.32609212,573.01495924)(862.22609712,572.99496298)
\lineto(861.92609712,572.93496298)
\curveto(861.89609255,572.93495932)(861.86109258,572.92995932)(861.82109712,572.91996298)
\lineto(861.71609712,572.91996298)
\curveto(861.56609288,572.87995937)(861.40109304,572.8549594)(861.22109712,572.84496298)
\curveto(861.05109339,572.84495941)(860.89109355,572.82495943)(860.74109712,572.78496298)
\curveto(860.66109378,572.76495949)(860.58609386,572.74495951)(860.51609712,572.72496298)
\curveto(860.45609399,572.71495954)(860.38609406,572.69995955)(860.30609712,572.67996298)
\curveto(860.1460943,572.62995962)(859.99609445,572.56495969)(859.85609712,572.48496298)
\curveto(859.71609473,572.41495984)(859.59609485,572.32495993)(859.49609712,572.21496298)
\curveto(859.39609505,572.10496015)(859.32109512,571.96996028)(859.27109712,571.80996298)
\curveto(859.22109522,571.65996059)(859.20109524,571.47496078)(859.21109712,571.25496298)
\curveto(859.21109523,571.1549611)(859.22609522,571.05996119)(859.25609712,570.96996298)
\curveto(859.29609515,570.88996136)(859.3410951,570.81496144)(859.39109712,570.74496298)
\curveto(859.47109497,570.63496162)(859.57609487,570.53996171)(859.70609712,570.45996298)
\curveto(859.83609461,570.38996186)(859.97609447,570.32996192)(860.12609712,570.27996298)
\curveto(860.17609427,570.26996198)(860.22609422,570.26496199)(860.27609712,570.26496298)
\curveto(860.32609412,570.26496199)(860.37609407,570.25996199)(860.42609712,570.24996298)
\curveto(860.49609395,570.22996202)(860.58109386,570.21496204)(860.68109712,570.20496298)
\curveto(860.79109365,570.20496205)(860.88109356,570.21496204)(860.95109712,570.23496298)
\curveto(861.01109343,570.254962)(861.07109337,570.25996199)(861.13109712,570.24996298)
\curveto(861.19109325,570.249962)(861.25109319,570.25996199)(861.31109712,570.27996298)
\curveto(861.39109305,570.29996195)(861.46609298,570.31496194)(861.53609712,570.32496298)
\curveto(861.61609283,570.33496192)(861.69109275,570.3549619)(861.76109712,570.38496298)
\curveto(862.05109239,570.50496175)(862.29609215,570.6499616)(862.49609712,570.81996298)
\curveto(862.70609174,570.98996126)(862.86609158,571.21996103)(862.97609712,571.50996298)
}
}
{
\newrgbcolor{curcolor}{0 0 0}
\pscustom[linestyle=none,fillstyle=solid,fillcolor=curcolor]
{
\newpath
\moveto(869.93773775,577.19496298)
\curveto(870.16773296,577.19495506)(870.29773283,577.13495512)(870.32773775,577.01496298)
\curveto(870.35773277,576.90495535)(870.37273275,576.73995551)(870.37273775,576.51996298)
\lineto(870.37273775,576.23496298)
\curveto(870.37273275,576.14495611)(870.34773278,576.06995618)(870.29773775,576.00996298)
\curveto(870.23773289,575.92995632)(870.15273297,575.88495637)(870.04273775,575.87496298)
\curveto(869.93273319,575.87495638)(869.8227333,575.85995639)(869.71273775,575.82996298)
\curveto(869.57273355,575.79995645)(869.43773369,575.76995648)(869.30773775,575.73996298)
\curveto(869.18773394,575.70995654)(869.07273405,575.66995658)(868.96273775,575.61996298)
\curveto(868.67273445,575.48995676)(868.43773469,575.30995694)(868.25773775,575.07996298)
\curveto(868.07773505,574.85995739)(867.9227352,574.60495765)(867.79273775,574.31496298)
\curveto(867.75273537,574.20495805)(867.7227354,574.08995816)(867.70273775,573.96996298)
\curveto(867.68273544,573.85995839)(867.65773547,573.74495851)(867.62773775,573.62496298)
\curveto(867.61773551,573.57495868)(867.61273551,573.52495873)(867.61273775,573.47496298)
\curveto(867.6227355,573.42495883)(867.6227355,573.37495888)(867.61273775,573.32496298)
\curveto(867.58273554,573.20495905)(867.56773556,573.06495919)(867.56773775,572.90496298)
\curveto(867.57773555,572.7549595)(867.58273554,572.60995964)(867.58273775,572.46996298)
\lineto(867.58273775,570.62496298)
\lineto(867.58273775,570.27996298)
\curveto(867.58273554,570.15996209)(867.57773555,570.04496221)(867.56773775,569.93496298)
\curveto(867.55773557,569.82496243)(867.55273557,569.72996252)(867.55273775,569.64996298)
\curveto(867.56273556,569.56996268)(867.54273558,569.49996275)(867.49273775,569.43996298)
\curveto(867.44273568,569.36996288)(867.36273576,569.32996292)(867.25273775,569.31996298)
\curveto(867.15273597,569.30996294)(867.04273608,569.30496295)(866.92273775,569.30496298)
\lineto(866.65273775,569.30496298)
\curveto(866.60273652,569.32496293)(866.55273657,569.33996291)(866.50273775,569.34996298)
\curveto(866.46273666,569.36996288)(866.43273669,569.39496286)(866.41273775,569.42496298)
\curveto(866.36273676,569.49496276)(866.33273679,569.57996267)(866.32273775,569.67996298)
\lineto(866.32273775,570.00996298)
\lineto(866.32273775,571.16496298)
\lineto(866.32273775,575.31996298)
\lineto(866.32273775,576.35496298)
\lineto(866.32273775,576.65496298)
\curveto(866.33273679,576.7549555)(866.36273676,576.83995541)(866.41273775,576.90996298)
\curveto(866.44273668,576.9499553)(866.49273663,576.97995527)(866.56273775,576.99996298)
\curveto(866.64273648,577.01995523)(866.7277364,577.02995522)(866.81773775,577.02996298)
\curveto(866.90773622,577.03995521)(866.99773613,577.03995521)(867.08773775,577.02996298)
\curveto(867.17773595,577.01995523)(867.24773588,577.00495525)(867.29773775,576.98496298)
\curveto(867.37773575,576.9549553)(867.4277357,576.89495536)(867.44773775,576.80496298)
\curveto(867.47773565,576.72495553)(867.49273563,576.63495562)(867.49273775,576.53496298)
\lineto(867.49273775,576.23496298)
\curveto(867.49273563,576.13495612)(867.51273561,576.04495621)(867.55273775,575.96496298)
\curveto(867.56273556,575.94495631)(867.57273555,575.92995632)(867.58273775,575.91996298)
\lineto(867.62773775,575.87496298)
\curveto(867.73773539,575.87495638)(867.8277353,575.91995633)(867.89773775,576.00996298)
\curveto(867.96773516,576.10995614)(868.0277351,576.18995606)(868.07773775,576.24996298)
\lineto(868.16773775,576.33996298)
\curveto(868.25773487,576.4499558)(868.38273474,576.56495569)(868.54273775,576.68496298)
\curveto(868.70273442,576.80495545)(868.85273427,576.89495536)(868.99273775,576.95496298)
\curveto(869.08273404,577.00495525)(869.17773395,577.03995521)(869.27773775,577.05996298)
\curveto(869.37773375,577.08995516)(869.48273364,577.11995513)(869.59273775,577.14996298)
\curveto(869.65273347,577.15995509)(869.71273341,577.16495509)(869.77273775,577.16496298)
\curveto(869.83273329,577.17495508)(869.88773324,577.18495507)(869.93773775,577.19496298)
}
}
{
\newrgbcolor{curcolor}{0 0 0}
\pscustom[linestyle=none,fillstyle=solid,fillcolor=curcolor]
{
\newpath
\moveto(874.94750337,577.19496298)
\curveto(875.17749858,577.19495506)(875.30749845,577.13495512)(875.33750337,577.01496298)
\curveto(875.36749839,576.90495535)(875.38249838,576.73995551)(875.38250337,576.51996298)
\lineto(875.38250337,576.23496298)
\curveto(875.38249838,576.14495611)(875.3574984,576.06995618)(875.30750337,576.00996298)
\curveto(875.24749851,575.92995632)(875.1624986,575.88495637)(875.05250337,575.87496298)
\curveto(874.94249882,575.87495638)(874.83249893,575.85995639)(874.72250337,575.82996298)
\curveto(874.58249918,575.79995645)(874.44749931,575.76995648)(874.31750337,575.73996298)
\curveto(874.19749956,575.70995654)(874.08249968,575.66995658)(873.97250337,575.61996298)
\curveto(873.68250008,575.48995676)(873.44750031,575.30995694)(873.26750337,575.07996298)
\curveto(873.08750067,574.85995739)(872.93250083,574.60495765)(872.80250337,574.31496298)
\curveto(872.762501,574.20495805)(872.73250103,574.08995816)(872.71250337,573.96996298)
\curveto(872.69250107,573.85995839)(872.66750109,573.74495851)(872.63750337,573.62496298)
\curveto(872.62750113,573.57495868)(872.62250114,573.52495873)(872.62250337,573.47496298)
\curveto(872.63250113,573.42495883)(872.63250113,573.37495888)(872.62250337,573.32496298)
\curveto(872.59250117,573.20495905)(872.57750118,573.06495919)(872.57750337,572.90496298)
\curveto(872.58750117,572.7549595)(872.59250117,572.60995964)(872.59250337,572.46996298)
\lineto(872.59250337,570.62496298)
\lineto(872.59250337,570.27996298)
\curveto(872.59250117,570.15996209)(872.58750117,570.04496221)(872.57750337,569.93496298)
\curveto(872.56750119,569.82496243)(872.5625012,569.72996252)(872.56250337,569.64996298)
\curveto(872.57250119,569.56996268)(872.55250121,569.49996275)(872.50250337,569.43996298)
\curveto(872.45250131,569.36996288)(872.37250139,569.32996292)(872.26250337,569.31996298)
\curveto(872.1625016,569.30996294)(872.05250171,569.30496295)(871.93250337,569.30496298)
\lineto(871.66250337,569.30496298)
\curveto(871.61250215,569.32496293)(871.5625022,569.33996291)(871.51250337,569.34996298)
\curveto(871.47250229,569.36996288)(871.44250232,569.39496286)(871.42250337,569.42496298)
\curveto(871.37250239,569.49496276)(871.34250242,569.57996267)(871.33250337,569.67996298)
\lineto(871.33250337,570.00996298)
\lineto(871.33250337,571.16496298)
\lineto(871.33250337,575.31996298)
\lineto(871.33250337,576.35496298)
\lineto(871.33250337,576.65496298)
\curveto(871.34250242,576.7549555)(871.37250239,576.83995541)(871.42250337,576.90996298)
\curveto(871.45250231,576.9499553)(871.50250226,576.97995527)(871.57250337,576.99996298)
\curveto(871.65250211,577.01995523)(871.73750202,577.02995522)(871.82750337,577.02996298)
\curveto(871.91750184,577.03995521)(872.00750175,577.03995521)(872.09750337,577.02996298)
\curveto(872.18750157,577.01995523)(872.2575015,577.00495525)(872.30750337,576.98496298)
\curveto(872.38750137,576.9549553)(872.43750132,576.89495536)(872.45750337,576.80496298)
\curveto(872.48750127,576.72495553)(872.50250126,576.63495562)(872.50250337,576.53496298)
\lineto(872.50250337,576.23496298)
\curveto(872.50250126,576.13495612)(872.52250124,576.04495621)(872.56250337,575.96496298)
\curveto(872.57250119,575.94495631)(872.58250118,575.92995632)(872.59250337,575.91996298)
\lineto(872.63750337,575.87496298)
\curveto(872.74750101,575.87495638)(872.83750092,575.91995633)(872.90750337,576.00996298)
\curveto(872.97750078,576.10995614)(873.03750072,576.18995606)(873.08750337,576.24996298)
\lineto(873.17750337,576.33996298)
\curveto(873.26750049,576.4499558)(873.39250037,576.56495569)(873.55250337,576.68496298)
\curveto(873.71250005,576.80495545)(873.8624999,576.89495536)(874.00250337,576.95496298)
\curveto(874.09249967,577.00495525)(874.18749957,577.03995521)(874.28750337,577.05996298)
\curveto(874.38749937,577.08995516)(874.49249927,577.11995513)(874.60250337,577.14996298)
\curveto(874.6624991,577.15995509)(874.72249904,577.16495509)(874.78250337,577.16496298)
\curveto(874.84249892,577.17495508)(874.89749886,577.18495507)(874.94750337,577.19496298)
}
}
{
\newrgbcolor{curcolor}{0 0 0}
\pscustom[linestyle=none,fillstyle=solid,fillcolor=curcolor]
{
\newpath
\moveto(883.437269,573.48996298)
\curveto(883.45726094,573.42995882)(883.46726093,573.33495892)(883.467269,573.20496298)
\curveto(883.46726093,573.08495917)(883.46226093,572.99995925)(883.452269,572.94996298)
\lineto(883.452269,572.79996298)
\curveto(883.44226095,572.71995953)(883.43226096,572.64495961)(883.422269,572.57496298)
\curveto(883.42226097,572.51495974)(883.41726098,572.44495981)(883.407269,572.36496298)
\curveto(883.38726101,572.30495995)(883.37226102,572.24496001)(883.362269,572.18496298)
\curveto(883.36226103,572.12496013)(883.35226104,572.06496019)(883.332269,572.00496298)
\curveto(883.2922611,571.87496038)(883.25726114,571.74496051)(883.227269,571.61496298)
\curveto(883.1972612,571.48496077)(883.15726124,571.36496089)(883.107269,571.25496298)
\curveto(882.8972615,570.77496148)(882.61726178,570.36996188)(882.267269,570.03996298)
\curveto(881.91726248,569.71996253)(881.48726291,569.47496278)(880.977269,569.30496298)
\curveto(880.86726353,569.26496299)(880.74726365,569.23496302)(880.617269,569.21496298)
\curveto(880.4972639,569.19496306)(880.37226402,569.17496308)(880.242269,569.15496298)
\curveto(880.18226421,569.14496311)(880.11726428,569.13996311)(880.047269,569.13996298)
\curveto(879.98726441,569.12996312)(879.92726447,569.12496313)(879.867269,569.12496298)
\curveto(879.82726457,569.11496314)(879.76726463,569.10996314)(879.687269,569.10996298)
\curveto(879.61726478,569.10996314)(879.56726483,569.11496314)(879.537269,569.12496298)
\curveto(879.4972649,569.13496312)(879.45726494,569.13996311)(879.417269,569.13996298)
\curveto(879.37726502,569.12996312)(879.34226505,569.12996312)(879.312269,569.13996298)
\lineto(879.222269,569.13996298)
\lineto(878.862269,569.18496298)
\curveto(878.72226567,569.22496303)(878.58726581,569.26496299)(878.457269,569.30496298)
\curveto(878.32726607,569.34496291)(878.20226619,569.38996286)(878.082269,569.43996298)
\curveto(877.63226676,569.63996261)(877.26226713,569.89996235)(876.972269,570.21996298)
\curveto(876.68226771,570.53996171)(876.44226795,570.92996132)(876.252269,571.38996298)
\curveto(876.20226819,571.48996076)(876.16226823,571.58996066)(876.132269,571.68996298)
\curveto(876.11226828,571.78996046)(876.0922683,571.89496036)(876.072269,572.00496298)
\curveto(876.05226834,572.04496021)(876.04226835,572.07496018)(876.042269,572.09496298)
\curveto(876.05226834,572.12496013)(876.05226834,572.15996009)(876.042269,572.19996298)
\curveto(876.02226837,572.27995997)(876.00726839,572.35995989)(875.997269,572.43996298)
\curveto(875.9972684,572.52995972)(875.98726841,572.61495964)(875.967269,572.69496298)
\lineto(875.967269,572.81496298)
\curveto(875.96726843,572.8549594)(875.96226843,572.89995935)(875.952269,572.94996298)
\curveto(875.94226845,572.99995925)(875.93726846,573.08495917)(875.937269,573.20496298)
\curveto(875.93726846,573.33495892)(875.94726845,573.42995882)(875.967269,573.48996298)
\curveto(875.98726841,573.55995869)(875.9922684,573.62995862)(875.982269,573.69996298)
\curveto(875.97226842,573.76995848)(875.97726842,573.83995841)(875.997269,573.90996298)
\curveto(876.00726839,573.95995829)(876.01226838,573.99995825)(876.012269,574.02996298)
\curveto(876.02226837,574.06995818)(876.03226836,574.11495814)(876.042269,574.16496298)
\curveto(876.07226832,574.28495797)(876.0972683,574.40495785)(876.117269,574.52496298)
\curveto(876.14726825,574.64495761)(876.18726821,574.75995749)(876.237269,574.86996298)
\curveto(876.38726801,575.23995701)(876.56726783,575.56995668)(876.777269,575.85996298)
\curveto(876.9972674,576.15995609)(877.26226713,576.40995584)(877.572269,576.60996298)
\curveto(877.6922667,576.68995556)(877.81726658,576.7549555)(877.947269,576.80496298)
\curveto(878.07726632,576.86495539)(878.21226618,576.92495533)(878.352269,576.98496298)
\curveto(878.47226592,577.03495522)(878.60226579,577.06495519)(878.742269,577.07496298)
\curveto(878.88226551,577.09495516)(879.02226537,577.12495513)(879.162269,577.16496298)
\lineto(879.357269,577.16496298)
\curveto(879.42726497,577.17495508)(879.4922649,577.18495507)(879.552269,577.19496298)
\curveto(880.44226395,577.20495505)(881.18226321,577.01995523)(881.772269,576.63996298)
\curveto(882.36226203,576.25995599)(882.78726161,575.76495649)(883.047269,575.15496298)
\curveto(883.0972613,575.0549572)(883.13726126,574.9549573)(883.167269,574.85496298)
\curveto(883.1972612,574.7549575)(883.23226116,574.6499576)(883.272269,574.53996298)
\curveto(883.30226109,574.42995782)(883.32726107,574.30995794)(883.347269,574.17996298)
\curveto(883.36726103,574.05995819)(883.392261,573.93495832)(883.422269,573.80496298)
\curveto(883.43226096,573.7549585)(883.43226096,573.69995855)(883.422269,573.63996298)
\curveto(883.42226097,573.58995866)(883.42726097,573.53995871)(883.437269,573.48996298)
\moveto(882.102269,572.63496298)
\curveto(882.12226227,572.70495955)(882.12726227,572.78495947)(882.117269,572.87496298)
\lineto(882.117269,573.12996298)
\curveto(882.11726228,573.51995873)(882.08226231,573.8499584)(882.012269,574.11996298)
\curveto(881.98226241,574.19995805)(881.95726244,574.27995797)(881.937269,574.35996298)
\curveto(881.91726248,574.43995781)(881.8922625,574.51495774)(881.862269,574.58496298)
\curveto(881.58226281,575.23495702)(881.13726326,575.68495657)(880.527269,575.93496298)
\curveto(880.45726394,575.96495629)(880.38226401,575.98495627)(880.302269,575.99496298)
\lineto(880.062269,576.05496298)
\curveto(879.98226441,576.07495618)(879.8972645,576.08495617)(879.807269,576.08496298)
\lineto(879.537269,576.08496298)
\lineto(879.267269,576.03996298)
\curveto(879.16726523,576.01995623)(879.07226532,575.99495626)(878.982269,575.96496298)
\curveto(878.90226549,575.94495631)(878.82226557,575.91495634)(878.742269,575.87496298)
\curveto(878.67226572,575.8549564)(878.60726579,575.82495643)(878.547269,575.78496298)
\curveto(878.48726591,575.74495651)(878.43226596,575.70495655)(878.382269,575.66496298)
\curveto(878.14226625,575.49495676)(877.94726645,575.28995696)(877.797269,575.04996298)
\curveto(877.64726675,574.80995744)(877.51726688,574.52995772)(877.407269,574.20996298)
\curveto(877.37726702,574.10995814)(877.35726704,574.00495825)(877.347269,573.89496298)
\curveto(877.33726706,573.79495846)(877.32226707,573.68995856)(877.302269,573.57996298)
\curveto(877.2922671,573.53995871)(877.28726711,573.47495878)(877.287269,573.38496298)
\curveto(877.27726712,573.3549589)(877.27226712,573.31995893)(877.272269,573.27996298)
\curveto(877.28226711,573.23995901)(877.28726711,573.19495906)(877.287269,573.14496298)
\lineto(877.287269,572.84496298)
\curveto(877.28726711,572.74495951)(877.2972671,572.6549596)(877.317269,572.57496298)
\lineto(877.347269,572.39496298)
\curveto(877.36726703,572.29495996)(877.38226701,572.19496006)(877.392269,572.09496298)
\curveto(877.41226698,572.00496025)(877.44226695,571.91996033)(877.482269,571.83996298)
\curveto(877.58226681,571.59996065)(877.6972667,571.37496088)(877.827269,571.16496298)
\curveto(877.96726643,570.9549613)(878.13726626,570.77996147)(878.337269,570.63996298)
\curveto(878.38726601,570.60996164)(878.43226596,570.58496167)(878.472269,570.56496298)
\curveto(878.51226588,570.54496171)(878.55726584,570.51996173)(878.607269,570.48996298)
\curveto(878.68726571,570.43996181)(878.77226562,570.39496186)(878.862269,570.35496298)
\curveto(878.96226543,570.32496193)(879.06726533,570.29496196)(879.177269,570.26496298)
\curveto(879.22726517,570.24496201)(879.27226512,570.23496202)(879.312269,570.23496298)
\curveto(879.36226503,570.24496201)(879.41226498,570.24496201)(879.462269,570.23496298)
\curveto(879.4922649,570.22496203)(879.55226484,570.21496204)(879.642269,570.20496298)
\curveto(879.74226465,570.19496206)(879.81726458,570.19996205)(879.867269,570.21996298)
\curveto(879.90726449,570.22996202)(879.94726445,570.22996202)(879.987269,570.21996298)
\curveto(880.02726437,570.21996203)(880.06726433,570.22996202)(880.107269,570.24996298)
\curveto(880.18726421,570.26996198)(880.26726413,570.28496197)(880.347269,570.29496298)
\curveto(880.42726397,570.31496194)(880.50226389,570.33996191)(880.572269,570.36996298)
\curveto(880.91226348,570.50996174)(881.18726321,570.70496155)(881.397269,570.95496298)
\curveto(881.60726279,571.20496105)(881.78226261,571.49996075)(881.922269,571.83996298)
\curveto(881.97226242,571.95996029)(882.00226239,572.08496017)(882.012269,572.21496298)
\curveto(882.03226236,572.3549599)(882.06226233,572.49495976)(882.102269,572.63496298)
}
}
{
\newrgbcolor{curcolor}{0 0 0}
\pscustom[linestyle=none,fillstyle=solid,fillcolor=curcolor]
{
\newpath
\moveto(885.49555025,579.96996298)
\curveto(885.62554863,579.96995228)(885.7605485,579.96995228)(885.90055025,579.96996298)
\curveto(886.05054821,579.96995228)(886.1605481,579.93495232)(886.23055025,579.86496298)
\curveto(886.28054798,579.79495246)(886.30554795,579.69995255)(886.30555025,579.57996298)
\curveto(886.31554794,579.46995278)(886.32054794,579.3549529)(886.32055025,579.23496298)
\lineto(886.32055025,577.89996298)
\lineto(886.32055025,571.82496298)
\lineto(886.32055025,570.14496298)
\lineto(886.32055025,569.75496298)
\curveto(886.32054794,569.61496264)(886.29554796,569.50496275)(886.24555025,569.42496298)
\curveto(886.21554804,569.37496288)(886.17054809,569.34496291)(886.11055025,569.33496298)
\curveto(886.0605482,569.32496293)(885.99554826,569.30996294)(885.91555025,569.28996298)
\lineto(885.70555025,569.28996298)
\lineto(885.39055025,569.28996298)
\curveto(885.29054897,569.29996295)(885.21554904,569.33496292)(885.16555025,569.39496298)
\curveto(885.11554914,569.47496278)(885.08554917,569.57496268)(885.07555025,569.69496298)
\lineto(885.07555025,570.06996298)
\lineto(885.07555025,571.44996298)
\lineto(885.07555025,577.68996298)
\lineto(885.07555025,579.15996298)
\curveto(885.07554918,579.26995298)(885.07054919,579.38495287)(885.06055025,579.50496298)
\curveto(885.0605492,579.63495262)(885.08554917,579.73495252)(885.13555025,579.80496298)
\curveto(885.17554908,579.86495239)(885.25054901,579.91495234)(885.36055025,579.95496298)
\curveto(885.38054888,579.96495229)(885.40054886,579.96495229)(885.42055025,579.95496298)
\curveto(885.45054881,579.9549523)(885.47554878,579.95995229)(885.49555025,579.96996298)
}
}
{
\newrgbcolor{curcolor}{0 0 0}
\pscustom[linestyle=none,fillstyle=solid,fillcolor=curcolor]
{
\newpath
\moveto(888.835394,579.96996298)
\curveto(888.96539238,579.96995228)(889.10039225,579.96995228)(889.240394,579.96996298)
\curveto(889.39039196,579.96995228)(889.50039185,579.93495232)(889.570394,579.86496298)
\curveto(889.62039173,579.79495246)(889.6453917,579.69995255)(889.645394,579.57996298)
\curveto(889.65539169,579.46995278)(889.66039169,579.3549529)(889.660394,579.23496298)
\lineto(889.660394,577.89996298)
\lineto(889.660394,571.82496298)
\lineto(889.660394,570.14496298)
\lineto(889.660394,569.75496298)
\curveto(889.66039169,569.61496264)(889.63539171,569.50496275)(889.585394,569.42496298)
\curveto(889.55539179,569.37496288)(889.51039184,569.34496291)(889.450394,569.33496298)
\curveto(889.40039195,569.32496293)(889.33539201,569.30996294)(889.255394,569.28996298)
\lineto(889.045394,569.28996298)
\lineto(888.730394,569.28996298)
\curveto(888.63039272,569.29996295)(888.55539279,569.33496292)(888.505394,569.39496298)
\curveto(888.45539289,569.47496278)(888.42539292,569.57496268)(888.415394,569.69496298)
\lineto(888.415394,570.06996298)
\lineto(888.415394,571.44996298)
\lineto(888.415394,577.68996298)
\lineto(888.415394,579.15996298)
\curveto(888.41539293,579.26995298)(888.41039294,579.38495287)(888.400394,579.50496298)
\curveto(888.40039295,579.63495262)(888.42539292,579.73495252)(888.475394,579.80496298)
\curveto(888.51539283,579.86495239)(888.59039276,579.91495234)(888.700394,579.95496298)
\curveto(888.72039263,579.96495229)(888.74039261,579.96495229)(888.760394,579.95496298)
\curveto(888.79039256,579.9549523)(888.81539253,579.95995229)(888.835394,579.96996298)
}
}
{
\newrgbcolor{curcolor}{0 0 0}
\pscustom[linestyle=none,fillstyle=solid,fillcolor=curcolor]
{
\newpath
\moveto(898.49023775,569.84496298)
\curveto(898.52022992,569.68496257)(898.50522993,569.5499627)(898.44523775,569.43996298)
\curveto(898.38523005,569.33996291)(898.30523013,569.26496299)(898.20523775,569.21496298)
\curveto(898.15523028,569.19496306)(898.10023034,569.18496307)(898.04023775,569.18496298)
\curveto(897.99023045,569.18496307)(897.9352305,569.17496308)(897.87523775,569.15496298)
\curveto(897.65523078,569.10496315)(897.435231,569.11996313)(897.21523775,569.19996298)
\curveto(897.00523143,569.26996298)(896.86023158,569.35996289)(896.78023775,569.46996298)
\curveto(896.73023171,569.53996271)(896.68523175,569.61996263)(896.64523775,569.70996298)
\curveto(896.60523183,569.80996244)(896.55523188,569.88996236)(896.49523775,569.94996298)
\curveto(896.47523196,569.96996228)(896.45023199,569.98996226)(896.42023775,570.00996298)
\curveto(896.40023204,570.02996222)(896.37023207,570.03496222)(896.33023775,570.02496298)
\curveto(896.22023222,569.99496226)(896.11523232,569.93996231)(896.01523775,569.85996298)
\curveto(895.92523251,569.77996247)(895.8352326,569.70996254)(895.74523775,569.64996298)
\curveto(895.61523282,569.56996268)(895.47523296,569.49496276)(895.32523775,569.42496298)
\curveto(895.17523326,569.36496289)(895.01523342,569.30996294)(894.84523775,569.25996298)
\curveto(894.74523369,569.22996302)(894.6352338,569.20996304)(894.51523775,569.19996298)
\curveto(894.40523403,569.18996306)(894.29523414,569.17496308)(894.18523775,569.15496298)
\curveto(894.1352343,569.14496311)(894.09023435,569.13996311)(894.05023775,569.13996298)
\lineto(893.94523775,569.13996298)
\curveto(893.8352346,569.11996313)(893.73023471,569.11996313)(893.63023775,569.13996298)
\lineto(893.49523775,569.13996298)
\curveto(893.44523499,569.1499631)(893.39523504,569.1549631)(893.34523775,569.15496298)
\curveto(893.29523514,569.1549631)(893.25023519,569.16496309)(893.21023775,569.18496298)
\curveto(893.17023527,569.19496306)(893.1352353,569.19996305)(893.10523775,569.19996298)
\curveto(893.08523535,569.18996306)(893.06023538,569.18996306)(893.03023775,569.19996298)
\lineto(892.79023775,569.25996298)
\curveto(892.71023573,569.26996298)(892.6352358,569.28996296)(892.56523775,569.31996298)
\curveto(892.26523617,569.4499628)(892.02023642,569.59496266)(891.83023775,569.75496298)
\curveto(891.65023679,569.92496233)(891.50023694,570.15996209)(891.38023775,570.45996298)
\curveto(891.29023715,570.67996157)(891.24523719,570.94496131)(891.24523775,571.25496298)
\lineto(891.24523775,571.56996298)
\curveto(891.25523718,571.61996063)(891.26023718,571.66996058)(891.26023775,571.71996298)
\lineto(891.29023775,571.89996298)
\lineto(891.41023775,572.22996298)
\curveto(891.45023699,572.33995991)(891.50023694,572.43995981)(891.56023775,572.52996298)
\curveto(891.7402367,572.81995943)(891.98523645,573.03495922)(892.29523775,573.17496298)
\curveto(892.60523583,573.31495894)(892.94523549,573.43995881)(893.31523775,573.54996298)
\curveto(893.45523498,573.58995866)(893.60023484,573.61995863)(893.75023775,573.63996298)
\curveto(893.90023454,573.65995859)(894.05023439,573.68495857)(894.20023775,573.71496298)
\curveto(894.27023417,573.73495852)(894.3352341,573.74495851)(894.39523775,573.74496298)
\curveto(894.46523397,573.74495851)(894.5402339,573.7549585)(894.62023775,573.77496298)
\curveto(894.69023375,573.79495846)(894.76023368,573.80495845)(894.83023775,573.80496298)
\curveto(894.90023354,573.81495844)(894.97523346,573.82995842)(895.05523775,573.84996298)
\curveto(895.30523313,573.90995834)(895.5402329,573.95995829)(895.76023775,573.99996298)
\curveto(895.98023246,574.0499582)(896.15523228,574.16495809)(896.28523775,574.34496298)
\curveto(896.34523209,574.42495783)(896.39523204,574.52495773)(896.43523775,574.64496298)
\curveto(896.47523196,574.77495748)(896.47523196,574.91495734)(896.43523775,575.06496298)
\curveto(896.37523206,575.30495695)(896.28523215,575.49495676)(896.16523775,575.63496298)
\curveto(896.05523238,575.77495648)(895.89523254,575.88495637)(895.68523775,575.96496298)
\curveto(895.56523287,576.01495624)(895.42023302,576.0499562)(895.25023775,576.06996298)
\curveto(895.09023335,576.08995616)(894.92023352,576.09995615)(894.74023775,576.09996298)
\curveto(894.56023388,576.09995615)(894.38523405,576.08995616)(894.21523775,576.06996298)
\curveto(894.04523439,576.0499562)(893.90023454,576.01995623)(893.78023775,575.97996298)
\curveto(893.61023483,575.91995633)(893.44523499,575.83495642)(893.28523775,575.72496298)
\curveto(893.20523523,575.66495659)(893.13023531,575.58495667)(893.06023775,575.48496298)
\curveto(893.00023544,575.39495686)(892.94523549,575.29495696)(892.89523775,575.18496298)
\curveto(892.86523557,575.10495715)(892.8352356,575.01995723)(892.80523775,574.92996298)
\curveto(892.78523565,574.83995741)(892.7402357,574.76995748)(892.67023775,574.71996298)
\curveto(892.63023581,574.68995756)(892.56023588,574.66495759)(892.46023775,574.64496298)
\curveto(892.37023607,574.63495762)(892.27523616,574.62995762)(892.17523775,574.62996298)
\curveto(892.07523636,574.62995762)(891.97523646,574.63495762)(891.87523775,574.64496298)
\curveto(891.78523665,574.66495759)(891.72023672,574.68995756)(891.68023775,574.71996298)
\curveto(891.6402368,574.7499575)(891.61023683,574.79995745)(891.59023775,574.86996298)
\curveto(891.57023687,574.93995731)(891.57023687,575.01495724)(891.59023775,575.09496298)
\curveto(891.62023682,575.22495703)(891.65023679,575.34495691)(891.68023775,575.45496298)
\curveto(891.72023672,575.57495668)(891.76523667,575.68995656)(891.81523775,575.79996298)
\curveto(892.00523643,576.1499561)(892.24523619,576.41995583)(892.53523775,576.60996298)
\curveto(892.82523561,576.80995544)(893.18523525,576.96995528)(893.61523775,577.08996298)
\curveto(893.71523472,577.10995514)(893.81523462,577.12495513)(893.91523775,577.13496298)
\curveto(894.02523441,577.14495511)(894.1352343,577.15995509)(894.24523775,577.17996298)
\curveto(894.28523415,577.18995506)(894.35023409,577.18995506)(894.44023775,577.17996298)
\curveto(894.53023391,577.17995507)(894.58523385,577.18995506)(894.60523775,577.20996298)
\curveto(895.30523313,577.21995503)(895.91523252,577.13995511)(896.43523775,576.96996298)
\curveto(896.95523148,576.79995545)(897.32023112,576.47495578)(897.53023775,575.99496298)
\curveto(897.62023082,575.79495646)(897.67023077,575.55995669)(897.68023775,575.28996298)
\curveto(897.70023074,575.02995722)(897.71023073,574.7549575)(897.71023775,574.46496298)
\lineto(897.71023775,571.14996298)
\curveto(897.71023073,571.00996124)(897.71523072,570.87496138)(897.72523775,570.74496298)
\curveto(897.7352307,570.61496164)(897.76523067,570.50996174)(897.81523775,570.42996298)
\curveto(897.86523057,570.35996189)(897.93023051,570.30996194)(898.01023775,570.27996298)
\curveto(898.10023034,570.23996201)(898.18523025,570.20996204)(898.26523775,570.18996298)
\curveto(898.34523009,570.17996207)(898.40523003,570.13496212)(898.44523775,570.05496298)
\curveto(898.46522997,570.02496223)(898.47522996,569.99496226)(898.47523775,569.96496298)
\curveto(898.47522996,569.93496232)(898.48022996,569.89496236)(898.49023775,569.84496298)
\moveto(896.34523775,571.50996298)
\curveto(896.40523203,571.6499606)(896.435232,571.80996044)(896.43523775,571.98996298)
\curveto(896.44523199,572.17996007)(896.45023199,572.37495988)(896.45023775,572.57496298)
\curveto(896.45023199,572.68495957)(896.44523199,572.78495947)(896.43523775,572.87496298)
\curveto(896.42523201,572.96495929)(896.38523205,573.03495922)(896.31523775,573.08496298)
\curveto(896.28523215,573.10495915)(896.21523222,573.11495914)(896.10523775,573.11496298)
\curveto(896.08523235,573.09495916)(896.05023239,573.08495917)(896.00023775,573.08496298)
\curveto(895.95023249,573.08495917)(895.90523253,573.07495918)(895.86523775,573.05496298)
\curveto(895.78523265,573.03495922)(895.69523274,573.01495924)(895.59523775,572.99496298)
\lineto(895.29523775,572.93496298)
\curveto(895.26523317,572.93495932)(895.23023321,572.92995932)(895.19023775,572.91996298)
\lineto(895.08523775,572.91996298)
\curveto(894.9352335,572.87995937)(894.77023367,572.8549594)(894.59023775,572.84496298)
\curveto(894.42023402,572.84495941)(894.26023418,572.82495943)(894.11023775,572.78496298)
\curveto(894.03023441,572.76495949)(893.95523448,572.74495951)(893.88523775,572.72496298)
\curveto(893.82523461,572.71495954)(893.75523468,572.69995955)(893.67523775,572.67996298)
\curveto(893.51523492,572.62995962)(893.36523507,572.56495969)(893.22523775,572.48496298)
\curveto(893.08523535,572.41495984)(892.96523547,572.32495993)(892.86523775,572.21496298)
\curveto(892.76523567,572.10496015)(892.69023575,571.96996028)(892.64023775,571.80996298)
\curveto(892.59023585,571.65996059)(892.57023587,571.47496078)(892.58023775,571.25496298)
\curveto(892.58023586,571.1549611)(892.59523584,571.05996119)(892.62523775,570.96996298)
\curveto(892.66523577,570.88996136)(892.71023573,570.81496144)(892.76023775,570.74496298)
\curveto(892.8402356,570.63496162)(892.94523549,570.53996171)(893.07523775,570.45996298)
\curveto(893.20523523,570.38996186)(893.34523509,570.32996192)(893.49523775,570.27996298)
\curveto(893.54523489,570.26996198)(893.59523484,570.26496199)(893.64523775,570.26496298)
\curveto(893.69523474,570.26496199)(893.74523469,570.25996199)(893.79523775,570.24996298)
\curveto(893.86523457,570.22996202)(893.95023449,570.21496204)(894.05023775,570.20496298)
\curveto(894.16023428,570.20496205)(894.25023419,570.21496204)(894.32023775,570.23496298)
\curveto(894.38023406,570.254962)(894.440234,570.25996199)(894.50023775,570.24996298)
\curveto(894.56023388,570.249962)(894.62023382,570.25996199)(894.68023775,570.27996298)
\curveto(894.76023368,570.29996195)(894.8352336,570.31496194)(894.90523775,570.32496298)
\curveto(894.98523345,570.33496192)(895.06023338,570.3549619)(895.13023775,570.38496298)
\curveto(895.42023302,570.50496175)(895.66523277,570.6499616)(895.86523775,570.81996298)
\curveto(896.07523236,570.98996126)(896.2352322,571.21996103)(896.34523775,571.50996298)
}
}
{
\newrgbcolor{curcolor}{0 0 0}
\pscustom[linestyle=none,fillstyle=solid,fillcolor=curcolor]
{
\newpath
\moveto(906.62187837,570.09996298)
\lineto(906.62187837,569.70996298)
\curveto(906.6218705,569.58996266)(906.59687052,569.48996276)(906.54687837,569.40996298)
\curveto(906.49687062,569.33996291)(906.41187071,569.29996295)(906.29187837,569.28996298)
\lineto(905.94687837,569.28996298)
\curveto(905.88687123,569.28996296)(905.82687129,569.28496297)(905.76687837,569.27496298)
\curveto(905.7168714,569.27496298)(905.67187145,569.28496297)(905.63187837,569.30496298)
\curveto(905.54187158,569.32496293)(905.48187164,569.36496289)(905.45187837,569.42496298)
\curveto(905.41187171,569.47496278)(905.38687173,569.53496272)(905.37687837,569.60496298)
\curveto(905.37687174,569.67496258)(905.36187176,569.74496251)(905.33187837,569.81496298)
\curveto(905.3218718,569.83496242)(905.30687181,569.8499624)(905.28687837,569.85996298)
\curveto(905.27687184,569.87996237)(905.26187186,569.89996235)(905.24187837,569.91996298)
\curveto(905.14187198,569.92996232)(905.06187206,569.90996234)(905.00187837,569.85996298)
\curveto(904.95187217,569.80996244)(904.89687222,569.75996249)(904.83687837,569.70996298)
\curveto(904.63687248,569.55996269)(904.43687268,569.44496281)(904.23687837,569.36496298)
\curveto(904.05687306,569.28496297)(903.84687327,569.22496303)(903.60687837,569.18496298)
\curveto(903.37687374,569.14496311)(903.13687398,569.12496313)(902.88687837,569.12496298)
\curveto(902.64687447,569.11496314)(902.40687471,569.12996312)(902.16687837,569.16996298)
\curveto(901.92687519,569.19996305)(901.7168754,569.254963)(901.53687837,569.33496298)
\curveto(901.0168761,569.5549627)(900.59687652,569.8499624)(900.27687837,570.21996298)
\curveto(899.95687716,570.59996165)(899.70687741,571.06996118)(899.52687837,571.62996298)
\curveto(899.48687763,571.71996053)(899.45687766,571.80996044)(899.43687837,571.89996298)
\curveto(899.42687769,571.99996025)(899.40687771,572.09996015)(899.37687837,572.19996298)
\curveto(899.36687775,572.24996)(899.36187776,572.29995995)(899.36187837,572.34996298)
\curveto(899.36187776,572.39995985)(899.35687776,572.4499598)(899.34687837,572.49996298)
\curveto(899.32687779,572.5499597)(899.3168778,572.59995965)(899.31687837,572.64996298)
\curveto(899.32687779,572.70995954)(899.32687779,572.76495949)(899.31687837,572.81496298)
\lineto(899.31687837,572.96496298)
\curveto(899.29687782,573.01495924)(899.28687783,573.07995917)(899.28687837,573.15996298)
\curveto(899.28687783,573.23995901)(899.29687782,573.30495895)(899.31687837,573.35496298)
\lineto(899.31687837,573.51996298)
\curveto(899.33687778,573.58995866)(899.34187778,573.65995859)(899.33187837,573.72996298)
\curveto(899.33187779,573.80995844)(899.34187778,573.88495837)(899.36187837,573.95496298)
\curveto(899.37187775,574.00495825)(899.37687774,574.0499582)(899.37687837,574.08996298)
\curveto(899.37687774,574.12995812)(899.38187774,574.17495808)(899.39187837,574.22496298)
\curveto(899.4218777,574.32495793)(899.44687767,574.41995783)(899.46687837,574.50996298)
\curveto(899.48687763,574.60995764)(899.51187761,574.70495755)(899.54187837,574.79496298)
\curveto(899.67187745,575.17495708)(899.83687728,575.51495674)(900.03687837,575.81496298)
\curveto(900.24687687,576.12495613)(900.49687662,576.37995587)(900.78687837,576.57996298)
\curveto(900.95687616,576.69995555)(901.13187599,576.79995545)(901.31187837,576.87996298)
\curveto(901.50187562,576.95995529)(901.70687541,577.02995522)(901.92687837,577.08996298)
\curveto(901.99687512,577.09995515)(902.06187506,577.10995514)(902.12187837,577.11996298)
\curveto(902.19187493,577.12995512)(902.26187486,577.14495511)(902.33187837,577.16496298)
\lineto(902.48187837,577.16496298)
\curveto(902.56187456,577.18495507)(902.67687444,577.19495506)(902.82687837,577.19496298)
\curveto(902.98687413,577.19495506)(903.10687401,577.18495507)(903.18687837,577.16496298)
\curveto(903.22687389,577.1549551)(903.28187384,577.1499551)(903.35187837,577.14996298)
\curveto(903.46187366,577.11995513)(903.57187355,577.09495516)(903.68187837,577.07496298)
\curveto(903.79187333,577.06495519)(903.89687322,577.03495522)(903.99687837,576.98496298)
\curveto(904.14687297,576.92495533)(904.28687283,576.85995539)(904.41687837,576.78996298)
\curveto(904.55687256,576.71995553)(904.68687243,576.63995561)(904.80687837,576.54996298)
\curveto(904.86687225,576.49995575)(904.92687219,576.44495581)(904.98687837,576.38496298)
\curveto(905.05687206,576.33495592)(905.14687197,576.31995593)(905.25687837,576.33996298)
\curveto(905.27687184,576.36995588)(905.29187183,576.39495586)(905.30187837,576.41496298)
\curveto(905.3218718,576.43495582)(905.33687178,576.46495579)(905.34687837,576.50496298)
\curveto(905.37687174,576.59495566)(905.38687173,576.70995554)(905.37687837,576.84996298)
\lineto(905.37687837,577.22496298)
\lineto(905.37687837,578.94996298)
\lineto(905.37687837,579.41496298)
\curveto(905.37687174,579.59495266)(905.40187172,579.72495253)(905.45187837,579.80496298)
\curveto(905.49187163,579.87495238)(905.55187157,579.91995233)(905.63187837,579.93996298)
\curveto(905.65187147,579.93995231)(905.67687144,579.93995231)(905.70687837,579.93996298)
\curveto(905.73687138,579.9499523)(905.76187136,579.9549523)(905.78187837,579.95496298)
\curveto(905.9218712,579.96495229)(906.06687105,579.96495229)(906.21687837,579.95496298)
\curveto(906.37687074,579.9549523)(906.48687063,579.91495234)(906.54687837,579.83496298)
\curveto(906.59687052,579.7549525)(906.6218705,579.6549526)(906.62187837,579.53496298)
\lineto(906.62187837,579.15996298)
\lineto(906.62187837,570.09996298)
\moveto(905.40687837,572.93496298)
\curveto(905.42687169,572.98495927)(905.43687168,573.0499592)(905.43687837,573.12996298)
\curveto(905.43687168,573.21995903)(905.42687169,573.28995896)(905.40687837,573.33996298)
\lineto(905.40687837,573.56496298)
\curveto(905.38687173,573.6549586)(905.37187175,573.74495851)(905.36187837,573.83496298)
\curveto(905.35187177,573.93495832)(905.33187179,574.02495823)(905.30187837,574.10496298)
\curveto(905.28187184,574.18495807)(905.26187186,574.25995799)(905.24187837,574.32996298)
\curveto(905.23187189,574.39995785)(905.21187191,574.46995778)(905.18187837,574.53996298)
\curveto(905.06187206,574.83995741)(904.90687221,575.10495715)(904.71687837,575.33496298)
\curveto(904.52687259,575.56495669)(904.28687283,575.74495651)(903.99687837,575.87496298)
\curveto(903.89687322,575.92495633)(903.79187333,575.95995629)(903.68187837,575.97996298)
\curveto(903.58187354,576.00995624)(903.47187365,576.03495622)(903.35187837,576.05496298)
\curveto(903.27187385,576.07495618)(903.18187394,576.08495617)(903.08187837,576.08496298)
\lineto(902.81187837,576.08496298)
\curveto(902.76187436,576.07495618)(902.7168744,576.06495619)(902.67687837,576.05496298)
\lineto(902.54187837,576.05496298)
\curveto(902.46187466,576.03495622)(902.37687474,576.01495624)(902.28687837,575.99496298)
\curveto(902.20687491,575.97495628)(902.12687499,575.9499563)(902.04687837,575.91996298)
\curveto(901.72687539,575.77995647)(901.46687565,575.57495668)(901.26687837,575.30496298)
\curveto(901.07687604,575.04495721)(900.9218762,574.73995751)(900.80187837,574.38996298)
\curveto(900.76187636,574.27995797)(900.73187639,574.16495809)(900.71187837,574.04496298)
\curveto(900.70187642,573.93495832)(900.68687643,573.82495843)(900.66687837,573.71496298)
\curveto(900.66687645,573.67495858)(900.66187646,573.63495862)(900.65187837,573.59496298)
\lineto(900.65187837,573.48996298)
\curveto(900.63187649,573.43995881)(900.6218765,573.38495887)(900.62187837,573.32496298)
\curveto(900.63187649,573.26495899)(900.63687648,573.20995904)(900.63687837,573.15996298)
\lineto(900.63687837,572.82996298)
\curveto(900.63687648,572.72995952)(900.64687647,572.63495962)(900.66687837,572.54496298)
\curveto(900.67687644,572.51495974)(900.68187644,572.46495979)(900.68187837,572.39496298)
\curveto(900.70187642,572.32495993)(900.7168764,572.25496)(900.72687837,572.18496298)
\lineto(900.78687837,571.97496298)
\curveto(900.89687622,571.62496063)(901.04687607,571.32496093)(901.23687837,571.07496298)
\curveto(901.42687569,570.82496143)(901.66687545,570.61996163)(901.95687837,570.45996298)
\curveto(902.04687507,570.40996184)(902.13687498,570.36996188)(902.22687837,570.33996298)
\curveto(902.3168748,570.30996194)(902.4168747,570.27996197)(902.52687837,570.24996298)
\curveto(902.57687454,570.22996202)(902.62687449,570.22496203)(902.67687837,570.23496298)
\curveto(902.73687438,570.24496201)(902.79187433,570.23996201)(902.84187837,570.21996298)
\curveto(902.88187424,570.20996204)(902.9218742,570.20496205)(902.96187837,570.20496298)
\lineto(903.09687837,570.20496298)
\lineto(903.23187837,570.20496298)
\curveto(903.26187386,570.21496204)(903.31187381,570.21996203)(903.38187837,570.21996298)
\curveto(903.46187366,570.23996201)(903.54187358,570.254962)(903.62187837,570.26496298)
\curveto(903.70187342,570.28496197)(903.77687334,570.30996194)(903.84687837,570.33996298)
\curveto(904.17687294,570.47996177)(904.44187268,570.6549616)(904.64187837,570.86496298)
\curveto(904.85187227,571.08496117)(905.02687209,571.35996089)(905.16687837,571.68996298)
\curveto(905.2168719,571.79996045)(905.25187187,571.90996034)(905.27187837,572.01996298)
\curveto(905.29187183,572.12996012)(905.3168718,572.23996001)(905.34687837,572.34996298)
\curveto(905.36687175,572.38995986)(905.37687174,572.42495983)(905.37687837,572.45496298)
\curveto(905.37687174,572.49495976)(905.38187174,572.53495972)(905.39187837,572.57496298)
\curveto(905.40187172,572.63495962)(905.40187172,572.69495956)(905.39187837,572.75496298)
\curveto(905.39187173,572.81495944)(905.39687172,572.87495938)(905.40687837,572.93496298)
}
}
{
\newrgbcolor{curcolor}{0 0 0}
\pscustom[linestyle=none,fillstyle=solid,fillcolor=curcolor]
{
\newpath
\moveto(915.69312837,573.48996298)
\curveto(915.71312031,573.42995882)(915.7231203,573.33495892)(915.72312837,573.20496298)
\curveto(915.7231203,573.08495917)(915.71812031,572.99995925)(915.70812837,572.94996298)
\lineto(915.70812837,572.79996298)
\curveto(915.69812033,572.71995953)(915.68812034,572.64495961)(915.67812837,572.57496298)
\curveto(915.67812035,572.51495974)(915.67312035,572.44495981)(915.66312837,572.36496298)
\curveto(915.64312038,572.30495995)(915.6281204,572.24496001)(915.61812837,572.18496298)
\curveto(915.61812041,572.12496013)(915.60812042,572.06496019)(915.58812837,572.00496298)
\curveto(915.54812048,571.87496038)(915.51312051,571.74496051)(915.48312837,571.61496298)
\curveto(915.45312057,571.48496077)(915.41312061,571.36496089)(915.36312837,571.25496298)
\curveto(915.15312087,570.77496148)(914.87312115,570.36996188)(914.52312837,570.03996298)
\curveto(914.17312185,569.71996253)(913.74312228,569.47496278)(913.23312837,569.30496298)
\curveto(913.1231229,569.26496299)(913.00312302,569.23496302)(912.87312837,569.21496298)
\curveto(912.75312327,569.19496306)(912.6281234,569.17496308)(912.49812837,569.15496298)
\curveto(912.43812359,569.14496311)(912.37312365,569.13996311)(912.30312837,569.13996298)
\curveto(912.24312378,569.12996312)(912.18312384,569.12496313)(912.12312837,569.12496298)
\curveto(912.08312394,569.11496314)(912.023124,569.10996314)(911.94312837,569.10996298)
\curveto(911.87312415,569.10996314)(911.8231242,569.11496314)(911.79312837,569.12496298)
\curveto(911.75312427,569.13496312)(911.71312431,569.13996311)(911.67312837,569.13996298)
\curveto(911.63312439,569.12996312)(911.59812443,569.12996312)(911.56812837,569.13996298)
\lineto(911.47812837,569.13996298)
\lineto(911.11812837,569.18496298)
\curveto(910.97812505,569.22496303)(910.84312518,569.26496299)(910.71312837,569.30496298)
\curveto(910.58312544,569.34496291)(910.45812557,569.38996286)(910.33812837,569.43996298)
\curveto(909.88812614,569.63996261)(909.51812651,569.89996235)(909.22812837,570.21996298)
\curveto(908.93812709,570.53996171)(908.69812733,570.92996132)(908.50812837,571.38996298)
\curveto(908.45812757,571.48996076)(908.41812761,571.58996066)(908.38812837,571.68996298)
\curveto(908.36812766,571.78996046)(908.34812768,571.89496036)(908.32812837,572.00496298)
\curveto(908.30812772,572.04496021)(908.29812773,572.07496018)(908.29812837,572.09496298)
\curveto(908.30812772,572.12496013)(908.30812772,572.15996009)(908.29812837,572.19996298)
\curveto(908.27812775,572.27995997)(908.26312776,572.35995989)(908.25312837,572.43996298)
\curveto(908.25312777,572.52995972)(908.24312778,572.61495964)(908.22312837,572.69496298)
\lineto(908.22312837,572.81496298)
\curveto(908.2231278,572.8549594)(908.21812781,572.89995935)(908.20812837,572.94996298)
\curveto(908.19812783,572.99995925)(908.19312783,573.08495917)(908.19312837,573.20496298)
\curveto(908.19312783,573.33495892)(908.20312782,573.42995882)(908.22312837,573.48996298)
\curveto(908.24312778,573.55995869)(908.24812778,573.62995862)(908.23812837,573.69996298)
\curveto(908.2281278,573.76995848)(908.23312779,573.83995841)(908.25312837,573.90996298)
\curveto(908.26312776,573.95995829)(908.26812776,573.99995825)(908.26812837,574.02996298)
\curveto(908.27812775,574.06995818)(908.28812774,574.11495814)(908.29812837,574.16496298)
\curveto(908.3281277,574.28495797)(908.35312767,574.40495785)(908.37312837,574.52496298)
\curveto(908.40312762,574.64495761)(908.44312758,574.75995749)(908.49312837,574.86996298)
\curveto(908.64312738,575.23995701)(908.8231272,575.56995668)(909.03312837,575.85996298)
\curveto(909.25312677,576.15995609)(909.51812651,576.40995584)(909.82812837,576.60996298)
\curveto(909.94812608,576.68995556)(910.07312595,576.7549555)(910.20312837,576.80496298)
\curveto(910.33312569,576.86495539)(910.46812556,576.92495533)(910.60812837,576.98496298)
\curveto(910.7281253,577.03495522)(910.85812517,577.06495519)(910.99812837,577.07496298)
\curveto(911.13812489,577.09495516)(911.27812475,577.12495513)(911.41812837,577.16496298)
\lineto(911.61312837,577.16496298)
\curveto(911.68312434,577.17495508)(911.74812428,577.18495507)(911.80812837,577.19496298)
\curveto(912.69812333,577.20495505)(913.43812259,577.01995523)(914.02812837,576.63996298)
\curveto(914.61812141,576.25995599)(915.04312098,575.76495649)(915.30312837,575.15496298)
\curveto(915.35312067,575.0549572)(915.39312063,574.9549573)(915.42312837,574.85496298)
\curveto(915.45312057,574.7549575)(915.48812054,574.6499576)(915.52812837,574.53996298)
\curveto(915.55812047,574.42995782)(915.58312044,574.30995794)(915.60312837,574.17996298)
\curveto(915.6231204,574.05995819)(915.64812038,573.93495832)(915.67812837,573.80496298)
\curveto(915.68812034,573.7549585)(915.68812034,573.69995855)(915.67812837,573.63996298)
\curveto(915.67812035,573.58995866)(915.68312034,573.53995871)(915.69312837,573.48996298)
\moveto(914.35812837,572.63496298)
\curveto(914.37812165,572.70495955)(914.38312164,572.78495947)(914.37312837,572.87496298)
\lineto(914.37312837,573.12996298)
\curveto(914.37312165,573.51995873)(914.33812169,573.8499584)(914.26812837,574.11996298)
\curveto(914.23812179,574.19995805)(914.21312181,574.27995797)(914.19312837,574.35996298)
\curveto(914.17312185,574.43995781)(914.14812188,574.51495774)(914.11812837,574.58496298)
\curveto(913.83812219,575.23495702)(913.39312263,575.68495657)(912.78312837,575.93496298)
\curveto(912.71312331,575.96495629)(912.63812339,575.98495627)(912.55812837,575.99496298)
\lineto(912.31812837,576.05496298)
\curveto(912.23812379,576.07495618)(912.15312387,576.08495617)(912.06312837,576.08496298)
\lineto(911.79312837,576.08496298)
\lineto(911.52312837,576.03996298)
\curveto(911.4231246,576.01995623)(911.3281247,575.99495626)(911.23812837,575.96496298)
\curveto(911.15812487,575.94495631)(911.07812495,575.91495634)(910.99812837,575.87496298)
\curveto(910.9281251,575.8549564)(910.86312516,575.82495643)(910.80312837,575.78496298)
\curveto(910.74312528,575.74495651)(910.68812534,575.70495655)(910.63812837,575.66496298)
\curveto(910.39812563,575.49495676)(910.20312582,575.28995696)(910.05312837,575.04996298)
\curveto(909.90312612,574.80995744)(909.77312625,574.52995772)(909.66312837,574.20996298)
\curveto(909.63312639,574.10995814)(909.61312641,574.00495825)(909.60312837,573.89496298)
\curveto(909.59312643,573.79495846)(909.57812645,573.68995856)(909.55812837,573.57996298)
\curveto(909.54812648,573.53995871)(909.54312648,573.47495878)(909.54312837,573.38496298)
\curveto(909.53312649,573.3549589)(909.5281265,573.31995893)(909.52812837,573.27996298)
\curveto(909.53812649,573.23995901)(909.54312648,573.19495906)(909.54312837,573.14496298)
\lineto(909.54312837,572.84496298)
\curveto(909.54312648,572.74495951)(909.55312647,572.6549596)(909.57312837,572.57496298)
\lineto(909.60312837,572.39496298)
\curveto(909.6231264,572.29495996)(909.63812639,572.19496006)(909.64812837,572.09496298)
\curveto(909.66812636,572.00496025)(909.69812633,571.91996033)(909.73812837,571.83996298)
\curveto(909.83812619,571.59996065)(909.95312607,571.37496088)(910.08312837,571.16496298)
\curveto(910.2231258,570.9549613)(910.39312563,570.77996147)(910.59312837,570.63996298)
\curveto(910.64312538,570.60996164)(910.68812534,570.58496167)(910.72812837,570.56496298)
\curveto(910.76812526,570.54496171)(910.81312521,570.51996173)(910.86312837,570.48996298)
\curveto(910.94312508,570.43996181)(911.028125,570.39496186)(911.11812837,570.35496298)
\curveto(911.21812481,570.32496193)(911.3231247,570.29496196)(911.43312837,570.26496298)
\curveto(911.48312454,570.24496201)(911.5281245,570.23496202)(911.56812837,570.23496298)
\curveto(911.61812441,570.24496201)(911.66812436,570.24496201)(911.71812837,570.23496298)
\curveto(911.74812428,570.22496203)(911.80812422,570.21496204)(911.89812837,570.20496298)
\curveto(911.99812403,570.19496206)(912.07312395,570.19996205)(912.12312837,570.21996298)
\curveto(912.16312386,570.22996202)(912.20312382,570.22996202)(912.24312837,570.21996298)
\curveto(912.28312374,570.21996203)(912.3231237,570.22996202)(912.36312837,570.24996298)
\curveto(912.44312358,570.26996198)(912.5231235,570.28496197)(912.60312837,570.29496298)
\curveto(912.68312334,570.31496194)(912.75812327,570.33996191)(912.82812837,570.36996298)
\curveto(913.16812286,570.50996174)(913.44312258,570.70496155)(913.65312837,570.95496298)
\curveto(913.86312216,571.20496105)(914.03812199,571.49996075)(914.17812837,571.83996298)
\curveto(914.2281218,571.95996029)(914.25812177,572.08496017)(914.26812837,572.21496298)
\curveto(914.28812174,572.3549599)(914.31812171,572.49495976)(914.35812837,572.63496298)
}
}
{
\newrgbcolor{curcolor}{0 0 0}
\pscustom[linestyle=none,fillstyle=solid,fillcolor=curcolor]
{
\newpath
\moveto(920.82640962,577.19496298)
\curveto(921.05640483,577.19495506)(921.1864047,577.13495512)(921.21640962,577.01496298)
\curveto(921.24640464,576.90495535)(921.26140463,576.73995551)(921.26140962,576.51996298)
\lineto(921.26140962,576.23496298)
\curveto(921.26140463,576.14495611)(921.23640465,576.06995618)(921.18640962,576.00996298)
\curveto(921.12640476,575.92995632)(921.04140485,575.88495637)(920.93140962,575.87496298)
\curveto(920.82140507,575.87495638)(920.71140518,575.85995639)(920.60140962,575.82996298)
\curveto(920.46140543,575.79995645)(920.32640556,575.76995648)(920.19640962,575.73996298)
\curveto(920.07640581,575.70995654)(919.96140593,575.66995658)(919.85140962,575.61996298)
\curveto(919.56140633,575.48995676)(919.32640656,575.30995694)(919.14640962,575.07996298)
\curveto(918.96640692,574.85995739)(918.81140708,574.60495765)(918.68140962,574.31496298)
\curveto(918.64140725,574.20495805)(918.61140728,574.08995816)(918.59140962,573.96996298)
\curveto(918.57140732,573.85995839)(918.54640734,573.74495851)(918.51640962,573.62496298)
\curveto(918.50640738,573.57495868)(918.50140739,573.52495873)(918.50140962,573.47496298)
\curveto(918.51140738,573.42495883)(918.51140738,573.37495888)(918.50140962,573.32496298)
\curveto(918.47140742,573.20495905)(918.45640743,573.06495919)(918.45640962,572.90496298)
\curveto(918.46640742,572.7549595)(918.47140742,572.60995964)(918.47140962,572.46996298)
\lineto(918.47140962,570.62496298)
\lineto(918.47140962,570.27996298)
\curveto(918.47140742,570.15996209)(918.46640742,570.04496221)(918.45640962,569.93496298)
\curveto(918.44640744,569.82496243)(918.44140745,569.72996252)(918.44140962,569.64996298)
\curveto(918.45140744,569.56996268)(918.43140746,569.49996275)(918.38140962,569.43996298)
\curveto(918.33140756,569.36996288)(918.25140764,569.32996292)(918.14140962,569.31996298)
\curveto(918.04140785,569.30996294)(917.93140796,569.30496295)(917.81140962,569.30496298)
\lineto(917.54140962,569.30496298)
\curveto(917.4914084,569.32496293)(917.44140845,569.33996291)(917.39140962,569.34996298)
\curveto(917.35140854,569.36996288)(917.32140857,569.39496286)(917.30140962,569.42496298)
\curveto(917.25140864,569.49496276)(917.22140867,569.57996267)(917.21140962,569.67996298)
\lineto(917.21140962,570.00996298)
\lineto(917.21140962,571.16496298)
\lineto(917.21140962,575.31996298)
\lineto(917.21140962,576.35496298)
\lineto(917.21140962,576.65496298)
\curveto(917.22140867,576.7549555)(917.25140864,576.83995541)(917.30140962,576.90996298)
\curveto(917.33140856,576.9499553)(917.38140851,576.97995527)(917.45140962,576.99996298)
\curveto(917.53140836,577.01995523)(917.61640827,577.02995522)(917.70640962,577.02996298)
\curveto(917.79640809,577.03995521)(917.886408,577.03995521)(917.97640962,577.02996298)
\curveto(918.06640782,577.01995523)(918.13640775,577.00495525)(918.18640962,576.98496298)
\curveto(918.26640762,576.9549553)(918.31640757,576.89495536)(918.33640962,576.80496298)
\curveto(918.36640752,576.72495553)(918.38140751,576.63495562)(918.38140962,576.53496298)
\lineto(918.38140962,576.23496298)
\curveto(918.38140751,576.13495612)(918.40140749,576.04495621)(918.44140962,575.96496298)
\curveto(918.45140744,575.94495631)(918.46140743,575.92995632)(918.47140962,575.91996298)
\lineto(918.51640962,575.87496298)
\curveto(918.62640726,575.87495638)(918.71640717,575.91995633)(918.78640962,576.00996298)
\curveto(918.85640703,576.10995614)(918.91640697,576.18995606)(918.96640962,576.24996298)
\lineto(919.05640962,576.33996298)
\curveto(919.14640674,576.4499558)(919.27140662,576.56495569)(919.43140962,576.68496298)
\curveto(919.5914063,576.80495545)(919.74140615,576.89495536)(919.88140962,576.95496298)
\curveto(919.97140592,577.00495525)(920.06640582,577.03995521)(920.16640962,577.05996298)
\curveto(920.26640562,577.08995516)(920.37140552,577.11995513)(920.48140962,577.14996298)
\curveto(920.54140535,577.15995509)(920.60140529,577.16495509)(920.66140962,577.16496298)
\curveto(920.72140517,577.17495508)(920.77640511,577.18495507)(920.82640962,577.19496298)
}
}
{
\newrgbcolor{curcolor}{0.40000001 0.40000001 0.40000001}
\pscustom[linestyle=none,fillstyle=solid,fillcolor=curcolor]
{
\newpath
\moveto(812.80437349,579.9999996)
\lineto(827.80437349,579.9999996)
\lineto(827.80437349,564.9999996)
\lineto(812.80437349,564.9999996)
\closepath
}
}
{
\newrgbcolor{curcolor}{0 0 0}
\pscustom[linestyle=none,fillstyle=solid,fillcolor=curcolor]
{
\newpath
\moveto(840.7425815,546.85548544)
\curveto(840.76257195,546.80548469)(840.78757193,546.74548475)(840.8175815,546.67548544)
\curveto(840.84757187,546.60548489)(840.86757185,546.53048497)(840.8775815,546.45048544)
\curveto(840.89757182,546.38048512)(840.89757182,546.31048519)(840.8775815,546.24048544)
\curveto(840.86757185,546.18048532)(840.82757189,546.13548536)(840.7575815,546.10548544)
\curveto(840.70757201,546.08548541)(840.64757207,546.07548542)(840.5775815,546.07548544)
\lineto(840.3675815,546.07548544)
\lineto(839.9175815,546.07548544)
\curveto(839.76757295,546.07548542)(839.64757307,546.1004854)(839.5575815,546.15048544)
\curveto(839.45757326,546.21048529)(839.38257333,546.31548518)(839.3325815,546.46548544)
\curveto(839.29257342,546.61548488)(839.24757347,546.75048475)(839.1975815,546.87048544)
\curveto(839.08757363,547.13048437)(838.98757373,547.4004841)(838.8975815,547.68048544)
\curveto(838.80757391,547.96048354)(838.70757401,548.23548326)(838.5975815,548.50548544)
\curveto(838.56757415,548.5954829)(838.53757418,548.68048282)(838.5075815,548.76048544)
\curveto(838.48757423,548.84048266)(838.45757426,548.91548258)(838.4175815,548.98548544)
\curveto(838.38757433,549.05548244)(838.34257437,549.11548238)(838.2825815,549.16548544)
\curveto(838.22257449,549.21548228)(838.14257457,549.25548224)(838.0425815,549.28548544)
\curveto(837.99257472,549.30548219)(837.93257478,549.31048219)(837.8625815,549.30048544)
\lineto(837.6675815,549.30048544)
\lineto(834.8325815,549.30048544)
\lineto(834.5325815,549.30048544)
\curveto(834.42257829,549.31048219)(834.3175784,549.31048219)(834.2175815,549.30048544)
\curveto(834.1175786,549.29048221)(834.02257869,549.27548222)(833.9325815,549.25548544)
\curveto(833.85257886,549.23548226)(833.79257892,549.1954823)(833.7525815,549.13548544)
\curveto(833.67257904,549.03548246)(833.6125791,548.92048258)(833.5725815,548.79048544)
\curveto(833.54257917,548.67048283)(833.50257921,548.54548295)(833.4525815,548.41548544)
\curveto(833.35257936,548.18548331)(833.25757946,547.94548355)(833.1675815,547.69548544)
\curveto(833.08757963,547.44548405)(832.99757972,547.20548429)(832.8975815,546.97548544)
\curveto(832.87757984,546.91548458)(832.85257986,546.84548465)(832.8225815,546.76548544)
\curveto(832.80257991,546.6954848)(832.77757994,546.62048488)(832.7475815,546.54048544)
\curveto(832.71758,546.46048504)(832.68258003,546.38548511)(832.6425815,546.31548544)
\curveto(832.6125801,546.25548524)(832.57758014,546.21048529)(832.5375815,546.18048544)
\curveto(832.45758026,546.12048538)(832.34758037,546.08548541)(832.2075815,546.07548544)
\lineto(831.7875815,546.07548544)
\lineto(831.5475815,546.07548544)
\curveto(831.47758124,546.08548541)(831.4175813,546.11048539)(831.3675815,546.15048544)
\curveto(831.3175814,546.18048532)(831.28758143,546.22548527)(831.2775815,546.28548544)
\curveto(831.27758144,546.34548515)(831.28258143,546.40548509)(831.2925815,546.46548544)
\curveto(831.3125814,546.53548496)(831.33258138,546.6004849)(831.3525815,546.66048544)
\curveto(831.38258133,546.73048477)(831.40758131,546.78048472)(831.4275815,546.81048544)
\curveto(831.56758115,547.13048437)(831.69258102,547.44548405)(831.8025815,547.75548544)
\curveto(831.9125808,548.07548342)(832.03258068,548.3954831)(832.1625815,548.71548544)
\curveto(832.25258046,548.93548256)(832.33758038,549.15048235)(832.4175815,549.36048544)
\curveto(832.49758022,549.58048192)(832.58258013,549.8004817)(832.6725815,550.02048544)
\curveto(832.97257974,550.74048076)(833.25757946,551.46548003)(833.5275815,552.19548544)
\curveto(833.79757892,552.93547856)(834.08257863,553.67047783)(834.3825815,554.40048544)
\curveto(834.49257822,554.66047684)(834.59257812,554.92547657)(834.6825815,555.19548544)
\curveto(834.78257793,555.46547603)(834.88757783,555.73047577)(834.9975815,555.99048544)
\curveto(835.04757767,556.1004754)(835.09257762,556.22047528)(835.1325815,556.35048544)
\curveto(835.18257753,556.49047501)(835.25257746,556.59047491)(835.3425815,556.65048544)
\curveto(835.38257733,556.69047481)(835.44757727,556.72047478)(835.5375815,556.74048544)
\curveto(835.55757716,556.75047475)(835.57757714,556.75047475)(835.5975815,556.74048544)
\curveto(835.62757709,556.74047476)(835.65257706,556.74547475)(835.6725815,556.75548544)
\curveto(835.85257686,556.75547474)(836.06257665,556.75547474)(836.3025815,556.75548544)
\curveto(836.54257617,556.76547473)(836.717576,556.73047477)(836.8275815,556.65048544)
\curveto(836.90757581,556.59047491)(836.96757575,556.49047501)(837.0075815,556.35048544)
\curveto(837.05757566,556.22047528)(837.10757561,556.1004754)(837.1575815,555.99048544)
\curveto(837.25757546,555.76047574)(837.34757537,555.53047597)(837.4275815,555.30048544)
\curveto(837.50757521,555.07047643)(837.59757512,554.84047666)(837.6975815,554.61048544)
\curveto(837.77757494,554.41047709)(837.85257486,554.20547729)(837.9225815,553.99548544)
\curveto(838.00257471,553.78547771)(838.08757463,553.58047792)(838.1775815,553.38048544)
\curveto(838.47757424,552.65047885)(838.76257395,551.91047959)(839.0325815,551.16048544)
\curveto(839.3125734,550.42048108)(839.60757311,549.68548181)(839.9175815,548.95548544)
\curveto(839.95757276,548.86548263)(839.98757273,548.78048272)(840.0075815,548.70048544)
\curveto(840.03757268,548.62048288)(840.06757265,548.53548296)(840.0975815,548.44548544)
\curveto(840.20757251,548.18548331)(840.3125724,547.92048358)(840.4125815,547.65048544)
\curveto(840.52257219,547.38048412)(840.63257208,547.11548438)(840.7425815,546.85548544)
\moveto(837.5325815,550.50048544)
\curveto(837.62257509,550.53048097)(837.67757504,550.58048092)(837.6975815,550.65048544)
\curveto(837.72757499,550.72048078)(837.73257498,550.7954807)(837.7125815,550.87548544)
\curveto(837.70257501,550.96548053)(837.67757504,551.05048045)(837.6375815,551.13048544)
\curveto(837.60757511,551.22048028)(837.57757514,551.2954802)(837.5475815,551.35548544)
\curveto(837.52757519,551.3954801)(837.5175752,551.43048007)(837.5175815,551.46048544)
\curveto(837.5175752,551.49048001)(837.50757521,551.52547997)(837.4875815,551.56548544)
\lineto(837.3975815,551.80548544)
\curveto(837.37757534,551.8954796)(837.34757537,551.98547951)(837.3075815,552.07548544)
\curveto(837.15757556,552.43547906)(837.02257569,552.8004787)(836.9025815,553.17048544)
\curveto(836.79257592,553.55047795)(836.66257605,553.92047758)(836.5125815,554.28048544)
\curveto(836.46257625,554.39047711)(836.4175763,554.500477)(836.3775815,554.61048544)
\curveto(836.34757637,554.72047678)(836.30757641,554.82547667)(836.2575815,554.92548544)
\curveto(836.23757648,554.97547652)(836.2125765,555.02047648)(836.1825815,555.06048544)
\curveto(836.16257655,555.11047639)(836.1125766,555.13547636)(836.0325815,555.13548544)
\curveto(836.0125767,555.11547638)(835.99257672,555.1004764)(835.9725815,555.09048544)
\curveto(835.95257676,555.08047642)(835.93257678,555.06547643)(835.9125815,555.04548544)
\curveto(835.87257684,554.9954765)(835.84257687,554.94047656)(835.8225815,554.88048544)
\curveto(835.80257691,554.83047667)(835.78257693,554.77547672)(835.7625815,554.71548544)
\curveto(835.712577,554.60547689)(835.67257704,554.495477)(835.6425815,554.38548544)
\curveto(835.6125771,554.27547722)(835.57257714,554.16547733)(835.5225815,554.05548544)
\curveto(835.35257736,553.66547783)(835.20257751,553.27047823)(835.0725815,552.87048544)
\curveto(834.95257776,552.47047903)(834.8125779,552.08047942)(834.6525815,551.70048544)
\lineto(834.5925815,551.55048544)
\curveto(834.58257813,551.50048)(834.56757815,551.45048005)(834.5475815,551.40048544)
\lineto(834.4575815,551.16048544)
\curveto(834.42757829,551.08048042)(834.40257831,551.0004805)(834.3825815,550.92048544)
\curveto(834.36257835,550.87048063)(834.35257836,550.81548068)(834.3525815,550.75548544)
\curveto(834.36257835,550.6954808)(834.37757834,550.64548085)(834.3975815,550.60548544)
\curveto(834.44757827,550.52548097)(834.55257816,550.48048102)(834.7125815,550.47048544)
\lineto(835.1625815,550.47048544)
\lineto(836.7675815,550.47048544)
\curveto(836.87757584,550.47048103)(837.0125757,550.46548103)(837.1725815,550.45548544)
\curveto(837.33257538,550.45548104)(837.45257526,550.47048103)(837.5325815,550.50048544)
}
}
{
\newrgbcolor{curcolor}{0 0 0}
\pscustom[linestyle=none,fillstyle=solid,fillcolor=curcolor]
{
\newpath
\moveto(848.819144,546.88548544)
\lineto(848.819144,546.49548544)
\curveto(848.81913612,546.37548512)(848.79413615,546.27548522)(848.744144,546.19548544)
\curveto(848.69413625,546.12548537)(848.60913633,546.08548541)(848.489144,546.07548544)
\lineto(848.144144,546.07548544)
\curveto(848.08413686,546.07548542)(848.02413692,546.07048543)(847.964144,546.06048544)
\curveto(847.91413703,546.06048544)(847.86913707,546.07048543)(847.829144,546.09048544)
\curveto(847.7391372,546.11048539)(847.67913726,546.15048535)(847.649144,546.21048544)
\curveto(847.60913733,546.26048524)(847.58413736,546.32048518)(847.574144,546.39048544)
\curveto(847.57413737,546.46048504)(847.55913738,546.53048497)(847.529144,546.60048544)
\curveto(847.51913742,546.62048488)(847.50413744,546.63548486)(847.484144,546.64548544)
\curveto(847.47413747,546.66548483)(847.45913748,546.68548481)(847.439144,546.70548544)
\curveto(847.3391376,546.71548478)(847.25913768,546.6954848)(847.199144,546.64548544)
\curveto(847.14913779,546.5954849)(847.09413785,546.54548495)(847.034144,546.49548544)
\curveto(846.83413811,546.34548515)(846.63413831,546.23048527)(846.434144,546.15048544)
\curveto(846.25413869,546.07048543)(846.0441389,546.01048549)(845.804144,545.97048544)
\curveto(845.57413937,545.93048557)(845.33413961,545.91048559)(845.084144,545.91048544)
\curveto(844.8441401,545.9004856)(844.60414034,545.91548558)(844.364144,545.95548544)
\curveto(844.12414082,545.98548551)(843.91414103,546.04048546)(843.734144,546.12048544)
\curveto(843.21414173,546.34048516)(842.79414215,546.63548486)(842.474144,547.00548544)
\curveto(842.15414279,547.38548411)(841.90414304,547.85548364)(841.724144,548.41548544)
\curveto(841.68414326,548.50548299)(841.65414329,548.5954829)(841.634144,548.68548544)
\curveto(841.62414332,548.78548271)(841.60414334,548.88548261)(841.574144,548.98548544)
\curveto(841.56414338,549.03548246)(841.55914338,549.08548241)(841.559144,549.13548544)
\curveto(841.55914338,549.18548231)(841.55414339,549.23548226)(841.544144,549.28548544)
\curveto(841.52414342,549.33548216)(841.51414343,549.38548211)(841.514144,549.43548544)
\curveto(841.52414342,549.495482)(841.52414342,549.55048195)(841.514144,549.60048544)
\lineto(841.514144,549.75048544)
\curveto(841.49414345,549.8004817)(841.48414346,549.86548163)(841.484144,549.94548544)
\curveto(841.48414346,550.02548147)(841.49414345,550.09048141)(841.514144,550.14048544)
\lineto(841.514144,550.30548544)
\curveto(841.53414341,550.37548112)(841.5391434,550.44548105)(841.529144,550.51548544)
\curveto(841.52914341,550.5954809)(841.5391434,550.67048083)(841.559144,550.74048544)
\curveto(841.56914337,550.79048071)(841.57414337,550.83548066)(841.574144,550.87548544)
\curveto(841.57414337,550.91548058)(841.57914336,550.96048054)(841.589144,551.01048544)
\curveto(841.61914332,551.11048039)(841.6441433,551.20548029)(841.664144,551.29548544)
\curveto(841.68414326,551.3954801)(841.70914323,551.49048001)(841.739144,551.58048544)
\curveto(841.86914307,551.96047954)(842.03414291,552.3004792)(842.234144,552.60048544)
\curveto(842.4441425,552.91047859)(842.69414225,553.16547833)(842.984144,553.36548544)
\curveto(843.15414179,553.48547801)(843.32914161,553.58547791)(843.509144,553.66548544)
\curveto(843.69914124,553.74547775)(843.90414104,553.81547768)(844.124144,553.87548544)
\curveto(844.19414075,553.88547761)(844.25914068,553.8954776)(844.319144,553.90548544)
\curveto(844.38914055,553.91547758)(844.45914048,553.93047757)(844.529144,553.95048544)
\lineto(844.679144,553.95048544)
\curveto(844.75914018,553.97047753)(844.87414007,553.98047752)(845.024144,553.98048544)
\curveto(845.18413976,553.98047752)(845.30413964,553.97047753)(845.384144,553.95048544)
\curveto(845.42413952,553.94047756)(845.47913946,553.93547756)(845.549144,553.93548544)
\curveto(845.65913928,553.90547759)(845.76913917,553.88047762)(845.879144,553.86048544)
\curveto(845.98913895,553.85047765)(846.09413885,553.82047768)(846.194144,553.77048544)
\curveto(846.3441386,553.71047779)(846.48413846,553.64547785)(846.614144,553.57548544)
\curveto(846.75413819,553.50547799)(846.88413806,553.42547807)(847.004144,553.33548544)
\curveto(847.06413788,553.28547821)(847.12413782,553.23047827)(847.184144,553.17048544)
\curveto(847.25413769,553.12047838)(847.3441376,553.10547839)(847.454144,553.12548544)
\curveto(847.47413747,553.15547834)(847.48913745,553.18047832)(847.499144,553.20048544)
\curveto(847.51913742,553.22047828)(847.53413741,553.25047825)(847.544144,553.29048544)
\curveto(847.57413737,553.38047812)(847.58413736,553.495478)(847.574144,553.63548544)
\lineto(847.574144,554.01048544)
\lineto(847.574144,555.73548544)
\lineto(847.574144,556.20048544)
\curveto(847.57413737,556.38047512)(847.59913734,556.51047499)(847.649144,556.59048544)
\curveto(847.68913725,556.66047484)(847.74913719,556.70547479)(847.829144,556.72548544)
\curveto(847.84913709,556.72547477)(847.87413707,556.72547477)(847.904144,556.72548544)
\curveto(847.93413701,556.73547476)(847.95913698,556.74047476)(847.979144,556.74048544)
\curveto(848.11913682,556.75047475)(848.26413668,556.75047475)(848.414144,556.74048544)
\curveto(848.57413637,556.74047476)(848.68413626,556.7004748)(848.744144,556.62048544)
\curveto(848.79413615,556.54047496)(848.81913612,556.44047506)(848.819144,556.32048544)
\lineto(848.819144,555.94548544)
\lineto(848.819144,546.88548544)
\moveto(847.604144,549.72048544)
\curveto(847.62413732,549.77048173)(847.63413731,549.83548166)(847.634144,549.91548544)
\curveto(847.63413731,550.00548149)(847.62413732,550.07548142)(847.604144,550.12548544)
\lineto(847.604144,550.35048544)
\curveto(847.58413736,550.44048106)(847.56913737,550.53048097)(847.559144,550.62048544)
\curveto(847.54913739,550.72048078)(847.52913741,550.81048069)(847.499144,550.89048544)
\curveto(847.47913746,550.97048053)(847.45913748,551.04548045)(847.439144,551.11548544)
\curveto(847.42913751,551.18548031)(847.40913753,551.25548024)(847.379144,551.32548544)
\curveto(847.25913768,551.62547987)(847.10413784,551.89047961)(846.914144,552.12048544)
\curveto(846.72413822,552.35047915)(846.48413846,552.53047897)(846.194144,552.66048544)
\curveto(846.09413885,552.71047879)(845.98913895,552.74547875)(845.879144,552.76548544)
\curveto(845.77913916,552.7954787)(845.66913927,552.82047868)(845.549144,552.84048544)
\curveto(845.46913947,552.86047864)(845.37913956,552.87047863)(845.279144,552.87048544)
\lineto(845.009144,552.87048544)
\curveto(844.95913998,552.86047864)(844.91414003,552.85047865)(844.874144,552.84048544)
\lineto(844.739144,552.84048544)
\curveto(844.65914028,552.82047868)(844.57414037,552.8004787)(844.484144,552.78048544)
\curveto(844.40414054,552.76047874)(844.32414062,552.73547876)(844.244144,552.70548544)
\curveto(843.92414102,552.56547893)(843.66414128,552.36047914)(843.464144,552.09048544)
\curveto(843.27414167,551.83047967)(843.11914182,551.52547997)(842.999144,551.17548544)
\curveto(842.95914198,551.06548043)(842.92914201,550.95048055)(842.909144,550.83048544)
\curveto(842.89914204,550.72048078)(842.88414206,550.61048089)(842.864144,550.50048544)
\curveto(842.86414208,550.46048104)(842.85914208,550.42048108)(842.849144,550.38048544)
\lineto(842.849144,550.27548544)
\curveto(842.82914211,550.22548127)(842.81914212,550.17048133)(842.819144,550.11048544)
\curveto(842.82914211,550.05048145)(842.83414211,549.9954815)(842.834144,549.94548544)
\lineto(842.834144,549.61548544)
\curveto(842.83414211,549.51548198)(842.8441421,549.42048208)(842.864144,549.33048544)
\curveto(842.87414207,549.3004822)(842.87914206,549.25048225)(842.879144,549.18048544)
\curveto(842.89914204,549.11048239)(842.91414203,549.04048246)(842.924144,548.97048544)
\lineto(842.984144,548.76048544)
\curveto(843.09414185,548.41048309)(843.2441417,548.11048339)(843.434144,547.86048544)
\curveto(843.62414132,547.61048389)(843.86414108,547.40548409)(844.154144,547.24548544)
\curveto(844.2441407,547.1954843)(844.33414061,547.15548434)(844.424144,547.12548544)
\curveto(844.51414043,547.0954844)(844.61414033,547.06548443)(844.724144,547.03548544)
\curveto(844.77414017,547.01548448)(844.82414012,547.01048449)(844.874144,547.02048544)
\curveto(844.93414001,547.03048447)(844.98913995,547.02548447)(845.039144,547.00548544)
\curveto(845.07913986,546.9954845)(845.11913982,546.99048451)(845.159144,546.99048544)
\lineto(845.294144,546.99048544)
\lineto(845.429144,546.99048544)
\curveto(845.45913948,547.0004845)(845.50913943,547.00548449)(845.579144,547.00548544)
\curveto(845.65913928,547.02548447)(845.7391392,547.04048446)(845.819144,547.05048544)
\curveto(845.89913904,547.07048443)(845.97413897,547.0954844)(846.044144,547.12548544)
\curveto(846.37413857,547.26548423)(846.6391383,547.44048406)(846.839144,547.65048544)
\curveto(847.04913789,547.87048363)(847.22413772,548.14548335)(847.364144,548.47548544)
\curveto(847.41413753,548.58548291)(847.44913749,548.6954828)(847.469144,548.80548544)
\curveto(847.48913745,548.91548258)(847.51413743,549.02548247)(847.544144,549.13548544)
\curveto(847.56413738,549.17548232)(847.57413737,549.21048229)(847.574144,549.24048544)
\curveto(847.57413737,549.28048222)(847.57913736,549.32048218)(847.589144,549.36048544)
\curveto(847.59913734,549.42048208)(847.59913734,549.48048202)(847.589144,549.54048544)
\curveto(847.58913735,549.6004819)(847.59413735,549.66048184)(847.604144,549.72048544)
}
}
{
\newrgbcolor{curcolor}{0 0 0}
\pscustom[linestyle=none,fillstyle=solid,fillcolor=curcolor]
{
\newpath
\moveto(854.455394,553.98048544)
\curveto(854.83538901,553.99047751)(855.15538869,553.95047755)(855.415394,553.86048544)
\curveto(855.68538816,553.77047773)(855.93038792,553.64047786)(856.150394,553.47048544)
\curveto(856.23038762,553.42047808)(856.29538755,553.35047815)(856.345394,553.26048544)
\curveto(856.40538744,553.18047832)(856.47038738,553.10547839)(856.540394,553.03548544)
\curveto(856.56038729,553.01547848)(856.59038726,552.99047851)(856.630394,552.96048544)
\curveto(856.67038718,552.93047857)(856.72038713,552.92047858)(856.780394,552.93048544)
\curveto(856.88038697,552.96047854)(856.96538688,553.02047848)(857.035394,553.11048544)
\curveto(857.11538673,553.21047829)(857.19538665,553.28547821)(857.275394,553.33548544)
\curveto(857.41538643,553.44547805)(857.56038629,553.54047796)(857.710394,553.62048544)
\curveto(857.86038599,553.71047779)(858.02538582,553.78547771)(858.205394,553.84548544)
\curveto(858.28538556,553.87547762)(858.37038548,553.8954776)(858.460394,553.90548544)
\curveto(858.56038529,553.92547757)(858.65538519,553.94547755)(858.745394,553.96548544)
\curveto(858.79538505,553.97547752)(858.84038501,553.98047752)(858.880394,553.98048544)
\lineto(859.030394,553.98048544)
\curveto(859.08038477,554.0004775)(859.1503847,554.00547749)(859.240394,553.99548544)
\curveto(859.33038452,553.9954775)(859.39538445,553.99047751)(859.435394,553.98048544)
\curveto(859.48538436,553.97047753)(859.56038429,553.96547753)(859.660394,553.96548544)
\curveto(859.7503841,553.94547755)(859.83538401,553.92547757)(859.915394,553.90548544)
\curveto(860.00538384,553.8954776)(860.09038376,553.87547762)(860.170394,553.84548544)
\curveto(860.22038363,553.82547767)(860.26538358,553.81047769)(860.305394,553.80048544)
\curveto(860.35538349,553.8004777)(860.40538344,553.79047771)(860.455394,553.77048544)
\curveto(860.95538289,553.55047795)(861.30038255,553.21047829)(861.490394,552.75048544)
\curveto(861.53038232,552.67047883)(861.56038229,552.58047892)(861.580394,552.48048544)
\curveto(861.60038225,552.39047911)(861.62038223,552.29047921)(861.640394,552.18048544)
\curveto(861.66038219,552.15047935)(861.66538218,552.11547938)(861.655394,552.07548544)
\curveto(861.65538219,552.04547945)(861.66038219,552.01547948)(861.670394,551.98548544)
\lineto(861.670394,551.85048544)
\curveto(861.68038217,551.81047969)(861.68038217,551.76547973)(861.670394,551.71548544)
\curveto(861.67038218,551.66547983)(861.67038218,551.61547988)(861.670394,551.56548544)
\lineto(861.670394,550.98048544)
\lineto(861.670394,550.02048544)
\lineto(861.670394,547.17048544)
\curveto(861.67038218,547.01048449)(861.67038218,546.82048468)(861.670394,546.60048544)
\curveto(861.68038217,546.38048512)(861.64038221,546.23548526)(861.550394,546.16548544)
\curveto(861.51038234,546.13548536)(861.4453824,546.11048539)(861.355394,546.09048544)
\curveto(861.26538258,546.08048542)(861.17038268,546.07548542)(861.070394,546.07548544)
\curveto(860.97038288,546.07548542)(860.87038298,546.08048542)(860.770394,546.09048544)
\curveto(860.68038317,546.1004854)(860.61538323,546.12048538)(860.575394,546.15048544)
\curveto(860.51538333,546.18048532)(860.47538337,546.24048526)(860.455394,546.33048544)
\curveto(860.43538341,546.39048511)(860.43038342,546.45048505)(860.440394,546.51048544)
\curveto(860.4503834,546.58048492)(860.4453834,546.64548485)(860.425394,546.70548544)
\curveto(860.41538343,546.75548474)(860.41038344,546.81048469)(860.410394,546.87048544)
\curveto(860.42038343,546.94048456)(860.42538342,547.00548449)(860.425394,547.06548544)
\lineto(860.425394,547.74048544)
\lineto(860.425394,550.60548544)
\curveto(860.42538342,550.93548056)(860.41538343,551.24548025)(860.395394,551.53548544)
\curveto(860.38538346,551.83547966)(860.31538353,552.08547941)(860.185394,552.28548544)
\curveto(860.03538381,552.52547897)(859.80538404,552.7004788)(859.495394,552.81048544)
\curveto(859.43538441,552.83047867)(859.37038448,552.84047866)(859.300394,552.84048544)
\curveto(859.24038461,552.85047865)(859.17538467,552.86547863)(859.105394,552.88548544)
\curveto(859.06538478,552.8954786)(859.00038485,552.8954786)(858.910394,552.88548544)
\curveto(858.82038503,552.88547861)(858.76038509,552.88047862)(858.730394,552.87048544)
\curveto(858.68038517,552.86047864)(858.63038522,552.85547864)(858.580394,552.85548544)
\curveto(858.53038532,552.86547863)(858.48038537,552.86047864)(858.430394,552.84048544)
\curveto(858.29038556,552.81047869)(858.15538569,552.77047873)(858.025394,552.72048544)
\curveto(857.50538634,552.500479)(857.15538669,552.11547938)(856.975394,551.56548544)
\curveto(856.92538692,551.3954801)(856.89538695,551.2004803)(856.885394,550.98048544)
\lineto(856.885394,550.30548544)
\lineto(856.885394,548.34048544)
\lineto(856.885394,546.88548544)
\lineto(856.885394,546.51048544)
\curveto(856.88538696,546.39048511)(856.86038699,546.2954852)(856.810394,546.22548544)
\curveto(856.76038709,546.14548535)(856.67538717,546.1004854)(856.555394,546.09048544)
\curveto(856.43538741,546.08048542)(856.31038754,546.07548542)(856.180394,546.07548544)
\curveto(856.01038784,546.07548542)(855.88538796,546.0954854)(855.805394,546.13548544)
\curveto(855.71538813,546.18548531)(855.66038819,546.26548523)(855.640394,546.37548544)
\curveto(855.63038822,546.495485)(855.62538822,546.62548487)(855.625394,546.76548544)
\lineto(855.625394,548.19048544)
\lineto(855.625394,550.66548544)
\curveto(855.62538822,550.98548051)(855.61538823,551.28048022)(855.595394,551.55048544)
\curveto(855.57538827,551.83047967)(855.50538834,552.07047943)(855.385394,552.27048544)
\curveto(855.27538857,552.45047905)(855.1503887,552.58047892)(855.010394,552.66048544)
\curveto(854.87038898,552.75047875)(854.68038917,552.82047868)(854.440394,552.87048544)
\curveto(854.40038945,552.88047862)(854.35538949,552.88547861)(854.305394,552.88548544)
\lineto(854.170394,552.88548544)
\curveto(853.9503899,552.88547861)(853.75539009,552.86047864)(853.585394,552.81048544)
\curveto(853.42539042,552.76047874)(853.28039057,552.6954788)(853.150394,552.61548544)
\curveto(852.64039121,552.30547919)(852.30039155,551.84047966)(852.130394,551.22048544)
\curveto(852.09039176,551.09048041)(852.07039178,550.94048056)(852.070394,550.77048544)
\curveto(852.08039177,550.61048089)(852.08539176,550.45048105)(852.085394,550.29048544)
\lineto(852.085394,548.59548544)
\lineto(852.085394,546.94548544)
\lineto(852.085394,546.54048544)
\curveto(852.08539176,546.4004851)(852.05539179,546.29048521)(851.995394,546.21048544)
\curveto(851.9453919,546.14048536)(851.87039198,546.1004854)(851.770394,546.09048544)
\curveto(851.67039218,546.08048542)(851.56539228,546.07548542)(851.455394,546.07548544)
\lineto(851.230394,546.07548544)
\curveto(851.17039268,546.0954854)(851.11039274,546.11048539)(851.050394,546.12048544)
\curveto(851.00039285,546.13048537)(850.95539289,546.16048534)(850.915394,546.21048544)
\curveto(850.86539298,546.27048523)(850.84039301,546.34548515)(850.840394,546.43548544)
\lineto(850.840394,546.75048544)
\lineto(850.840394,547.72548544)
\lineto(850.840394,552.01548544)
\lineto(850.840394,553.12548544)
\lineto(850.840394,553.41048544)
\curveto(850.84039301,553.51047799)(850.86039299,553.59047791)(850.900394,553.65048544)
\curveto(850.93039292,553.71047779)(850.97539287,553.75047775)(851.035394,553.77048544)
\curveto(851.11539273,553.8004777)(851.24039261,553.81547768)(851.410394,553.81548544)
\curveto(851.59039226,553.81547768)(851.72039213,553.8004777)(851.800394,553.77048544)
\curveto(851.88039197,553.73047777)(851.93539191,553.68047782)(851.965394,553.62048544)
\curveto(851.98539186,553.57047793)(851.99539185,553.51047799)(851.995394,553.44048544)
\curveto(852.00539184,553.37047813)(852.01539183,553.30547819)(852.025394,553.24548544)
\curveto(852.03539181,553.18547831)(852.05539179,553.13547836)(852.085394,553.09548544)
\curveto(852.11539173,553.05547844)(852.16539168,553.03547846)(852.235394,553.03548544)
\curveto(852.25539159,553.05547844)(852.27539157,553.06547843)(852.295394,553.06548544)
\curveto(852.32539152,553.06547843)(852.3503915,553.07547842)(852.370394,553.09548544)
\curveto(852.43039142,553.14547835)(852.48539136,553.1954783)(852.535394,553.24548544)
\lineto(852.715394,553.39548544)
\curveto(852.93539091,553.55547794)(853.18539066,553.6954778)(853.465394,553.81548544)
\curveto(853.56539028,553.85547764)(853.66539018,553.88047762)(853.765394,553.89048544)
\curveto(853.86538998,553.91047759)(853.97038988,553.93547756)(854.080394,553.96548544)
\lineto(854.260394,553.96548544)
\curveto(854.33038952,553.97547752)(854.39538945,553.98047752)(854.455394,553.98048544)
}
}
{
\newrgbcolor{curcolor}{0 0 0}
\pscustom[linestyle=none,fillstyle=solid,fillcolor=curcolor]
{
\newpath
\moveto(863.85312837,555.30048544)
\curveto(863.77312725,555.36047614)(863.7281273,555.46547603)(863.71812837,555.61548544)
\lineto(863.71812837,556.08048544)
\lineto(863.71812837,556.33548544)
\curveto(863.71812731,556.42547507)(863.73312729,556.500475)(863.76312837,556.56048544)
\curveto(863.80312722,556.64047486)(863.88312714,556.7004748)(864.00312837,556.74048544)
\curveto(864.023127,556.75047475)(864.04312698,556.75047475)(864.06312837,556.74048544)
\curveto(864.09312693,556.74047476)(864.11812691,556.74547475)(864.13812837,556.75548544)
\curveto(864.30812672,556.75547474)(864.46812656,556.75047475)(864.61812837,556.74048544)
\curveto(864.76812626,556.73047477)(864.86812616,556.67047483)(864.91812837,556.56048544)
\curveto(864.94812608,556.500475)(864.96312606,556.42547507)(864.96312837,556.33548544)
\lineto(864.96312837,556.08048544)
\curveto(864.96312606,555.9004756)(864.95812607,555.73047577)(864.94812837,555.57048544)
\curveto(864.94812608,555.41047609)(864.88312614,555.30547619)(864.75312837,555.25548544)
\curveto(864.70312632,555.23547626)(864.64812638,555.22547627)(864.58812837,555.22548544)
\lineto(864.42312837,555.22548544)
\lineto(864.10812837,555.22548544)
\curveto(864.00812702,555.22547627)(863.9231271,555.25047625)(863.85312837,555.30048544)
\moveto(864.96312837,546.79548544)
\lineto(864.96312837,546.48048544)
\curveto(864.97312605,546.38048512)(864.95312607,546.3004852)(864.90312837,546.24048544)
\curveto(864.87312615,546.18048532)(864.8281262,546.14048536)(864.76812837,546.12048544)
\curveto(864.70812632,546.11048539)(864.63812639,546.0954854)(864.55812837,546.07548544)
\lineto(864.33312837,546.07548544)
\curveto(864.20312682,546.07548542)(864.08812694,546.08048542)(863.98812837,546.09048544)
\curveto(863.89812713,546.11048539)(863.8281272,546.16048534)(863.77812837,546.24048544)
\curveto(863.73812729,546.3004852)(863.71812731,546.37548512)(863.71812837,546.46548544)
\lineto(863.71812837,546.75048544)
\lineto(863.71812837,553.09548544)
\lineto(863.71812837,553.41048544)
\curveto(863.71812731,553.52047798)(863.74312728,553.60547789)(863.79312837,553.66548544)
\curveto(863.8231272,553.71547778)(863.86312716,553.74547775)(863.91312837,553.75548544)
\curveto(863.96312706,553.76547773)(864.01812701,553.78047772)(864.07812837,553.80048544)
\curveto(864.09812693,553.8004777)(864.11812691,553.7954777)(864.13812837,553.78548544)
\curveto(864.16812686,553.78547771)(864.19312683,553.79047771)(864.21312837,553.80048544)
\curveto(864.34312668,553.8004777)(864.47312655,553.7954777)(864.60312837,553.78548544)
\curveto(864.74312628,553.78547771)(864.83812619,553.74547775)(864.88812837,553.66548544)
\curveto(864.93812609,553.60547789)(864.96312606,553.52547797)(864.96312837,553.42548544)
\lineto(864.96312837,553.14048544)
\lineto(864.96312837,546.79548544)
}
}
{
\newrgbcolor{curcolor}{0 0 0}
\pscustom[linestyle=none,fillstyle=solid,fillcolor=curcolor]
{
\newpath
\moveto(870.59797212,553.95048544)
\curveto(871.22796689,553.97047753)(871.73296638,553.88547761)(872.11297212,553.69548544)
\curveto(872.49296562,553.50547799)(872.79796532,553.22047828)(873.02797212,552.84048544)
\curveto(873.08796503,552.74047876)(873.13296498,552.63047887)(873.16297212,552.51048544)
\curveto(873.20296491,552.4004791)(873.23796488,552.28547921)(873.26797212,552.16548544)
\curveto(873.3179648,551.97547952)(873.34796477,551.77047973)(873.35797212,551.55048544)
\curveto(873.36796475,551.33048017)(873.37296474,551.10548039)(873.37297212,550.87548544)
\lineto(873.37297212,549.27048544)
\lineto(873.37297212,546.93048544)
\curveto(873.37296474,546.76048474)(873.36796475,546.59048491)(873.35797212,546.42048544)
\curveto(873.35796476,546.25048525)(873.29296482,546.14048536)(873.16297212,546.09048544)
\curveto(873.112965,546.07048543)(873.05796506,546.06048544)(872.99797212,546.06048544)
\curveto(872.94796517,546.05048545)(872.89296522,546.04548545)(872.83297212,546.04548544)
\curveto(872.70296541,546.04548545)(872.57796554,546.05048545)(872.45797212,546.06048544)
\curveto(872.33796578,546.06048544)(872.25296586,546.1004854)(872.20297212,546.18048544)
\curveto(872.15296596,546.25048525)(872.12796599,546.34048516)(872.12797212,546.45048544)
\lineto(872.12797212,546.78048544)
\lineto(872.12797212,548.07048544)
\lineto(872.12797212,550.51548544)
\curveto(872.12796599,550.78548071)(872.12296599,551.05048045)(872.11297212,551.31048544)
\curveto(872.10296601,551.58047992)(872.05796606,551.81047969)(871.97797212,552.00048544)
\curveto(871.89796622,552.2004793)(871.77796634,552.36047914)(871.61797212,552.48048544)
\curveto(871.45796666,552.61047889)(871.27296684,552.71047879)(871.06297212,552.78048544)
\curveto(871.00296711,552.8004787)(870.93796718,552.81047869)(870.86797212,552.81048544)
\curveto(870.80796731,552.82047868)(870.74796737,552.83547866)(870.68797212,552.85548544)
\curveto(870.63796748,552.86547863)(870.55796756,552.86547863)(870.44797212,552.85548544)
\curveto(870.34796777,552.85547864)(870.27796784,552.85047865)(870.23797212,552.84048544)
\curveto(870.19796792,552.82047868)(870.16296795,552.81047869)(870.13297212,552.81048544)
\curveto(870.10296801,552.82047868)(870.06796805,552.82047868)(870.02797212,552.81048544)
\curveto(869.89796822,552.78047872)(869.77296834,552.74547875)(869.65297212,552.70548544)
\curveto(869.54296857,552.67547882)(869.43796868,552.63047887)(869.33797212,552.57048544)
\curveto(869.29796882,552.55047895)(869.26296885,552.53047897)(869.23297212,552.51048544)
\curveto(869.20296891,552.49047901)(869.16796895,552.47047903)(869.12797212,552.45048544)
\curveto(868.77796934,552.2004793)(868.52296959,551.82547967)(868.36297212,551.32548544)
\curveto(868.33296978,551.24548025)(868.3129698,551.16048034)(868.30297212,551.07048544)
\curveto(868.29296982,550.99048051)(868.27796984,550.91048059)(868.25797212,550.83048544)
\curveto(868.23796988,550.78048072)(868.23296988,550.73048077)(868.24297212,550.68048544)
\curveto(868.25296986,550.64048086)(868.24796987,550.6004809)(868.22797212,550.56048544)
\lineto(868.22797212,550.24548544)
\curveto(868.2179699,550.21548128)(868.2129699,550.18048132)(868.21297212,550.14048544)
\curveto(868.22296989,550.1004814)(868.22796989,550.05548144)(868.22797212,550.00548544)
\lineto(868.22797212,549.55548544)
\lineto(868.22797212,548.11548544)
\lineto(868.22797212,546.79548544)
\lineto(868.22797212,546.45048544)
\curveto(868.22796989,546.34048516)(868.20296991,546.25048525)(868.15297212,546.18048544)
\curveto(868.10297001,546.1004854)(868.0129701,546.06048544)(867.88297212,546.06048544)
\curveto(867.76297035,546.05048545)(867.63797048,546.04548545)(867.50797212,546.04548544)
\curveto(867.42797069,546.04548545)(867.35297076,546.05048545)(867.28297212,546.06048544)
\curveto(867.2129709,546.07048543)(867.15297096,546.0954854)(867.10297212,546.13548544)
\curveto(867.02297109,546.18548531)(866.98297113,546.28048522)(866.98297212,546.42048544)
\lineto(866.98297212,546.82548544)
\lineto(866.98297212,548.59548544)
\lineto(866.98297212,552.22548544)
\lineto(866.98297212,553.14048544)
\lineto(866.98297212,553.41048544)
\curveto(866.98297113,553.500478)(867.00297111,553.57047793)(867.04297212,553.62048544)
\curveto(867.07297104,553.68047782)(867.12297099,553.72047778)(867.19297212,553.74048544)
\curveto(867.23297088,553.75047775)(867.28797083,553.76047774)(867.35797212,553.77048544)
\curveto(867.43797068,553.78047772)(867.5179706,553.78547771)(867.59797212,553.78548544)
\curveto(867.67797044,553.78547771)(867.75297036,553.78047772)(867.82297212,553.77048544)
\curveto(867.90297021,553.76047774)(867.95797016,553.74547775)(867.98797212,553.72548544)
\curveto(868.09797002,553.65547784)(868.14796997,553.56547793)(868.13797212,553.45548544)
\curveto(868.12796999,553.35547814)(868.14296997,553.24047826)(868.18297212,553.11048544)
\curveto(868.20296991,553.05047845)(868.24296987,553.0004785)(868.30297212,552.96048544)
\curveto(868.42296969,552.95047855)(868.5179696,552.9954785)(868.58797212,553.09548544)
\curveto(868.66796945,553.1954783)(868.74796937,553.27547822)(868.82797212,553.33548544)
\curveto(868.96796915,553.43547806)(869.10796901,553.52547797)(869.24797212,553.60548544)
\curveto(869.39796872,553.6954778)(869.56796855,553.77047773)(869.75797212,553.83048544)
\curveto(869.83796828,553.86047764)(869.92296819,553.88047762)(870.01297212,553.89048544)
\curveto(870.112968,553.9004776)(870.20796791,553.91547758)(870.29797212,553.93548544)
\curveto(870.34796777,553.94547755)(870.39796772,553.95047755)(870.44797212,553.95048544)
\lineto(870.59797212,553.95048544)
}
}
{
\newrgbcolor{curcolor}{0 0 0}
\pscustom[linestyle=none,fillstyle=solid,fillcolor=curcolor]
{
\newpath
\moveto(875.5425815,555.30048544)
\curveto(875.46258038,555.36047614)(875.41758042,555.46547603)(875.4075815,555.61548544)
\lineto(875.4075815,556.08048544)
\lineto(875.4075815,556.33548544)
\curveto(875.40758043,556.42547507)(875.42258042,556.500475)(875.4525815,556.56048544)
\curveto(875.49258035,556.64047486)(875.57258027,556.7004748)(875.6925815,556.74048544)
\curveto(875.71258013,556.75047475)(875.73258011,556.75047475)(875.7525815,556.74048544)
\curveto(875.78258006,556.74047476)(875.80758003,556.74547475)(875.8275815,556.75548544)
\curveto(875.99757984,556.75547474)(876.15757968,556.75047475)(876.3075815,556.74048544)
\curveto(876.45757938,556.73047477)(876.55757928,556.67047483)(876.6075815,556.56048544)
\curveto(876.6375792,556.500475)(876.65257919,556.42547507)(876.6525815,556.33548544)
\lineto(876.6525815,556.08048544)
\curveto(876.65257919,555.9004756)(876.64757919,555.73047577)(876.6375815,555.57048544)
\curveto(876.6375792,555.41047609)(876.57257927,555.30547619)(876.4425815,555.25548544)
\curveto(876.39257945,555.23547626)(876.3375795,555.22547627)(876.2775815,555.22548544)
\lineto(876.1125815,555.22548544)
\lineto(875.7975815,555.22548544)
\curveto(875.69758014,555.22547627)(875.61258023,555.25047625)(875.5425815,555.30048544)
\moveto(876.6525815,546.79548544)
\lineto(876.6525815,546.48048544)
\curveto(876.66257918,546.38048512)(876.6425792,546.3004852)(876.5925815,546.24048544)
\curveto(876.56257928,546.18048532)(876.51757932,546.14048536)(876.4575815,546.12048544)
\curveto(876.39757944,546.11048539)(876.32757951,546.0954854)(876.2475815,546.07548544)
\lineto(876.0225815,546.07548544)
\curveto(875.89257995,546.07548542)(875.77758006,546.08048542)(875.6775815,546.09048544)
\curveto(875.58758025,546.11048539)(875.51758032,546.16048534)(875.4675815,546.24048544)
\curveto(875.42758041,546.3004852)(875.40758043,546.37548512)(875.4075815,546.46548544)
\lineto(875.4075815,546.75048544)
\lineto(875.4075815,553.09548544)
\lineto(875.4075815,553.41048544)
\curveto(875.40758043,553.52047798)(875.43258041,553.60547789)(875.4825815,553.66548544)
\curveto(875.51258033,553.71547778)(875.55258029,553.74547775)(875.6025815,553.75548544)
\curveto(875.65258019,553.76547773)(875.70758013,553.78047772)(875.7675815,553.80048544)
\curveto(875.78758005,553.8004777)(875.80758003,553.7954777)(875.8275815,553.78548544)
\curveto(875.85757998,553.78547771)(875.88257996,553.79047771)(875.9025815,553.80048544)
\curveto(876.03257981,553.8004777)(876.16257968,553.7954777)(876.2925815,553.78548544)
\curveto(876.43257941,553.78547771)(876.52757931,553.74547775)(876.5775815,553.66548544)
\curveto(876.62757921,553.60547789)(876.65257919,553.52547797)(876.6525815,553.42548544)
\lineto(876.6525815,553.14048544)
\lineto(876.6525815,546.79548544)
}
}
{
\newrgbcolor{curcolor}{0 0 0}
\pscustom[linestyle=none,fillstyle=solid,fillcolor=curcolor]
{
\newpath
\moveto(881.02742525,553.98048544)
\curveto(881.74742118,553.99047751)(882.35242058,553.90547759)(882.84242525,553.72548544)
\curveto(883.3324196,553.55547794)(883.71241922,553.25047825)(883.98242525,552.81048544)
\curveto(884.05241888,552.7004788)(884.10741882,552.58547891)(884.14742525,552.46548544)
\curveto(884.18741874,552.35547914)(884.2274187,552.23047927)(884.26742525,552.09048544)
\curveto(884.28741864,552.02047948)(884.29241864,551.94547955)(884.28242525,551.86548544)
\curveto(884.27241866,551.7954797)(884.25741867,551.74047976)(884.23742525,551.70048544)
\curveto(884.21741871,551.68047982)(884.19241874,551.66047984)(884.16242525,551.64048544)
\curveto(884.1324188,551.63047987)(884.10741882,551.61547988)(884.08742525,551.59548544)
\curveto(884.03741889,551.57547992)(883.98741894,551.57047993)(883.93742525,551.58048544)
\curveto(883.88741904,551.59047991)(883.83741909,551.59047991)(883.78742525,551.58048544)
\curveto(883.70741922,551.56047994)(883.60241933,551.55547994)(883.47242525,551.56548544)
\curveto(883.34241959,551.58547991)(883.25241968,551.61047989)(883.20242525,551.64048544)
\curveto(883.12241981,551.69047981)(883.06741986,551.75547974)(883.03742525,551.83548544)
\curveto(883.01741991,551.92547957)(882.98241995,552.01047949)(882.93242525,552.09048544)
\curveto(882.84242009,552.25047925)(882.71742021,552.3954791)(882.55742525,552.52548544)
\curveto(882.44742048,552.60547889)(882.3274206,552.66547883)(882.19742525,552.70548544)
\curveto(882.06742086,552.74547875)(881.927421,552.78547871)(881.77742525,552.82548544)
\curveto(881.7274212,552.84547865)(881.67742125,552.85047865)(881.62742525,552.84048544)
\curveto(881.57742135,552.84047866)(881.5274214,552.84547865)(881.47742525,552.85548544)
\curveto(881.41742151,552.87547862)(881.34242159,552.88547861)(881.25242525,552.88548544)
\curveto(881.16242177,552.88547861)(881.08742184,552.87547862)(881.02742525,552.85548544)
\lineto(880.93742525,552.85548544)
\lineto(880.78742525,552.82548544)
\curveto(880.73742219,552.82547867)(880.68742224,552.82047868)(880.63742525,552.81048544)
\curveto(880.37742255,552.75047875)(880.16242277,552.66547883)(879.99242525,552.55548544)
\curveto(879.82242311,552.44547905)(879.70742322,552.26047924)(879.64742525,552.00048544)
\curveto(879.6274233,551.93047957)(879.62242331,551.86047964)(879.63242525,551.79048544)
\curveto(879.65242328,551.72047978)(879.67242326,551.66047984)(879.69242525,551.61048544)
\curveto(879.75242318,551.46048004)(879.82242311,551.35048015)(879.90242525,551.28048544)
\curveto(879.99242294,551.22048028)(880.10242283,551.15048035)(880.23242525,551.07048544)
\curveto(880.39242254,550.97048053)(880.57242236,550.8954806)(880.77242525,550.84548544)
\curveto(880.97242196,550.80548069)(881.17242176,550.75548074)(881.37242525,550.69548544)
\curveto(881.50242143,550.65548084)(881.6324213,550.62548087)(881.76242525,550.60548544)
\curveto(881.89242104,550.58548091)(882.02242091,550.55548094)(882.15242525,550.51548544)
\curveto(882.36242057,550.45548104)(882.56742036,550.3954811)(882.76742525,550.33548544)
\curveto(882.96741996,550.28548121)(883.16741976,550.22048128)(883.36742525,550.14048544)
\lineto(883.51742525,550.08048544)
\curveto(883.56741936,550.06048144)(883.61741931,550.03548146)(883.66742525,550.00548544)
\curveto(883.86741906,549.88548161)(884.04241889,549.75048175)(884.19242525,549.60048544)
\curveto(884.34241859,549.45048205)(884.46741846,549.26048224)(884.56742525,549.03048544)
\curveto(884.58741834,548.96048254)(884.60741832,548.86548263)(884.62742525,548.74548544)
\curveto(884.64741828,548.67548282)(884.65741827,548.6004829)(884.65742525,548.52048544)
\curveto(884.66741826,548.45048305)(884.67241826,548.37048313)(884.67242525,548.28048544)
\lineto(884.67242525,548.13048544)
\curveto(884.65241828,548.06048344)(884.64241829,547.99048351)(884.64242525,547.92048544)
\curveto(884.64241829,547.85048365)(884.6324183,547.78048372)(884.61242525,547.71048544)
\curveto(884.58241835,547.6004839)(884.54741838,547.495484)(884.50742525,547.39548544)
\curveto(884.46741846,547.2954842)(884.42241851,547.20548429)(884.37242525,547.12548544)
\curveto(884.21241872,546.86548463)(884.00741892,546.65548484)(883.75742525,546.49548544)
\curveto(883.50741942,546.34548515)(883.2274197,546.21548528)(882.91742525,546.10548544)
\curveto(882.8274201,546.07548542)(882.7324202,546.05548544)(882.63242525,546.04548544)
\curveto(882.54242039,546.02548547)(882.45242048,546.0004855)(882.36242525,545.97048544)
\curveto(882.26242067,545.95048555)(882.16242077,545.94048556)(882.06242525,545.94048544)
\curveto(881.96242097,545.94048556)(881.86242107,545.93048557)(881.76242525,545.91048544)
\lineto(881.61242525,545.91048544)
\curveto(881.56242137,545.9004856)(881.49242144,545.8954856)(881.40242525,545.89548544)
\curveto(881.31242162,545.8954856)(881.24242169,545.9004856)(881.19242525,545.91048544)
\lineto(881.02742525,545.91048544)
\curveto(880.96742196,545.93048557)(880.90242203,545.94048556)(880.83242525,545.94048544)
\curveto(880.76242217,545.93048557)(880.70242223,545.93548556)(880.65242525,545.95548544)
\curveto(880.60242233,545.96548553)(880.53742239,545.97048553)(880.45742525,545.97048544)
\lineto(880.21742525,546.03048544)
\curveto(880.14742278,546.04048546)(880.07242286,546.06048544)(879.99242525,546.09048544)
\curveto(879.68242325,546.19048531)(879.41242352,546.31548518)(879.18242525,546.46548544)
\curveto(878.95242398,546.61548488)(878.75242418,546.81048469)(878.58242525,547.05048544)
\curveto(878.49242444,547.18048432)(878.41742451,547.31548418)(878.35742525,547.45548544)
\curveto(878.29742463,547.5954839)(878.24242469,547.75048375)(878.19242525,547.92048544)
\curveto(878.17242476,547.98048352)(878.16242477,548.05048345)(878.16242525,548.13048544)
\curveto(878.17242476,548.22048328)(878.18742474,548.29048321)(878.20742525,548.34048544)
\curveto(878.23742469,548.38048312)(878.28742464,548.42048308)(878.35742525,548.46048544)
\curveto(878.40742452,548.48048302)(878.47742445,548.49048301)(878.56742525,548.49048544)
\curveto(878.65742427,548.500483)(878.74742418,548.500483)(878.83742525,548.49048544)
\curveto(878.927424,548.48048302)(879.01242392,548.46548303)(879.09242525,548.44548544)
\curveto(879.18242375,548.43548306)(879.24242369,548.42048308)(879.27242525,548.40048544)
\curveto(879.34242359,548.35048315)(879.38742354,548.27548322)(879.40742525,548.17548544)
\curveto(879.43742349,548.08548341)(879.47242346,548.0004835)(879.51242525,547.92048544)
\curveto(879.61242332,547.7004838)(879.74742318,547.53048397)(879.91742525,547.41048544)
\curveto(880.03742289,547.32048418)(880.17242276,547.25048425)(880.32242525,547.20048544)
\curveto(880.47242246,547.15048435)(880.6324223,547.1004844)(880.80242525,547.05048544)
\lineto(881.11742525,547.00548544)
\lineto(881.20742525,547.00548544)
\curveto(881.27742165,546.98548451)(881.36742156,546.97548452)(881.47742525,546.97548544)
\curveto(881.59742133,546.97548452)(881.69742123,546.98548451)(881.77742525,547.00548544)
\curveto(881.84742108,547.00548449)(881.90242103,547.01048449)(881.94242525,547.02048544)
\curveto(882.00242093,547.03048447)(882.06242087,547.03548446)(882.12242525,547.03548544)
\curveto(882.18242075,547.04548445)(882.23742069,547.05548444)(882.28742525,547.06548544)
\curveto(882.57742035,547.14548435)(882.80742012,547.25048425)(882.97742525,547.38048544)
\curveto(883.14741978,547.51048399)(883.26741966,547.73048377)(883.33742525,548.04048544)
\curveto(883.35741957,548.09048341)(883.36241957,548.14548335)(883.35242525,548.20548544)
\curveto(883.34241959,548.26548323)(883.3324196,548.31048319)(883.32242525,548.34048544)
\curveto(883.27241966,548.53048297)(883.20241973,548.67048283)(883.11242525,548.76048544)
\curveto(883.02241991,548.86048264)(882.90742002,548.95048255)(882.76742525,549.03048544)
\curveto(882.67742025,549.09048241)(882.57742035,549.14048236)(882.46742525,549.18048544)
\lineto(882.13742525,549.30048544)
\curveto(882.10742082,549.31048219)(882.07742085,549.31548218)(882.04742525,549.31548544)
\curveto(882.0274209,549.31548218)(882.00242093,549.32548217)(881.97242525,549.34548544)
\curveto(881.6324213,549.45548204)(881.27742165,549.53548196)(880.90742525,549.58548544)
\curveto(880.54742238,549.64548185)(880.20742272,549.74048176)(879.88742525,549.87048544)
\curveto(879.78742314,549.91048159)(879.69242324,549.94548155)(879.60242525,549.97548544)
\curveto(879.51242342,550.00548149)(879.4274235,550.04548145)(879.34742525,550.09548544)
\curveto(879.15742377,550.20548129)(878.98242395,550.33048117)(878.82242525,550.47048544)
\curveto(878.66242427,550.61048089)(878.53742439,550.78548071)(878.44742525,550.99548544)
\curveto(878.41742451,551.06548043)(878.39242454,551.13548036)(878.37242525,551.20548544)
\curveto(878.36242457,551.27548022)(878.34742458,551.35048015)(878.32742525,551.43048544)
\curveto(878.29742463,551.55047995)(878.28742464,551.68547981)(878.29742525,551.83548544)
\curveto(878.30742462,551.9954795)(878.32242461,552.13047937)(878.34242525,552.24048544)
\curveto(878.36242457,552.29047921)(878.37242456,552.33047917)(878.37242525,552.36048544)
\curveto(878.38242455,552.4004791)(878.39742453,552.44047906)(878.41742525,552.48048544)
\curveto(878.50742442,552.71047879)(878.6274243,552.91047859)(878.77742525,553.08048544)
\curveto(878.93742399,553.25047825)(879.11742381,553.4004781)(879.31742525,553.53048544)
\curveto(879.46742346,553.62047788)(879.6324233,553.69047781)(879.81242525,553.74048544)
\curveto(879.99242294,553.8004777)(880.18242275,553.85547764)(880.38242525,553.90548544)
\curveto(880.45242248,553.91547758)(880.51742241,553.92547757)(880.57742525,553.93548544)
\curveto(880.64742228,553.94547755)(880.72242221,553.95547754)(880.80242525,553.96548544)
\curveto(880.8324221,553.97547752)(880.87242206,553.97547752)(880.92242525,553.96548544)
\curveto(880.97242196,553.95547754)(881.00742192,553.96047754)(881.02742525,553.98048544)
}
}
{
\newrgbcolor{curcolor}{0 0 0}
\pscustom[linestyle=none,fillstyle=solid,fillcolor=curcolor]
{
\newpath
\moveto(887.04242525,556.14048544)
\curveto(887.19242324,556.14047536)(887.34242309,556.13547536)(887.49242525,556.12548544)
\curveto(887.64242279,556.12547537)(887.74742268,556.08547541)(887.80742525,556.00548544)
\curveto(887.85742257,555.94547555)(887.88242255,555.86047564)(887.88242525,555.75048544)
\curveto(887.89242254,555.65047585)(887.89742253,555.54547595)(887.89742525,555.43548544)
\lineto(887.89742525,554.56548544)
\curveto(887.89742253,554.48547701)(887.89242254,554.4004771)(887.88242525,554.31048544)
\curveto(887.88242255,554.23047727)(887.89242254,554.16047734)(887.91242525,554.10048544)
\curveto(887.95242248,553.96047754)(888.04242239,553.87047763)(888.18242525,553.83048544)
\curveto(888.2324222,553.82047768)(888.27742215,553.81547768)(888.31742525,553.81548544)
\lineto(888.46742525,553.81548544)
\lineto(888.87242525,553.81548544)
\curveto(889.0324214,553.82547767)(889.14742128,553.81547768)(889.21742525,553.78548544)
\curveto(889.30742112,553.72547777)(889.36742106,553.66547783)(889.39742525,553.60548544)
\curveto(889.41742101,553.56547793)(889.427421,553.52047798)(889.42742525,553.47048544)
\lineto(889.42742525,553.32048544)
\curveto(889.427421,553.21047829)(889.42242101,553.10547839)(889.41242525,553.00548544)
\curveto(889.40242103,552.91547858)(889.36742106,552.84547865)(889.30742525,552.79548544)
\curveto(889.24742118,552.74547875)(889.16242127,552.71547878)(889.05242525,552.70548544)
\lineto(888.72242525,552.70548544)
\curveto(888.61242182,552.71547878)(888.50242193,552.72047878)(888.39242525,552.72048544)
\curveto(888.28242215,552.72047878)(888.18742224,552.70547879)(888.10742525,552.67548544)
\curveto(888.03742239,552.64547885)(887.98742244,552.5954789)(887.95742525,552.52548544)
\curveto(887.9274225,552.45547904)(887.90742252,552.37047913)(887.89742525,552.27048544)
\curveto(887.88742254,552.18047932)(887.88242255,552.08047942)(887.88242525,551.97048544)
\curveto(887.89242254,551.87047963)(887.89742253,551.77047973)(887.89742525,551.67048544)
\lineto(887.89742525,548.70048544)
\curveto(887.89742253,548.48048302)(887.89242254,548.24548325)(887.88242525,547.99548544)
\curveto(887.88242255,547.75548374)(887.9274225,547.57048393)(888.01742525,547.44048544)
\curveto(888.06742236,547.36048414)(888.1324223,547.30548419)(888.21242525,547.27548544)
\curveto(888.29242214,547.24548425)(888.38742204,547.22048428)(888.49742525,547.20048544)
\curveto(888.5274219,547.19048431)(888.55742187,547.18548431)(888.58742525,547.18548544)
\curveto(888.6274218,547.1954843)(888.66242177,547.1954843)(888.69242525,547.18548544)
\lineto(888.88742525,547.18548544)
\curveto(888.98742144,547.18548431)(889.07742135,547.17548432)(889.15742525,547.15548544)
\curveto(889.24742118,547.14548435)(889.31242112,547.11048439)(889.35242525,547.05048544)
\curveto(889.37242106,547.02048448)(889.38742104,546.96548453)(889.39742525,546.88548544)
\curveto(889.41742101,546.81548468)(889.427421,546.74048476)(889.42742525,546.66048544)
\curveto(889.43742099,546.58048492)(889.43742099,546.500485)(889.42742525,546.42048544)
\curveto(889.41742101,546.35048515)(889.39742103,546.2954852)(889.36742525,546.25548544)
\curveto(889.3274211,546.18548531)(889.25242118,546.13548536)(889.14242525,546.10548544)
\curveto(889.06242137,546.08548541)(888.97242146,546.07548542)(888.87242525,546.07548544)
\curveto(888.77242166,546.08548541)(888.68242175,546.09048541)(888.60242525,546.09048544)
\curveto(888.54242189,546.09048541)(888.48242195,546.08548541)(888.42242525,546.07548544)
\curveto(888.36242207,546.07548542)(888.30742212,546.08048542)(888.25742525,546.09048544)
\lineto(888.07742525,546.09048544)
\curveto(888.0274224,546.1004854)(887.97742245,546.10548539)(887.92742525,546.10548544)
\curveto(887.88742254,546.11548538)(887.84242259,546.12048538)(887.79242525,546.12048544)
\curveto(887.59242284,546.17048533)(887.41742301,546.22548527)(887.26742525,546.28548544)
\curveto(887.1274233,546.34548515)(887.00742342,546.45048505)(886.90742525,546.60048544)
\curveto(886.76742366,546.8004847)(886.68742374,547.05048445)(886.66742525,547.35048544)
\curveto(886.64742378,547.66048384)(886.63742379,547.99048351)(886.63742525,548.34048544)
\lineto(886.63742525,552.27048544)
\curveto(886.60742382,552.4004791)(886.57742385,552.495479)(886.54742525,552.55548544)
\curveto(886.5274239,552.61547888)(886.45742397,552.66547883)(886.33742525,552.70548544)
\curveto(886.29742413,552.71547878)(886.25742417,552.71547878)(886.21742525,552.70548544)
\curveto(886.17742425,552.6954788)(886.13742429,552.7004788)(886.09742525,552.72048544)
\lineto(885.85742525,552.72048544)
\curveto(885.7274247,552.72047878)(885.61742481,552.73047877)(885.52742525,552.75048544)
\curveto(885.44742498,552.78047872)(885.39242504,552.84047866)(885.36242525,552.93048544)
\curveto(885.34242509,552.97047853)(885.3274251,553.01547848)(885.31742525,553.06548544)
\lineto(885.31742525,553.21548544)
\curveto(885.31742511,553.35547814)(885.3274251,553.47047803)(885.34742525,553.56048544)
\curveto(885.36742506,553.66047784)(885.427425,553.73547776)(885.52742525,553.78548544)
\curveto(885.63742479,553.82547767)(885.77742465,553.83547766)(885.94742525,553.81548544)
\curveto(886.1274243,553.7954777)(886.27742415,553.80547769)(886.39742525,553.84548544)
\curveto(886.48742394,553.8954776)(886.55742387,553.96547753)(886.60742525,554.05548544)
\curveto(886.6274238,554.11547738)(886.63742379,554.19047731)(886.63742525,554.28048544)
\lineto(886.63742525,554.53548544)
\lineto(886.63742525,555.46548544)
\lineto(886.63742525,555.70548544)
\curveto(886.63742379,555.7954757)(886.64742378,555.87047563)(886.66742525,555.93048544)
\curveto(886.70742372,556.01047549)(886.78242365,556.07547542)(886.89242525,556.12548544)
\curveto(886.92242351,556.12547537)(886.94742348,556.12547537)(886.96742525,556.12548544)
\curveto(886.99742343,556.13547536)(887.02242341,556.14047536)(887.04242525,556.14048544)
}
}
{
\newrgbcolor{curcolor}{0 0 0}
\pscustom[linestyle=none,fillstyle=solid,fillcolor=curcolor]
{
\newpath
\moveto(894.45922212,553.98048544)
\curveto(894.68921733,553.98047752)(894.8192172,553.92047758)(894.84922212,553.80048544)
\curveto(894.87921714,553.69047781)(894.89421713,553.52547797)(894.89422212,553.30548544)
\lineto(894.89422212,553.02048544)
\curveto(894.89421713,552.93047857)(894.86921715,552.85547864)(894.81922212,552.79548544)
\curveto(894.75921726,552.71547878)(894.67421735,552.67047883)(894.56422212,552.66048544)
\curveto(894.45421757,552.66047884)(894.34421768,552.64547885)(894.23422212,552.61548544)
\curveto(894.09421793,552.58547891)(893.95921806,552.55547894)(893.82922212,552.52548544)
\curveto(893.70921831,552.495479)(893.59421843,552.45547904)(893.48422212,552.40548544)
\curveto(893.19421883,552.27547922)(892.95921906,552.0954794)(892.77922212,551.86548544)
\curveto(892.59921942,551.64547985)(892.44421958,551.39048011)(892.31422212,551.10048544)
\curveto(892.27421975,550.99048051)(892.24421978,550.87548062)(892.22422212,550.75548544)
\curveto(892.20421982,550.64548085)(892.17921984,550.53048097)(892.14922212,550.41048544)
\curveto(892.13921988,550.36048114)(892.13421989,550.31048119)(892.13422212,550.26048544)
\curveto(892.14421988,550.21048129)(892.14421988,550.16048134)(892.13422212,550.11048544)
\curveto(892.10421992,549.99048151)(892.08921993,549.85048165)(892.08922212,549.69048544)
\curveto(892.09921992,549.54048196)(892.10421992,549.3954821)(892.10422212,549.25548544)
\lineto(892.10422212,547.41048544)
\lineto(892.10422212,547.06548544)
\curveto(892.10421992,546.94548455)(892.09921992,546.83048467)(892.08922212,546.72048544)
\curveto(892.07921994,546.61048489)(892.07421995,546.51548498)(892.07422212,546.43548544)
\curveto(892.08421994,546.35548514)(892.06421996,546.28548521)(892.01422212,546.22548544)
\curveto(891.96422006,546.15548534)(891.88422014,546.11548538)(891.77422212,546.10548544)
\curveto(891.67422035,546.0954854)(891.56422046,546.09048541)(891.44422212,546.09048544)
\lineto(891.17422212,546.09048544)
\curveto(891.1242209,546.11048539)(891.07422095,546.12548537)(891.02422212,546.13548544)
\curveto(890.98422104,546.15548534)(890.95422107,546.18048532)(890.93422212,546.21048544)
\curveto(890.88422114,546.28048522)(890.85422117,546.36548513)(890.84422212,546.46548544)
\lineto(890.84422212,546.79548544)
\lineto(890.84422212,547.95048544)
\lineto(890.84422212,552.10548544)
\lineto(890.84422212,553.14048544)
\lineto(890.84422212,553.44048544)
\curveto(890.85422117,553.54047796)(890.88422114,553.62547787)(890.93422212,553.69548544)
\curveto(890.96422106,553.73547776)(891.01422101,553.76547773)(891.08422212,553.78548544)
\curveto(891.16422086,553.80547769)(891.24922077,553.81547768)(891.33922212,553.81548544)
\curveto(891.42922059,553.82547767)(891.5192205,553.82547767)(891.60922212,553.81548544)
\curveto(891.69922032,553.80547769)(891.76922025,553.79047771)(891.81922212,553.77048544)
\curveto(891.89922012,553.74047776)(891.94922007,553.68047782)(891.96922212,553.59048544)
\curveto(891.99922002,553.51047799)(892.01422001,553.42047808)(892.01422212,553.32048544)
\lineto(892.01422212,553.02048544)
\curveto(892.01422001,552.92047858)(892.03421999,552.83047867)(892.07422212,552.75048544)
\curveto(892.08421994,552.73047877)(892.09421993,552.71547878)(892.10422212,552.70548544)
\lineto(892.14922212,552.66048544)
\curveto(892.25921976,552.66047884)(892.34921967,552.70547879)(892.41922212,552.79548544)
\curveto(892.48921953,552.8954786)(892.54921947,552.97547852)(892.59922212,553.03548544)
\lineto(892.68922212,553.12548544)
\curveto(892.77921924,553.23547826)(892.90421912,553.35047815)(893.06422212,553.47048544)
\curveto(893.2242188,553.59047791)(893.37421865,553.68047782)(893.51422212,553.74048544)
\curveto(893.60421842,553.79047771)(893.69921832,553.82547767)(893.79922212,553.84548544)
\curveto(893.89921812,553.87547762)(894.00421802,553.90547759)(894.11422212,553.93548544)
\curveto(894.17421785,553.94547755)(894.23421779,553.95047755)(894.29422212,553.95048544)
\curveto(894.35421767,553.96047754)(894.40921761,553.97047753)(894.45922212,553.98048544)
}
}
{
\newrgbcolor{curcolor}{0 0 0}
\pscustom[linestyle=none,fillstyle=solid,fillcolor=curcolor]
{
\newpath
\moveto(902.70898775,546.63048544)
\curveto(902.73897992,546.47048503)(902.72397993,546.33548516)(902.66398775,546.22548544)
\curveto(902.60398005,546.12548537)(902.52398013,546.05048545)(902.42398775,546.00048544)
\curveto(902.37398028,545.98048552)(902.31898034,545.97048553)(902.25898775,545.97048544)
\curveto(902.20898045,545.97048553)(902.1539805,545.96048554)(902.09398775,545.94048544)
\curveto(901.87398078,545.89048561)(901.653981,545.90548559)(901.43398775,545.98548544)
\curveto(901.22398143,546.05548544)(901.07898158,546.14548535)(900.99898775,546.25548544)
\curveto(900.94898171,546.32548517)(900.90398175,546.40548509)(900.86398775,546.49548544)
\curveto(900.82398183,546.5954849)(900.77398188,546.67548482)(900.71398775,546.73548544)
\curveto(900.69398196,546.75548474)(900.66898199,546.77548472)(900.63898775,546.79548544)
\curveto(900.61898204,546.81548468)(900.58898207,546.82048468)(900.54898775,546.81048544)
\curveto(900.43898222,546.78048472)(900.33398232,546.72548477)(900.23398775,546.64548544)
\curveto(900.14398251,546.56548493)(900.0539826,546.495485)(899.96398775,546.43548544)
\curveto(899.83398282,546.35548514)(899.69398296,546.28048522)(899.54398775,546.21048544)
\curveto(899.39398326,546.15048535)(899.23398342,546.0954854)(899.06398775,546.04548544)
\curveto(898.96398369,546.01548548)(898.8539838,545.9954855)(898.73398775,545.98548544)
\curveto(898.62398403,545.97548552)(898.51398414,545.96048554)(898.40398775,545.94048544)
\curveto(898.3539843,545.93048557)(898.30898435,545.92548557)(898.26898775,545.92548544)
\lineto(898.16398775,545.92548544)
\curveto(898.0539846,545.90548559)(897.94898471,545.90548559)(897.84898775,545.92548544)
\lineto(897.71398775,545.92548544)
\curveto(897.66398499,545.93548556)(897.61398504,545.94048556)(897.56398775,545.94048544)
\curveto(897.51398514,545.94048556)(897.46898519,545.95048555)(897.42898775,545.97048544)
\curveto(897.38898527,545.98048552)(897.3539853,545.98548551)(897.32398775,545.98548544)
\curveto(897.30398535,545.97548552)(897.27898538,545.97548552)(897.24898775,545.98548544)
\lineto(897.00898775,546.04548544)
\curveto(896.92898573,546.05548544)(896.8539858,546.07548542)(896.78398775,546.10548544)
\curveto(896.48398617,546.23548526)(896.23898642,546.38048512)(896.04898775,546.54048544)
\curveto(895.86898679,546.71048479)(895.71898694,546.94548455)(895.59898775,547.24548544)
\curveto(895.50898715,547.46548403)(895.46398719,547.73048377)(895.46398775,548.04048544)
\lineto(895.46398775,548.35548544)
\curveto(895.47398718,548.40548309)(895.47898718,548.45548304)(895.47898775,548.50548544)
\lineto(895.50898775,548.68548544)
\lineto(895.62898775,549.01548544)
\curveto(895.66898699,549.12548237)(895.71898694,549.22548227)(895.77898775,549.31548544)
\curveto(895.9589867,549.60548189)(896.20398645,549.82048168)(896.51398775,549.96048544)
\curveto(896.82398583,550.1004814)(897.16398549,550.22548127)(897.53398775,550.33548544)
\curveto(897.67398498,550.37548112)(897.81898484,550.40548109)(897.96898775,550.42548544)
\curveto(898.11898454,550.44548105)(898.26898439,550.47048103)(898.41898775,550.50048544)
\curveto(898.48898417,550.52048098)(898.5539841,550.53048097)(898.61398775,550.53048544)
\curveto(898.68398397,550.53048097)(898.7589839,550.54048096)(898.83898775,550.56048544)
\curveto(898.90898375,550.58048092)(898.97898368,550.59048091)(899.04898775,550.59048544)
\curveto(899.11898354,550.6004809)(899.19398346,550.61548088)(899.27398775,550.63548544)
\curveto(899.52398313,550.6954808)(899.7589829,550.74548075)(899.97898775,550.78548544)
\curveto(900.19898246,550.83548066)(900.37398228,550.95048055)(900.50398775,551.13048544)
\curveto(900.56398209,551.21048029)(900.61398204,551.31048019)(900.65398775,551.43048544)
\curveto(900.69398196,551.56047994)(900.69398196,551.7004798)(900.65398775,551.85048544)
\curveto(900.59398206,552.09047941)(900.50398215,552.28047922)(900.38398775,552.42048544)
\curveto(900.27398238,552.56047894)(900.11398254,552.67047883)(899.90398775,552.75048544)
\curveto(899.78398287,552.8004787)(899.63898302,552.83547866)(899.46898775,552.85548544)
\curveto(899.30898335,552.87547862)(899.13898352,552.88547861)(898.95898775,552.88548544)
\curveto(898.77898388,552.88547861)(898.60398405,552.87547862)(898.43398775,552.85548544)
\curveto(898.26398439,552.83547866)(898.11898454,552.80547869)(897.99898775,552.76548544)
\curveto(897.82898483,552.70547879)(897.66398499,552.62047888)(897.50398775,552.51048544)
\curveto(897.42398523,552.45047905)(897.34898531,552.37047913)(897.27898775,552.27048544)
\curveto(897.21898544,552.18047932)(897.16398549,552.08047942)(897.11398775,551.97048544)
\curveto(897.08398557,551.89047961)(897.0539856,551.80547969)(897.02398775,551.71548544)
\curveto(897.00398565,551.62547987)(896.9589857,551.55547994)(896.88898775,551.50548544)
\curveto(896.84898581,551.47548002)(896.77898588,551.45048005)(896.67898775,551.43048544)
\curveto(896.58898607,551.42048008)(896.49398616,551.41548008)(896.39398775,551.41548544)
\curveto(896.29398636,551.41548008)(896.19398646,551.42048008)(896.09398775,551.43048544)
\curveto(896.00398665,551.45048005)(895.93898672,551.47548002)(895.89898775,551.50548544)
\curveto(895.8589868,551.53547996)(895.82898683,551.58547991)(895.80898775,551.65548544)
\curveto(895.78898687,551.72547977)(895.78898687,551.8004797)(895.80898775,551.88048544)
\curveto(895.83898682,552.01047949)(895.86898679,552.13047937)(895.89898775,552.24048544)
\curveto(895.93898672,552.36047914)(895.98398667,552.47547902)(896.03398775,552.58548544)
\curveto(896.22398643,552.93547856)(896.46398619,553.20547829)(896.75398775,553.39548544)
\curveto(897.04398561,553.5954779)(897.40398525,553.75547774)(897.83398775,553.87548544)
\curveto(897.93398472,553.8954776)(898.03398462,553.91047759)(898.13398775,553.92048544)
\curveto(898.24398441,553.93047757)(898.3539843,553.94547755)(898.46398775,553.96548544)
\curveto(898.50398415,553.97547752)(898.56898409,553.97547752)(898.65898775,553.96548544)
\curveto(898.74898391,553.96547753)(898.80398385,553.97547752)(898.82398775,553.99548544)
\curveto(899.52398313,554.00547749)(900.13398252,553.92547757)(900.65398775,553.75548544)
\curveto(901.17398148,553.58547791)(901.53898112,553.26047824)(901.74898775,552.78048544)
\curveto(901.83898082,552.58047892)(901.88898077,552.34547915)(901.89898775,552.07548544)
\curveto(901.91898074,551.81547968)(901.92898073,551.54047996)(901.92898775,551.25048544)
\lineto(901.92898775,547.93548544)
\curveto(901.92898073,547.7954837)(901.93398072,547.66048384)(901.94398775,547.53048544)
\curveto(901.9539807,547.4004841)(901.98398067,547.2954842)(902.03398775,547.21548544)
\curveto(902.08398057,547.14548435)(902.14898051,547.0954844)(902.22898775,547.06548544)
\curveto(902.31898034,547.02548447)(902.40398025,546.9954845)(902.48398775,546.97548544)
\curveto(902.56398009,546.96548453)(902.62398003,546.92048458)(902.66398775,546.84048544)
\curveto(902.68397997,546.81048469)(902.69397996,546.78048472)(902.69398775,546.75048544)
\curveto(902.69397996,546.72048478)(902.69897996,546.68048482)(902.70898775,546.63048544)
\moveto(900.56398775,548.29548544)
\curveto(900.62398203,548.43548306)(900.653982,548.5954829)(900.65398775,548.77548544)
\curveto(900.66398199,548.96548253)(900.66898199,549.16048234)(900.66898775,549.36048544)
\curveto(900.66898199,549.47048203)(900.66398199,549.57048193)(900.65398775,549.66048544)
\curveto(900.64398201,549.75048175)(900.60398205,549.82048168)(900.53398775,549.87048544)
\curveto(900.50398215,549.89048161)(900.43398222,549.9004816)(900.32398775,549.90048544)
\curveto(900.30398235,549.88048162)(900.26898239,549.87048163)(900.21898775,549.87048544)
\curveto(900.16898249,549.87048163)(900.12398253,549.86048164)(900.08398775,549.84048544)
\curveto(900.00398265,549.82048168)(899.91398274,549.8004817)(899.81398775,549.78048544)
\lineto(899.51398775,549.72048544)
\curveto(899.48398317,549.72048178)(899.44898321,549.71548178)(899.40898775,549.70548544)
\lineto(899.30398775,549.70548544)
\curveto(899.1539835,549.66548183)(898.98898367,549.64048186)(898.80898775,549.63048544)
\curveto(898.63898402,549.63048187)(898.47898418,549.61048189)(898.32898775,549.57048544)
\curveto(898.24898441,549.55048195)(898.17398448,549.53048197)(898.10398775,549.51048544)
\curveto(898.04398461,549.500482)(897.97398468,549.48548201)(897.89398775,549.46548544)
\curveto(897.73398492,549.41548208)(897.58398507,549.35048215)(897.44398775,549.27048544)
\curveto(897.30398535,549.2004823)(897.18398547,549.11048239)(897.08398775,549.00048544)
\curveto(896.98398567,548.89048261)(896.90898575,548.75548274)(896.85898775,548.59548544)
\curveto(896.80898585,548.44548305)(896.78898587,548.26048324)(896.79898775,548.04048544)
\curveto(896.79898586,547.94048356)(896.81398584,547.84548365)(896.84398775,547.75548544)
\curveto(896.88398577,547.67548382)(896.92898573,547.6004839)(896.97898775,547.53048544)
\curveto(897.0589856,547.42048408)(897.16398549,547.32548417)(897.29398775,547.24548544)
\curveto(897.42398523,547.17548432)(897.56398509,547.11548438)(897.71398775,547.06548544)
\curveto(897.76398489,547.05548444)(897.81398484,547.05048445)(897.86398775,547.05048544)
\curveto(897.91398474,547.05048445)(897.96398469,547.04548445)(898.01398775,547.03548544)
\curveto(898.08398457,547.01548448)(898.16898449,547.0004845)(898.26898775,546.99048544)
\curveto(898.37898428,546.99048451)(898.46898419,547.0004845)(898.53898775,547.02048544)
\curveto(898.59898406,547.04048446)(898.658984,547.04548445)(898.71898775,547.03548544)
\curveto(898.77898388,547.03548446)(898.83898382,547.04548445)(898.89898775,547.06548544)
\curveto(898.97898368,547.08548441)(899.0539836,547.1004844)(899.12398775,547.11048544)
\curveto(899.20398345,547.12048438)(899.27898338,547.14048436)(899.34898775,547.17048544)
\curveto(899.63898302,547.29048421)(899.88398277,547.43548406)(900.08398775,547.60548544)
\curveto(900.29398236,547.77548372)(900.4539822,548.00548349)(900.56398775,548.29548544)
}
}
{
\newrgbcolor{curcolor}{0 0 0}
\pscustom[linestyle=none,fillstyle=solid,fillcolor=curcolor]
{
\newpath
\moveto(910.84062837,546.88548544)
\lineto(910.84062837,546.49548544)
\curveto(910.8406205,546.37548512)(910.81562052,546.27548522)(910.76562837,546.19548544)
\curveto(910.71562062,546.12548537)(910.63062071,546.08548541)(910.51062837,546.07548544)
\lineto(910.16562837,546.07548544)
\curveto(910.10562123,546.07548542)(910.04562129,546.07048543)(909.98562837,546.06048544)
\curveto(909.9356214,546.06048544)(909.89062145,546.07048543)(909.85062837,546.09048544)
\curveto(909.76062158,546.11048539)(909.70062164,546.15048535)(909.67062837,546.21048544)
\curveto(909.63062171,546.26048524)(909.60562173,546.32048518)(909.59562837,546.39048544)
\curveto(909.59562174,546.46048504)(909.58062176,546.53048497)(909.55062837,546.60048544)
\curveto(909.5406218,546.62048488)(909.52562181,546.63548486)(909.50562837,546.64548544)
\curveto(909.49562184,546.66548483)(909.48062186,546.68548481)(909.46062837,546.70548544)
\curveto(909.36062198,546.71548478)(909.28062206,546.6954848)(909.22062837,546.64548544)
\curveto(909.17062217,546.5954849)(909.11562222,546.54548495)(909.05562837,546.49548544)
\curveto(908.85562248,546.34548515)(908.65562268,546.23048527)(908.45562837,546.15048544)
\curveto(908.27562306,546.07048543)(908.06562327,546.01048549)(907.82562837,545.97048544)
\curveto(907.59562374,545.93048557)(907.35562398,545.91048559)(907.10562837,545.91048544)
\curveto(906.86562447,545.9004856)(906.62562471,545.91548558)(906.38562837,545.95548544)
\curveto(906.14562519,545.98548551)(905.9356254,546.04048546)(905.75562837,546.12048544)
\curveto(905.2356261,546.34048516)(904.81562652,546.63548486)(904.49562837,547.00548544)
\curveto(904.17562716,547.38548411)(903.92562741,547.85548364)(903.74562837,548.41548544)
\curveto(903.70562763,548.50548299)(903.67562766,548.5954829)(903.65562837,548.68548544)
\curveto(903.64562769,548.78548271)(903.62562771,548.88548261)(903.59562837,548.98548544)
\curveto(903.58562775,549.03548246)(903.58062776,549.08548241)(903.58062837,549.13548544)
\curveto(903.58062776,549.18548231)(903.57562776,549.23548226)(903.56562837,549.28548544)
\curveto(903.54562779,549.33548216)(903.5356278,549.38548211)(903.53562837,549.43548544)
\curveto(903.54562779,549.495482)(903.54562779,549.55048195)(903.53562837,549.60048544)
\lineto(903.53562837,549.75048544)
\curveto(903.51562782,549.8004817)(903.50562783,549.86548163)(903.50562837,549.94548544)
\curveto(903.50562783,550.02548147)(903.51562782,550.09048141)(903.53562837,550.14048544)
\lineto(903.53562837,550.30548544)
\curveto(903.55562778,550.37548112)(903.56062778,550.44548105)(903.55062837,550.51548544)
\curveto(903.55062779,550.5954809)(903.56062778,550.67048083)(903.58062837,550.74048544)
\curveto(903.59062775,550.79048071)(903.59562774,550.83548066)(903.59562837,550.87548544)
\curveto(903.59562774,550.91548058)(903.60062774,550.96048054)(903.61062837,551.01048544)
\curveto(903.6406277,551.11048039)(903.66562767,551.20548029)(903.68562837,551.29548544)
\curveto(903.70562763,551.3954801)(903.73062761,551.49048001)(903.76062837,551.58048544)
\curveto(903.89062745,551.96047954)(904.05562728,552.3004792)(904.25562837,552.60048544)
\curveto(904.46562687,552.91047859)(904.71562662,553.16547833)(905.00562837,553.36548544)
\curveto(905.17562616,553.48547801)(905.35062599,553.58547791)(905.53062837,553.66548544)
\curveto(905.72062562,553.74547775)(905.92562541,553.81547768)(906.14562837,553.87548544)
\curveto(906.21562512,553.88547761)(906.28062506,553.8954776)(906.34062837,553.90548544)
\curveto(906.41062493,553.91547758)(906.48062486,553.93047757)(906.55062837,553.95048544)
\lineto(906.70062837,553.95048544)
\curveto(906.78062456,553.97047753)(906.89562444,553.98047752)(907.04562837,553.98048544)
\curveto(907.20562413,553.98047752)(907.32562401,553.97047753)(907.40562837,553.95048544)
\curveto(907.44562389,553.94047756)(907.50062384,553.93547756)(907.57062837,553.93548544)
\curveto(907.68062366,553.90547759)(907.79062355,553.88047762)(907.90062837,553.86048544)
\curveto(908.01062333,553.85047765)(908.11562322,553.82047768)(908.21562837,553.77048544)
\curveto(908.36562297,553.71047779)(908.50562283,553.64547785)(908.63562837,553.57548544)
\curveto(908.77562256,553.50547799)(908.90562243,553.42547807)(909.02562837,553.33548544)
\curveto(909.08562225,553.28547821)(909.14562219,553.23047827)(909.20562837,553.17048544)
\curveto(909.27562206,553.12047838)(909.36562197,553.10547839)(909.47562837,553.12548544)
\curveto(909.49562184,553.15547834)(909.51062183,553.18047832)(909.52062837,553.20048544)
\curveto(909.5406218,553.22047828)(909.55562178,553.25047825)(909.56562837,553.29048544)
\curveto(909.59562174,553.38047812)(909.60562173,553.495478)(909.59562837,553.63548544)
\lineto(909.59562837,554.01048544)
\lineto(909.59562837,555.73548544)
\lineto(909.59562837,556.20048544)
\curveto(909.59562174,556.38047512)(909.62062172,556.51047499)(909.67062837,556.59048544)
\curveto(909.71062163,556.66047484)(909.77062157,556.70547479)(909.85062837,556.72548544)
\curveto(909.87062147,556.72547477)(909.89562144,556.72547477)(909.92562837,556.72548544)
\curveto(909.95562138,556.73547476)(909.98062136,556.74047476)(910.00062837,556.74048544)
\curveto(910.1406212,556.75047475)(910.28562105,556.75047475)(910.43562837,556.74048544)
\curveto(910.59562074,556.74047476)(910.70562063,556.7004748)(910.76562837,556.62048544)
\curveto(910.81562052,556.54047496)(910.8406205,556.44047506)(910.84062837,556.32048544)
\lineto(910.84062837,555.94548544)
\lineto(910.84062837,546.88548544)
\moveto(909.62562837,549.72048544)
\curveto(909.64562169,549.77048173)(909.65562168,549.83548166)(909.65562837,549.91548544)
\curveto(909.65562168,550.00548149)(909.64562169,550.07548142)(909.62562837,550.12548544)
\lineto(909.62562837,550.35048544)
\curveto(909.60562173,550.44048106)(909.59062175,550.53048097)(909.58062837,550.62048544)
\curveto(909.57062177,550.72048078)(909.55062179,550.81048069)(909.52062837,550.89048544)
\curveto(909.50062184,550.97048053)(909.48062186,551.04548045)(909.46062837,551.11548544)
\curveto(909.45062189,551.18548031)(909.43062191,551.25548024)(909.40062837,551.32548544)
\curveto(909.28062206,551.62547987)(909.12562221,551.89047961)(908.93562837,552.12048544)
\curveto(908.74562259,552.35047915)(908.50562283,552.53047897)(908.21562837,552.66048544)
\curveto(908.11562322,552.71047879)(908.01062333,552.74547875)(907.90062837,552.76548544)
\curveto(907.80062354,552.7954787)(907.69062365,552.82047868)(907.57062837,552.84048544)
\curveto(907.49062385,552.86047864)(907.40062394,552.87047863)(907.30062837,552.87048544)
\lineto(907.03062837,552.87048544)
\curveto(906.98062436,552.86047864)(906.9356244,552.85047865)(906.89562837,552.84048544)
\lineto(906.76062837,552.84048544)
\curveto(906.68062466,552.82047868)(906.59562474,552.8004787)(906.50562837,552.78048544)
\curveto(906.42562491,552.76047874)(906.34562499,552.73547876)(906.26562837,552.70548544)
\curveto(905.94562539,552.56547893)(905.68562565,552.36047914)(905.48562837,552.09048544)
\curveto(905.29562604,551.83047967)(905.1406262,551.52547997)(905.02062837,551.17548544)
\curveto(904.98062636,551.06548043)(904.95062639,550.95048055)(904.93062837,550.83048544)
\curveto(904.92062642,550.72048078)(904.90562643,550.61048089)(904.88562837,550.50048544)
\curveto(904.88562645,550.46048104)(904.88062646,550.42048108)(904.87062837,550.38048544)
\lineto(904.87062837,550.27548544)
\curveto(904.85062649,550.22548127)(904.8406265,550.17048133)(904.84062837,550.11048544)
\curveto(904.85062649,550.05048145)(904.85562648,549.9954815)(904.85562837,549.94548544)
\lineto(904.85562837,549.61548544)
\curveto(904.85562648,549.51548198)(904.86562647,549.42048208)(904.88562837,549.33048544)
\curveto(904.89562644,549.3004822)(904.90062644,549.25048225)(904.90062837,549.18048544)
\curveto(904.92062642,549.11048239)(904.9356264,549.04048246)(904.94562837,548.97048544)
\lineto(905.00562837,548.76048544)
\curveto(905.11562622,548.41048309)(905.26562607,548.11048339)(905.45562837,547.86048544)
\curveto(905.64562569,547.61048389)(905.88562545,547.40548409)(906.17562837,547.24548544)
\curveto(906.26562507,547.1954843)(906.35562498,547.15548434)(906.44562837,547.12548544)
\curveto(906.5356248,547.0954844)(906.6356247,547.06548443)(906.74562837,547.03548544)
\curveto(906.79562454,547.01548448)(906.84562449,547.01048449)(906.89562837,547.02048544)
\curveto(906.95562438,547.03048447)(907.01062433,547.02548447)(907.06062837,547.00548544)
\curveto(907.10062424,546.9954845)(907.1406242,546.99048451)(907.18062837,546.99048544)
\lineto(907.31562837,546.99048544)
\lineto(907.45062837,546.99048544)
\curveto(907.48062386,547.0004845)(907.53062381,547.00548449)(907.60062837,547.00548544)
\curveto(907.68062366,547.02548447)(907.76062358,547.04048446)(907.84062837,547.05048544)
\curveto(907.92062342,547.07048443)(907.99562334,547.0954844)(908.06562837,547.12548544)
\curveto(908.39562294,547.26548423)(908.66062268,547.44048406)(908.86062837,547.65048544)
\curveto(909.07062227,547.87048363)(909.24562209,548.14548335)(909.38562837,548.47548544)
\curveto(909.4356219,548.58548291)(909.47062187,548.6954828)(909.49062837,548.80548544)
\curveto(909.51062183,548.91548258)(909.5356218,549.02548247)(909.56562837,549.13548544)
\curveto(909.58562175,549.17548232)(909.59562174,549.21048229)(909.59562837,549.24048544)
\curveto(909.59562174,549.28048222)(909.60062174,549.32048218)(909.61062837,549.36048544)
\curveto(909.62062172,549.42048208)(909.62062172,549.48048202)(909.61062837,549.54048544)
\curveto(909.61062173,549.6004819)(909.61562172,549.66048184)(909.62562837,549.72048544)
}
}
{
\newrgbcolor{curcolor}{0 0 0}
\pscustom[linestyle=none,fillstyle=solid,fillcolor=curcolor]
{
\newpath
\moveto(919.91187837,550.27548544)
\curveto(919.93187031,550.21548128)(919.9418703,550.12048138)(919.94187837,549.99048544)
\curveto(919.9418703,549.87048163)(919.93687031,549.78548171)(919.92687837,549.73548544)
\lineto(919.92687837,549.58548544)
\curveto(919.91687033,549.50548199)(919.90687034,549.43048207)(919.89687837,549.36048544)
\curveto(919.89687035,549.3004822)(919.89187035,549.23048227)(919.88187837,549.15048544)
\curveto(919.86187038,549.09048241)(919.8468704,549.03048247)(919.83687837,548.97048544)
\curveto(919.83687041,548.91048259)(919.82687042,548.85048265)(919.80687837,548.79048544)
\curveto(919.76687048,548.66048284)(919.73187051,548.53048297)(919.70187837,548.40048544)
\curveto(919.67187057,548.27048323)(919.63187061,548.15048335)(919.58187837,548.04048544)
\curveto(919.37187087,547.56048394)(919.09187115,547.15548434)(918.74187837,546.82548544)
\curveto(918.39187185,546.50548499)(917.96187228,546.26048524)(917.45187837,546.09048544)
\curveto(917.3418729,546.05048545)(917.22187302,546.02048548)(917.09187837,546.00048544)
\curveto(916.97187327,545.98048552)(916.8468734,545.96048554)(916.71687837,545.94048544)
\curveto(916.65687359,545.93048557)(916.59187365,545.92548557)(916.52187837,545.92548544)
\curveto(916.46187378,545.91548558)(916.40187384,545.91048559)(916.34187837,545.91048544)
\curveto(916.30187394,545.9004856)(916.241874,545.8954856)(916.16187837,545.89548544)
\curveto(916.09187415,545.8954856)(916.0418742,545.9004856)(916.01187837,545.91048544)
\curveto(915.97187427,545.92048558)(915.93187431,545.92548557)(915.89187837,545.92548544)
\curveto(915.85187439,545.91548558)(915.81687443,545.91548558)(915.78687837,545.92548544)
\lineto(915.69687837,545.92548544)
\lineto(915.33687837,545.97048544)
\curveto(915.19687505,546.01048549)(915.06187518,546.05048545)(914.93187837,546.09048544)
\curveto(914.80187544,546.13048537)(914.67687557,546.17548532)(914.55687837,546.22548544)
\curveto(914.10687614,546.42548507)(913.73687651,546.68548481)(913.44687837,547.00548544)
\curveto(913.15687709,547.32548417)(912.91687733,547.71548378)(912.72687837,548.17548544)
\curveto(912.67687757,548.27548322)(912.63687761,548.37548312)(912.60687837,548.47548544)
\curveto(912.58687766,548.57548292)(912.56687768,548.68048282)(912.54687837,548.79048544)
\curveto(912.52687772,548.83048267)(912.51687773,548.86048264)(912.51687837,548.88048544)
\curveto(912.52687772,548.91048259)(912.52687772,548.94548255)(912.51687837,548.98548544)
\curveto(912.49687775,549.06548243)(912.48187776,549.14548235)(912.47187837,549.22548544)
\curveto(912.47187777,549.31548218)(912.46187778,549.4004821)(912.44187837,549.48048544)
\lineto(912.44187837,549.60048544)
\curveto(912.4418778,549.64048186)(912.43687781,549.68548181)(912.42687837,549.73548544)
\curveto(912.41687783,549.78548171)(912.41187783,549.87048163)(912.41187837,549.99048544)
\curveto(912.41187783,550.12048138)(912.42187782,550.21548128)(912.44187837,550.27548544)
\curveto(912.46187778,550.34548115)(912.46687778,550.41548108)(912.45687837,550.48548544)
\curveto(912.4468778,550.55548094)(912.45187779,550.62548087)(912.47187837,550.69548544)
\curveto(912.48187776,550.74548075)(912.48687776,550.78548071)(912.48687837,550.81548544)
\curveto(912.49687775,550.85548064)(912.50687774,550.9004806)(912.51687837,550.95048544)
\curveto(912.5468777,551.07048043)(912.57187767,551.19048031)(912.59187837,551.31048544)
\curveto(912.62187762,551.43048007)(912.66187758,551.54547995)(912.71187837,551.65548544)
\curveto(912.86187738,552.02547947)(913.0418772,552.35547914)(913.25187837,552.64548544)
\curveto(913.47187677,552.94547855)(913.73687651,553.1954783)(914.04687837,553.39548544)
\curveto(914.16687608,553.47547802)(914.29187595,553.54047796)(914.42187837,553.59048544)
\curveto(914.55187569,553.65047785)(914.68687556,553.71047779)(914.82687837,553.77048544)
\curveto(914.9468753,553.82047768)(915.07687517,553.85047765)(915.21687837,553.86048544)
\curveto(915.35687489,553.88047762)(915.49687475,553.91047759)(915.63687837,553.95048544)
\lineto(915.83187837,553.95048544)
\curveto(915.90187434,553.96047754)(915.96687428,553.97047753)(916.02687837,553.98048544)
\curveto(916.91687333,553.99047751)(917.65687259,553.80547769)(918.24687837,553.42548544)
\curveto(918.83687141,553.04547845)(919.26187098,552.55047895)(919.52187837,551.94048544)
\curveto(919.57187067,551.84047966)(919.61187063,551.74047976)(919.64187837,551.64048544)
\curveto(919.67187057,551.54047996)(919.70687054,551.43548006)(919.74687837,551.32548544)
\curveto(919.77687047,551.21548028)(919.80187044,551.0954804)(919.82187837,550.96548544)
\curveto(919.8418704,550.84548065)(919.86687038,550.72048078)(919.89687837,550.59048544)
\curveto(919.90687034,550.54048096)(919.90687034,550.48548101)(919.89687837,550.42548544)
\curveto(919.89687035,550.37548112)(919.90187034,550.32548117)(919.91187837,550.27548544)
\moveto(918.57687837,549.42048544)
\curveto(918.59687165,549.49048201)(918.60187164,549.57048193)(918.59187837,549.66048544)
\lineto(918.59187837,549.91548544)
\curveto(918.59187165,550.30548119)(918.55687169,550.63548086)(918.48687837,550.90548544)
\curveto(918.45687179,550.98548051)(918.43187181,551.06548043)(918.41187837,551.14548544)
\curveto(918.39187185,551.22548027)(918.36687188,551.3004802)(918.33687837,551.37048544)
\curveto(918.05687219,552.02047948)(917.61187263,552.47047903)(917.00187837,552.72048544)
\curveto(916.93187331,552.75047875)(916.85687339,552.77047873)(916.77687837,552.78048544)
\lineto(916.53687837,552.84048544)
\curveto(916.45687379,552.86047864)(916.37187387,552.87047863)(916.28187837,552.87048544)
\lineto(916.01187837,552.87048544)
\lineto(915.74187837,552.82548544)
\curveto(915.6418746,552.80547869)(915.5468747,552.78047872)(915.45687837,552.75048544)
\curveto(915.37687487,552.73047877)(915.29687495,552.7004788)(915.21687837,552.66048544)
\curveto(915.1468751,552.64047886)(915.08187516,552.61047889)(915.02187837,552.57048544)
\curveto(914.96187528,552.53047897)(914.90687534,552.49047901)(914.85687837,552.45048544)
\curveto(914.61687563,552.28047922)(914.42187582,552.07547942)(914.27187837,551.83548544)
\curveto(914.12187612,551.5954799)(913.99187625,551.31548018)(913.88187837,550.99548544)
\curveto(913.85187639,550.8954806)(913.83187641,550.79048071)(913.82187837,550.68048544)
\curveto(913.81187643,550.58048092)(913.79687645,550.47548102)(913.77687837,550.36548544)
\curveto(913.76687648,550.32548117)(913.76187648,550.26048124)(913.76187837,550.17048544)
\curveto(913.75187649,550.14048136)(913.7468765,550.10548139)(913.74687837,550.06548544)
\curveto(913.75687649,550.02548147)(913.76187648,549.98048152)(913.76187837,549.93048544)
\lineto(913.76187837,549.63048544)
\curveto(913.76187648,549.53048197)(913.77187647,549.44048206)(913.79187837,549.36048544)
\lineto(913.82187837,549.18048544)
\curveto(913.8418764,549.08048242)(913.85687639,548.98048252)(913.86687837,548.88048544)
\curveto(913.88687636,548.79048271)(913.91687633,548.70548279)(913.95687837,548.62548544)
\curveto(914.05687619,548.38548311)(914.17187607,548.16048334)(914.30187837,547.95048544)
\curveto(914.4418758,547.74048376)(914.61187563,547.56548393)(914.81187837,547.42548544)
\curveto(914.86187538,547.3954841)(914.90687534,547.37048413)(914.94687837,547.35048544)
\curveto(914.98687526,547.33048417)(915.03187521,547.30548419)(915.08187837,547.27548544)
\curveto(915.16187508,547.22548427)(915.246875,547.18048432)(915.33687837,547.14048544)
\curveto(915.43687481,547.11048439)(915.5418747,547.08048442)(915.65187837,547.05048544)
\curveto(915.70187454,547.03048447)(915.7468745,547.02048448)(915.78687837,547.02048544)
\curveto(915.83687441,547.03048447)(915.88687436,547.03048447)(915.93687837,547.02048544)
\curveto(915.96687428,547.01048449)(916.02687422,547.0004845)(916.11687837,546.99048544)
\curveto(916.21687403,546.98048452)(916.29187395,546.98548451)(916.34187837,547.00548544)
\curveto(916.38187386,547.01548448)(916.42187382,547.01548448)(916.46187837,547.00548544)
\curveto(916.50187374,547.00548449)(916.5418737,547.01548448)(916.58187837,547.03548544)
\curveto(916.66187358,547.05548444)(916.7418735,547.07048443)(916.82187837,547.08048544)
\curveto(916.90187334,547.1004844)(916.97687327,547.12548437)(917.04687837,547.15548544)
\curveto(917.38687286,547.2954842)(917.66187258,547.49048401)(917.87187837,547.74048544)
\curveto(918.08187216,547.99048351)(918.25687199,548.28548321)(918.39687837,548.62548544)
\curveto(918.4468718,548.74548275)(918.47687177,548.87048263)(918.48687837,549.00048544)
\curveto(918.50687174,549.14048236)(918.53687171,549.28048222)(918.57687837,549.42048544)
}
}
{
\newrgbcolor{curcolor}{0 0 0}
\pscustom[linestyle=none,fillstyle=solid,fillcolor=curcolor]
{
\newpath
\moveto(925.04515962,553.98048544)
\curveto(925.27515483,553.98047752)(925.4051547,553.92047758)(925.43515962,553.80048544)
\curveto(925.46515464,553.69047781)(925.48015463,553.52547797)(925.48015962,553.30548544)
\lineto(925.48015962,553.02048544)
\curveto(925.48015463,552.93047857)(925.45515465,552.85547864)(925.40515962,552.79548544)
\curveto(925.34515476,552.71547878)(925.26015485,552.67047883)(925.15015962,552.66048544)
\curveto(925.04015507,552.66047884)(924.93015518,552.64547885)(924.82015962,552.61548544)
\curveto(924.68015543,552.58547891)(924.54515556,552.55547894)(924.41515962,552.52548544)
\curveto(924.29515581,552.495479)(924.18015593,552.45547904)(924.07015962,552.40548544)
\curveto(923.78015633,552.27547922)(923.54515656,552.0954794)(923.36515962,551.86548544)
\curveto(923.18515692,551.64547985)(923.03015708,551.39048011)(922.90015962,551.10048544)
\curveto(922.86015725,550.99048051)(922.83015728,550.87548062)(922.81015962,550.75548544)
\curveto(922.79015732,550.64548085)(922.76515734,550.53048097)(922.73515962,550.41048544)
\curveto(922.72515738,550.36048114)(922.72015739,550.31048119)(922.72015962,550.26048544)
\curveto(922.73015738,550.21048129)(922.73015738,550.16048134)(922.72015962,550.11048544)
\curveto(922.69015742,549.99048151)(922.67515743,549.85048165)(922.67515962,549.69048544)
\curveto(922.68515742,549.54048196)(922.69015742,549.3954821)(922.69015962,549.25548544)
\lineto(922.69015962,547.41048544)
\lineto(922.69015962,547.06548544)
\curveto(922.69015742,546.94548455)(922.68515742,546.83048467)(922.67515962,546.72048544)
\curveto(922.66515744,546.61048489)(922.66015745,546.51548498)(922.66015962,546.43548544)
\curveto(922.67015744,546.35548514)(922.65015746,546.28548521)(922.60015962,546.22548544)
\curveto(922.55015756,546.15548534)(922.47015764,546.11548538)(922.36015962,546.10548544)
\curveto(922.26015785,546.0954854)(922.15015796,546.09048541)(922.03015962,546.09048544)
\lineto(921.76015962,546.09048544)
\curveto(921.7101584,546.11048539)(921.66015845,546.12548537)(921.61015962,546.13548544)
\curveto(921.57015854,546.15548534)(921.54015857,546.18048532)(921.52015962,546.21048544)
\curveto(921.47015864,546.28048522)(921.44015867,546.36548513)(921.43015962,546.46548544)
\lineto(921.43015962,546.79548544)
\lineto(921.43015962,547.95048544)
\lineto(921.43015962,552.10548544)
\lineto(921.43015962,553.14048544)
\lineto(921.43015962,553.44048544)
\curveto(921.44015867,553.54047796)(921.47015864,553.62547787)(921.52015962,553.69548544)
\curveto(921.55015856,553.73547776)(921.60015851,553.76547773)(921.67015962,553.78548544)
\curveto(921.75015836,553.80547769)(921.83515827,553.81547768)(921.92515962,553.81548544)
\curveto(922.01515809,553.82547767)(922.105158,553.82547767)(922.19515962,553.81548544)
\curveto(922.28515782,553.80547769)(922.35515775,553.79047771)(922.40515962,553.77048544)
\curveto(922.48515762,553.74047776)(922.53515757,553.68047782)(922.55515962,553.59048544)
\curveto(922.58515752,553.51047799)(922.60015751,553.42047808)(922.60015962,553.32048544)
\lineto(922.60015962,553.02048544)
\curveto(922.60015751,552.92047858)(922.62015749,552.83047867)(922.66015962,552.75048544)
\curveto(922.67015744,552.73047877)(922.68015743,552.71547878)(922.69015962,552.70548544)
\lineto(922.73515962,552.66048544)
\curveto(922.84515726,552.66047884)(922.93515717,552.70547879)(923.00515962,552.79548544)
\curveto(923.07515703,552.8954786)(923.13515697,552.97547852)(923.18515962,553.03548544)
\lineto(923.27515962,553.12548544)
\curveto(923.36515674,553.23547826)(923.49015662,553.35047815)(923.65015962,553.47048544)
\curveto(923.8101563,553.59047791)(923.96015615,553.68047782)(924.10015962,553.74048544)
\curveto(924.19015592,553.79047771)(924.28515582,553.82547767)(924.38515962,553.84548544)
\curveto(924.48515562,553.87547762)(924.59015552,553.90547759)(924.70015962,553.93548544)
\curveto(924.76015535,553.94547755)(924.82015529,553.95047755)(924.88015962,553.95048544)
\curveto(924.94015517,553.96047754)(924.99515511,553.97047753)(925.04515962,553.98048544)
}
}
{
\newrgbcolor{curcolor}{0.3019608 0.3019608 0.3019608}
\pscustom[linestyle=none,fillstyle=solid,fillcolor=curcolor]
{
\newpath
\moveto(812.80437349,556.7857662)
\lineto(827.80437349,556.7857662)
\lineto(827.80437349,541.7857662)
\lineto(812.80437349,541.7857662)
\closepath
}
}
{
\newrgbcolor{curcolor}{0.80000001 0.80000001 0.80000001}
\pscustom[linestyle=none,fillstyle=solid,fillcolor=curcolor]
{
\newpath
\moveto(317.29887519,272.12944313)
\curveto(377.21848815,217.36870856)(381.40058538,124.40191832)(326.63985081,64.48230536)
\curveto(321.76005274,59.14279314)(316.49273834,54.17080219)(310.88084865,49.60686595)
\lineto(218.14579411,163.63538644)
\closepath
}
}
{
\newrgbcolor{curcolor}{0.90196079 0.90196079 0.90196079}
\pscustom[linestyle=none,fillstyle=solid,fillcolor=curcolor]
{
\newpath
\moveto(218.14579699,310.61257455)
\curveto(254.58548008,310.61257383)(289.72594016,297.07611118)(316.74863917,272.6297568)
\lineto(218.14579411,163.63538644)
\closepath
}
}
{
\newrgbcolor{curcolor}{0.7019608 0.7019608 0.7019608}
\pscustom[linestyle=none,fillstyle=solid,fillcolor=curcolor]
{
\newpath
\moveto(310.85346121,49.58459825)
\curveto(297.3355099,38.59635565)(281.97174055,30.09880393)(265.47926459,24.48857046)
\lineto(218.14579411,163.63538644)
\closepath
}
}
{
\newrgbcolor{curcolor}{0.60000002 0.60000002 0.60000002}
\pscustom[linestyle=none,fillstyle=solid,fillcolor=curcolor]
{
\newpath
\moveto(265.48725154,24.49128763)
\curveto(257.17700069,21.66386347)(248.63016006,19.58660354)(239.94920957,18.28441499)
\lineto(218.14579411,163.63538644)
\closepath
}
}
{
\newrgbcolor{curcolor}{0.50196081 0.50196081 0.50196081}
\pscustom[linestyle=none,fillstyle=solid,fillcolor=curcolor]
{
\newpath
\moveto(239.86453893,18.27173916)
\curveto(232.67482369,17.19752568)(225.41531934,16.65819852)(218.14579805,16.65819833)
\lineto(218.14579411,163.63538644)
\closepath
}
}
{
\newrgbcolor{curcolor}{0.40000001 0.40000001 0.40000001}
\pscustom[linestyle=none,fillstyle=solid,fillcolor=curcolor]
{
\newpath
\moveto(218.14579805,16.65819833)
\curveto(136.97253848,16.65819615)(71.16860818,82.46212293)(71.168606,163.6353825)
\curveto(71.16860427,228.30317931)(113.43635648,285.37343078)(175.29471517,304.22727403)
\lineto(218.14579411,163.63538644)
\closepath
}
}
{
\newrgbcolor{curcolor}{0.3019608 0.3019608 0.3019608}
\pscustom[linestyle=none,fillstyle=solid,fillcolor=curcolor]
{
\newpath
\moveto(175.26755362,304.21899258)
\curveto(189.16569111,308.4579483)(203.61558807,310.61257483)(218.14579699,310.61257455)
\lineto(218.14579411,163.63538644)
\closepath
}
}
{
\newrgbcolor{curcolor}{0.80000001 0.80000001 0.80000001}
\pscustom[linestyle=none,fillstyle=solid,fillcolor=curcolor]
{
\newpath
\moveto(603.81799469,671.49475455)
\curveto(651.63672622,671.49475361)(696.4662482,648.23185622)(723.99549649,609.13234762)
\lineto(603.81799181,524.51756644)
\closepath
}
}
{
\newrgbcolor{curcolor}{0.40000001 0.40000001 0.40000001}
\pscustom[linestyle=none,fillstyle=solid,fillcolor=curcolor]
{
\newpath
\moveto(723.98235436,609.15101016)
\curveto(770.72411466,542.78606525)(754.81638045,451.09496418)(688.45143554,404.35320388)
\curveto(622.08649063,357.61144359)(530.39538955,373.5191778)(483.65362926,439.88412271)
\curveto(436.91186896,506.24906762)(452.81960317,597.94016869)(519.18454808,644.68192899)
\curveto(536.11320049,656.605016)(555.37008691,664.81665272)(575.69348709,668.77880884)
\lineto(603.81799181,524.51756644)
\closepath
}
}
{
\newrgbcolor{curcolor}{0.3019608 0.3019608 0.3019608}
\pscustom[linestyle=none,fillstyle=solid,fillcolor=curcolor]
{
\newpath
\moveto(575.65047067,668.7704159)
\curveto(584.93016201,670.58241409)(594.36304843,671.49475473)(603.81799469,671.49475455)
\lineto(603.81799181,524.51756644)
\closepath
}
}
{
\newrgbcolor{curcolor}{0.80000001 0.80000001 0.80000001}
\pscustom[linestyle=none,fillstyle=solid,fillcolor=curcolor]
{
\newpath
\moveto(237.31022677,670.80385389)
\curveto(246.031435,669.77926905)(254.64327676,667.97517266)(263.0424198,665.41321182)
\lineto(220.16099661,524.83057644)
\closepath
}
}
{
\newrgbcolor{curcolor}{0.90196079 0.90196079 0.90196079}
\pscustom[linestyle=none,fillstyle=solid,fillcolor=curcolor]
{
\newpath
\moveto(220.16099949,671.80776455)
\curveto(225.90945545,671.80776443)(231.6529609,671.47052095)(237.36191483,670.79777219)
\lineto(220.16099661,524.83057644)
\closepath
}
}
{
\newrgbcolor{curcolor}{0.7019608 0.7019608 0.7019608}
\pscustom[linestyle=none,fillstyle=solid,fillcolor=curcolor]
{
\newpath
\moveto(262.89364534,665.45850594)
\curveto(283.08702225,659.32233879)(301.7220603,648.90922784)(317.53134213,634.92739582)
\lineto(220.16099661,524.83057644)
\closepath
}
}
{
\newrgbcolor{curcolor}{0.60000002 0.60000002 0.60000002}
\pscustom[linestyle=none,fillstyle=solid,fillcolor=curcolor]
{
\newpath
\moveto(317.3703317,635.06958495)
\curveto(326.25486867,627.23515371)(334.15973508,618.35541144)(340.91290045,608.62360445)
\lineto(220.16099661,524.83057644)
\closepath
}
}
{
\newrgbcolor{curcolor}{0.50196081 0.50196081 0.50196081}
\pscustom[linestyle=none,fillstyle=solid,fillcolor=curcolor]
{
\newpath
\moveto(340.69887792,608.93118877)
\curveto(387.14636356,542.35995515)(370.83284257,450.74018077)(304.26160895,404.29269513)
\curveto(301.19237467,402.15125645)(298.04246957,400.12782605)(294.81878123,398.22682838)
\lineto(220.16099661,524.83057644)
\closepath
}
}
{
\newrgbcolor{curcolor}{0.40000001 0.40000001 0.40000001}
\pscustom[linestyle=none,fillstyle=solid,fillcolor=curcolor]
{
\newpath
\moveto(294.93351593,398.29455714)
\curveto(225.04960216,356.99883501)(134.92069944,380.17414336)(93.62497731,450.05805712)
\curveto(52.32925519,519.94197088)(75.50456353,610.07087361)(145.3884773,651.36659573)
\curveto(168.03443706,664.74852029)(193.8567147,671.80776506)(220.16099949,671.80776455)
\lineto(220.16099661,524.83057644)
\closepath
}
}
{
\newrgbcolor{curcolor}{0 0 0}
\pscustom[linestyle=none,fillstyle=solid,fillcolor=curcolor]
{
\newpath
\moveto(92.48718721,500.63789022)
\lineto(96.08718721,500.63789022)
\lineto(96.73218721,500.63789022)
\curveto(96.81218068,500.6378798)(96.88718061,500.6328798)(96.95718721,500.62289022)
\curveto(97.02718047,500.62287981)(97.08718041,500.61287982)(97.13718721,500.59289022)
\curveto(97.20718029,500.56287987)(97.26218023,500.50287993)(97.30218721,500.41289022)
\curveto(97.32218017,500.38288005)(97.33218016,500.34288009)(97.33218721,500.29289022)
\lineto(97.33218721,500.15789022)
\curveto(97.34218015,500.04788039)(97.33718016,499.94288049)(97.31718721,499.84289022)
\curveto(97.30718019,499.74288069)(97.27218022,499.67288076)(97.21218721,499.63289022)
\curveto(97.12218037,499.56288087)(96.98718051,499.52788091)(96.80718721,499.52789022)
\curveto(96.62718087,499.5378809)(96.46218103,499.54288089)(96.31218721,499.54289022)
\lineto(94.31718721,499.54289022)
\lineto(93.82218721,499.54289022)
\lineto(93.68718721,499.54289022)
\curveto(93.64718385,499.54288089)(93.60718389,499.5378809)(93.56718721,499.52789022)
\lineto(93.35718721,499.52789022)
\curveto(93.24718425,499.49788094)(93.16718433,499.45788098)(93.11718721,499.40789022)
\curveto(93.06718443,499.36788107)(93.03218446,499.31288112)(93.01218721,499.24289022)
\curveto(92.9921845,499.18288125)(92.97718452,499.11288132)(92.96718721,499.03289022)
\curveto(92.95718454,498.95288148)(92.93718456,498.86288157)(92.90718721,498.76289022)
\curveto(92.85718464,498.56288187)(92.81718468,498.35788208)(92.78718721,498.14789022)
\curveto(92.75718474,497.9378825)(92.71718478,497.7328827)(92.66718721,497.53289022)
\curveto(92.64718485,497.46288297)(92.63718486,497.39288304)(92.63718721,497.32289022)
\curveto(92.63718486,497.26288317)(92.62718487,497.19788324)(92.60718721,497.12789022)
\curveto(92.5971849,497.09788334)(92.58718491,497.05788338)(92.57718721,497.00789022)
\curveto(92.57718492,496.96788347)(92.58218491,496.92788351)(92.59218721,496.88789022)
\curveto(92.61218488,496.8378836)(92.63718486,496.79288364)(92.66718721,496.75289022)
\curveto(92.70718479,496.72288371)(92.76718473,496.71788372)(92.84718721,496.73789022)
\curveto(92.90718459,496.75788368)(92.96718453,496.78288365)(93.02718721,496.81289022)
\curveto(93.08718441,496.85288358)(93.14718435,496.88788355)(93.20718721,496.91789022)
\curveto(93.26718423,496.9378835)(93.31718418,496.95288348)(93.35718721,496.96289022)
\curveto(93.54718395,497.04288339)(93.75218374,497.09788334)(93.97218721,497.12789022)
\curveto(94.20218329,497.15788328)(94.43218306,497.16788327)(94.66218721,497.15789022)
\curveto(94.90218259,497.15788328)(95.13218236,497.1328833)(95.35218721,497.08289022)
\curveto(95.57218192,497.04288339)(95.77218172,496.98288345)(95.95218721,496.90289022)
\curveto(96.00218149,496.88288355)(96.04718145,496.86288357)(96.08718721,496.84289022)
\curveto(96.13718136,496.82288361)(96.18718131,496.79788364)(96.23718721,496.76789022)
\curveto(96.58718091,496.55788388)(96.86718063,496.32788411)(97.07718721,496.07789022)
\curveto(97.2971802,495.82788461)(97.49218,495.50288493)(97.66218721,495.10289022)
\curveto(97.71217978,494.99288544)(97.74717975,494.88288555)(97.76718721,494.77289022)
\curveto(97.78717971,494.66288577)(97.81217968,494.54788589)(97.84218721,494.42789022)
\curveto(97.85217964,494.39788604)(97.85717964,494.35288608)(97.85718721,494.29289022)
\curveto(97.87717962,494.2328862)(97.88717961,494.16288627)(97.88718721,494.08289022)
\curveto(97.88717961,494.01288642)(97.8971796,493.94788649)(97.91718721,493.88789022)
\lineto(97.91718721,493.72289022)
\curveto(97.92717957,493.67288676)(97.93217956,493.60288683)(97.93218721,493.51289022)
\curveto(97.93217956,493.42288701)(97.92217957,493.35288708)(97.90218721,493.30289022)
\curveto(97.88217961,493.24288719)(97.87717962,493.18288725)(97.88718721,493.12289022)
\curveto(97.8971796,493.07288736)(97.8921796,493.02288741)(97.87218721,492.97289022)
\curveto(97.83217966,492.81288762)(97.7971797,492.66288777)(97.76718721,492.52289022)
\curveto(97.73717976,492.38288805)(97.6921798,492.24788819)(97.63218721,492.11789022)
\curveto(97.47218002,491.74788869)(97.25218024,491.41288902)(96.97218721,491.11289022)
\curveto(96.6921808,490.81288962)(96.37218112,490.58288985)(96.01218721,490.42289022)
\curveto(95.84218165,490.34289009)(95.64218185,490.26789017)(95.41218721,490.19789022)
\curveto(95.30218219,490.15789028)(95.18718231,490.1328903)(95.06718721,490.12289022)
\curveto(94.94718255,490.11289032)(94.82718267,490.09289034)(94.70718721,490.06289022)
\curveto(94.65718284,490.04289039)(94.60218289,490.04289039)(94.54218721,490.06289022)
\curveto(94.48218301,490.07289036)(94.42218307,490.06789037)(94.36218721,490.04789022)
\curveto(94.26218323,490.02789041)(94.16218333,490.02789041)(94.06218721,490.04789022)
\lineto(93.92718721,490.04789022)
\curveto(93.87718362,490.06789037)(93.81718368,490.07789036)(93.74718721,490.07789022)
\curveto(93.68718381,490.06789037)(93.63218386,490.07289036)(93.58218721,490.09289022)
\curveto(93.54218395,490.10289033)(93.50718399,490.10789033)(93.47718721,490.10789022)
\curveto(93.44718405,490.10789033)(93.41218408,490.11289032)(93.37218721,490.12289022)
\lineto(93.10218721,490.18289022)
\curveto(93.01218448,490.20289023)(92.92718457,490.2328902)(92.84718721,490.27289022)
\curveto(92.50718499,490.41289002)(92.21718528,490.56788987)(91.97718721,490.73789022)
\curveto(91.73718576,490.91788952)(91.51718598,491.14788929)(91.31718721,491.42789022)
\curveto(91.16718633,491.65788878)(91.05218644,491.89788854)(90.97218721,492.14789022)
\curveto(90.95218654,492.19788824)(90.94218655,492.24288819)(90.94218721,492.28289022)
\curveto(90.94218655,492.3328881)(90.93218656,492.38288805)(90.91218721,492.43289022)
\curveto(90.8921866,492.49288794)(90.87718662,492.57288786)(90.86718721,492.67289022)
\curveto(90.86718663,492.77288766)(90.88718661,492.84788759)(90.92718721,492.89789022)
\curveto(90.97718652,492.97788746)(91.05718644,493.02288741)(91.16718721,493.03289022)
\curveto(91.27718622,493.04288739)(91.3921861,493.04788739)(91.51218721,493.04789022)
\lineto(91.67718721,493.04789022)
\curveto(91.73718576,493.04788739)(91.7921857,493.0378874)(91.84218721,493.01789022)
\curveto(91.93218556,492.99788744)(92.00218549,492.95788748)(92.05218721,492.89789022)
\curveto(92.12218537,492.80788763)(92.16718533,492.69788774)(92.18718721,492.56789022)
\curveto(92.21718528,492.44788799)(92.26218523,492.34288809)(92.32218721,492.25289022)
\curveto(92.51218498,491.91288852)(92.77218472,491.64288879)(93.10218721,491.44289022)
\curveto(93.20218429,491.38288905)(93.30718419,491.3328891)(93.41718721,491.29289022)
\curveto(93.53718396,491.26288917)(93.65718384,491.22788921)(93.77718721,491.18789022)
\curveto(93.94718355,491.1378893)(94.15218334,491.11788932)(94.39218721,491.12789022)
\curveto(94.64218285,491.14788929)(94.84218265,491.18288925)(94.99218721,491.23289022)
\curveto(95.36218213,491.35288908)(95.65218184,491.51288892)(95.86218721,491.71289022)
\curveto(96.08218141,491.92288851)(96.26218123,492.20288823)(96.40218721,492.55289022)
\curveto(96.45218104,492.65288778)(96.48218101,492.75788768)(96.49218721,492.86789022)
\curveto(96.51218098,492.97788746)(96.53718096,493.09288734)(96.56718721,493.21289022)
\lineto(96.56718721,493.31789022)
\curveto(96.57718092,493.35788708)(96.58218091,493.39788704)(96.58218721,493.43789022)
\curveto(96.5921809,493.46788697)(96.5921809,493.50288693)(96.58218721,493.54289022)
\lineto(96.58218721,493.66289022)
\curveto(96.58218091,493.92288651)(96.55218094,494.16788627)(96.49218721,494.39789022)
\curveto(96.38218111,494.74788569)(96.22718127,495.04288539)(96.02718721,495.28289022)
\curveto(95.82718167,495.5328849)(95.56718193,495.72788471)(95.24718721,495.86789022)
\lineto(95.06718721,495.92789022)
\curveto(95.01718248,495.94788449)(94.95718254,495.96788447)(94.88718721,495.98789022)
\curveto(94.83718266,496.00788443)(94.77718272,496.01788442)(94.70718721,496.01789022)
\curveto(94.64718285,496.02788441)(94.58218291,496.04288439)(94.51218721,496.06289022)
\lineto(94.36218721,496.06289022)
\curveto(94.32218317,496.08288435)(94.26718323,496.09288434)(94.19718721,496.09289022)
\curveto(94.13718336,496.09288434)(94.08218341,496.08288435)(94.03218721,496.06289022)
\lineto(93.92718721,496.06289022)
\curveto(93.8971836,496.06288437)(93.86218363,496.05788438)(93.82218721,496.04789022)
\lineto(93.58218721,495.98789022)
\curveto(93.50218399,495.97788446)(93.42218407,495.95788448)(93.34218721,495.92789022)
\curveto(93.10218439,495.82788461)(92.87218462,495.69288474)(92.65218721,495.52289022)
\curveto(92.56218493,495.45288498)(92.47718502,495.37788506)(92.39718721,495.29789022)
\curveto(92.31718518,495.22788521)(92.21718528,495.17288526)(92.09718721,495.13289022)
\curveto(92.00718549,495.10288533)(91.86718563,495.09288534)(91.67718721,495.10289022)
\curveto(91.497186,495.11288532)(91.37718612,495.1378853)(91.31718721,495.17789022)
\curveto(91.26718623,495.21788522)(91.22718627,495.27788516)(91.19718721,495.35789022)
\curveto(91.17718632,495.437885)(91.17718632,495.52288491)(91.19718721,495.61289022)
\curveto(91.22718627,495.7328847)(91.24718625,495.85288458)(91.25718721,495.97289022)
\curveto(91.27718622,496.10288433)(91.30218619,496.22788421)(91.33218721,496.34789022)
\curveto(91.35218614,496.38788405)(91.35718614,496.42288401)(91.34718721,496.45289022)
\curveto(91.34718615,496.49288394)(91.35718614,496.5378839)(91.37718721,496.58789022)
\curveto(91.3971861,496.67788376)(91.41218608,496.76788367)(91.42218721,496.85789022)
\curveto(91.43218606,496.95788348)(91.45218604,497.05288338)(91.48218721,497.14289022)
\curveto(91.492186,497.20288323)(91.497186,497.26288317)(91.49718721,497.32289022)
\curveto(91.50718599,497.38288305)(91.52218597,497.44288299)(91.54218721,497.50289022)
\curveto(91.5921859,497.70288273)(91.62718587,497.90788253)(91.64718721,498.11789022)
\curveto(91.67718582,498.3378821)(91.71718578,498.54788189)(91.76718721,498.74789022)
\curveto(91.7971857,498.84788159)(91.81718568,498.94788149)(91.82718721,499.04789022)
\curveto(91.83718566,499.14788129)(91.85218564,499.24788119)(91.87218721,499.34789022)
\curveto(91.88218561,499.37788106)(91.88718561,499.41788102)(91.88718721,499.46789022)
\curveto(91.91718558,499.57788086)(91.93718556,499.68288075)(91.94718721,499.78289022)
\curveto(91.96718553,499.89288054)(91.9921855,500.00288043)(92.02218721,500.11289022)
\curveto(92.04218545,500.19288024)(92.05718544,500.26288017)(92.06718721,500.32289022)
\curveto(92.07718542,500.39288004)(92.10218539,500.45287998)(92.14218721,500.50289022)
\curveto(92.16218533,500.5328799)(92.1921853,500.55287988)(92.23218721,500.56289022)
\curveto(92.27218522,500.58287985)(92.31718518,500.60287983)(92.36718721,500.62289022)
\curveto(92.42718507,500.62287981)(92.46718503,500.62787981)(92.48718721,500.63789022)
}
}
{
\newrgbcolor{curcolor}{0 0 0}
\pscustom[linestyle=none,fillstyle=solid,fillcolor=curcolor]
{
\newpath
\moveto(106.34179659,493.30289022)
\curveto(106.35178887,493.26288717)(106.35178887,493.21288722)(106.34179659,493.15289022)
\curveto(106.34178888,493.09288734)(106.33678888,493.04288739)(106.32679659,493.00289022)
\curveto(106.32678889,492.96288747)(106.3217889,492.92288751)(106.31179659,492.88289022)
\lineto(106.31179659,492.77789022)
\curveto(106.29178893,492.69788774)(106.27678894,492.61788782)(106.26679659,492.53789022)
\curveto(106.25678896,492.45788798)(106.23678898,492.38288805)(106.20679659,492.31289022)
\curveto(106.18678903,492.2328882)(106.16678905,492.15788828)(106.14679659,492.08789022)
\curveto(106.12678909,492.01788842)(106.09678912,491.94288849)(106.05679659,491.86289022)
\curveto(105.87678934,491.44288899)(105.6217896,491.10288933)(105.29179659,490.84289022)
\curveto(104.96179026,490.58288985)(104.57179065,490.37789006)(104.12179659,490.22789022)
\curveto(104.00179122,490.18789025)(103.87679134,490.16289027)(103.74679659,490.15289022)
\curveto(103.62679159,490.1328903)(103.50179172,490.10789033)(103.37179659,490.07789022)
\curveto(103.31179191,490.06789037)(103.24679197,490.06289037)(103.17679659,490.06289022)
\curveto(103.1167921,490.06289037)(103.05179217,490.05789038)(102.98179659,490.04789022)
\lineto(102.86179659,490.04789022)
\lineto(102.66679659,490.04789022)
\curveto(102.60679261,490.0378904)(102.55179267,490.04289039)(102.50179659,490.06289022)
\curveto(102.43179279,490.08289035)(102.36679285,490.08789035)(102.30679659,490.07789022)
\curveto(102.24679297,490.06789037)(102.18679303,490.07289036)(102.12679659,490.09289022)
\curveto(102.07679314,490.10289033)(102.03179319,490.10789033)(101.99179659,490.10789022)
\curveto(101.95179327,490.10789033)(101.90679331,490.11789032)(101.85679659,490.13789022)
\curveto(101.77679344,490.15789028)(101.70179352,490.17789026)(101.63179659,490.19789022)
\curveto(101.56179366,490.20789023)(101.49179373,490.22289021)(101.42179659,490.24289022)
\curveto(100.94179428,490.41289002)(100.54179468,490.62288981)(100.22179659,490.87289022)
\curveto(99.91179531,491.1328893)(99.66179556,491.48788895)(99.47179659,491.93789022)
\curveto(99.44179578,491.99788844)(99.4167958,492.05788838)(99.39679659,492.11789022)
\curveto(99.38679583,492.18788825)(99.37179585,492.26288817)(99.35179659,492.34289022)
\curveto(99.33179589,492.40288803)(99.3167959,492.46788797)(99.30679659,492.53789022)
\curveto(99.29679592,492.60788783)(99.28179594,492.67788776)(99.26179659,492.74789022)
\curveto(99.25179597,492.79788764)(99.24679597,492.8378876)(99.24679659,492.86789022)
\lineto(99.24679659,492.98789022)
\curveto(99.23679598,493.02788741)(99.22679599,493.07788736)(99.21679659,493.13789022)
\curveto(99.216796,493.19788724)(99.221796,493.24788719)(99.23179659,493.28789022)
\lineto(99.23179659,493.42289022)
\curveto(99.24179598,493.47288696)(99.24679597,493.52288691)(99.24679659,493.57289022)
\curveto(99.26679595,493.67288676)(99.28179594,493.76788667)(99.29179659,493.85789022)
\curveto(99.30179592,493.95788648)(99.3217959,494.05288638)(99.35179659,494.14289022)
\curveto(99.40179582,494.29288614)(99.45679576,494.432886)(99.51679659,494.56289022)
\curveto(99.57679564,494.69288574)(99.64679557,494.81288562)(99.72679659,494.92289022)
\curveto(99.75679546,494.97288546)(99.78679543,495.01288542)(99.81679659,495.04289022)
\curveto(99.85679536,495.07288536)(99.89179533,495.10788533)(99.92179659,495.14789022)
\curveto(99.98179524,495.22788521)(100.05179517,495.29788514)(100.13179659,495.35789022)
\curveto(100.19179503,495.40788503)(100.25179497,495.45288498)(100.31179659,495.49289022)
\lineto(100.52179659,495.64289022)
\curveto(100.57179465,495.68288475)(100.6217946,495.71788472)(100.67179659,495.74789022)
\curveto(100.7217945,495.78788465)(100.75679446,495.84288459)(100.77679659,495.91289022)
\curveto(100.77679444,495.94288449)(100.76679445,495.96788447)(100.74679659,495.98789022)
\curveto(100.73679448,496.01788442)(100.72679449,496.04288439)(100.71679659,496.06289022)
\curveto(100.67679454,496.11288432)(100.62679459,496.15788428)(100.56679659,496.19789022)
\curveto(100.5167947,496.24788419)(100.46679475,496.29288414)(100.41679659,496.33289022)
\curveto(100.37679484,496.36288407)(100.32679489,496.41788402)(100.26679659,496.49789022)
\curveto(100.24679497,496.52788391)(100.216795,496.55288388)(100.17679659,496.57289022)
\curveto(100.14679507,496.60288383)(100.1217951,496.6378838)(100.10179659,496.67789022)
\curveto(99.93179529,496.88788355)(99.80179542,497.1328833)(99.71179659,497.41289022)
\curveto(99.69179553,497.49288294)(99.67679554,497.57288286)(99.66679659,497.65289022)
\curveto(99.65679556,497.7328827)(99.64179558,497.81288262)(99.62179659,497.89289022)
\curveto(99.60179562,497.94288249)(99.59179563,498.00788243)(99.59179659,498.08789022)
\curveto(99.59179563,498.17788226)(99.60179562,498.24788219)(99.62179659,498.29789022)
\curveto(99.6217956,498.39788204)(99.62679559,498.46788197)(99.63679659,498.50789022)
\curveto(99.65679556,498.58788185)(99.67179555,498.65788178)(99.68179659,498.71789022)
\curveto(99.69179553,498.78788165)(99.70679551,498.85788158)(99.72679659,498.92789022)
\curveto(99.87679534,499.35788108)(100.09179513,499.70288073)(100.37179659,499.96289022)
\curveto(100.66179456,500.22288021)(101.01179421,500.43788)(101.42179659,500.60789022)
\curveto(101.53179369,500.65787978)(101.64679357,500.68787975)(101.76679659,500.69789022)
\curveto(101.89679332,500.71787972)(102.02679319,500.74787969)(102.15679659,500.78789022)
\curveto(102.23679298,500.78787965)(102.30679291,500.78787965)(102.36679659,500.78789022)
\curveto(102.43679278,500.79787964)(102.51179271,500.80787963)(102.59179659,500.81789022)
\curveto(103.38179184,500.8378796)(104.03679118,500.70787973)(104.55679659,500.42789022)
\curveto(105.08679013,500.14788029)(105.46678975,499.7378807)(105.69679659,499.19789022)
\curveto(105.80678941,498.96788147)(105.87678934,498.68288175)(105.90679659,498.34289022)
\curveto(105.94678927,498.01288242)(105.9167893,497.70788273)(105.81679659,497.42789022)
\curveto(105.77678944,497.29788314)(105.72678949,497.17788326)(105.66679659,497.06789022)
\curveto(105.6167896,496.95788348)(105.55678966,496.85288358)(105.48679659,496.75289022)
\curveto(105.46678975,496.71288372)(105.43678978,496.67788376)(105.39679659,496.64789022)
\lineto(105.30679659,496.55789022)
\curveto(105.25678996,496.46788397)(105.19679002,496.40288403)(105.12679659,496.36289022)
\curveto(105.07679014,496.31288412)(105.0217902,496.26288417)(104.96179659,496.21289022)
\curveto(104.91179031,496.17288426)(104.86679035,496.12788431)(104.82679659,496.07789022)
\curveto(104.80679041,496.05788438)(104.78679043,496.0328844)(104.76679659,496.00289022)
\curveto(104.75679046,495.98288445)(104.75679046,495.95788448)(104.76679659,495.92789022)
\curveto(104.77679044,495.87788456)(104.80679041,495.82788461)(104.85679659,495.77789022)
\curveto(104.90679031,495.7378847)(104.96179026,495.69788474)(105.02179659,495.65789022)
\lineto(105.20179659,495.53789022)
\curveto(105.26178996,495.50788493)(105.31178991,495.47788496)(105.35179659,495.44789022)
\curveto(105.68178954,495.20788523)(105.93178929,494.89788554)(106.10179659,494.51789022)
\curveto(106.14178908,494.437886)(106.17178905,494.35288608)(106.19179659,494.26289022)
\curveto(106.221789,494.17288626)(106.24678897,494.08288635)(106.26679659,493.99289022)
\curveto(106.27678894,493.94288649)(106.28678893,493.88788655)(106.29679659,493.82789022)
\lineto(106.32679659,493.67789022)
\curveto(106.33678888,493.61788682)(106.33678888,493.55288688)(106.32679659,493.48289022)
\curveto(106.3167889,493.42288701)(106.3217889,493.36288707)(106.34179659,493.30289022)
\moveto(100.95679659,498.34289022)
\curveto(100.92679429,498.2328822)(100.9217943,498.09288234)(100.94179659,497.92289022)
\curveto(100.96179426,497.76288267)(100.98679423,497.6378828)(101.01679659,497.54789022)
\curveto(101.12679409,497.22788321)(101.27679394,496.98288345)(101.46679659,496.81289022)
\curveto(101.65679356,496.65288378)(101.9217933,496.52288391)(102.26179659,496.42289022)
\curveto(102.39179283,496.39288404)(102.55679266,496.36788407)(102.75679659,496.34789022)
\curveto(102.95679226,496.3378841)(103.12679209,496.35288408)(103.26679659,496.39289022)
\curveto(103.55679166,496.47288396)(103.79679142,496.58288385)(103.98679659,496.72289022)
\curveto(104.18679103,496.87288356)(104.34179088,497.07288336)(104.45179659,497.32289022)
\curveto(104.47179075,497.37288306)(104.48179074,497.41788302)(104.48179659,497.45789022)
\curveto(104.49179073,497.49788294)(104.50679071,497.54288289)(104.52679659,497.59289022)
\curveto(104.55679066,497.70288273)(104.57679064,497.84288259)(104.58679659,498.01289022)
\curveto(104.59679062,498.18288225)(104.58679063,498.32788211)(104.55679659,498.44789022)
\curveto(104.53679068,498.5378819)(104.51179071,498.62288181)(104.48179659,498.70289022)
\curveto(104.46179076,498.78288165)(104.42679079,498.86288157)(104.37679659,498.94289022)
\curveto(104.20679101,499.21288122)(103.98179124,499.40788103)(103.70179659,499.52789022)
\curveto(103.43179179,499.64788079)(103.07179215,499.70788073)(102.62179659,499.70789022)
\curveto(102.60179262,499.68788075)(102.57179265,499.68288075)(102.53179659,499.69289022)
\curveto(102.49179273,499.70288073)(102.45679276,499.70288073)(102.42679659,499.69289022)
\curveto(102.37679284,499.67288076)(102.3217929,499.65788078)(102.26179659,499.64789022)
\curveto(102.21179301,499.64788079)(102.16179306,499.6378808)(102.11179659,499.61789022)
\curveto(101.87179335,499.52788091)(101.66179356,499.41288102)(101.48179659,499.27289022)
\curveto(101.30179392,499.14288129)(101.16179406,498.96288147)(101.06179659,498.73289022)
\curveto(101.04179418,498.67288176)(101.0217942,498.60788183)(101.00179659,498.53789022)
\curveto(100.99179423,498.47788196)(100.97679424,498.41288202)(100.95679659,498.34289022)
\moveto(104.97679659,492.80789022)
\curveto(105.02679019,492.99788744)(105.03179019,493.20288723)(104.99179659,493.42289022)
\curveto(104.96179026,493.64288679)(104.9167903,493.82288661)(104.85679659,493.96289022)
\curveto(104.68679053,494.3328861)(104.42679079,494.6378858)(104.07679659,494.87789022)
\curveto(103.73679148,495.11788532)(103.30179192,495.2378852)(102.77179659,495.23789022)
\curveto(102.74179248,495.21788522)(102.70179252,495.21288522)(102.65179659,495.22289022)
\curveto(102.60179262,495.24288519)(102.56179266,495.24788519)(102.53179659,495.23789022)
\lineto(102.26179659,495.17789022)
\curveto(102.18179304,495.16788527)(102.10179312,495.15288528)(102.02179659,495.13289022)
\curveto(101.7217935,495.02288541)(101.45679376,494.87788556)(101.22679659,494.69789022)
\curveto(101.00679421,494.51788592)(100.83679438,494.28788615)(100.71679659,494.00789022)
\curveto(100.68679453,493.92788651)(100.66179456,493.84788659)(100.64179659,493.76789022)
\curveto(100.6217946,493.68788675)(100.60179462,493.60288683)(100.58179659,493.51289022)
\curveto(100.55179467,493.39288704)(100.54179468,493.24288719)(100.55179659,493.06289022)
\curveto(100.57179465,492.88288755)(100.59679462,492.74288769)(100.62679659,492.64289022)
\curveto(100.64679457,492.59288784)(100.65679456,492.54788789)(100.65679659,492.50789022)
\curveto(100.66679455,492.47788796)(100.68179454,492.437888)(100.70179659,492.38789022)
\curveto(100.80179442,492.16788827)(100.93179429,491.96788847)(101.09179659,491.78789022)
\curveto(101.26179396,491.60788883)(101.45679376,491.47288896)(101.67679659,491.38289022)
\curveto(101.74679347,491.34288909)(101.84179338,491.30788913)(101.96179659,491.27789022)
\curveto(102.18179304,491.18788925)(102.43679278,491.14288929)(102.72679659,491.14289022)
\lineto(103.01179659,491.14289022)
\curveto(103.11179211,491.16288927)(103.20679201,491.17788926)(103.29679659,491.18789022)
\curveto(103.38679183,491.19788924)(103.47679174,491.21788922)(103.56679659,491.24789022)
\curveto(103.82679139,491.32788911)(104.06679115,491.45788898)(104.28679659,491.63789022)
\curveto(104.5167907,491.82788861)(104.68679053,492.04288839)(104.79679659,492.28289022)
\curveto(104.83679038,492.36288807)(104.86679035,492.44288799)(104.88679659,492.52289022)
\curveto(104.9167903,492.61288782)(104.94679027,492.70788773)(104.97679659,492.80789022)
}
}
{
\newrgbcolor{curcolor}{0 0 0}
\pscustom[linestyle=none,fillstyle=solid,fillcolor=curcolor]
{
\newpath
\moveto(108.63140596,491.86289022)
\lineto(108.93140596,491.86289022)
\curveto(109.0414039,491.87288856)(109.1464038,491.87288856)(109.24640596,491.86289022)
\curveto(109.35640359,491.86288857)(109.45640349,491.85288858)(109.54640596,491.83289022)
\curveto(109.63640331,491.82288861)(109.70640324,491.79788864)(109.75640596,491.75789022)
\curveto(109.77640317,491.7378887)(109.79140315,491.70788873)(109.80140596,491.66789022)
\curveto(109.82140312,491.62788881)(109.8414031,491.58288885)(109.86140596,491.53289022)
\lineto(109.86140596,491.45789022)
\curveto(109.87140307,491.40788903)(109.87140307,491.35288908)(109.86140596,491.29289022)
\lineto(109.86140596,491.14289022)
\lineto(109.86140596,490.66289022)
\curveto(109.86140308,490.49288994)(109.82140312,490.37289006)(109.74140596,490.30289022)
\curveto(109.67140327,490.25289018)(109.58140336,490.22789021)(109.47140596,490.22789022)
\lineto(109.14140596,490.22789022)
\lineto(108.69140596,490.22789022)
\curveto(108.5414044,490.22789021)(108.42640452,490.25789018)(108.34640596,490.31789022)
\curveto(108.30640464,490.34789009)(108.27640467,490.39789004)(108.25640596,490.46789022)
\curveto(108.23640471,490.54788989)(108.22140472,490.6328898)(108.21140596,490.72289022)
\lineto(108.21140596,491.00789022)
\curveto(108.22140472,491.10788933)(108.22640472,491.19288924)(108.22640596,491.26289022)
\lineto(108.22640596,491.45789022)
\curveto(108.22640472,491.51788892)(108.23640471,491.57288886)(108.25640596,491.62289022)
\curveto(108.29640465,491.7328887)(108.36640458,491.80288863)(108.46640596,491.83289022)
\curveto(108.49640445,491.8328886)(108.55140439,491.84288859)(108.63140596,491.86289022)
}
}
{
\newrgbcolor{curcolor}{0 0 0}
\pscustom[linestyle=none,fillstyle=solid,fillcolor=curcolor]
{
\newpath
\moveto(112.29656221,500.63789022)
\lineto(117.09656221,500.63789022)
\lineto(118.10156221,500.63789022)
\curveto(118.24155511,500.6378798)(118.36155499,500.62787981)(118.46156221,500.60789022)
\curveto(118.57155478,500.59787984)(118.6515547,500.55287988)(118.70156221,500.47289022)
\curveto(118.72155463,500.43288)(118.73155462,500.38288005)(118.73156221,500.32289022)
\curveto(118.74155461,500.26288017)(118.74655461,500.19788024)(118.74656221,500.12789022)
\lineto(118.74656221,499.85789022)
\curveto(118.74655461,499.76788067)(118.73655462,499.68788075)(118.71656221,499.61789022)
\curveto(118.67655468,499.5378809)(118.63155472,499.46788097)(118.58156221,499.40789022)
\lineto(118.43156221,499.22789022)
\curveto(118.40155495,499.17788126)(118.36655499,499.1378813)(118.32656221,499.10789022)
\curveto(118.28655507,499.07788136)(118.24655511,499.0378814)(118.20656221,498.98789022)
\curveto(118.12655523,498.87788156)(118.04155531,498.76788167)(117.95156221,498.65789022)
\curveto(117.86155549,498.55788188)(117.77655558,498.45288198)(117.69656221,498.34289022)
\curveto(117.5565558,498.14288229)(117.41655594,497.9328825)(117.27656221,497.71289022)
\curveto(117.13655622,497.50288293)(116.99655636,497.28788315)(116.85656221,497.06789022)
\curveto(116.80655655,496.97788346)(116.7565566,496.88288355)(116.70656221,496.78289022)
\curveto(116.6565567,496.68288375)(116.60155675,496.58788385)(116.54156221,496.49789022)
\curveto(116.52155683,496.47788396)(116.51155684,496.45288398)(116.51156221,496.42289022)
\curveto(116.51155684,496.39288404)(116.50155685,496.36788407)(116.48156221,496.34789022)
\curveto(116.41155694,496.24788419)(116.34655701,496.1328843)(116.28656221,496.00289022)
\curveto(116.22655713,495.88288455)(116.17155718,495.76788467)(116.12156221,495.65789022)
\curveto(116.02155733,495.42788501)(115.92655743,495.19288524)(115.83656221,494.95289022)
\curveto(115.74655761,494.71288572)(115.64655771,494.47288596)(115.53656221,494.23289022)
\curveto(115.51655784,494.18288625)(115.50155785,494.1378863)(115.49156221,494.09789022)
\curveto(115.49155786,494.05788638)(115.48155787,494.01288642)(115.46156221,493.96289022)
\curveto(115.41155794,493.84288659)(115.36655799,493.71788672)(115.32656221,493.58789022)
\curveto(115.29655806,493.46788697)(115.26155809,493.34788709)(115.22156221,493.22789022)
\curveto(115.14155821,492.99788744)(115.07655828,492.75788768)(115.02656221,492.50789022)
\curveto(114.98655837,492.26788817)(114.93655842,492.02788841)(114.87656221,491.78789022)
\curveto(114.83655852,491.6378888)(114.81155854,491.48788895)(114.80156221,491.33789022)
\curveto(114.79155856,491.18788925)(114.77155858,491.0378894)(114.74156221,490.88789022)
\curveto(114.73155862,490.84788959)(114.72655863,490.78788965)(114.72656221,490.70789022)
\curveto(114.69655866,490.58788985)(114.66655869,490.48788995)(114.63656221,490.40789022)
\curveto(114.60655875,490.32789011)(114.53655882,490.27289016)(114.42656221,490.24289022)
\curveto(114.37655898,490.22289021)(114.32155903,490.21289022)(114.26156221,490.21289022)
\lineto(114.06656221,490.21289022)
\curveto(113.92655943,490.21289022)(113.78655957,490.21789022)(113.64656221,490.22789022)
\curveto(113.51655984,490.2378902)(113.42155993,490.28289015)(113.36156221,490.36289022)
\curveto(113.32156003,490.42289001)(113.30156005,490.50788993)(113.30156221,490.61789022)
\curveto(113.31156004,490.72788971)(113.32656003,490.82288961)(113.34656221,490.90289022)
\lineto(113.34656221,490.97789022)
\curveto(113.35656,491.00788943)(113.36155999,491.0378894)(113.36156221,491.06789022)
\curveto(113.38155997,491.14788929)(113.39155996,491.22288921)(113.39156221,491.29289022)
\curveto(113.39155996,491.36288907)(113.40155995,491.432889)(113.42156221,491.50289022)
\curveto(113.47155988,491.69288874)(113.51155984,491.87788856)(113.54156221,492.05789022)
\curveto(113.57155978,492.24788819)(113.61155974,492.42788801)(113.66156221,492.59789022)
\curveto(113.68155967,492.64788779)(113.69155966,492.68788775)(113.69156221,492.71789022)
\curveto(113.69155966,492.74788769)(113.69655966,492.78288765)(113.70656221,492.82289022)
\curveto(113.80655955,493.12288731)(113.89655946,493.41788702)(113.97656221,493.70789022)
\curveto(114.06655929,493.99788644)(114.17155918,494.27788616)(114.29156221,494.54789022)
\curveto(114.5515588,495.12788531)(114.82155853,495.67788476)(115.10156221,496.19789022)
\curveto(115.38155797,496.72788371)(115.69155766,497.2328832)(116.03156221,497.71289022)
\curveto(116.17155718,497.91288252)(116.32155703,498.10288233)(116.48156221,498.28289022)
\curveto(116.64155671,498.47288196)(116.79155656,498.66288177)(116.93156221,498.85289022)
\curveto(116.97155638,498.90288153)(117.00655635,498.94788149)(117.03656221,498.98789022)
\curveto(117.07655628,499.0378814)(117.11155624,499.08788135)(117.14156221,499.13789022)
\curveto(117.1515562,499.15788128)(117.16155619,499.18288125)(117.17156221,499.21289022)
\curveto(117.19155616,499.24288119)(117.19155616,499.27288116)(117.17156221,499.30289022)
\curveto(117.1515562,499.36288107)(117.11655624,499.39788104)(117.06656221,499.40789022)
\curveto(117.01655634,499.42788101)(116.96655639,499.44788099)(116.91656221,499.46789022)
\lineto(116.81156221,499.46789022)
\curveto(116.77155658,499.47788096)(116.72155663,499.47788096)(116.66156221,499.46789022)
\lineto(116.51156221,499.46789022)
\lineto(115.91156221,499.46789022)
\lineto(113.27156221,499.46789022)
\lineto(112.53656221,499.46789022)
\lineto(112.29656221,499.46789022)
\curveto(112.22656113,499.47788096)(112.16656119,499.49288094)(112.11656221,499.51289022)
\curveto(112.02656133,499.55288088)(111.96656139,499.61288082)(111.93656221,499.69289022)
\curveto(111.88656147,499.79288064)(111.87156148,499.9378805)(111.89156221,500.12789022)
\curveto(111.91156144,500.32788011)(111.94656141,500.46287997)(111.99656221,500.53289022)
\curveto(112.01656134,500.55287988)(112.04156131,500.56787987)(112.07156221,500.57789022)
\lineto(112.19156221,500.63789022)
\curveto(112.21156114,500.6378798)(112.22656113,500.6328798)(112.23656221,500.62289022)
\curveto(112.2565611,500.62287981)(112.27656108,500.62787981)(112.29656221,500.63789022)
}
}
{
\newrgbcolor{curcolor}{0 0 0}
\pscustom[linestyle=none,fillstyle=solid,fillcolor=curcolor]
{
\newpath
\moveto(129.99117159,498.74789022)
\curveto(129.79116129,498.45788198)(129.5811615,498.17288226)(129.36117159,497.89289022)
\curveto(129.15116193,497.61288282)(128.94616213,497.32788311)(128.74617159,497.03789022)
\curveto(128.14616293,496.18788425)(127.54116354,495.34788509)(126.93117159,494.51789022)
\curveto(126.32116476,493.69788674)(125.71616536,492.86288757)(125.11617159,492.01289022)
\lineto(124.60617159,491.29289022)
\lineto(124.09617159,490.60289022)
\curveto(124.01616706,490.49288994)(123.93616714,490.37789006)(123.85617159,490.25789022)
\curveto(123.7761673,490.1378903)(123.6811674,490.04289039)(123.57117159,489.97289022)
\curveto(123.53116755,489.95289048)(123.46616761,489.9378905)(123.37617159,489.92789022)
\curveto(123.29616778,489.90789053)(123.20616787,489.89789054)(123.10617159,489.89789022)
\curveto(123.00616807,489.89789054)(122.91116817,489.90289053)(122.82117159,489.91289022)
\curveto(122.74116834,489.92289051)(122.6811684,489.94289049)(122.64117159,489.97289022)
\curveto(122.61116847,489.99289044)(122.58616849,490.02789041)(122.56617159,490.07789022)
\curveto(122.55616852,490.11789032)(122.56116852,490.16289027)(122.58117159,490.21289022)
\curveto(122.62116846,490.29289014)(122.66616841,490.36789007)(122.71617159,490.43789022)
\curveto(122.7761683,490.51788992)(122.83116825,490.59788984)(122.88117159,490.67789022)
\curveto(123.12116796,491.01788942)(123.36616771,491.35288908)(123.61617159,491.68289022)
\curveto(123.86616721,492.01288842)(124.10616697,492.34788809)(124.33617159,492.68789022)
\curveto(124.49616658,492.90788753)(124.65616642,493.12288731)(124.81617159,493.33289022)
\curveto(124.9761661,493.54288689)(125.13616594,493.75788668)(125.29617159,493.97789022)
\curveto(125.65616542,494.49788594)(126.02116506,495.00788543)(126.39117159,495.50789022)
\curveto(126.76116432,496.00788443)(127.13116395,496.51788392)(127.50117159,497.03789022)
\curveto(127.64116344,497.2378832)(127.7811633,497.432883)(127.92117159,497.62289022)
\curveto(128.07116301,497.81288262)(128.21616286,498.00788243)(128.35617159,498.20789022)
\curveto(128.56616251,498.50788193)(128.7811623,498.80788163)(129.00117159,499.10789022)
\lineto(129.66117159,500.00789022)
\lineto(129.84117159,500.27789022)
\lineto(130.05117159,500.54789022)
\lineto(130.17117159,500.72789022)
\curveto(130.22116086,500.78787965)(130.27116081,500.84287959)(130.32117159,500.89289022)
\curveto(130.39116069,500.94287949)(130.46616061,500.97787946)(130.54617159,500.99789022)
\curveto(130.56616051,501.00787943)(130.59116049,501.00787943)(130.62117159,500.99789022)
\curveto(130.66116042,500.99787944)(130.69116039,501.00787943)(130.71117159,501.02789022)
\curveto(130.83116025,501.02787941)(130.96616011,501.02287941)(131.11617159,501.01289022)
\curveto(131.26615981,501.01287942)(131.35615972,500.96787947)(131.38617159,500.87789022)
\curveto(131.40615967,500.84787959)(131.41115967,500.81287962)(131.40117159,500.77289022)
\curveto(131.39115969,500.7328797)(131.3761597,500.70287973)(131.35617159,500.68289022)
\curveto(131.31615976,500.60287983)(131.2761598,500.5328799)(131.23617159,500.47289022)
\curveto(131.19615988,500.41288002)(131.15115993,500.35288008)(131.10117159,500.29289022)
\lineto(130.53117159,499.51289022)
\curveto(130.35116073,499.26288117)(130.17116091,499.00788143)(129.99117159,498.74789022)
\moveto(123.13617159,494.84789022)
\curveto(123.08616799,494.86788557)(123.03616804,494.87288556)(122.98617159,494.86289022)
\curveto(122.93616814,494.85288558)(122.88616819,494.85788558)(122.83617159,494.87789022)
\curveto(122.72616835,494.89788554)(122.62116846,494.91788552)(122.52117159,494.93789022)
\curveto(122.43116865,494.96788547)(122.33616874,495.00788543)(122.23617159,495.05789022)
\curveto(121.90616917,495.19788524)(121.65116943,495.39288504)(121.47117159,495.64289022)
\curveto(121.29116979,495.90288453)(121.14616993,496.21288422)(121.03617159,496.57289022)
\curveto(121.00617007,496.65288378)(120.98617009,496.7328837)(120.97617159,496.81289022)
\curveto(120.96617011,496.90288353)(120.95117013,496.98788345)(120.93117159,497.06789022)
\curveto(120.92117016,497.11788332)(120.91617016,497.18288325)(120.91617159,497.26289022)
\curveto(120.90617017,497.29288314)(120.90117018,497.32288311)(120.90117159,497.35289022)
\curveto(120.90117018,497.39288304)(120.89617018,497.42788301)(120.88617159,497.45789022)
\lineto(120.88617159,497.60789022)
\curveto(120.8761702,497.65788278)(120.87117021,497.71788272)(120.87117159,497.78789022)
\curveto(120.87117021,497.86788257)(120.8761702,497.9328825)(120.88617159,497.98289022)
\lineto(120.88617159,498.14789022)
\curveto(120.90617017,498.19788224)(120.91117017,498.24288219)(120.90117159,498.28289022)
\curveto(120.90117018,498.3328821)(120.90617017,498.37788206)(120.91617159,498.41789022)
\curveto(120.92617015,498.45788198)(120.93117015,498.49288194)(120.93117159,498.52289022)
\curveto(120.93117015,498.56288187)(120.93617014,498.60288183)(120.94617159,498.64289022)
\curveto(120.9761701,498.75288168)(120.99617008,498.86288157)(121.00617159,498.97289022)
\curveto(121.02617005,499.09288134)(121.06117002,499.20788123)(121.11117159,499.31789022)
\curveto(121.25116983,499.65788078)(121.41116967,499.9328805)(121.59117159,500.14289022)
\curveto(121.7811693,500.36288007)(122.05116903,500.54287989)(122.40117159,500.68289022)
\curveto(122.4811686,500.71287972)(122.56616851,500.7328797)(122.65617159,500.74289022)
\curveto(122.74616833,500.76287967)(122.84116824,500.78287965)(122.94117159,500.80289022)
\curveto(122.97116811,500.81287962)(123.02616805,500.81287962)(123.10617159,500.80289022)
\curveto(123.18616789,500.80287963)(123.23616784,500.81287962)(123.25617159,500.83289022)
\curveto(123.81616726,500.84287959)(124.26616681,500.7328797)(124.60617159,500.50289022)
\curveto(124.95616612,500.27288016)(125.21616586,499.96788047)(125.38617159,499.58789022)
\curveto(125.42616565,499.49788094)(125.46116562,499.40288103)(125.49117159,499.30289022)
\curveto(125.52116556,499.20288123)(125.54616553,499.10288133)(125.56617159,499.00289022)
\curveto(125.58616549,498.97288146)(125.59116549,498.94288149)(125.58117159,498.91289022)
\curveto(125.5811655,498.88288155)(125.58616549,498.85288158)(125.59617159,498.82289022)
\curveto(125.62616545,498.71288172)(125.64616543,498.58788185)(125.65617159,498.44789022)
\curveto(125.66616541,498.31788212)(125.6761654,498.18288225)(125.68617159,498.04289022)
\lineto(125.68617159,497.87789022)
\curveto(125.69616538,497.81788262)(125.69616538,497.76288267)(125.68617159,497.71289022)
\curveto(125.6761654,497.66288277)(125.67116541,497.61288282)(125.67117159,497.56289022)
\lineto(125.67117159,497.42789022)
\curveto(125.66116542,497.38788305)(125.65616542,497.34788309)(125.65617159,497.30789022)
\curveto(125.66616541,497.26788317)(125.66116542,497.22288321)(125.64117159,497.17289022)
\curveto(125.62116546,497.06288337)(125.60116548,496.95788348)(125.58117159,496.85789022)
\curveto(125.57116551,496.75788368)(125.55116553,496.65788378)(125.52117159,496.55789022)
\curveto(125.39116569,496.19788424)(125.22616585,495.88288455)(125.02617159,495.61289022)
\curveto(124.82616625,495.34288509)(124.55116653,495.1378853)(124.20117159,494.99789022)
\curveto(124.12116696,494.96788547)(124.03616704,494.94288549)(123.94617159,494.92289022)
\lineto(123.67617159,494.86289022)
\curveto(123.62616745,494.85288558)(123.5811675,494.84788559)(123.54117159,494.84789022)
\curveto(123.50116758,494.85788558)(123.46116762,494.85788558)(123.42117159,494.84789022)
\curveto(123.32116776,494.82788561)(123.22616785,494.82788561)(123.13617159,494.84789022)
\moveto(122.29617159,496.24289022)
\curveto(122.33616874,496.17288426)(122.3761687,496.10788433)(122.41617159,496.04789022)
\curveto(122.45616862,495.99788444)(122.50616857,495.94788449)(122.56617159,495.89789022)
\lineto(122.71617159,495.77789022)
\curveto(122.7761683,495.74788469)(122.84116824,495.72288471)(122.91117159,495.70289022)
\curveto(122.95116813,495.68288475)(122.98616809,495.67288476)(123.01617159,495.67289022)
\curveto(123.05616802,495.68288475)(123.09616798,495.67788476)(123.13617159,495.65789022)
\curveto(123.16616791,495.65788478)(123.20616787,495.65288478)(123.25617159,495.64289022)
\curveto(123.30616777,495.64288479)(123.34616773,495.64788479)(123.37617159,495.65789022)
\lineto(123.60117159,495.70289022)
\curveto(123.85116723,495.78288465)(124.03616704,495.90788453)(124.15617159,496.07789022)
\curveto(124.23616684,496.17788426)(124.30616677,496.30788413)(124.36617159,496.46789022)
\curveto(124.44616663,496.64788379)(124.50616657,496.87288356)(124.54617159,497.14289022)
\curveto(124.58616649,497.42288301)(124.60116648,497.70288273)(124.59117159,497.98289022)
\curveto(124.5811665,498.27288216)(124.55116653,498.54788189)(124.50117159,498.80789022)
\curveto(124.45116663,499.06788137)(124.3761667,499.27788116)(124.27617159,499.43789022)
\curveto(124.15616692,499.6378808)(124.00616707,499.78788065)(123.82617159,499.88789022)
\curveto(123.74616733,499.9378805)(123.65616742,499.96788047)(123.55617159,499.97789022)
\curveto(123.45616762,499.99788044)(123.35116773,500.00788043)(123.24117159,500.00789022)
\curveto(123.22116786,499.99788044)(123.19616788,499.99288044)(123.16617159,499.99289022)
\curveto(123.14616793,500.00288043)(123.12616795,500.00288043)(123.10617159,499.99289022)
\curveto(123.05616802,499.98288045)(123.01116807,499.97288046)(122.97117159,499.96289022)
\curveto(122.93116815,499.96288047)(122.89116819,499.95288048)(122.85117159,499.93289022)
\curveto(122.67116841,499.85288058)(122.52116856,499.7328807)(122.40117159,499.57289022)
\curveto(122.29116879,499.41288102)(122.20116888,499.2328812)(122.13117159,499.03289022)
\curveto(122.07116901,498.84288159)(122.02616905,498.61788182)(121.99617159,498.35789022)
\curveto(121.9761691,498.09788234)(121.97116911,497.8328826)(121.98117159,497.56289022)
\curveto(121.99116909,497.30288313)(122.02116906,497.05288338)(122.07117159,496.81289022)
\curveto(122.13116895,496.58288385)(122.20616887,496.39288404)(122.29617159,496.24289022)
\moveto(133.09617159,493.25789022)
\curveto(133.10615797,493.20788723)(133.11115797,493.11788732)(133.11117159,492.98789022)
\curveto(133.11115797,492.85788758)(133.10115798,492.76788767)(133.08117159,492.71789022)
\curveto(133.06115802,492.66788777)(133.05615802,492.61288782)(133.06617159,492.55289022)
\curveto(133.076158,492.50288793)(133.076158,492.45288798)(133.06617159,492.40289022)
\curveto(133.02615805,492.26288817)(132.99615808,492.12788831)(132.97617159,491.99789022)
\curveto(132.96615811,491.86788857)(132.93615814,491.74788869)(132.88617159,491.63789022)
\curveto(132.74615833,491.28788915)(132.5811585,490.99288944)(132.39117159,490.75289022)
\curveto(132.20115888,490.52288991)(131.93115915,490.3378901)(131.58117159,490.19789022)
\curveto(131.50115958,490.16789027)(131.41615966,490.14789029)(131.32617159,490.13789022)
\curveto(131.23615984,490.11789032)(131.15115993,490.09789034)(131.07117159,490.07789022)
\curveto(131.02116006,490.06789037)(130.97116011,490.06289037)(130.92117159,490.06289022)
\curveto(130.87116021,490.06289037)(130.82116026,490.05789038)(130.77117159,490.04789022)
\curveto(130.74116034,490.0378904)(130.69116039,490.0378904)(130.62117159,490.04789022)
\curveto(130.55116053,490.04789039)(130.50116058,490.05289038)(130.47117159,490.06289022)
\curveto(130.41116067,490.08289035)(130.35116073,490.09289034)(130.29117159,490.09289022)
\curveto(130.24116084,490.08289035)(130.19116089,490.08789035)(130.14117159,490.10789022)
\curveto(130.05116103,490.12789031)(129.96116112,490.15289028)(129.87117159,490.18289022)
\curveto(129.79116129,490.20289023)(129.71116137,490.2328902)(129.63117159,490.27289022)
\curveto(129.31116177,490.41289002)(129.06116202,490.60788983)(128.88117159,490.85789022)
\curveto(128.70116238,491.11788932)(128.55116253,491.42288901)(128.43117159,491.77289022)
\curveto(128.41116267,491.85288858)(128.39616268,491.9378885)(128.38617159,492.02789022)
\curveto(128.3761627,492.11788832)(128.36116272,492.20288823)(128.34117159,492.28289022)
\curveto(128.33116275,492.31288812)(128.32616275,492.34288809)(128.32617159,492.37289022)
\lineto(128.32617159,492.47789022)
\curveto(128.30616277,492.55788788)(128.29616278,492.6378878)(128.29617159,492.71789022)
\lineto(128.29617159,492.85289022)
\curveto(128.2761628,492.95288748)(128.2761628,493.05288738)(128.29617159,493.15289022)
\lineto(128.29617159,493.33289022)
\curveto(128.30616277,493.38288705)(128.31116277,493.42788701)(128.31117159,493.46789022)
\curveto(128.31116277,493.51788692)(128.31616276,493.56288687)(128.32617159,493.60289022)
\curveto(128.33616274,493.64288679)(128.34116274,493.67788676)(128.34117159,493.70789022)
\curveto(128.34116274,493.74788669)(128.34616273,493.78788665)(128.35617159,493.82789022)
\lineto(128.41617159,494.15789022)
\curveto(128.43616264,494.27788616)(128.46616261,494.38788605)(128.50617159,494.48789022)
\curveto(128.64616243,494.81788562)(128.80616227,495.09288534)(128.98617159,495.31289022)
\curveto(129.1761619,495.54288489)(129.43616164,495.72788471)(129.76617159,495.86789022)
\curveto(129.84616123,495.90788453)(129.93116115,495.9328845)(130.02117159,495.94289022)
\lineto(130.32117159,496.00289022)
\lineto(130.45617159,496.00289022)
\curveto(130.50616057,496.01288442)(130.55616052,496.01788442)(130.60617159,496.01789022)
\curveto(131.1761599,496.0378844)(131.63615944,495.9328845)(131.98617159,495.70289022)
\curveto(132.34615873,495.48288495)(132.61115847,495.18288525)(132.78117159,494.80289022)
\curveto(132.83115825,494.70288573)(132.87115821,494.60288583)(132.90117159,494.50289022)
\curveto(132.93115815,494.40288603)(132.96115812,494.29788614)(132.99117159,494.18789022)
\curveto(133.00115808,494.14788629)(133.00615807,494.11288632)(133.00617159,494.08289022)
\curveto(133.00615807,494.06288637)(133.01115807,494.0328864)(133.02117159,493.99289022)
\curveto(133.04115804,493.92288651)(133.05115803,493.84788659)(133.05117159,493.76789022)
\curveto(133.05115803,493.68788675)(133.06115802,493.60788683)(133.08117159,493.52789022)
\curveto(133.081158,493.47788696)(133.081158,493.432887)(133.08117159,493.39289022)
\curveto(133.081158,493.35288708)(133.08615799,493.30788713)(133.09617159,493.25789022)
\moveto(131.98617159,492.82289022)
\curveto(131.99615908,492.87288756)(132.00115908,492.94788749)(132.00117159,493.04789022)
\curveto(132.01115907,493.14788729)(132.00615907,493.22288721)(131.98617159,493.27289022)
\curveto(131.96615911,493.3328871)(131.96115912,493.38788705)(131.97117159,493.43789022)
\curveto(131.99115909,493.49788694)(131.99115909,493.55788688)(131.97117159,493.61789022)
\curveto(131.96115912,493.64788679)(131.95615912,493.68288675)(131.95617159,493.72289022)
\curveto(131.95615912,493.76288667)(131.95115913,493.80288663)(131.94117159,493.84289022)
\curveto(131.92115916,493.92288651)(131.90115918,493.99788644)(131.88117159,494.06789022)
\curveto(131.87115921,494.14788629)(131.85615922,494.22788621)(131.83617159,494.30789022)
\curveto(131.80615927,494.36788607)(131.7811593,494.42788601)(131.76117159,494.48789022)
\curveto(131.74115934,494.54788589)(131.71115937,494.60788583)(131.67117159,494.66789022)
\curveto(131.57115951,494.8378856)(131.44115964,494.97288546)(131.28117159,495.07289022)
\curveto(131.20115988,495.12288531)(131.10615997,495.15788528)(130.99617159,495.17789022)
\curveto(130.88616019,495.19788524)(130.76116032,495.20788523)(130.62117159,495.20789022)
\curveto(130.60116048,495.19788524)(130.5761605,495.19288524)(130.54617159,495.19289022)
\curveto(130.51616056,495.20288523)(130.48616059,495.20288523)(130.45617159,495.19289022)
\lineto(130.30617159,495.13289022)
\curveto(130.25616082,495.12288531)(130.21116087,495.10788533)(130.17117159,495.08789022)
\curveto(129.9811611,494.97788546)(129.83616124,494.8328856)(129.73617159,494.65289022)
\curveto(129.64616143,494.47288596)(129.56616151,494.26788617)(129.49617159,494.03789022)
\curveto(129.45616162,493.90788653)(129.43616164,493.77288666)(129.43617159,493.63289022)
\curveto(129.43616164,493.50288693)(129.42616165,493.35788708)(129.40617159,493.19789022)
\curveto(129.39616168,493.14788729)(129.38616169,493.08788735)(129.37617159,493.01789022)
\curveto(129.3761617,492.94788749)(129.38616169,492.88788755)(129.40617159,492.83789022)
\lineto(129.40617159,492.67289022)
\lineto(129.40617159,492.49289022)
\curveto(129.41616166,492.44288799)(129.42616165,492.38788805)(129.43617159,492.32789022)
\curveto(129.44616163,492.27788816)(129.45116163,492.22288821)(129.45117159,492.16289022)
\curveto(129.46116162,492.10288833)(129.4761616,492.04788839)(129.49617159,491.99789022)
\curveto(129.54616153,491.80788863)(129.60616147,491.6328888)(129.67617159,491.47289022)
\curveto(129.74616133,491.31288912)(129.85116123,491.18288925)(129.99117159,491.08289022)
\curveto(130.12116096,490.98288945)(130.26116082,490.91288952)(130.41117159,490.87289022)
\curveto(130.44116064,490.86288957)(130.46616061,490.85788958)(130.48617159,490.85789022)
\curveto(130.51616056,490.86788957)(130.54616053,490.86788957)(130.57617159,490.85789022)
\curveto(130.59616048,490.85788958)(130.62616045,490.85288958)(130.66617159,490.84289022)
\curveto(130.70616037,490.84288959)(130.74116034,490.84788959)(130.77117159,490.85789022)
\curveto(130.81116027,490.86788957)(130.85116023,490.87288956)(130.89117159,490.87289022)
\curveto(130.93116015,490.87288956)(130.97116011,490.88288955)(131.01117159,490.90289022)
\curveto(131.25115983,490.98288945)(131.44615963,491.11788932)(131.59617159,491.30789022)
\curveto(131.71615936,491.48788895)(131.80615927,491.69288874)(131.86617159,491.92289022)
\curveto(131.88615919,491.99288844)(131.90115918,492.06288837)(131.91117159,492.13289022)
\curveto(131.92115916,492.21288822)(131.93615914,492.29288814)(131.95617159,492.37289022)
\curveto(131.95615912,492.432888)(131.96115912,492.47788796)(131.97117159,492.50789022)
\curveto(131.97115911,492.52788791)(131.97115911,492.55288788)(131.97117159,492.58289022)
\curveto(131.97115911,492.62288781)(131.9761591,492.65288778)(131.98617159,492.67289022)
\lineto(131.98617159,492.82289022)
}
}
{
\newrgbcolor{curcolor}{0 0 0}
\pscustom[linestyle=none,fillstyle=solid,fillcolor=curcolor]
{
\newpath
\moveto(406.14953951,709.52799154)
\lineto(407.43953951,709.52799154)
\curveto(407.54953668,709.52798086)(407.65453658,709.52298087)(407.75453951,709.51299154)
\curveto(407.85453638,709.51298088)(407.9295363,709.47798091)(407.97953951,709.40799154)
\curveto(408.0295362,709.33798105)(408.05453618,709.24798114)(408.05453951,709.13799154)
\curveto(408.06453617,709.02798136)(408.06953616,708.90798148)(408.06953951,708.77799154)
\lineto(408.06953951,707.47299154)
\lineto(408.06953951,702.26799154)
\lineto(408.06953951,699.80799154)
\lineto(408.06953951,699.37299154)
\curveto(408.07953615,699.21299118)(408.05953617,699.0929913)(408.00953951,699.01299154)
\curveto(407.96953626,698.94299145)(407.87953635,698.8879915)(407.73953951,698.84799154)
\curveto(407.66953656,698.82799156)(407.59453664,698.82299157)(407.51453951,698.83299154)
\curveto(407.4345368,698.84299155)(407.35453688,698.84799154)(407.27453951,698.84799154)
\lineto(406.38953951,698.84799154)
\curveto(406.27953795,698.84799154)(406.17453806,698.85299154)(406.07453951,698.86299154)
\curveto(405.98453825,698.87299152)(405.90953832,698.90299149)(405.84953951,698.95299154)
\curveto(405.79953843,699.00299139)(405.76953846,699.07799131)(405.75953951,699.17799154)
\curveto(405.74953848,699.27799111)(405.74453849,699.38299101)(405.74453951,699.49299154)
\lineto(405.74453951,700.79799154)
\lineto(405.74453951,706.27299154)
\lineto(405.74453951,708.46299154)
\curveto(405.74453849,708.60298179)(405.73953849,708.76798162)(405.72953951,708.95799154)
\curveto(405.7295385,709.14798124)(405.75453848,709.28298111)(405.80453951,709.36299154)
\curveto(405.84453839,709.42298097)(405.90953832,709.47298092)(405.99953951,709.51299154)
\curveto(406.0295382,709.51298088)(406.05453818,709.51298088)(406.07453951,709.51299154)
\curveto(406.10453813,709.52298087)(406.1295381,709.52798086)(406.14953951,709.52799154)
}
}
{
\newrgbcolor{curcolor}{0 0 0}
\pscustom[linestyle=none,fillstyle=solid,fillcolor=curcolor]
{
\newpath
\moveto(414.35336763,706.78299154)
\curveto(414.95336183,706.80298359)(415.45336133,706.71798367)(415.85336763,706.52799154)
\curveto(416.25336053,706.33798405)(416.56836021,706.05798433)(416.79836763,705.68799154)
\curveto(416.86835991,705.57798481)(416.92335986,705.45798493)(416.96336763,705.32799154)
\curveto(417.00335978,705.20798518)(417.04335974,705.08298531)(417.08336763,704.95299154)
\curveto(417.10335968,704.87298552)(417.11335967,704.79798559)(417.11336763,704.72799154)
\curveto(417.12335966,704.65798573)(417.13835964,704.5879858)(417.15836763,704.51799154)
\curveto(417.15835962,704.45798593)(417.16335962,704.41798597)(417.17336763,704.39799154)
\curveto(417.19335959,704.25798613)(417.20335958,704.11298628)(417.20336763,703.96299154)
\lineto(417.20336763,703.52799154)
\lineto(417.20336763,702.19299154)
\lineto(417.20336763,699.76299154)
\curveto(417.20335958,699.57299082)(417.19835958,699.387991)(417.18836763,699.20799154)
\curveto(417.18835959,699.03799135)(417.11835966,698.92799146)(416.97836763,698.87799154)
\curveto(416.91835986,698.85799153)(416.84835993,698.84799154)(416.76836763,698.84799154)
\lineto(416.52836763,698.84799154)
\lineto(415.71836763,698.84799154)
\curveto(415.59836118,698.84799154)(415.48836129,698.85299154)(415.38836763,698.86299154)
\curveto(415.29836148,698.88299151)(415.22836155,698.92799146)(415.17836763,698.99799154)
\curveto(415.13836164,699.05799133)(415.11336167,699.13299126)(415.10336763,699.22299154)
\lineto(415.10336763,699.53799154)
\lineto(415.10336763,700.58799154)
\lineto(415.10336763,702.82299154)
\curveto(415.10336168,703.1929872)(415.08836169,703.53298686)(415.05836763,703.84299154)
\curveto(415.02836175,704.16298623)(414.93836184,704.43298596)(414.78836763,704.65299154)
\curveto(414.64836213,704.85298554)(414.44336234,704.9929854)(414.17336763,705.07299154)
\curveto(414.12336266,705.0929853)(414.06836271,705.10298529)(414.00836763,705.10299154)
\curveto(413.95836282,705.10298529)(413.90336288,705.11298528)(413.84336763,705.13299154)
\curveto(413.79336299,705.14298525)(413.72836305,705.14298525)(413.64836763,705.13299154)
\curveto(413.5783632,705.13298526)(413.52336326,705.12798526)(413.48336763,705.11799154)
\curveto(413.44336334,705.10798528)(413.40836337,705.10298529)(413.37836763,705.10299154)
\curveto(413.34836343,705.10298529)(413.31836346,705.09798529)(413.28836763,705.08799154)
\curveto(413.05836372,705.02798536)(412.87336391,704.94798544)(412.73336763,704.84799154)
\curveto(412.41336437,704.61798577)(412.22336456,704.28298611)(412.16336763,703.84299154)
\curveto(412.10336468,703.40298699)(412.07336471,702.90798748)(412.07336763,702.35799154)
\lineto(412.07336763,700.48299154)
\lineto(412.07336763,699.56799154)
\lineto(412.07336763,699.29799154)
\curveto(412.07336471,699.20799118)(412.05836472,699.13299126)(412.02836763,699.07299154)
\curveto(411.9783648,698.96299143)(411.89836488,698.89799149)(411.78836763,698.87799154)
\curveto(411.6783651,698.85799153)(411.54336524,698.84799154)(411.38336763,698.84799154)
\lineto(410.63336763,698.84799154)
\curveto(410.52336626,698.84799154)(410.41336637,698.85299154)(410.30336763,698.86299154)
\curveto(410.19336659,698.87299152)(410.11336667,698.90799148)(410.06336763,698.96799154)
\curveto(409.99336679,699.05799133)(409.95836682,699.1879912)(409.95836763,699.35799154)
\curveto(409.96836681,699.52799086)(409.97336681,699.6879907)(409.97336763,699.83799154)
\lineto(409.97336763,701.87799154)
\lineto(409.97336763,705.17799154)
\lineto(409.97336763,705.94299154)
\lineto(409.97336763,706.24299154)
\curveto(409.9833668,706.33298406)(410.01336677,706.40798398)(410.06336763,706.46799154)
\curveto(410.0833667,706.49798389)(410.11336667,706.51798387)(410.15336763,706.52799154)
\curveto(410.20336658,706.54798384)(410.25336653,706.56298383)(410.30336763,706.57299154)
\lineto(410.37836763,706.57299154)
\curveto(410.42836635,706.58298381)(410.4783663,706.5879838)(410.52836763,706.58799154)
\lineto(410.69336763,706.58799154)
\lineto(411.32336763,706.58799154)
\curveto(411.40336538,706.5879838)(411.4783653,706.58298381)(411.54836763,706.57299154)
\curveto(411.62836515,706.57298382)(411.69836508,706.56298383)(411.75836763,706.54299154)
\curveto(411.82836495,706.51298388)(411.87336491,706.46798392)(411.89336763,706.40799154)
\curveto(411.92336486,706.34798404)(411.94836483,706.27798411)(411.96836763,706.19799154)
\curveto(411.9783648,706.15798423)(411.9783648,706.12298427)(411.96836763,706.09299154)
\curveto(411.96836481,706.06298433)(411.9783648,706.03298436)(411.99836763,706.00299154)
\curveto(412.01836476,705.95298444)(412.03336475,705.92298447)(412.04336763,705.91299154)
\curveto(412.06336472,705.90298449)(412.08836469,705.8879845)(412.11836763,705.86799154)
\curveto(412.22836455,705.85798453)(412.31836446,705.8929845)(412.38836763,705.97299154)
\curveto(412.45836432,706.06298433)(412.53336425,706.13298426)(412.61336763,706.18299154)
\curveto(412.8833639,706.38298401)(413.1833636,706.54298385)(413.51336763,706.66299154)
\curveto(413.60336318,706.6929837)(413.69336309,706.71298368)(413.78336763,706.72299154)
\curveto(413.8833629,706.73298366)(413.98836279,706.74798364)(414.09836763,706.76799154)
\curveto(414.12836265,706.77798361)(414.17336261,706.77798361)(414.23336763,706.76799154)
\curveto(414.29336249,706.76798362)(414.33336245,706.77298362)(414.35336763,706.78299154)
}
}
{
\newrgbcolor{curcolor}{0 0 0}
\pscustom[linestyle=none,fillstyle=solid,fillcolor=curcolor]
{
\newpath
\moveto(426.42461763,699.70299154)
\lineto(426.42461763,699.28299154)
\curveto(426.42460926,699.15299124)(426.39460929,699.04799134)(426.33461763,698.96799154)
\curveto(426.2846094,698.91799147)(426.21960947,698.88299151)(426.13961763,698.86299154)
\curveto(426.05960963,698.85299154)(425.96960972,698.84799154)(425.86961763,698.84799154)
\lineto(425.04461763,698.84799154)
\lineto(424.75961763,698.84799154)
\curveto(424.67961101,698.85799153)(424.61461107,698.88299151)(424.56461763,698.92299154)
\curveto(424.49461119,698.97299142)(424.45461123,699.03799135)(424.44461763,699.11799154)
\curveto(424.43461125,699.19799119)(424.41461127,699.27799111)(424.38461763,699.35799154)
\curveto(424.36461132,699.37799101)(424.34461134,699.392991)(424.32461763,699.40299154)
\curveto(424.31461137,699.42299097)(424.29961139,699.44299095)(424.27961763,699.46299154)
\curveto(424.16961152,699.46299093)(424.0896116,699.43799095)(424.03961763,699.38799154)
\lineto(423.88961763,699.23799154)
\curveto(423.81961187,699.1879912)(423.75461193,699.14299125)(423.69461763,699.10299154)
\curveto(423.63461205,699.07299132)(423.56961212,699.03299136)(423.49961763,698.98299154)
\curveto(423.45961223,698.96299143)(423.41461227,698.94299145)(423.36461763,698.92299154)
\curveto(423.32461236,698.90299149)(423.27961241,698.88299151)(423.22961763,698.86299154)
\curveto(423.0896126,698.81299158)(422.93961275,698.76799162)(422.77961763,698.72799154)
\curveto(422.72961296,698.70799168)(422.684613,698.69799169)(422.64461763,698.69799154)
\curveto(422.60461308,698.69799169)(422.56461312,698.6929917)(422.52461763,698.68299154)
\lineto(422.38961763,698.68299154)
\curveto(422.35961333,698.67299172)(422.31961337,698.66799172)(422.26961763,698.66799154)
\lineto(422.13461763,698.66799154)
\curveto(422.07461361,698.64799174)(421.9846137,698.64299175)(421.86461763,698.65299154)
\curveto(421.74461394,698.65299174)(421.65961403,698.66299173)(421.60961763,698.68299154)
\curveto(421.53961415,698.70299169)(421.47461421,698.71299168)(421.41461763,698.71299154)
\curveto(421.36461432,698.70299169)(421.30961438,698.70799168)(421.24961763,698.72799154)
\lineto(420.88961763,698.84799154)
\curveto(420.77961491,698.87799151)(420.66961502,698.91799147)(420.55961763,698.96799154)
\curveto(420.20961548,699.11799127)(419.89461579,699.34799104)(419.61461763,699.65799154)
\curveto(419.34461634,699.97799041)(419.12961656,700.31299008)(418.96961763,700.66299154)
\curveto(418.91961677,700.77298962)(418.87961681,700.87798951)(418.84961763,700.97799154)
\curveto(418.81961687,701.0879893)(418.7846169,701.19798919)(418.74461763,701.30799154)
\curveto(418.73461695,701.34798904)(418.72961696,701.38298901)(418.72961763,701.41299154)
\curveto(418.72961696,701.45298894)(418.71961697,701.49798889)(418.69961763,701.54799154)
\curveto(418.67961701,701.62798876)(418.65961703,701.71298868)(418.63961763,701.80299154)
\curveto(418.62961706,701.90298849)(418.61461707,702.00298839)(418.59461763,702.10299154)
\curveto(418.5846171,702.13298826)(418.57961711,702.16798822)(418.57961763,702.20799154)
\curveto(418.5896171,702.24798814)(418.5896171,702.28298811)(418.57961763,702.31299154)
\lineto(418.57961763,702.44799154)
\curveto(418.57961711,702.49798789)(418.57461711,702.54798784)(418.56461763,702.59799154)
\curveto(418.55461713,702.64798774)(418.54961714,702.70298769)(418.54961763,702.76299154)
\curveto(418.54961714,702.83298756)(418.55461713,702.8879875)(418.56461763,702.92799154)
\curveto(418.57461711,702.97798741)(418.57961711,703.02298737)(418.57961763,703.06299154)
\lineto(418.57961763,703.21299154)
\curveto(418.5896171,703.26298713)(418.5896171,703.30798708)(418.57961763,703.34799154)
\curveto(418.57961711,703.39798699)(418.5896171,703.44798694)(418.60961763,703.49799154)
\curveto(418.62961706,703.60798678)(418.64461704,703.71298668)(418.65461763,703.81299154)
\curveto(418.67461701,703.91298648)(418.69961699,704.01298638)(418.72961763,704.11299154)
\curveto(418.76961692,704.23298616)(418.80461688,704.34798604)(418.83461763,704.45799154)
\curveto(418.86461682,704.56798582)(418.90461678,704.67798571)(418.95461763,704.78799154)
\curveto(419.09461659,705.0879853)(419.26961642,705.37298502)(419.47961763,705.64299154)
\curveto(419.49961619,705.67298472)(419.52461616,705.69798469)(419.55461763,705.71799154)
\curveto(419.59461609,705.74798464)(419.62461606,705.77798461)(419.64461763,705.80799154)
\curveto(419.684616,705.85798453)(419.72461596,705.90298449)(419.76461763,705.94299154)
\curveto(419.80461588,705.98298441)(419.84961584,706.02298437)(419.89961763,706.06299154)
\curveto(419.93961575,706.08298431)(419.97461571,706.10798428)(420.00461763,706.13799154)
\curveto(420.03461565,706.17798421)(420.06961562,706.20798418)(420.10961763,706.22799154)
\curveto(420.35961533,706.39798399)(420.64961504,706.53798385)(420.97961763,706.64799154)
\curveto(421.04961464,706.66798372)(421.11961457,706.68298371)(421.18961763,706.69299154)
\curveto(421.26961442,706.70298369)(421.34961434,706.71798367)(421.42961763,706.73799154)
\curveto(421.49961419,706.75798363)(421.5896141,706.76798362)(421.69961763,706.76799154)
\curveto(421.80961388,706.77798361)(421.91961377,706.78298361)(422.02961763,706.78299154)
\curveto(422.13961355,706.78298361)(422.24461344,706.77798361)(422.34461763,706.76799154)
\curveto(422.45461323,706.75798363)(422.54461314,706.74298365)(422.61461763,706.72299154)
\curveto(422.76461292,706.67298372)(422.90961278,706.62798376)(423.04961763,706.58799154)
\curveto(423.1896125,706.54798384)(423.31961237,706.4929839)(423.43961763,706.42299154)
\curveto(423.50961218,706.37298402)(423.57461211,706.32298407)(423.63461763,706.27299154)
\curveto(423.69461199,706.23298416)(423.75961193,706.1879842)(423.82961763,706.13799154)
\curveto(423.86961182,706.10798428)(423.92461176,706.06798432)(423.99461763,706.01799154)
\curveto(424.07461161,705.96798442)(424.14961154,705.96798442)(424.21961763,706.01799154)
\curveto(424.25961143,706.03798435)(424.27961141,706.07298432)(424.27961763,706.12299154)
\curveto(424.27961141,706.17298422)(424.2896114,706.22298417)(424.30961763,706.27299154)
\lineto(424.30961763,706.42299154)
\curveto(424.31961137,706.45298394)(424.32461136,706.4879839)(424.32461763,706.52799154)
\lineto(424.32461763,706.64799154)
\lineto(424.32461763,708.68799154)
\curveto(424.32461136,708.79798159)(424.31961137,708.91798147)(424.30961763,709.04799154)
\curveto(424.30961138,709.1879812)(424.33461135,709.2929811)(424.38461763,709.36299154)
\curveto(424.42461126,709.44298095)(424.49961119,709.4929809)(424.60961763,709.51299154)
\curveto(424.62961106,709.52298087)(424.64961104,709.52298087)(424.66961763,709.51299154)
\curveto(424.689611,709.51298088)(424.70961098,709.51798087)(424.72961763,709.52799154)
\lineto(425.79461763,709.52799154)
\curveto(425.91460977,709.52798086)(426.02460966,709.52298087)(426.12461763,709.51299154)
\curveto(426.22460946,709.50298089)(426.29960939,709.46298093)(426.34961763,709.39299154)
\curveto(426.39960929,709.31298108)(426.42460926,709.20798118)(426.42461763,709.07799154)
\lineto(426.42461763,708.71799154)
\lineto(426.42461763,699.70299154)
\moveto(424.38461763,702.64299154)
\curveto(424.39461129,702.68298771)(424.39461129,702.72298767)(424.38461763,702.76299154)
\lineto(424.38461763,702.89799154)
\curveto(424.3846113,702.99798739)(424.37961131,703.09798729)(424.36961763,703.19799154)
\curveto(424.35961133,703.29798709)(424.34461134,703.387987)(424.32461763,703.46799154)
\curveto(424.30461138,703.57798681)(424.2846114,703.67798671)(424.26461763,703.76799154)
\curveto(424.25461143,703.85798653)(424.22961146,703.94298645)(424.18961763,704.02299154)
\curveto(424.04961164,704.38298601)(423.84461184,704.66798572)(423.57461763,704.87799154)
\curveto(423.31461237,705.0879853)(422.93461275,705.1929852)(422.43461763,705.19299154)
\curveto(422.37461331,705.1929852)(422.29461339,705.18298521)(422.19461763,705.16299154)
\curveto(422.11461357,705.14298525)(422.03961365,705.12298527)(421.96961763,705.10299154)
\curveto(421.90961378,705.0929853)(421.84961384,705.07298532)(421.78961763,705.04299154)
\curveto(421.51961417,704.93298546)(421.30961438,704.76298563)(421.15961763,704.53299154)
\curveto(421.00961468,704.30298609)(420.8896148,704.04298635)(420.79961763,703.75299154)
\curveto(420.76961492,703.65298674)(420.74961494,703.55298684)(420.73961763,703.45299154)
\curveto(420.72961496,703.35298704)(420.70961498,703.24798714)(420.67961763,703.13799154)
\lineto(420.67961763,702.92799154)
\curveto(420.65961503,702.83798755)(420.65461503,702.71298768)(420.66461763,702.55299154)
\curveto(420.67461501,702.40298799)(420.689615,702.2929881)(420.70961763,702.22299154)
\lineto(420.70961763,702.13299154)
\curveto(420.71961497,702.11298828)(420.72461496,702.0929883)(420.72461763,702.07299154)
\curveto(420.74461494,701.9929884)(420.75961493,701.91798847)(420.76961763,701.84799154)
\curveto(420.7896149,701.77798861)(420.80961488,701.70298869)(420.82961763,701.62299154)
\curveto(420.99961469,701.10298929)(421.2896144,700.71798967)(421.69961763,700.46799154)
\curveto(421.82961386,700.37799001)(422.00961368,700.30799008)(422.23961763,700.25799154)
\curveto(422.27961341,700.24799014)(422.33961335,700.24299015)(422.41961763,700.24299154)
\curveto(422.44961324,700.23299016)(422.49461319,700.22299017)(422.55461763,700.21299154)
\curveto(422.62461306,700.21299018)(422.67961301,700.21799017)(422.71961763,700.22799154)
\curveto(422.79961289,700.24799014)(422.87961281,700.26299013)(422.95961763,700.27299154)
\curveto(423.03961265,700.28299011)(423.11961257,700.30299009)(423.19961763,700.33299154)
\curveto(423.44961224,700.44298995)(423.64961204,700.58298981)(423.79961763,700.75299154)
\curveto(423.94961174,700.92298947)(424.07961161,701.13798925)(424.18961763,701.39799154)
\curveto(424.22961146,701.4879889)(424.25961143,701.57798881)(424.27961763,701.66799154)
\curveto(424.29961139,701.76798862)(424.31961137,701.87298852)(424.33961763,701.98299154)
\curveto(424.34961134,702.03298836)(424.34961134,702.07798831)(424.33961763,702.11799154)
\curveto(424.33961135,702.16798822)(424.34961134,702.21798817)(424.36961763,702.26799154)
\curveto(424.37961131,702.29798809)(424.3846113,702.33298806)(424.38461763,702.37299154)
\lineto(424.38461763,702.50799154)
\lineto(424.38461763,702.64299154)
}
}
{
\newrgbcolor{curcolor}{0 0 0}
\pscustom[linestyle=none,fillstyle=solid,fillcolor=curcolor]
{
\newpath
\moveto(430.10453951,709.43799154)
\curveto(430.17453656,709.35798103)(430.20953652,709.23798115)(430.20953951,709.07799154)
\lineto(430.20953951,708.61299154)
\lineto(430.20953951,708.20799154)
\curveto(430.20953652,708.06798232)(430.17453656,707.97298242)(430.10453951,707.92299154)
\curveto(430.04453669,707.87298252)(429.96453677,707.84298255)(429.86453951,707.83299154)
\curveto(429.77453696,707.82298257)(429.67453706,707.81798257)(429.56453951,707.81799154)
\lineto(428.72453951,707.81799154)
\curveto(428.61453812,707.81798257)(428.51453822,707.82298257)(428.42453951,707.83299154)
\curveto(428.34453839,707.84298255)(428.27453846,707.87298252)(428.21453951,707.92299154)
\curveto(428.17453856,707.95298244)(428.14453859,708.00798238)(428.12453951,708.08799154)
\curveto(428.11453862,708.17798221)(428.10453863,708.27298212)(428.09453951,708.37299154)
\lineto(428.09453951,708.70299154)
\curveto(428.10453863,708.81298158)(428.10953862,708.90798148)(428.10953951,708.98799154)
\lineto(428.10953951,709.19799154)
\curveto(428.11953861,709.26798112)(428.13953859,709.32798106)(428.16953951,709.37799154)
\curveto(428.18953854,709.41798097)(428.21453852,709.44798094)(428.24453951,709.46799154)
\lineto(428.36453951,709.52799154)
\curveto(428.38453835,709.52798086)(428.40953832,709.52798086)(428.43953951,709.52799154)
\curveto(428.46953826,709.53798085)(428.49453824,709.54298085)(428.51453951,709.54299154)
\lineto(429.60953951,709.54299154)
\curveto(429.70953702,709.54298085)(429.80453693,709.53798085)(429.89453951,709.52799154)
\curveto(429.98453675,709.51798087)(430.05453668,709.4879809)(430.10453951,709.43799154)
\moveto(430.20953951,699.67299154)
\curveto(430.20953652,699.47299092)(430.20453653,699.30299109)(430.19453951,699.16299154)
\curveto(430.18453655,699.02299137)(430.09453664,698.92799146)(429.92453951,698.87799154)
\curveto(429.86453687,698.85799153)(429.79953693,698.84799154)(429.72953951,698.84799154)
\curveto(429.65953707,698.85799153)(429.58453715,698.86299153)(429.50453951,698.86299154)
\lineto(428.66453951,698.86299154)
\curveto(428.57453816,698.86299153)(428.48453825,698.86799152)(428.39453951,698.87799154)
\curveto(428.31453842,698.8879915)(428.25453848,698.91799147)(428.21453951,698.96799154)
\curveto(428.15453858,699.03799135)(428.11953861,699.12299127)(428.10953951,699.22299154)
\lineto(428.10953951,699.56799154)
\lineto(428.10953951,705.89799154)
\lineto(428.10953951,706.19799154)
\curveto(428.10953862,706.29798409)(428.1295386,706.37798401)(428.16953951,706.43799154)
\curveto(428.2295385,706.50798388)(428.31453842,706.55298384)(428.42453951,706.57299154)
\curveto(428.44453829,706.58298381)(428.46953826,706.58298381)(428.49953951,706.57299154)
\curveto(428.53953819,706.57298382)(428.56953816,706.57798381)(428.58953951,706.58799154)
\lineto(429.33953951,706.58799154)
\lineto(429.53453951,706.58799154)
\curveto(429.61453712,706.59798379)(429.67953705,706.59798379)(429.72953951,706.58799154)
\lineto(429.84953951,706.58799154)
\curveto(429.90953682,706.56798382)(429.96453677,706.55298384)(430.01453951,706.54299154)
\curveto(430.06453667,706.53298386)(430.10453663,706.50298389)(430.13453951,706.45299154)
\curveto(430.17453656,706.40298399)(430.19453654,706.33298406)(430.19453951,706.24299154)
\curveto(430.20453653,706.15298424)(430.20953652,706.05798433)(430.20953951,705.95799154)
\lineto(430.20953951,699.67299154)
}
}
{
\newrgbcolor{curcolor}{0 0 0}
\pscustom[linestyle=none,fillstyle=solid,fillcolor=curcolor]
{
\newpath
\moveto(435.44172701,706.79799154)
\curveto(436.25172185,706.81798357)(436.92672117,706.69798369)(437.46672701,706.43799154)
\curveto(438.01672008,706.17798421)(438.45171965,705.80798458)(438.77172701,705.32799154)
\curveto(438.93171917,705.0879853)(439.05171905,704.81298558)(439.13172701,704.50299154)
\curveto(439.15171895,704.45298594)(439.16671893,704.387986)(439.17672701,704.30799154)
\curveto(439.1967189,704.22798616)(439.1967189,704.15798623)(439.17672701,704.09799154)
\curveto(439.13671896,703.9879864)(439.06671903,703.92298647)(438.96672701,703.90299154)
\curveto(438.86671923,703.8929865)(438.74671935,703.8879865)(438.60672701,703.88799154)
\lineto(437.82672701,703.88799154)
\lineto(437.54172701,703.88799154)
\curveto(437.45172065,703.8879865)(437.37672072,703.90798648)(437.31672701,703.94799154)
\curveto(437.23672086,703.9879864)(437.18172092,704.04798634)(437.15172701,704.12799154)
\curveto(437.12172098,704.21798617)(437.08172102,704.30798608)(437.03172701,704.39799154)
\curveto(436.97172113,704.50798588)(436.90672119,704.60798578)(436.83672701,704.69799154)
\curveto(436.76672133,704.7879856)(436.68672141,704.86798552)(436.59672701,704.93799154)
\curveto(436.45672164,705.02798536)(436.3017218,705.09798529)(436.13172701,705.14799154)
\curveto(436.07172203,705.16798522)(436.01172209,705.17798521)(435.95172701,705.17799154)
\curveto(435.89172221,705.17798521)(435.83672226,705.1879852)(435.78672701,705.20799154)
\lineto(435.63672701,705.20799154)
\curveto(435.43672266,705.20798518)(435.27672282,705.1879852)(435.15672701,705.14799154)
\curveto(434.86672323,705.05798533)(434.63172347,704.91798547)(434.45172701,704.72799154)
\curveto(434.27172383,704.54798584)(434.12672397,704.32798606)(434.01672701,704.06799154)
\curveto(433.96672413,703.95798643)(433.92672417,703.83798655)(433.89672701,703.70799154)
\curveto(433.87672422,703.5879868)(433.85172425,703.45798693)(433.82172701,703.31799154)
\curveto(433.81172429,703.27798711)(433.80672429,703.23798715)(433.80672701,703.19799154)
\curveto(433.80672429,703.15798723)(433.8017243,703.11798727)(433.79172701,703.07799154)
\curveto(433.77172433,702.97798741)(433.76172434,702.83798755)(433.76172701,702.65799154)
\curveto(433.77172433,702.47798791)(433.78672431,702.33798805)(433.80672701,702.23799154)
\curveto(433.80672429,702.15798823)(433.81172429,702.10298829)(433.82172701,702.07299154)
\curveto(433.84172426,702.00298839)(433.85172425,701.93298846)(433.85172701,701.86299154)
\curveto(433.86172424,701.7929886)(433.87672422,701.72298867)(433.89672701,701.65299154)
\curveto(433.97672412,701.42298897)(434.07172403,701.21298918)(434.18172701,701.02299154)
\curveto(434.29172381,700.83298956)(434.43172367,700.67298972)(434.60172701,700.54299154)
\curveto(434.64172346,700.51298988)(434.7017234,700.47798991)(434.78172701,700.43799154)
\curveto(434.89172321,700.36799002)(435.0017231,700.32299007)(435.11172701,700.30299154)
\curveto(435.23172287,700.28299011)(435.37672272,700.26299013)(435.54672701,700.24299154)
\lineto(435.63672701,700.24299154)
\curveto(435.67672242,700.24299015)(435.70672239,700.24799014)(435.72672701,700.25799154)
\lineto(435.86172701,700.25799154)
\curveto(435.93172217,700.27799011)(435.9967221,700.2929901)(436.05672701,700.30299154)
\curveto(436.12672197,700.32299007)(436.19172191,700.34299005)(436.25172701,700.36299154)
\curveto(436.55172155,700.4929899)(436.78172132,700.68298971)(436.94172701,700.93299154)
\curveto(436.98172112,700.98298941)(437.01672108,701.03798935)(437.04672701,701.09799154)
\curveto(437.07672102,701.16798922)(437.101721,701.22798916)(437.12172701,701.27799154)
\curveto(437.16172094,701.387989)(437.1967209,701.48298891)(437.22672701,701.56299154)
\curveto(437.25672084,701.65298874)(437.32672077,701.72298867)(437.43672701,701.77299154)
\curveto(437.52672057,701.81298858)(437.67172043,701.82798856)(437.87172701,701.81799154)
\lineto(438.36672701,701.81799154)
\lineto(438.57672701,701.81799154)
\curveto(438.65671944,701.82798856)(438.72171938,701.82298857)(438.77172701,701.80299154)
\lineto(438.89172701,701.80299154)
\lineto(439.01172701,701.77299154)
\curveto(439.05171905,701.77298862)(439.08171902,701.76298863)(439.10172701,701.74299154)
\curveto(439.15171895,701.70298869)(439.18171892,701.64298875)(439.19172701,701.56299154)
\curveto(439.21171889,701.4929889)(439.21171889,701.41798897)(439.19172701,701.33799154)
\curveto(439.101719,701.00798938)(438.99171911,700.71298968)(438.86172701,700.45299154)
\curveto(438.45171965,699.68299071)(437.7967203,699.14799124)(436.89672701,698.84799154)
\curveto(436.7967213,698.81799157)(436.69172141,698.79799159)(436.58172701,698.78799154)
\curveto(436.47172163,698.76799162)(436.36172174,698.74299165)(436.25172701,698.71299154)
\curveto(436.19172191,698.70299169)(436.13172197,698.69799169)(436.07172701,698.69799154)
\curveto(436.01172209,698.69799169)(435.95172215,698.6929917)(435.89172701,698.68299154)
\lineto(435.72672701,698.68299154)
\curveto(435.67672242,698.66299173)(435.6017225,698.65799173)(435.50172701,698.66799154)
\curveto(435.4017227,698.66799172)(435.32672277,698.67299172)(435.27672701,698.68299154)
\curveto(435.1967229,698.70299169)(435.12172298,698.71299168)(435.05172701,698.71299154)
\curveto(434.99172311,698.70299169)(434.92672317,698.70799168)(434.85672701,698.72799154)
\lineto(434.70672701,698.75799154)
\curveto(434.65672344,698.75799163)(434.60672349,698.76299163)(434.55672701,698.77299154)
\curveto(434.44672365,698.80299159)(434.34172376,698.83299156)(434.24172701,698.86299154)
\curveto(434.14172396,698.8929915)(434.04672405,698.92799146)(433.95672701,698.96799154)
\curveto(433.48672461,699.16799122)(433.09172501,699.42299097)(432.77172701,699.73299154)
\curveto(432.45172565,700.05299034)(432.19172591,700.44798994)(431.99172701,700.91799154)
\curveto(431.94172616,701.00798938)(431.9017262,701.10298929)(431.87172701,701.20299154)
\lineto(431.78172701,701.53299154)
\curveto(431.77172633,701.57298882)(431.76672633,701.60798878)(431.76672701,701.63799154)
\curveto(431.76672633,701.67798871)(431.75672634,701.72298867)(431.73672701,701.77299154)
\curveto(431.71672638,701.84298855)(431.70672639,701.91298848)(431.70672701,701.98299154)
\curveto(431.70672639,702.06298833)(431.6967264,702.13798825)(431.67672701,702.20799154)
\lineto(431.67672701,702.46299154)
\curveto(431.65672644,702.51298788)(431.64672645,702.56798782)(431.64672701,702.62799154)
\curveto(431.64672645,702.69798769)(431.65672644,702.75798763)(431.67672701,702.80799154)
\curveto(431.68672641,702.85798753)(431.68672641,702.90298749)(431.67672701,702.94299154)
\curveto(431.66672643,702.98298741)(431.66672643,703.02298737)(431.67672701,703.06299154)
\curveto(431.6967264,703.13298726)(431.7017264,703.19798719)(431.69172701,703.25799154)
\curveto(431.69172641,703.31798707)(431.7017264,703.37798701)(431.72172701,703.43799154)
\curveto(431.77172633,703.61798677)(431.81172629,703.7879866)(431.84172701,703.94799154)
\curveto(431.87172623,704.11798627)(431.91672618,704.28298611)(431.97672701,704.44299154)
\curveto(432.1967259,704.95298544)(432.47172563,705.37798501)(432.80172701,705.71799154)
\curveto(433.14172496,706.05798433)(433.57172453,706.33298406)(434.09172701,706.54299154)
\curveto(434.23172387,706.60298379)(434.37672372,706.64298375)(434.52672701,706.66299154)
\curveto(434.67672342,706.6929837)(434.83172327,706.72798366)(434.99172701,706.76799154)
\curveto(435.07172303,706.77798361)(435.14672295,706.78298361)(435.21672701,706.78299154)
\curveto(435.28672281,706.78298361)(435.36172274,706.7879836)(435.44172701,706.79799154)
}
}
{
\newrgbcolor{curcolor}{0 0 0}
\pscustom[linestyle=none,fillstyle=solid,fillcolor=curcolor]
{
\newpath
\moveto(447.53500826,699.44799154)
\curveto(447.55500041,699.33799105)(447.5650004,699.22799116)(447.56500826,699.11799154)
\curveto(447.57500039,699.00799138)(447.52500044,698.93299146)(447.41500826,698.89299154)
\curveto(447.35500061,698.86299153)(447.28500068,698.84799154)(447.20500826,698.84799154)
\lineto(446.96500826,698.84799154)
\lineto(446.15500826,698.84799154)
\lineto(445.88500826,698.84799154)
\curveto(445.80500216,698.85799153)(445.74000222,698.88299151)(445.69000826,698.92299154)
\curveto(445.62000234,698.96299143)(445.5650024,699.01799137)(445.52500826,699.08799154)
\curveto(445.49500247,699.16799122)(445.45000251,699.23299116)(445.39000826,699.28299154)
\curveto(445.37000259,699.30299109)(445.34500262,699.31799107)(445.31500826,699.32799154)
\curveto(445.28500268,699.34799104)(445.24500272,699.35299104)(445.19500826,699.34299154)
\curveto(445.14500282,699.32299107)(445.09500287,699.29799109)(445.04500826,699.26799154)
\curveto(445.00500296,699.23799115)(444.960003,699.21299118)(444.91000826,699.19299154)
\curveto(444.8600031,699.15299124)(444.80500316,699.11799127)(444.74500826,699.08799154)
\lineto(444.56500826,698.99799154)
\curveto(444.43500353,698.93799145)(444.30000366,698.8879915)(444.16000826,698.84799154)
\curveto(444.02000394,698.81799157)(443.87500409,698.78299161)(443.72500826,698.74299154)
\curveto(443.65500431,698.72299167)(443.58500438,698.71299168)(443.51500826,698.71299154)
\curveto(443.45500451,698.70299169)(443.39000457,698.6929917)(443.32000826,698.68299154)
\lineto(443.23000826,698.68299154)
\curveto(443.20000476,698.67299172)(443.17000479,698.66799172)(443.14000826,698.66799154)
\lineto(442.97500826,698.66799154)
\curveto(442.87500509,698.64799174)(442.77500519,698.64799174)(442.67500826,698.66799154)
\lineto(442.54000826,698.66799154)
\curveto(442.47000549,698.6879917)(442.40000556,698.69799169)(442.33000826,698.69799154)
\curveto(442.27000569,698.6879917)(442.21000575,698.6929917)(442.15000826,698.71299154)
\curveto(442.05000591,698.73299166)(441.95500601,698.75299164)(441.86500826,698.77299154)
\curveto(441.77500619,698.78299161)(441.69000627,698.80799158)(441.61000826,698.84799154)
\curveto(441.32000664,698.95799143)(441.07000689,699.09799129)(440.86000826,699.26799154)
\curveto(440.6600073,699.44799094)(440.50000746,699.68299071)(440.38000826,699.97299154)
\curveto(440.35000761,700.04299035)(440.32000764,700.11799027)(440.29000826,700.19799154)
\curveto(440.27000769,700.27799011)(440.25000771,700.36299003)(440.23000826,700.45299154)
\curveto(440.21000775,700.50298989)(440.20000776,700.55298984)(440.20000826,700.60299154)
\curveto(440.21000775,700.65298974)(440.21000775,700.70298969)(440.20000826,700.75299154)
\curveto(440.19000777,700.78298961)(440.18000778,700.84298955)(440.17000826,700.93299154)
\curveto(440.17000779,701.03298936)(440.17500779,701.10298929)(440.18500826,701.14299154)
\curveto(440.20500776,701.24298915)(440.21500775,701.32798906)(440.21500826,701.39799154)
\lineto(440.30500826,701.72799154)
\curveto(440.33500763,701.84798854)(440.37500759,701.95298844)(440.42500826,702.04299154)
\curveto(440.59500737,702.33298806)(440.79000717,702.55298784)(441.01000826,702.70299154)
\curveto(441.23000673,702.85298754)(441.51000645,702.98298741)(441.85000826,703.09299154)
\curveto(441.98000598,703.14298725)(442.11500585,703.17798721)(442.25500826,703.19799154)
\curveto(442.39500557,703.21798717)(442.53500543,703.24298715)(442.67500826,703.27299154)
\curveto(442.75500521,703.2929871)(442.84000512,703.30298709)(442.93000826,703.30299154)
\curveto(443.02000494,703.31298708)(443.11000485,703.32798706)(443.20000826,703.34799154)
\curveto(443.27000469,703.36798702)(443.34000462,703.37298702)(443.41000826,703.36299154)
\curveto(443.48000448,703.36298703)(443.55500441,703.37298702)(443.63500826,703.39299154)
\curveto(443.70500426,703.41298698)(443.77500419,703.42298697)(443.84500826,703.42299154)
\curveto(443.91500405,703.42298697)(443.99000397,703.43298696)(444.07000826,703.45299154)
\curveto(444.28000368,703.50298689)(444.47000349,703.54298685)(444.64000826,703.57299154)
\curveto(444.82000314,703.61298678)(444.98000298,703.70298669)(445.12000826,703.84299154)
\curveto(445.21000275,703.93298646)(445.27000269,704.03298636)(445.30000826,704.14299154)
\curveto(445.31000265,704.17298622)(445.31000265,704.19798619)(445.30000826,704.21799154)
\curveto(445.30000266,704.23798615)(445.30500266,704.25798613)(445.31500826,704.27799154)
\curveto(445.32500264,704.29798609)(445.33000263,704.32798606)(445.33000826,704.36799154)
\lineto(445.33000826,704.45799154)
\lineto(445.30000826,704.57799154)
\curveto(445.30000266,704.61798577)(445.29500267,704.65298574)(445.28500826,704.68299154)
\curveto(445.18500278,704.98298541)(444.97500299,705.1879852)(444.65500826,705.29799154)
\curveto(444.5650034,705.32798506)(444.45500351,705.34798504)(444.32500826,705.35799154)
\curveto(444.20500376,705.37798501)(444.08000388,705.38298501)(443.95000826,705.37299154)
\curveto(443.82000414,705.37298502)(443.69500427,705.36298503)(443.57500826,705.34299154)
\curveto(443.45500451,705.32298507)(443.35000461,705.29798509)(443.26000826,705.26799154)
\curveto(443.20000476,705.24798514)(443.14000482,705.21798517)(443.08000826,705.17799154)
\curveto(443.03000493,705.14798524)(442.98000498,705.11298528)(442.93000826,705.07299154)
\curveto(442.88000508,705.03298536)(442.82500514,704.97798541)(442.76500826,704.90799154)
\curveto(442.71500525,704.83798555)(442.68000528,704.77298562)(442.66000826,704.71299154)
\curveto(442.61000535,704.61298578)(442.5650054,704.51798587)(442.52500826,704.42799154)
\curveto(442.49500547,704.33798605)(442.42500554,704.27798611)(442.31500826,704.24799154)
\curveto(442.23500573,704.22798616)(442.15000581,704.21798617)(442.06000826,704.21799154)
\lineto(441.79000826,704.21799154)
\lineto(441.22000826,704.21799154)
\curveto(441.17000679,704.21798617)(441.12000684,704.21298618)(441.07000826,704.20299154)
\curveto(441.02000694,704.20298619)(440.97500699,704.20798618)(440.93500826,704.21799154)
\lineto(440.80000826,704.21799154)
\curveto(440.78000718,704.22798616)(440.75500721,704.23298616)(440.72500826,704.23299154)
\curveto(440.69500727,704.23298616)(440.67000729,704.24298615)(440.65000826,704.26299154)
\curveto(440.57000739,704.28298611)(440.51500745,704.34798604)(440.48500826,704.45799154)
\curveto(440.47500749,704.50798588)(440.47500749,704.55798583)(440.48500826,704.60799154)
\curveto(440.49500747,704.65798573)(440.50500746,704.70298569)(440.51500826,704.74299154)
\curveto(440.54500742,704.85298554)(440.57500739,704.95298544)(440.60500826,705.04299154)
\curveto(440.64500732,705.14298525)(440.69000727,705.23298516)(440.74000826,705.31299154)
\lineto(440.83000826,705.46299154)
\lineto(440.92000826,705.61299154)
\curveto(441.00000696,705.72298467)(441.10000686,705.82798456)(441.22000826,705.92799154)
\curveto(441.24000672,705.93798445)(441.27000669,705.96298443)(441.31000826,706.00299154)
\curveto(441.3600066,706.04298435)(441.40500656,706.07798431)(441.44500826,706.10799154)
\curveto(441.48500648,706.13798425)(441.53000643,706.16798422)(441.58000826,706.19799154)
\curveto(441.75000621,706.30798408)(441.93000603,706.392984)(442.12000826,706.45299154)
\curveto(442.31000565,706.52298387)(442.50500546,706.5879838)(442.70500826,706.64799154)
\curveto(442.82500514,706.67798371)(442.95000501,706.69798369)(443.08000826,706.70799154)
\curveto(443.21000475,706.71798367)(443.34000462,706.73798365)(443.47000826,706.76799154)
\curveto(443.51000445,706.77798361)(443.57000439,706.77798361)(443.65000826,706.76799154)
\curveto(443.74000422,706.75798363)(443.79500417,706.76298363)(443.81500826,706.78299154)
\curveto(444.22500374,706.7929836)(444.61500335,706.77798361)(444.98500826,706.73799154)
\curveto(445.3650026,706.69798369)(445.70500226,706.62298377)(446.00500826,706.51299154)
\curveto(446.31500165,706.40298399)(446.58000138,706.25298414)(446.80000826,706.06299154)
\curveto(447.02000094,705.88298451)(447.19000077,705.64798474)(447.31000826,705.35799154)
\curveto(447.38000058,705.1879852)(447.42000054,704.9929854)(447.43000826,704.77299154)
\curveto(447.44000052,704.55298584)(447.44500052,704.32798606)(447.44500826,704.09799154)
\lineto(447.44500826,700.75299154)
\lineto(447.44500826,700.16799154)
\curveto(447.44500052,699.97799041)(447.4650005,699.80299059)(447.50500826,699.64299154)
\curveto(447.51500045,699.61299078)(447.52000044,699.57799081)(447.52000826,699.53799154)
\curveto(447.52000044,699.50799088)(447.52500044,699.47799091)(447.53500826,699.44799154)
\moveto(445.33000826,701.75799154)
\curveto(445.34000262,701.80798858)(445.34500262,701.86298853)(445.34500826,701.92299154)
\curveto(445.34500262,701.9929884)(445.34000262,702.05298834)(445.33000826,702.10299154)
\curveto(445.31000265,702.16298823)(445.30000266,702.21798817)(445.30000826,702.26799154)
\curveto(445.30000266,702.31798807)(445.28000268,702.35798803)(445.24000826,702.38799154)
\curveto(445.19000277,702.42798796)(445.11500285,702.44798794)(445.01500826,702.44799154)
\curveto(444.97500299,702.43798795)(444.94000302,702.42798796)(444.91000826,702.41799154)
\curveto(444.88000308,702.41798797)(444.84500312,702.41298798)(444.80500826,702.40299154)
\curveto(444.73500323,702.38298801)(444.6600033,702.36798802)(444.58000826,702.35799154)
\curveto(444.50000346,702.34798804)(444.42000354,702.33298806)(444.34000826,702.31299154)
\curveto(444.31000365,702.30298809)(444.2650037,702.29798809)(444.20500826,702.29799154)
\curveto(444.07500389,702.26798812)(443.94500402,702.24798814)(443.81500826,702.23799154)
\curveto(443.68500428,702.22798816)(443.5600044,702.20298819)(443.44000826,702.16299154)
\curveto(443.3600046,702.14298825)(443.28500468,702.12298827)(443.21500826,702.10299154)
\curveto(443.14500482,702.0929883)(443.07500489,702.07298832)(443.00500826,702.04299154)
\curveto(442.79500517,701.95298844)(442.61500535,701.81798857)(442.46500826,701.63799154)
\curveto(442.32500564,701.45798893)(442.27500569,701.20798918)(442.31500826,700.88799154)
\curveto(442.33500563,700.71798967)(442.39000557,700.57798981)(442.48000826,700.46799154)
\curveto(442.55000541,700.35799003)(442.65500531,700.26799012)(442.79500826,700.19799154)
\curveto(442.93500503,700.13799025)(443.08500488,700.0929903)(443.24500826,700.06299154)
\curveto(443.41500455,700.03299036)(443.59000437,700.02299037)(443.77000826,700.03299154)
\curveto(443.960004,700.05299034)(444.13500383,700.0879903)(444.29500826,700.13799154)
\curveto(444.55500341,700.21799017)(444.7600032,700.34299005)(444.91000826,700.51299154)
\curveto(445.0600029,700.6929897)(445.17500279,700.91298948)(445.25500826,701.17299154)
\curveto(445.27500269,701.24298915)(445.28500268,701.31298908)(445.28500826,701.38299154)
\curveto(445.29500267,701.46298893)(445.31000265,701.54298885)(445.33000826,701.62299154)
\lineto(445.33000826,701.75799154)
}
}
{
\newrgbcolor{curcolor}{0 0 0}
\pscustom[linestyle=none,fillstyle=solid,fillcolor=curcolor]
{
\newpath
\moveto(456.68828951,699.70299154)
\lineto(456.68828951,699.28299154)
\curveto(456.68828114,699.15299124)(456.65828117,699.04799134)(456.59828951,698.96799154)
\curveto(456.54828128,698.91799147)(456.48328134,698.88299151)(456.40328951,698.86299154)
\curveto(456.3232815,698.85299154)(456.23328159,698.84799154)(456.13328951,698.84799154)
\lineto(455.30828951,698.84799154)
\lineto(455.02328951,698.84799154)
\curveto(454.94328288,698.85799153)(454.87828295,698.88299151)(454.82828951,698.92299154)
\curveto(454.75828307,698.97299142)(454.71828311,699.03799135)(454.70828951,699.11799154)
\curveto(454.69828313,699.19799119)(454.67828315,699.27799111)(454.64828951,699.35799154)
\curveto(454.6282832,699.37799101)(454.60828322,699.392991)(454.58828951,699.40299154)
\curveto(454.57828325,699.42299097)(454.56328326,699.44299095)(454.54328951,699.46299154)
\curveto(454.43328339,699.46299093)(454.35328347,699.43799095)(454.30328951,699.38799154)
\lineto(454.15328951,699.23799154)
\curveto(454.08328374,699.1879912)(454.01828381,699.14299125)(453.95828951,699.10299154)
\curveto(453.89828393,699.07299132)(453.83328399,699.03299136)(453.76328951,698.98299154)
\curveto(453.7232841,698.96299143)(453.67828415,698.94299145)(453.62828951,698.92299154)
\curveto(453.58828424,698.90299149)(453.54328428,698.88299151)(453.49328951,698.86299154)
\curveto(453.35328447,698.81299158)(453.20328462,698.76799162)(453.04328951,698.72799154)
\curveto(452.99328483,698.70799168)(452.94828488,698.69799169)(452.90828951,698.69799154)
\curveto(452.86828496,698.69799169)(452.828285,698.6929917)(452.78828951,698.68299154)
\lineto(452.65328951,698.68299154)
\curveto(452.6232852,698.67299172)(452.58328524,698.66799172)(452.53328951,698.66799154)
\lineto(452.39828951,698.66799154)
\curveto(452.33828549,698.64799174)(452.24828558,698.64299175)(452.12828951,698.65299154)
\curveto(452.00828582,698.65299174)(451.9232859,698.66299173)(451.87328951,698.68299154)
\curveto(451.80328602,698.70299169)(451.73828609,698.71299168)(451.67828951,698.71299154)
\curveto(451.6282862,698.70299169)(451.57328625,698.70799168)(451.51328951,698.72799154)
\lineto(451.15328951,698.84799154)
\curveto(451.04328678,698.87799151)(450.93328689,698.91799147)(450.82328951,698.96799154)
\curveto(450.47328735,699.11799127)(450.15828767,699.34799104)(449.87828951,699.65799154)
\curveto(449.60828822,699.97799041)(449.39328843,700.31299008)(449.23328951,700.66299154)
\curveto(449.18328864,700.77298962)(449.14328868,700.87798951)(449.11328951,700.97799154)
\curveto(449.08328874,701.0879893)(449.04828878,701.19798919)(449.00828951,701.30799154)
\curveto(448.99828883,701.34798904)(448.99328883,701.38298901)(448.99328951,701.41299154)
\curveto(448.99328883,701.45298894)(448.98328884,701.49798889)(448.96328951,701.54799154)
\curveto(448.94328888,701.62798876)(448.9232889,701.71298868)(448.90328951,701.80299154)
\curveto(448.89328893,701.90298849)(448.87828895,702.00298839)(448.85828951,702.10299154)
\curveto(448.84828898,702.13298826)(448.84328898,702.16798822)(448.84328951,702.20799154)
\curveto(448.85328897,702.24798814)(448.85328897,702.28298811)(448.84328951,702.31299154)
\lineto(448.84328951,702.44799154)
\curveto(448.84328898,702.49798789)(448.83828899,702.54798784)(448.82828951,702.59799154)
\curveto(448.81828901,702.64798774)(448.81328901,702.70298769)(448.81328951,702.76299154)
\curveto(448.81328901,702.83298756)(448.81828901,702.8879875)(448.82828951,702.92799154)
\curveto(448.83828899,702.97798741)(448.84328898,703.02298737)(448.84328951,703.06299154)
\lineto(448.84328951,703.21299154)
\curveto(448.85328897,703.26298713)(448.85328897,703.30798708)(448.84328951,703.34799154)
\curveto(448.84328898,703.39798699)(448.85328897,703.44798694)(448.87328951,703.49799154)
\curveto(448.89328893,703.60798678)(448.90828892,703.71298668)(448.91828951,703.81299154)
\curveto(448.93828889,703.91298648)(448.96328886,704.01298638)(448.99328951,704.11299154)
\curveto(449.03328879,704.23298616)(449.06828876,704.34798604)(449.09828951,704.45799154)
\curveto(449.1282887,704.56798582)(449.16828866,704.67798571)(449.21828951,704.78799154)
\curveto(449.35828847,705.0879853)(449.53328829,705.37298502)(449.74328951,705.64299154)
\curveto(449.76328806,705.67298472)(449.78828804,705.69798469)(449.81828951,705.71799154)
\curveto(449.85828797,705.74798464)(449.88828794,705.77798461)(449.90828951,705.80799154)
\curveto(449.94828788,705.85798453)(449.98828784,705.90298449)(450.02828951,705.94299154)
\curveto(450.06828776,705.98298441)(450.11328771,706.02298437)(450.16328951,706.06299154)
\curveto(450.20328762,706.08298431)(450.23828759,706.10798428)(450.26828951,706.13799154)
\curveto(450.29828753,706.17798421)(450.33328749,706.20798418)(450.37328951,706.22799154)
\curveto(450.6232872,706.39798399)(450.91328691,706.53798385)(451.24328951,706.64799154)
\curveto(451.31328651,706.66798372)(451.38328644,706.68298371)(451.45328951,706.69299154)
\curveto(451.53328629,706.70298369)(451.61328621,706.71798367)(451.69328951,706.73799154)
\curveto(451.76328606,706.75798363)(451.85328597,706.76798362)(451.96328951,706.76799154)
\curveto(452.07328575,706.77798361)(452.18328564,706.78298361)(452.29328951,706.78299154)
\curveto(452.40328542,706.78298361)(452.50828532,706.77798361)(452.60828951,706.76799154)
\curveto(452.71828511,706.75798363)(452.80828502,706.74298365)(452.87828951,706.72299154)
\curveto(453.0282848,706.67298372)(453.17328465,706.62798376)(453.31328951,706.58799154)
\curveto(453.45328437,706.54798384)(453.58328424,706.4929839)(453.70328951,706.42299154)
\curveto(453.77328405,706.37298402)(453.83828399,706.32298407)(453.89828951,706.27299154)
\curveto(453.95828387,706.23298416)(454.0232838,706.1879842)(454.09328951,706.13799154)
\curveto(454.13328369,706.10798428)(454.18828364,706.06798432)(454.25828951,706.01799154)
\curveto(454.33828349,705.96798442)(454.41328341,705.96798442)(454.48328951,706.01799154)
\curveto(454.5232833,706.03798435)(454.54328328,706.07298432)(454.54328951,706.12299154)
\curveto(454.54328328,706.17298422)(454.55328327,706.22298417)(454.57328951,706.27299154)
\lineto(454.57328951,706.42299154)
\curveto(454.58328324,706.45298394)(454.58828324,706.4879839)(454.58828951,706.52799154)
\lineto(454.58828951,706.64799154)
\lineto(454.58828951,708.68799154)
\curveto(454.58828324,708.79798159)(454.58328324,708.91798147)(454.57328951,709.04799154)
\curveto(454.57328325,709.1879812)(454.59828323,709.2929811)(454.64828951,709.36299154)
\curveto(454.68828314,709.44298095)(454.76328306,709.4929809)(454.87328951,709.51299154)
\curveto(454.89328293,709.52298087)(454.91328291,709.52298087)(454.93328951,709.51299154)
\curveto(454.95328287,709.51298088)(454.97328285,709.51798087)(454.99328951,709.52799154)
\lineto(456.05828951,709.52799154)
\curveto(456.17828165,709.52798086)(456.28828154,709.52298087)(456.38828951,709.51299154)
\curveto(456.48828134,709.50298089)(456.56328126,709.46298093)(456.61328951,709.39299154)
\curveto(456.66328116,709.31298108)(456.68828114,709.20798118)(456.68828951,709.07799154)
\lineto(456.68828951,708.71799154)
\lineto(456.68828951,699.70299154)
\moveto(454.64828951,702.64299154)
\curveto(454.65828317,702.68298771)(454.65828317,702.72298767)(454.64828951,702.76299154)
\lineto(454.64828951,702.89799154)
\curveto(454.64828318,702.99798739)(454.64328318,703.09798729)(454.63328951,703.19799154)
\curveto(454.6232832,703.29798709)(454.60828322,703.387987)(454.58828951,703.46799154)
\curveto(454.56828326,703.57798681)(454.54828328,703.67798671)(454.52828951,703.76799154)
\curveto(454.51828331,703.85798653)(454.49328333,703.94298645)(454.45328951,704.02299154)
\curveto(454.31328351,704.38298601)(454.10828372,704.66798572)(453.83828951,704.87799154)
\curveto(453.57828425,705.0879853)(453.19828463,705.1929852)(452.69828951,705.19299154)
\curveto(452.63828519,705.1929852)(452.55828527,705.18298521)(452.45828951,705.16299154)
\curveto(452.37828545,705.14298525)(452.30328552,705.12298527)(452.23328951,705.10299154)
\curveto(452.17328565,705.0929853)(452.11328571,705.07298532)(452.05328951,705.04299154)
\curveto(451.78328604,704.93298546)(451.57328625,704.76298563)(451.42328951,704.53299154)
\curveto(451.27328655,704.30298609)(451.15328667,704.04298635)(451.06328951,703.75299154)
\curveto(451.03328679,703.65298674)(451.01328681,703.55298684)(451.00328951,703.45299154)
\curveto(450.99328683,703.35298704)(450.97328685,703.24798714)(450.94328951,703.13799154)
\lineto(450.94328951,702.92799154)
\curveto(450.9232869,702.83798755)(450.91828691,702.71298768)(450.92828951,702.55299154)
\curveto(450.93828689,702.40298799)(450.95328687,702.2929881)(450.97328951,702.22299154)
\lineto(450.97328951,702.13299154)
\curveto(450.98328684,702.11298828)(450.98828684,702.0929883)(450.98828951,702.07299154)
\curveto(451.00828682,701.9929884)(451.0232868,701.91798847)(451.03328951,701.84799154)
\curveto(451.05328677,701.77798861)(451.07328675,701.70298869)(451.09328951,701.62299154)
\curveto(451.26328656,701.10298929)(451.55328627,700.71798967)(451.96328951,700.46799154)
\curveto(452.09328573,700.37799001)(452.27328555,700.30799008)(452.50328951,700.25799154)
\curveto(452.54328528,700.24799014)(452.60328522,700.24299015)(452.68328951,700.24299154)
\curveto(452.71328511,700.23299016)(452.75828507,700.22299017)(452.81828951,700.21299154)
\curveto(452.88828494,700.21299018)(452.94328488,700.21799017)(452.98328951,700.22799154)
\curveto(453.06328476,700.24799014)(453.14328468,700.26299013)(453.22328951,700.27299154)
\curveto(453.30328452,700.28299011)(453.38328444,700.30299009)(453.46328951,700.33299154)
\curveto(453.71328411,700.44298995)(453.91328391,700.58298981)(454.06328951,700.75299154)
\curveto(454.21328361,700.92298947)(454.34328348,701.13798925)(454.45328951,701.39799154)
\curveto(454.49328333,701.4879889)(454.5232833,701.57798881)(454.54328951,701.66799154)
\curveto(454.56328326,701.76798862)(454.58328324,701.87298852)(454.60328951,701.98299154)
\curveto(454.61328321,702.03298836)(454.61328321,702.07798831)(454.60328951,702.11799154)
\curveto(454.60328322,702.16798822)(454.61328321,702.21798817)(454.63328951,702.26799154)
\curveto(454.64328318,702.29798809)(454.64828318,702.33298806)(454.64828951,702.37299154)
\lineto(454.64828951,702.50799154)
\lineto(454.64828951,702.64299154)
}
}
{
\newrgbcolor{curcolor}{0 0 0}
\pscustom[linestyle=none,fillstyle=solid,fillcolor=curcolor]
{
\newpath
\moveto(466.03821138,703.03299154)
\curveto(466.05820281,702.97298742)(466.0682028,702.8879875)(466.06821138,702.77799154)
\curveto(466.0682028,702.66798772)(466.05820281,702.58298781)(466.03821138,702.52299154)
\lineto(466.03821138,702.37299154)
\curveto(466.01820285,702.2929881)(466.00820286,702.21298818)(466.00821138,702.13299154)
\curveto(466.01820285,702.05298834)(466.01320286,701.97298842)(465.99321138,701.89299154)
\curveto(465.9732029,701.82298857)(465.95820291,701.75798863)(465.94821138,701.69799154)
\curveto(465.93820293,701.63798875)(465.92820294,701.57298882)(465.91821138,701.50299154)
\curveto(465.87820299,701.392989)(465.84320303,701.27798911)(465.81321138,701.15799154)
\curveto(465.78320309,701.04798934)(465.74320313,700.94298945)(465.69321138,700.84299154)
\curveto(465.48320339,700.36299003)(465.20820366,699.97299042)(464.86821138,699.67299154)
\curveto(464.52820434,699.37299102)(464.11820475,699.12299127)(463.63821138,698.92299154)
\curveto(463.51820535,698.87299152)(463.39320548,698.83799155)(463.26321138,698.81799154)
\curveto(463.14320573,698.7879916)(463.01820585,698.75799163)(462.88821138,698.72799154)
\curveto(462.83820603,698.70799168)(462.78320609,698.69799169)(462.72321138,698.69799154)
\curveto(462.66320621,698.69799169)(462.60820626,698.6929917)(462.55821138,698.68299154)
\lineto(462.45321138,698.68299154)
\curveto(462.42320645,698.67299172)(462.39320648,698.66799172)(462.36321138,698.66799154)
\curveto(462.31320656,698.65799173)(462.23320664,698.65299174)(462.12321138,698.65299154)
\curveto(462.01320686,698.64299175)(461.92820694,698.64799174)(461.86821138,698.66799154)
\lineto(461.71821138,698.66799154)
\curveto(461.6682072,698.67799171)(461.61320726,698.68299171)(461.55321138,698.68299154)
\curveto(461.50320737,698.67299172)(461.45320742,698.67799171)(461.40321138,698.69799154)
\curveto(461.36320751,698.70799168)(461.32320755,698.71299168)(461.28321138,698.71299154)
\curveto(461.25320762,698.71299168)(461.21320766,698.71799167)(461.16321138,698.72799154)
\curveto(461.06320781,698.75799163)(460.96320791,698.78299161)(460.86321138,698.80299154)
\curveto(460.76320811,698.82299157)(460.6682082,698.85299154)(460.57821138,698.89299154)
\curveto(460.45820841,698.93299146)(460.34320853,698.97299142)(460.23321138,699.01299154)
\curveto(460.13320874,699.05299134)(460.02820884,699.10299129)(459.91821138,699.16299154)
\curveto(459.5682093,699.37299102)(459.2682096,699.61799077)(459.01821138,699.89799154)
\curveto(458.7682101,700.17799021)(458.55821031,700.51298988)(458.38821138,700.90299154)
\curveto(458.33821053,700.9929894)(458.29821057,701.0879893)(458.26821138,701.18799154)
\curveto(458.24821062,701.2879891)(458.22321065,701.392989)(458.19321138,701.50299154)
\curveto(458.1732107,701.55298884)(458.16321071,701.59798879)(458.16321138,701.63799154)
\curveto(458.16321071,701.67798871)(458.15321072,701.72298867)(458.13321138,701.77299154)
\curveto(458.11321076,701.85298854)(458.10321077,701.93298846)(458.10321138,702.01299154)
\curveto(458.10321077,702.10298829)(458.09321078,702.1879882)(458.07321138,702.26799154)
\curveto(458.06321081,702.31798807)(458.05821081,702.36298803)(458.05821138,702.40299154)
\lineto(458.05821138,702.53799154)
\curveto(458.03821083,702.59798779)(458.02821084,702.68298771)(458.02821138,702.79299154)
\curveto(458.03821083,702.90298749)(458.05321082,702.9879874)(458.07321138,703.04799154)
\lineto(458.07321138,703.15299154)
\curveto(458.08321079,703.20298719)(458.08321079,703.25298714)(458.07321138,703.30299154)
\curveto(458.0732108,703.36298703)(458.08321079,703.41798697)(458.10321138,703.46799154)
\curveto(458.11321076,703.51798687)(458.11821075,703.56298683)(458.11821138,703.60299154)
\curveto(458.11821075,703.65298674)(458.12821074,703.70298669)(458.14821138,703.75299154)
\curveto(458.18821068,703.88298651)(458.22321065,704.00798638)(458.25321138,704.12799154)
\curveto(458.28321059,704.25798613)(458.32321055,704.38298601)(458.37321138,704.50299154)
\curveto(458.55321032,704.91298548)(458.7682101,705.25298514)(459.01821138,705.52299154)
\curveto(459.2682096,705.80298459)(459.5732093,706.05798433)(459.93321138,706.28799154)
\curveto(460.03320884,706.33798405)(460.13820873,706.38298401)(460.24821138,706.42299154)
\curveto(460.35820851,706.46298393)(460.4682084,706.50798388)(460.57821138,706.55799154)
\curveto(460.70820816,706.60798378)(460.84320803,706.64298375)(460.98321138,706.66299154)
\curveto(461.12320775,706.68298371)(461.2682076,706.71298368)(461.41821138,706.75299154)
\curveto(461.49820737,706.76298363)(461.5732073,706.76798362)(461.64321138,706.76799154)
\curveto(461.71320716,706.76798362)(461.78320709,706.77298362)(461.85321138,706.78299154)
\curveto(462.43320644,706.7929836)(462.93320594,706.73298366)(463.35321138,706.60299154)
\curveto(463.78320509,706.47298392)(464.16320471,706.2929841)(464.49321138,706.06299154)
\curveto(464.60320427,705.98298441)(464.71320416,705.8929845)(464.82321138,705.79299154)
\curveto(464.94320393,705.70298469)(465.04320383,705.60298479)(465.12321138,705.49299154)
\curveto(465.20320367,705.392985)(465.2732036,705.2929851)(465.33321138,705.19299154)
\curveto(465.40320347,705.0929853)(465.4732034,704.9879854)(465.54321138,704.87799154)
\curveto(465.61320326,704.76798562)(465.6682032,704.64798574)(465.70821138,704.51799154)
\curveto(465.74820312,704.39798599)(465.79320308,704.26798612)(465.84321138,704.12799154)
\curveto(465.873203,704.04798634)(465.89820297,703.96298643)(465.91821138,703.87299154)
\lineto(465.97821138,703.60299154)
\curveto(465.98820288,703.56298683)(465.99320288,703.52298687)(465.99321138,703.48299154)
\curveto(465.99320288,703.44298695)(465.99820287,703.40298699)(466.00821138,703.36299154)
\curveto(466.02820284,703.31298708)(466.03320284,703.25798713)(466.02321138,703.19799154)
\curveto(466.01320286,703.13798725)(466.01820285,703.08298731)(466.03821138,703.03299154)
\moveto(463.93821138,702.49299154)
\curveto(463.94820492,702.54298785)(463.95320492,702.61298778)(463.95321138,702.70299154)
\curveto(463.95320492,702.80298759)(463.94820492,702.87798751)(463.93821138,702.92799154)
\lineto(463.93821138,703.04799154)
\curveto(463.91820495,703.09798729)(463.90820496,703.15298724)(463.90821138,703.21299154)
\curveto(463.90820496,703.27298712)(463.90320497,703.32798706)(463.89321138,703.37799154)
\curveto(463.89320498,703.41798697)(463.88820498,703.44798694)(463.87821138,703.46799154)
\lineto(463.81821138,703.70799154)
\curveto(463.80820506,703.79798659)(463.78820508,703.88298651)(463.75821138,703.96299154)
\curveto(463.64820522,704.22298617)(463.51820535,704.44298595)(463.36821138,704.62299154)
\curveto(463.21820565,704.81298558)(463.01820585,704.96298543)(462.76821138,705.07299154)
\curveto(462.70820616,705.0929853)(462.64820622,705.10798528)(462.58821138,705.11799154)
\curveto(462.52820634,705.13798525)(462.46320641,705.15798523)(462.39321138,705.17799154)
\curveto(462.31320656,705.19798519)(462.22820664,705.20298519)(462.13821138,705.19299154)
\lineto(461.86821138,705.19299154)
\curveto(461.83820703,705.17298522)(461.80320707,705.16298523)(461.76321138,705.16299154)
\curveto(461.72320715,705.17298522)(461.68820718,705.17298522)(461.65821138,705.16299154)
\lineto(461.44821138,705.10299154)
\curveto(461.38820748,705.0929853)(461.33320754,705.07298532)(461.28321138,705.04299154)
\curveto(461.03320784,704.93298546)(460.82820804,704.77298562)(460.66821138,704.56299154)
\curveto(460.51820835,704.36298603)(460.39820847,704.12798626)(460.30821138,703.85799154)
\curveto(460.27820859,703.75798663)(460.25320862,703.65298674)(460.23321138,703.54299154)
\curveto(460.22320865,703.43298696)(460.20820866,703.32298707)(460.18821138,703.21299154)
\curveto(460.17820869,703.16298723)(460.1732087,703.11298728)(460.17321138,703.06299154)
\lineto(460.17321138,702.91299154)
\curveto(460.15320872,702.84298755)(460.14320873,702.73798765)(460.14321138,702.59799154)
\curveto(460.15320872,702.45798793)(460.1682087,702.35298804)(460.18821138,702.28299154)
\lineto(460.18821138,702.14799154)
\curveto(460.20820866,702.06798832)(460.22320865,701.9879884)(460.23321138,701.90799154)
\curveto(460.24320863,701.83798855)(460.25820861,701.76298863)(460.27821138,701.68299154)
\curveto(460.37820849,701.38298901)(460.48320839,701.13798925)(460.59321138,700.94799154)
\curveto(460.71320816,700.76798962)(460.89820797,700.60298979)(461.14821138,700.45299154)
\curveto(461.21820765,700.40298999)(461.29320758,700.36299003)(461.37321138,700.33299154)
\curveto(461.46320741,700.30299009)(461.55320732,700.27799011)(461.64321138,700.25799154)
\curveto(461.68320719,700.24799014)(461.71820715,700.24299015)(461.74821138,700.24299154)
\curveto(461.77820709,700.25299014)(461.81320706,700.25299014)(461.85321138,700.24299154)
\lineto(461.97321138,700.21299154)
\curveto(462.02320685,700.21299018)(462.0682068,700.21799017)(462.10821138,700.22799154)
\lineto(462.22821138,700.22799154)
\curveto(462.30820656,700.24799014)(462.38820648,700.26299013)(462.46821138,700.27299154)
\curveto(462.54820632,700.28299011)(462.62320625,700.30299009)(462.69321138,700.33299154)
\curveto(462.95320592,700.43298996)(463.16320571,700.56798982)(463.32321138,700.73799154)
\curveto(463.48320539,700.90798948)(463.61820525,701.11798927)(463.72821138,701.36799154)
\curveto(463.7682051,701.46798892)(463.79820507,701.56798882)(463.81821138,701.66799154)
\curveto(463.83820503,701.76798862)(463.86320501,701.87298852)(463.89321138,701.98299154)
\curveto(463.90320497,702.02298837)(463.90820496,702.05798833)(463.90821138,702.08799154)
\curveto(463.90820496,702.12798826)(463.91320496,702.16798822)(463.92321138,702.20799154)
\lineto(463.92321138,702.34299154)
\curveto(463.92320495,702.392988)(463.92820494,702.44298795)(463.93821138,702.49299154)
}
}
{
\newrgbcolor{curcolor}{0 0 0}
\pscustom[linestyle=none,fillstyle=solid,fillcolor=curcolor]
{
\newpath
\moveto(471.86313326,706.78299154)
\curveto(471.97312794,706.78298361)(472.06812785,706.77298362)(472.14813326,706.75299154)
\curveto(472.23812768,706.73298366)(472.30812761,706.6879837)(472.35813326,706.61799154)
\curveto(472.4181275,706.53798385)(472.44812747,706.39798399)(472.44813326,706.19799154)
\lineto(472.44813326,705.68799154)
\lineto(472.44813326,705.31299154)
\curveto(472.45812746,705.17298522)(472.44312747,705.06298533)(472.40313326,704.98299154)
\curveto(472.36312755,704.91298548)(472.30312761,704.86798552)(472.22313326,704.84799154)
\curveto(472.15312776,704.82798556)(472.06812785,704.81798557)(471.96813326,704.81799154)
\curveto(471.87812804,704.81798557)(471.77812814,704.82298557)(471.66813326,704.83299154)
\curveto(471.56812835,704.84298555)(471.47312844,704.83798555)(471.38313326,704.81799154)
\curveto(471.3131286,704.79798559)(471.24312867,704.78298561)(471.17313326,704.77299154)
\curveto(471.10312881,704.77298562)(471.03812888,704.76298563)(470.97813326,704.74299154)
\curveto(470.8181291,704.6929857)(470.65812926,704.61798577)(470.49813326,704.51799154)
\curveto(470.33812958,704.42798596)(470.2131297,704.32298607)(470.12313326,704.20299154)
\curveto(470.07312984,704.12298627)(470.0181299,704.03798635)(469.95813326,703.94799154)
\curveto(469.90813001,703.86798652)(469.85813006,703.78298661)(469.80813326,703.69299154)
\curveto(469.77813014,703.61298678)(469.74813017,703.52798686)(469.71813326,703.43799154)
\lineto(469.65813326,703.19799154)
\curveto(469.63813028,703.12798726)(469.62813029,703.05298734)(469.62813326,702.97299154)
\curveto(469.62813029,702.90298749)(469.6181303,702.83298756)(469.59813326,702.76299154)
\curveto(469.58813033,702.72298767)(469.58313033,702.68298771)(469.58313326,702.64299154)
\curveto(469.59313032,702.61298778)(469.59313032,702.58298781)(469.58313326,702.55299154)
\lineto(469.58313326,702.31299154)
\curveto(469.56313035,702.24298815)(469.55813036,702.16298823)(469.56813326,702.07299154)
\curveto(469.57813034,701.9929884)(469.58313033,701.91298848)(469.58313326,701.83299154)
\lineto(469.58313326,700.87299154)
\lineto(469.58313326,699.59799154)
\curveto(469.58313033,699.46799092)(469.57813034,699.34799104)(469.56813326,699.23799154)
\curveto(469.55813036,699.12799126)(469.52813039,699.03799135)(469.47813326,698.96799154)
\curveto(469.45813046,698.93799145)(469.42313049,698.91299148)(469.37313326,698.89299154)
\curveto(469.33313058,698.88299151)(469.28813063,698.87299152)(469.23813326,698.86299154)
\lineto(469.16313326,698.86299154)
\curveto(469.1131308,698.85299154)(469.05813086,698.84799154)(468.99813326,698.84799154)
\lineto(468.83313326,698.84799154)
\lineto(468.18813326,698.84799154)
\curveto(468.12813179,698.85799153)(468.06313185,698.86299153)(467.99313326,698.86299154)
\lineto(467.79813326,698.86299154)
\curveto(467.74813217,698.88299151)(467.69813222,698.89799149)(467.64813326,698.90799154)
\curveto(467.59813232,698.92799146)(467.56313235,698.96299143)(467.54313326,699.01299154)
\curveto(467.50313241,699.06299133)(467.47813244,699.13299126)(467.46813326,699.22299154)
\lineto(467.46813326,699.52299154)
\lineto(467.46813326,700.54299154)
\lineto(467.46813326,704.77299154)
\lineto(467.46813326,705.88299154)
\lineto(467.46813326,706.16799154)
\curveto(467.46813245,706.26798412)(467.48813243,706.34798404)(467.52813326,706.40799154)
\curveto(467.57813234,706.4879839)(467.65313226,706.53798385)(467.75313326,706.55799154)
\curveto(467.85313206,706.57798381)(467.97313194,706.5879838)(468.11313326,706.58799154)
\lineto(468.87813326,706.58799154)
\curveto(468.99813092,706.5879838)(469.10313081,706.57798381)(469.19313326,706.55799154)
\curveto(469.28313063,706.54798384)(469.35313056,706.50298389)(469.40313326,706.42299154)
\curveto(469.43313048,706.37298402)(469.44813047,706.30298409)(469.44813326,706.21299154)
\lineto(469.47813326,705.94299154)
\curveto(469.48813043,705.86298453)(469.50313041,705.7879846)(469.52313326,705.71799154)
\curveto(469.55313036,705.64798474)(469.60313031,705.61298478)(469.67313326,705.61299154)
\curveto(469.69313022,705.63298476)(469.7131302,705.64298475)(469.73313326,705.64299154)
\curveto(469.75313016,705.64298475)(469.77313014,705.65298474)(469.79313326,705.67299154)
\curveto(469.85313006,705.72298467)(469.90313001,705.77798461)(469.94313326,705.83799154)
\curveto(469.99312992,705.90798448)(470.05312986,705.96798442)(470.12313326,706.01799154)
\curveto(470.16312975,706.04798434)(470.19812972,706.07798431)(470.22813326,706.10799154)
\curveto(470.25812966,706.14798424)(470.29312962,706.18298421)(470.33313326,706.21299154)
\lineto(470.60313326,706.39299154)
\curveto(470.70312921,706.45298394)(470.80312911,706.50798388)(470.90313326,706.55799154)
\curveto(471.00312891,706.59798379)(471.10312881,706.63298376)(471.20313326,706.66299154)
\lineto(471.53313326,706.75299154)
\curveto(471.56312835,706.76298363)(471.6181283,706.76298363)(471.69813326,706.75299154)
\curveto(471.78812813,706.75298364)(471.84312807,706.76298363)(471.86313326,706.78299154)
}
}
{
\newrgbcolor{curcolor}{0 0 0}
\pscustom[linestyle=none,fillstyle=solid,fillcolor=curcolor]
{
\newpath
\moveto(480.36953951,702.79299154)
\curveto(480.38953134,702.71298768)(480.38953134,702.62298777)(480.36953951,702.52299154)
\curveto(480.34953138,702.42298797)(480.31453142,702.35798803)(480.26453951,702.32799154)
\curveto(480.21453152,702.2879881)(480.13953159,702.25798813)(480.03953951,702.23799154)
\curveto(479.94953178,702.22798816)(479.84453189,702.21798817)(479.72453951,702.20799154)
\lineto(479.37953951,702.20799154)
\curveto(479.26953246,702.21798817)(479.16953256,702.22298817)(479.07953951,702.22299154)
\lineto(475.41953951,702.22299154)
\lineto(475.20953951,702.22299154)
\curveto(475.14953658,702.22298817)(475.09453664,702.21298818)(475.04453951,702.19299154)
\curveto(474.96453677,702.15298824)(474.91453682,702.11298828)(474.89453951,702.07299154)
\curveto(474.87453686,702.05298834)(474.85453688,702.01298838)(474.83453951,701.95299154)
\curveto(474.81453692,701.90298849)(474.80953692,701.85298854)(474.81953951,701.80299154)
\curveto(474.83953689,701.74298865)(474.84953688,701.68298871)(474.84953951,701.62299154)
\curveto(474.85953687,701.57298882)(474.87453686,701.51798887)(474.89453951,701.45799154)
\curveto(474.97453676,701.21798917)(475.06953666,701.01798937)(475.17953951,700.85799154)
\curveto(475.29953643,700.70798968)(475.45953627,700.57298982)(475.65953951,700.45299154)
\curveto(475.73953599,700.40298999)(475.81953591,700.36799002)(475.89953951,700.34799154)
\curveto(475.98953574,700.33799005)(476.07953565,700.31799007)(476.16953951,700.28799154)
\curveto(476.24953548,700.26799012)(476.35953537,700.25299014)(476.49953951,700.24299154)
\curveto(476.63953509,700.23299016)(476.75953497,700.23799015)(476.85953951,700.25799154)
\lineto(476.99453951,700.25799154)
\curveto(477.09453464,700.27799011)(477.18453455,700.29799009)(477.26453951,700.31799154)
\curveto(477.35453438,700.34799004)(477.43953429,700.37799001)(477.51953951,700.40799154)
\curveto(477.61953411,700.45798993)(477.729534,700.52298987)(477.84953951,700.60299154)
\curveto(477.97953375,700.68298971)(478.07453366,700.76298963)(478.13453951,700.84299154)
\curveto(478.18453355,700.91298948)(478.2345335,700.97798941)(478.28453951,701.03799154)
\curveto(478.34453339,701.10798928)(478.41453332,701.15798923)(478.49453951,701.18799154)
\curveto(478.59453314,701.23798915)(478.71953301,701.25798913)(478.86953951,701.24799154)
\lineto(479.30453951,701.24799154)
\lineto(479.48453951,701.24799154)
\curveto(479.55453218,701.25798913)(479.61453212,701.25298914)(479.66453951,701.23299154)
\lineto(479.81453951,701.23299154)
\curveto(479.91453182,701.21298918)(479.98453175,701.1879892)(480.02453951,701.15799154)
\curveto(480.06453167,701.13798925)(480.08453165,701.0929893)(480.08453951,701.02299154)
\curveto(480.09453164,700.95298944)(480.08953164,700.8929895)(480.06953951,700.84299154)
\curveto(480.01953171,700.70298969)(479.96453177,700.57798981)(479.90453951,700.46799154)
\curveto(479.84453189,700.35799003)(479.77453196,700.24799014)(479.69453951,700.13799154)
\curveto(479.47453226,699.80799058)(479.22453251,699.54299085)(478.94453951,699.34299154)
\curveto(478.66453307,699.14299125)(478.31453342,698.97299142)(477.89453951,698.83299154)
\curveto(477.78453395,698.7929916)(477.67453406,698.76799162)(477.56453951,698.75799154)
\curveto(477.45453428,698.74799164)(477.33953439,698.72799166)(477.21953951,698.69799154)
\curveto(477.17953455,698.6879917)(477.1345346,698.6879917)(477.08453951,698.69799154)
\curveto(477.04453469,698.69799169)(477.00453473,698.6929917)(476.96453951,698.68299154)
\lineto(476.79953951,698.68299154)
\curveto(476.74953498,698.66299173)(476.68953504,698.65799173)(476.61953951,698.66799154)
\curveto(476.55953517,698.66799172)(476.50453523,698.67299172)(476.45453951,698.68299154)
\curveto(476.37453536,698.6929917)(476.30453543,698.6929917)(476.24453951,698.68299154)
\curveto(476.18453555,698.67299172)(476.11953561,698.67799171)(476.04953951,698.69799154)
\curveto(475.99953573,698.71799167)(475.94453579,698.72799166)(475.88453951,698.72799154)
\curveto(475.82453591,698.72799166)(475.76953596,698.73799165)(475.71953951,698.75799154)
\curveto(475.60953612,698.77799161)(475.49953623,698.80299159)(475.38953951,698.83299154)
\curveto(475.27953645,698.85299154)(475.17953655,698.8879915)(475.08953951,698.93799154)
\curveto(474.97953675,698.97799141)(474.87453686,699.01299138)(474.77453951,699.04299154)
\curveto(474.68453705,699.08299131)(474.59953713,699.12799126)(474.51953951,699.17799154)
\curveto(474.19953753,699.37799101)(473.91453782,699.60799078)(473.66453951,699.86799154)
\curveto(473.41453832,700.13799025)(473.20953852,700.44798994)(473.04953951,700.79799154)
\curveto(472.99953873,700.90798948)(472.95953877,701.01798937)(472.92953951,701.12799154)
\curveto(472.89953883,701.24798914)(472.85953887,701.36798902)(472.80953951,701.48799154)
\curveto(472.79953893,701.52798886)(472.79453894,701.56298883)(472.79453951,701.59299154)
\curveto(472.79453894,701.63298876)(472.78953894,701.67298872)(472.77953951,701.71299154)
\curveto(472.73953899,701.83298856)(472.71453902,701.96298843)(472.70453951,702.10299154)
\lineto(472.67453951,702.52299154)
\curveto(472.67453906,702.57298782)(472.66953906,702.62798776)(472.65953951,702.68799154)
\curveto(472.65953907,702.74798764)(472.66453907,702.80298759)(472.67453951,702.85299154)
\lineto(472.67453951,703.03299154)
\lineto(472.71953951,703.39299154)
\curveto(472.75953897,703.56298683)(472.79453894,703.72798666)(472.82453951,703.88799154)
\curveto(472.85453888,704.04798634)(472.89953883,704.19798619)(472.95953951,704.33799154)
\curveto(473.38953834,705.37798501)(474.11953761,706.11298428)(475.14953951,706.54299154)
\curveto(475.28953644,706.60298379)(475.4295363,706.64298375)(475.56953951,706.66299154)
\curveto(475.71953601,706.6929837)(475.87453586,706.72798366)(476.03453951,706.76799154)
\curveto(476.11453562,706.77798361)(476.18953554,706.78298361)(476.25953951,706.78299154)
\curveto(476.3295354,706.78298361)(476.40453533,706.7879836)(476.48453951,706.79799154)
\curveto(476.99453474,706.80798358)(477.4295343,706.74798364)(477.78953951,706.61799154)
\curveto(478.15953357,706.49798389)(478.48953324,706.33798405)(478.77953951,706.13799154)
\curveto(478.86953286,706.07798431)(478.95953277,706.00798438)(479.04953951,705.92799154)
\curveto(479.13953259,705.85798453)(479.21953251,705.78298461)(479.28953951,705.70299154)
\curveto(479.31953241,705.65298474)(479.35953237,705.61298478)(479.40953951,705.58299154)
\curveto(479.48953224,705.47298492)(479.56453217,705.35798503)(479.63453951,705.23799154)
\curveto(479.70453203,705.12798526)(479.77953195,705.01298538)(479.85953951,704.89299154)
\curveto(479.90953182,704.80298559)(479.94953178,704.70798568)(479.97953951,704.60799154)
\curveto(480.01953171,704.51798587)(480.05953167,704.41798597)(480.09953951,704.30799154)
\curveto(480.14953158,704.17798621)(480.18953154,704.04298635)(480.21953951,703.90299154)
\curveto(480.24953148,703.76298663)(480.28453145,703.62298677)(480.32453951,703.48299154)
\curveto(480.34453139,703.40298699)(480.34953138,703.31298708)(480.33953951,703.21299154)
\curveto(480.33953139,703.12298727)(480.34953138,703.03798735)(480.36953951,702.95799154)
\lineto(480.36953951,702.79299154)
\moveto(478.11953951,703.67799154)
\curveto(478.18953354,703.77798661)(478.19453354,703.89798649)(478.13453951,704.03799154)
\curveto(478.08453365,704.1879862)(478.04453369,704.29798609)(478.01453951,704.36799154)
\curveto(477.87453386,704.63798575)(477.68953404,704.84298555)(477.45953951,704.98299154)
\curveto(477.2295345,705.13298526)(476.90953482,705.21298518)(476.49953951,705.22299154)
\curveto(476.46953526,705.20298519)(476.4345353,705.19798519)(476.39453951,705.20799154)
\curveto(476.35453538,705.21798517)(476.31953541,705.21798517)(476.28953951,705.20799154)
\curveto(476.23953549,705.1879852)(476.18453555,705.17298522)(476.12453951,705.16299154)
\curveto(476.06453567,705.16298523)(476.00953572,705.15298524)(475.95953951,705.13299154)
\curveto(475.51953621,704.9929854)(475.19453654,704.71798567)(474.98453951,704.30799154)
\curveto(474.96453677,704.26798612)(474.93953679,704.21298618)(474.90953951,704.14299154)
\curveto(474.88953684,704.08298631)(474.87453686,704.01798637)(474.86453951,703.94799154)
\curveto(474.85453688,703.8879865)(474.85453688,703.82798656)(474.86453951,703.76799154)
\curveto(474.88453685,703.70798668)(474.91953681,703.65798673)(474.96953951,703.61799154)
\curveto(475.04953668,703.56798682)(475.15953657,703.54298685)(475.29953951,703.54299154)
\lineto(475.70453951,703.54299154)
\lineto(477.36953951,703.54299154)
\lineto(477.80453951,703.54299154)
\curveto(477.96453377,703.55298684)(478.06953366,703.59798679)(478.11953951,703.67799154)
}
}
{
\newrgbcolor{curcolor}{0 0 0}
\pscustom[linestyle=none,fillstyle=solid,fillcolor=curcolor]
{
\newpath
\moveto(484.58782076,706.79799154)
\curveto(485.33781626,706.81798357)(485.98781561,706.73298366)(486.53782076,706.54299154)
\curveto(487.0978145,706.36298403)(487.52281407,706.04798434)(487.81282076,705.59799154)
\curveto(487.88281371,705.4879849)(487.94281365,705.37298502)(487.99282076,705.25299154)
\curveto(488.05281354,705.14298525)(488.10281349,705.01798537)(488.14282076,704.87799154)
\curveto(488.16281343,704.81798557)(488.17281342,704.75298564)(488.17282076,704.68299154)
\curveto(488.17281342,704.61298578)(488.16281343,704.55298584)(488.14282076,704.50299154)
\curveto(488.10281349,704.44298595)(488.04781355,704.40298599)(487.97782076,704.38299154)
\curveto(487.92781367,704.36298603)(487.86781373,704.35298604)(487.79782076,704.35299154)
\lineto(487.58782076,704.35299154)
\lineto(486.92782076,704.35299154)
\curveto(486.85781474,704.35298604)(486.78781481,704.34798604)(486.71782076,704.33799154)
\curveto(486.64781495,704.33798605)(486.58281501,704.34798604)(486.52282076,704.36799154)
\curveto(486.42281517,704.387986)(486.34781525,704.42798596)(486.29782076,704.48799154)
\curveto(486.24781535,704.54798584)(486.20281539,704.60798578)(486.16282076,704.66799154)
\lineto(486.04282076,704.87799154)
\curveto(486.01281558,704.95798543)(485.96281563,705.02298537)(485.89282076,705.07299154)
\curveto(485.7928158,705.15298524)(485.6928159,705.21298518)(485.59282076,705.25299154)
\curveto(485.50281609,705.2929851)(485.38781621,705.32798506)(485.24782076,705.35799154)
\curveto(485.17781642,705.37798501)(485.07281652,705.392985)(484.93282076,705.40299154)
\curveto(484.80281679,705.41298498)(484.70281689,705.40798498)(484.63282076,705.38799154)
\lineto(484.52782076,705.38799154)
\lineto(484.37782076,705.35799154)
\curveto(484.33781726,705.35798503)(484.2928173,705.35298504)(484.24282076,705.34299154)
\curveto(484.07281752,705.2929851)(483.93281766,705.22298517)(483.82282076,705.13299154)
\curveto(483.72281787,705.05298534)(483.65281794,704.92798546)(483.61282076,704.75799154)
\curveto(483.592818,704.6879857)(483.592818,704.62298577)(483.61282076,704.56299154)
\curveto(483.63281796,704.50298589)(483.65281794,704.45298594)(483.67282076,704.41299154)
\curveto(483.74281785,704.2929861)(483.82281777,704.19798619)(483.91282076,704.12799154)
\curveto(484.01281758,704.05798633)(484.12781747,703.99798639)(484.25782076,703.94799154)
\curveto(484.44781715,703.86798652)(484.65281694,703.79798659)(484.87282076,703.73799154)
\lineto(485.56282076,703.58799154)
\curveto(485.80281579,703.54798684)(486.03281556,703.49798689)(486.25282076,703.43799154)
\curveto(486.48281511,703.387987)(486.6978149,703.32298707)(486.89782076,703.24299154)
\curveto(486.98781461,703.20298719)(487.07281452,703.16798722)(487.15282076,703.13799154)
\curveto(487.24281435,703.11798727)(487.32781427,703.08298731)(487.40782076,703.03299154)
\curveto(487.597814,702.91298748)(487.76781383,702.78298761)(487.91782076,702.64299154)
\curveto(488.07781352,702.50298789)(488.20281339,702.32798806)(488.29282076,702.11799154)
\curveto(488.32281327,702.04798834)(488.34781325,701.97798841)(488.36782076,701.90799154)
\curveto(488.38781321,701.83798855)(488.40781319,701.76298863)(488.42782076,701.68299154)
\curveto(488.43781316,701.62298877)(488.44281315,701.52798886)(488.44282076,701.39799154)
\curveto(488.45281314,701.27798911)(488.45281314,701.18298921)(488.44282076,701.11299154)
\lineto(488.44282076,701.03799154)
\curveto(488.42281317,700.97798941)(488.40781319,700.91798947)(488.39782076,700.85799154)
\curveto(488.3978132,700.80798958)(488.3928132,700.75798963)(488.38282076,700.70799154)
\curveto(488.31281328,700.40798998)(488.20281339,700.14299025)(488.05282076,699.91299154)
\curveto(487.8928137,699.67299072)(487.6978139,699.47799091)(487.46782076,699.32799154)
\curveto(487.23781436,699.17799121)(486.97781462,699.04799134)(486.68782076,698.93799154)
\curveto(486.57781502,698.8879915)(486.45781514,698.85299154)(486.32782076,698.83299154)
\curveto(486.20781539,698.81299158)(486.08781551,698.7879916)(485.96782076,698.75799154)
\curveto(485.87781572,698.73799165)(485.78281581,698.72799166)(485.68282076,698.72799154)
\curveto(485.592816,698.71799167)(485.50281609,698.70299169)(485.41282076,698.68299154)
\lineto(485.14282076,698.68299154)
\curveto(485.08281651,698.66299173)(484.97781662,698.65299174)(484.82782076,698.65299154)
\curveto(484.68781691,698.65299174)(484.58781701,698.66299173)(484.52782076,698.68299154)
\curveto(484.4978171,698.68299171)(484.46281713,698.6879917)(484.42282076,698.69799154)
\lineto(484.31782076,698.69799154)
\curveto(484.1978174,698.71799167)(484.07781752,698.73299166)(483.95782076,698.74299154)
\curveto(483.83781776,698.75299164)(483.72281787,698.77299162)(483.61282076,698.80299154)
\curveto(483.22281837,698.91299148)(482.87781872,699.03799135)(482.57782076,699.17799154)
\curveto(482.27781932,699.32799106)(482.02281957,699.54799084)(481.81282076,699.83799154)
\curveto(481.67281992,700.02799036)(481.55282004,700.24799014)(481.45282076,700.49799154)
\curveto(481.43282016,700.55798983)(481.41282018,700.63798975)(481.39282076,700.73799154)
\curveto(481.37282022,700.7879896)(481.35782024,700.85798953)(481.34782076,700.94799154)
\curveto(481.33782026,701.03798935)(481.34282025,701.11298928)(481.36282076,701.17299154)
\curveto(481.3928202,701.24298915)(481.44282015,701.2929891)(481.51282076,701.32299154)
\curveto(481.56282003,701.34298905)(481.62281997,701.35298904)(481.69282076,701.35299154)
\lineto(481.91782076,701.35299154)
\lineto(482.62282076,701.35299154)
\lineto(482.86282076,701.35299154)
\curveto(482.94281865,701.35298904)(483.01281858,701.34298905)(483.07282076,701.32299154)
\curveto(483.18281841,701.28298911)(483.25281834,701.21798917)(483.28282076,701.12799154)
\curveto(483.32281827,701.03798935)(483.36781823,700.94298945)(483.41782076,700.84299154)
\curveto(483.43781816,700.7929896)(483.47281812,700.72798966)(483.52282076,700.64799154)
\curveto(483.58281801,700.56798982)(483.63281796,700.51798987)(483.67282076,700.49799154)
\curveto(483.7928178,700.39798999)(483.90781769,700.31799007)(484.01782076,700.25799154)
\curveto(484.12781747,700.20799018)(484.26781733,700.15799023)(484.43782076,700.10799154)
\curveto(484.48781711,700.0879903)(484.53781706,700.07799031)(484.58782076,700.07799154)
\curveto(484.63781696,700.0879903)(484.68781691,700.0879903)(484.73782076,700.07799154)
\curveto(484.81781678,700.05799033)(484.90281669,700.04799034)(484.99282076,700.04799154)
\curveto(485.0928165,700.05799033)(485.17781642,700.07299032)(485.24782076,700.09299154)
\curveto(485.2978163,700.10299029)(485.34281625,700.10799028)(485.38282076,700.10799154)
\curveto(485.43281616,700.10799028)(485.48281611,700.11799027)(485.53282076,700.13799154)
\curveto(485.67281592,700.1879902)(485.7978158,700.24799014)(485.90782076,700.31799154)
\curveto(486.02781557,700.38799)(486.12281547,700.47798991)(486.19282076,700.58799154)
\curveto(486.24281535,700.66798972)(486.28281531,700.7929896)(486.31282076,700.96299154)
\curveto(486.33281526,701.03298936)(486.33281526,701.09798929)(486.31282076,701.15799154)
\curveto(486.2928153,701.21798917)(486.27281532,701.26798912)(486.25282076,701.30799154)
\curveto(486.18281541,701.44798894)(486.0928155,701.55298884)(485.98282076,701.62299154)
\curveto(485.88281571,701.6929887)(485.76281583,701.75798863)(485.62282076,701.81799154)
\curveto(485.43281616,701.89798849)(485.23281636,701.96298843)(485.02282076,702.01299154)
\curveto(484.81281678,702.06298833)(484.60281699,702.11798827)(484.39282076,702.17799154)
\curveto(484.31281728,702.19798819)(484.22781737,702.21298818)(484.13782076,702.22299154)
\curveto(484.05781754,702.23298816)(483.97781762,702.24798814)(483.89782076,702.26799154)
\curveto(483.57781802,702.35798803)(483.27281832,702.44298795)(482.98282076,702.52299154)
\curveto(482.6928189,702.61298778)(482.42781917,702.74298765)(482.18782076,702.91299154)
\curveto(481.90781969,703.11298728)(481.70281989,703.38298701)(481.57282076,703.72299154)
\curveto(481.55282004,703.7929866)(481.53282006,703.8879865)(481.51282076,704.00799154)
\curveto(481.4928201,704.07798631)(481.47782012,704.16298623)(481.46782076,704.26299154)
\curveto(481.45782014,704.36298603)(481.46282013,704.45298594)(481.48282076,704.53299154)
\curveto(481.50282009,704.58298581)(481.50782009,704.62298577)(481.49782076,704.65299154)
\curveto(481.48782011,704.6929857)(481.4928201,704.73798565)(481.51282076,704.78799154)
\curveto(481.53282006,704.89798549)(481.55282004,704.99798539)(481.57282076,705.08799154)
\curveto(481.60281999,705.1879852)(481.63781996,705.28298511)(481.67782076,705.37299154)
\curveto(481.80781979,705.66298473)(481.98781961,705.89798449)(482.21782076,706.07799154)
\curveto(482.44781915,706.25798413)(482.70781889,706.40298399)(482.99782076,706.51299154)
\curveto(483.10781849,706.56298383)(483.22281837,706.59798379)(483.34282076,706.61799154)
\curveto(483.46281813,706.64798374)(483.58781801,706.67798371)(483.71782076,706.70799154)
\curveto(483.77781782,706.72798366)(483.83781776,706.73798365)(483.89782076,706.73799154)
\lineto(484.07782076,706.76799154)
\curveto(484.15781744,706.77798361)(484.24281735,706.78298361)(484.33282076,706.78299154)
\curveto(484.42281717,706.78298361)(484.50781709,706.7879836)(484.58782076,706.79799154)
}
}
{
\newrgbcolor{curcolor}{0 0 0}
\pscustom[linestyle=none,fillstyle=solid,fillcolor=curcolor]
{
}
}
{
\newrgbcolor{curcolor}{0 0 0}
\pscustom[linestyle=none,fillstyle=solid,fillcolor=curcolor]
{
\newpath
\moveto(501.42461763,699.70299154)
\lineto(501.42461763,699.28299154)
\curveto(501.42460926,699.15299124)(501.39460929,699.04799134)(501.33461763,698.96799154)
\curveto(501.2846094,698.91799147)(501.21960947,698.88299151)(501.13961763,698.86299154)
\curveto(501.05960963,698.85299154)(500.96960972,698.84799154)(500.86961763,698.84799154)
\lineto(500.04461763,698.84799154)
\lineto(499.75961763,698.84799154)
\curveto(499.67961101,698.85799153)(499.61461107,698.88299151)(499.56461763,698.92299154)
\curveto(499.49461119,698.97299142)(499.45461123,699.03799135)(499.44461763,699.11799154)
\curveto(499.43461125,699.19799119)(499.41461127,699.27799111)(499.38461763,699.35799154)
\curveto(499.36461132,699.37799101)(499.34461134,699.392991)(499.32461763,699.40299154)
\curveto(499.31461137,699.42299097)(499.29961139,699.44299095)(499.27961763,699.46299154)
\curveto(499.16961152,699.46299093)(499.0896116,699.43799095)(499.03961763,699.38799154)
\lineto(498.88961763,699.23799154)
\curveto(498.81961187,699.1879912)(498.75461193,699.14299125)(498.69461763,699.10299154)
\curveto(498.63461205,699.07299132)(498.56961212,699.03299136)(498.49961763,698.98299154)
\curveto(498.45961223,698.96299143)(498.41461227,698.94299145)(498.36461763,698.92299154)
\curveto(498.32461236,698.90299149)(498.27961241,698.88299151)(498.22961763,698.86299154)
\curveto(498.0896126,698.81299158)(497.93961275,698.76799162)(497.77961763,698.72799154)
\curveto(497.72961296,698.70799168)(497.684613,698.69799169)(497.64461763,698.69799154)
\curveto(497.60461308,698.69799169)(497.56461312,698.6929917)(497.52461763,698.68299154)
\lineto(497.38961763,698.68299154)
\curveto(497.35961333,698.67299172)(497.31961337,698.66799172)(497.26961763,698.66799154)
\lineto(497.13461763,698.66799154)
\curveto(497.07461361,698.64799174)(496.9846137,698.64299175)(496.86461763,698.65299154)
\curveto(496.74461394,698.65299174)(496.65961403,698.66299173)(496.60961763,698.68299154)
\curveto(496.53961415,698.70299169)(496.47461421,698.71299168)(496.41461763,698.71299154)
\curveto(496.36461432,698.70299169)(496.30961438,698.70799168)(496.24961763,698.72799154)
\lineto(495.88961763,698.84799154)
\curveto(495.77961491,698.87799151)(495.66961502,698.91799147)(495.55961763,698.96799154)
\curveto(495.20961548,699.11799127)(494.89461579,699.34799104)(494.61461763,699.65799154)
\curveto(494.34461634,699.97799041)(494.12961656,700.31299008)(493.96961763,700.66299154)
\curveto(493.91961677,700.77298962)(493.87961681,700.87798951)(493.84961763,700.97799154)
\curveto(493.81961687,701.0879893)(493.7846169,701.19798919)(493.74461763,701.30799154)
\curveto(493.73461695,701.34798904)(493.72961696,701.38298901)(493.72961763,701.41299154)
\curveto(493.72961696,701.45298894)(493.71961697,701.49798889)(493.69961763,701.54799154)
\curveto(493.67961701,701.62798876)(493.65961703,701.71298868)(493.63961763,701.80299154)
\curveto(493.62961706,701.90298849)(493.61461707,702.00298839)(493.59461763,702.10299154)
\curveto(493.5846171,702.13298826)(493.57961711,702.16798822)(493.57961763,702.20799154)
\curveto(493.5896171,702.24798814)(493.5896171,702.28298811)(493.57961763,702.31299154)
\lineto(493.57961763,702.44799154)
\curveto(493.57961711,702.49798789)(493.57461711,702.54798784)(493.56461763,702.59799154)
\curveto(493.55461713,702.64798774)(493.54961714,702.70298769)(493.54961763,702.76299154)
\curveto(493.54961714,702.83298756)(493.55461713,702.8879875)(493.56461763,702.92799154)
\curveto(493.57461711,702.97798741)(493.57961711,703.02298737)(493.57961763,703.06299154)
\lineto(493.57961763,703.21299154)
\curveto(493.5896171,703.26298713)(493.5896171,703.30798708)(493.57961763,703.34799154)
\curveto(493.57961711,703.39798699)(493.5896171,703.44798694)(493.60961763,703.49799154)
\curveto(493.62961706,703.60798678)(493.64461704,703.71298668)(493.65461763,703.81299154)
\curveto(493.67461701,703.91298648)(493.69961699,704.01298638)(493.72961763,704.11299154)
\curveto(493.76961692,704.23298616)(493.80461688,704.34798604)(493.83461763,704.45799154)
\curveto(493.86461682,704.56798582)(493.90461678,704.67798571)(493.95461763,704.78799154)
\curveto(494.09461659,705.0879853)(494.26961642,705.37298502)(494.47961763,705.64299154)
\curveto(494.49961619,705.67298472)(494.52461616,705.69798469)(494.55461763,705.71799154)
\curveto(494.59461609,705.74798464)(494.62461606,705.77798461)(494.64461763,705.80799154)
\curveto(494.684616,705.85798453)(494.72461596,705.90298449)(494.76461763,705.94299154)
\curveto(494.80461588,705.98298441)(494.84961584,706.02298437)(494.89961763,706.06299154)
\curveto(494.93961575,706.08298431)(494.97461571,706.10798428)(495.00461763,706.13799154)
\curveto(495.03461565,706.17798421)(495.06961562,706.20798418)(495.10961763,706.22799154)
\curveto(495.35961533,706.39798399)(495.64961504,706.53798385)(495.97961763,706.64799154)
\curveto(496.04961464,706.66798372)(496.11961457,706.68298371)(496.18961763,706.69299154)
\curveto(496.26961442,706.70298369)(496.34961434,706.71798367)(496.42961763,706.73799154)
\curveto(496.49961419,706.75798363)(496.5896141,706.76798362)(496.69961763,706.76799154)
\curveto(496.80961388,706.77798361)(496.91961377,706.78298361)(497.02961763,706.78299154)
\curveto(497.13961355,706.78298361)(497.24461344,706.77798361)(497.34461763,706.76799154)
\curveto(497.45461323,706.75798363)(497.54461314,706.74298365)(497.61461763,706.72299154)
\curveto(497.76461292,706.67298372)(497.90961278,706.62798376)(498.04961763,706.58799154)
\curveto(498.1896125,706.54798384)(498.31961237,706.4929839)(498.43961763,706.42299154)
\curveto(498.50961218,706.37298402)(498.57461211,706.32298407)(498.63461763,706.27299154)
\curveto(498.69461199,706.23298416)(498.75961193,706.1879842)(498.82961763,706.13799154)
\curveto(498.86961182,706.10798428)(498.92461176,706.06798432)(498.99461763,706.01799154)
\curveto(499.07461161,705.96798442)(499.14961154,705.96798442)(499.21961763,706.01799154)
\curveto(499.25961143,706.03798435)(499.27961141,706.07298432)(499.27961763,706.12299154)
\curveto(499.27961141,706.17298422)(499.2896114,706.22298417)(499.30961763,706.27299154)
\lineto(499.30961763,706.42299154)
\curveto(499.31961137,706.45298394)(499.32461136,706.4879839)(499.32461763,706.52799154)
\lineto(499.32461763,706.64799154)
\lineto(499.32461763,708.68799154)
\curveto(499.32461136,708.79798159)(499.31961137,708.91798147)(499.30961763,709.04799154)
\curveto(499.30961138,709.1879812)(499.33461135,709.2929811)(499.38461763,709.36299154)
\curveto(499.42461126,709.44298095)(499.49961119,709.4929809)(499.60961763,709.51299154)
\curveto(499.62961106,709.52298087)(499.64961104,709.52298087)(499.66961763,709.51299154)
\curveto(499.689611,709.51298088)(499.70961098,709.51798087)(499.72961763,709.52799154)
\lineto(500.79461763,709.52799154)
\curveto(500.91460977,709.52798086)(501.02460966,709.52298087)(501.12461763,709.51299154)
\curveto(501.22460946,709.50298089)(501.29960939,709.46298093)(501.34961763,709.39299154)
\curveto(501.39960929,709.31298108)(501.42460926,709.20798118)(501.42461763,709.07799154)
\lineto(501.42461763,708.71799154)
\lineto(501.42461763,699.70299154)
\moveto(499.38461763,702.64299154)
\curveto(499.39461129,702.68298771)(499.39461129,702.72298767)(499.38461763,702.76299154)
\lineto(499.38461763,702.89799154)
\curveto(499.3846113,702.99798739)(499.37961131,703.09798729)(499.36961763,703.19799154)
\curveto(499.35961133,703.29798709)(499.34461134,703.387987)(499.32461763,703.46799154)
\curveto(499.30461138,703.57798681)(499.2846114,703.67798671)(499.26461763,703.76799154)
\curveto(499.25461143,703.85798653)(499.22961146,703.94298645)(499.18961763,704.02299154)
\curveto(499.04961164,704.38298601)(498.84461184,704.66798572)(498.57461763,704.87799154)
\curveto(498.31461237,705.0879853)(497.93461275,705.1929852)(497.43461763,705.19299154)
\curveto(497.37461331,705.1929852)(497.29461339,705.18298521)(497.19461763,705.16299154)
\curveto(497.11461357,705.14298525)(497.03961365,705.12298527)(496.96961763,705.10299154)
\curveto(496.90961378,705.0929853)(496.84961384,705.07298532)(496.78961763,705.04299154)
\curveto(496.51961417,704.93298546)(496.30961438,704.76298563)(496.15961763,704.53299154)
\curveto(496.00961468,704.30298609)(495.8896148,704.04298635)(495.79961763,703.75299154)
\curveto(495.76961492,703.65298674)(495.74961494,703.55298684)(495.73961763,703.45299154)
\curveto(495.72961496,703.35298704)(495.70961498,703.24798714)(495.67961763,703.13799154)
\lineto(495.67961763,702.92799154)
\curveto(495.65961503,702.83798755)(495.65461503,702.71298768)(495.66461763,702.55299154)
\curveto(495.67461501,702.40298799)(495.689615,702.2929881)(495.70961763,702.22299154)
\lineto(495.70961763,702.13299154)
\curveto(495.71961497,702.11298828)(495.72461496,702.0929883)(495.72461763,702.07299154)
\curveto(495.74461494,701.9929884)(495.75961493,701.91798847)(495.76961763,701.84799154)
\curveto(495.7896149,701.77798861)(495.80961488,701.70298869)(495.82961763,701.62299154)
\curveto(495.99961469,701.10298929)(496.2896144,700.71798967)(496.69961763,700.46799154)
\curveto(496.82961386,700.37799001)(497.00961368,700.30799008)(497.23961763,700.25799154)
\curveto(497.27961341,700.24799014)(497.33961335,700.24299015)(497.41961763,700.24299154)
\curveto(497.44961324,700.23299016)(497.49461319,700.22299017)(497.55461763,700.21299154)
\curveto(497.62461306,700.21299018)(497.67961301,700.21799017)(497.71961763,700.22799154)
\curveto(497.79961289,700.24799014)(497.87961281,700.26299013)(497.95961763,700.27299154)
\curveto(498.03961265,700.28299011)(498.11961257,700.30299009)(498.19961763,700.33299154)
\curveto(498.44961224,700.44298995)(498.64961204,700.58298981)(498.79961763,700.75299154)
\curveto(498.94961174,700.92298947)(499.07961161,701.13798925)(499.18961763,701.39799154)
\curveto(499.22961146,701.4879889)(499.25961143,701.57798881)(499.27961763,701.66799154)
\curveto(499.29961139,701.76798862)(499.31961137,701.87298852)(499.33961763,701.98299154)
\curveto(499.34961134,702.03298836)(499.34961134,702.07798831)(499.33961763,702.11799154)
\curveto(499.33961135,702.16798822)(499.34961134,702.21798817)(499.36961763,702.26799154)
\curveto(499.37961131,702.29798809)(499.3846113,702.33298806)(499.38461763,702.37299154)
\lineto(499.38461763,702.50799154)
\lineto(499.38461763,702.64299154)
}
}
{
\newrgbcolor{curcolor}{0 0 0}
\pscustom[linestyle=none,fillstyle=solid,fillcolor=curcolor]
{
\newpath
\moveto(510.36953951,702.79299154)
\curveto(510.38953134,702.71298768)(510.38953134,702.62298777)(510.36953951,702.52299154)
\curveto(510.34953138,702.42298797)(510.31453142,702.35798803)(510.26453951,702.32799154)
\curveto(510.21453152,702.2879881)(510.13953159,702.25798813)(510.03953951,702.23799154)
\curveto(509.94953178,702.22798816)(509.84453189,702.21798817)(509.72453951,702.20799154)
\lineto(509.37953951,702.20799154)
\curveto(509.26953246,702.21798817)(509.16953256,702.22298817)(509.07953951,702.22299154)
\lineto(505.41953951,702.22299154)
\lineto(505.20953951,702.22299154)
\curveto(505.14953658,702.22298817)(505.09453664,702.21298818)(505.04453951,702.19299154)
\curveto(504.96453677,702.15298824)(504.91453682,702.11298828)(504.89453951,702.07299154)
\curveto(504.87453686,702.05298834)(504.85453688,702.01298838)(504.83453951,701.95299154)
\curveto(504.81453692,701.90298849)(504.80953692,701.85298854)(504.81953951,701.80299154)
\curveto(504.83953689,701.74298865)(504.84953688,701.68298871)(504.84953951,701.62299154)
\curveto(504.85953687,701.57298882)(504.87453686,701.51798887)(504.89453951,701.45799154)
\curveto(504.97453676,701.21798917)(505.06953666,701.01798937)(505.17953951,700.85799154)
\curveto(505.29953643,700.70798968)(505.45953627,700.57298982)(505.65953951,700.45299154)
\curveto(505.73953599,700.40298999)(505.81953591,700.36799002)(505.89953951,700.34799154)
\curveto(505.98953574,700.33799005)(506.07953565,700.31799007)(506.16953951,700.28799154)
\curveto(506.24953548,700.26799012)(506.35953537,700.25299014)(506.49953951,700.24299154)
\curveto(506.63953509,700.23299016)(506.75953497,700.23799015)(506.85953951,700.25799154)
\lineto(506.99453951,700.25799154)
\curveto(507.09453464,700.27799011)(507.18453455,700.29799009)(507.26453951,700.31799154)
\curveto(507.35453438,700.34799004)(507.43953429,700.37799001)(507.51953951,700.40799154)
\curveto(507.61953411,700.45798993)(507.729534,700.52298987)(507.84953951,700.60299154)
\curveto(507.97953375,700.68298971)(508.07453366,700.76298963)(508.13453951,700.84299154)
\curveto(508.18453355,700.91298948)(508.2345335,700.97798941)(508.28453951,701.03799154)
\curveto(508.34453339,701.10798928)(508.41453332,701.15798923)(508.49453951,701.18799154)
\curveto(508.59453314,701.23798915)(508.71953301,701.25798913)(508.86953951,701.24799154)
\lineto(509.30453951,701.24799154)
\lineto(509.48453951,701.24799154)
\curveto(509.55453218,701.25798913)(509.61453212,701.25298914)(509.66453951,701.23299154)
\lineto(509.81453951,701.23299154)
\curveto(509.91453182,701.21298918)(509.98453175,701.1879892)(510.02453951,701.15799154)
\curveto(510.06453167,701.13798925)(510.08453165,701.0929893)(510.08453951,701.02299154)
\curveto(510.09453164,700.95298944)(510.08953164,700.8929895)(510.06953951,700.84299154)
\curveto(510.01953171,700.70298969)(509.96453177,700.57798981)(509.90453951,700.46799154)
\curveto(509.84453189,700.35799003)(509.77453196,700.24799014)(509.69453951,700.13799154)
\curveto(509.47453226,699.80799058)(509.22453251,699.54299085)(508.94453951,699.34299154)
\curveto(508.66453307,699.14299125)(508.31453342,698.97299142)(507.89453951,698.83299154)
\curveto(507.78453395,698.7929916)(507.67453406,698.76799162)(507.56453951,698.75799154)
\curveto(507.45453428,698.74799164)(507.33953439,698.72799166)(507.21953951,698.69799154)
\curveto(507.17953455,698.6879917)(507.1345346,698.6879917)(507.08453951,698.69799154)
\curveto(507.04453469,698.69799169)(507.00453473,698.6929917)(506.96453951,698.68299154)
\lineto(506.79953951,698.68299154)
\curveto(506.74953498,698.66299173)(506.68953504,698.65799173)(506.61953951,698.66799154)
\curveto(506.55953517,698.66799172)(506.50453523,698.67299172)(506.45453951,698.68299154)
\curveto(506.37453536,698.6929917)(506.30453543,698.6929917)(506.24453951,698.68299154)
\curveto(506.18453555,698.67299172)(506.11953561,698.67799171)(506.04953951,698.69799154)
\curveto(505.99953573,698.71799167)(505.94453579,698.72799166)(505.88453951,698.72799154)
\curveto(505.82453591,698.72799166)(505.76953596,698.73799165)(505.71953951,698.75799154)
\curveto(505.60953612,698.77799161)(505.49953623,698.80299159)(505.38953951,698.83299154)
\curveto(505.27953645,698.85299154)(505.17953655,698.8879915)(505.08953951,698.93799154)
\curveto(504.97953675,698.97799141)(504.87453686,699.01299138)(504.77453951,699.04299154)
\curveto(504.68453705,699.08299131)(504.59953713,699.12799126)(504.51953951,699.17799154)
\curveto(504.19953753,699.37799101)(503.91453782,699.60799078)(503.66453951,699.86799154)
\curveto(503.41453832,700.13799025)(503.20953852,700.44798994)(503.04953951,700.79799154)
\curveto(502.99953873,700.90798948)(502.95953877,701.01798937)(502.92953951,701.12799154)
\curveto(502.89953883,701.24798914)(502.85953887,701.36798902)(502.80953951,701.48799154)
\curveto(502.79953893,701.52798886)(502.79453894,701.56298883)(502.79453951,701.59299154)
\curveto(502.79453894,701.63298876)(502.78953894,701.67298872)(502.77953951,701.71299154)
\curveto(502.73953899,701.83298856)(502.71453902,701.96298843)(502.70453951,702.10299154)
\lineto(502.67453951,702.52299154)
\curveto(502.67453906,702.57298782)(502.66953906,702.62798776)(502.65953951,702.68799154)
\curveto(502.65953907,702.74798764)(502.66453907,702.80298759)(502.67453951,702.85299154)
\lineto(502.67453951,703.03299154)
\lineto(502.71953951,703.39299154)
\curveto(502.75953897,703.56298683)(502.79453894,703.72798666)(502.82453951,703.88799154)
\curveto(502.85453888,704.04798634)(502.89953883,704.19798619)(502.95953951,704.33799154)
\curveto(503.38953834,705.37798501)(504.11953761,706.11298428)(505.14953951,706.54299154)
\curveto(505.28953644,706.60298379)(505.4295363,706.64298375)(505.56953951,706.66299154)
\curveto(505.71953601,706.6929837)(505.87453586,706.72798366)(506.03453951,706.76799154)
\curveto(506.11453562,706.77798361)(506.18953554,706.78298361)(506.25953951,706.78299154)
\curveto(506.3295354,706.78298361)(506.40453533,706.7879836)(506.48453951,706.79799154)
\curveto(506.99453474,706.80798358)(507.4295343,706.74798364)(507.78953951,706.61799154)
\curveto(508.15953357,706.49798389)(508.48953324,706.33798405)(508.77953951,706.13799154)
\curveto(508.86953286,706.07798431)(508.95953277,706.00798438)(509.04953951,705.92799154)
\curveto(509.13953259,705.85798453)(509.21953251,705.78298461)(509.28953951,705.70299154)
\curveto(509.31953241,705.65298474)(509.35953237,705.61298478)(509.40953951,705.58299154)
\curveto(509.48953224,705.47298492)(509.56453217,705.35798503)(509.63453951,705.23799154)
\curveto(509.70453203,705.12798526)(509.77953195,705.01298538)(509.85953951,704.89299154)
\curveto(509.90953182,704.80298559)(509.94953178,704.70798568)(509.97953951,704.60799154)
\curveto(510.01953171,704.51798587)(510.05953167,704.41798597)(510.09953951,704.30799154)
\curveto(510.14953158,704.17798621)(510.18953154,704.04298635)(510.21953951,703.90299154)
\curveto(510.24953148,703.76298663)(510.28453145,703.62298677)(510.32453951,703.48299154)
\curveto(510.34453139,703.40298699)(510.34953138,703.31298708)(510.33953951,703.21299154)
\curveto(510.33953139,703.12298727)(510.34953138,703.03798735)(510.36953951,702.95799154)
\lineto(510.36953951,702.79299154)
\moveto(508.11953951,703.67799154)
\curveto(508.18953354,703.77798661)(508.19453354,703.89798649)(508.13453951,704.03799154)
\curveto(508.08453365,704.1879862)(508.04453369,704.29798609)(508.01453951,704.36799154)
\curveto(507.87453386,704.63798575)(507.68953404,704.84298555)(507.45953951,704.98299154)
\curveto(507.2295345,705.13298526)(506.90953482,705.21298518)(506.49953951,705.22299154)
\curveto(506.46953526,705.20298519)(506.4345353,705.19798519)(506.39453951,705.20799154)
\curveto(506.35453538,705.21798517)(506.31953541,705.21798517)(506.28953951,705.20799154)
\curveto(506.23953549,705.1879852)(506.18453555,705.17298522)(506.12453951,705.16299154)
\curveto(506.06453567,705.16298523)(506.00953572,705.15298524)(505.95953951,705.13299154)
\curveto(505.51953621,704.9929854)(505.19453654,704.71798567)(504.98453951,704.30799154)
\curveto(504.96453677,704.26798612)(504.93953679,704.21298618)(504.90953951,704.14299154)
\curveto(504.88953684,704.08298631)(504.87453686,704.01798637)(504.86453951,703.94799154)
\curveto(504.85453688,703.8879865)(504.85453688,703.82798656)(504.86453951,703.76799154)
\curveto(504.88453685,703.70798668)(504.91953681,703.65798673)(504.96953951,703.61799154)
\curveto(505.04953668,703.56798682)(505.15953657,703.54298685)(505.29953951,703.54299154)
\lineto(505.70453951,703.54299154)
\lineto(507.36953951,703.54299154)
\lineto(507.80453951,703.54299154)
\curveto(507.96453377,703.55298684)(508.06953366,703.59798679)(508.11953951,703.67799154)
}
}
{
\newrgbcolor{curcolor}{0 0 0}
\pscustom[linestyle=none,fillstyle=solid,fillcolor=curcolor]
{
}
}
{
\newrgbcolor{curcolor}{0 0 0}
\pscustom[linestyle=none,fillstyle=solid,fillcolor=curcolor]
{
\newpath
\moveto(523.66797701,702.80799154)
\curveto(523.67796833,702.74798764)(523.68296832,702.65798773)(523.68297701,702.53799154)
\curveto(523.68296832,702.41798797)(523.67296833,702.33298806)(523.65297701,702.28299154)
\lineto(523.65297701,702.08799154)
\curveto(523.62296838,701.97798841)(523.6029684,701.87298852)(523.59297701,701.77299154)
\curveto(523.59296841,701.67298872)(523.57796843,701.57298882)(523.54797701,701.47299154)
\curveto(523.52796848,701.38298901)(523.5079685,701.2879891)(523.48797701,701.18799154)
\curveto(523.46796854,701.09798929)(523.43796857,701.00798938)(523.39797701,700.91799154)
\curveto(523.32796868,700.74798964)(523.25796875,700.5879898)(523.18797701,700.43799154)
\curveto(523.11796889,700.29799009)(523.03796897,700.15799023)(522.94797701,700.01799154)
\curveto(522.88796912,699.92799046)(522.82296918,699.84299055)(522.75297701,699.76299154)
\curveto(522.69296931,699.6929907)(522.62296938,699.61799077)(522.54297701,699.53799154)
\lineto(522.43797701,699.43299154)
\curveto(522.38796962,699.38299101)(522.33296967,699.33799105)(522.27297701,699.29799154)
\lineto(522.12297701,699.17799154)
\curveto(522.04296996,699.11799127)(521.95297005,699.06299133)(521.85297701,699.01299154)
\curveto(521.76297024,698.97299142)(521.66797034,698.92799146)(521.56797701,698.87799154)
\curveto(521.46797054,698.82799156)(521.36297064,698.7929916)(521.25297701,698.77299154)
\curveto(521.15297085,698.75299164)(521.04797096,698.73299166)(520.93797701,698.71299154)
\curveto(520.87797113,698.6929917)(520.81297119,698.68299171)(520.74297701,698.68299154)
\curveto(520.68297132,698.68299171)(520.61797139,698.67299172)(520.54797701,698.65299154)
\lineto(520.41297701,698.65299154)
\curveto(520.33297167,698.63299176)(520.25797175,698.63299176)(520.18797701,698.65299154)
\lineto(520.03797701,698.65299154)
\curveto(519.97797203,698.67299172)(519.91297209,698.68299171)(519.84297701,698.68299154)
\curveto(519.78297222,698.67299172)(519.72297228,698.67799171)(519.66297701,698.69799154)
\curveto(519.5029725,698.74799164)(519.34797266,698.7929916)(519.19797701,698.83299154)
\curveto(519.05797295,698.87299152)(518.92797308,698.93299146)(518.80797701,699.01299154)
\curveto(518.73797327,699.05299134)(518.67297333,699.0929913)(518.61297701,699.13299154)
\curveto(518.55297345,699.18299121)(518.48797352,699.23299116)(518.41797701,699.28299154)
\lineto(518.23797701,699.41799154)
\curveto(518.15797385,699.47799091)(518.08797392,699.48299091)(518.02797701,699.43299154)
\curveto(517.97797403,699.40299099)(517.95297405,699.36299103)(517.95297701,699.31299154)
\curveto(517.95297405,699.27299112)(517.94297406,699.22299117)(517.92297701,699.16299154)
\curveto(517.9029741,699.06299133)(517.89297411,698.94799144)(517.89297701,698.81799154)
\curveto(517.9029741,698.6879917)(517.9079741,698.56799182)(517.90797701,698.45799154)
\lineto(517.90797701,696.92799154)
\curveto(517.9079741,696.79799359)(517.9029741,696.67299372)(517.89297701,696.55299154)
\curveto(517.89297411,696.42299397)(517.86797414,696.31799407)(517.81797701,696.23799154)
\curveto(517.78797422,696.19799419)(517.73297427,696.16799422)(517.65297701,696.14799154)
\curveto(517.57297443,696.12799426)(517.48297452,696.11799427)(517.38297701,696.11799154)
\curveto(517.28297472,696.10799428)(517.18297482,696.10799428)(517.08297701,696.11799154)
\lineto(516.82797701,696.11799154)
\lineto(516.42297701,696.11799154)
\lineto(516.31797701,696.11799154)
\curveto(516.27797573,696.11799427)(516.24297576,696.12299427)(516.21297701,696.13299154)
\lineto(516.09297701,696.13299154)
\curveto(515.92297608,696.18299421)(515.83297617,696.28299411)(515.82297701,696.43299154)
\curveto(515.81297619,696.57299382)(515.8079762,696.74299365)(515.80797701,696.94299154)
\lineto(515.80797701,705.74799154)
\curveto(515.8079762,705.85798453)(515.8029762,705.97298442)(515.79297701,706.09299154)
\curveto(515.79297621,706.22298417)(515.81797619,706.32298407)(515.86797701,706.39299154)
\curveto(515.9079761,706.46298393)(515.96297604,706.50798388)(516.03297701,706.52799154)
\curveto(516.08297592,706.54798384)(516.14297586,706.55798383)(516.21297701,706.55799154)
\lineto(516.43797701,706.55799154)
\lineto(517.15797701,706.55799154)
\lineto(517.44297701,706.55799154)
\curveto(517.53297447,706.55798383)(517.6079744,706.53298386)(517.66797701,706.48299154)
\curveto(517.73797427,706.43298396)(517.77297423,706.36798402)(517.77297701,706.28799154)
\curveto(517.78297422,706.21798417)(517.8079742,706.14298425)(517.84797701,706.06299154)
\curveto(517.85797415,706.03298436)(517.86797414,706.00798438)(517.87797701,705.98799154)
\curveto(517.89797411,705.97798441)(517.91797409,705.96298443)(517.93797701,705.94299154)
\curveto(518.04797396,705.93298446)(518.13797387,705.96298443)(518.20797701,706.03299154)
\curveto(518.27797373,706.10298429)(518.34797366,706.16298423)(518.41797701,706.21299154)
\curveto(518.54797346,706.30298409)(518.68297332,706.38298401)(518.82297701,706.45299154)
\curveto(518.96297304,706.53298386)(519.11797289,706.59798379)(519.28797701,706.64799154)
\curveto(519.36797264,706.67798371)(519.45297255,706.69798369)(519.54297701,706.70799154)
\curveto(519.64297236,706.71798367)(519.73797227,706.73298366)(519.82797701,706.75299154)
\curveto(519.86797214,706.76298363)(519.9079721,706.76298363)(519.94797701,706.75299154)
\curveto(519.99797201,706.74298365)(520.03797197,706.74798364)(520.06797701,706.76799154)
\curveto(520.63797137,706.7879836)(521.11797089,706.70798368)(521.50797701,706.52799154)
\curveto(521.9079701,706.35798403)(522.24796976,706.13298426)(522.52797701,705.85299154)
\curveto(522.57796943,705.80298459)(522.62296938,705.75298464)(522.66297701,705.70299154)
\curveto(522.7029693,705.66298473)(522.74296926,705.61798477)(522.78297701,705.56799154)
\curveto(522.85296915,705.47798491)(522.91296909,705.387985)(522.96297701,705.29799154)
\curveto(523.02296898,705.20798518)(523.07796893,705.11798527)(523.12797701,705.02799154)
\curveto(523.14796886,705.00798538)(523.15796885,704.98298541)(523.15797701,704.95299154)
\curveto(523.16796884,704.92298547)(523.18296882,704.8879855)(523.20297701,704.84799154)
\curveto(523.26296874,704.74798564)(523.31796869,704.62798576)(523.36797701,704.48799154)
\curveto(523.38796862,704.42798596)(523.4079686,704.36298603)(523.42797701,704.29299154)
\curveto(523.44796856,704.23298616)(523.46796854,704.16798622)(523.48797701,704.09799154)
\curveto(523.52796848,703.97798641)(523.55296845,703.85298654)(523.56297701,703.72299154)
\curveto(523.58296842,703.5929868)(523.6079684,703.45798693)(523.63797701,703.31799154)
\lineto(523.63797701,703.15299154)
\lineto(523.66797701,702.97299154)
\lineto(523.66797701,702.80799154)
\moveto(521.55297701,702.46299154)
\curveto(521.56297044,702.51298788)(521.56797044,702.57798781)(521.56797701,702.65799154)
\curveto(521.56797044,702.74798764)(521.56297044,702.81798757)(521.55297701,702.86799154)
\lineto(521.55297701,703.00299154)
\curveto(521.53297047,703.06298733)(521.52297048,703.12798726)(521.52297701,703.19799154)
\curveto(521.52297048,703.26798712)(521.51297049,703.33798705)(521.49297701,703.40799154)
\curveto(521.47297053,703.50798688)(521.45297055,703.60298679)(521.43297701,703.69299154)
\curveto(521.41297059,703.7929866)(521.38297062,703.88298651)(521.34297701,703.96299154)
\curveto(521.22297078,704.28298611)(521.06797094,704.53798585)(520.87797701,704.72799154)
\curveto(520.68797132,704.91798547)(520.41797159,705.05798533)(520.06797701,705.14799154)
\curveto(519.98797202,705.16798522)(519.89797211,705.17798521)(519.79797701,705.17799154)
\lineto(519.52797701,705.17799154)
\curveto(519.48797252,705.16798522)(519.45297255,705.16298523)(519.42297701,705.16299154)
\curveto(519.39297261,705.16298523)(519.35797265,705.15798523)(519.31797701,705.14799154)
\lineto(519.10797701,705.08799154)
\curveto(519.04797296,705.07798531)(518.98797302,705.05798533)(518.92797701,705.02799154)
\curveto(518.66797334,704.91798547)(518.46297354,704.74798564)(518.31297701,704.51799154)
\curveto(518.17297383,704.2879861)(518.05797395,704.03298636)(517.96797701,703.75299154)
\curveto(517.94797406,703.67298672)(517.93297407,703.5879868)(517.92297701,703.49799154)
\curveto(517.91297409,703.41798697)(517.89797411,703.33798705)(517.87797701,703.25799154)
\curveto(517.86797414,703.21798717)(517.86297414,703.15298724)(517.86297701,703.06299154)
\curveto(517.84297416,703.02298737)(517.83797417,702.97298742)(517.84797701,702.91299154)
\curveto(517.85797415,702.86298753)(517.85797415,702.81298758)(517.84797701,702.76299154)
\curveto(517.82797418,702.70298769)(517.82797418,702.64798774)(517.84797701,702.59799154)
\lineto(517.84797701,702.41799154)
\lineto(517.84797701,702.28299154)
\curveto(517.84797416,702.24298815)(517.85797415,702.20298819)(517.87797701,702.16299154)
\curveto(517.87797413,702.0929883)(517.88297412,702.03798835)(517.89297701,701.99799154)
\lineto(517.92297701,701.81799154)
\curveto(517.93297407,701.75798863)(517.94797406,701.69798869)(517.96797701,701.63799154)
\curveto(518.05797395,701.34798904)(518.16297384,701.10798928)(518.28297701,700.91799154)
\curveto(518.41297359,700.73798965)(518.59297341,700.57798981)(518.82297701,700.43799154)
\curveto(518.96297304,700.35799003)(519.12797288,700.2929901)(519.31797701,700.24299154)
\curveto(519.35797265,700.23299016)(519.39297261,700.22799016)(519.42297701,700.22799154)
\curveto(519.45297255,700.23799015)(519.48797252,700.23799015)(519.52797701,700.22799154)
\curveto(519.56797244,700.21799017)(519.62797238,700.20799018)(519.70797701,700.19799154)
\curveto(519.78797222,700.19799019)(519.85297215,700.20299019)(519.90297701,700.21299154)
\curveto(519.98297202,700.23299016)(520.06297194,700.24799014)(520.14297701,700.25799154)
\curveto(520.23297177,700.27799011)(520.31797169,700.30299009)(520.39797701,700.33299154)
\curveto(520.63797137,700.43298996)(520.83297117,700.57298982)(520.98297701,700.75299154)
\curveto(521.13297087,700.93298946)(521.25797075,701.14298925)(521.35797701,701.38299154)
\curveto(521.4079706,701.50298889)(521.44297056,701.62798876)(521.46297701,701.75799154)
\curveto(521.48297052,701.8879885)(521.5079705,702.02298837)(521.53797701,702.16299154)
\lineto(521.53797701,702.31299154)
\curveto(521.54797046,702.36298803)(521.55297045,702.41298798)(521.55297701,702.46299154)
}
}
{
\newrgbcolor{curcolor}{0 0 0}
\pscustom[linestyle=none,fillstyle=solid,fillcolor=curcolor]
{
\newpath
\moveto(531.99789888,699.44799154)
\curveto(532.01789103,699.33799105)(532.02789102,699.22799116)(532.02789888,699.11799154)
\curveto(532.03789101,699.00799138)(531.98789106,698.93299146)(531.87789888,698.89299154)
\curveto(531.81789123,698.86299153)(531.7478913,698.84799154)(531.66789888,698.84799154)
\lineto(531.42789888,698.84799154)
\lineto(530.61789888,698.84799154)
\lineto(530.34789888,698.84799154)
\curveto(530.26789278,698.85799153)(530.20289285,698.88299151)(530.15289888,698.92299154)
\curveto(530.08289297,698.96299143)(530.02789302,699.01799137)(529.98789888,699.08799154)
\curveto(529.95789309,699.16799122)(529.91289314,699.23299116)(529.85289888,699.28299154)
\curveto(529.83289322,699.30299109)(529.80789324,699.31799107)(529.77789888,699.32799154)
\curveto(529.7478933,699.34799104)(529.70789334,699.35299104)(529.65789888,699.34299154)
\curveto(529.60789344,699.32299107)(529.55789349,699.29799109)(529.50789888,699.26799154)
\curveto(529.46789358,699.23799115)(529.42289363,699.21299118)(529.37289888,699.19299154)
\curveto(529.32289373,699.15299124)(529.26789378,699.11799127)(529.20789888,699.08799154)
\lineto(529.02789888,698.99799154)
\curveto(528.89789415,698.93799145)(528.76289429,698.8879915)(528.62289888,698.84799154)
\curveto(528.48289457,698.81799157)(528.33789471,698.78299161)(528.18789888,698.74299154)
\curveto(528.11789493,698.72299167)(528.047895,698.71299168)(527.97789888,698.71299154)
\curveto(527.91789513,698.70299169)(527.8528952,698.6929917)(527.78289888,698.68299154)
\lineto(527.69289888,698.68299154)
\curveto(527.66289539,698.67299172)(527.63289542,698.66799172)(527.60289888,698.66799154)
\lineto(527.43789888,698.66799154)
\curveto(527.33789571,698.64799174)(527.23789581,698.64799174)(527.13789888,698.66799154)
\lineto(527.00289888,698.66799154)
\curveto(526.93289612,698.6879917)(526.86289619,698.69799169)(526.79289888,698.69799154)
\curveto(526.73289632,698.6879917)(526.67289638,698.6929917)(526.61289888,698.71299154)
\curveto(526.51289654,698.73299166)(526.41789663,698.75299164)(526.32789888,698.77299154)
\curveto(526.23789681,698.78299161)(526.1528969,698.80799158)(526.07289888,698.84799154)
\curveto(525.78289727,698.95799143)(525.53289752,699.09799129)(525.32289888,699.26799154)
\curveto(525.12289793,699.44799094)(524.96289809,699.68299071)(524.84289888,699.97299154)
\curveto(524.81289824,700.04299035)(524.78289827,700.11799027)(524.75289888,700.19799154)
\curveto(524.73289832,700.27799011)(524.71289834,700.36299003)(524.69289888,700.45299154)
\curveto(524.67289838,700.50298989)(524.66289839,700.55298984)(524.66289888,700.60299154)
\curveto(524.67289838,700.65298974)(524.67289838,700.70298969)(524.66289888,700.75299154)
\curveto(524.6528984,700.78298961)(524.64289841,700.84298955)(524.63289888,700.93299154)
\curveto(524.63289842,701.03298936)(524.63789841,701.10298929)(524.64789888,701.14299154)
\curveto(524.66789838,701.24298915)(524.67789837,701.32798906)(524.67789888,701.39799154)
\lineto(524.76789888,701.72799154)
\curveto(524.79789825,701.84798854)(524.83789821,701.95298844)(524.88789888,702.04299154)
\curveto(525.05789799,702.33298806)(525.2528978,702.55298784)(525.47289888,702.70299154)
\curveto(525.69289736,702.85298754)(525.97289708,702.98298741)(526.31289888,703.09299154)
\curveto(526.44289661,703.14298725)(526.57789647,703.17798721)(526.71789888,703.19799154)
\curveto(526.85789619,703.21798717)(526.99789605,703.24298715)(527.13789888,703.27299154)
\curveto(527.21789583,703.2929871)(527.30289575,703.30298709)(527.39289888,703.30299154)
\curveto(527.48289557,703.31298708)(527.57289548,703.32798706)(527.66289888,703.34799154)
\curveto(527.73289532,703.36798702)(527.80289525,703.37298702)(527.87289888,703.36299154)
\curveto(527.94289511,703.36298703)(528.01789503,703.37298702)(528.09789888,703.39299154)
\curveto(528.16789488,703.41298698)(528.23789481,703.42298697)(528.30789888,703.42299154)
\curveto(528.37789467,703.42298697)(528.4528946,703.43298696)(528.53289888,703.45299154)
\curveto(528.74289431,703.50298689)(528.93289412,703.54298685)(529.10289888,703.57299154)
\curveto(529.28289377,703.61298678)(529.44289361,703.70298669)(529.58289888,703.84299154)
\curveto(529.67289338,703.93298646)(529.73289332,704.03298636)(529.76289888,704.14299154)
\curveto(529.77289328,704.17298622)(529.77289328,704.19798619)(529.76289888,704.21799154)
\curveto(529.76289329,704.23798615)(529.76789328,704.25798613)(529.77789888,704.27799154)
\curveto(529.78789326,704.29798609)(529.79289326,704.32798606)(529.79289888,704.36799154)
\lineto(529.79289888,704.45799154)
\lineto(529.76289888,704.57799154)
\curveto(529.76289329,704.61798577)(529.75789329,704.65298574)(529.74789888,704.68299154)
\curveto(529.6478934,704.98298541)(529.43789361,705.1879852)(529.11789888,705.29799154)
\curveto(529.02789402,705.32798506)(528.91789413,705.34798504)(528.78789888,705.35799154)
\curveto(528.66789438,705.37798501)(528.54289451,705.38298501)(528.41289888,705.37299154)
\curveto(528.28289477,705.37298502)(528.15789489,705.36298503)(528.03789888,705.34299154)
\curveto(527.91789513,705.32298507)(527.81289524,705.29798509)(527.72289888,705.26799154)
\curveto(527.66289539,705.24798514)(527.60289545,705.21798517)(527.54289888,705.17799154)
\curveto(527.49289556,705.14798524)(527.44289561,705.11298528)(527.39289888,705.07299154)
\curveto(527.34289571,705.03298536)(527.28789576,704.97798541)(527.22789888,704.90799154)
\curveto(527.17789587,704.83798555)(527.14289591,704.77298562)(527.12289888,704.71299154)
\curveto(527.07289598,704.61298578)(527.02789602,704.51798587)(526.98789888,704.42799154)
\curveto(526.95789609,704.33798605)(526.88789616,704.27798611)(526.77789888,704.24799154)
\curveto(526.69789635,704.22798616)(526.61289644,704.21798617)(526.52289888,704.21799154)
\lineto(526.25289888,704.21799154)
\lineto(525.68289888,704.21799154)
\curveto(525.63289742,704.21798617)(525.58289747,704.21298618)(525.53289888,704.20299154)
\curveto(525.48289757,704.20298619)(525.43789761,704.20798618)(525.39789888,704.21799154)
\lineto(525.26289888,704.21799154)
\curveto(525.24289781,704.22798616)(525.21789783,704.23298616)(525.18789888,704.23299154)
\curveto(525.15789789,704.23298616)(525.13289792,704.24298615)(525.11289888,704.26299154)
\curveto(525.03289802,704.28298611)(524.97789807,704.34798604)(524.94789888,704.45799154)
\curveto(524.93789811,704.50798588)(524.93789811,704.55798583)(524.94789888,704.60799154)
\curveto(524.95789809,704.65798573)(524.96789808,704.70298569)(524.97789888,704.74299154)
\curveto(525.00789804,704.85298554)(525.03789801,704.95298544)(525.06789888,705.04299154)
\curveto(525.10789794,705.14298525)(525.1528979,705.23298516)(525.20289888,705.31299154)
\lineto(525.29289888,705.46299154)
\lineto(525.38289888,705.61299154)
\curveto(525.46289759,705.72298467)(525.56289749,705.82798456)(525.68289888,705.92799154)
\curveto(525.70289735,705.93798445)(525.73289732,705.96298443)(525.77289888,706.00299154)
\curveto(525.82289723,706.04298435)(525.86789718,706.07798431)(525.90789888,706.10799154)
\curveto(525.9478971,706.13798425)(525.99289706,706.16798422)(526.04289888,706.19799154)
\curveto(526.21289684,706.30798408)(526.39289666,706.392984)(526.58289888,706.45299154)
\curveto(526.77289628,706.52298387)(526.96789608,706.5879838)(527.16789888,706.64799154)
\curveto(527.28789576,706.67798371)(527.41289564,706.69798369)(527.54289888,706.70799154)
\curveto(527.67289538,706.71798367)(527.80289525,706.73798365)(527.93289888,706.76799154)
\curveto(527.97289508,706.77798361)(528.03289502,706.77798361)(528.11289888,706.76799154)
\curveto(528.20289485,706.75798363)(528.25789479,706.76298363)(528.27789888,706.78299154)
\curveto(528.68789436,706.7929836)(529.07789397,706.77798361)(529.44789888,706.73799154)
\curveto(529.82789322,706.69798369)(530.16789288,706.62298377)(530.46789888,706.51299154)
\curveto(530.77789227,706.40298399)(531.04289201,706.25298414)(531.26289888,706.06299154)
\curveto(531.48289157,705.88298451)(531.6528914,705.64798474)(531.77289888,705.35799154)
\curveto(531.84289121,705.1879852)(531.88289117,704.9929854)(531.89289888,704.77299154)
\curveto(531.90289115,704.55298584)(531.90789114,704.32798606)(531.90789888,704.09799154)
\lineto(531.90789888,700.75299154)
\lineto(531.90789888,700.16799154)
\curveto(531.90789114,699.97799041)(531.92789112,699.80299059)(531.96789888,699.64299154)
\curveto(531.97789107,699.61299078)(531.98289107,699.57799081)(531.98289888,699.53799154)
\curveto(531.98289107,699.50799088)(531.98789106,699.47799091)(531.99789888,699.44799154)
\moveto(529.79289888,701.75799154)
\curveto(529.80289325,701.80798858)(529.80789324,701.86298853)(529.80789888,701.92299154)
\curveto(529.80789324,701.9929884)(529.80289325,702.05298834)(529.79289888,702.10299154)
\curveto(529.77289328,702.16298823)(529.76289329,702.21798817)(529.76289888,702.26799154)
\curveto(529.76289329,702.31798807)(529.74289331,702.35798803)(529.70289888,702.38799154)
\curveto(529.6528934,702.42798796)(529.57789347,702.44798794)(529.47789888,702.44799154)
\curveto(529.43789361,702.43798795)(529.40289365,702.42798796)(529.37289888,702.41799154)
\curveto(529.34289371,702.41798797)(529.30789374,702.41298798)(529.26789888,702.40299154)
\curveto(529.19789385,702.38298801)(529.12289393,702.36798802)(529.04289888,702.35799154)
\curveto(528.96289409,702.34798804)(528.88289417,702.33298806)(528.80289888,702.31299154)
\curveto(528.77289428,702.30298809)(528.72789432,702.29798809)(528.66789888,702.29799154)
\curveto(528.53789451,702.26798812)(528.40789464,702.24798814)(528.27789888,702.23799154)
\curveto(528.1478949,702.22798816)(528.02289503,702.20298819)(527.90289888,702.16299154)
\curveto(527.82289523,702.14298825)(527.7478953,702.12298827)(527.67789888,702.10299154)
\curveto(527.60789544,702.0929883)(527.53789551,702.07298832)(527.46789888,702.04299154)
\curveto(527.25789579,701.95298844)(527.07789597,701.81798857)(526.92789888,701.63799154)
\curveto(526.78789626,701.45798893)(526.73789631,701.20798918)(526.77789888,700.88799154)
\curveto(526.79789625,700.71798967)(526.8528962,700.57798981)(526.94289888,700.46799154)
\curveto(527.01289604,700.35799003)(527.11789593,700.26799012)(527.25789888,700.19799154)
\curveto(527.39789565,700.13799025)(527.5478955,700.0929903)(527.70789888,700.06299154)
\curveto(527.87789517,700.03299036)(528.052895,700.02299037)(528.23289888,700.03299154)
\curveto(528.42289463,700.05299034)(528.59789445,700.0879903)(528.75789888,700.13799154)
\curveto(529.01789403,700.21799017)(529.22289383,700.34299005)(529.37289888,700.51299154)
\curveto(529.52289353,700.6929897)(529.63789341,700.91298948)(529.71789888,701.17299154)
\curveto(529.73789331,701.24298915)(529.7478933,701.31298908)(529.74789888,701.38299154)
\curveto(529.75789329,701.46298893)(529.77289328,701.54298885)(529.79289888,701.62299154)
\lineto(529.79289888,701.75799154)
}
}
{
\newrgbcolor{curcolor}{0 0 0}
\pscustom[linestyle=none,fillstyle=solid,fillcolor=curcolor]
{
\newpath
\moveto(537.98618013,706.78299154)
\curveto(538.09617482,706.78298361)(538.19117472,706.77298362)(538.27118013,706.75299154)
\curveto(538.36117455,706.73298366)(538.43117448,706.6879837)(538.48118013,706.61799154)
\curveto(538.54117437,706.53798385)(538.57117434,706.39798399)(538.57118013,706.19799154)
\lineto(538.57118013,705.68799154)
\lineto(538.57118013,705.31299154)
\curveto(538.58117433,705.17298522)(538.56617435,705.06298533)(538.52618013,704.98299154)
\curveto(538.48617443,704.91298548)(538.42617449,704.86798552)(538.34618013,704.84799154)
\curveto(538.27617464,704.82798556)(538.19117472,704.81798557)(538.09118013,704.81799154)
\curveto(538.00117491,704.81798557)(537.90117501,704.82298557)(537.79118013,704.83299154)
\curveto(537.69117522,704.84298555)(537.59617532,704.83798555)(537.50618013,704.81799154)
\curveto(537.43617548,704.79798559)(537.36617555,704.78298561)(537.29618013,704.77299154)
\curveto(537.22617569,704.77298562)(537.16117575,704.76298563)(537.10118013,704.74299154)
\curveto(536.94117597,704.6929857)(536.78117613,704.61798577)(536.62118013,704.51799154)
\curveto(536.46117645,704.42798596)(536.33617658,704.32298607)(536.24618013,704.20299154)
\curveto(536.19617672,704.12298627)(536.14117677,704.03798635)(536.08118013,703.94799154)
\curveto(536.03117688,703.86798652)(535.98117693,703.78298661)(535.93118013,703.69299154)
\curveto(535.90117701,703.61298678)(535.87117704,703.52798686)(535.84118013,703.43799154)
\lineto(535.78118013,703.19799154)
\curveto(535.76117715,703.12798726)(535.75117716,703.05298734)(535.75118013,702.97299154)
\curveto(535.75117716,702.90298749)(535.74117717,702.83298756)(535.72118013,702.76299154)
\curveto(535.7111772,702.72298767)(535.70617721,702.68298771)(535.70618013,702.64299154)
\curveto(535.7161772,702.61298778)(535.7161772,702.58298781)(535.70618013,702.55299154)
\lineto(535.70618013,702.31299154)
\curveto(535.68617723,702.24298815)(535.68117723,702.16298823)(535.69118013,702.07299154)
\curveto(535.70117721,701.9929884)(535.70617721,701.91298848)(535.70618013,701.83299154)
\lineto(535.70618013,700.87299154)
\lineto(535.70618013,699.59799154)
\curveto(535.70617721,699.46799092)(535.70117721,699.34799104)(535.69118013,699.23799154)
\curveto(535.68117723,699.12799126)(535.65117726,699.03799135)(535.60118013,698.96799154)
\curveto(535.58117733,698.93799145)(535.54617737,698.91299148)(535.49618013,698.89299154)
\curveto(535.45617746,698.88299151)(535.4111775,698.87299152)(535.36118013,698.86299154)
\lineto(535.28618013,698.86299154)
\curveto(535.23617768,698.85299154)(535.18117773,698.84799154)(535.12118013,698.84799154)
\lineto(534.95618013,698.84799154)
\lineto(534.31118013,698.84799154)
\curveto(534.25117866,698.85799153)(534.18617873,698.86299153)(534.11618013,698.86299154)
\lineto(533.92118013,698.86299154)
\curveto(533.87117904,698.88299151)(533.82117909,698.89799149)(533.77118013,698.90799154)
\curveto(533.72117919,698.92799146)(533.68617923,698.96299143)(533.66618013,699.01299154)
\curveto(533.62617929,699.06299133)(533.60117931,699.13299126)(533.59118013,699.22299154)
\lineto(533.59118013,699.52299154)
\lineto(533.59118013,700.54299154)
\lineto(533.59118013,704.77299154)
\lineto(533.59118013,705.88299154)
\lineto(533.59118013,706.16799154)
\curveto(533.59117932,706.26798412)(533.6111793,706.34798404)(533.65118013,706.40799154)
\curveto(533.70117921,706.4879839)(533.77617914,706.53798385)(533.87618013,706.55799154)
\curveto(533.97617894,706.57798381)(534.09617882,706.5879838)(534.23618013,706.58799154)
\lineto(535.00118013,706.58799154)
\curveto(535.12117779,706.5879838)(535.22617769,706.57798381)(535.31618013,706.55799154)
\curveto(535.40617751,706.54798384)(535.47617744,706.50298389)(535.52618013,706.42299154)
\curveto(535.55617736,706.37298402)(535.57117734,706.30298409)(535.57118013,706.21299154)
\lineto(535.60118013,705.94299154)
\curveto(535.6111773,705.86298453)(535.62617729,705.7879846)(535.64618013,705.71799154)
\curveto(535.67617724,705.64798474)(535.72617719,705.61298478)(535.79618013,705.61299154)
\curveto(535.8161771,705.63298476)(535.83617708,705.64298475)(535.85618013,705.64299154)
\curveto(535.87617704,705.64298475)(535.89617702,705.65298474)(535.91618013,705.67299154)
\curveto(535.97617694,705.72298467)(536.02617689,705.77798461)(536.06618013,705.83799154)
\curveto(536.1161768,705.90798448)(536.17617674,705.96798442)(536.24618013,706.01799154)
\curveto(536.28617663,706.04798434)(536.32117659,706.07798431)(536.35118013,706.10799154)
\curveto(536.38117653,706.14798424)(536.4161765,706.18298421)(536.45618013,706.21299154)
\lineto(536.72618013,706.39299154)
\curveto(536.82617609,706.45298394)(536.92617599,706.50798388)(537.02618013,706.55799154)
\curveto(537.12617579,706.59798379)(537.22617569,706.63298376)(537.32618013,706.66299154)
\lineto(537.65618013,706.75299154)
\curveto(537.68617523,706.76298363)(537.74117517,706.76298363)(537.82118013,706.75299154)
\curveto(537.911175,706.75298364)(537.96617495,706.76298363)(537.98618013,706.78299154)
}
}
{
\newrgbcolor{curcolor}{0 0 0}
\pscustom[linestyle=none,fillstyle=solid,fillcolor=curcolor]
{
\newpath
\moveto(540.44125826,708.89799154)
\lineto(541.44625826,708.89799154)
\curveto(541.59625527,708.89798149)(541.72625514,708.8879815)(541.83625826,708.86799154)
\curveto(541.95625491,708.85798153)(542.04125483,708.79798159)(542.09125826,708.68799154)
\curveto(542.11125476,708.63798175)(542.12125475,708.57798181)(542.12125826,708.50799154)
\lineto(542.12125826,708.29799154)
\lineto(542.12125826,707.62299154)
\curveto(542.12125475,707.57298282)(542.11625475,707.51298288)(542.10625826,707.44299154)
\curveto(542.10625476,707.38298301)(542.11125476,707.32798306)(542.12125826,707.27799154)
\lineto(542.12125826,707.11299154)
\curveto(542.12125475,707.03298336)(542.12625474,706.95798343)(542.13625826,706.88799154)
\curveto(542.14625472,706.82798356)(542.1712547,706.77298362)(542.21125826,706.72299154)
\curveto(542.28125459,706.63298376)(542.40625446,706.58298381)(542.58625826,706.57299154)
\lineto(543.12625826,706.57299154)
\lineto(543.30625826,706.57299154)
\curveto(543.3662535,706.57298382)(543.42125345,706.56298383)(543.47125826,706.54299154)
\curveto(543.58125329,706.4929839)(543.64125323,706.40298399)(543.65125826,706.27299154)
\curveto(543.6712532,706.14298425)(543.68125319,705.99798439)(543.68125826,705.83799154)
\lineto(543.68125826,705.62799154)
\curveto(543.69125318,705.55798483)(543.68625318,705.49798489)(543.66625826,705.44799154)
\curveto(543.61625325,705.2879851)(543.51125336,705.20298519)(543.35125826,705.19299154)
\curveto(543.19125368,705.18298521)(543.01125386,705.17798521)(542.81125826,705.17799154)
\lineto(542.67625826,705.17799154)
\curveto(542.63625423,705.1879852)(542.60125427,705.1879852)(542.57125826,705.17799154)
\curveto(542.53125434,705.16798522)(542.49625437,705.16298523)(542.46625826,705.16299154)
\curveto(542.43625443,705.17298522)(542.40625446,705.16798522)(542.37625826,705.14799154)
\curveto(542.29625457,705.12798526)(542.23625463,705.08298531)(542.19625826,705.01299154)
\curveto(542.1662547,704.95298544)(542.14125473,704.87798551)(542.12125826,704.78799154)
\curveto(542.11125476,704.73798565)(542.11125476,704.68298571)(542.12125826,704.62299154)
\curveto(542.13125474,704.56298583)(542.13125474,704.50798588)(542.12125826,704.45799154)
\lineto(542.12125826,703.52799154)
\lineto(542.12125826,701.77299154)
\curveto(542.12125475,701.52298887)(542.12625474,701.30298909)(542.13625826,701.11299154)
\curveto(542.15625471,700.93298946)(542.22125465,700.77298962)(542.33125826,700.63299154)
\curveto(542.38125449,700.57298982)(542.44625442,700.52798986)(542.52625826,700.49799154)
\lineto(542.79625826,700.43799154)
\curveto(542.82625404,700.42798996)(542.85625401,700.42298997)(542.88625826,700.42299154)
\curveto(542.92625394,700.43298996)(542.95625391,700.43298996)(542.97625826,700.42299154)
\lineto(543.14125826,700.42299154)
\curveto(543.25125362,700.42298997)(543.34625352,700.41798997)(543.42625826,700.40799154)
\curveto(543.50625336,700.39798999)(543.5712533,700.35799003)(543.62125826,700.28799154)
\curveto(543.66125321,700.22799016)(543.68125319,700.14799024)(543.68125826,700.04799154)
\lineto(543.68125826,699.76299154)
\curveto(543.68125319,699.55299084)(543.67625319,699.35799103)(543.66625826,699.17799154)
\curveto(543.6662532,699.00799138)(543.58625328,698.8929915)(543.42625826,698.83299154)
\curveto(543.37625349,698.81299158)(543.33125354,698.80799158)(543.29125826,698.81799154)
\curveto(543.25125362,698.81799157)(543.20625366,698.80799158)(543.15625826,698.78799154)
\lineto(543.00625826,698.78799154)
\curveto(542.98625388,698.7879916)(542.95625391,698.7929916)(542.91625826,698.80299154)
\curveto(542.87625399,698.80299159)(542.84125403,698.79799159)(542.81125826,698.78799154)
\curveto(542.76125411,698.77799161)(542.70625416,698.77799161)(542.64625826,698.78799154)
\lineto(542.49625826,698.78799154)
\lineto(542.34625826,698.78799154)
\curveto(542.29625457,698.77799161)(542.25125462,698.77799161)(542.21125826,698.78799154)
\lineto(542.04625826,698.78799154)
\curveto(541.99625487,698.79799159)(541.94125493,698.80299159)(541.88125826,698.80299154)
\curveto(541.82125505,698.80299159)(541.7662551,698.80799158)(541.71625826,698.81799154)
\curveto(541.64625522,698.82799156)(541.58125529,698.83799155)(541.52125826,698.84799154)
\lineto(541.34125826,698.87799154)
\curveto(541.23125564,698.90799148)(541.12625574,698.94299145)(541.02625826,698.98299154)
\curveto(540.92625594,699.02299137)(540.83125604,699.06799132)(540.74125826,699.11799154)
\lineto(540.65125826,699.17799154)
\curveto(540.62125625,699.20799118)(540.58625628,699.23799115)(540.54625826,699.26799154)
\curveto(540.52625634,699.2879911)(540.50125637,699.30799108)(540.47125826,699.32799154)
\lineto(540.39625826,699.40299154)
\curveto(540.25625661,699.5929908)(540.15125672,699.80299059)(540.08125826,700.03299154)
\curveto(540.06125681,700.07299032)(540.05125682,700.10799028)(540.05125826,700.13799154)
\curveto(540.06125681,700.17799021)(540.06125681,700.22299017)(540.05125826,700.27299154)
\curveto(540.04125683,700.2929901)(540.03625683,700.31799007)(540.03625826,700.34799154)
\curveto(540.03625683,700.37799001)(540.03125684,700.40298999)(540.02125826,700.42299154)
\lineto(540.02125826,700.57299154)
\curveto(540.01125686,700.61298978)(540.00625686,700.65798973)(540.00625826,700.70799154)
\curveto(540.01625685,700.75798963)(540.02125685,700.80798958)(540.02125826,700.85799154)
\lineto(540.02125826,701.42799154)
\lineto(540.02125826,703.66299154)
\lineto(540.02125826,704.45799154)
\lineto(540.02125826,704.66799154)
\curveto(540.03125684,704.73798565)(540.02625684,704.80298559)(540.00625826,704.86299154)
\curveto(539.9662569,705.00298539)(539.89625697,705.0929853)(539.79625826,705.13299154)
\curveto(539.68625718,705.18298521)(539.54625732,705.19798519)(539.37625826,705.17799154)
\curveto(539.20625766,705.15798523)(539.06125781,705.17298522)(538.94125826,705.22299154)
\curveto(538.86125801,705.25298514)(538.81125806,705.29798509)(538.79125826,705.35799154)
\curveto(538.7712581,705.41798497)(538.75125812,705.4929849)(538.73125826,705.58299154)
\lineto(538.73125826,705.89799154)
\curveto(538.73125814,706.07798431)(538.74125813,706.22298417)(538.76125826,706.33299154)
\curveto(538.78125809,706.44298395)(538.866258,706.51798387)(539.01625826,706.55799154)
\curveto(539.05625781,706.57798381)(539.09625777,706.58298381)(539.13625826,706.57299154)
\lineto(539.27125826,706.57299154)
\curveto(539.42125745,706.57298382)(539.56125731,706.57798381)(539.69125826,706.58799154)
\curveto(539.82125705,706.60798378)(539.91125696,706.66798372)(539.96125826,706.76799154)
\curveto(539.99125688,706.83798355)(540.00625686,706.91798347)(540.00625826,707.00799154)
\curveto(540.01625685,707.09798329)(540.02125685,707.1879832)(540.02125826,707.27799154)
\lineto(540.02125826,708.20799154)
\lineto(540.02125826,708.46299154)
\curveto(540.02125685,708.55298184)(540.03125684,708.62798176)(540.05125826,708.68799154)
\curveto(540.10125677,708.7879816)(540.17625669,708.85298154)(540.27625826,708.88299154)
\curveto(540.29625657,708.8929815)(540.32125655,708.8929815)(540.35125826,708.88299154)
\curveto(540.39125648,708.88298151)(540.42125645,708.8879815)(540.44125826,708.89799154)
}
}
{
\newrgbcolor{curcolor}{0 0 0}
\pscustom[linestyle=none,fillstyle=solid,fillcolor=curcolor]
{
\newpath
\moveto(546.76469576,709.43799154)
\curveto(546.83469281,709.35798103)(546.86969277,709.23798115)(546.86969576,709.07799154)
\lineto(546.86969576,708.61299154)
\lineto(546.86969576,708.20799154)
\curveto(546.86969277,708.06798232)(546.83469281,707.97298242)(546.76469576,707.92299154)
\curveto(546.70469294,707.87298252)(546.62469302,707.84298255)(546.52469576,707.83299154)
\curveto(546.43469321,707.82298257)(546.33469331,707.81798257)(546.22469576,707.81799154)
\lineto(545.38469576,707.81799154)
\curveto(545.27469437,707.81798257)(545.17469447,707.82298257)(545.08469576,707.83299154)
\curveto(545.00469464,707.84298255)(544.93469471,707.87298252)(544.87469576,707.92299154)
\curveto(544.83469481,707.95298244)(544.80469484,708.00798238)(544.78469576,708.08799154)
\curveto(544.77469487,708.17798221)(544.76469488,708.27298212)(544.75469576,708.37299154)
\lineto(544.75469576,708.70299154)
\curveto(544.76469488,708.81298158)(544.76969487,708.90798148)(544.76969576,708.98799154)
\lineto(544.76969576,709.19799154)
\curveto(544.77969486,709.26798112)(544.79969484,709.32798106)(544.82969576,709.37799154)
\curveto(544.84969479,709.41798097)(544.87469477,709.44798094)(544.90469576,709.46799154)
\lineto(545.02469576,709.52799154)
\curveto(545.0446946,709.52798086)(545.06969457,709.52798086)(545.09969576,709.52799154)
\curveto(545.12969451,709.53798085)(545.15469449,709.54298085)(545.17469576,709.54299154)
\lineto(546.26969576,709.54299154)
\curveto(546.36969327,709.54298085)(546.46469318,709.53798085)(546.55469576,709.52799154)
\curveto(546.644693,709.51798087)(546.71469293,709.4879809)(546.76469576,709.43799154)
\moveto(546.86969576,699.67299154)
\curveto(546.86969277,699.47299092)(546.86469278,699.30299109)(546.85469576,699.16299154)
\curveto(546.8446928,699.02299137)(546.75469289,698.92799146)(546.58469576,698.87799154)
\curveto(546.52469312,698.85799153)(546.45969318,698.84799154)(546.38969576,698.84799154)
\curveto(546.31969332,698.85799153)(546.2446934,698.86299153)(546.16469576,698.86299154)
\lineto(545.32469576,698.86299154)
\curveto(545.23469441,698.86299153)(545.1446945,698.86799152)(545.05469576,698.87799154)
\curveto(544.97469467,698.8879915)(544.91469473,698.91799147)(544.87469576,698.96799154)
\curveto(544.81469483,699.03799135)(544.77969486,699.12299127)(544.76969576,699.22299154)
\lineto(544.76969576,699.56799154)
\lineto(544.76969576,705.89799154)
\lineto(544.76969576,706.19799154)
\curveto(544.76969487,706.29798409)(544.78969485,706.37798401)(544.82969576,706.43799154)
\curveto(544.88969475,706.50798388)(544.97469467,706.55298384)(545.08469576,706.57299154)
\curveto(545.10469454,706.58298381)(545.12969451,706.58298381)(545.15969576,706.57299154)
\curveto(545.19969444,706.57298382)(545.22969441,706.57798381)(545.24969576,706.58799154)
\lineto(545.99969576,706.58799154)
\lineto(546.19469576,706.58799154)
\curveto(546.27469337,706.59798379)(546.3396933,706.59798379)(546.38969576,706.58799154)
\lineto(546.50969576,706.58799154)
\curveto(546.56969307,706.56798382)(546.62469302,706.55298384)(546.67469576,706.54299154)
\curveto(546.72469292,706.53298386)(546.76469288,706.50298389)(546.79469576,706.45299154)
\curveto(546.83469281,706.40298399)(546.85469279,706.33298406)(546.85469576,706.24299154)
\curveto(546.86469278,706.15298424)(546.86969277,706.05798433)(546.86969576,705.95799154)
\lineto(546.86969576,699.67299154)
}
}
{
\newrgbcolor{curcolor}{0 0 0}
\pscustom[linestyle=none,fillstyle=solid,fillcolor=curcolor]
{
\newpath
\moveto(552.10188326,706.79799154)
\curveto(552.9118781,706.81798357)(553.58687742,706.69798369)(554.12688326,706.43799154)
\curveto(554.67687633,706.17798421)(555.1118759,705.80798458)(555.43188326,705.32799154)
\curveto(555.59187542,705.0879853)(555.7118753,704.81298558)(555.79188326,704.50299154)
\curveto(555.8118752,704.45298594)(555.82687518,704.387986)(555.83688326,704.30799154)
\curveto(555.85687515,704.22798616)(555.85687515,704.15798623)(555.83688326,704.09799154)
\curveto(555.79687521,703.9879864)(555.72687528,703.92298647)(555.62688326,703.90299154)
\curveto(555.52687548,703.8929865)(555.4068756,703.8879865)(555.26688326,703.88799154)
\lineto(554.48688326,703.88799154)
\lineto(554.20188326,703.88799154)
\curveto(554.1118769,703.8879865)(554.03687697,703.90798648)(553.97688326,703.94799154)
\curveto(553.89687711,703.9879864)(553.84187717,704.04798634)(553.81188326,704.12799154)
\curveto(553.78187723,704.21798617)(553.74187727,704.30798608)(553.69188326,704.39799154)
\curveto(553.63187738,704.50798588)(553.56687744,704.60798578)(553.49688326,704.69799154)
\curveto(553.42687758,704.7879856)(553.34687766,704.86798552)(553.25688326,704.93799154)
\curveto(553.11687789,705.02798536)(552.96187805,705.09798529)(552.79188326,705.14799154)
\curveto(552.73187828,705.16798522)(552.67187834,705.17798521)(552.61188326,705.17799154)
\curveto(552.55187846,705.17798521)(552.49687851,705.1879852)(552.44688326,705.20799154)
\lineto(552.29688326,705.20799154)
\curveto(552.09687891,705.20798518)(551.93687907,705.1879852)(551.81688326,705.14799154)
\curveto(551.52687948,705.05798533)(551.29187972,704.91798547)(551.11188326,704.72799154)
\curveto(550.93188008,704.54798584)(550.78688022,704.32798606)(550.67688326,704.06799154)
\curveto(550.62688038,703.95798643)(550.58688042,703.83798655)(550.55688326,703.70799154)
\curveto(550.53688047,703.5879868)(550.5118805,703.45798693)(550.48188326,703.31799154)
\curveto(550.47188054,703.27798711)(550.46688054,703.23798715)(550.46688326,703.19799154)
\curveto(550.46688054,703.15798723)(550.46188055,703.11798727)(550.45188326,703.07799154)
\curveto(550.43188058,702.97798741)(550.42188059,702.83798755)(550.42188326,702.65799154)
\curveto(550.43188058,702.47798791)(550.44688056,702.33798805)(550.46688326,702.23799154)
\curveto(550.46688054,702.15798823)(550.47188054,702.10298829)(550.48188326,702.07299154)
\curveto(550.50188051,702.00298839)(550.5118805,701.93298846)(550.51188326,701.86299154)
\curveto(550.52188049,701.7929886)(550.53688047,701.72298867)(550.55688326,701.65299154)
\curveto(550.63688037,701.42298897)(550.73188028,701.21298918)(550.84188326,701.02299154)
\curveto(550.95188006,700.83298956)(551.09187992,700.67298972)(551.26188326,700.54299154)
\curveto(551.30187971,700.51298988)(551.36187965,700.47798991)(551.44188326,700.43799154)
\curveto(551.55187946,700.36799002)(551.66187935,700.32299007)(551.77188326,700.30299154)
\curveto(551.89187912,700.28299011)(552.03687897,700.26299013)(552.20688326,700.24299154)
\lineto(552.29688326,700.24299154)
\curveto(552.33687867,700.24299015)(552.36687864,700.24799014)(552.38688326,700.25799154)
\lineto(552.52188326,700.25799154)
\curveto(552.59187842,700.27799011)(552.65687835,700.2929901)(552.71688326,700.30299154)
\curveto(552.78687822,700.32299007)(552.85187816,700.34299005)(552.91188326,700.36299154)
\curveto(553.2118778,700.4929899)(553.44187757,700.68298971)(553.60188326,700.93299154)
\curveto(553.64187737,700.98298941)(553.67687733,701.03798935)(553.70688326,701.09799154)
\curveto(553.73687727,701.16798922)(553.76187725,701.22798916)(553.78188326,701.27799154)
\curveto(553.82187719,701.387989)(553.85687715,701.48298891)(553.88688326,701.56299154)
\curveto(553.91687709,701.65298874)(553.98687702,701.72298867)(554.09688326,701.77299154)
\curveto(554.18687682,701.81298858)(554.33187668,701.82798856)(554.53188326,701.81799154)
\lineto(555.02688326,701.81799154)
\lineto(555.23688326,701.81799154)
\curveto(555.31687569,701.82798856)(555.38187563,701.82298857)(555.43188326,701.80299154)
\lineto(555.55188326,701.80299154)
\lineto(555.67188326,701.77299154)
\curveto(555.7118753,701.77298862)(555.74187527,701.76298863)(555.76188326,701.74299154)
\curveto(555.8118752,701.70298869)(555.84187517,701.64298875)(555.85188326,701.56299154)
\curveto(555.87187514,701.4929889)(555.87187514,701.41798897)(555.85188326,701.33799154)
\curveto(555.76187525,701.00798938)(555.65187536,700.71298968)(555.52188326,700.45299154)
\curveto(555.1118759,699.68299071)(554.45687655,699.14799124)(553.55688326,698.84799154)
\curveto(553.45687755,698.81799157)(553.35187766,698.79799159)(553.24188326,698.78799154)
\curveto(553.13187788,698.76799162)(553.02187799,698.74299165)(552.91188326,698.71299154)
\curveto(552.85187816,698.70299169)(552.79187822,698.69799169)(552.73188326,698.69799154)
\curveto(552.67187834,698.69799169)(552.6118784,698.6929917)(552.55188326,698.68299154)
\lineto(552.38688326,698.68299154)
\curveto(552.33687867,698.66299173)(552.26187875,698.65799173)(552.16188326,698.66799154)
\curveto(552.06187895,698.66799172)(551.98687902,698.67299172)(551.93688326,698.68299154)
\curveto(551.85687915,698.70299169)(551.78187923,698.71299168)(551.71188326,698.71299154)
\curveto(551.65187936,698.70299169)(551.58687942,698.70799168)(551.51688326,698.72799154)
\lineto(551.36688326,698.75799154)
\curveto(551.31687969,698.75799163)(551.26687974,698.76299163)(551.21688326,698.77299154)
\curveto(551.1068799,698.80299159)(551.00188001,698.83299156)(550.90188326,698.86299154)
\curveto(550.80188021,698.8929915)(550.7068803,698.92799146)(550.61688326,698.96799154)
\curveto(550.14688086,699.16799122)(549.75188126,699.42299097)(549.43188326,699.73299154)
\curveto(549.1118819,700.05299034)(548.85188216,700.44798994)(548.65188326,700.91799154)
\curveto(548.60188241,701.00798938)(548.56188245,701.10298929)(548.53188326,701.20299154)
\lineto(548.44188326,701.53299154)
\curveto(548.43188258,701.57298882)(548.42688258,701.60798878)(548.42688326,701.63799154)
\curveto(548.42688258,701.67798871)(548.41688259,701.72298867)(548.39688326,701.77299154)
\curveto(548.37688263,701.84298855)(548.36688264,701.91298848)(548.36688326,701.98299154)
\curveto(548.36688264,702.06298833)(548.35688265,702.13798825)(548.33688326,702.20799154)
\lineto(548.33688326,702.46299154)
\curveto(548.31688269,702.51298788)(548.3068827,702.56798782)(548.30688326,702.62799154)
\curveto(548.3068827,702.69798769)(548.31688269,702.75798763)(548.33688326,702.80799154)
\curveto(548.34688266,702.85798753)(548.34688266,702.90298749)(548.33688326,702.94299154)
\curveto(548.32688268,702.98298741)(548.32688268,703.02298737)(548.33688326,703.06299154)
\curveto(548.35688265,703.13298726)(548.36188265,703.19798719)(548.35188326,703.25799154)
\curveto(548.35188266,703.31798707)(548.36188265,703.37798701)(548.38188326,703.43799154)
\curveto(548.43188258,703.61798677)(548.47188254,703.7879866)(548.50188326,703.94799154)
\curveto(548.53188248,704.11798627)(548.57688243,704.28298611)(548.63688326,704.44299154)
\curveto(548.85688215,704.95298544)(549.13188188,705.37798501)(549.46188326,705.71799154)
\curveto(549.80188121,706.05798433)(550.23188078,706.33298406)(550.75188326,706.54299154)
\curveto(550.89188012,706.60298379)(551.03687997,706.64298375)(551.18688326,706.66299154)
\curveto(551.33687967,706.6929837)(551.49187952,706.72798366)(551.65188326,706.76799154)
\curveto(551.73187928,706.77798361)(551.8068792,706.78298361)(551.87688326,706.78299154)
\curveto(551.94687906,706.78298361)(552.02187899,706.7879836)(552.10188326,706.79799154)
}
}
{
\newrgbcolor{curcolor}{0 0 0}
\pscustom[linestyle=none,fillstyle=solid,fillcolor=curcolor]
{
\newpath
\moveto(559.24516451,709.43799154)
\curveto(559.31516156,709.35798103)(559.35016152,709.23798115)(559.35016451,709.07799154)
\lineto(559.35016451,708.61299154)
\lineto(559.35016451,708.20799154)
\curveto(559.35016152,708.06798232)(559.31516156,707.97298242)(559.24516451,707.92299154)
\curveto(559.18516169,707.87298252)(559.10516177,707.84298255)(559.00516451,707.83299154)
\curveto(558.91516196,707.82298257)(558.81516206,707.81798257)(558.70516451,707.81799154)
\lineto(557.86516451,707.81799154)
\curveto(557.75516312,707.81798257)(557.65516322,707.82298257)(557.56516451,707.83299154)
\curveto(557.48516339,707.84298255)(557.41516346,707.87298252)(557.35516451,707.92299154)
\curveto(557.31516356,707.95298244)(557.28516359,708.00798238)(557.26516451,708.08799154)
\curveto(557.25516362,708.17798221)(557.24516363,708.27298212)(557.23516451,708.37299154)
\lineto(557.23516451,708.70299154)
\curveto(557.24516363,708.81298158)(557.25016362,708.90798148)(557.25016451,708.98799154)
\lineto(557.25016451,709.19799154)
\curveto(557.26016361,709.26798112)(557.28016359,709.32798106)(557.31016451,709.37799154)
\curveto(557.33016354,709.41798097)(557.35516352,709.44798094)(557.38516451,709.46799154)
\lineto(557.50516451,709.52799154)
\curveto(557.52516335,709.52798086)(557.55016332,709.52798086)(557.58016451,709.52799154)
\curveto(557.61016326,709.53798085)(557.63516324,709.54298085)(557.65516451,709.54299154)
\lineto(558.75016451,709.54299154)
\curveto(558.85016202,709.54298085)(558.94516193,709.53798085)(559.03516451,709.52799154)
\curveto(559.12516175,709.51798087)(559.19516168,709.4879809)(559.24516451,709.43799154)
\moveto(559.35016451,699.67299154)
\curveto(559.35016152,699.47299092)(559.34516153,699.30299109)(559.33516451,699.16299154)
\curveto(559.32516155,699.02299137)(559.23516164,698.92799146)(559.06516451,698.87799154)
\curveto(559.00516187,698.85799153)(558.94016193,698.84799154)(558.87016451,698.84799154)
\curveto(558.80016207,698.85799153)(558.72516215,698.86299153)(558.64516451,698.86299154)
\lineto(557.80516451,698.86299154)
\curveto(557.71516316,698.86299153)(557.62516325,698.86799152)(557.53516451,698.87799154)
\curveto(557.45516342,698.8879915)(557.39516348,698.91799147)(557.35516451,698.96799154)
\curveto(557.29516358,699.03799135)(557.26016361,699.12299127)(557.25016451,699.22299154)
\lineto(557.25016451,699.56799154)
\lineto(557.25016451,705.89799154)
\lineto(557.25016451,706.19799154)
\curveto(557.25016362,706.29798409)(557.2701636,706.37798401)(557.31016451,706.43799154)
\curveto(557.3701635,706.50798388)(557.45516342,706.55298384)(557.56516451,706.57299154)
\curveto(557.58516329,706.58298381)(557.61016326,706.58298381)(557.64016451,706.57299154)
\curveto(557.68016319,706.57298382)(557.71016316,706.57798381)(557.73016451,706.58799154)
\lineto(558.48016451,706.58799154)
\lineto(558.67516451,706.58799154)
\curveto(558.75516212,706.59798379)(558.82016205,706.59798379)(558.87016451,706.58799154)
\lineto(558.99016451,706.58799154)
\curveto(559.05016182,706.56798382)(559.10516177,706.55298384)(559.15516451,706.54299154)
\curveto(559.20516167,706.53298386)(559.24516163,706.50298389)(559.27516451,706.45299154)
\curveto(559.31516156,706.40298399)(559.33516154,706.33298406)(559.33516451,706.24299154)
\curveto(559.34516153,706.15298424)(559.35016152,706.05798433)(559.35016451,705.95799154)
\lineto(559.35016451,699.67299154)
}
}
{
\newrgbcolor{curcolor}{0 0 0}
\pscustom[linestyle=none,fillstyle=solid,fillcolor=curcolor]
{
\newpath
\moveto(568.90235201,702.80799154)
\curveto(568.91234333,702.74798764)(568.91734332,702.65798773)(568.91735201,702.53799154)
\curveto(568.91734332,702.41798797)(568.90734333,702.33298806)(568.88735201,702.28299154)
\lineto(568.88735201,702.08799154)
\curveto(568.85734338,701.97798841)(568.8373434,701.87298852)(568.82735201,701.77299154)
\curveto(568.82734341,701.67298872)(568.81234343,701.57298882)(568.78235201,701.47299154)
\curveto(568.76234348,701.38298901)(568.7423435,701.2879891)(568.72235201,701.18799154)
\curveto(568.70234354,701.09798929)(568.67234357,701.00798938)(568.63235201,700.91799154)
\curveto(568.56234368,700.74798964)(568.49234375,700.5879898)(568.42235201,700.43799154)
\curveto(568.35234389,700.29799009)(568.27234397,700.15799023)(568.18235201,700.01799154)
\curveto(568.12234412,699.92799046)(568.05734418,699.84299055)(567.98735201,699.76299154)
\curveto(567.92734431,699.6929907)(567.85734438,699.61799077)(567.77735201,699.53799154)
\lineto(567.67235201,699.43299154)
\curveto(567.62234462,699.38299101)(567.56734467,699.33799105)(567.50735201,699.29799154)
\lineto(567.35735201,699.17799154)
\curveto(567.27734496,699.11799127)(567.18734505,699.06299133)(567.08735201,699.01299154)
\curveto(566.99734524,698.97299142)(566.90234534,698.92799146)(566.80235201,698.87799154)
\curveto(566.70234554,698.82799156)(566.59734564,698.7929916)(566.48735201,698.77299154)
\curveto(566.38734585,698.75299164)(566.28234596,698.73299166)(566.17235201,698.71299154)
\curveto(566.11234613,698.6929917)(566.04734619,698.68299171)(565.97735201,698.68299154)
\curveto(565.91734632,698.68299171)(565.85234639,698.67299172)(565.78235201,698.65299154)
\lineto(565.64735201,698.65299154)
\curveto(565.56734667,698.63299176)(565.49234675,698.63299176)(565.42235201,698.65299154)
\lineto(565.27235201,698.65299154)
\curveto(565.21234703,698.67299172)(565.14734709,698.68299171)(565.07735201,698.68299154)
\curveto(565.01734722,698.67299172)(564.95734728,698.67799171)(564.89735201,698.69799154)
\curveto(564.7373475,698.74799164)(564.58234766,698.7929916)(564.43235201,698.83299154)
\curveto(564.29234795,698.87299152)(564.16234808,698.93299146)(564.04235201,699.01299154)
\curveto(563.97234827,699.05299134)(563.90734833,699.0929913)(563.84735201,699.13299154)
\curveto(563.78734845,699.18299121)(563.72234852,699.23299116)(563.65235201,699.28299154)
\lineto(563.47235201,699.41799154)
\curveto(563.39234885,699.47799091)(563.32234892,699.48299091)(563.26235201,699.43299154)
\curveto(563.21234903,699.40299099)(563.18734905,699.36299103)(563.18735201,699.31299154)
\curveto(563.18734905,699.27299112)(563.17734906,699.22299117)(563.15735201,699.16299154)
\curveto(563.1373491,699.06299133)(563.12734911,698.94799144)(563.12735201,698.81799154)
\curveto(563.1373491,698.6879917)(563.1423491,698.56799182)(563.14235201,698.45799154)
\lineto(563.14235201,696.92799154)
\curveto(563.1423491,696.79799359)(563.1373491,696.67299372)(563.12735201,696.55299154)
\curveto(563.12734911,696.42299397)(563.10234914,696.31799407)(563.05235201,696.23799154)
\curveto(563.02234922,696.19799419)(562.96734927,696.16799422)(562.88735201,696.14799154)
\curveto(562.80734943,696.12799426)(562.71734952,696.11799427)(562.61735201,696.11799154)
\curveto(562.51734972,696.10799428)(562.41734982,696.10799428)(562.31735201,696.11799154)
\lineto(562.06235201,696.11799154)
\lineto(561.65735201,696.11799154)
\lineto(561.55235201,696.11799154)
\curveto(561.51235073,696.11799427)(561.47735076,696.12299427)(561.44735201,696.13299154)
\lineto(561.32735201,696.13299154)
\curveto(561.15735108,696.18299421)(561.06735117,696.28299411)(561.05735201,696.43299154)
\curveto(561.04735119,696.57299382)(561.0423512,696.74299365)(561.04235201,696.94299154)
\lineto(561.04235201,705.74799154)
\curveto(561.0423512,705.85798453)(561.0373512,705.97298442)(561.02735201,706.09299154)
\curveto(561.02735121,706.22298417)(561.05235119,706.32298407)(561.10235201,706.39299154)
\curveto(561.1423511,706.46298393)(561.19735104,706.50798388)(561.26735201,706.52799154)
\curveto(561.31735092,706.54798384)(561.37735086,706.55798383)(561.44735201,706.55799154)
\lineto(561.67235201,706.55799154)
\lineto(562.39235201,706.55799154)
\lineto(562.67735201,706.55799154)
\curveto(562.76734947,706.55798383)(562.8423494,706.53298386)(562.90235201,706.48299154)
\curveto(562.97234927,706.43298396)(563.00734923,706.36798402)(563.00735201,706.28799154)
\curveto(563.01734922,706.21798417)(563.0423492,706.14298425)(563.08235201,706.06299154)
\curveto(563.09234915,706.03298436)(563.10234914,706.00798438)(563.11235201,705.98799154)
\curveto(563.13234911,705.97798441)(563.15234909,705.96298443)(563.17235201,705.94299154)
\curveto(563.28234896,705.93298446)(563.37234887,705.96298443)(563.44235201,706.03299154)
\curveto(563.51234873,706.10298429)(563.58234866,706.16298423)(563.65235201,706.21299154)
\curveto(563.78234846,706.30298409)(563.91734832,706.38298401)(564.05735201,706.45299154)
\curveto(564.19734804,706.53298386)(564.35234789,706.59798379)(564.52235201,706.64799154)
\curveto(564.60234764,706.67798371)(564.68734755,706.69798369)(564.77735201,706.70799154)
\curveto(564.87734736,706.71798367)(564.97234727,706.73298366)(565.06235201,706.75299154)
\curveto(565.10234714,706.76298363)(565.1423471,706.76298363)(565.18235201,706.75299154)
\curveto(565.23234701,706.74298365)(565.27234697,706.74798364)(565.30235201,706.76799154)
\curveto(565.87234637,706.7879836)(566.35234589,706.70798368)(566.74235201,706.52799154)
\curveto(567.1423451,706.35798403)(567.48234476,706.13298426)(567.76235201,705.85299154)
\curveto(567.81234443,705.80298459)(567.85734438,705.75298464)(567.89735201,705.70299154)
\curveto(567.9373443,705.66298473)(567.97734426,705.61798477)(568.01735201,705.56799154)
\curveto(568.08734415,705.47798491)(568.14734409,705.387985)(568.19735201,705.29799154)
\curveto(568.25734398,705.20798518)(568.31234393,705.11798527)(568.36235201,705.02799154)
\curveto(568.38234386,705.00798538)(568.39234385,704.98298541)(568.39235201,704.95299154)
\curveto(568.40234384,704.92298547)(568.41734382,704.8879855)(568.43735201,704.84799154)
\curveto(568.49734374,704.74798564)(568.55234369,704.62798576)(568.60235201,704.48799154)
\curveto(568.62234362,704.42798596)(568.6423436,704.36298603)(568.66235201,704.29299154)
\curveto(568.68234356,704.23298616)(568.70234354,704.16798622)(568.72235201,704.09799154)
\curveto(568.76234348,703.97798641)(568.78734345,703.85298654)(568.79735201,703.72299154)
\curveto(568.81734342,703.5929868)(568.8423434,703.45798693)(568.87235201,703.31799154)
\lineto(568.87235201,703.15299154)
\lineto(568.90235201,702.97299154)
\lineto(568.90235201,702.80799154)
\moveto(566.78735201,702.46299154)
\curveto(566.79734544,702.51298788)(566.80234544,702.57798781)(566.80235201,702.65799154)
\curveto(566.80234544,702.74798764)(566.79734544,702.81798757)(566.78735201,702.86799154)
\lineto(566.78735201,703.00299154)
\curveto(566.76734547,703.06298733)(566.75734548,703.12798726)(566.75735201,703.19799154)
\curveto(566.75734548,703.26798712)(566.74734549,703.33798705)(566.72735201,703.40799154)
\curveto(566.70734553,703.50798688)(566.68734555,703.60298679)(566.66735201,703.69299154)
\curveto(566.64734559,703.7929866)(566.61734562,703.88298651)(566.57735201,703.96299154)
\curveto(566.45734578,704.28298611)(566.30234594,704.53798585)(566.11235201,704.72799154)
\curveto(565.92234632,704.91798547)(565.65234659,705.05798533)(565.30235201,705.14799154)
\curveto(565.22234702,705.16798522)(565.13234711,705.17798521)(565.03235201,705.17799154)
\lineto(564.76235201,705.17799154)
\curveto(564.72234752,705.16798522)(564.68734755,705.16298523)(564.65735201,705.16299154)
\curveto(564.62734761,705.16298523)(564.59234765,705.15798523)(564.55235201,705.14799154)
\lineto(564.34235201,705.08799154)
\curveto(564.28234796,705.07798531)(564.22234802,705.05798533)(564.16235201,705.02799154)
\curveto(563.90234834,704.91798547)(563.69734854,704.74798564)(563.54735201,704.51799154)
\curveto(563.40734883,704.2879861)(563.29234895,704.03298636)(563.20235201,703.75299154)
\curveto(563.18234906,703.67298672)(563.16734907,703.5879868)(563.15735201,703.49799154)
\curveto(563.14734909,703.41798697)(563.13234911,703.33798705)(563.11235201,703.25799154)
\curveto(563.10234914,703.21798717)(563.09734914,703.15298724)(563.09735201,703.06299154)
\curveto(563.07734916,703.02298737)(563.07234917,702.97298742)(563.08235201,702.91299154)
\curveto(563.09234915,702.86298753)(563.09234915,702.81298758)(563.08235201,702.76299154)
\curveto(563.06234918,702.70298769)(563.06234918,702.64798774)(563.08235201,702.59799154)
\lineto(563.08235201,702.41799154)
\lineto(563.08235201,702.28299154)
\curveto(563.08234916,702.24298815)(563.09234915,702.20298819)(563.11235201,702.16299154)
\curveto(563.11234913,702.0929883)(563.11734912,702.03798835)(563.12735201,701.99799154)
\lineto(563.15735201,701.81799154)
\curveto(563.16734907,701.75798863)(563.18234906,701.69798869)(563.20235201,701.63799154)
\curveto(563.29234895,701.34798904)(563.39734884,701.10798928)(563.51735201,700.91799154)
\curveto(563.64734859,700.73798965)(563.82734841,700.57798981)(564.05735201,700.43799154)
\curveto(564.19734804,700.35799003)(564.36234788,700.2929901)(564.55235201,700.24299154)
\curveto(564.59234765,700.23299016)(564.62734761,700.22799016)(564.65735201,700.22799154)
\curveto(564.68734755,700.23799015)(564.72234752,700.23799015)(564.76235201,700.22799154)
\curveto(564.80234744,700.21799017)(564.86234738,700.20799018)(564.94235201,700.19799154)
\curveto(565.02234722,700.19799019)(565.08734715,700.20299019)(565.13735201,700.21299154)
\curveto(565.21734702,700.23299016)(565.29734694,700.24799014)(565.37735201,700.25799154)
\curveto(565.46734677,700.27799011)(565.55234669,700.30299009)(565.63235201,700.33299154)
\curveto(565.87234637,700.43298996)(566.06734617,700.57298982)(566.21735201,700.75299154)
\curveto(566.36734587,700.93298946)(566.49234575,701.14298925)(566.59235201,701.38299154)
\curveto(566.6423456,701.50298889)(566.67734556,701.62798876)(566.69735201,701.75799154)
\curveto(566.71734552,701.8879885)(566.7423455,702.02298837)(566.77235201,702.16299154)
\lineto(566.77235201,702.31299154)
\curveto(566.78234546,702.36298803)(566.78734545,702.41298798)(566.78735201,702.46299154)
}
}
{
\newrgbcolor{curcolor}{0 0 0}
\pscustom[linestyle=none,fillstyle=solid,fillcolor=curcolor]
{
\newpath
\moveto(577.23227388,699.44799154)
\curveto(577.25226603,699.33799105)(577.26226602,699.22799116)(577.26227388,699.11799154)
\curveto(577.27226601,699.00799138)(577.22226606,698.93299146)(577.11227388,698.89299154)
\curveto(577.05226623,698.86299153)(576.9822663,698.84799154)(576.90227388,698.84799154)
\lineto(576.66227388,698.84799154)
\lineto(575.85227388,698.84799154)
\lineto(575.58227388,698.84799154)
\curveto(575.50226778,698.85799153)(575.43726785,698.88299151)(575.38727388,698.92299154)
\curveto(575.31726797,698.96299143)(575.26226802,699.01799137)(575.22227388,699.08799154)
\curveto(575.19226809,699.16799122)(575.14726814,699.23299116)(575.08727388,699.28299154)
\curveto(575.06726822,699.30299109)(575.04226824,699.31799107)(575.01227388,699.32799154)
\curveto(574.9822683,699.34799104)(574.94226834,699.35299104)(574.89227388,699.34299154)
\curveto(574.84226844,699.32299107)(574.79226849,699.29799109)(574.74227388,699.26799154)
\curveto(574.70226858,699.23799115)(574.65726863,699.21299118)(574.60727388,699.19299154)
\curveto(574.55726873,699.15299124)(574.50226878,699.11799127)(574.44227388,699.08799154)
\lineto(574.26227388,698.99799154)
\curveto(574.13226915,698.93799145)(573.99726929,698.8879915)(573.85727388,698.84799154)
\curveto(573.71726957,698.81799157)(573.57226971,698.78299161)(573.42227388,698.74299154)
\curveto(573.35226993,698.72299167)(573.28227,698.71299168)(573.21227388,698.71299154)
\curveto(573.15227013,698.70299169)(573.0872702,698.6929917)(573.01727388,698.68299154)
\lineto(572.92727388,698.68299154)
\curveto(572.89727039,698.67299172)(572.86727042,698.66799172)(572.83727388,698.66799154)
\lineto(572.67227388,698.66799154)
\curveto(572.57227071,698.64799174)(572.47227081,698.64799174)(572.37227388,698.66799154)
\lineto(572.23727388,698.66799154)
\curveto(572.16727112,698.6879917)(572.09727119,698.69799169)(572.02727388,698.69799154)
\curveto(571.96727132,698.6879917)(571.90727138,698.6929917)(571.84727388,698.71299154)
\curveto(571.74727154,698.73299166)(571.65227163,698.75299164)(571.56227388,698.77299154)
\curveto(571.47227181,698.78299161)(571.3872719,698.80799158)(571.30727388,698.84799154)
\curveto(571.01727227,698.95799143)(570.76727252,699.09799129)(570.55727388,699.26799154)
\curveto(570.35727293,699.44799094)(570.19727309,699.68299071)(570.07727388,699.97299154)
\curveto(570.04727324,700.04299035)(570.01727327,700.11799027)(569.98727388,700.19799154)
\curveto(569.96727332,700.27799011)(569.94727334,700.36299003)(569.92727388,700.45299154)
\curveto(569.90727338,700.50298989)(569.89727339,700.55298984)(569.89727388,700.60299154)
\curveto(569.90727338,700.65298974)(569.90727338,700.70298969)(569.89727388,700.75299154)
\curveto(569.8872734,700.78298961)(569.87727341,700.84298955)(569.86727388,700.93299154)
\curveto(569.86727342,701.03298936)(569.87227341,701.10298929)(569.88227388,701.14299154)
\curveto(569.90227338,701.24298915)(569.91227337,701.32798906)(569.91227388,701.39799154)
\lineto(570.00227388,701.72799154)
\curveto(570.03227325,701.84798854)(570.07227321,701.95298844)(570.12227388,702.04299154)
\curveto(570.29227299,702.33298806)(570.4872728,702.55298784)(570.70727388,702.70299154)
\curveto(570.92727236,702.85298754)(571.20727208,702.98298741)(571.54727388,703.09299154)
\curveto(571.67727161,703.14298725)(571.81227147,703.17798721)(571.95227388,703.19799154)
\curveto(572.09227119,703.21798717)(572.23227105,703.24298715)(572.37227388,703.27299154)
\curveto(572.45227083,703.2929871)(572.53727075,703.30298709)(572.62727388,703.30299154)
\curveto(572.71727057,703.31298708)(572.80727048,703.32798706)(572.89727388,703.34799154)
\curveto(572.96727032,703.36798702)(573.03727025,703.37298702)(573.10727388,703.36299154)
\curveto(573.17727011,703.36298703)(573.25227003,703.37298702)(573.33227388,703.39299154)
\curveto(573.40226988,703.41298698)(573.47226981,703.42298697)(573.54227388,703.42299154)
\curveto(573.61226967,703.42298697)(573.6872696,703.43298696)(573.76727388,703.45299154)
\curveto(573.97726931,703.50298689)(574.16726912,703.54298685)(574.33727388,703.57299154)
\curveto(574.51726877,703.61298678)(574.67726861,703.70298669)(574.81727388,703.84299154)
\curveto(574.90726838,703.93298646)(574.96726832,704.03298636)(574.99727388,704.14299154)
\curveto(575.00726828,704.17298622)(575.00726828,704.19798619)(574.99727388,704.21799154)
\curveto(574.99726829,704.23798615)(575.00226828,704.25798613)(575.01227388,704.27799154)
\curveto(575.02226826,704.29798609)(575.02726826,704.32798606)(575.02727388,704.36799154)
\lineto(575.02727388,704.45799154)
\lineto(574.99727388,704.57799154)
\curveto(574.99726829,704.61798577)(574.99226829,704.65298574)(574.98227388,704.68299154)
\curveto(574.8822684,704.98298541)(574.67226861,705.1879852)(574.35227388,705.29799154)
\curveto(574.26226902,705.32798506)(574.15226913,705.34798504)(574.02227388,705.35799154)
\curveto(573.90226938,705.37798501)(573.77726951,705.38298501)(573.64727388,705.37299154)
\curveto(573.51726977,705.37298502)(573.39226989,705.36298503)(573.27227388,705.34299154)
\curveto(573.15227013,705.32298507)(573.04727024,705.29798509)(572.95727388,705.26799154)
\curveto(572.89727039,705.24798514)(572.83727045,705.21798517)(572.77727388,705.17799154)
\curveto(572.72727056,705.14798524)(572.67727061,705.11298528)(572.62727388,705.07299154)
\curveto(572.57727071,705.03298536)(572.52227076,704.97798541)(572.46227388,704.90799154)
\curveto(572.41227087,704.83798555)(572.37727091,704.77298562)(572.35727388,704.71299154)
\curveto(572.30727098,704.61298578)(572.26227102,704.51798587)(572.22227388,704.42799154)
\curveto(572.19227109,704.33798605)(572.12227116,704.27798611)(572.01227388,704.24799154)
\curveto(571.93227135,704.22798616)(571.84727144,704.21798617)(571.75727388,704.21799154)
\lineto(571.48727388,704.21799154)
\lineto(570.91727388,704.21799154)
\curveto(570.86727242,704.21798617)(570.81727247,704.21298618)(570.76727388,704.20299154)
\curveto(570.71727257,704.20298619)(570.67227261,704.20798618)(570.63227388,704.21799154)
\lineto(570.49727388,704.21799154)
\curveto(570.47727281,704.22798616)(570.45227283,704.23298616)(570.42227388,704.23299154)
\curveto(570.39227289,704.23298616)(570.36727292,704.24298615)(570.34727388,704.26299154)
\curveto(570.26727302,704.28298611)(570.21227307,704.34798604)(570.18227388,704.45799154)
\curveto(570.17227311,704.50798588)(570.17227311,704.55798583)(570.18227388,704.60799154)
\curveto(570.19227309,704.65798573)(570.20227308,704.70298569)(570.21227388,704.74299154)
\curveto(570.24227304,704.85298554)(570.27227301,704.95298544)(570.30227388,705.04299154)
\curveto(570.34227294,705.14298525)(570.3872729,705.23298516)(570.43727388,705.31299154)
\lineto(570.52727388,705.46299154)
\lineto(570.61727388,705.61299154)
\curveto(570.69727259,705.72298467)(570.79727249,705.82798456)(570.91727388,705.92799154)
\curveto(570.93727235,705.93798445)(570.96727232,705.96298443)(571.00727388,706.00299154)
\curveto(571.05727223,706.04298435)(571.10227218,706.07798431)(571.14227388,706.10799154)
\curveto(571.1822721,706.13798425)(571.22727206,706.16798422)(571.27727388,706.19799154)
\curveto(571.44727184,706.30798408)(571.62727166,706.392984)(571.81727388,706.45299154)
\curveto(572.00727128,706.52298387)(572.20227108,706.5879838)(572.40227388,706.64799154)
\curveto(572.52227076,706.67798371)(572.64727064,706.69798369)(572.77727388,706.70799154)
\curveto(572.90727038,706.71798367)(573.03727025,706.73798365)(573.16727388,706.76799154)
\curveto(573.20727008,706.77798361)(573.26727002,706.77798361)(573.34727388,706.76799154)
\curveto(573.43726985,706.75798363)(573.49226979,706.76298363)(573.51227388,706.78299154)
\curveto(573.92226936,706.7929836)(574.31226897,706.77798361)(574.68227388,706.73799154)
\curveto(575.06226822,706.69798369)(575.40226788,706.62298377)(575.70227388,706.51299154)
\curveto(576.01226727,706.40298399)(576.27726701,706.25298414)(576.49727388,706.06299154)
\curveto(576.71726657,705.88298451)(576.8872664,705.64798474)(577.00727388,705.35799154)
\curveto(577.07726621,705.1879852)(577.11726617,704.9929854)(577.12727388,704.77299154)
\curveto(577.13726615,704.55298584)(577.14226614,704.32798606)(577.14227388,704.09799154)
\lineto(577.14227388,700.75299154)
\lineto(577.14227388,700.16799154)
\curveto(577.14226614,699.97799041)(577.16226612,699.80299059)(577.20227388,699.64299154)
\curveto(577.21226607,699.61299078)(577.21726607,699.57799081)(577.21727388,699.53799154)
\curveto(577.21726607,699.50799088)(577.22226606,699.47799091)(577.23227388,699.44799154)
\moveto(575.02727388,701.75799154)
\curveto(575.03726825,701.80798858)(575.04226824,701.86298853)(575.04227388,701.92299154)
\curveto(575.04226824,701.9929884)(575.03726825,702.05298834)(575.02727388,702.10299154)
\curveto(575.00726828,702.16298823)(574.99726829,702.21798817)(574.99727388,702.26799154)
\curveto(574.99726829,702.31798807)(574.97726831,702.35798803)(574.93727388,702.38799154)
\curveto(574.8872684,702.42798796)(574.81226847,702.44798794)(574.71227388,702.44799154)
\curveto(574.67226861,702.43798795)(574.63726865,702.42798796)(574.60727388,702.41799154)
\curveto(574.57726871,702.41798797)(574.54226874,702.41298798)(574.50227388,702.40299154)
\curveto(574.43226885,702.38298801)(574.35726893,702.36798802)(574.27727388,702.35799154)
\curveto(574.19726909,702.34798804)(574.11726917,702.33298806)(574.03727388,702.31299154)
\curveto(574.00726928,702.30298809)(573.96226932,702.29798809)(573.90227388,702.29799154)
\curveto(573.77226951,702.26798812)(573.64226964,702.24798814)(573.51227388,702.23799154)
\curveto(573.3822699,702.22798816)(573.25727003,702.20298819)(573.13727388,702.16299154)
\curveto(573.05727023,702.14298825)(572.9822703,702.12298827)(572.91227388,702.10299154)
\curveto(572.84227044,702.0929883)(572.77227051,702.07298832)(572.70227388,702.04299154)
\curveto(572.49227079,701.95298844)(572.31227097,701.81798857)(572.16227388,701.63799154)
\curveto(572.02227126,701.45798893)(571.97227131,701.20798918)(572.01227388,700.88799154)
\curveto(572.03227125,700.71798967)(572.0872712,700.57798981)(572.17727388,700.46799154)
\curveto(572.24727104,700.35799003)(572.35227093,700.26799012)(572.49227388,700.19799154)
\curveto(572.63227065,700.13799025)(572.7822705,700.0929903)(572.94227388,700.06299154)
\curveto(573.11227017,700.03299036)(573.28727,700.02299037)(573.46727388,700.03299154)
\curveto(573.65726963,700.05299034)(573.83226945,700.0879903)(573.99227388,700.13799154)
\curveto(574.25226903,700.21799017)(574.45726883,700.34299005)(574.60727388,700.51299154)
\curveto(574.75726853,700.6929897)(574.87226841,700.91298948)(574.95227388,701.17299154)
\curveto(574.97226831,701.24298915)(574.9822683,701.31298908)(574.98227388,701.38299154)
\curveto(574.99226829,701.46298893)(575.00726828,701.54298885)(575.02727388,701.62299154)
\lineto(575.02727388,701.75799154)
}
}
{
\newrgbcolor{curcolor}{0 0 0}
\pscustom[linestyle=none,fillstyle=solid,fillcolor=curcolor]
{
\newpath
\moveto(582.36555513,706.79799154)
\curveto(583.17554997,706.81798357)(583.8505493,706.69798369)(584.39055513,706.43799154)
\curveto(584.94054821,706.17798421)(585.37554777,705.80798458)(585.69555513,705.32799154)
\curveto(585.85554729,705.0879853)(585.97554717,704.81298558)(586.05555513,704.50299154)
\curveto(586.07554707,704.45298594)(586.09054706,704.387986)(586.10055513,704.30799154)
\curveto(586.12054703,704.22798616)(586.12054703,704.15798623)(586.10055513,704.09799154)
\curveto(586.06054709,703.9879864)(585.99054716,703.92298647)(585.89055513,703.90299154)
\curveto(585.79054736,703.8929865)(585.67054748,703.8879865)(585.53055513,703.88799154)
\lineto(584.75055513,703.88799154)
\lineto(584.46555513,703.88799154)
\curveto(584.37554877,703.8879865)(584.30054885,703.90798648)(584.24055513,703.94799154)
\curveto(584.16054899,703.9879864)(584.10554904,704.04798634)(584.07555513,704.12799154)
\curveto(584.0455491,704.21798617)(584.00554914,704.30798608)(583.95555513,704.39799154)
\curveto(583.89554925,704.50798588)(583.83054932,704.60798578)(583.76055513,704.69799154)
\curveto(583.69054946,704.7879856)(583.61054954,704.86798552)(583.52055513,704.93799154)
\curveto(583.38054977,705.02798536)(583.22554992,705.09798529)(583.05555513,705.14799154)
\curveto(582.99555015,705.16798522)(582.93555021,705.17798521)(582.87555513,705.17799154)
\curveto(582.81555033,705.17798521)(582.76055039,705.1879852)(582.71055513,705.20799154)
\lineto(582.56055513,705.20799154)
\curveto(582.36055079,705.20798518)(582.20055095,705.1879852)(582.08055513,705.14799154)
\curveto(581.79055136,705.05798533)(581.55555159,704.91798547)(581.37555513,704.72799154)
\curveto(581.19555195,704.54798584)(581.0505521,704.32798606)(580.94055513,704.06799154)
\curveto(580.89055226,703.95798643)(580.8505523,703.83798655)(580.82055513,703.70799154)
\curveto(580.80055235,703.5879868)(580.77555237,703.45798693)(580.74555513,703.31799154)
\curveto(580.73555241,703.27798711)(580.73055242,703.23798715)(580.73055513,703.19799154)
\curveto(580.73055242,703.15798723)(580.72555242,703.11798727)(580.71555513,703.07799154)
\curveto(580.69555245,702.97798741)(580.68555246,702.83798755)(580.68555513,702.65799154)
\curveto(580.69555245,702.47798791)(580.71055244,702.33798805)(580.73055513,702.23799154)
\curveto(580.73055242,702.15798823)(580.73555241,702.10298829)(580.74555513,702.07299154)
\curveto(580.76555238,702.00298839)(580.77555237,701.93298846)(580.77555513,701.86299154)
\curveto(580.78555236,701.7929886)(580.80055235,701.72298867)(580.82055513,701.65299154)
\curveto(580.90055225,701.42298897)(580.99555215,701.21298918)(581.10555513,701.02299154)
\curveto(581.21555193,700.83298956)(581.35555179,700.67298972)(581.52555513,700.54299154)
\curveto(581.56555158,700.51298988)(581.62555152,700.47798991)(581.70555513,700.43799154)
\curveto(581.81555133,700.36799002)(581.92555122,700.32299007)(582.03555513,700.30299154)
\curveto(582.15555099,700.28299011)(582.30055085,700.26299013)(582.47055513,700.24299154)
\lineto(582.56055513,700.24299154)
\curveto(582.60055055,700.24299015)(582.63055052,700.24799014)(582.65055513,700.25799154)
\lineto(582.78555513,700.25799154)
\curveto(582.85555029,700.27799011)(582.92055023,700.2929901)(582.98055513,700.30299154)
\curveto(583.0505501,700.32299007)(583.11555003,700.34299005)(583.17555513,700.36299154)
\curveto(583.47554967,700.4929899)(583.70554944,700.68298971)(583.86555513,700.93299154)
\curveto(583.90554924,700.98298941)(583.94054921,701.03798935)(583.97055513,701.09799154)
\curveto(584.00054915,701.16798922)(584.02554912,701.22798916)(584.04555513,701.27799154)
\curveto(584.08554906,701.387989)(584.12054903,701.48298891)(584.15055513,701.56299154)
\curveto(584.18054897,701.65298874)(584.2505489,701.72298867)(584.36055513,701.77299154)
\curveto(584.4505487,701.81298858)(584.59554855,701.82798856)(584.79555513,701.81799154)
\lineto(585.29055513,701.81799154)
\lineto(585.50055513,701.81799154)
\curveto(585.58054757,701.82798856)(585.6455475,701.82298857)(585.69555513,701.80299154)
\lineto(585.81555513,701.80299154)
\lineto(585.93555513,701.77299154)
\curveto(585.97554717,701.77298862)(586.00554714,701.76298863)(586.02555513,701.74299154)
\curveto(586.07554707,701.70298869)(586.10554704,701.64298875)(586.11555513,701.56299154)
\curveto(586.13554701,701.4929889)(586.13554701,701.41798897)(586.11555513,701.33799154)
\curveto(586.02554712,701.00798938)(585.91554723,700.71298968)(585.78555513,700.45299154)
\curveto(585.37554777,699.68299071)(584.72054843,699.14799124)(583.82055513,698.84799154)
\curveto(583.72054943,698.81799157)(583.61554953,698.79799159)(583.50555513,698.78799154)
\curveto(583.39554975,698.76799162)(583.28554986,698.74299165)(583.17555513,698.71299154)
\curveto(583.11555003,698.70299169)(583.05555009,698.69799169)(582.99555513,698.69799154)
\curveto(582.93555021,698.69799169)(582.87555027,698.6929917)(582.81555513,698.68299154)
\lineto(582.65055513,698.68299154)
\curveto(582.60055055,698.66299173)(582.52555062,698.65799173)(582.42555513,698.66799154)
\curveto(582.32555082,698.66799172)(582.2505509,698.67299172)(582.20055513,698.68299154)
\curveto(582.12055103,698.70299169)(582.0455511,698.71299168)(581.97555513,698.71299154)
\curveto(581.91555123,698.70299169)(581.8505513,698.70799168)(581.78055513,698.72799154)
\lineto(581.63055513,698.75799154)
\curveto(581.58055157,698.75799163)(581.53055162,698.76299163)(581.48055513,698.77299154)
\curveto(581.37055178,698.80299159)(581.26555188,698.83299156)(581.16555513,698.86299154)
\curveto(581.06555208,698.8929915)(580.97055218,698.92799146)(580.88055513,698.96799154)
\curveto(580.41055274,699.16799122)(580.01555313,699.42299097)(579.69555513,699.73299154)
\curveto(579.37555377,700.05299034)(579.11555403,700.44798994)(578.91555513,700.91799154)
\curveto(578.86555428,701.00798938)(578.82555432,701.10298929)(578.79555513,701.20299154)
\lineto(578.70555513,701.53299154)
\curveto(578.69555445,701.57298882)(578.69055446,701.60798878)(578.69055513,701.63799154)
\curveto(578.69055446,701.67798871)(578.68055447,701.72298867)(578.66055513,701.77299154)
\curveto(578.64055451,701.84298855)(578.63055452,701.91298848)(578.63055513,701.98299154)
\curveto(578.63055452,702.06298833)(578.62055453,702.13798825)(578.60055513,702.20799154)
\lineto(578.60055513,702.46299154)
\curveto(578.58055457,702.51298788)(578.57055458,702.56798782)(578.57055513,702.62799154)
\curveto(578.57055458,702.69798769)(578.58055457,702.75798763)(578.60055513,702.80799154)
\curveto(578.61055454,702.85798753)(578.61055454,702.90298749)(578.60055513,702.94299154)
\curveto(578.59055456,702.98298741)(578.59055456,703.02298737)(578.60055513,703.06299154)
\curveto(578.62055453,703.13298726)(578.62555452,703.19798719)(578.61555513,703.25799154)
\curveto(578.61555453,703.31798707)(578.62555452,703.37798701)(578.64555513,703.43799154)
\curveto(578.69555445,703.61798677)(578.73555441,703.7879866)(578.76555513,703.94799154)
\curveto(578.79555435,704.11798627)(578.84055431,704.28298611)(578.90055513,704.44299154)
\curveto(579.12055403,704.95298544)(579.39555375,705.37798501)(579.72555513,705.71799154)
\curveto(580.06555308,706.05798433)(580.49555265,706.33298406)(581.01555513,706.54299154)
\curveto(581.15555199,706.60298379)(581.30055185,706.64298375)(581.45055513,706.66299154)
\curveto(581.60055155,706.6929837)(581.75555139,706.72798366)(581.91555513,706.76799154)
\curveto(581.99555115,706.77798361)(582.07055108,706.78298361)(582.14055513,706.78299154)
\curveto(582.21055094,706.78298361)(582.28555086,706.7879836)(582.36555513,706.79799154)
}
}
{
\newrgbcolor{curcolor}{0 0 0}
\pscustom[linestyle=none,fillstyle=solid,fillcolor=curcolor]
{
\newpath
\moveto(589.50883638,709.43799154)
\curveto(589.57883343,709.35798103)(589.6138334,709.23798115)(589.61383638,709.07799154)
\lineto(589.61383638,708.61299154)
\lineto(589.61383638,708.20799154)
\curveto(589.6138334,708.06798232)(589.57883343,707.97298242)(589.50883638,707.92299154)
\curveto(589.44883356,707.87298252)(589.36883364,707.84298255)(589.26883638,707.83299154)
\curveto(589.17883383,707.82298257)(589.07883393,707.81798257)(588.96883638,707.81799154)
\lineto(588.12883638,707.81799154)
\curveto(588.01883499,707.81798257)(587.91883509,707.82298257)(587.82883638,707.83299154)
\curveto(587.74883526,707.84298255)(587.67883533,707.87298252)(587.61883638,707.92299154)
\curveto(587.57883543,707.95298244)(587.54883546,708.00798238)(587.52883638,708.08799154)
\curveto(587.51883549,708.17798221)(587.5088355,708.27298212)(587.49883638,708.37299154)
\lineto(587.49883638,708.70299154)
\curveto(587.5088355,708.81298158)(587.5138355,708.90798148)(587.51383638,708.98799154)
\lineto(587.51383638,709.19799154)
\curveto(587.52383549,709.26798112)(587.54383547,709.32798106)(587.57383638,709.37799154)
\curveto(587.59383542,709.41798097)(587.61883539,709.44798094)(587.64883638,709.46799154)
\lineto(587.76883638,709.52799154)
\curveto(587.78883522,709.52798086)(587.8138352,709.52798086)(587.84383638,709.52799154)
\curveto(587.87383514,709.53798085)(587.89883511,709.54298085)(587.91883638,709.54299154)
\lineto(589.01383638,709.54299154)
\curveto(589.1138339,709.54298085)(589.2088338,709.53798085)(589.29883638,709.52799154)
\curveto(589.38883362,709.51798087)(589.45883355,709.4879809)(589.50883638,709.43799154)
\moveto(589.61383638,699.67299154)
\curveto(589.6138334,699.47299092)(589.6088334,699.30299109)(589.59883638,699.16299154)
\curveto(589.58883342,699.02299137)(589.49883351,698.92799146)(589.32883638,698.87799154)
\curveto(589.26883374,698.85799153)(589.20383381,698.84799154)(589.13383638,698.84799154)
\curveto(589.06383395,698.85799153)(588.98883402,698.86299153)(588.90883638,698.86299154)
\lineto(588.06883638,698.86299154)
\curveto(587.97883503,698.86299153)(587.88883512,698.86799152)(587.79883638,698.87799154)
\curveto(587.71883529,698.8879915)(587.65883535,698.91799147)(587.61883638,698.96799154)
\curveto(587.55883545,699.03799135)(587.52383549,699.12299127)(587.51383638,699.22299154)
\lineto(587.51383638,699.56799154)
\lineto(587.51383638,705.89799154)
\lineto(587.51383638,706.19799154)
\curveto(587.5138355,706.29798409)(587.53383548,706.37798401)(587.57383638,706.43799154)
\curveto(587.63383538,706.50798388)(587.71883529,706.55298384)(587.82883638,706.57299154)
\curveto(587.84883516,706.58298381)(587.87383514,706.58298381)(587.90383638,706.57299154)
\curveto(587.94383507,706.57298382)(587.97383504,706.57798381)(587.99383638,706.58799154)
\lineto(588.74383638,706.58799154)
\lineto(588.93883638,706.58799154)
\curveto(589.01883399,706.59798379)(589.08383393,706.59798379)(589.13383638,706.58799154)
\lineto(589.25383638,706.58799154)
\curveto(589.3138337,706.56798382)(589.36883364,706.55298384)(589.41883638,706.54299154)
\curveto(589.46883354,706.53298386)(589.5088335,706.50298389)(589.53883638,706.45299154)
\curveto(589.57883343,706.40298399)(589.59883341,706.33298406)(589.59883638,706.24299154)
\curveto(589.6088334,706.15298424)(589.6138334,706.05798433)(589.61383638,705.95799154)
\lineto(589.61383638,699.67299154)
}
}
{
\newrgbcolor{curcolor}{0 0 0}
\pscustom[linestyle=none,fillstyle=solid,fillcolor=curcolor]
{
\newpath
\moveto(599.04602388,703.03299154)
\curveto(599.02601535,703.08298731)(599.02101536,703.13798725)(599.03102388,703.19799154)
\curveto(599.04101534,703.25798713)(599.03601534,703.31298708)(599.01602388,703.36299154)
\curveto(599.00601537,703.40298699)(599.00101538,703.44298695)(599.00102388,703.48299154)
\curveto(599.00101538,703.52298687)(598.99601538,703.56298683)(598.98602388,703.60299154)
\lineto(598.92602388,703.87299154)
\curveto(598.90601547,703.96298643)(598.8810155,704.04798634)(598.85102388,704.12799154)
\curveto(598.80101558,704.26798612)(598.75601562,704.39798599)(598.71602388,704.51799154)
\curveto(598.6760157,704.64798574)(598.62101576,704.76798562)(598.55102388,704.87799154)
\curveto(598.4810159,704.9879854)(598.41101597,705.0929853)(598.34102388,705.19299154)
\curveto(598.2810161,705.2929851)(598.21101617,705.392985)(598.13102388,705.49299154)
\curveto(598.05101633,705.60298479)(597.95101643,705.70298469)(597.83102388,705.79299154)
\curveto(597.72101666,705.8929845)(597.61101677,705.98298441)(597.50102388,706.06299154)
\curveto(597.17101721,706.2929841)(596.79101759,706.47298392)(596.36102388,706.60299154)
\curveto(595.94101844,706.73298366)(595.44101894,706.7929836)(594.86102388,706.78299154)
\curveto(594.79101959,706.77298362)(594.72101966,706.76798362)(594.65102388,706.76799154)
\curveto(594.5810198,706.76798362)(594.50601987,706.76298363)(594.42602388,706.75299154)
\curveto(594.2760201,706.71298368)(594.13102025,706.68298371)(593.99102388,706.66299154)
\curveto(593.85102053,706.64298375)(593.71602066,706.60798378)(593.58602388,706.55799154)
\curveto(593.4760209,706.50798388)(593.36602101,706.46298393)(593.25602388,706.42299154)
\curveto(593.14602123,706.38298401)(593.04102134,706.33798405)(592.94102388,706.28799154)
\curveto(592.5810218,706.05798433)(592.2760221,705.80298459)(592.02602388,705.52299154)
\curveto(591.7760226,705.25298514)(591.56102282,704.91298548)(591.38102388,704.50299154)
\curveto(591.33102305,704.38298601)(591.29102309,704.25798613)(591.26102388,704.12799154)
\curveto(591.23102315,704.00798638)(591.19602318,703.88298651)(591.15602388,703.75299154)
\curveto(591.13602324,703.70298669)(591.12602325,703.65298674)(591.12602388,703.60299154)
\curveto(591.12602325,703.56298683)(591.12102326,703.51798687)(591.11102388,703.46799154)
\curveto(591.09102329,703.41798697)(591.0810233,703.36298703)(591.08102388,703.30299154)
\curveto(591.09102329,703.25298714)(591.09102329,703.20298719)(591.08102388,703.15299154)
\lineto(591.08102388,703.04799154)
\curveto(591.06102332,702.9879874)(591.04602333,702.90298749)(591.03602388,702.79299154)
\curveto(591.03602334,702.68298771)(591.04602333,702.59798779)(591.06602388,702.53799154)
\lineto(591.06602388,702.40299154)
\curveto(591.06602331,702.36298803)(591.07102331,702.31798807)(591.08102388,702.26799154)
\curveto(591.10102328,702.1879882)(591.11102327,702.10298829)(591.11102388,702.01299154)
\curveto(591.11102327,701.93298846)(591.12102326,701.85298854)(591.14102388,701.77299154)
\curveto(591.16102322,701.72298867)(591.17102321,701.67798871)(591.17102388,701.63799154)
\curveto(591.17102321,701.59798879)(591.1810232,701.55298884)(591.20102388,701.50299154)
\curveto(591.23102315,701.392989)(591.25602312,701.2879891)(591.27602388,701.18799154)
\curveto(591.30602307,701.0879893)(591.34602303,700.9929894)(591.39602388,700.90299154)
\curveto(591.56602281,700.51298988)(591.7760226,700.17799021)(592.02602388,699.89799154)
\curveto(592.2760221,699.61799077)(592.5760218,699.37299102)(592.92602388,699.16299154)
\curveto(593.03602134,699.10299129)(593.14102124,699.05299134)(593.24102388,699.01299154)
\curveto(593.35102103,698.97299142)(593.46602091,698.93299146)(593.58602388,698.89299154)
\curveto(593.6760207,698.85299154)(593.77102061,698.82299157)(593.87102388,698.80299154)
\curveto(593.97102041,698.78299161)(594.07102031,698.75799163)(594.17102388,698.72799154)
\curveto(594.22102016,698.71799167)(594.26102012,698.71299168)(594.29102388,698.71299154)
\curveto(594.33102005,698.71299168)(594.37102001,698.70799168)(594.41102388,698.69799154)
\curveto(594.46101992,698.67799171)(594.51101987,698.67299172)(594.56102388,698.68299154)
\curveto(594.62101976,698.68299171)(594.6760197,698.67799171)(594.72602388,698.66799154)
\lineto(594.87602388,698.66799154)
\curveto(594.93601944,698.64799174)(595.02101936,698.64299175)(595.13102388,698.65299154)
\curveto(595.24101914,698.65299174)(595.32101906,698.65799173)(595.37102388,698.66799154)
\curveto(595.40101898,698.66799172)(595.43101895,698.67299172)(595.46102388,698.68299154)
\lineto(595.56602388,698.68299154)
\curveto(595.61601876,698.6929917)(595.67101871,698.69799169)(595.73102388,698.69799154)
\curveto(595.79101859,698.69799169)(595.84601853,698.70799168)(595.89602388,698.72799154)
\curveto(596.02601835,698.75799163)(596.15101823,698.7879916)(596.27102388,698.81799154)
\curveto(596.40101798,698.83799155)(596.52601785,698.87299152)(596.64602388,698.92299154)
\curveto(597.12601725,699.12299127)(597.53601684,699.37299102)(597.87602388,699.67299154)
\curveto(598.21601616,699.97299042)(598.49101589,700.36299003)(598.70102388,700.84299154)
\curveto(598.75101563,700.94298945)(598.79101559,701.04798934)(598.82102388,701.15799154)
\curveto(598.85101553,701.27798911)(598.88601549,701.392989)(598.92602388,701.50299154)
\curveto(598.93601544,701.57298882)(598.94601543,701.63798875)(598.95602388,701.69799154)
\curveto(598.96601541,701.75798863)(598.9810154,701.82298857)(599.00102388,701.89299154)
\curveto(599.02101536,701.97298842)(599.02601535,702.05298834)(599.01602388,702.13299154)
\curveto(599.01601536,702.21298818)(599.02601535,702.2929881)(599.04602388,702.37299154)
\lineto(599.04602388,702.52299154)
\curveto(599.06601531,702.58298781)(599.0760153,702.66798772)(599.07602388,702.77799154)
\curveto(599.0760153,702.8879875)(599.06601531,702.97298742)(599.04602388,703.03299154)
\moveto(596.94602388,702.49299154)
\curveto(596.93601744,702.44298795)(596.93101745,702.392988)(596.93102388,702.34299154)
\lineto(596.93102388,702.20799154)
\curveto(596.92101746,702.16798822)(596.91601746,702.12798826)(596.91602388,702.08799154)
\curveto(596.91601746,702.05798833)(596.91101747,702.02298837)(596.90102388,701.98299154)
\curveto(596.87101751,701.87298852)(596.84601753,701.76798862)(596.82602388,701.66799154)
\curveto(596.80601757,701.56798882)(596.7760176,701.46798892)(596.73602388,701.36799154)
\curveto(596.62601775,701.11798927)(596.49101789,700.90798948)(596.33102388,700.73799154)
\curveto(596.17101821,700.56798982)(595.96101842,700.43298996)(595.70102388,700.33299154)
\curveto(595.63101875,700.30299009)(595.55601882,700.28299011)(595.47602388,700.27299154)
\curveto(595.39601898,700.26299013)(595.31601906,700.24799014)(595.23602388,700.22799154)
\lineto(595.11602388,700.22799154)
\curveto(595.0760193,700.21799017)(595.03101935,700.21299018)(594.98102388,700.21299154)
\lineto(594.86102388,700.24299154)
\curveto(594.82101956,700.25299014)(594.78601959,700.25299014)(594.75602388,700.24299154)
\curveto(594.72601965,700.24299015)(594.69101969,700.24799014)(594.65102388,700.25799154)
\curveto(594.56101982,700.27799011)(594.47101991,700.30299009)(594.38102388,700.33299154)
\curveto(594.30102008,700.36299003)(594.22602015,700.40298999)(594.15602388,700.45299154)
\curveto(593.90602047,700.60298979)(593.72102066,700.76798962)(593.60102388,700.94799154)
\curveto(593.49102089,701.13798925)(593.38602099,701.38298901)(593.28602388,701.68299154)
\curveto(593.26602111,701.76298863)(593.25102113,701.83798855)(593.24102388,701.90799154)
\curveto(593.23102115,701.9879884)(593.21602116,702.06798832)(593.19602388,702.14799154)
\lineto(593.19602388,702.28299154)
\curveto(593.1760212,702.35298804)(593.16102122,702.45798793)(593.15102388,702.59799154)
\curveto(593.15102123,702.73798765)(593.16102122,702.84298755)(593.18102388,702.91299154)
\lineto(593.18102388,703.06299154)
\curveto(593.1810212,703.11298728)(593.18602119,703.16298723)(593.19602388,703.21299154)
\curveto(593.21602116,703.32298707)(593.23102115,703.43298696)(593.24102388,703.54299154)
\curveto(593.26102112,703.65298674)(593.28602109,703.75798663)(593.31602388,703.85799154)
\curveto(593.40602097,704.12798626)(593.52602085,704.36298603)(593.67602388,704.56299154)
\curveto(593.83602054,704.77298562)(594.04102034,704.93298546)(594.29102388,705.04299154)
\curveto(594.34102004,705.07298532)(594.39601998,705.0929853)(594.45602388,705.10299154)
\lineto(594.66602388,705.16299154)
\curveto(594.69601968,705.17298522)(594.73101965,705.17298522)(594.77102388,705.16299154)
\curveto(594.81101957,705.16298523)(594.84601953,705.17298522)(594.87602388,705.19299154)
\lineto(595.14602388,705.19299154)
\curveto(595.23601914,705.20298519)(595.32101906,705.19798519)(595.40102388,705.17799154)
\curveto(595.47101891,705.15798523)(595.53601884,705.13798525)(595.59602388,705.11799154)
\curveto(595.65601872,705.10798528)(595.71601866,705.0929853)(595.77602388,705.07299154)
\curveto(596.02601835,704.96298543)(596.22601815,704.81298558)(596.37602388,704.62299154)
\curveto(596.52601785,704.44298595)(596.65601772,704.22298617)(596.76602388,703.96299154)
\curveto(596.79601758,703.88298651)(596.81601756,703.79798659)(596.82602388,703.70799154)
\lineto(596.88602388,703.46799154)
\curveto(596.89601748,703.44798694)(596.90101748,703.41798697)(596.90102388,703.37799154)
\curveto(596.91101747,703.32798706)(596.91601746,703.27298712)(596.91602388,703.21299154)
\curveto(596.91601746,703.15298724)(596.92601745,703.09798729)(596.94602388,703.04799154)
\lineto(596.94602388,702.92799154)
\curveto(596.95601742,702.87798751)(596.96101742,702.80298759)(596.96102388,702.70299154)
\curveto(596.96101742,702.61298778)(596.95601742,702.54298785)(596.94602388,702.49299154)
\moveto(595.71602388,709.66299154)
\lineto(596.78102388,709.66299154)
\curveto(596.86101752,709.66298073)(596.95601742,709.66298073)(597.06602388,709.66299154)
\curveto(597.1760172,709.66298073)(597.25601712,709.64798074)(597.30602388,709.61799154)
\curveto(597.32601705,709.60798078)(597.33601704,709.5929808)(597.33602388,709.57299154)
\curveto(597.34601703,709.56298083)(597.36101702,709.55298084)(597.38102388,709.54299154)
\curveto(597.39101699,709.42298097)(597.34101704,709.31798107)(597.23102388,709.22799154)
\curveto(597.13101725,709.13798125)(597.04601733,709.05798133)(596.97602388,708.98799154)
\curveto(596.89601748,708.91798147)(596.81601756,708.84298155)(596.73602388,708.76299154)
\curveto(596.66601771,708.6929817)(596.59101779,708.62798176)(596.51102388,708.56799154)
\curveto(596.47101791,708.53798185)(596.43601794,708.50298189)(596.40602388,708.46299154)
\curveto(596.38601799,708.43298196)(596.35601802,708.40798198)(596.31602388,708.38799154)
\curveto(596.29601808,708.35798203)(596.27101811,708.33298206)(596.24102388,708.31299154)
\lineto(596.09102388,708.16299154)
\lineto(595.94102388,708.04299154)
\lineto(595.89602388,707.99799154)
\curveto(595.89601848,707.9879824)(595.88601849,707.97298242)(595.86602388,707.95299154)
\curveto(595.78601859,707.8929825)(595.70601867,707.82798256)(595.62602388,707.75799154)
\curveto(595.55601882,707.6879827)(595.46601891,707.63298276)(595.35602388,707.59299154)
\curveto(595.31601906,707.58298281)(595.2760191,707.57798281)(595.23602388,707.57799154)
\curveto(595.20601917,707.57798281)(595.16601921,707.57298282)(595.11602388,707.56299154)
\curveto(595.08601929,707.55298284)(595.04601933,707.54798284)(594.99602388,707.54799154)
\curveto(594.94601943,707.55798283)(594.90101948,707.56298283)(594.86102388,707.56299154)
\lineto(594.51602388,707.56299154)
\curveto(594.39601998,707.56298283)(594.30602007,707.5879828)(594.24602388,707.63799154)
\curveto(594.18602019,707.67798271)(594.17102021,707.74798264)(594.20102388,707.84799154)
\curveto(594.22102016,707.92798246)(594.25602012,707.99798239)(594.30602388,708.05799154)
\curveto(594.35602002,708.12798226)(594.40101998,708.19798219)(594.44102388,708.26799154)
\curveto(594.54101984,708.40798198)(594.63601974,708.54298185)(594.72602388,708.67299154)
\curveto(594.81601956,708.80298159)(594.90601947,708.93798145)(594.99602388,709.07799154)
\curveto(595.04601933,709.15798123)(595.09601928,709.24298115)(595.14602388,709.33299154)
\curveto(595.20601917,709.42298097)(595.27101911,709.4929809)(595.34102388,709.54299154)
\curveto(595.381019,709.57298082)(595.45101893,709.60798078)(595.55102388,709.64799154)
\curveto(595.57101881,709.65798073)(595.59601878,709.65798073)(595.62602388,709.64799154)
\curveto(595.66601871,709.64798074)(595.69601868,709.65298074)(595.71602388,709.66299154)
}
}
{
\newrgbcolor{curcolor}{0 0 0}
\pscustom[linestyle=none,fillstyle=solid,fillcolor=curcolor]
{
\newpath
\moveto(604.87094576,706.78299154)
\curveto(605.47093995,706.80298359)(605.97093945,706.71798367)(606.37094576,706.52799154)
\curveto(606.77093865,706.33798405)(607.08593834,706.05798433)(607.31594576,705.68799154)
\curveto(607.38593804,705.57798481)(607.44093798,705.45798493)(607.48094576,705.32799154)
\curveto(607.5209379,705.20798518)(607.56093786,705.08298531)(607.60094576,704.95299154)
\curveto(607.6209378,704.87298552)(607.63093779,704.79798559)(607.63094576,704.72799154)
\curveto(607.64093778,704.65798573)(607.65593777,704.5879858)(607.67594576,704.51799154)
\curveto(607.67593775,704.45798593)(607.68093774,704.41798597)(607.69094576,704.39799154)
\curveto(607.71093771,704.25798613)(607.7209377,704.11298628)(607.72094576,703.96299154)
\lineto(607.72094576,703.52799154)
\lineto(607.72094576,702.19299154)
\lineto(607.72094576,699.76299154)
\curveto(607.7209377,699.57299082)(607.71593771,699.387991)(607.70594576,699.20799154)
\curveto(607.70593772,699.03799135)(607.63593779,698.92799146)(607.49594576,698.87799154)
\curveto(607.43593799,698.85799153)(607.36593806,698.84799154)(607.28594576,698.84799154)
\lineto(607.04594576,698.84799154)
\lineto(606.23594576,698.84799154)
\curveto(606.11593931,698.84799154)(606.00593942,698.85299154)(605.90594576,698.86299154)
\curveto(605.81593961,698.88299151)(605.74593968,698.92799146)(605.69594576,698.99799154)
\curveto(605.65593977,699.05799133)(605.63093979,699.13299126)(605.62094576,699.22299154)
\lineto(605.62094576,699.53799154)
\lineto(605.62094576,700.58799154)
\lineto(605.62094576,702.82299154)
\curveto(605.6209398,703.1929872)(605.60593982,703.53298686)(605.57594576,703.84299154)
\curveto(605.54593988,704.16298623)(605.45593997,704.43298596)(605.30594576,704.65299154)
\curveto(605.16594026,704.85298554)(604.96094046,704.9929854)(604.69094576,705.07299154)
\curveto(604.64094078,705.0929853)(604.58594084,705.10298529)(604.52594576,705.10299154)
\curveto(604.47594095,705.10298529)(604.420941,705.11298528)(604.36094576,705.13299154)
\curveto(604.31094111,705.14298525)(604.24594118,705.14298525)(604.16594576,705.13299154)
\curveto(604.09594133,705.13298526)(604.04094138,705.12798526)(604.00094576,705.11799154)
\curveto(603.96094146,705.10798528)(603.9259415,705.10298529)(603.89594576,705.10299154)
\curveto(603.86594156,705.10298529)(603.83594159,705.09798529)(603.80594576,705.08799154)
\curveto(603.57594185,705.02798536)(603.39094203,704.94798544)(603.25094576,704.84799154)
\curveto(602.93094249,704.61798577)(602.74094268,704.28298611)(602.68094576,703.84299154)
\curveto(602.6209428,703.40298699)(602.59094283,702.90798748)(602.59094576,702.35799154)
\lineto(602.59094576,700.48299154)
\lineto(602.59094576,699.56799154)
\lineto(602.59094576,699.29799154)
\curveto(602.59094283,699.20799118)(602.57594285,699.13299126)(602.54594576,699.07299154)
\curveto(602.49594293,698.96299143)(602.41594301,698.89799149)(602.30594576,698.87799154)
\curveto(602.19594323,698.85799153)(602.06094336,698.84799154)(601.90094576,698.84799154)
\lineto(601.15094576,698.84799154)
\curveto(601.04094438,698.84799154)(600.93094449,698.85299154)(600.82094576,698.86299154)
\curveto(600.71094471,698.87299152)(600.63094479,698.90799148)(600.58094576,698.96799154)
\curveto(600.51094491,699.05799133)(600.47594495,699.1879912)(600.47594576,699.35799154)
\curveto(600.48594494,699.52799086)(600.49094493,699.6879907)(600.49094576,699.83799154)
\lineto(600.49094576,701.87799154)
\lineto(600.49094576,705.17799154)
\lineto(600.49094576,705.94299154)
\lineto(600.49094576,706.24299154)
\curveto(600.50094492,706.33298406)(600.53094489,706.40798398)(600.58094576,706.46799154)
\curveto(600.60094482,706.49798389)(600.63094479,706.51798387)(600.67094576,706.52799154)
\curveto(600.7209447,706.54798384)(600.77094465,706.56298383)(600.82094576,706.57299154)
\lineto(600.89594576,706.57299154)
\curveto(600.94594448,706.58298381)(600.99594443,706.5879838)(601.04594576,706.58799154)
\lineto(601.21094576,706.58799154)
\lineto(601.84094576,706.58799154)
\curveto(601.9209435,706.5879838)(601.99594343,706.58298381)(602.06594576,706.57299154)
\curveto(602.14594328,706.57298382)(602.21594321,706.56298383)(602.27594576,706.54299154)
\curveto(602.34594308,706.51298388)(602.39094303,706.46798392)(602.41094576,706.40799154)
\curveto(602.44094298,706.34798404)(602.46594296,706.27798411)(602.48594576,706.19799154)
\curveto(602.49594293,706.15798423)(602.49594293,706.12298427)(602.48594576,706.09299154)
\curveto(602.48594294,706.06298433)(602.49594293,706.03298436)(602.51594576,706.00299154)
\curveto(602.53594289,705.95298444)(602.55094287,705.92298447)(602.56094576,705.91299154)
\curveto(602.58094284,705.90298449)(602.60594282,705.8879845)(602.63594576,705.86799154)
\curveto(602.74594268,705.85798453)(602.83594259,705.8929845)(602.90594576,705.97299154)
\curveto(602.97594245,706.06298433)(603.05094237,706.13298426)(603.13094576,706.18299154)
\curveto(603.40094202,706.38298401)(603.70094172,706.54298385)(604.03094576,706.66299154)
\curveto(604.1209413,706.6929837)(604.21094121,706.71298368)(604.30094576,706.72299154)
\curveto(604.40094102,706.73298366)(604.50594092,706.74798364)(604.61594576,706.76799154)
\curveto(604.64594078,706.77798361)(604.69094073,706.77798361)(604.75094576,706.76799154)
\curveto(604.81094061,706.76798362)(604.85094057,706.77298362)(604.87094576,706.78299154)
}
}
{
\newrgbcolor{curcolor}{0 0 0}
\pscustom[linestyle=none,fillstyle=solid,fillcolor=curcolor]
{
\newpath
\moveto(19.26111654,349.66083456)
\lineto(20.55111654,349.66083456)
\curveto(20.66111372,349.66082388)(20.76611361,349.65582389)(20.86611654,349.64583456)
\curveto(20.96611341,349.6458239)(21.04111334,349.61082393)(21.09111654,349.54083456)
\curveto(21.14111324,349.47082407)(21.16611321,349.38082416)(21.16611654,349.27083456)
\curveto(21.1761132,349.16082438)(21.1811132,349.0408245)(21.18111654,348.91083456)
\lineto(21.18111654,347.60583456)
\lineto(21.18111654,342.40083456)
\lineto(21.18111654,339.94083456)
\lineto(21.18111654,339.50583456)
\curveto(21.19111319,339.3458342)(21.17111321,339.22583432)(21.12111654,339.14583456)
\curveto(21.0811133,339.07583447)(20.99111339,339.02083452)(20.85111654,338.98083456)
\curveto(20.7811136,338.96083458)(20.70611367,338.95583459)(20.62611654,338.96583456)
\curveto(20.54611383,338.97583457)(20.46611391,338.98083456)(20.38611654,338.98083456)
\lineto(19.50111654,338.98083456)
\curveto(19.39111499,338.98083456)(19.28611509,338.98583456)(19.18611654,338.99583456)
\curveto(19.09611528,339.00583454)(19.02111536,339.03583451)(18.96111654,339.08583456)
\curveto(18.91111547,339.13583441)(18.8811155,339.21083433)(18.87111654,339.31083456)
\curveto(18.86111552,339.41083413)(18.85611552,339.51583403)(18.85611654,339.62583456)
\lineto(18.85611654,340.93083456)
\lineto(18.85611654,346.40583456)
\lineto(18.85611654,348.59583456)
\curveto(18.85611552,348.73582481)(18.85111553,348.90082464)(18.84111654,349.09083456)
\curveto(18.84111554,349.28082426)(18.86611551,349.41582413)(18.91611654,349.49583456)
\curveto(18.95611542,349.55582399)(19.02111536,349.60582394)(19.11111654,349.64583456)
\curveto(19.14111524,349.6458239)(19.16611521,349.6458239)(19.18611654,349.64583456)
\curveto(19.21611516,349.65582389)(19.24111514,349.66082388)(19.26111654,349.66083456)
}
}
{
\newrgbcolor{curcolor}{0 0 0}
\pscustom[linestyle=none,fillstyle=solid,fillcolor=curcolor]
{
\newpath
\moveto(27.46494467,346.91583456)
\curveto(28.06493886,346.93582661)(28.56493836,346.85082669)(28.96494467,346.66083456)
\curveto(29.36493756,346.47082707)(29.67993725,346.19082735)(29.90994467,345.82083456)
\curveto(29.97993695,345.71082783)(30.03493689,345.59082795)(30.07494467,345.46083456)
\curveto(30.11493681,345.3408282)(30.15493677,345.21582833)(30.19494467,345.08583456)
\curveto(30.21493671,345.00582854)(30.2249367,344.93082861)(30.22494467,344.86083456)
\curveto(30.23493669,344.79082875)(30.24993668,344.72082882)(30.26994467,344.65083456)
\curveto(30.26993666,344.59082895)(30.27493665,344.55082899)(30.28494467,344.53083456)
\curveto(30.30493662,344.39082915)(30.31493661,344.2458293)(30.31494467,344.09583456)
\lineto(30.31494467,343.66083456)
\lineto(30.31494467,342.32583456)
\lineto(30.31494467,339.89583456)
\curveto(30.31493661,339.70583384)(30.30993662,339.52083402)(30.29994467,339.34083456)
\curveto(30.29993663,339.17083437)(30.2299367,339.06083448)(30.08994467,339.01083456)
\curveto(30.0299369,338.99083455)(29.95993697,338.98083456)(29.87994467,338.98083456)
\lineto(29.63994467,338.98083456)
\lineto(28.82994467,338.98083456)
\curveto(28.70993822,338.98083456)(28.59993833,338.98583456)(28.49994467,338.99583456)
\curveto(28.40993852,339.01583453)(28.33993859,339.06083448)(28.28994467,339.13083456)
\curveto(28.24993868,339.19083435)(28.2249387,339.26583428)(28.21494467,339.35583456)
\lineto(28.21494467,339.67083456)
\lineto(28.21494467,340.72083456)
\lineto(28.21494467,342.95583456)
\curveto(28.21493871,343.32583022)(28.19993873,343.66582988)(28.16994467,343.97583456)
\curveto(28.13993879,344.29582925)(28.04993888,344.56582898)(27.89994467,344.78583456)
\curveto(27.75993917,344.98582856)(27.55493937,345.12582842)(27.28494467,345.20583456)
\curveto(27.23493969,345.22582832)(27.17993975,345.23582831)(27.11994467,345.23583456)
\curveto(27.06993986,345.23582831)(27.01493991,345.2458283)(26.95494467,345.26583456)
\curveto(26.90494002,345.27582827)(26.83994009,345.27582827)(26.75994467,345.26583456)
\curveto(26.68994024,345.26582828)(26.63494029,345.26082828)(26.59494467,345.25083456)
\curveto(26.55494037,345.2408283)(26.51994041,345.23582831)(26.48994467,345.23583456)
\curveto(26.45994047,345.23582831)(26.4299405,345.23082831)(26.39994467,345.22083456)
\curveto(26.16994076,345.16082838)(25.98494094,345.08082846)(25.84494467,344.98083456)
\curveto(25.5249414,344.75082879)(25.33494159,344.41582913)(25.27494467,343.97583456)
\curveto(25.21494171,343.53583001)(25.18494174,343.0408305)(25.18494467,342.49083456)
\lineto(25.18494467,340.61583456)
\lineto(25.18494467,339.70083456)
\lineto(25.18494467,339.43083456)
\curveto(25.18494174,339.3408342)(25.16994176,339.26583428)(25.13994467,339.20583456)
\curveto(25.08994184,339.09583445)(25.00994192,339.03083451)(24.89994467,339.01083456)
\curveto(24.78994214,338.99083455)(24.65494227,338.98083456)(24.49494467,338.98083456)
\lineto(23.74494467,338.98083456)
\curveto(23.63494329,338.98083456)(23.5249434,338.98583456)(23.41494467,338.99583456)
\curveto(23.30494362,339.00583454)(23.2249437,339.0408345)(23.17494467,339.10083456)
\curveto(23.10494382,339.19083435)(23.06994386,339.32083422)(23.06994467,339.49083456)
\curveto(23.07994385,339.66083388)(23.08494384,339.82083372)(23.08494467,339.97083456)
\lineto(23.08494467,342.01083456)
\lineto(23.08494467,345.31083456)
\lineto(23.08494467,346.07583456)
\lineto(23.08494467,346.37583456)
\curveto(23.09494383,346.46582708)(23.1249438,346.540827)(23.17494467,346.60083456)
\curveto(23.19494373,346.63082691)(23.2249437,346.65082689)(23.26494467,346.66083456)
\curveto(23.31494361,346.68082686)(23.36494356,346.69582685)(23.41494467,346.70583456)
\lineto(23.48994467,346.70583456)
\curveto(23.53994339,346.71582683)(23.58994334,346.72082682)(23.63994467,346.72083456)
\lineto(23.80494467,346.72083456)
\lineto(24.43494467,346.72083456)
\curveto(24.51494241,346.72082682)(24.58994234,346.71582683)(24.65994467,346.70583456)
\curveto(24.73994219,346.70582684)(24.80994212,346.69582685)(24.86994467,346.67583456)
\curveto(24.93994199,346.6458269)(24.98494194,346.60082694)(25.00494467,346.54083456)
\curveto(25.03494189,346.48082706)(25.05994187,346.41082713)(25.07994467,346.33083456)
\curveto(25.08994184,346.29082725)(25.08994184,346.25582729)(25.07994467,346.22583456)
\curveto(25.07994185,346.19582735)(25.08994184,346.16582738)(25.10994467,346.13583456)
\curveto(25.1299418,346.08582746)(25.14494178,346.05582749)(25.15494467,346.04583456)
\curveto(25.17494175,346.03582751)(25.19994173,346.02082752)(25.22994467,346.00083456)
\curveto(25.33994159,345.99082755)(25.4299415,346.02582752)(25.49994467,346.10583456)
\curveto(25.56994136,346.19582735)(25.64494128,346.26582728)(25.72494467,346.31583456)
\curveto(25.99494093,346.51582703)(26.29494063,346.67582687)(26.62494467,346.79583456)
\curveto(26.71494021,346.82582672)(26.80494012,346.8458267)(26.89494467,346.85583456)
\curveto(26.99493993,346.86582668)(27.09993983,346.88082666)(27.20994467,346.90083456)
\curveto(27.23993969,346.91082663)(27.28493964,346.91082663)(27.34494467,346.90083456)
\curveto(27.40493952,346.90082664)(27.44493948,346.90582664)(27.46494467,346.91583456)
}
}
{
\newrgbcolor{curcolor}{0 0 0}
\pscustom[linestyle=none,fillstyle=solid,fillcolor=curcolor]
{
\newpath
\moveto(39.53619467,339.83583456)
\lineto(39.53619467,339.41583456)
\curveto(39.5361863,339.28583426)(39.50618633,339.18083436)(39.44619467,339.10083456)
\curveto(39.39618644,339.05083449)(39.3311865,339.01583453)(39.25119467,338.99583456)
\curveto(39.17118666,338.98583456)(39.08118675,338.98083456)(38.98119467,338.98083456)
\lineto(38.15619467,338.98083456)
\lineto(37.87119467,338.98083456)
\curveto(37.79118804,338.99083455)(37.72618811,339.01583453)(37.67619467,339.05583456)
\curveto(37.60618823,339.10583444)(37.56618827,339.17083437)(37.55619467,339.25083456)
\curveto(37.54618829,339.33083421)(37.52618831,339.41083413)(37.49619467,339.49083456)
\curveto(37.47618836,339.51083403)(37.45618838,339.52583402)(37.43619467,339.53583456)
\curveto(37.42618841,339.55583399)(37.41118842,339.57583397)(37.39119467,339.59583456)
\curveto(37.28118855,339.59583395)(37.20118863,339.57083397)(37.15119467,339.52083456)
\lineto(37.00119467,339.37083456)
\curveto(36.9311889,339.32083422)(36.86618897,339.27583427)(36.80619467,339.23583456)
\curveto(36.74618909,339.20583434)(36.68118915,339.16583438)(36.61119467,339.11583456)
\curveto(36.57118926,339.09583445)(36.52618931,339.07583447)(36.47619467,339.05583456)
\curveto(36.4361894,339.03583451)(36.39118944,339.01583453)(36.34119467,338.99583456)
\curveto(36.20118963,338.9458346)(36.05118978,338.90083464)(35.89119467,338.86083456)
\curveto(35.84118999,338.8408347)(35.79619004,338.83083471)(35.75619467,338.83083456)
\curveto(35.71619012,338.83083471)(35.67619016,338.82583472)(35.63619467,338.81583456)
\lineto(35.50119467,338.81583456)
\curveto(35.47119036,338.80583474)(35.4311904,338.80083474)(35.38119467,338.80083456)
\lineto(35.24619467,338.80083456)
\curveto(35.18619065,338.78083476)(35.09619074,338.77583477)(34.97619467,338.78583456)
\curveto(34.85619098,338.78583476)(34.77119106,338.79583475)(34.72119467,338.81583456)
\curveto(34.65119118,338.83583471)(34.58619125,338.8458347)(34.52619467,338.84583456)
\curveto(34.47619136,338.83583471)(34.42119141,338.8408347)(34.36119467,338.86083456)
\lineto(34.00119467,338.98083456)
\curveto(33.89119194,339.01083453)(33.78119205,339.05083449)(33.67119467,339.10083456)
\curveto(33.32119251,339.25083429)(33.00619283,339.48083406)(32.72619467,339.79083456)
\curveto(32.45619338,340.11083343)(32.24119359,340.4458331)(32.08119467,340.79583456)
\curveto(32.0311938,340.90583264)(31.99119384,341.01083253)(31.96119467,341.11083456)
\curveto(31.9311939,341.22083232)(31.89619394,341.33083221)(31.85619467,341.44083456)
\curveto(31.84619399,341.48083206)(31.84119399,341.51583203)(31.84119467,341.54583456)
\curveto(31.84119399,341.58583196)(31.831194,341.63083191)(31.81119467,341.68083456)
\curveto(31.79119404,341.76083178)(31.77119406,341.8458317)(31.75119467,341.93583456)
\curveto(31.74119409,342.03583151)(31.72619411,342.13583141)(31.70619467,342.23583456)
\curveto(31.69619414,342.26583128)(31.69119414,342.30083124)(31.69119467,342.34083456)
\curveto(31.70119413,342.38083116)(31.70119413,342.41583113)(31.69119467,342.44583456)
\lineto(31.69119467,342.58083456)
\curveto(31.69119414,342.63083091)(31.68619415,342.68083086)(31.67619467,342.73083456)
\curveto(31.66619417,342.78083076)(31.66119417,342.83583071)(31.66119467,342.89583456)
\curveto(31.66119417,342.96583058)(31.66619417,343.02083052)(31.67619467,343.06083456)
\curveto(31.68619415,343.11083043)(31.69119414,343.15583039)(31.69119467,343.19583456)
\lineto(31.69119467,343.34583456)
\curveto(31.70119413,343.39583015)(31.70119413,343.4408301)(31.69119467,343.48083456)
\curveto(31.69119414,343.53083001)(31.70119413,343.58082996)(31.72119467,343.63083456)
\curveto(31.74119409,343.7408298)(31.75619408,343.8458297)(31.76619467,343.94583456)
\curveto(31.78619405,344.0458295)(31.81119402,344.1458294)(31.84119467,344.24583456)
\curveto(31.88119395,344.36582918)(31.91619392,344.48082906)(31.94619467,344.59083456)
\curveto(31.97619386,344.70082884)(32.01619382,344.81082873)(32.06619467,344.92083456)
\curveto(32.20619363,345.22082832)(32.38119345,345.50582804)(32.59119467,345.77583456)
\curveto(32.61119322,345.80582774)(32.6361932,345.83082771)(32.66619467,345.85083456)
\curveto(32.70619313,345.88082766)(32.7361931,345.91082763)(32.75619467,345.94083456)
\curveto(32.79619304,345.99082755)(32.836193,346.03582751)(32.87619467,346.07583456)
\curveto(32.91619292,346.11582743)(32.96119287,346.15582739)(33.01119467,346.19583456)
\curveto(33.05119278,346.21582733)(33.08619275,346.2408273)(33.11619467,346.27083456)
\curveto(33.14619269,346.31082723)(33.18119265,346.3408272)(33.22119467,346.36083456)
\curveto(33.47119236,346.53082701)(33.76119207,346.67082687)(34.09119467,346.78083456)
\curveto(34.16119167,346.80082674)(34.2311916,346.81582673)(34.30119467,346.82583456)
\curveto(34.38119145,346.83582671)(34.46119137,346.85082669)(34.54119467,346.87083456)
\curveto(34.61119122,346.89082665)(34.70119113,346.90082664)(34.81119467,346.90083456)
\curveto(34.92119091,346.91082663)(35.0311908,346.91582663)(35.14119467,346.91583456)
\curveto(35.25119058,346.91582663)(35.35619048,346.91082663)(35.45619467,346.90083456)
\curveto(35.56619027,346.89082665)(35.65619018,346.87582667)(35.72619467,346.85583456)
\curveto(35.87618996,346.80582674)(36.02118981,346.76082678)(36.16119467,346.72083456)
\curveto(36.30118953,346.68082686)(36.4311894,346.62582692)(36.55119467,346.55583456)
\curveto(36.62118921,346.50582704)(36.68618915,346.45582709)(36.74619467,346.40583456)
\curveto(36.80618903,346.36582718)(36.87118896,346.32082722)(36.94119467,346.27083456)
\curveto(36.98118885,346.2408273)(37.0361888,346.20082734)(37.10619467,346.15083456)
\curveto(37.18618865,346.10082744)(37.26118857,346.10082744)(37.33119467,346.15083456)
\curveto(37.37118846,346.17082737)(37.39118844,346.20582734)(37.39119467,346.25583456)
\curveto(37.39118844,346.30582724)(37.40118843,346.35582719)(37.42119467,346.40583456)
\lineto(37.42119467,346.55583456)
\curveto(37.4311884,346.58582696)(37.4361884,346.62082692)(37.43619467,346.66083456)
\lineto(37.43619467,346.78083456)
\lineto(37.43619467,348.82083456)
\curveto(37.4361884,348.93082461)(37.4311884,349.05082449)(37.42119467,349.18083456)
\curveto(37.42118841,349.32082422)(37.44618839,349.42582412)(37.49619467,349.49583456)
\curveto(37.5361883,349.57582397)(37.61118822,349.62582392)(37.72119467,349.64583456)
\curveto(37.74118809,349.65582389)(37.76118807,349.65582389)(37.78119467,349.64583456)
\curveto(37.80118803,349.6458239)(37.82118801,349.65082389)(37.84119467,349.66083456)
\lineto(38.90619467,349.66083456)
\curveto(39.02618681,349.66082388)(39.1361867,349.65582389)(39.23619467,349.64583456)
\curveto(39.3361865,349.63582391)(39.41118642,349.59582395)(39.46119467,349.52583456)
\curveto(39.51118632,349.4458241)(39.5361863,349.3408242)(39.53619467,349.21083456)
\lineto(39.53619467,348.85083456)
\lineto(39.53619467,339.83583456)
\moveto(37.49619467,342.77583456)
\curveto(37.50618833,342.81583073)(37.50618833,342.85583069)(37.49619467,342.89583456)
\lineto(37.49619467,343.03083456)
\curveto(37.49618834,343.13083041)(37.49118834,343.23083031)(37.48119467,343.33083456)
\curveto(37.47118836,343.43083011)(37.45618838,343.52083002)(37.43619467,343.60083456)
\curveto(37.41618842,343.71082983)(37.39618844,343.81082973)(37.37619467,343.90083456)
\curveto(37.36618847,343.99082955)(37.34118849,344.07582947)(37.30119467,344.15583456)
\curveto(37.16118867,344.51582903)(36.95618888,344.80082874)(36.68619467,345.01083456)
\curveto(36.42618941,345.22082832)(36.04618979,345.32582822)(35.54619467,345.32583456)
\curveto(35.48619035,345.32582822)(35.40619043,345.31582823)(35.30619467,345.29583456)
\curveto(35.22619061,345.27582827)(35.15119068,345.25582829)(35.08119467,345.23583456)
\curveto(35.02119081,345.22582832)(34.96119087,345.20582834)(34.90119467,345.17583456)
\curveto(34.6311912,345.06582848)(34.42119141,344.89582865)(34.27119467,344.66583456)
\curveto(34.12119171,344.43582911)(34.00119183,344.17582937)(33.91119467,343.88583456)
\curveto(33.88119195,343.78582976)(33.86119197,343.68582986)(33.85119467,343.58583456)
\curveto(33.84119199,343.48583006)(33.82119201,343.38083016)(33.79119467,343.27083456)
\lineto(33.79119467,343.06083456)
\curveto(33.77119206,342.97083057)(33.76619207,342.8458307)(33.77619467,342.68583456)
\curveto(33.78619205,342.53583101)(33.80119203,342.42583112)(33.82119467,342.35583456)
\lineto(33.82119467,342.26583456)
\curveto(33.831192,342.2458313)(33.836192,342.22583132)(33.83619467,342.20583456)
\curveto(33.85619198,342.12583142)(33.87119196,342.05083149)(33.88119467,341.98083456)
\curveto(33.90119193,341.91083163)(33.92119191,341.83583171)(33.94119467,341.75583456)
\curveto(34.11119172,341.23583231)(34.40119143,340.85083269)(34.81119467,340.60083456)
\curveto(34.94119089,340.51083303)(35.12119071,340.4408331)(35.35119467,340.39083456)
\curveto(35.39119044,340.38083316)(35.45119038,340.37583317)(35.53119467,340.37583456)
\curveto(35.56119027,340.36583318)(35.60619023,340.35583319)(35.66619467,340.34583456)
\curveto(35.7361901,340.3458332)(35.79119004,340.35083319)(35.83119467,340.36083456)
\curveto(35.91118992,340.38083316)(35.99118984,340.39583315)(36.07119467,340.40583456)
\curveto(36.15118968,340.41583313)(36.2311896,340.43583311)(36.31119467,340.46583456)
\curveto(36.56118927,340.57583297)(36.76118907,340.71583283)(36.91119467,340.88583456)
\curveto(37.06118877,341.05583249)(37.19118864,341.27083227)(37.30119467,341.53083456)
\curveto(37.34118849,341.62083192)(37.37118846,341.71083183)(37.39119467,341.80083456)
\curveto(37.41118842,341.90083164)(37.4311884,342.00583154)(37.45119467,342.11583456)
\curveto(37.46118837,342.16583138)(37.46118837,342.21083133)(37.45119467,342.25083456)
\curveto(37.45118838,342.30083124)(37.46118837,342.35083119)(37.48119467,342.40083456)
\curveto(37.49118834,342.43083111)(37.49618834,342.46583108)(37.49619467,342.50583456)
\lineto(37.49619467,342.64083456)
\lineto(37.49619467,342.77583456)
}
}
{
\newrgbcolor{curcolor}{0 0 0}
\pscustom[linestyle=none,fillstyle=solid,fillcolor=curcolor]
{
\newpath
\moveto(43.21611654,349.57083456)
\curveto(43.28611359,349.49082405)(43.32111356,349.37082417)(43.32111654,349.21083456)
\lineto(43.32111654,348.74583456)
\lineto(43.32111654,348.34083456)
\curveto(43.32111356,348.20082534)(43.28611359,348.10582544)(43.21611654,348.05583456)
\curveto(43.15611372,348.00582554)(43.0761138,347.97582557)(42.97611654,347.96583456)
\curveto(42.88611399,347.95582559)(42.78611409,347.95082559)(42.67611654,347.95083456)
\lineto(41.83611654,347.95083456)
\curveto(41.72611515,347.95082559)(41.62611525,347.95582559)(41.53611654,347.96583456)
\curveto(41.45611542,347.97582557)(41.38611549,348.00582554)(41.32611654,348.05583456)
\curveto(41.28611559,348.08582546)(41.25611562,348.1408254)(41.23611654,348.22083456)
\curveto(41.22611565,348.31082523)(41.21611566,348.40582514)(41.20611654,348.50583456)
\lineto(41.20611654,348.83583456)
\curveto(41.21611566,348.9458246)(41.22111566,349.0408245)(41.22111654,349.12083456)
\lineto(41.22111654,349.33083456)
\curveto(41.23111565,349.40082414)(41.25111563,349.46082408)(41.28111654,349.51083456)
\curveto(41.30111558,349.55082399)(41.32611555,349.58082396)(41.35611654,349.60083456)
\lineto(41.47611654,349.66083456)
\curveto(41.49611538,349.66082388)(41.52111536,349.66082388)(41.55111654,349.66083456)
\curveto(41.5811153,349.67082387)(41.60611527,349.67582387)(41.62611654,349.67583456)
\lineto(42.72111654,349.67583456)
\curveto(42.82111406,349.67582387)(42.91611396,349.67082387)(43.00611654,349.66083456)
\curveto(43.09611378,349.65082389)(43.16611371,349.62082392)(43.21611654,349.57083456)
\moveto(43.32111654,339.80583456)
\curveto(43.32111356,339.60583394)(43.31611356,339.43583411)(43.30611654,339.29583456)
\curveto(43.29611358,339.15583439)(43.20611367,339.06083448)(43.03611654,339.01083456)
\curveto(42.9761139,338.99083455)(42.91111397,338.98083456)(42.84111654,338.98083456)
\curveto(42.77111411,338.99083455)(42.69611418,338.99583455)(42.61611654,338.99583456)
\lineto(41.77611654,338.99583456)
\curveto(41.68611519,338.99583455)(41.59611528,339.00083454)(41.50611654,339.01083456)
\curveto(41.42611545,339.02083452)(41.36611551,339.05083449)(41.32611654,339.10083456)
\curveto(41.26611561,339.17083437)(41.23111565,339.25583429)(41.22111654,339.35583456)
\lineto(41.22111654,339.70083456)
\lineto(41.22111654,346.03083456)
\lineto(41.22111654,346.33083456)
\curveto(41.22111566,346.43082711)(41.24111564,346.51082703)(41.28111654,346.57083456)
\curveto(41.34111554,346.6408269)(41.42611545,346.68582686)(41.53611654,346.70583456)
\curveto(41.55611532,346.71582683)(41.5811153,346.71582683)(41.61111654,346.70583456)
\curveto(41.65111523,346.70582684)(41.6811152,346.71082683)(41.70111654,346.72083456)
\lineto(42.45111654,346.72083456)
\lineto(42.64611654,346.72083456)
\curveto(42.72611415,346.73082681)(42.79111409,346.73082681)(42.84111654,346.72083456)
\lineto(42.96111654,346.72083456)
\curveto(43.02111386,346.70082684)(43.0761138,346.68582686)(43.12611654,346.67583456)
\curveto(43.1761137,346.66582688)(43.21611366,346.63582691)(43.24611654,346.58583456)
\curveto(43.28611359,346.53582701)(43.30611357,346.46582708)(43.30611654,346.37583456)
\curveto(43.31611356,346.28582726)(43.32111356,346.19082735)(43.32111654,346.09083456)
\lineto(43.32111654,339.80583456)
}
}
{
\newrgbcolor{curcolor}{0 0 0}
\pscustom[linestyle=none,fillstyle=solid,fillcolor=curcolor]
{
\newpath
\moveto(48.55330404,346.93083456)
\curveto(49.36329888,346.95082659)(50.03829821,346.83082671)(50.57830404,346.57083456)
\curveto(51.12829712,346.31082723)(51.56329668,345.9408276)(51.88330404,345.46083456)
\curveto(52.0432962,345.22082832)(52.16329608,344.9458286)(52.24330404,344.63583456)
\curveto(52.26329598,344.58582896)(52.27829597,344.52082902)(52.28830404,344.44083456)
\curveto(52.30829594,344.36082918)(52.30829594,344.29082925)(52.28830404,344.23083456)
\curveto(52.248296,344.12082942)(52.17829607,344.05582949)(52.07830404,344.03583456)
\curveto(51.97829627,344.02582952)(51.85829639,344.02082952)(51.71830404,344.02083456)
\lineto(50.93830404,344.02083456)
\lineto(50.65330404,344.02083456)
\curveto(50.56329768,344.02082952)(50.48829776,344.0408295)(50.42830404,344.08083456)
\curveto(50.3482979,344.12082942)(50.29329795,344.18082936)(50.26330404,344.26083456)
\curveto(50.23329801,344.35082919)(50.19329805,344.4408291)(50.14330404,344.53083456)
\curveto(50.08329816,344.6408289)(50.01829823,344.7408288)(49.94830404,344.83083456)
\curveto(49.87829837,344.92082862)(49.79829845,345.00082854)(49.70830404,345.07083456)
\curveto(49.56829868,345.16082838)(49.41329883,345.23082831)(49.24330404,345.28083456)
\curveto(49.18329906,345.30082824)(49.12329912,345.31082823)(49.06330404,345.31083456)
\curveto(49.00329924,345.31082823)(48.9482993,345.32082822)(48.89830404,345.34083456)
\lineto(48.74830404,345.34083456)
\curveto(48.5482997,345.3408282)(48.38829986,345.32082822)(48.26830404,345.28083456)
\curveto(47.97830027,345.19082835)(47.7433005,345.05082849)(47.56330404,344.86083456)
\curveto(47.38330086,344.68082886)(47.23830101,344.46082908)(47.12830404,344.20083456)
\curveto(47.07830117,344.09082945)(47.03830121,343.97082957)(47.00830404,343.84083456)
\curveto(46.98830126,343.72082982)(46.96330128,343.59082995)(46.93330404,343.45083456)
\curveto(46.92330132,343.41083013)(46.91830133,343.37083017)(46.91830404,343.33083456)
\curveto(46.91830133,343.29083025)(46.91330133,343.25083029)(46.90330404,343.21083456)
\curveto(46.88330136,343.11083043)(46.87330137,342.97083057)(46.87330404,342.79083456)
\curveto(46.88330136,342.61083093)(46.89830135,342.47083107)(46.91830404,342.37083456)
\curveto(46.91830133,342.29083125)(46.92330132,342.23583131)(46.93330404,342.20583456)
\curveto(46.95330129,342.13583141)(46.96330128,342.06583148)(46.96330404,341.99583456)
\curveto(46.97330127,341.92583162)(46.98830126,341.85583169)(47.00830404,341.78583456)
\curveto(47.08830116,341.55583199)(47.18330106,341.3458322)(47.29330404,341.15583456)
\curveto(47.40330084,340.96583258)(47.5433007,340.80583274)(47.71330404,340.67583456)
\curveto(47.75330049,340.6458329)(47.81330043,340.61083293)(47.89330404,340.57083456)
\curveto(48.00330024,340.50083304)(48.11330013,340.45583309)(48.22330404,340.43583456)
\curveto(48.3432999,340.41583313)(48.48829976,340.39583315)(48.65830404,340.37583456)
\lineto(48.74830404,340.37583456)
\curveto(48.78829946,340.37583317)(48.81829943,340.38083316)(48.83830404,340.39083456)
\lineto(48.97330404,340.39083456)
\curveto(49.0432992,340.41083313)(49.10829914,340.42583312)(49.16830404,340.43583456)
\curveto(49.23829901,340.45583309)(49.30329894,340.47583307)(49.36330404,340.49583456)
\curveto(49.66329858,340.62583292)(49.89329835,340.81583273)(50.05330404,341.06583456)
\curveto(50.09329815,341.11583243)(50.12829812,341.17083237)(50.15830404,341.23083456)
\curveto(50.18829806,341.30083224)(50.21329803,341.36083218)(50.23330404,341.41083456)
\curveto(50.27329797,341.52083202)(50.30829794,341.61583193)(50.33830404,341.69583456)
\curveto(50.36829788,341.78583176)(50.43829781,341.85583169)(50.54830404,341.90583456)
\curveto(50.63829761,341.9458316)(50.78329746,341.96083158)(50.98330404,341.95083456)
\lineto(51.47830404,341.95083456)
\lineto(51.68830404,341.95083456)
\curveto(51.76829648,341.96083158)(51.83329641,341.95583159)(51.88330404,341.93583456)
\lineto(52.00330404,341.93583456)
\lineto(52.12330404,341.90583456)
\curveto(52.16329608,341.90583164)(52.19329605,341.89583165)(52.21330404,341.87583456)
\curveto(52.26329598,341.83583171)(52.29329595,341.77583177)(52.30330404,341.69583456)
\curveto(52.32329592,341.62583192)(52.32329592,341.55083199)(52.30330404,341.47083456)
\curveto(52.21329603,341.1408324)(52.10329614,340.8458327)(51.97330404,340.58583456)
\curveto(51.56329668,339.81583373)(50.90829734,339.28083426)(50.00830404,338.98083456)
\curveto(49.90829834,338.95083459)(49.80329844,338.93083461)(49.69330404,338.92083456)
\curveto(49.58329866,338.90083464)(49.47329877,338.87583467)(49.36330404,338.84583456)
\curveto(49.30329894,338.83583471)(49.243299,338.83083471)(49.18330404,338.83083456)
\curveto(49.12329912,338.83083471)(49.06329918,338.82583472)(49.00330404,338.81583456)
\lineto(48.83830404,338.81583456)
\curveto(48.78829946,338.79583475)(48.71329953,338.79083475)(48.61330404,338.80083456)
\curveto(48.51329973,338.80083474)(48.43829981,338.80583474)(48.38830404,338.81583456)
\curveto(48.30829994,338.83583471)(48.23330001,338.8458347)(48.16330404,338.84583456)
\curveto(48.10330014,338.83583471)(48.03830021,338.8408347)(47.96830404,338.86083456)
\lineto(47.81830404,338.89083456)
\curveto(47.76830048,338.89083465)(47.71830053,338.89583465)(47.66830404,338.90583456)
\curveto(47.55830069,338.93583461)(47.45330079,338.96583458)(47.35330404,338.99583456)
\curveto(47.25330099,339.02583452)(47.15830109,339.06083448)(47.06830404,339.10083456)
\curveto(46.59830165,339.30083424)(46.20330204,339.55583399)(45.88330404,339.86583456)
\curveto(45.56330268,340.18583336)(45.30330294,340.58083296)(45.10330404,341.05083456)
\curveto(45.05330319,341.1408324)(45.01330323,341.23583231)(44.98330404,341.33583456)
\lineto(44.89330404,341.66583456)
\curveto(44.88330336,341.70583184)(44.87830337,341.7408318)(44.87830404,341.77083456)
\curveto(44.87830337,341.81083173)(44.86830338,341.85583169)(44.84830404,341.90583456)
\curveto(44.82830342,341.97583157)(44.81830343,342.0458315)(44.81830404,342.11583456)
\curveto(44.81830343,342.19583135)(44.80830344,342.27083127)(44.78830404,342.34083456)
\lineto(44.78830404,342.59583456)
\curveto(44.76830348,342.6458309)(44.75830349,342.70083084)(44.75830404,342.76083456)
\curveto(44.75830349,342.83083071)(44.76830348,342.89083065)(44.78830404,342.94083456)
\curveto(44.79830345,342.99083055)(44.79830345,343.03583051)(44.78830404,343.07583456)
\curveto(44.77830347,343.11583043)(44.77830347,343.15583039)(44.78830404,343.19583456)
\curveto(44.80830344,343.26583028)(44.81330343,343.33083021)(44.80330404,343.39083456)
\curveto(44.80330344,343.45083009)(44.81330343,343.51083003)(44.83330404,343.57083456)
\curveto(44.88330336,343.75082979)(44.92330332,343.92082962)(44.95330404,344.08083456)
\curveto(44.98330326,344.25082929)(45.02830322,344.41582913)(45.08830404,344.57583456)
\curveto(45.30830294,345.08582846)(45.58330266,345.51082803)(45.91330404,345.85083456)
\curveto(46.25330199,346.19082735)(46.68330156,346.46582708)(47.20330404,346.67583456)
\curveto(47.3433009,346.73582681)(47.48830076,346.77582677)(47.63830404,346.79583456)
\curveto(47.78830046,346.82582672)(47.9433003,346.86082668)(48.10330404,346.90083456)
\curveto(48.18330006,346.91082663)(48.25829999,346.91582663)(48.32830404,346.91583456)
\curveto(48.39829985,346.91582663)(48.47329977,346.92082662)(48.55330404,346.93083456)
}
}
{
\newrgbcolor{curcolor}{0 0 0}
\pscustom[linestyle=none,fillstyle=solid,fillcolor=curcolor]
{
\newpath
\moveto(60.64658529,339.58083456)
\curveto(60.66657744,339.47083407)(60.67657743,339.36083418)(60.67658529,339.25083456)
\curveto(60.68657742,339.1408344)(60.63657747,339.06583448)(60.52658529,339.02583456)
\curveto(60.46657764,338.99583455)(60.39657771,338.98083456)(60.31658529,338.98083456)
\lineto(60.07658529,338.98083456)
\lineto(59.26658529,338.98083456)
\lineto(58.99658529,338.98083456)
\curveto(58.91657919,338.99083455)(58.85157926,339.01583453)(58.80158529,339.05583456)
\curveto(58.73157938,339.09583445)(58.67657943,339.15083439)(58.63658529,339.22083456)
\curveto(58.6065795,339.30083424)(58.56157955,339.36583418)(58.50158529,339.41583456)
\curveto(58.48157963,339.43583411)(58.45657965,339.45083409)(58.42658529,339.46083456)
\curveto(58.39657971,339.48083406)(58.35657975,339.48583406)(58.30658529,339.47583456)
\curveto(58.25657985,339.45583409)(58.2065799,339.43083411)(58.15658529,339.40083456)
\curveto(58.11657999,339.37083417)(58.07158004,339.3458342)(58.02158529,339.32583456)
\curveto(57.97158014,339.28583426)(57.91658019,339.25083429)(57.85658529,339.22083456)
\lineto(57.67658529,339.13083456)
\curveto(57.54658056,339.07083447)(57.4115807,339.02083452)(57.27158529,338.98083456)
\curveto(57.13158098,338.95083459)(56.98658112,338.91583463)(56.83658529,338.87583456)
\curveto(56.76658134,338.85583469)(56.69658141,338.8458347)(56.62658529,338.84583456)
\curveto(56.56658154,338.83583471)(56.50158161,338.82583472)(56.43158529,338.81583456)
\lineto(56.34158529,338.81583456)
\curveto(56.3115818,338.80583474)(56.28158183,338.80083474)(56.25158529,338.80083456)
\lineto(56.08658529,338.80083456)
\curveto(55.98658212,338.78083476)(55.88658222,338.78083476)(55.78658529,338.80083456)
\lineto(55.65158529,338.80083456)
\curveto(55.58158253,338.82083472)(55.5115826,338.83083471)(55.44158529,338.83083456)
\curveto(55.38158273,338.82083472)(55.32158279,338.82583472)(55.26158529,338.84583456)
\curveto(55.16158295,338.86583468)(55.06658304,338.88583466)(54.97658529,338.90583456)
\curveto(54.88658322,338.91583463)(54.80158331,338.9408346)(54.72158529,338.98083456)
\curveto(54.43158368,339.09083445)(54.18158393,339.23083431)(53.97158529,339.40083456)
\curveto(53.77158434,339.58083396)(53.6115845,339.81583373)(53.49158529,340.10583456)
\curveto(53.46158465,340.17583337)(53.43158468,340.25083329)(53.40158529,340.33083456)
\curveto(53.38158473,340.41083313)(53.36158475,340.49583305)(53.34158529,340.58583456)
\curveto(53.32158479,340.63583291)(53.3115848,340.68583286)(53.31158529,340.73583456)
\curveto(53.32158479,340.78583276)(53.32158479,340.83583271)(53.31158529,340.88583456)
\curveto(53.30158481,340.91583263)(53.29158482,340.97583257)(53.28158529,341.06583456)
\curveto(53.28158483,341.16583238)(53.28658482,341.23583231)(53.29658529,341.27583456)
\curveto(53.31658479,341.37583217)(53.32658478,341.46083208)(53.32658529,341.53083456)
\lineto(53.41658529,341.86083456)
\curveto(53.44658466,341.98083156)(53.48658462,342.08583146)(53.53658529,342.17583456)
\curveto(53.7065844,342.46583108)(53.90158421,342.68583086)(54.12158529,342.83583456)
\curveto(54.34158377,342.98583056)(54.62158349,343.11583043)(54.96158529,343.22583456)
\curveto(55.09158302,343.27583027)(55.22658288,343.31083023)(55.36658529,343.33083456)
\curveto(55.5065826,343.35083019)(55.64658246,343.37583017)(55.78658529,343.40583456)
\curveto(55.86658224,343.42583012)(55.95158216,343.43583011)(56.04158529,343.43583456)
\curveto(56.13158198,343.4458301)(56.22158189,343.46083008)(56.31158529,343.48083456)
\curveto(56.38158173,343.50083004)(56.45158166,343.50583004)(56.52158529,343.49583456)
\curveto(56.59158152,343.49583005)(56.66658144,343.50583004)(56.74658529,343.52583456)
\curveto(56.81658129,343.54583)(56.88658122,343.55582999)(56.95658529,343.55583456)
\curveto(57.02658108,343.55582999)(57.10158101,343.56582998)(57.18158529,343.58583456)
\curveto(57.39158072,343.63582991)(57.58158053,343.67582987)(57.75158529,343.70583456)
\curveto(57.93158018,343.7458298)(58.09158002,343.83582971)(58.23158529,343.97583456)
\curveto(58.32157979,344.06582948)(58.38157973,344.16582938)(58.41158529,344.27583456)
\curveto(58.42157969,344.30582924)(58.42157969,344.33082921)(58.41158529,344.35083456)
\curveto(58.4115797,344.37082917)(58.41657969,344.39082915)(58.42658529,344.41083456)
\curveto(58.43657967,344.43082911)(58.44157967,344.46082908)(58.44158529,344.50083456)
\lineto(58.44158529,344.59083456)
\lineto(58.41158529,344.71083456)
\curveto(58.4115797,344.75082879)(58.4065797,344.78582876)(58.39658529,344.81583456)
\curveto(58.29657981,345.11582843)(58.08658002,345.32082822)(57.76658529,345.43083456)
\curveto(57.67658043,345.46082808)(57.56658054,345.48082806)(57.43658529,345.49083456)
\curveto(57.31658079,345.51082803)(57.19158092,345.51582803)(57.06158529,345.50583456)
\curveto(56.93158118,345.50582804)(56.8065813,345.49582805)(56.68658529,345.47583456)
\curveto(56.56658154,345.45582809)(56.46158165,345.43082811)(56.37158529,345.40083456)
\curveto(56.3115818,345.38082816)(56.25158186,345.35082819)(56.19158529,345.31083456)
\curveto(56.14158197,345.28082826)(56.09158202,345.2458283)(56.04158529,345.20583456)
\curveto(55.99158212,345.16582838)(55.93658217,345.11082843)(55.87658529,345.04083456)
\curveto(55.82658228,344.97082857)(55.79158232,344.90582864)(55.77158529,344.84583456)
\curveto(55.72158239,344.7458288)(55.67658243,344.65082889)(55.63658529,344.56083456)
\curveto(55.6065825,344.47082907)(55.53658257,344.41082913)(55.42658529,344.38083456)
\curveto(55.34658276,344.36082918)(55.26158285,344.35082919)(55.17158529,344.35083456)
\lineto(54.90158529,344.35083456)
\lineto(54.33158529,344.35083456)
\curveto(54.28158383,344.35082919)(54.23158388,344.3458292)(54.18158529,344.33583456)
\curveto(54.13158398,344.33582921)(54.08658402,344.3408292)(54.04658529,344.35083456)
\lineto(53.91158529,344.35083456)
\curveto(53.89158422,344.36082918)(53.86658424,344.36582918)(53.83658529,344.36583456)
\curveto(53.8065843,344.36582918)(53.78158433,344.37582917)(53.76158529,344.39583456)
\curveto(53.68158443,344.41582913)(53.62658448,344.48082906)(53.59658529,344.59083456)
\curveto(53.58658452,344.6408289)(53.58658452,344.69082885)(53.59658529,344.74083456)
\curveto(53.6065845,344.79082875)(53.61658449,344.83582871)(53.62658529,344.87583456)
\curveto(53.65658445,344.98582856)(53.68658442,345.08582846)(53.71658529,345.17583456)
\curveto(53.75658435,345.27582827)(53.80158431,345.36582818)(53.85158529,345.44583456)
\lineto(53.94158529,345.59583456)
\lineto(54.03158529,345.74583456)
\curveto(54.111584,345.85582769)(54.2115839,345.96082758)(54.33158529,346.06083456)
\curveto(54.35158376,346.07082747)(54.38158373,346.09582745)(54.42158529,346.13583456)
\curveto(54.47158364,346.17582737)(54.51658359,346.21082733)(54.55658529,346.24083456)
\curveto(54.59658351,346.27082727)(54.64158347,346.30082724)(54.69158529,346.33083456)
\curveto(54.86158325,346.4408271)(55.04158307,346.52582702)(55.23158529,346.58583456)
\curveto(55.42158269,346.65582689)(55.61658249,346.72082682)(55.81658529,346.78083456)
\curveto(55.93658217,346.81082673)(56.06158205,346.83082671)(56.19158529,346.84083456)
\curveto(56.32158179,346.85082669)(56.45158166,346.87082667)(56.58158529,346.90083456)
\curveto(56.62158149,346.91082663)(56.68158143,346.91082663)(56.76158529,346.90083456)
\curveto(56.85158126,346.89082665)(56.9065812,346.89582665)(56.92658529,346.91583456)
\curveto(57.33658077,346.92582662)(57.72658038,346.91082663)(58.09658529,346.87083456)
\curveto(58.47657963,346.83082671)(58.81657929,346.75582679)(59.11658529,346.64583456)
\curveto(59.42657868,346.53582701)(59.69157842,346.38582716)(59.91158529,346.19583456)
\curveto(60.13157798,346.01582753)(60.30157781,345.78082776)(60.42158529,345.49083456)
\curveto(60.49157762,345.32082822)(60.53157758,345.12582842)(60.54158529,344.90583456)
\curveto(60.55157756,344.68582886)(60.55657755,344.46082908)(60.55658529,344.23083456)
\lineto(60.55658529,340.88583456)
\lineto(60.55658529,340.30083456)
\curveto(60.55657755,340.11083343)(60.57657753,339.93583361)(60.61658529,339.77583456)
\curveto(60.62657748,339.7458338)(60.63157748,339.71083383)(60.63158529,339.67083456)
\curveto(60.63157748,339.6408339)(60.63657747,339.61083393)(60.64658529,339.58083456)
\moveto(58.44158529,341.89083456)
\curveto(58.45157966,341.9408316)(58.45657965,341.99583155)(58.45658529,342.05583456)
\curveto(58.45657965,342.12583142)(58.45157966,342.18583136)(58.44158529,342.23583456)
\curveto(58.42157969,342.29583125)(58.4115797,342.35083119)(58.41158529,342.40083456)
\curveto(58.4115797,342.45083109)(58.39157972,342.49083105)(58.35158529,342.52083456)
\curveto(58.30157981,342.56083098)(58.22657988,342.58083096)(58.12658529,342.58083456)
\curveto(58.08658002,342.57083097)(58.05158006,342.56083098)(58.02158529,342.55083456)
\curveto(57.99158012,342.55083099)(57.95658015,342.545831)(57.91658529,342.53583456)
\curveto(57.84658026,342.51583103)(57.77158034,342.50083104)(57.69158529,342.49083456)
\curveto(57.6115805,342.48083106)(57.53158058,342.46583108)(57.45158529,342.44583456)
\curveto(57.42158069,342.43583111)(57.37658073,342.43083111)(57.31658529,342.43083456)
\curveto(57.18658092,342.40083114)(57.05658105,342.38083116)(56.92658529,342.37083456)
\curveto(56.79658131,342.36083118)(56.67158144,342.33583121)(56.55158529,342.29583456)
\curveto(56.47158164,342.27583127)(56.39658171,342.25583129)(56.32658529,342.23583456)
\curveto(56.25658185,342.22583132)(56.18658192,342.20583134)(56.11658529,342.17583456)
\curveto(55.9065822,342.08583146)(55.72658238,341.95083159)(55.57658529,341.77083456)
\curveto(55.43658267,341.59083195)(55.38658272,341.3408322)(55.42658529,341.02083456)
\curveto(55.44658266,340.85083269)(55.50158261,340.71083283)(55.59158529,340.60083456)
\curveto(55.66158245,340.49083305)(55.76658234,340.40083314)(55.90658529,340.33083456)
\curveto(56.04658206,340.27083327)(56.19658191,340.22583332)(56.35658529,340.19583456)
\curveto(56.52658158,340.16583338)(56.70158141,340.15583339)(56.88158529,340.16583456)
\curveto(57.07158104,340.18583336)(57.24658086,340.22083332)(57.40658529,340.27083456)
\curveto(57.66658044,340.35083319)(57.87158024,340.47583307)(58.02158529,340.64583456)
\curveto(58.17157994,340.82583272)(58.28657982,341.0458325)(58.36658529,341.30583456)
\curveto(58.38657972,341.37583217)(58.39657971,341.4458321)(58.39658529,341.51583456)
\curveto(58.4065797,341.59583195)(58.42157969,341.67583187)(58.44158529,341.75583456)
\lineto(58.44158529,341.89083456)
}
}
{
\newrgbcolor{curcolor}{0 0 0}
\pscustom[linestyle=none,fillstyle=solid,fillcolor=curcolor]
{
\newpath
\moveto(69.79986654,339.83583456)
\lineto(69.79986654,339.41583456)
\curveto(69.79985817,339.28583426)(69.7698582,339.18083436)(69.70986654,339.10083456)
\curveto(69.65985831,339.05083449)(69.59485838,339.01583453)(69.51486654,338.99583456)
\curveto(69.43485854,338.98583456)(69.34485863,338.98083456)(69.24486654,338.98083456)
\lineto(68.41986654,338.98083456)
\lineto(68.13486654,338.98083456)
\curveto(68.05485992,338.99083455)(67.98985998,339.01583453)(67.93986654,339.05583456)
\curveto(67.8698601,339.10583444)(67.82986014,339.17083437)(67.81986654,339.25083456)
\curveto(67.80986016,339.33083421)(67.78986018,339.41083413)(67.75986654,339.49083456)
\curveto(67.73986023,339.51083403)(67.71986025,339.52583402)(67.69986654,339.53583456)
\curveto(67.68986028,339.55583399)(67.6748603,339.57583397)(67.65486654,339.59583456)
\curveto(67.54486043,339.59583395)(67.46486051,339.57083397)(67.41486654,339.52083456)
\lineto(67.26486654,339.37083456)
\curveto(67.19486078,339.32083422)(67.12986084,339.27583427)(67.06986654,339.23583456)
\curveto(67.00986096,339.20583434)(66.94486103,339.16583438)(66.87486654,339.11583456)
\curveto(66.83486114,339.09583445)(66.78986118,339.07583447)(66.73986654,339.05583456)
\curveto(66.69986127,339.03583451)(66.65486132,339.01583453)(66.60486654,338.99583456)
\curveto(66.46486151,338.9458346)(66.31486166,338.90083464)(66.15486654,338.86083456)
\curveto(66.10486187,338.8408347)(66.05986191,338.83083471)(66.01986654,338.83083456)
\curveto(65.97986199,338.83083471)(65.93986203,338.82583472)(65.89986654,338.81583456)
\lineto(65.76486654,338.81583456)
\curveto(65.73486224,338.80583474)(65.69486228,338.80083474)(65.64486654,338.80083456)
\lineto(65.50986654,338.80083456)
\curveto(65.44986252,338.78083476)(65.35986261,338.77583477)(65.23986654,338.78583456)
\curveto(65.11986285,338.78583476)(65.03486294,338.79583475)(64.98486654,338.81583456)
\curveto(64.91486306,338.83583471)(64.84986312,338.8458347)(64.78986654,338.84583456)
\curveto(64.73986323,338.83583471)(64.68486329,338.8408347)(64.62486654,338.86083456)
\lineto(64.26486654,338.98083456)
\curveto(64.15486382,339.01083453)(64.04486393,339.05083449)(63.93486654,339.10083456)
\curveto(63.58486439,339.25083429)(63.2698647,339.48083406)(62.98986654,339.79083456)
\curveto(62.71986525,340.11083343)(62.50486547,340.4458331)(62.34486654,340.79583456)
\curveto(62.29486568,340.90583264)(62.25486572,341.01083253)(62.22486654,341.11083456)
\curveto(62.19486578,341.22083232)(62.15986581,341.33083221)(62.11986654,341.44083456)
\curveto(62.10986586,341.48083206)(62.10486587,341.51583203)(62.10486654,341.54583456)
\curveto(62.10486587,341.58583196)(62.09486588,341.63083191)(62.07486654,341.68083456)
\curveto(62.05486592,341.76083178)(62.03486594,341.8458317)(62.01486654,341.93583456)
\curveto(62.00486597,342.03583151)(61.98986598,342.13583141)(61.96986654,342.23583456)
\curveto(61.95986601,342.26583128)(61.95486602,342.30083124)(61.95486654,342.34083456)
\curveto(61.96486601,342.38083116)(61.96486601,342.41583113)(61.95486654,342.44583456)
\lineto(61.95486654,342.58083456)
\curveto(61.95486602,342.63083091)(61.94986602,342.68083086)(61.93986654,342.73083456)
\curveto(61.92986604,342.78083076)(61.92486605,342.83583071)(61.92486654,342.89583456)
\curveto(61.92486605,342.96583058)(61.92986604,343.02083052)(61.93986654,343.06083456)
\curveto(61.94986602,343.11083043)(61.95486602,343.15583039)(61.95486654,343.19583456)
\lineto(61.95486654,343.34583456)
\curveto(61.96486601,343.39583015)(61.96486601,343.4408301)(61.95486654,343.48083456)
\curveto(61.95486602,343.53083001)(61.96486601,343.58082996)(61.98486654,343.63083456)
\curveto(62.00486597,343.7408298)(62.01986595,343.8458297)(62.02986654,343.94583456)
\curveto(62.04986592,344.0458295)(62.0748659,344.1458294)(62.10486654,344.24583456)
\curveto(62.14486583,344.36582918)(62.17986579,344.48082906)(62.20986654,344.59083456)
\curveto(62.23986573,344.70082884)(62.27986569,344.81082873)(62.32986654,344.92083456)
\curveto(62.4698655,345.22082832)(62.64486533,345.50582804)(62.85486654,345.77583456)
\curveto(62.8748651,345.80582774)(62.89986507,345.83082771)(62.92986654,345.85083456)
\curveto(62.969865,345.88082766)(62.99986497,345.91082763)(63.01986654,345.94083456)
\curveto(63.05986491,345.99082755)(63.09986487,346.03582751)(63.13986654,346.07583456)
\curveto(63.17986479,346.11582743)(63.22486475,346.15582739)(63.27486654,346.19583456)
\curveto(63.31486466,346.21582733)(63.34986462,346.2408273)(63.37986654,346.27083456)
\curveto(63.40986456,346.31082723)(63.44486453,346.3408272)(63.48486654,346.36083456)
\curveto(63.73486424,346.53082701)(64.02486395,346.67082687)(64.35486654,346.78083456)
\curveto(64.42486355,346.80082674)(64.49486348,346.81582673)(64.56486654,346.82583456)
\curveto(64.64486333,346.83582671)(64.72486325,346.85082669)(64.80486654,346.87083456)
\curveto(64.8748631,346.89082665)(64.96486301,346.90082664)(65.07486654,346.90083456)
\curveto(65.18486279,346.91082663)(65.29486268,346.91582663)(65.40486654,346.91583456)
\curveto(65.51486246,346.91582663)(65.61986235,346.91082663)(65.71986654,346.90083456)
\curveto(65.82986214,346.89082665)(65.91986205,346.87582667)(65.98986654,346.85583456)
\curveto(66.13986183,346.80582674)(66.28486169,346.76082678)(66.42486654,346.72083456)
\curveto(66.56486141,346.68082686)(66.69486128,346.62582692)(66.81486654,346.55583456)
\curveto(66.88486109,346.50582704)(66.94986102,346.45582709)(67.00986654,346.40583456)
\curveto(67.0698609,346.36582718)(67.13486084,346.32082722)(67.20486654,346.27083456)
\curveto(67.24486073,346.2408273)(67.29986067,346.20082734)(67.36986654,346.15083456)
\curveto(67.44986052,346.10082744)(67.52486045,346.10082744)(67.59486654,346.15083456)
\curveto(67.63486034,346.17082737)(67.65486032,346.20582734)(67.65486654,346.25583456)
\curveto(67.65486032,346.30582724)(67.66486031,346.35582719)(67.68486654,346.40583456)
\lineto(67.68486654,346.55583456)
\curveto(67.69486028,346.58582696)(67.69986027,346.62082692)(67.69986654,346.66083456)
\lineto(67.69986654,346.78083456)
\lineto(67.69986654,348.82083456)
\curveto(67.69986027,348.93082461)(67.69486028,349.05082449)(67.68486654,349.18083456)
\curveto(67.68486029,349.32082422)(67.70986026,349.42582412)(67.75986654,349.49583456)
\curveto(67.79986017,349.57582397)(67.8748601,349.62582392)(67.98486654,349.64583456)
\curveto(68.00485997,349.65582389)(68.02485995,349.65582389)(68.04486654,349.64583456)
\curveto(68.06485991,349.6458239)(68.08485989,349.65082389)(68.10486654,349.66083456)
\lineto(69.16986654,349.66083456)
\curveto(69.28985868,349.66082388)(69.39985857,349.65582389)(69.49986654,349.64583456)
\curveto(69.59985837,349.63582391)(69.6748583,349.59582395)(69.72486654,349.52583456)
\curveto(69.7748582,349.4458241)(69.79985817,349.3408242)(69.79986654,349.21083456)
\lineto(69.79986654,348.85083456)
\lineto(69.79986654,339.83583456)
\moveto(67.75986654,342.77583456)
\curveto(67.7698602,342.81583073)(67.7698602,342.85583069)(67.75986654,342.89583456)
\lineto(67.75986654,343.03083456)
\curveto(67.75986021,343.13083041)(67.75486022,343.23083031)(67.74486654,343.33083456)
\curveto(67.73486024,343.43083011)(67.71986025,343.52083002)(67.69986654,343.60083456)
\curveto(67.67986029,343.71082983)(67.65986031,343.81082973)(67.63986654,343.90083456)
\curveto(67.62986034,343.99082955)(67.60486037,344.07582947)(67.56486654,344.15583456)
\curveto(67.42486055,344.51582903)(67.21986075,344.80082874)(66.94986654,345.01083456)
\curveto(66.68986128,345.22082832)(66.30986166,345.32582822)(65.80986654,345.32583456)
\curveto(65.74986222,345.32582822)(65.6698623,345.31582823)(65.56986654,345.29583456)
\curveto(65.48986248,345.27582827)(65.41486256,345.25582829)(65.34486654,345.23583456)
\curveto(65.28486269,345.22582832)(65.22486275,345.20582834)(65.16486654,345.17583456)
\curveto(64.89486308,345.06582848)(64.68486329,344.89582865)(64.53486654,344.66583456)
\curveto(64.38486359,344.43582911)(64.26486371,344.17582937)(64.17486654,343.88583456)
\curveto(64.14486383,343.78582976)(64.12486385,343.68582986)(64.11486654,343.58583456)
\curveto(64.10486387,343.48583006)(64.08486389,343.38083016)(64.05486654,343.27083456)
\lineto(64.05486654,343.06083456)
\curveto(64.03486394,342.97083057)(64.02986394,342.8458307)(64.03986654,342.68583456)
\curveto(64.04986392,342.53583101)(64.06486391,342.42583112)(64.08486654,342.35583456)
\lineto(64.08486654,342.26583456)
\curveto(64.09486388,342.2458313)(64.09986387,342.22583132)(64.09986654,342.20583456)
\curveto(64.11986385,342.12583142)(64.13486384,342.05083149)(64.14486654,341.98083456)
\curveto(64.16486381,341.91083163)(64.18486379,341.83583171)(64.20486654,341.75583456)
\curveto(64.3748636,341.23583231)(64.66486331,340.85083269)(65.07486654,340.60083456)
\curveto(65.20486277,340.51083303)(65.38486259,340.4408331)(65.61486654,340.39083456)
\curveto(65.65486232,340.38083316)(65.71486226,340.37583317)(65.79486654,340.37583456)
\curveto(65.82486215,340.36583318)(65.8698621,340.35583319)(65.92986654,340.34583456)
\curveto(65.99986197,340.3458332)(66.05486192,340.35083319)(66.09486654,340.36083456)
\curveto(66.1748618,340.38083316)(66.25486172,340.39583315)(66.33486654,340.40583456)
\curveto(66.41486156,340.41583313)(66.49486148,340.43583311)(66.57486654,340.46583456)
\curveto(66.82486115,340.57583297)(67.02486095,340.71583283)(67.17486654,340.88583456)
\curveto(67.32486065,341.05583249)(67.45486052,341.27083227)(67.56486654,341.53083456)
\curveto(67.60486037,341.62083192)(67.63486034,341.71083183)(67.65486654,341.80083456)
\curveto(67.6748603,341.90083164)(67.69486028,342.00583154)(67.71486654,342.11583456)
\curveto(67.72486025,342.16583138)(67.72486025,342.21083133)(67.71486654,342.25083456)
\curveto(67.71486026,342.30083124)(67.72486025,342.35083119)(67.74486654,342.40083456)
\curveto(67.75486022,342.43083111)(67.75986021,342.46583108)(67.75986654,342.50583456)
\lineto(67.75986654,342.64083456)
\lineto(67.75986654,342.77583456)
}
}
{
\newrgbcolor{curcolor}{0 0 0}
\pscustom[linestyle=none,fillstyle=solid,fillcolor=curcolor]
{
\newpath
\moveto(79.14978842,343.16583456)
\curveto(79.16977985,343.10583044)(79.17977984,343.02083052)(79.17978842,342.91083456)
\curveto(79.17977984,342.80083074)(79.16977985,342.71583083)(79.14978842,342.65583456)
\lineto(79.14978842,342.50583456)
\curveto(79.12977989,342.42583112)(79.1197799,342.3458312)(79.11978842,342.26583456)
\curveto(79.12977989,342.18583136)(79.12477989,342.10583144)(79.10478842,342.02583456)
\curveto(79.08477993,341.95583159)(79.06977995,341.89083165)(79.05978842,341.83083456)
\curveto(79.04977997,341.77083177)(79.03977998,341.70583184)(79.02978842,341.63583456)
\curveto(78.98978003,341.52583202)(78.95478006,341.41083213)(78.92478842,341.29083456)
\curveto(78.89478012,341.18083236)(78.85478016,341.07583247)(78.80478842,340.97583456)
\curveto(78.59478042,340.49583305)(78.3197807,340.10583344)(77.97978842,339.80583456)
\curveto(77.63978138,339.50583404)(77.22978179,339.25583429)(76.74978842,339.05583456)
\curveto(76.62978239,339.00583454)(76.50478251,338.97083457)(76.37478842,338.95083456)
\curveto(76.25478276,338.92083462)(76.12978289,338.89083465)(75.99978842,338.86083456)
\curveto(75.94978307,338.8408347)(75.89478312,338.83083471)(75.83478842,338.83083456)
\curveto(75.77478324,338.83083471)(75.7197833,338.82583472)(75.66978842,338.81583456)
\lineto(75.56478842,338.81583456)
\curveto(75.53478348,338.80583474)(75.50478351,338.80083474)(75.47478842,338.80083456)
\curveto(75.42478359,338.79083475)(75.34478367,338.78583476)(75.23478842,338.78583456)
\curveto(75.12478389,338.77583477)(75.03978398,338.78083476)(74.97978842,338.80083456)
\lineto(74.82978842,338.80083456)
\curveto(74.77978424,338.81083473)(74.72478429,338.81583473)(74.66478842,338.81583456)
\curveto(74.6147844,338.80583474)(74.56478445,338.81083473)(74.51478842,338.83083456)
\curveto(74.47478454,338.8408347)(74.43478458,338.8458347)(74.39478842,338.84583456)
\curveto(74.36478465,338.8458347)(74.32478469,338.85083469)(74.27478842,338.86083456)
\curveto(74.17478484,338.89083465)(74.07478494,338.91583463)(73.97478842,338.93583456)
\curveto(73.87478514,338.95583459)(73.77978524,338.98583456)(73.68978842,339.02583456)
\curveto(73.56978545,339.06583448)(73.45478556,339.10583444)(73.34478842,339.14583456)
\curveto(73.24478577,339.18583436)(73.13978588,339.23583431)(73.02978842,339.29583456)
\curveto(72.67978634,339.50583404)(72.37978664,339.75083379)(72.12978842,340.03083456)
\curveto(71.87978714,340.31083323)(71.66978735,340.6458329)(71.49978842,341.03583456)
\curveto(71.44978757,341.12583242)(71.40978761,341.22083232)(71.37978842,341.32083456)
\curveto(71.35978766,341.42083212)(71.33478768,341.52583202)(71.30478842,341.63583456)
\curveto(71.28478773,341.68583186)(71.27478774,341.73083181)(71.27478842,341.77083456)
\curveto(71.27478774,341.81083173)(71.26478775,341.85583169)(71.24478842,341.90583456)
\curveto(71.22478779,341.98583156)(71.2147878,342.06583148)(71.21478842,342.14583456)
\curveto(71.2147878,342.23583131)(71.20478781,342.32083122)(71.18478842,342.40083456)
\curveto(71.17478784,342.45083109)(71.16978785,342.49583105)(71.16978842,342.53583456)
\lineto(71.16978842,342.67083456)
\curveto(71.14978787,342.73083081)(71.13978788,342.81583073)(71.13978842,342.92583456)
\curveto(71.14978787,343.03583051)(71.16478785,343.12083042)(71.18478842,343.18083456)
\lineto(71.18478842,343.28583456)
\curveto(71.19478782,343.33583021)(71.19478782,343.38583016)(71.18478842,343.43583456)
\curveto(71.18478783,343.49583005)(71.19478782,343.55082999)(71.21478842,343.60083456)
\curveto(71.22478779,343.65082989)(71.22978779,343.69582985)(71.22978842,343.73583456)
\curveto(71.22978779,343.78582976)(71.23978778,343.83582971)(71.25978842,343.88583456)
\curveto(71.29978772,344.01582953)(71.33478768,344.1408294)(71.36478842,344.26083456)
\curveto(71.39478762,344.39082915)(71.43478758,344.51582903)(71.48478842,344.63583456)
\curveto(71.66478735,345.0458285)(71.87978714,345.38582816)(72.12978842,345.65583456)
\curveto(72.37978664,345.93582761)(72.68478633,346.19082735)(73.04478842,346.42083456)
\curveto(73.14478587,346.47082707)(73.24978577,346.51582703)(73.35978842,346.55583456)
\curveto(73.46978555,346.59582695)(73.57978544,346.6408269)(73.68978842,346.69083456)
\curveto(73.8197852,346.7408268)(73.95478506,346.77582677)(74.09478842,346.79583456)
\curveto(74.23478478,346.81582673)(74.37978464,346.8458267)(74.52978842,346.88583456)
\curveto(74.60978441,346.89582665)(74.68478433,346.90082664)(74.75478842,346.90083456)
\curveto(74.82478419,346.90082664)(74.89478412,346.90582664)(74.96478842,346.91583456)
\curveto(75.54478347,346.92582662)(76.04478297,346.86582668)(76.46478842,346.73583456)
\curveto(76.89478212,346.60582694)(77.27478174,346.42582712)(77.60478842,346.19583456)
\curveto(77.7147813,346.11582743)(77.82478119,346.02582752)(77.93478842,345.92583456)
\curveto(78.05478096,345.83582771)(78.15478086,345.73582781)(78.23478842,345.62583456)
\curveto(78.3147807,345.52582802)(78.38478063,345.42582812)(78.44478842,345.32583456)
\curveto(78.5147805,345.22582832)(78.58478043,345.12082842)(78.65478842,345.01083456)
\curveto(78.72478029,344.90082864)(78.77978024,344.78082876)(78.81978842,344.65083456)
\curveto(78.85978016,344.53082901)(78.90478011,344.40082914)(78.95478842,344.26083456)
\curveto(78.98478003,344.18082936)(79.00978001,344.09582945)(79.02978842,344.00583456)
\lineto(79.08978842,343.73583456)
\curveto(79.09977992,343.69582985)(79.10477991,343.65582989)(79.10478842,343.61583456)
\curveto(79.10477991,343.57582997)(79.10977991,343.53583001)(79.11978842,343.49583456)
\curveto(79.13977988,343.4458301)(79.14477987,343.39083015)(79.13478842,343.33083456)
\curveto(79.12477989,343.27083027)(79.12977989,343.21583033)(79.14978842,343.16583456)
\moveto(77.04978842,342.62583456)
\curveto(77.05978196,342.67583087)(77.06478195,342.7458308)(77.06478842,342.83583456)
\curveto(77.06478195,342.93583061)(77.05978196,343.01083053)(77.04978842,343.06083456)
\lineto(77.04978842,343.18083456)
\curveto(77.02978199,343.23083031)(77.019782,343.28583026)(77.01978842,343.34583456)
\curveto(77.019782,343.40583014)(77.014782,343.46083008)(77.00478842,343.51083456)
\curveto(77.00478201,343.55082999)(76.99978202,343.58082996)(76.98978842,343.60083456)
\lineto(76.92978842,343.84083456)
\curveto(76.9197821,343.93082961)(76.89978212,344.01582953)(76.86978842,344.09583456)
\curveto(76.75978226,344.35582919)(76.62978239,344.57582897)(76.47978842,344.75583456)
\curveto(76.32978269,344.9458286)(76.12978289,345.09582845)(75.87978842,345.20583456)
\curveto(75.8197832,345.22582832)(75.75978326,345.2408283)(75.69978842,345.25083456)
\curveto(75.63978338,345.27082827)(75.57478344,345.29082825)(75.50478842,345.31083456)
\curveto(75.42478359,345.33082821)(75.33978368,345.33582821)(75.24978842,345.32583456)
\lineto(74.97978842,345.32583456)
\curveto(74.94978407,345.30582824)(74.9147841,345.29582825)(74.87478842,345.29583456)
\curveto(74.83478418,345.30582824)(74.79978422,345.30582824)(74.76978842,345.29583456)
\lineto(74.55978842,345.23583456)
\curveto(74.49978452,345.22582832)(74.44478457,345.20582834)(74.39478842,345.17583456)
\curveto(74.14478487,345.06582848)(73.93978508,344.90582864)(73.77978842,344.69583456)
\curveto(73.62978539,344.49582905)(73.50978551,344.26082928)(73.41978842,343.99083456)
\curveto(73.38978563,343.89082965)(73.36478565,343.78582976)(73.34478842,343.67583456)
\curveto(73.33478568,343.56582998)(73.3197857,343.45583009)(73.29978842,343.34583456)
\curveto(73.28978573,343.29583025)(73.28478573,343.2458303)(73.28478842,343.19583456)
\lineto(73.28478842,343.04583456)
\curveto(73.26478575,342.97583057)(73.25478576,342.87083067)(73.25478842,342.73083456)
\curveto(73.26478575,342.59083095)(73.27978574,342.48583106)(73.29978842,342.41583456)
\lineto(73.29978842,342.28083456)
\curveto(73.3197857,342.20083134)(73.33478568,342.12083142)(73.34478842,342.04083456)
\curveto(73.35478566,341.97083157)(73.36978565,341.89583165)(73.38978842,341.81583456)
\curveto(73.48978553,341.51583203)(73.59478542,341.27083227)(73.70478842,341.08083456)
\curveto(73.82478519,340.90083264)(74.00978501,340.73583281)(74.25978842,340.58583456)
\curveto(74.32978469,340.53583301)(74.40478461,340.49583305)(74.48478842,340.46583456)
\curveto(74.57478444,340.43583311)(74.66478435,340.41083313)(74.75478842,340.39083456)
\curveto(74.79478422,340.38083316)(74.82978419,340.37583317)(74.85978842,340.37583456)
\curveto(74.88978413,340.38583316)(74.92478409,340.38583316)(74.96478842,340.37583456)
\lineto(75.08478842,340.34583456)
\curveto(75.13478388,340.3458332)(75.17978384,340.35083319)(75.21978842,340.36083456)
\lineto(75.33978842,340.36083456)
\curveto(75.4197836,340.38083316)(75.49978352,340.39583315)(75.57978842,340.40583456)
\curveto(75.65978336,340.41583313)(75.73478328,340.43583311)(75.80478842,340.46583456)
\curveto(76.06478295,340.56583298)(76.27478274,340.70083284)(76.43478842,340.87083456)
\curveto(76.59478242,341.0408325)(76.72978229,341.25083229)(76.83978842,341.50083456)
\curveto(76.87978214,341.60083194)(76.90978211,341.70083184)(76.92978842,341.80083456)
\curveto(76.94978207,341.90083164)(76.97478204,342.00583154)(77.00478842,342.11583456)
\curveto(77.014782,342.15583139)(77.019782,342.19083135)(77.01978842,342.22083456)
\curveto(77.019782,342.26083128)(77.02478199,342.30083124)(77.03478842,342.34083456)
\lineto(77.03478842,342.47583456)
\curveto(77.03478198,342.52583102)(77.03978198,342.57583097)(77.04978842,342.62583456)
}
}
{
\newrgbcolor{curcolor}{0 0 0}
\pscustom[linestyle=none,fillstyle=solid,fillcolor=curcolor]
{
\newpath
\moveto(84.97471029,346.91583456)
\curveto(85.08470498,346.91582663)(85.17970488,346.90582664)(85.25971029,346.88583456)
\curveto(85.34970471,346.86582668)(85.41970464,346.82082672)(85.46971029,346.75083456)
\curveto(85.52970453,346.67082687)(85.5597045,346.53082701)(85.55971029,346.33083456)
\lineto(85.55971029,345.82083456)
\lineto(85.55971029,345.44583456)
\curveto(85.56970449,345.30582824)(85.55470451,345.19582835)(85.51471029,345.11583456)
\curveto(85.47470459,345.0458285)(85.41470465,345.00082854)(85.33471029,344.98083456)
\curveto(85.2647048,344.96082858)(85.17970488,344.95082859)(85.07971029,344.95083456)
\curveto(84.98970507,344.95082859)(84.88970517,344.95582859)(84.77971029,344.96583456)
\curveto(84.67970538,344.97582857)(84.58470548,344.97082857)(84.49471029,344.95083456)
\curveto(84.42470564,344.93082861)(84.35470571,344.91582863)(84.28471029,344.90583456)
\curveto(84.21470585,344.90582864)(84.14970591,344.89582865)(84.08971029,344.87583456)
\curveto(83.92970613,344.82582872)(83.76970629,344.75082879)(83.60971029,344.65083456)
\curveto(83.44970661,344.56082898)(83.32470674,344.45582909)(83.23471029,344.33583456)
\curveto(83.18470688,344.25582929)(83.12970693,344.17082937)(83.06971029,344.08083456)
\curveto(83.01970704,344.00082954)(82.96970709,343.91582963)(82.91971029,343.82583456)
\curveto(82.88970717,343.7458298)(82.8597072,343.66082988)(82.82971029,343.57083456)
\lineto(82.76971029,343.33083456)
\curveto(82.74970731,343.26083028)(82.73970732,343.18583036)(82.73971029,343.10583456)
\curveto(82.73970732,343.03583051)(82.72970733,342.96583058)(82.70971029,342.89583456)
\curveto(82.69970736,342.85583069)(82.69470737,342.81583073)(82.69471029,342.77583456)
\curveto(82.70470736,342.7458308)(82.70470736,342.71583083)(82.69471029,342.68583456)
\lineto(82.69471029,342.44583456)
\curveto(82.67470739,342.37583117)(82.66970739,342.29583125)(82.67971029,342.20583456)
\curveto(82.68970737,342.12583142)(82.69470737,342.0458315)(82.69471029,341.96583456)
\lineto(82.69471029,341.00583456)
\lineto(82.69471029,339.73083456)
\curveto(82.69470737,339.60083394)(82.68970737,339.48083406)(82.67971029,339.37083456)
\curveto(82.66970739,339.26083428)(82.63970742,339.17083437)(82.58971029,339.10083456)
\curveto(82.56970749,339.07083447)(82.53470753,339.0458345)(82.48471029,339.02583456)
\curveto(82.44470762,339.01583453)(82.39970766,339.00583454)(82.34971029,338.99583456)
\lineto(82.27471029,338.99583456)
\curveto(82.22470784,338.98583456)(82.16970789,338.98083456)(82.10971029,338.98083456)
\lineto(81.94471029,338.98083456)
\lineto(81.29971029,338.98083456)
\curveto(81.23970882,338.99083455)(81.17470889,338.99583455)(81.10471029,338.99583456)
\lineto(80.90971029,338.99583456)
\curveto(80.8597092,339.01583453)(80.80970925,339.03083451)(80.75971029,339.04083456)
\curveto(80.70970935,339.06083448)(80.67470939,339.09583445)(80.65471029,339.14583456)
\curveto(80.61470945,339.19583435)(80.58970947,339.26583428)(80.57971029,339.35583456)
\lineto(80.57971029,339.65583456)
\lineto(80.57971029,340.67583456)
\lineto(80.57971029,344.90583456)
\lineto(80.57971029,346.01583456)
\lineto(80.57971029,346.30083456)
\curveto(80.57970948,346.40082714)(80.59970946,346.48082706)(80.63971029,346.54083456)
\curveto(80.68970937,346.62082692)(80.7647093,346.67082687)(80.86471029,346.69083456)
\curveto(80.9647091,346.71082683)(81.08470898,346.72082682)(81.22471029,346.72083456)
\lineto(81.98971029,346.72083456)
\curveto(82.10970795,346.72082682)(82.21470785,346.71082683)(82.30471029,346.69083456)
\curveto(82.39470767,346.68082686)(82.4647076,346.63582691)(82.51471029,346.55583456)
\curveto(82.54470752,346.50582704)(82.5597075,346.43582711)(82.55971029,346.34583456)
\lineto(82.58971029,346.07583456)
\curveto(82.59970746,345.99582755)(82.61470745,345.92082762)(82.63471029,345.85083456)
\curveto(82.6647074,345.78082776)(82.71470735,345.7458278)(82.78471029,345.74583456)
\curveto(82.80470726,345.76582778)(82.82470724,345.77582777)(82.84471029,345.77583456)
\curveto(82.8647072,345.77582777)(82.88470718,345.78582776)(82.90471029,345.80583456)
\curveto(82.9647071,345.85582769)(83.01470705,345.91082763)(83.05471029,345.97083456)
\curveto(83.10470696,346.0408275)(83.1647069,346.10082744)(83.23471029,346.15083456)
\curveto(83.27470679,346.18082736)(83.30970675,346.21082733)(83.33971029,346.24083456)
\curveto(83.36970669,346.28082726)(83.40470666,346.31582723)(83.44471029,346.34583456)
\lineto(83.71471029,346.52583456)
\curveto(83.81470625,346.58582696)(83.91470615,346.6408269)(84.01471029,346.69083456)
\curveto(84.11470595,346.73082681)(84.21470585,346.76582678)(84.31471029,346.79583456)
\lineto(84.64471029,346.88583456)
\curveto(84.67470539,346.89582665)(84.72970533,346.89582665)(84.80971029,346.88583456)
\curveto(84.89970516,346.88582666)(84.95470511,346.89582665)(84.97471029,346.91583456)
}
}
{
\newrgbcolor{curcolor}{0 0 0}
\pscustom[linestyle=none,fillstyle=solid,fillcolor=curcolor]
{
\newpath
\moveto(93.48111654,342.92583456)
\curveto(93.50110838,342.8458307)(93.50110838,342.75583079)(93.48111654,342.65583456)
\curveto(93.46110842,342.55583099)(93.42610845,342.49083105)(93.37611654,342.46083456)
\curveto(93.32610855,342.42083112)(93.25110863,342.39083115)(93.15111654,342.37083456)
\curveto(93.06110882,342.36083118)(92.95610892,342.35083119)(92.83611654,342.34083456)
\lineto(92.49111654,342.34083456)
\curveto(92.3811095,342.35083119)(92.2811096,342.35583119)(92.19111654,342.35583456)
\lineto(88.53111654,342.35583456)
\lineto(88.32111654,342.35583456)
\curveto(88.26111362,342.35583119)(88.20611367,342.3458312)(88.15611654,342.32583456)
\curveto(88.0761138,342.28583126)(88.02611385,342.2458313)(88.00611654,342.20583456)
\curveto(87.98611389,342.18583136)(87.96611391,342.1458314)(87.94611654,342.08583456)
\curveto(87.92611395,342.03583151)(87.92111396,341.98583156)(87.93111654,341.93583456)
\curveto(87.95111393,341.87583167)(87.96111392,341.81583173)(87.96111654,341.75583456)
\curveto(87.97111391,341.70583184)(87.98611389,341.65083189)(88.00611654,341.59083456)
\curveto(88.08611379,341.35083219)(88.1811137,341.15083239)(88.29111654,340.99083456)
\curveto(88.41111347,340.8408327)(88.57111331,340.70583284)(88.77111654,340.58583456)
\curveto(88.85111303,340.53583301)(88.93111295,340.50083304)(89.01111654,340.48083456)
\curveto(89.10111278,340.47083307)(89.19111269,340.45083309)(89.28111654,340.42083456)
\curveto(89.36111252,340.40083314)(89.47111241,340.38583316)(89.61111654,340.37583456)
\curveto(89.75111213,340.36583318)(89.87111201,340.37083317)(89.97111654,340.39083456)
\lineto(90.10611654,340.39083456)
\curveto(90.20611167,340.41083313)(90.29611158,340.43083311)(90.37611654,340.45083456)
\curveto(90.46611141,340.48083306)(90.55111133,340.51083303)(90.63111654,340.54083456)
\curveto(90.73111115,340.59083295)(90.84111104,340.65583289)(90.96111654,340.73583456)
\curveto(91.09111079,340.81583273)(91.18611069,340.89583265)(91.24611654,340.97583456)
\curveto(91.29611058,341.0458325)(91.34611053,341.11083243)(91.39611654,341.17083456)
\curveto(91.45611042,341.2408323)(91.52611035,341.29083225)(91.60611654,341.32083456)
\curveto(91.70611017,341.37083217)(91.83111005,341.39083215)(91.98111654,341.38083456)
\lineto(92.41611654,341.38083456)
\lineto(92.59611654,341.38083456)
\curveto(92.66610921,341.39083215)(92.72610915,341.38583216)(92.77611654,341.36583456)
\lineto(92.92611654,341.36583456)
\curveto(93.02610885,341.3458322)(93.09610878,341.32083222)(93.13611654,341.29083456)
\curveto(93.1761087,341.27083227)(93.19610868,341.22583232)(93.19611654,341.15583456)
\curveto(93.20610867,341.08583246)(93.20110868,341.02583252)(93.18111654,340.97583456)
\curveto(93.13110875,340.83583271)(93.0761088,340.71083283)(93.01611654,340.60083456)
\curveto(92.95610892,340.49083305)(92.88610899,340.38083316)(92.80611654,340.27083456)
\curveto(92.58610929,339.9408336)(92.33610954,339.67583387)(92.05611654,339.47583456)
\curveto(91.7761101,339.27583427)(91.42611045,339.10583444)(91.00611654,338.96583456)
\curveto(90.89611098,338.92583462)(90.78611109,338.90083464)(90.67611654,338.89083456)
\curveto(90.56611131,338.88083466)(90.45111143,338.86083468)(90.33111654,338.83083456)
\curveto(90.29111159,338.82083472)(90.24611163,338.82083472)(90.19611654,338.83083456)
\curveto(90.15611172,338.83083471)(90.11611176,338.82583472)(90.07611654,338.81583456)
\lineto(89.91111654,338.81583456)
\curveto(89.86111202,338.79583475)(89.80111208,338.79083475)(89.73111654,338.80083456)
\curveto(89.67111221,338.80083474)(89.61611226,338.80583474)(89.56611654,338.81583456)
\curveto(89.48611239,338.82583472)(89.41611246,338.82583472)(89.35611654,338.81583456)
\curveto(89.29611258,338.80583474)(89.23111265,338.81083473)(89.16111654,338.83083456)
\curveto(89.11111277,338.85083469)(89.05611282,338.86083468)(88.99611654,338.86083456)
\curveto(88.93611294,338.86083468)(88.881113,338.87083467)(88.83111654,338.89083456)
\curveto(88.72111316,338.91083463)(88.61111327,338.93583461)(88.50111654,338.96583456)
\curveto(88.39111349,338.98583456)(88.29111359,339.02083452)(88.20111654,339.07083456)
\curveto(88.09111379,339.11083443)(87.98611389,339.1458344)(87.88611654,339.17583456)
\curveto(87.79611408,339.21583433)(87.71111417,339.26083428)(87.63111654,339.31083456)
\curveto(87.31111457,339.51083403)(87.02611485,339.7408338)(86.77611654,340.00083456)
\curveto(86.52611535,340.27083327)(86.32111556,340.58083296)(86.16111654,340.93083456)
\curveto(86.11111577,341.0408325)(86.07111581,341.15083239)(86.04111654,341.26083456)
\curveto(86.01111587,341.38083216)(85.97111591,341.50083204)(85.92111654,341.62083456)
\curveto(85.91111597,341.66083188)(85.90611597,341.69583185)(85.90611654,341.72583456)
\curveto(85.90611597,341.76583178)(85.90111598,341.80583174)(85.89111654,341.84583456)
\curveto(85.85111603,341.96583158)(85.82611605,342.09583145)(85.81611654,342.23583456)
\lineto(85.78611654,342.65583456)
\curveto(85.78611609,342.70583084)(85.7811161,342.76083078)(85.77111654,342.82083456)
\curveto(85.77111611,342.88083066)(85.7761161,342.93583061)(85.78611654,342.98583456)
\lineto(85.78611654,343.16583456)
\lineto(85.83111654,343.52583456)
\curveto(85.87111601,343.69582985)(85.90611597,343.86082968)(85.93611654,344.02083456)
\curveto(85.96611591,344.18082936)(86.01111587,344.33082921)(86.07111654,344.47083456)
\curveto(86.50111538,345.51082803)(87.23111465,346.2458273)(88.26111654,346.67583456)
\curveto(88.40111348,346.73582681)(88.54111334,346.77582677)(88.68111654,346.79583456)
\curveto(88.83111305,346.82582672)(88.98611289,346.86082668)(89.14611654,346.90083456)
\curveto(89.22611265,346.91082663)(89.30111258,346.91582663)(89.37111654,346.91583456)
\curveto(89.44111244,346.91582663)(89.51611236,346.92082662)(89.59611654,346.93083456)
\curveto(90.10611177,346.9408266)(90.54111134,346.88082666)(90.90111654,346.75083456)
\curveto(91.27111061,346.63082691)(91.60111028,346.47082707)(91.89111654,346.27083456)
\curveto(91.9811099,346.21082733)(92.07110981,346.1408274)(92.16111654,346.06083456)
\curveto(92.25110963,345.99082755)(92.33110955,345.91582763)(92.40111654,345.83583456)
\curveto(92.43110945,345.78582776)(92.47110941,345.7458278)(92.52111654,345.71583456)
\curveto(92.60110928,345.60582794)(92.6761092,345.49082805)(92.74611654,345.37083456)
\curveto(92.81610906,345.26082828)(92.89110899,345.1458284)(92.97111654,345.02583456)
\curveto(93.02110886,344.93582861)(93.06110882,344.8408287)(93.09111654,344.74083456)
\curveto(93.13110875,344.65082889)(93.17110871,344.55082899)(93.21111654,344.44083456)
\curveto(93.26110862,344.31082923)(93.30110858,344.17582937)(93.33111654,344.03583456)
\curveto(93.36110852,343.89582965)(93.39610848,343.75582979)(93.43611654,343.61583456)
\curveto(93.45610842,343.53583001)(93.46110842,343.4458301)(93.45111654,343.34583456)
\curveto(93.45110843,343.25583029)(93.46110842,343.17083037)(93.48111654,343.09083456)
\lineto(93.48111654,342.92583456)
\moveto(91.23111654,343.81083456)
\curveto(91.30111058,343.91082963)(91.30611057,344.03082951)(91.24611654,344.17083456)
\curveto(91.19611068,344.32082922)(91.15611072,344.43082911)(91.12611654,344.50083456)
\curveto(90.98611089,344.77082877)(90.80111108,344.97582857)(90.57111654,345.11583456)
\curveto(90.34111154,345.26582828)(90.02111186,345.3458282)(89.61111654,345.35583456)
\curveto(89.5811123,345.33582821)(89.54611233,345.33082821)(89.50611654,345.34083456)
\curveto(89.46611241,345.35082819)(89.43111245,345.35082819)(89.40111654,345.34083456)
\curveto(89.35111253,345.32082822)(89.29611258,345.30582824)(89.23611654,345.29583456)
\curveto(89.1761127,345.29582825)(89.12111276,345.28582826)(89.07111654,345.26583456)
\curveto(88.63111325,345.12582842)(88.30611357,344.85082869)(88.09611654,344.44083456)
\curveto(88.0761138,344.40082914)(88.05111383,344.3458292)(88.02111654,344.27583456)
\curveto(88.00111388,344.21582933)(87.98611389,344.15082939)(87.97611654,344.08083456)
\curveto(87.96611391,344.02082952)(87.96611391,343.96082958)(87.97611654,343.90083456)
\curveto(87.99611388,343.8408297)(88.03111385,343.79082975)(88.08111654,343.75083456)
\curveto(88.16111372,343.70082984)(88.27111361,343.67582987)(88.41111654,343.67583456)
\lineto(88.81611654,343.67583456)
\lineto(90.48111654,343.67583456)
\lineto(90.91611654,343.67583456)
\curveto(91.0761108,343.68582986)(91.1811107,343.73082981)(91.23111654,343.81083456)
}
}
{
\newrgbcolor{curcolor}{0 0 0}
\pscustom[linestyle=none,fillstyle=solid,fillcolor=curcolor]
{
\newpath
\moveto(97.69939779,346.93083456)
\curveto(98.44939329,346.95082659)(99.09939264,346.86582668)(99.64939779,346.67583456)
\curveto(100.20939153,346.49582705)(100.63439111,346.18082736)(100.92439779,345.73083456)
\curveto(100.99439075,345.62082792)(101.05439069,345.50582804)(101.10439779,345.38583456)
\curveto(101.16439058,345.27582827)(101.21439053,345.15082839)(101.25439779,345.01083456)
\curveto(101.27439047,344.95082859)(101.28439046,344.88582866)(101.28439779,344.81583456)
\curveto(101.28439046,344.7458288)(101.27439047,344.68582886)(101.25439779,344.63583456)
\curveto(101.21439053,344.57582897)(101.15939058,344.53582901)(101.08939779,344.51583456)
\curveto(101.0393907,344.49582905)(100.97939076,344.48582906)(100.90939779,344.48583456)
\lineto(100.69939779,344.48583456)
\lineto(100.03939779,344.48583456)
\curveto(99.96939177,344.48582906)(99.89939184,344.48082906)(99.82939779,344.47083456)
\curveto(99.75939198,344.47082907)(99.69439205,344.48082906)(99.63439779,344.50083456)
\curveto(99.53439221,344.52082902)(99.45939228,344.56082898)(99.40939779,344.62083456)
\curveto(99.35939238,344.68082886)(99.31439243,344.7408288)(99.27439779,344.80083456)
\lineto(99.15439779,345.01083456)
\curveto(99.12439262,345.09082845)(99.07439267,345.15582839)(99.00439779,345.20583456)
\curveto(98.90439284,345.28582826)(98.80439294,345.3458282)(98.70439779,345.38583456)
\curveto(98.61439313,345.42582812)(98.49939324,345.46082808)(98.35939779,345.49083456)
\curveto(98.28939345,345.51082803)(98.18439356,345.52582802)(98.04439779,345.53583456)
\curveto(97.91439383,345.545828)(97.81439393,345.540828)(97.74439779,345.52083456)
\lineto(97.63939779,345.52083456)
\lineto(97.48939779,345.49083456)
\curveto(97.44939429,345.49082805)(97.40439434,345.48582806)(97.35439779,345.47583456)
\curveto(97.18439456,345.42582812)(97.0443947,345.35582819)(96.93439779,345.26583456)
\curveto(96.83439491,345.18582836)(96.76439498,345.06082848)(96.72439779,344.89083456)
\curveto(96.70439504,344.82082872)(96.70439504,344.75582879)(96.72439779,344.69583456)
\curveto(96.744395,344.63582891)(96.76439498,344.58582896)(96.78439779,344.54583456)
\curveto(96.85439489,344.42582912)(96.93439481,344.33082921)(97.02439779,344.26083456)
\curveto(97.12439462,344.19082935)(97.2393945,344.13082941)(97.36939779,344.08083456)
\curveto(97.55939418,344.00082954)(97.76439398,343.93082961)(97.98439779,343.87083456)
\lineto(98.67439779,343.72083456)
\curveto(98.91439283,343.68082986)(99.1443926,343.63082991)(99.36439779,343.57083456)
\curveto(99.59439215,343.52083002)(99.80939193,343.45583009)(100.00939779,343.37583456)
\curveto(100.09939164,343.33583021)(100.18439156,343.30083024)(100.26439779,343.27083456)
\curveto(100.35439139,343.25083029)(100.4393913,343.21583033)(100.51939779,343.16583456)
\curveto(100.70939103,343.0458305)(100.87939086,342.91583063)(101.02939779,342.77583456)
\curveto(101.18939055,342.63583091)(101.31439043,342.46083108)(101.40439779,342.25083456)
\curveto(101.43439031,342.18083136)(101.45939028,342.11083143)(101.47939779,342.04083456)
\curveto(101.49939024,341.97083157)(101.51939022,341.89583165)(101.53939779,341.81583456)
\curveto(101.54939019,341.75583179)(101.55439019,341.66083188)(101.55439779,341.53083456)
\curveto(101.56439018,341.41083213)(101.56439018,341.31583223)(101.55439779,341.24583456)
\lineto(101.55439779,341.17083456)
\curveto(101.53439021,341.11083243)(101.51939022,341.05083249)(101.50939779,340.99083456)
\curveto(101.50939023,340.9408326)(101.50439024,340.89083265)(101.49439779,340.84083456)
\curveto(101.42439032,340.540833)(101.31439043,340.27583327)(101.16439779,340.04583456)
\curveto(101.00439074,339.80583374)(100.80939093,339.61083393)(100.57939779,339.46083456)
\curveto(100.34939139,339.31083423)(100.08939165,339.18083436)(99.79939779,339.07083456)
\curveto(99.68939205,339.02083452)(99.56939217,338.98583456)(99.43939779,338.96583456)
\curveto(99.31939242,338.9458346)(99.19939254,338.92083462)(99.07939779,338.89083456)
\curveto(98.98939275,338.87083467)(98.89439285,338.86083468)(98.79439779,338.86083456)
\curveto(98.70439304,338.85083469)(98.61439313,338.83583471)(98.52439779,338.81583456)
\lineto(98.25439779,338.81583456)
\curveto(98.19439355,338.79583475)(98.08939365,338.78583476)(97.93939779,338.78583456)
\curveto(97.79939394,338.78583476)(97.69939404,338.79583475)(97.63939779,338.81583456)
\curveto(97.60939413,338.81583473)(97.57439417,338.82083472)(97.53439779,338.83083456)
\lineto(97.42939779,338.83083456)
\curveto(97.30939443,338.85083469)(97.18939455,338.86583468)(97.06939779,338.87583456)
\curveto(96.94939479,338.88583466)(96.83439491,338.90583464)(96.72439779,338.93583456)
\curveto(96.33439541,339.0458345)(95.98939575,339.17083437)(95.68939779,339.31083456)
\curveto(95.38939635,339.46083408)(95.13439661,339.68083386)(94.92439779,339.97083456)
\curveto(94.78439696,340.16083338)(94.66439708,340.38083316)(94.56439779,340.63083456)
\curveto(94.5443972,340.69083285)(94.52439722,340.77083277)(94.50439779,340.87083456)
\curveto(94.48439726,340.92083262)(94.46939727,340.99083255)(94.45939779,341.08083456)
\curveto(94.44939729,341.17083237)(94.45439729,341.2458323)(94.47439779,341.30583456)
\curveto(94.50439724,341.37583217)(94.55439719,341.42583212)(94.62439779,341.45583456)
\curveto(94.67439707,341.47583207)(94.73439701,341.48583206)(94.80439779,341.48583456)
\lineto(95.02939779,341.48583456)
\lineto(95.73439779,341.48583456)
\lineto(95.97439779,341.48583456)
\curveto(96.05439569,341.48583206)(96.12439562,341.47583207)(96.18439779,341.45583456)
\curveto(96.29439545,341.41583213)(96.36439538,341.35083219)(96.39439779,341.26083456)
\curveto(96.43439531,341.17083237)(96.47939526,341.07583247)(96.52939779,340.97583456)
\curveto(96.54939519,340.92583262)(96.58439516,340.86083268)(96.63439779,340.78083456)
\curveto(96.69439505,340.70083284)(96.744395,340.65083289)(96.78439779,340.63083456)
\curveto(96.90439484,340.53083301)(97.01939472,340.45083309)(97.12939779,340.39083456)
\curveto(97.2393945,340.3408332)(97.37939436,340.29083325)(97.54939779,340.24083456)
\curveto(97.59939414,340.22083332)(97.64939409,340.21083333)(97.69939779,340.21083456)
\curveto(97.74939399,340.22083332)(97.79939394,340.22083332)(97.84939779,340.21083456)
\curveto(97.92939381,340.19083335)(98.01439373,340.18083336)(98.10439779,340.18083456)
\curveto(98.20439354,340.19083335)(98.28939345,340.20583334)(98.35939779,340.22583456)
\curveto(98.40939333,340.23583331)(98.45439329,340.2408333)(98.49439779,340.24083456)
\curveto(98.5443932,340.2408333)(98.59439315,340.25083329)(98.64439779,340.27083456)
\curveto(98.78439296,340.32083322)(98.90939283,340.38083316)(99.01939779,340.45083456)
\curveto(99.1393926,340.52083302)(99.23439251,340.61083293)(99.30439779,340.72083456)
\curveto(99.35439239,340.80083274)(99.39439235,340.92583262)(99.42439779,341.09583456)
\curveto(99.4443923,341.16583238)(99.4443923,341.23083231)(99.42439779,341.29083456)
\curveto(99.40439234,341.35083219)(99.38439236,341.40083214)(99.36439779,341.44083456)
\curveto(99.29439245,341.58083196)(99.20439254,341.68583186)(99.09439779,341.75583456)
\curveto(98.99439275,341.82583172)(98.87439287,341.89083165)(98.73439779,341.95083456)
\curveto(98.5443932,342.03083151)(98.3443934,342.09583145)(98.13439779,342.14583456)
\curveto(97.92439382,342.19583135)(97.71439403,342.25083129)(97.50439779,342.31083456)
\curveto(97.42439432,342.33083121)(97.3393944,342.3458312)(97.24939779,342.35583456)
\curveto(97.16939457,342.36583118)(97.08939465,342.38083116)(97.00939779,342.40083456)
\curveto(96.68939505,342.49083105)(96.38439536,342.57583097)(96.09439779,342.65583456)
\curveto(95.80439594,342.7458308)(95.5393962,342.87583067)(95.29939779,343.04583456)
\curveto(95.01939672,343.2458303)(94.81439693,343.51583003)(94.68439779,343.85583456)
\curveto(94.66439708,343.92582962)(94.6443971,344.02082952)(94.62439779,344.14083456)
\curveto(94.60439714,344.21082933)(94.58939715,344.29582925)(94.57939779,344.39583456)
\curveto(94.56939717,344.49582905)(94.57439717,344.58582896)(94.59439779,344.66583456)
\curveto(94.61439713,344.71582883)(94.61939712,344.75582879)(94.60939779,344.78583456)
\curveto(94.59939714,344.82582872)(94.60439714,344.87082867)(94.62439779,344.92083456)
\curveto(94.6443971,345.03082851)(94.66439708,345.13082841)(94.68439779,345.22083456)
\curveto(94.71439703,345.32082822)(94.74939699,345.41582813)(94.78939779,345.50583456)
\curveto(94.91939682,345.79582775)(95.09939664,346.03082751)(95.32939779,346.21083456)
\curveto(95.55939618,346.39082715)(95.81939592,346.53582701)(96.10939779,346.64583456)
\curveto(96.21939552,346.69582685)(96.33439541,346.73082681)(96.45439779,346.75083456)
\curveto(96.57439517,346.78082676)(96.69939504,346.81082673)(96.82939779,346.84083456)
\curveto(96.88939485,346.86082668)(96.94939479,346.87082667)(97.00939779,346.87083456)
\lineto(97.18939779,346.90083456)
\curveto(97.26939447,346.91082663)(97.35439439,346.91582663)(97.44439779,346.91583456)
\curveto(97.53439421,346.91582663)(97.61939412,346.92082662)(97.69939779,346.93083456)
}
}
{
\newrgbcolor{curcolor}{0 0 0}
\pscustom[linestyle=none,fillstyle=solid,fillcolor=curcolor]
{
}
}
{
\newrgbcolor{curcolor}{0 0 0}
\pscustom[linestyle=none,fillstyle=solid,fillcolor=curcolor]
{
\newpath
\moveto(114.53619467,339.83583456)
\lineto(114.53619467,339.41583456)
\curveto(114.5361863,339.28583426)(114.50618633,339.18083436)(114.44619467,339.10083456)
\curveto(114.39618644,339.05083449)(114.3311865,339.01583453)(114.25119467,338.99583456)
\curveto(114.17118666,338.98583456)(114.08118675,338.98083456)(113.98119467,338.98083456)
\lineto(113.15619467,338.98083456)
\lineto(112.87119467,338.98083456)
\curveto(112.79118804,338.99083455)(112.72618811,339.01583453)(112.67619467,339.05583456)
\curveto(112.60618823,339.10583444)(112.56618827,339.17083437)(112.55619467,339.25083456)
\curveto(112.54618829,339.33083421)(112.52618831,339.41083413)(112.49619467,339.49083456)
\curveto(112.47618836,339.51083403)(112.45618838,339.52583402)(112.43619467,339.53583456)
\curveto(112.42618841,339.55583399)(112.41118842,339.57583397)(112.39119467,339.59583456)
\curveto(112.28118855,339.59583395)(112.20118863,339.57083397)(112.15119467,339.52083456)
\lineto(112.00119467,339.37083456)
\curveto(111.9311889,339.32083422)(111.86618897,339.27583427)(111.80619467,339.23583456)
\curveto(111.74618909,339.20583434)(111.68118915,339.16583438)(111.61119467,339.11583456)
\curveto(111.57118926,339.09583445)(111.52618931,339.07583447)(111.47619467,339.05583456)
\curveto(111.4361894,339.03583451)(111.39118944,339.01583453)(111.34119467,338.99583456)
\curveto(111.20118963,338.9458346)(111.05118978,338.90083464)(110.89119467,338.86083456)
\curveto(110.84118999,338.8408347)(110.79619004,338.83083471)(110.75619467,338.83083456)
\curveto(110.71619012,338.83083471)(110.67619016,338.82583472)(110.63619467,338.81583456)
\lineto(110.50119467,338.81583456)
\curveto(110.47119036,338.80583474)(110.4311904,338.80083474)(110.38119467,338.80083456)
\lineto(110.24619467,338.80083456)
\curveto(110.18619065,338.78083476)(110.09619074,338.77583477)(109.97619467,338.78583456)
\curveto(109.85619098,338.78583476)(109.77119106,338.79583475)(109.72119467,338.81583456)
\curveto(109.65119118,338.83583471)(109.58619125,338.8458347)(109.52619467,338.84583456)
\curveto(109.47619136,338.83583471)(109.42119141,338.8408347)(109.36119467,338.86083456)
\lineto(109.00119467,338.98083456)
\curveto(108.89119194,339.01083453)(108.78119205,339.05083449)(108.67119467,339.10083456)
\curveto(108.32119251,339.25083429)(108.00619283,339.48083406)(107.72619467,339.79083456)
\curveto(107.45619338,340.11083343)(107.24119359,340.4458331)(107.08119467,340.79583456)
\curveto(107.0311938,340.90583264)(106.99119384,341.01083253)(106.96119467,341.11083456)
\curveto(106.9311939,341.22083232)(106.89619394,341.33083221)(106.85619467,341.44083456)
\curveto(106.84619399,341.48083206)(106.84119399,341.51583203)(106.84119467,341.54583456)
\curveto(106.84119399,341.58583196)(106.831194,341.63083191)(106.81119467,341.68083456)
\curveto(106.79119404,341.76083178)(106.77119406,341.8458317)(106.75119467,341.93583456)
\curveto(106.74119409,342.03583151)(106.72619411,342.13583141)(106.70619467,342.23583456)
\curveto(106.69619414,342.26583128)(106.69119414,342.30083124)(106.69119467,342.34083456)
\curveto(106.70119413,342.38083116)(106.70119413,342.41583113)(106.69119467,342.44583456)
\lineto(106.69119467,342.58083456)
\curveto(106.69119414,342.63083091)(106.68619415,342.68083086)(106.67619467,342.73083456)
\curveto(106.66619417,342.78083076)(106.66119417,342.83583071)(106.66119467,342.89583456)
\curveto(106.66119417,342.96583058)(106.66619417,343.02083052)(106.67619467,343.06083456)
\curveto(106.68619415,343.11083043)(106.69119414,343.15583039)(106.69119467,343.19583456)
\lineto(106.69119467,343.34583456)
\curveto(106.70119413,343.39583015)(106.70119413,343.4408301)(106.69119467,343.48083456)
\curveto(106.69119414,343.53083001)(106.70119413,343.58082996)(106.72119467,343.63083456)
\curveto(106.74119409,343.7408298)(106.75619408,343.8458297)(106.76619467,343.94583456)
\curveto(106.78619405,344.0458295)(106.81119402,344.1458294)(106.84119467,344.24583456)
\curveto(106.88119395,344.36582918)(106.91619392,344.48082906)(106.94619467,344.59083456)
\curveto(106.97619386,344.70082884)(107.01619382,344.81082873)(107.06619467,344.92083456)
\curveto(107.20619363,345.22082832)(107.38119345,345.50582804)(107.59119467,345.77583456)
\curveto(107.61119322,345.80582774)(107.6361932,345.83082771)(107.66619467,345.85083456)
\curveto(107.70619313,345.88082766)(107.7361931,345.91082763)(107.75619467,345.94083456)
\curveto(107.79619304,345.99082755)(107.836193,346.03582751)(107.87619467,346.07583456)
\curveto(107.91619292,346.11582743)(107.96119287,346.15582739)(108.01119467,346.19583456)
\curveto(108.05119278,346.21582733)(108.08619275,346.2408273)(108.11619467,346.27083456)
\curveto(108.14619269,346.31082723)(108.18119265,346.3408272)(108.22119467,346.36083456)
\curveto(108.47119236,346.53082701)(108.76119207,346.67082687)(109.09119467,346.78083456)
\curveto(109.16119167,346.80082674)(109.2311916,346.81582673)(109.30119467,346.82583456)
\curveto(109.38119145,346.83582671)(109.46119137,346.85082669)(109.54119467,346.87083456)
\curveto(109.61119122,346.89082665)(109.70119113,346.90082664)(109.81119467,346.90083456)
\curveto(109.92119091,346.91082663)(110.0311908,346.91582663)(110.14119467,346.91583456)
\curveto(110.25119058,346.91582663)(110.35619048,346.91082663)(110.45619467,346.90083456)
\curveto(110.56619027,346.89082665)(110.65619018,346.87582667)(110.72619467,346.85583456)
\curveto(110.87618996,346.80582674)(111.02118981,346.76082678)(111.16119467,346.72083456)
\curveto(111.30118953,346.68082686)(111.4311894,346.62582692)(111.55119467,346.55583456)
\curveto(111.62118921,346.50582704)(111.68618915,346.45582709)(111.74619467,346.40583456)
\curveto(111.80618903,346.36582718)(111.87118896,346.32082722)(111.94119467,346.27083456)
\curveto(111.98118885,346.2408273)(112.0361888,346.20082734)(112.10619467,346.15083456)
\curveto(112.18618865,346.10082744)(112.26118857,346.10082744)(112.33119467,346.15083456)
\curveto(112.37118846,346.17082737)(112.39118844,346.20582734)(112.39119467,346.25583456)
\curveto(112.39118844,346.30582724)(112.40118843,346.35582719)(112.42119467,346.40583456)
\lineto(112.42119467,346.55583456)
\curveto(112.4311884,346.58582696)(112.4361884,346.62082692)(112.43619467,346.66083456)
\lineto(112.43619467,346.78083456)
\lineto(112.43619467,348.82083456)
\curveto(112.4361884,348.93082461)(112.4311884,349.05082449)(112.42119467,349.18083456)
\curveto(112.42118841,349.32082422)(112.44618839,349.42582412)(112.49619467,349.49583456)
\curveto(112.5361883,349.57582397)(112.61118822,349.62582392)(112.72119467,349.64583456)
\curveto(112.74118809,349.65582389)(112.76118807,349.65582389)(112.78119467,349.64583456)
\curveto(112.80118803,349.6458239)(112.82118801,349.65082389)(112.84119467,349.66083456)
\lineto(113.90619467,349.66083456)
\curveto(114.02618681,349.66082388)(114.1361867,349.65582389)(114.23619467,349.64583456)
\curveto(114.3361865,349.63582391)(114.41118642,349.59582395)(114.46119467,349.52583456)
\curveto(114.51118632,349.4458241)(114.5361863,349.3408242)(114.53619467,349.21083456)
\lineto(114.53619467,348.85083456)
\lineto(114.53619467,339.83583456)
\moveto(112.49619467,342.77583456)
\curveto(112.50618833,342.81583073)(112.50618833,342.85583069)(112.49619467,342.89583456)
\lineto(112.49619467,343.03083456)
\curveto(112.49618834,343.13083041)(112.49118834,343.23083031)(112.48119467,343.33083456)
\curveto(112.47118836,343.43083011)(112.45618838,343.52083002)(112.43619467,343.60083456)
\curveto(112.41618842,343.71082983)(112.39618844,343.81082973)(112.37619467,343.90083456)
\curveto(112.36618847,343.99082955)(112.34118849,344.07582947)(112.30119467,344.15583456)
\curveto(112.16118867,344.51582903)(111.95618888,344.80082874)(111.68619467,345.01083456)
\curveto(111.42618941,345.22082832)(111.04618979,345.32582822)(110.54619467,345.32583456)
\curveto(110.48619035,345.32582822)(110.40619043,345.31582823)(110.30619467,345.29583456)
\curveto(110.22619061,345.27582827)(110.15119068,345.25582829)(110.08119467,345.23583456)
\curveto(110.02119081,345.22582832)(109.96119087,345.20582834)(109.90119467,345.17583456)
\curveto(109.6311912,345.06582848)(109.42119141,344.89582865)(109.27119467,344.66583456)
\curveto(109.12119171,344.43582911)(109.00119183,344.17582937)(108.91119467,343.88583456)
\curveto(108.88119195,343.78582976)(108.86119197,343.68582986)(108.85119467,343.58583456)
\curveto(108.84119199,343.48583006)(108.82119201,343.38083016)(108.79119467,343.27083456)
\lineto(108.79119467,343.06083456)
\curveto(108.77119206,342.97083057)(108.76619207,342.8458307)(108.77619467,342.68583456)
\curveto(108.78619205,342.53583101)(108.80119203,342.42583112)(108.82119467,342.35583456)
\lineto(108.82119467,342.26583456)
\curveto(108.831192,342.2458313)(108.836192,342.22583132)(108.83619467,342.20583456)
\curveto(108.85619198,342.12583142)(108.87119196,342.05083149)(108.88119467,341.98083456)
\curveto(108.90119193,341.91083163)(108.92119191,341.83583171)(108.94119467,341.75583456)
\curveto(109.11119172,341.23583231)(109.40119143,340.85083269)(109.81119467,340.60083456)
\curveto(109.94119089,340.51083303)(110.12119071,340.4408331)(110.35119467,340.39083456)
\curveto(110.39119044,340.38083316)(110.45119038,340.37583317)(110.53119467,340.37583456)
\curveto(110.56119027,340.36583318)(110.60619023,340.35583319)(110.66619467,340.34583456)
\curveto(110.7361901,340.3458332)(110.79119004,340.35083319)(110.83119467,340.36083456)
\curveto(110.91118992,340.38083316)(110.99118984,340.39583315)(111.07119467,340.40583456)
\curveto(111.15118968,340.41583313)(111.2311896,340.43583311)(111.31119467,340.46583456)
\curveto(111.56118927,340.57583297)(111.76118907,340.71583283)(111.91119467,340.88583456)
\curveto(112.06118877,341.05583249)(112.19118864,341.27083227)(112.30119467,341.53083456)
\curveto(112.34118849,341.62083192)(112.37118846,341.71083183)(112.39119467,341.80083456)
\curveto(112.41118842,341.90083164)(112.4311884,342.00583154)(112.45119467,342.11583456)
\curveto(112.46118837,342.16583138)(112.46118837,342.21083133)(112.45119467,342.25083456)
\curveto(112.45118838,342.30083124)(112.46118837,342.35083119)(112.48119467,342.40083456)
\curveto(112.49118834,342.43083111)(112.49618834,342.46583108)(112.49619467,342.50583456)
\lineto(112.49619467,342.64083456)
\lineto(112.49619467,342.77583456)
}
}
{
\newrgbcolor{curcolor}{0 0 0}
\pscustom[linestyle=none,fillstyle=solid,fillcolor=curcolor]
{
\newpath
\moveto(123.48111654,342.92583456)
\curveto(123.50110838,342.8458307)(123.50110838,342.75583079)(123.48111654,342.65583456)
\curveto(123.46110842,342.55583099)(123.42610845,342.49083105)(123.37611654,342.46083456)
\curveto(123.32610855,342.42083112)(123.25110863,342.39083115)(123.15111654,342.37083456)
\curveto(123.06110882,342.36083118)(122.95610892,342.35083119)(122.83611654,342.34083456)
\lineto(122.49111654,342.34083456)
\curveto(122.3811095,342.35083119)(122.2811096,342.35583119)(122.19111654,342.35583456)
\lineto(118.53111654,342.35583456)
\lineto(118.32111654,342.35583456)
\curveto(118.26111362,342.35583119)(118.20611367,342.3458312)(118.15611654,342.32583456)
\curveto(118.0761138,342.28583126)(118.02611385,342.2458313)(118.00611654,342.20583456)
\curveto(117.98611389,342.18583136)(117.96611391,342.1458314)(117.94611654,342.08583456)
\curveto(117.92611395,342.03583151)(117.92111396,341.98583156)(117.93111654,341.93583456)
\curveto(117.95111393,341.87583167)(117.96111392,341.81583173)(117.96111654,341.75583456)
\curveto(117.97111391,341.70583184)(117.98611389,341.65083189)(118.00611654,341.59083456)
\curveto(118.08611379,341.35083219)(118.1811137,341.15083239)(118.29111654,340.99083456)
\curveto(118.41111347,340.8408327)(118.57111331,340.70583284)(118.77111654,340.58583456)
\curveto(118.85111303,340.53583301)(118.93111295,340.50083304)(119.01111654,340.48083456)
\curveto(119.10111278,340.47083307)(119.19111269,340.45083309)(119.28111654,340.42083456)
\curveto(119.36111252,340.40083314)(119.47111241,340.38583316)(119.61111654,340.37583456)
\curveto(119.75111213,340.36583318)(119.87111201,340.37083317)(119.97111654,340.39083456)
\lineto(120.10611654,340.39083456)
\curveto(120.20611167,340.41083313)(120.29611158,340.43083311)(120.37611654,340.45083456)
\curveto(120.46611141,340.48083306)(120.55111133,340.51083303)(120.63111654,340.54083456)
\curveto(120.73111115,340.59083295)(120.84111104,340.65583289)(120.96111654,340.73583456)
\curveto(121.09111079,340.81583273)(121.18611069,340.89583265)(121.24611654,340.97583456)
\curveto(121.29611058,341.0458325)(121.34611053,341.11083243)(121.39611654,341.17083456)
\curveto(121.45611042,341.2408323)(121.52611035,341.29083225)(121.60611654,341.32083456)
\curveto(121.70611017,341.37083217)(121.83111005,341.39083215)(121.98111654,341.38083456)
\lineto(122.41611654,341.38083456)
\lineto(122.59611654,341.38083456)
\curveto(122.66610921,341.39083215)(122.72610915,341.38583216)(122.77611654,341.36583456)
\lineto(122.92611654,341.36583456)
\curveto(123.02610885,341.3458322)(123.09610878,341.32083222)(123.13611654,341.29083456)
\curveto(123.1761087,341.27083227)(123.19610868,341.22583232)(123.19611654,341.15583456)
\curveto(123.20610867,341.08583246)(123.20110868,341.02583252)(123.18111654,340.97583456)
\curveto(123.13110875,340.83583271)(123.0761088,340.71083283)(123.01611654,340.60083456)
\curveto(122.95610892,340.49083305)(122.88610899,340.38083316)(122.80611654,340.27083456)
\curveto(122.58610929,339.9408336)(122.33610954,339.67583387)(122.05611654,339.47583456)
\curveto(121.7761101,339.27583427)(121.42611045,339.10583444)(121.00611654,338.96583456)
\curveto(120.89611098,338.92583462)(120.78611109,338.90083464)(120.67611654,338.89083456)
\curveto(120.56611131,338.88083466)(120.45111143,338.86083468)(120.33111654,338.83083456)
\curveto(120.29111159,338.82083472)(120.24611163,338.82083472)(120.19611654,338.83083456)
\curveto(120.15611172,338.83083471)(120.11611176,338.82583472)(120.07611654,338.81583456)
\lineto(119.91111654,338.81583456)
\curveto(119.86111202,338.79583475)(119.80111208,338.79083475)(119.73111654,338.80083456)
\curveto(119.67111221,338.80083474)(119.61611226,338.80583474)(119.56611654,338.81583456)
\curveto(119.48611239,338.82583472)(119.41611246,338.82583472)(119.35611654,338.81583456)
\curveto(119.29611258,338.80583474)(119.23111265,338.81083473)(119.16111654,338.83083456)
\curveto(119.11111277,338.85083469)(119.05611282,338.86083468)(118.99611654,338.86083456)
\curveto(118.93611294,338.86083468)(118.881113,338.87083467)(118.83111654,338.89083456)
\curveto(118.72111316,338.91083463)(118.61111327,338.93583461)(118.50111654,338.96583456)
\curveto(118.39111349,338.98583456)(118.29111359,339.02083452)(118.20111654,339.07083456)
\curveto(118.09111379,339.11083443)(117.98611389,339.1458344)(117.88611654,339.17583456)
\curveto(117.79611408,339.21583433)(117.71111417,339.26083428)(117.63111654,339.31083456)
\curveto(117.31111457,339.51083403)(117.02611485,339.7408338)(116.77611654,340.00083456)
\curveto(116.52611535,340.27083327)(116.32111556,340.58083296)(116.16111654,340.93083456)
\curveto(116.11111577,341.0408325)(116.07111581,341.15083239)(116.04111654,341.26083456)
\curveto(116.01111587,341.38083216)(115.97111591,341.50083204)(115.92111654,341.62083456)
\curveto(115.91111597,341.66083188)(115.90611597,341.69583185)(115.90611654,341.72583456)
\curveto(115.90611597,341.76583178)(115.90111598,341.80583174)(115.89111654,341.84583456)
\curveto(115.85111603,341.96583158)(115.82611605,342.09583145)(115.81611654,342.23583456)
\lineto(115.78611654,342.65583456)
\curveto(115.78611609,342.70583084)(115.7811161,342.76083078)(115.77111654,342.82083456)
\curveto(115.77111611,342.88083066)(115.7761161,342.93583061)(115.78611654,342.98583456)
\lineto(115.78611654,343.16583456)
\lineto(115.83111654,343.52583456)
\curveto(115.87111601,343.69582985)(115.90611597,343.86082968)(115.93611654,344.02083456)
\curveto(115.96611591,344.18082936)(116.01111587,344.33082921)(116.07111654,344.47083456)
\curveto(116.50111538,345.51082803)(117.23111465,346.2458273)(118.26111654,346.67583456)
\curveto(118.40111348,346.73582681)(118.54111334,346.77582677)(118.68111654,346.79583456)
\curveto(118.83111305,346.82582672)(118.98611289,346.86082668)(119.14611654,346.90083456)
\curveto(119.22611265,346.91082663)(119.30111258,346.91582663)(119.37111654,346.91583456)
\curveto(119.44111244,346.91582663)(119.51611236,346.92082662)(119.59611654,346.93083456)
\curveto(120.10611177,346.9408266)(120.54111134,346.88082666)(120.90111654,346.75083456)
\curveto(121.27111061,346.63082691)(121.60111028,346.47082707)(121.89111654,346.27083456)
\curveto(121.9811099,346.21082733)(122.07110981,346.1408274)(122.16111654,346.06083456)
\curveto(122.25110963,345.99082755)(122.33110955,345.91582763)(122.40111654,345.83583456)
\curveto(122.43110945,345.78582776)(122.47110941,345.7458278)(122.52111654,345.71583456)
\curveto(122.60110928,345.60582794)(122.6761092,345.49082805)(122.74611654,345.37083456)
\curveto(122.81610906,345.26082828)(122.89110899,345.1458284)(122.97111654,345.02583456)
\curveto(123.02110886,344.93582861)(123.06110882,344.8408287)(123.09111654,344.74083456)
\curveto(123.13110875,344.65082889)(123.17110871,344.55082899)(123.21111654,344.44083456)
\curveto(123.26110862,344.31082923)(123.30110858,344.17582937)(123.33111654,344.03583456)
\curveto(123.36110852,343.89582965)(123.39610848,343.75582979)(123.43611654,343.61583456)
\curveto(123.45610842,343.53583001)(123.46110842,343.4458301)(123.45111654,343.34583456)
\curveto(123.45110843,343.25583029)(123.46110842,343.17083037)(123.48111654,343.09083456)
\lineto(123.48111654,342.92583456)
\moveto(121.23111654,343.81083456)
\curveto(121.30111058,343.91082963)(121.30611057,344.03082951)(121.24611654,344.17083456)
\curveto(121.19611068,344.32082922)(121.15611072,344.43082911)(121.12611654,344.50083456)
\curveto(120.98611089,344.77082877)(120.80111108,344.97582857)(120.57111654,345.11583456)
\curveto(120.34111154,345.26582828)(120.02111186,345.3458282)(119.61111654,345.35583456)
\curveto(119.5811123,345.33582821)(119.54611233,345.33082821)(119.50611654,345.34083456)
\curveto(119.46611241,345.35082819)(119.43111245,345.35082819)(119.40111654,345.34083456)
\curveto(119.35111253,345.32082822)(119.29611258,345.30582824)(119.23611654,345.29583456)
\curveto(119.1761127,345.29582825)(119.12111276,345.28582826)(119.07111654,345.26583456)
\curveto(118.63111325,345.12582842)(118.30611357,344.85082869)(118.09611654,344.44083456)
\curveto(118.0761138,344.40082914)(118.05111383,344.3458292)(118.02111654,344.27583456)
\curveto(118.00111388,344.21582933)(117.98611389,344.15082939)(117.97611654,344.08083456)
\curveto(117.96611391,344.02082952)(117.96611391,343.96082958)(117.97611654,343.90083456)
\curveto(117.99611388,343.8408297)(118.03111385,343.79082975)(118.08111654,343.75083456)
\curveto(118.16111372,343.70082984)(118.27111361,343.67582987)(118.41111654,343.67583456)
\lineto(118.81611654,343.67583456)
\lineto(120.48111654,343.67583456)
\lineto(120.91611654,343.67583456)
\curveto(121.0761108,343.68582986)(121.1811107,343.73082981)(121.23111654,343.81083456)
}
}
{
\newrgbcolor{curcolor}{0 0 0}
\pscustom[linestyle=none,fillstyle=solid,fillcolor=curcolor]
{
}
}
{
\newrgbcolor{curcolor}{0 0 0}
\pscustom[linestyle=none,fillstyle=solid,fillcolor=curcolor]
{
\newpath
\moveto(131.85955404,346.93083456)
\curveto(132.60954954,346.95082659)(133.25954889,346.86582668)(133.80955404,346.67583456)
\curveto(134.36954778,346.49582705)(134.79454736,346.18082736)(135.08455404,345.73083456)
\curveto(135.154547,345.62082792)(135.21454694,345.50582804)(135.26455404,345.38583456)
\curveto(135.32454683,345.27582827)(135.37454678,345.15082839)(135.41455404,345.01083456)
\curveto(135.43454672,344.95082859)(135.44454671,344.88582866)(135.44455404,344.81583456)
\curveto(135.44454671,344.7458288)(135.43454672,344.68582886)(135.41455404,344.63583456)
\curveto(135.37454678,344.57582897)(135.31954683,344.53582901)(135.24955404,344.51583456)
\curveto(135.19954695,344.49582905)(135.13954701,344.48582906)(135.06955404,344.48583456)
\lineto(134.85955404,344.48583456)
\lineto(134.19955404,344.48583456)
\curveto(134.12954802,344.48582906)(134.05954809,344.48082906)(133.98955404,344.47083456)
\curveto(133.91954823,344.47082907)(133.8545483,344.48082906)(133.79455404,344.50083456)
\curveto(133.69454846,344.52082902)(133.61954853,344.56082898)(133.56955404,344.62083456)
\curveto(133.51954863,344.68082886)(133.47454868,344.7408288)(133.43455404,344.80083456)
\lineto(133.31455404,345.01083456)
\curveto(133.28454887,345.09082845)(133.23454892,345.15582839)(133.16455404,345.20583456)
\curveto(133.06454909,345.28582826)(132.96454919,345.3458282)(132.86455404,345.38583456)
\curveto(132.77454938,345.42582812)(132.65954949,345.46082808)(132.51955404,345.49083456)
\curveto(132.4495497,345.51082803)(132.34454981,345.52582802)(132.20455404,345.53583456)
\curveto(132.07455008,345.545828)(131.97455018,345.540828)(131.90455404,345.52083456)
\lineto(131.79955404,345.52083456)
\lineto(131.64955404,345.49083456)
\curveto(131.60955054,345.49082805)(131.56455059,345.48582806)(131.51455404,345.47583456)
\curveto(131.34455081,345.42582812)(131.20455095,345.35582819)(131.09455404,345.26583456)
\curveto(130.99455116,345.18582836)(130.92455123,345.06082848)(130.88455404,344.89083456)
\curveto(130.86455129,344.82082872)(130.86455129,344.75582879)(130.88455404,344.69583456)
\curveto(130.90455125,344.63582891)(130.92455123,344.58582896)(130.94455404,344.54583456)
\curveto(131.01455114,344.42582912)(131.09455106,344.33082921)(131.18455404,344.26083456)
\curveto(131.28455087,344.19082935)(131.39955075,344.13082941)(131.52955404,344.08083456)
\curveto(131.71955043,344.00082954)(131.92455023,343.93082961)(132.14455404,343.87083456)
\lineto(132.83455404,343.72083456)
\curveto(133.07454908,343.68082986)(133.30454885,343.63082991)(133.52455404,343.57083456)
\curveto(133.7545484,343.52083002)(133.96954818,343.45583009)(134.16955404,343.37583456)
\curveto(134.25954789,343.33583021)(134.34454781,343.30083024)(134.42455404,343.27083456)
\curveto(134.51454764,343.25083029)(134.59954755,343.21583033)(134.67955404,343.16583456)
\curveto(134.86954728,343.0458305)(135.03954711,342.91583063)(135.18955404,342.77583456)
\curveto(135.3495468,342.63583091)(135.47454668,342.46083108)(135.56455404,342.25083456)
\curveto(135.59454656,342.18083136)(135.61954653,342.11083143)(135.63955404,342.04083456)
\curveto(135.65954649,341.97083157)(135.67954647,341.89583165)(135.69955404,341.81583456)
\curveto(135.70954644,341.75583179)(135.71454644,341.66083188)(135.71455404,341.53083456)
\curveto(135.72454643,341.41083213)(135.72454643,341.31583223)(135.71455404,341.24583456)
\lineto(135.71455404,341.17083456)
\curveto(135.69454646,341.11083243)(135.67954647,341.05083249)(135.66955404,340.99083456)
\curveto(135.66954648,340.9408326)(135.66454649,340.89083265)(135.65455404,340.84083456)
\curveto(135.58454657,340.540833)(135.47454668,340.27583327)(135.32455404,340.04583456)
\curveto(135.16454699,339.80583374)(134.96954718,339.61083393)(134.73955404,339.46083456)
\curveto(134.50954764,339.31083423)(134.2495479,339.18083436)(133.95955404,339.07083456)
\curveto(133.8495483,339.02083452)(133.72954842,338.98583456)(133.59955404,338.96583456)
\curveto(133.47954867,338.9458346)(133.35954879,338.92083462)(133.23955404,338.89083456)
\curveto(133.149549,338.87083467)(133.0545491,338.86083468)(132.95455404,338.86083456)
\curveto(132.86454929,338.85083469)(132.77454938,338.83583471)(132.68455404,338.81583456)
\lineto(132.41455404,338.81583456)
\curveto(132.3545498,338.79583475)(132.2495499,338.78583476)(132.09955404,338.78583456)
\curveto(131.95955019,338.78583476)(131.85955029,338.79583475)(131.79955404,338.81583456)
\curveto(131.76955038,338.81583473)(131.73455042,338.82083472)(131.69455404,338.83083456)
\lineto(131.58955404,338.83083456)
\curveto(131.46955068,338.85083469)(131.3495508,338.86583468)(131.22955404,338.87583456)
\curveto(131.10955104,338.88583466)(130.99455116,338.90583464)(130.88455404,338.93583456)
\curveto(130.49455166,339.0458345)(130.149552,339.17083437)(129.84955404,339.31083456)
\curveto(129.5495526,339.46083408)(129.29455286,339.68083386)(129.08455404,339.97083456)
\curveto(128.94455321,340.16083338)(128.82455333,340.38083316)(128.72455404,340.63083456)
\curveto(128.70455345,340.69083285)(128.68455347,340.77083277)(128.66455404,340.87083456)
\curveto(128.64455351,340.92083262)(128.62955352,340.99083255)(128.61955404,341.08083456)
\curveto(128.60955354,341.17083237)(128.61455354,341.2458323)(128.63455404,341.30583456)
\curveto(128.66455349,341.37583217)(128.71455344,341.42583212)(128.78455404,341.45583456)
\curveto(128.83455332,341.47583207)(128.89455326,341.48583206)(128.96455404,341.48583456)
\lineto(129.18955404,341.48583456)
\lineto(129.89455404,341.48583456)
\lineto(130.13455404,341.48583456)
\curveto(130.21455194,341.48583206)(130.28455187,341.47583207)(130.34455404,341.45583456)
\curveto(130.4545517,341.41583213)(130.52455163,341.35083219)(130.55455404,341.26083456)
\curveto(130.59455156,341.17083237)(130.63955151,341.07583247)(130.68955404,340.97583456)
\curveto(130.70955144,340.92583262)(130.74455141,340.86083268)(130.79455404,340.78083456)
\curveto(130.8545513,340.70083284)(130.90455125,340.65083289)(130.94455404,340.63083456)
\curveto(131.06455109,340.53083301)(131.17955097,340.45083309)(131.28955404,340.39083456)
\curveto(131.39955075,340.3408332)(131.53955061,340.29083325)(131.70955404,340.24083456)
\curveto(131.75955039,340.22083332)(131.80955034,340.21083333)(131.85955404,340.21083456)
\curveto(131.90955024,340.22083332)(131.95955019,340.22083332)(132.00955404,340.21083456)
\curveto(132.08955006,340.19083335)(132.17454998,340.18083336)(132.26455404,340.18083456)
\curveto(132.36454979,340.19083335)(132.4495497,340.20583334)(132.51955404,340.22583456)
\curveto(132.56954958,340.23583331)(132.61454954,340.2408333)(132.65455404,340.24083456)
\curveto(132.70454945,340.2408333)(132.7545494,340.25083329)(132.80455404,340.27083456)
\curveto(132.94454921,340.32083322)(133.06954908,340.38083316)(133.17955404,340.45083456)
\curveto(133.29954885,340.52083302)(133.39454876,340.61083293)(133.46455404,340.72083456)
\curveto(133.51454864,340.80083274)(133.5545486,340.92583262)(133.58455404,341.09583456)
\curveto(133.60454855,341.16583238)(133.60454855,341.23083231)(133.58455404,341.29083456)
\curveto(133.56454859,341.35083219)(133.54454861,341.40083214)(133.52455404,341.44083456)
\curveto(133.4545487,341.58083196)(133.36454879,341.68583186)(133.25455404,341.75583456)
\curveto(133.154549,341.82583172)(133.03454912,341.89083165)(132.89455404,341.95083456)
\curveto(132.70454945,342.03083151)(132.50454965,342.09583145)(132.29455404,342.14583456)
\curveto(132.08455007,342.19583135)(131.87455028,342.25083129)(131.66455404,342.31083456)
\curveto(131.58455057,342.33083121)(131.49955065,342.3458312)(131.40955404,342.35583456)
\curveto(131.32955082,342.36583118)(131.2495509,342.38083116)(131.16955404,342.40083456)
\curveto(130.8495513,342.49083105)(130.54455161,342.57583097)(130.25455404,342.65583456)
\curveto(129.96455219,342.7458308)(129.69955245,342.87583067)(129.45955404,343.04583456)
\curveto(129.17955297,343.2458303)(128.97455318,343.51583003)(128.84455404,343.85583456)
\curveto(128.82455333,343.92582962)(128.80455335,344.02082952)(128.78455404,344.14083456)
\curveto(128.76455339,344.21082933)(128.7495534,344.29582925)(128.73955404,344.39583456)
\curveto(128.72955342,344.49582905)(128.73455342,344.58582896)(128.75455404,344.66583456)
\curveto(128.77455338,344.71582883)(128.77955337,344.75582879)(128.76955404,344.78583456)
\curveto(128.75955339,344.82582872)(128.76455339,344.87082867)(128.78455404,344.92083456)
\curveto(128.80455335,345.03082851)(128.82455333,345.13082841)(128.84455404,345.22083456)
\curveto(128.87455328,345.32082822)(128.90955324,345.41582813)(128.94955404,345.50583456)
\curveto(129.07955307,345.79582775)(129.25955289,346.03082751)(129.48955404,346.21083456)
\curveto(129.71955243,346.39082715)(129.97955217,346.53582701)(130.26955404,346.64583456)
\curveto(130.37955177,346.69582685)(130.49455166,346.73082681)(130.61455404,346.75083456)
\curveto(130.73455142,346.78082676)(130.85955129,346.81082673)(130.98955404,346.84083456)
\curveto(131.0495511,346.86082668)(131.10955104,346.87082667)(131.16955404,346.87083456)
\lineto(131.34955404,346.90083456)
\curveto(131.42955072,346.91082663)(131.51455064,346.91582663)(131.60455404,346.91583456)
\curveto(131.69455046,346.91582663)(131.77955037,346.92082662)(131.85955404,346.93083456)
}
}
{
\newrgbcolor{curcolor}{0 0 0}
\pscustom[linestyle=none,fillstyle=solid,fillcolor=curcolor]
{
\newpath
\moveto(144.71619467,343.16583456)
\curveto(144.7361861,343.10583044)(144.74618609,343.02083052)(144.74619467,342.91083456)
\curveto(144.74618609,342.80083074)(144.7361861,342.71583083)(144.71619467,342.65583456)
\lineto(144.71619467,342.50583456)
\curveto(144.69618614,342.42583112)(144.68618615,342.3458312)(144.68619467,342.26583456)
\curveto(144.69618614,342.18583136)(144.69118614,342.10583144)(144.67119467,342.02583456)
\curveto(144.65118618,341.95583159)(144.6361862,341.89083165)(144.62619467,341.83083456)
\curveto(144.61618622,341.77083177)(144.60618623,341.70583184)(144.59619467,341.63583456)
\curveto(144.55618628,341.52583202)(144.52118631,341.41083213)(144.49119467,341.29083456)
\curveto(144.46118637,341.18083236)(144.42118641,341.07583247)(144.37119467,340.97583456)
\curveto(144.16118667,340.49583305)(143.88618695,340.10583344)(143.54619467,339.80583456)
\curveto(143.20618763,339.50583404)(142.79618804,339.25583429)(142.31619467,339.05583456)
\curveto(142.19618864,339.00583454)(142.07118876,338.97083457)(141.94119467,338.95083456)
\curveto(141.82118901,338.92083462)(141.69618914,338.89083465)(141.56619467,338.86083456)
\curveto(141.51618932,338.8408347)(141.46118937,338.83083471)(141.40119467,338.83083456)
\curveto(141.34118949,338.83083471)(141.28618955,338.82583472)(141.23619467,338.81583456)
\lineto(141.13119467,338.81583456)
\curveto(141.10118973,338.80583474)(141.07118976,338.80083474)(141.04119467,338.80083456)
\curveto(140.99118984,338.79083475)(140.91118992,338.78583476)(140.80119467,338.78583456)
\curveto(140.69119014,338.77583477)(140.60619023,338.78083476)(140.54619467,338.80083456)
\lineto(140.39619467,338.80083456)
\curveto(140.34619049,338.81083473)(140.29119054,338.81583473)(140.23119467,338.81583456)
\curveto(140.18119065,338.80583474)(140.1311907,338.81083473)(140.08119467,338.83083456)
\curveto(140.04119079,338.8408347)(140.00119083,338.8458347)(139.96119467,338.84583456)
\curveto(139.9311909,338.8458347)(139.89119094,338.85083469)(139.84119467,338.86083456)
\curveto(139.74119109,338.89083465)(139.64119119,338.91583463)(139.54119467,338.93583456)
\curveto(139.44119139,338.95583459)(139.34619149,338.98583456)(139.25619467,339.02583456)
\curveto(139.1361917,339.06583448)(139.02119181,339.10583444)(138.91119467,339.14583456)
\curveto(138.81119202,339.18583436)(138.70619213,339.23583431)(138.59619467,339.29583456)
\curveto(138.24619259,339.50583404)(137.94619289,339.75083379)(137.69619467,340.03083456)
\curveto(137.44619339,340.31083323)(137.2361936,340.6458329)(137.06619467,341.03583456)
\curveto(137.01619382,341.12583242)(136.97619386,341.22083232)(136.94619467,341.32083456)
\curveto(136.92619391,341.42083212)(136.90119393,341.52583202)(136.87119467,341.63583456)
\curveto(136.85119398,341.68583186)(136.84119399,341.73083181)(136.84119467,341.77083456)
\curveto(136.84119399,341.81083173)(136.831194,341.85583169)(136.81119467,341.90583456)
\curveto(136.79119404,341.98583156)(136.78119405,342.06583148)(136.78119467,342.14583456)
\curveto(136.78119405,342.23583131)(136.77119406,342.32083122)(136.75119467,342.40083456)
\curveto(136.74119409,342.45083109)(136.7361941,342.49583105)(136.73619467,342.53583456)
\lineto(136.73619467,342.67083456)
\curveto(136.71619412,342.73083081)(136.70619413,342.81583073)(136.70619467,342.92583456)
\curveto(136.71619412,343.03583051)(136.7311941,343.12083042)(136.75119467,343.18083456)
\lineto(136.75119467,343.28583456)
\curveto(136.76119407,343.33583021)(136.76119407,343.38583016)(136.75119467,343.43583456)
\curveto(136.75119408,343.49583005)(136.76119407,343.55082999)(136.78119467,343.60083456)
\curveto(136.79119404,343.65082989)(136.79619404,343.69582985)(136.79619467,343.73583456)
\curveto(136.79619404,343.78582976)(136.80619403,343.83582971)(136.82619467,343.88583456)
\curveto(136.86619397,344.01582953)(136.90119393,344.1408294)(136.93119467,344.26083456)
\curveto(136.96119387,344.39082915)(137.00119383,344.51582903)(137.05119467,344.63583456)
\curveto(137.2311936,345.0458285)(137.44619339,345.38582816)(137.69619467,345.65583456)
\curveto(137.94619289,345.93582761)(138.25119258,346.19082735)(138.61119467,346.42083456)
\curveto(138.71119212,346.47082707)(138.81619202,346.51582703)(138.92619467,346.55583456)
\curveto(139.0361918,346.59582695)(139.14619169,346.6408269)(139.25619467,346.69083456)
\curveto(139.38619145,346.7408268)(139.52119131,346.77582677)(139.66119467,346.79583456)
\curveto(139.80119103,346.81582673)(139.94619089,346.8458267)(140.09619467,346.88583456)
\curveto(140.17619066,346.89582665)(140.25119058,346.90082664)(140.32119467,346.90083456)
\curveto(140.39119044,346.90082664)(140.46119037,346.90582664)(140.53119467,346.91583456)
\curveto(141.11118972,346.92582662)(141.61118922,346.86582668)(142.03119467,346.73583456)
\curveto(142.46118837,346.60582694)(142.84118799,346.42582712)(143.17119467,346.19583456)
\curveto(143.28118755,346.11582743)(143.39118744,346.02582752)(143.50119467,345.92583456)
\curveto(143.62118721,345.83582771)(143.72118711,345.73582781)(143.80119467,345.62583456)
\curveto(143.88118695,345.52582802)(143.95118688,345.42582812)(144.01119467,345.32583456)
\curveto(144.08118675,345.22582832)(144.15118668,345.12082842)(144.22119467,345.01083456)
\curveto(144.29118654,344.90082864)(144.34618649,344.78082876)(144.38619467,344.65083456)
\curveto(144.42618641,344.53082901)(144.47118636,344.40082914)(144.52119467,344.26083456)
\curveto(144.55118628,344.18082936)(144.57618626,344.09582945)(144.59619467,344.00583456)
\lineto(144.65619467,343.73583456)
\curveto(144.66618617,343.69582985)(144.67118616,343.65582989)(144.67119467,343.61583456)
\curveto(144.67118616,343.57582997)(144.67618616,343.53583001)(144.68619467,343.49583456)
\curveto(144.70618613,343.4458301)(144.71118612,343.39083015)(144.70119467,343.33083456)
\curveto(144.69118614,343.27083027)(144.69618614,343.21583033)(144.71619467,343.16583456)
\moveto(142.61619467,342.62583456)
\curveto(142.62618821,342.67583087)(142.6311882,342.7458308)(142.63119467,342.83583456)
\curveto(142.6311882,342.93583061)(142.62618821,343.01083053)(142.61619467,343.06083456)
\lineto(142.61619467,343.18083456)
\curveto(142.59618824,343.23083031)(142.58618825,343.28583026)(142.58619467,343.34583456)
\curveto(142.58618825,343.40583014)(142.58118825,343.46083008)(142.57119467,343.51083456)
\curveto(142.57118826,343.55082999)(142.56618827,343.58082996)(142.55619467,343.60083456)
\lineto(142.49619467,343.84083456)
\curveto(142.48618835,343.93082961)(142.46618837,344.01582953)(142.43619467,344.09583456)
\curveto(142.32618851,344.35582919)(142.19618864,344.57582897)(142.04619467,344.75583456)
\curveto(141.89618894,344.9458286)(141.69618914,345.09582845)(141.44619467,345.20583456)
\curveto(141.38618945,345.22582832)(141.32618951,345.2408283)(141.26619467,345.25083456)
\curveto(141.20618963,345.27082827)(141.14118969,345.29082825)(141.07119467,345.31083456)
\curveto(140.99118984,345.33082821)(140.90618993,345.33582821)(140.81619467,345.32583456)
\lineto(140.54619467,345.32583456)
\curveto(140.51619032,345.30582824)(140.48119035,345.29582825)(140.44119467,345.29583456)
\curveto(140.40119043,345.30582824)(140.36619047,345.30582824)(140.33619467,345.29583456)
\lineto(140.12619467,345.23583456)
\curveto(140.06619077,345.22582832)(140.01119082,345.20582834)(139.96119467,345.17583456)
\curveto(139.71119112,345.06582848)(139.50619133,344.90582864)(139.34619467,344.69583456)
\curveto(139.19619164,344.49582905)(139.07619176,344.26082928)(138.98619467,343.99083456)
\curveto(138.95619188,343.89082965)(138.9311919,343.78582976)(138.91119467,343.67583456)
\curveto(138.90119193,343.56582998)(138.88619195,343.45583009)(138.86619467,343.34583456)
\curveto(138.85619198,343.29583025)(138.85119198,343.2458303)(138.85119467,343.19583456)
\lineto(138.85119467,343.04583456)
\curveto(138.831192,342.97583057)(138.82119201,342.87083067)(138.82119467,342.73083456)
\curveto(138.831192,342.59083095)(138.84619199,342.48583106)(138.86619467,342.41583456)
\lineto(138.86619467,342.28083456)
\curveto(138.88619195,342.20083134)(138.90119193,342.12083142)(138.91119467,342.04083456)
\curveto(138.92119191,341.97083157)(138.9361919,341.89583165)(138.95619467,341.81583456)
\curveto(139.05619178,341.51583203)(139.16119167,341.27083227)(139.27119467,341.08083456)
\curveto(139.39119144,340.90083264)(139.57619126,340.73583281)(139.82619467,340.58583456)
\curveto(139.89619094,340.53583301)(139.97119086,340.49583305)(140.05119467,340.46583456)
\curveto(140.14119069,340.43583311)(140.2311906,340.41083313)(140.32119467,340.39083456)
\curveto(140.36119047,340.38083316)(140.39619044,340.37583317)(140.42619467,340.37583456)
\curveto(140.45619038,340.38583316)(140.49119034,340.38583316)(140.53119467,340.37583456)
\lineto(140.65119467,340.34583456)
\curveto(140.70119013,340.3458332)(140.74619009,340.35083319)(140.78619467,340.36083456)
\lineto(140.90619467,340.36083456)
\curveto(140.98618985,340.38083316)(141.06618977,340.39583315)(141.14619467,340.40583456)
\curveto(141.22618961,340.41583313)(141.30118953,340.43583311)(141.37119467,340.46583456)
\curveto(141.6311892,340.56583298)(141.84118899,340.70083284)(142.00119467,340.87083456)
\curveto(142.16118867,341.0408325)(142.29618854,341.25083229)(142.40619467,341.50083456)
\curveto(142.44618839,341.60083194)(142.47618836,341.70083184)(142.49619467,341.80083456)
\curveto(142.51618832,341.90083164)(142.54118829,342.00583154)(142.57119467,342.11583456)
\curveto(142.58118825,342.15583139)(142.58618825,342.19083135)(142.58619467,342.22083456)
\curveto(142.58618825,342.26083128)(142.59118824,342.30083124)(142.60119467,342.34083456)
\lineto(142.60119467,342.47583456)
\curveto(142.60118823,342.52583102)(142.60618823,342.57583097)(142.61619467,342.62583456)
}
}
{
\newrgbcolor{curcolor}{0 0 0}
\pscustom[linestyle=none,fillstyle=solid,fillcolor=curcolor]
{
\newpath
\moveto(149.68611654,346.93083456)
\curveto(150.49611138,346.95082659)(151.17111071,346.83082671)(151.71111654,346.57083456)
\curveto(152.26110962,346.31082723)(152.69610918,345.9408276)(153.01611654,345.46083456)
\curveto(153.1761087,345.22082832)(153.29610858,344.9458286)(153.37611654,344.63583456)
\curveto(153.39610848,344.58582896)(153.41110847,344.52082902)(153.42111654,344.44083456)
\curveto(153.44110844,344.36082918)(153.44110844,344.29082925)(153.42111654,344.23083456)
\curveto(153.3811085,344.12082942)(153.31110857,344.05582949)(153.21111654,344.03583456)
\curveto(153.11110877,344.02582952)(152.99110889,344.02082952)(152.85111654,344.02083456)
\lineto(152.07111654,344.02083456)
\lineto(151.78611654,344.02083456)
\curveto(151.69611018,344.02082952)(151.62111026,344.0408295)(151.56111654,344.08083456)
\curveto(151.4811104,344.12082942)(151.42611045,344.18082936)(151.39611654,344.26083456)
\curveto(151.36611051,344.35082919)(151.32611055,344.4408291)(151.27611654,344.53083456)
\curveto(151.21611066,344.6408289)(151.15111073,344.7408288)(151.08111654,344.83083456)
\curveto(151.01111087,344.92082862)(150.93111095,345.00082854)(150.84111654,345.07083456)
\curveto(150.70111118,345.16082838)(150.54611133,345.23082831)(150.37611654,345.28083456)
\curveto(150.31611156,345.30082824)(150.25611162,345.31082823)(150.19611654,345.31083456)
\curveto(150.13611174,345.31082823)(150.0811118,345.32082822)(150.03111654,345.34083456)
\lineto(149.88111654,345.34083456)
\curveto(149.6811122,345.3408282)(149.52111236,345.32082822)(149.40111654,345.28083456)
\curveto(149.11111277,345.19082835)(148.876113,345.05082849)(148.69611654,344.86083456)
\curveto(148.51611336,344.68082886)(148.37111351,344.46082908)(148.26111654,344.20083456)
\curveto(148.21111367,344.09082945)(148.17111371,343.97082957)(148.14111654,343.84083456)
\curveto(148.12111376,343.72082982)(148.09611378,343.59082995)(148.06611654,343.45083456)
\curveto(148.05611382,343.41083013)(148.05111383,343.37083017)(148.05111654,343.33083456)
\curveto(148.05111383,343.29083025)(148.04611383,343.25083029)(148.03611654,343.21083456)
\curveto(148.01611386,343.11083043)(148.00611387,342.97083057)(148.00611654,342.79083456)
\curveto(148.01611386,342.61083093)(148.03111385,342.47083107)(148.05111654,342.37083456)
\curveto(148.05111383,342.29083125)(148.05611382,342.23583131)(148.06611654,342.20583456)
\curveto(148.08611379,342.13583141)(148.09611378,342.06583148)(148.09611654,341.99583456)
\curveto(148.10611377,341.92583162)(148.12111376,341.85583169)(148.14111654,341.78583456)
\curveto(148.22111366,341.55583199)(148.31611356,341.3458322)(148.42611654,341.15583456)
\curveto(148.53611334,340.96583258)(148.6761132,340.80583274)(148.84611654,340.67583456)
\curveto(148.88611299,340.6458329)(148.94611293,340.61083293)(149.02611654,340.57083456)
\curveto(149.13611274,340.50083304)(149.24611263,340.45583309)(149.35611654,340.43583456)
\curveto(149.4761124,340.41583313)(149.62111226,340.39583315)(149.79111654,340.37583456)
\lineto(149.88111654,340.37583456)
\curveto(149.92111196,340.37583317)(149.95111193,340.38083316)(149.97111654,340.39083456)
\lineto(150.10611654,340.39083456)
\curveto(150.1761117,340.41083313)(150.24111164,340.42583312)(150.30111654,340.43583456)
\curveto(150.37111151,340.45583309)(150.43611144,340.47583307)(150.49611654,340.49583456)
\curveto(150.79611108,340.62583292)(151.02611085,340.81583273)(151.18611654,341.06583456)
\curveto(151.22611065,341.11583243)(151.26111062,341.17083237)(151.29111654,341.23083456)
\curveto(151.32111056,341.30083224)(151.34611053,341.36083218)(151.36611654,341.41083456)
\curveto(151.40611047,341.52083202)(151.44111044,341.61583193)(151.47111654,341.69583456)
\curveto(151.50111038,341.78583176)(151.57111031,341.85583169)(151.68111654,341.90583456)
\curveto(151.77111011,341.9458316)(151.91610996,341.96083158)(152.11611654,341.95083456)
\lineto(152.61111654,341.95083456)
\lineto(152.82111654,341.95083456)
\curveto(152.90110898,341.96083158)(152.96610891,341.95583159)(153.01611654,341.93583456)
\lineto(153.13611654,341.93583456)
\lineto(153.25611654,341.90583456)
\curveto(153.29610858,341.90583164)(153.32610855,341.89583165)(153.34611654,341.87583456)
\curveto(153.39610848,341.83583171)(153.42610845,341.77583177)(153.43611654,341.69583456)
\curveto(153.45610842,341.62583192)(153.45610842,341.55083199)(153.43611654,341.47083456)
\curveto(153.34610853,341.1408324)(153.23610864,340.8458327)(153.10611654,340.58583456)
\curveto(152.69610918,339.81583373)(152.04110984,339.28083426)(151.14111654,338.98083456)
\curveto(151.04111084,338.95083459)(150.93611094,338.93083461)(150.82611654,338.92083456)
\curveto(150.71611116,338.90083464)(150.60611127,338.87583467)(150.49611654,338.84583456)
\curveto(150.43611144,338.83583471)(150.3761115,338.83083471)(150.31611654,338.83083456)
\curveto(150.25611162,338.83083471)(150.19611168,338.82583472)(150.13611654,338.81583456)
\lineto(149.97111654,338.81583456)
\curveto(149.92111196,338.79583475)(149.84611203,338.79083475)(149.74611654,338.80083456)
\curveto(149.64611223,338.80083474)(149.57111231,338.80583474)(149.52111654,338.81583456)
\curveto(149.44111244,338.83583471)(149.36611251,338.8458347)(149.29611654,338.84583456)
\curveto(149.23611264,338.83583471)(149.17111271,338.8408347)(149.10111654,338.86083456)
\lineto(148.95111654,338.89083456)
\curveto(148.90111298,338.89083465)(148.85111303,338.89583465)(148.80111654,338.90583456)
\curveto(148.69111319,338.93583461)(148.58611329,338.96583458)(148.48611654,338.99583456)
\curveto(148.38611349,339.02583452)(148.29111359,339.06083448)(148.20111654,339.10083456)
\curveto(147.73111415,339.30083424)(147.33611454,339.55583399)(147.01611654,339.86583456)
\curveto(146.69611518,340.18583336)(146.43611544,340.58083296)(146.23611654,341.05083456)
\curveto(146.18611569,341.1408324)(146.14611573,341.23583231)(146.11611654,341.33583456)
\lineto(146.02611654,341.66583456)
\curveto(146.01611586,341.70583184)(146.01111587,341.7408318)(146.01111654,341.77083456)
\curveto(146.01111587,341.81083173)(146.00111588,341.85583169)(145.98111654,341.90583456)
\curveto(145.96111592,341.97583157)(145.95111593,342.0458315)(145.95111654,342.11583456)
\curveto(145.95111593,342.19583135)(145.94111594,342.27083127)(145.92111654,342.34083456)
\lineto(145.92111654,342.59583456)
\curveto(145.90111598,342.6458309)(145.89111599,342.70083084)(145.89111654,342.76083456)
\curveto(145.89111599,342.83083071)(145.90111598,342.89083065)(145.92111654,342.94083456)
\curveto(145.93111595,342.99083055)(145.93111595,343.03583051)(145.92111654,343.07583456)
\curveto(145.91111597,343.11583043)(145.91111597,343.15583039)(145.92111654,343.19583456)
\curveto(145.94111594,343.26583028)(145.94611593,343.33083021)(145.93611654,343.39083456)
\curveto(145.93611594,343.45083009)(145.94611593,343.51083003)(145.96611654,343.57083456)
\curveto(146.01611586,343.75082979)(146.05611582,343.92082962)(146.08611654,344.08083456)
\curveto(146.11611576,344.25082929)(146.16111572,344.41582913)(146.22111654,344.57583456)
\curveto(146.44111544,345.08582846)(146.71611516,345.51082803)(147.04611654,345.85083456)
\curveto(147.38611449,346.19082735)(147.81611406,346.46582708)(148.33611654,346.67583456)
\curveto(148.4761134,346.73582681)(148.62111326,346.77582677)(148.77111654,346.79583456)
\curveto(148.92111296,346.82582672)(149.0761128,346.86082668)(149.23611654,346.90083456)
\curveto(149.31611256,346.91082663)(149.39111249,346.91582663)(149.46111654,346.91583456)
\curveto(149.53111235,346.91582663)(149.60611227,346.92082662)(149.68611654,346.93083456)
}
}
{
\newrgbcolor{curcolor}{0 0 0}
\pscustom[linestyle=none,fillstyle=solid,fillcolor=curcolor]
{
\newpath
\moveto(156.82939779,349.57083456)
\curveto(156.89939484,349.49082405)(156.93439481,349.37082417)(156.93439779,349.21083456)
\lineto(156.93439779,348.74583456)
\lineto(156.93439779,348.34083456)
\curveto(156.93439481,348.20082534)(156.89939484,348.10582544)(156.82939779,348.05583456)
\curveto(156.76939497,348.00582554)(156.68939505,347.97582557)(156.58939779,347.96583456)
\curveto(156.49939524,347.95582559)(156.39939534,347.95082559)(156.28939779,347.95083456)
\lineto(155.44939779,347.95083456)
\curveto(155.3393964,347.95082559)(155.2393965,347.95582559)(155.14939779,347.96583456)
\curveto(155.06939667,347.97582557)(154.99939674,348.00582554)(154.93939779,348.05583456)
\curveto(154.89939684,348.08582546)(154.86939687,348.1408254)(154.84939779,348.22083456)
\curveto(154.8393969,348.31082523)(154.82939691,348.40582514)(154.81939779,348.50583456)
\lineto(154.81939779,348.83583456)
\curveto(154.82939691,348.9458246)(154.83439691,349.0408245)(154.83439779,349.12083456)
\lineto(154.83439779,349.33083456)
\curveto(154.8443969,349.40082414)(154.86439688,349.46082408)(154.89439779,349.51083456)
\curveto(154.91439683,349.55082399)(154.9393968,349.58082396)(154.96939779,349.60083456)
\lineto(155.08939779,349.66083456)
\curveto(155.10939663,349.66082388)(155.13439661,349.66082388)(155.16439779,349.66083456)
\curveto(155.19439655,349.67082387)(155.21939652,349.67582387)(155.23939779,349.67583456)
\lineto(156.33439779,349.67583456)
\curveto(156.43439531,349.67582387)(156.52939521,349.67082387)(156.61939779,349.66083456)
\curveto(156.70939503,349.65082389)(156.77939496,349.62082392)(156.82939779,349.57083456)
\moveto(156.93439779,339.80583456)
\curveto(156.93439481,339.60583394)(156.92939481,339.43583411)(156.91939779,339.29583456)
\curveto(156.90939483,339.15583439)(156.81939492,339.06083448)(156.64939779,339.01083456)
\curveto(156.58939515,338.99083455)(156.52439522,338.98083456)(156.45439779,338.98083456)
\curveto(156.38439536,338.99083455)(156.30939543,338.99583455)(156.22939779,338.99583456)
\lineto(155.38939779,338.99583456)
\curveto(155.29939644,338.99583455)(155.20939653,339.00083454)(155.11939779,339.01083456)
\curveto(155.0393967,339.02083452)(154.97939676,339.05083449)(154.93939779,339.10083456)
\curveto(154.87939686,339.17083437)(154.8443969,339.25583429)(154.83439779,339.35583456)
\lineto(154.83439779,339.70083456)
\lineto(154.83439779,346.03083456)
\lineto(154.83439779,346.33083456)
\curveto(154.83439691,346.43082711)(154.85439689,346.51082703)(154.89439779,346.57083456)
\curveto(154.95439679,346.6408269)(155.0393967,346.68582686)(155.14939779,346.70583456)
\curveto(155.16939657,346.71582683)(155.19439655,346.71582683)(155.22439779,346.70583456)
\curveto(155.26439648,346.70582684)(155.29439645,346.71082683)(155.31439779,346.72083456)
\lineto(156.06439779,346.72083456)
\lineto(156.25939779,346.72083456)
\curveto(156.3393954,346.73082681)(156.40439534,346.73082681)(156.45439779,346.72083456)
\lineto(156.57439779,346.72083456)
\curveto(156.63439511,346.70082684)(156.68939505,346.68582686)(156.73939779,346.67583456)
\curveto(156.78939495,346.66582688)(156.82939491,346.63582691)(156.85939779,346.58583456)
\curveto(156.89939484,346.53582701)(156.91939482,346.46582708)(156.91939779,346.37583456)
\curveto(156.92939481,346.28582726)(156.93439481,346.19082735)(156.93439779,346.09083456)
\lineto(156.93439779,339.80583456)
}
}
{
\newrgbcolor{curcolor}{0 0 0}
\pscustom[linestyle=none,fillstyle=solid,fillcolor=curcolor]
{
\newpath
\moveto(165.64658529,339.58083456)
\curveto(165.66657744,339.47083407)(165.67657743,339.36083418)(165.67658529,339.25083456)
\curveto(165.68657742,339.1408344)(165.63657747,339.06583448)(165.52658529,339.02583456)
\curveto(165.46657764,338.99583455)(165.39657771,338.98083456)(165.31658529,338.98083456)
\lineto(165.07658529,338.98083456)
\lineto(164.26658529,338.98083456)
\lineto(163.99658529,338.98083456)
\curveto(163.91657919,338.99083455)(163.85157926,339.01583453)(163.80158529,339.05583456)
\curveto(163.73157938,339.09583445)(163.67657943,339.15083439)(163.63658529,339.22083456)
\curveto(163.6065795,339.30083424)(163.56157955,339.36583418)(163.50158529,339.41583456)
\curveto(163.48157963,339.43583411)(163.45657965,339.45083409)(163.42658529,339.46083456)
\curveto(163.39657971,339.48083406)(163.35657975,339.48583406)(163.30658529,339.47583456)
\curveto(163.25657985,339.45583409)(163.2065799,339.43083411)(163.15658529,339.40083456)
\curveto(163.11657999,339.37083417)(163.07158004,339.3458342)(163.02158529,339.32583456)
\curveto(162.97158014,339.28583426)(162.91658019,339.25083429)(162.85658529,339.22083456)
\lineto(162.67658529,339.13083456)
\curveto(162.54658056,339.07083447)(162.4115807,339.02083452)(162.27158529,338.98083456)
\curveto(162.13158098,338.95083459)(161.98658112,338.91583463)(161.83658529,338.87583456)
\curveto(161.76658134,338.85583469)(161.69658141,338.8458347)(161.62658529,338.84583456)
\curveto(161.56658154,338.83583471)(161.50158161,338.82583472)(161.43158529,338.81583456)
\lineto(161.34158529,338.81583456)
\curveto(161.3115818,338.80583474)(161.28158183,338.80083474)(161.25158529,338.80083456)
\lineto(161.08658529,338.80083456)
\curveto(160.98658212,338.78083476)(160.88658222,338.78083476)(160.78658529,338.80083456)
\lineto(160.65158529,338.80083456)
\curveto(160.58158253,338.82083472)(160.5115826,338.83083471)(160.44158529,338.83083456)
\curveto(160.38158273,338.82083472)(160.32158279,338.82583472)(160.26158529,338.84583456)
\curveto(160.16158295,338.86583468)(160.06658304,338.88583466)(159.97658529,338.90583456)
\curveto(159.88658322,338.91583463)(159.80158331,338.9408346)(159.72158529,338.98083456)
\curveto(159.43158368,339.09083445)(159.18158393,339.23083431)(158.97158529,339.40083456)
\curveto(158.77158434,339.58083396)(158.6115845,339.81583373)(158.49158529,340.10583456)
\curveto(158.46158465,340.17583337)(158.43158468,340.25083329)(158.40158529,340.33083456)
\curveto(158.38158473,340.41083313)(158.36158475,340.49583305)(158.34158529,340.58583456)
\curveto(158.32158479,340.63583291)(158.3115848,340.68583286)(158.31158529,340.73583456)
\curveto(158.32158479,340.78583276)(158.32158479,340.83583271)(158.31158529,340.88583456)
\curveto(158.30158481,340.91583263)(158.29158482,340.97583257)(158.28158529,341.06583456)
\curveto(158.28158483,341.16583238)(158.28658482,341.23583231)(158.29658529,341.27583456)
\curveto(158.31658479,341.37583217)(158.32658478,341.46083208)(158.32658529,341.53083456)
\lineto(158.41658529,341.86083456)
\curveto(158.44658466,341.98083156)(158.48658462,342.08583146)(158.53658529,342.17583456)
\curveto(158.7065844,342.46583108)(158.90158421,342.68583086)(159.12158529,342.83583456)
\curveto(159.34158377,342.98583056)(159.62158349,343.11583043)(159.96158529,343.22583456)
\curveto(160.09158302,343.27583027)(160.22658288,343.31083023)(160.36658529,343.33083456)
\curveto(160.5065826,343.35083019)(160.64658246,343.37583017)(160.78658529,343.40583456)
\curveto(160.86658224,343.42583012)(160.95158216,343.43583011)(161.04158529,343.43583456)
\curveto(161.13158198,343.4458301)(161.22158189,343.46083008)(161.31158529,343.48083456)
\curveto(161.38158173,343.50083004)(161.45158166,343.50583004)(161.52158529,343.49583456)
\curveto(161.59158152,343.49583005)(161.66658144,343.50583004)(161.74658529,343.52583456)
\curveto(161.81658129,343.54583)(161.88658122,343.55582999)(161.95658529,343.55583456)
\curveto(162.02658108,343.55582999)(162.10158101,343.56582998)(162.18158529,343.58583456)
\curveto(162.39158072,343.63582991)(162.58158053,343.67582987)(162.75158529,343.70583456)
\curveto(162.93158018,343.7458298)(163.09158002,343.83582971)(163.23158529,343.97583456)
\curveto(163.32157979,344.06582948)(163.38157973,344.16582938)(163.41158529,344.27583456)
\curveto(163.42157969,344.30582924)(163.42157969,344.33082921)(163.41158529,344.35083456)
\curveto(163.4115797,344.37082917)(163.41657969,344.39082915)(163.42658529,344.41083456)
\curveto(163.43657967,344.43082911)(163.44157967,344.46082908)(163.44158529,344.50083456)
\lineto(163.44158529,344.59083456)
\lineto(163.41158529,344.71083456)
\curveto(163.4115797,344.75082879)(163.4065797,344.78582876)(163.39658529,344.81583456)
\curveto(163.29657981,345.11582843)(163.08658002,345.32082822)(162.76658529,345.43083456)
\curveto(162.67658043,345.46082808)(162.56658054,345.48082806)(162.43658529,345.49083456)
\curveto(162.31658079,345.51082803)(162.19158092,345.51582803)(162.06158529,345.50583456)
\curveto(161.93158118,345.50582804)(161.8065813,345.49582805)(161.68658529,345.47583456)
\curveto(161.56658154,345.45582809)(161.46158165,345.43082811)(161.37158529,345.40083456)
\curveto(161.3115818,345.38082816)(161.25158186,345.35082819)(161.19158529,345.31083456)
\curveto(161.14158197,345.28082826)(161.09158202,345.2458283)(161.04158529,345.20583456)
\curveto(160.99158212,345.16582838)(160.93658217,345.11082843)(160.87658529,345.04083456)
\curveto(160.82658228,344.97082857)(160.79158232,344.90582864)(160.77158529,344.84583456)
\curveto(160.72158239,344.7458288)(160.67658243,344.65082889)(160.63658529,344.56083456)
\curveto(160.6065825,344.47082907)(160.53658257,344.41082913)(160.42658529,344.38083456)
\curveto(160.34658276,344.36082918)(160.26158285,344.35082919)(160.17158529,344.35083456)
\lineto(159.90158529,344.35083456)
\lineto(159.33158529,344.35083456)
\curveto(159.28158383,344.35082919)(159.23158388,344.3458292)(159.18158529,344.33583456)
\curveto(159.13158398,344.33582921)(159.08658402,344.3408292)(159.04658529,344.35083456)
\lineto(158.91158529,344.35083456)
\curveto(158.89158422,344.36082918)(158.86658424,344.36582918)(158.83658529,344.36583456)
\curveto(158.8065843,344.36582918)(158.78158433,344.37582917)(158.76158529,344.39583456)
\curveto(158.68158443,344.41582913)(158.62658448,344.48082906)(158.59658529,344.59083456)
\curveto(158.58658452,344.6408289)(158.58658452,344.69082885)(158.59658529,344.74083456)
\curveto(158.6065845,344.79082875)(158.61658449,344.83582871)(158.62658529,344.87583456)
\curveto(158.65658445,344.98582856)(158.68658442,345.08582846)(158.71658529,345.17583456)
\curveto(158.75658435,345.27582827)(158.80158431,345.36582818)(158.85158529,345.44583456)
\lineto(158.94158529,345.59583456)
\lineto(159.03158529,345.74583456)
\curveto(159.111584,345.85582769)(159.2115839,345.96082758)(159.33158529,346.06083456)
\curveto(159.35158376,346.07082747)(159.38158373,346.09582745)(159.42158529,346.13583456)
\curveto(159.47158364,346.17582737)(159.51658359,346.21082733)(159.55658529,346.24083456)
\curveto(159.59658351,346.27082727)(159.64158347,346.30082724)(159.69158529,346.33083456)
\curveto(159.86158325,346.4408271)(160.04158307,346.52582702)(160.23158529,346.58583456)
\curveto(160.42158269,346.65582689)(160.61658249,346.72082682)(160.81658529,346.78083456)
\curveto(160.93658217,346.81082673)(161.06158205,346.83082671)(161.19158529,346.84083456)
\curveto(161.32158179,346.85082669)(161.45158166,346.87082667)(161.58158529,346.90083456)
\curveto(161.62158149,346.91082663)(161.68158143,346.91082663)(161.76158529,346.90083456)
\curveto(161.85158126,346.89082665)(161.9065812,346.89582665)(161.92658529,346.91583456)
\curveto(162.33658077,346.92582662)(162.72658038,346.91082663)(163.09658529,346.87083456)
\curveto(163.47657963,346.83082671)(163.81657929,346.75582679)(164.11658529,346.64583456)
\curveto(164.42657868,346.53582701)(164.69157842,346.38582716)(164.91158529,346.19583456)
\curveto(165.13157798,346.01582753)(165.30157781,345.78082776)(165.42158529,345.49083456)
\curveto(165.49157762,345.32082822)(165.53157758,345.12582842)(165.54158529,344.90583456)
\curveto(165.55157756,344.68582886)(165.55657755,344.46082908)(165.55658529,344.23083456)
\lineto(165.55658529,340.88583456)
\lineto(165.55658529,340.30083456)
\curveto(165.55657755,340.11083343)(165.57657753,339.93583361)(165.61658529,339.77583456)
\curveto(165.62657748,339.7458338)(165.63157748,339.71083383)(165.63158529,339.67083456)
\curveto(165.63157748,339.6408339)(165.63657747,339.61083393)(165.64658529,339.58083456)
\moveto(163.44158529,341.89083456)
\curveto(163.45157966,341.9408316)(163.45657965,341.99583155)(163.45658529,342.05583456)
\curveto(163.45657965,342.12583142)(163.45157966,342.18583136)(163.44158529,342.23583456)
\curveto(163.42157969,342.29583125)(163.4115797,342.35083119)(163.41158529,342.40083456)
\curveto(163.4115797,342.45083109)(163.39157972,342.49083105)(163.35158529,342.52083456)
\curveto(163.30157981,342.56083098)(163.22657988,342.58083096)(163.12658529,342.58083456)
\curveto(163.08658002,342.57083097)(163.05158006,342.56083098)(163.02158529,342.55083456)
\curveto(162.99158012,342.55083099)(162.95658015,342.545831)(162.91658529,342.53583456)
\curveto(162.84658026,342.51583103)(162.77158034,342.50083104)(162.69158529,342.49083456)
\curveto(162.6115805,342.48083106)(162.53158058,342.46583108)(162.45158529,342.44583456)
\curveto(162.42158069,342.43583111)(162.37658073,342.43083111)(162.31658529,342.43083456)
\curveto(162.18658092,342.40083114)(162.05658105,342.38083116)(161.92658529,342.37083456)
\curveto(161.79658131,342.36083118)(161.67158144,342.33583121)(161.55158529,342.29583456)
\curveto(161.47158164,342.27583127)(161.39658171,342.25583129)(161.32658529,342.23583456)
\curveto(161.25658185,342.22583132)(161.18658192,342.20583134)(161.11658529,342.17583456)
\curveto(160.9065822,342.08583146)(160.72658238,341.95083159)(160.57658529,341.77083456)
\curveto(160.43658267,341.59083195)(160.38658272,341.3408322)(160.42658529,341.02083456)
\curveto(160.44658266,340.85083269)(160.50158261,340.71083283)(160.59158529,340.60083456)
\curveto(160.66158245,340.49083305)(160.76658234,340.40083314)(160.90658529,340.33083456)
\curveto(161.04658206,340.27083327)(161.19658191,340.22583332)(161.35658529,340.19583456)
\curveto(161.52658158,340.16583338)(161.70158141,340.15583339)(161.88158529,340.16583456)
\curveto(162.07158104,340.18583336)(162.24658086,340.22083332)(162.40658529,340.27083456)
\curveto(162.66658044,340.35083319)(162.87158024,340.47583307)(163.02158529,340.64583456)
\curveto(163.17157994,340.82583272)(163.28657982,341.0458325)(163.36658529,341.30583456)
\curveto(163.38657972,341.37583217)(163.39657971,341.4458321)(163.39658529,341.51583456)
\curveto(163.4065797,341.59583195)(163.42157969,341.67583187)(163.44158529,341.75583456)
\lineto(163.44158529,341.89083456)
}
}
{
\newrgbcolor{curcolor}{0 0 0}
\pscustom[linestyle=none,fillstyle=solid,fillcolor=curcolor]
{
\newpath
\moveto(175.00986654,343.24083456)
\curveto(175.02985794,343.18083036)(175.03985793,343.07583047)(175.03986654,342.92583456)
\curveto(175.03985793,342.78583076)(175.03485794,342.68583086)(175.02486654,342.62583456)
\curveto(175.02485795,342.57583097)(175.01985795,342.53083101)(175.00986654,342.49083456)
\lineto(175.00986654,342.37083456)
\curveto(174.98985798,342.29083125)(174.97985799,342.21083133)(174.97986654,342.13083456)
\curveto(174.97985799,342.06083148)(174.969858,341.98583156)(174.94986654,341.90583456)
\curveto(174.94985802,341.86583168)(174.93985803,341.79583175)(174.91986654,341.69583456)
\curveto(174.88985808,341.57583197)(174.85985811,341.45083209)(174.82986654,341.32083456)
\curveto(174.80985816,341.20083234)(174.7748582,341.08583246)(174.72486654,340.97583456)
\curveto(174.54485843,340.52583302)(174.31985865,340.13583341)(174.04986654,339.80583456)
\curveto(173.77985919,339.47583407)(173.42485955,339.21583433)(172.98486654,339.02583456)
\curveto(172.89486008,338.98583456)(172.79986017,338.95583459)(172.69986654,338.93583456)
\curveto(172.60986036,338.90583464)(172.50986046,338.87583467)(172.39986654,338.84583456)
\curveto(172.33986063,338.82583472)(172.2748607,338.81583473)(172.20486654,338.81583456)
\curveto(172.14486083,338.81583473)(172.08486089,338.81083473)(172.02486654,338.80083456)
\lineto(171.88986654,338.80083456)
\curveto(171.82986114,338.78083476)(171.74986122,338.77583477)(171.64986654,338.78583456)
\curveto(171.54986142,338.78583476)(171.4698615,338.79583475)(171.40986654,338.81583456)
\lineto(171.31986654,338.81583456)
\curveto(171.2698617,338.82583472)(171.21486176,338.83583471)(171.15486654,338.84583456)
\curveto(171.09486188,338.8458347)(171.03486194,338.85083469)(170.97486654,338.86083456)
\curveto(170.78486219,338.91083463)(170.60986236,338.96083458)(170.44986654,339.01083456)
\curveto(170.28986268,339.06083448)(170.13986283,339.13083441)(169.99986654,339.22083456)
\lineto(169.81986654,339.34083456)
\curveto(169.7698632,339.38083416)(169.71986325,339.42583412)(169.66986654,339.47583456)
\lineto(169.57986654,339.53583456)
\curveto(169.54986342,339.55583399)(169.51986345,339.57083397)(169.48986654,339.58083456)
\curveto(169.39986357,339.61083393)(169.34486363,339.59083395)(169.32486654,339.52083456)
\curveto(169.2748637,339.45083409)(169.23986373,339.36583418)(169.21986654,339.26583456)
\curveto(169.20986376,339.17583437)(169.1748638,339.10583444)(169.11486654,339.05583456)
\curveto(169.05486392,339.01583453)(168.98486399,338.99083455)(168.90486654,338.98083456)
\lineto(168.63486654,338.98083456)
\lineto(167.91486654,338.98083456)
\lineto(167.68986654,338.98083456)
\curveto(167.61986535,338.97083457)(167.55486542,338.97583457)(167.49486654,338.99583456)
\curveto(167.35486562,339.0458345)(167.2748657,339.13583441)(167.25486654,339.26583456)
\curveto(167.24486573,339.40583414)(167.23986573,339.56083398)(167.23986654,339.73083456)
\lineto(167.23986654,348.88083456)
\lineto(167.23986654,349.22583456)
\curveto(167.23986573,349.3458242)(167.26486571,349.4408241)(167.31486654,349.51083456)
\curveto(167.35486562,349.58082396)(167.42486555,349.62582392)(167.52486654,349.64583456)
\curveto(167.54486543,349.65582389)(167.56486541,349.65582389)(167.58486654,349.64583456)
\curveto(167.61486536,349.6458239)(167.63986533,349.65082389)(167.65986654,349.66083456)
\lineto(168.60486654,349.66083456)
\curveto(168.78486419,349.66082388)(168.93986403,349.65082389)(169.06986654,349.63083456)
\curveto(169.19986377,349.62082392)(169.28486369,349.545824)(169.32486654,349.40583456)
\curveto(169.35486362,349.30582424)(169.36486361,349.17082437)(169.35486654,349.00083456)
\curveto(169.34486363,348.8408247)(169.33986363,348.70082484)(169.33986654,348.58083456)
\lineto(169.33986654,346.94583456)
\lineto(169.33986654,346.61583456)
\curveto(169.33986363,346.50582704)(169.34986362,346.41082713)(169.36986654,346.33083456)
\curveto(169.37986359,346.28082726)(169.38986358,346.23582731)(169.39986654,346.19583456)
\curveto(169.40986356,346.16582738)(169.43486354,346.1458274)(169.47486654,346.13583456)
\curveto(169.49486348,346.11582743)(169.51986345,346.10582744)(169.54986654,346.10583456)
\curveto(169.58986338,346.10582744)(169.61986335,346.11082743)(169.63986654,346.12083456)
\curveto(169.70986326,346.16082738)(169.7748632,346.20082734)(169.83486654,346.24083456)
\curveto(169.89486308,346.29082725)(169.95986301,346.3408272)(170.02986654,346.39083456)
\curveto(170.15986281,346.48082706)(170.29486268,346.55582699)(170.43486654,346.61583456)
\curveto(170.5748624,346.68582686)(170.72986224,346.7458268)(170.89986654,346.79583456)
\curveto(170.97986199,346.82582672)(171.05986191,346.8408267)(171.13986654,346.84083456)
\curveto(171.21986175,346.85082669)(171.29986167,346.86582668)(171.37986654,346.88583456)
\curveto(171.44986152,346.90582664)(171.52486145,346.91582663)(171.60486654,346.91583456)
\lineto(171.84486654,346.91583456)
\lineto(171.99486654,346.91583456)
\curveto(172.02486095,346.90582664)(172.05986091,346.90082664)(172.09986654,346.90083456)
\curveto(172.13986083,346.91082663)(172.17986079,346.91082663)(172.21986654,346.90083456)
\curveto(172.32986064,346.87082667)(172.42986054,346.8458267)(172.51986654,346.82583456)
\curveto(172.61986035,346.81582673)(172.71486026,346.79082675)(172.80486654,346.75083456)
\curveto(173.26485971,346.56082698)(173.63985933,346.31582723)(173.92986654,346.01583456)
\curveto(174.21985875,345.71582783)(174.46485851,345.3408282)(174.66486654,344.89083456)
\curveto(174.71485826,344.77082877)(174.75485822,344.6458289)(174.78486654,344.51583456)
\curveto(174.82485815,344.38582916)(174.86485811,344.25082929)(174.90486654,344.11083456)
\curveto(174.92485805,344.0408295)(174.93485804,343.97082957)(174.93486654,343.90083456)
\curveto(174.94485803,343.8408297)(174.95985801,343.77082977)(174.97986654,343.69083456)
\curveto(174.99985797,343.6408299)(175.00485797,343.58582996)(174.99486654,343.52583456)
\curveto(174.99485798,343.46583008)(174.99985797,343.40583014)(175.00986654,343.34583456)
\lineto(175.00986654,343.24083456)
\moveto(172.78986654,341.83083456)
\curveto(172.81986015,341.93083161)(172.84486013,342.05583149)(172.86486654,342.20583456)
\curveto(172.89486008,342.35583119)(172.90986006,342.50583104)(172.90986654,342.65583456)
\curveto(172.91986005,342.81583073)(172.91986005,342.97083057)(172.90986654,343.12083456)
\curveto(172.90986006,343.28083026)(172.89486008,343.41583013)(172.86486654,343.52583456)
\curveto(172.83486014,343.62582992)(172.81486016,343.72082982)(172.80486654,343.81083456)
\curveto(172.79486018,343.90082964)(172.7698602,343.98582956)(172.72986654,344.06583456)
\curveto(172.58986038,344.41582913)(172.38986058,344.71082883)(172.12986654,344.95083456)
\curveto(171.87986109,345.20082834)(171.50986146,345.32582822)(171.01986654,345.32583456)
\curveto(170.97986199,345.32582822)(170.94486203,345.32082822)(170.91486654,345.31083456)
\lineto(170.80986654,345.31083456)
\curveto(170.73986223,345.29082825)(170.6748623,345.27082827)(170.61486654,345.25083456)
\curveto(170.55486242,345.2408283)(170.49486248,345.22582832)(170.43486654,345.20583456)
\curveto(170.14486283,345.07582847)(169.92486305,344.89082865)(169.77486654,344.65083456)
\curveto(169.62486335,344.42082912)(169.49986347,344.15582939)(169.39986654,343.85583456)
\curveto(169.3698636,343.77582977)(169.34986362,343.69082985)(169.33986654,343.60083456)
\curveto(169.33986363,343.52083002)(169.32986364,343.4408301)(169.30986654,343.36083456)
\curveto(169.29986367,343.33083021)(169.29486368,343.28083026)(169.29486654,343.21083456)
\curveto(169.28486369,343.17083037)(169.27986369,343.13083041)(169.27986654,343.09083456)
\curveto(169.28986368,343.05083049)(169.28986368,343.01083053)(169.27986654,342.97083456)
\curveto(169.25986371,342.89083065)(169.25486372,342.78083076)(169.26486654,342.64083456)
\curveto(169.2748637,342.50083104)(169.28986368,342.40083114)(169.30986654,342.34083456)
\curveto(169.32986364,342.25083129)(169.33986363,342.16583138)(169.33986654,342.08583456)
\curveto(169.34986362,342.00583154)(169.3698636,341.92583162)(169.39986654,341.84583456)
\curveto(169.48986348,341.56583198)(169.59486338,341.32083222)(169.71486654,341.11083456)
\curveto(169.84486313,340.91083263)(170.02486295,340.7408328)(170.25486654,340.60083456)
\curveto(170.41486256,340.50083304)(170.57986239,340.43083311)(170.74986654,340.39083456)
\curveto(170.7698622,340.39083315)(170.78986218,340.38583316)(170.80986654,340.37583456)
\lineto(170.89986654,340.37583456)
\curveto(170.92986204,340.36583318)(170.97986199,340.35583319)(171.04986654,340.34583456)
\curveto(171.11986185,340.3458332)(171.17986179,340.35083319)(171.22986654,340.36083456)
\curveto(171.32986164,340.38083316)(171.41986155,340.39583315)(171.49986654,340.40583456)
\curveto(171.58986138,340.42583312)(171.6748613,340.45083309)(171.75486654,340.48083456)
\curveto(172.03486094,340.61083293)(172.24986072,340.79083275)(172.39986654,341.02083456)
\curveto(172.55986041,341.25083229)(172.68986028,341.52083202)(172.78986654,341.83083456)
}
}
{
\newrgbcolor{curcolor}{0 0 0}
\pscustom[linestyle=none,fillstyle=solid,fillcolor=curcolor]
{
\newpath
\moveto(178.47978842,349.57083456)
\curveto(178.54978547,349.49082405)(178.58478543,349.37082417)(178.58478842,349.21083456)
\lineto(178.58478842,348.74583456)
\lineto(178.58478842,348.34083456)
\curveto(178.58478543,348.20082534)(178.54978547,348.10582544)(178.47978842,348.05583456)
\curveto(178.4197856,348.00582554)(178.33978568,347.97582557)(178.23978842,347.96583456)
\curveto(178.14978587,347.95582559)(178.04978597,347.95082559)(177.93978842,347.95083456)
\lineto(177.09978842,347.95083456)
\curveto(176.98978703,347.95082559)(176.88978713,347.95582559)(176.79978842,347.96583456)
\curveto(176.7197873,347.97582557)(176.64978737,348.00582554)(176.58978842,348.05583456)
\curveto(176.54978747,348.08582546)(176.5197875,348.1408254)(176.49978842,348.22083456)
\curveto(176.48978753,348.31082523)(176.47978754,348.40582514)(176.46978842,348.50583456)
\lineto(176.46978842,348.83583456)
\curveto(176.47978754,348.9458246)(176.48478753,349.0408245)(176.48478842,349.12083456)
\lineto(176.48478842,349.33083456)
\curveto(176.49478752,349.40082414)(176.5147875,349.46082408)(176.54478842,349.51083456)
\curveto(176.56478745,349.55082399)(176.58978743,349.58082396)(176.61978842,349.60083456)
\lineto(176.73978842,349.66083456)
\curveto(176.75978726,349.66082388)(176.78478723,349.66082388)(176.81478842,349.66083456)
\curveto(176.84478717,349.67082387)(176.86978715,349.67582387)(176.88978842,349.67583456)
\lineto(177.98478842,349.67583456)
\curveto(178.08478593,349.67582387)(178.17978584,349.67082387)(178.26978842,349.66083456)
\curveto(178.35978566,349.65082389)(178.42978559,349.62082392)(178.47978842,349.57083456)
\moveto(178.58478842,339.80583456)
\curveto(178.58478543,339.60583394)(178.57978544,339.43583411)(178.56978842,339.29583456)
\curveto(178.55978546,339.15583439)(178.46978555,339.06083448)(178.29978842,339.01083456)
\curveto(178.23978578,338.99083455)(178.17478584,338.98083456)(178.10478842,338.98083456)
\curveto(178.03478598,338.99083455)(177.95978606,338.99583455)(177.87978842,338.99583456)
\lineto(177.03978842,338.99583456)
\curveto(176.94978707,338.99583455)(176.85978716,339.00083454)(176.76978842,339.01083456)
\curveto(176.68978733,339.02083452)(176.62978739,339.05083449)(176.58978842,339.10083456)
\curveto(176.52978749,339.17083437)(176.49478752,339.25583429)(176.48478842,339.35583456)
\lineto(176.48478842,339.70083456)
\lineto(176.48478842,346.03083456)
\lineto(176.48478842,346.33083456)
\curveto(176.48478753,346.43082711)(176.50478751,346.51082703)(176.54478842,346.57083456)
\curveto(176.60478741,346.6408269)(176.68978733,346.68582686)(176.79978842,346.70583456)
\curveto(176.8197872,346.71582683)(176.84478717,346.71582683)(176.87478842,346.70583456)
\curveto(176.9147871,346.70582684)(176.94478707,346.71082683)(176.96478842,346.72083456)
\lineto(177.71478842,346.72083456)
\lineto(177.90978842,346.72083456)
\curveto(177.98978603,346.73082681)(178.05478596,346.73082681)(178.10478842,346.72083456)
\lineto(178.22478842,346.72083456)
\curveto(178.28478573,346.70082684)(178.33978568,346.68582686)(178.38978842,346.67583456)
\curveto(178.43978558,346.66582688)(178.47978554,346.63582691)(178.50978842,346.58583456)
\curveto(178.54978547,346.53582701)(178.56978545,346.46582708)(178.56978842,346.37583456)
\curveto(178.57978544,346.28582726)(178.58478543,346.19082735)(178.58478842,346.09083456)
\lineto(178.58478842,339.80583456)
}
}
{
\newrgbcolor{curcolor}{0 0 0}
\pscustom[linestyle=none,fillstyle=solid,fillcolor=curcolor]
{
\newpath
\moveto(180.75697592,349.67583456)
\lineto(181.85197592,349.67583456)
\curveto(181.95197343,349.67582387)(182.04697334,349.67082387)(182.13697592,349.66083456)
\curveto(182.22697316,349.65082389)(182.29697309,349.62082392)(182.34697592,349.57083456)
\curveto(182.40697298,349.50082404)(182.43697295,349.40582414)(182.43697592,349.28583456)
\curveto(182.44697294,349.17582437)(182.45197293,349.06082448)(182.45197592,348.94083456)
\lineto(182.45197592,347.60583456)
\lineto(182.45197592,342.22083456)
\lineto(182.45197592,339.92583456)
\lineto(182.45197592,339.50583456)
\curveto(182.46197292,339.35583419)(182.44197294,339.2408343)(182.39197592,339.16083456)
\curveto(182.34197304,339.08083446)(182.25197313,339.02583452)(182.12197592,338.99583456)
\curveto(182.06197332,338.97583457)(181.99197339,338.97083457)(181.91197592,338.98083456)
\curveto(181.84197354,338.99083455)(181.77197361,338.99583455)(181.70197592,338.99583456)
\lineto(180.98197592,338.99583456)
\curveto(180.87197451,338.99583455)(180.77197461,339.00083454)(180.68197592,339.01083456)
\curveto(180.59197479,339.02083452)(180.51697487,339.05083449)(180.45697592,339.10083456)
\curveto(180.39697499,339.15083439)(180.36197502,339.22583432)(180.35197592,339.32583456)
\lineto(180.35197592,339.65583456)
\lineto(180.35197592,340.99083456)
\lineto(180.35197592,346.61583456)
\lineto(180.35197592,348.65583456)
\curveto(180.35197503,348.78582476)(180.34697504,348.9408246)(180.33697592,349.12083456)
\curveto(180.33697505,349.30082424)(180.36197502,349.43082411)(180.41197592,349.51083456)
\curveto(180.43197495,349.55082399)(180.45697493,349.58082396)(180.48697592,349.60083456)
\lineto(180.60697592,349.66083456)
\curveto(180.62697476,349.66082388)(180.65197473,349.66082388)(180.68197592,349.66083456)
\curveto(180.71197467,349.67082387)(180.73697465,349.67582387)(180.75697592,349.67583456)
}
}
{
\newrgbcolor{curcolor}{0 0 0}
\pscustom[linestyle=none,fillstyle=solid,fillcolor=curcolor]
{
\newpath
\moveto(186.21416342,349.57083456)
\curveto(186.28416047,349.49082405)(186.31916043,349.37082417)(186.31916342,349.21083456)
\lineto(186.31916342,348.74583456)
\lineto(186.31916342,348.34083456)
\curveto(186.31916043,348.20082534)(186.28416047,348.10582544)(186.21416342,348.05583456)
\curveto(186.1541606,348.00582554)(186.07416068,347.97582557)(185.97416342,347.96583456)
\curveto(185.88416087,347.95582559)(185.78416097,347.95082559)(185.67416342,347.95083456)
\lineto(184.83416342,347.95083456)
\curveto(184.72416203,347.95082559)(184.62416213,347.95582559)(184.53416342,347.96583456)
\curveto(184.4541623,347.97582557)(184.38416237,348.00582554)(184.32416342,348.05583456)
\curveto(184.28416247,348.08582546)(184.2541625,348.1408254)(184.23416342,348.22083456)
\curveto(184.22416253,348.31082523)(184.21416254,348.40582514)(184.20416342,348.50583456)
\lineto(184.20416342,348.83583456)
\curveto(184.21416254,348.9458246)(184.21916253,349.0408245)(184.21916342,349.12083456)
\lineto(184.21916342,349.33083456)
\curveto(184.22916252,349.40082414)(184.2491625,349.46082408)(184.27916342,349.51083456)
\curveto(184.29916245,349.55082399)(184.32416243,349.58082396)(184.35416342,349.60083456)
\lineto(184.47416342,349.66083456)
\curveto(184.49416226,349.66082388)(184.51916223,349.66082388)(184.54916342,349.66083456)
\curveto(184.57916217,349.67082387)(184.60416215,349.67582387)(184.62416342,349.67583456)
\lineto(185.71916342,349.67583456)
\curveto(185.81916093,349.67582387)(185.91416084,349.67082387)(186.00416342,349.66083456)
\curveto(186.09416066,349.65082389)(186.16416059,349.62082392)(186.21416342,349.57083456)
\moveto(186.31916342,339.80583456)
\curveto(186.31916043,339.60583394)(186.31416044,339.43583411)(186.30416342,339.29583456)
\curveto(186.29416046,339.15583439)(186.20416055,339.06083448)(186.03416342,339.01083456)
\curveto(185.97416078,338.99083455)(185.90916084,338.98083456)(185.83916342,338.98083456)
\curveto(185.76916098,338.99083455)(185.69416106,338.99583455)(185.61416342,338.99583456)
\lineto(184.77416342,338.99583456)
\curveto(184.68416207,338.99583455)(184.59416216,339.00083454)(184.50416342,339.01083456)
\curveto(184.42416233,339.02083452)(184.36416239,339.05083449)(184.32416342,339.10083456)
\curveto(184.26416249,339.17083437)(184.22916252,339.25583429)(184.21916342,339.35583456)
\lineto(184.21916342,339.70083456)
\lineto(184.21916342,346.03083456)
\lineto(184.21916342,346.33083456)
\curveto(184.21916253,346.43082711)(184.23916251,346.51082703)(184.27916342,346.57083456)
\curveto(184.33916241,346.6408269)(184.42416233,346.68582686)(184.53416342,346.70583456)
\curveto(184.5541622,346.71582683)(184.57916217,346.71582683)(184.60916342,346.70583456)
\curveto(184.6491621,346.70582684)(184.67916207,346.71082683)(184.69916342,346.72083456)
\lineto(185.44916342,346.72083456)
\lineto(185.64416342,346.72083456)
\curveto(185.72416103,346.73082681)(185.78916096,346.73082681)(185.83916342,346.72083456)
\lineto(185.95916342,346.72083456)
\curveto(186.01916073,346.70082684)(186.07416068,346.68582686)(186.12416342,346.67583456)
\curveto(186.17416058,346.66582688)(186.21416054,346.63582691)(186.24416342,346.58583456)
\curveto(186.28416047,346.53582701)(186.30416045,346.46582708)(186.30416342,346.37583456)
\curveto(186.31416044,346.28582726)(186.31916043,346.19082735)(186.31916342,346.09083456)
\lineto(186.31916342,339.80583456)
}
}
{
\newrgbcolor{curcolor}{0 0 0}
\pscustom[linestyle=none,fillstyle=solid,fillcolor=curcolor]
{
\newpath
\moveto(195.57135092,339.83583456)
\lineto(195.57135092,339.41583456)
\curveto(195.57134255,339.28583426)(195.54134258,339.18083436)(195.48135092,339.10083456)
\curveto(195.43134269,339.05083449)(195.36634275,339.01583453)(195.28635092,338.99583456)
\curveto(195.20634291,338.98583456)(195.116343,338.98083456)(195.01635092,338.98083456)
\lineto(194.19135092,338.98083456)
\lineto(193.90635092,338.98083456)
\curveto(193.82634429,338.99083455)(193.76134436,339.01583453)(193.71135092,339.05583456)
\curveto(193.64134448,339.10583444)(193.60134452,339.17083437)(193.59135092,339.25083456)
\curveto(193.58134454,339.33083421)(193.56134456,339.41083413)(193.53135092,339.49083456)
\curveto(193.51134461,339.51083403)(193.49134463,339.52583402)(193.47135092,339.53583456)
\curveto(193.46134466,339.55583399)(193.44634467,339.57583397)(193.42635092,339.59583456)
\curveto(193.3163448,339.59583395)(193.23634488,339.57083397)(193.18635092,339.52083456)
\lineto(193.03635092,339.37083456)
\curveto(192.96634515,339.32083422)(192.90134522,339.27583427)(192.84135092,339.23583456)
\curveto(192.78134534,339.20583434)(192.7163454,339.16583438)(192.64635092,339.11583456)
\curveto(192.60634551,339.09583445)(192.56134556,339.07583447)(192.51135092,339.05583456)
\curveto(192.47134565,339.03583451)(192.42634569,339.01583453)(192.37635092,338.99583456)
\curveto(192.23634588,338.9458346)(192.08634603,338.90083464)(191.92635092,338.86083456)
\curveto(191.87634624,338.8408347)(191.83134629,338.83083471)(191.79135092,338.83083456)
\curveto(191.75134637,338.83083471)(191.71134641,338.82583472)(191.67135092,338.81583456)
\lineto(191.53635092,338.81583456)
\curveto(191.50634661,338.80583474)(191.46634665,338.80083474)(191.41635092,338.80083456)
\lineto(191.28135092,338.80083456)
\curveto(191.2213469,338.78083476)(191.13134699,338.77583477)(191.01135092,338.78583456)
\curveto(190.89134723,338.78583476)(190.80634731,338.79583475)(190.75635092,338.81583456)
\curveto(190.68634743,338.83583471)(190.6213475,338.8458347)(190.56135092,338.84583456)
\curveto(190.51134761,338.83583471)(190.45634766,338.8408347)(190.39635092,338.86083456)
\lineto(190.03635092,338.98083456)
\curveto(189.92634819,339.01083453)(189.8163483,339.05083449)(189.70635092,339.10083456)
\curveto(189.35634876,339.25083429)(189.04134908,339.48083406)(188.76135092,339.79083456)
\curveto(188.49134963,340.11083343)(188.27634984,340.4458331)(188.11635092,340.79583456)
\curveto(188.06635005,340.90583264)(188.02635009,341.01083253)(187.99635092,341.11083456)
\curveto(187.96635015,341.22083232)(187.93135019,341.33083221)(187.89135092,341.44083456)
\curveto(187.88135024,341.48083206)(187.87635024,341.51583203)(187.87635092,341.54583456)
\curveto(187.87635024,341.58583196)(187.86635025,341.63083191)(187.84635092,341.68083456)
\curveto(187.82635029,341.76083178)(187.80635031,341.8458317)(187.78635092,341.93583456)
\curveto(187.77635034,342.03583151)(187.76135036,342.13583141)(187.74135092,342.23583456)
\curveto(187.73135039,342.26583128)(187.72635039,342.30083124)(187.72635092,342.34083456)
\curveto(187.73635038,342.38083116)(187.73635038,342.41583113)(187.72635092,342.44583456)
\lineto(187.72635092,342.58083456)
\curveto(187.72635039,342.63083091)(187.7213504,342.68083086)(187.71135092,342.73083456)
\curveto(187.70135042,342.78083076)(187.69635042,342.83583071)(187.69635092,342.89583456)
\curveto(187.69635042,342.96583058)(187.70135042,343.02083052)(187.71135092,343.06083456)
\curveto(187.7213504,343.11083043)(187.72635039,343.15583039)(187.72635092,343.19583456)
\lineto(187.72635092,343.34583456)
\curveto(187.73635038,343.39583015)(187.73635038,343.4408301)(187.72635092,343.48083456)
\curveto(187.72635039,343.53083001)(187.73635038,343.58082996)(187.75635092,343.63083456)
\curveto(187.77635034,343.7408298)(187.79135033,343.8458297)(187.80135092,343.94583456)
\curveto(187.8213503,344.0458295)(187.84635027,344.1458294)(187.87635092,344.24583456)
\curveto(187.9163502,344.36582918)(187.95135017,344.48082906)(187.98135092,344.59083456)
\curveto(188.01135011,344.70082884)(188.05135007,344.81082873)(188.10135092,344.92083456)
\curveto(188.24134988,345.22082832)(188.4163497,345.50582804)(188.62635092,345.77583456)
\curveto(188.64634947,345.80582774)(188.67134945,345.83082771)(188.70135092,345.85083456)
\curveto(188.74134938,345.88082766)(188.77134935,345.91082763)(188.79135092,345.94083456)
\curveto(188.83134929,345.99082755)(188.87134925,346.03582751)(188.91135092,346.07583456)
\curveto(188.95134917,346.11582743)(188.99634912,346.15582739)(189.04635092,346.19583456)
\curveto(189.08634903,346.21582733)(189.121349,346.2408273)(189.15135092,346.27083456)
\curveto(189.18134894,346.31082723)(189.2163489,346.3408272)(189.25635092,346.36083456)
\curveto(189.50634861,346.53082701)(189.79634832,346.67082687)(190.12635092,346.78083456)
\curveto(190.19634792,346.80082674)(190.26634785,346.81582673)(190.33635092,346.82583456)
\curveto(190.4163477,346.83582671)(190.49634762,346.85082669)(190.57635092,346.87083456)
\curveto(190.64634747,346.89082665)(190.73634738,346.90082664)(190.84635092,346.90083456)
\curveto(190.95634716,346.91082663)(191.06634705,346.91582663)(191.17635092,346.91583456)
\curveto(191.28634683,346.91582663)(191.39134673,346.91082663)(191.49135092,346.90083456)
\curveto(191.60134652,346.89082665)(191.69134643,346.87582667)(191.76135092,346.85583456)
\curveto(191.91134621,346.80582674)(192.05634606,346.76082678)(192.19635092,346.72083456)
\curveto(192.33634578,346.68082686)(192.46634565,346.62582692)(192.58635092,346.55583456)
\curveto(192.65634546,346.50582704)(192.7213454,346.45582709)(192.78135092,346.40583456)
\curveto(192.84134528,346.36582718)(192.90634521,346.32082722)(192.97635092,346.27083456)
\curveto(193.0163451,346.2408273)(193.07134505,346.20082734)(193.14135092,346.15083456)
\curveto(193.2213449,346.10082744)(193.29634482,346.10082744)(193.36635092,346.15083456)
\curveto(193.40634471,346.17082737)(193.42634469,346.20582734)(193.42635092,346.25583456)
\curveto(193.42634469,346.30582724)(193.43634468,346.35582719)(193.45635092,346.40583456)
\lineto(193.45635092,346.55583456)
\curveto(193.46634465,346.58582696)(193.47134465,346.62082692)(193.47135092,346.66083456)
\lineto(193.47135092,346.78083456)
\lineto(193.47135092,348.82083456)
\curveto(193.47134465,348.93082461)(193.46634465,349.05082449)(193.45635092,349.18083456)
\curveto(193.45634466,349.32082422)(193.48134464,349.42582412)(193.53135092,349.49583456)
\curveto(193.57134455,349.57582397)(193.64634447,349.62582392)(193.75635092,349.64583456)
\curveto(193.77634434,349.65582389)(193.79634432,349.65582389)(193.81635092,349.64583456)
\curveto(193.83634428,349.6458239)(193.85634426,349.65082389)(193.87635092,349.66083456)
\lineto(194.94135092,349.66083456)
\curveto(195.06134306,349.66082388)(195.17134295,349.65582389)(195.27135092,349.64583456)
\curveto(195.37134275,349.63582391)(195.44634267,349.59582395)(195.49635092,349.52583456)
\curveto(195.54634257,349.4458241)(195.57134255,349.3408242)(195.57135092,349.21083456)
\lineto(195.57135092,348.85083456)
\lineto(195.57135092,339.83583456)
\moveto(193.53135092,342.77583456)
\curveto(193.54134458,342.81583073)(193.54134458,342.85583069)(193.53135092,342.89583456)
\lineto(193.53135092,343.03083456)
\curveto(193.53134459,343.13083041)(193.52634459,343.23083031)(193.51635092,343.33083456)
\curveto(193.50634461,343.43083011)(193.49134463,343.52083002)(193.47135092,343.60083456)
\curveto(193.45134467,343.71082983)(193.43134469,343.81082973)(193.41135092,343.90083456)
\curveto(193.40134472,343.99082955)(193.37634474,344.07582947)(193.33635092,344.15583456)
\curveto(193.19634492,344.51582903)(192.99134513,344.80082874)(192.72135092,345.01083456)
\curveto(192.46134566,345.22082832)(192.08134604,345.32582822)(191.58135092,345.32583456)
\curveto(191.5213466,345.32582822)(191.44134668,345.31582823)(191.34135092,345.29583456)
\curveto(191.26134686,345.27582827)(191.18634693,345.25582829)(191.11635092,345.23583456)
\curveto(191.05634706,345.22582832)(190.99634712,345.20582834)(190.93635092,345.17583456)
\curveto(190.66634745,345.06582848)(190.45634766,344.89582865)(190.30635092,344.66583456)
\curveto(190.15634796,344.43582911)(190.03634808,344.17582937)(189.94635092,343.88583456)
\curveto(189.9163482,343.78582976)(189.89634822,343.68582986)(189.88635092,343.58583456)
\curveto(189.87634824,343.48583006)(189.85634826,343.38083016)(189.82635092,343.27083456)
\lineto(189.82635092,343.06083456)
\curveto(189.80634831,342.97083057)(189.80134832,342.8458307)(189.81135092,342.68583456)
\curveto(189.8213483,342.53583101)(189.83634828,342.42583112)(189.85635092,342.35583456)
\lineto(189.85635092,342.26583456)
\curveto(189.86634825,342.2458313)(189.87134825,342.22583132)(189.87135092,342.20583456)
\curveto(189.89134823,342.12583142)(189.90634821,342.05083149)(189.91635092,341.98083456)
\curveto(189.93634818,341.91083163)(189.95634816,341.83583171)(189.97635092,341.75583456)
\curveto(190.14634797,341.23583231)(190.43634768,340.85083269)(190.84635092,340.60083456)
\curveto(190.97634714,340.51083303)(191.15634696,340.4408331)(191.38635092,340.39083456)
\curveto(191.42634669,340.38083316)(191.48634663,340.37583317)(191.56635092,340.37583456)
\curveto(191.59634652,340.36583318)(191.64134648,340.35583319)(191.70135092,340.34583456)
\curveto(191.77134635,340.3458332)(191.82634629,340.35083319)(191.86635092,340.36083456)
\curveto(191.94634617,340.38083316)(192.02634609,340.39583315)(192.10635092,340.40583456)
\curveto(192.18634593,340.41583313)(192.26634585,340.43583311)(192.34635092,340.46583456)
\curveto(192.59634552,340.57583297)(192.79634532,340.71583283)(192.94635092,340.88583456)
\curveto(193.09634502,341.05583249)(193.22634489,341.27083227)(193.33635092,341.53083456)
\curveto(193.37634474,341.62083192)(193.40634471,341.71083183)(193.42635092,341.80083456)
\curveto(193.44634467,341.90083164)(193.46634465,342.00583154)(193.48635092,342.11583456)
\curveto(193.49634462,342.16583138)(193.49634462,342.21083133)(193.48635092,342.25083456)
\curveto(193.48634463,342.30083124)(193.49634462,342.35083119)(193.51635092,342.40083456)
\curveto(193.52634459,342.43083111)(193.53134459,342.46583108)(193.53135092,342.50583456)
\lineto(193.53135092,342.64083456)
\lineto(193.53135092,342.77583456)
}
}
{
\newrgbcolor{curcolor}{0 0 0}
\pscustom[linestyle=none,fillstyle=solid,fillcolor=curcolor]
{
\newpath
\moveto(204.20127279,339.58083456)
\curveto(204.22126494,339.47083407)(204.23126493,339.36083418)(204.23127279,339.25083456)
\curveto(204.24126492,339.1408344)(204.19126497,339.06583448)(204.08127279,339.02583456)
\curveto(204.02126514,338.99583455)(203.95126521,338.98083456)(203.87127279,338.98083456)
\lineto(203.63127279,338.98083456)
\lineto(202.82127279,338.98083456)
\lineto(202.55127279,338.98083456)
\curveto(202.47126669,338.99083455)(202.40626676,339.01583453)(202.35627279,339.05583456)
\curveto(202.28626688,339.09583445)(202.23126693,339.15083439)(202.19127279,339.22083456)
\curveto(202.161267,339.30083424)(202.11626705,339.36583418)(202.05627279,339.41583456)
\curveto(202.03626713,339.43583411)(202.01126715,339.45083409)(201.98127279,339.46083456)
\curveto(201.95126721,339.48083406)(201.91126725,339.48583406)(201.86127279,339.47583456)
\curveto(201.81126735,339.45583409)(201.7612674,339.43083411)(201.71127279,339.40083456)
\curveto(201.67126749,339.37083417)(201.62626754,339.3458342)(201.57627279,339.32583456)
\curveto(201.52626764,339.28583426)(201.47126769,339.25083429)(201.41127279,339.22083456)
\lineto(201.23127279,339.13083456)
\curveto(201.10126806,339.07083447)(200.9662682,339.02083452)(200.82627279,338.98083456)
\curveto(200.68626848,338.95083459)(200.54126862,338.91583463)(200.39127279,338.87583456)
\curveto(200.32126884,338.85583469)(200.25126891,338.8458347)(200.18127279,338.84583456)
\curveto(200.12126904,338.83583471)(200.05626911,338.82583472)(199.98627279,338.81583456)
\lineto(199.89627279,338.81583456)
\curveto(199.8662693,338.80583474)(199.83626933,338.80083474)(199.80627279,338.80083456)
\lineto(199.64127279,338.80083456)
\curveto(199.54126962,338.78083476)(199.44126972,338.78083476)(199.34127279,338.80083456)
\lineto(199.20627279,338.80083456)
\curveto(199.13627003,338.82083472)(199.0662701,338.83083471)(198.99627279,338.83083456)
\curveto(198.93627023,338.82083472)(198.87627029,338.82583472)(198.81627279,338.84583456)
\curveto(198.71627045,338.86583468)(198.62127054,338.88583466)(198.53127279,338.90583456)
\curveto(198.44127072,338.91583463)(198.35627081,338.9408346)(198.27627279,338.98083456)
\curveto(197.98627118,339.09083445)(197.73627143,339.23083431)(197.52627279,339.40083456)
\curveto(197.32627184,339.58083396)(197.166272,339.81583373)(197.04627279,340.10583456)
\curveto(197.01627215,340.17583337)(196.98627218,340.25083329)(196.95627279,340.33083456)
\curveto(196.93627223,340.41083313)(196.91627225,340.49583305)(196.89627279,340.58583456)
\curveto(196.87627229,340.63583291)(196.8662723,340.68583286)(196.86627279,340.73583456)
\curveto(196.87627229,340.78583276)(196.87627229,340.83583271)(196.86627279,340.88583456)
\curveto(196.85627231,340.91583263)(196.84627232,340.97583257)(196.83627279,341.06583456)
\curveto(196.83627233,341.16583238)(196.84127232,341.23583231)(196.85127279,341.27583456)
\curveto(196.87127229,341.37583217)(196.88127228,341.46083208)(196.88127279,341.53083456)
\lineto(196.97127279,341.86083456)
\curveto(197.00127216,341.98083156)(197.04127212,342.08583146)(197.09127279,342.17583456)
\curveto(197.2612719,342.46583108)(197.45627171,342.68583086)(197.67627279,342.83583456)
\curveto(197.89627127,342.98583056)(198.17627099,343.11583043)(198.51627279,343.22583456)
\curveto(198.64627052,343.27583027)(198.78127038,343.31083023)(198.92127279,343.33083456)
\curveto(199.0612701,343.35083019)(199.20126996,343.37583017)(199.34127279,343.40583456)
\curveto(199.42126974,343.42583012)(199.50626966,343.43583011)(199.59627279,343.43583456)
\curveto(199.68626948,343.4458301)(199.77626939,343.46083008)(199.86627279,343.48083456)
\curveto(199.93626923,343.50083004)(200.00626916,343.50583004)(200.07627279,343.49583456)
\curveto(200.14626902,343.49583005)(200.22126894,343.50583004)(200.30127279,343.52583456)
\curveto(200.37126879,343.54583)(200.44126872,343.55582999)(200.51127279,343.55583456)
\curveto(200.58126858,343.55582999)(200.65626851,343.56582998)(200.73627279,343.58583456)
\curveto(200.94626822,343.63582991)(201.13626803,343.67582987)(201.30627279,343.70583456)
\curveto(201.48626768,343.7458298)(201.64626752,343.83582971)(201.78627279,343.97583456)
\curveto(201.87626729,344.06582948)(201.93626723,344.16582938)(201.96627279,344.27583456)
\curveto(201.97626719,344.30582924)(201.97626719,344.33082921)(201.96627279,344.35083456)
\curveto(201.9662672,344.37082917)(201.97126719,344.39082915)(201.98127279,344.41083456)
\curveto(201.99126717,344.43082911)(201.99626717,344.46082908)(201.99627279,344.50083456)
\lineto(201.99627279,344.59083456)
\lineto(201.96627279,344.71083456)
\curveto(201.9662672,344.75082879)(201.9612672,344.78582876)(201.95127279,344.81583456)
\curveto(201.85126731,345.11582843)(201.64126752,345.32082822)(201.32127279,345.43083456)
\curveto(201.23126793,345.46082808)(201.12126804,345.48082806)(200.99127279,345.49083456)
\curveto(200.87126829,345.51082803)(200.74626842,345.51582803)(200.61627279,345.50583456)
\curveto(200.48626868,345.50582804)(200.3612688,345.49582805)(200.24127279,345.47583456)
\curveto(200.12126904,345.45582809)(200.01626915,345.43082811)(199.92627279,345.40083456)
\curveto(199.8662693,345.38082816)(199.80626936,345.35082819)(199.74627279,345.31083456)
\curveto(199.69626947,345.28082826)(199.64626952,345.2458283)(199.59627279,345.20583456)
\curveto(199.54626962,345.16582838)(199.49126967,345.11082843)(199.43127279,345.04083456)
\curveto(199.38126978,344.97082857)(199.34626982,344.90582864)(199.32627279,344.84583456)
\curveto(199.27626989,344.7458288)(199.23126993,344.65082889)(199.19127279,344.56083456)
\curveto(199.16127,344.47082907)(199.09127007,344.41082913)(198.98127279,344.38083456)
\curveto(198.90127026,344.36082918)(198.81627035,344.35082919)(198.72627279,344.35083456)
\lineto(198.45627279,344.35083456)
\lineto(197.88627279,344.35083456)
\curveto(197.83627133,344.35082919)(197.78627138,344.3458292)(197.73627279,344.33583456)
\curveto(197.68627148,344.33582921)(197.64127152,344.3408292)(197.60127279,344.35083456)
\lineto(197.46627279,344.35083456)
\curveto(197.44627172,344.36082918)(197.42127174,344.36582918)(197.39127279,344.36583456)
\curveto(197.3612718,344.36582918)(197.33627183,344.37582917)(197.31627279,344.39583456)
\curveto(197.23627193,344.41582913)(197.18127198,344.48082906)(197.15127279,344.59083456)
\curveto(197.14127202,344.6408289)(197.14127202,344.69082885)(197.15127279,344.74083456)
\curveto(197.161272,344.79082875)(197.17127199,344.83582871)(197.18127279,344.87583456)
\curveto(197.21127195,344.98582856)(197.24127192,345.08582846)(197.27127279,345.17583456)
\curveto(197.31127185,345.27582827)(197.35627181,345.36582818)(197.40627279,345.44583456)
\lineto(197.49627279,345.59583456)
\lineto(197.58627279,345.74583456)
\curveto(197.6662715,345.85582769)(197.7662714,345.96082758)(197.88627279,346.06083456)
\curveto(197.90627126,346.07082747)(197.93627123,346.09582745)(197.97627279,346.13583456)
\curveto(198.02627114,346.17582737)(198.07127109,346.21082733)(198.11127279,346.24083456)
\curveto(198.15127101,346.27082727)(198.19627097,346.30082724)(198.24627279,346.33083456)
\curveto(198.41627075,346.4408271)(198.59627057,346.52582702)(198.78627279,346.58583456)
\curveto(198.97627019,346.65582689)(199.17126999,346.72082682)(199.37127279,346.78083456)
\curveto(199.49126967,346.81082673)(199.61626955,346.83082671)(199.74627279,346.84083456)
\curveto(199.87626929,346.85082669)(200.00626916,346.87082667)(200.13627279,346.90083456)
\curveto(200.17626899,346.91082663)(200.23626893,346.91082663)(200.31627279,346.90083456)
\curveto(200.40626876,346.89082665)(200.4612687,346.89582665)(200.48127279,346.91583456)
\curveto(200.89126827,346.92582662)(201.28126788,346.91082663)(201.65127279,346.87083456)
\curveto(202.03126713,346.83082671)(202.37126679,346.75582679)(202.67127279,346.64583456)
\curveto(202.98126618,346.53582701)(203.24626592,346.38582716)(203.46627279,346.19583456)
\curveto(203.68626548,346.01582753)(203.85626531,345.78082776)(203.97627279,345.49083456)
\curveto(204.04626512,345.32082822)(204.08626508,345.12582842)(204.09627279,344.90583456)
\curveto(204.10626506,344.68582886)(204.11126505,344.46082908)(204.11127279,344.23083456)
\lineto(204.11127279,340.88583456)
\lineto(204.11127279,340.30083456)
\curveto(204.11126505,340.11083343)(204.13126503,339.93583361)(204.17127279,339.77583456)
\curveto(204.18126498,339.7458338)(204.18626498,339.71083383)(204.18627279,339.67083456)
\curveto(204.18626498,339.6408339)(204.19126497,339.61083393)(204.20127279,339.58083456)
\moveto(201.99627279,341.89083456)
\curveto(202.00626716,341.9408316)(202.01126715,341.99583155)(202.01127279,342.05583456)
\curveto(202.01126715,342.12583142)(202.00626716,342.18583136)(201.99627279,342.23583456)
\curveto(201.97626719,342.29583125)(201.9662672,342.35083119)(201.96627279,342.40083456)
\curveto(201.9662672,342.45083109)(201.94626722,342.49083105)(201.90627279,342.52083456)
\curveto(201.85626731,342.56083098)(201.78126738,342.58083096)(201.68127279,342.58083456)
\curveto(201.64126752,342.57083097)(201.60626756,342.56083098)(201.57627279,342.55083456)
\curveto(201.54626762,342.55083099)(201.51126765,342.545831)(201.47127279,342.53583456)
\curveto(201.40126776,342.51583103)(201.32626784,342.50083104)(201.24627279,342.49083456)
\curveto(201.166268,342.48083106)(201.08626808,342.46583108)(201.00627279,342.44583456)
\curveto(200.97626819,342.43583111)(200.93126823,342.43083111)(200.87127279,342.43083456)
\curveto(200.74126842,342.40083114)(200.61126855,342.38083116)(200.48127279,342.37083456)
\curveto(200.35126881,342.36083118)(200.22626894,342.33583121)(200.10627279,342.29583456)
\curveto(200.02626914,342.27583127)(199.95126921,342.25583129)(199.88127279,342.23583456)
\curveto(199.81126935,342.22583132)(199.74126942,342.20583134)(199.67127279,342.17583456)
\curveto(199.4612697,342.08583146)(199.28126988,341.95083159)(199.13127279,341.77083456)
\curveto(198.99127017,341.59083195)(198.94127022,341.3408322)(198.98127279,341.02083456)
\curveto(199.00127016,340.85083269)(199.05627011,340.71083283)(199.14627279,340.60083456)
\curveto(199.21626995,340.49083305)(199.32126984,340.40083314)(199.46127279,340.33083456)
\curveto(199.60126956,340.27083327)(199.75126941,340.22583332)(199.91127279,340.19583456)
\curveto(200.08126908,340.16583338)(200.25626891,340.15583339)(200.43627279,340.16583456)
\curveto(200.62626854,340.18583336)(200.80126836,340.22083332)(200.96127279,340.27083456)
\curveto(201.22126794,340.35083319)(201.42626774,340.47583307)(201.57627279,340.64583456)
\curveto(201.72626744,340.82583272)(201.84126732,341.0458325)(201.92127279,341.30583456)
\curveto(201.94126722,341.37583217)(201.95126721,341.4458321)(201.95127279,341.51583456)
\curveto(201.9612672,341.59583195)(201.97626719,341.67583187)(201.99627279,341.75583456)
\lineto(201.99627279,341.89083456)
}
}
{
\newrgbcolor{curcolor}{0 0 0}
\pscustom[linestyle=none,fillstyle=solid,fillcolor=curcolor]
{
\newpath
\moveto(213.35455404,339.83583456)
\lineto(213.35455404,339.41583456)
\curveto(213.35454567,339.28583426)(213.3245457,339.18083436)(213.26455404,339.10083456)
\curveto(213.21454581,339.05083449)(213.14954588,339.01583453)(213.06955404,338.99583456)
\curveto(212.98954604,338.98583456)(212.89954613,338.98083456)(212.79955404,338.98083456)
\lineto(211.97455404,338.98083456)
\lineto(211.68955404,338.98083456)
\curveto(211.60954742,338.99083455)(211.54454748,339.01583453)(211.49455404,339.05583456)
\curveto(211.4245476,339.10583444)(211.38454764,339.17083437)(211.37455404,339.25083456)
\curveto(211.36454766,339.33083421)(211.34454768,339.41083413)(211.31455404,339.49083456)
\curveto(211.29454773,339.51083403)(211.27454775,339.52583402)(211.25455404,339.53583456)
\curveto(211.24454778,339.55583399)(211.2295478,339.57583397)(211.20955404,339.59583456)
\curveto(211.09954793,339.59583395)(211.01954801,339.57083397)(210.96955404,339.52083456)
\lineto(210.81955404,339.37083456)
\curveto(210.74954828,339.32083422)(210.68454834,339.27583427)(210.62455404,339.23583456)
\curveto(210.56454846,339.20583434)(210.49954853,339.16583438)(210.42955404,339.11583456)
\curveto(210.38954864,339.09583445)(210.34454868,339.07583447)(210.29455404,339.05583456)
\curveto(210.25454877,339.03583451)(210.20954882,339.01583453)(210.15955404,338.99583456)
\curveto(210.01954901,338.9458346)(209.86954916,338.90083464)(209.70955404,338.86083456)
\curveto(209.65954937,338.8408347)(209.61454941,338.83083471)(209.57455404,338.83083456)
\curveto(209.53454949,338.83083471)(209.49454953,338.82583472)(209.45455404,338.81583456)
\lineto(209.31955404,338.81583456)
\curveto(209.28954974,338.80583474)(209.24954978,338.80083474)(209.19955404,338.80083456)
\lineto(209.06455404,338.80083456)
\curveto(209.00455002,338.78083476)(208.91455011,338.77583477)(208.79455404,338.78583456)
\curveto(208.67455035,338.78583476)(208.58955044,338.79583475)(208.53955404,338.81583456)
\curveto(208.46955056,338.83583471)(208.40455062,338.8458347)(208.34455404,338.84583456)
\curveto(208.29455073,338.83583471)(208.23955079,338.8408347)(208.17955404,338.86083456)
\lineto(207.81955404,338.98083456)
\curveto(207.70955132,339.01083453)(207.59955143,339.05083449)(207.48955404,339.10083456)
\curveto(207.13955189,339.25083429)(206.8245522,339.48083406)(206.54455404,339.79083456)
\curveto(206.27455275,340.11083343)(206.05955297,340.4458331)(205.89955404,340.79583456)
\curveto(205.84955318,340.90583264)(205.80955322,341.01083253)(205.77955404,341.11083456)
\curveto(205.74955328,341.22083232)(205.71455331,341.33083221)(205.67455404,341.44083456)
\curveto(205.66455336,341.48083206)(205.65955337,341.51583203)(205.65955404,341.54583456)
\curveto(205.65955337,341.58583196)(205.64955338,341.63083191)(205.62955404,341.68083456)
\curveto(205.60955342,341.76083178)(205.58955344,341.8458317)(205.56955404,341.93583456)
\curveto(205.55955347,342.03583151)(205.54455348,342.13583141)(205.52455404,342.23583456)
\curveto(205.51455351,342.26583128)(205.50955352,342.30083124)(205.50955404,342.34083456)
\curveto(205.51955351,342.38083116)(205.51955351,342.41583113)(205.50955404,342.44583456)
\lineto(205.50955404,342.58083456)
\curveto(205.50955352,342.63083091)(205.50455352,342.68083086)(205.49455404,342.73083456)
\curveto(205.48455354,342.78083076)(205.47955355,342.83583071)(205.47955404,342.89583456)
\curveto(205.47955355,342.96583058)(205.48455354,343.02083052)(205.49455404,343.06083456)
\curveto(205.50455352,343.11083043)(205.50955352,343.15583039)(205.50955404,343.19583456)
\lineto(205.50955404,343.34583456)
\curveto(205.51955351,343.39583015)(205.51955351,343.4408301)(205.50955404,343.48083456)
\curveto(205.50955352,343.53083001)(205.51955351,343.58082996)(205.53955404,343.63083456)
\curveto(205.55955347,343.7408298)(205.57455345,343.8458297)(205.58455404,343.94583456)
\curveto(205.60455342,344.0458295)(205.6295534,344.1458294)(205.65955404,344.24583456)
\curveto(205.69955333,344.36582918)(205.73455329,344.48082906)(205.76455404,344.59083456)
\curveto(205.79455323,344.70082884)(205.83455319,344.81082873)(205.88455404,344.92083456)
\curveto(206.024553,345.22082832)(206.19955283,345.50582804)(206.40955404,345.77583456)
\curveto(206.4295526,345.80582774)(206.45455257,345.83082771)(206.48455404,345.85083456)
\curveto(206.5245525,345.88082766)(206.55455247,345.91082763)(206.57455404,345.94083456)
\curveto(206.61455241,345.99082755)(206.65455237,346.03582751)(206.69455404,346.07583456)
\curveto(206.73455229,346.11582743)(206.77955225,346.15582739)(206.82955404,346.19583456)
\curveto(206.86955216,346.21582733)(206.90455212,346.2408273)(206.93455404,346.27083456)
\curveto(206.96455206,346.31082723)(206.99955203,346.3408272)(207.03955404,346.36083456)
\curveto(207.28955174,346.53082701)(207.57955145,346.67082687)(207.90955404,346.78083456)
\curveto(207.97955105,346.80082674)(208.04955098,346.81582673)(208.11955404,346.82583456)
\curveto(208.19955083,346.83582671)(208.27955075,346.85082669)(208.35955404,346.87083456)
\curveto(208.4295506,346.89082665)(208.51955051,346.90082664)(208.62955404,346.90083456)
\curveto(208.73955029,346.91082663)(208.84955018,346.91582663)(208.95955404,346.91583456)
\curveto(209.06954996,346.91582663)(209.17454985,346.91082663)(209.27455404,346.90083456)
\curveto(209.38454964,346.89082665)(209.47454955,346.87582667)(209.54455404,346.85583456)
\curveto(209.69454933,346.80582674)(209.83954919,346.76082678)(209.97955404,346.72083456)
\curveto(210.11954891,346.68082686)(210.24954878,346.62582692)(210.36955404,346.55583456)
\curveto(210.43954859,346.50582704)(210.50454852,346.45582709)(210.56455404,346.40583456)
\curveto(210.6245484,346.36582718)(210.68954834,346.32082722)(210.75955404,346.27083456)
\curveto(210.79954823,346.2408273)(210.85454817,346.20082734)(210.92455404,346.15083456)
\curveto(211.00454802,346.10082744)(211.07954795,346.10082744)(211.14955404,346.15083456)
\curveto(211.18954784,346.17082737)(211.20954782,346.20582734)(211.20955404,346.25583456)
\curveto(211.20954782,346.30582724)(211.21954781,346.35582719)(211.23955404,346.40583456)
\lineto(211.23955404,346.55583456)
\curveto(211.24954778,346.58582696)(211.25454777,346.62082692)(211.25455404,346.66083456)
\lineto(211.25455404,346.78083456)
\lineto(211.25455404,348.82083456)
\curveto(211.25454777,348.93082461)(211.24954778,349.05082449)(211.23955404,349.18083456)
\curveto(211.23954779,349.32082422)(211.26454776,349.42582412)(211.31455404,349.49583456)
\curveto(211.35454767,349.57582397)(211.4295476,349.62582392)(211.53955404,349.64583456)
\curveto(211.55954747,349.65582389)(211.57954745,349.65582389)(211.59955404,349.64583456)
\curveto(211.61954741,349.6458239)(211.63954739,349.65082389)(211.65955404,349.66083456)
\lineto(212.72455404,349.66083456)
\curveto(212.84454618,349.66082388)(212.95454607,349.65582389)(213.05455404,349.64583456)
\curveto(213.15454587,349.63582391)(213.2295458,349.59582395)(213.27955404,349.52583456)
\curveto(213.3295457,349.4458241)(213.35454567,349.3408242)(213.35455404,349.21083456)
\lineto(213.35455404,348.85083456)
\lineto(213.35455404,339.83583456)
\moveto(211.31455404,342.77583456)
\curveto(211.3245477,342.81583073)(211.3245477,342.85583069)(211.31455404,342.89583456)
\lineto(211.31455404,343.03083456)
\curveto(211.31454771,343.13083041)(211.30954772,343.23083031)(211.29955404,343.33083456)
\curveto(211.28954774,343.43083011)(211.27454775,343.52083002)(211.25455404,343.60083456)
\curveto(211.23454779,343.71082983)(211.21454781,343.81082973)(211.19455404,343.90083456)
\curveto(211.18454784,343.99082955)(211.15954787,344.07582947)(211.11955404,344.15583456)
\curveto(210.97954805,344.51582903)(210.77454825,344.80082874)(210.50455404,345.01083456)
\curveto(210.24454878,345.22082832)(209.86454916,345.32582822)(209.36455404,345.32583456)
\curveto(209.30454972,345.32582822)(209.2245498,345.31582823)(209.12455404,345.29583456)
\curveto(209.04454998,345.27582827)(208.96955006,345.25582829)(208.89955404,345.23583456)
\curveto(208.83955019,345.22582832)(208.77955025,345.20582834)(208.71955404,345.17583456)
\curveto(208.44955058,345.06582848)(208.23955079,344.89582865)(208.08955404,344.66583456)
\curveto(207.93955109,344.43582911)(207.81955121,344.17582937)(207.72955404,343.88583456)
\curveto(207.69955133,343.78582976)(207.67955135,343.68582986)(207.66955404,343.58583456)
\curveto(207.65955137,343.48583006)(207.63955139,343.38083016)(207.60955404,343.27083456)
\lineto(207.60955404,343.06083456)
\curveto(207.58955144,342.97083057)(207.58455144,342.8458307)(207.59455404,342.68583456)
\curveto(207.60455142,342.53583101)(207.61955141,342.42583112)(207.63955404,342.35583456)
\lineto(207.63955404,342.26583456)
\curveto(207.64955138,342.2458313)(207.65455137,342.22583132)(207.65455404,342.20583456)
\curveto(207.67455135,342.12583142)(207.68955134,342.05083149)(207.69955404,341.98083456)
\curveto(207.71955131,341.91083163)(207.73955129,341.83583171)(207.75955404,341.75583456)
\curveto(207.9295511,341.23583231)(208.21955081,340.85083269)(208.62955404,340.60083456)
\curveto(208.75955027,340.51083303)(208.93955009,340.4408331)(209.16955404,340.39083456)
\curveto(209.20954982,340.38083316)(209.26954976,340.37583317)(209.34955404,340.37583456)
\curveto(209.37954965,340.36583318)(209.4245496,340.35583319)(209.48455404,340.34583456)
\curveto(209.55454947,340.3458332)(209.60954942,340.35083319)(209.64955404,340.36083456)
\curveto(209.7295493,340.38083316)(209.80954922,340.39583315)(209.88955404,340.40583456)
\curveto(209.96954906,340.41583313)(210.04954898,340.43583311)(210.12955404,340.46583456)
\curveto(210.37954865,340.57583297)(210.57954845,340.71583283)(210.72955404,340.88583456)
\curveto(210.87954815,341.05583249)(211.00954802,341.27083227)(211.11955404,341.53083456)
\curveto(211.15954787,341.62083192)(211.18954784,341.71083183)(211.20955404,341.80083456)
\curveto(211.2295478,341.90083164)(211.24954778,342.00583154)(211.26955404,342.11583456)
\curveto(211.27954775,342.16583138)(211.27954775,342.21083133)(211.26955404,342.25083456)
\curveto(211.26954776,342.30083124)(211.27954775,342.35083119)(211.29955404,342.40083456)
\curveto(211.30954772,342.43083111)(211.31454771,342.46583108)(211.31455404,342.50583456)
\lineto(211.31455404,342.64083456)
\lineto(211.31455404,342.77583456)
}
}
{
\newrgbcolor{curcolor}{0 0 0}
\pscustom[linestyle=none,fillstyle=solid,fillcolor=curcolor]
{
\newpath
\moveto(406.14953951,349.66083456)
\lineto(407.43953951,349.66083456)
\curveto(407.54953668,349.66082388)(407.65453658,349.65582389)(407.75453951,349.64583456)
\curveto(407.85453638,349.6458239)(407.9295363,349.61082393)(407.97953951,349.54083456)
\curveto(408.0295362,349.47082407)(408.05453618,349.38082416)(408.05453951,349.27083456)
\curveto(408.06453617,349.16082438)(408.06953616,349.0408245)(408.06953951,348.91083456)
\lineto(408.06953951,347.60583456)
\lineto(408.06953951,342.40083456)
\lineto(408.06953951,339.94083456)
\lineto(408.06953951,339.50583456)
\curveto(408.07953615,339.3458342)(408.05953617,339.22583432)(408.00953951,339.14583456)
\curveto(407.96953626,339.07583447)(407.87953635,339.02083452)(407.73953951,338.98083456)
\curveto(407.66953656,338.96083458)(407.59453664,338.95583459)(407.51453951,338.96583456)
\curveto(407.4345368,338.97583457)(407.35453688,338.98083456)(407.27453951,338.98083456)
\lineto(406.38953951,338.98083456)
\curveto(406.27953795,338.98083456)(406.17453806,338.98583456)(406.07453951,338.99583456)
\curveto(405.98453825,339.00583454)(405.90953832,339.03583451)(405.84953951,339.08583456)
\curveto(405.79953843,339.13583441)(405.76953846,339.21083433)(405.75953951,339.31083456)
\curveto(405.74953848,339.41083413)(405.74453849,339.51583403)(405.74453951,339.62583456)
\lineto(405.74453951,340.93083456)
\lineto(405.74453951,346.40583456)
\lineto(405.74453951,348.59583456)
\curveto(405.74453849,348.73582481)(405.73953849,348.90082464)(405.72953951,349.09083456)
\curveto(405.7295385,349.28082426)(405.75453848,349.41582413)(405.80453951,349.49583456)
\curveto(405.84453839,349.55582399)(405.90953832,349.60582394)(405.99953951,349.64583456)
\curveto(406.0295382,349.6458239)(406.05453818,349.6458239)(406.07453951,349.64583456)
\curveto(406.10453813,349.65582389)(406.1295381,349.66082388)(406.14953951,349.66083456)
}
}
{
\newrgbcolor{curcolor}{0 0 0}
\pscustom[linestyle=none,fillstyle=solid,fillcolor=curcolor]
{
\newpath
\moveto(414.35336763,346.91583456)
\curveto(414.95336183,346.93582661)(415.45336133,346.85082669)(415.85336763,346.66083456)
\curveto(416.25336053,346.47082707)(416.56836021,346.19082735)(416.79836763,345.82083456)
\curveto(416.86835991,345.71082783)(416.92335986,345.59082795)(416.96336763,345.46083456)
\curveto(417.00335978,345.3408282)(417.04335974,345.21582833)(417.08336763,345.08583456)
\curveto(417.10335968,345.00582854)(417.11335967,344.93082861)(417.11336763,344.86083456)
\curveto(417.12335966,344.79082875)(417.13835964,344.72082882)(417.15836763,344.65083456)
\curveto(417.15835962,344.59082895)(417.16335962,344.55082899)(417.17336763,344.53083456)
\curveto(417.19335959,344.39082915)(417.20335958,344.2458293)(417.20336763,344.09583456)
\lineto(417.20336763,343.66083456)
\lineto(417.20336763,342.32583456)
\lineto(417.20336763,339.89583456)
\curveto(417.20335958,339.70583384)(417.19835958,339.52083402)(417.18836763,339.34083456)
\curveto(417.18835959,339.17083437)(417.11835966,339.06083448)(416.97836763,339.01083456)
\curveto(416.91835986,338.99083455)(416.84835993,338.98083456)(416.76836763,338.98083456)
\lineto(416.52836763,338.98083456)
\lineto(415.71836763,338.98083456)
\curveto(415.59836118,338.98083456)(415.48836129,338.98583456)(415.38836763,338.99583456)
\curveto(415.29836148,339.01583453)(415.22836155,339.06083448)(415.17836763,339.13083456)
\curveto(415.13836164,339.19083435)(415.11336167,339.26583428)(415.10336763,339.35583456)
\lineto(415.10336763,339.67083456)
\lineto(415.10336763,340.72083456)
\lineto(415.10336763,342.95583456)
\curveto(415.10336168,343.32583022)(415.08836169,343.66582988)(415.05836763,343.97583456)
\curveto(415.02836175,344.29582925)(414.93836184,344.56582898)(414.78836763,344.78583456)
\curveto(414.64836213,344.98582856)(414.44336234,345.12582842)(414.17336763,345.20583456)
\curveto(414.12336266,345.22582832)(414.06836271,345.23582831)(414.00836763,345.23583456)
\curveto(413.95836282,345.23582831)(413.90336288,345.2458283)(413.84336763,345.26583456)
\curveto(413.79336299,345.27582827)(413.72836305,345.27582827)(413.64836763,345.26583456)
\curveto(413.5783632,345.26582828)(413.52336326,345.26082828)(413.48336763,345.25083456)
\curveto(413.44336334,345.2408283)(413.40836337,345.23582831)(413.37836763,345.23583456)
\curveto(413.34836343,345.23582831)(413.31836346,345.23082831)(413.28836763,345.22083456)
\curveto(413.05836372,345.16082838)(412.87336391,345.08082846)(412.73336763,344.98083456)
\curveto(412.41336437,344.75082879)(412.22336456,344.41582913)(412.16336763,343.97583456)
\curveto(412.10336468,343.53583001)(412.07336471,343.0408305)(412.07336763,342.49083456)
\lineto(412.07336763,340.61583456)
\lineto(412.07336763,339.70083456)
\lineto(412.07336763,339.43083456)
\curveto(412.07336471,339.3408342)(412.05836472,339.26583428)(412.02836763,339.20583456)
\curveto(411.9783648,339.09583445)(411.89836488,339.03083451)(411.78836763,339.01083456)
\curveto(411.6783651,338.99083455)(411.54336524,338.98083456)(411.38336763,338.98083456)
\lineto(410.63336763,338.98083456)
\curveto(410.52336626,338.98083456)(410.41336637,338.98583456)(410.30336763,338.99583456)
\curveto(410.19336659,339.00583454)(410.11336667,339.0408345)(410.06336763,339.10083456)
\curveto(409.99336679,339.19083435)(409.95836682,339.32083422)(409.95836763,339.49083456)
\curveto(409.96836681,339.66083388)(409.97336681,339.82083372)(409.97336763,339.97083456)
\lineto(409.97336763,342.01083456)
\lineto(409.97336763,345.31083456)
\lineto(409.97336763,346.07583456)
\lineto(409.97336763,346.37583456)
\curveto(409.9833668,346.46582708)(410.01336677,346.540827)(410.06336763,346.60083456)
\curveto(410.0833667,346.63082691)(410.11336667,346.65082689)(410.15336763,346.66083456)
\curveto(410.20336658,346.68082686)(410.25336653,346.69582685)(410.30336763,346.70583456)
\lineto(410.37836763,346.70583456)
\curveto(410.42836635,346.71582683)(410.4783663,346.72082682)(410.52836763,346.72083456)
\lineto(410.69336763,346.72083456)
\lineto(411.32336763,346.72083456)
\curveto(411.40336538,346.72082682)(411.4783653,346.71582683)(411.54836763,346.70583456)
\curveto(411.62836515,346.70582684)(411.69836508,346.69582685)(411.75836763,346.67583456)
\curveto(411.82836495,346.6458269)(411.87336491,346.60082694)(411.89336763,346.54083456)
\curveto(411.92336486,346.48082706)(411.94836483,346.41082713)(411.96836763,346.33083456)
\curveto(411.9783648,346.29082725)(411.9783648,346.25582729)(411.96836763,346.22583456)
\curveto(411.96836481,346.19582735)(411.9783648,346.16582738)(411.99836763,346.13583456)
\curveto(412.01836476,346.08582746)(412.03336475,346.05582749)(412.04336763,346.04583456)
\curveto(412.06336472,346.03582751)(412.08836469,346.02082752)(412.11836763,346.00083456)
\curveto(412.22836455,345.99082755)(412.31836446,346.02582752)(412.38836763,346.10583456)
\curveto(412.45836432,346.19582735)(412.53336425,346.26582728)(412.61336763,346.31583456)
\curveto(412.8833639,346.51582703)(413.1833636,346.67582687)(413.51336763,346.79583456)
\curveto(413.60336318,346.82582672)(413.69336309,346.8458267)(413.78336763,346.85583456)
\curveto(413.8833629,346.86582668)(413.98836279,346.88082666)(414.09836763,346.90083456)
\curveto(414.12836265,346.91082663)(414.17336261,346.91082663)(414.23336763,346.90083456)
\curveto(414.29336249,346.90082664)(414.33336245,346.90582664)(414.35336763,346.91583456)
}
}
{
\newrgbcolor{curcolor}{0 0 0}
\pscustom[linestyle=none,fillstyle=solid,fillcolor=curcolor]
{
\newpath
\moveto(426.42461763,339.83583456)
\lineto(426.42461763,339.41583456)
\curveto(426.42460926,339.28583426)(426.39460929,339.18083436)(426.33461763,339.10083456)
\curveto(426.2846094,339.05083449)(426.21960947,339.01583453)(426.13961763,338.99583456)
\curveto(426.05960963,338.98583456)(425.96960972,338.98083456)(425.86961763,338.98083456)
\lineto(425.04461763,338.98083456)
\lineto(424.75961763,338.98083456)
\curveto(424.67961101,338.99083455)(424.61461107,339.01583453)(424.56461763,339.05583456)
\curveto(424.49461119,339.10583444)(424.45461123,339.17083437)(424.44461763,339.25083456)
\curveto(424.43461125,339.33083421)(424.41461127,339.41083413)(424.38461763,339.49083456)
\curveto(424.36461132,339.51083403)(424.34461134,339.52583402)(424.32461763,339.53583456)
\curveto(424.31461137,339.55583399)(424.29961139,339.57583397)(424.27961763,339.59583456)
\curveto(424.16961152,339.59583395)(424.0896116,339.57083397)(424.03961763,339.52083456)
\lineto(423.88961763,339.37083456)
\curveto(423.81961187,339.32083422)(423.75461193,339.27583427)(423.69461763,339.23583456)
\curveto(423.63461205,339.20583434)(423.56961212,339.16583438)(423.49961763,339.11583456)
\curveto(423.45961223,339.09583445)(423.41461227,339.07583447)(423.36461763,339.05583456)
\curveto(423.32461236,339.03583451)(423.27961241,339.01583453)(423.22961763,338.99583456)
\curveto(423.0896126,338.9458346)(422.93961275,338.90083464)(422.77961763,338.86083456)
\curveto(422.72961296,338.8408347)(422.684613,338.83083471)(422.64461763,338.83083456)
\curveto(422.60461308,338.83083471)(422.56461312,338.82583472)(422.52461763,338.81583456)
\lineto(422.38961763,338.81583456)
\curveto(422.35961333,338.80583474)(422.31961337,338.80083474)(422.26961763,338.80083456)
\lineto(422.13461763,338.80083456)
\curveto(422.07461361,338.78083476)(421.9846137,338.77583477)(421.86461763,338.78583456)
\curveto(421.74461394,338.78583476)(421.65961403,338.79583475)(421.60961763,338.81583456)
\curveto(421.53961415,338.83583471)(421.47461421,338.8458347)(421.41461763,338.84583456)
\curveto(421.36461432,338.83583471)(421.30961438,338.8408347)(421.24961763,338.86083456)
\lineto(420.88961763,338.98083456)
\curveto(420.77961491,339.01083453)(420.66961502,339.05083449)(420.55961763,339.10083456)
\curveto(420.20961548,339.25083429)(419.89461579,339.48083406)(419.61461763,339.79083456)
\curveto(419.34461634,340.11083343)(419.12961656,340.4458331)(418.96961763,340.79583456)
\curveto(418.91961677,340.90583264)(418.87961681,341.01083253)(418.84961763,341.11083456)
\curveto(418.81961687,341.22083232)(418.7846169,341.33083221)(418.74461763,341.44083456)
\curveto(418.73461695,341.48083206)(418.72961696,341.51583203)(418.72961763,341.54583456)
\curveto(418.72961696,341.58583196)(418.71961697,341.63083191)(418.69961763,341.68083456)
\curveto(418.67961701,341.76083178)(418.65961703,341.8458317)(418.63961763,341.93583456)
\curveto(418.62961706,342.03583151)(418.61461707,342.13583141)(418.59461763,342.23583456)
\curveto(418.5846171,342.26583128)(418.57961711,342.30083124)(418.57961763,342.34083456)
\curveto(418.5896171,342.38083116)(418.5896171,342.41583113)(418.57961763,342.44583456)
\lineto(418.57961763,342.58083456)
\curveto(418.57961711,342.63083091)(418.57461711,342.68083086)(418.56461763,342.73083456)
\curveto(418.55461713,342.78083076)(418.54961714,342.83583071)(418.54961763,342.89583456)
\curveto(418.54961714,342.96583058)(418.55461713,343.02083052)(418.56461763,343.06083456)
\curveto(418.57461711,343.11083043)(418.57961711,343.15583039)(418.57961763,343.19583456)
\lineto(418.57961763,343.34583456)
\curveto(418.5896171,343.39583015)(418.5896171,343.4408301)(418.57961763,343.48083456)
\curveto(418.57961711,343.53083001)(418.5896171,343.58082996)(418.60961763,343.63083456)
\curveto(418.62961706,343.7408298)(418.64461704,343.8458297)(418.65461763,343.94583456)
\curveto(418.67461701,344.0458295)(418.69961699,344.1458294)(418.72961763,344.24583456)
\curveto(418.76961692,344.36582918)(418.80461688,344.48082906)(418.83461763,344.59083456)
\curveto(418.86461682,344.70082884)(418.90461678,344.81082873)(418.95461763,344.92083456)
\curveto(419.09461659,345.22082832)(419.26961642,345.50582804)(419.47961763,345.77583456)
\curveto(419.49961619,345.80582774)(419.52461616,345.83082771)(419.55461763,345.85083456)
\curveto(419.59461609,345.88082766)(419.62461606,345.91082763)(419.64461763,345.94083456)
\curveto(419.684616,345.99082755)(419.72461596,346.03582751)(419.76461763,346.07583456)
\curveto(419.80461588,346.11582743)(419.84961584,346.15582739)(419.89961763,346.19583456)
\curveto(419.93961575,346.21582733)(419.97461571,346.2408273)(420.00461763,346.27083456)
\curveto(420.03461565,346.31082723)(420.06961562,346.3408272)(420.10961763,346.36083456)
\curveto(420.35961533,346.53082701)(420.64961504,346.67082687)(420.97961763,346.78083456)
\curveto(421.04961464,346.80082674)(421.11961457,346.81582673)(421.18961763,346.82583456)
\curveto(421.26961442,346.83582671)(421.34961434,346.85082669)(421.42961763,346.87083456)
\curveto(421.49961419,346.89082665)(421.5896141,346.90082664)(421.69961763,346.90083456)
\curveto(421.80961388,346.91082663)(421.91961377,346.91582663)(422.02961763,346.91583456)
\curveto(422.13961355,346.91582663)(422.24461344,346.91082663)(422.34461763,346.90083456)
\curveto(422.45461323,346.89082665)(422.54461314,346.87582667)(422.61461763,346.85583456)
\curveto(422.76461292,346.80582674)(422.90961278,346.76082678)(423.04961763,346.72083456)
\curveto(423.1896125,346.68082686)(423.31961237,346.62582692)(423.43961763,346.55583456)
\curveto(423.50961218,346.50582704)(423.57461211,346.45582709)(423.63461763,346.40583456)
\curveto(423.69461199,346.36582718)(423.75961193,346.32082722)(423.82961763,346.27083456)
\curveto(423.86961182,346.2408273)(423.92461176,346.20082734)(423.99461763,346.15083456)
\curveto(424.07461161,346.10082744)(424.14961154,346.10082744)(424.21961763,346.15083456)
\curveto(424.25961143,346.17082737)(424.27961141,346.20582734)(424.27961763,346.25583456)
\curveto(424.27961141,346.30582724)(424.2896114,346.35582719)(424.30961763,346.40583456)
\lineto(424.30961763,346.55583456)
\curveto(424.31961137,346.58582696)(424.32461136,346.62082692)(424.32461763,346.66083456)
\lineto(424.32461763,346.78083456)
\lineto(424.32461763,348.82083456)
\curveto(424.32461136,348.93082461)(424.31961137,349.05082449)(424.30961763,349.18083456)
\curveto(424.30961138,349.32082422)(424.33461135,349.42582412)(424.38461763,349.49583456)
\curveto(424.42461126,349.57582397)(424.49961119,349.62582392)(424.60961763,349.64583456)
\curveto(424.62961106,349.65582389)(424.64961104,349.65582389)(424.66961763,349.64583456)
\curveto(424.689611,349.6458239)(424.70961098,349.65082389)(424.72961763,349.66083456)
\lineto(425.79461763,349.66083456)
\curveto(425.91460977,349.66082388)(426.02460966,349.65582389)(426.12461763,349.64583456)
\curveto(426.22460946,349.63582391)(426.29960939,349.59582395)(426.34961763,349.52583456)
\curveto(426.39960929,349.4458241)(426.42460926,349.3408242)(426.42461763,349.21083456)
\lineto(426.42461763,348.85083456)
\lineto(426.42461763,339.83583456)
\moveto(424.38461763,342.77583456)
\curveto(424.39461129,342.81583073)(424.39461129,342.85583069)(424.38461763,342.89583456)
\lineto(424.38461763,343.03083456)
\curveto(424.3846113,343.13083041)(424.37961131,343.23083031)(424.36961763,343.33083456)
\curveto(424.35961133,343.43083011)(424.34461134,343.52083002)(424.32461763,343.60083456)
\curveto(424.30461138,343.71082983)(424.2846114,343.81082973)(424.26461763,343.90083456)
\curveto(424.25461143,343.99082955)(424.22961146,344.07582947)(424.18961763,344.15583456)
\curveto(424.04961164,344.51582903)(423.84461184,344.80082874)(423.57461763,345.01083456)
\curveto(423.31461237,345.22082832)(422.93461275,345.32582822)(422.43461763,345.32583456)
\curveto(422.37461331,345.32582822)(422.29461339,345.31582823)(422.19461763,345.29583456)
\curveto(422.11461357,345.27582827)(422.03961365,345.25582829)(421.96961763,345.23583456)
\curveto(421.90961378,345.22582832)(421.84961384,345.20582834)(421.78961763,345.17583456)
\curveto(421.51961417,345.06582848)(421.30961438,344.89582865)(421.15961763,344.66583456)
\curveto(421.00961468,344.43582911)(420.8896148,344.17582937)(420.79961763,343.88583456)
\curveto(420.76961492,343.78582976)(420.74961494,343.68582986)(420.73961763,343.58583456)
\curveto(420.72961496,343.48583006)(420.70961498,343.38083016)(420.67961763,343.27083456)
\lineto(420.67961763,343.06083456)
\curveto(420.65961503,342.97083057)(420.65461503,342.8458307)(420.66461763,342.68583456)
\curveto(420.67461501,342.53583101)(420.689615,342.42583112)(420.70961763,342.35583456)
\lineto(420.70961763,342.26583456)
\curveto(420.71961497,342.2458313)(420.72461496,342.22583132)(420.72461763,342.20583456)
\curveto(420.74461494,342.12583142)(420.75961493,342.05083149)(420.76961763,341.98083456)
\curveto(420.7896149,341.91083163)(420.80961488,341.83583171)(420.82961763,341.75583456)
\curveto(420.99961469,341.23583231)(421.2896144,340.85083269)(421.69961763,340.60083456)
\curveto(421.82961386,340.51083303)(422.00961368,340.4408331)(422.23961763,340.39083456)
\curveto(422.27961341,340.38083316)(422.33961335,340.37583317)(422.41961763,340.37583456)
\curveto(422.44961324,340.36583318)(422.49461319,340.35583319)(422.55461763,340.34583456)
\curveto(422.62461306,340.3458332)(422.67961301,340.35083319)(422.71961763,340.36083456)
\curveto(422.79961289,340.38083316)(422.87961281,340.39583315)(422.95961763,340.40583456)
\curveto(423.03961265,340.41583313)(423.11961257,340.43583311)(423.19961763,340.46583456)
\curveto(423.44961224,340.57583297)(423.64961204,340.71583283)(423.79961763,340.88583456)
\curveto(423.94961174,341.05583249)(424.07961161,341.27083227)(424.18961763,341.53083456)
\curveto(424.22961146,341.62083192)(424.25961143,341.71083183)(424.27961763,341.80083456)
\curveto(424.29961139,341.90083164)(424.31961137,342.00583154)(424.33961763,342.11583456)
\curveto(424.34961134,342.16583138)(424.34961134,342.21083133)(424.33961763,342.25083456)
\curveto(424.33961135,342.30083124)(424.34961134,342.35083119)(424.36961763,342.40083456)
\curveto(424.37961131,342.43083111)(424.3846113,342.46583108)(424.38461763,342.50583456)
\lineto(424.38461763,342.64083456)
\lineto(424.38461763,342.77583456)
}
}
{
\newrgbcolor{curcolor}{0 0 0}
\pscustom[linestyle=none,fillstyle=solid,fillcolor=curcolor]
{
\newpath
\moveto(430.10453951,349.57083456)
\curveto(430.17453656,349.49082405)(430.20953652,349.37082417)(430.20953951,349.21083456)
\lineto(430.20953951,348.74583456)
\lineto(430.20953951,348.34083456)
\curveto(430.20953652,348.20082534)(430.17453656,348.10582544)(430.10453951,348.05583456)
\curveto(430.04453669,348.00582554)(429.96453677,347.97582557)(429.86453951,347.96583456)
\curveto(429.77453696,347.95582559)(429.67453706,347.95082559)(429.56453951,347.95083456)
\lineto(428.72453951,347.95083456)
\curveto(428.61453812,347.95082559)(428.51453822,347.95582559)(428.42453951,347.96583456)
\curveto(428.34453839,347.97582557)(428.27453846,348.00582554)(428.21453951,348.05583456)
\curveto(428.17453856,348.08582546)(428.14453859,348.1408254)(428.12453951,348.22083456)
\curveto(428.11453862,348.31082523)(428.10453863,348.40582514)(428.09453951,348.50583456)
\lineto(428.09453951,348.83583456)
\curveto(428.10453863,348.9458246)(428.10953862,349.0408245)(428.10953951,349.12083456)
\lineto(428.10953951,349.33083456)
\curveto(428.11953861,349.40082414)(428.13953859,349.46082408)(428.16953951,349.51083456)
\curveto(428.18953854,349.55082399)(428.21453852,349.58082396)(428.24453951,349.60083456)
\lineto(428.36453951,349.66083456)
\curveto(428.38453835,349.66082388)(428.40953832,349.66082388)(428.43953951,349.66083456)
\curveto(428.46953826,349.67082387)(428.49453824,349.67582387)(428.51453951,349.67583456)
\lineto(429.60953951,349.67583456)
\curveto(429.70953702,349.67582387)(429.80453693,349.67082387)(429.89453951,349.66083456)
\curveto(429.98453675,349.65082389)(430.05453668,349.62082392)(430.10453951,349.57083456)
\moveto(430.20953951,339.80583456)
\curveto(430.20953652,339.60583394)(430.20453653,339.43583411)(430.19453951,339.29583456)
\curveto(430.18453655,339.15583439)(430.09453664,339.06083448)(429.92453951,339.01083456)
\curveto(429.86453687,338.99083455)(429.79953693,338.98083456)(429.72953951,338.98083456)
\curveto(429.65953707,338.99083455)(429.58453715,338.99583455)(429.50453951,338.99583456)
\lineto(428.66453951,338.99583456)
\curveto(428.57453816,338.99583455)(428.48453825,339.00083454)(428.39453951,339.01083456)
\curveto(428.31453842,339.02083452)(428.25453848,339.05083449)(428.21453951,339.10083456)
\curveto(428.15453858,339.17083437)(428.11953861,339.25583429)(428.10953951,339.35583456)
\lineto(428.10953951,339.70083456)
\lineto(428.10953951,346.03083456)
\lineto(428.10953951,346.33083456)
\curveto(428.10953862,346.43082711)(428.1295386,346.51082703)(428.16953951,346.57083456)
\curveto(428.2295385,346.6408269)(428.31453842,346.68582686)(428.42453951,346.70583456)
\curveto(428.44453829,346.71582683)(428.46953826,346.71582683)(428.49953951,346.70583456)
\curveto(428.53953819,346.70582684)(428.56953816,346.71082683)(428.58953951,346.72083456)
\lineto(429.33953951,346.72083456)
\lineto(429.53453951,346.72083456)
\curveto(429.61453712,346.73082681)(429.67953705,346.73082681)(429.72953951,346.72083456)
\lineto(429.84953951,346.72083456)
\curveto(429.90953682,346.70082684)(429.96453677,346.68582686)(430.01453951,346.67583456)
\curveto(430.06453667,346.66582688)(430.10453663,346.63582691)(430.13453951,346.58583456)
\curveto(430.17453656,346.53582701)(430.19453654,346.46582708)(430.19453951,346.37583456)
\curveto(430.20453653,346.28582726)(430.20953652,346.19082735)(430.20953951,346.09083456)
\lineto(430.20953951,339.80583456)
}
}
{
\newrgbcolor{curcolor}{0 0 0}
\pscustom[linestyle=none,fillstyle=solid,fillcolor=curcolor]
{
\newpath
\moveto(435.44172701,346.93083456)
\curveto(436.25172185,346.95082659)(436.92672117,346.83082671)(437.46672701,346.57083456)
\curveto(438.01672008,346.31082723)(438.45171965,345.9408276)(438.77172701,345.46083456)
\curveto(438.93171917,345.22082832)(439.05171905,344.9458286)(439.13172701,344.63583456)
\curveto(439.15171895,344.58582896)(439.16671893,344.52082902)(439.17672701,344.44083456)
\curveto(439.1967189,344.36082918)(439.1967189,344.29082925)(439.17672701,344.23083456)
\curveto(439.13671896,344.12082942)(439.06671903,344.05582949)(438.96672701,344.03583456)
\curveto(438.86671923,344.02582952)(438.74671935,344.02082952)(438.60672701,344.02083456)
\lineto(437.82672701,344.02083456)
\lineto(437.54172701,344.02083456)
\curveto(437.45172065,344.02082952)(437.37672072,344.0408295)(437.31672701,344.08083456)
\curveto(437.23672086,344.12082942)(437.18172092,344.18082936)(437.15172701,344.26083456)
\curveto(437.12172098,344.35082919)(437.08172102,344.4408291)(437.03172701,344.53083456)
\curveto(436.97172113,344.6408289)(436.90672119,344.7408288)(436.83672701,344.83083456)
\curveto(436.76672133,344.92082862)(436.68672141,345.00082854)(436.59672701,345.07083456)
\curveto(436.45672164,345.16082838)(436.3017218,345.23082831)(436.13172701,345.28083456)
\curveto(436.07172203,345.30082824)(436.01172209,345.31082823)(435.95172701,345.31083456)
\curveto(435.89172221,345.31082823)(435.83672226,345.32082822)(435.78672701,345.34083456)
\lineto(435.63672701,345.34083456)
\curveto(435.43672266,345.3408282)(435.27672282,345.32082822)(435.15672701,345.28083456)
\curveto(434.86672323,345.19082835)(434.63172347,345.05082849)(434.45172701,344.86083456)
\curveto(434.27172383,344.68082886)(434.12672397,344.46082908)(434.01672701,344.20083456)
\curveto(433.96672413,344.09082945)(433.92672417,343.97082957)(433.89672701,343.84083456)
\curveto(433.87672422,343.72082982)(433.85172425,343.59082995)(433.82172701,343.45083456)
\curveto(433.81172429,343.41083013)(433.80672429,343.37083017)(433.80672701,343.33083456)
\curveto(433.80672429,343.29083025)(433.8017243,343.25083029)(433.79172701,343.21083456)
\curveto(433.77172433,343.11083043)(433.76172434,342.97083057)(433.76172701,342.79083456)
\curveto(433.77172433,342.61083093)(433.78672431,342.47083107)(433.80672701,342.37083456)
\curveto(433.80672429,342.29083125)(433.81172429,342.23583131)(433.82172701,342.20583456)
\curveto(433.84172426,342.13583141)(433.85172425,342.06583148)(433.85172701,341.99583456)
\curveto(433.86172424,341.92583162)(433.87672422,341.85583169)(433.89672701,341.78583456)
\curveto(433.97672412,341.55583199)(434.07172403,341.3458322)(434.18172701,341.15583456)
\curveto(434.29172381,340.96583258)(434.43172367,340.80583274)(434.60172701,340.67583456)
\curveto(434.64172346,340.6458329)(434.7017234,340.61083293)(434.78172701,340.57083456)
\curveto(434.89172321,340.50083304)(435.0017231,340.45583309)(435.11172701,340.43583456)
\curveto(435.23172287,340.41583313)(435.37672272,340.39583315)(435.54672701,340.37583456)
\lineto(435.63672701,340.37583456)
\curveto(435.67672242,340.37583317)(435.70672239,340.38083316)(435.72672701,340.39083456)
\lineto(435.86172701,340.39083456)
\curveto(435.93172217,340.41083313)(435.9967221,340.42583312)(436.05672701,340.43583456)
\curveto(436.12672197,340.45583309)(436.19172191,340.47583307)(436.25172701,340.49583456)
\curveto(436.55172155,340.62583292)(436.78172132,340.81583273)(436.94172701,341.06583456)
\curveto(436.98172112,341.11583243)(437.01672108,341.17083237)(437.04672701,341.23083456)
\curveto(437.07672102,341.30083224)(437.101721,341.36083218)(437.12172701,341.41083456)
\curveto(437.16172094,341.52083202)(437.1967209,341.61583193)(437.22672701,341.69583456)
\curveto(437.25672084,341.78583176)(437.32672077,341.85583169)(437.43672701,341.90583456)
\curveto(437.52672057,341.9458316)(437.67172043,341.96083158)(437.87172701,341.95083456)
\lineto(438.36672701,341.95083456)
\lineto(438.57672701,341.95083456)
\curveto(438.65671944,341.96083158)(438.72171938,341.95583159)(438.77172701,341.93583456)
\lineto(438.89172701,341.93583456)
\lineto(439.01172701,341.90583456)
\curveto(439.05171905,341.90583164)(439.08171902,341.89583165)(439.10172701,341.87583456)
\curveto(439.15171895,341.83583171)(439.18171892,341.77583177)(439.19172701,341.69583456)
\curveto(439.21171889,341.62583192)(439.21171889,341.55083199)(439.19172701,341.47083456)
\curveto(439.101719,341.1408324)(438.99171911,340.8458327)(438.86172701,340.58583456)
\curveto(438.45171965,339.81583373)(437.7967203,339.28083426)(436.89672701,338.98083456)
\curveto(436.7967213,338.95083459)(436.69172141,338.93083461)(436.58172701,338.92083456)
\curveto(436.47172163,338.90083464)(436.36172174,338.87583467)(436.25172701,338.84583456)
\curveto(436.19172191,338.83583471)(436.13172197,338.83083471)(436.07172701,338.83083456)
\curveto(436.01172209,338.83083471)(435.95172215,338.82583472)(435.89172701,338.81583456)
\lineto(435.72672701,338.81583456)
\curveto(435.67672242,338.79583475)(435.6017225,338.79083475)(435.50172701,338.80083456)
\curveto(435.4017227,338.80083474)(435.32672277,338.80583474)(435.27672701,338.81583456)
\curveto(435.1967229,338.83583471)(435.12172298,338.8458347)(435.05172701,338.84583456)
\curveto(434.99172311,338.83583471)(434.92672317,338.8408347)(434.85672701,338.86083456)
\lineto(434.70672701,338.89083456)
\curveto(434.65672344,338.89083465)(434.60672349,338.89583465)(434.55672701,338.90583456)
\curveto(434.44672365,338.93583461)(434.34172376,338.96583458)(434.24172701,338.99583456)
\curveto(434.14172396,339.02583452)(434.04672405,339.06083448)(433.95672701,339.10083456)
\curveto(433.48672461,339.30083424)(433.09172501,339.55583399)(432.77172701,339.86583456)
\curveto(432.45172565,340.18583336)(432.19172591,340.58083296)(431.99172701,341.05083456)
\curveto(431.94172616,341.1408324)(431.9017262,341.23583231)(431.87172701,341.33583456)
\lineto(431.78172701,341.66583456)
\curveto(431.77172633,341.70583184)(431.76672633,341.7408318)(431.76672701,341.77083456)
\curveto(431.76672633,341.81083173)(431.75672634,341.85583169)(431.73672701,341.90583456)
\curveto(431.71672638,341.97583157)(431.70672639,342.0458315)(431.70672701,342.11583456)
\curveto(431.70672639,342.19583135)(431.6967264,342.27083127)(431.67672701,342.34083456)
\lineto(431.67672701,342.59583456)
\curveto(431.65672644,342.6458309)(431.64672645,342.70083084)(431.64672701,342.76083456)
\curveto(431.64672645,342.83083071)(431.65672644,342.89083065)(431.67672701,342.94083456)
\curveto(431.68672641,342.99083055)(431.68672641,343.03583051)(431.67672701,343.07583456)
\curveto(431.66672643,343.11583043)(431.66672643,343.15583039)(431.67672701,343.19583456)
\curveto(431.6967264,343.26583028)(431.7017264,343.33083021)(431.69172701,343.39083456)
\curveto(431.69172641,343.45083009)(431.7017264,343.51083003)(431.72172701,343.57083456)
\curveto(431.77172633,343.75082979)(431.81172629,343.92082962)(431.84172701,344.08083456)
\curveto(431.87172623,344.25082929)(431.91672618,344.41582913)(431.97672701,344.57583456)
\curveto(432.1967259,345.08582846)(432.47172563,345.51082803)(432.80172701,345.85083456)
\curveto(433.14172496,346.19082735)(433.57172453,346.46582708)(434.09172701,346.67583456)
\curveto(434.23172387,346.73582681)(434.37672372,346.77582677)(434.52672701,346.79583456)
\curveto(434.67672342,346.82582672)(434.83172327,346.86082668)(434.99172701,346.90083456)
\curveto(435.07172303,346.91082663)(435.14672295,346.91582663)(435.21672701,346.91583456)
\curveto(435.28672281,346.91582663)(435.36172274,346.92082662)(435.44172701,346.93083456)
}
}
{
\newrgbcolor{curcolor}{0 0 0}
\pscustom[linestyle=none,fillstyle=solid,fillcolor=curcolor]
{
\newpath
\moveto(447.53500826,339.58083456)
\curveto(447.55500041,339.47083407)(447.5650004,339.36083418)(447.56500826,339.25083456)
\curveto(447.57500039,339.1408344)(447.52500044,339.06583448)(447.41500826,339.02583456)
\curveto(447.35500061,338.99583455)(447.28500068,338.98083456)(447.20500826,338.98083456)
\lineto(446.96500826,338.98083456)
\lineto(446.15500826,338.98083456)
\lineto(445.88500826,338.98083456)
\curveto(445.80500216,338.99083455)(445.74000222,339.01583453)(445.69000826,339.05583456)
\curveto(445.62000234,339.09583445)(445.5650024,339.15083439)(445.52500826,339.22083456)
\curveto(445.49500247,339.30083424)(445.45000251,339.36583418)(445.39000826,339.41583456)
\curveto(445.37000259,339.43583411)(445.34500262,339.45083409)(445.31500826,339.46083456)
\curveto(445.28500268,339.48083406)(445.24500272,339.48583406)(445.19500826,339.47583456)
\curveto(445.14500282,339.45583409)(445.09500287,339.43083411)(445.04500826,339.40083456)
\curveto(445.00500296,339.37083417)(444.960003,339.3458342)(444.91000826,339.32583456)
\curveto(444.8600031,339.28583426)(444.80500316,339.25083429)(444.74500826,339.22083456)
\lineto(444.56500826,339.13083456)
\curveto(444.43500353,339.07083447)(444.30000366,339.02083452)(444.16000826,338.98083456)
\curveto(444.02000394,338.95083459)(443.87500409,338.91583463)(443.72500826,338.87583456)
\curveto(443.65500431,338.85583469)(443.58500438,338.8458347)(443.51500826,338.84583456)
\curveto(443.45500451,338.83583471)(443.39000457,338.82583472)(443.32000826,338.81583456)
\lineto(443.23000826,338.81583456)
\curveto(443.20000476,338.80583474)(443.17000479,338.80083474)(443.14000826,338.80083456)
\lineto(442.97500826,338.80083456)
\curveto(442.87500509,338.78083476)(442.77500519,338.78083476)(442.67500826,338.80083456)
\lineto(442.54000826,338.80083456)
\curveto(442.47000549,338.82083472)(442.40000556,338.83083471)(442.33000826,338.83083456)
\curveto(442.27000569,338.82083472)(442.21000575,338.82583472)(442.15000826,338.84583456)
\curveto(442.05000591,338.86583468)(441.95500601,338.88583466)(441.86500826,338.90583456)
\curveto(441.77500619,338.91583463)(441.69000627,338.9408346)(441.61000826,338.98083456)
\curveto(441.32000664,339.09083445)(441.07000689,339.23083431)(440.86000826,339.40083456)
\curveto(440.6600073,339.58083396)(440.50000746,339.81583373)(440.38000826,340.10583456)
\curveto(440.35000761,340.17583337)(440.32000764,340.25083329)(440.29000826,340.33083456)
\curveto(440.27000769,340.41083313)(440.25000771,340.49583305)(440.23000826,340.58583456)
\curveto(440.21000775,340.63583291)(440.20000776,340.68583286)(440.20000826,340.73583456)
\curveto(440.21000775,340.78583276)(440.21000775,340.83583271)(440.20000826,340.88583456)
\curveto(440.19000777,340.91583263)(440.18000778,340.97583257)(440.17000826,341.06583456)
\curveto(440.17000779,341.16583238)(440.17500779,341.23583231)(440.18500826,341.27583456)
\curveto(440.20500776,341.37583217)(440.21500775,341.46083208)(440.21500826,341.53083456)
\lineto(440.30500826,341.86083456)
\curveto(440.33500763,341.98083156)(440.37500759,342.08583146)(440.42500826,342.17583456)
\curveto(440.59500737,342.46583108)(440.79000717,342.68583086)(441.01000826,342.83583456)
\curveto(441.23000673,342.98583056)(441.51000645,343.11583043)(441.85000826,343.22583456)
\curveto(441.98000598,343.27583027)(442.11500585,343.31083023)(442.25500826,343.33083456)
\curveto(442.39500557,343.35083019)(442.53500543,343.37583017)(442.67500826,343.40583456)
\curveto(442.75500521,343.42583012)(442.84000512,343.43583011)(442.93000826,343.43583456)
\curveto(443.02000494,343.4458301)(443.11000485,343.46083008)(443.20000826,343.48083456)
\curveto(443.27000469,343.50083004)(443.34000462,343.50583004)(443.41000826,343.49583456)
\curveto(443.48000448,343.49583005)(443.55500441,343.50583004)(443.63500826,343.52583456)
\curveto(443.70500426,343.54583)(443.77500419,343.55582999)(443.84500826,343.55583456)
\curveto(443.91500405,343.55582999)(443.99000397,343.56582998)(444.07000826,343.58583456)
\curveto(444.28000368,343.63582991)(444.47000349,343.67582987)(444.64000826,343.70583456)
\curveto(444.82000314,343.7458298)(444.98000298,343.83582971)(445.12000826,343.97583456)
\curveto(445.21000275,344.06582948)(445.27000269,344.16582938)(445.30000826,344.27583456)
\curveto(445.31000265,344.30582924)(445.31000265,344.33082921)(445.30000826,344.35083456)
\curveto(445.30000266,344.37082917)(445.30500266,344.39082915)(445.31500826,344.41083456)
\curveto(445.32500264,344.43082911)(445.33000263,344.46082908)(445.33000826,344.50083456)
\lineto(445.33000826,344.59083456)
\lineto(445.30000826,344.71083456)
\curveto(445.30000266,344.75082879)(445.29500267,344.78582876)(445.28500826,344.81583456)
\curveto(445.18500278,345.11582843)(444.97500299,345.32082822)(444.65500826,345.43083456)
\curveto(444.5650034,345.46082808)(444.45500351,345.48082806)(444.32500826,345.49083456)
\curveto(444.20500376,345.51082803)(444.08000388,345.51582803)(443.95000826,345.50583456)
\curveto(443.82000414,345.50582804)(443.69500427,345.49582805)(443.57500826,345.47583456)
\curveto(443.45500451,345.45582809)(443.35000461,345.43082811)(443.26000826,345.40083456)
\curveto(443.20000476,345.38082816)(443.14000482,345.35082819)(443.08000826,345.31083456)
\curveto(443.03000493,345.28082826)(442.98000498,345.2458283)(442.93000826,345.20583456)
\curveto(442.88000508,345.16582838)(442.82500514,345.11082843)(442.76500826,345.04083456)
\curveto(442.71500525,344.97082857)(442.68000528,344.90582864)(442.66000826,344.84583456)
\curveto(442.61000535,344.7458288)(442.5650054,344.65082889)(442.52500826,344.56083456)
\curveto(442.49500547,344.47082907)(442.42500554,344.41082913)(442.31500826,344.38083456)
\curveto(442.23500573,344.36082918)(442.15000581,344.35082919)(442.06000826,344.35083456)
\lineto(441.79000826,344.35083456)
\lineto(441.22000826,344.35083456)
\curveto(441.17000679,344.35082919)(441.12000684,344.3458292)(441.07000826,344.33583456)
\curveto(441.02000694,344.33582921)(440.97500699,344.3408292)(440.93500826,344.35083456)
\lineto(440.80000826,344.35083456)
\curveto(440.78000718,344.36082918)(440.75500721,344.36582918)(440.72500826,344.36583456)
\curveto(440.69500727,344.36582918)(440.67000729,344.37582917)(440.65000826,344.39583456)
\curveto(440.57000739,344.41582913)(440.51500745,344.48082906)(440.48500826,344.59083456)
\curveto(440.47500749,344.6408289)(440.47500749,344.69082885)(440.48500826,344.74083456)
\curveto(440.49500747,344.79082875)(440.50500746,344.83582871)(440.51500826,344.87583456)
\curveto(440.54500742,344.98582856)(440.57500739,345.08582846)(440.60500826,345.17583456)
\curveto(440.64500732,345.27582827)(440.69000727,345.36582818)(440.74000826,345.44583456)
\lineto(440.83000826,345.59583456)
\lineto(440.92000826,345.74583456)
\curveto(441.00000696,345.85582769)(441.10000686,345.96082758)(441.22000826,346.06083456)
\curveto(441.24000672,346.07082747)(441.27000669,346.09582745)(441.31000826,346.13583456)
\curveto(441.3600066,346.17582737)(441.40500656,346.21082733)(441.44500826,346.24083456)
\curveto(441.48500648,346.27082727)(441.53000643,346.30082724)(441.58000826,346.33083456)
\curveto(441.75000621,346.4408271)(441.93000603,346.52582702)(442.12000826,346.58583456)
\curveto(442.31000565,346.65582689)(442.50500546,346.72082682)(442.70500826,346.78083456)
\curveto(442.82500514,346.81082673)(442.95000501,346.83082671)(443.08000826,346.84083456)
\curveto(443.21000475,346.85082669)(443.34000462,346.87082667)(443.47000826,346.90083456)
\curveto(443.51000445,346.91082663)(443.57000439,346.91082663)(443.65000826,346.90083456)
\curveto(443.74000422,346.89082665)(443.79500417,346.89582665)(443.81500826,346.91583456)
\curveto(444.22500374,346.92582662)(444.61500335,346.91082663)(444.98500826,346.87083456)
\curveto(445.3650026,346.83082671)(445.70500226,346.75582679)(446.00500826,346.64583456)
\curveto(446.31500165,346.53582701)(446.58000138,346.38582716)(446.80000826,346.19583456)
\curveto(447.02000094,346.01582753)(447.19000077,345.78082776)(447.31000826,345.49083456)
\curveto(447.38000058,345.32082822)(447.42000054,345.12582842)(447.43000826,344.90583456)
\curveto(447.44000052,344.68582886)(447.44500052,344.46082908)(447.44500826,344.23083456)
\lineto(447.44500826,340.88583456)
\lineto(447.44500826,340.30083456)
\curveto(447.44500052,340.11083343)(447.4650005,339.93583361)(447.50500826,339.77583456)
\curveto(447.51500045,339.7458338)(447.52000044,339.71083383)(447.52000826,339.67083456)
\curveto(447.52000044,339.6408339)(447.52500044,339.61083393)(447.53500826,339.58083456)
\moveto(445.33000826,341.89083456)
\curveto(445.34000262,341.9408316)(445.34500262,341.99583155)(445.34500826,342.05583456)
\curveto(445.34500262,342.12583142)(445.34000262,342.18583136)(445.33000826,342.23583456)
\curveto(445.31000265,342.29583125)(445.30000266,342.35083119)(445.30000826,342.40083456)
\curveto(445.30000266,342.45083109)(445.28000268,342.49083105)(445.24000826,342.52083456)
\curveto(445.19000277,342.56083098)(445.11500285,342.58083096)(445.01500826,342.58083456)
\curveto(444.97500299,342.57083097)(444.94000302,342.56083098)(444.91000826,342.55083456)
\curveto(444.88000308,342.55083099)(444.84500312,342.545831)(444.80500826,342.53583456)
\curveto(444.73500323,342.51583103)(444.6600033,342.50083104)(444.58000826,342.49083456)
\curveto(444.50000346,342.48083106)(444.42000354,342.46583108)(444.34000826,342.44583456)
\curveto(444.31000365,342.43583111)(444.2650037,342.43083111)(444.20500826,342.43083456)
\curveto(444.07500389,342.40083114)(443.94500402,342.38083116)(443.81500826,342.37083456)
\curveto(443.68500428,342.36083118)(443.5600044,342.33583121)(443.44000826,342.29583456)
\curveto(443.3600046,342.27583127)(443.28500468,342.25583129)(443.21500826,342.23583456)
\curveto(443.14500482,342.22583132)(443.07500489,342.20583134)(443.00500826,342.17583456)
\curveto(442.79500517,342.08583146)(442.61500535,341.95083159)(442.46500826,341.77083456)
\curveto(442.32500564,341.59083195)(442.27500569,341.3408322)(442.31500826,341.02083456)
\curveto(442.33500563,340.85083269)(442.39000557,340.71083283)(442.48000826,340.60083456)
\curveto(442.55000541,340.49083305)(442.65500531,340.40083314)(442.79500826,340.33083456)
\curveto(442.93500503,340.27083327)(443.08500488,340.22583332)(443.24500826,340.19583456)
\curveto(443.41500455,340.16583338)(443.59000437,340.15583339)(443.77000826,340.16583456)
\curveto(443.960004,340.18583336)(444.13500383,340.22083332)(444.29500826,340.27083456)
\curveto(444.55500341,340.35083319)(444.7600032,340.47583307)(444.91000826,340.64583456)
\curveto(445.0600029,340.82583272)(445.17500279,341.0458325)(445.25500826,341.30583456)
\curveto(445.27500269,341.37583217)(445.28500268,341.4458321)(445.28500826,341.51583456)
\curveto(445.29500267,341.59583195)(445.31000265,341.67583187)(445.33000826,341.75583456)
\lineto(445.33000826,341.89083456)
}
}
{
\newrgbcolor{curcolor}{0 0 0}
\pscustom[linestyle=none,fillstyle=solid,fillcolor=curcolor]
{
\newpath
\moveto(456.68828951,339.83583456)
\lineto(456.68828951,339.41583456)
\curveto(456.68828114,339.28583426)(456.65828117,339.18083436)(456.59828951,339.10083456)
\curveto(456.54828128,339.05083449)(456.48328134,339.01583453)(456.40328951,338.99583456)
\curveto(456.3232815,338.98583456)(456.23328159,338.98083456)(456.13328951,338.98083456)
\lineto(455.30828951,338.98083456)
\lineto(455.02328951,338.98083456)
\curveto(454.94328288,338.99083455)(454.87828295,339.01583453)(454.82828951,339.05583456)
\curveto(454.75828307,339.10583444)(454.71828311,339.17083437)(454.70828951,339.25083456)
\curveto(454.69828313,339.33083421)(454.67828315,339.41083413)(454.64828951,339.49083456)
\curveto(454.6282832,339.51083403)(454.60828322,339.52583402)(454.58828951,339.53583456)
\curveto(454.57828325,339.55583399)(454.56328326,339.57583397)(454.54328951,339.59583456)
\curveto(454.43328339,339.59583395)(454.35328347,339.57083397)(454.30328951,339.52083456)
\lineto(454.15328951,339.37083456)
\curveto(454.08328374,339.32083422)(454.01828381,339.27583427)(453.95828951,339.23583456)
\curveto(453.89828393,339.20583434)(453.83328399,339.16583438)(453.76328951,339.11583456)
\curveto(453.7232841,339.09583445)(453.67828415,339.07583447)(453.62828951,339.05583456)
\curveto(453.58828424,339.03583451)(453.54328428,339.01583453)(453.49328951,338.99583456)
\curveto(453.35328447,338.9458346)(453.20328462,338.90083464)(453.04328951,338.86083456)
\curveto(452.99328483,338.8408347)(452.94828488,338.83083471)(452.90828951,338.83083456)
\curveto(452.86828496,338.83083471)(452.828285,338.82583472)(452.78828951,338.81583456)
\lineto(452.65328951,338.81583456)
\curveto(452.6232852,338.80583474)(452.58328524,338.80083474)(452.53328951,338.80083456)
\lineto(452.39828951,338.80083456)
\curveto(452.33828549,338.78083476)(452.24828558,338.77583477)(452.12828951,338.78583456)
\curveto(452.00828582,338.78583476)(451.9232859,338.79583475)(451.87328951,338.81583456)
\curveto(451.80328602,338.83583471)(451.73828609,338.8458347)(451.67828951,338.84583456)
\curveto(451.6282862,338.83583471)(451.57328625,338.8408347)(451.51328951,338.86083456)
\lineto(451.15328951,338.98083456)
\curveto(451.04328678,339.01083453)(450.93328689,339.05083449)(450.82328951,339.10083456)
\curveto(450.47328735,339.25083429)(450.15828767,339.48083406)(449.87828951,339.79083456)
\curveto(449.60828822,340.11083343)(449.39328843,340.4458331)(449.23328951,340.79583456)
\curveto(449.18328864,340.90583264)(449.14328868,341.01083253)(449.11328951,341.11083456)
\curveto(449.08328874,341.22083232)(449.04828878,341.33083221)(449.00828951,341.44083456)
\curveto(448.99828883,341.48083206)(448.99328883,341.51583203)(448.99328951,341.54583456)
\curveto(448.99328883,341.58583196)(448.98328884,341.63083191)(448.96328951,341.68083456)
\curveto(448.94328888,341.76083178)(448.9232889,341.8458317)(448.90328951,341.93583456)
\curveto(448.89328893,342.03583151)(448.87828895,342.13583141)(448.85828951,342.23583456)
\curveto(448.84828898,342.26583128)(448.84328898,342.30083124)(448.84328951,342.34083456)
\curveto(448.85328897,342.38083116)(448.85328897,342.41583113)(448.84328951,342.44583456)
\lineto(448.84328951,342.58083456)
\curveto(448.84328898,342.63083091)(448.83828899,342.68083086)(448.82828951,342.73083456)
\curveto(448.81828901,342.78083076)(448.81328901,342.83583071)(448.81328951,342.89583456)
\curveto(448.81328901,342.96583058)(448.81828901,343.02083052)(448.82828951,343.06083456)
\curveto(448.83828899,343.11083043)(448.84328898,343.15583039)(448.84328951,343.19583456)
\lineto(448.84328951,343.34583456)
\curveto(448.85328897,343.39583015)(448.85328897,343.4408301)(448.84328951,343.48083456)
\curveto(448.84328898,343.53083001)(448.85328897,343.58082996)(448.87328951,343.63083456)
\curveto(448.89328893,343.7408298)(448.90828892,343.8458297)(448.91828951,343.94583456)
\curveto(448.93828889,344.0458295)(448.96328886,344.1458294)(448.99328951,344.24583456)
\curveto(449.03328879,344.36582918)(449.06828876,344.48082906)(449.09828951,344.59083456)
\curveto(449.1282887,344.70082884)(449.16828866,344.81082873)(449.21828951,344.92083456)
\curveto(449.35828847,345.22082832)(449.53328829,345.50582804)(449.74328951,345.77583456)
\curveto(449.76328806,345.80582774)(449.78828804,345.83082771)(449.81828951,345.85083456)
\curveto(449.85828797,345.88082766)(449.88828794,345.91082763)(449.90828951,345.94083456)
\curveto(449.94828788,345.99082755)(449.98828784,346.03582751)(450.02828951,346.07583456)
\curveto(450.06828776,346.11582743)(450.11328771,346.15582739)(450.16328951,346.19583456)
\curveto(450.20328762,346.21582733)(450.23828759,346.2408273)(450.26828951,346.27083456)
\curveto(450.29828753,346.31082723)(450.33328749,346.3408272)(450.37328951,346.36083456)
\curveto(450.6232872,346.53082701)(450.91328691,346.67082687)(451.24328951,346.78083456)
\curveto(451.31328651,346.80082674)(451.38328644,346.81582673)(451.45328951,346.82583456)
\curveto(451.53328629,346.83582671)(451.61328621,346.85082669)(451.69328951,346.87083456)
\curveto(451.76328606,346.89082665)(451.85328597,346.90082664)(451.96328951,346.90083456)
\curveto(452.07328575,346.91082663)(452.18328564,346.91582663)(452.29328951,346.91583456)
\curveto(452.40328542,346.91582663)(452.50828532,346.91082663)(452.60828951,346.90083456)
\curveto(452.71828511,346.89082665)(452.80828502,346.87582667)(452.87828951,346.85583456)
\curveto(453.0282848,346.80582674)(453.17328465,346.76082678)(453.31328951,346.72083456)
\curveto(453.45328437,346.68082686)(453.58328424,346.62582692)(453.70328951,346.55583456)
\curveto(453.77328405,346.50582704)(453.83828399,346.45582709)(453.89828951,346.40583456)
\curveto(453.95828387,346.36582718)(454.0232838,346.32082722)(454.09328951,346.27083456)
\curveto(454.13328369,346.2408273)(454.18828364,346.20082734)(454.25828951,346.15083456)
\curveto(454.33828349,346.10082744)(454.41328341,346.10082744)(454.48328951,346.15083456)
\curveto(454.5232833,346.17082737)(454.54328328,346.20582734)(454.54328951,346.25583456)
\curveto(454.54328328,346.30582724)(454.55328327,346.35582719)(454.57328951,346.40583456)
\lineto(454.57328951,346.55583456)
\curveto(454.58328324,346.58582696)(454.58828324,346.62082692)(454.58828951,346.66083456)
\lineto(454.58828951,346.78083456)
\lineto(454.58828951,348.82083456)
\curveto(454.58828324,348.93082461)(454.58328324,349.05082449)(454.57328951,349.18083456)
\curveto(454.57328325,349.32082422)(454.59828323,349.42582412)(454.64828951,349.49583456)
\curveto(454.68828314,349.57582397)(454.76328306,349.62582392)(454.87328951,349.64583456)
\curveto(454.89328293,349.65582389)(454.91328291,349.65582389)(454.93328951,349.64583456)
\curveto(454.95328287,349.6458239)(454.97328285,349.65082389)(454.99328951,349.66083456)
\lineto(456.05828951,349.66083456)
\curveto(456.17828165,349.66082388)(456.28828154,349.65582389)(456.38828951,349.64583456)
\curveto(456.48828134,349.63582391)(456.56328126,349.59582395)(456.61328951,349.52583456)
\curveto(456.66328116,349.4458241)(456.68828114,349.3408242)(456.68828951,349.21083456)
\lineto(456.68828951,348.85083456)
\lineto(456.68828951,339.83583456)
\moveto(454.64828951,342.77583456)
\curveto(454.65828317,342.81583073)(454.65828317,342.85583069)(454.64828951,342.89583456)
\lineto(454.64828951,343.03083456)
\curveto(454.64828318,343.13083041)(454.64328318,343.23083031)(454.63328951,343.33083456)
\curveto(454.6232832,343.43083011)(454.60828322,343.52083002)(454.58828951,343.60083456)
\curveto(454.56828326,343.71082983)(454.54828328,343.81082973)(454.52828951,343.90083456)
\curveto(454.51828331,343.99082955)(454.49328333,344.07582947)(454.45328951,344.15583456)
\curveto(454.31328351,344.51582903)(454.10828372,344.80082874)(453.83828951,345.01083456)
\curveto(453.57828425,345.22082832)(453.19828463,345.32582822)(452.69828951,345.32583456)
\curveto(452.63828519,345.32582822)(452.55828527,345.31582823)(452.45828951,345.29583456)
\curveto(452.37828545,345.27582827)(452.30328552,345.25582829)(452.23328951,345.23583456)
\curveto(452.17328565,345.22582832)(452.11328571,345.20582834)(452.05328951,345.17583456)
\curveto(451.78328604,345.06582848)(451.57328625,344.89582865)(451.42328951,344.66583456)
\curveto(451.27328655,344.43582911)(451.15328667,344.17582937)(451.06328951,343.88583456)
\curveto(451.03328679,343.78582976)(451.01328681,343.68582986)(451.00328951,343.58583456)
\curveto(450.99328683,343.48583006)(450.97328685,343.38083016)(450.94328951,343.27083456)
\lineto(450.94328951,343.06083456)
\curveto(450.9232869,342.97083057)(450.91828691,342.8458307)(450.92828951,342.68583456)
\curveto(450.93828689,342.53583101)(450.95328687,342.42583112)(450.97328951,342.35583456)
\lineto(450.97328951,342.26583456)
\curveto(450.98328684,342.2458313)(450.98828684,342.22583132)(450.98828951,342.20583456)
\curveto(451.00828682,342.12583142)(451.0232868,342.05083149)(451.03328951,341.98083456)
\curveto(451.05328677,341.91083163)(451.07328675,341.83583171)(451.09328951,341.75583456)
\curveto(451.26328656,341.23583231)(451.55328627,340.85083269)(451.96328951,340.60083456)
\curveto(452.09328573,340.51083303)(452.27328555,340.4408331)(452.50328951,340.39083456)
\curveto(452.54328528,340.38083316)(452.60328522,340.37583317)(452.68328951,340.37583456)
\curveto(452.71328511,340.36583318)(452.75828507,340.35583319)(452.81828951,340.34583456)
\curveto(452.88828494,340.3458332)(452.94328488,340.35083319)(452.98328951,340.36083456)
\curveto(453.06328476,340.38083316)(453.14328468,340.39583315)(453.22328951,340.40583456)
\curveto(453.30328452,340.41583313)(453.38328444,340.43583311)(453.46328951,340.46583456)
\curveto(453.71328411,340.57583297)(453.91328391,340.71583283)(454.06328951,340.88583456)
\curveto(454.21328361,341.05583249)(454.34328348,341.27083227)(454.45328951,341.53083456)
\curveto(454.49328333,341.62083192)(454.5232833,341.71083183)(454.54328951,341.80083456)
\curveto(454.56328326,341.90083164)(454.58328324,342.00583154)(454.60328951,342.11583456)
\curveto(454.61328321,342.16583138)(454.61328321,342.21083133)(454.60328951,342.25083456)
\curveto(454.60328322,342.30083124)(454.61328321,342.35083119)(454.63328951,342.40083456)
\curveto(454.64328318,342.43083111)(454.64828318,342.46583108)(454.64828951,342.50583456)
\lineto(454.64828951,342.64083456)
\lineto(454.64828951,342.77583456)
}
}
{
\newrgbcolor{curcolor}{0 0 0}
\pscustom[linestyle=none,fillstyle=solid,fillcolor=curcolor]
{
\newpath
\moveto(466.03821138,343.16583456)
\curveto(466.05820281,343.10583044)(466.0682028,343.02083052)(466.06821138,342.91083456)
\curveto(466.0682028,342.80083074)(466.05820281,342.71583083)(466.03821138,342.65583456)
\lineto(466.03821138,342.50583456)
\curveto(466.01820285,342.42583112)(466.00820286,342.3458312)(466.00821138,342.26583456)
\curveto(466.01820285,342.18583136)(466.01320286,342.10583144)(465.99321138,342.02583456)
\curveto(465.9732029,341.95583159)(465.95820291,341.89083165)(465.94821138,341.83083456)
\curveto(465.93820293,341.77083177)(465.92820294,341.70583184)(465.91821138,341.63583456)
\curveto(465.87820299,341.52583202)(465.84320303,341.41083213)(465.81321138,341.29083456)
\curveto(465.78320309,341.18083236)(465.74320313,341.07583247)(465.69321138,340.97583456)
\curveto(465.48320339,340.49583305)(465.20820366,340.10583344)(464.86821138,339.80583456)
\curveto(464.52820434,339.50583404)(464.11820475,339.25583429)(463.63821138,339.05583456)
\curveto(463.51820535,339.00583454)(463.39320548,338.97083457)(463.26321138,338.95083456)
\curveto(463.14320573,338.92083462)(463.01820585,338.89083465)(462.88821138,338.86083456)
\curveto(462.83820603,338.8408347)(462.78320609,338.83083471)(462.72321138,338.83083456)
\curveto(462.66320621,338.83083471)(462.60820626,338.82583472)(462.55821138,338.81583456)
\lineto(462.45321138,338.81583456)
\curveto(462.42320645,338.80583474)(462.39320648,338.80083474)(462.36321138,338.80083456)
\curveto(462.31320656,338.79083475)(462.23320664,338.78583476)(462.12321138,338.78583456)
\curveto(462.01320686,338.77583477)(461.92820694,338.78083476)(461.86821138,338.80083456)
\lineto(461.71821138,338.80083456)
\curveto(461.6682072,338.81083473)(461.61320726,338.81583473)(461.55321138,338.81583456)
\curveto(461.50320737,338.80583474)(461.45320742,338.81083473)(461.40321138,338.83083456)
\curveto(461.36320751,338.8408347)(461.32320755,338.8458347)(461.28321138,338.84583456)
\curveto(461.25320762,338.8458347)(461.21320766,338.85083469)(461.16321138,338.86083456)
\curveto(461.06320781,338.89083465)(460.96320791,338.91583463)(460.86321138,338.93583456)
\curveto(460.76320811,338.95583459)(460.6682082,338.98583456)(460.57821138,339.02583456)
\curveto(460.45820841,339.06583448)(460.34320853,339.10583444)(460.23321138,339.14583456)
\curveto(460.13320874,339.18583436)(460.02820884,339.23583431)(459.91821138,339.29583456)
\curveto(459.5682093,339.50583404)(459.2682096,339.75083379)(459.01821138,340.03083456)
\curveto(458.7682101,340.31083323)(458.55821031,340.6458329)(458.38821138,341.03583456)
\curveto(458.33821053,341.12583242)(458.29821057,341.22083232)(458.26821138,341.32083456)
\curveto(458.24821062,341.42083212)(458.22321065,341.52583202)(458.19321138,341.63583456)
\curveto(458.1732107,341.68583186)(458.16321071,341.73083181)(458.16321138,341.77083456)
\curveto(458.16321071,341.81083173)(458.15321072,341.85583169)(458.13321138,341.90583456)
\curveto(458.11321076,341.98583156)(458.10321077,342.06583148)(458.10321138,342.14583456)
\curveto(458.10321077,342.23583131)(458.09321078,342.32083122)(458.07321138,342.40083456)
\curveto(458.06321081,342.45083109)(458.05821081,342.49583105)(458.05821138,342.53583456)
\lineto(458.05821138,342.67083456)
\curveto(458.03821083,342.73083081)(458.02821084,342.81583073)(458.02821138,342.92583456)
\curveto(458.03821083,343.03583051)(458.05321082,343.12083042)(458.07321138,343.18083456)
\lineto(458.07321138,343.28583456)
\curveto(458.08321079,343.33583021)(458.08321079,343.38583016)(458.07321138,343.43583456)
\curveto(458.0732108,343.49583005)(458.08321079,343.55082999)(458.10321138,343.60083456)
\curveto(458.11321076,343.65082989)(458.11821075,343.69582985)(458.11821138,343.73583456)
\curveto(458.11821075,343.78582976)(458.12821074,343.83582971)(458.14821138,343.88583456)
\curveto(458.18821068,344.01582953)(458.22321065,344.1408294)(458.25321138,344.26083456)
\curveto(458.28321059,344.39082915)(458.32321055,344.51582903)(458.37321138,344.63583456)
\curveto(458.55321032,345.0458285)(458.7682101,345.38582816)(459.01821138,345.65583456)
\curveto(459.2682096,345.93582761)(459.5732093,346.19082735)(459.93321138,346.42083456)
\curveto(460.03320884,346.47082707)(460.13820873,346.51582703)(460.24821138,346.55583456)
\curveto(460.35820851,346.59582695)(460.4682084,346.6408269)(460.57821138,346.69083456)
\curveto(460.70820816,346.7408268)(460.84320803,346.77582677)(460.98321138,346.79583456)
\curveto(461.12320775,346.81582673)(461.2682076,346.8458267)(461.41821138,346.88583456)
\curveto(461.49820737,346.89582665)(461.5732073,346.90082664)(461.64321138,346.90083456)
\curveto(461.71320716,346.90082664)(461.78320709,346.90582664)(461.85321138,346.91583456)
\curveto(462.43320644,346.92582662)(462.93320594,346.86582668)(463.35321138,346.73583456)
\curveto(463.78320509,346.60582694)(464.16320471,346.42582712)(464.49321138,346.19583456)
\curveto(464.60320427,346.11582743)(464.71320416,346.02582752)(464.82321138,345.92583456)
\curveto(464.94320393,345.83582771)(465.04320383,345.73582781)(465.12321138,345.62583456)
\curveto(465.20320367,345.52582802)(465.2732036,345.42582812)(465.33321138,345.32583456)
\curveto(465.40320347,345.22582832)(465.4732034,345.12082842)(465.54321138,345.01083456)
\curveto(465.61320326,344.90082864)(465.6682032,344.78082876)(465.70821138,344.65083456)
\curveto(465.74820312,344.53082901)(465.79320308,344.40082914)(465.84321138,344.26083456)
\curveto(465.873203,344.18082936)(465.89820297,344.09582945)(465.91821138,344.00583456)
\lineto(465.97821138,343.73583456)
\curveto(465.98820288,343.69582985)(465.99320288,343.65582989)(465.99321138,343.61583456)
\curveto(465.99320288,343.57582997)(465.99820287,343.53583001)(466.00821138,343.49583456)
\curveto(466.02820284,343.4458301)(466.03320284,343.39083015)(466.02321138,343.33083456)
\curveto(466.01320286,343.27083027)(466.01820285,343.21583033)(466.03821138,343.16583456)
\moveto(463.93821138,342.62583456)
\curveto(463.94820492,342.67583087)(463.95320492,342.7458308)(463.95321138,342.83583456)
\curveto(463.95320492,342.93583061)(463.94820492,343.01083053)(463.93821138,343.06083456)
\lineto(463.93821138,343.18083456)
\curveto(463.91820495,343.23083031)(463.90820496,343.28583026)(463.90821138,343.34583456)
\curveto(463.90820496,343.40583014)(463.90320497,343.46083008)(463.89321138,343.51083456)
\curveto(463.89320498,343.55082999)(463.88820498,343.58082996)(463.87821138,343.60083456)
\lineto(463.81821138,343.84083456)
\curveto(463.80820506,343.93082961)(463.78820508,344.01582953)(463.75821138,344.09583456)
\curveto(463.64820522,344.35582919)(463.51820535,344.57582897)(463.36821138,344.75583456)
\curveto(463.21820565,344.9458286)(463.01820585,345.09582845)(462.76821138,345.20583456)
\curveto(462.70820616,345.22582832)(462.64820622,345.2408283)(462.58821138,345.25083456)
\curveto(462.52820634,345.27082827)(462.46320641,345.29082825)(462.39321138,345.31083456)
\curveto(462.31320656,345.33082821)(462.22820664,345.33582821)(462.13821138,345.32583456)
\lineto(461.86821138,345.32583456)
\curveto(461.83820703,345.30582824)(461.80320707,345.29582825)(461.76321138,345.29583456)
\curveto(461.72320715,345.30582824)(461.68820718,345.30582824)(461.65821138,345.29583456)
\lineto(461.44821138,345.23583456)
\curveto(461.38820748,345.22582832)(461.33320754,345.20582834)(461.28321138,345.17583456)
\curveto(461.03320784,345.06582848)(460.82820804,344.90582864)(460.66821138,344.69583456)
\curveto(460.51820835,344.49582905)(460.39820847,344.26082928)(460.30821138,343.99083456)
\curveto(460.27820859,343.89082965)(460.25320862,343.78582976)(460.23321138,343.67583456)
\curveto(460.22320865,343.56582998)(460.20820866,343.45583009)(460.18821138,343.34583456)
\curveto(460.17820869,343.29583025)(460.1732087,343.2458303)(460.17321138,343.19583456)
\lineto(460.17321138,343.04583456)
\curveto(460.15320872,342.97583057)(460.14320873,342.87083067)(460.14321138,342.73083456)
\curveto(460.15320872,342.59083095)(460.1682087,342.48583106)(460.18821138,342.41583456)
\lineto(460.18821138,342.28083456)
\curveto(460.20820866,342.20083134)(460.22320865,342.12083142)(460.23321138,342.04083456)
\curveto(460.24320863,341.97083157)(460.25820861,341.89583165)(460.27821138,341.81583456)
\curveto(460.37820849,341.51583203)(460.48320839,341.27083227)(460.59321138,341.08083456)
\curveto(460.71320816,340.90083264)(460.89820797,340.73583281)(461.14821138,340.58583456)
\curveto(461.21820765,340.53583301)(461.29320758,340.49583305)(461.37321138,340.46583456)
\curveto(461.46320741,340.43583311)(461.55320732,340.41083313)(461.64321138,340.39083456)
\curveto(461.68320719,340.38083316)(461.71820715,340.37583317)(461.74821138,340.37583456)
\curveto(461.77820709,340.38583316)(461.81320706,340.38583316)(461.85321138,340.37583456)
\lineto(461.97321138,340.34583456)
\curveto(462.02320685,340.3458332)(462.0682068,340.35083319)(462.10821138,340.36083456)
\lineto(462.22821138,340.36083456)
\curveto(462.30820656,340.38083316)(462.38820648,340.39583315)(462.46821138,340.40583456)
\curveto(462.54820632,340.41583313)(462.62320625,340.43583311)(462.69321138,340.46583456)
\curveto(462.95320592,340.56583298)(463.16320571,340.70083284)(463.32321138,340.87083456)
\curveto(463.48320539,341.0408325)(463.61820525,341.25083229)(463.72821138,341.50083456)
\curveto(463.7682051,341.60083194)(463.79820507,341.70083184)(463.81821138,341.80083456)
\curveto(463.83820503,341.90083164)(463.86320501,342.00583154)(463.89321138,342.11583456)
\curveto(463.90320497,342.15583139)(463.90820496,342.19083135)(463.90821138,342.22083456)
\curveto(463.90820496,342.26083128)(463.91320496,342.30083124)(463.92321138,342.34083456)
\lineto(463.92321138,342.47583456)
\curveto(463.92320495,342.52583102)(463.92820494,342.57583097)(463.93821138,342.62583456)
}
}
{
\newrgbcolor{curcolor}{0 0 0}
\pscustom[linestyle=none,fillstyle=solid,fillcolor=curcolor]
{
\newpath
\moveto(471.86313326,346.91583456)
\curveto(471.97312794,346.91582663)(472.06812785,346.90582664)(472.14813326,346.88583456)
\curveto(472.23812768,346.86582668)(472.30812761,346.82082672)(472.35813326,346.75083456)
\curveto(472.4181275,346.67082687)(472.44812747,346.53082701)(472.44813326,346.33083456)
\lineto(472.44813326,345.82083456)
\lineto(472.44813326,345.44583456)
\curveto(472.45812746,345.30582824)(472.44312747,345.19582835)(472.40313326,345.11583456)
\curveto(472.36312755,345.0458285)(472.30312761,345.00082854)(472.22313326,344.98083456)
\curveto(472.15312776,344.96082858)(472.06812785,344.95082859)(471.96813326,344.95083456)
\curveto(471.87812804,344.95082859)(471.77812814,344.95582859)(471.66813326,344.96583456)
\curveto(471.56812835,344.97582857)(471.47312844,344.97082857)(471.38313326,344.95083456)
\curveto(471.3131286,344.93082861)(471.24312867,344.91582863)(471.17313326,344.90583456)
\curveto(471.10312881,344.90582864)(471.03812888,344.89582865)(470.97813326,344.87583456)
\curveto(470.8181291,344.82582872)(470.65812926,344.75082879)(470.49813326,344.65083456)
\curveto(470.33812958,344.56082898)(470.2131297,344.45582909)(470.12313326,344.33583456)
\curveto(470.07312984,344.25582929)(470.0181299,344.17082937)(469.95813326,344.08083456)
\curveto(469.90813001,344.00082954)(469.85813006,343.91582963)(469.80813326,343.82583456)
\curveto(469.77813014,343.7458298)(469.74813017,343.66082988)(469.71813326,343.57083456)
\lineto(469.65813326,343.33083456)
\curveto(469.63813028,343.26083028)(469.62813029,343.18583036)(469.62813326,343.10583456)
\curveto(469.62813029,343.03583051)(469.6181303,342.96583058)(469.59813326,342.89583456)
\curveto(469.58813033,342.85583069)(469.58313033,342.81583073)(469.58313326,342.77583456)
\curveto(469.59313032,342.7458308)(469.59313032,342.71583083)(469.58313326,342.68583456)
\lineto(469.58313326,342.44583456)
\curveto(469.56313035,342.37583117)(469.55813036,342.29583125)(469.56813326,342.20583456)
\curveto(469.57813034,342.12583142)(469.58313033,342.0458315)(469.58313326,341.96583456)
\lineto(469.58313326,341.00583456)
\lineto(469.58313326,339.73083456)
\curveto(469.58313033,339.60083394)(469.57813034,339.48083406)(469.56813326,339.37083456)
\curveto(469.55813036,339.26083428)(469.52813039,339.17083437)(469.47813326,339.10083456)
\curveto(469.45813046,339.07083447)(469.42313049,339.0458345)(469.37313326,339.02583456)
\curveto(469.33313058,339.01583453)(469.28813063,339.00583454)(469.23813326,338.99583456)
\lineto(469.16313326,338.99583456)
\curveto(469.1131308,338.98583456)(469.05813086,338.98083456)(468.99813326,338.98083456)
\lineto(468.83313326,338.98083456)
\lineto(468.18813326,338.98083456)
\curveto(468.12813179,338.99083455)(468.06313185,338.99583455)(467.99313326,338.99583456)
\lineto(467.79813326,338.99583456)
\curveto(467.74813217,339.01583453)(467.69813222,339.03083451)(467.64813326,339.04083456)
\curveto(467.59813232,339.06083448)(467.56313235,339.09583445)(467.54313326,339.14583456)
\curveto(467.50313241,339.19583435)(467.47813244,339.26583428)(467.46813326,339.35583456)
\lineto(467.46813326,339.65583456)
\lineto(467.46813326,340.67583456)
\lineto(467.46813326,344.90583456)
\lineto(467.46813326,346.01583456)
\lineto(467.46813326,346.30083456)
\curveto(467.46813245,346.40082714)(467.48813243,346.48082706)(467.52813326,346.54083456)
\curveto(467.57813234,346.62082692)(467.65313226,346.67082687)(467.75313326,346.69083456)
\curveto(467.85313206,346.71082683)(467.97313194,346.72082682)(468.11313326,346.72083456)
\lineto(468.87813326,346.72083456)
\curveto(468.99813092,346.72082682)(469.10313081,346.71082683)(469.19313326,346.69083456)
\curveto(469.28313063,346.68082686)(469.35313056,346.63582691)(469.40313326,346.55583456)
\curveto(469.43313048,346.50582704)(469.44813047,346.43582711)(469.44813326,346.34583456)
\lineto(469.47813326,346.07583456)
\curveto(469.48813043,345.99582755)(469.50313041,345.92082762)(469.52313326,345.85083456)
\curveto(469.55313036,345.78082776)(469.60313031,345.7458278)(469.67313326,345.74583456)
\curveto(469.69313022,345.76582778)(469.7131302,345.77582777)(469.73313326,345.77583456)
\curveto(469.75313016,345.77582777)(469.77313014,345.78582776)(469.79313326,345.80583456)
\curveto(469.85313006,345.85582769)(469.90313001,345.91082763)(469.94313326,345.97083456)
\curveto(469.99312992,346.0408275)(470.05312986,346.10082744)(470.12313326,346.15083456)
\curveto(470.16312975,346.18082736)(470.19812972,346.21082733)(470.22813326,346.24083456)
\curveto(470.25812966,346.28082726)(470.29312962,346.31582723)(470.33313326,346.34583456)
\lineto(470.60313326,346.52583456)
\curveto(470.70312921,346.58582696)(470.80312911,346.6408269)(470.90313326,346.69083456)
\curveto(471.00312891,346.73082681)(471.10312881,346.76582678)(471.20313326,346.79583456)
\lineto(471.53313326,346.88583456)
\curveto(471.56312835,346.89582665)(471.6181283,346.89582665)(471.69813326,346.88583456)
\curveto(471.78812813,346.88582666)(471.84312807,346.89582665)(471.86313326,346.91583456)
}
}
{
\newrgbcolor{curcolor}{0 0 0}
\pscustom[linestyle=none,fillstyle=solid,fillcolor=curcolor]
{
\newpath
\moveto(480.36953951,342.92583456)
\curveto(480.38953134,342.8458307)(480.38953134,342.75583079)(480.36953951,342.65583456)
\curveto(480.34953138,342.55583099)(480.31453142,342.49083105)(480.26453951,342.46083456)
\curveto(480.21453152,342.42083112)(480.13953159,342.39083115)(480.03953951,342.37083456)
\curveto(479.94953178,342.36083118)(479.84453189,342.35083119)(479.72453951,342.34083456)
\lineto(479.37953951,342.34083456)
\curveto(479.26953246,342.35083119)(479.16953256,342.35583119)(479.07953951,342.35583456)
\lineto(475.41953951,342.35583456)
\lineto(475.20953951,342.35583456)
\curveto(475.14953658,342.35583119)(475.09453664,342.3458312)(475.04453951,342.32583456)
\curveto(474.96453677,342.28583126)(474.91453682,342.2458313)(474.89453951,342.20583456)
\curveto(474.87453686,342.18583136)(474.85453688,342.1458314)(474.83453951,342.08583456)
\curveto(474.81453692,342.03583151)(474.80953692,341.98583156)(474.81953951,341.93583456)
\curveto(474.83953689,341.87583167)(474.84953688,341.81583173)(474.84953951,341.75583456)
\curveto(474.85953687,341.70583184)(474.87453686,341.65083189)(474.89453951,341.59083456)
\curveto(474.97453676,341.35083219)(475.06953666,341.15083239)(475.17953951,340.99083456)
\curveto(475.29953643,340.8408327)(475.45953627,340.70583284)(475.65953951,340.58583456)
\curveto(475.73953599,340.53583301)(475.81953591,340.50083304)(475.89953951,340.48083456)
\curveto(475.98953574,340.47083307)(476.07953565,340.45083309)(476.16953951,340.42083456)
\curveto(476.24953548,340.40083314)(476.35953537,340.38583316)(476.49953951,340.37583456)
\curveto(476.63953509,340.36583318)(476.75953497,340.37083317)(476.85953951,340.39083456)
\lineto(476.99453951,340.39083456)
\curveto(477.09453464,340.41083313)(477.18453455,340.43083311)(477.26453951,340.45083456)
\curveto(477.35453438,340.48083306)(477.43953429,340.51083303)(477.51953951,340.54083456)
\curveto(477.61953411,340.59083295)(477.729534,340.65583289)(477.84953951,340.73583456)
\curveto(477.97953375,340.81583273)(478.07453366,340.89583265)(478.13453951,340.97583456)
\curveto(478.18453355,341.0458325)(478.2345335,341.11083243)(478.28453951,341.17083456)
\curveto(478.34453339,341.2408323)(478.41453332,341.29083225)(478.49453951,341.32083456)
\curveto(478.59453314,341.37083217)(478.71953301,341.39083215)(478.86953951,341.38083456)
\lineto(479.30453951,341.38083456)
\lineto(479.48453951,341.38083456)
\curveto(479.55453218,341.39083215)(479.61453212,341.38583216)(479.66453951,341.36583456)
\lineto(479.81453951,341.36583456)
\curveto(479.91453182,341.3458322)(479.98453175,341.32083222)(480.02453951,341.29083456)
\curveto(480.06453167,341.27083227)(480.08453165,341.22583232)(480.08453951,341.15583456)
\curveto(480.09453164,341.08583246)(480.08953164,341.02583252)(480.06953951,340.97583456)
\curveto(480.01953171,340.83583271)(479.96453177,340.71083283)(479.90453951,340.60083456)
\curveto(479.84453189,340.49083305)(479.77453196,340.38083316)(479.69453951,340.27083456)
\curveto(479.47453226,339.9408336)(479.22453251,339.67583387)(478.94453951,339.47583456)
\curveto(478.66453307,339.27583427)(478.31453342,339.10583444)(477.89453951,338.96583456)
\curveto(477.78453395,338.92583462)(477.67453406,338.90083464)(477.56453951,338.89083456)
\curveto(477.45453428,338.88083466)(477.33953439,338.86083468)(477.21953951,338.83083456)
\curveto(477.17953455,338.82083472)(477.1345346,338.82083472)(477.08453951,338.83083456)
\curveto(477.04453469,338.83083471)(477.00453473,338.82583472)(476.96453951,338.81583456)
\lineto(476.79953951,338.81583456)
\curveto(476.74953498,338.79583475)(476.68953504,338.79083475)(476.61953951,338.80083456)
\curveto(476.55953517,338.80083474)(476.50453523,338.80583474)(476.45453951,338.81583456)
\curveto(476.37453536,338.82583472)(476.30453543,338.82583472)(476.24453951,338.81583456)
\curveto(476.18453555,338.80583474)(476.11953561,338.81083473)(476.04953951,338.83083456)
\curveto(475.99953573,338.85083469)(475.94453579,338.86083468)(475.88453951,338.86083456)
\curveto(475.82453591,338.86083468)(475.76953596,338.87083467)(475.71953951,338.89083456)
\curveto(475.60953612,338.91083463)(475.49953623,338.93583461)(475.38953951,338.96583456)
\curveto(475.27953645,338.98583456)(475.17953655,339.02083452)(475.08953951,339.07083456)
\curveto(474.97953675,339.11083443)(474.87453686,339.1458344)(474.77453951,339.17583456)
\curveto(474.68453705,339.21583433)(474.59953713,339.26083428)(474.51953951,339.31083456)
\curveto(474.19953753,339.51083403)(473.91453782,339.7408338)(473.66453951,340.00083456)
\curveto(473.41453832,340.27083327)(473.20953852,340.58083296)(473.04953951,340.93083456)
\curveto(472.99953873,341.0408325)(472.95953877,341.15083239)(472.92953951,341.26083456)
\curveto(472.89953883,341.38083216)(472.85953887,341.50083204)(472.80953951,341.62083456)
\curveto(472.79953893,341.66083188)(472.79453894,341.69583185)(472.79453951,341.72583456)
\curveto(472.79453894,341.76583178)(472.78953894,341.80583174)(472.77953951,341.84583456)
\curveto(472.73953899,341.96583158)(472.71453902,342.09583145)(472.70453951,342.23583456)
\lineto(472.67453951,342.65583456)
\curveto(472.67453906,342.70583084)(472.66953906,342.76083078)(472.65953951,342.82083456)
\curveto(472.65953907,342.88083066)(472.66453907,342.93583061)(472.67453951,342.98583456)
\lineto(472.67453951,343.16583456)
\lineto(472.71953951,343.52583456)
\curveto(472.75953897,343.69582985)(472.79453894,343.86082968)(472.82453951,344.02083456)
\curveto(472.85453888,344.18082936)(472.89953883,344.33082921)(472.95953951,344.47083456)
\curveto(473.38953834,345.51082803)(474.11953761,346.2458273)(475.14953951,346.67583456)
\curveto(475.28953644,346.73582681)(475.4295363,346.77582677)(475.56953951,346.79583456)
\curveto(475.71953601,346.82582672)(475.87453586,346.86082668)(476.03453951,346.90083456)
\curveto(476.11453562,346.91082663)(476.18953554,346.91582663)(476.25953951,346.91583456)
\curveto(476.3295354,346.91582663)(476.40453533,346.92082662)(476.48453951,346.93083456)
\curveto(476.99453474,346.9408266)(477.4295343,346.88082666)(477.78953951,346.75083456)
\curveto(478.15953357,346.63082691)(478.48953324,346.47082707)(478.77953951,346.27083456)
\curveto(478.86953286,346.21082733)(478.95953277,346.1408274)(479.04953951,346.06083456)
\curveto(479.13953259,345.99082755)(479.21953251,345.91582763)(479.28953951,345.83583456)
\curveto(479.31953241,345.78582776)(479.35953237,345.7458278)(479.40953951,345.71583456)
\curveto(479.48953224,345.60582794)(479.56453217,345.49082805)(479.63453951,345.37083456)
\curveto(479.70453203,345.26082828)(479.77953195,345.1458284)(479.85953951,345.02583456)
\curveto(479.90953182,344.93582861)(479.94953178,344.8408287)(479.97953951,344.74083456)
\curveto(480.01953171,344.65082889)(480.05953167,344.55082899)(480.09953951,344.44083456)
\curveto(480.14953158,344.31082923)(480.18953154,344.17582937)(480.21953951,344.03583456)
\curveto(480.24953148,343.89582965)(480.28453145,343.75582979)(480.32453951,343.61583456)
\curveto(480.34453139,343.53583001)(480.34953138,343.4458301)(480.33953951,343.34583456)
\curveto(480.33953139,343.25583029)(480.34953138,343.17083037)(480.36953951,343.09083456)
\lineto(480.36953951,342.92583456)
\moveto(478.11953951,343.81083456)
\curveto(478.18953354,343.91082963)(478.19453354,344.03082951)(478.13453951,344.17083456)
\curveto(478.08453365,344.32082922)(478.04453369,344.43082911)(478.01453951,344.50083456)
\curveto(477.87453386,344.77082877)(477.68953404,344.97582857)(477.45953951,345.11583456)
\curveto(477.2295345,345.26582828)(476.90953482,345.3458282)(476.49953951,345.35583456)
\curveto(476.46953526,345.33582821)(476.4345353,345.33082821)(476.39453951,345.34083456)
\curveto(476.35453538,345.35082819)(476.31953541,345.35082819)(476.28953951,345.34083456)
\curveto(476.23953549,345.32082822)(476.18453555,345.30582824)(476.12453951,345.29583456)
\curveto(476.06453567,345.29582825)(476.00953572,345.28582826)(475.95953951,345.26583456)
\curveto(475.51953621,345.12582842)(475.19453654,344.85082869)(474.98453951,344.44083456)
\curveto(474.96453677,344.40082914)(474.93953679,344.3458292)(474.90953951,344.27583456)
\curveto(474.88953684,344.21582933)(474.87453686,344.15082939)(474.86453951,344.08083456)
\curveto(474.85453688,344.02082952)(474.85453688,343.96082958)(474.86453951,343.90083456)
\curveto(474.88453685,343.8408297)(474.91953681,343.79082975)(474.96953951,343.75083456)
\curveto(475.04953668,343.70082984)(475.15953657,343.67582987)(475.29953951,343.67583456)
\lineto(475.70453951,343.67583456)
\lineto(477.36953951,343.67583456)
\lineto(477.80453951,343.67583456)
\curveto(477.96453377,343.68582986)(478.06953366,343.73082981)(478.11953951,343.81083456)
}
}
{
\newrgbcolor{curcolor}{0 0 0}
\pscustom[linestyle=none,fillstyle=solid,fillcolor=curcolor]
{
\newpath
\moveto(484.58782076,346.93083456)
\curveto(485.33781626,346.95082659)(485.98781561,346.86582668)(486.53782076,346.67583456)
\curveto(487.0978145,346.49582705)(487.52281407,346.18082736)(487.81282076,345.73083456)
\curveto(487.88281371,345.62082792)(487.94281365,345.50582804)(487.99282076,345.38583456)
\curveto(488.05281354,345.27582827)(488.10281349,345.15082839)(488.14282076,345.01083456)
\curveto(488.16281343,344.95082859)(488.17281342,344.88582866)(488.17282076,344.81583456)
\curveto(488.17281342,344.7458288)(488.16281343,344.68582886)(488.14282076,344.63583456)
\curveto(488.10281349,344.57582897)(488.04781355,344.53582901)(487.97782076,344.51583456)
\curveto(487.92781367,344.49582905)(487.86781373,344.48582906)(487.79782076,344.48583456)
\lineto(487.58782076,344.48583456)
\lineto(486.92782076,344.48583456)
\curveto(486.85781474,344.48582906)(486.78781481,344.48082906)(486.71782076,344.47083456)
\curveto(486.64781495,344.47082907)(486.58281501,344.48082906)(486.52282076,344.50083456)
\curveto(486.42281517,344.52082902)(486.34781525,344.56082898)(486.29782076,344.62083456)
\curveto(486.24781535,344.68082886)(486.20281539,344.7408288)(486.16282076,344.80083456)
\lineto(486.04282076,345.01083456)
\curveto(486.01281558,345.09082845)(485.96281563,345.15582839)(485.89282076,345.20583456)
\curveto(485.7928158,345.28582826)(485.6928159,345.3458282)(485.59282076,345.38583456)
\curveto(485.50281609,345.42582812)(485.38781621,345.46082808)(485.24782076,345.49083456)
\curveto(485.17781642,345.51082803)(485.07281652,345.52582802)(484.93282076,345.53583456)
\curveto(484.80281679,345.545828)(484.70281689,345.540828)(484.63282076,345.52083456)
\lineto(484.52782076,345.52083456)
\lineto(484.37782076,345.49083456)
\curveto(484.33781726,345.49082805)(484.2928173,345.48582806)(484.24282076,345.47583456)
\curveto(484.07281752,345.42582812)(483.93281766,345.35582819)(483.82282076,345.26583456)
\curveto(483.72281787,345.18582836)(483.65281794,345.06082848)(483.61282076,344.89083456)
\curveto(483.592818,344.82082872)(483.592818,344.75582879)(483.61282076,344.69583456)
\curveto(483.63281796,344.63582891)(483.65281794,344.58582896)(483.67282076,344.54583456)
\curveto(483.74281785,344.42582912)(483.82281777,344.33082921)(483.91282076,344.26083456)
\curveto(484.01281758,344.19082935)(484.12781747,344.13082941)(484.25782076,344.08083456)
\curveto(484.44781715,344.00082954)(484.65281694,343.93082961)(484.87282076,343.87083456)
\lineto(485.56282076,343.72083456)
\curveto(485.80281579,343.68082986)(486.03281556,343.63082991)(486.25282076,343.57083456)
\curveto(486.48281511,343.52083002)(486.6978149,343.45583009)(486.89782076,343.37583456)
\curveto(486.98781461,343.33583021)(487.07281452,343.30083024)(487.15282076,343.27083456)
\curveto(487.24281435,343.25083029)(487.32781427,343.21583033)(487.40782076,343.16583456)
\curveto(487.597814,343.0458305)(487.76781383,342.91583063)(487.91782076,342.77583456)
\curveto(488.07781352,342.63583091)(488.20281339,342.46083108)(488.29282076,342.25083456)
\curveto(488.32281327,342.18083136)(488.34781325,342.11083143)(488.36782076,342.04083456)
\curveto(488.38781321,341.97083157)(488.40781319,341.89583165)(488.42782076,341.81583456)
\curveto(488.43781316,341.75583179)(488.44281315,341.66083188)(488.44282076,341.53083456)
\curveto(488.45281314,341.41083213)(488.45281314,341.31583223)(488.44282076,341.24583456)
\lineto(488.44282076,341.17083456)
\curveto(488.42281317,341.11083243)(488.40781319,341.05083249)(488.39782076,340.99083456)
\curveto(488.3978132,340.9408326)(488.3928132,340.89083265)(488.38282076,340.84083456)
\curveto(488.31281328,340.540833)(488.20281339,340.27583327)(488.05282076,340.04583456)
\curveto(487.8928137,339.80583374)(487.6978139,339.61083393)(487.46782076,339.46083456)
\curveto(487.23781436,339.31083423)(486.97781462,339.18083436)(486.68782076,339.07083456)
\curveto(486.57781502,339.02083452)(486.45781514,338.98583456)(486.32782076,338.96583456)
\curveto(486.20781539,338.9458346)(486.08781551,338.92083462)(485.96782076,338.89083456)
\curveto(485.87781572,338.87083467)(485.78281581,338.86083468)(485.68282076,338.86083456)
\curveto(485.592816,338.85083469)(485.50281609,338.83583471)(485.41282076,338.81583456)
\lineto(485.14282076,338.81583456)
\curveto(485.08281651,338.79583475)(484.97781662,338.78583476)(484.82782076,338.78583456)
\curveto(484.68781691,338.78583476)(484.58781701,338.79583475)(484.52782076,338.81583456)
\curveto(484.4978171,338.81583473)(484.46281713,338.82083472)(484.42282076,338.83083456)
\lineto(484.31782076,338.83083456)
\curveto(484.1978174,338.85083469)(484.07781752,338.86583468)(483.95782076,338.87583456)
\curveto(483.83781776,338.88583466)(483.72281787,338.90583464)(483.61282076,338.93583456)
\curveto(483.22281837,339.0458345)(482.87781872,339.17083437)(482.57782076,339.31083456)
\curveto(482.27781932,339.46083408)(482.02281957,339.68083386)(481.81282076,339.97083456)
\curveto(481.67281992,340.16083338)(481.55282004,340.38083316)(481.45282076,340.63083456)
\curveto(481.43282016,340.69083285)(481.41282018,340.77083277)(481.39282076,340.87083456)
\curveto(481.37282022,340.92083262)(481.35782024,340.99083255)(481.34782076,341.08083456)
\curveto(481.33782026,341.17083237)(481.34282025,341.2458323)(481.36282076,341.30583456)
\curveto(481.3928202,341.37583217)(481.44282015,341.42583212)(481.51282076,341.45583456)
\curveto(481.56282003,341.47583207)(481.62281997,341.48583206)(481.69282076,341.48583456)
\lineto(481.91782076,341.48583456)
\lineto(482.62282076,341.48583456)
\lineto(482.86282076,341.48583456)
\curveto(482.94281865,341.48583206)(483.01281858,341.47583207)(483.07282076,341.45583456)
\curveto(483.18281841,341.41583213)(483.25281834,341.35083219)(483.28282076,341.26083456)
\curveto(483.32281827,341.17083237)(483.36781823,341.07583247)(483.41782076,340.97583456)
\curveto(483.43781816,340.92583262)(483.47281812,340.86083268)(483.52282076,340.78083456)
\curveto(483.58281801,340.70083284)(483.63281796,340.65083289)(483.67282076,340.63083456)
\curveto(483.7928178,340.53083301)(483.90781769,340.45083309)(484.01782076,340.39083456)
\curveto(484.12781747,340.3408332)(484.26781733,340.29083325)(484.43782076,340.24083456)
\curveto(484.48781711,340.22083332)(484.53781706,340.21083333)(484.58782076,340.21083456)
\curveto(484.63781696,340.22083332)(484.68781691,340.22083332)(484.73782076,340.21083456)
\curveto(484.81781678,340.19083335)(484.90281669,340.18083336)(484.99282076,340.18083456)
\curveto(485.0928165,340.19083335)(485.17781642,340.20583334)(485.24782076,340.22583456)
\curveto(485.2978163,340.23583331)(485.34281625,340.2408333)(485.38282076,340.24083456)
\curveto(485.43281616,340.2408333)(485.48281611,340.25083329)(485.53282076,340.27083456)
\curveto(485.67281592,340.32083322)(485.7978158,340.38083316)(485.90782076,340.45083456)
\curveto(486.02781557,340.52083302)(486.12281547,340.61083293)(486.19282076,340.72083456)
\curveto(486.24281535,340.80083274)(486.28281531,340.92583262)(486.31282076,341.09583456)
\curveto(486.33281526,341.16583238)(486.33281526,341.23083231)(486.31282076,341.29083456)
\curveto(486.2928153,341.35083219)(486.27281532,341.40083214)(486.25282076,341.44083456)
\curveto(486.18281541,341.58083196)(486.0928155,341.68583186)(485.98282076,341.75583456)
\curveto(485.88281571,341.82583172)(485.76281583,341.89083165)(485.62282076,341.95083456)
\curveto(485.43281616,342.03083151)(485.23281636,342.09583145)(485.02282076,342.14583456)
\curveto(484.81281678,342.19583135)(484.60281699,342.25083129)(484.39282076,342.31083456)
\curveto(484.31281728,342.33083121)(484.22781737,342.3458312)(484.13782076,342.35583456)
\curveto(484.05781754,342.36583118)(483.97781762,342.38083116)(483.89782076,342.40083456)
\curveto(483.57781802,342.49083105)(483.27281832,342.57583097)(482.98282076,342.65583456)
\curveto(482.6928189,342.7458308)(482.42781917,342.87583067)(482.18782076,343.04583456)
\curveto(481.90781969,343.2458303)(481.70281989,343.51583003)(481.57282076,343.85583456)
\curveto(481.55282004,343.92582962)(481.53282006,344.02082952)(481.51282076,344.14083456)
\curveto(481.4928201,344.21082933)(481.47782012,344.29582925)(481.46782076,344.39583456)
\curveto(481.45782014,344.49582905)(481.46282013,344.58582896)(481.48282076,344.66583456)
\curveto(481.50282009,344.71582883)(481.50782009,344.75582879)(481.49782076,344.78583456)
\curveto(481.48782011,344.82582872)(481.4928201,344.87082867)(481.51282076,344.92083456)
\curveto(481.53282006,345.03082851)(481.55282004,345.13082841)(481.57282076,345.22083456)
\curveto(481.60281999,345.32082822)(481.63781996,345.41582813)(481.67782076,345.50583456)
\curveto(481.80781979,345.79582775)(481.98781961,346.03082751)(482.21782076,346.21083456)
\curveto(482.44781915,346.39082715)(482.70781889,346.53582701)(482.99782076,346.64583456)
\curveto(483.10781849,346.69582685)(483.22281837,346.73082681)(483.34282076,346.75083456)
\curveto(483.46281813,346.78082676)(483.58781801,346.81082673)(483.71782076,346.84083456)
\curveto(483.77781782,346.86082668)(483.83781776,346.87082667)(483.89782076,346.87083456)
\lineto(484.07782076,346.90083456)
\curveto(484.15781744,346.91082663)(484.24281735,346.91582663)(484.33282076,346.91583456)
\curveto(484.42281717,346.91582663)(484.50781709,346.92082662)(484.58782076,346.93083456)
}
}
{
\newrgbcolor{curcolor}{0 0 0}
\pscustom[linestyle=none,fillstyle=solid,fillcolor=curcolor]
{
}
}
{
\newrgbcolor{curcolor}{0 0 0}
\pscustom[linestyle=none,fillstyle=solid,fillcolor=curcolor]
{
\newpath
\moveto(501.42461763,339.83583456)
\lineto(501.42461763,339.41583456)
\curveto(501.42460926,339.28583426)(501.39460929,339.18083436)(501.33461763,339.10083456)
\curveto(501.2846094,339.05083449)(501.21960947,339.01583453)(501.13961763,338.99583456)
\curveto(501.05960963,338.98583456)(500.96960972,338.98083456)(500.86961763,338.98083456)
\lineto(500.04461763,338.98083456)
\lineto(499.75961763,338.98083456)
\curveto(499.67961101,338.99083455)(499.61461107,339.01583453)(499.56461763,339.05583456)
\curveto(499.49461119,339.10583444)(499.45461123,339.17083437)(499.44461763,339.25083456)
\curveto(499.43461125,339.33083421)(499.41461127,339.41083413)(499.38461763,339.49083456)
\curveto(499.36461132,339.51083403)(499.34461134,339.52583402)(499.32461763,339.53583456)
\curveto(499.31461137,339.55583399)(499.29961139,339.57583397)(499.27961763,339.59583456)
\curveto(499.16961152,339.59583395)(499.0896116,339.57083397)(499.03961763,339.52083456)
\lineto(498.88961763,339.37083456)
\curveto(498.81961187,339.32083422)(498.75461193,339.27583427)(498.69461763,339.23583456)
\curveto(498.63461205,339.20583434)(498.56961212,339.16583438)(498.49961763,339.11583456)
\curveto(498.45961223,339.09583445)(498.41461227,339.07583447)(498.36461763,339.05583456)
\curveto(498.32461236,339.03583451)(498.27961241,339.01583453)(498.22961763,338.99583456)
\curveto(498.0896126,338.9458346)(497.93961275,338.90083464)(497.77961763,338.86083456)
\curveto(497.72961296,338.8408347)(497.684613,338.83083471)(497.64461763,338.83083456)
\curveto(497.60461308,338.83083471)(497.56461312,338.82583472)(497.52461763,338.81583456)
\lineto(497.38961763,338.81583456)
\curveto(497.35961333,338.80583474)(497.31961337,338.80083474)(497.26961763,338.80083456)
\lineto(497.13461763,338.80083456)
\curveto(497.07461361,338.78083476)(496.9846137,338.77583477)(496.86461763,338.78583456)
\curveto(496.74461394,338.78583476)(496.65961403,338.79583475)(496.60961763,338.81583456)
\curveto(496.53961415,338.83583471)(496.47461421,338.8458347)(496.41461763,338.84583456)
\curveto(496.36461432,338.83583471)(496.30961438,338.8408347)(496.24961763,338.86083456)
\lineto(495.88961763,338.98083456)
\curveto(495.77961491,339.01083453)(495.66961502,339.05083449)(495.55961763,339.10083456)
\curveto(495.20961548,339.25083429)(494.89461579,339.48083406)(494.61461763,339.79083456)
\curveto(494.34461634,340.11083343)(494.12961656,340.4458331)(493.96961763,340.79583456)
\curveto(493.91961677,340.90583264)(493.87961681,341.01083253)(493.84961763,341.11083456)
\curveto(493.81961687,341.22083232)(493.7846169,341.33083221)(493.74461763,341.44083456)
\curveto(493.73461695,341.48083206)(493.72961696,341.51583203)(493.72961763,341.54583456)
\curveto(493.72961696,341.58583196)(493.71961697,341.63083191)(493.69961763,341.68083456)
\curveto(493.67961701,341.76083178)(493.65961703,341.8458317)(493.63961763,341.93583456)
\curveto(493.62961706,342.03583151)(493.61461707,342.13583141)(493.59461763,342.23583456)
\curveto(493.5846171,342.26583128)(493.57961711,342.30083124)(493.57961763,342.34083456)
\curveto(493.5896171,342.38083116)(493.5896171,342.41583113)(493.57961763,342.44583456)
\lineto(493.57961763,342.58083456)
\curveto(493.57961711,342.63083091)(493.57461711,342.68083086)(493.56461763,342.73083456)
\curveto(493.55461713,342.78083076)(493.54961714,342.83583071)(493.54961763,342.89583456)
\curveto(493.54961714,342.96583058)(493.55461713,343.02083052)(493.56461763,343.06083456)
\curveto(493.57461711,343.11083043)(493.57961711,343.15583039)(493.57961763,343.19583456)
\lineto(493.57961763,343.34583456)
\curveto(493.5896171,343.39583015)(493.5896171,343.4408301)(493.57961763,343.48083456)
\curveto(493.57961711,343.53083001)(493.5896171,343.58082996)(493.60961763,343.63083456)
\curveto(493.62961706,343.7408298)(493.64461704,343.8458297)(493.65461763,343.94583456)
\curveto(493.67461701,344.0458295)(493.69961699,344.1458294)(493.72961763,344.24583456)
\curveto(493.76961692,344.36582918)(493.80461688,344.48082906)(493.83461763,344.59083456)
\curveto(493.86461682,344.70082884)(493.90461678,344.81082873)(493.95461763,344.92083456)
\curveto(494.09461659,345.22082832)(494.26961642,345.50582804)(494.47961763,345.77583456)
\curveto(494.49961619,345.80582774)(494.52461616,345.83082771)(494.55461763,345.85083456)
\curveto(494.59461609,345.88082766)(494.62461606,345.91082763)(494.64461763,345.94083456)
\curveto(494.684616,345.99082755)(494.72461596,346.03582751)(494.76461763,346.07583456)
\curveto(494.80461588,346.11582743)(494.84961584,346.15582739)(494.89961763,346.19583456)
\curveto(494.93961575,346.21582733)(494.97461571,346.2408273)(495.00461763,346.27083456)
\curveto(495.03461565,346.31082723)(495.06961562,346.3408272)(495.10961763,346.36083456)
\curveto(495.35961533,346.53082701)(495.64961504,346.67082687)(495.97961763,346.78083456)
\curveto(496.04961464,346.80082674)(496.11961457,346.81582673)(496.18961763,346.82583456)
\curveto(496.26961442,346.83582671)(496.34961434,346.85082669)(496.42961763,346.87083456)
\curveto(496.49961419,346.89082665)(496.5896141,346.90082664)(496.69961763,346.90083456)
\curveto(496.80961388,346.91082663)(496.91961377,346.91582663)(497.02961763,346.91583456)
\curveto(497.13961355,346.91582663)(497.24461344,346.91082663)(497.34461763,346.90083456)
\curveto(497.45461323,346.89082665)(497.54461314,346.87582667)(497.61461763,346.85583456)
\curveto(497.76461292,346.80582674)(497.90961278,346.76082678)(498.04961763,346.72083456)
\curveto(498.1896125,346.68082686)(498.31961237,346.62582692)(498.43961763,346.55583456)
\curveto(498.50961218,346.50582704)(498.57461211,346.45582709)(498.63461763,346.40583456)
\curveto(498.69461199,346.36582718)(498.75961193,346.32082722)(498.82961763,346.27083456)
\curveto(498.86961182,346.2408273)(498.92461176,346.20082734)(498.99461763,346.15083456)
\curveto(499.07461161,346.10082744)(499.14961154,346.10082744)(499.21961763,346.15083456)
\curveto(499.25961143,346.17082737)(499.27961141,346.20582734)(499.27961763,346.25583456)
\curveto(499.27961141,346.30582724)(499.2896114,346.35582719)(499.30961763,346.40583456)
\lineto(499.30961763,346.55583456)
\curveto(499.31961137,346.58582696)(499.32461136,346.62082692)(499.32461763,346.66083456)
\lineto(499.32461763,346.78083456)
\lineto(499.32461763,348.82083456)
\curveto(499.32461136,348.93082461)(499.31961137,349.05082449)(499.30961763,349.18083456)
\curveto(499.30961138,349.32082422)(499.33461135,349.42582412)(499.38461763,349.49583456)
\curveto(499.42461126,349.57582397)(499.49961119,349.62582392)(499.60961763,349.64583456)
\curveto(499.62961106,349.65582389)(499.64961104,349.65582389)(499.66961763,349.64583456)
\curveto(499.689611,349.6458239)(499.70961098,349.65082389)(499.72961763,349.66083456)
\lineto(500.79461763,349.66083456)
\curveto(500.91460977,349.66082388)(501.02460966,349.65582389)(501.12461763,349.64583456)
\curveto(501.22460946,349.63582391)(501.29960939,349.59582395)(501.34961763,349.52583456)
\curveto(501.39960929,349.4458241)(501.42460926,349.3408242)(501.42461763,349.21083456)
\lineto(501.42461763,348.85083456)
\lineto(501.42461763,339.83583456)
\moveto(499.38461763,342.77583456)
\curveto(499.39461129,342.81583073)(499.39461129,342.85583069)(499.38461763,342.89583456)
\lineto(499.38461763,343.03083456)
\curveto(499.3846113,343.13083041)(499.37961131,343.23083031)(499.36961763,343.33083456)
\curveto(499.35961133,343.43083011)(499.34461134,343.52083002)(499.32461763,343.60083456)
\curveto(499.30461138,343.71082983)(499.2846114,343.81082973)(499.26461763,343.90083456)
\curveto(499.25461143,343.99082955)(499.22961146,344.07582947)(499.18961763,344.15583456)
\curveto(499.04961164,344.51582903)(498.84461184,344.80082874)(498.57461763,345.01083456)
\curveto(498.31461237,345.22082832)(497.93461275,345.32582822)(497.43461763,345.32583456)
\curveto(497.37461331,345.32582822)(497.29461339,345.31582823)(497.19461763,345.29583456)
\curveto(497.11461357,345.27582827)(497.03961365,345.25582829)(496.96961763,345.23583456)
\curveto(496.90961378,345.22582832)(496.84961384,345.20582834)(496.78961763,345.17583456)
\curveto(496.51961417,345.06582848)(496.30961438,344.89582865)(496.15961763,344.66583456)
\curveto(496.00961468,344.43582911)(495.8896148,344.17582937)(495.79961763,343.88583456)
\curveto(495.76961492,343.78582976)(495.74961494,343.68582986)(495.73961763,343.58583456)
\curveto(495.72961496,343.48583006)(495.70961498,343.38083016)(495.67961763,343.27083456)
\lineto(495.67961763,343.06083456)
\curveto(495.65961503,342.97083057)(495.65461503,342.8458307)(495.66461763,342.68583456)
\curveto(495.67461501,342.53583101)(495.689615,342.42583112)(495.70961763,342.35583456)
\lineto(495.70961763,342.26583456)
\curveto(495.71961497,342.2458313)(495.72461496,342.22583132)(495.72461763,342.20583456)
\curveto(495.74461494,342.12583142)(495.75961493,342.05083149)(495.76961763,341.98083456)
\curveto(495.7896149,341.91083163)(495.80961488,341.83583171)(495.82961763,341.75583456)
\curveto(495.99961469,341.23583231)(496.2896144,340.85083269)(496.69961763,340.60083456)
\curveto(496.82961386,340.51083303)(497.00961368,340.4408331)(497.23961763,340.39083456)
\curveto(497.27961341,340.38083316)(497.33961335,340.37583317)(497.41961763,340.37583456)
\curveto(497.44961324,340.36583318)(497.49461319,340.35583319)(497.55461763,340.34583456)
\curveto(497.62461306,340.3458332)(497.67961301,340.35083319)(497.71961763,340.36083456)
\curveto(497.79961289,340.38083316)(497.87961281,340.39583315)(497.95961763,340.40583456)
\curveto(498.03961265,340.41583313)(498.11961257,340.43583311)(498.19961763,340.46583456)
\curveto(498.44961224,340.57583297)(498.64961204,340.71583283)(498.79961763,340.88583456)
\curveto(498.94961174,341.05583249)(499.07961161,341.27083227)(499.18961763,341.53083456)
\curveto(499.22961146,341.62083192)(499.25961143,341.71083183)(499.27961763,341.80083456)
\curveto(499.29961139,341.90083164)(499.31961137,342.00583154)(499.33961763,342.11583456)
\curveto(499.34961134,342.16583138)(499.34961134,342.21083133)(499.33961763,342.25083456)
\curveto(499.33961135,342.30083124)(499.34961134,342.35083119)(499.36961763,342.40083456)
\curveto(499.37961131,342.43083111)(499.3846113,342.46583108)(499.38461763,342.50583456)
\lineto(499.38461763,342.64083456)
\lineto(499.38461763,342.77583456)
}
}
{
\newrgbcolor{curcolor}{0 0 0}
\pscustom[linestyle=none,fillstyle=solid,fillcolor=curcolor]
{
\newpath
\moveto(510.36953951,342.92583456)
\curveto(510.38953134,342.8458307)(510.38953134,342.75583079)(510.36953951,342.65583456)
\curveto(510.34953138,342.55583099)(510.31453142,342.49083105)(510.26453951,342.46083456)
\curveto(510.21453152,342.42083112)(510.13953159,342.39083115)(510.03953951,342.37083456)
\curveto(509.94953178,342.36083118)(509.84453189,342.35083119)(509.72453951,342.34083456)
\lineto(509.37953951,342.34083456)
\curveto(509.26953246,342.35083119)(509.16953256,342.35583119)(509.07953951,342.35583456)
\lineto(505.41953951,342.35583456)
\lineto(505.20953951,342.35583456)
\curveto(505.14953658,342.35583119)(505.09453664,342.3458312)(505.04453951,342.32583456)
\curveto(504.96453677,342.28583126)(504.91453682,342.2458313)(504.89453951,342.20583456)
\curveto(504.87453686,342.18583136)(504.85453688,342.1458314)(504.83453951,342.08583456)
\curveto(504.81453692,342.03583151)(504.80953692,341.98583156)(504.81953951,341.93583456)
\curveto(504.83953689,341.87583167)(504.84953688,341.81583173)(504.84953951,341.75583456)
\curveto(504.85953687,341.70583184)(504.87453686,341.65083189)(504.89453951,341.59083456)
\curveto(504.97453676,341.35083219)(505.06953666,341.15083239)(505.17953951,340.99083456)
\curveto(505.29953643,340.8408327)(505.45953627,340.70583284)(505.65953951,340.58583456)
\curveto(505.73953599,340.53583301)(505.81953591,340.50083304)(505.89953951,340.48083456)
\curveto(505.98953574,340.47083307)(506.07953565,340.45083309)(506.16953951,340.42083456)
\curveto(506.24953548,340.40083314)(506.35953537,340.38583316)(506.49953951,340.37583456)
\curveto(506.63953509,340.36583318)(506.75953497,340.37083317)(506.85953951,340.39083456)
\lineto(506.99453951,340.39083456)
\curveto(507.09453464,340.41083313)(507.18453455,340.43083311)(507.26453951,340.45083456)
\curveto(507.35453438,340.48083306)(507.43953429,340.51083303)(507.51953951,340.54083456)
\curveto(507.61953411,340.59083295)(507.729534,340.65583289)(507.84953951,340.73583456)
\curveto(507.97953375,340.81583273)(508.07453366,340.89583265)(508.13453951,340.97583456)
\curveto(508.18453355,341.0458325)(508.2345335,341.11083243)(508.28453951,341.17083456)
\curveto(508.34453339,341.2408323)(508.41453332,341.29083225)(508.49453951,341.32083456)
\curveto(508.59453314,341.37083217)(508.71953301,341.39083215)(508.86953951,341.38083456)
\lineto(509.30453951,341.38083456)
\lineto(509.48453951,341.38083456)
\curveto(509.55453218,341.39083215)(509.61453212,341.38583216)(509.66453951,341.36583456)
\lineto(509.81453951,341.36583456)
\curveto(509.91453182,341.3458322)(509.98453175,341.32083222)(510.02453951,341.29083456)
\curveto(510.06453167,341.27083227)(510.08453165,341.22583232)(510.08453951,341.15583456)
\curveto(510.09453164,341.08583246)(510.08953164,341.02583252)(510.06953951,340.97583456)
\curveto(510.01953171,340.83583271)(509.96453177,340.71083283)(509.90453951,340.60083456)
\curveto(509.84453189,340.49083305)(509.77453196,340.38083316)(509.69453951,340.27083456)
\curveto(509.47453226,339.9408336)(509.22453251,339.67583387)(508.94453951,339.47583456)
\curveto(508.66453307,339.27583427)(508.31453342,339.10583444)(507.89453951,338.96583456)
\curveto(507.78453395,338.92583462)(507.67453406,338.90083464)(507.56453951,338.89083456)
\curveto(507.45453428,338.88083466)(507.33953439,338.86083468)(507.21953951,338.83083456)
\curveto(507.17953455,338.82083472)(507.1345346,338.82083472)(507.08453951,338.83083456)
\curveto(507.04453469,338.83083471)(507.00453473,338.82583472)(506.96453951,338.81583456)
\lineto(506.79953951,338.81583456)
\curveto(506.74953498,338.79583475)(506.68953504,338.79083475)(506.61953951,338.80083456)
\curveto(506.55953517,338.80083474)(506.50453523,338.80583474)(506.45453951,338.81583456)
\curveto(506.37453536,338.82583472)(506.30453543,338.82583472)(506.24453951,338.81583456)
\curveto(506.18453555,338.80583474)(506.11953561,338.81083473)(506.04953951,338.83083456)
\curveto(505.99953573,338.85083469)(505.94453579,338.86083468)(505.88453951,338.86083456)
\curveto(505.82453591,338.86083468)(505.76953596,338.87083467)(505.71953951,338.89083456)
\curveto(505.60953612,338.91083463)(505.49953623,338.93583461)(505.38953951,338.96583456)
\curveto(505.27953645,338.98583456)(505.17953655,339.02083452)(505.08953951,339.07083456)
\curveto(504.97953675,339.11083443)(504.87453686,339.1458344)(504.77453951,339.17583456)
\curveto(504.68453705,339.21583433)(504.59953713,339.26083428)(504.51953951,339.31083456)
\curveto(504.19953753,339.51083403)(503.91453782,339.7408338)(503.66453951,340.00083456)
\curveto(503.41453832,340.27083327)(503.20953852,340.58083296)(503.04953951,340.93083456)
\curveto(502.99953873,341.0408325)(502.95953877,341.15083239)(502.92953951,341.26083456)
\curveto(502.89953883,341.38083216)(502.85953887,341.50083204)(502.80953951,341.62083456)
\curveto(502.79953893,341.66083188)(502.79453894,341.69583185)(502.79453951,341.72583456)
\curveto(502.79453894,341.76583178)(502.78953894,341.80583174)(502.77953951,341.84583456)
\curveto(502.73953899,341.96583158)(502.71453902,342.09583145)(502.70453951,342.23583456)
\lineto(502.67453951,342.65583456)
\curveto(502.67453906,342.70583084)(502.66953906,342.76083078)(502.65953951,342.82083456)
\curveto(502.65953907,342.88083066)(502.66453907,342.93583061)(502.67453951,342.98583456)
\lineto(502.67453951,343.16583456)
\lineto(502.71953951,343.52583456)
\curveto(502.75953897,343.69582985)(502.79453894,343.86082968)(502.82453951,344.02083456)
\curveto(502.85453888,344.18082936)(502.89953883,344.33082921)(502.95953951,344.47083456)
\curveto(503.38953834,345.51082803)(504.11953761,346.2458273)(505.14953951,346.67583456)
\curveto(505.28953644,346.73582681)(505.4295363,346.77582677)(505.56953951,346.79583456)
\curveto(505.71953601,346.82582672)(505.87453586,346.86082668)(506.03453951,346.90083456)
\curveto(506.11453562,346.91082663)(506.18953554,346.91582663)(506.25953951,346.91583456)
\curveto(506.3295354,346.91582663)(506.40453533,346.92082662)(506.48453951,346.93083456)
\curveto(506.99453474,346.9408266)(507.4295343,346.88082666)(507.78953951,346.75083456)
\curveto(508.15953357,346.63082691)(508.48953324,346.47082707)(508.77953951,346.27083456)
\curveto(508.86953286,346.21082733)(508.95953277,346.1408274)(509.04953951,346.06083456)
\curveto(509.13953259,345.99082755)(509.21953251,345.91582763)(509.28953951,345.83583456)
\curveto(509.31953241,345.78582776)(509.35953237,345.7458278)(509.40953951,345.71583456)
\curveto(509.48953224,345.60582794)(509.56453217,345.49082805)(509.63453951,345.37083456)
\curveto(509.70453203,345.26082828)(509.77953195,345.1458284)(509.85953951,345.02583456)
\curveto(509.90953182,344.93582861)(509.94953178,344.8408287)(509.97953951,344.74083456)
\curveto(510.01953171,344.65082889)(510.05953167,344.55082899)(510.09953951,344.44083456)
\curveto(510.14953158,344.31082923)(510.18953154,344.17582937)(510.21953951,344.03583456)
\curveto(510.24953148,343.89582965)(510.28453145,343.75582979)(510.32453951,343.61583456)
\curveto(510.34453139,343.53583001)(510.34953138,343.4458301)(510.33953951,343.34583456)
\curveto(510.33953139,343.25583029)(510.34953138,343.17083037)(510.36953951,343.09083456)
\lineto(510.36953951,342.92583456)
\moveto(508.11953951,343.81083456)
\curveto(508.18953354,343.91082963)(508.19453354,344.03082951)(508.13453951,344.17083456)
\curveto(508.08453365,344.32082922)(508.04453369,344.43082911)(508.01453951,344.50083456)
\curveto(507.87453386,344.77082877)(507.68953404,344.97582857)(507.45953951,345.11583456)
\curveto(507.2295345,345.26582828)(506.90953482,345.3458282)(506.49953951,345.35583456)
\curveto(506.46953526,345.33582821)(506.4345353,345.33082821)(506.39453951,345.34083456)
\curveto(506.35453538,345.35082819)(506.31953541,345.35082819)(506.28953951,345.34083456)
\curveto(506.23953549,345.32082822)(506.18453555,345.30582824)(506.12453951,345.29583456)
\curveto(506.06453567,345.29582825)(506.00953572,345.28582826)(505.95953951,345.26583456)
\curveto(505.51953621,345.12582842)(505.19453654,344.85082869)(504.98453951,344.44083456)
\curveto(504.96453677,344.40082914)(504.93953679,344.3458292)(504.90953951,344.27583456)
\curveto(504.88953684,344.21582933)(504.87453686,344.15082939)(504.86453951,344.08083456)
\curveto(504.85453688,344.02082952)(504.85453688,343.96082958)(504.86453951,343.90083456)
\curveto(504.88453685,343.8408297)(504.91953681,343.79082975)(504.96953951,343.75083456)
\curveto(505.04953668,343.70082984)(505.15953657,343.67582987)(505.29953951,343.67583456)
\lineto(505.70453951,343.67583456)
\lineto(507.36953951,343.67583456)
\lineto(507.80453951,343.67583456)
\curveto(507.96453377,343.68582986)(508.06953366,343.73082981)(508.11953951,343.81083456)
}
}
{
\newrgbcolor{curcolor}{0 0 0}
\pscustom[linestyle=none,fillstyle=solid,fillcolor=curcolor]
{
}
}
{
\newrgbcolor{curcolor}{0 0 0}
\pscustom[linestyle=none,fillstyle=solid,fillcolor=curcolor]
{
\newpath
\moveto(523.66797701,342.94083456)
\curveto(523.67796833,342.88083066)(523.68296832,342.79083075)(523.68297701,342.67083456)
\curveto(523.68296832,342.55083099)(523.67296833,342.46583108)(523.65297701,342.41583456)
\lineto(523.65297701,342.22083456)
\curveto(523.62296838,342.11083143)(523.6029684,342.00583154)(523.59297701,341.90583456)
\curveto(523.59296841,341.80583174)(523.57796843,341.70583184)(523.54797701,341.60583456)
\curveto(523.52796848,341.51583203)(523.5079685,341.42083212)(523.48797701,341.32083456)
\curveto(523.46796854,341.23083231)(523.43796857,341.1408324)(523.39797701,341.05083456)
\curveto(523.32796868,340.88083266)(523.25796875,340.72083282)(523.18797701,340.57083456)
\curveto(523.11796889,340.43083311)(523.03796897,340.29083325)(522.94797701,340.15083456)
\curveto(522.88796912,340.06083348)(522.82296918,339.97583357)(522.75297701,339.89583456)
\curveto(522.69296931,339.82583372)(522.62296938,339.75083379)(522.54297701,339.67083456)
\lineto(522.43797701,339.56583456)
\curveto(522.38796962,339.51583403)(522.33296967,339.47083407)(522.27297701,339.43083456)
\lineto(522.12297701,339.31083456)
\curveto(522.04296996,339.25083429)(521.95297005,339.19583435)(521.85297701,339.14583456)
\curveto(521.76297024,339.10583444)(521.66797034,339.06083448)(521.56797701,339.01083456)
\curveto(521.46797054,338.96083458)(521.36297064,338.92583462)(521.25297701,338.90583456)
\curveto(521.15297085,338.88583466)(521.04797096,338.86583468)(520.93797701,338.84583456)
\curveto(520.87797113,338.82583472)(520.81297119,338.81583473)(520.74297701,338.81583456)
\curveto(520.68297132,338.81583473)(520.61797139,338.80583474)(520.54797701,338.78583456)
\lineto(520.41297701,338.78583456)
\curveto(520.33297167,338.76583478)(520.25797175,338.76583478)(520.18797701,338.78583456)
\lineto(520.03797701,338.78583456)
\curveto(519.97797203,338.80583474)(519.91297209,338.81583473)(519.84297701,338.81583456)
\curveto(519.78297222,338.80583474)(519.72297228,338.81083473)(519.66297701,338.83083456)
\curveto(519.5029725,338.88083466)(519.34797266,338.92583462)(519.19797701,338.96583456)
\curveto(519.05797295,339.00583454)(518.92797308,339.06583448)(518.80797701,339.14583456)
\curveto(518.73797327,339.18583436)(518.67297333,339.22583432)(518.61297701,339.26583456)
\curveto(518.55297345,339.31583423)(518.48797352,339.36583418)(518.41797701,339.41583456)
\lineto(518.23797701,339.55083456)
\curveto(518.15797385,339.61083393)(518.08797392,339.61583393)(518.02797701,339.56583456)
\curveto(517.97797403,339.53583401)(517.95297405,339.49583405)(517.95297701,339.44583456)
\curveto(517.95297405,339.40583414)(517.94297406,339.35583419)(517.92297701,339.29583456)
\curveto(517.9029741,339.19583435)(517.89297411,339.08083446)(517.89297701,338.95083456)
\curveto(517.9029741,338.82083472)(517.9079741,338.70083484)(517.90797701,338.59083456)
\lineto(517.90797701,337.06083456)
\curveto(517.9079741,336.93083661)(517.9029741,336.80583674)(517.89297701,336.68583456)
\curveto(517.89297411,336.55583699)(517.86797414,336.45083709)(517.81797701,336.37083456)
\curveto(517.78797422,336.33083721)(517.73297427,336.30083724)(517.65297701,336.28083456)
\curveto(517.57297443,336.26083728)(517.48297452,336.25083729)(517.38297701,336.25083456)
\curveto(517.28297472,336.2408373)(517.18297482,336.2408373)(517.08297701,336.25083456)
\lineto(516.82797701,336.25083456)
\lineto(516.42297701,336.25083456)
\lineto(516.31797701,336.25083456)
\curveto(516.27797573,336.25083729)(516.24297576,336.25583729)(516.21297701,336.26583456)
\lineto(516.09297701,336.26583456)
\curveto(515.92297608,336.31583723)(515.83297617,336.41583713)(515.82297701,336.56583456)
\curveto(515.81297619,336.70583684)(515.8079762,336.87583667)(515.80797701,337.07583456)
\lineto(515.80797701,345.88083456)
\curveto(515.8079762,345.99082755)(515.8029762,346.10582744)(515.79297701,346.22583456)
\curveto(515.79297621,346.35582719)(515.81797619,346.45582709)(515.86797701,346.52583456)
\curveto(515.9079761,346.59582695)(515.96297604,346.6408269)(516.03297701,346.66083456)
\curveto(516.08297592,346.68082686)(516.14297586,346.69082685)(516.21297701,346.69083456)
\lineto(516.43797701,346.69083456)
\lineto(517.15797701,346.69083456)
\lineto(517.44297701,346.69083456)
\curveto(517.53297447,346.69082685)(517.6079744,346.66582688)(517.66797701,346.61583456)
\curveto(517.73797427,346.56582698)(517.77297423,346.50082704)(517.77297701,346.42083456)
\curveto(517.78297422,346.35082719)(517.8079742,346.27582727)(517.84797701,346.19583456)
\curveto(517.85797415,346.16582738)(517.86797414,346.1408274)(517.87797701,346.12083456)
\curveto(517.89797411,346.11082743)(517.91797409,346.09582745)(517.93797701,346.07583456)
\curveto(518.04797396,346.06582748)(518.13797387,346.09582745)(518.20797701,346.16583456)
\curveto(518.27797373,346.23582731)(518.34797366,346.29582725)(518.41797701,346.34583456)
\curveto(518.54797346,346.43582711)(518.68297332,346.51582703)(518.82297701,346.58583456)
\curveto(518.96297304,346.66582688)(519.11797289,346.73082681)(519.28797701,346.78083456)
\curveto(519.36797264,346.81082673)(519.45297255,346.83082671)(519.54297701,346.84083456)
\curveto(519.64297236,346.85082669)(519.73797227,346.86582668)(519.82797701,346.88583456)
\curveto(519.86797214,346.89582665)(519.9079721,346.89582665)(519.94797701,346.88583456)
\curveto(519.99797201,346.87582667)(520.03797197,346.88082666)(520.06797701,346.90083456)
\curveto(520.63797137,346.92082662)(521.11797089,346.8408267)(521.50797701,346.66083456)
\curveto(521.9079701,346.49082705)(522.24796976,346.26582728)(522.52797701,345.98583456)
\curveto(522.57796943,345.93582761)(522.62296938,345.88582766)(522.66297701,345.83583456)
\curveto(522.7029693,345.79582775)(522.74296926,345.75082779)(522.78297701,345.70083456)
\curveto(522.85296915,345.61082793)(522.91296909,345.52082802)(522.96297701,345.43083456)
\curveto(523.02296898,345.3408282)(523.07796893,345.25082829)(523.12797701,345.16083456)
\curveto(523.14796886,345.1408284)(523.15796885,345.11582843)(523.15797701,345.08583456)
\curveto(523.16796884,345.05582849)(523.18296882,345.02082852)(523.20297701,344.98083456)
\curveto(523.26296874,344.88082866)(523.31796869,344.76082878)(523.36797701,344.62083456)
\curveto(523.38796862,344.56082898)(523.4079686,344.49582905)(523.42797701,344.42583456)
\curveto(523.44796856,344.36582918)(523.46796854,344.30082924)(523.48797701,344.23083456)
\curveto(523.52796848,344.11082943)(523.55296845,343.98582956)(523.56297701,343.85583456)
\curveto(523.58296842,343.72582982)(523.6079684,343.59082995)(523.63797701,343.45083456)
\lineto(523.63797701,343.28583456)
\lineto(523.66797701,343.10583456)
\lineto(523.66797701,342.94083456)
\moveto(521.55297701,342.59583456)
\curveto(521.56297044,342.6458309)(521.56797044,342.71083083)(521.56797701,342.79083456)
\curveto(521.56797044,342.88083066)(521.56297044,342.95083059)(521.55297701,343.00083456)
\lineto(521.55297701,343.13583456)
\curveto(521.53297047,343.19583035)(521.52297048,343.26083028)(521.52297701,343.33083456)
\curveto(521.52297048,343.40083014)(521.51297049,343.47083007)(521.49297701,343.54083456)
\curveto(521.47297053,343.6408299)(521.45297055,343.73582981)(521.43297701,343.82583456)
\curveto(521.41297059,343.92582962)(521.38297062,344.01582953)(521.34297701,344.09583456)
\curveto(521.22297078,344.41582913)(521.06797094,344.67082887)(520.87797701,344.86083456)
\curveto(520.68797132,345.05082849)(520.41797159,345.19082835)(520.06797701,345.28083456)
\curveto(519.98797202,345.30082824)(519.89797211,345.31082823)(519.79797701,345.31083456)
\lineto(519.52797701,345.31083456)
\curveto(519.48797252,345.30082824)(519.45297255,345.29582825)(519.42297701,345.29583456)
\curveto(519.39297261,345.29582825)(519.35797265,345.29082825)(519.31797701,345.28083456)
\lineto(519.10797701,345.22083456)
\curveto(519.04797296,345.21082833)(518.98797302,345.19082835)(518.92797701,345.16083456)
\curveto(518.66797334,345.05082849)(518.46297354,344.88082866)(518.31297701,344.65083456)
\curveto(518.17297383,344.42082912)(518.05797395,344.16582938)(517.96797701,343.88583456)
\curveto(517.94797406,343.80582974)(517.93297407,343.72082982)(517.92297701,343.63083456)
\curveto(517.91297409,343.55082999)(517.89797411,343.47083007)(517.87797701,343.39083456)
\curveto(517.86797414,343.35083019)(517.86297414,343.28583026)(517.86297701,343.19583456)
\curveto(517.84297416,343.15583039)(517.83797417,343.10583044)(517.84797701,343.04583456)
\curveto(517.85797415,342.99583055)(517.85797415,342.9458306)(517.84797701,342.89583456)
\curveto(517.82797418,342.83583071)(517.82797418,342.78083076)(517.84797701,342.73083456)
\lineto(517.84797701,342.55083456)
\lineto(517.84797701,342.41583456)
\curveto(517.84797416,342.37583117)(517.85797415,342.33583121)(517.87797701,342.29583456)
\curveto(517.87797413,342.22583132)(517.88297412,342.17083137)(517.89297701,342.13083456)
\lineto(517.92297701,341.95083456)
\curveto(517.93297407,341.89083165)(517.94797406,341.83083171)(517.96797701,341.77083456)
\curveto(518.05797395,341.48083206)(518.16297384,341.2408323)(518.28297701,341.05083456)
\curveto(518.41297359,340.87083267)(518.59297341,340.71083283)(518.82297701,340.57083456)
\curveto(518.96297304,340.49083305)(519.12797288,340.42583312)(519.31797701,340.37583456)
\curveto(519.35797265,340.36583318)(519.39297261,340.36083318)(519.42297701,340.36083456)
\curveto(519.45297255,340.37083317)(519.48797252,340.37083317)(519.52797701,340.36083456)
\curveto(519.56797244,340.35083319)(519.62797238,340.3408332)(519.70797701,340.33083456)
\curveto(519.78797222,340.33083321)(519.85297215,340.33583321)(519.90297701,340.34583456)
\curveto(519.98297202,340.36583318)(520.06297194,340.38083316)(520.14297701,340.39083456)
\curveto(520.23297177,340.41083313)(520.31797169,340.43583311)(520.39797701,340.46583456)
\curveto(520.63797137,340.56583298)(520.83297117,340.70583284)(520.98297701,340.88583456)
\curveto(521.13297087,341.06583248)(521.25797075,341.27583227)(521.35797701,341.51583456)
\curveto(521.4079706,341.63583191)(521.44297056,341.76083178)(521.46297701,341.89083456)
\curveto(521.48297052,342.02083152)(521.5079705,342.15583139)(521.53797701,342.29583456)
\lineto(521.53797701,342.44583456)
\curveto(521.54797046,342.49583105)(521.55297045,342.545831)(521.55297701,342.59583456)
}
}
{
\newrgbcolor{curcolor}{0 0 0}
\pscustom[linestyle=none,fillstyle=solid,fillcolor=curcolor]
{
\newpath
\moveto(532.71789888,343.16583456)
\curveto(532.73789031,343.10583044)(532.7478903,343.02083052)(532.74789888,342.91083456)
\curveto(532.7478903,342.80083074)(532.73789031,342.71583083)(532.71789888,342.65583456)
\lineto(532.71789888,342.50583456)
\curveto(532.69789035,342.42583112)(532.68789036,342.3458312)(532.68789888,342.26583456)
\curveto(532.69789035,342.18583136)(532.69289036,342.10583144)(532.67289888,342.02583456)
\curveto(532.6528904,341.95583159)(532.63789041,341.89083165)(532.62789888,341.83083456)
\curveto(532.61789043,341.77083177)(532.60789044,341.70583184)(532.59789888,341.63583456)
\curveto(532.55789049,341.52583202)(532.52289053,341.41083213)(532.49289888,341.29083456)
\curveto(532.46289059,341.18083236)(532.42289063,341.07583247)(532.37289888,340.97583456)
\curveto(532.16289089,340.49583305)(531.88789116,340.10583344)(531.54789888,339.80583456)
\curveto(531.20789184,339.50583404)(530.79789225,339.25583429)(530.31789888,339.05583456)
\curveto(530.19789285,339.00583454)(530.07289298,338.97083457)(529.94289888,338.95083456)
\curveto(529.82289323,338.92083462)(529.69789335,338.89083465)(529.56789888,338.86083456)
\curveto(529.51789353,338.8408347)(529.46289359,338.83083471)(529.40289888,338.83083456)
\curveto(529.34289371,338.83083471)(529.28789376,338.82583472)(529.23789888,338.81583456)
\lineto(529.13289888,338.81583456)
\curveto(529.10289395,338.80583474)(529.07289398,338.80083474)(529.04289888,338.80083456)
\curveto(528.99289406,338.79083475)(528.91289414,338.78583476)(528.80289888,338.78583456)
\curveto(528.69289436,338.77583477)(528.60789444,338.78083476)(528.54789888,338.80083456)
\lineto(528.39789888,338.80083456)
\curveto(528.3478947,338.81083473)(528.29289476,338.81583473)(528.23289888,338.81583456)
\curveto(528.18289487,338.80583474)(528.13289492,338.81083473)(528.08289888,338.83083456)
\curveto(528.04289501,338.8408347)(528.00289505,338.8458347)(527.96289888,338.84583456)
\curveto(527.93289512,338.8458347)(527.89289516,338.85083469)(527.84289888,338.86083456)
\curveto(527.74289531,338.89083465)(527.64289541,338.91583463)(527.54289888,338.93583456)
\curveto(527.44289561,338.95583459)(527.3478957,338.98583456)(527.25789888,339.02583456)
\curveto(527.13789591,339.06583448)(527.02289603,339.10583444)(526.91289888,339.14583456)
\curveto(526.81289624,339.18583436)(526.70789634,339.23583431)(526.59789888,339.29583456)
\curveto(526.2478968,339.50583404)(525.9478971,339.75083379)(525.69789888,340.03083456)
\curveto(525.4478976,340.31083323)(525.23789781,340.6458329)(525.06789888,341.03583456)
\curveto(525.01789803,341.12583242)(524.97789807,341.22083232)(524.94789888,341.32083456)
\curveto(524.92789812,341.42083212)(524.90289815,341.52583202)(524.87289888,341.63583456)
\curveto(524.8528982,341.68583186)(524.84289821,341.73083181)(524.84289888,341.77083456)
\curveto(524.84289821,341.81083173)(524.83289822,341.85583169)(524.81289888,341.90583456)
\curveto(524.79289826,341.98583156)(524.78289827,342.06583148)(524.78289888,342.14583456)
\curveto(524.78289827,342.23583131)(524.77289828,342.32083122)(524.75289888,342.40083456)
\curveto(524.74289831,342.45083109)(524.73789831,342.49583105)(524.73789888,342.53583456)
\lineto(524.73789888,342.67083456)
\curveto(524.71789833,342.73083081)(524.70789834,342.81583073)(524.70789888,342.92583456)
\curveto(524.71789833,343.03583051)(524.73289832,343.12083042)(524.75289888,343.18083456)
\lineto(524.75289888,343.28583456)
\curveto(524.76289829,343.33583021)(524.76289829,343.38583016)(524.75289888,343.43583456)
\curveto(524.7528983,343.49583005)(524.76289829,343.55082999)(524.78289888,343.60083456)
\curveto(524.79289826,343.65082989)(524.79789825,343.69582985)(524.79789888,343.73583456)
\curveto(524.79789825,343.78582976)(524.80789824,343.83582971)(524.82789888,343.88583456)
\curveto(524.86789818,344.01582953)(524.90289815,344.1408294)(524.93289888,344.26083456)
\curveto(524.96289809,344.39082915)(525.00289805,344.51582903)(525.05289888,344.63583456)
\curveto(525.23289782,345.0458285)(525.4478976,345.38582816)(525.69789888,345.65583456)
\curveto(525.9478971,345.93582761)(526.2528968,346.19082735)(526.61289888,346.42083456)
\curveto(526.71289634,346.47082707)(526.81789623,346.51582703)(526.92789888,346.55583456)
\curveto(527.03789601,346.59582695)(527.1478959,346.6408269)(527.25789888,346.69083456)
\curveto(527.38789566,346.7408268)(527.52289553,346.77582677)(527.66289888,346.79583456)
\curveto(527.80289525,346.81582673)(527.9478951,346.8458267)(528.09789888,346.88583456)
\curveto(528.17789487,346.89582665)(528.2528948,346.90082664)(528.32289888,346.90083456)
\curveto(528.39289466,346.90082664)(528.46289459,346.90582664)(528.53289888,346.91583456)
\curveto(529.11289394,346.92582662)(529.61289344,346.86582668)(530.03289888,346.73583456)
\curveto(530.46289259,346.60582694)(530.84289221,346.42582712)(531.17289888,346.19583456)
\curveto(531.28289177,346.11582743)(531.39289166,346.02582752)(531.50289888,345.92583456)
\curveto(531.62289143,345.83582771)(531.72289133,345.73582781)(531.80289888,345.62583456)
\curveto(531.88289117,345.52582802)(531.9528911,345.42582812)(532.01289888,345.32583456)
\curveto(532.08289097,345.22582832)(532.1528909,345.12082842)(532.22289888,345.01083456)
\curveto(532.29289076,344.90082864)(532.3478907,344.78082876)(532.38789888,344.65083456)
\curveto(532.42789062,344.53082901)(532.47289058,344.40082914)(532.52289888,344.26083456)
\curveto(532.5528905,344.18082936)(532.57789047,344.09582945)(532.59789888,344.00583456)
\lineto(532.65789888,343.73583456)
\curveto(532.66789038,343.69582985)(532.67289038,343.65582989)(532.67289888,343.61583456)
\curveto(532.67289038,343.57582997)(532.67789037,343.53583001)(532.68789888,343.49583456)
\curveto(532.70789034,343.4458301)(532.71289034,343.39083015)(532.70289888,343.33083456)
\curveto(532.69289036,343.27083027)(532.69789035,343.21583033)(532.71789888,343.16583456)
\moveto(530.61789888,342.62583456)
\curveto(530.62789242,342.67583087)(530.63289242,342.7458308)(530.63289888,342.83583456)
\curveto(530.63289242,342.93583061)(530.62789242,343.01083053)(530.61789888,343.06083456)
\lineto(530.61789888,343.18083456)
\curveto(530.59789245,343.23083031)(530.58789246,343.28583026)(530.58789888,343.34583456)
\curveto(530.58789246,343.40583014)(530.58289247,343.46083008)(530.57289888,343.51083456)
\curveto(530.57289248,343.55082999)(530.56789248,343.58082996)(530.55789888,343.60083456)
\lineto(530.49789888,343.84083456)
\curveto(530.48789256,343.93082961)(530.46789258,344.01582953)(530.43789888,344.09583456)
\curveto(530.32789272,344.35582919)(530.19789285,344.57582897)(530.04789888,344.75583456)
\curveto(529.89789315,344.9458286)(529.69789335,345.09582845)(529.44789888,345.20583456)
\curveto(529.38789366,345.22582832)(529.32789372,345.2408283)(529.26789888,345.25083456)
\curveto(529.20789384,345.27082827)(529.14289391,345.29082825)(529.07289888,345.31083456)
\curveto(528.99289406,345.33082821)(528.90789414,345.33582821)(528.81789888,345.32583456)
\lineto(528.54789888,345.32583456)
\curveto(528.51789453,345.30582824)(528.48289457,345.29582825)(528.44289888,345.29583456)
\curveto(528.40289465,345.30582824)(528.36789468,345.30582824)(528.33789888,345.29583456)
\lineto(528.12789888,345.23583456)
\curveto(528.06789498,345.22582832)(528.01289504,345.20582834)(527.96289888,345.17583456)
\curveto(527.71289534,345.06582848)(527.50789554,344.90582864)(527.34789888,344.69583456)
\curveto(527.19789585,344.49582905)(527.07789597,344.26082928)(526.98789888,343.99083456)
\curveto(526.95789609,343.89082965)(526.93289612,343.78582976)(526.91289888,343.67583456)
\curveto(526.90289615,343.56582998)(526.88789616,343.45583009)(526.86789888,343.34583456)
\curveto(526.85789619,343.29583025)(526.8528962,343.2458303)(526.85289888,343.19583456)
\lineto(526.85289888,343.04583456)
\curveto(526.83289622,342.97583057)(526.82289623,342.87083067)(526.82289888,342.73083456)
\curveto(526.83289622,342.59083095)(526.8478962,342.48583106)(526.86789888,342.41583456)
\lineto(526.86789888,342.28083456)
\curveto(526.88789616,342.20083134)(526.90289615,342.12083142)(526.91289888,342.04083456)
\curveto(526.92289613,341.97083157)(526.93789611,341.89583165)(526.95789888,341.81583456)
\curveto(527.05789599,341.51583203)(527.16289589,341.27083227)(527.27289888,341.08083456)
\curveto(527.39289566,340.90083264)(527.57789547,340.73583281)(527.82789888,340.58583456)
\curveto(527.89789515,340.53583301)(527.97289508,340.49583305)(528.05289888,340.46583456)
\curveto(528.14289491,340.43583311)(528.23289482,340.41083313)(528.32289888,340.39083456)
\curveto(528.36289469,340.38083316)(528.39789465,340.37583317)(528.42789888,340.37583456)
\curveto(528.45789459,340.38583316)(528.49289456,340.38583316)(528.53289888,340.37583456)
\lineto(528.65289888,340.34583456)
\curveto(528.70289435,340.3458332)(528.7478943,340.35083319)(528.78789888,340.36083456)
\lineto(528.90789888,340.36083456)
\curveto(528.98789406,340.38083316)(529.06789398,340.39583315)(529.14789888,340.40583456)
\curveto(529.22789382,340.41583313)(529.30289375,340.43583311)(529.37289888,340.46583456)
\curveto(529.63289342,340.56583298)(529.84289321,340.70083284)(530.00289888,340.87083456)
\curveto(530.16289289,341.0408325)(530.29789275,341.25083229)(530.40789888,341.50083456)
\curveto(530.4478926,341.60083194)(530.47789257,341.70083184)(530.49789888,341.80083456)
\curveto(530.51789253,341.90083164)(530.54289251,342.00583154)(530.57289888,342.11583456)
\curveto(530.58289247,342.15583139)(530.58789246,342.19083135)(530.58789888,342.22083456)
\curveto(530.58789246,342.26083128)(530.59289246,342.30083124)(530.60289888,342.34083456)
\lineto(530.60289888,342.47583456)
\curveto(530.60289245,342.52583102)(530.60789244,342.57583097)(530.61789888,342.62583456)
}
}
{
\newrgbcolor{curcolor}{0 0 0}
\pscustom[linestyle=none,fillstyle=solid,fillcolor=curcolor]
{
\newpath
\moveto(542.00782076,342.94083456)
\curveto(542.01781208,342.88083066)(542.02281207,342.79083075)(542.02282076,342.67083456)
\curveto(542.02281207,342.55083099)(542.01281208,342.46583108)(541.99282076,342.41583456)
\lineto(541.99282076,342.22083456)
\curveto(541.96281213,342.11083143)(541.94281215,342.00583154)(541.93282076,341.90583456)
\curveto(541.93281216,341.80583174)(541.91781218,341.70583184)(541.88782076,341.60583456)
\curveto(541.86781223,341.51583203)(541.84781225,341.42083212)(541.82782076,341.32083456)
\curveto(541.80781229,341.23083231)(541.77781232,341.1408324)(541.73782076,341.05083456)
\curveto(541.66781243,340.88083266)(541.5978125,340.72083282)(541.52782076,340.57083456)
\curveto(541.45781264,340.43083311)(541.37781272,340.29083325)(541.28782076,340.15083456)
\curveto(541.22781287,340.06083348)(541.16281293,339.97583357)(541.09282076,339.89583456)
\curveto(541.03281306,339.82583372)(540.96281313,339.75083379)(540.88282076,339.67083456)
\lineto(540.77782076,339.56583456)
\curveto(540.72781337,339.51583403)(540.67281342,339.47083407)(540.61282076,339.43083456)
\lineto(540.46282076,339.31083456)
\curveto(540.38281371,339.25083429)(540.2928138,339.19583435)(540.19282076,339.14583456)
\curveto(540.10281399,339.10583444)(540.00781409,339.06083448)(539.90782076,339.01083456)
\curveto(539.80781429,338.96083458)(539.70281439,338.92583462)(539.59282076,338.90583456)
\curveto(539.4928146,338.88583466)(539.38781471,338.86583468)(539.27782076,338.84583456)
\curveto(539.21781488,338.82583472)(539.15281494,338.81583473)(539.08282076,338.81583456)
\curveto(539.02281507,338.81583473)(538.95781514,338.80583474)(538.88782076,338.78583456)
\lineto(538.75282076,338.78583456)
\curveto(538.67281542,338.76583478)(538.5978155,338.76583478)(538.52782076,338.78583456)
\lineto(538.37782076,338.78583456)
\curveto(538.31781578,338.80583474)(538.25281584,338.81583473)(538.18282076,338.81583456)
\curveto(538.12281597,338.80583474)(538.06281603,338.81083473)(538.00282076,338.83083456)
\curveto(537.84281625,338.88083466)(537.68781641,338.92583462)(537.53782076,338.96583456)
\curveto(537.3978167,339.00583454)(537.26781683,339.06583448)(537.14782076,339.14583456)
\curveto(537.07781702,339.18583436)(537.01281708,339.22583432)(536.95282076,339.26583456)
\curveto(536.8928172,339.31583423)(536.82781727,339.36583418)(536.75782076,339.41583456)
\lineto(536.57782076,339.55083456)
\curveto(536.4978176,339.61083393)(536.42781767,339.61583393)(536.36782076,339.56583456)
\curveto(536.31781778,339.53583401)(536.2928178,339.49583405)(536.29282076,339.44583456)
\curveto(536.2928178,339.40583414)(536.28281781,339.35583419)(536.26282076,339.29583456)
\curveto(536.24281785,339.19583435)(536.23281786,339.08083446)(536.23282076,338.95083456)
\curveto(536.24281785,338.82083472)(536.24781785,338.70083484)(536.24782076,338.59083456)
\lineto(536.24782076,337.06083456)
\curveto(536.24781785,336.93083661)(536.24281785,336.80583674)(536.23282076,336.68583456)
\curveto(536.23281786,336.55583699)(536.20781789,336.45083709)(536.15782076,336.37083456)
\curveto(536.12781797,336.33083721)(536.07281802,336.30083724)(535.99282076,336.28083456)
\curveto(535.91281818,336.26083728)(535.82281827,336.25083729)(535.72282076,336.25083456)
\curveto(535.62281847,336.2408373)(535.52281857,336.2408373)(535.42282076,336.25083456)
\lineto(535.16782076,336.25083456)
\lineto(534.76282076,336.25083456)
\lineto(534.65782076,336.25083456)
\curveto(534.61781948,336.25083729)(534.58281951,336.25583729)(534.55282076,336.26583456)
\lineto(534.43282076,336.26583456)
\curveto(534.26281983,336.31583723)(534.17281992,336.41583713)(534.16282076,336.56583456)
\curveto(534.15281994,336.70583684)(534.14781995,336.87583667)(534.14782076,337.07583456)
\lineto(534.14782076,345.88083456)
\curveto(534.14781995,345.99082755)(534.14281995,346.10582744)(534.13282076,346.22583456)
\curveto(534.13281996,346.35582719)(534.15781994,346.45582709)(534.20782076,346.52583456)
\curveto(534.24781985,346.59582695)(534.30281979,346.6408269)(534.37282076,346.66083456)
\curveto(534.42281967,346.68082686)(534.48281961,346.69082685)(534.55282076,346.69083456)
\lineto(534.77782076,346.69083456)
\lineto(535.49782076,346.69083456)
\lineto(535.78282076,346.69083456)
\curveto(535.87281822,346.69082685)(535.94781815,346.66582688)(536.00782076,346.61583456)
\curveto(536.07781802,346.56582698)(536.11281798,346.50082704)(536.11282076,346.42083456)
\curveto(536.12281797,346.35082719)(536.14781795,346.27582727)(536.18782076,346.19583456)
\curveto(536.1978179,346.16582738)(536.20781789,346.1408274)(536.21782076,346.12083456)
\curveto(536.23781786,346.11082743)(536.25781784,346.09582745)(536.27782076,346.07583456)
\curveto(536.38781771,346.06582748)(536.47781762,346.09582745)(536.54782076,346.16583456)
\curveto(536.61781748,346.23582731)(536.68781741,346.29582725)(536.75782076,346.34583456)
\curveto(536.88781721,346.43582711)(537.02281707,346.51582703)(537.16282076,346.58583456)
\curveto(537.30281679,346.66582688)(537.45781664,346.73082681)(537.62782076,346.78083456)
\curveto(537.70781639,346.81082673)(537.7928163,346.83082671)(537.88282076,346.84083456)
\curveto(537.98281611,346.85082669)(538.07781602,346.86582668)(538.16782076,346.88583456)
\curveto(538.20781589,346.89582665)(538.24781585,346.89582665)(538.28782076,346.88583456)
\curveto(538.33781576,346.87582667)(538.37781572,346.88082666)(538.40782076,346.90083456)
\curveto(538.97781512,346.92082662)(539.45781464,346.8408267)(539.84782076,346.66083456)
\curveto(540.24781385,346.49082705)(540.58781351,346.26582728)(540.86782076,345.98583456)
\curveto(540.91781318,345.93582761)(540.96281313,345.88582766)(541.00282076,345.83583456)
\curveto(541.04281305,345.79582775)(541.08281301,345.75082779)(541.12282076,345.70083456)
\curveto(541.1928129,345.61082793)(541.25281284,345.52082802)(541.30282076,345.43083456)
\curveto(541.36281273,345.3408282)(541.41781268,345.25082829)(541.46782076,345.16083456)
\curveto(541.48781261,345.1408284)(541.4978126,345.11582843)(541.49782076,345.08583456)
\curveto(541.50781259,345.05582849)(541.52281257,345.02082852)(541.54282076,344.98083456)
\curveto(541.60281249,344.88082866)(541.65781244,344.76082878)(541.70782076,344.62083456)
\curveto(541.72781237,344.56082898)(541.74781235,344.49582905)(541.76782076,344.42583456)
\curveto(541.78781231,344.36582918)(541.80781229,344.30082924)(541.82782076,344.23083456)
\curveto(541.86781223,344.11082943)(541.8928122,343.98582956)(541.90282076,343.85583456)
\curveto(541.92281217,343.72582982)(541.94781215,343.59082995)(541.97782076,343.45083456)
\lineto(541.97782076,343.28583456)
\lineto(542.00782076,343.10583456)
\lineto(542.00782076,342.94083456)
\moveto(539.89282076,342.59583456)
\curveto(539.90281419,342.6458309)(539.90781419,342.71083083)(539.90782076,342.79083456)
\curveto(539.90781419,342.88083066)(539.90281419,342.95083059)(539.89282076,343.00083456)
\lineto(539.89282076,343.13583456)
\curveto(539.87281422,343.19583035)(539.86281423,343.26083028)(539.86282076,343.33083456)
\curveto(539.86281423,343.40083014)(539.85281424,343.47083007)(539.83282076,343.54083456)
\curveto(539.81281428,343.6408299)(539.7928143,343.73582981)(539.77282076,343.82583456)
\curveto(539.75281434,343.92582962)(539.72281437,344.01582953)(539.68282076,344.09583456)
\curveto(539.56281453,344.41582913)(539.40781469,344.67082887)(539.21782076,344.86083456)
\curveto(539.02781507,345.05082849)(538.75781534,345.19082835)(538.40782076,345.28083456)
\curveto(538.32781577,345.30082824)(538.23781586,345.31082823)(538.13782076,345.31083456)
\lineto(537.86782076,345.31083456)
\curveto(537.82781627,345.30082824)(537.7928163,345.29582825)(537.76282076,345.29583456)
\curveto(537.73281636,345.29582825)(537.6978164,345.29082825)(537.65782076,345.28083456)
\lineto(537.44782076,345.22083456)
\curveto(537.38781671,345.21082833)(537.32781677,345.19082835)(537.26782076,345.16083456)
\curveto(537.00781709,345.05082849)(536.80281729,344.88082866)(536.65282076,344.65083456)
\curveto(536.51281758,344.42082912)(536.3978177,344.16582938)(536.30782076,343.88583456)
\curveto(536.28781781,343.80582974)(536.27281782,343.72082982)(536.26282076,343.63083456)
\curveto(536.25281784,343.55082999)(536.23781786,343.47083007)(536.21782076,343.39083456)
\curveto(536.20781789,343.35083019)(536.20281789,343.28583026)(536.20282076,343.19583456)
\curveto(536.18281791,343.15583039)(536.17781792,343.10583044)(536.18782076,343.04583456)
\curveto(536.1978179,342.99583055)(536.1978179,342.9458306)(536.18782076,342.89583456)
\curveto(536.16781793,342.83583071)(536.16781793,342.78083076)(536.18782076,342.73083456)
\lineto(536.18782076,342.55083456)
\lineto(536.18782076,342.41583456)
\curveto(536.18781791,342.37583117)(536.1978179,342.33583121)(536.21782076,342.29583456)
\curveto(536.21781788,342.22583132)(536.22281787,342.17083137)(536.23282076,342.13083456)
\lineto(536.26282076,341.95083456)
\curveto(536.27281782,341.89083165)(536.28781781,341.83083171)(536.30782076,341.77083456)
\curveto(536.3978177,341.48083206)(536.50281759,341.2408323)(536.62282076,341.05083456)
\curveto(536.75281734,340.87083267)(536.93281716,340.71083283)(537.16282076,340.57083456)
\curveto(537.30281679,340.49083305)(537.46781663,340.42583312)(537.65782076,340.37583456)
\curveto(537.6978164,340.36583318)(537.73281636,340.36083318)(537.76282076,340.36083456)
\curveto(537.7928163,340.37083317)(537.82781627,340.37083317)(537.86782076,340.36083456)
\curveto(537.90781619,340.35083319)(537.96781613,340.3408332)(538.04782076,340.33083456)
\curveto(538.12781597,340.33083321)(538.1928159,340.33583321)(538.24282076,340.34583456)
\curveto(538.32281577,340.36583318)(538.40281569,340.38083316)(538.48282076,340.39083456)
\curveto(538.57281552,340.41083313)(538.65781544,340.43583311)(538.73782076,340.46583456)
\curveto(538.97781512,340.56583298)(539.17281492,340.70583284)(539.32282076,340.88583456)
\curveto(539.47281462,341.06583248)(539.5978145,341.27583227)(539.69782076,341.51583456)
\curveto(539.74781435,341.63583191)(539.78281431,341.76083178)(539.80282076,341.89083456)
\curveto(539.82281427,342.02083152)(539.84781425,342.15583139)(539.87782076,342.29583456)
\lineto(539.87782076,342.44583456)
\curveto(539.88781421,342.49583105)(539.8928142,342.545831)(539.89282076,342.59583456)
}
}
{
\newrgbcolor{curcolor}{0 0 0}
\pscustom[linestyle=none,fillstyle=solid,fillcolor=curcolor]
{
\newpath
\moveto(543.70774263,346.70583456)
\lineto(544.83274263,346.70583456)
\curveto(544.9427402,346.70582684)(545.0427401,346.70082684)(545.13274263,346.69083456)
\curveto(545.22273992,346.68082686)(545.28773985,346.6458269)(545.32774263,346.58583456)
\curveto(545.37773976,346.52582702)(545.40773973,346.4408271)(545.41774263,346.33083456)
\curveto(545.42773971,346.23082731)(545.43273971,346.12582742)(545.43274263,346.01583456)
\lineto(545.43274263,344.96583456)
\lineto(545.43274263,342.73083456)
\curveto(545.43273971,342.37083117)(545.44773969,342.03083151)(545.47774263,341.71083456)
\curveto(545.50773963,341.39083215)(545.59773954,341.12583242)(545.74774263,340.91583456)
\curveto(545.88773925,340.70583284)(546.11273903,340.55583299)(546.42274263,340.46583456)
\curveto(546.47273867,340.45583309)(546.51273863,340.45083309)(546.54274263,340.45083456)
\curveto(546.58273856,340.45083309)(546.62773851,340.4458331)(546.67774263,340.43583456)
\curveto(546.72773841,340.42583312)(546.78273836,340.42083312)(546.84274263,340.42083456)
\curveto(546.90273824,340.42083312)(546.94773819,340.42583312)(546.97774263,340.43583456)
\curveto(547.02773811,340.45583309)(547.06773807,340.46083308)(547.09774263,340.45083456)
\curveto(547.137738,340.4408331)(547.17773796,340.4458331)(547.21774263,340.46583456)
\curveto(547.42773771,340.51583303)(547.59273755,340.58083296)(547.71274263,340.66083456)
\curveto(547.89273725,340.77083277)(548.03273711,340.91083263)(548.13274263,341.08083456)
\curveto(548.2427369,341.26083228)(548.31773682,341.45583209)(548.35774263,341.66583456)
\curveto(548.40773673,341.88583166)(548.4377367,342.12583142)(548.44774263,342.38583456)
\curveto(548.45773668,342.65583089)(548.46273668,342.93583061)(548.46274263,343.22583456)
\lineto(548.46274263,345.04083456)
\lineto(548.46274263,346.01583456)
\lineto(548.46274263,346.28583456)
\curveto(548.46273668,346.38582716)(548.48273666,346.46582708)(548.52274263,346.52583456)
\curveto(548.57273657,346.61582693)(548.64773649,346.66582688)(548.74774263,346.67583456)
\curveto(548.84773629,346.69582685)(548.96773617,346.70582684)(549.10774263,346.70583456)
\lineto(549.90274263,346.70583456)
\lineto(550.18774263,346.70583456)
\curveto(550.27773486,346.70582684)(550.35273479,346.68582686)(550.41274263,346.64583456)
\curveto(550.49273465,346.59582695)(550.5377346,346.52082702)(550.54774263,346.42083456)
\curveto(550.55773458,346.32082722)(550.56273458,346.20582734)(550.56274263,346.07583456)
\lineto(550.56274263,344.93583456)
\lineto(550.56274263,340.72083456)
\lineto(550.56274263,339.65583456)
\lineto(550.56274263,339.35583456)
\curveto(550.56273458,339.25583429)(550.5427346,339.18083436)(550.50274263,339.13083456)
\curveto(550.45273469,339.05083449)(550.37773476,339.00583454)(550.27774263,338.99583456)
\curveto(550.17773496,338.98583456)(550.07273507,338.98083456)(549.96274263,338.98083456)
\lineto(549.15274263,338.98083456)
\curveto(549.0427361,338.98083456)(548.9427362,338.98583456)(548.85274263,338.99583456)
\curveto(548.77273637,339.00583454)(548.70773643,339.0458345)(548.65774263,339.11583456)
\curveto(548.6377365,339.1458344)(548.61773652,339.19083435)(548.59774263,339.25083456)
\curveto(548.58773655,339.31083423)(548.57273657,339.37083417)(548.55274263,339.43083456)
\curveto(548.5427366,339.49083405)(548.52773661,339.545834)(548.50774263,339.59583456)
\curveto(548.48773665,339.6458339)(548.45773668,339.67583387)(548.41774263,339.68583456)
\curveto(548.39773674,339.70583384)(548.37273677,339.71083383)(548.34274263,339.70083456)
\curveto(548.31273683,339.69083385)(548.28773685,339.68083386)(548.26774263,339.67083456)
\curveto(548.19773694,339.63083391)(548.137737,339.58583396)(548.08774263,339.53583456)
\curveto(548.0377371,339.48583406)(547.98273716,339.4408341)(547.92274263,339.40083456)
\curveto(547.88273726,339.37083417)(547.8427373,339.33583421)(547.80274263,339.29583456)
\curveto(547.77273737,339.26583428)(547.73273741,339.23583431)(547.68274263,339.20583456)
\curveto(547.45273769,339.06583448)(547.18273796,338.95583459)(546.87274263,338.87583456)
\curveto(546.80273834,338.85583469)(546.73273841,338.8458347)(546.66274263,338.84583456)
\curveto(546.59273855,338.83583471)(546.51773862,338.82083472)(546.43774263,338.80083456)
\curveto(546.39773874,338.79083475)(546.35273879,338.79083475)(546.30274263,338.80083456)
\curveto(546.26273888,338.80083474)(546.22273892,338.79583475)(546.18274263,338.78583456)
\curveto(546.15273899,338.77583477)(546.08773905,338.77583477)(545.98774263,338.78583456)
\curveto(545.89773924,338.78583476)(545.8377393,338.79083475)(545.80774263,338.80083456)
\curveto(545.75773938,338.80083474)(545.70773943,338.80583474)(545.65774263,338.81583456)
\lineto(545.50774263,338.81583456)
\curveto(545.38773975,338.8458347)(545.27273987,338.87083467)(545.16274263,338.89083456)
\curveto(545.05274009,338.91083463)(544.9427402,338.9408346)(544.83274263,338.98083456)
\curveto(544.78274036,339.00083454)(544.7377404,339.01583453)(544.69774263,339.02583456)
\curveto(544.66774047,339.0458345)(544.62774051,339.06583448)(544.57774263,339.08583456)
\curveto(544.22774091,339.27583427)(543.94774119,339.540834)(543.73774263,339.88083456)
\curveto(543.60774153,340.09083345)(543.51274163,340.3408332)(543.45274263,340.63083456)
\curveto(543.39274175,340.93083261)(543.35274179,341.2458323)(543.33274263,341.57583456)
\curveto(543.32274182,341.91583163)(543.31774182,342.26083128)(543.31774263,342.61083456)
\curveto(543.32774181,342.97083057)(543.33274181,343.32583022)(543.33274263,343.67583456)
\lineto(543.33274263,345.71583456)
\curveto(543.33274181,345.8458277)(543.32774181,345.99582755)(543.31774263,346.16583456)
\curveto(543.31774182,346.3458272)(543.3427418,346.47582707)(543.39274263,346.55583456)
\curveto(543.42274172,346.60582694)(543.48274166,346.65082689)(543.57274263,346.69083456)
\curveto(543.63274151,346.69082685)(543.67774146,346.69582685)(543.70774263,346.70583456)
}
}
{
\newrgbcolor{curcolor}{0 0 0}
\pscustom[linestyle=none,fillstyle=solid,fillcolor=curcolor]
{
\newpath
\moveto(552.70399263,349.67583456)
\lineto(553.79899263,349.67583456)
\curveto(553.89899015,349.67582387)(553.99399005,349.67082387)(554.08399263,349.66083456)
\curveto(554.17398987,349.65082389)(554.2439898,349.62082392)(554.29399263,349.57083456)
\curveto(554.35398969,349.50082404)(554.38398966,349.40582414)(554.38399263,349.28583456)
\curveto(554.39398965,349.17582437)(554.39898965,349.06082448)(554.39899263,348.94083456)
\lineto(554.39899263,347.60583456)
\lineto(554.39899263,342.22083456)
\lineto(554.39899263,339.92583456)
\lineto(554.39899263,339.50583456)
\curveto(554.40898964,339.35583419)(554.38898966,339.2408343)(554.33899263,339.16083456)
\curveto(554.28898976,339.08083446)(554.19898985,339.02583452)(554.06899263,338.99583456)
\curveto(554.00899004,338.97583457)(553.93899011,338.97083457)(553.85899263,338.98083456)
\curveto(553.78899026,338.99083455)(553.71899033,338.99583455)(553.64899263,338.99583456)
\lineto(552.92899263,338.99583456)
\curveto(552.81899123,338.99583455)(552.71899133,339.00083454)(552.62899263,339.01083456)
\curveto(552.53899151,339.02083452)(552.46399158,339.05083449)(552.40399263,339.10083456)
\curveto(552.3439917,339.15083439)(552.30899174,339.22583432)(552.29899263,339.32583456)
\lineto(552.29899263,339.65583456)
\lineto(552.29899263,340.99083456)
\lineto(552.29899263,346.61583456)
\lineto(552.29899263,348.65583456)
\curveto(552.29899175,348.78582476)(552.29399175,348.9408246)(552.28399263,349.12083456)
\curveto(552.28399176,349.30082424)(552.30899174,349.43082411)(552.35899263,349.51083456)
\curveto(552.37899167,349.55082399)(552.40399164,349.58082396)(552.43399263,349.60083456)
\lineto(552.55399263,349.66083456)
\curveto(552.57399147,349.66082388)(552.59899145,349.66082388)(552.62899263,349.66083456)
\curveto(552.65899139,349.67082387)(552.68399136,349.67582387)(552.70399263,349.67583456)
}
}
{
\newrgbcolor{curcolor}{0 0 0}
\pscustom[linestyle=none,fillstyle=solid,fillcolor=curcolor]
{
\newpath
\moveto(563.11118013,339.58083456)
\curveto(563.13117228,339.47083407)(563.14117227,339.36083418)(563.14118013,339.25083456)
\curveto(563.15117226,339.1408344)(563.10117231,339.06583448)(562.99118013,339.02583456)
\curveto(562.93117248,338.99583455)(562.86117255,338.98083456)(562.78118013,338.98083456)
\lineto(562.54118013,338.98083456)
\lineto(561.73118013,338.98083456)
\lineto(561.46118013,338.98083456)
\curveto(561.38117403,338.99083455)(561.3161741,339.01583453)(561.26618013,339.05583456)
\curveto(561.19617422,339.09583445)(561.14117427,339.15083439)(561.10118013,339.22083456)
\curveto(561.07117434,339.30083424)(561.02617439,339.36583418)(560.96618013,339.41583456)
\curveto(560.94617447,339.43583411)(560.92117449,339.45083409)(560.89118013,339.46083456)
\curveto(560.86117455,339.48083406)(560.82117459,339.48583406)(560.77118013,339.47583456)
\curveto(560.72117469,339.45583409)(560.67117474,339.43083411)(560.62118013,339.40083456)
\curveto(560.58117483,339.37083417)(560.53617488,339.3458342)(560.48618013,339.32583456)
\curveto(560.43617498,339.28583426)(560.38117503,339.25083429)(560.32118013,339.22083456)
\lineto(560.14118013,339.13083456)
\curveto(560.0111754,339.07083447)(559.87617554,339.02083452)(559.73618013,338.98083456)
\curveto(559.59617582,338.95083459)(559.45117596,338.91583463)(559.30118013,338.87583456)
\curveto(559.23117618,338.85583469)(559.16117625,338.8458347)(559.09118013,338.84583456)
\curveto(559.03117638,338.83583471)(558.96617645,338.82583472)(558.89618013,338.81583456)
\lineto(558.80618013,338.81583456)
\curveto(558.77617664,338.80583474)(558.74617667,338.80083474)(558.71618013,338.80083456)
\lineto(558.55118013,338.80083456)
\curveto(558.45117696,338.78083476)(558.35117706,338.78083476)(558.25118013,338.80083456)
\lineto(558.11618013,338.80083456)
\curveto(558.04617737,338.82083472)(557.97617744,338.83083471)(557.90618013,338.83083456)
\curveto(557.84617757,338.82083472)(557.78617763,338.82583472)(557.72618013,338.84583456)
\curveto(557.62617779,338.86583468)(557.53117788,338.88583466)(557.44118013,338.90583456)
\curveto(557.35117806,338.91583463)(557.26617815,338.9408346)(557.18618013,338.98083456)
\curveto(556.89617852,339.09083445)(556.64617877,339.23083431)(556.43618013,339.40083456)
\curveto(556.23617918,339.58083396)(556.07617934,339.81583373)(555.95618013,340.10583456)
\curveto(555.92617949,340.17583337)(555.89617952,340.25083329)(555.86618013,340.33083456)
\curveto(555.84617957,340.41083313)(555.82617959,340.49583305)(555.80618013,340.58583456)
\curveto(555.78617963,340.63583291)(555.77617964,340.68583286)(555.77618013,340.73583456)
\curveto(555.78617963,340.78583276)(555.78617963,340.83583271)(555.77618013,340.88583456)
\curveto(555.76617965,340.91583263)(555.75617966,340.97583257)(555.74618013,341.06583456)
\curveto(555.74617967,341.16583238)(555.75117966,341.23583231)(555.76118013,341.27583456)
\curveto(555.78117963,341.37583217)(555.79117962,341.46083208)(555.79118013,341.53083456)
\lineto(555.88118013,341.86083456)
\curveto(555.9111795,341.98083156)(555.95117946,342.08583146)(556.00118013,342.17583456)
\curveto(556.17117924,342.46583108)(556.36617905,342.68583086)(556.58618013,342.83583456)
\curveto(556.80617861,342.98583056)(557.08617833,343.11583043)(557.42618013,343.22583456)
\curveto(557.55617786,343.27583027)(557.69117772,343.31083023)(557.83118013,343.33083456)
\curveto(557.97117744,343.35083019)(558.1111773,343.37583017)(558.25118013,343.40583456)
\curveto(558.33117708,343.42583012)(558.416177,343.43583011)(558.50618013,343.43583456)
\curveto(558.59617682,343.4458301)(558.68617673,343.46083008)(558.77618013,343.48083456)
\curveto(558.84617657,343.50083004)(558.9161765,343.50583004)(558.98618013,343.49583456)
\curveto(559.05617636,343.49583005)(559.13117628,343.50583004)(559.21118013,343.52583456)
\curveto(559.28117613,343.54583)(559.35117606,343.55582999)(559.42118013,343.55583456)
\curveto(559.49117592,343.55582999)(559.56617585,343.56582998)(559.64618013,343.58583456)
\curveto(559.85617556,343.63582991)(560.04617537,343.67582987)(560.21618013,343.70583456)
\curveto(560.39617502,343.7458298)(560.55617486,343.83582971)(560.69618013,343.97583456)
\curveto(560.78617463,344.06582948)(560.84617457,344.16582938)(560.87618013,344.27583456)
\curveto(560.88617453,344.30582924)(560.88617453,344.33082921)(560.87618013,344.35083456)
\curveto(560.87617454,344.37082917)(560.88117453,344.39082915)(560.89118013,344.41083456)
\curveto(560.90117451,344.43082911)(560.90617451,344.46082908)(560.90618013,344.50083456)
\lineto(560.90618013,344.59083456)
\lineto(560.87618013,344.71083456)
\curveto(560.87617454,344.75082879)(560.87117454,344.78582876)(560.86118013,344.81583456)
\curveto(560.76117465,345.11582843)(560.55117486,345.32082822)(560.23118013,345.43083456)
\curveto(560.14117527,345.46082808)(560.03117538,345.48082806)(559.90118013,345.49083456)
\curveto(559.78117563,345.51082803)(559.65617576,345.51582803)(559.52618013,345.50583456)
\curveto(559.39617602,345.50582804)(559.27117614,345.49582805)(559.15118013,345.47583456)
\curveto(559.03117638,345.45582809)(558.92617649,345.43082811)(558.83618013,345.40083456)
\curveto(558.77617664,345.38082816)(558.7161767,345.35082819)(558.65618013,345.31083456)
\curveto(558.60617681,345.28082826)(558.55617686,345.2458283)(558.50618013,345.20583456)
\curveto(558.45617696,345.16582838)(558.40117701,345.11082843)(558.34118013,345.04083456)
\curveto(558.29117712,344.97082857)(558.25617716,344.90582864)(558.23618013,344.84583456)
\curveto(558.18617723,344.7458288)(558.14117727,344.65082889)(558.10118013,344.56083456)
\curveto(558.07117734,344.47082907)(558.00117741,344.41082913)(557.89118013,344.38083456)
\curveto(557.8111776,344.36082918)(557.72617769,344.35082919)(557.63618013,344.35083456)
\lineto(557.36618013,344.35083456)
\lineto(556.79618013,344.35083456)
\curveto(556.74617867,344.35082919)(556.69617872,344.3458292)(556.64618013,344.33583456)
\curveto(556.59617882,344.33582921)(556.55117886,344.3408292)(556.51118013,344.35083456)
\lineto(556.37618013,344.35083456)
\curveto(556.35617906,344.36082918)(556.33117908,344.36582918)(556.30118013,344.36583456)
\curveto(556.27117914,344.36582918)(556.24617917,344.37582917)(556.22618013,344.39583456)
\curveto(556.14617927,344.41582913)(556.09117932,344.48082906)(556.06118013,344.59083456)
\curveto(556.05117936,344.6408289)(556.05117936,344.69082885)(556.06118013,344.74083456)
\curveto(556.07117934,344.79082875)(556.08117933,344.83582871)(556.09118013,344.87583456)
\curveto(556.12117929,344.98582856)(556.15117926,345.08582846)(556.18118013,345.17583456)
\curveto(556.22117919,345.27582827)(556.26617915,345.36582818)(556.31618013,345.44583456)
\lineto(556.40618013,345.59583456)
\lineto(556.49618013,345.74583456)
\curveto(556.57617884,345.85582769)(556.67617874,345.96082758)(556.79618013,346.06083456)
\curveto(556.8161786,346.07082747)(556.84617857,346.09582745)(556.88618013,346.13583456)
\curveto(556.93617848,346.17582737)(556.98117843,346.21082733)(557.02118013,346.24083456)
\curveto(557.06117835,346.27082727)(557.10617831,346.30082724)(557.15618013,346.33083456)
\curveto(557.32617809,346.4408271)(557.50617791,346.52582702)(557.69618013,346.58583456)
\curveto(557.88617753,346.65582689)(558.08117733,346.72082682)(558.28118013,346.78083456)
\curveto(558.40117701,346.81082673)(558.52617689,346.83082671)(558.65618013,346.84083456)
\curveto(558.78617663,346.85082669)(558.9161765,346.87082667)(559.04618013,346.90083456)
\curveto(559.08617633,346.91082663)(559.14617627,346.91082663)(559.22618013,346.90083456)
\curveto(559.3161761,346.89082665)(559.37117604,346.89582665)(559.39118013,346.91583456)
\curveto(559.80117561,346.92582662)(560.19117522,346.91082663)(560.56118013,346.87083456)
\curveto(560.94117447,346.83082671)(561.28117413,346.75582679)(561.58118013,346.64583456)
\curveto(561.89117352,346.53582701)(562.15617326,346.38582716)(562.37618013,346.19583456)
\curveto(562.59617282,346.01582753)(562.76617265,345.78082776)(562.88618013,345.49083456)
\curveto(562.95617246,345.32082822)(562.99617242,345.12582842)(563.00618013,344.90583456)
\curveto(563.0161724,344.68582886)(563.02117239,344.46082908)(563.02118013,344.23083456)
\lineto(563.02118013,340.88583456)
\lineto(563.02118013,340.30083456)
\curveto(563.02117239,340.11083343)(563.04117237,339.93583361)(563.08118013,339.77583456)
\curveto(563.09117232,339.7458338)(563.09617232,339.71083383)(563.09618013,339.67083456)
\curveto(563.09617232,339.6408339)(563.10117231,339.61083393)(563.11118013,339.58083456)
\moveto(560.90618013,341.89083456)
\curveto(560.9161745,341.9408316)(560.92117449,341.99583155)(560.92118013,342.05583456)
\curveto(560.92117449,342.12583142)(560.9161745,342.18583136)(560.90618013,342.23583456)
\curveto(560.88617453,342.29583125)(560.87617454,342.35083119)(560.87618013,342.40083456)
\curveto(560.87617454,342.45083109)(560.85617456,342.49083105)(560.81618013,342.52083456)
\curveto(560.76617465,342.56083098)(560.69117472,342.58083096)(560.59118013,342.58083456)
\curveto(560.55117486,342.57083097)(560.5161749,342.56083098)(560.48618013,342.55083456)
\curveto(560.45617496,342.55083099)(560.42117499,342.545831)(560.38118013,342.53583456)
\curveto(560.3111751,342.51583103)(560.23617518,342.50083104)(560.15618013,342.49083456)
\curveto(560.07617534,342.48083106)(559.99617542,342.46583108)(559.91618013,342.44583456)
\curveto(559.88617553,342.43583111)(559.84117557,342.43083111)(559.78118013,342.43083456)
\curveto(559.65117576,342.40083114)(559.52117589,342.38083116)(559.39118013,342.37083456)
\curveto(559.26117615,342.36083118)(559.13617628,342.33583121)(559.01618013,342.29583456)
\curveto(558.93617648,342.27583127)(558.86117655,342.25583129)(558.79118013,342.23583456)
\curveto(558.72117669,342.22583132)(558.65117676,342.20583134)(558.58118013,342.17583456)
\curveto(558.37117704,342.08583146)(558.19117722,341.95083159)(558.04118013,341.77083456)
\curveto(557.90117751,341.59083195)(557.85117756,341.3408322)(557.89118013,341.02083456)
\curveto(557.9111775,340.85083269)(557.96617745,340.71083283)(558.05618013,340.60083456)
\curveto(558.12617729,340.49083305)(558.23117718,340.40083314)(558.37118013,340.33083456)
\curveto(558.5111769,340.27083327)(558.66117675,340.22583332)(558.82118013,340.19583456)
\curveto(558.99117642,340.16583338)(559.16617625,340.15583339)(559.34618013,340.16583456)
\curveto(559.53617588,340.18583336)(559.7111757,340.22083332)(559.87118013,340.27083456)
\curveto(560.13117528,340.35083319)(560.33617508,340.47583307)(560.48618013,340.64583456)
\curveto(560.63617478,340.82583272)(560.75117466,341.0458325)(560.83118013,341.30583456)
\curveto(560.85117456,341.37583217)(560.86117455,341.4458321)(560.86118013,341.51583456)
\curveto(560.87117454,341.59583195)(560.88617453,341.67583187)(560.90618013,341.75583456)
\lineto(560.90618013,341.89083456)
}
}
{
\newrgbcolor{curcolor}{0 0 0}
\pscustom[linestyle=none,fillstyle=solid,fillcolor=curcolor]
{
\newpath
\moveto(569.09946138,346.91583456)
\curveto(569.20945607,346.91582663)(569.30445597,346.90582664)(569.38446138,346.88583456)
\curveto(569.4744558,346.86582668)(569.54445573,346.82082672)(569.59446138,346.75083456)
\curveto(569.65445562,346.67082687)(569.68445559,346.53082701)(569.68446138,346.33083456)
\lineto(569.68446138,345.82083456)
\lineto(569.68446138,345.44583456)
\curveto(569.69445558,345.30582824)(569.6794556,345.19582835)(569.63946138,345.11583456)
\curveto(569.59945568,345.0458285)(569.53945574,345.00082854)(569.45946138,344.98083456)
\curveto(569.38945589,344.96082858)(569.30445597,344.95082859)(569.20446138,344.95083456)
\curveto(569.11445616,344.95082859)(569.01445626,344.95582859)(568.90446138,344.96583456)
\curveto(568.80445647,344.97582857)(568.70945657,344.97082857)(568.61946138,344.95083456)
\curveto(568.54945673,344.93082861)(568.4794568,344.91582863)(568.40946138,344.90583456)
\curveto(568.33945694,344.90582864)(568.274457,344.89582865)(568.21446138,344.87583456)
\curveto(568.05445722,344.82582872)(567.89445738,344.75082879)(567.73446138,344.65083456)
\curveto(567.5744577,344.56082898)(567.44945783,344.45582909)(567.35946138,344.33583456)
\curveto(567.30945797,344.25582929)(567.25445802,344.17082937)(567.19446138,344.08083456)
\curveto(567.14445813,344.00082954)(567.09445818,343.91582963)(567.04446138,343.82583456)
\curveto(567.01445826,343.7458298)(566.98445829,343.66082988)(566.95446138,343.57083456)
\lineto(566.89446138,343.33083456)
\curveto(566.8744584,343.26083028)(566.86445841,343.18583036)(566.86446138,343.10583456)
\curveto(566.86445841,343.03583051)(566.85445842,342.96583058)(566.83446138,342.89583456)
\curveto(566.82445845,342.85583069)(566.81945846,342.81583073)(566.81946138,342.77583456)
\curveto(566.82945845,342.7458308)(566.82945845,342.71583083)(566.81946138,342.68583456)
\lineto(566.81946138,342.44583456)
\curveto(566.79945848,342.37583117)(566.79445848,342.29583125)(566.80446138,342.20583456)
\curveto(566.81445846,342.12583142)(566.81945846,342.0458315)(566.81946138,341.96583456)
\lineto(566.81946138,341.00583456)
\lineto(566.81946138,339.73083456)
\curveto(566.81945846,339.60083394)(566.81445846,339.48083406)(566.80446138,339.37083456)
\curveto(566.79445848,339.26083428)(566.76445851,339.17083437)(566.71446138,339.10083456)
\curveto(566.69445858,339.07083447)(566.65945862,339.0458345)(566.60946138,339.02583456)
\curveto(566.56945871,339.01583453)(566.52445875,339.00583454)(566.47446138,338.99583456)
\lineto(566.39946138,338.99583456)
\curveto(566.34945893,338.98583456)(566.29445898,338.98083456)(566.23446138,338.98083456)
\lineto(566.06946138,338.98083456)
\lineto(565.42446138,338.98083456)
\curveto(565.36445991,338.99083455)(565.29945998,338.99583455)(565.22946138,338.99583456)
\lineto(565.03446138,338.99583456)
\curveto(564.98446029,339.01583453)(564.93446034,339.03083451)(564.88446138,339.04083456)
\curveto(564.83446044,339.06083448)(564.79946048,339.09583445)(564.77946138,339.14583456)
\curveto(564.73946054,339.19583435)(564.71446056,339.26583428)(564.70446138,339.35583456)
\lineto(564.70446138,339.65583456)
\lineto(564.70446138,340.67583456)
\lineto(564.70446138,344.90583456)
\lineto(564.70446138,346.01583456)
\lineto(564.70446138,346.30083456)
\curveto(564.70446057,346.40082714)(564.72446055,346.48082706)(564.76446138,346.54083456)
\curveto(564.81446046,346.62082692)(564.88946039,346.67082687)(564.98946138,346.69083456)
\curveto(565.08946019,346.71082683)(565.20946007,346.72082682)(565.34946138,346.72083456)
\lineto(566.11446138,346.72083456)
\curveto(566.23445904,346.72082682)(566.33945894,346.71082683)(566.42946138,346.69083456)
\curveto(566.51945876,346.68082686)(566.58945869,346.63582691)(566.63946138,346.55583456)
\curveto(566.66945861,346.50582704)(566.68445859,346.43582711)(566.68446138,346.34583456)
\lineto(566.71446138,346.07583456)
\curveto(566.72445855,345.99582755)(566.73945854,345.92082762)(566.75946138,345.85083456)
\curveto(566.78945849,345.78082776)(566.83945844,345.7458278)(566.90946138,345.74583456)
\curveto(566.92945835,345.76582778)(566.94945833,345.77582777)(566.96946138,345.77583456)
\curveto(566.98945829,345.77582777)(567.00945827,345.78582776)(567.02946138,345.80583456)
\curveto(567.08945819,345.85582769)(567.13945814,345.91082763)(567.17946138,345.97083456)
\curveto(567.22945805,346.0408275)(567.28945799,346.10082744)(567.35946138,346.15083456)
\curveto(567.39945788,346.18082736)(567.43445784,346.21082733)(567.46446138,346.24083456)
\curveto(567.49445778,346.28082726)(567.52945775,346.31582723)(567.56946138,346.34583456)
\lineto(567.83946138,346.52583456)
\curveto(567.93945734,346.58582696)(568.03945724,346.6408269)(568.13946138,346.69083456)
\curveto(568.23945704,346.73082681)(568.33945694,346.76582678)(568.43946138,346.79583456)
\lineto(568.76946138,346.88583456)
\curveto(568.79945648,346.89582665)(568.85445642,346.89582665)(568.93446138,346.88583456)
\curveto(569.02445625,346.88582666)(569.0794562,346.89582665)(569.09946138,346.91583456)
}
}
{
\newrgbcolor{curcolor}{0 0 0}
\pscustom[linestyle=none,fillstyle=solid,fillcolor=curcolor]
{
\newpath
\moveto(572.60453951,349.57083456)
\curveto(572.67453656,349.49082405)(572.70953652,349.37082417)(572.70953951,349.21083456)
\lineto(572.70953951,348.74583456)
\lineto(572.70953951,348.34083456)
\curveto(572.70953652,348.20082534)(572.67453656,348.10582544)(572.60453951,348.05583456)
\curveto(572.54453669,348.00582554)(572.46453677,347.97582557)(572.36453951,347.96583456)
\curveto(572.27453696,347.95582559)(572.17453706,347.95082559)(572.06453951,347.95083456)
\lineto(571.22453951,347.95083456)
\curveto(571.11453812,347.95082559)(571.01453822,347.95582559)(570.92453951,347.96583456)
\curveto(570.84453839,347.97582557)(570.77453846,348.00582554)(570.71453951,348.05583456)
\curveto(570.67453856,348.08582546)(570.64453859,348.1408254)(570.62453951,348.22083456)
\curveto(570.61453862,348.31082523)(570.60453863,348.40582514)(570.59453951,348.50583456)
\lineto(570.59453951,348.83583456)
\curveto(570.60453863,348.9458246)(570.60953862,349.0408245)(570.60953951,349.12083456)
\lineto(570.60953951,349.33083456)
\curveto(570.61953861,349.40082414)(570.63953859,349.46082408)(570.66953951,349.51083456)
\curveto(570.68953854,349.55082399)(570.71453852,349.58082396)(570.74453951,349.60083456)
\lineto(570.86453951,349.66083456)
\curveto(570.88453835,349.66082388)(570.90953832,349.66082388)(570.93953951,349.66083456)
\curveto(570.96953826,349.67082387)(570.99453824,349.67582387)(571.01453951,349.67583456)
\lineto(572.10953951,349.67583456)
\curveto(572.20953702,349.67582387)(572.30453693,349.67082387)(572.39453951,349.66083456)
\curveto(572.48453675,349.65082389)(572.55453668,349.62082392)(572.60453951,349.57083456)
\moveto(572.70953951,339.80583456)
\curveto(572.70953652,339.60583394)(572.70453653,339.43583411)(572.69453951,339.29583456)
\curveto(572.68453655,339.15583439)(572.59453664,339.06083448)(572.42453951,339.01083456)
\curveto(572.36453687,338.99083455)(572.29953693,338.98083456)(572.22953951,338.98083456)
\curveto(572.15953707,338.99083455)(572.08453715,338.99583455)(572.00453951,338.99583456)
\lineto(571.16453951,338.99583456)
\curveto(571.07453816,338.99583455)(570.98453825,339.00083454)(570.89453951,339.01083456)
\curveto(570.81453842,339.02083452)(570.75453848,339.05083449)(570.71453951,339.10083456)
\curveto(570.65453858,339.17083437)(570.61953861,339.25583429)(570.60953951,339.35583456)
\lineto(570.60953951,339.70083456)
\lineto(570.60953951,346.03083456)
\lineto(570.60953951,346.33083456)
\curveto(570.60953862,346.43082711)(570.6295386,346.51082703)(570.66953951,346.57083456)
\curveto(570.7295385,346.6408269)(570.81453842,346.68582686)(570.92453951,346.70583456)
\curveto(570.94453829,346.71582683)(570.96953826,346.71582683)(570.99953951,346.70583456)
\curveto(571.03953819,346.70582684)(571.06953816,346.71082683)(571.08953951,346.72083456)
\lineto(571.83953951,346.72083456)
\lineto(572.03453951,346.72083456)
\curveto(572.11453712,346.73082681)(572.17953705,346.73082681)(572.22953951,346.72083456)
\lineto(572.34953951,346.72083456)
\curveto(572.40953682,346.70082684)(572.46453677,346.68582686)(572.51453951,346.67583456)
\curveto(572.56453667,346.66582688)(572.60453663,346.63582691)(572.63453951,346.58583456)
\curveto(572.67453656,346.53582701)(572.69453654,346.46582708)(572.69453951,346.37583456)
\curveto(572.70453653,346.28582726)(572.70953652,346.19082735)(572.70953951,346.09083456)
\lineto(572.70953951,339.80583456)
}
}
{
\newrgbcolor{curcolor}{0 0 0}
\pscustom[linestyle=none,fillstyle=solid,fillcolor=curcolor]
{
\newpath
\moveto(581.96172701,339.83583456)
\lineto(581.96172701,339.41583456)
\curveto(581.96171864,339.28583426)(581.93171867,339.18083436)(581.87172701,339.10083456)
\curveto(581.82171878,339.05083449)(581.75671884,339.01583453)(581.67672701,338.99583456)
\curveto(581.596719,338.98583456)(581.50671909,338.98083456)(581.40672701,338.98083456)
\lineto(580.58172701,338.98083456)
\lineto(580.29672701,338.98083456)
\curveto(580.21672038,338.99083455)(580.15172045,339.01583453)(580.10172701,339.05583456)
\curveto(580.03172057,339.10583444)(579.99172061,339.17083437)(579.98172701,339.25083456)
\curveto(579.97172063,339.33083421)(579.95172065,339.41083413)(579.92172701,339.49083456)
\curveto(579.9017207,339.51083403)(579.88172072,339.52583402)(579.86172701,339.53583456)
\curveto(579.85172075,339.55583399)(579.83672076,339.57583397)(579.81672701,339.59583456)
\curveto(579.70672089,339.59583395)(579.62672097,339.57083397)(579.57672701,339.52083456)
\lineto(579.42672701,339.37083456)
\curveto(579.35672124,339.32083422)(579.29172131,339.27583427)(579.23172701,339.23583456)
\curveto(579.17172143,339.20583434)(579.10672149,339.16583438)(579.03672701,339.11583456)
\curveto(578.9967216,339.09583445)(578.95172165,339.07583447)(578.90172701,339.05583456)
\curveto(578.86172174,339.03583451)(578.81672178,339.01583453)(578.76672701,338.99583456)
\curveto(578.62672197,338.9458346)(578.47672212,338.90083464)(578.31672701,338.86083456)
\curveto(578.26672233,338.8408347)(578.22172238,338.83083471)(578.18172701,338.83083456)
\curveto(578.14172246,338.83083471)(578.1017225,338.82583472)(578.06172701,338.81583456)
\lineto(577.92672701,338.81583456)
\curveto(577.8967227,338.80583474)(577.85672274,338.80083474)(577.80672701,338.80083456)
\lineto(577.67172701,338.80083456)
\curveto(577.61172299,338.78083476)(577.52172308,338.77583477)(577.40172701,338.78583456)
\curveto(577.28172332,338.78583476)(577.1967234,338.79583475)(577.14672701,338.81583456)
\curveto(577.07672352,338.83583471)(577.01172359,338.8458347)(576.95172701,338.84583456)
\curveto(576.9017237,338.83583471)(576.84672375,338.8408347)(576.78672701,338.86083456)
\lineto(576.42672701,338.98083456)
\curveto(576.31672428,339.01083453)(576.20672439,339.05083449)(576.09672701,339.10083456)
\curveto(575.74672485,339.25083429)(575.43172517,339.48083406)(575.15172701,339.79083456)
\curveto(574.88172572,340.11083343)(574.66672593,340.4458331)(574.50672701,340.79583456)
\curveto(574.45672614,340.90583264)(574.41672618,341.01083253)(574.38672701,341.11083456)
\curveto(574.35672624,341.22083232)(574.32172628,341.33083221)(574.28172701,341.44083456)
\curveto(574.27172633,341.48083206)(574.26672633,341.51583203)(574.26672701,341.54583456)
\curveto(574.26672633,341.58583196)(574.25672634,341.63083191)(574.23672701,341.68083456)
\curveto(574.21672638,341.76083178)(574.1967264,341.8458317)(574.17672701,341.93583456)
\curveto(574.16672643,342.03583151)(574.15172645,342.13583141)(574.13172701,342.23583456)
\curveto(574.12172648,342.26583128)(574.11672648,342.30083124)(574.11672701,342.34083456)
\curveto(574.12672647,342.38083116)(574.12672647,342.41583113)(574.11672701,342.44583456)
\lineto(574.11672701,342.58083456)
\curveto(574.11672648,342.63083091)(574.11172649,342.68083086)(574.10172701,342.73083456)
\curveto(574.09172651,342.78083076)(574.08672651,342.83583071)(574.08672701,342.89583456)
\curveto(574.08672651,342.96583058)(574.09172651,343.02083052)(574.10172701,343.06083456)
\curveto(574.11172649,343.11083043)(574.11672648,343.15583039)(574.11672701,343.19583456)
\lineto(574.11672701,343.34583456)
\curveto(574.12672647,343.39583015)(574.12672647,343.4408301)(574.11672701,343.48083456)
\curveto(574.11672648,343.53083001)(574.12672647,343.58082996)(574.14672701,343.63083456)
\curveto(574.16672643,343.7408298)(574.18172642,343.8458297)(574.19172701,343.94583456)
\curveto(574.21172639,344.0458295)(574.23672636,344.1458294)(574.26672701,344.24583456)
\curveto(574.30672629,344.36582918)(574.34172626,344.48082906)(574.37172701,344.59083456)
\curveto(574.4017262,344.70082884)(574.44172616,344.81082873)(574.49172701,344.92083456)
\curveto(574.63172597,345.22082832)(574.80672579,345.50582804)(575.01672701,345.77583456)
\curveto(575.03672556,345.80582774)(575.06172554,345.83082771)(575.09172701,345.85083456)
\curveto(575.13172547,345.88082766)(575.16172544,345.91082763)(575.18172701,345.94083456)
\curveto(575.22172538,345.99082755)(575.26172534,346.03582751)(575.30172701,346.07583456)
\curveto(575.34172526,346.11582743)(575.38672521,346.15582739)(575.43672701,346.19583456)
\curveto(575.47672512,346.21582733)(575.51172509,346.2408273)(575.54172701,346.27083456)
\curveto(575.57172503,346.31082723)(575.60672499,346.3408272)(575.64672701,346.36083456)
\curveto(575.8967247,346.53082701)(576.18672441,346.67082687)(576.51672701,346.78083456)
\curveto(576.58672401,346.80082674)(576.65672394,346.81582673)(576.72672701,346.82583456)
\curveto(576.80672379,346.83582671)(576.88672371,346.85082669)(576.96672701,346.87083456)
\curveto(577.03672356,346.89082665)(577.12672347,346.90082664)(577.23672701,346.90083456)
\curveto(577.34672325,346.91082663)(577.45672314,346.91582663)(577.56672701,346.91583456)
\curveto(577.67672292,346.91582663)(577.78172282,346.91082663)(577.88172701,346.90083456)
\curveto(577.99172261,346.89082665)(578.08172252,346.87582667)(578.15172701,346.85583456)
\curveto(578.3017223,346.80582674)(578.44672215,346.76082678)(578.58672701,346.72083456)
\curveto(578.72672187,346.68082686)(578.85672174,346.62582692)(578.97672701,346.55583456)
\curveto(579.04672155,346.50582704)(579.11172149,346.45582709)(579.17172701,346.40583456)
\curveto(579.23172137,346.36582718)(579.2967213,346.32082722)(579.36672701,346.27083456)
\curveto(579.40672119,346.2408273)(579.46172114,346.20082734)(579.53172701,346.15083456)
\curveto(579.61172099,346.10082744)(579.68672091,346.10082744)(579.75672701,346.15083456)
\curveto(579.7967208,346.17082737)(579.81672078,346.20582734)(579.81672701,346.25583456)
\curveto(579.81672078,346.30582724)(579.82672077,346.35582719)(579.84672701,346.40583456)
\lineto(579.84672701,346.55583456)
\curveto(579.85672074,346.58582696)(579.86172074,346.62082692)(579.86172701,346.66083456)
\lineto(579.86172701,346.78083456)
\lineto(579.86172701,348.82083456)
\curveto(579.86172074,348.93082461)(579.85672074,349.05082449)(579.84672701,349.18083456)
\curveto(579.84672075,349.32082422)(579.87172073,349.42582412)(579.92172701,349.49583456)
\curveto(579.96172064,349.57582397)(580.03672056,349.62582392)(580.14672701,349.64583456)
\curveto(580.16672043,349.65582389)(580.18672041,349.65582389)(580.20672701,349.64583456)
\curveto(580.22672037,349.6458239)(580.24672035,349.65082389)(580.26672701,349.66083456)
\lineto(581.33172701,349.66083456)
\curveto(581.45171915,349.66082388)(581.56171904,349.65582389)(581.66172701,349.64583456)
\curveto(581.76171884,349.63582391)(581.83671876,349.59582395)(581.88672701,349.52583456)
\curveto(581.93671866,349.4458241)(581.96171864,349.3408242)(581.96172701,349.21083456)
\lineto(581.96172701,348.85083456)
\lineto(581.96172701,339.83583456)
\moveto(579.92172701,342.77583456)
\curveto(579.93172067,342.81583073)(579.93172067,342.85583069)(579.92172701,342.89583456)
\lineto(579.92172701,343.03083456)
\curveto(579.92172068,343.13083041)(579.91672068,343.23083031)(579.90672701,343.33083456)
\curveto(579.8967207,343.43083011)(579.88172072,343.52083002)(579.86172701,343.60083456)
\curveto(579.84172076,343.71082983)(579.82172078,343.81082973)(579.80172701,343.90083456)
\curveto(579.79172081,343.99082955)(579.76672083,344.07582947)(579.72672701,344.15583456)
\curveto(579.58672101,344.51582903)(579.38172122,344.80082874)(579.11172701,345.01083456)
\curveto(578.85172175,345.22082832)(578.47172213,345.32582822)(577.97172701,345.32583456)
\curveto(577.91172269,345.32582822)(577.83172277,345.31582823)(577.73172701,345.29583456)
\curveto(577.65172295,345.27582827)(577.57672302,345.25582829)(577.50672701,345.23583456)
\curveto(577.44672315,345.22582832)(577.38672321,345.20582834)(577.32672701,345.17583456)
\curveto(577.05672354,345.06582848)(576.84672375,344.89582865)(576.69672701,344.66583456)
\curveto(576.54672405,344.43582911)(576.42672417,344.17582937)(576.33672701,343.88583456)
\curveto(576.30672429,343.78582976)(576.28672431,343.68582986)(576.27672701,343.58583456)
\curveto(576.26672433,343.48583006)(576.24672435,343.38083016)(576.21672701,343.27083456)
\lineto(576.21672701,343.06083456)
\curveto(576.1967244,342.97083057)(576.19172441,342.8458307)(576.20172701,342.68583456)
\curveto(576.21172439,342.53583101)(576.22672437,342.42583112)(576.24672701,342.35583456)
\lineto(576.24672701,342.26583456)
\curveto(576.25672434,342.2458313)(576.26172434,342.22583132)(576.26172701,342.20583456)
\curveto(576.28172432,342.12583142)(576.2967243,342.05083149)(576.30672701,341.98083456)
\curveto(576.32672427,341.91083163)(576.34672425,341.83583171)(576.36672701,341.75583456)
\curveto(576.53672406,341.23583231)(576.82672377,340.85083269)(577.23672701,340.60083456)
\curveto(577.36672323,340.51083303)(577.54672305,340.4408331)(577.77672701,340.39083456)
\curveto(577.81672278,340.38083316)(577.87672272,340.37583317)(577.95672701,340.37583456)
\curveto(577.98672261,340.36583318)(578.03172257,340.35583319)(578.09172701,340.34583456)
\curveto(578.16172244,340.3458332)(578.21672238,340.35083319)(578.25672701,340.36083456)
\curveto(578.33672226,340.38083316)(578.41672218,340.39583315)(578.49672701,340.40583456)
\curveto(578.57672202,340.41583313)(578.65672194,340.43583311)(578.73672701,340.46583456)
\curveto(578.98672161,340.57583297)(579.18672141,340.71583283)(579.33672701,340.88583456)
\curveto(579.48672111,341.05583249)(579.61672098,341.27083227)(579.72672701,341.53083456)
\curveto(579.76672083,341.62083192)(579.7967208,341.71083183)(579.81672701,341.80083456)
\curveto(579.83672076,341.90083164)(579.85672074,342.00583154)(579.87672701,342.11583456)
\curveto(579.88672071,342.16583138)(579.88672071,342.21083133)(579.87672701,342.25083456)
\curveto(579.87672072,342.30083124)(579.88672071,342.35083119)(579.90672701,342.40083456)
\curveto(579.91672068,342.43083111)(579.92172068,342.46583108)(579.92172701,342.50583456)
\lineto(579.92172701,342.64083456)
\lineto(579.92172701,342.77583456)
}
}
{
\newrgbcolor{curcolor}{0 0 0}
\pscustom[linestyle=none,fillstyle=solid,fillcolor=curcolor]
{
\newpath
\moveto(590.59164888,339.58083456)
\curveto(590.61164103,339.47083407)(590.62164102,339.36083418)(590.62164888,339.25083456)
\curveto(590.63164101,339.1408344)(590.58164106,339.06583448)(590.47164888,339.02583456)
\curveto(590.41164123,338.99583455)(590.3416413,338.98083456)(590.26164888,338.98083456)
\lineto(590.02164888,338.98083456)
\lineto(589.21164888,338.98083456)
\lineto(588.94164888,338.98083456)
\curveto(588.86164278,338.99083455)(588.79664285,339.01583453)(588.74664888,339.05583456)
\curveto(588.67664297,339.09583445)(588.62164302,339.15083439)(588.58164888,339.22083456)
\curveto(588.55164309,339.30083424)(588.50664314,339.36583418)(588.44664888,339.41583456)
\curveto(588.42664322,339.43583411)(588.40164324,339.45083409)(588.37164888,339.46083456)
\curveto(588.3416433,339.48083406)(588.30164334,339.48583406)(588.25164888,339.47583456)
\curveto(588.20164344,339.45583409)(588.15164349,339.43083411)(588.10164888,339.40083456)
\curveto(588.06164358,339.37083417)(588.01664363,339.3458342)(587.96664888,339.32583456)
\curveto(587.91664373,339.28583426)(587.86164378,339.25083429)(587.80164888,339.22083456)
\lineto(587.62164888,339.13083456)
\curveto(587.49164415,339.07083447)(587.35664429,339.02083452)(587.21664888,338.98083456)
\curveto(587.07664457,338.95083459)(586.93164471,338.91583463)(586.78164888,338.87583456)
\curveto(586.71164493,338.85583469)(586.641645,338.8458347)(586.57164888,338.84583456)
\curveto(586.51164513,338.83583471)(586.4466452,338.82583472)(586.37664888,338.81583456)
\lineto(586.28664888,338.81583456)
\curveto(586.25664539,338.80583474)(586.22664542,338.80083474)(586.19664888,338.80083456)
\lineto(586.03164888,338.80083456)
\curveto(585.93164571,338.78083476)(585.83164581,338.78083476)(585.73164888,338.80083456)
\lineto(585.59664888,338.80083456)
\curveto(585.52664612,338.82083472)(585.45664619,338.83083471)(585.38664888,338.83083456)
\curveto(585.32664632,338.82083472)(585.26664638,338.82583472)(585.20664888,338.84583456)
\curveto(585.10664654,338.86583468)(585.01164663,338.88583466)(584.92164888,338.90583456)
\curveto(584.83164681,338.91583463)(584.7466469,338.9408346)(584.66664888,338.98083456)
\curveto(584.37664727,339.09083445)(584.12664752,339.23083431)(583.91664888,339.40083456)
\curveto(583.71664793,339.58083396)(583.55664809,339.81583373)(583.43664888,340.10583456)
\curveto(583.40664824,340.17583337)(583.37664827,340.25083329)(583.34664888,340.33083456)
\curveto(583.32664832,340.41083313)(583.30664834,340.49583305)(583.28664888,340.58583456)
\curveto(583.26664838,340.63583291)(583.25664839,340.68583286)(583.25664888,340.73583456)
\curveto(583.26664838,340.78583276)(583.26664838,340.83583271)(583.25664888,340.88583456)
\curveto(583.2466484,340.91583263)(583.23664841,340.97583257)(583.22664888,341.06583456)
\curveto(583.22664842,341.16583238)(583.23164841,341.23583231)(583.24164888,341.27583456)
\curveto(583.26164838,341.37583217)(583.27164837,341.46083208)(583.27164888,341.53083456)
\lineto(583.36164888,341.86083456)
\curveto(583.39164825,341.98083156)(583.43164821,342.08583146)(583.48164888,342.17583456)
\curveto(583.65164799,342.46583108)(583.8466478,342.68583086)(584.06664888,342.83583456)
\curveto(584.28664736,342.98583056)(584.56664708,343.11583043)(584.90664888,343.22583456)
\curveto(585.03664661,343.27583027)(585.17164647,343.31083023)(585.31164888,343.33083456)
\curveto(585.45164619,343.35083019)(585.59164605,343.37583017)(585.73164888,343.40583456)
\curveto(585.81164583,343.42583012)(585.89664575,343.43583011)(585.98664888,343.43583456)
\curveto(586.07664557,343.4458301)(586.16664548,343.46083008)(586.25664888,343.48083456)
\curveto(586.32664532,343.50083004)(586.39664525,343.50583004)(586.46664888,343.49583456)
\curveto(586.53664511,343.49583005)(586.61164503,343.50583004)(586.69164888,343.52583456)
\curveto(586.76164488,343.54583)(586.83164481,343.55582999)(586.90164888,343.55583456)
\curveto(586.97164467,343.55582999)(587.0466446,343.56582998)(587.12664888,343.58583456)
\curveto(587.33664431,343.63582991)(587.52664412,343.67582987)(587.69664888,343.70583456)
\curveto(587.87664377,343.7458298)(588.03664361,343.83582971)(588.17664888,343.97583456)
\curveto(588.26664338,344.06582948)(588.32664332,344.16582938)(588.35664888,344.27583456)
\curveto(588.36664328,344.30582924)(588.36664328,344.33082921)(588.35664888,344.35083456)
\curveto(588.35664329,344.37082917)(588.36164328,344.39082915)(588.37164888,344.41083456)
\curveto(588.38164326,344.43082911)(588.38664326,344.46082908)(588.38664888,344.50083456)
\lineto(588.38664888,344.59083456)
\lineto(588.35664888,344.71083456)
\curveto(588.35664329,344.75082879)(588.35164329,344.78582876)(588.34164888,344.81583456)
\curveto(588.2416434,345.11582843)(588.03164361,345.32082822)(587.71164888,345.43083456)
\curveto(587.62164402,345.46082808)(587.51164413,345.48082806)(587.38164888,345.49083456)
\curveto(587.26164438,345.51082803)(587.13664451,345.51582803)(587.00664888,345.50583456)
\curveto(586.87664477,345.50582804)(586.75164489,345.49582805)(586.63164888,345.47583456)
\curveto(586.51164513,345.45582809)(586.40664524,345.43082811)(586.31664888,345.40083456)
\curveto(586.25664539,345.38082816)(586.19664545,345.35082819)(586.13664888,345.31083456)
\curveto(586.08664556,345.28082826)(586.03664561,345.2458283)(585.98664888,345.20583456)
\curveto(585.93664571,345.16582838)(585.88164576,345.11082843)(585.82164888,345.04083456)
\curveto(585.77164587,344.97082857)(585.73664591,344.90582864)(585.71664888,344.84583456)
\curveto(585.66664598,344.7458288)(585.62164602,344.65082889)(585.58164888,344.56083456)
\curveto(585.55164609,344.47082907)(585.48164616,344.41082913)(585.37164888,344.38083456)
\curveto(585.29164635,344.36082918)(585.20664644,344.35082919)(585.11664888,344.35083456)
\lineto(584.84664888,344.35083456)
\lineto(584.27664888,344.35083456)
\curveto(584.22664742,344.35082919)(584.17664747,344.3458292)(584.12664888,344.33583456)
\curveto(584.07664757,344.33582921)(584.03164761,344.3408292)(583.99164888,344.35083456)
\lineto(583.85664888,344.35083456)
\curveto(583.83664781,344.36082918)(583.81164783,344.36582918)(583.78164888,344.36583456)
\curveto(583.75164789,344.36582918)(583.72664792,344.37582917)(583.70664888,344.39583456)
\curveto(583.62664802,344.41582913)(583.57164807,344.48082906)(583.54164888,344.59083456)
\curveto(583.53164811,344.6408289)(583.53164811,344.69082885)(583.54164888,344.74083456)
\curveto(583.55164809,344.79082875)(583.56164808,344.83582871)(583.57164888,344.87583456)
\curveto(583.60164804,344.98582856)(583.63164801,345.08582846)(583.66164888,345.17583456)
\curveto(583.70164794,345.27582827)(583.7466479,345.36582818)(583.79664888,345.44583456)
\lineto(583.88664888,345.59583456)
\lineto(583.97664888,345.74583456)
\curveto(584.05664759,345.85582769)(584.15664749,345.96082758)(584.27664888,346.06083456)
\curveto(584.29664735,346.07082747)(584.32664732,346.09582745)(584.36664888,346.13583456)
\curveto(584.41664723,346.17582737)(584.46164718,346.21082733)(584.50164888,346.24083456)
\curveto(584.5416471,346.27082727)(584.58664706,346.30082724)(584.63664888,346.33083456)
\curveto(584.80664684,346.4408271)(584.98664666,346.52582702)(585.17664888,346.58583456)
\curveto(585.36664628,346.65582689)(585.56164608,346.72082682)(585.76164888,346.78083456)
\curveto(585.88164576,346.81082673)(586.00664564,346.83082671)(586.13664888,346.84083456)
\curveto(586.26664538,346.85082669)(586.39664525,346.87082667)(586.52664888,346.90083456)
\curveto(586.56664508,346.91082663)(586.62664502,346.91082663)(586.70664888,346.90083456)
\curveto(586.79664485,346.89082665)(586.85164479,346.89582665)(586.87164888,346.91583456)
\curveto(587.28164436,346.92582662)(587.67164397,346.91082663)(588.04164888,346.87083456)
\curveto(588.42164322,346.83082671)(588.76164288,346.75582679)(589.06164888,346.64583456)
\curveto(589.37164227,346.53582701)(589.63664201,346.38582716)(589.85664888,346.19583456)
\curveto(590.07664157,346.01582753)(590.2466414,345.78082776)(590.36664888,345.49083456)
\curveto(590.43664121,345.32082822)(590.47664117,345.12582842)(590.48664888,344.90583456)
\curveto(590.49664115,344.68582886)(590.50164114,344.46082908)(590.50164888,344.23083456)
\lineto(590.50164888,340.88583456)
\lineto(590.50164888,340.30083456)
\curveto(590.50164114,340.11083343)(590.52164112,339.93583361)(590.56164888,339.77583456)
\curveto(590.57164107,339.7458338)(590.57664107,339.71083383)(590.57664888,339.67083456)
\curveto(590.57664107,339.6408339)(590.58164106,339.61083393)(590.59164888,339.58083456)
\moveto(588.38664888,341.89083456)
\curveto(588.39664325,341.9408316)(588.40164324,341.99583155)(588.40164888,342.05583456)
\curveto(588.40164324,342.12583142)(588.39664325,342.18583136)(588.38664888,342.23583456)
\curveto(588.36664328,342.29583125)(588.35664329,342.35083119)(588.35664888,342.40083456)
\curveto(588.35664329,342.45083109)(588.33664331,342.49083105)(588.29664888,342.52083456)
\curveto(588.2466434,342.56083098)(588.17164347,342.58083096)(588.07164888,342.58083456)
\curveto(588.03164361,342.57083097)(587.99664365,342.56083098)(587.96664888,342.55083456)
\curveto(587.93664371,342.55083099)(587.90164374,342.545831)(587.86164888,342.53583456)
\curveto(587.79164385,342.51583103)(587.71664393,342.50083104)(587.63664888,342.49083456)
\curveto(587.55664409,342.48083106)(587.47664417,342.46583108)(587.39664888,342.44583456)
\curveto(587.36664428,342.43583111)(587.32164432,342.43083111)(587.26164888,342.43083456)
\curveto(587.13164451,342.40083114)(587.00164464,342.38083116)(586.87164888,342.37083456)
\curveto(586.7416449,342.36083118)(586.61664503,342.33583121)(586.49664888,342.29583456)
\curveto(586.41664523,342.27583127)(586.3416453,342.25583129)(586.27164888,342.23583456)
\curveto(586.20164544,342.22583132)(586.13164551,342.20583134)(586.06164888,342.17583456)
\curveto(585.85164579,342.08583146)(585.67164597,341.95083159)(585.52164888,341.77083456)
\curveto(585.38164626,341.59083195)(585.33164631,341.3408322)(585.37164888,341.02083456)
\curveto(585.39164625,340.85083269)(585.4466462,340.71083283)(585.53664888,340.60083456)
\curveto(585.60664604,340.49083305)(585.71164593,340.40083314)(585.85164888,340.33083456)
\curveto(585.99164565,340.27083327)(586.1416455,340.22583332)(586.30164888,340.19583456)
\curveto(586.47164517,340.16583338)(586.646645,340.15583339)(586.82664888,340.16583456)
\curveto(587.01664463,340.18583336)(587.19164445,340.22083332)(587.35164888,340.27083456)
\curveto(587.61164403,340.35083319)(587.81664383,340.47583307)(587.96664888,340.64583456)
\curveto(588.11664353,340.82583272)(588.23164341,341.0458325)(588.31164888,341.30583456)
\curveto(588.33164331,341.37583217)(588.3416433,341.4458321)(588.34164888,341.51583456)
\curveto(588.35164329,341.59583195)(588.36664328,341.67583187)(588.38664888,341.75583456)
\lineto(588.38664888,341.89083456)
}
}
{
\newrgbcolor{curcolor}{0 0 0}
\pscustom[linestyle=none,fillstyle=solid,fillcolor=curcolor]
{
\newpath
\moveto(599.74493013,339.83583456)
\lineto(599.74493013,339.41583456)
\curveto(599.74492176,339.28583426)(599.71492179,339.18083436)(599.65493013,339.10083456)
\curveto(599.6049219,339.05083449)(599.53992197,339.01583453)(599.45993013,338.99583456)
\curveto(599.37992213,338.98583456)(599.28992222,338.98083456)(599.18993013,338.98083456)
\lineto(598.36493013,338.98083456)
\lineto(598.07993013,338.98083456)
\curveto(597.99992351,338.99083455)(597.93492357,339.01583453)(597.88493013,339.05583456)
\curveto(597.81492369,339.10583444)(597.77492373,339.17083437)(597.76493013,339.25083456)
\curveto(597.75492375,339.33083421)(597.73492377,339.41083413)(597.70493013,339.49083456)
\curveto(597.68492382,339.51083403)(597.66492384,339.52583402)(597.64493013,339.53583456)
\curveto(597.63492387,339.55583399)(597.61992389,339.57583397)(597.59993013,339.59583456)
\curveto(597.48992402,339.59583395)(597.4099241,339.57083397)(597.35993013,339.52083456)
\lineto(597.20993013,339.37083456)
\curveto(597.13992437,339.32083422)(597.07492443,339.27583427)(597.01493013,339.23583456)
\curveto(596.95492455,339.20583434)(596.88992462,339.16583438)(596.81993013,339.11583456)
\curveto(596.77992473,339.09583445)(596.73492477,339.07583447)(596.68493013,339.05583456)
\curveto(596.64492486,339.03583451)(596.59992491,339.01583453)(596.54993013,338.99583456)
\curveto(596.4099251,338.9458346)(596.25992525,338.90083464)(596.09993013,338.86083456)
\curveto(596.04992546,338.8408347)(596.0049255,338.83083471)(595.96493013,338.83083456)
\curveto(595.92492558,338.83083471)(595.88492562,338.82583472)(595.84493013,338.81583456)
\lineto(595.70993013,338.81583456)
\curveto(595.67992583,338.80583474)(595.63992587,338.80083474)(595.58993013,338.80083456)
\lineto(595.45493013,338.80083456)
\curveto(595.39492611,338.78083476)(595.3049262,338.77583477)(595.18493013,338.78583456)
\curveto(595.06492644,338.78583476)(594.97992653,338.79583475)(594.92993013,338.81583456)
\curveto(594.85992665,338.83583471)(594.79492671,338.8458347)(594.73493013,338.84583456)
\curveto(594.68492682,338.83583471)(594.62992688,338.8408347)(594.56993013,338.86083456)
\lineto(594.20993013,338.98083456)
\curveto(594.09992741,339.01083453)(593.98992752,339.05083449)(593.87993013,339.10083456)
\curveto(593.52992798,339.25083429)(593.21492829,339.48083406)(592.93493013,339.79083456)
\curveto(592.66492884,340.11083343)(592.44992906,340.4458331)(592.28993013,340.79583456)
\curveto(592.23992927,340.90583264)(592.19992931,341.01083253)(592.16993013,341.11083456)
\curveto(592.13992937,341.22083232)(592.1049294,341.33083221)(592.06493013,341.44083456)
\curveto(592.05492945,341.48083206)(592.04992946,341.51583203)(592.04993013,341.54583456)
\curveto(592.04992946,341.58583196)(592.03992947,341.63083191)(592.01993013,341.68083456)
\curveto(591.99992951,341.76083178)(591.97992953,341.8458317)(591.95993013,341.93583456)
\curveto(591.94992956,342.03583151)(591.93492957,342.13583141)(591.91493013,342.23583456)
\curveto(591.9049296,342.26583128)(591.89992961,342.30083124)(591.89993013,342.34083456)
\curveto(591.9099296,342.38083116)(591.9099296,342.41583113)(591.89993013,342.44583456)
\lineto(591.89993013,342.58083456)
\curveto(591.89992961,342.63083091)(591.89492961,342.68083086)(591.88493013,342.73083456)
\curveto(591.87492963,342.78083076)(591.86992964,342.83583071)(591.86993013,342.89583456)
\curveto(591.86992964,342.96583058)(591.87492963,343.02083052)(591.88493013,343.06083456)
\curveto(591.89492961,343.11083043)(591.89992961,343.15583039)(591.89993013,343.19583456)
\lineto(591.89993013,343.34583456)
\curveto(591.9099296,343.39583015)(591.9099296,343.4408301)(591.89993013,343.48083456)
\curveto(591.89992961,343.53083001)(591.9099296,343.58082996)(591.92993013,343.63083456)
\curveto(591.94992956,343.7408298)(591.96492954,343.8458297)(591.97493013,343.94583456)
\curveto(591.99492951,344.0458295)(592.01992949,344.1458294)(592.04993013,344.24583456)
\curveto(592.08992942,344.36582918)(592.12492938,344.48082906)(592.15493013,344.59083456)
\curveto(592.18492932,344.70082884)(592.22492928,344.81082873)(592.27493013,344.92083456)
\curveto(592.41492909,345.22082832)(592.58992892,345.50582804)(592.79993013,345.77583456)
\curveto(592.81992869,345.80582774)(592.84492866,345.83082771)(592.87493013,345.85083456)
\curveto(592.91492859,345.88082766)(592.94492856,345.91082763)(592.96493013,345.94083456)
\curveto(593.0049285,345.99082755)(593.04492846,346.03582751)(593.08493013,346.07583456)
\curveto(593.12492838,346.11582743)(593.16992834,346.15582739)(593.21993013,346.19583456)
\curveto(593.25992825,346.21582733)(593.29492821,346.2408273)(593.32493013,346.27083456)
\curveto(593.35492815,346.31082723)(593.38992812,346.3408272)(593.42993013,346.36083456)
\curveto(593.67992783,346.53082701)(593.96992754,346.67082687)(594.29993013,346.78083456)
\curveto(594.36992714,346.80082674)(594.43992707,346.81582673)(594.50993013,346.82583456)
\curveto(594.58992692,346.83582671)(594.66992684,346.85082669)(594.74993013,346.87083456)
\curveto(594.81992669,346.89082665)(594.9099266,346.90082664)(595.01993013,346.90083456)
\curveto(595.12992638,346.91082663)(595.23992627,346.91582663)(595.34993013,346.91583456)
\curveto(595.45992605,346.91582663)(595.56492594,346.91082663)(595.66493013,346.90083456)
\curveto(595.77492573,346.89082665)(595.86492564,346.87582667)(595.93493013,346.85583456)
\curveto(596.08492542,346.80582674)(596.22992528,346.76082678)(596.36993013,346.72083456)
\curveto(596.509925,346.68082686)(596.63992487,346.62582692)(596.75993013,346.55583456)
\curveto(596.82992468,346.50582704)(596.89492461,346.45582709)(596.95493013,346.40583456)
\curveto(597.01492449,346.36582718)(597.07992443,346.32082722)(597.14993013,346.27083456)
\curveto(597.18992432,346.2408273)(597.24492426,346.20082734)(597.31493013,346.15083456)
\curveto(597.39492411,346.10082744)(597.46992404,346.10082744)(597.53993013,346.15083456)
\curveto(597.57992393,346.17082737)(597.59992391,346.20582734)(597.59993013,346.25583456)
\curveto(597.59992391,346.30582724)(597.6099239,346.35582719)(597.62993013,346.40583456)
\lineto(597.62993013,346.55583456)
\curveto(597.63992387,346.58582696)(597.64492386,346.62082692)(597.64493013,346.66083456)
\lineto(597.64493013,346.78083456)
\lineto(597.64493013,348.82083456)
\curveto(597.64492386,348.93082461)(597.63992387,349.05082449)(597.62993013,349.18083456)
\curveto(597.62992388,349.32082422)(597.65492385,349.42582412)(597.70493013,349.49583456)
\curveto(597.74492376,349.57582397)(597.81992369,349.62582392)(597.92993013,349.64583456)
\curveto(597.94992356,349.65582389)(597.96992354,349.65582389)(597.98993013,349.64583456)
\curveto(598.0099235,349.6458239)(598.02992348,349.65082389)(598.04993013,349.66083456)
\lineto(599.11493013,349.66083456)
\curveto(599.23492227,349.66082388)(599.34492216,349.65582389)(599.44493013,349.64583456)
\curveto(599.54492196,349.63582391)(599.61992189,349.59582395)(599.66993013,349.52583456)
\curveto(599.71992179,349.4458241)(599.74492176,349.3408242)(599.74493013,349.21083456)
\lineto(599.74493013,348.85083456)
\lineto(599.74493013,339.83583456)
\moveto(597.70493013,342.77583456)
\curveto(597.71492379,342.81583073)(597.71492379,342.85583069)(597.70493013,342.89583456)
\lineto(597.70493013,343.03083456)
\curveto(597.7049238,343.13083041)(597.69992381,343.23083031)(597.68993013,343.33083456)
\curveto(597.67992383,343.43083011)(597.66492384,343.52083002)(597.64493013,343.60083456)
\curveto(597.62492388,343.71082983)(597.6049239,343.81082973)(597.58493013,343.90083456)
\curveto(597.57492393,343.99082955)(597.54992396,344.07582947)(597.50993013,344.15583456)
\curveto(597.36992414,344.51582903)(597.16492434,344.80082874)(596.89493013,345.01083456)
\curveto(596.63492487,345.22082832)(596.25492525,345.32582822)(595.75493013,345.32583456)
\curveto(595.69492581,345.32582822)(595.61492589,345.31582823)(595.51493013,345.29583456)
\curveto(595.43492607,345.27582827)(595.35992615,345.25582829)(595.28993013,345.23583456)
\curveto(595.22992628,345.22582832)(595.16992634,345.20582834)(595.10993013,345.17583456)
\curveto(594.83992667,345.06582848)(594.62992688,344.89582865)(594.47993013,344.66583456)
\curveto(594.32992718,344.43582911)(594.2099273,344.17582937)(594.11993013,343.88583456)
\curveto(594.08992742,343.78582976)(594.06992744,343.68582986)(594.05993013,343.58583456)
\curveto(594.04992746,343.48583006)(594.02992748,343.38083016)(593.99993013,343.27083456)
\lineto(593.99993013,343.06083456)
\curveto(593.97992753,342.97083057)(593.97492753,342.8458307)(593.98493013,342.68583456)
\curveto(593.99492751,342.53583101)(594.0099275,342.42583112)(594.02993013,342.35583456)
\lineto(594.02993013,342.26583456)
\curveto(594.03992747,342.2458313)(594.04492746,342.22583132)(594.04493013,342.20583456)
\curveto(594.06492744,342.12583142)(594.07992743,342.05083149)(594.08993013,341.98083456)
\curveto(594.1099274,341.91083163)(594.12992738,341.83583171)(594.14993013,341.75583456)
\curveto(594.31992719,341.23583231)(594.6099269,340.85083269)(595.01993013,340.60083456)
\curveto(595.14992636,340.51083303)(595.32992618,340.4408331)(595.55993013,340.39083456)
\curveto(595.59992591,340.38083316)(595.65992585,340.37583317)(595.73993013,340.37583456)
\curveto(595.76992574,340.36583318)(595.81492569,340.35583319)(595.87493013,340.34583456)
\curveto(595.94492556,340.3458332)(595.99992551,340.35083319)(596.03993013,340.36083456)
\curveto(596.11992539,340.38083316)(596.19992531,340.39583315)(596.27993013,340.40583456)
\curveto(596.35992515,340.41583313)(596.43992507,340.43583311)(596.51993013,340.46583456)
\curveto(596.76992474,340.57583297)(596.96992454,340.71583283)(597.11993013,340.88583456)
\curveto(597.26992424,341.05583249)(597.39992411,341.27083227)(597.50993013,341.53083456)
\curveto(597.54992396,341.62083192)(597.57992393,341.71083183)(597.59993013,341.80083456)
\curveto(597.61992389,341.90083164)(597.63992387,342.00583154)(597.65993013,342.11583456)
\curveto(597.66992384,342.16583138)(597.66992384,342.21083133)(597.65993013,342.25083456)
\curveto(597.65992385,342.30083124)(597.66992384,342.35083119)(597.68993013,342.40083456)
\curveto(597.69992381,342.43083111)(597.7049238,342.46583108)(597.70493013,342.50583456)
\lineto(597.70493013,342.64083456)
\lineto(597.70493013,342.77583456)
}
}
{
\newrgbcolor{curcolor}{0.90196079 0.90196079 0.90196079}
\pscustom[linestyle=none,fillstyle=solid,fillcolor=curcolor]
{
\newpath
\moveto(604.87146469,311.68518455)
\curveto(618.18443631,311.68518429)(631.43568874,309.87641892)(644.26164977,306.30851521)
\lineto(604.87146181,164.70799644)
\closepath
}
}
{
\newrgbcolor{curcolor}{0.50196081 0.50196081 0.50196081}
\pscustom[linestyle=none,fillstyle=solid,fillcolor=curcolor]
{
\newpath
\moveto(644.17091824,306.33372344)
\curveto(683.15464853,295.51620149)(715.95559718,269.10799023)(734.84486692,233.33166429)
\lineto(604.87146181,164.70799644)
\closepath
}
}
{
\newrgbcolor{curcolor}{0.40000001 0.40000001 0.40000001}
\pscustom[linestyle=none,fillstyle=solid,fillcolor=curcolor]
{
\newpath
\moveto(734.7311168,233.5466762)
\curveto(772.74966983,161.82716913)(745.42964864,72.86689448)(673.71014157,34.84834145)
\curveto(601.99063451,-3.17021158)(513.03035986,24.14980961)(475.01180682,95.86931667)
\curveto(436.99325379,167.58882374)(464.31327498,256.54909839)(536.03278205,294.56765142)
\curveto(557.23730521,305.80818167)(580.87180866,311.68517764)(604.87141955,311.68518455)
\lineto(604.87146181,164.70799644)
\closepath
}
}
{
\newrgbcolor{curcolor}{0 0 0}
\pscustom[linestyle=none,fillstyle=solid,fillcolor=curcolor]
{
\newpath
\moveto(267.97718721,630.26215414)
\curveto(268.00717949,630.14214993)(268.03217946,630.00215007)(268.05218721,629.84215414)
\curveto(268.07217942,629.68215039)(268.08217941,629.51715055)(268.08218721,629.34715414)
\curveto(268.08217941,629.17715089)(268.07217942,629.01215106)(268.05218721,628.85215414)
\curveto(268.03217946,628.69215138)(268.00717949,628.55215152)(267.97718721,628.43215414)
\curveto(267.93717956,628.29215178)(267.90217959,628.1671519)(267.87218721,628.05715414)
\curveto(267.84217965,627.94715212)(267.80217969,627.83715223)(267.75218721,627.72715414)
\curveto(267.48218001,627.08715298)(267.06718043,626.60215347)(266.50718721,626.27215414)
\curveto(266.42718107,626.21215386)(266.34218115,626.16215391)(266.25218721,626.12215414)
\curveto(266.16218133,626.09215398)(266.06218143,626.05715401)(265.95218721,626.01715414)
\curveto(265.84218165,625.9671541)(265.72218177,625.93215414)(265.59218721,625.91215414)
\curveto(265.47218202,625.88215419)(265.34218215,625.85215422)(265.20218721,625.82215414)
\curveto(265.14218235,625.80215427)(265.08218241,625.79715427)(265.02218721,625.80715414)
\curveto(264.97218252,625.81715425)(264.91218258,625.81215426)(264.84218721,625.79215414)
\curveto(264.82218267,625.78215429)(264.7971827,625.78215429)(264.76718721,625.79215414)
\curveto(264.73718276,625.79215428)(264.71218278,625.78715428)(264.69218721,625.77715414)
\lineto(264.54218721,625.77715414)
\curveto(264.47218302,625.7671543)(264.42218307,625.7671543)(264.39218721,625.77715414)
\curveto(264.35218314,625.78715428)(264.30718319,625.79215428)(264.25718721,625.79215414)
\curveto(264.21718328,625.78215429)(264.17718332,625.78215429)(264.13718721,625.79215414)
\curveto(264.04718345,625.81215426)(263.95718354,625.82715424)(263.86718721,625.83715414)
\curveto(263.77718372,625.83715423)(263.68718381,625.84715422)(263.59718721,625.86715414)
\curveto(263.50718399,625.89715417)(263.41718408,625.92215415)(263.32718721,625.94215414)
\curveto(263.23718426,625.96215411)(263.15218434,625.99215408)(263.07218721,626.03215414)
\curveto(262.83218466,626.14215393)(262.60718489,626.2721538)(262.39718721,626.42215414)
\curveto(262.18718531,626.58215349)(262.00718549,626.76215331)(261.85718721,626.96215414)
\curveto(261.73718576,627.13215294)(261.63218586,627.30715276)(261.54218721,627.48715414)
\curveto(261.45218604,627.6671524)(261.36218613,627.85715221)(261.27218721,628.05715414)
\curveto(261.23218626,628.15715191)(261.1971863,628.25715181)(261.16718721,628.35715414)
\curveto(261.14718635,628.4671516)(261.12218637,628.57715149)(261.09218721,628.68715414)
\curveto(261.05218644,628.82715124)(261.02718647,628.9671511)(261.01718721,629.10715414)
\curveto(261.00718649,629.24715082)(260.98718651,629.38715068)(260.95718721,629.52715414)
\curveto(260.94718655,629.63715043)(260.93718656,629.73715033)(260.92718721,629.82715414)
\curveto(260.92718657,629.92715014)(260.91718658,630.02715004)(260.89718721,630.12715414)
\lineto(260.89718721,630.21715414)
\curveto(260.90718659,630.24714982)(260.90718659,630.2721498)(260.89718721,630.29215414)
\lineto(260.89718721,630.50215414)
\curveto(260.87718662,630.56214951)(260.86718663,630.62714944)(260.86718721,630.69715414)
\curveto(260.87718662,630.77714929)(260.88218661,630.85214922)(260.88218721,630.92215414)
\lineto(260.88218721,631.07215414)
\curveto(260.88218661,631.12214895)(260.88718661,631.1721489)(260.89718721,631.22215414)
\lineto(260.89718721,631.59715414)
\curveto(260.90718659,631.62714844)(260.90718659,631.66214841)(260.89718721,631.70215414)
\curveto(260.8971866,631.74214833)(260.90218659,631.78214829)(260.91218721,631.82215414)
\curveto(260.93218656,631.93214814)(260.94718655,632.04214803)(260.95718721,632.15215414)
\curveto(260.96718653,632.2721478)(260.97718652,632.38714768)(260.98718721,632.49715414)
\curveto(261.02718647,632.64714742)(261.05218644,632.79214728)(261.06218721,632.93215414)
\curveto(261.08218641,633.08214699)(261.11218638,633.22714684)(261.15218721,633.36715414)
\curveto(261.24218625,633.6671464)(261.33718616,633.95214612)(261.43718721,634.22215414)
\curveto(261.53718596,634.49214558)(261.66218583,634.74214533)(261.81218721,634.97215414)
\curveto(262.01218548,635.29214478)(262.25718524,635.5721445)(262.54718721,635.81215414)
\curveto(262.83718466,636.05214402)(263.17718432,636.23714383)(263.56718721,636.36715414)
\curveto(263.67718382,636.40714366)(263.78718371,636.43214364)(263.89718721,636.44215414)
\curveto(264.01718348,636.46214361)(264.13718336,636.48714358)(264.25718721,636.51715414)
\curveto(264.32718317,636.52714354)(264.3921831,636.53214354)(264.45218721,636.53215414)
\curveto(264.51218298,636.53214354)(264.57718292,636.53714353)(264.64718721,636.54715414)
\curveto(265.34718215,636.5671435)(265.92218157,636.45214362)(266.37218721,636.20215414)
\curveto(266.82218067,635.95214412)(267.16718033,635.60214447)(267.40718721,635.15215414)
\curveto(267.51717998,634.92214515)(267.61717988,634.64714542)(267.70718721,634.32715414)
\curveto(267.72717977,634.25714581)(267.72717977,634.18214589)(267.70718721,634.10215414)
\curveto(267.6971798,634.03214604)(267.67217982,633.98214609)(267.63218721,633.95215414)
\curveto(267.60217989,633.92214615)(267.54217995,633.89714617)(267.45218721,633.87715414)
\curveto(267.36218013,633.8671462)(267.26218023,633.85714621)(267.15218721,633.84715414)
\curveto(267.05218044,633.84714622)(266.95218054,633.85214622)(266.85218721,633.86215414)
\curveto(266.76218073,633.8721462)(266.6971808,633.89214618)(266.65718721,633.92215414)
\curveto(266.54718095,633.99214608)(266.46718103,634.10214597)(266.41718721,634.25215414)
\curveto(266.37718112,634.40214567)(266.32218117,634.53214554)(266.25218721,634.64215414)
\curveto(266.06218143,634.95214512)(265.78218171,635.18214489)(265.41218721,635.33215414)
\curveto(265.34218215,635.36214471)(265.26718223,635.38214469)(265.18718721,635.39215414)
\curveto(265.11718238,635.40214467)(265.04218245,635.41714465)(264.96218721,635.43715414)
\curveto(264.91218258,635.44714462)(264.84218265,635.45214462)(264.75218721,635.45215414)
\curveto(264.67218282,635.45214462)(264.60718289,635.44714462)(264.55718721,635.43715414)
\curveto(264.51718298,635.41714465)(264.48218301,635.41214466)(264.45218721,635.42215414)
\curveto(264.42218307,635.43214464)(264.38718311,635.43214464)(264.34718721,635.42215414)
\lineto(264.10718721,635.36215414)
\curveto(264.03718346,635.34214473)(263.96718353,635.31714475)(263.89718721,635.28715414)
\curveto(263.51718398,635.12714494)(263.22718427,634.91714515)(263.02718721,634.65715414)
\curveto(262.83718466,634.39714567)(262.66218483,634.08214599)(262.50218721,633.71215414)
\curveto(262.47218502,633.63214644)(262.44718505,633.55214652)(262.42718721,633.47215414)
\curveto(262.41718508,633.39214668)(262.3971851,633.31214676)(262.36718721,633.23215414)
\curveto(262.33718516,633.12214695)(262.31218518,633.00714706)(262.29218721,632.88715414)
\curveto(262.28218521,632.7671473)(262.26218523,632.64714742)(262.23218721,632.52715414)
\curveto(262.21218528,632.47714759)(262.20218529,632.42714764)(262.20218721,632.37715414)
\curveto(262.21218528,632.32714774)(262.20718529,632.27714779)(262.18718721,632.22715414)
\curveto(262.17718532,632.1671479)(262.17718532,632.08714798)(262.18718721,631.98715414)
\curveto(262.1971853,631.89714817)(262.21218528,631.84214823)(262.23218721,631.82215414)
\curveto(262.25218524,631.78214829)(262.28218521,631.76214831)(262.32218721,631.76215414)
\curveto(262.37218512,631.76214831)(262.41718508,631.7721483)(262.45718721,631.79215414)
\curveto(262.52718497,631.83214824)(262.58718491,631.87714819)(262.63718721,631.92715414)
\curveto(262.68718481,631.97714809)(262.74718475,632.02714804)(262.81718721,632.07715414)
\lineto(262.87718721,632.13715414)
\curveto(262.90718459,632.1671479)(262.93718456,632.19214788)(262.96718721,632.21215414)
\curveto(263.1971843,632.3721477)(263.47218402,632.50714756)(263.79218721,632.61715414)
\curveto(263.86218363,632.63714743)(263.93218356,632.65214742)(264.00218721,632.66215414)
\curveto(264.07218342,632.6721474)(264.14718335,632.68714738)(264.22718721,632.70715414)
\curveto(264.26718323,632.70714736)(264.30218319,632.71214736)(264.33218721,632.72215414)
\curveto(264.36218313,632.73214734)(264.3971831,632.73214734)(264.43718721,632.72215414)
\curveto(264.48718301,632.72214735)(264.52718297,632.73214734)(264.55718721,632.75215414)
\lineto(264.72218721,632.75215414)
\lineto(264.81218721,632.75215414)
\curveto(264.86218263,632.76214731)(264.90218259,632.76214731)(264.93218721,632.75215414)
\curveto(264.98218251,632.74214733)(265.03218246,632.73714733)(265.08218721,632.73715414)
\curveto(265.14218235,632.74714732)(265.1971823,632.74714732)(265.24718721,632.73715414)
\curveto(265.35718214,632.70714736)(265.46218203,632.68714738)(265.56218721,632.67715414)
\curveto(265.67218182,632.6671474)(265.77718172,632.64214743)(265.87718721,632.60215414)
\curveto(266.2971812,632.46214761)(266.64218085,632.27714779)(266.91218721,632.04715414)
\curveto(267.18218031,631.82714824)(267.42218007,631.54214853)(267.63218721,631.19215414)
\curveto(267.71217978,631.05214902)(267.77717972,630.90214917)(267.82718721,630.74215414)
\curveto(267.87717962,630.59214948)(267.92717957,630.43214964)(267.97718721,630.26215414)
\moveto(266.73218721,628.95715414)
\curveto(266.74218075,629.00715106)(266.74718075,629.05215102)(266.74718721,629.09215414)
\lineto(266.74718721,629.24215414)
\curveto(266.74718075,629.55215052)(266.70718079,629.83715023)(266.62718721,630.09715414)
\curveto(266.60718089,630.15714991)(266.58718091,630.21214986)(266.56718721,630.26215414)
\curveto(266.55718094,630.32214975)(266.54218095,630.37714969)(266.52218721,630.42715414)
\curveto(266.30218119,630.91714915)(265.95718154,631.2671488)(265.48718721,631.47715414)
\curveto(265.40718209,631.50714856)(265.32718217,631.53214854)(265.24718721,631.55215414)
\lineto(265.00718721,631.61215414)
\curveto(264.92718257,631.63214844)(264.83718266,631.64214843)(264.73718721,631.64215414)
\lineto(264.42218721,631.64215414)
\curveto(264.40218309,631.62214845)(264.36218313,631.61214846)(264.30218721,631.61215414)
\curveto(264.25218324,631.62214845)(264.20718329,631.62214845)(264.16718721,631.61215414)
\lineto(263.92718721,631.55215414)
\curveto(263.85718364,631.54214853)(263.78718371,631.52214855)(263.71718721,631.49215414)
\curveto(263.11718438,631.23214884)(262.71218478,630.7671493)(262.50218721,630.09715414)
\curveto(262.47218502,630.01715005)(262.45218504,629.93715013)(262.44218721,629.85715414)
\curveto(262.43218506,629.77715029)(262.41718508,629.69215038)(262.39718721,629.60215414)
\lineto(262.39718721,629.45215414)
\curveto(262.38718511,629.41215066)(262.38218511,629.34215073)(262.38218721,629.24215414)
\curveto(262.38218511,629.01215106)(262.40218509,628.81715125)(262.44218721,628.65715414)
\curveto(262.46218503,628.58715148)(262.47718502,628.52215155)(262.48718721,628.46215414)
\curveto(262.497185,628.40215167)(262.51718498,628.33715173)(262.54718721,628.26715414)
\curveto(262.65718484,627.98715208)(262.80218469,627.74215233)(262.98218721,627.53215414)
\curveto(263.16218433,627.33215274)(263.3971841,627.1721529)(263.68718721,627.05215414)
\lineto(263.92718721,626.96215414)
\lineto(264.16718721,626.90215414)
\curveto(264.21718328,626.88215319)(264.25718324,626.87715319)(264.28718721,626.88715414)
\curveto(264.32718317,626.89715317)(264.37218312,626.89215318)(264.42218721,626.87215414)
\curveto(264.45218304,626.86215321)(264.50718299,626.85715321)(264.58718721,626.85715414)
\curveto(264.66718283,626.85715321)(264.72718277,626.86215321)(264.76718721,626.87215414)
\curveto(264.87718262,626.89215318)(264.98218251,626.90715316)(265.08218721,626.91715414)
\curveto(265.18218231,626.92715314)(265.27718222,626.95715311)(265.36718721,627.00715414)
\curveto(265.8971816,627.20715286)(266.28718121,627.58215249)(266.53718721,628.13215414)
\curveto(266.57718092,628.23215184)(266.60718089,628.33715173)(266.62718721,628.44715414)
\lineto(266.71718721,628.77715414)
\curveto(266.71718078,628.85715121)(266.72218077,628.91715115)(266.73218721,628.95715414)
}
}
{
\newrgbcolor{curcolor}{0 0 0}
\pscustom[linestyle=none,fillstyle=solid,fillcolor=curcolor]
{
\newpath
\moveto(270.28179659,627.57715414)
\lineto(270.58179659,627.57715414)
\curveto(270.69179453,627.58715248)(270.79679442,627.58715248)(270.89679659,627.57715414)
\curveto(271.00679421,627.57715249)(271.10679411,627.5671525)(271.19679659,627.54715414)
\curveto(271.28679393,627.53715253)(271.35679386,627.51215256)(271.40679659,627.47215414)
\curveto(271.42679379,627.45215262)(271.44179378,627.42215265)(271.45179659,627.38215414)
\curveto(271.47179375,627.34215273)(271.49179373,627.29715277)(271.51179659,627.24715414)
\lineto(271.51179659,627.17215414)
\curveto(271.5217937,627.12215295)(271.5217937,627.067153)(271.51179659,627.00715414)
\lineto(271.51179659,626.85715414)
\lineto(271.51179659,626.37715414)
\curveto(271.51179371,626.20715386)(271.47179375,626.08715398)(271.39179659,626.01715414)
\curveto(271.3217939,625.9671541)(271.23179399,625.94215413)(271.12179659,625.94215414)
\lineto(270.79179659,625.94215414)
\lineto(270.34179659,625.94215414)
\curveto(270.19179503,625.94215413)(270.07679514,625.9721541)(269.99679659,626.03215414)
\curveto(269.95679526,626.06215401)(269.92679529,626.11215396)(269.90679659,626.18215414)
\curveto(269.88679533,626.26215381)(269.87179535,626.34715372)(269.86179659,626.43715414)
\lineto(269.86179659,626.72215414)
\curveto(269.87179535,626.82215325)(269.87679534,626.90715316)(269.87679659,626.97715414)
\lineto(269.87679659,627.17215414)
\curveto(269.87679534,627.23215284)(269.88679533,627.28715278)(269.90679659,627.33715414)
\curveto(269.94679527,627.44715262)(270.0167952,627.51715255)(270.11679659,627.54715414)
\curveto(270.14679507,627.54715252)(270.20179502,627.55715251)(270.28179659,627.57715414)
}
}
{
\newrgbcolor{curcolor}{0 0 0}
\pscustom[linestyle=none,fillstyle=solid,fillcolor=curcolor]
{
\newpath
\moveto(273.94695284,636.35215414)
\lineto(278.74695284,636.35215414)
\lineto(279.75195284,636.35215414)
\curveto(279.89194574,636.35214372)(280.01194562,636.34214373)(280.11195284,636.32215414)
\curveto(280.22194541,636.31214376)(280.30194533,636.2671438)(280.35195284,636.18715414)
\curveto(280.37194526,636.14714392)(280.38194525,636.09714397)(280.38195284,636.03715414)
\curveto(280.39194524,635.97714409)(280.39694523,635.91214416)(280.39695284,635.84215414)
\lineto(280.39695284,635.57215414)
\curveto(280.39694523,635.48214459)(280.38694524,635.40214467)(280.36695284,635.33215414)
\curveto(280.3269453,635.25214482)(280.28194535,635.18214489)(280.23195284,635.12215414)
\lineto(280.08195284,634.94215414)
\curveto(280.05194558,634.89214518)(280.01694561,634.85214522)(279.97695284,634.82215414)
\curveto(279.93694569,634.79214528)(279.89694573,634.75214532)(279.85695284,634.70215414)
\curveto(279.77694585,634.59214548)(279.69194594,634.48214559)(279.60195284,634.37215414)
\curveto(279.51194612,634.2721458)(279.4269462,634.1671459)(279.34695284,634.05715414)
\curveto(279.20694642,633.85714621)(279.06694656,633.64714642)(278.92695284,633.42715414)
\curveto(278.78694684,633.21714685)(278.64694698,633.00214707)(278.50695284,632.78215414)
\curveto(278.45694717,632.69214738)(278.40694722,632.59714747)(278.35695284,632.49715414)
\curveto(278.30694732,632.39714767)(278.25194738,632.30214777)(278.19195284,632.21215414)
\curveto(278.17194746,632.19214788)(278.16194747,632.1671479)(278.16195284,632.13715414)
\curveto(278.16194747,632.10714796)(278.15194748,632.08214799)(278.13195284,632.06215414)
\curveto(278.06194757,631.96214811)(277.99694763,631.84714822)(277.93695284,631.71715414)
\curveto(277.87694775,631.59714847)(277.82194781,631.48214859)(277.77195284,631.37215414)
\curveto(277.67194796,631.14214893)(277.57694805,630.90714916)(277.48695284,630.66715414)
\curveto(277.39694823,630.42714964)(277.29694833,630.18714988)(277.18695284,629.94715414)
\curveto(277.16694846,629.89715017)(277.15194848,629.85215022)(277.14195284,629.81215414)
\curveto(277.14194849,629.7721503)(277.1319485,629.72715034)(277.11195284,629.67715414)
\curveto(277.06194857,629.55715051)(277.01694861,629.43215064)(276.97695284,629.30215414)
\curveto(276.94694868,629.18215089)(276.91194872,629.06215101)(276.87195284,628.94215414)
\curveto(276.79194884,628.71215136)(276.7269489,628.4721516)(276.67695284,628.22215414)
\curveto(276.63694899,627.98215209)(276.58694904,627.74215233)(276.52695284,627.50215414)
\curveto(276.48694914,627.35215272)(276.46194917,627.20215287)(276.45195284,627.05215414)
\curveto(276.44194919,626.90215317)(276.42194921,626.75215332)(276.39195284,626.60215414)
\curveto(276.38194925,626.56215351)(276.37694925,626.50215357)(276.37695284,626.42215414)
\curveto(276.34694928,626.30215377)(276.31694931,626.20215387)(276.28695284,626.12215414)
\curveto(276.25694937,626.04215403)(276.18694944,625.98715408)(276.07695284,625.95715414)
\curveto(276.0269496,625.93715413)(275.97194966,625.92715414)(275.91195284,625.92715414)
\lineto(275.71695284,625.92715414)
\curveto(275.57695005,625.92715414)(275.43695019,625.93215414)(275.29695284,625.94215414)
\curveto(275.16695046,625.95215412)(275.07195056,625.99715407)(275.01195284,626.07715414)
\curveto(274.97195066,626.13715393)(274.95195068,626.22215385)(274.95195284,626.33215414)
\curveto(274.96195067,626.44215363)(274.97695065,626.53715353)(274.99695284,626.61715414)
\lineto(274.99695284,626.69215414)
\curveto(275.00695062,626.72215335)(275.01195062,626.75215332)(275.01195284,626.78215414)
\curveto(275.0319506,626.86215321)(275.04195059,626.93715313)(275.04195284,627.00715414)
\curveto(275.04195059,627.07715299)(275.05195058,627.14715292)(275.07195284,627.21715414)
\curveto(275.12195051,627.40715266)(275.16195047,627.59215248)(275.19195284,627.77215414)
\curveto(275.22195041,627.96215211)(275.26195037,628.14215193)(275.31195284,628.31215414)
\curveto(275.3319503,628.36215171)(275.34195029,628.40215167)(275.34195284,628.43215414)
\curveto(275.34195029,628.46215161)(275.34695028,628.49715157)(275.35695284,628.53715414)
\curveto(275.45695017,628.83715123)(275.54695008,629.13215094)(275.62695284,629.42215414)
\curveto(275.71694991,629.71215036)(275.82194981,629.99215008)(275.94195284,630.26215414)
\curveto(276.20194943,630.84214923)(276.47194916,631.39214868)(276.75195284,631.91215414)
\curveto(277.0319486,632.44214763)(277.34194829,632.94714712)(277.68195284,633.42715414)
\curveto(277.82194781,633.62714644)(277.97194766,633.81714625)(278.13195284,633.99715414)
\curveto(278.29194734,634.18714588)(278.44194719,634.37714569)(278.58195284,634.56715414)
\curveto(278.62194701,634.61714545)(278.65694697,634.66214541)(278.68695284,634.70215414)
\curveto(278.7269469,634.75214532)(278.76194687,634.80214527)(278.79195284,634.85215414)
\curveto(278.80194683,634.8721452)(278.81194682,634.89714517)(278.82195284,634.92715414)
\curveto(278.84194679,634.95714511)(278.84194679,634.98714508)(278.82195284,635.01715414)
\curveto(278.80194683,635.07714499)(278.76694686,635.11214496)(278.71695284,635.12215414)
\curveto(278.66694696,635.14214493)(278.61694701,635.16214491)(278.56695284,635.18215414)
\lineto(278.46195284,635.18215414)
\curveto(278.42194721,635.19214488)(278.37194726,635.19214488)(278.31195284,635.18215414)
\lineto(278.16195284,635.18215414)
\lineto(277.56195284,635.18215414)
\lineto(274.92195284,635.18215414)
\lineto(274.18695284,635.18215414)
\lineto(273.94695284,635.18215414)
\curveto(273.87695175,635.19214488)(273.81695181,635.20714486)(273.76695284,635.22715414)
\curveto(273.67695195,635.2671448)(273.61695201,635.32714474)(273.58695284,635.40715414)
\curveto(273.53695209,635.50714456)(273.52195211,635.65214442)(273.54195284,635.84215414)
\curveto(273.56195207,636.04214403)(273.59695203,636.17714389)(273.64695284,636.24715414)
\curveto(273.66695196,636.2671438)(273.69195194,636.28214379)(273.72195284,636.29215414)
\lineto(273.84195284,636.35215414)
\curveto(273.86195177,636.35214372)(273.87695175,636.34714372)(273.88695284,636.33715414)
\curveto(273.90695172,636.33714373)(273.9269517,636.34214373)(273.94695284,636.35215414)
}
}
{
\newrgbcolor{curcolor}{0 0 0}
\pscustom[linestyle=none,fillstyle=solid,fillcolor=curcolor]
{
\newpath
\moveto(291.64156221,634.46215414)
\curveto(291.44155191,634.1721459)(291.23155212,633.88714618)(291.01156221,633.60715414)
\curveto(290.80155255,633.32714674)(290.59655276,633.04214703)(290.39656221,632.75215414)
\curveto(289.79655356,631.90214817)(289.19155416,631.06214901)(288.58156221,630.23215414)
\curveto(287.97155538,629.41215066)(287.36655599,628.57715149)(286.76656221,627.72715414)
\lineto(286.25656221,627.00715414)
\lineto(285.74656221,626.31715414)
\curveto(285.66655769,626.20715386)(285.58655777,626.09215398)(285.50656221,625.97215414)
\curveto(285.42655793,625.85215422)(285.33155802,625.75715431)(285.22156221,625.68715414)
\curveto(285.18155817,625.6671544)(285.11655824,625.65215442)(285.02656221,625.64215414)
\curveto(284.94655841,625.62215445)(284.8565585,625.61215446)(284.75656221,625.61215414)
\curveto(284.6565587,625.61215446)(284.56155879,625.61715445)(284.47156221,625.62715414)
\curveto(284.39155896,625.63715443)(284.33155902,625.65715441)(284.29156221,625.68715414)
\curveto(284.26155909,625.70715436)(284.23655912,625.74215433)(284.21656221,625.79215414)
\curveto(284.20655915,625.83215424)(284.21155914,625.87715419)(284.23156221,625.92715414)
\curveto(284.27155908,626.00715406)(284.31655904,626.08215399)(284.36656221,626.15215414)
\curveto(284.42655893,626.23215384)(284.48155887,626.31215376)(284.53156221,626.39215414)
\curveto(284.77155858,626.73215334)(285.01655834,627.067153)(285.26656221,627.39715414)
\curveto(285.51655784,627.72715234)(285.7565576,628.06215201)(285.98656221,628.40215414)
\curveto(286.14655721,628.62215145)(286.30655705,628.83715123)(286.46656221,629.04715414)
\curveto(286.62655673,629.25715081)(286.78655657,629.4721506)(286.94656221,629.69215414)
\curveto(287.30655605,630.21214986)(287.67155568,630.72214935)(288.04156221,631.22215414)
\curveto(288.41155494,631.72214835)(288.78155457,632.23214784)(289.15156221,632.75215414)
\curveto(289.29155406,632.95214712)(289.43155392,633.14714692)(289.57156221,633.33715414)
\curveto(289.72155363,633.52714654)(289.86655349,633.72214635)(290.00656221,633.92215414)
\curveto(290.21655314,634.22214585)(290.43155292,634.52214555)(290.65156221,634.82215414)
\lineto(291.31156221,635.72215414)
\lineto(291.49156221,635.99215414)
\lineto(291.70156221,636.26215414)
\lineto(291.82156221,636.44215414)
\curveto(291.87155148,636.50214357)(291.92155143,636.55714351)(291.97156221,636.60715414)
\curveto(292.04155131,636.65714341)(292.11655124,636.69214338)(292.19656221,636.71215414)
\curveto(292.21655114,636.72214335)(292.24155111,636.72214335)(292.27156221,636.71215414)
\curveto(292.31155104,636.71214336)(292.34155101,636.72214335)(292.36156221,636.74215414)
\curveto(292.48155087,636.74214333)(292.61655074,636.73714333)(292.76656221,636.72715414)
\curveto(292.91655044,636.72714334)(293.00655035,636.68214339)(293.03656221,636.59215414)
\curveto(293.0565503,636.56214351)(293.06155029,636.52714354)(293.05156221,636.48715414)
\curveto(293.04155031,636.44714362)(293.02655033,636.41714365)(293.00656221,636.39715414)
\curveto(292.96655039,636.31714375)(292.92655043,636.24714382)(292.88656221,636.18715414)
\curveto(292.84655051,636.12714394)(292.80155055,636.067144)(292.75156221,636.00715414)
\lineto(292.18156221,635.22715414)
\curveto(292.00155135,634.97714509)(291.82155153,634.72214535)(291.64156221,634.46215414)
\moveto(284.78656221,630.56215414)
\curveto(284.73655862,630.58214949)(284.68655867,630.58714948)(284.63656221,630.57715414)
\curveto(284.58655877,630.5671495)(284.53655882,630.5721495)(284.48656221,630.59215414)
\curveto(284.37655898,630.61214946)(284.27155908,630.63214944)(284.17156221,630.65215414)
\curveto(284.08155927,630.68214939)(283.98655937,630.72214935)(283.88656221,630.77215414)
\curveto(283.5565598,630.91214916)(283.30156005,631.10714896)(283.12156221,631.35715414)
\curveto(282.94156041,631.61714845)(282.79656056,631.92714814)(282.68656221,632.28715414)
\curveto(282.6565607,632.3671477)(282.63656072,632.44714762)(282.62656221,632.52715414)
\curveto(282.61656074,632.61714745)(282.60156075,632.70214737)(282.58156221,632.78215414)
\curveto(282.57156078,632.83214724)(282.56656079,632.89714717)(282.56656221,632.97715414)
\curveto(282.5565608,633.00714706)(282.5515608,633.03714703)(282.55156221,633.06715414)
\curveto(282.5515608,633.10714696)(282.54656081,633.14214693)(282.53656221,633.17215414)
\lineto(282.53656221,633.32215414)
\curveto(282.52656083,633.3721467)(282.52156083,633.43214664)(282.52156221,633.50215414)
\curveto(282.52156083,633.58214649)(282.52656083,633.64714642)(282.53656221,633.69715414)
\lineto(282.53656221,633.86215414)
\curveto(282.5565608,633.91214616)(282.56156079,633.95714611)(282.55156221,633.99715414)
\curveto(282.5515608,634.04714602)(282.5565608,634.09214598)(282.56656221,634.13215414)
\curveto(282.57656078,634.1721459)(282.58156077,634.20714586)(282.58156221,634.23715414)
\curveto(282.58156077,634.27714579)(282.58656077,634.31714575)(282.59656221,634.35715414)
\curveto(282.62656073,634.4671456)(282.64656071,634.57714549)(282.65656221,634.68715414)
\curveto(282.67656068,634.80714526)(282.71156064,634.92214515)(282.76156221,635.03215414)
\curveto(282.90156045,635.3721447)(283.06156029,635.64714442)(283.24156221,635.85715414)
\curveto(283.43155992,636.07714399)(283.70155965,636.25714381)(284.05156221,636.39715414)
\curveto(284.13155922,636.42714364)(284.21655914,636.44714362)(284.30656221,636.45715414)
\curveto(284.39655896,636.47714359)(284.49155886,636.49714357)(284.59156221,636.51715414)
\curveto(284.62155873,636.52714354)(284.67655868,636.52714354)(284.75656221,636.51715414)
\curveto(284.83655852,636.51714355)(284.88655847,636.52714354)(284.90656221,636.54715414)
\curveto(285.46655789,636.55714351)(285.91655744,636.44714362)(286.25656221,636.21715414)
\curveto(286.60655675,635.98714408)(286.86655649,635.68214439)(287.03656221,635.30215414)
\curveto(287.07655628,635.21214486)(287.11155624,635.11714495)(287.14156221,635.01715414)
\curveto(287.17155618,634.91714515)(287.19655616,634.81714525)(287.21656221,634.71715414)
\curveto(287.23655612,634.68714538)(287.24155611,634.65714541)(287.23156221,634.62715414)
\curveto(287.23155612,634.59714547)(287.23655612,634.5671455)(287.24656221,634.53715414)
\curveto(287.27655608,634.42714564)(287.29655606,634.30214577)(287.30656221,634.16215414)
\curveto(287.31655604,634.03214604)(287.32655603,633.89714617)(287.33656221,633.75715414)
\lineto(287.33656221,633.59215414)
\curveto(287.34655601,633.53214654)(287.34655601,633.47714659)(287.33656221,633.42715414)
\curveto(287.32655603,633.37714669)(287.32155603,633.32714674)(287.32156221,633.27715414)
\lineto(287.32156221,633.14215414)
\curveto(287.31155604,633.10214697)(287.30655605,633.06214701)(287.30656221,633.02215414)
\curveto(287.31655604,632.98214709)(287.31155604,632.93714713)(287.29156221,632.88715414)
\curveto(287.27155608,632.77714729)(287.2515561,632.6721474)(287.23156221,632.57215414)
\curveto(287.22155613,632.4721476)(287.20155615,632.3721477)(287.17156221,632.27215414)
\curveto(287.04155631,631.91214816)(286.87655648,631.59714847)(286.67656221,631.32715414)
\curveto(286.47655688,631.05714901)(286.20155715,630.85214922)(285.85156221,630.71215414)
\curveto(285.77155758,630.68214939)(285.68655767,630.65714941)(285.59656221,630.63715414)
\lineto(285.32656221,630.57715414)
\curveto(285.27655808,630.5671495)(285.23155812,630.56214951)(285.19156221,630.56215414)
\curveto(285.1515582,630.5721495)(285.11155824,630.5721495)(285.07156221,630.56215414)
\curveto(284.97155838,630.54214953)(284.87655848,630.54214953)(284.78656221,630.56215414)
\moveto(283.94656221,631.95715414)
\curveto(283.98655937,631.88714818)(284.02655933,631.82214825)(284.06656221,631.76215414)
\curveto(284.10655925,631.71214836)(284.1565592,631.66214841)(284.21656221,631.61215414)
\lineto(284.36656221,631.49215414)
\curveto(284.42655893,631.46214861)(284.49155886,631.43714863)(284.56156221,631.41715414)
\curveto(284.60155875,631.39714867)(284.63655872,631.38714868)(284.66656221,631.38715414)
\curveto(284.70655865,631.39714867)(284.74655861,631.39214868)(284.78656221,631.37215414)
\curveto(284.81655854,631.3721487)(284.8565585,631.3671487)(284.90656221,631.35715414)
\curveto(284.9565584,631.35714871)(284.99655836,631.36214871)(285.02656221,631.37215414)
\lineto(285.25156221,631.41715414)
\curveto(285.50155785,631.49714857)(285.68655767,631.62214845)(285.80656221,631.79215414)
\curveto(285.88655747,631.89214818)(285.9565574,632.02214805)(286.01656221,632.18215414)
\curveto(286.09655726,632.36214771)(286.1565572,632.58714748)(286.19656221,632.85715414)
\curveto(286.23655712,633.13714693)(286.2515571,633.41714665)(286.24156221,633.69715414)
\curveto(286.23155712,633.98714608)(286.20155715,634.26214581)(286.15156221,634.52215414)
\curveto(286.10155725,634.78214529)(286.02655733,634.99214508)(285.92656221,635.15215414)
\curveto(285.80655755,635.35214472)(285.6565577,635.50214457)(285.47656221,635.60215414)
\curveto(285.39655796,635.65214442)(285.30655805,635.68214439)(285.20656221,635.69215414)
\curveto(285.10655825,635.71214436)(285.00155835,635.72214435)(284.89156221,635.72215414)
\curveto(284.87155848,635.71214436)(284.84655851,635.70714436)(284.81656221,635.70715414)
\curveto(284.79655856,635.71714435)(284.77655858,635.71714435)(284.75656221,635.70715414)
\curveto(284.70655865,635.69714437)(284.66155869,635.68714438)(284.62156221,635.67715414)
\curveto(284.58155877,635.67714439)(284.54155881,635.6671444)(284.50156221,635.64715414)
\curveto(284.32155903,635.5671445)(284.17155918,635.44714462)(284.05156221,635.28715414)
\curveto(283.94155941,635.12714494)(283.8515595,634.94714512)(283.78156221,634.74715414)
\curveto(283.72155963,634.55714551)(283.67655968,634.33214574)(283.64656221,634.07215414)
\curveto(283.62655973,633.81214626)(283.62155973,633.54714652)(283.63156221,633.27715414)
\curveto(283.64155971,633.01714705)(283.67155968,632.7671473)(283.72156221,632.52715414)
\curveto(283.78155957,632.29714777)(283.8565595,632.10714796)(283.94656221,631.95715414)
\moveto(294.74656221,628.97215414)
\curveto(294.7565486,628.92215115)(294.76154859,628.83215124)(294.76156221,628.70215414)
\curveto(294.76154859,628.5721515)(294.7515486,628.48215159)(294.73156221,628.43215414)
\curveto(294.71154864,628.38215169)(294.70654865,628.32715174)(294.71656221,628.26715414)
\curveto(294.72654863,628.21715185)(294.72654863,628.1671519)(294.71656221,628.11715414)
\curveto(294.67654868,627.97715209)(294.64654871,627.84215223)(294.62656221,627.71215414)
\curveto(294.61654874,627.58215249)(294.58654877,627.46215261)(294.53656221,627.35215414)
\curveto(294.39654896,627.00215307)(294.23154912,626.70715336)(294.04156221,626.46715414)
\curveto(293.8515495,626.23715383)(293.58154977,626.05215402)(293.23156221,625.91215414)
\curveto(293.1515502,625.88215419)(293.06655029,625.86215421)(292.97656221,625.85215414)
\curveto(292.88655047,625.83215424)(292.80155055,625.81215426)(292.72156221,625.79215414)
\curveto(292.67155068,625.78215429)(292.62155073,625.77715429)(292.57156221,625.77715414)
\curveto(292.52155083,625.77715429)(292.47155088,625.7721543)(292.42156221,625.76215414)
\curveto(292.39155096,625.75215432)(292.34155101,625.75215432)(292.27156221,625.76215414)
\curveto(292.20155115,625.76215431)(292.1515512,625.7671543)(292.12156221,625.77715414)
\curveto(292.06155129,625.79715427)(292.00155135,625.80715426)(291.94156221,625.80715414)
\curveto(291.89155146,625.79715427)(291.84155151,625.80215427)(291.79156221,625.82215414)
\curveto(291.70155165,625.84215423)(291.61155174,625.8671542)(291.52156221,625.89715414)
\curveto(291.44155191,625.91715415)(291.36155199,625.94715412)(291.28156221,625.98715414)
\curveto(290.96155239,626.12715394)(290.71155264,626.32215375)(290.53156221,626.57215414)
\curveto(290.351553,626.83215324)(290.20155315,627.13715293)(290.08156221,627.48715414)
\curveto(290.06155329,627.5671525)(290.04655331,627.65215242)(290.03656221,627.74215414)
\curveto(290.02655333,627.83215224)(290.01155334,627.91715215)(289.99156221,627.99715414)
\curveto(289.98155337,628.02715204)(289.97655338,628.05715201)(289.97656221,628.08715414)
\lineto(289.97656221,628.19215414)
\curveto(289.9565534,628.2721518)(289.94655341,628.35215172)(289.94656221,628.43215414)
\lineto(289.94656221,628.56715414)
\curveto(289.92655343,628.6671514)(289.92655343,628.7671513)(289.94656221,628.86715414)
\lineto(289.94656221,629.04715414)
\curveto(289.9565534,629.09715097)(289.96155339,629.14215093)(289.96156221,629.18215414)
\curveto(289.96155339,629.23215084)(289.96655339,629.27715079)(289.97656221,629.31715414)
\curveto(289.98655337,629.35715071)(289.99155336,629.39215068)(289.99156221,629.42215414)
\curveto(289.99155336,629.46215061)(289.99655336,629.50215057)(290.00656221,629.54215414)
\lineto(290.06656221,629.87215414)
\curveto(290.08655327,629.99215008)(290.11655324,630.10214997)(290.15656221,630.20215414)
\curveto(290.29655306,630.53214954)(290.4565529,630.80714926)(290.63656221,631.02715414)
\curveto(290.82655253,631.25714881)(291.08655227,631.44214863)(291.41656221,631.58215414)
\curveto(291.49655186,631.62214845)(291.58155177,631.64714842)(291.67156221,631.65715414)
\lineto(291.97156221,631.71715414)
\lineto(292.10656221,631.71715414)
\curveto(292.1565512,631.72714834)(292.20655115,631.73214834)(292.25656221,631.73215414)
\curveto(292.82655053,631.75214832)(293.28655007,631.64714842)(293.63656221,631.41715414)
\curveto(293.99654936,631.19714887)(294.26154909,630.89714917)(294.43156221,630.51715414)
\curveto(294.48154887,630.41714965)(294.52154883,630.31714975)(294.55156221,630.21715414)
\curveto(294.58154877,630.11714995)(294.61154874,630.01215006)(294.64156221,629.90215414)
\curveto(294.6515487,629.86215021)(294.6565487,629.82715024)(294.65656221,629.79715414)
\curveto(294.6565487,629.77715029)(294.66154869,629.74715032)(294.67156221,629.70715414)
\curveto(294.69154866,629.63715043)(294.70154865,629.56215051)(294.70156221,629.48215414)
\curveto(294.70154865,629.40215067)(294.71154864,629.32215075)(294.73156221,629.24215414)
\curveto(294.73154862,629.19215088)(294.73154862,629.14715092)(294.73156221,629.10715414)
\curveto(294.73154862,629.067151)(294.73654862,629.02215105)(294.74656221,628.97215414)
\moveto(293.63656221,628.53715414)
\curveto(293.64654971,628.58715148)(293.6515497,628.66215141)(293.65156221,628.76215414)
\curveto(293.66154969,628.86215121)(293.6565497,628.93715113)(293.63656221,628.98715414)
\curveto(293.61654974,629.04715102)(293.61154974,629.10215097)(293.62156221,629.15215414)
\curveto(293.64154971,629.21215086)(293.64154971,629.2721508)(293.62156221,629.33215414)
\curveto(293.61154974,629.36215071)(293.60654975,629.39715067)(293.60656221,629.43715414)
\curveto(293.60654975,629.47715059)(293.60154975,629.51715055)(293.59156221,629.55715414)
\curveto(293.57154978,629.63715043)(293.5515498,629.71215036)(293.53156221,629.78215414)
\curveto(293.52154983,629.86215021)(293.50654985,629.94215013)(293.48656221,630.02215414)
\curveto(293.4565499,630.08214999)(293.43154992,630.14214993)(293.41156221,630.20215414)
\curveto(293.39154996,630.26214981)(293.36154999,630.32214975)(293.32156221,630.38215414)
\curveto(293.22155013,630.55214952)(293.09155026,630.68714938)(292.93156221,630.78715414)
\curveto(292.8515505,630.83714923)(292.7565506,630.8721492)(292.64656221,630.89215414)
\curveto(292.53655082,630.91214916)(292.41155094,630.92214915)(292.27156221,630.92215414)
\curveto(292.2515511,630.91214916)(292.22655113,630.90714916)(292.19656221,630.90715414)
\curveto(292.16655119,630.91714915)(292.13655122,630.91714915)(292.10656221,630.90715414)
\lineto(291.95656221,630.84715414)
\curveto(291.90655145,630.83714923)(291.86155149,630.82214925)(291.82156221,630.80215414)
\curveto(291.63155172,630.69214938)(291.48655187,630.54714952)(291.38656221,630.36715414)
\curveto(291.29655206,630.18714988)(291.21655214,629.98215009)(291.14656221,629.75215414)
\curveto(291.10655225,629.62215045)(291.08655227,629.48715058)(291.08656221,629.34715414)
\curveto(291.08655227,629.21715085)(291.07655228,629.072151)(291.05656221,628.91215414)
\curveto(291.04655231,628.86215121)(291.03655232,628.80215127)(291.02656221,628.73215414)
\curveto(291.02655233,628.66215141)(291.03655232,628.60215147)(291.05656221,628.55215414)
\lineto(291.05656221,628.38715414)
\lineto(291.05656221,628.20715414)
\curveto(291.06655229,628.15715191)(291.07655228,628.10215197)(291.08656221,628.04215414)
\curveto(291.09655226,627.99215208)(291.10155225,627.93715213)(291.10156221,627.87715414)
\curveto(291.11155224,627.81715225)(291.12655223,627.76215231)(291.14656221,627.71215414)
\curveto(291.19655216,627.52215255)(291.2565521,627.34715272)(291.32656221,627.18715414)
\curveto(291.39655196,627.02715304)(291.50155185,626.89715317)(291.64156221,626.79715414)
\curveto(291.77155158,626.69715337)(291.91155144,626.62715344)(292.06156221,626.58715414)
\curveto(292.09155126,626.57715349)(292.11655124,626.5721535)(292.13656221,626.57215414)
\curveto(292.16655119,626.58215349)(292.19655116,626.58215349)(292.22656221,626.57215414)
\curveto(292.24655111,626.5721535)(292.27655108,626.5671535)(292.31656221,626.55715414)
\curveto(292.356551,626.55715351)(292.39155096,626.56215351)(292.42156221,626.57215414)
\curveto(292.46155089,626.58215349)(292.50155085,626.58715348)(292.54156221,626.58715414)
\curveto(292.58155077,626.58715348)(292.62155073,626.59715347)(292.66156221,626.61715414)
\curveto(292.90155045,626.69715337)(293.09655026,626.83215324)(293.24656221,627.02215414)
\curveto(293.36654999,627.20215287)(293.4565499,627.40715266)(293.51656221,627.63715414)
\curveto(293.53654982,627.70715236)(293.5515498,627.77715229)(293.56156221,627.84715414)
\curveto(293.57154978,627.92715214)(293.58654977,628.00715206)(293.60656221,628.08715414)
\curveto(293.60654975,628.14715192)(293.61154974,628.19215188)(293.62156221,628.22215414)
\curveto(293.62154973,628.24215183)(293.62154973,628.2671518)(293.62156221,628.29715414)
\curveto(293.62154973,628.33715173)(293.62654973,628.3671517)(293.63656221,628.38715414)
\lineto(293.63656221,628.53715414)
}
}
{
\newrgbcolor{curcolor}{0 0 0}
\pscustom[linestyle=none,fillstyle=solid,fillcolor=curcolor]
{
\newpath
\moveto(320.65574556,506.54712362)
\curveto(321.34574093,506.55711299)(321.94574033,506.43711311)(322.45574556,506.18712362)
\curveto(322.9757393,505.93711361)(323.3707389,505.60211395)(323.64074556,505.18212362)
\curveto(323.69073858,505.10211445)(323.73573854,505.01211454)(323.77574556,504.91212362)
\curveto(323.81573846,504.82211473)(323.86073841,504.72711482)(323.91074556,504.62712362)
\curveto(323.95073832,504.52711502)(323.98073829,504.42711512)(324.00074556,504.32712362)
\curveto(324.02073825,504.22711532)(324.04073823,504.12211543)(324.06074556,504.01212362)
\curveto(324.08073819,503.96211559)(324.08573819,503.91711563)(324.07574556,503.87712362)
\curveto(324.06573821,503.83711571)(324.0707382,503.79211576)(324.09074556,503.74212362)
\curveto(324.10073817,503.69211586)(324.10573817,503.60711594)(324.10574556,503.48712362)
\curveto(324.10573817,503.37711617)(324.10073817,503.29211626)(324.09074556,503.23212362)
\curveto(324.0707382,503.17211638)(324.06073821,503.11211644)(324.06074556,503.05212362)
\curveto(324.0707382,502.99211656)(324.06573821,502.93211662)(324.04574556,502.87212362)
\curveto(324.00573827,502.73211682)(323.9707383,502.59711695)(323.94074556,502.46712362)
\curveto(323.91073836,502.33711721)(323.8707384,502.21211734)(323.82074556,502.09212362)
\curveto(323.76073851,501.9521176)(323.69073858,501.82711772)(323.61074556,501.71712362)
\curveto(323.54073873,501.60711794)(323.46573881,501.49711805)(323.38574556,501.38712362)
\lineto(323.32574556,501.32712362)
\curveto(323.31573896,501.30711824)(323.30073897,501.28711826)(323.28074556,501.26712362)
\curveto(323.16073911,501.10711844)(323.02573925,500.96211859)(322.87574556,500.83212362)
\curveto(322.72573955,500.70211885)(322.56573971,500.57711897)(322.39574556,500.45712362)
\curveto(322.08574019,500.23711931)(321.79074048,500.03211952)(321.51074556,499.84212362)
\curveto(321.28074099,499.70211985)(321.05074122,499.56711998)(320.82074556,499.43712362)
\curveto(320.60074167,499.30712024)(320.38074189,499.17212038)(320.16074556,499.03212362)
\curveto(319.91074236,498.86212069)(319.6707426,498.68212087)(319.44074556,498.49212362)
\curveto(319.22074305,498.30212125)(319.03074324,498.07712147)(318.87074556,497.81712362)
\curveto(318.83074344,497.75712179)(318.79574348,497.69712185)(318.76574556,497.63712362)
\curveto(318.73574354,497.58712196)(318.70574357,497.52212203)(318.67574556,497.44212362)
\curveto(318.65574362,497.37212218)(318.65074362,497.31212224)(318.66074556,497.26212362)
\curveto(318.68074359,497.19212236)(318.71574356,497.13712241)(318.76574556,497.09712362)
\curveto(318.81574346,497.06712248)(318.8757434,497.0471225)(318.94574556,497.03712362)
\lineto(319.18574556,497.03712362)
\lineto(319.93574556,497.03712362)
\lineto(322.74074556,497.03712362)
\lineto(323.40074556,497.03712362)
\curveto(323.49073878,497.03712251)(323.5757387,497.03212252)(323.65574556,497.02212362)
\curveto(323.73573854,497.02212253)(323.80073847,497.00212255)(323.85074556,496.96212362)
\curveto(323.90073837,496.92212263)(323.94073833,496.8471227)(323.97074556,496.73712362)
\curveto(324.01073826,496.63712291)(324.02073825,496.53712301)(324.00074556,496.43712362)
\lineto(324.00074556,496.30212362)
\curveto(323.98073829,496.23212332)(323.96073831,496.17212338)(323.94074556,496.12212362)
\curveto(323.92073835,496.07212348)(323.88573839,496.03212352)(323.83574556,496.00212362)
\curveto(323.78573849,495.96212359)(323.71573856,495.94212361)(323.62574556,495.94212362)
\lineto(323.35574556,495.94212362)
\lineto(322.45574556,495.94212362)
\lineto(318.94574556,495.94212362)
\lineto(317.88074556,495.94212362)
\curveto(317.80074447,495.94212361)(317.71074456,495.93712361)(317.61074556,495.92712362)
\curveto(317.51074476,495.92712362)(317.42574485,495.93712361)(317.35574556,495.95712362)
\curveto(317.14574513,496.02712352)(317.08074519,496.20712334)(317.16074556,496.49712362)
\curveto(317.1707451,496.53712301)(317.1707451,496.57212298)(317.16074556,496.60212362)
\curveto(317.16074511,496.64212291)(317.1707451,496.68712286)(317.19074556,496.73712362)
\curveto(317.21074506,496.81712273)(317.23074504,496.90212265)(317.25074556,496.99212362)
\curveto(317.270745,497.08212247)(317.29574498,497.16712238)(317.32574556,497.24712362)
\curveto(317.48574479,497.73712181)(317.68574459,498.1521214)(317.92574556,498.49212362)
\curveto(318.10574417,498.74212081)(318.31074396,498.96712058)(318.54074556,499.16712362)
\curveto(318.7707435,499.37712017)(319.01074326,499.57211998)(319.26074556,499.75212362)
\curveto(319.52074275,499.93211962)(319.78574249,500.10211945)(320.05574556,500.26212362)
\curveto(320.33574194,500.43211912)(320.60574167,500.60711894)(320.86574556,500.78712362)
\curveto(320.9757413,500.86711868)(321.08074119,500.94211861)(321.18074556,501.01212362)
\curveto(321.29074098,501.08211847)(321.40074087,501.15711839)(321.51074556,501.23712362)
\curveto(321.55074072,501.26711828)(321.58574069,501.29711825)(321.61574556,501.32712362)
\curveto(321.65574062,501.36711818)(321.69574058,501.39711815)(321.73574556,501.41712362)
\curveto(321.8757404,501.52711802)(322.00074027,501.6521179)(322.11074556,501.79212362)
\curveto(322.13074014,501.82211773)(322.15574012,501.8471177)(322.18574556,501.86712362)
\curveto(322.21574006,501.89711765)(322.24074003,501.92711762)(322.26074556,501.95712362)
\curveto(322.34073993,502.05711749)(322.40573987,502.15711739)(322.45574556,502.25712362)
\curveto(322.51573976,502.35711719)(322.5707397,502.46711708)(322.62074556,502.58712362)
\curveto(322.65073962,502.65711689)(322.6707396,502.73211682)(322.68074556,502.81212362)
\lineto(322.74074556,503.05212362)
\lineto(322.74074556,503.14212362)
\curveto(322.75073952,503.17211638)(322.75573952,503.20211635)(322.75574556,503.23212362)
\curveto(322.7757395,503.30211625)(322.78073949,503.39711615)(322.77074556,503.51712362)
\curveto(322.7707395,503.6471159)(322.76073951,503.7471158)(322.74074556,503.81712362)
\curveto(322.72073955,503.89711565)(322.70073957,503.97211558)(322.68074556,504.04212362)
\curveto(322.6707396,504.12211543)(322.65073962,504.20211535)(322.62074556,504.28212362)
\curveto(322.51073976,504.52211503)(322.36073991,504.72211483)(322.17074556,504.88212362)
\curveto(321.99074028,505.0521145)(321.7707405,505.19211436)(321.51074556,505.30212362)
\curveto(321.44074083,505.32211423)(321.3707409,505.33711421)(321.30074556,505.34712362)
\curveto(321.23074104,505.36711418)(321.15574112,505.38711416)(321.07574556,505.40712362)
\curveto(320.99574128,505.42711412)(320.88574139,505.43711411)(320.74574556,505.43712362)
\curveto(320.61574166,505.43711411)(320.51074176,505.42711412)(320.43074556,505.40712362)
\curveto(320.3707419,505.39711415)(320.31574196,505.39211416)(320.26574556,505.39212362)
\curveto(320.21574206,505.39211416)(320.16574211,505.38211417)(320.11574556,505.36212362)
\curveto(320.01574226,505.32211423)(319.92074235,505.28211427)(319.83074556,505.24212362)
\curveto(319.75074252,505.20211435)(319.6707426,505.15711439)(319.59074556,505.10712362)
\curveto(319.56074271,505.08711446)(319.53074274,505.06211449)(319.50074556,505.03212362)
\curveto(319.48074279,505.00211455)(319.45574282,504.97711457)(319.42574556,504.95712362)
\lineto(319.35074556,504.88212362)
\curveto(319.32074295,504.86211469)(319.29574298,504.84211471)(319.27574556,504.82212362)
\lineto(319.12574556,504.61212362)
\curveto(319.08574319,504.552115)(319.04074323,504.48711506)(318.99074556,504.41712362)
\curveto(318.93074334,504.32711522)(318.88074339,504.22211533)(318.84074556,504.10212362)
\curveto(318.81074346,503.99211556)(318.7757435,503.88211567)(318.73574556,503.77212362)
\curveto(318.69574358,503.66211589)(318.6707436,503.51711603)(318.66074556,503.33712362)
\curveto(318.65074362,503.16711638)(318.62074365,503.04211651)(318.57074556,502.96212362)
\curveto(318.52074375,502.88211667)(318.44574383,502.83711671)(318.34574556,502.82712362)
\curveto(318.24574403,502.81711673)(318.13574414,502.81211674)(318.01574556,502.81212362)
\curveto(317.9757443,502.81211674)(317.93574434,502.80711674)(317.89574556,502.79712362)
\curveto(317.85574442,502.79711675)(317.82074445,502.80211675)(317.79074556,502.81212362)
\curveto(317.74074453,502.83211672)(317.69074458,502.84211671)(317.64074556,502.84212362)
\curveto(317.60074467,502.84211671)(317.56074471,502.8521167)(317.52074556,502.87212362)
\curveto(317.43074484,502.93211662)(317.38574489,503.06711648)(317.38574556,503.27712362)
\lineto(317.38574556,503.39712362)
\curveto(317.39574488,503.45711609)(317.40074487,503.51711603)(317.40074556,503.57712362)
\curveto(317.41074486,503.6471159)(317.42074485,503.71211584)(317.43074556,503.77212362)
\curveto(317.45074482,503.88211567)(317.4707448,503.98211557)(317.49074556,504.07212362)
\curveto(317.51074476,504.17211538)(317.54074473,504.26711528)(317.58074556,504.35712362)
\curveto(317.60074467,504.42711512)(317.62074465,504.48711506)(317.64074556,504.53712362)
\lineto(317.70074556,504.71712362)
\curveto(317.82074445,504.97711457)(317.9757443,505.22211433)(318.16574556,505.45212362)
\curveto(318.36574391,505.68211387)(318.58074369,505.86711368)(318.81074556,506.00712362)
\curveto(318.92074335,506.08711346)(319.03574324,506.1521134)(319.15574556,506.20212362)
\lineto(319.54574556,506.35212362)
\curveto(319.65574262,506.40211315)(319.7707425,506.43211312)(319.89074556,506.44212362)
\curveto(320.01074226,506.46211309)(320.13574214,506.48711306)(320.26574556,506.51712362)
\curveto(320.33574194,506.51711303)(320.40074187,506.51711303)(320.46074556,506.51712362)
\curveto(320.52074175,506.52711302)(320.58574169,506.53711301)(320.65574556,506.54712362)
}
}
{
\newrgbcolor{curcolor}{0 0 0}
\pscustom[linestyle=none,fillstyle=solid,fillcolor=curcolor]
{
\newpath
\moveto(332.75535494,500.26212362)
\curveto(332.78534721,500.14211941)(332.81034719,500.00211955)(332.83035494,499.84212362)
\curveto(332.85034715,499.68211987)(332.86034714,499.51712003)(332.86035494,499.34712362)
\curveto(332.86034714,499.17712037)(332.85034715,499.01212054)(332.83035494,498.85212362)
\curveto(332.81034719,498.69212086)(332.78534721,498.552121)(332.75535494,498.43212362)
\curveto(332.71534728,498.29212126)(332.68034732,498.16712138)(332.65035494,498.05712362)
\curveto(332.62034738,497.9471216)(332.58034742,497.83712171)(332.53035494,497.72712362)
\curveto(332.26034774,497.08712246)(331.84534815,496.60212295)(331.28535494,496.27212362)
\curveto(331.20534879,496.21212334)(331.12034888,496.16212339)(331.03035494,496.12212362)
\curveto(330.94034906,496.09212346)(330.84034916,496.05712349)(330.73035494,496.01712362)
\curveto(330.62034938,495.96712358)(330.5003495,495.93212362)(330.37035494,495.91212362)
\curveto(330.25034975,495.88212367)(330.12034988,495.8521237)(329.98035494,495.82212362)
\curveto(329.92035008,495.80212375)(329.86035014,495.79712375)(329.80035494,495.80712362)
\curveto(329.75035025,495.81712373)(329.69035031,495.81212374)(329.62035494,495.79212362)
\curveto(329.6003504,495.78212377)(329.57535042,495.78212377)(329.54535494,495.79212362)
\curveto(329.51535048,495.79212376)(329.49035051,495.78712376)(329.47035494,495.77712362)
\lineto(329.32035494,495.77712362)
\curveto(329.25035075,495.76712378)(329.2003508,495.76712378)(329.17035494,495.77712362)
\curveto(329.13035087,495.78712376)(329.08535091,495.79212376)(329.03535494,495.79212362)
\curveto(328.995351,495.78212377)(328.95535104,495.78212377)(328.91535494,495.79212362)
\curveto(328.82535117,495.81212374)(328.73535126,495.82712372)(328.64535494,495.83712362)
\curveto(328.55535144,495.83712371)(328.46535153,495.8471237)(328.37535494,495.86712362)
\curveto(328.28535171,495.89712365)(328.1953518,495.92212363)(328.10535494,495.94212362)
\curveto(328.01535198,495.96212359)(327.93035207,495.99212356)(327.85035494,496.03212362)
\curveto(327.61035239,496.14212341)(327.38535261,496.27212328)(327.17535494,496.42212362)
\curveto(326.96535303,496.58212297)(326.78535321,496.76212279)(326.63535494,496.96212362)
\curveto(326.51535348,497.13212242)(326.41035359,497.30712224)(326.32035494,497.48712362)
\curveto(326.23035377,497.66712188)(326.14035386,497.85712169)(326.05035494,498.05712362)
\curveto(326.01035399,498.15712139)(325.97535402,498.25712129)(325.94535494,498.35712362)
\curveto(325.92535407,498.46712108)(325.9003541,498.57712097)(325.87035494,498.68712362)
\curveto(325.83035417,498.82712072)(325.80535419,498.96712058)(325.79535494,499.10712362)
\curveto(325.78535421,499.2471203)(325.76535423,499.38712016)(325.73535494,499.52712362)
\curveto(325.72535427,499.63711991)(325.71535428,499.73711981)(325.70535494,499.82712362)
\curveto(325.70535429,499.92711962)(325.6953543,500.02711952)(325.67535494,500.12712362)
\lineto(325.67535494,500.21712362)
\curveto(325.68535431,500.2471193)(325.68535431,500.27211928)(325.67535494,500.29212362)
\lineto(325.67535494,500.50212362)
\curveto(325.65535434,500.56211899)(325.64535435,500.62711892)(325.64535494,500.69712362)
\curveto(325.65535434,500.77711877)(325.66035434,500.8521187)(325.66035494,500.92212362)
\lineto(325.66035494,501.07212362)
\curveto(325.66035434,501.12211843)(325.66535433,501.17211838)(325.67535494,501.22212362)
\lineto(325.67535494,501.59712362)
\curveto(325.68535431,501.62711792)(325.68535431,501.66211789)(325.67535494,501.70212362)
\curveto(325.67535432,501.74211781)(325.68035432,501.78211777)(325.69035494,501.82212362)
\curveto(325.71035429,501.93211762)(325.72535427,502.04211751)(325.73535494,502.15212362)
\curveto(325.74535425,502.27211728)(325.75535424,502.38711716)(325.76535494,502.49712362)
\curveto(325.80535419,502.6471169)(325.83035417,502.79211676)(325.84035494,502.93212362)
\curveto(325.86035414,503.08211647)(325.89035411,503.22711632)(325.93035494,503.36712362)
\curveto(326.02035398,503.66711588)(326.11535388,503.9521156)(326.21535494,504.22212362)
\curveto(326.31535368,504.49211506)(326.44035356,504.74211481)(326.59035494,504.97212362)
\curveto(326.79035321,505.29211426)(327.03535296,505.57211398)(327.32535494,505.81212362)
\curveto(327.61535238,506.0521135)(327.95535204,506.23711331)(328.34535494,506.36712362)
\curveto(328.45535154,506.40711314)(328.56535143,506.43211312)(328.67535494,506.44212362)
\curveto(328.7953512,506.46211309)(328.91535108,506.48711306)(329.03535494,506.51712362)
\curveto(329.10535089,506.52711302)(329.17035083,506.53211302)(329.23035494,506.53212362)
\curveto(329.29035071,506.53211302)(329.35535064,506.53711301)(329.42535494,506.54712362)
\curveto(330.12534987,506.56711298)(330.7003493,506.4521131)(331.15035494,506.20212362)
\curveto(331.6003484,505.9521136)(331.94534805,505.60211395)(332.18535494,505.15212362)
\curveto(332.2953477,504.92211463)(332.3953476,504.6471149)(332.48535494,504.32712362)
\curveto(332.50534749,504.25711529)(332.50534749,504.18211537)(332.48535494,504.10212362)
\curveto(332.47534752,504.03211552)(332.45034755,503.98211557)(332.41035494,503.95212362)
\curveto(332.38034762,503.92211563)(332.32034768,503.89711565)(332.23035494,503.87712362)
\curveto(332.14034786,503.86711568)(332.04034796,503.85711569)(331.93035494,503.84712362)
\curveto(331.83034817,503.8471157)(331.73034827,503.8521157)(331.63035494,503.86212362)
\curveto(331.54034846,503.87211568)(331.47534852,503.89211566)(331.43535494,503.92212362)
\curveto(331.32534867,503.99211556)(331.24534875,504.10211545)(331.19535494,504.25212362)
\curveto(331.15534884,504.40211515)(331.1003489,504.53211502)(331.03035494,504.64212362)
\curveto(330.84034916,504.9521146)(330.56034944,505.18211437)(330.19035494,505.33212362)
\curveto(330.12034988,505.36211419)(330.04534995,505.38211417)(329.96535494,505.39212362)
\curveto(329.8953501,505.40211415)(329.82035018,505.41711413)(329.74035494,505.43712362)
\curveto(329.69035031,505.4471141)(329.62035038,505.4521141)(329.53035494,505.45212362)
\curveto(329.45035055,505.4521141)(329.38535061,505.4471141)(329.33535494,505.43712362)
\curveto(329.2953507,505.41711413)(329.26035074,505.41211414)(329.23035494,505.42212362)
\curveto(329.2003508,505.43211412)(329.16535083,505.43211412)(329.12535494,505.42212362)
\lineto(328.88535494,505.36212362)
\curveto(328.81535118,505.34211421)(328.74535125,505.31711423)(328.67535494,505.28712362)
\curveto(328.2953517,505.12711442)(328.00535199,504.91711463)(327.80535494,504.65712362)
\curveto(327.61535238,504.39711515)(327.44035256,504.08211547)(327.28035494,503.71212362)
\curveto(327.25035275,503.63211592)(327.22535277,503.552116)(327.20535494,503.47212362)
\curveto(327.1953528,503.39211616)(327.17535282,503.31211624)(327.14535494,503.23212362)
\curveto(327.11535288,503.12211643)(327.09035291,503.00711654)(327.07035494,502.88712362)
\curveto(327.06035294,502.76711678)(327.04035296,502.6471169)(327.01035494,502.52712362)
\curveto(326.99035301,502.47711707)(326.98035302,502.42711712)(326.98035494,502.37712362)
\curveto(326.99035301,502.32711722)(326.98535301,502.27711727)(326.96535494,502.22712362)
\curveto(326.95535304,502.16711738)(326.95535304,502.08711746)(326.96535494,501.98712362)
\curveto(326.97535302,501.89711765)(326.99035301,501.84211771)(327.01035494,501.82212362)
\curveto(327.03035297,501.78211777)(327.06035294,501.76211779)(327.10035494,501.76212362)
\curveto(327.15035285,501.76211779)(327.1953528,501.77211778)(327.23535494,501.79212362)
\curveto(327.30535269,501.83211772)(327.36535263,501.87711767)(327.41535494,501.92712362)
\curveto(327.46535253,501.97711757)(327.52535247,502.02711752)(327.59535494,502.07712362)
\lineto(327.65535494,502.13712362)
\curveto(327.68535231,502.16711738)(327.71535228,502.19211736)(327.74535494,502.21212362)
\curveto(327.97535202,502.37211718)(328.25035175,502.50711704)(328.57035494,502.61712362)
\curveto(328.64035136,502.63711691)(328.71035129,502.6521169)(328.78035494,502.66212362)
\curveto(328.85035115,502.67211688)(328.92535107,502.68711686)(329.00535494,502.70712362)
\curveto(329.04535095,502.70711684)(329.08035092,502.71211684)(329.11035494,502.72212362)
\curveto(329.14035086,502.73211682)(329.17535082,502.73211682)(329.21535494,502.72212362)
\curveto(329.26535073,502.72211683)(329.30535069,502.73211682)(329.33535494,502.75212362)
\lineto(329.50035494,502.75212362)
\lineto(329.59035494,502.75212362)
\curveto(329.64035036,502.76211679)(329.68035032,502.76211679)(329.71035494,502.75212362)
\curveto(329.76035024,502.74211681)(329.81035019,502.73711681)(329.86035494,502.73712362)
\curveto(329.92035008,502.7471168)(329.97535002,502.7471168)(330.02535494,502.73712362)
\curveto(330.13534986,502.70711684)(330.24034976,502.68711686)(330.34035494,502.67712362)
\curveto(330.45034955,502.66711688)(330.55534944,502.64211691)(330.65535494,502.60212362)
\curveto(331.07534892,502.46211709)(331.42034858,502.27711727)(331.69035494,502.04712362)
\curveto(331.96034804,501.82711772)(332.2003478,501.54211801)(332.41035494,501.19212362)
\curveto(332.49034751,501.0521185)(332.55534744,500.90211865)(332.60535494,500.74212362)
\curveto(332.65534734,500.59211896)(332.70534729,500.43211912)(332.75535494,500.26212362)
\moveto(331.51035494,498.95712362)
\curveto(331.52034848,499.00712054)(331.52534847,499.0521205)(331.52535494,499.09212362)
\lineto(331.52535494,499.24212362)
\curveto(331.52534847,499.55212)(331.48534851,499.83711971)(331.40535494,500.09712362)
\curveto(331.38534861,500.15711939)(331.36534863,500.21211934)(331.34535494,500.26212362)
\curveto(331.33534866,500.32211923)(331.32034868,500.37711917)(331.30035494,500.42712362)
\curveto(331.08034892,500.91711863)(330.73534926,501.26711828)(330.26535494,501.47712362)
\curveto(330.18534981,501.50711804)(330.10534989,501.53211802)(330.02535494,501.55212362)
\lineto(329.78535494,501.61212362)
\curveto(329.70535029,501.63211792)(329.61535038,501.64211791)(329.51535494,501.64212362)
\lineto(329.20035494,501.64212362)
\curveto(329.18035082,501.62211793)(329.14035086,501.61211794)(329.08035494,501.61212362)
\curveto(329.03035097,501.62211793)(328.98535101,501.62211793)(328.94535494,501.61212362)
\lineto(328.70535494,501.55212362)
\curveto(328.63535136,501.54211801)(328.56535143,501.52211803)(328.49535494,501.49212362)
\curveto(327.8953521,501.23211832)(327.49035251,500.76711878)(327.28035494,500.09712362)
\curveto(327.25035275,500.01711953)(327.23035277,499.93711961)(327.22035494,499.85712362)
\curveto(327.21035279,499.77711977)(327.1953528,499.69211986)(327.17535494,499.60212362)
\lineto(327.17535494,499.45212362)
\curveto(327.16535283,499.41212014)(327.16035284,499.34212021)(327.16035494,499.24212362)
\curveto(327.16035284,499.01212054)(327.18035282,498.81712073)(327.22035494,498.65712362)
\curveto(327.24035276,498.58712096)(327.25535274,498.52212103)(327.26535494,498.46212362)
\curveto(327.27535272,498.40212115)(327.2953527,498.33712121)(327.32535494,498.26712362)
\curveto(327.43535256,497.98712156)(327.58035242,497.74212181)(327.76035494,497.53212362)
\curveto(327.94035206,497.33212222)(328.17535182,497.17212238)(328.46535494,497.05212362)
\lineto(328.70535494,496.96212362)
\lineto(328.94535494,496.90212362)
\curveto(328.995351,496.88212267)(329.03535096,496.87712267)(329.06535494,496.88712362)
\curveto(329.10535089,496.89712265)(329.15035085,496.89212266)(329.20035494,496.87212362)
\curveto(329.23035077,496.86212269)(329.28535071,496.85712269)(329.36535494,496.85712362)
\curveto(329.44535055,496.85712269)(329.50535049,496.86212269)(329.54535494,496.87212362)
\curveto(329.65535034,496.89212266)(329.76035024,496.90712264)(329.86035494,496.91712362)
\curveto(329.96035004,496.92712262)(330.05534994,496.95712259)(330.14535494,497.00712362)
\curveto(330.67534932,497.20712234)(331.06534893,497.58212197)(331.31535494,498.13212362)
\curveto(331.35534864,498.23212132)(331.38534861,498.33712121)(331.40535494,498.44712362)
\lineto(331.49535494,498.77712362)
\curveto(331.4953485,498.85712069)(331.5003485,498.91712063)(331.51035494,498.95712362)
}
}
{
\newrgbcolor{curcolor}{0 0 0}
\pscustom[linestyle=none,fillstyle=solid,fillcolor=curcolor]
{
\newpath
\moveto(343.90996431,504.46212362)
\curveto(343.70995401,504.17211538)(343.49995422,503.88711566)(343.27996431,503.60712362)
\curveto(343.06995465,503.32711622)(342.86495486,503.04211651)(342.66496431,502.75212362)
\curveto(342.06495566,501.90211765)(341.45995626,501.06211849)(340.84996431,500.23212362)
\curveto(340.23995748,499.41212014)(339.63495809,498.57712097)(339.03496431,497.72712362)
\lineto(338.52496431,497.00712362)
\lineto(338.01496431,496.31712362)
\curveto(337.93495979,496.20712334)(337.85495987,496.09212346)(337.77496431,495.97212362)
\curveto(337.69496003,495.8521237)(337.59996012,495.75712379)(337.48996431,495.68712362)
\curveto(337.44996027,495.66712388)(337.38496034,495.6521239)(337.29496431,495.64212362)
\curveto(337.21496051,495.62212393)(337.1249606,495.61212394)(337.02496431,495.61212362)
\curveto(336.9249608,495.61212394)(336.82996089,495.61712393)(336.73996431,495.62712362)
\curveto(336.65996106,495.63712391)(336.59996112,495.65712389)(336.55996431,495.68712362)
\curveto(336.52996119,495.70712384)(336.50496122,495.74212381)(336.48496431,495.79212362)
\curveto(336.47496125,495.83212372)(336.47996124,495.87712367)(336.49996431,495.92712362)
\curveto(336.53996118,496.00712354)(336.58496114,496.08212347)(336.63496431,496.15212362)
\curveto(336.69496103,496.23212332)(336.74996097,496.31212324)(336.79996431,496.39212362)
\curveto(337.03996068,496.73212282)(337.28496044,497.06712248)(337.53496431,497.39712362)
\curveto(337.78495994,497.72712182)(338.0249597,498.06212149)(338.25496431,498.40212362)
\curveto(338.41495931,498.62212093)(338.57495915,498.83712071)(338.73496431,499.04712362)
\curveto(338.89495883,499.25712029)(339.05495867,499.47212008)(339.21496431,499.69212362)
\curveto(339.57495815,500.21211934)(339.93995778,500.72211883)(340.30996431,501.22212362)
\curveto(340.67995704,501.72211783)(341.04995667,502.23211732)(341.41996431,502.75212362)
\curveto(341.55995616,502.9521166)(341.69995602,503.1471164)(341.83996431,503.33712362)
\curveto(341.98995573,503.52711602)(342.13495559,503.72211583)(342.27496431,503.92212362)
\curveto(342.48495524,504.22211533)(342.69995502,504.52211503)(342.91996431,504.82212362)
\lineto(343.57996431,505.72212362)
\lineto(343.75996431,505.99212362)
\lineto(343.96996431,506.26212362)
\lineto(344.08996431,506.44212362)
\curveto(344.13995358,506.50211305)(344.18995353,506.55711299)(344.23996431,506.60712362)
\curveto(344.30995341,506.65711289)(344.38495334,506.69211286)(344.46496431,506.71212362)
\curveto(344.48495324,506.72211283)(344.50995321,506.72211283)(344.53996431,506.71212362)
\curveto(344.57995314,506.71211284)(344.60995311,506.72211283)(344.62996431,506.74212362)
\curveto(344.74995297,506.74211281)(344.88495284,506.73711281)(345.03496431,506.72712362)
\curveto(345.18495254,506.72711282)(345.27495245,506.68211287)(345.30496431,506.59212362)
\curveto(345.3249524,506.56211299)(345.32995239,506.52711302)(345.31996431,506.48712362)
\curveto(345.30995241,506.4471131)(345.29495243,506.41711313)(345.27496431,506.39712362)
\curveto(345.23495249,506.31711323)(345.19495253,506.2471133)(345.15496431,506.18712362)
\curveto(345.11495261,506.12711342)(345.06995265,506.06711348)(345.01996431,506.00712362)
\lineto(344.44996431,505.22712362)
\curveto(344.26995345,504.97711457)(344.08995363,504.72211483)(343.90996431,504.46212362)
\moveto(337.05496431,500.56212362)
\curveto(337.00496072,500.58211897)(336.95496077,500.58711896)(336.90496431,500.57712362)
\curveto(336.85496087,500.56711898)(336.80496092,500.57211898)(336.75496431,500.59212362)
\curveto(336.64496108,500.61211894)(336.53996118,500.63211892)(336.43996431,500.65212362)
\curveto(336.34996137,500.68211887)(336.25496147,500.72211883)(336.15496431,500.77212362)
\curveto(335.8249619,500.91211864)(335.56996215,501.10711844)(335.38996431,501.35712362)
\curveto(335.20996251,501.61711793)(335.06496266,501.92711762)(334.95496431,502.28712362)
\curveto(334.9249628,502.36711718)(334.90496282,502.4471171)(334.89496431,502.52712362)
\curveto(334.88496284,502.61711693)(334.86996285,502.70211685)(334.84996431,502.78212362)
\curveto(334.83996288,502.83211672)(334.83496289,502.89711665)(334.83496431,502.97712362)
\curveto(334.8249629,503.00711654)(334.8199629,503.03711651)(334.81996431,503.06712362)
\curveto(334.8199629,503.10711644)(334.81496291,503.14211641)(334.80496431,503.17212362)
\lineto(334.80496431,503.32212362)
\curveto(334.79496293,503.37211618)(334.78996293,503.43211612)(334.78996431,503.50212362)
\curveto(334.78996293,503.58211597)(334.79496293,503.6471159)(334.80496431,503.69712362)
\lineto(334.80496431,503.86212362)
\curveto(334.8249629,503.91211564)(334.82996289,503.95711559)(334.81996431,503.99712362)
\curveto(334.8199629,504.0471155)(334.8249629,504.09211546)(334.83496431,504.13212362)
\curveto(334.84496288,504.17211538)(334.84996287,504.20711534)(334.84996431,504.23712362)
\curveto(334.84996287,504.27711527)(334.85496287,504.31711523)(334.86496431,504.35712362)
\curveto(334.89496283,504.46711508)(334.91496281,504.57711497)(334.92496431,504.68712362)
\curveto(334.94496278,504.80711474)(334.97996274,504.92211463)(335.02996431,505.03212362)
\curveto(335.16996255,505.37211418)(335.32996239,505.6471139)(335.50996431,505.85712362)
\curveto(335.69996202,506.07711347)(335.96996175,506.25711329)(336.31996431,506.39712362)
\curveto(336.39996132,506.42711312)(336.48496124,506.4471131)(336.57496431,506.45712362)
\curveto(336.66496106,506.47711307)(336.75996096,506.49711305)(336.85996431,506.51712362)
\curveto(336.88996083,506.52711302)(336.94496078,506.52711302)(337.02496431,506.51712362)
\curveto(337.10496062,506.51711303)(337.15496057,506.52711302)(337.17496431,506.54712362)
\curveto(337.73495999,506.55711299)(338.18495954,506.4471131)(338.52496431,506.21712362)
\curveto(338.87495885,505.98711356)(339.13495859,505.68211387)(339.30496431,505.30212362)
\curveto(339.34495838,505.21211434)(339.37995834,505.11711443)(339.40996431,505.01712362)
\curveto(339.43995828,504.91711463)(339.46495826,504.81711473)(339.48496431,504.71712362)
\curveto(339.50495822,504.68711486)(339.50995821,504.65711489)(339.49996431,504.62712362)
\curveto(339.49995822,504.59711495)(339.50495822,504.56711498)(339.51496431,504.53712362)
\curveto(339.54495818,504.42711512)(339.56495816,504.30211525)(339.57496431,504.16212362)
\curveto(339.58495814,504.03211552)(339.59495813,503.89711565)(339.60496431,503.75712362)
\lineto(339.60496431,503.59212362)
\curveto(339.61495811,503.53211602)(339.61495811,503.47711607)(339.60496431,503.42712362)
\curveto(339.59495813,503.37711617)(339.58995813,503.32711622)(339.58996431,503.27712362)
\lineto(339.58996431,503.14212362)
\curveto(339.57995814,503.10211645)(339.57495815,503.06211649)(339.57496431,503.02212362)
\curveto(339.58495814,502.98211657)(339.57995814,502.93711661)(339.55996431,502.88712362)
\curveto(339.53995818,502.77711677)(339.5199582,502.67211688)(339.49996431,502.57212362)
\curveto(339.48995823,502.47211708)(339.46995825,502.37211718)(339.43996431,502.27212362)
\curveto(339.30995841,501.91211764)(339.14495858,501.59711795)(338.94496431,501.32712362)
\curveto(338.74495898,501.05711849)(338.46995925,500.8521187)(338.11996431,500.71212362)
\curveto(338.03995968,500.68211887)(337.95495977,500.65711889)(337.86496431,500.63712362)
\lineto(337.59496431,500.57712362)
\curveto(337.54496018,500.56711898)(337.49996022,500.56211899)(337.45996431,500.56212362)
\curveto(337.4199603,500.57211898)(337.37996034,500.57211898)(337.33996431,500.56212362)
\curveto(337.23996048,500.54211901)(337.14496058,500.54211901)(337.05496431,500.56212362)
\moveto(336.21496431,501.95712362)
\curveto(336.25496147,501.88711766)(336.29496143,501.82211773)(336.33496431,501.76212362)
\curveto(336.37496135,501.71211784)(336.4249613,501.66211789)(336.48496431,501.61212362)
\lineto(336.63496431,501.49212362)
\curveto(336.69496103,501.46211809)(336.75996096,501.43711811)(336.82996431,501.41712362)
\curveto(336.86996085,501.39711815)(336.90496082,501.38711816)(336.93496431,501.38712362)
\curveto(336.97496075,501.39711815)(337.01496071,501.39211816)(337.05496431,501.37212362)
\curveto(337.08496064,501.37211818)(337.1249606,501.36711818)(337.17496431,501.35712362)
\curveto(337.2249605,501.35711819)(337.26496046,501.36211819)(337.29496431,501.37212362)
\lineto(337.51996431,501.41712362)
\curveto(337.76995995,501.49711805)(337.95495977,501.62211793)(338.07496431,501.79212362)
\curveto(338.15495957,501.89211766)(338.2249595,502.02211753)(338.28496431,502.18212362)
\curveto(338.36495936,502.36211719)(338.4249593,502.58711696)(338.46496431,502.85712362)
\curveto(338.50495922,503.13711641)(338.5199592,503.41711613)(338.50996431,503.69712362)
\curveto(338.49995922,503.98711556)(338.46995925,504.26211529)(338.41996431,504.52212362)
\curveto(338.36995935,504.78211477)(338.29495943,504.99211456)(338.19496431,505.15212362)
\curveto(338.07495965,505.3521142)(337.9249598,505.50211405)(337.74496431,505.60212362)
\curveto(337.66496006,505.6521139)(337.57496015,505.68211387)(337.47496431,505.69212362)
\curveto(337.37496035,505.71211384)(337.26996045,505.72211383)(337.15996431,505.72212362)
\curveto(337.13996058,505.71211384)(337.11496061,505.70711384)(337.08496431,505.70712362)
\curveto(337.06496066,505.71711383)(337.04496068,505.71711383)(337.02496431,505.70712362)
\curveto(336.97496075,505.69711385)(336.92996079,505.68711386)(336.88996431,505.67712362)
\curveto(336.84996087,505.67711387)(336.80996091,505.66711388)(336.76996431,505.64712362)
\curveto(336.58996113,505.56711398)(336.43996128,505.4471141)(336.31996431,505.28712362)
\curveto(336.20996151,505.12711442)(336.1199616,504.9471146)(336.04996431,504.74712362)
\curveto(335.98996173,504.55711499)(335.94496178,504.33211522)(335.91496431,504.07212362)
\curveto(335.89496183,503.81211574)(335.88996183,503.547116)(335.89996431,503.27712362)
\curveto(335.90996181,503.01711653)(335.93996178,502.76711678)(335.98996431,502.52712362)
\curveto(336.04996167,502.29711725)(336.1249616,502.10711744)(336.21496431,501.95712362)
\moveto(347.01496431,498.97212362)
\curveto(347.0249507,498.92212063)(347.02995069,498.83212072)(347.02996431,498.70212362)
\curveto(347.02995069,498.57212098)(347.0199507,498.48212107)(346.99996431,498.43212362)
\curveto(346.97995074,498.38212117)(346.97495075,498.32712122)(346.98496431,498.26712362)
\curveto(346.99495073,498.21712133)(346.99495073,498.16712138)(346.98496431,498.11712362)
\curveto(346.94495078,497.97712157)(346.91495081,497.84212171)(346.89496431,497.71212362)
\curveto(346.88495084,497.58212197)(346.85495087,497.46212209)(346.80496431,497.35212362)
\curveto(346.66495106,497.00212255)(346.49995122,496.70712284)(346.30996431,496.46712362)
\curveto(346.1199516,496.23712331)(345.84995187,496.0521235)(345.49996431,495.91212362)
\curveto(345.4199523,495.88212367)(345.33495239,495.86212369)(345.24496431,495.85212362)
\curveto(345.15495257,495.83212372)(345.06995265,495.81212374)(344.98996431,495.79212362)
\curveto(344.93995278,495.78212377)(344.88995283,495.77712377)(344.83996431,495.77712362)
\curveto(344.78995293,495.77712377)(344.73995298,495.77212378)(344.68996431,495.76212362)
\curveto(344.65995306,495.7521238)(344.60995311,495.7521238)(344.53996431,495.76212362)
\curveto(344.46995325,495.76212379)(344.4199533,495.76712378)(344.38996431,495.77712362)
\curveto(344.32995339,495.79712375)(344.26995345,495.80712374)(344.20996431,495.80712362)
\curveto(344.15995356,495.79712375)(344.10995361,495.80212375)(344.05996431,495.82212362)
\curveto(343.96995375,495.84212371)(343.87995384,495.86712368)(343.78996431,495.89712362)
\curveto(343.70995401,495.91712363)(343.62995409,495.9471236)(343.54996431,495.98712362)
\curveto(343.22995449,496.12712342)(342.97995474,496.32212323)(342.79996431,496.57212362)
\curveto(342.6199551,496.83212272)(342.46995525,497.13712241)(342.34996431,497.48712362)
\curveto(342.32995539,497.56712198)(342.31495541,497.6521219)(342.30496431,497.74212362)
\curveto(342.29495543,497.83212172)(342.27995544,497.91712163)(342.25996431,497.99712362)
\curveto(342.24995547,498.02712152)(342.24495548,498.05712149)(342.24496431,498.08712362)
\lineto(342.24496431,498.19212362)
\curveto(342.2249555,498.27212128)(342.21495551,498.3521212)(342.21496431,498.43212362)
\lineto(342.21496431,498.56712362)
\curveto(342.19495553,498.66712088)(342.19495553,498.76712078)(342.21496431,498.86712362)
\lineto(342.21496431,499.04712362)
\curveto(342.2249555,499.09712045)(342.22995549,499.14212041)(342.22996431,499.18212362)
\curveto(342.22995549,499.23212032)(342.23495549,499.27712027)(342.24496431,499.31712362)
\curveto(342.25495547,499.35712019)(342.25995546,499.39212016)(342.25996431,499.42212362)
\curveto(342.25995546,499.46212009)(342.26495546,499.50212005)(342.27496431,499.54212362)
\lineto(342.33496431,499.87212362)
\curveto(342.35495537,499.99211956)(342.38495534,500.10211945)(342.42496431,500.20212362)
\curveto(342.56495516,500.53211902)(342.724955,500.80711874)(342.90496431,501.02712362)
\curveto(343.09495463,501.25711829)(343.35495437,501.44211811)(343.68496431,501.58212362)
\curveto(343.76495396,501.62211793)(343.84995387,501.6471179)(343.93996431,501.65712362)
\lineto(344.23996431,501.71712362)
\lineto(344.37496431,501.71712362)
\curveto(344.4249533,501.72711782)(344.47495325,501.73211782)(344.52496431,501.73212362)
\curveto(345.09495263,501.7521178)(345.55495217,501.6471179)(345.90496431,501.41712362)
\curveto(346.26495146,501.19711835)(346.52995119,500.89711865)(346.69996431,500.51712362)
\curveto(346.74995097,500.41711913)(346.78995093,500.31711923)(346.81996431,500.21712362)
\curveto(346.84995087,500.11711943)(346.87995084,500.01211954)(346.90996431,499.90212362)
\curveto(346.9199508,499.86211969)(346.9249508,499.82711972)(346.92496431,499.79712362)
\curveto(346.9249508,499.77711977)(346.92995079,499.7471198)(346.93996431,499.70712362)
\curveto(346.95995076,499.63711991)(346.96995075,499.56211999)(346.96996431,499.48212362)
\curveto(346.96995075,499.40212015)(346.97995074,499.32212023)(346.99996431,499.24212362)
\curveto(346.99995072,499.19212036)(346.99995072,499.1471204)(346.99996431,499.10712362)
\curveto(346.99995072,499.06712048)(347.00495072,499.02212053)(347.01496431,498.97212362)
\moveto(345.90496431,498.53712362)
\curveto(345.91495181,498.58712096)(345.9199518,498.66212089)(345.91996431,498.76212362)
\curveto(345.92995179,498.86212069)(345.9249518,498.93712061)(345.90496431,498.98712362)
\curveto(345.88495184,499.0471205)(345.87995184,499.10212045)(345.88996431,499.15212362)
\curveto(345.90995181,499.21212034)(345.90995181,499.27212028)(345.88996431,499.33212362)
\curveto(345.87995184,499.36212019)(345.87495185,499.39712015)(345.87496431,499.43712362)
\curveto(345.87495185,499.47712007)(345.86995185,499.51712003)(345.85996431,499.55712362)
\curveto(345.83995188,499.63711991)(345.8199519,499.71211984)(345.79996431,499.78212362)
\curveto(345.78995193,499.86211969)(345.77495195,499.94211961)(345.75496431,500.02212362)
\curveto(345.724952,500.08211947)(345.69995202,500.14211941)(345.67996431,500.20212362)
\curveto(345.65995206,500.26211929)(345.62995209,500.32211923)(345.58996431,500.38212362)
\curveto(345.48995223,500.552119)(345.35995236,500.68711886)(345.19996431,500.78712362)
\curveto(345.1199526,500.83711871)(345.0249527,500.87211868)(344.91496431,500.89212362)
\curveto(344.80495292,500.91211864)(344.67995304,500.92211863)(344.53996431,500.92212362)
\curveto(344.5199532,500.91211864)(344.49495323,500.90711864)(344.46496431,500.90712362)
\curveto(344.43495329,500.91711863)(344.40495332,500.91711863)(344.37496431,500.90712362)
\lineto(344.22496431,500.84712362)
\curveto(344.17495355,500.83711871)(344.12995359,500.82211873)(344.08996431,500.80212362)
\curveto(343.89995382,500.69211886)(343.75495397,500.547119)(343.65496431,500.36712362)
\curveto(343.56495416,500.18711936)(343.48495424,499.98211957)(343.41496431,499.75212362)
\curveto(343.37495435,499.62211993)(343.35495437,499.48712006)(343.35496431,499.34712362)
\curveto(343.35495437,499.21712033)(343.34495438,499.07212048)(343.32496431,498.91212362)
\curveto(343.31495441,498.86212069)(343.30495442,498.80212075)(343.29496431,498.73212362)
\curveto(343.29495443,498.66212089)(343.30495442,498.60212095)(343.32496431,498.55212362)
\lineto(343.32496431,498.38712362)
\lineto(343.32496431,498.20712362)
\curveto(343.33495439,498.15712139)(343.34495438,498.10212145)(343.35496431,498.04212362)
\curveto(343.36495436,497.99212156)(343.36995435,497.93712161)(343.36996431,497.87712362)
\curveto(343.37995434,497.81712173)(343.39495433,497.76212179)(343.41496431,497.71212362)
\curveto(343.46495426,497.52212203)(343.5249542,497.3471222)(343.59496431,497.18712362)
\curveto(343.66495406,497.02712252)(343.76995395,496.89712265)(343.90996431,496.79712362)
\curveto(344.03995368,496.69712285)(344.17995354,496.62712292)(344.32996431,496.58712362)
\curveto(344.35995336,496.57712297)(344.38495334,496.57212298)(344.40496431,496.57212362)
\curveto(344.43495329,496.58212297)(344.46495326,496.58212297)(344.49496431,496.57212362)
\curveto(344.51495321,496.57212298)(344.54495318,496.56712298)(344.58496431,496.55712362)
\curveto(344.6249531,496.55712299)(344.65995306,496.56212299)(344.68996431,496.57212362)
\curveto(344.72995299,496.58212297)(344.76995295,496.58712296)(344.80996431,496.58712362)
\curveto(344.84995287,496.58712296)(344.88995283,496.59712295)(344.92996431,496.61712362)
\curveto(345.16995255,496.69712285)(345.36495236,496.83212272)(345.51496431,497.02212362)
\curveto(345.63495209,497.20212235)(345.724952,497.40712214)(345.78496431,497.63712362)
\curveto(345.80495192,497.70712184)(345.8199519,497.77712177)(345.82996431,497.84712362)
\curveto(345.83995188,497.92712162)(345.85495187,498.00712154)(345.87496431,498.08712362)
\curveto(345.87495185,498.1471214)(345.87995184,498.19212136)(345.88996431,498.22212362)
\curveto(345.88995183,498.24212131)(345.88995183,498.26712128)(345.88996431,498.29712362)
\curveto(345.88995183,498.33712121)(345.89495183,498.36712118)(345.90496431,498.38712362)
\lineto(345.90496431,498.53712362)
}
}
{
\newrgbcolor{curcolor}{0 0 0}
\pscustom[linestyle=none,fillstyle=solid,fillcolor=curcolor]
{
\newpath
\moveto(641.46574556,635.83282919)
\curveto(641.56574071,635.83281857)(641.66074061,635.82281858)(641.75074556,635.80282919)
\curveto(641.84074043,635.79281861)(641.90574037,635.76281864)(641.94574556,635.71282919)
\curveto(642.00574027,635.63281877)(642.03574024,635.52781887)(642.03574556,635.39782919)
\lineto(642.03574556,635.00782919)
\lineto(642.03574556,633.50782919)
\lineto(642.03574556,627.11782919)
\lineto(642.03574556,625.94782919)
\lineto(642.03574556,625.63282919)
\curveto(642.04574023,625.53282887)(642.03074024,625.45282895)(641.99074556,625.39282919)
\curveto(641.94074033,625.31282909)(641.86574041,625.26282914)(641.76574556,625.24282919)
\curveto(641.6757406,625.23282917)(641.56574071,625.22782917)(641.43574556,625.22782919)
\lineto(641.21074556,625.22782919)
\curveto(641.13074114,625.24782915)(641.06074121,625.26282914)(641.00074556,625.27282919)
\curveto(640.94074133,625.29282911)(640.89074138,625.33282907)(640.85074556,625.39282919)
\curveto(640.81074146,625.45282895)(640.79074148,625.52782887)(640.79074556,625.61782919)
\lineto(640.79074556,625.91782919)
\lineto(640.79074556,627.01282919)
\lineto(640.79074556,632.35282919)
\curveto(640.7707415,632.44282196)(640.75574152,632.51782188)(640.74574556,632.57782919)
\curveto(640.74574153,632.64782175)(640.71574156,632.70782169)(640.65574556,632.75782919)
\curveto(640.58574169,632.80782159)(640.49574178,632.83282157)(640.38574556,632.83282919)
\curveto(640.28574199,632.84282156)(640.1757421,632.84782155)(640.05574556,632.84782919)
\lineto(638.91574556,632.84782919)
\lineto(638.42074556,632.84782919)
\curveto(638.26074401,632.85782154)(638.15074412,632.91782148)(638.09074556,633.02782919)
\curveto(638.0707442,633.05782134)(638.06074421,633.08782131)(638.06074556,633.11782919)
\curveto(638.06074421,633.15782124)(638.05574422,633.2028212)(638.04574556,633.25282919)
\curveto(638.02574425,633.37282103)(638.03074424,633.48282092)(638.06074556,633.58282919)
\curveto(638.10074417,633.68282072)(638.15574412,633.75282065)(638.22574556,633.79282919)
\curveto(638.30574397,633.84282056)(638.42574385,633.86782053)(638.58574556,633.86782919)
\curveto(638.74574353,633.86782053)(638.88074339,633.88282052)(638.99074556,633.91282919)
\curveto(639.04074323,633.92282048)(639.09574318,633.92782047)(639.15574556,633.92782919)
\curveto(639.21574306,633.93782046)(639.275743,633.95282045)(639.33574556,633.97282919)
\curveto(639.48574279,634.02282038)(639.63074264,634.07282033)(639.77074556,634.12282919)
\curveto(639.91074236,634.18282022)(640.04574223,634.25282015)(640.17574556,634.33282919)
\curveto(640.31574196,634.42281998)(640.43574184,634.52781987)(640.53574556,634.64782919)
\curveto(640.63574164,634.76781963)(640.73074154,634.8978195)(640.82074556,635.03782919)
\curveto(640.88074139,635.13781926)(640.92574135,635.24781915)(640.95574556,635.36782919)
\curveto(640.99574128,635.48781891)(641.04574123,635.59281881)(641.10574556,635.68282919)
\curveto(641.15574112,635.74281866)(641.22574105,635.78281862)(641.31574556,635.80282919)
\curveto(641.33574094,635.81281859)(641.36074091,635.81781858)(641.39074556,635.81782919)
\curveto(641.42074085,635.81781858)(641.44574083,635.82281858)(641.46574556,635.83282919)
}
}
{
\newrgbcolor{curcolor}{0 0 0}
\pscustom[linestyle=none,fillstyle=solid,fillcolor=curcolor]
{
\newpath
\moveto(647.26535494,635.63782919)
\lineto(650.86535494,635.63782919)
\lineto(651.51035494,635.63782919)
\curveto(651.59034841,635.63781876)(651.66534833,635.63281877)(651.73535494,635.62282919)
\curveto(651.80534819,635.62281878)(651.86534813,635.61281879)(651.91535494,635.59282919)
\curveto(651.98534801,635.56281884)(652.04034796,635.5028189)(652.08035494,635.41282919)
\curveto(652.1003479,635.38281902)(652.11034789,635.34281906)(652.11035494,635.29282919)
\lineto(652.11035494,635.15782919)
\curveto(652.12034788,635.04781935)(652.11534788,634.94281946)(652.09535494,634.84282919)
\curveto(652.08534791,634.74281966)(652.05034795,634.67281973)(651.99035494,634.63282919)
\curveto(651.9003481,634.56281984)(651.76534823,634.52781987)(651.58535494,634.52782919)
\curveto(651.40534859,634.53781986)(651.24034876,634.54281986)(651.09035494,634.54282919)
\lineto(649.09535494,634.54282919)
\lineto(648.60035494,634.54282919)
\lineto(648.46535494,634.54282919)
\curveto(648.42535157,634.54281986)(648.38535161,634.53781986)(648.34535494,634.52782919)
\lineto(648.13535494,634.52782919)
\curveto(648.02535197,634.4978199)(647.94535205,634.45781994)(647.89535494,634.40782919)
\curveto(647.84535215,634.36782003)(647.81035219,634.31282009)(647.79035494,634.24282919)
\curveto(647.77035223,634.18282022)(647.75535224,634.11282029)(647.74535494,634.03282919)
\curveto(647.73535226,633.95282045)(647.71535228,633.86282054)(647.68535494,633.76282919)
\curveto(647.63535236,633.56282084)(647.5953524,633.35782104)(647.56535494,633.14782919)
\curveto(647.53535246,632.93782146)(647.4953525,632.73282167)(647.44535494,632.53282919)
\curveto(647.42535257,632.46282194)(647.41535258,632.39282201)(647.41535494,632.32282919)
\curveto(647.41535258,632.26282214)(647.40535259,632.1978222)(647.38535494,632.12782919)
\curveto(647.37535262,632.0978223)(647.36535263,632.05782234)(647.35535494,632.00782919)
\curveto(647.35535264,631.96782243)(647.36035264,631.92782247)(647.37035494,631.88782919)
\curveto(647.39035261,631.83782256)(647.41535258,631.79282261)(647.44535494,631.75282919)
\curveto(647.48535251,631.72282268)(647.54535245,631.71782268)(647.62535494,631.73782919)
\curveto(647.68535231,631.75782264)(647.74535225,631.78282262)(647.80535494,631.81282919)
\curveto(647.86535213,631.85282255)(647.92535207,631.88782251)(647.98535494,631.91782919)
\curveto(648.04535195,631.93782246)(648.0953519,631.95282245)(648.13535494,631.96282919)
\curveto(648.32535167,632.04282236)(648.53035147,632.0978223)(648.75035494,632.12782919)
\curveto(648.98035102,632.15782224)(649.21035079,632.16782223)(649.44035494,632.15782919)
\curveto(649.68035032,632.15782224)(649.91035009,632.13282227)(650.13035494,632.08282919)
\curveto(650.35034965,632.04282236)(650.55034945,631.98282242)(650.73035494,631.90282919)
\curveto(650.78034922,631.88282252)(650.82534917,631.86282254)(650.86535494,631.84282919)
\curveto(650.91534908,631.82282258)(650.96534903,631.7978226)(651.01535494,631.76782919)
\curveto(651.36534863,631.55782284)(651.64534835,631.32782307)(651.85535494,631.07782919)
\curveto(652.07534792,630.82782357)(652.27034773,630.5028239)(652.44035494,630.10282919)
\curveto(652.49034751,629.99282441)(652.52534747,629.88282452)(652.54535494,629.77282919)
\curveto(652.56534743,629.66282474)(652.59034741,629.54782485)(652.62035494,629.42782919)
\curveto(652.63034737,629.397825)(652.63534736,629.35282505)(652.63535494,629.29282919)
\curveto(652.65534734,629.23282517)(652.66534733,629.16282524)(652.66535494,629.08282919)
\curveto(652.66534733,629.01282539)(652.67534732,628.94782545)(652.69535494,628.88782919)
\lineto(652.69535494,628.72282919)
\curveto(652.70534729,628.67282573)(652.71034729,628.6028258)(652.71035494,628.51282919)
\curveto(652.71034729,628.42282598)(652.7003473,628.35282605)(652.68035494,628.30282919)
\curveto(652.66034734,628.24282616)(652.65534734,628.18282622)(652.66535494,628.12282919)
\curveto(652.67534732,628.07282633)(652.67034733,628.02282638)(652.65035494,627.97282919)
\curveto(652.61034739,627.81282659)(652.57534742,627.66282674)(652.54535494,627.52282919)
\curveto(652.51534748,627.38282702)(652.47034753,627.24782715)(652.41035494,627.11782919)
\curveto(652.25034775,626.74782765)(652.03034797,626.41282799)(651.75035494,626.11282919)
\curveto(651.47034853,625.81282859)(651.15034885,625.58282882)(650.79035494,625.42282919)
\curveto(650.62034938,625.34282906)(650.42034958,625.26782913)(650.19035494,625.19782919)
\curveto(650.08034992,625.15782924)(649.96535003,625.13282927)(649.84535494,625.12282919)
\curveto(649.72535027,625.11282929)(649.60535039,625.09282931)(649.48535494,625.06282919)
\curveto(649.43535056,625.04282936)(649.38035062,625.04282936)(649.32035494,625.06282919)
\curveto(649.26035074,625.07282933)(649.2003508,625.06782933)(649.14035494,625.04782919)
\curveto(649.04035096,625.02782937)(648.94035106,625.02782937)(648.84035494,625.04782919)
\lineto(648.70535494,625.04782919)
\curveto(648.65535134,625.06782933)(648.5953514,625.07782932)(648.52535494,625.07782919)
\curveto(648.46535153,625.06782933)(648.41035159,625.07282933)(648.36035494,625.09282919)
\curveto(648.32035168,625.1028293)(648.28535171,625.10782929)(648.25535494,625.10782919)
\curveto(648.22535177,625.10782929)(648.19035181,625.11282929)(648.15035494,625.12282919)
\lineto(647.88035494,625.18282919)
\curveto(647.79035221,625.2028292)(647.70535229,625.23282917)(647.62535494,625.27282919)
\curveto(647.28535271,625.41282899)(646.995353,625.56782883)(646.75535494,625.73782919)
\curveto(646.51535348,625.91782848)(646.2953537,626.14782825)(646.09535494,626.42782919)
\curveto(645.94535405,626.65782774)(645.83035417,626.8978275)(645.75035494,627.14782919)
\curveto(645.73035427,627.1978272)(645.72035428,627.24282716)(645.72035494,627.28282919)
\curveto(645.72035428,627.33282707)(645.71035429,627.38282702)(645.69035494,627.43282919)
\curveto(645.67035433,627.49282691)(645.65535434,627.57282683)(645.64535494,627.67282919)
\curveto(645.64535435,627.77282663)(645.66535433,627.84782655)(645.70535494,627.89782919)
\curveto(645.75535424,627.97782642)(645.83535416,628.02282638)(645.94535494,628.03282919)
\curveto(646.05535394,628.04282636)(646.17035383,628.04782635)(646.29035494,628.04782919)
\lineto(646.45535494,628.04782919)
\curveto(646.51535348,628.04782635)(646.57035343,628.03782636)(646.62035494,628.01782919)
\curveto(646.71035329,627.9978264)(646.78035322,627.95782644)(646.83035494,627.89782919)
\curveto(646.9003531,627.80782659)(646.94535305,627.6978267)(646.96535494,627.56782919)
\curveto(646.995353,627.44782695)(647.04035296,627.34282706)(647.10035494,627.25282919)
\curveto(647.29035271,626.91282749)(647.55035245,626.64282776)(647.88035494,626.44282919)
\curveto(647.98035202,626.38282802)(648.08535191,626.33282807)(648.19535494,626.29282919)
\curveto(648.31535168,626.26282814)(648.43535156,626.22782817)(648.55535494,626.18782919)
\curveto(648.72535127,626.13782826)(648.93035107,626.11782828)(649.17035494,626.12782919)
\curveto(649.42035058,626.14782825)(649.62035038,626.18282822)(649.77035494,626.23282919)
\curveto(650.14034986,626.35282805)(650.43034957,626.51282789)(650.64035494,626.71282919)
\curveto(650.86034914,626.92282748)(651.04034896,627.2028272)(651.18035494,627.55282919)
\curveto(651.23034877,627.65282675)(651.26034874,627.75782664)(651.27035494,627.86782919)
\curveto(651.29034871,627.97782642)(651.31534868,628.09282631)(651.34535494,628.21282919)
\lineto(651.34535494,628.31782919)
\curveto(651.35534864,628.35782604)(651.36034864,628.397826)(651.36035494,628.43782919)
\curveto(651.37034863,628.46782593)(651.37034863,628.5028259)(651.36035494,628.54282919)
\lineto(651.36035494,628.66282919)
\curveto(651.36034864,628.92282548)(651.33034867,629.16782523)(651.27035494,629.39782919)
\curveto(651.16034884,629.74782465)(651.00534899,630.04282436)(650.80535494,630.28282919)
\curveto(650.60534939,630.53282387)(650.34534965,630.72782367)(650.02535494,630.86782919)
\lineto(649.84535494,630.92782919)
\curveto(649.7953502,630.94782345)(649.73535026,630.96782343)(649.66535494,630.98782919)
\curveto(649.61535038,631.00782339)(649.55535044,631.01782338)(649.48535494,631.01782919)
\curveto(649.42535057,631.02782337)(649.36035064,631.04282336)(649.29035494,631.06282919)
\lineto(649.14035494,631.06282919)
\curveto(649.1003509,631.08282332)(649.04535095,631.09282331)(648.97535494,631.09282919)
\curveto(648.91535108,631.09282331)(648.86035114,631.08282332)(648.81035494,631.06282919)
\lineto(648.70535494,631.06282919)
\curveto(648.67535132,631.06282334)(648.64035136,631.05782334)(648.60035494,631.04782919)
\lineto(648.36035494,630.98782919)
\curveto(648.28035172,630.97782342)(648.2003518,630.95782344)(648.12035494,630.92782919)
\curveto(647.88035212,630.82782357)(647.65035235,630.69282371)(647.43035494,630.52282919)
\curveto(647.34035266,630.45282395)(647.25535274,630.37782402)(647.17535494,630.29782919)
\curveto(647.0953529,630.22782417)(646.995353,630.17282423)(646.87535494,630.13282919)
\curveto(646.78535321,630.1028243)(646.64535335,630.09282431)(646.45535494,630.10282919)
\curveto(646.27535372,630.11282429)(646.15535384,630.13782426)(646.09535494,630.17782919)
\curveto(646.04535395,630.21782418)(646.00535399,630.27782412)(645.97535494,630.35782919)
\curveto(645.95535404,630.43782396)(645.95535404,630.52282388)(645.97535494,630.61282919)
\curveto(646.00535399,630.73282367)(646.02535397,630.85282355)(646.03535494,630.97282919)
\curveto(646.05535394,631.1028233)(646.08035392,631.22782317)(646.11035494,631.34782919)
\curveto(646.13035387,631.38782301)(646.13535386,631.42282298)(646.12535494,631.45282919)
\curveto(646.12535387,631.49282291)(646.13535386,631.53782286)(646.15535494,631.58782919)
\curveto(646.17535382,631.67782272)(646.19035381,631.76782263)(646.20035494,631.85782919)
\curveto(646.21035379,631.95782244)(646.23035377,632.05282235)(646.26035494,632.14282919)
\curveto(646.27035373,632.2028222)(646.27535372,632.26282214)(646.27535494,632.32282919)
\curveto(646.28535371,632.38282202)(646.3003537,632.44282196)(646.32035494,632.50282919)
\curveto(646.37035363,632.7028217)(646.40535359,632.90782149)(646.42535494,633.11782919)
\curveto(646.45535354,633.33782106)(646.4953535,633.54782085)(646.54535494,633.74782919)
\curveto(646.57535342,633.84782055)(646.5953534,633.94782045)(646.60535494,634.04782919)
\curveto(646.61535338,634.14782025)(646.63035337,634.24782015)(646.65035494,634.34782919)
\curveto(646.66035334,634.37782002)(646.66535333,634.41781998)(646.66535494,634.46782919)
\curveto(646.6953533,634.57781982)(646.71535328,634.68281972)(646.72535494,634.78282919)
\curveto(646.74535325,634.89281951)(646.77035323,635.0028194)(646.80035494,635.11282919)
\curveto(646.82035318,635.19281921)(646.83535316,635.26281914)(646.84535494,635.32282919)
\curveto(646.85535314,635.39281901)(646.88035312,635.45281895)(646.92035494,635.50282919)
\curveto(646.94035306,635.53281887)(646.97035303,635.55281885)(647.01035494,635.56282919)
\curveto(647.05035295,635.58281882)(647.0953529,635.6028188)(647.14535494,635.62282919)
\curveto(647.20535279,635.62281878)(647.24535275,635.62781877)(647.26535494,635.63782919)
}
}
{
\newrgbcolor{curcolor}{0 0 0}
\pscustom[linestyle=none,fillstyle=solid,fillcolor=curcolor]
{
\newpath
\moveto(655.05996431,626.86282919)
\lineto(655.35996431,626.86282919)
\curveto(655.46996225,626.87282753)(655.57496215,626.87282753)(655.67496431,626.86282919)
\curveto(655.78496194,626.86282754)(655.88496184,626.85282755)(655.97496431,626.83282919)
\curveto(656.06496166,626.82282758)(656.13496159,626.7978276)(656.18496431,626.75782919)
\curveto(656.20496152,626.73782766)(656.2199615,626.70782769)(656.22996431,626.66782919)
\curveto(656.24996147,626.62782777)(656.26996145,626.58282782)(656.28996431,626.53282919)
\lineto(656.28996431,626.45782919)
\curveto(656.29996142,626.40782799)(656.29996142,626.35282805)(656.28996431,626.29282919)
\lineto(656.28996431,626.14282919)
\lineto(656.28996431,625.66282919)
\curveto(656.28996143,625.49282891)(656.24996147,625.37282903)(656.16996431,625.30282919)
\curveto(656.09996162,625.25282915)(656.00996171,625.22782917)(655.89996431,625.22782919)
\lineto(655.56996431,625.22782919)
\lineto(655.11996431,625.22782919)
\curveto(654.96996275,625.22782917)(654.85496287,625.25782914)(654.77496431,625.31782919)
\curveto(654.73496299,625.34782905)(654.70496302,625.397829)(654.68496431,625.46782919)
\curveto(654.66496306,625.54782885)(654.64996307,625.63282877)(654.63996431,625.72282919)
\lineto(654.63996431,626.00782919)
\curveto(654.64996307,626.10782829)(654.65496307,626.19282821)(654.65496431,626.26282919)
\lineto(654.65496431,626.45782919)
\curveto(654.65496307,626.51782788)(654.66496306,626.57282783)(654.68496431,626.62282919)
\curveto(654.724963,626.73282767)(654.79496293,626.8028276)(654.89496431,626.83282919)
\curveto(654.9249628,626.83282757)(654.97996274,626.84282756)(655.05996431,626.86282919)
}
}
{
\newrgbcolor{curcolor}{0 0 0}
\pscustom[linestyle=none,fillstyle=solid,fillcolor=curcolor]
{
\newpath
\moveto(661.51512056,635.83282919)
\curveto(662.20511593,635.84281856)(662.80511533,635.72281868)(663.31512056,635.47282919)
\curveto(663.8351143,635.22281918)(664.2301139,634.88781951)(664.50012056,634.46782919)
\curveto(664.55011358,634.38782001)(664.59511354,634.2978201)(664.63512056,634.19782919)
\curveto(664.67511346,634.10782029)(664.72011341,634.01282039)(664.77012056,633.91282919)
\curveto(664.81011332,633.81282059)(664.84011329,633.71282069)(664.86012056,633.61282919)
\curveto(664.88011325,633.51282089)(664.90011323,633.40782099)(664.92012056,633.29782919)
\curveto(664.94011319,633.24782115)(664.94511319,633.2028212)(664.93512056,633.16282919)
\curveto(664.92511321,633.12282128)(664.9301132,633.07782132)(664.95012056,633.02782919)
\curveto(664.96011317,632.97782142)(664.96511317,632.89282151)(664.96512056,632.77282919)
\curveto(664.96511317,632.66282174)(664.96011317,632.57782182)(664.95012056,632.51782919)
\curveto(664.9301132,632.45782194)(664.92011321,632.397822)(664.92012056,632.33782919)
\curveto(664.9301132,632.27782212)(664.92511321,632.21782218)(664.90512056,632.15782919)
\curveto(664.86511327,632.01782238)(664.8301133,631.88282252)(664.80012056,631.75282919)
\curveto(664.77011336,631.62282278)(664.7301134,631.4978229)(664.68012056,631.37782919)
\curveto(664.62011351,631.23782316)(664.55011358,631.11282329)(664.47012056,631.00282919)
\curveto(664.40011373,630.89282351)(664.32511381,630.78282362)(664.24512056,630.67282919)
\lineto(664.18512056,630.61282919)
\curveto(664.17511396,630.59282381)(664.16011397,630.57282383)(664.14012056,630.55282919)
\curveto(664.02011411,630.39282401)(663.88511425,630.24782415)(663.73512056,630.11782919)
\curveto(663.58511455,629.98782441)(663.42511471,629.86282454)(663.25512056,629.74282919)
\curveto(662.94511519,629.52282488)(662.65011548,629.31782508)(662.37012056,629.12782919)
\curveto(662.14011599,628.98782541)(661.91011622,628.85282555)(661.68012056,628.72282919)
\curveto(661.46011667,628.59282581)(661.24011689,628.45782594)(661.02012056,628.31782919)
\curveto(660.77011736,628.14782625)(660.5301176,627.96782643)(660.30012056,627.77782919)
\curveto(660.08011805,627.58782681)(659.89011824,627.36282704)(659.73012056,627.10282919)
\curveto(659.69011844,627.04282736)(659.65511848,626.98282742)(659.62512056,626.92282919)
\curveto(659.59511854,626.87282753)(659.56511857,626.80782759)(659.53512056,626.72782919)
\curveto(659.51511862,626.65782774)(659.51011862,626.5978278)(659.52012056,626.54782919)
\curveto(659.54011859,626.47782792)(659.57511856,626.42282798)(659.62512056,626.38282919)
\curveto(659.67511846,626.35282805)(659.7351184,626.33282807)(659.80512056,626.32282919)
\lineto(660.04512056,626.32282919)
\lineto(660.79512056,626.32282919)
\lineto(663.60012056,626.32282919)
\lineto(664.26012056,626.32282919)
\curveto(664.35011378,626.32282808)(664.4351137,626.31782808)(664.51512056,626.30782919)
\curveto(664.59511354,626.30782809)(664.66011347,626.28782811)(664.71012056,626.24782919)
\curveto(664.76011337,626.20782819)(664.80011333,626.13282827)(664.83012056,626.02282919)
\curveto(664.87011326,625.92282848)(664.88011325,625.82282858)(664.86012056,625.72282919)
\lineto(664.86012056,625.58782919)
\curveto(664.84011329,625.51782888)(664.82011331,625.45782894)(664.80012056,625.40782919)
\curveto(664.78011335,625.35782904)(664.74511339,625.31782908)(664.69512056,625.28782919)
\curveto(664.64511349,625.24782915)(664.57511356,625.22782917)(664.48512056,625.22782919)
\lineto(664.21512056,625.22782919)
\lineto(663.31512056,625.22782919)
\lineto(659.80512056,625.22782919)
\lineto(658.74012056,625.22782919)
\curveto(658.66011947,625.22782917)(658.57011956,625.22282918)(658.47012056,625.21282919)
\curveto(658.37011976,625.21282919)(658.28511985,625.22282918)(658.21512056,625.24282919)
\curveto(658.00512013,625.31282909)(657.94012019,625.49282891)(658.02012056,625.78282919)
\curveto(658.0301201,625.82282858)(658.0301201,625.85782854)(658.02012056,625.88782919)
\curveto(658.02012011,625.92782847)(658.0301201,625.97282843)(658.05012056,626.02282919)
\curveto(658.07012006,626.1028283)(658.09012004,626.18782821)(658.11012056,626.27782919)
\curveto(658.13012,626.36782803)(658.15511998,626.45282795)(658.18512056,626.53282919)
\curveto(658.34511979,627.02282738)(658.54511959,627.43782696)(658.78512056,627.77782919)
\curveto(658.96511917,628.02782637)(659.17011896,628.25282615)(659.40012056,628.45282919)
\curveto(659.6301185,628.66282574)(659.87011826,628.85782554)(660.12012056,629.03782919)
\curveto(660.38011775,629.21782518)(660.64511749,629.38782501)(660.91512056,629.54782919)
\curveto(661.19511694,629.71782468)(661.46511667,629.89282451)(661.72512056,630.07282919)
\curveto(661.8351163,630.15282425)(661.94011619,630.22782417)(662.04012056,630.29782919)
\curveto(662.15011598,630.36782403)(662.26011587,630.44282396)(662.37012056,630.52282919)
\curveto(662.41011572,630.55282385)(662.44511569,630.58282382)(662.47512056,630.61282919)
\curveto(662.51511562,630.65282375)(662.55511558,630.68282372)(662.59512056,630.70282919)
\curveto(662.7351154,630.81282359)(662.86011527,630.93782346)(662.97012056,631.07782919)
\curveto(662.99011514,631.10782329)(663.01511512,631.13282327)(663.04512056,631.15282919)
\curveto(663.07511506,631.18282322)(663.10011503,631.21282319)(663.12012056,631.24282919)
\curveto(663.20011493,631.34282306)(663.26511487,631.44282296)(663.31512056,631.54282919)
\curveto(663.37511476,631.64282276)(663.4301147,631.75282265)(663.48012056,631.87282919)
\curveto(663.51011462,631.94282246)(663.5301146,632.01782238)(663.54012056,632.09782919)
\lineto(663.60012056,632.33782919)
\lineto(663.60012056,632.42782919)
\curveto(663.61011452,632.45782194)(663.61511452,632.48782191)(663.61512056,632.51782919)
\curveto(663.6351145,632.58782181)(663.64011449,632.68282172)(663.63012056,632.80282919)
\curveto(663.6301145,632.93282147)(663.62011451,633.03282137)(663.60012056,633.10282919)
\curveto(663.58011455,633.18282122)(663.56011457,633.25782114)(663.54012056,633.32782919)
\curveto(663.5301146,633.40782099)(663.51011462,633.48782091)(663.48012056,633.56782919)
\curveto(663.37011476,633.80782059)(663.22011491,634.00782039)(663.03012056,634.16782919)
\curveto(662.85011528,634.33782006)(662.6301155,634.47781992)(662.37012056,634.58782919)
\curveto(662.30011583,634.60781979)(662.2301159,634.62281978)(662.16012056,634.63282919)
\curveto(662.09011604,634.65281975)(662.01511612,634.67281973)(661.93512056,634.69282919)
\curveto(661.85511628,634.71281969)(661.74511639,634.72281968)(661.60512056,634.72282919)
\curveto(661.47511666,634.72281968)(661.37011676,634.71281969)(661.29012056,634.69282919)
\curveto(661.2301169,634.68281972)(661.17511696,634.67781972)(661.12512056,634.67782919)
\curveto(661.07511706,634.67781972)(661.02511711,634.66781973)(660.97512056,634.64782919)
\curveto(660.87511726,634.60781979)(660.78011735,634.56781983)(660.69012056,634.52782919)
\curveto(660.61011752,634.48781991)(660.5301176,634.44281996)(660.45012056,634.39282919)
\curveto(660.42011771,634.37282003)(660.39011774,634.34782005)(660.36012056,634.31782919)
\curveto(660.34011779,634.28782011)(660.31511782,634.26282014)(660.28512056,634.24282919)
\lineto(660.21012056,634.16782919)
\curveto(660.18011795,634.14782025)(660.15511798,634.12782027)(660.13512056,634.10782919)
\lineto(659.98512056,633.89782919)
\curveto(659.94511819,633.83782056)(659.90011823,633.77282063)(659.85012056,633.70282919)
\curveto(659.79011834,633.61282079)(659.74011839,633.50782089)(659.70012056,633.38782919)
\curveto(659.67011846,633.27782112)(659.6351185,633.16782123)(659.59512056,633.05782919)
\curveto(659.55511858,632.94782145)(659.5301186,632.8028216)(659.52012056,632.62282919)
\curveto(659.51011862,632.45282195)(659.48011865,632.32782207)(659.43012056,632.24782919)
\curveto(659.38011875,632.16782223)(659.30511883,632.12282228)(659.20512056,632.11282919)
\curveto(659.10511903,632.1028223)(658.99511914,632.0978223)(658.87512056,632.09782919)
\curveto(658.8351193,632.0978223)(658.79511934,632.09282231)(658.75512056,632.08282919)
\curveto(658.71511942,632.08282232)(658.68011945,632.08782231)(658.65012056,632.09782919)
\curveto(658.60011953,632.11782228)(658.55011958,632.12782227)(658.50012056,632.12782919)
\curveto(658.46011967,632.12782227)(658.42011971,632.13782226)(658.38012056,632.15782919)
\curveto(658.29011984,632.21782218)(658.24511989,632.35282205)(658.24512056,632.56282919)
\lineto(658.24512056,632.68282919)
\curveto(658.25511988,632.74282166)(658.26011987,632.8028216)(658.26012056,632.86282919)
\curveto(658.27011986,632.93282147)(658.28011985,632.9978214)(658.29012056,633.05782919)
\curveto(658.31011982,633.16782123)(658.3301198,633.26782113)(658.35012056,633.35782919)
\curveto(658.37011976,633.45782094)(658.40011973,633.55282085)(658.44012056,633.64282919)
\curveto(658.46011967,633.71282069)(658.48011965,633.77282063)(658.50012056,633.82282919)
\lineto(658.56012056,634.00282919)
\curveto(658.68011945,634.26282014)(658.8351193,634.50781989)(659.02512056,634.73782919)
\curveto(659.22511891,634.96781943)(659.44011869,635.15281925)(659.67012056,635.29282919)
\curveto(659.78011835,635.37281903)(659.89511824,635.43781896)(660.01512056,635.48782919)
\lineto(660.40512056,635.63782919)
\curveto(660.51511762,635.68781871)(660.6301175,635.71781868)(660.75012056,635.72782919)
\curveto(660.87011726,635.74781865)(660.99511714,635.77281863)(661.12512056,635.80282919)
\curveto(661.19511694,635.8028186)(661.26011687,635.8028186)(661.32012056,635.80282919)
\curveto(661.38011675,635.81281859)(661.44511669,635.82281858)(661.51512056,635.83282919)
}
}
{
\newrgbcolor{curcolor}{0 0 0}
\pscustom[linestyle=none,fillstyle=solid,fillcolor=curcolor]
{
\newpath
\moveto(676.41972994,633.74782919)
\curveto(676.21971964,633.45782094)(676.00971985,633.17282123)(675.78972994,632.89282919)
\curveto(675.57972028,632.61282179)(675.37472048,632.32782207)(675.17472994,632.03782919)
\curveto(674.57472128,631.18782321)(673.96972189,630.34782405)(673.35972994,629.51782919)
\curveto(672.74972311,628.6978257)(672.14472371,627.86282654)(671.54472994,627.01282919)
\lineto(671.03472994,626.29282919)
\lineto(670.52472994,625.60282919)
\curveto(670.44472541,625.49282891)(670.36472549,625.37782902)(670.28472994,625.25782919)
\curveto(670.20472565,625.13782926)(670.10972575,625.04282936)(669.99972994,624.97282919)
\curveto(669.9597259,624.95282945)(669.89472596,624.93782946)(669.80472994,624.92782919)
\curveto(669.72472613,624.90782949)(669.63472622,624.8978295)(669.53472994,624.89782919)
\curveto(669.43472642,624.8978295)(669.33972652,624.9028295)(669.24972994,624.91282919)
\curveto(669.16972669,624.92282948)(669.10972675,624.94282946)(669.06972994,624.97282919)
\curveto(669.03972682,624.99282941)(669.01472684,625.02782937)(668.99472994,625.07782919)
\curveto(668.98472687,625.11782928)(668.98972687,625.16282924)(669.00972994,625.21282919)
\curveto(669.04972681,625.29282911)(669.09472676,625.36782903)(669.14472994,625.43782919)
\curveto(669.20472665,625.51782888)(669.2597266,625.5978288)(669.30972994,625.67782919)
\curveto(669.54972631,626.01782838)(669.79472606,626.35282805)(670.04472994,626.68282919)
\curveto(670.29472556,627.01282739)(670.53472532,627.34782705)(670.76472994,627.68782919)
\curveto(670.92472493,627.90782649)(671.08472477,628.12282628)(671.24472994,628.33282919)
\curveto(671.40472445,628.54282586)(671.56472429,628.75782564)(671.72472994,628.97782919)
\curveto(672.08472377,629.4978249)(672.44972341,630.00782439)(672.81972994,630.50782919)
\curveto(673.18972267,631.00782339)(673.5597223,631.51782288)(673.92972994,632.03782919)
\curveto(674.06972179,632.23782216)(674.20972165,632.43282197)(674.34972994,632.62282919)
\curveto(674.49972136,632.81282159)(674.64472121,633.00782139)(674.78472994,633.20782919)
\curveto(674.99472086,633.50782089)(675.20972065,633.80782059)(675.42972994,634.10782919)
\lineto(676.08972994,635.00782919)
\lineto(676.26972994,635.27782919)
\lineto(676.47972994,635.54782919)
\lineto(676.59972994,635.72782919)
\curveto(676.64971921,635.78781861)(676.69971916,635.84281856)(676.74972994,635.89282919)
\curveto(676.81971904,635.94281846)(676.89471896,635.97781842)(676.97472994,635.99782919)
\curveto(676.99471886,636.00781839)(677.01971884,636.00781839)(677.04972994,635.99782919)
\curveto(677.08971877,635.9978184)(677.11971874,636.00781839)(677.13972994,636.02782919)
\curveto(677.2597186,636.02781837)(677.39471846,636.02281838)(677.54472994,636.01282919)
\curveto(677.69471816,636.01281839)(677.78471807,635.96781843)(677.81472994,635.87782919)
\curveto(677.83471802,635.84781855)(677.83971802,635.81281859)(677.82972994,635.77282919)
\curveto(677.81971804,635.73281867)(677.80471805,635.7028187)(677.78472994,635.68282919)
\curveto(677.74471811,635.6028188)(677.70471815,635.53281887)(677.66472994,635.47282919)
\curveto(677.62471823,635.41281899)(677.57971828,635.35281905)(677.52972994,635.29282919)
\lineto(676.95972994,634.51282919)
\curveto(676.77971908,634.26282014)(676.59971926,634.00782039)(676.41972994,633.74782919)
\moveto(669.56472994,629.84782919)
\curveto(669.51472634,629.86782453)(669.46472639,629.87282453)(669.41472994,629.86282919)
\curveto(669.36472649,629.85282455)(669.31472654,629.85782454)(669.26472994,629.87782919)
\curveto(669.1547267,629.8978245)(669.04972681,629.91782448)(668.94972994,629.93782919)
\curveto(668.859727,629.96782443)(668.76472709,630.00782439)(668.66472994,630.05782919)
\curveto(668.33472752,630.1978242)(668.07972778,630.39282401)(667.89972994,630.64282919)
\curveto(667.71972814,630.9028235)(667.57472828,631.21282319)(667.46472994,631.57282919)
\curveto(667.43472842,631.65282275)(667.41472844,631.73282267)(667.40472994,631.81282919)
\curveto(667.39472846,631.9028225)(667.37972848,631.98782241)(667.35972994,632.06782919)
\curveto(667.34972851,632.11782228)(667.34472851,632.18282222)(667.34472994,632.26282919)
\curveto(667.33472852,632.29282211)(667.32972853,632.32282208)(667.32972994,632.35282919)
\curveto(667.32972853,632.39282201)(667.32472853,632.42782197)(667.31472994,632.45782919)
\lineto(667.31472994,632.60782919)
\curveto(667.30472855,632.65782174)(667.29972856,632.71782168)(667.29972994,632.78782919)
\curveto(667.29972856,632.86782153)(667.30472855,632.93282147)(667.31472994,632.98282919)
\lineto(667.31472994,633.14782919)
\curveto(667.33472852,633.1978212)(667.33972852,633.24282116)(667.32972994,633.28282919)
\curveto(667.32972853,633.33282107)(667.33472852,633.37782102)(667.34472994,633.41782919)
\curveto(667.3547285,633.45782094)(667.3597285,633.49282091)(667.35972994,633.52282919)
\curveto(667.3597285,633.56282084)(667.36472849,633.6028208)(667.37472994,633.64282919)
\curveto(667.40472845,633.75282065)(667.42472843,633.86282054)(667.43472994,633.97282919)
\curveto(667.4547284,634.09282031)(667.48972837,634.20782019)(667.53972994,634.31782919)
\curveto(667.67972818,634.65781974)(667.83972802,634.93281947)(668.01972994,635.14282919)
\curveto(668.20972765,635.36281904)(668.47972738,635.54281886)(668.82972994,635.68282919)
\curveto(668.90972695,635.71281869)(668.99472686,635.73281867)(669.08472994,635.74282919)
\curveto(669.17472668,635.76281864)(669.26972659,635.78281862)(669.36972994,635.80282919)
\curveto(669.39972646,635.81281859)(669.4547264,635.81281859)(669.53472994,635.80282919)
\curveto(669.61472624,635.8028186)(669.66472619,635.81281859)(669.68472994,635.83282919)
\curveto(670.24472561,635.84281856)(670.69472516,635.73281867)(671.03472994,635.50282919)
\curveto(671.38472447,635.27281913)(671.64472421,634.96781943)(671.81472994,634.58782919)
\curveto(671.854724,634.4978199)(671.88972397,634.40282)(671.91972994,634.30282919)
\curveto(671.94972391,634.2028202)(671.97472388,634.1028203)(671.99472994,634.00282919)
\curveto(672.01472384,633.97282043)(672.01972384,633.94282046)(672.00972994,633.91282919)
\curveto(672.00972385,633.88282052)(672.01472384,633.85282055)(672.02472994,633.82282919)
\curveto(672.0547238,633.71282069)(672.07472378,633.58782081)(672.08472994,633.44782919)
\curveto(672.09472376,633.31782108)(672.10472375,633.18282122)(672.11472994,633.04282919)
\lineto(672.11472994,632.87782919)
\curveto(672.12472373,632.81782158)(672.12472373,632.76282164)(672.11472994,632.71282919)
\curveto(672.10472375,632.66282174)(672.09972376,632.61282179)(672.09972994,632.56282919)
\lineto(672.09972994,632.42782919)
\curveto(672.08972377,632.38782201)(672.08472377,632.34782205)(672.08472994,632.30782919)
\curveto(672.09472376,632.26782213)(672.08972377,632.22282218)(672.06972994,632.17282919)
\curveto(672.04972381,632.06282234)(672.02972383,631.95782244)(672.00972994,631.85782919)
\curveto(671.99972386,631.75782264)(671.97972388,631.65782274)(671.94972994,631.55782919)
\curveto(671.81972404,631.1978232)(671.6547242,630.88282352)(671.45472994,630.61282919)
\curveto(671.2547246,630.34282406)(670.97972488,630.13782426)(670.62972994,629.99782919)
\curveto(670.54972531,629.96782443)(670.46472539,629.94282446)(670.37472994,629.92282919)
\lineto(670.10472994,629.86282919)
\curveto(670.0547258,629.85282455)(670.00972585,629.84782455)(669.96972994,629.84782919)
\curveto(669.92972593,629.85782454)(669.88972597,629.85782454)(669.84972994,629.84782919)
\curveto(669.74972611,629.82782457)(669.6547262,629.82782457)(669.56472994,629.84782919)
\moveto(668.72472994,631.24282919)
\curveto(668.76472709,631.17282323)(668.80472705,631.10782329)(668.84472994,631.04782919)
\curveto(668.88472697,630.9978234)(668.93472692,630.94782345)(668.99472994,630.89782919)
\lineto(669.14472994,630.77782919)
\curveto(669.20472665,630.74782365)(669.26972659,630.72282368)(669.33972994,630.70282919)
\curveto(669.37972648,630.68282372)(669.41472644,630.67282373)(669.44472994,630.67282919)
\curveto(669.48472637,630.68282372)(669.52472633,630.67782372)(669.56472994,630.65782919)
\curveto(669.59472626,630.65782374)(669.63472622,630.65282375)(669.68472994,630.64282919)
\curveto(669.73472612,630.64282376)(669.77472608,630.64782375)(669.80472994,630.65782919)
\lineto(670.02972994,630.70282919)
\curveto(670.27972558,630.78282362)(670.46472539,630.90782349)(670.58472994,631.07782919)
\curveto(670.66472519,631.17782322)(670.73472512,631.30782309)(670.79472994,631.46782919)
\curveto(670.87472498,631.64782275)(670.93472492,631.87282253)(670.97472994,632.14282919)
\curveto(671.01472484,632.42282198)(671.02972483,632.7028217)(671.01972994,632.98282919)
\curveto(671.00972485,633.27282113)(670.97972488,633.54782085)(670.92972994,633.80782919)
\curveto(670.87972498,634.06782033)(670.80472505,634.27782012)(670.70472994,634.43782919)
\curveto(670.58472527,634.63781976)(670.43472542,634.78781961)(670.25472994,634.88782919)
\curveto(670.17472568,634.93781946)(670.08472577,634.96781943)(669.98472994,634.97782919)
\curveto(669.88472597,634.9978194)(669.77972608,635.00781939)(669.66972994,635.00782919)
\curveto(669.64972621,634.9978194)(669.62472623,634.99281941)(669.59472994,634.99282919)
\curveto(669.57472628,635.0028194)(669.5547263,635.0028194)(669.53472994,634.99282919)
\curveto(669.48472637,634.98281942)(669.43972642,634.97281943)(669.39972994,634.96282919)
\curveto(669.3597265,634.96281944)(669.31972654,634.95281945)(669.27972994,634.93282919)
\curveto(669.09972676,634.85281955)(668.94972691,634.73281967)(668.82972994,634.57282919)
\curveto(668.71972714,634.41281999)(668.62972723,634.23282017)(668.55972994,634.03282919)
\curveto(668.49972736,633.84282056)(668.4547274,633.61782078)(668.42472994,633.35782919)
\curveto(668.40472745,633.0978213)(668.39972746,632.83282157)(668.40972994,632.56282919)
\curveto(668.41972744,632.3028221)(668.44972741,632.05282235)(668.49972994,631.81282919)
\curveto(668.5597273,631.58282282)(668.63472722,631.39282301)(668.72472994,631.24282919)
\moveto(679.52472994,628.25782919)
\curveto(679.53471632,628.20782619)(679.53971632,628.11782628)(679.53972994,627.98782919)
\curveto(679.53971632,627.85782654)(679.52971633,627.76782663)(679.50972994,627.71782919)
\curveto(679.48971637,627.66782673)(679.48471637,627.61282679)(679.49472994,627.55282919)
\curveto(679.50471635,627.5028269)(679.50471635,627.45282695)(679.49472994,627.40282919)
\curveto(679.4547164,627.26282714)(679.42471643,627.12782727)(679.40472994,626.99782919)
\curveto(679.39471646,626.86782753)(679.36471649,626.74782765)(679.31472994,626.63782919)
\curveto(679.17471668,626.28782811)(679.00971685,625.99282841)(678.81972994,625.75282919)
\curveto(678.62971723,625.52282888)(678.3597175,625.33782906)(678.00972994,625.19782919)
\curveto(677.92971793,625.16782923)(677.84471801,625.14782925)(677.75472994,625.13782919)
\curveto(677.66471819,625.11782928)(677.57971828,625.0978293)(677.49972994,625.07782919)
\curveto(677.44971841,625.06782933)(677.39971846,625.06282934)(677.34972994,625.06282919)
\curveto(677.29971856,625.06282934)(677.24971861,625.05782934)(677.19972994,625.04782919)
\curveto(677.16971869,625.03782936)(677.11971874,625.03782936)(677.04972994,625.04782919)
\curveto(676.97971888,625.04782935)(676.92971893,625.05282935)(676.89972994,625.06282919)
\curveto(676.83971902,625.08282932)(676.77971908,625.09282931)(676.71972994,625.09282919)
\curveto(676.66971919,625.08282932)(676.61971924,625.08782931)(676.56972994,625.10782919)
\curveto(676.47971938,625.12782927)(676.38971947,625.15282925)(676.29972994,625.18282919)
\curveto(676.21971964,625.2028292)(676.13971972,625.23282917)(676.05972994,625.27282919)
\curveto(675.73972012,625.41282899)(675.48972037,625.60782879)(675.30972994,625.85782919)
\curveto(675.12972073,626.11782828)(674.97972088,626.42282798)(674.85972994,626.77282919)
\curveto(674.83972102,626.85282755)(674.82472103,626.93782746)(674.81472994,627.02782919)
\curveto(674.80472105,627.11782728)(674.78972107,627.2028272)(674.76972994,627.28282919)
\curveto(674.7597211,627.31282709)(674.7547211,627.34282706)(674.75472994,627.37282919)
\lineto(674.75472994,627.47782919)
\curveto(674.73472112,627.55782684)(674.72472113,627.63782676)(674.72472994,627.71782919)
\lineto(674.72472994,627.85282919)
\curveto(674.70472115,627.95282645)(674.70472115,628.05282635)(674.72472994,628.15282919)
\lineto(674.72472994,628.33282919)
\curveto(674.73472112,628.38282602)(674.73972112,628.42782597)(674.73972994,628.46782919)
\curveto(674.73972112,628.51782588)(674.74472111,628.56282584)(674.75472994,628.60282919)
\curveto(674.76472109,628.64282576)(674.76972109,628.67782572)(674.76972994,628.70782919)
\curveto(674.76972109,628.74782565)(674.77472108,628.78782561)(674.78472994,628.82782919)
\lineto(674.84472994,629.15782919)
\curveto(674.86472099,629.27782512)(674.89472096,629.38782501)(674.93472994,629.48782919)
\curveto(675.07472078,629.81782458)(675.23472062,630.09282431)(675.41472994,630.31282919)
\curveto(675.60472025,630.54282386)(675.86471999,630.72782367)(676.19472994,630.86782919)
\curveto(676.27471958,630.90782349)(676.3597195,630.93282347)(676.44972994,630.94282919)
\lineto(676.74972994,631.00282919)
\lineto(676.88472994,631.00282919)
\curveto(676.93471892,631.01282339)(676.98471887,631.01782338)(677.03472994,631.01782919)
\curveto(677.60471825,631.03782336)(678.06471779,630.93282347)(678.41472994,630.70282919)
\curveto(678.77471708,630.48282392)(679.03971682,630.18282422)(679.20972994,629.80282919)
\curveto(679.2597166,629.7028247)(679.29971656,629.6028248)(679.32972994,629.50282919)
\curveto(679.3597165,629.402825)(679.38971647,629.2978251)(679.41972994,629.18782919)
\curveto(679.42971643,629.14782525)(679.43471642,629.11282529)(679.43472994,629.08282919)
\curveto(679.43471642,629.06282534)(679.43971642,629.03282537)(679.44972994,628.99282919)
\curveto(679.46971639,628.92282548)(679.47971638,628.84782555)(679.47972994,628.76782919)
\curveto(679.47971638,628.68782571)(679.48971637,628.60782579)(679.50972994,628.52782919)
\curveto(679.50971635,628.47782592)(679.50971635,628.43282597)(679.50972994,628.39282919)
\curveto(679.50971635,628.35282605)(679.51471634,628.30782609)(679.52472994,628.25782919)
\moveto(678.41472994,627.82282919)
\curveto(678.42471743,627.87282653)(678.42971743,627.94782645)(678.42972994,628.04782919)
\curveto(678.43971742,628.14782625)(678.43471742,628.22282618)(678.41472994,628.27282919)
\curveto(678.39471746,628.33282607)(678.38971747,628.38782601)(678.39972994,628.43782919)
\curveto(678.41971744,628.4978259)(678.41971744,628.55782584)(678.39972994,628.61782919)
\curveto(678.38971747,628.64782575)(678.38471747,628.68282572)(678.38472994,628.72282919)
\curveto(678.38471747,628.76282564)(678.37971748,628.8028256)(678.36972994,628.84282919)
\curveto(678.34971751,628.92282548)(678.32971753,628.9978254)(678.30972994,629.06782919)
\curveto(678.29971756,629.14782525)(678.28471757,629.22782517)(678.26472994,629.30782919)
\curveto(678.23471762,629.36782503)(678.20971765,629.42782497)(678.18972994,629.48782919)
\curveto(678.16971769,629.54782485)(678.13971772,629.60782479)(678.09972994,629.66782919)
\curveto(677.99971786,629.83782456)(677.86971799,629.97282443)(677.70972994,630.07282919)
\curveto(677.62971823,630.12282428)(677.53471832,630.15782424)(677.42472994,630.17782919)
\curveto(677.31471854,630.1978242)(677.18971867,630.20782419)(677.04972994,630.20782919)
\curveto(677.02971883,630.1978242)(677.00471885,630.19282421)(676.97472994,630.19282919)
\curveto(676.94471891,630.2028242)(676.91471894,630.2028242)(676.88472994,630.19282919)
\lineto(676.73472994,630.13282919)
\curveto(676.68471917,630.12282428)(676.63971922,630.10782429)(676.59972994,630.08782919)
\curveto(676.40971945,629.97782442)(676.26471959,629.83282457)(676.16472994,629.65282919)
\curveto(676.07471978,629.47282493)(675.99471986,629.26782513)(675.92472994,629.03782919)
\curveto(675.88471997,628.90782549)(675.86471999,628.77282563)(675.86472994,628.63282919)
\curveto(675.86471999,628.5028259)(675.85472,628.35782604)(675.83472994,628.19782919)
\curveto(675.82472003,628.14782625)(675.81472004,628.08782631)(675.80472994,628.01782919)
\curveto(675.80472005,627.94782645)(675.81472004,627.88782651)(675.83472994,627.83782919)
\lineto(675.83472994,627.67282919)
\lineto(675.83472994,627.49282919)
\curveto(675.84472001,627.44282696)(675.85472,627.38782701)(675.86472994,627.32782919)
\curveto(675.87471998,627.27782712)(675.87971998,627.22282718)(675.87972994,627.16282919)
\curveto(675.88971997,627.1028273)(675.90471995,627.04782735)(675.92472994,626.99782919)
\curveto(675.97471988,626.80782759)(676.03471982,626.63282777)(676.10472994,626.47282919)
\curveto(676.17471968,626.31282809)(676.27971958,626.18282822)(676.41972994,626.08282919)
\curveto(676.54971931,625.98282842)(676.68971917,625.91282849)(676.83972994,625.87282919)
\curveto(676.86971899,625.86282854)(676.89471896,625.85782854)(676.91472994,625.85782919)
\curveto(676.94471891,625.86782853)(676.97471888,625.86782853)(677.00472994,625.85782919)
\curveto(677.02471883,625.85782854)(677.0547188,625.85282855)(677.09472994,625.84282919)
\curveto(677.13471872,625.84282856)(677.16971869,625.84782855)(677.19972994,625.85782919)
\curveto(677.23971862,625.86782853)(677.27971858,625.87282853)(677.31972994,625.87282919)
\curveto(677.3597185,625.87282853)(677.39971846,625.88282852)(677.43972994,625.90282919)
\curveto(677.67971818,625.98282842)(677.87471798,626.11782828)(678.02472994,626.30782919)
\curveto(678.14471771,626.48782791)(678.23471762,626.69282771)(678.29472994,626.92282919)
\curveto(678.31471754,626.99282741)(678.32971753,627.06282734)(678.33972994,627.13282919)
\curveto(678.34971751,627.21282719)(678.36471749,627.29282711)(678.38472994,627.37282919)
\curveto(678.38471747,627.43282697)(678.38971747,627.47782692)(678.39972994,627.50782919)
\curveto(678.39971746,627.52782687)(678.39971746,627.55282685)(678.39972994,627.58282919)
\curveto(678.39971746,627.62282678)(678.40471745,627.65282675)(678.41472994,627.67282919)
\lineto(678.41472994,627.82282919)
}
}
{
\newrgbcolor{curcolor}{0 0 0}
\pscustom[linestyle=none,fillstyle=solid,fillcolor=curcolor]
{
\newpath
\moveto(547.2779233,411.87430136)
\curveto(547.28791558,411.83429831)(547.28791558,411.78429836)(547.2779233,411.72430136)
\curveto(547.27791559,411.66429848)(547.27291559,411.61429853)(547.2629233,411.57430136)
\curveto(547.2629156,411.53429861)(547.25791561,411.49429865)(547.2479233,411.45430136)
\lineto(547.2479233,411.34930136)
\curveto(547.22791564,411.26929887)(547.21291565,411.18929895)(547.2029233,411.10930136)
\curveto(547.19291567,411.02929911)(547.17291569,410.95429919)(547.1429233,410.88430136)
\curveto(547.12291574,410.80429934)(547.10291576,410.72929941)(547.0829233,410.65930136)
\curveto(547.0629158,410.58929955)(547.03291583,410.51429963)(546.9929233,410.43430136)
\curveto(546.81291605,410.01430013)(546.55791631,409.67430047)(546.2279233,409.41430136)
\curveto(545.89791697,409.15430099)(545.50791736,408.94930119)(545.0579233,408.79930136)
\curveto(544.93791793,408.75930138)(544.81291805,408.73430141)(544.6829233,408.72430136)
\curveto(544.5629183,408.70430144)(544.43791843,408.67930146)(544.3079233,408.64930136)
\curveto(544.24791862,408.6393015)(544.18291868,408.63430151)(544.1129233,408.63430136)
\curveto(544.05291881,408.63430151)(543.98791888,408.62930151)(543.9179233,408.61930136)
\lineto(543.7979233,408.61930136)
\lineto(543.6029233,408.61930136)
\curveto(543.54291932,408.60930153)(543.48791938,408.61430153)(543.4379233,408.63430136)
\curveto(543.3679195,408.65430149)(543.30291956,408.65930148)(543.2429233,408.64930136)
\curveto(543.18291968,408.6393015)(543.12291974,408.6443015)(543.0629233,408.66430136)
\curveto(543.01291985,408.67430147)(542.9679199,408.67930146)(542.9279233,408.67930136)
\curveto(542.88791998,408.67930146)(542.84292002,408.68930145)(542.7929233,408.70930136)
\curveto(542.71292015,408.72930141)(542.63792023,408.74930139)(542.5679233,408.76930136)
\curveto(542.49792037,408.77930136)(542.42792044,408.79430135)(542.3579233,408.81430136)
\curveto(541.87792099,408.98430116)(541.47792139,409.19430095)(541.1579233,409.44430136)
\curveto(540.84792202,409.70430044)(540.59792227,410.05930008)(540.4079233,410.50930136)
\curveto(540.37792249,410.56929957)(540.35292251,410.62929951)(540.3329233,410.68930136)
\curveto(540.32292254,410.75929938)(540.30792256,410.83429931)(540.2879233,410.91430136)
\curveto(540.2679226,410.97429917)(540.25292261,411.0392991)(540.2429233,411.10930136)
\curveto(540.23292263,411.17929896)(540.21792265,411.24929889)(540.1979233,411.31930136)
\curveto(540.18792268,411.36929877)(540.18292268,411.40929873)(540.1829233,411.43930136)
\lineto(540.1829233,411.55930136)
\curveto(540.17292269,411.59929854)(540.1629227,411.64929849)(540.1529233,411.70930136)
\curveto(540.15292271,411.76929837)(540.15792271,411.81929832)(540.1679233,411.85930136)
\lineto(540.1679233,411.99430136)
\curveto(540.17792269,412.0442981)(540.18292268,412.09429805)(540.1829233,412.14430136)
\curveto(540.20292266,412.2442979)(540.21792265,412.3392978)(540.2279233,412.42930136)
\curveto(540.23792263,412.52929761)(540.25792261,412.62429752)(540.2879233,412.71430136)
\curveto(540.33792253,412.86429728)(540.39292247,413.00429714)(540.4529233,413.13430136)
\curveto(540.51292235,413.26429688)(540.58292228,413.38429676)(540.6629233,413.49430136)
\curveto(540.69292217,413.5442966)(540.72292214,413.58429656)(540.7529233,413.61430136)
\curveto(540.79292207,413.6442965)(540.82792204,413.67929646)(540.8579233,413.71930136)
\curveto(540.91792195,413.79929634)(540.98792188,413.86929627)(541.0679233,413.92930136)
\curveto(541.12792174,413.97929616)(541.18792168,414.02429612)(541.2479233,414.06430136)
\lineto(541.4579233,414.21430136)
\curveto(541.50792136,414.25429589)(541.55792131,414.28929585)(541.6079233,414.31930136)
\curveto(541.65792121,414.35929578)(541.69292117,414.41429573)(541.7129233,414.48430136)
\curveto(541.71292115,414.51429563)(541.70292116,414.5392956)(541.6829233,414.55930136)
\curveto(541.67292119,414.58929555)(541.6629212,414.61429553)(541.6529233,414.63430136)
\curveto(541.61292125,414.68429546)(541.5629213,414.72929541)(541.5029233,414.76930136)
\curveto(541.45292141,414.81929532)(541.40292146,414.86429528)(541.3529233,414.90430136)
\curveto(541.31292155,414.93429521)(541.2629216,414.98929515)(541.2029233,415.06930136)
\curveto(541.18292168,415.09929504)(541.15292171,415.12429502)(541.1129233,415.14430136)
\curveto(541.08292178,415.17429497)(541.05792181,415.20929493)(541.0379233,415.24930136)
\curveto(540.867922,415.45929468)(540.73792213,415.70429444)(540.6479233,415.98430136)
\curveto(540.62792224,416.06429408)(540.61292225,416.144294)(540.6029233,416.22430136)
\curveto(540.59292227,416.30429384)(540.57792229,416.38429376)(540.5579233,416.46430136)
\curveto(540.53792233,416.51429363)(540.52792234,416.57929356)(540.5279233,416.65930136)
\curveto(540.52792234,416.74929339)(540.53792233,416.81929332)(540.5579233,416.86930136)
\curveto(540.55792231,416.96929317)(540.5629223,417.0392931)(540.5729233,417.07930136)
\curveto(540.59292227,417.15929298)(540.60792226,417.22929291)(540.6179233,417.28930136)
\curveto(540.62792224,417.35929278)(540.64292222,417.42929271)(540.6629233,417.49930136)
\curveto(540.81292205,417.92929221)(541.02792184,418.27429187)(541.3079233,418.53430136)
\curveto(541.59792127,418.79429135)(541.94792092,419.00929113)(542.3579233,419.17930136)
\curveto(542.4679204,419.22929091)(542.58292028,419.25929088)(542.7029233,419.26930136)
\curveto(542.83292003,419.28929085)(542.9629199,419.31929082)(543.0929233,419.35930136)
\curveto(543.17291969,419.35929078)(543.24291962,419.35929078)(543.3029233,419.35930136)
\curveto(543.37291949,419.36929077)(543.44791942,419.37929076)(543.5279233,419.38930136)
\curveto(544.31791855,419.40929073)(544.97291789,419.27929086)(545.4929233,418.99930136)
\curveto(546.02291684,418.71929142)(546.40291646,418.30929183)(546.6329233,417.76930136)
\curveto(546.74291612,417.5392926)(546.81291605,417.25429289)(546.8429233,416.91430136)
\curveto(546.88291598,416.58429356)(546.85291601,416.27929386)(546.7529233,415.99930136)
\curveto(546.71291615,415.86929427)(546.6629162,415.74929439)(546.6029233,415.63930136)
\curveto(546.55291631,415.52929461)(546.49291637,415.42429472)(546.4229233,415.32430136)
\curveto(546.40291646,415.28429486)(546.37291649,415.24929489)(546.3329233,415.21930136)
\lineto(546.2429233,415.12930136)
\curveto(546.19291667,415.0392951)(546.13291673,414.97429517)(546.0629233,414.93430136)
\curveto(546.01291685,414.88429526)(545.95791691,414.83429531)(545.8979233,414.78430136)
\curveto(545.84791702,414.7442954)(545.80291706,414.69929544)(545.7629233,414.64930136)
\curveto(545.74291712,414.62929551)(545.72291714,414.60429554)(545.7029233,414.57430136)
\curveto(545.69291717,414.55429559)(545.69291717,414.52929561)(545.7029233,414.49930136)
\curveto(545.71291715,414.44929569)(545.74291712,414.39929574)(545.7929233,414.34930136)
\curveto(545.84291702,414.30929583)(545.89791697,414.26929587)(545.9579233,414.22930136)
\lineto(546.1379233,414.10930136)
\curveto(546.19791667,414.07929606)(546.24791662,414.04929609)(546.2879233,414.01930136)
\curveto(546.61791625,413.77929636)(546.867916,413.46929667)(547.0379233,413.08930136)
\curveto(547.07791579,413.00929713)(547.10791576,412.92429722)(547.1279233,412.83430136)
\curveto(547.15791571,412.7442974)(547.18291568,412.65429749)(547.2029233,412.56430136)
\curveto(547.21291565,412.51429763)(547.22291564,412.45929768)(547.2329233,412.39930136)
\lineto(547.2629233,412.24930136)
\curveto(547.27291559,412.18929795)(547.27291559,412.12429802)(547.2629233,412.05430136)
\curveto(547.25291561,411.99429815)(547.25791561,411.93429821)(547.2779233,411.87430136)
\moveto(541.8929233,416.91430136)
\curveto(541.862921,416.80429334)(541.85792101,416.66429348)(541.8779233,416.49430136)
\curveto(541.89792097,416.33429381)(541.92292094,416.20929393)(541.9529233,416.11930136)
\curveto(542.0629208,415.79929434)(542.21292065,415.55429459)(542.4029233,415.38430136)
\curveto(542.59292027,415.22429492)(542.85792001,415.09429505)(543.1979233,414.99430136)
\curveto(543.32791954,414.96429518)(543.49291937,414.9392952)(543.6929233,414.91930136)
\curveto(543.89291897,414.90929523)(544.0629188,414.92429522)(544.2029233,414.96430136)
\curveto(544.49291837,415.0442951)(544.73291813,415.15429499)(544.9229233,415.29430136)
\curveto(545.12291774,415.4442947)(545.27791759,415.6442945)(545.3879233,415.89430136)
\curveto(545.40791746,415.9442942)(545.41791745,415.98929415)(545.4179233,416.02930136)
\curveto(545.42791744,416.06929407)(545.44291742,416.11429403)(545.4629233,416.16430136)
\curveto(545.49291737,416.27429387)(545.51291735,416.41429373)(545.5229233,416.58430136)
\curveto(545.53291733,416.75429339)(545.52291734,416.89929324)(545.4929233,417.01930136)
\curveto(545.47291739,417.10929303)(545.44791742,417.19429295)(545.4179233,417.27430136)
\curveto(545.39791747,417.35429279)(545.3629175,417.43429271)(545.3129233,417.51430136)
\curveto(545.14291772,417.78429236)(544.91791795,417.97929216)(544.6379233,418.09930136)
\curveto(544.3679185,418.21929192)(544.00791886,418.27929186)(543.5579233,418.27930136)
\curveto(543.53791933,418.25929188)(543.50791936,418.25429189)(543.4679233,418.26430136)
\curveto(543.42791944,418.27429187)(543.39291947,418.27429187)(543.3629233,418.26430136)
\curveto(543.31291955,418.2442919)(543.25791961,418.22929191)(543.1979233,418.21930136)
\curveto(543.14791972,418.21929192)(543.09791977,418.20929193)(543.0479233,418.18930136)
\curveto(542.80792006,418.09929204)(542.59792027,417.98429216)(542.4179233,417.84430136)
\curveto(542.23792063,417.71429243)(542.09792077,417.53429261)(541.9979233,417.30430136)
\curveto(541.97792089,417.2442929)(541.95792091,417.17929296)(541.9379233,417.10930136)
\curveto(541.92792094,417.04929309)(541.91292095,416.98429316)(541.8929233,416.91430136)
\moveto(545.9129233,411.37930136)
\curveto(545.9629169,411.56929857)(545.9679169,411.77429837)(545.9279233,411.99430136)
\curveto(545.89791697,412.21429793)(545.85291701,412.39429775)(545.7929233,412.53430136)
\curveto(545.62291724,412.90429724)(545.3629175,413.20929693)(545.0129233,413.44930136)
\curveto(544.67291819,413.68929645)(544.23791863,413.80929633)(543.7079233,413.80930136)
\curveto(543.67791919,413.78929635)(543.63791923,413.78429636)(543.5879233,413.79430136)
\curveto(543.53791933,413.81429633)(543.49791937,413.81929632)(543.4679233,413.80930136)
\lineto(543.1979233,413.74930136)
\curveto(543.11791975,413.7392964)(543.03791983,413.72429642)(542.9579233,413.70430136)
\curveto(542.65792021,413.59429655)(542.39292047,413.44929669)(542.1629233,413.26930136)
\curveto(541.94292092,413.08929705)(541.77292109,412.85929728)(541.6529233,412.57930136)
\curveto(541.62292124,412.49929764)(541.59792127,412.41929772)(541.5779233,412.33930136)
\curveto(541.55792131,412.25929788)(541.53792133,412.17429797)(541.5179233,412.08430136)
\curveto(541.48792138,411.96429818)(541.47792139,411.81429833)(541.4879233,411.63430136)
\curveto(541.50792136,411.45429869)(541.53292133,411.31429883)(541.5629233,411.21430136)
\curveto(541.58292128,411.16429898)(541.59292127,411.11929902)(541.5929233,411.07930136)
\curveto(541.60292126,411.04929909)(541.61792125,411.00929913)(541.6379233,410.95930136)
\curveto(541.73792113,410.7392994)(541.867921,410.5392996)(542.0279233,410.35930136)
\curveto(542.19792067,410.17929996)(542.39292047,410.0443001)(542.6129233,409.95430136)
\curveto(542.68292018,409.91430023)(542.77792009,409.87930026)(542.8979233,409.84930136)
\curveto(543.11791975,409.75930038)(543.37291949,409.71430043)(543.6629233,409.71430136)
\lineto(543.9479233,409.71430136)
\curveto(544.04791882,409.73430041)(544.14291872,409.74930039)(544.2329233,409.75930136)
\curveto(544.32291854,409.76930037)(544.41291845,409.78930035)(544.5029233,409.81930136)
\curveto(544.7629181,409.89930024)(545.00291786,410.02930011)(545.2229233,410.20930136)
\curveto(545.45291741,410.39929974)(545.62291724,410.61429953)(545.7329233,410.85430136)
\curveto(545.77291709,410.93429921)(545.80291706,411.01429913)(545.8229233,411.09430136)
\curveto(545.85291701,411.18429896)(545.88291698,411.27929886)(545.9129233,411.37930136)
}
}
{
\newrgbcolor{curcolor}{0 0 0}
\pscustom[linestyle=none,fillstyle=solid,fillcolor=curcolor]
{
\newpath
\moveto(552.67253267,419.40430136)
\curveto(552.77252782,419.40429074)(552.86752772,419.39429075)(552.95753267,419.37430136)
\curveto(553.04752754,419.36429078)(553.11252748,419.33429081)(553.15253267,419.28430136)
\curveto(553.21252738,419.20429094)(553.24252735,419.09929104)(553.24253267,418.96930136)
\lineto(553.24253267,418.57930136)
\lineto(553.24253267,417.07930136)
\lineto(553.24253267,410.68930136)
\lineto(553.24253267,409.51930136)
\lineto(553.24253267,409.20430136)
\curveto(553.25252734,409.10430104)(553.23752735,409.02430112)(553.19753267,408.96430136)
\curveto(553.14752744,408.88430126)(553.07252752,408.83430131)(552.97253267,408.81430136)
\curveto(552.88252771,408.80430134)(552.77252782,408.79930134)(552.64253267,408.79930136)
\lineto(552.41753267,408.79930136)
\curveto(552.33752825,408.81930132)(552.26752832,408.83430131)(552.20753267,408.84430136)
\curveto(552.14752844,408.86430128)(552.09752849,408.90430124)(552.05753267,408.96430136)
\curveto(552.01752857,409.02430112)(551.99752859,409.09930104)(551.99753267,409.18930136)
\lineto(551.99753267,409.48930136)
\lineto(551.99753267,410.58430136)
\lineto(551.99753267,415.92430136)
\curveto(551.97752861,416.01429413)(551.96252863,416.08929405)(551.95253267,416.14930136)
\curveto(551.95252864,416.21929392)(551.92252867,416.27929386)(551.86253267,416.32930136)
\curveto(551.7925288,416.37929376)(551.70252889,416.40429374)(551.59253267,416.40430136)
\curveto(551.4925291,416.41429373)(551.38252921,416.41929372)(551.26253267,416.41930136)
\lineto(550.12253267,416.41930136)
\lineto(549.62753267,416.41930136)
\curveto(549.46753112,416.42929371)(549.35753123,416.48929365)(549.29753267,416.59930136)
\curveto(549.27753131,416.62929351)(549.26753132,416.65929348)(549.26753267,416.68930136)
\curveto(549.26753132,416.72929341)(549.26253133,416.77429337)(549.25253267,416.82430136)
\curveto(549.23253136,416.9442932)(549.23753135,417.05429309)(549.26753267,417.15430136)
\curveto(549.30753128,417.25429289)(549.36253123,417.32429282)(549.43253267,417.36430136)
\curveto(549.51253108,417.41429273)(549.63253096,417.4392927)(549.79253267,417.43930136)
\curveto(549.95253064,417.4392927)(550.0875305,417.45429269)(550.19753267,417.48430136)
\curveto(550.24753034,417.49429265)(550.30253029,417.49929264)(550.36253267,417.49930136)
\curveto(550.42253017,417.50929263)(550.48253011,417.52429262)(550.54253267,417.54430136)
\curveto(550.6925299,417.59429255)(550.83752975,417.6442925)(550.97753267,417.69430136)
\curveto(551.11752947,417.75429239)(551.25252934,417.82429232)(551.38253267,417.90430136)
\curveto(551.52252907,417.99429215)(551.64252895,418.09929204)(551.74253267,418.21930136)
\curveto(551.84252875,418.3392918)(551.93752865,418.46929167)(552.02753267,418.60930136)
\curveto(552.0875285,418.70929143)(552.13252846,418.81929132)(552.16253267,418.93930136)
\curveto(552.20252839,419.05929108)(552.25252834,419.16429098)(552.31253267,419.25430136)
\curveto(552.36252823,419.31429083)(552.43252816,419.35429079)(552.52253267,419.37430136)
\curveto(552.54252805,419.38429076)(552.56752802,419.38929075)(552.59753267,419.38930136)
\curveto(552.62752796,419.38929075)(552.65252794,419.39429075)(552.67253267,419.40430136)
}
}
{
\newrgbcolor{curcolor}{0 0 0}
\pscustom[linestyle=none,fillstyle=solid,fillcolor=curcolor]
{
\newpath
\moveto(557.91714205,410.43430136)
\lineto(558.21714205,410.43430136)
\curveto(558.32713999,410.4442997)(558.43213988,410.4442997)(558.53214205,410.43430136)
\curveto(558.64213967,410.43429971)(558.74213957,410.42429972)(558.83214205,410.40430136)
\curveto(558.92213939,410.39429975)(558.99213932,410.36929977)(559.04214205,410.32930136)
\curveto(559.06213925,410.30929983)(559.07713924,410.27929986)(559.08714205,410.23930136)
\curveto(559.10713921,410.19929994)(559.12713919,410.15429999)(559.14714205,410.10430136)
\lineto(559.14714205,410.02930136)
\curveto(559.15713916,409.97930016)(559.15713916,409.92430022)(559.14714205,409.86430136)
\lineto(559.14714205,409.71430136)
\lineto(559.14714205,409.23430136)
\curveto(559.14713917,409.06430108)(559.10713921,408.9443012)(559.02714205,408.87430136)
\curveto(558.95713936,408.82430132)(558.86713945,408.79930134)(558.75714205,408.79930136)
\lineto(558.42714205,408.79930136)
\lineto(557.97714205,408.79930136)
\curveto(557.82714049,408.79930134)(557.7121406,408.82930131)(557.63214205,408.88930136)
\curveto(557.59214072,408.91930122)(557.56214075,408.96930117)(557.54214205,409.03930136)
\curveto(557.52214079,409.11930102)(557.50714081,409.20430094)(557.49714205,409.29430136)
\lineto(557.49714205,409.57930136)
\curveto(557.50714081,409.67930046)(557.5121408,409.76430038)(557.51214205,409.83430136)
\lineto(557.51214205,410.02930136)
\curveto(557.5121408,410.08930005)(557.52214079,410.1443)(557.54214205,410.19430136)
\curveto(557.58214073,410.30429984)(557.65214066,410.37429977)(557.75214205,410.40430136)
\curveto(557.78214053,410.40429974)(557.83714048,410.41429973)(557.91714205,410.43430136)
}
}
{
\newrgbcolor{curcolor}{0 0 0}
\pscustom[linestyle=none,fillstyle=solid,fillcolor=curcolor]
{
\newpath
\moveto(568.1372983,411.87430136)
\curveto(568.14729058,411.83429831)(568.14729058,411.78429836)(568.1372983,411.72430136)
\curveto(568.13729059,411.66429848)(568.13229059,411.61429853)(568.1222983,411.57430136)
\curveto(568.1222906,411.53429861)(568.11729061,411.49429865)(568.1072983,411.45430136)
\lineto(568.1072983,411.34930136)
\curveto(568.08729064,411.26929887)(568.07229065,411.18929895)(568.0622983,411.10930136)
\curveto(568.05229067,411.02929911)(568.03229069,410.95429919)(568.0022983,410.88430136)
\curveto(567.98229074,410.80429934)(567.96229076,410.72929941)(567.9422983,410.65930136)
\curveto(567.9222908,410.58929955)(567.89229083,410.51429963)(567.8522983,410.43430136)
\curveto(567.67229105,410.01430013)(567.41729131,409.67430047)(567.0872983,409.41430136)
\curveto(566.75729197,409.15430099)(566.36729236,408.94930119)(565.9172983,408.79930136)
\curveto(565.79729293,408.75930138)(565.67229305,408.73430141)(565.5422983,408.72430136)
\curveto(565.4222933,408.70430144)(565.29729343,408.67930146)(565.1672983,408.64930136)
\curveto(565.10729362,408.6393015)(565.04229368,408.63430151)(564.9722983,408.63430136)
\curveto(564.91229381,408.63430151)(564.84729388,408.62930151)(564.7772983,408.61930136)
\lineto(564.6572983,408.61930136)
\lineto(564.4622983,408.61930136)
\curveto(564.40229432,408.60930153)(564.34729438,408.61430153)(564.2972983,408.63430136)
\curveto(564.2272945,408.65430149)(564.16229456,408.65930148)(564.1022983,408.64930136)
\curveto(564.04229468,408.6393015)(563.98229474,408.6443015)(563.9222983,408.66430136)
\curveto(563.87229485,408.67430147)(563.8272949,408.67930146)(563.7872983,408.67930136)
\curveto(563.74729498,408.67930146)(563.70229502,408.68930145)(563.6522983,408.70930136)
\curveto(563.57229515,408.72930141)(563.49729523,408.74930139)(563.4272983,408.76930136)
\curveto(563.35729537,408.77930136)(563.28729544,408.79430135)(563.2172983,408.81430136)
\curveto(562.73729599,408.98430116)(562.33729639,409.19430095)(562.0172983,409.44430136)
\curveto(561.70729702,409.70430044)(561.45729727,410.05930008)(561.2672983,410.50930136)
\curveto(561.23729749,410.56929957)(561.21229751,410.62929951)(561.1922983,410.68930136)
\curveto(561.18229754,410.75929938)(561.16729756,410.83429931)(561.1472983,410.91430136)
\curveto(561.1272976,410.97429917)(561.11229761,411.0392991)(561.1022983,411.10930136)
\curveto(561.09229763,411.17929896)(561.07729765,411.24929889)(561.0572983,411.31930136)
\curveto(561.04729768,411.36929877)(561.04229768,411.40929873)(561.0422983,411.43930136)
\lineto(561.0422983,411.55930136)
\curveto(561.03229769,411.59929854)(561.0222977,411.64929849)(561.0122983,411.70930136)
\curveto(561.01229771,411.76929837)(561.01729771,411.81929832)(561.0272983,411.85930136)
\lineto(561.0272983,411.99430136)
\curveto(561.03729769,412.0442981)(561.04229768,412.09429805)(561.0422983,412.14430136)
\curveto(561.06229766,412.2442979)(561.07729765,412.3392978)(561.0872983,412.42930136)
\curveto(561.09729763,412.52929761)(561.11729761,412.62429752)(561.1472983,412.71430136)
\curveto(561.19729753,412.86429728)(561.25229747,413.00429714)(561.3122983,413.13430136)
\curveto(561.37229735,413.26429688)(561.44229728,413.38429676)(561.5222983,413.49430136)
\curveto(561.55229717,413.5442966)(561.58229714,413.58429656)(561.6122983,413.61430136)
\curveto(561.65229707,413.6442965)(561.68729704,413.67929646)(561.7172983,413.71930136)
\curveto(561.77729695,413.79929634)(561.84729688,413.86929627)(561.9272983,413.92930136)
\curveto(561.98729674,413.97929616)(562.04729668,414.02429612)(562.1072983,414.06430136)
\lineto(562.3172983,414.21430136)
\curveto(562.36729636,414.25429589)(562.41729631,414.28929585)(562.4672983,414.31930136)
\curveto(562.51729621,414.35929578)(562.55229617,414.41429573)(562.5722983,414.48430136)
\curveto(562.57229615,414.51429563)(562.56229616,414.5392956)(562.5422983,414.55930136)
\curveto(562.53229619,414.58929555)(562.5222962,414.61429553)(562.5122983,414.63430136)
\curveto(562.47229625,414.68429546)(562.4222963,414.72929541)(562.3622983,414.76930136)
\curveto(562.31229641,414.81929532)(562.26229646,414.86429528)(562.2122983,414.90430136)
\curveto(562.17229655,414.93429521)(562.1222966,414.98929515)(562.0622983,415.06930136)
\curveto(562.04229668,415.09929504)(562.01229671,415.12429502)(561.9722983,415.14430136)
\curveto(561.94229678,415.17429497)(561.91729681,415.20929493)(561.8972983,415.24930136)
\curveto(561.727297,415.45929468)(561.59729713,415.70429444)(561.5072983,415.98430136)
\curveto(561.48729724,416.06429408)(561.47229725,416.144294)(561.4622983,416.22430136)
\curveto(561.45229727,416.30429384)(561.43729729,416.38429376)(561.4172983,416.46430136)
\curveto(561.39729733,416.51429363)(561.38729734,416.57929356)(561.3872983,416.65930136)
\curveto(561.38729734,416.74929339)(561.39729733,416.81929332)(561.4172983,416.86930136)
\curveto(561.41729731,416.96929317)(561.4222973,417.0392931)(561.4322983,417.07930136)
\curveto(561.45229727,417.15929298)(561.46729726,417.22929291)(561.4772983,417.28930136)
\curveto(561.48729724,417.35929278)(561.50229722,417.42929271)(561.5222983,417.49930136)
\curveto(561.67229705,417.92929221)(561.88729684,418.27429187)(562.1672983,418.53430136)
\curveto(562.45729627,418.79429135)(562.80729592,419.00929113)(563.2172983,419.17930136)
\curveto(563.3272954,419.22929091)(563.44229528,419.25929088)(563.5622983,419.26930136)
\curveto(563.69229503,419.28929085)(563.8222949,419.31929082)(563.9522983,419.35930136)
\curveto(564.03229469,419.35929078)(564.10229462,419.35929078)(564.1622983,419.35930136)
\curveto(564.23229449,419.36929077)(564.30729442,419.37929076)(564.3872983,419.38930136)
\curveto(565.17729355,419.40929073)(565.83229289,419.27929086)(566.3522983,418.99930136)
\curveto(566.88229184,418.71929142)(567.26229146,418.30929183)(567.4922983,417.76930136)
\curveto(567.60229112,417.5392926)(567.67229105,417.25429289)(567.7022983,416.91430136)
\curveto(567.74229098,416.58429356)(567.71229101,416.27929386)(567.6122983,415.99930136)
\curveto(567.57229115,415.86929427)(567.5222912,415.74929439)(567.4622983,415.63930136)
\curveto(567.41229131,415.52929461)(567.35229137,415.42429472)(567.2822983,415.32430136)
\curveto(567.26229146,415.28429486)(567.23229149,415.24929489)(567.1922983,415.21930136)
\lineto(567.1022983,415.12930136)
\curveto(567.05229167,415.0392951)(566.99229173,414.97429517)(566.9222983,414.93430136)
\curveto(566.87229185,414.88429526)(566.81729191,414.83429531)(566.7572983,414.78430136)
\curveto(566.70729202,414.7442954)(566.66229206,414.69929544)(566.6222983,414.64930136)
\curveto(566.60229212,414.62929551)(566.58229214,414.60429554)(566.5622983,414.57430136)
\curveto(566.55229217,414.55429559)(566.55229217,414.52929561)(566.5622983,414.49930136)
\curveto(566.57229215,414.44929569)(566.60229212,414.39929574)(566.6522983,414.34930136)
\curveto(566.70229202,414.30929583)(566.75729197,414.26929587)(566.8172983,414.22930136)
\lineto(566.9972983,414.10930136)
\curveto(567.05729167,414.07929606)(567.10729162,414.04929609)(567.1472983,414.01930136)
\curveto(567.47729125,413.77929636)(567.727291,413.46929667)(567.8972983,413.08930136)
\curveto(567.93729079,413.00929713)(567.96729076,412.92429722)(567.9872983,412.83430136)
\curveto(568.01729071,412.7442974)(568.04229068,412.65429749)(568.0622983,412.56430136)
\curveto(568.07229065,412.51429763)(568.08229064,412.45929768)(568.0922983,412.39930136)
\lineto(568.1222983,412.24930136)
\curveto(568.13229059,412.18929795)(568.13229059,412.12429802)(568.1222983,412.05430136)
\curveto(568.11229061,411.99429815)(568.11729061,411.93429821)(568.1372983,411.87430136)
\moveto(562.7522983,416.91430136)
\curveto(562.722296,416.80429334)(562.71729601,416.66429348)(562.7372983,416.49430136)
\curveto(562.75729597,416.33429381)(562.78229594,416.20929393)(562.8122983,416.11930136)
\curveto(562.9222958,415.79929434)(563.07229565,415.55429459)(563.2622983,415.38430136)
\curveto(563.45229527,415.22429492)(563.71729501,415.09429505)(564.0572983,414.99430136)
\curveto(564.18729454,414.96429518)(564.35229437,414.9392952)(564.5522983,414.91930136)
\curveto(564.75229397,414.90929523)(564.9222938,414.92429522)(565.0622983,414.96430136)
\curveto(565.35229337,415.0442951)(565.59229313,415.15429499)(565.7822983,415.29430136)
\curveto(565.98229274,415.4442947)(566.13729259,415.6442945)(566.2472983,415.89430136)
\curveto(566.26729246,415.9442942)(566.27729245,415.98929415)(566.2772983,416.02930136)
\curveto(566.28729244,416.06929407)(566.30229242,416.11429403)(566.3222983,416.16430136)
\curveto(566.35229237,416.27429387)(566.37229235,416.41429373)(566.3822983,416.58430136)
\curveto(566.39229233,416.75429339)(566.38229234,416.89929324)(566.3522983,417.01930136)
\curveto(566.33229239,417.10929303)(566.30729242,417.19429295)(566.2772983,417.27430136)
\curveto(566.25729247,417.35429279)(566.2222925,417.43429271)(566.1722983,417.51430136)
\curveto(566.00229272,417.78429236)(565.77729295,417.97929216)(565.4972983,418.09930136)
\curveto(565.2272935,418.21929192)(564.86729386,418.27929186)(564.4172983,418.27930136)
\curveto(564.39729433,418.25929188)(564.36729436,418.25429189)(564.3272983,418.26430136)
\curveto(564.28729444,418.27429187)(564.25229447,418.27429187)(564.2222983,418.26430136)
\curveto(564.17229455,418.2442919)(564.11729461,418.22929191)(564.0572983,418.21930136)
\curveto(564.00729472,418.21929192)(563.95729477,418.20929193)(563.9072983,418.18930136)
\curveto(563.66729506,418.09929204)(563.45729527,417.98429216)(563.2772983,417.84430136)
\curveto(563.09729563,417.71429243)(562.95729577,417.53429261)(562.8572983,417.30430136)
\curveto(562.83729589,417.2442929)(562.81729591,417.17929296)(562.7972983,417.10930136)
\curveto(562.78729594,417.04929309)(562.77229595,416.98429316)(562.7522983,416.91430136)
\moveto(566.7722983,411.37930136)
\curveto(566.8222919,411.56929857)(566.8272919,411.77429837)(566.7872983,411.99430136)
\curveto(566.75729197,412.21429793)(566.71229201,412.39429775)(566.6522983,412.53430136)
\curveto(566.48229224,412.90429724)(566.2222925,413.20929693)(565.8722983,413.44930136)
\curveto(565.53229319,413.68929645)(565.09729363,413.80929633)(564.5672983,413.80930136)
\curveto(564.53729419,413.78929635)(564.49729423,413.78429636)(564.4472983,413.79430136)
\curveto(564.39729433,413.81429633)(564.35729437,413.81929632)(564.3272983,413.80930136)
\lineto(564.0572983,413.74930136)
\curveto(563.97729475,413.7392964)(563.89729483,413.72429642)(563.8172983,413.70430136)
\curveto(563.51729521,413.59429655)(563.25229547,413.44929669)(563.0222983,413.26930136)
\curveto(562.80229592,413.08929705)(562.63229609,412.85929728)(562.5122983,412.57930136)
\curveto(562.48229624,412.49929764)(562.45729627,412.41929772)(562.4372983,412.33930136)
\curveto(562.41729631,412.25929788)(562.39729633,412.17429797)(562.3772983,412.08430136)
\curveto(562.34729638,411.96429818)(562.33729639,411.81429833)(562.3472983,411.63430136)
\curveto(562.36729636,411.45429869)(562.39229633,411.31429883)(562.4222983,411.21430136)
\curveto(562.44229628,411.16429898)(562.45229627,411.11929902)(562.4522983,411.07930136)
\curveto(562.46229626,411.04929909)(562.47729625,411.00929913)(562.4972983,410.95930136)
\curveto(562.59729613,410.7392994)(562.727296,410.5392996)(562.8872983,410.35930136)
\curveto(563.05729567,410.17929996)(563.25229547,410.0443001)(563.4722983,409.95430136)
\curveto(563.54229518,409.91430023)(563.63729509,409.87930026)(563.7572983,409.84930136)
\curveto(563.97729475,409.75930038)(564.23229449,409.71430043)(564.5222983,409.71430136)
\lineto(564.8072983,409.71430136)
\curveto(564.90729382,409.73430041)(565.00229372,409.74930039)(565.0922983,409.75930136)
\curveto(565.18229354,409.76930037)(565.27229345,409.78930035)(565.3622983,409.81930136)
\curveto(565.6222931,409.89930024)(565.86229286,410.02930011)(566.0822983,410.20930136)
\curveto(566.31229241,410.39929974)(566.48229224,410.61429953)(566.5922983,410.85430136)
\curveto(566.63229209,410.93429921)(566.66229206,411.01429913)(566.6822983,411.09430136)
\curveto(566.71229201,411.18429896)(566.74229198,411.27929886)(566.7722983,411.37930136)
}
}
{
\newrgbcolor{curcolor}{0 0 0}
\pscustom[linestyle=none,fillstyle=solid,fillcolor=curcolor]
{
\newpath
\moveto(579.27690767,417.31930136)
\curveto(579.07689737,417.02929311)(578.86689758,416.7442934)(578.64690767,416.46430136)
\curveto(578.43689801,416.18429396)(578.23189822,415.89929424)(578.03190767,415.60930136)
\curveto(577.43189902,414.75929538)(576.82689962,413.91929622)(576.21690767,413.08930136)
\curveto(575.60690084,412.26929787)(575.00190145,411.43429871)(574.40190767,410.58430136)
\lineto(573.89190767,409.86430136)
\lineto(573.38190767,409.17430136)
\curveto(573.30190315,409.06430108)(573.22190323,408.94930119)(573.14190767,408.82930136)
\curveto(573.06190339,408.70930143)(572.96690348,408.61430153)(572.85690767,408.54430136)
\curveto(572.81690363,408.52430162)(572.7519037,408.50930163)(572.66190767,408.49930136)
\curveto(572.58190387,408.47930166)(572.49190396,408.46930167)(572.39190767,408.46930136)
\curveto(572.29190416,408.46930167)(572.19690425,408.47430167)(572.10690767,408.48430136)
\curveto(572.02690442,408.49430165)(571.96690448,408.51430163)(571.92690767,408.54430136)
\curveto(571.89690455,408.56430158)(571.87190458,408.59930154)(571.85190767,408.64930136)
\curveto(571.84190461,408.68930145)(571.8469046,408.73430141)(571.86690767,408.78430136)
\curveto(571.90690454,408.86430128)(571.9519045,408.9393012)(572.00190767,409.00930136)
\curveto(572.06190439,409.08930105)(572.11690433,409.16930097)(572.16690767,409.24930136)
\curveto(572.40690404,409.58930055)(572.6519038,409.92430022)(572.90190767,410.25430136)
\curveto(573.1519033,410.58429956)(573.39190306,410.91929922)(573.62190767,411.25930136)
\curveto(573.78190267,411.47929866)(573.94190251,411.69429845)(574.10190767,411.90430136)
\curveto(574.26190219,412.11429803)(574.42190203,412.32929781)(574.58190767,412.54930136)
\curveto(574.94190151,413.06929707)(575.30690114,413.57929656)(575.67690767,414.07930136)
\curveto(576.0469004,414.57929556)(576.41690003,415.08929505)(576.78690767,415.60930136)
\curveto(576.92689952,415.80929433)(577.06689938,416.00429414)(577.20690767,416.19430136)
\curveto(577.35689909,416.38429376)(577.50189895,416.57929356)(577.64190767,416.77930136)
\curveto(577.8518986,417.07929306)(578.06689838,417.37929276)(578.28690767,417.67930136)
\lineto(578.94690767,418.57930136)
\lineto(579.12690767,418.84930136)
\lineto(579.33690767,419.11930136)
\lineto(579.45690767,419.29930136)
\curveto(579.50689694,419.35929078)(579.55689689,419.41429073)(579.60690767,419.46430136)
\curveto(579.67689677,419.51429063)(579.7518967,419.54929059)(579.83190767,419.56930136)
\curveto(579.8518966,419.57929056)(579.87689657,419.57929056)(579.90690767,419.56930136)
\curveto(579.9468965,419.56929057)(579.97689647,419.57929056)(579.99690767,419.59930136)
\curveto(580.11689633,419.59929054)(580.2518962,419.59429055)(580.40190767,419.58430136)
\curveto(580.5518959,419.58429056)(580.64189581,419.5392906)(580.67190767,419.44930136)
\curveto(580.69189576,419.41929072)(580.69689575,419.38429076)(580.68690767,419.34430136)
\curveto(580.67689577,419.30429084)(580.66189579,419.27429087)(580.64190767,419.25430136)
\curveto(580.60189585,419.17429097)(580.56189589,419.10429104)(580.52190767,419.04430136)
\curveto(580.48189597,418.98429116)(580.43689601,418.92429122)(580.38690767,418.86430136)
\lineto(579.81690767,418.08430136)
\curveto(579.63689681,417.83429231)(579.45689699,417.57929256)(579.27690767,417.31930136)
\moveto(572.42190767,413.41930136)
\curveto(572.37190408,413.4392967)(572.32190413,413.4442967)(572.27190767,413.43430136)
\curveto(572.22190423,413.42429672)(572.17190428,413.42929671)(572.12190767,413.44930136)
\curveto(572.01190444,413.46929667)(571.90690454,413.48929665)(571.80690767,413.50930136)
\curveto(571.71690473,413.5392966)(571.62190483,413.57929656)(571.52190767,413.62930136)
\curveto(571.19190526,413.76929637)(570.93690551,413.96429618)(570.75690767,414.21430136)
\curveto(570.57690587,414.47429567)(570.43190602,414.78429536)(570.32190767,415.14430136)
\curveto(570.29190616,415.22429492)(570.27190618,415.30429484)(570.26190767,415.38430136)
\curveto(570.2519062,415.47429467)(570.23690621,415.55929458)(570.21690767,415.63930136)
\curveto(570.20690624,415.68929445)(570.20190625,415.75429439)(570.20190767,415.83430136)
\curveto(570.19190626,415.86429428)(570.18690626,415.89429425)(570.18690767,415.92430136)
\curveto(570.18690626,415.96429418)(570.18190627,415.99929414)(570.17190767,416.02930136)
\lineto(570.17190767,416.17930136)
\curveto(570.16190629,416.22929391)(570.15690629,416.28929385)(570.15690767,416.35930136)
\curveto(570.15690629,416.4392937)(570.16190629,416.50429364)(570.17190767,416.55430136)
\lineto(570.17190767,416.71930136)
\curveto(570.19190626,416.76929337)(570.19690625,416.81429333)(570.18690767,416.85430136)
\curveto(570.18690626,416.90429324)(570.19190626,416.94929319)(570.20190767,416.98930136)
\curveto(570.21190624,417.02929311)(570.21690623,417.06429308)(570.21690767,417.09430136)
\curveto(570.21690623,417.13429301)(570.22190623,417.17429297)(570.23190767,417.21430136)
\curveto(570.26190619,417.32429282)(570.28190617,417.43429271)(570.29190767,417.54430136)
\curveto(570.31190614,417.66429248)(570.3469061,417.77929236)(570.39690767,417.88930136)
\curveto(570.53690591,418.22929191)(570.69690575,418.50429164)(570.87690767,418.71430136)
\curveto(571.06690538,418.93429121)(571.33690511,419.11429103)(571.68690767,419.25430136)
\curveto(571.76690468,419.28429086)(571.8519046,419.30429084)(571.94190767,419.31430136)
\curveto(572.03190442,419.33429081)(572.12690432,419.35429079)(572.22690767,419.37430136)
\curveto(572.25690419,419.38429076)(572.31190414,419.38429076)(572.39190767,419.37430136)
\curveto(572.47190398,419.37429077)(572.52190393,419.38429076)(572.54190767,419.40430136)
\curveto(573.10190335,419.41429073)(573.5519029,419.30429084)(573.89190767,419.07430136)
\curveto(574.24190221,418.8442913)(574.50190195,418.5392916)(574.67190767,418.15930136)
\curveto(574.71190174,418.06929207)(574.7469017,417.97429217)(574.77690767,417.87430136)
\curveto(574.80690164,417.77429237)(574.83190162,417.67429247)(574.85190767,417.57430136)
\curveto(574.87190158,417.5442926)(574.87690157,417.51429263)(574.86690767,417.48430136)
\curveto(574.86690158,417.45429269)(574.87190158,417.42429272)(574.88190767,417.39430136)
\curveto(574.91190154,417.28429286)(574.93190152,417.15929298)(574.94190767,417.01930136)
\curveto(574.9519015,416.88929325)(574.96190149,416.75429339)(574.97190767,416.61430136)
\lineto(574.97190767,416.44930136)
\curveto(574.98190147,416.38929375)(574.98190147,416.33429381)(574.97190767,416.28430136)
\curveto(574.96190149,416.23429391)(574.95690149,416.18429396)(574.95690767,416.13430136)
\lineto(574.95690767,415.99930136)
\curveto(574.9469015,415.95929418)(574.94190151,415.91929422)(574.94190767,415.87930136)
\curveto(574.9519015,415.8392943)(574.9469015,415.79429435)(574.92690767,415.74430136)
\curveto(574.90690154,415.63429451)(574.88690156,415.52929461)(574.86690767,415.42930136)
\curveto(574.85690159,415.32929481)(574.83690161,415.22929491)(574.80690767,415.12930136)
\curveto(574.67690177,414.76929537)(574.51190194,414.45429569)(574.31190767,414.18430136)
\curveto(574.11190234,413.91429623)(573.83690261,413.70929643)(573.48690767,413.56930136)
\curveto(573.40690304,413.5392966)(573.32190313,413.51429663)(573.23190767,413.49430136)
\lineto(572.96190767,413.43430136)
\curveto(572.91190354,413.42429672)(572.86690358,413.41929672)(572.82690767,413.41930136)
\curveto(572.78690366,413.42929671)(572.7469037,413.42929671)(572.70690767,413.41930136)
\curveto(572.60690384,413.39929674)(572.51190394,413.39929674)(572.42190767,413.41930136)
\moveto(571.58190767,414.81430136)
\curveto(571.62190483,414.7442954)(571.66190479,414.67929546)(571.70190767,414.61930136)
\curveto(571.74190471,414.56929557)(571.79190466,414.51929562)(571.85190767,414.46930136)
\lineto(572.00190767,414.34930136)
\curveto(572.06190439,414.31929582)(572.12690432,414.29429585)(572.19690767,414.27430136)
\curveto(572.23690421,414.25429589)(572.27190418,414.2442959)(572.30190767,414.24430136)
\curveto(572.34190411,414.25429589)(572.38190407,414.24929589)(572.42190767,414.22930136)
\curveto(572.451904,414.22929591)(572.49190396,414.22429592)(572.54190767,414.21430136)
\curveto(572.59190386,414.21429593)(572.63190382,414.21929592)(572.66190767,414.22930136)
\lineto(572.88690767,414.27430136)
\curveto(573.13690331,414.35429579)(573.32190313,414.47929566)(573.44190767,414.64930136)
\curveto(573.52190293,414.74929539)(573.59190286,414.87929526)(573.65190767,415.03930136)
\curveto(573.73190272,415.21929492)(573.79190266,415.4442947)(573.83190767,415.71430136)
\curveto(573.87190258,415.99429415)(573.88690256,416.27429387)(573.87690767,416.55430136)
\curveto(573.86690258,416.8442933)(573.83690261,417.11929302)(573.78690767,417.37930136)
\curveto(573.73690271,417.6392925)(573.66190279,417.84929229)(573.56190767,418.00930136)
\curveto(573.44190301,418.20929193)(573.29190316,418.35929178)(573.11190767,418.45930136)
\curveto(573.03190342,418.50929163)(572.94190351,418.5392916)(572.84190767,418.54930136)
\curveto(572.74190371,418.56929157)(572.63690381,418.57929156)(572.52690767,418.57930136)
\curveto(572.50690394,418.56929157)(572.48190397,418.56429158)(572.45190767,418.56430136)
\curveto(572.43190402,418.57429157)(572.41190404,418.57429157)(572.39190767,418.56430136)
\curveto(572.34190411,418.55429159)(572.29690415,418.5442916)(572.25690767,418.53430136)
\curveto(572.21690423,418.53429161)(572.17690427,418.52429162)(572.13690767,418.50430136)
\curveto(571.95690449,418.42429172)(571.80690464,418.30429184)(571.68690767,418.14430136)
\curveto(571.57690487,417.98429216)(571.48690496,417.80429234)(571.41690767,417.60430136)
\curveto(571.35690509,417.41429273)(571.31190514,417.18929295)(571.28190767,416.92930136)
\curveto(571.26190519,416.66929347)(571.25690519,416.40429374)(571.26690767,416.13430136)
\curveto(571.27690517,415.87429427)(571.30690514,415.62429452)(571.35690767,415.38430136)
\curveto(571.41690503,415.15429499)(571.49190496,414.96429518)(571.58190767,414.81430136)
\moveto(582.38190767,411.82930136)
\curveto(582.39189406,411.77929836)(582.39689405,411.68929845)(582.39690767,411.55930136)
\curveto(582.39689405,411.42929871)(582.38689406,411.3392988)(582.36690767,411.28930136)
\curveto(582.3468941,411.2392989)(582.34189411,411.18429896)(582.35190767,411.12430136)
\curveto(582.36189409,411.07429907)(582.36189409,411.02429912)(582.35190767,410.97430136)
\curveto(582.31189414,410.83429931)(582.28189417,410.69929944)(582.26190767,410.56930136)
\curveto(582.2518942,410.4392997)(582.22189423,410.31929982)(582.17190767,410.20930136)
\curveto(582.03189442,409.85930028)(581.86689458,409.56430058)(581.67690767,409.32430136)
\curveto(581.48689496,409.09430105)(581.21689523,408.90930123)(580.86690767,408.76930136)
\curveto(580.78689566,408.7393014)(580.70189575,408.71930142)(580.61190767,408.70930136)
\curveto(580.52189593,408.68930145)(580.43689601,408.66930147)(580.35690767,408.64930136)
\curveto(580.30689614,408.6393015)(580.25689619,408.63430151)(580.20690767,408.63430136)
\curveto(580.15689629,408.63430151)(580.10689634,408.62930151)(580.05690767,408.61930136)
\curveto(580.02689642,408.60930153)(579.97689647,408.60930153)(579.90690767,408.61930136)
\curveto(579.83689661,408.61930152)(579.78689666,408.62430152)(579.75690767,408.63430136)
\curveto(579.69689675,408.65430149)(579.63689681,408.66430148)(579.57690767,408.66430136)
\curveto(579.52689692,408.65430149)(579.47689697,408.65930148)(579.42690767,408.67930136)
\curveto(579.33689711,408.69930144)(579.2468972,408.72430142)(579.15690767,408.75430136)
\curveto(579.07689737,408.77430137)(578.99689745,408.80430134)(578.91690767,408.84430136)
\curveto(578.59689785,408.98430116)(578.3468981,409.17930096)(578.16690767,409.42930136)
\curveto(577.98689846,409.68930045)(577.83689861,409.99430015)(577.71690767,410.34430136)
\curveto(577.69689875,410.42429972)(577.68189877,410.50929963)(577.67190767,410.59930136)
\curveto(577.66189879,410.68929945)(577.6468988,410.77429937)(577.62690767,410.85430136)
\curveto(577.61689883,410.88429926)(577.61189884,410.91429923)(577.61190767,410.94430136)
\lineto(577.61190767,411.04930136)
\curveto(577.59189886,411.12929901)(577.58189887,411.20929893)(577.58190767,411.28930136)
\lineto(577.58190767,411.42430136)
\curveto(577.56189889,411.52429862)(577.56189889,411.62429852)(577.58190767,411.72430136)
\lineto(577.58190767,411.90430136)
\curveto(577.59189886,411.95429819)(577.59689885,411.99929814)(577.59690767,412.03930136)
\curveto(577.59689885,412.08929805)(577.60189885,412.13429801)(577.61190767,412.17430136)
\curveto(577.62189883,412.21429793)(577.62689882,412.24929789)(577.62690767,412.27930136)
\curveto(577.62689882,412.31929782)(577.63189882,412.35929778)(577.64190767,412.39930136)
\lineto(577.70190767,412.72930136)
\curveto(577.72189873,412.84929729)(577.7518987,412.95929718)(577.79190767,413.05930136)
\curveto(577.93189852,413.38929675)(578.09189836,413.66429648)(578.27190767,413.88430136)
\curveto(578.46189799,414.11429603)(578.72189773,414.29929584)(579.05190767,414.43930136)
\curveto(579.13189732,414.47929566)(579.21689723,414.50429564)(579.30690767,414.51430136)
\lineto(579.60690767,414.57430136)
\lineto(579.74190767,414.57430136)
\curveto(579.79189666,414.58429556)(579.84189661,414.58929555)(579.89190767,414.58930136)
\curveto(580.46189599,414.60929553)(580.92189553,414.50429564)(581.27190767,414.27430136)
\curveto(581.63189482,414.05429609)(581.89689455,413.75429639)(582.06690767,413.37430136)
\curveto(582.11689433,413.27429687)(582.15689429,413.17429697)(582.18690767,413.07430136)
\curveto(582.21689423,412.97429717)(582.2468942,412.86929727)(582.27690767,412.75930136)
\curveto(582.28689416,412.71929742)(582.29189416,412.68429746)(582.29190767,412.65430136)
\curveto(582.29189416,412.63429751)(582.29689415,412.60429754)(582.30690767,412.56430136)
\curveto(582.32689412,412.49429765)(582.33689411,412.41929772)(582.33690767,412.33930136)
\curveto(582.33689411,412.25929788)(582.3468941,412.17929796)(582.36690767,412.09930136)
\curveto(582.36689408,412.04929809)(582.36689408,412.00429814)(582.36690767,411.96430136)
\curveto(582.36689408,411.92429822)(582.37189408,411.87929826)(582.38190767,411.82930136)
\moveto(581.27190767,411.39430136)
\curveto(581.28189517,411.4442987)(581.28689516,411.51929862)(581.28690767,411.61930136)
\curveto(581.29689515,411.71929842)(581.29189516,411.79429835)(581.27190767,411.84430136)
\curveto(581.2518952,411.90429824)(581.2468952,411.95929818)(581.25690767,412.00930136)
\curveto(581.27689517,412.06929807)(581.27689517,412.12929801)(581.25690767,412.18930136)
\curveto(581.2468952,412.21929792)(581.24189521,412.25429789)(581.24190767,412.29430136)
\curveto(581.24189521,412.33429781)(581.23689521,412.37429777)(581.22690767,412.41430136)
\curveto(581.20689524,412.49429765)(581.18689526,412.56929757)(581.16690767,412.63930136)
\curveto(581.15689529,412.71929742)(581.14189531,412.79929734)(581.12190767,412.87930136)
\curveto(581.09189536,412.9392972)(581.06689538,412.99929714)(581.04690767,413.05930136)
\curveto(581.02689542,413.11929702)(580.99689545,413.17929696)(580.95690767,413.23930136)
\curveto(580.85689559,413.40929673)(580.72689572,413.5442966)(580.56690767,413.64430136)
\curveto(580.48689596,413.69429645)(580.39189606,413.72929641)(580.28190767,413.74930136)
\curveto(580.17189628,413.76929637)(580.0468964,413.77929636)(579.90690767,413.77930136)
\curveto(579.88689656,413.76929637)(579.86189659,413.76429638)(579.83190767,413.76430136)
\curveto(579.80189665,413.77429637)(579.77189668,413.77429637)(579.74190767,413.76430136)
\lineto(579.59190767,413.70430136)
\curveto(579.54189691,413.69429645)(579.49689695,413.67929646)(579.45690767,413.65930136)
\curveto(579.26689718,413.54929659)(579.12189733,413.40429674)(579.02190767,413.22430136)
\curveto(578.93189752,413.0442971)(578.8518976,412.8392973)(578.78190767,412.60930136)
\curveto(578.74189771,412.47929766)(578.72189773,412.3442978)(578.72190767,412.20430136)
\curveto(578.72189773,412.07429807)(578.71189774,411.92929821)(578.69190767,411.76930136)
\curveto(578.68189777,411.71929842)(578.67189778,411.65929848)(578.66190767,411.58930136)
\curveto(578.66189779,411.51929862)(578.67189778,411.45929868)(578.69190767,411.40930136)
\lineto(578.69190767,411.24430136)
\lineto(578.69190767,411.06430136)
\curveto(578.70189775,411.01429913)(578.71189774,410.95929918)(578.72190767,410.89930136)
\curveto(578.73189772,410.84929929)(578.73689771,410.79429935)(578.73690767,410.73430136)
\curveto(578.7468977,410.67429947)(578.76189769,410.61929952)(578.78190767,410.56930136)
\curveto(578.83189762,410.37929976)(578.89189756,410.20429994)(578.96190767,410.04430136)
\curveto(579.03189742,409.88430026)(579.13689731,409.75430039)(579.27690767,409.65430136)
\curveto(579.40689704,409.55430059)(579.5468969,409.48430066)(579.69690767,409.44430136)
\curveto(579.72689672,409.43430071)(579.7518967,409.42930071)(579.77190767,409.42930136)
\curveto(579.80189665,409.4393007)(579.83189662,409.4393007)(579.86190767,409.42930136)
\curveto(579.88189657,409.42930071)(579.91189654,409.42430072)(579.95190767,409.41430136)
\curveto(579.99189646,409.41430073)(580.02689642,409.41930072)(580.05690767,409.42930136)
\curveto(580.09689635,409.4393007)(580.13689631,409.4443007)(580.17690767,409.44430136)
\curveto(580.21689623,409.4443007)(580.25689619,409.45430069)(580.29690767,409.47430136)
\curveto(580.53689591,409.55430059)(580.73189572,409.68930045)(580.88190767,409.87930136)
\curveto(581.00189545,410.05930008)(581.09189536,410.26429988)(581.15190767,410.49430136)
\curveto(581.17189528,410.56429958)(581.18689526,410.63429951)(581.19690767,410.70430136)
\curveto(581.20689524,410.78429936)(581.22189523,410.86429928)(581.24190767,410.94430136)
\curveto(581.24189521,411.00429914)(581.2468952,411.04929909)(581.25690767,411.07930136)
\curveto(581.25689519,411.09929904)(581.25689519,411.12429902)(581.25690767,411.15430136)
\curveto(581.25689519,411.19429895)(581.26189519,411.22429892)(581.27190767,411.24430136)
\lineto(581.27190767,411.39430136)
}
}
{
\newrgbcolor{curcolor}{0 0 0}
\pscustom[linestyle=none,fillstyle=solid,fillcolor=curcolor]
{
\newpath
\moveto(245.03718721,280.83289022)
\curveto(245.13718236,280.8328796)(245.23218226,280.82287961)(245.32218721,280.80289022)
\curveto(245.41218208,280.79287964)(245.47718202,280.76287967)(245.51718721,280.71289022)
\curveto(245.57718192,280.6328798)(245.60718189,280.52787991)(245.60718721,280.39789022)
\lineto(245.60718721,280.00789022)
\lineto(245.60718721,278.50789022)
\lineto(245.60718721,272.11789022)
\lineto(245.60718721,270.94789022)
\lineto(245.60718721,270.63289022)
\curveto(245.61718188,270.5328899)(245.60218189,270.45288998)(245.56218721,270.39289022)
\curveto(245.51218198,270.31289012)(245.43718206,270.26289017)(245.33718721,270.24289022)
\curveto(245.24718225,270.2328902)(245.13718236,270.22789021)(245.00718721,270.22789022)
\lineto(244.78218721,270.22789022)
\curveto(244.70218279,270.24789019)(244.63218286,270.26289017)(244.57218721,270.27289022)
\curveto(244.51218298,270.29289014)(244.46218303,270.3328901)(244.42218721,270.39289022)
\curveto(244.38218311,270.45288998)(244.36218313,270.52788991)(244.36218721,270.61789022)
\lineto(244.36218721,270.91789022)
\lineto(244.36218721,272.01289022)
\lineto(244.36218721,277.35289022)
\curveto(244.34218315,277.44288299)(244.32718317,277.51788292)(244.31718721,277.57789022)
\curveto(244.31718318,277.64788279)(244.28718321,277.70788273)(244.22718721,277.75789022)
\curveto(244.15718334,277.80788263)(244.06718343,277.8328826)(243.95718721,277.83289022)
\curveto(243.85718364,277.84288259)(243.74718375,277.84788259)(243.62718721,277.84789022)
\lineto(242.48718721,277.84789022)
\lineto(241.99218721,277.84789022)
\curveto(241.83218566,277.85788258)(241.72218577,277.91788252)(241.66218721,278.02789022)
\curveto(241.64218585,278.05788238)(241.63218586,278.08788235)(241.63218721,278.11789022)
\curveto(241.63218586,278.15788228)(241.62718587,278.20288223)(241.61718721,278.25289022)
\curveto(241.5971859,278.37288206)(241.60218589,278.48288195)(241.63218721,278.58289022)
\curveto(241.67218582,278.68288175)(241.72718577,278.75288168)(241.79718721,278.79289022)
\curveto(241.87718562,278.84288159)(241.9971855,278.86788157)(242.15718721,278.86789022)
\curveto(242.31718518,278.86788157)(242.45218504,278.88288155)(242.56218721,278.91289022)
\curveto(242.61218488,278.92288151)(242.66718483,278.92788151)(242.72718721,278.92789022)
\curveto(242.78718471,278.9378815)(242.84718465,278.95288148)(242.90718721,278.97289022)
\curveto(243.05718444,279.02288141)(243.20218429,279.07288136)(243.34218721,279.12289022)
\curveto(243.48218401,279.18288125)(243.61718388,279.25288118)(243.74718721,279.33289022)
\curveto(243.88718361,279.42288101)(244.00718349,279.52788091)(244.10718721,279.64789022)
\curveto(244.20718329,279.76788067)(244.30218319,279.89788054)(244.39218721,280.03789022)
\curveto(244.45218304,280.1378803)(244.497183,280.24788019)(244.52718721,280.36789022)
\curveto(244.56718293,280.48787995)(244.61718288,280.59287984)(244.67718721,280.68289022)
\curveto(244.72718277,280.74287969)(244.7971827,280.78287965)(244.88718721,280.80289022)
\curveto(244.90718259,280.81287962)(244.93218256,280.81787962)(244.96218721,280.81789022)
\curveto(244.9921825,280.81787962)(245.01718248,280.82287961)(245.03718721,280.83289022)
}
}
{
\newrgbcolor{curcolor}{0 0 0}
\pscustom[linestyle=none,fillstyle=solid,fillcolor=curcolor]
{
\newpath
\moveto(253.38679659,280.83289022)
\curveto(253.48679173,280.8328796)(253.58179164,280.82287961)(253.67179659,280.80289022)
\curveto(253.76179146,280.79287964)(253.82679139,280.76287967)(253.86679659,280.71289022)
\curveto(253.92679129,280.6328798)(253.95679126,280.52787991)(253.95679659,280.39789022)
\lineto(253.95679659,280.00789022)
\lineto(253.95679659,278.50789022)
\lineto(253.95679659,272.11789022)
\lineto(253.95679659,270.94789022)
\lineto(253.95679659,270.63289022)
\curveto(253.96679125,270.5328899)(253.95179127,270.45288998)(253.91179659,270.39289022)
\curveto(253.86179136,270.31289012)(253.78679143,270.26289017)(253.68679659,270.24289022)
\curveto(253.59679162,270.2328902)(253.48679173,270.22789021)(253.35679659,270.22789022)
\lineto(253.13179659,270.22789022)
\curveto(253.05179217,270.24789019)(252.98179224,270.26289017)(252.92179659,270.27289022)
\curveto(252.86179236,270.29289014)(252.81179241,270.3328901)(252.77179659,270.39289022)
\curveto(252.73179249,270.45288998)(252.71179251,270.52788991)(252.71179659,270.61789022)
\lineto(252.71179659,270.91789022)
\lineto(252.71179659,272.01289022)
\lineto(252.71179659,277.35289022)
\curveto(252.69179253,277.44288299)(252.67679254,277.51788292)(252.66679659,277.57789022)
\curveto(252.66679255,277.64788279)(252.63679258,277.70788273)(252.57679659,277.75789022)
\curveto(252.50679271,277.80788263)(252.4167928,277.8328826)(252.30679659,277.83289022)
\curveto(252.20679301,277.84288259)(252.09679312,277.84788259)(251.97679659,277.84789022)
\lineto(250.83679659,277.84789022)
\lineto(250.34179659,277.84789022)
\curveto(250.18179504,277.85788258)(250.07179515,277.91788252)(250.01179659,278.02789022)
\curveto(249.99179523,278.05788238)(249.98179524,278.08788235)(249.98179659,278.11789022)
\curveto(249.98179524,278.15788228)(249.97679524,278.20288223)(249.96679659,278.25289022)
\curveto(249.94679527,278.37288206)(249.95179527,278.48288195)(249.98179659,278.58289022)
\curveto(250.0217952,278.68288175)(250.07679514,278.75288168)(250.14679659,278.79289022)
\curveto(250.22679499,278.84288159)(250.34679487,278.86788157)(250.50679659,278.86789022)
\curveto(250.66679455,278.86788157)(250.80179442,278.88288155)(250.91179659,278.91289022)
\curveto(250.96179426,278.92288151)(251.0167942,278.92788151)(251.07679659,278.92789022)
\curveto(251.13679408,278.9378815)(251.19679402,278.95288148)(251.25679659,278.97289022)
\curveto(251.40679381,279.02288141)(251.55179367,279.07288136)(251.69179659,279.12289022)
\curveto(251.83179339,279.18288125)(251.96679325,279.25288118)(252.09679659,279.33289022)
\curveto(252.23679298,279.42288101)(252.35679286,279.52788091)(252.45679659,279.64789022)
\curveto(252.55679266,279.76788067)(252.65179257,279.89788054)(252.74179659,280.03789022)
\curveto(252.80179242,280.1378803)(252.84679237,280.24788019)(252.87679659,280.36789022)
\curveto(252.9167923,280.48787995)(252.96679225,280.59287984)(253.02679659,280.68289022)
\curveto(253.07679214,280.74287969)(253.14679207,280.78287965)(253.23679659,280.80289022)
\curveto(253.25679196,280.81287962)(253.28179194,280.81787962)(253.31179659,280.81789022)
\curveto(253.34179188,280.81787962)(253.36679185,280.82287961)(253.38679659,280.83289022)
}
}
{
\newrgbcolor{curcolor}{0 0 0}
\pscustom[linestyle=none,fillstyle=solid,fillcolor=curcolor]
{
\newpath
\moveto(258.63140596,271.86289022)
\lineto(258.93140596,271.86289022)
\curveto(259.0414039,271.87288856)(259.1464038,271.87288856)(259.24640596,271.86289022)
\curveto(259.35640359,271.86288857)(259.45640349,271.85288858)(259.54640596,271.83289022)
\curveto(259.63640331,271.82288861)(259.70640324,271.79788864)(259.75640596,271.75789022)
\curveto(259.77640317,271.7378887)(259.79140315,271.70788873)(259.80140596,271.66789022)
\curveto(259.82140312,271.62788881)(259.8414031,271.58288885)(259.86140596,271.53289022)
\lineto(259.86140596,271.45789022)
\curveto(259.87140307,271.40788903)(259.87140307,271.35288908)(259.86140596,271.29289022)
\lineto(259.86140596,271.14289022)
\lineto(259.86140596,270.66289022)
\curveto(259.86140308,270.49288994)(259.82140312,270.37289006)(259.74140596,270.30289022)
\curveto(259.67140327,270.25289018)(259.58140336,270.22789021)(259.47140596,270.22789022)
\lineto(259.14140596,270.22789022)
\lineto(258.69140596,270.22789022)
\curveto(258.5414044,270.22789021)(258.42640452,270.25789018)(258.34640596,270.31789022)
\curveto(258.30640464,270.34789009)(258.27640467,270.39789004)(258.25640596,270.46789022)
\curveto(258.23640471,270.54788989)(258.22140472,270.6328898)(258.21140596,270.72289022)
\lineto(258.21140596,271.00789022)
\curveto(258.22140472,271.10788933)(258.22640472,271.19288924)(258.22640596,271.26289022)
\lineto(258.22640596,271.45789022)
\curveto(258.22640472,271.51788892)(258.23640471,271.57288886)(258.25640596,271.62289022)
\curveto(258.29640465,271.7328887)(258.36640458,271.80288863)(258.46640596,271.83289022)
\curveto(258.49640445,271.8328886)(258.55140439,271.84288859)(258.63140596,271.86289022)
}
}
{
\newrgbcolor{curcolor}{0 0 0}
\pscustom[linestyle=none,fillstyle=solid,fillcolor=curcolor]
{
\newpath
\moveto(268.85156221,273.30289022)
\curveto(268.86155449,273.26288717)(268.86155449,273.21288722)(268.85156221,273.15289022)
\curveto(268.8515545,273.09288734)(268.84655451,273.04288739)(268.83656221,273.00289022)
\curveto(268.83655452,272.96288747)(268.83155452,272.92288751)(268.82156221,272.88289022)
\lineto(268.82156221,272.77789022)
\curveto(268.80155455,272.69788774)(268.78655457,272.61788782)(268.77656221,272.53789022)
\curveto(268.76655459,272.45788798)(268.74655461,272.38288805)(268.71656221,272.31289022)
\curveto(268.69655466,272.2328882)(268.67655468,272.15788828)(268.65656221,272.08789022)
\curveto(268.63655472,272.01788842)(268.60655475,271.94288849)(268.56656221,271.86289022)
\curveto(268.38655497,271.44288899)(268.13155522,271.10288933)(267.80156221,270.84289022)
\curveto(267.47155588,270.58288985)(267.08155627,270.37789006)(266.63156221,270.22789022)
\curveto(266.51155684,270.18789025)(266.38655697,270.16289027)(266.25656221,270.15289022)
\curveto(266.13655722,270.1328903)(266.01155734,270.10789033)(265.88156221,270.07789022)
\curveto(265.82155753,270.06789037)(265.7565576,270.06289037)(265.68656221,270.06289022)
\curveto(265.62655773,270.06289037)(265.56155779,270.05789038)(265.49156221,270.04789022)
\lineto(265.37156221,270.04789022)
\lineto(265.17656221,270.04789022)
\curveto(265.11655824,270.0378904)(265.06155829,270.04289039)(265.01156221,270.06289022)
\curveto(264.94155841,270.08289035)(264.87655848,270.08789035)(264.81656221,270.07789022)
\curveto(264.7565586,270.06789037)(264.69655866,270.07289036)(264.63656221,270.09289022)
\curveto(264.58655877,270.10289033)(264.54155881,270.10789033)(264.50156221,270.10789022)
\curveto(264.46155889,270.10789033)(264.41655894,270.11789032)(264.36656221,270.13789022)
\curveto(264.28655907,270.15789028)(264.21155914,270.17789026)(264.14156221,270.19789022)
\curveto(264.07155928,270.20789023)(264.00155935,270.22289021)(263.93156221,270.24289022)
\curveto(263.4515599,270.41289002)(263.0515603,270.62288981)(262.73156221,270.87289022)
\curveto(262.42156093,271.1328893)(262.17156118,271.48788895)(261.98156221,271.93789022)
\curveto(261.9515614,271.99788844)(261.92656143,272.05788838)(261.90656221,272.11789022)
\curveto(261.89656146,272.18788825)(261.88156147,272.26288817)(261.86156221,272.34289022)
\curveto(261.84156151,272.40288803)(261.82656153,272.46788797)(261.81656221,272.53789022)
\curveto(261.80656155,272.60788783)(261.79156156,272.67788776)(261.77156221,272.74789022)
\curveto(261.76156159,272.79788764)(261.7565616,272.8378876)(261.75656221,272.86789022)
\lineto(261.75656221,272.98789022)
\curveto(261.74656161,273.02788741)(261.73656162,273.07788736)(261.72656221,273.13789022)
\curveto(261.72656163,273.19788724)(261.73156162,273.24788719)(261.74156221,273.28789022)
\lineto(261.74156221,273.42289022)
\curveto(261.7515616,273.47288696)(261.7565616,273.52288691)(261.75656221,273.57289022)
\curveto(261.77656158,273.67288676)(261.79156156,273.76788667)(261.80156221,273.85789022)
\curveto(261.81156154,273.95788648)(261.83156152,274.05288638)(261.86156221,274.14289022)
\curveto(261.91156144,274.29288614)(261.96656139,274.432886)(262.02656221,274.56289022)
\curveto(262.08656127,274.69288574)(262.1565612,274.81288562)(262.23656221,274.92289022)
\curveto(262.26656109,274.97288546)(262.29656106,275.01288542)(262.32656221,275.04289022)
\curveto(262.36656099,275.07288536)(262.40156095,275.10788533)(262.43156221,275.14789022)
\curveto(262.49156086,275.22788521)(262.56156079,275.29788514)(262.64156221,275.35789022)
\curveto(262.70156065,275.40788503)(262.76156059,275.45288498)(262.82156221,275.49289022)
\lineto(263.03156221,275.64289022)
\curveto(263.08156027,275.68288475)(263.13156022,275.71788472)(263.18156221,275.74789022)
\curveto(263.23156012,275.78788465)(263.26656009,275.84288459)(263.28656221,275.91289022)
\curveto(263.28656007,275.94288449)(263.27656008,275.96788447)(263.25656221,275.98789022)
\curveto(263.24656011,276.01788442)(263.23656012,276.04288439)(263.22656221,276.06289022)
\curveto(263.18656017,276.11288432)(263.13656022,276.15788428)(263.07656221,276.19789022)
\curveto(263.02656033,276.24788419)(262.97656038,276.29288414)(262.92656221,276.33289022)
\curveto(262.88656047,276.36288407)(262.83656052,276.41788402)(262.77656221,276.49789022)
\curveto(262.7565606,276.52788391)(262.72656063,276.55288388)(262.68656221,276.57289022)
\curveto(262.6565607,276.60288383)(262.63156072,276.6378838)(262.61156221,276.67789022)
\curveto(262.44156091,276.88788355)(262.31156104,277.1328833)(262.22156221,277.41289022)
\curveto(262.20156115,277.49288294)(262.18656117,277.57288286)(262.17656221,277.65289022)
\curveto(262.16656119,277.7328827)(262.1515612,277.81288262)(262.13156221,277.89289022)
\curveto(262.11156124,277.94288249)(262.10156125,278.00788243)(262.10156221,278.08789022)
\curveto(262.10156125,278.17788226)(262.11156124,278.24788219)(262.13156221,278.29789022)
\curveto(262.13156122,278.39788204)(262.13656122,278.46788197)(262.14656221,278.50789022)
\curveto(262.16656119,278.58788185)(262.18156117,278.65788178)(262.19156221,278.71789022)
\curveto(262.20156115,278.78788165)(262.21656114,278.85788158)(262.23656221,278.92789022)
\curveto(262.38656097,279.35788108)(262.60156075,279.70288073)(262.88156221,279.96289022)
\curveto(263.17156018,280.22288021)(263.52155983,280.43788)(263.93156221,280.60789022)
\curveto(264.04155931,280.65787978)(264.1565592,280.68787975)(264.27656221,280.69789022)
\curveto(264.40655895,280.71787972)(264.53655882,280.74787969)(264.66656221,280.78789022)
\curveto(264.74655861,280.78787965)(264.81655854,280.78787965)(264.87656221,280.78789022)
\curveto(264.94655841,280.79787964)(265.02155833,280.80787963)(265.10156221,280.81789022)
\curveto(265.89155746,280.8378796)(266.54655681,280.70787973)(267.06656221,280.42789022)
\curveto(267.59655576,280.14788029)(267.97655538,279.7378807)(268.20656221,279.19789022)
\curveto(268.31655504,278.96788147)(268.38655497,278.68288175)(268.41656221,278.34289022)
\curveto(268.4565549,278.01288242)(268.42655493,277.70788273)(268.32656221,277.42789022)
\curveto(268.28655507,277.29788314)(268.23655512,277.17788326)(268.17656221,277.06789022)
\curveto(268.12655523,276.95788348)(268.06655529,276.85288358)(267.99656221,276.75289022)
\curveto(267.97655538,276.71288372)(267.94655541,276.67788376)(267.90656221,276.64789022)
\lineto(267.81656221,276.55789022)
\curveto(267.76655559,276.46788397)(267.70655565,276.40288403)(267.63656221,276.36289022)
\curveto(267.58655577,276.31288412)(267.53155582,276.26288417)(267.47156221,276.21289022)
\curveto(267.42155593,276.17288426)(267.37655598,276.12788431)(267.33656221,276.07789022)
\curveto(267.31655604,276.05788438)(267.29655606,276.0328844)(267.27656221,276.00289022)
\curveto(267.26655609,275.98288445)(267.26655609,275.95788448)(267.27656221,275.92789022)
\curveto(267.28655607,275.87788456)(267.31655604,275.82788461)(267.36656221,275.77789022)
\curveto(267.41655594,275.7378847)(267.47155588,275.69788474)(267.53156221,275.65789022)
\lineto(267.71156221,275.53789022)
\curveto(267.77155558,275.50788493)(267.82155553,275.47788496)(267.86156221,275.44789022)
\curveto(268.19155516,275.20788523)(268.44155491,274.89788554)(268.61156221,274.51789022)
\curveto(268.6515547,274.437886)(268.68155467,274.35288608)(268.70156221,274.26289022)
\curveto(268.73155462,274.17288626)(268.7565546,274.08288635)(268.77656221,273.99289022)
\curveto(268.78655457,273.94288649)(268.79655456,273.88788655)(268.80656221,273.82789022)
\lineto(268.83656221,273.67789022)
\curveto(268.84655451,273.61788682)(268.84655451,273.55288688)(268.83656221,273.48289022)
\curveto(268.82655453,273.42288701)(268.83155452,273.36288707)(268.85156221,273.30289022)
\moveto(263.46656221,278.34289022)
\curveto(263.43655992,278.2328822)(263.43155992,278.09288234)(263.45156221,277.92289022)
\curveto(263.47155988,277.76288267)(263.49655986,277.6378828)(263.52656221,277.54789022)
\curveto(263.63655972,277.22788321)(263.78655957,276.98288345)(263.97656221,276.81289022)
\curveto(264.16655919,276.65288378)(264.43155892,276.52288391)(264.77156221,276.42289022)
\curveto(264.90155845,276.39288404)(265.06655829,276.36788407)(265.26656221,276.34789022)
\curveto(265.46655789,276.3378841)(265.63655772,276.35288408)(265.77656221,276.39289022)
\curveto(266.06655729,276.47288396)(266.30655705,276.58288385)(266.49656221,276.72289022)
\curveto(266.69655666,276.87288356)(266.8515565,277.07288336)(266.96156221,277.32289022)
\curveto(266.98155637,277.37288306)(266.99155636,277.41788302)(266.99156221,277.45789022)
\curveto(267.00155635,277.49788294)(267.01655634,277.54288289)(267.03656221,277.59289022)
\curveto(267.06655629,277.70288273)(267.08655627,277.84288259)(267.09656221,278.01289022)
\curveto(267.10655625,278.18288225)(267.09655626,278.32788211)(267.06656221,278.44789022)
\curveto(267.04655631,278.5378819)(267.02155633,278.62288181)(266.99156221,278.70289022)
\curveto(266.97155638,278.78288165)(266.93655642,278.86288157)(266.88656221,278.94289022)
\curveto(266.71655664,279.21288122)(266.49155686,279.40788103)(266.21156221,279.52789022)
\curveto(265.94155741,279.64788079)(265.58155777,279.70788073)(265.13156221,279.70789022)
\curveto(265.11155824,279.68788075)(265.08155827,279.68288075)(265.04156221,279.69289022)
\curveto(265.00155835,279.70288073)(264.96655839,279.70288073)(264.93656221,279.69289022)
\curveto(264.88655847,279.67288076)(264.83155852,279.65788078)(264.77156221,279.64789022)
\curveto(264.72155863,279.64788079)(264.67155868,279.6378808)(264.62156221,279.61789022)
\curveto(264.38155897,279.52788091)(264.17155918,279.41288102)(263.99156221,279.27289022)
\curveto(263.81155954,279.14288129)(263.67155968,278.96288147)(263.57156221,278.73289022)
\curveto(263.5515598,278.67288176)(263.53155982,278.60788183)(263.51156221,278.53789022)
\curveto(263.50155985,278.47788196)(263.48655987,278.41288202)(263.46656221,278.34289022)
\moveto(267.48656221,272.80789022)
\curveto(267.53655582,272.99788744)(267.54155581,273.20288723)(267.50156221,273.42289022)
\curveto(267.47155588,273.64288679)(267.42655593,273.82288661)(267.36656221,273.96289022)
\curveto(267.19655616,274.3328861)(266.93655642,274.6378858)(266.58656221,274.87789022)
\curveto(266.24655711,275.11788532)(265.81155754,275.2378852)(265.28156221,275.23789022)
\curveto(265.2515581,275.21788522)(265.21155814,275.21288522)(265.16156221,275.22289022)
\curveto(265.11155824,275.24288519)(265.07155828,275.24788519)(265.04156221,275.23789022)
\lineto(264.77156221,275.17789022)
\curveto(264.69155866,275.16788527)(264.61155874,275.15288528)(264.53156221,275.13289022)
\curveto(264.23155912,275.02288541)(263.96655939,274.87788556)(263.73656221,274.69789022)
\curveto(263.51655984,274.51788592)(263.34656001,274.28788615)(263.22656221,274.00789022)
\curveto(263.19656016,273.92788651)(263.17156018,273.84788659)(263.15156221,273.76789022)
\curveto(263.13156022,273.68788675)(263.11156024,273.60288683)(263.09156221,273.51289022)
\curveto(263.06156029,273.39288704)(263.0515603,273.24288719)(263.06156221,273.06289022)
\curveto(263.08156027,272.88288755)(263.10656025,272.74288769)(263.13656221,272.64289022)
\curveto(263.1565602,272.59288784)(263.16656019,272.54788789)(263.16656221,272.50789022)
\curveto(263.17656018,272.47788796)(263.19156016,272.437888)(263.21156221,272.38789022)
\curveto(263.31156004,272.16788827)(263.44155991,271.96788847)(263.60156221,271.78789022)
\curveto(263.77155958,271.60788883)(263.96655939,271.47288896)(264.18656221,271.38289022)
\curveto(264.2565591,271.34288909)(264.351559,271.30788913)(264.47156221,271.27789022)
\curveto(264.69155866,271.18788925)(264.94655841,271.14288929)(265.23656221,271.14289022)
\lineto(265.52156221,271.14289022)
\curveto(265.62155773,271.16288927)(265.71655764,271.17788926)(265.80656221,271.18789022)
\curveto(265.89655746,271.19788924)(265.98655737,271.21788922)(266.07656221,271.24789022)
\curveto(266.33655702,271.32788911)(266.57655678,271.45788898)(266.79656221,271.63789022)
\curveto(267.02655633,271.82788861)(267.19655616,272.04288839)(267.30656221,272.28289022)
\curveto(267.34655601,272.36288807)(267.37655598,272.44288799)(267.39656221,272.52289022)
\curveto(267.42655593,272.61288782)(267.4565559,272.70788773)(267.48656221,272.80789022)
}
}
{
\newrgbcolor{curcolor}{0 0 0}
\pscustom[linestyle=none,fillstyle=solid,fillcolor=curcolor]
{
\newpath
\moveto(279.99117159,278.74789022)
\curveto(279.79116129,278.45788198)(279.5811615,278.17288226)(279.36117159,277.89289022)
\curveto(279.15116193,277.61288282)(278.94616213,277.32788311)(278.74617159,277.03789022)
\curveto(278.14616293,276.18788425)(277.54116354,275.34788509)(276.93117159,274.51789022)
\curveto(276.32116476,273.69788674)(275.71616536,272.86288757)(275.11617159,272.01289022)
\lineto(274.60617159,271.29289022)
\lineto(274.09617159,270.60289022)
\curveto(274.01616706,270.49288994)(273.93616714,270.37789006)(273.85617159,270.25789022)
\curveto(273.7761673,270.1378903)(273.6811674,270.04289039)(273.57117159,269.97289022)
\curveto(273.53116755,269.95289048)(273.46616761,269.9378905)(273.37617159,269.92789022)
\curveto(273.29616778,269.90789053)(273.20616787,269.89789054)(273.10617159,269.89789022)
\curveto(273.00616807,269.89789054)(272.91116817,269.90289053)(272.82117159,269.91289022)
\curveto(272.74116834,269.92289051)(272.6811684,269.94289049)(272.64117159,269.97289022)
\curveto(272.61116847,269.99289044)(272.58616849,270.02789041)(272.56617159,270.07789022)
\curveto(272.55616852,270.11789032)(272.56116852,270.16289027)(272.58117159,270.21289022)
\curveto(272.62116846,270.29289014)(272.66616841,270.36789007)(272.71617159,270.43789022)
\curveto(272.7761683,270.51788992)(272.83116825,270.59788984)(272.88117159,270.67789022)
\curveto(273.12116796,271.01788942)(273.36616771,271.35288908)(273.61617159,271.68289022)
\curveto(273.86616721,272.01288842)(274.10616697,272.34788809)(274.33617159,272.68789022)
\curveto(274.49616658,272.90788753)(274.65616642,273.12288731)(274.81617159,273.33289022)
\curveto(274.9761661,273.54288689)(275.13616594,273.75788668)(275.29617159,273.97789022)
\curveto(275.65616542,274.49788594)(276.02116506,275.00788543)(276.39117159,275.50789022)
\curveto(276.76116432,276.00788443)(277.13116395,276.51788392)(277.50117159,277.03789022)
\curveto(277.64116344,277.2378832)(277.7811633,277.432883)(277.92117159,277.62289022)
\curveto(278.07116301,277.81288262)(278.21616286,278.00788243)(278.35617159,278.20789022)
\curveto(278.56616251,278.50788193)(278.7811623,278.80788163)(279.00117159,279.10789022)
\lineto(279.66117159,280.00789022)
\lineto(279.84117159,280.27789022)
\lineto(280.05117159,280.54789022)
\lineto(280.17117159,280.72789022)
\curveto(280.22116086,280.78787965)(280.27116081,280.84287959)(280.32117159,280.89289022)
\curveto(280.39116069,280.94287949)(280.46616061,280.97787946)(280.54617159,280.99789022)
\curveto(280.56616051,281.00787943)(280.59116049,281.00787943)(280.62117159,280.99789022)
\curveto(280.66116042,280.99787944)(280.69116039,281.00787943)(280.71117159,281.02789022)
\curveto(280.83116025,281.02787941)(280.96616011,281.02287941)(281.11617159,281.01289022)
\curveto(281.26615981,281.01287942)(281.35615972,280.96787947)(281.38617159,280.87789022)
\curveto(281.40615967,280.84787959)(281.41115967,280.81287962)(281.40117159,280.77289022)
\curveto(281.39115969,280.7328797)(281.3761597,280.70287973)(281.35617159,280.68289022)
\curveto(281.31615976,280.60287983)(281.2761598,280.5328799)(281.23617159,280.47289022)
\curveto(281.19615988,280.41288002)(281.15115993,280.35288008)(281.10117159,280.29289022)
\lineto(280.53117159,279.51289022)
\curveto(280.35116073,279.26288117)(280.17116091,279.00788143)(279.99117159,278.74789022)
\moveto(273.13617159,274.84789022)
\curveto(273.08616799,274.86788557)(273.03616804,274.87288556)(272.98617159,274.86289022)
\curveto(272.93616814,274.85288558)(272.88616819,274.85788558)(272.83617159,274.87789022)
\curveto(272.72616835,274.89788554)(272.62116846,274.91788552)(272.52117159,274.93789022)
\curveto(272.43116865,274.96788547)(272.33616874,275.00788543)(272.23617159,275.05789022)
\curveto(271.90616917,275.19788524)(271.65116943,275.39288504)(271.47117159,275.64289022)
\curveto(271.29116979,275.90288453)(271.14616993,276.21288422)(271.03617159,276.57289022)
\curveto(271.00617007,276.65288378)(270.98617009,276.7328837)(270.97617159,276.81289022)
\curveto(270.96617011,276.90288353)(270.95117013,276.98788345)(270.93117159,277.06789022)
\curveto(270.92117016,277.11788332)(270.91617016,277.18288325)(270.91617159,277.26289022)
\curveto(270.90617017,277.29288314)(270.90117018,277.32288311)(270.90117159,277.35289022)
\curveto(270.90117018,277.39288304)(270.89617018,277.42788301)(270.88617159,277.45789022)
\lineto(270.88617159,277.60789022)
\curveto(270.8761702,277.65788278)(270.87117021,277.71788272)(270.87117159,277.78789022)
\curveto(270.87117021,277.86788257)(270.8761702,277.9328825)(270.88617159,277.98289022)
\lineto(270.88617159,278.14789022)
\curveto(270.90617017,278.19788224)(270.91117017,278.24288219)(270.90117159,278.28289022)
\curveto(270.90117018,278.3328821)(270.90617017,278.37788206)(270.91617159,278.41789022)
\curveto(270.92617015,278.45788198)(270.93117015,278.49288194)(270.93117159,278.52289022)
\curveto(270.93117015,278.56288187)(270.93617014,278.60288183)(270.94617159,278.64289022)
\curveto(270.9761701,278.75288168)(270.99617008,278.86288157)(271.00617159,278.97289022)
\curveto(271.02617005,279.09288134)(271.06117002,279.20788123)(271.11117159,279.31789022)
\curveto(271.25116983,279.65788078)(271.41116967,279.9328805)(271.59117159,280.14289022)
\curveto(271.7811693,280.36288007)(272.05116903,280.54287989)(272.40117159,280.68289022)
\curveto(272.4811686,280.71287972)(272.56616851,280.7328797)(272.65617159,280.74289022)
\curveto(272.74616833,280.76287967)(272.84116824,280.78287965)(272.94117159,280.80289022)
\curveto(272.97116811,280.81287962)(273.02616805,280.81287962)(273.10617159,280.80289022)
\curveto(273.18616789,280.80287963)(273.23616784,280.81287962)(273.25617159,280.83289022)
\curveto(273.81616726,280.84287959)(274.26616681,280.7328797)(274.60617159,280.50289022)
\curveto(274.95616612,280.27288016)(275.21616586,279.96788047)(275.38617159,279.58789022)
\curveto(275.42616565,279.49788094)(275.46116562,279.40288103)(275.49117159,279.30289022)
\curveto(275.52116556,279.20288123)(275.54616553,279.10288133)(275.56617159,279.00289022)
\curveto(275.58616549,278.97288146)(275.59116549,278.94288149)(275.58117159,278.91289022)
\curveto(275.5811655,278.88288155)(275.58616549,278.85288158)(275.59617159,278.82289022)
\curveto(275.62616545,278.71288172)(275.64616543,278.58788185)(275.65617159,278.44789022)
\curveto(275.66616541,278.31788212)(275.6761654,278.18288225)(275.68617159,278.04289022)
\lineto(275.68617159,277.87789022)
\curveto(275.69616538,277.81788262)(275.69616538,277.76288267)(275.68617159,277.71289022)
\curveto(275.6761654,277.66288277)(275.67116541,277.61288282)(275.67117159,277.56289022)
\lineto(275.67117159,277.42789022)
\curveto(275.66116542,277.38788305)(275.65616542,277.34788309)(275.65617159,277.30789022)
\curveto(275.66616541,277.26788317)(275.66116542,277.22288321)(275.64117159,277.17289022)
\curveto(275.62116546,277.06288337)(275.60116548,276.95788348)(275.58117159,276.85789022)
\curveto(275.57116551,276.75788368)(275.55116553,276.65788378)(275.52117159,276.55789022)
\curveto(275.39116569,276.19788424)(275.22616585,275.88288455)(275.02617159,275.61289022)
\curveto(274.82616625,275.34288509)(274.55116653,275.1378853)(274.20117159,274.99789022)
\curveto(274.12116696,274.96788547)(274.03616704,274.94288549)(273.94617159,274.92289022)
\lineto(273.67617159,274.86289022)
\curveto(273.62616745,274.85288558)(273.5811675,274.84788559)(273.54117159,274.84789022)
\curveto(273.50116758,274.85788558)(273.46116762,274.85788558)(273.42117159,274.84789022)
\curveto(273.32116776,274.82788561)(273.22616785,274.82788561)(273.13617159,274.84789022)
\moveto(272.29617159,276.24289022)
\curveto(272.33616874,276.17288426)(272.3761687,276.10788433)(272.41617159,276.04789022)
\curveto(272.45616862,275.99788444)(272.50616857,275.94788449)(272.56617159,275.89789022)
\lineto(272.71617159,275.77789022)
\curveto(272.7761683,275.74788469)(272.84116824,275.72288471)(272.91117159,275.70289022)
\curveto(272.95116813,275.68288475)(272.98616809,275.67288476)(273.01617159,275.67289022)
\curveto(273.05616802,275.68288475)(273.09616798,275.67788476)(273.13617159,275.65789022)
\curveto(273.16616791,275.65788478)(273.20616787,275.65288478)(273.25617159,275.64289022)
\curveto(273.30616777,275.64288479)(273.34616773,275.64788479)(273.37617159,275.65789022)
\lineto(273.60117159,275.70289022)
\curveto(273.85116723,275.78288465)(274.03616704,275.90788453)(274.15617159,276.07789022)
\curveto(274.23616684,276.17788426)(274.30616677,276.30788413)(274.36617159,276.46789022)
\curveto(274.44616663,276.64788379)(274.50616657,276.87288356)(274.54617159,277.14289022)
\curveto(274.58616649,277.42288301)(274.60116648,277.70288273)(274.59117159,277.98289022)
\curveto(274.5811665,278.27288216)(274.55116653,278.54788189)(274.50117159,278.80789022)
\curveto(274.45116663,279.06788137)(274.3761667,279.27788116)(274.27617159,279.43789022)
\curveto(274.15616692,279.6378808)(274.00616707,279.78788065)(273.82617159,279.88789022)
\curveto(273.74616733,279.9378805)(273.65616742,279.96788047)(273.55617159,279.97789022)
\curveto(273.45616762,279.99788044)(273.35116773,280.00788043)(273.24117159,280.00789022)
\curveto(273.22116786,279.99788044)(273.19616788,279.99288044)(273.16617159,279.99289022)
\curveto(273.14616793,280.00288043)(273.12616795,280.00288043)(273.10617159,279.99289022)
\curveto(273.05616802,279.98288045)(273.01116807,279.97288046)(272.97117159,279.96289022)
\curveto(272.93116815,279.96288047)(272.89116819,279.95288048)(272.85117159,279.93289022)
\curveto(272.67116841,279.85288058)(272.52116856,279.7328807)(272.40117159,279.57289022)
\curveto(272.29116879,279.41288102)(272.20116888,279.2328812)(272.13117159,279.03289022)
\curveto(272.07116901,278.84288159)(272.02616905,278.61788182)(271.99617159,278.35789022)
\curveto(271.9761691,278.09788234)(271.97116911,277.8328826)(271.98117159,277.56289022)
\curveto(271.99116909,277.30288313)(272.02116906,277.05288338)(272.07117159,276.81289022)
\curveto(272.13116895,276.58288385)(272.20616887,276.39288404)(272.29617159,276.24289022)
\moveto(283.09617159,273.25789022)
\curveto(283.10615797,273.20788723)(283.11115797,273.11788732)(283.11117159,272.98789022)
\curveto(283.11115797,272.85788758)(283.10115798,272.76788767)(283.08117159,272.71789022)
\curveto(283.06115802,272.66788777)(283.05615802,272.61288782)(283.06617159,272.55289022)
\curveto(283.076158,272.50288793)(283.076158,272.45288798)(283.06617159,272.40289022)
\curveto(283.02615805,272.26288817)(282.99615808,272.12788831)(282.97617159,271.99789022)
\curveto(282.96615811,271.86788857)(282.93615814,271.74788869)(282.88617159,271.63789022)
\curveto(282.74615833,271.28788915)(282.5811585,270.99288944)(282.39117159,270.75289022)
\curveto(282.20115888,270.52288991)(281.93115915,270.3378901)(281.58117159,270.19789022)
\curveto(281.50115958,270.16789027)(281.41615966,270.14789029)(281.32617159,270.13789022)
\curveto(281.23615984,270.11789032)(281.15115993,270.09789034)(281.07117159,270.07789022)
\curveto(281.02116006,270.06789037)(280.97116011,270.06289037)(280.92117159,270.06289022)
\curveto(280.87116021,270.06289037)(280.82116026,270.05789038)(280.77117159,270.04789022)
\curveto(280.74116034,270.0378904)(280.69116039,270.0378904)(280.62117159,270.04789022)
\curveto(280.55116053,270.04789039)(280.50116058,270.05289038)(280.47117159,270.06289022)
\curveto(280.41116067,270.08289035)(280.35116073,270.09289034)(280.29117159,270.09289022)
\curveto(280.24116084,270.08289035)(280.19116089,270.08789035)(280.14117159,270.10789022)
\curveto(280.05116103,270.12789031)(279.96116112,270.15289028)(279.87117159,270.18289022)
\curveto(279.79116129,270.20289023)(279.71116137,270.2328902)(279.63117159,270.27289022)
\curveto(279.31116177,270.41289002)(279.06116202,270.60788983)(278.88117159,270.85789022)
\curveto(278.70116238,271.11788932)(278.55116253,271.42288901)(278.43117159,271.77289022)
\curveto(278.41116267,271.85288858)(278.39616268,271.9378885)(278.38617159,272.02789022)
\curveto(278.3761627,272.11788832)(278.36116272,272.20288823)(278.34117159,272.28289022)
\curveto(278.33116275,272.31288812)(278.32616275,272.34288809)(278.32617159,272.37289022)
\lineto(278.32617159,272.47789022)
\curveto(278.30616277,272.55788788)(278.29616278,272.6378878)(278.29617159,272.71789022)
\lineto(278.29617159,272.85289022)
\curveto(278.2761628,272.95288748)(278.2761628,273.05288738)(278.29617159,273.15289022)
\lineto(278.29617159,273.33289022)
\curveto(278.30616277,273.38288705)(278.31116277,273.42788701)(278.31117159,273.46789022)
\curveto(278.31116277,273.51788692)(278.31616276,273.56288687)(278.32617159,273.60289022)
\curveto(278.33616274,273.64288679)(278.34116274,273.67788676)(278.34117159,273.70789022)
\curveto(278.34116274,273.74788669)(278.34616273,273.78788665)(278.35617159,273.82789022)
\lineto(278.41617159,274.15789022)
\curveto(278.43616264,274.27788616)(278.46616261,274.38788605)(278.50617159,274.48789022)
\curveto(278.64616243,274.81788562)(278.80616227,275.09288534)(278.98617159,275.31289022)
\curveto(279.1761619,275.54288489)(279.43616164,275.72788471)(279.76617159,275.86789022)
\curveto(279.84616123,275.90788453)(279.93116115,275.9328845)(280.02117159,275.94289022)
\lineto(280.32117159,276.00289022)
\lineto(280.45617159,276.00289022)
\curveto(280.50616057,276.01288442)(280.55616052,276.01788442)(280.60617159,276.01789022)
\curveto(281.1761599,276.0378844)(281.63615944,275.9328845)(281.98617159,275.70289022)
\curveto(282.34615873,275.48288495)(282.61115847,275.18288525)(282.78117159,274.80289022)
\curveto(282.83115825,274.70288573)(282.87115821,274.60288583)(282.90117159,274.50289022)
\curveto(282.93115815,274.40288603)(282.96115812,274.29788614)(282.99117159,274.18789022)
\curveto(283.00115808,274.14788629)(283.00615807,274.11288632)(283.00617159,274.08289022)
\curveto(283.00615807,274.06288637)(283.01115807,274.0328864)(283.02117159,273.99289022)
\curveto(283.04115804,273.92288651)(283.05115803,273.84788659)(283.05117159,273.76789022)
\curveto(283.05115803,273.68788675)(283.06115802,273.60788683)(283.08117159,273.52789022)
\curveto(283.081158,273.47788696)(283.081158,273.432887)(283.08117159,273.39289022)
\curveto(283.081158,273.35288708)(283.08615799,273.30788713)(283.09617159,273.25789022)
\moveto(281.98617159,272.82289022)
\curveto(281.99615908,272.87288756)(282.00115908,272.94788749)(282.00117159,273.04789022)
\curveto(282.01115907,273.14788729)(282.00615907,273.22288721)(281.98617159,273.27289022)
\curveto(281.96615911,273.3328871)(281.96115912,273.38788705)(281.97117159,273.43789022)
\curveto(281.99115909,273.49788694)(281.99115909,273.55788688)(281.97117159,273.61789022)
\curveto(281.96115912,273.64788679)(281.95615912,273.68288675)(281.95617159,273.72289022)
\curveto(281.95615912,273.76288667)(281.95115913,273.80288663)(281.94117159,273.84289022)
\curveto(281.92115916,273.92288651)(281.90115918,273.99788644)(281.88117159,274.06789022)
\curveto(281.87115921,274.14788629)(281.85615922,274.22788621)(281.83617159,274.30789022)
\curveto(281.80615927,274.36788607)(281.7811593,274.42788601)(281.76117159,274.48789022)
\curveto(281.74115934,274.54788589)(281.71115937,274.60788583)(281.67117159,274.66789022)
\curveto(281.57115951,274.8378856)(281.44115964,274.97288546)(281.28117159,275.07289022)
\curveto(281.20115988,275.12288531)(281.10615997,275.15788528)(280.99617159,275.17789022)
\curveto(280.88616019,275.19788524)(280.76116032,275.20788523)(280.62117159,275.20789022)
\curveto(280.60116048,275.19788524)(280.5761605,275.19288524)(280.54617159,275.19289022)
\curveto(280.51616056,275.20288523)(280.48616059,275.20288523)(280.45617159,275.19289022)
\lineto(280.30617159,275.13289022)
\curveto(280.25616082,275.12288531)(280.21116087,275.10788533)(280.17117159,275.08789022)
\curveto(279.9811611,274.97788546)(279.83616124,274.8328856)(279.73617159,274.65289022)
\curveto(279.64616143,274.47288596)(279.56616151,274.26788617)(279.49617159,274.03789022)
\curveto(279.45616162,273.90788653)(279.43616164,273.77288666)(279.43617159,273.63289022)
\curveto(279.43616164,273.50288693)(279.42616165,273.35788708)(279.40617159,273.19789022)
\curveto(279.39616168,273.14788729)(279.38616169,273.08788735)(279.37617159,273.01789022)
\curveto(279.3761617,272.94788749)(279.38616169,272.88788755)(279.40617159,272.83789022)
\lineto(279.40617159,272.67289022)
\lineto(279.40617159,272.49289022)
\curveto(279.41616166,272.44288799)(279.42616165,272.38788805)(279.43617159,272.32789022)
\curveto(279.44616163,272.27788816)(279.45116163,272.22288821)(279.45117159,272.16289022)
\curveto(279.46116162,272.10288833)(279.4761616,272.04788839)(279.49617159,271.99789022)
\curveto(279.54616153,271.80788863)(279.60616147,271.6328888)(279.67617159,271.47289022)
\curveto(279.74616133,271.31288912)(279.85116123,271.18288925)(279.99117159,271.08289022)
\curveto(280.12116096,270.98288945)(280.26116082,270.91288952)(280.41117159,270.87289022)
\curveto(280.44116064,270.86288957)(280.46616061,270.85788958)(280.48617159,270.85789022)
\curveto(280.51616056,270.86788957)(280.54616053,270.86788957)(280.57617159,270.85789022)
\curveto(280.59616048,270.85788958)(280.62616045,270.85288958)(280.66617159,270.84289022)
\curveto(280.70616037,270.84288959)(280.74116034,270.84788959)(280.77117159,270.85789022)
\curveto(280.81116027,270.86788957)(280.85116023,270.87288956)(280.89117159,270.87289022)
\curveto(280.93116015,270.87288956)(280.97116011,270.88288955)(281.01117159,270.90289022)
\curveto(281.25115983,270.98288945)(281.44615963,271.11788932)(281.59617159,271.30789022)
\curveto(281.71615936,271.48788895)(281.80615927,271.69288874)(281.86617159,271.92289022)
\curveto(281.88615919,271.99288844)(281.90115918,272.06288837)(281.91117159,272.13289022)
\curveto(281.92115916,272.21288822)(281.93615914,272.29288814)(281.95617159,272.37289022)
\curveto(281.95615912,272.432888)(281.96115912,272.47788796)(281.97117159,272.50789022)
\curveto(281.97115911,272.52788791)(281.97115911,272.55288788)(281.97117159,272.58289022)
\curveto(281.97115911,272.62288781)(281.9761591,272.65288778)(281.98617159,272.67289022)
\lineto(281.98617159,272.82289022)
}
}
{
\newrgbcolor{curcolor}{0 0 0}
\pscustom[linestyle=none,fillstyle=solid,fillcolor=curcolor]
{
\newpath
\moveto(313.51289278,167.26147909)
\curveto(314.20288814,167.27146846)(314.80288754,167.15146858)(315.31289278,166.90147909)
\curveto(315.83288651,166.65146908)(316.22788612,166.31646942)(316.49789278,165.89647909)
\curveto(316.5478858,165.81646992)(316.59288575,165.72647001)(316.63289278,165.62647909)
\curveto(316.67288567,165.5364702)(316.71788563,165.44147029)(316.76789278,165.34147909)
\curveto(316.80788554,165.24147049)(316.83788551,165.14147059)(316.85789278,165.04147909)
\curveto(316.87788547,164.94147079)(316.89788545,164.8364709)(316.91789278,164.72647909)
\curveto(316.93788541,164.67647106)(316.9428854,164.6314711)(316.93289278,164.59147909)
\curveto(316.92288542,164.55147118)(316.92788542,164.50647123)(316.94789278,164.45647909)
\curveto(316.95788539,164.40647133)(316.96288538,164.32147141)(316.96289278,164.20147909)
\curveto(316.96288538,164.09147164)(316.95788539,164.00647173)(316.94789278,163.94647909)
\curveto(316.92788542,163.88647185)(316.91788543,163.82647191)(316.91789278,163.76647909)
\curveto(316.92788542,163.70647203)(316.92288542,163.64647209)(316.90289278,163.58647909)
\curveto(316.86288548,163.44647229)(316.82788552,163.31147242)(316.79789278,163.18147909)
\curveto(316.76788558,163.05147268)(316.72788562,162.92647281)(316.67789278,162.80647909)
\curveto(316.61788573,162.66647307)(316.5478858,162.54147319)(316.46789278,162.43147909)
\curveto(316.39788595,162.32147341)(316.32288602,162.21147352)(316.24289278,162.10147909)
\lineto(316.18289278,162.04147909)
\curveto(316.17288617,162.02147371)(316.15788619,162.00147373)(316.13789278,161.98147909)
\curveto(316.01788633,161.82147391)(315.88288646,161.67647406)(315.73289278,161.54647909)
\curveto(315.58288676,161.41647432)(315.42288692,161.29147444)(315.25289278,161.17147909)
\curveto(314.9428874,160.95147478)(314.6478877,160.74647499)(314.36789278,160.55647909)
\curveto(314.13788821,160.41647532)(313.90788844,160.28147545)(313.67789278,160.15147909)
\curveto(313.45788889,160.02147571)(313.23788911,159.88647585)(313.01789278,159.74647909)
\curveto(312.76788958,159.57647616)(312.52788982,159.39647634)(312.29789278,159.20647909)
\curveto(312.07789027,159.01647672)(311.88789046,158.79147694)(311.72789278,158.53147909)
\curveto(311.68789066,158.47147726)(311.65289069,158.41147732)(311.62289278,158.35147909)
\curveto(311.59289075,158.30147743)(311.56289078,158.2364775)(311.53289278,158.15647909)
\curveto(311.51289083,158.08647765)(311.50789084,158.02647771)(311.51789278,157.97647909)
\curveto(311.53789081,157.90647783)(311.57289077,157.85147788)(311.62289278,157.81147909)
\curveto(311.67289067,157.78147795)(311.73289061,157.76147797)(311.80289278,157.75147909)
\lineto(312.04289278,157.75147909)
\lineto(312.79289278,157.75147909)
\lineto(315.59789278,157.75147909)
\lineto(316.25789278,157.75147909)
\curveto(316.347886,157.75147798)(316.43288591,157.74647799)(316.51289278,157.73647909)
\curveto(316.59288575,157.736478)(316.65788569,157.71647802)(316.70789278,157.67647909)
\curveto(316.75788559,157.6364781)(316.79788555,157.56147817)(316.82789278,157.45147909)
\curveto(316.86788548,157.35147838)(316.87788547,157.25147848)(316.85789278,157.15147909)
\lineto(316.85789278,157.01647909)
\curveto(316.83788551,156.94647879)(316.81788553,156.88647885)(316.79789278,156.83647909)
\curveto(316.77788557,156.78647895)(316.7428856,156.74647899)(316.69289278,156.71647909)
\curveto(316.6428857,156.67647906)(316.57288577,156.65647908)(316.48289278,156.65647909)
\lineto(316.21289278,156.65647909)
\lineto(315.31289278,156.65647909)
\lineto(311.80289278,156.65647909)
\lineto(310.73789278,156.65647909)
\curveto(310.65789169,156.65647908)(310.56789178,156.65147908)(310.46789278,156.64147909)
\curveto(310.36789198,156.64147909)(310.28289206,156.65147908)(310.21289278,156.67147909)
\curveto(310.00289234,156.74147899)(309.93789241,156.92147881)(310.01789278,157.21147909)
\curveto(310.02789232,157.25147848)(310.02789232,157.28647845)(310.01789278,157.31647909)
\curveto(310.01789233,157.35647838)(310.02789232,157.40147833)(310.04789278,157.45147909)
\curveto(310.06789228,157.5314782)(310.08789226,157.61647812)(310.10789278,157.70647909)
\curveto(310.12789222,157.79647794)(310.15289219,157.88147785)(310.18289278,157.96147909)
\curveto(310.342892,158.45147728)(310.5428918,158.86647687)(310.78289278,159.20647909)
\curveto(310.96289138,159.45647628)(311.16789118,159.68147605)(311.39789278,159.88147909)
\curveto(311.62789072,160.09147564)(311.86789048,160.28647545)(312.11789278,160.46647909)
\curveto(312.37788997,160.64647509)(312.6428897,160.81647492)(312.91289278,160.97647909)
\curveto(313.19288915,161.14647459)(313.46288888,161.32147441)(313.72289278,161.50147909)
\curveto(313.83288851,161.58147415)(313.93788841,161.65647408)(314.03789278,161.72647909)
\curveto(314.1478882,161.79647394)(314.25788809,161.87147386)(314.36789278,161.95147909)
\curveto(314.40788794,161.98147375)(314.4428879,162.01147372)(314.47289278,162.04147909)
\curveto(314.51288783,162.08147365)(314.55288779,162.11147362)(314.59289278,162.13147909)
\curveto(314.73288761,162.24147349)(314.85788749,162.36647337)(314.96789278,162.50647909)
\curveto(314.98788736,162.5364732)(315.01288733,162.56147317)(315.04289278,162.58147909)
\curveto(315.07288727,162.61147312)(315.09788725,162.64147309)(315.11789278,162.67147909)
\curveto(315.19788715,162.77147296)(315.26288708,162.87147286)(315.31289278,162.97147909)
\curveto(315.37288697,163.07147266)(315.42788692,163.18147255)(315.47789278,163.30147909)
\curveto(315.50788684,163.37147236)(315.52788682,163.44647229)(315.53789278,163.52647909)
\lineto(315.59789278,163.76647909)
\lineto(315.59789278,163.85647909)
\curveto(315.60788674,163.88647185)(315.61288673,163.91647182)(315.61289278,163.94647909)
\curveto(315.63288671,164.01647172)(315.63788671,164.11147162)(315.62789278,164.23147909)
\curveto(315.62788672,164.36147137)(315.61788673,164.46147127)(315.59789278,164.53147909)
\curveto(315.57788677,164.61147112)(315.55788679,164.68647105)(315.53789278,164.75647909)
\curveto(315.52788682,164.8364709)(315.50788684,164.91647082)(315.47789278,164.99647909)
\curveto(315.36788698,165.2364705)(315.21788713,165.4364703)(315.02789278,165.59647909)
\curveto(314.8478875,165.76646997)(314.62788772,165.90646983)(314.36789278,166.01647909)
\curveto(314.29788805,166.0364697)(314.22788812,166.05146968)(314.15789278,166.06147909)
\curveto(314.08788826,166.08146965)(314.01288833,166.10146963)(313.93289278,166.12147909)
\curveto(313.85288849,166.14146959)(313.7428886,166.15146958)(313.60289278,166.15147909)
\curveto(313.47288887,166.15146958)(313.36788898,166.14146959)(313.28789278,166.12147909)
\curveto(313.22788912,166.11146962)(313.17288917,166.10646963)(313.12289278,166.10647909)
\curveto(313.07288927,166.10646963)(313.02288932,166.09646964)(312.97289278,166.07647909)
\curveto(312.87288947,166.0364697)(312.77788957,165.99646974)(312.68789278,165.95647909)
\curveto(312.60788974,165.91646982)(312.52788982,165.87146986)(312.44789278,165.82147909)
\curveto(312.41788993,165.80146993)(312.38788996,165.77646996)(312.35789278,165.74647909)
\curveto(312.33789001,165.71647002)(312.31289003,165.69147004)(312.28289278,165.67147909)
\lineto(312.20789278,165.59647909)
\curveto(312.17789017,165.57647016)(312.15289019,165.55647018)(312.13289278,165.53647909)
\lineto(311.98289278,165.32647909)
\curveto(311.9428904,165.26647047)(311.89789045,165.20147053)(311.84789278,165.13147909)
\curveto(311.78789056,165.04147069)(311.73789061,164.9364708)(311.69789278,164.81647909)
\curveto(311.66789068,164.70647103)(311.63289071,164.59647114)(311.59289278,164.48647909)
\curveto(311.55289079,164.37647136)(311.52789082,164.2314715)(311.51789278,164.05147909)
\curveto(311.50789084,163.88147185)(311.47789087,163.75647198)(311.42789278,163.67647909)
\curveto(311.37789097,163.59647214)(311.30289104,163.55147218)(311.20289278,163.54147909)
\curveto(311.10289124,163.5314722)(310.99289135,163.52647221)(310.87289278,163.52647909)
\curveto(310.83289151,163.52647221)(310.79289155,163.52147221)(310.75289278,163.51147909)
\curveto(310.71289163,163.51147222)(310.67789167,163.51647222)(310.64789278,163.52647909)
\curveto(310.59789175,163.54647219)(310.5478918,163.55647218)(310.49789278,163.55647909)
\curveto(310.45789189,163.55647218)(310.41789193,163.56647217)(310.37789278,163.58647909)
\curveto(310.28789206,163.64647209)(310.2428921,163.78147195)(310.24289278,163.99147909)
\lineto(310.24289278,164.11147909)
\curveto(310.25289209,164.17147156)(310.25789209,164.2314715)(310.25789278,164.29147909)
\curveto(310.26789208,164.36147137)(310.27789207,164.42647131)(310.28789278,164.48647909)
\curveto(310.30789204,164.59647114)(310.32789202,164.69647104)(310.34789278,164.78647909)
\curveto(310.36789198,164.88647085)(310.39789195,164.98147075)(310.43789278,165.07147909)
\curveto(310.45789189,165.14147059)(310.47789187,165.20147053)(310.49789278,165.25147909)
\lineto(310.55789278,165.43147909)
\curveto(310.67789167,165.69147004)(310.83289151,165.9364698)(311.02289278,166.16647909)
\curveto(311.22289112,166.39646934)(311.43789091,166.58146915)(311.66789278,166.72147909)
\curveto(311.77789057,166.80146893)(311.89289045,166.86646887)(312.01289278,166.91647909)
\lineto(312.40289278,167.06647909)
\curveto(312.51288983,167.11646862)(312.62788972,167.14646859)(312.74789278,167.15647909)
\curveto(312.86788948,167.17646856)(312.99288935,167.20146853)(313.12289278,167.23147909)
\curveto(313.19288915,167.2314685)(313.25788909,167.2314685)(313.31789278,167.23147909)
\curveto(313.37788897,167.24146849)(313.4428889,167.25146848)(313.51289278,167.26147909)
}
}
{
\newrgbcolor{curcolor}{0 0 0}
\pscustom[linestyle=none,fillstyle=solid,fillcolor=curcolor]
{
\newpath
\moveto(319.07250215,167.06647909)
\lineto(323.87250215,167.06647909)
\lineto(324.87750215,167.06647909)
\curveto(325.01749505,167.06646867)(325.13749493,167.05646868)(325.23750215,167.03647909)
\curveto(325.34749472,167.02646871)(325.42749464,166.98146875)(325.47750215,166.90147909)
\curveto(325.49749457,166.86146887)(325.50749456,166.81146892)(325.50750215,166.75147909)
\curveto(325.51749455,166.69146904)(325.52249455,166.62646911)(325.52250215,166.55647909)
\lineto(325.52250215,166.28647909)
\curveto(325.52249455,166.19646954)(325.51249456,166.11646962)(325.49250215,166.04647909)
\curveto(325.45249462,165.96646977)(325.40749466,165.89646984)(325.35750215,165.83647909)
\lineto(325.20750215,165.65647909)
\curveto(325.17749489,165.60647013)(325.14249493,165.56647017)(325.10250215,165.53647909)
\curveto(325.06249501,165.50647023)(325.02249505,165.46647027)(324.98250215,165.41647909)
\curveto(324.90249517,165.30647043)(324.81749525,165.19647054)(324.72750215,165.08647909)
\curveto(324.63749543,164.98647075)(324.55249552,164.88147085)(324.47250215,164.77147909)
\curveto(324.33249574,164.57147116)(324.19249588,164.36147137)(324.05250215,164.14147909)
\curveto(323.91249616,163.9314718)(323.7724963,163.71647202)(323.63250215,163.49647909)
\curveto(323.58249649,163.40647233)(323.53249654,163.31147242)(323.48250215,163.21147909)
\curveto(323.43249664,163.11147262)(323.37749669,163.01647272)(323.31750215,162.92647909)
\curveto(323.29749677,162.90647283)(323.28749678,162.88147285)(323.28750215,162.85147909)
\curveto(323.28749678,162.82147291)(323.27749679,162.79647294)(323.25750215,162.77647909)
\curveto(323.18749688,162.67647306)(323.12249695,162.56147317)(323.06250215,162.43147909)
\curveto(323.00249707,162.31147342)(322.94749712,162.19647354)(322.89750215,162.08647909)
\curveto(322.79749727,161.85647388)(322.70249737,161.62147411)(322.61250215,161.38147909)
\curveto(322.52249755,161.14147459)(322.42249765,160.90147483)(322.31250215,160.66147909)
\curveto(322.29249778,160.61147512)(322.27749779,160.56647517)(322.26750215,160.52647909)
\curveto(322.2674978,160.48647525)(322.25749781,160.44147529)(322.23750215,160.39147909)
\curveto(322.18749788,160.27147546)(322.14249793,160.14647559)(322.10250215,160.01647909)
\curveto(322.072498,159.89647584)(322.03749803,159.77647596)(321.99750215,159.65647909)
\curveto(321.91749815,159.42647631)(321.85249822,159.18647655)(321.80250215,158.93647909)
\curveto(321.76249831,158.69647704)(321.71249836,158.45647728)(321.65250215,158.21647909)
\curveto(321.61249846,158.06647767)(321.58749848,157.91647782)(321.57750215,157.76647909)
\curveto(321.5674985,157.61647812)(321.54749852,157.46647827)(321.51750215,157.31647909)
\curveto(321.50749856,157.27647846)(321.50249857,157.21647852)(321.50250215,157.13647909)
\curveto(321.4724986,157.01647872)(321.44249863,156.91647882)(321.41250215,156.83647909)
\curveto(321.38249869,156.75647898)(321.31249876,156.70147903)(321.20250215,156.67147909)
\curveto(321.15249892,156.65147908)(321.09749897,156.64147909)(321.03750215,156.64147909)
\lineto(320.84250215,156.64147909)
\curveto(320.70249937,156.64147909)(320.56249951,156.64647909)(320.42250215,156.65647909)
\curveto(320.29249978,156.66647907)(320.19749987,156.71147902)(320.13750215,156.79147909)
\curveto(320.09749997,156.85147888)(320.07749999,156.9364788)(320.07750215,157.04647909)
\curveto(320.08749998,157.15647858)(320.10249997,157.25147848)(320.12250215,157.33147909)
\lineto(320.12250215,157.40647909)
\curveto(320.13249994,157.4364783)(320.13749993,157.46647827)(320.13750215,157.49647909)
\curveto(320.15749991,157.57647816)(320.1674999,157.65147808)(320.16750215,157.72147909)
\curveto(320.1674999,157.79147794)(320.17749989,157.86147787)(320.19750215,157.93147909)
\curveto(320.24749982,158.12147761)(320.28749978,158.30647743)(320.31750215,158.48647909)
\curveto(320.34749972,158.67647706)(320.38749968,158.85647688)(320.43750215,159.02647909)
\curveto(320.45749961,159.07647666)(320.4674996,159.11647662)(320.46750215,159.14647909)
\curveto(320.4674996,159.17647656)(320.4724996,159.21147652)(320.48250215,159.25147909)
\curveto(320.58249949,159.55147618)(320.6724994,159.84647589)(320.75250215,160.13647909)
\curveto(320.84249923,160.42647531)(320.94749912,160.70647503)(321.06750215,160.97647909)
\curveto(321.32749874,161.55647418)(321.59749847,162.10647363)(321.87750215,162.62647909)
\curveto(322.15749791,163.15647258)(322.4674976,163.66147207)(322.80750215,164.14147909)
\curveto(322.94749712,164.34147139)(323.09749697,164.5314712)(323.25750215,164.71147909)
\curveto(323.41749665,164.90147083)(323.5674965,165.09147064)(323.70750215,165.28147909)
\curveto(323.74749632,165.3314704)(323.78249629,165.37647036)(323.81250215,165.41647909)
\curveto(323.85249622,165.46647027)(323.88749618,165.51647022)(323.91750215,165.56647909)
\curveto(323.92749614,165.58647015)(323.93749613,165.61147012)(323.94750215,165.64147909)
\curveto(323.9674961,165.67147006)(323.9674961,165.70147003)(323.94750215,165.73147909)
\curveto(323.92749614,165.79146994)(323.89249618,165.82646991)(323.84250215,165.83647909)
\curveto(323.79249628,165.85646988)(323.74249633,165.87646986)(323.69250215,165.89647909)
\lineto(323.58750215,165.89647909)
\curveto(323.54749652,165.90646983)(323.49749657,165.90646983)(323.43750215,165.89647909)
\lineto(323.28750215,165.89647909)
\lineto(322.68750215,165.89647909)
\lineto(320.04750215,165.89647909)
\lineto(319.31250215,165.89647909)
\lineto(319.07250215,165.89647909)
\curveto(319.00250107,165.90646983)(318.94250113,165.92146981)(318.89250215,165.94147909)
\curveto(318.80250127,165.98146975)(318.74250133,166.04146969)(318.71250215,166.12147909)
\curveto(318.66250141,166.22146951)(318.64750142,166.36646937)(318.66750215,166.55647909)
\curveto(318.68750138,166.75646898)(318.72250135,166.89146884)(318.77250215,166.96147909)
\curveto(318.79250128,166.98146875)(318.81750125,166.99646874)(318.84750215,167.00647909)
\lineto(318.96750215,167.06647909)
\curveto(318.98750108,167.06646867)(319.00250107,167.06146867)(319.01250215,167.05147909)
\curveto(319.03250104,167.05146868)(319.05250102,167.05646868)(319.07250215,167.06647909)
}
}
{
\newrgbcolor{curcolor}{0 0 0}
\pscustom[linestyle=none,fillstyle=solid,fillcolor=curcolor]
{
\newpath
\moveto(327.91711153,158.29147909)
\lineto(328.21711153,158.29147909)
\curveto(328.32710947,158.30147743)(328.43210936,158.30147743)(328.53211153,158.29147909)
\curveto(328.64210915,158.29147744)(328.74210905,158.28147745)(328.83211153,158.26147909)
\curveto(328.92210887,158.25147748)(328.9921088,158.22647751)(329.04211153,158.18647909)
\curveto(329.06210873,158.16647757)(329.07710872,158.1364776)(329.08711153,158.09647909)
\curveto(329.10710869,158.05647768)(329.12710867,158.01147772)(329.14711153,157.96147909)
\lineto(329.14711153,157.88647909)
\curveto(329.15710864,157.8364779)(329.15710864,157.78147795)(329.14711153,157.72147909)
\lineto(329.14711153,157.57147909)
\lineto(329.14711153,157.09147909)
\curveto(329.14710865,156.92147881)(329.10710869,156.80147893)(329.02711153,156.73147909)
\curveto(328.95710884,156.68147905)(328.86710893,156.65647908)(328.75711153,156.65647909)
\lineto(328.42711153,156.65647909)
\lineto(327.97711153,156.65647909)
\curveto(327.82710997,156.65647908)(327.71211008,156.68647905)(327.63211153,156.74647909)
\curveto(327.5921102,156.77647896)(327.56211023,156.82647891)(327.54211153,156.89647909)
\curveto(327.52211027,156.97647876)(327.50711029,157.06147867)(327.49711153,157.15147909)
\lineto(327.49711153,157.43647909)
\curveto(327.50711029,157.5364782)(327.51211028,157.62147811)(327.51211153,157.69147909)
\lineto(327.51211153,157.88647909)
\curveto(327.51211028,157.94647779)(327.52211027,158.00147773)(327.54211153,158.05147909)
\curveto(327.58211021,158.16147757)(327.65211014,158.2314775)(327.75211153,158.26147909)
\curveto(327.78211001,158.26147747)(327.83710996,158.27147746)(327.91711153,158.29147909)
}
}
{
\newrgbcolor{curcolor}{0 0 0}
\pscustom[linestyle=none,fillstyle=solid,fillcolor=curcolor]
{
\newpath
\moveto(338.00226778,160.15147909)
\curveto(338.07226013,160.10147563)(338.11226009,160.0314757)(338.12226778,159.94147909)
\curveto(338.14226006,159.85147588)(338.15226005,159.74647599)(338.15226778,159.62647909)
\curveto(338.15226005,159.57647616)(338.14726006,159.52647621)(338.13726778,159.47647909)
\curveto(338.13726007,159.42647631)(338.12726008,159.38147635)(338.10726778,159.34147909)
\curveto(338.07726013,159.25147648)(338.01726019,159.19147654)(337.92726778,159.16147909)
\curveto(337.84726036,159.14147659)(337.75226045,159.1314766)(337.64226778,159.13147909)
\lineto(337.32726778,159.13147909)
\curveto(337.21726099,159.14147659)(337.11226109,159.1314766)(337.01226778,159.10147909)
\curveto(336.87226133,159.07147666)(336.78226142,158.99147674)(336.74226778,158.86147909)
\curveto(336.72226148,158.79147694)(336.71226149,158.70647703)(336.71226778,158.60647909)
\lineto(336.71226778,158.33647909)
\lineto(336.71226778,157.39147909)
\lineto(336.71226778,157.06147909)
\curveto(336.71226149,156.95147878)(336.69226151,156.86647887)(336.65226778,156.80647909)
\curveto(336.61226159,156.74647899)(336.56226164,156.70647903)(336.50226778,156.68647909)
\curveto(336.45226175,156.67647906)(336.38726182,156.66147907)(336.30726778,156.64147909)
\lineto(336.11226778,156.64147909)
\curveto(335.99226221,156.64147909)(335.88726232,156.64647909)(335.79726778,156.65647909)
\curveto(335.7072625,156.67647906)(335.63726257,156.72647901)(335.58726778,156.80647909)
\curveto(335.55726265,156.85647888)(335.54226266,156.92647881)(335.54226778,157.01647909)
\lineto(335.54226778,157.31647909)
\lineto(335.54226778,158.35147909)
\curveto(335.54226266,158.51147722)(335.53226267,158.65647708)(335.51226778,158.78647909)
\curveto(335.5022627,158.92647681)(335.44726276,159.02147671)(335.34726778,159.07147909)
\curveto(335.29726291,159.09147664)(335.22726298,159.10647663)(335.13726778,159.11647909)
\curveto(335.05726315,159.12647661)(334.96726324,159.1314766)(334.86726778,159.13147909)
\lineto(334.58226778,159.13147909)
\lineto(334.34226778,159.13147909)
\lineto(332.07726778,159.13147909)
\curveto(331.98726622,159.1314766)(331.88226632,159.12647661)(331.76226778,159.11647909)
\lineto(331.43226778,159.11647909)
\curveto(331.32226688,159.11647662)(331.22226698,159.12647661)(331.13226778,159.14647909)
\curveto(331.04226716,159.16647657)(330.98226722,159.20147653)(330.95226778,159.25147909)
\curveto(330.9022673,159.32147641)(330.87726733,159.41647632)(330.87726778,159.53647909)
\lineto(330.87726778,159.88147909)
\lineto(330.87726778,160.15147909)
\curveto(330.91726729,160.32147541)(330.97226723,160.46147527)(331.04226778,160.57147909)
\curveto(331.11226709,160.68147505)(331.19226701,160.79647494)(331.28226778,160.91647909)
\lineto(331.64226778,161.45647909)
\curveto(332.08226612,162.08647365)(332.51726569,162.70647303)(332.94726778,163.31647909)
\lineto(334.26726778,165.17647909)
\curveto(334.42726378,165.40647033)(334.58226362,165.62647011)(334.73226778,165.83647909)
\curveto(334.88226332,166.05646968)(335.03726317,166.28146945)(335.19726778,166.51147909)
\curveto(335.24726296,166.58146915)(335.29726291,166.64646909)(335.34726778,166.70647909)
\curveto(335.39726281,166.77646896)(335.44726276,166.85146888)(335.49726778,166.93147909)
\lineto(335.55726778,167.02147909)
\curveto(335.58726262,167.06146867)(335.61726259,167.09146864)(335.64726778,167.11147909)
\curveto(335.68726252,167.14146859)(335.72726248,167.16146857)(335.76726778,167.17147909)
\curveto(335.8072624,167.19146854)(335.85226235,167.21146852)(335.90226778,167.23147909)
\curveto(335.92226228,167.2314685)(335.94226226,167.22646851)(335.96226778,167.21647909)
\curveto(335.99226221,167.21646852)(336.01726219,167.22646851)(336.03726778,167.24647909)
\curveto(336.16726204,167.24646849)(336.28726192,167.24146849)(336.39726778,167.23147909)
\curveto(336.5072617,167.22146851)(336.58726162,167.17646856)(336.63726778,167.09647909)
\curveto(336.67726153,167.04646869)(336.69726151,166.97646876)(336.69726778,166.88647909)
\curveto(336.7072615,166.79646894)(336.71226149,166.70146903)(336.71226778,166.60147909)
\lineto(336.71226778,161.14147909)
\curveto(336.71226149,161.07147466)(336.7072615,160.99647474)(336.69726778,160.91647909)
\curveto(336.69726151,160.84647489)(336.7022615,160.77647496)(336.71226778,160.70647909)
\lineto(336.71226778,160.60147909)
\curveto(336.73226147,160.55147518)(336.74726146,160.49647524)(336.75726778,160.43647909)
\curveto(336.76726144,160.38647535)(336.79226141,160.34647539)(336.83226778,160.31647909)
\curveto(336.9022613,160.26647547)(336.98726122,160.2364755)(337.08726778,160.22647909)
\lineto(337.41726778,160.22647909)
\curveto(337.52726068,160.22647551)(337.63226057,160.22147551)(337.73226778,160.21147909)
\curveto(337.84226036,160.21147552)(337.93226027,160.19147554)(338.00226778,160.15147909)
\moveto(335.43726778,160.34647909)
\curveto(335.51726269,160.45647528)(335.55226265,160.62647511)(335.54226778,160.85647909)
\lineto(335.54226778,161.47147909)
\lineto(335.54226778,163.94647909)
\lineto(335.54226778,164.26147909)
\curveto(335.55226265,164.38147135)(335.54726266,164.48147125)(335.52726778,164.56147909)
\lineto(335.52726778,164.71147909)
\curveto(335.52726268,164.80147093)(335.51226269,164.88647085)(335.48226778,164.96647909)
\curveto(335.47226273,164.98647075)(335.46226274,164.99647074)(335.45226778,164.99647909)
\lineto(335.40726778,165.04147909)
\curveto(335.38726282,165.05147068)(335.35726285,165.05647068)(335.31726778,165.05647909)
\curveto(335.29726291,165.0364707)(335.27726293,165.02147071)(335.25726778,165.01147909)
\curveto(335.24726296,165.01147072)(335.23226297,165.00647073)(335.21226778,164.99647909)
\curveto(335.15226305,164.94647079)(335.09226311,164.87647086)(335.03226778,164.78647909)
\curveto(334.97226323,164.69647104)(334.91726329,164.61647112)(334.86726778,164.54647909)
\curveto(334.76726344,164.40647133)(334.67226353,164.26147147)(334.58226778,164.11147909)
\curveto(334.49226371,163.97147176)(334.39726381,163.8314719)(334.29726778,163.69147909)
\lineto(333.75726778,162.91147909)
\curveto(333.58726462,162.65147308)(333.41226479,162.39147334)(333.23226778,162.13147909)
\curveto(333.15226505,162.02147371)(333.07726513,161.91647382)(333.00726778,161.81647909)
\lineto(332.79726778,161.51647909)
\curveto(332.74726546,161.4364743)(332.69726551,161.36147437)(332.64726778,161.29147909)
\curveto(332.6072656,161.22147451)(332.56226564,161.14647459)(332.51226778,161.06647909)
\curveto(332.46226574,161.00647473)(332.41226579,160.94147479)(332.36226778,160.87147909)
\curveto(332.32226588,160.81147492)(332.28226592,160.74147499)(332.24226778,160.66147909)
\curveto(332.202266,160.60147513)(332.17726603,160.5314752)(332.16726778,160.45147909)
\curveto(332.15726605,160.38147535)(332.19226601,160.32647541)(332.27226778,160.28647909)
\curveto(332.34226586,160.2364755)(332.45226575,160.21147552)(332.60226778,160.21147909)
\curveto(332.76226544,160.22147551)(332.89726531,160.22647551)(333.00726778,160.22647909)
\lineto(334.68726778,160.22647909)
\lineto(335.12226778,160.22647909)
\curveto(335.27226293,160.22647551)(335.37726283,160.26647547)(335.43726778,160.34647909)
}
}
{
\newrgbcolor{curcolor}{0 0 0}
\pscustom[linestyle=none,fillstyle=solid,fillcolor=curcolor]
{
\newpath
\moveto(349.27687715,165.17647909)
\curveto(349.07686685,164.88647085)(348.86686706,164.60147113)(348.64687715,164.32147909)
\curveto(348.43686749,164.04147169)(348.2318677,163.75647198)(348.03187715,163.46647909)
\curveto(347.4318685,162.61647312)(346.8268691,161.77647396)(346.21687715,160.94647909)
\curveto(345.60687032,160.12647561)(345.00187093,159.29147644)(344.40187715,158.44147909)
\lineto(343.89187715,157.72147909)
\lineto(343.38187715,157.03147909)
\curveto(343.30187263,156.92147881)(343.22187271,156.80647893)(343.14187715,156.68647909)
\curveto(343.06187287,156.56647917)(342.96687296,156.47147926)(342.85687715,156.40147909)
\curveto(342.81687311,156.38147935)(342.75187318,156.36647937)(342.66187715,156.35647909)
\curveto(342.58187335,156.3364794)(342.49187344,156.32647941)(342.39187715,156.32647909)
\curveto(342.29187364,156.32647941)(342.19687373,156.3314794)(342.10687715,156.34147909)
\curveto(342.0268739,156.35147938)(341.96687396,156.37147936)(341.92687715,156.40147909)
\curveto(341.89687403,156.42147931)(341.87187406,156.45647928)(341.85187715,156.50647909)
\curveto(341.84187409,156.54647919)(341.84687408,156.59147914)(341.86687715,156.64147909)
\curveto(341.90687402,156.72147901)(341.95187398,156.79647894)(342.00187715,156.86647909)
\curveto(342.06187387,156.94647879)(342.11687381,157.02647871)(342.16687715,157.10647909)
\curveto(342.40687352,157.44647829)(342.65187328,157.78147795)(342.90187715,158.11147909)
\curveto(343.15187278,158.44147729)(343.39187254,158.77647696)(343.62187715,159.11647909)
\curveto(343.78187215,159.3364764)(343.94187199,159.55147618)(344.10187715,159.76147909)
\curveto(344.26187167,159.97147576)(344.42187151,160.18647555)(344.58187715,160.40647909)
\curveto(344.94187099,160.92647481)(345.30687062,161.4364743)(345.67687715,161.93647909)
\curveto(346.04686988,162.4364733)(346.41686951,162.94647279)(346.78687715,163.46647909)
\curveto(346.926869,163.66647207)(347.06686886,163.86147187)(347.20687715,164.05147909)
\curveto(347.35686857,164.24147149)(347.50186843,164.4364713)(347.64187715,164.63647909)
\curveto(347.85186808,164.9364708)(348.06686786,165.2364705)(348.28687715,165.53647909)
\lineto(348.94687715,166.43647909)
\lineto(349.12687715,166.70647909)
\lineto(349.33687715,166.97647909)
\lineto(349.45687715,167.15647909)
\curveto(349.50686642,167.21646852)(349.55686637,167.27146846)(349.60687715,167.32147909)
\curveto(349.67686625,167.37146836)(349.75186618,167.40646833)(349.83187715,167.42647909)
\curveto(349.85186608,167.4364683)(349.87686605,167.4364683)(349.90687715,167.42647909)
\curveto(349.94686598,167.42646831)(349.97686595,167.4364683)(349.99687715,167.45647909)
\curveto(350.11686581,167.45646828)(350.25186568,167.45146828)(350.40187715,167.44147909)
\curveto(350.55186538,167.44146829)(350.64186529,167.39646834)(350.67187715,167.30647909)
\curveto(350.69186524,167.27646846)(350.69686523,167.24146849)(350.68687715,167.20147909)
\curveto(350.67686525,167.16146857)(350.66186527,167.1314686)(350.64187715,167.11147909)
\curveto(350.60186533,167.0314687)(350.56186537,166.96146877)(350.52187715,166.90147909)
\curveto(350.48186545,166.84146889)(350.43686549,166.78146895)(350.38687715,166.72147909)
\lineto(349.81687715,165.94147909)
\curveto(349.63686629,165.69147004)(349.45686647,165.4364703)(349.27687715,165.17647909)
\moveto(342.42187715,161.27647909)
\curveto(342.37187356,161.29647444)(342.32187361,161.30147443)(342.27187715,161.29147909)
\curveto(342.22187371,161.28147445)(342.17187376,161.28647445)(342.12187715,161.30647909)
\curveto(342.01187392,161.32647441)(341.90687402,161.34647439)(341.80687715,161.36647909)
\curveto(341.71687421,161.39647434)(341.62187431,161.4364743)(341.52187715,161.48647909)
\curveto(341.19187474,161.62647411)(340.93687499,161.82147391)(340.75687715,162.07147909)
\curveto(340.57687535,162.3314734)(340.4318755,162.64147309)(340.32187715,163.00147909)
\curveto(340.29187564,163.08147265)(340.27187566,163.16147257)(340.26187715,163.24147909)
\curveto(340.25187568,163.3314724)(340.23687569,163.41647232)(340.21687715,163.49647909)
\curveto(340.20687572,163.54647219)(340.20187573,163.61147212)(340.20187715,163.69147909)
\curveto(340.19187574,163.72147201)(340.18687574,163.75147198)(340.18687715,163.78147909)
\curveto(340.18687574,163.82147191)(340.18187575,163.85647188)(340.17187715,163.88647909)
\lineto(340.17187715,164.03647909)
\curveto(340.16187577,164.08647165)(340.15687577,164.14647159)(340.15687715,164.21647909)
\curveto(340.15687577,164.29647144)(340.16187577,164.36147137)(340.17187715,164.41147909)
\lineto(340.17187715,164.57647909)
\curveto(340.19187574,164.62647111)(340.19687573,164.67147106)(340.18687715,164.71147909)
\curveto(340.18687574,164.76147097)(340.19187574,164.80647093)(340.20187715,164.84647909)
\curveto(340.21187572,164.88647085)(340.21687571,164.92147081)(340.21687715,164.95147909)
\curveto(340.21687571,164.99147074)(340.22187571,165.0314707)(340.23187715,165.07147909)
\curveto(340.26187567,165.18147055)(340.28187565,165.29147044)(340.29187715,165.40147909)
\curveto(340.31187562,165.52147021)(340.34687558,165.6364701)(340.39687715,165.74647909)
\curveto(340.53687539,166.08646965)(340.69687523,166.36146937)(340.87687715,166.57147909)
\curveto(341.06687486,166.79146894)(341.33687459,166.97146876)(341.68687715,167.11147909)
\curveto(341.76687416,167.14146859)(341.85187408,167.16146857)(341.94187715,167.17147909)
\curveto(342.0318739,167.19146854)(342.1268738,167.21146852)(342.22687715,167.23147909)
\curveto(342.25687367,167.24146849)(342.31187362,167.24146849)(342.39187715,167.23147909)
\curveto(342.47187346,167.2314685)(342.52187341,167.24146849)(342.54187715,167.26147909)
\curveto(343.10187283,167.27146846)(343.55187238,167.16146857)(343.89187715,166.93147909)
\curveto(344.24187169,166.70146903)(344.50187143,166.39646934)(344.67187715,166.01647909)
\curveto(344.71187122,165.92646981)(344.74687118,165.8314699)(344.77687715,165.73147909)
\curveto(344.80687112,165.6314701)(344.8318711,165.5314702)(344.85187715,165.43147909)
\curveto(344.87187106,165.40147033)(344.87687105,165.37147036)(344.86687715,165.34147909)
\curveto(344.86687106,165.31147042)(344.87187106,165.28147045)(344.88187715,165.25147909)
\curveto(344.91187102,165.14147059)(344.931871,165.01647072)(344.94187715,164.87647909)
\curveto(344.95187098,164.74647099)(344.96187097,164.61147112)(344.97187715,164.47147909)
\lineto(344.97187715,164.30647909)
\curveto(344.98187095,164.24647149)(344.98187095,164.19147154)(344.97187715,164.14147909)
\curveto(344.96187097,164.09147164)(344.95687097,164.04147169)(344.95687715,163.99147909)
\lineto(344.95687715,163.85647909)
\curveto(344.94687098,163.81647192)(344.94187099,163.77647196)(344.94187715,163.73647909)
\curveto(344.95187098,163.69647204)(344.94687098,163.65147208)(344.92687715,163.60147909)
\curveto(344.90687102,163.49147224)(344.88687104,163.38647235)(344.86687715,163.28647909)
\curveto(344.85687107,163.18647255)(344.83687109,163.08647265)(344.80687715,162.98647909)
\curveto(344.67687125,162.62647311)(344.51187142,162.31147342)(344.31187715,162.04147909)
\curveto(344.11187182,161.77147396)(343.83687209,161.56647417)(343.48687715,161.42647909)
\curveto(343.40687252,161.39647434)(343.32187261,161.37147436)(343.23187715,161.35147909)
\lineto(342.96187715,161.29147909)
\curveto(342.91187302,161.28147445)(342.86687306,161.27647446)(342.82687715,161.27647909)
\curveto(342.78687314,161.28647445)(342.74687318,161.28647445)(342.70687715,161.27647909)
\curveto(342.60687332,161.25647448)(342.51187342,161.25647448)(342.42187715,161.27647909)
\moveto(341.58187715,162.67147909)
\curveto(341.62187431,162.60147313)(341.66187427,162.5364732)(341.70187715,162.47647909)
\curveto(341.74187419,162.42647331)(341.79187414,162.37647336)(341.85187715,162.32647909)
\lineto(342.00187715,162.20647909)
\curveto(342.06187387,162.17647356)(342.1268738,162.15147358)(342.19687715,162.13147909)
\curveto(342.23687369,162.11147362)(342.27187366,162.10147363)(342.30187715,162.10147909)
\curveto(342.34187359,162.11147362)(342.38187355,162.10647363)(342.42187715,162.08647909)
\curveto(342.45187348,162.08647365)(342.49187344,162.08147365)(342.54187715,162.07147909)
\curveto(342.59187334,162.07147366)(342.6318733,162.07647366)(342.66187715,162.08647909)
\lineto(342.88687715,162.13147909)
\curveto(343.13687279,162.21147352)(343.32187261,162.3364734)(343.44187715,162.50647909)
\curveto(343.52187241,162.60647313)(343.59187234,162.736473)(343.65187715,162.89647909)
\curveto(343.7318722,163.07647266)(343.79187214,163.30147243)(343.83187715,163.57147909)
\curveto(343.87187206,163.85147188)(343.88687204,164.1314716)(343.87687715,164.41147909)
\curveto(343.86687206,164.70147103)(343.83687209,164.97647076)(343.78687715,165.23647909)
\curveto(343.73687219,165.49647024)(343.66187227,165.70647003)(343.56187715,165.86647909)
\curveto(343.44187249,166.06646967)(343.29187264,166.21646952)(343.11187715,166.31647909)
\curveto(343.0318729,166.36646937)(342.94187299,166.39646934)(342.84187715,166.40647909)
\curveto(342.74187319,166.42646931)(342.63687329,166.4364693)(342.52687715,166.43647909)
\curveto(342.50687342,166.42646931)(342.48187345,166.42146931)(342.45187715,166.42147909)
\curveto(342.4318735,166.4314693)(342.41187352,166.4314693)(342.39187715,166.42147909)
\curveto(342.34187359,166.41146932)(342.29687363,166.40146933)(342.25687715,166.39147909)
\curveto(342.21687371,166.39146934)(342.17687375,166.38146935)(342.13687715,166.36147909)
\curveto(341.95687397,166.28146945)(341.80687412,166.16146957)(341.68687715,166.00147909)
\curveto(341.57687435,165.84146989)(341.48687444,165.66147007)(341.41687715,165.46147909)
\curveto(341.35687457,165.27147046)(341.31187462,165.04647069)(341.28187715,164.78647909)
\curveto(341.26187467,164.52647121)(341.25687467,164.26147147)(341.26687715,163.99147909)
\curveto(341.27687465,163.731472)(341.30687462,163.48147225)(341.35687715,163.24147909)
\curveto(341.41687451,163.01147272)(341.49187444,162.82147291)(341.58187715,162.67147909)
\moveto(352.38187715,159.68647909)
\curveto(352.39186354,159.6364761)(352.39686353,159.54647619)(352.39687715,159.41647909)
\curveto(352.39686353,159.28647645)(352.38686354,159.19647654)(352.36687715,159.14647909)
\curveto(352.34686358,159.09647664)(352.34186359,159.04147669)(352.35187715,158.98147909)
\curveto(352.36186357,158.9314768)(352.36186357,158.88147685)(352.35187715,158.83147909)
\curveto(352.31186362,158.69147704)(352.28186365,158.55647718)(352.26187715,158.42647909)
\curveto(352.25186368,158.29647744)(352.22186371,158.17647756)(352.17187715,158.06647909)
\curveto(352.0318639,157.71647802)(351.86686406,157.42147831)(351.67687715,157.18147909)
\curveto(351.48686444,156.95147878)(351.21686471,156.76647897)(350.86687715,156.62647909)
\curveto(350.78686514,156.59647914)(350.70186523,156.57647916)(350.61187715,156.56647909)
\curveto(350.52186541,156.54647919)(350.43686549,156.52647921)(350.35687715,156.50647909)
\curveto(350.30686562,156.49647924)(350.25686567,156.49147924)(350.20687715,156.49147909)
\curveto(350.15686577,156.49147924)(350.10686582,156.48647925)(350.05687715,156.47647909)
\curveto(350.0268659,156.46647927)(349.97686595,156.46647927)(349.90687715,156.47647909)
\curveto(349.83686609,156.47647926)(349.78686614,156.48147925)(349.75687715,156.49147909)
\curveto(349.69686623,156.51147922)(349.63686629,156.52147921)(349.57687715,156.52147909)
\curveto(349.5268664,156.51147922)(349.47686645,156.51647922)(349.42687715,156.53647909)
\curveto(349.33686659,156.55647918)(349.24686668,156.58147915)(349.15687715,156.61147909)
\curveto(349.07686685,156.6314791)(348.99686693,156.66147907)(348.91687715,156.70147909)
\curveto(348.59686733,156.84147889)(348.34686758,157.0364787)(348.16687715,157.28647909)
\curveto(347.98686794,157.54647819)(347.83686809,157.85147788)(347.71687715,158.20147909)
\curveto(347.69686823,158.28147745)(347.68186825,158.36647737)(347.67187715,158.45647909)
\curveto(347.66186827,158.54647719)(347.64686828,158.6314771)(347.62687715,158.71147909)
\curveto(347.61686831,158.74147699)(347.61186832,158.77147696)(347.61187715,158.80147909)
\lineto(347.61187715,158.90647909)
\curveto(347.59186834,158.98647675)(347.58186835,159.06647667)(347.58187715,159.14647909)
\lineto(347.58187715,159.28147909)
\curveto(347.56186837,159.38147635)(347.56186837,159.48147625)(347.58187715,159.58147909)
\lineto(347.58187715,159.76147909)
\curveto(347.59186834,159.81147592)(347.59686833,159.85647588)(347.59687715,159.89647909)
\curveto(347.59686833,159.94647579)(347.60186833,159.99147574)(347.61187715,160.03147909)
\curveto(347.62186831,160.07147566)(347.6268683,160.10647563)(347.62687715,160.13647909)
\curveto(347.6268683,160.17647556)(347.6318683,160.21647552)(347.64187715,160.25647909)
\lineto(347.70187715,160.58647909)
\curveto(347.72186821,160.70647503)(347.75186818,160.81647492)(347.79187715,160.91647909)
\curveto(347.931868,161.24647449)(348.09186784,161.52147421)(348.27187715,161.74147909)
\curveto(348.46186747,161.97147376)(348.72186721,162.15647358)(349.05187715,162.29647909)
\curveto(349.1318668,162.3364734)(349.21686671,162.36147337)(349.30687715,162.37147909)
\lineto(349.60687715,162.43147909)
\lineto(349.74187715,162.43147909)
\curveto(349.79186614,162.44147329)(349.84186609,162.44647329)(349.89187715,162.44647909)
\curveto(350.46186547,162.46647327)(350.92186501,162.36147337)(351.27187715,162.13147909)
\curveto(351.6318643,161.91147382)(351.89686403,161.61147412)(352.06687715,161.23147909)
\curveto(352.11686381,161.1314746)(352.15686377,161.0314747)(352.18687715,160.93147909)
\curveto(352.21686371,160.8314749)(352.24686368,160.72647501)(352.27687715,160.61647909)
\curveto(352.28686364,160.57647516)(352.29186364,160.54147519)(352.29187715,160.51147909)
\curveto(352.29186364,160.49147524)(352.29686363,160.46147527)(352.30687715,160.42147909)
\curveto(352.3268636,160.35147538)(352.33686359,160.27647546)(352.33687715,160.19647909)
\curveto(352.33686359,160.11647562)(352.34686358,160.0364757)(352.36687715,159.95647909)
\curveto(352.36686356,159.90647583)(352.36686356,159.86147587)(352.36687715,159.82147909)
\curveto(352.36686356,159.78147595)(352.37186356,159.736476)(352.38187715,159.68647909)
\moveto(351.27187715,159.25147909)
\curveto(351.28186465,159.30147643)(351.28686464,159.37647636)(351.28687715,159.47647909)
\curveto(351.29686463,159.57647616)(351.29186464,159.65147608)(351.27187715,159.70147909)
\curveto(351.25186468,159.76147597)(351.24686468,159.81647592)(351.25687715,159.86647909)
\curveto(351.27686465,159.92647581)(351.27686465,159.98647575)(351.25687715,160.04647909)
\curveto(351.24686468,160.07647566)(351.24186469,160.11147562)(351.24187715,160.15147909)
\curveto(351.24186469,160.19147554)(351.23686469,160.2314755)(351.22687715,160.27147909)
\curveto(351.20686472,160.35147538)(351.18686474,160.42647531)(351.16687715,160.49647909)
\curveto(351.15686477,160.57647516)(351.14186479,160.65647508)(351.12187715,160.73647909)
\curveto(351.09186484,160.79647494)(351.06686486,160.85647488)(351.04687715,160.91647909)
\curveto(351.0268649,160.97647476)(350.99686493,161.0364747)(350.95687715,161.09647909)
\curveto(350.85686507,161.26647447)(350.7268652,161.40147433)(350.56687715,161.50147909)
\curveto(350.48686544,161.55147418)(350.39186554,161.58647415)(350.28187715,161.60647909)
\curveto(350.17186576,161.62647411)(350.04686588,161.6364741)(349.90687715,161.63647909)
\curveto(349.88686604,161.62647411)(349.86186607,161.62147411)(349.83187715,161.62147909)
\curveto(349.80186613,161.6314741)(349.77186616,161.6314741)(349.74187715,161.62147909)
\lineto(349.59187715,161.56147909)
\curveto(349.54186639,161.55147418)(349.49686643,161.5364742)(349.45687715,161.51647909)
\curveto(349.26686666,161.40647433)(349.12186681,161.26147447)(349.02187715,161.08147909)
\curveto(348.931867,160.90147483)(348.85186708,160.69647504)(348.78187715,160.46647909)
\curveto(348.74186719,160.3364754)(348.72186721,160.20147553)(348.72187715,160.06147909)
\curveto(348.72186721,159.9314758)(348.71186722,159.78647595)(348.69187715,159.62647909)
\curveto(348.68186725,159.57647616)(348.67186726,159.51647622)(348.66187715,159.44647909)
\curveto(348.66186727,159.37647636)(348.67186726,159.31647642)(348.69187715,159.26647909)
\lineto(348.69187715,159.10147909)
\lineto(348.69187715,158.92147909)
\curveto(348.70186723,158.87147686)(348.71186722,158.81647692)(348.72187715,158.75647909)
\curveto(348.7318672,158.70647703)(348.73686719,158.65147708)(348.73687715,158.59147909)
\curveto(348.74686718,158.5314772)(348.76186717,158.47647726)(348.78187715,158.42647909)
\curveto(348.8318671,158.2364775)(348.89186704,158.06147767)(348.96187715,157.90147909)
\curveto(349.0318669,157.74147799)(349.13686679,157.61147812)(349.27687715,157.51147909)
\curveto(349.40686652,157.41147832)(349.54686638,157.34147839)(349.69687715,157.30147909)
\curveto(349.7268662,157.29147844)(349.75186618,157.28647845)(349.77187715,157.28647909)
\curveto(349.80186613,157.29647844)(349.8318661,157.29647844)(349.86187715,157.28647909)
\curveto(349.88186605,157.28647845)(349.91186602,157.28147845)(349.95187715,157.27147909)
\curveto(349.99186594,157.27147846)(350.0268659,157.27647846)(350.05687715,157.28647909)
\curveto(350.09686583,157.29647844)(350.13686579,157.30147843)(350.17687715,157.30147909)
\curveto(350.21686571,157.30147843)(350.25686567,157.31147842)(350.29687715,157.33147909)
\curveto(350.53686539,157.41147832)(350.7318652,157.54647819)(350.88187715,157.73647909)
\curveto(351.00186493,157.91647782)(351.09186484,158.12147761)(351.15187715,158.35147909)
\curveto(351.17186476,158.42147731)(351.18686474,158.49147724)(351.19687715,158.56147909)
\curveto(351.20686472,158.64147709)(351.22186471,158.72147701)(351.24187715,158.80147909)
\curveto(351.24186469,158.86147687)(351.24686468,158.90647683)(351.25687715,158.93647909)
\curveto(351.25686467,158.95647678)(351.25686467,158.98147675)(351.25687715,159.01147909)
\curveto(351.25686467,159.05147668)(351.26186467,159.08147665)(351.27187715,159.10147909)
\lineto(351.27187715,159.25147909)
}
}
{
\newrgbcolor{curcolor}{0 0 0}
\pscustom[linestyle=none,fillstyle=solid,fillcolor=curcolor]
{
\newpath
\moveto(261.0586136,64.20930136)
\lineto(264.6586136,64.20930136)
\lineto(265.3036136,64.20930136)
\curveto(265.38360707,64.20929093)(265.458607,64.20429094)(265.5286136,64.19430136)
\curveto(265.59860686,64.19429095)(265.6586068,64.18429096)(265.7086136,64.16430136)
\curveto(265.77860668,64.13429101)(265.83360662,64.07429107)(265.8736136,63.98430136)
\curveto(265.89360656,63.95429119)(265.90360655,63.91429123)(265.9036136,63.86430136)
\lineto(265.9036136,63.72930136)
\curveto(265.91360654,63.61929152)(265.90860655,63.51429163)(265.8886136,63.41430136)
\curveto(265.87860658,63.31429183)(265.84360661,63.2442919)(265.7836136,63.20430136)
\curveto(265.69360676,63.13429201)(265.5586069,63.09929204)(265.3786136,63.09930136)
\curveto(265.19860726,63.10929203)(265.03360742,63.11429203)(264.8836136,63.11430136)
\lineto(262.8886136,63.11430136)
\lineto(262.3936136,63.11430136)
\lineto(262.2586136,63.11430136)
\curveto(262.21861024,63.11429203)(262.17861028,63.10929203)(262.1386136,63.09930136)
\lineto(261.9286136,63.09930136)
\curveto(261.81861064,63.06929207)(261.73861072,63.02929211)(261.6886136,62.97930136)
\curveto(261.63861082,62.9392922)(261.60361085,62.88429226)(261.5836136,62.81430136)
\curveto(261.56361089,62.75429239)(261.54861091,62.68429246)(261.5386136,62.60430136)
\curveto(261.52861093,62.52429262)(261.50861095,62.43429271)(261.4786136,62.33430136)
\curveto(261.42861103,62.13429301)(261.38861107,61.92929321)(261.3586136,61.71930136)
\curveto(261.32861113,61.50929363)(261.28861117,61.30429384)(261.2386136,61.10430136)
\curveto(261.21861124,61.03429411)(261.20861125,60.96429418)(261.2086136,60.89430136)
\curveto(261.20861125,60.83429431)(261.19861126,60.76929437)(261.1786136,60.69930136)
\curveto(261.16861129,60.66929447)(261.1586113,60.62929451)(261.1486136,60.57930136)
\curveto(261.14861131,60.5392946)(261.1536113,60.49929464)(261.1636136,60.45930136)
\curveto(261.18361127,60.40929473)(261.20861125,60.36429478)(261.2386136,60.32430136)
\curveto(261.27861118,60.29429485)(261.33861112,60.28929485)(261.4186136,60.30930136)
\curveto(261.47861098,60.32929481)(261.53861092,60.35429479)(261.5986136,60.38430136)
\curveto(261.6586108,60.42429472)(261.71861074,60.45929468)(261.7786136,60.48930136)
\curveto(261.83861062,60.50929463)(261.88861057,60.52429462)(261.9286136,60.53430136)
\curveto(262.11861034,60.61429453)(262.32361013,60.66929447)(262.5436136,60.69930136)
\curveto(262.77360968,60.72929441)(263.00360945,60.7392944)(263.2336136,60.72930136)
\curveto(263.47360898,60.72929441)(263.70360875,60.70429444)(263.9236136,60.65430136)
\curveto(264.14360831,60.61429453)(264.34360811,60.55429459)(264.5236136,60.47430136)
\curveto(264.57360788,60.45429469)(264.61860784,60.43429471)(264.6586136,60.41430136)
\curveto(264.70860775,60.39429475)(264.7586077,60.36929477)(264.8086136,60.33930136)
\curveto(265.1586073,60.12929501)(265.43860702,59.89929524)(265.6486136,59.64930136)
\curveto(265.86860659,59.39929574)(266.06360639,59.07429607)(266.2336136,58.67430136)
\curveto(266.28360617,58.56429658)(266.31860614,58.45429669)(266.3386136,58.34430136)
\curveto(266.3586061,58.23429691)(266.38360607,58.11929702)(266.4136136,57.99930136)
\curveto(266.42360603,57.96929717)(266.42860603,57.92429722)(266.4286136,57.86430136)
\curveto(266.44860601,57.80429734)(266.458606,57.73429741)(266.4586136,57.65430136)
\curveto(266.458606,57.58429756)(266.46860599,57.51929762)(266.4886136,57.45930136)
\lineto(266.4886136,57.29430136)
\curveto(266.49860596,57.2442979)(266.50360595,57.17429797)(266.5036136,57.08430136)
\curveto(266.50360595,56.99429815)(266.49360596,56.92429822)(266.4736136,56.87430136)
\curveto(266.453606,56.81429833)(266.44860601,56.75429839)(266.4586136,56.69430136)
\curveto(266.46860599,56.6442985)(266.46360599,56.59429855)(266.4436136,56.54430136)
\curveto(266.40360605,56.38429876)(266.36860609,56.23429891)(266.3386136,56.09430136)
\curveto(266.30860615,55.95429919)(266.26360619,55.81929932)(266.2036136,55.68930136)
\curveto(266.04360641,55.31929982)(265.82360663,54.98430016)(265.5436136,54.68430136)
\curveto(265.26360719,54.38430076)(264.94360751,54.15430099)(264.5836136,53.99430136)
\curveto(264.41360804,53.91430123)(264.21360824,53.8393013)(263.9836136,53.76930136)
\curveto(263.87360858,53.72930141)(263.7586087,53.70430144)(263.6386136,53.69430136)
\curveto(263.51860894,53.68430146)(263.39860906,53.66430148)(263.2786136,53.63430136)
\curveto(263.22860923,53.61430153)(263.17360928,53.61430153)(263.1136136,53.63430136)
\curveto(263.0536094,53.6443015)(262.99360946,53.6393015)(262.9336136,53.61930136)
\curveto(262.83360962,53.59930154)(262.73360972,53.59930154)(262.6336136,53.61930136)
\lineto(262.4986136,53.61930136)
\curveto(262.44861001,53.6393015)(262.38861007,53.64930149)(262.3186136,53.64930136)
\curveto(262.2586102,53.6393015)(262.20361025,53.6443015)(262.1536136,53.66430136)
\curveto(262.11361034,53.67430147)(262.07861038,53.67930146)(262.0486136,53.67930136)
\curveto(262.01861044,53.67930146)(261.98361047,53.68430146)(261.9436136,53.69430136)
\lineto(261.6736136,53.75430136)
\curveto(261.58361087,53.77430137)(261.49861096,53.80430134)(261.4186136,53.84430136)
\curveto(261.07861138,53.98430116)(260.78861167,54.139301)(260.5486136,54.30930136)
\curveto(260.30861215,54.48930065)(260.08861237,54.71930042)(259.8886136,54.99930136)
\curveto(259.73861272,55.22929991)(259.62361283,55.46929967)(259.5436136,55.71930136)
\curveto(259.52361293,55.76929937)(259.51361294,55.81429933)(259.5136136,55.85430136)
\curveto(259.51361294,55.90429924)(259.50361295,55.95429919)(259.4836136,56.00430136)
\curveto(259.46361299,56.06429908)(259.44861301,56.144299)(259.4386136,56.24430136)
\curveto(259.43861302,56.3442988)(259.458613,56.41929872)(259.4986136,56.46930136)
\curveto(259.54861291,56.54929859)(259.62861283,56.59429855)(259.7386136,56.60430136)
\curveto(259.84861261,56.61429853)(259.96361249,56.61929852)(260.0836136,56.61930136)
\lineto(260.2486136,56.61930136)
\curveto(260.30861215,56.61929852)(260.36361209,56.60929853)(260.4136136,56.58930136)
\curveto(260.50361195,56.56929857)(260.57361188,56.52929861)(260.6236136,56.46930136)
\curveto(260.69361176,56.37929876)(260.73861172,56.26929887)(260.7586136,56.13930136)
\curveto(260.78861167,56.01929912)(260.83361162,55.91429923)(260.8936136,55.82430136)
\curveto(261.08361137,55.48429966)(261.34361111,55.21429993)(261.6736136,55.01430136)
\curveto(261.77361068,54.95430019)(261.87861058,54.90430024)(261.9886136,54.86430136)
\curveto(262.10861035,54.83430031)(262.22861023,54.79930034)(262.3486136,54.75930136)
\curveto(262.51860994,54.70930043)(262.72360973,54.68930045)(262.9636136,54.69930136)
\curveto(263.21360924,54.71930042)(263.41360904,54.75430039)(263.5636136,54.80430136)
\curveto(263.93360852,54.92430022)(264.22360823,55.08430006)(264.4336136,55.28430136)
\curveto(264.6536078,55.49429965)(264.83360762,55.77429937)(264.9736136,56.12430136)
\curveto(265.02360743,56.22429892)(265.0536074,56.32929881)(265.0636136,56.43930136)
\curveto(265.08360737,56.54929859)(265.10860735,56.66429848)(265.1386136,56.78430136)
\lineto(265.1386136,56.88930136)
\curveto(265.14860731,56.92929821)(265.1536073,56.96929817)(265.1536136,57.00930136)
\curveto(265.16360729,57.0392981)(265.16360729,57.07429807)(265.1536136,57.11430136)
\lineto(265.1536136,57.23430136)
\curveto(265.1536073,57.49429765)(265.12360733,57.7392974)(265.0636136,57.96930136)
\curveto(264.9536075,58.31929682)(264.79860766,58.61429653)(264.5986136,58.85430136)
\curveto(264.39860806,59.10429604)(264.13860832,59.29929584)(263.8186136,59.43930136)
\lineto(263.6386136,59.49930136)
\curveto(263.58860887,59.51929562)(263.52860893,59.5392956)(263.4586136,59.55930136)
\curveto(263.40860905,59.57929556)(263.34860911,59.58929555)(263.2786136,59.58930136)
\curveto(263.21860924,59.59929554)(263.1536093,59.61429553)(263.0836136,59.63430136)
\lineto(262.9336136,59.63430136)
\curveto(262.89360956,59.65429549)(262.83860962,59.66429548)(262.7686136,59.66430136)
\curveto(262.70860975,59.66429548)(262.6536098,59.65429549)(262.6036136,59.63430136)
\lineto(262.4986136,59.63430136)
\curveto(262.46860999,59.63429551)(262.43361002,59.62929551)(262.3936136,59.61930136)
\lineto(262.1536136,59.55930136)
\curveto(262.07361038,59.54929559)(261.99361046,59.52929561)(261.9136136,59.49930136)
\curveto(261.67361078,59.39929574)(261.44361101,59.26429588)(261.2236136,59.09430136)
\curveto(261.13361132,59.02429612)(261.04861141,58.94929619)(260.9686136,58.86930136)
\curveto(260.88861157,58.79929634)(260.78861167,58.7442964)(260.6686136,58.70430136)
\curveto(260.57861188,58.67429647)(260.43861202,58.66429648)(260.2486136,58.67430136)
\curveto(260.06861239,58.68429646)(259.94861251,58.70929643)(259.8886136,58.74930136)
\curveto(259.83861262,58.78929635)(259.79861266,58.84929629)(259.7686136,58.92930136)
\curveto(259.74861271,59.00929613)(259.74861271,59.09429605)(259.7686136,59.18430136)
\curveto(259.79861266,59.30429584)(259.81861264,59.42429572)(259.8286136,59.54430136)
\curveto(259.84861261,59.67429547)(259.87361258,59.79929534)(259.9036136,59.91930136)
\curveto(259.92361253,59.95929518)(259.92861253,59.99429515)(259.9186136,60.02430136)
\curveto(259.91861254,60.06429508)(259.92861253,60.10929503)(259.9486136,60.15930136)
\curveto(259.96861249,60.24929489)(259.98361247,60.3392948)(259.9936136,60.42930136)
\curveto(260.00361245,60.52929461)(260.02361243,60.62429452)(260.0536136,60.71430136)
\curveto(260.06361239,60.77429437)(260.06861239,60.83429431)(260.0686136,60.89430136)
\curveto(260.07861238,60.95429419)(260.09361236,61.01429413)(260.1136136,61.07430136)
\curveto(260.16361229,61.27429387)(260.19861226,61.47929366)(260.2186136,61.68930136)
\curveto(260.24861221,61.90929323)(260.28861217,62.11929302)(260.3386136,62.31930136)
\curveto(260.36861209,62.41929272)(260.38861207,62.51929262)(260.3986136,62.61930136)
\curveto(260.40861205,62.71929242)(260.42361203,62.81929232)(260.4436136,62.91930136)
\curveto(260.453612,62.94929219)(260.458612,62.98929215)(260.4586136,63.03930136)
\curveto(260.48861197,63.14929199)(260.50861195,63.25429189)(260.5186136,63.35430136)
\curveto(260.53861192,63.46429168)(260.56361189,63.57429157)(260.5936136,63.68430136)
\curveto(260.61361184,63.76429138)(260.62861183,63.83429131)(260.6386136,63.89430136)
\curveto(260.64861181,63.96429118)(260.67361178,64.02429112)(260.7136136,64.07430136)
\curveto(260.73361172,64.10429104)(260.76361169,64.12429102)(260.8036136,64.13430136)
\curveto(260.84361161,64.15429099)(260.88861157,64.17429097)(260.9386136,64.19430136)
\curveto(260.99861146,64.19429095)(261.03861142,64.19929094)(261.0586136,64.20930136)
}
}
{
\newrgbcolor{curcolor}{0 0 0}
\pscustom[linestyle=none,fillstyle=solid,fillcolor=curcolor]
{
\newpath
\moveto(268.85322298,55.43430136)
\lineto(269.15322298,55.43430136)
\curveto(269.26322092,55.4442997)(269.36822081,55.4442997)(269.46822298,55.43430136)
\curveto(269.5782206,55.43429971)(269.6782205,55.42429972)(269.76822298,55.40430136)
\curveto(269.85822032,55.39429975)(269.92822025,55.36929977)(269.97822298,55.32930136)
\curveto(269.99822018,55.30929983)(270.01322017,55.27929986)(270.02322298,55.23930136)
\curveto(270.04322014,55.19929994)(270.06322012,55.15429999)(270.08322298,55.10430136)
\lineto(270.08322298,55.02930136)
\curveto(270.09322009,54.97930016)(270.09322009,54.92430022)(270.08322298,54.86430136)
\lineto(270.08322298,54.71430136)
\lineto(270.08322298,54.23430136)
\curveto(270.0832201,54.06430108)(270.04322014,53.9443012)(269.96322298,53.87430136)
\curveto(269.89322029,53.82430132)(269.80322038,53.79930134)(269.69322298,53.79930136)
\lineto(269.36322298,53.79930136)
\lineto(268.91322298,53.79930136)
\curveto(268.76322142,53.79930134)(268.64822153,53.82930131)(268.56822298,53.88930136)
\curveto(268.52822165,53.91930122)(268.49822168,53.96930117)(268.47822298,54.03930136)
\curveto(268.45822172,54.11930102)(268.44322174,54.20430094)(268.43322298,54.29430136)
\lineto(268.43322298,54.57930136)
\curveto(268.44322174,54.67930046)(268.44822173,54.76430038)(268.44822298,54.83430136)
\lineto(268.44822298,55.02930136)
\curveto(268.44822173,55.08930005)(268.45822172,55.1443)(268.47822298,55.19430136)
\curveto(268.51822166,55.30429984)(268.58822159,55.37429977)(268.68822298,55.40430136)
\curveto(268.71822146,55.40429974)(268.77322141,55.41429973)(268.85322298,55.43430136)
}
}
{
\newrgbcolor{curcolor}{0 0 0}
\pscustom[linestyle=none,fillstyle=solid,fillcolor=curcolor]
{
\newpath
\moveto(272.51837923,64.20930136)
\lineto(277.31837923,64.20930136)
\lineto(278.32337923,64.20930136)
\curveto(278.46337213,64.20929093)(278.58337201,64.19929094)(278.68337923,64.17930136)
\curveto(278.7933718,64.16929097)(278.87337172,64.12429102)(278.92337923,64.04430136)
\curveto(278.94337165,64.00429114)(278.95337164,63.95429119)(278.95337923,63.89430136)
\curveto(278.96337163,63.83429131)(278.96837162,63.76929137)(278.96837923,63.69930136)
\lineto(278.96837923,63.42930136)
\curveto(278.96837162,63.3392918)(278.95837163,63.25929188)(278.93837923,63.18930136)
\curveto(278.89837169,63.10929203)(278.85337174,63.0392921)(278.80337923,62.97930136)
\lineto(278.65337923,62.79930136)
\curveto(278.62337197,62.74929239)(278.588372,62.70929243)(278.54837923,62.67930136)
\curveto(278.50837208,62.64929249)(278.46837212,62.60929253)(278.42837923,62.55930136)
\curveto(278.34837224,62.44929269)(278.26337233,62.3392928)(278.17337923,62.22930136)
\curveto(278.08337251,62.12929301)(277.99837259,62.02429312)(277.91837923,61.91430136)
\curveto(277.77837281,61.71429343)(277.63837295,61.50429364)(277.49837923,61.28430136)
\curveto(277.35837323,61.07429407)(277.21837337,60.85929428)(277.07837923,60.63930136)
\curveto(277.02837356,60.54929459)(276.97837361,60.45429469)(276.92837923,60.35430136)
\curveto(276.87837371,60.25429489)(276.82337377,60.15929498)(276.76337923,60.06930136)
\curveto(276.74337385,60.04929509)(276.73337386,60.02429512)(276.73337923,59.99430136)
\curveto(276.73337386,59.96429518)(276.72337387,59.9392952)(276.70337923,59.91930136)
\curveto(276.63337396,59.81929532)(276.56837402,59.70429544)(276.50837923,59.57430136)
\curveto(276.44837414,59.45429569)(276.3933742,59.3392958)(276.34337923,59.22930136)
\curveto(276.24337435,58.99929614)(276.14837444,58.76429638)(276.05837923,58.52430136)
\curveto(275.96837462,58.28429686)(275.86837472,58.0442971)(275.75837923,57.80430136)
\curveto(275.73837485,57.75429739)(275.72337487,57.70929743)(275.71337923,57.66930136)
\curveto(275.71337488,57.62929751)(275.70337489,57.58429756)(275.68337923,57.53430136)
\curveto(275.63337496,57.41429773)(275.588375,57.28929785)(275.54837923,57.15930136)
\curveto(275.51837507,57.0392981)(275.48337511,56.91929822)(275.44337923,56.79930136)
\curveto(275.36337523,56.56929857)(275.29837529,56.32929881)(275.24837923,56.07930136)
\curveto(275.20837538,55.8392993)(275.15837543,55.59929954)(275.09837923,55.35930136)
\curveto(275.05837553,55.20929993)(275.03337556,55.05930008)(275.02337923,54.90930136)
\curveto(275.01337558,54.75930038)(274.9933756,54.60930053)(274.96337923,54.45930136)
\curveto(274.95337564,54.41930072)(274.94837564,54.35930078)(274.94837923,54.27930136)
\curveto(274.91837567,54.15930098)(274.8883757,54.05930108)(274.85837923,53.97930136)
\curveto(274.82837576,53.89930124)(274.75837583,53.8443013)(274.64837923,53.81430136)
\curveto(274.59837599,53.79430135)(274.54337605,53.78430136)(274.48337923,53.78430136)
\lineto(274.28837923,53.78430136)
\curveto(274.14837644,53.78430136)(274.00837658,53.78930135)(273.86837923,53.79930136)
\curveto(273.73837685,53.80930133)(273.64337695,53.85430129)(273.58337923,53.93430136)
\curveto(273.54337705,53.99430115)(273.52337707,54.07930106)(273.52337923,54.18930136)
\curveto(273.53337706,54.29930084)(273.54837704,54.39430075)(273.56837923,54.47430136)
\lineto(273.56837923,54.54930136)
\curveto(273.57837701,54.57930056)(273.58337701,54.60930053)(273.58337923,54.63930136)
\curveto(273.60337699,54.71930042)(273.61337698,54.79430035)(273.61337923,54.86430136)
\curveto(273.61337698,54.93430021)(273.62337697,55.00430014)(273.64337923,55.07430136)
\curveto(273.6933769,55.26429988)(273.73337686,55.44929969)(273.76337923,55.62930136)
\curveto(273.7933768,55.81929932)(273.83337676,55.99929914)(273.88337923,56.16930136)
\curveto(273.90337669,56.21929892)(273.91337668,56.25929888)(273.91337923,56.28930136)
\curveto(273.91337668,56.31929882)(273.91837667,56.35429879)(273.92837923,56.39430136)
\curveto(274.02837656,56.69429845)(274.11837647,56.98929815)(274.19837923,57.27930136)
\curveto(274.2883763,57.56929757)(274.3933762,57.84929729)(274.51337923,58.11930136)
\curveto(274.77337582,58.69929644)(275.04337555,59.24929589)(275.32337923,59.76930136)
\curveto(275.60337499,60.29929484)(275.91337468,60.80429434)(276.25337923,61.28430136)
\curveto(276.3933742,61.48429366)(276.54337405,61.67429347)(276.70337923,61.85430136)
\curveto(276.86337373,62.0442931)(277.01337358,62.23429291)(277.15337923,62.42430136)
\curveto(277.1933734,62.47429267)(277.22837336,62.51929262)(277.25837923,62.55930136)
\curveto(277.29837329,62.60929253)(277.33337326,62.65929248)(277.36337923,62.70930136)
\curveto(277.37337322,62.72929241)(277.38337321,62.75429239)(277.39337923,62.78430136)
\curveto(277.41337318,62.81429233)(277.41337318,62.8442923)(277.39337923,62.87430136)
\curveto(277.37337322,62.93429221)(277.33837325,62.96929217)(277.28837923,62.97930136)
\curveto(277.23837335,62.99929214)(277.1883734,63.01929212)(277.13837923,63.03930136)
\lineto(277.03337923,63.03930136)
\curveto(276.9933736,63.04929209)(276.94337365,63.04929209)(276.88337923,63.03930136)
\lineto(276.73337923,63.03930136)
\lineto(276.13337923,63.03930136)
\lineto(273.49337923,63.03930136)
\lineto(272.75837923,63.03930136)
\lineto(272.51837923,63.03930136)
\curveto(272.44837814,63.04929209)(272.3883782,63.06429208)(272.33837923,63.08430136)
\curveto(272.24837834,63.12429202)(272.1883784,63.18429196)(272.15837923,63.26430136)
\curveto(272.10837848,63.36429178)(272.0933785,63.50929163)(272.11337923,63.69930136)
\curveto(272.13337846,63.89929124)(272.16837842,64.03429111)(272.21837923,64.10430136)
\curveto(272.23837835,64.12429102)(272.26337833,64.139291)(272.29337923,64.14930136)
\lineto(272.41337923,64.20930136)
\curveto(272.43337816,64.20929093)(272.44837814,64.20429094)(272.45837923,64.19430136)
\curveto(272.47837811,64.19429095)(272.49837809,64.19929094)(272.51837923,64.20930136)
}
}
{
\newrgbcolor{curcolor}{0 0 0}
\pscustom[linestyle=none,fillstyle=solid,fillcolor=curcolor]
{
\newpath
\moveto(290.2129886,62.31930136)
\curveto(290.0129783,62.02929311)(289.80297851,61.7442934)(289.5829886,61.46430136)
\curveto(289.37297894,61.18429396)(289.16797915,60.89929424)(288.9679886,60.60930136)
\curveto(288.36797995,59.75929538)(287.76298055,58.91929622)(287.1529886,58.08930136)
\curveto(286.54298177,57.26929787)(285.93798238,56.43429871)(285.3379886,55.58430136)
\lineto(284.8279886,54.86430136)
\lineto(284.3179886,54.17430136)
\curveto(284.23798408,54.06430108)(284.15798416,53.94930119)(284.0779886,53.82930136)
\curveto(283.99798432,53.70930143)(283.90298441,53.61430153)(283.7929886,53.54430136)
\curveto(283.75298456,53.52430162)(283.68798463,53.50930163)(283.5979886,53.49930136)
\curveto(283.5179848,53.47930166)(283.42798489,53.46930167)(283.3279886,53.46930136)
\curveto(283.22798509,53.46930167)(283.13298518,53.47430167)(283.0429886,53.48430136)
\curveto(282.96298535,53.49430165)(282.90298541,53.51430163)(282.8629886,53.54430136)
\curveto(282.83298548,53.56430158)(282.80798551,53.59930154)(282.7879886,53.64930136)
\curveto(282.77798554,53.68930145)(282.78298553,53.73430141)(282.8029886,53.78430136)
\curveto(282.84298547,53.86430128)(282.88798543,53.9393012)(282.9379886,54.00930136)
\curveto(282.99798532,54.08930105)(283.05298526,54.16930097)(283.1029886,54.24930136)
\curveto(283.34298497,54.58930055)(283.58798473,54.92430022)(283.8379886,55.25430136)
\curveto(284.08798423,55.58429956)(284.32798399,55.91929922)(284.5579886,56.25930136)
\curveto(284.7179836,56.47929866)(284.87798344,56.69429845)(285.0379886,56.90430136)
\curveto(285.19798312,57.11429803)(285.35798296,57.32929781)(285.5179886,57.54930136)
\curveto(285.87798244,58.06929707)(286.24298207,58.57929656)(286.6129886,59.07930136)
\curveto(286.98298133,59.57929556)(287.35298096,60.08929505)(287.7229886,60.60930136)
\curveto(287.86298045,60.80929433)(288.00298031,61.00429414)(288.1429886,61.19430136)
\curveto(288.29298002,61.38429376)(288.43797988,61.57929356)(288.5779886,61.77930136)
\curveto(288.78797953,62.07929306)(289.00297931,62.37929276)(289.2229886,62.67930136)
\lineto(289.8829886,63.57930136)
\lineto(290.0629886,63.84930136)
\lineto(290.2729886,64.11930136)
\lineto(290.3929886,64.29930136)
\curveto(290.44297787,64.35929078)(290.49297782,64.41429073)(290.5429886,64.46430136)
\curveto(290.6129777,64.51429063)(290.68797763,64.54929059)(290.7679886,64.56930136)
\curveto(290.78797753,64.57929056)(290.8129775,64.57929056)(290.8429886,64.56930136)
\curveto(290.88297743,64.56929057)(290.9129774,64.57929056)(290.9329886,64.59930136)
\curveto(291.05297726,64.59929054)(291.18797713,64.59429055)(291.3379886,64.58430136)
\curveto(291.48797683,64.58429056)(291.57797674,64.5392906)(291.6079886,64.44930136)
\curveto(291.62797669,64.41929072)(291.63297668,64.38429076)(291.6229886,64.34430136)
\curveto(291.6129767,64.30429084)(291.59797672,64.27429087)(291.5779886,64.25430136)
\curveto(291.53797678,64.17429097)(291.49797682,64.10429104)(291.4579886,64.04430136)
\curveto(291.4179769,63.98429116)(291.37297694,63.92429122)(291.3229886,63.86430136)
\lineto(290.7529886,63.08430136)
\curveto(290.57297774,62.83429231)(290.39297792,62.57929256)(290.2129886,62.31930136)
\moveto(283.3579886,58.41930136)
\curveto(283.30798501,58.4392967)(283.25798506,58.4442967)(283.2079886,58.43430136)
\curveto(283.15798516,58.42429672)(283.10798521,58.42929671)(283.0579886,58.44930136)
\curveto(282.94798537,58.46929667)(282.84298547,58.48929665)(282.7429886,58.50930136)
\curveto(282.65298566,58.5392966)(282.55798576,58.57929656)(282.4579886,58.62930136)
\curveto(282.12798619,58.76929637)(281.87298644,58.96429618)(281.6929886,59.21430136)
\curveto(281.5129868,59.47429567)(281.36798695,59.78429536)(281.2579886,60.14430136)
\curveto(281.22798709,60.22429492)(281.20798711,60.30429484)(281.1979886,60.38430136)
\curveto(281.18798713,60.47429467)(281.17298714,60.55929458)(281.1529886,60.63930136)
\curveto(281.14298717,60.68929445)(281.13798718,60.75429439)(281.1379886,60.83430136)
\curveto(281.12798719,60.86429428)(281.12298719,60.89429425)(281.1229886,60.92430136)
\curveto(281.12298719,60.96429418)(281.1179872,60.99929414)(281.1079886,61.02930136)
\lineto(281.1079886,61.17930136)
\curveto(281.09798722,61.22929391)(281.09298722,61.28929385)(281.0929886,61.35930136)
\curveto(281.09298722,61.4392937)(281.09798722,61.50429364)(281.1079886,61.55430136)
\lineto(281.1079886,61.71930136)
\curveto(281.12798719,61.76929337)(281.13298718,61.81429333)(281.1229886,61.85430136)
\curveto(281.12298719,61.90429324)(281.12798719,61.94929319)(281.1379886,61.98930136)
\curveto(281.14798717,62.02929311)(281.15298716,62.06429308)(281.1529886,62.09430136)
\curveto(281.15298716,62.13429301)(281.15798716,62.17429297)(281.1679886,62.21430136)
\curveto(281.19798712,62.32429282)(281.2179871,62.43429271)(281.2279886,62.54430136)
\curveto(281.24798707,62.66429248)(281.28298703,62.77929236)(281.3329886,62.88930136)
\curveto(281.47298684,63.22929191)(281.63298668,63.50429164)(281.8129886,63.71430136)
\curveto(282.00298631,63.93429121)(282.27298604,64.11429103)(282.6229886,64.25430136)
\curveto(282.70298561,64.28429086)(282.78798553,64.30429084)(282.8779886,64.31430136)
\curveto(282.96798535,64.33429081)(283.06298525,64.35429079)(283.1629886,64.37430136)
\curveto(283.19298512,64.38429076)(283.24798507,64.38429076)(283.3279886,64.37430136)
\curveto(283.40798491,64.37429077)(283.45798486,64.38429076)(283.4779886,64.40430136)
\curveto(284.03798428,64.41429073)(284.48798383,64.30429084)(284.8279886,64.07430136)
\curveto(285.17798314,63.8442913)(285.43798288,63.5392916)(285.6079886,63.15930136)
\curveto(285.64798267,63.06929207)(285.68298263,62.97429217)(285.7129886,62.87430136)
\curveto(285.74298257,62.77429237)(285.76798255,62.67429247)(285.7879886,62.57430136)
\curveto(285.80798251,62.5442926)(285.8129825,62.51429263)(285.8029886,62.48430136)
\curveto(285.80298251,62.45429269)(285.80798251,62.42429272)(285.8179886,62.39430136)
\curveto(285.84798247,62.28429286)(285.86798245,62.15929298)(285.8779886,62.01930136)
\curveto(285.88798243,61.88929325)(285.89798242,61.75429339)(285.9079886,61.61430136)
\lineto(285.9079886,61.44930136)
\curveto(285.9179824,61.38929375)(285.9179824,61.33429381)(285.9079886,61.28430136)
\curveto(285.89798242,61.23429391)(285.89298242,61.18429396)(285.8929886,61.13430136)
\lineto(285.8929886,60.99930136)
\curveto(285.88298243,60.95929418)(285.87798244,60.91929422)(285.8779886,60.87930136)
\curveto(285.88798243,60.8392943)(285.88298243,60.79429435)(285.8629886,60.74430136)
\curveto(285.84298247,60.63429451)(285.82298249,60.52929461)(285.8029886,60.42930136)
\curveto(285.79298252,60.32929481)(285.77298254,60.22929491)(285.7429886,60.12930136)
\curveto(285.6129827,59.76929537)(285.44798287,59.45429569)(285.2479886,59.18430136)
\curveto(285.04798327,58.91429623)(284.77298354,58.70929643)(284.4229886,58.56930136)
\curveto(284.34298397,58.5392966)(284.25798406,58.51429663)(284.1679886,58.49430136)
\lineto(283.8979886,58.43430136)
\curveto(283.84798447,58.42429672)(283.80298451,58.41929672)(283.7629886,58.41930136)
\curveto(283.72298459,58.42929671)(283.68298463,58.42929671)(283.6429886,58.41930136)
\curveto(283.54298477,58.39929674)(283.44798487,58.39929674)(283.3579886,58.41930136)
\moveto(282.5179886,59.81430136)
\curveto(282.55798576,59.7442954)(282.59798572,59.67929546)(282.6379886,59.61930136)
\curveto(282.67798564,59.56929557)(282.72798559,59.51929562)(282.7879886,59.46930136)
\lineto(282.9379886,59.34930136)
\curveto(282.99798532,59.31929582)(283.06298525,59.29429585)(283.1329886,59.27430136)
\curveto(283.17298514,59.25429589)(283.20798511,59.2442959)(283.2379886,59.24430136)
\curveto(283.27798504,59.25429589)(283.317985,59.24929589)(283.3579886,59.22930136)
\curveto(283.38798493,59.22929591)(283.42798489,59.22429592)(283.4779886,59.21430136)
\curveto(283.52798479,59.21429593)(283.56798475,59.21929592)(283.5979886,59.22930136)
\lineto(283.8229886,59.27430136)
\curveto(284.07298424,59.35429579)(284.25798406,59.47929566)(284.3779886,59.64930136)
\curveto(284.45798386,59.74929539)(284.52798379,59.87929526)(284.5879886,60.03930136)
\curveto(284.66798365,60.21929492)(284.72798359,60.4442947)(284.7679886,60.71430136)
\curveto(284.80798351,60.99429415)(284.82298349,61.27429387)(284.8129886,61.55430136)
\curveto(284.80298351,61.8442933)(284.77298354,62.11929302)(284.7229886,62.37930136)
\curveto(284.67298364,62.6392925)(284.59798372,62.84929229)(284.4979886,63.00930136)
\curveto(284.37798394,63.20929193)(284.22798409,63.35929178)(284.0479886,63.45930136)
\curveto(283.96798435,63.50929163)(283.87798444,63.5392916)(283.7779886,63.54930136)
\curveto(283.67798464,63.56929157)(283.57298474,63.57929156)(283.4629886,63.57930136)
\curveto(283.44298487,63.56929157)(283.4179849,63.56429158)(283.3879886,63.56430136)
\curveto(283.36798495,63.57429157)(283.34798497,63.57429157)(283.3279886,63.56430136)
\curveto(283.27798504,63.55429159)(283.23298508,63.5442916)(283.1929886,63.53430136)
\curveto(283.15298516,63.53429161)(283.1129852,63.52429162)(283.0729886,63.50430136)
\curveto(282.89298542,63.42429172)(282.74298557,63.30429184)(282.6229886,63.14430136)
\curveto(282.5129858,62.98429216)(282.42298589,62.80429234)(282.3529886,62.60430136)
\curveto(282.29298602,62.41429273)(282.24798607,62.18929295)(282.2179886,61.92930136)
\curveto(282.19798612,61.66929347)(282.19298612,61.40429374)(282.2029886,61.13430136)
\curveto(282.2129861,60.87429427)(282.24298607,60.62429452)(282.2929886,60.38430136)
\curveto(282.35298596,60.15429499)(282.42798589,59.96429518)(282.5179886,59.81430136)
\moveto(293.3179886,56.82930136)
\curveto(293.32797499,56.77929836)(293.33297498,56.68929845)(293.3329886,56.55930136)
\curveto(293.33297498,56.42929871)(293.32297499,56.3392988)(293.3029886,56.28930136)
\curveto(293.28297503,56.2392989)(293.27797504,56.18429896)(293.2879886,56.12430136)
\curveto(293.29797502,56.07429907)(293.29797502,56.02429912)(293.2879886,55.97430136)
\curveto(293.24797507,55.83429931)(293.2179751,55.69929944)(293.1979886,55.56930136)
\curveto(293.18797513,55.4392997)(293.15797516,55.31929982)(293.1079886,55.20930136)
\curveto(292.96797535,54.85930028)(292.80297551,54.56430058)(292.6129886,54.32430136)
\curveto(292.42297589,54.09430105)(292.15297616,53.90930123)(291.8029886,53.76930136)
\curveto(291.72297659,53.7393014)(291.63797668,53.71930142)(291.5479886,53.70930136)
\curveto(291.45797686,53.68930145)(291.37297694,53.66930147)(291.2929886,53.64930136)
\curveto(291.24297707,53.6393015)(291.19297712,53.63430151)(291.1429886,53.63430136)
\curveto(291.09297722,53.63430151)(291.04297727,53.62930151)(290.9929886,53.61930136)
\curveto(290.96297735,53.60930153)(290.9129774,53.60930153)(290.8429886,53.61930136)
\curveto(290.77297754,53.61930152)(290.72297759,53.62430152)(290.6929886,53.63430136)
\curveto(290.63297768,53.65430149)(290.57297774,53.66430148)(290.5129886,53.66430136)
\curveto(290.46297785,53.65430149)(290.4129779,53.65930148)(290.3629886,53.67930136)
\curveto(290.27297804,53.69930144)(290.18297813,53.72430142)(290.0929886,53.75430136)
\curveto(290.0129783,53.77430137)(289.93297838,53.80430134)(289.8529886,53.84430136)
\curveto(289.53297878,53.98430116)(289.28297903,54.17930096)(289.1029886,54.42930136)
\curveto(288.92297939,54.68930045)(288.77297954,54.99430015)(288.6529886,55.34430136)
\curveto(288.63297968,55.42429972)(288.6179797,55.50929963)(288.6079886,55.59930136)
\curveto(288.59797972,55.68929945)(288.58297973,55.77429937)(288.5629886,55.85430136)
\curveto(288.55297976,55.88429926)(288.54797977,55.91429923)(288.5479886,55.94430136)
\lineto(288.5479886,56.04930136)
\curveto(288.52797979,56.12929901)(288.5179798,56.20929893)(288.5179886,56.28930136)
\lineto(288.5179886,56.42430136)
\curveto(288.49797982,56.52429862)(288.49797982,56.62429852)(288.5179886,56.72430136)
\lineto(288.5179886,56.90430136)
\curveto(288.52797979,56.95429819)(288.53297978,56.99929814)(288.5329886,57.03930136)
\curveto(288.53297978,57.08929805)(288.53797978,57.13429801)(288.5479886,57.17430136)
\curveto(288.55797976,57.21429793)(288.56297975,57.24929789)(288.5629886,57.27930136)
\curveto(288.56297975,57.31929782)(288.56797975,57.35929778)(288.5779886,57.39930136)
\lineto(288.6379886,57.72930136)
\curveto(288.65797966,57.84929729)(288.68797963,57.95929718)(288.7279886,58.05930136)
\curveto(288.86797945,58.38929675)(289.02797929,58.66429648)(289.2079886,58.88430136)
\curveto(289.39797892,59.11429603)(289.65797866,59.29929584)(289.9879886,59.43930136)
\curveto(290.06797825,59.47929566)(290.15297816,59.50429564)(290.2429886,59.51430136)
\lineto(290.5429886,59.57430136)
\lineto(290.6779886,59.57430136)
\curveto(290.72797759,59.58429556)(290.77797754,59.58929555)(290.8279886,59.58930136)
\curveto(291.39797692,59.60929553)(291.85797646,59.50429564)(292.2079886,59.27430136)
\curveto(292.56797575,59.05429609)(292.83297548,58.75429639)(293.0029886,58.37430136)
\curveto(293.05297526,58.27429687)(293.09297522,58.17429697)(293.1229886,58.07430136)
\curveto(293.15297516,57.97429717)(293.18297513,57.86929727)(293.2129886,57.75930136)
\curveto(293.22297509,57.71929742)(293.22797509,57.68429746)(293.2279886,57.65430136)
\curveto(293.22797509,57.63429751)(293.23297508,57.60429754)(293.2429886,57.56430136)
\curveto(293.26297505,57.49429765)(293.27297504,57.41929772)(293.2729886,57.33930136)
\curveto(293.27297504,57.25929788)(293.28297503,57.17929796)(293.3029886,57.09930136)
\curveto(293.30297501,57.04929809)(293.30297501,57.00429814)(293.3029886,56.96430136)
\curveto(293.30297501,56.92429822)(293.30797501,56.87929826)(293.3179886,56.82930136)
\moveto(292.2079886,56.39430136)
\curveto(292.2179761,56.4442987)(292.22297609,56.51929862)(292.2229886,56.61930136)
\curveto(292.23297608,56.71929842)(292.22797609,56.79429835)(292.2079886,56.84430136)
\curveto(292.18797613,56.90429824)(292.18297613,56.95929818)(292.1929886,57.00930136)
\curveto(292.2129761,57.06929807)(292.2129761,57.12929801)(292.1929886,57.18930136)
\curveto(292.18297613,57.21929792)(292.17797614,57.25429789)(292.1779886,57.29430136)
\curveto(292.17797614,57.33429781)(292.17297614,57.37429777)(292.1629886,57.41430136)
\curveto(292.14297617,57.49429765)(292.12297619,57.56929757)(292.1029886,57.63930136)
\curveto(292.09297622,57.71929742)(292.07797624,57.79929734)(292.0579886,57.87930136)
\curveto(292.02797629,57.9392972)(292.00297631,57.99929714)(291.9829886,58.05930136)
\curveto(291.96297635,58.11929702)(291.93297638,58.17929696)(291.8929886,58.23930136)
\curveto(291.79297652,58.40929673)(291.66297665,58.5442966)(291.5029886,58.64430136)
\curveto(291.42297689,58.69429645)(291.32797699,58.72929641)(291.2179886,58.74930136)
\curveto(291.10797721,58.76929637)(290.98297733,58.77929636)(290.8429886,58.77930136)
\curveto(290.82297749,58.76929637)(290.79797752,58.76429638)(290.7679886,58.76430136)
\curveto(290.73797758,58.77429637)(290.70797761,58.77429637)(290.6779886,58.76430136)
\lineto(290.5279886,58.70430136)
\curveto(290.47797784,58.69429645)(290.43297788,58.67929646)(290.3929886,58.65930136)
\curveto(290.20297811,58.54929659)(290.05797826,58.40429674)(289.9579886,58.22430136)
\curveto(289.86797845,58.0442971)(289.78797853,57.8392973)(289.7179886,57.60930136)
\curveto(289.67797864,57.47929766)(289.65797866,57.3442978)(289.6579886,57.20430136)
\curveto(289.65797866,57.07429807)(289.64797867,56.92929821)(289.6279886,56.76930136)
\curveto(289.6179787,56.71929842)(289.60797871,56.65929848)(289.5979886,56.58930136)
\curveto(289.59797872,56.51929862)(289.60797871,56.45929868)(289.6279886,56.40930136)
\lineto(289.6279886,56.24430136)
\lineto(289.6279886,56.06430136)
\curveto(289.63797868,56.01429913)(289.64797867,55.95929918)(289.6579886,55.89930136)
\curveto(289.66797865,55.84929929)(289.67297864,55.79429935)(289.6729886,55.73430136)
\curveto(289.68297863,55.67429947)(289.69797862,55.61929952)(289.7179886,55.56930136)
\curveto(289.76797855,55.37929976)(289.82797849,55.20429994)(289.8979886,55.04430136)
\curveto(289.96797835,54.88430026)(290.07297824,54.75430039)(290.2129886,54.65430136)
\curveto(290.34297797,54.55430059)(290.48297783,54.48430066)(290.6329886,54.44430136)
\curveto(290.66297765,54.43430071)(290.68797763,54.42930071)(290.7079886,54.42930136)
\curveto(290.73797758,54.4393007)(290.76797755,54.4393007)(290.7979886,54.42930136)
\curveto(290.8179775,54.42930071)(290.84797747,54.42430072)(290.8879886,54.41430136)
\curveto(290.92797739,54.41430073)(290.96297735,54.41930072)(290.9929886,54.42930136)
\curveto(291.03297728,54.4393007)(291.07297724,54.4443007)(291.1129886,54.44430136)
\curveto(291.15297716,54.4443007)(291.19297712,54.45430069)(291.2329886,54.47430136)
\curveto(291.47297684,54.55430059)(291.66797665,54.68930045)(291.8179886,54.87930136)
\curveto(291.93797638,55.05930008)(292.02797629,55.26429988)(292.0879886,55.49430136)
\curveto(292.10797621,55.56429958)(292.12297619,55.63429951)(292.1329886,55.70430136)
\curveto(292.14297617,55.78429936)(292.15797616,55.86429928)(292.1779886,55.94430136)
\curveto(292.17797614,56.00429914)(292.18297613,56.04929909)(292.1929886,56.07930136)
\curveto(292.19297612,56.09929904)(292.19297612,56.12429902)(292.1929886,56.15430136)
\curveto(292.19297612,56.19429895)(292.19797612,56.22429892)(292.2079886,56.24430136)
\lineto(292.2079886,56.39430136)
}
}
{
\newrgbcolor{curcolor}{0 0 0}
\pscustom[linestyle=none,fillstyle=solid,fillcolor=curcolor]
{
\newpath
\moveto(92.14290804,145.86571249)
\curveto(92.21290039,145.81570903)(92.25290035,145.7457091)(92.26290804,145.65571249)
\curveto(92.28290032,145.56570928)(92.29290031,145.46070939)(92.29290804,145.34071249)
\curveto(92.29290031,145.29070956)(92.28790032,145.24070961)(92.27790804,145.19071249)
\curveto(92.27790033,145.14070971)(92.26790034,145.09570975)(92.24790804,145.05571249)
\curveto(92.21790039,144.96570988)(92.15790045,144.90570994)(92.06790804,144.87571249)
\curveto(91.98790062,144.85570999)(91.89290071,144.84571)(91.78290804,144.84571249)
\lineto(91.46790804,144.84571249)
\curveto(91.35790125,144.85570999)(91.25290135,144.84571)(91.15290804,144.81571249)
\curveto(91.01290159,144.78571006)(90.92290168,144.70571014)(90.88290804,144.57571249)
\curveto(90.86290174,144.50571034)(90.85290175,144.42071043)(90.85290804,144.32071249)
\lineto(90.85290804,144.05071249)
\lineto(90.85290804,143.10571249)
\lineto(90.85290804,142.77571249)
\curveto(90.85290175,142.66571218)(90.83290177,142.58071227)(90.79290804,142.52071249)
\curveto(90.75290185,142.46071239)(90.7029019,142.42071243)(90.64290804,142.40071249)
\curveto(90.59290201,142.39071246)(90.52790208,142.37571247)(90.44790804,142.35571249)
\lineto(90.25290804,142.35571249)
\curveto(90.13290247,142.35571249)(90.02790258,142.36071249)(89.93790804,142.37071249)
\curveto(89.84790276,142.39071246)(89.77790283,142.44071241)(89.72790804,142.52071249)
\curveto(89.69790291,142.57071228)(89.68290292,142.64071221)(89.68290804,142.73071249)
\lineto(89.68290804,143.03071249)
\lineto(89.68290804,144.06571249)
\curveto(89.68290292,144.22571062)(89.67290293,144.37071048)(89.65290804,144.50071249)
\curveto(89.64290296,144.64071021)(89.58790302,144.73571011)(89.48790804,144.78571249)
\curveto(89.43790317,144.80571004)(89.36790324,144.82071003)(89.27790804,144.83071249)
\curveto(89.19790341,144.84071001)(89.1079035,144.84571)(89.00790804,144.84571249)
\lineto(88.72290804,144.84571249)
\lineto(88.48290804,144.84571249)
\lineto(86.21790804,144.84571249)
\curveto(86.12790648,144.84571)(86.02290658,144.84071001)(85.90290804,144.83071249)
\lineto(85.57290804,144.83071249)
\curveto(85.46290714,144.83071002)(85.36290724,144.84071001)(85.27290804,144.86071249)
\curveto(85.18290742,144.88070997)(85.12290748,144.91570993)(85.09290804,144.96571249)
\curveto(85.04290756,145.03570981)(85.01790759,145.13070972)(85.01790804,145.25071249)
\lineto(85.01790804,145.59571249)
\lineto(85.01790804,145.86571249)
\curveto(85.05790755,146.03570881)(85.11290749,146.17570867)(85.18290804,146.28571249)
\curveto(85.25290735,146.39570845)(85.33290727,146.51070834)(85.42290804,146.63071249)
\lineto(85.78290804,147.17071249)
\curveto(86.22290638,147.80070705)(86.65790595,148.42070643)(87.08790804,149.03071249)
\lineto(88.40790804,150.89071249)
\curveto(88.56790404,151.12070373)(88.72290388,151.34070351)(88.87290804,151.55071249)
\curveto(89.02290358,151.77070308)(89.17790343,151.99570285)(89.33790804,152.22571249)
\curveto(89.38790322,152.29570255)(89.43790317,152.36070249)(89.48790804,152.42071249)
\curveto(89.53790307,152.49070236)(89.58790302,152.56570228)(89.63790804,152.64571249)
\lineto(89.69790804,152.73571249)
\curveto(89.72790288,152.77570207)(89.75790285,152.80570204)(89.78790804,152.82571249)
\curveto(89.82790278,152.85570199)(89.86790274,152.87570197)(89.90790804,152.88571249)
\curveto(89.94790266,152.90570194)(89.99290261,152.92570192)(90.04290804,152.94571249)
\curveto(90.06290254,152.9457019)(90.08290252,152.94070191)(90.10290804,152.93071249)
\curveto(90.13290247,152.93070192)(90.15790245,152.94070191)(90.17790804,152.96071249)
\curveto(90.3079023,152.96070189)(90.42790218,152.95570189)(90.53790804,152.94571249)
\curveto(90.64790196,152.93570191)(90.72790188,152.89070196)(90.77790804,152.81071249)
\curveto(90.81790179,152.76070209)(90.83790177,152.69070216)(90.83790804,152.60071249)
\curveto(90.84790176,152.51070234)(90.85290175,152.41570243)(90.85290804,152.31571249)
\lineto(90.85290804,146.85571249)
\curveto(90.85290175,146.78570806)(90.84790176,146.71070814)(90.83790804,146.63071249)
\curveto(90.83790177,146.56070829)(90.84290176,146.49070836)(90.85290804,146.42071249)
\lineto(90.85290804,146.31571249)
\curveto(90.87290173,146.26570858)(90.88790172,146.21070864)(90.89790804,146.15071249)
\curveto(90.9079017,146.10070875)(90.93290167,146.06070879)(90.97290804,146.03071249)
\curveto(91.04290156,145.98070887)(91.12790148,145.9507089)(91.22790804,145.94071249)
\lineto(91.55790804,145.94071249)
\curveto(91.66790094,145.94070891)(91.77290083,145.93570891)(91.87290804,145.92571249)
\curveto(91.98290062,145.92570892)(92.07290053,145.90570894)(92.14290804,145.86571249)
\moveto(89.57790804,146.06071249)
\curveto(89.65790295,146.17070868)(89.69290291,146.34070851)(89.68290804,146.57071249)
\lineto(89.68290804,147.18571249)
\lineto(89.68290804,149.66071249)
\lineto(89.68290804,149.97571249)
\curveto(89.69290291,150.09570475)(89.68790292,150.19570465)(89.66790804,150.27571249)
\lineto(89.66790804,150.42571249)
\curveto(89.66790294,150.51570433)(89.65290295,150.60070425)(89.62290804,150.68071249)
\curveto(89.61290299,150.70070415)(89.602903,150.71070414)(89.59290804,150.71071249)
\lineto(89.54790804,150.75571249)
\curveto(89.52790308,150.76570408)(89.49790311,150.77070408)(89.45790804,150.77071249)
\curveto(89.43790317,150.7507041)(89.41790319,150.73570411)(89.39790804,150.72571249)
\curveto(89.38790322,150.72570412)(89.37290323,150.72070413)(89.35290804,150.71071249)
\curveto(89.29290331,150.66070419)(89.23290337,150.59070426)(89.17290804,150.50071249)
\curveto(89.11290349,150.41070444)(89.05790355,150.33070452)(89.00790804,150.26071249)
\curveto(88.9079037,150.12070473)(88.81290379,149.97570487)(88.72290804,149.82571249)
\curveto(88.63290397,149.68570516)(88.53790407,149.5457053)(88.43790804,149.40571249)
\lineto(87.89790804,148.62571249)
\curveto(87.72790488,148.36570648)(87.55290505,148.10570674)(87.37290804,147.84571249)
\curveto(87.29290531,147.73570711)(87.21790539,147.63070722)(87.14790804,147.53071249)
\lineto(86.93790804,147.23071249)
\curveto(86.88790572,147.1507077)(86.83790577,147.07570777)(86.78790804,147.00571249)
\curveto(86.74790586,146.93570791)(86.7029059,146.86070799)(86.65290804,146.78071249)
\curveto(86.602906,146.72070813)(86.55290605,146.65570819)(86.50290804,146.58571249)
\curveto(86.46290614,146.52570832)(86.42290618,146.45570839)(86.38290804,146.37571249)
\curveto(86.34290626,146.31570853)(86.31790629,146.2457086)(86.30790804,146.16571249)
\curveto(86.29790631,146.09570875)(86.33290627,146.04070881)(86.41290804,146.00071249)
\curveto(86.48290612,145.9507089)(86.59290601,145.92570892)(86.74290804,145.92571249)
\curveto(86.9029057,145.93570891)(87.03790557,145.94070891)(87.14790804,145.94071249)
\lineto(88.82790804,145.94071249)
\lineto(89.26290804,145.94071249)
\curveto(89.41290319,145.94070891)(89.51790309,145.98070887)(89.57790804,146.06071249)
}
}
{
\newrgbcolor{curcolor}{0 0 0}
\pscustom[linestyle=none,fillstyle=solid,fillcolor=curcolor]
{
\newpath
\moveto(95.12251741,152.78071249)
\lineto(98.72251741,152.78071249)
\lineto(99.36751741,152.78071249)
\curveto(99.44751088,152.78070207)(99.52251081,152.77570207)(99.59251741,152.76571249)
\curveto(99.66251067,152.76570208)(99.72251061,152.75570209)(99.77251741,152.73571249)
\curveto(99.84251049,152.70570214)(99.89751043,152.6457022)(99.93751741,152.55571249)
\curveto(99.95751037,152.52570232)(99.96751036,152.48570236)(99.96751741,152.43571249)
\lineto(99.96751741,152.30071249)
\curveto(99.97751035,152.19070266)(99.97251036,152.08570276)(99.95251741,151.98571249)
\curveto(99.94251039,151.88570296)(99.90751042,151.81570303)(99.84751741,151.77571249)
\curveto(99.75751057,151.70570314)(99.62251071,151.67070318)(99.44251741,151.67071249)
\curveto(99.26251107,151.68070317)(99.09751123,151.68570316)(98.94751741,151.68571249)
\lineto(96.95251741,151.68571249)
\lineto(96.45751741,151.68571249)
\lineto(96.32251741,151.68571249)
\curveto(96.28251405,151.68570316)(96.24251409,151.68070317)(96.20251741,151.67071249)
\lineto(95.99251741,151.67071249)
\curveto(95.88251445,151.64070321)(95.80251453,151.60070325)(95.75251741,151.55071249)
\curveto(95.70251463,151.51070334)(95.66751466,151.45570339)(95.64751741,151.38571249)
\curveto(95.6275147,151.32570352)(95.61251472,151.25570359)(95.60251741,151.17571249)
\curveto(95.59251474,151.09570375)(95.57251476,151.00570384)(95.54251741,150.90571249)
\curveto(95.49251484,150.70570414)(95.45251488,150.50070435)(95.42251741,150.29071249)
\curveto(95.39251494,150.08070477)(95.35251498,149.87570497)(95.30251741,149.67571249)
\curveto(95.28251505,149.60570524)(95.27251506,149.53570531)(95.27251741,149.46571249)
\curveto(95.27251506,149.40570544)(95.26251507,149.34070551)(95.24251741,149.27071249)
\curveto(95.2325151,149.24070561)(95.22251511,149.20070565)(95.21251741,149.15071249)
\curveto(95.21251512,149.11070574)(95.21751511,149.07070578)(95.22751741,149.03071249)
\curveto(95.24751508,148.98070587)(95.27251506,148.93570591)(95.30251741,148.89571249)
\curveto(95.34251499,148.86570598)(95.40251493,148.86070599)(95.48251741,148.88071249)
\curveto(95.54251479,148.90070595)(95.60251473,148.92570592)(95.66251741,148.95571249)
\curveto(95.72251461,148.99570585)(95.78251455,149.03070582)(95.84251741,149.06071249)
\curveto(95.90251443,149.08070577)(95.95251438,149.09570575)(95.99251741,149.10571249)
\curveto(96.18251415,149.18570566)(96.38751394,149.24070561)(96.60751741,149.27071249)
\curveto(96.83751349,149.30070555)(97.06751326,149.31070554)(97.29751741,149.30071249)
\curveto(97.53751279,149.30070555)(97.76751256,149.27570557)(97.98751741,149.22571249)
\curveto(98.20751212,149.18570566)(98.40751192,149.12570572)(98.58751741,149.04571249)
\curveto(98.63751169,149.02570582)(98.68251165,149.00570584)(98.72251741,148.98571249)
\curveto(98.77251156,148.96570588)(98.82251151,148.94070591)(98.87251741,148.91071249)
\curveto(99.22251111,148.70070615)(99.50251083,148.47070638)(99.71251741,148.22071249)
\curveto(99.9325104,147.97070688)(100.1275102,147.6457072)(100.29751741,147.24571249)
\curveto(100.34750998,147.13570771)(100.38250995,147.02570782)(100.40251741,146.91571249)
\curveto(100.42250991,146.80570804)(100.44750988,146.69070816)(100.47751741,146.57071249)
\curveto(100.48750984,146.54070831)(100.49250984,146.49570835)(100.49251741,146.43571249)
\curveto(100.51250982,146.37570847)(100.52250981,146.30570854)(100.52251741,146.22571249)
\curveto(100.52250981,146.15570869)(100.5325098,146.09070876)(100.55251741,146.03071249)
\lineto(100.55251741,145.86571249)
\curveto(100.56250977,145.81570903)(100.56750976,145.7457091)(100.56751741,145.65571249)
\curveto(100.56750976,145.56570928)(100.55750977,145.49570935)(100.53751741,145.44571249)
\curveto(100.51750981,145.38570946)(100.51250982,145.32570952)(100.52251741,145.26571249)
\curveto(100.5325098,145.21570963)(100.5275098,145.16570968)(100.50751741,145.11571249)
\curveto(100.46750986,144.95570989)(100.4325099,144.80571004)(100.40251741,144.66571249)
\curveto(100.37250996,144.52571032)(100.32751,144.39071046)(100.26751741,144.26071249)
\curveto(100.10751022,143.89071096)(99.88751044,143.55571129)(99.60751741,143.25571249)
\curveto(99.327511,142.95571189)(99.00751132,142.72571212)(98.64751741,142.56571249)
\curveto(98.47751185,142.48571236)(98.27751205,142.41071244)(98.04751741,142.34071249)
\curveto(97.93751239,142.30071255)(97.82251251,142.27571257)(97.70251741,142.26571249)
\curveto(97.58251275,142.25571259)(97.46251287,142.23571261)(97.34251741,142.20571249)
\curveto(97.29251304,142.18571266)(97.23751309,142.18571266)(97.17751741,142.20571249)
\curveto(97.11751321,142.21571263)(97.05751327,142.21071264)(96.99751741,142.19071249)
\curveto(96.89751343,142.17071268)(96.79751353,142.17071268)(96.69751741,142.19071249)
\lineto(96.56251741,142.19071249)
\curveto(96.51251382,142.21071264)(96.45251388,142.22071263)(96.38251741,142.22071249)
\curveto(96.32251401,142.21071264)(96.26751406,142.21571263)(96.21751741,142.23571249)
\curveto(96.17751415,142.2457126)(96.14251419,142.2507126)(96.11251741,142.25071249)
\curveto(96.08251425,142.2507126)(96.04751428,142.25571259)(96.00751741,142.26571249)
\lineto(95.73751741,142.32571249)
\curveto(95.64751468,142.3457125)(95.56251477,142.37571247)(95.48251741,142.41571249)
\curveto(95.14251519,142.55571229)(94.85251548,142.71071214)(94.61251741,142.88071249)
\curveto(94.37251596,143.06071179)(94.15251618,143.29071156)(93.95251741,143.57071249)
\curveto(93.80251653,143.80071105)(93.68751664,144.04071081)(93.60751741,144.29071249)
\curveto(93.58751674,144.34071051)(93.57751675,144.38571046)(93.57751741,144.42571249)
\curveto(93.57751675,144.47571037)(93.56751676,144.52571032)(93.54751741,144.57571249)
\curveto(93.5275168,144.63571021)(93.51251682,144.71571013)(93.50251741,144.81571249)
\curveto(93.50251683,144.91570993)(93.52251681,144.99070986)(93.56251741,145.04071249)
\curveto(93.61251672,145.12070973)(93.69251664,145.16570968)(93.80251741,145.17571249)
\curveto(93.91251642,145.18570966)(94.0275163,145.19070966)(94.14751741,145.19071249)
\lineto(94.31251741,145.19071249)
\curveto(94.37251596,145.19070966)(94.4275159,145.18070967)(94.47751741,145.16071249)
\curveto(94.56751576,145.14070971)(94.63751569,145.10070975)(94.68751741,145.04071249)
\curveto(94.75751557,144.9507099)(94.80251553,144.84071001)(94.82251741,144.71071249)
\curveto(94.85251548,144.59071026)(94.89751543,144.48571036)(94.95751741,144.39571249)
\curveto(95.14751518,144.05571079)(95.40751492,143.78571106)(95.73751741,143.58571249)
\curveto(95.83751449,143.52571132)(95.94251439,143.47571137)(96.05251741,143.43571249)
\curveto(96.17251416,143.40571144)(96.29251404,143.37071148)(96.41251741,143.33071249)
\curveto(96.58251375,143.28071157)(96.78751354,143.26071159)(97.02751741,143.27071249)
\curveto(97.27751305,143.29071156)(97.47751285,143.32571152)(97.62751741,143.37571249)
\curveto(97.99751233,143.49571135)(98.28751204,143.65571119)(98.49751741,143.85571249)
\curveto(98.71751161,144.06571078)(98.89751143,144.3457105)(99.03751741,144.69571249)
\curveto(99.08751124,144.79571005)(99.11751121,144.90070995)(99.12751741,145.01071249)
\curveto(99.14751118,145.12070973)(99.17251116,145.23570961)(99.20251741,145.35571249)
\lineto(99.20251741,145.46071249)
\curveto(99.21251112,145.50070935)(99.21751111,145.54070931)(99.21751741,145.58071249)
\curveto(99.2275111,145.61070924)(99.2275111,145.6457092)(99.21751741,145.68571249)
\lineto(99.21751741,145.80571249)
\curveto(99.21751111,146.06570878)(99.18751114,146.31070854)(99.12751741,146.54071249)
\curveto(99.01751131,146.89070796)(98.86251147,147.18570766)(98.66251741,147.42571249)
\curveto(98.46251187,147.67570717)(98.20251213,147.87070698)(97.88251741,148.01071249)
\lineto(97.70251741,148.07071249)
\curveto(97.65251268,148.09070676)(97.59251274,148.11070674)(97.52251741,148.13071249)
\curveto(97.47251286,148.1507067)(97.41251292,148.16070669)(97.34251741,148.16071249)
\curveto(97.28251305,148.17070668)(97.21751311,148.18570666)(97.14751741,148.20571249)
\lineto(96.99751741,148.20571249)
\curveto(96.95751337,148.22570662)(96.90251343,148.23570661)(96.83251741,148.23571249)
\curveto(96.77251356,148.23570661)(96.71751361,148.22570662)(96.66751741,148.20571249)
\lineto(96.56251741,148.20571249)
\curveto(96.5325138,148.20570664)(96.49751383,148.20070665)(96.45751741,148.19071249)
\lineto(96.21751741,148.13071249)
\curveto(96.13751419,148.12070673)(96.05751427,148.10070675)(95.97751741,148.07071249)
\curveto(95.73751459,147.97070688)(95.50751482,147.83570701)(95.28751741,147.66571249)
\curveto(95.19751513,147.59570725)(95.11251522,147.52070733)(95.03251741,147.44071249)
\curveto(94.95251538,147.37070748)(94.85251548,147.31570753)(94.73251741,147.27571249)
\curveto(94.64251569,147.2457076)(94.50251583,147.23570761)(94.31251741,147.24571249)
\curveto(94.1325162,147.25570759)(94.01251632,147.28070757)(93.95251741,147.32071249)
\curveto(93.90251643,147.36070749)(93.86251647,147.42070743)(93.83251741,147.50071249)
\curveto(93.81251652,147.58070727)(93.81251652,147.66570718)(93.83251741,147.75571249)
\curveto(93.86251647,147.87570697)(93.88251645,147.99570685)(93.89251741,148.11571249)
\curveto(93.91251642,148.2457066)(93.93751639,148.37070648)(93.96751741,148.49071249)
\curveto(93.98751634,148.53070632)(93.99251634,148.56570628)(93.98251741,148.59571249)
\curveto(93.98251635,148.63570621)(93.99251634,148.68070617)(94.01251741,148.73071249)
\curveto(94.0325163,148.82070603)(94.04751628,148.91070594)(94.05751741,149.00071249)
\curveto(94.06751626,149.10070575)(94.08751624,149.19570565)(94.11751741,149.28571249)
\curveto(94.1275162,149.3457055)(94.1325162,149.40570544)(94.13251741,149.46571249)
\curveto(94.14251619,149.52570532)(94.15751617,149.58570526)(94.17751741,149.64571249)
\curveto(94.2275161,149.845705)(94.26251607,150.0507048)(94.28251741,150.26071249)
\curveto(94.31251602,150.48070437)(94.35251598,150.69070416)(94.40251741,150.89071249)
\curveto(94.4325159,150.99070386)(94.45251588,151.09070376)(94.46251741,151.19071249)
\curveto(94.47251586,151.29070356)(94.48751584,151.39070346)(94.50751741,151.49071249)
\curveto(94.51751581,151.52070333)(94.52251581,151.56070329)(94.52251741,151.61071249)
\curveto(94.55251578,151.72070313)(94.57251576,151.82570302)(94.58251741,151.92571249)
\curveto(94.60251573,152.03570281)(94.6275157,152.1457027)(94.65751741,152.25571249)
\curveto(94.67751565,152.33570251)(94.69251564,152.40570244)(94.70251741,152.46571249)
\curveto(94.71251562,152.53570231)(94.73751559,152.59570225)(94.77751741,152.64571249)
\curveto(94.79751553,152.67570217)(94.8275155,152.69570215)(94.86751741,152.70571249)
\curveto(94.90751542,152.72570212)(94.95251538,152.7457021)(95.00251741,152.76571249)
\curveto(95.06251527,152.76570208)(95.10251523,152.77070208)(95.12251741,152.78071249)
}
}
{
\newrgbcolor{curcolor}{0 0 0}
\pscustom[linestyle=none,fillstyle=solid,fillcolor=curcolor]
{
\newpath
\moveto(102.91712679,144.00571249)
\lineto(103.21712679,144.00571249)
\curveto(103.32712473,144.01571083)(103.43212462,144.01571083)(103.53212679,144.00571249)
\curveto(103.64212441,144.00571084)(103.74212431,143.99571085)(103.83212679,143.97571249)
\curveto(103.92212413,143.96571088)(103.99212406,143.94071091)(104.04212679,143.90071249)
\curveto(104.06212399,143.88071097)(104.07712398,143.850711)(104.08712679,143.81071249)
\curveto(104.10712395,143.77071108)(104.12712393,143.72571112)(104.14712679,143.67571249)
\lineto(104.14712679,143.60071249)
\curveto(104.1571239,143.5507113)(104.1571239,143.49571135)(104.14712679,143.43571249)
\lineto(104.14712679,143.28571249)
\lineto(104.14712679,142.80571249)
\curveto(104.14712391,142.63571221)(104.10712395,142.51571233)(104.02712679,142.44571249)
\curveto(103.9571241,142.39571245)(103.86712419,142.37071248)(103.75712679,142.37071249)
\lineto(103.42712679,142.37071249)
\lineto(102.97712679,142.37071249)
\curveto(102.82712523,142.37071248)(102.71212534,142.40071245)(102.63212679,142.46071249)
\curveto(102.59212546,142.49071236)(102.56212549,142.54071231)(102.54212679,142.61071249)
\curveto(102.52212553,142.69071216)(102.50712555,142.77571207)(102.49712679,142.86571249)
\lineto(102.49712679,143.15071249)
\curveto(102.50712555,143.2507116)(102.51212554,143.33571151)(102.51212679,143.40571249)
\lineto(102.51212679,143.60071249)
\curveto(102.51212554,143.66071119)(102.52212553,143.71571113)(102.54212679,143.76571249)
\curveto(102.58212547,143.87571097)(102.6521254,143.9457109)(102.75212679,143.97571249)
\curveto(102.78212527,143.97571087)(102.83712522,143.98571086)(102.91712679,144.00571249)
}
}
{
\newrgbcolor{curcolor}{0 0 0}
\pscustom[linestyle=none,fillstyle=solid,fillcolor=curcolor]
{
\newpath
\moveto(109.23728304,152.97571249)
\curveto(110.8672776,153.00570184)(111.91727655,152.4507024)(112.38728304,151.31071249)
\curveto(112.48727598,151.08070377)(112.55227591,150.79070406)(112.58228304,150.44071249)
\curveto(112.62227584,150.10070475)(112.59727587,149.79070506)(112.50728304,149.51071249)
\curveto(112.41727605,149.2507056)(112.29727617,149.02570582)(112.14728304,148.83571249)
\curveto(112.12727634,148.79570605)(112.10227636,148.76070609)(112.07228304,148.73071249)
\curveto(112.04227642,148.71070614)(112.01727645,148.68570616)(111.99728304,148.65571249)
\lineto(111.90728304,148.53571249)
\curveto(111.87727659,148.50570634)(111.84227662,148.48070637)(111.80228304,148.46071249)
\curveto(111.75227671,148.41070644)(111.69727677,148.36570648)(111.63728304,148.32571249)
\curveto(111.58727688,148.28570656)(111.54227692,148.23570661)(111.50228304,148.17571249)
\curveto(111.462277,148.13570671)(111.44727702,148.08570676)(111.45728304,148.02571249)
\curveto(111.467277,147.97570687)(111.49727697,147.93070692)(111.54728304,147.89071249)
\curveto(111.59727687,147.850707)(111.65227681,147.81070704)(111.71228304,147.77071249)
\curveto(111.78227668,147.74070711)(111.84727662,147.71070714)(111.90728304,147.68071249)
\curveto(111.9672765,147.6507072)(112.01727645,147.62070723)(112.05728304,147.59071249)
\curveto(112.37727609,147.37070748)(112.63227583,147.06070779)(112.82228304,146.66071249)
\curveto(112.8622756,146.57070828)(112.89227557,146.47570837)(112.91228304,146.37571249)
\curveto(112.94227552,146.28570856)(112.9672755,146.19570865)(112.98728304,146.10571249)
\curveto(112.99727547,146.05570879)(113.00227546,146.00570884)(113.00228304,145.95571249)
\curveto(113.01227545,145.91570893)(113.02227544,145.87070898)(113.03228304,145.82071249)
\curveto(113.04227542,145.77070908)(113.04227542,145.72070913)(113.03228304,145.67071249)
\curveto(113.02227544,145.62070923)(113.02727544,145.57070928)(113.04728304,145.52071249)
\curveto(113.05727541,145.47070938)(113.0622754,145.41070944)(113.06228304,145.34071249)
\curveto(113.0622754,145.27070958)(113.05227541,145.21070964)(113.03228304,145.16071249)
\lineto(113.03228304,144.93571249)
\lineto(112.97228304,144.69571249)
\curveto(112.9622755,144.62571022)(112.94727552,144.55571029)(112.92728304,144.48571249)
\curveto(112.89727557,144.39571045)(112.8672756,144.31071054)(112.83728304,144.23071249)
\curveto(112.81727565,144.1507107)(112.78727568,144.07071078)(112.74728304,143.99071249)
\curveto(112.72727574,143.93071092)(112.69727577,143.87071098)(112.65728304,143.81071249)
\curveto(112.62727584,143.76071109)(112.59227587,143.71071114)(112.55228304,143.66071249)
\curveto(112.35227611,143.3507115)(112.10227636,143.09071176)(111.80228304,142.88071249)
\curveto(111.50227696,142.68071217)(111.15727731,142.51571233)(110.76728304,142.38571249)
\curveto(110.64727782,142.3457125)(110.51727795,142.32071253)(110.37728304,142.31071249)
\curveto(110.24727822,142.29071256)(110.11227835,142.26571258)(109.97228304,142.23571249)
\curveto(109.90227856,142.22571262)(109.83227863,142.22071263)(109.76228304,142.22071249)
\curveto(109.70227876,142.22071263)(109.63727883,142.21571263)(109.56728304,142.20571249)
\curveto(109.52727894,142.19571265)(109.467279,142.19071266)(109.38728304,142.19071249)
\curveto(109.31727915,142.19071266)(109.2672792,142.19571265)(109.23728304,142.20571249)
\curveto(109.18727928,142.21571263)(109.14227932,142.22071263)(109.10228304,142.22071249)
\lineto(108.98228304,142.22071249)
\curveto(108.88227958,142.24071261)(108.78227968,142.25571259)(108.68228304,142.26571249)
\curveto(108.58227988,142.27571257)(108.48727998,142.29071256)(108.39728304,142.31071249)
\curveto(108.28728018,142.34071251)(108.17728029,142.36571248)(108.06728304,142.38571249)
\curveto(107.9672805,142.41571243)(107.8622806,142.45571239)(107.75228304,142.50571249)
\curveto(107.38228108,142.66571218)(107.0672814,142.86571198)(106.80728304,143.10571249)
\curveto(106.54728192,143.35571149)(106.33728213,143.66571118)(106.17728304,144.03571249)
\curveto(106.13728233,144.12571072)(106.10228236,144.22071063)(106.07228304,144.32071249)
\curveto(106.04228242,144.42071043)(106.01228245,144.52571032)(105.98228304,144.63571249)
\curveto(105.9622825,144.68571016)(105.95228251,144.73571011)(105.95228304,144.78571249)
\curveto(105.95228251,144.84571)(105.94228252,144.90570994)(105.92228304,144.96571249)
\curveto(105.90228256,145.02570982)(105.89228257,145.10570974)(105.89228304,145.20571249)
\curveto(105.89228257,145.30570954)(105.90728256,145.38070947)(105.93728304,145.43071249)
\curveto(105.94728252,145.46070939)(105.9622825,145.48570936)(105.98228304,145.50571249)
\lineto(106.04228304,145.56571249)
\curveto(106.08228238,145.58570926)(106.14228232,145.60070925)(106.22228304,145.61071249)
\curveto(106.31228215,145.62070923)(106.40228206,145.62570922)(106.49228304,145.62571249)
\curveto(106.58228188,145.62570922)(106.6672818,145.62070923)(106.74728304,145.61071249)
\curveto(106.83728163,145.60070925)(106.90228156,145.59070926)(106.94228304,145.58071249)
\curveto(106.9622815,145.56070929)(106.98228148,145.5457093)(107.00228304,145.53571249)
\curveto(107.02228144,145.53570931)(107.04228142,145.52570932)(107.06228304,145.50571249)
\curveto(107.13228133,145.41570943)(107.17228129,145.30070955)(107.18228304,145.16071249)
\curveto(107.20228126,145.02070983)(107.23228123,144.89570995)(107.27228304,144.78571249)
\lineto(107.42228304,144.42571249)
\curveto(107.47228099,144.31571053)(107.53728093,144.21071064)(107.61728304,144.11071249)
\curveto(107.63728083,144.08071077)(107.65728081,144.05571079)(107.67728304,144.03571249)
\curveto(107.70728076,144.01571083)(107.73228073,143.99071086)(107.75228304,143.96071249)
\curveto(107.79228067,143.90071095)(107.82728064,143.85571099)(107.85728304,143.82571249)
\curveto(107.89728057,143.79571105)(107.93228053,143.76571108)(107.96228304,143.73571249)
\curveto(108.00228046,143.70571114)(108.04728042,143.67571117)(108.09728304,143.64571249)
\curveto(108.18728028,143.58571126)(108.28228018,143.53571131)(108.38228304,143.49571249)
\lineto(108.71228304,143.37571249)
\curveto(108.8622796,143.32571152)(109.0622794,143.29571155)(109.31228304,143.28571249)
\curveto(109.5622789,143.27571157)(109.77227869,143.29571155)(109.94228304,143.34571249)
\curveto(110.02227844,143.36571148)(110.09227837,143.38071147)(110.15228304,143.39071249)
\lineto(110.36228304,143.45071249)
\curveto(110.64227782,143.57071128)(110.88227758,143.72071113)(111.08228304,143.90071249)
\curveto(111.29227717,144.08071077)(111.45727701,144.31071054)(111.57728304,144.59071249)
\curveto(111.60727686,144.66071019)(111.62727684,144.73071012)(111.63728304,144.80071249)
\lineto(111.69728304,145.04071249)
\curveto(111.73727673,145.18070967)(111.74727672,145.34070951)(111.72728304,145.52071249)
\curveto(111.70727676,145.71070914)(111.67727679,145.86070899)(111.63728304,145.97071249)
\curveto(111.50727696,146.3507085)(111.32227714,146.64070821)(111.08228304,146.84071249)
\curveto(110.85227761,147.04070781)(110.54227792,147.20070765)(110.15228304,147.32071249)
\curveto(110.04227842,147.3507075)(109.92227854,147.37070748)(109.79228304,147.38071249)
\curveto(109.67227879,147.39070746)(109.54727892,147.39570745)(109.41728304,147.39571249)
\curveto(109.25727921,147.39570745)(109.11727935,147.40070745)(108.99728304,147.41071249)
\curveto(108.87727959,147.42070743)(108.79227967,147.48070737)(108.74228304,147.59071249)
\curveto(108.72227974,147.62070723)(108.71227975,147.65570719)(108.71228304,147.69571249)
\lineto(108.71228304,147.83071249)
\curveto(108.70227976,147.93070692)(108.70227976,148.02570682)(108.71228304,148.11571249)
\curveto(108.73227973,148.20570664)(108.77227969,148.27070658)(108.83228304,148.31071249)
\curveto(108.87227959,148.34070651)(108.91227955,148.36070649)(108.95228304,148.37071249)
\curveto(109.00227946,148.38070647)(109.05727941,148.39070646)(109.11728304,148.40071249)
\curveto(109.13727933,148.41070644)(109.1622793,148.41070644)(109.19228304,148.40071249)
\curveto(109.22227924,148.40070645)(109.24727922,148.40570644)(109.26728304,148.41571249)
\lineto(109.40228304,148.41571249)
\curveto(109.51227895,148.43570641)(109.61227885,148.4457064)(109.70228304,148.44571249)
\curveto(109.80227866,148.45570639)(109.89727857,148.47570637)(109.98728304,148.50571249)
\curveto(110.30727816,148.61570623)(110.5622779,148.76070609)(110.75228304,148.94071249)
\curveto(110.94227752,149.12070573)(111.09227737,149.37070548)(111.20228304,149.69071249)
\curveto(111.23227723,149.79070506)(111.25227721,149.91570493)(111.26228304,150.06571249)
\curveto(111.28227718,150.22570462)(111.27727719,150.37070448)(111.24728304,150.50071249)
\curveto(111.22727724,150.57070428)(111.20727726,150.63570421)(111.18728304,150.69571249)
\curveto(111.17727729,150.76570408)(111.15727731,150.83070402)(111.12728304,150.89071249)
\curveto(111.02727744,151.13070372)(110.88227758,151.32070353)(110.69228304,151.46071249)
\curveto(110.50227796,151.60070325)(110.27727819,151.71070314)(110.01728304,151.79071249)
\curveto(109.95727851,151.81070304)(109.89727857,151.82070303)(109.83728304,151.82071249)
\curveto(109.77727869,151.82070303)(109.71227875,151.83070302)(109.64228304,151.85071249)
\curveto(109.5622789,151.87070298)(109.467279,151.88070297)(109.35728304,151.88071249)
\curveto(109.24727922,151.88070297)(109.15227931,151.87070298)(109.07228304,151.85071249)
\curveto(109.02227944,151.83070302)(108.97227949,151.82070303)(108.92228304,151.82071249)
\curveto(108.88227958,151.82070303)(108.83727963,151.81070304)(108.78728304,151.79071249)
\curveto(108.60727986,151.74070311)(108.43728003,151.66570318)(108.27728304,151.56571249)
\curveto(108.12728034,151.47570337)(107.99728047,151.36070349)(107.88728304,151.22071249)
\curveto(107.79728067,151.10070375)(107.71728075,150.97070388)(107.64728304,150.83071249)
\curveto(107.57728089,150.69070416)(107.51228095,150.53570431)(107.45228304,150.36571249)
\curveto(107.42228104,150.25570459)(107.40228106,150.13570471)(107.39228304,150.00571249)
\curveto(107.38228108,149.88570496)(107.34728112,149.78570506)(107.28728304,149.70571249)
\curveto(107.2672812,149.66570518)(107.20728126,149.62570522)(107.10728304,149.58571249)
\curveto(107.0672814,149.57570527)(107.00728146,149.56570528)(106.92728304,149.55571249)
\lineto(106.67228304,149.55571249)
\curveto(106.58228188,149.56570528)(106.49728197,149.57570527)(106.41728304,149.58571249)
\curveto(106.34728212,149.59570525)(106.29728217,149.61070524)(106.26728304,149.63071249)
\curveto(106.22728224,149.66070519)(106.19228227,149.71570513)(106.16228304,149.79571249)
\curveto(106.13228233,149.87570497)(106.12728234,149.96070489)(106.14728304,150.05071249)
\curveto(106.15728231,150.10070475)(106.1622823,150.1507047)(106.16228304,150.20071249)
\lineto(106.19228304,150.38071249)
\curveto(106.22228224,150.48070437)(106.24728222,150.58070427)(106.26728304,150.68071249)
\curveto(106.29728217,150.78070407)(106.33228213,150.87070398)(106.37228304,150.95071249)
\curveto(106.42228204,151.06070379)(106.467282,151.16570368)(106.50728304,151.26571249)
\curveto(106.54728192,151.37570347)(106.59728187,151.48070337)(106.65728304,151.58071249)
\curveto(106.98728148,152.12070273)(107.45728101,152.51570233)(108.06728304,152.76571249)
\curveto(108.18728028,152.81570203)(108.31228015,152.850702)(108.44228304,152.87071249)
\curveto(108.58227988,152.89070196)(108.72227974,152.91570193)(108.86228304,152.94571249)
\curveto(108.92227954,152.95570189)(108.98227948,152.96070189)(109.04228304,152.96071249)
\curveto(109.11227935,152.96070189)(109.17727929,152.96570188)(109.23728304,152.97571249)
}
}
{
\newrgbcolor{curcolor}{0 0 0}
\pscustom[linestyle=none,fillstyle=solid,fillcolor=curcolor]
{
\newpath
\moveto(124.27689241,150.89071249)
\curveto(124.07688211,150.60070425)(123.86688232,150.31570453)(123.64689241,150.03571249)
\curveto(123.43688275,149.75570509)(123.23188296,149.47070538)(123.03189241,149.18071249)
\curveto(122.43188376,148.33070652)(121.82688436,147.49070736)(121.21689241,146.66071249)
\curveto(120.60688558,145.84070901)(120.00188619,145.00570984)(119.40189241,144.15571249)
\lineto(118.89189241,143.43571249)
\lineto(118.38189241,142.74571249)
\curveto(118.30188789,142.63571221)(118.22188797,142.52071233)(118.14189241,142.40071249)
\curveto(118.06188813,142.28071257)(117.96688822,142.18571266)(117.85689241,142.11571249)
\curveto(117.81688837,142.09571275)(117.75188844,142.08071277)(117.66189241,142.07071249)
\curveto(117.58188861,142.0507128)(117.4918887,142.04071281)(117.39189241,142.04071249)
\curveto(117.2918889,142.04071281)(117.19688899,142.0457128)(117.10689241,142.05571249)
\curveto(117.02688916,142.06571278)(116.96688922,142.08571276)(116.92689241,142.11571249)
\curveto(116.89688929,142.13571271)(116.87188932,142.17071268)(116.85189241,142.22071249)
\curveto(116.84188935,142.26071259)(116.84688934,142.30571254)(116.86689241,142.35571249)
\curveto(116.90688928,142.43571241)(116.95188924,142.51071234)(117.00189241,142.58071249)
\curveto(117.06188913,142.66071219)(117.11688907,142.74071211)(117.16689241,142.82071249)
\curveto(117.40688878,143.16071169)(117.65188854,143.49571135)(117.90189241,143.82571249)
\curveto(118.15188804,144.15571069)(118.3918878,144.49071036)(118.62189241,144.83071249)
\curveto(118.78188741,145.0507098)(118.94188725,145.26570958)(119.10189241,145.47571249)
\curveto(119.26188693,145.68570916)(119.42188677,145.90070895)(119.58189241,146.12071249)
\curveto(119.94188625,146.64070821)(120.30688588,147.1507077)(120.67689241,147.65071249)
\curveto(121.04688514,148.1507067)(121.41688477,148.66070619)(121.78689241,149.18071249)
\curveto(121.92688426,149.38070547)(122.06688412,149.57570527)(122.20689241,149.76571249)
\curveto(122.35688383,149.95570489)(122.50188369,150.1507047)(122.64189241,150.35071249)
\curveto(122.85188334,150.6507042)(123.06688312,150.9507039)(123.28689241,151.25071249)
\lineto(123.94689241,152.15071249)
\lineto(124.12689241,152.42071249)
\lineto(124.33689241,152.69071249)
\lineto(124.45689241,152.87071249)
\curveto(124.50688168,152.93070192)(124.55688163,152.98570186)(124.60689241,153.03571249)
\curveto(124.67688151,153.08570176)(124.75188144,153.12070173)(124.83189241,153.14071249)
\curveto(124.85188134,153.1507017)(124.87688131,153.1507017)(124.90689241,153.14071249)
\curveto(124.94688124,153.14070171)(124.97688121,153.1507017)(124.99689241,153.17071249)
\curveto(125.11688107,153.17070168)(125.25188094,153.16570168)(125.40189241,153.15571249)
\curveto(125.55188064,153.15570169)(125.64188055,153.11070174)(125.67189241,153.02071249)
\curveto(125.6918805,152.99070186)(125.69688049,152.95570189)(125.68689241,152.91571249)
\curveto(125.67688051,152.87570197)(125.66188053,152.845702)(125.64189241,152.82571249)
\curveto(125.60188059,152.7457021)(125.56188063,152.67570217)(125.52189241,152.61571249)
\curveto(125.48188071,152.55570229)(125.43688075,152.49570235)(125.38689241,152.43571249)
\lineto(124.81689241,151.65571249)
\curveto(124.63688155,151.40570344)(124.45688173,151.1507037)(124.27689241,150.89071249)
\moveto(117.42189241,146.99071249)
\curveto(117.37188882,147.01070784)(117.32188887,147.01570783)(117.27189241,147.00571249)
\curveto(117.22188897,146.99570785)(117.17188902,147.00070785)(117.12189241,147.02071249)
\curveto(117.01188918,147.04070781)(116.90688928,147.06070779)(116.80689241,147.08071249)
\curveto(116.71688947,147.11070774)(116.62188957,147.1507077)(116.52189241,147.20071249)
\curveto(116.19189,147.34070751)(115.93689025,147.53570731)(115.75689241,147.78571249)
\curveto(115.57689061,148.0457068)(115.43189076,148.35570649)(115.32189241,148.71571249)
\curveto(115.2918909,148.79570605)(115.27189092,148.87570597)(115.26189241,148.95571249)
\curveto(115.25189094,149.0457058)(115.23689095,149.13070572)(115.21689241,149.21071249)
\curveto(115.20689098,149.26070559)(115.20189099,149.32570552)(115.20189241,149.40571249)
\curveto(115.191891,149.43570541)(115.186891,149.46570538)(115.18689241,149.49571249)
\curveto(115.186891,149.53570531)(115.18189101,149.57070528)(115.17189241,149.60071249)
\lineto(115.17189241,149.75071249)
\curveto(115.16189103,149.80070505)(115.15689103,149.86070499)(115.15689241,149.93071249)
\curveto(115.15689103,150.01070484)(115.16189103,150.07570477)(115.17189241,150.12571249)
\lineto(115.17189241,150.29071249)
\curveto(115.191891,150.34070451)(115.19689099,150.38570446)(115.18689241,150.42571249)
\curveto(115.186891,150.47570437)(115.191891,150.52070433)(115.20189241,150.56071249)
\curveto(115.21189098,150.60070425)(115.21689097,150.63570421)(115.21689241,150.66571249)
\curveto(115.21689097,150.70570414)(115.22189097,150.7457041)(115.23189241,150.78571249)
\curveto(115.26189093,150.89570395)(115.28189091,151.00570384)(115.29189241,151.11571249)
\curveto(115.31189088,151.23570361)(115.34689084,151.3507035)(115.39689241,151.46071249)
\curveto(115.53689065,151.80070305)(115.69689049,152.07570277)(115.87689241,152.28571249)
\curveto(116.06689012,152.50570234)(116.33688985,152.68570216)(116.68689241,152.82571249)
\curveto(116.76688942,152.85570199)(116.85188934,152.87570197)(116.94189241,152.88571249)
\curveto(117.03188916,152.90570194)(117.12688906,152.92570192)(117.22689241,152.94571249)
\curveto(117.25688893,152.95570189)(117.31188888,152.95570189)(117.39189241,152.94571249)
\curveto(117.47188872,152.9457019)(117.52188867,152.95570189)(117.54189241,152.97571249)
\curveto(118.10188809,152.98570186)(118.55188764,152.87570197)(118.89189241,152.64571249)
\curveto(119.24188695,152.41570243)(119.50188669,152.11070274)(119.67189241,151.73071249)
\curveto(119.71188648,151.64070321)(119.74688644,151.5457033)(119.77689241,151.44571249)
\curveto(119.80688638,151.3457035)(119.83188636,151.2457036)(119.85189241,151.14571249)
\curveto(119.87188632,151.11570373)(119.87688631,151.08570376)(119.86689241,151.05571249)
\curveto(119.86688632,151.02570382)(119.87188632,150.99570385)(119.88189241,150.96571249)
\curveto(119.91188628,150.85570399)(119.93188626,150.73070412)(119.94189241,150.59071249)
\curveto(119.95188624,150.46070439)(119.96188623,150.32570452)(119.97189241,150.18571249)
\lineto(119.97189241,150.02071249)
\curveto(119.98188621,149.96070489)(119.98188621,149.90570494)(119.97189241,149.85571249)
\curveto(119.96188623,149.80570504)(119.95688623,149.75570509)(119.95689241,149.70571249)
\lineto(119.95689241,149.57071249)
\curveto(119.94688624,149.53070532)(119.94188625,149.49070536)(119.94189241,149.45071249)
\curveto(119.95188624,149.41070544)(119.94688624,149.36570548)(119.92689241,149.31571249)
\curveto(119.90688628,149.20570564)(119.8868863,149.10070575)(119.86689241,149.00071249)
\curveto(119.85688633,148.90070595)(119.83688635,148.80070605)(119.80689241,148.70071249)
\curveto(119.67688651,148.34070651)(119.51188668,148.02570682)(119.31189241,147.75571249)
\curveto(119.11188708,147.48570736)(118.83688735,147.28070757)(118.48689241,147.14071249)
\curveto(118.40688778,147.11070774)(118.32188787,147.08570776)(118.23189241,147.06571249)
\lineto(117.96189241,147.00571249)
\curveto(117.91188828,146.99570785)(117.86688832,146.99070786)(117.82689241,146.99071249)
\curveto(117.7868884,147.00070785)(117.74688844,147.00070785)(117.70689241,146.99071249)
\curveto(117.60688858,146.97070788)(117.51188868,146.97070788)(117.42189241,146.99071249)
\moveto(116.58189241,148.38571249)
\curveto(116.62188957,148.31570653)(116.66188953,148.2507066)(116.70189241,148.19071249)
\curveto(116.74188945,148.14070671)(116.7918894,148.09070676)(116.85189241,148.04071249)
\lineto(117.00189241,147.92071249)
\curveto(117.06188913,147.89070696)(117.12688906,147.86570698)(117.19689241,147.84571249)
\curveto(117.23688895,147.82570702)(117.27188892,147.81570703)(117.30189241,147.81571249)
\curveto(117.34188885,147.82570702)(117.38188881,147.82070703)(117.42189241,147.80071249)
\curveto(117.45188874,147.80070705)(117.4918887,147.79570705)(117.54189241,147.78571249)
\curveto(117.5918886,147.78570706)(117.63188856,147.79070706)(117.66189241,147.80071249)
\lineto(117.88689241,147.84571249)
\curveto(118.13688805,147.92570692)(118.32188787,148.0507068)(118.44189241,148.22071249)
\curveto(118.52188767,148.32070653)(118.5918876,148.4507064)(118.65189241,148.61071249)
\curveto(118.73188746,148.79070606)(118.7918874,149.01570583)(118.83189241,149.28571249)
\curveto(118.87188732,149.56570528)(118.8868873,149.845705)(118.87689241,150.12571249)
\curveto(118.86688732,150.41570443)(118.83688735,150.69070416)(118.78689241,150.95071249)
\curveto(118.73688745,151.21070364)(118.66188753,151.42070343)(118.56189241,151.58071249)
\curveto(118.44188775,151.78070307)(118.2918879,151.93070292)(118.11189241,152.03071249)
\curveto(118.03188816,152.08070277)(117.94188825,152.11070274)(117.84189241,152.12071249)
\curveto(117.74188845,152.14070271)(117.63688855,152.1507027)(117.52689241,152.15071249)
\curveto(117.50688868,152.14070271)(117.48188871,152.13570271)(117.45189241,152.13571249)
\curveto(117.43188876,152.1457027)(117.41188878,152.1457027)(117.39189241,152.13571249)
\curveto(117.34188885,152.12570272)(117.29688889,152.11570273)(117.25689241,152.10571249)
\curveto(117.21688897,152.10570274)(117.17688901,152.09570275)(117.13689241,152.07571249)
\curveto(116.95688923,151.99570285)(116.80688938,151.87570297)(116.68689241,151.71571249)
\curveto(116.57688961,151.55570329)(116.4868897,151.37570347)(116.41689241,151.17571249)
\curveto(116.35688983,150.98570386)(116.31188988,150.76070409)(116.28189241,150.50071249)
\curveto(116.26188993,150.24070461)(116.25688993,149.97570487)(116.26689241,149.70571249)
\curveto(116.27688991,149.4457054)(116.30688988,149.19570565)(116.35689241,148.95571249)
\curveto(116.41688977,148.72570612)(116.4918897,148.53570631)(116.58189241,148.38571249)
\moveto(127.38189241,145.40071249)
\curveto(127.3918788,145.3507095)(127.39687879,145.26070959)(127.39689241,145.13071249)
\curveto(127.39687879,145.00070985)(127.3868788,144.91070994)(127.36689241,144.86071249)
\curveto(127.34687884,144.81071004)(127.34187885,144.75571009)(127.35189241,144.69571249)
\curveto(127.36187883,144.6457102)(127.36187883,144.59571025)(127.35189241,144.54571249)
\curveto(127.31187888,144.40571044)(127.28187891,144.27071058)(127.26189241,144.14071249)
\curveto(127.25187894,144.01071084)(127.22187897,143.89071096)(127.17189241,143.78071249)
\curveto(127.03187916,143.43071142)(126.86687932,143.13571171)(126.67689241,142.89571249)
\curveto(126.4868797,142.66571218)(126.21687997,142.48071237)(125.86689241,142.34071249)
\curveto(125.7868804,142.31071254)(125.70188049,142.29071256)(125.61189241,142.28071249)
\curveto(125.52188067,142.26071259)(125.43688075,142.24071261)(125.35689241,142.22071249)
\curveto(125.30688088,142.21071264)(125.25688093,142.20571264)(125.20689241,142.20571249)
\curveto(125.15688103,142.20571264)(125.10688108,142.20071265)(125.05689241,142.19071249)
\curveto(125.02688116,142.18071267)(124.97688121,142.18071267)(124.90689241,142.19071249)
\curveto(124.83688135,142.19071266)(124.7868814,142.19571265)(124.75689241,142.20571249)
\curveto(124.69688149,142.22571262)(124.63688155,142.23571261)(124.57689241,142.23571249)
\curveto(124.52688166,142.22571262)(124.47688171,142.23071262)(124.42689241,142.25071249)
\curveto(124.33688185,142.27071258)(124.24688194,142.29571255)(124.15689241,142.32571249)
\curveto(124.07688211,142.3457125)(123.99688219,142.37571247)(123.91689241,142.41571249)
\curveto(123.59688259,142.55571229)(123.34688284,142.7507121)(123.16689241,143.00071249)
\curveto(122.9868832,143.26071159)(122.83688335,143.56571128)(122.71689241,143.91571249)
\curveto(122.69688349,143.99571085)(122.68188351,144.08071077)(122.67189241,144.17071249)
\curveto(122.66188353,144.26071059)(122.64688354,144.3457105)(122.62689241,144.42571249)
\curveto(122.61688357,144.45571039)(122.61188358,144.48571036)(122.61189241,144.51571249)
\lineto(122.61189241,144.62071249)
\curveto(122.5918836,144.70071015)(122.58188361,144.78071007)(122.58189241,144.86071249)
\lineto(122.58189241,144.99571249)
\curveto(122.56188363,145.09570975)(122.56188363,145.19570965)(122.58189241,145.29571249)
\lineto(122.58189241,145.47571249)
\curveto(122.5918836,145.52570932)(122.59688359,145.57070928)(122.59689241,145.61071249)
\curveto(122.59688359,145.66070919)(122.60188359,145.70570914)(122.61189241,145.74571249)
\curveto(122.62188357,145.78570906)(122.62688356,145.82070903)(122.62689241,145.85071249)
\curveto(122.62688356,145.89070896)(122.63188356,145.93070892)(122.64189241,145.97071249)
\lineto(122.70189241,146.30071249)
\curveto(122.72188347,146.42070843)(122.75188344,146.53070832)(122.79189241,146.63071249)
\curveto(122.93188326,146.96070789)(123.0918831,147.23570761)(123.27189241,147.45571249)
\curveto(123.46188273,147.68570716)(123.72188247,147.87070698)(124.05189241,148.01071249)
\curveto(124.13188206,148.0507068)(124.21688197,148.07570677)(124.30689241,148.08571249)
\lineto(124.60689241,148.14571249)
\lineto(124.74189241,148.14571249)
\curveto(124.7918814,148.15570669)(124.84188135,148.16070669)(124.89189241,148.16071249)
\curveto(125.46188073,148.18070667)(125.92188027,148.07570677)(126.27189241,147.84571249)
\curveto(126.63187956,147.62570722)(126.89687929,147.32570752)(127.06689241,146.94571249)
\curveto(127.11687907,146.845708)(127.15687903,146.7457081)(127.18689241,146.64571249)
\curveto(127.21687897,146.5457083)(127.24687894,146.44070841)(127.27689241,146.33071249)
\curveto(127.2868789,146.29070856)(127.2918789,146.25570859)(127.29189241,146.22571249)
\curveto(127.2918789,146.20570864)(127.29687889,146.17570867)(127.30689241,146.13571249)
\curveto(127.32687886,146.06570878)(127.33687885,145.99070886)(127.33689241,145.91071249)
\curveto(127.33687885,145.83070902)(127.34687884,145.7507091)(127.36689241,145.67071249)
\curveto(127.36687882,145.62070923)(127.36687882,145.57570927)(127.36689241,145.53571249)
\curveto(127.36687882,145.49570935)(127.37187882,145.4507094)(127.38189241,145.40071249)
\moveto(126.27189241,144.96571249)
\curveto(126.28187991,145.01570983)(126.2868799,145.09070976)(126.28689241,145.19071249)
\curveto(126.29687989,145.29070956)(126.2918799,145.36570948)(126.27189241,145.41571249)
\curveto(126.25187994,145.47570937)(126.24687994,145.53070932)(126.25689241,145.58071249)
\curveto(126.27687991,145.64070921)(126.27687991,145.70070915)(126.25689241,145.76071249)
\curveto(126.24687994,145.79070906)(126.24187995,145.82570902)(126.24189241,145.86571249)
\curveto(126.24187995,145.90570894)(126.23687995,145.9457089)(126.22689241,145.98571249)
\curveto(126.20687998,146.06570878)(126.18688,146.14070871)(126.16689241,146.21071249)
\curveto(126.15688003,146.29070856)(126.14188005,146.37070848)(126.12189241,146.45071249)
\curveto(126.0918801,146.51070834)(126.06688012,146.57070828)(126.04689241,146.63071249)
\curveto(126.02688016,146.69070816)(125.99688019,146.7507081)(125.95689241,146.81071249)
\curveto(125.85688033,146.98070787)(125.72688046,147.11570773)(125.56689241,147.21571249)
\curveto(125.4868807,147.26570758)(125.3918808,147.30070755)(125.28189241,147.32071249)
\curveto(125.17188102,147.34070751)(125.04688114,147.3507075)(124.90689241,147.35071249)
\curveto(124.8868813,147.34070751)(124.86188133,147.33570751)(124.83189241,147.33571249)
\curveto(124.80188139,147.3457075)(124.77188142,147.3457075)(124.74189241,147.33571249)
\lineto(124.59189241,147.27571249)
\curveto(124.54188165,147.26570758)(124.49688169,147.2507076)(124.45689241,147.23071249)
\curveto(124.26688192,147.12070773)(124.12188207,146.97570787)(124.02189241,146.79571249)
\curveto(123.93188226,146.61570823)(123.85188234,146.41070844)(123.78189241,146.18071249)
\curveto(123.74188245,146.0507088)(123.72188247,145.91570893)(123.72189241,145.77571249)
\curveto(123.72188247,145.6457092)(123.71188248,145.50070935)(123.69189241,145.34071249)
\curveto(123.68188251,145.29070956)(123.67188252,145.23070962)(123.66189241,145.16071249)
\curveto(123.66188253,145.09070976)(123.67188252,145.03070982)(123.69189241,144.98071249)
\lineto(123.69189241,144.81571249)
\lineto(123.69189241,144.63571249)
\curveto(123.70188249,144.58571026)(123.71188248,144.53071032)(123.72189241,144.47071249)
\curveto(123.73188246,144.42071043)(123.73688245,144.36571048)(123.73689241,144.30571249)
\curveto(123.74688244,144.2457106)(123.76188243,144.19071066)(123.78189241,144.14071249)
\curveto(123.83188236,143.9507109)(123.8918823,143.77571107)(123.96189241,143.61571249)
\curveto(124.03188216,143.45571139)(124.13688205,143.32571152)(124.27689241,143.22571249)
\curveto(124.40688178,143.12571172)(124.54688164,143.05571179)(124.69689241,143.01571249)
\curveto(124.72688146,143.00571184)(124.75188144,143.00071185)(124.77189241,143.00071249)
\curveto(124.80188139,143.01071184)(124.83188136,143.01071184)(124.86189241,143.00071249)
\curveto(124.88188131,143.00071185)(124.91188128,142.99571185)(124.95189241,142.98571249)
\curveto(124.9918812,142.98571186)(125.02688116,142.99071186)(125.05689241,143.00071249)
\curveto(125.09688109,143.01071184)(125.13688105,143.01571183)(125.17689241,143.01571249)
\curveto(125.21688097,143.01571183)(125.25688093,143.02571182)(125.29689241,143.04571249)
\curveto(125.53688065,143.12571172)(125.73188046,143.26071159)(125.88189241,143.45071249)
\curveto(126.00188019,143.63071122)(126.0918801,143.83571101)(126.15189241,144.06571249)
\curveto(126.17188002,144.13571071)(126.18688,144.20571064)(126.19689241,144.27571249)
\curveto(126.20687998,144.35571049)(126.22187997,144.43571041)(126.24189241,144.51571249)
\curveto(126.24187995,144.57571027)(126.24687994,144.62071023)(126.25689241,144.65071249)
\curveto(126.25687993,144.67071018)(126.25687993,144.69571015)(126.25689241,144.72571249)
\curveto(126.25687993,144.76571008)(126.26187993,144.79571005)(126.27189241,144.81571249)
\lineto(126.27189241,144.96571249)
}
}
{
\newrgbcolor{curcolor}{0 0 0}
\pscustom[linestyle=none,fillstyle=solid,fillcolor=curcolor]
{
\newpath
\moveto(670.03721773,263.69000692)
\curveto(670.13721287,263.6899963)(670.23221278,263.67999631)(670.32221773,263.66000692)
\curveto(670.4122126,263.64999634)(670.47721253,263.61999637)(670.51721773,263.57000692)
\curveto(670.57721243,263.4899965)(670.6072124,263.38499661)(670.60721773,263.25500692)
\lineto(670.60721773,262.86500692)
\lineto(670.60721773,261.36500692)
\lineto(670.60721773,254.97500692)
\lineto(670.60721773,253.80500692)
\lineto(670.60721773,253.49000692)
\curveto(670.61721239,253.3900066)(670.60221241,253.31000668)(670.56221773,253.25000692)
\curveto(670.5122125,253.17000682)(670.43721257,253.12000687)(670.33721773,253.10000692)
\curveto(670.24721276,253.0900069)(670.13721287,253.08500691)(670.00721773,253.08500692)
\lineto(669.78221773,253.08500692)
\curveto(669.70221331,253.10500689)(669.63221338,253.12000687)(669.57221773,253.13000692)
\curveto(669.5122135,253.15000684)(669.46221355,253.1900068)(669.42221773,253.25000692)
\curveto(669.38221363,253.31000668)(669.36221365,253.38500661)(669.36221773,253.47500692)
\lineto(669.36221773,253.77500692)
\lineto(669.36221773,254.87000692)
\lineto(669.36221773,260.21000692)
\curveto(669.34221367,260.29999969)(669.32721368,260.37499962)(669.31721773,260.43500692)
\curveto(669.31721369,260.50499949)(669.28721372,260.56499943)(669.22721773,260.61500692)
\curveto(669.15721385,260.66499933)(669.06721394,260.6899993)(668.95721773,260.69000692)
\curveto(668.85721415,260.69999929)(668.74721426,260.70499929)(668.62721773,260.70500692)
\lineto(667.48721773,260.70500692)
\lineto(666.99221773,260.70500692)
\curveto(666.83221618,260.71499928)(666.72221629,260.77499922)(666.66221773,260.88500692)
\curveto(666.64221637,260.91499908)(666.63221638,260.94499905)(666.63221773,260.97500692)
\curveto(666.63221638,261.01499898)(666.62721638,261.05999893)(666.61721773,261.11000692)
\curveto(666.59721641,261.22999876)(666.60221641,261.33999865)(666.63221773,261.44000692)
\curveto(666.67221634,261.53999845)(666.72721628,261.60999838)(666.79721773,261.65000692)
\curveto(666.87721613,261.69999829)(666.99721601,261.72499827)(667.15721773,261.72500692)
\curveto(667.31721569,261.72499827)(667.45221556,261.73999825)(667.56221773,261.77000692)
\curveto(667.6122154,261.77999821)(667.66721534,261.78499821)(667.72721773,261.78500692)
\curveto(667.78721522,261.7949982)(667.84721516,261.80999818)(667.90721773,261.83000692)
\curveto(668.05721495,261.87999811)(668.20221481,261.92999806)(668.34221773,261.98000692)
\curveto(668.48221453,262.03999795)(668.61721439,262.10999788)(668.74721773,262.19000692)
\curveto(668.88721412,262.27999771)(669.007214,262.38499761)(669.10721773,262.50500692)
\curveto(669.2072138,262.62499737)(669.30221371,262.75499724)(669.39221773,262.89500692)
\curveto(669.45221356,262.994997)(669.49721351,263.10499689)(669.52721773,263.22500692)
\curveto(669.56721344,263.34499665)(669.61721339,263.44999654)(669.67721773,263.54000692)
\curveto(669.72721328,263.59999639)(669.79721321,263.63999635)(669.88721773,263.66000692)
\curveto(669.9072131,263.66999632)(669.93221308,263.67499632)(669.96221773,263.67500692)
\curveto(669.99221302,263.67499632)(670.01721299,263.67999631)(670.03721773,263.69000692)
}
}
{
\newrgbcolor{curcolor}{0 0 0}
\pscustom[linestyle=none,fillstyle=solid,fillcolor=curcolor]
{
\newpath
\moveto(677.4418271,263.69000692)
\curveto(679.07182166,263.71999627)(680.12182061,263.16499683)(680.5918271,262.02500692)
\curveto(680.69182004,261.7949982)(680.75681998,261.50499849)(680.7868271,261.15500692)
\curveto(680.82681991,260.81499918)(680.80181993,260.50499949)(680.7118271,260.22500692)
\curveto(680.62182011,259.96500003)(680.50182023,259.74000025)(680.3518271,259.55000692)
\curveto(680.3318204,259.51000048)(680.30682043,259.47500052)(680.2768271,259.44500692)
\curveto(680.24682049,259.42500057)(680.22182051,259.40000059)(680.2018271,259.37000692)
\lineto(680.1118271,259.25000692)
\curveto(680.08182065,259.22000077)(680.04682069,259.1950008)(680.0068271,259.17500692)
\curveto(679.95682078,259.12500087)(679.90182083,259.08000091)(679.8418271,259.04000692)
\curveto(679.79182094,259.00000099)(679.74682099,258.95000104)(679.7068271,258.89000692)
\curveto(679.66682107,258.85000114)(679.65182108,258.80000119)(679.6618271,258.74000692)
\curveto(679.67182106,258.6900013)(679.70182103,258.64500135)(679.7518271,258.60500692)
\curveto(679.80182093,258.56500143)(679.85682088,258.52500147)(679.9168271,258.48500692)
\curveto(679.98682075,258.45500154)(680.05182068,258.42500157)(680.1118271,258.39500692)
\curveto(680.17182056,258.36500163)(680.22182051,258.33500166)(680.2618271,258.30500692)
\curveto(680.58182015,258.08500191)(680.8368199,257.77500222)(681.0268271,257.37500692)
\curveto(681.06681967,257.28500271)(681.09681964,257.1900028)(681.1168271,257.09000692)
\curveto(681.14681959,257.00000299)(681.17181956,256.91000308)(681.1918271,256.82000692)
\curveto(681.20181953,256.77000322)(681.20681953,256.72000327)(681.2068271,256.67000692)
\curveto(681.21681952,256.63000336)(681.22681951,256.58500341)(681.2368271,256.53500692)
\curveto(681.24681949,256.48500351)(681.24681949,256.43500356)(681.2368271,256.38500692)
\curveto(681.22681951,256.33500366)(681.2318195,256.28500371)(681.2518271,256.23500692)
\curveto(681.26181947,256.18500381)(681.26681947,256.12500387)(681.2668271,256.05500692)
\curveto(681.26681947,255.98500401)(681.25681948,255.92500407)(681.2368271,255.87500692)
\lineto(681.2368271,255.65000692)
\lineto(681.1768271,255.41000692)
\curveto(681.16681957,255.34000465)(681.15181958,255.27000472)(681.1318271,255.20000692)
\curveto(681.10181963,255.11000488)(681.07181966,255.02500497)(681.0418271,254.94500692)
\curveto(681.02181971,254.86500513)(680.99181974,254.78500521)(680.9518271,254.70500692)
\curveto(680.9318198,254.64500535)(680.90181983,254.58500541)(680.8618271,254.52500692)
\curveto(680.8318199,254.47500552)(680.79681994,254.42500557)(680.7568271,254.37500692)
\curveto(680.55682018,254.06500593)(680.30682043,253.80500619)(680.0068271,253.59500692)
\curveto(679.70682103,253.3950066)(679.36182137,253.23000676)(678.9718271,253.10000692)
\curveto(678.85182188,253.06000693)(678.72182201,253.03500696)(678.5818271,253.02500692)
\curveto(678.45182228,253.00500699)(678.31682242,252.98000701)(678.1768271,252.95000692)
\curveto(678.10682263,252.94000705)(678.0368227,252.93500706)(677.9668271,252.93500692)
\curveto(677.90682283,252.93500706)(677.84182289,252.93000706)(677.7718271,252.92000692)
\curveto(677.731823,252.91000708)(677.67182306,252.90500709)(677.5918271,252.90500692)
\curveto(677.52182321,252.90500709)(677.47182326,252.91000708)(677.4418271,252.92000692)
\curveto(677.39182334,252.93000706)(677.34682339,252.93500706)(677.3068271,252.93500692)
\lineto(677.1868271,252.93500692)
\curveto(677.08682365,252.95500704)(676.98682375,252.97000702)(676.8868271,252.98000692)
\curveto(676.78682395,252.990007)(676.69182404,253.00500699)(676.6018271,253.02500692)
\curveto(676.49182424,253.05500694)(676.38182435,253.08000691)(676.2718271,253.10000692)
\curveto(676.17182456,253.13000686)(676.06682467,253.17000682)(675.9568271,253.22000692)
\curveto(675.58682515,253.38000661)(675.27182546,253.58000641)(675.0118271,253.82000692)
\curveto(674.75182598,254.07000592)(674.54182619,254.38000561)(674.3818271,254.75000692)
\curveto(674.34182639,254.84000515)(674.30682643,254.93500506)(674.2768271,255.03500692)
\curveto(674.24682649,255.13500486)(674.21682652,255.24000475)(674.1868271,255.35000692)
\curveto(674.16682657,255.40000459)(674.15682658,255.45000454)(674.1568271,255.50000692)
\curveto(674.15682658,255.56000443)(674.14682659,255.62000437)(674.1268271,255.68000692)
\curveto(674.10682663,255.74000425)(674.09682664,255.82000417)(674.0968271,255.92000692)
\curveto(674.09682664,256.02000397)(674.11182662,256.0950039)(674.1418271,256.14500692)
\curveto(674.15182658,256.17500382)(674.16682657,256.20000379)(674.1868271,256.22000692)
\lineto(674.2468271,256.28000692)
\curveto(674.28682645,256.30000369)(674.34682639,256.31500368)(674.4268271,256.32500692)
\curveto(674.51682622,256.33500366)(674.60682613,256.34000365)(674.6968271,256.34000692)
\curveto(674.78682595,256.34000365)(674.87182586,256.33500366)(674.9518271,256.32500692)
\curveto(675.04182569,256.31500368)(675.10682563,256.30500369)(675.1468271,256.29500692)
\curveto(675.16682557,256.27500372)(675.18682555,256.26000373)(675.2068271,256.25000692)
\curveto(675.22682551,256.25000374)(675.24682549,256.24000375)(675.2668271,256.22000692)
\curveto(675.3368254,256.13000386)(675.37682536,256.01500398)(675.3868271,255.87500692)
\curveto(675.40682533,255.73500426)(675.4368253,255.61000438)(675.4768271,255.50000692)
\lineto(675.6268271,255.14000692)
\curveto(675.67682506,255.03000496)(675.74182499,254.92500507)(675.8218271,254.82500692)
\curveto(675.84182489,254.7950052)(675.86182487,254.77000522)(675.8818271,254.75000692)
\curveto(675.91182482,254.73000526)(675.9368248,254.70500529)(675.9568271,254.67500692)
\curveto(675.99682474,254.61500538)(676.0318247,254.57000542)(676.0618271,254.54000692)
\curveto(676.10182463,254.51000548)(676.1368246,254.48000551)(676.1668271,254.45000692)
\curveto(676.20682453,254.42000557)(676.25182448,254.3900056)(676.3018271,254.36000692)
\curveto(676.39182434,254.30000569)(676.48682425,254.25000574)(676.5868271,254.21000692)
\lineto(676.9168271,254.09000692)
\curveto(677.06682367,254.04000595)(677.26682347,254.01000598)(677.5168271,254.00000692)
\curveto(677.76682297,253.990006)(677.97682276,254.01000598)(678.1468271,254.06000692)
\curveto(678.22682251,254.08000591)(678.29682244,254.0950059)(678.3568271,254.10500692)
\lineto(678.5668271,254.16500692)
\curveto(678.84682189,254.28500571)(679.08682165,254.43500556)(679.2868271,254.61500692)
\curveto(679.49682124,254.7950052)(679.66182107,255.02500497)(679.7818271,255.30500692)
\curveto(679.81182092,255.37500462)(679.8318209,255.44500455)(679.8418271,255.51500692)
\lineto(679.9018271,255.75500692)
\curveto(679.94182079,255.8950041)(679.95182078,256.05500394)(679.9318271,256.23500692)
\curveto(679.91182082,256.42500357)(679.88182085,256.57500342)(679.8418271,256.68500692)
\curveto(679.71182102,257.06500293)(679.52682121,257.35500264)(679.2868271,257.55500692)
\curveto(679.05682168,257.75500224)(678.74682199,257.91500208)(678.3568271,258.03500692)
\curveto(678.24682249,258.06500193)(678.12682261,258.08500191)(677.9968271,258.09500692)
\curveto(677.87682286,258.10500189)(677.75182298,258.11000188)(677.6218271,258.11000692)
\curveto(677.46182327,258.11000188)(677.32182341,258.11500188)(677.2018271,258.12500692)
\curveto(677.08182365,258.13500186)(676.99682374,258.1950018)(676.9468271,258.30500692)
\curveto(676.92682381,258.33500166)(676.91682382,258.37000162)(676.9168271,258.41000692)
\lineto(676.9168271,258.54500692)
\curveto(676.90682383,258.64500135)(676.90682383,258.74000125)(676.9168271,258.83000692)
\curveto(676.9368238,258.92000107)(676.97682376,258.98500101)(677.0368271,259.02500692)
\curveto(677.07682366,259.05500094)(677.11682362,259.07500092)(677.1568271,259.08500692)
\curveto(677.20682353,259.0950009)(677.26182347,259.10500089)(677.3218271,259.11500692)
\curveto(677.34182339,259.12500087)(677.36682337,259.12500087)(677.3968271,259.11500692)
\curveto(677.42682331,259.11500088)(677.45182328,259.12000087)(677.4718271,259.13000692)
\lineto(677.6068271,259.13000692)
\curveto(677.71682302,259.15000084)(677.81682292,259.16000083)(677.9068271,259.16000692)
\curveto(678.00682273,259.17000082)(678.10182263,259.1900008)(678.1918271,259.22000692)
\curveto(678.51182222,259.33000066)(678.76682197,259.47500052)(678.9568271,259.65500692)
\curveto(679.14682159,259.83500016)(679.29682144,260.08499991)(679.4068271,260.40500692)
\curveto(679.4368213,260.50499949)(679.45682128,260.62999936)(679.4668271,260.78000692)
\curveto(679.48682125,260.93999905)(679.48182125,261.08499891)(679.4518271,261.21500692)
\curveto(679.4318213,261.28499871)(679.41182132,261.34999864)(679.3918271,261.41000692)
\curveto(679.38182135,261.47999851)(679.36182137,261.54499845)(679.3318271,261.60500692)
\curveto(679.2318215,261.84499815)(679.08682165,262.03499796)(678.8968271,262.17500692)
\curveto(678.70682203,262.31499768)(678.48182225,262.42499757)(678.2218271,262.50500692)
\curveto(678.16182257,262.52499747)(678.10182263,262.53499746)(678.0418271,262.53500692)
\curveto(677.98182275,262.53499746)(677.91682282,262.54499745)(677.8468271,262.56500692)
\curveto(677.76682297,262.58499741)(677.67182306,262.5949974)(677.5618271,262.59500692)
\curveto(677.45182328,262.5949974)(677.35682338,262.58499741)(677.2768271,262.56500692)
\curveto(677.22682351,262.54499745)(677.17682356,262.53499746)(677.1268271,262.53500692)
\curveto(677.08682365,262.53499746)(677.04182369,262.52499747)(676.9918271,262.50500692)
\curveto(676.81182392,262.45499754)(676.64182409,262.37999761)(676.4818271,262.28000692)
\curveto(676.3318244,262.1899978)(676.20182453,262.07499792)(676.0918271,261.93500692)
\curveto(676.00182473,261.81499818)(675.92182481,261.68499831)(675.8518271,261.54500692)
\curveto(675.78182495,261.40499859)(675.71682502,261.24999874)(675.6568271,261.08000692)
\curveto(675.62682511,260.96999902)(675.60682513,260.84999914)(675.5968271,260.72000692)
\curveto(675.58682515,260.59999939)(675.55182518,260.49999949)(675.4918271,260.42000692)
\curveto(675.47182526,260.37999961)(675.41182532,260.33999965)(675.3118271,260.30000692)
\curveto(675.27182546,260.2899997)(675.21182552,260.27999971)(675.1318271,260.27000692)
\lineto(674.8768271,260.27000692)
\curveto(674.78682595,260.27999971)(674.70182603,260.2899997)(674.6218271,260.30000692)
\curveto(674.55182618,260.30999968)(674.50182623,260.32499967)(674.4718271,260.34500692)
\curveto(674.4318263,260.37499962)(674.39682634,260.42999956)(674.3668271,260.51000692)
\curveto(674.3368264,260.5899994)(674.3318264,260.67499932)(674.3518271,260.76500692)
\curveto(674.36182637,260.81499918)(674.36682637,260.86499913)(674.3668271,260.91500692)
\lineto(674.3968271,261.09500692)
\curveto(674.42682631,261.1949988)(674.45182628,261.2949987)(674.4718271,261.39500692)
\curveto(674.50182623,261.4949985)(674.5368262,261.58499841)(674.5768271,261.66500692)
\curveto(674.62682611,261.77499822)(674.67182606,261.87999811)(674.7118271,261.98000692)
\curveto(674.75182598,262.0899979)(674.80182593,262.1949978)(674.8618271,262.29500692)
\curveto(675.19182554,262.83499716)(675.66182507,263.22999676)(676.2718271,263.48000692)
\curveto(676.39182434,263.52999646)(676.51682422,263.56499643)(676.6468271,263.58500692)
\curveto(676.78682395,263.60499639)(676.92682381,263.62999636)(677.0668271,263.66000692)
\curveto(677.12682361,263.66999632)(677.18682355,263.67499632)(677.2468271,263.67500692)
\curveto(677.31682342,263.67499632)(677.38182335,263.67999631)(677.4418271,263.69000692)
}
}
{
\newrgbcolor{curcolor}{0 0 0}
\pscustom[linestyle=none,fillstyle=solid,fillcolor=curcolor]
{
\newpath
\moveto(692.48143648,261.60500692)
\curveto(692.28142618,261.31499868)(692.07142639,261.02999896)(691.85143648,260.75000692)
\curveto(691.64142682,260.46999952)(691.43642702,260.18499981)(691.23643648,259.89500692)
\curveto(690.63642782,259.04500095)(690.03142843,258.20500179)(689.42143648,257.37500692)
\curveto(688.81142965,256.55500344)(688.20643025,255.72000427)(687.60643648,254.87000692)
\lineto(687.09643648,254.15000692)
\lineto(686.58643648,253.46000692)
\curveto(686.50643195,253.35000664)(686.42643203,253.23500676)(686.34643648,253.11500692)
\curveto(686.26643219,252.995007)(686.17143229,252.90000709)(686.06143648,252.83000692)
\curveto(686.02143244,252.81000718)(685.9564325,252.7950072)(685.86643648,252.78500692)
\curveto(685.78643267,252.76500723)(685.69643276,252.75500724)(685.59643648,252.75500692)
\curveto(685.49643296,252.75500724)(685.40143306,252.76000723)(685.31143648,252.77000692)
\curveto(685.23143323,252.78000721)(685.17143329,252.80000719)(685.13143648,252.83000692)
\curveto(685.10143336,252.85000714)(685.07643338,252.88500711)(685.05643648,252.93500692)
\curveto(685.04643341,252.97500702)(685.05143341,253.02000697)(685.07143648,253.07000692)
\curveto(685.11143335,253.15000684)(685.1564333,253.22500677)(685.20643648,253.29500692)
\curveto(685.26643319,253.37500662)(685.32143314,253.45500654)(685.37143648,253.53500692)
\curveto(685.61143285,253.87500612)(685.8564326,254.21000578)(686.10643648,254.54000692)
\curveto(686.3564321,254.87000512)(686.59643186,255.20500479)(686.82643648,255.54500692)
\curveto(686.98643147,255.76500423)(687.14643131,255.98000401)(687.30643648,256.19000692)
\curveto(687.46643099,256.40000359)(687.62643083,256.61500338)(687.78643648,256.83500692)
\curveto(688.14643031,257.35500264)(688.51142995,257.86500213)(688.88143648,258.36500692)
\curveto(689.25142921,258.86500113)(689.62142884,259.37500062)(689.99143648,259.89500692)
\curveto(690.13142833,260.0949999)(690.27142819,260.2899997)(690.41143648,260.48000692)
\curveto(690.5614279,260.66999932)(690.70642775,260.86499913)(690.84643648,261.06500692)
\curveto(691.0564274,261.36499863)(691.27142719,261.66499833)(691.49143648,261.96500692)
\lineto(692.15143648,262.86500692)
\lineto(692.33143648,263.13500692)
\lineto(692.54143648,263.40500692)
\lineto(692.66143648,263.58500692)
\curveto(692.71142575,263.64499635)(692.7614257,263.69999629)(692.81143648,263.75000692)
\curveto(692.88142558,263.79999619)(692.9564255,263.83499616)(693.03643648,263.85500692)
\curveto(693.0564254,263.86499613)(693.08142538,263.86499613)(693.11143648,263.85500692)
\curveto(693.15142531,263.85499614)(693.18142528,263.86499613)(693.20143648,263.88500692)
\curveto(693.32142514,263.88499611)(693.456425,263.87999611)(693.60643648,263.87000692)
\curveto(693.7564247,263.86999612)(693.84642461,263.82499617)(693.87643648,263.73500692)
\curveto(693.89642456,263.70499629)(693.90142456,263.66999632)(693.89143648,263.63000692)
\curveto(693.88142458,263.5899964)(693.86642459,263.55999643)(693.84643648,263.54000692)
\curveto(693.80642465,263.45999653)(693.76642469,263.3899966)(693.72643648,263.33000692)
\curveto(693.68642477,263.26999672)(693.64142482,263.20999678)(693.59143648,263.15000692)
\lineto(693.02143648,262.37000692)
\curveto(692.84142562,262.11999787)(692.6614258,261.86499813)(692.48143648,261.60500692)
\moveto(685.62643648,257.70500692)
\curveto(685.57643288,257.72500227)(685.52643293,257.73000226)(685.47643648,257.72000692)
\curveto(685.42643303,257.71000228)(685.37643308,257.71500228)(685.32643648,257.73500692)
\curveto(685.21643324,257.75500224)(685.11143335,257.77500222)(685.01143648,257.79500692)
\curveto(684.92143354,257.82500217)(684.82643363,257.86500213)(684.72643648,257.91500692)
\curveto(684.39643406,258.05500194)(684.14143432,258.25000174)(683.96143648,258.50000692)
\curveto(683.78143468,258.76000123)(683.63643482,259.07000092)(683.52643648,259.43000692)
\curveto(683.49643496,259.51000048)(683.47643498,259.5900004)(683.46643648,259.67000692)
\curveto(683.456435,259.76000023)(683.44143502,259.84500015)(683.42143648,259.92500692)
\curveto(683.41143505,259.97500002)(683.40643505,260.03999995)(683.40643648,260.12000692)
\curveto(683.39643506,260.14999984)(683.39143507,260.17999981)(683.39143648,260.21000692)
\curveto(683.39143507,260.24999974)(683.38643507,260.28499971)(683.37643648,260.31500692)
\lineto(683.37643648,260.46500692)
\curveto(683.36643509,260.51499948)(683.3614351,260.57499942)(683.36143648,260.64500692)
\curveto(683.3614351,260.72499927)(683.36643509,260.7899992)(683.37643648,260.84000692)
\lineto(683.37643648,261.00500692)
\curveto(683.39643506,261.05499894)(683.40143506,261.09999889)(683.39143648,261.14000692)
\curveto(683.39143507,261.1899988)(683.39643506,261.23499876)(683.40643648,261.27500692)
\curveto(683.41643504,261.31499868)(683.42143504,261.34999864)(683.42143648,261.38000692)
\curveto(683.42143504,261.41999857)(683.42643503,261.45999853)(683.43643648,261.50000692)
\curveto(683.46643499,261.60999838)(683.48643497,261.71999827)(683.49643648,261.83000692)
\curveto(683.51643494,261.94999804)(683.55143491,262.06499793)(683.60143648,262.17500692)
\curveto(683.74143472,262.51499748)(683.90143456,262.7899972)(684.08143648,263.00000692)
\curveto(684.27143419,263.21999677)(684.54143392,263.39999659)(684.89143648,263.54000692)
\curveto(684.97143349,263.56999642)(685.0564334,263.5899964)(685.14643648,263.60000692)
\curveto(685.23643322,263.61999637)(685.33143313,263.63999635)(685.43143648,263.66000692)
\curveto(685.461433,263.66999632)(685.51643294,263.66999632)(685.59643648,263.66000692)
\curveto(685.67643278,263.65999633)(685.72643273,263.66999632)(685.74643648,263.69000692)
\curveto(686.30643215,263.69999629)(686.7564317,263.5899964)(687.09643648,263.36000692)
\curveto(687.44643101,263.12999686)(687.70643075,262.82499717)(687.87643648,262.44500692)
\curveto(687.91643054,262.35499764)(687.95143051,262.25999773)(687.98143648,262.16000692)
\curveto(688.01143045,262.05999793)(688.03643042,261.95999803)(688.05643648,261.86000692)
\curveto(688.07643038,261.82999816)(688.08143038,261.79999819)(688.07143648,261.77000692)
\curveto(688.07143039,261.73999825)(688.07643038,261.70999828)(688.08643648,261.68000692)
\curveto(688.11643034,261.56999842)(688.13643032,261.44499855)(688.14643648,261.30500692)
\curveto(688.1564303,261.17499882)(688.16643029,261.03999895)(688.17643648,260.90000692)
\lineto(688.17643648,260.73500692)
\curveto(688.18643027,260.67499932)(688.18643027,260.61999937)(688.17643648,260.57000692)
\curveto(688.16643029,260.51999947)(688.1614303,260.46999952)(688.16143648,260.42000692)
\lineto(688.16143648,260.28500692)
\curveto(688.15143031,260.24499975)(688.14643031,260.20499979)(688.14643648,260.16500692)
\curveto(688.1564303,260.12499987)(688.15143031,260.07999991)(688.13143648,260.03000692)
\curveto(688.11143035,259.92000007)(688.09143037,259.81500018)(688.07143648,259.71500692)
\curveto(688.0614304,259.61500038)(688.04143042,259.51500048)(688.01143648,259.41500692)
\curveto(687.88143058,259.05500094)(687.71643074,258.74000125)(687.51643648,258.47000692)
\curveto(687.31643114,258.20000179)(687.04143142,257.995002)(686.69143648,257.85500692)
\curveto(686.61143185,257.82500217)(686.52643193,257.80000219)(686.43643648,257.78000692)
\lineto(686.16643648,257.72000692)
\curveto(686.11643234,257.71000228)(686.07143239,257.70500229)(686.03143648,257.70500692)
\curveto(685.99143247,257.71500228)(685.95143251,257.71500228)(685.91143648,257.70500692)
\curveto(685.81143265,257.68500231)(685.71643274,257.68500231)(685.62643648,257.70500692)
\moveto(684.78643648,259.10000692)
\curveto(684.82643363,259.03000096)(684.86643359,258.96500103)(684.90643648,258.90500692)
\curveto(684.94643351,258.85500114)(684.99643346,258.80500119)(685.05643648,258.75500692)
\lineto(685.20643648,258.63500692)
\curveto(685.26643319,258.60500139)(685.33143313,258.58000141)(685.40143648,258.56000692)
\curveto(685.44143302,258.54000145)(685.47643298,258.53000146)(685.50643648,258.53000692)
\curveto(685.54643291,258.54000145)(685.58643287,258.53500146)(685.62643648,258.51500692)
\curveto(685.6564328,258.51500148)(685.69643276,258.51000148)(685.74643648,258.50000692)
\curveto(685.79643266,258.50000149)(685.83643262,258.50500149)(685.86643648,258.51500692)
\lineto(686.09143648,258.56000692)
\curveto(686.34143212,258.64000135)(686.52643193,258.76500123)(686.64643648,258.93500692)
\curveto(686.72643173,259.03500096)(686.79643166,259.16500083)(686.85643648,259.32500692)
\curveto(686.93643152,259.50500049)(686.99643146,259.73000026)(687.03643648,260.00000692)
\curveto(687.07643138,260.27999971)(687.09143137,260.55999943)(687.08143648,260.84000692)
\curveto(687.07143139,261.12999886)(687.04143142,261.40499859)(686.99143648,261.66500692)
\curveto(686.94143152,261.92499807)(686.86643159,262.13499786)(686.76643648,262.29500692)
\curveto(686.64643181,262.4949975)(686.49643196,262.64499735)(686.31643648,262.74500692)
\curveto(686.23643222,262.7949972)(686.14643231,262.82499717)(686.04643648,262.83500692)
\curveto(685.94643251,262.85499714)(685.84143262,262.86499713)(685.73143648,262.86500692)
\curveto(685.71143275,262.85499714)(685.68643277,262.84999714)(685.65643648,262.85000692)
\curveto(685.63643282,262.85999713)(685.61643284,262.85999713)(685.59643648,262.85000692)
\curveto(685.54643291,262.83999715)(685.50143296,262.82999716)(685.46143648,262.82000692)
\curveto(685.42143304,262.81999717)(685.38143308,262.80999718)(685.34143648,262.79000692)
\curveto(685.1614333,262.70999728)(685.01143345,262.5899974)(684.89143648,262.43000692)
\curveto(684.78143368,262.26999772)(684.69143377,262.0899979)(684.62143648,261.89000692)
\curveto(684.5614339,261.69999829)(684.51643394,261.47499852)(684.48643648,261.21500692)
\curveto(684.46643399,260.95499904)(684.461434,260.6899993)(684.47143648,260.42000692)
\curveto(684.48143398,260.15999983)(684.51143395,259.91000008)(684.56143648,259.67000692)
\curveto(684.62143384,259.44000055)(684.69643376,259.25000074)(684.78643648,259.10000692)
\moveto(695.58643648,256.11500692)
\curveto(695.59642286,256.06500393)(695.60142286,255.97500402)(695.60143648,255.84500692)
\curveto(695.60142286,255.71500428)(695.59142287,255.62500437)(695.57143648,255.57500692)
\curveto(695.55142291,255.52500447)(695.54642291,255.47000452)(695.55643648,255.41000692)
\curveto(695.56642289,255.36000463)(695.56642289,255.31000468)(695.55643648,255.26000692)
\curveto(695.51642294,255.12000487)(695.48642297,254.98500501)(695.46643648,254.85500692)
\curveto(695.456423,254.72500527)(695.42642303,254.60500539)(695.37643648,254.49500692)
\curveto(695.23642322,254.14500585)(695.07142339,253.85000614)(694.88143648,253.61000692)
\curveto(694.69142377,253.38000661)(694.42142404,253.1950068)(694.07143648,253.05500692)
\curveto(693.99142447,253.02500697)(693.90642455,253.00500699)(693.81643648,252.99500692)
\curveto(693.72642473,252.97500702)(693.64142482,252.95500704)(693.56143648,252.93500692)
\curveto(693.51142495,252.92500707)(693.461425,252.92000707)(693.41143648,252.92000692)
\curveto(693.3614251,252.92000707)(693.31142515,252.91500708)(693.26143648,252.90500692)
\curveto(693.23142523,252.8950071)(693.18142528,252.8950071)(693.11143648,252.90500692)
\curveto(693.04142542,252.90500709)(692.99142547,252.91000708)(692.96143648,252.92000692)
\curveto(692.90142556,252.94000705)(692.84142562,252.95000704)(692.78143648,252.95000692)
\curveto(692.73142573,252.94000705)(692.68142578,252.94500705)(692.63143648,252.96500692)
\curveto(692.54142592,252.98500701)(692.45142601,253.01000698)(692.36143648,253.04000692)
\curveto(692.28142618,253.06000693)(692.20142626,253.0900069)(692.12143648,253.13000692)
\curveto(691.80142666,253.27000672)(691.55142691,253.46500653)(691.37143648,253.71500692)
\curveto(691.19142727,253.97500602)(691.04142742,254.28000571)(690.92143648,254.63000692)
\curveto(690.90142756,254.71000528)(690.88642757,254.7950052)(690.87643648,254.88500692)
\curveto(690.86642759,254.97500502)(690.85142761,255.06000493)(690.83143648,255.14000692)
\curveto(690.82142764,255.17000482)(690.81642764,255.20000479)(690.81643648,255.23000692)
\lineto(690.81643648,255.33500692)
\curveto(690.79642766,255.41500458)(690.78642767,255.4950045)(690.78643648,255.57500692)
\lineto(690.78643648,255.71000692)
\curveto(690.76642769,255.81000418)(690.76642769,255.91000408)(690.78643648,256.01000692)
\lineto(690.78643648,256.19000692)
\curveto(690.79642766,256.24000375)(690.80142766,256.28500371)(690.80143648,256.32500692)
\curveto(690.80142766,256.37500362)(690.80642765,256.42000357)(690.81643648,256.46000692)
\curveto(690.82642763,256.50000349)(690.83142763,256.53500346)(690.83143648,256.56500692)
\curveto(690.83142763,256.60500339)(690.83642762,256.64500335)(690.84643648,256.68500692)
\lineto(690.90643648,257.01500692)
\curveto(690.92642753,257.13500286)(690.9564275,257.24500275)(690.99643648,257.34500692)
\curveto(691.13642732,257.67500232)(691.29642716,257.95000204)(691.47643648,258.17000692)
\curveto(691.66642679,258.40000159)(691.92642653,258.58500141)(692.25643648,258.72500692)
\curveto(692.33642612,258.76500123)(692.42142604,258.7900012)(692.51143648,258.80000692)
\lineto(692.81143648,258.86000692)
\lineto(692.94643648,258.86000692)
\curveto(692.99642546,258.87000112)(693.04642541,258.87500112)(693.09643648,258.87500692)
\curveto(693.66642479,258.8950011)(694.12642433,258.7900012)(694.47643648,258.56000692)
\curveto(694.83642362,258.34000165)(695.10142336,258.04000195)(695.27143648,257.66000692)
\curveto(695.32142314,257.56000243)(695.3614231,257.46000253)(695.39143648,257.36000692)
\curveto(695.42142304,257.26000273)(695.45142301,257.15500284)(695.48143648,257.04500692)
\curveto(695.49142297,257.00500299)(695.49642296,256.97000302)(695.49643648,256.94000692)
\curveto(695.49642296,256.92000307)(695.50142296,256.8900031)(695.51143648,256.85000692)
\curveto(695.53142293,256.78000321)(695.54142292,256.70500329)(695.54143648,256.62500692)
\curveto(695.54142292,256.54500345)(695.55142291,256.46500353)(695.57143648,256.38500692)
\curveto(695.57142289,256.33500366)(695.57142289,256.2900037)(695.57143648,256.25000692)
\curveto(695.57142289,256.21000378)(695.57642288,256.16500383)(695.58643648,256.11500692)
\moveto(694.47643648,255.68000692)
\curveto(694.48642397,255.73000426)(694.49142397,255.80500419)(694.49143648,255.90500692)
\curveto(694.50142396,256.00500399)(694.49642396,256.08000391)(694.47643648,256.13000692)
\curveto(694.456424,256.1900038)(694.45142401,256.24500375)(694.46143648,256.29500692)
\curveto(694.48142398,256.35500364)(694.48142398,256.41500358)(694.46143648,256.47500692)
\curveto(694.45142401,256.50500349)(694.44642401,256.54000345)(694.44643648,256.58000692)
\curveto(694.44642401,256.62000337)(694.44142402,256.66000333)(694.43143648,256.70000692)
\curveto(694.41142405,256.78000321)(694.39142407,256.85500314)(694.37143648,256.92500692)
\curveto(694.3614241,257.00500299)(694.34642411,257.08500291)(694.32643648,257.16500692)
\curveto(694.29642416,257.22500277)(694.27142419,257.28500271)(694.25143648,257.34500692)
\curveto(694.23142423,257.40500259)(694.20142426,257.46500253)(694.16143648,257.52500692)
\curveto(694.0614244,257.6950023)(693.93142453,257.83000216)(693.77143648,257.93000692)
\curveto(693.69142477,257.98000201)(693.59642486,258.01500198)(693.48643648,258.03500692)
\curveto(693.37642508,258.05500194)(693.25142521,258.06500193)(693.11143648,258.06500692)
\curveto(693.09142537,258.05500194)(693.06642539,258.05000194)(693.03643648,258.05000692)
\curveto(693.00642545,258.06000193)(692.97642548,258.06000193)(692.94643648,258.05000692)
\lineto(692.79643648,257.99000692)
\curveto(692.74642571,257.98000201)(692.70142576,257.96500203)(692.66143648,257.94500692)
\curveto(692.47142599,257.83500216)(692.32642613,257.6900023)(692.22643648,257.51000692)
\curveto(692.13642632,257.33000266)(692.0564264,257.12500287)(691.98643648,256.89500692)
\curveto(691.94642651,256.76500323)(691.92642653,256.63000336)(691.92643648,256.49000692)
\curveto(691.92642653,256.36000363)(691.91642654,256.21500378)(691.89643648,256.05500692)
\curveto(691.88642657,256.00500399)(691.87642658,255.94500405)(691.86643648,255.87500692)
\curveto(691.86642659,255.80500419)(691.87642658,255.74500425)(691.89643648,255.69500692)
\lineto(691.89643648,255.53000692)
\lineto(691.89643648,255.35000692)
\curveto(691.90642655,255.30000469)(691.91642654,255.24500475)(691.92643648,255.18500692)
\curveto(691.93642652,255.13500486)(691.94142652,255.08000491)(691.94143648,255.02000692)
\curveto(691.95142651,254.96000503)(691.96642649,254.90500509)(691.98643648,254.85500692)
\curveto(692.03642642,254.66500533)(692.09642636,254.4900055)(692.16643648,254.33000692)
\curveto(692.23642622,254.17000582)(692.34142612,254.04000595)(692.48143648,253.94000692)
\curveto(692.61142585,253.84000615)(692.75142571,253.77000622)(692.90143648,253.73000692)
\curveto(692.93142553,253.72000627)(692.9564255,253.71500628)(692.97643648,253.71500692)
\curveto(693.00642545,253.72500627)(693.03642542,253.72500627)(693.06643648,253.71500692)
\curveto(693.08642537,253.71500628)(693.11642534,253.71000628)(693.15643648,253.70000692)
\curveto(693.19642526,253.70000629)(693.23142523,253.70500629)(693.26143648,253.71500692)
\curveto(693.30142516,253.72500627)(693.34142512,253.73000626)(693.38143648,253.73000692)
\curveto(693.42142504,253.73000626)(693.461425,253.74000625)(693.50143648,253.76000692)
\curveto(693.74142472,253.84000615)(693.93642452,253.97500602)(694.08643648,254.16500692)
\curveto(694.20642425,254.34500565)(694.29642416,254.55000544)(694.35643648,254.78000692)
\curveto(694.37642408,254.85000514)(694.39142407,254.92000507)(694.40143648,254.99000692)
\curveto(694.41142405,255.07000492)(694.42642403,255.15000484)(694.44643648,255.23000692)
\curveto(694.44642401,255.2900047)(694.45142401,255.33500466)(694.46143648,255.36500692)
\curveto(694.461424,255.38500461)(694.461424,255.41000458)(694.46143648,255.44000692)
\curveto(694.461424,255.48000451)(694.46642399,255.51000448)(694.47643648,255.53000692)
\lineto(694.47643648,255.68000692)
}
}
{
\newrgbcolor{curcolor}{0 0 0}
\pscustom[linestyle=none,fillstyle=solid,fillcolor=curcolor]
{
\newpath
\moveto(529.4208066,62.58859579)
\curveto(529.43079888,62.54859274)(529.43079888,62.49859279)(529.4208066,62.43859579)
\curveto(529.42079889,62.37859291)(529.41579889,62.32859296)(529.4058066,62.28859579)
\curveto(529.4057989,62.24859304)(529.40079891,62.20859308)(529.3908066,62.16859579)
\lineto(529.3908066,62.06359579)
\curveto(529.37079894,61.98359331)(529.35579895,61.90359339)(529.3458066,61.82359579)
\curveto(529.33579897,61.74359355)(529.31579899,61.66859362)(529.2858066,61.59859579)
\curveto(529.26579904,61.51859377)(529.24579906,61.44359385)(529.2258066,61.37359579)
\curveto(529.2057991,61.30359399)(529.17579913,61.22859406)(529.1358066,61.14859579)
\curveto(528.95579935,60.72859456)(528.70079961,60.3885949)(528.3708066,60.12859579)
\curveto(528.04080027,59.86859542)(527.65080066,59.66359563)(527.2008066,59.51359579)
\curveto(527.08080123,59.47359582)(526.95580135,59.44859584)(526.8258066,59.43859579)
\curveto(526.7058016,59.41859587)(526.58080173,59.3935959)(526.4508066,59.36359579)
\curveto(526.39080192,59.35359594)(526.32580198,59.34859594)(526.2558066,59.34859579)
\curveto(526.19580211,59.34859594)(526.13080218,59.34359595)(526.0608066,59.33359579)
\lineto(525.9408066,59.33359579)
\lineto(525.7458066,59.33359579)
\curveto(525.68580262,59.32359597)(525.63080268,59.32859596)(525.5808066,59.34859579)
\curveto(525.5108028,59.36859592)(525.44580286,59.37359592)(525.3858066,59.36359579)
\curveto(525.32580298,59.35359594)(525.26580304,59.35859593)(525.2058066,59.37859579)
\curveto(525.15580315,59.3885959)(525.1108032,59.3935959)(525.0708066,59.39359579)
\curveto(525.03080328,59.3935959)(524.98580332,59.40359589)(524.9358066,59.42359579)
\curveto(524.85580345,59.44359585)(524.78080353,59.46359583)(524.7108066,59.48359579)
\curveto(524.64080367,59.4935958)(524.57080374,59.50859578)(524.5008066,59.52859579)
\curveto(524.02080429,59.69859559)(523.62080469,59.90859538)(523.3008066,60.15859579)
\curveto(522.99080532,60.41859487)(522.74080557,60.77359452)(522.5508066,61.22359579)
\curveto(522.52080579,61.28359401)(522.49580581,61.34359395)(522.4758066,61.40359579)
\curveto(522.46580584,61.47359382)(522.45080586,61.54859374)(522.4308066,61.62859579)
\curveto(522.4108059,61.6885936)(522.39580591,61.75359354)(522.3858066,61.82359579)
\curveto(522.37580593,61.8935934)(522.36080595,61.96359333)(522.3408066,62.03359579)
\curveto(522.33080598,62.08359321)(522.32580598,62.12359317)(522.3258066,62.15359579)
\lineto(522.3258066,62.27359579)
\curveto(522.31580599,62.31359298)(522.305806,62.36359293)(522.2958066,62.42359579)
\curveto(522.29580601,62.48359281)(522.30080601,62.53359276)(522.3108066,62.57359579)
\lineto(522.3108066,62.70859579)
\curveto(522.32080599,62.75859253)(522.32580598,62.80859248)(522.3258066,62.85859579)
\curveto(522.34580596,62.95859233)(522.36080595,63.05359224)(522.3708066,63.14359579)
\curveto(522.38080593,63.24359205)(522.40080591,63.33859195)(522.4308066,63.42859579)
\curveto(522.48080583,63.57859171)(522.53580577,63.71859157)(522.5958066,63.84859579)
\curveto(522.65580565,63.97859131)(522.72580558,64.09859119)(522.8058066,64.20859579)
\curveto(522.83580547,64.25859103)(522.86580544,64.29859099)(522.8958066,64.32859579)
\curveto(522.93580537,64.35859093)(522.97080534,64.3935909)(523.0008066,64.43359579)
\curveto(523.06080525,64.51359078)(523.13080518,64.58359071)(523.2108066,64.64359579)
\curveto(523.27080504,64.6935906)(523.33080498,64.73859055)(523.3908066,64.77859579)
\lineto(523.6008066,64.92859579)
\curveto(523.65080466,64.96859032)(523.70080461,65.00359029)(523.7508066,65.03359579)
\curveto(523.80080451,65.07359022)(523.83580447,65.12859016)(523.8558066,65.19859579)
\curveto(523.85580445,65.22859006)(523.84580446,65.25359004)(523.8258066,65.27359579)
\curveto(523.81580449,65.30358999)(523.8058045,65.32858996)(523.7958066,65.34859579)
\curveto(523.75580455,65.39858989)(523.7058046,65.44358985)(523.6458066,65.48359579)
\curveto(523.59580471,65.53358976)(523.54580476,65.57858971)(523.4958066,65.61859579)
\curveto(523.45580485,65.64858964)(523.4058049,65.70358959)(523.3458066,65.78359579)
\curveto(523.32580498,65.81358948)(523.29580501,65.83858945)(523.2558066,65.85859579)
\curveto(523.22580508,65.8885894)(523.20080511,65.92358937)(523.1808066,65.96359579)
\curveto(523.0108053,66.17358912)(522.88080543,66.41858887)(522.7908066,66.69859579)
\curveto(522.77080554,66.77858851)(522.75580555,66.85858843)(522.7458066,66.93859579)
\curveto(522.73580557,67.01858827)(522.72080559,67.09858819)(522.7008066,67.17859579)
\curveto(522.68080563,67.22858806)(522.67080564,67.293588)(522.6708066,67.37359579)
\curveto(522.67080564,67.46358783)(522.68080563,67.53358776)(522.7008066,67.58359579)
\curveto(522.70080561,67.68358761)(522.7058056,67.75358754)(522.7158066,67.79359579)
\curveto(522.73580557,67.87358742)(522.75080556,67.94358735)(522.7608066,68.00359579)
\curveto(522.77080554,68.07358722)(522.78580552,68.14358715)(522.8058066,68.21359579)
\curveto(522.95580535,68.64358665)(523.17080514,68.9885863)(523.4508066,69.24859579)
\curveto(523.74080457,69.50858578)(524.09080422,69.72358557)(524.5008066,69.89359579)
\curveto(524.6108037,69.94358535)(524.72580358,69.97358532)(524.8458066,69.98359579)
\curveto(524.97580333,70.00358529)(525.1058032,70.03358526)(525.2358066,70.07359579)
\curveto(525.31580299,70.07358522)(525.38580292,70.07358522)(525.4458066,70.07359579)
\curveto(525.51580279,70.08358521)(525.59080272,70.0935852)(525.6708066,70.10359579)
\curveto(526.46080185,70.12358517)(527.11580119,69.9935853)(527.6358066,69.71359579)
\curveto(528.16580014,69.43358586)(528.54579976,69.02358627)(528.7758066,68.48359579)
\curveto(528.88579942,68.25358704)(528.95579935,67.96858732)(528.9858066,67.62859579)
\curveto(529.02579928,67.29858799)(528.99579931,66.9935883)(528.8958066,66.71359579)
\curveto(528.85579945,66.58358871)(528.8057995,66.46358883)(528.7458066,66.35359579)
\curveto(528.69579961,66.24358905)(528.63579967,66.13858915)(528.5658066,66.03859579)
\curveto(528.54579976,65.99858929)(528.51579979,65.96358933)(528.4758066,65.93359579)
\lineto(528.3858066,65.84359579)
\curveto(528.33579997,65.75358954)(528.27580003,65.6885896)(528.2058066,65.64859579)
\curveto(528.15580015,65.59858969)(528.10080021,65.54858974)(528.0408066,65.49859579)
\curveto(527.99080032,65.45858983)(527.94580036,65.41358988)(527.9058066,65.36359579)
\curveto(527.88580042,65.34358995)(527.86580044,65.31858997)(527.8458066,65.28859579)
\curveto(527.83580047,65.26859002)(527.83580047,65.24359005)(527.8458066,65.21359579)
\curveto(527.85580045,65.16359013)(527.88580042,65.11359018)(527.9358066,65.06359579)
\curveto(527.98580032,65.02359027)(528.04080027,64.98359031)(528.1008066,64.94359579)
\lineto(528.2808066,64.82359579)
\curveto(528.34079997,64.7935905)(528.39079992,64.76359053)(528.4308066,64.73359579)
\curveto(528.76079955,64.4935908)(529.0107993,64.18359111)(529.1808066,63.80359579)
\curveto(529.22079909,63.72359157)(529.25079906,63.63859165)(529.2708066,63.54859579)
\curveto(529.30079901,63.45859183)(529.32579898,63.36859192)(529.3458066,63.27859579)
\curveto(529.35579895,63.22859206)(529.36579894,63.17359212)(529.3758066,63.11359579)
\lineto(529.4058066,62.96359579)
\curveto(529.41579889,62.90359239)(529.41579889,62.83859245)(529.4058066,62.76859579)
\curveto(529.39579891,62.70859258)(529.40079891,62.64859264)(529.4208066,62.58859579)
\moveto(524.0358066,67.62859579)
\curveto(524.0058043,67.51858777)(524.00080431,67.37858791)(524.0208066,67.20859579)
\curveto(524.04080427,67.04858824)(524.06580424,66.92358837)(524.0958066,66.83359579)
\curveto(524.2058041,66.51358878)(524.35580395,66.26858902)(524.5458066,66.09859579)
\curveto(524.73580357,65.93858935)(525.00080331,65.80858948)(525.3408066,65.70859579)
\curveto(525.47080284,65.67858961)(525.63580267,65.65358964)(525.8358066,65.63359579)
\curveto(526.03580227,65.62358967)(526.2058021,65.63858965)(526.3458066,65.67859579)
\curveto(526.63580167,65.75858953)(526.87580143,65.86858942)(527.0658066,66.00859579)
\curveto(527.26580104,66.15858913)(527.42080089,66.35858893)(527.5308066,66.60859579)
\curveto(527.55080076,66.65858863)(527.56080075,66.70358859)(527.5608066,66.74359579)
\curveto(527.57080074,66.78358851)(527.58580072,66.82858846)(527.6058066,66.87859579)
\curveto(527.63580067,66.9885883)(527.65580065,67.12858816)(527.6658066,67.29859579)
\curveto(527.67580063,67.46858782)(527.66580064,67.61358768)(527.6358066,67.73359579)
\curveto(527.61580069,67.82358747)(527.59080072,67.90858738)(527.5608066,67.98859579)
\curveto(527.54080077,68.06858722)(527.5058008,68.14858714)(527.4558066,68.22859579)
\curveto(527.28580102,68.49858679)(527.06080125,68.6935866)(526.7808066,68.81359579)
\curveto(526.5108018,68.93358636)(526.15080216,68.9935863)(525.7008066,68.99359579)
\curveto(525.68080263,68.97358632)(525.65080266,68.96858632)(525.6108066,68.97859579)
\curveto(525.57080274,68.9885863)(525.53580277,68.9885863)(525.5058066,68.97859579)
\curveto(525.45580285,68.95858633)(525.40080291,68.94358635)(525.3408066,68.93359579)
\curveto(525.29080302,68.93358636)(525.24080307,68.92358637)(525.1908066,68.90359579)
\curveto(524.95080336,68.81358648)(524.74080357,68.69858659)(524.5608066,68.55859579)
\curveto(524.38080393,68.42858686)(524.24080407,68.24858704)(524.1408066,68.01859579)
\curveto(524.12080419,67.95858733)(524.10080421,67.8935874)(524.0808066,67.82359579)
\curveto(524.07080424,67.76358753)(524.05580425,67.69858759)(524.0358066,67.62859579)
\moveto(528.0558066,62.09359579)
\curveto(528.1058002,62.28359301)(528.1108002,62.4885928)(528.0708066,62.70859579)
\curveto(528.04080027,62.92859236)(527.99580031,63.10859218)(527.9358066,63.24859579)
\curveto(527.76580054,63.61859167)(527.5058008,63.92359137)(527.1558066,64.16359579)
\curveto(526.81580149,64.40359089)(526.38080193,64.52359077)(525.8508066,64.52359579)
\curveto(525.82080249,64.50359079)(525.78080253,64.49859079)(525.7308066,64.50859579)
\curveto(525.68080263,64.52859076)(525.64080267,64.53359076)(525.6108066,64.52359579)
\lineto(525.3408066,64.46359579)
\curveto(525.26080305,64.45359084)(525.18080313,64.43859085)(525.1008066,64.41859579)
\curveto(524.80080351,64.30859098)(524.53580377,64.16359113)(524.3058066,63.98359579)
\curveto(524.08580422,63.80359149)(523.91580439,63.57359172)(523.7958066,63.29359579)
\curveto(523.76580454,63.21359208)(523.74080457,63.13359216)(523.7208066,63.05359579)
\curveto(523.70080461,62.97359232)(523.68080463,62.8885924)(523.6608066,62.79859579)
\curveto(523.63080468,62.67859261)(523.62080469,62.52859276)(523.6308066,62.34859579)
\curveto(523.65080466,62.16859312)(523.67580463,62.02859326)(523.7058066,61.92859579)
\curveto(523.72580458,61.87859341)(523.73580457,61.83359346)(523.7358066,61.79359579)
\curveto(523.74580456,61.76359353)(523.76080455,61.72359357)(523.7808066,61.67359579)
\curveto(523.88080443,61.45359384)(524.0108043,61.25359404)(524.1708066,61.07359579)
\curveto(524.34080397,60.8935944)(524.53580377,60.75859453)(524.7558066,60.66859579)
\curveto(524.82580348,60.62859466)(524.92080339,60.5935947)(525.0408066,60.56359579)
\curveto(525.26080305,60.47359482)(525.51580279,60.42859486)(525.8058066,60.42859579)
\lineto(526.0908066,60.42859579)
\curveto(526.19080212,60.44859484)(526.28580202,60.46359483)(526.3758066,60.47359579)
\curveto(526.46580184,60.48359481)(526.55580175,60.50359479)(526.6458066,60.53359579)
\curveto(526.9058014,60.61359468)(527.14580116,60.74359455)(527.3658066,60.92359579)
\curveto(527.59580071,61.11359418)(527.76580054,61.32859396)(527.8758066,61.56859579)
\curveto(527.91580039,61.64859364)(527.94580036,61.72859356)(527.9658066,61.80859579)
\curveto(527.99580031,61.89859339)(528.02580028,61.9935933)(528.0558066,62.09359579)
}
}
{
\newrgbcolor{curcolor}{0 0 0}
\pscustom[linestyle=none,fillstyle=solid,fillcolor=curcolor]
{
\newpath
\moveto(534.00541597,70.11859579)
\curveto(534.69541134,70.12858516)(535.29541074,70.00858528)(535.80541597,69.75859579)
\curveto(536.32540971,69.50858578)(536.72040931,69.17358612)(536.99041597,68.75359579)
\curveto(537.04040899,68.67358662)(537.08540895,68.58358671)(537.12541597,68.48359579)
\curveto(537.16540887,68.3935869)(537.21040882,68.29858699)(537.26041597,68.19859579)
\curveto(537.30040873,68.09858719)(537.3304087,67.99858729)(537.35041597,67.89859579)
\curveto(537.37040866,67.79858749)(537.39040864,67.6935876)(537.41041597,67.58359579)
\curveto(537.4304086,67.53358776)(537.4354086,67.4885878)(537.42541597,67.44859579)
\curveto(537.41540862,67.40858788)(537.42040861,67.36358793)(537.44041597,67.31359579)
\curveto(537.45040858,67.26358803)(537.45540858,67.17858811)(537.45541597,67.05859579)
\curveto(537.45540858,66.94858834)(537.45040858,66.86358843)(537.44041597,66.80359579)
\curveto(537.42040861,66.74358855)(537.41040862,66.68358861)(537.41041597,66.62359579)
\curveto(537.42040861,66.56358873)(537.41540862,66.50358879)(537.39541597,66.44359579)
\curveto(537.35540868,66.30358899)(537.32040871,66.16858912)(537.29041597,66.03859579)
\curveto(537.26040877,65.90858938)(537.22040881,65.78358951)(537.17041597,65.66359579)
\curveto(537.11040892,65.52358977)(537.04040899,65.39858989)(536.96041597,65.28859579)
\curveto(536.89040914,65.17859011)(536.81540922,65.06859022)(536.73541597,64.95859579)
\lineto(536.67541597,64.89859579)
\curveto(536.66540937,64.87859041)(536.65040938,64.85859043)(536.63041597,64.83859579)
\curveto(536.51040952,64.67859061)(536.37540966,64.53359076)(536.22541597,64.40359579)
\curveto(536.07540996,64.27359102)(535.91541012,64.14859114)(535.74541597,64.02859579)
\curveto(535.4354106,63.80859148)(535.14041089,63.60359169)(534.86041597,63.41359579)
\curveto(534.6304114,63.27359202)(534.40041163,63.13859215)(534.17041597,63.00859579)
\curveto(533.95041208,62.87859241)(533.7304123,62.74359255)(533.51041597,62.60359579)
\curveto(533.26041277,62.43359286)(533.02041301,62.25359304)(532.79041597,62.06359579)
\curveto(532.57041346,61.87359342)(532.38041365,61.64859364)(532.22041597,61.38859579)
\curveto(532.18041385,61.32859396)(532.14541389,61.26859402)(532.11541597,61.20859579)
\curveto(532.08541395,61.15859413)(532.05541398,61.0935942)(532.02541597,61.01359579)
\curveto(532.00541403,60.94359435)(532.00041403,60.88359441)(532.01041597,60.83359579)
\curveto(532.030414,60.76359453)(532.06541397,60.70859458)(532.11541597,60.66859579)
\curveto(532.16541387,60.63859465)(532.22541381,60.61859467)(532.29541597,60.60859579)
\lineto(532.53541597,60.60859579)
\lineto(533.28541597,60.60859579)
\lineto(536.09041597,60.60859579)
\lineto(536.75041597,60.60859579)
\curveto(536.84040919,60.60859468)(536.92540911,60.60359469)(537.00541597,60.59359579)
\curveto(537.08540895,60.5935947)(537.15040888,60.57359472)(537.20041597,60.53359579)
\curveto(537.25040878,60.4935948)(537.29040874,60.41859487)(537.32041597,60.30859579)
\curveto(537.36040867,60.20859508)(537.37040866,60.10859518)(537.35041597,60.00859579)
\lineto(537.35041597,59.87359579)
\curveto(537.3304087,59.80359549)(537.31040872,59.74359555)(537.29041597,59.69359579)
\curveto(537.27040876,59.64359565)(537.2354088,59.60359569)(537.18541597,59.57359579)
\curveto(537.1354089,59.53359576)(537.06540897,59.51359578)(536.97541597,59.51359579)
\lineto(536.70541597,59.51359579)
\lineto(535.80541597,59.51359579)
\lineto(532.29541597,59.51359579)
\lineto(531.23041597,59.51359579)
\curveto(531.15041488,59.51359578)(531.06041497,59.50859578)(530.96041597,59.49859579)
\curveto(530.86041517,59.49859579)(530.77541526,59.50859578)(530.70541597,59.52859579)
\curveto(530.49541554,59.59859569)(530.4304156,59.77859551)(530.51041597,60.06859579)
\curveto(530.52041551,60.10859518)(530.52041551,60.14359515)(530.51041597,60.17359579)
\curveto(530.51041552,60.21359508)(530.52041551,60.25859503)(530.54041597,60.30859579)
\curveto(530.56041547,60.3885949)(530.58041545,60.47359482)(530.60041597,60.56359579)
\curveto(530.62041541,60.65359464)(530.64541539,60.73859455)(530.67541597,60.81859579)
\curveto(530.8354152,61.30859398)(531.035415,61.72359357)(531.27541597,62.06359579)
\curveto(531.45541458,62.31359298)(531.66041437,62.53859275)(531.89041597,62.73859579)
\curveto(532.12041391,62.94859234)(532.36041367,63.14359215)(532.61041597,63.32359579)
\curveto(532.87041316,63.50359179)(533.1354129,63.67359162)(533.40541597,63.83359579)
\curveto(533.68541235,64.00359129)(533.95541208,64.17859111)(534.21541597,64.35859579)
\curveto(534.32541171,64.43859085)(534.4304116,64.51359078)(534.53041597,64.58359579)
\curveto(534.64041139,64.65359064)(534.75041128,64.72859056)(534.86041597,64.80859579)
\curveto(534.90041113,64.83859045)(534.9354111,64.86859042)(534.96541597,64.89859579)
\curveto(535.00541103,64.93859035)(535.04541099,64.96859032)(535.08541597,64.98859579)
\curveto(535.22541081,65.09859019)(535.35041068,65.22359007)(535.46041597,65.36359579)
\curveto(535.48041055,65.3935899)(535.50541053,65.41858987)(535.53541597,65.43859579)
\curveto(535.56541047,65.46858982)(535.59041044,65.49858979)(535.61041597,65.52859579)
\curveto(535.69041034,65.62858966)(535.75541028,65.72858956)(535.80541597,65.82859579)
\curveto(535.86541017,65.92858936)(535.92041011,66.03858925)(535.97041597,66.15859579)
\curveto(536.00041003,66.22858906)(536.02041001,66.30358899)(536.03041597,66.38359579)
\lineto(536.09041597,66.62359579)
\lineto(536.09041597,66.71359579)
\curveto(536.10040993,66.74358855)(536.10540993,66.77358852)(536.10541597,66.80359579)
\curveto(536.12540991,66.87358842)(536.1304099,66.96858832)(536.12041597,67.08859579)
\curveto(536.12040991,67.21858807)(536.11040992,67.31858797)(536.09041597,67.38859579)
\curveto(536.07040996,67.46858782)(536.05040998,67.54358775)(536.03041597,67.61359579)
\curveto(536.02041001,67.6935876)(536.00041003,67.77358752)(535.97041597,67.85359579)
\curveto(535.86041017,68.0935872)(535.71041032,68.293587)(535.52041597,68.45359579)
\curveto(535.34041069,68.62358667)(535.12041091,68.76358653)(534.86041597,68.87359579)
\curveto(534.79041124,68.8935864)(534.72041131,68.90858638)(534.65041597,68.91859579)
\curveto(534.58041145,68.93858635)(534.50541153,68.95858633)(534.42541597,68.97859579)
\curveto(534.34541169,68.99858629)(534.2354118,69.00858628)(534.09541597,69.00859579)
\curveto(533.96541207,69.00858628)(533.86041217,68.99858629)(533.78041597,68.97859579)
\curveto(533.72041231,68.96858632)(533.66541237,68.96358633)(533.61541597,68.96359579)
\curveto(533.56541247,68.96358633)(533.51541252,68.95358634)(533.46541597,68.93359579)
\curveto(533.36541267,68.8935864)(533.27041276,68.85358644)(533.18041597,68.81359579)
\curveto(533.10041293,68.77358652)(533.02041301,68.72858656)(532.94041597,68.67859579)
\curveto(532.91041312,68.65858663)(532.88041315,68.63358666)(532.85041597,68.60359579)
\curveto(532.8304132,68.57358672)(532.80541323,68.54858674)(532.77541597,68.52859579)
\lineto(532.70041597,68.45359579)
\curveto(532.67041336,68.43358686)(532.64541339,68.41358688)(532.62541597,68.39359579)
\lineto(532.47541597,68.18359579)
\curveto(532.4354136,68.12358717)(532.39041364,68.05858723)(532.34041597,67.98859579)
\curveto(532.28041375,67.89858739)(532.2304138,67.7935875)(532.19041597,67.67359579)
\curveto(532.16041387,67.56358773)(532.12541391,67.45358784)(532.08541597,67.34359579)
\curveto(532.04541399,67.23358806)(532.02041401,67.0885882)(532.01041597,66.90859579)
\curveto(532.00041403,66.73858855)(531.97041406,66.61358868)(531.92041597,66.53359579)
\curveto(531.87041416,66.45358884)(531.79541424,66.40858888)(531.69541597,66.39859579)
\curveto(531.59541444,66.3885889)(531.48541455,66.38358891)(531.36541597,66.38359579)
\curveto(531.32541471,66.38358891)(531.28541475,66.37858891)(531.24541597,66.36859579)
\curveto(531.20541483,66.36858892)(531.17041486,66.37358892)(531.14041597,66.38359579)
\curveto(531.09041494,66.40358889)(531.04041499,66.41358888)(530.99041597,66.41359579)
\curveto(530.95041508,66.41358888)(530.91041512,66.42358887)(530.87041597,66.44359579)
\curveto(530.78041525,66.50358879)(530.7354153,66.63858865)(530.73541597,66.84859579)
\lineto(530.73541597,66.96859579)
\curveto(530.74541529,67.02858826)(530.75041528,67.0885882)(530.75041597,67.14859579)
\curveto(530.76041527,67.21858807)(530.77041526,67.28358801)(530.78041597,67.34359579)
\curveto(530.80041523,67.45358784)(530.82041521,67.55358774)(530.84041597,67.64359579)
\curveto(530.86041517,67.74358755)(530.89041514,67.83858745)(530.93041597,67.92859579)
\curveto(530.95041508,67.99858729)(530.97041506,68.05858723)(530.99041597,68.10859579)
\lineto(531.05041597,68.28859579)
\curveto(531.17041486,68.54858674)(531.32541471,68.7935865)(531.51541597,69.02359579)
\curveto(531.71541432,69.25358604)(531.9304141,69.43858585)(532.16041597,69.57859579)
\curveto(532.27041376,69.65858563)(532.38541365,69.72358557)(532.50541597,69.77359579)
\lineto(532.89541597,69.92359579)
\curveto(533.00541303,69.97358532)(533.12041291,70.00358529)(533.24041597,70.01359579)
\curveto(533.36041267,70.03358526)(533.48541255,70.05858523)(533.61541597,70.08859579)
\curveto(533.68541235,70.0885852)(533.75041228,70.0885852)(533.81041597,70.08859579)
\curveto(533.87041216,70.09858519)(533.9354121,70.10858518)(534.00541597,70.11859579)
}
}
{
\newrgbcolor{curcolor}{0 0 0}
\pscustom[linestyle=none,fillstyle=solid,fillcolor=curcolor]
{
\newpath
\moveto(540.06002535,61.14859579)
\lineto(540.36002535,61.14859579)
\curveto(540.47002329,61.15859413)(540.57502318,61.15859413)(540.67502535,61.14859579)
\curveto(540.78502297,61.14859414)(540.88502287,61.13859415)(540.97502535,61.11859579)
\curveto(541.06502269,61.10859418)(541.13502262,61.08359421)(541.18502535,61.04359579)
\curveto(541.20502255,61.02359427)(541.22002254,60.9935943)(541.23002535,60.95359579)
\curveto(541.25002251,60.91359438)(541.27002249,60.86859442)(541.29002535,60.81859579)
\lineto(541.29002535,60.74359579)
\curveto(541.30002246,60.6935946)(541.30002246,60.63859465)(541.29002535,60.57859579)
\lineto(541.29002535,60.42859579)
\lineto(541.29002535,59.94859579)
\curveto(541.29002247,59.77859551)(541.25002251,59.65859563)(541.17002535,59.58859579)
\curveto(541.10002266,59.53859575)(541.01002275,59.51359578)(540.90002535,59.51359579)
\lineto(540.57002535,59.51359579)
\lineto(540.12002535,59.51359579)
\curveto(539.97002379,59.51359578)(539.8550239,59.54359575)(539.77502535,59.60359579)
\curveto(539.73502402,59.63359566)(539.70502405,59.68359561)(539.68502535,59.75359579)
\curveto(539.66502409,59.83359546)(539.65002411,59.91859537)(539.64002535,60.00859579)
\lineto(539.64002535,60.29359579)
\curveto(539.65002411,60.3935949)(539.6550241,60.47859481)(539.65502535,60.54859579)
\lineto(539.65502535,60.74359579)
\curveto(539.6550241,60.80359449)(539.66502409,60.85859443)(539.68502535,60.90859579)
\curveto(539.72502403,61.01859427)(539.79502396,61.0885942)(539.89502535,61.11859579)
\curveto(539.92502383,61.11859417)(539.98002378,61.12859416)(540.06002535,61.14859579)
}
}
{
\newrgbcolor{curcolor}{0 0 0}
\pscustom[linestyle=none,fillstyle=solid,fillcolor=curcolor]
{
\newpath
\moveto(550.2651816,63.83359579)
\curveto(550.29517387,63.71359158)(550.32017385,63.57359172)(550.3401816,63.41359579)
\curveto(550.36017381,63.25359204)(550.3701738,63.0885922)(550.3701816,62.91859579)
\curveto(550.3701738,62.74859254)(550.36017381,62.58359271)(550.3401816,62.42359579)
\curveto(550.32017385,62.26359303)(550.29517387,62.12359317)(550.2651816,62.00359579)
\curveto(550.22517394,61.86359343)(550.19017398,61.73859355)(550.1601816,61.62859579)
\curveto(550.13017404,61.51859377)(550.09017408,61.40859388)(550.0401816,61.29859579)
\curveto(549.7701744,60.65859463)(549.35517481,60.17359512)(548.7951816,59.84359579)
\curveto(548.71517545,59.78359551)(548.63017554,59.73359556)(548.5401816,59.69359579)
\curveto(548.45017572,59.66359563)(548.35017582,59.62859566)(548.2401816,59.58859579)
\curveto(548.13017604,59.53859575)(548.01017616,59.50359579)(547.8801816,59.48359579)
\curveto(547.76017641,59.45359584)(547.63017654,59.42359587)(547.4901816,59.39359579)
\curveto(547.43017674,59.37359592)(547.3701768,59.36859592)(547.3101816,59.37859579)
\curveto(547.26017691,59.3885959)(547.20017697,59.38359591)(547.1301816,59.36359579)
\curveto(547.11017706,59.35359594)(547.08517708,59.35359594)(547.0551816,59.36359579)
\curveto(547.02517714,59.36359593)(547.00017717,59.35859593)(546.9801816,59.34859579)
\lineto(546.8301816,59.34859579)
\curveto(546.76017741,59.33859595)(546.71017746,59.33859595)(546.6801816,59.34859579)
\curveto(546.64017753,59.35859593)(546.59517757,59.36359593)(546.5451816,59.36359579)
\curveto(546.50517766,59.35359594)(546.4651777,59.35359594)(546.4251816,59.36359579)
\curveto(546.33517783,59.38359591)(546.24517792,59.39859589)(546.1551816,59.40859579)
\curveto(546.0651781,59.40859588)(545.97517819,59.41859587)(545.8851816,59.43859579)
\curveto(545.79517837,59.46859582)(545.70517846,59.4935958)(545.6151816,59.51359579)
\curveto(545.52517864,59.53359576)(545.44017873,59.56359573)(545.3601816,59.60359579)
\curveto(545.12017905,59.71359558)(544.89517927,59.84359545)(544.6851816,59.99359579)
\curveto(544.47517969,60.15359514)(544.29517987,60.33359496)(544.1451816,60.53359579)
\curveto(544.02518014,60.70359459)(543.92018025,60.87859441)(543.8301816,61.05859579)
\curveto(543.74018043,61.23859405)(543.65018052,61.42859386)(543.5601816,61.62859579)
\curveto(543.52018065,61.72859356)(543.48518068,61.82859346)(543.4551816,61.92859579)
\curveto(543.43518073,62.03859325)(543.41018076,62.14859314)(543.3801816,62.25859579)
\curveto(543.34018083,62.39859289)(543.31518085,62.53859275)(543.3051816,62.67859579)
\curveto(543.29518087,62.81859247)(543.27518089,62.95859233)(543.2451816,63.09859579)
\curveto(543.23518093,63.20859208)(543.22518094,63.30859198)(543.2151816,63.39859579)
\curveto(543.21518095,63.49859179)(543.20518096,63.59859169)(543.1851816,63.69859579)
\lineto(543.1851816,63.78859579)
\curveto(543.19518097,63.81859147)(543.19518097,63.84359145)(543.1851816,63.86359579)
\lineto(543.1851816,64.07359579)
\curveto(543.165181,64.13359116)(543.15518101,64.19859109)(543.1551816,64.26859579)
\curveto(543.165181,64.34859094)(543.170181,64.42359087)(543.1701816,64.49359579)
\lineto(543.1701816,64.64359579)
\curveto(543.170181,64.6935906)(543.17518099,64.74359055)(543.1851816,64.79359579)
\lineto(543.1851816,65.16859579)
\curveto(543.19518097,65.19859009)(543.19518097,65.23359006)(543.1851816,65.27359579)
\curveto(543.18518098,65.31358998)(543.19018098,65.35358994)(543.2001816,65.39359579)
\curveto(543.22018095,65.50358979)(543.23518093,65.61358968)(543.2451816,65.72359579)
\curveto(543.25518091,65.84358945)(543.2651809,65.95858933)(543.2751816,66.06859579)
\curveto(543.31518085,66.21858907)(543.34018083,66.36358893)(543.3501816,66.50359579)
\curveto(543.3701808,66.65358864)(543.40018077,66.79858849)(543.4401816,66.93859579)
\curveto(543.53018064,67.23858805)(543.62518054,67.52358777)(543.7251816,67.79359579)
\curveto(543.82518034,68.06358723)(543.95018022,68.31358698)(544.1001816,68.54359579)
\curveto(544.30017987,68.86358643)(544.54517962,69.14358615)(544.8351816,69.38359579)
\curveto(545.12517904,69.62358567)(545.4651787,69.80858548)(545.8551816,69.93859579)
\curveto(545.9651782,69.97858531)(546.07517809,70.00358529)(546.1851816,70.01359579)
\curveto(546.30517786,70.03358526)(546.42517774,70.05858523)(546.5451816,70.08859579)
\curveto(546.61517755,70.09858519)(546.68017749,70.10358519)(546.7401816,70.10359579)
\curveto(546.80017737,70.10358519)(546.8651773,70.10858518)(546.9351816,70.11859579)
\curveto(547.63517653,70.13858515)(548.21017596,70.02358527)(548.6601816,69.77359579)
\curveto(549.11017506,69.52358577)(549.45517471,69.17358612)(549.6951816,68.72359579)
\curveto(549.80517436,68.4935868)(549.90517426,68.21858707)(549.9951816,67.89859579)
\curveto(550.01517415,67.82858746)(550.01517415,67.75358754)(549.9951816,67.67359579)
\curveto(549.98517418,67.60358769)(549.96017421,67.55358774)(549.9201816,67.52359579)
\curveto(549.89017428,67.4935878)(549.83017434,67.46858782)(549.7401816,67.44859579)
\curveto(549.65017452,67.43858785)(549.55017462,67.42858786)(549.4401816,67.41859579)
\curveto(549.34017483,67.41858787)(549.24017493,67.42358787)(549.1401816,67.43359579)
\curveto(549.05017512,67.44358785)(548.98517518,67.46358783)(548.9451816,67.49359579)
\curveto(548.83517533,67.56358773)(548.75517541,67.67358762)(548.7051816,67.82359579)
\curveto(548.6651755,67.97358732)(548.61017556,68.10358719)(548.5401816,68.21359579)
\curveto(548.35017582,68.52358677)(548.0701761,68.75358654)(547.7001816,68.90359579)
\curveto(547.63017654,68.93358636)(547.55517661,68.95358634)(547.4751816,68.96359579)
\curveto(547.40517676,68.97358632)(547.33017684,68.9885863)(547.2501816,69.00859579)
\curveto(547.20017697,69.01858627)(547.13017704,69.02358627)(547.0401816,69.02359579)
\curveto(546.96017721,69.02358627)(546.89517727,69.01858627)(546.8451816,69.00859579)
\curveto(546.80517736,68.9885863)(546.7701774,68.98358631)(546.7401816,68.99359579)
\curveto(546.71017746,69.00358629)(546.67517749,69.00358629)(546.6351816,68.99359579)
\lineto(546.3951816,68.93359579)
\curveto(546.32517784,68.91358638)(546.25517791,68.8885864)(546.1851816,68.85859579)
\curveto(545.80517836,68.69858659)(545.51517865,68.4885868)(545.3151816,68.22859579)
\curveto(545.12517904,67.96858732)(544.95017922,67.65358764)(544.7901816,67.28359579)
\curveto(544.76017941,67.20358809)(544.73517943,67.12358817)(544.7151816,67.04359579)
\curveto(544.70517946,66.96358833)(544.68517948,66.88358841)(544.6551816,66.80359579)
\curveto(544.62517954,66.6935886)(544.60017957,66.57858871)(544.5801816,66.45859579)
\curveto(544.5701796,66.33858895)(544.55017962,66.21858907)(544.5201816,66.09859579)
\curveto(544.50017967,66.04858924)(544.49017968,65.99858929)(544.4901816,65.94859579)
\curveto(544.50017967,65.89858939)(544.49517967,65.84858944)(544.4751816,65.79859579)
\curveto(544.4651797,65.73858955)(544.4651797,65.65858963)(544.4751816,65.55859579)
\curveto(544.48517968,65.46858982)(544.50017967,65.41358988)(544.5201816,65.39359579)
\curveto(544.54017963,65.35358994)(544.5701796,65.33358996)(544.6101816,65.33359579)
\curveto(544.66017951,65.33358996)(544.70517946,65.34358995)(544.7451816,65.36359579)
\curveto(544.81517935,65.40358989)(544.87517929,65.44858984)(544.9251816,65.49859579)
\curveto(544.97517919,65.54858974)(545.03517913,65.59858969)(545.1051816,65.64859579)
\lineto(545.1651816,65.70859579)
\curveto(545.19517897,65.73858955)(545.22517894,65.76358953)(545.2551816,65.78359579)
\curveto(545.48517868,65.94358935)(545.76017841,66.07858921)(546.0801816,66.18859579)
\curveto(546.15017802,66.20858908)(546.22017795,66.22358907)(546.2901816,66.23359579)
\curveto(546.36017781,66.24358905)(546.43517773,66.25858903)(546.5151816,66.27859579)
\curveto(546.55517761,66.27858901)(546.59017758,66.28358901)(546.6201816,66.29359579)
\curveto(546.65017752,66.30358899)(546.68517748,66.30358899)(546.7251816,66.29359579)
\curveto(546.77517739,66.293589)(546.81517735,66.30358899)(546.8451816,66.32359579)
\lineto(547.0101816,66.32359579)
\lineto(547.1001816,66.32359579)
\curveto(547.15017702,66.33358896)(547.19017698,66.33358896)(547.2201816,66.32359579)
\curveto(547.2701769,66.31358898)(547.32017685,66.30858898)(547.3701816,66.30859579)
\curveto(547.43017674,66.31858897)(547.48517668,66.31858897)(547.5351816,66.30859579)
\curveto(547.64517652,66.27858901)(547.75017642,66.25858903)(547.8501816,66.24859579)
\curveto(547.96017621,66.23858905)(548.0651761,66.21358908)(548.1651816,66.17359579)
\curveto(548.58517558,66.03358926)(548.93017524,65.84858944)(549.2001816,65.61859579)
\curveto(549.4701747,65.39858989)(549.71017446,65.11359018)(549.9201816,64.76359579)
\curveto(550.00017417,64.62359067)(550.0651741,64.47359082)(550.1151816,64.31359579)
\curveto(550.165174,64.16359113)(550.21517395,64.00359129)(550.2651816,63.83359579)
\moveto(549.0201816,62.52859579)
\curveto(549.03017514,62.57859271)(549.03517513,62.62359267)(549.0351816,62.66359579)
\lineto(549.0351816,62.81359579)
\curveto(549.03517513,63.12359217)(548.99517517,63.40859188)(548.9151816,63.66859579)
\curveto(548.89517527,63.72859156)(548.87517529,63.78359151)(548.8551816,63.83359579)
\curveto(548.84517532,63.8935914)(548.83017534,63.94859134)(548.8101816,63.99859579)
\curveto(548.59017558,64.4885908)(548.24517592,64.83859045)(547.7751816,65.04859579)
\curveto(547.69517647,65.07859021)(547.61517655,65.10359019)(547.5351816,65.12359579)
\lineto(547.2951816,65.18359579)
\curveto(547.21517695,65.20359009)(547.12517704,65.21359008)(547.0251816,65.21359579)
\lineto(546.7101816,65.21359579)
\curveto(546.69017748,65.1935901)(546.65017752,65.18359011)(546.5901816,65.18359579)
\curveto(546.54017763,65.1935901)(546.49517767,65.1935901)(546.4551816,65.18359579)
\lineto(546.2151816,65.12359579)
\curveto(546.14517802,65.11359018)(546.07517809,65.0935902)(546.0051816,65.06359579)
\curveto(545.40517876,64.80359049)(545.00017917,64.33859095)(544.7901816,63.66859579)
\curveto(544.76017941,63.5885917)(544.74017943,63.50859178)(544.7301816,63.42859579)
\curveto(544.72017945,63.34859194)(544.70517946,63.26359203)(544.6851816,63.17359579)
\lineto(544.6851816,63.02359579)
\curveto(544.67517949,62.98359231)(544.6701795,62.91359238)(544.6701816,62.81359579)
\curveto(544.6701795,62.58359271)(544.69017948,62.3885929)(544.7301816,62.22859579)
\curveto(544.75017942,62.15859313)(544.7651794,62.0935932)(544.7751816,62.03359579)
\curveto(544.78517938,61.97359332)(544.80517936,61.90859338)(544.8351816,61.83859579)
\curveto(544.94517922,61.55859373)(545.09017908,61.31359398)(545.2701816,61.10359579)
\curveto(545.45017872,60.90359439)(545.68517848,60.74359455)(545.9751816,60.62359579)
\lineto(546.2151816,60.53359579)
\lineto(546.4551816,60.47359579)
\curveto(546.50517766,60.45359484)(546.54517762,60.44859484)(546.5751816,60.45859579)
\curveto(546.61517755,60.46859482)(546.66017751,60.46359483)(546.7101816,60.44359579)
\curveto(546.74017743,60.43359486)(546.79517737,60.42859486)(546.8751816,60.42859579)
\curveto(546.95517721,60.42859486)(547.01517715,60.43359486)(547.0551816,60.44359579)
\curveto(547.165177,60.46359483)(547.2701769,60.47859481)(547.3701816,60.48859579)
\curveto(547.4701767,60.49859479)(547.5651766,60.52859476)(547.6551816,60.57859579)
\curveto(548.18517598,60.77859451)(548.57517559,61.15359414)(548.8251816,61.70359579)
\curveto(548.8651753,61.80359349)(548.89517527,61.90859338)(548.9151816,62.01859579)
\lineto(549.0051816,62.34859579)
\curveto(549.00517516,62.42859286)(549.01017516,62.4885928)(549.0201816,62.52859579)
}
}
{
\newrgbcolor{curcolor}{0 0 0}
\pscustom[linestyle=none,fillstyle=solid,fillcolor=curcolor]
{
\newpath
\moveto(561.41979097,68.03359579)
\curveto(561.21978067,67.74358755)(561.00978088,67.45858783)(560.78979097,67.17859579)
\curveto(560.57978131,66.89858839)(560.37478152,66.61358868)(560.17479097,66.32359579)
\curveto(559.57478232,65.47358982)(558.96978292,64.63359066)(558.35979097,63.80359579)
\curveto(557.74978414,62.98359231)(557.14478475,62.14859314)(556.54479097,61.29859579)
\lineto(556.03479097,60.57859579)
\lineto(555.52479097,59.88859579)
\curveto(555.44478645,59.77859551)(555.36478653,59.66359563)(555.28479097,59.54359579)
\curveto(555.20478669,59.42359587)(555.10978678,59.32859596)(554.99979097,59.25859579)
\curveto(554.95978693,59.23859605)(554.894787,59.22359607)(554.80479097,59.21359579)
\curveto(554.72478717,59.1935961)(554.63478726,59.18359611)(554.53479097,59.18359579)
\curveto(554.43478746,59.18359611)(554.33978755,59.1885961)(554.24979097,59.19859579)
\curveto(554.16978772,59.20859608)(554.10978778,59.22859606)(554.06979097,59.25859579)
\curveto(554.03978785,59.27859601)(554.01478788,59.31359598)(553.99479097,59.36359579)
\curveto(553.98478791,59.40359589)(553.9897879,59.44859584)(554.00979097,59.49859579)
\curveto(554.04978784,59.57859571)(554.0947878,59.65359564)(554.14479097,59.72359579)
\curveto(554.20478769,59.80359549)(554.25978763,59.88359541)(554.30979097,59.96359579)
\curveto(554.54978734,60.30359499)(554.7947871,60.63859465)(555.04479097,60.96859579)
\curveto(555.2947866,61.29859399)(555.53478636,61.63359366)(555.76479097,61.97359579)
\curveto(555.92478597,62.1935931)(556.08478581,62.40859288)(556.24479097,62.61859579)
\curveto(556.40478549,62.82859246)(556.56478533,63.04359225)(556.72479097,63.26359579)
\curveto(557.08478481,63.78359151)(557.44978444,64.293591)(557.81979097,64.79359579)
\curveto(558.1897837,65.29359)(558.55978333,65.80358949)(558.92979097,66.32359579)
\curveto(559.06978282,66.52358877)(559.20978268,66.71858857)(559.34979097,66.90859579)
\curveto(559.49978239,67.09858819)(559.64478225,67.293588)(559.78479097,67.49359579)
\curveto(559.9947819,67.7935875)(560.20978168,68.0935872)(560.42979097,68.39359579)
\lineto(561.08979097,69.29359579)
\lineto(561.26979097,69.56359579)
\lineto(561.47979097,69.83359579)
\lineto(561.59979097,70.01359579)
\curveto(561.64978024,70.07358522)(561.69978019,70.12858516)(561.74979097,70.17859579)
\curveto(561.81978007,70.22858506)(561.89478,70.26358503)(561.97479097,70.28359579)
\curveto(561.9947799,70.293585)(562.01977987,70.293585)(562.04979097,70.28359579)
\curveto(562.0897798,70.28358501)(562.11977977,70.293585)(562.13979097,70.31359579)
\curveto(562.25977963,70.31358498)(562.3947795,70.30858498)(562.54479097,70.29859579)
\curveto(562.6947792,70.29858499)(562.78477911,70.25358504)(562.81479097,70.16359579)
\curveto(562.83477906,70.13358516)(562.83977905,70.09858519)(562.82979097,70.05859579)
\curveto(562.81977907,70.01858527)(562.80477909,69.9885853)(562.78479097,69.96859579)
\curveto(562.74477915,69.8885854)(562.70477919,69.81858547)(562.66479097,69.75859579)
\curveto(562.62477927,69.69858559)(562.57977931,69.63858565)(562.52979097,69.57859579)
\lineto(561.95979097,68.79859579)
\curveto(561.77978011,68.54858674)(561.59978029,68.293587)(561.41979097,68.03359579)
\moveto(554.56479097,64.13359579)
\curveto(554.51478738,64.15359114)(554.46478743,64.15859113)(554.41479097,64.14859579)
\curveto(554.36478753,64.13859115)(554.31478758,64.14359115)(554.26479097,64.16359579)
\curveto(554.15478774,64.18359111)(554.04978784,64.20359109)(553.94979097,64.22359579)
\curveto(553.85978803,64.25359104)(553.76478813,64.293591)(553.66479097,64.34359579)
\curveto(553.33478856,64.48359081)(553.07978881,64.67859061)(552.89979097,64.92859579)
\curveto(552.71978917,65.1885901)(552.57478932,65.49858979)(552.46479097,65.85859579)
\curveto(552.43478946,65.93858935)(552.41478948,66.01858927)(552.40479097,66.09859579)
\curveto(552.3947895,66.1885891)(552.37978951,66.27358902)(552.35979097,66.35359579)
\curveto(552.34978954,66.40358889)(552.34478955,66.46858882)(552.34479097,66.54859579)
\curveto(552.33478956,66.57858871)(552.32978956,66.60858868)(552.32979097,66.63859579)
\curveto(552.32978956,66.67858861)(552.32478957,66.71358858)(552.31479097,66.74359579)
\lineto(552.31479097,66.89359579)
\curveto(552.30478959,66.94358835)(552.29978959,67.00358829)(552.29979097,67.07359579)
\curveto(552.29978959,67.15358814)(552.30478959,67.21858807)(552.31479097,67.26859579)
\lineto(552.31479097,67.43359579)
\curveto(552.33478956,67.48358781)(552.33978955,67.52858776)(552.32979097,67.56859579)
\curveto(552.32978956,67.61858767)(552.33478956,67.66358763)(552.34479097,67.70359579)
\curveto(552.35478954,67.74358755)(552.35978953,67.77858751)(552.35979097,67.80859579)
\curveto(552.35978953,67.84858744)(552.36478953,67.8885874)(552.37479097,67.92859579)
\curveto(552.40478949,68.03858725)(552.42478947,68.14858714)(552.43479097,68.25859579)
\curveto(552.45478944,68.37858691)(552.4897894,68.4935868)(552.53979097,68.60359579)
\curveto(552.67978921,68.94358635)(552.83978905,69.21858607)(553.01979097,69.42859579)
\curveto(553.20978868,69.64858564)(553.47978841,69.82858546)(553.82979097,69.96859579)
\curveto(553.90978798,69.99858529)(553.9947879,70.01858527)(554.08479097,70.02859579)
\curveto(554.17478772,70.04858524)(554.26978762,70.06858522)(554.36979097,70.08859579)
\curveto(554.39978749,70.09858519)(554.45478744,70.09858519)(554.53479097,70.08859579)
\curveto(554.61478728,70.0885852)(554.66478723,70.09858519)(554.68479097,70.11859579)
\curveto(555.24478665,70.12858516)(555.6947862,70.01858527)(556.03479097,69.78859579)
\curveto(556.38478551,69.55858573)(556.64478525,69.25358604)(556.81479097,68.87359579)
\curveto(556.85478504,68.78358651)(556.889785,68.6885866)(556.91979097,68.58859579)
\curveto(556.94978494,68.4885868)(556.97478492,68.3885869)(556.99479097,68.28859579)
\curveto(557.01478488,68.25858703)(557.01978487,68.22858706)(557.00979097,68.19859579)
\curveto(557.00978488,68.16858712)(557.01478488,68.13858715)(557.02479097,68.10859579)
\curveto(557.05478484,67.99858729)(557.07478482,67.87358742)(557.08479097,67.73359579)
\curveto(557.0947848,67.60358769)(557.10478479,67.46858782)(557.11479097,67.32859579)
\lineto(557.11479097,67.16359579)
\curveto(557.12478477,67.10358819)(557.12478477,67.04858824)(557.11479097,66.99859579)
\curveto(557.10478479,66.94858834)(557.09978479,66.89858839)(557.09979097,66.84859579)
\lineto(557.09979097,66.71359579)
\curveto(557.0897848,66.67358862)(557.08478481,66.63358866)(557.08479097,66.59359579)
\curveto(557.0947848,66.55358874)(557.0897848,66.50858878)(557.06979097,66.45859579)
\curveto(557.04978484,66.34858894)(557.02978486,66.24358905)(557.00979097,66.14359579)
\curveto(556.99978489,66.04358925)(556.97978491,65.94358935)(556.94979097,65.84359579)
\curveto(556.81978507,65.48358981)(556.65478524,65.16859012)(556.45479097,64.89859579)
\curveto(556.25478564,64.62859066)(555.97978591,64.42359087)(555.62979097,64.28359579)
\curveto(555.54978634,64.25359104)(555.46478643,64.22859106)(555.37479097,64.20859579)
\lineto(555.10479097,64.14859579)
\curveto(555.05478684,64.13859115)(555.00978688,64.13359116)(554.96979097,64.13359579)
\curveto(554.92978696,64.14359115)(554.889787,64.14359115)(554.84979097,64.13359579)
\curveto(554.74978714,64.11359118)(554.65478724,64.11359118)(554.56479097,64.13359579)
\moveto(553.72479097,65.52859579)
\curveto(553.76478813,65.45858983)(553.80478809,65.3935899)(553.84479097,65.33359579)
\curveto(553.88478801,65.28359001)(553.93478796,65.23359006)(553.99479097,65.18359579)
\lineto(554.14479097,65.06359579)
\curveto(554.20478769,65.03359026)(554.26978762,65.00859028)(554.33979097,64.98859579)
\curveto(554.37978751,64.96859032)(554.41478748,64.95859033)(554.44479097,64.95859579)
\curveto(554.48478741,64.96859032)(554.52478737,64.96359033)(554.56479097,64.94359579)
\curveto(554.5947873,64.94359035)(554.63478726,64.93859035)(554.68479097,64.92859579)
\curveto(554.73478716,64.92859036)(554.77478712,64.93359036)(554.80479097,64.94359579)
\lineto(555.02979097,64.98859579)
\curveto(555.27978661,65.06859022)(555.46478643,65.1935901)(555.58479097,65.36359579)
\curveto(555.66478623,65.46358983)(555.73478616,65.5935897)(555.79479097,65.75359579)
\curveto(555.87478602,65.93358936)(555.93478596,66.15858913)(555.97479097,66.42859579)
\curveto(556.01478588,66.70858858)(556.02978586,66.9885883)(556.01979097,67.26859579)
\curveto(556.00978588,67.55858773)(555.97978591,67.83358746)(555.92979097,68.09359579)
\curveto(555.87978601,68.35358694)(555.80478609,68.56358673)(555.70479097,68.72359579)
\curveto(555.58478631,68.92358637)(555.43478646,69.07358622)(555.25479097,69.17359579)
\curveto(555.17478672,69.22358607)(555.08478681,69.25358604)(554.98479097,69.26359579)
\curveto(554.88478701,69.28358601)(554.77978711,69.293586)(554.66979097,69.29359579)
\curveto(554.64978724,69.28358601)(554.62478727,69.27858601)(554.59479097,69.27859579)
\curveto(554.57478732,69.288586)(554.55478734,69.288586)(554.53479097,69.27859579)
\curveto(554.48478741,69.26858602)(554.43978745,69.25858603)(554.39979097,69.24859579)
\curveto(554.35978753,69.24858604)(554.31978757,69.23858605)(554.27979097,69.21859579)
\curveto(554.09978779,69.13858615)(553.94978794,69.01858627)(553.82979097,68.85859579)
\curveto(553.71978817,68.69858659)(553.62978826,68.51858677)(553.55979097,68.31859579)
\curveto(553.49978839,68.12858716)(553.45478844,67.90358739)(553.42479097,67.64359579)
\curveto(553.40478849,67.38358791)(553.39978849,67.11858817)(553.40979097,66.84859579)
\curveto(553.41978847,66.5885887)(553.44978844,66.33858895)(553.49979097,66.09859579)
\curveto(553.55978833,65.86858942)(553.63478826,65.67858961)(553.72479097,65.52859579)
\moveto(564.52479097,62.54359579)
\curveto(564.53477736,62.4935928)(564.53977735,62.40359289)(564.53979097,62.27359579)
\curveto(564.53977735,62.14359315)(564.52977736,62.05359324)(564.50979097,62.00359579)
\curveto(564.4897774,61.95359334)(564.48477741,61.89859339)(564.49479097,61.83859579)
\curveto(564.50477739,61.7885935)(564.50477739,61.73859355)(564.49479097,61.68859579)
\curveto(564.45477744,61.54859374)(564.42477747,61.41359388)(564.40479097,61.28359579)
\curveto(564.3947775,61.15359414)(564.36477753,61.03359426)(564.31479097,60.92359579)
\curveto(564.17477772,60.57359472)(564.00977788,60.27859501)(563.81979097,60.03859579)
\curveto(563.62977826,59.80859548)(563.35977853,59.62359567)(563.00979097,59.48359579)
\curveto(562.92977896,59.45359584)(562.84477905,59.43359586)(562.75479097,59.42359579)
\curveto(562.66477923,59.40359589)(562.57977931,59.38359591)(562.49979097,59.36359579)
\curveto(562.44977944,59.35359594)(562.39977949,59.34859594)(562.34979097,59.34859579)
\curveto(562.29977959,59.34859594)(562.24977964,59.34359595)(562.19979097,59.33359579)
\curveto(562.16977972,59.32359597)(562.11977977,59.32359597)(562.04979097,59.33359579)
\curveto(561.97977991,59.33359596)(561.92977996,59.33859595)(561.89979097,59.34859579)
\curveto(561.83978005,59.36859592)(561.77978011,59.37859591)(561.71979097,59.37859579)
\curveto(561.66978022,59.36859592)(561.61978027,59.37359592)(561.56979097,59.39359579)
\curveto(561.47978041,59.41359588)(561.3897805,59.43859585)(561.29979097,59.46859579)
\curveto(561.21978067,59.4885958)(561.13978075,59.51859577)(561.05979097,59.55859579)
\curveto(560.73978115,59.69859559)(560.4897814,59.8935954)(560.30979097,60.14359579)
\curveto(560.12978176,60.40359489)(559.97978191,60.70859458)(559.85979097,61.05859579)
\curveto(559.83978205,61.13859415)(559.82478207,61.22359407)(559.81479097,61.31359579)
\curveto(559.80478209,61.40359389)(559.7897821,61.4885938)(559.76979097,61.56859579)
\curveto(559.75978213,61.59859369)(559.75478214,61.62859366)(559.75479097,61.65859579)
\lineto(559.75479097,61.76359579)
\curveto(559.73478216,61.84359345)(559.72478217,61.92359337)(559.72479097,62.00359579)
\lineto(559.72479097,62.13859579)
\curveto(559.70478219,62.23859305)(559.70478219,62.33859295)(559.72479097,62.43859579)
\lineto(559.72479097,62.61859579)
\curveto(559.73478216,62.66859262)(559.73978215,62.71359258)(559.73979097,62.75359579)
\curveto(559.73978215,62.80359249)(559.74478215,62.84859244)(559.75479097,62.88859579)
\curveto(559.76478213,62.92859236)(559.76978212,62.96359233)(559.76979097,62.99359579)
\curveto(559.76978212,63.03359226)(559.77478212,63.07359222)(559.78479097,63.11359579)
\lineto(559.84479097,63.44359579)
\curveto(559.86478203,63.56359173)(559.894782,63.67359162)(559.93479097,63.77359579)
\curveto(560.07478182,64.10359119)(560.23478166,64.37859091)(560.41479097,64.59859579)
\curveto(560.60478129,64.82859046)(560.86478103,65.01359028)(561.19479097,65.15359579)
\curveto(561.27478062,65.1935901)(561.35978053,65.21859007)(561.44979097,65.22859579)
\lineto(561.74979097,65.28859579)
\lineto(561.88479097,65.28859579)
\curveto(561.93477996,65.29858999)(561.98477991,65.30358999)(562.03479097,65.30359579)
\curveto(562.60477929,65.32358997)(563.06477883,65.21859007)(563.41479097,64.98859579)
\curveto(563.77477812,64.76859052)(564.03977785,64.46859082)(564.20979097,64.08859579)
\curveto(564.25977763,63.9885913)(564.29977759,63.8885914)(564.32979097,63.78859579)
\curveto(564.35977753,63.6885916)(564.3897775,63.58359171)(564.41979097,63.47359579)
\curveto(564.42977746,63.43359186)(564.43477746,63.39859189)(564.43479097,63.36859579)
\curveto(564.43477746,63.34859194)(564.43977745,63.31859197)(564.44979097,63.27859579)
\curveto(564.46977742,63.20859208)(564.47977741,63.13359216)(564.47979097,63.05359579)
\curveto(564.47977741,62.97359232)(564.4897774,62.8935924)(564.50979097,62.81359579)
\curveto(564.50977738,62.76359253)(564.50977738,62.71859257)(564.50979097,62.67859579)
\curveto(564.50977738,62.63859265)(564.51477738,62.5935927)(564.52479097,62.54359579)
\moveto(563.41479097,62.10859579)
\curveto(563.42477847,62.15859313)(563.42977846,62.23359306)(563.42979097,62.33359579)
\curveto(563.43977845,62.43359286)(563.43477846,62.50859278)(563.41479097,62.55859579)
\curveto(563.3947785,62.61859267)(563.3897785,62.67359262)(563.39979097,62.72359579)
\curveto(563.41977847,62.78359251)(563.41977847,62.84359245)(563.39979097,62.90359579)
\curveto(563.3897785,62.93359236)(563.38477851,62.96859232)(563.38479097,63.00859579)
\curveto(563.38477851,63.04859224)(563.37977851,63.0885922)(563.36979097,63.12859579)
\curveto(563.34977854,63.20859208)(563.32977856,63.28359201)(563.30979097,63.35359579)
\curveto(563.29977859,63.43359186)(563.28477861,63.51359178)(563.26479097,63.59359579)
\curveto(563.23477866,63.65359164)(563.20977868,63.71359158)(563.18979097,63.77359579)
\curveto(563.16977872,63.83359146)(563.13977875,63.8935914)(563.09979097,63.95359579)
\curveto(562.99977889,64.12359117)(562.86977902,64.25859103)(562.70979097,64.35859579)
\curveto(562.62977926,64.40859088)(562.53477936,64.44359085)(562.42479097,64.46359579)
\curveto(562.31477958,64.48359081)(562.1897797,64.4935908)(562.04979097,64.49359579)
\curveto(562.02977986,64.48359081)(562.00477989,64.47859081)(561.97479097,64.47859579)
\curveto(561.94477995,64.4885908)(561.91477998,64.4885908)(561.88479097,64.47859579)
\lineto(561.73479097,64.41859579)
\curveto(561.68478021,64.40859088)(561.63978025,64.3935909)(561.59979097,64.37359579)
\curveto(561.40978048,64.26359103)(561.26478063,64.11859117)(561.16479097,63.93859579)
\curveto(561.07478082,63.75859153)(560.9947809,63.55359174)(560.92479097,63.32359579)
\curveto(560.88478101,63.1935921)(560.86478103,63.05859223)(560.86479097,62.91859579)
\curveto(560.86478103,62.7885925)(560.85478104,62.64359265)(560.83479097,62.48359579)
\curveto(560.82478107,62.43359286)(560.81478108,62.37359292)(560.80479097,62.30359579)
\curveto(560.80478109,62.23359306)(560.81478108,62.17359312)(560.83479097,62.12359579)
\lineto(560.83479097,61.95859579)
\lineto(560.83479097,61.77859579)
\curveto(560.84478105,61.72859356)(560.85478104,61.67359362)(560.86479097,61.61359579)
\curveto(560.87478102,61.56359373)(560.87978101,61.50859378)(560.87979097,61.44859579)
\curveto(560.889781,61.3885939)(560.90478099,61.33359396)(560.92479097,61.28359579)
\curveto(560.97478092,61.0935942)(561.03478086,60.91859437)(561.10479097,60.75859579)
\curveto(561.17478072,60.59859469)(561.27978061,60.46859482)(561.41979097,60.36859579)
\curveto(561.54978034,60.26859502)(561.6897802,60.19859509)(561.83979097,60.15859579)
\curveto(561.86978002,60.14859514)(561.89478,60.14359515)(561.91479097,60.14359579)
\curveto(561.94477995,60.15359514)(561.97477992,60.15359514)(562.00479097,60.14359579)
\curveto(562.02477987,60.14359515)(562.05477984,60.13859515)(562.09479097,60.12859579)
\curveto(562.13477976,60.12859516)(562.16977972,60.13359516)(562.19979097,60.14359579)
\curveto(562.23977965,60.15359514)(562.27977961,60.15859513)(562.31979097,60.15859579)
\curveto(562.35977953,60.15859513)(562.39977949,60.16859512)(562.43979097,60.18859579)
\curveto(562.67977921,60.26859502)(562.87477902,60.40359489)(563.02479097,60.59359579)
\curveto(563.14477875,60.77359452)(563.23477866,60.97859431)(563.29479097,61.20859579)
\curveto(563.31477858,61.27859401)(563.32977856,61.34859394)(563.33979097,61.41859579)
\curveto(563.34977854,61.49859379)(563.36477853,61.57859371)(563.38479097,61.65859579)
\curveto(563.38477851,61.71859357)(563.3897785,61.76359353)(563.39979097,61.79359579)
\curveto(563.39977849,61.81359348)(563.39977849,61.83859345)(563.39979097,61.86859579)
\curveto(563.39977849,61.90859338)(563.40477849,61.93859335)(563.41479097,61.95859579)
\lineto(563.41479097,62.10859579)
}
}
\end{pspicture}

\caption{Porcentajes de actividad clasificados por rol}
\label{usuarios_pie_2}
\end{figure}

\subsection{Contactos}
Considerando los indicadores de sociabilidad, puede apreciarse la matriz de
adyacencias (figura \ref{contactos_matriz}) de la red social, puede verse que
los enlaces fuertes son casi exclusividad propia de los desarrolladores, siendo
entre los otros roles predominantes los enlaces débiles. Puede verse también
una sutil relación entre los usuarios que establecieron el nombre de usuario en
su perfil, y los niveles de sociabilidad. Cuya interrelación, es motivo de
seguimiento e intención de demostración.

%\begin{figure}
%\centering
%%LaTeX with PSTricks extensions
%%Creator: inkscape 0.48.5
%%Please note this file requires PSTricks extensions
\psset{xunit=.5pt,yunit=.5pt,runit=.5pt}
\begin{pspicture}(880,545)
{
\newrgbcolor{curcolor}{0.80000001 0.80000001 0.80000001}
\pscustom[linestyle=none,fillstyle=solid,fillcolor=curcolor]
{
\newpath
\moveto(102.59358215,527.46856428)
\lineto(135.89073563,527.46856428)
\lineto(135.89073563,510.44859434)
\lineto(102.59358215,510.44859434)
\closepath
}
}
{
\newrgbcolor{curcolor}{0.80000001 0.80000001 0.80000001}
\pscustom[linestyle=none,fillstyle=solid,fillcolor=curcolor]
{
\newpath
\moveto(102.59358215,510.47228742)
\lineto(135.89073563,510.47228742)
\lineto(135.89073563,493.45231748)
\lineto(102.59358215,493.45231748)
\closepath
}
}
{
\newrgbcolor{curcolor}{0.80000001 0.80000001 0.80000001}
\pscustom[linestyle=none,fillstyle=solid,fillcolor=curcolor]
{
\newpath
\moveto(102.59358215,493.47613264)
\lineto(135.89073563,493.47613264)
\lineto(135.89073563,476.4561627)
\lineto(102.59358215,476.4561627)
\closepath
}
}
{
\newrgbcolor{curcolor}{0.80000001 0.80000001 0.80000001}
\pscustom[linestyle=none,fillstyle=solid,fillcolor=curcolor]
{
\newpath
\moveto(169.12477112,527.46899152)
\lineto(202.42192459,527.46899152)
\lineto(202.42192459,510.44902158)
\lineto(169.12477112,510.44902158)
\closepath
}
}
{
\newrgbcolor{curcolor}{0.80000001 0.80000001 0.80000001}
\pscustom[linestyle=none,fillstyle=solid,fillcolor=curcolor]
{
\newpath
\moveto(169.12480164,510.44903303)
\lineto(202.42195511,510.44903303)
\lineto(202.42195511,493.42906309)
\lineto(169.12480164,493.42906309)
\closepath
}
}
{
\newrgbcolor{curcolor}{0.80000001 0.80000001 0.80000001}
\pscustom[linestyle=none,fillstyle=solid,fillcolor=curcolor]
{
\newpath
\moveto(102.59358215,476.47991682)
\lineto(135.89073563,476.47991682)
\lineto(135.89073563,459.45994688)
\lineto(102.59358215,459.45994688)
\closepath
}
}
{
\newrgbcolor{curcolor}{0.80000001 0.80000001 0.80000001}
\pscustom[linestyle=none,fillstyle=solid,fillcolor=curcolor]
{
\newpath
\moveto(102.59358215,459.483701)
\lineto(135.89073563,459.483701)
\lineto(135.89073563,442.46373106)
\lineto(102.59358215,442.46373106)
\closepath
}
}
{
\newrgbcolor{curcolor}{0.80000001 0.80000001 0.80000001}
\pscustom[linestyle=none,fillstyle=solid,fillcolor=curcolor]
{
\newpath
\moveto(102.59358215,442.48748518)
\lineto(135.89073563,442.48748518)
\lineto(135.89073563,425.46751524)
\lineto(102.59358215,425.46751524)
\closepath
}
}
{
\newrgbcolor{curcolor}{0.80000001 0.80000001 0.80000001}
\pscustom[linestyle=none,fillstyle=solid,fillcolor=curcolor]
{
\newpath
\moveto(102.59358215,425.49126936)
\lineto(135.89073563,425.49126936)
\lineto(135.89073563,408.47129942)
\lineto(102.59358215,408.47129942)
\closepath
}
}
{
\newrgbcolor{curcolor}{0.80000001 0.80000001 0.80000001}
\pscustom[linestyle=none,fillstyle=solid,fillcolor=curcolor]
{
\newpath
\moveto(102.59358215,408.49505354)
\lineto(135.89073563,408.49505354)
\lineto(135.89073563,391.47508359)
\lineto(102.59358215,391.47508359)
\closepath
}
}
{
\newrgbcolor{curcolor}{0.80000001 0.80000001 0.80000001}
\pscustom[linestyle=none,fillstyle=solid,fillcolor=curcolor]
{
\newpath
\moveto(102.59358215,391.49883771)
\lineto(135.89073563,391.49883771)
\lineto(135.89073563,374.47886777)
\lineto(102.59358215,374.47886777)
\closepath
}
}
{
\newrgbcolor{curcolor}{0.80000001 0.80000001 0.80000001}
\pscustom[linestyle=none,fillstyle=solid,fillcolor=curcolor]
{
\newpath
\moveto(102.59358215,374.50262189)
\lineto(135.89073563,374.50262189)
\lineto(135.89073563,357.48265195)
\lineto(102.59358215,357.48265195)
\closepath
}
}
{
\newrgbcolor{curcolor}{0.80000001 0.80000001 0.80000001}
\pscustom[linestyle=none,fillstyle=solid,fillcolor=curcolor]
{
\newpath
\moveto(102.59358215,357.50640607)
\lineto(135.89073563,357.50640607)
\lineto(135.89073563,340.48643613)
\lineto(102.59358215,340.48643613)
\closepath
}
}
{
\newrgbcolor{curcolor}{0.80000001 0.80000001 0.80000001}
\pscustom[linestyle=none,fillstyle=solid,fillcolor=curcolor]
{
\newpath
\moveto(102.59358215,340.51019025)
\lineto(135.89073563,340.51019025)
\lineto(135.89073563,323.49022031)
\lineto(102.59358215,323.49022031)
\closepath
}
}
{
\newrgbcolor{curcolor}{0.80000001 0.80000001 0.80000001}
\pscustom[linestyle=none,fillstyle=solid,fillcolor=curcolor]
{
\newpath
\moveto(102.59358215,323.51397443)
\lineto(135.89073563,323.51397443)
\lineto(135.89073563,306.49400449)
\lineto(102.59358215,306.49400449)
\closepath
}
}
{
\newrgbcolor{curcolor}{0.80000001 0.80000001 0.80000001}
\pscustom[linestyle=none,fillstyle=solid,fillcolor=curcolor]
{
\newpath
\moveto(102.59358215,306.51775861)
\lineto(135.89073563,306.51775861)
\lineto(135.89073563,289.49778867)
\lineto(102.59358215,289.49778867)
\closepath
}
}
{
\newrgbcolor{curcolor}{0.80000001 0.80000001 0.80000001}
\pscustom[linestyle=none,fillstyle=solid,fillcolor=curcolor]
{
\newpath
\moveto(135.89073181,527.46856428)
\lineto(169.18788528,527.46856428)
\lineto(169.18788528,510.44859434)
\lineto(135.89073181,510.44859434)
\closepath
}
}
{
\newrgbcolor{curcolor}{0.80000001 0.80000001 0.80000001}
\pscustom[linestyle=none,fillstyle=solid,fillcolor=curcolor]
{
\newpath
\moveto(135.89073181,510.47228742)
\lineto(169.18788528,510.47228742)
\lineto(169.18788528,493.45231748)
\lineto(135.89073181,493.45231748)
\closepath
}
}
{
\newrgbcolor{curcolor}{0.80000001 0.80000001 0.80000001}
\pscustom[linestyle=none,fillstyle=solid,fillcolor=curcolor]
{
\newpath
\moveto(135.89073181,493.47613264)
\lineto(169.18788528,493.47613264)
\lineto(169.18788528,476.4561627)
\lineto(135.89073181,476.4561627)
\closepath
}
}
{
\newrgbcolor{curcolor}{0.80000001 0.80000001 0.80000001}
\pscustom[linestyle=none,fillstyle=solid,fillcolor=curcolor]
{
\newpath
\moveto(135.89073181,476.47991682)
\lineto(169.18788528,476.47991682)
\lineto(169.18788528,459.45994688)
\lineto(135.89073181,459.45994688)
\closepath
}
}
{
\newrgbcolor{curcolor}{0.80000001 0.80000001 0.80000001}
\pscustom[linestyle=none,fillstyle=solid,fillcolor=curcolor]
{
\newpath
\moveto(135.89073181,459.483701)
\lineto(169.18788528,459.483701)
\lineto(169.18788528,442.46373106)
\lineto(135.89073181,442.46373106)
\closepath
}
}
{
\newrgbcolor{curcolor}{0.80000001 0.80000001 0.80000001}
\pscustom[linestyle=none,fillstyle=solid,fillcolor=curcolor]
{
\newpath
\moveto(135.89073181,442.48748518)
\lineto(169.18788528,442.48748518)
\lineto(169.18788528,425.46751524)
\lineto(135.89073181,425.46751524)
\closepath
}
}
{
\newrgbcolor{curcolor}{0.80000001 0.80000001 0.80000001}
\pscustom[linestyle=none,fillstyle=solid,fillcolor=curcolor]
{
\newpath
\moveto(135.89073181,425.49126936)
\lineto(169.18788528,425.49126936)
\lineto(169.18788528,408.47129942)
\lineto(135.89073181,408.47129942)
\closepath
}
}
{
\newrgbcolor{curcolor}{0.80000001 0.80000001 0.80000001}
\pscustom[linestyle=none,fillstyle=solid,fillcolor=curcolor]
{
\newpath
\moveto(135.89073181,408.49505354)
\lineto(169.18788528,408.49505354)
\lineto(169.18788528,391.47508359)
\lineto(135.89073181,391.47508359)
\closepath
}
}
{
\newrgbcolor{curcolor}{0.80000001 0.80000001 0.80000001}
\pscustom[linestyle=none,fillstyle=solid,fillcolor=curcolor]
{
\newpath
\moveto(135.89073181,391.49883771)
\lineto(169.18788528,391.49883771)
\lineto(169.18788528,374.47886777)
\lineto(135.89073181,374.47886777)
\closepath
}
}
{
\newrgbcolor{curcolor}{0.80000001 0.80000001 0.80000001}
\pscustom[linestyle=none,fillstyle=solid,fillcolor=curcolor]
{
\newpath
\moveto(135.89073181,374.50262189)
\lineto(169.18788528,374.50262189)
\lineto(169.18788528,357.48265195)
\lineto(135.89073181,357.48265195)
\closepath
}
}
{
\newrgbcolor{curcolor}{0.80000001 0.80000001 0.80000001}
\pscustom[linestyle=none,fillstyle=solid,fillcolor=curcolor]
{
\newpath
\moveto(135.89073181,357.50640607)
\lineto(169.18788528,357.50640607)
\lineto(169.18788528,340.48643613)
\lineto(135.89073181,340.48643613)
\closepath
}
}
{
\newrgbcolor{curcolor}{0.80000001 0.80000001 0.80000001}
\pscustom[linestyle=none,fillstyle=solid,fillcolor=curcolor]
{
\newpath
\moveto(135.89073181,340.51019025)
\lineto(169.18788528,340.51019025)
\lineto(169.18788528,323.49022031)
\lineto(135.89073181,323.49022031)
\closepath
}
}
{
\newrgbcolor{curcolor}{0.80000001 0.80000001 0.80000001}
\pscustom[linestyle=none,fillstyle=solid,fillcolor=curcolor]
{
\newpath
\moveto(135.89073181,323.51397443)
\lineto(169.18788528,323.51397443)
\lineto(169.18788528,306.49400449)
\lineto(135.89073181,306.49400449)
\closepath
}
}
{
\newrgbcolor{curcolor}{0.80000001 0.80000001 0.80000001}
\pscustom[linestyle=none,fillstyle=solid,fillcolor=curcolor]
{
\newpath
\moveto(135.89073181,306.51775861)
\lineto(169.18788528,306.51775861)
\lineto(169.18788528,289.49778867)
\lineto(135.89073181,289.49778867)
\closepath
}
}
{
\newrgbcolor{curcolor}{0.80000001 0.80000001 0.80000001}
\pscustom[linestyle=none,fillstyle=solid,fillcolor=curcolor]
{
\newpath
\moveto(135.89073181,289.46856428)
\lineto(169.18788528,289.46856428)
\lineto(169.18788528,272.44859434)
\lineto(135.89073181,272.44859434)
\closepath
}
}
{
\newrgbcolor{curcolor}{0.80000001 0.80000001 0.80000001}
\pscustom[linestyle=none,fillstyle=solid,fillcolor=curcolor]
{
\newpath
\moveto(135.89073181,272.47234846)
\lineto(169.18788528,272.47234846)
\lineto(169.18788528,255.45237852)
\lineto(135.89073181,255.45237852)
\closepath
}
}
{
\newrgbcolor{curcolor}{0.80000001 0.80000001 0.80000001}
\pscustom[linestyle=none,fillstyle=solid,fillcolor=curcolor]
{
\newpath
\moveto(135.89073181,255.47613264)
\lineto(169.18788528,255.47613264)
\lineto(169.18788528,238.4561627)
\lineto(135.89073181,238.4561627)
\closepath
}
}
{
\newrgbcolor{curcolor}{0.80000001 0.80000001 0.80000001}
\pscustom[linestyle=none,fillstyle=solid,fillcolor=curcolor]
{
\newpath
\moveto(135.89073181,238.48003889)
\lineto(169.18788528,238.48003889)
\lineto(169.18788528,221.46006895)
\lineto(135.89073181,221.46006895)
\closepath
}
}
{
\newrgbcolor{curcolor}{0.80000001 0.80000001 0.80000001}
\pscustom[linestyle=none,fillstyle=solid,fillcolor=curcolor]
{
\newpath
\moveto(135.89073181,221.48382307)
\lineto(169.18788528,221.48382307)
\lineto(169.18788528,204.46385313)
\lineto(135.89073181,204.46385313)
\closepath
}
}
{
\newrgbcolor{curcolor}{0.80000001 0.80000001 0.80000001}
\pscustom[linestyle=none,fillstyle=solid,fillcolor=curcolor]
{
\newpath
\moveto(135.89073181,204.48760725)
\lineto(169.18788528,204.48760725)
\lineto(169.18788528,187.46763731)
\lineto(135.89073181,187.46763731)
\closepath
}
}
{
\newrgbcolor{curcolor}{0.80000001 0.80000001 0.80000001}
\pscustom[linestyle=none,fillstyle=solid,fillcolor=curcolor]
{
\newpath
\moveto(135.89073181,187.49139143)
\lineto(169.18788528,187.49139143)
\lineto(169.18788528,170.47142149)
\lineto(135.89073181,170.47142149)
\closepath
}
}
{
\newrgbcolor{curcolor}{0.80000001 0.80000001 0.80000001}
\pscustom[linestyle=none,fillstyle=solid,fillcolor=curcolor]
{
\newpath
\moveto(135.89073181,170.49517561)
\lineto(169.18788528,170.49517561)
\lineto(169.18788528,153.47520567)
\lineto(135.89073181,153.47520567)
\closepath
}
}
{
\newrgbcolor{curcolor}{0.80000001 0.80000001 0.80000001}
\pscustom[linestyle=none,fillstyle=solid,fillcolor=curcolor]
{
\newpath
\moveto(135.89073181,153.49895979)
\lineto(169.18788528,153.49895979)
\lineto(169.18788528,136.47898984)
\lineto(135.89073181,136.47898984)
\closepath
}
}
{
\newrgbcolor{curcolor}{0.80000001 0.80000001 0.80000001}
\pscustom[linestyle=none,fillstyle=solid,fillcolor=curcolor]
{
\newpath
\moveto(135.89073181,136.50274396)
\lineto(169.18788528,136.50274396)
\lineto(169.18788528,119.48277402)
\lineto(135.89073181,119.48277402)
\closepath
}
}
{
\newrgbcolor{curcolor}{0.80000001 0.80000001 0.80000001}
\pscustom[linestyle=none,fillstyle=solid,fillcolor=curcolor]
{
\newpath
\moveto(135.89073181,119.50652814)
\lineto(169.18788528,119.50652814)
\lineto(169.18788528,102.4865582)
\lineto(135.89073181,102.4865582)
\closepath
}
}
{
\newrgbcolor{curcolor}{0.80000001 0.80000001 0.80000001}
\pscustom[linestyle=none,fillstyle=solid,fillcolor=curcolor]
{
\newpath
\moveto(135.89073181,102.51031232)
\lineto(169.18788528,102.51031232)
\lineto(169.18788528,85.49034238)
\lineto(135.89073181,85.49034238)
\closepath
}
}
{
\newrgbcolor{curcolor}{0.80000001 0.80000001 0.80000001}
\pscustom[linestyle=none,fillstyle=solid,fillcolor=curcolor]
{
\newpath
\moveto(135.89073181,68.51788068)
\lineto(169.18788528,68.51788068)
\lineto(169.18788528,51.49791074)
\lineto(135.89073181,51.49791074)
\closepath
}
}
{
\newrgbcolor{curcolor}{0.80000001 0.80000001 0.80000001}
\pscustom[linestyle=none,fillstyle=solid,fillcolor=curcolor]
{
\newpath
\moveto(169.12477112,425.46905256)
\lineto(202.42192459,425.46905256)
\lineto(202.42192459,408.44908262)
\lineto(169.12477112,408.44908262)
\closepath
}
}
{
\newrgbcolor{curcolor}{0.80000001 0.80000001 0.80000001}
\pscustom[linestyle=none,fillstyle=solid,fillcolor=curcolor]
{
\newpath
\moveto(169.12480164,408.44903303)
\lineto(202.42195511,408.44903303)
\lineto(202.42195511,391.42906309)
\lineto(169.12480164,391.42906309)
\closepath
}
}
{
\newrgbcolor{curcolor}{0.80000001 0.80000001 0.80000001}
\pscustom[linestyle=none,fillstyle=solid,fillcolor=curcolor]
{
\newpath
\moveto(169.12480164,374.50799299)
\lineto(202.42195511,374.50799299)
\lineto(202.42195511,357.48802305)
\lineto(169.12480164,357.48802305)
\closepath
}
}
{
\newrgbcolor{curcolor}{0.80000001 0.80000001 0.80000001}
\pscustom[linestyle=none,fillstyle=solid,fillcolor=curcolor]
{
\newpath
\moveto(169.39073181,255.00274396)
\lineto(202.68788528,255.00274396)
\lineto(202.68788528,237.98277402)
\lineto(169.39073181,237.98277402)
\closepath
}
}
{
\newrgbcolor{curcolor}{0.80000001 0.80000001 0.80000001}
\pscustom[linestyle=none,fillstyle=solid,fillcolor=curcolor]
{
\newpath
\moveto(169.39073181,238.00652814)
\lineto(202.68788528,238.00652814)
\lineto(202.68788528,220.9865582)
\lineto(169.39073181,220.9865582)
\closepath
}
}
{
\newrgbcolor{curcolor}{0.80000001 0.80000001 0.80000001}
\pscustom[linestyle=none,fillstyle=solid,fillcolor=curcolor]
{
\newpath
\moveto(169.39073181,221.01031232)
\lineto(202.68788528,221.01031232)
\lineto(202.68788528,203.99034238)
\lineto(169.39073181,203.99034238)
\closepath
}
}
{
\newrgbcolor{curcolor}{0.80000001 0.80000001 0.80000001}
\pscustom[linestyle=none,fillstyle=solid,fillcolor=curcolor]
{
\newpath
\moveto(169.39073181,170.75652814)
\lineto(202.68788528,170.75652814)
\lineto(202.68788528,153.7365582)
\lineto(169.39073181,153.7365582)
\closepath
}
}
{
\newrgbcolor{curcolor}{0.80000001 0.80000001 0.80000001}
\pscustom[linestyle=none,fillstyle=solid,fillcolor=curcolor]
{
\newpath
\moveto(169.39073181,153.76031232)
\lineto(202.68788528,153.76031232)
\lineto(202.68788528,136.74034238)
\lineto(169.39073181,136.74034238)
\closepath
}
}
{
\newrgbcolor{curcolor}{0.80000001 0.80000001 0.80000001}
\pscustom[linestyle=none,fillstyle=solid,fillcolor=curcolor]
{
\newpath
\moveto(169.39073181,85.76031232)
\lineto(202.68788528,85.76031232)
\lineto(202.68788528,68.74034238)
\lineto(169.39073181,68.74034238)
\closepath
}
}
{
\newrgbcolor{curcolor}{0.80000001 0.80000001 0.80000001}
\pscustom[linestyle=none,fillstyle=solid,fillcolor=curcolor]
{
\newpath
\moveto(169.39073181,34.26025129)
\lineto(202.68788528,34.26025129)
\lineto(202.68788528,17.24028135)
\lineto(169.39073181,17.24028135)
\closepath
}
}
{
\newrgbcolor{curcolor}{0.80000001 0.80000001 0.80000001}
\pscustom[linestyle=none,fillstyle=solid,fillcolor=curcolor]
{
\newpath
\moveto(135.89073181,51.52898908)
\lineto(169.18788528,51.52898908)
\lineto(169.18788528,34.50901914)
\lineto(135.89073181,34.50901914)
\closepath
}
}
{
\newrgbcolor{curcolor}{0.80000001 0.80000001 0.80000001}
\pscustom[linestyle=none,fillstyle=solid,fillcolor=curcolor]
{
\newpath
\moveto(135.89073181,34.53277326)
\lineto(169.18788528,34.53277326)
\lineto(169.18788528,17.51280332)
\lineto(135.89073181,17.51280332)
\closepath
}
}
{
\newrgbcolor{curcolor}{0.80000001 0.80000001 0.80000001}
\pscustom[linestyle=none,fillstyle=solid,fillcolor=curcolor]
{
\newpath
\moveto(135.89073181,17.53661848)
\lineto(169.18788528,17.53661848)
\lineto(169.18788528,0.51664854)
\lineto(135.89073181,0.51664854)
\closepath
}
}
{
\newrgbcolor{curcolor}{0.80000001 0.80000001 0.80000001}
\pscustom[linestyle=none,fillstyle=solid,fillcolor=curcolor]
{
\newpath
\moveto(102.59358215,34.53277326)
\lineto(135.89073563,34.53277326)
\lineto(135.89073563,17.51280332)
\lineto(102.59358215,17.51280332)
\closepath
}
}
{
\newrgbcolor{curcolor}{0.80000001 0.80000001 0.80000001}
\pscustom[linestyle=none,fillstyle=solid,fillcolor=curcolor]
{
\newpath
\moveto(102.59358215,17.53661848)
\lineto(135.89073563,17.53661848)
\lineto(135.89073563,0.51664854)
\lineto(102.59358215,0.51664854)
\closepath
}
}
{
\newrgbcolor{curcolor}{0.80000001 0.80000001 0.80000001}
\pscustom[linestyle=none,fillstyle=solid,fillcolor=curcolor]
{
\newpath
\moveto(102.59358215,85.51043439)
\lineto(135.89073563,85.51043439)
\lineto(135.89073563,68.49046445)
\lineto(102.59358215,68.49046445)
\closepath
}
}
{
\newrgbcolor{curcolor}{0.80000001 0.80000001 0.80000001}
\pscustom[linestyle=none,fillstyle=solid,fillcolor=curcolor]
{
\newpath
\moveto(102.59358215,68.51421857)
\lineto(135.89073563,68.51421857)
\lineto(135.89073563,51.49424863)
\lineto(102.59358215,51.49424863)
\closepath
}
}
{
\newrgbcolor{curcolor}{0.80000001 0.80000001 0.80000001}
\pscustom[linestyle=none,fillstyle=solid,fillcolor=curcolor]
{
\newpath
\moveto(102.59358215,119.48028303)
\lineto(135.89073563,119.48028303)
\lineto(135.89073563,102.46031309)
\lineto(102.59358215,102.46031309)
\closepath
}
}
{
\newrgbcolor{curcolor}{0.80000001 0.80000001 0.80000001}
\pscustom[linestyle=none,fillstyle=solid,fillcolor=curcolor]
{
\newpath
\moveto(102.59358215,102.48406721)
\lineto(135.89073563,102.48406721)
\lineto(135.89073563,85.46409727)
\lineto(102.59358215,85.46409727)
\closepath
}
}
{
\newrgbcolor{curcolor}{0.80000001 0.80000001 0.80000001}
\pscustom[linestyle=none,fillstyle=solid,fillcolor=curcolor]
{
\newpath
\moveto(102.59358215,153.44524885)
\lineto(135.89073563,153.44524885)
\lineto(135.89073563,136.42527891)
\lineto(102.59358215,136.42527891)
\closepath
}
}
{
\newrgbcolor{curcolor}{0.80000001 0.80000001 0.80000001}
\pscustom[linestyle=none,fillstyle=solid,fillcolor=curcolor]
{
\newpath
\moveto(102.59358215,136.44903303)
\lineto(135.89073563,136.44903303)
\lineto(135.89073563,119.42906309)
\lineto(102.59358215,119.42906309)
\closepath
}
}
{
\newrgbcolor{curcolor}{0.80000001 0.80000001 0.80000001}
\pscustom[linestyle=none,fillstyle=solid,fillcolor=curcolor]
{
\newpath
\moveto(102.59358215,187.41521955)
\lineto(135.89073563,187.41521955)
\lineto(135.89073563,170.39524961)
\lineto(102.59358215,170.39524961)
\closepath
}
}
{
\newrgbcolor{curcolor}{0.80000001 0.80000001 0.80000001}
\pscustom[linestyle=none,fillstyle=solid,fillcolor=curcolor]
{
\newpath
\moveto(102.59358215,170.41900373)
\lineto(135.89073563,170.41900373)
\lineto(135.89073563,153.39903379)
\lineto(102.59358215,153.39903379)
\closepath
}
}
{
\newrgbcolor{curcolor}{0.80000001 0.80000001 0.80000001}
\pscustom[linestyle=none,fillstyle=solid,fillcolor=curcolor]
{
\newpath
\moveto(102.59358215,272.48028303)
\lineto(135.89073563,272.48028303)
\lineto(135.89073563,255.46031309)
\lineto(102.59358215,255.46031309)
\closepath
}
}
{
\newrgbcolor{curcolor}{0.80000001 0.80000001 0.80000001}
\pscustom[linestyle=none,fillstyle=solid,fillcolor=curcolor]
{
\newpath
\moveto(102.59358215,221.50030256)
\lineto(135.89073563,221.50030256)
\lineto(135.89073563,204.48033262)
\lineto(102.59358215,204.48033262)
\closepath
}
}
{
\newrgbcolor{curcolor}{0.3019608 0.3019608 0.3019608}
\pscustom[linestyle=none,fillstyle=solid,fillcolor=curcolor]
{
\newpath
\moveto(335.53973389,527.50445295)
\lineto(353.08185196,527.50445295)
\lineto(353.08185196,510.48211408)
\lineto(335.53973389,510.48211408)
\closepath
}
}
{
\newrgbcolor{curcolor}{0.3019608 0.3019608 0.3019608}
\pscustom[linestyle=none,fillstyle=solid,fillcolor=curcolor]
{
\newpath
\moveto(353.0246582,510.53686262)
\lineto(370.56677628,510.53686262)
\lineto(370.56677628,493.51452375)
\lineto(353.0246582,493.51452375)
\closepath
}
}
{
\newrgbcolor{curcolor}{0.3019608 0.3019608 0.3019608}
\pscustom[linestyle=none,fillstyle=solid,fillcolor=curcolor]
{
\newpath
\moveto(370.67959595,493.50915266)
\lineto(388.22171402,493.50915266)
\lineto(388.22171402,476.48681379)
\lineto(370.67959595,476.48681379)
\closepath
}
}
{
\newrgbcolor{curcolor}{0.3019608 0.3019608 0.3019608}
\pscustom[linestyle=none,fillstyle=solid,fillcolor=curcolor]
{
\newpath
\moveto(388.16455078,476.54150129)
\lineto(405.70666885,476.54150129)
\lineto(405.70666885,459.51916242)
\lineto(388.16455078,459.51916242)
\closepath
}
}
{
\newrgbcolor{curcolor}{0.3019608 0.3019608 0.3019608}
\pscustom[linestyle=none,fillstyle=solid,fillcolor=curcolor]
{
\newpath
\moveto(405.78027344,459.52435041)
\lineto(423.32239151,459.52435041)
\lineto(423.32239151,442.50201154)
\lineto(405.78027344,442.50201154)
\closepath
}
}
{
\newrgbcolor{curcolor}{0.3019608 0.3019608 0.3019608}
\pscustom[linestyle=none,fillstyle=solid,fillcolor=curcolor]
{
\newpath
\moveto(423.26513672,442.55669904)
\lineto(440.80725479,442.55669904)
\lineto(440.80725479,425.53436018)
\lineto(423.26513672,425.53436018)
\closepath
}
}
{
\newrgbcolor{curcolor}{0.3019608 0.3019608 0.3019608}
\pscustom[linestyle=none,fillstyle=solid,fillcolor=curcolor]
{
\newpath
\moveto(440.9201355,425.52917219)
\lineto(458.46225357,425.52917219)
\lineto(458.46225357,408.50683332)
\lineto(440.9201355,408.50683332)
\closepath
}
}
{
\newrgbcolor{curcolor}{0.3019608 0.3019608 0.3019608}
\pscustom[linestyle=none,fillstyle=solid,fillcolor=curcolor]
{
\newpath
\moveto(458.4050293,408.56145979)
\lineto(475.94714737,408.56145979)
\lineto(475.94714737,391.53912092)
\lineto(458.4050293,391.53912092)
\closepath
}
}
{
\newrgbcolor{curcolor}{0.3019608 0.3019608 0.3019608}
\pscustom[linestyle=none,fillstyle=solid,fillcolor=curcolor]
{
\newpath
\moveto(475.95471191,391.48327375)
\lineto(493.49682999,391.48327375)
\lineto(493.49682999,374.46093488)
\lineto(475.95471191,374.46093488)
\closepath
}
}
{
\newrgbcolor{curcolor}{0.3019608 0.3019608 0.3019608}
\pscustom[linestyle=none,fillstyle=solid,fillcolor=curcolor]
{
\newpath
\moveto(493.4395752,374.51543928)
\lineto(510.98169327,374.51543928)
\lineto(510.98169327,357.49310041)
\lineto(493.4395752,357.49310041)
\closepath
}
}
{
\newrgbcolor{curcolor}{0.3019608 0.3019608 0.3019608}
\pscustom[linestyle=none,fillstyle=solid,fillcolor=curcolor]
{
\newpath
\moveto(511.09457397,357.48797346)
\lineto(528.63669205,357.48797346)
\lineto(528.63669205,340.46563459)
\lineto(511.09457397,340.46563459)
\closepath
}
}
{
\newrgbcolor{curcolor}{0.3019608 0.3019608 0.3019608}
\pscustom[linestyle=none,fillstyle=solid,fillcolor=curcolor]
{
\newpath
\moveto(528.57946777,340.52026105)
\lineto(546.12158585,340.52026105)
\lineto(546.12158585,323.49792219)
\lineto(528.57946777,323.49792219)
\closepath
}
}
{
\newrgbcolor{curcolor}{0.3019608 0.3019608 0.3019608}
\pscustom[linestyle=none,fillstyle=solid,fillcolor=curcolor]
{
\newpath
\moveto(546.19525146,323.50311018)
\lineto(563.73736954,323.50311018)
\lineto(563.73736954,306.48077131)
\lineto(546.19525146,306.48077131)
\closepath
}
}
{
\newrgbcolor{curcolor}{0.3019608 0.3019608 0.3019608}
\pscustom[linestyle=none,fillstyle=solid,fillcolor=curcolor]
{
\newpath
\moveto(563.68011475,306.53545881)
\lineto(581.22223282,306.53545881)
\lineto(581.22223282,289.51311994)
\lineto(563.68011475,289.51311994)
\closepath
}
}
{
\newrgbcolor{curcolor}{0.3019608 0.3019608 0.3019608}
\pscustom[linestyle=none,fillstyle=solid,fillcolor=curcolor]
{
\newpath
\moveto(581.33514404,289.50793195)
\lineto(598.87726212,289.50793195)
\lineto(598.87726212,272.48559309)
\lineto(581.33514404,272.48559309)
\closepath
}
}
{
\newrgbcolor{curcolor}{0.3019608 0.3019608 0.3019608}
\pscustom[linestyle=none,fillstyle=solid,fillcolor=curcolor]
{
\newpath
\moveto(598.82000732,272.54021955)
\lineto(616.3621254,272.54021955)
\lineto(616.3621254,255.51788068)
\lineto(598.82000732,255.51788068)
\closepath
}
}
{
\newrgbcolor{curcolor}{0.3019608 0.3019608 0.3019608}
\pscustom[linestyle=none,fillstyle=solid,fillcolor=curcolor]
{
\newpath
\moveto(616.3694458,255.42443586)
\lineto(633.91156387,255.42443586)
\lineto(633.91156387,238.40209699)
\lineto(616.3694458,238.40209699)
\closepath
}
}
{
\newrgbcolor{curcolor}{0.3019608 0.3019608 0.3019608}
\pscustom[linestyle=none,fillstyle=solid,fillcolor=curcolor]
{
\newpath
\moveto(633.85437012,238.45672346)
\lineto(651.39648819,238.45672346)
\lineto(651.39648819,221.43438459)
\lineto(633.85437012,221.43438459)
\closepath
}
}
{
\newrgbcolor{curcolor}{0.3019608 0.3019608 0.3019608}
\pscustom[linestyle=none,fillstyle=solid,fillcolor=curcolor]
{
\newpath
\moveto(651.50933838,221.42913557)
\lineto(669.05145645,221.42913557)
\lineto(669.05145645,204.4067967)
\lineto(651.50933838,204.4067967)
\closepath
}
}
{
\newrgbcolor{curcolor}{0.3019608 0.3019608 0.3019608}
\pscustom[linestyle=none,fillstyle=solid,fillcolor=curcolor]
{
\newpath
\moveto(668.99420166,204.46142316)
\lineto(686.53631973,204.46142316)
\lineto(686.53631973,187.4390843)
\lineto(668.99420166,187.4390843)
\closepath
}
}
{
\newrgbcolor{curcolor}{0.3019608 0.3019608 0.3019608}
\pscustom[linestyle=none,fillstyle=solid,fillcolor=curcolor]
{
\newpath
\moveto(686.61004639,187.44427229)
\lineto(704.15216446,187.44427229)
\lineto(704.15216446,170.42193342)
\lineto(686.61004639,170.42193342)
\closepath
}
}
{
\newrgbcolor{curcolor}{0.3019608 0.3019608 0.3019608}
\pscustom[linestyle=none,fillstyle=solid,fillcolor=curcolor]
{
\newpath
\moveto(704.09490967,170.47655988)
\lineto(721.63702774,170.47655988)
\lineto(721.63702774,153.45422102)
\lineto(704.09490967,153.45422102)
\closepath
}
}
{
\newrgbcolor{curcolor}{0.3019608 0.3019608 0.3019608}
\pscustom[linestyle=none,fillstyle=solid,fillcolor=curcolor]
{
\newpath
\moveto(721.74987793,153.44909406)
\lineto(739.291996,153.44909406)
\lineto(739.291996,136.4267552)
\lineto(721.74987793,136.4267552)
\closepath
}
}
{
\newrgbcolor{curcolor}{0.3019608 0.3019608 0.3019608}
\pscustom[linestyle=none,fillstyle=solid,fillcolor=curcolor]
{
\newpath
\moveto(739.23480225,136.48138166)
\lineto(756.77692032,136.48138166)
\lineto(756.77692032,119.45904279)
\lineto(739.23480225,119.45904279)
\closepath
}
}
{
\newrgbcolor{curcolor}{0.3019608 0.3019608 0.3019608}
\pscustom[linestyle=none,fillstyle=solid,fillcolor=curcolor]
{
\newpath
\moveto(756.83251953,119.45898176)
\lineto(774.3746376,119.45898176)
\lineto(774.3746376,102.43664289)
\lineto(756.83251953,102.43664289)
\closepath
}
}
{
\newrgbcolor{curcolor}{0.3019608 0.3019608 0.3019608}
\pscustom[linestyle=none,fillstyle=solid,fillcolor=curcolor]
{
\newpath
\moveto(774.31738281,102.49126936)
\lineto(791.85950089,102.49126936)
\lineto(791.85950089,85.46893049)
\lineto(774.31738281,85.46893049)
\closepath
}
}
{
\newrgbcolor{curcolor}{0.3019608 0.3019608 0.3019608}
\pscustom[linestyle=none,fillstyle=solid,fillcolor=curcolor]
{
\newpath
\moveto(791.97241211,85.46380354)
\lineto(809.51453018,85.46380354)
\lineto(809.51453018,68.44146467)
\lineto(791.97241211,68.44146467)
\closepath
}
}
{
\newrgbcolor{curcolor}{0.3019608 0.3019608 0.3019608}
\pscustom[linestyle=none,fillstyle=solid,fillcolor=curcolor]
{
\newpath
\moveto(809.45727539,68.49609113)
\lineto(826.99939346,68.49609113)
\lineto(826.99939346,51.47375227)
\lineto(809.45727539,51.47375227)
\closepath
}
}
{
\newrgbcolor{curcolor}{0.3019608 0.3019608 0.3019608}
\pscustom[linestyle=none,fillstyle=solid,fillcolor=curcolor]
{
\newpath
\moveto(826.92254639,51.4501927)
\lineto(844.46466446,51.4501927)
\lineto(844.46466446,34.42785383)
\lineto(826.92254639,34.42785383)
\closepath
}
}
{
\newrgbcolor{curcolor}{0.3019608 0.3019608 0.3019608}
\pscustom[linestyle=none,fillstyle=solid,fillcolor=curcolor]
{
\newpath
\moveto(844.4074707,34.48248029)
\lineto(861.94958878,34.48248029)
\lineto(861.94958878,17.46014143)
\lineto(844.4074707,17.46014143)
\closepath
}
}
{
\newrgbcolor{curcolor}{0.3019608 0.3019608 0.3019608}
\pscustom[linestyle=none,fillstyle=solid,fillcolor=curcolor]
{
\newpath
\moveto(862.06243896,17.45495344)
\lineto(879.60455704,17.45495344)
\lineto(879.60455704,0.43261457)
\lineto(862.06243896,0.43261457)
\closepath
}
}
{
\newrgbcolor{curcolor}{0.80000001 0.80000001 0.80000001}
\pscustom[linestyle=none,fillstyle=solid,fillcolor=curcolor]
{
\newpath
\moveto(353.0246582,527.50445295)
\lineto(370.56677628,527.50445295)
\lineto(370.56677628,510.48211408)
\lineto(353.0246582,510.48211408)
\closepath
}
}
{
\newrgbcolor{curcolor}{0.80000001 0.80000001 0.80000001}
\pscustom[linestyle=none,fillstyle=solid,fillcolor=curcolor]
{
\newpath
\moveto(370.62237549,527.50445295)
\lineto(388.16449356,527.50445295)
\lineto(388.16449356,510.48211408)
\lineto(370.62237549,510.48211408)
\closepath
}
}
{
\newrgbcolor{curcolor}{0.80000001 0.80000001 0.80000001}
\pscustom[linestyle=none,fillstyle=solid,fillcolor=curcolor]
{
\newpath
\moveto(405.74261475,510.53686262)
\lineto(423.28473282,510.53686262)
\lineto(423.28473282,493.51452375)
\lineto(405.74261475,493.51452375)
\closepath
}
}
{
\newrgbcolor{curcolor}{0.80000001 0.80000001 0.80000001}
\pscustom[linestyle=none,fillstyle=solid,fillcolor=curcolor]
{
\newpath
\moveto(335.53973389,510.53686262)
\lineto(353.08185196,510.53686262)
\lineto(353.08185196,493.51452375)
\lineto(335.53973389,493.51452375)
\closepath
}
}
{
\newrgbcolor{curcolor}{0.80000001 0.80000001 0.80000001}
\pscustom[linestyle=none,fillstyle=solid,fillcolor=curcolor]
{
\newpath
\moveto(335.53973389,493.56384016)
\lineto(353.08185196,493.56384016)
\lineto(353.08185196,476.54150129)
\lineto(335.53973389,476.54150129)
\closepath
}
}
{
\newrgbcolor{curcolor}{0.80000001 0.80000001 0.80000001}
\pscustom[linestyle=none,fillstyle=solid,fillcolor=curcolor]
{
\newpath
\moveto(440.82525635,510.53686262)
\lineto(458.36737442,510.53686262)
\lineto(458.36737442,493.51452375)
\lineto(440.82525635,493.51452375)
\closepath
}
}
{
\newrgbcolor{curcolor}{0.80000001 0.80000001 0.80000001}
\pscustom[linestyle=none,fillstyle=solid,fillcolor=curcolor]
{
\newpath
\moveto(458.38543701,510.53686262)
\lineto(475.92755508,510.53686262)
\lineto(475.92755508,493.51452375)
\lineto(458.38543701,493.51452375)
\closepath
}
}
{
\newrgbcolor{curcolor}{0.80000001 0.80000001 0.80000001}
\pscustom[linestyle=none,fillstyle=solid,fillcolor=curcolor]
{
\newpath
\moveto(458.38543701,527.50445295)
\lineto(475.92755508,527.50445295)
\lineto(475.92755508,510.48211408)
\lineto(458.38543701,510.48211408)
\closepath
}
}
{
\newrgbcolor{curcolor}{0.80000001 0.80000001 0.80000001}
\pscustom[linestyle=none,fillstyle=solid,fillcolor=curcolor]
{
\newpath
\moveto(598.80004883,527.50445295)
\lineto(616.3421669,527.50445295)
\lineto(616.3421669,510.48211408)
\lineto(598.80004883,510.48211408)
\closepath
}
}
{
\newrgbcolor{curcolor}{0.80000001 0.80000001 0.80000001}
\pscustom[linestyle=none,fillstyle=solid,fillcolor=curcolor]
{
\newpath
\moveto(651.47216797,527.50445295)
\lineto(669.01428604,527.50445295)
\lineto(669.01428604,510.48211408)
\lineto(651.47216797,510.48211408)
\closepath
}
}
{
\newrgbcolor{curcolor}{0.80000001 0.80000001 0.80000001}
\pscustom[linestyle=none,fillstyle=solid,fillcolor=curcolor]
{
\newpath
\moveto(669.01428223,527.50445295)
\lineto(686.5564003,527.50445295)
\lineto(686.5564003,510.48211408)
\lineto(669.01428223,510.48211408)
\closepath
}
}
{
\newrgbcolor{curcolor}{0.80000001 0.80000001 0.80000001}
\pscustom[linestyle=none,fillstyle=solid,fillcolor=curcolor]
{
\newpath
\moveto(651.46575928,510.53686262)
\lineto(669.00787735,510.53686262)
\lineto(669.00787735,493.51452375)
\lineto(651.46575928,493.51452375)
\closepath
}
}
{
\newrgbcolor{curcolor}{0.80000001 0.80000001 0.80000001}
\pscustom[linestyle=none,fillstyle=solid,fillcolor=curcolor]
{
\newpath
\moveto(809.43774414,527.50445295)
\lineto(826.97986221,527.50445295)
\lineto(826.97986221,510.48211408)
\lineto(809.43774414,510.48211408)
\closepath
}
}
{
\newrgbcolor{curcolor}{0.80000001 0.80000001 0.80000001}
\pscustom[linestyle=none,fillstyle=solid,fillcolor=curcolor]
{
\newpath
\moveto(862.0993042,510.53686262)
\lineto(879.64142227,510.53686262)
\lineto(879.64142227,493.51452375)
\lineto(862.0993042,493.51452375)
\closepath
}
}
{
\newrgbcolor{curcolor}{0.80000001 0.80000001 0.80000001}
\pscustom[linestyle=none,fillstyle=solid,fillcolor=curcolor]
{
\newpath
\moveto(353.0246582,459.50750471)
\lineto(370.56677628,459.50750471)
\lineto(370.56677628,442.48516584)
\lineto(353.0246582,442.48516584)
\closepath
}
}
{
\newrgbcolor{curcolor}{0.80000001 0.80000001 0.80000001}
\pscustom[linestyle=none,fillstyle=solid,fillcolor=curcolor]
{
\newpath
\moveto(353.08187866,425.51745344)
\lineto(370.62399673,425.51745344)
\lineto(370.62399673,408.49511457)
\lineto(353.08187866,408.49511457)
\closepath
}
}
{
\newrgbcolor{curcolor}{0.80000001 0.80000001 0.80000001}
\pscustom[linestyle=none,fillstyle=solid,fillcolor=curcolor]
{
\newpath
\moveto(353.08187866,408.47820783)
\lineto(370.62399673,408.47820783)
\lineto(370.62399673,391.45586896)
\lineto(353.08187866,391.45586896)
\closepath
}
}
{
\newrgbcolor{curcolor}{0.80000001 0.80000001 0.80000001}
\pscustom[linestyle=none,fillstyle=solid,fillcolor=curcolor]
{
\newpath
\moveto(335.53973389,408.47820783)
\lineto(353.08185196,408.47820783)
\lineto(353.08185196,391.45586896)
\lineto(335.53973389,391.45586896)
\closepath
}
}
{
\newrgbcolor{curcolor}{0.80000001 0.80000001 0.80000001}
\pscustom[linestyle=none,fillstyle=solid,fillcolor=curcolor]
{
\newpath
\moveto(335.53973389,272.48028303)
\lineto(353.08185196,272.48028303)
\lineto(353.08185196,255.45794416)
\lineto(335.53973389,255.45794416)
\closepath
}
}
{
\newrgbcolor{curcolor}{0.80000001 0.80000001 0.80000001}
\pscustom[linestyle=none,fillstyle=solid,fillcolor=curcolor]
{
\newpath
\moveto(335.53973389,221.47320295)
\lineto(353.08185196,221.47320295)
\lineto(353.08185196,204.45086408)
\lineto(335.53973389,204.45086408)
\closepath
}
}
{
\newrgbcolor{curcolor}{0.80000001 0.80000001 0.80000001}
\pscustom[linestyle=none,fillstyle=solid,fillcolor=curcolor]
{
\newpath
\moveto(353.0246582,221.47320295)
\lineto(370.56677628,221.47320295)
\lineto(370.56677628,204.45086408)
\lineto(353.0246582,204.45086408)
\closepath
}
}
{
\newrgbcolor{curcolor}{0.80000001 0.80000001 0.80000001}
\pscustom[linestyle=none,fillstyle=solid,fillcolor=curcolor]
{
\newpath
\moveto(335.53973389,204.49084211)
\lineto(353.08185196,204.49084211)
\lineto(353.08185196,187.46850324)
\lineto(335.53973389,187.46850324)
\closepath
}
}
{
\newrgbcolor{curcolor}{0.80000001 0.80000001 0.80000001}
\pscustom[linestyle=none,fillstyle=solid,fillcolor=curcolor]
{
\newpath
\moveto(335.53973389,68.44518781)
\lineto(353.08185196,68.44518781)
\lineto(353.08185196,51.42284895)
\lineto(335.53973389,51.42284895)
\closepath
}
}
{
\newrgbcolor{curcolor}{0.80000001 0.80000001 0.80000001}
\pscustom[linestyle=none,fillstyle=solid,fillcolor=curcolor]
{
\newpath
\moveto(353.0246582,17.45495344)
\lineto(370.56677628,17.45495344)
\lineto(370.56677628,0.43261457)
\lineto(353.0246582,0.43261457)
\closepath
}
}
{
\newrgbcolor{curcolor}{0.80000001 0.80000001 0.80000001}
\pscustom[linestyle=none,fillstyle=solid,fillcolor=curcolor]
{
\newpath
\moveto(616.34851074,238.47552229)
\lineto(633.89062881,238.47552229)
\lineto(633.89062881,221.45318342)
\lineto(616.34851074,221.45318342)
\closepath
}
}
{
\newrgbcolor{curcolor}{0.80000001 0.80000001 0.80000001}
\pscustom[linestyle=none,fillstyle=solid,fillcolor=curcolor]
{
\newpath
\moveto(633.92364502,255.50915266)
\lineto(651.46576309,255.50915266)
\lineto(651.46576309,238.48681379)
\lineto(633.92364502,238.48681379)
\closepath
}
}
{
\newrgbcolor{curcolor}{0.50196081 0.50196081 0.50196081}
\pscustom[linestyle=none,fillstyle=solid,fillcolor=curcolor]
{
\newpath
\moveto(370.62237549,510.53686262)
\lineto(388.16449356,510.53686262)
\lineto(388.16449356,493.51452375)
\lineto(370.62237549,493.51452375)
\closepath
}
}
{
\newrgbcolor{curcolor}{0.50196081 0.50196081 0.50196081}
\pscustom[linestyle=none,fillstyle=solid,fillcolor=curcolor]
{
\newpath
\moveto(388.14489746,510.53686262)
\lineto(405.68701553,510.53686262)
\lineto(405.68701553,493.51452375)
\lineto(388.14489746,493.51452375)
\closepath
}
}
{
\newrgbcolor{curcolor}{0.50196081 0.50196081 0.50196081}
\pscustom[linestyle=none,fillstyle=solid,fillcolor=curcolor]
{
\newpath
\moveto(388.14489746,527.50445295)
\lineto(405.68701553,527.50445295)
\lineto(405.68701553,510.48211408)
\lineto(388.14489746,510.48211408)
\closepath
}
}
{
\newrgbcolor{curcolor}{0.50196081 0.50196081 0.50196081}
\pscustom[linestyle=none,fillstyle=solid,fillcolor=curcolor]
{
\newpath
\moveto(423.26513672,510.53686262)
\lineto(440.80725479,510.53686262)
\lineto(440.80725479,493.51452375)
\lineto(423.26513672,493.51452375)
\closepath
}
}
{
\newrgbcolor{curcolor}{0.50196081 0.50196081 0.50196081}
\pscustom[linestyle=none,fillstyle=solid,fillcolor=curcolor]
{
\newpath
\moveto(475.92755127,510.53686262)
\lineto(493.46966934,510.53686262)
\lineto(493.46966934,493.51452375)
\lineto(475.92755127,493.51452375)
\closepath
}
}
{
\newrgbcolor{curcolor}{0.50196081 0.50196081 0.50196081}
\pscustom[linestyle=none,fillstyle=solid,fillcolor=curcolor]
{
\newpath
\moveto(493.47613525,510.53686262)
\lineto(511.01825333,510.53686262)
\lineto(511.01825333,493.51452375)
\lineto(493.47613525,493.51452375)
\closepath
}
}
{
\newrgbcolor{curcolor}{0.50196081 0.50196081 0.50196081}
\pscustom[linestyle=none,fillstyle=solid,fillcolor=curcolor]
{
\newpath
\moveto(493.48254395,527.50445295)
\lineto(511.02466202,527.50445295)
\lineto(511.02466202,510.48211408)
\lineto(493.48254395,510.48211408)
\closepath
}
}
{
\newrgbcolor{curcolor}{0.50196081 0.50196081 0.50196081}
\pscustom[linestyle=none,fillstyle=solid,fillcolor=curcolor]
{
\newpath
\moveto(511.03112793,527.50445295)
\lineto(528.573246,527.50445295)
\lineto(528.573246,510.48211408)
\lineto(511.03112793,510.48211408)
\closepath
}
}
{
\newrgbcolor{curcolor}{0.50196081 0.50196081 0.50196081}
\pscustom[linestyle=none,fillstyle=solid,fillcolor=curcolor]
{
\newpath
\moveto(528.57952881,527.50445295)
\lineto(546.12164688,527.50445295)
\lineto(546.12164688,510.48211408)
\lineto(528.57952881,510.48211408)
\closepath
}
}
{
\newrgbcolor{curcolor}{0.50196081 0.50196081 0.50196081}
\pscustom[linestyle=none,fillstyle=solid,fillcolor=curcolor]
{
\newpath
\moveto(546.12811279,527.50445295)
\lineto(563.67023087,527.50445295)
\lineto(563.67023087,510.48211408)
\lineto(546.12811279,510.48211408)
\closepath
}
}
{
\newrgbcolor{curcolor}{0.50196081 0.50196081 0.50196081}
\pscustom[linestyle=none,fillstyle=solid,fillcolor=curcolor]
{
\newpath
\moveto(511.02462769,510.53686262)
\lineto(528.56674576,510.53686262)
\lineto(528.56674576,493.51452375)
\lineto(511.02462769,493.51452375)
\closepath
}
}
{
\newrgbcolor{curcolor}{0.50196081 0.50196081 0.50196081}
\pscustom[linestyle=none,fillstyle=solid,fillcolor=curcolor]
{
\newpath
\moveto(563.70318604,527.44988752)
\lineto(581.24530411,527.44988752)
\lineto(581.24530411,510.42754865)
\lineto(563.70318604,510.42754865)
\closepath
}
}
{
\newrgbcolor{curcolor}{0.50196081 0.50196081 0.50196081}
\pscustom[linestyle=none,fillstyle=solid,fillcolor=curcolor]
{
\newpath
\moveto(563.69671631,510.48211408)
\lineto(581.23883438,510.48211408)
\lineto(581.23883438,493.45977521)
\lineto(563.69671631,493.45977521)
\closepath
}
}
{
\newrgbcolor{curcolor}{0.50196081 0.50196081 0.50196081}
\pscustom[linestyle=none,fillstyle=solid,fillcolor=curcolor]
{
\newpath
\moveto(598.80004883,510.48211408)
\lineto(616.3421669,510.48211408)
\lineto(616.3421669,493.45977521)
\lineto(598.80004883,493.45977521)
\closepath
}
}
{
\newrgbcolor{curcolor}{0.50196081 0.50196081 0.50196081}
\pscustom[linestyle=none,fillstyle=solid,fillcolor=curcolor]
{
\newpath
\moveto(686.56915283,510.53686262)
\lineto(704.1112709,510.53686262)
\lineto(704.1112709,493.51452375)
\lineto(686.56915283,493.51452375)
\closepath
}
}
{
\newrgbcolor{curcolor}{0.50196081 0.50196081 0.50196081}
\pscustom[linestyle=none,fillstyle=solid,fillcolor=curcolor]
{
\newpath
\moveto(704.11767578,527.50445295)
\lineto(721.65979385,527.50445295)
\lineto(721.65979385,510.48211408)
\lineto(704.11767578,510.48211408)
\closepath
}
}
{
\newrgbcolor{curcolor}{0.50196081 0.50196081 0.50196081}
\pscustom[linestyle=none,fillstyle=solid,fillcolor=curcolor]
{
\newpath
\moveto(739.23480225,527.50445295)
\lineto(756.77692032,527.50445295)
\lineto(756.77692032,510.48211408)
\lineto(739.23480225,510.48211408)
\closepath
}
}
{
\newrgbcolor{curcolor}{0.50196081 0.50196081 0.50196081}
\pscustom[linestyle=none,fillstyle=solid,fillcolor=curcolor]
{
\newpath
\moveto(756.83251953,527.50445295)
\lineto(774.3746376,527.50445295)
\lineto(774.3746376,510.48211408)
\lineto(756.83251953,510.48211408)
\closepath
}
}
{
\newrgbcolor{curcolor}{0.50196081 0.50196081 0.50196081}
\pscustom[linestyle=none,fillstyle=solid,fillcolor=curcolor]
{
\newpath
\moveto(774.31750488,527.50445295)
\lineto(791.85962296,527.50445295)
\lineto(791.85962296,510.48211408)
\lineto(774.31750488,510.48211408)
\closepath
}
}
{
\newrgbcolor{curcolor}{0.50196081 0.50196081 0.50196081}
\pscustom[linestyle=none,fillstyle=solid,fillcolor=curcolor]
{
\newpath
\moveto(475.93405151,408.47820783)
\lineto(493.47616959,408.47820783)
\lineto(493.47616959,391.45586896)
\lineto(475.93405151,391.45586896)
\closepath
}
}
{
\newrgbcolor{curcolor}{0.50196081 0.50196081 0.50196081}
\pscustom[linestyle=none,fillstyle=solid,fillcolor=curcolor]
{
\newpath
\moveto(563.703125,289.50268293)
\lineto(581.24524307,289.50268293)
\lineto(581.24524307,272.48034406)
\lineto(563.703125,272.48034406)
\closepath
}
}
{
\newrgbcolor{curcolor}{0.50196081 0.50196081 0.50196081}
\pscustom[linestyle=none,fillstyle=solid,fillcolor=curcolor]
{
\newpath
\moveto(563.69671631,272.53155256)
\lineto(581.23883438,272.53155256)
\lineto(581.23883438,255.50921369)
\lineto(563.69671631,255.50921369)
\closepath
}
}
{
\newrgbcolor{curcolor}{0.50196081 0.50196081 0.50196081}
\pscustom[linestyle=none,fillstyle=solid,fillcolor=curcolor]
{
\newpath
\moveto(581.2845459,255.44665266)
\lineto(598.82666397,255.44665266)
\lineto(598.82666397,238.42431379)
\lineto(581.2845459,238.42431379)
\closepath
}
}
{
\newrgbcolor{curcolor}{0.50196081 0.50196081 0.50196081}
\pscustom[linestyle=none,fillstyle=solid,fillcolor=curcolor]
{
\newpath
\moveto(581.27819824,238.47552229)
\lineto(598.82031631,238.47552229)
\lineto(598.82031631,221.45318342)
\lineto(581.27819824,221.45318342)
\closepath
}
}
{
\newrgbcolor{curcolor}{0.50196081 0.50196081 0.50196081}
\pscustom[linestyle=none,fillstyle=solid,fillcolor=curcolor]
{
\newpath
\moveto(704.11767578,238.49554182)
\lineto(721.65979385,238.49554182)
\lineto(721.65979385,221.47320295)
\lineto(704.11767578,221.47320295)
\closepath
}
}
{
\newrgbcolor{curcolor}{0.50196081 0.50196081 0.50196081}
\pscustom[linestyle=none,fillstyle=solid,fillcolor=curcolor]
{
\newpath
\moveto(704.11126709,204.49084211)
\lineto(721.65338516,204.49084211)
\lineto(721.65338516,187.46850324)
\lineto(704.11126709,187.46850324)
\closepath
}
}
{
\newrgbcolor{curcolor}{0.50196081 0.50196081 0.50196081}
\pscustom[linestyle=none,fillstyle=solid,fillcolor=curcolor]
{
\newpath
\moveto(721.67468262,238.49554182)
\lineto(739.21680069,238.49554182)
\lineto(739.21680069,221.47320295)
\lineto(721.67468262,221.47320295)
\closepath
}
}
{
\newrgbcolor{curcolor}{0.50196081 0.50196081 0.50196081}
\pscustom[linestyle=none,fillstyle=solid,fillcolor=curcolor]
{
\newpath
\moveto(826.97979736,68.44518781)
\lineto(844.52191544,68.44518781)
\lineto(844.52191544,51.42284895)
\lineto(826.97979736,51.42284895)
\closepath
}
}
{
\newrgbcolor{curcolor}{0.50196081 0.50196081 0.50196081}
\pscustom[linestyle=none,fillstyle=solid,fillcolor=curcolor]
{
\newpath
\moveto(844.5368042,68.44518781)
\lineto(862.07892227,68.44518781)
\lineto(862.07892227,51.42284895)
\lineto(844.5368042,51.42284895)
\closepath
}
}
{
\newrgbcolor{curcolor}{0.50196081 0.50196081 0.50196081}
\pscustom[linestyle=none,fillstyle=solid,fillcolor=curcolor]
{
\newpath
\moveto(704.11126709,68.46203352)
\lineto(721.65338516,68.46203352)
\lineto(721.65338516,51.43969465)
\lineto(704.11126709,51.43969465)
\closepath
}
}
{
\newrgbcolor{curcolor}{0.50196081 0.50196081 0.50196081}
\pscustom[linestyle=none,fillstyle=solid,fillcolor=curcolor]
{
\newpath
\moveto(721.66827393,68.46203352)
\lineto(739.210392,68.46203352)
\lineto(739.210392,51.43969465)
\lineto(721.66827393,51.43969465)
\closepath
}
}
{
\newrgbcolor{curcolor}{0.50196081 0.50196081 0.50196081}
\pscustom[linestyle=none,fillstyle=solid,fillcolor=curcolor]
{
\newpath
\moveto(704.11767578,85.48443342)
\lineto(721.65979385,85.48443342)
\lineto(721.65979385,68.46209455)
\lineto(704.11767578,68.46209455)
\closepath
}
}
{
\newrgbcolor{curcolor}{0.50196081 0.50196081 0.50196081}
\pscustom[linestyle=none,fillstyle=solid,fillcolor=curcolor]
{
\newpath
\moveto(563.69671631,68.46203352)
\lineto(581.23883438,68.46203352)
\lineto(581.23883438,51.43969465)
\lineto(563.69671631,51.43969465)
\closepath
}
}
{
\newrgbcolor{curcolor}{0.50196081 0.50196081 0.50196081}
\pscustom[linestyle=none,fillstyle=solid,fillcolor=curcolor]
{
\newpath
\moveto(493.47613525,68.44518781)
\lineto(511.01825333,68.44518781)
\lineto(511.01825333,51.42284895)
\lineto(493.47613525,51.42284895)
\closepath
}
}
{
\newrgbcolor{curcolor}{0.50196081 0.50196081 0.50196081}
\pscustom[linestyle=none,fillstyle=solid,fillcolor=curcolor]
{
\newpath
\moveto(353.08187866,68.44518781)
\lineto(370.62399673,68.44518781)
\lineto(370.62399673,51.42284895)
\lineto(353.08187866,51.42284895)
\closepath
}
}
{
\newrgbcolor{curcolor}{0 0 0}
\pscustom[linewidth=0.97635722,linecolor=curcolor]
{
\newpath
\moveto(102.62741,544.5349)
\lineto(102.62741,0.48818)
}
}
{
\newrgbcolor{curcolor}{0 0 0}
\pscustom[linewidth=0.96543622,linecolor=curcolor]
{
\newpath
\moveto(0.48271811,527.48601)
\lineto(879.60683,527.48601)
}
}
{
\newrgbcolor{curcolor}{0 0 0}
\pscustom[linewidth=0.96543622,linecolor=curcolor]
{
\newpath
\moveto(0.48271811,0.49928)
\lineto(879.60683,0.49928)
}
}
{
\newrgbcolor{curcolor}{0 0 0}
\pscustom[linewidth=0.90066206,linecolor=curcolor]
{
\newpath
\moveto(335.51919,544.57275)
\lineto(335.51919,0.45038)
}
}
{
\newrgbcolor{curcolor}{0 0 0}
\pscustom[linewidth=0.90066206,linecolor=curcolor]
{
\newpath
\moveto(879.64153,527.51372)
\lineto(879.64153,0.45038)
}
}
{
\newrgbcolor{curcolor}{0 0 0}
\pscustom[linewidth=0.97635722,linecolor=curcolor]
{
\newpath
\moveto(302.2814,544.5349)
\lineto(302.2814,0.48818)
}
}
{
\newrgbcolor{curcolor}{0 0 0}
\pscustom[linewidth=0.97635722,linecolor=curcolor]
{
\newpath
\moveto(269.00574,544.5349)
\lineto(269.00574,0.48818)
}
}
{
\newrgbcolor{curcolor}{0 0 0}
\pscustom[linewidth=0.97635722,linecolor=curcolor]
{
\newpath
\moveto(235.73006,544.5349)
\lineto(235.73006,0.48818)
}
}
{
\newrgbcolor{curcolor}{0 0 0}
\pscustom[linewidth=0.97635722,linecolor=curcolor]
{
\newpath
\moveto(202.45439,544.5349)
\lineto(202.45439,0.48818)
}
}
{
\newrgbcolor{curcolor}{0 0 0}
\pscustom[linewidth=0.97635722,linecolor=curcolor]
{
\newpath
\moveto(169.17873,544.5349)
\lineto(169.17873,0.48818)
}
}
{
\newrgbcolor{curcolor}{0 0 0}
\pscustom[linewidth=0.97635722,linecolor=curcolor]
{
\newpath
\moveto(135.90306,544.5349)
\lineto(135.90306,0.48818)
}
}
{
\newrgbcolor{curcolor}{0 0 0}
\pscustom[linewidth=0.90066206,linecolor=curcolor]
{
\newpath
\moveto(353.07153,527.50448)
\lineto(353.07153,0.45038)
}
}
{
\newrgbcolor{curcolor}{0 0 0}
\pscustom[linewidth=0.90066206,linecolor=curcolor]
{
\newpath
\moveto(370.62386,527.50448)
\lineto(370.62386,0.45038)
}
}
{
\newrgbcolor{curcolor}{0 0 0}
\pscustom[linewidth=0.90066206,linecolor=curcolor]
{
\newpath
\moveto(388.17619,527.50448)
\lineto(388.17619,0.45038)
}
}
{
\newrgbcolor{curcolor}{0 0 0}
\pscustom[linewidth=0.90066206,linecolor=curcolor]
{
\newpath
\moveto(405.72853,527.50448)
\lineto(405.72853,0.45038)
}
}
{
\newrgbcolor{curcolor}{0 0 0}
\pscustom[linewidth=0.90066206,linecolor=curcolor]
{
\newpath
\moveto(423.28086,527.50448)
\lineto(423.28086,0.45038)
}
}
{
\newrgbcolor{curcolor}{0 0 0}
\pscustom[linewidth=0.90066206,linecolor=curcolor]
{
\newpath
\moveto(440.83319,527.50448)
\lineto(440.83319,0.45038)
}
}
{
\newrgbcolor{curcolor}{0 0 0}
\pscustom[linewidth=0.90066206,linecolor=curcolor]
{
\newpath
\moveto(458.38553,527.50448)
\lineto(458.38553,0.45038)
}
}
{
\newrgbcolor{curcolor}{0 0 0}
\pscustom[linewidth=0.90066206,linecolor=curcolor]
{
\newpath
\moveto(475.93786,527.50448)
\lineto(475.93786,0.45038)
}
}
{
\newrgbcolor{curcolor}{0 0 0}
\pscustom[linewidth=0.90066206,linecolor=curcolor]
{
\newpath
\moveto(493.4902,527.50448)
\lineto(493.4902,0.45038)
}
}
{
\newrgbcolor{curcolor}{0 0 0}
\pscustom[linewidth=0.90066206,linecolor=curcolor]
{
\newpath
\moveto(511.04253,527.50448)
\lineto(511.04253,0.45038)
}
}
{
\newrgbcolor{curcolor}{0 0 0}
\pscustom[linewidth=0.90066206,linecolor=curcolor]
{
\newpath
\moveto(528.59486,527.50448)
\lineto(528.59486,0.45038)
}
}
{
\newrgbcolor{curcolor}{0 0 0}
\pscustom[linewidth=0.90066206,linecolor=curcolor]
{
\newpath
\moveto(546.1472,527.50448)
\lineto(546.1472,0.45038)
}
}
{
\newrgbcolor{curcolor}{0 0 0}
\pscustom[linewidth=0.90066206,linecolor=curcolor]
{
\newpath
\moveto(563.69953,527.50448)
\lineto(563.69953,0.45038)
}
}
{
\newrgbcolor{curcolor}{0 0 0}
\pscustom[linewidth=0.90066206,linecolor=curcolor]
{
\newpath
\moveto(581.25186,527.50448)
\lineto(581.25186,0.45038)
}
}
{
\newrgbcolor{curcolor}{0 0 0}
\pscustom[linewidth=0.90066206,linecolor=curcolor]
{
\newpath
\moveto(598.8042,527.50448)
\lineto(598.8042,0.45038)
}
}
{
\newrgbcolor{curcolor}{0 0 0}
\pscustom[linewidth=0.90066206,linecolor=curcolor]
{
\newpath
\moveto(616.35653,527.50448)
\lineto(616.35653,0.45038)
}
}
{
\newrgbcolor{curcolor}{0 0 0}
\pscustom[linewidth=0.90066206,linecolor=curcolor]
{
\newpath
\moveto(633.90886,527.50448)
\lineto(633.90886,0.45038)
}
}
{
\newrgbcolor{curcolor}{0 0 0}
\pscustom[linewidth=0.90066206,linecolor=curcolor]
{
\newpath
\moveto(651.4612,527.50448)
\lineto(651.4612,0.45038)
}
}
{
\newrgbcolor{curcolor}{0 0 0}
\pscustom[linewidth=0.90066206,linecolor=curcolor]
{
\newpath
\moveto(669.01353,527.50448)
\lineto(669.01353,0.45038)
}
}
{
\newrgbcolor{curcolor}{0 0 0}
\pscustom[linewidth=0.90066206,linecolor=curcolor]
{
\newpath
\moveto(686.56586,527.50448)
\lineto(686.56586,0.45038)
}
}
{
\newrgbcolor{curcolor}{0 0 0}
\pscustom[linewidth=0.90066206,linecolor=curcolor]
{
\newpath
\moveto(704.1182,527.50448)
\lineto(704.1182,0.45038)
}
}
{
\newrgbcolor{curcolor}{0 0 0}
\pscustom[linewidth=0.90066206,linecolor=curcolor]
{
\newpath
\moveto(721.67053,527.50448)
\lineto(721.67053,0.45038)
}
}
{
\newrgbcolor{curcolor}{0 0 0}
\pscustom[linewidth=0.90066206,linecolor=curcolor]
{
\newpath
\moveto(739.22286,527.50448)
\lineto(739.22286,0.45038)
}
}
{
\newrgbcolor{curcolor}{0 0 0}
\pscustom[linewidth=0.90066206,linecolor=curcolor]
{
\newpath
\moveto(756.77521,527.50448)
\lineto(756.77521,0.45038)
}
}
{
\newrgbcolor{curcolor}{0 0 0}
\pscustom[linewidth=0.90066206,linecolor=curcolor]
{
\newpath
\moveto(774.32753,527.50448)
\lineto(774.32753,0.45038)
}
}
{
\newrgbcolor{curcolor}{0 0 0}
\pscustom[linewidth=0.90066206,linecolor=curcolor]
{
\newpath
\moveto(791.87985,527.50448)
\lineto(791.87985,0.45038)
}
}
{
\newrgbcolor{curcolor}{0 0 0}
\pscustom[linewidth=0.90066206,linecolor=curcolor]
{
\newpath
\moveto(809.43217,527.50448)
\lineto(809.43217,0.45038)
}
}
{
\newrgbcolor{curcolor}{0 0 0}
\pscustom[linewidth=0.90066206,linecolor=curcolor]
{
\newpath
\moveto(826.98457,527.50448)
\lineto(826.98457,0.45038)
}
}
{
\newrgbcolor{curcolor}{0 0 0}
\pscustom[linewidth=0.90066206,linecolor=curcolor]
{
\newpath
\moveto(844.53689,527.50448)
\lineto(844.53689,0.45038)
}
}
{
\newrgbcolor{curcolor}{0 0 0}
\pscustom[linewidth=0.90066206,linecolor=curcolor]
{
\newpath
\moveto(862.08921,527.50448)
\lineto(862.08921,0.45038)
}
}
{
\newrgbcolor{curcolor}{0 0 0}
\pscustom[linestyle=none,fillstyle=solid,fillcolor=curcolor]
{
\newpath
\moveto(7.10569843,513.8486302)
\lineto(11.41069843,513.8486302)
\curveto(14.95069489,513.8486302)(16.00069843,516.96863261)(16.00069843,519.3836302)
\curveto(16.00069843,522.48862709)(14.27569563,524.6186302)(11.47069843,524.6186302)
\lineto(7.10569843,524.6186302)
\lineto(7.10569843,513.8486302)
\moveto(8.56069843,523.3736302)
\lineto(11.27569843,523.3736302)
\curveto(13.25569645,523.3736302)(14.50069843,522.00862748)(14.50069843,519.2936302)
\curveto(14.50069843,516.57863291)(13.27069654,515.0936302)(11.38069843,515.0936302)
\lineto(8.56069843,515.0936302)
\lineto(8.56069843,523.3736302)
}
}
{
\newrgbcolor{curcolor}{0 0 0}
\pscustom[linestyle=none,fillstyle=solid,fillcolor=curcolor]
{
\newpath
\moveto(23.04554218,516.3086302)
\curveto(23.00054223,515.72363078)(22.26554094,514.7636302)(21.02054218,514.7636302)
\curveto(19.5055437,514.7636302)(18.74054218,515.70863183)(18.74054218,517.3436302)
\lineto(24.47054218,517.3436302)
\curveto(24.47054218,520.11862742)(23.36053992,521.9186302)(21.09554218,521.9186302)
\curveto(18.50054478,521.9186302)(17.33054218,519.98362777)(17.33054218,517.5536302)
\curveto(17.33054218,515.28863246)(18.63554439,513.6236302)(20.84054218,513.6236302)
\curveto(22.10054092,513.6236302)(22.61054254,513.92363044)(22.97054218,514.1636302)
\curveto(23.96054119,514.82362954)(24.32054223,515.93363057)(24.36554218,516.3086302)
\lineto(23.04554218,516.3086302)
\moveto(18.74054218,518.3936302)
\curveto(18.74054218,519.60862898)(19.7005434,520.7336302)(20.91554218,520.7336302)
\curveto(22.52054058,520.7336302)(23.03054226,519.60862898)(23.10554218,518.3936302)
\lineto(18.74054218,518.3936302)
}
}
{
\newrgbcolor{curcolor}{0 0 0}
\pscustom[linestyle=none,fillstyle=solid,fillcolor=curcolor]
{
\newpath
\moveto(31.78515156,519.4586302)
\curveto(31.78515156,519.84862981)(31.59014875,521.9186302)(28.78515156,521.9186302)
\curveto(27.2401531,521.9186302)(25.81515156,521.13862847)(25.81515156,519.4136302)
\curveto(25.81515156,518.33363128)(26.53515265,517.77862993)(27.63015156,517.5086302)
\lineto(29.16015156,517.1336302)
\curveto(30.28515043,516.84863048)(30.72015156,516.63862957)(30.72015156,516.0086302)
\curveto(30.72015156,515.13863107)(29.86515061,514.7636302)(28.92015156,514.7636302)
\curveto(27.06015342,514.7636302)(26.88015151,515.75363081)(26.83515156,516.3686302)
\lineto(25.56015156,516.3686302)
\curveto(25.60515151,515.42363114)(25.83015466,513.6236302)(28.93515156,513.6236302)
\curveto(30.70514979,513.6236302)(32.04015156,514.59863182)(32.04015156,516.2186302)
\curveto(32.04015156,517.28362913)(31.47014992,517.8836306)(29.83515156,518.2886302)
\lineto(28.51515156,518.6186302)
\curveto(27.49515258,518.87362994)(27.09015156,519.02363084)(27.09015156,519.6686302)
\curveto(27.09015156,520.64362922)(28.24515196,520.7786302)(28.65015156,520.7786302)
\curveto(30.31514989,520.7786302)(30.49515157,519.9536297)(30.51015156,519.4586302)
\lineto(31.78515156,519.4586302)
}
}
{
\newrgbcolor{curcolor}{0 0 0}
\pscustom[linestyle=none,fillstyle=solid,fillcolor=curcolor]
{
\newpath
\moveto(34.72515156,519.3086302)
\curveto(34.81515147,519.9086296)(35.02515306,520.8236302)(36.52515156,520.8236302)
\curveto(37.77015031,520.8236302)(38.37015156,520.37362937)(38.37015156,519.5486302)
\curveto(38.37015156,518.76863098)(37.99515124,518.64863017)(37.68015156,518.6186302)
\lineto(35.50515156,518.3486302)
\curveto(33.31515375,518.07863047)(33.12015156,516.54862954)(33.12015156,515.8886302)
\curveto(33.12015156,514.53863155)(34.140153,513.6236302)(35.58015156,513.6236302)
\curveto(37.11015003,513.6236302)(37.90515207,514.34363075)(38.41515156,514.8986302)
\curveto(38.46015151,514.2986308)(38.64015273,513.6986302)(39.81015156,513.6986302)
\curveto(40.11015126,513.6986302)(40.30515178,513.78863026)(40.53015156,513.8486302)
\lineto(40.53015156,514.8086302)
\curveto(40.38015171,514.77863023)(40.21515144,514.7486302)(40.09515156,514.7486302)
\curveto(39.82515183,514.7486302)(39.66015156,514.88363053)(39.66015156,515.2136302)
\lineto(39.66015156,519.7286302)
\curveto(39.66015156,521.73862819)(37.38015093,521.9186302)(36.75015156,521.9186302)
\curveto(34.81515349,521.9186302)(33.5701515,521.18362832)(33.51015156,519.3086302)
\lineto(34.72515156,519.3086302)
\moveto(38.34015156,516.5636302)
\curveto(38.34015156,515.51363125)(37.14015033,514.7186302)(35.91015156,514.7186302)
\curveto(34.92015255,514.7186302)(34.48515156,515.22863105)(34.48515156,516.0836302)
\curveto(34.48515156,517.07362921)(35.5201522,517.26863029)(36.16515156,517.3586302)
\curveto(37.80014992,517.56862999)(38.13015177,517.68863036)(38.34015156,517.8536302)
\lineto(38.34015156,516.5636302)
}
}
{
\newrgbcolor{curcolor}{0 0 0}
\pscustom[linestyle=none,fillstyle=solid,fillcolor=curcolor]
{
\newpath
\moveto(43.40476093,518.4086302)
\curveto(43.40476093,519.54862906)(44.18476216,520.5086302)(45.41476093,520.5086302)
\lineto(45.90976093,520.5086302)
\lineto(45.90976093,521.8736302)
\curveto(45.80476104,521.90363017)(45.72976077,521.9186302)(45.56476093,521.9186302)
\curveto(44.57476192,521.9186302)(43.88476041,521.30362928)(43.35976093,520.3886302)
\lineto(43.32976093,520.3886302)
\lineto(43.32976093,521.6936302)
\lineto(42.08476093,521.6936302)
\lineto(42.08476093,513.8486302)
\lineto(43.40476093,513.8486302)
\lineto(43.40476093,518.4086302)
}
}
{
\newrgbcolor{curcolor}{0 0 0}
\pscustom[linestyle=none,fillstyle=solid,fillcolor=curcolor]
{
\newpath
\moveto(48.41452656,518.4086302)
\curveto(48.41452656,519.54862906)(49.19452779,520.5086302)(50.42452656,520.5086302)
\lineto(50.91952656,520.5086302)
\lineto(50.91952656,521.8736302)
\curveto(50.81452666,521.90363017)(50.73952639,521.9186302)(50.57452656,521.9186302)
\curveto(49.58452755,521.9186302)(48.89452603,521.30362928)(48.36952656,520.3886302)
\lineto(48.33952656,520.3886302)
\lineto(48.33952656,521.6936302)
\lineto(47.09452656,521.6936302)
\lineto(47.09452656,513.8486302)
\lineto(48.41452656,513.8486302)
\lineto(48.41452656,518.4086302)
}
}
{
\newrgbcolor{curcolor}{0 0 0}
\pscustom[linestyle=none,fillstyle=solid,fillcolor=curcolor]
{
\newpath
\moveto(51.18132343,517.7786302)
\curveto(51.18132343,515.75363222)(52.32132594,513.6386302)(54.82632343,513.6386302)
\curveto(57.33132093,513.6386302)(58.47132343,515.75363222)(58.47132343,517.7786302)
\curveto(58.47132343,519.80362817)(57.33132093,521.9186302)(54.82632343,521.9186302)
\curveto(52.32132594,521.9186302)(51.18132343,519.80362817)(51.18132343,517.7786302)
\moveto(52.54632343,517.7786302)
\curveto(52.54632343,518.82862915)(52.93632532,520.7786302)(54.82632343,520.7786302)
\curveto(56.71632154,520.7786302)(57.10632343,518.82862915)(57.10632343,517.7786302)
\curveto(57.10632343,516.72863125)(56.71632154,514.7786302)(54.82632343,514.7786302)
\curveto(52.93632532,514.7786302)(52.54632343,516.72863125)(52.54632343,517.7786302)
}
}
{
\newrgbcolor{curcolor}{0 0 0}
\pscustom[linestyle=none,fillstyle=solid,fillcolor=curcolor]
{
\newpath
\moveto(61.33093281,524.6186302)
\lineto(60.01093281,524.6186302)
\lineto(60.01093281,513.8486302)
\lineto(61.33093281,513.8486302)
\lineto(61.33093281,524.6186302)
}
}
{
\newrgbcolor{curcolor}{0 0 0}
\pscustom[linestyle=none,fillstyle=solid,fillcolor=curcolor]
{
\newpath
\moveto(64.67077656,524.6186302)
\lineto(63.35077656,524.6186302)
\lineto(63.35077656,513.8486302)
\lineto(64.67077656,513.8486302)
\lineto(64.67077656,524.6186302)
}
}
{
\newrgbcolor{curcolor}{0 0 0}
\pscustom[linestyle=none,fillstyle=solid,fillcolor=curcolor]
{
\newpath
\moveto(67.83062031,519.3086302)
\curveto(67.92062022,519.9086296)(68.13062181,520.8236302)(69.63062031,520.8236302)
\curveto(70.87561906,520.8236302)(71.47562031,520.37362937)(71.47562031,519.5486302)
\curveto(71.47562031,518.76863098)(71.10061999,518.64863017)(70.78562031,518.6186302)
\lineto(68.61062031,518.3486302)
\curveto(66.4206225,518.07863047)(66.22562031,516.54862954)(66.22562031,515.8886302)
\curveto(66.22562031,514.53863155)(67.24562175,513.6236302)(68.68562031,513.6236302)
\curveto(70.21561878,513.6236302)(71.01062082,514.34363075)(71.52062031,514.8986302)
\curveto(71.56562026,514.2986308)(71.74562148,513.6986302)(72.91562031,513.6986302)
\curveto(73.21562001,513.6986302)(73.41062053,513.78863026)(73.63562031,513.8486302)
\lineto(73.63562031,514.8086302)
\curveto(73.48562046,514.77863023)(73.32062019,514.7486302)(73.20062031,514.7486302)
\curveto(72.93062058,514.7486302)(72.76562031,514.88363053)(72.76562031,515.2136302)
\lineto(72.76562031,519.7286302)
\curveto(72.76562031,521.73862819)(70.48561968,521.9186302)(69.85562031,521.9186302)
\curveto(67.92062224,521.9186302)(66.67562025,521.18362832)(66.61562031,519.3086302)
\lineto(67.83062031,519.3086302)
\moveto(71.44562031,516.5636302)
\curveto(71.44562031,515.51363125)(70.24561908,514.7186302)(69.01562031,514.7186302)
\curveto(68.0256213,514.7186302)(67.59062031,515.22863105)(67.59062031,516.0836302)
\curveto(67.59062031,517.07362921)(68.62562095,517.26863029)(69.27062031,517.3586302)
\curveto(70.90561867,517.56862999)(71.23562052,517.68863036)(71.44562031,517.8536302)
\lineto(71.44562031,516.5636302)
}
}
{
\newrgbcolor{curcolor}{0 0 0}
\pscustom[linestyle=none,fillstyle=solid,fillcolor=curcolor]
{
\newpath
\moveto(81.52022968,524.6186302)
\lineto(80.20022968,524.6186302)
\lineto(80.20022968,520.6886302)
\lineto(80.17022968,520.5836302)
\curveto(79.85523,521.03362975)(79.25522826,521.9186302)(77.83022968,521.9186302)
\curveto(75.74523177,521.9186302)(74.56022968,520.20862799)(74.56022968,518.0036302)
\curveto(74.56022968,516.12863207)(75.34023235,513.6236302)(78.01022968,513.6236302)
\curveto(78.77522892,513.6236302)(79.67523025,513.86363126)(80.24522968,514.9286302)
\lineto(80.27522968,514.9286302)
\lineto(80.27522968,513.8486302)
\lineto(81.52022968,513.8486302)
\lineto(81.52022968,524.6186302)
\moveto(75.92522968,517.7936302)
\curveto(75.92522968,518.79862919)(76.03023172,520.7336302)(78.07022968,520.7336302)
\curveto(79.97522778,520.7336302)(80.18522968,518.67862892)(80.18522968,517.4036302)
\curveto(80.18522968,515.31863228)(78.88022884,514.7636302)(78.04022968,514.7636302)
\curveto(76.60023112,514.7636302)(75.92522968,516.06863192)(75.92522968,517.7936302)
}
}
{
\newrgbcolor{curcolor}{0 0 0}
\pscustom[linestyle=none,fillstyle=solid,fillcolor=curcolor]
{
\newpath
\moveto(82.90983906,517.7786302)
\curveto(82.90983906,515.75363222)(84.04984156,513.6386302)(86.55483906,513.6386302)
\curveto(89.05983655,513.6386302)(90.19983906,515.75363222)(90.19983906,517.7786302)
\curveto(90.19983906,519.80362817)(89.05983655,521.9186302)(86.55483906,521.9186302)
\curveto(84.04984156,521.9186302)(82.90983906,519.80362817)(82.90983906,517.7786302)
\moveto(84.27483906,517.7786302)
\curveto(84.27483906,518.82862915)(84.66484095,520.7786302)(86.55483906,520.7786302)
\curveto(88.44483717,520.7786302)(88.83483906,518.82862915)(88.83483906,517.7786302)
\curveto(88.83483906,516.72863125)(88.44483717,514.7786302)(86.55483906,514.7786302)
\curveto(84.66484095,514.7786302)(84.27483906,516.72863125)(84.27483906,517.7786302)
}
}
{
\newrgbcolor{curcolor}{0 0 0}
\pscustom[linestyle=none,fillstyle=solid,fillcolor=curcolor]
{
\newpath
\moveto(93.20944843,518.4086302)
\curveto(93.20944843,519.54862906)(93.98944966,520.5086302)(95.21944843,520.5086302)
\lineto(95.71444843,520.5086302)
\lineto(95.71444843,521.8736302)
\curveto(95.60944854,521.90363017)(95.53444827,521.9186302)(95.36944843,521.9186302)
\curveto(94.37944942,521.9186302)(93.68944791,521.30362928)(93.16444843,520.3886302)
\lineto(93.13444843,520.3886302)
\lineto(93.13444843,521.6936302)
\lineto(91.88944843,521.6936302)
\lineto(91.88944843,513.8486302)
\lineto(93.20944843,513.8486302)
\lineto(93.20944843,518.4086302)
}
}
{
\newrgbcolor{curcolor}{0 0 0}
\pscustom[linestyle=none,fillstyle=solid,fillcolor=curcolor]
{
\newpath
\moveto(7.10569843,496.85839582)
\lineto(11.41069843,496.85839582)
\curveto(14.95069489,496.85839582)(16.00069843,499.97839824)(16.00069843,502.39339582)
\curveto(16.00069843,505.49839272)(14.27569563,507.62839582)(11.47069843,507.62839582)
\lineto(7.10569843,507.62839582)
\lineto(7.10569843,496.85839582)
\moveto(8.56069843,506.38339582)
\lineto(11.27569843,506.38339582)
\curveto(13.25569645,506.38339582)(14.50069843,505.01839311)(14.50069843,502.30339582)
\curveto(14.50069843,499.58839854)(13.27069654,498.10339582)(11.38069843,498.10339582)
\lineto(8.56069843,498.10339582)
\lineto(8.56069843,506.38339582)
}
}
{
\newrgbcolor{curcolor}{0 0 0}
\pscustom[linestyle=none,fillstyle=solid,fillcolor=curcolor]
{
\newpath
\moveto(23.04554218,499.31839582)
\curveto(23.00054223,498.73339641)(22.26554094,497.77339582)(21.02054218,497.77339582)
\curveto(19.5055437,497.77339582)(18.74054218,498.71839746)(18.74054218,500.35339582)
\lineto(24.47054218,500.35339582)
\curveto(24.47054218,503.12839305)(23.36053992,504.92839582)(21.09554218,504.92839582)
\curveto(18.50054478,504.92839582)(17.33054218,502.99339339)(17.33054218,500.56339582)
\curveto(17.33054218,498.29839809)(18.63554439,496.63339582)(20.84054218,496.63339582)
\curveto(22.10054092,496.63339582)(22.61054254,496.93339606)(22.97054218,497.17339582)
\curveto(23.96054119,497.83339516)(24.32054223,498.9433962)(24.36554218,499.31839582)
\lineto(23.04554218,499.31839582)
\moveto(18.74054218,501.40339582)
\curveto(18.74054218,502.61839461)(19.7005434,503.74339582)(20.91554218,503.74339582)
\curveto(22.52054058,503.74339582)(23.03054226,502.61839461)(23.10554218,501.40339582)
\lineto(18.74054218,501.40339582)
}
}
{
\newrgbcolor{curcolor}{0 0 0}
\pscustom[linestyle=none,fillstyle=solid,fillcolor=curcolor]
{
\newpath
\moveto(31.78515156,502.46839582)
\curveto(31.78515156,502.85839543)(31.59014875,504.92839582)(28.78515156,504.92839582)
\curveto(27.2401531,504.92839582)(25.81515156,504.1483941)(25.81515156,502.42339582)
\curveto(25.81515156,501.3433969)(26.53515265,500.78839555)(27.63015156,500.51839582)
\lineto(29.16015156,500.14339582)
\curveto(30.28515043,499.85839611)(30.72015156,499.64839519)(30.72015156,499.01839582)
\curveto(30.72015156,498.14839669)(29.86515061,497.77339582)(28.92015156,497.77339582)
\curveto(27.06015342,497.77339582)(26.88015151,498.76339644)(26.83515156,499.37839582)
\lineto(25.56015156,499.37839582)
\curveto(25.60515151,498.43339677)(25.83015466,496.63339582)(28.93515156,496.63339582)
\curveto(30.70514979,496.63339582)(32.04015156,497.60839744)(32.04015156,499.22839582)
\curveto(32.04015156,500.29339476)(31.47014992,500.89339623)(29.83515156,501.29839582)
\lineto(28.51515156,501.62839582)
\curveto(27.49515258,501.88339557)(27.09015156,502.03339647)(27.09015156,502.67839582)
\curveto(27.09015156,503.65339485)(28.24515196,503.78839582)(28.65015156,503.78839582)
\curveto(30.31514989,503.78839582)(30.49515157,502.96339533)(30.51015156,502.46839582)
\lineto(31.78515156,502.46839582)
}
}
{
\newrgbcolor{curcolor}{0 0 0}
\pscustom[linestyle=none,fillstyle=solid,fillcolor=curcolor]
{
\newpath
\moveto(34.72515156,502.31839582)
\curveto(34.81515147,502.91839522)(35.02515306,503.83339582)(36.52515156,503.83339582)
\curveto(37.77015031,503.83339582)(38.37015156,503.383395)(38.37015156,502.55839582)
\curveto(38.37015156,501.7783966)(37.99515124,501.65839579)(37.68015156,501.62839582)
\lineto(35.50515156,501.35839582)
\curveto(33.31515375,501.08839609)(33.12015156,499.55839516)(33.12015156,498.89839582)
\curveto(33.12015156,497.54839717)(34.140153,496.63339582)(35.58015156,496.63339582)
\curveto(37.11015003,496.63339582)(37.90515207,497.35339638)(38.41515156,497.90839582)
\curveto(38.46015151,497.30839642)(38.64015273,496.70839582)(39.81015156,496.70839582)
\curveto(40.11015126,496.70839582)(40.30515178,496.79839588)(40.53015156,496.85839582)
\lineto(40.53015156,497.81839582)
\curveto(40.38015171,497.78839585)(40.21515144,497.75839582)(40.09515156,497.75839582)
\curveto(39.82515183,497.75839582)(39.66015156,497.89339615)(39.66015156,498.22339582)
\lineto(39.66015156,502.73839582)
\curveto(39.66015156,504.74839381)(37.38015093,504.92839582)(36.75015156,504.92839582)
\curveto(34.81515349,504.92839582)(33.5701515,504.19339395)(33.51015156,502.31839582)
\lineto(34.72515156,502.31839582)
\moveto(38.34015156,499.57339582)
\curveto(38.34015156,498.52339687)(37.14015033,497.72839582)(35.91015156,497.72839582)
\curveto(34.92015255,497.72839582)(34.48515156,498.23839668)(34.48515156,499.09339582)
\curveto(34.48515156,500.08339483)(35.5201522,500.27839591)(36.16515156,500.36839582)
\curveto(37.80014992,500.57839561)(38.13015177,500.69839599)(38.34015156,500.86339582)
\lineto(38.34015156,499.57339582)
}
}
{
\newrgbcolor{curcolor}{0 0 0}
\pscustom[linestyle=none,fillstyle=solid,fillcolor=curcolor]
{
\newpath
\moveto(43.40476093,501.41839582)
\curveto(43.40476093,502.55839468)(44.18476216,503.51839582)(45.41476093,503.51839582)
\lineto(45.90976093,503.51839582)
\lineto(45.90976093,504.88339582)
\curveto(45.80476104,504.91339579)(45.72976077,504.92839582)(45.56476093,504.92839582)
\curveto(44.57476192,504.92839582)(43.88476041,504.31339491)(43.35976093,503.39839582)
\lineto(43.32976093,503.39839582)
\lineto(43.32976093,504.70339582)
\lineto(42.08476093,504.70339582)
\lineto(42.08476093,496.85839582)
\lineto(43.40476093,496.85839582)
\lineto(43.40476093,501.41839582)
}
}
{
\newrgbcolor{curcolor}{0 0 0}
\pscustom[linestyle=none,fillstyle=solid,fillcolor=curcolor]
{
\newpath
\moveto(48.41452656,501.41839582)
\curveto(48.41452656,502.55839468)(49.19452779,503.51839582)(50.42452656,503.51839582)
\lineto(50.91952656,503.51839582)
\lineto(50.91952656,504.88339582)
\curveto(50.81452666,504.91339579)(50.73952639,504.92839582)(50.57452656,504.92839582)
\curveto(49.58452755,504.92839582)(48.89452603,504.31339491)(48.36952656,503.39839582)
\lineto(48.33952656,503.39839582)
\lineto(48.33952656,504.70339582)
\lineto(47.09452656,504.70339582)
\lineto(47.09452656,496.85839582)
\lineto(48.41452656,496.85839582)
\lineto(48.41452656,501.41839582)
}
}
{
\newrgbcolor{curcolor}{0 0 0}
\pscustom[linestyle=none,fillstyle=solid,fillcolor=curcolor]
{
\newpath
\moveto(51.18132343,500.78839582)
\curveto(51.18132343,498.76339785)(52.32132594,496.64839582)(54.82632343,496.64839582)
\curveto(57.33132093,496.64839582)(58.47132343,498.76339785)(58.47132343,500.78839582)
\curveto(58.47132343,502.8133938)(57.33132093,504.92839582)(54.82632343,504.92839582)
\curveto(52.32132594,504.92839582)(51.18132343,502.8133938)(51.18132343,500.78839582)
\moveto(52.54632343,500.78839582)
\curveto(52.54632343,501.83839477)(52.93632532,503.78839582)(54.82632343,503.78839582)
\curveto(56.71632154,503.78839582)(57.10632343,501.83839477)(57.10632343,500.78839582)
\curveto(57.10632343,499.73839687)(56.71632154,497.78839582)(54.82632343,497.78839582)
\curveto(52.93632532,497.78839582)(52.54632343,499.73839687)(52.54632343,500.78839582)
}
}
{
\newrgbcolor{curcolor}{0 0 0}
\pscustom[linestyle=none,fillstyle=solid,fillcolor=curcolor]
{
\newpath
\moveto(61.33093281,507.62839582)
\lineto(60.01093281,507.62839582)
\lineto(60.01093281,496.85839582)
\lineto(61.33093281,496.85839582)
\lineto(61.33093281,507.62839582)
}
}
{
\newrgbcolor{curcolor}{0 0 0}
\pscustom[linestyle=none,fillstyle=solid,fillcolor=curcolor]
{
\newpath
\moveto(64.67077656,507.62839582)
\lineto(63.35077656,507.62839582)
\lineto(63.35077656,496.85839582)
\lineto(64.67077656,496.85839582)
\lineto(64.67077656,507.62839582)
}
}
{
\newrgbcolor{curcolor}{0 0 0}
\pscustom[linestyle=none,fillstyle=solid,fillcolor=curcolor]
{
\newpath
\moveto(67.83062031,502.31839582)
\curveto(67.92062022,502.91839522)(68.13062181,503.83339582)(69.63062031,503.83339582)
\curveto(70.87561906,503.83339582)(71.47562031,503.383395)(71.47562031,502.55839582)
\curveto(71.47562031,501.7783966)(71.10061999,501.65839579)(70.78562031,501.62839582)
\lineto(68.61062031,501.35839582)
\curveto(66.4206225,501.08839609)(66.22562031,499.55839516)(66.22562031,498.89839582)
\curveto(66.22562031,497.54839717)(67.24562175,496.63339582)(68.68562031,496.63339582)
\curveto(70.21561878,496.63339582)(71.01062082,497.35339638)(71.52062031,497.90839582)
\curveto(71.56562026,497.30839642)(71.74562148,496.70839582)(72.91562031,496.70839582)
\curveto(73.21562001,496.70839582)(73.41062053,496.79839588)(73.63562031,496.85839582)
\lineto(73.63562031,497.81839582)
\curveto(73.48562046,497.78839585)(73.32062019,497.75839582)(73.20062031,497.75839582)
\curveto(72.93062058,497.75839582)(72.76562031,497.89339615)(72.76562031,498.22339582)
\lineto(72.76562031,502.73839582)
\curveto(72.76562031,504.74839381)(70.48561968,504.92839582)(69.85562031,504.92839582)
\curveto(67.92062224,504.92839582)(66.67562025,504.19339395)(66.61562031,502.31839582)
\lineto(67.83062031,502.31839582)
\moveto(71.44562031,499.57339582)
\curveto(71.44562031,498.52339687)(70.24561908,497.72839582)(69.01562031,497.72839582)
\curveto(68.0256213,497.72839582)(67.59062031,498.23839668)(67.59062031,499.09339582)
\curveto(67.59062031,500.08339483)(68.62562095,500.27839591)(69.27062031,500.36839582)
\curveto(70.90561867,500.57839561)(71.23562052,500.69839599)(71.44562031,500.86339582)
\lineto(71.44562031,499.57339582)
}
}
{
\newrgbcolor{curcolor}{0 0 0}
\pscustom[linestyle=none,fillstyle=solid,fillcolor=curcolor]
{
\newpath
\moveto(81.52022968,507.62839582)
\lineto(80.20022968,507.62839582)
\lineto(80.20022968,503.69839582)
\lineto(80.17022968,503.59339582)
\curveto(79.85523,504.04339537)(79.25522826,504.92839582)(77.83022968,504.92839582)
\curveto(75.74523177,504.92839582)(74.56022968,503.21839362)(74.56022968,501.01339582)
\curveto(74.56022968,499.1383977)(75.34023235,496.63339582)(78.01022968,496.63339582)
\curveto(78.77522892,496.63339582)(79.67523025,496.87339689)(80.24522968,497.93839582)
\lineto(80.27522968,497.93839582)
\lineto(80.27522968,496.85839582)
\lineto(81.52022968,496.85839582)
\lineto(81.52022968,507.62839582)
\moveto(75.92522968,500.80339582)
\curveto(75.92522968,501.80839482)(76.03023172,503.74339582)(78.07022968,503.74339582)
\curveto(79.97522778,503.74339582)(80.18522968,501.68839455)(80.18522968,500.41339582)
\curveto(80.18522968,498.32839791)(78.88022884,497.77339582)(78.04022968,497.77339582)
\curveto(76.60023112,497.77339582)(75.92522968,499.07839755)(75.92522968,500.80339582)
}
}
{
\newrgbcolor{curcolor}{0 0 0}
\pscustom[linestyle=none,fillstyle=solid,fillcolor=curcolor]
{
\newpath
\moveto(82.90983906,500.78839582)
\curveto(82.90983906,498.76339785)(84.04984156,496.64839582)(86.55483906,496.64839582)
\curveto(89.05983655,496.64839582)(90.19983906,498.76339785)(90.19983906,500.78839582)
\curveto(90.19983906,502.8133938)(89.05983655,504.92839582)(86.55483906,504.92839582)
\curveto(84.04984156,504.92839582)(82.90983906,502.8133938)(82.90983906,500.78839582)
\moveto(84.27483906,500.78839582)
\curveto(84.27483906,501.83839477)(84.66484095,503.78839582)(86.55483906,503.78839582)
\curveto(88.44483717,503.78839582)(88.83483906,501.83839477)(88.83483906,500.78839582)
\curveto(88.83483906,499.73839687)(88.44483717,497.78839582)(86.55483906,497.78839582)
\curveto(84.66484095,497.78839582)(84.27483906,499.73839687)(84.27483906,500.78839582)
}
}
{
\newrgbcolor{curcolor}{0 0 0}
\pscustom[linestyle=none,fillstyle=solid,fillcolor=curcolor]
{
\newpath
\moveto(93.20944843,501.41839582)
\curveto(93.20944843,502.55839468)(93.98944966,503.51839582)(95.21944843,503.51839582)
\lineto(95.71444843,503.51839582)
\lineto(95.71444843,504.88339582)
\curveto(95.60944854,504.91339579)(95.53444827,504.92839582)(95.36944843,504.92839582)
\curveto(94.37944942,504.92839582)(93.68944791,504.31339491)(93.16444843,503.39839582)
\lineto(93.13444843,503.39839582)
\lineto(93.13444843,504.70339582)
\lineto(91.88944843,504.70339582)
\lineto(91.88944843,496.85839582)
\lineto(93.20944843,496.85839582)
\lineto(93.20944843,501.41839582)
}
}
{
\newrgbcolor{curcolor}{0 0 0}
\pscustom[linestyle=none,fillstyle=solid,fillcolor=curcolor]
{
\newpath
\moveto(14.17069843,398.05205061)
\lineto(15.23569843,394.91705061)
\lineto(16.82569843,394.91705061)
\lineto(12.92569843,405.68705061)
\lineto(11.27569843,405.68705061)
\lineto(7.22569843,394.91705061)
\lineto(8.72569843,394.91705061)
\lineto(9.85069843,398.05205061)
\lineto(14.17069843,398.05205061)
\moveto(10.30069843,399.34205061)
\lineto(12.02569843,404.08205061)
\lineto(12.05569843,404.08205061)
\lineto(13.64569843,399.34205061)
\lineto(10.30069843,399.34205061)
}
}
{
\newrgbcolor{curcolor}{0 0 0}
\pscustom[linestyle=none,fillstyle=solid,fillcolor=curcolor]
{
\newpath
\moveto(24.22726093,405.68705061)
\lineto(22.90726093,405.68705061)
\lineto(22.90726093,401.75705061)
\lineto(22.87726093,401.65205061)
\curveto(22.56226125,402.10205016)(21.96225951,402.98705061)(20.53726093,402.98705061)
\curveto(18.45226302,402.98705061)(17.26726093,401.2770484)(17.26726093,399.07205061)
\curveto(17.26726093,397.19705248)(18.0472636,394.69205061)(20.71726093,394.69205061)
\curveto(21.48226017,394.69205061)(22.3822615,394.93205167)(22.95226093,395.99705061)
\lineto(22.98226093,395.99705061)
\lineto(22.98226093,394.91705061)
\lineto(24.22726093,394.91705061)
\lineto(24.22726093,405.68705061)
\moveto(18.63226093,398.86205061)
\curveto(18.63226093,399.8670496)(18.73726297,401.80205061)(20.77726093,401.80205061)
\curveto(22.68225903,401.80205061)(22.89226093,399.74704933)(22.89226093,398.47205061)
\curveto(22.89226093,396.38705269)(21.58726009,395.83205061)(20.74726093,395.83205061)
\curveto(19.30726237,395.83205061)(18.63226093,397.13705233)(18.63226093,398.86205061)
}
}
{
\newrgbcolor{curcolor}{0 0 0}
\pscustom[linestyle=none,fillstyle=solid,fillcolor=curcolor]
{
\newpath
\moveto(26.06687031,394.91705061)
\lineto(27.38687031,394.91705061)
\lineto(27.38687031,399.19205061)
\curveto(27.38687031,401.32204848)(28.70687106,401.80205061)(29.45687031,401.80205061)
\curveto(30.43186933,401.80205061)(30.68687031,401.00704995)(30.68687031,400.34705061)
\lineto(30.68687031,394.91705061)
\lineto(32.00687031,394.91705061)
\lineto(32.00687031,399.70205061)
\curveto(32.00687031,400.75204956)(32.74187142,401.80205061)(33.85187031,401.80205061)
\curveto(34.97686918,401.80205061)(35.30687031,401.06704953)(35.30687031,399.98705061)
\lineto(35.30687031,394.91705061)
\lineto(36.62687031,394.91705061)
\lineto(36.62687031,400.34705061)
\curveto(36.62687031,402.5520484)(35.03686947,402.98705061)(34.19687031,402.98705061)
\curveto(32.98187152,402.98705061)(32.45686965,402.44704987)(31.79687031,401.71205061)
\curveto(31.57187053,402.13205019)(31.1218689,402.98705061)(29.71187031,402.98705061)
\curveto(28.30187172,402.98705061)(27.62687002,402.07205019)(27.34187031,401.65205061)
\lineto(27.31187031,401.65205061)
\lineto(27.31187031,402.76205061)
\lineto(26.06687031,402.76205061)
\lineto(26.06687031,394.91705061)
}
}
{
\newrgbcolor{curcolor}{0 0 0}
\pscustom[linestyle=none,fillstyle=solid,fillcolor=curcolor]
{
\newpath
\moveto(39.92663593,402.76205061)
\lineto(38.60663593,402.76205061)
\lineto(38.60663593,394.91705061)
\lineto(39.92663593,394.91705061)
\lineto(39.92663593,402.76205061)
\moveto(39.92663593,404.18705061)
\lineto(39.92663593,405.68705061)
\lineto(38.60663593,405.68705061)
\lineto(38.60663593,404.18705061)
\lineto(39.92663593,404.18705061)
}
}
{
\newrgbcolor{curcolor}{0 0 0}
\pscustom[linestyle=none,fillstyle=solid,fillcolor=curcolor]
{
\newpath
\moveto(48.30647968,400.25705061)
\curveto(48.30647968,402.49204837)(46.77647847,402.98705061)(45.56147968,402.98705061)
\curveto(44.21148103,402.98705061)(43.4764794,402.07205019)(43.19147968,401.65205061)
\lineto(43.16147968,401.65205061)
\lineto(43.16147968,402.76205061)
\lineto(41.91647968,402.76205061)
\lineto(41.91647968,394.91705061)
\lineto(43.23647968,394.91705061)
\lineto(43.23647968,399.19205061)
\curveto(43.23647968,401.32204848)(44.55648043,401.80205061)(45.30647968,401.80205061)
\curveto(46.59647839,401.80205061)(46.98647968,401.11204924)(46.98647968,399.74705061)
\lineto(46.98647968,394.91705061)
\lineto(48.30647968,394.91705061)
\lineto(48.30647968,400.25705061)
}
}
{
\newrgbcolor{curcolor}{0 0 0}
\pscustom[linestyle=none,fillstyle=solid,fillcolor=curcolor]
{
\newpath
\moveto(51.61608906,402.76205061)
\lineto(50.29608906,402.76205061)
\lineto(50.29608906,394.91705061)
\lineto(51.61608906,394.91705061)
\lineto(51.61608906,402.76205061)
\moveto(51.61608906,404.18705061)
\lineto(51.61608906,405.68705061)
\lineto(50.29608906,405.68705061)
\lineto(50.29608906,404.18705061)
\lineto(51.61608906,404.18705061)
}
}
{
\newrgbcolor{curcolor}{0 0 0}
\pscustom[linestyle=none,fillstyle=solid,fillcolor=curcolor]
{
\newpath
\moveto(59.33593281,400.52705061)
\curveto(59.33593281,400.91705022)(59.14093,402.98705061)(56.33593281,402.98705061)
\curveto(54.79093435,402.98705061)(53.36593281,402.20704888)(53.36593281,400.48205061)
\curveto(53.36593281,399.40205169)(54.0859339,398.84705034)(55.18093281,398.57705061)
\lineto(56.71093281,398.20205061)
\curveto(57.83593168,397.91705089)(58.27093281,397.70704998)(58.27093281,397.07705061)
\curveto(58.27093281,396.20705148)(57.41593186,395.83205061)(56.47093281,395.83205061)
\curveto(54.61093467,395.83205061)(54.43093276,396.82205122)(54.38593281,397.43705061)
\lineto(53.11093281,397.43705061)
\curveto(53.15593276,396.49205155)(53.38093591,394.69205061)(56.48593281,394.69205061)
\curveto(58.25593104,394.69205061)(59.59093281,395.66705223)(59.59093281,397.28705061)
\curveto(59.59093281,398.35204954)(59.02093117,398.95205101)(57.38593281,399.35705061)
\lineto(56.06593281,399.68705061)
\curveto(55.04593383,399.94205035)(54.64093281,400.09205125)(54.64093281,400.73705061)
\curveto(54.64093281,401.71204963)(55.79593321,401.84705061)(56.20093281,401.84705061)
\curveto(57.86593114,401.84705061)(58.04593282,401.02205011)(58.06093281,400.52705061)
\lineto(59.33593281,400.52705061)
}
}
{
\newrgbcolor{curcolor}{0 0 0}
\pscustom[linestyle=none,fillstyle=solid,fillcolor=curcolor]
{
\newpath
\moveto(63.98593281,401.66705061)
\lineto(63.98593281,402.76205061)
\lineto(62.72593281,402.76205061)
\lineto(62.72593281,404.95205061)
\lineto(61.40593281,404.95205061)
\lineto(61.40593281,402.76205061)
\lineto(60.34093281,402.76205061)
\lineto(60.34093281,401.66705061)
\lineto(61.40593281,401.66705061)
\lineto(61.40593281,396.49205061)
\curveto(61.40593281,395.54705155)(61.69093411,394.81205061)(62.99593281,394.81205061)
\curveto(63.13093267,394.81205061)(63.50593329,394.87205065)(63.98593281,394.91705061)
\lineto(63.98593281,395.95205061)
\lineto(63.52093281,395.95205061)
\curveto(63.25093308,395.95205061)(62.72593281,395.95205122)(62.72593281,396.56705061)
\lineto(62.72593281,401.66705061)
\lineto(63.98593281,401.66705061)
}
}
{
\newrgbcolor{curcolor}{0 0 0}
\pscustom[linestyle=none,fillstyle=solid,fillcolor=curcolor]
{
\newpath
\moveto(66.76608906,399.47705061)
\curveto(66.76608906,400.61704947)(67.54609029,401.57705061)(68.77608906,401.57705061)
\lineto(69.27108906,401.57705061)
\lineto(69.27108906,402.94205061)
\curveto(69.16608916,402.97205058)(69.09108889,402.98705061)(68.92608906,402.98705061)
\curveto(67.93609005,402.98705061)(67.24608853,402.37204969)(66.72108906,401.45705061)
\lineto(66.69108906,401.45705061)
\lineto(66.69108906,402.76205061)
\lineto(65.44608906,402.76205061)
\lineto(65.44608906,394.91705061)
\lineto(66.76608906,394.91705061)
\lineto(66.76608906,399.47705061)
}
}
{
\newrgbcolor{curcolor}{0 0 0}
\pscustom[linestyle=none,fillstyle=solid,fillcolor=curcolor]
{
\newpath
\moveto(71.29937031,400.37705061)
\curveto(71.38937022,400.97705001)(71.59937181,401.89205061)(73.09937031,401.89205061)
\curveto(74.34436906,401.89205061)(74.94437031,401.44204978)(74.94437031,400.61705061)
\curveto(74.94437031,399.83705139)(74.56936999,399.71705058)(74.25437031,399.68705061)
\lineto(72.07937031,399.41705061)
\curveto(69.8893725,399.14705088)(69.69437031,397.61704995)(69.69437031,396.95705061)
\curveto(69.69437031,395.60705196)(70.71437175,394.69205061)(72.15437031,394.69205061)
\curveto(73.68436878,394.69205061)(74.47937082,395.41205116)(74.98937031,395.96705061)
\curveto(75.03437026,395.36705121)(75.21437148,394.76705061)(76.38437031,394.76705061)
\curveto(76.68437001,394.76705061)(76.87937053,394.85705067)(77.10437031,394.91705061)
\lineto(77.10437031,395.87705061)
\curveto(76.95437046,395.84705064)(76.78937019,395.81705061)(76.66937031,395.81705061)
\curveto(76.39937058,395.81705061)(76.23437031,395.95205094)(76.23437031,396.28205061)
\lineto(76.23437031,400.79705061)
\curveto(76.23437031,402.8070486)(73.95436968,402.98705061)(73.32437031,402.98705061)
\curveto(71.38937224,402.98705061)(70.14437025,402.25204873)(70.08437031,400.37705061)
\lineto(71.29937031,400.37705061)
\moveto(74.91437031,397.63205061)
\curveto(74.91437031,396.58205166)(73.71436908,395.78705061)(72.48437031,395.78705061)
\curveto(71.4943713,395.78705061)(71.05937031,396.29705146)(71.05937031,397.15205061)
\curveto(71.05937031,398.14204962)(72.09437095,398.3370507)(72.73937031,398.42705061)
\curveto(74.37436867,398.6370504)(74.70437052,398.75705077)(74.91437031,398.92205061)
\lineto(74.91437031,397.63205061)
}
}
{
\newrgbcolor{curcolor}{0 0 0}
\pscustom[linestyle=none,fillstyle=solid,fillcolor=curcolor]
{
\newpath
\moveto(84.98897968,405.68705061)
\lineto(83.66897968,405.68705061)
\lineto(83.66897968,401.75705061)
\lineto(83.63897968,401.65205061)
\curveto(83.32398,402.10205016)(82.72397826,402.98705061)(81.29897968,402.98705061)
\curveto(79.21398177,402.98705061)(78.02897968,401.2770484)(78.02897968,399.07205061)
\curveto(78.02897968,397.19705248)(78.80898235,394.69205061)(81.47897968,394.69205061)
\curveto(82.24397892,394.69205061)(83.14398025,394.93205167)(83.71397968,395.99705061)
\lineto(83.74397968,395.99705061)
\lineto(83.74397968,394.91705061)
\lineto(84.98897968,394.91705061)
\lineto(84.98897968,405.68705061)
\moveto(79.39397968,398.86205061)
\curveto(79.39397968,399.8670496)(79.49898172,401.80205061)(81.53897968,401.80205061)
\curveto(83.44397778,401.80205061)(83.65397968,399.74704933)(83.65397968,398.47205061)
\curveto(83.65397968,396.38705269)(82.34897884,395.83205061)(81.50897968,395.83205061)
\curveto(80.06898112,395.83205061)(79.39397968,397.13705233)(79.39397968,398.86205061)
}
}
{
\newrgbcolor{curcolor}{0 0 0}
\pscustom[linestyle=none,fillstyle=solid,fillcolor=curcolor]
{
\newpath
\moveto(86.37858906,398.84705061)
\curveto(86.37858906,396.82205263)(87.51859156,394.70705061)(90.02358906,394.70705061)
\curveto(92.52858655,394.70705061)(93.66858906,396.82205263)(93.66858906,398.84705061)
\curveto(93.66858906,400.87204858)(92.52858655,402.98705061)(90.02358906,402.98705061)
\curveto(87.51859156,402.98705061)(86.37858906,400.87204858)(86.37858906,398.84705061)
\moveto(87.74358906,398.84705061)
\curveto(87.74358906,399.89704956)(88.13359095,401.84705061)(90.02358906,401.84705061)
\curveto(91.91358717,401.84705061)(92.30358906,399.89704956)(92.30358906,398.84705061)
\curveto(92.30358906,397.79705166)(91.91358717,395.84705061)(90.02358906,395.84705061)
\curveto(88.13359095,395.84705061)(87.74358906,397.79705166)(87.74358906,398.84705061)
}
}
{
\newrgbcolor{curcolor}{0 0 0}
\pscustom[linestyle=none,fillstyle=solid,fillcolor=curcolor]
{
\newpath
\moveto(96.67819843,399.47705061)
\curveto(96.67819843,400.61704947)(97.45819966,401.57705061)(98.68819843,401.57705061)
\lineto(99.18319843,401.57705061)
\lineto(99.18319843,402.94205061)
\curveto(99.07819854,402.97205058)(99.00319827,402.98705061)(98.83819843,402.98705061)
\curveto(97.84819942,402.98705061)(97.15819791,402.37204969)(96.63319843,401.45705061)
\lineto(96.60319843,401.45705061)
\lineto(96.60319843,402.76205061)
\lineto(95.35819843,402.76205061)
\lineto(95.35819843,394.91705061)
\lineto(96.67819843,394.91705061)
\lineto(96.67819843,399.47705061)
}
}
{
\newrgbcolor{curcolor}{0 0 0}
\pscustom[linestyle=none,fillstyle=solid,fillcolor=curcolor]
{
\newpath
\moveto(7.10569843,377.92681623)
\lineto(11.41069843,377.92681623)
\curveto(14.95069489,377.92681623)(16.00069843,381.04681865)(16.00069843,383.46181623)
\curveto(16.00069843,386.56681313)(14.27569563,388.69681623)(11.47069843,388.69681623)
\lineto(7.10569843,388.69681623)
\lineto(7.10569843,377.92681623)
\moveto(8.56069843,387.45181623)
\lineto(11.27569843,387.45181623)
\curveto(13.25569645,387.45181623)(14.50069843,386.08681352)(14.50069843,383.37181623)
\curveto(14.50069843,380.65681895)(13.27069654,379.17181623)(11.38069843,379.17181623)
\lineto(8.56069843,379.17181623)
\lineto(8.56069843,387.45181623)
}
}
{
\newrgbcolor{curcolor}{0 0 0}
\pscustom[linestyle=none,fillstyle=solid,fillcolor=curcolor]
{
\newpath
\moveto(23.04554218,380.38681623)
\curveto(23.00054223,379.80181682)(22.26554094,378.84181623)(21.02054218,378.84181623)
\curveto(19.5055437,378.84181623)(18.74054218,379.78681787)(18.74054218,381.42181623)
\lineto(24.47054218,381.42181623)
\curveto(24.47054218,384.19681346)(23.36053992,385.99681623)(21.09554218,385.99681623)
\curveto(18.50054478,385.99681623)(17.33054218,384.0618138)(17.33054218,381.63181623)
\curveto(17.33054218,379.3668185)(18.63554439,377.70181623)(20.84054218,377.70181623)
\curveto(22.10054092,377.70181623)(22.61054254,378.00181647)(22.97054218,378.24181623)
\curveto(23.96054119,378.90181557)(24.32054223,380.01181661)(24.36554218,380.38681623)
\lineto(23.04554218,380.38681623)
\moveto(18.74054218,382.47181623)
\curveto(18.74054218,383.68681502)(19.7005434,384.81181623)(20.91554218,384.81181623)
\curveto(22.52054058,384.81181623)(23.03054226,383.68681502)(23.10554218,382.47181623)
\lineto(18.74054218,382.47181623)
}
}
{
\newrgbcolor{curcolor}{0 0 0}
\pscustom[linestyle=none,fillstyle=solid,fillcolor=curcolor]
{
\newpath
\moveto(31.78515156,383.53681623)
\curveto(31.78515156,383.92681584)(31.59014875,385.99681623)(28.78515156,385.99681623)
\curveto(27.2401531,385.99681623)(25.81515156,385.21681451)(25.81515156,383.49181623)
\curveto(25.81515156,382.41181731)(26.53515265,381.85681596)(27.63015156,381.58681623)
\lineto(29.16015156,381.21181623)
\curveto(30.28515043,380.92681652)(30.72015156,380.7168156)(30.72015156,380.08681623)
\curveto(30.72015156,379.2168171)(29.86515061,378.84181623)(28.92015156,378.84181623)
\curveto(27.06015342,378.84181623)(26.88015151,379.83181685)(26.83515156,380.44681623)
\lineto(25.56015156,380.44681623)
\curveto(25.60515151,379.50181718)(25.83015466,377.70181623)(28.93515156,377.70181623)
\curveto(30.70514979,377.70181623)(32.04015156,378.67681785)(32.04015156,380.29681623)
\curveto(32.04015156,381.36181517)(31.47014992,381.96181664)(29.83515156,382.36681623)
\lineto(28.51515156,382.69681623)
\curveto(27.49515258,382.95181598)(27.09015156,383.10181688)(27.09015156,383.74681623)
\curveto(27.09015156,384.72181526)(28.24515196,384.85681623)(28.65015156,384.85681623)
\curveto(30.31514989,384.85681623)(30.49515157,384.03181574)(30.51015156,383.53681623)
\lineto(31.78515156,383.53681623)
}
}
{
\newrgbcolor{curcolor}{0 0 0}
\pscustom[linestyle=none,fillstyle=solid,fillcolor=curcolor]
{
\newpath
\moveto(34.72515156,383.38681623)
\curveto(34.81515147,383.98681563)(35.02515306,384.90181623)(36.52515156,384.90181623)
\curveto(37.77015031,384.90181623)(38.37015156,384.45181541)(38.37015156,383.62681623)
\curveto(38.37015156,382.84681701)(37.99515124,382.7268162)(37.68015156,382.69681623)
\lineto(35.50515156,382.42681623)
\curveto(33.31515375,382.1568165)(33.12015156,380.62681557)(33.12015156,379.96681623)
\curveto(33.12015156,378.61681758)(34.140153,377.70181623)(35.58015156,377.70181623)
\curveto(37.11015003,377.70181623)(37.90515207,378.42181679)(38.41515156,378.97681623)
\curveto(38.46015151,378.37681683)(38.64015273,377.77681623)(39.81015156,377.77681623)
\curveto(40.11015126,377.77681623)(40.30515178,377.86681629)(40.53015156,377.92681623)
\lineto(40.53015156,378.88681623)
\curveto(40.38015171,378.85681626)(40.21515144,378.82681623)(40.09515156,378.82681623)
\curveto(39.82515183,378.82681623)(39.66015156,378.96181656)(39.66015156,379.29181623)
\lineto(39.66015156,383.80681623)
\curveto(39.66015156,385.81681422)(37.38015093,385.99681623)(36.75015156,385.99681623)
\curveto(34.81515349,385.99681623)(33.5701515,385.26181436)(33.51015156,383.38681623)
\lineto(34.72515156,383.38681623)
\moveto(38.34015156,380.64181623)
\curveto(38.34015156,379.59181728)(37.14015033,378.79681623)(35.91015156,378.79681623)
\curveto(34.92015255,378.79681623)(34.48515156,379.30681709)(34.48515156,380.16181623)
\curveto(34.48515156,381.15181524)(35.5201522,381.34681632)(36.16515156,381.43681623)
\curveto(37.80014992,381.64681602)(38.13015177,381.7668164)(38.34015156,381.93181623)
\lineto(38.34015156,380.64181623)
}
}
{
\newrgbcolor{curcolor}{0 0 0}
\pscustom[linestyle=none,fillstyle=solid,fillcolor=curcolor]
{
\newpath
\moveto(43.40476093,382.48681623)
\curveto(43.40476093,383.62681509)(44.18476216,384.58681623)(45.41476093,384.58681623)
\lineto(45.90976093,384.58681623)
\lineto(45.90976093,385.95181623)
\curveto(45.80476104,385.9818162)(45.72976077,385.99681623)(45.56476093,385.99681623)
\curveto(44.57476192,385.99681623)(43.88476041,385.38181532)(43.35976093,384.46681623)
\lineto(43.32976093,384.46681623)
\lineto(43.32976093,385.77181623)
\lineto(42.08476093,385.77181623)
\lineto(42.08476093,377.92681623)
\lineto(43.40476093,377.92681623)
\lineto(43.40476093,382.48681623)
}
}
{
\newrgbcolor{curcolor}{0 0 0}
\pscustom[linestyle=none,fillstyle=solid,fillcolor=curcolor]
{
\newpath
\moveto(48.41452656,382.48681623)
\curveto(48.41452656,383.62681509)(49.19452779,384.58681623)(50.42452656,384.58681623)
\lineto(50.91952656,384.58681623)
\lineto(50.91952656,385.95181623)
\curveto(50.81452666,385.9818162)(50.73952639,385.99681623)(50.57452656,385.99681623)
\curveto(49.58452755,385.99681623)(48.89452603,385.38181532)(48.36952656,384.46681623)
\lineto(48.33952656,384.46681623)
\lineto(48.33952656,385.77181623)
\lineto(47.09452656,385.77181623)
\lineto(47.09452656,377.92681623)
\lineto(48.41452656,377.92681623)
\lineto(48.41452656,382.48681623)
}
}
{
\newrgbcolor{curcolor}{0 0 0}
\pscustom[linestyle=none,fillstyle=solid,fillcolor=curcolor]
{
\newpath
\moveto(51.18132343,381.85681623)
\curveto(51.18132343,379.83181826)(52.32132594,377.71681623)(54.82632343,377.71681623)
\curveto(57.33132093,377.71681623)(58.47132343,379.83181826)(58.47132343,381.85681623)
\curveto(58.47132343,383.88181421)(57.33132093,385.99681623)(54.82632343,385.99681623)
\curveto(52.32132594,385.99681623)(51.18132343,383.88181421)(51.18132343,381.85681623)
\moveto(52.54632343,381.85681623)
\curveto(52.54632343,382.90681518)(52.93632532,384.85681623)(54.82632343,384.85681623)
\curveto(56.71632154,384.85681623)(57.10632343,382.90681518)(57.10632343,381.85681623)
\curveto(57.10632343,380.80681728)(56.71632154,378.85681623)(54.82632343,378.85681623)
\curveto(52.93632532,378.85681623)(52.54632343,380.80681728)(52.54632343,381.85681623)
}
}
{
\newrgbcolor{curcolor}{0 0 0}
\pscustom[linestyle=none,fillstyle=solid,fillcolor=curcolor]
{
\newpath
\moveto(61.33093281,388.69681623)
\lineto(60.01093281,388.69681623)
\lineto(60.01093281,377.92681623)
\lineto(61.33093281,377.92681623)
\lineto(61.33093281,388.69681623)
}
}
{
\newrgbcolor{curcolor}{0 0 0}
\pscustom[linestyle=none,fillstyle=solid,fillcolor=curcolor]
{
\newpath
\moveto(64.67077656,388.69681623)
\lineto(63.35077656,388.69681623)
\lineto(63.35077656,377.92681623)
\lineto(64.67077656,377.92681623)
\lineto(64.67077656,388.69681623)
}
}
{
\newrgbcolor{curcolor}{0 0 0}
\pscustom[linestyle=none,fillstyle=solid,fillcolor=curcolor]
{
\newpath
\moveto(67.83062031,383.38681623)
\curveto(67.92062022,383.98681563)(68.13062181,384.90181623)(69.63062031,384.90181623)
\curveto(70.87561906,384.90181623)(71.47562031,384.45181541)(71.47562031,383.62681623)
\curveto(71.47562031,382.84681701)(71.10061999,382.7268162)(70.78562031,382.69681623)
\lineto(68.61062031,382.42681623)
\curveto(66.4206225,382.1568165)(66.22562031,380.62681557)(66.22562031,379.96681623)
\curveto(66.22562031,378.61681758)(67.24562175,377.70181623)(68.68562031,377.70181623)
\curveto(70.21561878,377.70181623)(71.01062082,378.42181679)(71.52062031,378.97681623)
\curveto(71.56562026,378.37681683)(71.74562148,377.77681623)(72.91562031,377.77681623)
\curveto(73.21562001,377.77681623)(73.41062053,377.86681629)(73.63562031,377.92681623)
\lineto(73.63562031,378.88681623)
\curveto(73.48562046,378.85681626)(73.32062019,378.82681623)(73.20062031,378.82681623)
\curveto(72.93062058,378.82681623)(72.76562031,378.96181656)(72.76562031,379.29181623)
\lineto(72.76562031,383.80681623)
\curveto(72.76562031,385.81681422)(70.48561968,385.99681623)(69.85562031,385.99681623)
\curveto(67.92062224,385.99681623)(66.67562025,385.26181436)(66.61562031,383.38681623)
\lineto(67.83062031,383.38681623)
\moveto(71.44562031,380.64181623)
\curveto(71.44562031,379.59181728)(70.24561908,378.79681623)(69.01562031,378.79681623)
\curveto(68.0256213,378.79681623)(67.59062031,379.30681709)(67.59062031,380.16181623)
\curveto(67.59062031,381.15181524)(68.62562095,381.34681632)(69.27062031,381.43681623)
\curveto(70.90561867,381.64681602)(71.23562052,381.7668164)(71.44562031,381.93181623)
\lineto(71.44562031,380.64181623)
}
}
{
\newrgbcolor{curcolor}{0 0 0}
\pscustom[linestyle=none,fillstyle=solid,fillcolor=curcolor]
{
\newpath
\moveto(81.52022968,388.69681623)
\lineto(80.20022968,388.69681623)
\lineto(80.20022968,384.76681623)
\lineto(80.17022968,384.66181623)
\curveto(79.85523,385.11181578)(79.25522826,385.99681623)(77.83022968,385.99681623)
\curveto(75.74523177,385.99681623)(74.56022968,384.28681403)(74.56022968,382.08181623)
\curveto(74.56022968,380.20681811)(75.34023235,377.70181623)(78.01022968,377.70181623)
\curveto(78.77522892,377.70181623)(79.67523025,377.9418173)(80.24522968,379.00681623)
\lineto(80.27522968,379.00681623)
\lineto(80.27522968,377.92681623)
\lineto(81.52022968,377.92681623)
\lineto(81.52022968,388.69681623)
\moveto(75.92522968,381.87181623)
\curveto(75.92522968,382.87681523)(76.03023172,384.81181623)(78.07022968,384.81181623)
\curveto(79.97522778,384.81181623)(80.18522968,382.75681496)(80.18522968,381.48181623)
\curveto(80.18522968,379.39681832)(78.88022884,378.84181623)(78.04022968,378.84181623)
\curveto(76.60023112,378.84181623)(75.92522968,380.14681796)(75.92522968,381.87181623)
}
}
{
\newrgbcolor{curcolor}{0 0 0}
\pscustom[linestyle=none,fillstyle=solid,fillcolor=curcolor]
{
\newpath
\moveto(82.90983906,381.85681623)
\curveto(82.90983906,379.83181826)(84.04984156,377.71681623)(86.55483906,377.71681623)
\curveto(89.05983655,377.71681623)(90.19983906,379.83181826)(90.19983906,381.85681623)
\curveto(90.19983906,383.88181421)(89.05983655,385.99681623)(86.55483906,385.99681623)
\curveto(84.04984156,385.99681623)(82.90983906,383.88181421)(82.90983906,381.85681623)
\moveto(84.27483906,381.85681623)
\curveto(84.27483906,382.90681518)(84.66484095,384.85681623)(86.55483906,384.85681623)
\curveto(88.44483717,384.85681623)(88.83483906,382.90681518)(88.83483906,381.85681623)
\curveto(88.83483906,380.80681728)(88.44483717,378.85681623)(86.55483906,378.85681623)
\curveto(84.66484095,378.85681623)(84.27483906,380.80681728)(84.27483906,381.85681623)
}
}
{
\newrgbcolor{curcolor}{0 0 0}
\pscustom[linestyle=none,fillstyle=solid,fillcolor=curcolor]
{
\newpath
\moveto(93.20944843,382.48681623)
\curveto(93.20944843,383.62681509)(93.98944966,384.58681623)(95.21944843,384.58681623)
\lineto(95.71444843,384.58681623)
\lineto(95.71444843,385.95181623)
\curveto(95.60944854,385.9818162)(95.53444827,385.99681623)(95.36944843,385.99681623)
\curveto(94.37944942,385.99681623)(93.68944791,385.38181532)(93.16444843,384.46681623)
\lineto(93.13444843,384.46681623)
\lineto(93.13444843,385.77181623)
\lineto(91.88944843,385.77181623)
\lineto(91.88944843,377.92681623)
\lineto(93.20944843,377.92681623)
\lineto(93.20944843,382.48681623)
}
}
{
\newrgbcolor{curcolor}{0 0 0}
\pscustom[linestyle=none,fillstyle=solid,fillcolor=curcolor]
{
\newpath
\moveto(8.8457082,490.63810041)
\lineto(7.3907082,490.63810041)
\lineto(7.3907082,479.86810041)
\lineto(8.8457082,479.86810041)
\lineto(8.8457082,490.63810041)
}
}
{
\newrgbcolor{curcolor}{0 0 0}
\pscustom[linestyle=none,fillstyle=solid,fillcolor=curcolor]
{
\newpath
\moveto(17.55086445,485.20810041)
\curveto(17.55086445,487.44309818)(16.02086323,487.93810041)(14.80586445,487.93810041)
\curveto(13.4558658,487.93810041)(12.72086416,487.02309999)(12.43586445,486.60310041)
\lineto(12.40586445,486.60310041)
\lineto(12.40586445,487.71310041)
\lineto(11.16086445,487.71310041)
\lineto(11.16086445,479.86810041)
\lineto(12.48086445,479.86810041)
\lineto(12.48086445,484.14310041)
\curveto(12.48086445,486.27309828)(13.8008652,486.75310041)(14.55086445,486.75310041)
\curveto(15.84086316,486.75310041)(16.23086445,486.06309905)(16.23086445,484.69810041)
\lineto(16.23086445,479.86810041)
\lineto(17.55086445,479.86810041)
\lineto(17.55086445,485.20810041)
}
}
{
\newrgbcolor{curcolor}{0 0 0}
\pscustom[linestyle=none,fillstyle=solid,fillcolor=curcolor]
{
\newpath
\moveto(21.96250507,481.32310041)
\lineto(21.93250507,481.32310041)
\lineto(19.89250507,487.71310041)
\lineto(18.36250507,487.71310041)
\lineto(21.22750507,479.86810041)
\lineto(22.63750507,479.86810041)
\lineto(25.62250507,487.71310041)
\lineto(24.18250507,487.71310041)
\lineto(21.96250507,481.32310041)
}
}
{
\newrgbcolor{curcolor}{0 0 0}
\pscustom[linestyle=none,fillstyle=solid,fillcolor=curcolor]
{
\newpath
\moveto(28.06750507,487.71310041)
\lineto(26.74750507,487.71310041)
\lineto(26.74750507,479.86810041)
\lineto(28.06750507,479.86810041)
\lineto(28.06750507,487.71310041)
\moveto(28.06750507,489.13810041)
\lineto(28.06750507,490.63810041)
\lineto(26.74750507,490.63810041)
\lineto(26.74750507,489.13810041)
\lineto(28.06750507,489.13810041)
}
}
{
\newrgbcolor{curcolor}{0 0 0}
\pscustom[linestyle=none,fillstyle=solid,fillcolor=curcolor]
{
\newpath
\moveto(32.93734882,486.61810041)
\lineto(32.93734882,487.71310041)
\lineto(31.67734882,487.71310041)
\lineto(31.67734882,489.90310041)
\lineto(30.35734882,489.90310041)
\lineto(30.35734882,487.71310041)
\lineto(29.29234882,487.71310041)
\lineto(29.29234882,486.61810041)
\lineto(30.35734882,486.61810041)
\lineto(30.35734882,481.44310041)
\curveto(30.35734882,480.49810136)(30.64235013,479.76310041)(31.94734882,479.76310041)
\curveto(32.08234869,479.76310041)(32.4573493,479.82310046)(32.93734882,479.86810041)
\lineto(32.93734882,480.90310041)
\lineto(32.47234882,480.90310041)
\curveto(32.20234909,480.90310041)(31.67734882,480.90310103)(31.67734882,481.51810041)
\lineto(31.67734882,486.61810041)
\lineto(32.93734882,486.61810041)
}
}
{
\newrgbcolor{curcolor}{0 0 0}
\pscustom[linestyle=none,fillstyle=solid,fillcolor=curcolor]
{
\newpath
\moveto(35.2410207,485.32810041)
\curveto(35.33102061,485.92809981)(35.5410222,486.84310041)(37.0410207,486.84310041)
\curveto(38.28601945,486.84310041)(38.8860207,486.39309959)(38.8860207,485.56810041)
\curveto(38.8860207,484.78810119)(38.51102038,484.66810038)(38.1960207,484.63810041)
\lineto(36.0210207,484.36810041)
\curveto(33.83102289,484.09810068)(33.6360207,482.56809975)(33.6360207,481.90810041)
\curveto(33.6360207,480.55810176)(34.65602214,479.64310041)(36.0960207,479.64310041)
\curveto(37.62601917,479.64310041)(38.42102121,480.36310097)(38.9310207,480.91810041)
\curveto(38.97602065,480.31810101)(39.15602187,479.71810041)(40.3260207,479.71810041)
\curveto(40.6260204,479.71810041)(40.82102092,479.80810047)(41.0460207,479.86810041)
\lineto(41.0460207,480.82810041)
\curveto(40.89602085,480.79810044)(40.73102058,480.76810041)(40.6110207,480.76810041)
\curveto(40.34102097,480.76810041)(40.1760207,480.90310074)(40.1760207,481.23310041)
\lineto(40.1760207,485.74810041)
\curveto(40.1760207,487.7580984)(37.89602007,487.93810041)(37.2660207,487.93810041)
\curveto(35.33102263,487.93810041)(34.08602064,487.20309854)(34.0260207,485.32810041)
\lineto(35.2410207,485.32810041)
\moveto(38.8560207,482.58310041)
\curveto(38.8560207,481.53310146)(37.65601947,480.73810041)(36.4260207,480.73810041)
\curveto(35.43602169,480.73810041)(35.0010207,481.24810127)(35.0010207,482.10310041)
\curveto(35.0010207,483.09309942)(36.03602134,483.2881005)(36.6810207,483.37810041)
\curveto(38.31601906,483.5881002)(38.64602091,483.70810058)(38.8560207,483.87310041)
\lineto(38.8560207,482.58310041)
}
}
{
\newrgbcolor{curcolor}{0 0 0}
\pscustom[linestyle=none,fillstyle=solid,fillcolor=curcolor]
{
\newpath
\moveto(48.93063007,490.63810041)
\lineto(47.61063007,490.63810041)
\lineto(47.61063007,486.70810041)
\lineto(47.58063007,486.60310041)
\curveto(47.26563039,487.05309996)(46.66562865,487.93810041)(45.24063007,487.93810041)
\curveto(43.15563216,487.93810041)(41.97063007,486.22809821)(41.97063007,484.02310041)
\curveto(41.97063007,482.14810229)(42.75063274,479.64310041)(45.42063007,479.64310041)
\curveto(46.18562931,479.64310041)(47.08563064,479.88310148)(47.65563007,480.94810041)
\lineto(47.68563007,480.94810041)
\lineto(47.68563007,479.86810041)
\lineto(48.93063007,479.86810041)
\lineto(48.93063007,490.63810041)
\moveto(43.33563007,483.81310041)
\curveto(43.33563007,484.81809941)(43.44063211,486.75310041)(45.48063007,486.75310041)
\curveto(47.38562817,486.75310041)(47.59563007,484.69809914)(47.59563007,483.42310041)
\curveto(47.59563007,481.3381025)(46.29062923,480.78310041)(45.45063007,480.78310041)
\curveto(44.01063151,480.78310041)(43.33563007,482.08810214)(43.33563007,483.81310041)
}
}
{
\newrgbcolor{curcolor}{0 0 0}
\pscustom[linestyle=none,fillstyle=solid,fillcolor=curcolor]
{
\newpath
\moveto(50.32023945,483.79810041)
\curveto(50.32023945,481.77310244)(51.46024195,479.65810041)(53.96523945,479.65810041)
\curveto(56.47023694,479.65810041)(57.61023945,481.77310244)(57.61023945,483.79810041)
\curveto(57.61023945,485.82309839)(56.47023694,487.93810041)(53.96523945,487.93810041)
\curveto(51.46024195,487.93810041)(50.32023945,485.82309839)(50.32023945,483.79810041)
\moveto(51.68523945,483.79810041)
\curveto(51.68523945,484.84809936)(52.07524134,486.79810041)(53.96523945,486.79810041)
\curveto(55.85523756,486.79810041)(56.24523945,484.84809936)(56.24523945,483.79810041)
\curveto(56.24523945,482.74810146)(55.85523756,480.79810041)(53.96523945,480.79810041)
\curveto(52.07524134,480.79810041)(51.68523945,482.74810146)(51.68523945,483.79810041)
}
}
{
\newrgbcolor{curcolor}{0 0 0}
\pscustom[linestyle=none,fillstyle=solid,fillcolor=curcolor]
{
\newpath
\moveto(8.8457082,473.64786604)
\lineto(7.3907082,473.64786604)
\lineto(7.3907082,462.87786604)
\lineto(8.8457082,462.87786604)
\lineto(8.8457082,473.64786604)
}
}
{
\newrgbcolor{curcolor}{0 0 0}
\pscustom[linestyle=none,fillstyle=solid,fillcolor=curcolor]
{
\newpath
\moveto(17.55086445,468.21786604)
\curveto(17.55086445,470.4528638)(16.02086323,470.94786604)(14.80586445,470.94786604)
\curveto(13.4558658,470.94786604)(12.72086416,470.03286562)(12.43586445,469.61286604)
\lineto(12.40586445,469.61286604)
\lineto(12.40586445,470.72286604)
\lineto(11.16086445,470.72286604)
\lineto(11.16086445,462.87786604)
\lineto(12.48086445,462.87786604)
\lineto(12.48086445,467.15286604)
\curveto(12.48086445,469.28286391)(13.8008652,469.76286604)(14.55086445,469.76286604)
\curveto(15.84086316,469.76286604)(16.23086445,469.07286467)(16.23086445,467.70786604)
\lineto(16.23086445,462.87786604)
\lineto(17.55086445,462.87786604)
\lineto(17.55086445,468.21786604)
}
}
{
\newrgbcolor{curcolor}{0 0 0}
\pscustom[linestyle=none,fillstyle=solid,fillcolor=curcolor]
{
\newpath
\moveto(21.96250507,464.33286604)
\lineto(21.93250507,464.33286604)
\lineto(19.89250507,470.72286604)
\lineto(18.36250507,470.72286604)
\lineto(21.22750507,462.87786604)
\lineto(22.63750507,462.87786604)
\lineto(25.62250507,470.72286604)
\lineto(24.18250507,470.72286604)
\lineto(21.96250507,464.33286604)
}
}
{
\newrgbcolor{curcolor}{0 0 0}
\pscustom[linestyle=none,fillstyle=solid,fillcolor=curcolor]
{
\newpath
\moveto(28.06750507,470.72286604)
\lineto(26.74750507,470.72286604)
\lineto(26.74750507,462.87786604)
\lineto(28.06750507,462.87786604)
\lineto(28.06750507,470.72286604)
\moveto(28.06750507,472.14786604)
\lineto(28.06750507,473.64786604)
\lineto(26.74750507,473.64786604)
\lineto(26.74750507,472.14786604)
\lineto(28.06750507,472.14786604)
}
}
{
\newrgbcolor{curcolor}{0 0 0}
\pscustom[linestyle=none,fillstyle=solid,fillcolor=curcolor]
{
\newpath
\moveto(32.93734882,469.62786604)
\lineto(32.93734882,470.72286604)
\lineto(31.67734882,470.72286604)
\lineto(31.67734882,472.91286604)
\lineto(30.35734882,472.91286604)
\lineto(30.35734882,470.72286604)
\lineto(29.29234882,470.72286604)
\lineto(29.29234882,469.62786604)
\lineto(30.35734882,469.62786604)
\lineto(30.35734882,464.45286604)
\curveto(30.35734882,463.50786698)(30.64235013,462.77286604)(31.94734882,462.77286604)
\curveto(32.08234869,462.77286604)(32.4573493,462.83286608)(32.93734882,462.87786604)
\lineto(32.93734882,463.91286604)
\lineto(32.47234882,463.91286604)
\curveto(32.20234909,463.91286604)(31.67734882,463.91286665)(31.67734882,464.52786604)
\lineto(31.67734882,469.62786604)
\lineto(32.93734882,469.62786604)
}
}
{
\newrgbcolor{curcolor}{0 0 0}
\pscustom[linestyle=none,fillstyle=solid,fillcolor=curcolor]
{
\newpath
\moveto(35.2410207,468.33786604)
\curveto(35.33102061,468.93786544)(35.5410222,469.85286604)(37.0410207,469.85286604)
\curveto(38.28601945,469.85286604)(38.8860207,469.40286521)(38.8860207,468.57786604)
\curveto(38.8860207,467.79786682)(38.51102038,467.67786601)(38.1960207,467.64786604)
\lineto(36.0210207,467.37786604)
\curveto(33.83102289,467.10786631)(33.6360207,465.57786538)(33.6360207,464.91786604)
\curveto(33.6360207,463.56786739)(34.65602214,462.65286604)(36.0960207,462.65286604)
\curveto(37.62601917,462.65286604)(38.42102121,463.37286659)(38.9310207,463.92786604)
\curveto(38.97602065,463.32786664)(39.15602187,462.72786604)(40.3260207,462.72786604)
\curveto(40.6260204,462.72786604)(40.82102092,462.8178661)(41.0460207,462.87786604)
\lineto(41.0460207,463.83786604)
\curveto(40.89602085,463.80786607)(40.73102058,463.77786604)(40.6110207,463.77786604)
\curveto(40.34102097,463.77786604)(40.1760207,463.91286637)(40.1760207,464.24286604)
\lineto(40.1760207,468.75786604)
\curveto(40.1760207,470.76786403)(37.89602007,470.94786604)(37.2660207,470.94786604)
\curveto(35.33102263,470.94786604)(34.08602064,470.21286416)(34.0260207,468.33786604)
\lineto(35.2410207,468.33786604)
\moveto(38.8560207,465.59286604)
\curveto(38.8560207,464.54286709)(37.65601947,463.74786604)(36.4260207,463.74786604)
\curveto(35.43602169,463.74786604)(35.0010207,464.25786689)(35.0010207,465.11286604)
\curveto(35.0010207,466.10286505)(36.03602134,466.29786613)(36.6810207,466.38786604)
\curveto(38.31601906,466.59786583)(38.64602091,466.7178662)(38.8560207,466.88286604)
\lineto(38.8560207,465.59286604)
}
}
{
\newrgbcolor{curcolor}{0 0 0}
\pscustom[linestyle=none,fillstyle=solid,fillcolor=curcolor]
{
\newpath
\moveto(48.93063007,473.64786604)
\lineto(47.61063007,473.64786604)
\lineto(47.61063007,469.71786604)
\lineto(47.58063007,469.61286604)
\curveto(47.26563039,470.06286559)(46.66562865,470.94786604)(45.24063007,470.94786604)
\curveto(43.15563216,470.94786604)(41.97063007,469.23786383)(41.97063007,467.03286604)
\curveto(41.97063007,465.15786791)(42.75063274,462.65286604)(45.42063007,462.65286604)
\curveto(46.18562931,462.65286604)(47.08563064,462.8928671)(47.65563007,463.95786604)
\lineto(47.68563007,463.95786604)
\lineto(47.68563007,462.87786604)
\lineto(48.93063007,462.87786604)
\lineto(48.93063007,473.64786604)
\moveto(43.33563007,466.82286604)
\curveto(43.33563007,467.82786503)(43.44063211,469.76286604)(45.48063007,469.76286604)
\curveto(47.38562817,469.76286604)(47.59563007,467.70786476)(47.59563007,466.43286604)
\curveto(47.59563007,464.34786812)(46.29062923,463.79286604)(45.45063007,463.79286604)
\curveto(44.01063151,463.79286604)(43.33563007,465.09786776)(43.33563007,466.82286604)
}
}
{
\newrgbcolor{curcolor}{0 0 0}
\pscustom[linestyle=none,fillstyle=solid,fillcolor=curcolor]
{
\newpath
\moveto(50.32023945,466.80786604)
\curveto(50.32023945,464.78286806)(51.46024195,462.66786604)(53.96523945,462.66786604)
\curveto(56.47023694,462.66786604)(57.61023945,464.78286806)(57.61023945,466.80786604)
\curveto(57.61023945,468.83286401)(56.47023694,470.94786604)(53.96523945,470.94786604)
\curveto(51.46024195,470.94786604)(50.32023945,468.83286401)(50.32023945,466.80786604)
\moveto(51.68523945,466.80786604)
\curveto(51.68523945,467.85786499)(52.07524134,469.80786604)(53.96523945,469.80786604)
\curveto(55.85523756,469.80786604)(56.24523945,467.85786499)(56.24523945,466.80786604)
\curveto(56.24523945,465.75786709)(55.85523756,463.80786604)(53.96523945,463.80786604)
\curveto(52.07524134,463.80786604)(51.68523945,465.75786709)(51.68523945,466.80786604)
}
}
{
\newrgbcolor{curcolor}{0 0 0}
\pscustom[linestyle=none,fillstyle=solid,fillcolor=curcolor]
{
\newpath
\moveto(8.8457082,456.65763166)
\lineto(7.3907082,456.65763166)
\lineto(7.3907082,445.88763166)
\lineto(8.8457082,445.88763166)
\lineto(8.8457082,456.65763166)
}
}
{
\newrgbcolor{curcolor}{0 0 0}
\pscustom[linestyle=none,fillstyle=solid,fillcolor=curcolor]
{
\newpath
\moveto(17.55086445,451.22763166)
\curveto(17.55086445,453.46262943)(16.02086323,453.95763166)(14.80586445,453.95763166)
\curveto(13.4558658,453.95763166)(12.72086416,453.04263124)(12.43586445,452.62263166)
\lineto(12.40586445,452.62263166)
\lineto(12.40586445,453.73263166)
\lineto(11.16086445,453.73263166)
\lineto(11.16086445,445.88763166)
\lineto(12.48086445,445.88763166)
\lineto(12.48086445,450.16263166)
\curveto(12.48086445,452.29262953)(13.8008652,452.77263166)(14.55086445,452.77263166)
\curveto(15.84086316,452.77263166)(16.23086445,452.0826303)(16.23086445,450.71763166)
\lineto(16.23086445,445.88763166)
\lineto(17.55086445,445.88763166)
\lineto(17.55086445,451.22763166)
}
}
{
\newrgbcolor{curcolor}{0 0 0}
\pscustom[linestyle=none,fillstyle=solid,fillcolor=curcolor]
{
\newpath
\moveto(21.96250507,447.34263166)
\lineto(21.93250507,447.34263166)
\lineto(19.89250507,453.73263166)
\lineto(18.36250507,453.73263166)
\lineto(21.22750507,445.88763166)
\lineto(22.63750507,445.88763166)
\lineto(25.62250507,453.73263166)
\lineto(24.18250507,453.73263166)
\lineto(21.96250507,447.34263166)
}
}
{
\newrgbcolor{curcolor}{0 0 0}
\pscustom[linestyle=none,fillstyle=solid,fillcolor=curcolor]
{
\newpath
\moveto(28.06750507,453.73263166)
\lineto(26.74750507,453.73263166)
\lineto(26.74750507,445.88763166)
\lineto(28.06750507,445.88763166)
\lineto(28.06750507,453.73263166)
\moveto(28.06750507,455.15763166)
\lineto(28.06750507,456.65763166)
\lineto(26.74750507,456.65763166)
\lineto(26.74750507,455.15763166)
\lineto(28.06750507,455.15763166)
}
}
{
\newrgbcolor{curcolor}{0 0 0}
\pscustom[linestyle=none,fillstyle=solid,fillcolor=curcolor]
{
\newpath
\moveto(32.93734882,452.63763166)
\lineto(32.93734882,453.73263166)
\lineto(31.67734882,453.73263166)
\lineto(31.67734882,455.92263166)
\lineto(30.35734882,455.92263166)
\lineto(30.35734882,453.73263166)
\lineto(29.29234882,453.73263166)
\lineto(29.29234882,452.63763166)
\lineto(30.35734882,452.63763166)
\lineto(30.35734882,447.46263166)
\curveto(30.35734882,446.51763261)(30.64235013,445.78263166)(31.94734882,445.78263166)
\curveto(32.08234869,445.78263166)(32.4573493,445.84263171)(32.93734882,445.88763166)
\lineto(32.93734882,446.92263166)
\lineto(32.47234882,446.92263166)
\curveto(32.20234909,446.92263166)(31.67734882,446.92263228)(31.67734882,447.53763166)
\lineto(31.67734882,452.63763166)
\lineto(32.93734882,452.63763166)
}
}
{
\newrgbcolor{curcolor}{0 0 0}
\pscustom[linestyle=none,fillstyle=solid,fillcolor=curcolor]
{
\newpath
\moveto(35.2410207,451.34763166)
\curveto(35.33102061,451.94763106)(35.5410222,452.86263166)(37.0410207,452.86263166)
\curveto(38.28601945,452.86263166)(38.8860207,452.41263084)(38.8860207,451.58763166)
\curveto(38.8860207,450.80763244)(38.51102038,450.68763163)(38.1960207,450.65763166)
\lineto(36.0210207,450.38763166)
\curveto(33.83102289,450.11763193)(33.6360207,448.587631)(33.6360207,447.92763166)
\curveto(33.6360207,446.57763301)(34.65602214,445.66263166)(36.0960207,445.66263166)
\curveto(37.62601917,445.66263166)(38.42102121,446.38263222)(38.9310207,446.93763166)
\curveto(38.97602065,446.33763226)(39.15602187,445.73763166)(40.3260207,445.73763166)
\curveto(40.6260204,445.73763166)(40.82102092,445.82763172)(41.0460207,445.88763166)
\lineto(41.0460207,446.84763166)
\curveto(40.89602085,446.81763169)(40.73102058,446.78763166)(40.6110207,446.78763166)
\curveto(40.34102097,446.78763166)(40.1760207,446.92263199)(40.1760207,447.25263166)
\lineto(40.1760207,451.76763166)
\curveto(40.1760207,453.77762965)(37.89602007,453.95763166)(37.2660207,453.95763166)
\curveto(35.33102263,453.95763166)(34.08602064,453.22262979)(34.0260207,451.34763166)
\lineto(35.2410207,451.34763166)
\moveto(38.8560207,448.60263166)
\curveto(38.8560207,447.55263271)(37.65601947,446.75763166)(36.4260207,446.75763166)
\curveto(35.43602169,446.75763166)(35.0010207,447.26763252)(35.0010207,448.12263166)
\curveto(35.0010207,449.11263067)(36.03602134,449.30763175)(36.6810207,449.39763166)
\curveto(38.31601906,449.60763145)(38.64602091,449.72763183)(38.8560207,449.89263166)
\lineto(38.8560207,448.60263166)
}
}
{
\newrgbcolor{curcolor}{0 0 0}
\pscustom[linestyle=none,fillstyle=solid,fillcolor=curcolor]
{
\newpath
\moveto(48.93063007,456.65763166)
\lineto(47.61063007,456.65763166)
\lineto(47.61063007,452.72763166)
\lineto(47.58063007,452.62263166)
\curveto(47.26563039,453.07263121)(46.66562865,453.95763166)(45.24063007,453.95763166)
\curveto(43.15563216,453.95763166)(41.97063007,452.24762946)(41.97063007,450.04263166)
\curveto(41.97063007,448.16763354)(42.75063274,445.66263166)(45.42063007,445.66263166)
\curveto(46.18562931,445.66263166)(47.08563064,445.90263273)(47.65563007,446.96763166)
\lineto(47.68563007,446.96763166)
\lineto(47.68563007,445.88763166)
\lineto(48.93063007,445.88763166)
\lineto(48.93063007,456.65763166)
\moveto(43.33563007,449.83263166)
\curveto(43.33563007,450.83763066)(43.44063211,452.77263166)(45.48063007,452.77263166)
\curveto(47.38562817,452.77263166)(47.59563007,450.71763039)(47.59563007,449.44263166)
\curveto(47.59563007,447.35763375)(46.29062923,446.80263166)(45.45063007,446.80263166)
\curveto(44.01063151,446.80263166)(43.33563007,448.10763339)(43.33563007,449.83263166)
}
}
{
\newrgbcolor{curcolor}{0 0 0}
\pscustom[linestyle=none,fillstyle=solid,fillcolor=curcolor]
{
\newpath
\moveto(50.32023945,449.81763166)
\curveto(50.32023945,447.79263369)(51.46024195,445.67763166)(53.96523945,445.67763166)
\curveto(56.47023694,445.67763166)(57.61023945,447.79263369)(57.61023945,449.81763166)
\curveto(57.61023945,451.84262964)(56.47023694,453.95763166)(53.96523945,453.95763166)
\curveto(51.46024195,453.95763166)(50.32023945,451.84262964)(50.32023945,449.81763166)
\moveto(51.68523945,449.81763166)
\curveto(51.68523945,450.86763061)(52.07524134,452.81763166)(53.96523945,452.81763166)
\curveto(55.85523756,452.81763166)(56.24523945,450.86763061)(56.24523945,449.81763166)
\curveto(56.24523945,448.76763271)(55.85523756,446.81763166)(53.96523945,446.81763166)
\curveto(52.07524134,446.81763166)(51.68523945,448.76763271)(51.68523945,449.81763166)
}
}
{
\newrgbcolor{curcolor}{0 0 0}
\pscustom[linestyle=none,fillstyle=solid,fillcolor=curcolor]
{
\newpath
\moveto(8.8457082,439.66739729)
\lineto(7.3907082,439.66739729)
\lineto(7.3907082,428.89739729)
\lineto(8.8457082,428.89739729)
\lineto(8.8457082,439.66739729)
}
}
{
\newrgbcolor{curcolor}{0 0 0}
\pscustom[linestyle=none,fillstyle=solid,fillcolor=curcolor]
{
\newpath
\moveto(17.55086445,434.23739729)
\curveto(17.55086445,436.47239505)(16.02086323,436.96739729)(14.80586445,436.96739729)
\curveto(13.4558658,436.96739729)(12.72086416,436.05239687)(12.43586445,435.63239729)
\lineto(12.40586445,435.63239729)
\lineto(12.40586445,436.74239729)
\lineto(11.16086445,436.74239729)
\lineto(11.16086445,428.89739729)
\lineto(12.48086445,428.89739729)
\lineto(12.48086445,433.17239729)
\curveto(12.48086445,435.30239516)(13.8008652,435.78239729)(14.55086445,435.78239729)
\curveto(15.84086316,435.78239729)(16.23086445,435.09239592)(16.23086445,433.72739729)
\lineto(16.23086445,428.89739729)
\lineto(17.55086445,428.89739729)
\lineto(17.55086445,434.23739729)
}
}
{
\newrgbcolor{curcolor}{0 0 0}
\pscustom[linestyle=none,fillstyle=solid,fillcolor=curcolor]
{
\newpath
\moveto(21.96250507,430.35239729)
\lineto(21.93250507,430.35239729)
\lineto(19.89250507,436.74239729)
\lineto(18.36250507,436.74239729)
\lineto(21.22750507,428.89739729)
\lineto(22.63750507,428.89739729)
\lineto(25.62250507,436.74239729)
\lineto(24.18250507,436.74239729)
\lineto(21.96250507,430.35239729)
}
}
{
\newrgbcolor{curcolor}{0 0 0}
\pscustom[linestyle=none,fillstyle=solid,fillcolor=curcolor]
{
\newpath
\moveto(28.06750507,436.74239729)
\lineto(26.74750507,436.74239729)
\lineto(26.74750507,428.89739729)
\lineto(28.06750507,428.89739729)
\lineto(28.06750507,436.74239729)
\moveto(28.06750507,438.16739729)
\lineto(28.06750507,439.66739729)
\lineto(26.74750507,439.66739729)
\lineto(26.74750507,438.16739729)
\lineto(28.06750507,438.16739729)
}
}
{
\newrgbcolor{curcolor}{0 0 0}
\pscustom[linestyle=none,fillstyle=solid,fillcolor=curcolor]
{
\newpath
\moveto(32.93734882,435.64739729)
\lineto(32.93734882,436.74239729)
\lineto(31.67734882,436.74239729)
\lineto(31.67734882,438.93239729)
\lineto(30.35734882,438.93239729)
\lineto(30.35734882,436.74239729)
\lineto(29.29234882,436.74239729)
\lineto(29.29234882,435.64739729)
\lineto(30.35734882,435.64739729)
\lineto(30.35734882,430.47239729)
\curveto(30.35734882,429.52739823)(30.64235013,428.79239729)(31.94734882,428.79239729)
\curveto(32.08234869,428.79239729)(32.4573493,428.85239733)(32.93734882,428.89739729)
\lineto(32.93734882,429.93239729)
\lineto(32.47234882,429.93239729)
\curveto(32.20234909,429.93239729)(31.67734882,429.9323979)(31.67734882,430.54739729)
\lineto(31.67734882,435.64739729)
\lineto(32.93734882,435.64739729)
}
}
{
\newrgbcolor{curcolor}{0 0 0}
\pscustom[linestyle=none,fillstyle=solid,fillcolor=curcolor]
{
\newpath
\moveto(35.2410207,434.35739729)
\curveto(35.33102061,434.95739669)(35.5410222,435.87239729)(37.0410207,435.87239729)
\curveto(38.28601945,435.87239729)(38.8860207,435.42239646)(38.8860207,434.59739729)
\curveto(38.8860207,433.81739807)(38.51102038,433.69739726)(38.1960207,433.66739729)
\lineto(36.0210207,433.39739729)
\curveto(33.83102289,433.12739756)(33.6360207,431.59739663)(33.6360207,430.93739729)
\curveto(33.6360207,429.58739864)(34.65602214,428.67239729)(36.0960207,428.67239729)
\curveto(37.62601917,428.67239729)(38.42102121,429.39239784)(38.9310207,429.94739729)
\curveto(38.97602065,429.34739789)(39.15602187,428.74739729)(40.3260207,428.74739729)
\curveto(40.6260204,428.74739729)(40.82102092,428.83739735)(41.0460207,428.89739729)
\lineto(41.0460207,429.85739729)
\curveto(40.89602085,429.82739732)(40.73102058,429.79739729)(40.6110207,429.79739729)
\curveto(40.34102097,429.79739729)(40.1760207,429.93239762)(40.1760207,430.26239729)
\lineto(40.1760207,434.77739729)
\curveto(40.1760207,436.78739528)(37.89602007,436.96739729)(37.2660207,436.96739729)
\curveto(35.33102263,436.96739729)(34.08602064,436.23239541)(34.0260207,434.35739729)
\lineto(35.2410207,434.35739729)
\moveto(38.8560207,431.61239729)
\curveto(38.8560207,430.56239834)(37.65601947,429.76739729)(36.4260207,429.76739729)
\curveto(35.43602169,429.76739729)(35.0010207,430.27739814)(35.0010207,431.13239729)
\curveto(35.0010207,432.1223963)(36.03602134,432.31739738)(36.6810207,432.40739729)
\curveto(38.31601906,432.61739708)(38.64602091,432.73739745)(38.8560207,432.90239729)
\lineto(38.8560207,431.61239729)
}
}
{
\newrgbcolor{curcolor}{0 0 0}
\pscustom[linestyle=none,fillstyle=solid,fillcolor=curcolor]
{
\newpath
\moveto(48.93063007,439.66739729)
\lineto(47.61063007,439.66739729)
\lineto(47.61063007,435.73739729)
\lineto(47.58063007,435.63239729)
\curveto(47.26563039,436.08239684)(46.66562865,436.96739729)(45.24063007,436.96739729)
\curveto(43.15563216,436.96739729)(41.97063007,435.25739508)(41.97063007,433.05239729)
\curveto(41.97063007,431.17739916)(42.75063274,428.67239729)(45.42063007,428.67239729)
\curveto(46.18562931,428.67239729)(47.08563064,428.91239835)(47.65563007,429.97739729)
\lineto(47.68563007,429.97739729)
\lineto(47.68563007,428.89739729)
\lineto(48.93063007,428.89739729)
\lineto(48.93063007,439.66739729)
\moveto(43.33563007,432.84239729)
\curveto(43.33563007,433.84739628)(43.44063211,435.78239729)(45.48063007,435.78239729)
\curveto(47.38562817,435.78239729)(47.59563007,433.72739601)(47.59563007,432.45239729)
\curveto(47.59563007,430.36739937)(46.29062923,429.81239729)(45.45063007,429.81239729)
\curveto(44.01063151,429.81239729)(43.33563007,431.11739901)(43.33563007,432.84239729)
}
}
{
\newrgbcolor{curcolor}{0 0 0}
\pscustom[linestyle=none,fillstyle=solid,fillcolor=curcolor]
{
\newpath
\moveto(50.32023945,432.82739729)
\curveto(50.32023945,430.80239931)(51.46024195,428.68739729)(53.96523945,428.68739729)
\curveto(56.47023694,428.68739729)(57.61023945,430.80239931)(57.61023945,432.82739729)
\curveto(57.61023945,434.85239526)(56.47023694,436.96739729)(53.96523945,436.96739729)
\curveto(51.46024195,436.96739729)(50.32023945,434.85239526)(50.32023945,432.82739729)
\moveto(51.68523945,432.82739729)
\curveto(51.68523945,433.87739624)(52.07524134,435.82739729)(53.96523945,435.82739729)
\curveto(55.85523756,435.82739729)(56.24523945,433.87739624)(56.24523945,432.82739729)
\curveto(56.24523945,431.77739834)(55.85523756,429.82739729)(53.96523945,429.82739729)
\curveto(52.07524134,429.82739729)(51.68523945,431.77739834)(51.68523945,432.82739729)
}
}
{
\newrgbcolor{curcolor}{0 0 0}
\pscustom[linestyle=none,fillstyle=solid,fillcolor=curcolor]
{
\newpath
\moveto(8.8457082,337.72611311)
\lineto(7.3907082,337.72611311)
\lineto(7.3907082,326.95611311)
\lineto(8.8457082,326.95611311)
\lineto(8.8457082,337.72611311)
}
}
{
\newrgbcolor{curcolor}{0 0 0}
\pscustom[linestyle=none,fillstyle=solid,fillcolor=curcolor]
{
\newpath
\moveto(17.55086445,332.29611311)
\curveto(17.55086445,334.53111087)(16.02086323,335.02611311)(14.80586445,335.02611311)
\curveto(13.4558658,335.02611311)(12.72086416,334.11111269)(12.43586445,333.69111311)
\lineto(12.40586445,333.69111311)
\lineto(12.40586445,334.80111311)
\lineto(11.16086445,334.80111311)
\lineto(11.16086445,326.95611311)
\lineto(12.48086445,326.95611311)
\lineto(12.48086445,331.23111311)
\curveto(12.48086445,333.36111098)(13.8008652,333.84111311)(14.55086445,333.84111311)
\curveto(15.84086316,333.84111311)(16.23086445,333.15111174)(16.23086445,331.78611311)
\lineto(16.23086445,326.95611311)
\lineto(17.55086445,326.95611311)
\lineto(17.55086445,332.29611311)
}
}
{
\newrgbcolor{curcolor}{0 0 0}
\pscustom[linestyle=none,fillstyle=solid,fillcolor=curcolor]
{
\newpath
\moveto(21.96250507,328.41111311)
\lineto(21.93250507,328.41111311)
\lineto(19.89250507,334.80111311)
\lineto(18.36250507,334.80111311)
\lineto(21.22750507,326.95611311)
\lineto(22.63750507,326.95611311)
\lineto(25.62250507,334.80111311)
\lineto(24.18250507,334.80111311)
\lineto(21.96250507,328.41111311)
}
}
{
\newrgbcolor{curcolor}{0 0 0}
\pscustom[linestyle=none,fillstyle=solid,fillcolor=curcolor]
{
\newpath
\moveto(28.06750507,334.80111311)
\lineto(26.74750507,334.80111311)
\lineto(26.74750507,326.95611311)
\lineto(28.06750507,326.95611311)
\lineto(28.06750507,334.80111311)
\moveto(28.06750507,336.22611311)
\lineto(28.06750507,337.72611311)
\lineto(26.74750507,337.72611311)
\lineto(26.74750507,336.22611311)
\lineto(28.06750507,336.22611311)
}
}
{
\newrgbcolor{curcolor}{0 0 0}
\pscustom[linestyle=none,fillstyle=solid,fillcolor=curcolor]
{
\newpath
\moveto(32.93734882,333.70611311)
\lineto(32.93734882,334.80111311)
\lineto(31.67734882,334.80111311)
\lineto(31.67734882,336.99111311)
\lineto(30.35734882,336.99111311)
\lineto(30.35734882,334.80111311)
\lineto(29.29234882,334.80111311)
\lineto(29.29234882,333.70611311)
\lineto(30.35734882,333.70611311)
\lineto(30.35734882,328.53111311)
\curveto(30.35734882,327.58611405)(30.64235013,326.85111311)(31.94734882,326.85111311)
\curveto(32.08234869,326.85111311)(32.4573493,326.91111315)(32.93734882,326.95611311)
\lineto(32.93734882,327.99111311)
\lineto(32.47234882,327.99111311)
\curveto(32.20234909,327.99111311)(31.67734882,327.99111372)(31.67734882,328.60611311)
\lineto(31.67734882,333.70611311)
\lineto(32.93734882,333.70611311)
}
}
{
\newrgbcolor{curcolor}{0 0 0}
\pscustom[linestyle=none,fillstyle=solid,fillcolor=curcolor]
{
\newpath
\moveto(35.2410207,332.41611311)
\curveto(35.33102061,333.01611251)(35.5410222,333.93111311)(37.0410207,333.93111311)
\curveto(38.28601945,333.93111311)(38.8860207,333.48111228)(38.8860207,332.65611311)
\curveto(38.8860207,331.87611389)(38.51102038,331.75611308)(38.1960207,331.72611311)
\lineto(36.0210207,331.45611311)
\curveto(33.83102289,331.18611338)(33.6360207,329.65611245)(33.6360207,328.99611311)
\curveto(33.6360207,327.64611446)(34.65602214,326.73111311)(36.0960207,326.73111311)
\curveto(37.62601917,326.73111311)(38.42102121,327.45111366)(38.9310207,328.00611311)
\curveto(38.97602065,327.40611371)(39.15602187,326.80611311)(40.3260207,326.80611311)
\curveto(40.6260204,326.80611311)(40.82102092,326.89611317)(41.0460207,326.95611311)
\lineto(41.0460207,327.91611311)
\curveto(40.89602085,327.88611314)(40.73102058,327.85611311)(40.6110207,327.85611311)
\curveto(40.34102097,327.85611311)(40.1760207,327.99111344)(40.1760207,328.32111311)
\lineto(40.1760207,332.83611311)
\curveto(40.1760207,334.8461111)(37.89602007,335.02611311)(37.2660207,335.02611311)
\curveto(35.33102263,335.02611311)(34.08602064,334.29111123)(34.0260207,332.41611311)
\lineto(35.2410207,332.41611311)
\moveto(38.8560207,329.67111311)
\curveto(38.8560207,328.62111416)(37.65601947,327.82611311)(36.4260207,327.82611311)
\curveto(35.43602169,327.82611311)(35.0010207,328.33611396)(35.0010207,329.19111311)
\curveto(35.0010207,330.18111212)(36.03602134,330.3761132)(36.6810207,330.46611311)
\curveto(38.31601906,330.6761129)(38.64602091,330.79611327)(38.8560207,330.96111311)
\lineto(38.8560207,329.67111311)
}
}
{
\newrgbcolor{curcolor}{0 0 0}
\pscustom[linestyle=none,fillstyle=solid,fillcolor=curcolor]
{
\newpath
\moveto(48.93063007,337.72611311)
\lineto(47.61063007,337.72611311)
\lineto(47.61063007,333.79611311)
\lineto(47.58063007,333.69111311)
\curveto(47.26563039,334.14111266)(46.66562865,335.02611311)(45.24063007,335.02611311)
\curveto(43.15563216,335.02611311)(41.97063007,333.3161109)(41.97063007,331.11111311)
\curveto(41.97063007,329.23611498)(42.75063274,326.73111311)(45.42063007,326.73111311)
\curveto(46.18562931,326.73111311)(47.08563064,326.97111417)(47.65563007,328.03611311)
\lineto(47.68563007,328.03611311)
\lineto(47.68563007,326.95611311)
\lineto(48.93063007,326.95611311)
\lineto(48.93063007,337.72611311)
\moveto(43.33563007,330.90111311)
\curveto(43.33563007,331.9061121)(43.44063211,333.84111311)(45.48063007,333.84111311)
\curveto(47.38562817,333.84111311)(47.59563007,331.78611183)(47.59563007,330.51111311)
\curveto(47.59563007,328.42611519)(46.29062923,327.87111311)(45.45063007,327.87111311)
\curveto(44.01063151,327.87111311)(43.33563007,329.17611483)(43.33563007,330.90111311)
}
}
{
\newrgbcolor{curcolor}{0 0 0}
\pscustom[linestyle=none,fillstyle=solid,fillcolor=curcolor]
{
\newpath
\moveto(50.32023945,330.88611311)
\curveto(50.32023945,328.86111513)(51.46024195,326.74611311)(53.96523945,326.74611311)
\curveto(56.47023694,326.74611311)(57.61023945,328.86111513)(57.61023945,330.88611311)
\curveto(57.61023945,332.91111108)(56.47023694,335.02611311)(53.96523945,335.02611311)
\curveto(51.46024195,335.02611311)(50.32023945,332.91111108)(50.32023945,330.88611311)
\moveto(51.68523945,330.88611311)
\curveto(51.68523945,331.93611206)(52.07524134,333.88611311)(53.96523945,333.88611311)
\curveto(55.85523756,333.88611311)(56.24523945,331.93611206)(56.24523945,330.88611311)
\curveto(56.24523945,329.83611416)(55.85523756,327.88611311)(53.96523945,327.88611311)
\curveto(52.07524134,327.88611311)(51.68523945,329.83611416)(51.68523945,330.88611311)
}
}
{
\newrgbcolor{curcolor}{0 0 0}
\pscustom[linestyle=none,fillstyle=solid,fillcolor=curcolor]
{
\newpath
\moveto(8.8457082,320.73587873)
\lineto(7.3907082,320.73587873)
\lineto(7.3907082,309.96587873)
\lineto(8.8457082,309.96587873)
\lineto(8.8457082,320.73587873)
}
}
{
\newrgbcolor{curcolor}{0 0 0}
\pscustom[linestyle=none,fillstyle=solid,fillcolor=curcolor]
{
\newpath
\moveto(17.55086445,315.30587873)
\curveto(17.55086445,317.5408765)(16.02086323,318.03587873)(14.80586445,318.03587873)
\curveto(13.4558658,318.03587873)(12.72086416,317.12087831)(12.43586445,316.70087873)
\lineto(12.40586445,316.70087873)
\lineto(12.40586445,317.81087873)
\lineto(11.16086445,317.81087873)
\lineto(11.16086445,309.96587873)
\lineto(12.48086445,309.96587873)
\lineto(12.48086445,314.24087873)
\curveto(12.48086445,316.3708766)(13.8008652,316.85087873)(14.55086445,316.85087873)
\curveto(15.84086316,316.85087873)(16.23086445,316.16087737)(16.23086445,314.79587873)
\lineto(16.23086445,309.96587873)
\lineto(17.55086445,309.96587873)
\lineto(17.55086445,315.30587873)
}
}
{
\newrgbcolor{curcolor}{0 0 0}
\pscustom[linestyle=none,fillstyle=solid,fillcolor=curcolor]
{
\newpath
\moveto(21.96250507,311.42087873)
\lineto(21.93250507,311.42087873)
\lineto(19.89250507,317.81087873)
\lineto(18.36250507,317.81087873)
\lineto(21.22750507,309.96587873)
\lineto(22.63750507,309.96587873)
\lineto(25.62250507,317.81087873)
\lineto(24.18250507,317.81087873)
\lineto(21.96250507,311.42087873)
}
}
{
\newrgbcolor{curcolor}{0 0 0}
\pscustom[linestyle=none,fillstyle=solid,fillcolor=curcolor]
{
\newpath
\moveto(28.06750507,317.81087873)
\lineto(26.74750507,317.81087873)
\lineto(26.74750507,309.96587873)
\lineto(28.06750507,309.96587873)
\lineto(28.06750507,317.81087873)
\moveto(28.06750507,319.23587873)
\lineto(28.06750507,320.73587873)
\lineto(26.74750507,320.73587873)
\lineto(26.74750507,319.23587873)
\lineto(28.06750507,319.23587873)
}
}
{
\newrgbcolor{curcolor}{0 0 0}
\pscustom[linestyle=none,fillstyle=solid,fillcolor=curcolor]
{
\newpath
\moveto(32.93734882,316.71587873)
\lineto(32.93734882,317.81087873)
\lineto(31.67734882,317.81087873)
\lineto(31.67734882,320.00087873)
\lineto(30.35734882,320.00087873)
\lineto(30.35734882,317.81087873)
\lineto(29.29234882,317.81087873)
\lineto(29.29234882,316.71587873)
\lineto(30.35734882,316.71587873)
\lineto(30.35734882,311.54087873)
\curveto(30.35734882,310.59587968)(30.64235013,309.86087873)(31.94734882,309.86087873)
\curveto(32.08234869,309.86087873)(32.4573493,309.92087878)(32.93734882,309.96587873)
\lineto(32.93734882,311.00087873)
\lineto(32.47234882,311.00087873)
\curveto(32.20234909,311.00087873)(31.67734882,311.00087935)(31.67734882,311.61587873)
\lineto(31.67734882,316.71587873)
\lineto(32.93734882,316.71587873)
}
}
{
\newrgbcolor{curcolor}{0 0 0}
\pscustom[linestyle=none,fillstyle=solid,fillcolor=curcolor]
{
\newpath
\moveto(35.2410207,315.42587873)
\curveto(35.33102061,316.02587813)(35.5410222,316.94087873)(37.0410207,316.94087873)
\curveto(38.28601945,316.94087873)(38.8860207,316.49087791)(38.8860207,315.66587873)
\curveto(38.8860207,314.88587951)(38.51102038,314.7658787)(38.1960207,314.73587873)
\lineto(36.0210207,314.46587873)
\curveto(33.83102289,314.195879)(33.6360207,312.66587807)(33.6360207,312.00587873)
\curveto(33.6360207,310.65588008)(34.65602214,309.74087873)(36.0960207,309.74087873)
\curveto(37.62601917,309.74087873)(38.42102121,310.46087929)(38.9310207,311.01587873)
\curveto(38.97602065,310.41587933)(39.15602187,309.81587873)(40.3260207,309.81587873)
\curveto(40.6260204,309.81587873)(40.82102092,309.90587879)(41.0460207,309.96587873)
\lineto(41.0460207,310.92587873)
\curveto(40.89602085,310.89587876)(40.73102058,310.86587873)(40.6110207,310.86587873)
\curveto(40.34102097,310.86587873)(40.1760207,311.00087906)(40.1760207,311.33087873)
\lineto(40.1760207,315.84587873)
\curveto(40.1760207,317.85587672)(37.89602007,318.03587873)(37.2660207,318.03587873)
\curveto(35.33102263,318.03587873)(34.08602064,317.30087686)(34.0260207,315.42587873)
\lineto(35.2410207,315.42587873)
\moveto(38.8560207,312.68087873)
\curveto(38.8560207,311.63087978)(37.65601947,310.83587873)(36.4260207,310.83587873)
\curveto(35.43602169,310.83587873)(35.0010207,311.34587959)(35.0010207,312.20087873)
\curveto(35.0010207,313.19087774)(36.03602134,313.38587882)(36.6810207,313.47587873)
\curveto(38.31601906,313.68587852)(38.64602091,313.8058789)(38.8560207,313.97087873)
\lineto(38.8560207,312.68087873)
}
}
{
\newrgbcolor{curcolor}{0 0 0}
\pscustom[linestyle=none,fillstyle=solid,fillcolor=curcolor]
{
\newpath
\moveto(48.93063007,320.73587873)
\lineto(47.61063007,320.73587873)
\lineto(47.61063007,316.80587873)
\lineto(47.58063007,316.70087873)
\curveto(47.26563039,317.15087828)(46.66562865,318.03587873)(45.24063007,318.03587873)
\curveto(43.15563216,318.03587873)(41.97063007,316.32587653)(41.97063007,314.12087873)
\curveto(41.97063007,312.24588061)(42.75063274,309.74087873)(45.42063007,309.74087873)
\curveto(46.18562931,309.74087873)(47.08563064,309.9808798)(47.65563007,311.04587873)
\lineto(47.68563007,311.04587873)
\lineto(47.68563007,309.96587873)
\lineto(48.93063007,309.96587873)
\lineto(48.93063007,320.73587873)
\moveto(43.33563007,313.91087873)
\curveto(43.33563007,314.91587773)(43.44063211,316.85087873)(45.48063007,316.85087873)
\curveto(47.38562817,316.85087873)(47.59563007,314.79587746)(47.59563007,313.52087873)
\curveto(47.59563007,311.43588082)(46.29062923,310.88087873)(45.45063007,310.88087873)
\curveto(44.01063151,310.88087873)(43.33563007,312.18588046)(43.33563007,313.91087873)
}
}
{
\newrgbcolor{curcolor}{0 0 0}
\pscustom[linestyle=none,fillstyle=solid,fillcolor=curcolor]
{
\newpath
\moveto(50.32023945,313.89587873)
\curveto(50.32023945,311.87088076)(51.46024195,309.75587873)(53.96523945,309.75587873)
\curveto(56.47023694,309.75587873)(57.61023945,311.87088076)(57.61023945,313.89587873)
\curveto(57.61023945,315.92087671)(56.47023694,318.03587873)(53.96523945,318.03587873)
\curveto(51.46024195,318.03587873)(50.32023945,315.92087671)(50.32023945,313.89587873)
\moveto(51.68523945,313.89587873)
\curveto(51.68523945,314.94587768)(52.07524134,316.89587873)(53.96523945,316.89587873)
\curveto(55.85523756,316.89587873)(56.24523945,314.94587768)(56.24523945,313.89587873)
\curveto(56.24523945,312.84587978)(55.85523756,310.89587873)(53.96523945,310.89587873)
\curveto(52.07524134,310.89587873)(51.68523945,312.84587978)(51.68523945,313.89587873)
}
}
{
\newrgbcolor{curcolor}{0 0 0}
\pscustom[linestyle=none,fillstyle=solid,fillcolor=curcolor]
{
\newpath
\moveto(15.2657082,413.19728498)
\lineto(8.7707082,413.19728498)
\lineto(8.7707082,416.79728498)
\lineto(14.6657082,416.79728498)
\lineto(14.6657082,418.08728498)
\lineto(8.7707082,418.08728498)
\lineto(8.7707082,421.38728498)
\lineto(15.1607082,421.38728498)
\lineto(15.1607082,422.67728498)
\lineto(7.3157082,422.67728498)
\lineto(7.3157082,411.90728498)
\lineto(15.2657082,411.90728498)
\lineto(15.2657082,413.19728498)
}
}
{
\newrgbcolor{curcolor}{0 0 0}
\pscustom[linestyle=none,fillstyle=solid,fillcolor=curcolor]
{
\newpath
\moveto(22.36938007,417.51728498)
\curveto(22.36938007,417.90728459)(22.17437727,419.97728498)(19.36938007,419.97728498)
\curveto(17.82438162,419.97728498)(16.39938007,419.19728326)(16.39938007,417.47228498)
\curveto(16.39938007,416.39228606)(17.11938117,415.83728471)(18.21438007,415.56728498)
\lineto(19.74438007,415.19228498)
\curveto(20.86937895,414.90728527)(21.30438007,414.69728435)(21.30438007,414.06728498)
\curveto(21.30438007,413.19728585)(20.44937913,412.82228498)(19.50438007,412.82228498)
\curveto(17.64438193,412.82228498)(17.46438003,413.8122856)(17.41938007,414.42728498)
\lineto(16.14438007,414.42728498)
\curveto(16.18938003,413.48228593)(16.41438318,411.68228498)(19.51938007,411.68228498)
\curveto(21.2893783,411.68228498)(22.62438007,412.6572866)(22.62438007,414.27728498)
\curveto(22.62438007,415.34228392)(22.05437844,415.94228539)(20.41938007,416.34728498)
\lineto(19.09938007,416.67728498)
\curveto(18.07938109,416.93228473)(17.67438007,417.08228563)(17.67438007,417.72728498)
\curveto(17.67438007,418.70228401)(18.82938048,418.83728498)(19.23438007,418.83728498)
\curveto(20.89937841,418.83728498)(21.07938009,418.01228449)(21.09438007,417.51728498)
\lineto(22.36938007,417.51728498)
}
}
{
\newrgbcolor{curcolor}{0 0 0}
\pscustom[linestyle=none,fillstyle=solid,fillcolor=curcolor]
{
\newpath
\moveto(27.01938007,418.65728498)
\lineto(27.01938007,419.75228498)
\lineto(25.75938007,419.75228498)
\lineto(25.75938007,421.94228498)
\lineto(24.43938007,421.94228498)
\lineto(24.43938007,419.75228498)
\lineto(23.37438007,419.75228498)
\lineto(23.37438007,418.65728498)
\lineto(24.43938007,418.65728498)
\lineto(24.43938007,413.48228498)
\curveto(24.43938007,412.53728593)(24.72438138,411.80228498)(26.02938007,411.80228498)
\curveto(26.16437994,411.80228498)(26.53938055,411.86228503)(27.01938007,411.90728498)
\lineto(27.01938007,412.94228498)
\lineto(26.55438007,412.94228498)
\curveto(26.28438034,412.94228498)(25.75938007,412.9422856)(25.75938007,413.55728498)
\lineto(25.75938007,418.65728498)
\lineto(27.01938007,418.65728498)
}
}
{
\newrgbcolor{curcolor}{0 0 0}
\pscustom[linestyle=none,fillstyle=solid,fillcolor=curcolor]
{
\newpath
\moveto(34.65953632,411.90728498)
\lineto(34.65953632,419.75228498)
\lineto(33.33953632,419.75228498)
\lineto(33.33953632,415.43228498)
\curveto(33.33953632,414.29228612)(32.84453466,412.82228498)(31.17953632,412.82228498)
\curveto(30.32453718,412.82228498)(29.66453632,413.25728627)(29.66453632,414.54728498)
\lineto(29.66453632,419.75228498)
\lineto(28.34453632,419.75228498)
\lineto(28.34453632,414.11228498)
\curveto(28.34453632,412.23728686)(29.73953748,411.68228498)(30.89453632,411.68228498)
\curveto(32.15453506,411.68228498)(32.82953688,412.1622859)(33.38453632,413.07728498)
\lineto(33.41453632,413.04728498)
\lineto(33.41453632,411.90728498)
\lineto(34.65953632,411.90728498)
}
}
{
\newrgbcolor{curcolor}{0 0 0}
\pscustom[linestyle=none,fillstyle=solid,fillcolor=curcolor]
{
\newpath
\moveto(43.1591457,422.67728498)
\lineto(41.8391457,422.67728498)
\lineto(41.8391457,418.74728498)
\lineto(41.8091457,418.64228498)
\curveto(41.49414601,419.09228453)(40.89414427,419.97728498)(39.4691457,419.97728498)
\curveto(37.38414778,419.97728498)(36.1991457,418.26728278)(36.1991457,416.06228498)
\curveto(36.1991457,414.18728686)(36.97914837,411.68228498)(39.6491457,411.68228498)
\curveto(40.41414493,411.68228498)(41.31414627,411.92228605)(41.8841457,412.98728498)
\lineto(41.9141457,412.98728498)
\lineto(41.9141457,411.90728498)
\lineto(43.1591457,411.90728498)
\lineto(43.1591457,422.67728498)
\moveto(37.5641457,415.85228498)
\curveto(37.5641457,416.85728398)(37.66914774,418.79228498)(39.7091457,418.79228498)
\curveto(41.61414379,418.79228498)(41.8241457,416.73728371)(41.8241457,415.46228498)
\curveto(41.8241457,413.37728707)(40.51914486,412.82228498)(39.6791457,412.82228498)
\curveto(38.23914714,412.82228498)(37.5641457,414.12728671)(37.5641457,415.85228498)
}
}
{
\newrgbcolor{curcolor}{0 0 0}
\pscustom[linestyle=none,fillstyle=solid,fillcolor=curcolor]
{
\newpath
\moveto(46.34875507,419.75228498)
\lineto(45.02875507,419.75228498)
\lineto(45.02875507,411.90728498)
\lineto(46.34875507,411.90728498)
\lineto(46.34875507,419.75228498)
\moveto(46.34875507,421.17728498)
\lineto(46.34875507,422.67728498)
\lineto(45.02875507,422.67728498)
\lineto(45.02875507,421.17728498)
\lineto(46.34875507,421.17728498)
}
}
{
\newrgbcolor{curcolor}{0 0 0}
\pscustom[linestyle=none,fillstyle=solid,fillcolor=curcolor]
{
\newpath
\moveto(49.50859882,417.36728498)
\curveto(49.59859873,417.96728438)(49.80860032,418.88228498)(51.30859882,418.88228498)
\curveto(52.55359758,418.88228498)(53.15359882,418.43228416)(53.15359882,417.60728498)
\curveto(53.15359882,416.82728576)(52.77859851,416.70728495)(52.46359882,416.67728498)
\lineto(50.28859882,416.40728498)
\curveto(48.09860101,416.13728525)(47.90359882,414.60728432)(47.90359882,413.94728498)
\curveto(47.90359882,412.59728633)(48.92360026,411.68228498)(50.36359882,411.68228498)
\curveto(51.89359729,411.68228498)(52.68859933,412.40228554)(53.19859882,412.95728498)
\curveto(53.24359878,412.35728558)(53.42359999,411.75728498)(54.59359882,411.75728498)
\curveto(54.89359852,411.75728498)(55.08859905,411.84728504)(55.31359882,411.90728498)
\lineto(55.31359882,412.86728498)
\curveto(55.16359897,412.83728501)(54.9985987,412.80728498)(54.87859882,412.80728498)
\curveto(54.60859909,412.80728498)(54.44359882,412.94228531)(54.44359882,413.27228498)
\lineto(54.44359882,417.78728498)
\curveto(54.44359882,419.79728297)(52.16359819,419.97728498)(51.53359882,419.97728498)
\curveto(49.59860076,419.97728498)(48.35359876,419.24228311)(48.29359882,417.36728498)
\lineto(49.50859882,417.36728498)
\moveto(53.12359882,414.62228498)
\curveto(53.12359882,413.57228603)(51.92359759,412.77728498)(50.69359882,412.77728498)
\curveto(49.70359981,412.77728498)(49.26859882,413.28728584)(49.26859882,414.14228498)
\curveto(49.26859882,415.13228399)(50.30359947,415.32728507)(50.94859882,415.41728498)
\curveto(52.58359719,415.62728477)(52.91359903,415.74728515)(53.12359882,415.91228498)
\lineto(53.12359882,414.62228498)
}
}
{
\newrgbcolor{curcolor}{0 0 0}
\pscustom[linestyle=none,fillstyle=solid,fillcolor=curcolor]
{
\newpath
\moveto(63.0782082,417.24728498)
\curveto(63.0782082,419.48228275)(61.54820698,419.97728498)(60.3332082,419.97728498)
\curveto(58.98320955,419.97728498)(58.24820791,419.06228456)(57.9632082,418.64228498)
\lineto(57.9332082,418.64228498)
\lineto(57.9332082,419.75228498)
\lineto(56.6882082,419.75228498)
\lineto(56.6882082,411.90728498)
\lineto(58.0082082,411.90728498)
\lineto(58.0082082,416.18228498)
\curveto(58.0082082,418.31228285)(59.32820895,418.79228498)(60.0782082,418.79228498)
\curveto(61.36820691,418.79228498)(61.7582082,418.10228362)(61.7582082,416.73728498)
\lineto(61.7582082,411.90728498)
\lineto(63.0782082,411.90728498)
\lineto(63.0782082,417.24728498)
}
}
{
\newrgbcolor{curcolor}{0 0 0}
\pscustom[linestyle=none,fillstyle=solid,fillcolor=curcolor]
{
\newpath
\moveto(67.91781757,418.65728498)
\lineto(67.91781757,419.75228498)
\lineto(66.65781757,419.75228498)
\lineto(66.65781757,421.94228498)
\lineto(65.33781757,421.94228498)
\lineto(65.33781757,419.75228498)
\lineto(64.27281757,419.75228498)
\lineto(64.27281757,418.65728498)
\lineto(65.33781757,418.65728498)
\lineto(65.33781757,413.48228498)
\curveto(65.33781757,412.53728593)(65.62281888,411.80228498)(66.92781757,411.80228498)
\curveto(67.06281744,411.80228498)(67.43781805,411.86228503)(67.91781757,411.90728498)
\lineto(67.91781757,412.94228498)
\lineto(67.45281757,412.94228498)
\curveto(67.18281784,412.94228498)(66.65781757,412.9422856)(66.65781757,413.55728498)
\lineto(66.65781757,418.65728498)
\lineto(67.91781757,418.65728498)
}
}
{
\newrgbcolor{curcolor}{0 0 0}
\pscustom[linestyle=none,fillstyle=solid,fillcolor=curcolor]
{
\newpath
\moveto(74.24500507,414.36728498)
\curveto(74.20000512,413.78228557)(73.46500383,412.82228498)(72.22000507,412.82228498)
\curveto(70.70500659,412.82228498)(69.94000507,413.76728662)(69.94000507,415.40228498)
\lineto(75.67000507,415.40228498)
\curveto(75.67000507,418.17728221)(74.56000281,419.97728498)(72.29500507,419.97728498)
\curveto(69.70000767,419.97728498)(68.53000507,418.04228255)(68.53000507,415.61228498)
\curveto(68.53000507,413.34728725)(69.83500728,411.68228498)(72.04000507,411.68228498)
\curveto(73.30000381,411.68228498)(73.81000543,411.98228522)(74.17000507,412.22228498)
\curveto(75.16000408,412.88228432)(75.52000512,413.99228536)(75.56500507,414.36728498)
\lineto(74.24500507,414.36728498)
\moveto(69.94000507,416.45228498)
\curveto(69.94000507,417.66728377)(70.90000629,418.79228498)(72.11500507,418.79228498)
\curveto(73.72000347,418.79228498)(74.23000515,417.66728377)(74.30500507,416.45228498)
\lineto(69.94000507,416.45228498)
}
}
{
\newrgbcolor{curcolor}{0 0 0}
\pscustom[linestyle=none,fillstyle=solid,fillcolor=curcolor]
{
\newpath
\moveto(15.2657082,277.27540998)
\lineto(8.7707082,277.27540998)
\lineto(8.7707082,280.87540998)
\lineto(14.6657082,280.87540998)
\lineto(14.6657082,282.16540998)
\lineto(8.7707082,282.16540998)
\lineto(8.7707082,285.46540998)
\lineto(15.1607082,285.46540998)
\lineto(15.1607082,286.75540998)
\lineto(7.3157082,286.75540998)
\lineto(7.3157082,275.98540998)
\lineto(15.2657082,275.98540998)
\lineto(15.2657082,277.27540998)
}
}
{
\newrgbcolor{curcolor}{0 0 0}
\pscustom[linestyle=none,fillstyle=solid,fillcolor=curcolor]
{
\newpath
\moveto(22.36938007,281.59540998)
\curveto(22.36938007,281.98540959)(22.17437727,284.05540998)(19.36938007,284.05540998)
\curveto(17.82438162,284.05540998)(16.39938007,283.27540826)(16.39938007,281.55040998)
\curveto(16.39938007,280.47041106)(17.11938117,279.91540971)(18.21438007,279.64540998)
\lineto(19.74438007,279.27040998)
\curveto(20.86937895,278.98541027)(21.30438007,278.77540935)(21.30438007,278.14540998)
\curveto(21.30438007,277.27541085)(20.44937913,276.90040998)(19.50438007,276.90040998)
\curveto(17.64438193,276.90040998)(17.46438003,277.8904106)(17.41938007,278.50540998)
\lineto(16.14438007,278.50540998)
\curveto(16.18938003,277.56041093)(16.41438318,275.76040998)(19.51938007,275.76040998)
\curveto(21.2893783,275.76040998)(22.62438007,276.7354116)(22.62438007,278.35540998)
\curveto(22.62438007,279.42040892)(22.05437844,280.02041039)(20.41938007,280.42540998)
\lineto(19.09938007,280.75540998)
\curveto(18.07938109,281.01040973)(17.67438007,281.16041063)(17.67438007,281.80540998)
\curveto(17.67438007,282.78040901)(18.82938048,282.91540998)(19.23438007,282.91540998)
\curveto(20.89937841,282.91540998)(21.07938009,282.09040949)(21.09438007,281.59540998)
\lineto(22.36938007,281.59540998)
}
}
{
\newrgbcolor{curcolor}{0 0 0}
\pscustom[linestyle=none,fillstyle=solid,fillcolor=curcolor]
{
\newpath
\moveto(27.01938007,282.73540998)
\lineto(27.01938007,283.83040998)
\lineto(25.75938007,283.83040998)
\lineto(25.75938007,286.02040998)
\lineto(24.43938007,286.02040998)
\lineto(24.43938007,283.83040998)
\lineto(23.37438007,283.83040998)
\lineto(23.37438007,282.73540998)
\lineto(24.43938007,282.73540998)
\lineto(24.43938007,277.56040998)
\curveto(24.43938007,276.61541093)(24.72438138,275.88040998)(26.02938007,275.88040998)
\curveto(26.16437994,275.88040998)(26.53938055,275.94041003)(27.01938007,275.98540998)
\lineto(27.01938007,277.02040998)
\lineto(26.55438007,277.02040998)
\curveto(26.28438034,277.02040998)(25.75938007,277.0204106)(25.75938007,277.63540998)
\lineto(25.75938007,282.73540998)
\lineto(27.01938007,282.73540998)
}
}
{
\newrgbcolor{curcolor}{0 0 0}
\pscustom[linestyle=none,fillstyle=solid,fillcolor=curcolor]
{
\newpath
\moveto(34.65953632,275.98540998)
\lineto(34.65953632,283.83040998)
\lineto(33.33953632,283.83040998)
\lineto(33.33953632,279.51040998)
\curveto(33.33953632,278.37041112)(32.84453466,276.90040998)(31.17953632,276.90040998)
\curveto(30.32453718,276.90040998)(29.66453632,277.33541127)(29.66453632,278.62540998)
\lineto(29.66453632,283.83040998)
\lineto(28.34453632,283.83040998)
\lineto(28.34453632,278.19040998)
\curveto(28.34453632,276.31541186)(29.73953748,275.76040998)(30.89453632,275.76040998)
\curveto(32.15453506,275.76040998)(32.82953688,276.2404109)(33.38453632,277.15540998)
\lineto(33.41453632,277.12540998)
\lineto(33.41453632,275.98540998)
\lineto(34.65953632,275.98540998)
}
}
{
\newrgbcolor{curcolor}{0 0 0}
\pscustom[linestyle=none,fillstyle=solid,fillcolor=curcolor]
{
\newpath
\moveto(43.1591457,286.75540998)
\lineto(41.8391457,286.75540998)
\lineto(41.8391457,282.82540998)
\lineto(41.8091457,282.72040998)
\curveto(41.49414601,283.17040953)(40.89414427,284.05540998)(39.4691457,284.05540998)
\curveto(37.38414778,284.05540998)(36.1991457,282.34540778)(36.1991457,280.14040998)
\curveto(36.1991457,278.26541186)(36.97914837,275.76040998)(39.6491457,275.76040998)
\curveto(40.41414493,275.76040998)(41.31414627,276.00041105)(41.8841457,277.06540998)
\lineto(41.9141457,277.06540998)
\lineto(41.9141457,275.98540998)
\lineto(43.1591457,275.98540998)
\lineto(43.1591457,286.75540998)
\moveto(37.5641457,279.93040998)
\curveto(37.5641457,280.93540898)(37.66914774,282.87040998)(39.7091457,282.87040998)
\curveto(41.61414379,282.87040998)(41.8241457,280.81540871)(41.8241457,279.54040998)
\curveto(41.8241457,277.45541207)(40.51914486,276.90040998)(39.6791457,276.90040998)
\curveto(38.23914714,276.90040998)(37.5641457,278.20541171)(37.5641457,279.93040998)
}
}
{
\newrgbcolor{curcolor}{0 0 0}
\pscustom[linestyle=none,fillstyle=solid,fillcolor=curcolor]
{
\newpath
\moveto(46.34875507,283.83040998)
\lineto(45.02875507,283.83040998)
\lineto(45.02875507,275.98540998)
\lineto(46.34875507,275.98540998)
\lineto(46.34875507,283.83040998)
\moveto(46.34875507,285.25540998)
\lineto(46.34875507,286.75540998)
\lineto(45.02875507,286.75540998)
\lineto(45.02875507,285.25540998)
\lineto(46.34875507,285.25540998)
}
}
{
\newrgbcolor{curcolor}{0 0 0}
\pscustom[linestyle=none,fillstyle=solid,fillcolor=curcolor]
{
\newpath
\moveto(49.50859882,281.44540998)
\curveto(49.59859873,282.04540938)(49.80860032,282.96040998)(51.30859882,282.96040998)
\curveto(52.55359758,282.96040998)(53.15359882,282.51040916)(53.15359882,281.68540998)
\curveto(53.15359882,280.90541076)(52.77859851,280.78540995)(52.46359882,280.75540998)
\lineto(50.28859882,280.48540998)
\curveto(48.09860101,280.21541025)(47.90359882,278.68540932)(47.90359882,278.02540998)
\curveto(47.90359882,276.67541133)(48.92360026,275.76040998)(50.36359882,275.76040998)
\curveto(51.89359729,275.76040998)(52.68859933,276.48041054)(53.19859882,277.03540998)
\curveto(53.24359878,276.43541058)(53.42359999,275.83540998)(54.59359882,275.83540998)
\curveto(54.89359852,275.83540998)(55.08859905,275.92541004)(55.31359882,275.98540998)
\lineto(55.31359882,276.94540998)
\curveto(55.16359897,276.91541001)(54.9985987,276.88540998)(54.87859882,276.88540998)
\curveto(54.60859909,276.88540998)(54.44359882,277.02041031)(54.44359882,277.35040998)
\lineto(54.44359882,281.86540998)
\curveto(54.44359882,283.87540797)(52.16359819,284.05540998)(51.53359882,284.05540998)
\curveto(49.59860076,284.05540998)(48.35359876,283.32040811)(48.29359882,281.44540998)
\lineto(49.50859882,281.44540998)
\moveto(53.12359882,278.70040998)
\curveto(53.12359882,277.65041103)(51.92359759,276.85540998)(50.69359882,276.85540998)
\curveto(49.70359981,276.85540998)(49.26859882,277.36541084)(49.26859882,278.22040998)
\curveto(49.26859882,279.21040899)(50.30359947,279.40541007)(50.94859882,279.49540998)
\curveto(52.58359719,279.70540977)(52.91359903,279.82541015)(53.12359882,279.99040998)
\lineto(53.12359882,278.70040998)
}
}
{
\newrgbcolor{curcolor}{0 0 0}
\pscustom[linestyle=none,fillstyle=solid,fillcolor=curcolor]
{
\newpath
\moveto(63.0782082,281.32540998)
\curveto(63.0782082,283.56040775)(61.54820698,284.05540998)(60.3332082,284.05540998)
\curveto(58.98320955,284.05540998)(58.24820791,283.14040956)(57.9632082,282.72040998)
\lineto(57.9332082,282.72040998)
\lineto(57.9332082,283.83040998)
\lineto(56.6882082,283.83040998)
\lineto(56.6882082,275.98540998)
\lineto(58.0082082,275.98540998)
\lineto(58.0082082,280.26040998)
\curveto(58.0082082,282.39040785)(59.32820895,282.87040998)(60.0782082,282.87040998)
\curveto(61.36820691,282.87040998)(61.7582082,282.18040862)(61.7582082,280.81540998)
\lineto(61.7582082,275.98540998)
\lineto(63.0782082,275.98540998)
\lineto(63.0782082,281.32540998)
}
}
{
\newrgbcolor{curcolor}{0 0 0}
\pscustom[linestyle=none,fillstyle=solid,fillcolor=curcolor]
{
\newpath
\moveto(67.91781757,282.73540998)
\lineto(67.91781757,283.83040998)
\lineto(66.65781757,283.83040998)
\lineto(66.65781757,286.02040998)
\lineto(65.33781757,286.02040998)
\lineto(65.33781757,283.83040998)
\lineto(64.27281757,283.83040998)
\lineto(64.27281757,282.73540998)
\lineto(65.33781757,282.73540998)
\lineto(65.33781757,277.56040998)
\curveto(65.33781757,276.61541093)(65.62281888,275.88040998)(66.92781757,275.88040998)
\curveto(67.06281744,275.88040998)(67.43781805,275.94041003)(67.91781757,275.98540998)
\lineto(67.91781757,277.02040998)
\lineto(67.45281757,277.02040998)
\curveto(67.18281784,277.02040998)(66.65781757,277.0204106)(66.65781757,277.63540998)
\lineto(66.65781757,282.73540998)
\lineto(67.91781757,282.73540998)
}
}
{
\newrgbcolor{curcolor}{0 0 0}
\pscustom[linestyle=none,fillstyle=solid,fillcolor=curcolor]
{
\newpath
\moveto(74.24500507,278.44540998)
\curveto(74.20000512,277.86041057)(73.46500383,276.90040998)(72.22000507,276.90040998)
\curveto(70.70500659,276.90040998)(69.94000507,277.84541162)(69.94000507,279.48040998)
\lineto(75.67000507,279.48040998)
\curveto(75.67000507,282.25540721)(74.56000281,284.05540998)(72.29500507,284.05540998)
\curveto(69.70000767,284.05540998)(68.53000507,282.12040755)(68.53000507,279.69040998)
\curveto(68.53000507,277.42541225)(69.83500728,275.76040998)(72.04000507,275.76040998)
\curveto(73.30000381,275.76040998)(73.81000543,276.06041022)(74.17000507,276.30040998)
\curveto(75.16000408,276.96040932)(75.52000512,278.07041036)(75.56500507,278.44540998)
\lineto(74.24500507,278.44540998)
\moveto(69.94000507,280.53040998)
\curveto(69.94000507,281.74540877)(70.90000629,282.87040998)(72.11500507,282.87040998)
\curveto(73.72000347,282.87040998)(74.23000515,281.74540877)(74.30500507,280.53040998)
\lineto(69.94000507,280.53040998)
}
}
{
\newrgbcolor{curcolor}{0 0 0}
\pscustom[linestyle=none,fillstyle=solid,fillcolor=curcolor]
{
\newpath
\moveto(15.2657082,243.29494123)
\lineto(8.7707082,243.29494123)
\lineto(8.7707082,246.89494123)
\lineto(14.6657082,246.89494123)
\lineto(14.6657082,248.18494123)
\lineto(8.7707082,248.18494123)
\lineto(8.7707082,251.48494123)
\lineto(15.1607082,251.48494123)
\lineto(15.1607082,252.77494123)
\lineto(7.3157082,252.77494123)
\lineto(7.3157082,242.00494123)
\lineto(15.2657082,242.00494123)
\lineto(15.2657082,243.29494123)
}
}
{
\newrgbcolor{curcolor}{0 0 0}
\pscustom[linestyle=none,fillstyle=solid,fillcolor=curcolor]
{
\newpath
\moveto(22.36938007,247.61494123)
\curveto(22.36938007,248.00494084)(22.17437727,250.07494123)(19.36938007,250.07494123)
\curveto(17.82438162,250.07494123)(16.39938007,249.29493951)(16.39938007,247.56994123)
\curveto(16.39938007,246.48994231)(17.11938117,245.93494096)(18.21438007,245.66494123)
\lineto(19.74438007,245.28994123)
\curveto(20.86937895,245.00494152)(21.30438007,244.7949406)(21.30438007,244.16494123)
\curveto(21.30438007,243.2949421)(20.44937913,242.91994123)(19.50438007,242.91994123)
\curveto(17.64438193,242.91994123)(17.46438003,243.90994185)(17.41938007,244.52494123)
\lineto(16.14438007,244.52494123)
\curveto(16.18938003,243.57994218)(16.41438318,241.77994123)(19.51938007,241.77994123)
\curveto(21.2893783,241.77994123)(22.62438007,242.75494285)(22.62438007,244.37494123)
\curveto(22.62438007,245.43994017)(22.05437844,246.03994164)(20.41938007,246.44494123)
\lineto(19.09938007,246.77494123)
\curveto(18.07938109,247.02994098)(17.67438007,247.17994188)(17.67438007,247.82494123)
\curveto(17.67438007,248.79994026)(18.82938048,248.93494123)(19.23438007,248.93494123)
\curveto(20.89937841,248.93494123)(21.07938009,248.10994074)(21.09438007,247.61494123)
\lineto(22.36938007,247.61494123)
}
}
{
\newrgbcolor{curcolor}{0 0 0}
\pscustom[linestyle=none,fillstyle=solid,fillcolor=curcolor]
{
\newpath
\moveto(27.01938007,248.75494123)
\lineto(27.01938007,249.84994123)
\lineto(25.75938007,249.84994123)
\lineto(25.75938007,252.03994123)
\lineto(24.43938007,252.03994123)
\lineto(24.43938007,249.84994123)
\lineto(23.37438007,249.84994123)
\lineto(23.37438007,248.75494123)
\lineto(24.43938007,248.75494123)
\lineto(24.43938007,243.57994123)
\curveto(24.43938007,242.63494218)(24.72438138,241.89994123)(26.02938007,241.89994123)
\curveto(26.16437994,241.89994123)(26.53938055,241.95994128)(27.01938007,242.00494123)
\lineto(27.01938007,243.03994123)
\lineto(26.55438007,243.03994123)
\curveto(26.28438034,243.03994123)(25.75938007,243.03994185)(25.75938007,243.65494123)
\lineto(25.75938007,248.75494123)
\lineto(27.01938007,248.75494123)
}
}
{
\newrgbcolor{curcolor}{0 0 0}
\pscustom[linestyle=none,fillstyle=solid,fillcolor=curcolor]
{
\newpath
\moveto(34.65953632,242.00494123)
\lineto(34.65953632,249.84994123)
\lineto(33.33953632,249.84994123)
\lineto(33.33953632,245.52994123)
\curveto(33.33953632,244.38994237)(32.84453466,242.91994123)(31.17953632,242.91994123)
\curveto(30.32453718,242.91994123)(29.66453632,243.35494252)(29.66453632,244.64494123)
\lineto(29.66453632,249.84994123)
\lineto(28.34453632,249.84994123)
\lineto(28.34453632,244.20994123)
\curveto(28.34453632,242.33494311)(29.73953748,241.77994123)(30.89453632,241.77994123)
\curveto(32.15453506,241.77994123)(32.82953688,242.25994215)(33.38453632,243.17494123)
\lineto(33.41453632,243.14494123)
\lineto(33.41453632,242.00494123)
\lineto(34.65953632,242.00494123)
}
}
{
\newrgbcolor{curcolor}{0 0 0}
\pscustom[linestyle=none,fillstyle=solid,fillcolor=curcolor]
{
\newpath
\moveto(43.1591457,252.77494123)
\lineto(41.8391457,252.77494123)
\lineto(41.8391457,248.84494123)
\lineto(41.8091457,248.73994123)
\curveto(41.49414601,249.18994078)(40.89414427,250.07494123)(39.4691457,250.07494123)
\curveto(37.38414778,250.07494123)(36.1991457,248.36493903)(36.1991457,246.15994123)
\curveto(36.1991457,244.28494311)(36.97914837,241.77994123)(39.6491457,241.77994123)
\curveto(40.41414493,241.77994123)(41.31414627,242.0199423)(41.8841457,243.08494123)
\lineto(41.9141457,243.08494123)
\lineto(41.9141457,242.00494123)
\lineto(43.1591457,242.00494123)
\lineto(43.1591457,252.77494123)
\moveto(37.5641457,245.94994123)
\curveto(37.5641457,246.95494023)(37.66914774,248.88994123)(39.7091457,248.88994123)
\curveto(41.61414379,248.88994123)(41.8241457,246.83493996)(41.8241457,245.55994123)
\curveto(41.8241457,243.47494332)(40.51914486,242.91994123)(39.6791457,242.91994123)
\curveto(38.23914714,242.91994123)(37.5641457,244.22494296)(37.5641457,245.94994123)
}
}
{
\newrgbcolor{curcolor}{0 0 0}
\pscustom[linestyle=none,fillstyle=solid,fillcolor=curcolor]
{
\newpath
\moveto(46.34875507,249.84994123)
\lineto(45.02875507,249.84994123)
\lineto(45.02875507,242.00494123)
\lineto(46.34875507,242.00494123)
\lineto(46.34875507,249.84994123)
\moveto(46.34875507,251.27494123)
\lineto(46.34875507,252.77494123)
\lineto(45.02875507,252.77494123)
\lineto(45.02875507,251.27494123)
\lineto(46.34875507,251.27494123)
}
}
{
\newrgbcolor{curcolor}{0 0 0}
\pscustom[linestyle=none,fillstyle=solid,fillcolor=curcolor]
{
\newpath
\moveto(49.50859882,247.46494123)
\curveto(49.59859873,248.06494063)(49.80860032,248.97994123)(51.30859882,248.97994123)
\curveto(52.55359758,248.97994123)(53.15359882,248.52994041)(53.15359882,247.70494123)
\curveto(53.15359882,246.92494201)(52.77859851,246.8049412)(52.46359882,246.77494123)
\lineto(50.28859882,246.50494123)
\curveto(48.09860101,246.2349415)(47.90359882,244.70494057)(47.90359882,244.04494123)
\curveto(47.90359882,242.69494258)(48.92360026,241.77994123)(50.36359882,241.77994123)
\curveto(51.89359729,241.77994123)(52.68859933,242.49994179)(53.19859882,243.05494123)
\curveto(53.24359878,242.45494183)(53.42359999,241.85494123)(54.59359882,241.85494123)
\curveto(54.89359852,241.85494123)(55.08859905,241.94494129)(55.31359882,242.00494123)
\lineto(55.31359882,242.96494123)
\curveto(55.16359897,242.93494126)(54.9985987,242.90494123)(54.87859882,242.90494123)
\curveto(54.60859909,242.90494123)(54.44359882,243.03994156)(54.44359882,243.36994123)
\lineto(54.44359882,247.88494123)
\curveto(54.44359882,249.89493922)(52.16359819,250.07494123)(51.53359882,250.07494123)
\curveto(49.59860076,250.07494123)(48.35359876,249.33993936)(48.29359882,247.46494123)
\lineto(49.50859882,247.46494123)
\moveto(53.12359882,244.71994123)
\curveto(53.12359882,243.66994228)(51.92359759,242.87494123)(50.69359882,242.87494123)
\curveto(49.70359981,242.87494123)(49.26859882,243.38494209)(49.26859882,244.23994123)
\curveto(49.26859882,245.22994024)(50.30359947,245.42494132)(50.94859882,245.51494123)
\curveto(52.58359719,245.72494102)(52.91359903,245.8449414)(53.12359882,246.00994123)
\lineto(53.12359882,244.71994123)
}
}
{
\newrgbcolor{curcolor}{0 0 0}
\pscustom[linestyle=none,fillstyle=solid,fillcolor=curcolor]
{
\newpath
\moveto(63.0782082,247.34494123)
\curveto(63.0782082,249.579939)(61.54820698,250.07494123)(60.3332082,250.07494123)
\curveto(58.98320955,250.07494123)(58.24820791,249.15994081)(57.9632082,248.73994123)
\lineto(57.9332082,248.73994123)
\lineto(57.9332082,249.84994123)
\lineto(56.6882082,249.84994123)
\lineto(56.6882082,242.00494123)
\lineto(58.0082082,242.00494123)
\lineto(58.0082082,246.27994123)
\curveto(58.0082082,248.4099391)(59.32820895,248.88994123)(60.0782082,248.88994123)
\curveto(61.36820691,248.88994123)(61.7582082,248.19993987)(61.7582082,246.83494123)
\lineto(61.7582082,242.00494123)
\lineto(63.0782082,242.00494123)
\lineto(63.0782082,247.34494123)
}
}
{
\newrgbcolor{curcolor}{0 0 0}
\pscustom[linestyle=none,fillstyle=solid,fillcolor=curcolor]
{
\newpath
\moveto(67.91781757,248.75494123)
\lineto(67.91781757,249.84994123)
\lineto(66.65781757,249.84994123)
\lineto(66.65781757,252.03994123)
\lineto(65.33781757,252.03994123)
\lineto(65.33781757,249.84994123)
\lineto(64.27281757,249.84994123)
\lineto(64.27281757,248.75494123)
\lineto(65.33781757,248.75494123)
\lineto(65.33781757,243.57994123)
\curveto(65.33781757,242.63494218)(65.62281888,241.89994123)(66.92781757,241.89994123)
\curveto(67.06281744,241.89994123)(67.43781805,241.95994128)(67.91781757,242.00494123)
\lineto(67.91781757,243.03994123)
\lineto(67.45281757,243.03994123)
\curveto(67.18281784,243.03994123)(66.65781757,243.03994185)(66.65781757,243.65494123)
\lineto(66.65781757,248.75494123)
\lineto(67.91781757,248.75494123)
}
}
{
\newrgbcolor{curcolor}{0 0 0}
\pscustom[linestyle=none,fillstyle=solid,fillcolor=curcolor]
{
\newpath
\moveto(74.24500507,244.46494123)
\curveto(74.20000512,243.87994182)(73.46500383,242.91994123)(72.22000507,242.91994123)
\curveto(70.70500659,242.91994123)(69.94000507,243.86494287)(69.94000507,245.49994123)
\lineto(75.67000507,245.49994123)
\curveto(75.67000507,248.27493846)(74.56000281,250.07494123)(72.29500507,250.07494123)
\curveto(69.70000767,250.07494123)(68.53000507,248.1399388)(68.53000507,245.70994123)
\curveto(68.53000507,243.4449435)(69.83500728,241.77994123)(72.04000507,241.77994123)
\curveto(73.30000381,241.77994123)(73.81000543,242.07994147)(74.17000507,242.31994123)
\curveto(75.16000408,242.97994057)(75.52000512,244.08994161)(75.56500507,244.46494123)
\lineto(74.24500507,244.46494123)
\moveto(69.94000507,246.54994123)
\curveto(69.94000507,247.76494002)(70.90000629,248.88994123)(72.11500507,248.88994123)
\curveto(73.72000347,248.88994123)(74.23000515,247.76494002)(74.30500507,246.54994123)
\lineto(69.94000507,246.54994123)
}
}
{
\newrgbcolor{curcolor}{0 0 0}
\pscustom[linestyle=none,fillstyle=solid,fillcolor=curcolor]
{
\newpath
\moveto(15.2657082,226.30470686)
\lineto(8.7707082,226.30470686)
\lineto(8.7707082,229.90470686)
\lineto(14.6657082,229.90470686)
\lineto(14.6657082,231.19470686)
\lineto(8.7707082,231.19470686)
\lineto(8.7707082,234.49470686)
\lineto(15.1607082,234.49470686)
\lineto(15.1607082,235.78470686)
\lineto(7.3157082,235.78470686)
\lineto(7.3157082,225.01470686)
\lineto(15.2657082,225.01470686)
\lineto(15.2657082,226.30470686)
}
}
{
\newrgbcolor{curcolor}{0 0 0}
\pscustom[linestyle=none,fillstyle=solid,fillcolor=curcolor]
{
\newpath
\moveto(22.36938007,230.62470686)
\curveto(22.36938007,231.01470647)(22.17437727,233.08470686)(19.36938007,233.08470686)
\curveto(17.82438162,233.08470686)(16.39938007,232.30470513)(16.39938007,230.57970686)
\curveto(16.39938007,229.49970794)(17.11938117,228.94470659)(18.21438007,228.67470686)
\lineto(19.74438007,228.29970686)
\curveto(20.86937895,228.01470714)(21.30438007,227.80470623)(21.30438007,227.17470686)
\curveto(21.30438007,226.30470773)(20.44937913,225.92970686)(19.50438007,225.92970686)
\curveto(17.64438193,225.92970686)(17.46438003,226.91970747)(17.41938007,227.53470686)
\lineto(16.14438007,227.53470686)
\curveto(16.18938003,226.5897078)(16.41438318,224.78970686)(19.51938007,224.78970686)
\curveto(21.2893783,224.78970686)(22.62438007,225.76470848)(22.62438007,227.38470686)
\curveto(22.62438007,228.44970579)(22.05437844,229.04970726)(20.41938007,229.45470686)
\lineto(19.09938007,229.78470686)
\curveto(18.07938109,230.0397066)(17.67438007,230.1897075)(17.67438007,230.83470686)
\curveto(17.67438007,231.80970588)(18.82938048,231.94470686)(19.23438007,231.94470686)
\curveto(20.89937841,231.94470686)(21.07938009,231.11970636)(21.09438007,230.62470686)
\lineto(22.36938007,230.62470686)
}
}
{
\newrgbcolor{curcolor}{0 0 0}
\pscustom[linestyle=none,fillstyle=solid,fillcolor=curcolor]
{
\newpath
\moveto(27.01938007,231.76470686)
\lineto(27.01938007,232.85970686)
\lineto(25.75938007,232.85970686)
\lineto(25.75938007,235.04970686)
\lineto(24.43938007,235.04970686)
\lineto(24.43938007,232.85970686)
\lineto(23.37438007,232.85970686)
\lineto(23.37438007,231.76470686)
\lineto(24.43938007,231.76470686)
\lineto(24.43938007,226.58970686)
\curveto(24.43938007,225.6447078)(24.72438138,224.90970686)(26.02938007,224.90970686)
\curveto(26.16437994,224.90970686)(26.53938055,224.9697069)(27.01938007,225.01470686)
\lineto(27.01938007,226.04970686)
\lineto(26.55438007,226.04970686)
\curveto(26.28438034,226.04970686)(25.75938007,226.04970747)(25.75938007,226.66470686)
\lineto(25.75938007,231.76470686)
\lineto(27.01938007,231.76470686)
}
}
{
\newrgbcolor{curcolor}{0 0 0}
\pscustom[linestyle=none,fillstyle=solid,fillcolor=curcolor]
{
\newpath
\moveto(34.65953632,225.01470686)
\lineto(34.65953632,232.85970686)
\lineto(33.33953632,232.85970686)
\lineto(33.33953632,228.53970686)
\curveto(33.33953632,227.399708)(32.84453466,225.92970686)(31.17953632,225.92970686)
\curveto(30.32453718,225.92970686)(29.66453632,226.36470815)(29.66453632,227.65470686)
\lineto(29.66453632,232.85970686)
\lineto(28.34453632,232.85970686)
\lineto(28.34453632,227.21970686)
\curveto(28.34453632,225.34470873)(29.73953748,224.78970686)(30.89453632,224.78970686)
\curveto(32.15453506,224.78970686)(32.82953688,225.26970777)(33.38453632,226.18470686)
\lineto(33.41453632,226.15470686)
\lineto(33.41453632,225.01470686)
\lineto(34.65953632,225.01470686)
}
}
{
\newrgbcolor{curcolor}{0 0 0}
\pscustom[linestyle=none,fillstyle=solid,fillcolor=curcolor]
{
\newpath
\moveto(43.1591457,235.78470686)
\lineto(41.8391457,235.78470686)
\lineto(41.8391457,231.85470686)
\lineto(41.8091457,231.74970686)
\curveto(41.49414601,232.19970641)(40.89414427,233.08470686)(39.4691457,233.08470686)
\curveto(37.38414778,233.08470686)(36.1991457,231.37470465)(36.1991457,229.16970686)
\curveto(36.1991457,227.29470873)(36.97914837,224.78970686)(39.6491457,224.78970686)
\curveto(40.41414493,224.78970686)(41.31414627,225.02970792)(41.8841457,226.09470686)
\lineto(41.9141457,226.09470686)
\lineto(41.9141457,225.01470686)
\lineto(43.1591457,225.01470686)
\lineto(43.1591457,235.78470686)
\moveto(37.5641457,228.95970686)
\curveto(37.5641457,229.96470585)(37.66914774,231.89970686)(39.7091457,231.89970686)
\curveto(41.61414379,231.89970686)(41.8241457,229.84470558)(41.8241457,228.56970686)
\curveto(41.8241457,226.48470894)(40.51914486,225.92970686)(39.6791457,225.92970686)
\curveto(38.23914714,225.92970686)(37.5641457,227.23470858)(37.5641457,228.95970686)
}
}
{
\newrgbcolor{curcolor}{0 0 0}
\pscustom[linestyle=none,fillstyle=solid,fillcolor=curcolor]
{
\newpath
\moveto(46.34875507,232.85970686)
\lineto(45.02875507,232.85970686)
\lineto(45.02875507,225.01470686)
\lineto(46.34875507,225.01470686)
\lineto(46.34875507,232.85970686)
\moveto(46.34875507,234.28470686)
\lineto(46.34875507,235.78470686)
\lineto(45.02875507,235.78470686)
\lineto(45.02875507,234.28470686)
\lineto(46.34875507,234.28470686)
}
}
{
\newrgbcolor{curcolor}{0 0 0}
\pscustom[linestyle=none,fillstyle=solid,fillcolor=curcolor]
{
\newpath
\moveto(49.50859882,230.47470686)
\curveto(49.59859873,231.07470626)(49.80860032,231.98970686)(51.30859882,231.98970686)
\curveto(52.55359758,231.98970686)(53.15359882,231.53970603)(53.15359882,230.71470686)
\curveto(53.15359882,229.93470764)(52.77859851,229.81470683)(52.46359882,229.78470686)
\lineto(50.28859882,229.51470686)
\curveto(48.09860101,229.24470713)(47.90359882,227.7147062)(47.90359882,227.05470686)
\curveto(47.90359882,225.70470821)(48.92360026,224.78970686)(50.36359882,224.78970686)
\curveto(51.89359729,224.78970686)(52.68859933,225.50970741)(53.19859882,226.06470686)
\curveto(53.24359878,225.46470746)(53.42359999,224.86470686)(54.59359882,224.86470686)
\curveto(54.89359852,224.86470686)(55.08859905,224.95470692)(55.31359882,225.01470686)
\lineto(55.31359882,225.97470686)
\curveto(55.16359897,225.94470689)(54.9985987,225.91470686)(54.87859882,225.91470686)
\curveto(54.60859909,225.91470686)(54.44359882,226.04970719)(54.44359882,226.37970686)
\lineto(54.44359882,230.89470686)
\curveto(54.44359882,232.90470485)(52.16359819,233.08470686)(51.53359882,233.08470686)
\curveto(49.59860076,233.08470686)(48.35359876,232.34970498)(48.29359882,230.47470686)
\lineto(49.50859882,230.47470686)
\moveto(53.12359882,227.72970686)
\curveto(53.12359882,226.67970791)(51.92359759,225.88470686)(50.69359882,225.88470686)
\curveto(49.70359981,225.88470686)(49.26859882,226.39470771)(49.26859882,227.24970686)
\curveto(49.26859882,228.23970587)(50.30359947,228.43470695)(50.94859882,228.52470686)
\curveto(52.58359719,228.73470665)(52.91359903,228.85470702)(53.12359882,229.01970686)
\lineto(53.12359882,227.72970686)
}
}
{
\newrgbcolor{curcolor}{0 0 0}
\pscustom[linestyle=none,fillstyle=solid,fillcolor=curcolor]
{
\newpath
\moveto(63.0782082,230.35470686)
\curveto(63.0782082,232.58970462)(61.54820698,233.08470686)(60.3332082,233.08470686)
\curveto(58.98320955,233.08470686)(58.24820791,232.16970644)(57.9632082,231.74970686)
\lineto(57.9332082,231.74970686)
\lineto(57.9332082,232.85970686)
\lineto(56.6882082,232.85970686)
\lineto(56.6882082,225.01470686)
\lineto(58.0082082,225.01470686)
\lineto(58.0082082,229.28970686)
\curveto(58.0082082,231.41970473)(59.32820895,231.89970686)(60.0782082,231.89970686)
\curveto(61.36820691,231.89970686)(61.7582082,231.20970549)(61.7582082,229.84470686)
\lineto(61.7582082,225.01470686)
\lineto(63.0782082,225.01470686)
\lineto(63.0782082,230.35470686)
}
}
{
\newrgbcolor{curcolor}{0 0 0}
\pscustom[linestyle=none,fillstyle=solid,fillcolor=curcolor]
{
\newpath
\moveto(67.91781757,231.76470686)
\lineto(67.91781757,232.85970686)
\lineto(66.65781757,232.85970686)
\lineto(66.65781757,235.04970686)
\lineto(65.33781757,235.04970686)
\lineto(65.33781757,232.85970686)
\lineto(64.27281757,232.85970686)
\lineto(64.27281757,231.76470686)
\lineto(65.33781757,231.76470686)
\lineto(65.33781757,226.58970686)
\curveto(65.33781757,225.6447078)(65.62281888,224.90970686)(66.92781757,224.90970686)
\curveto(67.06281744,224.90970686)(67.43781805,224.9697069)(67.91781757,225.01470686)
\lineto(67.91781757,226.04970686)
\lineto(67.45281757,226.04970686)
\curveto(67.18281784,226.04970686)(66.65781757,226.04970747)(66.65781757,226.66470686)
\lineto(66.65781757,231.76470686)
\lineto(67.91781757,231.76470686)
}
}
{
\newrgbcolor{curcolor}{0 0 0}
\pscustom[linestyle=none,fillstyle=solid,fillcolor=curcolor]
{
\newpath
\moveto(74.24500507,227.47470686)
\curveto(74.20000512,226.88970744)(73.46500383,225.92970686)(72.22000507,225.92970686)
\curveto(70.70500659,225.92970686)(69.94000507,226.87470849)(69.94000507,228.50970686)
\lineto(75.67000507,228.50970686)
\curveto(75.67000507,231.28470408)(74.56000281,233.08470686)(72.29500507,233.08470686)
\curveto(69.70000767,233.08470686)(68.53000507,231.14970443)(68.53000507,228.71970686)
\curveto(68.53000507,226.45470912)(69.83500728,224.78970686)(72.04000507,224.78970686)
\curveto(73.30000381,224.78970686)(73.81000543,225.0897071)(74.17000507,225.32970686)
\curveto(75.16000408,225.9897062)(75.52000512,227.09970723)(75.56500507,227.47470686)
\lineto(74.24500507,227.47470686)
\moveto(69.94000507,229.55970686)
\curveto(69.94000507,230.77470564)(70.90000629,231.89970686)(72.11500507,231.89970686)
\curveto(73.72000347,231.89970686)(74.23000515,230.77470564)(74.30500507,229.55970686)
\lineto(69.94000507,229.55970686)
}
}
{
\newrgbcolor{curcolor}{0 0 0}
\pscustom[linestyle=none,fillstyle=solid,fillcolor=curcolor]
{
\newpath
\moveto(15.2657082,192.32423811)
\lineto(8.7707082,192.32423811)
\lineto(8.7707082,195.92423811)
\lineto(14.6657082,195.92423811)
\lineto(14.6657082,197.21423811)
\lineto(8.7707082,197.21423811)
\lineto(8.7707082,200.51423811)
\lineto(15.1607082,200.51423811)
\lineto(15.1607082,201.80423811)
\lineto(7.3157082,201.80423811)
\lineto(7.3157082,191.03423811)
\lineto(15.2657082,191.03423811)
\lineto(15.2657082,192.32423811)
}
}
{
\newrgbcolor{curcolor}{0 0 0}
\pscustom[linestyle=none,fillstyle=solid,fillcolor=curcolor]
{
\newpath
\moveto(22.36938007,196.64423811)
\curveto(22.36938007,197.03423772)(22.17437727,199.10423811)(19.36938007,199.10423811)
\curveto(17.82438162,199.10423811)(16.39938007,198.32423638)(16.39938007,196.59923811)
\curveto(16.39938007,195.51923919)(17.11938117,194.96423784)(18.21438007,194.69423811)
\lineto(19.74438007,194.31923811)
\curveto(20.86937895,194.03423839)(21.30438007,193.82423748)(21.30438007,193.19423811)
\curveto(21.30438007,192.32423898)(20.44937913,191.94923811)(19.50438007,191.94923811)
\curveto(17.64438193,191.94923811)(17.46438003,192.93923872)(17.41938007,193.55423811)
\lineto(16.14438007,193.55423811)
\curveto(16.18938003,192.60923905)(16.41438318,190.80923811)(19.51938007,190.80923811)
\curveto(21.2893783,190.80923811)(22.62438007,191.78423973)(22.62438007,193.40423811)
\curveto(22.62438007,194.46923704)(22.05437844,195.06923851)(20.41938007,195.47423811)
\lineto(19.09938007,195.80423811)
\curveto(18.07938109,196.05923785)(17.67438007,196.20923875)(17.67438007,196.85423811)
\curveto(17.67438007,197.82923713)(18.82938048,197.96423811)(19.23438007,197.96423811)
\curveto(20.89937841,197.96423811)(21.07938009,197.13923761)(21.09438007,196.64423811)
\lineto(22.36938007,196.64423811)
}
}
{
\newrgbcolor{curcolor}{0 0 0}
\pscustom[linestyle=none,fillstyle=solid,fillcolor=curcolor]
{
\newpath
\moveto(27.01938007,197.78423811)
\lineto(27.01938007,198.87923811)
\lineto(25.75938007,198.87923811)
\lineto(25.75938007,201.06923811)
\lineto(24.43938007,201.06923811)
\lineto(24.43938007,198.87923811)
\lineto(23.37438007,198.87923811)
\lineto(23.37438007,197.78423811)
\lineto(24.43938007,197.78423811)
\lineto(24.43938007,192.60923811)
\curveto(24.43938007,191.66423905)(24.72438138,190.92923811)(26.02938007,190.92923811)
\curveto(26.16437994,190.92923811)(26.53938055,190.98923815)(27.01938007,191.03423811)
\lineto(27.01938007,192.06923811)
\lineto(26.55438007,192.06923811)
\curveto(26.28438034,192.06923811)(25.75938007,192.06923872)(25.75938007,192.68423811)
\lineto(25.75938007,197.78423811)
\lineto(27.01938007,197.78423811)
}
}
{
\newrgbcolor{curcolor}{0 0 0}
\pscustom[linestyle=none,fillstyle=solid,fillcolor=curcolor]
{
\newpath
\moveto(34.65953632,191.03423811)
\lineto(34.65953632,198.87923811)
\lineto(33.33953632,198.87923811)
\lineto(33.33953632,194.55923811)
\curveto(33.33953632,193.41923925)(32.84453466,191.94923811)(31.17953632,191.94923811)
\curveto(30.32453718,191.94923811)(29.66453632,192.3842394)(29.66453632,193.67423811)
\lineto(29.66453632,198.87923811)
\lineto(28.34453632,198.87923811)
\lineto(28.34453632,193.23923811)
\curveto(28.34453632,191.36423998)(29.73953748,190.80923811)(30.89453632,190.80923811)
\curveto(32.15453506,190.80923811)(32.82953688,191.28923902)(33.38453632,192.20423811)
\lineto(33.41453632,192.17423811)
\lineto(33.41453632,191.03423811)
\lineto(34.65953632,191.03423811)
}
}
{
\newrgbcolor{curcolor}{0 0 0}
\pscustom[linestyle=none,fillstyle=solid,fillcolor=curcolor]
{
\newpath
\moveto(43.1591457,201.80423811)
\lineto(41.8391457,201.80423811)
\lineto(41.8391457,197.87423811)
\lineto(41.8091457,197.76923811)
\curveto(41.49414601,198.21923766)(40.89414427,199.10423811)(39.4691457,199.10423811)
\curveto(37.38414778,199.10423811)(36.1991457,197.3942359)(36.1991457,195.18923811)
\curveto(36.1991457,193.31423998)(36.97914837,190.80923811)(39.6491457,190.80923811)
\curveto(40.41414493,190.80923811)(41.31414627,191.04923917)(41.8841457,192.11423811)
\lineto(41.9141457,192.11423811)
\lineto(41.9141457,191.03423811)
\lineto(43.1591457,191.03423811)
\lineto(43.1591457,201.80423811)
\moveto(37.5641457,194.97923811)
\curveto(37.5641457,195.9842371)(37.66914774,197.91923811)(39.7091457,197.91923811)
\curveto(41.61414379,197.91923811)(41.8241457,195.86423683)(41.8241457,194.58923811)
\curveto(41.8241457,192.50424019)(40.51914486,191.94923811)(39.6791457,191.94923811)
\curveto(38.23914714,191.94923811)(37.5641457,193.25423983)(37.5641457,194.97923811)
}
}
{
\newrgbcolor{curcolor}{0 0 0}
\pscustom[linestyle=none,fillstyle=solid,fillcolor=curcolor]
{
\newpath
\moveto(46.34875507,198.87923811)
\lineto(45.02875507,198.87923811)
\lineto(45.02875507,191.03423811)
\lineto(46.34875507,191.03423811)
\lineto(46.34875507,198.87923811)
\moveto(46.34875507,200.30423811)
\lineto(46.34875507,201.80423811)
\lineto(45.02875507,201.80423811)
\lineto(45.02875507,200.30423811)
\lineto(46.34875507,200.30423811)
}
}
{
\newrgbcolor{curcolor}{0 0 0}
\pscustom[linestyle=none,fillstyle=solid,fillcolor=curcolor]
{
\newpath
\moveto(49.50859882,196.49423811)
\curveto(49.59859873,197.09423751)(49.80860032,198.00923811)(51.30859882,198.00923811)
\curveto(52.55359758,198.00923811)(53.15359882,197.55923728)(53.15359882,196.73423811)
\curveto(53.15359882,195.95423889)(52.77859851,195.83423808)(52.46359882,195.80423811)
\lineto(50.28859882,195.53423811)
\curveto(48.09860101,195.26423838)(47.90359882,193.73423745)(47.90359882,193.07423811)
\curveto(47.90359882,191.72423946)(48.92360026,190.80923811)(50.36359882,190.80923811)
\curveto(51.89359729,190.80923811)(52.68859933,191.52923866)(53.19859882,192.08423811)
\curveto(53.24359878,191.48423871)(53.42359999,190.88423811)(54.59359882,190.88423811)
\curveto(54.89359852,190.88423811)(55.08859905,190.97423817)(55.31359882,191.03423811)
\lineto(55.31359882,191.99423811)
\curveto(55.16359897,191.96423814)(54.9985987,191.93423811)(54.87859882,191.93423811)
\curveto(54.60859909,191.93423811)(54.44359882,192.06923844)(54.44359882,192.39923811)
\lineto(54.44359882,196.91423811)
\curveto(54.44359882,198.9242361)(52.16359819,199.10423811)(51.53359882,199.10423811)
\curveto(49.59860076,199.10423811)(48.35359876,198.36923623)(48.29359882,196.49423811)
\lineto(49.50859882,196.49423811)
\moveto(53.12359882,193.74923811)
\curveto(53.12359882,192.69923916)(51.92359759,191.90423811)(50.69359882,191.90423811)
\curveto(49.70359981,191.90423811)(49.26859882,192.41423896)(49.26859882,193.26923811)
\curveto(49.26859882,194.25923712)(50.30359947,194.4542382)(50.94859882,194.54423811)
\curveto(52.58359719,194.7542379)(52.91359903,194.87423827)(53.12359882,195.03923811)
\lineto(53.12359882,193.74923811)
}
}
{
\newrgbcolor{curcolor}{0 0 0}
\pscustom[linestyle=none,fillstyle=solid,fillcolor=curcolor]
{
\newpath
\moveto(63.0782082,196.37423811)
\curveto(63.0782082,198.60923587)(61.54820698,199.10423811)(60.3332082,199.10423811)
\curveto(58.98320955,199.10423811)(58.24820791,198.18923769)(57.9632082,197.76923811)
\lineto(57.9332082,197.76923811)
\lineto(57.9332082,198.87923811)
\lineto(56.6882082,198.87923811)
\lineto(56.6882082,191.03423811)
\lineto(58.0082082,191.03423811)
\lineto(58.0082082,195.30923811)
\curveto(58.0082082,197.43923598)(59.32820895,197.91923811)(60.0782082,197.91923811)
\curveto(61.36820691,197.91923811)(61.7582082,197.22923674)(61.7582082,195.86423811)
\lineto(61.7582082,191.03423811)
\lineto(63.0782082,191.03423811)
\lineto(63.0782082,196.37423811)
}
}
{
\newrgbcolor{curcolor}{0 0 0}
\pscustom[linestyle=none,fillstyle=solid,fillcolor=curcolor]
{
\newpath
\moveto(67.91781757,197.78423811)
\lineto(67.91781757,198.87923811)
\lineto(66.65781757,198.87923811)
\lineto(66.65781757,201.06923811)
\lineto(65.33781757,201.06923811)
\lineto(65.33781757,198.87923811)
\lineto(64.27281757,198.87923811)
\lineto(64.27281757,197.78423811)
\lineto(65.33781757,197.78423811)
\lineto(65.33781757,192.60923811)
\curveto(65.33781757,191.66423905)(65.62281888,190.92923811)(66.92781757,190.92923811)
\curveto(67.06281744,190.92923811)(67.43781805,190.98923815)(67.91781757,191.03423811)
\lineto(67.91781757,192.06923811)
\lineto(67.45281757,192.06923811)
\curveto(67.18281784,192.06923811)(66.65781757,192.06923872)(66.65781757,192.68423811)
\lineto(66.65781757,197.78423811)
\lineto(67.91781757,197.78423811)
}
}
{
\newrgbcolor{curcolor}{0 0 0}
\pscustom[linestyle=none,fillstyle=solid,fillcolor=curcolor]
{
\newpath
\moveto(74.24500507,193.49423811)
\curveto(74.20000512,192.90923869)(73.46500383,191.94923811)(72.22000507,191.94923811)
\curveto(70.70500659,191.94923811)(69.94000507,192.89423974)(69.94000507,194.52923811)
\lineto(75.67000507,194.52923811)
\curveto(75.67000507,197.30423533)(74.56000281,199.10423811)(72.29500507,199.10423811)
\curveto(69.70000767,199.10423811)(68.53000507,197.16923568)(68.53000507,194.73923811)
\curveto(68.53000507,192.47424037)(69.83500728,190.80923811)(72.04000507,190.80923811)
\curveto(73.30000381,190.80923811)(73.81000543,191.10923835)(74.17000507,191.34923811)
\curveto(75.16000408,192.00923745)(75.52000512,193.11923848)(75.56500507,193.49423811)
\lineto(74.24500507,193.49423811)
\moveto(69.94000507,195.57923811)
\curveto(69.94000507,196.79423689)(70.90000629,197.91923811)(72.11500507,197.91923811)
\curveto(73.72000347,197.91923811)(74.23000515,196.79423689)(74.30500507,195.57923811)
\lineto(69.94000507,195.57923811)
}
}
{
\newrgbcolor{curcolor}{0 0 0}
\pscustom[linestyle=none,fillstyle=solid,fillcolor=curcolor]
{
\newpath
\moveto(15.2657082,141.35353498)
\lineto(8.7707082,141.35353498)
\lineto(8.7707082,144.95353498)
\lineto(14.6657082,144.95353498)
\lineto(14.6657082,146.24353498)
\lineto(8.7707082,146.24353498)
\lineto(8.7707082,149.54353498)
\lineto(15.1607082,149.54353498)
\lineto(15.1607082,150.83353498)
\lineto(7.3157082,150.83353498)
\lineto(7.3157082,140.06353498)
\lineto(15.2657082,140.06353498)
\lineto(15.2657082,141.35353498)
}
}
{
\newrgbcolor{curcolor}{0 0 0}
\pscustom[linestyle=none,fillstyle=solid,fillcolor=curcolor]
{
\newpath
\moveto(22.36938007,145.67353498)
\curveto(22.36938007,146.06353459)(22.17437727,148.13353498)(19.36938007,148.13353498)
\curveto(17.82438162,148.13353498)(16.39938007,147.35353326)(16.39938007,145.62853498)
\curveto(16.39938007,144.54853606)(17.11938117,143.99353471)(18.21438007,143.72353498)
\lineto(19.74438007,143.34853498)
\curveto(20.86937895,143.06353527)(21.30438007,142.85353435)(21.30438007,142.22353498)
\curveto(21.30438007,141.35353585)(20.44937913,140.97853498)(19.50438007,140.97853498)
\curveto(17.64438193,140.97853498)(17.46438003,141.9685356)(17.41938007,142.58353498)
\lineto(16.14438007,142.58353498)
\curveto(16.18938003,141.63853593)(16.41438318,139.83853498)(19.51938007,139.83853498)
\curveto(21.2893783,139.83853498)(22.62438007,140.8135366)(22.62438007,142.43353498)
\curveto(22.62438007,143.49853392)(22.05437844,144.09853539)(20.41938007,144.50353498)
\lineto(19.09938007,144.83353498)
\curveto(18.07938109,145.08853473)(17.67438007,145.23853563)(17.67438007,145.88353498)
\curveto(17.67438007,146.85853401)(18.82938048,146.99353498)(19.23438007,146.99353498)
\curveto(20.89937841,146.99353498)(21.07938009,146.16853449)(21.09438007,145.67353498)
\lineto(22.36938007,145.67353498)
}
}
{
\newrgbcolor{curcolor}{0 0 0}
\pscustom[linestyle=none,fillstyle=solid,fillcolor=curcolor]
{
\newpath
\moveto(27.01938007,146.81353498)
\lineto(27.01938007,147.90853498)
\lineto(25.75938007,147.90853498)
\lineto(25.75938007,150.09853498)
\lineto(24.43938007,150.09853498)
\lineto(24.43938007,147.90853498)
\lineto(23.37438007,147.90853498)
\lineto(23.37438007,146.81353498)
\lineto(24.43938007,146.81353498)
\lineto(24.43938007,141.63853498)
\curveto(24.43938007,140.69353593)(24.72438138,139.95853498)(26.02938007,139.95853498)
\curveto(26.16437994,139.95853498)(26.53938055,140.01853503)(27.01938007,140.06353498)
\lineto(27.01938007,141.09853498)
\lineto(26.55438007,141.09853498)
\curveto(26.28438034,141.09853498)(25.75938007,141.0985356)(25.75938007,141.71353498)
\lineto(25.75938007,146.81353498)
\lineto(27.01938007,146.81353498)
}
}
{
\newrgbcolor{curcolor}{0 0 0}
\pscustom[linestyle=none,fillstyle=solid,fillcolor=curcolor]
{
\newpath
\moveto(34.65953632,140.06353498)
\lineto(34.65953632,147.90853498)
\lineto(33.33953632,147.90853498)
\lineto(33.33953632,143.58853498)
\curveto(33.33953632,142.44853612)(32.84453466,140.97853498)(31.17953632,140.97853498)
\curveto(30.32453718,140.97853498)(29.66453632,141.41353627)(29.66453632,142.70353498)
\lineto(29.66453632,147.90853498)
\lineto(28.34453632,147.90853498)
\lineto(28.34453632,142.26853498)
\curveto(28.34453632,140.39353686)(29.73953748,139.83853498)(30.89453632,139.83853498)
\curveto(32.15453506,139.83853498)(32.82953688,140.3185359)(33.38453632,141.23353498)
\lineto(33.41453632,141.20353498)
\lineto(33.41453632,140.06353498)
\lineto(34.65953632,140.06353498)
}
}
{
\newrgbcolor{curcolor}{0 0 0}
\pscustom[linestyle=none,fillstyle=solid,fillcolor=curcolor]
{
\newpath
\moveto(43.1591457,150.83353498)
\lineto(41.8391457,150.83353498)
\lineto(41.8391457,146.90353498)
\lineto(41.8091457,146.79853498)
\curveto(41.49414601,147.24853453)(40.89414427,148.13353498)(39.4691457,148.13353498)
\curveto(37.38414778,148.13353498)(36.1991457,146.42353278)(36.1991457,144.21853498)
\curveto(36.1991457,142.34353686)(36.97914837,139.83853498)(39.6491457,139.83853498)
\curveto(40.41414493,139.83853498)(41.31414627,140.07853605)(41.8841457,141.14353498)
\lineto(41.9141457,141.14353498)
\lineto(41.9141457,140.06353498)
\lineto(43.1591457,140.06353498)
\lineto(43.1591457,150.83353498)
\moveto(37.5641457,144.00853498)
\curveto(37.5641457,145.01353398)(37.66914774,146.94853498)(39.7091457,146.94853498)
\curveto(41.61414379,146.94853498)(41.8241457,144.89353371)(41.8241457,143.61853498)
\curveto(41.8241457,141.53353707)(40.51914486,140.97853498)(39.6791457,140.97853498)
\curveto(38.23914714,140.97853498)(37.5641457,142.28353671)(37.5641457,144.00853498)
}
}
{
\newrgbcolor{curcolor}{0 0 0}
\pscustom[linestyle=none,fillstyle=solid,fillcolor=curcolor]
{
\newpath
\moveto(46.34875507,147.90853498)
\lineto(45.02875507,147.90853498)
\lineto(45.02875507,140.06353498)
\lineto(46.34875507,140.06353498)
\lineto(46.34875507,147.90853498)
\moveto(46.34875507,149.33353498)
\lineto(46.34875507,150.83353498)
\lineto(45.02875507,150.83353498)
\lineto(45.02875507,149.33353498)
\lineto(46.34875507,149.33353498)
}
}
{
\newrgbcolor{curcolor}{0 0 0}
\pscustom[linestyle=none,fillstyle=solid,fillcolor=curcolor]
{
\newpath
\moveto(49.50859882,145.52353498)
\curveto(49.59859873,146.12353438)(49.80860032,147.03853498)(51.30859882,147.03853498)
\curveto(52.55359758,147.03853498)(53.15359882,146.58853416)(53.15359882,145.76353498)
\curveto(53.15359882,144.98353576)(52.77859851,144.86353495)(52.46359882,144.83353498)
\lineto(50.28859882,144.56353498)
\curveto(48.09860101,144.29353525)(47.90359882,142.76353432)(47.90359882,142.10353498)
\curveto(47.90359882,140.75353633)(48.92360026,139.83853498)(50.36359882,139.83853498)
\curveto(51.89359729,139.83853498)(52.68859933,140.55853554)(53.19859882,141.11353498)
\curveto(53.24359878,140.51353558)(53.42359999,139.91353498)(54.59359882,139.91353498)
\curveto(54.89359852,139.91353498)(55.08859905,140.00353504)(55.31359882,140.06353498)
\lineto(55.31359882,141.02353498)
\curveto(55.16359897,140.99353501)(54.9985987,140.96353498)(54.87859882,140.96353498)
\curveto(54.60859909,140.96353498)(54.44359882,141.09853531)(54.44359882,141.42853498)
\lineto(54.44359882,145.94353498)
\curveto(54.44359882,147.95353297)(52.16359819,148.13353498)(51.53359882,148.13353498)
\curveto(49.59860076,148.13353498)(48.35359876,147.39853311)(48.29359882,145.52353498)
\lineto(49.50859882,145.52353498)
\moveto(53.12359882,142.77853498)
\curveto(53.12359882,141.72853603)(51.92359759,140.93353498)(50.69359882,140.93353498)
\curveto(49.70359981,140.93353498)(49.26859882,141.44353584)(49.26859882,142.29853498)
\curveto(49.26859882,143.28853399)(50.30359947,143.48353507)(50.94859882,143.57353498)
\curveto(52.58359719,143.78353477)(52.91359903,143.90353515)(53.12359882,144.06853498)
\lineto(53.12359882,142.77853498)
}
}
{
\newrgbcolor{curcolor}{0 0 0}
\pscustom[linestyle=none,fillstyle=solid,fillcolor=curcolor]
{
\newpath
\moveto(63.0782082,145.40353498)
\curveto(63.0782082,147.63853275)(61.54820698,148.13353498)(60.3332082,148.13353498)
\curveto(58.98320955,148.13353498)(58.24820791,147.21853456)(57.9632082,146.79853498)
\lineto(57.9332082,146.79853498)
\lineto(57.9332082,147.90853498)
\lineto(56.6882082,147.90853498)
\lineto(56.6882082,140.06353498)
\lineto(58.0082082,140.06353498)
\lineto(58.0082082,144.33853498)
\curveto(58.0082082,146.46853285)(59.32820895,146.94853498)(60.0782082,146.94853498)
\curveto(61.36820691,146.94853498)(61.7582082,146.25853362)(61.7582082,144.89353498)
\lineto(61.7582082,140.06353498)
\lineto(63.0782082,140.06353498)
\lineto(63.0782082,145.40353498)
}
}
{
\newrgbcolor{curcolor}{0 0 0}
\pscustom[linestyle=none,fillstyle=solid,fillcolor=curcolor]
{
\newpath
\moveto(67.91781757,146.81353498)
\lineto(67.91781757,147.90853498)
\lineto(66.65781757,147.90853498)
\lineto(66.65781757,150.09853498)
\lineto(65.33781757,150.09853498)
\lineto(65.33781757,147.90853498)
\lineto(64.27281757,147.90853498)
\lineto(64.27281757,146.81353498)
\lineto(65.33781757,146.81353498)
\lineto(65.33781757,141.63853498)
\curveto(65.33781757,140.69353593)(65.62281888,139.95853498)(66.92781757,139.95853498)
\curveto(67.06281744,139.95853498)(67.43781805,140.01853503)(67.91781757,140.06353498)
\lineto(67.91781757,141.09853498)
\lineto(67.45281757,141.09853498)
\curveto(67.18281784,141.09853498)(66.65781757,141.0985356)(66.65781757,141.71353498)
\lineto(66.65781757,146.81353498)
\lineto(67.91781757,146.81353498)
}
}
{
\newrgbcolor{curcolor}{0 0 0}
\pscustom[linestyle=none,fillstyle=solid,fillcolor=curcolor]
{
\newpath
\moveto(74.24500507,142.52353498)
\curveto(74.20000512,141.93853557)(73.46500383,140.97853498)(72.22000507,140.97853498)
\curveto(70.70500659,140.97853498)(69.94000507,141.92353662)(69.94000507,143.55853498)
\lineto(75.67000507,143.55853498)
\curveto(75.67000507,146.33353221)(74.56000281,148.13353498)(72.29500507,148.13353498)
\curveto(69.70000767,148.13353498)(68.53000507,146.19853255)(68.53000507,143.76853498)
\curveto(68.53000507,141.50353725)(69.83500728,139.83853498)(72.04000507,139.83853498)
\curveto(73.30000381,139.83853498)(73.81000543,140.13853522)(74.17000507,140.37853498)
\curveto(75.16000408,141.03853432)(75.52000512,142.14853536)(75.56500507,142.52353498)
\lineto(74.24500507,142.52353498)
\moveto(69.94000507,144.60853498)
\curveto(69.94000507,145.82353377)(70.90000629,146.94853498)(72.11500507,146.94853498)
\curveto(73.72000347,146.94853498)(74.23000515,145.82353377)(74.30500507,144.60853498)
\lineto(69.94000507,144.60853498)
}
}
{
\newrgbcolor{curcolor}{0 0 0}
\pscustom[linestyle=none,fillstyle=solid,fillcolor=curcolor]
{
\newpath
\moveto(15.2657082,73.39259748)
\lineto(8.7707082,73.39259748)
\lineto(8.7707082,76.99259748)
\lineto(14.6657082,76.99259748)
\lineto(14.6657082,78.28259748)
\lineto(8.7707082,78.28259748)
\lineto(8.7707082,81.58259748)
\lineto(15.1607082,81.58259748)
\lineto(15.1607082,82.87259748)
\lineto(7.3157082,82.87259748)
\lineto(7.3157082,72.10259748)
\lineto(15.2657082,72.10259748)
\lineto(15.2657082,73.39259748)
}
}
{
\newrgbcolor{curcolor}{0 0 0}
\pscustom[linestyle=none,fillstyle=solid,fillcolor=curcolor]
{
\newpath
\moveto(22.36938007,77.71259748)
\curveto(22.36938007,78.10259709)(22.17437727,80.17259748)(19.36938007,80.17259748)
\curveto(17.82438162,80.17259748)(16.39938007,79.39259576)(16.39938007,77.66759748)
\curveto(16.39938007,76.58759856)(17.11938117,76.03259721)(18.21438007,75.76259748)
\lineto(19.74438007,75.38759748)
\curveto(20.86937895,75.10259777)(21.30438007,74.89259685)(21.30438007,74.26259748)
\curveto(21.30438007,73.39259835)(20.44937913,73.01759748)(19.50438007,73.01759748)
\curveto(17.64438193,73.01759748)(17.46438003,74.0075981)(17.41938007,74.62259748)
\lineto(16.14438007,74.62259748)
\curveto(16.18938003,73.67759843)(16.41438318,71.87759748)(19.51938007,71.87759748)
\curveto(21.2893783,71.87759748)(22.62438007,72.8525991)(22.62438007,74.47259748)
\curveto(22.62438007,75.53759642)(22.05437844,76.13759789)(20.41938007,76.54259748)
\lineto(19.09938007,76.87259748)
\curveto(18.07938109,77.12759723)(17.67438007,77.27759813)(17.67438007,77.92259748)
\curveto(17.67438007,78.89759651)(18.82938048,79.03259748)(19.23438007,79.03259748)
\curveto(20.89937841,79.03259748)(21.07938009,78.20759699)(21.09438007,77.71259748)
\lineto(22.36938007,77.71259748)
}
}
{
\newrgbcolor{curcolor}{0 0 0}
\pscustom[linestyle=none,fillstyle=solid,fillcolor=curcolor]
{
\newpath
\moveto(27.01938007,78.85259748)
\lineto(27.01938007,79.94759748)
\lineto(25.75938007,79.94759748)
\lineto(25.75938007,82.13759748)
\lineto(24.43938007,82.13759748)
\lineto(24.43938007,79.94759748)
\lineto(23.37438007,79.94759748)
\lineto(23.37438007,78.85259748)
\lineto(24.43938007,78.85259748)
\lineto(24.43938007,73.67759748)
\curveto(24.43938007,72.73259843)(24.72438138,71.99759748)(26.02938007,71.99759748)
\curveto(26.16437994,71.99759748)(26.53938055,72.05759753)(27.01938007,72.10259748)
\lineto(27.01938007,73.13759748)
\lineto(26.55438007,73.13759748)
\curveto(26.28438034,73.13759748)(25.75938007,73.1375981)(25.75938007,73.75259748)
\lineto(25.75938007,78.85259748)
\lineto(27.01938007,78.85259748)
}
}
{
\newrgbcolor{curcolor}{0 0 0}
\pscustom[linestyle=none,fillstyle=solid,fillcolor=curcolor]
{
\newpath
\moveto(34.65953632,72.10259748)
\lineto(34.65953632,79.94759748)
\lineto(33.33953632,79.94759748)
\lineto(33.33953632,75.62759748)
\curveto(33.33953632,74.48759862)(32.84453466,73.01759748)(31.17953632,73.01759748)
\curveto(30.32453718,73.01759748)(29.66453632,73.45259877)(29.66453632,74.74259748)
\lineto(29.66453632,79.94759748)
\lineto(28.34453632,79.94759748)
\lineto(28.34453632,74.30759748)
\curveto(28.34453632,72.43259936)(29.73953748,71.87759748)(30.89453632,71.87759748)
\curveto(32.15453506,71.87759748)(32.82953688,72.3575984)(33.38453632,73.27259748)
\lineto(33.41453632,73.24259748)
\lineto(33.41453632,72.10259748)
\lineto(34.65953632,72.10259748)
}
}
{
\newrgbcolor{curcolor}{0 0 0}
\pscustom[linestyle=none,fillstyle=solid,fillcolor=curcolor]
{
\newpath
\moveto(43.1591457,82.87259748)
\lineto(41.8391457,82.87259748)
\lineto(41.8391457,78.94259748)
\lineto(41.8091457,78.83759748)
\curveto(41.49414601,79.28759703)(40.89414427,80.17259748)(39.4691457,80.17259748)
\curveto(37.38414778,80.17259748)(36.1991457,78.46259528)(36.1991457,76.25759748)
\curveto(36.1991457,74.38259936)(36.97914837,71.87759748)(39.6491457,71.87759748)
\curveto(40.41414493,71.87759748)(41.31414627,72.11759855)(41.8841457,73.18259748)
\lineto(41.9141457,73.18259748)
\lineto(41.9141457,72.10259748)
\lineto(43.1591457,72.10259748)
\lineto(43.1591457,82.87259748)
\moveto(37.5641457,76.04759748)
\curveto(37.5641457,77.05259648)(37.66914774,78.98759748)(39.7091457,78.98759748)
\curveto(41.61414379,78.98759748)(41.8241457,76.93259621)(41.8241457,75.65759748)
\curveto(41.8241457,73.57259957)(40.51914486,73.01759748)(39.6791457,73.01759748)
\curveto(38.23914714,73.01759748)(37.5641457,74.32259921)(37.5641457,76.04759748)
}
}
{
\newrgbcolor{curcolor}{0 0 0}
\pscustom[linestyle=none,fillstyle=solid,fillcolor=curcolor]
{
\newpath
\moveto(46.34875507,79.94759748)
\lineto(45.02875507,79.94759748)
\lineto(45.02875507,72.10259748)
\lineto(46.34875507,72.10259748)
\lineto(46.34875507,79.94759748)
\moveto(46.34875507,81.37259748)
\lineto(46.34875507,82.87259748)
\lineto(45.02875507,82.87259748)
\lineto(45.02875507,81.37259748)
\lineto(46.34875507,81.37259748)
}
}
{
\newrgbcolor{curcolor}{0 0 0}
\pscustom[linestyle=none,fillstyle=solid,fillcolor=curcolor]
{
\newpath
\moveto(49.50859882,77.56259748)
\curveto(49.59859873,78.16259688)(49.80860032,79.07759748)(51.30859882,79.07759748)
\curveto(52.55359758,79.07759748)(53.15359882,78.62759666)(53.15359882,77.80259748)
\curveto(53.15359882,77.02259826)(52.77859851,76.90259745)(52.46359882,76.87259748)
\lineto(50.28859882,76.60259748)
\curveto(48.09860101,76.33259775)(47.90359882,74.80259682)(47.90359882,74.14259748)
\curveto(47.90359882,72.79259883)(48.92360026,71.87759748)(50.36359882,71.87759748)
\curveto(51.89359729,71.87759748)(52.68859933,72.59759804)(53.19859882,73.15259748)
\curveto(53.24359878,72.55259808)(53.42359999,71.95259748)(54.59359882,71.95259748)
\curveto(54.89359852,71.95259748)(55.08859905,72.04259754)(55.31359882,72.10259748)
\lineto(55.31359882,73.06259748)
\curveto(55.16359897,73.03259751)(54.9985987,73.00259748)(54.87859882,73.00259748)
\curveto(54.60859909,73.00259748)(54.44359882,73.13759781)(54.44359882,73.46759748)
\lineto(54.44359882,77.98259748)
\curveto(54.44359882,79.99259547)(52.16359819,80.17259748)(51.53359882,80.17259748)
\curveto(49.59860076,80.17259748)(48.35359876,79.43759561)(48.29359882,77.56259748)
\lineto(49.50859882,77.56259748)
\moveto(53.12359882,74.81759748)
\curveto(53.12359882,73.76759853)(51.92359759,72.97259748)(50.69359882,72.97259748)
\curveto(49.70359981,72.97259748)(49.26859882,73.48259834)(49.26859882,74.33759748)
\curveto(49.26859882,75.32759649)(50.30359947,75.52259757)(50.94859882,75.61259748)
\curveto(52.58359719,75.82259727)(52.91359903,75.94259765)(53.12359882,76.10759748)
\lineto(53.12359882,74.81759748)
}
}
{
\newrgbcolor{curcolor}{0 0 0}
\pscustom[linestyle=none,fillstyle=solid,fillcolor=curcolor]
{
\newpath
\moveto(63.0782082,77.44259748)
\curveto(63.0782082,79.67759525)(61.54820698,80.17259748)(60.3332082,80.17259748)
\curveto(58.98320955,80.17259748)(58.24820791,79.25759706)(57.9632082,78.83759748)
\lineto(57.9332082,78.83759748)
\lineto(57.9332082,79.94759748)
\lineto(56.6882082,79.94759748)
\lineto(56.6882082,72.10259748)
\lineto(58.0082082,72.10259748)
\lineto(58.0082082,76.37759748)
\curveto(58.0082082,78.50759535)(59.32820895,78.98759748)(60.0782082,78.98759748)
\curveto(61.36820691,78.98759748)(61.7582082,78.29759612)(61.7582082,76.93259748)
\lineto(61.7582082,72.10259748)
\lineto(63.0782082,72.10259748)
\lineto(63.0782082,77.44259748)
}
}
{
\newrgbcolor{curcolor}{0 0 0}
\pscustom[linestyle=none,fillstyle=solid,fillcolor=curcolor]
{
\newpath
\moveto(67.91781757,78.85259748)
\lineto(67.91781757,79.94759748)
\lineto(66.65781757,79.94759748)
\lineto(66.65781757,82.13759748)
\lineto(65.33781757,82.13759748)
\lineto(65.33781757,79.94759748)
\lineto(64.27281757,79.94759748)
\lineto(64.27281757,78.85259748)
\lineto(65.33781757,78.85259748)
\lineto(65.33781757,73.67759748)
\curveto(65.33781757,72.73259843)(65.62281888,71.99759748)(66.92781757,71.99759748)
\curveto(67.06281744,71.99759748)(67.43781805,72.05759753)(67.91781757,72.10259748)
\lineto(67.91781757,73.13759748)
\lineto(67.45281757,73.13759748)
\curveto(67.18281784,73.13759748)(66.65781757,73.1375981)(66.65781757,73.75259748)
\lineto(66.65781757,78.85259748)
\lineto(67.91781757,78.85259748)
}
}
{
\newrgbcolor{curcolor}{0 0 0}
\pscustom[linestyle=none,fillstyle=solid,fillcolor=curcolor]
{
\newpath
\moveto(74.24500507,74.56259748)
\curveto(74.20000512,73.97759807)(73.46500383,73.01759748)(72.22000507,73.01759748)
\curveto(70.70500659,73.01759748)(69.94000507,73.96259912)(69.94000507,75.59759748)
\lineto(75.67000507,75.59759748)
\curveto(75.67000507,78.37259471)(74.56000281,80.17259748)(72.29500507,80.17259748)
\curveto(69.70000767,80.17259748)(68.53000507,78.23759505)(68.53000507,75.80759748)
\curveto(68.53000507,73.54259975)(69.83500728,71.87759748)(72.04000507,71.87759748)
\curveto(73.30000381,71.87759748)(73.81000543,72.17759772)(74.17000507,72.41759748)
\curveto(75.16000408,73.07759682)(75.52000512,74.18759786)(75.56500507,74.56259748)
\lineto(74.24500507,74.56259748)
\moveto(69.94000507,76.64759748)
\curveto(69.94000507,77.86259627)(70.90000629,78.98759748)(72.11500507,78.98759748)
\curveto(73.72000347,78.98759748)(74.23000515,77.86259627)(74.30500507,76.64759748)
\lineto(69.94000507,76.64759748)
}
}
{
\newrgbcolor{curcolor}{0 0 0}
\pscustom[linestyle=none,fillstyle=solid,fillcolor=curcolor]
{
\newpath
\moveto(15.2657082,56.40236311)
\lineto(8.7707082,56.40236311)
\lineto(8.7707082,60.00236311)
\lineto(14.6657082,60.00236311)
\lineto(14.6657082,61.29236311)
\lineto(8.7707082,61.29236311)
\lineto(8.7707082,64.59236311)
\lineto(15.1607082,64.59236311)
\lineto(15.1607082,65.88236311)
\lineto(7.3157082,65.88236311)
\lineto(7.3157082,55.11236311)
\lineto(15.2657082,55.11236311)
\lineto(15.2657082,56.40236311)
}
}
{
\newrgbcolor{curcolor}{0 0 0}
\pscustom[linestyle=none,fillstyle=solid,fillcolor=curcolor]
{
\newpath
\moveto(22.36938007,60.72236311)
\curveto(22.36938007,61.11236272)(22.17437727,63.18236311)(19.36938007,63.18236311)
\curveto(17.82438162,63.18236311)(16.39938007,62.40236138)(16.39938007,60.67736311)
\curveto(16.39938007,59.59736419)(17.11938117,59.04236284)(18.21438007,58.77236311)
\lineto(19.74438007,58.39736311)
\curveto(20.86937895,58.11236339)(21.30438007,57.90236248)(21.30438007,57.27236311)
\curveto(21.30438007,56.40236398)(20.44937913,56.02736311)(19.50438007,56.02736311)
\curveto(17.64438193,56.02736311)(17.46438003,57.01736372)(17.41938007,57.63236311)
\lineto(16.14438007,57.63236311)
\curveto(16.18938003,56.68736405)(16.41438318,54.88736311)(19.51938007,54.88736311)
\curveto(21.2893783,54.88736311)(22.62438007,55.86236473)(22.62438007,57.48236311)
\curveto(22.62438007,58.54736204)(22.05437844,59.14736351)(20.41938007,59.55236311)
\lineto(19.09938007,59.88236311)
\curveto(18.07938109,60.13736285)(17.67438007,60.28736375)(17.67438007,60.93236311)
\curveto(17.67438007,61.90736213)(18.82938048,62.04236311)(19.23438007,62.04236311)
\curveto(20.89937841,62.04236311)(21.07938009,61.21736261)(21.09438007,60.72236311)
\lineto(22.36938007,60.72236311)
}
}
{
\newrgbcolor{curcolor}{0 0 0}
\pscustom[linestyle=none,fillstyle=solid,fillcolor=curcolor]
{
\newpath
\moveto(27.01938007,61.86236311)
\lineto(27.01938007,62.95736311)
\lineto(25.75938007,62.95736311)
\lineto(25.75938007,65.14736311)
\lineto(24.43938007,65.14736311)
\lineto(24.43938007,62.95736311)
\lineto(23.37438007,62.95736311)
\lineto(23.37438007,61.86236311)
\lineto(24.43938007,61.86236311)
\lineto(24.43938007,56.68736311)
\curveto(24.43938007,55.74236405)(24.72438138,55.00736311)(26.02938007,55.00736311)
\curveto(26.16437994,55.00736311)(26.53938055,55.06736315)(27.01938007,55.11236311)
\lineto(27.01938007,56.14736311)
\lineto(26.55438007,56.14736311)
\curveto(26.28438034,56.14736311)(25.75938007,56.14736372)(25.75938007,56.76236311)
\lineto(25.75938007,61.86236311)
\lineto(27.01938007,61.86236311)
}
}
{
\newrgbcolor{curcolor}{0 0 0}
\pscustom[linestyle=none,fillstyle=solid,fillcolor=curcolor]
{
\newpath
\moveto(34.65953632,55.11236311)
\lineto(34.65953632,62.95736311)
\lineto(33.33953632,62.95736311)
\lineto(33.33953632,58.63736311)
\curveto(33.33953632,57.49736425)(32.84453466,56.02736311)(31.17953632,56.02736311)
\curveto(30.32453718,56.02736311)(29.66453632,56.4623644)(29.66453632,57.75236311)
\lineto(29.66453632,62.95736311)
\lineto(28.34453632,62.95736311)
\lineto(28.34453632,57.31736311)
\curveto(28.34453632,55.44236498)(29.73953748,54.88736311)(30.89453632,54.88736311)
\curveto(32.15453506,54.88736311)(32.82953688,55.36736402)(33.38453632,56.28236311)
\lineto(33.41453632,56.25236311)
\lineto(33.41453632,55.11236311)
\lineto(34.65953632,55.11236311)
}
}
{
\newrgbcolor{curcolor}{0 0 0}
\pscustom[linestyle=none,fillstyle=solid,fillcolor=curcolor]
{
\newpath
\moveto(43.1591457,65.88236311)
\lineto(41.8391457,65.88236311)
\lineto(41.8391457,61.95236311)
\lineto(41.8091457,61.84736311)
\curveto(41.49414601,62.29736266)(40.89414427,63.18236311)(39.4691457,63.18236311)
\curveto(37.38414778,63.18236311)(36.1991457,61.4723609)(36.1991457,59.26736311)
\curveto(36.1991457,57.39236498)(36.97914837,54.88736311)(39.6491457,54.88736311)
\curveto(40.41414493,54.88736311)(41.31414627,55.12736417)(41.8841457,56.19236311)
\lineto(41.9141457,56.19236311)
\lineto(41.9141457,55.11236311)
\lineto(43.1591457,55.11236311)
\lineto(43.1591457,65.88236311)
\moveto(37.5641457,59.05736311)
\curveto(37.5641457,60.0623621)(37.66914774,61.99736311)(39.7091457,61.99736311)
\curveto(41.61414379,61.99736311)(41.8241457,59.94236183)(41.8241457,58.66736311)
\curveto(41.8241457,56.58236519)(40.51914486,56.02736311)(39.6791457,56.02736311)
\curveto(38.23914714,56.02736311)(37.5641457,57.33236483)(37.5641457,59.05736311)
}
}
{
\newrgbcolor{curcolor}{0 0 0}
\pscustom[linestyle=none,fillstyle=solid,fillcolor=curcolor]
{
\newpath
\moveto(46.34875507,62.95736311)
\lineto(45.02875507,62.95736311)
\lineto(45.02875507,55.11236311)
\lineto(46.34875507,55.11236311)
\lineto(46.34875507,62.95736311)
\moveto(46.34875507,64.38236311)
\lineto(46.34875507,65.88236311)
\lineto(45.02875507,65.88236311)
\lineto(45.02875507,64.38236311)
\lineto(46.34875507,64.38236311)
}
}
{
\newrgbcolor{curcolor}{0 0 0}
\pscustom[linestyle=none,fillstyle=solid,fillcolor=curcolor]
{
\newpath
\moveto(49.50859882,60.57236311)
\curveto(49.59859873,61.17236251)(49.80860032,62.08736311)(51.30859882,62.08736311)
\curveto(52.55359758,62.08736311)(53.15359882,61.63736228)(53.15359882,60.81236311)
\curveto(53.15359882,60.03236389)(52.77859851,59.91236308)(52.46359882,59.88236311)
\lineto(50.28859882,59.61236311)
\curveto(48.09860101,59.34236338)(47.90359882,57.81236245)(47.90359882,57.15236311)
\curveto(47.90359882,55.80236446)(48.92360026,54.88736311)(50.36359882,54.88736311)
\curveto(51.89359729,54.88736311)(52.68859933,55.60736366)(53.19859882,56.16236311)
\curveto(53.24359878,55.56236371)(53.42359999,54.96236311)(54.59359882,54.96236311)
\curveto(54.89359852,54.96236311)(55.08859905,55.05236317)(55.31359882,55.11236311)
\lineto(55.31359882,56.07236311)
\curveto(55.16359897,56.04236314)(54.9985987,56.01236311)(54.87859882,56.01236311)
\curveto(54.60859909,56.01236311)(54.44359882,56.14736344)(54.44359882,56.47736311)
\lineto(54.44359882,60.99236311)
\curveto(54.44359882,63.0023611)(52.16359819,63.18236311)(51.53359882,63.18236311)
\curveto(49.59860076,63.18236311)(48.35359876,62.44736123)(48.29359882,60.57236311)
\lineto(49.50859882,60.57236311)
\moveto(53.12359882,57.82736311)
\curveto(53.12359882,56.77736416)(51.92359759,55.98236311)(50.69359882,55.98236311)
\curveto(49.70359981,55.98236311)(49.26859882,56.49236396)(49.26859882,57.34736311)
\curveto(49.26859882,58.33736212)(50.30359947,58.5323632)(50.94859882,58.62236311)
\curveto(52.58359719,58.8323629)(52.91359903,58.95236327)(53.12359882,59.11736311)
\lineto(53.12359882,57.82736311)
}
}
{
\newrgbcolor{curcolor}{0 0 0}
\pscustom[linestyle=none,fillstyle=solid,fillcolor=curcolor]
{
\newpath
\moveto(63.0782082,60.45236311)
\curveto(63.0782082,62.68736087)(61.54820698,63.18236311)(60.3332082,63.18236311)
\curveto(58.98320955,63.18236311)(58.24820791,62.26736269)(57.9632082,61.84736311)
\lineto(57.9332082,61.84736311)
\lineto(57.9332082,62.95736311)
\lineto(56.6882082,62.95736311)
\lineto(56.6882082,55.11236311)
\lineto(58.0082082,55.11236311)
\lineto(58.0082082,59.38736311)
\curveto(58.0082082,61.51736098)(59.32820895,61.99736311)(60.0782082,61.99736311)
\curveto(61.36820691,61.99736311)(61.7582082,61.30736174)(61.7582082,59.94236311)
\lineto(61.7582082,55.11236311)
\lineto(63.0782082,55.11236311)
\lineto(63.0782082,60.45236311)
}
}
{
\newrgbcolor{curcolor}{0 0 0}
\pscustom[linestyle=none,fillstyle=solid,fillcolor=curcolor]
{
\newpath
\moveto(67.91781757,61.86236311)
\lineto(67.91781757,62.95736311)
\lineto(66.65781757,62.95736311)
\lineto(66.65781757,65.14736311)
\lineto(65.33781757,65.14736311)
\lineto(65.33781757,62.95736311)
\lineto(64.27281757,62.95736311)
\lineto(64.27281757,61.86236311)
\lineto(65.33781757,61.86236311)
\lineto(65.33781757,56.68736311)
\curveto(65.33781757,55.74236405)(65.62281888,55.00736311)(66.92781757,55.00736311)
\curveto(67.06281744,55.00736311)(67.43781805,55.06736315)(67.91781757,55.11236311)
\lineto(67.91781757,56.14736311)
\lineto(67.45281757,56.14736311)
\curveto(67.18281784,56.14736311)(66.65781757,56.14736372)(66.65781757,56.76236311)
\lineto(66.65781757,61.86236311)
\lineto(67.91781757,61.86236311)
}
}
{
\newrgbcolor{curcolor}{0 0 0}
\pscustom[linestyle=none,fillstyle=solid,fillcolor=curcolor]
{
\newpath
\moveto(74.24500507,57.57236311)
\curveto(74.20000512,56.98736369)(73.46500383,56.02736311)(72.22000507,56.02736311)
\curveto(70.70500659,56.02736311)(69.94000507,56.97236474)(69.94000507,58.60736311)
\lineto(75.67000507,58.60736311)
\curveto(75.67000507,61.38236033)(74.56000281,63.18236311)(72.29500507,63.18236311)
\curveto(69.70000767,63.18236311)(68.53000507,61.24736068)(68.53000507,58.81736311)
\curveto(68.53000507,56.55236537)(69.83500728,54.88736311)(72.04000507,54.88736311)
\curveto(73.30000381,54.88736311)(73.81000543,55.18736335)(74.17000507,55.42736311)
\curveto(75.16000408,56.08736245)(75.52000512,57.19736348)(75.56500507,57.57236311)
\lineto(74.24500507,57.57236311)
\moveto(69.94000507,59.65736311)
\curveto(69.94000507,60.87236189)(70.90000629,61.99736311)(72.11500507,61.99736311)
\curveto(73.72000347,61.99736311)(74.23000515,60.87236189)(74.30500507,59.65736311)
\lineto(69.94000507,59.65736311)
}
}
{
\newrgbcolor{curcolor}{0 0 0}
\pscustom[linestyle=none,fillstyle=solid,fillcolor=curcolor]
{
\newpath
\moveto(15.2657082,39.4120677)
\lineto(8.7707082,39.4120677)
\lineto(8.7707082,43.0120677)
\lineto(14.6657082,43.0120677)
\lineto(14.6657082,44.3020677)
\lineto(8.7707082,44.3020677)
\lineto(8.7707082,47.6020677)
\lineto(15.1607082,47.6020677)
\lineto(15.1607082,48.8920677)
\lineto(7.3157082,48.8920677)
\lineto(7.3157082,38.1220677)
\lineto(15.2657082,38.1220677)
\lineto(15.2657082,39.4120677)
}
}
{
\newrgbcolor{curcolor}{0 0 0}
\pscustom[linestyle=none,fillstyle=solid,fillcolor=curcolor]
{
\newpath
\moveto(22.36938007,43.7320677)
\curveto(22.36938007,44.12206731)(22.17437727,46.1920677)(19.36938007,46.1920677)
\curveto(17.82438162,46.1920677)(16.39938007,45.41206597)(16.39938007,43.6870677)
\curveto(16.39938007,42.60706878)(17.11938117,42.05206743)(18.21438007,41.7820677)
\lineto(19.74438007,41.4070677)
\curveto(20.86937895,41.12206798)(21.30438007,40.91206707)(21.30438007,40.2820677)
\curveto(21.30438007,39.41206857)(20.44937913,39.0370677)(19.50438007,39.0370677)
\curveto(17.64438193,39.0370677)(17.46438003,40.02706831)(17.41938007,40.6420677)
\lineto(16.14438007,40.6420677)
\curveto(16.18938003,39.69706864)(16.41438318,37.8970677)(19.51938007,37.8970677)
\curveto(21.2893783,37.8970677)(22.62438007,38.87206932)(22.62438007,40.4920677)
\curveto(22.62438007,41.55706663)(22.05437844,42.1570681)(20.41938007,42.5620677)
\lineto(19.09938007,42.8920677)
\curveto(18.07938109,43.14706744)(17.67438007,43.29706834)(17.67438007,43.9420677)
\curveto(17.67438007,44.91706672)(18.82938048,45.0520677)(19.23438007,45.0520677)
\curveto(20.89937841,45.0520677)(21.07938009,44.2270672)(21.09438007,43.7320677)
\lineto(22.36938007,43.7320677)
}
}
{
\newrgbcolor{curcolor}{0 0 0}
\pscustom[linestyle=none,fillstyle=solid,fillcolor=curcolor]
{
\newpath
\moveto(27.01938007,44.8720677)
\lineto(27.01938007,45.9670677)
\lineto(25.75938007,45.9670677)
\lineto(25.75938007,48.1570677)
\lineto(24.43938007,48.1570677)
\lineto(24.43938007,45.9670677)
\lineto(23.37438007,45.9670677)
\lineto(23.37438007,44.8720677)
\lineto(24.43938007,44.8720677)
\lineto(24.43938007,39.6970677)
\curveto(24.43938007,38.75206864)(24.72438138,38.0170677)(26.02938007,38.0170677)
\curveto(26.16437994,38.0170677)(26.53938055,38.07706774)(27.01938007,38.1220677)
\lineto(27.01938007,39.1570677)
\lineto(26.55438007,39.1570677)
\curveto(26.28438034,39.1570677)(25.75938007,39.15706831)(25.75938007,39.7720677)
\lineto(25.75938007,44.8720677)
\lineto(27.01938007,44.8720677)
}
}
{
\newrgbcolor{curcolor}{0 0 0}
\pscustom[linestyle=none,fillstyle=solid,fillcolor=curcolor]
{
\newpath
\moveto(34.65953632,38.1220677)
\lineto(34.65953632,45.9670677)
\lineto(33.33953632,45.9670677)
\lineto(33.33953632,41.6470677)
\curveto(33.33953632,40.50706884)(32.84453466,39.0370677)(31.17953632,39.0370677)
\curveto(30.32453718,39.0370677)(29.66453632,39.47206899)(29.66453632,40.7620677)
\lineto(29.66453632,45.9670677)
\lineto(28.34453632,45.9670677)
\lineto(28.34453632,40.3270677)
\curveto(28.34453632,38.45206957)(29.73953748,37.8970677)(30.89453632,37.8970677)
\curveto(32.15453506,37.8970677)(32.82953688,38.37706861)(33.38453632,39.2920677)
\lineto(33.41453632,39.2620677)
\lineto(33.41453632,38.1220677)
\lineto(34.65953632,38.1220677)
}
}
{
\newrgbcolor{curcolor}{0 0 0}
\pscustom[linestyle=none,fillstyle=solid,fillcolor=curcolor]
{
\newpath
\moveto(43.1591457,48.8920677)
\lineto(41.8391457,48.8920677)
\lineto(41.8391457,44.9620677)
\lineto(41.8091457,44.8570677)
\curveto(41.49414601,45.30706725)(40.89414427,46.1920677)(39.4691457,46.1920677)
\curveto(37.38414778,46.1920677)(36.1991457,44.48206549)(36.1991457,42.2770677)
\curveto(36.1991457,40.40206957)(36.97914837,37.8970677)(39.6491457,37.8970677)
\curveto(40.41414493,37.8970677)(41.31414627,38.13706876)(41.8841457,39.2020677)
\lineto(41.9141457,39.2020677)
\lineto(41.9141457,38.1220677)
\lineto(43.1591457,38.1220677)
\lineto(43.1591457,48.8920677)
\moveto(37.5641457,42.0670677)
\curveto(37.5641457,43.07206669)(37.66914774,45.0070677)(39.7091457,45.0070677)
\curveto(41.61414379,45.0070677)(41.8241457,42.95206642)(41.8241457,41.6770677)
\curveto(41.8241457,39.59206978)(40.51914486,39.0370677)(39.6791457,39.0370677)
\curveto(38.23914714,39.0370677)(37.5641457,40.34206942)(37.5641457,42.0670677)
}
}
{
\newrgbcolor{curcolor}{0 0 0}
\pscustom[linestyle=none,fillstyle=solid,fillcolor=curcolor]
{
\newpath
\moveto(46.34875507,45.9670677)
\lineto(45.02875507,45.9670677)
\lineto(45.02875507,38.1220677)
\lineto(46.34875507,38.1220677)
\lineto(46.34875507,45.9670677)
\moveto(46.34875507,47.3920677)
\lineto(46.34875507,48.8920677)
\lineto(45.02875507,48.8920677)
\lineto(45.02875507,47.3920677)
\lineto(46.34875507,47.3920677)
}
}
{
\newrgbcolor{curcolor}{0 0 0}
\pscustom[linestyle=none,fillstyle=solid,fillcolor=curcolor]
{
\newpath
\moveto(49.50859882,43.5820677)
\curveto(49.59859873,44.1820671)(49.80860032,45.0970677)(51.30859882,45.0970677)
\curveto(52.55359758,45.0970677)(53.15359882,44.64706687)(53.15359882,43.8220677)
\curveto(53.15359882,43.04206848)(52.77859851,42.92206767)(52.46359882,42.8920677)
\lineto(50.28859882,42.6220677)
\curveto(48.09860101,42.35206797)(47.90359882,40.82206704)(47.90359882,40.1620677)
\curveto(47.90359882,38.81206905)(48.92360026,37.8970677)(50.36359882,37.8970677)
\curveto(51.89359729,37.8970677)(52.68859933,38.61706825)(53.19859882,39.1720677)
\curveto(53.24359878,38.5720683)(53.42359999,37.9720677)(54.59359882,37.9720677)
\curveto(54.89359852,37.9720677)(55.08859905,38.06206776)(55.31359882,38.1220677)
\lineto(55.31359882,39.0820677)
\curveto(55.16359897,39.05206773)(54.9985987,39.0220677)(54.87859882,39.0220677)
\curveto(54.60859909,39.0220677)(54.44359882,39.15706803)(54.44359882,39.4870677)
\lineto(54.44359882,44.0020677)
\curveto(54.44359882,46.01206569)(52.16359819,46.1920677)(51.53359882,46.1920677)
\curveto(49.59860076,46.1920677)(48.35359876,45.45706582)(48.29359882,43.5820677)
\lineto(49.50859882,43.5820677)
\moveto(53.12359882,40.8370677)
\curveto(53.12359882,39.78706875)(51.92359759,38.9920677)(50.69359882,38.9920677)
\curveto(49.70359981,38.9920677)(49.26859882,39.50206855)(49.26859882,40.3570677)
\curveto(49.26859882,41.34706671)(50.30359947,41.54206779)(50.94859882,41.6320677)
\curveto(52.58359719,41.84206749)(52.91359903,41.96206786)(53.12359882,42.1270677)
\lineto(53.12359882,40.8370677)
}
}
{
\newrgbcolor{curcolor}{0 0 0}
\pscustom[linestyle=none,fillstyle=solid,fillcolor=curcolor]
{
\newpath
\moveto(63.0782082,43.4620677)
\curveto(63.0782082,45.69706546)(61.54820698,46.1920677)(60.3332082,46.1920677)
\curveto(58.98320955,46.1920677)(58.24820791,45.27706728)(57.9632082,44.8570677)
\lineto(57.9332082,44.8570677)
\lineto(57.9332082,45.9670677)
\lineto(56.6882082,45.9670677)
\lineto(56.6882082,38.1220677)
\lineto(58.0082082,38.1220677)
\lineto(58.0082082,42.3970677)
\curveto(58.0082082,44.52706557)(59.32820895,45.0070677)(60.0782082,45.0070677)
\curveto(61.36820691,45.0070677)(61.7582082,44.31706633)(61.7582082,42.9520677)
\lineto(61.7582082,38.1220677)
\lineto(63.0782082,38.1220677)
\lineto(63.0782082,43.4620677)
}
}
{
\newrgbcolor{curcolor}{0 0 0}
\pscustom[linestyle=none,fillstyle=solid,fillcolor=curcolor]
{
\newpath
\moveto(67.91781757,44.8720677)
\lineto(67.91781757,45.9670677)
\lineto(66.65781757,45.9670677)
\lineto(66.65781757,48.1570677)
\lineto(65.33781757,48.1570677)
\lineto(65.33781757,45.9670677)
\lineto(64.27281757,45.9670677)
\lineto(64.27281757,44.8720677)
\lineto(65.33781757,44.8720677)
\lineto(65.33781757,39.6970677)
\curveto(65.33781757,38.75206864)(65.62281888,38.0170677)(66.92781757,38.0170677)
\curveto(67.06281744,38.0170677)(67.43781805,38.07706774)(67.91781757,38.1220677)
\lineto(67.91781757,39.1570677)
\lineto(67.45281757,39.1570677)
\curveto(67.18281784,39.1570677)(66.65781757,39.15706831)(66.65781757,39.7720677)
\lineto(66.65781757,44.8720677)
\lineto(67.91781757,44.8720677)
}
}
{
\newrgbcolor{curcolor}{0 0 0}
\pscustom[linestyle=none,fillstyle=solid,fillcolor=curcolor]
{
\newpath
\moveto(74.24500507,40.5820677)
\curveto(74.20000512,39.99706828)(73.46500383,39.0370677)(72.22000507,39.0370677)
\curveto(70.70500659,39.0370677)(69.94000507,39.98206933)(69.94000507,41.6170677)
\lineto(75.67000507,41.6170677)
\curveto(75.67000507,44.39206492)(74.56000281,46.1920677)(72.29500507,46.1920677)
\curveto(69.70000767,46.1920677)(68.53000507,44.25706527)(68.53000507,41.8270677)
\curveto(68.53000507,39.56206996)(69.83500728,37.8970677)(72.04000507,37.8970677)
\curveto(73.30000381,37.8970677)(73.81000543,38.19706794)(74.17000507,38.4370677)
\curveto(75.16000408,39.09706704)(75.52000512,40.20706807)(75.56500507,40.5820677)
\lineto(74.24500507,40.5820677)
\moveto(69.94000507,42.6670677)
\curveto(69.94000507,43.88206648)(70.90000629,45.0070677)(72.11500507,45.0070677)
\curveto(73.72000347,45.0070677)(74.23000515,43.88206648)(74.30500507,42.6670677)
\lineto(69.94000507,42.6670677)
}
}
{
\newrgbcolor{curcolor}{0 0 0}
\pscustom[linestyle=none,fillstyle=solid,fillcolor=curcolor]
{
\newpath
\moveto(15.2657082,22.42183332)
\lineto(8.7707082,22.42183332)
\lineto(8.7707082,26.02183332)
\lineto(14.6657082,26.02183332)
\lineto(14.6657082,27.31183332)
\lineto(8.7707082,27.31183332)
\lineto(8.7707082,30.61183332)
\lineto(15.1607082,30.61183332)
\lineto(15.1607082,31.90183332)
\lineto(7.3157082,31.90183332)
\lineto(7.3157082,21.13183332)
\lineto(15.2657082,21.13183332)
\lineto(15.2657082,22.42183332)
}
}
{
\newrgbcolor{curcolor}{0 0 0}
\pscustom[linestyle=none,fillstyle=solid,fillcolor=curcolor]
{
\newpath
\moveto(22.36938007,26.74183332)
\curveto(22.36938007,27.13183293)(22.17437727,29.20183332)(19.36938007,29.20183332)
\curveto(17.82438162,29.20183332)(16.39938007,28.4218316)(16.39938007,26.69683332)
\curveto(16.39938007,25.6168344)(17.11938117,25.06183305)(18.21438007,24.79183332)
\lineto(19.74438007,24.41683332)
\curveto(20.86937895,24.13183361)(21.30438007,23.92183269)(21.30438007,23.29183332)
\curveto(21.30438007,22.42183419)(20.44937913,22.04683332)(19.50438007,22.04683332)
\curveto(17.64438193,22.04683332)(17.46438003,23.03683394)(17.41938007,23.65183332)
\lineto(16.14438007,23.65183332)
\curveto(16.18938003,22.70683427)(16.41438318,20.90683332)(19.51938007,20.90683332)
\curveto(21.2893783,20.90683332)(22.62438007,21.88183494)(22.62438007,23.50183332)
\curveto(22.62438007,24.56683226)(22.05437844,25.16683373)(20.41938007,25.57183332)
\lineto(19.09938007,25.90183332)
\curveto(18.07938109,26.15683307)(17.67438007,26.30683397)(17.67438007,26.95183332)
\curveto(17.67438007,27.92683235)(18.82938048,28.06183332)(19.23438007,28.06183332)
\curveto(20.89937841,28.06183332)(21.07938009,27.23683283)(21.09438007,26.74183332)
\lineto(22.36938007,26.74183332)
}
}
{
\newrgbcolor{curcolor}{0 0 0}
\pscustom[linestyle=none,fillstyle=solid,fillcolor=curcolor]
{
\newpath
\moveto(27.01938007,27.88183332)
\lineto(27.01938007,28.97683332)
\lineto(25.75938007,28.97683332)
\lineto(25.75938007,31.16683332)
\lineto(24.43938007,31.16683332)
\lineto(24.43938007,28.97683332)
\lineto(23.37438007,28.97683332)
\lineto(23.37438007,27.88183332)
\lineto(24.43938007,27.88183332)
\lineto(24.43938007,22.70683332)
\curveto(24.43938007,21.76183427)(24.72438138,21.02683332)(26.02938007,21.02683332)
\curveto(26.16437994,21.02683332)(26.53938055,21.08683337)(27.01938007,21.13183332)
\lineto(27.01938007,22.16683332)
\lineto(26.55438007,22.16683332)
\curveto(26.28438034,22.16683332)(25.75938007,22.16683394)(25.75938007,22.78183332)
\lineto(25.75938007,27.88183332)
\lineto(27.01938007,27.88183332)
}
}
{
\newrgbcolor{curcolor}{0 0 0}
\pscustom[linestyle=none,fillstyle=solid,fillcolor=curcolor]
{
\newpath
\moveto(34.65953632,21.13183332)
\lineto(34.65953632,28.97683332)
\lineto(33.33953632,28.97683332)
\lineto(33.33953632,24.65683332)
\curveto(33.33953632,23.51683446)(32.84453466,22.04683332)(31.17953632,22.04683332)
\curveto(30.32453718,22.04683332)(29.66453632,22.48183461)(29.66453632,23.77183332)
\lineto(29.66453632,28.97683332)
\lineto(28.34453632,28.97683332)
\lineto(28.34453632,23.33683332)
\curveto(28.34453632,21.4618352)(29.73953748,20.90683332)(30.89453632,20.90683332)
\curveto(32.15453506,20.90683332)(32.82953688,21.38683424)(33.38453632,22.30183332)
\lineto(33.41453632,22.27183332)
\lineto(33.41453632,21.13183332)
\lineto(34.65953632,21.13183332)
}
}
{
\newrgbcolor{curcolor}{0 0 0}
\pscustom[linestyle=none,fillstyle=solid,fillcolor=curcolor]
{
\newpath
\moveto(43.1591457,31.90183332)
\lineto(41.8391457,31.90183332)
\lineto(41.8391457,27.97183332)
\lineto(41.8091457,27.86683332)
\curveto(41.49414601,28.31683287)(40.89414427,29.20183332)(39.4691457,29.20183332)
\curveto(37.38414778,29.20183332)(36.1991457,27.49183112)(36.1991457,25.28683332)
\curveto(36.1991457,23.4118352)(36.97914837,20.90683332)(39.6491457,20.90683332)
\curveto(40.41414493,20.90683332)(41.31414627,21.14683439)(41.8841457,22.21183332)
\lineto(41.9141457,22.21183332)
\lineto(41.9141457,21.13183332)
\lineto(43.1591457,21.13183332)
\lineto(43.1591457,31.90183332)
\moveto(37.5641457,25.07683332)
\curveto(37.5641457,26.08183232)(37.66914774,28.01683332)(39.7091457,28.01683332)
\curveto(41.61414379,28.01683332)(41.8241457,25.96183205)(41.8241457,24.68683332)
\curveto(41.8241457,22.60183541)(40.51914486,22.04683332)(39.6791457,22.04683332)
\curveto(38.23914714,22.04683332)(37.5641457,23.35183505)(37.5641457,25.07683332)
}
}
{
\newrgbcolor{curcolor}{0 0 0}
\pscustom[linestyle=none,fillstyle=solid,fillcolor=curcolor]
{
\newpath
\moveto(46.34875507,28.97683332)
\lineto(45.02875507,28.97683332)
\lineto(45.02875507,21.13183332)
\lineto(46.34875507,21.13183332)
\lineto(46.34875507,28.97683332)
\moveto(46.34875507,30.40183332)
\lineto(46.34875507,31.90183332)
\lineto(45.02875507,31.90183332)
\lineto(45.02875507,30.40183332)
\lineto(46.34875507,30.40183332)
}
}
{
\newrgbcolor{curcolor}{0 0 0}
\pscustom[linestyle=none,fillstyle=solid,fillcolor=curcolor]
{
\newpath
\moveto(49.50859882,26.59183332)
\curveto(49.59859873,27.19183272)(49.80860032,28.10683332)(51.30859882,28.10683332)
\curveto(52.55359758,28.10683332)(53.15359882,27.6568325)(53.15359882,26.83183332)
\curveto(53.15359882,26.0518341)(52.77859851,25.93183329)(52.46359882,25.90183332)
\lineto(50.28859882,25.63183332)
\curveto(48.09860101,25.36183359)(47.90359882,23.83183266)(47.90359882,23.17183332)
\curveto(47.90359882,21.82183467)(48.92360026,20.90683332)(50.36359882,20.90683332)
\curveto(51.89359729,20.90683332)(52.68859933,21.62683388)(53.19859882,22.18183332)
\curveto(53.24359878,21.58183392)(53.42359999,20.98183332)(54.59359882,20.98183332)
\curveto(54.89359852,20.98183332)(55.08859905,21.07183338)(55.31359882,21.13183332)
\lineto(55.31359882,22.09183332)
\curveto(55.16359897,22.06183335)(54.9985987,22.03183332)(54.87859882,22.03183332)
\curveto(54.60859909,22.03183332)(54.44359882,22.16683365)(54.44359882,22.49683332)
\lineto(54.44359882,27.01183332)
\curveto(54.44359882,29.02183131)(52.16359819,29.20183332)(51.53359882,29.20183332)
\curveto(49.59860076,29.20183332)(48.35359876,28.46683145)(48.29359882,26.59183332)
\lineto(49.50859882,26.59183332)
\moveto(53.12359882,23.84683332)
\curveto(53.12359882,22.79683437)(51.92359759,22.00183332)(50.69359882,22.00183332)
\curveto(49.70359981,22.00183332)(49.26859882,22.51183418)(49.26859882,23.36683332)
\curveto(49.26859882,24.35683233)(50.30359947,24.55183341)(50.94859882,24.64183332)
\curveto(52.58359719,24.85183311)(52.91359903,24.97183349)(53.12359882,25.13683332)
\lineto(53.12359882,23.84683332)
}
}
{
\newrgbcolor{curcolor}{0 0 0}
\pscustom[linestyle=none,fillstyle=solid,fillcolor=curcolor]
{
\newpath
\moveto(63.0782082,26.47183332)
\curveto(63.0782082,28.70683109)(61.54820698,29.20183332)(60.3332082,29.20183332)
\curveto(58.98320955,29.20183332)(58.24820791,28.2868329)(57.9632082,27.86683332)
\lineto(57.9332082,27.86683332)
\lineto(57.9332082,28.97683332)
\lineto(56.6882082,28.97683332)
\lineto(56.6882082,21.13183332)
\lineto(58.0082082,21.13183332)
\lineto(58.0082082,25.40683332)
\curveto(58.0082082,27.53683119)(59.32820895,28.01683332)(60.0782082,28.01683332)
\curveto(61.36820691,28.01683332)(61.7582082,27.32683196)(61.7582082,25.96183332)
\lineto(61.7582082,21.13183332)
\lineto(63.0782082,21.13183332)
\lineto(63.0782082,26.47183332)
}
}
{
\newrgbcolor{curcolor}{0 0 0}
\pscustom[linestyle=none,fillstyle=solid,fillcolor=curcolor]
{
\newpath
\moveto(67.91781757,27.88183332)
\lineto(67.91781757,28.97683332)
\lineto(66.65781757,28.97683332)
\lineto(66.65781757,31.16683332)
\lineto(65.33781757,31.16683332)
\lineto(65.33781757,28.97683332)
\lineto(64.27281757,28.97683332)
\lineto(64.27281757,27.88183332)
\lineto(65.33781757,27.88183332)
\lineto(65.33781757,22.70683332)
\curveto(65.33781757,21.76183427)(65.62281888,21.02683332)(66.92781757,21.02683332)
\curveto(67.06281744,21.02683332)(67.43781805,21.08683337)(67.91781757,21.13183332)
\lineto(67.91781757,22.16683332)
\lineto(67.45281757,22.16683332)
\curveto(67.18281784,22.16683332)(66.65781757,22.16683394)(66.65781757,22.78183332)
\lineto(66.65781757,27.88183332)
\lineto(67.91781757,27.88183332)
}
}
{
\newrgbcolor{curcolor}{0 0 0}
\pscustom[linestyle=none,fillstyle=solid,fillcolor=curcolor]
{
\newpath
\moveto(74.24500507,23.59183332)
\curveto(74.20000512,23.00683391)(73.46500383,22.04683332)(72.22000507,22.04683332)
\curveto(70.70500659,22.04683332)(69.94000507,22.99183496)(69.94000507,24.62683332)
\lineto(75.67000507,24.62683332)
\curveto(75.67000507,27.40183055)(74.56000281,29.20183332)(72.29500507,29.20183332)
\curveto(69.70000767,29.20183332)(68.53000507,27.26683089)(68.53000507,24.83683332)
\curveto(68.53000507,22.57183559)(69.83500728,20.90683332)(72.04000507,20.90683332)
\curveto(73.30000381,20.90683332)(73.81000543,21.20683356)(74.17000507,21.44683332)
\curveto(75.16000408,22.10683266)(75.52000512,23.2168337)(75.56500507,23.59183332)
\lineto(74.24500507,23.59183332)
\moveto(69.94000507,25.67683332)
\curveto(69.94000507,26.89183211)(70.90000629,28.01683332)(72.11500507,28.01683332)
\curveto(73.72000347,28.01683332)(74.23000515,26.89183211)(74.30500507,25.67683332)
\lineto(69.94000507,25.67683332)
}
}
{
\newrgbcolor{curcolor}{0 0 0}
\pscustom[linestyle=none,fillstyle=solid,fillcolor=curcolor]
{
\newpath
\moveto(15.2657082,5.43172102)
\lineto(8.7707082,5.43172102)
\lineto(8.7707082,9.03172102)
\lineto(14.6657082,9.03172102)
\lineto(14.6657082,10.32172102)
\lineto(8.7707082,10.32172102)
\lineto(8.7707082,13.62172102)
\lineto(15.1607082,13.62172102)
\lineto(15.1607082,14.91172102)
\lineto(7.3157082,14.91172102)
\lineto(7.3157082,4.14172102)
\lineto(15.2657082,4.14172102)
\lineto(15.2657082,5.43172102)
}
}
{
\newrgbcolor{curcolor}{0 0 0}
\pscustom[linestyle=none,fillstyle=solid,fillcolor=curcolor]
{
\newpath
\moveto(22.36938007,9.75172102)
\curveto(22.36938007,10.14172063)(22.17437727,12.21172102)(19.36938007,12.21172102)
\curveto(17.82438162,12.21172102)(16.39938007,11.43171929)(16.39938007,9.70672102)
\curveto(16.39938007,8.6267221)(17.11938117,8.07172075)(18.21438007,7.80172102)
\lineto(19.74438007,7.42672102)
\curveto(20.86937895,7.1417213)(21.30438007,6.93172039)(21.30438007,6.30172102)
\curveto(21.30438007,5.43172189)(20.44937913,5.05672102)(19.50438007,5.05672102)
\curveto(17.64438193,5.05672102)(17.46438003,6.04672163)(17.41938007,6.66172102)
\lineto(16.14438007,6.66172102)
\curveto(16.18938003,5.71672196)(16.41438318,3.91672102)(19.51938007,3.91672102)
\curveto(21.2893783,3.91672102)(22.62438007,4.89172264)(22.62438007,6.51172102)
\curveto(22.62438007,7.57671995)(22.05437844,8.17672142)(20.41938007,8.58172102)
\lineto(19.09938007,8.91172102)
\curveto(18.07938109,9.16672076)(17.67438007,9.31672166)(17.67438007,9.96172102)
\curveto(17.67438007,10.93672004)(18.82938048,11.07172102)(19.23438007,11.07172102)
\curveto(20.89937841,11.07172102)(21.07938009,10.24672052)(21.09438007,9.75172102)
\lineto(22.36938007,9.75172102)
}
}
{
\newrgbcolor{curcolor}{0 0 0}
\pscustom[linestyle=none,fillstyle=solid,fillcolor=curcolor]
{
\newpath
\moveto(27.01938007,10.89172102)
\lineto(27.01938007,11.98672102)
\lineto(25.75938007,11.98672102)
\lineto(25.75938007,14.17672102)
\lineto(24.43938007,14.17672102)
\lineto(24.43938007,11.98672102)
\lineto(23.37438007,11.98672102)
\lineto(23.37438007,10.89172102)
\lineto(24.43938007,10.89172102)
\lineto(24.43938007,5.71672102)
\curveto(24.43938007,4.77172196)(24.72438138,4.03672102)(26.02938007,4.03672102)
\curveto(26.16437994,4.03672102)(26.53938055,4.09672106)(27.01938007,4.14172102)
\lineto(27.01938007,5.17672102)
\lineto(26.55438007,5.17672102)
\curveto(26.28438034,5.17672102)(25.75938007,5.17672163)(25.75938007,5.79172102)
\lineto(25.75938007,10.89172102)
\lineto(27.01938007,10.89172102)
}
}
{
\newrgbcolor{curcolor}{0 0 0}
\pscustom[linestyle=none,fillstyle=solid,fillcolor=curcolor]
{
\newpath
\moveto(34.65953632,4.14172102)
\lineto(34.65953632,11.98672102)
\lineto(33.33953632,11.98672102)
\lineto(33.33953632,7.66672102)
\curveto(33.33953632,6.52672216)(32.84453466,5.05672102)(31.17953632,5.05672102)
\curveto(30.32453718,5.05672102)(29.66453632,5.49172231)(29.66453632,6.78172102)
\lineto(29.66453632,11.98672102)
\lineto(28.34453632,11.98672102)
\lineto(28.34453632,6.34672102)
\curveto(28.34453632,4.47172289)(29.73953748,3.91672102)(30.89453632,3.91672102)
\curveto(32.15453506,3.91672102)(32.82953688,4.39672193)(33.38453632,5.31172102)
\lineto(33.41453632,5.28172102)
\lineto(33.41453632,4.14172102)
\lineto(34.65953632,4.14172102)
}
}
{
\newrgbcolor{curcolor}{0 0 0}
\pscustom[linestyle=none,fillstyle=solid,fillcolor=curcolor]
{
\newpath
\moveto(43.1591457,14.91172102)
\lineto(41.8391457,14.91172102)
\lineto(41.8391457,10.98172102)
\lineto(41.8091457,10.87672102)
\curveto(41.49414601,11.32672057)(40.89414427,12.21172102)(39.4691457,12.21172102)
\curveto(37.38414778,12.21172102)(36.1991457,10.50171881)(36.1991457,8.29672102)
\curveto(36.1991457,6.42172289)(36.97914837,3.91672102)(39.6491457,3.91672102)
\curveto(40.41414493,3.91672102)(41.31414627,4.15672208)(41.8841457,5.22172102)
\lineto(41.9141457,5.22172102)
\lineto(41.9141457,4.14172102)
\lineto(43.1591457,4.14172102)
\lineto(43.1591457,14.91172102)
\moveto(37.5641457,8.08672102)
\curveto(37.5641457,9.09172001)(37.66914774,11.02672102)(39.7091457,11.02672102)
\curveto(41.61414379,11.02672102)(41.8241457,8.97171974)(41.8241457,7.69672102)
\curveto(41.8241457,5.6117231)(40.51914486,5.05672102)(39.6791457,5.05672102)
\curveto(38.23914714,5.05672102)(37.5641457,6.36172274)(37.5641457,8.08672102)
}
}
{
\newrgbcolor{curcolor}{0 0 0}
\pscustom[linestyle=none,fillstyle=solid,fillcolor=curcolor]
{
\newpath
\moveto(46.34875507,11.98672102)
\lineto(45.02875507,11.98672102)
\lineto(45.02875507,4.14172102)
\lineto(46.34875507,4.14172102)
\lineto(46.34875507,11.98672102)
\moveto(46.34875507,13.41172102)
\lineto(46.34875507,14.91172102)
\lineto(45.02875507,14.91172102)
\lineto(45.02875507,13.41172102)
\lineto(46.34875507,13.41172102)
}
}
{
\newrgbcolor{curcolor}{0 0 0}
\pscustom[linestyle=none,fillstyle=solid,fillcolor=curcolor]
{
\newpath
\moveto(49.50859882,9.60172102)
\curveto(49.59859873,10.20172042)(49.80860032,11.11672102)(51.30859882,11.11672102)
\curveto(52.55359758,11.11672102)(53.15359882,10.66672019)(53.15359882,9.84172102)
\curveto(53.15359882,9.0617218)(52.77859851,8.94172099)(52.46359882,8.91172102)
\lineto(50.28859882,8.64172102)
\curveto(48.09860101,8.37172129)(47.90359882,6.84172036)(47.90359882,6.18172102)
\curveto(47.90359882,4.83172237)(48.92360026,3.91672102)(50.36359882,3.91672102)
\curveto(51.89359729,3.91672102)(52.68859933,4.63672157)(53.19859882,5.19172102)
\curveto(53.24359878,4.59172162)(53.42359999,3.99172102)(54.59359882,3.99172102)
\curveto(54.89359852,3.99172102)(55.08859905,4.08172108)(55.31359882,4.14172102)
\lineto(55.31359882,5.10172102)
\curveto(55.16359897,5.07172105)(54.9985987,5.04172102)(54.87859882,5.04172102)
\curveto(54.60859909,5.04172102)(54.44359882,5.17672135)(54.44359882,5.50672102)
\lineto(54.44359882,10.02172102)
\curveto(54.44359882,12.03171901)(52.16359819,12.21172102)(51.53359882,12.21172102)
\curveto(49.59860076,12.21172102)(48.35359876,11.47671914)(48.29359882,9.60172102)
\lineto(49.50859882,9.60172102)
\moveto(53.12359882,6.85672102)
\curveto(53.12359882,5.80672207)(51.92359759,5.01172102)(50.69359882,5.01172102)
\curveto(49.70359981,5.01172102)(49.26859882,5.52172187)(49.26859882,6.37672102)
\curveto(49.26859882,7.36672003)(50.30359947,7.56172111)(50.94859882,7.65172102)
\curveto(52.58359719,7.86172081)(52.91359903,7.98172118)(53.12359882,8.14672102)
\lineto(53.12359882,6.85672102)
}
}
{
\newrgbcolor{curcolor}{0 0 0}
\pscustom[linestyle=none,fillstyle=solid,fillcolor=curcolor]
{
\newpath
\moveto(63.0782082,9.48172102)
\curveto(63.0782082,11.71671878)(61.54820698,12.21172102)(60.3332082,12.21172102)
\curveto(58.98320955,12.21172102)(58.24820791,11.2967206)(57.9632082,10.87672102)
\lineto(57.9332082,10.87672102)
\lineto(57.9332082,11.98672102)
\lineto(56.6882082,11.98672102)
\lineto(56.6882082,4.14172102)
\lineto(58.0082082,4.14172102)
\lineto(58.0082082,8.41672102)
\curveto(58.0082082,10.54671889)(59.32820895,11.02672102)(60.0782082,11.02672102)
\curveto(61.36820691,11.02672102)(61.7582082,10.33671965)(61.7582082,8.97172102)
\lineto(61.7582082,4.14172102)
\lineto(63.0782082,4.14172102)
\lineto(63.0782082,9.48172102)
}
}
{
\newrgbcolor{curcolor}{0 0 0}
\pscustom[linestyle=none,fillstyle=solid,fillcolor=curcolor]
{
\newpath
\moveto(67.91781757,10.89172102)
\lineto(67.91781757,11.98672102)
\lineto(66.65781757,11.98672102)
\lineto(66.65781757,14.17672102)
\lineto(65.33781757,14.17672102)
\lineto(65.33781757,11.98672102)
\lineto(64.27281757,11.98672102)
\lineto(64.27281757,10.89172102)
\lineto(65.33781757,10.89172102)
\lineto(65.33781757,5.71672102)
\curveto(65.33781757,4.77172196)(65.62281888,4.03672102)(66.92781757,4.03672102)
\curveto(67.06281744,4.03672102)(67.43781805,4.09672106)(67.91781757,4.14172102)
\lineto(67.91781757,5.17672102)
\lineto(67.45281757,5.17672102)
\curveto(67.18281784,5.17672102)(66.65781757,5.17672163)(66.65781757,5.79172102)
\lineto(66.65781757,10.89172102)
\lineto(67.91781757,10.89172102)
}
}
{
\newrgbcolor{curcolor}{0 0 0}
\pscustom[linestyle=none,fillstyle=solid,fillcolor=curcolor]
{
\newpath
\moveto(74.24500507,6.60172102)
\curveto(74.20000512,6.0167216)(73.46500383,5.05672102)(72.22000507,5.05672102)
\curveto(70.70500659,5.05672102)(69.94000507,6.00172265)(69.94000507,7.63672102)
\lineto(75.67000507,7.63672102)
\curveto(75.67000507,10.41171824)(74.56000281,12.21172102)(72.29500507,12.21172102)
\curveto(69.70000767,12.21172102)(68.53000507,10.27671859)(68.53000507,7.84672102)
\curveto(68.53000507,5.58172328)(69.83500728,3.91672102)(72.04000507,3.91672102)
\curveto(73.30000381,3.91672102)(73.81000543,4.21672126)(74.17000507,4.45672102)
\curveto(75.16000408,5.11672036)(75.52000512,6.22672139)(75.56500507,6.60172102)
\lineto(74.24500507,6.60172102)
\moveto(69.94000507,8.68672102)
\curveto(69.94000507,9.9017198)(70.90000629,11.02672102)(72.11500507,11.02672102)
\curveto(73.72000347,11.02672102)(74.23000515,9.9017198)(74.30500507,8.68672102)
\lineto(69.94000507,8.68672102)
}
}
{
\newrgbcolor{curcolor}{0 0 0}
\pscustom[linestyle=none,fillstyle=solid,fillcolor=curcolor]
{
\newpath
\moveto(8.8457082,133.84330061)
\lineto(7.3907082,133.84330061)
\lineto(7.3907082,123.07330061)
\lineto(8.8457082,123.07330061)
\lineto(8.8457082,133.84330061)
}
}
{
\newrgbcolor{curcolor}{0 0 0}
\pscustom[linestyle=none,fillstyle=solid,fillcolor=curcolor]
{
\newpath
\moveto(17.55086445,128.41330061)
\curveto(17.55086445,130.64829837)(16.02086323,131.14330061)(14.80586445,131.14330061)
\curveto(13.4558658,131.14330061)(12.72086416,130.22830019)(12.43586445,129.80830061)
\lineto(12.40586445,129.80830061)
\lineto(12.40586445,130.91830061)
\lineto(11.16086445,130.91830061)
\lineto(11.16086445,123.07330061)
\lineto(12.48086445,123.07330061)
\lineto(12.48086445,127.34830061)
\curveto(12.48086445,129.47829848)(13.8008652,129.95830061)(14.55086445,129.95830061)
\curveto(15.84086316,129.95830061)(16.23086445,129.26829924)(16.23086445,127.90330061)
\lineto(16.23086445,123.07330061)
\lineto(17.55086445,123.07330061)
\lineto(17.55086445,128.41330061)
}
}
{
\newrgbcolor{curcolor}{0 0 0}
\pscustom[linestyle=none,fillstyle=solid,fillcolor=curcolor]
{
\newpath
\moveto(21.96250507,124.52830061)
\lineto(21.93250507,124.52830061)
\lineto(19.89250507,130.91830061)
\lineto(18.36250507,130.91830061)
\lineto(21.22750507,123.07330061)
\lineto(22.63750507,123.07330061)
\lineto(25.62250507,130.91830061)
\lineto(24.18250507,130.91830061)
\lineto(21.96250507,124.52830061)
}
}
{
\newrgbcolor{curcolor}{0 0 0}
\pscustom[linestyle=none,fillstyle=solid,fillcolor=curcolor]
{
\newpath
\moveto(28.06750507,130.91830061)
\lineto(26.74750507,130.91830061)
\lineto(26.74750507,123.07330061)
\lineto(28.06750507,123.07330061)
\lineto(28.06750507,130.91830061)
\moveto(28.06750507,132.34330061)
\lineto(28.06750507,133.84330061)
\lineto(26.74750507,133.84330061)
\lineto(26.74750507,132.34330061)
\lineto(28.06750507,132.34330061)
}
}
{
\newrgbcolor{curcolor}{0 0 0}
\pscustom[linestyle=none,fillstyle=solid,fillcolor=curcolor]
{
\newpath
\moveto(32.93734882,129.82330061)
\lineto(32.93734882,130.91830061)
\lineto(31.67734882,130.91830061)
\lineto(31.67734882,133.10830061)
\lineto(30.35734882,133.10830061)
\lineto(30.35734882,130.91830061)
\lineto(29.29234882,130.91830061)
\lineto(29.29234882,129.82330061)
\lineto(30.35734882,129.82330061)
\lineto(30.35734882,124.64830061)
\curveto(30.35734882,123.70330155)(30.64235013,122.96830061)(31.94734882,122.96830061)
\curveto(32.08234869,122.96830061)(32.4573493,123.02830065)(32.93734882,123.07330061)
\lineto(32.93734882,124.10830061)
\lineto(32.47234882,124.10830061)
\curveto(32.20234909,124.10830061)(31.67734882,124.10830122)(31.67734882,124.72330061)
\lineto(31.67734882,129.82330061)
\lineto(32.93734882,129.82330061)
}
}
{
\newrgbcolor{curcolor}{0 0 0}
\pscustom[linestyle=none,fillstyle=solid,fillcolor=curcolor]
{
\newpath
\moveto(35.2410207,128.53330061)
\curveto(35.33102061,129.13330001)(35.5410222,130.04830061)(37.0410207,130.04830061)
\curveto(38.28601945,130.04830061)(38.8860207,129.59829978)(38.8860207,128.77330061)
\curveto(38.8860207,127.99330139)(38.51102038,127.87330058)(38.1960207,127.84330061)
\lineto(36.0210207,127.57330061)
\curveto(33.83102289,127.30330088)(33.6360207,125.77329995)(33.6360207,125.11330061)
\curveto(33.6360207,123.76330196)(34.65602214,122.84830061)(36.0960207,122.84830061)
\curveto(37.62601917,122.84830061)(38.42102121,123.56830116)(38.9310207,124.12330061)
\curveto(38.97602065,123.52330121)(39.15602187,122.92330061)(40.3260207,122.92330061)
\curveto(40.6260204,122.92330061)(40.82102092,123.01330067)(41.0460207,123.07330061)
\lineto(41.0460207,124.03330061)
\curveto(40.89602085,124.00330064)(40.73102058,123.97330061)(40.6110207,123.97330061)
\curveto(40.34102097,123.97330061)(40.1760207,124.10830094)(40.1760207,124.43830061)
\lineto(40.1760207,128.95330061)
\curveto(40.1760207,130.9632986)(37.89602007,131.14330061)(37.2660207,131.14330061)
\curveto(35.33102263,131.14330061)(34.08602064,130.40829873)(34.0260207,128.53330061)
\lineto(35.2410207,128.53330061)
\moveto(38.8560207,125.78830061)
\curveto(38.8560207,124.73830166)(37.65601947,123.94330061)(36.4260207,123.94330061)
\curveto(35.43602169,123.94330061)(35.0010207,124.45330146)(35.0010207,125.30830061)
\curveto(35.0010207,126.29829962)(36.03602134,126.4933007)(36.6810207,126.58330061)
\curveto(38.31601906,126.7933004)(38.64602091,126.91330077)(38.8560207,127.07830061)
\lineto(38.8560207,125.78830061)
}
}
{
\newrgbcolor{curcolor}{0 0 0}
\pscustom[linestyle=none,fillstyle=solid,fillcolor=curcolor]
{
\newpath
\moveto(48.93063007,133.84330061)
\lineto(47.61063007,133.84330061)
\lineto(47.61063007,129.91330061)
\lineto(47.58063007,129.80830061)
\curveto(47.26563039,130.25830016)(46.66562865,131.14330061)(45.24063007,131.14330061)
\curveto(43.15563216,131.14330061)(41.97063007,129.4332984)(41.97063007,127.22830061)
\curveto(41.97063007,125.35330248)(42.75063274,122.84830061)(45.42063007,122.84830061)
\curveto(46.18562931,122.84830061)(47.08563064,123.08830167)(47.65563007,124.15330061)
\lineto(47.68563007,124.15330061)
\lineto(47.68563007,123.07330061)
\lineto(48.93063007,123.07330061)
\lineto(48.93063007,133.84330061)
\moveto(43.33563007,127.01830061)
\curveto(43.33563007,128.0232996)(43.44063211,129.95830061)(45.48063007,129.95830061)
\curveto(47.38562817,129.95830061)(47.59563007,127.90329933)(47.59563007,126.62830061)
\curveto(47.59563007,124.54330269)(46.29062923,123.98830061)(45.45063007,123.98830061)
\curveto(44.01063151,123.98830061)(43.33563007,125.29330233)(43.33563007,127.01830061)
}
}
{
\newrgbcolor{curcolor}{0 0 0}
\pscustom[linestyle=none,fillstyle=solid,fillcolor=curcolor]
{
\newpath
\moveto(50.32023945,127.00330061)
\curveto(50.32023945,124.97830263)(51.46024195,122.86330061)(53.96523945,122.86330061)
\curveto(56.47023694,122.86330061)(57.61023945,124.97830263)(57.61023945,127.00330061)
\curveto(57.61023945,129.02829858)(56.47023694,131.14330061)(53.96523945,131.14330061)
\curveto(51.46024195,131.14330061)(50.32023945,129.02829858)(50.32023945,127.00330061)
\moveto(51.68523945,127.00330061)
\curveto(51.68523945,128.05329956)(52.07524134,130.00330061)(53.96523945,130.00330061)
\curveto(55.85523756,130.00330061)(56.24523945,128.05329956)(56.24523945,127.00330061)
\curveto(56.24523945,125.95330166)(55.85523756,124.00330061)(53.96523945,124.00330061)
\curveto(52.07524134,124.00330061)(51.68523945,125.95330166)(51.68523945,127.00330061)
}
}
{
\newrgbcolor{curcolor}{0 0 0}
\pscustom[linestyle=none,fillstyle=solid,fillcolor=curcolor]
{
\newpath
\moveto(8.8457082,116.85306623)
\lineto(7.3907082,116.85306623)
\lineto(7.3907082,106.08306623)
\lineto(8.8457082,106.08306623)
\lineto(8.8457082,116.85306623)
}
}
{
\newrgbcolor{curcolor}{0 0 0}
\pscustom[linestyle=none,fillstyle=solid,fillcolor=curcolor]
{
\newpath
\moveto(17.55086445,111.42306623)
\curveto(17.55086445,113.658064)(16.02086323,114.15306623)(14.80586445,114.15306623)
\curveto(13.4558658,114.15306623)(12.72086416,113.23806581)(12.43586445,112.81806623)
\lineto(12.40586445,112.81806623)
\lineto(12.40586445,113.92806623)
\lineto(11.16086445,113.92806623)
\lineto(11.16086445,106.08306623)
\lineto(12.48086445,106.08306623)
\lineto(12.48086445,110.35806623)
\curveto(12.48086445,112.4880641)(13.8008652,112.96806623)(14.55086445,112.96806623)
\curveto(15.84086316,112.96806623)(16.23086445,112.27806487)(16.23086445,110.91306623)
\lineto(16.23086445,106.08306623)
\lineto(17.55086445,106.08306623)
\lineto(17.55086445,111.42306623)
}
}
{
\newrgbcolor{curcolor}{0 0 0}
\pscustom[linestyle=none,fillstyle=solid,fillcolor=curcolor]
{
\newpath
\moveto(21.96250507,107.53806623)
\lineto(21.93250507,107.53806623)
\lineto(19.89250507,113.92806623)
\lineto(18.36250507,113.92806623)
\lineto(21.22750507,106.08306623)
\lineto(22.63750507,106.08306623)
\lineto(25.62250507,113.92806623)
\lineto(24.18250507,113.92806623)
\lineto(21.96250507,107.53806623)
}
}
{
\newrgbcolor{curcolor}{0 0 0}
\pscustom[linestyle=none,fillstyle=solid,fillcolor=curcolor]
{
\newpath
\moveto(28.06750507,113.92806623)
\lineto(26.74750507,113.92806623)
\lineto(26.74750507,106.08306623)
\lineto(28.06750507,106.08306623)
\lineto(28.06750507,113.92806623)
\moveto(28.06750507,115.35306623)
\lineto(28.06750507,116.85306623)
\lineto(26.74750507,116.85306623)
\lineto(26.74750507,115.35306623)
\lineto(28.06750507,115.35306623)
}
}
{
\newrgbcolor{curcolor}{0 0 0}
\pscustom[linestyle=none,fillstyle=solid,fillcolor=curcolor]
{
\newpath
\moveto(32.93734882,112.83306623)
\lineto(32.93734882,113.92806623)
\lineto(31.67734882,113.92806623)
\lineto(31.67734882,116.11806623)
\lineto(30.35734882,116.11806623)
\lineto(30.35734882,113.92806623)
\lineto(29.29234882,113.92806623)
\lineto(29.29234882,112.83306623)
\lineto(30.35734882,112.83306623)
\lineto(30.35734882,107.65806623)
\curveto(30.35734882,106.71306718)(30.64235013,105.97806623)(31.94734882,105.97806623)
\curveto(32.08234869,105.97806623)(32.4573493,106.03806628)(32.93734882,106.08306623)
\lineto(32.93734882,107.11806623)
\lineto(32.47234882,107.11806623)
\curveto(32.20234909,107.11806623)(31.67734882,107.11806685)(31.67734882,107.73306623)
\lineto(31.67734882,112.83306623)
\lineto(32.93734882,112.83306623)
}
}
{
\newrgbcolor{curcolor}{0 0 0}
\pscustom[linestyle=none,fillstyle=solid,fillcolor=curcolor]
{
\newpath
\moveto(35.2410207,111.54306623)
\curveto(35.33102061,112.14306563)(35.5410222,113.05806623)(37.0410207,113.05806623)
\curveto(38.28601945,113.05806623)(38.8860207,112.60806541)(38.8860207,111.78306623)
\curveto(38.8860207,111.00306701)(38.51102038,110.8830662)(38.1960207,110.85306623)
\lineto(36.0210207,110.58306623)
\curveto(33.83102289,110.3130665)(33.6360207,108.78306557)(33.6360207,108.12306623)
\curveto(33.6360207,106.77306758)(34.65602214,105.85806623)(36.0960207,105.85806623)
\curveto(37.62601917,105.85806623)(38.42102121,106.57806679)(38.9310207,107.13306623)
\curveto(38.97602065,106.53306683)(39.15602187,105.93306623)(40.3260207,105.93306623)
\curveto(40.6260204,105.93306623)(40.82102092,106.02306629)(41.0460207,106.08306623)
\lineto(41.0460207,107.04306623)
\curveto(40.89602085,107.01306626)(40.73102058,106.98306623)(40.6110207,106.98306623)
\curveto(40.34102097,106.98306623)(40.1760207,107.11806656)(40.1760207,107.44806623)
\lineto(40.1760207,111.96306623)
\curveto(40.1760207,113.97306422)(37.89602007,114.15306623)(37.2660207,114.15306623)
\curveto(35.33102263,114.15306623)(34.08602064,113.41806436)(34.0260207,111.54306623)
\lineto(35.2410207,111.54306623)
\moveto(38.8560207,108.79806623)
\curveto(38.8560207,107.74806728)(37.65601947,106.95306623)(36.4260207,106.95306623)
\curveto(35.43602169,106.95306623)(35.0010207,107.46306709)(35.0010207,108.31806623)
\curveto(35.0010207,109.30806524)(36.03602134,109.50306632)(36.6810207,109.59306623)
\curveto(38.31601906,109.80306602)(38.64602091,109.9230664)(38.8560207,110.08806623)
\lineto(38.8560207,108.79806623)
}
}
{
\newrgbcolor{curcolor}{0 0 0}
\pscustom[linestyle=none,fillstyle=solid,fillcolor=curcolor]
{
\newpath
\moveto(48.93063007,116.85306623)
\lineto(47.61063007,116.85306623)
\lineto(47.61063007,112.92306623)
\lineto(47.58063007,112.81806623)
\curveto(47.26563039,113.26806578)(46.66562865,114.15306623)(45.24063007,114.15306623)
\curveto(43.15563216,114.15306623)(41.97063007,112.44306403)(41.97063007,110.23806623)
\curveto(41.97063007,108.36306811)(42.75063274,105.85806623)(45.42063007,105.85806623)
\curveto(46.18562931,105.85806623)(47.08563064,106.0980673)(47.65563007,107.16306623)
\lineto(47.68563007,107.16306623)
\lineto(47.68563007,106.08306623)
\lineto(48.93063007,106.08306623)
\lineto(48.93063007,116.85306623)
\moveto(43.33563007,110.02806623)
\curveto(43.33563007,111.03306523)(43.44063211,112.96806623)(45.48063007,112.96806623)
\curveto(47.38562817,112.96806623)(47.59563007,110.91306496)(47.59563007,109.63806623)
\curveto(47.59563007,107.55306832)(46.29062923,106.99806623)(45.45063007,106.99806623)
\curveto(44.01063151,106.99806623)(43.33563007,108.30306796)(43.33563007,110.02806623)
}
}
{
\newrgbcolor{curcolor}{0 0 0}
\pscustom[linestyle=none,fillstyle=solid,fillcolor=curcolor]
{
\newpath
\moveto(50.32023945,110.01306623)
\curveto(50.32023945,107.98806826)(51.46024195,105.87306623)(53.96523945,105.87306623)
\curveto(56.47023694,105.87306623)(57.61023945,107.98806826)(57.61023945,110.01306623)
\curveto(57.61023945,112.03806421)(56.47023694,114.15306623)(53.96523945,114.15306623)
\curveto(51.46024195,114.15306623)(50.32023945,112.03806421)(50.32023945,110.01306623)
\moveto(51.68523945,110.01306623)
\curveto(51.68523945,111.06306518)(52.07524134,113.01306623)(53.96523945,113.01306623)
\curveto(55.85523756,113.01306623)(56.24523945,111.06306518)(56.24523945,110.01306623)
\curveto(56.24523945,108.96306728)(55.85523756,107.01306623)(53.96523945,107.01306623)
\curveto(52.07524134,107.01306623)(51.68523945,108.96306728)(51.68523945,110.01306623)
}
}
{
\newrgbcolor{curcolor}{0 0 0}
\pscustom[linestyle=none,fillstyle=solid,fillcolor=curcolor]
{
\newpath
\moveto(8.8457082,184.81400373)
\lineto(7.3907082,184.81400373)
\lineto(7.3907082,174.04400373)
\lineto(8.8457082,174.04400373)
\lineto(8.8457082,184.81400373)
}
}
{
\newrgbcolor{curcolor}{0 0 0}
\pscustom[linestyle=none,fillstyle=solid,fillcolor=curcolor]
{
\newpath
\moveto(17.55086445,179.38400373)
\curveto(17.55086445,181.6190015)(16.02086323,182.11400373)(14.80586445,182.11400373)
\curveto(13.4558658,182.11400373)(12.72086416,181.19900331)(12.43586445,180.77900373)
\lineto(12.40586445,180.77900373)
\lineto(12.40586445,181.88900373)
\lineto(11.16086445,181.88900373)
\lineto(11.16086445,174.04400373)
\lineto(12.48086445,174.04400373)
\lineto(12.48086445,178.31900373)
\curveto(12.48086445,180.4490016)(13.8008652,180.92900373)(14.55086445,180.92900373)
\curveto(15.84086316,180.92900373)(16.23086445,180.23900237)(16.23086445,178.87400373)
\lineto(16.23086445,174.04400373)
\lineto(17.55086445,174.04400373)
\lineto(17.55086445,179.38400373)
}
}
{
\newrgbcolor{curcolor}{0 0 0}
\pscustom[linestyle=none,fillstyle=solid,fillcolor=curcolor]
{
\newpath
\moveto(21.96250507,175.49900373)
\lineto(21.93250507,175.49900373)
\lineto(19.89250507,181.88900373)
\lineto(18.36250507,181.88900373)
\lineto(21.22750507,174.04400373)
\lineto(22.63750507,174.04400373)
\lineto(25.62250507,181.88900373)
\lineto(24.18250507,181.88900373)
\lineto(21.96250507,175.49900373)
}
}
{
\newrgbcolor{curcolor}{0 0 0}
\pscustom[linestyle=none,fillstyle=solid,fillcolor=curcolor]
{
\newpath
\moveto(28.06750507,181.88900373)
\lineto(26.74750507,181.88900373)
\lineto(26.74750507,174.04400373)
\lineto(28.06750507,174.04400373)
\lineto(28.06750507,181.88900373)
\moveto(28.06750507,183.31400373)
\lineto(28.06750507,184.81400373)
\lineto(26.74750507,184.81400373)
\lineto(26.74750507,183.31400373)
\lineto(28.06750507,183.31400373)
}
}
{
\newrgbcolor{curcolor}{0 0 0}
\pscustom[linestyle=none,fillstyle=solid,fillcolor=curcolor]
{
\newpath
\moveto(32.93734882,180.79400373)
\lineto(32.93734882,181.88900373)
\lineto(31.67734882,181.88900373)
\lineto(31.67734882,184.07900373)
\lineto(30.35734882,184.07900373)
\lineto(30.35734882,181.88900373)
\lineto(29.29234882,181.88900373)
\lineto(29.29234882,180.79400373)
\lineto(30.35734882,180.79400373)
\lineto(30.35734882,175.61900373)
\curveto(30.35734882,174.67400468)(30.64235013,173.93900373)(31.94734882,173.93900373)
\curveto(32.08234869,173.93900373)(32.4573493,173.99900378)(32.93734882,174.04400373)
\lineto(32.93734882,175.07900373)
\lineto(32.47234882,175.07900373)
\curveto(32.20234909,175.07900373)(31.67734882,175.07900435)(31.67734882,175.69400373)
\lineto(31.67734882,180.79400373)
\lineto(32.93734882,180.79400373)
}
}
{
\newrgbcolor{curcolor}{0 0 0}
\pscustom[linestyle=none,fillstyle=solid,fillcolor=curcolor]
{
\newpath
\moveto(35.2410207,179.50400373)
\curveto(35.33102061,180.10400313)(35.5410222,181.01900373)(37.0410207,181.01900373)
\curveto(38.28601945,181.01900373)(38.8860207,180.56900291)(38.8860207,179.74400373)
\curveto(38.8860207,178.96400451)(38.51102038,178.8440037)(38.1960207,178.81400373)
\lineto(36.0210207,178.54400373)
\curveto(33.83102289,178.274004)(33.6360207,176.74400307)(33.6360207,176.08400373)
\curveto(33.6360207,174.73400508)(34.65602214,173.81900373)(36.0960207,173.81900373)
\curveto(37.62601917,173.81900373)(38.42102121,174.53900429)(38.9310207,175.09400373)
\curveto(38.97602065,174.49400433)(39.15602187,173.89400373)(40.3260207,173.89400373)
\curveto(40.6260204,173.89400373)(40.82102092,173.98400379)(41.0460207,174.04400373)
\lineto(41.0460207,175.00400373)
\curveto(40.89602085,174.97400376)(40.73102058,174.94400373)(40.6110207,174.94400373)
\curveto(40.34102097,174.94400373)(40.1760207,175.07900406)(40.1760207,175.40900373)
\lineto(40.1760207,179.92400373)
\curveto(40.1760207,181.93400172)(37.89602007,182.11400373)(37.2660207,182.11400373)
\curveto(35.33102263,182.11400373)(34.08602064,181.37900186)(34.0260207,179.50400373)
\lineto(35.2410207,179.50400373)
\moveto(38.8560207,176.75900373)
\curveto(38.8560207,175.70900478)(37.65601947,174.91400373)(36.4260207,174.91400373)
\curveto(35.43602169,174.91400373)(35.0010207,175.42400459)(35.0010207,176.27900373)
\curveto(35.0010207,177.26900274)(36.03602134,177.46400382)(36.6810207,177.55400373)
\curveto(38.31601906,177.76400352)(38.64602091,177.8840039)(38.8560207,178.04900373)
\lineto(38.8560207,176.75900373)
}
}
{
\newrgbcolor{curcolor}{0 0 0}
\pscustom[linestyle=none,fillstyle=solid,fillcolor=curcolor]
{
\newpath
\moveto(48.93063007,184.81400373)
\lineto(47.61063007,184.81400373)
\lineto(47.61063007,180.88400373)
\lineto(47.58063007,180.77900373)
\curveto(47.26563039,181.22900328)(46.66562865,182.11400373)(45.24063007,182.11400373)
\curveto(43.15563216,182.11400373)(41.97063007,180.40400153)(41.97063007,178.19900373)
\curveto(41.97063007,176.32400561)(42.75063274,173.81900373)(45.42063007,173.81900373)
\curveto(46.18562931,173.81900373)(47.08563064,174.0590048)(47.65563007,175.12400373)
\lineto(47.68563007,175.12400373)
\lineto(47.68563007,174.04400373)
\lineto(48.93063007,174.04400373)
\lineto(48.93063007,184.81400373)
\moveto(43.33563007,177.98900373)
\curveto(43.33563007,178.99400273)(43.44063211,180.92900373)(45.48063007,180.92900373)
\curveto(47.38562817,180.92900373)(47.59563007,178.87400246)(47.59563007,177.59900373)
\curveto(47.59563007,175.51400582)(46.29062923,174.95900373)(45.45063007,174.95900373)
\curveto(44.01063151,174.95900373)(43.33563007,176.26400546)(43.33563007,177.98900373)
}
}
{
\newrgbcolor{curcolor}{0 0 0}
\pscustom[linestyle=none,fillstyle=solid,fillcolor=curcolor]
{
\newpath
\moveto(50.32023945,177.97400373)
\curveto(50.32023945,175.94900576)(51.46024195,173.83400373)(53.96523945,173.83400373)
\curveto(56.47023694,173.83400373)(57.61023945,175.94900576)(57.61023945,177.97400373)
\curveto(57.61023945,179.99900171)(56.47023694,182.11400373)(53.96523945,182.11400373)
\curveto(51.46024195,182.11400373)(50.32023945,179.99900171)(50.32023945,177.97400373)
\moveto(51.68523945,177.97400373)
\curveto(51.68523945,179.02400268)(52.07524134,180.97400373)(53.96523945,180.97400373)
\curveto(55.85523756,180.97400373)(56.24523945,179.02400268)(56.24523945,177.97400373)
\curveto(56.24523945,176.92400478)(55.85523756,174.97400373)(53.96523945,174.97400373)
\curveto(52.07524134,174.97400373)(51.68523945,176.92400478)(51.68523945,177.97400373)
}
}
{
\newrgbcolor{curcolor}{0 0 0}
\pscustom[linestyle=none,fillstyle=solid,fillcolor=curcolor]
{
\newpath
\moveto(8.8457082,218.79447248)
\lineto(7.3907082,218.79447248)
\lineto(7.3907082,208.02447248)
\lineto(8.8457082,208.02447248)
\lineto(8.8457082,218.79447248)
}
}
{
\newrgbcolor{curcolor}{0 0 0}
\pscustom[linestyle=none,fillstyle=solid,fillcolor=curcolor]
{
\newpath
\moveto(17.55086445,213.36447248)
\curveto(17.55086445,215.59947025)(16.02086323,216.09447248)(14.80586445,216.09447248)
\curveto(13.4558658,216.09447248)(12.72086416,215.17947206)(12.43586445,214.75947248)
\lineto(12.40586445,214.75947248)
\lineto(12.40586445,215.86947248)
\lineto(11.16086445,215.86947248)
\lineto(11.16086445,208.02447248)
\lineto(12.48086445,208.02447248)
\lineto(12.48086445,212.29947248)
\curveto(12.48086445,214.42947035)(13.8008652,214.90947248)(14.55086445,214.90947248)
\curveto(15.84086316,214.90947248)(16.23086445,214.21947112)(16.23086445,212.85447248)
\lineto(16.23086445,208.02447248)
\lineto(17.55086445,208.02447248)
\lineto(17.55086445,213.36447248)
}
}
{
\newrgbcolor{curcolor}{0 0 0}
\pscustom[linestyle=none,fillstyle=solid,fillcolor=curcolor]
{
\newpath
\moveto(21.96250507,209.47947248)
\lineto(21.93250507,209.47947248)
\lineto(19.89250507,215.86947248)
\lineto(18.36250507,215.86947248)
\lineto(21.22750507,208.02447248)
\lineto(22.63750507,208.02447248)
\lineto(25.62250507,215.86947248)
\lineto(24.18250507,215.86947248)
\lineto(21.96250507,209.47947248)
}
}
{
\newrgbcolor{curcolor}{0 0 0}
\pscustom[linestyle=none,fillstyle=solid,fillcolor=curcolor]
{
\newpath
\moveto(28.06750507,215.86947248)
\lineto(26.74750507,215.86947248)
\lineto(26.74750507,208.02447248)
\lineto(28.06750507,208.02447248)
\lineto(28.06750507,215.86947248)
\moveto(28.06750507,217.29447248)
\lineto(28.06750507,218.79447248)
\lineto(26.74750507,218.79447248)
\lineto(26.74750507,217.29447248)
\lineto(28.06750507,217.29447248)
}
}
{
\newrgbcolor{curcolor}{0 0 0}
\pscustom[linestyle=none,fillstyle=solid,fillcolor=curcolor]
{
\newpath
\moveto(32.93734882,214.77447248)
\lineto(32.93734882,215.86947248)
\lineto(31.67734882,215.86947248)
\lineto(31.67734882,218.05947248)
\lineto(30.35734882,218.05947248)
\lineto(30.35734882,215.86947248)
\lineto(29.29234882,215.86947248)
\lineto(29.29234882,214.77447248)
\lineto(30.35734882,214.77447248)
\lineto(30.35734882,209.59947248)
\curveto(30.35734882,208.65447343)(30.64235013,207.91947248)(31.94734882,207.91947248)
\curveto(32.08234869,207.91947248)(32.4573493,207.97947253)(32.93734882,208.02447248)
\lineto(32.93734882,209.05947248)
\lineto(32.47234882,209.05947248)
\curveto(32.20234909,209.05947248)(31.67734882,209.0594731)(31.67734882,209.67447248)
\lineto(31.67734882,214.77447248)
\lineto(32.93734882,214.77447248)
}
}
{
\newrgbcolor{curcolor}{0 0 0}
\pscustom[linestyle=none,fillstyle=solid,fillcolor=curcolor]
{
\newpath
\moveto(35.2410207,213.48447248)
\curveto(35.33102061,214.08447188)(35.5410222,214.99947248)(37.0410207,214.99947248)
\curveto(38.28601945,214.99947248)(38.8860207,214.54947166)(38.8860207,213.72447248)
\curveto(38.8860207,212.94447326)(38.51102038,212.82447245)(38.1960207,212.79447248)
\lineto(36.0210207,212.52447248)
\curveto(33.83102289,212.25447275)(33.6360207,210.72447182)(33.6360207,210.06447248)
\curveto(33.6360207,208.71447383)(34.65602214,207.79947248)(36.0960207,207.79947248)
\curveto(37.62601917,207.79947248)(38.42102121,208.51947304)(38.9310207,209.07447248)
\curveto(38.97602065,208.47447308)(39.15602187,207.87447248)(40.3260207,207.87447248)
\curveto(40.6260204,207.87447248)(40.82102092,207.96447254)(41.0460207,208.02447248)
\lineto(41.0460207,208.98447248)
\curveto(40.89602085,208.95447251)(40.73102058,208.92447248)(40.6110207,208.92447248)
\curveto(40.34102097,208.92447248)(40.1760207,209.05947281)(40.1760207,209.38947248)
\lineto(40.1760207,213.90447248)
\curveto(40.1760207,215.91447047)(37.89602007,216.09447248)(37.2660207,216.09447248)
\curveto(35.33102263,216.09447248)(34.08602064,215.35947061)(34.0260207,213.48447248)
\lineto(35.2410207,213.48447248)
\moveto(38.8560207,210.73947248)
\curveto(38.8560207,209.68947353)(37.65601947,208.89447248)(36.4260207,208.89447248)
\curveto(35.43602169,208.89447248)(35.0010207,209.40447334)(35.0010207,210.25947248)
\curveto(35.0010207,211.24947149)(36.03602134,211.44447257)(36.6810207,211.53447248)
\curveto(38.31601906,211.74447227)(38.64602091,211.86447265)(38.8560207,212.02947248)
\lineto(38.8560207,210.73947248)
}
}
{
\newrgbcolor{curcolor}{0 0 0}
\pscustom[linestyle=none,fillstyle=solid,fillcolor=curcolor]
{
\newpath
\moveto(48.93063007,218.79447248)
\lineto(47.61063007,218.79447248)
\lineto(47.61063007,214.86447248)
\lineto(47.58063007,214.75947248)
\curveto(47.26563039,215.20947203)(46.66562865,216.09447248)(45.24063007,216.09447248)
\curveto(43.15563216,216.09447248)(41.97063007,214.38447028)(41.97063007,212.17947248)
\curveto(41.97063007,210.30447436)(42.75063274,207.79947248)(45.42063007,207.79947248)
\curveto(46.18562931,207.79947248)(47.08563064,208.03947355)(47.65563007,209.10447248)
\lineto(47.68563007,209.10447248)
\lineto(47.68563007,208.02447248)
\lineto(48.93063007,208.02447248)
\lineto(48.93063007,218.79447248)
\moveto(43.33563007,211.96947248)
\curveto(43.33563007,212.97447148)(43.44063211,214.90947248)(45.48063007,214.90947248)
\curveto(47.38562817,214.90947248)(47.59563007,212.85447121)(47.59563007,211.57947248)
\curveto(47.59563007,209.49447457)(46.29062923,208.93947248)(45.45063007,208.93947248)
\curveto(44.01063151,208.93947248)(43.33563007,210.24447421)(43.33563007,211.96947248)
}
}
{
\newrgbcolor{curcolor}{0 0 0}
\pscustom[linestyle=none,fillstyle=solid,fillcolor=curcolor]
{
\newpath
\moveto(50.32023945,211.95447248)
\curveto(50.32023945,209.92947451)(51.46024195,207.81447248)(53.96523945,207.81447248)
\curveto(56.47023694,207.81447248)(57.61023945,209.92947451)(57.61023945,211.95447248)
\curveto(57.61023945,213.97947046)(56.47023694,216.09447248)(53.96523945,216.09447248)
\curveto(51.46024195,216.09447248)(50.32023945,213.97947046)(50.32023945,211.95447248)
\moveto(51.68523945,211.95447248)
\curveto(51.68523945,213.00447143)(52.07524134,214.95447248)(53.96523945,214.95447248)
\curveto(55.85523756,214.95447248)(56.24523945,213.00447143)(56.24523945,211.95447248)
\curveto(56.24523945,210.90447353)(55.85523756,208.95447248)(53.96523945,208.95447248)
\curveto(52.07524134,208.95447248)(51.68523945,210.90447353)(51.68523945,211.95447248)
}
}
{
\newrgbcolor{curcolor}{0 0 0}
\pscustom[linestyle=none,fillstyle=solid,fillcolor=curcolor]
{
\newpath
\moveto(7.10569843,343.94634748)
\lineto(11.41069843,343.94634748)
\curveto(14.95069489,343.94634748)(16.00069843,347.0663499)(16.00069843,349.48134748)
\curveto(16.00069843,352.58634438)(14.27569563,354.71634748)(11.47069843,354.71634748)
\lineto(7.10569843,354.71634748)
\lineto(7.10569843,343.94634748)
\moveto(8.56069843,353.47134748)
\lineto(11.27569843,353.47134748)
\curveto(13.25569645,353.47134748)(14.50069843,352.10634477)(14.50069843,349.39134748)
\curveto(14.50069843,346.6763502)(13.27069654,345.19134748)(11.38069843,345.19134748)
\lineto(8.56069843,345.19134748)
\lineto(8.56069843,353.47134748)
}
}
{
\newrgbcolor{curcolor}{0 0 0}
\pscustom[linestyle=none,fillstyle=solid,fillcolor=curcolor]
{
\newpath
\moveto(17.25554218,347.87634748)
\curveto(17.25554218,345.85134951)(18.39554469,343.73634748)(20.90054218,343.73634748)
\curveto(23.40553968,343.73634748)(24.54554218,345.85134951)(24.54554218,347.87634748)
\curveto(24.54554218,349.90134546)(23.40553968,352.01634748)(20.90054218,352.01634748)
\curveto(18.39554469,352.01634748)(17.25554218,349.90134546)(17.25554218,347.87634748)
\moveto(18.62054218,347.87634748)
\curveto(18.62054218,348.92634643)(19.01054407,350.87634748)(20.90054218,350.87634748)
\curveto(22.79054029,350.87634748)(23.18054218,348.92634643)(23.18054218,347.87634748)
\curveto(23.18054218,346.82634853)(22.79054029,344.87634748)(20.90054218,344.87634748)
\curveto(19.01054407,344.87634748)(18.62054218,346.82634853)(18.62054218,347.87634748)
}
}
{
\newrgbcolor{curcolor}{0 0 0}
\pscustom[linestyle=none,fillstyle=solid,fillcolor=curcolor]
{
\newpath
\moveto(32.23515156,349.15134748)
\curveto(32.13015166,350.51634612)(31.3501495,352.01634748)(29.29515156,352.01634748)
\curveto(26.70015415,352.01634748)(25.53015156,350.08134505)(25.53015156,347.65134748)
\curveto(25.53015156,345.38634975)(26.83515376,343.72134748)(29.04015156,343.72134748)
\curveto(31.33514926,343.72134748)(32.10015169,345.47634873)(32.23515156,346.72134748)
\lineto(30.96015156,346.72134748)
\curveto(30.73515178,345.52134868)(29.97015067,344.86134748)(29.08515156,344.86134748)
\curveto(27.27015337,344.86134748)(26.94015156,346.52634883)(26.94015156,347.87634748)
\curveto(26.94015156,349.27134609)(27.46515319,350.83134748)(29.10015156,350.83134748)
\curveto(30.21015045,350.83134748)(30.79515172,350.20134643)(30.96015156,349.15134748)
\lineto(32.23515156,349.15134748)
}
}
{
\newrgbcolor{curcolor}{0 0 0}
\pscustom[linestyle=none,fillstyle=solid,fillcolor=curcolor]
{
\newpath
\moveto(38.89515156,346.40634748)
\curveto(38.8501516,345.82134807)(38.11515031,344.86134748)(36.87015156,344.86134748)
\curveto(35.35515307,344.86134748)(34.59015156,345.80634912)(34.59015156,347.44134748)
\lineto(40.32015156,347.44134748)
\curveto(40.32015156,350.21634471)(39.21014929,352.01634748)(36.94515156,352.01634748)
\curveto(34.35015415,352.01634748)(33.18015156,350.08134505)(33.18015156,347.65134748)
\curveto(33.18015156,345.38634975)(34.48515376,343.72134748)(36.69015156,343.72134748)
\curveto(37.9501503,343.72134748)(38.46015192,344.02134772)(38.82015156,344.26134748)
\curveto(39.81015057,344.92134682)(40.1701516,346.03134786)(40.21515156,346.40634748)
\lineto(38.89515156,346.40634748)
\moveto(34.59015156,348.49134748)
\curveto(34.59015156,349.70634627)(35.55015277,350.83134748)(36.76515156,350.83134748)
\curveto(38.37014995,350.83134748)(38.88015163,349.70634627)(38.95515156,348.49134748)
\lineto(34.59015156,348.49134748)
}
}
{
\newrgbcolor{curcolor}{0 0 0}
\pscustom[linestyle=none,fillstyle=solid,fillcolor=curcolor]
{
\newpath
\moveto(48.29476093,349.28634748)
\curveto(48.29476093,351.52134525)(46.76475972,352.01634748)(45.54976093,352.01634748)
\curveto(44.19976228,352.01634748)(43.46476065,351.10134706)(43.17976093,350.68134748)
\lineto(43.14976093,350.68134748)
\lineto(43.14976093,351.79134748)
\lineto(41.90476093,351.79134748)
\lineto(41.90476093,343.94634748)
\lineto(43.22476093,343.94634748)
\lineto(43.22476093,348.22134748)
\curveto(43.22476093,350.35134535)(44.54476168,350.83134748)(45.29476093,350.83134748)
\curveto(46.58475964,350.83134748)(46.97476093,350.14134612)(46.97476093,348.77634748)
\lineto(46.97476093,343.94634748)
\lineto(48.29476093,343.94634748)
\lineto(48.29476093,349.28634748)
}
}
{
\newrgbcolor{curcolor}{0 0 0}
\pscustom[linestyle=none,fillstyle=solid,fillcolor=curcolor]
{
\newpath
\moveto(53.13437031,350.69634748)
\lineto(53.13437031,351.79134748)
\lineto(51.87437031,351.79134748)
\lineto(51.87437031,353.98134748)
\lineto(50.55437031,353.98134748)
\lineto(50.55437031,351.79134748)
\lineto(49.48937031,351.79134748)
\lineto(49.48937031,350.69634748)
\lineto(50.55437031,350.69634748)
\lineto(50.55437031,345.52134748)
\curveto(50.55437031,344.57634843)(50.83937161,343.84134748)(52.14437031,343.84134748)
\curveto(52.27937017,343.84134748)(52.65437079,343.90134753)(53.13437031,343.94634748)
\lineto(53.13437031,344.98134748)
\lineto(52.66937031,344.98134748)
\curveto(52.39937058,344.98134748)(51.87437031,344.9813481)(51.87437031,345.59634748)
\lineto(51.87437031,350.69634748)
\lineto(53.13437031,350.69634748)
}
}
{
\newrgbcolor{curcolor}{0 0 0}
\pscustom[linestyle=none,fillstyle=solid,fillcolor=curcolor]
{
\newpath
\moveto(59.46155781,346.40634748)
\curveto(59.41655785,345.82134807)(58.68155656,344.86134748)(57.43655781,344.86134748)
\curveto(55.92155932,344.86134748)(55.15655781,345.80634912)(55.15655781,347.44134748)
\lineto(60.88655781,347.44134748)
\curveto(60.88655781,350.21634471)(59.77655554,352.01634748)(57.51155781,352.01634748)
\curveto(54.9165604,352.01634748)(53.74655781,350.08134505)(53.74655781,347.65134748)
\curveto(53.74655781,345.38634975)(55.05156001,343.72134748)(57.25655781,343.72134748)
\curveto(58.51655655,343.72134748)(59.02655817,344.02134772)(59.38655781,344.26134748)
\curveto(60.37655682,344.92134682)(60.73655785,346.03134786)(60.78155781,346.40634748)
\lineto(59.46155781,346.40634748)
\moveto(55.15655781,348.49134748)
\curveto(55.15655781,349.70634627)(56.11655902,350.83134748)(57.33155781,350.83134748)
\curveto(58.9365562,350.83134748)(59.44655788,349.70634627)(59.52155781,348.49134748)
\lineto(55.15655781,348.49134748)
}
}
{
\newrgbcolor{curcolor}{0 0 0}
\pscustom[linestyle=none,fillstyle=solid,fillcolor=curcolor]
{
\newpath
\moveto(7.10569843,360.93658186)
\lineto(11.41069843,360.93658186)
\curveto(14.95069489,360.93658186)(16.00069843,364.05658427)(16.00069843,366.47158186)
\curveto(16.00069843,369.57657875)(14.27569563,371.70658186)(11.47069843,371.70658186)
\lineto(7.10569843,371.70658186)
\lineto(7.10569843,360.93658186)
\moveto(8.56069843,370.46158186)
\lineto(11.27569843,370.46158186)
\curveto(13.25569645,370.46158186)(14.50069843,369.09657914)(14.50069843,366.38158186)
\curveto(14.50069843,363.66658457)(13.27069654,362.18158186)(11.38069843,362.18158186)
\lineto(8.56069843,362.18158186)
\lineto(8.56069843,370.46158186)
}
}
{
\newrgbcolor{curcolor}{0 0 0}
\pscustom[linestyle=none,fillstyle=solid,fillcolor=curcolor]
{
\newpath
\moveto(17.25554218,364.86658186)
\curveto(17.25554218,362.84158388)(18.39554469,360.72658186)(20.90054218,360.72658186)
\curveto(23.40553968,360.72658186)(24.54554218,362.84158388)(24.54554218,364.86658186)
\curveto(24.54554218,366.89157983)(23.40553968,369.00658186)(20.90054218,369.00658186)
\curveto(18.39554469,369.00658186)(17.25554218,366.89157983)(17.25554218,364.86658186)
\moveto(18.62054218,364.86658186)
\curveto(18.62054218,365.91658081)(19.01054407,367.86658186)(20.90054218,367.86658186)
\curveto(22.79054029,367.86658186)(23.18054218,365.91658081)(23.18054218,364.86658186)
\curveto(23.18054218,363.81658291)(22.79054029,361.86658186)(20.90054218,361.86658186)
\curveto(19.01054407,361.86658186)(18.62054218,363.81658291)(18.62054218,364.86658186)
}
}
{
\newrgbcolor{curcolor}{0 0 0}
\pscustom[linestyle=none,fillstyle=solid,fillcolor=curcolor]
{
\newpath
\moveto(32.23515156,366.14158186)
\curveto(32.13015166,367.50658049)(31.3501495,369.00658186)(29.29515156,369.00658186)
\curveto(26.70015415,369.00658186)(25.53015156,367.07157943)(25.53015156,364.64158186)
\curveto(25.53015156,362.37658412)(26.83515376,360.71158186)(29.04015156,360.71158186)
\curveto(31.33514926,360.71158186)(32.10015169,362.4665831)(32.23515156,363.71158186)
\lineto(30.96015156,363.71158186)
\curveto(30.73515178,362.51158306)(29.97015067,361.85158186)(29.08515156,361.85158186)
\curveto(27.27015337,361.85158186)(26.94015156,363.51658321)(26.94015156,364.86658186)
\curveto(26.94015156,366.26158046)(27.46515319,367.82158186)(29.10015156,367.82158186)
\curveto(30.21015045,367.82158186)(30.79515172,367.19158081)(30.96015156,366.14158186)
\lineto(32.23515156,366.14158186)
}
}
{
\newrgbcolor{curcolor}{0 0 0}
\pscustom[linestyle=none,fillstyle=solid,fillcolor=curcolor]
{
\newpath
\moveto(38.89515156,363.39658186)
\curveto(38.8501516,362.81158244)(38.11515031,361.85158186)(36.87015156,361.85158186)
\curveto(35.35515307,361.85158186)(34.59015156,362.79658349)(34.59015156,364.43158186)
\lineto(40.32015156,364.43158186)
\curveto(40.32015156,367.20657908)(39.21014929,369.00658186)(36.94515156,369.00658186)
\curveto(34.35015415,369.00658186)(33.18015156,367.07157943)(33.18015156,364.64158186)
\curveto(33.18015156,362.37658412)(34.48515376,360.71158186)(36.69015156,360.71158186)
\curveto(37.9501503,360.71158186)(38.46015192,361.0115821)(38.82015156,361.25158186)
\curveto(39.81015057,361.9115812)(40.1701516,363.02158223)(40.21515156,363.39658186)
\lineto(38.89515156,363.39658186)
\moveto(34.59015156,365.48158186)
\curveto(34.59015156,366.69658064)(35.55015277,367.82158186)(36.76515156,367.82158186)
\curveto(38.37014995,367.82158186)(38.88015163,366.69658064)(38.95515156,365.48158186)
\lineto(34.59015156,365.48158186)
}
}
{
\newrgbcolor{curcolor}{0 0 0}
\pscustom[linestyle=none,fillstyle=solid,fillcolor=curcolor]
{
\newpath
\moveto(48.29476093,366.27658186)
\curveto(48.29476093,368.51157962)(46.76475972,369.00658186)(45.54976093,369.00658186)
\curveto(44.19976228,369.00658186)(43.46476065,368.09158144)(43.17976093,367.67158186)
\lineto(43.14976093,367.67158186)
\lineto(43.14976093,368.78158186)
\lineto(41.90476093,368.78158186)
\lineto(41.90476093,360.93658186)
\lineto(43.22476093,360.93658186)
\lineto(43.22476093,365.21158186)
\curveto(43.22476093,367.34157973)(44.54476168,367.82158186)(45.29476093,367.82158186)
\curveto(46.58475964,367.82158186)(46.97476093,367.13158049)(46.97476093,365.76658186)
\lineto(46.97476093,360.93658186)
\lineto(48.29476093,360.93658186)
\lineto(48.29476093,366.27658186)
}
}
{
\newrgbcolor{curcolor}{0 0 0}
\pscustom[linestyle=none,fillstyle=solid,fillcolor=curcolor]
{
\newpath
\moveto(53.13437031,367.68658186)
\lineto(53.13437031,368.78158186)
\lineto(51.87437031,368.78158186)
\lineto(51.87437031,370.97158186)
\lineto(50.55437031,370.97158186)
\lineto(50.55437031,368.78158186)
\lineto(49.48937031,368.78158186)
\lineto(49.48937031,367.68658186)
\lineto(50.55437031,367.68658186)
\lineto(50.55437031,362.51158186)
\curveto(50.55437031,361.5665828)(50.83937161,360.83158186)(52.14437031,360.83158186)
\curveto(52.27937017,360.83158186)(52.65437079,360.8915819)(53.13437031,360.93658186)
\lineto(53.13437031,361.97158186)
\lineto(52.66937031,361.97158186)
\curveto(52.39937058,361.97158186)(51.87437031,361.97158247)(51.87437031,362.58658186)
\lineto(51.87437031,367.68658186)
\lineto(53.13437031,367.68658186)
}
}
{
\newrgbcolor{curcolor}{0 0 0}
\pscustom[linestyle=none,fillstyle=solid,fillcolor=curcolor]
{
\newpath
\moveto(59.46155781,363.39658186)
\curveto(59.41655785,362.81158244)(58.68155656,361.85158186)(57.43655781,361.85158186)
\curveto(55.92155932,361.85158186)(55.15655781,362.79658349)(55.15655781,364.43158186)
\lineto(60.88655781,364.43158186)
\curveto(60.88655781,367.20657908)(59.77655554,369.00658186)(57.51155781,369.00658186)
\curveto(54.9165604,369.00658186)(53.74655781,367.07157943)(53.74655781,364.64158186)
\curveto(53.74655781,362.37658412)(55.05156001,360.71158186)(57.25655781,360.71158186)
\curveto(58.51655655,360.71158186)(59.02655817,361.0115821)(59.38655781,361.25158186)
\curveto(60.37655682,361.9115812)(60.73655785,363.02158223)(60.78155781,363.39658186)
\lineto(59.46155781,363.39658186)
\moveto(55.15655781,365.48158186)
\curveto(55.15655781,366.69658064)(56.11655902,367.82158186)(57.33155781,367.82158186)
\curveto(58.9365562,367.82158186)(59.44655788,366.69658064)(59.52155781,365.48158186)
\lineto(55.15655781,365.48158186)
}
}
{
\newrgbcolor{curcolor}{0 0 0}
\pscustom[linestyle=none,fillstyle=solid,fillcolor=curcolor]
{
\newpath
\moveto(7.10569843,89.09283186)
\lineto(11.41069843,89.09283186)
\curveto(14.95069489,89.09283186)(16.00069843,92.21283427)(16.00069843,94.62783186)
\curveto(16.00069843,97.73282875)(14.27569563,99.86283186)(11.47069843,99.86283186)
\lineto(7.10569843,99.86283186)
\lineto(7.10569843,89.09283186)
\moveto(8.56069843,98.61783186)
\lineto(11.27569843,98.61783186)
\curveto(13.25569645,98.61783186)(14.50069843,97.25282914)(14.50069843,94.53783186)
\curveto(14.50069843,91.82283457)(13.27069654,90.33783186)(11.38069843,90.33783186)
\lineto(8.56069843,90.33783186)
\lineto(8.56069843,98.61783186)
}
}
{
\newrgbcolor{curcolor}{0 0 0}
\pscustom[linestyle=none,fillstyle=solid,fillcolor=curcolor]
{
\newpath
\moveto(17.25554218,93.02283186)
\curveto(17.25554218,90.99783388)(18.39554469,88.88283186)(20.90054218,88.88283186)
\curveto(23.40553968,88.88283186)(24.54554218,90.99783388)(24.54554218,93.02283186)
\curveto(24.54554218,95.04782983)(23.40553968,97.16283186)(20.90054218,97.16283186)
\curveto(18.39554469,97.16283186)(17.25554218,95.04782983)(17.25554218,93.02283186)
\moveto(18.62054218,93.02283186)
\curveto(18.62054218,94.07283081)(19.01054407,96.02283186)(20.90054218,96.02283186)
\curveto(22.79054029,96.02283186)(23.18054218,94.07283081)(23.18054218,93.02283186)
\curveto(23.18054218,91.97283291)(22.79054029,90.02283186)(20.90054218,90.02283186)
\curveto(19.01054407,90.02283186)(18.62054218,91.97283291)(18.62054218,93.02283186)
}
}
{
\newrgbcolor{curcolor}{0 0 0}
\pscustom[linestyle=none,fillstyle=solid,fillcolor=curcolor]
{
\newpath
\moveto(32.23515156,94.29783186)
\curveto(32.13015166,95.66283049)(31.3501495,97.16283186)(29.29515156,97.16283186)
\curveto(26.70015415,97.16283186)(25.53015156,95.22782943)(25.53015156,92.79783186)
\curveto(25.53015156,90.53283412)(26.83515376,88.86783186)(29.04015156,88.86783186)
\curveto(31.33514926,88.86783186)(32.10015169,90.6228331)(32.23515156,91.86783186)
\lineto(30.96015156,91.86783186)
\curveto(30.73515178,90.66783306)(29.97015067,90.00783186)(29.08515156,90.00783186)
\curveto(27.27015337,90.00783186)(26.94015156,91.67283321)(26.94015156,93.02283186)
\curveto(26.94015156,94.41783046)(27.46515319,95.97783186)(29.10015156,95.97783186)
\curveto(30.21015045,95.97783186)(30.79515172,95.34783081)(30.96015156,94.29783186)
\lineto(32.23515156,94.29783186)
}
}
{
\newrgbcolor{curcolor}{0 0 0}
\pscustom[linestyle=none,fillstyle=solid,fillcolor=curcolor]
{
\newpath
\moveto(38.89515156,91.55283186)
\curveto(38.8501516,90.96783244)(38.11515031,90.00783186)(36.87015156,90.00783186)
\curveto(35.35515307,90.00783186)(34.59015156,90.95283349)(34.59015156,92.58783186)
\lineto(40.32015156,92.58783186)
\curveto(40.32015156,95.36282908)(39.21014929,97.16283186)(36.94515156,97.16283186)
\curveto(34.35015415,97.16283186)(33.18015156,95.22782943)(33.18015156,92.79783186)
\curveto(33.18015156,90.53283412)(34.48515376,88.86783186)(36.69015156,88.86783186)
\curveto(37.9501503,88.86783186)(38.46015192,89.1678321)(38.82015156,89.40783186)
\curveto(39.81015057,90.0678312)(40.1701516,91.17783223)(40.21515156,91.55283186)
\lineto(38.89515156,91.55283186)
\moveto(34.59015156,93.63783186)
\curveto(34.59015156,94.85283064)(35.55015277,95.97783186)(36.76515156,95.97783186)
\curveto(38.37014995,95.97783186)(38.88015163,94.85283064)(38.95515156,93.63783186)
\lineto(34.59015156,93.63783186)
}
}
{
\newrgbcolor{curcolor}{0 0 0}
\pscustom[linestyle=none,fillstyle=solid,fillcolor=curcolor]
{
\newpath
\moveto(48.29476093,94.43283186)
\curveto(48.29476093,96.66782962)(46.76475972,97.16283186)(45.54976093,97.16283186)
\curveto(44.19976228,97.16283186)(43.46476065,96.24783144)(43.17976093,95.82783186)
\lineto(43.14976093,95.82783186)
\lineto(43.14976093,96.93783186)
\lineto(41.90476093,96.93783186)
\lineto(41.90476093,89.09283186)
\lineto(43.22476093,89.09283186)
\lineto(43.22476093,93.36783186)
\curveto(43.22476093,95.49782973)(44.54476168,95.97783186)(45.29476093,95.97783186)
\curveto(46.58475964,95.97783186)(46.97476093,95.28783049)(46.97476093,93.92283186)
\lineto(46.97476093,89.09283186)
\lineto(48.29476093,89.09283186)
\lineto(48.29476093,94.43283186)
}
}
{
\newrgbcolor{curcolor}{0 0 0}
\pscustom[linestyle=none,fillstyle=solid,fillcolor=curcolor]
{
\newpath
\moveto(53.13437031,95.84283186)
\lineto(53.13437031,96.93783186)
\lineto(51.87437031,96.93783186)
\lineto(51.87437031,99.12783186)
\lineto(50.55437031,99.12783186)
\lineto(50.55437031,96.93783186)
\lineto(49.48937031,96.93783186)
\lineto(49.48937031,95.84283186)
\lineto(50.55437031,95.84283186)
\lineto(50.55437031,90.66783186)
\curveto(50.55437031,89.7228328)(50.83937161,88.98783186)(52.14437031,88.98783186)
\curveto(52.27937017,88.98783186)(52.65437079,89.0478319)(53.13437031,89.09283186)
\lineto(53.13437031,90.12783186)
\lineto(52.66937031,90.12783186)
\curveto(52.39937058,90.12783186)(51.87437031,90.12783247)(51.87437031,90.74283186)
\lineto(51.87437031,95.84283186)
\lineto(53.13437031,95.84283186)
}
}
{
\newrgbcolor{curcolor}{0 0 0}
\pscustom[linestyle=none,fillstyle=solid,fillcolor=curcolor]
{
\newpath
\moveto(59.46155781,91.55283186)
\curveto(59.41655785,90.96783244)(58.68155656,90.00783186)(57.43655781,90.00783186)
\curveto(55.92155932,90.00783186)(55.15655781,90.95283349)(55.15655781,92.58783186)
\lineto(60.88655781,92.58783186)
\curveto(60.88655781,95.36282908)(59.77655554,97.16283186)(57.51155781,97.16283186)
\curveto(54.9165604,97.16283186)(53.74655781,95.22782943)(53.74655781,92.79783186)
\curveto(53.74655781,90.53283412)(55.05156001,88.86783186)(57.25655781,88.86783186)
\curveto(58.51655655,88.86783186)(59.02655817,89.1678321)(59.38655781,89.40783186)
\curveto(60.37655682,90.0678312)(60.73655785,91.17783223)(60.78155781,91.55283186)
\lineto(59.46155781,91.55283186)
\moveto(55.15655781,93.63783186)
\curveto(55.15655781,94.85283064)(56.11655902,95.97783186)(57.33155781,95.97783186)
\curveto(58.9365562,95.97783186)(59.44655788,94.85283064)(59.52155781,93.63783186)
\lineto(55.15655781,93.63783186)
}
}
{
\newrgbcolor{curcolor}{0 0 0}
\pscustom[linestyle=none,fillstyle=solid,fillcolor=curcolor]
{
\newpath
\moveto(14.17069843,160.18876936)
\lineto(15.23569843,157.05376936)
\lineto(16.82569843,157.05376936)
\lineto(12.92569843,167.82376936)
\lineto(11.27569843,167.82376936)
\lineto(7.22569843,157.05376936)
\lineto(8.72569843,157.05376936)
\lineto(9.85069843,160.18876936)
\lineto(14.17069843,160.18876936)
\moveto(10.30069843,161.47876936)
\lineto(12.02569843,166.21876936)
\lineto(12.05569843,166.21876936)
\lineto(13.64569843,161.47876936)
\lineto(10.30069843,161.47876936)
}
}
{
\newrgbcolor{curcolor}{0 0 0}
\pscustom[linestyle=none,fillstyle=solid,fillcolor=curcolor]
{
\newpath
\moveto(23.93077656,157.05376936)
\lineto(23.93077656,164.89876936)
\lineto(22.61077656,164.89876936)
\lineto(22.61077656,160.57876936)
\curveto(22.61077656,159.4387705)(22.11577489,157.96876936)(20.45077656,157.96876936)
\curveto(19.59577741,157.96876936)(18.93577656,158.40377065)(18.93577656,159.69376936)
\lineto(18.93577656,164.89876936)
\lineto(17.61577656,164.89876936)
\lineto(17.61577656,159.25876936)
\curveto(17.61577656,157.38377123)(19.01077771,156.82876936)(20.16577656,156.82876936)
\curveto(21.4257753,156.82876936)(22.10077711,157.30877027)(22.65577656,158.22376936)
\lineto(22.68577656,158.19376936)
\lineto(22.68577656,157.05376936)
\lineto(23.93077656,157.05376936)
}
}
{
\newrgbcolor{curcolor}{0 0 0}
\pscustom[linestyle=none,fillstyle=solid,fillcolor=curcolor]
{
\newpath
\moveto(29.53538593,161.08876936)
\lineto(32.19038593,164.89876936)
\lineto(30.57038593,164.89876936)
\lineto(28.75538593,162.13876936)
\lineto(26.94038593,164.89876936)
\lineto(25.24538593,164.89876936)
\lineto(27.87038593,161.08876936)
\lineto(25.11038593,157.05376936)
\lineto(26.77538593,157.05376936)
\lineto(28.66538593,160.00876936)
\lineto(30.61538593,157.05376936)
\lineto(32.29538593,157.05376936)
\lineto(29.53538593,161.08876936)
}
}
{
\newrgbcolor{curcolor}{0 0 0}
\pscustom[linestyle=none,fillstyle=solid,fillcolor=curcolor]
{
\newpath
\moveto(34.77038593,164.89876936)
\lineto(33.45038593,164.89876936)
\lineto(33.45038593,157.05376936)
\lineto(34.77038593,157.05376936)
\lineto(34.77038593,164.89876936)
\moveto(34.77038593,166.32376936)
\lineto(34.77038593,167.82376936)
\lineto(33.45038593,167.82376936)
\lineto(33.45038593,166.32376936)
\lineto(34.77038593,166.32376936)
}
}
{
\newrgbcolor{curcolor}{0 0 0}
\pscustom[linestyle=none,fillstyle=solid,fillcolor=curcolor]
{
\newpath
\moveto(38.11022968,167.82376936)
\lineto(36.79022968,167.82376936)
\lineto(36.79022968,157.05376936)
\lineto(38.11022968,157.05376936)
\lineto(38.11022968,167.82376936)
}
}
{
\newrgbcolor{curcolor}{0 0 0}
\pscustom[linestyle=none,fillstyle=solid,fillcolor=curcolor]
{
\newpath
\moveto(41.45007343,164.89876936)
\lineto(40.13007343,164.89876936)
\lineto(40.13007343,157.05376936)
\lineto(41.45007343,157.05376936)
\lineto(41.45007343,164.89876936)
\moveto(41.45007343,166.32376936)
\lineto(41.45007343,167.82376936)
\lineto(40.13007343,167.82376936)
\lineto(40.13007343,166.32376936)
\lineto(41.45007343,166.32376936)
}
}
{
\newrgbcolor{curcolor}{0 0 0}
\pscustom[linestyle=none,fillstyle=solid,fillcolor=curcolor]
{
\newpath
\moveto(44.60991718,162.51376936)
\curveto(44.69991709,163.11376876)(44.90991868,164.02876936)(46.40991718,164.02876936)
\curveto(47.65491594,164.02876936)(48.25491718,163.57876853)(48.25491718,162.75376936)
\curveto(48.25491718,161.97377014)(47.87991687,161.85376933)(47.56491718,161.82376936)
\lineto(45.38991718,161.55376936)
\curveto(43.19991937,161.28376963)(43.00491718,159.7537687)(43.00491718,159.09376936)
\curveto(43.00491718,157.74377071)(44.02491862,156.82876936)(45.46491718,156.82876936)
\curveto(46.99491565,156.82876936)(47.78991769,157.54876991)(48.29991718,158.10376936)
\curveto(48.34491714,157.50376996)(48.52491835,156.90376936)(49.69491718,156.90376936)
\curveto(49.99491688,156.90376936)(50.18991741,156.99376942)(50.41491718,157.05376936)
\lineto(50.41491718,158.01376936)
\curveto(50.26491733,157.98376939)(50.09991706,157.95376936)(49.97991718,157.95376936)
\curveto(49.70991745,157.95376936)(49.54491718,158.08876969)(49.54491718,158.41876936)
\lineto(49.54491718,162.93376936)
\curveto(49.54491718,164.94376735)(47.26491655,165.12376936)(46.63491718,165.12376936)
\curveto(44.69991912,165.12376936)(43.45491712,164.38876748)(43.39491718,162.51376936)
\lineto(44.60991718,162.51376936)
\moveto(48.22491718,159.76876936)
\curveto(48.22491718,158.71877041)(47.02491595,157.92376936)(45.79491718,157.92376936)
\curveto(44.80491817,157.92376936)(44.36991718,158.43377021)(44.36991718,159.28876936)
\curveto(44.36991718,160.27876837)(45.40491783,160.47376945)(46.04991718,160.56376936)
\curveto(47.68491555,160.77376915)(48.01491739,160.89376952)(48.22491718,161.05876936)
\lineto(48.22491718,159.76876936)
}
}
{
\newrgbcolor{curcolor}{0 0 0}
\pscustom[linestyle=none,fillstyle=solid,fillcolor=curcolor]
{
\newpath
\moveto(53.28952656,161.61376936)
\curveto(53.28952656,162.75376822)(54.06952779,163.71376936)(55.29952656,163.71376936)
\lineto(55.79452656,163.71376936)
\lineto(55.79452656,165.07876936)
\curveto(55.68952666,165.10876933)(55.61452639,165.12376936)(55.44952656,165.12376936)
\curveto(54.45952755,165.12376936)(53.76952603,164.50876844)(53.24452656,163.59376936)
\lineto(53.21452656,163.59376936)
\lineto(53.21452656,164.89876936)
\lineto(51.96952656,164.89876936)
\lineto(51.96952656,157.05376936)
\lineto(53.28952656,157.05376936)
\lineto(53.28952656,161.61376936)
}
}
{
\newrgbcolor{curcolor}{0 0 0}
\pscustom[linestyle=none,fillstyle=solid,fillcolor=curcolor]
{
\newpath
\moveto(14.17069843,262.13005354)
\lineto(15.23569843,258.99505354)
\lineto(16.82569843,258.99505354)
\lineto(12.92569843,269.76505354)
\lineto(11.27569843,269.76505354)
\lineto(7.22569843,258.99505354)
\lineto(8.72569843,258.99505354)
\lineto(9.85069843,262.13005354)
\lineto(14.17069843,262.13005354)
\moveto(10.30069843,263.42005354)
\lineto(12.02569843,268.16005354)
\lineto(12.05569843,268.16005354)
\lineto(13.64569843,263.42005354)
\lineto(10.30069843,263.42005354)
}
}
{
\newrgbcolor{curcolor}{0 0 0}
\pscustom[linestyle=none,fillstyle=solid,fillcolor=curcolor]
{
\newpath
\moveto(23.93077656,258.99505354)
\lineto(23.93077656,266.84005354)
\lineto(22.61077656,266.84005354)
\lineto(22.61077656,262.52005354)
\curveto(22.61077656,261.38005468)(22.11577489,259.91005354)(20.45077656,259.91005354)
\curveto(19.59577741,259.91005354)(18.93577656,260.34505483)(18.93577656,261.63505354)
\lineto(18.93577656,266.84005354)
\lineto(17.61577656,266.84005354)
\lineto(17.61577656,261.20005354)
\curveto(17.61577656,259.32505541)(19.01077771,258.77005354)(20.16577656,258.77005354)
\curveto(21.4257753,258.77005354)(22.10077711,259.25005445)(22.65577656,260.16505354)
\lineto(22.68577656,260.13505354)
\lineto(22.68577656,258.99505354)
\lineto(23.93077656,258.99505354)
}
}
{
\newrgbcolor{curcolor}{0 0 0}
\pscustom[linestyle=none,fillstyle=solid,fillcolor=curcolor]
{
\newpath
\moveto(29.53538593,263.03005354)
\lineto(32.19038593,266.84005354)
\lineto(30.57038593,266.84005354)
\lineto(28.75538593,264.08005354)
\lineto(26.94038593,266.84005354)
\lineto(25.24538593,266.84005354)
\lineto(27.87038593,263.03005354)
\lineto(25.11038593,258.99505354)
\lineto(26.77538593,258.99505354)
\lineto(28.66538593,261.95005354)
\lineto(30.61538593,258.99505354)
\lineto(32.29538593,258.99505354)
\lineto(29.53538593,263.03005354)
}
}
{
\newrgbcolor{curcolor}{0 0 0}
\pscustom[linestyle=none,fillstyle=solid,fillcolor=curcolor]
{
\newpath
\moveto(34.77038593,266.84005354)
\lineto(33.45038593,266.84005354)
\lineto(33.45038593,258.99505354)
\lineto(34.77038593,258.99505354)
\lineto(34.77038593,266.84005354)
\moveto(34.77038593,268.26505354)
\lineto(34.77038593,269.76505354)
\lineto(33.45038593,269.76505354)
\lineto(33.45038593,268.26505354)
\lineto(34.77038593,268.26505354)
}
}
{
\newrgbcolor{curcolor}{0 0 0}
\pscustom[linestyle=none,fillstyle=solid,fillcolor=curcolor]
{
\newpath
\moveto(38.11022968,269.76505354)
\lineto(36.79022968,269.76505354)
\lineto(36.79022968,258.99505354)
\lineto(38.11022968,258.99505354)
\lineto(38.11022968,269.76505354)
}
}
{
\newrgbcolor{curcolor}{0 0 0}
\pscustom[linestyle=none,fillstyle=solid,fillcolor=curcolor]
{
\newpath
\moveto(41.45007343,266.84005354)
\lineto(40.13007343,266.84005354)
\lineto(40.13007343,258.99505354)
\lineto(41.45007343,258.99505354)
\lineto(41.45007343,266.84005354)
\moveto(41.45007343,268.26505354)
\lineto(41.45007343,269.76505354)
\lineto(40.13007343,269.76505354)
\lineto(40.13007343,268.26505354)
\lineto(41.45007343,268.26505354)
}
}
{
\newrgbcolor{curcolor}{0 0 0}
\pscustom[linestyle=none,fillstyle=solid,fillcolor=curcolor]
{
\newpath
\moveto(44.60991718,264.45505354)
\curveto(44.69991709,265.05505294)(44.90991868,265.97005354)(46.40991718,265.97005354)
\curveto(47.65491594,265.97005354)(48.25491718,265.52005271)(48.25491718,264.69505354)
\curveto(48.25491718,263.91505432)(47.87991687,263.79505351)(47.56491718,263.76505354)
\lineto(45.38991718,263.49505354)
\curveto(43.19991937,263.22505381)(43.00491718,261.69505288)(43.00491718,261.03505354)
\curveto(43.00491718,259.68505489)(44.02491862,258.77005354)(45.46491718,258.77005354)
\curveto(46.99491565,258.77005354)(47.78991769,259.49005409)(48.29991718,260.04505354)
\curveto(48.34491714,259.44505414)(48.52491835,258.84505354)(49.69491718,258.84505354)
\curveto(49.99491688,258.84505354)(50.18991741,258.9350536)(50.41491718,258.99505354)
\lineto(50.41491718,259.95505354)
\curveto(50.26491733,259.92505357)(50.09991706,259.89505354)(49.97991718,259.89505354)
\curveto(49.70991745,259.89505354)(49.54491718,260.03005387)(49.54491718,260.36005354)
\lineto(49.54491718,264.87505354)
\curveto(49.54491718,266.88505153)(47.26491655,267.06505354)(46.63491718,267.06505354)
\curveto(44.69991912,267.06505354)(43.45491712,266.33005166)(43.39491718,264.45505354)
\lineto(44.60991718,264.45505354)
\moveto(48.22491718,261.71005354)
\curveto(48.22491718,260.66005459)(47.02491595,259.86505354)(45.79491718,259.86505354)
\curveto(44.80491817,259.86505354)(44.36991718,260.37505439)(44.36991718,261.23005354)
\curveto(44.36991718,262.22005255)(45.40491783,262.41505363)(46.04991718,262.50505354)
\curveto(47.68491555,262.71505333)(48.01491739,262.8350537)(48.22491718,263.00005354)
\lineto(48.22491718,261.71005354)
}
}
{
\newrgbcolor{curcolor}{0 0 0}
\pscustom[linestyle=none,fillstyle=solid,fillcolor=curcolor]
{
\newpath
\moveto(53.28952656,263.55505354)
\curveto(53.28952656,264.6950524)(54.06952779,265.65505354)(55.29952656,265.65505354)
\lineto(55.79452656,265.65505354)
\lineto(55.79452656,267.02005354)
\curveto(55.68952666,267.05005351)(55.61452639,267.06505354)(55.44952656,267.06505354)
\curveto(54.45952755,267.06505354)(53.76952603,266.45005262)(53.24452656,265.53505354)
\lineto(53.21452656,265.53505354)
\lineto(53.21452656,266.84005354)
\lineto(51.96952656,266.84005354)
\lineto(51.96952656,258.99505354)
\lineto(53.28952656,258.99505354)
\lineto(53.28952656,263.55505354)
}
}
{
\newrgbcolor{curcolor}{0 0 0}
\pscustom[linestyle=none,fillstyle=solid,fillcolor=curcolor]
{
\newpath
\moveto(17.41070942,292.97564436)
\lineto(17.41070942,303.74564436)
\lineto(15.34070942,303.74564436)
\lineto(12.28070942,294.64064436)
\lineto(12.25070942,294.64064436)
\lineto(9.17570942,303.74564436)
\lineto(7.09070942,303.74564436)
\lineto(7.09070942,292.97564436)
\lineto(8.50070942,292.97564436)
\lineto(8.50070942,299.33564436)
\curveto(8.50070942,299.65064404)(8.47070942,301.01564535)(8.47070942,302.00564436)
\lineto(8.50070942,302.00564436)
\lineto(11.53070942,292.97564436)
\lineto(12.97070942,292.97564436)
\lineto(16.00070942,302.02064436)
\lineto(16.03070942,302.02064436)
\curveto(16.03070942,301.01564536)(16.00070942,299.65064404)(16.00070942,299.33564436)
\lineto(16.00070942,292.97564436)
\lineto(17.41070942,292.97564436)
}
}
{
\newrgbcolor{curcolor}{0 0 0}
\pscustom[linestyle=none,fillstyle=solid,fillcolor=curcolor]
{
\newpath
\moveto(19.03047504,296.90564436)
\curveto(19.03047504,294.88064638)(20.17047755,292.76564436)(22.67547504,292.76564436)
\curveto(25.18047254,292.76564436)(26.32047504,294.88064638)(26.32047504,296.90564436)
\curveto(26.32047504,298.93064233)(25.18047254,301.04564436)(22.67547504,301.04564436)
\curveto(20.17047755,301.04564436)(19.03047504,298.93064233)(19.03047504,296.90564436)
\moveto(20.39547504,296.90564436)
\curveto(20.39547504,297.95564331)(20.78547693,299.90564436)(22.67547504,299.90564436)
\curveto(24.56547315,299.90564436)(24.95547504,297.95564331)(24.95547504,296.90564436)
\curveto(24.95547504,295.85564541)(24.56547315,293.90564436)(22.67547504,293.90564436)
\curveto(20.78547693,293.90564436)(20.39547504,295.85564541)(20.39547504,296.90564436)
}
}
{
\newrgbcolor{curcolor}{0 0 0}
\pscustom[linestyle=none,fillstyle=solid,fillcolor=curcolor]
{
\newpath
\moveto(34.34008442,303.74564436)
\lineto(33.02008442,303.74564436)
\lineto(33.02008442,299.81564436)
\lineto(32.99008442,299.71064436)
\curveto(32.67508473,300.16064391)(32.07508299,301.04564436)(30.65008442,301.04564436)
\curveto(28.5650865,301.04564436)(27.38008442,299.33564215)(27.38008442,297.13064436)
\curveto(27.38008442,295.25564623)(28.16008709,292.75064436)(30.83008442,292.75064436)
\curveto(31.59508365,292.75064436)(32.49508499,292.99064542)(33.06508442,294.05564436)
\lineto(33.09508442,294.05564436)
\lineto(33.09508442,292.97564436)
\lineto(34.34008442,292.97564436)
\lineto(34.34008442,303.74564436)
\moveto(28.74508442,296.92064436)
\curveto(28.74508442,297.92564335)(28.85008646,299.86064436)(30.89008442,299.86064436)
\curveto(32.79508251,299.86064436)(33.00508442,297.80564308)(33.00508442,296.53064436)
\curveto(33.00508442,294.44564644)(31.70008358,293.89064436)(30.86008442,293.89064436)
\curveto(29.42008586,293.89064436)(28.74508442,295.19564608)(28.74508442,296.92064436)
}
}
{
\newrgbcolor{curcolor}{0 0 0}
\pscustom[linestyle=none,fillstyle=solid,fillcolor=curcolor]
{
\newpath
\moveto(41.51969379,295.43564436)
\curveto(41.47469384,294.85064494)(40.73969255,293.89064436)(39.49469379,293.89064436)
\curveto(37.97969531,293.89064436)(37.21469379,294.83564599)(37.21469379,296.47064436)
\lineto(42.94469379,296.47064436)
\curveto(42.94469379,299.24564158)(41.83469153,301.04564436)(39.56969379,301.04564436)
\curveto(36.97469639,301.04564436)(35.80469379,299.11064193)(35.80469379,296.68064436)
\curveto(35.80469379,294.41564662)(37.109696,292.75064436)(39.31469379,292.75064436)
\curveto(40.57469253,292.75064436)(41.08469415,293.0506446)(41.44469379,293.29064436)
\curveto(42.4346928,293.9506437)(42.79469384,295.06064473)(42.83969379,295.43564436)
\lineto(41.51969379,295.43564436)
\moveto(37.21469379,297.52064436)
\curveto(37.21469379,298.73564314)(38.17469501,299.86064436)(39.38969379,299.86064436)
\curveto(40.99469219,299.86064436)(41.50469387,298.73564314)(41.57969379,297.52064436)
\lineto(37.21469379,297.52064436)
}
}
{
\newrgbcolor{curcolor}{0 0 0}
\pscustom[linestyle=none,fillstyle=solid,fillcolor=curcolor]
{
\newpath
\moveto(46.02930317,297.53564436)
\curveto(46.02930317,298.67564322)(46.8093044,299.63564436)(48.03930317,299.63564436)
\lineto(48.53430317,299.63564436)
\lineto(48.53430317,301.00064436)
\curveto(48.42930327,301.03064433)(48.354303,301.04564436)(48.18930317,301.04564436)
\curveto(47.19930416,301.04564436)(46.50930264,300.43064344)(45.98430317,299.51564436)
\lineto(45.95430317,299.51564436)
\lineto(45.95430317,300.82064436)
\lineto(44.70930317,300.82064436)
\lineto(44.70930317,292.97564436)
\lineto(46.02930317,292.97564436)
\lineto(46.02930317,297.53564436)
}
}
{
\newrgbcolor{curcolor}{0 0 0}
\pscustom[linestyle=none,fillstyle=solid,fillcolor=curcolor]
{
\newpath
\moveto(50.56258442,298.43564436)
\curveto(50.65258433,299.03564376)(50.86258592,299.95064436)(52.36258442,299.95064436)
\curveto(53.60758317,299.95064436)(54.20758442,299.50064353)(54.20758442,298.67564436)
\curveto(54.20758442,297.89564514)(53.8325841,297.77564433)(53.51758442,297.74564436)
\lineto(51.34258442,297.47564436)
\curveto(49.15258661,297.20564463)(48.95758442,295.6756437)(48.95758442,295.01564436)
\curveto(48.95758442,293.66564571)(49.97758586,292.75064436)(51.41758442,292.75064436)
\curveto(52.94758289,292.75064436)(53.74258493,293.47064491)(54.25258442,294.02564436)
\curveto(54.29758437,293.42564496)(54.47758559,292.82564436)(55.64758442,292.82564436)
\curveto(55.94758412,292.82564436)(56.14258464,292.91564442)(56.36758442,292.97564436)
\lineto(56.36758442,293.93564436)
\curveto(56.21758457,293.90564439)(56.0525843,293.87564436)(55.93258442,293.87564436)
\curveto(55.66258469,293.87564436)(55.49758442,294.01064469)(55.49758442,294.34064436)
\lineto(55.49758442,298.85564436)
\curveto(55.49758442,300.86564235)(53.21758379,301.04564436)(52.58758442,301.04564436)
\curveto(50.65258635,301.04564436)(49.40758436,300.31064248)(49.34758442,298.43564436)
\lineto(50.56258442,298.43564436)
\moveto(54.17758442,295.69064436)
\curveto(54.17758442,294.64064541)(52.97758319,293.84564436)(51.74758442,293.84564436)
\curveto(50.75758541,293.84564436)(50.32258442,294.35564521)(50.32258442,295.21064436)
\curveto(50.32258442,296.20064337)(51.35758506,296.39564445)(52.00258442,296.48564436)
\curveto(53.63758278,296.69564415)(53.96758463,296.81564452)(54.17758442,296.98064436)
\lineto(54.17758442,295.69064436)
}
}
{
\newrgbcolor{curcolor}{0 0 0}
\pscustom[linestyle=none,fillstyle=solid,fillcolor=curcolor]
{
\newpath
\moveto(64.25219379,303.74564436)
\lineto(62.93219379,303.74564436)
\lineto(62.93219379,299.81564436)
\lineto(62.90219379,299.71064436)
\curveto(62.58719411,300.16064391)(61.98719237,301.04564436)(60.56219379,301.04564436)
\curveto(58.47719588,301.04564436)(57.29219379,299.33564215)(57.29219379,297.13064436)
\curveto(57.29219379,295.25564623)(58.07219646,292.75064436)(60.74219379,292.75064436)
\curveto(61.50719303,292.75064436)(62.40719436,292.99064542)(62.97719379,294.05564436)
\lineto(63.00719379,294.05564436)
\lineto(63.00719379,292.97564436)
\lineto(64.25219379,292.97564436)
\lineto(64.25219379,303.74564436)
\moveto(58.65719379,296.92064436)
\curveto(58.65719379,297.92564335)(58.76219583,299.86064436)(60.80219379,299.86064436)
\curveto(62.70719189,299.86064436)(62.91719379,297.80564308)(62.91719379,296.53064436)
\curveto(62.91719379,294.44564644)(61.61219295,293.89064436)(60.77219379,293.89064436)
\curveto(59.33219523,293.89064436)(58.65719379,295.19564608)(58.65719379,296.92064436)
}
}
{
\newrgbcolor{curcolor}{0 0 0}
\pscustom[linestyle=none,fillstyle=solid,fillcolor=curcolor]
{
\newpath
\moveto(65.64180317,296.90564436)
\curveto(65.64180317,294.88064638)(66.78180567,292.76564436)(69.28680317,292.76564436)
\curveto(71.79180066,292.76564436)(72.93180317,294.88064638)(72.93180317,296.90564436)
\curveto(72.93180317,298.93064233)(71.79180066,301.04564436)(69.28680317,301.04564436)
\curveto(66.78180567,301.04564436)(65.64180317,298.93064233)(65.64180317,296.90564436)
\moveto(67.00680317,296.90564436)
\curveto(67.00680317,297.95564331)(67.39680506,299.90564436)(69.28680317,299.90564436)
\curveto(71.17680128,299.90564436)(71.56680317,297.95564331)(71.56680317,296.90564436)
\curveto(71.56680317,295.85564541)(71.17680128,293.90564436)(69.28680317,293.90564436)
\curveto(67.39680506,293.90564436)(67.00680317,295.85564541)(67.00680317,296.90564436)
}
}
{
\newrgbcolor{curcolor}{0 0 0}
\pscustom[linestyle=none,fillstyle=solid,fillcolor=curcolor]
{
\newpath
\moveto(75.94141254,297.53564436)
\curveto(75.94141254,298.67564322)(76.72141377,299.63564436)(77.95141254,299.63564436)
\lineto(78.44641254,299.63564436)
\lineto(78.44641254,301.00064436)
\curveto(78.34141265,301.03064433)(78.26641238,301.04564436)(78.10141254,301.04564436)
\curveto(77.11141353,301.04564436)(76.42141202,300.43064344)(75.89641254,299.51564436)
\lineto(75.86641254,299.51564436)
\lineto(75.86641254,300.82064436)
\lineto(74.62141254,300.82064436)
\lineto(74.62141254,292.97564436)
\lineto(75.94141254,292.97564436)
\lineto(75.94141254,297.53564436)
}
}
{
\newrgbcolor{curcolor}{0 0 0}
\pscustom[linestyle=none,fillstyle=solid,fillcolor=curcolor]
{
\newpath
\moveto(111.68000793,541.71457502)
\lineto(113.00000793,541.71457502)
\lineto(113.00000793,534.18457502)
\curveto(113.00000586,533.39457263)(113.13500573,532.80957321)(113.40500793,532.42957502)
\curveto(113.68500518,532.05957396)(114.15500471,531.87457415)(114.81500793,531.87457502)
\curveto(115.51500335,531.87457415)(115.99500287,532.06957395)(116.25500793,532.45957502)
\curveto(116.51500235,532.85957316)(116.64500222,533.43457259)(116.64500793,534.18457502)
\lineto(116.64500793,541.71457502)
\lineto(117.96500793,541.71457502)
\lineto(117.96500793,534.18457502)
\curveto(117.9650009,533.15457287)(117.70000116,532.32957369)(117.17000793,531.70957502)
\curveto(116.65000221,531.09957492)(115.865003,530.79457523)(114.81500793,530.79457502)
\curveto(113.72500514,530.79457523)(112.93000593,531.07957494)(112.43000793,531.64957502)
\curveto(111.93000693,532.2195738)(111.68000718,533.06457296)(111.68000793,534.18457502)
\lineto(111.68000793,541.71457502)
}
}
{
\newrgbcolor{curcolor}{0 0 0}
\pscustom[linestyle=none,fillstyle=solid,fillcolor=curcolor]
{
\newpath
\moveto(119.59297668,541.71457502)
\lineto(120.89797668,541.71457502)
\lineto(121.30297668,541.71457502)
\lineto(124.84297668,532.77457502)
\lineto(124.87297668,532.77457502)
\lineto(124.87297668,541.71457502)
\lineto(126.19297668,541.71457502)
\lineto(126.19297668,531.00457502)
\lineto(124.88797668,531.00457502)
\lineto(124.37797668,531.00457502)
\lineto(120.94297668,539.67457502)
\lineto(120.91297668,539.67457502)
\lineto(120.91297668,531.00457502)
\lineto(119.59297668,531.00457502)
\lineto(119.59297668,541.71457502)
}
}
{
\newrgbcolor{curcolor}{0 0 0}
\pscustom[linestyle=none,fillstyle=solid,fillcolor=curcolor]
{
\newpath
\moveto(144.96510803,541.50455305)
\lineto(150.50010803,541.50455305)
\lineto(150.50010803,540.33455305)
\lineto(146.28510803,540.33455305)
\lineto(146.28510803,536.97455305)
\lineto(150.26010803,536.97455305)
\lineto(150.26010803,535.80455305)
\lineto(146.28510803,535.80455305)
\lineto(146.28510803,531.96455305)
\lineto(150.68010803,531.96455305)
\lineto(150.68010803,530.79455305)
\lineto(144.96510803,530.79455305)
\lineto(144.96510803,541.50455305)
}
}
{
\newrgbcolor{curcolor}{0 0 0}
\pscustom[linestyle=none,fillstyle=solid,fillcolor=curcolor]
{
\newpath
\moveto(151.93846741,541.50455305)
\lineto(154.15846741,541.50455305)
\lineto(156.30346741,533.01455305)
\lineto(156.33346741,533.01455305)
\lineto(158.47846741,541.50455305)
\lineto(160.69846741,541.50455305)
\lineto(160.69846741,530.79455305)
\lineto(159.37846741,530.79455305)
\lineto(159.37846741,540.15455305)
\lineto(159.34846741,540.15455305)
\lineto(156.97846741,530.79455305)
\lineto(155.65846741,530.79455305)
\lineto(153.28846741,540.15455305)
\lineto(153.25846741,540.15455305)
\lineto(153.25846741,530.79455305)
\lineto(151.93846741,530.79455305)
\lineto(151.93846741,541.50455305)
}
}
{
\newrgbcolor{curcolor}{0 0 0}
\pscustom[linestyle=none,fillstyle=solid,fillcolor=curcolor]
{
\newpath
\moveto(181.9155304,534.81455305)
\lineto(184.7955304,534.81455305)
\lineto(183.4305304,539.98955305)
\lineto(183.4005304,539.98955305)
\lineto(181.9155304,534.81455305)
\moveto(182.5605304,541.50455305)
\lineto(184.3305304,541.50455305)
\lineto(187.2105304,530.79455305)
\lineto(185.8305304,530.79455305)
\lineto(185.0655304,533.73455305)
\lineto(181.6455304,533.73455305)
\lineto(180.8505304,530.79455305)
\lineto(179.4705304,530.79455305)
\lineto(182.5605304,541.50455305)
}
}
{
\newrgbcolor{curcolor}{0 0 0}
\pscustom[linestyle=none,fillstyle=solid,fillcolor=curcolor]
{
\newpath
\moveto(186.76756165,541.50455305)
\lineto(188.14756165,541.50455305)
\lineto(190.24756165,532.30955305)
\lineto(190.27756165,532.30955305)
\lineto(192.37756165,541.50455305)
\lineto(193.75756165,541.50455305)
\lineto(191.05756165,530.79455305)
\lineto(189.37756165,530.79455305)
\lineto(186.76756165,541.50455305)
}
}
{
\newrgbcolor{curcolor}{0 0 0}
\pscustom[linestyle=none,fillstyle=solid,fillcolor=curcolor]
{
\newpath
\moveto(214.97467346,535.02457502)
\lineto(217.85467346,535.02457502)
\lineto(216.48967346,540.19957502)
\lineto(216.45967346,540.19957502)
\lineto(214.97467346,535.02457502)
\moveto(215.61967346,541.71457502)
\lineto(217.38967346,541.71457502)
\lineto(220.26967346,531.00457502)
\lineto(218.88967346,531.00457502)
\lineto(218.12467346,533.94457502)
\lineto(214.70467346,533.94457502)
\lineto(213.90967346,531.00457502)
\lineto(212.52967346,531.00457502)
\lineto(215.61967346,541.71457502)
}
}
{
\newrgbcolor{curcolor}{0 0 0}
\pscustom[linestyle=none,fillstyle=solid,fillcolor=curcolor]
{
\newpath
\moveto(227.57467346,534.67957502)
\curveto(227.54466607,534.16957185)(227.46466615,533.67957234)(227.33467346,533.20957502)
\curveto(227.2146664,532.74957327)(227.02966658,532.33957368)(226.77967346,531.97957502)
\curveto(226.52966708,531.6195744)(226.19966741,531.32957469)(225.78967346,531.10957502)
\curveto(225.38966822,530.89957512)(224.89966871,530.79457523)(224.31967346,530.79457502)
\curveto(223.55967005,530.79457523)(222.94967066,530.95457507)(222.48967346,531.27457502)
\curveto(222.03967157,531.60457442)(221.69467192,532.03457399)(221.45467346,532.56457502)
\curveto(221.2146724,533.09457293)(221.05467256,533.68957233)(220.97467346,534.34957502)
\curveto(220.90467271,535.00957101)(220.86967274,535.67957034)(220.86967346,536.35957502)
\curveto(220.86967274,537.02956899)(220.9096727,537.69456833)(220.98967346,538.35457502)
\curveto(221.06967254,539.024567)(221.23467238,539.6245664)(221.48467346,540.15457502)
\curveto(221.73467188,540.68456534)(222.08467153,541.10956491)(222.53467346,541.42957502)
\curveto(222.98467063,541.75956426)(223.57967003,541.9245641)(224.31967346,541.92457502)
\curveto(225.4096682,541.9245641)(226.19966741,541.63456439)(226.68967346,541.05457502)
\curveto(227.18966642,540.47456555)(227.45466616,539.65456637)(227.48467346,538.59457502)
\lineto(226.10467346,538.59457502)
\curveto(226.09466752,538.89456713)(226.05966755,539.17956684)(225.99967346,539.44957502)
\curveto(225.93966767,539.72956629)(225.83966777,539.96956605)(225.69967346,540.16957502)
\curveto(225.56966804,540.37956564)(225.38966822,540.54456548)(225.15967346,540.66457502)
\curveto(224.93966867,540.78456524)(224.65966895,540.84456518)(224.31967346,540.84457502)
\curveto(223.85966975,540.84456518)(223.49467012,540.7245653)(223.22467346,540.48457502)
\curveto(222.95467066,540.25456577)(222.74467087,539.93456609)(222.59467346,539.52457502)
\curveto(222.45467116,539.1245669)(222.35967125,538.64956737)(222.30967346,538.09957502)
\curveto(222.26967134,537.55956846)(222.24967136,536.97956904)(222.24967346,536.35957502)
\curveto(222.24967136,535.73957028)(222.26967134,535.15457087)(222.30967346,534.60457502)
\curveto(222.35967125,534.06457196)(222.45467116,533.58957243)(222.59467346,533.17957502)
\curveto(222.74467087,532.77957324)(222.95467066,532.45957356)(223.22467346,532.21957502)
\curveto(223.49467012,531.98957403)(223.85966975,531.87457415)(224.31967346,531.87457502)
\curveto(224.71966889,531.87457415)(225.03966857,531.95957406)(225.27967346,532.12957502)
\curveto(225.51966809,532.29957372)(225.70466791,532.5195735)(225.83467346,532.78957502)
\curveto(225.97466764,533.05957296)(226.06466755,533.35957266)(226.10467346,533.68957502)
\curveto(226.15466746,534.019572)(226.18466743,534.34957167)(226.19467346,534.67957502)
\lineto(227.57467346,534.67957502)
}
}
{
\newrgbcolor{curcolor}{0 0 0}
\pscustom[linestyle=none,fillstyle=solid,fillcolor=curcolor]
{
\newpath
\moveto(247.66676086,536.40455305)
\lineto(249.24176086,536.40455305)
\curveto(249.72175662,536.40454744)(250.1367562,536.57454727)(250.48676086,536.91455305)
\curveto(250.8367555,537.25454659)(251.01175533,537.77954606)(251.01176086,538.48955305)
\curveto(251.01175533,539.08954475)(250.86675547,539.55954428)(250.57676086,539.89955305)
\curveto(250.28675605,540.24954359)(249.81175653,540.42454342)(249.15176086,540.42455305)
\lineto(247.66676086,540.42455305)
\lineto(247.66676086,536.40455305)
\moveto(246.34676086,541.50455305)
\lineto(249.07676086,541.50455305)
\curveto(249.22675711,541.50454234)(249.41175693,541.49954234)(249.63176086,541.48955305)
\curveto(249.86175648,541.47954236)(250.09675624,541.4445424)(250.33676086,541.38455305)
\curveto(250.58675575,541.33454251)(250.83175551,541.2445426)(251.07176086,541.11455305)
\curveto(251.32175502,540.99454285)(251.5417548,540.81954302)(251.73176086,540.58955305)
\curveto(251.93175441,540.35954348)(252.09175425,540.06954377)(252.21176086,539.71955305)
\curveto(252.33175401,539.36954447)(252.39175395,538.9395449)(252.39176086,538.42955305)
\curveto(252.39175395,537.92954591)(252.31675402,537.48454636)(252.16676086,537.09455305)
\curveto(252.01675432,536.71454713)(251.80175454,536.38954745)(251.52176086,536.11955305)
\curveto(251.25175509,535.85954798)(250.92675541,535.65954818)(250.54676086,535.51955305)
\curveto(250.16675617,535.38954845)(249.75175659,535.32454852)(249.30176086,535.32455305)
\lineto(247.66676086,535.32455305)
\lineto(247.66676086,530.79455305)
\lineto(246.34676086,530.79455305)
\lineto(246.34676086,541.50455305)
}
}
{
\newrgbcolor{curcolor}{0 0 0}
\pscustom[linestyle=none,fillstyle=solid,fillcolor=curcolor]
{
\newpath
\moveto(254.46679993,534.81455305)
\lineto(257.34679993,534.81455305)
\lineto(255.98179993,539.98955305)
\lineto(255.95179993,539.98955305)
\lineto(254.46679993,534.81455305)
\moveto(255.11179993,541.50455305)
\lineto(256.88179993,541.50455305)
\lineto(259.76179993,530.79455305)
\lineto(258.38179993,530.79455305)
\lineto(257.61679993,533.73455305)
\lineto(254.19679993,533.73455305)
\lineto(253.40179993,530.79455305)
\lineto(252.02179993,530.79455305)
\lineto(255.11179993,541.50455305)
}
}
{
\newrgbcolor{curcolor}{0 0 0}
\pscustom[linestyle=none,fillstyle=solid,fillcolor=curcolor]
{
\newpath
\moveto(279.85390991,536.61457502)
\lineto(281.42890991,536.61457502)
\curveto(281.90890567,536.61456941)(282.32390525,536.78456924)(282.67390991,537.12457502)
\curveto(283.02390455,537.46456856)(283.19890438,537.98956803)(283.19890991,538.69957502)
\curveto(283.19890438,539.29956672)(283.05390452,539.76956625)(282.76390991,540.10957502)
\curveto(282.4739051,540.45956556)(281.99890558,540.63456539)(281.33890991,540.63457502)
\lineto(279.85390991,540.63457502)
\lineto(279.85390991,536.61457502)
\moveto(278.53390991,541.71457502)
\lineto(281.26390991,541.71457502)
\curveto(281.41390616,541.71456431)(281.59890598,541.70956431)(281.81890991,541.69957502)
\curveto(282.04890553,541.68956433)(282.28390529,541.65456437)(282.52390991,541.59457502)
\curveto(282.7739048,541.54456448)(283.01890456,541.45456457)(283.25890991,541.32457502)
\curveto(283.50890407,541.20456482)(283.72890385,541.02956499)(283.91890991,540.79957502)
\curveto(284.11890346,540.56956545)(284.2789033,540.27956574)(284.39890991,539.92957502)
\curveto(284.51890306,539.57956644)(284.578903,539.14956687)(284.57890991,538.63957502)
\curveto(284.578903,538.13956788)(284.50390307,537.69456833)(284.35390991,537.30457502)
\curveto(284.20390337,536.9245691)(283.98890359,536.59956942)(283.70890991,536.32957502)
\curveto(283.43890414,536.06956995)(283.11390446,535.86957015)(282.73390991,535.72957502)
\curveto(282.35390522,535.59957042)(281.93890564,535.53457049)(281.48890991,535.53457502)
\lineto(279.85390991,535.53457502)
\lineto(279.85390991,531.00457502)
\lineto(278.53390991,531.00457502)
\lineto(278.53390991,541.71457502)
}
}
{
\newrgbcolor{curcolor}{0 0 0}
\pscustom[linestyle=none,fillstyle=solid,fillcolor=curcolor]
{
\newpath
\moveto(289.04094116,540.84457502)
\curveto(288.58093745,540.84456518)(288.21593782,540.7245653)(287.94594116,540.48457502)
\curveto(287.67593836,540.25456577)(287.46593857,539.93456609)(287.31594116,539.52457502)
\curveto(287.17593886,539.1245669)(287.08093895,538.64956737)(287.03094116,538.09957502)
\curveto(286.99093904,537.55956846)(286.97093906,536.97956904)(286.97094116,536.35957502)
\curveto(286.97093906,535.73957028)(286.99093904,535.15457087)(287.03094116,534.60457502)
\curveto(287.08093895,534.06457196)(287.17593886,533.58957243)(287.31594116,533.17957502)
\curveto(287.46593857,532.77957324)(287.67593836,532.45957356)(287.94594116,532.21957502)
\curveto(288.21593782,531.98957403)(288.58093745,531.87457415)(289.04094116,531.87457502)
\curveto(289.50093653,531.87457415)(289.86593617,531.98957403)(290.13594116,532.21957502)
\curveto(290.40593563,532.45957356)(290.61093542,532.77957324)(290.75094116,533.17957502)
\curveto(290.90093513,533.58957243)(290.99593504,534.06457196)(291.03594116,534.60457502)
\curveto(291.08593495,535.15457087)(291.11093492,535.73957028)(291.11094116,536.35957502)
\curveto(291.11093492,536.97956904)(291.08593495,537.55956846)(291.03594116,538.09957502)
\curveto(290.99593504,538.64956737)(290.90093513,539.1245669)(290.75094116,539.52457502)
\curveto(290.61093542,539.93456609)(290.40593563,540.25456577)(290.13594116,540.48457502)
\curveto(289.86593617,540.7245653)(289.50093653,540.84456518)(289.04094116,540.84457502)
\moveto(289.04094116,541.92457502)
\curveto(289.78093625,541.9245641)(290.37593566,541.75956426)(290.82594116,541.42957502)
\curveto(291.27593476,541.10956491)(291.62593441,540.68456534)(291.87594116,540.15457502)
\curveto(292.12593391,539.6245664)(292.29093374,539.024567)(292.37094116,538.35457502)
\curveto(292.45093358,537.69456833)(292.49093354,537.02956899)(292.49094116,536.35957502)
\curveto(292.49093354,535.67957034)(292.45093358,535.00957101)(292.37094116,534.34957502)
\curveto(292.29093374,533.68957233)(292.12593391,533.09457293)(291.87594116,532.56457502)
\curveto(291.62593441,532.03457399)(291.27593476,531.60457442)(290.82594116,531.27457502)
\curveto(290.37593566,530.95457507)(289.78093625,530.79457523)(289.04094116,530.79457502)
\curveto(288.30093773,530.79457523)(287.70593833,530.95457507)(287.25594116,531.27457502)
\curveto(286.80593923,531.60457442)(286.45593958,532.03457399)(286.20594116,532.56457502)
\curveto(285.95594008,533.09457293)(285.79094024,533.68957233)(285.71094116,534.34957502)
\curveto(285.6309404,535.00957101)(285.59094044,535.67957034)(285.59094116,536.35957502)
\curveto(285.59094044,537.02956899)(285.6309404,537.69456833)(285.71094116,538.35457502)
\curveto(285.79094024,539.024567)(285.95594008,539.6245664)(286.20594116,540.15457502)
\curveto(286.45593958,540.68456534)(286.80593923,541.10956491)(287.25594116,541.42957502)
\curveto(287.70593833,541.75956426)(288.30093773,541.9245641)(289.04094116,541.92457502)
}
}
{
\newrgbcolor{curcolor}{0 0 0}
\pscustom[linestyle=none,fillstyle=solid,fillcolor=curcolor]
{
\newpath
\moveto(316.21300415,538.93957502)
\curveto(316.21299874,539.2195668)(316.18299877,539.47456655)(316.12300415,539.70457502)
\curveto(316.07299888,539.94456608)(315.98299897,540.14456588)(315.85300415,540.30457502)
\curveto(315.72299923,540.47456555)(315.5479994,540.60456542)(315.32800415,540.69457502)
\curveto(315.11799983,540.79456523)(314.85800009,540.84456518)(314.54800415,540.84457502)
\curveto(313.97800097,540.84456518)(313.53800141,540.69456533)(313.22800415,540.39457502)
\curveto(312.92800202,540.10456592)(312.77800217,539.67456635)(312.77800415,539.10457502)
\curveto(312.77800217,538.60456742)(312.90300205,538.2195678)(313.15300415,537.94957502)
\curveto(313.40300155,537.67956834)(313.71300124,537.46456856)(314.08300415,537.30457502)
\curveto(314.46300049,537.14456888)(314.86800008,536.99956902)(315.29800415,536.86957502)
\curveto(315.73799921,536.74956927)(316.14299881,536.58456944)(316.51300415,536.37457502)
\curveto(316.89299806,536.16456986)(317.20799774,535.87457015)(317.45800415,535.50457502)
\curveto(317.70799724,535.13457089)(317.83299712,534.6245714)(317.83300415,533.97457502)
\curveto(317.83299712,533.35457267)(317.72799722,532.83957318)(317.51800415,532.42957502)
\curveto(317.31799763,532.019574)(317.05799789,531.69457433)(316.73800415,531.45457502)
\curveto(316.41799853,531.21457481)(316.05799889,531.04457498)(315.65800415,530.94457502)
\curveto(315.26799968,530.84457518)(314.88300007,530.79457523)(314.50300415,530.79457502)
\curveto(313.87300108,530.79457523)(313.3480016,530.87457515)(312.92800415,531.03457502)
\curveto(312.51800243,531.19457483)(312.18800276,531.4245746)(311.93800415,531.72457502)
\curveto(311.69800325,532.024574)(311.52300343,532.39457363)(311.41300415,532.83457502)
\curveto(311.31300364,533.28457274)(311.26300369,533.79457223)(311.26300415,534.36457502)
\lineto(312.58300415,534.36457502)
\curveto(312.58300237,534.06457196)(312.59800235,533.76457226)(312.62800415,533.46457502)
\curveto(312.65800229,533.17457285)(312.73800221,532.90957311)(312.86800415,532.66957502)
\curveto(312.99800195,532.42957359)(313.19800175,532.23457379)(313.46800415,532.08457502)
\curveto(313.73800121,531.94457408)(314.11300084,531.87457415)(314.59300415,531.87457502)
\curveto(314.8530001,531.87457415)(315.09799985,531.9195741)(315.32800415,532.00957502)
\curveto(315.55799939,532.09957392)(315.7529992,532.2245738)(315.91300415,532.38457502)
\curveto(316.08299887,532.55457347)(316.21299874,532.75457327)(316.30300415,532.98457502)
\curveto(316.40299855,533.21457281)(316.4529985,533.47457255)(316.45300415,533.76457502)
\curveto(316.4529985,534.14457188)(316.37799857,534.45457157)(316.22800415,534.69457502)
\curveto(316.08799886,534.93457109)(315.89799905,535.13457089)(315.65800415,535.29457502)
\curveto(315.42799952,535.46457056)(315.15799979,535.59957042)(314.84800415,535.69957502)
\curveto(314.5480004,535.80957021)(314.23800071,535.9195701)(313.91800415,536.02957502)
\curveto(313.60800134,536.13956988)(313.29800165,536.25956976)(312.98800415,536.38957502)
\curveto(312.68800226,536.52956949)(312.41800253,536.70456932)(312.17800415,536.91457502)
\curveto(311.948003,537.13456889)(311.75800319,537.40956861)(311.60800415,537.73957502)
\curveto(311.46800348,538.06956795)(311.39800355,538.47956754)(311.39800415,538.96957502)
\curveto(311.39800355,539.2195668)(311.43300352,539.50956651)(311.50300415,539.83957502)
\curveto(311.57300338,540.17956584)(311.71800323,540.50456552)(311.93800415,540.81457502)
\curveto(312.16800278,541.1245649)(312.48800246,541.38456464)(312.89800415,541.59457502)
\curveto(313.30800164,541.81456421)(313.8530011,541.9245641)(314.53300415,541.92457502)
\curveto(315.56299939,541.9245641)(316.31299864,541.67456435)(316.78300415,541.17457502)
\curveto(317.26299769,540.68456534)(317.51299744,539.93956608)(317.53300415,538.93957502)
\lineto(316.21300415,538.93957502)
}
}
{
\newrgbcolor{curcolor}{0 0 0}
\pscustom[linestyle=none,fillstyle=solid,fillcolor=curcolor]
{
\newpath
\moveto(322.46800415,540.84457502)
\curveto(322.00800044,540.84456518)(321.64300081,540.7245653)(321.37300415,540.48457502)
\curveto(321.10300135,540.25456577)(320.89300156,539.93456609)(320.74300415,539.52457502)
\curveto(320.60300185,539.1245669)(320.50800194,538.64956737)(320.45800415,538.09957502)
\curveto(320.41800203,537.55956846)(320.39800205,536.97956904)(320.39800415,536.35957502)
\curveto(320.39800205,535.73957028)(320.41800203,535.15457087)(320.45800415,534.60457502)
\curveto(320.50800194,534.06457196)(320.60300185,533.58957243)(320.74300415,533.17957502)
\curveto(320.89300156,532.77957324)(321.10300135,532.45957356)(321.37300415,532.21957502)
\curveto(321.64300081,531.98957403)(322.00800044,531.87457415)(322.46800415,531.87457502)
\curveto(322.92799952,531.87457415)(323.29299916,531.98957403)(323.56300415,532.21957502)
\curveto(323.83299862,532.45957356)(324.03799841,532.77957324)(324.17800415,533.17957502)
\curveto(324.32799812,533.58957243)(324.42299803,534.06457196)(324.46300415,534.60457502)
\curveto(324.51299794,535.15457087)(324.53799791,535.73957028)(324.53800415,536.35957502)
\curveto(324.53799791,536.97956904)(324.51299794,537.55956846)(324.46300415,538.09957502)
\curveto(324.42299803,538.64956737)(324.32799812,539.1245669)(324.17800415,539.52457502)
\curveto(324.03799841,539.93456609)(323.83299862,540.25456577)(323.56300415,540.48457502)
\curveto(323.29299916,540.7245653)(322.92799952,540.84456518)(322.46800415,540.84457502)
\moveto(322.46800415,541.92457502)
\curveto(323.20799924,541.9245641)(323.80299865,541.75956426)(324.25300415,541.42957502)
\curveto(324.70299775,541.10956491)(325.0529974,540.68456534)(325.30300415,540.15457502)
\curveto(325.5529969,539.6245664)(325.71799673,539.024567)(325.79800415,538.35457502)
\curveto(325.87799657,537.69456833)(325.91799653,537.02956899)(325.91800415,536.35957502)
\curveto(325.91799653,535.67957034)(325.87799657,535.00957101)(325.79800415,534.34957502)
\curveto(325.71799673,533.68957233)(325.5529969,533.09457293)(325.30300415,532.56457502)
\curveto(325.0529974,532.03457399)(324.70299775,531.60457442)(324.25300415,531.27457502)
\curveto(323.80299865,530.95457507)(323.20799924,530.79457523)(322.46800415,530.79457502)
\curveto(321.72800072,530.79457523)(321.13300132,530.95457507)(320.68300415,531.27457502)
\curveto(320.23300222,531.60457442)(319.88300257,532.03457399)(319.63300415,532.56457502)
\curveto(319.38300307,533.09457293)(319.21800323,533.68957233)(319.13800415,534.34957502)
\curveto(319.05800339,535.00957101)(319.01800343,535.67957034)(319.01800415,536.35957502)
\curveto(319.01800343,537.02956899)(319.05800339,537.69456833)(319.13800415,538.35457502)
\curveto(319.21800323,539.024567)(319.38300307,539.6245664)(319.63300415,540.15457502)
\curveto(319.88300257,540.68456534)(320.23300222,541.10956491)(320.68300415,541.42957502)
\curveto(321.13300132,541.75956426)(321.72800072,541.9245641)(322.46800415,541.92457502)
}
}
{
\newrgbcolor{curcolor}{0 0 0}
\pscustom[linestyle=none,fillstyle=solid,fillcolor=curcolor]
{
\newpath
\moveto(250.0216217,518.91546369)
\curveto(250.59583533,518.76311541)(251.03528802,518.49163131)(251.33998108,518.10101057)
\curveto(251.64466241,517.71428834)(251.79700601,517.22991382)(251.79701233,516.64788557)
\curveto(251.79700601,515.84319646)(251.5255219,515.21038459)(250.9825592,514.74944807)
\curveto(250.44349174,514.29241676)(249.69544561,514.06390136)(248.73841858,514.06390119)
\curveto(248.33607197,514.06390136)(247.92591613,514.1010107)(247.50794983,514.17522932)
\curveto(247.08997946,514.24944805)(246.67982362,514.35686982)(246.27748108,514.49749494)
\lineto(246.27748108,515.67522932)
\curveto(246.67591738,515.46819683)(247.06849511,515.31390011)(247.45521545,515.21233869)
\curveto(247.84193184,515.11077531)(248.22669708,515.05999412)(248.60951233,515.05999494)
\curveto(249.25794605,515.05999412)(249.75599242,515.20647834)(250.10365295,515.49944807)
\curveto(250.45130423,515.79241526)(250.62513218,516.21428984)(250.62513733,516.76507307)
\curveto(250.62513218,517.27288253)(250.45130423,517.67522588)(250.10365295,517.97210432)
\curveto(249.75599242,518.27288153)(249.28528977,518.423272)(248.69154358,518.42327619)
\lineto(247.78919983,518.42327619)
\lineto(247.78919983,519.39593244)
\lineto(248.69154358,519.39593244)
\curveto(249.23450857,519.39592728)(249.65833627,519.51506779)(249.96302795,519.75335432)
\curveto(250.26771066,519.99162981)(250.42005426,520.32366073)(250.4200592,520.74944807)
\curveto(250.42005426,521.19865985)(250.27747628,521.54240951)(249.99232483,521.78069807)
\curveto(249.71107059,522.02287778)(249.30872725,522.14397141)(248.78529358,522.14397932)
\curveto(248.43763437,522.14397141)(248.07825973,522.10490895)(247.70716858,522.02679182)
\curveto(247.33607297,521.9486591)(246.94740148,521.83147172)(246.54115295,521.67522932)
\lineto(246.54115295,522.76507307)
\curveto(247.01380767,522.89006441)(247.43372912,522.98381432)(247.80091858,523.04632307)
\curveto(248.17200963,523.10881419)(248.5001343,523.14006416)(248.78529358,523.14007307)
\curveto(249.63685192,523.14006416)(250.31653874,522.92522063)(250.82435608,522.49554182)
\curveto(251.33606897,522.06975273)(251.59192809,521.50334705)(251.5919342,520.79632307)
\curveto(251.59192809,520.31584823)(251.4571626,519.91545801)(251.18763733,519.59515119)
\curveto(250.92200688,519.27483365)(250.5333354,519.04827138)(250.0216217,518.91546369)
}
}
{
\newrgbcolor{curcolor}{0 0 0}
\pscustom[linestyle=none,fillstyle=solid,fillcolor=curcolor]
{
\newpath
\moveto(255.52357483,518.62835432)
\curveto(255.523572,518.84319346)(255.59779067,519.02874015)(255.74623108,519.18499494)
\curveto(255.89857162,519.34123983)(256.08021207,519.41936476)(256.29115295,519.41936994)
\curveto(256.50989914,519.41936476)(256.69739895,519.34123983)(256.85365295,519.18499494)
\curveto(257.00989864,519.02874015)(257.08802356,518.84319346)(257.08802795,518.62835432)
\curveto(257.08802356,518.40960014)(257.00989864,518.22405345)(256.85365295,518.07171369)
\curveto(256.7013052,517.91936626)(256.51380538,517.84319446)(256.29115295,517.84319807)
\curveto(256.07239958,517.84319446)(255.88880601,517.91741313)(255.7403717,518.06585432)
\curveto(255.59583755,518.21428784)(255.523572,518.40178765)(255.52357483,518.62835432)
\moveto(256.3028717,522.20257307)
\curveto(255.7520874,522.2025651)(255.33997843,521.90569039)(255.06654358,521.31194807)
\curveto(254.79701023,520.71819158)(254.66224474,519.81389561)(254.6622467,518.59905744)
\curveto(254.66224474,517.38811679)(254.79701023,516.48577394)(255.06654358,515.89202619)
\curveto(255.33997843,515.29827513)(255.7520874,515.00140042)(256.3028717,515.00140119)
\curveto(256.85755504,515.00140042)(257.269664,515.29827513)(257.53919983,515.89202619)
\curveto(257.81263221,516.48577394)(257.94935082,517.38811679)(257.94935608,518.59905744)
\curveto(257.94935082,519.81389561)(257.81263221,520.71819158)(257.53919983,521.31194807)
\curveto(257.269664,521.90569039)(256.85755504,522.2025651)(256.3028717,522.20257307)
\moveto(256.3028717,523.14007307)
\curveto(257.23646091,523.14006416)(257.94153833,522.75725204)(258.41810608,521.99163557)
\curveto(258.89856862,521.22600357)(259.13880276,520.09514533)(259.1388092,518.59905744)
\curveto(259.13880276,517.10686707)(258.89856862,515.97796195)(258.41810608,515.21233869)
\curveto(257.94153833,514.44671348)(257.23646091,514.06390136)(256.3028717,514.06390119)
\curveto(255.36927528,514.06390136)(254.66419786,514.44671348)(254.18763733,515.21233869)
\curveto(253.71107381,515.97796195)(253.4727928,517.10686707)(253.47279358,518.59905744)
\curveto(253.4727928,520.09514533)(253.71107381,521.22600357)(254.18763733,521.99163557)
\curveto(254.66419786,522.75725204)(255.36927528,523.14006416)(256.3028717,523.14007307)
}
}
{
\newrgbcolor{curcolor}{0 0 0}
\pscustom[linestyle=none,fillstyle=solid,fillcolor=curcolor]
{
\newpath
\moveto(217.62138367,521.73345686)
\lineto(214.86161804,517.11040998)
\lineto(217.62138367,517.11040998)
\lineto(217.62138367,521.73345686)
\moveto(217.42802429,522.81158186)
\lineto(218.79911804,522.81158186)
\lineto(218.79911804,517.11040998)
\lineto(219.96513367,517.11040998)
\lineto(219.96513367,516.14947248)
\lineto(218.79911804,516.14947248)
\lineto(218.79911804,514.06353498)
\lineto(217.62138367,514.06353498)
\lineto(217.62138367,516.14947248)
\lineto(213.91239929,516.14947248)
\lineto(213.91239929,517.26861311)
\lineto(217.42802429,522.81158186)
}
}
{
\newrgbcolor{curcolor}{0 0 0}
\pscustom[linestyle=none,fillstyle=solid,fillcolor=curcolor]
{
\newpath
\moveto(222.11552429,515.05962873)
\lineto(223.95536804,515.05962873)
\lineto(223.95536804,521.74517561)
\lineto(221.97489929,521.29986311)
\lineto(221.97489929,522.37798811)
\lineto(223.94364929,522.81158186)
\lineto(225.12724304,522.81158186)
\lineto(225.12724304,515.05962873)
\lineto(226.94364929,515.05962873)
\lineto(226.94364929,514.06353498)
\lineto(222.11552429,514.06353498)
\lineto(222.11552429,515.05962873)
}
}
{
\newrgbcolor{curcolor}{0 0 0}
\pscustom[linestyle=none,fillstyle=solid,fillcolor=curcolor]
{
\newpath
\moveto(215.46513367,498.17779279)
\lineto(219.48466492,498.17779279)
\lineto(219.48466492,497.18169904)
\lineto(214.17021179,497.18169904)
\lineto(214.17021179,498.17779279)
\curveto(214.90067892,498.94732228)(215.53935016,499.6270091)(216.08622742,500.21685529)
\curveto(216.63309906,500.80669542)(217.01005181,501.22271063)(217.21708679,501.46490217)
\curveto(217.60770746,501.94145991)(217.87137908,502.32622515)(218.00810242,502.61919904)
\curveto(218.1448163,502.91606831)(218.21317561,503.21880238)(218.21318054,503.52740217)
\curveto(218.21317561,504.01567658)(218.0686445,504.3984887)(217.77958679,504.67583967)
\curveto(217.49442633,504.95317565)(217.10184859,505.09184738)(216.60185242,505.09185529)
\curveto(216.2463807,505.09184738)(215.8733342,505.02739432)(215.48271179,504.89849592)
\curveto(215.09208498,504.76958208)(214.67802289,504.57426978)(214.24052429,504.31255842)
\lineto(214.24052429,505.50787092)
\curveto(214.64286668,505.69926865)(215.03739753,505.84379976)(215.42411804,505.94146467)
\curveto(215.81474051,506.03911206)(216.19950575,506.08794014)(216.57841492,506.08794904)
\curveto(217.43387951,506.08794014)(218.12137883,505.85942474)(218.64091492,505.40240217)
\curveto(219.16434653,504.9492694)(219.42606502,504.35356687)(219.42607117,503.61529279)
\curveto(219.42606502,503.24028673)(219.33817448,502.86528711)(219.16239929,502.49029279)
\curveto(218.99051858,502.11528786)(218.70926886,501.70122577)(218.31864929,501.24810529)
\curveto(218.09989447,500.99419523)(217.78153542,500.64263308)(217.36357117,500.19341779)
\curveto(216.949505,499.74419648)(216.31669313,499.07232215)(215.46513367,498.17779279)
}
}
{
\newrgbcolor{curcolor}{0 0 0}
\pscustom[linestyle=none,fillstyle=solid,fillcolor=curcolor]
{
\newpath
\moveto(223.32841492,501.57623029)
\curveto(223.32841209,501.79106943)(223.40263076,501.97661612)(223.55107117,502.13287092)
\curveto(223.70341171,502.28911581)(223.88505216,502.36724073)(224.09599304,502.36724592)
\curveto(224.31473923,502.36724073)(224.50223904,502.28911581)(224.65849304,502.13287092)
\curveto(224.81473873,501.97661612)(224.89286365,501.79106943)(224.89286804,501.57623029)
\curveto(224.89286365,501.35747612)(224.81473873,501.17192943)(224.65849304,501.01958967)
\curveto(224.50614528,500.86724223)(224.31864547,500.79107043)(224.09599304,500.79107404)
\curveto(223.87723966,500.79107043)(223.6936461,500.86528911)(223.54521179,501.01373029)
\curveto(223.40067764,501.16216381)(223.32841209,501.34966363)(223.32841492,501.57623029)
\moveto(224.10771179,505.15044904)
\curveto(223.55692748,505.15044107)(223.14481852,504.85356637)(222.87138367,504.25982404)
\curveto(222.60185031,503.66606756)(222.46708482,502.76177159)(222.46708679,501.54693342)
\curveto(222.46708482,500.33599276)(222.60185031,499.43364992)(222.87138367,498.83990217)
\curveto(223.14481852,498.2461511)(223.55692748,497.9492764)(224.10771179,497.94927717)
\curveto(224.66239513,497.9492764)(225.07450409,498.2461511)(225.34403992,498.83990217)
\curveto(225.6174723,499.43364992)(225.75419091,500.33599276)(225.75419617,501.54693342)
\curveto(225.75419091,502.76177159)(225.6174723,503.66606756)(225.34403992,504.25982404)
\curveto(225.07450409,504.85356637)(224.66239513,505.15044107)(224.10771179,505.15044904)
\moveto(224.10771179,506.08794904)
\curveto(225.041301,506.08794014)(225.74637842,505.70512802)(226.22294617,504.93951154)
\curveto(226.70340871,504.17387955)(226.94364285,503.04302131)(226.94364929,501.54693342)
\curveto(226.94364285,500.05474304)(226.70340871,498.92583792)(226.22294617,498.16021467)
\curveto(225.74637842,497.39458946)(225.041301,497.01177734)(224.10771179,497.01177717)
\curveto(223.17411537,497.01177734)(222.46903795,497.39458946)(221.99247742,498.16021467)
\curveto(221.5159139,498.92583792)(221.27763289,500.05474304)(221.27763367,501.54693342)
\curveto(221.27763289,503.04302131)(221.5159139,504.17387955)(221.99247742,504.93951154)
\curveto(222.46903795,505.70512802)(223.17411537,506.08794014)(224.10771179,506.08794904)
}
}
{
\newrgbcolor{curcolor}{0 0 0}
\pscustom[linestyle=none,fillstyle=solid,fillcolor=curcolor]
{
\newpath
\moveto(247.45521545,498.20562482)
\lineto(251.4747467,498.20562482)
\lineto(251.4747467,497.20953107)
\lineto(246.16029358,497.20953107)
\lineto(246.16029358,498.20562482)
\curveto(246.89076071,498.97515431)(247.52943194,499.65484113)(248.0763092,500.24468732)
\curveto(248.62318085,500.83452745)(249.0001336,501.25054266)(249.20716858,501.4927342)
\curveto(249.59778925,501.96929194)(249.86146086,502.35405718)(249.9981842,502.64703107)
\curveto(250.13489809,502.94390034)(250.2032574,503.24663441)(250.20326233,503.5552342)
\curveto(250.2032574,504.04350862)(250.05872629,504.42632073)(249.76966858,504.7036717)
\curveto(249.48450811,504.98100768)(249.09193038,505.11967941)(248.5919342,505.11968732)
\curveto(248.23646249,505.11967941)(247.86341599,505.05522635)(247.47279358,504.92632795)
\curveto(247.08216677,504.79741411)(246.66810468,504.60210181)(246.23060608,504.34039045)
\lineto(246.23060608,505.53570295)
\curveto(246.63294847,505.72710068)(247.02747932,505.87163179)(247.41419983,505.9692967)
\curveto(247.80482229,506.06694409)(248.18958753,506.11577217)(248.5684967,506.11578107)
\curveto(249.4239613,506.11577217)(250.11146061,505.88725677)(250.6309967,505.4302342)
\curveto(251.15442832,504.97710143)(251.41614681,504.3813989)(251.41615295,503.64312482)
\curveto(251.41614681,503.26811877)(251.32825627,502.89311914)(251.15248108,502.51812482)
\curveto(250.98060037,502.14311989)(250.69935065,501.7290578)(250.30873108,501.27593732)
\curveto(250.08997626,501.02202726)(249.7716172,500.67046511)(249.35365295,500.22124982)
\curveto(248.93958678,499.77202851)(248.30677492,499.10015418)(247.45521545,498.20562482)
}
}
{
\newrgbcolor{curcolor}{0 0 0}
\pscustom[linestyle=none,fillstyle=solid,fillcolor=curcolor]
{
\newpath
\moveto(256.7950592,504.87945295)
\lineto(254.03529358,500.25640607)
\lineto(256.7950592,500.25640607)
\lineto(256.7950592,504.87945295)
\moveto(256.60169983,505.95757795)
\lineto(257.97279358,505.95757795)
\lineto(257.97279358,500.25640607)
\lineto(259.1388092,500.25640607)
\lineto(259.1388092,499.29546857)
\lineto(257.97279358,499.29546857)
\lineto(257.97279358,497.20953107)
\lineto(256.7950592,497.20953107)
\lineto(256.7950592,499.29546857)
\lineto(253.08607483,499.29546857)
\lineto(253.08607483,500.4146092)
\lineto(256.60169983,505.95757795)
}
}
{
\newrgbcolor{curcolor}{0 0 0}
\pscustom[linestyle=none,fillstyle=solid,fillcolor=curcolor]
{
\newpath
\moveto(288.01174927,501.37011457)
\curveto(287.48440244,501.37011042)(287.07619972,501.22167306)(286.78713989,500.92480207)
\curveto(286.50198154,500.6318299)(286.35940356,500.2158147)(286.35940552,499.6767552)
\curveto(286.35940356,499.13769077)(286.50393467,498.71776932)(286.79299927,498.41698957)
\curveto(287.08596533,498.12011367)(287.49221493,497.97167631)(288.01174927,497.97167707)
\curveto(288.54299513,497.97167631)(288.95119784,498.11816054)(289.23635864,498.4111302)
\curveto(289.52541602,498.7080037)(289.66994713,499.12987828)(289.66995239,499.6767552)
\curveto(289.66994713,500.21190845)(289.5234629,500.62792366)(289.23049927,500.92480207)
\curveto(288.94143223,501.22167306)(288.53518263,501.37011042)(288.01174927,501.37011457)
\moveto(286.98049927,501.86230207)
\curveto(286.47659094,501.99120354)(286.08206009,502.23143768)(285.79690552,502.5830052)
\curveto(285.5156544,502.93456198)(285.37502954,503.35838968)(285.37503052,503.85448957)
\curveto(285.37502954,504.54979474)(285.61135743,505.10057544)(286.08401489,505.50683332)
\curveto(286.55666899,505.91698087)(287.19924647,506.12205879)(288.01174927,506.1220677)
\curveto(288.82815109,506.12205879)(289.4726817,505.91698087)(289.94534302,505.50683332)
\curveto(290.41799325,505.10057544)(290.65432114,504.54979474)(290.65432739,503.85448957)
\curveto(290.65432114,503.35838968)(290.51174316,502.93456198)(290.22659302,502.5830052)
\curveto(289.94533747,502.23143768)(289.55275974,501.99120354)(289.04885864,501.86230207)
\curveto(289.63479091,501.7333913)(290.08205609,501.47362594)(290.39065552,501.0830052)
\curveto(290.70314922,500.69237672)(290.85939906,500.18651785)(290.85940552,499.56542707)
\curveto(290.85939906,498.77636301)(290.60744619,498.15917613)(290.10354614,497.71386457)
\curveto(289.5996347,497.26855202)(288.90236977,497.04589599)(288.01174927,497.04589582)
\curveto(287.12112155,497.04589599)(286.42385662,497.26659889)(285.91995239,497.7080052)
\curveto(285.41995137,498.15331676)(285.16995163,498.76855052)(285.16995239,499.55370832)
\curveto(285.16995163,500.17870536)(285.32424835,500.68651735)(285.63284302,501.07714582)
\curveto(285.94534147,501.47167281)(286.39455978,501.7333913)(286.98049927,501.86230207)
\moveto(286.55276489,503.74316145)
\curveto(286.55276274,503.27440539)(286.67776262,502.91698387)(286.92776489,502.67089582)
\curveto(287.17776212,502.42479686)(287.53908988,502.30175011)(288.01174927,502.3017552)
\curveto(288.48830768,502.30175011)(288.85158857,502.42479686)(289.10159302,502.67089582)
\curveto(289.35158807,502.91698387)(289.47658794,503.27440539)(289.47659302,503.74316145)
\curveto(289.47658794,504.21971694)(289.35158807,504.58299783)(289.10159302,504.8330052)
\curveto(288.85549481,505.08299733)(288.49221393,505.2079972)(288.01174927,505.2080052)
\curveto(287.53908988,505.2079972)(287.17776212,505.08104421)(286.92776489,504.82714582)
\curveto(286.67776262,504.57713846)(286.55276274,504.2158107)(286.55276489,503.74316145)
}
}
{
\newrgbcolor{curcolor}{0 0 0}
\pscustom[linestyle=none,fillstyle=solid,fillcolor=curcolor]
{
\newpath
\moveto(278.81253052,515.06011701)
\lineto(280.65237427,515.06011701)
\lineto(280.65237427,521.74566389)
\lineto(278.67190552,521.30035139)
\lineto(278.67190552,522.37847639)
\lineto(280.64065552,522.81207014)
\lineto(281.82424927,522.81207014)
\lineto(281.82424927,515.06011701)
\lineto(283.64065552,515.06011701)
\lineto(283.64065552,514.06402326)
\lineto(278.81253052,514.06402326)
\lineto(278.81253052,515.06011701)
}
}
{
\newrgbcolor{curcolor}{0 0 0}
\pscustom[linestyle=none,fillstyle=solid,fillcolor=curcolor]
{
\newpath
\moveto(286.03128052,515.06011701)
\lineto(287.87112427,515.06011701)
\lineto(287.87112427,521.74566389)
\lineto(285.89065552,521.30035139)
\lineto(285.89065552,522.37847639)
\lineto(287.85940552,522.81207014)
\lineto(289.04299927,522.81207014)
\lineto(289.04299927,515.06011701)
\lineto(290.85940552,515.06011701)
\lineto(290.85940552,514.06402326)
\lineto(286.03128052,514.06402326)
\lineto(286.03128052,515.06011701)
}
}
{
\newrgbcolor{curcolor}{0 0 0}
\pscustom[linestyle=none,fillstyle=solid,fillcolor=curcolor]
{
\newpath
\moveto(312.26507568,522.98193098)
\lineto(316.69476318,522.98193098)
\lineto(316.69476318,521.98583723)
\lineto(313.34320068,521.98583723)
\lineto(313.34320068,519.8354466)
\curveto(313.51116697,519.89794094)(313.67913556,519.94286277)(313.84710693,519.97021223)
\curveto(314.01897897,520.00145646)(314.19085379,520.01708144)(314.36273193,520.01708723)
\curveto(315.26897772,520.01708144)(315.987727,519.74950359)(316.51898193,519.21435285)
\curveto(317.05022594,518.67919216)(317.31585067,517.95458351)(317.31585693,517.04052473)
\curveto(317.31585067,516.11864784)(317.03655407,515.39208607)(316.47796631,514.86083723)
\curveto(315.92327394,514.32958713)(315.16350907,514.0639624)(314.19866943,514.06396223)
\curveto(313.733823,514.0639624)(313.30804218,514.09521237)(312.92132568,514.15771223)
\curveto(312.5385117,514.22021224)(312.19476204,514.31396215)(311.89007568,514.43896223)
\lineto(311.89007568,515.6401341)
\curveto(312.24944949,515.44482039)(312.61077725,515.29833616)(312.97406006,515.20068098)
\curveto(313.33733902,515.1069301)(313.7084324,515.06005515)(314.08734131,515.06005598)
\curveto(314.73968137,515.06005515)(315.24163399,515.23192998)(315.59320068,515.57568098)
\curveto(315.94866454,515.91942929)(316.12639873,516.40771005)(316.12640381,517.04052473)
\curveto(316.12639873,517.66552129)(315.94280517,518.15184893)(315.57562256,518.4995091)
\curveto(315.21233715,518.84716074)(314.70452516,519.02098869)(314.05218506,519.02099348)
\curveto(313.73577612,519.02098869)(313.42718268,518.98387935)(313.12640381,518.90966535)
\curveto(312.82562079,518.83934825)(312.5385117,518.73192648)(312.26507568,518.58739973)
\lineto(312.26507568,522.98193098)
}
}
{
\newrgbcolor{curcolor}{0 0 0}
\pscustom[linestyle=none,fillstyle=solid,fillcolor=curcolor]
{
\newpath
\moveto(321.10101318,518.62841535)
\curveto(321.10101035,518.84325449)(321.17522903,519.02880118)(321.32366943,519.18505598)
\curveto(321.47600998,519.34130087)(321.65765042,519.41942579)(321.86859131,519.41943098)
\curveto(322.08733749,519.41942579)(322.2748373,519.34130087)(322.43109131,519.18505598)
\curveto(322.58733699,519.02880118)(322.66546191,518.84325449)(322.66546631,518.62841535)
\curveto(322.66546191,518.40966118)(322.58733699,518.22411449)(322.43109131,518.07177473)
\curveto(322.27874355,517.91942729)(322.09124374,517.84325549)(321.86859131,517.8432591)
\curveto(321.64983793,517.84325549)(321.46624436,517.91747417)(321.31781006,518.06591535)
\curveto(321.17327591,518.21434887)(321.10101035,518.40184868)(321.10101318,518.62841535)
\moveto(321.88031006,522.2026341)
\curveto(321.32952575,522.20262613)(320.91741679,521.90575143)(320.64398193,521.3120091)
\curveto(320.37444858,520.71825262)(320.23968309,519.81395665)(320.23968506,518.59911848)
\curveto(320.23968309,517.38817782)(320.37444858,516.48583497)(320.64398193,515.89208723)
\curveto(320.91741679,515.29833616)(321.32952575,515.00146146)(321.88031006,515.00146223)
\curveto(322.43499339,515.00146146)(322.84710236,515.29833616)(323.11663818,515.89208723)
\curveto(323.39007056,516.48583497)(323.52678918,517.38817782)(323.52679443,518.59911848)
\curveto(323.52678918,519.81395665)(323.39007056,520.71825262)(323.11663818,521.3120091)
\curveto(322.84710236,521.90575143)(322.43499339,522.20262613)(321.88031006,522.2026341)
\moveto(321.88031006,523.1401341)
\curveto(322.81389927,523.1401252)(323.51897669,522.75731308)(323.99554443,521.9916966)
\curveto(324.47600698,521.22606461)(324.71624111,520.09520637)(324.71624756,518.59911848)
\curveto(324.71624111,517.1069281)(324.47600698,515.97802298)(323.99554443,515.21239973)
\curveto(323.51897669,514.44677451)(322.81389927,514.0639624)(321.88031006,514.06396223)
\curveto(320.94671363,514.0639624)(320.24163621,514.44677451)(319.76507568,515.21239973)
\curveto(319.28851217,515.97802298)(319.05023115,517.1069281)(319.05023193,518.59911848)
\curveto(319.05023115,520.09520637)(319.28851217,521.22606461)(319.76507568,521.9916966)
\curveto(320.24163621,522.75731308)(320.94671363,523.1401252)(321.88031006,523.1401341)
}
}
{
\newrgbcolor{curcolor}{0 0 0}
\pscustom[linestyle=none,fillstyle=solid,fillcolor=curcolor]
{
\newpath
\moveto(315.15374756,504.84661604)
\lineto(312.39398193,500.22356916)
\lineto(315.15374756,500.22356916)
\lineto(315.15374756,504.84661604)
\moveto(314.96038818,505.92474104)
\lineto(316.33148193,505.92474104)
\lineto(316.33148193,500.22356916)
\lineto(317.49749756,500.22356916)
\lineto(317.49749756,499.26263166)
\lineto(316.33148193,499.26263166)
\lineto(316.33148193,497.17669416)
\lineto(315.15374756,497.17669416)
\lineto(315.15374756,499.26263166)
\lineto(311.44476318,499.26263166)
\lineto(311.44476318,500.38177229)
\lineto(314.96038818,505.92474104)
}
}
{
\newrgbcolor{curcolor}{0 0 0}
\pscustom[linestyle=none,fillstyle=solid,fillcolor=curcolor]
{
\newpath
\moveto(322.37249756,504.84661604)
\lineto(319.61273193,500.22356916)
\lineto(322.37249756,500.22356916)
\lineto(322.37249756,504.84661604)
\moveto(322.17913818,505.92474104)
\lineto(323.55023193,505.92474104)
\lineto(323.55023193,500.22356916)
\lineto(324.71624756,500.22356916)
\lineto(324.71624756,499.26263166)
\lineto(323.55023193,499.26263166)
\lineto(323.55023193,497.17669416)
\lineto(322.37249756,497.17669416)
\lineto(322.37249756,499.26263166)
\lineto(318.66351318,499.26263166)
\lineto(318.66351318,500.38177229)
\lineto(322.17913818,505.92474104)
}
}
{
\newrgbcolor{curcolor}{0 0 0}
\pscustom[linestyle=none,fillstyle=solid,fillcolor=curcolor]
{
\newpath
\moveto(223.32841492,484.52447248)
\curveto(223.32841209,484.73931162)(223.40263076,484.92485831)(223.55107117,485.08111311)
\curveto(223.70341171,485.237358)(223.88505216,485.31548292)(224.09599304,485.31548811)
\curveto(224.31473923,485.31548292)(224.50223904,485.237358)(224.65849304,485.08111311)
\curveto(224.81473873,484.92485831)(224.89286365,484.73931162)(224.89286804,484.52447248)
\curveto(224.89286365,484.3057183)(224.81473873,484.12017162)(224.65849304,483.96783186)
\curveto(224.50614528,483.81548442)(224.31864547,483.73931262)(224.09599304,483.73931623)
\curveto(223.87723966,483.73931262)(223.6936461,483.8135313)(223.54521179,483.96197248)
\curveto(223.40067764,484.110406)(223.32841209,484.29790581)(223.32841492,484.52447248)
\moveto(224.10771179,488.09869123)
\curveto(223.55692748,488.09868326)(223.14481852,487.80180856)(222.87138367,487.20806623)
\curveto(222.60185031,486.61430975)(222.46708482,485.71001378)(222.46708679,484.49517561)
\curveto(222.46708482,483.28423495)(222.60185031,482.3818921)(222.87138367,481.78814436)
\curveto(223.14481852,481.19439329)(223.55692748,480.89751859)(224.10771179,480.89751936)
\curveto(224.66239513,480.89751859)(225.07450409,481.19439329)(225.34403992,481.78814436)
\curveto(225.6174723,482.3818921)(225.75419091,483.28423495)(225.75419617,484.49517561)
\curveto(225.75419091,485.71001378)(225.6174723,486.61430975)(225.34403992,487.20806623)
\curveto(225.07450409,487.80180856)(224.66239513,488.09868326)(224.10771179,488.09869123)
\moveto(224.10771179,489.03619123)
\curveto(225.041301,489.03618232)(225.74637842,488.65337021)(226.22294617,487.88775373)
\curveto(226.70340871,487.12212174)(226.94364285,485.99126349)(226.94364929,484.49517561)
\curveto(226.94364285,483.00298523)(226.70340871,481.87408011)(226.22294617,481.10845686)
\curveto(225.74637842,480.34283164)(225.041301,479.96001953)(224.10771179,479.96001936)
\curveto(223.17411537,479.96001953)(222.46903795,480.34283164)(221.99247742,481.10845686)
\curveto(221.5159139,481.87408011)(221.27763289,483.00298523)(221.27763367,484.49517561)
\curveto(221.27763289,485.99126349)(221.5159139,487.12212174)(221.99247742,487.88775373)
\curveto(222.46903795,488.65337021)(223.17411537,489.03618232)(224.10771179,489.03619123)
}
}
{
\newrgbcolor{curcolor}{0 0 0}
\pscustom[linestyle=none,fillstyle=solid,fillcolor=curcolor]
{
\newpath
\moveto(255.52357483,484.74981428)
\curveto(255.523572,484.96465342)(255.59779067,485.15020011)(255.74623108,485.3064549)
\curveto(255.89857162,485.46269979)(256.08021207,485.54082472)(256.29115295,485.5408299)
\curveto(256.50989914,485.54082472)(256.69739895,485.46269979)(256.85365295,485.3064549)
\curveto(257.00989864,485.15020011)(257.08802356,484.96465342)(257.08802795,484.74981428)
\curveto(257.08802356,484.5310601)(257.00989864,484.34551341)(256.85365295,484.19317365)
\curveto(256.7013052,484.04082622)(256.51380538,483.96465442)(256.29115295,483.96465803)
\curveto(256.07239958,483.96465442)(255.88880601,484.03887309)(255.7403717,484.18731428)
\curveto(255.59583755,484.3357478)(255.523572,484.52324761)(255.52357483,484.74981428)
\moveto(256.3028717,488.32403303)
\curveto(255.7520874,488.32402506)(255.33997843,488.02715036)(255.06654358,487.43340803)
\curveto(254.79701023,486.83965154)(254.66224474,485.93535557)(254.6622467,484.7205174)
\curveto(254.66224474,483.50957675)(254.79701023,482.6072339)(255.06654358,482.01348615)
\curveto(255.33997843,481.41973509)(255.7520874,481.12286038)(256.3028717,481.12286115)
\curveto(256.85755504,481.12286038)(257.269664,481.41973509)(257.53919983,482.01348615)
\curveto(257.81263221,482.6072339)(257.94935082,483.50957675)(257.94935608,484.7205174)
\curveto(257.94935082,485.93535557)(257.81263221,486.83965154)(257.53919983,487.43340803)
\curveto(257.269664,488.02715036)(256.85755504,488.32402506)(256.3028717,488.32403303)
\moveto(256.3028717,489.26153303)
\curveto(257.23646091,489.26152412)(257.94153833,488.878712)(258.41810608,488.11309553)
\curveto(258.89856862,487.34746354)(259.13880276,486.21660529)(259.1388092,484.7205174)
\curveto(259.13880276,483.22832703)(258.89856862,482.09942191)(258.41810608,481.33379865)
\curveto(257.94153833,480.56817344)(257.23646091,480.18536132)(256.3028717,480.18536115)
\curveto(255.36927528,480.18536132)(254.66419786,480.56817344)(254.18763733,481.33379865)
\curveto(253.71107381,482.09942191)(253.4727928,483.22832703)(253.47279358,484.7205174)
\curveto(253.4727928,486.21660529)(253.71107381,487.34746354)(254.18763733,488.11309553)
\curveto(254.66419786,488.878712)(255.36927528,489.26152412)(256.3028717,489.26153303)
}
}
{
\newrgbcolor{curcolor}{0 0 0}
\pscustom[linestyle=none,fillstyle=solid,fillcolor=curcolor]
{
\newpath
\moveto(287.24414062,484.59209943)
\curveto(287.24413779,484.80693857)(287.31835647,484.99248526)(287.46679688,485.14874006)
\curveto(287.61913742,485.30498495)(287.80077786,485.38310987)(288.01171875,485.38311506)
\curveto(288.23046493,485.38310987)(288.41796475,485.30498495)(288.57421875,485.14874006)
\curveto(288.73046443,484.99248526)(288.80858936,484.80693857)(288.80859375,484.59209943)
\curveto(288.80858936,484.37334526)(288.73046443,484.18779857)(288.57421875,484.03545881)
\curveto(288.42187099,483.88311137)(288.23437118,483.80693957)(288.01171875,483.80694318)
\curveto(287.79296537,483.80693957)(287.6093718,483.88115825)(287.4609375,484.02959943)
\curveto(287.31640335,484.17803295)(287.24413779,484.36553277)(287.24414062,484.59209943)
\moveto(288.0234375,488.16631818)
\curveto(287.47265319,488.16631021)(287.06054423,487.86943551)(286.78710938,487.27569318)
\curveto(286.51757602,486.6819367)(286.38281053,485.77764073)(286.3828125,484.56280256)
\curveto(286.38281053,483.3518619)(286.51757602,482.44951906)(286.78710938,481.85577131)
\curveto(287.06054423,481.26202024)(287.47265319,480.96514554)(288.0234375,480.96514631)
\curveto(288.57812084,480.96514554)(288.9902298,481.26202024)(289.25976562,481.85577131)
\curveto(289.53319801,482.44951906)(289.66991662,483.3518619)(289.66992188,484.56280256)
\curveto(289.66991662,485.77764073)(289.53319801,486.6819367)(289.25976562,487.27569318)
\curveto(288.9902298,487.86943551)(288.57812084,488.16631021)(288.0234375,488.16631818)
\moveto(288.0234375,489.10381818)
\curveto(288.95702671,489.10380928)(289.66210413,488.72099716)(290.13867188,487.95538068)
\curveto(290.61913442,487.18974869)(290.85936855,486.05889045)(290.859375,484.56280256)
\curveto(290.85936855,483.07061219)(290.61913442,481.94170706)(290.13867188,481.17608381)
\curveto(289.66210413,480.4104586)(288.95702671,480.02764648)(288.0234375,480.02764631)
\curveto(287.08984107,480.02764648)(286.38476365,480.4104586)(285.90820312,481.17608381)
\curveto(285.43163961,481.94170706)(285.1933586,483.07061219)(285.19335938,484.56280256)
\curveto(285.1933586,486.05889045)(285.43163961,487.18974869)(285.90820312,487.95538068)
\curveto(286.38476365,488.72099716)(287.08984107,489.10380928)(288.0234375,489.10381818)
}
}
{
\newrgbcolor{curcolor}{0 0 0}
\pscustom[linestyle=none,fillstyle=solid,fillcolor=curcolor]
{
\newpath
\moveto(319.66546631,489.03741193)
\lineto(324.09515381,489.03741193)
\lineto(324.09515381,488.04131818)
\lineto(320.74359131,488.04131818)
\lineto(320.74359131,485.89092756)
\curveto(320.9115576,485.95342189)(321.07952618,485.99834372)(321.24749756,486.02569318)
\curveto(321.41936959,486.05693742)(321.59124442,486.0725624)(321.76312256,486.07256818)
\curveto(322.66936834,486.0725624)(323.38811762,485.80498454)(323.91937256,485.26983381)
\curveto(324.45061656,484.73467311)(324.71624129,484.01006446)(324.71624756,483.09600568)
\curveto(324.71624129,482.1741288)(324.4369447,481.44756703)(323.87835693,480.91631818)
\curveto(323.32366456,480.38506809)(322.5638997,480.11944335)(321.59906006,480.11944318)
\curveto(321.13421363,480.11944335)(320.7084328,480.15069332)(320.32171631,480.21319318)
\curveto(319.93890232,480.2756932)(319.59515267,480.3694431)(319.29046631,480.49444318)
\lineto(319.29046631,481.69561506)
\curveto(319.64984011,481.50030135)(320.01116788,481.35381712)(320.37445068,481.25616193)
\curveto(320.73772965,481.16241106)(321.10882303,481.11553611)(321.48773193,481.11553693)
\curveto(322.140072,481.11553611)(322.64202462,481.28741094)(322.99359131,481.63116193)
\curveto(323.34905516,481.97491025)(323.52678936,482.46319101)(323.52679443,483.09600568)
\curveto(323.52678936,483.72100225)(323.34319579,484.20732989)(322.97601318,484.55499006)
\curveto(322.61272777,484.9026417)(322.10491578,485.07646965)(321.45257568,485.07647443)
\curveto(321.13616675,485.07646965)(320.82757331,485.03936031)(320.52679443,484.96514631)
\curveto(320.22601141,484.8948292)(319.93890232,484.78740744)(319.66546631,484.64288068)
\lineto(319.66546631,489.03741193)
}
}
{
\newrgbcolor{curcolor}{0 0 0}
\pscustom[linestyle=none,fillstyle=solid,fillcolor=curcolor]
{
\newpath
\moveto(223.32841492,467.47271467)
\curveto(223.32841209,467.68755381)(223.40263076,467.8731005)(223.55107117,468.02935529)
\curveto(223.70341171,468.18560019)(223.88505216,468.26372511)(224.09599304,468.26373029)
\curveto(224.31473923,468.26372511)(224.50223904,468.18560019)(224.65849304,468.02935529)
\curveto(224.81473873,467.8731005)(224.89286365,467.68755381)(224.89286804,467.47271467)
\curveto(224.89286365,467.25396049)(224.81473873,467.0684138)(224.65849304,466.91607404)
\curveto(224.50614528,466.76372661)(224.31864547,466.68755481)(224.09599304,466.68755842)
\curveto(223.87723966,466.68755481)(223.6936461,466.76177348)(223.54521179,466.91021467)
\curveto(223.40067764,467.05864819)(223.32841209,467.246148)(223.32841492,467.47271467)
\moveto(224.10771179,471.04693342)
\curveto(223.55692748,471.04692545)(223.14481852,470.75005075)(222.87138367,470.15630842)
\curveto(222.60185031,469.56255193)(222.46708482,468.65825596)(222.46708679,467.44341779)
\curveto(222.46708482,466.23247714)(222.60185031,465.33013429)(222.87138367,464.73638654)
\curveto(223.14481852,464.14263548)(223.55692748,463.84576078)(224.10771179,463.84576154)
\curveto(224.66239513,463.84576078)(225.07450409,464.14263548)(225.34403992,464.73638654)
\curveto(225.6174723,465.33013429)(225.75419091,466.23247714)(225.75419617,467.44341779)
\curveto(225.75419091,468.65825596)(225.6174723,469.56255193)(225.34403992,470.15630842)
\curveto(225.07450409,470.75005075)(224.66239513,471.04692545)(224.10771179,471.04693342)
\moveto(224.10771179,471.98443342)
\curveto(225.041301,471.98442451)(225.74637842,471.60161239)(226.22294617,470.83599592)
\curveto(226.70340871,470.07036393)(226.94364285,468.93950568)(226.94364929,467.44341779)
\curveto(226.94364285,465.95122742)(226.70340871,464.8223223)(226.22294617,464.05669904)
\curveto(225.74637842,463.29107383)(225.041301,462.90826171)(224.10771179,462.90826154)
\curveto(223.17411537,462.90826171)(222.46903795,463.29107383)(221.99247742,464.05669904)
\curveto(221.5159139,464.8223223)(221.27763289,465.95122742)(221.27763367,467.44341779)
\curveto(221.27763289,468.93950568)(221.5159139,470.07036393)(221.99247742,470.83599592)
\curveto(222.46903795,471.60161239)(223.17411537,471.98442451)(224.10771179,471.98443342)
}
}
{
\newrgbcolor{curcolor}{0 0 0}
\pscustom[linestyle=none,fillstyle=solid,fillcolor=curcolor]
{
\newpath
\moveto(255.52357483,467.72552229)
\curveto(255.523572,467.94036143)(255.59779067,468.12590812)(255.74623108,468.28216291)
\curveto(255.89857162,468.4384078)(256.08021207,468.51653272)(256.29115295,468.51653791)
\curveto(256.50989914,468.51653272)(256.69739895,468.4384078)(256.85365295,468.28216291)
\curveto(257.00989864,468.12590812)(257.08802356,467.94036143)(257.08802795,467.72552229)
\curveto(257.08802356,467.50676811)(257.00989864,467.32122142)(256.85365295,467.16888166)
\curveto(256.7013052,467.01653422)(256.51380538,466.94036243)(256.29115295,466.94036604)
\curveto(256.07239958,466.94036243)(255.88880601,467.0145811)(255.7403717,467.16302229)
\curveto(255.59583755,467.3114558)(255.523572,467.49895562)(255.52357483,467.72552229)
\moveto(256.3028717,471.29974104)
\curveto(255.7520874,471.29973307)(255.33997843,471.00285836)(255.06654358,470.40911604)
\curveto(254.79701023,469.81535955)(254.66224474,468.91106358)(254.6622467,467.69622541)
\curveto(254.66224474,466.48528476)(254.79701023,465.58294191)(255.06654358,464.98919416)
\curveto(255.33997843,464.3954431)(255.7520874,464.09856839)(256.3028717,464.09856916)
\curveto(256.85755504,464.09856839)(257.269664,464.3954431)(257.53919983,464.98919416)
\curveto(257.81263221,465.58294191)(257.94935082,466.48528476)(257.94935608,467.69622541)
\curveto(257.94935082,468.91106358)(257.81263221,469.81535955)(257.53919983,470.40911604)
\curveto(257.269664,471.00285836)(256.85755504,471.29973307)(256.3028717,471.29974104)
\moveto(256.3028717,472.23724104)
\curveto(257.23646091,472.23723213)(257.94153833,471.85442001)(258.41810608,471.08880354)
\curveto(258.89856862,470.32317154)(259.13880276,469.1923133)(259.1388092,467.69622541)
\curveto(259.13880276,466.20403504)(258.89856862,465.07512992)(258.41810608,464.30950666)
\curveto(257.94153833,463.54388145)(257.23646091,463.16106933)(256.3028717,463.16106916)
\curveto(255.36927528,463.16106933)(254.66419786,463.54388145)(254.18763733,464.30950666)
\curveto(253.71107381,465.07512992)(253.4727928,466.20403504)(253.47279358,467.69622541)
\curveto(253.4727928,469.1923133)(253.71107381,470.32317154)(254.18763733,471.08880354)
\curveto(254.66419786,471.85442001)(255.36927528,472.23723213)(256.3028717,472.23724104)
}
}
{
\newrgbcolor{curcolor}{0 0 0}
\pscustom[linestyle=none,fillstyle=solid,fillcolor=curcolor]
{
\newpath
\moveto(287.24414062,467.57391096)
\curveto(287.24413779,467.7887501)(287.31835647,467.97429679)(287.46679688,468.13055158)
\curveto(287.61913742,468.28679647)(287.80077786,468.3649214)(288.01171875,468.36492658)
\curveto(288.23046493,468.3649214)(288.41796475,468.28679647)(288.57421875,468.13055158)
\curveto(288.73046443,467.97429679)(288.80858936,467.7887501)(288.80859375,467.57391096)
\curveto(288.80858936,467.35515678)(288.73046443,467.16961009)(288.57421875,467.01727033)
\curveto(288.42187099,466.8649229)(288.23437118,466.7887511)(288.01171875,466.78875471)
\curveto(287.79296537,466.7887511)(287.6093718,466.86296977)(287.4609375,467.01141096)
\curveto(287.31640335,467.15984448)(287.24413779,467.34734429)(287.24414062,467.57391096)
\moveto(288.0234375,471.14812971)
\curveto(287.47265319,471.14812174)(287.06054423,470.85124704)(286.78710938,470.25750471)
\curveto(286.51757602,469.66374822)(286.38281053,468.75945225)(286.3828125,467.54461408)
\curveto(286.38281053,466.33367343)(286.51757602,465.43133058)(286.78710938,464.83758283)
\curveto(287.06054423,464.24383177)(287.47265319,463.94695706)(288.0234375,463.94695783)
\curveto(288.57812084,463.94695706)(288.9902298,464.24383177)(289.25976562,464.83758283)
\curveto(289.53319801,465.43133058)(289.66991662,466.33367343)(289.66992188,467.54461408)
\curveto(289.66991662,468.75945225)(289.53319801,469.66374822)(289.25976562,470.25750471)
\curveto(288.9902298,470.85124704)(288.57812084,471.14812174)(288.0234375,471.14812971)
\moveto(288.0234375,472.08562971)
\curveto(288.95702671,472.0856208)(289.66210413,471.70280868)(290.13867188,470.93719221)
\curveto(290.61913442,470.17156021)(290.85936855,469.04070197)(290.859375,467.54461408)
\curveto(290.85936855,466.05242371)(290.61913442,464.92351859)(290.13867188,464.15789533)
\curveto(289.66210413,463.39227012)(288.95702671,463.009458)(288.0234375,463.00945783)
\curveto(287.08984107,463.009458)(286.38476365,463.39227012)(285.90820312,464.15789533)
\curveto(285.43163961,464.92351859)(285.1933586,466.05242371)(285.19335938,467.54461408)
\curveto(285.1933586,469.04070197)(285.43163961,470.17156021)(285.90820312,470.93719221)
\curveto(286.38476365,471.70280868)(287.08984107,472.0856208)(288.0234375,472.08562971)
}
}
{
\newrgbcolor{curcolor}{0 0 0}
\pscustom[linestyle=none,fillstyle=solid,fillcolor=curcolor]
{
\newpath
\moveto(320.69671631,464.07000471)
\lineto(324.71624756,464.07000471)
\lineto(324.71624756,463.07391096)
\lineto(319.40179443,463.07391096)
\lineto(319.40179443,464.07000471)
\curveto(320.13226156,464.83953419)(320.7709328,465.51922101)(321.31781006,466.10906721)
\curveto(321.86468171,466.69890733)(322.24163445,467.11492254)(322.44866943,467.35711408)
\curveto(322.83929011,467.83367182)(323.10296172,468.21843706)(323.23968506,468.51141096)
\curveto(323.37639894,468.80828022)(323.44475825,469.11101429)(323.44476318,469.41961408)
\curveto(323.44475825,469.9078885)(323.30022714,470.29070062)(323.01116943,470.56805158)
\curveto(322.72600897,470.84538756)(322.33343124,470.9840593)(321.83343506,470.98406721)
\curveto(321.47796334,470.9840593)(321.10491684,470.91960624)(320.71429443,470.79070783)
\curveto(320.32366762,470.66179399)(319.90960554,470.46648169)(319.47210693,470.20477033)
\lineto(319.47210693,471.40008283)
\curveto(319.87444932,471.59148056)(320.26898018,471.73601167)(320.65570068,471.83367658)
\curveto(321.04632315,471.93132397)(321.43108839,471.98015205)(321.80999756,471.98016096)
\curveto(322.66546215,471.98015205)(323.35296147,471.75163665)(323.87249756,471.29461408)
\curveto(324.39592917,470.84148131)(324.65764766,470.24577879)(324.65765381,469.50750471)
\curveto(324.65764766,469.13249865)(324.56975713,468.75749902)(324.39398193,468.38250471)
\curveto(324.22210122,468.00749977)(323.9408515,467.59343769)(323.55023193,467.14031721)
\curveto(323.33147711,466.88640714)(323.01311806,466.534845)(322.59515381,466.08562971)
\curveto(322.18108764,465.63640839)(321.54827577,464.96453407)(320.69671631,464.07000471)
}
}
{
\newrgbcolor{curcolor}{0 0 0}
\pscustom[linestyle=none,fillstyle=solid,fillcolor=curcolor]
{
\newpath
\moveto(223.32841492,450.42095686)
\curveto(223.32841209,450.635796)(223.40263076,450.82134269)(223.55107117,450.97759748)
\curveto(223.70341171,451.13384237)(223.88505216,451.21196729)(224.09599304,451.21197248)
\curveto(224.31473923,451.21196729)(224.50223904,451.13384237)(224.65849304,450.97759748)
\curveto(224.81473873,450.82134269)(224.89286365,450.635796)(224.89286804,450.42095686)
\curveto(224.89286365,450.20220268)(224.81473873,450.01665599)(224.65849304,449.86431623)
\curveto(224.50614528,449.71196879)(224.31864547,449.635797)(224.09599304,449.63580061)
\curveto(223.87723966,449.635797)(223.6936461,449.71001567)(223.54521179,449.85845686)
\curveto(223.40067764,450.00689038)(223.32841209,450.19439019)(223.32841492,450.42095686)
\moveto(224.10771179,453.99517561)
\curveto(223.55692748,453.99516764)(223.14481852,453.69829293)(222.87138367,453.10455061)
\curveto(222.60185031,452.51079412)(222.46708482,451.60649815)(222.46708679,450.39165998)
\curveto(222.46708482,449.18071933)(222.60185031,448.27837648)(222.87138367,447.68462873)
\curveto(223.14481852,447.09087767)(223.55692748,446.79400296)(224.10771179,446.79400373)
\curveto(224.66239513,446.79400296)(225.07450409,447.09087767)(225.34403992,447.68462873)
\curveto(225.6174723,448.27837648)(225.75419091,449.18071933)(225.75419617,450.39165998)
\curveto(225.75419091,451.60649815)(225.6174723,452.51079412)(225.34403992,453.10455061)
\curveto(225.07450409,453.69829293)(224.66239513,453.99516764)(224.10771179,453.99517561)
\moveto(224.10771179,454.93267561)
\curveto(225.041301,454.9326667)(225.74637842,454.54985458)(226.22294617,453.78423811)
\curveto(226.70340871,453.01860611)(226.94364285,451.88774787)(226.94364929,450.39165998)
\curveto(226.94364285,448.89946961)(226.70340871,447.77056449)(226.22294617,447.00494123)
\curveto(225.74637842,446.23931602)(225.041301,445.8565039)(224.10771179,445.85650373)
\curveto(223.17411537,445.8565039)(222.46903795,446.23931602)(221.99247742,447.00494123)
\curveto(221.5159139,447.77056449)(221.27763289,448.89946961)(221.27763367,450.39165998)
\curveto(221.27763289,451.88774787)(221.5159139,453.01860611)(221.99247742,453.78423811)
\curveto(222.46903795,454.54985458)(223.17411537,454.9326667)(224.10771179,454.93267561)
}
}
{
\newrgbcolor{curcolor}{0 0 0}
\pscustom[linestyle=none,fillstyle=solid,fillcolor=curcolor]
{
\newpath
\moveto(255.52357483,450.70135236)
\curveto(255.523572,450.9161915)(255.59779067,451.10173819)(255.74623108,451.25799299)
\curveto(255.89857162,451.41423788)(256.08021207,451.4923628)(256.29115295,451.49236799)
\curveto(256.50989914,451.4923628)(256.69739895,451.41423788)(256.85365295,451.25799299)
\curveto(257.00989864,451.10173819)(257.08802356,450.9161915)(257.08802795,450.70135236)
\curveto(257.08802356,450.48259819)(257.00989864,450.2970515)(256.85365295,450.14471174)
\curveto(256.7013052,449.9923643)(256.51380538,449.9161925)(256.29115295,449.91619611)
\curveto(256.07239958,449.9161925)(255.88880601,449.99041118)(255.7403717,450.13885236)
\curveto(255.59583755,450.28728588)(255.523572,450.4747857)(255.52357483,450.70135236)
\moveto(256.3028717,454.27557111)
\curveto(255.7520874,454.27556314)(255.33997843,453.97868844)(255.06654358,453.38494611)
\curveto(254.79701023,452.79118963)(254.66224474,451.88689366)(254.6622467,450.67205549)
\curveto(254.66224474,449.46111483)(254.79701023,448.55877199)(255.06654358,447.96502424)
\curveto(255.33997843,447.37127317)(255.7520874,447.07439847)(256.3028717,447.07439924)
\curveto(256.85755504,447.07439847)(257.269664,447.37127317)(257.53919983,447.96502424)
\curveto(257.81263221,448.55877199)(257.94935082,449.46111483)(257.94935608,450.67205549)
\curveto(257.94935082,451.88689366)(257.81263221,452.79118963)(257.53919983,453.38494611)
\curveto(257.269664,453.97868844)(256.85755504,454.27556314)(256.3028717,454.27557111)
\moveto(256.3028717,455.21307111)
\curveto(257.23646091,455.21306221)(257.94153833,454.83025009)(258.41810608,454.06463361)
\curveto(258.89856862,453.29900162)(259.13880276,452.16814338)(259.1388092,450.67205549)
\curveto(259.13880276,449.17986512)(258.89856862,448.05095999)(258.41810608,447.28533674)
\curveto(257.94153833,446.51971153)(257.23646091,446.13689941)(256.3028717,446.13689924)
\curveto(255.36927528,446.13689941)(254.66419786,446.51971153)(254.18763733,447.28533674)
\curveto(253.71107381,448.05095999)(253.4727928,449.17986512)(253.47279358,450.67205549)
\curveto(253.4727928,452.16814338)(253.71107381,453.29900162)(254.18763733,454.06463361)
\curveto(254.66419786,454.83025009)(255.36927528,455.21306221)(256.3028717,455.21307111)
}
}
{
\newrgbcolor{curcolor}{0 0 0}
\pscustom[linestyle=none,fillstyle=solid,fillcolor=curcolor]
{
\newpath
\moveto(287.24414062,450.55572248)
\curveto(287.24413779,450.77056162)(287.31835647,450.95610831)(287.46679688,451.11236311)
\curveto(287.61913742,451.268608)(287.80077786,451.34673292)(288.01171875,451.34673811)
\curveto(288.23046493,451.34673292)(288.41796475,451.268608)(288.57421875,451.11236311)
\curveto(288.73046443,450.95610831)(288.80858936,450.77056162)(288.80859375,450.55572248)
\curveto(288.80858936,450.3369683)(288.73046443,450.15142162)(288.57421875,449.99908186)
\curveto(288.42187099,449.84673442)(288.23437118,449.77056262)(288.01171875,449.77056623)
\curveto(287.79296537,449.77056262)(287.6093718,449.8447813)(287.4609375,449.99322248)
\curveto(287.31640335,450.141656)(287.24413779,450.32915581)(287.24414062,450.55572248)
\moveto(288.0234375,454.12994123)
\curveto(287.47265319,454.12993326)(287.06054423,453.83305856)(286.78710938,453.23931623)
\curveto(286.51757602,452.64555975)(286.38281053,451.74126378)(286.3828125,450.52642561)
\curveto(286.38281053,449.31548495)(286.51757602,448.4131421)(286.78710938,447.81939436)
\curveto(287.06054423,447.22564329)(287.47265319,446.92876859)(288.0234375,446.92876936)
\curveto(288.57812084,446.92876859)(288.9902298,447.22564329)(289.25976562,447.81939436)
\curveto(289.53319801,448.4131421)(289.66991662,449.31548495)(289.66992188,450.52642561)
\curveto(289.66991662,451.74126378)(289.53319801,452.64555975)(289.25976562,453.23931623)
\curveto(288.9902298,453.83305856)(288.57812084,454.12993326)(288.0234375,454.12994123)
\moveto(288.0234375,455.06744123)
\curveto(288.95702671,455.06743232)(289.66210413,454.68462021)(290.13867188,453.91900373)
\curveto(290.61913442,453.15337174)(290.85936855,452.02251349)(290.859375,450.52642561)
\curveto(290.85936855,449.03423523)(290.61913442,447.90533011)(290.13867188,447.13970686)
\curveto(289.66210413,446.37408164)(288.95702671,445.99126953)(288.0234375,445.99126936)
\curveto(287.08984107,445.99126953)(286.38476365,446.37408164)(285.90820312,447.13970686)
\curveto(285.43163961,447.90533011)(285.1933586,449.03423523)(285.19335938,450.52642561)
\curveto(285.1933586,452.02251349)(285.43163961,453.15337174)(285.90820312,453.91900373)
\curveto(286.38476365,454.68462021)(287.08984107,455.06743232)(288.0234375,455.06744123)
}
}
{
\newrgbcolor{curcolor}{0 0 0}
\pscustom[linestyle=none,fillstyle=solid,fillcolor=curcolor]
{
\newpath
\moveto(322.37249756,453.85650373)
\lineto(319.61273193,449.23345686)
\lineto(322.37249756,449.23345686)
\lineto(322.37249756,453.85650373)
\moveto(322.17913818,454.93462873)
\lineto(323.55023193,454.93462873)
\lineto(323.55023193,449.23345686)
\lineto(324.71624756,449.23345686)
\lineto(324.71624756,448.27251936)
\lineto(323.55023193,448.27251936)
\lineto(323.55023193,446.18658186)
\lineto(322.37249756,446.18658186)
\lineto(322.37249756,448.27251936)
\lineto(318.66351318,448.27251936)
\lineto(318.66351318,449.39165998)
\lineto(322.17913818,454.93462873)
}
}
{
\newrgbcolor{curcolor}{0 0 0}
\pscustom[linestyle=none,fillstyle=solid,fillcolor=curcolor]
{
\newpath
\moveto(223.32841492,433.36919904)
\curveto(223.32841209,433.58403818)(223.40263076,433.76958487)(223.55107117,433.92583967)
\curveto(223.70341171,434.08208456)(223.88505216,434.16020948)(224.09599304,434.16021467)
\curveto(224.31473923,434.16020948)(224.50223904,434.08208456)(224.65849304,433.92583967)
\curveto(224.81473873,433.76958487)(224.89286365,433.58403818)(224.89286804,433.36919904)
\curveto(224.89286365,433.15044487)(224.81473873,432.96489818)(224.65849304,432.81255842)
\curveto(224.50614528,432.66021098)(224.31864547,432.58403918)(224.09599304,432.58404279)
\curveto(223.87723966,432.58403918)(223.6936461,432.65825786)(223.54521179,432.80669904)
\curveto(223.40067764,432.95513256)(223.32841209,433.14263238)(223.32841492,433.36919904)
\moveto(224.10771179,436.94341779)
\curveto(223.55692748,436.94340982)(223.14481852,436.64653512)(222.87138367,436.05279279)
\curveto(222.60185031,435.45903631)(222.46708482,434.55474034)(222.46708679,433.33990217)
\curveto(222.46708482,432.12896151)(222.60185031,431.22661867)(222.87138367,430.63287092)
\curveto(223.14481852,430.03911985)(223.55692748,429.74224515)(224.10771179,429.74224592)
\curveto(224.66239513,429.74224515)(225.07450409,430.03911985)(225.34403992,430.63287092)
\curveto(225.6174723,431.22661867)(225.75419091,432.12896151)(225.75419617,433.33990217)
\curveto(225.75419091,434.55474034)(225.6174723,435.45903631)(225.34403992,436.05279279)
\curveto(225.07450409,436.64653512)(224.66239513,436.94340982)(224.10771179,436.94341779)
\moveto(224.10771179,437.88091779)
\curveto(225.041301,437.88090889)(225.74637842,437.49809677)(226.22294617,436.73248029)
\curveto(226.70340871,435.9668483)(226.94364285,434.83599006)(226.94364929,433.33990217)
\curveto(226.94364285,431.84771179)(226.70340871,430.71880667)(226.22294617,429.95318342)
\curveto(225.74637842,429.18755821)(225.041301,428.80474609)(224.10771179,428.80474592)
\curveto(223.17411537,428.80474609)(222.46903795,429.18755821)(221.99247742,429.95318342)
\curveto(221.5159139,430.71880667)(221.27763289,431.84771179)(221.27763367,433.33990217)
\curveto(221.27763289,434.83599006)(221.5159139,435.9668483)(221.99247742,436.73248029)
\curveto(222.46903795,437.49809677)(223.17411537,437.88090889)(224.10771179,437.88091779)
}
}
{
\newrgbcolor{curcolor}{0 0 0}
\pscustom[linestyle=none,fillstyle=solid,fillcolor=curcolor]
{
\newpath
\moveto(255.52357483,433.67706037)
\curveto(255.523572,433.89189951)(255.59779067,434.0774462)(255.74623108,434.233701)
\curveto(255.89857162,434.38994589)(256.08021207,434.46807081)(256.29115295,434.468076)
\curveto(256.50989914,434.46807081)(256.69739895,434.38994589)(256.85365295,434.233701)
\curveto(257.00989864,434.0774462)(257.08802356,433.89189951)(257.08802795,433.67706037)
\curveto(257.08802356,433.4583062)(257.00989864,433.27275951)(256.85365295,433.12041975)
\curveto(256.7013052,432.96807231)(256.51380538,432.89190051)(256.29115295,432.89190412)
\curveto(256.07239958,432.89190051)(255.88880601,432.96611919)(255.7403717,433.11456037)
\curveto(255.59583755,433.26299389)(255.523572,433.4504937)(255.52357483,433.67706037)
\moveto(256.3028717,437.25127912)
\curveto(255.7520874,437.25127115)(255.33997843,436.95439645)(255.06654358,436.36065412)
\curveto(254.79701023,435.76689764)(254.66224474,434.86260167)(254.6622467,433.6477635)
\curveto(254.66224474,432.43682284)(254.79701023,431.53447999)(255.06654358,430.94073225)
\curveto(255.33997843,430.34698118)(255.7520874,430.05010648)(256.3028717,430.05010725)
\curveto(256.85755504,430.05010648)(257.269664,430.34698118)(257.53919983,430.94073225)
\curveto(257.81263221,431.53447999)(257.94935082,432.43682284)(257.94935608,433.6477635)
\curveto(257.94935082,434.86260167)(257.81263221,435.76689764)(257.53919983,436.36065412)
\curveto(257.269664,436.95439645)(256.85755504,437.25127115)(256.3028717,437.25127912)
\moveto(256.3028717,438.18877912)
\curveto(257.23646091,438.18877021)(257.94153833,437.8059581)(258.41810608,437.04034162)
\curveto(258.89856862,436.27470963)(259.13880276,435.14385138)(259.1388092,433.6477635)
\curveto(259.13880276,432.15557312)(258.89856862,431.026668)(258.41810608,430.26104475)
\curveto(257.94153833,429.49541953)(257.23646091,429.11260742)(256.3028717,429.11260725)
\curveto(255.36927528,429.11260742)(254.66419786,429.49541953)(254.18763733,430.26104475)
\curveto(253.71107381,431.026668)(253.4727928,432.15557312)(253.47279358,433.6477635)
\curveto(253.4727928,435.14385138)(253.71107381,436.27470963)(254.18763733,437.04034162)
\curveto(254.66419786,437.8059581)(255.36927528,438.18877021)(256.3028717,438.18877912)
}
}
{
\newrgbcolor{curcolor}{0 0 0}
\pscustom[linestyle=none,fillstyle=solid,fillcolor=curcolor]
{
\newpath
\moveto(287.24414062,433.537534)
\curveto(287.24413779,433.75237314)(287.31835647,433.93791983)(287.46679688,434.09417463)
\curveto(287.61913742,434.25041952)(287.80077786,434.32854444)(288.01171875,434.32854963)
\curveto(288.23046493,434.32854444)(288.41796475,434.25041952)(288.57421875,434.09417463)
\curveto(288.73046443,433.93791983)(288.80858936,433.75237314)(288.80859375,433.537534)
\curveto(288.80858936,433.31877983)(288.73046443,433.13323314)(288.57421875,432.98089338)
\curveto(288.42187099,432.82854594)(288.23437118,432.75237414)(288.01171875,432.75237775)
\curveto(287.79296537,432.75237414)(287.6093718,432.82659282)(287.4609375,432.975034)
\curveto(287.31640335,433.12346752)(287.24413779,433.31096734)(287.24414062,433.537534)
\moveto(288.0234375,437.11175275)
\curveto(287.47265319,437.11174479)(287.06054423,436.81487008)(286.78710938,436.22112775)
\curveto(286.51757602,435.62737127)(286.38281053,434.7230753)(286.3828125,433.50823713)
\curveto(286.38281053,432.29729647)(286.51757602,431.39495363)(286.78710938,430.80120588)
\curveto(287.06054423,430.20745481)(287.47265319,429.91058011)(288.0234375,429.91058088)
\curveto(288.57812084,429.91058011)(288.9902298,430.20745481)(289.25976562,430.80120588)
\curveto(289.53319801,431.39495363)(289.66991662,432.29729647)(289.66992188,433.50823713)
\curveto(289.66991662,434.7230753)(289.53319801,435.62737127)(289.25976562,436.22112775)
\curveto(288.9902298,436.81487008)(288.57812084,437.11174479)(288.0234375,437.11175275)
\moveto(288.0234375,438.04925275)
\curveto(288.95702671,438.04924385)(289.66210413,437.66643173)(290.13867188,436.90081525)
\curveto(290.61913442,436.13518326)(290.85936855,435.00432502)(290.859375,433.50823713)
\curveto(290.85936855,432.01604676)(290.61913442,430.88714163)(290.13867188,430.12151838)
\curveto(289.66210413,429.35589317)(288.95702671,428.97308105)(288.0234375,428.97308088)
\curveto(287.08984107,428.97308105)(286.38476365,429.35589317)(285.90820312,430.12151838)
\curveto(285.43163961,430.88714163)(285.1933586,432.01604676)(285.19335938,433.50823713)
\curveto(285.1933586,435.00432502)(285.43163961,436.13518326)(285.90820312,436.90081525)
\curveto(286.38476365,437.66643173)(287.08984107,438.04924385)(288.0234375,438.04925275)
}
}
{
\newrgbcolor{curcolor}{0 0 0}
\pscustom[linestyle=none,fillstyle=solid,fillcolor=curcolor]
{
\newpath
\moveto(319.88812256,430.2953465)
\lineto(321.72796631,430.2953465)
\lineto(321.72796631,436.98089338)
\lineto(319.74749756,436.53558088)
\lineto(319.74749756,437.61370588)
\lineto(321.71624756,438.04729963)
\lineto(322.89984131,438.04729963)
\lineto(322.89984131,430.2953465)
\lineto(324.71624756,430.2953465)
\lineto(324.71624756,429.29925275)
\lineto(319.88812256,429.29925275)
\lineto(319.88812256,430.2953465)
}
}
{
\newrgbcolor{curcolor}{0 0 0}
\pscustom[linestyle=none,fillstyle=solid,fillcolor=curcolor]
{
\newpath
\moveto(222.92411804,412.91900373)
\lineto(226.94364929,412.91900373)
\lineto(226.94364929,411.92290998)
\lineto(221.62919617,411.92290998)
\lineto(221.62919617,412.91900373)
\curveto(222.3596633,413.68853321)(222.99833453,414.36822004)(223.54521179,414.95806623)
\curveto(224.09208344,415.54790636)(224.46903619,415.96392156)(224.67607117,416.20611311)
\curveto(225.06669184,416.68267085)(225.33036345,417.06743609)(225.46708679,417.36040998)
\curveto(225.60380068,417.65727925)(225.67215998,417.96001332)(225.67216492,418.26861311)
\curveto(225.67215998,418.75688752)(225.52762888,419.13969964)(225.23857117,419.41705061)
\curveto(224.9534107,419.69438658)(224.56083297,419.83305832)(224.06083679,419.83306623)
\curveto(223.70536508,419.83305832)(223.33231857,419.76860526)(222.94169617,419.63970686)
\curveto(222.55106935,419.51079302)(222.13700727,419.31548071)(221.69950867,419.05376936)
\lineto(221.69950867,420.24908186)
\curveto(222.10185105,420.44047959)(222.49638191,420.58501069)(222.88310242,420.68267561)
\curveto(223.27372488,420.780323)(223.65849012,420.82915107)(224.03739929,420.82915998)
\curveto(224.89286389,420.82915107)(225.5803632,420.60063568)(226.09989929,420.14361311)
\curveto(226.62333091,419.69048034)(226.8850494,419.09477781)(226.88505554,418.35650373)
\curveto(226.8850494,417.98149767)(226.79715886,417.60649805)(226.62138367,417.23150373)
\curveto(226.44950296,416.8564988)(226.16825324,416.44243671)(225.77763367,415.98931623)
\curveto(225.55887885,415.73540617)(225.24051979,415.38384402)(224.82255554,414.93462873)
\curveto(224.40848937,414.48540742)(223.7756775,413.81353309)(222.92411804,412.91900373)
}
}
{
\newrgbcolor{curcolor}{0 0 0}
\pscustom[linestyle=none,fillstyle=solid,fillcolor=curcolor]
{
\newpath
\moveto(255.52357483,416.65289045)
\curveto(255.523572,416.86772959)(255.59779067,417.05327628)(255.74623108,417.20953107)
\curveto(255.89857162,417.36577597)(256.08021207,417.44390089)(256.29115295,417.44390607)
\curveto(256.50989914,417.44390089)(256.69739895,417.36577597)(256.85365295,417.20953107)
\curveto(257.00989864,417.05327628)(257.08802356,416.86772959)(257.08802795,416.65289045)
\curveto(257.08802356,416.43413627)(257.00989864,416.24858958)(256.85365295,416.09624982)
\curveto(256.7013052,415.94390239)(256.51380538,415.86773059)(256.29115295,415.8677342)
\curveto(256.07239958,415.86773059)(255.88880601,415.94194927)(255.7403717,416.09039045)
\curveto(255.59583755,416.23882397)(255.523572,416.42632378)(255.52357483,416.65289045)
\moveto(256.3028717,420.2271092)
\curveto(255.7520874,420.22710123)(255.33997843,419.93022653)(255.06654358,419.3364842)
\curveto(254.79701023,418.74272771)(254.66224474,417.83843174)(254.6622467,416.62359357)
\curveto(254.66224474,415.41265292)(254.79701023,414.51031007)(255.06654358,413.91656232)
\curveto(255.33997843,413.32281126)(255.7520874,413.02593656)(256.3028717,413.02593732)
\curveto(256.85755504,413.02593656)(257.269664,413.32281126)(257.53919983,413.91656232)
\curveto(257.81263221,414.51031007)(257.94935082,415.41265292)(257.94935608,416.62359357)
\curveto(257.94935082,417.83843174)(257.81263221,418.74272771)(257.53919983,419.3364842)
\curveto(257.269664,419.93022653)(256.85755504,420.22710123)(256.3028717,420.2271092)
\moveto(256.3028717,421.1646092)
\curveto(257.23646091,421.16460029)(257.94153833,420.78178818)(258.41810608,420.0161717)
\curveto(258.89856862,419.25053971)(259.13880276,418.11968146)(259.1388092,416.62359357)
\curveto(259.13880276,415.1314032)(258.89856862,414.00249808)(258.41810608,413.23687482)
\curveto(257.94153833,412.47124961)(257.23646091,412.08843749)(256.3028717,412.08843732)
\curveto(255.36927528,412.08843749)(254.66419786,412.47124961)(254.18763733,413.23687482)
\curveto(253.71107381,414.00249808)(253.4727928,415.1314032)(253.47279358,416.62359357)
\curveto(253.4727928,418.11968146)(253.71107381,419.25053971)(254.18763733,420.0161717)
\curveto(254.66419786,420.78178818)(255.36927528,421.16460029)(256.3028717,421.1646092)
}
}
{
\newrgbcolor{curcolor}{0 0 0}
\pscustom[linestyle=none,fillstyle=solid,fillcolor=curcolor]
{
\newpath
\moveto(287.24414062,416.51934553)
\curveto(287.24413779,416.73418467)(287.31835647,416.91973136)(287.46679688,417.07598615)
\curveto(287.61913742,417.23223104)(287.80077786,417.31035597)(288.01171875,417.31036115)
\curveto(288.23046493,417.31035597)(288.41796475,417.23223104)(288.57421875,417.07598615)
\curveto(288.73046443,416.91973136)(288.80858936,416.73418467)(288.80859375,416.51934553)
\curveto(288.80858936,416.30059135)(288.73046443,416.11504466)(288.57421875,415.9627049)
\curveto(288.42187099,415.81035747)(288.23437118,415.73418567)(288.01171875,415.73418928)
\curveto(287.79296537,415.73418567)(287.6093718,415.80840434)(287.4609375,415.95684553)
\curveto(287.31640335,416.10527905)(287.24413779,416.29277886)(287.24414062,416.51934553)
\moveto(288.0234375,420.09356428)
\curveto(287.47265319,420.09355631)(287.06054423,419.79668161)(286.78710938,419.20293928)
\curveto(286.51757602,418.60918279)(286.38281053,417.70488682)(286.3828125,416.49004865)
\curveto(286.38281053,415.279108)(286.51757602,414.37676515)(286.78710938,413.7830174)
\curveto(287.06054423,413.18926634)(287.47265319,412.89239163)(288.0234375,412.8923924)
\curveto(288.57812084,412.89239163)(288.9902298,413.18926634)(289.25976562,413.7830174)
\curveto(289.53319801,414.37676515)(289.66991662,415.279108)(289.66992188,416.49004865)
\curveto(289.66991662,417.70488682)(289.53319801,418.60918279)(289.25976562,419.20293928)
\curveto(288.9902298,419.79668161)(288.57812084,420.09355631)(288.0234375,420.09356428)
\moveto(288.0234375,421.03106428)
\curveto(288.95702671,421.03105537)(289.66210413,420.64824325)(290.13867188,419.88262678)
\curveto(290.61913442,419.11699479)(290.85936855,417.98613654)(290.859375,416.49004865)
\curveto(290.85936855,414.99785828)(290.61913442,413.86895316)(290.13867188,413.1033299)
\curveto(289.66210413,412.33770469)(288.95702671,411.95489257)(288.0234375,411.9548924)
\curveto(287.08984107,411.95489257)(286.38476365,412.33770469)(285.90820312,413.1033299)
\curveto(285.43163961,413.86895316)(285.1933586,414.99785828)(285.19335938,416.49004865)
\curveto(285.1933586,417.98613654)(285.43163961,419.11699479)(285.90820312,419.88262678)
\curveto(286.38476365,420.64824325)(287.08984107,421.03105537)(288.0234375,421.03106428)
}
}
{
\newrgbcolor{curcolor}{0 0 0}
\pscustom[linestyle=none,fillstyle=solid,fillcolor=curcolor]
{
\newpath
\moveto(322.37249756,420.08184553)
\lineto(319.61273193,415.45879865)
\lineto(322.37249756,415.45879865)
\lineto(322.37249756,420.08184553)
\moveto(322.17913818,421.15997053)
\lineto(323.55023193,421.15997053)
\lineto(323.55023193,415.45879865)
\lineto(324.71624756,415.45879865)
\lineto(324.71624756,414.49786115)
\lineto(323.55023193,414.49786115)
\lineto(323.55023193,412.41192365)
\lineto(322.37249756,412.41192365)
\lineto(322.37249756,414.49786115)
\lineto(318.66351318,414.49786115)
\lineto(318.66351318,415.61700178)
\lineto(322.17913818,421.15997053)
}
}
{
\newrgbcolor{curcolor}{0 0 0}
\pscustom[linestyle=none,fillstyle=solid,fillcolor=curcolor]
{
\newpath
\moveto(223.32841492,399.43560529)
\curveto(223.32841209,399.65044443)(223.40263076,399.83599112)(223.55107117,399.99224592)
\curveto(223.70341171,400.14849081)(223.88505216,400.22661573)(224.09599304,400.22662092)
\curveto(224.31473923,400.22661573)(224.50223904,400.14849081)(224.65849304,399.99224592)
\curveto(224.81473873,399.83599112)(224.89286365,399.65044443)(224.89286804,399.43560529)
\curveto(224.89286365,399.21685112)(224.81473873,399.03130443)(224.65849304,398.87896467)
\curveto(224.50614528,398.72661723)(224.31864547,398.65044543)(224.09599304,398.65044904)
\curveto(223.87723966,398.65044543)(223.6936461,398.72466411)(223.54521179,398.87310529)
\curveto(223.40067764,399.02153881)(223.32841209,399.20903863)(223.32841492,399.43560529)
\moveto(224.10771179,403.00982404)
\curveto(223.55692748,403.00981607)(223.14481852,402.71294137)(222.87138367,402.11919904)
\curveto(222.60185031,401.52544256)(222.46708482,400.62114659)(222.46708679,399.40630842)
\curveto(222.46708482,398.19536776)(222.60185031,397.29302492)(222.87138367,396.69927717)
\curveto(223.14481852,396.1055261)(223.55692748,395.8086514)(224.10771179,395.80865217)
\curveto(224.66239513,395.8086514)(225.07450409,396.1055261)(225.34403992,396.69927717)
\curveto(225.6174723,397.29302492)(225.75419091,398.19536776)(225.75419617,399.40630842)
\curveto(225.75419091,400.62114659)(225.6174723,401.52544256)(225.34403992,402.11919904)
\curveto(225.07450409,402.71294137)(224.66239513,403.00981607)(224.10771179,403.00982404)
\moveto(224.10771179,403.94732404)
\curveto(225.041301,403.94731514)(225.74637842,403.56450302)(226.22294617,402.79888654)
\curveto(226.70340871,402.03325455)(226.94364285,400.90239631)(226.94364929,399.40630842)
\curveto(226.94364285,397.91411804)(226.70340871,396.78521292)(226.22294617,396.01958967)
\curveto(225.74637842,395.25396446)(225.041301,394.87115234)(224.10771179,394.87115217)
\curveto(223.17411537,394.87115234)(222.46903795,395.25396446)(221.99247742,396.01958967)
\curveto(221.5159139,396.78521292)(221.27763289,397.91411804)(221.27763367,399.40630842)
\curveto(221.27763289,400.90239631)(221.5159139,402.03325455)(221.99247742,402.79888654)
\curveto(222.46903795,403.56450302)(223.17411537,403.94731514)(224.10771179,403.94732404)
}
}
{
\newrgbcolor{curcolor}{0 0 0}
\pscustom[linestyle=none,fillstyle=solid,fillcolor=curcolor]
{
\newpath
\moveto(255.11927795,396.23016096)
\lineto(259.1388092,396.23016096)
\lineto(259.1388092,395.23406721)
\lineto(253.82435608,395.23406721)
\lineto(253.82435608,396.23016096)
\curveto(254.55482321,396.99969044)(255.19349444,397.67937726)(255.7403717,398.26922346)
\curveto(256.28724335,398.85906358)(256.6641961,399.27507879)(256.87123108,399.51727033)
\curveto(257.26185175,399.99382807)(257.52552336,400.37859331)(257.6622467,400.67156721)
\curveto(257.79896059,400.96843647)(257.8673199,401.27117054)(257.86732483,401.57977033)
\curveto(257.8673199,402.06804475)(257.72278879,402.45085687)(257.43373108,402.72820783)
\curveto(257.14857061,403.00554381)(256.75599288,403.14421555)(256.2559967,403.14422346)
\curveto(255.90052499,403.14421555)(255.52747849,403.07976249)(255.13685608,402.95086408)
\curveto(254.74622927,402.82195024)(254.33216718,402.62663794)(253.89466858,402.36492658)
\lineto(253.89466858,403.56023908)
\curveto(254.29701097,403.75163681)(254.69154182,403.89616792)(255.07826233,403.99383283)
\curveto(255.46888479,404.09148022)(255.85365003,404.1403083)(256.2325592,404.14031721)
\curveto(257.0880238,404.1403083)(257.77552311,403.9117929)(258.2950592,403.45477033)
\curveto(258.81849082,403.00163756)(259.08020931,402.40593504)(259.08021545,401.66766096)
\curveto(259.08020931,401.2926549)(258.99231877,400.91765527)(258.81654358,400.54266096)
\curveto(258.64466287,400.16765602)(258.36341315,399.75359394)(257.97279358,399.30047346)
\curveto(257.75403876,399.04656339)(257.4356797,398.69500125)(257.01771545,398.24578596)
\curveto(256.60364928,397.79656464)(255.97083742,397.12469032)(255.11927795,396.23016096)
}
}
{
\newrgbcolor{curcolor}{0 0 0}
\pscustom[linestyle=none,fillstyle=solid,fillcolor=curcolor]
{
\newpath
\moveto(287.24414062,399.50115705)
\curveto(287.24413779,399.71599619)(287.31835647,399.90154288)(287.46679688,400.05779768)
\curveto(287.61913742,400.21404257)(287.80077786,400.29216749)(288.01171875,400.29217268)
\curveto(288.23046493,400.29216749)(288.41796475,400.21404257)(288.57421875,400.05779768)
\curveto(288.73046443,399.90154288)(288.80858936,399.71599619)(288.80859375,399.50115705)
\curveto(288.80858936,399.28240288)(288.73046443,399.09685619)(288.57421875,398.94451643)
\curveto(288.42187099,398.79216899)(288.23437118,398.71599719)(288.01171875,398.7160008)
\curveto(287.79296537,398.71599719)(287.6093718,398.79021587)(287.4609375,398.93865705)
\curveto(287.31640335,399.08709057)(287.24413779,399.27459038)(287.24414062,399.50115705)
\moveto(288.0234375,403.0753758)
\curveto(287.47265319,403.07536783)(287.06054423,402.77849313)(286.78710938,402.1847508)
\curveto(286.51757602,401.59099432)(286.38281053,400.68669835)(286.3828125,399.47186018)
\curveto(286.38281053,398.26091952)(286.51757602,397.35857667)(286.78710938,396.76482893)
\curveto(287.06054423,396.17107786)(287.47265319,395.87420316)(288.0234375,395.87420393)
\curveto(288.57812084,395.87420316)(288.9902298,396.17107786)(289.25976562,396.76482893)
\curveto(289.53319801,397.35857667)(289.66991662,398.26091952)(289.66992188,399.47186018)
\curveto(289.66991662,400.68669835)(289.53319801,401.59099432)(289.25976562,402.1847508)
\curveto(288.9902298,402.77849313)(288.57812084,403.07536783)(288.0234375,403.0753758)
\moveto(288.0234375,404.0128758)
\curveto(288.95702671,404.01286689)(289.66210413,403.63005478)(290.13867188,402.8644383)
\curveto(290.61913442,402.09880631)(290.85936855,400.96794806)(290.859375,399.47186018)
\curveto(290.85936855,397.9796698)(290.61913442,396.85076468)(290.13867188,396.08514143)
\curveto(289.66210413,395.31951621)(288.95702671,394.9367041)(288.0234375,394.93670393)
\curveto(287.08984107,394.9367041)(286.38476365,395.31951621)(285.90820312,396.08514143)
\curveto(285.43163961,396.85076468)(285.1933586,397.9796698)(285.19335938,399.47186018)
\curveto(285.1933586,400.96794806)(285.43163961,402.09880631)(285.90820312,402.8644383)
\curveto(286.38476365,403.63005478)(287.08984107,404.01286689)(288.0234375,404.0128758)
}
}
{
\newrgbcolor{curcolor}{0 0 0}
\pscustom[linestyle=none,fillstyle=solid,fillcolor=curcolor]
{
\newpath
\moveto(312.63421631,396.36248518)
\lineto(314.47406006,396.36248518)
\lineto(314.47406006,403.04803205)
\lineto(312.49359131,402.60271955)
\lineto(312.49359131,403.68084455)
\lineto(314.46234131,404.1144383)
\lineto(315.64593506,404.1144383)
\lineto(315.64593506,396.36248518)
\lineto(317.46234131,396.36248518)
\lineto(317.46234131,395.36639143)
\lineto(312.63421631,395.36639143)
\lineto(312.63421631,396.36248518)
}
}
{
\newrgbcolor{curcolor}{0 0 0}
\pscustom[linestyle=none,fillstyle=solid,fillcolor=curcolor]
{
\newpath
\moveto(321.10101318,399.76092268)
\curveto(321.10101035,399.97576182)(321.17522903,400.16130851)(321.32366943,400.3175633)
\curveto(321.47600998,400.47380819)(321.65765042,400.55193312)(321.86859131,400.5519383)
\curveto(322.08733749,400.55193312)(322.2748373,400.47380819)(322.43109131,400.3175633)
\curveto(322.58733699,400.16130851)(322.66546191,399.97576182)(322.66546631,399.76092268)
\curveto(322.66546191,399.5421685)(322.58733699,399.35662181)(322.43109131,399.20428205)
\curveto(322.27874355,399.05193462)(322.09124374,398.97576282)(321.86859131,398.97576643)
\curveto(321.64983793,398.97576282)(321.46624436,399.04998149)(321.31781006,399.19842268)
\curveto(321.17327591,399.3468562)(321.10101035,399.53435601)(321.10101318,399.76092268)
\moveto(321.88031006,403.33514143)
\curveto(321.32952575,403.33513346)(320.91741679,403.03825875)(320.64398193,402.44451643)
\curveto(320.37444858,401.85075994)(320.23968309,400.94646397)(320.23968506,399.7316258)
\curveto(320.23968309,398.52068515)(320.37444858,397.6183423)(320.64398193,397.02459455)
\curveto(320.91741679,396.43084349)(321.32952575,396.13396878)(321.88031006,396.13396955)
\curveto(322.43499339,396.13396878)(322.84710236,396.43084349)(323.11663818,397.02459455)
\curveto(323.39007056,397.6183423)(323.52678918,398.52068515)(323.52679443,399.7316258)
\curveto(323.52678918,400.94646397)(323.39007056,401.85075994)(323.11663818,402.44451643)
\curveto(322.84710236,403.03825875)(322.43499339,403.33513346)(321.88031006,403.33514143)
\moveto(321.88031006,404.27264143)
\curveto(322.81389927,404.27263252)(323.51897669,403.8898204)(323.99554443,403.12420393)
\curveto(324.47600698,402.35857193)(324.71624111,401.22771369)(324.71624756,399.7316258)
\curveto(324.71624111,398.23943543)(324.47600698,397.11053031)(323.99554443,396.34490705)
\curveto(323.51897669,395.57928184)(322.81389927,395.19646972)(321.88031006,395.19646955)
\curveto(320.94671363,395.19646972)(320.24163621,395.57928184)(319.76507568,396.34490705)
\curveto(319.28851217,397.11053031)(319.05023115,398.23943543)(319.05023193,399.7316258)
\curveto(319.05023115,401.22771369)(319.28851217,402.35857193)(319.76507568,403.12420393)
\curveto(320.24163621,403.8898204)(320.94671363,404.27263252)(321.88031006,404.27264143)
}
}
{
\newrgbcolor{curcolor}{0 0 0}
\pscustom[linestyle=none,fillstyle=solid,fillcolor=curcolor]
{
\newpath
\moveto(223.32841492,382.38384748)
\curveto(223.32841209,382.59868662)(223.40263076,382.78423331)(223.55107117,382.94048811)
\curveto(223.70341171,383.096733)(223.88505216,383.17485792)(224.09599304,383.17486311)
\curveto(224.31473923,383.17485792)(224.50223904,383.096733)(224.65849304,382.94048811)
\curveto(224.81473873,382.78423331)(224.89286365,382.59868662)(224.89286804,382.38384748)
\curveto(224.89286365,382.1650933)(224.81473873,381.97954662)(224.65849304,381.82720686)
\curveto(224.50614528,381.67485942)(224.31864547,381.59868762)(224.09599304,381.59869123)
\curveto(223.87723966,381.59868762)(223.6936461,381.6729063)(223.54521179,381.82134748)
\curveto(223.40067764,381.969781)(223.32841209,382.15728081)(223.32841492,382.38384748)
\moveto(224.10771179,385.95806623)
\curveto(223.55692748,385.95805826)(223.14481852,385.66118356)(222.87138367,385.06744123)
\curveto(222.60185031,384.47368475)(222.46708482,383.56938878)(222.46708679,382.35455061)
\curveto(222.46708482,381.14360995)(222.60185031,380.2412671)(222.87138367,379.64751936)
\curveto(223.14481852,379.05376829)(223.55692748,378.75689359)(224.10771179,378.75689436)
\curveto(224.66239513,378.75689359)(225.07450409,379.05376829)(225.34403992,379.64751936)
\curveto(225.6174723,380.2412671)(225.75419091,381.14360995)(225.75419617,382.35455061)
\curveto(225.75419091,383.56938878)(225.6174723,384.47368475)(225.34403992,385.06744123)
\curveto(225.07450409,385.66118356)(224.66239513,385.95805826)(224.10771179,385.95806623)
\moveto(224.10771179,386.89556623)
\curveto(225.041301,386.89555732)(225.74637842,386.51274521)(226.22294617,385.74712873)
\curveto(226.70340871,384.98149674)(226.94364285,383.85063849)(226.94364929,382.35455061)
\curveto(226.94364285,380.86236023)(226.70340871,379.73345511)(226.22294617,378.96783186)
\curveto(225.74637842,378.20220664)(225.041301,377.81939453)(224.10771179,377.81939436)
\curveto(223.17411537,377.81939453)(222.46903795,378.20220664)(221.99247742,378.96783186)
\curveto(221.5159139,379.73345511)(221.27763289,380.86236023)(221.27763367,382.35455061)
\curveto(221.27763289,383.85063849)(221.5159139,384.98149674)(221.99247742,385.74712873)
\curveto(222.46903795,386.51274521)(223.17411537,386.89555732)(224.10771179,386.89556623)
}
}
{
\newrgbcolor{curcolor}{0 0 0}
\pscustom[linestyle=none,fillstyle=solid,fillcolor=curcolor]
{
\newpath
\moveto(255.52357483,382.77435041)
\curveto(255.523572,382.98918955)(255.59779067,383.17473624)(255.74623108,383.33099104)
\curveto(255.89857162,383.48723593)(256.08021207,383.56536085)(256.29115295,383.56536604)
\curveto(256.50989914,383.56536085)(256.69739895,383.48723593)(256.85365295,383.33099104)
\curveto(257.00989864,383.17473624)(257.08802356,382.98918955)(257.08802795,382.77435041)
\curveto(257.08802356,382.55559623)(257.00989864,382.37004954)(256.85365295,382.21770979)
\curveto(256.7013052,382.06536235)(256.51380538,381.98919055)(256.29115295,381.98919416)
\curveto(256.07239958,381.98919055)(255.88880601,382.06340923)(255.7403717,382.21185041)
\curveto(255.59583755,382.36028393)(255.523572,382.54778374)(255.52357483,382.77435041)
\moveto(256.3028717,386.34856916)
\curveto(255.7520874,386.34856119)(255.33997843,386.05168649)(255.06654358,385.45794416)
\curveto(254.79701023,384.86418768)(254.66224474,383.95989171)(254.6622467,382.74505354)
\curveto(254.66224474,381.53411288)(254.79701023,380.63177003)(255.06654358,380.03802229)
\curveto(255.33997843,379.44427122)(255.7520874,379.14739652)(256.3028717,379.14739729)
\curveto(256.85755504,379.14739652)(257.269664,379.44427122)(257.53919983,380.03802229)
\curveto(257.81263221,380.63177003)(257.94935082,381.53411288)(257.94935608,382.74505354)
\curveto(257.94935082,383.95989171)(257.81263221,384.86418768)(257.53919983,385.45794416)
\curveto(257.269664,386.05168649)(256.85755504,386.34856119)(256.3028717,386.34856916)
\moveto(256.3028717,387.28606916)
\curveto(257.23646091,387.28606025)(257.94153833,386.90324814)(258.41810608,386.13763166)
\curveto(258.89856862,385.37199967)(259.13880276,384.24114142)(259.1388092,382.74505354)
\curveto(259.13880276,381.25286316)(258.89856862,380.12395804)(258.41810608,379.35833479)
\curveto(257.94153833,378.59270957)(257.23646091,378.20989746)(256.3028717,378.20989729)
\curveto(255.36927528,378.20989746)(254.66419786,378.59270957)(254.18763733,379.35833479)
\curveto(253.71107381,380.12395804)(253.4727928,381.25286316)(253.47279358,382.74505354)
\curveto(253.4727928,384.24114142)(253.71107381,385.37199967)(254.18763733,386.13763166)
\curveto(254.66419786,386.90324814)(255.36927528,387.28606025)(256.3028717,387.28606916)
}
}
{
\newrgbcolor{curcolor}{0 0 0}
\pscustom[linestyle=none,fillstyle=solid,fillcolor=curcolor]
{
\newpath
\moveto(287.24414062,382.48296857)
\curveto(287.24413779,382.69780771)(287.31835647,382.8833544)(287.46679688,383.0396092)
\curveto(287.61913742,383.19585409)(287.80077786,383.27397901)(288.01171875,383.2739842)
\curveto(288.23046493,383.27397901)(288.41796475,383.19585409)(288.57421875,383.0396092)
\curveto(288.73046443,382.8833544)(288.80858936,382.69780771)(288.80859375,382.48296857)
\curveto(288.80858936,382.2642144)(288.73046443,382.07866771)(288.57421875,381.92632795)
\curveto(288.42187099,381.77398051)(288.23437118,381.69780871)(288.01171875,381.69781232)
\curveto(287.79296537,381.69780871)(287.6093718,381.77202739)(287.4609375,381.92046857)
\curveto(287.31640335,382.06890209)(287.24413779,382.25640191)(287.24414062,382.48296857)
\moveto(288.0234375,386.05718732)
\curveto(287.47265319,386.05717936)(287.06054423,385.76030465)(286.78710938,385.16656232)
\curveto(286.51757602,384.57280584)(286.38281053,383.66850987)(286.3828125,382.4536717)
\curveto(286.38281053,381.24273104)(286.51757602,380.3403882)(286.78710938,379.74664045)
\curveto(287.06054423,379.15288938)(287.47265319,378.85601468)(288.0234375,378.85601545)
\curveto(288.57812084,378.85601468)(288.9902298,379.15288938)(289.25976562,379.74664045)
\curveto(289.53319801,380.3403882)(289.66991662,381.24273104)(289.66992188,382.4536717)
\curveto(289.66991662,383.66850987)(289.53319801,384.57280584)(289.25976562,385.16656232)
\curveto(288.9902298,385.76030465)(288.57812084,386.05717936)(288.0234375,386.05718732)
\moveto(288.0234375,386.99468732)
\curveto(288.95702671,386.99467842)(289.66210413,386.6118663)(290.13867188,385.84624982)
\curveto(290.61913442,385.08061783)(290.85936855,383.94975959)(290.859375,382.4536717)
\curveto(290.85936855,380.96148133)(290.61913442,379.83257621)(290.13867188,379.06695295)
\curveto(289.66210413,378.30132774)(288.95702671,377.91851562)(288.0234375,377.91851545)
\curveto(287.08984107,377.91851562)(286.38476365,378.30132774)(285.90820312,379.06695295)
\curveto(285.43163961,379.83257621)(285.1933586,380.96148133)(285.19335938,382.4536717)
\curveto(285.1933586,383.94975959)(285.43163961,385.08061783)(285.90820312,385.84624982)
\curveto(286.38476365,386.6118663)(287.08984107,386.99467842)(288.0234375,386.99468732)
}
}
{
\newrgbcolor{curcolor}{0 0 0}
\pscustom[linestyle=none,fillstyle=solid,fillcolor=curcolor]
{
\newpath
\moveto(320.69671631,379.14703107)
\lineto(324.71624756,379.14703107)
\lineto(324.71624756,378.15093732)
\lineto(319.40179443,378.15093732)
\lineto(319.40179443,379.14703107)
\curveto(320.13226156,379.91656056)(320.7709328,380.59624738)(321.31781006,381.18609357)
\curveto(321.86468171,381.7759337)(322.24163445,382.19194891)(322.44866943,382.43414045)
\curveto(322.83929011,382.91069819)(323.10296172,383.29546343)(323.23968506,383.58843732)
\curveto(323.37639894,383.88530659)(323.44475825,384.18804066)(323.44476318,384.49664045)
\curveto(323.44475825,384.98491487)(323.30022714,385.36772698)(323.01116943,385.64507795)
\curveto(322.72600897,385.92241393)(322.33343124,386.06108566)(321.83343506,386.06109357)
\curveto(321.47796334,386.06108566)(321.10491684,385.9966326)(320.71429443,385.8677342)
\curveto(320.32366762,385.73882036)(319.90960554,385.54350806)(319.47210693,385.2817967)
\lineto(319.47210693,386.4771092)
\curveto(319.87444932,386.66850693)(320.26898018,386.81303804)(320.65570068,386.91070295)
\curveto(321.04632315,387.00835034)(321.43108839,387.05717842)(321.80999756,387.05718732)
\curveto(322.66546215,387.05717842)(323.35296147,386.82866302)(323.87249756,386.37164045)
\curveto(324.39592917,385.91850768)(324.65764766,385.32280515)(324.65765381,384.58453107)
\curveto(324.65764766,384.20952502)(324.56975713,383.83452539)(324.39398193,383.45953107)
\curveto(324.22210122,383.08452614)(323.9408515,382.67046405)(323.55023193,382.21734357)
\curveto(323.33147711,381.96343351)(323.01311806,381.61187136)(322.59515381,381.16265607)
\curveto(322.18108764,380.71343476)(321.54827577,380.04156043)(320.69671631,379.14703107)
}
}
{
\newrgbcolor{curcolor}{0 0 0}
\pscustom[linestyle=none,fillstyle=solid,fillcolor=curcolor]
{
\newpath
\moveto(222.92411804,361.93365217)
\lineto(226.94364929,361.93365217)
\lineto(226.94364929,360.93755842)
\lineto(221.62919617,360.93755842)
\lineto(221.62919617,361.93365217)
\curveto(222.3596633,362.70318165)(222.99833453,363.38286847)(223.54521179,363.97271467)
\curveto(224.09208344,364.56255479)(224.46903619,364.97857)(224.67607117,365.22076154)
\curveto(225.06669184,365.69731928)(225.33036345,366.08208452)(225.46708679,366.37505842)
\curveto(225.60380068,366.67192768)(225.67215998,366.97466176)(225.67216492,367.28326154)
\curveto(225.67215998,367.77153596)(225.52762888,368.15434808)(225.23857117,368.43169904)
\curveto(224.9534107,368.70903502)(224.56083297,368.84770676)(224.06083679,368.84771467)
\curveto(223.70536508,368.84770676)(223.33231857,368.7832537)(222.94169617,368.65435529)
\curveto(222.55106935,368.52544146)(222.13700727,368.33012915)(221.69950867,368.06841779)
\lineto(221.69950867,369.26373029)
\curveto(222.10185105,369.45512803)(222.49638191,369.59965913)(222.88310242,369.69732404)
\curveto(223.27372488,369.79497144)(223.65849012,369.84379951)(224.03739929,369.84380842)
\curveto(224.89286389,369.84379951)(225.5803632,369.61528412)(226.09989929,369.15826154)
\curveto(226.62333091,368.70512878)(226.8850494,368.10942625)(226.88505554,367.37115217)
\curveto(226.8850494,366.99614611)(226.79715886,366.62114648)(226.62138367,366.24615217)
\curveto(226.44950296,365.87114723)(226.16825324,365.45708515)(225.77763367,365.00396467)
\curveto(225.55887885,364.75005461)(225.24051979,364.39849246)(224.82255554,363.94927717)
\curveto(224.40848937,363.50005586)(223.7756775,362.82818153)(222.92411804,361.93365217)
}
}
{
\newrgbcolor{curcolor}{0 0 0}
\pscustom[linestyle=none,fillstyle=solid,fillcolor=curcolor]
{
\newpath
\moveto(255.52357483,365.75005842)
\curveto(255.523572,365.96489756)(255.59779067,366.15044425)(255.74623108,366.30669904)
\curveto(255.89857162,366.46294394)(256.08021207,366.54106886)(256.29115295,366.54107404)
\curveto(256.50989914,366.54106886)(256.69739895,366.46294394)(256.85365295,366.30669904)
\curveto(257.00989864,366.15044425)(257.08802356,365.96489756)(257.08802795,365.75005842)
\curveto(257.08802356,365.53130424)(257.00989864,365.34575755)(256.85365295,365.19341779)
\curveto(256.7013052,365.04107036)(256.51380538,364.96489856)(256.29115295,364.96490217)
\curveto(256.07239958,364.96489856)(255.88880601,365.03911723)(255.7403717,365.18755842)
\curveto(255.59583755,365.33599194)(255.523572,365.52349175)(255.52357483,365.75005842)
\moveto(256.3028717,369.32427717)
\curveto(255.7520874,369.3242692)(255.33997843,369.0273945)(255.06654358,368.43365217)
\curveto(254.79701023,367.83989568)(254.66224474,366.93559971)(254.6622467,365.72076154)
\curveto(254.66224474,364.50982089)(254.79701023,363.60747804)(255.06654358,363.01373029)
\curveto(255.33997843,362.41997923)(255.7520874,362.12310453)(256.3028717,362.12310529)
\curveto(256.85755504,362.12310453)(257.269664,362.41997923)(257.53919983,363.01373029)
\curveto(257.81263221,363.60747804)(257.94935082,364.50982089)(257.94935608,365.72076154)
\curveto(257.94935082,366.93559971)(257.81263221,367.83989568)(257.53919983,368.43365217)
\curveto(257.269664,369.0273945)(256.85755504,369.3242692)(256.3028717,369.32427717)
\moveto(256.3028717,370.26177717)
\curveto(257.23646091,370.26176826)(257.94153833,369.87895614)(258.41810608,369.11333967)
\curveto(258.89856862,368.34770768)(259.13880276,367.21684943)(259.1388092,365.72076154)
\curveto(259.13880276,364.22857117)(258.89856862,363.09966605)(258.41810608,362.33404279)
\curveto(257.94153833,361.56841758)(257.23646091,361.18560546)(256.3028717,361.18560529)
\curveto(255.36927528,361.18560546)(254.66419786,361.56841758)(254.18763733,362.33404279)
\curveto(253.71107381,363.09966605)(253.4727928,364.22857117)(253.47279358,365.72076154)
\curveto(253.4727928,367.21684943)(253.71107381,368.34770768)(254.18763733,369.11333967)
\curveto(254.66419786,369.87895614)(255.36927528,370.26176826)(256.3028717,370.26177717)
}
}
{
\newrgbcolor{curcolor}{0 0 0}
\pscustom[linestyle=none,fillstyle=solid,fillcolor=curcolor]
{
\newpath
\moveto(287.24414062,365.4647801)
\curveto(287.24413779,365.67961924)(287.31835647,365.86516593)(287.46679688,366.02142072)
\curveto(287.61913742,366.17766562)(287.80077786,366.25579054)(288.01171875,366.25579572)
\curveto(288.23046493,366.25579054)(288.41796475,366.17766562)(288.57421875,366.02142072)
\curveto(288.73046443,365.86516593)(288.80858936,365.67961924)(288.80859375,365.4647801)
\curveto(288.80858936,365.24602592)(288.73046443,365.06047923)(288.57421875,364.90813947)
\curveto(288.42187099,364.75579204)(288.23437118,364.67962024)(288.01171875,364.67962385)
\curveto(287.79296537,364.67962024)(287.6093718,364.75383891)(287.4609375,364.9022801)
\curveto(287.31640335,365.05071362)(287.24413779,365.23821343)(287.24414062,365.4647801)
\moveto(288.0234375,369.03899885)
\curveto(287.47265319,369.03899088)(287.06054423,368.74211618)(286.78710938,368.14837385)
\curveto(286.51757602,367.55461736)(286.38281053,366.65032139)(286.3828125,365.43548322)
\curveto(286.38281053,364.22454257)(286.51757602,363.32219972)(286.78710938,362.72845197)
\curveto(287.06054423,362.13470091)(287.47265319,361.83782621)(288.0234375,361.83782697)
\curveto(288.57812084,361.83782621)(288.9902298,362.13470091)(289.25976562,362.72845197)
\curveto(289.53319801,363.32219972)(289.66991662,364.22454257)(289.66992188,365.43548322)
\curveto(289.66991662,366.65032139)(289.53319801,367.55461736)(289.25976562,368.14837385)
\curveto(288.9902298,368.74211618)(288.57812084,369.03899088)(288.0234375,369.03899885)
\moveto(288.0234375,369.97649885)
\curveto(288.95702671,369.97648994)(289.66210413,369.59367782)(290.13867188,368.82806135)
\curveto(290.61913442,368.06242936)(290.85936855,366.93157111)(290.859375,365.43548322)
\curveto(290.85936855,363.94329285)(290.61913442,362.81438773)(290.13867188,362.04876447)
\curveto(289.66210413,361.28313926)(288.95702671,360.90032714)(288.0234375,360.90032697)
\curveto(287.08984107,360.90032714)(286.38476365,361.28313926)(285.90820312,362.04876447)
\curveto(285.43163961,362.81438773)(285.1933586,363.94329285)(285.19335938,365.43548322)
\curveto(285.1933586,366.93157111)(285.43163961,368.06242936)(285.90820312,368.82806135)
\curveto(286.38476365,369.59367782)(287.08984107,369.97648994)(288.0234375,369.97649885)
}
}
{
\newrgbcolor{curcolor}{0 0 0}
\pscustom[linestyle=none,fillstyle=solid,fillcolor=curcolor]
{
\newpath
\moveto(322.94085693,365.78704572)
\curveto(323.51507056,365.63469744)(323.95452325,365.36321334)(324.25921631,364.9725926)
\curveto(324.56389764,364.58587037)(324.71624124,364.10149585)(324.71624756,363.5194676)
\curveto(324.71624124,362.71477849)(324.44475713,362.08196662)(323.90179443,361.6210301)
\curveto(323.36272696,361.16399879)(322.61468084,360.93548339)(321.65765381,360.93548322)
\curveto(321.2553072,360.93548339)(320.84515136,360.97259273)(320.42718506,361.04681135)
\curveto(320.00921469,361.12103008)(319.59905885,361.22845185)(319.19671631,361.36907697)
\lineto(319.19671631,362.54681135)
\curveto(319.59515261,362.33977886)(319.98773034,362.18548214)(320.37445068,362.08392072)
\curveto(320.76116707,361.98235735)(321.14593231,361.93157615)(321.52874756,361.93157697)
\curveto(322.17718128,361.93157615)(322.67522765,362.07806038)(323.02288818,362.3710301)
\curveto(323.37053946,362.66399729)(323.54436741,363.08587187)(323.54437256,363.6366551)
\curveto(323.54436741,364.14446456)(323.37053946,364.54680791)(323.02288818,364.84368635)
\curveto(322.67522765,365.14446356)(322.204525,365.29485403)(321.61077881,365.29485822)
\lineto(320.70843506,365.29485822)
\lineto(320.70843506,366.26751447)
\lineto(321.61077881,366.26751447)
\curveto(322.1537438,366.26750931)(322.5775715,366.38664982)(322.88226318,366.62493635)
\curveto(323.18694589,366.86321184)(323.33928949,367.19524276)(323.33929443,367.6210301)
\curveto(323.33928949,368.07024188)(323.19671151,368.41399154)(322.91156006,368.6522801)
\curveto(322.63030582,368.89445981)(322.22796247,369.01555344)(321.70452881,369.01556135)
\curveto(321.3568696,369.01555344)(320.99749496,368.97649098)(320.62640381,368.89837385)
\curveto(320.2553082,368.82024113)(319.86663671,368.70305375)(319.46038818,368.54681135)
\lineto(319.46038818,369.6366551)
\curveto(319.93304289,369.76164644)(320.35296435,369.85539635)(320.72015381,369.9179051)
\curveto(321.09124486,369.98039622)(321.41936953,370.01164619)(321.70452881,370.0116551)
\curveto(322.55608715,370.01164619)(323.23577397,369.79680266)(323.74359131,369.36712385)
\curveto(324.2553042,368.94133476)(324.51116332,368.37492908)(324.51116943,367.6679051)
\curveto(324.51116332,367.18743027)(324.37639783,366.78704004)(324.10687256,366.46673322)
\curveto(323.84124211,366.14641568)(323.45257063,365.91985341)(322.94085693,365.78704572)
}
}
{
\newrgbcolor{curcolor}{0 0 0}
\pscustom[linestyle=none,fillstyle=solid,fillcolor=curcolor]
{
\newpath
\moveto(222.11552429,345.21001936)
\lineto(223.95536804,345.21001936)
\lineto(223.95536804,351.89556623)
\lineto(221.97489929,351.45025373)
\lineto(221.97489929,352.52837873)
\lineto(223.94364929,352.96197248)
\lineto(225.12724304,352.96197248)
\lineto(225.12724304,345.21001936)
\lineto(226.94364929,345.21001936)
\lineto(226.94364929,344.21392561)
\lineto(222.11552429,344.21392561)
\lineto(222.11552429,345.21001936)
}
}
{
\newrgbcolor{curcolor}{0 0 0}
\pscustom[linestyle=none,fillstyle=solid,fillcolor=curcolor]
{
\newpath
\moveto(255.52357483,348.7258885)
\curveto(255.523572,348.94072764)(255.59779067,349.12627433)(255.74623108,349.28252912)
\curveto(255.89857162,349.43877401)(256.08021207,349.51689894)(256.29115295,349.51690412)
\curveto(256.50989914,349.51689894)(256.69739895,349.43877401)(256.85365295,349.28252912)
\curveto(257.00989864,349.12627433)(257.08802356,348.94072764)(257.08802795,348.7258885)
\curveto(257.08802356,348.50713432)(257.00989864,348.32158763)(256.85365295,348.16924787)
\curveto(256.7013052,348.01690044)(256.51380538,347.94072864)(256.29115295,347.94073225)
\curveto(256.07239958,347.94072864)(255.88880601,348.01494731)(255.7403717,348.1633885)
\curveto(255.59583755,348.31182202)(255.523572,348.49932183)(255.52357483,348.7258885)
\moveto(256.3028717,352.30010725)
\curveto(255.7520874,352.30009928)(255.33997843,352.00322457)(255.06654358,351.40948225)
\curveto(254.79701023,350.81572576)(254.66224474,349.91142979)(254.6622467,348.69659162)
\curveto(254.66224474,347.48565097)(254.79701023,346.58330812)(255.06654358,345.98956037)
\curveto(255.33997843,345.39580931)(255.7520874,345.0989346)(256.3028717,345.09893537)
\curveto(256.85755504,345.0989346)(257.269664,345.39580931)(257.53919983,345.98956037)
\curveto(257.81263221,346.58330812)(257.94935082,347.48565097)(257.94935608,348.69659162)
\curveto(257.94935082,349.91142979)(257.81263221,350.81572576)(257.53919983,351.40948225)
\curveto(257.269664,352.00322457)(256.85755504,352.30009928)(256.3028717,352.30010725)
\moveto(256.3028717,353.23760725)
\curveto(257.23646091,353.23759834)(257.94153833,352.85478622)(258.41810608,352.08916975)
\curveto(258.89856862,351.32353775)(259.13880276,350.19267951)(259.1388092,348.69659162)
\curveto(259.13880276,347.20440125)(258.89856862,346.07549613)(258.41810608,345.30987287)
\curveto(257.94153833,344.54424766)(257.23646091,344.16143554)(256.3028717,344.16143537)
\curveto(255.36927528,344.16143554)(254.66419786,344.54424766)(254.18763733,345.30987287)
\curveto(253.71107381,346.07549613)(253.4727928,347.20440125)(253.47279358,348.69659162)
\curveto(253.4727928,350.19267951)(253.71107381,351.32353775)(254.18763733,352.08916975)
\curveto(254.66419786,352.85478622)(255.36927528,353.23759834)(256.3028717,353.23760725)
}
}
{
\newrgbcolor{curcolor}{0 0 0}
\pscustom[linestyle=none,fillstyle=solid,fillcolor=curcolor]
{
\newpath
\moveto(287.24414062,348.44659162)
\curveto(287.24413779,348.66143076)(287.31835647,348.84697745)(287.46679688,349.00323225)
\curveto(287.61913742,349.15947714)(287.80077786,349.23760206)(288.01171875,349.23760725)
\curveto(288.23046493,349.23760206)(288.41796475,349.15947714)(288.57421875,349.00323225)
\curveto(288.73046443,348.84697745)(288.80858936,348.66143076)(288.80859375,348.44659162)
\curveto(288.80858936,348.22783745)(288.73046443,348.04229076)(288.57421875,347.889951)
\curveto(288.42187099,347.73760356)(288.23437118,347.66143176)(288.01171875,347.66143537)
\curveto(287.79296537,347.66143176)(287.6093718,347.73565044)(287.4609375,347.88409162)
\curveto(287.31640335,348.03252514)(287.24413779,348.22002495)(287.24414062,348.44659162)
\moveto(288.0234375,352.02081037)
\curveto(287.47265319,352.0208024)(287.06054423,351.7239277)(286.78710938,351.13018537)
\curveto(286.51757602,350.53642889)(286.38281053,349.63213292)(286.3828125,348.41729475)
\curveto(286.38281053,347.20635409)(286.51757602,346.30401124)(286.78710938,345.7102635)
\curveto(287.06054423,345.11651243)(287.47265319,344.81963773)(288.0234375,344.8196385)
\curveto(288.57812084,344.81963773)(288.9902298,345.11651243)(289.25976562,345.7102635)
\curveto(289.53319801,346.30401124)(289.66991662,347.20635409)(289.66992188,348.41729475)
\curveto(289.66991662,349.63213292)(289.53319801,350.53642889)(289.25976562,351.13018537)
\curveto(288.9902298,351.7239277)(288.57812084,352.0208024)(288.0234375,352.02081037)
\moveto(288.0234375,352.95831037)
\curveto(288.95702671,352.95830146)(289.66210413,352.57548935)(290.13867188,351.80987287)
\curveto(290.61913442,351.04424088)(290.85936855,349.91338263)(290.859375,348.41729475)
\curveto(290.85936855,346.92510437)(290.61913442,345.79619925)(290.13867188,345.030576)
\curveto(289.66210413,344.26495078)(288.95702671,343.88213867)(288.0234375,343.8821385)
\curveto(287.08984107,343.88213867)(286.38476365,344.26495078)(285.90820312,345.030576)
\curveto(285.43163961,345.79619925)(285.1933586,346.92510437)(285.19335938,348.41729475)
\curveto(285.1933586,349.91338263)(285.43163961,351.04424088)(285.90820312,351.80987287)
\curveto(286.38476365,352.57548935)(287.08984107,352.95830146)(288.0234375,352.95831037)
}
}
{
\newrgbcolor{curcolor}{0 0 0}
\pscustom[linestyle=none,fillstyle=solid,fillcolor=curcolor]
{
\newpath
\moveto(320.69671631,344.88604475)
\lineto(324.71624756,344.88604475)
\lineto(324.71624756,343.889951)
\lineto(319.40179443,343.889951)
\lineto(319.40179443,344.88604475)
\curveto(320.13226156,345.65557423)(320.7709328,346.33526105)(321.31781006,346.92510725)
\curveto(321.86468171,347.51494737)(322.24163445,347.93096258)(322.44866943,348.17315412)
\curveto(322.83929011,348.64971186)(323.10296172,349.0344771)(323.23968506,349.327451)
\curveto(323.37639894,349.62432026)(323.44475825,349.92705433)(323.44476318,350.23565412)
\curveto(323.44475825,350.72392854)(323.30022714,351.10674065)(323.01116943,351.38409162)
\curveto(322.72600897,351.6614276)(322.33343124,351.80009934)(321.83343506,351.80010725)
\curveto(321.47796334,351.80009934)(321.10491684,351.73564628)(320.71429443,351.60674787)
\curveto(320.32366762,351.47783403)(319.90960554,351.28252173)(319.47210693,351.02081037)
\lineto(319.47210693,352.21612287)
\curveto(319.87444932,352.4075206)(320.26898018,352.55205171)(320.65570068,352.64971662)
\curveto(321.04632315,352.74736401)(321.43108839,352.79619209)(321.80999756,352.796201)
\curveto(322.66546215,352.79619209)(323.35296147,352.56767669)(323.87249756,352.11065412)
\curveto(324.39592917,351.65752135)(324.65764766,351.06181882)(324.65765381,350.32354475)
\curveto(324.65764766,349.94853869)(324.56975713,349.57353906)(324.39398193,349.19854475)
\curveto(324.22210122,348.82353981)(323.9408515,348.40947773)(323.55023193,347.95635725)
\curveto(323.33147711,347.70244718)(323.01311806,347.35088504)(322.59515381,346.90166975)
\curveto(322.18108764,346.45244843)(321.54827577,345.78057411)(320.69671631,344.88604475)
}
}
{
\newrgbcolor{curcolor}{0 0 0}
\pscustom[linestyle=none,fillstyle=solid,fillcolor=curcolor]
{
\newpath
\moveto(223.32841492,331.72662092)
\curveto(223.32841209,331.94146006)(223.40263076,332.12700675)(223.55107117,332.28326154)
\curveto(223.70341171,332.43950644)(223.88505216,332.51763136)(224.09599304,332.51763654)
\curveto(224.31473923,332.51763136)(224.50223904,332.43950644)(224.65849304,332.28326154)
\curveto(224.81473873,332.12700675)(224.89286365,331.94146006)(224.89286804,331.72662092)
\curveto(224.89286365,331.50786674)(224.81473873,331.32232005)(224.65849304,331.16998029)
\curveto(224.50614528,331.01763286)(224.31864547,330.94146106)(224.09599304,330.94146467)
\curveto(223.87723966,330.94146106)(223.6936461,331.01567973)(223.54521179,331.16412092)
\curveto(223.40067764,331.31255444)(223.32841209,331.50005425)(223.32841492,331.72662092)
\moveto(224.10771179,335.30083967)
\curveto(223.55692748,335.3008317)(223.14481852,335.003957)(222.87138367,334.41021467)
\curveto(222.60185031,333.81645818)(222.46708482,332.91216221)(222.46708679,331.69732404)
\curveto(222.46708482,330.48638339)(222.60185031,329.58404054)(222.87138367,328.99029279)
\curveto(223.14481852,328.39654173)(223.55692748,328.09966703)(224.10771179,328.09966779)
\curveto(224.66239513,328.09966703)(225.07450409,328.39654173)(225.34403992,328.99029279)
\curveto(225.6174723,329.58404054)(225.75419091,330.48638339)(225.75419617,331.69732404)
\curveto(225.75419091,332.91216221)(225.6174723,333.81645818)(225.34403992,334.41021467)
\curveto(225.07450409,335.003957)(224.66239513,335.3008317)(224.10771179,335.30083967)
\moveto(224.10771179,336.23833967)
\curveto(225.041301,336.23833076)(225.74637842,335.85551864)(226.22294617,335.08990217)
\curveto(226.70340871,334.32427018)(226.94364285,333.19341193)(226.94364929,331.69732404)
\curveto(226.94364285,330.20513367)(226.70340871,329.07622855)(226.22294617,328.31060529)
\curveto(225.74637842,327.54498008)(225.041301,327.16216796)(224.10771179,327.16216779)
\curveto(223.17411537,327.16216796)(222.46903795,327.54498008)(221.99247742,328.31060529)
\curveto(221.5159139,329.07622855)(221.27763289,330.20513367)(221.27763367,331.69732404)
\curveto(221.27763289,333.19341193)(221.5159139,334.32427018)(221.99247742,335.08990217)
\curveto(222.46903795,335.85551864)(223.17411537,336.23833076)(224.10771179,336.23833967)
}
}
{
\newrgbcolor{curcolor}{0 0 0}
\pscustom[linestyle=none,fillstyle=solid,fillcolor=curcolor]
{
\newpath
\moveto(255.52357483,331.7015965)
\curveto(255.523572,331.91643564)(255.59779067,332.10198233)(255.74623108,332.25823713)
\curveto(255.89857162,332.41448202)(256.08021207,332.49260694)(256.29115295,332.49261213)
\curveto(256.50989914,332.49260694)(256.69739895,332.41448202)(256.85365295,332.25823713)
\curveto(257.00989864,332.10198233)(257.08802356,331.91643564)(257.08802795,331.7015965)
\curveto(257.08802356,331.48284233)(257.00989864,331.29729564)(256.85365295,331.14495588)
\curveto(256.7013052,330.99260844)(256.51380538,330.91643664)(256.29115295,330.91644025)
\curveto(256.07239958,330.91643664)(255.88880601,330.99065532)(255.7403717,331.1390965)
\curveto(255.59583755,331.28753002)(255.523572,331.47502984)(255.52357483,331.7015965)
\moveto(256.3028717,335.27581525)
\curveto(255.7520874,335.27580729)(255.33997843,334.97893258)(255.06654358,334.38519025)
\curveto(254.79701023,333.79143377)(254.66224474,332.8871378)(254.6622467,331.67229963)
\curveto(254.66224474,330.46135897)(254.79701023,329.55901613)(255.06654358,328.96526838)
\curveto(255.33997843,328.37151731)(255.7520874,328.07464261)(256.3028717,328.07464338)
\curveto(256.85755504,328.07464261)(257.269664,328.37151731)(257.53919983,328.96526838)
\curveto(257.81263221,329.55901613)(257.94935082,330.46135897)(257.94935608,331.67229963)
\curveto(257.94935082,332.8871378)(257.81263221,333.79143377)(257.53919983,334.38519025)
\curveto(257.269664,334.97893258)(256.85755504,335.27580729)(256.3028717,335.27581525)
\moveto(256.3028717,336.21331525)
\curveto(257.23646091,336.21330635)(257.94153833,335.83049423)(258.41810608,335.06487775)
\curveto(258.89856862,334.29924576)(259.13880276,333.16838752)(259.1388092,331.67229963)
\curveto(259.13880276,330.18010926)(258.89856862,329.05120413)(258.41810608,328.28558088)
\curveto(257.94153833,327.51995567)(257.23646091,327.13714355)(256.3028717,327.13714338)
\curveto(255.36927528,327.13714355)(254.66419786,327.51995567)(254.18763733,328.28558088)
\curveto(253.71107381,329.05120413)(253.4727928,330.18010926)(253.47279358,331.67229963)
\curveto(253.4727928,333.16838752)(253.71107381,334.29924576)(254.18763733,335.06487775)
\curveto(254.66419786,335.83049423)(255.36927528,336.21330635)(256.3028717,336.21331525)
}
}
{
\newrgbcolor{curcolor}{0 0 0}
\pscustom[linestyle=none,fillstyle=solid,fillcolor=curcolor]
{
\newpath
\moveto(287.24414062,331.42840314)
\curveto(287.24413779,331.64324229)(287.31835647,331.82878897)(287.46679688,331.98504377)
\curveto(287.61913742,332.14128866)(287.80077786,332.21941358)(288.01171875,332.21941877)
\curveto(288.23046493,332.21941358)(288.41796475,332.14128866)(288.57421875,331.98504377)
\curveto(288.73046443,331.82878897)(288.80858936,331.64324229)(288.80859375,331.42840314)
\curveto(288.80858936,331.20964897)(288.73046443,331.02410228)(288.57421875,330.87176252)
\curveto(288.42187099,330.71941508)(288.23437118,330.64324329)(288.01171875,330.64324689)
\curveto(287.79296537,330.64324329)(287.6093718,330.71746196)(287.4609375,330.86590314)
\curveto(287.31640335,331.01433666)(287.24413779,331.20183648)(287.24414062,331.42840314)
\moveto(288.0234375,335.00262189)
\curveto(287.47265319,335.00261393)(287.06054423,334.70573922)(286.78710938,334.11199689)
\curveto(286.51757602,333.51824041)(286.38281053,332.61394444)(286.3828125,331.39910627)
\curveto(286.38281053,330.18816562)(286.51757602,329.28582277)(286.78710938,328.69207502)
\curveto(287.06054423,328.09832396)(287.47265319,327.80144925)(288.0234375,327.80145002)
\curveto(288.57812084,327.80144925)(288.9902298,328.09832396)(289.25976562,328.69207502)
\curveto(289.53319801,329.28582277)(289.66991662,330.18816562)(289.66992188,331.39910627)
\curveto(289.66991662,332.61394444)(289.53319801,333.51824041)(289.25976562,334.11199689)
\curveto(288.9902298,334.70573922)(288.57812084,335.00261393)(288.0234375,335.00262189)
\moveto(288.0234375,335.94012189)
\curveto(288.95702671,335.94011299)(289.66210413,335.55730087)(290.13867188,334.79168439)
\curveto(290.61913442,334.0260524)(290.85936855,332.89519416)(290.859375,331.39910627)
\curveto(290.85936855,329.9069159)(290.61913442,328.77801078)(290.13867188,328.01238752)
\curveto(289.66210413,327.24676231)(288.95702671,326.86395019)(288.0234375,326.86395002)
\curveto(287.08984107,326.86395019)(286.38476365,327.24676231)(285.90820312,328.01238752)
\curveto(285.43163961,328.77801078)(285.1933586,329.9069159)(285.19335938,331.39910627)
\curveto(285.1933586,332.89519416)(285.43163961,334.0260524)(285.90820312,334.79168439)
\curveto(286.38476365,335.55730087)(287.08984107,335.94011299)(288.0234375,335.94012189)
}
}
{
\newrgbcolor{curcolor}{0 0 0}
\pscustom[linestyle=none,fillstyle=solid,fillcolor=curcolor]
{
\newpath
\moveto(319.88812256,327.99871564)
\lineto(321.72796631,327.99871564)
\lineto(321.72796631,334.68426252)
\lineto(319.74749756,334.23895002)
\lineto(319.74749756,335.31707502)
\lineto(321.71624756,335.75066877)
\lineto(322.89984131,335.75066877)
\lineto(322.89984131,327.99871564)
\lineto(324.71624756,327.99871564)
\lineto(324.71624756,327.00262189)
\lineto(319.88812256,327.00262189)
\lineto(319.88812256,327.99871564)
}
}
{
\newrgbcolor{curcolor}{0 0 0}
\pscustom[linestyle=none,fillstyle=solid,fillcolor=curcolor]
{
\newpath
\moveto(223.32841492,314.67486311)
\curveto(223.32841209,314.88970225)(223.40263076,315.07524894)(223.55107117,315.23150373)
\curveto(223.70341171,315.38774862)(223.88505216,315.46587354)(224.09599304,315.46587873)
\curveto(224.31473923,315.46587354)(224.50223904,315.38774862)(224.65849304,315.23150373)
\curveto(224.81473873,315.07524894)(224.89286365,314.88970225)(224.89286804,314.67486311)
\curveto(224.89286365,314.45610893)(224.81473873,314.27056224)(224.65849304,314.11822248)
\curveto(224.50614528,313.96587504)(224.31864547,313.88970325)(224.09599304,313.88970686)
\curveto(223.87723966,313.88970325)(223.6936461,313.96392192)(223.54521179,314.11236311)
\curveto(223.40067764,314.26079663)(223.32841209,314.44829644)(223.32841492,314.67486311)
\moveto(224.10771179,318.24908186)
\curveto(223.55692748,318.24907389)(223.14481852,317.95219918)(222.87138367,317.35845686)
\curveto(222.60185031,316.76470037)(222.46708482,315.8604044)(222.46708679,314.64556623)
\curveto(222.46708482,313.43462558)(222.60185031,312.53228273)(222.87138367,311.93853498)
\curveto(223.14481852,311.34478392)(223.55692748,311.04790921)(224.10771179,311.04790998)
\curveto(224.66239513,311.04790921)(225.07450409,311.34478392)(225.34403992,311.93853498)
\curveto(225.6174723,312.53228273)(225.75419091,313.43462558)(225.75419617,314.64556623)
\curveto(225.75419091,315.8604044)(225.6174723,316.76470037)(225.34403992,317.35845686)
\curveto(225.07450409,317.95219918)(224.66239513,318.24907389)(224.10771179,318.24908186)
\moveto(224.10771179,319.18658186)
\curveto(225.041301,319.18657295)(225.74637842,318.80376083)(226.22294617,318.03814436)
\curveto(226.70340871,317.27251236)(226.94364285,316.14165412)(226.94364929,314.64556623)
\curveto(226.94364285,313.15337586)(226.70340871,312.02447074)(226.22294617,311.25884748)
\curveto(225.74637842,310.49322227)(225.041301,310.11041015)(224.10771179,310.11040998)
\curveto(223.17411537,310.11041015)(222.46903795,310.49322227)(221.99247742,311.25884748)
\curveto(221.5159139,312.02447074)(221.27763289,313.15337586)(221.27763367,314.64556623)
\curveto(221.27763289,316.14165412)(221.5159139,317.27251236)(221.99247742,318.03814436)
\curveto(222.46903795,318.80376083)(223.17411537,319.18657295)(224.10771179,319.18658186)
}
}
{
\newrgbcolor{curcolor}{0 0 0}
\pscustom[linestyle=none,fillstyle=solid,fillcolor=curcolor]
{
\newpath
\moveto(255.52357483,314.67742658)
\curveto(255.523572,314.89226572)(255.59779067,315.07781241)(255.74623108,315.23406721)
\curveto(255.89857162,315.3903121)(256.08021207,315.46843702)(256.29115295,315.46844221)
\curveto(256.50989914,315.46843702)(256.69739895,315.3903121)(256.85365295,315.23406721)
\curveto(257.00989864,315.07781241)(257.08802356,314.89226572)(257.08802795,314.67742658)
\curveto(257.08802356,314.45867241)(257.00989864,314.27312572)(256.85365295,314.12078596)
\curveto(256.7013052,313.96843852)(256.51380538,313.89226672)(256.29115295,313.89227033)
\curveto(256.07239958,313.89226672)(255.88880601,313.9664854)(255.7403717,314.11492658)
\curveto(255.59583755,314.2633601)(255.523572,314.45085991)(255.52357483,314.67742658)
\moveto(256.3028717,318.25164533)
\curveto(255.7520874,318.25163736)(255.33997843,317.95476266)(255.06654358,317.36102033)
\curveto(254.79701023,316.76726385)(254.66224474,315.86296788)(254.6622467,314.64812971)
\curveto(254.66224474,313.43718905)(254.79701023,312.53484621)(255.06654358,311.94109846)
\curveto(255.33997843,311.34734739)(255.7520874,311.05047269)(256.3028717,311.05047346)
\curveto(256.85755504,311.05047269)(257.269664,311.34734739)(257.53919983,311.94109846)
\curveto(257.81263221,312.53484621)(257.94935082,313.43718905)(257.94935608,314.64812971)
\curveto(257.94935082,315.86296788)(257.81263221,316.76726385)(257.53919983,317.36102033)
\curveto(257.269664,317.95476266)(256.85755504,318.25163736)(256.3028717,318.25164533)
\moveto(256.3028717,319.18914533)
\curveto(257.23646091,319.18913643)(257.94153833,318.80632431)(258.41810608,318.04070783)
\curveto(258.89856862,317.27507584)(259.13880276,316.1442176)(259.1388092,314.64812971)
\curveto(259.13880276,313.15593933)(258.89856862,312.02703421)(258.41810608,311.26141096)
\curveto(257.94153833,310.49578574)(257.23646091,310.11297363)(256.3028717,310.11297346)
\curveto(255.36927528,310.11297363)(254.66419786,310.49578574)(254.18763733,311.26141096)
\curveto(253.71107381,312.02703421)(253.4727928,313.15593933)(253.47279358,314.64812971)
\curveto(253.4727928,316.1442176)(253.71107381,317.27507584)(254.18763733,318.04070783)
\curveto(254.66419786,318.80632431)(255.36927528,319.18913643)(256.3028717,319.18914533)
}
}
{
\newrgbcolor{curcolor}{0 0 0}
\pscustom[linestyle=none,fillstyle=solid,fillcolor=curcolor]
{
\newpath
\moveto(287.24414062,314.41021467)
\curveto(287.24413779,314.62505381)(287.31835647,314.8106005)(287.46679688,314.96685529)
\curveto(287.61913742,315.12310019)(287.80077786,315.20122511)(288.01171875,315.20123029)
\curveto(288.23046493,315.20122511)(288.41796475,315.12310019)(288.57421875,314.96685529)
\curveto(288.73046443,314.8106005)(288.80858936,314.62505381)(288.80859375,314.41021467)
\curveto(288.80858936,314.19146049)(288.73046443,314.0059138)(288.57421875,313.85357404)
\curveto(288.42187099,313.70122661)(288.23437118,313.62505481)(288.01171875,313.62505842)
\curveto(287.79296537,313.62505481)(287.6093718,313.69927348)(287.4609375,313.84771467)
\curveto(287.31640335,313.99614819)(287.24413779,314.183648)(287.24414062,314.41021467)
\moveto(288.0234375,317.98443342)
\curveto(287.47265319,317.98442545)(287.06054423,317.68755075)(286.78710938,317.09380842)
\curveto(286.51757602,316.50005193)(286.38281053,315.59575596)(286.3828125,314.38091779)
\curveto(286.38281053,313.16997714)(286.51757602,312.26763429)(286.78710938,311.67388654)
\curveto(287.06054423,311.08013548)(287.47265319,310.78326078)(288.0234375,310.78326154)
\curveto(288.57812084,310.78326078)(288.9902298,311.08013548)(289.25976562,311.67388654)
\curveto(289.53319801,312.26763429)(289.66991662,313.16997714)(289.66992188,314.38091779)
\curveto(289.66991662,315.59575596)(289.53319801,316.50005193)(289.25976562,317.09380842)
\curveto(288.9902298,317.68755075)(288.57812084,317.98442545)(288.0234375,317.98443342)
\moveto(288.0234375,318.92193342)
\curveto(288.95702671,318.92192451)(289.66210413,318.53911239)(290.13867188,317.77349592)
\curveto(290.61913442,317.00786393)(290.85936855,315.87700568)(290.859375,314.38091779)
\curveto(290.85936855,312.88872742)(290.61913442,311.7598223)(290.13867188,310.99419904)
\curveto(289.66210413,310.22857383)(288.95702671,309.84576171)(288.0234375,309.84576154)
\curveto(287.08984107,309.84576171)(286.38476365,310.22857383)(285.90820312,310.99419904)
\curveto(285.43163961,311.7598223)(285.1933586,312.88872742)(285.19335938,314.38091779)
\curveto(285.1933586,315.87700568)(285.43163961,317.00786393)(285.90820312,317.77349592)
\curveto(286.38476365,318.53911239)(287.08984107,318.92192451)(288.0234375,318.92193342)
}
}
{
\newrgbcolor{curcolor}{0 0 0}
\pscustom[linestyle=none,fillstyle=solid,fillcolor=curcolor]
{
\newpath
\moveto(319.88812256,311.11138654)
\lineto(321.72796631,311.11138654)
\lineto(321.72796631,317.79693342)
\lineto(319.74749756,317.35162092)
\lineto(319.74749756,318.42974592)
\lineto(321.71624756,318.86333967)
\lineto(322.89984131,318.86333967)
\lineto(322.89984131,311.11138654)
\lineto(324.71624756,311.11138654)
\lineto(324.71624756,310.11529279)
\lineto(319.88812256,310.11529279)
\lineto(319.88812256,311.11138654)
}
}
{
\newrgbcolor{curcolor}{0 0 0}
\pscustom[linestyle=none,fillstyle=solid,fillcolor=curcolor]
{
\newpath
\moveto(215.58818054,294.22466779)
\lineto(219.60771179,294.22466779)
\lineto(219.60771179,293.22857404)
\lineto(214.29325867,293.22857404)
\lineto(214.29325867,294.22466779)
\curveto(215.0237258,294.99419728)(215.66239703,295.6738841)(216.20927429,296.26373029)
\curveto(216.75614594,296.85357042)(217.13309869,297.26958563)(217.34013367,297.51177717)
\curveto(217.73075434,297.98833491)(217.99442595,298.37310015)(218.13114929,298.66607404)
\curveto(218.26786318,298.96294331)(218.33622248,299.26567738)(218.33622742,299.57427717)
\curveto(218.33622248,300.06255158)(218.19169138,300.4453637)(217.90263367,300.72271467)
\curveto(217.6174732,301.00005065)(217.22489547,301.13872238)(216.72489929,301.13873029)
\curveto(216.36942758,301.13872238)(215.99638107,301.07426932)(215.60575867,300.94537092)
\curveto(215.21513185,300.81645708)(214.80106977,300.62114478)(214.36357117,300.35943342)
\lineto(214.36357117,301.55474592)
\curveto(214.76591355,301.74614365)(215.16044441,301.89067476)(215.54716492,301.98833967)
\curveto(215.93778738,302.08598706)(216.32255262,302.13481514)(216.70146179,302.13482404)
\curveto(217.55692639,302.13481514)(218.2444257,301.90629974)(218.76396179,301.44927717)
\curveto(219.28739341,300.9961444)(219.5491119,300.40044187)(219.54911804,299.66216779)
\curveto(219.5491119,299.28716173)(219.46122136,298.91216211)(219.28544617,298.53716779)
\curveto(219.11356546,298.16216286)(218.83231574,297.74810077)(218.44169617,297.29498029)
\curveto(218.22294135,297.04107023)(217.90458229,296.68950808)(217.48661804,296.24029279)
\curveto(217.07255187,295.79107148)(216.43974,295.11919715)(215.58818054,294.22466779)
}
}
{
\newrgbcolor{curcolor}{0 0 0}
\pscustom[linestyle=none,fillstyle=solid,fillcolor=curcolor]
{
\newpath
\moveto(221.43583679,301.97662092)
\lineto(226.94364929,301.97662092)
\lineto(226.94364929,301.47271467)
\lineto(223.81474304,293.22857404)
\lineto(222.57841492,293.22857404)
\lineto(225.62528992,300.98052717)
\lineto(221.43583679,300.98052717)
\lineto(221.43583679,301.97662092)
}
}
{
\newrgbcolor{curcolor}{0 0 0}
\pscustom[linestyle=none,fillstyle=solid,fillcolor=curcolor]
{
\newpath
\moveto(255.52357483,297.65313459)
\curveto(255.523572,297.86797373)(255.59779067,298.05352042)(255.74623108,298.20977521)
\curveto(255.89857162,298.36602011)(256.08021207,298.44414503)(256.29115295,298.44415021)
\curveto(256.50989914,298.44414503)(256.69739895,298.36602011)(256.85365295,298.20977521)
\curveto(257.00989864,298.05352042)(257.08802356,297.86797373)(257.08802795,297.65313459)
\curveto(257.08802356,297.43438041)(257.00989864,297.24883372)(256.85365295,297.09649396)
\curveto(256.7013052,296.94414653)(256.51380538,296.86797473)(256.29115295,296.86797834)
\curveto(256.07239958,296.86797473)(255.88880601,296.94219341)(255.7403717,297.09063459)
\curveto(255.59583755,297.23906811)(255.523572,297.42656792)(255.52357483,297.65313459)
\moveto(256.3028717,301.22735334)
\curveto(255.7520874,301.22734537)(255.33997843,300.93047067)(255.06654358,300.33672834)
\curveto(254.79701023,299.74297186)(254.66224474,298.83867588)(254.6622467,297.62383771)
\curveto(254.66224474,296.41289706)(254.79701023,295.51055421)(255.06654358,294.91680646)
\curveto(255.33997843,294.3230554)(255.7520874,294.0261807)(256.3028717,294.02618146)
\curveto(256.85755504,294.0261807)(257.269664,294.3230554)(257.53919983,294.91680646)
\curveto(257.81263221,295.51055421)(257.94935082,296.41289706)(257.94935608,297.62383771)
\curveto(257.94935082,298.83867588)(257.81263221,299.74297186)(257.53919983,300.33672834)
\curveto(257.269664,300.93047067)(256.85755504,301.22734537)(256.3028717,301.22735334)
\moveto(256.3028717,302.16485334)
\curveto(257.23646091,302.16484443)(257.94153833,301.78203232)(258.41810608,301.01641584)
\curveto(258.89856862,300.25078385)(259.13880276,299.1199256)(259.1388092,297.62383771)
\curveto(259.13880276,296.13164734)(258.89856862,295.00274222)(258.41810608,294.23711896)
\curveto(257.94153833,293.47149375)(257.23646091,293.08868163)(256.3028717,293.08868146)
\curveto(255.36927528,293.08868163)(254.66419786,293.47149375)(254.18763733,294.23711896)
\curveto(253.71107381,295.00274222)(253.4727928,296.13164734)(253.47279358,297.62383771)
\curveto(253.4727928,299.1199256)(253.71107381,300.25078385)(254.18763733,301.01641584)
\curveto(254.66419786,301.78203232)(255.36927528,302.16484443)(256.3028717,302.16485334)
}
}
{
\newrgbcolor{curcolor}{0 0 0}
\pscustom[linestyle=none,fillstyle=solid,fillcolor=curcolor]
{
\newpath
\moveto(289.08401489,297.67913557)
\curveto(289.65822852,297.52678729)(290.09768121,297.25530318)(290.40237427,296.86468244)
\curveto(290.7070556,296.47796021)(290.8593992,295.9935857)(290.85940552,295.41155744)
\curveto(290.8593992,294.60686833)(290.58791509,293.97405646)(290.04495239,293.51311994)
\curveto(289.50588492,293.05608863)(288.7578388,292.82757324)(287.80081177,292.82757307)
\curveto(287.39846516,292.82757324)(286.98830932,292.86468257)(286.57034302,292.93890119)
\curveto(286.15237265,293.01311993)(285.74221681,293.12054169)(285.33987427,293.26116682)
\lineto(285.33987427,294.43890119)
\curveto(285.73831057,294.23186871)(286.1308883,294.07757199)(286.51760864,293.97601057)
\curveto(286.90432503,293.87444719)(287.28909027,293.82366599)(287.67190552,293.82366682)
\curveto(288.32033923,293.82366599)(288.81838561,293.97015022)(289.16604614,294.26311994)
\curveto(289.51369742,294.55608713)(289.68752537,294.97796171)(289.68753052,295.52874494)
\curveto(289.68752537,296.0365544)(289.51369742,296.43889775)(289.16604614,296.73577619)
\curveto(288.81838561,297.0365534)(288.34768296,297.18694388)(287.75393677,297.18694807)
\lineto(286.85159302,297.18694807)
\lineto(286.85159302,298.15960432)
\lineto(287.75393677,298.15960432)
\curveto(288.29690176,298.15959915)(288.72072946,298.27873966)(289.02542114,298.51702619)
\curveto(289.33010385,298.75530168)(289.48244745,299.0873326)(289.48245239,299.51311994)
\curveto(289.48244745,299.96233173)(289.33986946,300.30608138)(289.05471802,300.54436994)
\curveto(288.77346378,300.78654965)(288.37112043,300.90764328)(287.84768677,300.90765119)
\curveto(287.50002755,300.90764328)(287.14065291,300.86858082)(286.76956177,300.79046369)
\curveto(286.39846616,300.71233098)(286.00979467,300.59514359)(285.60354614,300.43890119)
\lineto(285.60354614,301.52874494)
\curveto(286.07620085,301.65373629)(286.49612231,301.74748619)(286.86331177,301.80999494)
\curveto(287.23440282,301.87248607)(287.56252749,301.90373604)(287.84768677,301.90374494)
\curveto(288.69924511,301.90373604)(289.37893193,301.6888925)(289.88674927,301.25921369)
\curveto(290.39846216,300.83342461)(290.65432128,300.26701892)(290.65432739,299.55999494)
\curveto(290.65432128,299.07952011)(290.51955579,298.67912988)(290.25003052,298.35882307)
\curveto(289.98440007,298.03850553)(289.59572858,297.81194325)(289.08401489,297.67913557)
}
}
{
\newrgbcolor{curcolor}{0 0 0}
\pscustom[linestyle=none,fillstyle=solid,fillcolor=curcolor]
{
\newpath
\moveto(319.66546631,301.97601057)
\lineto(324.09515381,301.97601057)
\lineto(324.09515381,300.97991682)
\lineto(320.74359131,300.97991682)
\lineto(320.74359131,298.82952619)
\curveto(320.9115576,298.89202053)(321.07952618,298.93694236)(321.24749756,298.96429182)
\curveto(321.41936959,298.99553605)(321.59124442,299.01116103)(321.76312256,299.01116682)
\curveto(322.66936834,299.01116103)(323.38811762,298.74358318)(323.91937256,298.20843244)
\curveto(324.45061656,297.67327175)(324.71624129,296.9486631)(324.71624756,296.03460432)
\curveto(324.71624129,295.11272743)(324.4369447,294.38616566)(323.87835693,293.85491682)
\curveto(323.32366456,293.32366672)(322.5638997,293.05804199)(321.59906006,293.05804182)
\curveto(321.13421363,293.05804199)(320.7084328,293.08929196)(320.32171631,293.15179182)
\curveto(319.93890232,293.21429183)(319.59515267,293.30804174)(319.29046631,293.43304182)
\lineto(319.29046631,294.63421369)
\curveto(319.64984011,294.43889998)(320.01116788,294.29241575)(320.37445068,294.19476057)
\curveto(320.73772965,294.10100969)(321.10882303,294.05413474)(321.48773193,294.05413557)
\curveto(322.140072,294.05413474)(322.64202462,294.22600957)(322.99359131,294.56976057)
\curveto(323.34905516,294.91350888)(323.52678936,295.40178964)(323.52679443,296.03460432)
\curveto(323.52678936,296.65960088)(323.34319579,297.14592852)(322.97601318,297.49358869)
\curveto(322.61272777,297.84124033)(322.10491578,298.01506828)(321.45257568,298.01507307)
\curveto(321.13616675,298.01506828)(320.82757331,297.97795894)(320.52679443,297.90374494)
\curveto(320.22601141,297.83342784)(319.93890232,297.72600607)(319.66546631,297.58147932)
\lineto(319.66546631,301.97601057)
}
}
{
\newrgbcolor{curcolor}{0 0 0}
\pscustom[linestyle=none,fillstyle=solid,fillcolor=curcolor]
{
\newpath
\moveto(223.32841492,280.74126936)
\curveto(223.32841209,280.9561085)(223.40263076,281.14165519)(223.55107117,281.29790998)
\curveto(223.70341171,281.45415487)(223.88505216,281.53227979)(224.09599304,281.53228498)
\curveto(224.31473923,281.53227979)(224.50223904,281.45415487)(224.65849304,281.29790998)
\curveto(224.81473873,281.14165519)(224.89286365,280.9561085)(224.89286804,280.74126936)
\curveto(224.89286365,280.52251518)(224.81473873,280.33696849)(224.65849304,280.18462873)
\curveto(224.50614528,280.03228129)(224.31864547,279.9561095)(224.09599304,279.95611311)
\curveto(223.87723966,279.9561095)(223.6936461,280.03032817)(223.54521179,280.17876936)
\curveto(223.40067764,280.32720288)(223.32841209,280.51470269)(223.32841492,280.74126936)
\moveto(224.10771179,284.31548811)
\curveto(223.55692748,284.31548014)(223.14481852,284.01860543)(222.87138367,283.42486311)
\curveto(222.60185031,282.83110662)(222.46708482,281.92681065)(222.46708679,280.71197248)
\curveto(222.46708482,279.50103183)(222.60185031,278.59868898)(222.87138367,278.00494123)
\curveto(223.14481852,277.41119017)(223.55692748,277.11431546)(224.10771179,277.11431623)
\curveto(224.66239513,277.11431546)(225.07450409,277.41119017)(225.34403992,278.00494123)
\curveto(225.6174723,278.59868898)(225.75419091,279.50103183)(225.75419617,280.71197248)
\curveto(225.75419091,281.92681065)(225.6174723,282.83110662)(225.34403992,283.42486311)
\curveto(225.07450409,284.01860543)(224.66239513,284.31548014)(224.10771179,284.31548811)
\moveto(224.10771179,285.25298811)
\curveto(225.041301,285.2529792)(225.74637842,284.87016708)(226.22294617,284.10455061)
\curveto(226.70340871,283.33891861)(226.94364285,282.20806037)(226.94364929,280.71197248)
\curveto(226.94364285,279.21978211)(226.70340871,278.09087699)(226.22294617,277.32525373)
\curveto(225.74637842,276.55962852)(225.041301,276.1768164)(224.10771179,276.17681623)
\curveto(223.17411537,276.1768164)(222.46903795,276.55962852)(221.99247742,277.32525373)
\curveto(221.5159139,278.09087699)(221.27763289,279.21978211)(221.27763367,280.71197248)
\curveto(221.27763289,282.20806037)(221.5159139,283.33891861)(221.99247742,284.10455061)
\curveto(222.46903795,284.87016708)(223.17411537,285.2529792)(224.10771179,285.25298811)
}
}
{
\newrgbcolor{curcolor}{0 0 0}
\pscustom[linestyle=none,fillstyle=solid,fillcolor=curcolor]
{
\newpath
\moveto(255.52357483,280.62896467)
\curveto(255.523572,280.84380381)(255.59779067,281.0293505)(255.74623108,281.18560529)
\curveto(255.89857162,281.34185019)(256.08021207,281.41997511)(256.29115295,281.41998029)
\curveto(256.50989914,281.41997511)(256.69739895,281.34185019)(256.85365295,281.18560529)
\curveto(257.00989864,281.0293505)(257.08802356,280.84380381)(257.08802795,280.62896467)
\curveto(257.08802356,280.41021049)(257.00989864,280.2246638)(256.85365295,280.07232404)
\curveto(256.7013052,279.91997661)(256.51380538,279.84380481)(256.29115295,279.84380842)
\curveto(256.07239958,279.84380481)(255.88880601,279.91802348)(255.7403717,280.06646467)
\curveto(255.59583755,280.21489819)(255.523572,280.402398)(255.52357483,280.62896467)
\moveto(256.3028717,284.20318342)
\curveto(255.7520874,284.20317545)(255.33997843,283.90630075)(255.06654358,283.31255842)
\curveto(254.79701023,282.71880193)(254.66224474,281.81450596)(254.6622467,280.59966779)
\curveto(254.66224474,279.38872714)(254.79701023,278.48638429)(255.06654358,277.89263654)
\curveto(255.33997843,277.29888548)(255.7520874,277.00201078)(256.3028717,277.00201154)
\curveto(256.85755504,277.00201078)(257.269664,277.29888548)(257.53919983,277.89263654)
\curveto(257.81263221,278.48638429)(257.94935082,279.38872714)(257.94935608,280.59966779)
\curveto(257.94935082,281.81450596)(257.81263221,282.71880193)(257.53919983,283.31255842)
\curveto(257.269664,283.90630075)(256.85755504,284.20317545)(256.3028717,284.20318342)
\moveto(256.3028717,285.14068342)
\curveto(257.23646091,285.14067451)(257.94153833,284.75786239)(258.41810608,283.99224592)
\curveto(258.89856862,283.22661393)(259.13880276,282.09575568)(259.1388092,280.59966779)
\curveto(259.13880276,279.10747742)(258.89856862,277.9785723)(258.41810608,277.21294904)
\curveto(257.94153833,276.44732383)(257.23646091,276.06451171)(256.3028717,276.06451154)
\curveto(255.36927528,276.06451171)(254.66419786,276.44732383)(254.18763733,277.21294904)
\curveto(253.71107381,277.9785723)(253.4727928,279.10747742)(253.47279358,280.59966779)
\curveto(253.4727928,282.09575568)(253.71107381,283.22661393)(254.18763733,283.99224592)
\curveto(254.66419786,284.75786239)(255.36927528,285.14067451)(256.3028717,285.14068342)
}
}
{
\newrgbcolor{curcolor}{0 0 0}
\pscustom[linestyle=none,fillstyle=solid,fillcolor=curcolor]
{
\newpath
\moveto(287.24414062,280.37383771)
\curveto(287.24413779,280.58867686)(287.31835647,280.77422354)(287.46679688,280.93047834)
\curveto(287.61913742,281.08672323)(287.80077786,281.16484815)(288.01171875,281.16485334)
\curveto(288.23046493,281.16484815)(288.41796475,281.08672323)(288.57421875,280.93047834)
\curveto(288.73046443,280.77422354)(288.80858936,280.58867686)(288.80859375,280.37383771)
\curveto(288.80858936,280.15508354)(288.73046443,279.96953685)(288.57421875,279.81719709)
\curveto(288.42187099,279.66484965)(288.23437118,279.58867786)(288.01171875,279.58868146)
\curveto(287.79296537,279.58867786)(287.6093718,279.66289653)(287.4609375,279.81133771)
\curveto(287.31640335,279.95977123)(287.24413779,280.14727105)(287.24414062,280.37383771)
\moveto(288.0234375,283.94805646)
\curveto(287.47265319,283.9480485)(287.06054423,283.65117379)(286.78710938,283.05743146)
\curveto(286.51757602,282.46367498)(286.38281053,281.55937901)(286.3828125,280.34454084)
\curveto(286.38281053,279.13360019)(286.51757602,278.23125734)(286.78710938,277.63750959)
\curveto(287.06054423,277.04375853)(287.47265319,276.74688382)(288.0234375,276.74688459)
\curveto(288.57812084,276.74688382)(288.9902298,277.04375853)(289.25976562,277.63750959)
\curveto(289.53319801,278.23125734)(289.66991662,279.13360019)(289.66992188,280.34454084)
\curveto(289.66991662,281.55937901)(289.53319801,282.46367498)(289.25976562,283.05743146)
\curveto(288.9902298,283.65117379)(288.57812084,283.9480485)(288.0234375,283.94805646)
\moveto(288.0234375,284.88555646)
\curveto(288.95702671,284.88554756)(289.66210413,284.50273544)(290.13867188,283.73711896)
\curveto(290.61913442,282.97148697)(290.85936855,281.84062873)(290.859375,280.34454084)
\curveto(290.85936855,278.85235047)(290.61913442,277.72344535)(290.13867188,276.95782209)
\curveto(289.66210413,276.19219688)(288.95702671,275.80938476)(288.0234375,275.80938459)
\curveto(287.08984107,275.80938476)(286.38476365,276.19219688)(285.90820312,276.95782209)
\curveto(285.43163961,277.72344535)(285.1933586,278.85235047)(285.19335938,280.34454084)
\curveto(285.1933586,281.84062873)(285.43163961,282.97148697)(285.90820312,283.73711896)
\curveto(286.38476365,284.50273544)(287.08984107,284.88554756)(288.0234375,284.88555646)
}
}
{
\newrgbcolor{curcolor}{0 0 0}
\pscustom[linestyle=none,fillstyle=solid,fillcolor=curcolor]
{
\newpath
\moveto(322.37249756,283.84063459)
\lineto(319.61273193,279.21758771)
\lineto(322.37249756,279.21758771)
\lineto(322.37249756,283.84063459)
\moveto(322.17913818,284.91875959)
\lineto(323.55023193,284.91875959)
\lineto(323.55023193,279.21758771)
\lineto(324.71624756,279.21758771)
\lineto(324.71624756,278.25665021)
\lineto(323.55023193,278.25665021)
\lineto(323.55023193,276.17071271)
\lineto(322.37249756,276.17071271)
\lineto(322.37249756,278.25665021)
\lineto(318.66351318,278.25665021)
\lineto(318.66351318,279.37579084)
\lineto(322.17913818,284.91875959)
}
}
{
\newrgbcolor{curcolor}{0 0 0}
\pscustom[linestyle=none,fillstyle=solid,fillcolor=curcolor]
{
\newpath
\moveto(223.32841492,263.68951154)
\curveto(223.32841209,263.90435068)(223.40263076,264.08989737)(223.55107117,264.24615217)
\curveto(223.70341171,264.40239706)(223.88505216,264.48052198)(224.09599304,264.48052717)
\curveto(224.31473923,264.48052198)(224.50223904,264.40239706)(224.65849304,264.24615217)
\curveto(224.81473873,264.08989737)(224.89286365,263.90435068)(224.89286804,263.68951154)
\curveto(224.89286365,263.47075737)(224.81473873,263.28521068)(224.65849304,263.13287092)
\curveto(224.50614528,262.98052348)(224.31864547,262.90435168)(224.09599304,262.90435529)
\curveto(223.87723966,262.90435168)(223.6936461,262.97857036)(223.54521179,263.12701154)
\curveto(223.40067764,263.27544506)(223.32841209,263.46294488)(223.32841492,263.68951154)
\moveto(224.10771179,267.26373029)
\curveto(223.55692748,267.26372232)(223.14481852,266.96684762)(222.87138367,266.37310529)
\curveto(222.60185031,265.77934881)(222.46708482,264.87505284)(222.46708679,263.66021467)
\curveto(222.46708482,262.44927401)(222.60185031,261.54693117)(222.87138367,260.95318342)
\curveto(223.14481852,260.35943235)(223.55692748,260.06255765)(224.10771179,260.06255842)
\curveto(224.66239513,260.06255765)(225.07450409,260.35943235)(225.34403992,260.95318342)
\curveto(225.6174723,261.54693117)(225.75419091,262.44927401)(225.75419617,263.66021467)
\curveto(225.75419091,264.87505284)(225.6174723,265.77934881)(225.34403992,266.37310529)
\curveto(225.07450409,266.96684762)(224.66239513,267.26372232)(224.10771179,267.26373029)
\moveto(224.10771179,268.20123029)
\curveto(225.041301,268.20122139)(225.74637842,267.81840927)(226.22294617,267.05279279)
\curveto(226.70340871,266.2871608)(226.94364285,265.15630256)(226.94364929,263.66021467)
\curveto(226.94364285,262.16802429)(226.70340871,261.03911917)(226.22294617,260.27349592)
\curveto(225.74637842,259.50787071)(225.041301,259.12505859)(224.10771179,259.12505842)
\curveto(223.17411537,259.12505859)(222.46903795,259.50787071)(221.99247742,260.27349592)
\curveto(221.5159139,261.03911917)(221.27763289,262.16802429)(221.27763367,263.66021467)
\curveto(221.27763289,265.15630256)(221.5159139,266.2871608)(221.99247742,267.05279279)
\curveto(222.46903795,267.81840927)(223.17411537,268.20122139)(224.10771179,268.20123029)
}
}
{
\newrgbcolor{curcolor}{0 0 0}
\pscustom[linestyle=none,fillstyle=solid,fillcolor=curcolor]
{
\newpath
\moveto(255.52357483,263.60467268)
\curveto(255.523572,263.81951182)(255.59779067,264.00505851)(255.74623108,264.1613133)
\curveto(255.89857162,264.31755819)(256.08021207,264.39568312)(256.29115295,264.3956883)
\curveto(256.50989914,264.39568312)(256.69739895,264.31755819)(256.85365295,264.1613133)
\curveto(257.00989864,264.00505851)(257.08802356,263.81951182)(257.08802795,263.60467268)
\curveto(257.08802356,263.3859185)(257.00989864,263.20037181)(256.85365295,263.04803205)
\curveto(256.7013052,262.89568462)(256.51380538,262.81951282)(256.29115295,262.81951643)
\curveto(256.07239958,262.81951282)(255.88880601,262.89373149)(255.7403717,263.04217268)
\curveto(255.59583755,263.1906062)(255.523572,263.37810601)(255.52357483,263.60467268)
\moveto(256.3028717,267.17889143)
\curveto(255.7520874,267.17888346)(255.33997843,266.88200875)(255.06654358,266.28826643)
\curveto(254.79701023,265.69450994)(254.66224474,264.79021397)(254.6622467,263.5753758)
\curveto(254.66224474,262.36443515)(254.79701023,261.4620923)(255.06654358,260.86834455)
\curveto(255.33997843,260.27459349)(255.7520874,259.97771878)(256.3028717,259.97771955)
\curveto(256.85755504,259.97771878)(257.269664,260.27459349)(257.53919983,260.86834455)
\curveto(257.81263221,261.4620923)(257.94935082,262.36443515)(257.94935608,263.5753758)
\curveto(257.94935082,264.79021397)(257.81263221,265.69450994)(257.53919983,266.28826643)
\curveto(257.269664,266.88200875)(256.85755504,267.17888346)(256.3028717,267.17889143)
\moveto(256.3028717,268.11639143)
\curveto(257.23646091,268.11638252)(257.94153833,267.7335704)(258.41810608,266.96795393)
\curveto(258.89856862,266.20232193)(259.13880276,265.07146369)(259.1388092,263.5753758)
\curveto(259.13880276,262.08318543)(258.89856862,260.95428031)(258.41810608,260.18865705)
\curveto(257.94153833,259.42303184)(257.23646091,259.04021972)(256.3028717,259.04021955)
\curveto(255.36927528,259.04021972)(254.66419786,259.42303184)(254.18763733,260.18865705)
\curveto(253.71107381,260.95428031)(253.4727928,262.08318543)(253.47279358,263.5753758)
\curveto(253.4727928,265.07146369)(253.71107381,266.20232193)(254.18763733,266.96795393)
\curveto(254.66419786,267.7335704)(255.36927528,268.11638252)(256.3028717,268.11639143)
}
}
{
\newrgbcolor{curcolor}{0 0 0}
\pscustom[linestyle=none,fillstyle=solid,fillcolor=curcolor]
{
\newpath
\moveto(287.24414062,263.35564924)
\curveto(287.24413779,263.57048838)(287.31835647,263.75603507)(287.46679688,263.91228986)
\curveto(287.61913742,264.06853476)(287.80077786,264.14665968)(288.01171875,264.14666486)
\curveto(288.23046493,264.14665968)(288.41796475,264.06853476)(288.57421875,263.91228986)
\curveto(288.73046443,263.75603507)(288.80858936,263.57048838)(288.80859375,263.35564924)
\curveto(288.80858936,263.13689506)(288.73046443,262.95134837)(288.57421875,262.79900861)
\curveto(288.42187099,262.64666118)(288.23437118,262.57048938)(288.01171875,262.57049299)
\curveto(287.79296537,262.57048938)(287.6093718,262.64470805)(287.4609375,262.79314924)
\curveto(287.31640335,262.94158276)(287.24413779,263.12908257)(287.24414062,263.35564924)
\moveto(288.0234375,266.92986799)
\curveto(287.47265319,266.92986002)(287.06054423,266.63298532)(286.78710938,266.03924299)
\curveto(286.51757602,265.4454865)(286.38281053,264.54119053)(286.3828125,263.32635236)
\curveto(286.38281053,262.11541171)(286.51757602,261.21306886)(286.78710938,260.61932111)
\curveto(287.06054423,260.02557005)(287.47265319,259.72869535)(288.0234375,259.72869611)
\curveto(288.57812084,259.72869535)(288.9902298,260.02557005)(289.25976562,260.61932111)
\curveto(289.53319801,261.21306886)(289.66991662,262.11541171)(289.66992188,263.32635236)
\curveto(289.66991662,264.54119053)(289.53319801,265.4454865)(289.25976562,266.03924299)
\curveto(288.9902298,266.63298532)(288.57812084,266.92986002)(288.0234375,266.92986799)
\moveto(288.0234375,267.86736799)
\curveto(288.95702671,267.86735908)(289.66210413,267.48454696)(290.13867188,266.71893049)
\curveto(290.61913442,265.9532985)(290.85936855,264.82244025)(290.859375,263.32635236)
\curveto(290.85936855,261.83416199)(290.61913442,260.70525687)(290.13867188,259.93963361)
\curveto(289.66210413,259.1740084)(288.95702671,258.79119628)(288.0234375,258.79119611)
\curveto(287.08984107,258.79119628)(286.38476365,259.1740084)(285.90820312,259.93963361)
\curveto(285.43163961,260.70525687)(285.1933586,261.83416199)(285.19335938,263.32635236)
\curveto(285.1933586,264.82244025)(285.43163961,265.9532985)(285.90820312,266.71893049)
\curveto(286.38476365,267.48454696)(287.08984107,267.86735908)(288.0234375,267.86736799)
}
}
{
\newrgbcolor{curcolor}{0 0 0}
\pscustom[linestyle=none,fillstyle=solid,fillcolor=curcolor]
{
\newpath
\moveto(319.20843506,268.03143049)
\lineto(324.71624756,268.03143049)
\lineto(324.71624756,267.52752424)
\lineto(321.58734131,259.28338361)
\lineto(320.35101318,259.28338361)
\lineto(323.39788818,267.03533674)
\lineto(319.20843506,267.03533674)
\lineto(319.20843506,268.03143049)
}
}
{
\newrgbcolor{curcolor}{0 0 0}
\pscustom[linestyle=none,fillstyle=solid,fillcolor=curcolor]
{
\newpath
\moveto(223.32841492,246.63775373)
\curveto(223.32841209,246.85259287)(223.40263076,247.03813956)(223.55107117,247.19439436)
\curveto(223.70341171,247.35063925)(223.88505216,247.42876417)(224.09599304,247.42876936)
\curveto(224.31473923,247.42876417)(224.50223904,247.35063925)(224.65849304,247.19439436)
\curveto(224.81473873,247.03813956)(224.89286365,246.85259287)(224.89286804,246.63775373)
\curveto(224.89286365,246.41899955)(224.81473873,246.23345287)(224.65849304,246.08111311)
\curveto(224.50614528,245.92876567)(224.31864547,245.85259387)(224.09599304,245.85259748)
\curveto(223.87723966,245.85259387)(223.6936461,245.92681255)(223.54521179,246.07525373)
\curveto(223.40067764,246.22368725)(223.32841209,246.41118706)(223.32841492,246.63775373)
\moveto(224.10771179,250.21197248)
\curveto(223.55692748,250.21196451)(223.14481852,249.91508981)(222.87138367,249.32134748)
\curveto(222.60185031,248.727591)(222.46708482,247.82329503)(222.46708679,246.60845686)
\curveto(222.46708482,245.3975162)(222.60185031,244.49517335)(222.87138367,243.90142561)
\curveto(223.14481852,243.30767454)(223.55692748,243.01079984)(224.10771179,243.01080061)
\curveto(224.66239513,243.01079984)(225.07450409,243.30767454)(225.34403992,243.90142561)
\curveto(225.6174723,244.49517335)(225.75419091,245.3975162)(225.75419617,246.60845686)
\curveto(225.75419091,247.82329503)(225.6174723,248.727591)(225.34403992,249.32134748)
\curveto(225.07450409,249.91508981)(224.66239513,250.21196451)(224.10771179,250.21197248)
\moveto(224.10771179,251.14947248)
\curveto(225.041301,251.14946357)(225.74637842,250.76665146)(226.22294617,250.00103498)
\curveto(226.70340871,249.23540299)(226.94364285,248.10454474)(226.94364929,246.60845686)
\curveto(226.94364285,245.11626648)(226.70340871,243.98736136)(226.22294617,243.22173811)
\curveto(225.74637842,242.45611289)(225.041301,242.07330078)(224.10771179,242.07330061)
\curveto(223.17411537,242.07330078)(222.46903795,242.45611289)(221.99247742,243.22173811)
\curveto(221.5159139,243.98736136)(221.27763289,245.11626648)(221.27763367,246.60845686)
\curveto(221.27763289,248.10454474)(221.5159139,249.23540299)(221.99247742,250.00103498)
\curveto(222.46903795,250.76665146)(223.17411537,251.14946357)(224.10771179,251.14947248)
}
}
{
\newrgbcolor{curcolor}{0 0 0}
\pscustom[linestyle=none,fillstyle=solid,fillcolor=curcolor]
{
\newpath
\moveto(255.52357483,246.58050275)
\curveto(255.523572,246.79534189)(255.59779067,246.98088858)(255.74623108,247.13714338)
\curveto(255.89857162,247.29338827)(256.08021207,247.37151319)(256.29115295,247.37151838)
\curveto(256.50989914,247.37151319)(256.69739895,247.29338827)(256.85365295,247.13714338)
\curveto(257.00989864,246.98088858)(257.08802356,246.79534189)(257.08802795,246.58050275)
\curveto(257.08802356,246.36174858)(257.00989864,246.17620189)(256.85365295,246.02386213)
\curveto(256.7013052,245.87151469)(256.51380538,245.79534289)(256.29115295,245.7953465)
\curveto(256.07239958,245.79534289)(255.88880601,245.86956157)(255.7403717,246.01800275)
\curveto(255.59583755,246.16643627)(255.523572,246.35393609)(255.52357483,246.58050275)
\moveto(256.3028717,250.1547215)
\curveto(255.7520874,250.15471354)(255.33997843,249.85783883)(255.06654358,249.2640965)
\curveto(254.79701023,248.67034002)(254.66224474,247.76604405)(254.6622467,246.55120588)
\curveto(254.66224474,245.34026522)(254.79701023,244.43792238)(255.06654358,243.84417463)
\curveto(255.33997843,243.25042356)(255.7520874,242.95354886)(256.3028717,242.95354963)
\curveto(256.85755504,242.95354886)(257.269664,243.25042356)(257.53919983,243.84417463)
\curveto(257.81263221,244.43792238)(257.94935082,245.34026522)(257.94935608,246.55120588)
\curveto(257.94935082,247.76604405)(257.81263221,248.67034002)(257.53919983,249.2640965)
\curveto(257.269664,249.85783883)(256.85755504,250.15471354)(256.3028717,250.1547215)
\moveto(256.3028717,251.0922215)
\curveto(257.23646091,251.0922126)(257.94153833,250.70940048)(258.41810608,249.943784)
\curveto(258.89856862,249.17815201)(259.13880276,248.04729377)(259.1388092,246.55120588)
\curveto(259.13880276,245.05901551)(258.89856862,243.93011038)(258.41810608,243.16448713)
\curveto(257.94153833,242.39886192)(257.23646091,242.0160498)(256.3028717,242.01604963)
\curveto(255.36927528,242.0160498)(254.66419786,242.39886192)(254.18763733,243.16448713)
\curveto(253.71107381,243.93011038)(253.4727928,245.05901551)(253.47279358,246.55120588)
\curveto(253.4727928,248.04729377)(253.71107381,249.17815201)(254.18763733,249.943784)
\curveto(254.66419786,250.70940048)(255.36927528,251.0922126)(256.3028717,251.0922215)
}
}
{
\newrgbcolor{curcolor}{0 0 0}
\pscustom[linestyle=none,fillstyle=solid,fillcolor=curcolor]
{
\newpath
\moveto(287.24414062,246.33746076)
\curveto(287.24413779,246.5522999)(287.31835647,246.73784659)(287.46679688,246.89410139)
\curveto(287.61913742,247.05034628)(287.80077786,247.1284712)(288.01171875,247.12847639)
\curveto(288.23046493,247.1284712)(288.41796475,247.05034628)(288.57421875,246.89410139)
\curveto(288.73046443,246.73784659)(288.80858936,246.5522999)(288.80859375,246.33746076)
\curveto(288.80858936,246.11870659)(288.73046443,245.9331599)(288.57421875,245.78082014)
\curveto(288.42187099,245.6284727)(288.23437118,245.5523009)(288.01171875,245.55230451)
\curveto(287.79296537,245.5523009)(287.6093718,245.62651958)(287.4609375,245.77496076)
\curveto(287.31640335,245.92339428)(287.24413779,246.11089409)(287.24414062,246.33746076)
\moveto(288.0234375,249.91167951)
\curveto(287.47265319,249.91167154)(287.06054423,249.61479684)(286.78710938,249.02105451)
\curveto(286.51757602,248.42729803)(286.38281053,247.52300206)(286.3828125,246.30816389)
\curveto(286.38281053,245.09722323)(286.51757602,244.19488038)(286.78710938,243.60113264)
\curveto(287.06054423,243.00738157)(287.47265319,242.71050687)(288.0234375,242.71050764)
\curveto(288.57812084,242.71050687)(288.9902298,243.00738157)(289.25976562,243.60113264)
\curveto(289.53319801,244.19488038)(289.66991662,245.09722323)(289.66992188,246.30816389)
\curveto(289.66991662,247.52300206)(289.53319801,248.42729803)(289.25976562,249.02105451)
\curveto(288.9902298,249.61479684)(288.57812084,249.91167154)(288.0234375,249.91167951)
\moveto(288.0234375,250.84917951)
\curveto(288.95702671,250.84917061)(289.66210413,250.46635849)(290.13867188,249.70074201)
\curveto(290.61913442,248.93511002)(290.85936855,247.80425178)(290.859375,246.30816389)
\curveto(290.85936855,244.81597351)(290.61913442,243.68706839)(290.13867188,242.92144514)
\curveto(289.66210413,242.15581992)(288.95702671,241.77300781)(288.0234375,241.77300764)
\curveto(287.08984107,241.77300781)(286.38476365,242.15581992)(285.90820312,242.92144514)
\curveto(285.43163961,243.68706839)(285.1933586,244.81597351)(285.19335938,246.30816389)
\curveto(285.1933586,247.80425178)(285.43163961,248.93511002)(285.90820312,249.70074201)
\curveto(286.38476365,250.46635849)(287.08984107,250.84917061)(288.0234375,250.84917951)
}
}
{
\newrgbcolor{curcolor}{0 0 0}
\pscustom[linestyle=none,fillstyle=solid,fillcolor=curcolor]
{
\newpath
\moveto(324.07757568,250.79253889)
\lineto(324.07757568,249.70269514)
\curveto(323.83147637,249.84721878)(323.56975788,249.95659367)(323.29241943,250.03082014)
\curveto(323.01507094,250.10893727)(322.72600873,250.14799973)(322.42523193,250.14800764)
\curveto(321.67522853,250.14799973)(321.10686972,249.86479688)(320.72015381,249.29839826)
\curveto(320.333433,248.73589176)(320.14007381,247.90581447)(320.14007568,246.80816389)
\curveto(320.32757363,247.19878393)(320.58733899,247.49761175)(320.91937256,247.70464826)
\curveto(321.25140083,247.91558008)(321.63225982,248.02104873)(322.06195068,248.02105451)
\curveto(322.90569605,248.02104873)(323.55803915,247.76128336)(324.01898193,247.24175764)
\curveto(324.48381947,246.72612815)(324.71624111,245.99370701)(324.71624756,245.04449201)
\curveto(324.71624111,244.09917765)(324.4779601,243.36675651)(324.00140381,242.84722639)
\curveto(323.52483605,242.32769505)(322.85491485,242.06792968)(321.99163818,242.06792951)
\curveto(320.97601048,242.06792968)(320.2318706,242.43121057)(319.75921631,243.15777326)
\curveto(319.28655904,243.88824036)(319.05023115,245.03667671)(319.05023193,246.60308576)
\curveto(319.05023115,248.07964242)(319.333434,249.20464129)(319.89984131,249.97808576)
\curveto(320.47015161,250.75542099)(321.29436954,251.14409248)(322.37249756,251.14410139)
\curveto(322.66155567,251.14409248)(322.95061788,251.11284251)(323.23968506,251.05035139)
\curveto(323.5287423,250.99174888)(323.8080389,250.90581147)(324.07757568,250.79253889)
\moveto(321.96820068,247.09527326)
\curveto(321.46429124,247.0952684)(321.06780726,246.91362796)(320.77874756,246.55035139)
\curveto(320.48968284,246.18706619)(320.34515173,245.68511356)(320.34515381,245.04449201)
\curveto(320.34515173,244.40386485)(320.48968284,243.90191222)(320.77874756,243.53863264)
\curveto(321.06780726,243.17535045)(321.46429124,242.99371001)(321.96820068,242.99371076)
\curveto(322.49163396,242.99371001)(322.88616482,243.16558483)(323.15179443,243.50933576)
\curveto(323.41741429,243.85699039)(323.55022665,244.36870863)(323.55023193,245.04449201)
\curveto(323.55022665,245.72417603)(323.41741429,246.23589426)(323.15179443,246.57964826)
\curveto(322.88616482,246.92339358)(322.49163396,247.0952684)(321.96820068,247.09527326)
}
}
{
\newrgbcolor{curcolor}{0 0 0}
\pscustom[linestyle=none,fillstyle=solid,fillcolor=curcolor]
{
\newpath
\moveto(223.32841492,229.58599592)
\curveto(223.32841209,229.80083506)(223.40263076,229.98638175)(223.55107117,230.14263654)
\curveto(223.70341171,230.29888144)(223.88505216,230.37700636)(224.09599304,230.37701154)
\curveto(224.31473923,230.37700636)(224.50223904,230.29888144)(224.65849304,230.14263654)
\curveto(224.81473873,229.98638175)(224.89286365,229.80083506)(224.89286804,229.58599592)
\curveto(224.89286365,229.36724174)(224.81473873,229.18169505)(224.65849304,229.02935529)
\curveto(224.50614528,228.87700786)(224.31864547,228.80083606)(224.09599304,228.80083967)
\curveto(223.87723966,228.80083606)(223.6936461,228.87505473)(223.54521179,229.02349592)
\curveto(223.40067764,229.17192944)(223.32841209,229.35942925)(223.32841492,229.58599592)
\moveto(224.10771179,233.16021467)
\curveto(223.55692748,233.1602067)(223.14481852,232.863332)(222.87138367,232.26958967)
\curveto(222.60185031,231.67583318)(222.46708482,230.77153721)(222.46708679,229.55669904)
\curveto(222.46708482,228.34575839)(222.60185031,227.44341554)(222.87138367,226.84966779)
\curveto(223.14481852,226.25591673)(223.55692748,225.95904203)(224.10771179,225.95904279)
\curveto(224.66239513,225.95904203)(225.07450409,226.25591673)(225.34403992,226.84966779)
\curveto(225.6174723,227.44341554)(225.75419091,228.34575839)(225.75419617,229.55669904)
\curveto(225.75419091,230.77153721)(225.6174723,231.67583318)(225.34403992,232.26958967)
\curveto(225.07450409,232.863332)(224.66239513,233.1602067)(224.10771179,233.16021467)
\moveto(224.10771179,234.09771467)
\curveto(225.041301,234.09770576)(225.74637842,233.71489364)(226.22294617,232.94927717)
\curveto(226.70340871,232.18364518)(226.94364285,231.05278693)(226.94364929,229.55669904)
\curveto(226.94364285,228.06450867)(226.70340871,226.93560355)(226.22294617,226.16998029)
\curveto(225.74637842,225.40435508)(225.041301,225.02154296)(224.10771179,225.02154279)
\curveto(223.17411537,225.02154296)(222.46903795,225.40435508)(221.99247742,226.16998029)
\curveto(221.5159139,226.93560355)(221.27763289,228.06450867)(221.27763367,229.55669904)
\curveto(221.27763289,231.05278693)(221.5159139,232.18364518)(221.99247742,232.94927717)
\curveto(222.46903795,233.71489364)(223.17411537,234.09770576)(224.10771179,234.09771467)
}
}
{
\newrgbcolor{curcolor}{0 0 0}
\pscustom[linestyle=none,fillstyle=solid,fillcolor=curcolor]
{
\newpath
\moveto(255.52357483,229.55621076)
\curveto(255.523572,229.7710499)(255.59779067,229.95659659)(255.74623108,230.11285139)
\curveto(255.89857162,230.26909628)(256.08021207,230.3472212)(256.29115295,230.34722639)
\curveto(256.50989914,230.3472212)(256.69739895,230.26909628)(256.85365295,230.11285139)
\curveto(257.00989864,229.95659659)(257.08802356,229.7710499)(257.08802795,229.55621076)
\curveto(257.08802356,229.33745659)(257.00989864,229.1519099)(256.85365295,228.99957014)
\curveto(256.7013052,228.8472227)(256.51380538,228.7710509)(256.29115295,228.77105451)
\curveto(256.07239958,228.7710509)(255.88880601,228.84526958)(255.7403717,228.99371076)
\curveto(255.59583755,229.14214428)(255.523572,229.32964409)(255.52357483,229.55621076)
\moveto(256.3028717,233.13042951)
\curveto(255.7520874,233.13042154)(255.33997843,232.83354684)(255.06654358,232.23980451)
\curveto(254.79701023,231.64604803)(254.66224474,230.74175206)(254.6622467,229.52691389)
\curveto(254.66224474,228.31597323)(254.79701023,227.41363038)(255.06654358,226.81988264)
\curveto(255.33997843,226.22613157)(255.7520874,225.92925687)(256.3028717,225.92925764)
\curveto(256.85755504,225.92925687)(257.269664,226.22613157)(257.53919983,226.81988264)
\curveto(257.81263221,227.41363038)(257.94935082,228.31597323)(257.94935608,229.52691389)
\curveto(257.94935082,230.74175206)(257.81263221,231.64604803)(257.53919983,232.23980451)
\curveto(257.269664,232.83354684)(256.85755504,233.13042154)(256.3028717,233.13042951)
\moveto(256.3028717,234.06792951)
\curveto(257.23646091,234.06792061)(257.94153833,233.68510849)(258.41810608,232.91949201)
\curveto(258.89856862,232.15386002)(259.13880276,231.02300178)(259.1388092,229.52691389)
\curveto(259.13880276,228.03472351)(258.89856862,226.90581839)(258.41810608,226.14019514)
\curveto(257.94153833,225.37456992)(257.23646091,224.99175781)(256.3028717,224.99175764)
\curveto(255.36927528,224.99175781)(254.66419786,225.37456992)(254.18763733,226.14019514)
\curveto(253.71107381,226.90581839)(253.4727928,228.03472351)(253.47279358,229.52691389)
\curveto(253.4727928,231.02300178)(253.71107381,232.15386002)(254.18763733,232.91949201)
\curveto(254.66419786,233.68510849)(255.36927528,234.06792061)(256.3028717,234.06792951)
}
}
{
\newrgbcolor{curcolor}{0 0 0}
\pscustom[linestyle=none,fillstyle=solid,fillcolor=curcolor]
{
\newpath
\moveto(287.24414062,229.31927229)
\curveto(287.24413779,229.53411143)(287.31835647,229.71965812)(287.46679688,229.87591291)
\curveto(287.61913742,230.0321578)(287.80077786,230.11028272)(288.01171875,230.11028791)
\curveto(288.23046493,230.11028272)(288.41796475,230.0321578)(288.57421875,229.87591291)
\curveto(288.73046443,229.71965812)(288.80858936,229.53411143)(288.80859375,229.31927229)
\curveto(288.80858936,229.10051811)(288.73046443,228.91497142)(288.57421875,228.76263166)
\curveto(288.42187099,228.61028422)(288.23437118,228.53411243)(288.01171875,228.53411604)
\curveto(287.79296537,228.53411243)(287.6093718,228.6083311)(287.4609375,228.75677229)
\curveto(287.31640335,228.9052058)(287.24413779,229.09270562)(287.24414062,229.31927229)
\moveto(288.0234375,232.89349104)
\curveto(287.47265319,232.89348307)(287.06054423,232.59660836)(286.78710938,232.00286604)
\curveto(286.51757602,231.40910955)(286.38281053,230.50481358)(286.3828125,229.28997541)
\curveto(286.38281053,228.07903476)(286.51757602,227.17669191)(286.78710938,226.58294416)
\curveto(287.06054423,225.9891931)(287.47265319,225.69231839)(288.0234375,225.69231916)
\curveto(288.57812084,225.69231839)(288.9902298,225.9891931)(289.25976562,226.58294416)
\curveto(289.53319801,227.17669191)(289.66991662,228.07903476)(289.66992188,229.28997541)
\curveto(289.66991662,230.50481358)(289.53319801,231.40910955)(289.25976562,232.00286604)
\curveto(288.9902298,232.59660836)(288.57812084,232.89348307)(288.0234375,232.89349104)
\moveto(288.0234375,233.83099104)
\curveto(288.95702671,233.83098213)(289.66210413,233.44817001)(290.13867188,232.68255354)
\curveto(290.61913442,231.91692154)(290.85936855,230.7860633)(290.859375,229.28997541)
\curveto(290.85936855,227.79778504)(290.61913442,226.66887992)(290.13867188,225.90325666)
\curveto(289.66210413,225.13763145)(288.95702671,224.75481933)(288.0234375,224.75481916)
\curveto(287.08984107,224.75481933)(286.38476365,225.13763145)(285.90820312,225.90325666)
\curveto(285.43163961,226.66887992)(285.1933586,227.79778504)(285.19335938,229.28997541)
\curveto(285.1933586,230.7860633)(285.43163961,231.91692154)(285.90820312,232.68255354)
\curveto(286.38476365,233.44817001)(287.08984107,233.83098213)(288.0234375,233.83099104)
}
}
{
\newrgbcolor{curcolor}{0 0 0}
\pscustom[linestyle=none,fillstyle=solid,fillcolor=curcolor]
{
\newpath
\moveto(312.63421631,226.01849104)
\lineto(314.47406006,226.01849104)
\lineto(314.47406006,232.70403791)
\lineto(312.49359131,232.25872541)
\lineto(312.49359131,233.33685041)
\lineto(314.46234131,233.77044416)
\lineto(315.64593506,233.77044416)
\lineto(315.64593506,226.01849104)
\lineto(317.46234131,226.01849104)
\lineto(317.46234131,225.02239729)
\lineto(312.63421631,225.02239729)
\lineto(312.63421631,226.01849104)
}
}
{
\newrgbcolor{curcolor}{0 0 0}
\pscustom[linestyle=none,fillstyle=solid,fillcolor=curcolor]
{
\newpath
\moveto(321.10101318,229.41692854)
\curveto(321.10101035,229.63176768)(321.17522903,229.81731437)(321.32366943,229.97356916)
\curveto(321.47600998,230.12981405)(321.65765042,230.20793897)(321.86859131,230.20794416)
\curveto(322.08733749,230.20793897)(322.2748373,230.12981405)(322.43109131,229.97356916)
\curveto(322.58733699,229.81731437)(322.66546191,229.63176768)(322.66546631,229.41692854)
\curveto(322.66546191,229.19817436)(322.58733699,229.01262767)(322.43109131,228.86028791)
\curveto(322.27874355,228.70794047)(322.09124374,228.63176868)(321.86859131,228.63177229)
\curveto(321.64983793,228.63176868)(321.46624436,228.70598735)(321.31781006,228.85442854)
\curveto(321.17327591,229.00286205)(321.10101035,229.19036187)(321.10101318,229.41692854)
\moveto(321.88031006,232.99114729)
\curveto(321.32952575,232.99113932)(320.91741679,232.69426461)(320.64398193,232.10052229)
\curveto(320.37444858,231.5067658)(320.23968309,230.60246983)(320.23968506,229.38763166)
\curveto(320.23968309,228.17669101)(320.37444858,227.27434816)(320.64398193,226.68060041)
\curveto(320.91741679,226.08684935)(321.32952575,225.78997464)(321.88031006,225.78997541)
\curveto(322.43499339,225.78997464)(322.84710236,226.08684935)(323.11663818,226.68060041)
\curveto(323.39007056,227.27434816)(323.52678918,228.17669101)(323.52679443,229.38763166)
\curveto(323.52678918,230.60246983)(323.39007056,231.5067658)(323.11663818,232.10052229)
\curveto(322.84710236,232.69426461)(322.43499339,232.99113932)(321.88031006,232.99114729)
\moveto(321.88031006,233.92864729)
\curveto(322.81389927,233.92863838)(323.51897669,233.54582626)(323.99554443,232.78020979)
\curveto(324.47600698,232.01457779)(324.71624111,230.88371955)(324.71624756,229.38763166)
\curveto(324.71624111,227.89544129)(324.47600698,226.76653617)(323.99554443,226.00091291)
\curveto(323.51897669,225.2352877)(322.81389927,224.85247558)(321.88031006,224.85247541)
\curveto(320.94671363,224.85247558)(320.24163621,225.2352877)(319.76507568,226.00091291)
\curveto(319.28851217,226.76653617)(319.05023115,227.89544129)(319.05023193,229.38763166)
\curveto(319.05023115,230.88371955)(319.28851217,232.01457779)(319.76507568,232.78020979)
\curveto(320.24163621,233.54582626)(320.94671363,233.92863838)(321.88031006,233.92864729)
}
}
{
\newrgbcolor{curcolor}{0 0 0}
\pscustom[linestyle=none,fillstyle=solid,fillcolor=curcolor]
{
\newpath
\moveto(222.92411804,209.13580061)
\lineto(226.94364929,209.13580061)
\lineto(226.94364929,208.13970686)
\lineto(221.62919617,208.13970686)
\lineto(221.62919617,209.13580061)
\curveto(222.3596633,209.90533009)(222.99833453,210.58501691)(223.54521179,211.17486311)
\curveto(224.09208344,211.76470323)(224.46903619,212.18071844)(224.67607117,212.42290998)
\curveto(225.06669184,212.89946772)(225.33036345,213.28423296)(225.46708679,213.57720686)
\curveto(225.60380068,213.87407612)(225.67215998,214.17681019)(225.67216492,214.48540998)
\curveto(225.67215998,214.9736844)(225.52762888,215.35649651)(225.23857117,215.63384748)
\curveto(224.9534107,215.91118346)(224.56083297,216.0498552)(224.06083679,216.04986311)
\curveto(223.70536508,216.0498552)(223.33231857,215.98540213)(222.94169617,215.85650373)
\curveto(222.55106935,215.72758989)(222.13700727,215.53227759)(221.69950867,215.27056623)
\lineto(221.69950867,216.46587873)
\curveto(222.10185105,216.65727646)(222.49638191,216.80180757)(222.88310242,216.89947248)
\curveto(223.27372488,216.99711987)(223.65849012,217.04594795)(224.03739929,217.04595686)
\curveto(224.89286389,217.04594795)(225.5803632,216.81743255)(226.09989929,216.36040998)
\curveto(226.62333091,215.90727721)(226.8850494,215.31157468)(226.88505554,214.57330061)
\curveto(226.8850494,214.19829455)(226.79715886,213.82329492)(226.62138367,213.44830061)
\curveto(226.44950296,213.07329567)(226.16825324,212.65923359)(225.77763367,212.20611311)
\curveto(225.55887885,211.95220304)(225.24051979,211.60064089)(224.82255554,211.15142561)
\curveto(224.40848937,210.70220429)(223.7756775,210.03032996)(222.92411804,209.13580061)
}
}
{
\newrgbcolor{curcolor}{0 0 0}
\pscustom[linestyle=none,fillstyle=solid,fillcolor=curcolor]
{
\newpath
\moveto(255.52357483,212.53204084)
\curveto(255.523572,212.74687998)(255.59779067,212.93242667)(255.74623108,213.08868146)
\curveto(255.89857162,213.24492636)(256.08021207,213.32305128)(256.29115295,213.32305646)
\curveto(256.50989914,213.32305128)(256.69739895,213.24492636)(256.85365295,213.08868146)
\curveto(257.00989864,212.93242667)(257.08802356,212.74687998)(257.08802795,212.53204084)
\curveto(257.08802356,212.31328666)(257.00989864,212.12773997)(256.85365295,211.97540021)
\curveto(256.7013052,211.82305278)(256.51380538,211.74688098)(256.29115295,211.74688459)
\curveto(256.07239958,211.74688098)(255.88880601,211.82109966)(255.7403717,211.96954084)
\curveto(255.59583755,212.11797436)(255.523572,212.30547417)(255.52357483,212.53204084)
\moveto(256.3028717,216.10625959)
\curveto(255.7520874,216.10625162)(255.33997843,215.80937692)(255.06654358,215.21563459)
\curveto(254.79701023,214.62187811)(254.66224474,213.71758213)(254.6622467,212.50274396)
\curveto(254.66224474,211.29180331)(254.79701023,210.38946046)(255.06654358,209.79571271)
\curveto(255.33997843,209.20196165)(255.7520874,208.90508695)(256.3028717,208.90508771)
\curveto(256.85755504,208.90508695)(257.269664,209.20196165)(257.53919983,209.79571271)
\curveto(257.81263221,210.38946046)(257.94935082,211.29180331)(257.94935608,212.50274396)
\curveto(257.94935082,213.71758213)(257.81263221,214.62187811)(257.53919983,215.21563459)
\curveto(257.269664,215.80937692)(256.85755504,216.10625162)(256.3028717,216.10625959)
\moveto(256.3028717,217.04375959)
\curveto(257.23646091,217.04375068)(257.94153833,216.66093857)(258.41810608,215.89532209)
\curveto(258.89856862,215.1296901)(259.13880276,213.99883185)(259.1388092,212.50274396)
\curveto(259.13880276,211.01055359)(258.89856862,209.88164847)(258.41810608,209.11602521)
\curveto(257.94153833,208.3504)(257.23646091,207.96758788)(256.3028717,207.96758771)
\curveto(255.36927528,207.96758788)(254.66419786,208.3504)(254.18763733,209.11602521)
\curveto(253.71107381,209.88164847)(253.4727928,211.01055359)(253.47279358,212.50274396)
\curveto(253.4727928,213.99883185)(253.71107381,215.1296901)(254.18763733,215.89532209)
\curveto(254.66419786,216.66093857)(255.36927528,217.04375068)(256.3028717,217.04375959)
}
}
{
\newrgbcolor{curcolor}{0 0 0}
\pscustom[linestyle=none,fillstyle=solid,fillcolor=curcolor]
{
\newpath
\moveto(286.03128052,209.06084943)
\lineto(287.87112427,209.06084943)
\lineto(287.87112427,215.74639631)
\lineto(285.89065552,215.30108381)
\lineto(285.89065552,216.37920881)
\lineto(287.85940552,216.81280256)
\lineto(289.04299927,216.81280256)
\lineto(289.04299927,209.06084943)
\lineto(290.85940552,209.06084943)
\lineto(290.85940552,208.06475568)
\lineto(286.03128052,208.06475568)
\lineto(286.03128052,209.06084943)
}
}
{
\newrgbcolor{curcolor}{0 0 0}
\pscustom[linestyle=none,fillstyle=solid,fillcolor=curcolor]
{
\newpath
\moveto(321.86859131,211.96124006)
\curveto(321.34124448,211.9612359)(320.93304176,211.81279855)(320.64398193,211.51592756)
\curveto(320.35882358,211.22295539)(320.2162456,210.80694018)(320.21624756,210.26788068)
\curveto(320.2162456,209.72881626)(320.36077671,209.30889481)(320.64984131,209.00811506)
\curveto(320.94280737,208.71123915)(321.34905697,208.5628018)(321.86859131,208.56280256)
\curveto(322.39983717,208.5628018)(322.80803988,208.70928603)(323.09320068,209.00225568)
\curveto(323.38225806,209.29912919)(323.52678917,209.72100377)(323.52679443,210.26788068)
\curveto(323.52678917,210.80303394)(323.38030494,211.21904915)(323.08734131,211.51592756)
\curveto(322.79827427,211.81279855)(322.39202468,211.9612359)(321.86859131,211.96124006)
\moveto(320.83734131,212.45342756)
\curveto(320.33343298,212.58232903)(319.93890213,212.82256317)(319.65374756,213.17413068)
\curveto(319.37249645,213.52568746)(319.23187159,213.94951517)(319.23187256,214.44561506)
\curveto(319.23187159,215.14092022)(319.46819947,215.69170092)(319.94085693,216.09795881)
\curveto(320.41351103,216.50810636)(321.05608851,216.71318428)(321.86859131,216.71319318)
\curveto(322.68499313,216.71318428)(323.32952374,216.50810636)(323.80218506,216.09795881)
\curveto(324.27483529,215.69170092)(324.51116318,215.14092022)(324.51116943,214.44561506)
\curveto(324.51116318,213.94951517)(324.3685852,213.52568746)(324.08343506,213.17413068)
\curveto(323.80217952,212.82256317)(323.40960178,212.58232903)(322.90570068,212.45342756)
\curveto(323.49163295,212.32451679)(323.93889813,212.06475143)(324.24749756,211.67413068)
\curveto(324.55999126,211.28350221)(324.7162411,210.77764334)(324.71624756,210.15655256)
\curveto(324.7162411,209.3674885)(324.46428823,208.75030162)(323.96038818,208.30499006)
\curveto(323.45647674,207.85967751)(322.75921181,207.63702148)(321.86859131,207.63702131)
\curveto(320.97796359,207.63702148)(320.28069866,207.85772438)(319.77679443,208.29913068)
\curveto(319.27679342,208.74444225)(319.02679367,209.35967601)(319.02679443,210.14483381)
\curveto(319.02679367,210.76983085)(319.18109039,211.27764284)(319.48968506,211.66827131)
\curveto(319.80218352,212.0627983)(320.25140182,212.32451679)(320.83734131,212.45342756)
\moveto(320.40960693,214.33428693)
\curveto(320.40960478,213.86553088)(320.53460466,213.50810936)(320.78460693,213.26202131)
\curveto(321.03460416,213.01592235)(321.39593192,212.8928756)(321.86859131,212.89288068)
\curveto(322.34514972,212.8928756)(322.70843061,213.01592235)(322.95843506,213.26202131)
\curveto(323.20843011,213.50810936)(323.33342998,213.86553088)(323.33343506,214.33428693)
\curveto(323.33342998,214.81084243)(323.20843011,215.17412332)(322.95843506,215.42413068)
\curveto(322.71233686,215.67412282)(322.34905597,215.79912269)(321.86859131,215.79913068)
\curveto(321.39593192,215.79912269)(321.03460416,215.67216969)(320.78460693,215.41827131)
\curveto(320.53460466,215.16826395)(320.40960478,214.80693618)(320.40960693,214.33428693)
}
}
{
\newrgbcolor{curcolor}{0 0 0}
\pscustom[linestyle=none,fillstyle=solid,fillcolor=curcolor]
{
\newpath
\moveto(223.32841492,195.65240217)
\curveto(223.32841209,195.86724131)(223.40263076,196.052788)(223.55107117,196.20904279)
\curveto(223.70341171,196.36528769)(223.88505216,196.44341261)(224.09599304,196.44341779)
\curveto(224.31473923,196.44341261)(224.50223904,196.36528769)(224.65849304,196.20904279)
\curveto(224.81473873,196.052788)(224.89286365,195.86724131)(224.89286804,195.65240217)
\curveto(224.89286365,195.43364799)(224.81473873,195.2481013)(224.65849304,195.09576154)
\curveto(224.50614528,194.94341411)(224.31864547,194.86724231)(224.09599304,194.86724592)
\curveto(223.87723966,194.86724231)(223.6936461,194.94146098)(223.54521179,195.08990217)
\curveto(223.40067764,195.23833569)(223.32841209,195.4258355)(223.32841492,195.65240217)
\moveto(224.10771179,199.22662092)
\curveto(223.55692748,199.22661295)(223.14481852,198.92973825)(222.87138367,198.33599592)
\curveto(222.60185031,197.74223943)(222.46708482,196.83794346)(222.46708679,195.62310529)
\curveto(222.46708482,194.41216464)(222.60185031,193.50982179)(222.87138367,192.91607404)
\curveto(223.14481852,192.32232298)(223.55692748,192.02544828)(224.10771179,192.02544904)
\curveto(224.66239513,192.02544828)(225.07450409,192.32232298)(225.34403992,192.91607404)
\curveto(225.6174723,193.50982179)(225.75419091,194.41216464)(225.75419617,195.62310529)
\curveto(225.75419091,196.83794346)(225.6174723,197.74223943)(225.34403992,198.33599592)
\curveto(225.07450409,198.92973825)(224.66239513,199.22661295)(224.10771179,199.22662092)
\moveto(224.10771179,200.16412092)
\curveto(225.041301,200.16411201)(225.74637842,199.78129989)(226.22294617,199.01568342)
\curveto(226.70340871,198.25005143)(226.94364285,197.11919318)(226.94364929,195.62310529)
\curveto(226.94364285,194.13091492)(226.70340871,193.0020098)(226.22294617,192.23638654)
\curveto(225.74637842,191.47076133)(225.041301,191.08794921)(224.10771179,191.08794904)
\curveto(223.17411537,191.08794921)(222.46903795,191.47076133)(221.99247742,192.23638654)
\curveto(221.5159139,193.0020098)(221.27763289,194.13091492)(221.27763367,195.62310529)
\curveto(221.27763289,197.11919318)(221.5159139,198.25005143)(221.99247742,199.01568342)
\curveto(222.46903795,199.78129989)(223.17411537,200.16411201)(224.10771179,200.16412092)
}
}
{
\newrgbcolor{curcolor}{0 0 0}
\pscustom[linestyle=none,fillstyle=solid,fillcolor=curcolor]
{
\newpath
\moveto(255.52357483,195.50774885)
\curveto(255.523572,195.72258799)(255.59779067,195.90813468)(255.74623108,196.06438947)
\curveto(255.89857162,196.22063437)(256.08021207,196.29875929)(256.29115295,196.29876447)
\curveto(256.50989914,196.29875929)(256.69739895,196.22063437)(256.85365295,196.06438947)
\curveto(257.00989864,195.90813468)(257.08802356,195.72258799)(257.08802795,195.50774885)
\curveto(257.08802356,195.28899467)(257.00989864,195.10344798)(256.85365295,194.95110822)
\curveto(256.7013052,194.79876079)(256.51380538,194.72258899)(256.29115295,194.7225926)
\curveto(256.07239958,194.72258899)(255.88880601,194.79680766)(255.7403717,194.94524885)
\curveto(255.59583755,195.09368237)(255.523572,195.28118218)(255.52357483,195.50774885)
\moveto(256.3028717,199.0819676)
\curveto(255.7520874,199.08195963)(255.33997843,198.78508493)(255.06654358,198.1913426)
\curveto(254.79701023,197.59758611)(254.66224474,196.69329014)(254.6622467,195.47845197)
\curveto(254.66224474,194.26751132)(254.79701023,193.36516847)(255.06654358,192.77142072)
\curveto(255.33997843,192.17766966)(255.7520874,191.88079496)(256.3028717,191.88079572)
\curveto(256.85755504,191.88079496)(257.269664,192.17766966)(257.53919983,192.77142072)
\curveto(257.81263221,193.36516847)(257.94935082,194.26751132)(257.94935608,195.47845197)
\curveto(257.94935082,196.69329014)(257.81263221,197.59758611)(257.53919983,198.1913426)
\curveto(257.269664,198.78508493)(256.85755504,199.08195963)(256.3028717,199.0819676)
\moveto(256.3028717,200.0194676)
\curveto(257.23646091,200.01945869)(257.94153833,199.63664657)(258.41810608,198.8710301)
\curveto(258.89856862,198.10539811)(259.13880276,196.97453986)(259.1388092,195.47845197)
\curveto(259.13880276,193.9862616)(258.89856862,192.85735648)(258.41810608,192.09173322)
\curveto(257.94153833,191.32610801)(257.23646091,190.94329589)(256.3028717,190.94329572)
\curveto(255.36927528,190.94329589)(254.66419786,191.32610801)(254.18763733,192.09173322)
\curveto(253.71107381,192.85735648)(253.4727928,193.9862616)(253.47279358,195.47845197)
\curveto(253.4727928,196.97453986)(253.71107381,198.10539811)(254.18763733,198.8710301)
\curveto(254.66419786,199.63664657)(255.36927528,200.01945869)(256.3028717,200.0194676)
}
}
{
\newrgbcolor{curcolor}{0 0 0}
\pscustom[linestyle=none,fillstyle=solid,fillcolor=curcolor]
{
\newpath
\moveto(287.24414062,195.61102033)
\curveto(287.24413779,195.82585947)(287.31835647,196.01140616)(287.46679688,196.16766096)
\curveto(287.61913742,196.32390585)(287.80077786,196.40203077)(288.01171875,196.40203596)
\curveto(288.23046493,196.40203077)(288.41796475,196.32390585)(288.57421875,196.16766096)
\curveto(288.73046443,196.01140616)(288.80858936,195.82585947)(288.80859375,195.61102033)
\curveto(288.80858936,195.39226616)(288.73046443,195.20671947)(288.57421875,195.05437971)
\curveto(288.42187099,194.90203227)(288.23437118,194.82586047)(288.01171875,194.82586408)
\curveto(287.79296537,194.82586047)(287.6093718,194.90007915)(287.4609375,195.04852033)
\curveto(287.31640335,195.19695385)(287.24413779,195.38445366)(287.24414062,195.61102033)
\moveto(288.0234375,199.18523908)
\curveto(287.47265319,199.18523111)(287.06054423,198.88835641)(286.78710938,198.29461408)
\curveto(286.51757602,197.7008576)(286.38281053,196.79656163)(286.3828125,195.58172346)
\curveto(286.38281053,194.3707828)(286.51757602,193.46843996)(286.78710938,192.87469221)
\curveto(287.06054423,192.28094114)(287.47265319,191.98406644)(288.0234375,191.98406721)
\curveto(288.57812084,191.98406644)(288.9902298,192.28094114)(289.25976562,192.87469221)
\curveto(289.53319801,193.46843996)(289.66991662,194.3707828)(289.66992188,195.58172346)
\curveto(289.66991662,196.79656163)(289.53319801,197.7008576)(289.25976562,198.29461408)
\curveto(288.9902298,198.88835641)(288.57812084,199.18523111)(288.0234375,199.18523908)
\moveto(288.0234375,200.12273908)
\curveto(288.95702671,200.12273018)(289.66210413,199.73991806)(290.13867188,198.97430158)
\curveto(290.61913442,198.20866959)(290.85936855,197.07781135)(290.859375,195.58172346)
\curveto(290.85936855,194.08953308)(290.61913442,192.96062796)(290.13867188,192.19500471)
\curveto(289.66210413,191.42937949)(288.95702671,191.04656738)(288.0234375,191.04656721)
\curveto(287.08984107,191.04656738)(286.38476365,191.42937949)(285.90820312,192.19500471)
\curveto(285.43163961,192.96062796)(285.1933586,194.08953308)(285.19335938,195.58172346)
\curveto(285.1933586,197.07781135)(285.43163961,198.20866959)(285.90820312,198.97430158)
\curveto(286.38476365,199.73991806)(287.08984107,200.12273018)(288.0234375,200.12273908)
}
}
{
\newrgbcolor{curcolor}{0 0 0}
\pscustom[linestyle=none,fillstyle=solid,fillcolor=curcolor]
{
\newpath
\moveto(324.07757568,199.14617658)
\lineto(324.07757568,198.05633283)
\curveto(323.83147637,198.20085647)(323.56975788,198.31023136)(323.29241943,198.38445783)
\curveto(323.01507094,198.46257496)(322.72600873,198.50163742)(322.42523193,198.50164533)
\curveto(321.67522853,198.50163742)(321.10686972,198.21843458)(320.72015381,197.65203596)
\curveto(320.333433,197.08952946)(320.14007381,196.25945216)(320.14007568,195.16180158)
\curveto(320.32757363,195.55242162)(320.58733899,195.85124945)(320.91937256,196.05828596)
\curveto(321.25140083,196.26921778)(321.63225982,196.37468642)(322.06195068,196.37469221)
\curveto(322.90569605,196.37468642)(323.55803915,196.11492106)(324.01898193,195.59539533)
\curveto(324.48381947,195.07976584)(324.71624111,194.3473447)(324.71624756,193.39812971)
\curveto(324.71624111,192.45281535)(324.4779601,191.7203942)(324.00140381,191.20086408)
\curveto(323.52483605,190.68133274)(322.85491485,190.42156738)(321.99163818,190.42156721)
\curveto(320.97601048,190.42156738)(320.2318706,190.78484826)(319.75921631,191.51141096)
\curveto(319.28655904,192.24187806)(319.05023115,193.39031441)(319.05023193,194.95672346)
\curveto(319.05023115,196.43328012)(319.333434,197.55827899)(319.89984131,198.33172346)
\curveto(320.47015161,199.10905869)(321.29436954,199.49773018)(322.37249756,199.49773908)
\curveto(322.66155567,199.49773018)(322.95061788,199.46648021)(323.23968506,199.40398908)
\curveto(323.5287423,199.34538658)(323.8080389,199.25944916)(324.07757568,199.14617658)
\moveto(321.96820068,195.44891096)
\curveto(321.46429124,195.4489061)(321.06780726,195.26726566)(320.77874756,194.90398908)
\curveto(320.48968284,194.54070388)(320.34515173,194.03875126)(320.34515381,193.39812971)
\curveto(320.34515173,192.75750254)(320.48968284,192.25554992)(320.77874756,191.89227033)
\curveto(321.06780726,191.52898814)(321.46429124,191.3473477)(321.96820068,191.34734846)
\curveto(322.49163396,191.3473477)(322.88616482,191.51922253)(323.15179443,191.86297346)
\curveto(323.41741429,192.21062809)(323.55022665,192.72234633)(323.55023193,193.39812971)
\curveto(323.55022665,194.07781372)(323.41741429,194.58953196)(323.15179443,194.93328596)
\curveto(322.88616482,195.27703127)(322.49163396,195.4489061)(321.96820068,195.44891096)
}
}
{
\newrgbcolor{curcolor}{0 0 0}
\pscustom[linestyle=none,fillstyle=solid,fillcolor=curcolor]
{
\newpath
\moveto(223.32841492,178.60064436)
\curveto(223.32841209,178.8154835)(223.40263076,179.00103019)(223.55107117,179.15728498)
\curveto(223.70341171,179.31352987)(223.88505216,179.39165479)(224.09599304,179.39165998)
\curveto(224.31473923,179.39165479)(224.50223904,179.31352987)(224.65849304,179.15728498)
\curveto(224.81473873,179.00103019)(224.89286365,178.8154835)(224.89286804,178.60064436)
\curveto(224.89286365,178.38189018)(224.81473873,178.19634349)(224.65849304,178.04400373)
\curveto(224.50614528,177.89165629)(224.31864547,177.8154845)(224.09599304,177.81548811)
\curveto(223.87723966,177.8154845)(223.6936461,177.88970317)(223.54521179,178.03814436)
\curveto(223.40067764,178.18657788)(223.32841209,178.37407769)(223.32841492,178.60064436)
\moveto(224.10771179,182.17486311)
\curveto(223.55692748,182.17485514)(223.14481852,181.87798043)(222.87138367,181.28423811)
\curveto(222.60185031,180.69048162)(222.46708482,179.78618565)(222.46708679,178.57134748)
\curveto(222.46708482,177.36040683)(222.60185031,176.45806398)(222.87138367,175.86431623)
\curveto(223.14481852,175.27056517)(223.55692748,174.97369046)(224.10771179,174.97369123)
\curveto(224.66239513,174.97369046)(225.07450409,175.27056517)(225.34403992,175.86431623)
\curveto(225.6174723,176.45806398)(225.75419091,177.36040683)(225.75419617,178.57134748)
\curveto(225.75419091,179.78618565)(225.6174723,180.69048162)(225.34403992,181.28423811)
\curveto(225.07450409,181.87798043)(224.66239513,182.17485514)(224.10771179,182.17486311)
\moveto(224.10771179,183.11236311)
\curveto(225.041301,183.1123542)(225.74637842,182.72954208)(226.22294617,181.96392561)
\curveto(226.70340871,181.19829361)(226.94364285,180.06743537)(226.94364929,178.57134748)
\curveto(226.94364285,177.07915711)(226.70340871,175.95025199)(226.22294617,175.18462873)
\curveto(225.74637842,174.41900352)(225.041301,174.0361914)(224.10771179,174.03619123)
\curveto(223.17411537,174.0361914)(222.46903795,174.41900352)(221.99247742,175.18462873)
\curveto(221.5159139,175.95025199)(221.27763289,177.07915711)(221.27763367,178.57134748)
\curveto(221.27763289,180.06743537)(221.5159139,181.19829361)(221.99247742,181.96392561)
\curveto(222.46903795,182.72954208)(223.17411537,183.1123542)(224.10771179,183.11236311)
}
}
{
\newrgbcolor{curcolor}{0 0 0}
\pscustom[linestyle=none,fillstyle=solid,fillcolor=curcolor]
{
\newpath
\moveto(255.52357483,178.48357893)
\curveto(255.523572,178.69841807)(255.59779067,178.88396476)(255.74623108,179.04021955)
\curveto(255.89857162,179.19646444)(256.08021207,179.27458937)(256.29115295,179.27459455)
\curveto(256.50989914,179.27458937)(256.69739895,179.19646444)(256.85365295,179.04021955)
\curveto(257.00989864,178.88396476)(257.08802356,178.69841807)(257.08802795,178.48357893)
\curveto(257.08802356,178.26482475)(257.00989864,178.07927806)(256.85365295,177.9269383)
\curveto(256.7013052,177.77459087)(256.51380538,177.69841907)(256.29115295,177.69842268)
\curveto(256.07239958,177.69841907)(255.88880601,177.77263774)(255.7403717,177.92107893)
\curveto(255.59583755,178.06951245)(255.523572,178.25701226)(255.52357483,178.48357893)
\moveto(256.3028717,182.05779768)
\curveto(255.7520874,182.05778971)(255.33997843,181.760915)(255.06654358,181.16717268)
\curveto(254.79701023,180.57341619)(254.66224474,179.66912022)(254.6622467,178.45428205)
\curveto(254.66224474,177.2433414)(254.79701023,176.34099855)(255.06654358,175.7472508)
\curveto(255.33997843,175.15349974)(255.7520874,174.85662503)(256.3028717,174.8566258)
\curveto(256.85755504,174.85662503)(257.269664,175.15349974)(257.53919983,175.7472508)
\curveto(257.81263221,176.34099855)(257.94935082,177.2433414)(257.94935608,178.45428205)
\curveto(257.94935082,179.66912022)(257.81263221,180.57341619)(257.53919983,181.16717268)
\curveto(257.269664,181.760915)(256.85755504,182.05778971)(256.3028717,182.05779768)
\moveto(256.3028717,182.99529768)
\curveto(257.23646091,182.99528877)(257.94153833,182.61247665)(258.41810608,181.84686018)
\curveto(258.89856862,181.08122818)(259.13880276,179.95036994)(259.1388092,178.45428205)
\curveto(259.13880276,176.96209168)(258.89856862,175.83318656)(258.41810608,175.0675633)
\curveto(257.94153833,174.30193809)(257.23646091,173.91912597)(256.3028717,173.9191258)
\curveto(255.36927528,173.91912597)(254.66419786,174.30193809)(254.18763733,175.0675633)
\curveto(253.71107381,175.83318656)(253.4727928,176.96209168)(253.47279358,178.45428205)
\curveto(253.4727928,179.95036994)(253.71107381,181.08122818)(254.18763733,181.84686018)
\curveto(254.66419786,182.61247665)(255.36927528,182.99528877)(256.3028717,182.99529768)
}
}
{
\newrgbcolor{curcolor}{0 0 0}
\pscustom[linestyle=none,fillstyle=solid,fillcolor=curcolor]
{
\newpath
\moveto(287.24414062,178.59295393)
\curveto(287.24413779,178.80779307)(287.31835647,178.99333976)(287.46679688,179.14959455)
\curveto(287.61913742,179.30583944)(287.80077786,179.38396437)(288.01171875,179.38396955)
\curveto(288.23046493,179.38396437)(288.41796475,179.30583944)(288.57421875,179.14959455)
\curveto(288.73046443,178.99333976)(288.80858936,178.80779307)(288.80859375,178.59295393)
\curveto(288.80858936,178.37419975)(288.73046443,178.18865306)(288.57421875,178.0363133)
\curveto(288.42187099,177.88396587)(288.23437118,177.80779407)(288.01171875,177.80779768)
\curveto(287.79296537,177.80779407)(287.6093718,177.88201274)(287.4609375,178.03045393)
\curveto(287.31640335,178.17888745)(287.24413779,178.36638726)(287.24414062,178.59295393)
\moveto(288.0234375,182.16717268)
\curveto(287.47265319,182.16716471)(287.06054423,181.87029)(286.78710938,181.27654768)
\curveto(286.51757602,180.68279119)(286.38281053,179.77849522)(286.3828125,178.56365705)
\curveto(286.38281053,177.3527164)(286.51757602,176.45037355)(286.78710938,175.8566258)
\curveto(287.06054423,175.26287474)(287.47265319,174.96600003)(288.0234375,174.9660008)
\curveto(288.57812084,174.96600003)(288.9902298,175.26287474)(289.25976562,175.8566258)
\curveto(289.53319801,176.45037355)(289.66991662,177.3527164)(289.66992188,178.56365705)
\curveto(289.66991662,179.77849522)(289.53319801,180.68279119)(289.25976562,181.27654768)
\curveto(288.9902298,181.87029)(288.57812084,182.16716471)(288.0234375,182.16717268)
\moveto(288.0234375,183.10467268)
\curveto(288.95702671,183.10466377)(289.66210413,182.72185165)(290.13867188,181.95623518)
\curveto(290.61913442,181.19060318)(290.85936855,180.05974494)(290.859375,178.56365705)
\curveto(290.85936855,177.07146668)(290.61913442,175.94256156)(290.13867188,175.1769383)
\curveto(289.66210413,174.41131309)(288.95702671,174.02850097)(288.0234375,174.0285008)
\curveto(287.08984107,174.02850097)(286.38476365,174.41131309)(285.90820312,175.1769383)
\curveto(285.43163961,175.94256156)(285.1933586,177.07146668)(285.19335938,178.56365705)
\curveto(285.1933586,180.05974494)(285.43163961,181.19060318)(285.90820312,181.95623518)
\curveto(286.38476365,182.72185165)(287.08984107,183.10466377)(288.0234375,183.10467268)
}
}
{
\newrgbcolor{curcolor}{0 0 0}
\pscustom[linestyle=none,fillstyle=solid,fillcolor=curcolor]
{
\newpath
\moveto(319.88812256,174.53045393)
\lineto(321.72796631,174.53045393)
\lineto(321.72796631,181.2160008)
\lineto(319.74749756,180.7706883)
\lineto(319.74749756,181.8488133)
\lineto(321.71624756,182.28240705)
\lineto(322.89984131,182.28240705)
\lineto(322.89984131,174.53045393)
\lineto(324.71624756,174.53045393)
\lineto(324.71624756,173.53436018)
\lineto(319.88812256,173.53436018)
\lineto(319.88812256,174.53045393)
}
}
{
\newrgbcolor{curcolor}{0 0 0}
\pscustom[linestyle=none,fillstyle=solid,fillcolor=curcolor]
{
\newpath
\moveto(223.32841492,161.54888654)
\curveto(223.32841209,161.76372568)(223.40263076,161.94927237)(223.55107117,162.10552717)
\curveto(223.70341171,162.26177206)(223.88505216,162.33989698)(224.09599304,162.33990217)
\curveto(224.31473923,162.33989698)(224.50223904,162.26177206)(224.65849304,162.10552717)
\curveto(224.81473873,161.94927237)(224.89286365,161.76372568)(224.89286804,161.54888654)
\curveto(224.89286365,161.33013237)(224.81473873,161.14458568)(224.65849304,160.99224592)
\curveto(224.50614528,160.83989848)(224.31864547,160.76372668)(224.09599304,160.76373029)
\curveto(223.87723966,160.76372668)(223.6936461,160.83794536)(223.54521179,160.98638654)
\curveto(223.40067764,161.13482006)(223.32841209,161.32231988)(223.32841492,161.54888654)
\moveto(224.10771179,165.12310529)
\curveto(223.55692748,165.12309732)(223.14481852,164.82622262)(222.87138367,164.23248029)
\curveto(222.60185031,163.63872381)(222.46708482,162.73442784)(222.46708679,161.51958967)
\curveto(222.46708482,160.30864901)(222.60185031,159.40630617)(222.87138367,158.81255842)
\curveto(223.14481852,158.21880735)(223.55692748,157.92193265)(224.10771179,157.92193342)
\curveto(224.66239513,157.92193265)(225.07450409,158.21880735)(225.34403992,158.81255842)
\curveto(225.6174723,159.40630617)(225.75419091,160.30864901)(225.75419617,161.51958967)
\curveto(225.75419091,162.73442784)(225.6174723,163.63872381)(225.34403992,164.23248029)
\curveto(225.07450409,164.82622262)(224.66239513,165.12309732)(224.10771179,165.12310529)
\moveto(224.10771179,166.06060529)
\curveto(225.041301,166.06059639)(225.74637842,165.67778427)(226.22294617,164.91216779)
\curveto(226.70340871,164.1465358)(226.94364285,163.01567756)(226.94364929,161.51958967)
\curveto(226.94364285,160.02739929)(226.70340871,158.89849417)(226.22294617,158.13287092)
\curveto(225.74637842,157.36724571)(225.041301,156.98443359)(224.10771179,156.98443342)
\curveto(223.17411537,156.98443359)(222.46903795,157.36724571)(221.99247742,158.13287092)
\curveto(221.5159139,158.89849417)(221.27763289,160.02739929)(221.27763367,161.51958967)
\curveto(221.27763289,163.01567756)(221.5159139,164.1465358)(221.99247742,164.91216779)
\curveto(222.46903795,165.67778427)(223.17411537,166.06059639)(224.10771179,166.06060529)
}
}
{
\newrgbcolor{curcolor}{0 0 0}
\pscustom[linestyle=none,fillstyle=solid,fillcolor=curcolor]
{
\newpath
\moveto(255.52357483,161.45928693)
\curveto(255.523572,161.67412607)(255.59779067,161.85967276)(255.74623108,162.01592756)
\curveto(255.89857162,162.17217245)(256.08021207,162.25029737)(256.29115295,162.25030256)
\curveto(256.50989914,162.25029737)(256.69739895,162.17217245)(256.85365295,162.01592756)
\curveto(257.00989864,161.85967276)(257.08802356,161.67412607)(257.08802795,161.45928693)
\curveto(257.08802356,161.24053276)(257.00989864,161.05498607)(256.85365295,160.90264631)
\curveto(256.7013052,160.75029887)(256.51380538,160.67412707)(256.29115295,160.67413068)
\curveto(256.07239958,160.67412707)(255.88880601,160.74834575)(255.7403717,160.89678693)
\curveto(255.59583755,161.04522045)(255.523572,161.23272027)(255.52357483,161.45928693)
\moveto(256.3028717,165.03350568)
\curveto(255.7520874,165.03349771)(255.33997843,164.73662301)(255.06654358,164.14288068)
\curveto(254.79701023,163.5491242)(254.66224474,162.64482823)(254.6622467,161.42999006)
\curveto(254.66224474,160.2190494)(254.79701023,159.31670656)(255.06654358,158.72295881)
\curveto(255.33997843,158.12920774)(255.7520874,157.83233304)(256.3028717,157.83233381)
\curveto(256.85755504,157.83233304)(257.269664,158.12920774)(257.53919983,158.72295881)
\curveto(257.81263221,159.31670656)(257.94935082,160.2190494)(257.94935608,161.42999006)
\curveto(257.94935082,162.64482823)(257.81263221,163.5491242)(257.53919983,164.14288068)
\curveto(257.269664,164.73662301)(256.85755504,165.03349771)(256.3028717,165.03350568)
\moveto(256.3028717,165.97100568)
\curveto(257.23646091,165.97099678)(257.94153833,165.58818466)(258.41810608,164.82256818)
\curveto(258.89856862,164.05693619)(259.13880276,162.92607795)(259.1388092,161.42999006)
\curveto(259.13880276,159.93779969)(258.89856862,158.80889456)(258.41810608,158.04327131)
\curveto(257.94153833,157.2776461)(257.23646091,156.89483398)(256.3028717,156.89483381)
\curveto(255.36927528,156.89483398)(254.66419786,157.2776461)(254.18763733,158.04327131)
\curveto(253.71107381,158.80889456)(253.4727928,159.93779969)(253.47279358,161.42999006)
\curveto(253.4727928,162.92607795)(253.71107381,164.05693619)(254.18763733,164.82256818)
\curveto(254.66419786,165.58818466)(255.36927528,165.97099678)(256.3028717,165.97100568)
}
}
{
\newrgbcolor{curcolor}{0 0 0}
\pscustom[linestyle=none,fillstyle=solid,fillcolor=curcolor]
{
\newpath
\moveto(287.24414062,161.57476545)
\curveto(287.24413779,161.78960459)(287.31835647,161.97515128)(287.46679688,162.13140607)
\curveto(287.61913742,162.28765097)(287.80077786,162.36577589)(288.01171875,162.36578107)
\curveto(288.23046493,162.36577589)(288.41796475,162.28765097)(288.57421875,162.13140607)
\curveto(288.73046443,161.97515128)(288.80858936,161.78960459)(288.80859375,161.57476545)
\curveto(288.80858936,161.35601127)(288.73046443,161.17046458)(288.57421875,161.01812482)
\curveto(288.42187099,160.86577739)(288.23437118,160.78960559)(288.01171875,160.7896092)
\curveto(287.79296537,160.78960559)(287.6093718,160.86382427)(287.4609375,161.01226545)
\curveto(287.31640335,161.16069897)(287.24413779,161.34819878)(287.24414062,161.57476545)
\moveto(288.0234375,165.1489842)
\curveto(287.47265319,165.14897623)(287.06054423,164.85210153)(286.78710938,164.2583592)
\curveto(286.51757602,163.66460271)(286.38281053,162.76030674)(286.3828125,161.54546857)
\curveto(286.38281053,160.33452792)(286.51757602,159.43218507)(286.78710938,158.83843732)
\curveto(287.06054423,158.24468626)(287.47265319,157.94781156)(288.0234375,157.94781232)
\curveto(288.57812084,157.94781156)(288.9902298,158.24468626)(289.25976562,158.83843732)
\curveto(289.53319801,159.43218507)(289.66991662,160.33452792)(289.66992188,161.54546857)
\curveto(289.66991662,162.76030674)(289.53319801,163.66460271)(289.25976562,164.2583592)
\curveto(288.9902298,164.85210153)(288.57812084,165.14897623)(288.0234375,165.1489842)
\moveto(288.0234375,166.0864842)
\curveto(288.95702671,166.08647529)(289.66210413,165.70366318)(290.13867188,164.9380467)
\curveto(290.61913442,164.17241471)(290.85936855,163.04155646)(290.859375,161.54546857)
\curveto(290.85936855,160.0532782)(290.61913442,158.92437308)(290.13867188,158.15874982)
\curveto(289.66210413,157.39312461)(288.95702671,157.01031249)(288.0234375,157.01031232)
\curveto(287.08984107,157.01031249)(286.38476365,157.39312461)(285.90820312,158.15874982)
\curveto(285.43163961,158.92437308)(285.1933586,160.0532782)(285.19335938,161.54546857)
\curveto(285.1933586,163.04155646)(285.43163961,164.17241471)(285.90820312,164.9380467)
\curveto(286.38476365,165.70366318)(287.08984107,166.08647529)(288.0234375,166.0864842)
}
}
{
\newrgbcolor{curcolor}{0 0 0}
\pscustom[linestyle=none,fillstyle=solid,fillcolor=curcolor]
{
\newpath
\moveto(319.66546631,165.39507795)
\lineto(324.09515381,165.39507795)
\lineto(324.09515381,164.3989842)
\lineto(320.74359131,164.3989842)
\lineto(320.74359131,162.24859357)
\curveto(320.9115576,162.31108791)(321.07952618,162.35600974)(321.24749756,162.3833592)
\curveto(321.41936959,162.41460343)(321.59124442,162.43022842)(321.76312256,162.4302342)
\curveto(322.66936834,162.43022842)(323.38811762,162.16265056)(323.91937256,161.62749982)
\curveto(324.45061656,161.09233913)(324.71624129,160.36773048)(324.71624756,159.4536717)
\curveto(324.71624129,158.53179481)(324.4369447,157.80523304)(323.87835693,157.2739842)
\curveto(323.32366456,156.7427341)(322.5638997,156.47710937)(321.59906006,156.4771092)
\curveto(321.13421363,156.47710937)(320.7084328,156.50835934)(320.32171631,156.5708592)
\curveto(319.93890232,156.63335921)(319.59515267,156.72710912)(319.29046631,156.8521092)
\lineto(319.29046631,158.05328107)
\curveto(319.64984011,157.85796736)(320.01116788,157.71148313)(320.37445068,157.61382795)
\curveto(320.73772965,157.52007708)(321.10882303,157.47320212)(321.48773193,157.47320295)
\curveto(322.140072,157.47320212)(322.64202462,157.64507695)(322.99359131,157.98882795)
\curveto(323.34905516,158.33257626)(323.52678936,158.82085703)(323.52679443,159.4536717)
\curveto(323.52678936,160.07866827)(323.34319579,160.56499591)(322.97601318,160.91265607)
\curveto(322.61272777,161.26030771)(322.10491578,161.43413566)(321.45257568,161.43414045)
\curveto(321.13616675,161.43413566)(320.82757331,161.39702632)(320.52679443,161.32281232)
\curveto(320.22601141,161.25249522)(319.93890232,161.14507345)(319.66546631,161.0005467)
\lineto(319.66546631,165.39507795)
}
}
{
\newrgbcolor{curcolor}{0 0 0}
\pscustom[linestyle=none,fillstyle=solid,fillcolor=curcolor]
{
\newpath
\moveto(223.32841492,144.49712873)
\curveto(223.32841209,144.71196787)(223.40263076,144.89751456)(223.55107117,145.05376936)
\curveto(223.70341171,145.21001425)(223.88505216,145.28813917)(224.09599304,145.28814436)
\curveto(224.31473923,145.28813917)(224.50223904,145.21001425)(224.65849304,145.05376936)
\curveto(224.81473873,144.89751456)(224.89286365,144.71196787)(224.89286804,144.49712873)
\curveto(224.89286365,144.27837455)(224.81473873,144.09282787)(224.65849304,143.94048811)
\curveto(224.50614528,143.78814067)(224.31864547,143.71196887)(224.09599304,143.71197248)
\curveto(223.87723966,143.71196887)(223.6936461,143.78618755)(223.54521179,143.93462873)
\curveto(223.40067764,144.08306225)(223.32841209,144.27056206)(223.32841492,144.49712873)
\moveto(224.10771179,148.07134748)
\curveto(223.55692748,148.07133951)(223.14481852,147.77446481)(222.87138367,147.18072248)
\curveto(222.60185031,146.586966)(222.46708482,145.68267003)(222.46708679,144.46783186)
\curveto(222.46708482,143.2568912)(222.60185031,142.35454835)(222.87138367,141.76080061)
\curveto(223.14481852,141.16704954)(223.55692748,140.87017484)(224.10771179,140.87017561)
\curveto(224.66239513,140.87017484)(225.07450409,141.16704954)(225.34403992,141.76080061)
\curveto(225.6174723,142.35454835)(225.75419091,143.2568912)(225.75419617,144.46783186)
\curveto(225.75419091,145.68267003)(225.6174723,146.586966)(225.34403992,147.18072248)
\curveto(225.07450409,147.77446481)(224.66239513,148.07133951)(224.10771179,148.07134748)
\moveto(224.10771179,149.00884748)
\curveto(225.041301,149.00883857)(225.74637842,148.62602646)(226.22294617,147.86040998)
\curveto(226.70340871,147.09477799)(226.94364285,145.96391974)(226.94364929,144.46783186)
\curveto(226.94364285,142.97564148)(226.70340871,141.84673636)(226.22294617,141.08111311)
\curveto(225.74637842,140.31548789)(225.041301,139.93267578)(224.10771179,139.93267561)
\curveto(223.17411537,139.93267578)(222.46903795,140.31548789)(221.99247742,141.08111311)
\curveto(221.5159139,141.84673636)(221.27763289,142.97564148)(221.27763367,144.46783186)
\curveto(221.27763289,145.96391974)(221.5159139,147.09477799)(221.99247742,147.86040998)
\curveto(222.46903795,148.62602646)(223.17411537,149.00883857)(224.10771179,149.00884748)
}
}
{
\newrgbcolor{curcolor}{0 0 0}
\pscustom[linestyle=none,fillstyle=solid,fillcolor=curcolor]
{
\newpath
\moveto(255.52357483,144.43511701)
\curveto(255.523572,144.64995615)(255.59779067,144.83550284)(255.74623108,144.99175764)
\curveto(255.89857162,145.14800253)(256.08021207,145.22612745)(256.29115295,145.22613264)
\curveto(256.50989914,145.22612745)(256.69739895,145.14800253)(256.85365295,144.99175764)
\curveto(257.00989864,144.83550284)(257.08802356,144.64995615)(257.08802795,144.43511701)
\curveto(257.08802356,144.21636284)(257.00989864,144.03081615)(256.85365295,143.87847639)
\curveto(256.7013052,143.72612895)(256.51380538,143.64995715)(256.29115295,143.64996076)
\curveto(256.07239958,143.64995715)(255.88880601,143.72417583)(255.7403717,143.87261701)
\curveto(255.59583755,144.02105053)(255.523572,144.20855034)(255.52357483,144.43511701)
\moveto(256.3028717,148.00933576)
\curveto(255.7520874,148.00932779)(255.33997843,147.71245309)(255.06654358,147.11871076)
\curveto(254.79701023,146.52495428)(254.66224474,145.62065831)(254.6622467,144.40582014)
\curveto(254.66224474,143.19487948)(254.79701023,142.29253663)(255.06654358,141.69878889)
\curveto(255.33997843,141.10503782)(255.7520874,140.80816312)(256.3028717,140.80816389)
\curveto(256.85755504,140.80816312)(257.269664,141.10503782)(257.53919983,141.69878889)
\curveto(257.81263221,142.29253663)(257.94935082,143.19487948)(257.94935608,144.40582014)
\curveto(257.94935082,145.62065831)(257.81263221,146.52495428)(257.53919983,147.11871076)
\curveto(257.269664,147.71245309)(256.85755504,148.00932779)(256.3028717,148.00933576)
\moveto(256.3028717,148.94683576)
\curveto(257.23646091,148.94682686)(257.94153833,148.56401474)(258.41810608,147.79839826)
\curveto(258.89856862,147.03276627)(259.13880276,145.90190803)(259.1388092,144.40582014)
\curveto(259.13880276,142.91362976)(258.89856862,141.78472464)(258.41810608,141.01910139)
\curveto(257.94153833,140.25347617)(257.23646091,139.87066406)(256.3028717,139.87066389)
\curveto(255.36927528,139.87066406)(254.66419786,140.25347617)(254.18763733,141.01910139)
\curveto(253.71107381,141.78472464)(253.4727928,142.91362976)(253.47279358,144.40582014)
\curveto(253.4727928,145.90190803)(253.71107381,147.03276627)(254.18763733,147.79839826)
\curveto(254.66419786,148.56401474)(255.36927528,148.94682686)(256.3028717,148.94683576)
}
}
{
\newrgbcolor{curcolor}{0 0 0}
\pscustom[linestyle=none,fillstyle=solid,fillcolor=curcolor]
{
\newpath
\moveto(287.24414062,144.55657697)
\curveto(287.24413779,144.77141611)(287.31835647,144.9569628)(287.46679688,145.1132176)
\curveto(287.61913742,145.26946249)(287.80077786,145.34758741)(288.01171875,145.3475926)
\curveto(288.23046493,145.34758741)(288.41796475,145.26946249)(288.57421875,145.1132176)
\curveto(288.73046443,144.9569628)(288.80858936,144.77141611)(288.80859375,144.55657697)
\curveto(288.80858936,144.3378228)(288.73046443,144.15227611)(288.57421875,143.99993635)
\curveto(288.42187099,143.84758891)(288.23437118,143.77141711)(288.01171875,143.77142072)
\curveto(287.79296537,143.77141711)(287.6093718,143.84563579)(287.4609375,143.99407697)
\curveto(287.31640335,144.14251049)(287.24413779,144.3300103)(287.24414062,144.55657697)
\moveto(288.0234375,148.13079572)
\curveto(287.47265319,148.13078775)(287.06054423,147.83391305)(286.78710938,147.24017072)
\curveto(286.51757602,146.64641424)(286.38281053,145.74211827)(286.3828125,144.5272801)
\curveto(286.38281053,143.31633944)(286.51757602,142.4139966)(286.78710938,141.82024885)
\curveto(287.06054423,141.22649778)(287.47265319,140.92962308)(288.0234375,140.92962385)
\curveto(288.57812084,140.92962308)(288.9902298,141.22649778)(289.25976562,141.82024885)
\curveto(289.53319801,142.4139966)(289.66991662,143.31633944)(289.66992188,144.5272801)
\curveto(289.66991662,145.74211827)(289.53319801,146.64641424)(289.25976562,147.24017072)
\curveto(288.9902298,147.83391305)(288.57812084,148.13078775)(288.0234375,148.13079572)
\moveto(288.0234375,149.06829572)
\curveto(288.95702671,149.06828682)(289.66210413,148.6854747)(290.13867188,147.91985822)
\curveto(290.61913442,147.15422623)(290.85936855,146.02336799)(290.859375,144.5272801)
\curveto(290.85936855,143.03508972)(290.61913442,141.9061846)(290.13867188,141.14056135)
\curveto(289.66210413,140.37493613)(288.95702671,139.99212402)(288.0234375,139.99212385)
\curveto(287.08984107,139.99212402)(286.38476365,140.37493613)(285.90820312,141.14056135)
\curveto(285.43163961,141.9061846)(285.1933586,143.03508972)(285.19335938,144.5272801)
\curveto(285.1933586,146.02336799)(285.43163961,147.15422623)(285.90820312,147.91985822)
\curveto(286.38476365,148.6854747)(287.08984107,149.06828682)(288.0234375,149.06829572)
}
}
{
\newrgbcolor{curcolor}{0 0 0}
\pscustom[linestyle=none,fillstyle=solid,fillcolor=curcolor]
{
\newpath
\moveto(320.69671631,140.42767072)
\lineto(324.71624756,140.42767072)
\lineto(324.71624756,139.43157697)
\lineto(319.40179443,139.43157697)
\lineto(319.40179443,140.42767072)
\curveto(320.13226156,141.19720021)(320.7709328,141.87688703)(321.31781006,142.46673322)
\curveto(321.86468171,143.05657335)(322.24163445,143.47258856)(322.44866943,143.7147801)
\curveto(322.83929011,144.19133784)(323.10296172,144.57610308)(323.23968506,144.86907697)
\curveto(323.37639894,145.16594624)(323.44475825,145.46868031)(323.44476318,145.7772801)
\curveto(323.44475825,146.26555451)(323.30022714,146.64836663)(323.01116943,146.9257176)
\curveto(322.72600897,147.20305358)(322.33343124,147.34172531)(321.83343506,147.34173322)
\curveto(321.47796334,147.34172531)(321.10491684,147.27727225)(320.71429443,147.14837385)
\curveto(320.32366762,147.01946001)(319.90960554,146.82414771)(319.47210693,146.56243635)
\lineto(319.47210693,147.75774885)
\curveto(319.87444932,147.94914658)(320.26898018,148.09367769)(320.65570068,148.1913426)
\curveto(321.04632315,148.28898999)(321.43108839,148.33781807)(321.80999756,148.33782697)
\curveto(322.66546215,148.33781807)(323.35296147,148.10930267)(323.87249756,147.6522801)
\curveto(324.39592917,147.19914733)(324.65764766,146.6034448)(324.65765381,145.86517072)
\curveto(324.65764766,145.49016466)(324.56975713,145.11516504)(324.39398193,144.74017072)
\curveto(324.22210122,144.36516579)(323.9408515,143.9511037)(323.55023193,143.49798322)
\curveto(323.33147711,143.24407316)(323.01311806,142.89251101)(322.59515381,142.44329572)
\curveto(322.18108764,141.99407441)(321.54827577,141.32220008)(320.69671631,140.42767072)
}
}
{
\newrgbcolor{curcolor}{0 0 0}
\pscustom[linestyle=none,fillstyle=solid,fillcolor=curcolor]
{
\newpath
\moveto(223.32841492,127.44537092)
\curveto(223.32841209,127.66021006)(223.40263076,127.84575675)(223.55107117,128.00201154)
\curveto(223.70341171,128.15825644)(223.88505216,128.23638136)(224.09599304,128.23638654)
\curveto(224.31473923,128.23638136)(224.50223904,128.15825644)(224.65849304,128.00201154)
\curveto(224.81473873,127.84575675)(224.89286365,127.66021006)(224.89286804,127.44537092)
\curveto(224.89286365,127.22661674)(224.81473873,127.04107005)(224.65849304,126.88873029)
\curveto(224.50614528,126.73638286)(224.31864547,126.66021106)(224.09599304,126.66021467)
\curveto(223.87723966,126.66021106)(223.6936461,126.73442973)(223.54521179,126.88287092)
\curveto(223.40067764,127.03130444)(223.32841209,127.21880425)(223.32841492,127.44537092)
\moveto(224.10771179,131.01958967)
\curveto(223.55692748,131.0195817)(223.14481852,130.722707)(222.87138367,130.12896467)
\curveto(222.60185031,129.53520818)(222.46708482,128.63091221)(222.46708679,127.41607404)
\curveto(222.46708482,126.20513339)(222.60185031,125.30279054)(222.87138367,124.70904279)
\curveto(223.14481852,124.11529173)(223.55692748,123.81841703)(224.10771179,123.81841779)
\curveto(224.66239513,123.81841703)(225.07450409,124.11529173)(225.34403992,124.70904279)
\curveto(225.6174723,125.30279054)(225.75419091,126.20513339)(225.75419617,127.41607404)
\curveto(225.75419091,128.63091221)(225.6174723,129.53520818)(225.34403992,130.12896467)
\curveto(225.07450409,130.722707)(224.66239513,131.0195817)(224.10771179,131.01958967)
\moveto(224.10771179,131.95708967)
\curveto(225.041301,131.95708076)(225.74637842,131.57426864)(226.22294617,130.80865217)
\curveto(226.70340871,130.04302018)(226.94364285,128.91216193)(226.94364929,127.41607404)
\curveto(226.94364285,125.92388367)(226.70340871,124.79497855)(226.22294617,124.02935529)
\curveto(225.74637842,123.26373008)(225.041301,122.88091796)(224.10771179,122.88091779)
\curveto(223.17411537,122.88091796)(222.46903795,123.26373008)(221.99247742,124.02935529)
\curveto(221.5159139,124.79497855)(221.27763289,125.92388367)(221.27763367,127.41607404)
\curveto(221.27763289,128.91216193)(221.5159139,130.04302018)(221.99247742,130.80865217)
\curveto(222.46903795,131.57426864)(223.17411537,131.95708076)(224.10771179,131.95708967)
}
}
{
\newrgbcolor{curcolor}{0 0 0}
\pscustom[linestyle=none,fillstyle=solid,fillcolor=curcolor]
{
\newpath
\moveto(255.52357483,127.41082502)
\curveto(255.523572,127.62566416)(255.59779067,127.81121085)(255.74623108,127.96746564)
\curveto(255.89857162,128.12371054)(256.08021207,128.20183546)(256.29115295,128.20184064)
\curveto(256.50989914,128.20183546)(256.69739895,128.12371054)(256.85365295,127.96746564)
\curveto(257.00989864,127.81121085)(257.08802356,127.62566416)(257.08802795,127.41082502)
\curveto(257.08802356,127.19207084)(257.00989864,127.00652415)(256.85365295,126.85418439)
\curveto(256.7013052,126.70183696)(256.51380538,126.62566516)(256.29115295,126.62566877)
\curveto(256.07239958,126.62566516)(255.88880601,126.69988384)(255.7403717,126.84832502)
\curveto(255.59583755,126.99675854)(255.523572,127.18425835)(255.52357483,127.41082502)
\moveto(256.3028717,130.98504377)
\curveto(255.7520874,130.9850358)(255.33997843,130.6881611)(255.06654358,130.09441877)
\curveto(254.79701023,129.50066229)(254.66224474,128.59636631)(254.6622467,127.38152814)
\curveto(254.66224474,126.17058749)(254.79701023,125.26824464)(255.06654358,124.67449689)
\curveto(255.33997843,124.08074583)(255.7520874,123.78387113)(256.3028717,123.78387189)
\curveto(256.85755504,123.78387113)(257.269664,124.08074583)(257.53919983,124.67449689)
\curveto(257.81263221,125.26824464)(257.94935082,126.17058749)(257.94935608,127.38152814)
\curveto(257.94935082,128.59636631)(257.81263221,129.50066229)(257.53919983,130.09441877)
\curveto(257.269664,130.6881611)(256.85755504,130.9850358)(256.3028717,130.98504377)
\moveto(256.3028717,131.92254377)
\curveto(257.23646091,131.92253486)(257.94153833,131.53972275)(258.41810608,130.77410627)
\curveto(258.89856862,130.00847428)(259.13880276,128.87761603)(259.1388092,127.38152814)
\curveto(259.13880276,125.88933777)(258.89856862,124.76043265)(258.41810608,123.99480939)
\curveto(257.94153833,123.22918418)(257.23646091,122.84637206)(256.3028717,122.84637189)
\curveto(255.36927528,122.84637206)(254.66419786,123.22918418)(254.18763733,123.99480939)
\curveto(253.71107381,124.76043265)(253.4727928,125.88933777)(253.47279358,127.38152814)
\curveto(253.4727928,128.87761603)(253.71107381,130.00847428)(254.18763733,130.77410627)
\curveto(254.66419786,131.53972275)(255.36927528,131.92253486)(256.3028717,131.92254377)
}
}
{
\newrgbcolor{curcolor}{0 0 0}
\pscustom[linestyle=none,fillstyle=solid,fillcolor=curcolor]
{
\newpath
\moveto(287.24414062,127.5383885)
\curveto(287.24413779,127.75322764)(287.31835647,127.93877433)(287.46679688,128.09502912)
\curveto(287.61913742,128.25127401)(287.80077786,128.32939894)(288.01171875,128.32940412)
\curveto(288.23046493,128.32939894)(288.41796475,128.25127401)(288.57421875,128.09502912)
\curveto(288.73046443,127.93877433)(288.80858936,127.75322764)(288.80859375,127.5383885)
\curveto(288.80858936,127.31963432)(288.73046443,127.13408763)(288.57421875,126.98174787)
\curveto(288.42187099,126.82940044)(288.23437118,126.75322864)(288.01171875,126.75323225)
\curveto(287.79296537,126.75322864)(287.6093718,126.82744731)(287.4609375,126.9758885)
\curveto(287.31640335,127.12432202)(287.24413779,127.31182183)(287.24414062,127.5383885)
\moveto(288.0234375,131.11260725)
\curveto(287.47265319,131.11259928)(287.06054423,130.81572457)(286.78710938,130.22198225)
\curveto(286.51757602,129.62822576)(286.38281053,128.72392979)(286.3828125,127.50909162)
\curveto(286.38281053,126.29815097)(286.51757602,125.39580812)(286.78710938,124.80206037)
\curveto(287.06054423,124.20830931)(287.47265319,123.9114346)(288.0234375,123.91143537)
\curveto(288.57812084,123.9114346)(288.9902298,124.20830931)(289.25976562,124.80206037)
\curveto(289.53319801,125.39580812)(289.66991662,126.29815097)(289.66992188,127.50909162)
\curveto(289.66991662,128.72392979)(289.53319801,129.62822576)(289.25976562,130.22198225)
\curveto(288.9902298,130.81572457)(288.57812084,131.11259928)(288.0234375,131.11260725)
\moveto(288.0234375,132.05010725)
\curveto(288.95702671,132.05009834)(289.66210413,131.66728622)(290.13867188,130.90166975)
\curveto(290.61913442,130.13603775)(290.85936855,129.00517951)(290.859375,127.50909162)
\curveto(290.85936855,126.01690125)(290.61913442,124.88799613)(290.13867188,124.12237287)
\curveto(289.66210413,123.35674766)(288.95702671,122.97393554)(288.0234375,122.97393537)
\curveto(287.08984107,122.97393554)(286.38476365,123.35674766)(285.90820312,124.12237287)
\curveto(285.43163961,124.88799613)(285.1933586,126.01690125)(285.19335938,127.50909162)
\curveto(285.1933586,129.00517951)(285.43163961,130.13603775)(285.90820312,130.90166975)
\curveto(286.38476365,131.66728622)(287.08984107,132.05009834)(288.0234375,132.05010725)
}
}
{
\newrgbcolor{curcolor}{0 0 0}
\pscustom[linestyle=none,fillstyle=solid,fillcolor=curcolor]
{
\newpath
\moveto(319.88812256,123.54034162)
\lineto(321.72796631,123.54034162)
\lineto(321.72796631,130.2258885)
\lineto(319.74749756,129.780576)
\lineto(319.74749756,130.858701)
\lineto(321.71624756,131.29229475)
\lineto(322.89984131,131.29229475)
\lineto(322.89984131,123.54034162)
\lineto(324.71624756,123.54034162)
\lineto(324.71624756,122.54424787)
\lineto(319.88812256,122.54424787)
\lineto(319.88812256,123.54034162)
}
}
{
\newrgbcolor{curcolor}{0 0 0}
\pscustom[linestyle=none,fillstyle=solid,fillcolor=curcolor]
{
\newpath
\moveto(223.32841492,110.39361311)
\curveto(223.32841209,110.60845225)(223.40263076,110.79399894)(223.55107117,110.95025373)
\curveto(223.70341171,111.10649862)(223.88505216,111.18462354)(224.09599304,111.18462873)
\curveto(224.31473923,111.18462354)(224.50223904,111.10649862)(224.65849304,110.95025373)
\curveto(224.81473873,110.79399894)(224.89286365,110.60845225)(224.89286804,110.39361311)
\curveto(224.89286365,110.17485893)(224.81473873,109.98931224)(224.65849304,109.83697248)
\curveto(224.50614528,109.68462504)(224.31864547,109.60845325)(224.09599304,109.60845686)
\curveto(223.87723966,109.60845325)(223.6936461,109.68267192)(223.54521179,109.83111311)
\curveto(223.40067764,109.97954663)(223.32841209,110.16704644)(223.32841492,110.39361311)
\moveto(224.10771179,113.96783186)
\curveto(223.55692748,113.96782389)(223.14481852,113.67094918)(222.87138367,113.07720686)
\curveto(222.60185031,112.48345037)(222.46708482,111.5791544)(222.46708679,110.36431623)
\curveto(222.46708482,109.15337558)(222.60185031,108.25103273)(222.87138367,107.65728498)
\curveto(223.14481852,107.06353392)(223.55692748,106.76665921)(224.10771179,106.76665998)
\curveto(224.66239513,106.76665921)(225.07450409,107.06353392)(225.34403992,107.65728498)
\curveto(225.6174723,108.25103273)(225.75419091,109.15337558)(225.75419617,110.36431623)
\curveto(225.75419091,111.5791544)(225.6174723,112.48345037)(225.34403992,113.07720686)
\curveto(225.07450409,113.67094918)(224.66239513,113.96782389)(224.10771179,113.96783186)
\moveto(224.10771179,114.90533186)
\curveto(225.041301,114.90532295)(225.74637842,114.52251083)(226.22294617,113.75689436)
\curveto(226.70340871,112.99126236)(226.94364285,111.86040412)(226.94364929,110.36431623)
\curveto(226.94364285,108.87212586)(226.70340871,107.74322074)(226.22294617,106.97759748)
\curveto(225.74637842,106.21197227)(225.041301,105.82916015)(224.10771179,105.82915998)
\curveto(223.17411537,105.82916015)(222.46903795,106.21197227)(221.99247742,106.97759748)
\curveto(221.5159139,107.74322074)(221.27763289,108.87212586)(221.27763367,110.36431623)
\curveto(221.27763289,111.86040412)(221.5159139,112.99126236)(221.99247742,113.75689436)
\curveto(222.46903795,114.52251083)(223.17411537,114.90532295)(224.10771179,114.90533186)
}
}
{
\newrgbcolor{curcolor}{0 0 0}
\pscustom[linestyle=none,fillstyle=solid,fillcolor=curcolor]
{
\newpath
\moveto(255.52357483,110.3866551)
\curveto(255.523572,110.60149424)(255.59779067,110.78704093)(255.74623108,110.94329572)
\curveto(255.89857162,111.09954062)(256.08021207,111.17766554)(256.29115295,111.17767072)
\curveto(256.50989914,111.17766554)(256.69739895,111.09954062)(256.85365295,110.94329572)
\curveto(257.00989864,110.78704093)(257.08802356,110.60149424)(257.08802795,110.3866551)
\curveto(257.08802356,110.16790092)(257.00989864,109.98235423)(256.85365295,109.83001447)
\curveto(256.7013052,109.67766704)(256.51380538,109.60149524)(256.29115295,109.60149885)
\curveto(256.07239958,109.60149524)(255.88880601,109.67571391)(255.7403717,109.8241551)
\curveto(255.59583755,109.97258862)(255.523572,110.16008843)(255.52357483,110.3866551)
\moveto(256.3028717,113.96087385)
\curveto(255.7520874,113.96086588)(255.33997843,113.66399118)(255.06654358,113.07024885)
\curveto(254.79701023,112.47649236)(254.66224474,111.57219639)(254.6622467,110.35735822)
\curveto(254.66224474,109.14641757)(254.79701023,108.24407472)(255.06654358,107.65032697)
\curveto(255.33997843,107.05657591)(255.7520874,106.75970121)(256.3028717,106.75970197)
\curveto(256.85755504,106.75970121)(257.269664,107.05657591)(257.53919983,107.65032697)
\curveto(257.81263221,108.24407472)(257.94935082,109.14641757)(257.94935608,110.35735822)
\curveto(257.94935082,111.57219639)(257.81263221,112.47649236)(257.53919983,113.07024885)
\curveto(257.269664,113.66399118)(256.85755504,113.96086588)(256.3028717,113.96087385)
\moveto(256.3028717,114.89837385)
\curveto(257.23646091,114.89836494)(257.94153833,114.51555282)(258.41810608,113.74993635)
\curveto(258.89856862,112.98430436)(259.13880276,111.85344611)(259.1388092,110.35735822)
\curveto(259.13880276,108.86516785)(258.89856862,107.73626273)(258.41810608,106.97063947)
\curveto(257.94153833,106.20501426)(257.23646091,105.82220214)(256.3028717,105.82220197)
\curveto(255.36927528,105.82220214)(254.66419786,106.20501426)(254.18763733,106.97063947)
\curveto(253.71107381,107.73626273)(253.4727928,108.86516785)(253.47279358,110.35735822)
\curveto(253.4727928,111.85344611)(253.71107381,112.98430436)(254.18763733,113.74993635)
\curveto(254.66419786,114.51555282)(255.36927528,114.89836494)(256.3028717,114.89837385)
}
}
{
\newrgbcolor{curcolor}{0 0 0}
\pscustom[linestyle=none,fillstyle=solid,fillcolor=curcolor]
{
\newpath
\moveto(287.24414062,110.52020002)
\curveto(287.24413779,110.73503916)(287.31835647,110.92058585)(287.46679688,111.07684064)
\curveto(287.61913742,111.23308554)(287.80077786,111.31121046)(288.01171875,111.31121564)
\curveto(288.23046493,111.31121046)(288.41796475,111.23308554)(288.57421875,111.07684064)
\curveto(288.73046443,110.92058585)(288.80858936,110.73503916)(288.80859375,110.52020002)
\curveto(288.80858936,110.30144584)(288.73046443,110.11589915)(288.57421875,109.96355939)
\curveto(288.42187099,109.81121196)(288.23437118,109.73504016)(288.01171875,109.73504377)
\curveto(287.79296537,109.73504016)(287.6093718,109.80925884)(287.4609375,109.95770002)
\curveto(287.31640335,110.10613354)(287.24413779,110.29363335)(287.24414062,110.52020002)
\moveto(288.0234375,114.09441877)
\curveto(287.47265319,114.0944108)(287.06054423,113.7975361)(286.78710938,113.20379377)
\curveto(286.51757602,112.61003729)(286.38281053,111.70574131)(286.3828125,110.49090314)
\curveto(286.38281053,109.27996249)(286.51757602,108.37761964)(286.78710938,107.78387189)
\curveto(287.06054423,107.19012083)(287.47265319,106.89324613)(288.0234375,106.89324689)
\curveto(288.57812084,106.89324613)(288.9902298,107.19012083)(289.25976562,107.78387189)
\curveto(289.53319801,108.37761964)(289.66991662,109.27996249)(289.66992188,110.49090314)
\curveto(289.66991662,111.70574131)(289.53319801,112.61003729)(289.25976562,113.20379377)
\curveto(288.9902298,113.7975361)(288.57812084,114.0944108)(288.0234375,114.09441877)
\moveto(288.0234375,115.03191877)
\curveto(288.95702671,115.03190986)(289.66210413,114.64909775)(290.13867188,113.88348127)
\curveto(290.61913442,113.11784928)(290.85936855,111.98699103)(290.859375,110.49090314)
\curveto(290.85936855,108.99871277)(290.61913442,107.86980765)(290.13867188,107.10418439)
\curveto(289.66210413,106.33855918)(288.95702671,105.95574706)(288.0234375,105.95574689)
\curveto(287.08984107,105.95574706)(286.38476365,106.33855918)(285.90820312,107.10418439)
\curveto(285.43163961,107.86980765)(285.1933586,108.99871277)(285.19335938,110.49090314)
\curveto(285.1933586,111.98699103)(285.43163961,113.11784928)(285.90820312,113.88348127)
\curveto(286.38476365,114.64909775)(287.08984107,115.03190986)(288.0234375,115.03191877)
}
}
{
\newrgbcolor{curcolor}{0 0 0}
\pscustom[linestyle=none,fillstyle=solid,fillcolor=curcolor]
{
\newpath
\moveto(319.88812256,106.65301252)
\lineto(321.72796631,106.65301252)
\lineto(321.72796631,113.33855939)
\lineto(319.74749756,112.89324689)
\lineto(319.74749756,113.97137189)
\lineto(321.71624756,114.40496564)
\lineto(322.89984131,114.40496564)
\lineto(322.89984131,106.65301252)
\lineto(324.71624756,106.65301252)
\lineto(324.71624756,105.65691877)
\lineto(319.88812256,105.65691877)
\lineto(319.88812256,106.65301252)
}
}
{
\newrgbcolor{curcolor}{0 0 0}
\pscustom[linestyle=none,fillstyle=solid,fillcolor=curcolor]
{
\newpath
\moveto(222.11552429,90.10162092)
\lineto(223.95536804,90.10162092)
\lineto(223.95536804,96.78716779)
\lineto(221.97489929,96.34185529)
\lineto(221.97489929,97.41998029)
\lineto(223.94364929,97.85357404)
\lineto(225.12724304,97.85357404)
\lineto(225.12724304,90.10162092)
\lineto(226.94364929,90.10162092)
\lineto(226.94364929,89.10552717)
\lineto(222.11552429,89.10552717)
\lineto(222.11552429,90.10162092)
}
}
{
\newrgbcolor{curcolor}{0 0 0}
\pscustom[linestyle=none,fillstyle=solid,fillcolor=curcolor]
{
\newpath
\moveto(255.52357483,93.36236311)
\curveto(255.523572,93.57720225)(255.59779067,93.76274894)(255.74623108,93.91900373)
\curveto(255.89857162,94.07524862)(256.08021207,94.15337354)(256.29115295,94.15337873)
\curveto(256.50989914,94.15337354)(256.69739895,94.07524862)(256.85365295,93.91900373)
\curveto(257.00989864,93.76274894)(257.08802356,93.57720225)(257.08802795,93.36236311)
\curveto(257.08802356,93.14360893)(257.00989864,92.95806224)(256.85365295,92.80572248)
\curveto(256.7013052,92.65337504)(256.51380538,92.57720325)(256.29115295,92.57720686)
\curveto(256.07239958,92.57720325)(255.88880601,92.65142192)(255.7403717,92.79986311)
\curveto(255.59583755,92.94829663)(255.523572,93.13579644)(255.52357483,93.36236311)
\moveto(256.3028717,96.93658186)
\curveto(255.7520874,96.93657389)(255.33997843,96.63969918)(255.06654358,96.04595686)
\curveto(254.79701023,95.45220037)(254.66224474,94.5479044)(254.6622467,93.33306623)
\curveto(254.66224474,92.12212558)(254.79701023,91.21978273)(255.06654358,90.62603498)
\curveto(255.33997843,90.03228392)(255.7520874,89.73540921)(256.3028717,89.73540998)
\curveto(256.85755504,89.73540921)(257.269664,90.03228392)(257.53919983,90.62603498)
\curveto(257.81263221,91.21978273)(257.94935082,92.12212558)(257.94935608,93.33306623)
\curveto(257.94935082,94.5479044)(257.81263221,95.45220037)(257.53919983,96.04595686)
\curveto(257.269664,96.63969918)(256.85755504,96.93657389)(256.3028717,96.93658186)
\moveto(256.3028717,97.87408186)
\curveto(257.23646091,97.87407295)(257.94153833,97.49126083)(258.41810608,96.72564436)
\curveto(258.89856862,95.96001236)(259.13880276,94.82915412)(259.1388092,93.33306623)
\curveto(259.13880276,91.84087586)(258.89856862,90.71197074)(258.41810608,89.94634748)
\curveto(257.94153833,89.18072227)(257.23646091,88.79791015)(256.3028717,88.79790998)
\curveto(255.36927528,88.79791015)(254.66419786,89.18072227)(254.18763733,89.94634748)
\curveto(253.71107381,90.71197074)(253.4727928,91.84087586)(253.47279358,93.33306623)
\curveto(253.4727928,94.82915412)(253.71107381,95.96001236)(254.18763733,96.72564436)
\curveto(254.66419786,97.49126083)(255.36927528,97.87407295)(256.3028717,97.87408186)
}
}
{
\newrgbcolor{curcolor}{0 0 0}
\pscustom[linestyle=none,fillstyle=solid,fillcolor=curcolor]
{
\newpath
\moveto(287.24414062,93.50201154)
\curveto(287.24413779,93.71685068)(287.31835647,93.90239737)(287.46679688,94.05865217)
\curveto(287.61913742,94.21489706)(287.80077786,94.29302198)(288.01171875,94.29302717)
\curveto(288.23046493,94.29302198)(288.41796475,94.21489706)(288.57421875,94.05865217)
\curveto(288.73046443,93.90239737)(288.80858936,93.71685068)(288.80859375,93.50201154)
\curveto(288.80858936,93.28325737)(288.73046443,93.09771068)(288.57421875,92.94537092)
\curveto(288.42187099,92.79302348)(288.23437118,92.71685168)(288.01171875,92.71685529)
\curveto(287.79296537,92.71685168)(287.6093718,92.79107036)(287.4609375,92.93951154)
\curveto(287.31640335,93.08794506)(287.24413779,93.27544488)(287.24414062,93.50201154)
\moveto(288.0234375,97.07623029)
\curveto(287.47265319,97.07622232)(287.06054423,96.77934762)(286.78710938,96.18560529)
\curveto(286.51757602,95.59184881)(286.38281053,94.68755284)(286.3828125,93.47271467)
\curveto(286.38281053,92.26177401)(286.51757602,91.35943117)(286.78710938,90.76568342)
\curveto(287.06054423,90.17193235)(287.47265319,89.87505765)(288.0234375,89.87505842)
\curveto(288.57812084,89.87505765)(288.9902298,90.17193235)(289.25976562,90.76568342)
\curveto(289.53319801,91.35943117)(289.66991662,92.26177401)(289.66992188,93.47271467)
\curveto(289.66991662,94.68755284)(289.53319801,95.59184881)(289.25976562,96.18560529)
\curveto(288.9902298,96.77934762)(288.57812084,97.07622232)(288.0234375,97.07623029)
\moveto(288.0234375,98.01373029)
\curveto(288.95702671,98.01372139)(289.66210413,97.63090927)(290.13867188,96.86529279)
\curveto(290.61913442,96.0996608)(290.85936855,94.96880256)(290.859375,93.47271467)
\curveto(290.85936855,91.98052429)(290.61913442,90.85161917)(290.13867188,90.08599592)
\curveto(289.66210413,89.32037071)(288.95702671,88.93755859)(288.0234375,88.93755842)
\curveto(287.08984107,88.93755859)(286.38476365,89.32037071)(285.90820312,90.08599592)
\curveto(285.43163961,90.85161917)(285.1933586,91.98052429)(285.19335938,93.47271467)
\curveto(285.1933586,94.96880256)(285.43163961,96.0996608)(285.90820312,96.86529279)
\curveto(286.38476365,97.63090927)(287.08984107,98.01372139)(288.0234375,98.01373029)
}
}
{
\newrgbcolor{curcolor}{0 0 0}
\pscustom[linestyle=none,fillstyle=solid,fillcolor=curcolor]
{
\newpath
\moveto(319.88812256,89.76568342)
\lineto(321.72796631,89.76568342)
\lineto(321.72796631,96.45123029)
\lineto(319.74749756,96.00591779)
\lineto(319.74749756,97.08404279)
\lineto(321.71624756,97.51763654)
\lineto(322.89984131,97.51763654)
\lineto(322.89984131,89.76568342)
\lineto(324.71624756,89.76568342)
\lineto(324.71624756,88.76958967)
\lineto(319.88812256,88.76958967)
\lineto(319.88812256,89.76568342)
}
}
{
\newrgbcolor{curcolor}{0 0 0}
\pscustom[linestyle=none,fillstyle=solid,fillcolor=curcolor]
{
\newpath
\moveto(223.32841492,76.61822248)
\curveto(223.32841209,76.83306162)(223.40263076,77.01860831)(223.55107117,77.17486311)
\curveto(223.70341171,77.331108)(223.88505216,77.40923292)(224.09599304,77.40923811)
\curveto(224.31473923,77.40923292)(224.50223904,77.331108)(224.65849304,77.17486311)
\curveto(224.81473873,77.01860831)(224.89286365,76.83306162)(224.89286804,76.61822248)
\curveto(224.89286365,76.3994683)(224.81473873,76.21392162)(224.65849304,76.06158186)
\curveto(224.50614528,75.90923442)(224.31864547,75.83306262)(224.09599304,75.83306623)
\curveto(223.87723966,75.83306262)(223.6936461,75.9072813)(223.54521179,76.05572248)
\curveto(223.40067764,76.204156)(223.32841209,76.39165581)(223.32841492,76.61822248)
\moveto(224.10771179,80.19244123)
\curveto(223.55692748,80.19243326)(223.14481852,79.89555856)(222.87138367,79.30181623)
\curveto(222.60185031,78.70805975)(222.46708482,77.80376378)(222.46708679,76.58892561)
\curveto(222.46708482,75.37798495)(222.60185031,74.4756421)(222.87138367,73.88189436)
\curveto(223.14481852,73.28814329)(223.55692748,72.99126859)(224.10771179,72.99126936)
\curveto(224.66239513,72.99126859)(225.07450409,73.28814329)(225.34403992,73.88189436)
\curveto(225.6174723,74.4756421)(225.75419091,75.37798495)(225.75419617,76.58892561)
\curveto(225.75419091,77.80376378)(225.6174723,78.70805975)(225.34403992,79.30181623)
\curveto(225.07450409,79.89555856)(224.66239513,80.19243326)(224.10771179,80.19244123)
\moveto(224.10771179,81.12994123)
\curveto(225.041301,81.12993232)(225.74637842,80.74712021)(226.22294617,79.98150373)
\curveto(226.70340871,79.21587174)(226.94364285,78.08501349)(226.94364929,76.58892561)
\curveto(226.94364285,75.09673523)(226.70340871,73.96783011)(226.22294617,73.20220686)
\curveto(225.74637842,72.43658164)(225.041301,72.05376953)(224.10771179,72.05376936)
\curveto(223.17411537,72.05376953)(222.46903795,72.43658164)(221.99247742,73.20220686)
\curveto(221.5159139,73.96783011)(221.27763289,75.09673523)(221.27763367,76.58892561)
\curveto(221.27763289,78.08501349)(221.5159139,79.21587174)(221.99247742,79.98150373)
\curveto(222.46903795,80.74712021)(223.17411537,81.12993232)(224.10771179,81.12994123)
}
}
{
\newrgbcolor{curcolor}{0 0 0}
\pscustom[linestyle=none,fillstyle=solid,fillcolor=curcolor]
{
\newpath
\moveto(255.11927795,72.93975568)
\lineto(259.1388092,72.93975568)
\lineto(259.1388092,71.94366193)
\lineto(253.82435608,71.94366193)
\lineto(253.82435608,72.93975568)
\curveto(254.55482321,73.70928517)(255.19349444,74.38897199)(255.7403717,74.97881818)
\curveto(256.28724335,75.56865831)(256.6641961,75.98467352)(256.87123108,76.22686506)
\curveto(257.26185175,76.7034228)(257.52552336,77.08818804)(257.6622467,77.38116193)
\curveto(257.79896059,77.6780312)(257.8673199,77.98076527)(257.86732483,78.28936506)
\curveto(257.8673199,78.77763947)(257.72278879,79.16045159)(257.43373108,79.43780256)
\curveto(257.14857061,79.71513854)(256.75599288,79.85381027)(256.2559967,79.85381818)
\curveto(255.90052499,79.85381027)(255.52747849,79.78935721)(255.13685608,79.66045881)
\curveto(254.74622927,79.53154497)(254.33216718,79.33623267)(253.89466858,79.07452131)
\lineto(253.89466858,80.26983381)
\curveto(254.29701097,80.46123154)(254.69154182,80.60576265)(255.07826233,80.70342756)
\curveto(255.46888479,80.80107495)(255.85365003,80.84990303)(256.2325592,80.84991193)
\curveto(257.0880238,80.84990303)(257.77552311,80.62138763)(258.2950592,80.16436506)
\curveto(258.81849082,79.71123229)(259.08020931,79.11552976)(259.08021545,78.37725568)
\curveto(259.08020931,78.00224963)(258.99231877,77.62725)(258.81654358,77.25225568)
\curveto(258.64466287,76.87725075)(258.36341315,76.46318866)(257.97279358,76.01006818)
\curveto(257.75403876,75.75615812)(257.4356797,75.40459597)(257.01771545,74.95538068)
\curveto(256.60364928,74.50615937)(255.97083742,73.83428504)(255.11927795,72.93975568)
}
}
{
\newrgbcolor{curcolor}{0 0 0}
\pscustom[linestyle=none,fillstyle=solid,fillcolor=curcolor]
{
\newpath
\moveto(287.24414062,76.48382307)
\curveto(287.24413779,76.69866221)(287.31835647,76.8842089)(287.46679688,77.04046369)
\curveto(287.61913742,77.19670858)(287.80077786,77.27483351)(288.01171875,77.27483869)
\curveto(288.23046493,77.27483351)(288.41796475,77.19670858)(288.57421875,77.04046369)
\curveto(288.73046443,76.8842089)(288.80858936,76.69866221)(288.80859375,76.48382307)
\curveto(288.80858936,76.26506889)(288.73046443,76.0795222)(288.57421875,75.92718244)
\curveto(288.42187099,75.77483501)(288.23437118,75.69866321)(288.01171875,75.69866682)
\curveto(287.79296537,75.69866321)(287.6093718,75.77288188)(287.4609375,75.92132307)
\curveto(287.31640335,76.06975659)(287.24413779,76.2572564)(287.24414062,76.48382307)
\moveto(288.0234375,80.05804182)
\curveto(287.47265319,80.05803385)(287.06054423,79.76115914)(286.78710938,79.16741682)
\curveto(286.51757602,78.57366033)(286.38281053,77.66936436)(286.3828125,76.45452619)
\curveto(286.38281053,75.24358554)(286.51757602,74.34124269)(286.78710938,73.74749494)
\curveto(287.06054423,73.15374388)(287.47265319,72.85686917)(288.0234375,72.85686994)
\curveto(288.57812084,72.85686917)(288.9902298,73.15374388)(289.25976562,73.74749494)
\curveto(289.53319801,74.34124269)(289.66991662,75.24358554)(289.66992188,76.45452619)
\curveto(289.66991662,77.66936436)(289.53319801,78.57366033)(289.25976562,79.16741682)
\curveto(288.9902298,79.76115914)(288.57812084,80.05803385)(288.0234375,80.05804182)
\moveto(288.0234375,80.99554182)
\curveto(288.95702671,80.99553291)(289.66210413,80.61272079)(290.13867188,79.84710432)
\curveto(290.61913442,79.08147232)(290.85936855,77.95061408)(290.859375,76.45452619)
\curveto(290.85936855,74.96233582)(290.61913442,73.8334307)(290.13867188,73.06780744)
\curveto(289.66210413,72.30218223)(288.95702671,71.91937011)(288.0234375,71.91936994)
\curveto(287.08984107,71.91937011)(286.38476365,72.30218223)(285.90820312,73.06780744)
\curveto(285.43163961,73.8334307)(285.1933586,74.96233582)(285.19335938,76.45452619)
\curveto(285.1933586,77.95061408)(285.43163961,79.08147232)(285.90820312,79.84710432)
\curveto(286.38476365,80.61272079)(287.08984107,80.99553291)(288.0234375,80.99554182)
}
}
{
\newrgbcolor{curcolor}{0 0 0}
\pscustom[linestyle=none,fillstyle=solid,fillcolor=curcolor]
{
\newpath
\moveto(320.69671631,72.72015119)
\lineto(324.71624756,72.72015119)
\lineto(324.71624756,71.72405744)
\lineto(319.40179443,71.72405744)
\lineto(319.40179443,72.72015119)
\curveto(320.13226156,73.48968068)(320.7709328,74.1693675)(321.31781006,74.75921369)
\curveto(321.86468171,75.34905382)(322.24163445,75.76506903)(322.44866943,76.00726057)
\curveto(322.83929011,76.48381831)(323.10296172,76.86858355)(323.23968506,77.16155744)
\curveto(323.37639894,77.45842671)(323.44475825,77.76116078)(323.44476318,78.06976057)
\curveto(323.44475825,78.55803498)(323.30022714,78.9408471)(323.01116943,79.21819807)
\curveto(322.72600897,79.49553404)(322.33343124,79.63420578)(321.83343506,79.63421369)
\curveto(321.47796334,79.63420578)(321.10491684,79.56975272)(320.71429443,79.44085432)
\curveto(320.32366762,79.31194048)(319.90960554,79.11662817)(319.47210693,78.85491682)
\lineto(319.47210693,80.05022932)
\curveto(319.87444932,80.24162705)(320.26898018,80.38615815)(320.65570068,80.48382307)
\curveto(321.04632315,80.58147046)(321.43108839,80.63029854)(321.80999756,80.63030744)
\curveto(322.66546215,80.63029854)(323.35296147,80.40178314)(323.87249756,79.94476057)
\curveto(324.39592917,79.4916278)(324.65764766,78.89592527)(324.65765381,78.15765119)
\curveto(324.65764766,77.78264513)(324.56975713,77.40764551)(324.39398193,77.03265119)
\curveto(324.22210122,76.65764626)(323.9408515,76.24358417)(323.55023193,75.79046369)
\curveto(323.33147711,75.53655363)(323.01311806,75.18499148)(322.59515381,74.73577619)
\curveto(322.18108764,74.28655488)(321.54827577,73.61468055)(320.69671631,72.72015119)
}
}
{
\newrgbcolor{curcolor}{0 0 0}
\pscustom[linestyle=none,fillstyle=solid,fillcolor=curcolor]
{
\newpath
\moveto(223.32841492,59.56646467)
\curveto(223.32841209,59.78130381)(223.40263076,59.9668505)(223.55107117,60.12310529)
\curveto(223.70341171,60.27935019)(223.88505216,60.35747511)(224.09599304,60.35748029)
\curveto(224.31473923,60.35747511)(224.50223904,60.27935019)(224.65849304,60.12310529)
\curveto(224.81473873,59.9668505)(224.89286365,59.78130381)(224.89286804,59.56646467)
\curveto(224.89286365,59.34771049)(224.81473873,59.1621638)(224.65849304,59.00982404)
\curveto(224.50614528,58.85747661)(224.31864547,58.78130481)(224.09599304,58.78130842)
\curveto(223.87723966,58.78130481)(223.6936461,58.85552348)(223.54521179,59.00396467)
\curveto(223.40067764,59.15239819)(223.32841209,59.339898)(223.32841492,59.56646467)
\moveto(224.10771179,63.14068342)
\curveto(223.55692748,63.14067545)(223.14481852,62.84380075)(222.87138367,62.25005842)
\curveto(222.60185031,61.65630193)(222.46708482,60.75200596)(222.46708679,59.53716779)
\curveto(222.46708482,58.32622714)(222.60185031,57.42388429)(222.87138367,56.83013654)
\curveto(223.14481852,56.23638548)(223.55692748,55.93951078)(224.10771179,55.93951154)
\curveto(224.66239513,55.93951078)(225.07450409,56.23638548)(225.34403992,56.83013654)
\curveto(225.6174723,57.42388429)(225.75419091,58.32622714)(225.75419617,59.53716779)
\curveto(225.75419091,60.75200596)(225.6174723,61.65630193)(225.34403992,62.25005842)
\curveto(225.07450409,62.84380075)(224.66239513,63.14067545)(224.10771179,63.14068342)
\moveto(224.10771179,64.07818342)
\curveto(225.041301,64.07817451)(225.74637842,63.69536239)(226.22294617,62.92974592)
\curveto(226.70340871,62.16411393)(226.94364285,61.03325568)(226.94364929,59.53716779)
\curveto(226.94364285,58.04497742)(226.70340871,56.9160723)(226.22294617,56.15044904)
\curveto(225.74637842,55.38482383)(225.041301,55.00201171)(224.10771179,55.00201154)
\curveto(223.17411537,55.00201171)(222.46903795,55.38482383)(221.99247742,56.15044904)
\curveto(221.5159139,56.9160723)(221.27763289,58.04497742)(221.27763367,59.53716779)
\curveto(221.27763289,61.03325568)(221.5159139,62.16411393)(221.99247742,62.92974592)
\curveto(222.46903795,63.69536239)(223.17411537,64.07817451)(224.10771179,64.07818342)
}
}
{
\newrgbcolor{curcolor}{0 0 0}
\pscustom[linestyle=none,fillstyle=solid,fillcolor=curcolor]
{
\newpath
\moveto(255.52357483,59.48382307)
\curveto(255.523572,59.69866221)(255.59779067,59.8842089)(255.74623108,60.04046369)
\curveto(255.89857162,60.19670858)(256.08021207,60.27483351)(256.29115295,60.27483869)
\curveto(256.50989914,60.27483351)(256.69739895,60.19670858)(256.85365295,60.04046369)
\curveto(257.00989864,59.8842089)(257.08802356,59.69866221)(257.08802795,59.48382307)
\curveto(257.08802356,59.26506889)(257.00989864,59.0795222)(256.85365295,58.92718244)
\curveto(256.7013052,58.77483501)(256.51380538,58.69866321)(256.29115295,58.69866682)
\curveto(256.07239958,58.69866321)(255.88880601,58.77288188)(255.7403717,58.92132307)
\curveto(255.59583755,59.06975659)(255.523572,59.2572564)(255.52357483,59.48382307)
\moveto(256.3028717,63.05804182)
\curveto(255.7520874,63.05803385)(255.33997843,62.76115914)(255.06654358,62.16741682)
\curveto(254.79701023,61.57366033)(254.66224474,60.66936436)(254.6622467,59.45452619)
\curveto(254.66224474,58.24358554)(254.79701023,57.34124269)(255.06654358,56.74749494)
\curveto(255.33997843,56.15374388)(255.7520874,55.85686917)(256.3028717,55.85686994)
\curveto(256.85755504,55.85686917)(257.269664,56.15374388)(257.53919983,56.74749494)
\curveto(257.81263221,57.34124269)(257.94935082,58.24358554)(257.94935608,59.45452619)
\curveto(257.94935082,60.66936436)(257.81263221,61.57366033)(257.53919983,62.16741682)
\curveto(257.269664,62.76115914)(256.85755504,63.05803385)(256.3028717,63.05804182)
\moveto(256.3028717,63.99554182)
\curveto(257.23646091,63.99553291)(257.94153833,63.61272079)(258.41810608,62.84710432)
\curveto(258.89856862,62.08147232)(259.13880276,60.95061408)(259.1388092,59.45452619)
\curveto(259.13880276,57.96233582)(258.89856862,56.8334307)(258.41810608,56.06780744)
\curveto(257.94153833,55.30218223)(257.23646091,54.91937011)(256.3028717,54.91936994)
\curveto(255.36927528,54.91937011)(254.66419786,55.30218223)(254.18763733,56.06780744)
\curveto(253.71107381,56.8334307)(253.4727928,57.96233582)(253.47279358,59.45452619)
\curveto(253.4727928,60.95061408)(253.71107381,62.08147232)(254.18763733,62.84710432)
\curveto(254.66419786,63.61272079)(255.36927528,63.99553291)(256.3028717,63.99554182)
}
}
{
\newrgbcolor{curcolor}{0 0 0}
\pscustom[linestyle=none,fillstyle=solid,fillcolor=curcolor]
{
\newpath
\moveto(287.24414062,59.46563459)
\curveto(287.24413779,59.68047373)(287.31835647,59.86602042)(287.46679688,60.02227521)
\curveto(287.61913742,60.17852011)(287.80077786,60.25664503)(288.01171875,60.25665021)
\curveto(288.23046493,60.25664503)(288.41796475,60.17852011)(288.57421875,60.02227521)
\curveto(288.73046443,59.86602042)(288.80858936,59.68047373)(288.80859375,59.46563459)
\curveto(288.80858936,59.24688041)(288.73046443,59.06133372)(288.57421875,58.90899396)
\curveto(288.42187099,58.75664653)(288.23437118,58.68047473)(288.01171875,58.68047834)
\curveto(287.79296537,58.68047473)(287.6093718,58.75469341)(287.4609375,58.90313459)
\curveto(287.31640335,59.05156811)(287.24413779,59.23906792)(287.24414062,59.46563459)
\moveto(288.0234375,63.03985334)
\curveto(287.47265319,63.03984537)(287.06054423,62.74297067)(286.78710938,62.14922834)
\curveto(286.51757602,61.55547186)(286.38281053,60.65117588)(286.3828125,59.43633771)
\curveto(286.38281053,58.22539706)(286.51757602,57.32305421)(286.78710938,56.72930646)
\curveto(287.06054423,56.1355554)(287.47265319,55.8386807)(288.0234375,55.83868146)
\curveto(288.57812084,55.8386807)(288.9902298,56.1355554)(289.25976562,56.72930646)
\curveto(289.53319801,57.32305421)(289.66991662,58.22539706)(289.66992188,59.43633771)
\curveto(289.66991662,60.65117588)(289.53319801,61.55547186)(289.25976562,62.14922834)
\curveto(288.9902298,62.74297067)(288.57812084,63.03984537)(288.0234375,63.03985334)
\moveto(288.0234375,63.97735334)
\curveto(288.95702671,63.97734443)(289.66210413,63.59453232)(290.13867188,62.82891584)
\curveto(290.61913442,62.06328385)(290.85936855,60.9324256)(290.859375,59.43633771)
\curveto(290.85936855,57.94414734)(290.61913442,56.81524222)(290.13867188,56.04961896)
\curveto(289.66210413,55.28399375)(288.95702671,54.90118163)(288.0234375,54.90118146)
\curveto(287.08984107,54.90118163)(286.38476365,55.28399375)(285.90820312,56.04961896)
\curveto(285.43163961,56.81524222)(285.1933586,57.94414734)(285.19335938,59.43633771)
\curveto(285.1933586,60.9324256)(285.43163961,62.06328385)(285.90820312,62.82891584)
\curveto(286.38476365,63.59453232)(287.08984107,63.97734443)(288.0234375,63.97735334)
}
}
{
\newrgbcolor{curcolor}{0 0 0}
\pscustom[linestyle=none,fillstyle=solid,fillcolor=curcolor]
{
\newpath
\moveto(312.62249756,55.67461896)
\lineto(314.46234131,55.67461896)
\lineto(314.46234131,62.36016584)
\lineto(312.48187256,61.91485334)
\lineto(312.48187256,62.99297834)
\lineto(314.45062256,63.42657209)
\lineto(315.63421631,63.42657209)
\lineto(315.63421631,55.67461896)
\lineto(317.45062256,55.67461896)
\lineto(317.45062256,54.67852521)
\lineto(312.62249756,54.67852521)
\lineto(312.62249756,55.67461896)
}
}
{
\newrgbcolor{curcolor}{0 0 0}
\pscustom[linestyle=none,fillstyle=solid,fillcolor=curcolor]
{
\newpath
\moveto(321.86859131,58.83282209)
\curveto(321.34124448,58.83281794)(320.93304176,58.68438058)(320.64398193,58.38750959)
\curveto(320.35882358,58.09453742)(320.2162456,57.67852221)(320.21624756,57.13946271)
\curveto(320.2162456,56.60039829)(320.36077671,56.18047684)(320.64984131,55.87969709)
\curveto(320.94280737,55.58282119)(321.34905697,55.43438383)(321.86859131,55.43438459)
\curveto(322.39983717,55.43438383)(322.80803988,55.58086806)(323.09320068,55.87383771)
\curveto(323.38225806,56.17071122)(323.52678917,56.5925858)(323.52679443,57.13946271)
\curveto(323.52678917,57.67461597)(323.38030494,58.09063118)(323.08734131,58.38750959)
\curveto(322.79827427,58.68438058)(322.39202468,58.83281794)(321.86859131,58.83282209)
\moveto(320.83734131,59.32500959)
\curveto(320.33343298,59.45391106)(319.93890213,59.6941452)(319.65374756,60.04571271)
\curveto(319.37249645,60.3972695)(319.23187159,60.8210972)(319.23187256,61.31719709)
\curveto(319.23187159,62.01250226)(319.46819947,62.56328296)(319.94085693,62.96954084)
\curveto(320.41351103,63.37968839)(321.05608851,63.58476631)(321.86859131,63.58477521)
\curveto(322.68499313,63.58476631)(323.32952374,63.37968839)(323.80218506,62.96954084)
\curveto(324.27483529,62.56328296)(324.51116318,62.01250226)(324.51116943,61.31719709)
\curveto(324.51116318,60.8210972)(324.3685852,60.3972695)(324.08343506,60.04571271)
\curveto(323.80217952,59.6941452)(323.40960178,59.45391106)(322.90570068,59.32500959)
\curveto(323.49163295,59.19609882)(323.93889813,58.93633346)(324.24749756,58.54571271)
\curveto(324.55999126,58.15508424)(324.7162411,57.64922537)(324.71624756,57.02813459)
\curveto(324.7162411,56.23907053)(324.46428823,55.62188365)(323.96038818,55.17657209)
\curveto(323.45647674,54.73125954)(322.75921181,54.50860351)(321.86859131,54.50860334)
\curveto(320.97796359,54.50860351)(320.28069866,54.72930641)(319.77679443,55.17071271)
\curveto(319.27679342,55.61602428)(319.02679367,56.23125804)(319.02679443,57.01641584)
\curveto(319.02679367,57.64141288)(319.18109039,58.14922487)(319.48968506,58.53985334)
\curveto(319.80218352,58.93438033)(320.25140182,59.19609882)(320.83734131,59.32500959)
\moveto(320.40960693,61.20586896)
\curveto(320.40960478,60.73711291)(320.53460466,60.37969139)(320.78460693,60.13360334)
\curveto(321.03460416,59.88750438)(321.39593192,59.76445763)(321.86859131,59.76446271)
\curveto(322.34514972,59.76445763)(322.70843061,59.88750438)(322.95843506,60.13360334)
\curveto(323.20843011,60.37969139)(323.33342998,60.73711291)(323.33343506,61.20586896)
\curveto(323.33342998,61.68242446)(323.20843011,62.04570535)(322.95843506,62.29571271)
\curveto(322.71233686,62.54570485)(322.34905597,62.67070472)(321.86859131,62.67071271)
\curveto(321.39593192,62.67070472)(321.03460416,62.54375172)(320.78460693,62.28985334)
\curveto(320.53460466,62.03984598)(320.40960478,61.67851821)(320.40960693,61.20586896)
}
}
{
\newrgbcolor{curcolor}{0 0 0}
\pscustom[linestyle=none,fillstyle=solid,fillcolor=curcolor]
{
\newpath
\moveto(223.32841492,42.51470686)
\curveto(223.32841209,42.729546)(223.40263076,42.91509269)(223.55107117,43.07134748)
\curveto(223.70341171,43.22759237)(223.88505216,43.30571729)(224.09599304,43.30572248)
\curveto(224.31473923,43.30571729)(224.50223904,43.22759237)(224.65849304,43.07134748)
\curveto(224.81473873,42.91509269)(224.89286365,42.729546)(224.89286804,42.51470686)
\curveto(224.89286365,42.29595268)(224.81473873,42.11040599)(224.65849304,41.95806623)
\curveto(224.50614528,41.80571879)(224.31864547,41.729547)(224.09599304,41.72955061)
\curveto(223.87723966,41.729547)(223.6936461,41.80376567)(223.54521179,41.95220686)
\curveto(223.40067764,42.10064038)(223.32841209,42.28814019)(223.32841492,42.51470686)
\moveto(224.10771179,46.08892561)
\curveto(223.55692748,46.08891764)(223.14481852,45.79204293)(222.87138367,45.19830061)
\curveto(222.60185031,44.60454412)(222.46708482,43.70024815)(222.46708679,42.48540998)
\curveto(222.46708482,41.27446933)(222.60185031,40.37212648)(222.87138367,39.77837873)
\curveto(223.14481852,39.18462767)(223.55692748,38.88775296)(224.10771179,38.88775373)
\curveto(224.66239513,38.88775296)(225.07450409,39.18462767)(225.34403992,39.77837873)
\curveto(225.6174723,40.37212648)(225.75419091,41.27446933)(225.75419617,42.48540998)
\curveto(225.75419091,43.70024815)(225.6174723,44.60454412)(225.34403992,45.19830061)
\curveto(225.07450409,45.79204293)(224.66239513,46.08891764)(224.10771179,46.08892561)
\moveto(224.10771179,47.02642561)
\curveto(225.041301,47.0264167)(225.74637842,46.64360458)(226.22294617,45.87798811)
\curveto(226.70340871,45.11235611)(226.94364285,43.98149787)(226.94364929,42.48540998)
\curveto(226.94364285,40.99321961)(226.70340871,39.86431449)(226.22294617,39.09869123)
\curveto(225.74637842,38.33306602)(225.041301,37.9502539)(224.10771179,37.95025373)
\curveto(223.17411537,37.9502539)(222.46903795,38.33306602)(221.99247742,39.09869123)
\curveto(221.5159139,39.86431449)(221.27763289,40.99321961)(221.27763367,42.48540998)
\curveto(221.27763289,43.98149787)(221.5159139,45.11235611)(221.99247742,45.87798811)
\curveto(222.46903795,46.64360458)(223.17411537,47.0264167)(224.10771179,47.02642561)
}
}
{
\newrgbcolor{curcolor}{0 0 0}
\pscustom[linestyle=none,fillstyle=solid,fillcolor=curcolor]
{
\newpath
\moveto(255.52357483,42.45959211)
\curveto(255.523572,42.67443125)(255.59779067,42.85997794)(255.74623108,43.01623273)
\curveto(255.89857162,43.17247763)(256.08021207,43.25060255)(256.29115295,43.25060773)
\curveto(256.50989914,43.25060255)(256.69739895,43.17247763)(256.85365295,43.01623273)
\curveto(257.00989864,42.85997794)(257.08802356,42.67443125)(257.08802795,42.45959211)
\curveto(257.08802356,42.24083793)(257.00989864,42.05529124)(256.85365295,41.90295148)
\curveto(256.7013052,41.75060405)(256.51380538,41.67443225)(256.29115295,41.67443586)
\curveto(256.07239958,41.67443225)(255.88880601,41.74865093)(255.7403717,41.89709211)
\curveto(255.59583755,42.04552563)(255.523572,42.23302544)(255.52357483,42.45959211)
\moveto(256.3028717,46.03381086)
\curveto(255.7520874,46.03380289)(255.33997843,45.73692819)(255.06654358,45.14318586)
\curveto(254.79701023,44.54942938)(254.66224474,43.6451334)(254.6622467,42.43029523)
\curveto(254.66224474,41.21935458)(254.79701023,40.31701173)(255.06654358,39.72326398)
\curveto(255.33997843,39.12951292)(255.7520874,38.83263822)(256.3028717,38.83263898)
\curveto(256.85755504,38.83263822)(257.269664,39.12951292)(257.53919983,39.72326398)
\curveto(257.81263221,40.31701173)(257.94935082,41.21935458)(257.94935608,42.43029523)
\curveto(257.94935082,43.6451334)(257.81263221,44.54942938)(257.53919983,45.14318586)
\curveto(257.269664,45.73692819)(256.85755504,46.03380289)(256.3028717,46.03381086)
\moveto(256.3028717,46.97131086)
\curveto(257.23646091,46.97130195)(257.94153833,46.58848984)(258.41810608,45.82287336)
\curveto(258.89856862,45.05724137)(259.13880276,43.92638312)(259.1388092,42.43029523)
\curveto(259.13880276,40.93810486)(258.89856862,39.80919974)(258.41810608,39.04357648)
\curveto(257.94153833,38.27795127)(257.23646091,37.89513915)(256.3028717,37.89513898)
\curveto(255.36927528,37.89513915)(254.66419786,38.27795127)(254.18763733,39.04357648)
\curveto(253.71107381,39.80919974)(253.4727928,40.93810486)(253.47279358,42.43029523)
\curveto(253.4727928,43.92638312)(253.71107381,45.05724137)(254.18763733,45.82287336)
\curveto(254.66419786,46.58848984)(255.36927528,46.97130195)(256.3028717,46.97131086)
}
}
{
\newrgbcolor{curcolor}{0 0 0}
\pscustom[linestyle=none,fillstyle=solid,fillcolor=curcolor]
{
\newpath
\moveto(287.24414062,42.44750715)
\curveto(287.24413779,42.66234629)(287.31835647,42.84789298)(287.46679688,43.00414777)
\curveto(287.61913742,43.16039267)(287.80077786,43.23851759)(288.01171875,43.23852277)
\curveto(288.23046493,43.23851759)(288.41796475,43.16039267)(288.57421875,43.00414777)
\curveto(288.73046443,42.84789298)(288.80858936,42.66234629)(288.80859375,42.44750715)
\curveto(288.80858936,42.22875297)(288.73046443,42.04320628)(288.57421875,41.89086652)
\curveto(288.42187099,41.73851909)(288.23437118,41.66234729)(288.01171875,41.6623509)
\curveto(287.79296537,41.66234729)(287.6093718,41.73656596)(287.4609375,41.88500715)
\curveto(287.31640335,42.03344067)(287.24413779,42.22094048)(287.24414062,42.44750715)
\moveto(288.0234375,46.0217259)
\curveto(287.47265319,46.02171793)(287.06054423,45.72484323)(286.78710938,45.1311009)
\curveto(286.51757602,44.53734441)(286.38281053,43.63304844)(286.3828125,42.41821027)
\curveto(286.38281053,41.20726962)(286.51757602,40.30492677)(286.78710938,39.71117902)
\curveto(287.06054423,39.11742796)(287.47265319,38.82055326)(288.0234375,38.82055402)
\curveto(288.57812084,38.82055326)(288.9902298,39.11742796)(289.25976562,39.71117902)
\curveto(289.53319801,40.30492677)(289.66991662,41.20726962)(289.66992188,42.41821027)
\curveto(289.66991662,43.63304844)(289.53319801,44.53734441)(289.25976562,45.1311009)
\curveto(288.9902298,45.72484323)(288.57812084,46.02171793)(288.0234375,46.0217259)
\moveto(288.0234375,46.9592259)
\curveto(288.95702671,46.95921699)(289.66210413,46.57640488)(290.13867188,45.8107884)
\curveto(290.61913442,45.04515641)(290.85936855,43.91429816)(290.859375,42.41821027)
\curveto(290.85936855,40.9260199)(290.61913442,39.79711478)(290.13867188,39.03149152)
\curveto(289.66210413,38.26586631)(288.95702671,37.88305419)(288.0234375,37.88305402)
\curveto(287.08984107,37.88305419)(286.38476365,38.26586631)(285.90820312,39.03149152)
\curveto(285.43163961,39.79711478)(285.1933586,40.9260199)(285.19335938,42.41821027)
\curveto(285.1933586,43.91429816)(285.43163961,45.04515641)(285.90820312,45.8107884)
\curveto(286.38476365,46.57640488)(287.08984107,46.95921699)(288.0234375,46.9592259)
}
}
{
\newrgbcolor{curcolor}{0 0 0}
\pscustom[linestyle=none,fillstyle=solid,fillcolor=curcolor]
{
\newpath
\moveto(319.88812256,38.61749006)
\lineto(321.72796631,38.61749006)
\lineto(321.72796631,45.30303693)
\lineto(319.74749756,44.85772443)
\lineto(319.74749756,45.93584943)
\lineto(321.71624756,46.36944318)
\lineto(322.89984131,46.36944318)
\lineto(322.89984131,38.61749006)
\lineto(324.71624756,38.61749006)
\lineto(324.71624756,37.62139631)
\lineto(319.88812256,37.62139631)
\lineto(319.88812256,38.61749006)
}
}
{
\newrgbcolor{curcolor}{0 0 0}
\pscustom[linestyle=none,fillstyle=solid,fillcolor=curcolor]
{
\newpath
\moveto(223.32841492,25.46288801)
\curveto(223.32841209,25.67772715)(223.40263076,25.86327384)(223.55107117,26.01952863)
\curveto(223.70341171,26.17577353)(223.88505216,26.25389845)(224.09599304,26.25390363)
\curveto(224.31473923,26.25389845)(224.50223904,26.17577353)(224.65849304,26.01952863)
\curveto(224.81473873,25.86327384)(224.89286365,25.67772715)(224.89286804,25.46288801)
\curveto(224.89286365,25.24413383)(224.81473873,25.05858714)(224.65849304,24.90624738)
\curveto(224.50614528,24.75389995)(224.31864547,24.67772815)(224.09599304,24.67773176)
\curveto(223.87723966,24.67772815)(223.6936461,24.75194682)(223.54521179,24.90038801)
\curveto(223.40067764,25.04882153)(223.32841209,25.23632134)(223.32841492,25.46288801)
\moveto(224.10771179,29.03710676)
\curveto(223.55692748,29.03709879)(223.14481852,28.74022409)(222.87138367,28.14648176)
\curveto(222.60185031,27.55272527)(222.46708482,26.6484293)(222.46708679,25.43359113)
\curveto(222.46708482,24.22265048)(222.60185031,23.32030763)(222.87138367,22.72655988)
\curveto(223.14481852,22.13280882)(223.55692748,21.83593412)(224.10771179,21.83593488)
\curveto(224.66239513,21.83593412)(225.07450409,22.13280882)(225.34403992,22.72655988)
\curveto(225.6174723,23.32030763)(225.75419091,24.22265048)(225.75419617,25.43359113)
\curveto(225.75419091,26.6484293)(225.6174723,27.55272527)(225.34403992,28.14648176)
\curveto(225.07450409,28.74022409)(224.66239513,29.03709879)(224.10771179,29.03710676)
\moveto(224.10771179,29.97460676)
\curveto(225.041301,29.97459785)(225.74637842,29.59178573)(226.22294617,28.82616926)
\curveto(226.70340871,28.06053727)(226.94364285,26.92967902)(226.94364929,25.43359113)
\curveto(226.94364285,23.94140076)(226.70340871,22.81249564)(226.22294617,22.04687238)
\curveto(225.74637842,21.28124717)(225.041301,20.89843505)(224.10771179,20.89843488)
\curveto(223.17411537,20.89843505)(222.46903795,21.28124717)(221.99247742,22.04687238)
\curveto(221.5159139,22.81249564)(221.27763289,23.94140076)(221.27763367,25.43359113)
\curveto(221.27763289,26.92967902)(221.5159139,28.06053727)(221.99247742,28.82616926)
\curveto(222.46903795,29.59178573)(223.17411537,29.97459785)(224.10771179,29.97460676)
}
}
{
\newrgbcolor{curcolor}{0 0 0}
\pscustom[linestyle=none,fillstyle=solid,fillcolor=curcolor]
{
\newpath
\moveto(255.52357483,25.43530012)
\curveto(255.523572,25.65013926)(255.59779067,25.83568595)(255.74623108,25.99194074)
\curveto(255.89857162,26.14818563)(256.08021207,26.22631056)(256.29115295,26.22631574)
\curveto(256.50989914,26.22631056)(256.69739895,26.14818563)(256.85365295,25.99194074)
\curveto(257.00989864,25.83568595)(257.08802356,25.65013926)(257.08802795,25.43530012)
\curveto(257.08802356,25.21654594)(257.00989864,25.03099925)(256.85365295,24.87865949)
\curveto(256.7013052,24.72631206)(256.51380538,24.65014026)(256.29115295,24.65014387)
\curveto(256.07239958,24.65014026)(255.88880601,24.72435893)(255.7403717,24.87280012)
\curveto(255.59583755,25.02123364)(255.523572,25.20873345)(255.52357483,25.43530012)
\moveto(256.3028717,29.00951887)
\curveto(255.7520874,29.0095109)(255.33997843,28.7126362)(255.06654358,28.11889387)
\curveto(254.79701023,27.52513738)(254.66224474,26.62084141)(254.6622467,25.40600324)
\curveto(254.66224474,24.19506259)(254.79701023,23.29271974)(255.06654358,22.69897199)
\curveto(255.33997843,22.10522093)(255.7520874,21.80834622)(256.3028717,21.80834699)
\curveto(256.85755504,21.80834622)(257.269664,22.10522093)(257.53919983,22.69897199)
\curveto(257.81263221,23.29271974)(257.94935082,24.19506259)(257.94935608,25.40600324)
\curveto(257.94935082,26.62084141)(257.81263221,27.52513738)(257.53919983,28.11889387)
\curveto(257.269664,28.7126362)(256.85755504,29.0095109)(256.3028717,29.00951887)
\moveto(256.3028717,29.94701887)
\curveto(257.23646091,29.94700996)(257.94153833,29.56419784)(258.41810608,28.79858137)
\curveto(258.89856862,28.03294938)(259.13880276,26.90209113)(259.1388092,25.40600324)
\curveto(259.13880276,23.91381287)(258.89856862,22.78490775)(258.41810608,22.01928449)
\curveto(257.94153833,21.25365928)(257.23646091,20.87084716)(256.3028717,20.87084699)
\curveto(255.36927528,20.87084716)(254.66419786,21.25365928)(254.18763733,22.01928449)
\curveto(253.71107381,22.78490775)(253.4727928,23.91381287)(253.47279358,25.40600324)
\curveto(253.4727928,26.90209113)(253.71107381,28.03294938)(254.18763733,28.79858137)
\curveto(254.66419786,29.56419784)(255.36927528,29.94700996)(256.3028717,29.94701887)
}
}
{
\newrgbcolor{curcolor}{0 0 0}
\pscustom[linestyle=none,fillstyle=solid,fillcolor=curcolor]
{
\newpath
\moveto(287.24414062,25.42931867)
\curveto(287.24413779,25.64415781)(287.31835647,25.8297045)(287.46679688,25.9859593)
\curveto(287.61913742,26.14220419)(287.80077786,26.22032911)(288.01171875,26.2203343)
\curveto(288.23046493,26.22032911)(288.41796475,26.14220419)(288.57421875,25.9859593)
\curveto(288.73046443,25.8297045)(288.80858936,25.64415781)(288.80859375,25.42931867)
\curveto(288.80858936,25.2105645)(288.73046443,25.02501781)(288.57421875,24.87267805)
\curveto(288.42187099,24.72033061)(288.23437118,24.64415881)(288.01171875,24.64416242)
\curveto(287.79296537,24.64415881)(287.6093718,24.71837749)(287.4609375,24.86681867)
\curveto(287.31640335,25.01525219)(287.24413779,25.202752)(287.24414062,25.42931867)
\moveto(288.0234375,29.00353742)
\curveto(287.47265319,29.00352945)(287.06054423,28.70665475)(286.78710938,28.11291242)
\curveto(286.51757602,27.51915594)(286.38281053,26.61485997)(286.3828125,25.4000218)
\curveto(286.38281053,24.18908114)(286.51757602,23.28673829)(286.78710938,22.69299055)
\curveto(287.06054423,22.09923948)(287.47265319,21.80236478)(288.0234375,21.80236555)
\curveto(288.57812084,21.80236478)(288.9902298,22.09923948)(289.25976562,22.69299055)
\curveto(289.53319801,23.28673829)(289.66991662,24.18908114)(289.66992188,25.4000218)
\curveto(289.66991662,26.61485997)(289.53319801,27.51915594)(289.25976562,28.11291242)
\curveto(288.9902298,28.70665475)(288.57812084,29.00352945)(288.0234375,29.00353742)
\moveto(288.0234375,29.94103742)
\curveto(288.95702671,29.94102852)(289.66210413,29.5582164)(290.13867188,28.79259992)
\curveto(290.61913442,28.02696793)(290.85936855,26.89610969)(290.859375,25.4000218)
\curveto(290.85936855,23.90783142)(290.61913442,22.7789263)(290.13867188,22.01330305)
\curveto(289.66210413,21.24767783)(288.95702671,20.86486572)(288.0234375,20.86486555)
\curveto(287.08984107,20.86486572)(286.38476365,21.24767783)(285.90820312,22.01330305)
\curveto(285.43163961,22.7789263)(285.1933586,23.90783142)(285.19335938,25.4000218)
\curveto(285.1933586,26.89610969)(285.43163961,28.02696793)(285.90820312,28.79259992)
\curveto(286.38476365,29.5582164)(287.08984107,29.94102852)(288.0234375,29.94103742)
}
}
{
\newrgbcolor{curcolor}{0 0 0}
\pscustom[linestyle=none,fillstyle=solid,fillcolor=curcolor]
{
\newpath
\moveto(319.88812256,21.73022199)
\lineto(321.72796631,21.73022199)
\lineto(321.72796631,28.41576887)
\lineto(319.74749756,27.97045637)
\lineto(319.74749756,29.04858137)
\lineto(321.71624756,29.48217512)
\lineto(322.89984131,29.48217512)
\lineto(322.89984131,21.73022199)
\lineto(324.71624756,21.73022199)
\lineto(324.71624756,20.73412824)
\lineto(319.88812256,20.73412824)
\lineto(319.88812256,21.73022199)
}
}
{
\newrgbcolor{curcolor}{0 0 0}
\pscustom[linestyle=none,fillstyle=solid,fillcolor=curcolor]
{
\newpath
\moveto(223.32841492,8.41125227)
\curveto(223.32841209,8.62609141)(223.40263076,8.8116381)(223.55107117,8.96789289)
\curveto(223.70341171,9.12413778)(223.88505216,9.20226271)(224.09599304,9.20226789)
\curveto(224.31473923,9.20226271)(224.50223904,9.12413778)(224.65849304,8.96789289)
\curveto(224.81473873,8.8116381)(224.89286365,8.62609141)(224.89286804,8.41125227)
\curveto(224.89286365,8.19249809)(224.81473873,8.0069514)(224.65849304,7.85461164)
\curveto(224.50614528,7.70226421)(224.31864547,7.62609241)(224.09599304,7.62609602)
\curveto(223.87723966,7.62609241)(223.6936461,7.70031108)(223.54521179,7.84875227)
\curveto(223.40067764,7.99718579)(223.32841209,8.1846856)(223.32841492,8.41125227)
\moveto(224.10771179,11.98547102)
\curveto(223.55692748,11.98546305)(223.14481852,11.68858834)(222.87138367,11.09484602)
\curveto(222.60185031,10.50108953)(222.46708482,9.59679356)(222.46708679,8.38195539)
\curveto(222.46708482,7.17101474)(222.60185031,6.26867189)(222.87138367,5.67492414)
\curveto(223.14481852,5.08117308)(223.55692748,4.78429837)(224.10771179,4.78429914)
\curveto(224.66239513,4.78429837)(225.07450409,5.08117308)(225.34403992,5.67492414)
\curveto(225.6174723,6.26867189)(225.75419091,7.17101474)(225.75419617,8.38195539)
\curveto(225.75419091,9.59679356)(225.6174723,10.50108953)(225.34403992,11.09484602)
\curveto(225.07450409,11.68858834)(224.66239513,11.98546305)(224.10771179,11.98547102)
\moveto(224.10771179,12.92297102)
\curveto(225.041301,12.92296211)(225.74637842,12.54014999)(226.22294617,11.77453352)
\curveto(226.70340871,11.00890152)(226.94364285,9.87804328)(226.94364929,8.38195539)
\curveto(226.94364285,6.88976502)(226.70340871,5.7608599)(226.22294617,4.99523664)
\curveto(225.74637842,4.22961143)(225.041301,3.84679931)(224.10771179,3.84679914)
\curveto(223.17411537,3.84679931)(222.46903795,4.22961143)(221.99247742,4.99523664)
\curveto(221.5159139,5.7608599)(221.27763289,6.88976502)(221.27763367,8.38195539)
\curveto(221.27763289,9.87804328)(221.5159139,11.00890152)(221.99247742,11.77453352)
\curveto(222.46903795,12.54014999)(223.17411537,12.92296211)(224.10771179,12.92297102)
}
}
{
\newrgbcolor{curcolor}{0 0 0}
\pscustom[linestyle=none,fillstyle=solid,fillcolor=curcolor]
{
\newpath
\moveto(255.52357483,8.41125227)
\curveto(255.523572,8.62609141)(255.59779067,8.8116381)(255.74623108,8.96789289)
\curveto(255.89857162,9.12413778)(256.08021207,9.20226271)(256.29115295,9.20226789)
\curveto(256.50989914,9.20226271)(256.69739895,9.12413778)(256.85365295,8.96789289)
\curveto(257.00989864,8.8116381)(257.08802356,8.62609141)(257.08802795,8.41125227)
\curveto(257.08802356,8.19249809)(257.00989864,8.0069514)(256.85365295,7.85461164)
\curveto(256.7013052,7.70226421)(256.51380538,7.62609241)(256.29115295,7.62609602)
\curveto(256.07239958,7.62609241)(255.88880601,7.70031108)(255.7403717,7.84875227)
\curveto(255.59583755,7.99718579)(255.523572,8.1846856)(255.52357483,8.41125227)
\moveto(256.3028717,11.98547102)
\curveto(255.7520874,11.98546305)(255.33997843,11.68858834)(255.06654358,11.09484602)
\curveto(254.79701023,10.50108953)(254.66224474,9.59679356)(254.6622467,8.38195539)
\curveto(254.66224474,7.17101474)(254.79701023,6.26867189)(255.06654358,5.67492414)
\curveto(255.33997843,5.08117308)(255.7520874,4.78429837)(256.3028717,4.78429914)
\curveto(256.85755504,4.78429837)(257.269664,5.08117308)(257.53919983,5.67492414)
\curveto(257.81263221,6.26867189)(257.94935082,7.17101474)(257.94935608,8.38195539)
\curveto(257.94935082,9.59679356)(257.81263221,10.50108953)(257.53919983,11.09484602)
\curveto(257.269664,11.68858834)(256.85755504,11.98546305)(256.3028717,11.98547102)
\moveto(256.3028717,12.92297102)
\curveto(257.23646091,12.92296211)(257.94153833,12.54014999)(258.41810608,11.77453352)
\curveto(258.89856862,11.00890152)(259.13880276,9.87804328)(259.1388092,8.38195539)
\curveto(259.13880276,6.88976502)(258.89856862,5.7608599)(258.41810608,4.99523664)
\curveto(257.94153833,4.22961143)(257.23646091,3.84679931)(256.3028717,3.84679914)
\curveto(255.36927528,3.84679931)(254.66419786,4.22961143)(254.18763733,4.99523664)
\curveto(253.71107381,5.7608599)(253.4727928,6.88976502)(253.47279358,8.38195539)
\curveto(253.4727928,9.87804328)(253.71107381,11.00890152)(254.18763733,11.77453352)
\curveto(254.66419786,12.54014999)(255.36927528,12.92296211)(256.3028717,12.92297102)
}
}
{
\newrgbcolor{curcolor}{0 0 0}
\pscustom[linestyle=none,fillstyle=solid,fillcolor=curcolor]
{
\newpath
\moveto(287.24414062,8.41125227)
\curveto(287.24413779,8.62609141)(287.31835647,8.8116381)(287.46679688,8.96789289)
\curveto(287.61913742,9.12413778)(287.80077786,9.20226271)(288.01171875,9.20226789)
\curveto(288.23046493,9.20226271)(288.41796475,9.12413778)(288.57421875,8.96789289)
\curveto(288.73046443,8.8116381)(288.80858936,8.62609141)(288.80859375,8.41125227)
\curveto(288.80858936,8.19249809)(288.73046443,8.0069514)(288.57421875,7.85461164)
\curveto(288.42187099,7.70226421)(288.23437118,7.62609241)(288.01171875,7.62609602)
\curveto(287.79296537,7.62609241)(287.6093718,7.70031108)(287.4609375,7.84875227)
\curveto(287.31640335,7.99718579)(287.24413779,8.1846856)(287.24414062,8.41125227)
\moveto(288.0234375,11.98547102)
\curveto(287.47265319,11.98546305)(287.06054423,11.68858834)(286.78710938,11.09484602)
\curveto(286.51757602,10.50108953)(286.38281053,9.59679356)(286.3828125,8.38195539)
\curveto(286.38281053,7.17101474)(286.51757602,6.26867189)(286.78710938,5.67492414)
\curveto(287.06054423,5.08117308)(287.47265319,4.78429837)(288.0234375,4.78429914)
\curveto(288.57812084,4.78429837)(288.9902298,5.08117308)(289.25976562,5.67492414)
\curveto(289.53319801,6.26867189)(289.66991662,7.17101474)(289.66992188,8.38195539)
\curveto(289.66991662,9.59679356)(289.53319801,10.50108953)(289.25976562,11.09484602)
\curveto(288.9902298,11.68858834)(288.57812084,11.98546305)(288.0234375,11.98547102)
\moveto(288.0234375,12.92297102)
\curveto(288.95702671,12.92296211)(289.66210413,12.54014999)(290.13867188,11.77453352)
\curveto(290.61913442,11.00890152)(290.85936855,9.87804328)(290.859375,8.38195539)
\curveto(290.85936855,6.88976502)(290.61913442,5.7608599)(290.13867188,4.99523664)
\curveto(289.66210413,4.22961143)(288.95702671,3.84679931)(288.0234375,3.84679914)
\curveto(287.08984107,3.84679931)(286.38476365,4.22961143)(285.90820312,4.99523664)
\curveto(285.43163961,5.7608599)(285.1933586,6.88976502)(285.19335938,8.38195539)
\curveto(285.1933586,9.87804328)(285.43163961,11.00890152)(285.90820312,11.77453352)
\curveto(286.38476365,12.54014999)(287.08984107,12.92296211)(288.0234375,12.92297102)
}
}
{
\newrgbcolor{curcolor}{0 0 0}
\pscustom[linestyle=none,fillstyle=solid,fillcolor=curcolor]
{
\newpath
\moveto(322.37249756,11.51672102)
\lineto(319.61273193,6.89367414)
\lineto(322.37249756,6.89367414)
\lineto(322.37249756,11.51672102)
\moveto(322.17913818,12.59484602)
\lineto(323.55023193,12.59484602)
\lineto(323.55023193,6.89367414)
\lineto(324.71624756,6.89367414)
\lineto(324.71624756,5.93273664)
\lineto(323.55023193,5.93273664)
\lineto(323.55023193,3.84679914)
\lineto(322.37249756,3.84679914)
\lineto(322.37249756,5.93273664)
\lineto(318.66351318,5.93273664)
\lineto(318.66351318,7.05187727)
\lineto(322.17913818,12.59484602)
}
}
{
\newrgbcolor{curcolor}{0 0 0}
\pscustom[linewidth=0.96543622,linecolor=curcolor]
{
\newpath
\moveto(0.48271811,510.48647)
\lineto(879.60683,510.48647)
}
}
{
\newrgbcolor{curcolor}{0 0 0}
\pscustom[linewidth=0.96543622,linecolor=curcolor]
{
\newpath
\moveto(0.48271811,493.4869)
\lineto(879.60683,493.4869)
}
}
{
\newrgbcolor{curcolor}{0 0 0}
\pscustom[linewidth=0.96543622,linecolor=curcolor]
{
\newpath
\moveto(0.48271811,476.48733)
\lineto(879.60683,476.48733)
}
}
{
\newrgbcolor{curcolor}{0 0 0}
\pscustom[linewidth=0.96543622,linecolor=curcolor]
{
\newpath
\moveto(0.48271811,459.48775)
\lineto(879.60683,459.48775)
}
}
{
\newrgbcolor{curcolor}{0 0 0}
\pscustom[linewidth=0.96543622,linecolor=curcolor]
{
\newpath
\moveto(0.48271811,442.48818)
\lineto(879.60683,442.48818)
}
}
{
\newrgbcolor{curcolor}{0 0 0}
\pscustom[linewidth=0.96543622,linecolor=curcolor]
{
\newpath
\moveto(0.48271811,425.48861)
\lineto(879.60683,425.48861)
}
}
{
\newrgbcolor{curcolor}{0 0 0}
\pscustom[linewidth=0.96543622,linecolor=curcolor]
{
\newpath
\moveto(0.48271811,408.48904)
\lineto(879.60683,408.48904)
}
}
{
\newrgbcolor{curcolor}{0 0 0}
\pscustom[linewidth=0.96543622,linecolor=curcolor]
{
\newpath
\moveto(0.48271811,391.48946)
\lineto(879.60683,391.48946)
}
}
{
\newrgbcolor{curcolor}{0 0 0}
\pscustom[linewidth=0.96543622,linecolor=curcolor]
{
\newpath
\moveto(0.48271811,374.48989)
\lineto(879.60683,374.48989)
}
}
{
\newrgbcolor{curcolor}{0 0 0}
\pscustom[linewidth=0.96543622,linecolor=curcolor]
{
\newpath
\moveto(0.48271811,357.49032)
\lineto(879.60683,357.49032)
}
}
{
\newrgbcolor{curcolor}{0 0 0}
\pscustom[linewidth=0.96543622,linecolor=curcolor]
{
\newpath
\moveto(0.48271811,340.49074)
\lineto(879.60683,340.49074)
}
}
{
\newrgbcolor{curcolor}{0 0 0}
\pscustom[linewidth=0.96543622,linecolor=curcolor]
{
\newpath
\moveto(0.48271811,323.49117)
\lineto(879.60683,323.49117)
}
}
{
\newrgbcolor{curcolor}{0 0 0}
\pscustom[linewidth=0.96543622,linecolor=curcolor]
{
\newpath
\moveto(0.48271811,306.4916)
\lineto(879.60683,306.4916)
}
}
{
\newrgbcolor{curcolor}{0 0 0}
\pscustom[linewidth=0.96543622,linecolor=curcolor]
{
\newpath
\moveto(0.48271811,289.49203)
\lineto(879.60683,289.49203)
}
}
{
\newrgbcolor{curcolor}{0 0 0}
\pscustom[linewidth=0.96543622,linecolor=curcolor]
{
\newpath
\moveto(0.48271811,272.49245)
\lineto(879.60683,272.49245)
}
}
{
\newrgbcolor{curcolor}{0 0 0}
\pscustom[linewidth=0.96543622,linecolor=curcolor]
{
\newpath
\moveto(0.48271811,255.49288)
\lineto(879.60683,255.49288)
}
}
{
\newrgbcolor{curcolor}{0 0 0}
\pscustom[linewidth=0.96543622,linecolor=curcolor]
{
\newpath
\moveto(0.48271811,238.49331)
\lineto(879.60683,238.49331)
}
}
{
\newrgbcolor{curcolor}{0 0 0}
\pscustom[linewidth=0.96543622,linecolor=curcolor]
{
\newpath
\moveto(0.48271811,221.49373)
\lineto(879.60683,221.49373)
}
}
{
\newrgbcolor{curcolor}{0 0 0}
\pscustom[linewidth=0.96543622,linecolor=curcolor]
{
\newpath
\moveto(0.48271811,204.49416)
\lineto(879.60683,204.49416)
}
}
{
\newrgbcolor{curcolor}{0 0 0}
\pscustom[linewidth=0.96543622,linecolor=curcolor]
{
\newpath
\moveto(0.48271811,187.49459)
\lineto(879.60683,187.49459)
}
}
{
\newrgbcolor{curcolor}{0 0 0}
\pscustom[linewidth=0.96543622,linecolor=curcolor]
{
\newpath
\moveto(0.48271811,170.49502)
\lineto(879.60683,170.49502)
}
}
{
\newrgbcolor{curcolor}{0 0 0}
\pscustom[linewidth=0.96543622,linecolor=curcolor]
{
\newpath
\moveto(0.48271811,153.49544)
\lineto(879.60683,153.49544)
}
}
{
\newrgbcolor{curcolor}{0 0 0}
\pscustom[linewidth=0.96543622,linecolor=curcolor]
{
\newpath
\moveto(0.48271811,136.49587)
\lineto(879.60683,136.49587)
}
}
{
\newrgbcolor{curcolor}{0 0 0}
\pscustom[linewidth=0.96543622,linecolor=curcolor]
{
\newpath
\moveto(0.48271811,119.49629)
\lineto(879.60683,119.49629)
}
}
{
\newrgbcolor{curcolor}{0 0 0}
\pscustom[linewidth=0.96543622,linecolor=curcolor]
{
\newpath
\moveto(0.48271811,102.49672)
\lineto(879.60683,102.49672)
}
}
{
\newrgbcolor{curcolor}{0 0 0}
\pscustom[linewidth=0.96543622,linecolor=curcolor]
{
\newpath
\moveto(0.48271811,85.49715)
\lineto(879.60683,85.49715)
}
}
{
\newrgbcolor{curcolor}{0 0 0}
\pscustom[linewidth=0.96543622,linecolor=curcolor]
{
\newpath
\moveto(0.48271811,68.49757)
\lineto(879.60683,68.49757)
}
}
{
\newrgbcolor{curcolor}{0 0 0}
\pscustom[linewidth=0.96543622,linecolor=curcolor]
{
\newpath
\moveto(0.48271811,51.49799979)
\lineto(879.60683,51.49799979)
}
}
{
\newrgbcolor{curcolor}{0 0 0}
\pscustom[linewidth=0.96543622,linecolor=curcolor]
{
\newpath
\moveto(0.48271811,34.49838)
\lineto(879.60683,34.49838)
}
}
{
\newrgbcolor{curcolor}{0 0 0}
\pscustom[linewidth=0.96543622,linecolor=curcolor]
{
\newpath
\moveto(0.48271811,17.49885237)
\lineto(879.60683,17.49885237)
}
}
\end{pspicture}

%\caption{Matriz de adyacencias de la red social}
%\label{contactos_matriz}
%\end{figure}

\subsection{Espacios Virtuales}
En el cuadro \ref{espacios_tabla_1} pueden verse los distintos tipos de
espacios virtuales y sus indicadores propios, entre los que destaca la
supremacía del espacio portada, por sobre cualquier otro espacio, siendo el que 
capta mas audiencia de entre los espacios. También es notorio el ausente uso de
espacios para equipos de trabajo en los grupos, cosa que puede ser debida a las
escasez de grupos registrados. Si bien la portada acapara la mayor audiencia,
no acapara la mayor cantidad de recursos (figura \ref{espacios_bars_1}),
llevándose los espacios de comunidades un 45\% del contenido del sitio, 
reforzando la teoría de fomento hacia los espacios menos formales (figura
\ref{espacios_pie_1}).

\begin{table}
\centering
\begin{tabular}{l|c c c c c}
$Tipo$ & $Cantidad$ & $Recursos$ & $Audiencia$ &
$Recursos/Cantidad$ & $Audiencia/Recursos$ \\
\hline
$Portada    $ & $ 1$ & $17$ & $1314$ & $17   $ & $77.29$ \\
$Carreras   $ & $ 4$ & $ 1$ & $  18$ & $ 0.25$ & $18   $ \\
$Areas      $ & $ 5$ & $ 2$ & $  15$ & $ 0.4 $ & $ 7.5 $ \\
$Materias   $ & $ 4$ & $13$ & $  85$ & $ 3.25$ & $ 6.54$ \\
$Grupos     $ & $ 8$ & $ 3$ & $   1$ & $ 0.38$ & $ 0.33$ \\
$Equipos    $ & $ 0$ & $ 0$ & $   0$ & $    -$ & $ -   $ \\
$Comunidades$ & $11$ & $30$ & $  96$ & $ 2.73$ & $ 3.2 $ \\
\end{tabular}
\caption{Clasificación de los espacios y su actividad}
\label{espacios_tabla_1}
\end{table}

\begin{figure}
\centering
%LaTeX with PSTricks extensions
%%Creator: inkscape 0.48.5
%%Please note this file requires PSTricks extensions
\psset{xunit=.5pt,yunit=.5pt,runit=.5pt}
\begin{pspicture}(854,432)
{
\newrgbcolor{curcolor}{0 0 0}
\pscustom[linestyle=none,fillstyle=solid,fillcolor=curcolor]
{
\newpath
\moveto(352.02975342,25.63030273)
\lineto(356.93475342,25.63030273)
\lineto(358.22475342,25.63030273)
\curveto(358.33474339,25.63029204)(358.44474328,25.63029204)(358.55475342,25.63030273)
\curveto(358.66474306,25.64029203)(358.74974298,25.62029205)(358.80975342,25.57030273)
\curveto(358.8297429,25.55029212)(358.84474288,25.52529214)(358.85475342,25.49530273)
\curveto(358.87474285,25.4652922)(358.89474283,25.43529223)(358.91475342,25.40530273)
\curveto(358.91474281,25.33529233)(358.89974283,25.22029245)(358.86975342,25.06030273)
\curveto(358.83974289,24.91029276)(358.80474292,24.79529287)(358.76475342,24.71530273)
\curveto(358.70474302,24.57529309)(358.60474312,24.49529317)(358.46475342,24.47530273)
\curveto(358.33474339,24.4652932)(358.17974355,24.46029321)(357.99975342,24.46030273)
\lineto(356.49975342,24.46030273)
\lineto(353.97975342,24.46030273)
\lineto(353.40975342,24.46030273)
\curveto(353.19974853,24.4702932)(353.03974869,24.44529322)(352.92975342,24.38530273)
\curveto(352.81974891,24.32529334)(352.74474898,24.22029345)(352.70475342,24.07030273)
\curveto(352.67474905,23.92029375)(352.64474908,23.7652939)(352.61475342,23.60530273)
\lineto(352.29975342,22.07530273)
\curveto(352.27974945,21.9652957)(352.24974948,21.83529583)(352.20975342,21.68530273)
\curveto(352.17974955,21.53529613)(352.16474956,21.41529625)(352.16475342,21.32530273)
\curveto(352.17474955,21.20529646)(352.21974951,21.12529654)(352.29975342,21.08530273)
\curveto(352.33974939,21.0652966)(352.40474932,21.04529662)(352.49475342,21.02530273)
\lineto(352.64475342,21.02530273)
\curveto(352.68474904,21.01529665)(352.724749,21.01029666)(352.76475342,21.01030273)
\curveto(352.81474891,21.02029665)(352.86474886,21.02529664)(352.91475342,21.02530273)
\lineto(353.42475342,21.02530273)
\lineto(356.36475342,21.02530273)
\lineto(356.66475342,21.02530273)
\curveto(356.77474495,21.03529663)(356.88474484,21.03529663)(356.99475342,21.02530273)
\curveto(357.11474461,21.02529664)(357.21974451,21.01529665)(357.30975342,20.99530273)
\curveto(357.40974432,20.98529668)(357.47974425,20.9652967)(357.51975342,20.93530273)
\curveto(357.54974418,20.91529675)(357.56474416,20.8702968)(357.56475342,20.80030273)
\curveto(357.57474415,20.73029694)(357.57474415,20.65529701)(357.56475342,20.57530273)
\curveto(357.56474416,20.49529717)(357.54974418,20.41029726)(357.51975342,20.32030273)
\curveto(357.49974423,20.24029743)(357.47474425,20.1702975)(357.44475342,20.11030273)
\curveto(357.40474432,20.02029765)(357.34474438,19.95529771)(357.26475342,19.91530273)
\curveto(357.24474448,19.89529777)(357.21474451,19.88029779)(357.17475342,19.87030273)
\curveto(357.14474458,19.8702978)(357.11474461,19.8652978)(357.08475342,19.85530273)
\lineto(356.99475342,19.85530273)
\curveto(356.93474479,19.84529782)(356.87974485,19.84029783)(356.82975342,19.84030273)
\curveto(356.78974494,19.85029782)(356.74474498,19.85529781)(356.69475342,19.85530273)
\lineto(356.13975342,19.85530273)
\lineto(352.97475342,19.85530273)
\lineto(352.61475342,19.85530273)
\curveto(352.50474922,19.8652978)(352.39474933,19.86029781)(352.28475342,19.84030273)
\curveto(352.18474954,19.83029784)(352.09474963,19.80529786)(352.01475342,19.76530273)
\curveto(351.93474979,19.72529794)(351.86974986,19.65529801)(351.81975342,19.55530273)
\curveto(351.77974995,19.49529817)(351.75474997,19.42529824)(351.74475342,19.34530273)
\lineto(351.71475342,19.10530273)
\lineto(351.53475342,18.26530273)
\lineto(351.24975342,16.84030273)
\curveto(351.2297505,16.70030097)(351.20975052,16.5703011)(351.18975342,16.45030273)
\curveto(351.17975055,16.34030133)(351.20475052,16.26030141)(351.26475342,16.21030273)
\curveto(351.3247504,16.16030151)(351.39975033,16.13030154)(351.48975342,16.12030273)
\lineto(351.78975342,16.12030273)
\lineto(352.74975342,16.12030273)
\lineto(355.52475342,16.12030273)
\lineto(356.37975342,16.12030273)
\lineto(356.61975342,16.12030273)
\curveto(356.69974503,16.13030154)(356.76974496,16.12530154)(356.82975342,16.10530273)
\curveto(356.93974479,16.0653016)(357.00474472,16.01030166)(357.02475342,15.94030273)
\curveto(357.04474468,15.91030176)(357.04974468,15.86030181)(357.03975342,15.79030273)
\curveto(357.03974469,15.72030195)(357.03474469,15.64530202)(357.02475342,15.56530273)
\curveto(357.01474471,15.49530217)(356.99474473,15.42030225)(356.96475342,15.34030273)
\curveto(356.94474478,15.2703024)(356.9247448,15.21530245)(356.90475342,15.17530273)
\curveto(356.85474487,15.09530257)(356.79974493,15.04030263)(356.73975342,15.01030273)
\curveto(356.66974506,14.9703027)(356.58474514,14.95030272)(356.48475342,14.95030273)
\lineto(356.21475342,14.95030273)
\lineto(355.16475342,14.95030273)
\lineto(351.17475342,14.95030273)
\lineto(350.12475342,14.95030273)
\curveto(349.98475174,14.95030272)(349.86475186,14.95530271)(349.76475342,14.96530273)
\curveto(349.67475205,14.98530268)(349.60975212,15.03530263)(349.56975342,15.11530273)
\curveto(349.54975218,15.17530249)(349.54475218,15.25030242)(349.55475342,15.34030273)
\curveto(349.57475215,15.44030223)(349.59475213,15.53530213)(349.61475342,15.62530273)
\lineto(349.82475342,16.67530273)
\lineto(350.63475342,20.69530273)
\lineto(351.30975342,24.05530273)
\lineto(351.48975342,24.98530273)
\curveto(351.50975022,25.07529259)(351.5247502,25.1652925)(351.53475342,25.25530273)
\curveto(351.55475017,25.34529232)(351.58975014,25.41529225)(351.63975342,25.46530273)
\curveto(351.69975003,25.53529213)(351.78474994,25.58529208)(351.89475342,25.61530273)
\curveto(351.9247498,25.62529204)(351.94474978,25.62529204)(351.95475342,25.61530273)
\curveto(351.97474975,25.61529205)(351.99974973,25.62029205)(352.02975342,25.63030273)
}
}
{
\newrgbcolor{curcolor}{0 0 0}
\pscustom[linestyle=none,fillstyle=solid,fillcolor=curcolor]
{
\newpath
\moveto(362.41467529,22.85530273)
\curveto(363.13466964,22.8652948)(363.71966905,22.78029489)(364.16967529,22.60030273)
\curveto(364.62966814,22.43029524)(364.94966782,22.12529554)(365.12967529,21.68530273)
\curveto(365.17966759,21.57529609)(365.20966756,21.46029621)(365.21967529,21.34030273)
\curveto(365.23966753,21.23029644)(365.25466752,21.10529656)(365.26467529,20.96530273)
\curveto(365.2746675,20.89529677)(365.26466751,20.82029685)(365.23467529,20.74030273)
\curveto(365.21466756,20.670297)(365.18966758,20.61529705)(365.15967529,20.57530273)
\curveto(365.13966763,20.55529711)(365.10966766,20.53529713)(365.06967529,20.51530273)
\curveto(365.03966773,20.50529716)(365.01466776,20.49029718)(364.99467529,20.47030273)
\curveto(364.93466784,20.45029722)(364.87966789,20.44529722)(364.82967529,20.45530273)
\curveto(364.78966798,20.4652972)(364.74466803,20.4652972)(364.69467529,20.45530273)
\curveto(364.60466817,20.43529723)(364.49466828,20.43029724)(364.36467529,20.44030273)
\curveto(364.24466853,20.46029721)(364.15966861,20.48529718)(364.10967529,20.51530273)
\curveto(364.03966873,20.5652971)(363.99966877,20.63029704)(363.98967529,20.71030273)
\curveto(363.98966878,20.80029687)(363.9696688,20.88529678)(363.92967529,20.96530273)
\curveto(363.87966889,21.12529654)(363.78466899,21.2702964)(363.64467529,21.40030273)
\curveto(363.55466922,21.48029619)(363.44466933,21.54029613)(363.31467529,21.58030273)
\curveto(363.19466958,21.62029605)(363.06466971,21.66029601)(362.92467529,21.70030273)
\curveto(362.88466989,21.72029595)(362.83466994,21.72529594)(362.77467529,21.71530273)
\curveto(362.72467005,21.71529595)(362.67967009,21.72029595)(362.63967529,21.73030273)
\curveto(362.57967019,21.75029592)(362.50467027,21.76029591)(362.41467529,21.76030273)
\curveto(362.32467045,21.76029591)(362.24967052,21.75029592)(362.18967529,21.73030273)
\lineto(362.09967529,21.73030273)
\curveto(362.03967073,21.72029595)(361.98467079,21.71029596)(361.93467529,21.70030273)
\curveto(361.88467089,21.70029597)(361.83467094,21.69529597)(361.78467529,21.68530273)
\curveto(361.51467126,21.62529604)(361.27967149,21.54029613)(361.07967529,21.43030273)
\curveto(360.88967188,21.32029635)(360.73967203,21.13529653)(360.62967529,20.87530273)
\curveto(360.59967217,20.80529686)(360.58467219,20.73529693)(360.58467529,20.66530273)
\curveto(360.58467219,20.59529707)(360.58967218,20.53529713)(360.59967529,20.48530273)
\curveto(360.62967214,20.33529733)(360.67967209,20.22529744)(360.74967529,20.15530273)
\curveto(360.81967195,20.09529757)(360.91467186,20.02529764)(361.03467529,19.94530273)
\curveto(361.1746716,19.84529782)(361.33967143,19.7702979)(361.52967529,19.72030273)
\curveto(361.71967105,19.68029799)(361.90967086,19.63029804)(362.09967529,19.57030273)
\curveto(362.21967055,19.53029814)(362.33967043,19.50029817)(362.45967529,19.48030273)
\curveto(362.58967018,19.46029821)(362.71467006,19.43029824)(362.83467529,19.39030273)
\curveto(363.03466974,19.33029834)(363.22966954,19.2702984)(363.41967529,19.21030273)
\curveto(363.60966916,19.16029851)(363.79466898,19.09529857)(363.97467529,19.01530273)
\curveto(364.02466875,18.99529867)(364.0696687,18.97529869)(364.10967529,18.95530273)
\curveto(364.15966861,18.93529873)(364.20966856,18.91029876)(364.25967529,18.88030273)
\curveto(364.42966834,18.76029891)(364.5746682,18.62529904)(364.69467529,18.47530273)
\curveto(364.81466796,18.32529934)(364.90466787,18.13529953)(364.96467529,17.90530273)
\lineto(364.96467529,17.62030273)
\curveto(364.96466781,17.55030012)(364.95966781,17.47530019)(364.94967529,17.39530273)
\curveto(364.93966783,17.32530034)(364.92966784,17.24530042)(364.91967529,17.15530273)
\lineto(364.88967529,17.00530273)
\curveto(364.84966792,16.93530073)(364.81966795,16.8653008)(364.79967529,16.79530273)
\curveto(364.78966798,16.72530094)(364.769668,16.65530101)(364.73967529,16.58530273)
\curveto(364.68966808,16.47530119)(364.63466814,16.3703013)(364.57467529,16.27030273)
\curveto(364.51466826,16.1703015)(364.44966832,16.08030159)(364.37967529,16.00030273)
\curveto(364.1696686,15.74030193)(363.92466885,15.53030214)(363.64467529,15.37030273)
\curveto(363.36466941,15.22030245)(363.05966971,15.09030258)(362.72967529,14.98030273)
\curveto(362.62967014,14.95030272)(362.52967024,14.93030274)(362.42967529,14.92030273)
\curveto(362.32967044,14.90030277)(362.23467054,14.87530279)(362.14467529,14.84530273)
\curveto(362.03467074,14.82530284)(361.92967084,14.81530285)(361.82967529,14.81530273)
\curveto(361.72967104,14.81530285)(361.62967114,14.80530286)(361.52967529,14.78530273)
\lineto(361.37967529,14.78530273)
\curveto(361.32967144,14.77530289)(361.25967151,14.7703029)(361.16967529,14.77030273)
\curveto(361.07967169,14.7703029)(361.00967176,14.77530289)(360.95967529,14.78530273)
\lineto(360.79467529,14.78530273)
\curveto(360.73467204,14.80530286)(360.6696721,14.81530285)(360.59967529,14.81530273)
\curveto(360.52967224,14.80530286)(360.4746723,14.81030286)(360.43467529,14.83030273)
\curveto(360.38467239,14.84030283)(360.31967245,14.84530282)(360.23967529,14.84530273)
\curveto(360.15967261,14.8653028)(360.08467269,14.88530278)(360.01467529,14.90530273)
\curveto(359.94467283,14.91530275)(359.8696729,14.93530273)(359.78967529,14.96530273)
\curveto(359.49967327,15.0653026)(359.25467352,15.19030248)(359.05467529,15.34030273)
\curveto(358.85467392,15.49030218)(358.69467408,15.68530198)(358.57467529,15.92530273)
\curveto(358.51467426,16.05530161)(358.46467431,16.19030148)(358.42467529,16.33030273)
\curveto(358.39467438,16.4703012)(358.3746744,16.62530104)(358.36467529,16.79530273)
\curveto(358.35467442,16.85530081)(358.35967441,16.92530074)(358.37967529,17.00530273)
\curveto(358.39967437,17.09530057)(358.42467435,17.1653005)(358.45467529,17.21530273)
\curveto(358.49467428,17.25530041)(358.55467422,17.29530037)(358.63467529,17.33530273)
\curveto(358.68467409,17.35530031)(358.75467402,17.3653003)(358.84467529,17.36530273)
\curveto(358.94467383,17.37530029)(359.03467374,17.37530029)(359.11467529,17.36530273)
\curveto(359.20467357,17.35530031)(359.28967348,17.34030033)(359.36967529,17.32030273)
\curveto(359.45967331,17.31030036)(359.51467326,17.29530037)(359.53467529,17.27530273)
\curveto(359.59467318,17.22530044)(359.62467315,17.15030052)(359.62467529,17.05030273)
\curveto(359.63467314,16.96030071)(359.65467312,16.87530079)(359.68467529,16.79530273)
\curveto(359.73467304,16.57530109)(359.83467294,16.40530126)(359.98467529,16.28530273)
\curveto(360.08467269,16.19530147)(360.20467257,16.12530154)(360.34467529,16.07530273)
\curveto(360.48467229,16.02530164)(360.63467214,15.97530169)(360.79467529,15.92530273)
\lineto(361.10967529,15.88030273)
\lineto(361.19967529,15.88030273)
\curveto(361.25967151,15.86030181)(361.34467143,15.85030182)(361.45467529,15.85030273)
\curveto(361.5746712,15.85030182)(361.67967109,15.86030181)(361.76967529,15.88030273)
\curveto(361.83967093,15.88030179)(361.89467088,15.88530178)(361.93467529,15.89530273)
\curveto(361.99467078,15.90530176)(362.05467072,15.91030176)(362.11467529,15.91030273)
\curveto(362.1746706,15.92030175)(362.22967054,15.93030174)(362.27967529,15.94030273)
\curveto(362.58967018,16.02030165)(362.83966993,16.12530154)(363.02967529,16.25530273)
\curveto(363.22966954,16.38530128)(363.39466938,16.60530106)(363.52467529,16.91530273)
\curveto(363.55466922,16.9653007)(363.5696692,17.02030065)(363.56967529,17.08030273)
\curveto(363.57966919,17.14030053)(363.57966919,17.18530048)(363.56967529,17.21530273)
\curveto(363.55966921,17.40530026)(363.51966925,17.54530012)(363.44967529,17.63530273)
\curveto(363.37966939,17.73529993)(363.28466949,17.82529984)(363.16467529,17.90530273)
\curveto(363.08466969,17.9652997)(362.98966978,18.01529965)(362.87967529,18.05530273)
\lineto(362.57967529,18.17530273)
\curveto(362.54967022,18.18529948)(362.51967025,18.19029948)(362.48967529,18.19030273)
\curveto(362.4696703,18.19029948)(362.44967032,18.20029947)(362.42967529,18.22030273)
\curveto(362.10967066,18.33029934)(361.769671,18.41029926)(361.40967529,18.46030273)
\curveto(361.05967171,18.52029915)(360.73967203,18.61529905)(360.44967529,18.74530273)
\curveto(360.35967241,18.78529888)(360.2696725,18.82029885)(360.17967529,18.85030273)
\curveto(360.09967267,18.88029879)(360.02467275,18.92029875)(359.95467529,18.97030273)
\curveto(359.78467299,19.08029859)(359.63467314,19.20529846)(359.50467529,19.34530273)
\curveto(359.3746734,19.48529818)(359.28467349,19.66029801)(359.23467529,19.87030273)
\curveto(359.21467356,19.94029773)(359.20467357,20.01029766)(359.20467529,20.08030273)
\lineto(359.20467529,20.30530273)
\curveto(359.19467358,20.42529724)(359.20967356,20.56029711)(359.24967529,20.71030273)
\curveto(359.28967348,20.8702968)(359.32967344,21.00529666)(359.36967529,21.11530273)
\curveto(359.39967337,21.1652965)(359.41967335,21.20529646)(359.42967529,21.23530273)
\curveto(359.44967332,21.27529639)(359.4746733,21.31529635)(359.50467529,21.35530273)
\curveto(359.63467314,21.58529608)(359.79467298,21.78529588)(359.98467529,21.95530273)
\curveto(360.1746726,22.12529554)(360.38467239,22.27529539)(360.61467529,22.40530273)
\curveto(360.774672,22.49529517)(360.94967182,22.5652951)(361.13967529,22.61530273)
\curveto(361.33967143,22.67529499)(361.54467123,22.73029494)(361.75467529,22.78030273)
\curveto(361.82467095,22.79029488)(361.88967088,22.80029487)(361.94967529,22.81030273)
\curveto(362.01967075,22.82029485)(362.09467068,22.83029484)(362.17467529,22.84030273)
\curveto(362.21467056,22.85029482)(362.25467052,22.85029482)(362.29467529,22.84030273)
\curveto(362.34467043,22.83029484)(362.38467039,22.83529483)(362.41467529,22.85530273)
}
}
{
\newrgbcolor{curcolor}{0 0 0}
\pscustom[linestyle=none,fillstyle=solid,fillcolor=curcolor]
{
\newpath
\moveto(374.12967529,19.01530273)
\curveto(374.12966614,18.9652987)(374.11966615,18.90029877)(374.09967529,18.82030273)
\curveto(374.08966618,18.74029893)(374.0746662,18.67529899)(374.05467529,18.62530273)
\curveto(374.02466625,18.57529909)(374.00966626,18.52529914)(374.00967529,18.47530273)
\curveto(374.00966626,18.43529923)(374.00466627,18.39529927)(373.99467529,18.35530273)
\curveto(373.9746663,18.28529938)(373.95466632,18.23029944)(373.93467529,18.19030273)
\lineto(373.84467529,17.92030273)
\curveto(373.82466645,17.83029984)(373.79466648,17.74029993)(373.75467529,17.65030273)
\curveto(373.72466655,17.5703001)(373.68966658,17.49030018)(373.64967529,17.41030273)
\curveto(373.61966665,17.34030033)(373.57966669,17.2653004)(373.52967529,17.18530273)
\curveto(373.33966693,16.81530085)(373.10966716,16.48030119)(372.83967529,16.18030273)
\curveto(372.75966751,16.09030158)(372.6746676,16.00030167)(372.58467529,15.91030273)
\curveto(372.49466778,15.83030184)(372.40466787,15.75530191)(372.31467529,15.68530273)
\lineto(372.22467529,15.61030273)
\curveto(372.14466813,15.56030211)(372.0696682,15.51030216)(371.99967529,15.46030273)
\curveto(371.92966834,15.41030226)(371.84966842,15.36030231)(371.75967529,15.31030273)
\curveto(371.62966864,15.23030244)(371.48966878,15.16030251)(371.33967529,15.10030273)
\curveto(371.19966907,15.05030262)(371.05466922,15.00030267)(370.90467529,14.95030273)
\curveto(370.82466945,14.93030274)(370.74466953,14.91530275)(370.66467529,14.90530273)
\curveto(370.58466969,14.89530277)(370.50466977,14.88030279)(370.42467529,14.86030273)
\lineto(370.36467529,14.86030273)
\curveto(370.35466992,14.85030282)(370.33966993,14.84530282)(370.31967529,14.84530273)
\curveto(370.21967005,14.82530284)(370.07967019,14.81530285)(369.89967529,14.81530273)
\curveto(369.72967054,14.80530286)(369.59967067,14.81030286)(369.50967529,14.83030273)
\lineto(369.43467529,14.83030273)
\curveto(369.36467091,14.84030283)(369.29967097,14.85030282)(369.23967529,14.86030273)
\curveto(369.17967109,14.86030281)(369.11967115,14.8703028)(369.05967529,14.89030273)
\curveto(368.89967137,14.94030273)(368.74967152,14.98530268)(368.60967529,15.02530273)
\curveto(368.4696718,15.0653026)(368.34467193,15.12530254)(368.23467529,15.20530273)
\curveto(368.08467219,15.29530237)(367.95967231,15.39030228)(367.85967529,15.49030273)
\curveto(367.82967244,15.52030215)(367.77967249,15.56030211)(367.70967529,15.61030273)
\curveto(367.63967263,15.670302)(367.56467271,15.67530199)(367.48467529,15.62530273)
\curveto(367.44467283,15.59530207)(367.41967285,15.55530211)(367.40967529,15.50530273)
\curveto(367.39967287,15.45530221)(367.3746729,15.40030227)(367.33467529,15.34030273)
\curveto(367.32467295,15.31030236)(367.31967295,15.27530239)(367.31967529,15.23530273)
\curveto(367.31967295,15.20530246)(367.31467296,15.1703025)(367.30467529,15.13030273)
\curveto(367.26467301,15.0703026)(367.23967303,15.00530266)(367.22967529,14.93530273)
\curveto(367.21967305,14.85530281)(367.20967306,14.78530288)(367.19967529,14.72530273)
\lineto(366.83967529,12.92530273)
\curveto(366.80967346,12.78530488)(366.77967349,12.64030503)(366.74967529,12.49030273)
\curveto(366.71967355,12.34030533)(366.6696736,12.22530544)(366.59967529,12.14530273)
\curveto(366.52967374,12.07530559)(366.43967383,12.04030563)(366.32967529,12.04030273)
\curveto(366.21967405,12.03030564)(366.10967416,12.02530564)(365.99967529,12.02530273)
\lineto(365.75967529,12.02530273)
\curveto(365.69967457,12.04530562)(365.64467463,12.0653056)(365.59467529,12.08530273)
\curveto(365.55467472,12.10530556)(365.52467475,12.14030553)(365.50467529,12.19030273)
\curveto(365.4746748,12.26030541)(365.4746748,12.3703053)(365.50467529,12.52030273)
\curveto(365.54467473,12.670305)(365.5746747,12.80030487)(365.59467529,12.91030273)
\lineto(367.39467529,21.91030273)
\curveto(367.41467286,22.03029564)(367.43967283,22.15029552)(367.46967529,22.27030273)
\curveto(367.49967277,22.40029527)(367.54967272,22.50529516)(367.61967529,22.58530273)
\curveto(367.65967261,22.62529504)(367.73467254,22.65529501)(367.84467529,22.67530273)
\curveto(367.95467232,22.70529496)(368.0696722,22.71529495)(368.18967529,22.70530273)
\curveto(368.30967196,22.70529496)(368.41967185,22.69029498)(368.51967529,22.66030273)
\curveto(368.61967165,22.64029503)(368.67967159,22.61029506)(368.69967529,22.57030273)
\curveto(368.72967154,22.52029515)(368.73967153,22.46029521)(368.72967529,22.39030273)
\curveto(368.71967155,22.32029535)(368.72467155,22.25029542)(368.74467529,22.18030273)
\curveto(368.75467152,22.15029552)(368.76467151,22.12529554)(368.77467529,22.10530273)
\lineto(368.81967529,22.06030273)
\curveto(368.92967134,22.05029562)(369.02967124,22.08529558)(369.11967529,22.16530273)
\curveto(369.20967106,22.24529542)(369.29467098,22.31029536)(369.37467529,22.36030273)
\curveto(369.6746706,22.54029513)(370.01467026,22.68029499)(370.39467529,22.78030273)
\curveto(370.48466979,22.80029487)(370.5746697,22.81529485)(370.66467529,22.82530273)
\curveto(370.76466951,22.83529483)(370.8696694,22.85029482)(370.97967529,22.87030273)
\curveto(371.01966925,22.88029479)(371.0696692,22.88029479)(371.12967529,22.87030273)
\curveto(371.18966908,22.86029481)(371.22966904,22.8652948)(371.24967529,22.88530273)
\curveto(371.67966859,22.89529477)(372.04966822,22.85029482)(372.35967529,22.75030273)
\curveto(372.6696676,22.66029501)(372.93966733,22.53029514)(373.16967529,22.36030273)
\curveto(373.20966706,22.32029535)(373.24966702,22.28029539)(373.28967529,22.24030273)
\curveto(373.33966693,22.21029546)(373.38466689,22.17529549)(373.42467529,22.13530273)
\curveto(373.44466683,22.11529555)(373.45966681,22.09529557)(373.46967529,22.07530273)
\curveto(373.47966679,22.0652956)(373.49466678,22.05029562)(373.51467529,22.03030273)
\curveto(373.55466672,21.98029569)(373.59466668,21.92529574)(373.63467529,21.86530273)
\curveto(373.68466659,21.80529586)(373.72966654,21.74529592)(373.76967529,21.68530273)
\curveto(373.85966641,21.51529615)(373.94966632,21.33029634)(374.03967529,21.13030273)
\curveto(374.08966618,21.00029667)(374.12466615,20.85529681)(374.14467529,20.69530273)
\curveto(374.1746661,20.53529713)(374.19466608,20.37529729)(374.20467529,20.21530273)
\curveto(374.21466606,20.13529753)(374.21466606,20.05029762)(374.20467529,19.96030273)
\curveto(374.20466607,19.8702978)(374.20966606,19.78529788)(374.21967529,19.70530273)
\lineto(374.18967529,19.58530273)
\lineto(374.18967529,19.49530273)
\curveto(374.19966607,19.44529822)(374.19466608,19.39029828)(374.17467529,19.33030273)
\curveto(374.15466612,19.2702984)(374.14966612,19.21529845)(374.15967529,19.16530273)
\lineto(374.12967529,19.01530273)
\moveto(372.71967529,18.61030273)
\curveto(372.74966752,18.66029901)(372.76466751,18.72029895)(372.76467529,18.79030273)
\curveto(372.7746675,18.8702988)(372.78466749,18.94029873)(372.79467529,19.00030273)
\curveto(372.83466744,19.1702985)(372.85966741,19.33029834)(372.86967529,19.48030273)
\curveto(372.88966738,19.63029804)(372.88466739,19.77529789)(372.85467529,19.91530273)
\curveto(372.85466742,19.97529769)(372.84966742,20.03529763)(372.83967529,20.09530273)
\curveto(372.83966743,20.1652975)(372.82966744,20.23029744)(372.80967529,20.29030273)
\curveto(372.75966751,20.56029711)(372.63966763,20.82029685)(372.44967529,21.07030273)
\curveto(372.269668,21.32029635)(372.08466819,21.49029618)(371.89467529,21.58030273)
\curveto(371.81466846,21.62029605)(371.73466854,21.65029602)(371.65467529,21.67030273)
\curveto(371.5746687,21.69029598)(371.49466878,21.71529595)(371.41467529,21.74530273)
\curveto(371.32466895,21.7652959)(371.21966905,21.77529589)(371.09967529,21.77530273)
\lineto(370.76967529,21.77530273)
\curveto(370.74966952,21.75529591)(370.70966956,21.74529592)(370.64967529,21.74530273)
\curveto(370.59966967,21.75529591)(370.55466972,21.75529591)(370.51467529,21.74530273)
\curveto(370.41466986,21.72529594)(370.31966995,21.70529596)(370.22967529,21.68530273)
\curveto(370.14967012,21.665296)(370.06467021,21.63529603)(369.97467529,21.59530273)
\curveto(369.62467065,21.45529621)(369.31967095,21.25029642)(369.05967529,20.98030273)
\curveto(368.79967147,20.72029695)(368.57967169,20.41529725)(368.39967529,20.06530273)
\curveto(368.33967193,19.95529771)(368.28967198,19.84529782)(368.24967529,19.73530273)
\curveto(368.21967205,19.62529804)(368.18467209,19.51529815)(368.14467529,19.40530273)
\curveto(368.12467215,19.3652983)(368.10967216,19.32529834)(368.09967529,19.28530273)
\curveto(368.08967218,19.25529841)(368.07967219,19.22029845)(368.06967529,19.18030273)
\lineto(368.03967529,19.06030273)
\curveto(368.01967225,19.01029866)(367.99967227,18.93529873)(367.97967529,18.83530273)
\curveto(367.95967231,18.74529892)(367.94967232,18.67529899)(367.94967529,18.62530273)
\lineto(367.93467529,18.50530273)
\curveto(367.93467234,18.4652992)(367.92967234,18.42529924)(367.91967529,18.38530273)
\curveto(367.90967236,18.34529932)(367.90967236,18.31029936)(367.91967529,18.28030273)
\curveto(367.91967235,18.25029942)(367.91467236,18.22029945)(367.90467529,18.19030273)
\lineto(367.90467529,18.08530273)
\lineto(367.90467529,17.84530273)
\curveto(367.90467237,17.7652999)(367.90967236,17.68529998)(367.91967529,17.60530273)
\curveto(367.95967231,17.2653004)(368.04967222,16.9653007)(368.18967529,16.70530273)
\curveto(368.33967193,16.45530121)(368.55967171,16.26030141)(368.84967529,16.12030273)
\curveto(369.01967125,16.04030163)(369.19967107,15.98030169)(369.38967529,15.94030273)
\curveto(369.42967084,15.92030175)(369.4696708,15.91030176)(369.50967529,15.91030273)
\curveto(369.54967072,15.92030175)(369.58967068,15.92030175)(369.62967529,15.91030273)
\lineto(369.74967529,15.91030273)
\curveto(369.81967045,15.89030178)(369.88967038,15.89030178)(369.95967529,15.91030273)
\lineto(370.07967529,15.91030273)
\curveto(370.18967008,15.93030174)(370.29466998,15.94530172)(370.39467529,15.95530273)
\curveto(370.50466977,15.9653017)(370.61466966,15.99030168)(370.72467529,16.03030273)
\curveto(371.05466922,16.16030151)(371.33966893,16.33030134)(371.57967529,16.54030273)
\curveto(371.81966845,16.76030091)(372.03466824,17.02530064)(372.22467529,17.33530273)
\curveto(372.30466797,17.47530019)(372.3696679,17.61530005)(372.41967529,17.75530273)
\curveto(372.47966779,17.90529976)(372.54466773,18.06029961)(372.61467529,18.22030273)
\curveto(372.63466764,18.2702994)(372.64466763,18.31529935)(372.64467529,18.35530273)
\curveto(372.65466762,18.39529927)(372.6696676,18.44029923)(372.68967529,18.49030273)
\lineto(372.71967529,18.61030273)
}
}
{
\newrgbcolor{curcolor}{0 0 0}
\pscustom[linestyle=none,fillstyle=solid,fillcolor=curcolor]
{
\newpath
\moveto(381.80592529,15.50530273)
\curveto(381.79591738,15.34530232)(381.75091743,15.21030246)(381.67092529,15.10030273)
\curveto(381.59091759,15.00030267)(381.49591768,14.92530274)(381.38592529,14.87530273)
\curveto(381.33591784,14.85530281)(381.2809179,14.84530282)(381.22092529,14.84530273)
\curveto(381.17091801,14.84530282)(381.11091807,14.83530283)(381.04092529,14.81530273)
\curveto(380.81091837,14.7653029)(380.59591858,14.78030289)(380.39592529,14.86030273)
\curveto(380.19591898,14.93030274)(380.07091911,15.02030265)(380.02092529,15.13030273)
\curveto(379.9809192,15.20030247)(379.95091923,15.28030239)(379.93092529,15.37030273)
\curveto(379.91091927,15.4703022)(379.8759193,15.55030212)(379.82592529,15.61030273)
\lineto(379.76592529,15.67030273)
\curveto(379.74591943,15.69030198)(379.71591946,15.69530197)(379.67592529,15.68530273)
\curveto(379.55591962,15.65530201)(379.44091974,15.60030207)(379.33092529,15.52030273)
\curveto(379.22091996,15.44030223)(379.11592006,15.3703023)(379.01592529,15.31030273)
\curveto(378.86592031,15.23030244)(378.71092047,15.15530251)(378.55092529,15.08530273)
\curveto(378.39092079,15.02530264)(378.22092096,14.9703027)(378.04092529,14.92030273)
\curveto(377.93092125,14.89030278)(377.81592136,14.8703028)(377.69592529,14.86030273)
\curveto(377.58592159,14.85030282)(377.47092171,14.83530283)(377.35092529,14.81530273)
\curveto(377.30092188,14.80530286)(377.25592192,14.80030287)(377.21592529,14.80030273)
\lineto(377.11092529,14.80030273)
\curveto(377.00092218,14.78030289)(376.89592228,14.78030289)(376.79592529,14.80030273)
\lineto(376.66092529,14.80030273)
\curveto(376.61092257,14.81030286)(376.56092262,14.81530285)(376.51092529,14.81530273)
\curveto(376.46092272,14.81530285)(376.42092276,14.82530284)(376.39092529,14.84530273)
\curveto(376.35092283,14.85530281)(376.31592286,14.86030281)(376.28592529,14.86030273)
\curveto(376.26592291,14.85030282)(376.24092294,14.85030282)(376.21092529,14.86030273)
\lineto(375.97092529,14.92030273)
\curveto(375.90092328,14.93030274)(375.83592334,14.95030272)(375.77592529,14.98030273)
\curveto(375.49592368,15.11030256)(375.2809239,15.25530241)(375.13092529,15.41530273)
\curveto(374.9809242,15.58530208)(374.8759243,15.82030185)(374.81592529,16.12030273)
\curveto(374.76592441,16.34030133)(374.77092441,16.60530106)(374.83092529,16.91530273)
\lineto(374.90592529,17.23030273)
\curveto(374.92592425,17.28030039)(374.94092424,17.33030034)(374.95092529,17.38030273)
\lineto(375.01092529,17.56030273)
\lineto(375.19092529,17.89030273)
\curveto(375.26092392,18.00029967)(375.33092385,18.10029957)(375.40092529,18.19030273)
\curveto(375.64092354,18.48029919)(375.93092325,18.69529897)(376.27092529,18.83530273)
\curveto(376.61092257,18.97529869)(376.9759222,19.10029857)(377.36592529,19.21030273)
\curveto(377.51592166,19.25029842)(377.66592151,19.28029839)(377.81592529,19.30030273)
\curveto(377.9759212,19.32029835)(378.13092105,19.34529832)(378.28092529,19.37530273)
\curveto(378.36092082,19.39529827)(378.43092075,19.40529826)(378.49092529,19.40530273)
\curveto(378.56092062,19.40529826)(378.63592054,19.41529825)(378.71592529,19.43530273)
\curveto(378.78592039,19.45529821)(378.85592032,19.4652982)(378.92592529,19.46530273)
\curveto(379.00592017,19.47529819)(379.08592009,19.49029818)(379.16592529,19.51030273)
\curveto(379.42591975,19.5702981)(379.67091951,19.62029805)(379.90092529,19.66030273)
\curveto(380.13091905,19.71029796)(380.33091885,19.82529784)(380.50092529,20.00530273)
\curveto(380.57091861,20.08529758)(380.63591854,20.18529748)(380.69592529,20.30530273)
\curveto(380.76591841,20.43529723)(380.79591838,20.57529709)(380.78592529,20.72530273)
\curveto(380.7759184,20.9652967)(380.72591845,21.15529651)(380.63592529,21.29530273)
\curveto(380.55591862,21.43529623)(380.41591876,21.54529612)(380.21592529,21.62530273)
\curveto(380.10591907,21.67529599)(379.97091921,21.71029596)(379.81092529,21.73030273)
\curveto(379.65091953,21.75029592)(379.4809197,21.76029591)(379.30092529,21.76030273)
\curveto(379.12092006,21.76029591)(378.94092024,21.75029592)(378.76092529,21.73030273)
\curveto(378.59092059,21.71029596)(378.44092074,21.68029599)(378.31092529,21.64030273)
\curveto(378.13092105,21.58029609)(377.95092123,21.49529617)(377.77092529,21.38530273)
\curveto(377.6809215,21.32529634)(377.59092159,21.24529642)(377.50092529,21.14530273)
\curveto(377.42092176,21.05529661)(377.34592183,20.95529671)(377.27592529,20.84530273)
\curveto(377.22592195,20.7652969)(377.180922,20.68029699)(377.14092529,20.59030273)
\curveto(377.10092208,20.50029717)(377.04092214,20.43029724)(376.96092529,20.38030273)
\curveto(376.91092227,20.35029732)(376.83592234,20.32529734)(376.73592529,20.30530273)
\curveto(376.63592254,20.29529737)(376.53592264,20.29029738)(376.43592529,20.29030273)
\curveto(376.33592284,20.29029738)(376.24092294,20.29529737)(376.15092529,20.30530273)
\curveto(376.06092312,20.32529734)(376.00092318,20.35029732)(375.97092529,20.38030273)
\curveto(375.93092325,20.41029726)(375.90592327,20.46029721)(375.89592529,20.53030273)
\curveto(375.89592328,20.60029707)(375.91592326,20.67529699)(375.95592529,20.75530273)
\curveto(376.00592317,20.88529678)(376.06092312,21.00529666)(376.12092529,21.11530273)
\curveto(376.180923,21.23529643)(376.24592293,21.35029632)(376.31592529,21.46030273)
\curveto(376.5759226,21.81029586)(376.87092231,22.08029559)(377.20092529,22.27030273)
\curveto(377.53092165,22.4702952)(377.92092126,22.63029504)(378.37092529,22.75030273)
\curveto(378.4809207,22.7702949)(378.58592059,22.78529488)(378.68592529,22.79530273)
\curveto(378.79592038,22.80529486)(378.90592027,22.82029485)(379.01592529,22.84030273)
\curveto(379.06592011,22.85029482)(379.13092005,22.85029482)(379.21092529,22.84030273)
\curveto(379.30091988,22.84029483)(379.36091982,22.85029482)(379.39092529,22.87030273)
\curveto(380.09091909,22.88029479)(380.6809185,22.80029487)(381.16092529,22.63030273)
\curveto(381.65091753,22.46029521)(381.95591722,22.13529553)(382.07592529,21.65530273)
\curveto(382.12591705,21.45529621)(382.13091705,21.22029645)(382.09092529,20.95030273)
\curveto(382.05091713,20.69029698)(382.00091718,20.41529725)(381.94092529,20.12530273)
\lineto(381.28092529,16.81030273)
\curveto(381.25091793,16.670301)(381.22591795,16.53530113)(381.20592529,16.40530273)
\curveto(381.19591798,16.27530139)(381.20591797,16.1703015)(381.23592529,16.09030273)
\curveto(381.2759179,16.02030165)(381.33091785,15.9703017)(381.40092529,15.94030273)
\curveto(381.49091769,15.90030177)(381.57091761,15.8703018)(381.64092529,15.85030273)
\curveto(381.72091746,15.84030183)(381.77091741,15.79530187)(381.79092529,15.71530273)
\curveto(381.81091737,15.68530198)(381.81591736,15.65530201)(381.80592529,15.62530273)
\lineto(381.80592529,15.50530273)
\moveto(379.99092529,17.17030273)
\curveto(380.0809191,17.31030036)(380.14591903,17.4703002)(380.18592529,17.65030273)
\curveto(380.22591895,17.84029983)(380.26591891,18.03529963)(380.30592529,18.23530273)
\curveto(380.32591885,18.34529932)(380.34091884,18.44529922)(380.35092529,18.53530273)
\curveto(380.36091882,18.62529904)(380.33591884,18.69529897)(380.27592529,18.74530273)
\curveto(380.24591893,18.7652989)(380.175919,18.77529889)(380.06592529,18.77530273)
\curveto(380.04591913,18.75529891)(380.01091917,18.74529892)(379.96092529,18.74530273)
\curveto(379.91091927,18.74529892)(379.86091932,18.73529893)(379.81092529,18.71530273)
\curveto(379.73091945,18.69529897)(379.63591954,18.67529899)(379.52592529,18.65530273)
\lineto(379.22592529,18.59530273)
\curveto(379.19591998,18.59529907)(379.16092002,18.59029908)(379.12092529,18.58030273)
\lineto(379.01592529,18.58030273)
\curveto(378.85592032,18.54029913)(378.68592049,18.51529915)(378.50592529,18.50530273)
\curveto(378.33592084,18.50529916)(378.17092101,18.48529918)(378.01092529,18.44530273)
\curveto(377.92092126,18.42529924)(377.84092134,18.40529926)(377.77092529,18.38530273)
\curveto(377.71092147,18.37529929)(377.63592154,18.36029931)(377.54592529,18.34030273)
\curveto(377.3759218,18.29029938)(377.21092197,18.22529944)(377.05092529,18.14530273)
\curveto(376.90092228,18.07529959)(376.76592241,17.98529968)(376.64592529,17.87530273)
\curveto(376.52592265,17.7652999)(376.42592275,17.63030004)(376.34592529,17.47030273)
\curveto(376.26592291,17.32030035)(376.20592297,17.13530053)(376.16592529,16.91530273)
\curveto(376.14592303,16.81530085)(376.14592303,16.72030095)(376.16592529,16.63030273)
\curveto(376.18592299,16.55030112)(376.21592296,16.47530119)(376.25592529,16.40530273)
\curveto(376.30592287,16.29530137)(376.38592279,16.20030147)(376.49592529,16.12030273)
\curveto(376.61592256,16.05030162)(376.74592243,15.99030168)(376.88592529,15.94030273)
\curveto(376.93592224,15.93030174)(376.98592219,15.92530174)(377.03592529,15.92530273)
\curveto(377.08592209,15.92530174)(377.13592204,15.92030175)(377.18592529,15.91030273)
\curveto(377.25592192,15.89030178)(377.34092184,15.87530179)(377.44092529,15.86530273)
\curveto(377.54092164,15.8653018)(377.63092155,15.87530179)(377.71092529,15.89530273)
\curveto(377.77092141,15.91530175)(377.83092135,15.92030175)(377.89092529,15.91030273)
\curveto(377.95092123,15.91030176)(378.01092117,15.92030175)(378.07092529,15.94030273)
\curveto(378.16092102,15.96030171)(378.24092094,15.97530169)(378.31092529,15.98530273)
\curveto(378.39092079,15.99530167)(378.47092071,16.01530165)(378.55092529,16.04530273)
\curveto(378.86092032,16.1653015)(379.13592004,16.31030136)(379.37592529,16.48030273)
\curveto(379.61591956,16.65030102)(379.82091936,16.88030079)(379.99092529,17.17030273)
}
}
{
\newrgbcolor{curcolor}{0 0 0}
\pscustom[linestyle=none,fillstyle=solid,fillcolor=curcolor]
{
\newpath
\moveto(387.58256592,22.85530273)
\curveto(388.32255954,22.8652948)(388.91755894,22.75529491)(389.36756592,22.52530273)
\curveto(389.81755804,22.30529536)(390.14255772,21.9702957)(390.34256592,21.52030273)
\curveto(390.43255743,21.32029635)(390.49255737,21.07529659)(390.52256592,20.78530273)
\curveto(390.53255733,20.73529693)(390.53255733,20.670297)(390.52256592,20.59030273)
\curveto(390.52255734,20.51029716)(390.50755735,20.44029723)(390.47756592,20.38030273)
\curveto(390.43755742,20.33029734)(390.37755748,20.28529738)(390.29756592,20.24530273)
\curveto(390.2575576,20.22529744)(390.22255764,20.21529745)(390.19256592,20.21530273)
\curveto(390.17255769,20.22529744)(390.13755772,20.22529744)(390.08756592,20.21530273)
\curveto(390.04755781,20.20529746)(390.00755785,20.20029747)(389.96756592,20.20030273)
\curveto(389.92755793,20.21029746)(389.88755797,20.21529745)(389.84756592,20.21530273)
\lineto(389.53256592,20.21530273)
\curveto(389.44255842,20.22529744)(389.36755849,20.25529741)(389.30756592,20.30530273)
\curveto(389.23755862,20.3652973)(389.19755866,20.45029722)(389.18756592,20.56030273)
\curveto(389.17755868,20.670297)(389.1575587,20.7652969)(389.12756592,20.84530273)
\curveto(389.02755883,21.10529656)(388.87255899,21.31029636)(388.66256592,21.46030273)
\curveto(388.59255927,21.51029616)(388.51255935,21.55029612)(388.42256592,21.58030273)
\curveto(388.34255952,21.62029605)(388.2575596,21.65529601)(388.16756592,21.68530273)
\curveto(388.03755982,21.72529594)(387.85756,21.74529592)(387.62756592,21.74530273)
\curveto(387.39756046,21.75529591)(387.20256066,21.73529593)(387.04256592,21.68530273)
\curveto(386.97256089,21.665296)(386.90256096,21.65029602)(386.83256592,21.64030273)
\curveto(386.77256109,21.63029604)(386.70756115,21.61529605)(386.63756592,21.59530273)
\curveto(386.3575615,21.48529618)(386.09756176,21.33529633)(385.85756592,21.14530273)
\curveto(385.61756224,20.95529671)(385.41756244,20.73029694)(385.25756592,20.47030273)
\curveto(385.19756266,20.38029729)(385.14256272,20.28529738)(385.09256592,20.18530273)
\curveto(385.04256282,20.09529757)(384.99256287,19.99529767)(384.94256592,19.88530273)
\lineto(384.77756592,19.48030273)
\curveto(384.7575631,19.43029824)(384.74256312,19.37529829)(384.73256592,19.31530273)
\curveto(384.72256314,19.25529841)(384.70256316,19.20029847)(384.67256592,19.15030273)
\lineto(384.65756592,19.03030273)
\curveto(384.63756322,18.99029868)(384.61256325,18.92529874)(384.58256592,18.83530273)
\curveto(384.5625633,18.74529892)(384.5575633,18.68029899)(384.56756592,18.64030273)
\curveto(384.56756329,18.59029908)(384.5575633,18.54029913)(384.53756592,18.49030273)
\curveto(384.51756334,18.44029923)(384.50756335,18.39029928)(384.50756592,18.34030273)
\curveto(384.51756334,18.30029937)(384.51256335,18.23029944)(384.49256592,18.13030273)
\curveto(384.49256337,18.05029962)(384.48756337,17.9652997)(384.47756592,17.87530273)
\curveto(384.47756338,17.78529988)(384.48256338,17.70029997)(384.49256592,17.62030273)
\curveto(384.53256333,17.30030037)(384.60256326,17.02030065)(384.70256592,16.78030273)
\curveto(384.80256306,16.55030112)(384.96756289,16.35030132)(385.19756592,16.18030273)
\curveto(385.27756258,16.13030154)(385.3575625,16.08530158)(385.43756592,16.04530273)
\curveto(385.52756233,16.00530166)(385.62256224,15.9653017)(385.72256592,15.92530273)
\curveto(385.77256209,15.91530175)(385.81256205,15.91030176)(385.84256592,15.91030273)
\curveto(385.87256199,15.91030176)(385.91256195,15.90530176)(385.96256592,15.89530273)
\curveto(385.99256187,15.88530178)(386.04256182,15.88030179)(386.11256592,15.88030273)
\lineto(386.27756592,15.88030273)
\curveto(386.26756159,15.8703018)(386.28256158,15.8653018)(386.32256592,15.86530273)
\curveto(386.35256151,15.87530179)(386.37756148,15.87530179)(386.39756592,15.86530273)
\curveto(386.42756143,15.8653018)(386.4625614,15.8703018)(386.50256592,15.88030273)
\curveto(386.57256129,15.90030177)(386.63756122,15.90530176)(386.69756592,15.89530273)
\curveto(386.76756109,15.89530177)(386.83756102,15.90530176)(386.90756592,15.92530273)
\curveto(387.18756067,16.00530166)(387.43256043,16.10530156)(387.64256592,16.22530273)
\curveto(387.86256,16.35530131)(388.0575598,16.52030115)(388.22756592,16.72030273)
\curveto(388.28755957,16.80030087)(388.34755951,16.88530078)(388.40756592,16.97530273)
\lineto(388.58756592,17.24530273)
\curveto(388.61755924,17.32530034)(388.65255921,17.40030027)(388.69256592,17.47030273)
\curveto(388.73255913,17.55030012)(388.79255907,17.61530005)(388.87256592,17.66530273)
\curveto(388.91255895,17.69529997)(388.97755888,17.71529995)(389.06756592,17.72530273)
\curveto(389.16755869,17.74529992)(389.26755859,17.75529991)(389.36756592,17.75530273)
\curveto(389.47755838,17.7652999)(389.57755828,17.7652999)(389.66756592,17.75530273)
\curveto(389.7575581,17.74529992)(389.82255804,17.72529994)(389.86256592,17.69530273)
\curveto(389.91255795,17.65530001)(389.93755792,17.59530007)(389.93756592,17.51530273)
\curveto(389.93755792,17.43530023)(389.91755794,17.35030032)(389.87756592,17.26030273)
\curveto(389.79755806,17.11030056)(389.72255814,16.9653007)(389.65256592,16.82530273)
\curveto(389.58255828,16.69530097)(389.49755836,16.5653011)(389.39756592,16.43530273)
\curveto(389.18755867,16.13530153)(388.94755891,15.8703018)(388.67756592,15.64030273)
\curveto(388.40755945,15.41030226)(388.09755976,15.22530244)(387.74756592,15.08530273)
\curveto(387.6575602,15.04530262)(387.5625603,15.01030266)(387.46256592,14.98030273)
\curveto(387.37256049,14.96030271)(387.27756058,14.93530273)(387.17756592,14.90530273)
\curveto(387.0575608,14.8653028)(386.94256092,14.84530282)(386.83256592,14.84530273)
\curveto(386.72256114,14.83530283)(386.60756125,14.82030285)(386.48756592,14.80030273)
\curveto(386.44756141,14.78030289)(386.40756145,14.77530289)(386.36756592,14.78530273)
\curveto(386.32756153,14.79530287)(386.28756157,14.79530287)(386.24756592,14.78530273)
\lineto(386.11256592,14.78530273)
\lineto(385.87256592,14.78530273)
\curveto(385.80256206,14.77530289)(385.73756212,14.78030289)(385.67756592,14.80030273)
\lineto(385.60256592,14.80030273)
\lineto(385.25756592,14.84530273)
\curveto(385.13756272,14.88530278)(385.01756284,14.92030275)(384.89756592,14.95030273)
\curveto(384.78756307,14.98030269)(384.68256318,15.02030265)(384.58256592,15.07030273)
\curveto(384.25256361,15.23030244)(383.99256387,15.42030225)(383.80256592,15.64030273)
\curveto(383.61256425,15.86030181)(383.44756441,16.13030154)(383.30756592,16.45030273)
\curveto(383.27756458,16.53030114)(383.25256461,16.62030105)(383.23256592,16.72030273)
\lineto(383.17256592,17.02030273)
\curveto(383.14256472,17.13030054)(383.12756473,17.24530042)(383.12756592,17.36530273)
\curveto(383.13756472,17.48530018)(383.13756472,17.60530006)(383.12756592,17.72530273)
\curveto(383.12756473,17.7652999)(383.13256473,17.80529986)(383.14256592,17.84530273)
\curveto(383.15256471,17.88529978)(383.15256471,17.92529974)(383.14256592,17.96530273)
\curveto(383.14256472,18.02529964)(383.14756471,18.09029958)(383.15756592,18.16030273)
\curveto(383.17756468,18.23029944)(383.18756467,18.29529937)(383.18756592,18.35530273)
\lineto(383.21756592,18.50530273)
\curveto(383.21756464,18.55529911)(383.22256464,18.62529904)(383.23256592,18.71530273)
\curveto(383.25256461,18.80529886)(383.27256459,18.87529879)(383.29256592,18.92530273)
\curveto(383.31256455,18.97529869)(383.32256454,19.02029865)(383.32256592,19.06030273)
\curveto(383.33256453,19.10029857)(383.34756451,19.14029853)(383.36756592,19.18030273)
\curveto(383.39756446,19.25029842)(383.41756444,19.32029835)(383.42756592,19.39030273)
\curveto(383.43756442,19.46029821)(383.4575644,19.52529814)(383.48756592,19.58530273)
\curveto(383.5575643,19.75529791)(383.62256424,19.92529774)(383.68256592,20.09530273)
\curveto(383.75256411,20.2652974)(383.83256403,20.42529724)(383.92256592,20.57530273)
\curveto(384.24256362,21.09529657)(384.58256328,21.51529615)(384.94256592,21.83530273)
\curveto(385.30256256,22.15529551)(385.76756209,22.42029525)(386.33756592,22.63030273)
\curveto(386.4575614,22.68029499)(386.58256128,22.71529495)(386.71256592,22.73530273)
\curveto(386.84256102,22.75529491)(386.98256088,22.78029489)(387.13256592,22.81030273)
\curveto(387.20256066,22.82029485)(387.27256059,22.82529484)(387.34256592,22.82530273)
\curveto(387.41256045,22.83529483)(387.49256037,22.84529482)(387.58256592,22.85530273)
}
}
{
\newrgbcolor{curcolor}{0 0 0}
\pscustom[linestyle=none,fillstyle=solid,fillcolor=curcolor]
{
\newpath
\moveto(393.04420654,24.17530273)
\curveto(392.97420357,24.23529343)(392.95420359,24.34029333)(392.98420654,24.49030273)
\curveto(393.01420353,24.65029302)(393.0442035,24.80529286)(393.07420654,24.95530273)
\curveto(393.08420346,25.03529263)(393.09920344,25.12029255)(393.11920654,25.21030273)
\curveto(393.1392034,25.30029237)(393.16920337,25.37529229)(393.20920654,25.43530273)
\curveto(393.26920327,25.51529215)(393.35920318,25.57529209)(393.47920654,25.61530273)
\curveto(393.50920303,25.62529204)(393.53420301,25.62529204)(393.55420654,25.61530273)
\curveto(393.57420297,25.61529205)(393.59920294,25.62029205)(393.62920654,25.63030273)
\curveto(393.79920274,25.63029204)(393.95420259,25.62529204)(394.09420654,25.61530273)
\curveto(394.2442023,25.60529206)(394.33420221,25.54529212)(394.36420654,25.43530273)
\curveto(394.38420216,25.37529229)(394.38420216,25.30029237)(394.36420654,25.21030273)
\curveto(394.3442022,25.13029254)(394.32920221,25.04529262)(394.31920654,24.95530273)
\curveto(394.27920226,24.77529289)(394.2392023,24.60529306)(394.19920654,24.44530273)
\curveto(394.16920237,24.28529338)(394.08420246,24.18029349)(393.94420654,24.13030273)
\curveto(393.88420266,24.11029356)(393.82420272,24.10029357)(393.76420654,24.10030273)
\lineto(393.59920654,24.10030273)
\lineto(393.28420654,24.10030273)
\curveto(393.18420336,24.10029357)(393.10420344,24.12529354)(393.04420654,24.17530273)
\moveto(392.45920654,15.67030273)
\curveto(392.4392041,15.5703021)(392.41920412,15.4653022)(392.39920654,15.35530273)
\curveto(392.38920415,15.25530241)(392.34920419,15.17530249)(392.27920654,15.11530273)
\curveto(392.2392043,15.05530261)(392.18920435,15.01530265)(392.12920654,14.99530273)
\curveto(392.06920447,14.98530268)(391.99420455,14.9703027)(391.90420654,14.95030273)
\lineto(391.67920654,14.95030273)
\curveto(391.54920499,14.95030272)(391.4392051,14.95530271)(391.34920654,14.96530273)
\curveto(391.25920528,14.98530268)(391.19420535,15.03530263)(391.15420654,15.11530273)
\curveto(391.13420541,15.17530249)(391.12920541,15.25030242)(391.13920654,15.34030273)
\curveto(391.15920538,15.44030223)(391.17920536,15.53530213)(391.19920654,15.62530273)
\lineto(392.47420654,21.97030273)
\curveto(392.49420405,22.08029559)(392.51420403,22.18529548)(392.53420654,22.28530273)
\curveto(392.55420399,22.39529527)(392.59420395,22.48029519)(392.65420654,22.54030273)
\curveto(392.69420385,22.59029508)(392.7392038,22.62029505)(392.78920654,22.63030273)
\curveto(392.84920369,22.64029503)(392.90920363,22.65529501)(392.96920654,22.67530273)
\curveto(392.98920355,22.67529499)(393.00920353,22.670295)(393.02920654,22.66030273)
\curveto(393.05920348,22.66029501)(393.08420346,22.665295)(393.10420654,22.67530273)
\curveto(393.23420331,22.67529499)(393.36420318,22.670295)(393.49420654,22.66030273)
\curveto(393.63420291,22.66029501)(393.71920282,22.62029505)(393.74920654,22.54030273)
\curveto(393.78920275,22.48029519)(393.79920274,22.40029527)(393.77920654,22.30030273)
\curveto(393.75920278,22.21029546)(393.7392028,22.11529555)(393.71920654,22.01530273)
\lineto(392.45920654,15.67030273)
}
}
{
\newrgbcolor{curcolor}{0 0 0}
\pscustom[linestyle=none,fillstyle=solid,fillcolor=curcolor]
{
\newpath
\moveto(402.21905029,19.15030273)
\curveto(402.2290414,19.09029858)(402.21904141,18.99529867)(402.18905029,18.86530273)
\curveto(402.16904146,18.74529892)(402.14904148,18.66029901)(402.12905029,18.61030273)
\lineto(402.09905029,18.46030273)
\curveto(402.06904156,18.38029929)(402.04404159,18.30529936)(402.02405029,18.23530273)
\curveto(402.01404162,18.17529949)(401.99404164,18.10529956)(401.96405029,18.02530273)
\curveto(401.9340417,17.9652997)(401.90904172,17.90529976)(401.88905029,17.84530273)
\curveto(401.87904175,17.78529988)(401.85404178,17.72529994)(401.81405029,17.66530273)
\lineto(401.63405029,17.27530273)
\curveto(401.58404205,17.14530052)(401.51904211,17.02530064)(401.43905029,16.91530273)
\curveto(401.13904249,16.43530123)(400.77904285,16.03030164)(400.35905029,15.70030273)
\curveto(399.94904368,15.38030229)(399.46904416,15.13530253)(398.91905029,14.96530273)
\curveto(398.80904482,14.92530274)(398.68904494,14.89530277)(398.55905029,14.87530273)
\curveto(398.4290452,14.85530281)(398.29404534,14.83530283)(398.15405029,14.81530273)
\curveto(398.09404554,14.80530286)(398.0290456,14.80030287)(397.95905029,14.80030273)
\curveto(397.89904573,14.79030288)(397.83904579,14.78530288)(397.77905029,14.78530273)
\curveto(397.73904589,14.77530289)(397.67904595,14.7703029)(397.59905029,14.77030273)
\curveto(397.5290461,14.7703029)(397.47904615,14.77530289)(397.44905029,14.78530273)
\curveto(397.40904622,14.79530287)(397.36904626,14.80030287)(397.32905029,14.80030273)
\curveto(397.28904634,14.79030288)(397.25404638,14.79030288)(397.22405029,14.80030273)
\lineto(397.13405029,14.80030273)
\lineto(396.78905029,14.84530273)
\lineto(396.39905029,14.96530273)
\curveto(396.27904735,15.00530266)(396.16404747,15.05030262)(396.05405029,15.10030273)
\curveto(395.64404799,15.30030237)(395.32404831,15.56030211)(395.09405029,15.88030273)
\curveto(394.87404876,16.20030147)(394.71404892,16.59030108)(394.61405029,17.05030273)
\curveto(394.58404905,17.15030052)(394.56404907,17.25030042)(394.55405029,17.35030273)
\lineto(394.55405029,17.66530273)
\curveto(394.54404909,17.70529996)(394.54404909,17.73529993)(394.55405029,17.75530273)
\curveto(394.56404907,17.78529988)(394.56904906,17.82029985)(394.56905029,17.86030273)
\curveto(394.56904906,17.94029973)(394.57404906,18.02029965)(394.58405029,18.10030273)
\curveto(394.59404904,18.19029948)(394.59904903,18.27529939)(394.59905029,18.35530273)
\curveto(394.60904902,18.40529926)(394.61404902,18.44529922)(394.61405029,18.47530273)
\curveto(394.62404901,18.51529915)(394.629049,18.56029911)(394.62905029,18.61030273)
\curveto(394.629049,18.66029901)(394.63904899,18.74529892)(394.65905029,18.86530273)
\curveto(394.68904894,18.99529867)(394.71904891,19.09029858)(394.74905029,19.15030273)
\curveto(394.78904884,19.22029845)(394.80904882,19.29029838)(394.80905029,19.36030273)
\curveto(394.80904882,19.43029824)(394.8290488,19.50029817)(394.86905029,19.57030273)
\curveto(394.88904874,19.62029805)(394.90404873,19.66029801)(394.91405029,19.69030273)
\curveto(394.92404871,19.73029794)(394.93904869,19.77529789)(394.95905029,19.82530273)
\curveto(395.01904861,19.94529772)(395.06904856,20.0652976)(395.10905029,20.18530273)
\curveto(395.15904847,20.30529736)(395.22404841,20.42029725)(395.30405029,20.53030273)
\curveto(395.52404811,20.90029677)(395.76904786,21.23029644)(396.03905029,21.52030273)
\curveto(396.31904731,21.82029585)(396.634047,22.0702956)(396.98405029,22.27030273)
\curveto(397.11404652,22.35029532)(397.24904638,22.41529525)(397.38905029,22.46530273)
\lineto(397.83905029,22.64530273)
\curveto(397.96904566,22.69529497)(398.10404553,22.72529494)(398.24405029,22.73530273)
\curveto(398.38404525,22.75529491)(398.5290451,22.78529488)(398.67905029,22.82530273)
\lineto(398.87405029,22.82530273)
\lineto(399.08405029,22.85530273)
\curveto(399.97404366,22.8652948)(400.67404296,22.68029499)(401.18405029,22.30030273)
\curveto(401.70404193,21.92029575)(402.0290416,21.42529624)(402.15905029,20.81530273)
\curveto(402.18904144,20.71529695)(402.20904142,20.61529705)(402.21905029,20.51530273)
\curveto(402.2290414,20.41529725)(402.24404139,20.31029736)(402.26405029,20.20030273)
\curveto(402.27404136,20.09029758)(402.27404136,19.9702977)(402.26405029,19.84030273)
\lineto(402.26405029,19.46530273)
\curveto(402.26404137,19.41529825)(402.25404138,19.36029831)(402.23405029,19.30030273)
\curveto(402.22404141,19.25029842)(402.21904141,19.20029847)(402.21905029,19.15030273)
\moveto(400.71905029,18.29530273)
\curveto(400.74904288,18.3652993)(400.76904286,18.44529922)(400.77905029,18.53530273)
\curveto(400.79904283,18.62529904)(400.81404282,18.71029896)(400.82405029,18.79030273)
\curveto(400.90404273,19.18029849)(400.93904269,19.51029816)(400.92905029,19.78030273)
\curveto(400.90904272,19.86029781)(400.89404274,19.94029773)(400.88405029,20.02030273)
\curveto(400.88404275,20.10029757)(400.87904275,20.17529749)(400.86905029,20.24530273)
\curveto(400.71904291,20.89529677)(400.36404327,21.34529632)(399.80405029,21.59530273)
\curveto(399.7340439,21.62529604)(399.65904397,21.64529602)(399.57905029,21.65530273)
\curveto(399.50904412,21.67529599)(399.4340442,21.69529597)(399.35405029,21.71530273)
\curveto(399.28404435,21.73529593)(399.20404443,21.74529592)(399.11405029,21.74530273)
\lineto(398.84405029,21.74530273)
\lineto(398.55905029,21.70030273)
\curveto(398.45904517,21.68029599)(398.36404527,21.65529601)(398.27405029,21.62530273)
\curveto(398.18404545,21.60529606)(398.09404554,21.57529609)(398.00405029,21.53530273)
\curveto(397.9340457,21.51529615)(397.86404577,21.48529618)(397.79405029,21.44530273)
\curveto(397.72404591,21.40529626)(397.65904597,21.3652963)(397.59905029,21.32530273)
\curveto(397.3290463,21.15529651)(397.09404654,20.95029672)(396.89405029,20.71030273)
\curveto(396.69404694,20.4702972)(396.50904712,20.19029748)(396.33905029,19.87030273)
\curveto(396.28904734,19.7702979)(396.24904738,19.665298)(396.21905029,19.55530273)
\curveto(396.18904744,19.45529821)(396.14904748,19.35029832)(396.09905029,19.24030273)
\curveto(396.08904754,19.20029847)(396.07404756,19.13529853)(396.05405029,19.04530273)
\curveto(396.0340476,19.01529865)(396.02404761,18.98029869)(396.02405029,18.94030273)
\curveto(396.02404761,18.90029877)(396.01904761,18.85529881)(396.00905029,18.80530273)
\lineto(395.94905029,18.50530273)
\curveto(395.9290477,18.40529926)(395.91904771,18.31529935)(395.91905029,18.23530273)
\lineto(395.91905029,18.05530273)
\curveto(395.91904771,17.95529971)(395.91404772,17.85529981)(395.90405029,17.75530273)
\curveto(395.90404773,17.6653)(395.91404772,17.58030009)(395.93405029,17.50030273)
\curveto(395.98404765,17.26030041)(396.05404758,17.03530063)(396.14405029,16.82530273)
\curveto(396.24404739,16.61530105)(396.37904725,16.44030123)(396.54905029,16.30030273)
\curveto(396.59904703,16.2703014)(396.63904699,16.24530142)(396.66905029,16.22530273)
\curveto(396.70904692,16.20530146)(396.74904688,16.18030149)(396.78905029,16.15030273)
\curveto(396.85904677,16.10030157)(396.93904669,16.05530161)(397.02905029,16.01530273)
\curveto(397.11904651,15.98530168)(397.21404642,15.95530171)(397.31405029,15.92530273)
\curveto(397.36404627,15.90530176)(397.40904622,15.89530177)(397.44905029,15.89530273)
\curveto(397.49904613,15.90530176)(397.54904608,15.90530176)(397.59905029,15.89530273)
\curveto(397.629046,15.88530178)(397.68904594,15.87530179)(397.77905029,15.86530273)
\curveto(397.86904576,15.85530181)(397.94404569,15.86030181)(398.00405029,15.88030273)
\curveto(398.04404559,15.89030178)(398.08404555,15.89030178)(398.12405029,15.88030273)
\curveto(398.16404547,15.88030179)(398.20404543,15.89030178)(398.24405029,15.91030273)
\curveto(398.32404531,15.93030174)(398.40404523,15.94530172)(398.48405029,15.95530273)
\curveto(398.57404506,15.97530169)(398.65904497,16.00030167)(398.73905029,16.03030273)
\curveto(399.09904453,16.1703015)(399.40904422,16.3653013)(399.66905029,16.61530273)
\curveto(399.9290437,16.8653008)(400.16404347,17.16030051)(400.37405029,17.50030273)
\curveto(400.45404318,17.62030005)(400.51404312,17.74529992)(400.55405029,17.87530273)
\curveto(400.59404304,18.01529965)(400.64904298,18.15529951)(400.71905029,18.29530273)
}
}
{
\newrgbcolor{curcolor}{0 0 0}
\pscustom[linestyle=none,fillstyle=solid,fillcolor=curcolor]
{
\newpath
\moveto(406.88733154,22.85530273)
\curveto(407.60732589,22.8652948)(408.1923253,22.78029489)(408.64233154,22.60030273)
\curveto(409.10232439,22.43029524)(409.42232407,22.12529554)(409.60233154,21.68530273)
\curveto(409.65232384,21.57529609)(409.68232381,21.46029621)(409.69233154,21.34030273)
\curveto(409.71232378,21.23029644)(409.72732377,21.10529656)(409.73733154,20.96530273)
\curveto(409.74732375,20.89529677)(409.73732376,20.82029685)(409.70733154,20.74030273)
\curveto(409.68732381,20.670297)(409.66232383,20.61529705)(409.63233154,20.57530273)
\curveto(409.61232388,20.55529711)(409.58232391,20.53529713)(409.54233154,20.51530273)
\curveto(409.51232398,20.50529716)(409.48732401,20.49029718)(409.46733154,20.47030273)
\curveto(409.40732409,20.45029722)(409.35232414,20.44529722)(409.30233154,20.45530273)
\curveto(409.26232423,20.4652972)(409.21732428,20.4652972)(409.16733154,20.45530273)
\curveto(409.07732442,20.43529723)(408.96732453,20.43029724)(408.83733154,20.44030273)
\curveto(408.71732478,20.46029721)(408.63232486,20.48529718)(408.58233154,20.51530273)
\curveto(408.51232498,20.5652971)(408.47232502,20.63029704)(408.46233154,20.71030273)
\curveto(408.46232503,20.80029687)(408.44232505,20.88529678)(408.40233154,20.96530273)
\curveto(408.35232514,21.12529654)(408.25732524,21.2702964)(408.11733154,21.40030273)
\curveto(408.02732547,21.48029619)(407.91732558,21.54029613)(407.78733154,21.58030273)
\curveto(407.66732583,21.62029605)(407.53732596,21.66029601)(407.39733154,21.70030273)
\curveto(407.35732614,21.72029595)(407.30732619,21.72529594)(407.24733154,21.71530273)
\curveto(407.1973263,21.71529595)(407.15232634,21.72029595)(407.11233154,21.73030273)
\curveto(407.05232644,21.75029592)(406.97732652,21.76029591)(406.88733154,21.76030273)
\curveto(406.7973267,21.76029591)(406.72232677,21.75029592)(406.66233154,21.73030273)
\lineto(406.57233154,21.73030273)
\curveto(406.51232698,21.72029595)(406.45732704,21.71029596)(406.40733154,21.70030273)
\curveto(406.35732714,21.70029597)(406.30732719,21.69529597)(406.25733154,21.68530273)
\curveto(405.98732751,21.62529604)(405.75232774,21.54029613)(405.55233154,21.43030273)
\curveto(405.36232813,21.32029635)(405.21232828,21.13529653)(405.10233154,20.87530273)
\curveto(405.07232842,20.80529686)(405.05732844,20.73529693)(405.05733154,20.66530273)
\curveto(405.05732844,20.59529707)(405.06232843,20.53529713)(405.07233154,20.48530273)
\curveto(405.10232839,20.33529733)(405.15232834,20.22529744)(405.22233154,20.15530273)
\curveto(405.2923282,20.09529757)(405.38732811,20.02529764)(405.50733154,19.94530273)
\curveto(405.64732785,19.84529782)(405.81232768,19.7702979)(406.00233154,19.72030273)
\curveto(406.1923273,19.68029799)(406.38232711,19.63029804)(406.57233154,19.57030273)
\curveto(406.6923268,19.53029814)(406.81232668,19.50029817)(406.93233154,19.48030273)
\curveto(407.06232643,19.46029821)(407.18732631,19.43029824)(407.30733154,19.39030273)
\curveto(407.50732599,19.33029834)(407.70232579,19.2702984)(407.89233154,19.21030273)
\curveto(408.08232541,19.16029851)(408.26732523,19.09529857)(408.44733154,19.01530273)
\curveto(408.497325,18.99529867)(408.54232495,18.97529869)(408.58233154,18.95530273)
\curveto(408.63232486,18.93529873)(408.68232481,18.91029876)(408.73233154,18.88030273)
\curveto(408.90232459,18.76029891)(409.04732445,18.62529904)(409.16733154,18.47530273)
\curveto(409.28732421,18.32529934)(409.37732412,18.13529953)(409.43733154,17.90530273)
\lineto(409.43733154,17.62030273)
\curveto(409.43732406,17.55030012)(409.43232406,17.47530019)(409.42233154,17.39530273)
\curveto(409.41232408,17.32530034)(409.40232409,17.24530042)(409.39233154,17.15530273)
\lineto(409.36233154,17.00530273)
\curveto(409.32232417,16.93530073)(409.2923242,16.8653008)(409.27233154,16.79530273)
\curveto(409.26232423,16.72530094)(409.24232425,16.65530101)(409.21233154,16.58530273)
\curveto(409.16232433,16.47530119)(409.10732439,16.3703013)(409.04733154,16.27030273)
\curveto(408.98732451,16.1703015)(408.92232457,16.08030159)(408.85233154,16.00030273)
\curveto(408.64232485,15.74030193)(408.3973251,15.53030214)(408.11733154,15.37030273)
\curveto(407.83732566,15.22030245)(407.53232596,15.09030258)(407.20233154,14.98030273)
\curveto(407.10232639,14.95030272)(407.00232649,14.93030274)(406.90233154,14.92030273)
\curveto(406.80232669,14.90030277)(406.70732679,14.87530279)(406.61733154,14.84530273)
\curveto(406.50732699,14.82530284)(406.40232709,14.81530285)(406.30233154,14.81530273)
\curveto(406.20232729,14.81530285)(406.10232739,14.80530286)(406.00233154,14.78530273)
\lineto(405.85233154,14.78530273)
\curveto(405.80232769,14.77530289)(405.73232776,14.7703029)(405.64233154,14.77030273)
\curveto(405.55232794,14.7703029)(405.48232801,14.77530289)(405.43233154,14.78530273)
\lineto(405.26733154,14.78530273)
\curveto(405.20732829,14.80530286)(405.14232835,14.81530285)(405.07233154,14.81530273)
\curveto(405.00232849,14.80530286)(404.94732855,14.81030286)(404.90733154,14.83030273)
\curveto(404.85732864,14.84030283)(404.7923287,14.84530282)(404.71233154,14.84530273)
\curveto(404.63232886,14.8653028)(404.55732894,14.88530278)(404.48733154,14.90530273)
\curveto(404.41732908,14.91530275)(404.34232915,14.93530273)(404.26233154,14.96530273)
\curveto(403.97232952,15.0653026)(403.72732977,15.19030248)(403.52733154,15.34030273)
\curveto(403.32733017,15.49030218)(403.16733033,15.68530198)(403.04733154,15.92530273)
\curveto(402.98733051,16.05530161)(402.93733056,16.19030148)(402.89733154,16.33030273)
\curveto(402.86733063,16.4703012)(402.84733065,16.62530104)(402.83733154,16.79530273)
\curveto(402.82733067,16.85530081)(402.83233066,16.92530074)(402.85233154,17.00530273)
\curveto(402.87233062,17.09530057)(402.8973306,17.1653005)(402.92733154,17.21530273)
\curveto(402.96733053,17.25530041)(403.02733047,17.29530037)(403.10733154,17.33530273)
\curveto(403.15733034,17.35530031)(403.22733027,17.3653003)(403.31733154,17.36530273)
\curveto(403.41733008,17.37530029)(403.50732999,17.37530029)(403.58733154,17.36530273)
\curveto(403.67732982,17.35530031)(403.76232973,17.34030033)(403.84233154,17.32030273)
\curveto(403.93232956,17.31030036)(403.98732951,17.29530037)(404.00733154,17.27530273)
\curveto(404.06732943,17.22530044)(404.0973294,17.15030052)(404.09733154,17.05030273)
\curveto(404.10732939,16.96030071)(404.12732937,16.87530079)(404.15733154,16.79530273)
\curveto(404.20732929,16.57530109)(404.30732919,16.40530126)(404.45733154,16.28530273)
\curveto(404.55732894,16.19530147)(404.67732882,16.12530154)(404.81733154,16.07530273)
\curveto(404.95732854,16.02530164)(405.10732839,15.97530169)(405.26733154,15.92530273)
\lineto(405.58233154,15.88030273)
\lineto(405.67233154,15.88030273)
\curveto(405.73232776,15.86030181)(405.81732768,15.85030182)(405.92733154,15.85030273)
\curveto(406.04732745,15.85030182)(406.15232734,15.86030181)(406.24233154,15.88030273)
\curveto(406.31232718,15.88030179)(406.36732713,15.88530178)(406.40733154,15.89530273)
\curveto(406.46732703,15.90530176)(406.52732697,15.91030176)(406.58733154,15.91030273)
\curveto(406.64732685,15.92030175)(406.70232679,15.93030174)(406.75233154,15.94030273)
\curveto(407.06232643,16.02030165)(407.31232618,16.12530154)(407.50233154,16.25530273)
\curveto(407.70232579,16.38530128)(407.86732563,16.60530106)(407.99733154,16.91530273)
\curveto(408.02732547,16.9653007)(408.04232545,17.02030065)(408.04233154,17.08030273)
\curveto(408.05232544,17.14030053)(408.05232544,17.18530048)(408.04233154,17.21530273)
\curveto(408.03232546,17.40530026)(407.9923255,17.54530012)(407.92233154,17.63530273)
\curveto(407.85232564,17.73529993)(407.75732574,17.82529984)(407.63733154,17.90530273)
\curveto(407.55732594,17.9652997)(407.46232603,18.01529965)(407.35233154,18.05530273)
\lineto(407.05233154,18.17530273)
\curveto(407.02232647,18.18529948)(406.9923265,18.19029948)(406.96233154,18.19030273)
\curveto(406.94232655,18.19029948)(406.92232657,18.20029947)(406.90233154,18.22030273)
\curveto(406.58232691,18.33029934)(406.24232725,18.41029926)(405.88233154,18.46030273)
\curveto(405.53232796,18.52029915)(405.21232828,18.61529905)(404.92233154,18.74530273)
\curveto(404.83232866,18.78529888)(404.74232875,18.82029885)(404.65233154,18.85030273)
\curveto(404.57232892,18.88029879)(404.497329,18.92029875)(404.42733154,18.97030273)
\curveto(404.25732924,19.08029859)(404.10732939,19.20529846)(403.97733154,19.34530273)
\curveto(403.84732965,19.48529818)(403.75732974,19.66029801)(403.70733154,19.87030273)
\curveto(403.68732981,19.94029773)(403.67732982,20.01029766)(403.67733154,20.08030273)
\lineto(403.67733154,20.30530273)
\curveto(403.66732983,20.42529724)(403.68232981,20.56029711)(403.72233154,20.71030273)
\curveto(403.76232973,20.8702968)(403.80232969,21.00529666)(403.84233154,21.11530273)
\curveto(403.87232962,21.1652965)(403.8923296,21.20529646)(403.90233154,21.23530273)
\curveto(403.92232957,21.27529639)(403.94732955,21.31529635)(403.97733154,21.35530273)
\curveto(404.10732939,21.58529608)(404.26732923,21.78529588)(404.45733154,21.95530273)
\curveto(404.64732885,22.12529554)(404.85732864,22.27529539)(405.08733154,22.40530273)
\curveto(405.24732825,22.49529517)(405.42232807,22.5652951)(405.61233154,22.61530273)
\curveto(405.81232768,22.67529499)(406.01732748,22.73029494)(406.22733154,22.78030273)
\curveto(406.2973272,22.79029488)(406.36232713,22.80029487)(406.42233154,22.81030273)
\curveto(406.492327,22.82029485)(406.56732693,22.83029484)(406.64733154,22.84030273)
\curveto(406.68732681,22.85029482)(406.72732677,22.85029482)(406.76733154,22.84030273)
\curveto(406.81732668,22.83029484)(406.85732664,22.83529483)(406.88733154,22.85530273)
}
}
{
\newrgbcolor{curcolor}{0 0 0}
\pscustom[linestyle=none,fillstyle=solid,fillcolor=curcolor]
{
}
}
{
\newrgbcolor{curcolor}{0 0 0}
\pscustom[linestyle=none,fillstyle=solid,fillcolor=curcolor]
{
\newpath
\moveto(421.65248779,15.76030273)
\lineto(421.56248779,15.37030273)
\curveto(421.54247986,15.25030242)(421.5024799,15.15030252)(421.44248779,15.07030273)
\curveto(421.37248003,15.00030267)(421.27748013,14.96030271)(421.15748779,14.95030273)
\lineto(420.81248779,14.95030273)
\curveto(420.75248065,14.95030272)(420.69248071,14.94530272)(420.63248779,14.93530273)
\curveto(420.58248082,14.93530273)(420.53748087,14.94530272)(420.49748779,14.96530273)
\curveto(420.41748099,14.98530268)(420.36748104,15.02530264)(420.34748779,15.08530273)
\curveto(420.31748109,15.13530253)(420.3074811,15.19530247)(420.31748779,15.26530273)
\curveto(420.32748108,15.33530233)(420.32248108,15.40530226)(420.30248779,15.47530273)
\curveto(420.3024811,15.49530217)(420.29248111,15.51030216)(420.27248779,15.52030273)
\lineto(420.24248779,15.58030273)
\curveto(420.14248126,15.59030208)(420.05748135,15.5703021)(419.98748779,15.52030273)
\curveto(419.92748148,15.4703022)(419.86248154,15.42030225)(419.79248779,15.37030273)
\curveto(419.56248184,15.22030245)(419.33748207,15.10530256)(419.11748779,15.02530273)
\curveto(418.92748248,14.94530272)(418.7074827,14.88530278)(418.45748779,14.84530273)
\curveto(418.21748319,14.80530286)(417.97248343,14.78530288)(417.72248779,14.78530273)
\curveto(417.48248392,14.77530289)(417.24248416,14.79030288)(417.00248779,14.83030273)
\curveto(416.77248463,14.86030281)(416.57748483,14.91530275)(416.41748779,14.99530273)
\curveto(415.93748547,15.21530245)(415.57248583,15.51030216)(415.32248779,15.88030273)
\curveto(415.08248632,16.26030141)(414.92748648,16.73030094)(414.85748779,17.29030273)
\curveto(414.83748657,17.38030029)(414.82748658,17.4703002)(414.82748779,17.56030273)
\curveto(414.83748657,17.66030001)(414.83748657,17.76029991)(414.82748779,17.86030273)
\curveto(414.82748658,17.91029976)(414.83248657,17.96029971)(414.84248779,18.01030273)
\curveto(414.85248655,18.06029961)(414.85748655,18.11029956)(414.85748779,18.16030273)
\curveto(414.84748656,18.21029946)(414.84748656,18.26029941)(414.85748779,18.31030273)
\curveto(414.87748653,18.3702993)(414.88748652,18.42529924)(414.88748779,18.47530273)
\lineto(414.91748779,18.62530273)
\curveto(414.9074865,18.67529899)(414.9074865,18.74029893)(414.91748779,18.82030273)
\curveto(414.93748647,18.90029877)(414.96248644,18.9652987)(414.99248779,19.01530273)
\lineto(415.03748779,19.18030273)
\curveto(415.06748634,19.25029842)(415.08748632,19.32029835)(415.09748779,19.39030273)
\curveto(415.1074863,19.4702982)(415.12748628,19.54529812)(415.15748779,19.61530273)
\curveto(415.17748623,19.665298)(415.19248621,19.71029796)(415.20248779,19.75030273)
\curveto(415.21248619,19.79029788)(415.22748618,19.83529783)(415.24748779,19.88530273)
\curveto(415.29748611,19.98529768)(415.34248606,20.08029759)(415.38248779,20.17030273)
\curveto(415.42248598,20.2702974)(415.46748594,20.3652973)(415.51748779,20.45530273)
\curveto(415.71748569,20.83529683)(415.94748546,21.17529649)(416.20748779,21.47530273)
\curveto(416.47748493,21.78529588)(416.77748463,22.04029563)(417.10748779,22.24030273)
\curveto(417.3074841,22.36029531)(417.5074839,22.46029521)(417.70748779,22.54030273)
\curveto(417.9074835,22.62029505)(418.12248328,22.69029498)(418.35248779,22.75030273)
\lineto(418.56248779,22.78030273)
\curveto(418.63248277,22.79029488)(418.7024827,22.80529486)(418.77248779,22.82530273)
\lineto(418.92248779,22.82530273)
\curveto(419.01248239,22.84529482)(419.13248227,22.85529481)(419.28248779,22.85530273)
\curveto(419.44248196,22.85529481)(419.55748185,22.84529482)(419.62748779,22.82530273)
\curveto(419.66748174,22.81529485)(419.72248168,22.81029486)(419.79248779,22.81030273)
\curveto(419.89248151,22.78029489)(419.99748141,22.75529491)(420.10748779,22.73530273)
\curveto(420.21748119,22.72529494)(420.31748109,22.69529497)(420.40748779,22.64530273)
\curveto(420.54748086,22.58529508)(420.67748073,22.52029515)(420.79748779,22.45030273)
\curveto(420.91748049,22.38029529)(421.02748038,22.30029537)(421.12748779,22.21030273)
\curveto(421.17748023,22.16029551)(421.22748018,22.10529556)(421.27748779,22.04530273)
\curveto(421.33748007,21.99529567)(421.42247998,21.98029569)(421.53248779,22.00030273)
\lineto(421.60748779,22.07530273)
\curveto(421.62747978,22.09529557)(421.64247976,22.12529554)(421.65248779,22.16530273)
\curveto(421.7024797,22.25529541)(421.73747967,22.3702953)(421.75748779,22.51030273)
\curveto(421.78747962,22.65029502)(421.81247959,22.77529489)(421.83248779,22.88530273)
\lineto(422.17748779,24.61030273)
\curveto(422.2074792,24.75029292)(422.23747917,24.90529276)(422.26748779,25.07530273)
\curveto(422.3074791,25.25529241)(422.35747905,25.38529228)(422.41748779,25.46530273)
\curveto(422.47747893,25.53529213)(422.54747886,25.58029209)(422.62748779,25.60030273)
\curveto(422.64747876,25.60029207)(422.67247873,25.60029207)(422.70248779,25.60030273)
\curveto(422.73247867,25.61029206)(422.75747865,25.61529205)(422.77748779,25.61530273)
\curveto(422.92747848,25.62529204)(423.07747833,25.62529204)(423.22748779,25.61530273)
\curveto(423.37747803,25.61529205)(423.47747793,25.57529209)(423.52748779,25.49530273)
\curveto(423.55747785,25.41529225)(423.55747785,25.31529235)(423.52748779,25.19530273)
\curveto(423.5074779,25.07529259)(423.48747792,24.95029272)(423.46748779,24.82030273)
\lineto(421.65248779,15.76030273)
\moveto(421.00748779,18.59530273)
\curveto(421.03748037,18.64529902)(421.05748035,18.71029896)(421.06748779,18.79030273)
\curveto(421.08748032,18.88029879)(421.09248031,18.95029872)(421.08248779,19.00030273)
\lineto(421.12748779,19.22530273)
\curveto(421.12748028,19.31529835)(421.13248027,19.40529826)(421.14248779,19.49530273)
\curveto(421.15248025,19.59529807)(421.14748026,19.68529798)(421.12748779,19.76530273)
\lineto(421.12748779,19.99030273)
\curveto(421.12748028,20.06029761)(421.11748029,20.13029754)(421.09748779,20.20030273)
\curveto(421.03748037,20.50029717)(420.93248047,20.7652969)(420.78248779,20.99530273)
\curveto(420.64248076,21.22529644)(420.44248096,21.40529626)(420.18248779,21.53530273)
\curveto(420.09248131,21.58529608)(419.99748141,21.62029605)(419.89748779,21.64030273)
\curveto(419.79748161,21.670296)(419.68748172,21.69529597)(419.56748779,21.71530273)
\curveto(419.49748191,21.73529593)(419.41248199,21.74529592)(419.31248779,21.74530273)
\lineto(419.04248779,21.74530273)
\lineto(418.89248779,21.71530273)
\lineto(418.75748779,21.71530273)
\curveto(418.67748273,21.69529597)(418.59248281,21.67529599)(418.50248779,21.65530273)
\curveto(418.41248299,21.63529603)(418.32748308,21.61029606)(418.24748779,21.58030273)
\curveto(417.89748351,21.44029623)(417.59748381,21.23529643)(417.34748779,20.96530273)
\curveto(417.09748431,20.70529696)(416.87748453,20.40029727)(416.68748779,20.05030273)
\curveto(416.62748478,19.94029773)(416.57748483,19.82529784)(416.53748779,19.70530273)
\lineto(416.41748779,19.37530273)
\lineto(416.38748779,19.25530273)
\curveto(416.37748503,19.22529844)(416.36748504,19.19029848)(416.35748779,19.15030273)
\curveto(416.32748508,19.10029857)(416.3074851,19.04529862)(416.29748779,18.98530273)
\curveto(416.29748511,18.92529874)(416.29248511,18.8702988)(416.28248779,18.82030273)
\curveto(416.26248514,18.71029896)(416.23748517,18.60029907)(416.20748779,18.49030273)
\curveto(416.18748522,18.39029928)(416.18248522,18.29529937)(416.19248779,18.20530273)
\curveto(416.19248521,18.17529949)(416.18748522,18.12529954)(416.17748779,18.05530273)
\lineto(416.17748779,17.84530273)
\curveto(416.17748523,17.77529989)(416.18248522,17.70529996)(416.19248779,17.63530273)
\curveto(416.23248517,17.28530038)(416.32248508,16.98530068)(416.46248779,16.73530273)
\curveto(416.6024848,16.48530118)(416.8024846,16.28030139)(417.06248779,16.12030273)
\curveto(417.14248426,16.0703016)(417.22248418,16.03030164)(417.30248779,16.00030273)
\curveto(417.39248401,15.9703017)(417.48748392,15.94030173)(417.58748779,15.91030273)
\curveto(417.63748377,15.89030178)(417.68748372,15.88530178)(417.73748779,15.89530273)
\curveto(417.79748361,15.90530176)(417.85248355,15.90030177)(417.90248779,15.88030273)
\curveto(417.93248347,15.8703018)(417.96748344,15.8653018)(418.00748779,15.86530273)
\lineto(418.14248779,15.86530273)
\lineto(418.27748779,15.86530273)
\curveto(418.31748309,15.87530179)(418.37248303,15.88030179)(418.44248779,15.88030273)
\curveto(418.52248288,15.90030177)(418.6024828,15.91530175)(418.68248779,15.92530273)
\curveto(418.77248263,15.94530172)(418.85248255,15.9703017)(418.92248779,16.00030273)
\curveto(419.28248212,16.14030153)(419.58748182,16.31530135)(419.83748779,16.52530273)
\curveto(420.08748132,16.74530092)(420.31248109,17.02030065)(420.51248779,17.35030273)
\curveto(420.58248082,17.46030021)(420.63748077,17.5703001)(420.67748779,17.68030273)
\lineto(420.82748779,18.01030273)
\curveto(420.85748055,18.05029962)(420.87248053,18.08529958)(420.87248779,18.11530273)
\curveto(420.88248052,18.15529951)(420.89748051,18.19529947)(420.91748779,18.23530273)
\curveto(420.93748047,18.29529937)(420.95248045,18.35529931)(420.96248779,18.41530273)
\curveto(420.97248043,18.47529919)(420.98748042,18.53529913)(421.00748779,18.59530273)
}
}
{
\newrgbcolor{curcolor}{0 0 0}
\pscustom[linestyle=none,fillstyle=solid,fillcolor=curcolor]
{
\newpath
\moveto(431.02373779,19.12030273)
\curveto(431.02372929,19.02029865)(431.00372931,18.90529876)(430.96373779,18.77530273)
\curveto(430.92372939,18.65529901)(430.87372944,18.5702991)(430.81373779,18.52030273)
\curveto(430.75372956,18.48029919)(430.67372964,18.45029922)(430.57373779,18.43030273)
\curveto(430.47372984,18.42029925)(430.36372995,18.41529925)(430.24373779,18.41530273)
\lineto(429.88373779,18.41530273)
\curveto(429.77373054,18.42529924)(429.67373064,18.43029924)(429.58373779,18.43030273)
\lineto(425.74373779,18.43030273)
\curveto(425.66373465,18.43029924)(425.57873473,18.42529924)(425.48873779,18.41530273)
\curveto(425.4087349,18.41529925)(425.34373497,18.40029927)(425.29373779,18.37030273)
\curveto(425.24373507,18.35029932)(425.19373512,18.31029936)(425.14373779,18.25030273)
\lineto(425.05373779,18.11530273)
\curveto(425.02373529,18.0652996)(425.0137353,18.01529965)(425.02373779,17.96530273)
\curveto(425.02373529,17.91529975)(425.01873529,17.8702998)(425.00873779,17.83030273)
\lineto(425.00873779,17.71030273)
\lineto(425.00873779,17.45530273)
\curveto(425.01873529,17.37530029)(425.03373528,17.29530037)(425.05373779,17.21530273)
\curveto(425.18373513,16.67530099)(425.48873482,16.29030138)(425.96873779,16.06030273)
\curveto(426.01873429,16.03030164)(426.07873423,16.00530166)(426.14873779,15.98530273)
\curveto(426.21873409,15.9653017)(426.28373403,15.94530172)(426.34373779,15.92530273)
\curveto(426.37373394,15.91530175)(426.42373389,15.91030176)(426.49373779,15.91030273)
\curveto(426.62373369,15.8703018)(426.80373351,15.85030182)(427.03373779,15.85030273)
\curveto(427.26373305,15.85030182)(427.45373286,15.8703018)(427.60373779,15.91030273)
\curveto(427.75373256,15.95030172)(427.88873242,15.99030168)(428.00873779,16.03030273)
\curveto(428.13873217,16.08030159)(428.25873205,16.14030153)(428.36873779,16.21030273)
\curveto(428.48873182,16.28030139)(428.59873171,16.36030131)(428.69873779,16.45030273)
\curveto(428.79873151,16.55030112)(428.88873142,16.65530101)(428.96873779,16.76530273)
\curveto(429.04873126,16.8653008)(429.12373119,16.9703007)(429.19373779,17.08030273)
\curveto(429.26373105,17.19030048)(429.35873095,17.2703004)(429.47873779,17.32030273)
\curveto(429.51873079,17.34030033)(429.58373073,17.35530031)(429.67373779,17.36530273)
\curveto(429.77373054,17.37530029)(429.86373045,17.37530029)(429.94373779,17.36530273)
\curveto(430.03373028,17.3653003)(430.11873019,17.36030031)(430.19873779,17.35030273)
\curveto(430.27873003,17.34030033)(430.32872998,17.32030035)(430.34873779,17.29030273)
\curveto(430.43872987,17.22030045)(430.44372987,17.10530056)(430.36373779,16.94530273)
\curveto(430.22373009,16.67530099)(430.06873024,16.43530123)(429.89873779,16.22530273)
\curveto(429.63873067,15.90530176)(429.35873095,15.64030203)(429.05873779,15.43030273)
\curveto(428.76873154,15.23030244)(428.4137319,15.0653026)(427.99373779,14.93530273)
\curveto(427.88373243,14.89530277)(427.77873253,14.8703028)(427.67873779,14.86030273)
\curveto(427.57873273,14.84030283)(427.46873284,14.82030285)(427.34873779,14.80030273)
\curveto(427.29873301,14.79030288)(427.24873306,14.78530288)(427.19873779,14.78530273)
\curveto(427.15873315,14.78530288)(427.1137332,14.78030289)(427.06373779,14.77030273)
\lineto(426.91373779,14.77030273)
\curveto(426.86373345,14.76030291)(426.80373351,14.75530291)(426.73373779,14.75530273)
\curveto(426.67373364,14.75530291)(426.62373369,14.76030291)(426.58373779,14.77030273)
\lineto(426.44873779,14.77030273)
\curveto(426.39873391,14.78030289)(426.35373396,14.78530288)(426.31373779,14.78530273)
\curveto(426.27373404,14.78530288)(426.23373408,14.79030288)(426.19373779,14.80030273)
\curveto(426.14373417,14.81030286)(426.08873422,14.82030285)(426.02873779,14.83030273)
\curveto(425.97873433,14.83030284)(425.92873438,14.83530283)(425.87873779,14.84530273)
\curveto(425.78873452,14.8653028)(425.69873461,14.89030278)(425.60873779,14.92030273)
\curveto(425.52873478,14.94030273)(425.45373486,14.9653027)(425.38373779,14.99530273)
\curveto(425.34373497,15.01530265)(425.308735,15.02530264)(425.27873779,15.02530273)
\curveto(425.24873506,15.03530263)(425.21873509,15.05030262)(425.18873779,15.07030273)
\curveto(425.04873526,15.14030253)(424.90373541,15.22530244)(424.75373779,15.32530273)
\curveto(424.50373581,15.51530215)(424.30373601,15.74530192)(424.15373779,16.01530273)
\curveto(424.00373631,16.29530137)(423.89373642,16.60530106)(423.82373779,16.94530273)
\curveto(423.79373652,17.05530061)(423.77873653,17.1703005)(423.77873779,17.29030273)
\curveto(423.77873653,17.41030026)(423.76873654,17.53030014)(423.74873779,17.65030273)
\lineto(423.74873779,17.75530273)
\curveto(423.75873655,17.78529988)(423.76373655,17.82529984)(423.76373779,17.87530273)
\lineto(423.76373779,18.13030273)
\curveto(423.77373654,18.22029945)(423.77873653,18.31029936)(423.77873779,18.40030273)
\lineto(423.82373779,18.61030273)
\curveto(423.82373649,18.65029902)(423.82873648,18.70529896)(423.83873779,18.77530273)
\curveto(423.84873646,18.85529881)(423.86373645,18.92029875)(423.88373779,18.97030273)
\lineto(423.91373779,19.13530273)
\curveto(423.94373637,19.18529848)(423.95873635,19.23529843)(423.95873779,19.28530273)
\curveto(423.96873634,19.34529832)(423.98373633,19.40029827)(424.00373779,19.45030273)
\curveto(424.07373624,19.61029806)(424.13873617,19.7702979)(424.19873779,19.93030273)
\curveto(424.25873605,20.09029758)(424.33373598,20.24029743)(424.42373779,20.38030273)
\curveto(424.49373582,20.49029718)(424.55873575,20.60029707)(424.61873779,20.71030273)
\curveto(424.68873562,20.83029684)(424.76873554,20.94529672)(424.85873779,21.05530273)
\curveto(425.14873516,21.40529626)(425.45873485,21.70529596)(425.78873779,21.95530273)
\curveto(426.11873419,22.21529545)(426.50373381,22.43029524)(426.94373779,22.60030273)
\curveto(427.07373324,22.65029502)(427.20373311,22.68529498)(427.33373779,22.70530273)
\curveto(427.46373285,22.73529493)(427.60373271,22.7652949)(427.75373779,22.79530273)
\curveto(427.80373251,22.80529486)(427.84873246,22.81029486)(427.88873779,22.81030273)
\curveto(427.92873238,22.82029485)(427.97373234,22.82529484)(428.02373779,22.82530273)
\curveto(428.04373227,22.83529483)(428.06873224,22.83529483)(428.09873779,22.82530273)
\curveto(428.12873218,22.81529485)(428.15373216,22.82029485)(428.17373779,22.84030273)
\curveto(428.60373171,22.85029482)(428.96373135,22.80529486)(429.25373779,22.70530273)
\curveto(429.54373077,22.61529505)(429.79873051,22.49029518)(430.01873779,22.33030273)
\curveto(430.05873025,22.31029536)(430.08873022,22.28029539)(430.10873779,22.24030273)
\curveto(430.13873017,22.21029546)(430.16873014,22.18529548)(430.19873779,22.16530273)
\curveto(430.26873004,22.10529556)(430.33872997,22.03529563)(430.40873779,21.95530273)
\curveto(430.47872983,21.87529579)(430.53372978,21.79529587)(430.57373779,21.71530273)
\curveto(430.69372962,21.50529616)(430.78872952,21.30529636)(430.85873779,21.11530273)
\curveto(430.9087294,21.00529666)(430.93872937,20.88529678)(430.94873779,20.75530273)
\lineto(431.00873779,20.36530273)
\curveto(431.03872927,20.23529743)(431.04872926,20.10029757)(431.03873779,19.96030273)
\curveto(431.03872927,19.82029785)(431.04372927,19.68029799)(431.05373779,19.54030273)
\curveto(431.05372926,19.4702982)(431.04872926,19.40029827)(431.03873779,19.33030273)
\curveto(431.02872928,19.26029841)(431.02372929,19.19029848)(431.02373779,19.12030273)
\moveto(429.67373779,19.63030273)
\curveto(429.70373061,19.670298)(429.73373058,19.72029795)(429.76373779,19.78030273)
\curveto(429.80373051,19.85029782)(429.81873049,19.92029775)(429.80873779,19.99030273)
\curveto(429.79873051,20.21029746)(429.75873055,20.41529725)(429.68873779,20.60530273)
\curveto(429.58873072,20.83529683)(429.46873084,21.03029664)(429.32873779,21.19030273)
\curveto(429.19873111,21.35029632)(429.0087313,21.48529618)(428.75873779,21.59530273)
\curveto(428.68873162,21.61529605)(428.61873169,21.63029604)(428.54873779,21.64030273)
\curveto(428.48873182,21.66029601)(428.41873189,21.68029599)(428.33873779,21.70030273)
\curveto(428.26873204,21.72029595)(428.18873212,21.73029594)(428.09873779,21.73030273)
\lineto(427.84373779,21.73030273)
\curveto(427.80373251,21.71029596)(427.76373255,21.70029597)(427.72373779,21.70030273)
\curveto(427.68373263,21.71029596)(427.64873266,21.71029596)(427.61873779,21.70030273)
\lineto(427.37873779,21.64030273)
\curveto(427.29873301,21.63029604)(427.22373309,21.61529605)(427.15373779,21.59530273)
\curveto(426.83373348,21.47529619)(426.56873374,21.32529634)(426.35873779,21.14530273)
\curveto(426.14873416,20.9652967)(425.94873436,20.74029693)(425.75873779,20.47030273)
\curveto(425.71873459,20.42029725)(425.67373464,20.35529731)(425.62373779,20.27530273)
\curveto(425.58373473,20.20529746)(425.54373477,20.12529754)(425.50373779,20.03530273)
\curveto(425.46373485,19.94529772)(425.43873487,19.86029781)(425.42873779,19.78030273)
\curveto(425.42873488,19.70029797)(425.45373486,19.64029803)(425.50373779,19.60030273)
\curveto(425.57373474,19.54029813)(425.70373461,19.51029816)(425.89373779,19.51030273)
\curveto(426.09373422,19.52029815)(426.26373405,19.52529814)(426.40373779,19.52530273)
\lineto(428.68373779,19.52530273)
\curveto(428.83373148,19.52529814)(429.0137313,19.52029815)(429.22373779,19.51030273)
\curveto(429.43373088,19.51029816)(429.58373073,19.55029812)(429.67373779,19.63030273)
}
}
{
\newrgbcolor{curcolor}{0 0 0}
\pscustom[linestyle=none,fillstyle=solid,fillcolor=curcolor]
{
\newpath
\moveto(436.09537842,25.75030273)
\curveto(436.27537272,25.76029191)(436.46537253,25.76029191)(436.66537842,25.75030273)
\curveto(436.86537213,25.74029193)(436.995372,25.68029199)(437.05537842,25.57030273)
\curveto(437.08537191,25.51029216)(437.0953719,25.43529223)(437.08537842,25.34530273)
\curveto(437.07537192,25.2652924)(437.06037193,25.17529249)(437.04037842,25.07530273)
\curveto(437.02037197,24.94529272)(436.97537202,24.84029283)(436.90537842,24.76030273)
\curveto(436.85537214,24.71029296)(436.7903722,24.67529299)(436.71037842,24.65530273)
\curveto(436.63037236,24.64529302)(436.54537245,24.64029303)(436.45537842,24.64030273)
\lineto(436.18537842,24.64030273)
\curveto(436.0953729,24.65029302)(436.01037298,24.65029302)(435.93037842,24.64030273)
\curveto(435.64037335,24.56029311)(435.43537356,24.43029324)(435.31537842,24.25030273)
\curveto(435.1953738,24.08029359)(435.10037389,23.82029385)(435.03037842,23.47030273)
\curveto(435.01037398,23.40029427)(434.98537401,23.32529434)(434.95537842,23.24530273)
\curveto(434.93537406,23.17529449)(434.93037406,23.11029456)(434.94037842,23.05030273)
\curveto(434.94037405,22.90029477)(434.98537401,22.79529487)(435.07537842,22.73530273)
\curveto(435.14537385,22.70529496)(435.24037375,22.69029498)(435.36037842,22.69030273)
\lineto(435.72037842,22.69030273)
\lineto(435.94537842,22.69030273)
\curveto(435.97537302,22.670295)(436.00537299,22.665295)(436.03537842,22.67530273)
\curveto(436.06537293,22.68529498)(436.0953729,22.68029499)(436.12537842,22.66030273)
\curveto(436.21537278,22.63029504)(436.26537273,22.5702951)(436.27537842,22.48030273)
\curveto(436.2953727,22.40029527)(436.2903727,22.29529537)(436.26037842,22.16530273)
\lineto(436.23037842,22.04530273)
\lineto(436.20037842,21.92530273)
\curveto(436.14037285,21.77529589)(436.05537294,21.67529599)(435.94537842,21.62530273)
\curveto(435.80537319,21.57529609)(435.63537336,21.56029611)(435.43537842,21.58030273)
\curveto(435.23537376,21.61029606)(435.06037393,21.60529606)(434.91037842,21.56530273)
\curveto(434.83037416,21.54529612)(434.76537423,21.50529616)(434.71537842,21.44530273)
\curveto(434.66537433,21.39529627)(434.62037437,21.32529634)(434.58037842,21.23530273)
\curveto(434.55037444,21.1652965)(434.53037446,21.08529658)(434.52037842,20.99530273)
\curveto(434.51037448,20.90529676)(434.4953745,20.82029685)(434.47537842,20.74030273)
\lineto(434.28037842,19.75030273)
\lineto(433.65037842,16.57030273)
\lineto(433.50037842,15.82030273)
\curveto(433.4903755,15.76030191)(433.48037551,15.69530197)(433.47037842,15.62530273)
\curveto(433.46037553,15.55530211)(433.44037555,15.49530217)(433.41037842,15.44530273)
\lineto(433.38037842,15.32530273)
\lineto(433.32037842,15.20530273)
\curveto(433.31037568,15.1653025)(433.2903757,15.13030254)(433.26037842,15.10030273)
\curveto(433.20037579,15.03030264)(433.11537588,14.99030268)(433.00537842,14.98030273)
\curveto(432.90537609,14.9703027)(432.7953762,14.9653027)(432.67537842,14.96530273)
\lineto(432.39037842,14.96530273)
\curveto(432.35037664,14.98530268)(432.30537669,15.00030267)(432.25537842,15.01030273)
\curveto(432.21537678,15.03030264)(432.18537681,15.0653026)(432.16537842,15.11530273)
\curveto(432.15537684,15.14530252)(432.15037684,15.21030246)(432.15037842,15.31030273)
\lineto(432.16537842,15.41530273)
\curveto(432.15537684,15.4653022)(432.16037683,15.51530215)(432.18037842,15.56530273)
\curveto(432.20037679,15.62530204)(432.21537678,15.68030199)(432.22537842,15.73030273)
\lineto(432.34537842,16.33030273)
\lineto(433.15537842,20.42530273)
\curveto(433.17537582,20.53529713)(433.20037579,20.65029702)(433.23037842,20.77030273)
\curveto(433.26037573,20.89029678)(433.28037571,21.00029667)(433.29037842,21.10030273)
\curveto(433.31037568,21.21029646)(433.31037568,21.30529636)(433.29037842,21.38530273)
\curveto(433.28037571,21.4652962)(433.23537576,21.52029615)(433.15537842,21.55030273)
\curveto(433.10537589,21.58029609)(433.04037595,21.59529607)(432.96037842,21.59530273)
\lineto(432.73537842,21.59530273)
\lineto(432.49537842,21.59530273)
\curveto(432.42537657,21.59529607)(432.36037663,21.60529606)(432.30037842,21.62530273)
\curveto(432.22037677,21.665296)(432.17537682,21.75029592)(432.16537842,21.88030273)
\lineto(432.16537842,22.01530273)
\curveto(432.17537682,22.05529561)(432.18537681,22.10029557)(432.19537842,22.15030273)
\curveto(432.22537677,22.29029538)(432.26037673,22.40029527)(432.30037842,22.48030273)
\curveto(432.35037664,22.5702951)(432.43037656,22.63029504)(432.54037842,22.66030273)
\curveto(432.62037637,22.69029498)(432.70537629,22.70029497)(432.79537842,22.69030273)
\lineto(433.06537842,22.69030273)
\curveto(433.16537583,22.69029498)(433.25537574,22.70029497)(433.33537842,22.72030273)
\curveto(433.41537558,22.74029493)(433.48537551,22.78029489)(433.54537842,22.84030273)
\curveto(433.63537536,22.92029475)(433.6953753,23.04529462)(433.72537842,23.21530273)
\curveto(433.75537524,23.38529428)(433.78537521,23.54529412)(433.81537842,23.69530273)
\curveto(433.85537514,23.89529377)(433.90537509,24.08029359)(433.96537842,24.25030273)
\curveto(434.02537497,24.43029324)(434.10037489,24.59029308)(434.19037842,24.73030273)
\curveto(434.34037465,24.9702927)(434.52037447,25.1652925)(434.73037842,25.31530273)
\curveto(434.95037404,25.4652922)(435.20037379,25.58029209)(435.48037842,25.66030273)
\curveto(435.54037345,25.68029199)(435.60537339,25.69029198)(435.67537842,25.69030273)
\curveto(435.74537325,25.70029197)(435.81537318,25.71529195)(435.88537842,25.73530273)
\curveto(435.90537309,25.74529192)(435.94037305,25.74529192)(435.99037842,25.73530273)
\curveto(436.04037295,25.73529193)(436.07537292,25.74029193)(436.09537842,25.75030273)
\moveto(438.04537842,24.17530273)
\curveto(438.10537089,24.12529354)(438.18537081,24.10029357)(438.28537842,24.10030273)
\lineto(438.60037842,24.10030273)
\lineto(438.76537842,24.10030273)
\curveto(438.82537017,24.10029357)(438.88537011,24.11029356)(438.94537842,24.13030273)
\curveto(439.08536991,24.18029349)(439.17036982,24.28529338)(439.20037842,24.44530273)
\curveto(439.24036975,24.60529306)(439.28036971,24.77529289)(439.32037842,24.95530273)
\curveto(439.33036966,25.04529262)(439.34536965,25.13029254)(439.36537842,25.21030273)
\curveto(439.38536961,25.30029237)(439.38536961,25.37529229)(439.36537842,25.43530273)
\curveto(439.33536966,25.54529212)(439.24536975,25.60529206)(439.09537842,25.61530273)
\curveto(438.95537004,25.62529204)(438.80037019,25.63029204)(438.63037842,25.63030273)
\curveto(438.60037039,25.62029205)(438.57537042,25.61529205)(438.55537842,25.61530273)
\curveto(438.53537046,25.62529204)(438.51037048,25.62529204)(438.48037842,25.61530273)
\curveto(438.36037063,25.57529209)(438.27037072,25.51529215)(438.21037842,25.43530273)
\curveto(438.17037082,25.37529229)(438.14037085,25.30029237)(438.12037842,25.21030273)
\curveto(438.10037089,25.12029255)(438.08537091,25.03529263)(438.07537842,24.95530273)
\curveto(438.04537095,24.80529286)(438.01537098,24.65029302)(437.98537842,24.49030273)
\curveto(437.95537104,24.34029333)(437.97537102,24.23529343)(438.04537842,24.17530273)
\moveto(438.72037842,22.01530273)
\curveto(438.74037025,22.11529555)(438.76037023,22.21029546)(438.78037842,22.30030273)
\curveto(438.80037019,22.40029527)(438.7903702,22.48029519)(438.75037842,22.54030273)
\curveto(438.72037027,22.62029505)(438.63537036,22.66029501)(438.49537842,22.66030273)
\curveto(438.36537063,22.670295)(438.23537076,22.67529499)(438.10537842,22.67530273)
\curveto(438.08537091,22.665295)(438.06037093,22.66029501)(438.03037842,22.66030273)
\curveto(438.01037098,22.670295)(437.990371,22.67529499)(437.97037842,22.67530273)
\curveto(437.91037108,22.65529501)(437.85037114,22.64029503)(437.79037842,22.63030273)
\curveto(437.74037125,22.62029505)(437.6953713,22.59029508)(437.65537842,22.54030273)
\curveto(437.5953714,22.48029519)(437.55537144,22.39529527)(437.53537842,22.28530273)
\curveto(437.51537148,22.18529548)(437.4953715,22.08029559)(437.47537842,21.97030273)
\lineto(436.20037842,15.62530273)
\curveto(436.18037281,15.53530213)(436.16037283,15.44030223)(436.14037842,15.34030273)
\curveto(436.13037286,15.25030242)(436.13537286,15.17530249)(436.15537842,15.11530273)
\curveto(436.1953728,15.03530263)(436.26037273,14.98530268)(436.35037842,14.96530273)
\curveto(436.44037255,14.95530271)(436.55037244,14.95030272)(436.68037842,14.95030273)
\lineto(436.90537842,14.95030273)
\curveto(436.995372,14.9703027)(437.07037192,14.98530268)(437.13037842,14.99530273)
\curveto(437.1903718,15.01530265)(437.24037175,15.05530261)(437.28037842,15.11530273)
\curveto(437.35037164,15.17530249)(437.3903716,15.25530241)(437.40037842,15.35530273)
\curveto(437.42037157,15.4653022)(437.44037155,15.5703021)(437.46037842,15.67030273)
\lineto(438.72037842,22.01530273)
}
}
{
\newrgbcolor{curcolor}{0 0 0}
\pscustom[linestyle=none,fillstyle=solid,fillcolor=curcolor]
{
\newpath
\moveto(444.51905029,22.82530273)
\curveto(445.15904347,22.84529482)(445.64904298,22.76029491)(445.98905029,22.57030273)
\curveto(446.3290423,22.38029529)(446.57404206,22.09529557)(446.72405029,21.71530273)
\curveto(446.76404187,21.61529605)(446.78904184,21.50529616)(446.79905029,21.38530273)
\curveto(446.81904181,21.27529639)(446.8290418,21.16029651)(446.82905029,21.04030273)
\curveto(446.84904178,20.85029682)(446.83904179,20.64529702)(446.79905029,20.42530273)
\curveto(446.76904186,20.20529746)(446.7290419,19.98029769)(446.67905029,19.75030273)
\lineto(446.36405029,18.14530273)
\lineto(445.89905029,15.80530273)
\lineto(445.77905029,15.29530273)
\curveto(445.73904289,15.12530254)(445.64904298,15.01530265)(445.50905029,14.96530273)
\curveto(445.45904317,14.94530272)(445.40404323,14.93530273)(445.34405029,14.93530273)
\curveto(445.29404334,14.92530274)(445.23904339,14.92030275)(445.17905029,14.92030273)
\curveto(445.04904358,14.92030275)(444.92404371,14.92530274)(444.80405029,14.93530273)
\curveto(444.68404395,14.93530273)(444.60904402,14.97530269)(444.57905029,15.05530273)
\curveto(444.53904409,15.12530254)(444.5290441,15.21530245)(444.54905029,15.32530273)
\curveto(444.56904406,15.43530223)(444.59404404,15.54530212)(444.62405029,15.65530273)
\lineto(444.87905029,16.94530273)
\lineto(445.35905029,19.39030273)
\curveto(445.41904321,19.66029801)(445.46904316,19.92529774)(445.50905029,20.18530273)
\curveto(445.54904308,20.45529721)(445.54904308,20.68529698)(445.50905029,20.87530273)
\curveto(445.46904316,21.07529659)(445.37904325,21.23529643)(445.23905029,21.35530273)
\curveto(445.10904352,21.48529618)(444.94904368,21.58529608)(444.75905029,21.65530273)
\curveto(444.69904393,21.67529599)(444.634044,21.68529598)(444.56405029,21.68530273)
\curveto(444.50404413,21.69529597)(444.44904418,21.71029596)(444.39905029,21.73030273)
\curveto(444.34904428,21.74029593)(444.26904436,21.74029593)(444.15905029,21.73030273)
\curveto(444.05904457,21.73029594)(443.98404465,21.72529594)(443.93405029,21.71530273)
\curveto(443.89404474,21.69529597)(443.85904477,21.68529598)(443.82905029,21.68530273)
\curveto(443.79904483,21.69529597)(443.76404487,21.69529597)(443.72405029,21.68530273)
\curveto(443.58404505,21.65529601)(443.45404518,21.62029605)(443.33405029,21.58030273)
\curveto(443.21404542,21.55029612)(443.09904553,21.50529616)(442.98905029,21.44530273)
\curveto(442.93904569,21.42529624)(442.89904573,21.40529626)(442.86905029,21.38530273)
\curveto(442.83904579,21.3652963)(442.79904583,21.34529632)(442.74905029,21.32530273)
\curveto(442.34904628,21.07529659)(442.01904661,20.70029697)(441.75905029,20.20030273)
\curveto(441.71904691,20.12029755)(441.68404695,20.03529763)(441.65405029,19.94530273)
\lineto(441.56405029,19.70530273)
\curveto(441.5340471,19.65529801)(441.51904711,19.60529806)(441.51905029,19.55530273)
\curveto(441.51904711,19.51529815)(441.50404713,19.47529819)(441.47405029,19.43530273)
\lineto(441.41405029,19.12030273)
\curveto(441.39404724,19.09029858)(441.38404725,19.05529861)(441.38405029,19.01530273)
\curveto(441.38404725,18.97529869)(441.37904725,18.93029874)(441.36905029,18.88030273)
\lineto(441.27905029,18.43030273)
\lineto(440.97905029,16.99030273)
\lineto(440.72405029,15.67030273)
\curveto(440.70404793,15.56030211)(440.67904795,15.44530222)(440.64905029,15.32530273)
\curveto(440.629048,15.21530245)(440.58904804,15.12530254)(440.52905029,15.05530273)
\curveto(440.45904817,14.97530269)(440.35904827,14.93530273)(440.22905029,14.93530273)
\curveto(440.10904852,14.92530274)(439.98404865,14.92030275)(439.85405029,14.92030273)
\curveto(439.77404886,14.92030275)(439.69904893,14.92530274)(439.62905029,14.93530273)
\curveto(439.55904907,14.94530272)(439.50404913,14.9703027)(439.46405029,15.01030273)
\curveto(439.39404924,15.06030261)(439.37404926,15.15530251)(439.40405029,15.29530273)
\curveto(439.4340492,15.43530223)(439.45904917,15.5703021)(439.47905029,15.70030273)
\lineto(439.83905029,17.47030273)
\lineto(440.55905029,21.10030273)
\lineto(440.73905029,22.01530273)
\lineto(440.79905029,22.28530273)
\curveto(440.81904781,22.37529529)(440.85404778,22.44529522)(440.90405029,22.49530273)
\curveto(440.94404769,22.55529511)(440.99904763,22.59529507)(441.06905029,22.61530273)
\curveto(441.11904751,22.62529504)(441.17904745,22.63529503)(441.24905029,22.64530273)
\curveto(441.3290473,22.65529501)(441.40904722,22.66029501)(441.48905029,22.66030273)
\curveto(441.56904706,22.66029501)(441.64404699,22.65529501)(441.71405029,22.64530273)
\curveto(441.79404684,22.63529503)(441.84404679,22.62029505)(441.86405029,22.60030273)
\curveto(441.96404667,22.53029514)(441.99904663,22.44029523)(441.96905029,22.33030273)
\curveto(441.93904669,22.23029544)(441.9290467,22.11529555)(441.93905029,21.98530273)
\curveto(441.94904668,21.92529574)(441.97904665,21.87529579)(442.02905029,21.83530273)
\curveto(442.14904648,21.82529584)(442.25404638,21.8702958)(442.34405029,21.97030273)
\curveto(442.44404619,22.0702956)(442.53904609,22.15029552)(442.62905029,22.21030273)
\curveto(442.78904584,22.31029536)(442.94904568,22.40029527)(443.10905029,22.48030273)
\curveto(443.26904536,22.5702951)(443.45404518,22.64529502)(443.66405029,22.70530273)
\curveto(443.74404489,22.73529493)(443.8340448,22.75529491)(443.93405029,22.76530273)
\curveto(444.0340446,22.77529489)(444.1290445,22.79029488)(444.21905029,22.81030273)
\curveto(444.26904436,22.82029485)(444.31904431,22.82529484)(444.36905029,22.82530273)
\lineto(444.51905029,22.82530273)
}
}
{
\newrgbcolor{curcolor}{0 0 0}
\pscustom[linestyle=none,fillstyle=solid,fillcolor=curcolor]
{
\newpath
\moveto(449.73365967,24.17530273)
\curveto(449.66365669,24.23529343)(449.64365671,24.34029333)(449.67365967,24.49030273)
\curveto(449.70365665,24.65029302)(449.73365662,24.80529286)(449.76365967,24.95530273)
\curveto(449.77365658,25.03529263)(449.78865657,25.12029255)(449.80865967,25.21030273)
\curveto(449.82865653,25.30029237)(449.8586565,25.37529229)(449.89865967,25.43530273)
\curveto(449.9586564,25.51529215)(450.04865631,25.57529209)(450.16865967,25.61530273)
\curveto(450.19865616,25.62529204)(450.22365613,25.62529204)(450.24365967,25.61530273)
\curveto(450.26365609,25.61529205)(450.28865607,25.62029205)(450.31865967,25.63030273)
\curveto(450.48865587,25.63029204)(450.64365571,25.62529204)(450.78365967,25.61530273)
\curveto(450.93365542,25.60529206)(451.02365533,25.54529212)(451.05365967,25.43530273)
\curveto(451.07365528,25.37529229)(451.07365528,25.30029237)(451.05365967,25.21030273)
\curveto(451.03365532,25.13029254)(451.01865534,25.04529262)(451.00865967,24.95530273)
\curveto(450.96865539,24.77529289)(450.92865543,24.60529306)(450.88865967,24.44530273)
\curveto(450.8586555,24.28529338)(450.77365558,24.18029349)(450.63365967,24.13030273)
\curveto(450.57365578,24.11029356)(450.51365584,24.10029357)(450.45365967,24.10030273)
\lineto(450.28865967,24.10030273)
\lineto(449.97365967,24.10030273)
\curveto(449.87365648,24.10029357)(449.79365656,24.12529354)(449.73365967,24.17530273)
\moveto(449.14865967,15.67030273)
\curveto(449.12865723,15.5703021)(449.10865725,15.4653022)(449.08865967,15.35530273)
\curveto(449.07865728,15.25530241)(449.03865732,15.17530249)(448.96865967,15.11530273)
\curveto(448.92865743,15.05530261)(448.87865748,15.01530265)(448.81865967,14.99530273)
\curveto(448.7586576,14.98530268)(448.68365767,14.9703027)(448.59365967,14.95030273)
\lineto(448.36865967,14.95030273)
\curveto(448.23865812,14.95030272)(448.12865823,14.95530271)(448.03865967,14.96530273)
\curveto(447.94865841,14.98530268)(447.88365847,15.03530263)(447.84365967,15.11530273)
\curveto(447.82365853,15.17530249)(447.81865854,15.25030242)(447.82865967,15.34030273)
\curveto(447.84865851,15.44030223)(447.86865849,15.53530213)(447.88865967,15.62530273)
\lineto(449.16365967,21.97030273)
\curveto(449.18365717,22.08029559)(449.20365715,22.18529548)(449.22365967,22.28530273)
\curveto(449.24365711,22.39529527)(449.28365707,22.48029519)(449.34365967,22.54030273)
\curveto(449.38365697,22.59029508)(449.42865693,22.62029505)(449.47865967,22.63030273)
\curveto(449.53865682,22.64029503)(449.59865676,22.65529501)(449.65865967,22.67530273)
\curveto(449.67865668,22.67529499)(449.69865666,22.670295)(449.71865967,22.66030273)
\curveto(449.74865661,22.66029501)(449.77365658,22.665295)(449.79365967,22.67530273)
\curveto(449.92365643,22.67529499)(450.0536563,22.670295)(450.18365967,22.66030273)
\curveto(450.32365603,22.66029501)(450.40865595,22.62029505)(450.43865967,22.54030273)
\curveto(450.47865588,22.48029519)(450.48865587,22.40029527)(450.46865967,22.30030273)
\curveto(450.44865591,22.21029546)(450.42865593,22.11529555)(450.40865967,22.01530273)
\lineto(449.14865967,15.67030273)
}
}
{
\newrgbcolor{curcolor}{0 0 0}
\pscustom[linestyle=none,fillstyle=solid,fillcolor=curcolor]
{
\newpath
\moveto(458.06850342,15.76030273)
\lineto(457.97850342,15.37030273)
\curveto(457.95849549,15.25030242)(457.91849553,15.15030252)(457.85850342,15.07030273)
\curveto(457.78849566,15.00030267)(457.69349575,14.96030271)(457.57350342,14.95030273)
\lineto(457.22850342,14.95030273)
\curveto(457.16849628,14.95030272)(457.10849634,14.94530272)(457.04850342,14.93530273)
\curveto(456.99849645,14.93530273)(456.95349649,14.94530272)(456.91350342,14.96530273)
\curveto(456.83349661,14.98530268)(456.78349666,15.02530264)(456.76350342,15.08530273)
\curveto(456.73349671,15.13530253)(456.72349672,15.19530247)(456.73350342,15.26530273)
\curveto(456.7434967,15.33530233)(456.73849671,15.40530226)(456.71850342,15.47530273)
\curveto(456.71849673,15.49530217)(456.70849674,15.51030216)(456.68850342,15.52030273)
\lineto(456.65850342,15.58030273)
\curveto(456.55849689,15.59030208)(456.47349697,15.5703021)(456.40350342,15.52030273)
\curveto(456.3434971,15.4703022)(456.27849717,15.42030225)(456.20850342,15.37030273)
\curveto(455.97849747,15.22030245)(455.75349769,15.10530256)(455.53350342,15.02530273)
\curveto(455.3434981,14.94530272)(455.12349832,14.88530278)(454.87350342,14.84530273)
\curveto(454.63349881,14.80530286)(454.38849906,14.78530288)(454.13850342,14.78530273)
\curveto(453.89849955,14.77530289)(453.65849979,14.79030288)(453.41850342,14.83030273)
\curveto(453.18850026,14.86030281)(452.99350045,14.91530275)(452.83350342,14.99530273)
\curveto(452.35350109,15.21530245)(451.98850146,15.51030216)(451.73850342,15.88030273)
\curveto(451.49850195,16.26030141)(451.3435021,16.73030094)(451.27350342,17.29030273)
\curveto(451.25350219,17.38030029)(451.2435022,17.4703002)(451.24350342,17.56030273)
\curveto(451.25350219,17.66030001)(451.25350219,17.76029991)(451.24350342,17.86030273)
\curveto(451.2435022,17.91029976)(451.2485022,17.96029971)(451.25850342,18.01030273)
\curveto(451.26850218,18.06029961)(451.27350217,18.11029956)(451.27350342,18.16030273)
\curveto(451.26350218,18.21029946)(451.26350218,18.26029941)(451.27350342,18.31030273)
\curveto(451.29350215,18.3702993)(451.30350214,18.42529924)(451.30350342,18.47530273)
\lineto(451.33350342,18.62530273)
\curveto(451.32350212,18.67529899)(451.32350212,18.74029893)(451.33350342,18.82030273)
\curveto(451.35350209,18.90029877)(451.37850207,18.9652987)(451.40850342,19.01530273)
\lineto(451.45350342,19.18030273)
\curveto(451.48350196,19.25029842)(451.50350194,19.32029835)(451.51350342,19.39030273)
\curveto(451.52350192,19.4702982)(451.5435019,19.54529812)(451.57350342,19.61530273)
\curveto(451.59350185,19.665298)(451.60850184,19.71029796)(451.61850342,19.75030273)
\curveto(451.62850182,19.79029788)(451.6435018,19.83529783)(451.66350342,19.88530273)
\curveto(451.71350173,19.98529768)(451.75850169,20.08029759)(451.79850342,20.17030273)
\curveto(451.83850161,20.2702974)(451.88350156,20.3652973)(451.93350342,20.45530273)
\curveto(452.13350131,20.83529683)(452.36350108,21.17529649)(452.62350342,21.47530273)
\curveto(452.89350055,21.78529588)(453.19350025,22.04029563)(453.52350342,22.24030273)
\curveto(453.72349972,22.36029531)(453.92349952,22.46029521)(454.12350342,22.54030273)
\curveto(454.32349912,22.62029505)(454.53849891,22.69029498)(454.76850342,22.75030273)
\lineto(454.97850342,22.78030273)
\curveto(455.0484984,22.79029488)(455.11849833,22.80529486)(455.18850342,22.82530273)
\lineto(455.33850342,22.82530273)
\curveto(455.42849802,22.84529482)(455.5484979,22.85529481)(455.69850342,22.85530273)
\curveto(455.85849759,22.85529481)(455.97349747,22.84529482)(456.04350342,22.82530273)
\curveto(456.08349736,22.81529485)(456.13849731,22.81029486)(456.20850342,22.81030273)
\curveto(456.30849714,22.78029489)(456.41349703,22.75529491)(456.52350342,22.73530273)
\curveto(456.63349681,22.72529494)(456.73349671,22.69529497)(456.82350342,22.64530273)
\curveto(456.96349648,22.58529508)(457.09349635,22.52029515)(457.21350342,22.45030273)
\curveto(457.33349611,22.38029529)(457.443496,22.30029537)(457.54350342,22.21030273)
\curveto(457.59349585,22.16029551)(457.6434958,22.10529556)(457.69350342,22.04530273)
\curveto(457.75349569,21.99529567)(457.83849561,21.98029569)(457.94850342,22.00030273)
\lineto(458.02350342,22.07530273)
\curveto(458.0434954,22.09529557)(458.05849539,22.12529554)(458.06850342,22.16530273)
\curveto(458.11849533,22.25529541)(458.15349529,22.3702953)(458.17350342,22.51030273)
\curveto(458.20349524,22.65029502)(458.22849522,22.77529489)(458.24850342,22.88530273)
\lineto(458.59350342,24.61030273)
\curveto(458.62349482,24.75029292)(458.65349479,24.90529276)(458.68350342,25.07530273)
\curveto(458.72349472,25.25529241)(458.77349467,25.38529228)(458.83350342,25.46530273)
\curveto(458.89349455,25.53529213)(458.96349448,25.58029209)(459.04350342,25.60030273)
\curveto(459.06349438,25.60029207)(459.08849436,25.60029207)(459.11850342,25.60030273)
\curveto(459.1484943,25.61029206)(459.17349427,25.61529205)(459.19350342,25.61530273)
\curveto(459.3434941,25.62529204)(459.49349395,25.62529204)(459.64350342,25.61530273)
\curveto(459.79349365,25.61529205)(459.89349355,25.57529209)(459.94350342,25.49530273)
\curveto(459.97349347,25.41529225)(459.97349347,25.31529235)(459.94350342,25.19530273)
\curveto(459.92349352,25.07529259)(459.90349354,24.95029272)(459.88350342,24.82030273)
\lineto(458.06850342,15.76030273)
\moveto(457.42350342,18.59530273)
\curveto(457.45349599,18.64529902)(457.47349597,18.71029896)(457.48350342,18.79030273)
\curveto(457.50349594,18.88029879)(457.50849594,18.95029872)(457.49850342,19.00030273)
\lineto(457.54350342,19.22530273)
\curveto(457.5434959,19.31529835)(457.5484959,19.40529826)(457.55850342,19.49530273)
\curveto(457.56849588,19.59529807)(457.56349588,19.68529798)(457.54350342,19.76530273)
\lineto(457.54350342,19.99030273)
\curveto(457.5434959,20.06029761)(457.53349591,20.13029754)(457.51350342,20.20030273)
\curveto(457.45349599,20.50029717)(457.3484961,20.7652969)(457.19850342,20.99530273)
\curveto(457.05849639,21.22529644)(456.85849659,21.40529626)(456.59850342,21.53530273)
\curveto(456.50849694,21.58529608)(456.41349703,21.62029605)(456.31350342,21.64030273)
\curveto(456.21349723,21.670296)(456.10349734,21.69529597)(455.98350342,21.71530273)
\curveto(455.91349753,21.73529593)(455.82849762,21.74529592)(455.72850342,21.74530273)
\lineto(455.45850342,21.74530273)
\lineto(455.30850342,21.71530273)
\lineto(455.17350342,21.71530273)
\curveto(455.09349835,21.69529597)(455.00849844,21.67529599)(454.91850342,21.65530273)
\curveto(454.82849862,21.63529603)(454.7434987,21.61029606)(454.66350342,21.58030273)
\curveto(454.31349913,21.44029623)(454.01349943,21.23529643)(453.76350342,20.96530273)
\curveto(453.51349993,20.70529696)(453.29350015,20.40029727)(453.10350342,20.05030273)
\curveto(453.0435004,19.94029773)(452.99350045,19.82529784)(452.95350342,19.70530273)
\lineto(452.83350342,19.37530273)
\lineto(452.80350342,19.25530273)
\curveto(452.79350065,19.22529844)(452.78350066,19.19029848)(452.77350342,19.15030273)
\curveto(452.7435007,19.10029857)(452.72350072,19.04529862)(452.71350342,18.98530273)
\curveto(452.71350073,18.92529874)(452.70850074,18.8702988)(452.69850342,18.82030273)
\curveto(452.67850077,18.71029896)(452.65350079,18.60029907)(452.62350342,18.49030273)
\curveto(452.60350084,18.39029928)(452.59850085,18.29529937)(452.60850342,18.20530273)
\curveto(452.60850084,18.17529949)(452.60350084,18.12529954)(452.59350342,18.05530273)
\lineto(452.59350342,17.84530273)
\curveto(452.59350085,17.77529989)(452.59850085,17.70529996)(452.60850342,17.63530273)
\curveto(452.6485008,17.28530038)(452.73850071,16.98530068)(452.87850342,16.73530273)
\curveto(453.01850043,16.48530118)(453.21850023,16.28030139)(453.47850342,16.12030273)
\curveto(453.55849989,16.0703016)(453.63849981,16.03030164)(453.71850342,16.00030273)
\curveto(453.80849964,15.9703017)(453.90349954,15.94030173)(454.00350342,15.91030273)
\curveto(454.05349939,15.89030178)(454.10349934,15.88530178)(454.15350342,15.89530273)
\curveto(454.21349923,15.90530176)(454.26849918,15.90030177)(454.31850342,15.88030273)
\curveto(454.3484991,15.8703018)(454.38349906,15.8653018)(454.42350342,15.86530273)
\lineto(454.55850342,15.86530273)
\lineto(454.69350342,15.86530273)
\curveto(454.73349871,15.87530179)(454.78849866,15.88030179)(454.85850342,15.88030273)
\curveto(454.93849851,15.90030177)(455.01849843,15.91530175)(455.09850342,15.92530273)
\curveto(455.18849826,15.94530172)(455.26849818,15.9703017)(455.33850342,16.00030273)
\curveto(455.69849775,16.14030153)(456.00349744,16.31530135)(456.25350342,16.52530273)
\curveto(456.50349694,16.74530092)(456.72849672,17.02030065)(456.92850342,17.35030273)
\curveto(456.99849645,17.46030021)(457.05349639,17.5703001)(457.09350342,17.68030273)
\lineto(457.24350342,18.01030273)
\curveto(457.27349617,18.05029962)(457.28849616,18.08529958)(457.28850342,18.11530273)
\curveto(457.29849615,18.15529951)(457.31349613,18.19529947)(457.33350342,18.23530273)
\curveto(457.35349609,18.29529937)(457.36849608,18.35529931)(457.37850342,18.41530273)
\curveto(457.38849606,18.47529919)(457.40349604,18.53529913)(457.42350342,18.59530273)
}
}
{
\newrgbcolor{curcolor}{0 0 0}
\pscustom[linestyle=none,fillstyle=solid,fillcolor=curcolor]
{
\newpath
\moveto(467.81475342,19.15030273)
\curveto(467.82474453,19.09029858)(467.81474454,18.99529867)(467.78475342,18.86530273)
\curveto(467.76474459,18.74529892)(467.74474461,18.66029901)(467.72475342,18.61030273)
\lineto(467.69475342,18.46030273)
\curveto(467.66474469,18.38029929)(467.63974471,18.30529936)(467.61975342,18.23530273)
\curveto(467.60974474,18.17529949)(467.58974476,18.10529956)(467.55975342,18.02530273)
\curveto(467.52974482,17.9652997)(467.50474485,17.90529976)(467.48475342,17.84530273)
\curveto(467.47474488,17.78529988)(467.4497449,17.72529994)(467.40975342,17.66530273)
\lineto(467.22975342,17.27530273)
\curveto(467.17974517,17.14530052)(467.11474524,17.02530064)(467.03475342,16.91530273)
\curveto(466.73474562,16.43530123)(466.37474598,16.03030164)(465.95475342,15.70030273)
\curveto(465.54474681,15.38030229)(465.06474729,15.13530253)(464.51475342,14.96530273)
\curveto(464.40474795,14.92530274)(464.28474807,14.89530277)(464.15475342,14.87530273)
\curveto(464.02474833,14.85530281)(463.88974846,14.83530283)(463.74975342,14.81530273)
\curveto(463.68974866,14.80530286)(463.62474873,14.80030287)(463.55475342,14.80030273)
\curveto(463.49474886,14.79030288)(463.43474892,14.78530288)(463.37475342,14.78530273)
\curveto(463.33474902,14.77530289)(463.27474908,14.7703029)(463.19475342,14.77030273)
\curveto(463.12474923,14.7703029)(463.07474928,14.77530289)(463.04475342,14.78530273)
\curveto(463.00474935,14.79530287)(462.96474939,14.80030287)(462.92475342,14.80030273)
\curveto(462.88474947,14.79030288)(462.8497495,14.79030288)(462.81975342,14.80030273)
\lineto(462.72975342,14.80030273)
\lineto(462.38475342,14.84530273)
\lineto(461.99475342,14.96530273)
\curveto(461.87475048,15.00530266)(461.75975059,15.05030262)(461.64975342,15.10030273)
\curveto(461.23975111,15.30030237)(460.91975143,15.56030211)(460.68975342,15.88030273)
\curveto(460.46975188,16.20030147)(460.30975204,16.59030108)(460.20975342,17.05030273)
\curveto(460.17975217,17.15030052)(460.15975219,17.25030042)(460.14975342,17.35030273)
\lineto(460.14975342,17.66530273)
\curveto(460.13975221,17.70529996)(460.13975221,17.73529993)(460.14975342,17.75530273)
\curveto(460.15975219,17.78529988)(460.16475219,17.82029985)(460.16475342,17.86030273)
\curveto(460.16475219,17.94029973)(460.16975218,18.02029965)(460.17975342,18.10030273)
\curveto(460.18975216,18.19029948)(460.19475216,18.27529939)(460.19475342,18.35530273)
\curveto(460.20475215,18.40529926)(460.20975214,18.44529922)(460.20975342,18.47530273)
\curveto(460.21975213,18.51529915)(460.22475213,18.56029911)(460.22475342,18.61030273)
\curveto(460.22475213,18.66029901)(460.23475212,18.74529892)(460.25475342,18.86530273)
\curveto(460.28475207,18.99529867)(460.31475204,19.09029858)(460.34475342,19.15030273)
\curveto(460.38475197,19.22029845)(460.40475195,19.29029838)(460.40475342,19.36030273)
\curveto(460.40475195,19.43029824)(460.42475193,19.50029817)(460.46475342,19.57030273)
\curveto(460.48475187,19.62029805)(460.49975185,19.66029801)(460.50975342,19.69030273)
\curveto(460.51975183,19.73029794)(460.53475182,19.77529789)(460.55475342,19.82530273)
\curveto(460.61475174,19.94529772)(460.66475169,20.0652976)(460.70475342,20.18530273)
\curveto(460.7547516,20.30529736)(460.81975153,20.42029725)(460.89975342,20.53030273)
\curveto(461.11975123,20.90029677)(461.36475099,21.23029644)(461.63475342,21.52030273)
\curveto(461.91475044,21.82029585)(462.22975012,22.0702956)(462.57975342,22.27030273)
\curveto(462.70974964,22.35029532)(462.84474951,22.41529525)(462.98475342,22.46530273)
\lineto(463.43475342,22.64530273)
\curveto(463.56474879,22.69529497)(463.69974865,22.72529494)(463.83975342,22.73530273)
\curveto(463.97974837,22.75529491)(464.12474823,22.78529488)(464.27475342,22.82530273)
\lineto(464.46975342,22.82530273)
\lineto(464.67975342,22.85530273)
\curveto(465.56974678,22.8652948)(466.26974608,22.68029499)(466.77975342,22.30030273)
\curveto(467.29974505,21.92029575)(467.62474473,21.42529624)(467.75475342,20.81530273)
\curveto(467.78474457,20.71529695)(467.80474455,20.61529705)(467.81475342,20.51530273)
\curveto(467.82474453,20.41529725)(467.83974451,20.31029736)(467.85975342,20.20030273)
\curveto(467.86974448,20.09029758)(467.86974448,19.9702977)(467.85975342,19.84030273)
\lineto(467.85975342,19.46530273)
\curveto(467.85974449,19.41529825)(467.8497445,19.36029831)(467.82975342,19.30030273)
\curveto(467.81974453,19.25029842)(467.81474454,19.20029847)(467.81475342,19.15030273)
\moveto(466.31475342,18.29530273)
\curveto(466.34474601,18.3652993)(466.36474599,18.44529922)(466.37475342,18.53530273)
\curveto(466.39474596,18.62529904)(466.40974594,18.71029896)(466.41975342,18.79030273)
\curveto(466.49974585,19.18029849)(466.53474582,19.51029816)(466.52475342,19.78030273)
\curveto(466.50474585,19.86029781)(466.48974586,19.94029773)(466.47975342,20.02030273)
\curveto(466.47974587,20.10029757)(466.47474588,20.17529749)(466.46475342,20.24530273)
\curveto(466.31474604,20.89529677)(465.95974639,21.34529632)(465.39975342,21.59530273)
\curveto(465.32974702,21.62529604)(465.2547471,21.64529602)(465.17475342,21.65530273)
\curveto(465.10474725,21.67529599)(465.02974732,21.69529597)(464.94975342,21.71530273)
\curveto(464.87974747,21.73529593)(464.79974755,21.74529592)(464.70975342,21.74530273)
\lineto(464.43975342,21.74530273)
\lineto(464.15475342,21.70030273)
\curveto(464.0547483,21.68029599)(463.95974839,21.65529601)(463.86975342,21.62530273)
\curveto(463.77974857,21.60529606)(463.68974866,21.57529609)(463.59975342,21.53530273)
\curveto(463.52974882,21.51529615)(463.45974889,21.48529618)(463.38975342,21.44530273)
\curveto(463.31974903,21.40529626)(463.2547491,21.3652963)(463.19475342,21.32530273)
\curveto(462.92474943,21.15529651)(462.68974966,20.95029672)(462.48975342,20.71030273)
\curveto(462.28975006,20.4702972)(462.10475025,20.19029748)(461.93475342,19.87030273)
\curveto(461.88475047,19.7702979)(461.84475051,19.665298)(461.81475342,19.55530273)
\curveto(461.78475057,19.45529821)(461.74475061,19.35029832)(461.69475342,19.24030273)
\curveto(461.68475067,19.20029847)(461.66975068,19.13529853)(461.64975342,19.04530273)
\curveto(461.62975072,19.01529865)(461.61975073,18.98029869)(461.61975342,18.94030273)
\curveto(461.61975073,18.90029877)(461.61475074,18.85529881)(461.60475342,18.80530273)
\lineto(461.54475342,18.50530273)
\curveto(461.52475083,18.40529926)(461.51475084,18.31529935)(461.51475342,18.23530273)
\lineto(461.51475342,18.05530273)
\curveto(461.51475084,17.95529971)(461.50975084,17.85529981)(461.49975342,17.75530273)
\curveto(461.49975085,17.6653)(461.50975084,17.58030009)(461.52975342,17.50030273)
\curveto(461.57975077,17.26030041)(461.6497507,17.03530063)(461.73975342,16.82530273)
\curveto(461.83975051,16.61530105)(461.97475038,16.44030123)(462.14475342,16.30030273)
\curveto(462.19475016,16.2703014)(462.23475012,16.24530142)(462.26475342,16.22530273)
\curveto(462.30475005,16.20530146)(462.34475001,16.18030149)(462.38475342,16.15030273)
\curveto(462.4547499,16.10030157)(462.53474982,16.05530161)(462.62475342,16.01530273)
\curveto(462.71474964,15.98530168)(462.80974954,15.95530171)(462.90975342,15.92530273)
\curveto(462.95974939,15.90530176)(463.00474935,15.89530177)(463.04475342,15.89530273)
\curveto(463.09474926,15.90530176)(463.14474921,15.90530176)(463.19475342,15.89530273)
\curveto(463.22474913,15.88530178)(463.28474907,15.87530179)(463.37475342,15.86530273)
\curveto(463.46474889,15.85530181)(463.53974881,15.86030181)(463.59975342,15.88030273)
\curveto(463.63974871,15.89030178)(463.67974867,15.89030178)(463.71975342,15.88030273)
\curveto(463.75974859,15.88030179)(463.79974855,15.89030178)(463.83975342,15.91030273)
\curveto(463.91974843,15.93030174)(463.99974835,15.94530172)(464.07975342,15.95530273)
\curveto(464.16974818,15.97530169)(464.2547481,16.00030167)(464.33475342,16.03030273)
\curveto(464.69474766,16.1703015)(465.00474735,16.3653013)(465.26475342,16.61530273)
\curveto(465.52474683,16.8653008)(465.75974659,17.16030051)(465.96975342,17.50030273)
\curveto(466.0497463,17.62030005)(466.10974624,17.74529992)(466.14975342,17.87530273)
\curveto(466.18974616,18.01529965)(466.24474611,18.15529951)(466.31475342,18.29530273)
}
}
{
\newrgbcolor{curcolor}{0 0 0}
\pscustom[linestyle=none,fillstyle=solid,fillcolor=curcolor]
{
\newpath
\moveto(472.48303467,22.85530273)
\curveto(473.20302901,22.8652948)(473.78802843,22.78029489)(474.23803467,22.60030273)
\curveto(474.69802752,22.43029524)(475.0180272,22.12529554)(475.19803467,21.68530273)
\curveto(475.24802697,21.57529609)(475.27802694,21.46029621)(475.28803467,21.34030273)
\curveto(475.30802691,21.23029644)(475.32302689,21.10529656)(475.33303467,20.96530273)
\curveto(475.34302687,20.89529677)(475.33302688,20.82029685)(475.30303467,20.74030273)
\curveto(475.28302693,20.670297)(475.25802696,20.61529705)(475.22803467,20.57530273)
\curveto(475.20802701,20.55529711)(475.17802704,20.53529713)(475.13803467,20.51530273)
\curveto(475.10802711,20.50529716)(475.08302713,20.49029718)(475.06303467,20.47030273)
\curveto(475.00302721,20.45029722)(474.94802727,20.44529722)(474.89803467,20.45530273)
\curveto(474.85802736,20.4652972)(474.8130274,20.4652972)(474.76303467,20.45530273)
\curveto(474.67302754,20.43529723)(474.56302765,20.43029724)(474.43303467,20.44030273)
\curveto(474.3130279,20.46029721)(474.22802799,20.48529718)(474.17803467,20.51530273)
\curveto(474.10802811,20.5652971)(474.06802815,20.63029704)(474.05803467,20.71030273)
\curveto(474.05802816,20.80029687)(474.03802818,20.88529678)(473.99803467,20.96530273)
\curveto(473.94802827,21.12529654)(473.85302836,21.2702964)(473.71303467,21.40030273)
\curveto(473.62302859,21.48029619)(473.5130287,21.54029613)(473.38303467,21.58030273)
\curveto(473.26302895,21.62029605)(473.13302908,21.66029601)(472.99303467,21.70030273)
\curveto(472.95302926,21.72029595)(472.90302931,21.72529594)(472.84303467,21.71530273)
\curveto(472.79302942,21.71529595)(472.74802947,21.72029595)(472.70803467,21.73030273)
\curveto(472.64802957,21.75029592)(472.57302964,21.76029591)(472.48303467,21.76030273)
\curveto(472.39302982,21.76029591)(472.3180299,21.75029592)(472.25803467,21.73030273)
\lineto(472.16803467,21.73030273)
\curveto(472.10803011,21.72029595)(472.05303016,21.71029596)(472.00303467,21.70030273)
\curveto(471.95303026,21.70029597)(471.90303031,21.69529597)(471.85303467,21.68530273)
\curveto(471.58303063,21.62529604)(471.34803087,21.54029613)(471.14803467,21.43030273)
\curveto(470.95803126,21.32029635)(470.80803141,21.13529653)(470.69803467,20.87530273)
\curveto(470.66803155,20.80529686)(470.65303156,20.73529693)(470.65303467,20.66530273)
\curveto(470.65303156,20.59529707)(470.65803156,20.53529713)(470.66803467,20.48530273)
\curveto(470.69803152,20.33529733)(470.74803147,20.22529744)(470.81803467,20.15530273)
\curveto(470.88803133,20.09529757)(470.98303123,20.02529764)(471.10303467,19.94530273)
\curveto(471.24303097,19.84529782)(471.40803081,19.7702979)(471.59803467,19.72030273)
\curveto(471.78803043,19.68029799)(471.97803024,19.63029804)(472.16803467,19.57030273)
\curveto(472.28802993,19.53029814)(472.40802981,19.50029817)(472.52803467,19.48030273)
\curveto(472.65802956,19.46029821)(472.78302943,19.43029824)(472.90303467,19.39030273)
\curveto(473.10302911,19.33029834)(473.29802892,19.2702984)(473.48803467,19.21030273)
\curveto(473.67802854,19.16029851)(473.86302835,19.09529857)(474.04303467,19.01530273)
\curveto(474.09302812,18.99529867)(474.13802808,18.97529869)(474.17803467,18.95530273)
\curveto(474.22802799,18.93529873)(474.27802794,18.91029876)(474.32803467,18.88030273)
\curveto(474.49802772,18.76029891)(474.64302757,18.62529904)(474.76303467,18.47530273)
\curveto(474.88302733,18.32529934)(474.97302724,18.13529953)(475.03303467,17.90530273)
\lineto(475.03303467,17.62030273)
\curveto(475.03302718,17.55030012)(475.02802719,17.47530019)(475.01803467,17.39530273)
\curveto(475.00802721,17.32530034)(474.99802722,17.24530042)(474.98803467,17.15530273)
\lineto(474.95803467,17.00530273)
\curveto(474.9180273,16.93530073)(474.88802733,16.8653008)(474.86803467,16.79530273)
\curveto(474.85802736,16.72530094)(474.83802738,16.65530101)(474.80803467,16.58530273)
\curveto(474.75802746,16.47530119)(474.70302751,16.3703013)(474.64303467,16.27030273)
\curveto(474.58302763,16.1703015)(474.5180277,16.08030159)(474.44803467,16.00030273)
\curveto(474.23802798,15.74030193)(473.99302822,15.53030214)(473.71303467,15.37030273)
\curveto(473.43302878,15.22030245)(473.12802909,15.09030258)(472.79803467,14.98030273)
\curveto(472.69802952,14.95030272)(472.59802962,14.93030274)(472.49803467,14.92030273)
\curveto(472.39802982,14.90030277)(472.30302991,14.87530279)(472.21303467,14.84530273)
\curveto(472.10303011,14.82530284)(471.99803022,14.81530285)(471.89803467,14.81530273)
\curveto(471.79803042,14.81530285)(471.69803052,14.80530286)(471.59803467,14.78530273)
\lineto(471.44803467,14.78530273)
\curveto(471.39803082,14.77530289)(471.32803089,14.7703029)(471.23803467,14.77030273)
\curveto(471.14803107,14.7703029)(471.07803114,14.77530289)(471.02803467,14.78530273)
\lineto(470.86303467,14.78530273)
\curveto(470.80303141,14.80530286)(470.73803148,14.81530285)(470.66803467,14.81530273)
\curveto(470.59803162,14.80530286)(470.54303167,14.81030286)(470.50303467,14.83030273)
\curveto(470.45303176,14.84030283)(470.38803183,14.84530282)(470.30803467,14.84530273)
\curveto(470.22803199,14.8653028)(470.15303206,14.88530278)(470.08303467,14.90530273)
\curveto(470.0130322,14.91530275)(469.93803228,14.93530273)(469.85803467,14.96530273)
\curveto(469.56803265,15.0653026)(469.32303289,15.19030248)(469.12303467,15.34030273)
\curveto(468.92303329,15.49030218)(468.76303345,15.68530198)(468.64303467,15.92530273)
\curveto(468.58303363,16.05530161)(468.53303368,16.19030148)(468.49303467,16.33030273)
\curveto(468.46303375,16.4703012)(468.44303377,16.62530104)(468.43303467,16.79530273)
\curveto(468.42303379,16.85530081)(468.42803379,16.92530074)(468.44803467,17.00530273)
\curveto(468.46803375,17.09530057)(468.49303372,17.1653005)(468.52303467,17.21530273)
\curveto(468.56303365,17.25530041)(468.62303359,17.29530037)(468.70303467,17.33530273)
\curveto(468.75303346,17.35530031)(468.82303339,17.3653003)(468.91303467,17.36530273)
\curveto(469.0130332,17.37530029)(469.10303311,17.37530029)(469.18303467,17.36530273)
\curveto(469.27303294,17.35530031)(469.35803286,17.34030033)(469.43803467,17.32030273)
\curveto(469.52803269,17.31030036)(469.58303263,17.29530037)(469.60303467,17.27530273)
\curveto(469.66303255,17.22530044)(469.69303252,17.15030052)(469.69303467,17.05030273)
\curveto(469.70303251,16.96030071)(469.72303249,16.87530079)(469.75303467,16.79530273)
\curveto(469.80303241,16.57530109)(469.90303231,16.40530126)(470.05303467,16.28530273)
\curveto(470.15303206,16.19530147)(470.27303194,16.12530154)(470.41303467,16.07530273)
\curveto(470.55303166,16.02530164)(470.70303151,15.97530169)(470.86303467,15.92530273)
\lineto(471.17803467,15.88030273)
\lineto(471.26803467,15.88030273)
\curveto(471.32803089,15.86030181)(471.4130308,15.85030182)(471.52303467,15.85030273)
\curveto(471.64303057,15.85030182)(471.74803047,15.86030181)(471.83803467,15.88030273)
\curveto(471.90803031,15.88030179)(471.96303025,15.88530178)(472.00303467,15.89530273)
\curveto(472.06303015,15.90530176)(472.12303009,15.91030176)(472.18303467,15.91030273)
\curveto(472.24302997,15.92030175)(472.29802992,15.93030174)(472.34803467,15.94030273)
\curveto(472.65802956,16.02030165)(472.90802931,16.12530154)(473.09803467,16.25530273)
\curveto(473.29802892,16.38530128)(473.46302875,16.60530106)(473.59303467,16.91530273)
\curveto(473.62302859,16.9653007)(473.63802858,17.02030065)(473.63803467,17.08030273)
\curveto(473.64802857,17.14030053)(473.64802857,17.18530048)(473.63803467,17.21530273)
\curveto(473.62802859,17.40530026)(473.58802863,17.54530012)(473.51803467,17.63530273)
\curveto(473.44802877,17.73529993)(473.35302886,17.82529984)(473.23303467,17.90530273)
\curveto(473.15302906,17.9652997)(473.05802916,18.01529965)(472.94803467,18.05530273)
\lineto(472.64803467,18.17530273)
\curveto(472.6180296,18.18529948)(472.58802963,18.19029948)(472.55803467,18.19030273)
\curveto(472.53802968,18.19029948)(472.5180297,18.20029947)(472.49803467,18.22030273)
\curveto(472.17803004,18.33029934)(471.83803038,18.41029926)(471.47803467,18.46030273)
\curveto(471.12803109,18.52029915)(470.80803141,18.61529905)(470.51803467,18.74530273)
\curveto(470.42803179,18.78529888)(470.33803188,18.82029885)(470.24803467,18.85030273)
\curveto(470.16803205,18.88029879)(470.09303212,18.92029875)(470.02303467,18.97030273)
\curveto(469.85303236,19.08029859)(469.70303251,19.20529846)(469.57303467,19.34530273)
\curveto(469.44303277,19.48529818)(469.35303286,19.66029801)(469.30303467,19.87030273)
\curveto(469.28303293,19.94029773)(469.27303294,20.01029766)(469.27303467,20.08030273)
\lineto(469.27303467,20.30530273)
\curveto(469.26303295,20.42529724)(469.27803294,20.56029711)(469.31803467,20.71030273)
\curveto(469.35803286,20.8702968)(469.39803282,21.00529666)(469.43803467,21.11530273)
\curveto(469.46803275,21.1652965)(469.48803273,21.20529646)(469.49803467,21.23530273)
\curveto(469.5180327,21.27529639)(469.54303267,21.31529635)(469.57303467,21.35530273)
\curveto(469.70303251,21.58529608)(469.86303235,21.78529588)(470.05303467,21.95530273)
\curveto(470.24303197,22.12529554)(470.45303176,22.27529539)(470.68303467,22.40530273)
\curveto(470.84303137,22.49529517)(471.0180312,22.5652951)(471.20803467,22.61530273)
\curveto(471.40803081,22.67529499)(471.6130306,22.73029494)(471.82303467,22.78030273)
\curveto(471.89303032,22.79029488)(471.95803026,22.80029487)(472.01803467,22.81030273)
\curveto(472.08803013,22.82029485)(472.16303005,22.83029484)(472.24303467,22.84030273)
\curveto(472.28302993,22.85029482)(472.32302989,22.85029482)(472.36303467,22.84030273)
\curveto(472.4130298,22.83029484)(472.45302976,22.83529483)(472.48303467,22.85530273)
}
}
{
\newrgbcolor{curcolor}{0 0 0}
\pscustom[linewidth=1,linecolor=curcolor]
{
\newpath
\moveto(89.01786,85.52252)
\lineto(726.99776,85.52252)
}
}
{
\newrgbcolor{curcolor}{0 0 0}
\pscustom[linewidth=1,linecolor=curcolor]
{
\newpath
\moveto(89.01786,159.51623)
\lineto(726.99776,159.51623)
}
}
{
\newrgbcolor{curcolor}{0 0 0}
\pscustom[linewidth=1,linecolor=curcolor]
{
\newpath
\moveto(89.01786,233.55822)
\lineto(726.99776,233.55822)
}
}
{
\newrgbcolor{curcolor}{0 0 0}
\pscustom[linewidth=1,linecolor=curcolor]
{
\newpath
\moveto(89.01786,308.62335)
\lineto(726.99776,308.62335)
}
}
{
\newrgbcolor{curcolor}{0 0 0}
\pscustom[linewidth=1,linecolor=curcolor]
{
\newpath
\moveto(89.01786,382.589017)
\lineto(726.99776,382.589017)
}
}
{
\newrgbcolor{curcolor}{0 0 0}
\pscustom[linestyle=none,fillstyle=solid,fillcolor=curcolor]
{
\newpath
\moveto(94.64285278,419.08210297)
\curveto(95.69284611,419.102092)(96.55784524,418.92209218)(97.23785278,418.54210297)
\curveto(97.91784388,418.16209294)(98.45784334,417.65709344)(98.85785278,417.02710297)
\curveto(98.96784283,416.85709424)(99.05784274,416.68209442)(99.12785278,416.50210297)
\curveto(99.1978426,416.33209477)(99.26284254,416.14209496)(99.32285278,415.93210297)
\curveto(99.34284246,415.86209524)(99.36284244,415.78209532)(99.38285278,415.69210297)
\curveto(99.4028424,415.6020955)(99.3978424,415.51709558)(99.36785278,415.43710297)
\curveto(99.34784245,415.37709572)(99.30784249,415.33709576)(99.24785278,415.31710297)
\curveto(99.1978426,415.30709579)(99.13784266,415.29209581)(99.06785278,415.27210297)
\lineto(98.94785278,415.27210297)
\curveto(98.88784291,415.25209585)(98.81784298,415.24209586)(98.73785278,415.24210297)
\curveto(98.66784313,415.25209585)(98.5978432,415.25709584)(98.52785278,415.25710297)
\curveto(98.43784336,415.25709584)(98.32784347,415.25209585)(98.19785278,415.24210297)
\lineto(97.83785278,415.24210297)
\curveto(97.71784408,415.25209585)(97.60784419,415.26209584)(97.50785278,415.27210297)
\curveto(97.40784439,415.29209581)(97.33284447,415.31709578)(97.28285278,415.34710297)
\curveto(97.2028446,415.41709568)(97.14284466,415.51209559)(97.10285278,415.63210297)
\curveto(97.06284474,415.75209535)(97.01284479,415.85709524)(96.95285278,415.94710297)
\curveto(96.76284504,416.27709482)(96.51284529,416.53709456)(96.20285278,416.72710297)
\curveto(95.99284581,416.85709424)(95.75784604,416.96209414)(95.49785278,417.04210297)
\curveto(95.32784647,417.102094)(95.11284669,417.13209397)(94.85285278,417.13210297)
\curveto(94.6028472,417.13209397)(94.38284742,417.10709399)(94.19285278,417.05710297)
\curveto(94.11284769,417.03709406)(94.03784776,417.01709408)(93.96785278,416.99710297)
\curveto(93.90784789,416.98709411)(93.84284796,416.96709413)(93.77285278,416.93710297)
\curveto(93.09284871,416.64709445)(92.61284919,416.16709493)(92.33285278,415.49710297)
\curveto(92.28284952,415.37709572)(92.23784956,415.25209585)(92.19785278,415.12210297)
\curveto(92.15784964,414.99209611)(92.11284969,414.85709624)(92.06285278,414.71710297)
\curveto(92.05284975,414.64709645)(92.04284976,414.58209652)(92.03285278,414.52210297)
\curveto(92.02284978,414.46209664)(92.01284979,414.3970967)(92.00285278,414.32710297)
\curveto(91.98284982,414.26709683)(91.97284983,414.2020969)(91.97285278,414.13210297)
\curveto(91.98284982,414.07209703)(91.97784982,414.00709709)(91.95785278,413.93710297)
\curveto(91.93784986,413.85709724)(91.92784987,413.77209733)(91.92785278,413.68210297)
\curveto(91.93784986,413.6020975)(91.94284986,413.52209758)(91.94285278,413.44210297)
\curveto(91.94284986,413.4020977)(91.93784986,413.36209774)(91.92785278,413.32210297)
\curveto(91.92784987,413.28209782)(91.93284987,413.24209786)(91.94285278,413.20210297)
\lineto(91.94285278,413.06710297)
\curveto(91.96284984,413.01709808)(91.96784983,412.96709813)(91.95785278,412.91710297)
\curveto(91.95784984,412.86709823)(91.96784983,412.81709828)(91.98785278,412.76710297)
\curveto(91.98784981,412.70709839)(91.9978498,412.62709847)(92.01785278,412.52710297)
\curveto(92.03784976,412.41709868)(92.05784974,412.31209879)(92.07785278,412.21210297)
\curveto(92.0978497,412.12209898)(92.12284968,412.03209907)(92.15285278,411.94210297)
\curveto(92.29284951,411.52209958)(92.47284933,411.16209994)(92.69285278,410.86210297)
\curveto(92.91284889,410.57210053)(93.2028486,410.33210077)(93.56285278,410.14210297)
\curveto(93.67284813,410.09210101)(93.78784801,410.04710105)(93.90785278,410.00710297)
\curveto(94.02784777,409.97710112)(94.15784764,409.94210116)(94.29785278,409.90210297)
\curveto(94.34784745,409.89210121)(94.39284741,409.88210122)(94.43285278,409.87210297)
\lineto(94.58285278,409.87210297)
\lineto(94.70285278,409.87210297)
\curveto(94.75284705,409.85210125)(94.81784698,409.84210126)(94.89785278,409.84210297)
\curveto(94.97784682,409.85210125)(95.04284676,409.86210124)(95.09285278,409.87210297)
\curveto(95.15284665,409.87210123)(95.1978466,409.87710122)(95.22785278,409.88710297)
\curveto(95.34784645,409.90710119)(95.45784634,409.92710117)(95.55785278,409.94710297)
\curveto(95.66784613,409.96710113)(95.77284603,410.0021011)(95.87285278,410.05210297)
\curveto(96.1028457,410.15210095)(96.2978455,410.28210082)(96.45785278,410.44210297)
\curveto(96.62784517,410.61210049)(96.77784502,410.8021003)(96.90785278,411.01210297)
\curveto(96.96784483,411.11209999)(97.01784478,411.22209988)(97.05785278,411.34210297)
\curveto(97.0978447,411.46209964)(97.14284466,411.58209952)(97.19285278,411.70210297)
\curveto(97.22284458,411.81209929)(97.25284455,411.91209919)(97.28285278,412.00210297)
\curveto(97.31284449,412.09209901)(97.37784442,412.16209894)(97.47785278,412.21210297)
\curveto(97.53784426,412.23209887)(97.61284419,412.24209886)(97.70285278,412.24210297)
\lineto(97.95785278,412.24210297)
\lineto(98.87285278,412.24210297)
\lineto(99.14285278,412.24210297)
\curveto(99.24284256,412.24209886)(99.32284248,412.22209888)(99.38285278,412.18210297)
\curveto(99.45284235,412.13209897)(99.48784231,412.05209905)(99.48785278,411.94210297)
\curveto(99.4978423,411.83209927)(99.48784231,411.72709937)(99.45785278,411.62710297)
\lineto(99.36785278,411.22210297)
\curveto(99.31784248,411.07210003)(99.26784253,410.92710017)(99.21785278,410.78710297)
\curveto(99.17784262,410.64710045)(99.12284268,410.51210059)(99.05285278,410.38210297)
\curveto(99.0028428,410.3021008)(98.96284284,410.22210088)(98.93285278,410.14210297)
\curveto(98.9028429,410.07210103)(98.86284294,410.0021011)(98.81285278,409.93210297)
\curveto(98.26284354,409.07210203)(97.46284434,408.47210263)(96.41285278,408.13210297)
\curveto(96.3028455,408.09210301)(96.19284561,408.06210304)(96.08285278,408.04210297)
\lineto(95.75285278,407.98210297)
\curveto(95.7028461,407.96210314)(95.65284615,407.95710314)(95.60285278,407.96710297)
\curveto(95.56284624,407.96710313)(95.51784628,407.95710314)(95.46785278,407.93710297)
\lineto(95.25785278,407.93710297)
\curveto(95.1978466,407.92710317)(95.13284667,407.91710318)(95.06285278,407.90710297)
\lineto(94.82285278,407.90710297)
\lineto(94.55285278,407.90710297)
\curveto(94.46284734,407.90710319)(94.37784742,407.91710318)(94.29785278,407.93710297)
\curveto(94.26784753,407.94710315)(94.21784758,407.95210315)(94.14785278,407.95210297)
\lineto(93.93785278,407.98210297)
\curveto(93.86784793,407.99210311)(93.79284801,408.0021031)(93.71285278,408.01210297)
\curveto(93.61284819,408.04210306)(93.51284829,408.06710303)(93.41285278,408.08710297)
\curveto(93.32284848,408.10710299)(93.22784857,408.13210297)(93.12785278,408.16210297)
\lineto(92.85785278,408.25210297)
\curveto(92.76784903,408.29210281)(92.68284912,408.33210277)(92.60285278,408.37210297)
\curveto(92.0028498,408.63210247)(91.49285031,408.97710212)(91.07285278,409.40710297)
\curveto(90.66285114,409.84710125)(90.32785147,410.36710073)(90.06785278,410.96710297)
\curveto(90.00785179,411.11709998)(89.95785184,411.26709983)(89.91785278,411.41710297)
\curveto(89.87785192,411.56709953)(89.83285197,411.72209938)(89.78285278,411.88210297)
\curveto(89.76285204,411.93209917)(89.75285205,411.97209913)(89.75285278,412.00210297)
\curveto(89.75285205,412.04209906)(89.74785205,412.08209902)(89.73785278,412.12210297)
\curveto(89.71785208,412.21209889)(89.7028521,412.30709879)(89.69285278,412.40710297)
\curveto(89.68285212,412.50709859)(89.66785213,412.6020985)(89.64785278,412.69210297)
\curveto(89.62785217,412.74209836)(89.61785218,412.78209832)(89.61785278,412.81210297)
\curveto(89.62785217,412.85209825)(89.62785217,412.89209821)(89.61785278,412.93210297)
\lineto(89.61785278,413.21710297)
\curveto(89.5978522,413.26709783)(89.58785221,413.34209776)(89.58785278,413.44210297)
\curveto(89.58785221,413.54209756)(89.5978522,413.61709748)(89.61785278,413.66710297)
\curveto(89.62785217,413.6970974)(89.62785217,413.72709737)(89.61785278,413.75710297)
\lineto(89.61785278,413.84710297)
\lineto(89.61785278,413.98210297)
\curveto(89.63785216,414.06209704)(89.64785215,414.14709695)(89.64785278,414.23710297)
\lineto(89.67785278,414.50710297)
\curveto(89.6978521,414.58709651)(89.71285209,414.66209644)(89.72285278,414.73210297)
\curveto(89.73285207,414.81209629)(89.74785205,414.89209621)(89.76785278,414.97210297)
\curveto(89.80785199,415.11209599)(89.84285196,415.24709585)(89.87285278,415.37710297)
\curveto(89.9028519,415.51709558)(89.94285186,415.64709545)(89.99285278,415.76710297)
\lineto(90.14285278,416.15710297)
\curveto(90.2028516,416.28709481)(90.26785153,416.40709469)(90.33785278,416.51710297)
\curveto(90.4978513,416.7970943)(90.66285114,417.04709405)(90.83285278,417.26710297)
\curveto(90.88285092,417.33709376)(90.93785086,417.4020937)(90.99785278,417.46210297)
\lineto(91.17785278,417.64210297)
\curveto(91.42785037,417.89209321)(91.68785011,418.102093)(91.95785278,418.27210297)
\curveto(92.23784956,418.45209265)(92.55284925,418.61209249)(92.90285278,418.75210297)
\curveto(93.02284878,418.8020923)(93.14784865,418.84209226)(93.27785278,418.87210297)
\curveto(93.40784839,418.91209219)(93.54284826,418.94709215)(93.68285278,418.97710297)
\curveto(93.74284806,418.9970921)(93.802848,419.00709209)(93.86285278,419.00710297)
\curveto(93.92284788,419.00709209)(93.97784782,419.01709208)(94.02785278,419.03710297)
\curveto(94.10784769,419.04709205)(94.18284762,419.05209205)(94.25285278,419.05210297)
\curveto(94.33284747,419.06209204)(94.41284739,419.07209203)(94.49285278,419.08210297)
\curveto(94.51284729,419.09209201)(94.53784726,419.09209201)(94.56785278,419.08210297)
\curveto(94.5978472,419.07209203)(94.62284718,419.07209203)(94.64285278,419.08210297)
}
}
{
\newrgbcolor{curcolor}{0 0 0}
\pscustom[linestyle=none,fillstyle=solid,fillcolor=curcolor]
{
\newpath
\moveto(107.95136841,408.73210297)
\curveto(107.97136056,408.62210248)(107.98136055,408.51210259)(107.98136841,408.40210297)
\curveto(107.99136054,408.29210281)(107.94136059,408.21710288)(107.83136841,408.17710297)
\curveto(107.77136076,408.14710295)(107.70136083,408.13210297)(107.62136841,408.13210297)
\lineto(107.38136841,408.13210297)
\lineto(106.57136841,408.13210297)
\lineto(106.30136841,408.13210297)
\curveto(106.22136231,408.14210296)(106.15636237,408.16710293)(106.10636841,408.20710297)
\curveto(106.03636249,408.24710285)(105.98136255,408.3021028)(105.94136841,408.37210297)
\curveto(105.91136262,408.45210265)(105.86636266,408.51710258)(105.80636841,408.56710297)
\curveto(105.78636274,408.58710251)(105.76136277,408.6021025)(105.73136841,408.61210297)
\curveto(105.70136283,408.63210247)(105.66136287,408.63710246)(105.61136841,408.62710297)
\curveto(105.56136297,408.60710249)(105.51136302,408.58210252)(105.46136841,408.55210297)
\curveto(105.42136311,408.52210258)(105.37636315,408.4971026)(105.32636841,408.47710297)
\curveto(105.27636325,408.43710266)(105.22136331,408.4021027)(105.16136841,408.37210297)
\lineto(104.98136841,408.28210297)
\curveto(104.85136368,408.22210288)(104.71636381,408.17210293)(104.57636841,408.13210297)
\curveto(104.43636409,408.102103)(104.29136424,408.06710303)(104.14136841,408.02710297)
\curveto(104.07136446,408.00710309)(104.00136453,407.9971031)(103.93136841,407.99710297)
\curveto(103.87136466,407.98710311)(103.80636472,407.97710312)(103.73636841,407.96710297)
\lineto(103.64636841,407.96710297)
\curveto(103.61636491,407.95710314)(103.58636494,407.95210315)(103.55636841,407.95210297)
\lineto(103.39136841,407.95210297)
\curveto(103.29136524,407.93210317)(103.19136534,407.93210317)(103.09136841,407.95210297)
\lineto(102.95636841,407.95210297)
\curveto(102.88636564,407.97210313)(102.81636571,407.98210312)(102.74636841,407.98210297)
\curveto(102.68636584,407.97210313)(102.6263659,407.97710312)(102.56636841,407.99710297)
\curveto(102.46636606,408.01710308)(102.37136616,408.03710306)(102.28136841,408.05710297)
\curveto(102.19136634,408.06710303)(102.10636642,408.09210301)(102.02636841,408.13210297)
\curveto(101.73636679,408.24210286)(101.48636704,408.38210272)(101.27636841,408.55210297)
\curveto(101.07636745,408.73210237)(100.91636761,408.96710213)(100.79636841,409.25710297)
\curveto(100.76636776,409.32710177)(100.73636779,409.4021017)(100.70636841,409.48210297)
\curveto(100.68636784,409.56210154)(100.66636786,409.64710145)(100.64636841,409.73710297)
\curveto(100.6263679,409.78710131)(100.61636791,409.83710126)(100.61636841,409.88710297)
\curveto(100.6263679,409.93710116)(100.6263679,409.98710111)(100.61636841,410.03710297)
\curveto(100.60636792,410.06710103)(100.59636793,410.12710097)(100.58636841,410.21710297)
\curveto(100.58636794,410.31710078)(100.59136794,410.38710071)(100.60136841,410.42710297)
\curveto(100.62136791,410.52710057)(100.6313679,410.61210049)(100.63136841,410.68210297)
\lineto(100.72136841,411.01210297)
\curveto(100.75136778,411.13209997)(100.79136774,411.23709986)(100.84136841,411.32710297)
\curveto(101.01136752,411.61709948)(101.20636732,411.83709926)(101.42636841,411.98710297)
\curveto(101.64636688,412.13709896)(101.9263666,412.26709883)(102.26636841,412.37710297)
\curveto(102.39636613,412.42709867)(102.531366,412.46209864)(102.67136841,412.48210297)
\curveto(102.81136572,412.5020986)(102.95136558,412.52709857)(103.09136841,412.55710297)
\curveto(103.17136536,412.57709852)(103.25636527,412.58709851)(103.34636841,412.58710297)
\curveto(103.43636509,412.5970985)(103.526365,412.61209849)(103.61636841,412.63210297)
\curveto(103.68636484,412.65209845)(103.75636477,412.65709844)(103.82636841,412.64710297)
\curveto(103.89636463,412.64709845)(103.97136456,412.65709844)(104.05136841,412.67710297)
\curveto(104.12136441,412.6970984)(104.19136434,412.70709839)(104.26136841,412.70710297)
\curveto(104.3313642,412.70709839)(104.40636412,412.71709838)(104.48636841,412.73710297)
\curveto(104.69636383,412.78709831)(104.88636364,412.82709827)(105.05636841,412.85710297)
\curveto(105.23636329,412.8970982)(105.39636313,412.98709811)(105.53636841,413.12710297)
\curveto(105.6263629,413.21709788)(105.68636284,413.31709778)(105.71636841,413.42710297)
\curveto(105.7263628,413.45709764)(105.7263628,413.48209762)(105.71636841,413.50210297)
\curveto(105.71636281,413.52209758)(105.72136281,413.54209756)(105.73136841,413.56210297)
\curveto(105.74136279,413.58209752)(105.74636278,413.61209749)(105.74636841,413.65210297)
\lineto(105.74636841,413.74210297)
\lineto(105.71636841,413.86210297)
\curveto(105.71636281,413.9020972)(105.71136282,413.93709716)(105.70136841,413.96710297)
\curveto(105.60136293,414.26709683)(105.39136314,414.47209663)(105.07136841,414.58210297)
\curveto(104.98136355,414.61209649)(104.87136366,414.63209647)(104.74136841,414.64210297)
\curveto(104.62136391,414.66209644)(104.49636403,414.66709643)(104.36636841,414.65710297)
\curveto(104.23636429,414.65709644)(104.11136442,414.64709645)(103.99136841,414.62710297)
\curveto(103.87136466,414.60709649)(103.76636476,414.58209652)(103.67636841,414.55210297)
\curveto(103.61636491,414.53209657)(103.55636497,414.5020966)(103.49636841,414.46210297)
\curveto(103.44636508,414.43209667)(103.39636513,414.3970967)(103.34636841,414.35710297)
\curveto(103.29636523,414.31709678)(103.24136529,414.26209684)(103.18136841,414.19210297)
\curveto(103.1313654,414.12209698)(103.09636543,414.05709704)(103.07636841,413.99710297)
\curveto(103.0263655,413.8970972)(102.98136555,413.8020973)(102.94136841,413.71210297)
\curveto(102.91136562,413.62209748)(102.84136569,413.56209754)(102.73136841,413.53210297)
\curveto(102.65136588,413.51209759)(102.56636596,413.5020976)(102.47636841,413.50210297)
\lineto(102.20636841,413.50210297)
\lineto(101.63636841,413.50210297)
\curveto(101.58636694,413.5020976)(101.53636699,413.4970976)(101.48636841,413.48710297)
\curveto(101.43636709,413.48709761)(101.39136714,413.49209761)(101.35136841,413.50210297)
\lineto(101.21636841,413.50210297)
\curveto(101.19636733,413.51209759)(101.17136736,413.51709758)(101.14136841,413.51710297)
\curveto(101.11136742,413.51709758)(101.08636744,413.52709757)(101.06636841,413.54710297)
\curveto(100.98636754,413.56709753)(100.9313676,413.63209747)(100.90136841,413.74210297)
\curveto(100.89136764,413.79209731)(100.89136764,413.84209726)(100.90136841,413.89210297)
\curveto(100.91136762,413.94209716)(100.92136761,413.98709711)(100.93136841,414.02710297)
\curveto(100.96136757,414.13709696)(100.99136754,414.23709686)(101.02136841,414.32710297)
\curveto(101.06136747,414.42709667)(101.10636742,414.51709658)(101.15636841,414.59710297)
\lineto(101.24636841,414.74710297)
\lineto(101.33636841,414.89710297)
\curveto(101.41636711,415.00709609)(101.51636701,415.11209599)(101.63636841,415.21210297)
\curveto(101.65636687,415.22209588)(101.68636684,415.24709585)(101.72636841,415.28710297)
\curveto(101.77636675,415.32709577)(101.82136671,415.36209574)(101.86136841,415.39210297)
\curveto(101.90136663,415.42209568)(101.94636658,415.45209565)(101.99636841,415.48210297)
\curveto(102.16636636,415.59209551)(102.34636618,415.67709542)(102.53636841,415.73710297)
\curveto(102.7263658,415.80709529)(102.92136561,415.87209523)(103.12136841,415.93210297)
\curveto(103.24136529,415.96209514)(103.36636516,415.98209512)(103.49636841,415.99210297)
\curveto(103.6263649,416.0020951)(103.75636477,416.02209508)(103.88636841,416.05210297)
\curveto(103.9263646,416.06209504)(103.98636454,416.06209504)(104.06636841,416.05210297)
\curveto(104.15636437,416.04209506)(104.21136432,416.04709505)(104.23136841,416.06710297)
\curveto(104.64136389,416.07709502)(105.0313635,416.06209504)(105.40136841,416.02210297)
\curveto(105.78136275,415.98209512)(106.12136241,415.90709519)(106.42136841,415.79710297)
\curveto(106.7313618,415.68709541)(106.99636153,415.53709556)(107.21636841,415.34710297)
\curveto(107.43636109,415.16709593)(107.60636092,414.93209617)(107.72636841,414.64210297)
\curveto(107.79636073,414.47209663)(107.83636069,414.27709682)(107.84636841,414.05710297)
\curveto(107.85636067,413.83709726)(107.86136067,413.61209749)(107.86136841,413.38210297)
\lineto(107.86136841,410.03710297)
\lineto(107.86136841,409.45210297)
\curveto(107.86136067,409.26210184)(107.88136065,409.08710201)(107.92136841,408.92710297)
\curveto(107.9313606,408.8971022)(107.93636059,408.86210224)(107.93636841,408.82210297)
\curveto(107.93636059,408.79210231)(107.94136059,408.76210234)(107.95136841,408.73210297)
\moveto(105.74636841,411.04210297)
\curveto(105.75636277,411.09210001)(105.76136277,411.14709995)(105.76136841,411.20710297)
\curveto(105.76136277,411.27709982)(105.75636277,411.33709976)(105.74636841,411.38710297)
\curveto(105.7263628,411.44709965)(105.71636281,411.5020996)(105.71636841,411.55210297)
\curveto(105.71636281,411.6020995)(105.69636283,411.64209946)(105.65636841,411.67210297)
\curveto(105.60636292,411.71209939)(105.531363,411.73209937)(105.43136841,411.73210297)
\curveto(105.39136314,411.72209938)(105.35636317,411.71209939)(105.32636841,411.70210297)
\curveto(105.29636323,411.7020994)(105.26136327,411.6970994)(105.22136841,411.68710297)
\curveto(105.15136338,411.66709943)(105.07636345,411.65209945)(104.99636841,411.64210297)
\curveto(104.91636361,411.63209947)(104.83636369,411.61709948)(104.75636841,411.59710297)
\curveto(104.7263638,411.58709951)(104.68136385,411.58209952)(104.62136841,411.58210297)
\curveto(104.49136404,411.55209955)(104.36136417,411.53209957)(104.23136841,411.52210297)
\curveto(104.10136443,411.51209959)(103.97636455,411.48709961)(103.85636841,411.44710297)
\curveto(103.77636475,411.42709967)(103.70136483,411.40709969)(103.63136841,411.38710297)
\curveto(103.56136497,411.37709972)(103.49136504,411.35709974)(103.42136841,411.32710297)
\curveto(103.21136532,411.23709986)(103.0313655,411.1021)(102.88136841,410.92210297)
\curveto(102.74136579,410.74210036)(102.69136584,410.49210061)(102.73136841,410.17210297)
\curveto(102.75136578,410.0021011)(102.80636572,409.86210124)(102.89636841,409.75210297)
\curveto(102.96636556,409.64210146)(103.07136546,409.55210155)(103.21136841,409.48210297)
\curveto(103.35136518,409.42210168)(103.50136503,409.37710172)(103.66136841,409.34710297)
\curveto(103.8313647,409.31710178)(104.00636452,409.30710179)(104.18636841,409.31710297)
\curveto(104.37636415,409.33710176)(104.55136398,409.37210173)(104.71136841,409.42210297)
\curveto(104.97136356,409.5021016)(105.17636335,409.62710147)(105.32636841,409.79710297)
\curveto(105.47636305,409.97710112)(105.59136294,410.1971009)(105.67136841,410.45710297)
\curveto(105.69136284,410.52710057)(105.70136283,410.5971005)(105.70136841,410.66710297)
\curveto(105.71136282,410.74710035)(105.7263628,410.82710027)(105.74636841,410.90710297)
\lineto(105.74636841,411.04210297)
}
}
{
\newrgbcolor{curcolor}{0 0 0}
\pscustom[linestyle=none,fillstyle=solid,fillcolor=curcolor]
{
\newpath
\moveto(113.93964966,416.06710297)
\curveto(114.53964385,416.08709501)(115.03964335,416.0020951)(115.43964966,415.81210297)
\curveto(115.83964255,415.62209548)(116.15464224,415.34209576)(116.38464966,414.97210297)
\curveto(116.45464194,414.86209624)(116.50964188,414.74209636)(116.54964966,414.61210297)
\curveto(116.5896418,414.49209661)(116.62964176,414.36709673)(116.66964966,414.23710297)
\curveto(116.6896417,414.15709694)(116.69964169,414.08209702)(116.69964966,414.01210297)
\curveto(116.70964168,413.94209716)(116.72464167,413.87209723)(116.74464966,413.80210297)
\curveto(116.74464165,413.74209736)(116.74964164,413.7020974)(116.75964966,413.68210297)
\curveto(116.77964161,413.54209756)(116.7896416,413.3970977)(116.78964966,413.24710297)
\lineto(116.78964966,412.81210297)
\lineto(116.78964966,411.47710297)
\lineto(116.78964966,409.04710297)
\curveto(116.7896416,408.85710224)(116.78464161,408.67210243)(116.77464966,408.49210297)
\curveto(116.77464162,408.32210278)(116.70464169,408.21210289)(116.56464966,408.16210297)
\curveto(116.50464189,408.14210296)(116.43464196,408.13210297)(116.35464966,408.13210297)
\lineto(116.11464966,408.13210297)
\lineto(115.30464966,408.13210297)
\curveto(115.18464321,408.13210297)(115.07464332,408.13710296)(114.97464966,408.14710297)
\curveto(114.88464351,408.16710293)(114.81464358,408.21210289)(114.76464966,408.28210297)
\curveto(114.72464367,408.34210276)(114.69964369,408.41710268)(114.68964966,408.50710297)
\lineto(114.68964966,408.82210297)
\lineto(114.68964966,409.87210297)
\lineto(114.68964966,412.10710297)
\curveto(114.6896437,412.47709862)(114.67464372,412.81709828)(114.64464966,413.12710297)
\curveto(114.61464378,413.44709765)(114.52464387,413.71709738)(114.37464966,413.93710297)
\curveto(114.23464416,414.13709696)(114.02964436,414.27709682)(113.75964966,414.35710297)
\curveto(113.70964468,414.37709672)(113.65464474,414.38709671)(113.59464966,414.38710297)
\curveto(113.54464485,414.38709671)(113.4896449,414.3970967)(113.42964966,414.41710297)
\curveto(113.37964501,414.42709667)(113.31464508,414.42709667)(113.23464966,414.41710297)
\curveto(113.16464523,414.41709668)(113.10964528,414.41209669)(113.06964966,414.40210297)
\curveto(113.02964536,414.39209671)(112.9946454,414.38709671)(112.96464966,414.38710297)
\curveto(112.93464546,414.38709671)(112.90464549,414.38209672)(112.87464966,414.37210297)
\curveto(112.64464575,414.31209679)(112.45964593,414.23209687)(112.31964966,414.13210297)
\curveto(111.99964639,413.9020972)(111.80964658,413.56709753)(111.74964966,413.12710297)
\curveto(111.6896467,412.68709841)(111.65964673,412.19209891)(111.65964966,411.64210297)
\lineto(111.65964966,409.76710297)
\lineto(111.65964966,408.85210297)
\lineto(111.65964966,408.58210297)
\curveto(111.65964673,408.49210261)(111.64464675,408.41710268)(111.61464966,408.35710297)
\curveto(111.56464683,408.24710285)(111.48464691,408.18210292)(111.37464966,408.16210297)
\curveto(111.26464713,408.14210296)(111.12964726,408.13210297)(110.96964966,408.13210297)
\lineto(110.21964966,408.13210297)
\curveto(110.10964828,408.13210297)(109.99964839,408.13710296)(109.88964966,408.14710297)
\curveto(109.77964861,408.15710294)(109.69964869,408.19210291)(109.64964966,408.25210297)
\curveto(109.57964881,408.34210276)(109.54464885,408.47210263)(109.54464966,408.64210297)
\curveto(109.55464884,408.81210229)(109.55964883,408.97210213)(109.55964966,409.12210297)
\lineto(109.55964966,411.16210297)
\lineto(109.55964966,414.46210297)
\lineto(109.55964966,415.22710297)
\lineto(109.55964966,415.52710297)
\curveto(109.56964882,415.61709548)(109.59964879,415.69209541)(109.64964966,415.75210297)
\curveto(109.66964872,415.78209532)(109.69964869,415.8020953)(109.73964966,415.81210297)
\curveto(109.7896486,415.83209527)(109.83964855,415.84709525)(109.88964966,415.85710297)
\lineto(109.96464966,415.85710297)
\curveto(110.01464838,415.86709523)(110.06464833,415.87209523)(110.11464966,415.87210297)
\lineto(110.27964966,415.87210297)
\lineto(110.90964966,415.87210297)
\curveto(110.9896474,415.87209523)(111.06464733,415.86709523)(111.13464966,415.85710297)
\curveto(111.21464718,415.85709524)(111.28464711,415.84709525)(111.34464966,415.82710297)
\curveto(111.41464698,415.7970953)(111.45964693,415.75209535)(111.47964966,415.69210297)
\curveto(111.50964688,415.63209547)(111.53464686,415.56209554)(111.55464966,415.48210297)
\curveto(111.56464683,415.44209566)(111.56464683,415.40709569)(111.55464966,415.37710297)
\curveto(111.55464684,415.34709575)(111.56464683,415.31709578)(111.58464966,415.28710297)
\curveto(111.60464679,415.23709586)(111.61964677,415.20709589)(111.62964966,415.19710297)
\curveto(111.64964674,415.18709591)(111.67464672,415.17209593)(111.70464966,415.15210297)
\curveto(111.81464658,415.14209596)(111.90464649,415.17709592)(111.97464966,415.25710297)
\curveto(112.04464635,415.34709575)(112.11964627,415.41709568)(112.19964966,415.46710297)
\curveto(112.46964592,415.66709543)(112.76964562,415.82709527)(113.09964966,415.94710297)
\curveto(113.1896452,415.97709512)(113.27964511,415.9970951)(113.36964966,416.00710297)
\curveto(113.46964492,416.01709508)(113.57464482,416.03209507)(113.68464966,416.05210297)
\curveto(113.71464468,416.06209504)(113.75964463,416.06209504)(113.81964966,416.05210297)
\curveto(113.87964451,416.05209505)(113.91964447,416.05709504)(113.93964966,416.06710297)
}
}
{
\newrgbcolor{curcolor}{0 0 0}
\pscustom[linestyle=none,fillstyle=solid,fillcolor=curcolor]
{
\newpath
\moveto(119.47089966,418.18210297)
\lineto(120.47589966,418.18210297)
\curveto(120.62589667,418.18209292)(120.75589654,418.17209293)(120.86589966,418.15210297)
\curveto(120.98589631,418.14209296)(121.07089623,418.08209302)(121.12089966,417.97210297)
\curveto(121.14089616,417.92209318)(121.15089615,417.86209324)(121.15089966,417.79210297)
\lineto(121.15089966,417.58210297)
\lineto(121.15089966,416.90710297)
\curveto(121.15089615,416.85709424)(121.14589615,416.7970943)(121.13589966,416.72710297)
\curveto(121.13589616,416.66709443)(121.14089616,416.61209449)(121.15089966,416.56210297)
\lineto(121.15089966,416.39710297)
\curveto(121.15089615,416.31709478)(121.15589614,416.24209486)(121.16589966,416.17210297)
\curveto(121.17589612,416.11209499)(121.2008961,416.05709504)(121.24089966,416.00710297)
\curveto(121.31089599,415.91709518)(121.43589586,415.86709523)(121.61589966,415.85710297)
\lineto(122.15589966,415.85710297)
\lineto(122.33589966,415.85710297)
\curveto(122.3958949,415.85709524)(122.45089485,415.84709525)(122.50089966,415.82710297)
\curveto(122.61089469,415.77709532)(122.67089463,415.68709541)(122.68089966,415.55710297)
\curveto(122.7008946,415.42709567)(122.71089459,415.28209582)(122.71089966,415.12210297)
\lineto(122.71089966,414.91210297)
\curveto(122.72089458,414.84209626)(122.71589458,414.78209632)(122.69589966,414.73210297)
\curveto(122.64589465,414.57209653)(122.54089476,414.48709661)(122.38089966,414.47710297)
\curveto(122.22089508,414.46709663)(122.04089526,414.46209664)(121.84089966,414.46210297)
\lineto(121.70589966,414.46210297)
\curveto(121.66589563,414.47209663)(121.63089567,414.47209663)(121.60089966,414.46210297)
\curveto(121.56089574,414.45209665)(121.52589577,414.44709665)(121.49589966,414.44710297)
\curveto(121.46589583,414.45709664)(121.43589586,414.45209665)(121.40589966,414.43210297)
\curveto(121.32589597,414.41209669)(121.26589603,414.36709673)(121.22589966,414.29710297)
\curveto(121.1958961,414.23709686)(121.17089613,414.16209694)(121.15089966,414.07210297)
\curveto(121.14089616,414.02209708)(121.14089616,413.96709713)(121.15089966,413.90710297)
\curveto(121.16089614,413.84709725)(121.16089614,413.79209731)(121.15089966,413.74210297)
\lineto(121.15089966,412.81210297)
\lineto(121.15089966,411.05710297)
\curveto(121.15089615,410.80710029)(121.15589614,410.58710051)(121.16589966,410.39710297)
\curveto(121.18589611,410.21710088)(121.25089605,410.05710104)(121.36089966,409.91710297)
\curveto(121.41089589,409.85710124)(121.47589582,409.81210129)(121.55589966,409.78210297)
\lineto(121.82589966,409.72210297)
\curveto(121.85589544,409.71210139)(121.88589541,409.70710139)(121.91589966,409.70710297)
\curveto(121.95589534,409.71710138)(121.98589531,409.71710138)(122.00589966,409.70710297)
\lineto(122.17089966,409.70710297)
\curveto(122.28089502,409.70710139)(122.37589492,409.7021014)(122.45589966,409.69210297)
\curveto(122.53589476,409.68210142)(122.6008947,409.64210146)(122.65089966,409.57210297)
\curveto(122.69089461,409.51210159)(122.71089459,409.43210167)(122.71089966,409.33210297)
\lineto(122.71089966,409.04710297)
\curveto(122.71089459,408.83710226)(122.70589459,408.64210246)(122.69589966,408.46210297)
\curveto(122.6958946,408.29210281)(122.61589468,408.17710292)(122.45589966,408.11710297)
\curveto(122.40589489,408.097103)(122.36089494,408.09210301)(122.32089966,408.10210297)
\curveto(122.28089502,408.102103)(122.23589506,408.09210301)(122.18589966,408.07210297)
\lineto(122.03589966,408.07210297)
\curveto(122.01589528,408.07210303)(121.98589531,408.07710302)(121.94589966,408.08710297)
\curveto(121.90589539,408.08710301)(121.87089543,408.08210302)(121.84089966,408.07210297)
\curveto(121.79089551,408.06210304)(121.73589556,408.06210304)(121.67589966,408.07210297)
\lineto(121.52589966,408.07210297)
\lineto(121.37589966,408.07210297)
\curveto(121.32589597,408.06210304)(121.28089602,408.06210304)(121.24089966,408.07210297)
\lineto(121.07589966,408.07210297)
\curveto(121.02589627,408.08210302)(120.97089633,408.08710301)(120.91089966,408.08710297)
\curveto(120.85089645,408.08710301)(120.7958965,408.09210301)(120.74589966,408.10210297)
\curveto(120.67589662,408.11210299)(120.61089669,408.12210298)(120.55089966,408.13210297)
\lineto(120.37089966,408.16210297)
\curveto(120.26089704,408.19210291)(120.15589714,408.22710287)(120.05589966,408.26710297)
\curveto(119.95589734,408.30710279)(119.86089744,408.35210275)(119.77089966,408.40210297)
\lineto(119.68089966,408.46210297)
\curveto(119.65089765,408.49210261)(119.61589768,408.52210258)(119.57589966,408.55210297)
\curveto(119.55589774,408.57210253)(119.53089777,408.59210251)(119.50089966,408.61210297)
\lineto(119.42589966,408.68710297)
\curveto(119.28589801,408.87710222)(119.18089812,409.08710201)(119.11089966,409.31710297)
\curveto(119.09089821,409.35710174)(119.08089822,409.39210171)(119.08089966,409.42210297)
\curveto(119.09089821,409.46210164)(119.09089821,409.50710159)(119.08089966,409.55710297)
\curveto(119.07089823,409.57710152)(119.06589823,409.6021015)(119.06589966,409.63210297)
\curveto(119.06589823,409.66210144)(119.06089824,409.68710141)(119.05089966,409.70710297)
\lineto(119.05089966,409.85710297)
\curveto(119.04089826,409.8971012)(119.03589826,409.94210116)(119.03589966,409.99210297)
\curveto(119.04589825,410.04210106)(119.05089825,410.09210101)(119.05089966,410.14210297)
\lineto(119.05089966,410.71210297)
\lineto(119.05089966,412.94710297)
\lineto(119.05089966,413.74210297)
\lineto(119.05089966,413.95210297)
\curveto(119.06089824,414.02209708)(119.05589824,414.08709701)(119.03589966,414.14710297)
\curveto(118.9958983,414.28709681)(118.92589837,414.37709672)(118.82589966,414.41710297)
\curveto(118.71589858,414.46709663)(118.57589872,414.48209662)(118.40589966,414.46210297)
\curveto(118.23589906,414.44209666)(118.09089921,414.45709664)(117.97089966,414.50710297)
\curveto(117.89089941,414.53709656)(117.84089946,414.58209652)(117.82089966,414.64210297)
\curveto(117.8008995,414.7020964)(117.78089952,414.77709632)(117.76089966,414.86710297)
\lineto(117.76089966,415.18210297)
\curveto(117.76089954,415.36209574)(117.77089953,415.50709559)(117.79089966,415.61710297)
\curveto(117.81089949,415.72709537)(117.8958994,415.8020953)(118.04589966,415.84210297)
\curveto(118.08589921,415.86209524)(118.12589917,415.86709523)(118.16589966,415.85710297)
\lineto(118.30089966,415.85710297)
\curveto(118.45089885,415.85709524)(118.59089871,415.86209524)(118.72089966,415.87210297)
\curveto(118.85089845,415.89209521)(118.94089836,415.95209515)(118.99089966,416.05210297)
\curveto(119.02089828,416.12209498)(119.03589826,416.2020949)(119.03589966,416.29210297)
\curveto(119.04589825,416.38209472)(119.05089825,416.47209463)(119.05089966,416.56210297)
\lineto(119.05089966,417.49210297)
\lineto(119.05089966,417.74710297)
\curveto(119.05089825,417.83709326)(119.06089824,417.91209319)(119.08089966,417.97210297)
\curveto(119.13089817,418.07209303)(119.20589809,418.13709296)(119.30589966,418.16710297)
\curveto(119.32589797,418.17709292)(119.35089795,418.17709292)(119.38089966,418.16710297)
\curveto(119.42089788,418.16709293)(119.45089785,418.17209293)(119.47089966,418.18210297)
}
}
{
\newrgbcolor{curcolor}{0 0 0}
\pscustom[linestyle=none,fillstyle=solid,fillcolor=curcolor]
{
\newpath
\moveto(125.79433716,418.72210297)
\curveto(125.86433421,418.64209246)(125.89933417,418.52209258)(125.89933716,418.36210297)
\lineto(125.89933716,417.89710297)
\lineto(125.89933716,417.49210297)
\curveto(125.89933417,417.35209375)(125.86433421,417.25709384)(125.79433716,417.20710297)
\curveto(125.73433434,417.15709394)(125.65433442,417.12709397)(125.55433716,417.11710297)
\curveto(125.46433461,417.10709399)(125.36433471,417.102094)(125.25433716,417.10210297)
\lineto(124.41433716,417.10210297)
\curveto(124.30433577,417.102094)(124.20433587,417.10709399)(124.11433716,417.11710297)
\curveto(124.03433604,417.12709397)(123.96433611,417.15709394)(123.90433716,417.20710297)
\curveto(123.86433621,417.23709386)(123.83433624,417.29209381)(123.81433716,417.37210297)
\curveto(123.80433627,417.46209364)(123.79433628,417.55709354)(123.78433716,417.65710297)
\lineto(123.78433716,417.98710297)
\curveto(123.79433628,418.097093)(123.79933627,418.19209291)(123.79933716,418.27210297)
\lineto(123.79933716,418.48210297)
\curveto(123.80933626,418.55209255)(123.82933624,418.61209249)(123.85933716,418.66210297)
\curveto(123.87933619,418.7020924)(123.90433617,418.73209237)(123.93433716,418.75210297)
\lineto(124.05433716,418.81210297)
\curveto(124.074336,418.81209229)(124.09933597,418.81209229)(124.12933716,418.81210297)
\curveto(124.15933591,418.82209228)(124.18433589,418.82709227)(124.20433716,418.82710297)
\lineto(125.29933716,418.82710297)
\curveto(125.39933467,418.82709227)(125.49433458,418.82209228)(125.58433716,418.81210297)
\curveto(125.6743344,418.8020923)(125.74433433,418.77209233)(125.79433716,418.72210297)
\moveto(125.89933716,408.95710297)
\curveto(125.89933417,408.75710234)(125.89433418,408.58710251)(125.88433716,408.44710297)
\curveto(125.8743342,408.30710279)(125.78433429,408.21210289)(125.61433716,408.16210297)
\curveto(125.55433452,408.14210296)(125.48933458,408.13210297)(125.41933716,408.13210297)
\curveto(125.34933472,408.14210296)(125.2743348,408.14710295)(125.19433716,408.14710297)
\lineto(124.35433716,408.14710297)
\curveto(124.26433581,408.14710295)(124.1743359,408.15210295)(124.08433716,408.16210297)
\curveto(124.00433607,408.17210293)(123.94433613,408.2021029)(123.90433716,408.25210297)
\curveto(123.84433623,408.32210278)(123.80933626,408.40710269)(123.79933716,408.50710297)
\lineto(123.79933716,408.85210297)
\lineto(123.79933716,415.18210297)
\lineto(123.79933716,415.48210297)
\curveto(123.79933627,415.58209552)(123.81933625,415.66209544)(123.85933716,415.72210297)
\curveto(123.91933615,415.79209531)(124.00433607,415.83709526)(124.11433716,415.85710297)
\curveto(124.13433594,415.86709523)(124.15933591,415.86709523)(124.18933716,415.85710297)
\curveto(124.22933584,415.85709524)(124.25933581,415.86209524)(124.27933716,415.87210297)
\lineto(125.02933716,415.87210297)
\lineto(125.22433716,415.87210297)
\curveto(125.30433477,415.88209522)(125.3693347,415.88209522)(125.41933716,415.87210297)
\lineto(125.53933716,415.87210297)
\curveto(125.59933447,415.85209525)(125.65433442,415.83709526)(125.70433716,415.82710297)
\curveto(125.75433432,415.81709528)(125.79433428,415.78709531)(125.82433716,415.73710297)
\curveto(125.86433421,415.68709541)(125.88433419,415.61709548)(125.88433716,415.52710297)
\curveto(125.89433418,415.43709566)(125.89933417,415.34209576)(125.89933716,415.24210297)
\lineto(125.89933716,408.95710297)
}
}
{
\newrgbcolor{curcolor}{0 0 0}
\pscustom[linestyle=none,fillstyle=solid,fillcolor=curcolor]
{
\newpath
\moveto(135.15152466,408.98710297)
\lineto(135.15152466,408.56710297)
\curveto(135.15151629,408.43710266)(135.12151632,408.33210277)(135.06152466,408.25210297)
\curveto(135.01151643,408.2021029)(134.94651649,408.16710293)(134.86652466,408.14710297)
\curveto(134.78651665,408.13710296)(134.69651674,408.13210297)(134.59652466,408.13210297)
\lineto(133.77152466,408.13210297)
\lineto(133.48652466,408.13210297)
\curveto(133.40651803,408.14210296)(133.3415181,408.16710293)(133.29152466,408.20710297)
\curveto(133.22151822,408.25710284)(133.18151826,408.32210278)(133.17152466,408.40210297)
\curveto(133.16151828,408.48210262)(133.1415183,408.56210254)(133.11152466,408.64210297)
\curveto(133.09151835,408.66210244)(133.07151837,408.67710242)(133.05152466,408.68710297)
\curveto(133.0415184,408.70710239)(133.02651841,408.72710237)(133.00652466,408.74710297)
\curveto(132.89651854,408.74710235)(132.81651862,408.72210238)(132.76652466,408.67210297)
\lineto(132.61652466,408.52210297)
\curveto(132.54651889,408.47210263)(132.48151896,408.42710267)(132.42152466,408.38710297)
\curveto(132.36151908,408.35710274)(132.29651914,408.31710278)(132.22652466,408.26710297)
\curveto(132.18651925,408.24710285)(132.1415193,408.22710287)(132.09152466,408.20710297)
\curveto(132.05151939,408.18710291)(132.00651943,408.16710293)(131.95652466,408.14710297)
\curveto(131.81651962,408.097103)(131.66651977,408.05210305)(131.50652466,408.01210297)
\curveto(131.45651998,407.99210311)(131.41152003,407.98210312)(131.37152466,407.98210297)
\curveto(131.33152011,407.98210312)(131.29152015,407.97710312)(131.25152466,407.96710297)
\lineto(131.11652466,407.96710297)
\curveto(131.08652035,407.95710314)(131.04652039,407.95210315)(130.99652466,407.95210297)
\lineto(130.86152466,407.95210297)
\curveto(130.80152064,407.93210317)(130.71152073,407.92710317)(130.59152466,407.93710297)
\curveto(130.47152097,407.93710316)(130.38652105,407.94710315)(130.33652466,407.96710297)
\curveto(130.26652117,407.98710311)(130.20152124,407.9971031)(130.14152466,407.99710297)
\curveto(130.09152135,407.98710311)(130.0365214,407.99210311)(129.97652466,408.01210297)
\lineto(129.61652466,408.13210297)
\curveto(129.50652193,408.16210294)(129.39652204,408.2021029)(129.28652466,408.25210297)
\curveto(128.9365225,408.4021027)(128.62152282,408.63210247)(128.34152466,408.94210297)
\curveto(128.07152337,409.26210184)(127.85652358,409.5971015)(127.69652466,409.94710297)
\curveto(127.64652379,410.05710104)(127.60652383,410.16210094)(127.57652466,410.26210297)
\curveto(127.54652389,410.37210073)(127.51152393,410.48210062)(127.47152466,410.59210297)
\curveto(127.46152398,410.63210047)(127.45652398,410.66710043)(127.45652466,410.69710297)
\curveto(127.45652398,410.73710036)(127.44652399,410.78210032)(127.42652466,410.83210297)
\curveto(127.40652403,410.91210019)(127.38652405,410.9971001)(127.36652466,411.08710297)
\curveto(127.35652408,411.18709991)(127.3415241,411.28709981)(127.32152466,411.38710297)
\curveto(127.31152413,411.41709968)(127.30652413,411.45209965)(127.30652466,411.49210297)
\curveto(127.31652412,411.53209957)(127.31652412,411.56709953)(127.30652466,411.59710297)
\lineto(127.30652466,411.73210297)
\curveto(127.30652413,411.78209932)(127.30152414,411.83209927)(127.29152466,411.88210297)
\curveto(127.28152416,411.93209917)(127.27652416,411.98709911)(127.27652466,412.04710297)
\curveto(127.27652416,412.11709898)(127.28152416,412.17209893)(127.29152466,412.21210297)
\curveto(127.30152414,412.26209884)(127.30652413,412.30709879)(127.30652466,412.34710297)
\lineto(127.30652466,412.49710297)
\curveto(127.31652412,412.54709855)(127.31652412,412.59209851)(127.30652466,412.63210297)
\curveto(127.30652413,412.68209842)(127.31652412,412.73209837)(127.33652466,412.78210297)
\curveto(127.35652408,412.89209821)(127.37152407,412.9970981)(127.38152466,413.09710297)
\curveto(127.40152404,413.1970979)(127.42652401,413.2970978)(127.45652466,413.39710297)
\curveto(127.49652394,413.51709758)(127.53152391,413.63209747)(127.56152466,413.74210297)
\curveto(127.59152385,413.85209725)(127.63152381,413.96209714)(127.68152466,414.07210297)
\curveto(127.82152362,414.37209673)(127.99652344,414.65709644)(128.20652466,414.92710297)
\curveto(128.22652321,414.95709614)(128.25152319,414.98209612)(128.28152466,415.00210297)
\curveto(128.32152312,415.03209607)(128.35152309,415.06209604)(128.37152466,415.09210297)
\curveto(128.41152303,415.14209596)(128.45152299,415.18709591)(128.49152466,415.22710297)
\curveto(128.53152291,415.26709583)(128.57652286,415.30709579)(128.62652466,415.34710297)
\curveto(128.66652277,415.36709573)(128.70152274,415.39209571)(128.73152466,415.42210297)
\curveto(128.76152268,415.46209564)(128.79652264,415.49209561)(128.83652466,415.51210297)
\curveto(129.08652235,415.68209542)(129.37652206,415.82209528)(129.70652466,415.93210297)
\curveto(129.77652166,415.95209515)(129.84652159,415.96709513)(129.91652466,415.97710297)
\curveto(129.99652144,415.98709511)(130.07652136,416.0020951)(130.15652466,416.02210297)
\curveto(130.22652121,416.04209506)(130.31652112,416.05209505)(130.42652466,416.05210297)
\curveto(130.5365209,416.06209504)(130.64652079,416.06709503)(130.75652466,416.06710297)
\curveto(130.86652057,416.06709503)(130.97152047,416.06209504)(131.07152466,416.05210297)
\curveto(131.18152026,416.04209506)(131.27152017,416.02709507)(131.34152466,416.00710297)
\curveto(131.49151995,415.95709514)(131.6365198,415.91209519)(131.77652466,415.87210297)
\curveto(131.91651952,415.83209527)(132.04651939,415.77709532)(132.16652466,415.70710297)
\curveto(132.2365192,415.65709544)(132.30151914,415.60709549)(132.36152466,415.55710297)
\curveto(132.42151902,415.51709558)(132.48651895,415.47209563)(132.55652466,415.42210297)
\curveto(132.59651884,415.39209571)(132.65151879,415.35209575)(132.72152466,415.30210297)
\curveto(132.80151864,415.25209585)(132.87651856,415.25209585)(132.94652466,415.30210297)
\curveto(132.98651845,415.32209578)(133.00651843,415.35709574)(133.00652466,415.40710297)
\curveto(133.00651843,415.45709564)(133.01651842,415.50709559)(133.03652466,415.55710297)
\lineto(133.03652466,415.70710297)
\curveto(133.04651839,415.73709536)(133.05151839,415.77209533)(133.05152466,415.81210297)
\lineto(133.05152466,415.93210297)
\lineto(133.05152466,417.97210297)
\curveto(133.05151839,418.08209302)(133.04651839,418.2020929)(133.03652466,418.33210297)
\curveto(133.0365184,418.47209263)(133.06151838,418.57709252)(133.11152466,418.64710297)
\curveto(133.15151829,418.72709237)(133.22651821,418.77709232)(133.33652466,418.79710297)
\curveto(133.35651808,418.80709229)(133.37651806,418.80709229)(133.39652466,418.79710297)
\curveto(133.41651802,418.7970923)(133.436518,418.8020923)(133.45652466,418.81210297)
\lineto(134.52152466,418.81210297)
\curveto(134.6415168,418.81209229)(134.75151669,418.80709229)(134.85152466,418.79710297)
\curveto(134.95151649,418.78709231)(135.02651641,418.74709235)(135.07652466,418.67710297)
\curveto(135.12651631,418.5970925)(135.15151629,418.49209261)(135.15152466,418.36210297)
\lineto(135.15152466,418.00210297)
\lineto(135.15152466,408.98710297)
\moveto(133.11152466,411.92710297)
\curveto(133.12151832,411.96709913)(133.12151832,412.00709909)(133.11152466,412.04710297)
\lineto(133.11152466,412.18210297)
\curveto(133.11151833,412.28209882)(133.10651833,412.38209872)(133.09652466,412.48210297)
\curveto(133.08651835,412.58209852)(133.07151837,412.67209843)(133.05152466,412.75210297)
\curveto(133.03151841,412.86209824)(133.01151843,412.96209814)(132.99152466,413.05210297)
\curveto(132.98151846,413.14209796)(132.95651848,413.22709787)(132.91652466,413.30710297)
\curveto(132.77651866,413.66709743)(132.57151887,413.95209715)(132.30152466,414.16210297)
\curveto(132.0415194,414.37209673)(131.66151978,414.47709662)(131.16152466,414.47710297)
\curveto(131.10152034,414.47709662)(131.02152042,414.46709663)(130.92152466,414.44710297)
\curveto(130.8415206,414.42709667)(130.76652067,414.40709669)(130.69652466,414.38710297)
\curveto(130.6365208,414.37709672)(130.57652086,414.35709674)(130.51652466,414.32710297)
\curveto(130.24652119,414.21709688)(130.0365214,414.04709705)(129.88652466,413.81710297)
\curveto(129.7365217,413.58709751)(129.61652182,413.32709777)(129.52652466,413.03710297)
\curveto(129.49652194,412.93709816)(129.47652196,412.83709826)(129.46652466,412.73710297)
\curveto(129.45652198,412.63709846)(129.436522,412.53209857)(129.40652466,412.42210297)
\lineto(129.40652466,412.21210297)
\curveto(129.38652205,412.12209898)(129.38152206,411.9970991)(129.39152466,411.83710297)
\curveto(129.40152204,411.68709941)(129.41652202,411.57709952)(129.43652466,411.50710297)
\lineto(129.43652466,411.41710297)
\curveto(129.44652199,411.3970997)(129.45152199,411.37709972)(129.45152466,411.35710297)
\curveto(129.47152197,411.27709982)(129.48652195,411.2020999)(129.49652466,411.13210297)
\curveto(129.51652192,411.06210004)(129.5365219,410.98710011)(129.55652466,410.90710297)
\curveto(129.72652171,410.38710071)(130.01652142,410.0021011)(130.42652466,409.75210297)
\curveto(130.55652088,409.66210144)(130.7365207,409.59210151)(130.96652466,409.54210297)
\curveto(131.00652043,409.53210157)(131.06652037,409.52710157)(131.14652466,409.52710297)
\curveto(131.17652026,409.51710158)(131.22152022,409.50710159)(131.28152466,409.49710297)
\curveto(131.35152009,409.4971016)(131.40652003,409.5021016)(131.44652466,409.51210297)
\curveto(131.52651991,409.53210157)(131.60651983,409.54710155)(131.68652466,409.55710297)
\curveto(131.76651967,409.56710153)(131.84651959,409.58710151)(131.92652466,409.61710297)
\curveto(132.17651926,409.72710137)(132.37651906,409.86710123)(132.52652466,410.03710297)
\curveto(132.67651876,410.20710089)(132.80651863,410.42210068)(132.91652466,410.68210297)
\curveto(132.95651848,410.77210033)(132.98651845,410.86210024)(133.00652466,410.95210297)
\curveto(133.02651841,411.05210005)(133.04651839,411.15709994)(133.06652466,411.26710297)
\curveto(133.07651836,411.31709978)(133.07651836,411.36209974)(133.06652466,411.40210297)
\curveto(133.06651837,411.45209965)(133.07651836,411.5020996)(133.09652466,411.55210297)
\curveto(133.10651833,411.58209952)(133.11151833,411.61709948)(133.11152466,411.65710297)
\lineto(133.11152466,411.79210297)
\lineto(133.11152466,411.92710297)
}
}
{
\newrgbcolor{curcolor}{0 0 0}
\pscustom[linestyle=none,fillstyle=solid,fillcolor=curcolor]
{
\newpath
\moveto(143.78144653,408.73210297)
\curveto(143.80143868,408.62210248)(143.81143867,408.51210259)(143.81144653,408.40210297)
\curveto(143.82143866,408.29210281)(143.77143871,408.21710288)(143.66144653,408.17710297)
\curveto(143.60143888,408.14710295)(143.53143895,408.13210297)(143.45144653,408.13210297)
\lineto(143.21144653,408.13210297)
\lineto(142.40144653,408.13210297)
\lineto(142.13144653,408.13210297)
\curveto(142.05144043,408.14210296)(141.9864405,408.16710293)(141.93644653,408.20710297)
\curveto(141.86644062,408.24710285)(141.81144067,408.3021028)(141.77144653,408.37210297)
\curveto(141.74144074,408.45210265)(141.69644079,408.51710258)(141.63644653,408.56710297)
\curveto(141.61644087,408.58710251)(141.59144089,408.6021025)(141.56144653,408.61210297)
\curveto(141.53144095,408.63210247)(141.49144099,408.63710246)(141.44144653,408.62710297)
\curveto(141.39144109,408.60710249)(141.34144114,408.58210252)(141.29144653,408.55210297)
\curveto(141.25144123,408.52210258)(141.20644128,408.4971026)(141.15644653,408.47710297)
\curveto(141.10644138,408.43710266)(141.05144143,408.4021027)(140.99144653,408.37210297)
\lineto(140.81144653,408.28210297)
\curveto(140.6814418,408.22210288)(140.54644194,408.17210293)(140.40644653,408.13210297)
\curveto(140.26644222,408.102103)(140.12144236,408.06710303)(139.97144653,408.02710297)
\curveto(139.90144258,408.00710309)(139.83144265,407.9971031)(139.76144653,407.99710297)
\curveto(139.70144278,407.98710311)(139.63644285,407.97710312)(139.56644653,407.96710297)
\lineto(139.47644653,407.96710297)
\curveto(139.44644304,407.95710314)(139.41644307,407.95210315)(139.38644653,407.95210297)
\lineto(139.22144653,407.95210297)
\curveto(139.12144336,407.93210317)(139.02144346,407.93210317)(138.92144653,407.95210297)
\lineto(138.78644653,407.95210297)
\curveto(138.71644377,407.97210313)(138.64644384,407.98210312)(138.57644653,407.98210297)
\curveto(138.51644397,407.97210313)(138.45644403,407.97710312)(138.39644653,407.99710297)
\curveto(138.29644419,408.01710308)(138.20144428,408.03710306)(138.11144653,408.05710297)
\curveto(138.02144446,408.06710303)(137.93644455,408.09210301)(137.85644653,408.13210297)
\curveto(137.56644492,408.24210286)(137.31644517,408.38210272)(137.10644653,408.55210297)
\curveto(136.90644558,408.73210237)(136.74644574,408.96710213)(136.62644653,409.25710297)
\curveto(136.59644589,409.32710177)(136.56644592,409.4021017)(136.53644653,409.48210297)
\curveto(136.51644597,409.56210154)(136.49644599,409.64710145)(136.47644653,409.73710297)
\curveto(136.45644603,409.78710131)(136.44644604,409.83710126)(136.44644653,409.88710297)
\curveto(136.45644603,409.93710116)(136.45644603,409.98710111)(136.44644653,410.03710297)
\curveto(136.43644605,410.06710103)(136.42644606,410.12710097)(136.41644653,410.21710297)
\curveto(136.41644607,410.31710078)(136.42144606,410.38710071)(136.43144653,410.42710297)
\curveto(136.45144603,410.52710057)(136.46144602,410.61210049)(136.46144653,410.68210297)
\lineto(136.55144653,411.01210297)
\curveto(136.5814459,411.13209997)(136.62144586,411.23709986)(136.67144653,411.32710297)
\curveto(136.84144564,411.61709948)(137.03644545,411.83709926)(137.25644653,411.98710297)
\curveto(137.47644501,412.13709896)(137.75644473,412.26709883)(138.09644653,412.37710297)
\curveto(138.22644426,412.42709867)(138.36144412,412.46209864)(138.50144653,412.48210297)
\curveto(138.64144384,412.5020986)(138.7814437,412.52709857)(138.92144653,412.55710297)
\curveto(139.00144348,412.57709852)(139.0864434,412.58709851)(139.17644653,412.58710297)
\curveto(139.26644322,412.5970985)(139.35644313,412.61209849)(139.44644653,412.63210297)
\curveto(139.51644297,412.65209845)(139.5864429,412.65709844)(139.65644653,412.64710297)
\curveto(139.72644276,412.64709845)(139.80144268,412.65709844)(139.88144653,412.67710297)
\curveto(139.95144253,412.6970984)(140.02144246,412.70709839)(140.09144653,412.70710297)
\curveto(140.16144232,412.70709839)(140.23644225,412.71709838)(140.31644653,412.73710297)
\curveto(140.52644196,412.78709831)(140.71644177,412.82709827)(140.88644653,412.85710297)
\curveto(141.06644142,412.8970982)(141.22644126,412.98709811)(141.36644653,413.12710297)
\curveto(141.45644103,413.21709788)(141.51644097,413.31709778)(141.54644653,413.42710297)
\curveto(141.55644093,413.45709764)(141.55644093,413.48209762)(141.54644653,413.50210297)
\curveto(141.54644094,413.52209758)(141.55144093,413.54209756)(141.56144653,413.56210297)
\curveto(141.57144091,413.58209752)(141.57644091,413.61209749)(141.57644653,413.65210297)
\lineto(141.57644653,413.74210297)
\lineto(141.54644653,413.86210297)
\curveto(141.54644094,413.9020972)(141.54144094,413.93709716)(141.53144653,413.96710297)
\curveto(141.43144105,414.26709683)(141.22144126,414.47209663)(140.90144653,414.58210297)
\curveto(140.81144167,414.61209649)(140.70144178,414.63209647)(140.57144653,414.64210297)
\curveto(140.45144203,414.66209644)(140.32644216,414.66709643)(140.19644653,414.65710297)
\curveto(140.06644242,414.65709644)(139.94144254,414.64709645)(139.82144653,414.62710297)
\curveto(139.70144278,414.60709649)(139.59644289,414.58209652)(139.50644653,414.55210297)
\curveto(139.44644304,414.53209657)(139.3864431,414.5020966)(139.32644653,414.46210297)
\curveto(139.27644321,414.43209667)(139.22644326,414.3970967)(139.17644653,414.35710297)
\curveto(139.12644336,414.31709678)(139.07144341,414.26209684)(139.01144653,414.19210297)
\curveto(138.96144352,414.12209698)(138.92644356,414.05709704)(138.90644653,413.99710297)
\curveto(138.85644363,413.8970972)(138.81144367,413.8020973)(138.77144653,413.71210297)
\curveto(138.74144374,413.62209748)(138.67144381,413.56209754)(138.56144653,413.53210297)
\curveto(138.481444,413.51209759)(138.39644409,413.5020976)(138.30644653,413.50210297)
\lineto(138.03644653,413.50210297)
\lineto(137.46644653,413.50210297)
\curveto(137.41644507,413.5020976)(137.36644512,413.4970976)(137.31644653,413.48710297)
\curveto(137.26644522,413.48709761)(137.22144526,413.49209761)(137.18144653,413.50210297)
\lineto(137.04644653,413.50210297)
\curveto(137.02644546,413.51209759)(137.00144548,413.51709758)(136.97144653,413.51710297)
\curveto(136.94144554,413.51709758)(136.91644557,413.52709757)(136.89644653,413.54710297)
\curveto(136.81644567,413.56709753)(136.76144572,413.63209747)(136.73144653,413.74210297)
\curveto(136.72144576,413.79209731)(136.72144576,413.84209726)(136.73144653,413.89210297)
\curveto(136.74144574,413.94209716)(136.75144573,413.98709711)(136.76144653,414.02710297)
\curveto(136.79144569,414.13709696)(136.82144566,414.23709686)(136.85144653,414.32710297)
\curveto(136.89144559,414.42709667)(136.93644555,414.51709658)(136.98644653,414.59710297)
\lineto(137.07644653,414.74710297)
\lineto(137.16644653,414.89710297)
\curveto(137.24644524,415.00709609)(137.34644514,415.11209599)(137.46644653,415.21210297)
\curveto(137.486445,415.22209588)(137.51644497,415.24709585)(137.55644653,415.28710297)
\curveto(137.60644488,415.32709577)(137.65144483,415.36209574)(137.69144653,415.39210297)
\curveto(137.73144475,415.42209568)(137.77644471,415.45209565)(137.82644653,415.48210297)
\curveto(137.99644449,415.59209551)(138.17644431,415.67709542)(138.36644653,415.73710297)
\curveto(138.55644393,415.80709529)(138.75144373,415.87209523)(138.95144653,415.93210297)
\curveto(139.07144341,415.96209514)(139.19644329,415.98209512)(139.32644653,415.99210297)
\curveto(139.45644303,416.0020951)(139.5864429,416.02209508)(139.71644653,416.05210297)
\curveto(139.75644273,416.06209504)(139.81644267,416.06209504)(139.89644653,416.05210297)
\curveto(139.9864425,416.04209506)(140.04144244,416.04709505)(140.06144653,416.06710297)
\curveto(140.47144201,416.07709502)(140.86144162,416.06209504)(141.23144653,416.02210297)
\curveto(141.61144087,415.98209512)(141.95144053,415.90709519)(142.25144653,415.79710297)
\curveto(142.56143992,415.68709541)(142.82643966,415.53709556)(143.04644653,415.34710297)
\curveto(143.26643922,415.16709593)(143.43643905,414.93209617)(143.55644653,414.64210297)
\curveto(143.62643886,414.47209663)(143.66643882,414.27709682)(143.67644653,414.05710297)
\curveto(143.6864388,413.83709726)(143.69143879,413.61209749)(143.69144653,413.38210297)
\lineto(143.69144653,410.03710297)
\lineto(143.69144653,409.45210297)
\curveto(143.69143879,409.26210184)(143.71143877,409.08710201)(143.75144653,408.92710297)
\curveto(143.76143872,408.8971022)(143.76643872,408.86210224)(143.76644653,408.82210297)
\curveto(143.76643872,408.79210231)(143.77143871,408.76210234)(143.78144653,408.73210297)
\moveto(141.57644653,411.04210297)
\curveto(141.5864409,411.09210001)(141.59144089,411.14709995)(141.59144653,411.20710297)
\curveto(141.59144089,411.27709982)(141.5864409,411.33709976)(141.57644653,411.38710297)
\curveto(141.55644093,411.44709965)(141.54644094,411.5020996)(141.54644653,411.55210297)
\curveto(141.54644094,411.6020995)(141.52644096,411.64209946)(141.48644653,411.67210297)
\curveto(141.43644105,411.71209939)(141.36144112,411.73209937)(141.26144653,411.73210297)
\curveto(141.22144126,411.72209938)(141.1864413,411.71209939)(141.15644653,411.70210297)
\curveto(141.12644136,411.7020994)(141.09144139,411.6970994)(141.05144653,411.68710297)
\curveto(140.9814415,411.66709943)(140.90644158,411.65209945)(140.82644653,411.64210297)
\curveto(140.74644174,411.63209947)(140.66644182,411.61709948)(140.58644653,411.59710297)
\curveto(140.55644193,411.58709951)(140.51144197,411.58209952)(140.45144653,411.58210297)
\curveto(140.32144216,411.55209955)(140.19144229,411.53209957)(140.06144653,411.52210297)
\curveto(139.93144255,411.51209959)(139.80644268,411.48709961)(139.68644653,411.44710297)
\curveto(139.60644288,411.42709967)(139.53144295,411.40709969)(139.46144653,411.38710297)
\curveto(139.39144309,411.37709972)(139.32144316,411.35709974)(139.25144653,411.32710297)
\curveto(139.04144344,411.23709986)(138.86144362,411.1021)(138.71144653,410.92210297)
\curveto(138.57144391,410.74210036)(138.52144396,410.49210061)(138.56144653,410.17210297)
\curveto(138.5814439,410.0021011)(138.63644385,409.86210124)(138.72644653,409.75210297)
\curveto(138.79644369,409.64210146)(138.90144358,409.55210155)(139.04144653,409.48210297)
\curveto(139.1814433,409.42210168)(139.33144315,409.37710172)(139.49144653,409.34710297)
\curveto(139.66144282,409.31710178)(139.83644265,409.30710179)(140.01644653,409.31710297)
\curveto(140.20644228,409.33710176)(140.3814421,409.37210173)(140.54144653,409.42210297)
\curveto(140.80144168,409.5021016)(141.00644148,409.62710147)(141.15644653,409.79710297)
\curveto(141.30644118,409.97710112)(141.42144106,410.1971009)(141.50144653,410.45710297)
\curveto(141.52144096,410.52710057)(141.53144095,410.5971005)(141.53144653,410.66710297)
\curveto(141.54144094,410.74710035)(141.55644093,410.82710027)(141.57644653,410.90710297)
\lineto(141.57644653,411.04210297)
}
}
{
\newrgbcolor{curcolor}{0 0 0}
\pscustom[linestyle=none,fillstyle=solid,fillcolor=curcolor]
{
\newpath
\moveto(152.93472778,408.98710297)
\lineto(152.93472778,408.56710297)
\curveto(152.93471941,408.43710266)(152.90471944,408.33210277)(152.84472778,408.25210297)
\curveto(152.79471955,408.2021029)(152.72971962,408.16710293)(152.64972778,408.14710297)
\curveto(152.56971978,408.13710296)(152.47971987,408.13210297)(152.37972778,408.13210297)
\lineto(151.55472778,408.13210297)
\lineto(151.26972778,408.13210297)
\curveto(151.18972116,408.14210296)(151.12472122,408.16710293)(151.07472778,408.20710297)
\curveto(151.00472134,408.25710284)(150.96472138,408.32210278)(150.95472778,408.40210297)
\curveto(150.9447214,408.48210262)(150.92472142,408.56210254)(150.89472778,408.64210297)
\curveto(150.87472147,408.66210244)(150.85472149,408.67710242)(150.83472778,408.68710297)
\curveto(150.82472152,408.70710239)(150.80972154,408.72710237)(150.78972778,408.74710297)
\curveto(150.67972167,408.74710235)(150.59972175,408.72210238)(150.54972778,408.67210297)
\lineto(150.39972778,408.52210297)
\curveto(150.32972202,408.47210263)(150.26472208,408.42710267)(150.20472778,408.38710297)
\curveto(150.1447222,408.35710274)(150.07972227,408.31710278)(150.00972778,408.26710297)
\curveto(149.96972238,408.24710285)(149.92472242,408.22710287)(149.87472778,408.20710297)
\curveto(149.83472251,408.18710291)(149.78972256,408.16710293)(149.73972778,408.14710297)
\curveto(149.59972275,408.097103)(149.4497229,408.05210305)(149.28972778,408.01210297)
\curveto(149.23972311,407.99210311)(149.19472315,407.98210312)(149.15472778,407.98210297)
\curveto(149.11472323,407.98210312)(149.07472327,407.97710312)(149.03472778,407.96710297)
\lineto(148.89972778,407.96710297)
\curveto(148.86972348,407.95710314)(148.82972352,407.95210315)(148.77972778,407.95210297)
\lineto(148.64472778,407.95210297)
\curveto(148.58472376,407.93210317)(148.49472385,407.92710317)(148.37472778,407.93710297)
\curveto(148.25472409,407.93710316)(148.16972418,407.94710315)(148.11972778,407.96710297)
\curveto(148.0497243,407.98710311)(147.98472436,407.9971031)(147.92472778,407.99710297)
\curveto(147.87472447,407.98710311)(147.81972453,407.99210311)(147.75972778,408.01210297)
\lineto(147.39972778,408.13210297)
\curveto(147.28972506,408.16210294)(147.17972517,408.2021029)(147.06972778,408.25210297)
\curveto(146.71972563,408.4021027)(146.40472594,408.63210247)(146.12472778,408.94210297)
\curveto(145.85472649,409.26210184)(145.63972671,409.5971015)(145.47972778,409.94710297)
\curveto(145.42972692,410.05710104)(145.38972696,410.16210094)(145.35972778,410.26210297)
\curveto(145.32972702,410.37210073)(145.29472705,410.48210062)(145.25472778,410.59210297)
\curveto(145.2447271,410.63210047)(145.23972711,410.66710043)(145.23972778,410.69710297)
\curveto(145.23972711,410.73710036)(145.22972712,410.78210032)(145.20972778,410.83210297)
\curveto(145.18972716,410.91210019)(145.16972718,410.9971001)(145.14972778,411.08710297)
\curveto(145.13972721,411.18709991)(145.12472722,411.28709981)(145.10472778,411.38710297)
\curveto(145.09472725,411.41709968)(145.08972726,411.45209965)(145.08972778,411.49210297)
\curveto(145.09972725,411.53209957)(145.09972725,411.56709953)(145.08972778,411.59710297)
\lineto(145.08972778,411.73210297)
\curveto(145.08972726,411.78209932)(145.08472726,411.83209927)(145.07472778,411.88210297)
\curveto(145.06472728,411.93209917)(145.05972729,411.98709911)(145.05972778,412.04710297)
\curveto(145.05972729,412.11709898)(145.06472728,412.17209893)(145.07472778,412.21210297)
\curveto(145.08472726,412.26209884)(145.08972726,412.30709879)(145.08972778,412.34710297)
\lineto(145.08972778,412.49710297)
\curveto(145.09972725,412.54709855)(145.09972725,412.59209851)(145.08972778,412.63210297)
\curveto(145.08972726,412.68209842)(145.09972725,412.73209837)(145.11972778,412.78210297)
\curveto(145.13972721,412.89209821)(145.15472719,412.9970981)(145.16472778,413.09710297)
\curveto(145.18472716,413.1970979)(145.20972714,413.2970978)(145.23972778,413.39710297)
\curveto(145.27972707,413.51709758)(145.31472703,413.63209747)(145.34472778,413.74210297)
\curveto(145.37472697,413.85209725)(145.41472693,413.96209714)(145.46472778,414.07210297)
\curveto(145.60472674,414.37209673)(145.77972657,414.65709644)(145.98972778,414.92710297)
\curveto(146.00972634,414.95709614)(146.03472631,414.98209612)(146.06472778,415.00210297)
\curveto(146.10472624,415.03209607)(146.13472621,415.06209604)(146.15472778,415.09210297)
\curveto(146.19472615,415.14209596)(146.23472611,415.18709591)(146.27472778,415.22710297)
\curveto(146.31472603,415.26709583)(146.35972599,415.30709579)(146.40972778,415.34710297)
\curveto(146.4497259,415.36709573)(146.48472586,415.39209571)(146.51472778,415.42210297)
\curveto(146.5447258,415.46209564)(146.57972577,415.49209561)(146.61972778,415.51210297)
\curveto(146.86972548,415.68209542)(147.15972519,415.82209528)(147.48972778,415.93210297)
\curveto(147.55972479,415.95209515)(147.62972472,415.96709513)(147.69972778,415.97710297)
\curveto(147.77972457,415.98709511)(147.85972449,416.0020951)(147.93972778,416.02210297)
\curveto(148.00972434,416.04209506)(148.09972425,416.05209505)(148.20972778,416.05210297)
\curveto(148.31972403,416.06209504)(148.42972392,416.06709503)(148.53972778,416.06710297)
\curveto(148.6497237,416.06709503)(148.75472359,416.06209504)(148.85472778,416.05210297)
\curveto(148.96472338,416.04209506)(149.05472329,416.02709507)(149.12472778,416.00710297)
\curveto(149.27472307,415.95709514)(149.41972293,415.91209519)(149.55972778,415.87210297)
\curveto(149.69972265,415.83209527)(149.82972252,415.77709532)(149.94972778,415.70710297)
\curveto(150.01972233,415.65709544)(150.08472226,415.60709549)(150.14472778,415.55710297)
\curveto(150.20472214,415.51709558)(150.26972208,415.47209563)(150.33972778,415.42210297)
\curveto(150.37972197,415.39209571)(150.43472191,415.35209575)(150.50472778,415.30210297)
\curveto(150.58472176,415.25209585)(150.65972169,415.25209585)(150.72972778,415.30210297)
\curveto(150.76972158,415.32209578)(150.78972156,415.35709574)(150.78972778,415.40710297)
\curveto(150.78972156,415.45709564)(150.79972155,415.50709559)(150.81972778,415.55710297)
\lineto(150.81972778,415.70710297)
\curveto(150.82972152,415.73709536)(150.83472151,415.77209533)(150.83472778,415.81210297)
\lineto(150.83472778,415.93210297)
\lineto(150.83472778,417.97210297)
\curveto(150.83472151,418.08209302)(150.82972152,418.2020929)(150.81972778,418.33210297)
\curveto(150.81972153,418.47209263)(150.8447215,418.57709252)(150.89472778,418.64710297)
\curveto(150.93472141,418.72709237)(151.00972134,418.77709232)(151.11972778,418.79710297)
\curveto(151.13972121,418.80709229)(151.15972119,418.80709229)(151.17972778,418.79710297)
\curveto(151.19972115,418.7970923)(151.21972113,418.8020923)(151.23972778,418.81210297)
\lineto(152.30472778,418.81210297)
\curveto(152.42471992,418.81209229)(152.53471981,418.80709229)(152.63472778,418.79710297)
\curveto(152.73471961,418.78709231)(152.80971954,418.74709235)(152.85972778,418.67710297)
\curveto(152.90971944,418.5970925)(152.93471941,418.49209261)(152.93472778,418.36210297)
\lineto(152.93472778,418.00210297)
\lineto(152.93472778,408.98710297)
\moveto(150.89472778,411.92710297)
\curveto(150.90472144,411.96709913)(150.90472144,412.00709909)(150.89472778,412.04710297)
\lineto(150.89472778,412.18210297)
\curveto(150.89472145,412.28209882)(150.88972146,412.38209872)(150.87972778,412.48210297)
\curveto(150.86972148,412.58209852)(150.85472149,412.67209843)(150.83472778,412.75210297)
\curveto(150.81472153,412.86209824)(150.79472155,412.96209814)(150.77472778,413.05210297)
\curveto(150.76472158,413.14209796)(150.73972161,413.22709787)(150.69972778,413.30710297)
\curveto(150.55972179,413.66709743)(150.35472199,413.95209715)(150.08472778,414.16210297)
\curveto(149.82472252,414.37209673)(149.4447229,414.47709662)(148.94472778,414.47710297)
\curveto(148.88472346,414.47709662)(148.80472354,414.46709663)(148.70472778,414.44710297)
\curveto(148.62472372,414.42709667)(148.5497238,414.40709669)(148.47972778,414.38710297)
\curveto(148.41972393,414.37709672)(148.35972399,414.35709674)(148.29972778,414.32710297)
\curveto(148.02972432,414.21709688)(147.81972453,414.04709705)(147.66972778,413.81710297)
\curveto(147.51972483,413.58709751)(147.39972495,413.32709777)(147.30972778,413.03710297)
\curveto(147.27972507,412.93709816)(147.25972509,412.83709826)(147.24972778,412.73710297)
\curveto(147.23972511,412.63709846)(147.21972513,412.53209857)(147.18972778,412.42210297)
\lineto(147.18972778,412.21210297)
\curveto(147.16972518,412.12209898)(147.16472518,411.9970991)(147.17472778,411.83710297)
\curveto(147.18472516,411.68709941)(147.19972515,411.57709952)(147.21972778,411.50710297)
\lineto(147.21972778,411.41710297)
\curveto(147.22972512,411.3970997)(147.23472511,411.37709972)(147.23472778,411.35710297)
\curveto(147.25472509,411.27709982)(147.26972508,411.2020999)(147.27972778,411.13210297)
\curveto(147.29972505,411.06210004)(147.31972503,410.98710011)(147.33972778,410.90710297)
\curveto(147.50972484,410.38710071)(147.79972455,410.0021011)(148.20972778,409.75210297)
\curveto(148.33972401,409.66210144)(148.51972383,409.59210151)(148.74972778,409.54210297)
\curveto(148.78972356,409.53210157)(148.8497235,409.52710157)(148.92972778,409.52710297)
\curveto(148.95972339,409.51710158)(149.00472334,409.50710159)(149.06472778,409.49710297)
\curveto(149.13472321,409.4971016)(149.18972316,409.5021016)(149.22972778,409.51210297)
\curveto(149.30972304,409.53210157)(149.38972296,409.54710155)(149.46972778,409.55710297)
\curveto(149.5497228,409.56710153)(149.62972272,409.58710151)(149.70972778,409.61710297)
\curveto(149.95972239,409.72710137)(150.15972219,409.86710123)(150.30972778,410.03710297)
\curveto(150.45972189,410.20710089)(150.58972176,410.42210068)(150.69972778,410.68210297)
\curveto(150.73972161,410.77210033)(150.76972158,410.86210024)(150.78972778,410.95210297)
\curveto(150.80972154,411.05210005)(150.82972152,411.15709994)(150.84972778,411.26710297)
\curveto(150.85972149,411.31709978)(150.85972149,411.36209974)(150.84972778,411.40210297)
\curveto(150.8497215,411.45209965)(150.85972149,411.5020996)(150.87972778,411.55210297)
\curveto(150.88972146,411.58209952)(150.89472145,411.61709948)(150.89472778,411.65710297)
\lineto(150.89472778,411.79210297)
\lineto(150.89472778,411.92710297)
}
}
{
\newrgbcolor{curcolor}{0 0 0}
\pscustom[linestyle=none,fillstyle=solid,fillcolor=curcolor]
{
}
}
{
\newrgbcolor{curcolor}{0 0 0}
\pscustom[linestyle=none,fillstyle=solid,fillcolor=curcolor]
{
\newpath
\moveto(166.26480591,408.98710297)
\lineto(166.26480591,408.56710297)
\curveto(166.26479754,408.43710266)(166.23479757,408.33210277)(166.17480591,408.25210297)
\curveto(166.12479768,408.2021029)(166.05979774,408.16710293)(165.97980591,408.14710297)
\curveto(165.8997979,408.13710296)(165.80979799,408.13210297)(165.70980591,408.13210297)
\lineto(164.88480591,408.13210297)
\lineto(164.59980591,408.13210297)
\curveto(164.51979928,408.14210296)(164.45479935,408.16710293)(164.40480591,408.20710297)
\curveto(164.33479947,408.25710284)(164.29479951,408.32210278)(164.28480591,408.40210297)
\curveto(164.27479953,408.48210262)(164.25479955,408.56210254)(164.22480591,408.64210297)
\curveto(164.2047996,408.66210244)(164.18479962,408.67710242)(164.16480591,408.68710297)
\curveto(164.15479965,408.70710239)(164.13979966,408.72710237)(164.11980591,408.74710297)
\curveto(164.00979979,408.74710235)(163.92979987,408.72210238)(163.87980591,408.67210297)
\lineto(163.72980591,408.52210297)
\curveto(163.65980014,408.47210263)(163.59480021,408.42710267)(163.53480591,408.38710297)
\curveto(163.47480033,408.35710274)(163.40980039,408.31710278)(163.33980591,408.26710297)
\curveto(163.2998005,408.24710285)(163.25480055,408.22710287)(163.20480591,408.20710297)
\curveto(163.16480064,408.18710291)(163.11980068,408.16710293)(163.06980591,408.14710297)
\curveto(162.92980087,408.097103)(162.77980102,408.05210305)(162.61980591,408.01210297)
\curveto(162.56980123,407.99210311)(162.52480128,407.98210312)(162.48480591,407.98210297)
\curveto(162.44480136,407.98210312)(162.4048014,407.97710312)(162.36480591,407.96710297)
\lineto(162.22980591,407.96710297)
\curveto(162.1998016,407.95710314)(162.15980164,407.95210315)(162.10980591,407.95210297)
\lineto(161.97480591,407.95210297)
\curveto(161.91480189,407.93210317)(161.82480198,407.92710317)(161.70480591,407.93710297)
\curveto(161.58480222,407.93710316)(161.4998023,407.94710315)(161.44980591,407.96710297)
\curveto(161.37980242,407.98710311)(161.31480249,407.9971031)(161.25480591,407.99710297)
\curveto(161.2048026,407.98710311)(161.14980265,407.99210311)(161.08980591,408.01210297)
\lineto(160.72980591,408.13210297)
\curveto(160.61980318,408.16210294)(160.50980329,408.2021029)(160.39980591,408.25210297)
\curveto(160.04980375,408.4021027)(159.73480407,408.63210247)(159.45480591,408.94210297)
\curveto(159.18480462,409.26210184)(158.96980483,409.5971015)(158.80980591,409.94710297)
\curveto(158.75980504,410.05710104)(158.71980508,410.16210094)(158.68980591,410.26210297)
\curveto(158.65980514,410.37210073)(158.62480518,410.48210062)(158.58480591,410.59210297)
\curveto(158.57480523,410.63210047)(158.56980523,410.66710043)(158.56980591,410.69710297)
\curveto(158.56980523,410.73710036)(158.55980524,410.78210032)(158.53980591,410.83210297)
\curveto(158.51980528,410.91210019)(158.4998053,410.9971001)(158.47980591,411.08710297)
\curveto(158.46980533,411.18709991)(158.45480535,411.28709981)(158.43480591,411.38710297)
\curveto(158.42480538,411.41709968)(158.41980538,411.45209965)(158.41980591,411.49210297)
\curveto(158.42980537,411.53209957)(158.42980537,411.56709953)(158.41980591,411.59710297)
\lineto(158.41980591,411.73210297)
\curveto(158.41980538,411.78209932)(158.41480539,411.83209927)(158.40480591,411.88210297)
\curveto(158.39480541,411.93209917)(158.38980541,411.98709911)(158.38980591,412.04710297)
\curveto(158.38980541,412.11709898)(158.39480541,412.17209893)(158.40480591,412.21210297)
\curveto(158.41480539,412.26209884)(158.41980538,412.30709879)(158.41980591,412.34710297)
\lineto(158.41980591,412.49710297)
\curveto(158.42980537,412.54709855)(158.42980537,412.59209851)(158.41980591,412.63210297)
\curveto(158.41980538,412.68209842)(158.42980537,412.73209837)(158.44980591,412.78210297)
\curveto(158.46980533,412.89209821)(158.48480532,412.9970981)(158.49480591,413.09710297)
\curveto(158.51480529,413.1970979)(158.53980526,413.2970978)(158.56980591,413.39710297)
\curveto(158.60980519,413.51709758)(158.64480516,413.63209747)(158.67480591,413.74210297)
\curveto(158.7048051,413.85209725)(158.74480506,413.96209714)(158.79480591,414.07210297)
\curveto(158.93480487,414.37209673)(159.10980469,414.65709644)(159.31980591,414.92710297)
\curveto(159.33980446,414.95709614)(159.36480444,414.98209612)(159.39480591,415.00210297)
\curveto(159.43480437,415.03209607)(159.46480434,415.06209604)(159.48480591,415.09210297)
\curveto(159.52480428,415.14209596)(159.56480424,415.18709591)(159.60480591,415.22710297)
\curveto(159.64480416,415.26709583)(159.68980411,415.30709579)(159.73980591,415.34710297)
\curveto(159.77980402,415.36709573)(159.81480399,415.39209571)(159.84480591,415.42210297)
\curveto(159.87480393,415.46209564)(159.90980389,415.49209561)(159.94980591,415.51210297)
\curveto(160.1998036,415.68209542)(160.48980331,415.82209528)(160.81980591,415.93210297)
\curveto(160.88980291,415.95209515)(160.95980284,415.96709513)(161.02980591,415.97710297)
\curveto(161.10980269,415.98709511)(161.18980261,416.0020951)(161.26980591,416.02210297)
\curveto(161.33980246,416.04209506)(161.42980237,416.05209505)(161.53980591,416.05210297)
\curveto(161.64980215,416.06209504)(161.75980204,416.06709503)(161.86980591,416.06710297)
\curveto(161.97980182,416.06709503)(162.08480172,416.06209504)(162.18480591,416.05210297)
\curveto(162.29480151,416.04209506)(162.38480142,416.02709507)(162.45480591,416.00710297)
\curveto(162.6048012,415.95709514)(162.74980105,415.91209519)(162.88980591,415.87210297)
\curveto(163.02980077,415.83209527)(163.15980064,415.77709532)(163.27980591,415.70710297)
\curveto(163.34980045,415.65709544)(163.41480039,415.60709549)(163.47480591,415.55710297)
\curveto(163.53480027,415.51709558)(163.5998002,415.47209563)(163.66980591,415.42210297)
\curveto(163.70980009,415.39209571)(163.76480004,415.35209575)(163.83480591,415.30210297)
\curveto(163.91479989,415.25209585)(163.98979981,415.25209585)(164.05980591,415.30210297)
\curveto(164.0997997,415.32209578)(164.11979968,415.35709574)(164.11980591,415.40710297)
\curveto(164.11979968,415.45709564)(164.12979967,415.50709559)(164.14980591,415.55710297)
\lineto(164.14980591,415.70710297)
\curveto(164.15979964,415.73709536)(164.16479964,415.77209533)(164.16480591,415.81210297)
\lineto(164.16480591,415.93210297)
\lineto(164.16480591,417.97210297)
\curveto(164.16479964,418.08209302)(164.15979964,418.2020929)(164.14980591,418.33210297)
\curveto(164.14979965,418.47209263)(164.17479963,418.57709252)(164.22480591,418.64710297)
\curveto(164.26479954,418.72709237)(164.33979946,418.77709232)(164.44980591,418.79710297)
\curveto(164.46979933,418.80709229)(164.48979931,418.80709229)(164.50980591,418.79710297)
\curveto(164.52979927,418.7970923)(164.54979925,418.8020923)(164.56980591,418.81210297)
\lineto(165.63480591,418.81210297)
\curveto(165.75479805,418.81209229)(165.86479794,418.80709229)(165.96480591,418.79710297)
\curveto(166.06479774,418.78709231)(166.13979766,418.74709235)(166.18980591,418.67710297)
\curveto(166.23979756,418.5970925)(166.26479754,418.49209261)(166.26480591,418.36210297)
\lineto(166.26480591,418.00210297)
\lineto(166.26480591,408.98710297)
\moveto(164.22480591,411.92710297)
\curveto(164.23479957,411.96709913)(164.23479957,412.00709909)(164.22480591,412.04710297)
\lineto(164.22480591,412.18210297)
\curveto(164.22479958,412.28209882)(164.21979958,412.38209872)(164.20980591,412.48210297)
\curveto(164.1997996,412.58209852)(164.18479962,412.67209843)(164.16480591,412.75210297)
\curveto(164.14479966,412.86209824)(164.12479968,412.96209814)(164.10480591,413.05210297)
\curveto(164.09479971,413.14209796)(164.06979973,413.22709787)(164.02980591,413.30710297)
\curveto(163.88979991,413.66709743)(163.68480012,413.95209715)(163.41480591,414.16210297)
\curveto(163.15480065,414.37209673)(162.77480103,414.47709662)(162.27480591,414.47710297)
\curveto(162.21480159,414.47709662)(162.13480167,414.46709663)(162.03480591,414.44710297)
\curveto(161.95480185,414.42709667)(161.87980192,414.40709669)(161.80980591,414.38710297)
\curveto(161.74980205,414.37709672)(161.68980211,414.35709674)(161.62980591,414.32710297)
\curveto(161.35980244,414.21709688)(161.14980265,414.04709705)(160.99980591,413.81710297)
\curveto(160.84980295,413.58709751)(160.72980307,413.32709777)(160.63980591,413.03710297)
\curveto(160.60980319,412.93709816)(160.58980321,412.83709826)(160.57980591,412.73710297)
\curveto(160.56980323,412.63709846)(160.54980325,412.53209857)(160.51980591,412.42210297)
\lineto(160.51980591,412.21210297)
\curveto(160.4998033,412.12209898)(160.49480331,411.9970991)(160.50480591,411.83710297)
\curveto(160.51480329,411.68709941)(160.52980327,411.57709952)(160.54980591,411.50710297)
\lineto(160.54980591,411.41710297)
\curveto(160.55980324,411.3970997)(160.56480324,411.37709972)(160.56480591,411.35710297)
\curveto(160.58480322,411.27709982)(160.5998032,411.2020999)(160.60980591,411.13210297)
\curveto(160.62980317,411.06210004)(160.64980315,410.98710011)(160.66980591,410.90710297)
\curveto(160.83980296,410.38710071)(161.12980267,410.0021011)(161.53980591,409.75210297)
\curveto(161.66980213,409.66210144)(161.84980195,409.59210151)(162.07980591,409.54210297)
\curveto(162.11980168,409.53210157)(162.17980162,409.52710157)(162.25980591,409.52710297)
\curveto(162.28980151,409.51710158)(162.33480147,409.50710159)(162.39480591,409.49710297)
\curveto(162.46480134,409.4971016)(162.51980128,409.5021016)(162.55980591,409.51210297)
\curveto(162.63980116,409.53210157)(162.71980108,409.54710155)(162.79980591,409.55710297)
\curveto(162.87980092,409.56710153)(162.95980084,409.58710151)(163.03980591,409.61710297)
\curveto(163.28980051,409.72710137)(163.48980031,409.86710123)(163.63980591,410.03710297)
\curveto(163.78980001,410.20710089)(163.91979988,410.42210068)(164.02980591,410.68210297)
\curveto(164.06979973,410.77210033)(164.0997997,410.86210024)(164.11980591,410.95210297)
\curveto(164.13979966,411.05210005)(164.15979964,411.15709994)(164.17980591,411.26710297)
\curveto(164.18979961,411.31709978)(164.18979961,411.36209974)(164.17980591,411.40210297)
\curveto(164.17979962,411.45209965)(164.18979961,411.5020996)(164.20980591,411.55210297)
\curveto(164.21979958,411.58209952)(164.22479958,411.61709948)(164.22480591,411.65710297)
\lineto(164.22480591,411.79210297)
\lineto(164.22480591,411.92710297)
}
}
{
\newrgbcolor{curcolor}{0 0 0}
\pscustom[linestyle=none,fillstyle=solid,fillcolor=curcolor]
{
\newpath
\moveto(175.20972778,412.07710297)
\curveto(175.22971962,411.9970991)(175.22971962,411.90709919)(175.20972778,411.80710297)
\curveto(175.18971966,411.70709939)(175.15471969,411.64209946)(175.10472778,411.61210297)
\curveto(175.05471979,411.57209953)(174.97971987,411.54209956)(174.87972778,411.52210297)
\curveto(174.78972006,411.51209959)(174.68472016,411.5020996)(174.56472778,411.49210297)
\lineto(174.21972778,411.49210297)
\curveto(174.10972074,411.5020996)(174.00972084,411.50709959)(173.91972778,411.50710297)
\lineto(170.25972778,411.50710297)
\lineto(170.04972778,411.50710297)
\curveto(169.98972486,411.50709959)(169.93472491,411.4970996)(169.88472778,411.47710297)
\curveto(169.80472504,411.43709966)(169.75472509,411.3970997)(169.73472778,411.35710297)
\curveto(169.71472513,411.33709976)(169.69472515,411.2970998)(169.67472778,411.23710297)
\curveto(169.65472519,411.18709991)(169.6497252,411.13709996)(169.65972778,411.08710297)
\curveto(169.67972517,411.02710007)(169.68972516,410.96710013)(169.68972778,410.90710297)
\curveto(169.69972515,410.85710024)(169.71472513,410.8021003)(169.73472778,410.74210297)
\curveto(169.81472503,410.5021006)(169.90972494,410.3021008)(170.01972778,410.14210297)
\curveto(170.13972471,409.99210111)(170.29972455,409.85710124)(170.49972778,409.73710297)
\curveto(170.57972427,409.68710141)(170.65972419,409.65210145)(170.73972778,409.63210297)
\curveto(170.82972402,409.62210148)(170.91972393,409.6021015)(171.00972778,409.57210297)
\curveto(171.08972376,409.55210155)(171.19972365,409.53710156)(171.33972778,409.52710297)
\curveto(171.47972337,409.51710158)(171.59972325,409.52210158)(171.69972778,409.54210297)
\lineto(171.83472778,409.54210297)
\curveto(171.93472291,409.56210154)(172.02472282,409.58210152)(172.10472778,409.60210297)
\curveto(172.19472265,409.63210147)(172.27972257,409.66210144)(172.35972778,409.69210297)
\curveto(172.45972239,409.74210136)(172.56972228,409.80710129)(172.68972778,409.88710297)
\curveto(172.81972203,409.96710113)(172.91472193,410.04710105)(172.97472778,410.12710297)
\curveto(173.02472182,410.1971009)(173.07472177,410.26210084)(173.12472778,410.32210297)
\curveto(173.18472166,410.39210071)(173.25472159,410.44210066)(173.33472778,410.47210297)
\curveto(173.43472141,410.52210058)(173.55972129,410.54210056)(173.70972778,410.53210297)
\lineto(174.14472778,410.53210297)
\lineto(174.32472778,410.53210297)
\curveto(174.39472045,410.54210056)(174.45472039,410.53710056)(174.50472778,410.51710297)
\lineto(174.65472778,410.51710297)
\curveto(174.75472009,410.4971006)(174.82472002,410.47210063)(174.86472778,410.44210297)
\curveto(174.90471994,410.42210068)(174.92471992,410.37710072)(174.92472778,410.30710297)
\curveto(174.93471991,410.23710086)(174.92971992,410.17710092)(174.90972778,410.12710297)
\curveto(174.85971999,409.98710111)(174.80472004,409.86210124)(174.74472778,409.75210297)
\curveto(174.68472016,409.64210146)(174.61472023,409.53210157)(174.53472778,409.42210297)
\curveto(174.31472053,409.09210201)(174.06472078,408.82710227)(173.78472778,408.62710297)
\curveto(173.50472134,408.42710267)(173.15472169,408.25710284)(172.73472778,408.11710297)
\curveto(172.62472222,408.07710302)(172.51472233,408.05210305)(172.40472778,408.04210297)
\curveto(172.29472255,408.03210307)(172.17972267,408.01210309)(172.05972778,407.98210297)
\curveto(172.01972283,407.97210313)(171.97472287,407.97210313)(171.92472778,407.98210297)
\curveto(171.88472296,407.98210312)(171.844723,407.97710312)(171.80472778,407.96710297)
\lineto(171.63972778,407.96710297)
\curveto(171.58972326,407.94710315)(171.52972332,407.94210316)(171.45972778,407.95210297)
\curveto(171.39972345,407.95210315)(171.3447235,407.95710314)(171.29472778,407.96710297)
\curveto(171.21472363,407.97710312)(171.1447237,407.97710312)(171.08472778,407.96710297)
\curveto(171.02472382,407.95710314)(170.95972389,407.96210314)(170.88972778,407.98210297)
\curveto(170.83972401,408.0021031)(170.78472406,408.01210309)(170.72472778,408.01210297)
\curveto(170.66472418,408.01210309)(170.60972424,408.02210308)(170.55972778,408.04210297)
\curveto(170.4497244,408.06210304)(170.33972451,408.08710301)(170.22972778,408.11710297)
\curveto(170.11972473,408.13710296)(170.01972483,408.17210293)(169.92972778,408.22210297)
\curveto(169.81972503,408.26210284)(169.71472513,408.2971028)(169.61472778,408.32710297)
\curveto(169.52472532,408.36710273)(169.43972541,408.41210269)(169.35972778,408.46210297)
\curveto(169.03972581,408.66210244)(168.75472609,408.89210221)(168.50472778,409.15210297)
\curveto(168.25472659,409.42210168)(168.0497268,409.73210137)(167.88972778,410.08210297)
\curveto(167.83972701,410.19210091)(167.79972705,410.3021008)(167.76972778,410.41210297)
\curveto(167.73972711,410.53210057)(167.69972715,410.65210045)(167.64972778,410.77210297)
\curveto(167.63972721,410.81210029)(167.63472721,410.84710025)(167.63472778,410.87710297)
\curveto(167.63472721,410.91710018)(167.62972722,410.95710014)(167.61972778,410.99710297)
\curveto(167.57972727,411.11709998)(167.55472729,411.24709985)(167.54472778,411.38710297)
\lineto(167.51472778,411.80710297)
\curveto(167.51472733,411.85709924)(167.50972734,411.91209919)(167.49972778,411.97210297)
\curveto(167.49972735,412.03209907)(167.50472734,412.08709901)(167.51472778,412.13710297)
\lineto(167.51472778,412.31710297)
\lineto(167.55972778,412.67710297)
\curveto(167.59972725,412.84709825)(167.63472721,413.01209809)(167.66472778,413.17210297)
\curveto(167.69472715,413.33209777)(167.73972711,413.48209762)(167.79972778,413.62210297)
\curveto(168.22972662,414.66209644)(168.95972589,415.3970957)(169.98972778,415.82710297)
\curveto(170.12972472,415.88709521)(170.26972458,415.92709517)(170.40972778,415.94710297)
\curveto(170.55972429,415.97709512)(170.71472413,416.01209509)(170.87472778,416.05210297)
\curveto(170.95472389,416.06209504)(171.02972382,416.06709503)(171.09972778,416.06710297)
\curveto(171.16972368,416.06709503)(171.2447236,416.07209503)(171.32472778,416.08210297)
\curveto(171.83472301,416.09209501)(172.26972258,416.03209507)(172.62972778,415.90210297)
\curveto(172.99972185,415.78209532)(173.32972152,415.62209548)(173.61972778,415.42210297)
\curveto(173.70972114,415.36209574)(173.79972105,415.29209581)(173.88972778,415.21210297)
\curveto(173.97972087,415.14209596)(174.05972079,415.06709603)(174.12972778,414.98710297)
\curveto(174.15972069,414.93709616)(174.19972065,414.8970962)(174.24972778,414.86710297)
\curveto(174.32972052,414.75709634)(174.40472044,414.64209646)(174.47472778,414.52210297)
\curveto(174.5447203,414.41209669)(174.61972023,414.2970968)(174.69972778,414.17710297)
\curveto(174.7497201,414.08709701)(174.78972006,413.99209711)(174.81972778,413.89210297)
\curveto(174.85971999,413.8020973)(174.89971995,413.7020974)(174.93972778,413.59210297)
\curveto(174.98971986,413.46209764)(175.02971982,413.32709777)(175.05972778,413.18710297)
\curveto(175.08971976,413.04709805)(175.12471972,412.90709819)(175.16472778,412.76710297)
\curveto(175.18471966,412.68709841)(175.18971966,412.5970985)(175.17972778,412.49710297)
\curveto(175.17971967,412.40709869)(175.18971966,412.32209878)(175.20972778,412.24210297)
\lineto(175.20972778,412.07710297)
\moveto(172.95972778,412.96210297)
\curveto(173.02972182,413.06209804)(173.03472181,413.18209792)(172.97472778,413.32210297)
\curveto(172.92472192,413.47209763)(172.88472196,413.58209752)(172.85472778,413.65210297)
\curveto(172.71472213,413.92209718)(172.52972232,414.12709697)(172.29972778,414.26710297)
\curveto(172.06972278,414.41709668)(171.7497231,414.4970966)(171.33972778,414.50710297)
\curveto(171.30972354,414.48709661)(171.27472357,414.48209662)(171.23472778,414.49210297)
\curveto(171.19472365,414.5020966)(171.15972369,414.5020966)(171.12972778,414.49210297)
\curveto(171.07972377,414.47209663)(171.02472382,414.45709664)(170.96472778,414.44710297)
\curveto(170.90472394,414.44709665)(170.849724,414.43709666)(170.79972778,414.41710297)
\curveto(170.35972449,414.27709682)(170.03472481,414.0020971)(169.82472778,413.59210297)
\curveto(169.80472504,413.55209755)(169.77972507,413.4970976)(169.74972778,413.42710297)
\curveto(169.72972512,413.36709773)(169.71472513,413.3020978)(169.70472778,413.23210297)
\curveto(169.69472515,413.17209793)(169.69472515,413.11209799)(169.70472778,413.05210297)
\curveto(169.72472512,412.99209811)(169.75972509,412.94209816)(169.80972778,412.90210297)
\curveto(169.88972496,412.85209825)(169.99972485,412.82709827)(170.13972778,412.82710297)
\lineto(170.54472778,412.82710297)
\lineto(172.20972778,412.82710297)
\lineto(172.64472778,412.82710297)
\curveto(172.80472204,412.83709826)(172.90972194,412.88209822)(172.95972778,412.96210297)
}
}
{
\newrgbcolor{curcolor}{0 0 0}
\pscustom[linestyle=none,fillstyle=solid,fillcolor=curcolor]
{
}
}
{
\newrgbcolor{curcolor}{0 0 0}
\pscustom[linestyle=none,fillstyle=solid,fillcolor=curcolor]
{
\newpath
\moveto(185.04316528,416.06710297)
\curveto(185.15315997,416.06709503)(185.24815987,416.05709504)(185.32816528,416.03710297)
\curveto(185.4181597,416.01709508)(185.48815963,415.97209513)(185.53816528,415.90210297)
\curveto(185.59815952,415.82209528)(185.62815949,415.68209542)(185.62816528,415.48210297)
\lineto(185.62816528,414.97210297)
\lineto(185.62816528,414.59710297)
\curveto(185.63815948,414.45709664)(185.6231595,414.34709675)(185.58316528,414.26710297)
\curveto(185.54315958,414.1970969)(185.48315964,414.15209695)(185.40316528,414.13210297)
\curveto(185.33315979,414.11209699)(185.24815987,414.102097)(185.14816528,414.10210297)
\curveto(185.05816006,414.102097)(184.95816016,414.10709699)(184.84816528,414.11710297)
\curveto(184.74816037,414.12709697)(184.65316047,414.12209698)(184.56316528,414.10210297)
\curveto(184.49316063,414.08209702)(184.4231607,414.06709703)(184.35316528,414.05710297)
\curveto(184.28316084,414.05709704)(184.2181609,414.04709705)(184.15816528,414.02710297)
\curveto(183.99816112,413.97709712)(183.83816128,413.9020972)(183.67816528,413.80210297)
\curveto(183.5181616,413.71209739)(183.39316173,413.60709749)(183.30316528,413.48710297)
\curveto(183.25316187,413.40709769)(183.19816192,413.32209778)(183.13816528,413.23210297)
\curveto(183.08816203,413.15209795)(183.03816208,413.06709803)(182.98816528,412.97710297)
\curveto(182.95816216,412.8970982)(182.92816219,412.81209829)(182.89816528,412.72210297)
\lineto(182.83816528,412.48210297)
\curveto(182.8181623,412.41209869)(182.80816231,412.33709876)(182.80816528,412.25710297)
\curveto(182.80816231,412.18709891)(182.79816232,412.11709898)(182.77816528,412.04710297)
\curveto(182.76816235,412.00709909)(182.76316236,411.96709913)(182.76316528,411.92710297)
\curveto(182.77316235,411.8970992)(182.77316235,411.86709923)(182.76316528,411.83710297)
\lineto(182.76316528,411.59710297)
\curveto(182.74316238,411.52709957)(182.73816238,411.44709965)(182.74816528,411.35710297)
\curveto(182.75816236,411.27709982)(182.76316236,411.1970999)(182.76316528,411.11710297)
\lineto(182.76316528,410.15710297)
\lineto(182.76316528,408.88210297)
\curveto(182.76316236,408.75210235)(182.75816236,408.63210247)(182.74816528,408.52210297)
\curveto(182.73816238,408.41210269)(182.70816241,408.32210278)(182.65816528,408.25210297)
\curveto(182.63816248,408.22210288)(182.60316252,408.1971029)(182.55316528,408.17710297)
\curveto(182.51316261,408.16710293)(182.46816265,408.15710294)(182.41816528,408.14710297)
\lineto(182.34316528,408.14710297)
\curveto(182.29316283,408.13710296)(182.23816288,408.13210297)(182.17816528,408.13210297)
\lineto(182.01316528,408.13210297)
\lineto(181.36816528,408.13210297)
\curveto(181.30816381,408.14210296)(181.24316388,408.14710295)(181.17316528,408.14710297)
\lineto(180.97816528,408.14710297)
\curveto(180.92816419,408.16710293)(180.87816424,408.18210292)(180.82816528,408.19210297)
\curveto(180.77816434,408.21210289)(180.74316438,408.24710285)(180.72316528,408.29710297)
\curveto(180.68316444,408.34710275)(180.65816446,408.41710268)(180.64816528,408.50710297)
\lineto(180.64816528,408.80710297)
\lineto(180.64816528,409.82710297)
\lineto(180.64816528,414.05710297)
\lineto(180.64816528,415.16710297)
\lineto(180.64816528,415.45210297)
\curveto(180.64816447,415.55209555)(180.66816445,415.63209547)(180.70816528,415.69210297)
\curveto(180.75816436,415.77209533)(180.83316429,415.82209528)(180.93316528,415.84210297)
\curveto(181.03316409,415.86209524)(181.15316397,415.87209523)(181.29316528,415.87210297)
\lineto(182.05816528,415.87210297)
\curveto(182.17816294,415.87209523)(182.28316284,415.86209524)(182.37316528,415.84210297)
\curveto(182.46316266,415.83209527)(182.53316259,415.78709531)(182.58316528,415.70710297)
\curveto(182.61316251,415.65709544)(182.62816249,415.58709551)(182.62816528,415.49710297)
\lineto(182.65816528,415.22710297)
\curveto(182.66816245,415.14709595)(182.68316244,415.07209603)(182.70316528,415.00210297)
\curveto(182.73316239,414.93209617)(182.78316234,414.8970962)(182.85316528,414.89710297)
\curveto(182.87316225,414.91709618)(182.89316223,414.92709617)(182.91316528,414.92710297)
\curveto(182.93316219,414.92709617)(182.95316217,414.93709616)(182.97316528,414.95710297)
\curveto(183.03316209,415.00709609)(183.08316204,415.06209604)(183.12316528,415.12210297)
\curveto(183.17316195,415.19209591)(183.23316189,415.25209585)(183.30316528,415.30210297)
\curveto(183.34316178,415.33209577)(183.37816174,415.36209574)(183.40816528,415.39210297)
\curveto(183.43816168,415.43209567)(183.47316165,415.46709563)(183.51316528,415.49710297)
\lineto(183.78316528,415.67710297)
\curveto(183.88316124,415.73709536)(183.98316114,415.79209531)(184.08316528,415.84210297)
\curveto(184.18316094,415.88209522)(184.28316084,415.91709518)(184.38316528,415.94710297)
\lineto(184.71316528,416.03710297)
\curveto(184.74316038,416.04709505)(184.79816032,416.04709505)(184.87816528,416.03710297)
\curveto(184.96816015,416.03709506)(185.0231601,416.04709505)(185.04316528,416.06710297)
}
}
{
\newrgbcolor{curcolor}{0 0 0}
\pscustom[linestyle=none,fillstyle=solid,fillcolor=curcolor]
{
\newpath
\moveto(193.54957153,412.07710297)
\curveto(193.56956337,411.9970991)(193.56956337,411.90709919)(193.54957153,411.80710297)
\curveto(193.52956341,411.70709939)(193.49456344,411.64209946)(193.44457153,411.61210297)
\curveto(193.39456354,411.57209953)(193.31956362,411.54209956)(193.21957153,411.52210297)
\curveto(193.12956381,411.51209959)(193.02456391,411.5020996)(192.90457153,411.49210297)
\lineto(192.55957153,411.49210297)
\curveto(192.44956449,411.5020996)(192.34956459,411.50709959)(192.25957153,411.50710297)
\lineto(188.59957153,411.50710297)
\lineto(188.38957153,411.50710297)
\curveto(188.32956861,411.50709959)(188.27456866,411.4970996)(188.22457153,411.47710297)
\curveto(188.14456879,411.43709966)(188.09456884,411.3970997)(188.07457153,411.35710297)
\curveto(188.05456888,411.33709976)(188.0345689,411.2970998)(188.01457153,411.23710297)
\curveto(187.99456894,411.18709991)(187.98956895,411.13709996)(187.99957153,411.08710297)
\curveto(188.01956892,411.02710007)(188.02956891,410.96710013)(188.02957153,410.90710297)
\curveto(188.0395689,410.85710024)(188.05456888,410.8021003)(188.07457153,410.74210297)
\curveto(188.15456878,410.5021006)(188.24956869,410.3021008)(188.35957153,410.14210297)
\curveto(188.47956846,409.99210111)(188.6395683,409.85710124)(188.83957153,409.73710297)
\curveto(188.91956802,409.68710141)(188.99956794,409.65210145)(189.07957153,409.63210297)
\curveto(189.16956777,409.62210148)(189.25956768,409.6021015)(189.34957153,409.57210297)
\curveto(189.42956751,409.55210155)(189.5395674,409.53710156)(189.67957153,409.52710297)
\curveto(189.81956712,409.51710158)(189.939567,409.52210158)(190.03957153,409.54210297)
\lineto(190.17457153,409.54210297)
\curveto(190.27456666,409.56210154)(190.36456657,409.58210152)(190.44457153,409.60210297)
\curveto(190.5345664,409.63210147)(190.61956632,409.66210144)(190.69957153,409.69210297)
\curveto(190.79956614,409.74210136)(190.90956603,409.80710129)(191.02957153,409.88710297)
\curveto(191.15956578,409.96710113)(191.25456568,410.04710105)(191.31457153,410.12710297)
\curveto(191.36456557,410.1971009)(191.41456552,410.26210084)(191.46457153,410.32210297)
\curveto(191.52456541,410.39210071)(191.59456534,410.44210066)(191.67457153,410.47210297)
\curveto(191.77456516,410.52210058)(191.89956504,410.54210056)(192.04957153,410.53210297)
\lineto(192.48457153,410.53210297)
\lineto(192.66457153,410.53210297)
\curveto(192.7345642,410.54210056)(192.79456414,410.53710056)(192.84457153,410.51710297)
\lineto(192.99457153,410.51710297)
\curveto(193.09456384,410.4971006)(193.16456377,410.47210063)(193.20457153,410.44210297)
\curveto(193.24456369,410.42210068)(193.26456367,410.37710072)(193.26457153,410.30710297)
\curveto(193.27456366,410.23710086)(193.26956367,410.17710092)(193.24957153,410.12710297)
\curveto(193.19956374,409.98710111)(193.14456379,409.86210124)(193.08457153,409.75210297)
\curveto(193.02456391,409.64210146)(192.95456398,409.53210157)(192.87457153,409.42210297)
\curveto(192.65456428,409.09210201)(192.40456453,408.82710227)(192.12457153,408.62710297)
\curveto(191.84456509,408.42710267)(191.49456544,408.25710284)(191.07457153,408.11710297)
\curveto(190.96456597,408.07710302)(190.85456608,408.05210305)(190.74457153,408.04210297)
\curveto(190.6345663,408.03210307)(190.51956642,408.01210309)(190.39957153,407.98210297)
\curveto(190.35956658,407.97210313)(190.31456662,407.97210313)(190.26457153,407.98210297)
\curveto(190.22456671,407.98210312)(190.18456675,407.97710312)(190.14457153,407.96710297)
\lineto(189.97957153,407.96710297)
\curveto(189.92956701,407.94710315)(189.86956707,407.94210316)(189.79957153,407.95210297)
\curveto(189.7395672,407.95210315)(189.68456725,407.95710314)(189.63457153,407.96710297)
\curveto(189.55456738,407.97710312)(189.48456745,407.97710312)(189.42457153,407.96710297)
\curveto(189.36456757,407.95710314)(189.29956764,407.96210314)(189.22957153,407.98210297)
\curveto(189.17956776,408.0021031)(189.12456781,408.01210309)(189.06457153,408.01210297)
\curveto(189.00456793,408.01210309)(188.94956799,408.02210308)(188.89957153,408.04210297)
\curveto(188.78956815,408.06210304)(188.67956826,408.08710301)(188.56957153,408.11710297)
\curveto(188.45956848,408.13710296)(188.35956858,408.17210293)(188.26957153,408.22210297)
\curveto(188.15956878,408.26210284)(188.05456888,408.2971028)(187.95457153,408.32710297)
\curveto(187.86456907,408.36710273)(187.77956916,408.41210269)(187.69957153,408.46210297)
\curveto(187.37956956,408.66210244)(187.09456984,408.89210221)(186.84457153,409.15210297)
\curveto(186.59457034,409.42210168)(186.38957055,409.73210137)(186.22957153,410.08210297)
\curveto(186.17957076,410.19210091)(186.1395708,410.3021008)(186.10957153,410.41210297)
\curveto(186.07957086,410.53210057)(186.0395709,410.65210045)(185.98957153,410.77210297)
\curveto(185.97957096,410.81210029)(185.97457096,410.84710025)(185.97457153,410.87710297)
\curveto(185.97457096,410.91710018)(185.96957097,410.95710014)(185.95957153,410.99710297)
\curveto(185.91957102,411.11709998)(185.89457104,411.24709985)(185.88457153,411.38710297)
\lineto(185.85457153,411.80710297)
\curveto(185.85457108,411.85709924)(185.84957109,411.91209919)(185.83957153,411.97210297)
\curveto(185.8395711,412.03209907)(185.84457109,412.08709901)(185.85457153,412.13710297)
\lineto(185.85457153,412.31710297)
\lineto(185.89957153,412.67710297)
\curveto(185.939571,412.84709825)(185.97457096,413.01209809)(186.00457153,413.17210297)
\curveto(186.0345709,413.33209777)(186.07957086,413.48209762)(186.13957153,413.62210297)
\curveto(186.56957037,414.66209644)(187.29956964,415.3970957)(188.32957153,415.82710297)
\curveto(188.46956847,415.88709521)(188.60956833,415.92709517)(188.74957153,415.94710297)
\curveto(188.89956804,415.97709512)(189.05456788,416.01209509)(189.21457153,416.05210297)
\curveto(189.29456764,416.06209504)(189.36956757,416.06709503)(189.43957153,416.06710297)
\curveto(189.50956743,416.06709503)(189.58456735,416.07209503)(189.66457153,416.08210297)
\curveto(190.17456676,416.09209501)(190.60956633,416.03209507)(190.96957153,415.90210297)
\curveto(191.3395656,415.78209532)(191.66956527,415.62209548)(191.95957153,415.42210297)
\curveto(192.04956489,415.36209574)(192.1395648,415.29209581)(192.22957153,415.21210297)
\curveto(192.31956462,415.14209596)(192.39956454,415.06709603)(192.46957153,414.98710297)
\curveto(192.49956444,414.93709616)(192.5395644,414.8970962)(192.58957153,414.86710297)
\curveto(192.66956427,414.75709634)(192.74456419,414.64209646)(192.81457153,414.52210297)
\curveto(192.88456405,414.41209669)(192.95956398,414.2970968)(193.03957153,414.17710297)
\curveto(193.08956385,414.08709701)(193.12956381,413.99209711)(193.15957153,413.89210297)
\curveto(193.19956374,413.8020973)(193.2395637,413.7020974)(193.27957153,413.59210297)
\curveto(193.32956361,413.46209764)(193.36956357,413.32709777)(193.39957153,413.18710297)
\curveto(193.42956351,413.04709805)(193.46456347,412.90709819)(193.50457153,412.76710297)
\curveto(193.52456341,412.68709841)(193.52956341,412.5970985)(193.51957153,412.49710297)
\curveto(193.51956342,412.40709869)(193.52956341,412.32209878)(193.54957153,412.24210297)
\lineto(193.54957153,412.07710297)
\moveto(191.29957153,412.96210297)
\curveto(191.36956557,413.06209804)(191.37456556,413.18209792)(191.31457153,413.32210297)
\curveto(191.26456567,413.47209763)(191.22456571,413.58209752)(191.19457153,413.65210297)
\curveto(191.05456588,413.92209718)(190.86956607,414.12709697)(190.63957153,414.26710297)
\curveto(190.40956653,414.41709668)(190.08956685,414.4970966)(189.67957153,414.50710297)
\curveto(189.64956729,414.48709661)(189.61456732,414.48209662)(189.57457153,414.49210297)
\curveto(189.5345674,414.5020966)(189.49956744,414.5020966)(189.46957153,414.49210297)
\curveto(189.41956752,414.47209663)(189.36456757,414.45709664)(189.30457153,414.44710297)
\curveto(189.24456769,414.44709665)(189.18956775,414.43709666)(189.13957153,414.41710297)
\curveto(188.69956824,414.27709682)(188.37456856,414.0020971)(188.16457153,413.59210297)
\curveto(188.14456879,413.55209755)(188.11956882,413.4970976)(188.08957153,413.42710297)
\curveto(188.06956887,413.36709773)(188.05456888,413.3020978)(188.04457153,413.23210297)
\curveto(188.0345689,413.17209793)(188.0345689,413.11209799)(188.04457153,413.05210297)
\curveto(188.06456887,412.99209811)(188.09956884,412.94209816)(188.14957153,412.90210297)
\curveto(188.22956871,412.85209825)(188.3395686,412.82709827)(188.47957153,412.82710297)
\lineto(188.88457153,412.82710297)
\lineto(190.54957153,412.82710297)
\lineto(190.98457153,412.82710297)
\curveto(191.14456579,412.83709826)(191.24956569,412.88209822)(191.29957153,412.96210297)
}
}
{
\newrgbcolor{curcolor}{0 0 0}
\pscustom[linestyle=none,fillstyle=solid,fillcolor=curcolor]
{
\newpath
\moveto(198.36785278,416.08210297)
\curveto(199.17784762,416.102095)(199.85284695,415.98209512)(200.39285278,415.72210297)
\curveto(200.94284586,415.46209564)(201.37784542,415.09209601)(201.69785278,414.61210297)
\curveto(201.85784494,414.37209673)(201.97784482,414.097097)(202.05785278,413.78710297)
\curveto(202.07784472,413.73709736)(202.09284471,413.67209743)(202.10285278,413.59210297)
\curveto(202.12284468,413.51209759)(202.12284468,413.44209766)(202.10285278,413.38210297)
\curveto(202.06284474,413.27209783)(201.99284481,413.20709789)(201.89285278,413.18710297)
\curveto(201.79284501,413.17709792)(201.67284513,413.17209793)(201.53285278,413.17210297)
\lineto(200.75285278,413.17210297)
\lineto(200.46785278,413.17210297)
\curveto(200.37784642,413.17209793)(200.3028465,413.19209791)(200.24285278,413.23210297)
\curveto(200.16284664,413.27209783)(200.10784669,413.33209777)(200.07785278,413.41210297)
\curveto(200.04784675,413.5020976)(200.00784679,413.59209751)(199.95785278,413.68210297)
\curveto(199.8978469,413.79209731)(199.83284697,413.89209721)(199.76285278,413.98210297)
\curveto(199.69284711,414.07209703)(199.61284719,414.15209695)(199.52285278,414.22210297)
\curveto(199.38284742,414.31209679)(199.22784757,414.38209672)(199.05785278,414.43210297)
\curveto(198.9978478,414.45209665)(198.93784786,414.46209664)(198.87785278,414.46210297)
\curveto(198.81784798,414.46209664)(198.76284804,414.47209663)(198.71285278,414.49210297)
\lineto(198.56285278,414.49210297)
\curveto(198.36284844,414.49209661)(198.2028486,414.47209663)(198.08285278,414.43210297)
\curveto(197.79284901,414.34209676)(197.55784924,414.2020969)(197.37785278,414.01210297)
\curveto(197.1978496,413.83209727)(197.05284975,413.61209749)(196.94285278,413.35210297)
\curveto(196.89284991,413.24209786)(196.85284995,413.12209798)(196.82285278,412.99210297)
\curveto(196.80285,412.87209823)(196.77785002,412.74209836)(196.74785278,412.60210297)
\curveto(196.73785006,412.56209854)(196.73285007,412.52209858)(196.73285278,412.48210297)
\curveto(196.73285007,412.44209866)(196.72785007,412.4020987)(196.71785278,412.36210297)
\curveto(196.6978501,412.26209884)(196.68785011,412.12209898)(196.68785278,411.94210297)
\curveto(196.6978501,411.76209934)(196.71285009,411.62209948)(196.73285278,411.52210297)
\curveto(196.73285007,411.44209966)(196.73785006,411.38709971)(196.74785278,411.35710297)
\curveto(196.76785003,411.28709981)(196.77785002,411.21709988)(196.77785278,411.14710297)
\curveto(196.78785001,411.07710002)(196.80285,411.00710009)(196.82285278,410.93710297)
\curveto(196.9028499,410.70710039)(196.9978498,410.4971006)(197.10785278,410.30710297)
\curveto(197.21784958,410.11710098)(197.35784944,409.95710114)(197.52785278,409.82710297)
\curveto(197.56784923,409.7971013)(197.62784917,409.76210134)(197.70785278,409.72210297)
\curveto(197.81784898,409.65210145)(197.92784887,409.60710149)(198.03785278,409.58710297)
\curveto(198.15784864,409.56710153)(198.3028485,409.54710155)(198.47285278,409.52710297)
\lineto(198.56285278,409.52710297)
\curveto(198.6028482,409.52710157)(198.63284817,409.53210157)(198.65285278,409.54210297)
\lineto(198.78785278,409.54210297)
\curveto(198.85784794,409.56210154)(198.92284788,409.57710152)(198.98285278,409.58710297)
\curveto(199.05284775,409.60710149)(199.11784768,409.62710147)(199.17785278,409.64710297)
\curveto(199.47784732,409.77710132)(199.70784709,409.96710113)(199.86785278,410.21710297)
\curveto(199.90784689,410.26710083)(199.94284686,410.32210078)(199.97285278,410.38210297)
\curveto(200.0028468,410.45210065)(200.02784677,410.51210059)(200.04785278,410.56210297)
\curveto(200.08784671,410.67210043)(200.12284668,410.76710033)(200.15285278,410.84710297)
\curveto(200.18284662,410.93710016)(200.25284655,411.00710009)(200.36285278,411.05710297)
\curveto(200.45284635,411.0971)(200.5978462,411.11209999)(200.79785278,411.10210297)
\lineto(201.29285278,411.10210297)
\lineto(201.50285278,411.10210297)
\curveto(201.58284522,411.11209999)(201.64784515,411.10709999)(201.69785278,411.08710297)
\lineto(201.81785278,411.08710297)
\lineto(201.93785278,411.05710297)
\curveto(201.97784482,411.05710004)(202.00784479,411.04710005)(202.02785278,411.02710297)
\curveto(202.07784472,410.98710011)(202.10784469,410.92710017)(202.11785278,410.84710297)
\curveto(202.13784466,410.77710032)(202.13784466,410.7021004)(202.11785278,410.62210297)
\curveto(202.02784477,410.29210081)(201.91784488,409.9971011)(201.78785278,409.73710297)
\curveto(201.37784542,408.96710213)(200.72284608,408.43210267)(199.82285278,408.13210297)
\curveto(199.72284708,408.102103)(199.61784718,408.08210302)(199.50785278,408.07210297)
\curveto(199.3978474,408.05210305)(199.28784751,408.02710307)(199.17785278,407.99710297)
\curveto(199.11784768,407.98710311)(199.05784774,407.98210312)(198.99785278,407.98210297)
\curveto(198.93784786,407.98210312)(198.87784792,407.97710312)(198.81785278,407.96710297)
\lineto(198.65285278,407.96710297)
\curveto(198.6028482,407.94710315)(198.52784827,407.94210316)(198.42785278,407.95210297)
\curveto(198.32784847,407.95210315)(198.25284855,407.95710314)(198.20285278,407.96710297)
\curveto(198.12284868,407.98710311)(198.04784875,407.9971031)(197.97785278,407.99710297)
\curveto(197.91784888,407.98710311)(197.85284895,407.99210311)(197.78285278,408.01210297)
\lineto(197.63285278,408.04210297)
\curveto(197.58284922,408.04210306)(197.53284927,408.04710305)(197.48285278,408.05710297)
\curveto(197.37284943,408.08710301)(197.26784953,408.11710298)(197.16785278,408.14710297)
\curveto(197.06784973,408.17710292)(196.97284983,408.21210289)(196.88285278,408.25210297)
\curveto(196.41285039,408.45210265)(196.01785078,408.70710239)(195.69785278,409.01710297)
\curveto(195.37785142,409.33710176)(195.11785168,409.73210137)(194.91785278,410.20210297)
\curveto(194.86785193,410.29210081)(194.82785197,410.38710071)(194.79785278,410.48710297)
\lineto(194.70785278,410.81710297)
\curveto(194.6978521,410.85710024)(194.69285211,410.89210021)(194.69285278,410.92210297)
\curveto(194.69285211,410.96210014)(194.68285212,411.00710009)(194.66285278,411.05710297)
\curveto(194.64285216,411.12709997)(194.63285217,411.1970999)(194.63285278,411.26710297)
\curveto(194.63285217,411.34709975)(194.62285218,411.42209968)(194.60285278,411.49210297)
\lineto(194.60285278,411.74710297)
\curveto(194.58285222,411.7970993)(194.57285223,411.85209925)(194.57285278,411.91210297)
\curveto(194.57285223,411.98209912)(194.58285222,412.04209906)(194.60285278,412.09210297)
\curveto(194.61285219,412.14209896)(194.61285219,412.18709891)(194.60285278,412.22710297)
\curveto(194.59285221,412.26709883)(194.59285221,412.30709879)(194.60285278,412.34710297)
\curveto(194.62285218,412.41709868)(194.62785217,412.48209862)(194.61785278,412.54210297)
\curveto(194.61785218,412.6020985)(194.62785217,412.66209844)(194.64785278,412.72210297)
\curveto(194.6978521,412.9020982)(194.73785206,413.07209803)(194.76785278,413.23210297)
\curveto(194.797852,413.4020977)(194.84285196,413.56709753)(194.90285278,413.72710297)
\curveto(195.12285168,414.23709686)(195.3978514,414.66209644)(195.72785278,415.00210297)
\curveto(196.06785073,415.34209576)(196.4978503,415.61709548)(197.01785278,415.82710297)
\curveto(197.15784964,415.88709521)(197.3028495,415.92709517)(197.45285278,415.94710297)
\curveto(197.6028492,415.97709512)(197.75784904,416.01209509)(197.91785278,416.05210297)
\curveto(197.9978488,416.06209504)(198.07284873,416.06709503)(198.14285278,416.06710297)
\curveto(198.21284859,416.06709503)(198.28784851,416.07209503)(198.36785278,416.08210297)
}
}
{
\newrgbcolor{curcolor}{0 0 0}
\pscustom[linestyle=none,fillstyle=solid,fillcolor=curcolor]
{
\newpath
\moveto(203.83113403,415.85710297)
\lineto(204.95613403,415.85710297)
\curveto(205.0661316,415.85709524)(205.1661315,415.85209525)(205.25613403,415.84210297)
\curveto(205.34613132,415.83209527)(205.41113125,415.7970953)(205.45113403,415.73710297)
\curveto(205.50113116,415.67709542)(205.53113113,415.59209551)(205.54113403,415.48210297)
\curveto(205.55113111,415.38209572)(205.55613111,415.27709582)(205.55613403,415.16710297)
\lineto(205.55613403,414.11710297)
\lineto(205.55613403,411.88210297)
\curveto(205.55613111,411.52209958)(205.57113109,411.18209992)(205.60113403,410.86210297)
\curveto(205.63113103,410.54210056)(205.72113094,410.27710082)(205.87113403,410.06710297)
\curveto(206.01113065,409.85710124)(206.23613043,409.70710139)(206.54613403,409.61710297)
\curveto(206.59613007,409.60710149)(206.63613003,409.6021015)(206.66613403,409.60210297)
\curveto(206.70612996,409.6021015)(206.75112991,409.5971015)(206.80113403,409.58710297)
\curveto(206.85112981,409.57710152)(206.90612976,409.57210153)(206.96613403,409.57210297)
\curveto(207.02612964,409.57210153)(207.07112959,409.57710152)(207.10113403,409.58710297)
\curveto(207.15112951,409.60710149)(207.19112947,409.61210149)(207.22113403,409.60210297)
\curveto(207.2611294,409.59210151)(207.30112936,409.5971015)(207.34113403,409.61710297)
\curveto(207.55112911,409.66710143)(207.71612895,409.73210137)(207.83613403,409.81210297)
\curveto(208.01612865,409.92210118)(208.15612851,410.06210104)(208.25613403,410.23210297)
\curveto(208.3661283,410.41210069)(208.44112822,410.60710049)(208.48113403,410.81710297)
\curveto(208.53112813,411.03710006)(208.5611281,411.27709982)(208.57113403,411.53710297)
\curveto(208.58112808,411.80709929)(208.58612808,412.08709901)(208.58613403,412.37710297)
\lineto(208.58613403,414.19210297)
\lineto(208.58613403,415.16710297)
\lineto(208.58613403,415.43710297)
\curveto(208.58612808,415.53709556)(208.60612806,415.61709548)(208.64613403,415.67710297)
\curveto(208.69612797,415.76709533)(208.77112789,415.81709528)(208.87113403,415.82710297)
\curveto(208.97112769,415.84709525)(209.09112757,415.85709524)(209.23113403,415.85710297)
\lineto(210.02613403,415.85710297)
\lineto(210.31113403,415.85710297)
\curveto(210.40112626,415.85709524)(210.47612619,415.83709526)(210.53613403,415.79710297)
\curveto(210.61612605,415.74709535)(210.661126,415.67209543)(210.67113403,415.57210297)
\curveto(210.68112598,415.47209563)(210.68612598,415.35709574)(210.68613403,415.22710297)
\lineto(210.68613403,414.08710297)
\lineto(210.68613403,409.87210297)
\lineto(210.68613403,408.80710297)
\lineto(210.68613403,408.50710297)
\curveto(210.68612598,408.40710269)(210.666126,408.33210277)(210.62613403,408.28210297)
\curveto(210.57612609,408.2021029)(210.50112616,408.15710294)(210.40113403,408.14710297)
\curveto(210.30112636,408.13710296)(210.19612647,408.13210297)(210.08613403,408.13210297)
\lineto(209.27613403,408.13210297)
\curveto(209.1661275,408.13210297)(209.0661276,408.13710296)(208.97613403,408.14710297)
\curveto(208.89612777,408.15710294)(208.83112783,408.1971029)(208.78113403,408.26710297)
\curveto(208.7611279,408.2971028)(208.74112792,408.34210276)(208.72113403,408.40210297)
\curveto(208.71112795,408.46210264)(208.69612797,408.52210258)(208.67613403,408.58210297)
\curveto(208.666128,408.64210246)(208.65112801,408.6971024)(208.63113403,408.74710297)
\curveto(208.61112805,408.7971023)(208.58112808,408.82710227)(208.54113403,408.83710297)
\curveto(208.52112814,408.85710224)(208.49612817,408.86210224)(208.46613403,408.85210297)
\curveto(208.43612823,408.84210226)(208.41112825,408.83210227)(208.39113403,408.82210297)
\curveto(208.32112834,408.78210232)(208.2611284,408.73710236)(208.21113403,408.68710297)
\curveto(208.1611285,408.63710246)(208.10612856,408.59210251)(208.04613403,408.55210297)
\curveto(208.00612866,408.52210258)(207.9661287,408.48710261)(207.92613403,408.44710297)
\curveto(207.89612877,408.41710268)(207.85612881,408.38710271)(207.80613403,408.35710297)
\curveto(207.57612909,408.21710288)(207.30612936,408.10710299)(206.99613403,408.02710297)
\curveto(206.92612974,408.00710309)(206.85612981,407.9971031)(206.78613403,407.99710297)
\curveto(206.71612995,407.98710311)(206.64113002,407.97210313)(206.56113403,407.95210297)
\curveto(206.52113014,407.94210316)(206.47613019,407.94210316)(206.42613403,407.95210297)
\curveto(206.38613028,407.95210315)(206.34613032,407.94710315)(206.30613403,407.93710297)
\curveto(206.27613039,407.92710317)(206.21113045,407.92710317)(206.11113403,407.93710297)
\curveto(206.02113064,407.93710316)(205.9611307,407.94210316)(205.93113403,407.95210297)
\curveto(205.88113078,407.95210315)(205.83113083,407.95710314)(205.78113403,407.96710297)
\lineto(205.63113403,407.96710297)
\curveto(205.51113115,407.9971031)(205.39613127,408.02210308)(205.28613403,408.04210297)
\curveto(205.17613149,408.06210304)(205.0661316,408.09210301)(204.95613403,408.13210297)
\curveto(204.90613176,408.15210295)(204.8611318,408.16710293)(204.82113403,408.17710297)
\curveto(204.79113187,408.1971029)(204.75113191,408.21710288)(204.70113403,408.23710297)
\curveto(204.35113231,408.42710267)(204.07113259,408.69210241)(203.86113403,409.03210297)
\curveto(203.73113293,409.24210186)(203.63613303,409.49210161)(203.57613403,409.78210297)
\curveto(203.51613315,410.08210102)(203.47613319,410.3971007)(203.45613403,410.72710297)
\curveto(203.44613322,411.06710003)(203.44113322,411.41209969)(203.44113403,411.76210297)
\curveto(203.45113321,412.12209898)(203.45613321,412.47709862)(203.45613403,412.82710297)
\lineto(203.45613403,414.86710297)
\curveto(203.45613321,414.9970961)(203.45113321,415.14709595)(203.44113403,415.31710297)
\curveto(203.44113322,415.4970956)(203.4661332,415.62709547)(203.51613403,415.70710297)
\curveto(203.54613312,415.75709534)(203.60613306,415.8020953)(203.69613403,415.84210297)
\curveto(203.75613291,415.84209526)(203.80113286,415.84709525)(203.83113403,415.85710297)
}
}
{
\newrgbcolor{curcolor}{0 0 0}
\pscustom[linestyle=none,fillstyle=solid,fillcolor=curcolor]
{
\newpath
\moveto(216.74238403,416.06710297)
\curveto(216.85237872,416.06709503)(216.94737862,416.05709504)(217.02738403,416.03710297)
\curveto(217.11737845,416.01709508)(217.18737838,415.97209513)(217.23738403,415.90210297)
\curveto(217.29737827,415.82209528)(217.32737824,415.68209542)(217.32738403,415.48210297)
\lineto(217.32738403,414.97210297)
\lineto(217.32738403,414.59710297)
\curveto(217.33737823,414.45709664)(217.32237825,414.34709675)(217.28238403,414.26710297)
\curveto(217.24237833,414.1970969)(217.18237839,414.15209695)(217.10238403,414.13210297)
\curveto(217.03237854,414.11209699)(216.94737862,414.102097)(216.84738403,414.10210297)
\curveto(216.75737881,414.102097)(216.65737891,414.10709699)(216.54738403,414.11710297)
\curveto(216.44737912,414.12709697)(216.35237922,414.12209698)(216.26238403,414.10210297)
\curveto(216.19237938,414.08209702)(216.12237945,414.06709703)(216.05238403,414.05710297)
\curveto(215.98237959,414.05709704)(215.91737965,414.04709705)(215.85738403,414.02710297)
\curveto(215.69737987,413.97709712)(215.53738003,413.9020972)(215.37738403,413.80210297)
\curveto(215.21738035,413.71209739)(215.09238048,413.60709749)(215.00238403,413.48710297)
\curveto(214.95238062,413.40709769)(214.89738067,413.32209778)(214.83738403,413.23210297)
\curveto(214.78738078,413.15209795)(214.73738083,413.06709803)(214.68738403,412.97710297)
\curveto(214.65738091,412.8970982)(214.62738094,412.81209829)(214.59738403,412.72210297)
\lineto(214.53738403,412.48210297)
\curveto(214.51738105,412.41209869)(214.50738106,412.33709876)(214.50738403,412.25710297)
\curveto(214.50738106,412.18709891)(214.49738107,412.11709898)(214.47738403,412.04710297)
\curveto(214.4673811,412.00709909)(214.46238111,411.96709913)(214.46238403,411.92710297)
\curveto(214.4723811,411.8970992)(214.4723811,411.86709923)(214.46238403,411.83710297)
\lineto(214.46238403,411.59710297)
\curveto(214.44238113,411.52709957)(214.43738113,411.44709965)(214.44738403,411.35710297)
\curveto(214.45738111,411.27709982)(214.46238111,411.1970999)(214.46238403,411.11710297)
\lineto(214.46238403,410.15710297)
\lineto(214.46238403,408.88210297)
\curveto(214.46238111,408.75210235)(214.45738111,408.63210247)(214.44738403,408.52210297)
\curveto(214.43738113,408.41210269)(214.40738116,408.32210278)(214.35738403,408.25210297)
\curveto(214.33738123,408.22210288)(214.30238127,408.1971029)(214.25238403,408.17710297)
\curveto(214.21238136,408.16710293)(214.1673814,408.15710294)(214.11738403,408.14710297)
\lineto(214.04238403,408.14710297)
\curveto(213.99238158,408.13710296)(213.93738163,408.13210297)(213.87738403,408.13210297)
\lineto(213.71238403,408.13210297)
\lineto(213.06738403,408.13210297)
\curveto(213.00738256,408.14210296)(212.94238263,408.14710295)(212.87238403,408.14710297)
\lineto(212.67738403,408.14710297)
\curveto(212.62738294,408.16710293)(212.57738299,408.18210292)(212.52738403,408.19210297)
\curveto(212.47738309,408.21210289)(212.44238313,408.24710285)(212.42238403,408.29710297)
\curveto(212.38238319,408.34710275)(212.35738321,408.41710268)(212.34738403,408.50710297)
\lineto(212.34738403,408.80710297)
\lineto(212.34738403,409.82710297)
\lineto(212.34738403,414.05710297)
\lineto(212.34738403,415.16710297)
\lineto(212.34738403,415.45210297)
\curveto(212.34738322,415.55209555)(212.3673832,415.63209547)(212.40738403,415.69210297)
\curveto(212.45738311,415.77209533)(212.53238304,415.82209528)(212.63238403,415.84210297)
\curveto(212.73238284,415.86209524)(212.85238272,415.87209523)(212.99238403,415.87210297)
\lineto(213.75738403,415.87210297)
\curveto(213.87738169,415.87209523)(213.98238159,415.86209524)(214.07238403,415.84210297)
\curveto(214.16238141,415.83209527)(214.23238134,415.78709531)(214.28238403,415.70710297)
\curveto(214.31238126,415.65709544)(214.32738124,415.58709551)(214.32738403,415.49710297)
\lineto(214.35738403,415.22710297)
\curveto(214.3673812,415.14709595)(214.38238119,415.07209603)(214.40238403,415.00210297)
\curveto(214.43238114,414.93209617)(214.48238109,414.8970962)(214.55238403,414.89710297)
\curveto(214.572381,414.91709618)(214.59238098,414.92709617)(214.61238403,414.92710297)
\curveto(214.63238094,414.92709617)(214.65238092,414.93709616)(214.67238403,414.95710297)
\curveto(214.73238084,415.00709609)(214.78238079,415.06209604)(214.82238403,415.12210297)
\curveto(214.8723807,415.19209591)(214.93238064,415.25209585)(215.00238403,415.30210297)
\curveto(215.04238053,415.33209577)(215.07738049,415.36209574)(215.10738403,415.39210297)
\curveto(215.13738043,415.43209567)(215.1723804,415.46709563)(215.21238403,415.49710297)
\lineto(215.48238403,415.67710297)
\curveto(215.58237999,415.73709536)(215.68237989,415.79209531)(215.78238403,415.84210297)
\curveto(215.88237969,415.88209522)(215.98237959,415.91709518)(216.08238403,415.94710297)
\lineto(216.41238403,416.03710297)
\curveto(216.44237913,416.04709505)(216.49737907,416.04709505)(216.57738403,416.03710297)
\curveto(216.6673789,416.03709506)(216.72237885,416.04709505)(216.74238403,416.06710297)
}
}
{
\newrgbcolor{curcolor}{0 0 0}
\pscustom[linestyle=none,fillstyle=solid,fillcolor=curcolor]
{
\newpath
\moveto(221.11746216,416.08210297)
\curveto(221.86745766,416.102095)(222.51745701,416.01709508)(223.06746216,415.82710297)
\curveto(223.6274559,415.64709545)(224.05245547,415.33209577)(224.34246216,414.88210297)
\curveto(224.41245511,414.77209633)(224.47245505,414.65709644)(224.52246216,414.53710297)
\curveto(224.58245494,414.42709667)(224.63245489,414.3020968)(224.67246216,414.16210297)
\curveto(224.69245483,414.102097)(224.70245482,414.03709706)(224.70246216,413.96710297)
\curveto(224.70245482,413.8970972)(224.69245483,413.83709726)(224.67246216,413.78710297)
\curveto(224.63245489,413.72709737)(224.57745495,413.68709741)(224.50746216,413.66710297)
\curveto(224.45745507,413.64709745)(224.39745513,413.63709746)(224.32746216,413.63710297)
\lineto(224.11746216,413.63710297)
\lineto(223.45746216,413.63710297)
\curveto(223.38745614,413.63709746)(223.31745621,413.63209747)(223.24746216,413.62210297)
\curveto(223.17745635,413.62209748)(223.11245641,413.63209747)(223.05246216,413.65210297)
\curveto(222.95245657,413.67209743)(222.87745665,413.71209739)(222.82746216,413.77210297)
\curveto(222.77745675,413.83209727)(222.73245679,413.89209721)(222.69246216,413.95210297)
\lineto(222.57246216,414.16210297)
\curveto(222.54245698,414.24209686)(222.49245703,414.30709679)(222.42246216,414.35710297)
\curveto(222.3224572,414.43709666)(222.2224573,414.4970966)(222.12246216,414.53710297)
\curveto(222.03245749,414.57709652)(221.91745761,414.61209649)(221.77746216,414.64210297)
\curveto(221.70745782,414.66209644)(221.60245792,414.67709642)(221.46246216,414.68710297)
\curveto(221.33245819,414.6970964)(221.23245829,414.69209641)(221.16246216,414.67210297)
\lineto(221.05746216,414.67210297)
\lineto(220.90746216,414.64210297)
\curveto(220.86745866,414.64209646)(220.8224587,414.63709646)(220.77246216,414.62710297)
\curveto(220.60245892,414.57709652)(220.46245906,414.50709659)(220.35246216,414.41710297)
\curveto(220.25245927,414.33709676)(220.18245934,414.21209689)(220.14246216,414.04210297)
\curveto(220.1224594,413.97209713)(220.1224594,413.90709719)(220.14246216,413.84710297)
\curveto(220.16245936,413.78709731)(220.18245934,413.73709736)(220.20246216,413.69710297)
\curveto(220.27245925,413.57709752)(220.35245917,413.48209762)(220.44246216,413.41210297)
\curveto(220.54245898,413.34209776)(220.65745887,413.28209782)(220.78746216,413.23210297)
\curveto(220.97745855,413.15209795)(221.18245834,413.08209802)(221.40246216,413.02210297)
\lineto(222.09246216,412.87210297)
\curveto(222.33245719,412.83209827)(222.56245696,412.78209832)(222.78246216,412.72210297)
\curveto(223.01245651,412.67209843)(223.2274563,412.60709849)(223.42746216,412.52710297)
\curveto(223.51745601,412.48709861)(223.60245592,412.45209865)(223.68246216,412.42210297)
\curveto(223.77245575,412.4020987)(223.85745567,412.36709873)(223.93746216,412.31710297)
\curveto(224.1274554,412.1970989)(224.29745523,412.06709903)(224.44746216,411.92710297)
\curveto(224.60745492,411.78709931)(224.73245479,411.61209949)(224.82246216,411.40210297)
\curveto(224.85245467,411.33209977)(224.87745465,411.26209984)(224.89746216,411.19210297)
\curveto(224.91745461,411.12209998)(224.93745459,411.04710005)(224.95746216,410.96710297)
\curveto(224.96745456,410.90710019)(224.97245455,410.81210029)(224.97246216,410.68210297)
\curveto(224.98245454,410.56210054)(224.98245454,410.46710063)(224.97246216,410.39710297)
\lineto(224.97246216,410.32210297)
\curveto(224.95245457,410.26210084)(224.93745459,410.2021009)(224.92746216,410.14210297)
\curveto(224.9274546,410.09210101)(224.9224546,410.04210106)(224.91246216,409.99210297)
\curveto(224.84245468,409.69210141)(224.73245479,409.42710167)(224.58246216,409.19710297)
\curveto(224.4224551,408.95710214)(224.2274553,408.76210234)(223.99746216,408.61210297)
\curveto(223.76745576,408.46210264)(223.50745602,408.33210277)(223.21746216,408.22210297)
\curveto(223.10745642,408.17210293)(222.98745654,408.13710296)(222.85746216,408.11710297)
\curveto(222.73745679,408.097103)(222.61745691,408.07210303)(222.49746216,408.04210297)
\curveto(222.40745712,408.02210308)(222.31245721,408.01210309)(222.21246216,408.01210297)
\curveto(222.1224574,408.0021031)(222.03245749,407.98710311)(221.94246216,407.96710297)
\lineto(221.67246216,407.96710297)
\curveto(221.61245791,407.94710315)(221.50745802,407.93710316)(221.35746216,407.93710297)
\curveto(221.21745831,407.93710316)(221.11745841,407.94710315)(221.05746216,407.96710297)
\curveto(221.0274585,407.96710313)(220.99245853,407.97210313)(220.95246216,407.98210297)
\lineto(220.84746216,407.98210297)
\curveto(220.7274588,408.0021031)(220.60745892,408.01710308)(220.48746216,408.02710297)
\curveto(220.36745916,408.03710306)(220.25245927,408.05710304)(220.14246216,408.08710297)
\curveto(219.75245977,408.1971029)(219.40746012,408.32210278)(219.10746216,408.46210297)
\curveto(218.80746072,408.61210249)(218.55246097,408.83210227)(218.34246216,409.12210297)
\curveto(218.20246132,409.31210179)(218.08246144,409.53210157)(217.98246216,409.78210297)
\curveto(217.96246156,409.84210126)(217.94246158,409.92210118)(217.92246216,410.02210297)
\curveto(217.90246162,410.07210103)(217.88746164,410.14210096)(217.87746216,410.23210297)
\curveto(217.86746166,410.32210078)(217.87246165,410.3971007)(217.89246216,410.45710297)
\curveto(217.9224616,410.52710057)(217.97246155,410.57710052)(218.04246216,410.60710297)
\curveto(218.09246143,410.62710047)(218.15246137,410.63710046)(218.22246216,410.63710297)
\lineto(218.44746216,410.63710297)
\lineto(219.15246216,410.63710297)
\lineto(219.39246216,410.63710297)
\curveto(219.47246005,410.63710046)(219.54245998,410.62710047)(219.60246216,410.60710297)
\curveto(219.71245981,410.56710053)(219.78245974,410.5021006)(219.81246216,410.41210297)
\curveto(219.85245967,410.32210078)(219.89745963,410.22710087)(219.94746216,410.12710297)
\curveto(219.96745956,410.07710102)(220.00245952,410.01210109)(220.05246216,409.93210297)
\curveto(220.11245941,409.85210125)(220.16245936,409.8021013)(220.20246216,409.78210297)
\curveto(220.3224592,409.68210142)(220.43745909,409.6021015)(220.54746216,409.54210297)
\curveto(220.65745887,409.49210161)(220.79745873,409.44210166)(220.96746216,409.39210297)
\curveto(221.01745851,409.37210173)(221.06745846,409.36210174)(221.11746216,409.36210297)
\curveto(221.16745836,409.37210173)(221.21745831,409.37210173)(221.26746216,409.36210297)
\curveto(221.34745818,409.34210176)(221.43245809,409.33210177)(221.52246216,409.33210297)
\curveto(221.6224579,409.34210176)(221.70745782,409.35710174)(221.77746216,409.37710297)
\curveto(221.8274577,409.38710171)(221.87245765,409.39210171)(221.91246216,409.39210297)
\curveto(221.96245756,409.39210171)(222.01245751,409.4021017)(222.06246216,409.42210297)
\curveto(222.20245732,409.47210163)(222.3274572,409.53210157)(222.43746216,409.60210297)
\curveto(222.55745697,409.67210143)(222.65245687,409.76210134)(222.72246216,409.87210297)
\curveto(222.77245675,409.95210115)(222.81245671,410.07710102)(222.84246216,410.24710297)
\curveto(222.86245666,410.31710078)(222.86245666,410.38210072)(222.84246216,410.44210297)
\curveto(222.8224567,410.5021006)(222.80245672,410.55210055)(222.78246216,410.59210297)
\curveto(222.71245681,410.73210037)(222.6224569,410.83710026)(222.51246216,410.90710297)
\curveto(222.41245711,410.97710012)(222.29245723,411.04210006)(222.15246216,411.10210297)
\curveto(221.96245756,411.18209992)(221.76245776,411.24709985)(221.55246216,411.29710297)
\curveto(221.34245818,411.34709975)(221.13245839,411.4020997)(220.92246216,411.46210297)
\curveto(220.84245868,411.48209962)(220.75745877,411.4970996)(220.66746216,411.50710297)
\curveto(220.58745894,411.51709958)(220.50745902,411.53209957)(220.42746216,411.55210297)
\curveto(220.10745942,411.64209946)(219.80245972,411.72709937)(219.51246216,411.80710297)
\curveto(219.2224603,411.8970992)(218.95746057,412.02709907)(218.71746216,412.19710297)
\curveto(218.43746109,412.3970987)(218.23246129,412.66709843)(218.10246216,413.00710297)
\curveto(218.08246144,413.07709802)(218.06246146,413.17209793)(218.04246216,413.29210297)
\curveto(218.0224615,413.36209774)(218.00746152,413.44709765)(217.99746216,413.54710297)
\curveto(217.98746154,413.64709745)(217.99246153,413.73709736)(218.01246216,413.81710297)
\curveto(218.03246149,413.86709723)(218.03746149,413.90709719)(218.02746216,413.93710297)
\curveto(218.01746151,413.97709712)(218.0224615,414.02209708)(218.04246216,414.07210297)
\curveto(218.06246146,414.18209692)(218.08246144,414.28209682)(218.10246216,414.37210297)
\curveto(218.13246139,414.47209663)(218.16746136,414.56709653)(218.20746216,414.65710297)
\curveto(218.33746119,414.94709615)(218.51746101,415.18209592)(218.74746216,415.36210297)
\curveto(218.97746055,415.54209556)(219.23746029,415.68709541)(219.52746216,415.79710297)
\curveto(219.63745989,415.84709525)(219.75245977,415.88209522)(219.87246216,415.90210297)
\curveto(219.99245953,415.93209517)(220.11745941,415.96209514)(220.24746216,415.99210297)
\curveto(220.30745922,416.01209509)(220.36745916,416.02209508)(220.42746216,416.02210297)
\lineto(220.60746216,416.05210297)
\curveto(220.68745884,416.06209504)(220.77245875,416.06709503)(220.86246216,416.06710297)
\curveto(220.95245857,416.06709503)(221.03745849,416.07209503)(221.11746216,416.08210297)
}
}
{
\newrgbcolor{curcolor}{0 0 0}
\pscustom[linestyle=none,fillstyle=solid,fillcolor=curcolor]
{
\newpath
\moveto(233.97410278,412.31710297)
\curveto(233.99409421,412.25709884)(234.0040942,412.17209893)(234.00410278,412.06210297)
\curveto(234.0040942,411.95209915)(233.99409421,411.86709923)(233.97410278,411.80710297)
\lineto(233.97410278,411.65710297)
\curveto(233.95409425,411.57709952)(233.94409426,411.4970996)(233.94410278,411.41710297)
\curveto(233.95409425,411.33709976)(233.94909426,411.25709984)(233.92910278,411.17710297)
\curveto(233.9090943,411.10709999)(233.89409431,411.04210006)(233.88410278,410.98210297)
\curveto(233.87409433,410.92210018)(233.86409434,410.85710024)(233.85410278,410.78710297)
\curveto(233.81409439,410.67710042)(233.77909443,410.56210054)(233.74910278,410.44210297)
\curveto(233.71909449,410.33210077)(233.67909453,410.22710087)(233.62910278,410.12710297)
\curveto(233.41909479,409.64710145)(233.14409506,409.25710184)(232.80410278,408.95710297)
\curveto(232.46409574,408.65710244)(232.05409615,408.40710269)(231.57410278,408.20710297)
\curveto(231.45409675,408.15710294)(231.32909688,408.12210298)(231.19910278,408.10210297)
\curveto(231.07909713,408.07210303)(230.95409725,408.04210306)(230.82410278,408.01210297)
\curveto(230.77409743,407.99210311)(230.71909749,407.98210312)(230.65910278,407.98210297)
\curveto(230.59909761,407.98210312)(230.54409766,407.97710312)(230.49410278,407.96710297)
\lineto(230.38910278,407.96710297)
\curveto(230.35909785,407.95710314)(230.32909788,407.95210315)(230.29910278,407.95210297)
\curveto(230.24909796,407.94210316)(230.16909804,407.93710316)(230.05910278,407.93710297)
\curveto(229.94909826,407.92710317)(229.86409834,407.93210317)(229.80410278,407.95210297)
\lineto(229.65410278,407.95210297)
\curveto(229.6040986,407.96210314)(229.54909866,407.96710313)(229.48910278,407.96710297)
\curveto(229.43909877,407.95710314)(229.38909882,407.96210314)(229.33910278,407.98210297)
\curveto(229.29909891,407.99210311)(229.25909895,407.9971031)(229.21910278,407.99710297)
\curveto(229.18909902,407.9971031)(229.14909906,408.0021031)(229.09910278,408.01210297)
\curveto(228.99909921,408.04210306)(228.89909931,408.06710303)(228.79910278,408.08710297)
\curveto(228.69909951,408.10710299)(228.6040996,408.13710296)(228.51410278,408.17710297)
\curveto(228.39409981,408.21710288)(228.27909993,408.25710284)(228.16910278,408.29710297)
\curveto(228.06910014,408.33710276)(227.96410024,408.38710271)(227.85410278,408.44710297)
\curveto(227.5041007,408.65710244)(227.204101,408.9021022)(226.95410278,409.18210297)
\curveto(226.7041015,409.46210164)(226.49410171,409.7971013)(226.32410278,410.18710297)
\curveto(226.27410193,410.27710082)(226.23410197,410.37210073)(226.20410278,410.47210297)
\curveto(226.18410202,410.57210053)(226.15910205,410.67710042)(226.12910278,410.78710297)
\curveto(226.1091021,410.83710026)(226.09910211,410.88210022)(226.09910278,410.92210297)
\curveto(226.09910211,410.96210014)(226.08910212,411.00710009)(226.06910278,411.05710297)
\curveto(226.04910216,411.13709996)(226.03910217,411.21709988)(226.03910278,411.29710297)
\curveto(226.03910217,411.38709971)(226.02910218,411.47209963)(226.00910278,411.55210297)
\curveto(225.99910221,411.6020995)(225.99410221,411.64709945)(225.99410278,411.68710297)
\lineto(225.99410278,411.82210297)
\curveto(225.97410223,411.88209922)(225.96410224,411.96709913)(225.96410278,412.07710297)
\curveto(225.97410223,412.18709891)(225.98910222,412.27209883)(226.00910278,412.33210297)
\lineto(226.00910278,412.43710297)
\curveto(226.01910219,412.48709861)(226.01910219,412.53709856)(226.00910278,412.58710297)
\curveto(226.0091022,412.64709845)(226.01910219,412.7020984)(226.03910278,412.75210297)
\curveto(226.04910216,412.8020983)(226.05410215,412.84709825)(226.05410278,412.88710297)
\curveto(226.05410215,412.93709816)(226.06410214,412.98709811)(226.08410278,413.03710297)
\curveto(226.12410208,413.16709793)(226.15910205,413.29209781)(226.18910278,413.41210297)
\curveto(226.21910199,413.54209756)(226.25910195,413.66709743)(226.30910278,413.78710297)
\curveto(226.48910172,414.1970969)(226.7041015,414.53709656)(226.95410278,414.80710297)
\curveto(227.204101,415.08709601)(227.5091007,415.34209576)(227.86910278,415.57210297)
\curveto(227.96910024,415.62209548)(228.07410013,415.66709543)(228.18410278,415.70710297)
\curveto(228.29409991,415.74709535)(228.4040998,415.79209531)(228.51410278,415.84210297)
\curveto(228.64409956,415.89209521)(228.77909943,415.92709517)(228.91910278,415.94710297)
\curveto(229.05909915,415.96709513)(229.204099,415.9970951)(229.35410278,416.03710297)
\curveto(229.43409877,416.04709505)(229.5090987,416.05209505)(229.57910278,416.05210297)
\curveto(229.64909856,416.05209505)(229.71909849,416.05709504)(229.78910278,416.06710297)
\curveto(230.36909784,416.07709502)(230.86909734,416.01709508)(231.28910278,415.88710297)
\curveto(231.71909649,415.75709534)(232.09909611,415.57709552)(232.42910278,415.34710297)
\curveto(232.53909567,415.26709583)(232.64909556,415.17709592)(232.75910278,415.07710297)
\curveto(232.87909533,414.98709611)(232.97909523,414.88709621)(233.05910278,414.77710297)
\curveto(233.13909507,414.67709642)(233.209095,414.57709652)(233.26910278,414.47710297)
\curveto(233.33909487,414.37709672)(233.4090948,414.27209683)(233.47910278,414.16210297)
\curveto(233.54909466,414.05209705)(233.6040946,413.93209717)(233.64410278,413.80210297)
\curveto(233.68409452,413.68209742)(233.72909448,413.55209755)(233.77910278,413.41210297)
\curveto(233.8090944,413.33209777)(233.83409437,413.24709785)(233.85410278,413.15710297)
\lineto(233.91410278,412.88710297)
\curveto(233.92409428,412.84709825)(233.92909428,412.80709829)(233.92910278,412.76710297)
\curveto(233.92909428,412.72709837)(233.93409427,412.68709841)(233.94410278,412.64710297)
\curveto(233.96409424,412.5970985)(233.96909424,412.54209856)(233.95910278,412.48210297)
\curveto(233.94909426,412.42209868)(233.95409425,412.36709873)(233.97410278,412.31710297)
\moveto(231.87410278,411.77710297)
\curveto(231.88409632,411.82709927)(231.88909632,411.8970992)(231.88910278,411.98710297)
\curveto(231.88909632,412.08709901)(231.88409632,412.16209894)(231.87410278,412.21210297)
\lineto(231.87410278,412.33210297)
\curveto(231.85409635,412.38209872)(231.84409636,412.43709866)(231.84410278,412.49710297)
\curveto(231.84409636,412.55709854)(231.83909637,412.61209849)(231.82910278,412.66210297)
\curveto(231.82909638,412.7020984)(231.82409638,412.73209837)(231.81410278,412.75210297)
\lineto(231.75410278,412.99210297)
\curveto(231.74409646,413.08209802)(231.72409648,413.16709793)(231.69410278,413.24710297)
\curveto(231.58409662,413.50709759)(231.45409675,413.72709737)(231.30410278,413.90710297)
\curveto(231.15409705,414.097097)(230.95409725,414.24709685)(230.70410278,414.35710297)
\curveto(230.64409756,414.37709672)(230.58409762,414.39209671)(230.52410278,414.40210297)
\curveto(230.46409774,414.42209668)(230.39909781,414.44209666)(230.32910278,414.46210297)
\curveto(230.24909796,414.48209662)(230.16409804,414.48709661)(230.07410278,414.47710297)
\lineto(229.80410278,414.47710297)
\curveto(229.77409843,414.45709664)(229.73909847,414.44709665)(229.69910278,414.44710297)
\curveto(229.65909855,414.45709664)(229.62409858,414.45709664)(229.59410278,414.44710297)
\lineto(229.38410278,414.38710297)
\curveto(229.32409888,414.37709672)(229.26909894,414.35709674)(229.21910278,414.32710297)
\curveto(228.96909924,414.21709688)(228.76409944,414.05709704)(228.60410278,413.84710297)
\curveto(228.45409975,413.64709745)(228.33409987,413.41209769)(228.24410278,413.14210297)
\curveto(228.21409999,413.04209806)(228.18910002,412.93709816)(228.16910278,412.82710297)
\curveto(228.15910005,412.71709838)(228.14410006,412.60709849)(228.12410278,412.49710297)
\curveto(228.11410009,412.44709865)(228.1091001,412.3970987)(228.10910278,412.34710297)
\lineto(228.10910278,412.19710297)
\curveto(228.08910012,412.12709897)(228.07910013,412.02209908)(228.07910278,411.88210297)
\curveto(228.08910012,411.74209936)(228.1041001,411.63709946)(228.12410278,411.56710297)
\lineto(228.12410278,411.43210297)
\curveto(228.14410006,411.35209975)(228.15910005,411.27209983)(228.16910278,411.19210297)
\curveto(228.17910003,411.12209998)(228.19410001,411.04710005)(228.21410278,410.96710297)
\curveto(228.31409989,410.66710043)(228.41909979,410.42210068)(228.52910278,410.23210297)
\curveto(228.64909956,410.05210105)(228.83409937,409.88710121)(229.08410278,409.73710297)
\curveto(229.15409905,409.68710141)(229.22909898,409.64710145)(229.30910278,409.61710297)
\curveto(229.39909881,409.58710151)(229.48909872,409.56210154)(229.57910278,409.54210297)
\curveto(229.61909859,409.53210157)(229.65409855,409.52710157)(229.68410278,409.52710297)
\curveto(229.71409849,409.53710156)(229.74909846,409.53710156)(229.78910278,409.52710297)
\lineto(229.90910278,409.49710297)
\curveto(229.95909825,409.4971016)(230.0040982,409.5021016)(230.04410278,409.51210297)
\lineto(230.16410278,409.51210297)
\curveto(230.24409796,409.53210157)(230.32409788,409.54710155)(230.40410278,409.55710297)
\curveto(230.48409772,409.56710153)(230.55909765,409.58710151)(230.62910278,409.61710297)
\curveto(230.88909732,409.71710138)(231.09909711,409.85210125)(231.25910278,410.02210297)
\curveto(231.41909679,410.19210091)(231.55409665,410.4021007)(231.66410278,410.65210297)
\curveto(231.7040965,410.75210035)(231.73409647,410.85210025)(231.75410278,410.95210297)
\curveto(231.77409643,411.05210005)(231.79909641,411.15709994)(231.82910278,411.26710297)
\curveto(231.83909637,411.30709979)(231.84409636,411.34209976)(231.84410278,411.37210297)
\curveto(231.84409636,411.41209969)(231.84909636,411.45209965)(231.85910278,411.49210297)
\lineto(231.85910278,411.62710297)
\curveto(231.85909635,411.67709942)(231.86409634,411.72709937)(231.87410278,411.77710297)
}
}
{
\newrgbcolor{curcolor}{0 0 0}
\pscustom[linestyle=none,fillstyle=solid,fillcolor=curcolor]
{
\newpath
\moveto(238.34402466,416.08210297)
\curveto(239.09402016,416.102095)(239.74401951,416.01709508)(240.29402466,415.82710297)
\curveto(240.8540184,415.64709545)(241.27901797,415.33209577)(241.56902466,414.88210297)
\curveto(241.63901761,414.77209633)(241.69901755,414.65709644)(241.74902466,414.53710297)
\curveto(241.80901744,414.42709667)(241.85901739,414.3020968)(241.89902466,414.16210297)
\curveto(241.91901733,414.102097)(241.92901732,414.03709706)(241.92902466,413.96710297)
\curveto(241.92901732,413.8970972)(241.91901733,413.83709726)(241.89902466,413.78710297)
\curveto(241.85901739,413.72709737)(241.80401745,413.68709741)(241.73402466,413.66710297)
\curveto(241.68401757,413.64709745)(241.62401763,413.63709746)(241.55402466,413.63710297)
\lineto(241.34402466,413.63710297)
\lineto(240.68402466,413.63710297)
\curveto(240.61401864,413.63709746)(240.54401871,413.63209747)(240.47402466,413.62210297)
\curveto(240.40401885,413.62209748)(240.33901891,413.63209747)(240.27902466,413.65210297)
\curveto(240.17901907,413.67209743)(240.10401915,413.71209739)(240.05402466,413.77210297)
\curveto(240.00401925,413.83209727)(239.95901929,413.89209721)(239.91902466,413.95210297)
\lineto(239.79902466,414.16210297)
\curveto(239.76901948,414.24209686)(239.71901953,414.30709679)(239.64902466,414.35710297)
\curveto(239.5490197,414.43709666)(239.4490198,414.4970966)(239.34902466,414.53710297)
\curveto(239.25901999,414.57709652)(239.14402011,414.61209649)(239.00402466,414.64210297)
\curveto(238.93402032,414.66209644)(238.82902042,414.67709642)(238.68902466,414.68710297)
\curveto(238.55902069,414.6970964)(238.45902079,414.69209641)(238.38902466,414.67210297)
\lineto(238.28402466,414.67210297)
\lineto(238.13402466,414.64210297)
\curveto(238.09402116,414.64209646)(238.0490212,414.63709646)(237.99902466,414.62710297)
\curveto(237.82902142,414.57709652)(237.68902156,414.50709659)(237.57902466,414.41710297)
\curveto(237.47902177,414.33709676)(237.40902184,414.21209689)(237.36902466,414.04210297)
\curveto(237.3490219,413.97209713)(237.3490219,413.90709719)(237.36902466,413.84710297)
\curveto(237.38902186,413.78709731)(237.40902184,413.73709736)(237.42902466,413.69710297)
\curveto(237.49902175,413.57709752)(237.57902167,413.48209762)(237.66902466,413.41210297)
\curveto(237.76902148,413.34209776)(237.88402137,413.28209782)(238.01402466,413.23210297)
\curveto(238.20402105,413.15209795)(238.40902084,413.08209802)(238.62902466,413.02210297)
\lineto(239.31902466,412.87210297)
\curveto(239.55901969,412.83209827)(239.78901946,412.78209832)(240.00902466,412.72210297)
\curveto(240.23901901,412.67209843)(240.4540188,412.60709849)(240.65402466,412.52710297)
\curveto(240.74401851,412.48709861)(240.82901842,412.45209865)(240.90902466,412.42210297)
\curveto(240.99901825,412.4020987)(241.08401817,412.36709873)(241.16402466,412.31710297)
\curveto(241.3540179,412.1970989)(241.52401773,412.06709903)(241.67402466,411.92710297)
\curveto(241.83401742,411.78709931)(241.95901729,411.61209949)(242.04902466,411.40210297)
\curveto(242.07901717,411.33209977)(242.10401715,411.26209984)(242.12402466,411.19210297)
\curveto(242.14401711,411.12209998)(242.16401709,411.04710005)(242.18402466,410.96710297)
\curveto(242.19401706,410.90710019)(242.19901705,410.81210029)(242.19902466,410.68210297)
\curveto(242.20901704,410.56210054)(242.20901704,410.46710063)(242.19902466,410.39710297)
\lineto(242.19902466,410.32210297)
\curveto(242.17901707,410.26210084)(242.16401709,410.2021009)(242.15402466,410.14210297)
\curveto(242.1540171,410.09210101)(242.1490171,410.04210106)(242.13902466,409.99210297)
\curveto(242.06901718,409.69210141)(241.95901729,409.42710167)(241.80902466,409.19710297)
\curveto(241.6490176,408.95710214)(241.4540178,408.76210234)(241.22402466,408.61210297)
\curveto(240.99401826,408.46210264)(240.73401852,408.33210277)(240.44402466,408.22210297)
\curveto(240.33401892,408.17210293)(240.21401904,408.13710296)(240.08402466,408.11710297)
\curveto(239.96401929,408.097103)(239.84401941,408.07210303)(239.72402466,408.04210297)
\curveto(239.63401962,408.02210308)(239.53901971,408.01210309)(239.43902466,408.01210297)
\curveto(239.3490199,408.0021031)(239.25901999,407.98710311)(239.16902466,407.96710297)
\lineto(238.89902466,407.96710297)
\curveto(238.83902041,407.94710315)(238.73402052,407.93710316)(238.58402466,407.93710297)
\curveto(238.44402081,407.93710316)(238.34402091,407.94710315)(238.28402466,407.96710297)
\curveto(238.254021,407.96710313)(238.21902103,407.97210313)(238.17902466,407.98210297)
\lineto(238.07402466,407.98210297)
\curveto(237.9540213,408.0021031)(237.83402142,408.01710308)(237.71402466,408.02710297)
\curveto(237.59402166,408.03710306)(237.47902177,408.05710304)(237.36902466,408.08710297)
\curveto(236.97902227,408.1971029)(236.63402262,408.32210278)(236.33402466,408.46210297)
\curveto(236.03402322,408.61210249)(235.77902347,408.83210227)(235.56902466,409.12210297)
\curveto(235.42902382,409.31210179)(235.30902394,409.53210157)(235.20902466,409.78210297)
\curveto(235.18902406,409.84210126)(235.16902408,409.92210118)(235.14902466,410.02210297)
\curveto(235.12902412,410.07210103)(235.11402414,410.14210096)(235.10402466,410.23210297)
\curveto(235.09402416,410.32210078)(235.09902415,410.3971007)(235.11902466,410.45710297)
\curveto(235.1490241,410.52710057)(235.19902405,410.57710052)(235.26902466,410.60710297)
\curveto(235.31902393,410.62710047)(235.37902387,410.63710046)(235.44902466,410.63710297)
\lineto(235.67402466,410.63710297)
\lineto(236.37902466,410.63710297)
\lineto(236.61902466,410.63710297)
\curveto(236.69902255,410.63710046)(236.76902248,410.62710047)(236.82902466,410.60710297)
\curveto(236.93902231,410.56710053)(237.00902224,410.5021006)(237.03902466,410.41210297)
\curveto(237.07902217,410.32210078)(237.12402213,410.22710087)(237.17402466,410.12710297)
\curveto(237.19402206,410.07710102)(237.22902202,410.01210109)(237.27902466,409.93210297)
\curveto(237.33902191,409.85210125)(237.38902186,409.8021013)(237.42902466,409.78210297)
\curveto(237.5490217,409.68210142)(237.66402159,409.6021015)(237.77402466,409.54210297)
\curveto(237.88402137,409.49210161)(238.02402123,409.44210166)(238.19402466,409.39210297)
\curveto(238.24402101,409.37210173)(238.29402096,409.36210174)(238.34402466,409.36210297)
\curveto(238.39402086,409.37210173)(238.44402081,409.37210173)(238.49402466,409.36210297)
\curveto(238.57402068,409.34210176)(238.65902059,409.33210177)(238.74902466,409.33210297)
\curveto(238.8490204,409.34210176)(238.93402032,409.35710174)(239.00402466,409.37710297)
\curveto(239.0540202,409.38710171)(239.09902015,409.39210171)(239.13902466,409.39210297)
\curveto(239.18902006,409.39210171)(239.23902001,409.4021017)(239.28902466,409.42210297)
\curveto(239.42901982,409.47210163)(239.5540197,409.53210157)(239.66402466,409.60210297)
\curveto(239.78401947,409.67210143)(239.87901937,409.76210134)(239.94902466,409.87210297)
\curveto(239.99901925,409.95210115)(240.03901921,410.07710102)(240.06902466,410.24710297)
\curveto(240.08901916,410.31710078)(240.08901916,410.38210072)(240.06902466,410.44210297)
\curveto(240.0490192,410.5021006)(240.02901922,410.55210055)(240.00902466,410.59210297)
\curveto(239.93901931,410.73210037)(239.8490194,410.83710026)(239.73902466,410.90710297)
\curveto(239.63901961,410.97710012)(239.51901973,411.04210006)(239.37902466,411.10210297)
\curveto(239.18902006,411.18209992)(238.98902026,411.24709985)(238.77902466,411.29710297)
\curveto(238.56902068,411.34709975)(238.35902089,411.4020997)(238.14902466,411.46210297)
\curveto(238.06902118,411.48209962)(237.98402127,411.4970996)(237.89402466,411.50710297)
\curveto(237.81402144,411.51709958)(237.73402152,411.53209957)(237.65402466,411.55210297)
\curveto(237.33402192,411.64209946)(237.02902222,411.72709937)(236.73902466,411.80710297)
\curveto(236.4490228,411.8970992)(236.18402307,412.02709907)(235.94402466,412.19710297)
\curveto(235.66402359,412.3970987)(235.45902379,412.66709843)(235.32902466,413.00710297)
\curveto(235.30902394,413.07709802)(235.28902396,413.17209793)(235.26902466,413.29210297)
\curveto(235.249024,413.36209774)(235.23402402,413.44709765)(235.22402466,413.54710297)
\curveto(235.21402404,413.64709745)(235.21902403,413.73709736)(235.23902466,413.81710297)
\curveto(235.25902399,413.86709723)(235.26402399,413.90709719)(235.25402466,413.93710297)
\curveto(235.24402401,413.97709712)(235.249024,414.02209708)(235.26902466,414.07210297)
\curveto(235.28902396,414.18209692)(235.30902394,414.28209682)(235.32902466,414.37210297)
\curveto(235.35902389,414.47209663)(235.39402386,414.56709653)(235.43402466,414.65710297)
\curveto(235.56402369,414.94709615)(235.74402351,415.18209592)(235.97402466,415.36210297)
\curveto(236.20402305,415.54209556)(236.46402279,415.68709541)(236.75402466,415.79710297)
\curveto(236.86402239,415.84709525)(236.97902227,415.88209522)(237.09902466,415.90210297)
\curveto(237.21902203,415.93209517)(237.34402191,415.96209514)(237.47402466,415.99210297)
\curveto(237.53402172,416.01209509)(237.59402166,416.02209508)(237.65402466,416.02210297)
\lineto(237.83402466,416.05210297)
\curveto(237.91402134,416.06209504)(237.99902125,416.06709503)(238.08902466,416.06710297)
\curveto(238.17902107,416.06709503)(238.26402099,416.07209503)(238.34402466,416.08210297)
}
}
{
\newrgbcolor{curcolor}{0 0 0}
\pscustom[linestyle=none,fillstyle=solid,fillcolor=curcolor]
{
}
}
{
\newrgbcolor{curcolor}{0 0 0}
\pscustom[linestyle=none,fillstyle=solid,fillcolor=curcolor]
{
\newpath
\moveto(255.48082153,412.09210297)
\curveto(255.49081285,412.03209907)(255.49581285,411.94209916)(255.49582153,411.82210297)
\curveto(255.49581285,411.7020994)(255.48581286,411.61709948)(255.46582153,411.56710297)
\lineto(255.46582153,411.37210297)
\curveto(255.43581291,411.26209984)(255.41581293,411.15709994)(255.40582153,411.05710297)
\curveto(255.40581294,410.95710014)(255.39081295,410.85710024)(255.36082153,410.75710297)
\curveto(255.340813,410.66710043)(255.32081302,410.57210053)(255.30082153,410.47210297)
\curveto(255.28081306,410.38210072)(255.25081309,410.29210081)(255.21082153,410.20210297)
\curveto(255.1408132,410.03210107)(255.07081327,409.87210123)(255.00082153,409.72210297)
\curveto(254.93081341,409.58210152)(254.85081349,409.44210166)(254.76082153,409.30210297)
\curveto(254.70081364,409.21210189)(254.63581371,409.12710197)(254.56582153,409.04710297)
\curveto(254.50581384,408.97710212)(254.43581391,408.9021022)(254.35582153,408.82210297)
\lineto(254.25082153,408.71710297)
\curveto(254.20081414,408.66710243)(254.1458142,408.62210248)(254.08582153,408.58210297)
\lineto(253.93582153,408.46210297)
\curveto(253.85581449,408.4021027)(253.76581458,408.34710275)(253.66582153,408.29710297)
\curveto(253.57581477,408.25710284)(253.48081486,408.21210289)(253.38082153,408.16210297)
\curveto(253.28081506,408.11210299)(253.17581517,408.07710302)(253.06582153,408.05710297)
\curveto(252.96581538,408.03710306)(252.86081548,408.01710308)(252.75082153,407.99710297)
\curveto(252.69081565,407.97710312)(252.62581572,407.96710313)(252.55582153,407.96710297)
\curveto(252.49581585,407.96710313)(252.43081591,407.95710314)(252.36082153,407.93710297)
\lineto(252.22582153,407.93710297)
\curveto(252.1458162,407.91710318)(252.07081627,407.91710318)(252.00082153,407.93710297)
\lineto(251.85082153,407.93710297)
\curveto(251.79081655,407.95710314)(251.72581662,407.96710313)(251.65582153,407.96710297)
\curveto(251.59581675,407.95710314)(251.53581681,407.96210314)(251.47582153,407.98210297)
\curveto(251.31581703,408.03210307)(251.16081718,408.07710302)(251.01082153,408.11710297)
\curveto(250.87081747,408.15710294)(250.7408176,408.21710288)(250.62082153,408.29710297)
\curveto(250.55081779,408.33710276)(250.48581786,408.37710272)(250.42582153,408.41710297)
\curveto(250.36581798,408.46710263)(250.30081804,408.51710258)(250.23082153,408.56710297)
\lineto(250.05082153,408.70210297)
\curveto(249.97081837,408.76210234)(249.90081844,408.76710233)(249.84082153,408.71710297)
\curveto(249.79081855,408.68710241)(249.76581858,408.64710245)(249.76582153,408.59710297)
\curveto(249.76581858,408.55710254)(249.75581859,408.50710259)(249.73582153,408.44710297)
\curveto(249.71581863,408.34710275)(249.70581864,408.23210287)(249.70582153,408.10210297)
\curveto(249.71581863,407.97210313)(249.72081862,407.85210325)(249.72082153,407.74210297)
\lineto(249.72082153,406.21210297)
\curveto(249.72081862,406.08210502)(249.71581863,405.95710514)(249.70582153,405.83710297)
\curveto(249.70581864,405.70710539)(249.68081866,405.6021055)(249.63082153,405.52210297)
\curveto(249.60081874,405.48210562)(249.5458188,405.45210565)(249.46582153,405.43210297)
\curveto(249.38581896,405.41210569)(249.29581905,405.4021057)(249.19582153,405.40210297)
\curveto(249.09581925,405.39210571)(248.99581935,405.39210571)(248.89582153,405.40210297)
\lineto(248.64082153,405.40210297)
\lineto(248.23582153,405.40210297)
\lineto(248.13082153,405.40210297)
\curveto(248.09082025,405.4021057)(248.05582029,405.40710569)(248.02582153,405.41710297)
\lineto(247.90582153,405.41710297)
\curveto(247.73582061,405.46710563)(247.6458207,405.56710553)(247.63582153,405.71710297)
\curveto(247.62582072,405.85710524)(247.62082072,406.02710507)(247.62082153,406.22710297)
\lineto(247.62082153,415.03210297)
\curveto(247.62082072,415.14209596)(247.61582073,415.25709584)(247.60582153,415.37710297)
\curveto(247.60582074,415.50709559)(247.63082071,415.60709549)(247.68082153,415.67710297)
\curveto(247.72082062,415.74709535)(247.77582057,415.79209531)(247.84582153,415.81210297)
\curveto(247.89582045,415.83209527)(247.95582039,415.84209526)(248.02582153,415.84210297)
\lineto(248.25082153,415.84210297)
\lineto(248.97082153,415.84210297)
\lineto(249.25582153,415.84210297)
\curveto(249.345819,415.84209526)(249.42081892,415.81709528)(249.48082153,415.76710297)
\curveto(249.55081879,415.71709538)(249.58581876,415.65209545)(249.58582153,415.57210297)
\curveto(249.59581875,415.5020956)(249.62081872,415.42709567)(249.66082153,415.34710297)
\curveto(249.67081867,415.31709578)(249.68081866,415.29209581)(249.69082153,415.27210297)
\curveto(249.71081863,415.26209584)(249.73081861,415.24709585)(249.75082153,415.22710297)
\curveto(249.86081848,415.21709588)(249.95081839,415.24709585)(250.02082153,415.31710297)
\curveto(250.09081825,415.38709571)(250.16081818,415.44709565)(250.23082153,415.49710297)
\curveto(250.36081798,415.58709551)(250.49581785,415.66709543)(250.63582153,415.73710297)
\curveto(250.77581757,415.81709528)(250.93081741,415.88209522)(251.10082153,415.93210297)
\curveto(251.18081716,415.96209514)(251.26581708,415.98209512)(251.35582153,415.99210297)
\curveto(251.45581689,416.0020951)(251.55081679,416.01709508)(251.64082153,416.03710297)
\curveto(251.68081666,416.04709505)(251.72081662,416.04709505)(251.76082153,416.03710297)
\curveto(251.81081653,416.02709507)(251.85081649,416.03209507)(251.88082153,416.05210297)
\curveto(252.45081589,416.07209503)(252.93081541,415.99209511)(253.32082153,415.81210297)
\curveto(253.72081462,415.64209546)(254.06081428,415.41709568)(254.34082153,415.13710297)
\curveto(254.39081395,415.08709601)(254.43581391,415.03709606)(254.47582153,414.98710297)
\curveto(254.51581383,414.94709615)(254.55581379,414.9020962)(254.59582153,414.85210297)
\curveto(254.66581368,414.76209634)(254.72581362,414.67209643)(254.77582153,414.58210297)
\curveto(254.83581351,414.49209661)(254.89081345,414.4020967)(254.94082153,414.31210297)
\curveto(254.96081338,414.29209681)(254.97081337,414.26709683)(254.97082153,414.23710297)
\curveto(254.98081336,414.20709689)(254.99581335,414.17209693)(255.01582153,414.13210297)
\curveto(255.07581327,414.03209707)(255.13081321,413.91209719)(255.18082153,413.77210297)
\curveto(255.20081314,413.71209739)(255.22081312,413.64709745)(255.24082153,413.57710297)
\curveto(255.26081308,413.51709758)(255.28081306,413.45209765)(255.30082153,413.38210297)
\curveto(255.340813,413.26209784)(255.36581298,413.13709796)(255.37582153,413.00710297)
\curveto(255.39581295,412.87709822)(255.42081292,412.74209836)(255.45082153,412.60210297)
\lineto(255.45082153,412.43710297)
\lineto(255.48082153,412.25710297)
\lineto(255.48082153,412.09210297)
\moveto(253.36582153,411.74710297)
\curveto(253.37581497,411.7970993)(253.38081496,411.86209924)(253.38082153,411.94210297)
\curveto(253.38081496,412.03209907)(253.37581497,412.102099)(253.36582153,412.15210297)
\lineto(253.36582153,412.28710297)
\curveto(253.345815,412.34709875)(253.33581501,412.41209869)(253.33582153,412.48210297)
\curveto(253.33581501,412.55209855)(253.32581502,412.62209848)(253.30582153,412.69210297)
\curveto(253.28581506,412.79209831)(253.26581508,412.88709821)(253.24582153,412.97710297)
\curveto(253.22581512,413.07709802)(253.19581515,413.16709793)(253.15582153,413.24710297)
\curveto(253.03581531,413.56709753)(252.88081546,413.82209728)(252.69082153,414.01210297)
\curveto(252.50081584,414.2020969)(252.23081611,414.34209676)(251.88082153,414.43210297)
\curveto(251.80081654,414.45209665)(251.71081663,414.46209664)(251.61082153,414.46210297)
\lineto(251.34082153,414.46210297)
\curveto(251.30081704,414.45209665)(251.26581708,414.44709665)(251.23582153,414.44710297)
\curveto(251.20581714,414.44709665)(251.17081717,414.44209666)(251.13082153,414.43210297)
\lineto(250.92082153,414.37210297)
\curveto(250.86081748,414.36209674)(250.80081754,414.34209676)(250.74082153,414.31210297)
\curveto(250.48081786,414.2020969)(250.27581807,414.03209707)(250.12582153,413.80210297)
\curveto(249.98581836,413.57209753)(249.87081847,413.31709778)(249.78082153,413.03710297)
\curveto(249.76081858,412.95709814)(249.7458186,412.87209823)(249.73582153,412.78210297)
\curveto(249.72581862,412.7020984)(249.71081863,412.62209848)(249.69082153,412.54210297)
\curveto(249.68081866,412.5020986)(249.67581867,412.43709866)(249.67582153,412.34710297)
\curveto(249.65581869,412.30709879)(249.65081869,412.25709884)(249.66082153,412.19710297)
\curveto(249.67081867,412.14709895)(249.67081867,412.097099)(249.66082153,412.04710297)
\curveto(249.6408187,411.98709911)(249.6408187,411.93209917)(249.66082153,411.88210297)
\lineto(249.66082153,411.70210297)
\lineto(249.66082153,411.56710297)
\curveto(249.66081868,411.52709957)(249.67081867,411.48709961)(249.69082153,411.44710297)
\curveto(249.69081865,411.37709972)(249.69581865,411.32209978)(249.70582153,411.28210297)
\lineto(249.73582153,411.10210297)
\curveto(249.7458186,411.04210006)(249.76081858,410.98210012)(249.78082153,410.92210297)
\curveto(249.87081847,410.63210047)(249.97581837,410.39210071)(250.09582153,410.20210297)
\curveto(250.22581812,410.02210108)(250.40581794,409.86210124)(250.63582153,409.72210297)
\curveto(250.77581757,409.64210146)(250.9408174,409.57710152)(251.13082153,409.52710297)
\curveto(251.17081717,409.51710158)(251.20581714,409.51210159)(251.23582153,409.51210297)
\curveto(251.26581708,409.52210158)(251.30081704,409.52210158)(251.34082153,409.51210297)
\curveto(251.38081696,409.5021016)(251.4408169,409.49210161)(251.52082153,409.48210297)
\curveto(251.60081674,409.48210162)(251.66581668,409.48710161)(251.71582153,409.49710297)
\curveto(251.79581655,409.51710158)(251.87581647,409.53210157)(251.95582153,409.54210297)
\curveto(252.0458163,409.56210154)(252.13081621,409.58710151)(252.21082153,409.61710297)
\curveto(252.45081589,409.71710138)(252.6458157,409.85710124)(252.79582153,410.03710297)
\curveto(252.9458154,410.21710088)(253.07081527,410.42710067)(253.17082153,410.66710297)
\curveto(253.22081512,410.78710031)(253.25581509,410.91210019)(253.27582153,411.04210297)
\curveto(253.29581505,411.17209993)(253.32081502,411.30709979)(253.35082153,411.44710297)
\lineto(253.35082153,411.59710297)
\curveto(253.36081498,411.64709945)(253.36581498,411.6970994)(253.36582153,411.74710297)
}
}
{
\newrgbcolor{curcolor}{0 0 0}
\pscustom[linestyle=none,fillstyle=solid,fillcolor=curcolor]
{
\newpath
\moveto(257.18074341,415.85710297)
\lineto(258.30574341,415.85710297)
\curveto(258.41574097,415.85709524)(258.51574087,415.85209525)(258.60574341,415.84210297)
\curveto(258.69574069,415.83209527)(258.76074063,415.7970953)(258.80074341,415.73710297)
\curveto(258.85074054,415.67709542)(258.88074051,415.59209551)(258.89074341,415.48210297)
\curveto(258.90074049,415.38209572)(258.90574048,415.27709582)(258.90574341,415.16710297)
\lineto(258.90574341,414.11710297)
\lineto(258.90574341,411.88210297)
\curveto(258.90574048,411.52209958)(258.92074047,411.18209992)(258.95074341,410.86210297)
\curveto(258.98074041,410.54210056)(259.07074032,410.27710082)(259.22074341,410.06710297)
\curveto(259.36074003,409.85710124)(259.5857398,409.70710139)(259.89574341,409.61710297)
\curveto(259.94573944,409.60710149)(259.9857394,409.6021015)(260.01574341,409.60210297)
\curveto(260.05573933,409.6021015)(260.10073929,409.5971015)(260.15074341,409.58710297)
\curveto(260.20073919,409.57710152)(260.25573913,409.57210153)(260.31574341,409.57210297)
\curveto(260.37573901,409.57210153)(260.42073897,409.57710152)(260.45074341,409.58710297)
\curveto(260.50073889,409.60710149)(260.54073885,409.61210149)(260.57074341,409.60210297)
\curveto(260.61073878,409.59210151)(260.65073874,409.5971015)(260.69074341,409.61710297)
\curveto(260.90073849,409.66710143)(261.06573832,409.73210137)(261.18574341,409.81210297)
\curveto(261.36573802,409.92210118)(261.50573788,410.06210104)(261.60574341,410.23210297)
\curveto(261.71573767,410.41210069)(261.7907376,410.60710049)(261.83074341,410.81710297)
\curveto(261.88073751,411.03710006)(261.91073748,411.27709982)(261.92074341,411.53710297)
\curveto(261.93073746,411.80709929)(261.93573745,412.08709901)(261.93574341,412.37710297)
\lineto(261.93574341,414.19210297)
\lineto(261.93574341,415.16710297)
\lineto(261.93574341,415.43710297)
\curveto(261.93573745,415.53709556)(261.95573743,415.61709548)(261.99574341,415.67710297)
\curveto(262.04573734,415.76709533)(262.12073727,415.81709528)(262.22074341,415.82710297)
\curveto(262.32073707,415.84709525)(262.44073695,415.85709524)(262.58074341,415.85710297)
\lineto(263.37574341,415.85710297)
\lineto(263.66074341,415.85710297)
\curveto(263.75073564,415.85709524)(263.82573556,415.83709526)(263.88574341,415.79710297)
\curveto(263.96573542,415.74709535)(264.01073538,415.67209543)(264.02074341,415.57210297)
\curveto(264.03073536,415.47209563)(264.03573535,415.35709574)(264.03574341,415.22710297)
\lineto(264.03574341,414.08710297)
\lineto(264.03574341,409.87210297)
\lineto(264.03574341,408.80710297)
\lineto(264.03574341,408.50710297)
\curveto(264.03573535,408.40710269)(264.01573537,408.33210277)(263.97574341,408.28210297)
\curveto(263.92573546,408.2021029)(263.85073554,408.15710294)(263.75074341,408.14710297)
\curveto(263.65073574,408.13710296)(263.54573584,408.13210297)(263.43574341,408.13210297)
\lineto(262.62574341,408.13210297)
\curveto(262.51573687,408.13210297)(262.41573697,408.13710296)(262.32574341,408.14710297)
\curveto(262.24573714,408.15710294)(262.18073721,408.1971029)(262.13074341,408.26710297)
\curveto(262.11073728,408.2971028)(262.0907373,408.34210276)(262.07074341,408.40210297)
\curveto(262.06073733,408.46210264)(262.04573734,408.52210258)(262.02574341,408.58210297)
\curveto(262.01573737,408.64210246)(262.00073739,408.6971024)(261.98074341,408.74710297)
\curveto(261.96073743,408.7971023)(261.93073746,408.82710227)(261.89074341,408.83710297)
\curveto(261.87073752,408.85710224)(261.84573754,408.86210224)(261.81574341,408.85210297)
\curveto(261.7857376,408.84210226)(261.76073763,408.83210227)(261.74074341,408.82210297)
\curveto(261.67073772,408.78210232)(261.61073778,408.73710236)(261.56074341,408.68710297)
\curveto(261.51073788,408.63710246)(261.45573793,408.59210251)(261.39574341,408.55210297)
\curveto(261.35573803,408.52210258)(261.31573807,408.48710261)(261.27574341,408.44710297)
\curveto(261.24573814,408.41710268)(261.20573818,408.38710271)(261.15574341,408.35710297)
\curveto(260.92573846,408.21710288)(260.65573873,408.10710299)(260.34574341,408.02710297)
\curveto(260.27573911,408.00710309)(260.20573918,407.9971031)(260.13574341,407.99710297)
\curveto(260.06573932,407.98710311)(259.9907394,407.97210313)(259.91074341,407.95210297)
\curveto(259.87073952,407.94210316)(259.82573956,407.94210316)(259.77574341,407.95210297)
\curveto(259.73573965,407.95210315)(259.69573969,407.94710315)(259.65574341,407.93710297)
\curveto(259.62573976,407.92710317)(259.56073983,407.92710317)(259.46074341,407.93710297)
\curveto(259.37074002,407.93710316)(259.31074008,407.94210316)(259.28074341,407.95210297)
\curveto(259.23074016,407.95210315)(259.18074021,407.95710314)(259.13074341,407.96710297)
\lineto(258.98074341,407.96710297)
\curveto(258.86074053,407.9971031)(258.74574064,408.02210308)(258.63574341,408.04210297)
\curveto(258.52574086,408.06210304)(258.41574097,408.09210301)(258.30574341,408.13210297)
\curveto(258.25574113,408.15210295)(258.21074118,408.16710293)(258.17074341,408.17710297)
\curveto(258.14074125,408.1971029)(258.10074129,408.21710288)(258.05074341,408.23710297)
\curveto(257.70074169,408.42710267)(257.42074197,408.69210241)(257.21074341,409.03210297)
\curveto(257.08074231,409.24210186)(256.9857424,409.49210161)(256.92574341,409.78210297)
\curveto(256.86574252,410.08210102)(256.82574256,410.3971007)(256.80574341,410.72710297)
\curveto(256.79574259,411.06710003)(256.7907426,411.41209969)(256.79074341,411.76210297)
\curveto(256.80074259,412.12209898)(256.80574258,412.47709862)(256.80574341,412.82710297)
\lineto(256.80574341,414.86710297)
\curveto(256.80574258,414.9970961)(256.80074259,415.14709595)(256.79074341,415.31710297)
\curveto(256.7907426,415.4970956)(256.81574257,415.62709547)(256.86574341,415.70710297)
\curveto(256.89574249,415.75709534)(256.95574243,415.8020953)(257.04574341,415.84210297)
\curveto(257.10574228,415.84209526)(257.15074224,415.84709525)(257.18074341,415.85710297)
}
}
{
\newrgbcolor{curcolor}{0 0 0}
\pscustom[linestyle=none,fillstyle=solid,fillcolor=curcolor]
{
\newpath
\moveto(273.46699341,412.39210297)
\curveto(273.48698481,412.33209877)(273.4969848,412.22709887)(273.49699341,412.07710297)
\curveto(273.4969848,411.93709916)(273.4919848,411.83709926)(273.48199341,411.77710297)
\curveto(273.48198481,411.72709937)(273.47698482,411.68209942)(273.46699341,411.64210297)
\lineto(273.46699341,411.52210297)
\curveto(273.44698485,411.44209966)(273.43698486,411.36209974)(273.43699341,411.28210297)
\curveto(273.43698486,411.21209989)(273.42698487,411.13709996)(273.40699341,411.05710297)
\curveto(273.40698489,411.01710008)(273.3969849,410.94710015)(273.37699341,410.84710297)
\curveto(273.34698495,410.72710037)(273.31698498,410.6021005)(273.28699341,410.47210297)
\curveto(273.26698503,410.35210075)(273.23198506,410.23710086)(273.18199341,410.12710297)
\curveto(273.00198529,409.67710142)(272.77698552,409.28710181)(272.50699341,408.95710297)
\curveto(272.23698606,408.62710247)(271.88198641,408.36710273)(271.44199341,408.17710297)
\curveto(271.35198694,408.13710296)(271.25698704,408.10710299)(271.15699341,408.08710297)
\curveto(271.06698723,408.05710304)(270.96698733,408.02710307)(270.85699341,407.99710297)
\curveto(270.7969875,407.97710312)(270.73198756,407.96710313)(270.66199341,407.96710297)
\curveto(270.60198769,407.96710313)(270.54198775,407.96210314)(270.48199341,407.95210297)
\lineto(270.34699341,407.95210297)
\curveto(270.28698801,407.93210317)(270.20698809,407.92710317)(270.10699341,407.93710297)
\curveto(270.00698829,407.93710316)(269.92698837,407.94710315)(269.86699341,407.96710297)
\lineto(269.77699341,407.96710297)
\curveto(269.72698857,407.97710312)(269.67198862,407.98710311)(269.61199341,407.99710297)
\curveto(269.55198874,407.9971031)(269.4919888,408.0021031)(269.43199341,408.01210297)
\curveto(269.24198905,408.06210304)(269.06698923,408.11210299)(268.90699341,408.16210297)
\curveto(268.74698955,408.21210289)(268.5969897,408.28210282)(268.45699341,408.37210297)
\lineto(268.27699341,408.49210297)
\curveto(268.22699007,408.53210257)(268.17699012,408.57710252)(268.12699341,408.62710297)
\lineto(268.03699341,408.68710297)
\curveto(268.00699029,408.70710239)(267.97699032,408.72210238)(267.94699341,408.73210297)
\curveto(267.85699044,408.76210234)(267.80199049,408.74210236)(267.78199341,408.67210297)
\curveto(267.73199056,408.6021025)(267.6969906,408.51710258)(267.67699341,408.41710297)
\curveto(267.66699063,408.32710277)(267.63199066,408.25710284)(267.57199341,408.20710297)
\curveto(267.51199078,408.16710293)(267.44199085,408.14210296)(267.36199341,408.13210297)
\lineto(267.09199341,408.13210297)
\lineto(266.37199341,408.13210297)
\lineto(266.14699341,408.13210297)
\curveto(266.07699222,408.12210298)(266.01199228,408.12710297)(265.95199341,408.14710297)
\curveto(265.81199248,408.1971029)(265.73199256,408.28710281)(265.71199341,408.41710297)
\curveto(265.70199259,408.55710254)(265.6969926,408.71210239)(265.69699341,408.88210297)
\lineto(265.69699341,418.03210297)
\lineto(265.69699341,418.37710297)
\curveto(265.6969926,418.4970926)(265.72199257,418.59209251)(265.77199341,418.66210297)
\curveto(265.81199248,418.73209237)(265.88199241,418.77709232)(265.98199341,418.79710297)
\curveto(266.00199229,418.80709229)(266.02199227,418.80709229)(266.04199341,418.79710297)
\curveto(266.07199222,418.7970923)(266.0969922,418.8020923)(266.11699341,418.81210297)
\lineto(267.06199341,418.81210297)
\curveto(267.24199105,418.81209229)(267.3969909,418.8020923)(267.52699341,418.78210297)
\curveto(267.65699064,418.77209233)(267.74199055,418.6970924)(267.78199341,418.55710297)
\curveto(267.81199048,418.45709264)(267.82199047,418.32209278)(267.81199341,418.15210297)
\curveto(267.80199049,417.99209311)(267.7969905,417.85209325)(267.79699341,417.73210297)
\lineto(267.79699341,416.09710297)
\lineto(267.79699341,415.76710297)
\curveto(267.7969905,415.65709544)(267.80699049,415.56209554)(267.82699341,415.48210297)
\curveto(267.83699046,415.43209567)(267.84699045,415.38709571)(267.85699341,415.34710297)
\curveto(267.86699043,415.31709578)(267.8919904,415.2970958)(267.93199341,415.28710297)
\curveto(267.95199034,415.26709583)(267.97699032,415.25709584)(268.00699341,415.25710297)
\curveto(268.04699025,415.25709584)(268.07699022,415.26209584)(268.09699341,415.27210297)
\curveto(268.16699013,415.31209579)(268.23199006,415.35209575)(268.29199341,415.39210297)
\curveto(268.35198994,415.44209566)(268.41698988,415.49209561)(268.48699341,415.54210297)
\curveto(268.61698968,415.63209547)(268.75198954,415.70709539)(268.89199341,415.76710297)
\curveto(269.03198926,415.83709526)(269.18698911,415.8970952)(269.35699341,415.94710297)
\curveto(269.43698886,415.97709512)(269.51698878,415.99209511)(269.59699341,415.99210297)
\curveto(269.67698862,416.0020951)(269.75698854,416.01709508)(269.83699341,416.03710297)
\curveto(269.90698839,416.05709504)(269.98198831,416.06709503)(270.06199341,416.06710297)
\lineto(270.30199341,416.06710297)
\lineto(270.45199341,416.06710297)
\curveto(270.48198781,416.05709504)(270.51698778,416.05209505)(270.55699341,416.05210297)
\curveto(270.5969877,416.06209504)(270.63698766,416.06209504)(270.67699341,416.05210297)
\curveto(270.78698751,416.02209508)(270.88698741,415.9970951)(270.97699341,415.97710297)
\curveto(271.07698722,415.96709513)(271.17198712,415.94209516)(271.26199341,415.90210297)
\curveto(271.72198657,415.71209539)(272.0969862,415.46709563)(272.38699341,415.16710297)
\curveto(272.67698562,414.86709623)(272.92198537,414.49209661)(273.12199341,414.04210297)
\curveto(273.17198512,413.92209718)(273.21198508,413.7970973)(273.24199341,413.66710297)
\curveto(273.28198501,413.53709756)(273.32198497,413.4020977)(273.36199341,413.26210297)
\curveto(273.38198491,413.19209791)(273.3919849,413.12209798)(273.39199341,413.05210297)
\curveto(273.40198489,412.99209811)(273.41698488,412.92209818)(273.43699341,412.84210297)
\curveto(273.45698484,412.79209831)(273.46198483,412.73709836)(273.45199341,412.67710297)
\curveto(273.45198484,412.61709848)(273.45698484,412.55709854)(273.46699341,412.49710297)
\lineto(273.46699341,412.39210297)
\moveto(271.24699341,410.98210297)
\curveto(271.27698702,411.08210002)(271.30198699,411.20709989)(271.32199341,411.35710297)
\curveto(271.35198694,411.50709959)(271.36698693,411.65709944)(271.36699341,411.80710297)
\curveto(271.37698692,411.96709913)(271.37698692,412.12209898)(271.36699341,412.27210297)
\curveto(271.36698693,412.43209867)(271.35198694,412.56709853)(271.32199341,412.67710297)
\curveto(271.291987,412.77709832)(271.27198702,412.87209823)(271.26199341,412.96210297)
\curveto(271.25198704,413.05209805)(271.22698707,413.13709796)(271.18699341,413.21710297)
\curveto(271.04698725,413.56709753)(270.84698745,413.86209724)(270.58699341,414.10210297)
\curveto(270.33698796,414.35209675)(269.96698833,414.47709662)(269.47699341,414.47710297)
\curveto(269.43698886,414.47709662)(269.40198889,414.47209663)(269.37199341,414.46210297)
\lineto(269.26699341,414.46210297)
\curveto(269.1969891,414.44209666)(269.13198916,414.42209668)(269.07199341,414.40210297)
\curveto(269.01198928,414.39209671)(268.95198934,414.37709672)(268.89199341,414.35710297)
\curveto(268.60198969,414.22709687)(268.38198991,414.04209706)(268.23199341,413.80210297)
\curveto(268.08199021,413.57209753)(267.95699034,413.30709779)(267.85699341,413.00710297)
\curveto(267.82699047,412.92709817)(267.80699049,412.84209826)(267.79699341,412.75210297)
\curveto(267.7969905,412.67209843)(267.78699051,412.59209851)(267.76699341,412.51210297)
\curveto(267.75699054,412.48209862)(267.75199054,412.43209867)(267.75199341,412.36210297)
\curveto(267.74199055,412.32209878)(267.73699056,412.28209882)(267.73699341,412.24210297)
\curveto(267.74699055,412.2020989)(267.74699055,412.16209894)(267.73699341,412.12210297)
\curveto(267.71699058,412.04209906)(267.71199058,411.93209917)(267.72199341,411.79210297)
\curveto(267.73199056,411.65209945)(267.74699055,411.55209955)(267.76699341,411.49210297)
\curveto(267.78699051,411.4020997)(267.7969905,411.31709978)(267.79699341,411.23710297)
\curveto(267.80699049,411.15709994)(267.82699047,411.07710002)(267.85699341,410.99710297)
\curveto(267.94699035,410.71710038)(268.05199024,410.47210063)(268.17199341,410.26210297)
\curveto(268.30198999,410.06210104)(268.48198981,409.89210121)(268.71199341,409.75210297)
\curveto(268.87198942,409.65210145)(269.03698926,409.58210152)(269.20699341,409.54210297)
\curveto(269.22698907,409.54210156)(269.24698905,409.53710156)(269.26699341,409.52710297)
\lineto(269.35699341,409.52710297)
\curveto(269.38698891,409.51710158)(269.43698886,409.50710159)(269.50699341,409.49710297)
\curveto(269.57698872,409.4971016)(269.63698866,409.5021016)(269.68699341,409.51210297)
\curveto(269.78698851,409.53210157)(269.87698842,409.54710155)(269.95699341,409.55710297)
\curveto(270.04698825,409.57710152)(270.13198816,409.6021015)(270.21199341,409.63210297)
\curveto(270.4919878,409.76210134)(270.70698759,409.94210116)(270.85699341,410.17210297)
\curveto(271.01698728,410.4021007)(271.14698715,410.67210043)(271.24699341,410.98210297)
}
}
{
\newrgbcolor{curcolor}{0 0 0}
\pscustom[linestyle=none,fillstyle=solid,fillcolor=curcolor]
{
\newpath
\moveto(275.34691528,418.82710297)
\lineto(276.44191528,418.82710297)
\curveto(276.5419128,418.82709227)(276.6369127,418.82209228)(276.72691528,418.81210297)
\curveto(276.81691252,418.8020923)(276.88691245,418.77209233)(276.93691528,418.72210297)
\curveto(276.99691234,418.65209245)(277.02691231,418.55709254)(277.02691528,418.43710297)
\curveto(277.0369123,418.32709277)(277.0419123,418.21209289)(277.04191528,418.09210297)
\lineto(277.04191528,416.75710297)
\lineto(277.04191528,411.37210297)
\lineto(277.04191528,409.07710297)
\lineto(277.04191528,408.65710297)
\curveto(277.05191229,408.50710259)(277.03191231,408.39210271)(276.98191528,408.31210297)
\curveto(276.93191241,408.23210287)(276.8419125,408.17710292)(276.71191528,408.14710297)
\curveto(276.65191269,408.12710297)(276.58191276,408.12210298)(276.50191528,408.13210297)
\curveto(276.43191291,408.14210296)(276.36191298,408.14710295)(276.29191528,408.14710297)
\lineto(275.57191528,408.14710297)
\curveto(275.46191388,408.14710295)(275.36191398,408.15210295)(275.27191528,408.16210297)
\curveto(275.18191416,408.17210293)(275.10691423,408.2021029)(275.04691528,408.25210297)
\curveto(274.98691435,408.3021028)(274.95191439,408.37710272)(274.94191528,408.47710297)
\lineto(274.94191528,408.80710297)
\lineto(274.94191528,410.14210297)
\lineto(274.94191528,415.76710297)
\lineto(274.94191528,417.80710297)
\curveto(274.9419144,417.93709316)(274.9369144,418.09209301)(274.92691528,418.27210297)
\curveto(274.92691441,418.45209265)(274.95191439,418.58209252)(275.00191528,418.66210297)
\curveto(275.02191432,418.7020924)(275.04691429,418.73209237)(275.07691528,418.75210297)
\lineto(275.19691528,418.81210297)
\curveto(275.21691412,418.81209229)(275.2419141,418.81209229)(275.27191528,418.81210297)
\curveto(275.30191404,418.82209228)(275.32691401,418.82709227)(275.34691528,418.82710297)
}
}
{
\newrgbcolor{curcolor}{0 0 0}
\pscustom[linestyle=none,fillstyle=solid,fillcolor=curcolor]
{
\newpath
\moveto(280.80410278,418.72210297)
\curveto(280.87409983,418.64209246)(280.9090998,418.52209258)(280.90910278,418.36210297)
\lineto(280.90910278,417.89710297)
\lineto(280.90910278,417.49210297)
\curveto(280.9090998,417.35209375)(280.87409983,417.25709384)(280.80410278,417.20710297)
\curveto(280.74409996,417.15709394)(280.66410004,417.12709397)(280.56410278,417.11710297)
\curveto(280.47410023,417.10709399)(280.37410033,417.102094)(280.26410278,417.10210297)
\lineto(279.42410278,417.10210297)
\curveto(279.31410139,417.102094)(279.21410149,417.10709399)(279.12410278,417.11710297)
\curveto(279.04410166,417.12709397)(278.97410173,417.15709394)(278.91410278,417.20710297)
\curveto(278.87410183,417.23709386)(278.84410186,417.29209381)(278.82410278,417.37210297)
\curveto(278.81410189,417.46209364)(278.8041019,417.55709354)(278.79410278,417.65710297)
\lineto(278.79410278,417.98710297)
\curveto(278.8041019,418.097093)(278.8091019,418.19209291)(278.80910278,418.27210297)
\lineto(278.80910278,418.48210297)
\curveto(278.81910189,418.55209255)(278.83910187,418.61209249)(278.86910278,418.66210297)
\curveto(278.88910182,418.7020924)(278.91410179,418.73209237)(278.94410278,418.75210297)
\lineto(279.06410278,418.81210297)
\curveto(279.08410162,418.81209229)(279.1091016,418.81209229)(279.13910278,418.81210297)
\curveto(279.16910154,418.82209228)(279.19410151,418.82709227)(279.21410278,418.82710297)
\lineto(280.30910278,418.82710297)
\curveto(280.4091003,418.82709227)(280.5041002,418.82209228)(280.59410278,418.81210297)
\curveto(280.68410002,418.8020923)(280.75409995,418.77209233)(280.80410278,418.72210297)
\moveto(280.90910278,408.95710297)
\curveto(280.9090998,408.75710234)(280.9040998,408.58710251)(280.89410278,408.44710297)
\curveto(280.88409982,408.30710279)(280.79409991,408.21210289)(280.62410278,408.16210297)
\curveto(280.56410014,408.14210296)(280.49910021,408.13210297)(280.42910278,408.13210297)
\curveto(280.35910035,408.14210296)(280.28410042,408.14710295)(280.20410278,408.14710297)
\lineto(279.36410278,408.14710297)
\curveto(279.27410143,408.14710295)(279.18410152,408.15210295)(279.09410278,408.16210297)
\curveto(279.01410169,408.17210293)(278.95410175,408.2021029)(278.91410278,408.25210297)
\curveto(278.85410185,408.32210278)(278.81910189,408.40710269)(278.80910278,408.50710297)
\lineto(278.80910278,408.85210297)
\lineto(278.80910278,415.18210297)
\lineto(278.80910278,415.48210297)
\curveto(278.8091019,415.58209552)(278.82910188,415.66209544)(278.86910278,415.72210297)
\curveto(278.92910178,415.79209531)(279.01410169,415.83709526)(279.12410278,415.85710297)
\curveto(279.14410156,415.86709523)(279.16910154,415.86709523)(279.19910278,415.85710297)
\curveto(279.23910147,415.85709524)(279.26910144,415.86209524)(279.28910278,415.87210297)
\lineto(280.03910278,415.87210297)
\lineto(280.23410278,415.87210297)
\curveto(280.31410039,415.88209522)(280.37910033,415.88209522)(280.42910278,415.87210297)
\lineto(280.54910278,415.87210297)
\curveto(280.6091001,415.85209525)(280.66410004,415.83709526)(280.71410278,415.82710297)
\curveto(280.76409994,415.81709528)(280.8040999,415.78709531)(280.83410278,415.73710297)
\curveto(280.87409983,415.68709541)(280.89409981,415.61709548)(280.89410278,415.52710297)
\curveto(280.9040998,415.43709566)(280.9090998,415.34209576)(280.90910278,415.24210297)
\lineto(280.90910278,408.95710297)
}
}
{
\newrgbcolor{curcolor}{0 0 0}
\pscustom[linestyle=none,fillstyle=solid,fillcolor=curcolor]
{
\newpath
\moveto(286.14129028,416.08210297)
\curveto(286.95128512,416.102095)(287.62628445,415.98209512)(288.16629028,415.72210297)
\curveto(288.71628336,415.46209564)(289.15128292,415.09209601)(289.47129028,414.61210297)
\curveto(289.63128244,414.37209673)(289.75128232,414.097097)(289.83129028,413.78710297)
\curveto(289.85128222,413.73709736)(289.86628221,413.67209743)(289.87629028,413.59210297)
\curveto(289.89628218,413.51209759)(289.89628218,413.44209766)(289.87629028,413.38210297)
\curveto(289.83628224,413.27209783)(289.76628231,413.20709789)(289.66629028,413.18710297)
\curveto(289.56628251,413.17709792)(289.44628263,413.17209793)(289.30629028,413.17210297)
\lineto(288.52629028,413.17210297)
\lineto(288.24129028,413.17210297)
\curveto(288.15128392,413.17209793)(288.076284,413.19209791)(288.01629028,413.23210297)
\curveto(287.93628414,413.27209783)(287.88128419,413.33209777)(287.85129028,413.41210297)
\curveto(287.82128425,413.5020976)(287.78128429,413.59209751)(287.73129028,413.68210297)
\curveto(287.6712844,413.79209731)(287.60628447,413.89209721)(287.53629028,413.98210297)
\curveto(287.46628461,414.07209703)(287.38628469,414.15209695)(287.29629028,414.22210297)
\curveto(287.15628492,414.31209679)(287.00128507,414.38209672)(286.83129028,414.43210297)
\curveto(286.7712853,414.45209665)(286.71128536,414.46209664)(286.65129028,414.46210297)
\curveto(286.59128548,414.46209664)(286.53628554,414.47209663)(286.48629028,414.49210297)
\lineto(286.33629028,414.49210297)
\curveto(286.13628594,414.49209661)(285.9762861,414.47209663)(285.85629028,414.43210297)
\curveto(285.56628651,414.34209676)(285.33128674,414.2020969)(285.15129028,414.01210297)
\curveto(284.9712871,413.83209727)(284.82628725,413.61209749)(284.71629028,413.35210297)
\curveto(284.66628741,413.24209786)(284.62628745,413.12209798)(284.59629028,412.99210297)
\curveto(284.5762875,412.87209823)(284.55128752,412.74209836)(284.52129028,412.60210297)
\curveto(284.51128756,412.56209854)(284.50628757,412.52209858)(284.50629028,412.48210297)
\curveto(284.50628757,412.44209866)(284.50128757,412.4020987)(284.49129028,412.36210297)
\curveto(284.4712876,412.26209884)(284.46128761,412.12209898)(284.46129028,411.94210297)
\curveto(284.4712876,411.76209934)(284.48628759,411.62209948)(284.50629028,411.52210297)
\curveto(284.50628757,411.44209966)(284.51128756,411.38709971)(284.52129028,411.35710297)
\curveto(284.54128753,411.28709981)(284.55128752,411.21709988)(284.55129028,411.14710297)
\curveto(284.56128751,411.07710002)(284.5762875,411.00710009)(284.59629028,410.93710297)
\curveto(284.6762874,410.70710039)(284.7712873,410.4971006)(284.88129028,410.30710297)
\curveto(284.99128708,410.11710098)(285.13128694,409.95710114)(285.30129028,409.82710297)
\curveto(285.34128673,409.7971013)(285.40128667,409.76210134)(285.48129028,409.72210297)
\curveto(285.59128648,409.65210145)(285.70128637,409.60710149)(285.81129028,409.58710297)
\curveto(285.93128614,409.56710153)(286.076286,409.54710155)(286.24629028,409.52710297)
\lineto(286.33629028,409.52710297)
\curveto(286.3762857,409.52710157)(286.40628567,409.53210157)(286.42629028,409.54210297)
\lineto(286.56129028,409.54210297)
\curveto(286.63128544,409.56210154)(286.69628538,409.57710152)(286.75629028,409.58710297)
\curveto(286.82628525,409.60710149)(286.89128518,409.62710147)(286.95129028,409.64710297)
\curveto(287.25128482,409.77710132)(287.48128459,409.96710113)(287.64129028,410.21710297)
\curveto(287.68128439,410.26710083)(287.71628436,410.32210078)(287.74629028,410.38210297)
\curveto(287.7762843,410.45210065)(287.80128427,410.51210059)(287.82129028,410.56210297)
\curveto(287.86128421,410.67210043)(287.89628418,410.76710033)(287.92629028,410.84710297)
\curveto(287.95628412,410.93710016)(288.02628405,411.00710009)(288.13629028,411.05710297)
\curveto(288.22628385,411.0971)(288.3712837,411.11209999)(288.57129028,411.10210297)
\lineto(289.06629028,411.10210297)
\lineto(289.27629028,411.10210297)
\curveto(289.35628272,411.11209999)(289.42128265,411.10709999)(289.47129028,411.08710297)
\lineto(289.59129028,411.08710297)
\lineto(289.71129028,411.05710297)
\curveto(289.75128232,411.05710004)(289.78128229,411.04710005)(289.80129028,411.02710297)
\curveto(289.85128222,410.98710011)(289.88128219,410.92710017)(289.89129028,410.84710297)
\curveto(289.91128216,410.77710032)(289.91128216,410.7021004)(289.89129028,410.62210297)
\curveto(289.80128227,410.29210081)(289.69128238,409.9971011)(289.56129028,409.73710297)
\curveto(289.15128292,408.96710213)(288.49628358,408.43210267)(287.59629028,408.13210297)
\curveto(287.49628458,408.102103)(287.39128468,408.08210302)(287.28129028,408.07210297)
\curveto(287.1712849,408.05210305)(287.06128501,408.02710307)(286.95129028,407.99710297)
\curveto(286.89128518,407.98710311)(286.83128524,407.98210312)(286.77129028,407.98210297)
\curveto(286.71128536,407.98210312)(286.65128542,407.97710312)(286.59129028,407.96710297)
\lineto(286.42629028,407.96710297)
\curveto(286.3762857,407.94710315)(286.30128577,407.94210316)(286.20129028,407.95210297)
\curveto(286.10128597,407.95210315)(286.02628605,407.95710314)(285.97629028,407.96710297)
\curveto(285.89628618,407.98710311)(285.82128625,407.9971031)(285.75129028,407.99710297)
\curveto(285.69128638,407.98710311)(285.62628645,407.99210311)(285.55629028,408.01210297)
\lineto(285.40629028,408.04210297)
\curveto(285.35628672,408.04210306)(285.30628677,408.04710305)(285.25629028,408.05710297)
\curveto(285.14628693,408.08710301)(285.04128703,408.11710298)(284.94129028,408.14710297)
\curveto(284.84128723,408.17710292)(284.74628733,408.21210289)(284.65629028,408.25210297)
\curveto(284.18628789,408.45210265)(283.79128828,408.70710239)(283.47129028,409.01710297)
\curveto(283.15128892,409.33710176)(282.89128918,409.73210137)(282.69129028,410.20210297)
\curveto(282.64128943,410.29210081)(282.60128947,410.38710071)(282.57129028,410.48710297)
\lineto(282.48129028,410.81710297)
\curveto(282.4712896,410.85710024)(282.46628961,410.89210021)(282.46629028,410.92210297)
\curveto(282.46628961,410.96210014)(282.45628962,411.00710009)(282.43629028,411.05710297)
\curveto(282.41628966,411.12709997)(282.40628967,411.1970999)(282.40629028,411.26710297)
\curveto(282.40628967,411.34709975)(282.39628968,411.42209968)(282.37629028,411.49210297)
\lineto(282.37629028,411.74710297)
\curveto(282.35628972,411.7970993)(282.34628973,411.85209925)(282.34629028,411.91210297)
\curveto(282.34628973,411.98209912)(282.35628972,412.04209906)(282.37629028,412.09210297)
\curveto(282.38628969,412.14209896)(282.38628969,412.18709891)(282.37629028,412.22710297)
\curveto(282.36628971,412.26709883)(282.36628971,412.30709879)(282.37629028,412.34710297)
\curveto(282.39628968,412.41709868)(282.40128967,412.48209862)(282.39129028,412.54210297)
\curveto(282.39128968,412.6020985)(282.40128967,412.66209844)(282.42129028,412.72210297)
\curveto(282.4712896,412.9020982)(282.51128956,413.07209803)(282.54129028,413.23210297)
\curveto(282.5712895,413.4020977)(282.61628946,413.56709753)(282.67629028,413.72710297)
\curveto(282.89628918,414.23709686)(283.1712889,414.66209644)(283.50129028,415.00210297)
\curveto(283.84128823,415.34209576)(284.2712878,415.61709548)(284.79129028,415.82710297)
\curveto(284.93128714,415.88709521)(285.076287,415.92709517)(285.22629028,415.94710297)
\curveto(285.3762867,415.97709512)(285.53128654,416.01209509)(285.69129028,416.05210297)
\curveto(285.7712863,416.06209504)(285.84628623,416.06709503)(285.91629028,416.06710297)
\curveto(285.98628609,416.06709503)(286.06128601,416.07209503)(286.14129028,416.08210297)
}
}
{
\newrgbcolor{curcolor}{0 0 0}
\pscustom[linestyle=none,fillstyle=solid,fillcolor=curcolor]
{
\newpath
\moveto(298.23457153,408.73210297)
\curveto(298.25456368,408.62210248)(298.26456367,408.51210259)(298.26457153,408.40210297)
\curveto(298.27456366,408.29210281)(298.22456371,408.21710288)(298.11457153,408.17710297)
\curveto(298.05456388,408.14710295)(297.98456395,408.13210297)(297.90457153,408.13210297)
\lineto(297.66457153,408.13210297)
\lineto(296.85457153,408.13210297)
\lineto(296.58457153,408.13210297)
\curveto(296.50456543,408.14210296)(296.4395655,408.16710293)(296.38957153,408.20710297)
\curveto(296.31956562,408.24710285)(296.26456567,408.3021028)(296.22457153,408.37210297)
\curveto(296.19456574,408.45210265)(296.14956579,408.51710258)(296.08957153,408.56710297)
\curveto(296.06956587,408.58710251)(296.04456589,408.6021025)(296.01457153,408.61210297)
\curveto(295.98456595,408.63210247)(295.94456599,408.63710246)(295.89457153,408.62710297)
\curveto(295.84456609,408.60710249)(295.79456614,408.58210252)(295.74457153,408.55210297)
\curveto(295.70456623,408.52210258)(295.65956628,408.4971026)(295.60957153,408.47710297)
\curveto(295.55956638,408.43710266)(295.50456643,408.4021027)(295.44457153,408.37210297)
\lineto(295.26457153,408.28210297)
\curveto(295.1345668,408.22210288)(294.99956694,408.17210293)(294.85957153,408.13210297)
\curveto(294.71956722,408.102103)(294.57456736,408.06710303)(294.42457153,408.02710297)
\curveto(294.35456758,408.00710309)(294.28456765,407.9971031)(294.21457153,407.99710297)
\curveto(294.15456778,407.98710311)(294.08956785,407.97710312)(294.01957153,407.96710297)
\lineto(293.92957153,407.96710297)
\curveto(293.89956804,407.95710314)(293.86956807,407.95210315)(293.83957153,407.95210297)
\lineto(293.67457153,407.95210297)
\curveto(293.57456836,407.93210317)(293.47456846,407.93210317)(293.37457153,407.95210297)
\lineto(293.23957153,407.95210297)
\curveto(293.16956877,407.97210313)(293.09956884,407.98210312)(293.02957153,407.98210297)
\curveto(292.96956897,407.97210313)(292.90956903,407.97710312)(292.84957153,407.99710297)
\curveto(292.74956919,408.01710308)(292.65456928,408.03710306)(292.56457153,408.05710297)
\curveto(292.47456946,408.06710303)(292.38956955,408.09210301)(292.30957153,408.13210297)
\curveto(292.01956992,408.24210286)(291.76957017,408.38210272)(291.55957153,408.55210297)
\curveto(291.35957058,408.73210237)(291.19957074,408.96710213)(291.07957153,409.25710297)
\curveto(291.04957089,409.32710177)(291.01957092,409.4021017)(290.98957153,409.48210297)
\curveto(290.96957097,409.56210154)(290.94957099,409.64710145)(290.92957153,409.73710297)
\curveto(290.90957103,409.78710131)(290.89957104,409.83710126)(290.89957153,409.88710297)
\curveto(290.90957103,409.93710116)(290.90957103,409.98710111)(290.89957153,410.03710297)
\curveto(290.88957105,410.06710103)(290.87957106,410.12710097)(290.86957153,410.21710297)
\curveto(290.86957107,410.31710078)(290.87457106,410.38710071)(290.88457153,410.42710297)
\curveto(290.90457103,410.52710057)(290.91457102,410.61210049)(290.91457153,410.68210297)
\lineto(291.00457153,411.01210297)
\curveto(291.0345709,411.13209997)(291.07457086,411.23709986)(291.12457153,411.32710297)
\curveto(291.29457064,411.61709948)(291.48957045,411.83709926)(291.70957153,411.98710297)
\curveto(291.92957001,412.13709896)(292.20956973,412.26709883)(292.54957153,412.37710297)
\curveto(292.67956926,412.42709867)(292.81456912,412.46209864)(292.95457153,412.48210297)
\curveto(293.09456884,412.5020986)(293.2345687,412.52709857)(293.37457153,412.55710297)
\curveto(293.45456848,412.57709852)(293.5395684,412.58709851)(293.62957153,412.58710297)
\curveto(293.71956822,412.5970985)(293.80956813,412.61209849)(293.89957153,412.63210297)
\curveto(293.96956797,412.65209845)(294.0395679,412.65709844)(294.10957153,412.64710297)
\curveto(294.17956776,412.64709845)(294.25456768,412.65709844)(294.33457153,412.67710297)
\curveto(294.40456753,412.6970984)(294.47456746,412.70709839)(294.54457153,412.70710297)
\curveto(294.61456732,412.70709839)(294.68956725,412.71709838)(294.76957153,412.73710297)
\curveto(294.97956696,412.78709831)(295.16956677,412.82709827)(295.33957153,412.85710297)
\curveto(295.51956642,412.8970982)(295.67956626,412.98709811)(295.81957153,413.12710297)
\curveto(295.90956603,413.21709788)(295.96956597,413.31709778)(295.99957153,413.42710297)
\curveto(296.00956593,413.45709764)(296.00956593,413.48209762)(295.99957153,413.50210297)
\curveto(295.99956594,413.52209758)(296.00456593,413.54209756)(296.01457153,413.56210297)
\curveto(296.02456591,413.58209752)(296.02956591,413.61209749)(296.02957153,413.65210297)
\lineto(296.02957153,413.74210297)
\lineto(295.99957153,413.86210297)
\curveto(295.99956594,413.9020972)(295.99456594,413.93709716)(295.98457153,413.96710297)
\curveto(295.88456605,414.26709683)(295.67456626,414.47209663)(295.35457153,414.58210297)
\curveto(295.26456667,414.61209649)(295.15456678,414.63209647)(295.02457153,414.64210297)
\curveto(294.90456703,414.66209644)(294.77956716,414.66709643)(294.64957153,414.65710297)
\curveto(294.51956742,414.65709644)(294.39456754,414.64709645)(294.27457153,414.62710297)
\curveto(294.15456778,414.60709649)(294.04956789,414.58209652)(293.95957153,414.55210297)
\curveto(293.89956804,414.53209657)(293.8395681,414.5020966)(293.77957153,414.46210297)
\curveto(293.72956821,414.43209667)(293.67956826,414.3970967)(293.62957153,414.35710297)
\curveto(293.57956836,414.31709678)(293.52456841,414.26209684)(293.46457153,414.19210297)
\curveto(293.41456852,414.12209698)(293.37956856,414.05709704)(293.35957153,413.99710297)
\curveto(293.30956863,413.8970972)(293.26456867,413.8020973)(293.22457153,413.71210297)
\curveto(293.19456874,413.62209748)(293.12456881,413.56209754)(293.01457153,413.53210297)
\curveto(292.934569,413.51209759)(292.84956909,413.5020976)(292.75957153,413.50210297)
\lineto(292.48957153,413.50210297)
\lineto(291.91957153,413.50210297)
\curveto(291.86957007,413.5020976)(291.81957012,413.4970976)(291.76957153,413.48710297)
\curveto(291.71957022,413.48709761)(291.67457026,413.49209761)(291.63457153,413.50210297)
\lineto(291.49957153,413.50210297)
\curveto(291.47957046,413.51209759)(291.45457048,413.51709758)(291.42457153,413.51710297)
\curveto(291.39457054,413.51709758)(291.36957057,413.52709757)(291.34957153,413.54710297)
\curveto(291.26957067,413.56709753)(291.21457072,413.63209747)(291.18457153,413.74210297)
\curveto(291.17457076,413.79209731)(291.17457076,413.84209726)(291.18457153,413.89210297)
\curveto(291.19457074,413.94209716)(291.20457073,413.98709711)(291.21457153,414.02710297)
\curveto(291.24457069,414.13709696)(291.27457066,414.23709686)(291.30457153,414.32710297)
\curveto(291.34457059,414.42709667)(291.38957055,414.51709658)(291.43957153,414.59710297)
\lineto(291.52957153,414.74710297)
\lineto(291.61957153,414.89710297)
\curveto(291.69957024,415.00709609)(291.79957014,415.11209599)(291.91957153,415.21210297)
\curveto(291.93957,415.22209588)(291.96956997,415.24709585)(292.00957153,415.28710297)
\curveto(292.05956988,415.32709577)(292.10456983,415.36209574)(292.14457153,415.39210297)
\curveto(292.18456975,415.42209568)(292.22956971,415.45209565)(292.27957153,415.48210297)
\curveto(292.44956949,415.59209551)(292.62956931,415.67709542)(292.81957153,415.73710297)
\curveto(293.00956893,415.80709529)(293.20456873,415.87209523)(293.40457153,415.93210297)
\curveto(293.52456841,415.96209514)(293.64956829,415.98209512)(293.77957153,415.99210297)
\curveto(293.90956803,416.0020951)(294.0395679,416.02209508)(294.16957153,416.05210297)
\curveto(294.20956773,416.06209504)(294.26956767,416.06209504)(294.34957153,416.05210297)
\curveto(294.4395675,416.04209506)(294.49456744,416.04709505)(294.51457153,416.06710297)
\curveto(294.92456701,416.07709502)(295.31456662,416.06209504)(295.68457153,416.02210297)
\curveto(296.06456587,415.98209512)(296.40456553,415.90709519)(296.70457153,415.79710297)
\curveto(297.01456492,415.68709541)(297.27956466,415.53709556)(297.49957153,415.34710297)
\curveto(297.71956422,415.16709593)(297.88956405,414.93209617)(298.00957153,414.64210297)
\curveto(298.07956386,414.47209663)(298.11956382,414.27709682)(298.12957153,414.05710297)
\curveto(298.1395638,413.83709726)(298.14456379,413.61209749)(298.14457153,413.38210297)
\lineto(298.14457153,410.03710297)
\lineto(298.14457153,409.45210297)
\curveto(298.14456379,409.26210184)(298.16456377,409.08710201)(298.20457153,408.92710297)
\curveto(298.21456372,408.8971022)(298.21956372,408.86210224)(298.21957153,408.82210297)
\curveto(298.21956372,408.79210231)(298.22456371,408.76210234)(298.23457153,408.73210297)
\moveto(296.02957153,411.04210297)
\curveto(296.0395659,411.09210001)(296.04456589,411.14709995)(296.04457153,411.20710297)
\curveto(296.04456589,411.27709982)(296.0395659,411.33709976)(296.02957153,411.38710297)
\curveto(296.00956593,411.44709965)(295.99956594,411.5020996)(295.99957153,411.55210297)
\curveto(295.99956594,411.6020995)(295.97956596,411.64209946)(295.93957153,411.67210297)
\curveto(295.88956605,411.71209939)(295.81456612,411.73209937)(295.71457153,411.73210297)
\curveto(295.67456626,411.72209938)(295.6395663,411.71209939)(295.60957153,411.70210297)
\curveto(295.57956636,411.7020994)(295.54456639,411.6970994)(295.50457153,411.68710297)
\curveto(295.4345665,411.66709943)(295.35956658,411.65209945)(295.27957153,411.64210297)
\curveto(295.19956674,411.63209947)(295.11956682,411.61709948)(295.03957153,411.59710297)
\curveto(295.00956693,411.58709951)(294.96456697,411.58209952)(294.90457153,411.58210297)
\curveto(294.77456716,411.55209955)(294.64456729,411.53209957)(294.51457153,411.52210297)
\curveto(294.38456755,411.51209959)(294.25956768,411.48709961)(294.13957153,411.44710297)
\curveto(294.05956788,411.42709967)(293.98456795,411.40709969)(293.91457153,411.38710297)
\curveto(293.84456809,411.37709972)(293.77456816,411.35709974)(293.70457153,411.32710297)
\curveto(293.49456844,411.23709986)(293.31456862,411.1021)(293.16457153,410.92210297)
\curveto(293.02456891,410.74210036)(292.97456896,410.49210061)(293.01457153,410.17210297)
\curveto(293.0345689,410.0021011)(293.08956885,409.86210124)(293.17957153,409.75210297)
\curveto(293.24956869,409.64210146)(293.35456858,409.55210155)(293.49457153,409.48210297)
\curveto(293.6345683,409.42210168)(293.78456815,409.37710172)(293.94457153,409.34710297)
\curveto(294.11456782,409.31710178)(294.28956765,409.30710179)(294.46957153,409.31710297)
\curveto(294.65956728,409.33710176)(294.8345671,409.37210173)(294.99457153,409.42210297)
\curveto(295.25456668,409.5021016)(295.45956648,409.62710147)(295.60957153,409.79710297)
\curveto(295.75956618,409.97710112)(295.87456606,410.1971009)(295.95457153,410.45710297)
\curveto(295.97456596,410.52710057)(295.98456595,410.5971005)(295.98457153,410.66710297)
\curveto(295.99456594,410.74710035)(296.00956593,410.82710027)(296.02957153,410.90710297)
\lineto(296.02957153,411.04210297)
}
}
{
\newrgbcolor{curcolor}{0 0 0}
\pscustom[linestyle=none,fillstyle=solid,fillcolor=curcolor]
{
\newpath
\moveto(307.38785278,408.98710297)
\lineto(307.38785278,408.56710297)
\curveto(307.38784441,408.43710266)(307.35784444,408.33210277)(307.29785278,408.25210297)
\curveto(307.24784455,408.2021029)(307.18284462,408.16710293)(307.10285278,408.14710297)
\curveto(307.02284478,408.13710296)(306.93284487,408.13210297)(306.83285278,408.13210297)
\lineto(306.00785278,408.13210297)
\lineto(305.72285278,408.13210297)
\curveto(305.64284616,408.14210296)(305.57784622,408.16710293)(305.52785278,408.20710297)
\curveto(305.45784634,408.25710284)(305.41784638,408.32210278)(305.40785278,408.40210297)
\curveto(305.3978464,408.48210262)(305.37784642,408.56210254)(305.34785278,408.64210297)
\curveto(305.32784647,408.66210244)(305.30784649,408.67710242)(305.28785278,408.68710297)
\curveto(305.27784652,408.70710239)(305.26284654,408.72710237)(305.24285278,408.74710297)
\curveto(305.13284667,408.74710235)(305.05284675,408.72210238)(305.00285278,408.67210297)
\lineto(304.85285278,408.52210297)
\curveto(304.78284702,408.47210263)(304.71784708,408.42710267)(304.65785278,408.38710297)
\curveto(304.5978472,408.35710274)(304.53284727,408.31710278)(304.46285278,408.26710297)
\curveto(304.42284738,408.24710285)(304.37784742,408.22710287)(304.32785278,408.20710297)
\curveto(304.28784751,408.18710291)(304.24284756,408.16710293)(304.19285278,408.14710297)
\curveto(304.05284775,408.097103)(303.9028479,408.05210305)(303.74285278,408.01210297)
\curveto(303.69284811,407.99210311)(303.64784815,407.98210312)(303.60785278,407.98210297)
\curveto(303.56784823,407.98210312)(303.52784827,407.97710312)(303.48785278,407.96710297)
\lineto(303.35285278,407.96710297)
\curveto(303.32284848,407.95710314)(303.28284852,407.95210315)(303.23285278,407.95210297)
\lineto(303.09785278,407.95210297)
\curveto(303.03784876,407.93210317)(302.94784885,407.92710317)(302.82785278,407.93710297)
\curveto(302.70784909,407.93710316)(302.62284918,407.94710315)(302.57285278,407.96710297)
\curveto(302.5028493,407.98710311)(302.43784936,407.9971031)(302.37785278,407.99710297)
\curveto(302.32784947,407.98710311)(302.27284953,407.99210311)(302.21285278,408.01210297)
\lineto(301.85285278,408.13210297)
\curveto(301.74285006,408.16210294)(301.63285017,408.2021029)(301.52285278,408.25210297)
\curveto(301.17285063,408.4021027)(300.85785094,408.63210247)(300.57785278,408.94210297)
\curveto(300.30785149,409.26210184)(300.09285171,409.5971015)(299.93285278,409.94710297)
\curveto(299.88285192,410.05710104)(299.84285196,410.16210094)(299.81285278,410.26210297)
\curveto(299.78285202,410.37210073)(299.74785205,410.48210062)(299.70785278,410.59210297)
\curveto(299.6978521,410.63210047)(299.69285211,410.66710043)(299.69285278,410.69710297)
\curveto(299.69285211,410.73710036)(299.68285212,410.78210032)(299.66285278,410.83210297)
\curveto(299.64285216,410.91210019)(299.62285218,410.9971001)(299.60285278,411.08710297)
\curveto(299.59285221,411.18709991)(299.57785222,411.28709981)(299.55785278,411.38710297)
\curveto(299.54785225,411.41709968)(299.54285226,411.45209965)(299.54285278,411.49210297)
\curveto(299.55285225,411.53209957)(299.55285225,411.56709953)(299.54285278,411.59710297)
\lineto(299.54285278,411.73210297)
\curveto(299.54285226,411.78209932)(299.53785226,411.83209927)(299.52785278,411.88210297)
\curveto(299.51785228,411.93209917)(299.51285229,411.98709911)(299.51285278,412.04710297)
\curveto(299.51285229,412.11709898)(299.51785228,412.17209893)(299.52785278,412.21210297)
\curveto(299.53785226,412.26209884)(299.54285226,412.30709879)(299.54285278,412.34710297)
\lineto(299.54285278,412.49710297)
\curveto(299.55285225,412.54709855)(299.55285225,412.59209851)(299.54285278,412.63210297)
\curveto(299.54285226,412.68209842)(299.55285225,412.73209837)(299.57285278,412.78210297)
\curveto(299.59285221,412.89209821)(299.60785219,412.9970981)(299.61785278,413.09710297)
\curveto(299.63785216,413.1970979)(299.66285214,413.2970978)(299.69285278,413.39710297)
\curveto(299.73285207,413.51709758)(299.76785203,413.63209747)(299.79785278,413.74210297)
\curveto(299.82785197,413.85209725)(299.86785193,413.96209714)(299.91785278,414.07210297)
\curveto(300.05785174,414.37209673)(300.23285157,414.65709644)(300.44285278,414.92710297)
\curveto(300.46285134,414.95709614)(300.48785131,414.98209612)(300.51785278,415.00210297)
\curveto(300.55785124,415.03209607)(300.58785121,415.06209604)(300.60785278,415.09210297)
\curveto(300.64785115,415.14209596)(300.68785111,415.18709591)(300.72785278,415.22710297)
\curveto(300.76785103,415.26709583)(300.81285099,415.30709579)(300.86285278,415.34710297)
\curveto(300.9028509,415.36709573)(300.93785086,415.39209571)(300.96785278,415.42210297)
\curveto(300.9978508,415.46209564)(301.03285077,415.49209561)(301.07285278,415.51210297)
\curveto(301.32285048,415.68209542)(301.61285019,415.82209528)(301.94285278,415.93210297)
\curveto(302.01284979,415.95209515)(302.08284972,415.96709513)(302.15285278,415.97710297)
\curveto(302.23284957,415.98709511)(302.31284949,416.0020951)(302.39285278,416.02210297)
\curveto(302.46284934,416.04209506)(302.55284925,416.05209505)(302.66285278,416.05210297)
\curveto(302.77284903,416.06209504)(302.88284892,416.06709503)(302.99285278,416.06710297)
\curveto(303.1028487,416.06709503)(303.20784859,416.06209504)(303.30785278,416.05210297)
\curveto(303.41784838,416.04209506)(303.50784829,416.02709507)(303.57785278,416.00710297)
\curveto(303.72784807,415.95709514)(303.87284793,415.91209519)(304.01285278,415.87210297)
\curveto(304.15284765,415.83209527)(304.28284752,415.77709532)(304.40285278,415.70710297)
\curveto(304.47284733,415.65709544)(304.53784726,415.60709549)(304.59785278,415.55710297)
\curveto(304.65784714,415.51709558)(304.72284708,415.47209563)(304.79285278,415.42210297)
\curveto(304.83284697,415.39209571)(304.88784691,415.35209575)(304.95785278,415.30210297)
\curveto(305.03784676,415.25209585)(305.11284669,415.25209585)(305.18285278,415.30210297)
\curveto(305.22284658,415.32209578)(305.24284656,415.35709574)(305.24285278,415.40710297)
\curveto(305.24284656,415.45709564)(305.25284655,415.50709559)(305.27285278,415.55710297)
\lineto(305.27285278,415.70710297)
\curveto(305.28284652,415.73709536)(305.28784651,415.77209533)(305.28785278,415.81210297)
\lineto(305.28785278,415.93210297)
\lineto(305.28785278,417.97210297)
\curveto(305.28784651,418.08209302)(305.28284652,418.2020929)(305.27285278,418.33210297)
\curveto(305.27284653,418.47209263)(305.2978465,418.57709252)(305.34785278,418.64710297)
\curveto(305.38784641,418.72709237)(305.46284634,418.77709232)(305.57285278,418.79710297)
\curveto(305.59284621,418.80709229)(305.61284619,418.80709229)(305.63285278,418.79710297)
\curveto(305.65284615,418.7970923)(305.67284613,418.8020923)(305.69285278,418.81210297)
\lineto(306.75785278,418.81210297)
\curveto(306.87784492,418.81209229)(306.98784481,418.80709229)(307.08785278,418.79710297)
\curveto(307.18784461,418.78709231)(307.26284454,418.74709235)(307.31285278,418.67710297)
\curveto(307.36284444,418.5970925)(307.38784441,418.49209261)(307.38785278,418.36210297)
\lineto(307.38785278,418.00210297)
\lineto(307.38785278,408.98710297)
\moveto(305.34785278,411.92710297)
\curveto(305.35784644,411.96709913)(305.35784644,412.00709909)(305.34785278,412.04710297)
\lineto(305.34785278,412.18210297)
\curveto(305.34784645,412.28209882)(305.34284646,412.38209872)(305.33285278,412.48210297)
\curveto(305.32284648,412.58209852)(305.30784649,412.67209843)(305.28785278,412.75210297)
\curveto(305.26784653,412.86209824)(305.24784655,412.96209814)(305.22785278,413.05210297)
\curveto(305.21784658,413.14209796)(305.19284661,413.22709787)(305.15285278,413.30710297)
\curveto(305.01284679,413.66709743)(304.80784699,413.95209715)(304.53785278,414.16210297)
\curveto(304.27784752,414.37209673)(303.8978479,414.47709662)(303.39785278,414.47710297)
\curveto(303.33784846,414.47709662)(303.25784854,414.46709663)(303.15785278,414.44710297)
\curveto(303.07784872,414.42709667)(303.0028488,414.40709669)(302.93285278,414.38710297)
\curveto(302.87284893,414.37709672)(302.81284899,414.35709674)(302.75285278,414.32710297)
\curveto(302.48284932,414.21709688)(302.27284953,414.04709705)(302.12285278,413.81710297)
\curveto(301.97284983,413.58709751)(301.85284995,413.32709777)(301.76285278,413.03710297)
\curveto(301.73285007,412.93709816)(301.71285009,412.83709826)(301.70285278,412.73710297)
\curveto(301.69285011,412.63709846)(301.67285013,412.53209857)(301.64285278,412.42210297)
\lineto(301.64285278,412.21210297)
\curveto(301.62285018,412.12209898)(301.61785018,411.9970991)(301.62785278,411.83710297)
\curveto(301.63785016,411.68709941)(301.65285015,411.57709952)(301.67285278,411.50710297)
\lineto(301.67285278,411.41710297)
\curveto(301.68285012,411.3970997)(301.68785011,411.37709972)(301.68785278,411.35710297)
\curveto(301.70785009,411.27709982)(301.72285008,411.2020999)(301.73285278,411.13210297)
\curveto(301.75285005,411.06210004)(301.77285003,410.98710011)(301.79285278,410.90710297)
\curveto(301.96284984,410.38710071)(302.25284955,410.0021011)(302.66285278,409.75210297)
\curveto(302.79284901,409.66210144)(302.97284883,409.59210151)(303.20285278,409.54210297)
\curveto(303.24284856,409.53210157)(303.3028485,409.52710157)(303.38285278,409.52710297)
\curveto(303.41284839,409.51710158)(303.45784834,409.50710159)(303.51785278,409.49710297)
\curveto(303.58784821,409.4971016)(303.64284816,409.5021016)(303.68285278,409.51210297)
\curveto(303.76284804,409.53210157)(303.84284796,409.54710155)(303.92285278,409.55710297)
\curveto(304.0028478,409.56710153)(304.08284772,409.58710151)(304.16285278,409.61710297)
\curveto(304.41284739,409.72710137)(304.61284719,409.86710123)(304.76285278,410.03710297)
\curveto(304.91284689,410.20710089)(305.04284676,410.42210068)(305.15285278,410.68210297)
\curveto(305.19284661,410.77210033)(305.22284658,410.86210024)(305.24285278,410.95210297)
\curveto(305.26284654,411.05210005)(305.28284652,411.15709994)(305.30285278,411.26710297)
\curveto(305.31284649,411.31709978)(305.31284649,411.36209974)(305.30285278,411.40210297)
\curveto(305.3028465,411.45209965)(305.31284649,411.5020996)(305.33285278,411.55210297)
\curveto(305.34284646,411.58209952)(305.34784645,411.61709948)(305.34785278,411.65710297)
\lineto(305.34785278,411.79210297)
\lineto(305.34785278,411.92710297)
}
}
{
\newrgbcolor{curcolor}{0 0 0}
\pscustom[linestyle=none,fillstyle=solid,fillcolor=curcolor]
{
\newpath
\moveto(316.73777466,412.31710297)
\curveto(316.75776609,412.25709884)(316.76776608,412.17209893)(316.76777466,412.06210297)
\curveto(316.76776608,411.95209915)(316.75776609,411.86709923)(316.73777466,411.80710297)
\lineto(316.73777466,411.65710297)
\curveto(316.71776613,411.57709952)(316.70776614,411.4970996)(316.70777466,411.41710297)
\curveto(316.71776613,411.33709976)(316.71276613,411.25709984)(316.69277466,411.17710297)
\curveto(316.67276617,411.10709999)(316.65776619,411.04210006)(316.64777466,410.98210297)
\curveto(316.63776621,410.92210018)(316.62776622,410.85710024)(316.61777466,410.78710297)
\curveto(316.57776627,410.67710042)(316.5427663,410.56210054)(316.51277466,410.44210297)
\curveto(316.48276636,410.33210077)(316.4427664,410.22710087)(316.39277466,410.12710297)
\curveto(316.18276666,409.64710145)(315.90776694,409.25710184)(315.56777466,408.95710297)
\curveto(315.22776762,408.65710244)(314.81776803,408.40710269)(314.33777466,408.20710297)
\curveto(314.21776863,408.15710294)(314.09276875,408.12210298)(313.96277466,408.10210297)
\curveto(313.842769,408.07210303)(313.71776913,408.04210306)(313.58777466,408.01210297)
\curveto(313.53776931,407.99210311)(313.48276936,407.98210312)(313.42277466,407.98210297)
\curveto(313.36276948,407.98210312)(313.30776954,407.97710312)(313.25777466,407.96710297)
\lineto(313.15277466,407.96710297)
\curveto(313.12276972,407.95710314)(313.09276975,407.95210315)(313.06277466,407.95210297)
\curveto(313.01276983,407.94210316)(312.93276991,407.93710316)(312.82277466,407.93710297)
\curveto(312.71277013,407.92710317)(312.62777022,407.93210317)(312.56777466,407.95210297)
\lineto(312.41777466,407.95210297)
\curveto(312.36777048,407.96210314)(312.31277053,407.96710313)(312.25277466,407.96710297)
\curveto(312.20277064,407.95710314)(312.15277069,407.96210314)(312.10277466,407.98210297)
\curveto(312.06277078,407.99210311)(312.02277082,407.9971031)(311.98277466,407.99710297)
\curveto(311.95277089,407.9971031)(311.91277093,408.0021031)(311.86277466,408.01210297)
\curveto(311.76277108,408.04210306)(311.66277118,408.06710303)(311.56277466,408.08710297)
\curveto(311.46277138,408.10710299)(311.36777148,408.13710296)(311.27777466,408.17710297)
\curveto(311.15777169,408.21710288)(311.0427718,408.25710284)(310.93277466,408.29710297)
\curveto(310.83277201,408.33710276)(310.72777212,408.38710271)(310.61777466,408.44710297)
\curveto(310.26777258,408.65710244)(309.96777288,408.9021022)(309.71777466,409.18210297)
\curveto(309.46777338,409.46210164)(309.25777359,409.7971013)(309.08777466,410.18710297)
\curveto(309.03777381,410.27710082)(308.99777385,410.37210073)(308.96777466,410.47210297)
\curveto(308.9477739,410.57210053)(308.92277392,410.67710042)(308.89277466,410.78710297)
\curveto(308.87277397,410.83710026)(308.86277398,410.88210022)(308.86277466,410.92210297)
\curveto(308.86277398,410.96210014)(308.85277399,411.00710009)(308.83277466,411.05710297)
\curveto(308.81277403,411.13709996)(308.80277404,411.21709988)(308.80277466,411.29710297)
\curveto(308.80277404,411.38709971)(308.79277405,411.47209963)(308.77277466,411.55210297)
\curveto(308.76277408,411.6020995)(308.75777409,411.64709945)(308.75777466,411.68710297)
\lineto(308.75777466,411.82210297)
\curveto(308.73777411,411.88209922)(308.72777412,411.96709913)(308.72777466,412.07710297)
\curveto(308.73777411,412.18709891)(308.75277409,412.27209883)(308.77277466,412.33210297)
\lineto(308.77277466,412.43710297)
\curveto(308.78277406,412.48709861)(308.78277406,412.53709856)(308.77277466,412.58710297)
\curveto(308.77277407,412.64709845)(308.78277406,412.7020984)(308.80277466,412.75210297)
\curveto(308.81277403,412.8020983)(308.81777403,412.84709825)(308.81777466,412.88710297)
\curveto(308.81777403,412.93709816)(308.82777402,412.98709811)(308.84777466,413.03710297)
\curveto(308.88777396,413.16709793)(308.92277392,413.29209781)(308.95277466,413.41210297)
\curveto(308.98277386,413.54209756)(309.02277382,413.66709743)(309.07277466,413.78710297)
\curveto(309.25277359,414.1970969)(309.46777338,414.53709656)(309.71777466,414.80710297)
\curveto(309.96777288,415.08709601)(310.27277257,415.34209576)(310.63277466,415.57210297)
\curveto(310.73277211,415.62209548)(310.83777201,415.66709543)(310.94777466,415.70710297)
\curveto(311.05777179,415.74709535)(311.16777168,415.79209531)(311.27777466,415.84210297)
\curveto(311.40777144,415.89209521)(311.5427713,415.92709517)(311.68277466,415.94710297)
\curveto(311.82277102,415.96709513)(311.96777088,415.9970951)(312.11777466,416.03710297)
\curveto(312.19777065,416.04709505)(312.27277057,416.05209505)(312.34277466,416.05210297)
\curveto(312.41277043,416.05209505)(312.48277036,416.05709504)(312.55277466,416.06710297)
\curveto(313.13276971,416.07709502)(313.63276921,416.01709508)(314.05277466,415.88710297)
\curveto(314.48276836,415.75709534)(314.86276798,415.57709552)(315.19277466,415.34710297)
\curveto(315.30276754,415.26709583)(315.41276743,415.17709592)(315.52277466,415.07710297)
\curveto(315.6427672,414.98709611)(315.7427671,414.88709621)(315.82277466,414.77710297)
\curveto(315.90276694,414.67709642)(315.97276687,414.57709652)(316.03277466,414.47710297)
\curveto(316.10276674,414.37709672)(316.17276667,414.27209683)(316.24277466,414.16210297)
\curveto(316.31276653,414.05209705)(316.36776648,413.93209717)(316.40777466,413.80210297)
\curveto(316.4477664,413.68209742)(316.49276635,413.55209755)(316.54277466,413.41210297)
\curveto(316.57276627,413.33209777)(316.59776625,413.24709785)(316.61777466,413.15710297)
\lineto(316.67777466,412.88710297)
\curveto(316.68776616,412.84709825)(316.69276615,412.80709829)(316.69277466,412.76710297)
\curveto(316.69276615,412.72709837)(316.69776615,412.68709841)(316.70777466,412.64710297)
\curveto(316.72776612,412.5970985)(316.73276611,412.54209856)(316.72277466,412.48210297)
\curveto(316.71276613,412.42209868)(316.71776613,412.36709873)(316.73777466,412.31710297)
\moveto(314.63777466,411.77710297)
\curveto(314.6477682,411.82709927)(314.65276819,411.8970992)(314.65277466,411.98710297)
\curveto(314.65276819,412.08709901)(314.6477682,412.16209894)(314.63777466,412.21210297)
\lineto(314.63777466,412.33210297)
\curveto(314.61776823,412.38209872)(314.60776824,412.43709866)(314.60777466,412.49710297)
\curveto(314.60776824,412.55709854)(314.60276824,412.61209849)(314.59277466,412.66210297)
\curveto(314.59276825,412.7020984)(314.58776826,412.73209837)(314.57777466,412.75210297)
\lineto(314.51777466,412.99210297)
\curveto(314.50776834,413.08209802)(314.48776836,413.16709793)(314.45777466,413.24710297)
\curveto(314.3477685,413.50709759)(314.21776863,413.72709737)(314.06777466,413.90710297)
\curveto(313.91776893,414.097097)(313.71776913,414.24709685)(313.46777466,414.35710297)
\curveto(313.40776944,414.37709672)(313.3477695,414.39209671)(313.28777466,414.40210297)
\curveto(313.22776962,414.42209668)(313.16276968,414.44209666)(313.09277466,414.46210297)
\curveto(313.01276983,414.48209662)(312.92776992,414.48709661)(312.83777466,414.47710297)
\lineto(312.56777466,414.47710297)
\curveto(312.53777031,414.45709664)(312.50277034,414.44709665)(312.46277466,414.44710297)
\curveto(312.42277042,414.45709664)(312.38777046,414.45709664)(312.35777466,414.44710297)
\lineto(312.14777466,414.38710297)
\curveto(312.08777076,414.37709672)(312.03277081,414.35709674)(311.98277466,414.32710297)
\curveto(311.73277111,414.21709688)(311.52777132,414.05709704)(311.36777466,413.84710297)
\curveto(311.21777163,413.64709745)(311.09777175,413.41209769)(311.00777466,413.14210297)
\curveto(310.97777187,413.04209806)(310.95277189,412.93709816)(310.93277466,412.82710297)
\curveto(310.92277192,412.71709838)(310.90777194,412.60709849)(310.88777466,412.49710297)
\curveto(310.87777197,412.44709865)(310.87277197,412.3970987)(310.87277466,412.34710297)
\lineto(310.87277466,412.19710297)
\curveto(310.85277199,412.12709897)(310.842772,412.02209908)(310.84277466,411.88210297)
\curveto(310.85277199,411.74209936)(310.86777198,411.63709946)(310.88777466,411.56710297)
\lineto(310.88777466,411.43210297)
\curveto(310.90777194,411.35209975)(310.92277192,411.27209983)(310.93277466,411.19210297)
\curveto(310.9427719,411.12209998)(310.95777189,411.04710005)(310.97777466,410.96710297)
\curveto(311.07777177,410.66710043)(311.18277166,410.42210068)(311.29277466,410.23210297)
\curveto(311.41277143,410.05210105)(311.59777125,409.88710121)(311.84777466,409.73710297)
\curveto(311.91777093,409.68710141)(311.99277085,409.64710145)(312.07277466,409.61710297)
\curveto(312.16277068,409.58710151)(312.25277059,409.56210154)(312.34277466,409.54210297)
\curveto(312.38277046,409.53210157)(312.41777043,409.52710157)(312.44777466,409.52710297)
\curveto(312.47777037,409.53710156)(312.51277033,409.53710156)(312.55277466,409.52710297)
\lineto(312.67277466,409.49710297)
\curveto(312.72277012,409.4971016)(312.76777008,409.5021016)(312.80777466,409.51210297)
\lineto(312.92777466,409.51210297)
\curveto(313.00776984,409.53210157)(313.08776976,409.54710155)(313.16777466,409.55710297)
\curveto(313.2477696,409.56710153)(313.32276952,409.58710151)(313.39277466,409.61710297)
\curveto(313.65276919,409.71710138)(313.86276898,409.85210125)(314.02277466,410.02210297)
\curveto(314.18276866,410.19210091)(314.31776853,410.4021007)(314.42777466,410.65210297)
\curveto(314.46776838,410.75210035)(314.49776835,410.85210025)(314.51777466,410.95210297)
\curveto(314.53776831,411.05210005)(314.56276828,411.15709994)(314.59277466,411.26710297)
\curveto(314.60276824,411.30709979)(314.60776824,411.34209976)(314.60777466,411.37210297)
\curveto(314.60776824,411.41209969)(314.61276823,411.45209965)(314.62277466,411.49210297)
\lineto(314.62277466,411.62710297)
\curveto(314.62276822,411.67709942)(314.62776822,411.72709937)(314.63777466,411.77710297)
}
}
{
\newrgbcolor{curcolor}{0 0 0}
\pscustom[linestyle=none,fillstyle=solid,fillcolor=curcolor]
{
\newpath
\moveto(321.10769653,416.08210297)
\curveto(321.85769203,416.102095)(322.50769138,416.01709508)(323.05769653,415.82710297)
\curveto(323.61769027,415.64709545)(324.04268985,415.33209577)(324.33269653,414.88210297)
\curveto(324.40268949,414.77209633)(324.46268943,414.65709644)(324.51269653,414.53710297)
\curveto(324.57268932,414.42709667)(324.62268927,414.3020968)(324.66269653,414.16210297)
\curveto(324.68268921,414.102097)(324.6926892,414.03709706)(324.69269653,413.96710297)
\curveto(324.6926892,413.8970972)(324.68268921,413.83709726)(324.66269653,413.78710297)
\curveto(324.62268927,413.72709737)(324.56768932,413.68709741)(324.49769653,413.66710297)
\curveto(324.44768944,413.64709745)(324.3876895,413.63709746)(324.31769653,413.63710297)
\lineto(324.10769653,413.63710297)
\lineto(323.44769653,413.63710297)
\curveto(323.37769051,413.63709746)(323.30769058,413.63209747)(323.23769653,413.62210297)
\curveto(323.16769072,413.62209748)(323.10269079,413.63209747)(323.04269653,413.65210297)
\curveto(322.94269095,413.67209743)(322.86769102,413.71209739)(322.81769653,413.77210297)
\curveto(322.76769112,413.83209727)(322.72269117,413.89209721)(322.68269653,413.95210297)
\lineto(322.56269653,414.16210297)
\curveto(322.53269136,414.24209686)(322.48269141,414.30709679)(322.41269653,414.35710297)
\curveto(322.31269158,414.43709666)(322.21269168,414.4970966)(322.11269653,414.53710297)
\curveto(322.02269187,414.57709652)(321.90769198,414.61209649)(321.76769653,414.64210297)
\curveto(321.69769219,414.66209644)(321.5926923,414.67709642)(321.45269653,414.68710297)
\curveto(321.32269257,414.6970964)(321.22269267,414.69209641)(321.15269653,414.67210297)
\lineto(321.04769653,414.67210297)
\lineto(320.89769653,414.64210297)
\curveto(320.85769303,414.64209646)(320.81269308,414.63709646)(320.76269653,414.62710297)
\curveto(320.5926933,414.57709652)(320.45269344,414.50709659)(320.34269653,414.41710297)
\curveto(320.24269365,414.33709676)(320.17269372,414.21209689)(320.13269653,414.04210297)
\curveto(320.11269378,413.97209713)(320.11269378,413.90709719)(320.13269653,413.84710297)
\curveto(320.15269374,413.78709731)(320.17269372,413.73709736)(320.19269653,413.69710297)
\curveto(320.26269363,413.57709752)(320.34269355,413.48209762)(320.43269653,413.41210297)
\curveto(320.53269336,413.34209776)(320.64769324,413.28209782)(320.77769653,413.23210297)
\curveto(320.96769292,413.15209795)(321.17269272,413.08209802)(321.39269653,413.02210297)
\lineto(322.08269653,412.87210297)
\curveto(322.32269157,412.83209827)(322.55269134,412.78209832)(322.77269653,412.72210297)
\curveto(323.00269089,412.67209843)(323.21769067,412.60709849)(323.41769653,412.52710297)
\curveto(323.50769038,412.48709861)(323.5926903,412.45209865)(323.67269653,412.42210297)
\curveto(323.76269013,412.4020987)(323.84769004,412.36709873)(323.92769653,412.31710297)
\curveto(324.11768977,412.1970989)(324.2876896,412.06709903)(324.43769653,411.92710297)
\curveto(324.59768929,411.78709931)(324.72268917,411.61209949)(324.81269653,411.40210297)
\curveto(324.84268905,411.33209977)(324.86768902,411.26209984)(324.88769653,411.19210297)
\curveto(324.90768898,411.12209998)(324.92768896,411.04710005)(324.94769653,410.96710297)
\curveto(324.95768893,410.90710019)(324.96268893,410.81210029)(324.96269653,410.68210297)
\curveto(324.97268892,410.56210054)(324.97268892,410.46710063)(324.96269653,410.39710297)
\lineto(324.96269653,410.32210297)
\curveto(324.94268895,410.26210084)(324.92768896,410.2021009)(324.91769653,410.14210297)
\curveto(324.91768897,410.09210101)(324.91268898,410.04210106)(324.90269653,409.99210297)
\curveto(324.83268906,409.69210141)(324.72268917,409.42710167)(324.57269653,409.19710297)
\curveto(324.41268948,408.95710214)(324.21768967,408.76210234)(323.98769653,408.61210297)
\curveto(323.75769013,408.46210264)(323.49769039,408.33210277)(323.20769653,408.22210297)
\curveto(323.09769079,408.17210293)(322.97769091,408.13710296)(322.84769653,408.11710297)
\curveto(322.72769116,408.097103)(322.60769128,408.07210303)(322.48769653,408.04210297)
\curveto(322.39769149,408.02210308)(322.30269159,408.01210309)(322.20269653,408.01210297)
\curveto(322.11269178,408.0021031)(322.02269187,407.98710311)(321.93269653,407.96710297)
\lineto(321.66269653,407.96710297)
\curveto(321.60269229,407.94710315)(321.49769239,407.93710316)(321.34769653,407.93710297)
\curveto(321.20769268,407.93710316)(321.10769278,407.94710315)(321.04769653,407.96710297)
\curveto(321.01769287,407.96710313)(320.98269291,407.97210313)(320.94269653,407.98210297)
\lineto(320.83769653,407.98210297)
\curveto(320.71769317,408.0021031)(320.59769329,408.01710308)(320.47769653,408.02710297)
\curveto(320.35769353,408.03710306)(320.24269365,408.05710304)(320.13269653,408.08710297)
\curveto(319.74269415,408.1971029)(319.39769449,408.32210278)(319.09769653,408.46210297)
\curveto(318.79769509,408.61210249)(318.54269535,408.83210227)(318.33269653,409.12210297)
\curveto(318.1926957,409.31210179)(318.07269582,409.53210157)(317.97269653,409.78210297)
\curveto(317.95269594,409.84210126)(317.93269596,409.92210118)(317.91269653,410.02210297)
\curveto(317.892696,410.07210103)(317.87769601,410.14210096)(317.86769653,410.23210297)
\curveto(317.85769603,410.32210078)(317.86269603,410.3971007)(317.88269653,410.45710297)
\curveto(317.91269598,410.52710057)(317.96269593,410.57710052)(318.03269653,410.60710297)
\curveto(318.08269581,410.62710047)(318.14269575,410.63710046)(318.21269653,410.63710297)
\lineto(318.43769653,410.63710297)
\lineto(319.14269653,410.63710297)
\lineto(319.38269653,410.63710297)
\curveto(319.46269443,410.63710046)(319.53269436,410.62710047)(319.59269653,410.60710297)
\curveto(319.70269419,410.56710053)(319.77269412,410.5021006)(319.80269653,410.41210297)
\curveto(319.84269405,410.32210078)(319.887694,410.22710087)(319.93769653,410.12710297)
\curveto(319.95769393,410.07710102)(319.9926939,410.01210109)(320.04269653,409.93210297)
\curveto(320.10269379,409.85210125)(320.15269374,409.8021013)(320.19269653,409.78210297)
\curveto(320.31269358,409.68210142)(320.42769346,409.6021015)(320.53769653,409.54210297)
\curveto(320.64769324,409.49210161)(320.7876931,409.44210166)(320.95769653,409.39210297)
\curveto(321.00769288,409.37210173)(321.05769283,409.36210174)(321.10769653,409.36210297)
\curveto(321.15769273,409.37210173)(321.20769268,409.37210173)(321.25769653,409.36210297)
\curveto(321.33769255,409.34210176)(321.42269247,409.33210177)(321.51269653,409.33210297)
\curveto(321.61269228,409.34210176)(321.69769219,409.35710174)(321.76769653,409.37710297)
\curveto(321.81769207,409.38710171)(321.86269203,409.39210171)(321.90269653,409.39210297)
\curveto(321.95269194,409.39210171)(322.00269189,409.4021017)(322.05269653,409.42210297)
\curveto(322.1926917,409.47210163)(322.31769157,409.53210157)(322.42769653,409.60210297)
\curveto(322.54769134,409.67210143)(322.64269125,409.76210134)(322.71269653,409.87210297)
\curveto(322.76269113,409.95210115)(322.80269109,410.07710102)(322.83269653,410.24710297)
\curveto(322.85269104,410.31710078)(322.85269104,410.38210072)(322.83269653,410.44210297)
\curveto(322.81269108,410.5021006)(322.7926911,410.55210055)(322.77269653,410.59210297)
\curveto(322.70269119,410.73210037)(322.61269128,410.83710026)(322.50269653,410.90710297)
\curveto(322.40269149,410.97710012)(322.28269161,411.04210006)(322.14269653,411.10210297)
\curveto(321.95269194,411.18209992)(321.75269214,411.24709985)(321.54269653,411.29710297)
\curveto(321.33269256,411.34709975)(321.12269277,411.4020997)(320.91269653,411.46210297)
\curveto(320.83269306,411.48209962)(320.74769314,411.4970996)(320.65769653,411.50710297)
\curveto(320.57769331,411.51709958)(320.49769339,411.53209957)(320.41769653,411.55210297)
\curveto(320.09769379,411.64209946)(319.7926941,411.72709937)(319.50269653,411.80710297)
\curveto(319.21269468,411.8970992)(318.94769494,412.02709907)(318.70769653,412.19710297)
\curveto(318.42769546,412.3970987)(318.22269567,412.66709843)(318.09269653,413.00710297)
\curveto(318.07269582,413.07709802)(318.05269584,413.17209793)(318.03269653,413.29210297)
\curveto(318.01269588,413.36209774)(317.99769589,413.44709765)(317.98769653,413.54710297)
\curveto(317.97769591,413.64709745)(317.98269591,413.73709736)(318.00269653,413.81710297)
\curveto(318.02269587,413.86709723)(318.02769586,413.90709719)(318.01769653,413.93710297)
\curveto(318.00769588,413.97709712)(318.01269588,414.02209708)(318.03269653,414.07210297)
\curveto(318.05269584,414.18209692)(318.07269582,414.28209682)(318.09269653,414.37210297)
\curveto(318.12269577,414.47209663)(318.15769573,414.56709653)(318.19769653,414.65710297)
\curveto(318.32769556,414.94709615)(318.50769538,415.18209592)(318.73769653,415.36210297)
\curveto(318.96769492,415.54209556)(319.22769466,415.68709541)(319.51769653,415.79710297)
\curveto(319.62769426,415.84709525)(319.74269415,415.88209522)(319.86269653,415.90210297)
\curveto(319.98269391,415.93209517)(320.10769378,415.96209514)(320.23769653,415.99210297)
\curveto(320.29769359,416.01209509)(320.35769353,416.02209508)(320.41769653,416.02210297)
\lineto(320.59769653,416.05210297)
\curveto(320.67769321,416.06209504)(320.76269313,416.06709503)(320.85269653,416.06710297)
\curveto(320.94269295,416.06709503)(321.02769286,416.07209503)(321.10769653,416.08210297)
}
}
{
\newrgbcolor{curcolor}{0 0 0}
\pscustom[linestyle=none,fillstyle=solid,fillcolor=curcolor]
{
}
}
{
\newrgbcolor{curcolor}{0 0 0}
\pscustom[linestyle=none,fillstyle=solid,fillcolor=curcolor]
{
\newpath
\moveto(338.24449341,412.09210297)
\curveto(338.25448473,412.03209907)(338.25948472,411.94209916)(338.25949341,411.82210297)
\curveto(338.25948472,411.7020994)(338.24948473,411.61709948)(338.22949341,411.56710297)
\lineto(338.22949341,411.37210297)
\curveto(338.19948478,411.26209984)(338.1794848,411.15709994)(338.16949341,411.05710297)
\curveto(338.16948481,410.95710014)(338.15448483,410.85710024)(338.12449341,410.75710297)
\curveto(338.10448488,410.66710043)(338.0844849,410.57210053)(338.06449341,410.47210297)
\curveto(338.04448494,410.38210072)(338.01448497,410.29210081)(337.97449341,410.20210297)
\curveto(337.90448508,410.03210107)(337.83448515,409.87210123)(337.76449341,409.72210297)
\curveto(337.69448529,409.58210152)(337.61448537,409.44210166)(337.52449341,409.30210297)
\curveto(337.46448552,409.21210189)(337.39948558,409.12710197)(337.32949341,409.04710297)
\curveto(337.26948571,408.97710212)(337.19948578,408.9021022)(337.11949341,408.82210297)
\lineto(337.01449341,408.71710297)
\curveto(336.96448602,408.66710243)(336.90948607,408.62210248)(336.84949341,408.58210297)
\lineto(336.69949341,408.46210297)
\curveto(336.61948636,408.4021027)(336.52948645,408.34710275)(336.42949341,408.29710297)
\curveto(336.33948664,408.25710284)(336.24448674,408.21210289)(336.14449341,408.16210297)
\curveto(336.04448694,408.11210299)(335.93948704,408.07710302)(335.82949341,408.05710297)
\curveto(335.72948725,408.03710306)(335.62448736,408.01710308)(335.51449341,407.99710297)
\curveto(335.45448753,407.97710312)(335.38948759,407.96710313)(335.31949341,407.96710297)
\curveto(335.25948772,407.96710313)(335.19448779,407.95710314)(335.12449341,407.93710297)
\lineto(334.98949341,407.93710297)
\curveto(334.90948807,407.91710318)(334.83448815,407.91710318)(334.76449341,407.93710297)
\lineto(334.61449341,407.93710297)
\curveto(334.55448843,407.95710314)(334.48948849,407.96710313)(334.41949341,407.96710297)
\curveto(334.35948862,407.95710314)(334.29948868,407.96210314)(334.23949341,407.98210297)
\curveto(334.0794889,408.03210307)(333.92448906,408.07710302)(333.77449341,408.11710297)
\curveto(333.63448935,408.15710294)(333.50448948,408.21710288)(333.38449341,408.29710297)
\curveto(333.31448967,408.33710276)(333.24948973,408.37710272)(333.18949341,408.41710297)
\curveto(333.12948985,408.46710263)(333.06448992,408.51710258)(332.99449341,408.56710297)
\lineto(332.81449341,408.70210297)
\curveto(332.73449025,408.76210234)(332.66449032,408.76710233)(332.60449341,408.71710297)
\curveto(332.55449043,408.68710241)(332.52949045,408.64710245)(332.52949341,408.59710297)
\curveto(332.52949045,408.55710254)(332.51949046,408.50710259)(332.49949341,408.44710297)
\curveto(332.4794905,408.34710275)(332.46949051,408.23210287)(332.46949341,408.10210297)
\curveto(332.4794905,407.97210313)(332.4844905,407.85210325)(332.48449341,407.74210297)
\lineto(332.48449341,406.21210297)
\curveto(332.4844905,406.08210502)(332.4794905,405.95710514)(332.46949341,405.83710297)
\curveto(332.46949051,405.70710539)(332.44449054,405.6021055)(332.39449341,405.52210297)
\curveto(332.36449062,405.48210562)(332.30949067,405.45210565)(332.22949341,405.43210297)
\curveto(332.14949083,405.41210569)(332.05949092,405.4021057)(331.95949341,405.40210297)
\curveto(331.85949112,405.39210571)(331.75949122,405.39210571)(331.65949341,405.40210297)
\lineto(331.40449341,405.40210297)
\lineto(330.99949341,405.40210297)
\lineto(330.89449341,405.40210297)
\curveto(330.85449213,405.4021057)(330.81949216,405.40710569)(330.78949341,405.41710297)
\lineto(330.66949341,405.41710297)
\curveto(330.49949248,405.46710563)(330.40949257,405.56710553)(330.39949341,405.71710297)
\curveto(330.38949259,405.85710524)(330.3844926,406.02710507)(330.38449341,406.22710297)
\lineto(330.38449341,415.03210297)
\curveto(330.3844926,415.14209596)(330.3794926,415.25709584)(330.36949341,415.37710297)
\curveto(330.36949261,415.50709559)(330.39449259,415.60709549)(330.44449341,415.67710297)
\curveto(330.4844925,415.74709535)(330.53949244,415.79209531)(330.60949341,415.81210297)
\curveto(330.65949232,415.83209527)(330.71949226,415.84209526)(330.78949341,415.84210297)
\lineto(331.01449341,415.84210297)
\lineto(331.73449341,415.84210297)
\lineto(332.01949341,415.84210297)
\curveto(332.10949087,415.84209526)(332.1844908,415.81709528)(332.24449341,415.76710297)
\curveto(332.31449067,415.71709538)(332.34949063,415.65209545)(332.34949341,415.57210297)
\curveto(332.35949062,415.5020956)(332.3844906,415.42709567)(332.42449341,415.34710297)
\curveto(332.43449055,415.31709578)(332.44449054,415.29209581)(332.45449341,415.27210297)
\curveto(332.47449051,415.26209584)(332.49449049,415.24709585)(332.51449341,415.22710297)
\curveto(332.62449036,415.21709588)(332.71449027,415.24709585)(332.78449341,415.31710297)
\curveto(332.85449013,415.38709571)(332.92449006,415.44709565)(332.99449341,415.49710297)
\curveto(333.12448986,415.58709551)(333.25948972,415.66709543)(333.39949341,415.73710297)
\curveto(333.53948944,415.81709528)(333.69448929,415.88209522)(333.86449341,415.93210297)
\curveto(333.94448904,415.96209514)(334.02948895,415.98209512)(334.11949341,415.99210297)
\curveto(334.21948876,416.0020951)(334.31448867,416.01709508)(334.40449341,416.03710297)
\curveto(334.44448854,416.04709505)(334.4844885,416.04709505)(334.52449341,416.03710297)
\curveto(334.57448841,416.02709507)(334.61448837,416.03209507)(334.64449341,416.05210297)
\curveto(335.21448777,416.07209503)(335.69448729,415.99209511)(336.08449341,415.81210297)
\curveto(336.4844865,415.64209546)(336.82448616,415.41709568)(337.10449341,415.13710297)
\curveto(337.15448583,415.08709601)(337.19948578,415.03709606)(337.23949341,414.98710297)
\curveto(337.2794857,414.94709615)(337.31948566,414.9020962)(337.35949341,414.85210297)
\curveto(337.42948555,414.76209634)(337.48948549,414.67209643)(337.53949341,414.58210297)
\curveto(337.59948538,414.49209661)(337.65448533,414.4020967)(337.70449341,414.31210297)
\curveto(337.72448526,414.29209681)(337.73448525,414.26709683)(337.73449341,414.23710297)
\curveto(337.74448524,414.20709689)(337.75948522,414.17209693)(337.77949341,414.13210297)
\curveto(337.83948514,414.03209707)(337.89448509,413.91209719)(337.94449341,413.77210297)
\curveto(337.96448502,413.71209739)(337.984485,413.64709745)(338.00449341,413.57710297)
\curveto(338.02448496,413.51709758)(338.04448494,413.45209765)(338.06449341,413.38210297)
\curveto(338.10448488,413.26209784)(338.12948485,413.13709796)(338.13949341,413.00710297)
\curveto(338.15948482,412.87709822)(338.1844848,412.74209836)(338.21449341,412.60210297)
\lineto(338.21449341,412.43710297)
\lineto(338.24449341,412.25710297)
\lineto(338.24449341,412.09210297)
\moveto(336.12949341,411.74710297)
\curveto(336.13948684,411.7970993)(336.14448684,411.86209924)(336.14449341,411.94210297)
\curveto(336.14448684,412.03209907)(336.13948684,412.102099)(336.12949341,412.15210297)
\lineto(336.12949341,412.28710297)
\curveto(336.10948687,412.34709875)(336.09948688,412.41209869)(336.09949341,412.48210297)
\curveto(336.09948688,412.55209855)(336.08948689,412.62209848)(336.06949341,412.69210297)
\curveto(336.04948693,412.79209831)(336.02948695,412.88709821)(336.00949341,412.97710297)
\curveto(335.98948699,413.07709802)(335.95948702,413.16709793)(335.91949341,413.24710297)
\curveto(335.79948718,413.56709753)(335.64448734,413.82209728)(335.45449341,414.01210297)
\curveto(335.26448772,414.2020969)(334.99448799,414.34209676)(334.64449341,414.43210297)
\curveto(334.56448842,414.45209665)(334.47448851,414.46209664)(334.37449341,414.46210297)
\lineto(334.10449341,414.46210297)
\curveto(334.06448892,414.45209665)(334.02948895,414.44709665)(333.99949341,414.44710297)
\curveto(333.96948901,414.44709665)(333.93448905,414.44209666)(333.89449341,414.43210297)
\lineto(333.68449341,414.37210297)
\curveto(333.62448936,414.36209674)(333.56448942,414.34209676)(333.50449341,414.31210297)
\curveto(333.24448974,414.2020969)(333.03948994,414.03209707)(332.88949341,413.80210297)
\curveto(332.74949023,413.57209753)(332.63449035,413.31709778)(332.54449341,413.03710297)
\curveto(332.52449046,412.95709814)(332.50949047,412.87209823)(332.49949341,412.78210297)
\curveto(332.48949049,412.7020984)(332.47449051,412.62209848)(332.45449341,412.54210297)
\curveto(332.44449054,412.5020986)(332.43949054,412.43709866)(332.43949341,412.34710297)
\curveto(332.41949056,412.30709879)(332.41449057,412.25709884)(332.42449341,412.19710297)
\curveto(332.43449055,412.14709895)(332.43449055,412.097099)(332.42449341,412.04710297)
\curveto(332.40449058,411.98709911)(332.40449058,411.93209917)(332.42449341,411.88210297)
\lineto(332.42449341,411.70210297)
\lineto(332.42449341,411.56710297)
\curveto(332.42449056,411.52709957)(332.43449055,411.48709961)(332.45449341,411.44710297)
\curveto(332.45449053,411.37709972)(332.45949052,411.32209978)(332.46949341,411.28210297)
\lineto(332.49949341,411.10210297)
\curveto(332.50949047,411.04210006)(332.52449046,410.98210012)(332.54449341,410.92210297)
\curveto(332.63449035,410.63210047)(332.73949024,410.39210071)(332.85949341,410.20210297)
\curveto(332.98948999,410.02210108)(333.16948981,409.86210124)(333.39949341,409.72210297)
\curveto(333.53948944,409.64210146)(333.70448928,409.57710152)(333.89449341,409.52710297)
\curveto(333.93448905,409.51710158)(333.96948901,409.51210159)(333.99949341,409.51210297)
\curveto(334.02948895,409.52210158)(334.06448892,409.52210158)(334.10449341,409.51210297)
\curveto(334.14448884,409.5021016)(334.20448878,409.49210161)(334.28449341,409.48210297)
\curveto(334.36448862,409.48210162)(334.42948855,409.48710161)(334.47949341,409.49710297)
\curveto(334.55948842,409.51710158)(334.63948834,409.53210157)(334.71949341,409.54210297)
\curveto(334.80948817,409.56210154)(334.89448809,409.58710151)(334.97449341,409.61710297)
\curveto(335.21448777,409.71710138)(335.40948757,409.85710124)(335.55949341,410.03710297)
\curveto(335.70948727,410.21710088)(335.83448715,410.42710067)(335.93449341,410.66710297)
\curveto(335.984487,410.78710031)(336.01948696,410.91210019)(336.03949341,411.04210297)
\curveto(336.05948692,411.17209993)(336.0844869,411.30709979)(336.11449341,411.44710297)
\lineto(336.11449341,411.59710297)
\curveto(336.12448686,411.64709945)(336.12948685,411.6970994)(336.12949341,411.74710297)
}
}
{
\newrgbcolor{curcolor}{0 0 0}
\pscustom[linestyle=none,fillstyle=solid,fillcolor=curcolor]
{
\newpath
\moveto(347.29441528,412.31710297)
\curveto(347.31440671,412.25709884)(347.3244067,412.17209893)(347.32441528,412.06210297)
\curveto(347.3244067,411.95209915)(347.31440671,411.86709923)(347.29441528,411.80710297)
\lineto(347.29441528,411.65710297)
\curveto(347.27440675,411.57709952)(347.26440676,411.4970996)(347.26441528,411.41710297)
\curveto(347.27440675,411.33709976)(347.26940676,411.25709984)(347.24941528,411.17710297)
\curveto(347.2294068,411.10709999)(347.21440681,411.04210006)(347.20441528,410.98210297)
\curveto(347.19440683,410.92210018)(347.18440684,410.85710024)(347.17441528,410.78710297)
\curveto(347.13440689,410.67710042)(347.09940693,410.56210054)(347.06941528,410.44210297)
\curveto(347.03940699,410.33210077)(346.99940703,410.22710087)(346.94941528,410.12710297)
\curveto(346.73940729,409.64710145)(346.46440756,409.25710184)(346.12441528,408.95710297)
\curveto(345.78440824,408.65710244)(345.37440865,408.40710269)(344.89441528,408.20710297)
\curveto(344.77440925,408.15710294)(344.64940938,408.12210298)(344.51941528,408.10210297)
\curveto(344.39940963,408.07210303)(344.27440975,408.04210306)(344.14441528,408.01210297)
\curveto(344.09440993,407.99210311)(344.03940999,407.98210312)(343.97941528,407.98210297)
\curveto(343.91941011,407.98210312)(343.86441016,407.97710312)(343.81441528,407.96710297)
\lineto(343.70941528,407.96710297)
\curveto(343.67941035,407.95710314)(343.64941038,407.95210315)(343.61941528,407.95210297)
\curveto(343.56941046,407.94210316)(343.48941054,407.93710316)(343.37941528,407.93710297)
\curveto(343.26941076,407.92710317)(343.18441084,407.93210317)(343.12441528,407.95210297)
\lineto(342.97441528,407.95210297)
\curveto(342.9244111,407.96210314)(342.86941116,407.96710313)(342.80941528,407.96710297)
\curveto(342.75941127,407.95710314)(342.70941132,407.96210314)(342.65941528,407.98210297)
\curveto(342.61941141,407.99210311)(342.57941145,407.9971031)(342.53941528,407.99710297)
\curveto(342.50941152,407.9971031)(342.46941156,408.0021031)(342.41941528,408.01210297)
\curveto(342.31941171,408.04210306)(342.21941181,408.06710303)(342.11941528,408.08710297)
\curveto(342.01941201,408.10710299)(341.9244121,408.13710296)(341.83441528,408.17710297)
\curveto(341.71441231,408.21710288)(341.59941243,408.25710284)(341.48941528,408.29710297)
\curveto(341.38941264,408.33710276)(341.28441274,408.38710271)(341.17441528,408.44710297)
\curveto(340.8244132,408.65710244)(340.5244135,408.9021022)(340.27441528,409.18210297)
\curveto(340.024414,409.46210164)(339.81441421,409.7971013)(339.64441528,410.18710297)
\curveto(339.59441443,410.27710082)(339.55441447,410.37210073)(339.52441528,410.47210297)
\curveto(339.50441452,410.57210053)(339.47941455,410.67710042)(339.44941528,410.78710297)
\curveto(339.4294146,410.83710026)(339.41941461,410.88210022)(339.41941528,410.92210297)
\curveto(339.41941461,410.96210014)(339.40941462,411.00710009)(339.38941528,411.05710297)
\curveto(339.36941466,411.13709996)(339.35941467,411.21709988)(339.35941528,411.29710297)
\curveto(339.35941467,411.38709971)(339.34941468,411.47209963)(339.32941528,411.55210297)
\curveto(339.31941471,411.6020995)(339.31441471,411.64709945)(339.31441528,411.68710297)
\lineto(339.31441528,411.82210297)
\curveto(339.29441473,411.88209922)(339.28441474,411.96709913)(339.28441528,412.07710297)
\curveto(339.29441473,412.18709891)(339.30941472,412.27209883)(339.32941528,412.33210297)
\lineto(339.32941528,412.43710297)
\curveto(339.33941469,412.48709861)(339.33941469,412.53709856)(339.32941528,412.58710297)
\curveto(339.3294147,412.64709845)(339.33941469,412.7020984)(339.35941528,412.75210297)
\curveto(339.36941466,412.8020983)(339.37441465,412.84709825)(339.37441528,412.88710297)
\curveto(339.37441465,412.93709816)(339.38441464,412.98709811)(339.40441528,413.03710297)
\curveto(339.44441458,413.16709793)(339.47941455,413.29209781)(339.50941528,413.41210297)
\curveto(339.53941449,413.54209756)(339.57941445,413.66709743)(339.62941528,413.78710297)
\curveto(339.80941422,414.1970969)(340.024414,414.53709656)(340.27441528,414.80710297)
\curveto(340.5244135,415.08709601)(340.8294132,415.34209576)(341.18941528,415.57210297)
\curveto(341.28941274,415.62209548)(341.39441263,415.66709543)(341.50441528,415.70710297)
\curveto(341.61441241,415.74709535)(341.7244123,415.79209531)(341.83441528,415.84210297)
\curveto(341.96441206,415.89209521)(342.09941193,415.92709517)(342.23941528,415.94710297)
\curveto(342.37941165,415.96709513)(342.5244115,415.9970951)(342.67441528,416.03710297)
\curveto(342.75441127,416.04709505)(342.8294112,416.05209505)(342.89941528,416.05210297)
\curveto(342.96941106,416.05209505)(343.03941099,416.05709504)(343.10941528,416.06710297)
\curveto(343.68941034,416.07709502)(344.18940984,416.01709508)(344.60941528,415.88710297)
\curveto(345.03940899,415.75709534)(345.41940861,415.57709552)(345.74941528,415.34710297)
\curveto(345.85940817,415.26709583)(345.96940806,415.17709592)(346.07941528,415.07710297)
\curveto(346.19940783,414.98709611)(346.29940773,414.88709621)(346.37941528,414.77710297)
\curveto(346.45940757,414.67709642)(346.5294075,414.57709652)(346.58941528,414.47710297)
\curveto(346.65940737,414.37709672)(346.7294073,414.27209683)(346.79941528,414.16210297)
\curveto(346.86940716,414.05209705)(346.9244071,413.93209717)(346.96441528,413.80210297)
\curveto(347.00440702,413.68209742)(347.04940698,413.55209755)(347.09941528,413.41210297)
\curveto(347.1294069,413.33209777)(347.15440687,413.24709785)(347.17441528,413.15710297)
\lineto(347.23441528,412.88710297)
\curveto(347.24440678,412.84709825)(347.24940678,412.80709829)(347.24941528,412.76710297)
\curveto(347.24940678,412.72709837)(347.25440677,412.68709841)(347.26441528,412.64710297)
\curveto(347.28440674,412.5970985)(347.28940674,412.54209856)(347.27941528,412.48210297)
\curveto(347.26940676,412.42209868)(347.27440675,412.36709873)(347.29441528,412.31710297)
\moveto(345.19441528,411.77710297)
\curveto(345.20440882,411.82709927)(345.20940882,411.8970992)(345.20941528,411.98710297)
\curveto(345.20940882,412.08709901)(345.20440882,412.16209894)(345.19441528,412.21210297)
\lineto(345.19441528,412.33210297)
\curveto(345.17440885,412.38209872)(345.16440886,412.43709866)(345.16441528,412.49710297)
\curveto(345.16440886,412.55709854)(345.15940887,412.61209849)(345.14941528,412.66210297)
\curveto(345.14940888,412.7020984)(345.14440888,412.73209837)(345.13441528,412.75210297)
\lineto(345.07441528,412.99210297)
\curveto(345.06440896,413.08209802)(345.04440898,413.16709793)(345.01441528,413.24710297)
\curveto(344.90440912,413.50709759)(344.77440925,413.72709737)(344.62441528,413.90710297)
\curveto(344.47440955,414.097097)(344.27440975,414.24709685)(344.02441528,414.35710297)
\curveto(343.96441006,414.37709672)(343.90441012,414.39209671)(343.84441528,414.40210297)
\curveto(343.78441024,414.42209668)(343.71941031,414.44209666)(343.64941528,414.46210297)
\curveto(343.56941046,414.48209662)(343.48441054,414.48709661)(343.39441528,414.47710297)
\lineto(343.12441528,414.47710297)
\curveto(343.09441093,414.45709664)(343.05941097,414.44709665)(343.01941528,414.44710297)
\curveto(342.97941105,414.45709664)(342.94441108,414.45709664)(342.91441528,414.44710297)
\lineto(342.70441528,414.38710297)
\curveto(342.64441138,414.37709672)(342.58941144,414.35709674)(342.53941528,414.32710297)
\curveto(342.28941174,414.21709688)(342.08441194,414.05709704)(341.92441528,413.84710297)
\curveto(341.77441225,413.64709745)(341.65441237,413.41209769)(341.56441528,413.14210297)
\curveto(341.53441249,413.04209806)(341.50941252,412.93709816)(341.48941528,412.82710297)
\curveto(341.47941255,412.71709838)(341.46441256,412.60709849)(341.44441528,412.49710297)
\curveto(341.43441259,412.44709865)(341.4294126,412.3970987)(341.42941528,412.34710297)
\lineto(341.42941528,412.19710297)
\curveto(341.40941262,412.12709897)(341.39941263,412.02209908)(341.39941528,411.88210297)
\curveto(341.40941262,411.74209936)(341.4244126,411.63709946)(341.44441528,411.56710297)
\lineto(341.44441528,411.43210297)
\curveto(341.46441256,411.35209975)(341.47941255,411.27209983)(341.48941528,411.19210297)
\curveto(341.49941253,411.12209998)(341.51441251,411.04710005)(341.53441528,410.96710297)
\curveto(341.63441239,410.66710043)(341.73941229,410.42210068)(341.84941528,410.23210297)
\curveto(341.96941206,410.05210105)(342.15441187,409.88710121)(342.40441528,409.73710297)
\curveto(342.47441155,409.68710141)(342.54941148,409.64710145)(342.62941528,409.61710297)
\curveto(342.71941131,409.58710151)(342.80941122,409.56210154)(342.89941528,409.54210297)
\curveto(342.93941109,409.53210157)(342.97441105,409.52710157)(343.00441528,409.52710297)
\curveto(343.03441099,409.53710156)(343.06941096,409.53710156)(343.10941528,409.52710297)
\lineto(343.22941528,409.49710297)
\curveto(343.27941075,409.4971016)(343.3244107,409.5021016)(343.36441528,409.51210297)
\lineto(343.48441528,409.51210297)
\curveto(343.56441046,409.53210157)(343.64441038,409.54710155)(343.72441528,409.55710297)
\curveto(343.80441022,409.56710153)(343.87941015,409.58710151)(343.94941528,409.61710297)
\curveto(344.20940982,409.71710138)(344.41940961,409.85210125)(344.57941528,410.02210297)
\curveto(344.73940929,410.19210091)(344.87440915,410.4021007)(344.98441528,410.65210297)
\curveto(345.024409,410.75210035)(345.05440897,410.85210025)(345.07441528,410.95210297)
\curveto(345.09440893,411.05210005)(345.11940891,411.15709994)(345.14941528,411.26710297)
\curveto(345.15940887,411.30709979)(345.16440886,411.34209976)(345.16441528,411.37210297)
\curveto(345.16440886,411.41209969)(345.16940886,411.45209965)(345.17941528,411.49210297)
\lineto(345.17941528,411.62710297)
\curveto(345.17940885,411.67709942)(345.18440884,411.72709937)(345.19441528,411.77710297)
}
}
{
\newrgbcolor{curcolor}{0 0 0}
\pscustom[linestyle=none,fillstyle=solid,fillcolor=curcolor]
{
\newpath
\moveto(353.11933716,416.06710297)
\curveto(353.22933184,416.06709503)(353.32433175,416.05709504)(353.40433716,416.03710297)
\curveto(353.49433158,416.01709508)(353.56433151,415.97209513)(353.61433716,415.90210297)
\curveto(353.6743314,415.82209528)(353.70433137,415.68209542)(353.70433716,415.48210297)
\lineto(353.70433716,414.97210297)
\lineto(353.70433716,414.59710297)
\curveto(353.71433136,414.45709664)(353.69933137,414.34709675)(353.65933716,414.26710297)
\curveto(353.61933145,414.1970969)(353.55933151,414.15209695)(353.47933716,414.13210297)
\curveto(353.40933166,414.11209699)(353.32433175,414.102097)(353.22433716,414.10210297)
\curveto(353.13433194,414.102097)(353.03433204,414.10709699)(352.92433716,414.11710297)
\curveto(352.82433225,414.12709697)(352.72933234,414.12209698)(352.63933716,414.10210297)
\curveto(352.5693325,414.08209702)(352.49933257,414.06709703)(352.42933716,414.05710297)
\curveto(352.35933271,414.05709704)(352.29433278,414.04709705)(352.23433716,414.02710297)
\curveto(352.074333,413.97709712)(351.91433316,413.9020972)(351.75433716,413.80210297)
\curveto(351.59433348,413.71209739)(351.4693336,413.60709749)(351.37933716,413.48710297)
\curveto(351.32933374,413.40709769)(351.2743338,413.32209778)(351.21433716,413.23210297)
\curveto(351.16433391,413.15209795)(351.11433396,413.06709803)(351.06433716,412.97710297)
\curveto(351.03433404,412.8970982)(351.00433407,412.81209829)(350.97433716,412.72210297)
\lineto(350.91433716,412.48210297)
\curveto(350.89433418,412.41209869)(350.88433419,412.33709876)(350.88433716,412.25710297)
\curveto(350.88433419,412.18709891)(350.8743342,412.11709898)(350.85433716,412.04710297)
\curveto(350.84433423,412.00709909)(350.83933423,411.96709913)(350.83933716,411.92710297)
\curveto(350.84933422,411.8970992)(350.84933422,411.86709923)(350.83933716,411.83710297)
\lineto(350.83933716,411.59710297)
\curveto(350.81933425,411.52709957)(350.81433426,411.44709965)(350.82433716,411.35710297)
\curveto(350.83433424,411.27709982)(350.83933423,411.1970999)(350.83933716,411.11710297)
\lineto(350.83933716,410.15710297)
\lineto(350.83933716,408.88210297)
\curveto(350.83933423,408.75210235)(350.83433424,408.63210247)(350.82433716,408.52210297)
\curveto(350.81433426,408.41210269)(350.78433429,408.32210278)(350.73433716,408.25210297)
\curveto(350.71433436,408.22210288)(350.67933439,408.1971029)(350.62933716,408.17710297)
\curveto(350.58933448,408.16710293)(350.54433453,408.15710294)(350.49433716,408.14710297)
\lineto(350.41933716,408.14710297)
\curveto(350.3693347,408.13710296)(350.31433476,408.13210297)(350.25433716,408.13210297)
\lineto(350.08933716,408.13210297)
\lineto(349.44433716,408.13210297)
\curveto(349.38433569,408.14210296)(349.31933575,408.14710295)(349.24933716,408.14710297)
\lineto(349.05433716,408.14710297)
\curveto(349.00433607,408.16710293)(348.95433612,408.18210292)(348.90433716,408.19210297)
\curveto(348.85433622,408.21210289)(348.81933625,408.24710285)(348.79933716,408.29710297)
\curveto(348.75933631,408.34710275)(348.73433634,408.41710268)(348.72433716,408.50710297)
\lineto(348.72433716,408.80710297)
\lineto(348.72433716,409.82710297)
\lineto(348.72433716,414.05710297)
\lineto(348.72433716,415.16710297)
\lineto(348.72433716,415.45210297)
\curveto(348.72433635,415.55209555)(348.74433633,415.63209547)(348.78433716,415.69210297)
\curveto(348.83433624,415.77209533)(348.90933616,415.82209528)(349.00933716,415.84210297)
\curveto(349.10933596,415.86209524)(349.22933584,415.87209523)(349.36933716,415.87210297)
\lineto(350.13433716,415.87210297)
\curveto(350.25433482,415.87209523)(350.35933471,415.86209524)(350.44933716,415.84210297)
\curveto(350.53933453,415.83209527)(350.60933446,415.78709531)(350.65933716,415.70710297)
\curveto(350.68933438,415.65709544)(350.70433437,415.58709551)(350.70433716,415.49710297)
\lineto(350.73433716,415.22710297)
\curveto(350.74433433,415.14709595)(350.75933431,415.07209603)(350.77933716,415.00210297)
\curveto(350.80933426,414.93209617)(350.85933421,414.8970962)(350.92933716,414.89710297)
\curveto(350.94933412,414.91709618)(350.9693341,414.92709617)(350.98933716,414.92710297)
\curveto(351.00933406,414.92709617)(351.02933404,414.93709616)(351.04933716,414.95710297)
\curveto(351.10933396,415.00709609)(351.15933391,415.06209604)(351.19933716,415.12210297)
\curveto(351.24933382,415.19209591)(351.30933376,415.25209585)(351.37933716,415.30210297)
\curveto(351.41933365,415.33209577)(351.45433362,415.36209574)(351.48433716,415.39210297)
\curveto(351.51433356,415.43209567)(351.54933352,415.46709563)(351.58933716,415.49710297)
\lineto(351.85933716,415.67710297)
\curveto(351.95933311,415.73709536)(352.05933301,415.79209531)(352.15933716,415.84210297)
\curveto(352.25933281,415.88209522)(352.35933271,415.91709518)(352.45933716,415.94710297)
\lineto(352.78933716,416.03710297)
\curveto(352.81933225,416.04709505)(352.8743322,416.04709505)(352.95433716,416.03710297)
\curveto(353.04433203,416.03709506)(353.09933197,416.04709505)(353.11933716,416.06710297)
}
}
{
\newrgbcolor{curcolor}{0 0 0}
\pscustom[linestyle=none,fillstyle=solid,fillcolor=curcolor]
{
}
}
{
\newrgbcolor{curcolor}{0 0 0}
\pscustom[linestyle=none,fillstyle=solid,fillcolor=curcolor]
{
\newpath
\moveto(359.73457153,418.18210297)
\lineto(360.73957153,418.18210297)
\curveto(360.88956855,418.18209292)(361.01956842,418.17209293)(361.12957153,418.15210297)
\curveto(361.24956819,418.14209296)(361.3345681,418.08209302)(361.38457153,417.97210297)
\curveto(361.40456803,417.92209318)(361.41456802,417.86209324)(361.41457153,417.79210297)
\lineto(361.41457153,417.58210297)
\lineto(361.41457153,416.90710297)
\curveto(361.41456802,416.85709424)(361.40956803,416.7970943)(361.39957153,416.72710297)
\curveto(361.39956804,416.66709443)(361.40456803,416.61209449)(361.41457153,416.56210297)
\lineto(361.41457153,416.39710297)
\curveto(361.41456802,416.31709478)(361.41956802,416.24209486)(361.42957153,416.17210297)
\curveto(361.439568,416.11209499)(361.46456797,416.05709504)(361.50457153,416.00710297)
\curveto(361.57456786,415.91709518)(361.69956774,415.86709523)(361.87957153,415.85710297)
\lineto(362.41957153,415.85710297)
\lineto(362.59957153,415.85710297)
\curveto(362.65956678,415.85709524)(362.71456672,415.84709525)(362.76457153,415.82710297)
\curveto(362.87456656,415.77709532)(362.9345665,415.68709541)(362.94457153,415.55710297)
\curveto(362.96456647,415.42709567)(362.97456646,415.28209582)(362.97457153,415.12210297)
\lineto(362.97457153,414.91210297)
\curveto(362.98456645,414.84209626)(362.97956646,414.78209632)(362.95957153,414.73210297)
\curveto(362.90956653,414.57209653)(362.80456663,414.48709661)(362.64457153,414.47710297)
\curveto(362.48456695,414.46709663)(362.30456713,414.46209664)(362.10457153,414.46210297)
\lineto(361.96957153,414.46210297)
\curveto(361.92956751,414.47209663)(361.89456754,414.47209663)(361.86457153,414.46210297)
\curveto(361.82456761,414.45209665)(361.78956765,414.44709665)(361.75957153,414.44710297)
\curveto(361.72956771,414.45709664)(361.69956774,414.45209665)(361.66957153,414.43210297)
\curveto(361.58956785,414.41209669)(361.52956791,414.36709673)(361.48957153,414.29710297)
\curveto(361.45956798,414.23709686)(361.434568,414.16209694)(361.41457153,414.07210297)
\curveto(361.40456803,414.02209708)(361.40456803,413.96709713)(361.41457153,413.90710297)
\curveto(361.42456801,413.84709725)(361.42456801,413.79209731)(361.41457153,413.74210297)
\lineto(361.41457153,412.81210297)
\lineto(361.41457153,411.05710297)
\curveto(361.41456802,410.80710029)(361.41956802,410.58710051)(361.42957153,410.39710297)
\curveto(361.44956799,410.21710088)(361.51456792,410.05710104)(361.62457153,409.91710297)
\curveto(361.67456776,409.85710124)(361.7395677,409.81210129)(361.81957153,409.78210297)
\lineto(362.08957153,409.72210297)
\curveto(362.11956732,409.71210139)(362.14956729,409.70710139)(362.17957153,409.70710297)
\curveto(362.21956722,409.71710138)(362.24956719,409.71710138)(362.26957153,409.70710297)
\lineto(362.43457153,409.70710297)
\curveto(362.54456689,409.70710139)(362.6395668,409.7021014)(362.71957153,409.69210297)
\curveto(362.79956664,409.68210142)(362.86456657,409.64210146)(362.91457153,409.57210297)
\curveto(362.95456648,409.51210159)(362.97456646,409.43210167)(362.97457153,409.33210297)
\lineto(362.97457153,409.04710297)
\curveto(362.97456646,408.83710226)(362.96956647,408.64210246)(362.95957153,408.46210297)
\curveto(362.95956648,408.29210281)(362.87956656,408.17710292)(362.71957153,408.11710297)
\curveto(362.66956677,408.097103)(362.62456681,408.09210301)(362.58457153,408.10210297)
\curveto(362.54456689,408.102103)(362.49956694,408.09210301)(362.44957153,408.07210297)
\lineto(362.29957153,408.07210297)
\curveto(362.27956716,408.07210303)(362.24956719,408.07710302)(362.20957153,408.08710297)
\curveto(362.16956727,408.08710301)(362.1345673,408.08210302)(362.10457153,408.07210297)
\curveto(362.05456738,408.06210304)(361.99956744,408.06210304)(361.93957153,408.07210297)
\lineto(361.78957153,408.07210297)
\lineto(361.63957153,408.07210297)
\curveto(361.58956785,408.06210304)(361.54456789,408.06210304)(361.50457153,408.07210297)
\lineto(361.33957153,408.07210297)
\curveto(361.28956815,408.08210302)(361.2345682,408.08710301)(361.17457153,408.08710297)
\curveto(361.11456832,408.08710301)(361.05956838,408.09210301)(361.00957153,408.10210297)
\curveto(360.9395685,408.11210299)(360.87456856,408.12210298)(360.81457153,408.13210297)
\lineto(360.63457153,408.16210297)
\curveto(360.52456891,408.19210291)(360.41956902,408.22710287)(360.31957153,408.26710297)
\curveto(360.21956922,408.30710279)(360.12456931,408.35210275)(360.03457153,408.40210297)
\lineto(359.94457153,408.46210297)
\curveto(359.91456952,408.49210261)(359.87956956,408.52210258)(359.83957153,408.55210297)
\curveto(359.81956962,408.57210253)(359.79456964,408.59210251)(359.76457153,408.61210297)
\lineto(359.68957153,408.68710297)
\curveto(359.54956989,408.87710222)(359.44456999,409.08710201)(359.37457153,409.31710297)
\curveto(359.35457008,409.35710174)(359.34457009,409.39210171)(359.34457153,409.42210297)
\curveto(359.35457008,409.46210164)(359.35457008,409.50710159)(359.34457153,409.55710297)
\curveto(359.3345701,409.57710152)(359.32957011,409.6021015)(359.32957153,409.63210297)
\curveto(359.32957011,409.66210144)(359.32457011,409.68710141)(359.31457153,409.70710297)
\lineto(359.31457153,409.85710297)
\curveto(359.30457013,409.8971012)(359.29957014,409.94210116)(359.29957153,409.99210297)
\curveto(359.30957013,410.04210106)(359.31457012,410.09210101)(359.31457153,410.14210297)
\lineto(359.31457153,410.71210297)
\lineto(359.31457153,412.94710297)
\lineto(359.31457153,413.74210297)
\lineto(359.31457153,413.95210297)
\curveto(359.32457011,414.02209708)(359.31957012,414.08709701)(359.29957153,414.14710297)
\curveto(359.25957018,414.28709681)(359.18957025,414.37709672)(359.08957153,414.41710297)
\curveto(358.97957046,414.46709663)(358.8395706,414.48209662)(358.66957153,414.46210297)
\curveto(358.49957094,414.44209666)(358.35457108,414.45709664)(358.23457153,414.50710297)
\curveto(358.15457128,414.53709656)(358.10457133,414.58209652)(358.08457153,414.64210297)
\curveto(358.06457137,414.7020964)(358.04457139,414.77709632)(358.02457153,414.86710297)
\lineto(358.02457153,415.18210297)
\curveto(358.02457141,415.36209574)(358.0345714,415.50709559)(358.05457153,415.61710297)
\curveto(358.07457136,415.72709537)(358.15957128,415.8020953)(358.30957153,415.84210297)
\curveto(358.34957109,415.86209524)(358.38957105,415.86709523)(358.42957153,415.85710297)
\lineto(358.56457153,415.85710297)
\curveto(358.71457072,415.85709524)(358.85457058,415.86209524)(358.98457153,415.87210297)
\curveto(359.11457032,415.89209521)(359.20457023,415.95209515)(359.25457153,416.05210297)
\curveto(359.28457015,416.12209498)(359.29957014,416.2020949)(359.29957153,416.29210297)
\curveto(359.30957013,416.38209472)(359.31457012,416.47209463)(359.31457153,416.56210297)
\lineto(359.31457153,417.49210297)
\lineto(359.31457153,417.74710297)
\curveto(359.31457012,417.83709326)(359.32457011,417.91209319)(359.34457153,417.97210297)
\curveto(359.39457004,418.07209303)(359.46956997,418.13709296)(359.56957153,418.16710297)
\curveto(359.58956985,418.17709292)(359.61456982,418.17709292)(359.64457153,418.16710297)
\curveto(359.68456975,418.16709293)(359.71456972,418.17209293)(359.73457153,418.18210297)
}
}
{
\newrgbcolor{curcolor}{0 0 0}
\pscustom[linestyle=none,fillstyle=solid,fillcolor=curcolor]
{
\newpath
\moveto(366.05800903,418.72210297)
\curveto(366.12800608,418.64209246)(366.16300605,418.52209258)(366.16300903,418.36210297)
\lineto(366.16300903,417.89710297)
\lineto(366.16300903,417.49210297)
\curveto(366.16300605,417.35209375)(366.12800608,417.25709384)(366.05800903,417.20710297)
\curveto(365.99800621,417.15709394)(365.91800629,417.12709397)(365.81800903,417.11710297)
\curveto(365.72800648,417.10709399)(365.62800658,417.102094)(365.51800903,417.10210297)
\lineto(364.67800903,417.10210297)
\curveto(364.56800764,417.102094)(364.46800774,417.10709399)(364.37800903,417.11710297)
\curveto(364.29800791,417.12709397)(364.22800798,417.15709394)(364.16800903,417.20710297)
\curveto(364.12800808,417.23709386)(364.09800811,417.29209381)(364.07800903,417.37210297)
\curveto(364.06800814,417.46209364)(364.05800815,417.55709354)(364.04800903,417.65710297)
\lineto(364.04800903,417.98710297)
\curveto(364.05800815,418.097093)(364.06300815,418.19209291)(364.06300903,418.27210297)
\lineto(364.06300903,418.48210297)
\curveto(364.07300814,418.55209255)(364.09300812,418.61209249)(364.12300903,418.66210297)
\curveto(364.14300807,418.7020924)(364.16800804,418.73209237)(364.19800903,418.75210297)
\lineto(364.31800903,418.81210297)
\curveto(364.33800787,418.81209229)(364.36300785,418.81209229)(364.39300903,418.81210297)
\curveto(364.42300779,418.82209228)(364.44800776,418.82709227)(364.46800903,418.82710297)
\lineto(365.56300903,418.82710297)
\curveto(365.66300655,418.82709227)(365.75800645,418.82209228)(365.84800903,418.81210297)
\curveto(365.93800627,418.8020923)(366.0080062,418.77209233)(366.05800903,418.72210297)
\moveto(366.16300903,408.95710297)
\curveto(366.16300605,408.75710234)(366.15800605,408.58710251)(366.14800903,408.44710297)
\curveto(366.13800607,408.30710279)(366.04800616,408.21210289)(365.87800903,408.16210297)
\curveto(365.81800639,408.14210296)(365.75300646,408.13210297)(365.68300903,408.13210297)
\curveto(365.6130066,408.14210296)(365.53800667,408.14710295)(365.45800903,408.14710297)
\lineto(364.61800903,408.14710297)
\curveto(364.52800768,408.14710295)(364.43800777,408.15210295)(364.34800903,408.16210297)
\curveto(364.26800794,408.17210293)(364.208008,408.2021029)(364.16800903,408.25210297)
\curveto(364.1080081,408.32210278)(364.07300814,408.40710269)(364.06300903,408.50710297)
\lineto(364.06300903,408.85210297)
\lineto(364.06300903,415.18210297)
\lineto(364.06300903,415.48210297)
\curveto(364.06300815,415.58209552)(364.08300813,415.66209544)(364.12300903,415.72210297)
\curveto(364.18300803,415.79209531)(364.26800794,415.83709526)(364.37800903,415.85710297)
\curveto(364.39800781,415.86709523)(364.42300779,415.86709523)(364.45300903,415.85710297)
\curveto(364.49300772,415.85709524)(364.52300769,415.86209524)(364.54300903,415.87210297)
\lineto(365.29300903,415.87210297)
\lineto(365.48800903,415.87210297)
\curveto(365.56800664,415.88209522)(365.63300658,415.88209522)(365.68300903,415.87210297)
\lineto(365.80300903,415.87210297)
\curveto(365.86300635,415.85209525)(365.91800629,415.83709526)(365.96800903,415.82710297)
\curveto(366.01800619,415.81709528)(366.05800615,415.78709531)(366.08800903,415.73710297)
\curveto(366.12800608,415.68709541)(366.14800606,415.61709548)(366.14800903,415.52710297)
\curveto(366.15800605,415.43709566)(366.16300605,415.34209576)(366.16300903,415.24210297)
\lineto(366.16300903,408.95710297)
}
}
{
\newrgbcolor{curcolor}{0 0 0}
\pscustom[linestyle=none,fillstyle=solid,fillcolor=curcolor]
{
\newpath
\moveto(375.71519653,412.09210297)
\curveto(375.72518785,412.03209907)(375.73018785,411.94209916)(375.73019653,411.82210297)
\curveto(375.73018785,411.7020994)(375.72018786,411.61709948)(375.70019653,411.56710297)
\lineto(375.70019653,411.37210297)
\curveto(375.67018791,411.26209984)(375.65018793,411.15709994)(375.64019653,411.05710297)
\curveto(375.64018794,410.95710014)(375.62518795,410.85710024)(375.59519653,410.75710297)
\curveto(375.575188,410.66710043)(375.55518802,410.57210053)(375.53519653,410.47210297)
\curveto(375.51518806,410.38210072)(375.48518809,410.29210081)(375.44519653,410.20210297)
\curveto(375.3751882,410.03210107)(375.30518827,409.87210123)(375.23519653,409.72210297)
\curveto(375.16518841,409.58210152)(375.08518849,409.44210166)(374.99519653,409.30210297)
\curveto(374.93518864,409.21210189)(374.87018871,409.12710197)(374.80019653,409.04710297)
\curveto(374.74018884,408.97710212)(374.67018891,408.9021022)(374.59019653,408.82210297)
\lineto(374.48519653,408.71710297)
\curveto(374.43518914,408.66710243)(374.3801892,408.62210248)(374.32019653,408.58210297)
\lineto(374.17019653,408.46210297)
\curveto(374.09018949,408.4021027)(374.00018958,408.34710275)(373.90019653,408.29710297)
\curveto(373.81018977,408.25710284)(373.71518986,408.21210289)(373.61519653,408.16210297)
\curveto(373.51519006,408.11210299)(373.41019017,408.07710302)(373.30019653,408.05710297)
\curveto(373.20019038,408.03710306)(373.09519048,408.01710308)(372.98519653,407.99710297)
\curveto(372.92519065,407.97710312)(372.86019072,407.96710313)(372.79019653,407.96710297)
\curveto(372.73019085,407.96710313)(372.66519091,407.95710314)(372.59519653,407.93710297)
\lineto(372.46019653,407.93710297)
\curveto(372.3801912,407.91710318)(372.30519127,407.91710318)(372.23519653,407.93710297)
\lineto(372.08519653,407.93710297)
\curveto(372.02519155,407.95710314)(371.96019162,407.96710313)(371.89019653,407.96710297)
\curveto(371.83019175,407.95710314)(371.77019181,407.96210314)(371.71019653,407.98210297)
\curveto(371.55019203,408.03210307)(371.39519218,408.07710302)(371.24519653,408.11710297)
\curveto(371.10519247,408.15710294)(370.9751926,408.21710288)(370.85519653,408.29710297)
\curveto(370.78519279,408.33710276)(370.72019286,408.37710272)(370.66019653,408.41710297)
\curveto(370.60019298,408.46710263)(370.53519304,408.51710258)(370.46519653,408.56710297)
\lineto(370.28519653,408.70210297)
\curveto(370.20519337,408.76210234)(370.13519344,408.76710233)(370.07519653,408.71710297)
\curveto(370.02519355,408.68710241)(370.00019358,408.64710245)(370.00019653,408.59710297)
\curveto(370.00019358,408.55710254)(369.99019359,408.50710259)(369.97019653,408.44710297)
\curveto(369.95019363,408.34710275)(369.94019364,408.23210287)(369.94019653,408.10210297)
\curveto(369.95019363,407.97210313)(369.95519362,407.85210325)(369.95519653,407.74210297)
\lineto(369.95519653,406.21210297)
\curveto(369.95519362,406.08210502)(369.95019363,405.95710514)(369.94019653,405.83710297)
\curveto(369.94019364,405.70710539)(369.91519366,405.6021055)(369.86519653,405.52210297)
\curveto(369.83519374,405.48210562)(369.7801938,405.45210565)(369.70019653,405.43210297)
\curveto(369.62019396,405.41210569)(369.53019405,405.4021057)(369.43019653,405.40210297)
\curveto(369.33019425,405.39210571)(369.23019435,405.39210571)(369.13019653,405.40210297)
\lineto(368.87519653,405.40210297)
\lineto(368.47019653,405.40210297)
\lineto(368.36519653,405.40210297)
\curveto(368.32519525,405.4021057)(368.29019529,405.40710569)(368.26019653,405.41710297)
\lineto(368.14019653,405.41710297)
\curveto(367.97019561,405.46710563)(367.8801957,405.56710553)(367.87019653,405.71710297)
\curveto(367.86019572,405.85710524)(367.85519572,406.02710507)(367.85519653,406.22710297)
\lineto(367.85519653,415.03210297)
\curveto(367.85519572,415.14209596)(367.85019573,415.25709584)(367.84019653,415.37710297)
\curveto(367.84019574,415.50709559)(367.86519571,415.60709549)(367.91519653,415.67710297)
\curveto(367.95519562,415.74709535)(368.01019557,415.79209531)(368.08019653,415.81210297)
\curveto(368.13019545,415.83209527)(368.19019539,415.84209526)(368.26019653,415.84210297)
\lineto(368.48519653,415.84210297)
\lineto(369.20519653,415.84210297)
\lineto(369.49019653,415.84210297)
\curveto(369.580194,415.84209526)(369.65519392,415.81709528)(369.71519653,415.76710297)
\curveto(369.78519379,415.71709538)(369.82019376,415.65209545)(369.82019653,415.57210297)
\curveto(369.83019375,415.5020956)(369.85519372,415.42709567)(369.89519653,415.34710297)
\curveto(369.90519367,415.31709578)(369.91519366,415.29209581)(369.92519653,415.27210297)
\curveto(369.94519363,415.26209584)(369.96519361,415.24709585)(369.98519653,415.22710297)
\curveto(370.09519348,415.21709588)(370.18519339,415.24709585)(370.25519653,415.31710297)
\curveto(370.32519325,415.38709571)(370.39519318,415.44709565)(370.46519653,415.49710297)
\curveto(370.59519298,415.58709551)(370.73019285,415.66709543)(370.87019653,415.73710297)
\curveto(371.01019257,415.81709528)(371.16519241,415.88209522)(371.33519653,415.93210297)
\curveto(371.41519216,415.96209514)(371.50019208,415.98209512)(371.59019653,415.99210297)
\curveto(371.69019189,416.0020951)(371.78519179,416.01709508)(371.87519653,416.03710297)
\curveto(371.91519166,416.04709505)(371.95519162,416.04709505)(371.99519653,416.03710297)
\curveto(372.04519153,416.02709507)(372.08519149,416.03209507)(372.11519653,416.05210297)
\curveto(372.68519089,416.07209503)(373.16519041,415.99209511)(373.55519653,415.81210297)
\curveto(373.95518962,415.64209546)(374.29518928,415.41709568)(374.57519653,415.13710297)
\curveto(374.62518895,415.08709601)(374.67018891,415.03709606)(374.71019653,414.98710297)
\curveto(374.75018883,414.94709615)(374.79018879,414.9020962)(374.83019653,414.85210297)
\curveto(374.90018868,414.76209634)(374.96018862,414.67209643)(375.01019653,414.58210297)
\curveto(375.07018851,414.49209661)(375.12518845,414.4020967)(375.17519653,414.31210297)
\curveto(375.19518838,414.29209681)(375.20518837,414.26709683)(375.20519653,414.23710297)
\curveto(375.21518836,414.20709689)(375.23018835,414.17209693)(375.25019653,414.13210297)
\curveto(375.31018827,414.03209707)(375.36518821,413.91209719)(375.41519653,413.77210297)
\curveto(375.43518814,413.71209739)(375.45518812,413.64709745)(375.47519653,413.57710297)
\curveto(375.49518808,413.51709758)(375.51518806,413.45209765)(375.53519653,413.38210297)
\curveto(375.575188,413.26209784)(375.60018798,413.13709796)(375.61019653,413.00710297)
\curveto(375.63018795,412.87709822)(375.65518792,412.74209836)(375.68519653,412.60210297)
\lineto(375.68519653,412.43710297)
\lineto(375.71519653,412.25710297)
\lineto(375.71519653,412.09210297)
\moveto(373.60019653,411.74710297)
\curveto(373.61018997,411.7970993)(373.61518996,411.86209924)(373.61519653,411.94210297)
\curveto(373.61518996,412.03209907)(373.61018997,412.102099)(373.60019653,412.15210297)
\lineto(373.60019653,412.28710297)
\curveto(373.58019,412.34709875)(373.57019001,412.41209869)(373.57019653,412.48210297)
\curveto(373.57019001,412.55209855)(373.56019002,412.62209848)(373.54019653,412.69210297)
\curveto(373.52019006,412.79209831)(373.50019008,412.88709821)(373.48019653,412.97710297)
\curveto(373.46019012,413.07709802)(373.43019015,413.16709793)(373.39019653,413.24710297)
\curveto(373.27019031,413.56709753)(373.11519046,413.82209728)(372.92519653,414.01210297)
\curveto(372.73519084,414.2020969)(372.46519111,414.34209676)(372.11519653,414.43210297)
\curveto(372.03519154,414.45209665)(371.94519163,414.46209664)(371.84519653,414.46210297)
\lineto(371.57519653,414.46210297)
\curveto(371.53519204,414.45209665)(371.50019208,414.44709665)(371.47019653,414.44710297)
\curveto(371.44019214,414.44709665)(371.40519217,414.44209666)(371.36519653,414.43210297)
\lineto(371.15519653,414.37210297)
\curveto(371.09519248,414.36209674)(371.03519254,414.34209676)(370.97519653,414.31210297)
\curveto(370.71519286,414.2020969)(370.51019307,414.03209707)(370.36019653,413.80210297)
\curveto(370.22019336,413.57209753)(370.10519347,413.31709778)(370.01519653,413.03710297)
\curveto(369.99519358,412.95709814)(369.9801936,412.87209823)(369.97019653,412.78210297)
\curveto(369.96019362,412.7020984)(369.94519363,412.62209848)(369.92519653,412.54210297)
\curveto(369.91519366,412.5020986)(369.91019367,412.43709866)(369.91019653,412.34710297)
\curveto(369.89019369,412.30709879)(369.88519369,412.25709884)(369.89519653,412.19710297)
\curveto(369.90519367,412.14709895)(369.90519367,412.097099)(369.89519653,412.04710297)
\curveto(369.8751937,411.98709911)(369.8751937,411.93209917)(369.89519653,411.88210297)
\lineto(369.89519653,411.70210297)
\lineto(369.89519653,411.56710297)
\curveto(369.89519368,411.52709957)(369.90519367,411.48709961)(369.92519653,411.44710297)
\curveto(369.92519365,411.37709972)(369.93019365,411.32209978)(369.94019653,411.28210297)
\lineto(369.97019653,411.10210297)
\curveto(369.9801936,411.04210006)(369.99519358,410.98210012)(370.01519653,410.92210297)
\curveto(370.10519347,410.63210047)(370.21019337,410.39210071)(370.33019653,410.20210297)
\curveto(370.46019312,410.02210108)(370.64019294,409.86210124)(370.87019653,409.72210297)
\curveto(371.01019257,409.64210146)(371.1751924,409.57710152)(371.36519653,409.52710297)
\curveto(371.40519217,409.51710158)(371.44019214,409.51210159)(371.47019653,409.51210297)
\curveto(371.50019208,409.52210158)(371.53519204,409.52210158)(371.57519653,409.51210297)
\curveto(371.61519196,409.5021016)(371.6751919,409.49210161)(371.75519653,409.48210297)
\curveto(371.83519174,409.48210162)(371.90019168,409.48710161)(371.95019653,409.49710297)
\curveto(372.03019155,409.51710158)(372.11019147,409.53210157)(372.19019653,409.54210297)
\curveto(372.2801913,409.56210154)(372.36519121,409.58710151)(372.44519653,409.61710297)
\curveto(372.68519089,409.71710138)(372.8801907,409.85710124)(373.03019653,410.03710297)
\curveto(373.1801904,410.21710088)(373.30519027,410.42710067)(373.40519653,410.66710297)
\curveto(373.45519012,410.78710031)(373.49019009,410.91210019)(373.51019653,411.04210297)
\curveto(373.53019005,411.17209993)(373.55519002,411.30709979)(373.58519653,411.44710297)
\lineto(373.58519653,411.59710297)
\curveto(373.59518998,411.64709945)(373.60018998,411.6970994)(373.60019653,411.74710297)
}
}
{
\newrgbcolor{curcolor}{0 0 0}
\pscustom[linestyle=none,fillstyle=solid,fillcolor=curcolor]
{
\newpath
\moveto(384.76511841,412.31710297)
\curveto(384.78510984,412.25709884)(384.79510983,412.17209893)(384.79511841,412.06210297)
\curveto(384.79510983,411.95209915)(384.78510984,411.86709923)(384.76511841,411.80710297)
\lineto(384.76511841,411.65710297)
\curveto(384.74510988,411.57709952)(384.73510989,411.4970996)(384.73511841,411.41710297)
\curveto(384.74510988,411.33709976)(384.74010988,411.25709984)(384.72011841,411.17710297)
\curveto(384.70010992,411.10709999)(384.68510994,411.04210006)(384.67511841,410.98210297)
\curveto(384.66510996,410.92210018)(384.65510997,410.85710024)(384.64511841,410.78710297)
\curveto(384.60511002,410.67710042)(384.57011005,410.56210054)(384.54011841,410.44210297)
\curveto(384.51011011,410.33210077)(384.47011015,410.22710087)(384.42011841,410.12710297)
\curveto(384.21011041,409.64710145)(383.93511069,409.25710184)(383.59511841,408.95710297)
\curveto(383.25511137,408.65710244)(382.84511178,408.40710269)(382.36511841,408.20710297)
\curveto(382.24511238,408.15710294)(382.1201125,408.12210298)(381.99011841,408.10210297)
\curveto(381.87011275,408.07210303)(381.74511288,408.04210306)(381.61511841,408.01210297)
\curveto(381.56511306,407.99210311)(381.51011311,407.98210312)(381.45011841,407.98210297)
\curveto(381.39011323,407.98210312)(381.33511329,407.97710312)(381.28511841,407.96710297)
\lineto(381.18011841,407.96710297)
\curveto(381.15011347,407.95710314)(381.1201135,407.95210315)(381.09011841,407.95210297)
\curveto(381.04011358,407.94210316)(380.96011366,407.93710316)(380.85011841,407.93710297)
\curveto(380.74011388,407.92710317)(380.65511397,407.93210317)(380.59511841,407.95210297)
\lineto(380.44511841,407.95210297)
\curveto(380.39511423,407.96210314)(380.34011428,407.96710313)(380.28011841,407.96710297)
\curveto(380.23011439,407.95710314)(380.18011444,407.96210314)(380.13011841,407.98210297)
\curveto(380.09011453,407.99210311)(380.05011457,407.9971031)(380.01011841,407.99710297)
\curveto(379.98011464,407.9971031)(379.94011468,408.0021031)(379.89011841,408.01210297)
\curveto(379.79011483,408.04210306)(379.69011493,408.06710303)(379.59011841,408.08710297)
\curveto(379.49011513,408.10710299)(379.39511523,408.13710296)(379.30511841,408.17710297)
\curveto(379.18511544,408.21710288)(379.07011555,408.25710284)(378.96011841,408.29710297)
\curveto(378.86011576,408.33710276)(378.75511587,408.38710271)(378.64511841,408.44710297)
\curveto(378.29511633,408.65710244)(377.99511663,408.9021022)(377.74511841,409.18210297)
\curveto(377.49511713,409.46210164)(377.28511734,409.7971013)(377.11511841,410.18710297)
\curveto(377.06511756,410.27710082)(377.0251176,410.37210073)(376.99511841,410.47210297)
\curveto(376.97511765,410.57210053)(376.95011767,410.67710042)(376.92011841,410.78710297)
\curveto(376.90011772,410.83710026)(376.89011773,410.88210022)(376.89011841,410.92210297)
\curveto(376.89011773,410.96210014)(376.88011774,411.00710009)(376.86011841,411.05710297)
\curveto(376.84011778,411.13709996)(376.83011779,411.21709988)(376.83011841,411.29710297)
\curveto(376.83011779,411.38709971)(376.8201178,411.47209963)(376.80011841,411.55210297)
\curveto(376.79011783,411.6020995)(376.78511784,411.64709945)(376.78511841,411.68710297)
\lineto(376.78511841,411.82210297)
\curveto(376.76511786,411.88209922)(376.75511787,411.96709913)(376.75511841,412.07710297)
\curveto(376.76511786,412.18709891)(376.78011784,412.27209883)(376.80011841,412.33210297)
\lineto(376.80011841,412.43710297)
\curveto(376.81011781,412.48709861)(376.81011781,412.53709856)(376.80011841,412.58710297)
\curveto(376.80011782,412.64709845)(376.81011781,412.7020984)(376.83011841,412.75210297)
\curveto(376.84011778,412.8020983)(376.84511778,412.84709825)(376.84511841,412.88710297)
\curveto(376.84511778,412.93709816)(376.85511777,412.98709811)(376.87511841,413.03710297)
\curveto(376.91511771,413.16709793)(376.95011767,413.29209781)(376.98011841,413.41210297)
\curveto(377.01011761,413.54209756)(377.05011757,413.66709743)(377.10011841,413.78710297)
\curveto(377.28011734,414.1970969)(377.49511713,414.53709656)(377.74511841,414.80710297)
\curveto(377.99511663,415.08709601)(378.30011632,415.34209576)(378.66011841,415.57210297)
\curveto(378.76011586,415.62209548)(378.86511576,415.66709543)(378.97511841,415.70710297)
\curveto(379.08511554,415.74709535)(379.19511543,415.79209531)(379.30511841,415.84210297)
\curveto(379.43511519,415.89209521)(379.57011505,415.92709517)(379.71011841,415.94710297)
\curveto(379.85011477,415.96709513)(379.99511463,415.9970951)(380.14511841,416.03710297)
\curveto(380.2251144,416.04709505)(380.30011432,416.05209505)(380.37011841,416.05210297)
\curveto(380.44011418,416.05209505)(380.51011411,416.05709504)(380.58011841,416.06710297)
\curveto(381.16011346,416.07709502)(381.66011296,416.01709508)(382.08011841,415.88710297)
\curveto(382.51011211,415.75709534)(382.89011173,415.57709552)(383.22011841,415.34710297)
\curveto(383.33011129,415.26709583)(383.44011118,415.17709592)(383.55011841,415.07710297)
\curveto(383.67011095,414.98709611)(383.77011085,414.88709621)(383.85011841,414.77710297)
\curveto(383.93011069,414.67709642)(384.00011062,414.57709652)(384.06011841,414.47710297)
\curveto(384.13011049,414.37709672)(384.20011042,414.27209683)(384.27011841,414.16210297)
\curveto(384.34011028,414.05209705)(384.39511023,413.93209717)(384.43511841,413.80210297)
\curveto(384.47511015,413.68209742)(384.5201101,413.55209755)(384.57011841,413.41210297)
\curveto(384.60011002,413.33209777)(384.62511,413.24709785)(384.64511841,413.15710297)
\lineto(384.70511841,412.88710297)
\curveto(384.71510991,412.84709825)(384.7201099,412.80709829)(384.72011841,412.76710297)
\curveto(384.7201099,412.72709837)(384.7251099,412.68709841)(384.73511841,412.64710297)
\curveto(384.75510987,412.5970985)(384.76010986,412.54209856)(384.75011841,412.48210297)
\curveto(384.74010988,412.42209868)(384.74510988,412.36709873)(384.76511841,412.31710297)
\moveto(382.66511841,411.77710297)
\curveto(382.67511195,411.82709927)(382.68011194,411.8970992)(382.68011841,411.98710297)
\curveto(382.68011194,412.08709901)(382.67511195,412.16209894)(382.66511841,412.21210297)
\lineto(382.66511841,412.33210297)
\curveto(382.64511198,412.38209872)(382.63511199,412.43709866)(382.63511841,412.49710297)
\curveto(382.63511199,412.55709854)(382.63011199,412.61209849)(382.62011841,412.66210297)
\curveto(382.620112,412.7020984)(382.61511201,412.73209837)(382.60511841,412.75210297)
\lineto(382.54511841,412.99210297)
\curveto(382.53511209,413.08209802)(382.51511211,413.16709793)(382.48511841,413.24710297)
\curveto(382.37511225,413.50709759)(382.24511238,413.72709737)(382.09511841,413.90710297)
\curveto(381.94511268,414.097097)(381.74511288,414.24709685)(381.49511841,414.35710297)
\curveto(381.43511319,414.37709672)(381.37511325,414.39209671)(381.31511841,414.40210297)
\curveto(381.25511337,414.42209668)(381.19011343,414.44209666)(381.12011841,414.46210297)
\curveto(381.04011358,414.48209662)(380.95511367,414.48709661)(380.86511841,414.47710297)
\lineto(380.59511841,414.47710297)
\curveto(380.56511406,414.45709664)(380.53011409,414.44709665)(380.49011841,414.44710297)
\curveto(380.45011417,414.45709664)(380.41511421,414.45709664)(380.38511841,414.44710297)
\lineto(380.17511841,414.38710297)
\curveto(380.11511451,414.37709672)(380.06011456,414.35709674)(380.01011841,414.32710297)
\curveto(379.76011486,414.21709688)(379.55511507,414.05709704)(379.39511841,413.84710297)
\curveto(379.24511538,413.64709745)(379.1251155,413.41209769)(379.03511841,413.14210297)
\curveto(379.00511562,413.04209806)(378.98011564,412.93709816)(378.96011841,412.82710297)
\curveto(378.95011567,412.71709838)(378.93511569,412.60709849)(378.91511841,412.49710297)
\curveto(378.90511572,412.44709865)(378.90011572,412.3970987)(378.90011841,412.34710297)
\lineto(378.90011841,412.19710297)
\curveto(378.88011574,412.12709897)(378.87011575,412.02209908)(378.87011841,411.88210297)
\curveto(378.88011574,411.74209936)(378.89511573,411.63709946)(378.91511841,411.56710297)
\lineto(378.91511841,411.43210297)
\curveto(378.93511569,411.35209975)(378.95011567,411.27209983)(378.96011841,411.19210297)
\curveto(378.97011565,411.12209998)(378.98511564,411.04710005)(379.00511841,410.96710297)
\curveto(379.10511552,410.66710043)(379.21011541,410.42210068)(379.32011841,410.23210297)
\curveto(379.44011518,410.05210105)(379.625115,409.88710121)(379.87511841,409.73710297)
\curveto(379.94511468,409.68710141)(380.0201146,409.64710145)(380.10011841,409.61710297)
\curveto(380.19011443,409.58710151)(380.28011434,409.56210154)(380.37011841,409.54210297)
\curveto(380.41011421,409.53210157)(380.44511418,409.52710157)(380.47511841,409.52710297)
\curveto(380.50511412,409.53710156)(380.54011408,409.53710156)(380.58011841,409.52710297)
\lineto(380.70011841,409.49710297)
\curveto(380.75011387,409.4971016)(380.79511383,409.5021016)(380.83511841,409.51210297)
\lineto(380.95511841,409.51210297)
\curveto(381.03511359,409.53210157)(381.11511351,409.54710155)(381.19511841,409.55710297)
\curveto(381.27511335,409.56710153)(381.35011327,409.58710151)(381.42011841,409.61710297)
\curveto(381.68011294,409.71710138)(381.89011273,409.85210125)(382.05011841,410.02210297)
\curveto(382.21011241,410.19210091)(382.34511228,410.4021007)(382.45511841,410.65210297)
\curveto(382.49511213,410.75210035)(382.5251121,410.85210025)(382.54511841,410.95210297)
\curveto(382.56511206,411.05210005)(382.59011203,411.15709994)(382.62011841,411.26710297)
\curveto(382.63011199,411.30709979)(382.63511199,411.34209976)(382.63511841,411.37210297)
\curveto(382.63511199,411.41209969)(382.64011198,411.45209965)(382.65011841,411.49210297)
\lineto(382.65011841,411.62710297)
\curveto(382.65011197,411.67709942)(382.65511197,411.72709937)(382.66511841,411.77710297)
}
}
{
\newrgbcolor{curcolor}{0 0 0}
\pscustom[linestyle=none,fillstyle=solid,fillcolor=curcolor]
{
}
}
{
\newrgbcolor{curcolor}{0 0 0}
\pscustom[linestyle=none,fillstyle=solid,fillcolor=curcolor]
{
\newpath
\moveto(397.91519653,408.98710297)
\lineto(397.91519653,408.56710297)
\curveto(397.91518816,408.43710266)(397.88518819,408.33210277)(397.82519653,408.25210297)
\curveto(397.7751883,408.2021029)(397.71018837,408.16710293)(397.63019653,408.14710297)
\curveto(397.55018853,408.13710296)(397.46018862,408.13210297)(397.36019653,408.13210297)
\lineto(396.53519653,408.13210297)
\lineto(396.25019653,408.13210297)
\curveto(396.17018991,408.14210296)(396.10518997,408.16710293)(396.05519653,408.20710297)
\curveto(395.98519009,408.25710284)(395.94519013,408.32210278)(395.93519653,408.40210297)
\curveto(395.92519015,408.48210262)(395.90519017,408.56210254)(395.87519653,408.64210297)
\curveto(395.85519022,408.66210244)(395.83519024,408.67710242)(395.81519653,408.68710297)
\curveto(395.80519027,408.70710239)(395.79019029,408.72710237)(395.77019653,408.74710297)
\curveto(395.66019042,408.74710235)(395.5801905,408.72210238)(395.53019653,408.67210297)
\lineto(395.38019653,408.52210297)
\curveto(395.31019077,408.47210263)(395.24519083,408.42710267)(395.18519653,408.38710297)
\curveto(395.12519095,408.35710274)(395.06019102,408.31710278)(394.99019653,408.26710297)
\curveto(394.95019113,408.24710285)(394.90519117,408.22710287)(394.85519653,408.20710297)
\curveto(394.81519126,408.18710291)(394.77019131,408.16710293)(394.72019653,408.14710297)
\curveto(394.5801915,408.097103)(394.43019165,408.05210305)(394.27019653,408.01210297)
\curveto(394.22019186,407.99210311)(394.1751919,407.98210312)(394.13519653,407.98210297)
\curveto(394.09519198,407.98210312)(394.05519202,407.97710312)(394.01519653,407.96710297)
\lineto(393.88019653,407.96710297)
\curveto(393.85019223,407.95710314)(393.81019227,407.95210315)(393.76019653,407.95210297)
\lineto(393.62519653,407.95210297)
\curveto(393.56519251,407.93210317)(393.4751926,407.92710317)(393.35519653,407.93710297)
\curveto(393.23519284,407.93710316)(393.15019293,407.94710315)(393.10019653,407.96710297)
\curveto(393.03019305,407.98710311)(392.96519311,407.9971031)(392.90519653,407.99710297)
\curveto(392.85519322,407.98710311)(392.80019328,407.99210311)(392.74019653,408.01210297)
\lineto(392.38019653,408.13210297)
\curveto(392.27019381,408.16210294)(392.16019392,408.2021029)(392.05019653,408.25210297)
\curveto(391.70019438,408.4021027)(391.38519469,408.63210247)(391.10519653,408.94210297)
\curveto(390.83519524,409.26210184)(390.62019546,409.5971015)(390.46019653,409.94710297)
\curveto(390.41019567,410.05710104)(390.37019571,410.16210094)(390.34019653,410.26210297)
\curveto(390.31019577,410.37210073)(390.2751958,410.48210062)(390.23519653,410.59210297)
\curveto(390.22519585,410.63210047)(390.22019586,410.66710043)(390.22019653,410.69710297)
\curveto(390.22019586,410.73710036)(390.21019587,410.78210032)(390.19019653,410.83210297)
\curveto(390.17019591,410.91210019)(390.15019593,410.9971001)(390.13019653,411.08710297)
\curveto(390.12019596,411.18709991)(390.10519597,411.28709981)(390.08519653,411.38710297)
\curveto(390.075196,411.41709968)(390.07019601,411.45209965)(390.07019653,411.49210297)
\curveto(390.080196,411.53209957)(390.080196,411.56709953)(390.07019653,411.59710297)
\lineto(390.07019653,411.73210297)
\curveto(390.07019601,411.78209932)(390.06519601,411.83209927)(390.05519653,411.88210297)
\curveto(390.04519603,411.93209917)(390.04019604,411.98709911)(390.04019653,412.04710297)
\curveto(390.04019604,412.11709898)(390.04519603,412.17209893)(390.05519653,412.21210297)
\curveto(390.06519601,412.26209884)(390.07019601,412.30709879)(390.07019653,412.34710297)
\lineto(390.07019653,412.49710297)
\curveto(390.080196,412.54709855)(390.080196,412.59209851)(390.07019653,412.63210297)
\curveto(390.07019601,412.68209842)(390.080196,412.73209837)(390.10019653,412.78210297)
\curveto(390.12019596,412.89209821)(390.13519594,412.9970981)(390.14519653,413.09710297)
\curveto(390.16519591,413.1970979)(390.19019589,413.2970978)(390.22019653,413.39710297)
\curveto(390.26019582,413.51709758)(390.29519578,413.63209747)(390.32519653,413.74210297)
\curveto(390.35519572,413.85209725)(390.39519568,413.96209714)(390.44519653,414.07210297)
\curveto(390.58519549,414.37209673)(390.76019532,414.65709644)(390.97019653,414.92710297)
\curveto(390.99019509,414.95709614)(391.01519506,414.98209612)(391.04519653,415.00210297)
\curveto(391.08519499,415.03209607)(391.11519496,415.06209604)(391.13519653,415.09210297)
\curveto(391.1751949,415.14209596)(391.21519486,415.18709591)(391.25519653,415.22710297)
\curveto(391.29519478,415.26709583)(391.34019474,415.30709579)(391.39019653,415.34710297)
\curveto(391.43019465,415.36709573)(391.46519461,415.39209571)(391.49519653,415.42210297)
\curveto(391.52519455,415.46209564)(391.56019452,415.49209561)(391.60019653,415.51210297)
\curveto(391.85019423,415.68209542)(392.14019394,415.82209528)(392.47019653,415.93210297)
\curveto(392.54019354,415.95209515)(392.61019347,415.96709513)(392.68019653,415.97710297)
\curveto(392.76019332,415.98709511)(392.84019324,416.0020951)(392.92019653,416.02210297)
\curveto(392.99019309,416.04209506)(393.080193,416.05209505)(393.19019653,416.05210297)
\curveto(393.30019278,416.06209504)(393.41019267,416.06709503)(393.52019653,416.06710297)
\curveto(393.63019245,416.06709503)(393.73519234,416.06209504)(393.83519653,416.05210297)
\curveto(393.94519213,416.04209506)(394.03519204,416.02709507)(394.10519653,416.00710297)
\curveto(394.25519182,415.95709514)(394.40019168,415.91209519)(394.54019653,415.87210297)
\curveto(394.6801914,415.83209527)(394.81019127,415.77709532)(394.93019653,415.70710297)
\curveto(395.00019108,415.65709544)(395.06519101,415.60709549)(395.12519653,415.55710297)
\curveto(395.18519089,415.51709558)(395.25019083,415.47209563)(395.32019653,415.42210297)
\curveto(395.36019072,415.39209571)(395.41519066,415.35209575)(395.48519653,415.30210297)
\curveto(395.56519051,415.25209585)(395.64019044,415.25209585)(395.71019653,415.30210297)
\curveto(395.75019033,415.32209578)(395.77019031,415.35709574)(395.77019653,415.40710297)
\curveto(395.77019031,415.45709564)(395.7801903,415.50709559)(395.80019653,415.55710297)
\lineto(395.80019653,415.70710297)
\curveto(395.81019027,415.73709536)(395.81519026,415.77209533)(395.81519653,415.81210297)
\lineto(395.81519653,415.93210297)
\lineto(395.81519653,417.97210297)
\curveto(395.81519026,418.08209302)(395.81019027,418.2020929)(395.80019653,418.33210297)
\curveto(395.80019028,418.47209263)(395.82519025,418.57709252)(395.87519653,418.64710297)
\curveto(395.91519016,418.72709237)(395.99019009,418.77709232)(396.10019653,418.79710297)
\curveto(396.12018996,418.80709229)(396.14018994,418.80709229)(396.16019653,418.79710297)
\curveto(396.1801899,418.7970923)(396.20018988,418.8020923)(396.22019653,418.81210297)
\lineto(397.28519653,418.81210297)
\curveto(397.40518867,418.81209229)(397.51518856,418.80709229)(397.61519653,418.79710297)
\curveto(397.71518836,418.78709231)(397.79018829,418.74709235)(397.84019653,418.67710297)
\curveto(397.89018819,418.5970925)(397.91518816,418.49209261)(397.91519653,418.36210297)
\lineto(397.91519653,418.00210297)
\lineto(397.91519653,408.98710297)
\moveto(395.87519653,411.92710297)
\curveto(395.88519019,411.96709913)(395.88519019,412.00709909)(395.87519653,412.04710297)
\lineto(395.87519653,412.18210297)
\curveto(395.8751902,412.28209882)(395.87019021,412.38209872)(395.86019653,412.48210297)
\curveto(395.85019023,412.58209852)(395.83519024,412.67209843)(395.81519653,412.75210297)
\curveto(395.79519028,412.86209824)(395.7751903,412.96209814)(395.75519653,413.05210297)
\curveto(395.74519033,413.14209796)(395.72019036,413.22709787)(395.68019653,413.30710297)
\curveto(395.54019054,413.66709743)(395.33519074,413.95209715)(395.06519653,414.16210297)
\curveto(394.80519127,414.37209673)(394.42519165,414.47709662)(393.92519653,414.47710297)
\curveto(393.86519221,414.47709662)(393.78519229,414.46709663)(393.68519653,414.44710297)
\curveto(393.60519247,414.42709667)(393.53019255,414.40709669)(393.46019653,414.38710297)
\curveto(393.40019268,414.37709672)(393.34019274,414.35709674)(393.28019653,414.32710297)
\curveto(393.01019307,414.21709688)(392.80019328,414.04709705)(392.65019653,413.81710297)
\curveto(392.50019358,413.58709751)(392.3801937,413.32709777)(392.29019653,413.03710297)
\curveto(392.26019382,412.93709816)(392.24019384,412.83709826)(392.23019653,412.73710297)
\curveto(392.22019386,412.63709846)(392.20019388,412.53209857)(392.17019653,412.42210297)
\lineto(392.17019653,412.21210297)
\curveto(392.15019393,412.12209898)(392.14519393,411.9970991)(392.15519653,411.83710297)
\curveto(392.16519391,411.68709941)(392.1801939,411.57709952)(392.20019653,411.50710297)
\lineto(392.20019653,411.41710297)
\curveto(392.21019387,411.3970997)(392.21519386,411.37709972)(392.21519653,411.35710297)
\curveto(392.23519384,411.27709982)(392.25019383,411.2020999)(392.26019653,411.13210297)
\curveto(392.2801938,411.06210004)(392.30019378,410.98710011)(392.32019653,410.90710297)
\curveto(392.49019359,410.38710071)(392.7801933,410.0021011)(393.19019653,409.75210297)
\curveto(393.32019276,409.66210144)(393.50019258,409.59210151)(393.73019653,409.54210297)
\curveto(393.77019231,409.53210157)(393.83019225,409.52710157)(393.91019653,409.52710297)
\curveto(393.94019214,409.51710158)(393.98519209,409.50710159)(394.04519653,409.49710297)
\curveto(394.11519196,409.4971016)(394.17019191,409.5021016)(394.21019653,409.51210297)
\curveto(394.29019179,409.53210157)(394.37019171,409.54710155)(394.45019653,409.55710297)
\curveto(394.53019155,409.56710153)(394.61019147,409.58710151)(394.69019653,409.61710297)
\curveto(394.94019114,409.72710137)(395.14019094,409.86710123)(395.29019653,410.03710297)
\curveto(395.44019064,410.20710089)(395.57019051,410.42210068)(395.68019653,410.68210297)
\curveto(395.72019036,410.77210033)(395.75019033,410.86210024)(395.77019653,410.95210297)
\curveto(395.79019029,411.05210005)(395.81019027,411.15709994)(395.83019653,411.26710297)
\curveto(395.84019024,411.31709978)(395.84019024,411.36209974)(395.83019653,411.40210297)
\curveto(395.83019025,411.45209965)(395.84019024,411.5020996)(395.86019653,411.55210297)
\curveto(395.87019021,411.58209952)(395.8751902,411.61709948)(395.87519653,411.65710297)
\lineto(395.87519653,411.79210297)
\lineto(395.87519653,411.92710297)
}
}
{
\newrgbcolor{curcolor}{0 0 0}
\pscustom[linestyle=none,fillstyle=solid,fillcolor=curcolor]
{
\newpath
\moveto(406.86011841,412.07710297)
\curveto(406.88011024,411.9970991)(406.88011024,411.90709919)(406.86011841,411.80710297)
\curveto(406.84011028,411.70709939)(406.80511032,411.64209946)(406.75511841,411.61210297)
\curveto(406.70511042,411.57209953)(406.63011049,411.54209956)(406.53011841,411.52210297)
\curveto(406.44011068,411.51209959)(406.33511079,411.5020996)(406.21511841,411.49210297)
\lineto(405.87011841,411.49210297)
\curveto(405.76011136,411.5020996)(405.66011146,411.50709959)(405.57011841,411.50710297)
\lineto(401.91011841,411.50710297)
\lineto(401.70011841,411.50710297)
\curveto(401.64011548,411.50709959)(401.58511554,411.4970996)(401.53511841,411.47710297)
\curveto(401.45511567,411.43709966)(401.40511572,411.3970997)(401.38511841,411.35710297)
\curveto(401.36511576,411.33709976)(401.34511578,411.2970998)(401.32511841,411.23710297)
\curveto(401.30511582,411.18709991)(401.30011582,411.13709996)(401.31011841,411.08710297)
\curveto(401.33011579,411.02710007)(401.34011578,410.96710013)(401.34011841,410.90710297)
\curveto(401.35011577,410.85710024)(401.36511576,410.8021003)(401.38511841,410.74210297)
\curveto(401.46511566,410.5021006)(401.56011556,410.3021008)(401.67011841,410.14210297)
\curveto(401.79011533,409.99210111)(401.95011517,409.85710124)(402.15011841,409.73710297)
\curveto(402.23011489,409.68710141)(402.31011481,409.65210145)(402.39011841,409.63210297)
\curveto(402.48011464,409.62210148)(402.57011455,409.6021015)(402.66011841,409.57210297)
\curveto(402.74011438,409.55210155)(402.85011427,409.53710156)(402.99011841,409.52710297)
\curveto(403.13011399,409.51710158)(403.25011387,409.52210158)(403.35011841,409.54210297)
\lineto(403.48511841,409.54210297)
\curveto(403.58511354,409.56210154)(403.67511345,409.58210152)(403.75511841,409.60210297)
\curveto(403.84511328,409.63210147)(403.93011319,409.66210144)(404.01011841,409.69210297)
\curveto(404.11011301,409.74210136)(404.2201129,409.80710129)(404.34011841,409.88710297)
\curveto(404.47011265,409.96710113)(404.56511256,410.04710105)(404.62511841,410.12710297)
\curveto(404.67511245,410.1971009)(404.7251124,410.26210084)(404.77511841,410.32210297)
\curveto(404.83511229,410.39210071)(404.90511222,410.44210066)(404.98511841,410.47210297)
\curveto(405.08511204,410.52210058)(405.21011191,410.54210056)(405.36011841,410.53210297)
\lineto(405.79511841,410.53210297)
\lineto(405.97511841,410.53210297)
\curveto(406.04511108,410.54210056)(406.10511102,410.53710056)(406.15511841,410.51710297)
\lineto(406.30511841,410.51710297)
\curveto(406.40511072,410.4971006)(406.47511065,410.47210063)(406.51511841,410.44210297)
\curveto(406.55511057,410.42210068)(406.57511055,410.37710072)(406.57511841,410.30710297)
\curveto(406.58511054,410.23710086)(406.58011054,410.17710092)(406.56011841,410.12710297)
\curveto(406.51011061,409.98710111)(406.45511067,409.86210124)(406.39511841,409.75210297)
\curveto(406.33511079,409.64210146)(406.26511086,409.53210157)(406.18511841,409.42210297)
\curveto(405.96511116,409.09210201)(405.71511141,408.82710227)(405.43511841,408.62710297)
\curveto(405.15511197,408.42710267)(404.80511232,408.25710284)(404.38511841,408.11710297)
\curveto(404.27511285,408.07710302)(404.16511296,408.05210305)(404.05511841,408.04210297)
\curveto(403.94511318,408.03210307)(403.83011329,408.01210309)(403.71011841,407.98210297)
\curveto(403.67011345,407.97210313)(403.6251135,407.97210313)(403.57511841,407.98210297)
\curveto(403.53511359,407.98210312)(403.49511363,407.97710312)(403.45511841,407.96710297)
\lineto(403.29011841,407.96710297)
\curveto(403.24011388,407.94710315)(403.18011394,407.94210316)(403.11011841,407.95210297)
\curveto(403.05011407,407.95210315)(402.99511413,407.95710314)(402.94511841,407.96710297)
\curveto(402.86511426,407.97710312)(402.79511433,407.97710312)(402.73511841,407.96710297)
\curveto(402.67511445,407.95710314)(402.61011451,407.96210314)(402.54011841,407.98210297)
\curveto(402.49011463,408.0021031)(402.43511469,408.01210309)(402.37511841,408.01210297)
\curveto(402.31511481,408.01210309)(402.26011486,408.02210308)(402.21011841,408.04210297)
\curveto(402.10011502,408.06210304)(401.99011513,408.08710301)(401.88011841,408.11710297)
\curveto(401.77011535,408.13710296)(401.67011545,408.17210293)(401.58011841,408.22210297)
\curveto(401.47011565,408.26210284)(401.36511576,408.2971028)(401.26511841,408.32710297)
\curveto(401.17511595,408.36710273)(401.09011603,408.41210269)(401.01011841,408.46210297)
\curveto(400.69011643,408.66210244)(400.40511672,408.89210221)(400.15511841,409.15210297)
\curveto(399.90511722,409.42210168)(399.70011742,409.73210137)(399.54011841,410.08210297)
\curveto(399.49011763,410.19210091)(399.45011767,410.3021008)(399.42011841,410.41210297)
\curveto(399.39011773,410.53210057)(399.35011777,410.65210045)(399.30011841,410.77210297)
\curveto(399.29011783,410.81210029)(399.28511784,410.84710025)(399.28511841,410.87710297)
\curveto(399.28511784,410.91710018)(399.28011784,410.95710014)(399.27011841,410.99710297)
\curveto(399.23011789,411.11709998)(399.20511792,411.24709985)(399.19511841,411.38710297)
\lineto(399.16511841,411.80710297)
\curveto(399.16511796,411.85709924)(399.16011796,411.91209919)(399.15011841,411.97210297)
\curveto(399.15011797,412.03209907)(399.15511797,412.08709901)(399.16511841,412.13710297)
\lineto(399.16511841,412.31710297)
\lineto(399.21011841,412.67710297)
\curveto(399.25011787,412.84709825)(399.28511784,413.01209809)(399.31511841,413.17210297)
\curveto(399.34511778,413.33209777)(399.39011773,413.48209762)(399.45011841,413.62210297)
\curveto(399.88011724,414.66209644)(400.61011651,415.3970957)(401.64011841,415.82710297)
\curveto(401.78011534,415.88709521)(401.9201152,415.92709517)(402.06011841,415.94710297)
\curveto(402.21011491,415.97709512)(402.36511476,416.01209509)(402.52511841,416.05210297)
\curveto(402.60511452,416.06209504)(402.68011444,416.06709503)(402.75011841,416.06710297)
\curveto(402.8201143,416.06709503)(402.89511423,416.07209503)(402.97511841,416.08210297)
\curveto(403.48511364,416.09209501)(403.9201132,416.03209507)(404.28011841,415.90210297)
\curveto(404.65011247,415.78209532)(404.98011214,415.62209548)(405.27011841,415.42210297)
\curveto(405.36011176,415.36209574)(405.45011167,415.29209581)(405.54011841,415.21210297)
\curveto(405.63011149,415.14209596)(405.71011141,415.06709603)(405.78011841,414.98710297)
\curveto(405.81011131,414.93709616)(405.85011127,414.8970962)(405.90011841,414.86710297)
\curveto(405.98011114,414.75709634)(406.05511107,414.64209646)(406.12511841,414.52210297)
\curveto(406.19511093,414.41209669)(406.27011085,414.2970968)(406.35011841,414.17710297)
\curveto(406.40011072,414.08709701)(406.44011068,413.99209711)(406.47011841,413.89210297)
\curveto(406.51011061,413.8020973)(406.55011057,413.7020974)(406.59011841,413.59210297)
\curveto(406.64011048,413.46209764)(406.68011044,413.32709777)(406.71011841,413.18710297)
\curveto(406.74011038,413.04709805)(406.77511035,412.90709819)(406.81511841,412.76710297)
\curveto(406.83511029,412.68709841)(406.84011028,412.5970985)(406.83011841,412.49710297)
\curveto(406.83011029,412.40709869)(406.84011028,412.32209878)(406.86011841,412.24210297)
\lineto(406.86011841,412.07710297)
\moveto(404.61011841,412.96210297)
\curveto(404.68011244,413.06209804)(404.68511244,413.18209792)(404.62511841,413.32210297)
\curveto(404.57511255,413.47209763)(404.53511259,413.58209752)(404.50511841,413.65210297)
\curveto(404.36511276,413.92209718)(404.18011294,414.12709697)(403.95011841,414.26710297)
\curveto(403.7201134,414.41709668)(403.40011372,414.4970966)(402.99011841,414.50710297)
\curveto(402.96011416,414.48709661)(402.9251142,414.48209662)(402.88511841,414.49210297)
\curveto(402.84511428,414.5020966)(402.81011431,414.5020966)(402.78011841,414.49210297)
\curveto(402.73011439,414.47209663)(402.67511445,414.45709664)(402.61511841,414.44710297)
\curveto(402.55511457,414.44709665)(402.50011462,414.43709666)(402.45011841,414.41710297)
\curveto(402.01011511,414.27709682)(401.68511544,414.0020971)(401.47511841,413.59210297)
\curveto(401.45511567,413.55209755)(401.43011569,413.4970976)(401.40011841,413.42710297)
\curveto(401.38011574,413.36709773)(401.36511576,413.3020978)(401.35511841,413.23210297)
\curveto(401.34511578,413.17209793)(401.34511578,413.11209799)(401.35511841,413.05210297)
\curveto(401.37511575,412.99209811)(401.41011571,412.94209816)(401.46011841,412.90210297)
\curveto(401.54011558,412.85209825)(401.65011547,412.82709827)(401.79011841,412.82710297)
\lineto(402.19511841,412.82710297)
\lineto(403.86011841,412.82710297)
\lineto(404.29511841,412.82710297)
\curveto(404.45511267,412.83709826)(404.56011256,412.88209822)(404.61011841,412.96210297)
}
}
{
\newrgbcolor{curcolor}{0 0 0}
\pscustom[linestyle=none,fillstyle=solid,fillcolor=curcolor]
{
}
}
{
\newrgbcolor{curcolor}{0 0 0}
\pscustom[linestyle=none,fillstyle=solid,fillcolor=curcolor]
{
\newpath
\moveto(419.63355591,412.07710297)
\curveto(419.65354774,411.9970991)(419.65354774,411.90709919)(419.63355591,411.80710297)
\curveto(419.61354778,411.70709939)(419.57854782,411.64209946)(419.52855591,411.61210297)
\curveto(419.47854792,411.57209953)(419.40354799,411.54209956)(419.30355591,411.52210297)
\curveto(419.21354818,411.51209959)(419.10854829,411.5020996)(418.98855591,411.49210297)
\lineto(418.64355591,411.49210297)
\curveto(418.53354886,411.5020996)(418.43354896,411.50709959)(418.34355591,411.50710297)
\lineto(414.68355591,411.50710297)
\lineto(414.47355591,411.50710297)
\curveto(414.41355298,411.50709959)(414.35855304,411.4970996)(414.30855591,411.47710297)
\curveto(414.22855317,411.43709966)(414.17855322,411.3970997)(414.15855591,411.35710297)
\curveto(414.13855326,411.33709976)(414.11855328,411.2970998)(414.09855591,411.23710297)
\curveto(414.07855332,411.18709991)(414.07355332,411.13709996)(414.08355591,411.08710297)
\curveto(414.10355329,411.02710007)(414.11355328,410.96710013)(414.11355591,410.90710297)
\curveto(414.12355327,410.85710024)(414.13855326,410.8021003)(414.15855591,410.74210297)
\curveto(414.23855316,410.5021006)(414.33355306,410.3021008)(414.44355591,410.14210297)
\curveto(414.56355283,409.99210111)(414.72355267,409.85710124)(414.92355591,409.73710297)
\curveto(415.00355239,409.68710141)(415.08355231,409.65210145)(415.16355591,409.63210297)
\curveto(415.25355214,409.62210148)(415.34355205,409.6021015)(415.43355591,409.57210297)
\curveto(415.51355188,409.55210155)(415.62355177,409.53710156)(415.76355591,409.52710297)
\curveto(415.90355149,409.51710158)(416.02355137,409.52210158)(416.12355591,409.54210297)
\lineto(416.25855591,409.54210297)
\curveto(416.35855104,409.56210154)(416.44855095,409.58210152)(416.52855591,409.60210297)
\curveto(416.61855078,409.63210147)(416.70355069,409.66210144)(416.78355591,409.69210297)
\curveto(416.88355051,409.74210136)(416.9935504,409.80710129)(417.11355591,409.88710297)
\curveto(417.24355015,409.96710113)(417.33855006,410.04710105)(417.39855591,410.12710297)
\curveto(417.44854995,410.1971009)(417.4985499,410.26210084)(417.54855591,410.32210297)
\curveto(417.60854979,410.39210071)(417.67854972,410.44210066)(417.75855591,410.47210297)
\curveto(417.85854954,410.52210058)(417.98354941,410.54210056)(418.13355591,410.53210297)
\lineto(418.56855591,410.53210297)
\lineto(418.74855591,410.53210297)
\curveto(418.81854858,410.54210056)(418.87854852,410.53710056)(418.92855591,410.51710297)
\lineto(419.07855591,410.51710297)
\curveto(419.17854822,410.4971006)(419.24854815,410.47210063)(419.28855591,410.44210297)
\curveto(419.32854807,410.42210068)(419.34854805,410.37710072)(419.34855591,410.30710297)
\curveto(419.35854804,410.23710086)(419.35354804,410.17710092)(419.33355591,410.12710297)
\curveto(419.28354811,409.98710111)(419.22854817,409.86210124)(419.16855591,409.75210297)
\curveto(419.10854829,409.64210146)(419.03854836,409.53210157)(418.95855591,409.42210297)
\curveto(418.73854866,409.09210201)(418.48854891,408.82710227)(418.20855591,408.62710297)
\curveto(417.92854947,408.42710267)(417.57854982,408.25710284)(417.15855591,408.11710297)
\curveto(417.04855035,408.07710302)(416.93855046,408.05210305)(416.82855591,408.04210297)
\curveto(416.71855068,408.03210307)(416.60355079,408.01210309)(416.48355591,407.98210297)
\curveto(416.44355095,407.97210313)(416.398551,407.97210313)(416.34855591,407.98210297)
\curveto(416.30855109,407.98210312)(416.26855113,407.97710312)(416.22855591,407.96710297)
\lineto(416.06355591,407.96710297)
\curveto(416.01355138,407.94710315)(415.95355144,407.94210316)(415.88355591,407.95210297)
\curveto(415.82355157,407.95210315)(415.76855163,407.95710314)(415.71855591,407.96710297)
\curveto(415.63855176,407.97710312)(415.56855183,407.97710312)(415.50855591,407.96710297)
\curveto(415.44855195,407.95710314)(415.38355201,407.96210314)(415.31355591,407.98210297)
\curveto(415.26355213,408.0021031)(415.20855219,408.01210309)(415.14855591,408.01210297)
\curveto(415.08855231,408.01210309)(415.03355236,408.02210308)(414.98355591,408.04210297)
\curveto(414.87355252,408.06210304)(414.76355263,408.08710301)(414.65355591,408.11710297)
\curveto(414.54355285,408.13710296)(414.44355295,408.17210293)(414.35355591,408.22210297)
\curveto(414.24355315,408.26210284)(414.13855326,408.2971028)(414.03855591,408.32710297)
\curveto(413.94855345,408.36710273)(413.86355353,408.41210269)(413.78355591,408.46210297)
\curveto(413.46355393,408.66210244)(413.17855422,408.89210221)(412.92855591,409.15210297)
\curveto(412.67855472,409.42210168)(412.47355492,409.73210137)(412.31355591,410.08210297)
\curveto(412.26355513,410.19210091)(412.22355517,410.3021008)(412.19355591,410.41210297)
\curveto(412.16355523,410.53210057)(412.12355527,410.65210045)(412.07355591,410.77210297)
\curveto(412.06355533,410.81210029)(412.05855534,410.84710025)(412.05855591,410.87710297)
\curveto(412.05855534,410.91710018)(412.05355534,410.95710014)(412.04355591,410.99710297)
\curveto(412.00355539,411.11709998)(411.97855542,411.24709985)(411.96855591,411.38710297)
\lineto(411.93855591,411.80710297)
\curveto(411.93855546,411.85709924)(411.93355546,411.91209919)(411.92355591,411.97210297)
\curveto(411.92355547,412.03209907)(411.92855547,412.08709901)(411.93855591,412.13710297)
\lineto(411.93855591,412.31710297)
\lineto(411.98355591,412.67710297)
\curveto(412.02355537,412.84709825)(412.05855534,413.01209809)(412.08855591,413.17210297)
\curveto(412.11855528,413.33209777)(412.16355523,413.48209762)(412.22355591,413.62210297)
\curveto(412.65355474,414.66209644)(413.38355401,415.3970957)(414.41355591,415.82710297)
\curveto(414.55355284,415.88709521)(414.6935527,415.92709517)(414.83355591,415.94710297)
\curveto(414.98355241,415.97709512)(415.13855226,416.01209509)(415.29855591,416.05210297)
\curveto(415.37855202,416.06209504)(415.45355194,416.06709503)(415.52355591,416.06710297)
\curveto(415.5935518,416.06709503)(415.66855173,416.07209503)(415.74855591,416.08210297)
\curveto(416.25855114,416.09209501)(416.6935507,416.03209507)(417.05355591,415.90210297)
\curveto(417.42354997,415.78209532)(417.75354964,415.62209548)(418.04355591,415.42210297)
\curveto(418.13354926,415.36209574)(418.22354917,415.29209581)(418.31355591,415.21210297)
\curveto(418.40354899,415.14209596)(418.48354891,415.06709603)(418.55355591,414.98710297)
\curveto(418.58354881,414.93709616)(418.62354877,414.8970962)(418.67355591,414.86710297)
\curveto(418.75354864,414.75709634)(418.82854857,414.64209646)(418.89855591,414.52210297)
\curveto(418.96854843,414.41209669)(419.04354835,414.2970968)(419.12355591,414.17710297)
\curveto(419.17354822,414.08709701)(419.21354818,413.99209711)(419.24355591,413.89210297)
\curveto(419.28354811,413.8020973)(419.32354807,413.7020974)(419.36355591,413.59210297)
\curveto(419.41354798,413.46209764)(419.45354794,413.32709777)(419.48355591,413.18710297)
\curveto(419.51354788,413.04709805)(419.54854785,412.90709819)(419.58855591,412.76710297)
\curveto(419.60854779,412.68709841)(419.61354778,412.5970985)(419.60355591,412.49710297)
\curveto(419.60354779,412.40709869)(419.61354778,412.32209878)(419.63355591,412.24210297)
\lineto(419.63355591,412.07710297)
\moveto(417.38355591,412.96210297)
\curveto(417.45354994,413.06209804)(417.45854994,413.18209792)(417.39855591,413.32210297)
\curveto(417.34855005,413.47209763)(417.30855009,413.58209752)(417.27855591,413.65210297)
\curveto(417.13855026,413.92209718)(416.95355044,414.12709697)(416.72355591,414.26710297)
\curveto(416.4935509,414.41709668)(416.17355122,414.4970966)(415.76355591,414.50710297)
\curveto(415.73355166,414.48709661)(415.6985517,414.48209662)(415.65855591,414.49210297)
\curveto(415.61855178,414.5020966)(415.58355181,414.5020966)(415.55355591,414.49210297)
\curveto(415.50355189,414.47209663)(415.44855195,414.45709664)(415.38855591,414.44710297)
\curveto(415.32855207,414.44709665)(415.27355212,414.43709666)(415.22355591,414.41710297)
\curveto(414.78355261,414.27709682)(414.45855294,414.0020971)(414.24855591,413.59210297)
\curveto(414.22855317,413.55209755)(414.20355319,413.4970976)(414.17355591,413.42710297)
\curveto(414.15355324,413.36709773)(414.13855326,413.3020978)(414.12855591,413.23210297)
\curveto(414.11855328,413.17209793)(414.11855328,413.11209799)(414.12855591,413.05210297)
\curveto(414.14855325,412.99209811)(414.18355321,412.94209816)(414.23355591,412.90210297)
\curveto(414.31355308,412.85209825)(414.42355297,412.82709827)(414.56355591,412.82710297)
\lineto(414.96855591,412.82710297)
\lineto(416.63355591,412.82710297)
\lineto(417.06855591,412.82710297)
\curveto(417.22855017,412.83709826)(417.33355006,412.88209822)(417.38355591,412.96210297)
}
}
{
\newrgbcolor{curcolor}{0 0 0}
\pscustom[linestyle=none,fillstyle=solid,fillcolor=curcolor]
{
\newpath
\moveto(423.85183716,416.08210297)
\curveto(424.60183266,416.102095)(425.25183201,416.01709508)(425.80183716,415.82710297)
\curveto(426.3618309,415.64709545)(426.78683047,415.33209577)(427.07683716,414.88210297)
\curveto(427.14683011,414.77209633)(427.20683005,414.65709644)(427.25683716,414.53710297)
\curveto(427.31682994,414.42709667)(427.36682989,414.3020968)(427.40683716,414.16210297)
\curveto(427.42682983,414.102097)(427.43682982,414.03709706)(427.43683716,413.96710297)
\curveto(427.43682982,413.8970972)(427.42682983,413.83709726)(427.40683716,413.78710297)
\curveto(427.36682989,413.72709737)(427.31182995,413.68709741)(427.24183716,413.66710297)
\curveto(427.19183007,413.64709745)(427.13183013,413.63709746)(427.06183716,413.63710297)
\lineto(426.85183716,413.63710297)
\lineto(426.19183716,413.63710297)
\curveto(426.12183114,413.63709746)(426.05183121,413.63209747)(425.98183716,413.62210297)
\curveto(425.91183135,413.62209748)(425.84683141,413.63209747)(425.78683716,413.65210297)
\curveto(425.68683157,413.67209743)(425.61183165,413.71209739)(425.56183716,413.77210297)
\curveto(425.51183175,413.83209727)(425.46683179,413.89209721)(425.42683716,413.95210297)
\lineto(425.30683716,414.16210297)
\curveto(425.27683198,414.24209686)(425.22683203,414.30709679)(425.15683716,414.35710297)
\curveto(425.0568322,414.43709666)(424.9568323,414.4970966)(424.85683716,414.53710297)
\curveto(424.76683249,414.57709652)(424.65183261,414.61209649)(424.51183716,414.64210297)
\curveto(424.44183282,414.66209644)(424.33683292,414.67709642)(424.19683716,414.68710297)
\curveto(424.06683319,414.6970964)(423.96683329,414.69209641)(423.89683716,414.67210297)
\lineto(423.79183716,414.67210297)
\lineto(423.64183716,414.64210297)
\curveto(423.60183366,414.64209646)(423.5568337,414.63709646)(423.50683716,414.62710297)
\curveto(423.33683392,414.57709652)(423.19683406,414.50709659)(423.08683716,414.41710297)
\curveto(422.98683427,414.33709676)(422.91683434,414.21209689)(422.87683716,414.04210297)
\curveto(422.8568344,413.97209713)(422.8568344,413.90709719)(422.87683716,413.84710297)
\curveto(422.89683436,413.78709731)(422.91683434,413.73709736)(422.93683716,413.69710297)
\curveto(423.00683425,413.57709752)(423.08683417,413.48209762)(423.17683716,413.41210297)
\curveto(423.27683398,413.34209776)(423.39183387,413.28209782)(423.52183716,413.23210297)
\curveto(423.71183355,413.15209795)(423.91683334,413.08209802)(424.13683716,413.02210297)
\lineto(424.82683716,412.87210297)
\curveto(425.06683219,412.83209827)(425.29683196,412.78209832)(425.51683716,412.72210297)
\curveto(425.74683151,412.67209843)(425.9618313,412.60709849)(426.16183716,412.52710297)
\curveto(426.25183101,412.48709861)(426.33683092,412.45209865)(426.41683716,412.42210297)
\curveto(426.50683075,412.4020987)(426.59183067,412.36709873)(426.67183716,412.31710297)
\curveto(426.8618304,412.1970989)(427.03183023,412.06709903)(427.18183716,411.92710297)
\curveto(427.34182992,411.78709931)(427.46682979,411.61209949)(427.55683716,411.40210297)
\curveto(427.58682967,411.33209977)(427.61182965,411.26209984)(427.63183716,411.19210297)
\curveto(427.65182961,411.12209998)(427.67182959,411.04710005)(427.69183716,410.96710297)
\curveto(427.70182956,410.90710019)(427.70682955,410.81210029)(427.70683716,410.68210297)
\curveto(427.71682954,410.56210054)(427.71682954,410.46710063)(427.70683716,410.39710297)
\lineto(427.70683716,410.32210297)
\curveto(427.68682957,410.26210084)(427.67182959,410.2021009)(427.66183716,410.14210297)
\curveto(427.6618296,410.09210101)(427.6568296,410.04210106)(427.64683716,409.99210297)
\curveto(427.57682968,409.69210141)(427.46682979,409.42710167)(427.31683716,409.19710297)
\curveto(427.1568301,408.95710214)(426.9618303,408.76210234)(426.73183716,408.61210297)
\curveto(426.50183076,408.46210264)(426.24183102,408.33210277)(425.95183716,408.22210297)
\curveto(425.84183142,408.17210293)(425.72183154,408.13710296)(425.59183716,408.11710297)
\curveto(425.47183179,408.097103)(425.35183191,408.07210303)(425.23183716,408.04210297)
\curveto(425.14183212,408.02210308)(425.04683221,408.01210309)(424.94683716,408.01210297)
\curveto(424.8568324,408.0021031)(424.76683249,407.98710311)(424.67683716,407.96710297)
\lineto(424.40683716,407.96710297)
\curveto(424.34683291,407.94710315)(424.24183302,407.93710316)(424.09183716,407.93710297)
\curveto(423.95183331,407.93710316)(423.85183341,407.94710315)(423.79183716,407.96710297)
\curveto(423.7618335,407.96710313)(423.72683353,407.97210313)(423.68683716,407.98210297)
\lineto(423.58183716,407.98210297)
\curveto(423.4618338,408.0021031)(423.34183392,408.01710308)(423.22183716,408.02710297)
\curveto(423.10183416,408.03710306)(422.98683427,408.05710304)(422.87683716,408.08710297)
\curveto(422.48683477,408.1971029)(422.14183512,408.32210278)(421.84183716,408.46210297)
\curveto(421.54183572,408.61210249)(421.28683597,408.83210227)(421.07683716,409.12210297)
\curveto(420.93683632,409.31210179)(420.81683644,409.53210157)(420.71683716,409.78210297)
\curveto(420.69683656,409.84210126)(420.67683658,409.92210118)(420.65683716,410.02210297)
\curveto(420.63683662,410.07210103)(420.62183664,410.14210096)(420.61183716,410.23210297)
\curveto(420.60183666,410.32210078)(420.60683665,410.3971007)(420.62683716,410.45710297)
\curveto(420.6568366,410.52710057)(420.70683655,410.57710052)(420.77683716,410.60710297)
\curveto(420.82683643,410.62710047)(420.88683637,410.63710046)(420.95683716,410.63710297)
\lineto(421.18183716,410.63710297)
\lineto(421.88683716,410.63710297)
\lineto(422.12683716,410.63710297)
\curveto(422.20683505,410.63710046)(422.27683498,410.62710047)(422.33683716,410.60710297)
\curveto(422.44683481,410.56710053)(422.51683474,410.5021006)(422.54683716,410.41210297)
\curveto(422.58683467,410.32210078)(422.63183463,410.22710087)(422.68183716,410.12710297)
\curveto(422.70183456,410.07710102)(422.73683452,410.01210109)(422.78683716,409.93210297)
\curveto(422.84683441,409.85210125)(422.89683436,409.8021013)(422.93683716,409.78210297)
\curveto(423.0568342,409.68210142)(423.17183409,409.6021015)(423.28183716,409.54210297)
\curveto(423.39183387,409.49210161)(423.53183373,409.44210166)(423.70183716,409.39210297)
\curveto(423.75183351,409.37210173)(423.80183346,409.36210174)(423.85183716,409.36210297)
\curveto(423.90183336,409.37210173)(423.95183331,409.37210173)(424.00183716,409.36210297)
\curveto(424.08183318,409.34210176)(424.16683309,409.33210177)(424.25683716,409.33210297)
\curveto(424.3568329,409.34210176)(424.44183282,409.35710174)(424.51183716,409.37710297)
\curveto(424.5618327,409.38710171)(424.60683265,409.39210171)(424.64683716,409.39210297)
\curveto(424.69683256,409.39210171)(424.74683251,409.4021017)(424.79683716,409.42210297)
\curveto(424.93683232,409.47210163)(425.0618322,409.53210157)(425.17183716,409.60210297)
\curveto(425.29183197,409.67210143)(425.38683187,409.76210134)(425.45683716,409.87210297)
\curveto(425.50683175,409.95210115)(425.54683171,410.07710102)(425.57683716,410.24710297)
\curveto(425.59683166,410.31710078)(425.59683166,410.38210072)(425.57683716,410.44210297)
\curveto(425.5568317,410.5021006)(425.53683172,410.55210055)(425.51683716,410.59210297)
\curveto(425.44683181,410.73210037)(425.3568319,410.83710026)(425.24683716,410.90710297)
\curveto(425.14683211,410.97710012)(425.02683223,411.04210006)(424.88683716,411.10210297)
\curveto(424.69683256,411.18209992)(424.49683276,411.24709985)(424.28683716,411.29710297)
\curveto(424.07683318,411.34709975)(423.86683339,411.4020997)(423.65683716,411.46210297)
\curveto(423.57683368,411.48209962)(423.49183377,411.4970996)(423.40183716,411.50710297)
\curveto(423.32183394,411.51709958)(423.24183402,411.53209957)(423.16183716,411.55210297)
\curveto(422.84183442,411.64209946)(422.53683472,411.72709937)(422.24683716,411.80710297)
\curveto(421.9568353,411.8970992)(421.69183557,412.02709907)(421.45183716,412.19710297)
\curveto(421.17183609,412.3970987)(420.96683629,412.66709843)(420.83683716,413.00710297)
\curveto(420.81683644,413.07709802)(420.79683646,413.17209793)(420.77683716,413.29210297)
\curveto(420.7568365,413.36209774)(420.74183652,413.44709765)(420.73183716,413.54710297)
\curveto(420.72183654,413.64709745)(420.72683653,413.73709736)(420.74683716,413.81710297)
\curveto(420.76683649,413.86709723)(420.77183649,413.90709719)(420.76183716,413.93710297)
\curveto(420.75183651,413.97709712)(420.7568365,414.02209708)(420.77683716,414.07210297)
\curveto(420.79683646,414.18209692)(420.81683644,414.28209682)(420.83683716,414.37210297)
\curveto(420.86683639,414.47209663)(420.90183636,414.56709653)(420.94183716,414.65710297)
\curveto(421.07183619,414.94709615)(421.25183601,415.18209592)(421.48183716,415.36210297)
\curveto(421.71183555,415.54209556)(421.97183529,415.68709541)(422.26183716,415.79710297)
\curveto(422.37183489,415.84709525)(422.48683477,415.88209522)(422.60683716,415.90210297)
\curveto(422.72683453,415.93209517)(422.85183441,415.96209514)(422.98183716,415.99210297)
\curveto(423.04183422,416.01209509)(423.10183416,416.02209508)(423.16183716,416.02210297)
\lineto(423.34183716,416.05210297)
\curveto(423.42183384,416.06209504)(423.50683375,416.06709503)(423.59683716,416.06710297)
\curveto(423.68683357,416.06709503)(423.77183349,416.07209503)(423.85183716,416.08210297)
}
}
{
\newrgbcolor{curcolor}{0 0 0}
\pscustom[linestyle=none,fillstyle=solid,fillcolor=curcolor]
{
\newpath
\moveto(436.82847778,412.09210297)
\curveto(436.8384691,412.03209907)(436.8434691,411.94209916)(436.84347778,411.82210297)
\curveto(436.8434691,411.7020994)(436.83346911,411.61709948)(436.81347778,411.56710297)
\lineto(436.81347778,411.37210297)
\curveto(436.78346916,411.26209984)(436.76346918,411.15709994)(436.75347778,411.05710297)
\curveto(436.75346919,410.95710014)(436.7384692,410.85710024)(436.70847778,410.75710297)
\curveto(436.68846925,410.66710043)(436.66846927,410.57210053)(436.64847778,410.47210297)
\curveto(436.62846931,410.38210072)(436.59846934,410.29210081)(436.55847778,410.20210297)
\curveto(436.48846945,410.03210107)(436.41846952,409.87210123)(436.34847778,409.72210297)
\curveto(436.27846966,409.58210152)(436.19846974,409.44210166)(436.10847778,409.30210297)
\curveto(436.04846989,409.21210189)(435.98346996,409.12710197)(435.91347778,409.04710297)
\curveto(435.85347009,408.97710212)(435.78347016,408.9021022)(435.70347778,408.82210297)
\lineto(435.59847778,408.71710297)
\curveto(435.54847039,408.66710243)(435.49347045,408.62210248)(435.43347778,408.58210297)
\lineto(435.28347778,408.46210297)
\curveto(435.20347074,408.4021027)(435.11347083,408.34710275)(435.01347778,408.29710297)
\curveto(434.92347102,408.25710284)(434.82847111,408.21210289)(434.72847778,408.16210297)
\curveto(434.62847131,408.11210299)(434.52347142,408.07710302)(434.41347778,408.05710297)
\curveto(434.31347163,408.03710306)(434.20847173,408.01710308)(434.09847778,407.99710297)
\curveto(434.0384719,407.97710312)(433.97347197,407.96710313)(433.90347778,407.96710297)
\curveto(433.8434721,407.96710313)(433.77847216,407.95710314)(433.70847778,407.93710297)
\lineto(433.57347778,407.93710297)
\curveto(433.49347245,407.91710318)(433.41847252,407.91710318)(433.34847778,407.93710297)
\lineto(433.19847778,407.93710297)
\curveto(433.1384728,407.95710314)(433.07347287,407.96710313)(433.00347778,407.96710297)
\curveto(432.943473,407.95710314)(432.88347306,407.96210314)(432.82347778,407.98210297)
\curveto(432.66347328,408.03210307)(432.50847343,408.07710302)(432.35847778,408.11710297)
\curveto(432.21847372,408.15710294)(432.08847385,408.21710288)(431.96847778,408.29710297)
\curveto(431.89847404,408.33710276)(431.83347411,408.37710272)(431.77347778,408.41710297)
\curveto(431.71347423,408.46710263)(431.64847429,408.51710258)(431.57847778,408.56710297)
\lineto(431.39847778,408.70210297)
\curveto(431.31847462,408.76210234)(431.24847469,408.76710233)(431.18847778,408.71710297)
\curveto(431.1384748,408.68710241)(431.11347483,408.64710245)(431.11347778,408.59710297)
\curveto(431.11347483,408.55710254)(431.10347484,408.50710259)(431.08347778,408.44710297)
\curveto(431.06347488,408.34710275)(431.05347489,408.23210287)(431.05347778,408.10210297)
\curveto(431.06347488,407.97210313)(431.06847487,407.85210325)(431.06847778,407.74210297)
\lineto(431.06847778,406.21210297)
\curveto(431.06847487,406.08210502)(431.06347488,405.95710514)(431.05347778,405.83710297)
\curveto(431.05347489,405.70710539)(431.02847491,405.6021055)(430.97847778,405.52210297)
\curveto(430.94847499,405.48210562)(430.89347505,405.45210565)(430.81347778,405.43210297)
\curveto(430.73347521,405.41210569)(430.6434753,405.4021057)(430.54347778,405.40210297)
\curveto(430.4434755,405.39210571)(430.3434756,405.39210571)(430.24347778,405.40210297)
\lineto(429.98847778,405.40210297)
\lineto(429.58347778,405.40210297)
\lineto(429.47847778,405.40210297)
\curveto(429.4384765,405.4021057)(429.40347654,405.40710569)(429.37347778,405.41710297)
\lineto(429.25347778,405.41710297)
\curveto(429.08347686,405.46710563)(428.99347695,405.56710553)(428.98347778,405.71710297)
\curveto(428.97347697,405.85710524)(428.96847697,406.02710507)(428.96847778,406.22710297)
\lineto(428.96847778,415.03210297)
\curveto(428.96847697,415.14209596)(428.96347698,415.25709584)(428.95347778,415.37710297)
\curveto(428.95347699,415.50709559)(428.97847696,415.60709549)(429.02847778,415.67710297)
\curveto(429.06847687,415.74709535)(429.12347682,415.79209531)(429.19347778,415.81210297)
\curveto(429.2434767,415.83209527)(429.30347664,415.84209526)(429.37347778,415.84210297)
\lineto(429.59847778,415.84210297)
\lineto(430.31847778,415.84210297)
\lineto(430.60347778,415.84210297)
\curveto(430.69347525,415.84209526)(430.76847517,415.81709528)(430.82847778,415.76710297)
\curveto(430.89847504,415.71709538)(430.93347501,415.65209545)(430.93347778,415.57210297)
\curveto(430.943475,415.5020956)(430.96847497,415.42709567)(431.00847778,415.34710297)
\curveto(431.01847492,415.31709578)(431.02847491,415.29209581)(431.03847778,415.27210297)
\curveto(431.05847488,415.26209584)(431.07847486,415.24709585)(431.09847778,415.22710297)
\curveto(431.20847473,415.21709588)(431.29847464,415.24709585)(431.36847778,415.31710297)
\curveto(431.4384745,415.38709571)(431.50847443,415.44709565)(431.57847778,415.49710297)
\curveto(431.70847423,415.58709551)(431.8434741,415.66709543)(431.98347778,415.73710297)
\curveto(432.12347382,415.81709528)(432.27847366,415.88209522)(432.44847778,415.93210297)
\curveto(432.52847341,415.96209514)(432.61347333,415.98209512)(432.70347778,415.99210297)
\curveto(432.80347314,416.0020951)(432.89847304,416.01709508)(432.98847778,416.03710297)
\curveto(433.02847291,416.04709505)(433.06847287,416.04709505)(433.10847778,416.03710297)
\curveto(433.15847278,416.02709507)(433.19847274,416.03209507)(433.22847778,416.05210297)
\curveto(433.79847214,416.07209503)(434.27847166,415.99209511)(434.66847778,415.81210297)
\curveto(435.06847087,415.64209546)(435.40847053,415.41709568)(435.68847778,415.13710297)
\curveto(435.7384702,415.08709601)(435.78347016,415.03709606)(435.82347778,414.98710297)
\curveto(435.86347008,414.94709615)(435.90347004,414.9020962)(435.94347778,414.85210297)
\curveto(436.01346993,414.76209634)(436.07346987,414.67209643)(436.12347778,414.58210297)
\curveto(436.18346976,414.49209661)(436.2384697,414.4020967)(436.28847778,414.31210297)
\curveto(436.30846963,414.29209681)(436.31846962,414.26709683)(436.31847778,414.23710297)
\curveto(436.32846961,414.20709689)(436.3434696,414.17209693)(436.36347778,414.13210297)
\curveto(436.42346952,414.03209707)(436.47846946,413.91209719)(436.52847778,413.77210297)
\curveto(436.54846939,413.71209739)(436.56846937,413.64709745)(436.58847778,413.57710297)
\curveto(436.60846933,413.51709758)(436.62846931,413.45209765)(436.64847778,413.38210297)
\curveto(436.68846925,413.26209784)(436.71346923,413.13709796)(436.72347778,413.00710297)
\curveto(436.7434692,412.87709822)(436.76846917,412.74209836)(436.79847778,412.60210297)
\lineto(436.79847778,412.43710297)
\lineto(436.82847778,412.25710297)
\lineto(436.82847778,412.09210297)
\moveto(434.71347778,411.74710297)
\curveto(434.72347122,411.7970993)(434.72847121,411.86209924)(434.72847778,411.94210297)
\curveto(434.72847121,412.03209907)(434.72347122,412.102099)(434.71347778,412.15210297)
\lineto(434.71347778,412.28710297)
\curveto(434.69347125,412.34709875)(434.68347126,412.41209869)(434.68347778,412.48210297)
\curveto(434.68347126,412.55209855)(434.67347127,412.62209848)(434.65347778,412.69210297)
\curveto(434.63347131,412.79209831)(434.61347133,412.88709821)(434.59347778,412.97710297)
\curveto(434.57347137,413.07709802)(434.5434714,413.16709793)(434.50347778,413.24710297)
\curveto(434.38347156,413.56709753)(434.22847171,413.82209728)(434.03847778,414.01210297)
\curveto(433.84847209,414.2020969)(433.57847236,414.34209676)(433.22847778,414.43210297)
\curveto(433.14847279,414.45209665)(433.05847288,414.46209664)(432.95847778,414.46210297)
\lineto(432.68847778,414.46210297)
\curveto(432.64847329,414.45209665)(432.61347333,414.44709665)(432.58347778,414.44710297)
\curveto(432.55347339,414.44709665)(432.51847342,414.44209666)(432.47847778,414.43210297)
\lineto(432.26847778,414.37210297)
\curveto(432.20847373,414.36209674)(432.14847379,414.34209676)(432.08847778,414.31210297)
\curveto(431.82847411,414.2020969)(431.62347432,414.03209707)(431.47347778,413.80210297)
\curveto(431.33347461,413.57209753)(431.21847472,413.31709778)(431.12847778,413.03710297)
\curveto(431.10847483,412.95709814)(431.09347485,412.87209823)(431.08347778,412.78210297)
\curveto(431.07347487,412.7020984)(431.05847488,412.62209848)(431.03847778,412.54210297)
\curveto(431.02847491,412.5020986)(431.02347492,412.43709866)(431.02347778,412.34710297)
\curveto(431.00347494,412.30709879)(430.99847494,412.25709884)(431.00847778,412.19710297)
\curveto(431.01847492,412.14709895)(431.01847492,412.097099)(431.00847778,412.04710297)
\curveto(430.98847495,411.98709911)(430.98847495,411.93209917)(431.00847778,411.88210297)
\lineto(431.00847778,411.70210297)
\lineto(431.00847778,411.56710297)
\curveto(431.00847493,411.52709957)(431.01847492,411.48709961)(431.03847778,411.44710297)
\curveto(431.0384749,411.37709972)(431.0434749,411.32209978)(431.05347778,411.28210297)
\lineto(431.08347778,411.10210297)
\curveto(431.09347485,411.04210006)(431.10847483,410.98210012)(431.12847778,410.92210297)
\curveto(431.21847472,410.63210047)(431.32347462,410.39210071)(431.44347778,410.20210297)
\curveto(431.57347437,410.02210108)(431.75347419,409.86210124)(431.98347778,409.72210297)
\curveto(432.12347382,409.64210146)(432.28847365,409.57710152)(432.47847778,409.52710297)
\curveto(432.51847342,409.51710158)(432.55347339,409.51210159)(432.58347778,409.51210297)
\curveto(432.61347333,409.52210158)(432.64847329,409.52210158)(432.68847778,409.51210297)
\curveto(432.72847321,409.5021016)(432.78847315,409.49210161)(432.86847778,409.48210297)
\curveto(432.94847299,409.48210162)(433.01347293,409.48710161)(433.06347778,409.49710297)
\curveto(433.1434728,409.51710158)(433.22347272,409.53210157)(433.30347778,409.54210297)
\curveto(433.39347255,409.56210154)(433.47847246,409.58710151)(433.55847778,409.61710297)
\curveto(433.79847214,409.71710138)(433.99347195,409.85710124)(434.14347778,410.03710297)
\curveto(434.29347165,410.21710088)(434.41847152,410.42710067)(434.51847778,410.66710297)
\curveto(434.56847137,410.78710031)(434.60347134,410.91210019)(434.62347778,411.04210297)
\curveto(434.6434713,411.17209993)(434.66847127,411.30709979)(434.69847778,411.44710297)
\lineto(434.69847778,411.59710297)
\curveto(434.70847123,411.64709945)(434.71347123,411.6970994)(434.71347778,411.74710297)
}
}
{
\newrgbcolor{curcolor}{0 0 0}
\pscustom[linestyle=none,fillstyle=solid,fillcolor=curcolor]
{
\newpath
\moveto(445.15839966,408.73210297)
\curveto(445.17839181,408.62210248)(445.1883918,408.51210259)(445.18839966,408.40210297)
\curveto(445.19839179,408.29210281)(445.14839184,408.21710288)(445.03839966,408.17710297)
\curveto(444.97839201,408.14710295)(444.90839208,408.13210297)(444.82839966,408.13210297)
\lineto(444.58839966,408.13210297)
\lineto(443.77839966,408.13210297)
\lineto(443.50839966,408.13210297)
\curveto(443.42839356,408.14210296)(443.36339362,408.16710293)(443.31339966,408.20710297)
\curveto(443.24339374,408.24710285)(443.1883938,408.3021028)(443.14839966,408.37210297)
\curveto(443.11839387,408.45210265)(443.07339391,408.51710258)(443.01339966,408.56710297)
\curveto(442.99339399,408.58710251)(442.96839402,408.6021025)(442.93839966,408.61210297)
\curveto(442.90839408,408.63210247)(442.86839412,408.63710246)(442.81839966,408.62710297)
\curveto(442.76839422,408.60710249)(442.71839427,408.58210252)(442.66839966,408.55210297)
\curveto(442.62839436,408.52210258)(442.5833944,408.4971026)(442.53339966,408.47710297)
\curveto(442.4833945,408.43710266)(442.42839456,408.4021027)(442.36839966,408.37210297)
\lineto(442.18839966,408.28210297)
\curveto(442.05839493,408.22210288)(441.92339506,408.17210293)(441.78339966,408.13210297)
\curveto(441.64339534,408.102103)(441.49839549,408.06710303)(441.34839966,408.02710297)
\curveto(441.27839571,408.00710309)(441.20839578,407.9971031)(441.13839966,407.99710297)
\curveto(441.07839591,407.98710311)(441.01339597,407.97710312)(440.94339966,407.96710297)
\lineto(440.85339966,407.96710297)
\curveto(440.82339616,407.95710314)(440.79339619,407.95210315)(440.76339966,407.95210297)
\lineto(440.59839966,407.95210297)
\curveto(440.49839649,407.93210317)(440.39839659,407.93210317)(440.29839966,407.95210297)
\lineto(440.16339966,407.95210297)
\curveto(440.09339689,407.97210313)(440.02339696,407.98210312)(439.95339966,407.98210297)
\curveto(439.89339709,407.97210313)(439.83339715,407.97710312)(439.77339966,407.99710297)
\curveto(439.67339731,408.01710308)(439.57839741,408.03710306)(439.48839966,408.05710297)
\curveto(439.39839759,408.06710303)(439.31339767,408.09210301)(439.23339966,408.13210297)
\curveto(438.94339804,408.24210286)(438.69339829,408.38210272)(438.48339966,408.55210297)
\curveto(438.2833987,408.73210237)(438.12339886,408.96710213)(438.00339966,409.25710297)
\curveto(437.97339901,409.32710177)(437.94339904,409.4021017)(437.91339966,409.48210297)
\curveto(437.89339909,409.56210154)(437.87339911,409.64710145)(437.85339966,409.73710297)
\curveto(437.83339915,409.78710131)(437.82339916,409.83710126)(437.82339966,409.88710297)
\curveto(437.83339915,409.93710116)(437.83339915,409.98710111)(437.82339966,410.03710297)
\curveto(437.81339917,410.06710103)(437.80339918,410.12710097)(437.79339966,410.21710297)
\curveto(437.79339919,410.31710078)(437.79839919,410.38710071)(437.80839966,410.42710297)
\curveto(437.82839916,410.52710057)(437.83839915,410.61210049)(437.83839966,410.68210297)
\lineto(437.92839966,411.01210297)
\curveto(437.95839903,411.13209997)(437.99839899,411.23709986)(438.04839966,411.32710297)
\curveto(438.21839877,411.61709948)(438.41339857,411.83709926)(438.63339966,411.98710297)
\curveto(438.85339813,412.13709896)(439.13339785,412.26709883)(439.47339966,412.37710297)
\curveto(439.60339738,412.42709867)(439.73839725,412.46209864)(439.87839966,412.48210297)
\curveto(440.01839697,412.5020986)(440.15839683,412.52709857)(440.29839966,412.55710297)
\curveto(440.37839661,412.57709852)(440.46339652,412.58709851)(440.55339966,412.58710297)
\curveto(440.64339634,412.5970985)(440.73339625,412.61209849)(440.82339966,412.63210297)
\curveto(440.89339609,412.65209845)(440.96339602,412.65709844)(441.03339966,412.64710297)
\curveto(441.10339588,412.64709845)(441.17839581,412.65709844)(441.25839966,412.67710297)
\curveto(441.32839566,412.6970984)(441.39839559,412.70709839)(441.46839966,412.70710297)
\curveto(441.53839545,412.70709839)(441.61339537,412.71709838)(441.69339966,412.73710297)
\curveto(441.90339508,412.78709831)(442.09339489,412.82709827)(442.26339966,412.85710297)
\curveto(442.44339454,412.8970982)(442.60339438,412.98709811)(442.74339966,413.12710297)
\curveto(442.83339415,413.21709788)(442.89339409,413.31709778)(442.92339966,413.42710297)
\curveto(442.93339405,413.45709764)(442.93339405,413.48209762)(442.92339966,413.50210297)
\curveto(442.92339406,413.52209758)(442.92839406,413.54209756)(442.93839966,413.56210297)
\curveto(442.94839404,413.58209752)(442.95339403,413.61209749)(442.95339966,413.65210297)
\lineto(442.95339966,413.74210297)
\lineto(442.92339966,413.86210297)
\curveto(442.92339406,413.9020972)(442.91839407,413.93709716)(442.90839966,413.96710297)
\curveto(442.80839418,414.26709683)(442.59839439,414.47209663)(442.27839966,414.58210297)
\curveto(442.1883948,414.61209649)(442.07839491,414.63209647)(441.94839966,414.64210297)
\curveto(441.82839516,414.66209644)(441.70339528,414.66709643)(441.57339966,414.65710297)
\curveto(441.44339554,414.65709644)(441.31839567,414.64709645)(441.19839966,414.62710297)
\curveto(441.07839591,414.60709649)(440.97339601,414.58209652)(440.88339966,414.55210297)
\curveto(440.82339616,414.53209657)(440.76339622,414.5020966)(440.70339966,414.46210297)
\curveto(440.65339633,414.43209667)(440.60339638,414.3970967)(440.55339966,414.35710297)
\curveto(440.50339648,414.31709678)(440.44839654,414.26209684)(440.38839966,414.19210297)
\curveto(440.33839665,414.12209698)(440.30339668,414.05709704)(440.28339966,413.99710297)
\curveto(440.23339675,413.8970972)(440.1883968,413.8020973)(440.14839966,413.71210297)
\curveto(440.11839687,413.62209748)(440.04839694,413.56209754)(439.93839966,413.53210297)
\curveto(439.85839713,413.51209759)(439.77339721,413.5020976)(439.68339966,413.50210297)
\lineto(439.41339966,413.50210297)
\lineto(438.84339966,413.50210297)
\curveto(438.79339819,413.5020976)(438.74339824,413.4970976)(438.69339966,413.48710297)
\curveto(438.64339834,413.48709761)(438.59839839,413.49209761)(438.55839966,413.50210297)
\lineto(438.42339966,413.50210297)
\curveto(438.40339858,413.51209759)(438.37839861,413.51709758)(438.34839966,413.51710297)
\curveto(438.31839867,413.51709758)(438.29339869,413.52709757)(438.27339966,413.54710297)
\curveto(438.19339879,413.56709753)(438.13839885,413.63209747)(438.10839966,413.74210297)
\curveto(438.09839889,413.79209731)(438.09839889,413.84209726)(438.10839966,413.89210297)
\curveto(438.11839887,413.94209716)(438.12839886,413.98709711)(438.13839966,414.02710297)
\curveto(438.16839882,414.13709696)(438.19839879,414.23709686)(438.22839966,414.32710297)
\curveto(438.26839872,414.42709667)(438.31339867,414.51709658)(438.36339966,414.59710297)
\lineto(438.45339966,414.74710297)
\lineto(438.54339966,414.89710297)
\curveto(438.62339836,415.00709609)(438.72339826,415.11209599)(438.84339966,415.21210297)
\curveto(438.86339812,415.22209588)(438.89339809,415.24709585)(438.93339966,415.28710297)
\curveto(438.983398,415.32709577)(439.02839796,415.36209574)(439.06839966,415.39210297)
\curveto(439.10839788,415.42209568)(439.15339783,415.45209565)(439.20339966,415.48210297)
\curveto(439.37339761,415.59209551)(439.55339743,415.67709542)(439.74339966,415.73710297)
\curveto(439.93339705,415.80709529)(440.12839686,415.87209523)(440.32839966,415.93210297)
\curveto(440.44839654,415.96209514)(440.57339641,415.98209512)(440.70339966,415.99210297)
\curveto(440.83339615,416.0020951)(440.96339602,416.02209508)(441.09339966,416.05210297)
\curveto(441.13339585,416.06209504)(441.19339579,416.06209504)(441.27339966,416.05210297)
\curveto(441.36339562,416.04209506)(441.41839557,416.04709505)(441.43839966,416.06710297)
\curveto(441.84839514,416.07709502)(442.23839475,416.06209504)(442.60839966,416.02210297)
\curveto(442.988394,415.98209512)(443.32839366,415.90709519)(443.62839966,415.79710297)
\curveto(443.93839305,415.68709541)(444.20339278,415.53709556)(444.42339966,415.34710297)
\curveto(444.64339234,415.16709593)(444.81339217,414.93209617)(444.93339966,414.64210297)
\curveto(445.00339198,414.47209663)(445.04339194,414.27709682)(445.05339966,414.05710297)
\curveto(445.06339192,413.83709726)(445.06839192,413.61209749)(445.06839966,413.38210297)
\lineto(445.06839966,410.03710297)
\lineto(445.06839966,409.45210297)
\curveto(445.06839192,409.26210184)(445.0883919,409.08710201)(445.12839966,408.92710297)
\curveto(445.13839185,408.8971022)(445.14339184,408.86210224)(445.14339966,408.82210297)
\curveto(445.14339184,408.79210231)(445.14839184,408.76210234)(445.15839966,408.73210297)
\moveto(442.95339966,411.04210297)
\curveto(442.96339402,411.09210001)(442.96839402,411.14709995)(442.96839966,411.20710297)
\curveto(442.96839402,411.27709982)(442.96339402,411.33709976)(442.95339966,411.38710297)
\curveto(442.93339405,411.44709965)(442.92339406,411.5020996)(442.92339966,411.55210297)
\curveto(442.92339406,411.6020995)(442.90339408,411.64209946)(442.86339966,411.67210297)
\curveto(442.81339417,411.71209939)(442.73839425,411.73209937)(442.63839966,411.73210297)
\curveto(442.59839439,411.72209938)(442.56339442,411.71209939)(442.53339966,411.70210297)
\curveto(442.50339448,411.7020994)(442.46839452,411.6970994)(442.42839966,411.68710297)
\curveto(442.35839463,411.66709943)(442.2833947,411.65209945)(442.20339966,411.64210297)
\curveto(442.12339486,411.63209947)(442.04339494,411.61709948)(441.96339966,411.59710297)
\curveto(441.93339505,411.58709951)(441.8883951,411.58209952)(441.82839966,411.58210297)
\curveto(441.69839529,411.55209955)(441.56839542,411.53209957)(441.43839966,411.52210297)
\curveto(441.30839568,411.51209959)(441.1833958,411.48709961)(441.06339966,411.44710297)
\curveto(440.983396,411.42709967)(440.90839608,411.40709969)(440.83839966,411.38710297)
\curveto(440.76839622,411.37709972)(440.69839629,411.35709974)(440.62839966,411.32710297)
\curveto(440.41839657,411.23709986)(440.23839675,411.1021)(440.08839966,410.92210297)
\curveto(439.94839704,410.74210036)(439.89839709,410.49210061)(439.93839966,410.17210297)
\curveto(439.95839703,410.0021011)(440.01339697,409.86210124)(440.10339966,409.75210297)
\curveto(440.17339681,409.64210146)(440.27839671,409.55210155)(440.41839966,409.48210297)
\curveto(440.55839643,409.42210168)(440.70839628,409.37710172)(440.86839966,409.34710297)
\curveto(441.03839595,409.31710178)(441.21339577,409.30710179)(441.39339966,409.31710297)
\curveto(441.5833954,409.33710176)(441.75839523,409.37210173)(441.91839966,409.42210297)
\curveto(442.17839481,409.5021016)(442.3833946,409.62710147)(442.53339966,409.79710297)
\curveto(442.6833943,409.97710112)(442.79839419,410.1971009)(442.87839966,410.45710297)
\curveto(442.89839409,410.52710057)(442.90839408,410.5971005)(442.90839966,410.66710297)
\curveto(442.91839407,410.74710035)(442.93339405,410.82710027)(442.95339966,410.90710297)
\lineto(442.95339966,411.04210297)
}
}
{
\newrgbcolor{curcolor}{0 0 0}
\pscustom[linestyle=none,fillstyle=solid,fillcolor=curcolor]
{
\newpath
\moveto(450.29168091,416.08210297)
\curveto(451.10167575,416.102095)(451.77667507,415.98209512)(452.31668091,415.72210297)
\curveto(452.86667398,415.46209564)(453.30167355,415.09209601)(453.62168091,414.61210297)
\curveto(453.78167307,414.37209673)(453.90167295,414.097097)(453.98168091,413.78710297)
\curveto(454.00167285,413.73709736)(454.01667283,413.67209743)(454.02668091,413.59210297)
\curveto(454.0466728,413.51209759)(454.0466728,413.44209766)(454.02668091,413.38210297)
\curveto(453.98667286,413.27209783)(453.91667293,413.20709789)(453.81668091,413.18710297)
\curveto(453.71667313,413.17709792)(453.59667325,413.17209793)(453.45668091,413.17210297)
\lineto(452.67668091,413.17210297)
\lineto(452.39168091,413.17210297)
\curveto(452.30167455,413.17209793)(452.22667462,413.19209791)(452.16668091,413.23210297)
\curveto(452.08667476,413.27209783)(452.03167482,413.33209777)(452.00168091,413.41210297)
\curveto(451.97167488,413.5020976)(451.93167492,413.59209751)(451.88168091,413.68210297)
\curveto(451.82167503,413.79209731)(451.75667509,413.89209721)(451.68668091,413.98210297)
\curveto(451.61667523,414.07209703)(451.53667531,414.15209695)(451.44668091,414.22210297)
\curveto(451.30667554,414.31209679)(451.1516757,414.38209672)(450.98168091,414.43210297)
\curveto(450.92167593,414.45209665)(450.86167599,414.46209664)(450.80168091,414.46210297)
\curveto(450.74167611,414.46209664)(450.68667616,414.47209663)(450.63668091,414.49210297)
\lineto(450.48668091,414.49210297)
\curveto(450.28667656,414.49209661)(450.12667672,414.47209663)(450.00668091,414.43210297)
\curveto(449.71667713,414.34209676)(449.48167737,414.2020969)(449.30168091,414.01210297)
\curveto(449.12167773,413.83209727)(448.97667787,413.61209749)(448.86668091,413.35210297)
\curveto(448.81667803,413.24209786)(448.77667807,413.12209798)(448.74668091,412.99210297)
\curveto(448.72667812,412.87209823)(448.70167815,412.74209836)(448.67168091,412.60210297)
\curveto(448.66167819,412.56209854)(448.65667819,412.52209858)(448.65668091,412.48210297)
\curveto(448.65667819,412.44209866)(448.6516782,412.4020987)(448.64168091,412.36210297)
\curveto(448.62167823,412.26209884)(448.61167824,412.12209898)(448.61168091,411.94210297)
\curveto(448.62167823,411.76209934)(448.63667821,411.62209948)(448.65668091,411.52210297)
\curveto(448.65667819,411.44209966)(448.66167819,411.38709971)(448.67168091,411.35710297)
\curveto(448.69167816,411.28709981)(448.70167815,411.21709988)(448.70168091,411.14710297)
\curveto(448.71167814,411.07710002)(448.72667812,411.00710009)(448.74668091,410.93710297)
\curveto(448.82667802,410.70710039)(448.92167793,410.4971006)(449.03168091,410.30710297)
\curveto(449.14167771,410.11710098)(449.28167757,409.95710114)(449.45168091,409.82710297)
\curveto(449.49167736,409.7971013)(449.5516773,409.76210134)(449.63168091,409.72210297)
\curveto(449.74167711,409.65210145)(449.851677,409.60710149)(449.96168091,409.58710297)
\curveto(450.08167677,409.56710153)(450.22667662,409.54710155)(450.39668091,409.52710297)
\lineto(450.48668091,409.52710297)
\curveto(450.52667632,409.52710157)(450.55667629,409.53210157)(450.57668091,409.54210297)
\lineto(450.71168091,409.54210297)
\curveto(450.78167607,409.56210154)(450.846676,409.57710152)(450.90668091,409.58710297)
\curveto(450.97667587,409.60710149)(451.04167581,409.62710147)(451.10168091,409.64710297)
\curveto(451.40167545,409.77710132)(451.63167522,409.96710113)(451.79168091,410.21710297)
\curveto(451.83167502,410.26710083)(451.86667498,410.32210078)(451.89668091,410.38210297)
\curveto(451.92667492,410.45210065)(451.9516749,410.51210059)(451.97168091,410.56210297)
\curveto(452.01167484,410.67210043)(452.0466748,410.76710033)(452.07668091,410.84710297)
\curveto(452.10667474,410.93710016)(452.17667467,411.00710009)(452.28668091,411.05710297)
\curveto(452.37667447,411.0971)(452.52167433,411.11209999)(452.72168091,411.10210297)
\lineto(453.21668091,411.10210297)
\lineto(453.42668091,411.10210297)
\curveto(453.50667334,411.11209999)(453.57167328,411.10709999)(453.62168091,411.08710297)
\lineto(453.74168091,411.08710297)
\lineto(453.86168091,411.05710297)
\curveto(453.90167295,411.05710004)(453.93167292,411.04710005)(453.95168091,411.02710297)
\curveto(454.00167285,410.98710011)(454.03167282,410.92710017)(454.04168091,410.84710297)
\curveto(454.06167279,410.77710032)(454.06167279,410.7021004)(454.04168091,410.62210297)
\curveto(453.9516729,410.29210081)(453.84167301,409.9971011)(453.71168091,409.73710297)
\curveto(453.30167355,408.96710213)(452.6466742,408.43210267)(451.74668091,408.13210297)
\curveto(451.6466752,408.102103)(451.54167531,408.08210302)(451.43168091,408.07210297)
\curveto(451.32167553,408.05210305)(451.21167564,408.02710307)(451.10168091,407.99710297)
\curveto(451.04167581,407.98710311)(450.98167587,407.98210312)(450.92168091,407.98210297)
\curveto(450.86167599,407.98210312)(450.80167605,407.97710312)(450.74168091,407.96710297)
\lineto(450.57668091,407.96710297)
\curveto(450.52667632,407.94710315)(450.4516764,407.94210316)(450.35168091,407.95210297)
\curveto(450.2516766,407.95210315)(450.17667667,407.95710314)(450.12668091,407.96710297)
\curveto(450.0466768,407.98710311)(449.97167688,407.9971031)(449.90168091,407.99710297)
\curveto(449.84167701,407.98710311)(449.77667707,407.99210311)(449.70668091,408.01210297)
\lineto(449.55668091,408.04210297)
\curveto(449.50667734,408.04210306)(449.45667739,408.04710305)(449.40668091,408.05710297)
\curveto(449.29667755,408.08710301)(449.19167766,408.11710298)(449.09168091,408.14710297)
\curveto(448.99167786,408.17710292)(448.89667795,408.21210289)(448.80668091,408.25210297)
\curveto(448.33667851,408.45210265)(447.94167891,408.70710239)(447.62168091,409.01710297)
\curveto(447.30167955,409.33710176)(447.04167981,409.73210137)(446.84168091,410.20210297)
\curveto(446.79168006,410.29210081)(446.7516801,410.38710071)(446.72168091,410.48710297)
\lineto(446.63168091,410.81710297)
\curveto(446.62168023,410.85710024)(446.61668023,410.89210021)(446.61668091,410.92210297)
\curveto(446.61668023,410.96210014)(446.60668024,411.00710009)(446.58668091,411.05710297)
\curveto(446.56668028,411.12709997)(446.55668029,411.1970999)(446.55668091,411.26710297)
\curveto(446.55668029,411.34709975)(446.5466803,411.42209968)(446.52668091,411.49210297)
\lineto(446.52668091,411.74710297)
\curveto(446.50668034,411.7970993)(446.49668035,411.85209925)(446.49668091,411.91210297)
\curveto(446.49668035,411.98209912)(446.50668034,412.04209906)(446.52668091,412.09210297)
\curveto(446.53668031,412.14209896)(446.53668031,412.18709891)(446.52668091,412.22710297)
\curveto(446.51668033,412.26709883)(446.51668033,412.30709879)(446.52668091,412.34710297)
\curveto(446.5466803,412.41709868)(446.5516803,412.48209862)(446.54168091,412.54210297)
\curveto(446.54168031,412.6020985)(446.5516803,412.66209844)(446.57168091,412.72210297)
\curveto(446.62168023,412.9020982)(446.66168019,413.07209803)(446.69168091,413.23210297)
\curveto(446.72168013,413.4020977)(446.76668008,413.56709753)(446.82668091,413.72710297)
\curveto(447.0466798,414.23709686)(447.32167953,414.66209644)(447.65168091,415.00210297)
\curveto(447.99167886,415.34209576)(448.42167843,415.61709548)(448.94168091,415.82710297)
\curveto(449.08167777,415.88709521)(449.22667762,415.92709517)(449.37668091,415.94710297)
\curveto(449.52667732,415.97709512)(449.68167717,416.01209509)(449.84168091,416.05210297)
\curveto(449.92167693,416.06209504)(449.99667685,416.06709503)(450.06668091,416.06710297)
\curveto(450.13667671,416.06709503)(450.21167664,416.07209503)(450.29168091,416.08210297)
}
}
{
\newrgbcolor{curcolor}{0 0 0}
\pscustom[linestyle=none,fillstyle=solid,fillcolor=curcolor]
{
\newpath
\moveto(457.43496216,418.72210297)
\curveto(457.50495921,418.64209246)(457.53995917,418.52209258)(457.53996216,418.36210297)
\lineto(457.53996216,417.89710297)
\lineto(457.53996216,417.49210297)
\curveto(457.53995917,417.35209375)(457.50495921,417.25709384)(457.43496216,417.20710297)
\curveto(457.37495934,417.15709394)(457.29495942,417.12709397)(457.19496216,417.11710297)
\curveto(457.10495961,417.10709399)(457.00495971,417.102094)(456.89496216,417.10210297)
\lineto(456.05496216,417.10210297)
\curveto(455.94496077,417.102094)(455.84496087,417.10709399)(455.75496216,417.11710297)
\curveto(455.67496104,417.12709397)(455.60496111,417.15709394)(455.54496216,417.20710297)
\curveto(455.50496121,417.23709386)(455.47496124,417.29209381)(455.45496216,417.37210297)
\curveto(455.44496127,417.46209364)(455.43496128,417.55709354)(455.42496216,417.65710297)
\lineto(455.42496216,417.98710297)
\curveto(455.43496128,418.097093)(455.43996127,418.19209291)(455.43996216,418.27210297)
\lineto(455.43996216,418.48210297)
\curveto(455.44996126,418.55209255)(455.46996124,418.61209249)(455.49996216,418.66210297)
\curveto(455.51996119,418.7020924)(455.54496117,418.73209237)(455.57496216,418.75210297)
\lineto(455.69496216,418.81210297)
\curveto(455.714961,418.81209229)(455.73996097,418.81209229)(455.76996216,418.81210297)
\curveto(455.79996091,418.82209228)(455.82496089,418.82709227)(455.84496216,418.82710297)
\lineto(456.93996216,418.82710297)
\curveto(457.03995967,418.82709227)(457.13495958,418.82209228)(457.22496216,418.81210297)
\curveto(457.3149594,418.8020923)(457.38495933,418.77209233)(457.43496216,418.72210297)
\moveto(457.53996216,408.95710297)
\curveto(457.53995917,408.75710234)(457.53495918,408.58710251)(457.52496216,408.44710297)
\curveto(457.5149592,408.30710279)(457.42495929,408.21210289)(457.25496216,408.16210297)
\curveto(457.19495952,408.14210296)(457.12995958,408.13210297)(457.05996216,408.13210297)
\curveto(456.98995972,408.14210296)(456.9149598,408.14710295)(456.83496216,408.14710297)
\lineto(455.99496216,408.14710297)
\curveto(455.90496081,408.14710295)(455.8149609,408.15210295)(455.72496216,408.16210297)
\curveto(455.64496107,408.17210293)(455.58496113,408.2021029)(455.54496216,408.25210297)
\curveto(455.48496123,408.32210278)(455.44996126,408.40710269)(455.43996216,408.50710297)
\lineto(455.43996216,408.85210297)
\lineto(455.43996216,415.18210297)
\lineto(455.43996216,415.48210297)
\curveto(455.43996127,415.58209552)(455.45996125,415.66209544)(455.49996216,415.72210297)
\curveto(455.55996115,415.79209531)(455.64496107,415.83709526)(455.75496216,415.85710297)
\curveto(455.77496094,415.86709523)(455.79996091,415.86709523)(455.82996216,415.85710297)
\curveto(455.86996084,415.85709524)(455.89996081,415.86209524)(455.91996216,415.87210297)
\lineto(456.66996216,415.87210297)
\lineto(456.86496216,415.87210297)
\curveto(456.94495977,415.88209522)(457.0099597,415.88209522)(457.05996216,415.87210297)
\lineto(457.17996216,415.87210297)
\curveto(457.23995947,415.85209525)(457.29495942,415.83709526)(457.34496216,415.82710297)
\curveto(457.39495932,415.81709528)(457.43495928,415.78709531)(457.46496216,415.73710297)
\curveto(457.50495921,415.68709541)(457.52495919,415.61709548)(457.52496216,415.52710297)
\curveto(457.53495918,415.43709566)(457.53995917,415.34209576)(457.53996216,415.24210297)
\lineto(457.53996216,408.95710297)
}
}
{
\newrgbcolor{curcolor}{0 0 0}
\pscustom[linestyle=none,fillstyle=solid,fillcolor=curcolor]
{
\newpath
\moveto(466.97214966,412.31710297)
\curveto(466.99214109,412.25709884)(467.00214108,412.17209893)(467.00214966,412.06210297)
\curveto(467.00214108,411.95209915)(466.99214109,411.86709923)(466.97214966,411.80710297)
\lineto(466.97214966,411.65710297)
\curveto(466.95214113,411.57709952)(466.94214114,411.4970996)(466.94214966,411.41710297)
\curveto(466.95214113,411.33709976)(466.94714113,411.25709984)(466.92714966,411.17710297)
\curveto(466.90714117,411.10709999)(466.89214119,411.04210006)(466.88214966,410.98210297)
\curveto(466.87214121,410.92210018)(466.86214122,410.85710024)(466.85214966,410.78710297)
\curveto(466.81214127,410.67710042)(466.7771413,410.56210054)(466.74714966,410.44210297)
\curveto(466.71714136,410.33210077)(466.6771414,410.22710087)(466.62714966,410.12710297)
\curveto(466.41714166,409.64710145)(466.14214194,409.25710184)(465.80214966,408.95710297)
\curveto(465.46214262,408.65710244)(465.05214303,408.40710269)(464.57214966,408.20710297)
\curveto(464.45214363,408.15710294)(464.32714375,408.12210298)(464.19714966,408.10210297)
\curveto(464.077144,408.07210303)(463.95214413,408.04210306)(463.82214966,408.01210297)
\curveto(463.77214431,407.99210311)(463.71714436,407.98210312)(463.65714966,407.98210297)
\curveto(463.59714448,407.98210312)(463.54214454,407.97710312)(463.49214966,407.96710297)
\lineto(463.38714966,407.96710297)
\curveto(463.35714472,407.95710314)(463.32714475,407.95210315)(463.29714966,407.95210297)
\curveto(463.24714483,407.94210316)(463.16714491,407.93710316)(463.05714966,407.93710297)
\curveto(462.94714513,407.92710317)(462.86214522,407.93210317)(462.80214966,407.95210297)
\lineto(462.65214966,407.95210297)
\curveto(462.60214548,407.96210314)(462.54714553,407.96710313)(462.48714966,407.96710297)
\curveto(462.43714564,407.95710314)(462.38714569,407.96210314)(462.33714966,407.98210297)
\curveto(462.29714578,407.99210311)(462.25714582,407.9971031)(462.21714966,407.99710297)
\curveto(462.18714589,407.9971031)(462.14714593,408.0021031)(462.09714966,408.01210297)
\curveto(461.99714608,408.04210306)(461.89714618,408.06710303)(461.79714966,408.08710297)
\curveto(461.69714638,408.10710299)(461.60214648,408.13710296)(461.51214966,408.17710297)
\curveto(461.39214669,408.21710288)(461.2771468,408.25710284)(461.16714966,408.29710297)
\curveto(461.06714701,408.33710276)(460.96214712,408.38710271)(460.85214966,408.44710297)
\curveto(460.50214758,408.65710244)(460.20214788,408.9021022)(459.95214966,409.18210297)
\curveto(459.70214838,409.46210164)(459.49214859,409.7971013)(459.32214966,410.18710297)
\curveto(459.27214881,410.27710082)(459.23214885,410.37210073)(459.20214966,410.47210297)
\curveto(459.1821489,410.57210053)(459.15714892,410.67710042)(459.12714966,410.78710297)
\curveto(459.10714897,410.83710026)(459.09714898,410.88210022)(459.09714966,410.92210297)
\curveto(459.09714898,410.96210014)(459.08714899,411.00710009)(459.06714966,411.05710297)
\curveto(459.04714903,411.13709996)(459.03714904,411.21709988)(459.03714966,411.29710297)
\curveto(459.03714904,411.38709971)(459.02714905,411.47209963)(459.00714966,411.55210297)
\curveto(458.99714908,411.6020995)(458.99214909,411.64709945)(458.99214966,411.68710297)
\lineto(458.99214966,411.82210297)
\curveto(458.97214911,411.88209922)(458.96214912,411.96709913)(458.96214966,412.07710297)
\curveto(458.97214911,412.18709891)(458.98714909,412.27209883)(459.00714966,412.33210297)
\lineto(459.00714966,412.43710297)
\curveto(459.01714906,412.48709861)(459.01714906,412.53709856)(459.00714966,412.58710297)
\curveto(459.00714907,412.64709845)(459.01714906,412.7020984)(459.03714966,412.75210297)
\curveto(459.04714903,412.8020983)(459.05214903,412.84709825)(459.05214966,412.88710297)
\curveto(459.05214903,412.93709816)(459.06214902,412.98709811)(459.08214966,413.03710297)
\curveto(459.12214896,413.16709793)(459.15714892,413.29209781)(459.18714966,413.41210297)
\curveto(459.21714886,413.54209756)(459.25714882,413.66709743)(459.30714966,413.78710297)
\curveto(459.48714859,414.1970969)(459.70214838,414.53709656)(459.95214966,414.80710297)
\curveto(460.20214788,415.08709601)(460.50714757,415.34209576)(460.86714966,415.57210297)
\curveto(460.96714711,415.62209548)(461.07214701,415.66709543)(461.18214966,415.70710297)
\curveto(461.29214679,415.74709535)(461.40214668,415.79209531)(461.51214966,415.84210297)
\curveto(461.64214644,415.89209521)(461.7771463,415.92709517)(461.91714966,415.94710297)
\curveto(462.05714602,415.96709513)(462.20214588,415.9970951)(462.35214966,416.03710297)
\curveto(462.43214565,416.04709505)(462.50714557,416.05209505)(462.57714966,416.05210297)
\curveto(462.64714543,416.05209505)(462.71714536,416.05709504)(462.78714966,416.06710297)
\curveto(463.36714471,416.07709502)(463.86714421,416.01709508)(464.28714966,415.88710297)
\curveto(464.71714336,415.75709534)(465.09714298,415.57709552)(465.42714966,415.34710297)
\curveto(465.53714254,415.26709583)(465.64714243,415.17709592)(465.75714966,415.07710297)
\curveto(465.8771422,414.98709611)(465.9771421,414.88709621)(466.05714966,414.77710297)
\curveto(466.13714194,414.67709642)(466.20714187,414.57709652)(466.26714966,414.47710297)
\curveto(466.33714174,414.37709672)(466.40714167,414.27209683)(466.47714966,414.16210297)
\curveto(466.54714153,414.05209705)(466.60214148,413.93209717)(466.64214966,413.80210297)
\curveto(466.6821414,413.68209742)(466.72714135,413.55209755)(466.77714966,413.41210297)
\curveto(466.80714127,413.33209777)(466.83214125,413.24709785)(466.85214966,413.15710297)
\lineto(466.91214966,412.88710297)
\curveto(466.92214116,412.84709825)(466.92714115,412.80709829)(466.92714966,412.76710297)
\curveto(466.92714115,412.72709837)(466.93214115,412.68709841)(466.94214966,412.64710297)
\curveto(466.96214112,412.5970985)(466.96714111,412.54209856)(466.95714966,412.48210297)
\curveto(466.94714113,412.42209868)(466.95214113,412.36709873)(466.97214966,412.31710297)
\moveto(464.87214966,411.77710297)
\curveto(464.8821432,411.82709927)(464.88714319,411.8970992)(464.88714966,411.98710297)
\curveto(464.88714319,412.08709901)(464.8821432,412.16209894)(464.87214966,412.21210297)
\lineto(464.87214966,412.33210297)
\curveto(464.85214323,412.38209872)(464.84214324,412.43709866)(464.84214966,412.49710297)
\curveto(464.84214324,412.55709854)(464.83714324,412.61209849)(464.82714966,412.66210297)
\curveto(464.82714325,412.7020984)(464.82214326,412.73209837)(464.81214966,412.75210297)
\lineto(464.75214966,412.99210297)
\curveto(464.74214334,413.08209802)(464.72214336,413.16709793)(464.69214966,413.24710297)
\curveto(464.5821435,413.50709759)(464.45214363,413.72709737)(464.30214966,413.90710297)
\curveto(464.15214393,414.097097)(463.95214413,414.24709685)(463.70214966,414.35710297)
\curveto(463.64214444,414.37709672)(463.5821445,414.39209671)(463.52214966,414.40210297)
\curveto(463.46214462,414.42209668)(463.39714468,414.44209666)(463.32714966,414.46210297)
\curveto(463.24714483,414.48209662)(463.16214492,414.48709661)(463.07214966,414.47710297)
\lineto(462.80214966,414.47710297)
\curveto(462.77214531,414.45709664)(462.73714534,414.44709665)(462.69714966,414.44710297)
\curveto(462.65714542,414.45709664)(462.62214546,414.45709664)(462.59214966,414.44710297)
\lineto(462.38214966,414.38710297)
\curveto(462.32214576,414.37709672)(462.26714581,414.35709674)(462.21714966,414.32710297)
\curveto(461.96714611,414.21709688)(461.76214632,414.05709704)(461.60214966,413.84710297)
\curveto(461.45214663,413.64709745)(461.33214675,413.41209769)(461.24214966,413.14210297)
\curveto(461.21214687,413.04209806)(461.18714689,412.93709816)(461.16714966,412.82710297)
\curveto(461.15714692,412.71709838)(461.14214694,412.60709849)(461.12214966,412.49710297)
\curveto(461.11214697,412.44709865)(461.10714697,412.3970987)(461.10714966,412.34710297)
\lineto(461.10714966,412.19710297)
\curveto(461.08714699,412.12709897)(461.077147,412.02209908)(461.07714966,411.88210297)
\curveto(461.08714699,411.74209936)(461.10214698,411.63709946)(461.12214966,411.56710297)
\lineto(461.12214966,411.43210297)
\curveto(461.14214694,411.35209975)(461.15714692,411.27209983)(461.16714966,411.19210297)
\curveto(461.1771469,411.12209998)(461.19214689,411.04710005)(461.21214966,410.96710297)
\curveto(461.31214677,410.66710043)(461.41714666,410.42210068)(461.52714966,410.23210297)
\curveto(461.64714643,410.05210105)(461.83214625,409.88710121)(462.08214966,409.73710297)
\curveto(462.15214593,409.68710141)(462.22714585,409.64710145)(462.30714966,409.61710297)
\curveto(462.39714568,409.58710151)(462.48714559,409.56210154)(462.57714966,409.54210297)
\curveto(462.61714546,409.53210157)(462.65214543,409.52710157)(462.68214966,409.52710297)
\curveto(462.71214537,409.53710156)(462.74714533,409.53710156)(462.78714966,409.52710297)
\lineto(462.90714966,409.49710297)
\curveto(462.95714512,409.4971016)(463.00214508,409.5021016)(463.04214966,409.51210297)
\lineto(463.16214966,409.51210297)
\curveto(463.24214484,409.53210157)(463.32214476,409.54710155)(463.40214966,409.55710297)
\curveto(463.4821446,409.56710153)(463.55714452,409.58710151)(463.62714966,409.61710297)
\curveto(463.88714419,409.71710138)(464.09714398,409.85210125)(464.25714966,410.02210297)
\curveto(464.41714366,410.19210091)(464.55214353,410.4021007)(464.66214966,410.65210297)
\curveto(464.70214338,410.75210035)(464.73214335,410.85210025)(464.75214966,410.95210297)
\curveto(464.77214331,411.05210005)(464.79714328,411.15709994)(464.82714966,411.26710297)
\curveto(464.83714324,411.30709979)(464.84214324,411.34209976)(464.84214966,411.37210297)
\curveto(464.84214324,411.41209969)(464.84714323,411.45209965)(464.85714966,411.49210297)
\lineto(464.85714966,411.62710297)
\curveto(464.85714322,411.67709942)(464.86214322,411.72709937)(464.87214966,411.77710297)
}
}
{
\newrgbcolor{curcolor}{0 0 0}
\pscustom[linestyle=none,fillstyle=solid,fillcolor=curcolor]
{
\newpath
\moveto(12.5808255,206.61524658)
\curveto(12.5808362,206.75524277)(12.5808362,206.9202426)(12.5808255,207.11024658)
\curveto(12.57083621,207.31024221)(12.60583617,207.43024209)(12.6858255,207.47024658)
\curveto(12.775836,207.53024199)(12.91083587,207.54024198)(13.0908255,207.50024658)
\curveto(13.27083551,207.46024206)(13.44083534,207.4252421)(13.6008255,207.39524658)
\lineto(15.8508255,206.94524658)
\lineto(18.6408255,206.39024658)
\curveto(18.99082979,206.3202432)(19.33582944,206.25524327)(19.6758255,206.19524658)
\curveto(20.00582877,206.13524339)(20.30082848,206.1252434)(20.5608255,206.16524658)
\curveto(20.95082783,206.21524331)(21.27082751,206.33524319)(21.5208255,206.52524658)
\curveto(21.77082701,206.71524281)(21.9758268,206.98024254)(22.1358255,207.32024658)
\curveto(22.18582659,207.4202421)(22.22082656,207.53024199)(22.2408255,207.65024658)
\curveto(22.25082653,207.77024175)(22.27082651,207.88524164)(22.3008255,207.99524658)
\lineto(22.3308255,208.17524658)
\curveto(22.33082645,208.24524128)(22.33582644,208.30524122)(22.3458255,208.35524658)
\lineto(22.3458255,208.50524658)
\curveto(22.34582643,208.56524096)(22.35082643,208.6252409)(22.3608255,208.68524658)
\curveto(22.36082642,208.75524077)(22.35082643,208.8202407)(22.3308255,208.88024658)
\curveto(22.32082646,208.93024059)(22.32082646,208.98024054)(22.3308255,209.03024658)
\curveto(22.33082645,209.09024043)(22.32582645,209.14524038)(22.3158255,209.19524658)
\curveto(22.28582649,209.31524021)(22.26082652,209.43024009)(22.2408255,209.54024658)
\curveto(22.22082656,209.66023986)(22.18582659,209.78023974)(22.1358255,209.90024658)
\curveto(21.98582679,210.31023921)(21.79082699,210.64523888)(21.5508255,210.90524658)
\curveto(21.31082747,211.17523835)(20.99082779,211.41523811)(20.5908255,211.62524658)
\curveto(20.34082844,211.75523777)(20.06082872,211.85523767)(19.7508255,211.92524658)
\curveto(19.43082935,212.00523752)(19.10582967,212.08023744)(18.7758255,212.15024658)
\lineto(16.3158255,212.64524658)
\lineto(13.7208255,213.15524658)
\curveto(13.55083523,213.19523633)(13.36083542,213.23023629)(13.1508255,213.26024658)
\curveto(12.93083585,213.30023622)(12.775836,213.37023615)(12.6858255,213.47024658)
\curveto(12.61583616,213.54023598)(12.5808362,213.64023588)(12.5808255,213.77024658)
\curveto(12.5808362,213.91023561)(12.5808362,214.04523548)(12.5808255,214.17524658)
\curveto(12.5808362,214.2252353)(12.58583619,214.27023525)(12.5958255,214.31024658)
\curveto(12.59583618,214.35023517)(12.59583618,214.39523513)(12.5958255,214.44524658)
\curveto(12.62583615,214.58523494)(12.6758361,214.66523486)(12.7458255,214.68524658)
\curveto(12.82583595,214.7252348)(12.94083584,214.7252348)(13.0908255,214.68524658)
\curveto(13.24083554,214.65523487)(13.3758354,214.6252349)(13.4958255,214.59524658)
\lineto(15.5658255,214.19024658)
\lineto(18.6858255,213.56024658)
\curveto(19.05582972,213.49023603)(19.42082936,213.41023611)(19.7808255,213.32024658)
\curveto(20.14082864,213.24023628)(20.46582831,213.13523639)(20.7558255,213.00524658)
\curveto(21.24582753,212.76523676)(21.66582711,212.49523703)(22.0158255,212.19524658)
\curveto(22.35582642,211.90523762)(22.64582613,211.54523798)(22.8858255,211.11524658)
\curveto(22.9758258,210.94523858)(23.05582572,210.77023875)(23.1258255,210.59024658)
\curveto(23.19582558,210.41023911)(23.26082552,210.21523931)(23.3208255,210.00524658)
\curveto(23.35082543,209.91523961)(23.37082541,209.8202397)(23.3808255,209.72024658)
\curveto(23.40082538,209.63023989)(23.42082536,209.53523999)(23.4408255,209.43524658)
\curveto(23.46082532,209.3252402)(23.47082531,209.21524031)(23.4708255,209.10524658)
\curveto(23.47082531,209.00524052)(23.4808253,208.90524062)(23.5008255,208.80524658)
\lineto(23.5008255,208.62524658)
\curveto(23.51082527,208.57524095)(23.51582526,208.49524103)(23.5158255,208.38524658)
\curveto(23.51582526,208.27524125)(23.51082527,208.19524133)(23.5008255,208.14524658)
\lineto(23.5008255,207.96524658)
\curveto(23.4808253,207.89524163)(23.47082531,207.8202417)(23.4708255,207.74024658)
\curveto(23.4808253,207.67024185)(23.4758253,207.60524192)(23.4558255,207.54524658)
\lineto(23.4558255,207.42524658)
\curveto(23.43582534,207.34524218)(23.42082536,207.26524226)(23.4108255,207.18524658)
\curveto(23.40082538,207.11524241)(23.38582539,207.04524248)(23.3658255,206.97524658)
\curveto(23.31582546,206.81524271)(23.27082551,206.65524287)(23.2308255,206.49524658)
\curveto(23.1808256,206.34524318)(23.12082566,206.20024332)(23.0508255,206.06024658)
\curveto(23.02082576,206.00024352)(22.98582579,205.94024358)(22.9458255,205.88024658)
\curveto(22.90582587,205.8202437)(22.86082592,205.76024376)(22.8108255,205.70024658)
\curveto(22.62082616,205.44024408)(22.39582638,205.23524429)(22.1358255,205.08524658)
\curveto(21.8758269,204.93524459)(21.57082721,204.8252447)(21.2208255,204.75524658)
\curveto(21.07082771,204.7252448)(20.91582786,204.71024481)(20.7558255,204.71024658)
\curveto(20.59582818,204.7202448)(20.43082835,204.7202448)(20.2608255,204.71024658)
\curveto(20.1808286,204.7202448)(20.10582867,204.73024479)(20.0358255,204.74024658)
\curveto(19.95582882,204.75024477)(19.8758289,204.75524477)(19.7958255,204.75524658)
\lineto(19.6458255,204.78524658)
\curveto(19.56582921,204.78524474)(19.48582929,204.79524473)(19.4058255,204.81524658)
\curveto(19.31582946,204.84524468)(19.23082955,204.87024465)(19.1508255,204.89024658)
\lineto(18.1758255,205.08524658)
\lineto(14.2608255,205.86524658)
\lineto(13.2408255,206.07524658)
\curveto(13.15083563,206.09524343)(13.06083572,206.11024341)(12.9708255,206.12024658)
\curveto(12.8808359,206.14024338)(12.80583597,206.17524335)(12.7458255,206.22524658)
\curveto(12.6758361,206.28524324)(12.62583615,206.37024315)(12.5958255,206.48024658)
\curveto(12.59583618,206.54024298)(12.59083619,206.58524294)(12.5808255,206.61524658)
}
}
{
\newrgbcolor{curcolor}{0 0 0}
\pscustom[linestyle=none,fillstyle=solid,fillcolor=curcolor]
{
\newpath
\moveto(15.3858255,219.96009033)
\curveto(15.36583341,220.60008351)(15.45083333,221.09008302)(15.6408255,221.43009033)
\curveto(15.83083295,221.77008234)(16.11583266,222.0150821)(16.4958255,222.16509033)
\curveto(16.59583218,222.20508191)(16.70583207,222.23008188)(16.8258255,222.24009033)
\curveto(16.93583184,222.26008185)(17.05083173,222.27008184)(17.1708255,222.27009033)
\curveto(17.36083142,222.29008182)(17.56583121,222.28008183)(17.7858255,222.24009033)
\curveto(18.00583077,222.2100819)(18.23083055,222.17008194)(18.4608255,222.12009033)
\lineto(20.0658255,221.80509033)
\lineto(22.4058255,221.34009033)
\lineto(22.9158255,221.22009033)
\curveto(23.08582569,221.18008293)(23.19582558,221.09008302)(23.2458255,220.95009033)
\curveto(23.26582551,220.90008321)(23.2758255,220.84508327)(23.2758255,220.78509033)
\curveto(23.28582549,220.73508338)(23.29082549,220.68008343)(23.2908255,220.62009033)
\curveto(23.29082549,220.49008362)(23.28582549,220.36508375)(23.2758255,220.24509033)
\curveto(23.2758255,220.12508399)(23.23582554,220.05008406)(23.1558255,220.02009033)
\curveto(23.08582569,219.98008413)(22.99582578,219.97008414)(22.8858255,219.99009033)
\curveto(22.775826,220.0100841)(22.66582611,220.03508408)(22.5558255,220.06509033)
\lineto(21.2658255,220.32009033)
\lineto(18.8208255,220.80009033)
\curveto(18.55083023,220.86008325)(18.28583049,220.9100832)(18.0258255,220.95009033)
\curveto(17.75583102,220.99008312)(17.52583125,220.99008312)(17.3358255,220.95009033)
\curveto(17.13583164,220.9100832)(16.9758318,220.82008329)(16.8558255,220.68009033)
\curveto(16.72583205,220.55008356)(16.62583215,220.39008372)(16.5558255,220.20009033)
\curveto(16.53583224,220.14008397)(16.52583225,220.07508404)(16.5258255,220.00509033)
\curveto(16.51583226,219.94508417)(16.50083228,219.89008422)(16.4808255,219.84009033)
\curveto(16.47083231,219.79008432)(16.47083231,219.7100844)(16.4808255,219.60009033)
\curveto(16.4808323,219.50008461)(16.48583229,219.42508469)(16.4958255,219.37509033)
\curveto(16.51583226,219.33508478)(16.52583225,219.30008481)(16.5258255,219.27009033)
\curveto(16.51583226,219.24008487)(16.51583226,219.20508491)(16.5258255,219.16509033)
\curveto(16.55583222,219.02508509)(16.59083219,218.89508522)(16.6308255,218.77509033)
\curveto(16.66083212,218.65508546)(16.70583207,218.54008557)(16.7658255,218.43009033)
\curveto(16.78583199,218.38008573)(16.80583197,218.34008577)(16.8258255,218.31009033)
\curveto(16.84583193,218.28008583)(16.86583191,218.24008587)(16.8858255,218.19009033)
\curveto(17.13583164,217.79008632)(17.51083127,217.46008665)(18.0108255,217.20009033)
\curveto(18.09083069,217.16008695)(18.1758306,217.12508699)(18.2658255,217.09509033)
\lineto(18.5058255,217.00509033)
\curveto(18.55583022,216.97508714)(18.60583017,216.96008715)(18.6558255,216.96009033)
\curveto(18.69583008,216.96008715)(18.73583004,216.94508717)(18.7758255,216.91509033)
\lineto(19.0908255,216.85509033)
\curveto(19.12082966,216.83508728)(19.15582962,216.82508729)(19.1958255,216.82509033)
\curveto(19.23582954,216.82508729)(19.2808295,216.82008729)(19.3308255,216.81009033)
\lineto(19.7808255,216.72009033)
\lineto(21.2208255,216.42009033)
\lineto(22.5408255,216.16509033)
\curveto(22.65082613,216.14508797)(22.76582601,216.12008799)(22.8858255,216.09009033)
\curveto(22.99582578,216.07008804)(23.08582569,216.03008808)(23.1558255,215.97009033)
\curveto(23.23582554,215.90008821)(23.2758255,215.80008831)(23.2758255,215.67009033)
\curveto(23.28582549,215.55008856)(23.29082549,215.42508869)(23.2908255,215.29509033)
\curveto(23.29082549,215.2150889)(23.28582549,215.14008897)(23.2758255,215.07009033)
\curveto(23.26582551,215.00008911)(23.24082554,214.94508917)(23.2008255,214.90509033)
\curveto(23.15082563,214.83508928)(23.05582572,214.8150893)(22.9158255,214.84509033)
\curveto(22.775826,214.87508924)(22.64082614,214.90008921)(22.5108255,214.92009033)
\lineto(20.7408255,215.28009033)
\lineto(17.1108255,216.00009033)
\lineto(16.1958255,216.18009033)
\lineto(15.9258255,216.24009033)
\curveto(15.83583294,216.26008785)(15.76583301,216.29508782)(15.7158255,216.34509033)
\curveto(15.65583312,216.38508773)(15.61583316,216.44008767)(15.5958255,216.51009033)
\curveto(15.58583319,216.56008755)(15.5758332,216.62008749)(15.5658255,216.69009033)
\curveto(15.55583322,216.77008734)(15.55083323,216.85008726)(15.5508255,216.93009033)
\curveto(15.55083323,217.0100871)(15.55583322,217.08508703)(15.5658255,217.15509033)
\curveto(15.5758332,217.23508688)(15.59083319,217.28508683)(15.6108255,217.30509033)
\curveto(15.6808331,217.40508671)(15.77083301,217.44008667)(15.8808255,217.41009033)
\curveto(15.9808328,217.38008673)(16.09583268,217.37008674)(16.2258255,217.38009033)
\curveto(16.28583249,217.39008672)(16.33583244,217.42008669)(16.3758255,217.47009033)
\curveto(16.38583239,217.59008652)(16.34083244,217.69508642)(16.2408255,217.78509033)
\curveto(16.14083264,217.88508623)(16.06083272,217.98008613)(16.0008255,218.07009033)
\curveto(15.90083288,218.23008588)(15.81083297,218.39008572)(15.7308255,218.55009033)
\curveto(15.64083314,218.7100854)(15.56583321,218.89508522)(15.5058255,219.10509033)
\curveto(15.4758333,219.18508493)(15.45583332,219.27508484)(15.4458255,219.37509033)
\curveto(15.43583334,219.47508464)(15.42083336,219.57008454)(15.4008255,219.66009033)
\curveto(15.39083339,219.7100844)(15.38583339,219.76008435)(15.3858255,219.81009033)
\lineto(15.3858255,219.96009033)
}
}
{
\newrgbcolor{curcolor}{0 0 0}
\pscustom[linestyle=none,fillstyle=solid,fillcolor=curcolor]
{
\newpath
\moveto(14.0358255,225.17469971)
\curveto(13.9758348,225.10469673)(13.87083491,225.08469675)(13.7208255,225.11469971)
\curveto(13.56083522,225.14469669)(13.40583537,225.17469666)(13.2558255,225.20469971)
\curveto(13.1758356,225.21469662)(13.09083569,225.22969661)(13.0008255,225.24969971)
\curveto(12.91083587,225.26969657)(12.83583594,225.29969654)(12.7758255,225.33969971)
\curveto(12.69583608,225.39969644)(12.63583614,225.48969635)(12.5958255,225.60969971)
\curveto(12.58583619,225.6396962)(12.58583619,225.66469617)(12.5958255,225.68469971)
\curveto(12.59583618,225.70469613)(12.59083619,225.72969611)(12.5808255,225.75969971)
\curveto(12.5808362,225.92969591)(12.58583619,226.08469575)(12.5958255,226.22469971)
\curveto(12.60583617,226.37469546)(12.66583611,226.46469537)(12.7758255,226.49469971)
\curveto(12.83583594,226.51469532)(12.91083587,226.51469532)(13.0008255,226.49469971)
\curveto(13.0808357,226.47469536)(13.16583561,226.45969538)(13.2558255,226.44969971)
\curveto(13.43583534,226.40969543)(13.60583517,226.36969547)(13.7658255,226.32969971)
\curveto(13.92583485,226.29969554)(14.03083475,226.21469562)(14.0808255,226.07469971)
\curveto(14.10083468,226.01469582)(14.11083467,225.95469588)(14.1108255,225.89469971)
\lineto(14.1108255,225.72969971)
\lineto(14.1108255,225.41469971)
\curveto(14.11083467,225.31469652)(14.08583469,225.2346966)(14.0358255,225.17469971)
\moveto(22.5408255,224.58969971)
\curveto(22.64082614,224.56969727)(22.74582603,224.54969729)(22.8558255,224.52969971)
\curveto(22.95582582,224.51969732)(23.03582574,224.47969736)(23.0958255,224.40969971)
\curveto(23.15582562,224.36969747)(23.19582558,224.31969752)(23.2158255,224.25969971)
\curveto(23.22582555,224.19969764)(23.24082554,224.12469771)(23.2608255,224.03469971)
\lineto(23.2608255,223.80969971)
\curveto(23.26082552,223.67969816)(23.25582552,223.56969827)(23.2458255,223.47969971)
\curveto(23.22582555,223.38969845)(23.1758256,223.32469851)(23.0958255,223.28469971)
\curveto(23.03582574,223.26469857)(22.96082582,223.25969858)(22.8708255,223.26969971)
\curveto(22.77082601,223.28969855)(22.6758261,223.30969853)(22.5858255,223.32969971)
\lineto(16.2408255,224.60469971)
\curveto(16.13083265,224.62469721)(16.02583275,224.64469719)(15.9258255,224.66469971)
\curveto(15.81583296,224.68469715)(15.73083305,224.72469711)(15.6708255,224.78469971)
\curveto(15.62083316,224.82469701)(15.59083319,224.86969697)(15.5808255,224.91969971)
\curveto(15.57083321,224.97969686)(15.55583322,225.0396968)(15.5358255,225.09969971)
\curveto(15.53583324,225.11969672)(15.54083324,225.1396967)(15.5508255,225.15969971)
\curveto(15.55083323,225.18969665)(15.54583323,225.21469662)(15.5358255,225.23469971)
\curveto(15.53583324,225.36469647)(15.54083324,225.49469634)(15.5508255,225.62469971)
\curveto(15.55083323,225.76469607)(15.59083319,225.84969599)(15.6708255,225.87969971)
\curveto(15.73083305,225.91969592)(15.81083297,225.92969591)(15.9108255,225.90969971)
\curveto(16.00083278,225.88969595)(16.09583268,225.86969597)(16.1958255,225.84969971)
\lineto(22.5408255,224.58969971)
}
}
{
\newrgbcolor{curcolor}{0 0 0}
\pscustom[linestyle=none,fillstyle=solid,fillcolor=curcolor]
{
\newpath
\moveto(22.4508255,233.50954346)
\lineto(22.8408255,233.41954346)
\curveto(22.96082582,233.39953553)(23.06082572,233.35953557)(23.1408255,233.29954346)
\curveto(23.21082557,233.2295357)(23.25082553,233.13453579)(23.2608255,233.01454346)
\lineto(23.2608255,232.66954346)
\curveto(23.26082552,232.60953632)(23.26582551,232.54953638)(23.2758255,232.48954346)
\curveto(23.2758255,232.43953649)(23.26582551,232.39453653)(23.2458255,232.35454346)
\curveto(23.22582555,232.27453665)(23.18582559,232.2245367)(23.1258255,232.20454346)
\curveto(23.0758257,232.17453675)(23.01582576,232.16453676)(22.9458255,232.17454346)
\curveto(22.8758259,232.18453674)(22.80582597,232.17953675)(22.7358255,232.15954346)
\curveto(22.71582606,232.15953677)(22.70082608,232.14953678)(22.6908255,232.12954346)
\lineto(22.6308255,232.09954346)
\curveto(22.62082616,231.99953693)(22.64082614,231.91453701)(22.6908255,231.84454346)
\curveto(22.74082604,231.78453714)(22.79082599,231.71953721)(22.8408255,231.64954346)
\curveto(22.99082579,231.41953751)(23.10582567,231.19453773)(23.1858255,230.97454346)
\curveto(23.26582551,230.78453814)(23.32582545,230.56453836)(23.3658255,230.31454346)
\curveto(23.40582537,230.07453885)(23.42582535,229.8295391)(23.4258255,229.57954346)
\curveto(23.43582534,229.33953959)(23.42082536,229.09953983)(23.3808255,228.85954346)
\curveto(23.35082543,228.6295403)(23.29582548,228.43454049)(23.2158255,228.27454346)
\curveto(22.99582578,227.79454113)(22.70082608,227.4295415)(22.3308255,227.17954346)
\curveto(21.95082683,226.93954199)(21.4808273,226.78454214)(20.9208255,226.71454346)
\curveto(20.83082795,226.69454223)(20.74082804,226.68454224)(20.6508255,226.68454346)
\curveto(20.55082823,226.69454223)(20.45082833,226.69454223)(20.3508255,226.68454346)
\curveto(20.30082848,226.68454224)(20.25082853,226.68954224)(20.2008255,226.69954346)
\curveto(20.15082863,226.70954222)(20.10082868,226.71454221)(20.0508255,226.71454346)
\curveto(20.00082878,226.70454222)(19.95082883,226.70454222)(19.9008255,226.71454346)
\curveto(19.84082894,226.73454219)(19.78582899,226.74454218)(19.7358255,226.74454346)
\lineto(19.5858255,226.77454346)
\curveto(19.53582924,226.76454216)(19.47082931,226.76454216)(19.3908255,226.77454346)
\curveto(19.31082947,226.79454213)(19.24582953,226.81954211)(19.1958255,226.84954346)
\lineto(19.0308255,226.89454346)
\curveto(18.96082982,226.924542)(18.89082989,226.94454198)(18.8208255,226.95454346)
\curveto(18.74083004,226.96454196)(18.66583011,226.98454194)(18.5958255,227.01454346)
\curveto(18.54583023,227.03454189)(18.50083028,227.04954188)(18.4608255,227.05954346)
\curveto(18.42083036,227.06954186)(18.3758304,227.08454184)(18.3258255,227.10454346)
\curveto(18.22583055,227.15454177)(18.13083065,227.19954173)(18.0408255,227.23954346)
\curveto(17.94083084,227.27954165)(17.84583093,227.3245416)(17.7558255,227.37454346)
\curveto(17.3758314,227.57454135)(17.03583174,227.80454112)(16.7358255,228.06454346)
\curveto(16.42583235,228.33454059)(16.17083261,228.63454029)(15.9708255,228.96454346)
\curveto(15.85083293,229.16453976)(15.75083303,229.36453956)(15.6708255,229.56454346)
\curveto(15.59083319,229.76453916)(15.52083326,229.97953895)(15.4608255,230.20954346)
\lineto(15.4308255,230.41954346)
\curveto(15.42083336,230.48953844)(15.40583337,230.55953837)(15.3858255,230.62954346)
\lineto(15.3858255,230.77954346)
\curveto(15.36583341,230.86953806)(15.35583342,230.98953794)(15.3558255,231.13954346)
\curveto(15.35583342,231.29953763)(15.36583341,231.41453751)(15.3858255,231.48454346)
\curveto(15.39583338,231.5245374)(15.40083338,231.57953735)(15.4008255,231.64954346)
\curveto(15.43083335,231.74953718)(15.45583332,231.85453707)(15.4758255,231.96454346)
\curveto(15.48583329,232.07453685)(15.51583326,232.17453675)(15.5658255,232.26454346)
\curveto(15.62583315,232.40453652)(15.69083309,232.53453639)(15.7608255,232.65454346)
\curveto(15.83083295,232.77453615)(15.91083287,232.88453604)(16.0008255,232.98454346)
\curveto(16.05083273,233.03453589)(16.10583267,233.08453584)(16.1658255,233.13454346)
\curveto(16.21583256,233.19453573)(16.23083255,233.27953565)(16.2108255,233.38954346)
\lineto(16.1358255,233.46454346)
\curveto(16.11583266,233.48453544)(16.08583269,233.49953543)(16.0458255,233.50954346)
\curveto(15.95583282,233.55953537)(15.84083294,233.59453533)(15.7008255,233.61454346)
\curveto(15.56083322,233.64453528)(15.43583334,233.66953526)(15.3258255,233.68954346)
\lineto(13.6008255,234.03454346)
\curveto(13.46083532,234.06453486)(13.30583547,234.09453483)(13.1358255,234.12454346)
\curveto(12.95583582,234.16453476)(12.82583595,234.21453471)(12.7458255,234.27454346)
\curveto(12.6758361,234.33453459)(12.63083615,234.40453452)(12.6108255,234.48454346)
\curveto(12.61083617,234.50453442)(12.61083617,234.5295344)(12.6108255,234.55954346)
\curveto(12.60083618,234.58953434)(12.59583618,234.61453431)(12.5958255,234.63454346)
\curveto(12.58583619,234.78453414)(12.58583619,234.93453399)(12.5958255,235.08454346)
\curveto(12.59583618,235.23453369)(12.63583614,235.33453359)(12.7158255,235.38454346)
\curveto(12.79583598,235.41453351)(12.89583588,235.41453351)(13.0158255,235.38454346)
\curveto(13.13583564,235.36453356)(13.26083552,235.34453358)(13.3908255,235.32454346)
\lineto(22.4508255,233.50954346)
\moveto(19.6158255,232.86454346)
\curveto(19.56582921,232.89453603)(19.50082928,232.91453601)(19.4208255,232.92454346)
\curveto(19.33082945,232.94453598)(19.26082952,232.94953598)(19.2108255,232.93954346)
\lineto(18.9858255,232.98454346)
\curveto(18.89582988,232.98453594)(18.80582997,232.98953594)(18.7158255,232.99954346)
\curveto(18.61583016,233.00953592)(18.52583025,233.00453592)(18.4458255,232.98454346)
\lineto(18.2208255,232.98454346)
\curveto(18.15083063,232.98453594)(18.0808307,232.97453595)(18.0108255,232.95454346)
\curveto(17.71083107,232.89453603)(17.44583133,232.78953614)(17.2158255,232.63954346)
\curveto(16.98583179,232.49953643)(16.80583197,232.29953663)(16.6758255,232.03954346)
\curveto(16.62583215,231.94953698)(16.59083219,231.85453707)(16.5708255,231.75454346)
\curveto(16.54083224,231.65453727)(16.51583226,231.54453738)(16.4958255,231.42454346)
\curveto(16.4758323,231.35453757)(16.46583231,231.26953766)(16.4658255,231.16954346)
\lineto(16.4658255,230.89954346)
\lineto(16.4958255,230.74954346)
\lineto(16.4958255,230.61454346)
\curveto(16.51583226,230.53453839)(16.53583224,230.44953848)(16.5558255,230.35954346)
\curveto(16.5758322,230.26953866)(16.60083218,230.18453874)(16.6308255,230.10454346)
\curveto(16.77083201,229.75453917)(16.9758318,229.45453947)(17.2458255,229.20454346)
\curveto(17.50583127,228.95453997)(17.81083097,228.73454019)(18.1608255,228.54454346)
\curveto(18.27083051,228.48454044)(18.38583039,228.43454049)(18.5058255,228.39454346)
\lineto(18.8358255,228.27454346)
\lineto(18.9558255,228.24454346)
\curveto(18.98582979,228.23454069)(19.02082976,228.2245407)(19.0608255,228.21454346)
\curveto(19.11082967,228.18454074)(19.16582961,228.16454076)(19.2258255,228.15454346)
\curveto(19.28582949,228.15454077)(19.34082944,228.14954078)(19.3908255,228.13954346)
\curveto(19.50082928,228.11954081)(19.61082917,228.09454083)(19.7208255,228.06454346)
\curveto(19.82082896,228.04454088)(19.91582886,228.03954089)(20.0058255,228.04954346)
\curveto(20.03582874,228.04954088)(20.08582869,228.04454088)(20.1558255,228.03454346)
\lineto(20.3658255,228.03454346)
\curveto(20.43582834,228.03454089)(20.50582827,228.03954089)(20.5758255,228.04954346)
\curveto(20.92582785,228.08954084)(21.22582755,228.17954075)(21.4758255,228.31954346)
\curveto(21.72582705,228.45954047)(21.93082685,228.65954027)(22.0908255,228.91954346)
\curveto(22.14082664,228.99953993)(22.1808266,229.07953985)(22.2108255,229.15954346)
\curveto(22.24082654,229.24953968)(22.27082651,229.34453958)(22.3008255,229.44454346)
\curveto(22.32082646,229.49453943)(22.32582645,229.54453938)(22.3158255,229.59454346)
\curveto(22.30582647,229.65453927)(22.31082647,229.70953922)(22.3308255,229.75954346)
\curveto(22.34082644,229.78953914)(22.34582643,229.8245391)(22.3458255,229.86454346)
\lineto(22.3458255,229.99954346)
\lineto(22.3458255,230.13454346)
\curveto(22.33582644,230.17453875)(22.33082645,230.2295387)(22.3308255,230.29954346)
\curveto(22.31082647,230.37953855)(22.29582648,230.45953847)(22.2858255,230.53954346)
\curveto(22.26582651,230.6295383)(22.24082654,230.70953822)(22.2108255,230.77954346)
\curveto(22.07082671,231.13953779)(21.89582688,231.44453748)(21.6858255,231.69454346)
\curveto(21.46582731,231.94453698)(21.19082759,232.16953676)(20.8608255,232.36954346)
\curveto(20.75082803,232.43953649)(20.64082814,232.49453643)(20.5308255,232.53454346)
\lineto(20.2008255,232.68454346)
\curveto(20.16082862,232.71453621)(20.12582865,232.7295362)(20.0958255,232.72954346)
\curveto(20.05582872,232.73953619)(20.01582876,232.75453617)(19.9758255,232.77454346)
\curveto(19.91582886,232.79453613)(19.85582892,232.80953612)(19.7958255,232.81954346)
\curveto(19.73582904,232.8295361)(19.6758291,232.84453608)(19.6158255,232.86454346)
}
}
{
\newrgbcolor{curcolor}{0 0 0}
\pscustom[linestyle=none,fillstyle=solid,fillcolor=curcolor]
{
\newpath
\moveto(22.7058255,242.29579346)
\curveto(22.86582591,242.28578555)(23.00082578,242.24078559)(23.1108255,242.16079346)
\curveto(23.21082557,242.08078575)(23.28582549,241.98578585)(23.3358255,241.87579346)
\curveto(23.35582542,241.82578601)(23.36582541,241.77078606)(23.3658255,241.71079346)
\curveto(23.36582541,241.66078617)(23.3758254,241.60078623)(23.3958255,241.53079346)
\curveto(23.44582533,241.30078653)(23.43082535,241.08578675)(23.3508255,240.88579346)
\curveto(23.2808255,240.68578715)(23.19082559,240.56078727)(23.0808255,240.51079346)
\curveto(23.01082577,240.47078736)(22.93082585,240.44078739)(22.8408255,240.42079346)
\curveto(22.74082604,240.40078743)(22.66082612,240.36578747)(22.6008255,240.31579346)
\lineto(22.5408255,240.25579346)
\curveto(22.52082626,240.2357876)(22.51582626,240.20578763)(22.5258255,240.16579346)
\curveto(22.55582622,240.04578779)(22.61082617,239.9307879)(22.6908255,239.82079346)
\curveto(22.77082601,239.71078812)(22.84082594,239.60578823)(22.9008255,239.50579346)
\curveto(22.9808258,239.35578848)(23.05582572,239.20078863)(23.1258255,239.04079346)
\curveto(23.18582559,238.88078895)(23.24082554,238.71078912)(23.2908255,238.53079346)
\curveto(23.32082546,238.42078941)(23.34082544,238.30578953)(23.3508255,238.18579346)
\curveto(23.36082542,238.07578976)(23.3758254,237.96078987)(23.3958255,237.84079346)
\curveto(23.40582537,237.79079004)(23.41082537,237.74579009)(23.4108255,237.70579346)
\lineto(23.4108255,237.60079346)
\curveto(23.43082535,237.49079034)(23.43082535,237.38579045)(23.4108255,237.28579346)
\lineto(23.4108255,237.15079346)
\curveto(23.40082538,237.10079073)(23.39582538,237.05079078)(23.3958255,237.00079346)
\curveto(23.39582538,236.95079088)(23.38582539,236.91079092)(23.3658255,236.88079346)
\curveto(23.35582542,236.84079099)(23.35082543,236.80579103)(23.3508255,236.77579346)
\curveto(23.36082542,236.75579108)(23.36082542,236.7307911)(23.3508255,236.70079346)
\lineto(23.2908255,236.46079346)
\curveto(23.2808255,236.39079144)(23.26082552,236.32579151)(23.2308255,236.26579346)
\curveto(23.10082568,235.98579185)(22.95582582,235.77079206)(22.7958255,235.62079346)
\curveto(22.62582615,235.47079236)(22.39082639,235.36579247)(22.0908255,235.30579346)
\curveto(21.87082691,235.25579258)(21.60582717,235.26079257)(21.2958255,235.32079346)
\lineto(20.9808255,235.39579346)
\curveto(20.93082785,235.41579242)(20.8808279,235.4307924)(20.8308255,235.44079346)
\lineto(20.6508255,235.50079346)
\lineto(20.3208255,235.68079346)
\curveto(20.21082857,235.75079208)(20.11082867,235.82079201)(20.0208255,235.89079346)
\curveto(19.73082905,236.1307917)(19.51582926,236.42079141)(19.3758255,236.76079346)
\curveto(19.23582954,237.10079073)(19.11082967,237.46579037)(19.0008255,237.85579346)
\curveto(18.96082982,238.00578983)(18.93082985,238.15578968)(18.9108255,238.30579346)
\curveto(18.89082989,238.46578937)(18.86582991,238.62078921)(18.8358255,238.77079346)
\curveto(18.81582996,238.85078898)(18.80582997,238.92078891)(18.8058255,238.98079346)
\curveto(18.80582997,239.05078878)(18.79582998,239.12578871)(18.7758255,239.20579346)
\curveto(18.75583002,239.27578856)(18.74583003,239.34578849)(18.7458255,239.41579346)
\curveto(18.73583004,239.49578834)(18.72083006,239.57578826)(18.7008255,239.65579346)
\curveto(18.64083014,239.91578792)(18.59083019,240.16078767)(18.5508255,240.39079346)
\curveto(18.50083028,240.62078721)(18.38583039,240.82078701)(18.2058255,240.99079346)
\curveto(18.12583065,241.06078677)(18.02583075,241.12578671)(17.9058255,241.18579346)
\curveto(17.775831,241.25578658)(17.63583114,241.28578655)(17.4858255,241.27579346)
\curveto(17.24583153,241.26578657)(17.05583172,241.21578662)(16.9158255,241.12579346)
\curveto(16.775832,241.04578679)(16.66583211,240.90578693)(16.5858255,240.70579346)
\curveto(16.53583224,240.59578724)(16.50083228,240.46078737)(16.4808255,240.30079346)
\curveto(16.46083232,240.14078769)(16.45083233,239.97078786)(16.4508255,239.79079346)
\curveto(16.45083233,239.61078822)(16.46083232,239.4307884)(16.4808255,239.25079346)
\curveto(16.50083228,239.08078875)(16.53083225,238.9307889)(16.5708255,238.80079346)
\curveto(16.63083215,238.62078921)(16.71583206,238.44078939)(16.8258255,238.26079346)
\curveto(16.88583189,238.17078966)(16.96583181,238.08078975)(17.0658255,237.99079346)
\curveto(17.15583162,237.91078992)(17.25583152,237.83579)(17.3658255,237.76579346)
\curveto(17.44583133,237.71579012)(17.53083125,237.67079016)(17.6208255,237.63079346)
\curveto(17.71083107,237.59079024)(17.780831,237.5307903)(17.8308255,237.45079346)
\curveto(17.86083092,237.40079043)(17.88583089,237.32579051)(17.9058255,237.22579346)
\curveto(17.91583086,237.12579071)(17.92083086,237.02579081)(17.9208255,236.92579346)
\curveto(17.92083086,236.82579101)(17.91583086,236.7307911)(17.9058255,236.64079346)
\curveto(17.88583089,236.55079128)(17.86083092,236.49079134)(17.8308255,236.46079346)
\curveto(17.80083098,236.42079141)(17.75083103,236.39579144)(17.6808255,236.38579346)
\curveto(17.61083117,236.38579145)(17.53583124,236.40579143)(17.4558255,236.44579346)
\curveto(17.32583145,236.49579134)(17.20583157,236.55079128)(17.0958255,236.61079346)
\curveto(16.9758318,236.67079116)(16.86083192,236.7357911)(16.7508255,236.80579346)
\curveto(16.40083238,237.06579077)(16.13083265,237.36079047)(15.9408255,237.69079346)
\curveto(15.74083304,238.02078981)(15.5808332,238.41078942)(15.4608255,238.86079346)
\curveto(15.44083334,238.97078886)(15.42583335,239.07578876)(15.4158255,239.17579346)
\curveto(15.40583337,239.28578855)(15.39083339,239.39578844)(15.3708255,239.50579346)
\curveto(15.36083342,239.55578828)(15.36083342,239.62078821)(15.3708255,239.70079346)
\curveto(15.37083341,239.79078804)(15.36083342,239.85078798)(15.3408255,239.88079346)
\curveto(15.33083345,240.58078725)(15.41083337,241.17078666)(15.5808255,241.65079346)
\curveto(15.75083303,242.14078569)(16.0758327,242.44578539)(16.5558255,242.56579346)
\curveto(16.75583202,242.61578522)(16.99083179,242.62078521)(17.2608255,242.58079346)
\curveto(17.52083126,242.54078529)(17.79583098,242.49078534)(18.0858255,242.43079346)
\lineto(21.4008255,241.77079346)
\curveto(21.54082724,241.74078609)(21.6758271,241.71578612)(21.8058255,241.69579346)
\curveto(21.93582684,241.68578615)(22.04082674,241.69578614)(22.1208255,241.72579346)
\curveto(22.19082659,241.76578607)(22.24082654,241.82078601)(22.2708255,241.89079346)
\curveto(22.31082647,241.98078585)(22.34082644,242.06078577)(22.3608255,242.13079346)
\curveto(22.37082641,242.21078562)(22.41582636,242.26078557)(22.4958255,242.28079346)
\curveto(22.52582625,242.30078553)(22.55582622,242.30578553)(22.5858255,242.29579346)
\lineto(22.7058255,242.29579346)
\moveto(21.0408255,240.48079346)
\curveto(20.90082788,240.57078726)(20.74082804,240.6357872)(20.5608255,240.67579346)
\curveto(20.37082841,240.71578712)(20.1758286,240.75578708)(19.9758255,240.79579346)
\curveto(19.86582891,240.81578702)(19.76582901,240.830787)(19.6758255,240.84079346)
\curveto(19.58582919,240.85078698)(19.51582926,240.82578701)(19.4658255,240.76579346)
\curveto(19.44582933,240.7357871)(19.43582934,240.66578717)(19.4358255,240.55579346)
\curveto(19.45582932,240.5357873)(19.46582931,240.50078733)(19.4658255,240.45079346)
\curveto(19.46582931,240.40078743)(19.4758293,240.35078748)(19.4958255,240.30079346)
\curveto(19.51582926,240.22078761)(19.53582924,240.12578771)(19.5558255,240.01579346)
\lineto(19.6158255,239.71579346)
\curveto(19.61582916,239.68578815)(19.62082916,239.65078818)(19.6308255,239.61079346)
\lineto(19.6308255,239.50579346)
\curveto(19.67082911,239.34578849)(19.69582908,239.17578866)(19.7058255,238.99579346)
\curveto(19.70582907,238.82578901)(19.72582905,238.66078917)(19.7658255,238.50079346)
\curveto(19.78582899,238.41078942)(19.80582897,238.3307895)(19.8258255,238.26079346)
\curveto(19.83582894,238.20078963)(19.85082893,238.12578971)(19.8708255,238.03579346)
\curveto(19.92082886,237.86578997)(19.98582879,237.70079013)(20.0658255,237.54079346)
\curveto(20.13582864,237.39079044)(20.22582855,237.25579058)(20.3358255,237.13579346)
\curveto(20.44582833,237.01579082)(20.5808282,236.91579092)(20.7408255,236.83579346)
\curveto(20.89082789,236.75579108)(21.0758277,236.69579114)(21.2958255,236.65579346)
\curveto(21.39582738,236.6357912)(21.49082729,236.6357912)(21.5808255,236.65579346)
\curveto(21.66082712,236.67579116)(21.73582704,236.70579113)(21.8058255,236.74579346)
\curveto(21.91582686,236.79579104)(22.01082677,236.87579096)(22.0908255,236.98579346)
\curveto(22.16082662,237.10579073)(22.22082656,237.2357906)(22.2708255,237.37579346)
\curveto(22.2808265,237.42579041)(22.28582649,237.47579036)(22.2858255,237.52579346)
\curveto(22.28582649,237.57579026)(22.29082649,237.62579021)(22.3008255,237.67579346)
\curveto(22.32082646,237.74579009)(22.33582644,237.83079)(22.3458255,237.93079346)
\curveto(22.34582643,238.0307898)(22.33582644,238.12078971)(22.3158255,238.20079346)
\curveto(22.29582648,238.26078957)(22.29082649,238.32078951)(22.3008255,238.38079346)
\curveto(22.30082648,238.44078939)(22.29082649,238.50078933)(22.2708255,238.56079346)
\curveto(22.25082653,238.65078918)(22.23582654,238.7307891)(22.2258255,238.80079346)
\curveto(22.21582656,238.88078895)(22.19582658,238.96078887)(22.1658255,239.04079346)
\curveto(22.04582673,239.35078848)(21.90082688,239.62578821)(21.7308255,239.86579346)
\curveto(21.56082722,240.10578773)(21.33082745,240.31078752)(21.0408255,240.48079346)
}
}
{
\newrgbcolor{curcolor}{0 0 0}
\pscustom[linestyle=none,fillstyle=solid,fillcolor=curcolor]
{
\newpath
\moveto(22.4508255,250.47243408)
\lineto(22.8408255,250.38243408)
\curveto(22.96082582,250.36242615)(23.06082572,250.32242619)(23.1408255,250.26243408)
\curveto(23.21082557,250.19242632)(23.25082553,250.09742642)(23.2608255,249.97743408)
\lineto(23.2608255,249.63243408)
\curveto(23.26082552,249.57242694)(23.26582551,249.512427)(23.2758255,249.45243408)
\curveto(23.2758255,249.40242711)(23.26582551,249.35742716)(23.2458255,249.31743408)
\curveto(23.22582555,249.23742728)(23.18582559,249.18742733)(23.1258255,249.16743408)
\curveto(23.0758257,249.13742738)(23.01582576,249.12742739)(22.9458255,249.13743408)
\curveto(22.8758259,249.14742737)(22.80582597,249.14242737)(22.7358255,249.12243408)
\curveto(22.71582606,249.12242739)(22.70082608,249.1124274)(22.6908255,249.09243408)
\lineto(22.6308255,249.06243408)
\curveto(22.62082616,248.96242755)(22.64082614,248.87742764)(22.6908255,248.80743408)
\curveto(22.74082604,248.74742777)(22.79082599,248.68242783)(22.8408255,248.61243408)
\curveto(22.99082579,248.38242813)(23.10582567,248.15742836)(23.1858255,247.93743408)
\curveto(23.26582551,247.74742877)(23.32582545,247.52742899)(23.3658255,247.27743408)
\curveto(23.40582537,247.03742948)(23.42582535,246.79242972)(23.4258255,246.54243408)
\curveto(23.43582534,246.30243021)(23.42082536,246.06243045)(23.3808255,245.82243408)
\curveto(23.35082543,245.59243092)(23.29582548,245.39743112)(23.2158255,245.23743408)
\curveto(22.99582578,244.75743176)(22.70082608,244.39243212)(22.3308255,244.14243408)
\curveto(21.95082683,243.90243261)(21.4808273,243.74743277)(20.9208255,243.67743408)
\curveto(20.83082795,243.65743286)(20.74082804,243.64743287)(20.6508255,243.64743408)
\curveto(20.55082823,243.65743286)(20.45082833,243.65743286)(20.3508255,243.64743408)
\curveto(20.30082848,243.64743287)(20.25082853,243.65243286)(20.2008255,243.66243408)
\curveto(20.15082863,243.67243284)(20.10082868,243.67743284)(20.0508255,243.67743408)
\curveto(20.00082878,243.66743285)(19.95082883,243.66743285)(19.9008255,243.67743408)
\curveto(19.84082894,243.69743282)(19.78582899,243.70743281)(19.7358255,243.70743408)
\lineto(19.5858255,243.73743408)
\curveto(19.53582924,243.72743279)(19.47082931,243.72743279)(19.3908255,243.73743408)
\curveto(19.31082947,243.75743276)(19.24582953,243.78243273)(19.1958255,243.81243408)
\lineto(19.0308255,243.85743408)
\curveto(18.96082982,243.88743263)(18.89082989,243.90743261)(18.8208255,243.91743408)
\curveto(18.74083004,243.92743259)(18.66583011,243.94743257)(18.5958255,243.97743408)
\curveto(18.54583023,243.99743252)(18.50083028,244.0124325)(18.4608255,244.02243408)
\curveto(18.42083036,244.03243248)(18.3758304,244.04743247)(18.3258255,244.06743408)
\curveto(18.22583055,244.1174324)(18.13083065,244.16243235)(18.0408255,244.20243408)
\curveto(17.94083084,244.24243227)(17.84583093,244.28743223)(17.7558255,244.33743408)
\curveto(17.3758314,244.53743198)(17.03583174,244.76743175)(16.7358255,245.02743408)
\curveto(16.42583235,245.29743122)(16.17083261,245.59743092)(15.9708255,245.92743408)
\curveto(15.85083293,246.12743039)(15.75083303,246.32743019)(15.6708255,246.52743408)
\curveto(15.59083319,246.72742979)(15.52083326,246.94242957)(15.4608255,247.17243408)
\lineto(15.4308255,247.38243408)
\curveto(15.42083336,247.45242906)(15.40583337,247.52242899)(15.3858255,247.59243408)
\lineto(15.3858255,247.74243408)
\curveto(15.36583341,247.83242868)(15.35583342,247.95242856)(15.3558255,248.10243408)
\curveto(15.35583342,248.26242825)(15.36583341,248.37742814)(15.3858255,248.44743408)
\curveto(15.39583338,248.48742803)(15.40083338,248.54242797)(15.4008255,248.61243408)
\curveto(15.43083335,248.7124278)(15.45583332,248.8174277)(15.4758255,248.92743408)
\curveto(15.48583329,249.03742748)(15.51583326,249.13742738)(15.5658255,249.22743408)
\curveto(15.62583315,249.36742715)(15.69083309,249.49742702)(15.7608255,249.61743408)
\curveto(15.83083295,249.73742678)(15.91083287,249.84742667)(16.0008255,249.94743408)
\curveto(16.05083273,249.99742652)(16.10583267,250.04742647)(16.1658255,250.09743408)
\curveto(16.21583256,250.15742636)(16.23083255,250.24242627)(16.2108255,250.35243408)
\lineto(16.1358255,250.42743408)
\curveto(16.11583266,250.44742607)(16.08583269,250.46242605)(16.0458255,250.47243408)
\curveto(15.95583282,250.52242599)(15.84083294,250.55742596)(15.7008255,250.57743408)
\curveto(15.56083322,250.60742591)(15.43583334,250.63242588)(15.3258255,250.65243408)
\lineto(13.6008255,250.99743408)
\curveto(13.46083532,251.02742549)(13.30583547,251.05742546)(13.1358255,251.08743408)
\curveto(12.95583582,251.12742539)(12.82583595,251.17742534)(12.7458255,251.23743408)
\curveto(12.6758361,251.29742522)(12.63083615,251.36742515)(12.6108255,251.44743408)
\curveto(12.61083617,251.46742505)(12.61083617,251.49242502)(12.6108255,251.52243408)
\curveto(12.60083618,251.55242496)(12.59583618,251.57742494)(12.5958255,251.59743408)
\curveto(12.58583619,251.74742477)(12.58583619,251.89742462)(12.5958255,252.04743408)
\curveto(12.59583618,252.19742432)(12.63583614,252.29742422)(12.7158255,252.34743408)
\curveto(12.79583598,252.37742414)(12.89583588,252.37742414)(13.0158255,252.34743408)
\curveto(13.13583564,252.32742419)(13.26083552,252.30742421)(13.3908255,252.28743408)
\lineto(22.4508255,250.47243408)
\moveto(19.6158255,249.82743408)
\curveto(19.56582921,249.85742666)(19.50082928,249.87742664)(19.4208255,249.88743408)
\curveto(19.33082945,249.90742661)(19.26082952,249.9124266)(19.2108255,249.90243408)
\lineto(18.9858255,249.94743408)
\curveto(18.89582988,249.94742657)(18.80582997,249.95242656)(18.7158255,249.96243408)
\curveto(18.61583016,249.97242654)(18.52583025,249.96742655)(18.4458255,249.94743408)
\lineto(18.2208255,249.94743408)
\curveto(18.15083063,249.94742657)(18.0808307,249.93742658)(18.0108255,249.91743408)
\curveto(17.71083107,249.85742666)(17.44583133,249.75242676)(17.2158255,249.60243408)
\curveto(16.98583179,249.46242705)(16.80583197,249.26242725)(16.6758255,249.00243408)
\curveto(16.62583215,248.9124276)(16.59083219,248.8174277)(16.5708255,248.71743408)
\curveto(16.54083224,248.6174279)(16.51583226,248.50742801)(16.4958255,248.38743408)
\curveto(16.4758323,248.3174282)(16.46583231,248.23242828)(16.4658255,248.13243408)
\lineto(16.4658255,247.86243408)
\lineto(16.4958255,247.71243408)
\lineto(16.4958255,247.57743408)
\curveto(16.51583226,247.49742902)(16.53583224,247.4124291)(16.5558255,247.32243408)
\curveto(16.5758322,247.23242928)(16.60083218,247.14742937)(16.6308255,247.06743408)
\curveto(16.77083201,246.7174298)(16.9758318,246.4174301)(17.2458255,246.16743408)
\curveto(17.50583127,245.9174306)(17.81083097,245.69743082)(18.1608255,245.50743408)
\curveto(18.27083051,245.44743107)(18.38583039,245.39743112)(18.5058255,245.35743408)
\lineto(18.8358255,245.23743408)
\lineto(18.9558255,245.20743408)
\curveto(18.98582979,245.19743132)(19.02082976,245.18743133)(19.0608255,245.17743408)
\curveto(19.11082967,245.14743137)(19.16582961,245.12743139)(19.2258255,245.11743408)
\curveto(19.28582949,245.1174314)(19.34082944,245.1124314)(19.3908255,245.10243408)
\curveto(19.50082928,245.08243143)(19.61082917,245.05743146)(19.7208255,245.02743408)
\curveto(19.82082896,245.00743151)(19.91582886,245.00243151)(20.0058255,245.01243408)
\curveto(20.03582874,245.0124315)(20.08582869,245.00743151)(20.1558255,244.99743408)
\lineto(20.3658255,244.99743408)
\curveto(20.43582834,244.99743152)(20.50582827,245.00243151)(20.5758255,245.01243408)
\curveto(20.92582785,245.05243146)(21.22582755,245.14243137)(21.4758255,245.28243408)
\curveto(21.72582705,245.42243109)(21.93082685,245.62243089)(22.0908255,245.88243408)
\curveto(22.14082664,245.96243055)(22.1808266,246.04243047)(22.2108255,246.12243408)
\curveto(22.24082654,246.2124303)(22.27082651,246.30743021)(22.3008255,246.40743408)
\curveto(22.32082646,246.45743006)(22.32582645,246.50743001)(22.3158255,246.55743408)
\curveto(22.30582647,246.6174299)(22.31082647,246.67242984)(22.3308255,246.72243408)
\curveto(22.34082644,246.75242976)(22.34582643,246.78742973)(22.3458255,246.82743408)
\lineto(22.3458255,246.96243408)
\lineto(22.3458255,247.09743408)
\curveto(22.33582644,247.13742938)(22.33082645,247.19242932)(22.3308255,247.26243408)
\curveto(22.31082647,247.34242917)(22.29582648,247.42242909)(22.2858255,247.50243408)
\curveto(22.26582651,247.59242892)(22.24082654,247.67242884)(22.2108255,247.74243408)
\curveto(22.07082671,248.10242841)(21.89582688,248.40742811)(21.6858255,248.65743408)
\curveto(21.46582731,248.90742761)(21.19082759,249.13242738)(20.8608255,249.33243408)
\curveto(20.75082803,249.40242711)(20.64082814,249.45742706)(20.5308255,249.49743408)
\lineto(20.2008255,249.64743408)
\curveto(20.16082862,249.67742684)(20.12582865,249.69242682)(20.0958255,249.69243408)
\curveto(20.05582872,249.70242681)(20.01582876,249.7174268)(19.9758255,249.73743408)
\curveto(19.91582886,249.75742676)(19.85582892,249.77242674)(19.7958255,249.78243408)
\curveto(19.73582904,249.79242672)(19.6758291,249.80742671)(19.6158255,249.82743408)
}
}
{
\newrgbcolor{curcolor}{0 0 0}
\pscustom[linestyle=none,fillstyle=solid,fillcolor=curcolor]
{
\newpath
\moveto(19.0908255,259.84368408)
\curveto(19.19082959,259.84367558)(19.30582947,259.8236756)(19.4358255,259.78368408)
\curveto(19.55582922,259.74367568)(19.64082914,259.69367573)(19.6908255,259.63368408)
\curveto(19.73082905,259.57367585)(19.76082902,259.49367593)(19.7808255,259.39368408)
\curveto(19.79082899,259.29367613)(19.79582898,259.18367624)(19.7958255,259.06368408)
\lineto(19.7958255,258.70368408)
\curveto(19.78582899,258.59367683)(19.780829,258.49367693)(19.7808255,258.40368408)
\lineto(19.7808255,254.56368408)
\curveto(19.780829,254.48368094)(19.78582899,254.39868102)(19.7958255,254.30868408)
\curveto(19.79582898,254.22868119)(19.81082897,254.16368126)(19.8408255,254.11368408)
\curveto(19.86082892,254.06368136)(19.90082888,254.01368141)(19.9608255,253.96368408)
\lineto(20.0958255,253.87368408)
\curveto(20.14582863,253.84368158)(20.19582858,253.83368159)(20.2458255,253.84368408)
\curveto(20.29582848,253.84368158)(20.34082844,253.83868158)(20.3808255,253.82868408)
\lineto(20.5008255,253.82868408)
\lineto(20.7558255,253.82868408)
\curveto(20.83582794,253.83868158)(20.91582786,253.85368157)(20.9958255,253.87368408)
\curveto(21.53582724,254.00368142)(21.92082686,254.30868111)(22.1508255,254.78868408)
\curveto(22.1808266,254.83868058)(22.20582657,254.89868052)(22.2258255,254.96868408)
\curveto(22.24582653,255.03868038)(22.26582651,255.10368032)(22.2858255,255.16368408)
\curveto(22.29582648,255.19368023)(22.30082648,255.24368018)(22.3008255,255.31368408)
\curveto(22.34082644,255.44367998)(22.36082642,255.6236798)(22.3608255,255.85368408)
\curveto(22.36082642,256.08367934)(22.34082644,256.27367915)(22.3008255,256.42368408)
\curveto(22.26082652,256.57367885)(22.22082656,256.70867871)(22.1808255,256.82868408)
\curveto(22.13082665,256.95867846)(22.07082671,257.07867834)(22.0008255,257.18868408)
\curveto(21.93082685,257.30867811)(21.85082693,257.418678)(21.7608255,257.51868408)
\curveto(21.66082712,257.6186778)(21.55582722,257.70867771)(21.4458255,257.78868408)
\curveto(21.34582743,257.86867755)(21.24082754,257.94367748)(21.1308255,258.01368408)
\curveto(21.02082776,258.08367734)(20.94082784,258.17867724)(20.8908255,258.29868408)
\curveto(20.87082791,258.33867708)(20.85582792,258.40367702)(20.8458255,258.49368408)
\curveto(20.83582794,258.59367683)(20.83582794,258.68367674)(20.8458255,258.76368408)
\curveto(20.84582793,258.85367657)(20.85082793,258.93867648)(20.8608255,259.01868408)
\curveto(20.87082791,259.09867632)(20.89082789,259.14867627)(20.9208255,259.16868408)
\curveto(20.99082779,259.25867616)(21.10582767,259.26367616)(21.2658255,259.18368408)
\curveto(21.53582724,259.04367638)(21.775827,258.88867653)(21.9858255,258.71868408)
\curveto(22.30582647,258.45867696)(22.57082621,258.17867724)(22.7808255,257.87868408)
\curveto(22.9808258,257.58867783)(23.14582563,257.23367819)(23.2758255,256.81368408)
\curveto(23.31582546,256.70367872)(23.34082544,256.59867882)(23.3508255,256.49868408)
\curveto(23.37082541,256.39867902)(23.39082539,256.28867913)(23.4108255,256.16868408)
\curveto(23.42082536,256.1186793)(23.42582535,256.06867935)(23.4258255,256.01868408)
\curveto(23.42582535,255.97867944)(23.43082535,255.93367949)(23.4408255,255.88368408)
\lineto(23.4408255,255.73368408)
\curveto(23.45082533,255.68367974)(23.45582532,255.6236798)(23.4558255,255.55368408)
\curveto(23.45582532,255.49367993)(23.45082533,255.44367998)(23.4408255,255.40368408)
\lineto(23.4408255,255.26868408)
\curveto(23.43082535,255.2186802)(23.42582535,255.17368025)(23.4258255,255.13368408)
\curveto(23.42582535,255.09368033)(23.42082536,255.05368037)(23.4108255,255.01368408)
\curveto(23.40082538,254.96368046)(23.39082539,254.90868051)(23.3808255,254.84868408)
\curveto(23.3808254,254.79868062)(23.3758254,254.74868067)(23.3658255,254.69868408)
\curveto(23.34582543,254.60868081)(23.32082546,254.5186809)(23.2908255,254.42868408)
\curveto(23.27082551,254.34868107)(23.24582553,254.27368115)(23.2158255,254.20368408)
\curveto(23.19582558,254.16368126)(23.18582559,254.12868129)(23.1858255,254.09868408)
\curveto(23.1758256,254.06868135)(23.16082562,254.03868138)(23.1408255,254.00868408)
\curveto(23.07082571,253.86868155)(22.98582579,253.7236817)(22.8858255,253.57368408)
\curveto(22.69582608,253.3236821)(22.46582631,253.1236823)(22.1958255,252.97368408)
\curveto(21.91582686,252.8236826)(21.60582717,252.71368271)(21.2658255,252.64368408)
\curveto(21.15582762,252.61368281)(21.04082774,252.59868282)(20.9208255,252.59868408)
\curveto(20.80082798,252.59868282)(20.6808281,252.58868283)(20.5608255,252.56868408)
\lineto(20.4558255,252.56868408)
\curveto(20.42582835,252.57868284)(20.38582839,252.58368284)(20.3358255,252.58368408)
\lineto(20.0808255,252.58368408)
\curveto(19.99082879,252.59368283)(19.90082888,252.59868282)(19.8108255,252.59868408)
\lineto(19.6008255,252.64368408)
\curveto(19.56082922,252.64368278)(19.50582927,252.64868277)(19.4358255,252.65868408)
\curveto(19.35582942,252.66868275)(19.29082949,252.68368274)(19.2408255,252.70368408)
\lineto(19.0758255,252.73368408)
\curveto(19.02582975,252.76368266)(18.9758298,252.77868264)(18.9258255,252.77868408)
\curveto(18.86582991,252.78868263)(18.81082997,252.80368262)(18.7608255,252.82368408)
\curveto(18.60083018,252.89368253)(18.44083034,252.95868246)(18.2808255,253.01868408)
\curveto(18.12083066,253.07868234)(17.97083081,253.15368227)(17.8308255,253.24368408)
\curveto(17.72083106,253.31368211)(17.61083117,253.37868204)(17.5008255,253.43868408)
\curveto(17.3808314,253.50868191)(17.26583151,253.58868183)(17.1558255,253.67868408)
\curveto(16.80583197,253.96868145)(16.50583227,254.27868114)(16.2558255,254.60868408)
\curveto(15.99583278,254.93868048)(15.780833,255.3236801)(15.6108255,255.76368408)
\curveto(15.56083322,255.89367953)(15.52583325,256.0236794)(15.5058255,256.15368408)
\curveto(15.4758333,256.28367914)(15.44583333,256.423679)(15.4158255,256.57368408)
\curveto(15.40583337,256.6236788)(15.40083338,256.66867875)(15.4008255,256.70868408)
\curveto(15.39083339,256.74867867)(15.38583339,256.79367863)(15.3858255,256.84368408)
\curveto(15.3758334,256.86367856)(15.3758334,256.88867853)(15.3858255,256.91868408)
\curveto(15.39583338,256.94867847)(15.39083339,256.97367845)(15.3708255,256.99368408)
\curveto(15.36083342,257.423678)(15.40583337,257.78367764)(15.5058255,258.07368408)
\curveto(15.59583318,258.36367706)(15.72083306,258.6186768)(15.8808255,258.83868408)
\curveto(15.90083288,258.87867654)(15.93083285,258.90867651)(15.9708255,258.92868408)
\curveto(16.00083278,258.95867646)(16.02583275,258.98867643)(16.0458255,259.01868408)
\curveto(16.10583267,259.08867633)(16.1758326,259.15867626)(16.2558255,259.22868408)
\curveto(16.33583244,259.29867612)(16.41583236,259.35367607)(16.4958255,259.39368408)
\curveto(16.70583207,259.51367591)(16.90583187,259.60867581)(17.0958255,259.67868408)
\curveto(17.20583157,259.72867569)(17.32583145,259.75867566)(17.4558255,259.76868408)
\lineto(17.8458255,259.82868408)
\curveto(17.9758308,259.85867556)(18.11083067,259.86867555)(18.2508255,259.85868408)
\curveto(18.39083039,259.85867556)(18.53083025,259.86367556)(18.6708255,259.87368408)
\curveto(18.74083004,259.87367555)(18.81082997,259.86867555)(18.8808255,259.85868408)
\curveto(18.95082983,259.84867557)(19.02082976,259.84367558)(19.0908255,259.84368408)
\moveto(18.5808255,258.49368408)
\curveto(18.54083024,258.5236769)(18.49083029,258.55367687)(18.4308255,258.58368408)
\curveto(18.36083042,258.6236768)(18.29083049,258.63867678)(18.2208255,258.62868408)
\curveto(18.00083078,258.6186768)(17.79583098,258.57867684)(17.6058255,258.50868408)
\curveto(17.3758314,258.40867701)(17.1808316,258.28867713)(17.0208255,258.14868408)
\curveto(16.86083192,258.0186774)(16.72583205,257.82867759)(16.6158255,257.57868408)
\curveto(16.59583218,257.50867791)(16.5808322,257.43867798)(16.5708255,257.36868408)
\curveto(16.55083223,257.30867811)(16.53083225,257.23867818)(16.5108255,257.15868408)
\curveto(16.49083229,257.08867833)(16.4808323,257.00867841)(16.4808255,256.91868408)
\lineto(16.4808255,256.66368408)
\curveto(16.50083228,256.6236788)(16.51083227,256.58367884)(16.5108255,256.54368408)
\curveto(16.50083228,256.50367892)(16.50083228,256.46867895)(16.5108255,256.43868408)
\lineto(16.5708255,256.19868408)
\curveto(16.5808322,256.1186793)(16.59583218,256.04367938)(16.6158255,255.97368408)
\curveto(16.73583204,255.65367977)(16.88583189,255.38868003)(17.0658255,255.17868408)
\curveto(17.24583153,254.96868045)(17.47083131,254.76868065)(17.7408255,254.57868408)
\curveto(17.79083099,254.53868088)(17.85583092,254.49368093)(17.9358255,254.44368408)
\curveto(18.00583077,254.40368102)(18.08583069,254.36368106)(18.1758255,254.32368408)
\curveto(18.26583051,254.28368114)(18.35083043,254.25868116)(18.4308255,254.24868408)
\curveto(18.51083027,254.24868117)(18.57083021,254.27368115)(18.6108255,254.32368408)
\curveto(18.67083011,254.39368103)(18.70083008,254.5236809)(18.7008255,254.71368408)
\curveto(18.69083009,254.91368051)(18.68583009,255.08368034)(18.6858255,255.22368408)
\lineto(18.6858255,257.50368408)
\curveto(18.68583009,257.65367777)(18.69083009,257.83367759)(18.7008255,258.04368408)
\curveto(18.70083008,258.25367717)(18.66083012,258.40367702)(18.5808255,258.49368408)
}
}
{
\newrgbcolor{curcolor}{0 0 0}
\pscustom[linestyle=none,fillstyle=solid,fillcolor=curcolor]
{
\newpath
\moveto(15.3558255,264.33032471)
\curveto(15.34583343,265.05031905)(15.43083335,265.63531847)(15.6108255,266.08532471)
\curveto(15.780833,266.54531756)(16.08583269,266.86531724)(16.5258255,267.04532471)
\curveto(16.63583214,267.09531701)(16.75083203,267.12531698)(16.8708255,267.13532471)
\curveto(16.9808318,267.15531695)(17.10583167,267.17031693)(17.2458255,267.18032471)
\curveto(17.31583146,267.19031691)(17.39083139,267.18031692)(17.4708255,267.15032471)
\curveto(17.54083124,267.13031697)(17.59583118,267.105317)(17.6358255,267.07532471)
\curveto(17.65583112,267.05531705)(17.6758311,267.02531708)(17.6958255,266.98532471)
\curveto(17.70583107,266.95531715)(17.72083106,266.93031717)(17.7408255,266.91032471)
\curveto(17.76083102,266.85031725)(17.76583101,266.79531731)(17.7558255,266.74532471)
\curveto(17.74583103,266.7053174)(17.74583103,266.66031744)(17.7558255,266.61032471)
\curveto(17.775831,266.52031758)(17.780831,266.41031769)(17.7708255,266.28032471)
\curveto(17.75083103,266.16031794)(17.72583105,266.07531803)(17.6958255,266.02532471)
\curveto(17.64583113,265.95531815)(17.5808312,265.91531819)(17.5008255,265.90532471)
\curveto(17.41083137,265.9053182)(17.32583145,265.88531822)(17.2458255,265.84532471)
\curveto(17.08583169,265.79531831)(16.94083184,265.7003184)(16.8108255,265.56032471)
\curveto(16.73083205,265.47031863)(16.67083211,265.36031874)(16.6308255,265.23032471)
\curveto(16.59083219,265.11031899)(16.55083223,264.98031912)(16.5108255,264.84032471)
\curveto(16.49083229,264.8003193)(16.48583229,264.75031935)(16.4958255,264.69032471)
\curveto(16.49583228,264.64031946)(16.49083229,264.59531951)(16.4808255,264.55532471)
\curveto(16.46083232,264.49531961)(16.45083233,264.42031968)(16.4508255,264.33032471)
\curveto(16.45083233,264.24031986)(16.46083232,264.16531994)(16.4808255,264.10532471)
\lineto(16.4808255,264.01532471)
\curveto(16.49083229,263.95532015)(16.50083228,263.9003202)(16.5108255,263.85032471)
\curveto(16.51083227,263.8003203)(16.51583226,263.75032035)(16.5258255,263.70032471)
\curveto(16.58583219,263.43032067)(16.67083211,263.19532091)(16.7808255,262.99532471)
\curveto(16.89083189,262.8053213)(17.0758317,262.65532145)(17.3358255,262.54532471)
\curveto(17.40583137,262.51532159)(17.4758313,262.5003216)(17.5458255,262.50032471)
\curveto(17.61583116,262.5003216)(17.6758311,262.5053216)(17.7258255,262.51532471)
\curveto(17.8758309,262.54532156)(17.98583079,262.59532151)(18.0558255,262.66532471)
\curveto(18.11583066,262.73532137)(18.18583059,262.83032127)(18.2658255,262.95032471)
\curveto(18.36583041,263.09032101)(18.44083034,263.25532085)(18.4908255,263.44532471)
\curveto(18.53083025,263.63532047)(18.5808302,263.82532028)(18.6408255,264.01532471)
\curveto(18.6808301,264.13531997)(18.71083007,264.25531985)(18.7308255,264.37532471)
\curveto(18.75083003,264.5053196)(18.78083,264.63031947)(18.8208255,264.75032471)
\curveto(18.8808299,264.95031915)(18.94082984,265.14531896)(19.0008255,265.33532471)
\curveto(19.05082973,265.52531858)(19.11582966,265.71031839)(19.1958255,265.89032471)
\curveto(19.21582956,265.94031816)(19.23582954,265.98531812)(19.2558255,266.02532471)
\curveto(19.2758295,266.07531803)(19.30082948,266.12531798)(19.3308255,266.17532471)
\curveto(19.45082933,266.34531776)(19.58582919,266.49031761)(19.7358255,266.61032471)
\curveto(19.88582889,266.73031737)(20.0758287,266.82031728)(20.3058255,266.88032471)
\lineto(20.5908255,266.88032471)
\curveto(20.66082812,266.88031722)(20.73582804,266.87531723)(20.8158255,266.86532471)
\curveto(20.88582789,266.85531725)(20.96582781,266.84531726)(21.0558255,266.83532471)
\lineto(21.2058255,266.80532471)
\curveto(21.2758275,266.76531734)(21.34582743,266.73531737)(21.4158255,266.71532471)
\curveto(21.48582729,266.7053174)(21.55582722,266.68531742)(21.6258255,266.65532471)
\curveto(21.73582704,266.6053175)(21.84082694,266.55031755)(21.9408255,266.49032471)
\curveto(22.04082674,266.43031767)(22.13082665,266.36531774)(22.2108255,266.29532471)
\curveto(22.47082631,266.08531802)(22.6808261,265.84031826)(22.8408255,265.56032471)
\curveto(22.99082579,265.28031882)(23.12082566,264.97531913)(23.2308255,264.64532471)
\curveto(23.26082552,264.54531956)(23.2808255,264.44531966)(23.2908255,264.34532471)
\curveto(23.31082547,264.24531986)(23.33582544,264.15031995)(23.3658255,264.06032471)
\curveto(23.38582539,263.95032015)(23.39582538,263.84532026)(23.3958255,263.74532471)
\curveto(23.39582538,263.64532046)(23.40582537,263.54532056)(23.4258255,263.44532471)
\lineto(23.4258255,263.29532471)
\curveto(23.43582534,263.24532086)(23.44082534,263.17532093)(23.4408255,263.08532471)
\curveto(23.44082534,262.99532111)(23.43582534,262.92532118)(23.4258255,262.87532471)
\lineto(23.4258255,262.71032471)
\curveto(23.40582537,262.65032145)(23.39582538,262.58532152)(23.3958255,262.51532471)
\curveto(23.40582537,262.44532166)(23.40082538,262.39032171)(23.3808255,262.35032471)
\curveto(23.37082541,262.3003218)(23.36582541,262.23532187)(23.3658255,262.15532471)
\curveto(23.34582543,262.07532203)(23.32582545,262.0003221)(23.3058255,261.93032471)
\curveto(23.29582548,261.86032224)(23.2758255,261.78532232)(23.2458255,261.70532471)
\curveto(23.14582563,261.41532269)(23.02082576,261.17032293)(22.8708255,260.97032471)
\curveto(22.72082606,260.77032333)(22.52582625,260.61032349)(22.2858255,260.49032471)
\curveto(22.15582662,260.43032367)(22.02082676,260.38032372)(21.8808255,260.34032471)
\curveto(21.74082704,260.31032379)(21.58582719,260.29032381)(21.4158255,260.28032471)
\curveto(21.35582742,260.27032383)(21.28582749,260.27532383)(21.2058255,260.29532471)
\curveto(21.11582766,260.31532379)(21.04582773,260.34032376)(20.9958255,260.37032471)
\curveto(20.95582782,260.41032369)(20.91582786,260.47032363)(20.8758255,260.55032471)
\curveto(20.85582792,260.6003235)(20.84582793,260.67032343)(20.8458255,260.76032471)
\curveto(20.83582794,260.86032324)(20.83582794,260.95032315)(20.8458255,261.03032471)
\curveto(20.85582792,261.12032298)(20.87082791,261.2053229)(20.8908255,261.28532471)
\curveto(20.90082788,261.37532273)(20.91582786,261.43032267)(20.9358255,261.45032471)
\curveto(20.98582779,261.51032259)(21.06082772,261.54032256)(21.1608255,261.54032471)
\curveto(21.25082753,261.55032255)(21.33582744,261.57032253)(21.4158255,261.60032471)
\curveto(21.63582714,261.65032245)(21.80582697,261.75032235)(21.9258255,261.90032471)
\curveto(22.01582676,262.0003221)(22.08582669,262.12032198)(22.1358255,262.26032471)
\curveto(22.18582659,262.4003217)(22.23582654,262.55032155)(22.2858255,262.71032471)
\lineto(22.3308255,263.02532471)
\lineto(22.3308255,263.11532471)
\curveto(22.35082643,263.17532093)(22.36082642,263.26032084)(22.3608255,263.37032471)
\curveto(22.36082642,263.49032061)(22.35082643,263.59532051)(22.3308255,263.68532471)
\curveto(22.33082645,263.75532035)(22.32582645,263.81032029)(22.3158255,263.85032471)
\curveto(22.30582647,263.91032019)(22.30082648,263.97032013)(22.3008255,264.03032471)
\curveto(22.29082649,264.09032001)(22.2808265,264.14531996)(22.2708255,264.19532471)
\curveto(22.19082659,264.5053196)(22.08582669,264.75531935)(21.9558255,264.94532471)
\curveto(21.82582695,265.14531896)(21.60582717,265.31031879)(21.2958255,265.44032471)
\curveto(21.24582753,265.47031863)(21.19082759,265.48531862)(21.1308255,265.48532471)
\curveto(21.07082771,265.49531861)(21.02582775,265.49531861)(20.9958255,265.48532471)
\curveto(20.80582797,265.47531863)(20.66582811,265.43531867)(20.5758255,265.36532471)
\curveto(20.4758283,265.29531881)(20.38582839,265.2003189)(20.3058255,265.08032471)
\curveto(20.24582853,265.0003191)(20.19582858,264.9053192)(20.1558255,264.79532471)
\lineto(20.0358255,264.49532471)
\curveto(20.02582875,264.46531964)(20.02082876,264.43531967)(20.0208255,264.40532471)
\curveto(20.02082876,264.38531972)(20.01082877,264.36531974)(19.9908255,264.34532471)
\curveto(19.8808289,264.02532008)(19.80082898,263.68532042)(19.7508255,263.32532471)
\curveto(19.69082909,262.97532113)(19.59582918,262.65532145)(19.4658255,262.36532471)
\curveto(19.42582935,262.27532183)(19.39082939,262.18532192)(19.3608255,262.09532471)
\curveto(19.33082945,262.01532209)(19.29082949,261.94032216)(19.2408255,261.87032471)
\curveto(19.13082965,261.7003224)(19.00582977,261.55032255)(18.8658255,261.42032471)
\curveto(18.72583005,261.29032281)(18.55083023,261.2003229)(18.3408255,261.15032471)
\curveto(18.27083051,261.13032297)(18.20083058,261.12032298)(18.1308255,261.12032471)
\lineto(17.9058255,261.12032471)
\curveto(17.78583099,261.11032299)(17.65083113,261.12532298)(17.5008255,261.16532471)
\curveto(17.34083144,261.2053229)(17.20583157,261.24532286)(17.0958255,261.28532471)
\curveto(17.04583173,261.31532279)(17.00583177,261.33532277)(16.9758255,261.34532471)
\curveto(16.93583184,261.36532274)(16.89583188,261.39032271)(16.8558255,261.42032471)
\curveto(16.62583215,261.55032255)(16.42583235,261.71032239)(16.2558255,261.90032471)
\curveto(16.08583269,262.09032201)(15.93583284,262.3003218)(15.8058255,262.53032471)
\curveto(15.71583306,262.69032141)(15.64583313,262.86532124)(15.5958255,263.05532471)
\curveto(15.53583324,263.25532085)(15.4808333,263.46032064)(15.4308255,263.67032471)
\curveto(15.42083336,263.74032036)(15.41083337,263.8053203)(15.4008255,263.86532471)
\curveto(15.39083339,263.93532017)(15.3808334,264.01032009)(15.3708255,264.09032471)
\curveto(15.36083342,264.13031997)(15.36083342,264.17031993)(15.3708255,264.21032471)
\curveto(15.3808334,264.26031984)(15.3758334,264.3003198)(15.3558255,264.33032471)
}
}
{
\newrgbcolor{curcolor}{0 0 0}
\pscustom[linestyle=none,fillstyle=solid,fillcolor=curcolor]
{
\newpath
\moveto(118.89923523,72.61)
\curveto(118.9192261,72.52999222)(118.92922609,72.41999233)(118.92923523,72.28)
\curveto(118.92922609,72.1499926)(118.9192261,72.0499927)(118.89923523,71.98)
\curveto(118.87922614,71.90999284)(118.87422615,71.84499291)(118.88423523,71.785)
\curveto(118.89422613,71.72499303)(118.88922613,71.65999309)(118.86923523,71.59)
\curveto(118.84922617,71.52999322)(118.83422619,71.46499328)(118.82423523,71.395)
\curveto(118.81422621,71.33499341)(118.79922622,71.27499347)(118.77923523,71.215)
\curveto(118.75922626,71.13499361)(118.73422629,71.05999369)(118.70423523,70.99)
\curveto(118.68422634,70.91999383)(118.65922636,70.8499939)(118.62923523,70.78)
\curveto(118.60922641,70.749994)(118.59422643,70.71999403)(118.58423523,70.69)
\curveto(118.58422644,70.66999408)(118.57422645,70.6499941)(118.55423523,70.63)
\curveto(118.44422658,70.42999432)(118.3242267,70.2499945)(118.19423523,70.09)
\curveto(118.17422685,70.0499947)(118.13922688,70.00999474)(118.08923523,69.97)
\curveto(118.04922697,69.92999482)(118.01422701,69.89999485)(117.98423523,69.88)
\curveto(117.94422708,69.85999489)(117.90922711,69.82999492)(117.87923523,69.79)
\curveto(117.84922717,69.75999499)(117.8192272,69.73499501)(117.78923523,69.715)
\lineto(117.47423523,69.535)
\curveto(117.36422766,69.4549953)(117.23422779,69.39499535)(117.08423523,69.355)
\lineto(116.63423523,69.235)
\curveto(116.55422847,69.21499554)(116.47422855,69.19999555)(116.39423523,69.19)
\curveto(116.31422871,69.18999556)(116.23422879,69.17999557)(116.15423523,69.16)
\curveto(116.11422891,69.1499956)(116.07422895,69.14499561)(116.03423523,69.145)
\curveto(116.00422902,69.1549956)(115.97422905,69.1549956)(115.94423523,69.145)
\curveto(115.89422913,69.13499561)(115.84422918,69.13499561)(115.79423523,69.145)
\curveto(115.75422927,69.1549956)(115.70922931,69.1549956)(115.65923523,69.145)
\lineto(113.39423523,69.145)
\lineto(112.89923523,69.145)
\curveto(112.72923229,69.1549956)(112.59923242,69.12499562)(112.50923523,69.055)
\curveto(112.39923262,68.97499578)(112.34423268,68.82999592)(112.34423523,68.62)
\curveto(112.35423267,68.40999634)(112.35923266,68.21499654)(112.35923523,68.035)
\lineto(112.35923523,65.83)
\lineto(112.35923523,65.335)
\curveto(112.36923265,65.14499961)(112.34923267,65.00999974)(112.29923523,64.93)
\curveto(112.25923276,64.86999988)(112.20923281,64.82999992)(112.14923523,64.81)
\curveto(112.09923292,64.79999995)(112.03423299,64.78499996)(111.95423523,64.765)
\lineto(111.68423523,64.765)
\curveto(111.53423349,64.76499998)(111.39923362,64.76999998)(111.27923523,64.78)
\curveto(111.15923386,64.78999996)(111.07423395,64.83999991)(111.02423523,64.93)
\curveto(110.98423404,64.98999976)(110.96423406,65.06999968)(110.96423523,65.17)
\lineto(110.96423523,65.485)
\lineto(110.96423523,74.59)
\curveto(110.96423406,74.69999005)(110.95923406,74.81998993)(110.94923523,74.95)
\curveto(110.94923407,75.08998966)(110.97423405,75.19998955)(111.02423523,75.28)
\curveto(111.06423396,75.33998941)(111.13923388,75.38998936)(111.24923523,75.43)
\curveto(111.26923375,75.43998931)(111.28923373,75.43998931)(111.30923523,75.43)
\curveto(111.32923369,75.42998932)(111.34923367,75.43498931)(111.36923523,75.445)
\lineto(114.77423523,75.445)
\curveto(115.15422987,75.4449893)(115.5242295,75.43998931)(115.88423523,75.43)
\curveto(116.25422877,75.42998932)(116.58422844,75.38498937)(116.87423523,75.295)
\curveto(117.3242277,75.1449896)(117.68922733,74.9499898)(117.96923523,74.71)
\curveto(118.24922677,74.46999028)(118.47922654,74.13999061)(118.65923523,73.72)
\curveto(118.70922631,73.60999114)(118.74422628,73.49499126)(118.76423523,73.375)
\curveto(118.79422623,73.2549915)(118.82922619,73.12999162)(118.86923523,73)
\curveto(118.88922613,72.92999182)(118.89422613,72.86499188)(118.88423523,72.805)
\curveto(118.87422615,72.744992)(118.87922614,72.67999207)(118.89923523,72.61)
\moveto(117.48923523,72.07)
\curveto(117.52922749,72.20999254)(117.53422749,72.36999238)(117.50423523,72.55)
\curveto(117.47422755,72.73999201)(117.44422758,72.88999186)(117.41423523,73)
\curveto(117.31422771,73.27999147)(117.17922784,73.49999125)(117.00923523,73.66)
\curveto(116.84922817,73.82999092)(116.63922838,73.96999078)(116.37923523,74.08)
\curveto(116.15922886,74.16999058)(115.90422912,74.22499053)(115.61423523,74.245)
\curveto(115.33422969,74.26499048)(115.03922998,74.27499047)(114.72923523,74.275)
\lineto(112.79423523,74.275)
\curveto(112.77423225,74.26499048)(112.74923227,74.25999049)(112.71923523,74.26)
\curveto(112.69923232,74.25999049)(112.67423235,74.2549905)(112.64423523,74.245)
\curveto(112.5242325,74.21499054)(112.44423258,74.1499906)(112.40423523,74.05)
\curveto(112.36423266,73.9499908)(112.34423268,73.81499094)(112.34423523,73.645)
\curveto(112.35423267,73.48499127)(112.35923266,73.33499141)(112.35923523,73.195)
\lineto(112.35923523,71.395)
\curveto(112.35923266,71.24499351)(112.35423267,71.07999367)(112.34423523,70.9)
\curveto(112.34423268,70.71999403)(112.37423265,70.57999417)(112.43423523,70.48)
\curveto(112.48423254,70.39999435)(112.55923246,70.3499944)(112.65923523,70.33)
\curveto(112.76923225,70.31999443)(112.88923213,70.31499444)(113.01923523,70.315)
\lineto(115.04423523,70.315)
\lineto(115.50923523,70.315)
\curveto(115.66922935,70.32499442)(115.80922921,70.34499441)(115.92923523,70.375)
\curveto(116.19922882,70.44499431)(116.43422859,70.52499423)(116.63423523,70.615)
\curveto(116.84422818,70.71499404)(117.019228,70.86499388)(117.15923523,71.065)
\curveto(117.23922778,71.18499357)(117.29922772,71.30999344)(117.33923523,71.44)
\curveto(117.38922763,71.56999318)(117.43422759,71.71499304)(117.47423523,71.875)
\curveto(117.48422754,71.91499284)(117.48922753,71.97999277)(117.48923523,72.07)
}
}
{
\newrgbcolor{curcolor}{0 0 0}
\pscustom[linestyle=none,fillstyle=solid,fillcolor=curcolor]
{
\newpath
\moveto(127.56079773,68.965)
\curveto(127.58078967,68.90499584)(127.59078966,68.80999594)(127.59079773,68.68)
\curveto(127.59078966,68.55999619)(127.58578966,68.47499628)(127.57579773,68.425)
\lineto(127.57579773,68.275)
\curveto(127.56578968,68.19499655)(127.55578969,68.11999663)(127.54579773,68.05)
\curveto(127.5457897,67.98999676)(127.54078971,67.91999683)(127.53079773,67.84)
\curveto(127.51078974,67.77999697)(127.49578975,67.71999703)(127.48579773,67.66)
\curveto(127.48578976,67.59999715)(127.47578977,67.53999721)(127.45579773,67.48)
\curveto(127.41578983,67.3499974)(127.38078987,67.21999753)(127.35079773,67.09)
\curveto(127.32078993,66.95999779)(127.28078997,66.83999791)(127.23079773,66.73)
\curveto(127.02079023,66.2499985)(126.74079051,65.84499891)(126.39079773,65.515)
\curveto(126.04079121,65.19499955)(125.61079164,64.9499998)(125.10079773,64.78)
\curveto(124.99079226,64.74000001)(124.87079238,64.71000004)(124.74079773,64.69)
\curveto(124.62079263,64.67000008)(124.49579275,64.6500001)(124.36579773,64.63)
\curveto(124.30579294,64.62000013)(124.24079301,64.61500014)(124.17079773,64.615)
\curveto(124.11079314,64.60500015)(124.0507932,64.60000015)(123.99079773,64.6)
\curveto(123.9507933,64.59000016)(123.89079336,64.58500016)(123.81079773,64.585)
\curveto(123.74079351,64.58500016)(123.69079356,64.59000016)(123.66079773,64.6)
\curveto(123.62079363,64.61000014)(123.58079367,64.61500014)(123.54079773,64.615)
\curveto(123.50079375,64.60500015)(123.46579378,64.60500015)(123.43579773,64.615)
\lineto(123.34579773,64.615)
\lineto(122.98579773,64.66)
\curveto(122.8457944,64.70000005)(122.71079454,64.74000001)(122.58079773,64.78)
\curveto(122.4507948,64.81999993)(122.32579492,64.86499988)(122.20579773,64.915)
\curveto(121.75579549,65.11499964)(121.38579586,65.37499938)(121.09579773,65.695)
\curveto(120.80579644,66.01499874)(120.56579668,66.40499835)(120.37579773,66.865)
\curveto(120.32579692,66.96499778)(120.28579696,67.06499768)(120.25579773,67.165)
\curveto(120.23579701,67.26499748)(120.21579703,67.36999738)(120.19579773,67.48)
\curveto(120.17579707,67.51999723)(120.16579708,67.5499972)(120.16579773,67.57)
\curveto(120.17579707,67.59999715)(120.17579707,67.63499711)(120.16579773,67.675)
\curveto(120.1457971,67.75499699)(120.13079712,67.83499691)(120.12079773,67.915)
\curveto(120.12079713,68.00499675)(120.11079714,68.08999666)(120.09079773,68.17)
\lineto(120.09079773,68.29)
\curveto(120.09079716,68.32999642)(120.08579716,68.37499638)(120.07579773,68.425)
\curveto(120.06579718,68.47499628)(120.06079719,68.55999619)(120.06079773,68.68)
\curveto(120.06079719,68.80999594)(120.07079718,68.90499584)(120.09079773,68.965)
\curveto(120.11079714,69.03499571)(120.11579713,69.10499564)(120.10579773,69.175)
\curveto(120.09579715,69.24499551)(120.10079715,69.31499544)(120.12079773,69.385)
\curveto(120.13079712,69.43499531)(120.13579711,69.47499528)(120.13579773,69.505)
\curveto(120.1457971,69.54499521)(120.15579709,69.58999516)(120.16579773,69.64)
\curveto(120.19579705,69.75999499)(120.22079703,69.87999487)(120.24079773,70)
\curveto(120.27079698,70.11999463)(120.31079694,70.23499451)(120.36079773,70.345)
\curveto(120.51079674,70.71499404)(120.69079656,71.04499371)(120.90079773,71.335)
\curveto(121.12079613,71.63499311)(121.38579586,71.88499287)(121.69579773,72.085)
\curveto(121.81579543,72.16499258)(121.94079531,72.22999252)(122.07079773,72.28)
\curveto(122.20079505,72.33999241)(122.33579491,72.39999235)(122.47579773,72.46)
\curveto(122.59579465,72.50999224)(122.72579452,72.53999221)(122.86579773,72.55)
\curveto(123.00579424,72.56999218)(123.1457941,72.59999215)(123.28579773,72.64)
\lineto(123.48079773,72.64)
\curveto(123.5507937,72.6499921)(123.61579363,72.65999209)(123.67579773,72.67)
\curveto(124.56579268,72.67999207)(125.30579194,72.49499225)(125.89579773,72.115)
\curveto(126.48579076,71.73499301)(126.91079034,71.23999351)(127.17079773,70.63)
\curveto(127.22079003,70.52999422)(127.26078999,70.42999432)(127.29079773,70.33)
\curveto(127.32078993,70.22999452)(127.35578989,70.12499462)(127.39579773,70.015)
\curveto(127.42578982,69.90499484)(127.4507898,69.78499497)(127.47079773,69.655)
\curveto(127.49078976,69.53499521)(127.51578973,69.40999534)(127.54579773,69.28)
\curveto(127.55578969,69.22999552)(127.55578969,69.17499558)(127.54579773,69.115)
\curveto(127.5457897,69.06499568)(127.5507897,69.01499574)(127.56079773,68.965)
\moveto(126.22579773,68.11)
\curveto(126.245791,68.17999657)(126.250791,68.25999649)(126.24079773,68.35)
\lineto(126.24079773,68.605)
\curveto(126.24079101,68.99499575)(126.20579104,69.32499542)(126.13579773,69.595)
\curveto(126.10579114,69.67499508)(126.08079117,69.754995)(126.06079773,69.835)
\curveto(126.04079121,69.91499484)(126.01579123,69.98999476)(125.98579773,70.06)
\curveto(125.70579154,70.70999404)(125.26079199,71.15999359)(124.65079773,71.41)
\curveto(124.58079267,71.43999331)(124.50579274,71.45999329)(124.42579773,71.47)
\lineto(124.18579773,71.53)
\curveto(124.10579314,71.5499932)(124.02079323,71.55999319)(123.93079773,71.56)
\lineto(123.66079773,71.56)
\lineto(123.39079773,71.515)
\curveto(123.29079396,71.49499325)(123.19579405,71.46999328)(123.10579773,71.44)
\curveto(123.02579422,71.41999333)(122.9457943,71.38999336)(122.86579773,71.35)
\curveto(122.79579445,71.32999342)(122.73079452,71.29999345)(122.67079773,71.26)
\curveto(122.61079464,71.21999353)(122.55579469,71.17999357)(122.50579773,71.14)
\curveto(122.26579498,70.96999378)(122.07079518,70.76499398)(121.92079773,70.525)
\curveto(121.77079548,70.28499447)(121.64079561,70.00499474)(121.53079773,69.685)
\curveto(121.50079575,69.58499517)(121.48079577,69.47999527)(121.47079773,69.37)
\curveto(121.46079579,69.26999548)(121.4457958,69.16499558)(121.42579773,69.055)
\curveto(121.41579583,69.01499574)(121.41079584,68.9499958)(121.41079773,68.86)
\curveto(121.40079585,68.82999592)(121.39579585,68.79499595)(121.39579773,68.755)
\curveto(121.40579584,68.71499604)(121.41079584,68.66999608)(121.41079773,68.62)
\lineto(121.41079773,68.32)
\curveto(121.41079584,68.21999653)(121.42079583,68.12999662)(121.44079773,68.05)
\lineto(121.47079773,67.87)
\curveto(121.49079576,67.76999698)(121.50579574,67.66999708)(121.51579773,67.57)
\curveto(121.53579571,67.47999727)(121.56579568,67.39499735)(121.60579773,67.315)
\curveto(121.70579554,67.07499768)(121.82079543,66.8499979)(121.95079773,66.64)
\curveto(122.09079516,66.42999832)(122.26079499,66.25499849)(122.46079773,66.115)
\curveto(122.51079474,66.08499867)(122.55579469,66.05999869)(122.59579773,66.04)
\curveto(122.63579461,66.01999873)(122.68079457,65.99499875)(122.73079773,65.965)
\curveto(122.81079444,65.91499884)(122.89579435,65.86999888)(122.98579773,65.83)
\curveto(123.08579416,65.79999895)(123.19079406,65.76999898)(123.30079773,65.74)
\curveto(123.3507939,65.71999903)(123.39579385,65.70999904)(123.43579773,65.71)
\curveto(123.48579376,65.71999903)(123.53579371,65.71999903)(123.58579773,65.71)
\curveto(123.61579363,65.69999905)(123.67579357,65.68999906)(123.76579773,65.68)
\curveto(123.86579338,65.66999908)(123.94079331,65.67499908)(123.99079773,65.695)
\curveto(124.03079322,65.70499905)(124.07079318,65.70499905)(124.11079773,65.695)
\curveto(124.1507931,65.69499905)(124.19079306,65.70499905)(124.23079773,65.725)
\curveto(124.31079294,65.74499901)(124.39079286,65.75999899)(124.47079773,65.77)
\curveto(124.5507927,65.78999896)(124.62579262,65.81499894)(124.69579773,65.845)
\curveto(125.03579221,65.98499877)(125.31079194,66.17999857)(125.52079773,66.43)
\curveto(125.73079152,66.67999807)(125.90579134,66.97499778)(126.04579773,67.315)
\curveto(126.09579115,67.43499731)(126.12579112,67.55999719)(126.13579773,67.69)
\curveto(126.15579109,67.82999692)(126.18579106,67.96999678)(126.22579773,68.11)
}
}
{
\newrgbcolor{curcolor}{0 0 0}
\pscustom[linestyle=none,fillstyle=solid,fillcolor=curcolor]
{
\newpath
\moveto(132.69407898,72.67)
\curveto(132.92407419,72.66999208)(133.05407406,72.60999214)(133.08407898,72.49)
\curveto(133.114074,72.37999237)(133.12907398,72.21499254)(133.12907898,71.995)
\lineto(133.12907898,71.71)
\curveto(133.12907398,71.61999313)(133.10407401,71.54499321)(133.05407898,71.485)
\curveto(132.99407412,71.40499334)(132.9090742,71.35999339)(132.79907898,71.35)
\curveto(132.68907442,71.3499934)(132.57907453,71.33499341)(132.46907898,71.305)
\curveto(132.32907478,71.27499347)(132.19407492,71.24499351)(132.06407898,71.215)
\curveto(131.94407517,71.18499357)(131.82907528,71.14499361)(131.71907898,71.095)
\curveto(131.42907568,70.96499378)(131.19407592,70.78499397)(131.01407898,70.555)
\curveto(130.83407628,70.33499441)(130.67907643,70.07999467)(130.54907898,69.79)
\curveto(130.5090766,69.67999507)(130.47907663,69.56499518)(130.45907898,69.445)
\curveto(130.43907667,69.33499541)(130.4140767,69.21999553)(130.38407898,69.1)
\curveto(130.37407674,69.0499957)(130.36907674,68.99999575)(130.36907898,68.95)
\curveto(130.37907673,68.89999585)(130.37907673,68.8499959)(130.36907898,68.8)
\curveto(130.33907677,68.67999607)(130.32407679,68.53999621)(130.32407898,68.38)
\curveto(130.33407678,68.22999652)(130.33907677,68.08499667)(130.33907898,67.945)
\lineto(130.33907898,66.1)
\lineto(130.33907898,65.755)
\curveto(130.33907677,65.63499912)(130.33407678,65.51999923)(130.32407898,65.41)
\curveto(130.3140768,65.29999945)(130.3090768,65.20499955)(130.30907898,65.125)
\curveto(130.31907679,65.04499971)(130.29907681,64.97499978)(130.24907898,64.915)
\curveto(130.19907691,64.84499991)(130.11907699,64.80499995)(130.00907898,64.795)
\curveto(129.9090772,64.78499996)(129.79907731,64.77999997)(129.67907898,64.78)
\lineto(129.40907898,64.78)
\curveto(129.35907775,64.79999995)(129.3090778,64.81499994)(129.25907898,64.825)
\curveto(129.21907789,64.84499991)(129.18907792,64.86999988)(129.16907898,64.9)
\curveto(129.11907799,64.96999978)(129.08907802,65.05499969)(129.07907898,65.155)
\lineto(129.07907898,65.485)
\lineto(129.07907898,66.64)
\lineto(129.07907898,70.795)
\lineto(129.07907898,71.83)
\lineto(129.07907898,72.13)
\curveto(129.08907802,72.22999252)(129.11907799,72.31499244)(129.16907898,72.385)
\curveto(129.19907791,72.42499233)(129.24907786,72.4549923)(129.31907898,72.475)
\curveto(129.39907771,72.49499225)(129.48407763,72.50499224)(129.57407898,72.505)
\curveto(129.66407745,72.51499224)(129.75407736,72.51499224)(129.84407898,72.505)
\curveto(129.93407718,72.49499225)(130.00407711,72.47999227)(130.05407898,72.46)
\curveto(130.13407698,72.42999232)(130.18407693,72.36999238)(130.20407898,72.28)
\curveto(130.23407688,72.19999255)(130.24907686,72.10999264)(130.24907898,72.01)
\lineto(130.24907898,71.71)
\curveto(130.24907686,71.60999314)(130.26907684,71.51999323)(130.30907898,71.44)
\curveto(130.31907679,71.41999333)(130.32907678,71.40499334)(130.33907898,71.395)
\lineto(130.38407898,71.35)
\curveto(130.49407662,71.3499934)(130.58407653,71.39499335)(130.65407898,71.485)
\curveto(130.72407639,71.58499317)(130.78407633,71.66499308)(130.83407898,71.725)
\lineto(130.92407898,71.815)
\curveto(131.0140761,71.92499283)(131.13907597,72.03999271)(131.29907898,72.16)
\curveto(131.45907565,72.27999247)(131.6090755,72.36999238)(131.74907898,72.43)
\curveto(131.83907527,72.47999227)(131.93407518,72.51499224)(132.03407898,72.535)
\curveto(132.13407498,72.56499218)(132.23907487,72.59499215)(132.34907898,72.625)
\curveto(132.4090747,72.63499211)(132.46907464,72.63999211)(132.52907898,72.64)
\curveto(132.58907452,72.6499921)(132.64407447,72.65999209)(132.69407898,72.67)
}
}
{
\newrgbcolor{curcolor}{0 0 0}
\pscustom[linestyle=none,fillstyle=solid,fillcolor=curcolor]
{
\newpath
\moveto(135.0038446,74.83)
\curveto(135.15384259,74.82998992)(135.30384244,74.82498993)(135.4538446,74.815)
\curveto(135.60384214,74.81498993)(135.70884204,74.77498997)(135.7688446,74.695)
\curveto(135.81884193,74.63499011)(135.8438419,74.5499902)(135.8438446,74.44)
\curveto(135.85384189,74.33999041)(135.85884189,74.23499051)(135.8588446,74.125)
\lineto(135.8588446,73.255)
\curveto(135.85884189,73.17499157)(135.85384189,73.08999166)(135.8438446,73)
\curveto(135.8438419,72.91999183)(135.85384189,72.8499919)(135.8738446,72.79)
\curveto(135.91384183,72.6499921)(136.00384174,72.55999219)(136.1438446,72.52)
\curveto(136.19384155,72.50999224)(136.23884151,72.50499224)(136.2788446,72.505)
\lineto(136.4288446,72.505)
\lineto(136.8338446,72.505)
\curveto(136.99384075,72.51499224)(137.10884064,72.50499224)(137.1788446,72.475)
\curveto(137.26884048,72.41499234)(137.32884042,72.3549924)(137.3588446,72.295)
\curveto(137.37884037,72.2549925)(137.38884036,72.20999254)(137.3888446,72.16)
\lineto(137.3888446,72.01)
\curveto(137.38884036,71.89999285)(137.38384036,71.79499295)(137.3738446,71.695)
\curveto(137.36384038,71.60499314)(137.32884042,71.53499321)(137.2688446,71.485)
\curveto(137.20884054,71.43499331)(137.12384062,71.40499334)(137.0138446,71.395)
\lineto(136.6838446,71.395)
\curveto(136.57384117,71.40499334)(136.46384128,71.40999334)(136.3538446,71.41)
\curveto(136.2438415,71.40999334)(136.1488416,71.39499335)(136.0688446,71.365)
\curveto(135.99884175,71.33499341)(135.9488418,71.28499347)(135.9188446,71.215)
\curveto(135.88884186,71.14499361)(135.86884188,71.05999369)(135.8588446,70.96)
\curveto(135.8488419,70.86999388)(135.8438419,70.76999398)(135.8438446,70.66)
\curveto(135.85384189,70.55999419)(135.85884189,70.45999429)(135.8588446,70.36)
\lineto(135.8588446,67.39)
\curveto(135.85884189,67.16999758)(135.85384189,66.93499781)(135.8438446,66.685)
\curveto(135.8438419,66.44499831)(135.88884186,66.25999849)(135.9788446,66.13)
\curveto(136.02884172,66.0499987)(136.09384165,65.99499875)(136.1738446,65.965)
\curveto(136.25384149,65.93499881)(136.3488414,65.90999884)(136.4588446,65.89)
\curveto(136.48884126,65.87999887)(136.51884123,65.87499888)(136.5488446,65.875)
\curveto(136.58884116,65.88499887)(136.62384112,65.88499887)(136.6538446,65.875)
\lineto(136.8488446,65.875)
\curveto(136.9488408,65.87499888)(137.03884071,65.86499888)(137.1188446,65.845)
\curveto(137.20884054,65.83499892)(137.27384047,65.79999895)(137.3138446,65.74)
\curveto(137.33384041,65.70999904)(137.3488404,65.65499909)(137.3588446,65.575)
\curveto(137.37884037,65.50499925)(137.38884036,65.42999932)(137.3888446,65.35)
\curveto(137.39884035,65.26999948)(137.39884035,65.18999956)(137.3888446,65.11)
\curveto(137.37884037,65.03999971)(137.35884039,64.98499976)(137.3288446,64.945)
\curveto(137.28884046,64.87499988)(137.21384053,64.82499992)(137.1038446,64.795)
\curveto(137.02384072,64.77499998)(136.93384081,64.76499998)(136.8338446,64.765)
\curveto(136.73384101,64.77499998)(136.6438411,64.77999997)(136.5638446,64.78)
\curveto(136.50384124,64.77999997)(136.4438413,64.77499998)(136.3838446,64.765)
\curveto(136.32384142,64.76499998)(136.26884148,64.76999998)(136.2188446,64.78)
\lineto(136.0388446,64.78)
\curveto(135.98884176,64.78999996)(135.93884181,64.79499995)(135.8888446,64.795)
\curveto(135.8488419,64.80499995)(135.80384194,64.80999994)(135.7538446,64.81)
\curveto(135.55384219,64.85999989)(135.37884237,64.91499984)(135.2288446,64.975)
\curveto(135.08884266,65.03499972)(134.96884278,65.13999961)(134.8688446,65.29)
\curveto(134.72884302,65.48999926)(134.6488431,65.73999901)(134.6288446,66.04)
\curveto(134.60884314,66.3499984)(134.59884315,66.67999807)(134.5988446,67.03)
\lineto(134.5988446,70.96)
\curveto(134.56884318,71.08999366)(134.53884321,71.18499357)(134.5088446,71.245)
\curveto(134.48884326,71.30499344)(134.41884333,71.3549934)(134.2988446,71.395)
\curveto(134.25884349,71.40499334)(134.21884353,71.40499334)(134.1788446,71.395)
\curveto(134.13884361,71.38499337)(134.09884365,71.38999336)(134.0588446,71.41)
\lineto(133.8188446,71.41)
\curveto(133.68884406,71.40999334)(133.57884417,71.41999333)(133.4888446,71.44)
\curveto(133.40884434,71.46999328)(133.35384439,71.52999322)(133.3238446,71.62)
\curveto(133.30384444,71.65999309)(133.28884446,71.70499304)(133.2788446,71.755)
\lineto(133.2788446,71.905)
\curveto(133.27884447,72.04499271)(133.28884446,72.15999259)(133.3088446,72.25)
\curveto(133.32884442,72.3499924)(133.38884436,72.42499233)(133.4888446,72.475)
\curveto(133.59884415,72.51499224)(133.73884401,72.52499223)(133.9088446,72.505)
\curveto(134.08884366,72.48499227)(134.23884351,72.49499225)(134.3588446,72.535)
\curveto(134.4488433,72.58499217)(134.51884323,72.6549921)(134.5688446,72.745)
\curveto(134.58884316,72.80499194)(134.59884315,72.87999187)(134.5988446,72.97)
\lineto(134.5988446,73.225)
\lineto(134.5988446,74.155)
\lineto(134.5988446,74.395)
\curveto(134.59884315,74.48499027)(134.60884314,74.55999019)(134.6288446,74.62)
\curveto(134.66884308,74.69999005)(134.743843,74.76498998)(134.8538446,74.815)
\curveto(134.88384286,74.81498993)(134.90884284,74.81498993)(134.9288446,74.815)
\curveto(134.95884279,74.82498993)(134.98384276,74.82998992)(135.0038446,74.83)
}
}
{
\newrgbcolor{curcolor}{0 0 0}
\pscustom[linestyle=none,fillstyle=solid,fillcolor=curcolor]
{
\newpath
\moveto(145.66064148,65.32)
\curveto(145.69063365,65.15999959)(145.67563366,65.02499972)(145.61564148,64.915)
\curveto(145.55563378,64.81499994)(145.47563386,64.74000001)(145.37564148,64.69)
\curveto(145.32563401,64.67000008)(145.27063407,64.66000009)(145.21064148,64.66)
\curveto(145.16063418,64.66000009)(145.10563423,64.6500001)(145.04564148,64.63)
\curveto(144.82563451,64.58000017)(144.60563473,64.59500015)(144.38564148,64.675)
\curveto(144.17563516,64.74500001)(144.03063531,64.83499992)(143.95064148,64.945)
\curveto(143.90063544,65.01499974)(143.85563548,65.09499965)(143.81564148,65.185)
\curveto(143.77563556,65.28499946)(143.72563561,65.36499938)(143.66564148,65.425)
\curveto(143.64563569,65.44499931)(143.62063572,65.46499928)(143.59064148,65.485)
\curveto(143.57063577,65.50499925)(143.5406358,65.50999924)(143.50064148,65.5)
\curveto(143.39063595,65.46999928)(143.28563605,65.41499934)(143.18564148,65.335)
\curveto(143.09563624,65.25499949)(143.00563633,65.18499956)(142.91564148,65.125)
\curveto(142.78563655,65.04499971)(142.64563669,64.96999978)(142.49564148,64.9)
\curveto(142.34563699,64.83999991)(142.18563715,64.78499996)(142.01564148,64.735)
\curveto(141.91563742,64.70500005)(141.80563753,64.68500006)(141.68564148,64.675)
\curveto(141.57563776,64.66500008)(141.46563787,64.6500001)(141.35564148,64.63)
\curveto(141.30563803,64.62000013)(141.26063808,64.61500014)(141.22064148,64.615)
\lineto(141.11564148,64.615)
\curveto(141.00563833,64.59500015)(140.90063844,64.59500015)(140.80064148,64.615)
\lineto(140.66564148,64.615)
\curveto(140.61563872,64.62500012)(140.56563877,64.63000012)(140.51564148,64.63)
\curveto(140.46563887,64.63000012)(140.42063892,64.64000011)(140.38064148,64.66)
\curveto(140.340639,64.67000008)(140.30563903,64.67500008)(140.27564148,64.675)
\curveto(140.25563908,64.66500008)(140.23063911,64.66500008)(140.20064148,64.675)
\lineto(139.96064148,64.735)
\curveto(139.88063946,64.74500001)(139.80563953,64.76499998)(139.73564148,64.795)
\curveto(139.4356399,64.92499982)(139.19064015,65.06999968)(139.00064148,65.23)
\curveto(138.82064052,65.39999935)(138.67064067,65.63499912)(138.55064148,65.935)
\curveto(138.46064088,66.15499859)(138.41564092,66.41999833)(138.41564148,66.73)
\lineto(138.41564148,67.045)
\curveto(138.42564091,67.09499765)(138.43064091,67.14499761)(138.43064148,67.195)
\lineto(138.46064148,67.375)
\lineto(138.58064148,67.705)
\curveto(138.62064072,67.81499694)(138.67064067,67.91499684)(138.73064148,68.005)
\curveto(138.91064043,68.29499645)(139.15564018,68.50999624)(139.46564148,68.65)
\curveto(139.77563956,68.78999596)(140.11563922,68.91499584)(140.48564148,69.025)
\curveto(140.62563871,69.06499568)(140.77063857,69.09499565)(140.92064148,69.115)
\curveto(141.07063827,69.13499561)(141.22063812,69.15999559)(141.37064148,69.19)
\curveto(141.4406379,69.20999554)(141.50563783,69.21999553)(141.56564148,69.22)
\curveto(141.6356377,69.21999553)(141.71063763,69.22999552)(141.79064148,69.25)
\curveto(141.86063748,69.26999548)(141.93063741,69.27999547)(142.00064148,69.28)
\curveto(142.07063727,69.28999546)(142.14563719,69.30499544)(142.22564148,69.325)
\curveto(142.47563686,69.38499537)(142.71063663,69.43499531)(142.93064148,69.475)
\curveto(143.15063619,69.52499522)(143.32563601,69.63999511)(143.45564148,69.82)
\curveto(143.51563582,69.89999485)(143.56563577,69.99999475)(143.60564148,70.12)
\curveto(143.64563569,70.2499945)(143.64563569,70.38999436)(143.60564148,70.54)
\curveto(143.54563579,70.77999397)(143.45563588,70.96999378)(143.33564148,71.11)
\curveto(143.22563611,71.2499935)(143.06563627,71.35999339)(142.85564148,71.44)
\curveto(142.7356366,71.48999326)(142.59063675,71.52499323)(142.42064148,71.545)
\curveto(142.26063708,71.56499318)(142.09063725,71.57499317)(141.91064148,71.575)
\curveto(141.73063761,71.57499317)(141.55563778,71.56499318)(141.38564148,71.545)
\curveto(141.21563812,71.52499323)(141.07063827,71.49499325)(140.95064148,71.455)
\curveto(140.78063856,71.39499335)(140.61563872,71.30999344)(140.45564148,71.2)
\curveto(140.37563896,71.13999361)(140.30063904,71.05999369)(140.23064148,70.96)
\curveto(140.17063917,70.86999388)(140.11563922,70.76999398)(140.06564148,70.66)
\curveto(140.0356393,70.57999417)(140.00563933,70.49499425)(139.97564148,70.405)
\curveto(139.95563938,70.31499444)(139.91063943,70.24499451)(139.84064148,70.195)
\curveto(139.80063954,70.16499458)(139.73063961,70.13999461)(139.63064148,70.12)
\curveto(139.5406398,70.10999464)(139.44563989,70.10499464)(139.34564148,70.105)
\curveto(139.24564009,70.10499464)(139.14564019,70.10999464)(139.04564148,70.12)
\curveto(138.95564038,70.13999461)(138.89064045,70.16499458)(138.85064148,70.195)
\curveto(138.81064053,70.22499452)(138.78064056,70.27499447)(138.76064148,70.345)
\curveto(138.7406406,70.41499434)(138.7406406,70.48999426)(138.76064148,70.57)
\curveto(138.79064055,70.69999405)(138.82064052,70.81999393)(138.85064148,70.93)
\curveto(138.89064045,71.0499937)(138.9356404,71.16499358)(138.98564148,71.275)
\curveto(139.17564016,71.62499313)(139.41563992,71.89499285)(139.70564148,72.085)
\curveto(139.99563934,72.28499247)(140.35563898,72.44499231)(140.78564148,72.565)
\curveto(140.88563845,72.58499217)(140.98563835,72.59999215)(141.08564148,72.61)
\curveto(141.19563814,72.61999213)(141.30563803,72.63499211)(141.41564148,72.655)
\curveto(141.45563788,72.66499208)(141.52063782,72.66499208)(141.61064148,72.655)
\curveto(141.70063764,72.6549921)(141.75563758,72.66499208)(141.77564148,72.685)
\curveto(142.47563686,72.69499206)(143.08563625,72.61499214)(143.60564148,72.445)
\curveto(144.12563521,72.27499247)(144.49063485,71.9499928)(144.70064148,71.47)
\curveto(144.79063455,71.26999348)(144.8406345,71.03499371)(144.85064148,70.765)
\curveto(144.87063447,70.50499424)(144.88063446,70.22999452)(144.88064148,69.94)
\lineto(144.88064148,66.625)
\curveto(144.88063446,66.48499827)(144.88563445,66.3499984)(144.89564148,66.22)
\curveto(144.90563443,66.08999866)(144.9356344,65.98499877)(144.98564148,65.905)
\curveto(145.0356343,65.83499892)(145.10063424,65.78499896)(145.18064148,65.755)
\curveto(145.27063407,65.71499904)(145.35563398,65.68499906)(145.43564148,65.665)
\curveto(145.51563382,65.65499909)(145.57563376,65.60999914)(145.61564148,65.53)
\curveto(145.6356337,65.49999925)(145.64563369,65.46999928)(145.64564148,65.44)
\curveto(145.64563369,65.40999934)(145.65063369,65.36999938)(145.66064148,65.32)
\moveto(143.51564148,66.985)
\curveto(143.57563576,67.12499762)(143.60563573,67.28499747)(143.60564148,67.465)
\curveto(143.61563572,67.65499709)(143.62063572,67.8499969)(143.62064148,68.05)
\curveto(143.62063572,68.15999659)(143.61563572,68.25999649)(143.60564148,68.35)
\curveto(143.59563574,68.43999631)(143.55563578,68.50999624)(143.48564148,68.56)
\curveto(143.45563588,68.57999617)(143.38563595,68.58999616)(143.27564148,68.59)
\curveto(143.25563608,68.56999618)(143.22063612,68.55999619)(143.17064148,68.56)
\curveto(143.12063622,68.55999619)(143.07563626,68.5499962)(143.03564148,68.53)
\curveto(142.95563638,68.50999624)(142.86563647,68.48999626)(142.76564148,68.47)
\lineto(142.46564148,68.41)
\curveto(142.4356369,68.40999634)(142.40063694,68.40499634)(142.36064148,68.395)
\lineto(142.25564148,68.395)
\curveto(142.10563723,68.3549964)(141.9406374,68.32999642)(141.76064148,68.32)
\curveto(141.59063775,68.31999643)(141.43063791,68.29999645)(141.28064148,68.26)
\curveto(141.20063814,68.23999651)(141.12563821,68.21999653)(141.05564148,68.2)
\curveto(140.99563834,68.18999656)(140.92563841,68.17499658)(140.84564148,68.155)
\curveto(140.68563865,68.10499665)(140.5356388,68.03999671)(140.39564148,67.96)
\curveto(140.25563908,67.88999686)(140.1356392,67.79999695)(140.03564148,67.69)
\curveto(139.9356394,67.57999717)(139.86063948,67.44499731)(139.81064148,67.285)
\curveto(139.76063958,67.13499761)(139.7406396,66.9499978)(139.75064148,66.73)
\curveto(139.75063959,66.62999812)(139.76563957,66.53499821)(139.79564148,66.445)
\curveto(139.8356395,66.36499838)(139.88063946,66.28999846)(139.93064148,66.22)
\curveto(140.01063933,66.10999864)(140.11563922,66.01499874)(140.24564148,65.935)
\curveto(140.37563896,65.86499888)(140.51563882,65.80499895)(140.66564148,65.755)
\curveto(140.71563862,65.74499901)(140.76563857,65.73999901)(140.81564148,65.74)
\curveto(140.86563847,65.73999901)(140.91563842,65.73499902)(140.96564148,65.725)
\curveto(141.0356383,65.70499905)(141.12063822,65.68999906)(141.22064148,65.68)
\curveto(141.33063801,65.67999907)(141.42063792,65.68999906)(141.49064148,65.71)
\curveto(141.55063779,65.72999902)(141.61063773,65.73499902)(141.67064148,65.725)
\curveto(141.73063761,65.72499902)(141.79063755,65.73499902)(141.85064148,65.755)
\curveto(141.93063741,65.77499898)(142.00563733,65.78999896)(142.07564148,65.8)
\curveto(142.15563718,65.80999894)(142.23063711,65.82999892)(142.30064148,65.86)
\curveto(142.59063675,65.97999877)(142.8356365,66.12499862)(143.03564148,66.295)
\curveto(143.24563609,66.46499828)(143.40563593,66.69499805)(143.51564148,66.985)
}
}
{
\newrgbcolor{curcolor}{0 0 0}
\pscustom[linestyle=none,fillstyle=solid,fillcolor=curcolor]
{
\newpath
\moveto(153.7922821,65.575)
\lineto(153.7922821,65.185)
\curveto(153.79227423,65.06499968)(153.76727425,64.96499978)(153.7172821,64.885)
\curveto(153.66727435,64.81499994)(153.58227444,64.77499998)(153.4622821,64.765)
\lineto(153.1172821,64.765)
\curveto(153.05727496,64.76499998)(152.99727502,64.75999999)(152.9372821,64.75)
\curveto(152.88727513,64.75)(152.84227518,64.75999999)(152.8022821,64.78)
\curveto(152.71227531,64.79999995)(152.65227537,64.83999991)(152.6222821,64.9)
\curveto(152.58227544,64.9499998)(152.55727546,65.00999974)(152.5472821,65.08)
\curveto(152.54727547,65.1499996)(152.53227549,65.21999953)(152.5022821,65.29)
\curveto(152.49227553,65.30999944)(152.47727554,65.32499942)(152.4572821,65.335)
\curveto(152.44727557,65.35499939)(152.43227559,65.37499938)(152.4122821,65.395)
\curveto(152.31227571,65.40499935)(152.23227579,65.38499936)(152.1722821,65.335)
\curveto(152.1222759,65.28499946)(152.06727595,65.23499952)(152.0072821,65.185)
\curveto(151.80727621,65.03499972)(151.60727641,64.91999983)(151.4072821,64.84)
\curveto(151.22727679,64.75999999)(151.017277,64.70000005)(150.7772821,64.66)
\curveto(150.54727747,64.62000013)(150.30727771,64.60000015)(150.0572821,64.6)
\curveto(149.8172782,64.59000016)(149.57727844,64.60500015)(149.3372821,64.645)
\curveto(149.09727892,64.67500008)(148.88727913,64.73000002)(148.7072821,64.81)
\curveto(148.18727983,65.02999972)(147.76728025,65.32499942)(147.4472821,65.695)
\curveto(147.12728089,66.07499868)(146.87728114,66.54499821)(146.6972821,67.105)
\curveto(146.65728136,67.19499755)(146.62728139,67.28499747)(146.6072821,67.375)
\curveto(146.59728142,67.47499728)(146.57728144,67.57499718)(146.5472821,67.675)
\curveto(146.53728148,67.72499702)(146.53228149,67.77499698)(146.5322821,67.825)
\curveto(146.53228149,67.87499688)(146.52728149,67.92499682)(146.5172821,67.975)
\curveto(146.49728152,68.02499672)(146.48728153,68.07499668)(146.4872821,68.125)
\curveto(146.49728152,68.18499657)(146.49728152,68.23999651)(146.4872821,68.29)
\lineto(146.4872821,68.44)
\curveto(146.46728155,68.48999626)(146.45728156,68.5549962)(146.4572821,68.635)
\curveto(146.45728156,68.71499604)(146.46728155,68.77999597)(146.4872821,68.83)
\lineto(146.4872821,68.995)
\curveto(146.50728151,69.06499568)(146.51228151,69.13499561)(146.5022821,69.205)
\curveto(146.50228152,69.28499547)(146.51228151,69.35999539)(146.5322821,69.43)
\curveto(146.54228148,69.47999527)(146.54728147,69.52499522)(146.5472821,69.565)
\curveto(146.54728147,69.60499514)(146.55228147,69.6499951)(146.5622821,69.7)
\curveto(146.59228143,69.79999495)(146.6172814,69.89499485)(146.6372821,69.985)
\curveto(146.65728136,70.08499467)(146.68228134,70.17999457)(146.7122821,70.27)
\curveto(146.84228118,70.6499941)(147.00728101,70.98999376)(147.2072821,71.29)
\curveto(147.4172806,71.59999315)(147.66728035,71.8549929)(147.9572821,72.055)
\curveto(148.12727989,72.17499257)(148.30227972,72.27499247)(148.4822821,72.355)
\curveto(148.67227935,72.43499231)(148.87727914,72.50499224)(149.0972821,72.565)
\curveto(149.16727885,72.57499217)(149.23227879,72.58499217)(149.2922821,72.595)
\curveto(149.36227866,72.60499214)(149.43227859,72.61999213)(149.5022821,72.64)
\lineto(149.6522821,72.64)
\curveto(149.73227829,72.65999209)(149.84727817,72.66999208)(149.9972821,72.67)
\curveto(150.15727786,72.66999208)(150.27727774,72.65999209)(150.3572821,72.64)
\curveto(150.39727762,72.62999212)(150.45227757,72.62499213)(150.5222821,72.625)
\curveto(150.63227739,72.59499215)(150.74227728,72.56999218)(150.8522821,72.55)
\curveto(150.96227706,72.53999221)(151.06727695,72.50999224)(151.1672821,72.46)
\curveto(151.3172767,72.39999235)(151.45727656,72.33499241)(151.5872821,72.265)
\curveto(151.72727629,72.19499255)(151.85727616,72.11499264)(151.9772821,72.025)
\curveto(152.03727598,71.97499277)(152.09727592,71.91999283)(152.1572821,71.86)
\curveto(152.22727579,71.80999294)(152.3172757,71.79499295)(152.4272821,71.815)
\curveto(152.44727557,71.84499291)(152.46227556,71.86999288)(152.4722821,71.89)
\curveto(152.49227553,71.90999284)(152.50727551,71.93999281)(152.5172821,71.98)
\curveto(152.54727547,72.06999268)(152.55727546,72.18499257)(152.5472821,72.325)
\lineto(152.5472821,72.7)
\lineto(152.5472821,74.425)
\lineto(152.5472821,74.89)
\curveto(152.54727547,75.06998968)(152.57227545,75.19998955)(152.6222821,75.28)
\curveto(152.66227536,75.3499894)(152.7222753,75.39498936)(152.8022821,75.415)
\curveto(152.8222752,75.41498933)(152.84727517,75.41498933)(152.8772821,75.415)
\curveto(152.90727511,75.42498933)(152.93227509,75.42998932)(152.9522821,75.43)
\curveto(153.09227493,75.43998931)(153.23727478,75.43998931)(153.3872821,75.43)
\curveto(153.54727447,75.42998932)(153.65727436,75.38998936)(153.7172821,75.31)
\curveto(153.76727425,75.22998952)(153.79227423,75.12998962)(153.7922821,75.01)
\lineto(153.7922821,74.635)
\lineto(153.7922821,65.575)
\moveto(152.5772821,68.41)
\curveto(152.59727542,68.45999629)(152.60727541,68.52499622)(152.6072821,68.605)
\curveto(152.60727541,68.69499605)(152.59727542,68.76499598)(152.5772821,68.815)
\lineto(152.5772821,69.04)
\curveto(152.55727546,69.12999562)(152.54227548,69.21999553)(152.5322821,69.31)
\curveto(152.5222755,69.40999534)(152.50227552,69.49999525)(152.4722821,69.58)
\curveto(152.45227557,69.65999509)(152.43227559,69.73499501)(152.4122821,69.805)
\curveto(152.40227562,69.87499488)(152.38227564,69.94499481)(152.3522821,70.015)
\curveto(152.23227579,70.31499444)(152.07727594,70.57999417)(151.8872821,70.81)
\curveto(151.69727632,71.03999371)(151.45727656,71.21999353)(151.1672821,71.35)
\curveto(151.06727695,71.39999335)(150.96227706,71.43499331)(150.8522821,71.455)
\curveto(150.75227727,71.48499327)(150.64227738,71.50999324)(150.5222821,71.53)
\curveto(150.44227758,71.5499932)(150.35227767,71.55999319)(150.2522821,71.56)
\lineto(149.9822821,71.56)
\curveto(149.93227809,71.5499932)(149.88727813,71.53999321)(149.8472821,71.53)
\lineto(149.7122821,71.53)
\curveto(149.63227839,71.50999324)(149.54727847,71.48999326)(149.4572821,71.47)
\curveto(149.37727864,71.4499933)(149.29727872,71.42499333)(149.2172821,71.395)
\curveto(148.89727912,71.2549935)(148.63727938,71.0499937)(148.4372821,70.78)
\curveto(148.24727977,70.51999423)(148.09227993,70.21499454)(147.9722821,69.865)
\curveto(147.93228009,69.754995)(147.90228012,69.63999511)(147.8822821,69.52)
\curveto(147.87228015,69.40999534)(147.85728016,69.29999545)(147.8372821,69.19)
\curveto(147.83728018,69.1499956)(147.83228019,69.10999564)(147.8222821,69.07)
\lineto(147.8222821,68.965)
\curveto(147.80228022,68.91499584)(147.79228023,68.85999589)(147.7922821,68.8)
\curveto(147.80228022,68.73999601)(147.80728021,68.68499607)(147.8072821,68.635)
\lineto(147.8072821,68.305)
\curveto(147.80728021,68.20499654)(147.8172802,68.10999664)(147.8372821,68.02)
\curveto(147.84728017,67.98999676)(147.85228017,67.93999681)(147.8522821,67.87)
\curveto(147.87228015,67.79999695)(147.88728013,67.72999702)(147.8972821,67.66)
\lineto(147.9572821,67.45)
\curveto(148.06727995,67.09999765)(148.2172798,66.79999795)(148.4072821,66.55)
\curveto(148.59727942,66.29999845)(148.83727918,66.09499865)(149.1272821,65.935)
\curveto(149.2172788,65.88499887)(149.30727871,65.84499891)(149.3972821,65.815)
\curveto(149.48727853,65.78499896)(149.58727843,65.75499899)(149.6972821,65.725)
\curveto(149.74727827,65.70499905)(149.79727822,65.69999905)(149.8472821,65.71)
\curveto(149.90727811,65.71999903)(149.96227806,65.71499904)(150.0122821,65.695)
\curveto(150.05227797,65.68499906)(150.09227793,65.67999907)(150.1322821,65.68)
\lineto(150.2672821,65.68)
\lineto(150.4022821,65.68)
\curveto(150.43227759,65.68999906)(150.48227754,65.69499905)(150.5522821,65.695)
\curveto(150.63227739,65.71499904)(150.71227731,65.72999902)(150.7922821,65.74)
\curveto(150.87227715,65.75999899)(150.94727707,65.78499896)(151.0172821,65.815)
\curveto(151.34727667,65.95499879)(151.61227641,66.12999862)(151.8122821,66.34)
\curveto(152.022276,66.55999819)(152.19727582,66.83499791)(152.3372821,67.165)
\curveto(152.38727563,67.27499748)(152.4222756,67.38499737)(152.4422821,67.495)
\curveto(152.46227556,67.60499715)(152.48727553,67.71499704)(152.5172821,67.825)
\curveto(152.53727548,67.86499688)(152.54727547,67.89999685)(152.5472821,67.93)
\curveto(152.54727547,67.96999678)(152.55227547,68.00999674)(152.5622821,68.05)
\curveto(152.57227545,68.10999664)(152.57227545,68.16999658)(152.5622821,68.23)
\curveto(152.56227546,68.28999646)(152.56727545,68.3499964)(152.5772821,68.41)
}
}
{
\newrgbcolor{curcolor}{0 0 0}
\pscustom[linestyle=none,fillstyle=solid,fillcolor=curcolor]
{
\newpath
\moveto(162.6235321,65.32)
\curveto(162.65352427,65.15999959)(162.63852429,65.02499972)(162.5785321,64.915)
\curveto(162.51852441,64.81499994)(162.43852449,64.74000001)(162.3385321,64.69)
\curveto(162.28852464,64.67000008)(162.23352469,64.66000009)(162.1735321,64.66)
\curveto(162.1235248,64.66000009)(162.06852486,64.6500001)(162.0085321,64.63)
\curveto(161.78852514,64.58000017)(161.56852536,64.59500015)(161.3485321,64.675)
\curveto(161.13852579,64.74500001)(160.99352593,64.83499992)(160.9135321,64.945)
\curveto(160.86352606,65.01499974)(160.81852611,65.09499965)(160.7785321,65.185)
\curveto(160.73852619,65.28499946)(160.68852624,65.36499938)(160.6285321,65.425)
\curveto(160.60852632,65.44499931)(160.58352634,65.46499928)(160.5535321,65.485)
\curveto(160.53352639,65.50499925)(160.50352642,65.50999924)(160.4635321,65.5)
\curveto(160.35352657,65.46999928)(160.24852668,65.41499934)(160.1485321,65.335)
\curveto(160.05852687,65.25499949)(159.96852696,65.18499956)(159.8785321,65.125)
\curveto(159.74852718,65.04499971)(159.60852732,64.96999978)(159.4585321,64.9)
\curveto(159.30852762,64.83999991)(159.14852778,64.78499996)(158.9785321,64.735)
\curveto(158.87852805,64.70500005)(158.76852816,64.68500006)(158.6485321,64.675)
\curveto(158.53852839,64.66500008)(158.4285285,64.6500001)(158.3185321,64.63)
\curveto(158.26852866,64.62000013)(158.2235287,64.61500014)(158.1835321,64.615)
\lineto(158.0785321,64.615)
\curveto(157.96852896,64.59500015)(157.86352906,64.59500015)(157.7635321,64.615)
\lineto(157.6285321,64.615)
\curveto(157.57852935,64.62500012)(157.5285294,64.63000012)(157.4785321,64.63)
\curveto(157.4285295,64.63000012)(157.38352954,64.64000011)(157.3435321,64.66)
\curveto(157.30352962,64.67000008)(157.26852966,64.67500008)(157.2385321,64.675)
\curveto(157.21852971,64.66500008)(157.19352973,64.66500008)(157.1635321,64.675)
\lineto(156.9235321,64.735)
\curveto(156.84353008,64.74500001)(156.76853016,64.76499998)(156.6985321,64.795)
\curveto(156.39853053,64.92499982)(156.15353077,65.06999968)(155.9635321,65.23)
\curveto(155.78353114,65.39999935)(155.63353129,65.63499912)(155.5135321,65.935)
\curveto(155.4235315,66.15499859)(155.37853155,66.41999833)(155.3785321,66.73)
\lineto(155.3785321,67.045)
\curveto(155.38853154,67.09499765)(155.39353153,67.14499761)(155.3935321,67.195)
\lineto(155.4235321,67.375)
\lineto(155.5435321,67.705)
\curveto(155.58353134,67.81499694)(155.63353129,67.91499684)(155.6935321,68.005)
\curveto(155.87353105,68.29499645)(156.11853081,68.50999624)(156.4285321,68.65)
\curveto(156.73853019,68.78999596)(157.07852985,68.91499584)(157.4485321,69.025)
\curveto(157.58852934,69.06499568)(157.73352919,69.09499565)(157.8835321,69.115)
\curveto(158.03352889,69.13499561)(158.18352874,69.15999559)(158.3335321,69.19)
\curveto(158.40352852,69.20999554)(158.46852846,69.21999553)(158.5285321,69.22)
\curveto(158.59852833,69.21999553)(158.67352825,69.22999552)(158.7535321,69.25)
\curveto(158.8235281,69.26999548)(158.89352803,69.27999547)(158.9635321,69.28)
\curveto(159.03352789,69.28999546)(159.10852782,69.30499544)(159.1885321,69.325)
\curveto(159.43852749,69.38499537)(159.67352725,69.43499531)(159.8935321,69.475)
\curveto(160.11352681,69.52499522)(160.28852664,69.63999511)(160.4185321,69.82)
\curveto(160.47852645,69.89999485)(160.5285264,69.99999475)(160.5685321,70.12)
\curveto(160.60852632,70.2499945)(160.60852632,70.38999436)(160.5685321,70.54)
\curveto(160.50852642,70.77999397)(160.41852651,70.96999378)(160.2985321,71.11)
\curveto(160.18852674,71.2499935)(160.0285269,71.35999339)(159.8185321,71.44)
\curveto(159.69852723,71.48999326)(159.55352737,71.52499323)(159.3835321,71.545)
\curveto(159.2235277,71.56499318)(159.05352787,71.57499317)(158.8735321,71.575)
\curveto(158.69352823,71.57499317)(158.51852841,71.56499318)(158.3485321,71.545)
\curveto(158.17852875,71.52499323)(158.03352889,71.49499325)(157.9135321,71.455)
\curveto(157.74352918,71.39499335)(157.57852935,71.30999344)(157.4185321,71.2)
\curveto(157.33852959,71.13999361)(157.26352966,71.05999369)(157.1935321,70.96)
\curveto(157.13352979,70.86999388)(157.07852985,70.76999398)(157.0285321,70.66)
\curveto(156.99852993,70.57999417)(156.96852996,70.49499425)(156.9385321,70.405)
\curveto(156.91853001,70.31499444)(156.87353005,70.24499451)(156.8035321,70.195)
\curveto(156.76353016,70.16499458)(156.69353023,70.13999461)(156.5935321,70.12)
\curveto(156.50353042,70.10999464)(156.40853052,70.10499464)(156.3085321,70.105)
\curveto(156.20853072,70.10499464)(156.10853082,70.10999464)(156.0085321,70.12)
\curveto(155.91853101,70.13999461)(155.85353107,70.16499458)(155.8135321,70.195)
\curveto(155.77353115,70.22499452)(155.74353118,70.27499447)(155.7235321,70.345)
\curveto(155.70353122,70.41499434)(155.70353122,70.48999426)(155.7235321,70.57)
\curveto(155.75353117,70.69999405)(155.78353114,70.81999393)(155.8135321,70.93)
\curveto(155.85353107,71.0499937)(155.89853103,71.16499358)(155.9485321,71.275)
\curveto(156.13853079,71.62499313)(156.37853055,71.89499285)(156.6685321,72.085)
\curveto(156.95852997,72.28499247)(157.31852961,72.44499231)(157.7485321,72.565)
\curveto(157.84852908,72.58499217)(157.94852898,72.59999215)(158.0485321,72.61)
\curveto(158.15852877,72.61999213)(158.26852866,72.63499211)(158.3785321,72.655)
\curveto(158.41852851,72.66499208)(158.48352844,72.66499208)(158.5735321,72.655)
\curveto(158.66352826,72.6549921)(158.71852821,72.66499208)(158.7385321,72.685)
\curveto(159.43852749,72.69499206)(160.04852688,72.61499214)(160.5685321,72.445)
\curveto(161.08852584,72.27499247)(161.45352547,71.9499928)(161.6635321,71.47)
\curveto(161.75352517,71.26999348)(161.80352512,71.03499371)(161.8135321,70.765)
\curveto(161.83352509,70.50499424)(161.84352508,70.22999452)(161.8435321,69.94)
\lineto(161.8435321,66.625)
\curveto(161.84352508,66.48499827)(161.84852508,66.3499984)(161.8585321,66.22)
\curveto(161.86852506,66.08999866)(161.89852503,65.98499877)(161.9485321,65.905)
\curveto(161.99852493,65.83499892)(162.06352486,65.78499896)(162.1435321,65.755)
\curveto(162.23352469,65.71499904)(162.31852461,65.68499906)(162.3985321,65.665)
\curveto(162.47852445,65.65499909)(162.53852439,65.60999914)(162.5785321,65.53)
\curveto(162.59852433,65.49999925)(162.60852432,65.46999928)(162.6085321,65.44)
\curveto(162.60852432,65.40999934)(162.61352431,65.36999938)(162.6235321,65.32)
\moveto(160.4785321,66.985)
\curveto(160.53852639,67.12499762)(160.56852636,67.28499747)(160.5685321,67.465)
\curveto(160.57852635,67.65499709)(160.58352634,67.8499969)(160.5835321,68.05)
\curveto(160.58352634,68.15999659)(160.57852635,68.25999649)(160.5685321,68.35)
\curveto(160.55852637,68.43999631)(160.51852641,68.50999624)(160.4485321,68.56)
\curveto(160.41852651,68.57999617)(160.34852658,68.58999616)(160.2385321,68.59)
\curveto(160.21852671,68.56999618)(160.18352674,68.55999619)(160.1335321,68.56)
\curveto(160.08352684,68.55999619)(160.03852689,68.5499962)(159.9985321,68.53)
\curveto(159.91852701,68.50999624)(159.8285271,68.48999626)(159.7285321,68.47)
\lineto(159.4285321,68.41)
\curveto(159.39852753,68.40999634)(159.36352756,68.40499634)(159.3235321,68.395)
\lineto(159.2185321,68.395)
\curveto(159.06852786,68.3549964)(158.90352802,68.32999642)(158.7235321,68.32)
\curveto(158.55352837,68.31999643)(158.39352853,68.29999645)(158.2435321,68.26)
\curveto(158.16352876,68.23999651)(158.08852884,68.21999653)(158.0185321,68.2)
\curveto(157.95852897,68.18999656)(157.88852904,68.17499658)(157.8085321,68.155)
\curveto(157.64852928,68.10499665)(157.49852943,68.03999671)(157.3585321,67.96)
\curveto(157.21852971,67.88999686)(157.09852983,67.79999695)(156.9985321,67.69)
\curveto(156.89853003,67.57999717)(156.8235301,67.44499731)(156.7735321,67.285)
\curveto(156.7235302,67.13499761)(156.70353022,66.9499978)(156.7135321,66.73)
\curveto(156.71353021,66.62999812)(156.7285302,66.53499821)(156.7585321,66.445)
\curveto(156.79853013,66.36499838)(156.84353008,66.28999846)(156.8935321,66.22)
\curveto(156.97352995,66.10999864)(157.07852985,66.01499874)(157.2085321,65.935)
\curveto(157.33852959,65.86499888)(157.47852945,65.80499895)(157.6285321,65.755)
\curveto(157.67852925,65.74499901)(157.7285292,65.73999901)(157.7785321,65.74)
\curveto(157.8285291,65.73999901)(157.87852905,65.73499902)(157.9285321,65.725)
\curveto(157.99852893,65.70499905)(158.08352884,65.68999906)(158.1835321,65.68)
\curveto(158.29352863,65.67999907)(158.38352854,65.68999906)(158.4535321,65.71)
\curveto(158.51352841,65.72999902)(158.57352835,65.73499902)(158.6335321,65.725)
\curveto(158.69352823,65.72499902)(158.75352817,65.73499902)(158.8135321,65.755)
\curveto(158.89352803,65.77499898)(158.96852796,65.78999896)(159.0385321,65.8)
\curveto(159.11852781,65.80999894)(159.19352773,65.82999892)(159.2635321,65.86)
\curveto(159.55352737,65.97999877)(159.79852713,66.12499862)(159.9985321,66.295)
\curveto(160.20852672,66.46499828)(160.36852656,66.69499805)(160.4785321,66.985)
}
}
{
\newrgbcolor{curcolor}{0 0 0}
\pscustom[linestyle=none,fillstyle=solid,fillcolor=curcolor]
{
\newpath
\moveto(308.24115601,65.52900879)
\curveto(308.26114646,65.47900804)(308.28614644,65.4190081)(308.31615601,65.34900879)
\curveto(308.34614638,65.27900824)(308.36614636,65.20400832)(308.37615601,65.12400879)
\curveto(308.39614633,65.05400847)(308.39614633,64.98400854)(308.37615601,64.91400879)
\curveto(308.36614636,64.85400867)(308.3261464,64.80900871)(308.25615601,64.77900879)
\curveto(308.20614652,64.75900876)(308.14614658,64.74900877)(308.07615601,64.74900879)
\lineto(307.86615601,64.74900879)
\lineto(307.41615601,64.74900879)
\curveto(307.26614746,64.74900877)(307.14614758,64.77400875)(307.05615601,64.82400879)
\curveto(306.95614777,64.88400864)(306.88114784,64.98900853)(306.83115601,65.13900879)
\curveto(306.79114793,65.28900823)(306.74614798,65.4240081)(306.69615601,65.54400879)
\curveto(306.58614814,65.80400772)(306.48614824,66.07400745)(306.39615601,66.35400879)
\curveto(306.30614842,66.63400689)(306.20614852,66.90900661)(306.09615601,67.17900879)
\curveto(306.06614866,67.26900625)(306.03614869,67.35400617)(306.00615601,67.43400879)
\curveto(305.98614874,67.51400601)(305.95614877,67.58900593)(305.91615601,67.65900879)
\curveto(305.88614884,67.72900579)(305.84114888,67.78900573)(305.78115601,67.83900879)
\curveto(305.721149,67.88900563)(305.64114908,67.92900559)(305.54115601,67.95900879)
\curveto(305.49114923,67.97900554)(305.43114929,67.98400554)(305.36115601,67.97400879)
\lineto(305.16615601,67.97400879)
\lineto(302.33115601,67.97400879)
\lineto(302.03115601,67.97400879)
\curveto(301.9211528,67.98400554)(301.81615291,67.98400554)(301.71615601,67.97400879)
\curveto(301.61615311,67.96400556)(301.5211532,67.94900557)(301.43115601,67.92900879)
\curveto(301.35115337,67.90900561)(301.29115343,67.86900565)(301.25115601,67.80900879)
\curveto(301.17115355,67.70900581)(301.11115361,67.59400593)(301.07115601,67.46400879)
\curveto(301.04115368,67.34400618)(301.00115372,67.2190063)(300.95115601,67.08900879)
\curveto(300.85115387,66.85900666)(300.75615397,66.6190069)(300.66615601,66.36900879)
\curveto(300.58615414,66.1190074)(300.49615423,65.87900764)(300.39615601,65.64900879)
\curveto(300.37615435,65.58900793)(300.35115437,65.519008)(300.32115601,65.43900879)
\curveto(300.30115442,65.36900815)(300.27615445,65.29400823)(300.24615601,65.21400879)
\curveto(300.21615451,65.13400839)(300.18115454,65.05900846)(300.14115601,64.98900879)
\curveto(300.11115461,64.92900859)(300.07615465,64.88400864)(300.03615601,64.85400879)
\curveto(299.95615477,64.79400873)(299.84615488,64.75900876)(299.70615601,64.74900879)
\lineto(299.28615601,64.74900879)
\lineto(299.04615601,64.74900879)
\curveto(298.97615575,64.75900876)(298.91615581,64.78400874)(298.86615601,64.82400879)
\curveto(298.81615591,64.85400867)(298.78615594,64.89900862)(298.77615601,64.95900879)
\curveto(298.77615595,65.0190085)(298.78115594,65.07900844)(298.79115601,65.13900879)
\curveto(298.81115591,65.20900831)(298.83115589,65.27400825)(298.85115601,65.33400879)
\curveto(298.88115584,65.40400812)(298.90615582,65.45400807)(298.92615601,65.48400879)
\curveto(299.06615566,65.80400772)(299.19115553,66.1190074)(299.30115601,66.42900879)
\curveto(299.41115531,66.74900677)(299.53115519,67.06900645)(299.66115601,67.38900879)
\curveto(299.75115497,67.60900591)(299.83615489,67.8240057)(299.91615601,68.03400879)
\curveto(299.99615473,68.25400527)(300.08115464,68.47400505)(300.17115601,68.69400879)
\curveto(300.47115425,69.41400411)(300.75615397,70.13900338)(301.02615601,70.86900879)
\curveto(301.29615343,71.60900191)(301.58115314,72.34400118)(301.88115601,73.07400879)
\curveto(301.99115273,73.33400019)(302.09115263,73.59899992)(302.18115601,73.86900879)
\curveto(302.28115244,74.13899938)(302.38615234,74.40399912)(302.49615601,74.66400879)
\curveto(302.54615218,74.77399875)(302.59115213,74.89399863)(302.63115601,75.02400879)
\curveto(302.68115204,75.16399836)(302.75115197,75.26399826)(302.84115601,75.32400879)
\curveto(302.88115184,75.36399816)(302.94615178,75.39399813)(303.03615601,75.41400879)
\curveto(303.05615167,75.4239981)(303.07615165,75.4239981)(303.09615601,75.41400879)
\curveto(303.1261516,75.41399811)(303.15115157,75.4189981)(303.17115601,75.42900879)
\curveto(303.35115137,75.42899809)(303.56115116,75.42899809)(303.80115601,75.42900879)
\curveto(304.04115068,75.43899808)(304.21615051,75.40399812)(304.32615601,75.32400879)
\curveto(304.40615032,75.26399826)(304.46615026,75.16399836)(304.50615601,75.02400879)
\curveto(304.55615017,74.89399863)(304.60615012,74.77399875)(304.65615601,74.66400879)
\curveto(304.75614997,74.43399909)(304.84614988,74.20399932)(304.92615601,73.97400879)
\curveto(305.00614972,73.74399978)(305.09614963,73.51400001)(305.19615601,73.28400879)
\curveto(305.27614945,73.08400044)(305.35114937,72.87900064)(305.42115601,72.66900879)
\curveto(305.50114922,72.45900106)(305.58614914,72.25400127)(305.67615601,72.05400879)
\curveto(305.97614875,71.3240022)(306.26114846,70.58400294)(306.53115601,69.83400879)
\curveto(306.81114791,69.09400443)(307.10614762,68.35900516)(307.41615601,67.62900879)
\curveto(307.45614727,67.53900598)(307.48614724,67.45400607)(307.50615601,67.37400879)
\curveto(307.53614719,67.29400623)(307.56614716,67.20900631)(307.59615601,67.11900879)
\curveto(307.70614702,66.85900666)(307.81114691,66.59400693)(307.91115601,66.32400879)
\curveto(308.0211467,66.05400747)(308.13114659,65.78900773)(308.24115601,65.52900879)
\moveto(305.03115601,69.17400879)
\curveto(305.1211496,69.20400432)(305.17614955,69.25400427)(305.19615601,69.32400879)
\curveto(305.2261495,69.39400413)(305.23114949,69.46900405)(305.21115601,69.54900879)
\curveto(305.20114952,69.63900388)(305.17614955,69.7240038)(305.13615601,69.80400879)
\curveto(305.10614962,69.89400363)(305.07614965,69.96900355)(305.04615601,70.02900879)
\curveto(305.0261497,70.06900345)(305.01614971,70.10400342)(305.01615601,70.13400879)
\curveto(305.01614971,70.16400336)(305.00614972,70.19900332)(304.98615601,70.23900879)
\lineto(304.89615601,70.47900879)
\curveto(304.87614985,70.56900295)(304.84614988,70.65900286)(304.80615601,70.74900879)
\curveto(304.65615007,71.10900241)(304.5211502,71.47400205)(304.40115601,71.84400879)
\curveto(304.29115043,72.2240013)(304.16115056,72.59400093)(304.01115601,72.95400879)
\curveto(303.96115076,73.06400046)(303.91615081,73.17400035)(303.87615601,73.28400879)
\curveto(303.84615088,73.39400013)(303.80615092,73.49900002)(303.75615601,73.59900879)
\curveto(303.73615099,73.64899987)(303.71115101,73.69399983)(303.68115601,73.73400879)
\curveto(303.66115106,73.78399974)(303.61115111,73.80899971)(303.53115601,73.80900879)
\curveto(303.51115121,73.78899973)(303.49115123,73.77399975)(303.47115601,73.76400879)
\curveto(303.45115127,73.75399977)(303.43115129,73.73899978)(303.41115601,73.71900879)
\curveto(303.37115135,73.66899985)(303.34115138,73.61399991)(303.32115601,73.55400879)
\curveto(303.30115142,73.50400002)(303.28115144,73.44900007)(303.26115601,73.38900879)
\curveto(303.21115151,73.27900024)(303.17115155,73.16900035)(303.14115601,73.05900879)
\curveto(303.11115161,72.94900057)(303.07115165,72.83900068)(303.02115601,72.72900879)
\curveto(302.85115187,72.33900118)(302.70115202,71.94400158)(302.57115601,71.54400879)
\curveto(302.45115227,71.14400238)(302.31115241,70.75400277)(302.15115601,70.37400879)
\lineto(302.09115601,70.22400879)
\curveto(302.08115264,70.17400335)(302.06615266,70.1240034)(302.04615601,70.07400879)
\lineto(301.95615601,69.83400879)
\curveto(301.9261528,69.75400377)(301.90115282,69.67400385)(301.88115601,69.59400879)
\curveto(301.86115286,69.54400398)(301.85115287,69.48900403)(301.85115601,69.42900879)
\curveto(301.86115286,69.36900415)(301.87615285,69.3190042)(301.89615601,69.27900879)
\curveto(301.94615278,69.19900432)(302.05115267,69.15400437)(302.21115601,69.14400879)
\lineto(302.66115601,69.14400879)
\lineto(304.26615601,69.14400879)
\curveto(304.37615035,69.14400438)(304.51115021,69.13900438)(304.67115601,69.12900879)
\curveto(304.83114989,69.12900439)(304.95114977,69.14400438)(305.03115601,69.17400879)
}
}
{
\newrgbcolor{curcolor}{0 0 0}
\pscustom[linestyle=none,fillstyle=solid,fillcolor=curcolor]
{
\newpath
\moveto(313.00271851,72.65400879)
\curveto(313.23271372,72.65400087)(313.36271359,72.59400093)(313.39271851,72.47400879)
\curveto(313.42271353,72.36400116)(313.43771351,72.19900132)(313.43771851,71.97900879)
\lineto(313.43771851,71.69400879)
\curveto(313.43771351,71.60400192)(313.41271354,71.52900199)(313.36271851,71.46900879)
\curveto(313.30271365,71.38900213)(313.21771373,71.34400218)(313.10771851,71.33400879)
\curveto(312.99771395,71.33400219)(312.88771406,71.3190022)(312.77771851,71.28900879)
\curveto(312.63771431,71.25900226)(312.50271445,71.22900229)(312.37271851,71.19900879)
\curveto(312.2527147,71.16900235)(312.13771481,71.12900239)(312.02771851,71.07900879)
\curveto(311.73771521,70.94900257)(311.50271545,70.76900275)(311.32271851,70.53900879)
\curveto(311.14271581,70.3190032)(310.98771596,70.06400346)(310.85771851,69.77400879)
\curveto(310.81771613,69.66400386)(310.78771616,69.54900397)(310.76771851,69.42900879)
\curveto(310.7477162,69.3190042)(310.72271623,69.20400432)(310.69271851,69.08400879)
\curveto(310.68271627,69.03400449)(310.67771627,68.98400454)(310.67771851,68.93400879)
\curveto(310.68771626,68.88400464)(310.68771626,68.83400469)(310.67771851,68.78400879)
\curveto(310.6477163,68.66400486)(310.63271632,68.524005)(310.63271851,68.36400879)
\curveto(310.64271631,68.21400531)(310.6477163,68.06900545)(310.64771851,67.92900879)
\lineto(310.64771851,66.08400879)
\lineto(310.64771851,65.73900879)
\curveto(310.6477163,65.6190079)(310.64271631,65.50400802)(310.63271851,65.39400879)
\curveto(310.62271633,65.28400824)(310.61771633,65.18900833)(310.61771851,65.10900879)
\curveto(310.62771632,65.02900849)(310.60771634,64.95900856)(310.55771851,64.89900879)
\curveto(310.50771644,64.82900869)(310.42771652,64.78900873)(310.31771851,64.77900879)
\curveto(310.21771673,64.76900875)(310.10771684,64.76400876)(309.98771851,64.76400879)
\lineto(309.71771851,64.76400879)
\curveto(309.66771728,64.78400874)(309.61771733,64.79900872)(309.56771851,64.80900879)
\curveto(309.52771742,64.82900869)(309.49771745,64.85400867)(309.47771851,64.88400879)
\curveto(309.42771752,64.95400857)(309.39771755,65.03900848)(309.38771851,65.13900879)
\lineto(309.38771851,65.46900879)
\lineto(309.38771851,66.62400879)
\lineto(309.38771851,70.77900879)
\lineto(309.38771851,71.81400879)
\lineto(309.38771851,72.11400879)
\curveto(309.39771755,72.21400131)(309.42771752,72.29900122)(309.47771851,72.36900879)
\curveto(309.50771744,72.40900111)(309.55771739,72.43900108)(309.62771851,72.45900879)
\curveto(309.70771724,72.47900104)(309.79271716,72.48900103)(309.88271851,72.48900879)
\curveto(309.97271698,72.49900102)(310.06271689,72.49900102)(310.15271851,72.48900879)
\curveto(310.24271671,72.47900104)(310.31271664,72.46400106)(310.36271851,72.44400879)
\curveto(310.44271651,72.41400111)(310.49271646,72.35400117)(310.51271851,72.26400879)
\curveto(310.54271641,72.18400134)(310.55771639,72.09400143)(310.55771851,71.99400879)
\lineto(310.55771851,71.69400879)
\curveto(310.55771639,71.59400193)(310.57771637,71.50400202)(310.61771851,71.42400879)
\curveto(310.62771632,71.40400212)(310.63771631,71.38900213)(310.64771851,71.37900879)
\lineto(310.69271851,71.33400879)
\curveto(310.80271615,71.33400219)(310.89271606,71.37900214)(310.96271851,71.46900879)
\curveto(311.03271592,71.56900195)(311.09271586,71.64900187)(311.14271851,71.70900879)
\lineto(311.23271851,71.79900879)
\curveto(311.32271563,71.90900161)(311.4477155,72.0240015)(311.60771851,72.14400879)
\curveto(311.76771518,72.26400126)(311.91771503,72.35400117)(312.05771851,72.41400879)
\curveto(312.1477148,72.46400106)(312.24271471,72.49900102)(312.34271851,72.51900879)
\curveto(312.44271451,72.54900097)(312.5477144,72.57900094)(312.65771851,72.60900879)
\curveto(312.71771423,72.6190009)(312.77771417,72.6240009)(312.83771851,72.62400879)
\curveto(312.89771405,72.63400089)(312.952714,72.64400088)(313.00271851,72.65400879)
}
}
{
\newrgbcolor{curcolor}{0 0 0}
\pscustom[linestyle=none,fillstyle=solid,fillcolor=curcolor]
{
\newpath
\moveto(321.11748413,68.91900879)
\curveto(321.13747645,68.8190047)(321.13747645,68.70400482)(321.11748413,68.57400879)
\curveto(321.10747648,68.45400507)(321.07747651,68.36900515)(321.02748413,68.31900879)
\curveto(320.97747661,68.27900524)(320.90247668,68.24900527)(320.80248413,68.22900879)
\curveto(320.71247687,68.2190053)(320.60747698,68.21400531)(320.48748413,68.21400879)
\lineto(320.12748413,68.21400879)
\curveto(320.00747758,68.2240053)(319.90247768,68.22900529)(319.81248413,68.22900879)
\lineto(315.97248413,68.22900879)
\curveto(315.89248169,68.22900529)(315.81248177,68.2240053)(315.73248413,68.21400879)
\curveto(315.65248193,68.21400531)(315.587482,68.19900532)(315.53748413,68.16900879)
\curveto(315.49748209,68.14900537)(315.45748213,68.10900541)(315.41748413,68.04900879)
\curveto(315.39748219,68.0190055)(315.37748221,67.97400555)(315.35748413,67.91400879)
\curveto(315.33748225,67.86400566)(315.33748225,67.81400571)(315.35748413,67.76400879)
\curveto(315.36748222,67.71400581)(315.37248221,67.66900585)(315.37248413,67.62900879)
\curveto(315.37248221,67.58900593)(315.37748221,67.54900597)(315.38748413,67.50900879)
\curveto(315.40748218,67.42900609)(315.42748216,67.34400618)(315.44748413,67.25400879)
\curveto(315.46748212,67.17400635)(315.49748209,67.09400643)(315.53748413,67.01400879)
\curveto(315.76748182,66.47400705)(316.14748144,66.08900743)(316.67748413,65.85900879)
\curveto(316.73748085,65.82900769)(316.80248078,65.80400772)(316.87248413,65.78400879)
\lineto(317.08248413,65.72400879)
\curveto(317.11248047,65.71400781)(317.16248042,65.70900781)(317.23248413,65.70900879)
\curveto(317.37248021,65.66900785)(317.55748003,65.64900787)(317.78748413,65.64900879)
\curveto(318.01747957,65.64900787)(318.20247938,65.66900785)(318.34248413,65.70900879)
\curveto(318.4824791,65.74900777)(318.60747898,65.78900773)(318.71748413,65.82900879)
\curveto(318.83747875,65.87900764)(318.94747864,65.93900758)(319.04748413,66.00900879)
\curveto(319.15747843,66.07900744)(319.25247833,66.15900736)(319.33248413,66.24900879)
\curveto(319.41247817,66.34900717)(319.4824781,66.45400707)(319.54248413,66.56400879)
\curveto(319.60247798,66.66400686)(319.65247793,66.76900675)(319.69248413,66.87900879)
\curveto(319.74247784,66.98900653)(319.82247776,67.06900645)(319.93248413,67.11900879)
\curveto(319.97247761,67.13900638)(320.03747755,67.15400637)(320.12748413,67.16400879)
\curveto(320.21747737,67.17400635)(320.30747728,67.17400635)(320.39748413,67.16400879)
\curveto(320.4874771,67.16400636)(320.57247701,67.15900636)(320.65248413,67.14900879)
\curveto(320.73247685,67.13900638)(320.7874768,67.1190064)(320.81748413,67.08900879)
\curveto(320.91747667,67.0190065)(320.94247664,66.90400662)(320.89248413,66.74400879)
\curveto(320.81247677,66.47400705)(320.70747688,66.23400729)(320.57748413,66.02400879)
\curveto(320.37747721,65.70400782)(320.14747744,65.43900808)(319.88748413,65.22900879)
\curveto(319.63747795,65.02900849)(319.31747827,64.86400866)(318.92748413,64.73400879)
\curveto(318.82747876,64.69400883)(318.72747886,64.66900885)(318.62748413,64.65900879)
\curveto(318.52747906,64.63900888)(318.42247916,64.6190089)(318.31248413,64.59900879)
\curveto(318.26247932,64.58900893)(318.21247937,64.58400894)(318.16248413,64.58400879)
\curveto(318.12247946,64.58400894)(318.07747951,64.57900894)(318.02748413,64.56900879)
\lineto(317.87748413,64.56900879)
\curveto(317.82747976,64.55900896)(317.76747982,64.55400897)(317.69748413,64.55400879)
\curveto(317.63747995,64.55400897)(317.58748,64.55900896)(317.54748413,64.56900879)
\lineto(317.41248413,64.56900879)
\curveto(317.36248022,64.57900894)(317.31748027,64.58400894)(317.27748413,64.58400879)
\curveto(317.23748035,64.58400894)(317.19748039,64.58900893)(317.15748413,64.59900879)
\curveto(317.10748048,64.60900891)(317.05248053,64.6190089)(316.99248413,64.62900879)
\curveto(316.93248065,64.62900889)(316.87748071,64.63400889)(316.82748413,64.64400879)
\curveto(316.73748085,64.66400886)(316.64748094,64.68900883)(316.55748413,64.71900879)
\curveto(316.46748112,64.73900878)(316.3824812,64.76400876)(316.30248413,64.79400879)
\curveto(316.26248132,64.81400871)(316.22748136,64.8240087)(316.19748413,64.82400879)
\curveto(316.16748142,64.83400869)(316.13248145,64.84900867)(316.09248413,64.86900879)
\curveto(315.94248164,64.93900858)(315.7824818,65.0240085)(315.61248413,65.12400879)
\curveto(315.32248226,65.31400821)(315.07248251,65.54400798)(314.86248413,65.81400879)
\curveto(314.66248292,66.09400743)(314.49248309,66.40400712)(314.35248413,66.74400879)
\curveto(314.30248328,66.85400667)(314.26248332,66.96900655)(314.23248413,67.08900879)
\curveto(314.21248337,67.20900631)(314.1824834,67.32900619)(314.14248413,67.44900879)
\curveto(314.13248345,67.48900603)(314.12748346,67.524006)(314.12748413,67.55400879)
\curveto(314.12748346,67.58400594)(314.12248346,67.6240059)(314.11248413,67.67400879)
\curveto(314.09248349,67.75400577)(314.07748351,67.83900568)(314.06748413,67.92900879)
\curveto(314.05748353,68.0190055)(314.04248354,68.10900541)(314.02248413,68.19900879)
\lineto(314.02248413,68.40900879)
\curveto(314.01248357,68.44900507)(314.00248358,68.50400502)(313.99248413,68.57400879)
\curveto(313.99248359,68.65400487)(313.99748359,68.7190048)(314.00748413,68.76900879)
\lineto(314.00748413,68.93400879)
\curveto(314.02748356,68.98400454)(314.03248355,69.03400449)(314.02248413,69.08400879)
\curveto(314.02248356,69.14400438)(314.02748356,69.19900432)(314.03748413,69.24900879)
\curveto(314.07748351,69.40900411)(314.10748348,69.56900395)(314.12748413,69.72900879)
\curveto(314.15748343,69.88900363)(314.20248338,70.03900348)(314.26248413,70.17900879)
\curveto(314.31248327,70.28900323)(314.35748323,70.39900312)(314.39748413,70.50900879)
\curveto(314.44748314,70.62900289)(314.50248308,70.74400278)(314.56248413,70.85400879)
\curveto(314.7824828,71.20400232)(315.03248255,71.50400202)(315.31248413,71.75400879)
\curveto(315.59248199,72.01400151)(315.93748165,72.22900129)(316.34748413,72.39900879)
\curveto(316.46748112,72.44900107)(316.587481,72.48400104)(316.70748413,72.50400879)
\curveto(316.83748075,72.53400099)(316.97248061,72.56400096)(317.11248413,72.59400879)
\curveto(317.16248042,72.60400092)(317.20748038,72.60900091)(317.24748413,72.60900879)
\curveto(317.2874803,72.6190009)(317.33248025,72.6240009)(317.38248413,72.62400879)
\curveto(317.40248018,72.63400089)(317.42748016,72.63400089)(317.45748413,72.62400879)
\curveto(317.4874801,72.61400091)(317.51248007,72.6190009)(317.53248413,72.63900879)
\curveto(317.95247963,72.64900087)(318.31747927,72.60400092)(318.62748413,72.50400879)
\curveto(318.93747865,72.41400111)(319.21747837,72.28900123)(319.46748413,72.12900879)
\curveto(319.51747807,72.10900141)(319.55747803,72.07900144)(319.58748413,72.03900879)
\curveto(319.61747797,72.00900151)(319.65247793,71.98400154)(319.69248413,71.96400879)
\curveto(319.77247781,71.90400162)(319.85247773,71.83400169)(319.93248413,71.75400879)
\curveto(320.02247756,71.67400185)(320.09747749,71.59400193)(320.15748413,71.51400879)
\curveto(320.31747727,71.30400222)(320.45247713,71.10400242)(320.56248413,70.91400879)
\curveto(320.63247695,70.80400272)(320.6874769,70.68400284)(320.72748413,70.55400879)
\curveto(320.76747682,70.4240031)(320.81247677,70.29400323)(320.86248413,70.16400879)
\curveto(320.91247667,70.03400349)(320.94747664,69.89900362)(320.96748413,69.75900879)
\curveto(320.99747659,69.6190039)(321.03247655,69.47900404)(321.07248413,69.33900879)
\curveto(321.0824765,69.26900425)(321.0874765,69.19900432)(321.08748413,69.12900879)
\lineto(321.11748413,68.91900879)
\moveto(319.66248413,69.42900879)
\curveto(319.69247789,69.46900405)(319.71747787,69.519004)(319.73748413,69.57900879)
\curveto(319.75747783,69.64900387)(319.75747783,69.7190038)(319.73748413,69.78900879)
\curveto(319.67747791,70.00900351)(319.59247799,70.21400331)(319.48248413,70.40400879)
\curveto(319.34247824,70.63400289)(319.1874784,70.82900269)(319.01748413,70.98900879)
\curveto(318.84747874,71.14900237)(318.62747896,71.28400224)(318.35748413,71.39400879)
\curveto(318.2874793,71.41400211)(318.21747937,71.42900209)(318.14748413,71.43900879)
\curveto(318.07747951,71.45900206)(318.00247958,71.47900204)(317.92248413,71.49900879)
\curveto(317.84247974,71.519002)(317.75747983,71.52900199)(317.66748413,71.52900879)
\lineto(317.41248413,71.52900879)
\curveto(317.3824802,71.50900201)(317.34748024,71.49900202)(317.30748413,71.49900879)
\curveto(317.26748032,71.50900201)(317.23248035,71.50900201)(317.20248413,71.49900879)
\lineto(316.96248413,71.43900879)
\curveto(316.89248069,71.42900209)(316.82248076,71.41400211)(316.75248413,71.39400879)
\curveto(316.46248112,71.27400225)(316.22748136,71.1240024)(316.04748413,70.94400879)
\curveto(315.87748171,70.76400276)(315.72248186,70.53900298)(315.58248413,70.26900879)
\curveto(315.55248203,70.2190033)(315.52248206,70.15400337)(315.49248413,70.07400879)
\curveto(315.46248212,70.00400352)(315.43748215,69.9240036)(315.41748413,69.83400879)
\curveto(315.39748219,69.74400378)(315.39248219,69.65900386)(315.40248413,69.57900879)
\curveto(315.41248217,69.49900402)(315.44748214,69.43900408)(315.50748413,69.39900879)
\curveto(315.587482,69.33900418)(315.72248186,69.30900421)(315.91248413,69.30900879)
\curveto(316.11248147,69.3190042)(316.2824813,69.3240042)(316.42248413,69.32400879)
\lineto(318.70248413,69.32400879)
\curveto(318.85247873,69.3240042)(319.03247855,69.3190042)(319.24248413,69.30900879)
\curveto(319.45247813,69.30900421)(319.59247799,69.34900417)(319.66248413,69.42900879)
}
}
{
\newrgbcolor{curcolor}{0 0 0}
\pscustom[linestyle=none,fillstyle=solid,fillcolor=curcolor]
{
\newpath
\moveto(329.30912476,65.30400879)
\curveto(329.33911693,65.14400838)(329.32411694,65.00900851)(329.26412476,64.89900879)
\curveto(329.20411706,64.79900872)(329.12411714,64.7240088)(329.02412476,64.67400879)
\curveto(328.97411729,64.65400887)(328.91911735,64.64400888)(328.85912476,64.64400879)
\curveto(328.80911746,64.64400888)(328.75411751,64.63400889)(328.69412476,64.61400879)
\curveto(328.47411779,64.56400896)(328.25411801,64.57900894)(328.03412476,64.65900879)
\curveto(327.82411844,64.72900879)(327.67911859,64.8190087)(327.59912476,64.92900879)
\curveto(327.54911872,64.99900852)(327.50411876,65.07900844)(327.46412476,65.16900879)
\curveto(327.42411884,65.26900825)(327.37411889,65.34900817)(327.31412476,65.40900879)
\curveto(327.29411897,65.42900809)(327.269119,65.44900807)(327.23912476,65.46900879)
\curveto(327.21911905,65.48900803)(327.18911908,65.49400803)(327.14912476,65.48400879)
\curveto(327.03911923,65.45400807)(326.93411933,65.39900812)(326.83412476,65.31900879)
\curveto(326.74411952,65.23900828)(326.65411961,65.16900835)(326.56412476,65.10900879)
\curveto(326.43411983,65.02900849)(326.29411997,64.95400857)(326.14412476,64.88400879)
\curveto(325.99412027,64.8240087)(325.83412043,64.76900875)(325.66412476,64.71900879)
\curveto(325.5641207,64.68900883)(325.45412081,64.66900885)(325.33412476,64.65900879)
\curveto(325.22412104,64.64900887)(325.11412115,64.63400889)(325.00412476,64.61400879)
\curveto(324.95412131,64.60400892)(324.90912136,64.59900892)(324.86912476,64.59900879)
\lineto(324.76412476,64.59900879)
\curveto(324.65412161,64.57900894)(324.54912172,64.57900894)(324.44912476,64.59900879)
\lineto(324.31412476,64.59900879)
\curveto(324.264122,64.60900891)(324.21412205,64.61400891)(324.16412476,64.61400879)
\curveto(324.11412215,64.61400891)(324.0691222,64.6240089)(324.02912476,64.64400879)
\curveto(323.98912228,64.65400887)(323.95412231,64.65900886)(323.92412476,64.65900879)
\curveto(323.90412236,64.64900887)(323.87912239,64.64900887)(323.84912476,64.65900879)
\lineto(323.60912476,64.71900879)
\curveto(323.52912274,64.72900879)(323.45412281,64.74900877)(323.38412476,64.77900879)
\curveto(323.08412318,64.90900861)(322.83912343,65.05400847)(322.64912476,65.21400879)
\curveto(322.4691238,65.38400814)(322.31912395,65.6190079)(322.19912476,65.91900879)
\curveto(322.10912416,66.13900738)(322.0641242,66.40400712)(322.06412476,66.71400879)
\lineto(322.06412476,67.02900879)
\curveto(322.07412419,67.07900644)(322.07912419,67.12900639)(322.07912476,67.17900879)
\lineto(322.10912476,67.35900879)
\lineto(322.22912476,67.68900879)
\curveto(322.269124,67.79900572)(322.31912395,67.89900562)(322.37912476,67.98900879)
\curveto(322.55912371,68.27900524)(322.80412346,68.49400503)(323.11412476,68.63400879)
\curveto(323.42412284,68.77400475)(323.7641225,68.89900462)(324.13412476,69.00900879)
\curveto(324.27412199,69.04900447)(324.41912185,69.07900444)(324.56912476,69.09900879)
\curveto(324.71912155,69.1190044)(324.8691214,69.14400438)(325.01912476,69.17400879)
\curveto(325.08912118,69.19400433)(325.15412111,69.20400432)(325.21412476,69.20400879)
\curveto(325.28412098,69.20400432)(325.35912091,69.21400431)(325.43912476,69.23400879)
\curveto(325.50912076,69.25400427)(325.57912069,69.26400426)(325.64912476,69.26400879)
\curveto(325.71912055,69.27400425)(325.79412047,69.28900423)(325.87412476,69.30900879)
\curveto(326.12412014,69.36900415)(326.35911991,69.4190041)(326.57912476,69.45900879)
\curveto(326.79911947,69.50900401)(326.97411929,69.6240039)(327.10412476,69.80400879)
\curveto(327.1641191,69.88400364)(327.21411905,69.98400354)(327.25412476,70.10400879)
\curveto(327.29411897,70.23400329)(327.29411897,70.37400315)(327.25412476,70.52400879)
\curveto(327.19411907,70.76400276)(327.10411916,70.95400257)(326.98412476,71.09400879)
\curveto(326.87411939,71.23400229)(326.71411955,71.34400218)(326.50412476,71.42400879)
\curveto(326.38411988,71.47400205)(326.23912003,71.50900201)(326.06912476,71.52900879)
\curveto(325.90912036,71.54900197)(325.73912053,71.55900196)(325.55912476,71.55900879)
\curveto(325.37912089,71.55900196)(325.20412106,71.54900197)(325.03412476,71.52900879)
\curveto(324.8641214,71.50900201)(324.71912155,71.47900204)(324.59912476,71.43900879)
\curveto(324.42912184,71.37900214)(324.264122,71.29400223)(324.10412476,71.18400879)
\curveto(324.02412224,71.1240024)(323.94912232,71.04400248)(323.87912476,70.94400879)
\curveto(323.81912245,70.85400267)(323.7641225,70.75400277)(323.71412476,70.64400879)
\curveto(323.68412258,70.56400296)(323.65412261,70.47900304)(323.62412476,70.38900879)
\curveto(323.60412266,70.29900322)(323.55912271,70.22900329)(323.48912476,70.17900879)
\curveto(323.44912282,70.14900337)(323.37912289,70.1240034)(323.27912476,70.10400879)
\curveto(323.18912308,70.09400343)(323.09412317,70.08900343)(322.99412476,70.08900879)
\curveto(322.89412337,70.08900343)(322.79412347,70.09400343)(322.69412476,70.10400879)
\curveto(322.60412366,70.1240034)(322.53912373,70.14900337)(322.49912476,70.17900879)
\curveto(322.45912381,70.20900331)(322.42912384,70.25900326)(322.40912476,70.32900879)
\curveto(322.38912388,70.39900312)(322.38912388,70.47400305)(322.40912476,70.55400879)
\curveto(322.43912383,70.68400284)(322.4691238,70.80400272)(322.49912476,70.91400879)
\curveto(322.53912373,71.03400249)(322.58412368,71.14900237)(322.63412476,71.25900879)
\curveto(322.82412344,71.60900191)(323.0641232,71.87900164)(323.35412476,72.06900879)
\curveto(323.64412262,72.26900125)(324.00412226,72.42900109)(324.43412476,72.54900879)
\curveto(324.53412173,72.56900095)(324.63412163,72.58400094)(324.73412476,72.59400879)
\curveto(324.84412142,72.60400092)(324.95412131,72.6190009)(325.06412476,72.63900879)
\curveto(325.10412116,72.64900087)(325.1691211,72.64900087)(325.25912476,72.63900879)
\curveto(325.34912092,72.63900088)(325.40412086,72.64900087)(325.42412476,72.66900879)
\curveto(326.12412014,72.67900084)(326.73411953,72.59900092)(327.25412476,72.42900879)
\curveto(327.77411849,72.25900126)(328.13911813,71.93400159)(328.34912476,71.45400879)
\curveto(328.43911783,71.25400227)(328.48911778,71.0190025)(328.49912476,70.74900879)
\curveto(328.51911775,70.48900303)(328.52911774,70.21400331)(328.52912476,69.92400879)
\lineto(328.52912476,66.60900879)
\curveto(328.52911774,66.46900705)(328.53411773,66.33400719)(328.54412476,66.20400879)
\curveto(328.55411771,66.07400745)(328.58411768,65.96900755)(328.63412476,65.88900879)
\curveto(328.68411758,65.8190077)(328.74911752,65.76900775)(328.82912476,65.73900879)
\curveto(328.91911735,65.69900782)(329.00411726,65.66900785)(329.08412476,65.64900879)
\curveto(329.1641171,65.63900788)(329.22411704,65.59400793)(329.26412476,65.51400879)
\curveto(329.28411698,65.48400804)(329.29411697,65.45400807)(329.29412476,65.42400879)
\curveto(329.29411697,65.39400813)(329.29911697,65.35400817)(329.30912476,65.30400879)
\moveto(327.16412476,66.96900879)
\curveto(327.22411904,67.10900641)(327.25411901,67.26900625)(327.25412476,67.44900879)
\curveto(327.264119,67.63900588)(327.269119,67.83400569)(327.26912476,68.03400879)
\curveto(327.269119,68.14400538)(327.264119,68.24400528)(327.25412476,68.33400879)
\curveto(327.24411902,68.4240051)(327.20411906,68.49400503)(327.13412476,68.54400879)
\curveto(327.10411916,68.56400496)(327.03411923,68.57400495)(326.92412476,68.57400879)
\curveto(326.90411936,68.55400497)(326.8691194,68.54400498)(326.81912476,68.54400879)
\curveto(326.7691195,68.54400498)(326.72411954,68.53400499)(326.68412476,68.51400879)
\curveto(326.60411966,68.49400503)(326.51411975,68.47400505)(326.41412476,68.45400879)
\lineto(326.11412476,68.39400879)
\curveto(326.08412018,68.39400513)(326.04912022,68.38900513)(326.00912476,68.37900879)
\lineto(325.90412476,68.37900879)
\curveto(325.75412051,68.33900518)(325.58912068,68.31400521)(325.40912476,68.30400879)
\curveto(325.23912103,68.30400522)(325.07912119,68.28400524)(324.92912476,68.24400879)
\curveto(324.84912142,68.2240053)(324.77412149,68.20400532)(324.70412476,68.18400879)
\curveto(324.64412162,68.17400535)(324.57412169,68.15900536)(324.49412476,68.13900879)
\curveto(324.33412193,68.08900543)(324.18412208,68.0240055)(324.04412476,67.94400879)
\curveto(323.90412236,67.87400565)(323.78412248,67.78400574)(323.68412476,67.67400879)
\curveto(323.58412268,67.56400596)(323.50912276,67.42900609)(323.45912476,67.26900879)
\curveto(323.40912286,67.1190064)(323.38912288,66.93400659)(323.39912476,66.71400879)
\curveto(323.39912287,66.61400691)(323.41412285,66.519007)(323.44412476,66.42900879)
\curveto(323.48412278,66.34900717)(323.52912274,66.27400725)(323.57912476,66.20400879)
\curveto(323.65912261,66.09400743)(323.7641225,65.99900752)(323.89412476,65.91900879)
\curveto(324.02412224,65.84900767)(324.1641221,65.78900773)(324.31412476,65.73900879)
\curveto(324.3641219,65.72900779)(324.41412185,65.7240078)(324.46412476,65.72400879)
\curveto(324.51412175,65.7240078)(324.5641217,65.7190078)(324.61412476,65.70900879)
\curveto(324.68412158,65.68900783)(324.7691215,65.67400785)(324.86912476,65.66400879)
\curveto(324.97912129,65.66400786)(325.0691212,65.67400785)(325.13912476,65.69400879)
\curveto(325.19912107,65.71400781)(325.25912101,65.7190078)(325.31912476,65.70900879)
\curveto(325.37912089,65.70900781)(325.43912083,65.7190078)(325.49912476,65.73900879)
\curveto(325.57912069,65.75900776)(325.65412061,65.77400775)(325.72412476,65.78400879)
\curveto(325.80412046,65.79400773)(325.87912039,65.81400771)(325.94912476,65.84400879)
\curveto(326.23912003,65.96400756)(326.48411978,66.10900741)(326.68412476,66.27900879)
\curveto(326.89411937,66.44900707)(327.05411921,66.67900684)(327.16412476,66.96900879)
}
}
{
\newrgbcolor{curcolor}{0 0 0}
\pscustom[linestyle=none,fillstyle=solid,fillcolor=curcolor]
{
\newpath
\moveto(332.91076538,72.65400879)
\curveto(333.63076132,72.66400086)(334.23576071,72.57900094)(334.72576538,72.39900879)
\curveto(335.21575973,72.22900129)(335.59575935,71.9240016)(335.86576538,71.48400879)
\curveto(335.93575901,71.37400215)(335.99075896,71.25900226)(336.03076538,71.13900879)
\curveto(336.07075888,71.02900249)(336.11075884,70.90400262)(336.15076538,70.76400879)
\curveto(336.17075878,70.69400283)(336.17575877,70.6190029)(336.16576538,70.53900879)
\curveto(336.15575879,70.46900305)(336.14075881,70.41400311)(336.12076538,70.37400879)
\curveto(336.10075885,70.35400317)(336.07575887,70.33400319)(336.04576538,70.31400879)
\curveto(336.01575893,70.30400322)(335.99075896,70.28900323)(335.97076538,70.26900879)
\curveto(335.92075903,70.24900327)(335.87075908,70.24400328)(335.82076538,70.25400879)
\curveto(335.77075918,70.26400326)(335.72075923,70.26400326)(335.67076538,70.25400879)
\curveto(335.59075936,70.23400329)(335.48575946,70.22900329)(335.35576538,70.23900879)
\curveto(335.22575972,70.25900326)(335.13575981,70.28400324)(335.08576538,70.31400879)
\curveto(335.00575994,70.36400316)(334.95076,70.42900309)(334.92076538,70.50900879)
\curveto(334.90076005,70.59900292)(334.86576008,70.68400284)(334.81576538,70.76400879)
\curveto(334.72576022,70.9240026)(334.60076035,71.06900245)(334.44076538,71.19900879)
\curveto(334.33076062,71.27900224)(334.21076074,71.33900218)(334.08076538,71.37900879)
\curveto(333.950761,71.4190021)(333.81076114,71.45900206)(333.66076538,71.49900879)
\curveto(333.61076134,71.519002)(333.56076139,71.524002)(333.51076538,71.51400879)
\curveto(333.46076149,71.51400201)(333.41076154,71.519002)(333.36076538,71.52900879)
\curveto(333.30076165,71.54900197)(333.22576172,71.55900196)(333.13576538,71.55900879)
\curveto(333.0457619,71.55900196)(332.97076198,71.54900197)(332.91076538,71.52900879)
\lineto(332.82076538,71.52900879)
\lineto(332.67076538,71.49900879)
\curveto(332.62076233,71.49900202)(332.57076238,71.49400203)(332.52076538,71.48400879)
\curveto(332.26076269,71.4240021)(332.0457629,71.33900218)(331.87576538,71.22900879)
\curveto(331.70576324,71.1190024)(331.59076336,70.93400259)(331.53076538,70.67400879)
\curveto(331.51076344,70.60400292)(331.50576344,70.53400299)(331.51576538,70.46400879)
\curveto(331.53576341,70.39400313)(331.55576339,70.33400319)(331.57576538,70.28400879)
\curveto(331.63576331,70.13400339)(331.70576324,70.0240035)(331.78576538,69.95400879)
\curveto(331.87576307,69.89400363)(331.98576296,69.8240037)(332.11576538,69.74400879)
\curveto(332.27576267,69.64400388)(332.45576249,69.56900395)(332.65576538,69.51900879)
\curveto(332.85576209,69.47900404)(333.05576189,69.42900409)(333.25576538,69.36900879)
\curveto(333.38576156,69.32900419)(333.51576143,69.29900422)(333.64576538,69.27900879)
\curveto(333.77576117,69.25900426)(333.90576104,69.22900429)(334.03576538,69.18900879)
\curveto(334.2457607,69.12900439)(334.4507605,69.06900445)(334.65076538,69.00900879)
\curveto(334.8507601,68.95900456)(335.0507599,68.89400463)(335.25076538,68.81400879)
\lineto(335.40076538,68.75400879)
\curveto(335.4507595,68.73400479)(335.50075945,68.70900481)(335.55076538,68.67900879)
\curveto(335.7507592,68.55900496)(335.92575902,68.4240051)(336.07576538,68.27400879)
\curveto(336.22575872,68.1240054)(336.3507586,67.93400559)(336.45076538,67.70400879)
\curveto(336.47075848,67.63400589)(336.49075846,67.53900598)(336.51076538,67.41900879)
\curveto(336.53075842,67.34900617)(336.54075841,67.27400625)(336.54076538,67.19400879)
\curveto(336.5507584,67.1240064)(336.55575839,67.04400648)(336.55576538,66.95400879)
\lineto(336.55576538,66.80400879)
\curveto(336.53575841,66.73400679)(336.52575842,66.66400686)(336.52576538,66.59400879)
\curveto(336.52575842,66.524007)(336.51575843,66.45400707)(336.49576538,66.38400879)
\curveto(336.46575848,66.27400725)(336.43075852,66.16900735)(336.39076538,66.06900879)
\curveto(336.3507586,65.96900755)(336.30575864,65.87900764)(336.25576538,65.79900879)
\curveto(336.09575885,65.53900798)(335.89075906,65.32900819)(335.64076538,65.16900879)
\curveto(335.39075956,65.0190085)(335.11075984,64.88900863)(334.80076538,64.77900879)
\curveto(334.71076024,64.74900877)(334.61576033,64.72900879)(334.51576538,64.71900879)
\curveto(334.42576052,64.69900882)(334.33576061,64.67400885)(334.24576538,64.64400879)
\curveto(334.1457608,64.6240089)(334.0457609,64.61400891)(333.94576538,64.61400879)
\curveto(333.8457611,64.61400891)(333.7457612,64.60400892)(333.64576538,64.58400879)
\lineto(333.49576538,64.58400879)
\curveto(333.4457615,64.57400895)(333.37576157,64.56900895)(333.28576538,64.56900879)
\curveto(333.19576175,64.56900895)(333.12576182,64.57400895)(333.07576538,64.58400879)
\lineto(332.91076538,64.58400879)
\curveto(332.8507621,64.60400892)(332.78576216,64.61400891)(332.71576538,64.61400879)
\curveto(332.6457623,64.60400892)(332.58576236,64.60900891)(332.53576538,64.62900879)
\curveto(332.48576246,64.63900888)(332.42076253,64.64400888)(332.34076538,64.64400879)
\lineto(332.10076538,64.70400879)
\curveto(332.03076292,64.71400881)(331.95576299,64.73400879)(331.87576538,64.76400879)
\curveto(331.56576338,64.86400866)(331.29576365,64.98900853)(331.06576538,65.13900879)
\curveto(330.83576411,65.28900823)(330.63576431,65.48400804)(330.46576538,65.72400879)
\curveto(330.37576457,65.85400767)(330.30076465,65.98900753)(330.24076538,66.12900879)
\curveto(330.18076477,66.26900725)(330.12576482,66.4240071)(330.07576538,66.59400879)
\curveto(330.05576489,66.65400687)(330.0457649,66.7240068)(330.04576538,66.80400879)
\curveto(330.05576489,66.89400663)(330.07076488,66.96400656)(330.09076538,67.01400879)
\curveto(330.12076483,67.05400647)(330.17076478,67.09400643)(330.24076538,67.13400879)
\curveto(330.29076466,67.15400637)(330.36076459,67.16400636)(330.45076538,67.16400879)
\curveto(330.54076441,67.17400635)(330.63076432,67.17400635)(330.72076538,67.16400879)
\curveto(330.81076414,67.15400637)(330.89576405,67.13900638)(330.97576538,67.11900879)
\curveto(331.06576388,67.10900641)(331.12576382,67.09400643)(331.15576538,67.07400879)
\curveto(331.22576372,67.0240065)(331.27076368,66.94900657)(331.29076538,66.84900879)
\curveto(331.32076363,66.75900676)(331.35576359,66.67400685)(331.39576538,66.59400879)
\curveto(331.49576345,66.37400715)(331.63076332,66.20400732)(331.80076538,66.08400879)
\curveto(331.92076303,65.99400753)(332.05576289,65.9240076)(332.20576538,65.87400879)
\curveto(332.35576259,65.8240077)(332.51576243,65.77400775)(332.68576538,65.72400879)
\lineto(333.00076538,65.67900879)
\lineto(333.09076538,65.67900879)
\curveto(333.16076179,65.65900786)(333.2507617,65.64900787)(333.36076538,65.64900879)
\curveto(333.48076147,65.64900787)(333.58076137,65.65900786)(333.66076538,65.67900879)
\curveto(333.73076122,65.67900784)(333.78576116,65.68400784)(333.82576538,65.69400879)
\curveto(333.88576106,65.70400782)(333.945761,65.70900781)(334.00576538,65.70900879)
\curveto(334.06576088,65.7190078)(334.12076083,65.72900779)(334.17076538,65.73900879)
\curveto(334.46076049,65.8190077)(334.69076026,65.9240076)(334.86076538,66.05400879)
\curveto(335.03075992,66.18400734)(335.1507598,66.40400712)(335.22076538,66.71400879)
\curveto(335.24075971,66.76400676)(335.2457597,66.8190067)(335.23576538,66.87900879)
\curveto(335.22575972,66.93900658)(335.21575973,66.98400654)(335.20576538,67.01400879)
\curveto(335.15575979,67.20400632)(335.08575986,67.34400618)(334.99576538,67.43400879)
\curveto(334.90576004,67.53400599)(334.79076016,67.6240059)(334.65076538,67.70400879)
\curveto(334.56076039,67.76400576)(334.46076049,67.81400571)(334.35076538,67.85400879)
\lineto(334.02076538,67.97400879)
\curveto(333.99076096,67.98400554)(333.96076099,67.98900553)(333.93076538,67.98900879)
\curveto(333.91076104,67.98900553)(333.88576106,67.99900552)(333.85576538,68.01900879)
\curveto(333.51576143,68.12900539)(333.16076179,68.20900531)(332.79076538,68.25900879)
\curveto(332.43076252,68.3190052)(332.09076286,68.41400511)(331.77076538,68.54400879)
\curveto(331.67076328,68.58400494)(331.57576337,68.6190049)(331.48576538,68.64900879)
\curveto(331.39576355,68.67900484)(331.31076364,68.7190048)(331.23076538,68.76900879)
\curveto(331.04076391,68.87900464)(330.86576408,69.00400452)(330.70576538,69.14400879)
\curveto(330.5457644,69.28400424)(330.42076453,69.45900406)(330.33076538,69.66900879)
\curveto(330.30076465,69.73900378)(330.27576467,69.80900371)(330.25576538,69.87900879)
\curveto(330.2457647,69.94900357)(330.23076472,70.0240035)(330.21076538,70.10400879)
\curveto(330.18076477,70.2240033)(330.17076478,70.35900316)(330.18076538,70.50900879)
\curveto(330.19076476,70.66900285)(330.20576474,70.80400272)(330.22576538,70.91400879)
\curveto(330.2457647,70.96400256)(330.25576469,71.00400252)(330.25576538,71.03400879)
\curveto(330.26576468,71.07400245)(330.28076467,71.11400241)(330.30076538,71.15400879)
\curveto(330.39076456,71.38400214)(330.51076444,71.58400194)(330.66076538,71.75400879)
\curveto(330.82076413,71.9240016)(331.00076395,72.07400145)(331.20076538,72.20400879)
\curveto(331.3507636,72.29400123)(331.51576343,72.36400116)(331.69576538,72.41400879)
\curveto(331.87576307,72.47400105)(332.06576288,72.52900099)(332.26576538,72.57900879)
\curveto(332.33576261,72.58900093)(332.40076255,72.59900092)(332.46076538,72.60900879)
\curveto(332.53076242,72.6190009)(332.60576234,72.62900089)(332.68576538,72.63900879)
\curveto(332.71576223,72.64900087)(332.75576219,72.64900087)(332.80576538,72.63900879)
\curveto(332.85576209,72.62900089)(332.89076206,72.63400089)(332.91076538,72.65400879)
}
}
{
\newrgbcolor{curcolor}{0 0 0}
\pscustom[linestyle=none,fillstyle=solid,fillcolor=curcolor]
{
\newpath
\moveto(482.74775391,75.85150391)
\curveto(483.43774767,75.87149295)(484.04274706,75.80649302)(484.56275391,75.65650391)
\curveto(485.08274602,75.51649331)(485.54274556,75.30649352)(485.94275391,75.02650391)
\curveto(486.12274498,74.90649392)(486.28774482,74.77149405)(486.43775391,74.62150391)
\curveto(486.45774465,74.60149422)(486.47774463,74.58149424)(486.49775391,74.56150391)
\curveto(486.51774459,74.54149428)(486.53774457,74.5214943)(486.55775391,74.50150391)
\curveto(486.6077445,74.4214944)(486.66274444,74.34649448)(486.72275391,74.27650391)
\curveto(486.78274432,74.21649461)(486.83774427,74.14649468)(486.88775391,74.06650391)
\curveto(486.99774411,73.89649493)(487.09274401,73.71649511)(487.17275391,73.52650391)
\curveto(487.25274385,73.33649549)(487.33274377,73.14149568)(487.41275391,72.94150391)
\curveto(487.44274366,72.84149598)(487.45774365,72.73649609)(487.45775391,72.62650391)
\curveto(487.45774365,72.51649631)(487.41774369,72.43649639)(487.33775391,72.38650391)
\curveto(487.31774379,72.36649646)(487.28774382,72.35649647)(487.24775391,72.35650391)
\curveto(487.21774389,72.35649647)(487.18774392,72.35149647)(487.15775391,72.34150391)
\lineto(487.05275391,72.34150391)
\curveto(487.0027441,72.3214965)(486.91774419,72.31149651)(486.79775391,72.31150391)
\curveto(486.68774442,72.31149651)(486.6077445,72.3214965)(486.55775391,72.34150391)
\curveto(486.52774458,72.35149647)(486.49774461,72.35149647)(486.46775391,72.34150391)
\curveto(486.43774467,72.33149649)(486.4027447,72.33649649)(486.36275391,72.35650391)
\curveto(486.3027448,72.36649646)(486.24774486,72.39149643)(486.19775391,72.43150391)
\curveto(486.13774497,72.48149634)(486.09274501,72.55149627)(486.06275391,72.64150391)
\curveto(486.04274506,72.73149609)(486.01274509,72.81649601)(485.97275391,72.89650391)
\curveto(485.92274518,73.0264958)(485.86274524,73.14649568)(485.79275391,73.25650391)
\curveto(485.73274537,73.37649545)(485.66274544,73.49149533)(485.58275391,73.60150391)
\curveto(485.56274554,73.6214952)(485.53774557,73.64149518)(485.50775391,73.66150391)
\curveto(485.47774563,73.69149513)(485.45274565,73.7214951)(485.43275391,73.75150391)
\curveto(485.34274576,73.85149497)(485.25274585,73.93649489)(485.16275391,74.00650391)
\curveto(485.11274599,74.03649479)(485.06774604,74.06649476)(485.02775391,74.09650391)
\curveto(484.98774612,74.13649469)(484.94274616,74.17149465)(484.89275391,74.20150391)
\curveto(484.75274635,74.28149454)(484.6027465,74.35149447)(484.44275391,74.41150391)
\curveto(484.28274682,74.47149435)(484.11774699,74.5264943)(483.94775391,74.57650391)
\curveto(483.85774725,74.59649423)(483.76774734,74.61149421)(483.67775391,74.62150391)
\curveto(483.58774752,74.63149419)(483.49774761,74.64649418)(483.40775391,74.66650391)
\curveto(483.36774774,74.67649415)(483.32774778,74.67649415)(483.28775391,74.66650391)
\curveto(483.25774785,74.66649416)(483.22774788,74.67149415)(483.19775391,74.68150391)
\lineto(483.01775391,74.68150391)
\lineto(482.80775391,74.68150391)
\curveto(482.73774837,74.68149414)(482.67274843,74.67649415)(482.61275391,74.66650391)
\curveto(482.59274851,74.66649416)(482.56774854,74.66149416)(482.53775391,74.65150391)
\lineto(482.46275391,74.65150391)
\curveto(482.4027487,74.64149418)(482.33774877,74.63149419)(482.26775391,74.62150391)
\lineto(482.08775391,74.59150391)
\curveto(481.73774937,74.50149432)(481.43274967,74.37649445)(481.17275391,74.21650391)
\curveto(480.75275035,73.95649487)(480.41275069,73.64649518)(480.15275391,73.28650391)
\curveto(479.9027512,72.93649589)(479.69275141,72.50649632)(479.52275391,71.99650391)
\curveto(479.48275162,71.88649694)(479.45275165,71.77149705)(479.43275391,71.65150391)
\curveto(479.41275169,71.53149729)(479.38775172,71.41149741)(479.35775391,71.29150391)
\curveto(479.33775177,71.24149758)(479.32775178,71.19149763)(479.32775391,71.14150391)
\curveto(479.33775177,71.10149772)(479.33275177,71.05649777)(479.31275391,71.00650391)
\curveto(479.29275181,70.93649789)(479.28275182,70.86149796)(479.28275391,70.78150391)
\curveto(479.29275181,70.71149811)(479.28775182,70.63649819)(479.26775391,70.55650391)
\lineto(479.26775391,70.39150391)
\curveto(479.25775185,70.33149849)(479.25275185,70.24649858)(479.25275391,70.13650391)
\curveto(479.25275185,70.0264988)(479.25775185,69.94649888)(479.26775391,69.89650391)
\lineto(479.26775391,69.74650391)
\curveto(479.27775183,69.7264991)(479.28275182,69.69649913)(479.28275391,69.65650391)
\curveto(479.28275182,69.6264992)(479.28775182,69.60149922)(479.29775391,69.58150391)
\curveto(479.31775179,69.51149931)(479.32275178,69.44649938)(479.31275391,69.38650391)
\curveto(479.3027518,69.3264995)(479.3077518,69.26149956)(479.32775391,69.19150391)
\curveto(479.34775176,69.11149971)(479.36275174,69.03149979)(479.37275391,68.95150391)
\curveto(479.39275171,68.88149994)(479.41275169,68.80650002)(479.43275391,68.72650391)
\lineto(479.58275391,68.27650391)
\lineto(479.76275391,67.85650391)
\curveto(479.81275129,67.73650109)(479.87275123,67.6215012)(479.94275391,67.51150391)
\curveto(480.02275108,67.40150142)(480.102751,67.29650153)(480.18275391,67.19650391)
\curveto(480.55275055,66.69650213)(481.03775007,66.3265025)(481.63775391,66.08650391)
\curveto(481.7077494,66.05650277)(481.77774933,66.03150279)(481.84775391,66.01150391)
\curveto(481.91774919,65.99150283)(481.99274911,65.97150285)(482.07275391,65.95150391)
\curveto(482.31274879,65.88150294)(482.59274851,65.84650298)(482.91275391,65.84650391)
\lineto(483.10775391,65.84650391)
\curveto(483.17774793,65.84650298)(483.24274786,65.85150297)(483.30275391,65.86150391)
\curveto(483.35274775,65.88150294)(483.4027477,65.88650294)(483.45275391,65.87650391)
\curveto(483.51274759,65.86650296)(483.56774754,65.87150295)(483.61775391,65.89150391)
\curveto(483.75774735,65.93150289)(483.88774722,65.96150286)(484.00775391,65.98150391)
\curveto(484.13774697,66.01150281)(484.25774685,66.05150277)(484.36775391,66.10150391)
\curveto(485.29774581,66.47150235)(485.92274518,67.13650169)(486.24275391,68.09650391)
\curveto(486.26274484,68.17650065)(486.27774483,68.25650057)(486.28775391,68.33650391)
\lineto(486.34775391,68.57650391)
\curveto(486.37774473,68.69650013)(486.38774472,68.8265)(486.37775391,68.96650391)
\curveto(486.36774474,69.11649971)(486.32274478,69.2214996)(486.24275391,69.28150391)
\curveto(486.16274494,69.33149949)(486.05274505,69.35649947)(485.91275391,69.35650391)
\lineto(485.50775391,69.35650391)
\lineto(483.84275391,69.35650391)
\lineto(483.48275391,69.35650391)
\curveto(483.35274775,69.35649947)(483.24774786,69.37149945)(483.16775391,69.40150391)
\curveto(483.08774802,69.44149938)(483.03774807,69.49649933)(483.01775391,69.56650391)
\curveto(482.99774811,69.60649922)(482.98274812,69.66149916)(482.97275391,69.73150391)
\curveto(482.96274814,69.81149901)(482.95774815,69.89149893)(482.95775391,69.97150391)
\curveto(482.95774815,70.05149877)(482.96274814,70.1264987)(482.97275391,70.19650391)
\curveto(482.99274811,70.27649855)(483.01274809,70.33149849)(483.03275391,70.36150391)
\curveto(483.07274803,70.43149839)(483.14274796,70.48149834)(483.24275391,70.51150391)
\curveto(483.29274781,70.53149829)(483.35274775,70.54149828)(483.42275391,70.54150391)
\lineto(483.63275391,70.54150391)
\lineto(484.30775391,70.54150391)
\lineto(486.58775391,70.54150391)
\lineto(486.91775391,70.54150391)
\curveto(487.02774408,70.55149827)(487.12774398,70.54649828)(487.21775391,70.52650391)
\curveto(487.31774379,70.51649831)(487.4027437,70.49149833)(487.47275391,70.45150391)
\curveto(487.54274356,70.4214984)(487.59274351,70.36149846)(487.62275391,70.27150391)
\curveto(487.64274346,70.21149861)(487.64774346,70.14149868)(487.63775391,70.06150391)
\curveto(487.63774347,69.98149884)(487.63774347,69.90149892)(487.63775391,69.82150391)
\lineto(487.63775391,68.99650391)
\lineto(487.63775391,66.22150391)
\lineto(487.63775391,65.53150391)
\curveto(487.63774347,65.46150336)(487.63774347,65.39150343)(487.63775391,65.32150391)
\curveto(487.63774347,65.25150357)(487.62774348,65.19150363)(487.60775391,65.14150391)
\curveto(487.57774353,65.06150376)(487.51774359,65.00150382)(487.42775391,64.96150391)
\curveto(487.39774371,64.94150388)(487.33274377,64.93150389)(487.23275391,64.93150391)
\curveto(487.09274401,64.93150389)(486.98774412,64.94650388)(486.91775391,64.97650391)
\curveto(486.84774426,65.00650382)(486.79274431,65.05150377)(486.75275391,65.11150391)
\curveto(486.71274439,65.17150365)(486.67774443,65.24150358)(486.64775391,65.32150391)
\curveto(486.62774448,65.40150342)(486.6027445,65.49150333)(486.57275391,65.59150391)
\curveto(486.55274455,65.66150316)(486.51774459,65.75150307)(486.46775391,65.86150391)
\curveto(486.42774468,65.97150285)(486.35274475,66.01150281)(486.24275391,65.98150391)
\curveto(486.17274493,65.96150286)(486.11774499,65.93150289)(486.07775391,65.89150391)
\curveto(486.03774507,65.85150297)(485.99274511,65.81150301)(485.94275391,65.77150391)
\curveto(485.86274524,65.71150311)(485.78774532,65.64650318)(485.71775391,65.57650391)
\curveto(485.64774546,65.51650331)(485.57274553,65.46150336)(485.49275391,65.41150391)
\curveto(485.28274582,65.27150355)(485.05774605,65.15650367)(484.81775391,65.06650391)
\curveto(484.58774652,64.97650385)(484.33774677,64.89150393)(484.06775391,64.81150391)
\curveto(483.99774711,64.79150403)(483.92774718,64.77650405)(483.85775391,64.76650391)
\curveto(483.78774732,64.75650407)(483.71274739,64.74150408)(483.63275391,64.72150391)
\curveto(483.55274755,64.7215041)(483.49274761,64.71650411)(483.45275391,64.70650391)
\lineto(483.34775391,64.70650391)
\curveto(483.31774779,64.70650412)(483.28774782,64.70150412)(483.25775391,64.69150391)
\lineto(483.10775391,64.69150391)
\curveto(483.06774804,64.68150414)(483.01274809,64.67650415)(482.94275391,64.67650391)
\curveto(482.87274823,64.67650415)(482.81274829,64.68150414)(482.76275391,64.69150391)
\lineto(482.50775391,64.69150391)
\curveto(482.39774871,64.71150411)(482.29274881,64.7265041)(482.19275391,64.73650391)
\curveto(482.102749,64.73650409)(482.0077491,64.75150407)(481.90775391,64.78150391)
\curveto(481.72774938,64.83150399)(481.55274955,64.87650395)(481.38275391,64.91650391)
\curveto(481.21274989,64.95650387)(481.04775006,65.01150381)(480.88775391,65.08150391)
\curveto(480.31775079,65.34150348)(479.81775129,65.68650314)(479.38775391,66.11650391)
\curveto(478.95775215,66.55650227)(478.61275249,67.06650176)(478.35275391,67.64650391)
\curveto(478.3027528,67.76650106)(478.25275285,67.89150093)(478.20275391,68.02150391)
\curveto(478.16275294,68.15150067)(478.11775299,68.28650054)(478.06775391,68.42650391)
\curveto(478.05775305,68.46650036)(478.05275305,68.50150032)(478.05275391,68.53150391)
\curveto(478.05275305,68.57150025)(478.04275306,68.61150021)(478.02275391,68.65150391)
\curveto(477.99275311,68.76150006)(477.96775314,68.87649995)(477.94775391,68.99650391)
\curveto(477.93775317,69.11649971)(477.91775319,69.23649959)(477.88775391,69.35650391)
\curveto(477.87775323,69.39649943)(477.87275323,69.43649939)(477.87275391,69.47650391)
\curveto(477.88275322,69.51649931)(477.88275322,69.55149927)(477.87275391,69.58150391)
\lineto(477.87275391,69.71650391)
\curveto(477.85275325,69.76649906)(477.84275326,69.86149896)(477.84275391,70.00150391)
\curveto(477.84275326,70.14149868)(477.85275325,70.24149858)(477.87275391,70.30150391)
\lineto(477.87275391,70.43650391)
\lineto(477.87275391,70.61650391)
\curveto(477.87275323,70.67649815)(477.87775323,70.73649809)(477.88775391,70.79650391)
\curveto(477.89775321,70.84649798)(477.9027532,70.89649793)(477.90275391,70.94650391)
\curveto(477.9027532,71.00649782)(477.9077532,71.06149776)(477.91775391,71.11150391)
\curveto(477.94775316,71.23149759)(477.96775314,71.35149747)(477.97775391,71.47150391)
\curveto(477.99775311,71.59149723)(478.02275308,71.71149711)(478.05275391,71.83150391)
\curveto(478.15275295,72.16149666)(478.25275285,72.47649635)(478.35275391,72.77650391)
\curveto(478.46275264,73.08649574)(478.6027525,73.37149545)(478.77275391,73.63150391)
\curveto(479.07275203,74.10149472)(479.42775168,74.50149432)(479.83775391,74.83150391)
\curveto(480.25775085,75.17149365)(480.75275035,75.43649339)(481.32275391,75.62650391)
\curveto(481.43274967,75.66649316)(481.53774957,75.69149313)(481.63775391,75.70150391)
\curveto(481.74774936,75.7214931)(481.85774925,75.74649308)(481.96775391,75.77650391)
\curveto(482.01774909,75.79649303)(482.06274904,75.80649302)(482.10275391,75.80650391)
\curveto(482.15274895,75.80649302)(482.2027489,75.81149301)(482.25275391,75.82150391)
\curveto(482.33274877,75.83149299)(482.41274869,75.83649299)(482.49275391,75.83650391)
\curveto(482.58274852,75.84649298)(482.66774844,75.85149297)(482.74775391,75.85150391)
}
}
{
\newrgbcolor{curcolor}{0 0 0}
\pscustom[linestyle=none,fillstyle=solid,fillcolor=curcolor]
{
\newpath
\moveto(493.15423828,72.83650391)
\curveto(493.38423349,72.83649599)(493.51423336,72.77649605)(493.54423828,72.65650391)
\curveto(493.5742333,72.54649628)(493.58923329,72.38149644)(493.58923828,72.16150391)
\lineto(493.58923828,71.87650391)
\curveto(493.58923329,71.78649704)(493.56423331,71.71149711)(493.51423828,71.65150391)
\curveto(493.45423342,71.57149725)(493.36923351,71.5264973)(493.25923828,71.51650391)
\curveto(493.14923373,71.51649731)(493.03923384,71.50149732)(492.92923828,71.47150391)
\curveto(492.78923409,71.44149738)(492.65423422,71.41149741)(492.52423828,71.38150391)
\curveto(492.40423447,71.35149747)(492.28923459,71.31149751)(492.17923828,71.26150391)
\curveto(491.88923499,71.13149769)(491.65423522,70.95149787)(491.47423828,70.72150391)
\curveto(491.29423558,70.50149832)(491.13923574,70.24649858)(491.00923828,69.95650391)
\curveto(490.96923591,69.84649898)(490.93923594,69.73149909)(490.91923828,69.61150391)
\curveto(490.89923598,69.50149932)(490.874236,69.38649944)(490.84423828,69.26650391)
\curveto(490.83423604,69.21649961)(490.82923605,69.16649966)(490.82923828,69.11650391)
\curveto(490.83923604,69.06649976)(490.83923604,69.01649981)(490.82923828,68.96650391)
\curveto(490.79923608,68.84649998)(490.78423609,68.70650012)(490.78423828,68.54650391)
\curveto(490.79423608,68.39650043)(490.79923608,68.25150057)(490.79923828,68.11150391)
\lineto(490.79923828,66.26650391)
\lineto(490.79923828,65.92150391)
\curveto(490.79923608,65.80150302)(490.79423608,65.68650314)(490.78423828,65.57650391)
\curveto(490.7742361,65.46650336)(490.76923611,65.37150345)(490.76923828,65.29150391)
\curveto(490.7792361,65.21150361)(490.75923612,65.14150368)(490.70923828,65.08150391)
\curveto(490.65923622,65.01150381)(490.5792363,64.97150385)(490.46923828,64.96150391)
\curveto(490.36923651,64.95150387)(490.25923662,64.94650388)(490.13923828,64.94650391)
\lineto(489.86923828,64.94650391)
\curveto(489.81923706,64.96650386)(489.76923711,64.98150384)(489.71923828,64.99150391)
\curveto(489.6792372,65.01150381)(489.64923723,65.03650379)(489.62923828,65.06650391)
\curveto(489.5792373,65.13650369)(489.54923733,65.2215036)(489.53923828,65.32150391)
\lineto(489.53923828,65.65150391)
\lineto(489.53923828,66.80650391)
\lineto(489.53923828,70.96150391)
\lineto(489.53923828,71.99650391)
\lineto(489.53923828,72.29650391)
\curveto(489.54923733,72.39649643)(489.5792373,72.48149634)(489.62923828,72.55150391)
\curveto(489.65923722,72.59149623)(489.70923717,72.6214962)(489.77923828,72.64150391)
\curveto(489.85923702,72.66149616)(489.94423693,72.67149615)(490.03423828,72.67150391)
\curveto(490.12423675,72.68149614)(490.21423666,72.68149614)(490.30423828,72.67150391)
\curveto(490.39423648,72.66149616)(490.46423641,72.64649618)(490.51423828,72.62650391)
\curveto(490.59423628,72.59649623)(490.64423623,72.53649629)(490.66423828,72.44650391)
\curveto(490.69423618,72.36649646)(490.70923617,72.27649655)(490.70923828,72.17650391)
\lineto(490.70923828,71.87650391)
\curveto(490.70923617,71.77649705)(490.72923615,71.68649714)(490.76923828,71.60650391)
\curveto(490.7792361,71.58649724)(490.78923609,71.57149725)(490.79923828,71.56150391)
\lineto(490.84423828,71.51650391)
\curveto(490.95423592,71.51649731)(491.04423583,71.56149726)(491.11423828,71.65150391)
\curveto(491.18423569,71.75149707)(491.24423563,71.83149699)(491.29423828,71.89150391)
\lineto(491.38423828,71.98150391)
\curveto(491.4742354,72.09149673)(491.59923528,72.20649662)(491.75923828,72.32650391)
\curveto(491.91923496,72.44649638)(492.06923481,72.53649629)(492.20923828,72.59650391)
\curveto(492.29923458,72.64649618)(492.39423448,72.68149614)(492.49423828,72.70150391)
\curveto(492.59423428,72.73149609)(492.69923418,72.76149606)(492.80923828,72.79150391)
\curveto(492.86923401,72.80149602)(492.92923395,72.80649602)(492.98923828,72.80650391)
\curveto(493.04923383,72.81649601)(493.10423377,72.826496)(493.15423828,72.83650391)
}
}
{
\newrgbcolor{curcolor}{0 0 0}
\pscustom[linestyle=none,fillstyle=solid,fillcolor=curcolor]
{
\newpath
\moveto(494.98400391,72.65650391)
\lineto(495.41900391,72.65650391)
\curveto(495.56900194,72.65649617)(495.67400184,72.61649621)(495.73400391,72.53650391)
\curveto(495.78400173,72.45649637)(495.8090017,72.35649647)(495.80900391,72.23650391)
\curveto(495.81900169,72.11649671)(495.82400169,71.99649683)(495.82400391,71.87650391)
\lineto(495.82400391,70.45150391)
\lineto(495.82400391,68.18650391)
\lineto(495.82400391,67.49650391)
\curveto(495.82400169,67.26650156)(495.84900166,67.06650176)(495.89900391,66.89650391)
\curveto(496.05900145,66.44650238)(496.35900115,66.13150269)(496.79900391,65.95150391)
\curveto(497.01900049,65.86150296)(497.28400023,65.826503)(497.59400391,65.84650391)
\curveto(497.90399961,65.87650295)(498.15399936,65.93150289)(498.34400391,66.01150391)
\curveto(498.67399884,66.15150267)(498.93399858,66.3265025)(499.12400391,66.53650391)
\curveto(499.32399819,66.75650207)(499.47899803,67.04150178)(499.58900391,67.39150391)
\curveto(499.61899789,67.47150135)(499.63899787,67.55150127)(499.64900391,67.63150391)
\curveto(499.65899785,67.71150111)(499.67399784,67.79650103)(499.69400391,67.88650391)
\curveto(499.70399781,67.93650089)(499.70399781,67.98150084)(499.69400391,68.02150391)
\curveto(499.69399782,68.06150076)(499.70399781,68.10650072)(499.72400391,68.15650391)
\lineto(499.72400391,68.47150391)
\curveto(499.74399777,68.55150027)(499.74899776,68.64150018)(499.73900391,68.74150391)
\curveto(499.72899778,68.85149997)(499.72399779,68.95149987)(499.72400391,69.04150391)
\lineto(499.72400391,70.21150391)
\lineto(499.72400391,71.80150391)
\curveto(499.72399779,71.9214969)(499.71899779,72.04649678)(499.70900391,72.17650391)
\curveto(499.7089978,72.31649651)(499.73399778,72.4264964)(499.78400391,72.50650391)
\curveto(499.82399769,72.55649627)(499.86899764,72.58649624)(499.91900391,72.59650391)
\curveto(499.97899753,72.61649621)(500.04899746,72.63649619)(500.12900391,72.65650391)
\lineto(500.35400391,72.65650391)
\curveto(500.47399704,72.65649617)(500.57899693,72.65149617)(500.66900391,72.64150391)
\curveto(500.76899674,72.63149619)(500.84399667,72.58649624)(500.89400391,72.50650391)
\curveto(500.94399657,72.45649637)(500.96899654,72.38149644)(500.96900391,72.28150391)
\lineto(500.96900391,71.99650391)
\lineto(500.96900391,70.97650391)
\lineto(500.96900391,66.94150391)
\lineto(500.96900391,65.59150391)
\curveto(500.96899654,65.47150335)(500.96399655,65.35650347)(500.95400391,65.24650391)
\curveto(500.95399656,65.14650368)(500.91899659,65.07150375)(500.84900391,65.02150391)
\curveto(500.8089967,64.99150383)(500.74899676,64.96650386)(500.66900391,64.94650391)
\curveto(500.58899692,64.93650389)(500.49899701,64.9265039)(500.39900391,64.91650391)
\curveto(500.3089972,64.91650391)(500.21899729,64.9215039)(500.12900391,64.93150391)
\curveto(500.04899746,64.94150388)(499.98899752,64.96150386)(499.94900391,64.99150391)
\curveto(499.89899761,65.03150379)(499.85399766,65.09650373)(499.81400391,65.18650391)
\curveto(499.80399771,65.2265036)(499.79399772,65.28150354)(499.78400391,65.35150391)
\curveto(499.78399773,65.4215034)(499.77899773,65.48650334)(499.76900391,65.54650391)
\curveto(499.75899775,65.61650321)(499.73899777,65.67150315)(499.70900391,65.71150391)
\curveto(499.67899783,65.75150307)(499.63399788,65.76650306)(499.57400391,65.75650391)
\curveto(499.49399802,65.73650309)(499.4139981,65.67650315)(499.33400391,65.57650391)
\curveto(499.25399826,65.48650334)(499.17899833,65.41650341)(499.10900391,65.36650391)
\curveto(498.88899862,65.20650362)(498.63899887,65.06650376)(498.35900391,64.94650391)
\curveto(498.24899926,64.89650393)(498.13399938,64.86650396)(498.01400391,64.85650391)
\curveto(497.90399961,64.83650399)(497.78899972,64.81150401)(497.66900391,64.78150391)
\curveto(497.61899989,64.77150405)(497.56399995,64.77150405)(497.50400391,64.78150391)
\curveto(497.45400006,64.79150403)(497.40400011,64.78650404)(497.35400391,64.76650391)
\curveto(497.25400026,64.74650408)(497.16400035,64.74650408)(497.08400391,64.76650391)
\lineto(496.93400391,64.76650391)
\curveto(496.88400063,64.78650404)(496.82400069,64.79650403)(496.75400391,64.79650391)
\curveto(496.69400082,64.79650403)(496.63900087,64.80150402)(496.58900391,64.81150391)
\curveto(496.54900096,64.83150399)(496.509001,64.84150398)(496.46900391,64.84150391)
\curveto(496.43900107,64.83150399)(496.39900111,64.83650399)(496.34900391,64.85650391)
\lineto(496.10900391,64.91650391)
\curveto(496.03900147,64.93650389)(495.96400155,64.96650386)(495.88400391,65.00650391)
\curveto(495.62400189,65.11650371)(495.40400211,65.26150356)(495.22400391,65.44150391)
\curveto(495.05400246,65.63150319)(494.9140026,65.85650297)(494.80400391,66.11650391)
\curveto(494.76400275,66.20650262)(494.73400278,66.29650253)(494.71400391,66.38650391)
\lineto(494.65400391,66.68650391)
\curveto(494.63400288,66.74650208)(494.62400289,66.80150202)(494.62400391,66.85150391)
\curveto(494.63400288,66.91150191)(494.62900288,66.97650185)(494.60900391,67.04650391)
\curveto(494.59900291,67.06650176)(494.59400292,67.09150173)(494.59400391,67.12150391)
\curveto(494.59400292,67.16150166)(494.58900292,67.19650163)(494.57900391,67.22650391)
\lineto(494.57900391,67.37650391)
\curveto(494.56900294,67.41650141)(494.56400295,67.46150136)(494.56400391,67.51150391)
\curveto(494.57400294,67.57150125)(494.57900293,67.6265012)(494.57900391,67.67650391)
\lineto(494.57900391,68.27650391)
\lineto(494.57900391,71.03650391)
\lineto(494.57900391,71.99650391)
\lineto(494.57900391,72.26650391)
\curveto(494.57900293,72.35649647)(494.59900291,72.43149639)(494.63900391,72.49150391)
\curveto(494.67900283,72.56149626)(494.75400276,72.61149621)(494.86400391,72.64150391)
\curveto(494.88400263,72.65149617)(494.90400261,72.65149617)(494.92400391,72.64150391)
\curveto(494.94400257,72.64149618)(494.96400255,72.64649618)(494.98400391,72.65650391)
}
}
{
\newrgbcolor{curcolor}{0 0 0}
\pscustom[linestyle=none,fillstyle=solid,fillcolor=curcolor]
{
\newpath
\moveto(510.29361328,68.99650391)
\curveto(510.30360493,68.94649988)(510.30860493,68.88149994)(510.30861328,68.80150391)
\curveto(510.30860493,68.7215001)(510.30360493,68.65650017)(510.29361328,68.60650391)
\curveto(510.27360496,68.55650027)(510.26860497,68.50650032)(510.27861328,68.45650391)
\curveto(510.28860495,68.41650041)(510.28860495,68.37650045)(510.27861328,68.33650391)
\curveto(510.27860496,68.26650056)(510.27360496,68.21150061)(510.26361328,68.17150391)
\curveto(510.24360499,68.08150074)(510.22860501,67.99150083)(510.21861328,67.90150391)
\curveto(510.21860502,67.81150101)(510.20860503,67.7215011)(510.18861328,67.63150391)
\lineto(510.12861328,67.39150391)
\curveto(510.10860513,67.3215015)(510.08360515,67.24650158)(510.05361328,67.16650391)
\curveto(509.9336053,66.79650203)(509.76860547,66.46150236)(509.55861328,66.16150391)
\curveto(509.49860574,66.07150275)(509.4336058,65.98150284)(509.36361328,65.89150391)
\curveto(509.29360594,65.81150301)(509.21860602,65.73650309)(509.13861328,65.66650391)
\lineto(509.06361328,65.59150391)
\curveto(508.99360624,65.54150328)(508.92860631,65.49150333)(508.86861328,65.44150391)
\curveto(508.80860643,65.39150343)(508.7386065,65.34150348)(508.65861328,65.29150391)
\curveto(508.54860669,65.21150361)(508.42360681,65.14150368)(508.28361328,65.08150391)
\curveto(508.15360708,65.03150379)(508.01860722,64.98150384)(507.87861328,64.93150391)
\curveto(507.79860744,64.91150391)(507.71860752,64.89650393)(507.63861328,64.88650391)
\curveto(507.56860767,64.87650395)(507.49360774,64.86150396)(507.41361328,64.84150391)
\lineto(507.35361328,64.84150391)
\curveto(507.34360789,64.83150399)(507.32860791,64.826504)(507.30861328,64.82650391)
\curveto(507.21860802,64.80650402)(507.08360815,64.79650403)(506.90361328,64.79650391)
\curveto(506.7336085,64.78650404)(506.59860864,64.79150403)(506.49861328,64.81150391)
\lineto(506.42361328,64.81150391)
\curveto(506.35360888,64.821504)(506.28860895,64.83150399)(506.22861328,64.84150391)
\curveto(506.16860907,64.84150398)(506.10860913,64.85150397)(506.04861328,64.87150391)
\curveto(505.87860936,64.9215039)(505.71860952,64.96650386)(505.56861328,65.00650391)
\curveto(505.41860982,65.04650378)(505.27860996,65.10650372)(505.14861328,65.18650391)
\curveto(504.98861025,65.27650355)(504.84861039,65.37150345)(504.72861328,65.47150391)
\curveto(504.68861055,65.50150332)(504.62861061,65.54150328)(504.54861328,65.59150391)
\curveto(504.46861077,65.65150317)(504.39361084,65.65650317)(504.32361328,65.60650391)
\curveto(504.28361095,65.57650325)(504.26361097,65.53650329)(504.26361328,65.48650391)
\curveto(504.26361097,65.43650339)(504.25361098,65.38150344)(504.23361328,65.32150391)
\curveto(504.22361101,65.29150353)(504.22361101,65.25650357)(504.23361328,65.21650391)
\curveto(504.24361099,65.18650364)(504.24361099,65.15150367)(504.23361328,65.11150391)
\curveto(504.21361102,65.05150377)(504.20361103,64.98650384)(504.20361328,64.91650391)
\curveto(504.21361102,64.83650399)(504.21861102,64.76650406)(504.21861328,64.70650391)
\lineto(504.21861328,62.90650391)
\lineto(504.21861328,62.47150391)
\curveto(504.21861102,62.3215065)(504.18861105,62.20650662)(504.12861328,62.12650391)
\curveto(504.07861116,62.05650677)(503.99861124,62.0215068)(503.88861328,62.02150391)
\curveto(503.77861146,62.01150681)(503.66861157,62.00650682)(503.55861328,62.00650391)
\lineto(503.31861328,62.00650391)
\curveto(503.24861199,62.0265068)(503.18861205,62.04650678)(503.13861328,62.06650391)
\curveto(503.09861214,62.08650674)(503.06361217,62.1215067)(503.03361328,62.17150391)
\curveto(502.98361225,62.24150658)(502.95861228,62.35150647)(502.95861328,62.50150391)
\curveto(502.96861227,62.65150617)(502.97361226,62.78150604)(502.97361328,62.89150391)
\lineto(502.97361328,71.89150391)
\lineto(502.97361328,72.25150391)
\curveto(502.98361225,72.38149644)(503.01361222,72.48649634)(503.06361328,72.56650391)
\curveto(503.09361214,72.60649622)(503.15861208,72.63649619)(503.25861328,72.65650391)
\curveto(503.36861187,72.68649614)(503.48361175,72.69649613)(503.60361328,72.68650391)
\curveto(503.72361151,72.68649614)(503.8336114,72.67149615)(503.93361328,72.64150391)
\curveto(504.04361119,72.6214962)(504.11361112,72.59149623)(504.14361328,72.55150391)
\curveto(504.18361105,72.50149632)(504.20361103,72.44149638)(504.20361328,72.37150391)
\curveto(504.21361102,72.30149652)(504.233611,72.23149659)(504.26361328,72.16150391)
\curveto(504.28361095,72.13149669)(504.29861094,72.10649672)(504.30861328,72.08650391)
\curveto(504.32861091,72.07649675)(504.34861089,72.06149676)(504.36861328,72.04150391)
\curveto(504.47861076,72.03149679)(504.56861067,72.06649676)(504.63861328,72.14650391)
\curveto(504.71861052,72.2264966)(504.79361044,72.29149653)(504.86361328,72.34150391)
\curveto(505.12361011,72.5214963)(505.4336098,72.66149616)(505.79361328,72.76150391)
\curveto(505.88360935,72.78149604)(505.97360926,72.79649603)(506.06361328,72.80650391)
\curveto(506.16360907,72.81649601)(506.26360897,72.83149599)(506.36361328,72.85150391)
\curveto(506.40360883,72.86149596)(506.45360878,72.86149596)(506.51361328,72.85150391)
\curveto(506.57360866,72.84149598)(506.61360862,72.84649598)(506.63361328,72.86650391)
\curveto(507.06360817,72.87649595)(507.44360779,72.83149599)(507.77361328,72.73150391)
\curveto(508.10360713,72.64149618)(508.39860684,72.51149631)(508.65861328,72.34150391)
\lineto(508.80861328,72.22150391)
\curveto(508.85860638,72.19149663)(508.90860633,72.15649667)(508.95861328,72.11650391)
\curveto(508.97860626,72.09649673)(508.99360624,72.07649675)(509.00361328,72.05650391)
\curveto(509.02360621,72.04649678)(509.04360619,72.03149679)(509.06361328,72.01150391)
\curveto(509.11360612,71.96149686)(509.16860607,71.90649692)(509.22861328,71.84650391)
\curveto(509.28860595,71.78649704)(509.34360589,71.7264971)(509.39361328,71.66650391)
\curveto(509.51360572,71.49649733)(509.6386056,71.31149751)(509.76861328,71.11150391)
\curveto(509.84860539,70.98149784)(509.91360532,70.83649799)(509.96361328,70.67650391)
\curveto(510.02360521,70.51649831)(510.07860516,70.35649847)(510.12861328,70.19650391)
\curveto(510.14860509,70.11649871)(510.16360507,70.03149879)(510.17361328,69.94150391)
\curveto(510.19360504,69.85149897)(510.21360502,69.76649906)(510.23361328,69.68650391)
\lineto(510.23361328,69.56650391)
\curveto(510.24360499,69.53649929)(510.24860499,69.50649932)(510.24861328,69.47650391)
\curveto(510.26860497,69.4264994)(510.27360496,69.37149945)(510.26361328,69.31150391)
\curveto(510.26360497,69.25149957)(510.27360496,69.19649963)(510.29361328,69.14650391)
\lineto(510.29361328,68.99650391)
\moveto(508.95861328,68.59150391)
\curveto(508.97860626,68.64150018)(508.98360625,68.70150012)(508.97361328,68.77150391)
\curveto(508.96360627,68.85149997)(508.95860628,68.9214999)(508.95861328,68.98150391)
\curveto(508.95860628,69.15149967)(508.94860629,69.31149951)(508.92861328,69.46150391)
\curveto(508.91860632,69.61149921)(508.88860635,69.75649907)(508.83861328,69.89650391)
\lineto(508.77861328,70.07650391)
\curveto(508.76860647,70.14649868)(508.74860649,70.21149861)(508.71861328,70.27150391)
\curveto(508.60860663,70.54149828)(508.4336068,70.80149802)(508.19361328,71.05150391)
\curveto(507.96360727,71.30149752)(507.74360749,71.47149735)(507.53361328,71.56150391)
\curveto(507.45360778,71.60149722)(507.36860787,71.63149719)(507.27861328,71.65150391)
\curveto(507.19860804,71.67149715)(507.11360812,71.69649713)(507.02361328,71.72650391)
\curveto(506.9336083,71.74649708)(506.82860841,71.75649707)(506.70861328,71.75650391)
\lineto(506.37861328,71.75650391)
\curveto(506.35860888,71.73649709)(506.31860892,71.7264971)(506.25861328,71.72650391)
\curveto(506.20860903,71.73649709)(506.16360907,71.73649709)(506.12361328,71.72650391)
\lineto(505.85361328,71.66650391)
\curveto(505.77360946,71.64649718)(505.69360954,71.61649721)(505.61361328,71.57650391)
\curveto(505.29360994,71.43649739)(505.02861021,71.23149759)(504.81861328,70.96150391)
\curveto(504.61861062,70.70149812)(504.46361077,70.39649843)(504.35361328,70.04650391)
\curveto(504.31361092,69.93649889)(504.28361095,69.826499)(504.26361328,69.71650391)
\curveto(504.25361098,69.60649922)(504.238611,69.49649933)(504.21861328,69.38650391)
\curveto(504.20861103,69.34649948)(504.20361103,69.30649952)(504.20361328,69.26650391)
\curveto(504.20361103,69.23649959)(504.19861104,69.20149962)(504.18861328,69.16150391)
\lineto(504.18861328,69.04150391)
\curveto(504.17861106,68.99149983)(504.17361106,68.91649991)(504.17361328,68.81650391)
\curveto(504.17361106,68.7265001)(504.17861106,68.65650017)(504.18861328,68.60650391)
\lineto(504.18861328,68.48650391)
\curveto(504.19861104,68.44650038)(504.20361103,68.40650042)(504.20361328,68.36650391)
\curveto(504.20361103,68.3265005)(504.20861103,68.29150053)(504.21861328,68.26150391)
\curveto(504.22861101,68.23150059)(504.233611,68.20150062)(504.23361328,68.17150391)
\curveto(504.233611,68.14150068)(504.238611,68.10650072)(504.24861328,68.06650391)
\curveto(504.26861097,67.98650084)(504.28361095,67.90650092)(504.29361328,67.82650391)
\lineto(504.35361328,67.58650391)
\curveto(504.46361077,67.24650158)(504.61361062,66.94650188)(504.80361328,66.68650391)
\curveto(505.00361023,66.43650239)(505.26360997,66.24150258)(505.58361328,66.10150391)
\curveto(505.77360946,66.0215028)(505.96860927,65.96150286)(506.16861328,65.92150391)
\curveto(506.20860903,65.90150292)(506.24860899,65.89150293)(506.28861328,65.89150391)
\curveto(506.32860891,65.90150292)(506.36860887,65.90150292)(506.40861328,65.89150391)
\lineto(506.52861328,65.89150391)
\curveto(506.59860864,65.87150295)(506.66860857,65.87150295)(506.73861328,65.89150391)
\lineto(506.85861328,65.89150391)
\curveto(506.96860827,65.91150291)(507.07360816,65.9265029)(507.17361328,65.93650391)
\curveto(507.27360796,65.94650288)(507.37360786,65.97150285)(507.47361328,66.01150391)
\curveto(507.78360745,66.14150268)(508.0336072,66.31150251)(508.22361328,66.52150391)
\curveto(508.42360681,66.74150208)(508.58860665,67.00650182)(508.71861328,67.31650391)
\curveto(508.76860647,67.45650137)(508.80360643,67.59650123)(508.82361328,67.73650391)
\curveto(508.85360638,67.88650094)(508.88860635,68.04150078)(508.92861328,68.20150391)
\curveto(508.9386063,68.25150057)(508.94360629,68.29650053)(508.94361328,68.33650391)
\curveto(508.94360629,68.37650045)(508.94860629,68.4215004)(508.95861328,68.47150391)
\lineto(508.95861328,68.59150391)
}
}
{
\newrgbcolor{curcolor}{0 0 0}
\pscustom[linestyle=none,fillstyle=solid,fillcolor=curcolor]
{
\newpath
\moveto(518.89986328,69.13150391)
\curveto(518.91985522,69.07149975)(518.92985521,68.97649985)(518.92986328,68.84650391)
\curveto(518.92985521,68.7265001)(518.92485522,68.64150018)(518.91486328,68.59150391)
\lineto(518.91486328,68.44150391)
\curveto(518.90485524,68.36150046)(518.89485525,68.28650054)(518.88486328,68.21650391)
\curveto(518.88485526,68.15650067)(518.87985526,68.08650074)(518.86986328,68.00650391)
\curveto(518.84985529,67.94650088)(518.83485531,67.88650094)(518.82486328,67.82650391)
\curveto(518.82485532,67.76650106)(518.81485533,67.70650112)(518.79486328,67.64650391)
\curveto(518.75485539,67.51650131)(518.71985542,67.38650144)(518.68986328,67.25650391)
\curveto(518.65985548,67.1265017)(518.61985552,67.00650182)(518.56986328,66.89650391)
\curveto(518.35985578,66.41650241)(518.07985606,66.01150281)(517.72986328,65.68150391)
\curveto(517.37985676,65.36150346)(516.94985719,65.11650371)(516.43986328,64.94650391)
\curveto(516.32985781,64.90650392)(516.20985793,64.87650395)(516.07986328,64.85650391)
\curveto(515.95985818,64.83650399)(515.83485831,64.81650401)(515.70486328,64.79650391)
\curveto(515.6448585,64.78650404)(515.57985856,64.78150404)(515.50986328,64.78150391)
\curveto(515.44985869,64.77150405)(515.38985875,64.76650406)(515.32986328,64.76650391)
\curveto(515.28985885,64.75650407)(515.22985891,64.75150407)(515.14986328,64.75150391)
\curveto(515.07985906,64.75150407)(515.02985911,64.75650407)(514.99986328,64.76650391)
\curveto(514.95985918,64.77650405)(514.91985922,64.78150404)(514.87986328,64.78150391)
\curveto(514.8398593,64.77150405)(514.80485934,64.77150405)(514.77486328,64.78150391)
\lineto(514.68486328,64.78150391)
\lineto(514.32486328,64.82650391)
\curveto(514.18485996,64.86650396)(514.04986009,64.90650392)(513.91986328,64.94650391)
\curveto(513.78986035,64.98650384)(513.66486048,65.03150379)(513.54486328,65.08150391)
\curveto(513.09486105,65.28150354)(512.72486142,65.54150328)(512.43486328,65.86150391)
\curveto(512.144862,66.18150264)(511.90486224,66.57150225)(511.71486328,67.03150391)
\curveto(511.66486248,67.13150169)(511.62486252,67.23150159)(511.59486328,67.33150391)
\curveto(511.57486257,67.43150139)(511.55486259,67.53650129)(511.53486328,67.64650391)
\curveto(511.51486263,67.68650114)(511.50486264,67.71650111)(511.50486328,67.73650391)
\curveto(511.51486263,67.76650106)(511.51486263,67.80150102)(511.50486328,67.84150391)
\curveto(511.48486266,67.9215009)(511.46986267,68.00150082)(511.45986328,68.08150391)
\curveto(511.45986268,68.17150065)(511.44986269,68.25650057)(511.42986328,68.33650391)
\lineto(511.42986328,68.45650391)
\curveto(511.42986271,68.49650033)(511.42486272,68.54150028)(511.41486328,68.59150391)
\curveto(511.40486274,68.64150018)(511.39986274,68.7265001)(511.39986328,68.84650391)
\curveto(511.39986274,68.97649985)(511.40986273,69.07149975)(511.42986328,69.13150391)
\curveto(511.44986269,69.20149962)(511.45486269,69.27149955)(511.44486328,69.34150391)
\curveto(511.43486271,69.41149941)(511.4398627,69.48149934)(511.45986328,69.55150391)
\curveto(511.46986267,69.60149922)(511.47486267,69.64149918)(511.47486328,69.67150391)
\curveto(511.48486266,69.71149911)(511.49486265,69.75649907)(511.50486328,69.80650391)
\curveto(511.53486261,69.9264989)(511.55986258,70.04649878)(511.57986328,70.16650391)
\curveto(511.60986253,70.28649854)(511.64986249,70.40149842)(511.69986328,70.51150391)
\curveto(511.84986229,70.88149794)(512.02986211,71.21149761)(512.23986328,71.50150391)
\curveto(512.45986168,71.80149702)(512.72486142,72.05149677)(513.03486328,72.25150391)
\curveto(513.15486099,72.33149649)(513.27986086,72.39649643)(513.40986328,72.44650391)
\curveto(513.5398606,72.50649632)(513.67486047,72.56649626)(513.81486328,72.62650391)
\curveto(513.93486021,72.67649615)(514.06486008,72.70649612)(514.20486328,72.71650391)
\curveto(514.3448598,72.73649609)(514.48485966,72.76649606)(514.62486328,72.80650391)
\lineto(514.81986328,72.80650391)
\curveto(514.88985925,72.81649601)(514.95485919,72.826496)(515.01486328,72.83650391)
\curveto(515.90485824,72.84649598)(516.6448575,72.66149616)(517.23486328,72.28150391)
\curveto(517.82485632,71.90149692)(518.24985589,71.40649742)(518.50986328,70.79650391)
\curveto(518.55985558,70.69649813)(518.59985554,70.59649823)(518.62986328,70.49650391)
\curveto(518.65985548,70.39649843)(518.69485545,70.29149853)(518.73486328,70.18150391)
\curveto(518.76485538,70.07149875)(518.78985535,69.95149887)(518.80986328,69.82150391)
\curveto(518.82985531,69.70149912)(518.85485529,69.57649925)(518.88486328,69.44650391)
\curveto(518.89485525,69.39649943)(518.89485525,69.34149948)(518.88486328,69.28150391)
\curveto(518.88485526,69.23149959)(518.88985525,69.18149964)(518.89986328,69.13150391)
\moveto(517.56486328,68.27650391)
\curveto(517.58485656,68.34650048)(517.58985655,68.4265004)(517.57986328,68.51650391)
\lineto(517.57986328,68.77150391)
\curveto(517.57985656,69.16149966)(517.5448566,69.49149933)(517.47486328,69.76150391)
\curveto(517.4448567,69.84149898)(517.41985672,69.9214989)(517.39986328,70.00150391)
\curveto(517.37985676,70.08149874)(517.35485679,70.15649867)(517.32486328,70.22650391)
\curveto(517.0448571,70.87649795)(516.59985754,71.3264975)(515.98986328,71.57650391)
\curveto(515.91985822,71.60649722)(515.8448583,71.6264972)(515.76486328,71.63650391)
\lineto(515.52486328,71.69650391)
\curveto(515.4448587,71.71649711)(515.35985878,71.7264971)(515.26986328,71.72650391)
\lineto(514.99986328,71.72650391)
\lineto(514.72986328,71.68150391)
\curveto(514.62985951,71.66149716)(514.53485961,71.63649719)(514.44486328,71.60650391)
\curveto(514.36485978,71.58649724)(514.28485986,71.55649727)(514.20486328,71.51650391)
\curveto(514.13486001,71.49649733)(514.06986007,71.46649736)(514.00986328,71.42650391)
\curveto(513.94986019,71.38649744)(513.89486025,71.34649748)(513.84486328,71.30650391)
\curveto(513.60486054,71.13649769)(513.40986073,70.93149789)(513.25986328,70.69150391)
\curveto(513.10986103,70.45149837)(512.97986116,70.17149865)(512.86986328,69.85150391)
\curveto(512.8398613,69.75149907)(512.81986132,69.64649918)(512.80986328,69.53650391)
\curveto(512.79986134,69.43649939)(512.78486136,69.33149949)(512.76486328,69.22150391)
\curveto(512.75486139,69.18149964)(512.74986139,69.11649971)(512.74986328,69.02650391)
\curveto(512.7398614,68.99649983)(512.73486141,68.96149986)(512.73486328,68.92150391)
\curveto(512.7448614,68.88149994)(512.74986139,68.83649999)(512.74986328,68.78650391)
\lineto(512.74986328,68.48650391)
\curveto(512.74986139,68.38650044)(512.75986138,68.29650053)(512.77986328,68.21650391)
\lineto(512.80986328,68.03650391)
\curveto(512.82986131,67.93650089)(512.8448613,67.83650099)(512.85486328,67.73650391)
\curveto(512.87486127,67.64650118)(512.90486124,67.56150126)(512.94486328,67.48150391)
\curveto(513.0448611,67.24150158)(513.15986098,67.01650181)(513.28986328,66.80650391)
\curveto(513.42986071,66.59650223)(513.59986054,66.4215024)(513.79986328,66.28150391)
\curveto(513.84986029,66.25150257)(513.89486025,66.2265026)(513.93486328,66.20650391)
\curveto(513.97486017,66.18650264)(514.01986012,66.16150266)(514.06986328,66.13150391)
\curveto(514.14985999,66.08150274)(514.23485991,66.03650279)(514.32486328,65.99650391)
\curveto(514.42485972,65.96650286)(514.52985961,65.93650289)(514.63986328,65.90650391)
\curveto(514.68985945,65.88650294)(514.73485941,65.87650295)(514.77486328,65.87650391)
\curveto(514.82485932,65.88650294)(514.87485927,65.88650294)(514.92486328,65.87650391)
\curveto(514.95485919,65.86650296)(515.01485913,65.85650297)(515.10486328,65.84650391)
\curveto(515.20485894,65.83650299)(515.27985886,65.84150298)(515.32986328,65.86150391)
\curveto(515.36985877,65.87150295)(515.40985873,65.87150295)(515.44986328,65.86150391)
\curveto(515.48985865,65.86150296)(515.52985861,65.87150295)(515.56986328,65.89150391)
\curveto(515.64985849,65.91150291)(515.72985841,65.9265029)(515.80986328,65.93650391)
\curveto(515.88985825,65.95650287)(515.96485818,65.98150284)(516.03486328,66.01150391)
\curveto(516.37485777,66.15150267)(516.64985749,66.34650248)(516.85986328,66.59650391)
\curveto(517.06985707,66.84650198)(517.2448569,67.14150168)(517.38486328,67.48150391)
\curveto(517.43485671,67.60150122)(517.46485668,67.7265011)(517.47486328,67.85650391)
\curveto(517.49485665,67.99650083)(517.52485662,68.13650069)(517.56486328,68.27650391)
}
}
{
\newrgbcolor{curcolor}{0 0 0}
\pscustom[linestyle=none,fillstyle=solid,fillcolor=curcolor]
{
\newpath
\moveto(522.81814453,72.83650391)
\curveto(523.53814047,72.84649598)(524.14313986,72.76149606)(524.63314453,72.58150391)
\curveto(525.12313888,72.41149641)(525.5031385,72.10649672)(525.77314453,71.66650391)
\curveto(525.84313816,71.55649727)(525.89813811,71.44149738)(525.93814453,71.32150391)
\curveto(525.97813803,71.21149761)(526.01813799,71.08649774)(526.05814453,70.94650391)
\curveto(526.07813793,70.87649795)(526.08313792,70.80149802)(526.07314453,70.72150391)
\curveto(526.06313794,70.65149817)(526.04813796,70.59649823)(526.02814453,70.55650391)
\curveto(526.008138,70.53649829)(525.98313802,70.51649831)(525.95314453,70.49650391)
\curveto(525.92313808,70.48649834)(525.89813811,70.47149835)(525.87814453,70.45150391)
\curveto(525.82813818,70.43149839)(525.77813823,70.4264984)(525.72814453,70.43650391)
\curveto(525.67813833,70.44649838)(525.62813838,70.44649838)(525.57814453,70.43650391)
\curveto(525.49813851,70.41649841)(525.39313861,70.41149841)(525.26314453,70.42150391)
\curveto(525.13313887,70.44149838)(525.04313896,70.46649836)(524.99314453,70.49650391)
\curveto(524.91313909,70.54649828)(524.85813915,70.61149821)(524.82814453,70.69150391)
\curveto(524.8081392,70.78149804)(524.77313923,70.86649796)(524.72314453,70.94650391)
\curveto(524.63313937,71.10649772)(524.5081395,71.25149757)(524.34814453,71.38150391)
\curveto(524.23813977,71.46149736)(524.11813989,71.5214973)(523.98814453,71.56150391)
\curveto(523.85814015,71.60149722)(523.71814029,71.64149718)(523.56814453,71.68150391)
\curveto(523.51814049,71.70149712)(523.46814054,71.70649712)(523.41814453,71.69650391)
\curveto(523.36814064,71.69649713)(523.31814069,71.70149712)(523.26814453,71.71150391)
\curveto(523.2081408,71.73149709)(523.13314087,71.74149708)(523.04314453,71.74150391)
\curveto(522.95314105,71.74149708)(522.87814113,71.73149709)(522.81814453,71.71150391)
\lineto(522.72814453,71.71150391)
\lineto(522.57814453,71.68150391)
\curveto(522.52814148,71.68149714)(522.47814153,71.67649715)(522.42814453,71.66650391)
\curveto(522.16814184,71.60649722)(521.95314205,71.5214973)(521.78314453,71.41150391)
\curveto(521.61314239,71.30149752)(521.49814251,71.11649771)(521.43814453,70.85650391)
\curveto(521.41814259,70.78649804)(521.41314259,70.71649811)(521.42314453,70.64650391)
\curveto(521.44314256,70.57649825)(521.46314254,70.51649831)(521.48314453,70.46650391)
\curveto(521.54314246,70.31649851)(521.61314239,70.20649862)(521.69314453,70.13650391)
\curveto(521.78314222,70.07649875)(521.89314211,70.00649882)(522.02314453,69.92650391)
\curveto(522.18314182,69.826499)(522.36314164,69.75149907)(522.56314453,69.70150391)
\curveto(522.76314124,69.66149916)(522.96314104,69.61149921)(523.16314453,69.55150391)
\curveto(523.29314071,69.51149931)(523.42314058,69.48149934)(523.55314453,69.46150391)
\curveto(523.68314032,69.44149938)(523.81314019,69.41149941)(523.94314453,69.37150391)
\curveto(524.15313985,69.31149951)(524.35813965,69.25149957)(524.55814453,69.19150391)
\curveto(524.75813925,69.14149968)(524.95813905,69.07649975)(525.15814453,68.99650391)
\lineto(525.30814453,68.93650391)
\curveto(525.35813865,68.91649991)(525.4081386,68.89149993)(525.45814453,68.86150391)
\curveto(525.65813835,68.74150008)(525.83313817,68.60650022)(525.98314453,68.45650391)
\curveto(526.13313787,68.30650052)(526.25813775,68.11650071)(526.35814453,67.88650391)
\curveto(526.37813763,67.81650101)(526.39813761,67.7215011)(526.41814453,67.60150391)
\curveto(526.43813757,67.53150129)(526.44813756,67.45650137)(526.44814453,67.37650391)
\curveto(526.45813755,67.30650152)(526.46313754,67.2265016)(526.46314453,67.13650391)
\lineto(526.46314453,66.98650391)
\curveto(526.44313756,66.91650191)(526.43313757,66.84650198)(526.43314453,66.77650391)
\curveto(526.43313757,66.70650212)(526.42313758,66.63650219)(526.40314453,66.56650391)
\curveto(526.37313763,66.45650237)(526.33813767,66.35150247)(526.29814453,66.25150391)
\curveto(526.25813775,66.15150267)(526.21313779,66.06150276)(526.16314453,65.98150391)
\curveto(526.003138,65.7215031)(525.79813821,65.51150331)(525.54814453,65.35150391)
\curveto(525.29813871,65.20150362)(525.01813899,65.07150375)(524.70814453,64.96150391)
\curveto(524.61813939,64.93150389)(524.52313948,64.91150391)(524.42314453,64.90150391)
\curveto(524.33313967,64.88150394)(524.24313976,64.85650397)(524.15314453,64.82650391)
\curveto(524.05313995,64.80650402)(523.95314005,64.79650403)(523.85314453,64.79650391)
\curveto(523.75314025,64.79650403)(523.65314035,64.78650404)(523.55314453,64.76650391)
\lineto(523.40314453,64.76650391)
\curveto(523.35314065,64.75650407)(523.28314072,64.75150407)(523.19314453,64.75150391)
\curveto(523.1031409,64.75150407)(523.03314097,64.75650407)(522.98314453,64.76650391)
\lineto(522.81814453,64.76650391)
\curveto(522.75814125,64.78650404)(522.69314131,64.79650403)(522.62314453,64.79650391)
\curveto(522.55314145,64.78650404)(522.49314151,64.79150403)(522.44314453,64.81150391)
\curveto(522.39314161,64.821504)(522.32814168,64.826504)(522.24814453,64.82650391)
\lineto(522.00814453,64.88650391)
\curveto(521.93814207,64.89650393)(521.86314214,64.91650391)(521.78314453,64.94650391)
\curveto(521.47314253,65.04650378)(521.2031428,65.17150365)(520.97314453,65.32150391)
\curveto(520.74314326,65.47150335)(520.54314346,65.66650316)(520.37314453,65.90650391)
\curveto(520.28314372,66.03650279)(520.2081438,66.17150265)(520.14814453,66.31150391)
\curveto(520.08814392,66.45150237)(520.03314397,66.60650222)(519.98314453,66.77650391)
\curveto(519.96314404,66.83650199)(519.95314405,66.90650192)(519.95314453,66.98650391)
\curveto(519.96314404,67.07650175)(519.97814403,67.14650168)(519.99814453,67.19650391)
\curveto(520.02814398,67.23650159)(520.07814393,67.27650155)(520.14814453,67.31650391)
\curveto(520.19814381,67.33650149)(520.26814374,67.34650148)(520.35814453,67.34650391)
\curveto(520.44814356,67.35650147)(520.53814347,67.35650147)(520.62814453,67.34650391)
\curveto(520.71814329,67.33650149)(520.8031432,67.3215015)(520.88314453,67.30150391)
\curveto(520.97314303,67.29150153)(521.03314297,67.27650155)(521.06314453,67.25650391)
\curveto(521.13314287,67.20650162)(521.17814283,67.13150169)(521.19814453,67.03150391)
\curveto(521.22814278,66.94150188)(521.26314274,66.85650197)(521.30314453,66.77650391)
\curveto(521.4031426,66.55650227)(521.53814247,66.38650244)(521.70814453,66.26650391)
\curveto(521.82814218,66.17650265)(521.96314204,66.10650272)(522.11314453,66.05650391)
\curveto(522.26314174,66.00650282)(522.42314158,65.95650287)(522.59314453,65.90650391)
\lineto(522.90814453,65.86150391)
\lineto(522.99814453,65.86150391)
\curveto(523.06814094,65.84150298)(523.15814085,65.83150299)(523.26814453,65.83150391)
\curveto(523.38814062,65.83150299)(523.48814052,65.84150298)(523.56814453,65.86150391)
\curveto(523.63814037,65.86150296)(523.69314031,65.86650296)(523.73314453,65.87650391)
\curveto(523.79314021,65.88650294)(523.85314015,65.89150293)(523.91314453,65.89150391)
\curveto(523.97314003,65.90150292)(524.02813998,65.91150291)(524.07814453,65.92150391)
\curveto(524.36813964,66.00150282)(524.59813941,66.10650272)(524.76814453,66.23650391)
\curveto(524.93813907,66.36650246)(525.05813895,66.58650224)(525.12814453,66.89650391)
\curveto(525.14813886,66.94650188)(525.15313885,67.00150182)(525.14314453,67.06150391)
\curveto(525.13313887,67.1215017)(525.12313888,67.16650166)(525.11314453,67.19650391)
\curveto(525.06313894,67.38650144)(524.99313901,67.5265013)(524.90314453,67.61650391)
\curveto(524.81313919,67.71650111)(524.69813931,67.80650102)(524.55814453,67.88650391)
\curveto(524.46813954,67.94650088)(524.36813964,67.99650083)(524.25814453,68.03650391)
\lineto(523.92814453,68.15650391)
\curveto(523.89814011,68.16650066)(523.86814014,68.17150065)(523.83814453,68.17150391)
\curveto(523.81814019,68.17150065)(523.79314021,68.18150064)(523.76314453,68.20150391)
\curveto(523.42314058,68.31150051)(523.06814094,68.39150043)(522.69814453,68.44150391)
\curveto(522.33814167,68.50150032)(521.99814201,68.59650023)(521.67814453,68.72650391)
\curveto(521.57814243,68.76650006)(521.48314252,68.80150002)(521.39314453,68.83150391)
\curveto(521.3031427,68.86149996)(521.21814279,68.90149992)(521.13814453,68.95150391)
\curveto(520.94814306,69.06149976)(520.77314323,69.18649964)(520.61314453,69.32650391)
\curveto(520.45314355,69.46649936)(520.32814368,69.64149918)(520.23814453,69.85150391)
\curveto(520.2081438,69.9214989)(520.18314382,69.99149883)(520.16314453,70.06150391)
\curveto(520.15314385,70.13149869)(520.13814387,70.20649862)(520.11814453,70.28650391)
\curveto(520.08814392,70.40649842)(520.07814393,70.54149828)(520.08814453,70.69150391)
\curveto(520.09814391,70.85149797)(520.11314389,70.98649784)(520.13314453,71.09650391)
\curveto(520.15314385,71.14649768)(520.16314384,71.18649764)(520.16314453,71.21650391)
\curveto(520.17314383,71.25649757)(520.18814382,71.29649753)(520.20814453,71.33650391)
\curveto(520.29814371,71.56649726)(520.41814359,71.76649706)(520.56814453,71.93650391)
\curveto(520.72814328,72.10649672)(520.9081431,72.25649657)(521.10814453,72.38650391)
\curveto(521.25814275,72.47649635)(521.42314258,72.54649628)(521.60314453,72.59650391)
\curveto(521.78314222,72.65649617)(521.97314203,72.71149611)(522.17314453,72.76150391)
\curveto(522.24314176,72.77149605)(522.3081417,72.78149604)(522.36814453,72.79150391)
\curveto(522.43814157,72.80149602)(522.51314149,72.81149601)(522.59314453,72.82150391)
\curveto(522.62314138,72.83149599)(522.66314134,72.83149599)(522.71314453,72.82150391)
\curveto(522.76314124,72.81149601)(522.79814121,72.81649601)(522.81814453,72.83650391)
}
}
{
\newrgbcolor{curcolor}{0 0 0}
\pscustom[linestyle=none,fillstyle=solid,fillcolor=curcolor]
{
\newpath
\moveto(644.0099292,75.92651611)
\curveto(644.9899227,75.94650516)(645.80992188,75.78650532)(646.4699292,75.44651611)
\curveto(647.13992055,75.11650599)(647.65992003,74.65650645)(648.0299292,74.06651611)
\curveto(648.12991956,73.9065072)(648.20991948,73.75150735)(648.2699292,73.60151611)
\curveto(648.33991935,73.46150764)(648.40491928,73.29150781)(648.4649292,73.09151611)
\curveto(648.4849192,73.04150806)(648.50491918,72.97150813)(648.5249292,72.88151611)
\curveto(648.54491914,72.8015083)(648.53991915,72.72650838)(648.5099292,72.65651611)
\curveto(648.4899192,72.59650851)(648.44991924,72.55650855)(648.3899292,72.53651611)
\curveto(648.33991935,72.52650858)(648.2849194,72.51150859)(648.2249292,72.49151611)
\lineto(648.0749292,72.49151611)
\curveto(648.04491964,72.48150862)(648.00491968,72.47650863)(647.9549292,72.47651611)
\lineto(647.8349292,72.47651611)
\curveto(647.69491999,72.47650863)(647.56492012,72.48150862)(647.4449292,72.49151611)
\curveto(647.33492035,72.51150859)(647.25492043,72.56150854)(647.2049292,72.64151611)
\curveto(647.13492055,72.74150836)(647.07992061,72.85650825)(647.0399292,72.98651611)
\curveto(646.99992069,73.11650799)(646.94492074,73.23650787)(646.8749292,73.34651611)
\curveto(646.74492094,73.56650754)(646.59492109,73.75650735)(646.4249292,73.91651611)
\curveto(646.26492142,74.07650703)(646.07492161,74.22650688)(645.8549292,74.36651611)
\curveto(645.73492195,74.44650666)(645.59992209,74.5065066)(645.4499292,74.54651611)
\curveto(645.30992238,74.58650652)(645.16492252,74.62650648)(645.0149292,74.66651611)
\curveto(644.90492278,74.69650641)(644.77992291,74.71650639)(644.6399292,74.72651611)
\curveto(644.49992319,74.74650636)(644.34992334,74.75650635)(644.1899292,74.75651611)
\curveto(644.03992365,74.75650635)(643.8899238,74.74650636)(643.7399292,74.72651611)
\curveto(643.59992409,74.71650639)(643.47992421,74.69650641)(643.3799292,74.66651611)
\curveto(643.27992441,74.64650646)(643.1849245,74.62650648)(643.0949292,74.60651611)
\curveto(643.00492468,74.58650652)(642.91492477,74.55650655)(642.8249292,74.51651611)
\curveto(641.9849257,74.16650694)(641.37992631,73.56650754)(641.0099292,72.71651611)
\curveto(640.93992675,72.57650853)(640.87992681,72.42650868)(640.8299292,72.26651611)
\curveto(640.7899269,72.11650899)(640.74492694,71.96150914)(640.6949292,71.80151611)
\curveto(640.67492701,71.74150936)(640.66492702,71.67650943)(640.6649292,71.60651611)
\curveto(640.66492702,71.54650956)(640.65492703,71.48650962)(640.6349292,71.42651611)
\curveto(640.62492706,71.38650972)(640.61992707,71.35150975)(640.6199292,71.32151611)
\curveto(640.61992707,71.29150981)(640.61492707,71.25650985)(640.6049292,71.21651611)
\curveto(640.5849271,71.10651)(640.56992712,70.99151011)(640.5599292,70.87151611)
\lineto(640.5599292,70.52651611)
\curveto(640.55992713,70.45651065)(640.55492713,70.38151072)(640.5449292,70.30151611)
\curveto(640.54492714,70.23151087)(640.54992714,70.16651094)(640.5599292,70.10651611)
\lineto(640.5599292,69.95651611)
\curveto(640.57992711,69.88651122)(640.5849271,69.81651129)(640.5749292,69.74651611)
\curveto(640.57492711,69.67651143)(640.5849271,69.6065115)(640.6049292,69.53651611)
\curveto(640.62492706,69.47651163)(640.62992706,69.41651169)(640.6199292,69.35651611)
\curveto(640.61992707,69.29651181)(640.62992706,69.24151186)(640.6499292,69.19151611)
\curveto(640.67992701,69.06151204)(640.70492698,68.93151217)(640.7249292,68.80151611)
\curveto(640.75492693,68.68151242)(640.7899269,68.56151254)(640.8299292,68.44151611)
\curveto(640.99992669,67.94151316)(641.21992647,67.51151359)(641.4899292,67.15151611)
\curveto(641.75992593,66.8015143)(642.11492557,66.51151459)(642.5549292,66.28151611)
\curveto(642.69492499,66.21151489)(642.83492485,66.15651495)(642.9749292,66.11651611)
\curveto(643.12492456,66.07651503)(643.2849244,66.03151507)(643.4549292,65.98151611)
\curveto(643.52492416,65.96151514)(643.5899241,65.95151515)(643.6499292,65.95151611)
\curveto(643.70992398,65.96151514)(643.77992391,65.95651515)(643.8599292,65.93651611)
\curveto(643.90992378,65.92651518)(643.99992369,65.91651519)(644.1299292,65.90651611)
\curveto(644.25992343,65.9065152)(644.35492333,65.91651519)(644.4149292,65.93651611)
\lineto(644.5199292,65.93651611)
\curveto(644.55992313,65.94651516)(644.59992309,65.94651516)(644.6399292,65.93651611)
\curveto(644.67992301,65.93651517)(644.71992297,65.94651516)(644.7599292,65.96651611)
\curveto(644.85992283,65.98651512)(644.95492273,66.0015151)(645.0449292,66.01151611)
\curveto(645.14492254,66.03151507)(645.23992245,66.06151504)(645.3299292,66.10151611)
\curveto(646.10992158,66.42151468)(646.65992103,66.94651416)(646.9799292,67.67651611)
\curveto(647.05992063,67.85651325)(647.13492055,68.07151303)(647.2049292,68.32151611)
\curveto(647.22492046,68.41151269)(647.23992045,68.5015126)(647.2499292,68.59151611)
\curveto(647.26992042,68.69151241)(647.30492038,68.78151232)(647.3549292,68.86151611)
\curveto(647.40492028,68.94151216)(647.4849202,68.98651212)(647.5949292,68.99651611)
\curveto(647.70491998,69.0065121)(647.82491986,69.01151209)(647.9549292,69.01151611)
\lineto(648.1049292,69.01151611)
\curveto(648.15491953,69.01151209)(648.19991949,69.0065121)(648.2399292,68.99651611)
\lineto(648.3449292,68.99651611)
\lineto(648.4349292,68.96651611)
\curveto(648.47491921,68.96651214)(648.50491918,68.95651215)(648.5249292,68.93651611)
\curveto(648.59491909,68.89651221)(648.63491905,68.82151228)(648.6449292,68.71151611)
\curveto(648.65491903,68.61151249)(648.64491904,68.51151259)(648.6149292,68.41151611)
\curveto(648.55491913,68.18151292)(648.49991919,67.96151314)(648.4499292,67.75151611)
\curveto(648.39991929,67.54151356)(648.32491936,67.34151376)(648.2249292,67.15151611)
\curveto(648.14491954,67.02151408)(648.06991962,66.89651421)(647.9999292,66.77651611)
\curveto(647.93991975,66.65651445)(647.86991982,66.53651457)(647.7899292,66.41651611)
\curveto(647.60992008,66.15651495)(647.3849203,65.91651519)(647.1149292,65.69651611)
\curveto(646.85492083,65.48651562)(646.56992112,65.31151579)(646.2599292,65.17151611)
\curveto(646.14992154,65.12151598)(646.03992165,65.08151602)(645.9299292,65.05151611)
\curveto(645.82992186,65.02151608)(645.72492196,64.98651612)(645.6149292,64.94651611)
\curveto(645.50492218,64.9065162)(645.3899223,64.88151622)(645.2699292,64.87151611)
\curveto(645.15992253,64.85151625)(645.04492264,64.83151627)(644.9249292,64.81151611)
\curveto(644.87492281,64.79151631)(644.82992286,64.78651632)(644.7899292,64.79651611)
\curveto(644.74992294,64.79651631)(644.70992298,64.79151631)(644.6699292,64.78151611)
\curveto(644.60992308,64.77151633)(644.54992314,64.76651634)(644.4899292,64.76651611)
\curveto(644.42992326,64.76651634)(644.36492332,64.76151634)(644.2949292,64.75151611)
\curveto(644.26492342,64.74151636)(644.19492349,64.74151636)(644.0849292,64.75151611)
\curveto(643.9849237,64.75151635)(643.91992377,64.75651635)(643.8899292,64.76651611)
\curveto(643.83992385,64.77651633)(643.7899239,64.78151632)(643.7399292,64.78151611)
\curveto(643.69992399,64.77151633)(643.65492403,64.77151633)(643.6049292,64.78151611)
\lineto(643.4549292,64.78151611)
\curveto(643.37492431,64.8015163)(643.29992439,64.81651629)(643.2299292,64.82651611)
\curveto(643.15992453,64.82651628)(643.0849246,64.83651627)(643.0049292,64.85651611)
\lineto(642.7349292,64.91651611)
\curveto(642.64492504,64.92651618)(642.55992513,64.94651616)(642.4799292,64.97651611)
\curveto(642.26992542,65.03651607)(642.07992561,65.11151599)(641.9099292,65.20151611)
\curveto(641.27992641,65.47151563)(640.76992692,65.85651525)(640.3799292,66.35651611)
\curveto(639.9899277,66.85651425)(639.67992801,67.44651366)(639.4499292,68.12651611)
\curveto(639.40992828,68.24651286)(639.37492831,68.37151273)(639.3449292,68.50151611)
\curveto(639.32492836,68.63151247)(639.29992839,68.76651234)(639.2699292,68.90651611)
\curveto(639.24992844,68.95651215)(639.23992845,69.0015121)(639.2399292,69.04151611)
\curveto(639.24992844,69.08151202)(639.24992844,69.12651198)(639.2399292,69.17651611)
\curveto(639.21992847,69.26651184)(639.20492848,69.36151174)(639.1949292,69.46151611)
\curveto(639.19492849,69.56151154)(639.1849285,69.65651145)(639.1649292,69.74651611)
\lineto(639.1649292,70.03151611)
\curveto(639.14492854,70.08151102)(639.13492855,70.16651094)(639.1349292,70.28651611)
\curveto(639.13492855,70.4065107)(639.14492854,70.49151061)(639.1649292,70.54151611)
\curveto(639.17492851,70.57151053)(639.17492851,70.6015105)(639.1649292,70.63151611)
\curveto(639.15492853,70.67151043)(639.15492853,70.7015104)(639.1649292,70.72151611)
\lineto(639.1649292,70.85651611)
\curveto(639.17492851,70.93651017)(639.17992851,71.01651009)(639.1799292,71.09651611)
\curveto(639.1899285,71.18650992)(639.20492848,71.27150983)(639.2249292,71.35151611)
\curveto(639.24492844,71.41150969)(639.25492843,71.47150963)(639.2549292,71.53151611)
\curveto(639.25492843,71.6015095)(639.26492842,71.67150943)(639.2849292,71.74151611)
\curveto(639.33492835,71.91150919)(639.37492831,72.07650903)(639.4049292,72.23651611)
\curveto(639.43492825,72.39650871)(639.47992821,72.54650856)(639.5399292,72.68651611)
\lineto(639.6899292,73.07651611)
\curveto(639.74992794,73.21650789)(639.81492787,73.34150776)(639.8849292,73.45151611)
\curveto(640.03492765,73.71150739)(640.1849275,73.94650716)(640.3349292,74.15651611)
\curveto(640.36492732,74.2065069)(640.39992729,74.24650686)(640.4399292,74.27651611)
\curveto(640.4899272,74.31650679)(640.52992716,74.36150674)(640.5599292,74.41151611)
\curveto(640.61992707,74.49150661)(640.67992701,74.56150654)(640.7399292,74.62151611)
\lineto(640.9499292,74.80151611)
\curveto(641.00992668,74.85150625)(641.06492662,74.89650621)(641.1149292,74.93651611)
\curveto(641.17492651,74.98650612)(641.23992645,75.03650607)(641.3099292,75.08651611)
\curveto(641.45992623,75.19650591)(641.61492607,75.29150581)(641.7749292,75.37151611)
\curveto(641.94492574,75.45150565)(642.11992557,75.53150557)(642.2999292,75.61151611)
\curveto(642.40992528,75.66150544)(642.52492516,75.69650541)(642.6449292,75.71651611)
\curveto(642.77492491,75.74650536)(642.89992479,75.78150532)(643.0199292,75.82151611)
\curveto(643.0899246,75.83150527)(643.15492453,75.84150526)(643.2149292,75.85151611)
\lineto(643.3949292,75.88151611)
\curveto(643.47492421,75.89150521)(643.54992414,75.89650521)(643.6199292,75.89651611)
\curveto(643.69992399,75.9065052)(643.77992391,75.91650519)(643.8599292,75.92651611)
\curveto(643.87992381,75.93650517)(643.90492378,75.93650517)(643.9349292,75.92651611)
\curveto(643.96492372,75.91650519)(643.9899237,75.91650519)(644.0099292,75.92651611)
}
}
{
\newrgbcolor{curcolor}{0 0 0}
\pscustom[linestyle=none,fillstyle=solid,fillcolor=curcolor]
{
\newpath
\moveto(657.36977295,69.20651611)
\curveto(657.38976489,69.14651196)(657.39976488,69.05151205)(657.39977295,68.92151611)
\curveto(657.39976488,68.8015123)(657.39476488,68.71651239)(657.38477295,68.66651611)
\lineto(657.38477295,68.51651611)
\curveto(657.3747649,68.43651267)(657.36476491,68.36151274)(657.35477295,68.29151611)
\curveto(657.35476492,68.23151287)(657.34976493,68.16151294)(657.33977295,68.08151611)
\curveto(657.31976496,68.02151308)(657.30476497,67.96151314)(657.29477295,67.90151611)
\curveto(657.29476498,67.84151326)(657.28476499,67.78151332)(657.26477295,67.72151611)
\curveto(657.22476505,67.59151351)(657.18976509,67.46151364)(657.15977295,67.33151611)
\curveto(657.12976515,67.2015139)(657.08976519,67.08151402)(657.03977295,66.97151611)
\curveto(656.82976545,66.49151461)(656.54976573,66.08651502)(656.19977295,65.75651611)
\curveto(655.84976643,65.43651567)(655.41976686,65.19151591)(654.90977295,65.02151611)
\curveto(654.79976748,64.98151612)(654.6797676,64.95151615)(654.54977295,64.93151611)
\curveto(654.42976785,64.91151619)(654.30476797,64.89151621)(654.17477295,64.87151611)
\curveto(654.11476816,64.86151624)(654.04976823,64.85651625)(653.97977295,64.85651611)
\curveto(653.91976836,64.84651626)(653.85976842,64.84151626)(653.79977295,64.84151611)
\curveto(653.75976852,64.83151627)(653.69976858,64.82651628)(653.61977295,64.82651611)
\curveto(653.54976873,64.82651628)(653.49976878,64.83151627)(653.46977295,64.84151611)
\curveto(653.42976885,64.85151625)(653.38976889,64.85651625)(653.34977295,64.85651611)
\curveto(653.30976897,64.84651626)(653.274769,64.84651626)(653.24477295,64.85651611)
\lineto(653.15477295,64.85651611)
\lineto(652.79477295,64.90151611)
\curveto(652.65476962,64.94151616)(652.51976976,64.98151612)(652.38977295,65.02151611)
\curveto(652.25977002,65.06151604)(652.13477014,65.106516)(652.01477295,65.15651611)
\curveto(651.56477071,65.35651575)(651.19477108,65.61651549)(650.90477295,65.93651611)
\curveto(650.61477166,66.25651485)(650.3747719,66.64651446)(650.18477295,67.10651611)
\curveto(650.13477214,67.2065139)(650.09477218,67.3065138)(650.06477295,67.40651611)
\curveto(650.04477223,67.5065136)(650.02477225,67.61151349)(650.00477295,67.72151611)
\curveto(649.98477229,67.76151334)(649.9747723,67.79151331)(649.97477295,67.81151611)
\curveto(649.98477229,67.84151326)(649.98477229,67.87651323)(649.97477295,67.91651611)
\curveto(649.95477232,67.99651311)(649.93977234,68.07651303)(649.92977295,68.15651611)
\curveto(649.92977235,68.24651286)(649.91977236,68.33151277)(649.89977295,68.41151611)
\lineto(649.89977295,68.53151611)
\curveto(649.89977238,68.57151253)(649.89477238,68.61651249)(649.88477295,68.66651611)
\curveto(649.8747724,68.71651239)(649.86977241,68.8015123)(649.86977295,68.92151611)
\curveto(649.86977241,69.05151205)(649.8797724,69.14651196)(649.89977295,69.20651611)
\curveto(649.91977236,69.27651183)(649.92477235,69.34651176)(649.91477295,69.41651611)
\curveto(649.90477237,69.48651162)(649.90977237,69.55651155)(649.92977295,69.62651611)
\curveto(649.93977234,69.67651143)(649.94477233,69.71651139)(649.94477295,69.74651611)
\curveto(649.95477232,69.78651132)(649.96477231,69.83151127)(649.97477295,69.88151611)
\curveto(650.00477227,70.0015111)(650.02977225,70.12151098)(650.04977295,70.24151611)
\curveto(650.0797722,70.36151074)(650.11977216,70.47651063)(650.16977295,70.58651611)
\curveto(650.31977196,70.95651015)(650.49977178,71.28650982)(650.70977295,71.57651611)
\curveto(650.92977135,71.87650923)(651.19477108,72.12650898)(651.50477295,72.32651611)
\curveto(651.62477065,72.4065087)(651.74977053,72.47150863)(651.87977295,72.52151611)
\curveto(652.00977027,72.58150852)(652.14477013,72.64150846)(652.28477295,72.70151611)
\curveto(652.40476987,72.75150835)(652.53476974,72.78150832)(652.67477295,72.79151611)
\curveto(652.81476946,72.81150829)(652.95476932,72.84150826)(653.09477295,72.88151611)
\lineto(653.28977295,72.88151611)
\curveto(653.35976892,72.89150821)(653.42476885,72.9015082)(653.48477295,72.91151611)
\curveto(654.3747679,72.92150818)(655.11476716,72.73650837)(655.70477295,72.35651611)
\curveto(656.29476598,71.97650913)(656.71976556,71.48150962)(656.97977295,70.87151611)
\curveto(657.02976525,70.77151033)(657.06976521,70.67151043)(657.09977295,70.57151611)
\curveto(657.12976515,70.47151063)(657.16476511,70.36651074)(657.20477295,70.25651611)
\curveto(657.23476504,70.14651096)(657.25976502,70.02651108)(657.27977295,69.89651611)
\curveto(657.29976498,69.77651133)(657.32476495,69.65151145)(657.35477295,69.52151611)
\curveto(657.36476491,69.47151163)(657.36476491,69.41651169)(657.35477295,69.35651611)
\curveto(657.35476492,69.3065118)(657.35976492,69.25651185)(657.36977295,69.20651611)
\moveto(656.03477295,68.35151611)
\curveto(656.05476622,68.42151268)(656.05976622,68.5015126)(656.04977295,68.59151611)
\lineto(656.04977295,68.84651611)
\curveto(656.04976623,69.23651187)(656.01476626,69.56651154)(655.94477295,69.83651611)
\curveto(655.91476636,69.91651119)(655.88976639,69.99651111)(655.86977295,70.07651611)
\curveto(655.84976643,70.15651095)(655.82476645,70.23151087)(655.79477295,70.30151611)
\curveto(655.51476676,70.95151015)(655.06976721,71.4015097)(654.45977295,71.65151611)
\curveto(654.38976789,71.68150942)(654.31476796,71.7015094)(654.23477295,71.71151611)
\lineto(653.99477295,71.77151611)
\curveto(653.91476836,71.79150931)(653.82976845,71.8015093)(653.73977295,71.80151611)
\lineto(653.46977295,71.80151611)
\lineto(653.19977295,71.75651611)
\curveto(653.09976918,71.73650937)(653.00476927,71.71150939)(652.91477295,71.68151611)
\curveto(652.83476944,71.66150944)(652.75476952,71.63150947)(652.67477295,71.59151611)
\curveto(652.60476967,71.57150953)(652.53976974,71.54150956)(652.47977295,71.50151611)
\curveto(652.41976986,71.46150964)(652.36476991,71.42150968)(652.31477295,71.38151611)
\curveto(652.0747702,71.21150989)(651.8797704,71.0065101)(651.72977295,70.76651611)
\curveto(651.5797707,70.52651058)(651.44977083,70.24651086)(651.33977295,69.92651611)
\curveto(651.30977097,69.82651128)(651.28977099,69.72151138)(651.27977295,69.61151611)
\curveto(651.26977101,69.51151159)(651.25477102,69.4065117)(651.23477295,69.29651611)
\curveto(651.22477105,69.25651185)(651.21977106,69.19151191)(651.21977295,69.10151611)
\curveto(651.20977107,69.07151203)(651.20477107,69.03651207)(651.20477295,68.99651611)
\curveto(651.21477106,68.95651215)(651.21977106,68.91151219)(651.21977295,68.86151611)
\lineto(651.21977295,68.56151611)
\curveto(651.21977106,68.46151264)(651.22977105,68.37151273)(651.24977295,68.29151611)
\lineto(651.27977295,68.11151611)
\curveto(651.29977098,68.01151309)(651.31477096,67.91151319)(651.32477295,67.81151611)
\curveto(651.34477093,67.72151338)(651.3747709,67.63651347)(651.41477295,67.55651611)
\curveto(651.51477076,67.31651379)(651.62977065,67.09151401)(651.75977295,66.88151611)
\curveto(651.89977038,66.67151443)(652.06977021,66.49651461)(652.26977295,66.35651611)
\curveto(652.31976996,66.32651478)(652.36476991,66.3015148)(652.40477295,66.28151611)
\curveto(652.44476983,66.26151484)(652.48976979,66.23651487)(652.53977295,66.20651611)
\curveto(652.61976966,66.15651495)(652.70476957,66.11151499)(652.79477295,66.07151611)
\curveto(652.89476938,66.04151506)(652.99976928,66.01151509)(653.10977295,65.98151611)
\curveto(653.15976912,65.96151514)(653.20476907,65.95151515)(653.24477295,65.95151611)
\curveto(653.29476898,65.96151514)(653.34476893,65.96151514)(653.39477295,65.95151611)
\curveto(653.42476885,65.94151516)(653.48476879,65.93151517)(653.57477295,65.92151611)
\curveto(653.6747686,65.91151519)(653.74976853,65.91651519)(653.79977295,65.93651611)
\curveto(653.83976844,65.94651516)(653.8797684,65.94651516)(653.91977295,65.93651611)
\curveto(653.95976832,65.93651517)(653.99976828,65.94651516)(654.03977295,65.96651611)
\curveto(654.11976816,65.98651512)(654.19976808,66.0015151)(654.27977295,66.01151611)
\curveto(654.35976792,66.03151507)(654.43476784,66.05651505)(654.50477295,66.08651611)
\curveto(654.84476743,66.22651488)(655.11976716,66.42151468)(655.32977295,66.67151611)
\curveto(655.53976674,66.92151418)(655.71476656,67.21651389)(655.85477295,67.55651611)
\curveto(655.90476637,67.67651343)(655.93476634,67.8015133)(655.94477295,67.93151611)
\curveto(655.96476631,68.07151303)(655.99476628,68.21151289)(656.03477295,68.35151611)
}
}
{
\newrgbcolor{curcolor}{0 0 0}
\pscustom[linestyle=none,fillstyle=solid,fillcolor=curcolor]
{
\newpath
\moveto(662.5480542,72.91151611)
\curveto(662.92804921,72.92150818)(663.24804889,72.88150822)(663.5080542,72.79151611)
\curveto(663.77804836,72.7015084)(664.02304812,72.57150853)(664.2430542,72.40151611)
\curveto(664.32304782,72.35150875)(664.38804775,72.28150882)(664.4380542,72.19151611)
\curveto(664.49804764,72.11150899)(664.56304758,72.03650907)(664.6330542,71.96651611)
\curveto(664.65304749,71.94650916)(664.68304746,71.92150918)(664.7230542,71.89151611)
\curveto(664.76304738,71.86150924)(664.81304733,71.85150925)(664.8730542,71.86151611)
\curveto(664.97304717,71.89150921)(665.05804708,71.95150915)(665.1280542,72.04151611)
\curveto(665.20804693,72.14150896)(665.28804685,72.21650889)(665.3680542,72.26651611)
\curveto(665.50804663,72.37650873)(665.65304649,72.47150863)(665.8030542,72.55151611)
\curveto(665.95304619,72.64150846)(666.11804602,72.71650839)(666.2980542,72.77651611)
\curveto(666.37804576,72.8065083)(666.46304568,72.82650828)(666.5530542,72.83651611)
\curveto(666.65304549,72.85650825)(666.74804539,72.87650823)(666.8380542,72.89651611)
\curveto(666.88804525,72.9065082)(666.93304521,72.91150819)(666.9730542,72.91151611)
\lineto(667.1230542,72.91151611)
\curveto(667.17304497,72.93150817)(667.2430449,72.93650817)(667.3330542,72.92651611)
\curveto(667.42304472,72.92650818)(667.48804465,72.92150818)(667.5280542,72.91151611)
\curveto(667.57804456,72.9015082)(667.65304449,72.89650821)(667.7530542,72.89651611)
\curveto(667.8430443,72.87650823)(667.92804421,72.85650825)(668.0080542,72.83651611)
\curveto(668.09804404,72.82650828)(668.18304396,72.8065083)(668.2630542,72.77651611)
\curveto(668.31304383,72.75650835)(668.35804378,72.74150836)(668.3980542,72.73151611)
\curveto(668.44804369,72.73150837)(668.49804364,72.72150838)(668.5480542,72.70151611)
\curveto(669.04804309,72.48150862)(669.39304275,72.14150896)(669.5830542,71.68151611)
\curveto(669.62304252,71.6015095)(669.65304249,71.51150959)(669.6730542,71.41151611)
\curveto(669.69304245,71.32150978)(669.71304243,71.22150988)(669.7330542,71.11151611)
\curveto(669.75304239,71.08151002)(669.75804238,71.04651006)(669.7480542,71.00651611)
\curveto(669.74804239,70.97651013)(669.75304239,70.94651016)(669.7630542,70.91651611)
\lineto(669.7630542,70.78151611)
\curveto(669.77304237,70.74151036)(669.77304237,70.69651041)(669.7630542,70.64651611)
\curveto(669.76304238,70.59651051)(669.76304238,70.54651056)(669.7630542,70.49651611)
\lineto(669.7630542,69.91151611)
\lineto(669.7630542,68.95151611)
\lineto(669.7630542,66.10151611)
\curveto(669.76304238,65.94151516)(669.76304238,65.75151535)(669.7630542,65.53151611)
\curveto(669.77304237,65.31151579)(669.73304241,65.16651594)(669.6430542,65.09651611)
\curveto(669.60304254,65.06651604)(669.5380426,65.04151606)(669.4480542,65.02151611)
\curveto(669.35804278,65.01151609)(669.26304288,65.0065161)(669.1630542,65.00651611)
\curveto(669.06304308,65.0065161)(668.96304318,65.01151609)(668.8630542,65.02151611)
\curveto(668.77304337,65.03151607)(668.70804343,65.05151605)(668.6680542,65.08151611)
\curveto(668.60804353,65.11151599)(668.56804357,65.17151593)(668.5480542,65.26151611)
\curveto(668.52804361,65.32151578)(668.52304362,65.38151572)(668.5330542,65.44151611)
\curveto(668.5430436,65.51151559)(668.5380436,65.57651553)(668.5180542,65.63651611)
\curveto(668.50804363,65.68651542)(668.50304364,65.74151536)(668.5030542,65.80151611)
\curveto(668.51304363,65.87151523)(668.51804362,65.93651517)(668.5180542,65.99651611)
\lineto(668.5180542,66.67151611)
\lineto(668.5180542,69.53651611)
\curveto(668.51804362,69.86651124)(668.50804363,70.17651093)(668.4880542,70.46651611)
\curveto(668.47804366,70.76651034)(668.40804373,71.01651009)(668.2780542,71.21651611)
\curveto(668.12804401,71.45650965)(667.89804424,71.63150947)(667.5880542,71.74151611)
\curveto(667.52804461,71.76150934)(667.46304468,71.77150933)(667.3930542,71.77151611)
\curveto(667.33304481,71.78150932)(667.26804487,71.79650931)(667.1980542,71.81651611)
\curveto(667.15804498,71.82650928)(667.09304505,71.82650928)(667.0030542,71.81651611)
\curveto(666.91304523,71.81650929)(666.85304529,71.81150929)(666.8230542,71.80151611)
\curveto(666.77304537,71.79150931)(666.72304542,71.78650932)(666.6730542,71.78651611)
\curveto(666.62304552,71.79650931)(666.57304557,71.79150931)(666.5230542,71.77151611)
\curveto(666.38304576,71.74150936)(666.24804589,71.7015094)(666.1180542,71.65151611)
\curveto(665.59804654,71.43150967)(665.24804689,71.04651006)(665.0680542,70.49651611)
\curveto(665.01804712,70.32651078)(664.98804715,70.13151097)(664.9780542,69.91151611)
\lineto(664.9780542,69.23651611)
\lineto(664.9780542,67.27151611)
\lineto(664.9780542,65.81651611)
\lineto(664.9780542,65.44151611)
\curveto(664.97804716,65.32151578)(664.95304719,65.22651588)(664.9030542,65.15651611)
\curveto(664.85304729,65.07651603)(664.76804737,65.03151607)(664.6480542,65.02151611)
\curveto(664.52804761,65.01151609)(664.40304774,65.0065161)(664.2730542,65.00651611)
\curveto(664.10304804,65.0065161)(663.97804816,65.02651608)(663.8980542,65.06651611)
\curveto(663.80804833,65.11651599)(663.75304839,65.19651591)(663.7330542,65.30651611)
\curveto(663.72304842,65.42651568)(663.71804842,65.55651555)(663.7180542,65.69651611)
\lineto(663.7180542,67.12151611)
\lineto(663.7180542,69.59651611)
\curveto(663.71804842,69.91651119)(663.70804843,70.21151089)(663.6880542,70.48151611)
\curveto(663.66804847,70.76151034)(663.59804854,71.0015101)(663.4780542,71.20151611)
\curveto(663.36804877,71.38150972)(663.2430489,71.51150959)(663.1030542,71.59151611)
\curveto(662.96304918,71.68150942)(662.77304937,71.75150935)(662.5330542,71.80151611)
\curveto(662.49304965,71.81150929)(662.44804969,71.81650929)(662.3980542,71.81651611)
\lineto(662.2630542,71.81651611)
\curveto(662.0430501,71.81650929)(661.84805029,71.79150931)(661.6780542,71.74151611)
\curveto(661.51805062,71.69150941)(661.37305077,71.62650948)(661.2430542,71.54651611)
\curveto(660.73305141,71.23650987)(660.39305175,70.77151033)(660.2230542,70.15151611)
\curveto(660.18305196,70.02151108)(660.16305198,69.87151123)(660.1630542,69.70151611)
\curveto(660.17305197,69.54151156)(660.17805196,69.38151172)(660.1780542,69.22151611)
\lineto(660.1780542,67.52651611)
\lineto(660.1780542,65.87651611)
\lineto(660.1780542,65.47151611)
\curveto(660.17805196,65.33151577)(660.14805199,65.22151588)(660.0880542,65.14151611)
\curveto(660.0380521,65.07151603)(659.96305218,65.03151607)(659.8630542,65.02151611)
\curveto(659.76305238,65.01151609)(659.65805248,65.0065161)(659.5480542,65.00651611)
\lineto(659.3230542,65.00651611)
\curveto(659.26305288,65.02651608)(659.20305294,65.04151606)(659.1430542,65.05151611)
\curveto(659.09305305,65.06151604)(659.04805309,65.09151601)(659.0080542,65.14151611)
\curveto(658.95805318,65.2015159)(658.93305321,65.27651583)(658.9330542,65.36651611)
\lineto(658.9330542,65.68151611)
\lineto(658.9330542,66.65651611)
\lineto(658.9330542,70.94651611)
\lineto(658.9330542,72.05651611)
\lineto(658.9330542,72.34151611)
\curveto(658.93305321,72.44150866)(658.95305319,72.52150858)(658.9930542,72.58151611)
\curveto(659.02305312,72.64150846)(659.06805307,72.68150842)(659.1280542,72.70151611)
\curveto(659.20805293,72.73150837)(659.33305281,72.74650836)(659.5030542,72.74651611)
\curveto(659.68305246,72.74650836)(659.81305233,72.73150837)(659.8930542,72.70151611)
\curveto(659.97305217,72.66150844)(660.02805211,72.61150849)(660.0580542,72.55151611)
\curveto(660.07805206,72.5015086)(660.08805205,72.44150866)(660.0880542,72.37151611)
\curveto(660.09805204,72.3015088)(660.10805203,72.23650887)(660.1180542,72.17651611)
\curveto(660.12805201,72.11650899)(660.14805199,72.06650904)(660.1780542,72.02651611)
\curveto(660.20805193,71.98650912)(660.25805188,71.96650914)(660.3280542,71.96651611)
\curveto(660.34805179,71.98650912)(660.36805177,71.99650911)(660.3880542,71.99651611)
\curveto(660.41805172,71.99650911)(660.4430517,72.0065091)(660.4630542,72.02651611)
\curveto(660.52305162,72.07650903)(660.57805156,72.12650898)(660.6280542,72.17651611)
\lineto(660.8080542,72.32651611)
\curveto(661.02805111,72.48650862)(661.27805086,72.62650848)(661.5580542,72.74651611)
\curveto(661.65805048,72.78650832)(661.75805038,72.81150829)(661.8580542,72.82151611)
\curveto(661.95805018,72.84150826)(662.06305008,72.86650824)(662.1730542,72.89651611)
\lineto(662.3530542,72.89651611)
\curveto(662.42304972,72.9065082)(662.48804965,72.91150819)(662.5480542,72.91151611)
}
}
{
\newrgbcolor{curcolor}{0 0 0}
\pscustom[linestyle=none,fillstyle=solid,fillcolor=curcolor]
{
\newpath
\moveto(672.12578857,72.73151611)
\lineto(672.56078857,72.73151611)
\curveto(672.71078661,72.73150837)(672.8157865,72.69150841)(672.87578857,72.61151611)
\curveto(672.92578639,72.53150857)(672.95078637,72.43150867)(672.95078857,72.31151611)
\curveto(672.96078636,72.19150891)(672.96578635,72.07150903)(672.96578857,71.95151611)
\lineto(672.96578857,70.52651611)
\lineto(672.96578857,68.26151611)
\lineto(672.96578857,67.57151611)
\curveto(672.96578635,67.34151376)(672.99078633,67.14151396)(673.04078857,66.97151611)
\curveto(673.20078612,66.52151458)(673.50078582,66.2065149)(673.94078857,66.02651611)
\curveto(674.16078516,65.93651517)(674.42578489,65.9015152)(674.73578857,65.92151611)
\curveto(675.04578427,65.95151515)(675.29578402,66.0065151)(675.48578857,66.08651611)
\curveto(675.8157835,66.22651488)(676.07578324,66.4015147)(676.26578857,66.61151611)
\curveto(676.46578285,66.83151427)(676.6207827,67.11651399)(676.73078857,67.46651611)
\curveto(676.76078256,67.54651356)(676.78078254,67.62651348)(676.79078857,67.70651611)
\curveto(676.80078252,67.78651332)(676.8157825,67.87151323)(676.83578857,67.96151611)
\curveto(676.84578247,68.01151309)(676.84578247,68.05651305)(676.83578857,68.09651611)
\curveto(676.83578248,68.13651297)(676.84578247,68.18151292)(676.86578857,68.23151611)
\lineto(676.86578857,68.54651611)
\curveto(676.88578243,68.62651248)(676.89078243,68.71651239)(676.88078857,68.81651611)
\curveto(676.87078245,68.92651218)(676.86578245,69.02651208)(676.86578857,69.11651611)
\lineto(676.86578857,70.28651611)
\lineto(676.86578857,71.87651611)
\curveto(676.86578245,71.99650911)(676.86078246,72.12150898)(676.85078857,72.25151611)
\curveto(676.85078247,72.39150871)(676.87578244,72.5015086)(676.92578857,72.58151611)
\curveto(676.96578235,72.63150847)(677.01078231,72.66150844)(677.06078857,72.67151611)
\curveto(677.1207822,72.69150841)(677.19078213,72.71150839)(677.27078857,72.73151611)
\lineto(677.49578857,72.73151611)
\curveto(677.6157817,72.73150837)(677.7207816,72.72650838)(677.81078857,72.71651611)
\curveto(677.91078141,72.7065084)(677.98578133,72.66150844)(678.03578857,72.58151611)
\curveto(678.08578123,72.53150857)(678.11078121,72.45650865)(678.11078857,72.35651611)
\lineto(678.11078857,72.07151611)
\lineto(678.11078857,71.05151611)
\lineto(678.11078857,67.01651611)
\lineto(678.11078857,65.66651611)
\curveto(678.11078121,65.54651556)(678.10578121,65.43151567)(678.09578857,65.32151611)
\curveto(678.09578122,65.22151588)(678.06078126,65.14651596)(677.99078857,65.09651611)
\curveto(677.95078137,65.06651604)(677.89078143,65.04151606)(677.81078857,65.02151611)
\curveto(677.73078159,65.01151609)(677.64078168,65.0015161)(677.54078857,64.99151611)
\curveto(677.45078187,64.99151611)(677.36078196,64.99651611)(677.27078857,65.00651611)
\curveto(677.19078213,65.01651609)(677.13078219,65.03651607)(677.09078857,65.06651611)
\curveto(677.04078228,65.106516)(676.99578232,65.17151593)(676.95578857,65.26151611)
\curveto(676.94578237,65.3015158)(676.93578238,65.35651575)(676.92578857,65.42651611)
\curveto(676.92578239,65.49651561)(676.9207824,65.56151554)(676.91078857,65.62151611)
\curveto(676.90078242,65.69151541)(676.88078244,65.74651536)(676.85078857,65.78651611)
\curveto(676.8207825,65.82651528)(676.77578254,65.84151526)(676.71578857,65.83151611)
\curveto(676.63578268,65.81151529)(676.55578276,65.75151535)(676.47578857,65.65151611)
\curveto(676.39578292,65.56151554)(676.320783,65.49151561)(676.25078857,65.44151611)
\curveto(676.03078329,65.28151582)(675.78078354,65.14151596)(675.50078857,65.02151611)
\curveto(675.39078393,64.97151613)(675.27578404,64.94151616)(675.15578857,64.93151611)
\curveto(675.04578427,64.91151619)(674.93078439,64.88651622)(674.81078857,64.85651611)
\curveto(674.76078456,64.84651626)(674.70578461,64.84651626)(674.64578857,64.85651611)
\curveto(674.59578472,64.86651624)(674.54578477,64.86151624)(674.49578857,64.84151611)
\curveto(674.39578492,64.82151628)(674.30578501,64.82151628)(674.22578857,64.84151611)
\lineto(674.07578857,64.84151611)
\curveto(674.02578529,64.86151624)(673.96578535,64.87151623)(673.89578857,64.87151611)
\curveto(673.83578548,64.87151623)(673.78078554,64.87651623)(673.73078857,64.88651611)
\curveto(673.69078563,64.9065162)(673.65078567,64.91651619)(673.61078857,64.91651611)
\curveto(673.58078574,64.9065162)(673.54078578,64.91151619)(673.49078857,64.93151611)
\lineto(673.25078857,64.99151611)
\curveto(673.18078614,65.01151609)(673.10578621,65.04151606)(673.02578857,65.08151611)
\curveto(672.76578655,65.19151591)(672.54578677,65.33651577)(672.36578857,65.51651611)
\curveto(672.19578712,65.7065154)(672.05578726,65.93151517)(671.94578857,66.19151611)
\curveto(671.90578741,66.28151482)(671.87578744,66.37151473)(671.85578857,66.46151611)
\lineto(671.79578857,66.76151611)
\curveto(671.77578754,66.82151428)(671.76578755,66.87651423)(671.76578857,66.92651611)
\curveto(671.77578754,66.98651412)(671.77078755,67.05151405)(671.75078857,67.12151611)
\curveto(671.74078758,67.14151396)(671.73578758,67.16651394)(671.73578857,67.19651611)
\curveto(671.73578758,67.23651387)(671.73078759,67.27151383)(671.72078857,67.30151611)
\lineto(671.72078857,67.45151611)
\curveto(671.71078761,67.49151361)(671.70578761,67.53651357)(671.70578857,67.58651611)
\curveto(671.7157876,67.64651346)(671.7207876,67.7015134)(671.72078857,67.75151611)
\lineto(671.72078857,68.35151611)
\lineto(671.72078857,71.11151611)
\lineto(671.72078857,72.07151611)
\lineto(671.72078857,72.34151611)
\curveto(671.7207876,72.43150867)(671.74078758,72.5065086)(671.78078857,72.56651611)
\curveto(671.8207875,72.63650847)(671.89578742,72.68650842)(672.00578857,72.71651611)
\curveto(672.02578729,72.72650838)(672.04578727,72.72650838)(672.06578857,72.71651611)
\curveto(672.08578723,72.71650839)(672.10578721,72.72150838)(672.12578857,72.73151611)
}
}
{
\newrgbcolor{curcolor}{0 0 0}
\pscustom[linestyle=none,fillstyle=solid,fillcolor=curcolor]
{
\newpath
\moveto(683.70039795,72.88151611)
\curveto(684.33039271,72.9015082)(684.83539221,72.81650829)(685.21539795,72.62651611)
\curveto(685.59539145,72.43650867)(685.90039114,72.15150895)(686.13039795,71.77151611)
\curveto(686.19039085,71.67150943)(686.23539081,71.56150954)(686.26539795,71.44151611)
\curveto(686.30539074,71.33150977)(686.3403907,71.21650989)(686.37039795,71.09651611)
\curveto(686.42039062,70.9065102)(686.45039059,70.7015104)(686.46039795,70.48151611)
\curveto(686.47039057,70.26151084)(686.47539057,70.03651107)(686.47539795,69.80651611)
\lineto(686.47539795,68.20151611)
\lineto(686.47539795,65.86151611)
\curveto(686.47539057,65.69151541)(686.47039057,65.52151558)(686.46039795,65.35151611)
\curveto(686.46039058,65.18151592)(686.39539065,65.07151603)(686.26539795,65.02151611)
\curveto(686.21539083,65.0015161)(686.16039088,64.99151611)(686.10039795,64.99151611)
\curveto(686.05039099,64.98151612)(685.99539105,64.97651613)(685.93539795,64.97651611)
\curveto(685.80539124,64.97651613)(685.68039136,64.98151612)(685.56039795,64.99151611)
\curveto(685.4403916,64.99151611)(685.35539169,65.03151607)(685.30539795,65.11151611)
\curveto(685.25539179,65.18151592)(685.23039181,65.27151583)(685.23039795,65.38151611)
\lineto(685.23039795,65.71151611)
\lineto(685.23039795,67.00151611)
\lineto(685.23039795,69.44651611)
\curveto(685.23039181,69.71651139)(685.22539182,69.98151112)(685.21539795,70.24151611)
\curveto(685.20539184,70.51151059)(685.16039188,70.74151036)(685.08039795,70.93151611)
\curveto(685.00039204,71.13150997)(684.88039216,71.29150981)(684.72039795,71.41151611)
\curveto(684.56039248,71.54150956)(684.37539267,71.64150946)(684.16539795,71.71151611)
\curveto(684.10539294,71.73150937)(684.040393,71.74150936)(683.97039795,71.74151611)
\curveto(683.91039313,71.75150935)(683.85039319,71.76650934)(683.79039795,71.78651611)
\curveto(683.7403933,71.79650931)(683.66039338,71.79650931)(683.55039795,71.78651611)
\curveto(683.45039359,71.78650932)(683.38039366,71.78150932)(683.34039795,71.77151611)
\curveto(683.30039374,71.75150935)(683.26539378,71.74150936)(683.23539795,71.74151611)
\curveto(683.20539384,71.75150935)(683.17039387,71.75150935)(683.13039795,71.74151611)
\curveto(683.00039404,71.71150939)(682.87539417,71.67650943)(682.75539795,71.63651611)
\curveto(682.6453944,71.6065095)(682.5403945,71.56150954)(682.44039795,71.50151611)
\curveto(682.40039464,71.48150962)(682.36539468,71.46150964)(682.33539795,71.44151611)
\curveto(682.30539474,71.42150968)(682.27039477,71.4015097)(682.23039795,71.38151611)
\curveto(681.88039516,71.13150997)(681.62539542,70.75651035)(681.46539795,70.25651611)
\curveto(681.43539561,70.17651093)(681.41539563,70.09151101)(681.40539795,70.00151611)
\curveto(681.39539565,69.92151118)(681.38039566,69.84151126)(681.36039795,69.76151611)
\curveto(681.3403957,69.71151139)(681.33539571,69.66151144)(681.34539795,69.61151611)
\curveto(681.35539569,69.57151153)(681.35039569,69.53151157)(681.33039795,69.49151611)
\lineto(681.33039795,69.17651611)
\curveto(681.32039572,69.14651196)(681.31539573,69.11151199)(681.31539795,69.07151611)
\curveto(681.32539572,69.03151207)(681.33039571,68.98651212)(681.33039795,68.93651611)
\lineto(681.33039795,68.48651611)
\lineto(681.33039795,67.04651611)
\lineto(681.33039795,65.72651611)
\lineto(681.33039795,65.38151611)
\curveto(681.33039571,65.27151583)(681.30539574,65.18151592)(681.25539795,65.11151611)
\curveto(681.20539584,65.03151607)(681.11539593,64.99151611)(680.98539795,64.99151611)
\curveto(680.86539618,64.98151612)(680.7403963,64.97651613)(680.61039795,64.97651611)
\curveto(680.53039651,64.97651613)(680.45539659,64.98151612)(680.38539795,64.99151611)
\curveto(680.31539673,65.0015161)(680.25539679,65.02651608)(680.20539795,65.06651611)
\curveto(680.12539692,65.11651599)(680.08539696,65.21151589)(680.08539795,65.35151611)
\lineto(680.08539795,65.75651611)
\lineto(680.08539795,67.52651611)
\lineto(680.08539795,71.15651611)
\lineto(680.08539795,72.07151611)
\lineto(680.08539795,72.34151611)
\curveto(680.08539696,72.43150867)(680.10539694,72.5015086)(680.14539795,72.55151611)
\curveto(680.17539687,72.61150849)(680.22539682,72.65150845)(680.29539795,72.67151611)
\curveto(680.33539671,72.68150842)(680.39039665,72.69150841)(680.46039795,72.70151611)
\curveto(680.5403965,72.71150839)(680.62039642,72.71650839)(680.70039795,72.71651611)
\curveto(680.78039626,72.71650839)(680.85539619,72.71150839)(680.92539795,72.70151611)
\curveto(681.00539604,72.69150841)(681.06039598,72.67650843)(681.09039795,72.65651611)
\curveto(681.20039584,72.58650852)(681.25039579,72.49650861)(681.24039795,72.38651611)
\curveto(681.23039581,72.28650882)(681.2453958,72.17150893)(681.28539795,72.04151611)
\curveto(681.30539574,71.98150912)(681.3453957,71.93150917)(681.40539795,71.89151611)
\curveto(681.52539552,71.88150922)(681.62039542,71.92650918)(681.69039795,72.02651611)
\curveto(681.77039527,72.12650898)(681.85039519,72.2065089)(681.93039795,72.26651611)
\curveto(682.07039497,72.36650874)(682.21039483,72.45650865)(682.35039795,72.53651611)
\curveto(682.50039454,72.62650848)(682.67039437,72.7015084)(682.86039795,72.76151611)
\curveto(682.9403941,72.79150831)(683.02539402,72.81150829)(683.11539795,72.82151611)
\curveto(683.21539383,72.83150827)(683.31039373,72.84650826)(683.40039795,72.86651611)
\curveto(683.45039359,72.87650823)(683.50039354,72.88150822)(683.55039795,72.88151611)
\lineto(683.70039795,72.88151611)
}
}
{
\newrgbcolor{curcolor}{0 0 0}
\pscustom[linestyle=none,fillstyle=solid,fillcolor=curcolor]
{
\newpath
\moveto(688.64500732,74.23151611)
\curveto(688.5650062,74.29150681)(688.52000625,74.39650671)(688.51000732,74.54651611)
\lineto(688.51000732,75.01151611)
\lineto(688.51000732,75.26651611)
\curveto(688.51000626,75.35650575)(688.52500624,75.43150567)(688.55500732,75.49151611)
\curveto(688.59500617,75.57150553)(688.67500609,75.63150547)(688.79500732,75.67151611)
\curveto(688.81500595,75.68150542)(688.83500593,75.68150542)(688.85500732,75.67151611)
\curveto(688.88500588,75.67150543)(688.91000586,75.67650543)(688.93000732,75.68651611)
\curveto(689.10000567,75.68650542)(689.26000551,75.68150542)(689.41000732,75.67151611)
\curveto(689.56000521,75.66150544)(689.66000511,75.6015055)(689.71000732,75.49151611)
\curveto(689.74000503,75.43150567)(689.75500501,75.35650575)(689.75500732,75.26651611)
\lineto(689.75500732,75.01151611)
\curveto(689.75500501,74.83150627)(689.75000502,74.66150644)(689.74000732,74.50151611)
\curveto(689.74000503,74.34150676)(689.67500509,74.23650687)(689.54500732,74.18651611)
\curveto(689.49500527,74.16650694)(689.44000533,74.15650695)(689.38000732,74.15651611)
\lineto(689.21500732,74.15651611)
\lineto(688.90000732,74.15651611)
\curveto(688.80000597,74.15650695)(688.71500605,74.18150692)(688.64500732,74.23151611)
\moveto(689.75500732,65.72651611)
\lineto(689.75500732,65.41151611)
\curveto(689.765005,65.31151579)(689.74500502,65.23151587)(689.69500732,65.17151611)
\curveto(689.6650051,65.11151599)(689.62000515,65.07151603)(689.56000732,65.05151611)
\curveto(689.50000527,65.04151606)(689.43000534,65.02651608)(689.35000732,65.00651611)
\lineto(689.12500732,65.00651611)
\curveto(688.99500577,65.0065161)(688.88000589,65.01151609)(688.78000732,65.02151611)
\curveto(688.69000608,65.04151606)(688.62000615,65.09151601)(688.57000732,65.17151611)
\curveto(688.53000624,65.23151587)(688.51000626,65.3065158)(688.51000732,65.39651611)
\lineto(688.51000732,65.68151611)
\lineto(688.51000732,72.02651611)
\lineto(688.51000732,72.34151611)
\curveto(688.51000626,72.45150865)(688.53500623,72.53650857)(688.58500732,72.59651611)
\curveto(688.61500615,72.64650846)(688.65500611,72.67650843)(688.70500732,72.68651611)
\curveto(688.75500601,72.69650841)(688.81000596,72.71150839)(688.87000732,72.73151611)
\curveto(688.89000588,72.73150837)(688.91000586,72.72650838)(688.93000732,72.71651611)
\curveto(688.96000581,72.71650839)(688.98500578,72.72150838)(689.00500732,72.73151611)
\curveto(689.13500563,72.73150837)(689.2650055,72.72650838)(689.39500732,72.71651611)
\curveto(689.53500523,72.71650839)(689.63000514,72.67650843)(689.68000732,72.59651611)
\curveto(689.73000504,72.53650857)(689.75500501,72.45650865)(689.75500732,72.35651611)
\lineto(689.75500732,72.07151611)
\lineto(689.75500732,65.72651611)
}
}
{
\newrgbcolor{curcolor}{0 0 0}
\pscustom[linestyle=none,fillstyle=solid,fillcolor=curcolor]
{
\newpath
\moveto(698.65985107,65.81651611)
\lineto(698.65985107,65.42651611)
\curveto(698.6598432,65.3065158)(698.63484322,65.2065159)(698.58485107,65.12651611)
\curveto(698.53484332,65.05651605)(698.44984341,65.01651609)(698.32985107,65.00651611)
\lineto(697.98485107,65.00651611)
\curveto(697.92484393,65.0065161)(697.86484399,65.0015161)(697.80485107,64.99151611)
\curveto(697.7548441,64.99151611)(697.70984415,65.0015161)(697.66985107,65.02151611)
\curveto(697.57984428,65.04151606)(697.51984434,65.08151602)(697.48985107,65.14151611)
\curveto(697.44984441,65.19151591)(697.42484443,65.25151585)(697.41485107,65.32151611)
\curveto(697.41484444,65.39151571)(697.39984446,65.46151564)(697.36985107,65.53151611)
\curveto(697.3598445,65.55151555)(697.34484451,65.56651554)(697.32485107,65.57651611)
\curveto(697.31484454,65.59651551)(697.29984456,65.61651549)(697.27985107,65.63651611)
\curveto(697.17984468,65.64651546)(697.09984476,65.62651548)(697.03985107,65.57651611)
\curveto(696.98984487,65.52651558)(696.93484492,65.47651563)(696.87485107,65.42651611)
\curveto(696.67484518,65.27651583)(696.47484538,65.16151594)(696.27485107,65.08151611)
\curveto(696.09484576,65.0015161)(695.88484597,64.94151616)(695.64485107,64.90151611)
\curveto(695.41484644,64.86151624)(695.17484668,64.84151626)(694.92485107,64.84151611)
\curveto(694.68484717,64.83151627)(694.44484741,64.84651626)(694.20485107,64.88651611)
\curveto(693.96484789,64.91651619)(693.7548481,64.97151613)(693.57485107,65.05151611)
\curveto(693.0548488,65.27151583)(692.63484922,65.56651554)(692.31485107,65.93651611)
\curveto(691.99484986,66.31651479)(691.74485011,66.78651432)(691.56485107,67.34651611)
\curveto(691.52485033,67.43651367)(691.49485036,67.52651358)(691.47485107,67.61651611)
\curveto(691.46485039,67.71651339)(691.44485041,67.81651329)(691.41485107,67.91651611)
\curveto(691.40485045,67.96651314)(691.39985046,68.01651309)(691.39985107,68.06651611)
\curveto(691.39985046,68.11651299)(691.39485046,68.16651294)(691.38485107,68.21651611)
\curveto(691.36485049,68.26651284)(691.3548505,68.31651279)(691.35485107,68.36651611)
\curveto(691.36485049,68.42651268)(691.36485049,68.48151262)(691.35485107,68.53151611)
\lineto(691.35485107,68.68151611)
\curveto(691.33485052,68.73151237)(691.32485053,68.79651231)(691.32485107,68.87651611)
\curveto(691.32485053,68.95651215)(691.33485052,69.02151208)(691.35485107,69.07151611)
\lineto(691.35485107,69.23651611)
\curveto(691.37485048,69.3065118)(691.37985048,69.37651173)(691.36985107,69.44651611)
\curveto(691.36985049,69.52651158)(691.37985048,69.6015115)(691.39985107,69.67151611)
\curveto(691.40985045,69.72151138)(691.41485044,69.76651134)(691.41485107,69.80651611)
\curveto(691.41485044,69.84651126)(691.41985044,69.89151121)(691.42985107,69.94151611)
\curveto(691.4598504,70.04151106)(691.48485037,70.13651097)(691.50485107,70.22651611)
\curveto(691.52485033,70.32651078)(691.54985031,70.42151068)(691.57985107,70.51151611)
\curveto(691.70985015,70.89151021)(691.87484998,71.23150987)(692.07485107,71.53151611)
\curveto(692.28484957,71.84150926)(692.53484932,72.09650901)(692.82485107,72.29651611)
\curveto(692.99484886,72.41650869)(693.16984869,72.51650859)(693.34985107,72.59651611)
\curveto(693.53984832,72.67650843)(693.74484811,72.74650836)(693.96485107,72.80651611)
\curveto(694.03484782,72.81650829)(694.09984776,72.82650828)(694.15985107,72.83651611)
\curveto(694.22984763,72.84650826)(694.29984756,72.86150824)(694.36985107,72.88151611)
\lineto(694.51985107,72.88151611)
\curveto(694.59984726,72.9015082)(694.71484714,72.91150819)(694.86485107,72.91151611)
\curveto(695.02484683,72.91150819)(695.14484671,72.9015082)(695.22485107,72.88151611)
\curveto(695.26484659,72.87150823)(695.31984654,72.86650824)(695.38985107,72.86651611)
\curveto(695.49984636,72.83650827)(695.60984625,72.81150829)(695.71985107,72.79151611)
\curveto(695.82984603,72.78150832)(695.93484592,72.75150835)(696.03485107,72.70151611)
\curveto(696.18484567,72.64150846)(696.32484553,72.57650853)(696.45485107,72.50651611)
\curveto(696.59484526,72.43650867)(696.72484513,72.35650875)(696.84485107,72.26651611)
\curveto(696.90484495,72.21650889)(696.96484489,72.16150894)(697.02485107,72.10151611)
\curveto(697.09484476,72.05150905)(697.18484467,72.03650907)(697.29485107,72.05651611)
\curveto(697.31484454,72.08650902)(697.32984453,72.11150899)(697.33985107,72.13151611)
\curveto(697.3598445,72.15150895)(697.37484448,72.18150892)(697.38485107,72.22151611)
\curveto(697.41484444,72.31150879)(697.42484443,72.42650868)(697.41485107,72.56651611)
\lineto(697.41485107,72.94151611)
\lineto(697.41485107,74.66651611)
\lineto(697.41485107,75.13151611)
\curveto(697.41484444,75.31150579)(697.43984442,75.44150566)(697.48985107,75.52151611)
\curveto(697.52984433,75.59150551)(697.58984427,75.63650547)(697.66985107,75.65651611)
\curveto(697.68984417,75.65650545)(697.71484414,75.65650545)(697.74485107,75.65651611)
\curveto(697.77484408,75.66650544)(697.79984406,75.67150543)(697.81985107,75.67151611)
\curveto(697.9598439,75.68150542)(698.10484375,75.68150542)(698.25485107,75.67151611)
\curveto(698.41484344,75.67150543)(698.52484333,75.63150547)(698.58485107,75.55151611)
\curveto(698.63484322,75.47150563)(698.6598432,75.37150573)(698.65985107,75.25151611)
\lineto(698.65985107,74.87651611)
\lineto(698.65985107,65.81651611)
\moveto(697.44485107,68.65151611)
\curveto(697.46484439,68.7015124)(697.47484438,68.76651234)(697.47485107,68.84651611)
\curveto(697.47484438,68.93651217)(697.46484439,69.0065121)(697.44485107,69.05651611)
\lineto(697.44485107,69.28151611)
\curveto(697.42484443,69.37151173)(697.40984445,69.46151164)(697.39985107,69.55151611)
\curveto(697.38984447,69.65151145)(697.36984449,69.74151136)(697.33985107,69.82151611)
\curveto(697.31984454,69.9015112)(697.29984456,69.97651113)(697.27985107,70.04651611)
\curveto(697.26984459,70.11651099)(697.24984461,70.18651092)(697.21985107,70.25651611)
\curveto(697.09984476,70.55651055)(696.94484491,70.82151028)(696.75485107,71.05151611)
\curveto(696.56484529,71.28150982)(696.32484553,71.46150964)(696.03485107,71.59151611)
\curveto(695.93484592,71.64150946)(695.82984603,71.67650943)(695.71985107,71.69651611)
\curveto(695.61984624,71.72650938)(695.50984635,71.75150935)(695.38985107,71.77151611)
\curveto(695.30984655,71.79150931)(695.21984664,71.8015093)(695.11985107,71.80151611)
\lineto(694.84985107,71.80151611)
\curveto(694.79984706,71.79150931)(694.7548471,71.78150932)(694.71485107,71.77151611)
\lineto(694.57985107,71.77151611)
\curveto(694.49984736,71.75150935)(694.41484744,71.73150937)(694.32485107,71.71151611)
\curveto(694.24484761,71.69150941)(694.16484769,71.66650944)(694.08485107,71.63651611)
\curveto(693.76484809,71.49650961)(693.50484835,71.29150981)(693.30485107,71.02151611)
\curveto(693.11484874,70.76151034)(692.9598489,70.45651065)(692.83985107,70.10651611)
\curveto(692.79984906,69.99651111)(692.76984909,69.88151122)(692.74985107,69.76151611)
\curveto(692.73984912,69.65151145)(692.72484913,69.54151156)(692.70485107,69.43151611)
\curveto(692.70484915,69.39151171)(692.69984916,69.35151175)(692.68985107,69.31151611)
\lineto(692.68985107,69.20651611)
\curveto(692.66984919,69.15651195)(692.6598492,69.101512)(692.65985107,69.04151611)
\curveto(692.66984919,68.98151212)(692.67484918,68.92651218)(692.67485107,68.87651611)
\lineto(692.67485107,68.54651611)
\curveto(692.67484918,68.44651266)(692.68484917,68.35151275)(692.70485107,68.26151611)
\curveto(692.71484914,68.23151287)(692.71984914,68.18151292)(692.71985107,68.11151611)
\curveto(692.73984912,68.04151306)(692.7548491,67.97151313)(692.76485107,67.90151611)
\lineto(692.82485107,67.69151611)
\curveto(692.93484892,67.34151376)(693.08484877,67.04151406)(693.27485107,66.79151611)
\curveto(693.46484839,66.54151456)(693.70484815,66.33651477)(693.99485107,66.17651611)
\curveto(694.08484777,66.12651498)(694.17484768,66.08651502)(694.26485107,66.05651611)
\curveto(694.3548475,66.02651508)(694.4548474,65.99651511)(694.56485107,65.96651611)
\curveto(694.61484724,65.94651516)(694.66484719,65.94151516)(694.71485107,65.95151611)
\curveto(694.77484708,65.96151514)(694.82984703,65.95651515)(694.87985107,65.93651611)
\curveto(694.91984694,65.92651518)(694.9598469,65.92151518)(694.99985107,65.92151611)
\lineto(695.13485107,65.92151611)
\lineto(695.26985107,65.92151611)
\curveto(695.29984656,65.93151517)(695.34984651,65.93651517)(695.41985107,65.93651611)
\curveto(695.49984636,65.95651515)(695.57984628,65.97151513)(695.65985107,65.98151611)
\curveto(695.73984612,66.0015151)(695.81484604,66.02651508)(695.88485107,66.05651611)
\curveto(696.21484564,66.19651491)(696.47984538,66.37151473)(696.67985107,66.58151611)
\curveto(696.88984497,66.8015143)(697.06484479,67.07651403)(697.20485107,67.40651611)
\curveto(697.2548446,67.51651359)(697.28984457,67.62651348)(697.30985107,67.73651611)
\curveto(697.32984453,67.84651326)(697.3548445,67.95651315)(697.38485107,68.06651611)
\curveto(697.40484445,68.106513)(697.41484444,68.14151296)(697.41485107,68.17151611)
\curveto(697.41484444,68.21151289)(697.41984444,68.25151285)(697.42985107,68.29151611)
\curveto(697.43984442,68.35151275)(697.43984442,68.41151269)(697.42985107,68.47151611)
\curveto(697.42984443,68.53151257)(697.43484442,68.59151251)(697.44485107,68.65151611)
}
}
{
\newrgbcolor{curcolor}{0 0 0}
\pscustom[linestyle=none,fillstyle=solid,fillcolor=curcolor]
{
\newpath
\moveto(707.49110107,65.56151611)
\curveto(707.52109324,65.4015157)(707.50609326,65.26651584)(707.44610107,65.15651611)
\curveto(707.38609338,65.05651605)(707.30609346,64.98151612)(707.20610107,64.93151611)
\curveto(707.15609361,64.91151619)(707.10109366,64.9015162)(707.04110107,64.90151611)
\curveto(706.99109377,64.9015162)(706.93609383,64.89151621)(706.87610107,64.87151611)
\curveto(706.65609411,64.82151628)(706.43609433,64.83651627)(706.21610107,64.91651611)
\curveto(706.00609476,64.98651612)(705.8610949,65.07651603)(705.78110107,65.18651611)
\curveto(705.73109503,65.25651585)(705.68609508,65.33651577)(705.64610107,65.42651611)
\curveto(705.60609516,65.52651558)(705.55609521,65.6065155)(705.49610107,65.66651611)
\curveto(705.47609529,65.68651542)(705.45109531,65.7065154)(705.42110107,65.72651611)
\curveto(705.40109536,65.74651536)(705.37109539,65.75151535)(705.33110107,65.74151611)
\curveto(705.22109554,65.71151539)(705.11609565,65.65651545)(705.01610107,65.57651611)
\curveto(704.92609584,65.49651561)(704.83609593,65.42651568)(704.74610107,65.36651611)
\curveto(704.61609615,65.28651582)(704.47609629,65.21151589)(704.32610107,65.14151611)
\curveto(704.17609659,65.08151602)(704.01609675,65.02651608)(703.84610107,64.97651611)
\curveto(703.74609702,64.94651616)(703.63609713,64.92651618)(703.51610107,64.91651611)
\curveto(703.40609736,64.9065162)(703.29609747,64.89151621)(703.18610107,64.87151611)
\curveto(703.13609763,64.86151624)(703.09109767,64.85651625)(703.05110107,64.85651611)
\lineto(702.94610107,64.85651611)
\curveto(702.83609793,64.83651627)(702.73109803,64.83651627)(702.63110107,64.85651611)
\lineto(702.49610107,64.85651611)
\curveto(702.44609832,64.86651624)(702.39609837,64.87151623)(702.34610107,64.87151611)
\curveto(702.29609847,64.87151623)(702.25109851,64.88151622)(702.21110107,64.90151611)
\curveto(702.17109859,64.91151619)(702.13609863,64.91651619)(702.10610107,64.91651611)
\curveto(702.08609868,64.9065162)(702.0610987,64.9065162)(702.03110107,64.91651611)
\lineto(701.79110107,64.97651611)
\curveto(701.71109905,64.98651612)(701.63609913,65.0065161)(701.56610107,65.03651611)
\curveto(701.2660995,65.16651594)(701.02109974,65.31151579)(700.83110107,65.47151611)
\curveto(700.65110011,65.64151546)(700.50110026,65.87651523)(700.38110107,66.17651611)
\curveto(700.29110047,66.39651471)(700.24610052,66.66151444)(700.24610107,66.97151611)
\lineto(700.24610107,67.28651611)
\curveto(700.25610051,67.33651377)(700.2611005,67.38651372)(700.26110107,67.43651611)
\lineto(700.29110107,67.61651611)
\lineto(700.41110107,67.94651611)
\curveto(700.45110031,68.05651305)(700.50110026,68.15651295)(700.56110107,68.24651611)
\curveto(700.74110002,68.53651257)(700.98609978,68.75151235)(701.29610107,68.89151611)
\curveto(701.60609916,69.03151207)(701.94609882,69.15651195)(702.31610107,69.26651611)
\curveto(702.45609831,69.3065118)(702.60109816,69.33651177)(702.75110107,69.35651611)
\curveto(702.90109786,69.37651173)(703.05109771,69.4015117)(703.20110107,69.43151611)
\curveto(703.27109749,69.45151165)(703.33609743,69.46151164)(703.39610107,69.46151611)
\curveto(703.4660973,69.46151164)(703.54109722,69.47151163)(703.62110107,69.49151611)
\curveto(703.69109707,69.51151159)(703.761097,69.52151158)(703.83110107,69.52151611)
\curveto(703.90109686,69.53151157)(703.97609679,69.54651156)(704.05610107,69.56651611)
\curveto(704.30609646,69.62651148)(704.54109622,69.67651143)(704.76110107,69.71651611)
\curveto(704.98109578,69.76651134)(705.15609561,69.88151122)(705.28610107,70.06151611)
\curveto(705.34609542,70.14151096)(705.39609537,70.24151086)(705.43610107,70.36151611)
\curveto(705.47609529,70.49151061)(705.47609529,70.63151047)(705.43610107,70.78151611)
\curveto(705.37609539,71.02151008)(705.28609548,71.21150989)(705.16610107,71.35151611)
\curveto(705.05609571,71.49150961)(704.89609587,71.6015095)(704.68610107,71.68151611)
\curveto(704.5660962,71.73150937)(704.42109634,71.76650934)(704.25110107,71.78651611)
\curveto(704.09109667,71.8065093)(703.92109684,71.81650929)(703.74110107,71.81651611)
\curveto(703.5610972,71.81650929)(703.38609738,71.8065093)(703.21610107,71.78651611)
\curveto(703.04609772,71.76650934)(702.90109786,71.73650937)(702.78110107,71.69651611)
\curveto(702.61109815,71.63650947)(702.44609832,71.55150955)(702.28610107,71.44151611)
\curveto(702.20609856,71.38150972)(702.13109863,71.3015098)(702.06110107,71.20151611)
\curveto(702.00109876,71.11150999)(701.94609882,71.01151009)(701.89610107,70.90151611)
\curveto(701.8660989,70.82151028)(701.83609893,70.73651037)(701.80610107,70.64651611)
\curveto(701.78609898,70.55651055)(701.74109902,70.48651062)(701.67110107,70.43651611)
\curveto(701.63109913,70.4065107)(701.5610992,70.38151072)(701.46110107,70.36151611)
\curveto(701.37109939,70.35151075)(701.27609949,70.34651076)(701.17610107,70.34651611)
\curveto(701.07609969,70.34651076)(700.97609979,70.35151075)(700.87610107,70.36151611)
\curveto(700.78609998,70.38151072)(700.72110004,70.4065107)(700.68110107,70.43651611)
\curveto(700.64110012,70.46651064)(700.61110015,70.51651059)(700.59110107,70.58651611)
\curveto(700.57110019,70.65651045)(700.57110019,70.73151037)(700.59110107,70.81151611)
\curveto(700.62110014,70.94151016)(700.65110011,71.06151004)(700.68110107,71.17151611)
\curveto(700.72110004,71.29150981)(700.7661,71.4065097)(700.81610107,71.51651611)
\curveto(701.00609976,71.86650924)(701.24609952,72.13650897)(701.53610107,72.32651611)
\curveto(701.82609894,72.52650858)(702.18609858,72.68650842)(702.61610107,72.80651611)
\curveto(702.71609805,72.82650828)(702.81609795,72.84150826)(702.91610107,72.85151611)
\curveto(703.02609774,72.86150824)(703.13609763,72.87650823)(703.24610107,72.89651611)
\curveto(703.28609748,72.9065082)(703.35109741,72.9065082)(703.44110107,72.89651611)
\curveto(703.53109723,72.89650821)(703.58609718,72.9065082)(703.60610107,72.92651611)
\curveto(704.30609646,72.93650817)(704.91609585,72.85650825)(705.43610107,72.68651611)
\curveto(705.95609481,72.51650859)(706.32109444,72.19150891)(706.53110107,71.71151611)
\curveto(706.62109414,71.51150959)(706.67109409,71.27650983)(706.68110107,71.00651611)
\curveto(706.70109406,70.74651036)(706.71109405,70.47151063)(706.71110107,70.18151611)
\lineto(706.71110107,66.86651611)
\curveto(706.71109405,66.72651438)(706.71609405,66.59151451)(706.72610107,66.46151611)
\curveto(706.73609403,66.33151477)(706.766094,66.22651488)(706.81610107,66.14651611)
\curveto(706.8660939,66.07651503)(706.93109383,66.02651508)(707.01110107,65.99651611)
\curveto(707.10109366,65.95651515)(707.18609358,65.92651518)(707.26610107,65.90651611)
\curveto(707.34609342,65.89651521)(707.40609336,65.85151525)(707.44610107,65.77151611)
\curveto(707.4660933,65.74151536)(707.47609329,65.71151539)(707.47610107,65.68151611)
\curveto(707.47609329,65.65151545)(707.48109328,65.61151549)(707.49110107,65.56151611)
\moveto(705.34610107,67.22651611)
\curveto(705.40609536,67.36651374)(705.43609533,67.52651358)(705.43610107,67.70651611)
\curveto(705.44609532,67.89651321)(705.45109531,68.09151301)(705.45110107,68.29151611)
\curveto(705.45109531,68.4015127)(705.44609532,68.5015126)(705.43610107,68.59151611)
\curveto(705.42609534,68.68151242)(705.38609538,68.75151235)(705.31610107,68.80151611)
\curveto(705.28609548,68.82151228)(705.21609555,68.83151227)(705.10610107,68.83151611)
\curveto(705.08609568,68.81151229)(705.05109571,68.8015123)(705.00110107,68.80151611)
\curveto(704.95109581,68.8015123)(704.90609586,68.79151231)(704.86610107,68.77151611)
\curveto(704.78609598,68.75151235)(704.69609607,68.73151237)(704.59610107,68.71151611)
\lineto(704.29610107,68.65151611)
\curveto(704.2660965,68.65151245)(704.23109653,68.64651246)(704.19110107,68.63651611)
\lineto(704.08610107,68.63651611)
\curveto(703.93609683,68.59651251)(703.77109699,68.57151253)(703.59110107,68.56151611)
\curveto(703.42109734,68.56151254)(703.2610975,68.54151256)(703.11110107,68.50151611)
\curveto(703.03109773,68.48151262)(702.95609781,68.46151264)(702.88610107,68.44151611)
\curveto(702.82609794,68.43151267)(702.75609801,68.41651269)(702.67610107,68.39651611)
\curveto(702.51609825,68.34651276)(702.3660984,68.28151282)(702.22610107,68.20151611)
\curveto(702.08609868,68.13151297)(701.9660988,68.04151306)(701.86610107,67.93151611)
\curveto(701.766099,67.82151328)(701.69109907,67.68651342)(701.64110107,67.52651611)
\curveto(701.59109917,67.37651373)(701.57109919,67.19151391)(701.58110107,66.97151611)
\curveto(701.58109918,66.87151423)(701.59609917,66.77651433)(701.62610107,66.68651611)
\curveto(701.6660991,66.6065145)(701.71109905,66.53151457)(701.76110107,66.46151611)
\curveto(701.84109892,66.35151475)(701.94609882,66.25651485)(702.07610107,66.17651611)
\curveto(702.20609856,66.106515)(702.34609842,66.04651506)(702.49610107,65.99651611)
\curveto(702.54609822,65.98651512)(702.59609817,65.98151512)(702.64610107,65.98151611)
\curveto(702.69609807,65.98151512)(702.74609802,65.97651513)(702.79610107,65.96651611)
\curveto(702.8660979,65.94651516)(702.95109781,65.93151517)(703.05110107,65.92151611)
\curveto(703.1610976,65.92151518)(703.25109751,65.93151517)(703.32110107,65.95151611)
\curveto(703.38109738,65.97151513)(703.44109732,65.97651513)(703.50110107,65.96651611)
\curveto(703.5610972,65.96651514)(703.62109714,65.97651513)(703.68110107,65.99651611)
\curveto(703.761097,66.01651509)(703.83609693,66.03151507)(703.90610107,66.04151611)
\curveto(703.98609678,66.05151505)(704.0610967,66.07151503)(704.13110107,66.10151611)
\curveto(704.42109634,66.22151488)(704.6660961,66.36651474)(704.86610107,66.53651611)
\curveto(705.07609569,66.7065144)(705.23609553,66.93651417)(705.34610107,67.22651611)
}
}
{
\newrgbcolor{curcolor}{0 0 0}
\pscustom[linestyle=none,fillstyle=solid,fillcolor=curcolor]
{
\newpath
\moveto(715.6227417,65.81651611)
\lineto(715.6227417,65.42651611)
\curveto(715.62273382,65.3065158)(715.59773385,65.2065159)(715.5477417,65.12651611)
\curveto(715.49773395,65.05651605)(715.41273403,65.01651609)(715.2927417,65.00651611)
\lineto(714.9477417,65.00651611)
\curveto(714.88773456,65.0065161)(714.82773462,65.0015161)(714.7677417,64.99151611)
\curveto(714.71773473,64.99151611)(714.67273477,65.0015161)(714.6327417,65.02151611)
\curveto(714.5427349,65.04151606)(714.48273496,65.08151602)(714.4527417,65.14151611)
\curveto(714.41273503,65.19151591)(714.38773506,65.25151585)(714.3777417,65.32151611)
\curveto(714.37773507,65.39151571)(714.36273508,65.46151564)(714.3327417,65.53151611)
\curveto(714.32273512,65.55151555)(714.30773514,65.56651554)(714.2877417,65.57651611)
\curveto(714.27773517,65.59651551)(714.26273518,65.61651549)(714.2427417,65.63651611)
\curveto(714.1427353,65.64651546)(714.06273538,65.62651548)(714.0027417,65.57651611)
\curveto(713.95273549,65.52651558)(713.89773555,65.47651563)(713.8377417,65.42651611)
\curveto(713.63773581,65.27651583)(713.43773601,65.16151594)(713.2377417,65.08151611)
\curveto(713.05773639,65.0015161)(712.8477366,64.94151616)(712.6077417,64.90151611)
\curveto(712.37773707,64.86151624)(712.13773731,64.84151626)(711.8877417,64.84151611)
\curveto(711.6477378,64.83151627)(711.40773804,64.84651626)(711.1677417,64.88651611)
\curveto(710.92773852,64.91651619)(710.71773873,64.97151613)(710.5377417,65.05151611)
\curveto(710.01773943,65.27151583)(709.59773985,65.56651554)(709.2777417,65.93651611)
\curveto(708.95774049,66.31651479)(708.70774074,66.78651432)(708.5277417,67.34651611)
\curveto(708.48774096,67.43651367)(708.45774099,67.52651358)(708.4377417,67.61651611)
\curveto(708.42774102,67.71651339)(708.40774104,67.81651329)(708.3777417,67.91651611)
\curveto(708.36774108,67.96651314)(708.36274108,68.01651309)(708.3627417,68.06651611)
\curveto(708.36274108,68.11651299)(708.35774109,68.16651294)(708.3477417,68.21651611)
\curveto(708.32774112,68.26651284)(708.31774113,68.31651279)(708.3177417,68.36651611)
\curveto(708.32774112,68.42651268)(708.32774112,68.48151262)(708.3177417,68.53151611)
\lineto(708.3177417,68.68151611)
\curveto(708.29774115,68.73151237)(708.28774116,68.79651231)(708.2877417,68.87651611)
\curveto(708.28774116,68.95651215)(708.29774115,69.02151208)(708.3177417,69.07151611)
\lineto(708.3177417,69.23651611)
\curveto(708.33774111,69.3065118)(708.3427411,69.37651173)(708.3327417,69.44651611)
\curveto(708.33274111,69.52651158)(708.3427411,69.6015115)(708.3627417,69.67151611)
\curveto(708.37274107,69.72151138)(708.37774107,69.76651134)(708.3777417,69.80651611)
\curveto(708.37774107,69.84651126)(708.38274106,69.89151121)(708.3927417,69.94151611)
\curveto(708.42274102,70.04151106)(708.447741,70.13651097)(708.4677417,70.22651611)
\curveto(708.48774096,70.32651078)(708.51274093,70.42151068)(708.5427417,70.51151611)
\curveto(708.67274077,70.89151021)(708.83774061,71.23150987)(709.0377417,71.53151611)
\curveto(709.2477402,71.84150926)(709.49773995,72.09650901)(709.7877417,72.29651611)
\curveto(709.95773949,72.41650869)(710.13273931,72.51650859)(710.3127417,72.59651611)
\curveto(710.50273894,72.67650843)(710.70773874,72.74650836)(710.9277417,72.80651611)
\curveto(710.99773845,72.81650829)(711.06273838,72.82650828)(711.1227417,72.83651611)
\curveto(711.19273825,72.84650826)(711.26273818,72.86150824)(711.3327417,72.88151611)
\lineto(711.4827417,72.88151611)
\curveto(711.56273788,72.9015082)(711.67773777,72.91150819)(711.8277417,72.91151611)
\curveto(711.98773746,72.91150819)(712.10773734,72.9015082)(712.1877417,72.88151611)
\curveto(712.22773722,72.87150823)(712.28273716,72.86650824)(712.3527417,72.86651611)
\curveto(712.46273698,72.83650827)(712.57273687,72.81150829)(712.6827417,72.79151611)
\curveto(712.79273665,72.78150832)(712.89773655,72.75150835)(712.9977417,72.70151611)
\curveto(713.1477363,72.64150846)(713.28773616,72.57650853)(713.4177417,72.50651611)
\curveto(713.55773589,72.43650867)(713.68773576,72.35650875)(713.8077417,72.26651611)
\curveto(713.86773558,72.21650889)(713.92773552,72.16150894)(713.9877417,72.10151611)
\curveto(714.05773539,72.05150905)(714.1477353,72.03650907)(714.2577417,72.05651611)
\curveto(714.27773517,72.08650902)(714.29273515,72.11150899)(714.3027417,72.13151611)
\curveto(714.32273512,72.15150895)(714.33773511,72.18150892)(714.3477417,72.22151611)
\curveto(714.37773507,72.31150879)(714.38773506,72.42650868)(714.3777417,72.56651611)
\lineto(714.3777417,72.94151611)
\lineto(714.3777417,74.66651611)
\lineto(714.3777417,75.13151611)
\curveto(714.37773507,75.31150579)(714.40273504,75.44150566)(714.4527417,75.52151611)
\curveto(714.49273495,75.59150551)(714.55273489,75.63650547)(714.6327417,75.65651611)
\curveto(714.65273479,75.65650545)(714.67773477,75.65650545)(714.7077417,75.65651611)
\curveto(714.73773471,75.66650544)(714.76273468,75.67150543)(714.7827417,75.67151611)
\curveto(714.92273452,75.68150542)(715.06773438,75.68150542)(715.2177417,75.67151611)
\curveto(715.37773407,75.67150543)(715.48773396,75.63150547)(715.5477417,75.55151611)
\curveto(715.59773385,75.47150563)(715.62273382,75.37150573)(715.6227417,75.25151611)
\lineto(715.6227417,74.87651611)
\lineto(715.6227417,65.81651611)
\moveto(714.4077417,68.65151611)
\curveto(714.42773502,68.7015124)(714.43773501,68.76651234)(714.4377417,68.84651611)
\curveto(714.43773501,68.93651217)(714.42773502,69.0065121)(714.4077417,69.05651611)
\lineto(714.4077417,69.28151611)
\curveto(714.38773506,69.37151173)(714.37273507,69.46151164)(714.3627417,69.55151611)
\curveto(714.35273509,69.65151145)(714.33273511,69.74151136)(714.3027417,69.82151611)
\curveto(714.28273516,69.9015112)(714.26273518,69.97651113)(714.2427417,70.04651611)
\curveto(714.23273521,70.11651099)(714.21273523,70.18651092)(714.1827417,70.25651611)
\curveto(714.06273538,70.55651055)(713.90773554,70.82151028)(713.7177417,71.05151611)
\curveto(713.52773592,71.28150982)(713.28773616,71.46150964)(712.9977417,71.59151611)
\curveto(712.89773655,71.64150946)(712.79273665,71.67650943)(712.6827417,71.69651611)
\curveto(712.58273686,71.72650938)(712.47273697,71.75150935)(712.3527417,71.77151611)
\curveto(712.27273717,71.79150931)(712.18273726,71.8015093)(712.0827417,71.80151611)
\lineto(711.8127417,71.80151611)
\curveto(711.76273768,71.79150931)(711.71773773,71.78150932)(711.6777417,71.77151611)
\lineto(711.5427417,71.77151611)
\curveto(711.46273798,71.75150935)(711.37773807,71.73150937)(711.2877417,71.71151611)
\curveto(711.20773824,71.69150941)(711.12773832,71.66650944)(711.0477417,71.63651611)
\curveto(710.72773872,71.49650961)(710.46773898,71.29150981)(710.2677417,71.02151611)
\curveto(710.07773937,70.76151034)(709.92273952,70.45651065)(709.8027417,70.10651611)
\curveto(709.76273968,69.99651111)(709.73273971,69.88151122)(709.7127417,69.76151611)
\curveto(709.70273974,69.65151145)(709.68773976,69.54151156)(709.6677417,69.43151611)
\curveto(709.66773978,69.39151171)(709.66273978,69.35151175)(709.6527417,69.31151611)
\lineto(709.6527417,69.20651611)
\curveto(709.63273981,69.15651195)(709.62273982,69.101512)(709.6227417,69.04151611)
\curveto(709.63273981,68.98151212)(709.63773981,68.92651218)(709.6377417,68.87651611)
\lineto(709.6377417,68.54651611)
\curveto(709.63773981,68.44651266)(709.6477398,68.35151275)(709.6677417,68.26151611)
\curveto(709.67773977,68.23151287)(709.68273976,68.18151292)(709.6827417,68.11151611)
\curveto(709.70273974,68.04151306)(709.71773973,67.97151313)(709.7277417,67.90151611)
\lineto(709.7877417,67.69151611)
\curveto(709.89773955,67.34151376)(710.0477394,67.04151406)(710.2377417,66.79151611)
\curveto(710.42773902,66.54151456)(710.66773878,66.33651477)(710.9577417,66.17651611)
\curveto(711.0477384,66.12651498)(711.13773831,66.08651502)(711.2277417,66.05651611)
\curveto(711.31773813,66.02651508)(711.41773803,65.99651511)(711.5277417,65.96651611)
\curveto(711.57773787,65.94651516)(711.62773782,65.94151516)(711.6777417,65.95151611)
\curveto(711.73773771,65.96151514)(711.79273765,65.95651515)(711.8427417,65.93651611)
\curveto(711.88273756,65.92651518)(711.92273752,65.92151518)(711.9627417,65.92151611)
\lineto(712.0977417,65.92151611)
\lineto(712.2327417,65.92151611)
\curveto(712.26273718,65.93151517)(712.31273713,65.93651517)(712.3827417,65.93651611)
\curveto(712.46273698,65.95651515)(712.5427369,65.97151513)(712.6227417,65.98151611)
\curveto(712.70273674,66.0015151)(712.77773667,66.02651508)(712.8477417,66.05651611)
\curveto(713.17773627,66.19651491)(713.442736,66.37151473)(713.6427417,66.58151611)
\curveto(713.85273559,66.8015143)(714.02773542,67.07651403)(714.1677417,67.40651611)
\curveto(714.21773523,67.51651359)(714.25273519,67.62651348)(714.2727417,67.73651611)
\curveto(714.29273515,67.84651326)(714.31773513,67.95651315)(714.3477417,68.06651611)
\curveto(714.36773508,68.106513)(714.37773507,68.14151296)(714.3777417,68.17151611)
\curveto(714.37773507,68.21151289)(714.38273506,68.25151285)(714.3927417,68.29151611)
\curveto(714.40273504,68.35151275)(714.40273504,68.41151269)(714.3927417,68.47151611)
\curveto(714.39273505,68.53151257)(714.39773505,68.59151251)(714.4077417,68.65151611)
}
}
{
\newrgbcolor{curcolor}{0 0 0}
\pscustom[linestyle=none,fillstyle=solid,fillcolor=curcolor]
{
\newpath
\moveto(724.3189917,69.17651611)
\curveto(724.33898401,69.07651203)(724.33898401,68.96151214)(724.3189917,68.83151611)
\curveto(724.30898404,68.71151239)(724.27898407,68.62651248)(724.2289917,68.57651611)
\curveto(724.17898417,68.53651257)(724.10398425,68.5065126)(724.0039917,68.48651611)
\curveto(723.91398444,68.47651263)(723.80898454,68.47151263)(723.6889917,68.47151611)
\lineto(723.3289917,68.47151611)
\curveto(723.20898514,68.48151262)(723.10398525,68.48651262)(723.0139917,68.48651611)
\lineto(719.1739917,68.48651611)
\curveto(719.09398926,68.48651262)(719.01398934,68.48151262)(718.9339917,68.47151611)
\curveto(718.8539895,68.47151263)(718.78898956,68.45651265)(718.7389917,68.42651611)
\curveto(718.69898965,68.4065127)(718.65898969,68.36651274)(718.6189917,68.30651611)
\curveto(718.59898975,68.27651283)(718.57898977,68.23151287)(718.5589917,68.17151611)
\curveto(718.53898981,68.12151298)(718.53898981,68.07151303)(718.5589917,68.02151611)
\curveto(718.56898978,67.97151313)(718.57398978,67.92651318)(718.5739917,67.88651611)
\curveto(718.57398978,67.84651326)(718.57898977,67.8065133)(718.5889917,67.76651611)
\curveto(718.60898974,67.68651342)(718.62898972,67.6015135)(718.6489917,67.51151611)
\curveto(718.66898968,67.43151367)(718.69898965,67.35151375)(718.7389917,67.27151611)
\curveto(718.96898938,66.73151437)(719.348989,66.34651476)(719.8789917,66.11651611)
\curveto(719.93898841,66.08651502)(720.00398835,66.06151504)(720.0739917,66.04151611)
\lineto(720.2839917,65.98151611)
\curveto(720.31398804,65.97151513)(720.36398799,65.96651514)(720.4339917,65.96651611)
\curveto(720.57398778,65.92651518)(720.75898759,65.9065152)(720.9889917,65.90651611)
\curveto(721.21898713,65.9065152)(721.40398695,65.92651518)(721.5439917,65.96651611)
\curveto(721.68398667,66.0065151)(721.80898654,66.04651506)(721.9189917,66.08651611)
\curveto(722.03898631,66.13651497)(722.1489862,66.19651491)(722.2489917,66.26651611)
\curveto(722.35898599,66.33651477)(722.4539859,66.41651469)(722.5339917,66.50651611)
\curveto(722.61398574,66.6065145)(722.68398567,66.71151439)(722.7439917,66.82151611)
\curveto(722.80398555,66.92151418)(722.8539855,67.02651408)(722.8939917,67.13651611)
\curveto(722.94398541,67.24651386)(723.02398533,67.32651378)(723.1339917,67.37651611)
\curveto(723.17398518,67.39651371)(723.23898511,67.41151369)(723.3289917,67.42151611)
\curveto(723.41898493,67.43151367)(723.50898484,67.43151367)(723.5989917,67.42151611)
\curveto(723.68898466,67.42151368)(723.77398458,67.41651369)(723.8539917,67.40651611)
\curveto(723.93398442,67.39651371)(723.98898436,67.37651373)(724.0189917,67.34651611)
\curveto(724.11898423,67.27651383)(724.14398421,67.16151394)(724.0939917,67.00151611)
\curveto(724.01398434,66.73151437)(723.90898444,66.49151461)(723.7789917,66.28151611)
\curveto(723.57898477,65.96151514)(723.348985,65.69651541)(723.0889917,65.48651611)
\curveto(722.83898551,65.28651582)(722.51898583,65.12151598)(722.1289917,64.99151611)
\curveto(722.02898632,64.95151615)(721.92898642,64.92651618)(721.8289917,64.91651611)
\curveto(721.72898662,64.89651621)(721.62398673,64.87651623)(721.5139917,64.85651611)
\curveto(721.46398689,64.84651626)(721.41398694,64.84151626)(721.3639917,64.84151611)
\curveto(721.32398703,64.84151626)(721.27898707,64.83651627)(721.2289917,64.82651611)
\lineto(721.0789917,64.82651611)
\curveto(721.02898732,64.81651629)(720.96898738,64.81151629)(720.8989917,64.81151611)
\curveto(720.83898751,64.81151629)(720.78898756,64.81651629)(720.7489917,64.82651611)
\lineto(720.6139917,64.82651611)
\curveto(720.56398779,64.83651627)(720.51898783,64.84151626)(720.4789917,64.84151611)
\curveto(720.43898791,64.84151626)(720.39898795,64.84651626)(720.3589917,64.85651611)
\curveto(720.30898804,64.86651624)(720.2539881,64.87651623)(720.1939917,64.88651611)
\curveto(720.13398822,64.88651622)(720.07898827,64.89151621)(720.0289917,64.90151611)
\curveto(719.93898841,64.92151618)(719.8489885,64.94651616)(719.7589917,64.97651611)
\curveto(719.66898868,64.99651611)(719.58398877,65.02151608)(719.5039917,65.05151611)
\curveto(719.46398889,65.07151603)(719.42898892,65.08151602)(719.3989917,65.08151611)
\curveto(719.36898898,65.09151601)(719.33398902,65.106516)(719.2939917,65.12651611)
\curveto(719.14398921,65.19651591)(718.98398937,65.28151582)(718.8139917,65.38151611)
\curveto(718.52398983,65.57151553)(718.27399008,65.8015153)(718.0639917,66.07151611)
\curveto(717.86399049,66.35151475)(717.69399066,66.66151444)(717.5539917,67.00151611)
\curveto(717.50399085,67.11151399)(717.46399089,67.22651388)(717.4339917,67.34651611)
\curveto(717.41399094,67.46651364)(717.38399097,67.58651352)(717.3439917,67.70651611)
\curveto(717.33399102,67.74651336)(717.32899102,67.78151332)(717.3289917,67.81151611)
\curveto(717.32899102,67.84151326)(717.32399103,67.88151322)(717.3139917,67.93151611)
\curveto(717.29399106,68.01151309)(717.27899107,68.09651301)(717.2689917,68.18651611)
\curveto(717.25899109,68.27651283)(717.24399111,68.36651274)(717.2239917,68.45651611)
\lineto(717.2239917,68.66651611)
\curveto(717.21399114,68.7065124)(717.20399115,68.76151234)(717.1939917,68.83151611)
\curveto(717.19399116,68.91151219)(717.19899115,68.97651213)(717.2089917,69.02651611)
\lineto(717.2089917,69.19151611)
\curveto(717.22899112,69.24151186)(717.23399112,69.29151181)(717.2239917,69.34151611)
\curveto(717.22399113,69.4015117)(717.22899112,69.45651165)(717.2389917,69.50651611)
\curveto(717.27899107,69.66651144)(717.30899104,69.82651128)(717.3289917,69.98651611)
\curveto(717.35899099,70.14651096)(717.40399095,70.29651081)(717.4639917,70.43651611)
\curveto(717.51399084,70.54651056)(717.55899079,70.65651045)(717.5989917,70.76651611)
\curveto(717.6489907,70.88651022)(717.70399065,71.0015101)(717.7639917,71.11151611)
\curveto(717.98399037,71.46150964)(718.23399012,71.76150934)(718.5139917,72.01151611)
\curveto(718.79398956,72.27150883)(719.13898921,72.48650862)(719.5489917,72.65651611)
\curveto(719.66898868,72.7065084)(719.78898856,72.74150836)(719.9089917,72.76151611)
\curveto(720.03898831,72.79150831)(720.17398818,72.82150828)(720.3139917,72.85151611)
\curveto(720.36398799,72.86150824)(720.40898794,72.86650824)(720.4489917,72.86651611)
\curveto(720.48898786,72.87650823)(720.53398782,72.88150822)(720.5839917,72.88151611)
\curveto(720.60398775,72.89150821)(720.62898772,72.89150821)(720.6589917,72.88151611)
\curveto(720.68898766,72.87150823)(720.71398764,72.87650823)(720.7339917,72.89651611)
\curveto(721.1539872,72.9065082)(721.51898683,72.86150824)(721.8289917,72.76151611)
\curveto(722.13898621,72.67150843)(722.41898593,72.54650856)(722.6689917,72.38651611)
\curveto(722.71898563,72.36650874)(722.75898559,72.33650877)(722.7889917,72.29651611)
\curveto(722.81898553,72.26650884)(722.8539855,72.24150886)(722.8939917,72.22151611)
\curveto(722.97398538,72.16150894)(723.0539853,72.09150901)(723.1339917,72.01151611)
\curveto(723.22398513,71.93150917)(723.29898505,71.85150925)(723.3589917,71.77151611)
\curveto(723.51898483,71.56150954)(723.6539847,71.36150974)(723.7639917,71.17151611)
\curveto(723.83398452,71.06151004)(723.88898446,70.94151016)(723.9289917,70.81151611)
\curveto(723.96898438,70.68151042)(724.01398434,70.55151055)(724.0639917,70.42151611)
\curveto(724.11398424,70.29151081)(724.1489842,70.15651095)(724.1689917,70.01651611)
\curveto(724.19898415,69.87651123)(724.23398412,69.73651137)(724.2739917,69.59651611)
\curveto(724.28398407,69.52651158)(724.28898406,69.45651165)(724.2889917,69.38651611)
\lineto(724.3189917,69.17651611)
\moveto(722.8639917,69.68651611)
\curveto(722.89398546,69.72651138)(722.91898543,69.77651133)(722.9389917,69.83651611)
\curveto(722.95898539,69.9065112)(722.95898539,69.97651113)(722.9389917,70.04651611)
\curveto(722.87898547,70.26651084)(722.79398556,70.47151063)(722.6839917,70.66151611)
\curveto(722.54398581,70.89151021)(722.38898596,71.08651002)(722.2189917,71.24651611)
\curveto(722.0489863,71.4065097)(721.82898652,71.54150956)(721.5589917,71.65151611)
\curveto(721.48898686,71.67150943)(721.41898693,71.68650942)(721.3489917,71.69651611)
\curveto(721.27898707,71.71650939)(721.20398715,71.73650937)(721.1239917,71.75651611)
\curveto(721.04398731,71.77650933)(720.95898739,71.78650932)(720.8689917,71.78651611)
\lineto(720.6139917,71.78651611)
\curveto(720.58398777,71.76650934)(720.5489878,71.75650935)(720.5089917,71.75651611)
\curveto(720.46898788,71.76650934)(720.43398792,71.76650934)(720.4039917,71.75651611)
\lineto(720.1639917,71.69651611)
\curveto(720.09398826,71.68650942)(720.02398833,71.67150943)(719.9539917,71.65151611)
\curveto(719.66398869,71.53150957)(719.42898892,71.38150972)(719.2489917,71.20151611)
\curveto(719.07898927,71.02151008)(718.92398943,70.79651031)(718.7839917,70.52651611)
\curveto(718.7539896,70.47651063)(718.72398963,70.41151069)(718.6939917,70.33151611)
\curveto(718.66398969,70.26151084)(718.63898971,70.18151092)(718.6189917,70.09151611)
\curveto(718.59898975,70.0015111)(718.59398976,69.91651119)(718.6039917,69.83651611)
\curveto(718.61398974,69.75651135)(718.6489897,69.69651141)(718.7089917,69.65651611)
\curveto(718.78898956,69.59651151)(718.92398943,69.56651154)(719.1139917,69.56651611)
\curveto(719.31398904,69.57651153)(719.48398887,69.58151152)(719.6239917,69.58151611)
\lineto(721.9039917,69.58151611)
\curveto(722.0539863,69.58151152)(722.23398612,69.57651153)(722.4439917,69.56651611)
\curveto(722.6539857,69.56651154)(722.79398556,69.6065115)(722.8639917,69.68651611)
}
}
{
\newrgbcolor{curcolor}{0 0 0}
\pscustom[linestyle=none,fillstyle=solid,fillcolor=curcolor]
{
\newpath
\moveto(728.05563232,72.91151611)
\curveto(728.77562826,72.92150818)(729.38062765,72.83650827)(729.87063232,72.65651611)
\curveto(730.36062667,72.48650862)(730.74062629,72.18150892)(731.01063232,71.74151611)
\curveto(731.08062595,71.63150947)(731.1356259,71.51650959)(731.17563232,71.39651611)
\curveto(731.21562582,71.28650982)(731.25562578,71.16150994)(731.29563232,71.02151611)
\curveto(731.31562572,70.95151015)(731.32062571,70.87651023)(731.31063232,70.79651611)
\curveto(731.30062573,70.72651038)(731.28562575,70.67151043)(731.26563232,70.63151611)
\curveto(731.24562579,70.61151049)(731.22062581,70.59151051)(731.19063232,70.57151611)
\curveto(731.16062587,70.56151054)(731.1356259,70.54651056)(731.11563232,70.52651611)
\curveto(731.06562597,70.5065106)(731.01562602,70.5015106)(730.96563232,70.51151611)
\curveto(730.91562612,70.52151058)(730.86562617,70.52151058)(730.81563232,70.51151611)
\curveto(730.7356263,70.49151061)(730.6306264,70.48651062)(730.50063232,70.49651611)
\curveto(730.37062666,70.51651059)(730.28062675,70.54151056)(730.23063232,70.57151611)
\curveto(730.15062688,70.62151048)(730.09562694,70.68651042)(730.06563232,70.76651611)
\curveto(730.04562699,70.85651025)(730.01062702,70.94151016)(729.96063232,71.02151611)
\curveto(729.87062716,71.18150992)(729.74562729,71.32650978)(729.58563232,71.45651611)
\curveto(729.47562756,71.53650957)(729.35562768,71.59650951)(729.22563232,71.63651611)
\curveto(729.09562794,71.67650943)(728.95562808,71.71650939)(728.80563232,71.75651611)
\curveto(728.75562828,71.77650933)(728.70562833,71.78150932)(728.65563232,71.77151611)
\curveto(728.60562843,71.77150933)(728.55562848,71.77650933)(728.50563232,71.78651611)
\curveto(728.44562859,71.8065093)(728.37062866,71.81650929)(728.28063232,71.81651611)
\curveto(728.19062884,71.81650929)(728.11562892,71.8065093)(728.05563232,71.78651611)
\lineto(727.96563232,71.78651611)
\lineto(727.81563232,71.75651611)
\curveto(727.76562927,71.75650935)(727.71562932,71.75150935)(727.66563232,71.74151611)
\curveto(727.40562963,71.68150942)(727.19062984,71.59650951)(727.02063232,71.48651611)
\curveto(726.85063018,71.37650973)(726.7356303,71.19150991)(726.67563232,70.93151611)
\curveto(726.65563038,70.86151024)(726.65063038,70.79151031)(726.66063232,70.72151611)
\curveto(726.68063035,70.65151045)(726.70063033,70.59151051)(726.72063232,70.54151611)
\curveto(726.78063025,70.39151071)(726.85063018,70.28151082)(726.93063232,70.21151611)
\curveto(727.02063001,70.15151095)(727.1306299,70.08151102)(727.26063232,70.00151611)
\curveto(727.42062961,69.9015112)(727.60062943,69.82651128)(727.80063232,69.77651611)
\curveto(728.00062903,69.73651137)(728.20062883,69.68651142)(728.40063232,69.62651611)
\curveto(728.5306285,69.58651152)(728.66062837,69.55651155)(728.79063232,69.53651611)
\curveto(728.92062811,69.51651159)(729.05062798,69.48651162)(729.18063232,69.44651611)
\curveto(729.39062764,69.38651172)(729.59562744,69.32651178)(729.79563232,69.26651611)
\curveto(729.99562704,69.21651189)(730.19562684,69.15151195)(730.39563232,69.07151611)
\lineto(730.54563232,69.01151611)
\curveto(730.59562644,68.99151211)(730.64562639,68.96651214)(730.69563232,68.93651611)
\curveto(730.89562614,68.81651229)(731.07062596,68.68151242)(731.22063232,68.53151611)
\curveto(731.37062566,68.38151272)(731.49562554,68.19151291)(731.59563232,67.96151611)
\curveto(731.61562542,67.89151321)(731.6356254,67.79651331)(731.65563232,67.67651611)
\curveto(731.67562536,67.6065135)(731.68562535,67.53151357)(731.68563232,67.45151611)
\curveto(731.69562534,67.38151372)(731.70062533,67.3015138)(731.70063232,67.21151611)
\lineto(731.70063232,67.06151611)
\curveto(731.68062535,66.99151411)(731.67062536,66.92151418)(731.67063232,66.85151611)
\curveto(731.67062536,66.78151432)(731.66062537,66.71151439)(731.64063232,66.64151611)
\curveto(731.61062542,66.53151457)(731.57562546,66.42651468)(731.53563232,66.32651611)
\curveto(731.49562554,66.22651488)(731.45062558,66.13651497)(731.40063232,66.05651611)
\curveto(731.24062579,65.79651531)(731.035626,65.58651552)(730.78563232,65.42651611)
\curveto(730.5356265,65.27651583)(730.25562678,65.14651596)(729.94563232,65.03651611)
\curveto(729.85562718,65.0065161)(729.76062727,64.98651612)(729.66063232,64.97651611)
\curveto(729.57062746,64.95651615)(729.48062755,64.93151617)(729.39063232,64.90151611)
\curveto(729.29062774,64.88151622)(729.19062784,64.87151623)(729.09063232,64.87151611)
\curveto(728.99062804,64.87151623)(728.89062814,64.86151624)(728.79063232,64.84151611)
\lineto(728.64063232,64.84151611)
\curveto(728.59062844,64.83151627)(728.52062851,64.82651628)(728.43063232,64.82651611)
\curveto(728.34062869,64.82651628)(728.27062876,64.83151627)(728.22063232,64.84151611)
\lineto(728.05563232,64.84151611)
\curveto(727.99562904,64.86151624)(727.9306291,64.87151623)(727.86063232,64.87151611)
\curveto(727.79062924,64.86151624)(727.7306293,64.86651624)(727.68063232,64.88651611)
\curveto(727.6306294,64.89651621)(727.56562947,64.9015162)(727.48563232,64.90151611)
\lineto(727.24563232,64.96151611)
\curveto(727.17562986,64.97151613)(727.10062993,64.99151611)(727.02063232,65.02151611)
\curveto(726.71063032,65.12151598)(726.44063059,65.24651586)(726.21063232,65.39651611)
\curveto(725.98063105,65.54651556)(725.78063125,65.74151536)(725.61063232,65.98151611)
\curveto(725.52063151,66.11151499)(725.44563159,66.24651486)(725.38563232,66.38651611)
\curveto(725.32563171,66.52651458)(725.27063176,66.68151442)(725.22063232,66.85151611)
\curveto(725.20063183,66.91151419)(725.19063184,66.98151412)(725.19063232,67.06151611)
\curveto(725.20063183,67.15151395)(725.21563182,67.22151388)(725.23563232,67.27151611)
\curveto(725.26563177,67.31151379)(725.31563172,67.35151375)(725.38563232,67.39151611)
\curveto(725.4356316,67.41151369)(725.50563153,67.42151368)(725.59563232,67.42151611)
\curveto(725.68563135,67.43151367)(725.77563126,67.43151367)(725.86563232,67.42151611)
\curveto(725.95563108,67.41151369)(726.04063099,67.39651371)(726.12063232,67.37651611)
\curveto(726.21063082,67.36651374)(726.27063076,67.35151375)(726.30063232,67.33151611)
\curveto(726.37063066,67.28151382)(726.41563062,67.2065139)(726.43563232,67.10651611)
\curveto(726.46563057,67.01651409)(726.50063053,66.93151417)(726.54063232,66.85151611)
\curveto(726.64063039,66.63151447)(726.77563026,66.46151464)(726.94563232,66.34151611)
\curveto(727.06562997,66.25151485)(727.20062983,66.18151492)(727.35063232,66.13151611)
\curveto(727.50062953,66.08151502)(727.66062937,66.03151507)(727.83063232,65.98151611)
\lineto(728.14563232,65.93651611)
\lineto(728.23563232,65.93651611)
\curveto(728.30562873,65.91651519)(728.39562864,65.9065152)(728.50563232,65.90651611)
\curveto(728.62562841,65.9065152)(728.72562831,65.91651519)(728.80563232,65.93651611)
\curveto(728.87562816,65.93651517)(728.9306281,65.94151516)(728.97063232,65.95151611)
\curveto(729.030628,65.96151514)(729.09062794,65.96651514)(729.15063232,65.96651611)
\curveto(729.21062782,65.97651513)(729.26562777,65.98651512)(729.31563232,65.99651611)
\curveto(729.60562743,66.07651503)(729.8356272,66.18151492)(730.00563232,66.31151611)
\curveto(730.17562686,66.44151466)(730.29562674,66.66151444)(730.36563232,66.97151611)
\curveto(730.38562665,67.02151408)(730.39062664,67.07651403)(730.38063232,67.13651611)
\curveto(730.37062666,67.19651391)(730.36062667,67.24151386)(730.35063232,67.27151611)
\curveto(730.30062673,67.46151364)(730.2306268,67.6015135)(730.14063232,67.69151611)
\curveto(730.05062698,67.79151331)(729.9356271,67.88151322)(729.79563232,67.96151611)
\curveto(729.70562733,68.02151308)(729.60562743,68.07151303)(729.49563232,68.11151611)
\lineto(729.16563232,68.23151611)
\curveto(729.1356279,68.24151286)(729.10562793,68.24651286)(729.07563232,68.24651611)
\curveto(729.05562798,68.24651286)(729.030628,68.25651285)(729.00063232,68.27651611)
\curveto(728.66062837,68.38651272)(728.30562873,68.46651264)(727.93563232,68.51651611)
\curveto(727.57562946,68.57651253)(727.2356298,68.67151243)(726.91563232,68.80151611)
\curveto(726.81563022,68.84151226)(726.72063031,68.87651223)(726.63063232,68.90651611)
\curveto(726.54063049,68.93651217)(726.45563058,68.97651213)(726.37563232,69.02651611)
\curveto(726.18563085,69.13651197)(726.01063102,69.26151184)(725.85063232,69.40151611)
\curveto(725.69063134,69.54151156)(725.56563147,69.71651139)(725.47563232,69.92651611)
\curveto(725.44563159,69.99651111)(725.42063161,70.06651104)(725.40063232,70.13651611)
\curveto(725.39063164,70.2065109)(725.37563166,70.28151082)(725.35563232,70.36151611)
\curveto(725.32563171,70.48151062)(725.31563172,70.61651049)(725.32563232,70.76651611)
\curveto(725.3356317,70.92651018)(725.35063168,71.06151004)(725.37063232,71.17151611)
\curveto(725.39063164,71.22150988)(725.40063163,71.26150984)(725.40063232,71.29151611)
\curveto(725.41063162,71.33150977)(725.42563161,71.37150973)(725.44563232,71.41151611)
\curveto(725.5356315,71.64150946)(725.65563138,71.84150926)(725.80563232,72.01151611)
\curveto(725.96563107,72.18150892)(726.14563089,72.33150877)(726.34563232,72.46151611)
\curveto(726.49563054,72.55150855)(726.66063037,72.62150848)(726.84063232,72.67151611)
\curveto(727.02063001,72.73150837)(727.21062982,72.78650832)(727.41063232,72.83651611)
\curveto(727.48062955,72.84650826)(727.54562949,72.85650825)(727.60563232,72.86651611)
\curveto(727.67562936,72.87650823)(727.75062928,72.88650822)(727.83063232,72.89651611)
\curveto(727.86062917,72.9065082)(727.90062913,72.9065082)(727.95063232,72.89651611)
\curveto(728.00062903,72.88650822)(728.035629,72.89150821)(728.05563232,72.91151611)
}
}
{
\newrgbcolor{curcolor}{0 0 0}
\pscustom[linestyle=none,fillstyle=solid,fillcolor=curcolor]
{
\newpath
\moveto(203.55070068,57.435)
\curveto(204.53069418,57.45498904)(205.35069336,57.2949892)(206.01070068,56.955)
\curveto(206.68069203,56.62498987)(207.20069151,56.16499033)(207.57070068,55.575)
\curveto(207.67069104,55.41499108)(207.75069096,55.25999124)(207.81070068,55.11)
\curveto(207.88069083,54.96999153)(207.94569077,54.7999917)(208.00570068,54.6)
\curveto(208.02569069,54.54999195)(208.04569067,54.47999202)(208.06570068,54.39)
\curveto(208.08569063,54.30999219)(208.08069063,54.23499227)(208.05070068,54.165)
\curveto(208.03069068,54.1049924)(207.99069072,54.06499244)(207.93070068,54.045)
\curveto(207.88069083,54.03499247)(207.82569089,54.01999248)(207.76570068,54)
\lineto(207.61570068,54)
\curveto(207.58569113,53.98999251)(207.54569117,53.98499251)(207.49570068,53.985)
\lineto(207.37570068,53.985)
\curveto(207.23569148,53.98499251)(207.10569161,53.98999251)(206.98570068,54)
\curveto(206.87569184,54.01999248)(206.79569192,54.06999243)(206.74570068,54.15)
\curveto(206.67569204,54.24999225)(206.62069209,54.36499214)(206.58070068,54.495)
\curveto(206.54069217,54.62499187)(206.48569223,54.74499176)(206.41570068,54.855)
\curveto(206.28569243,55.07499143)(206.13569258,55.26499124)(205.96570068,55.425)
\curveto(205.80569291,55.58499091)(205.6156931,55.73499077)(205.39570068,55.875)
\curveto(205.27569344,55.95499054)(205.14069357,56.01499048)(204.99070068,56.055)
\curveto(204.85069386,56.0949904)(204.70569401,56.13499037)(204.55570068,56.175)
\curveto(204.44569427,56.2049903)(204.32069439,56.22499027)(204.18070068,56.235)
\curveto(204.04069467,56.25499024)(203.89069482,56.26499024)(203.73070068,56.265)
\curveto(203.58069513,56.26499024)(203.43069528,56.25499024)(203.28070068,56.235)
\curveto(203.14069557,56.22499027)(203.02069569,56.2049903)(202.92070068,56.175)
\curveto(202.82069589,56.15499034)(202.72569599,56.13499037)(202.63570068,56.115)
\curveto(202.54569617,56.0949904)(202.45569626,56.06499044)(202.36570068,56.025)
\curveto(201.52569719,55.67499083)(200.92069779,55.07499143)(200.55070068,54.225)
\curveto(200.48069823,54.08499241)(200.42069829,53.93499257)(200.37070068,53.775)
\curveto(200.33069838,53.62499287)(200.28569843,53.46999303)(200.23570068,53.31)
\curveto(200.2156985,53.24999325)(200.20569851,53.18499331)(200.20570068,53.115)
\curveto(200.20569851,53.05499344)(200.19569852,52.99499351)(200.17570068,52.935)
\curveto(200.16569855,52.89499361)(200.16069855,52.85999364)(200.16070068,52.83)
\curveto(200.16069855,52.7999937)(200.15569856,52.76499374)(200.14570068,52.725)
\curveto(200.12569859,52.61499388)(200.1106986,52.499994)(200.10070068,52.38)
\lineto(200.10070068,52.035)
\curveto(200.10069861,51.96499454)(200.09569862,51.88999461)(200.08570068,51.81)
\curveto(200.08569863,51.73999476)(200.09069862,51.67499482)(200.10070068,51.615)
\lineto(200.10070068,51.465)
\curveto(200.12069859,51.39499511)(200.12569859,51.32499518)(200.11570068,51.255)
\curveto(200.1156986,51.18499531)(200.12569859,51.11499538)(200.14570068,51.045)
\curveto(200.16569855,50.98499551)(200.17069854,50.92499558)(200.16070068,50.865)
\curveto(200.16069855,50.8049957)(200.17069854,50.74999575)(200.19070068,50.7)
\curveto(200.22069849,50.56999593)(200.24569847,50.43999606)(200.26570068,50.31)
\curveto(200.29569842,50.18999631)(200.33069838,50.06999643)(200.37070068,49.95)
\curveto(200.54069817,49.44999705)(200.76069795,49.01999748)(201.03070068,48.66)
\curveto(201.30069741,48.30999819)(201.65569706,48.01999848)(202.09570068,47.79)
\curveto(202.23569648,47.71999878)(202.37569634,47.66499884)(202.51570068,47.625)
\curveto(202.66569605,47.58499892)(202.82569589,47.53999896)(202.99570068,47.49)
\curveto(203.06569565,47.46999903)(203.13069558,47.45999904)(203.19070068,47.46)
\curveto(203.25069546,47.46999903)(203.32069539,47.46499904)(203.40070068,47.445)
\curveto(203.45069526,47.43499906)(203.54069517,47.42499908)(203.67070068,47.415)
\curveto(203.80069491,47.41499908)(203.89569482,47.42499908)(203.95570068,47.445)
\lineto(204.06070068,47.445)
\curveto(204.10069461,47.45499905)(204.14069457,47.45499905)(204.18070068,47.445)
\curveto(204.22069449,47.44499905)(204.26069445,47.45499905)(204.30070068,47.475)
\curveto(204.40069431,47.49499901)(204.49569422,47.50999899)(204.58570068,47.52)
\curveto(204.68569403,47.53999896)(204.78069393,47.56999893)(204.87070068,47.61)
\curveto(205.65069306,47.92999857)(206.20069251,48.45499805)(206.52070068,49.185)
\curveto(206.60069211,49.36499714)(206.67569204,49.57999692)(206.74570068,49.83)
\curveto(206.76569195,49.91999658)(206.78069193,50.00999649)(206.79070068,50.1)
\curveto(206.8106919,50.1999963)(206.84569187,50.28999621)(206.89570068,50.37)
\curveto(206.94569177,50.44999605)(207.02569169,50.49499601)(207.13570068,50.505)
\curveto(207.24569147,50.51499598)(207.36569135,50.51999598)(207.49570068,50.52)
\lineto(207.64570068,50.52)
\curveto(207.69569102,50.51999598)(207.74069097,50.51499598)(207.78070068,50.505)
\lineto(207.88570068,50.505)
\lineto(207.97570068,50.475)
\curveto(208.0156907,50.47499602)(208.04569067,50.46499604)(208.06570068,50.445)
\curveto(208.13569058,50.4049961)(208.17569054,50.32999617)(208.18570068,50.22)
\curveto(208.19569052,50.11999638)(208.18569053,50.01999648)(208.15570068,49.92)
\curveto(208.09569062,49.68999681)(208.04069067,49.46999703)(207.99070068,49.26)
\curveto(207.94069077,49.04999745)(207.86569085,48.84999765)(207.76570068,48.66)
\curveto(207.68569103,48.52999797)(207.6106911,48.40499809)(207.54070068,48.285)
\curveto(207.48069123,48.16499834)(207.4106913,48.04499845)(207.33070068,47.925)
\curveto(207.15069156,47.66499884)(206.92569179,47.42499908)(206.65570068,47.205)
\curveto(206.39569232,46.99499951)(206.1106926,46.81999968)(205.80070068,46.68)
\curveto(205.69069302,46.62999987)(205.58069313,46.58999991)(205.47070068,46.56)
\curveto(205.37069334,46.52999997)(205.26569345,46.49500001)(205.15570068,46.455)
\curveto(205.04569367,46.41500008)(204.93069378,46.39000011)(204.81070068,46.38)
\curveto(204.70069401,46.36000014)(204.58569413,46.34000016)(204.46570068,46.32)
\curveto(204.4156943,46.3000002)(204.37069434,46.29500021)(204.33070068,46.305)
\curveto(204.29069442,46.30500019)(204.25069446,46.3000002)(204.21070068,46.29)
\curveto(204.15069456,46.28000022)(204.09069462,46.27500022)(204.03070068,46.275)
\curveto(203.97069474,46.27500022)(203.90569481,46.27000023)(203.83570068,46.26)
\curveto(203.80569491,46.25000025)(203.73569498,46.25000025)(203.62570068,46.26)
\curveto(203.52569519,46.26000024)(203.46069525,46.26500024)(203.43070068,46.275)
\curveto(203.38069533,46.28500022)(203.33069538,46.29000021)(203.28070068,46.29)
\curveto(203.24069547,46.28000022)(203.19569552,46.28000022)(203.14570068,46.29)
\lineto(202.99570068,46.29)
\curveto(202.9156958,46.31000019)(202.84069587,46.32500018)(202.77070068,46.335)
\curveto(202.70069601,46.33500016)(202.62569609,46.34500015)(202.54570068,46.365)
\lineto(202.27570068,46.425)
\curveto(202.18569653,46.43500006)(202.10069661,46.45500005)(202.02070068,46.485)
\curveto(201.8106969,46.54499995)(201.62069709,46.61999988)(201.45070068,46.71)
\curveto(200.82069789,46.97999952)(200.3106984,47.36499914)(199.92070068,47.865)
\curveto(199.53069918,48.36499814)(199.22069949,48.95499755)(198.99070068,49.635)
\curveto(198.95069976,49.75499675)(198.9156998,49.87999662)(198.88570068,50.01)
\curveto(198.86569985,50.13999636)(198.84069987,50.27499622)(198.81070068,50.415)
\curveto(198.79069992,50.46499604)(198.78069993,50.50999599)(198.78070068,50.55)
\curveto(198.79069992,50.58999591)(198.79069992,50.63499587)(198.78070068,50.685)
\curveto(198.76069995,50.77499572)(198.74569997,50.86999563)(198.73570068,50.97)
\curveto(198.73569998,51.06999543)(198.72569999,51.16499534)(198.70570068,51.255)
\lineto(198.70570068,51.54)
\curveto(198.68570003,51.58999491)(198.67570004,51.67499482)(198.67570068,51.795)
\curveto(198.67570004,51.91499458)(198.68570003,51.9999945)(198.70570068,52.05)
\curveto(198.7157,52.07999442)(198.7157,52.10999439)(198.70570068,52.14)
\curveto(198.69570002,52.17999432)(198.69570002,52.20999429)(198.70570068,52.23)
\lineto(198.70570068,52.365)
\curveto(198.7157,52.44499405)(198.72069999,52.52499397)(198.72070068,52.605)
\curveto(198.73069998,52.69499381)(198.74569997,52.77999372)(198.76570068,52.86)
\curveto(198.78569993,52.91999358)(198.79569992,52.97999352)(198.79570068,53.04)
\curveto(198.79569992,53.10999339)(198.80569991,53.17999332)(198.82570068,53.25)
\curveto(198.87569984,53.41999308)(198.9156998,53.58499291)(198.94570068,53.745)
\curveto(198.97569974,53.9049926)(199.02069969,54.05499244)(199.08070068,54.195)
\lineto(199.23070068,54.585)
\curveto(199.29069942,54.72499177)(199.35569936,54.84999165)(199.42570068,54.96)
\curveto(199.57569914,55.21999128)(199.72569899,55.45499104)(199.87570068,55.665)
\curveto(199.90569881,55.71499078)(199.94069877,55.75499074)(199.98070068,55.785)
\curveto(200.03069868,55.82499067)(200.07069864,55.86999063)(200.10070068,55.92)
\curveto(200.16069855,55.9999905)(200.22069849,56.06999043)(200.28070068,56.13)
\lineto(200.49070068,56.31)
\curveto(200.55069816,56.35999014)(200.60569811,56.4049901)(200.65570068,56.445)
\curveto(200.715698,56.49499)(200.78069793,56.54498996)(200.85070068,56.595)
\curveto(201.00069771,56.7049898)(201.15569756,56.7999897)(201.31570068,56.88)
\curveto(201.48569723,56.95998954)(201.66069705,57.03998946)(201.84070068,57.12)
\curveto(201.95069676,57.16998933)(202.06569665,57.2049893)(202.18570068,57.225)
\curveto(202.3156964,57.25498924)(202.44069627,57.28998921)(202.56070068,57.33)
\curveto(202.63069608,57.33998916)(202.69569602,57.34998915)(202.75570068,57.36)
\lineto(202.93570068,57.39)
\curveto(203.0156957,57.3999891)(203.09069562,57.4049891)(203.16070068,57.405)
\curveto(203.24069547,57.41498909)(203.32069539,57.42498907)(203.40070068,57.435)
\curveto(203.42069529,57.44498906)(203.44569527,57.44498906)(203.47570068,57.435)
\curveto(203.50569521,57.42498907)(203.53069518,57.42498907)(203.55070068,57.435)
}
}
{
\newrgbcolor{curcolor}{0 0 0}
\pscustom[linestyle=none,fillstyle=solid,fillcolor=curcolor]
{
\newpath
\moveto(216.67054443,47.07)
\curveto(216.7005366,46.90999959)(216.68553662,46.77499972)(216.62554443,46.665)
\curveto(216.56553674,46.56499994)(216.48553682,46.49000001)(216.38554443,46.44)
\curveto(216.33553697,46.42000008)(216.28053702,46.41000009)(216.22054443,46.41)
\curveto(216.17053713,46.41000009)(216.11553719,46.4000001)(216.05554443,46.38)
\curveto(215.83553747,46.33000017)(215.61553769,46.34500015)(215.39554443,46.425)
\curveto(215.18553812,46.49500001)(215.04053826,46.58499992)(214.96054443,46.695)
\curveto(214.91053839,46.76499974)(214.86553844,46.84499965)(214.82554443,46.935)
\curveto(214.78553852,47.03499946)(214.73553857,47.11499938)(214.67554443,47.175)
\curveto(214.65553865,47.19499931)(214.63053867,47.21499928)(214.60054443,47.235)
\curveto(214.58053872,47.25499925)(214.55053875,47.25999924)(214.51054443,47.25)
\curveto(214.4005389,47.21999928)(214.29553901,47.16499934)(214.19554443,47.085)
\curveto(214.1055392,47.00499949)(214.01553929,46.93499956)(213.92554443,46.875)
\curveto(213.79553951,46.79499971)(213.65553965,46.71999978)(213.50554443,46.65)
\curveto(213.35553995,46.58999991)(213.19554011,46.53499996)(213.02554443,46.485)
\curveto(212.92554038,46.45500005)(212.81554049,46.43500006)(212.69554443,46.425)
\curveto(212.58554072,46.41500008)(212.47554083,46.4000001)(212.36554443,46.38)
\curveto(212.31554099,46.37000013)(212.27054103,46.36500014)(212.23054443,46.365)
\lineto(212.12554443,46.365)
\curveto(212.01554129,46.34500015)(211.91054139,46.34500015)(211.81054443,46.365)
\lineto(211.67554443,46.365)
\curveto(211.62554168,46.37500012)(211.57554173,46.38000012)(211.52554443,46.38)
\curveto(211.47554183,46.38000012)(211.43054187,46.39000011)(211.39054443,46.41)
\curveto(211.35054195,46.42000008)(211.31554199,46.42500008)(211.28554443,46.425)
\curveto(211.26554204,46.41500008)(211.24054206,46.41500008)(211.21054443,46.425)
\lineto(210.97054443,46.485)
\curveto(210.89054241,46.49500001)(210.81554249,46.51499998)(210.74554443,46.545)
\curveto(210.44554286,46.67499982)(210.2005431,46.81999968)(210.01054443,46.98)
\curveto(209.83054347,47.14999935)(209.68054362,47.38499912)(209.56054443,47.685)
\curveto(209.47054383,47.90499859)(209.42554388,48.16999833)(209.42554443,48.48)
\lineto(209.42554443,48.795)
\curveto(209.43554387,48.84499765)(209.44054386,48.89499761)(209.44054443,48.945)
\lineto(209.47054443,49.125)
\lineto(209.59054443,49.455)
\curveto(209.63054367,49.56499694)(209.68054362,49.66499684)(209.74054443,49.755)
\curveto(209.92054338,50.04499645)(210.16554314,50.25999624)(210.47554443,50.4)
\curveto(210.78554252,50.53999596)(211.12554218,50.66499584)(211.49554443,50.775)
\curveto(211.63554167,50.81499568)(211.78054152,50.84499565)(211.93054443,50.865)
\curveto(212.08054122,50.88499561)(212.23054107,50.90999559)(212.38054443,50.94)
\curveto(212.45054085,50.95999554)(212.51554079,50.96999553)(212.57554443,50.97)
\curveto(212.64554066,50.96999553)(212.72054058,50.97999552)(212.80054443,51)
\curveto(212.87054043,51.01999548)(212.94054036,51.02999547)(213.01054443,51.03)
\curveto(213.08054022,51.03999546)(213.15554015,51.05499544)(213.23554443,51.075)
\curveto(213.48553982,51.13499537)(213.72053958,51.18499531)(213.94054443,51.225)
\curveto(214.16053914,51.27499522)(214.33553897,51.38999511)(214.46554443,51.57)
\curveto(214.52553878,51.64999485)(214.57553873,51.74999475)(214.61554443,51.87)
\curveto(214.65553865,51.9999945)(214.65553865,52.13999436)(214.61554443,52.29)
\curveto(214.55553875,52.52999397)(214.46553884,52.71999378)(214.34554443,52.86)
\curveto(214.23553907,52.9999935)(214.07553923,53.10999339)(213.86554443,53.19)
\curveto(213.74553956,53.23999326)(213.6005397,53.27499323)(213.43054443,53.295)
\curveto(213.27054003,53.31499318)(213.1005402,53.32499317)(212.92054443,53.325)
\curveto(212.74054056,53.32499317)(212.56554074,53.31499318)(212.39554443,53.295)
\curveto(212.22554108,53.27499323)(212.08054122,53.24499325)(211.96054443,53.205)
\curveto(211.79054151,53.14499335)(211.62554168,53.05999344)(211.46554443,52.95)
\curveto(211.38554192,52.88999361)(211.31054199,52.80999369)(211.24054443,52.71)
\curveto(211.18054212,52.61999388)(211.12554218,52.51999398)(211.07554443,52.41)
\curveto(211.04554226,52.32999417)(211.01554229,52.24499425)(210.98554443,52.155)
\curveto(210.96554234,52.06499444)(210.92054238,51.99499451)(210.85054443,51.945)
\curveto(210.81054249,51.91499458)(210.74054256,51.88999461)(210.64054443,51.87)
\curveto(210.55054275,51.85999464)(210.45554285,51.85499464)(210.35554443,51.855)
\curveto(210.25554305,51.85499464)(210.15554315,51.85999464)(210.05554443,51.87)
\curveto(209.96554334,51.88999461)(209.9005434,51.91499458)(209.86054443,51.945)
\curveto(209.82054348,51.97499452)(209.79054351,52.02499447)(209.77054443,52.095)
\curveto(209.75054355,52.16499434)(209.75054355,52.23999426)(209.77054443,52.32)
\curveto(209.8005435,52.44999405)(209.83054347,52.56999393)(209.86054443,52.68)
\curveto(209.9005434,52.7999937)(209.94554336,52.91499358)(209.99554443,53.025)
\curveto(210.18554312,53.37499313)(210.42554288,53.64499285)(210.71554443,53.835)
\curveto(211.0055423,54.03499247)(211.36554194,54.19499231)(211.79554443,54.315)
\curveto(211.89554141,54.33499217)(211.99554131,54.34999215)(212.09554443,54.36)
\curveto(212.2055411,54.36999213)(212.31554099,54.38499211)(212.42554443,54.405)
\curveto(212.46554084,54.41499208)(212.53054077,54.41499208)(212.62054443,54.405)
\curveto(212.71054059,54.4049921)(212.76554054,54.41499208)(212.78554443,54.435)
\curveto(213.48553982,54.44499206)(214.09553921,54.36499214)(214.61554443,54.195)
\curveto(215.13553817,54.02499247)(215.5005378,53.6999928)(215.71054443,53.22)
\curveto(215.8005375,53.01999348)(215.85053745,52.78499371)(215.86054443,52.515)
\curveto(215.88053742,52.25499424)(215.89053741,51.97999452)(215.89054443,51.69)
\lineto(215.89054443,48.375)
\curveto(215.89053741,48.23499827)(215.89553741,48.0999984)(215.90554443,47.97)
\curveto(215.91553739,47.83999866)(215.94553736,47.73499877)(215.99554443,47.655)
\curveto(216.04553726,47.58499892)(216.11053719,47.53499896)(216.19054443,47.505)
\curveto(216.28053702,47.46499904)(216.36553694,47.43499906)(216.44554443,47.415)
\curveto(216.52553678,47.40499909)(216.58553672,47.35999914)(216.62554443,47.28)
\curveto(216.64553666,47.24999925)(216.65553665,47.21999928)(216.65554443,47.19)
\curveto(216.65553665,47.15999934)(216.66053664,47.11999938)(216.67054443,47.07)
\moveto(214.52554443,48.735)
\curveto(214.58553872,48.87499762)(214.61553869,49.03499747)(214.61554443,49.215)
\curveto(214.62553868,49.40499709)(214.63053867,49.5999969)(214.63054443,49.8)
\curveto(214.63053867,49.90999659)(214.62553868,50.00999649)(214.61554443,50.1)
\curveto(214.6055387,50.18999631)(214.56553874,50.25999624)(214.49554443,50.31)
\curveto(214.46553884,50.32999617)(214.39553891,50.33999616)(214.28554443,50.34)
\curveto(214.26553904,50.31999618)(214.23053907,50.30999619)(214.18054443,50.31)
\curveto(214.13053917,50.30999619)(214.08553922,50.2999962)(214.04554443,50.28)
\curveto(213.96553934,50.25999624)(213.87553943,50.23999626)(213.77554443,50.22)
\lineto(213.47554443,50.16)
\curveto(213.44553986,50.15999634)(213.41053989,50.15499634)(213.37054443,50.145)
\lineto(213.26554443,50.145)
\curveto(213.11554019,50.1049964)(212.95054035,50.07999642)(212.77054443,50.07)
\curveto(212.6005407,50.06999643)(212.44054086,50.04999645)(212.29054443,50.01)
\curveto(212.21054109,49.98999651)(212.13554117,49.96999653)(212.06554443,49.95)
\curveto(212.0055413,49.93999656)(211.93554137,49.92499658)(211.85554443,49.905)
\curveto(211.69554161,49.85499665)(211.54554176,49.78999671)(211.40554443,49.71)
\curveto(211.26554204,49.63999686)(211.14554216,49.54999695)(211.04554443,49.44)
\curveto(210.94554236,49.32999717)(210.87054243,49.19499731)(210.82054443,49.035)
\curveto(210.77054253,48.88499761)(210.75054255,48.6999978)(210.76054443,48.48)
\curveto(210.76054254,48.37999812)(210.77554253,48.28499821)(210.80554443,48.195)
\curveto(210.84554246,48.11499838)(210.89054241,48.03999846)(210.94054443,47.97)
\curveto(211.02054228,47.85999864)(211.12554218,47.76499874)(211.25554443,47.685)
\curveto(211.38554192,47.61499888)(211.52554178,47.55499895)(211.67554443,47.505)
\curveto(211.72554158,47.49499901)(211.77554153,47.48999901)(211.82554443,47.49)
\curveto(211.87554143,47.48999901)(211.92554138,47.48499902)(211.97554443,47.475)
\curveto(212.04554126,47.45499905)(212.13054117,47.43999906)(212.23054443,47.43)
\curveto(212.34054096,47.42999907)(212.43054087,47.43999906)(212.50054443,47.46)
\curveto(212.56054074,47.47999902)(212.62054068,47.48499902)(212.68054443,47.475)
\curveto(212.74054056,47.47499902)(212.8005405,47.48499902)(212.86054443,47.505)
\curveto(212.94054036,47.52499898)(213.01554029,47.53999896)(213.08554443,47.55)
\curveto(213.16554014,47.55999894)(213.24054006,47.57999892)(213.31054443,47.61)
\curveto(213.6005397,47.72999877)(213.84553946,47.87499862)(214.04554443,48.045)
\curveto(214.25553905,48.21499828)(214.41553889,48.44499805)(214.52554443,48.735)
}
}
{
\newrgbcolor{curcolor}{0 0 0}
\pscustom[linestyle=none,fillstyle=solid,fillcolor=curcolor]
{
\newpath
\moveto(221.48718506,54.42)
\curveto(221.71718027,54.41999208)(221.84718014,54.35999214)(221.87718506,54.24)
\curveto(221.90718008,54.12999237)(221.92218006,53.96499254)(221.92218506,53.745)
\lineto(221.92218506,53.46)
\curveto(221.92218006,53.36999313)(221.89718009,53.29499321)(221.84718506,53.235)
\curveto(221.7871802,53.15499334)(221.70218028,53.10999339)(221.59218506,53.1)
\curveto(221.4821805,53.0999934)(221.37218061,53.08499341)(221.26218506,53.055)
\curveto(221.12218086,53.02499347)(220.987181,52.99499351)(220.85718506,52.965)
\curveto(220.73718125,52.93499357)(220.62218136,52.89499361)(220.51218506,52.845)
\curveto(220.22218176,52.71499378)(219.987182,52.53499397)(219.80718506,52.305)
\curveto(219.62718236,52.08499441)(219.47218251,51.82999467)(219.34218506,51.54)
\curveto(219.30218268,51.42999507)(219.27218271,51.31499518)(219.25218506,51.195)
\curveto(219.23218275,51.08499541)(219.20718278,50.96999553)(219.17718506,50.85)
\curveto(219.16718282,50.7999957)(219.16218282,50.74999575)(219.16218506,50.7)
\curveto(219.17218281,50.64999585)(219.17218281,50.5999959)(219.16218506,50.55)
\curveto(219.13218285,50.42999607)(219.11718287,50.28999621)(219.11718506,50.13)
\curveto(219.12718286,49.97999652)(219.13218285,49.83499667)(219.13218506,49.695)
\lineto(219.13218506,47.85)
\lineto(219.13218506,47.505)
\curveto(219.13218285,47.38499912)(219.12718286,47.26999923)(219.11718506,47.16)
\curveto(219.10718288,47.04999945)(219.10218288,46.95499955)(219.10218506,46.875)
\curveto(219.11218287,46.79499971)(219.09218289,46.72499978)(219.04218506,46.665)
\curveto(218.99218299,46.59499991)(218.91218307,46.55499995)(218.80218506,46.545)
\curveto(218.70218328,46.53499996)(218.59218339,46.52999997)(218.47218506,46.53)
\lineto(218.20218506,46.53)
\curveto(218.15218383,46.54999995)(218.10218388,46.56499994)(218.05218506,46.575)
\curveto(218.01218397,46.59499991)(217.982184,46.61999988)(217.96218506,46.65)
\curveto(217.91218407,46.71999978)(217.8821841,46.80499969)(217.87218506,46.905)
\lineto(217.87218506,47.235)
\lineto(217.87218506,48.39)
\lineto(217.87218506,52.545)
\lineto(217.87218506,53.58)
\lineto(217.87218506,53.88)
\curveto(217.8821841,53.97999252)(217.91218407,54.06499244)(217.96218506,54.135)
\curveto(217.99218399,54.17499233)(218.04218394,54.2049923)(218.11218506,54.225)
\curveto(218.19218379,54.24499225)(218.27718371,54.25499224)(218.36718506,54.255)
\curveto(218.45718353,54.26499224)(218.54718344,54.26499224)(218.63718506,54.255)
\curveto(218.72718326,54.24499225)(218.79718319,54.22999227)(218.84718506,54.21)
\curveto(218.92718306,54.17999232)(218.97718301,54.11999238)(218.99718506,54.03)
\curveto(219.02718296,53.94999255)(219.04218294,53.85999264)(219.04218506,53.76)
\lineto(219.04218506,53.46)
\curveto(219.04218294,53.35999314)(219.06218292,53.26999323)(219.10218506,53.19)
\curveto(219.11218287,53.16999333)(219.12218286,53.15499334)(219.13218506,53.145)
\lineto(219.17718506,53.1)
\curveto(219.2871827,53.0999934)(219.37718261,53.14499335)(219.44718506,53.235)
\curveto(219.51718247,53.33499317)(219.57718241,53.41499308)(219.62718506,53.475)
\lineto(219.71718506,53.565)
\curveto(219.80718218,53.67499283)(219.93218205,53.78999271)(220.09218506,53.91)
\curveto(220.25218173,54.02999247)(220.40218158,54.11999238)(220.54218506,54.18)
\curveto(220.63218135,54.22999227)(220.72718126,54.26499224)(220.82718506,54.285)
\curveto(220.92718106,54.31499218)(221.03218095,54.34499215)(221.14218506,54.375)
\curveto(221.20218078,54.38499211)(221.26218072,54.38999211)(221.32218506,54.39)
\curveto(221.3821806,54.3999921)(221.43718055,54.40999209)(221.48718506,54.42)
}
}
{
\newrgbcolor{curcolor}{0 0 0}
\pscustom[linestyle=none,fillstyle=solid,fillcolor=curcolor]
{
\newpath
\moveto(226.49695068,54.42)
\curveto(226.72694589,54.41999208)(226.85694576,54.35999214)(226.88695068,54.24)
\curveto(226.9169457,54.12999237)(226.93194569,53.96499254)(226.93195068,53.745)
\lineto(226.93195068,53.46)
\curveto(226.93194569,53.36999313)(226.90694571,53.29499321)(226.85695068,53.235)
\curveto(226.79694582,53.15499334)(226.71194591,53.10999339)(226.60195068,53.1)
\curveto(226.49194613,53.0999934)(226.38194624,53.08499341)(226.27195068,53.055)
\curveto(226.13194649,53.02499347)(225.99694662,52.99499351)(225.86695068,52.965)
\curveto(225.74694687,52.93499357)(225.63194699,52.89499361)(225.52195068,52.845)
\curveto(225.23194739,52.71499378)(224.99694762,52.53499397)(224.81695068,52.305)
\curveto(224.63694798,52.08499441)(224.48194814,51.82999467)(224.35195068,51.54)
\curveto(224.31194831,51.42999507)(224.28194834,51.31499518)(224.26195068,51.195)
\curveto(224.24194838,51.08499541)(224.2169484,50.96999553)(224.18695068,50.85)
\curveto(224.17694844,50.7999957)(224.17194845,50.74999575)(224.17195068,50.7)
\curveto(224.18194844,50.64999585)(224.18194844,50.5999959)(224.17195068,50.55)
\curveto(224.14194848,50.42999607)(224.12694849,50.28999621)(224.12695068,50.13)
\curveto(224.13694848,49.97999652)(224.14194848,49.83499667)(224.14195068,49.695)
\lineto(224.14195068,47.85)
\lineto(224.14195068,47.505)
\curveto(224.14194848,47.38499912)(224.13694848,47.26999923)(224.12695068,47.16)
\curveto(224.1169485,47.04999945)(224.11194851,46.95499955)(224.11195068,46.875)
\curveto(224.1219485,46.79499971)(224.10194852,46.72499978)(224.05195068,46.665)
\curveto(224.00194862,46.59499991)(223.9219487,46.55499995)(223.81195068,46.545)
\curveto(223.71194891,46.53499996)(223.60194902,46.52999997)(223.48195068,46.53)
\lineto(223.21195068,46.53)
\curveto(223.16194946,46.54999995)(223.11194951,46.56499994)(223.06195068,46.575)
\curveto(223.0219496,46.59499991)(222.99194963,46.61999988)(222.97195068,46.65)
\curveto(222.9219497,46.71999978)(222.89194973,46.80499969)(222.88195068,46.905)
\lineto(222.88195068,47.235)
\lineto(222.88195068,48.39)
\lineto(222.88195068,52.545)
\lineto(222.88195068,53.58)
\lineto(222.88195068,53.88)
\curveto(222.89194973,53.97999252)(222.9219497,54.06499244)(222.97195068,54.135)
\curveto(223.00194962,54.17499233)(223.05194957,54.2049923)(223.12195068,54.225)
\curveto(223.20194942,54.24499225)(223.28694933,54.25499224)(223.37695068,54.255)
\curveto(223.46694915,54.26499224)(223.55694906,54.26499224)(223.64695068,54.255)
\curveto(223.73694888,54.24499225)(223.80694881,54.22999227)(223.85695068,54.21)
\curveto(223.93694868,54.17999232)(223.98694863,54.11999238)(224.00695068,54.03)
\curveto(224.03694858,53.94999255)(224.05194857,53.85999264)(224.05195068,53.76)
\lineto(224.05195068,53.46)
\curveto(224.05194857,53.35999314)(224.07194855,53.26999323)(224.11195068,53.19)
\curveto(224.1219485,53.16999333)(224.13194849,53.15499334)(224.14195068,53.145)
\lineto(224.18695068,53.1)
\curveto(224.29694832,53.0999934)(224.38694823,53.14499335)(224.45695068,53.235)
\curveto(224.52694809,53.33499317)(224.58694803,53.41499308)(224.63695068,53.475)
\lineto(224.72695068,53.565)
\curveto(224.8169478,53.67499283)(224.94194768,53.78999271)(225.10195068,53.91)
\curveto(225.26194736,54.02999247)(225.41194721,54.11999238)(225.55195068,54.18)
\curveto(225.64194698,54.22999227)(225.73694688,54.26499224)(225.83695068,54.285)
\curveto(225.93694668,54.31499218)(226.04194658,54.34499215)(226.15195068,54.375)
\curveto(226.21194641,54.38499211)(226.27194635,54.38999211)(226.33195068,54.39)
\curveto(226.39194623,54.3999921)(226.44694617,54.40999209)(226.49695068,54.42)
}
}
{
\newrgbcolor{curcolor}{0 0 0}
\pscustom[linestyle=none,fillstyle=solid,fillcolor=curcolor]
{
\newpath
\moveto(234.61171631,50.685)
\curveto(234.63170862,50.58499591)(234.63170862,50.46999603)(234.61171631,50.34)
\curveto(234.60170865,50.21999628)(234.57170868,50.13499637)(234.52171631,50.085)
\curveto(234.47170878,50.04499645)(234.39670886,50.01499648)(234.29671631,49.995)
\curveto(234.20670905,49.98499651)(234.10170915,49.97999652)(233.98171631,49.98)
\lineto(233.62171631,49.98)
\curveto(233.50170975,49.98999651)(233.39670986,49.99499651)(233.30671631,49.995)
\lineto(229.46671631,49.995)
\curveto(229.38671387,49.99499651)(229.30671395,49.98999651)(229.22671631,49.98)
\curveto(229.14671411,49.97999652)(229.08171417,49.96499654)(229.03171631,49.935)
\curveto(228.99171426,49.91499658)(228.9517143,49.87499662)(228.91171631,49.815)
\curveto(228.89171436,49.78499671)(228.87171438,49.73999676)(228.85171631,49.68)
\curveto(228.83171442,49.62999687)(228.83171442,49.57999692)(228.85171631,49.53)
\curveto(228.86171439,49.47999702)(228.86671439,49.43499707)(228.86671631,49.395)
\curveto(228.86671439,49.35499715)(228.87171438,49.31499718)(228.88171631,49.275)
\curveto(228.90171435,49.19499731)(228.92171433,49.10999739)(228.94171631,49.02)
\curveto(228.96171429,48.93999756)(228.99171426,48.85999764)(229.03171631,48.78)
\curveto(229.26171399,48.23999826)(229.64171361,47.85499865)(230.17171631,47.625)
\curveto(230.23171302,47.59499891)(230.29671296,47.56999893)(230.36671631,47.55)
\lineto(230.57671631,47.49)
\curveto(230.60671265,47.47999902)(230.6567126,47.47499902)(230.72671631,47.475)
\curveto(230.86671239,47.43499906)(231.0517122,47.41499908)(231.28171631,47.415)
\curveto(231.51171174,47.41499908)(231.69671156,47.43499906)(231.83671631,47.475)
\curveto(231.97671128,47.51499898)(232.10171115,47.55499895)(232.21171631,47.595)
\curveto(232.33171092,47.64499885)(232.44171081,47.70499879)(232.54171631,47.775)
\curveto(232.6517106,47.84499865)(232.74671051,47.92499858)(232.82671631,48.015)
\curveto(232.90671035,48.11499838)(232.97671028,48.21999828)(233.03671631,48.33)
\curveto(233.09671016,48.42999807)(233.14671011,48.53499797)(233.18671631,48.645)
\curveto(233.23671002,48.75499775)(233.31670994,48.83499767)(233.42671631,48.885)
\curveto(233.46670979,48.90499759)(233.53170972,48.91999758)(233.62171631,48.93)
\curveto(233.71170954,48.93999756)(233.80170945,48.93999756)(233.89171631,48.93)
\curveto(233.98170927,48.92999757)(234.06670919,48.92499758)(234.14671631,48.915)
\curveto(234.22670903,48.90499759)(234.28170897,48.88499761)(234.31171631,48.855)
\curveto(234.41170884,48.78499771)(234.43670882,48.66999783)(234.38671631,48.51)
\curveto(234.30670895,48.23999826)(234.20170905,47.9999985)(234.07171631,47.79)
\curveto(233.87170938,47.46999903)(233.64170961,47.20499929)(233.38171631,46.995)
\curveto(233.13171012,46.79499971)(232.81171044,46.62999987)(232.42171631,46.5)
\curveto(232.32171093,46.46000004)(232.22171103,46.43500006)(232.12171631,46.425)
\curveto(232.02171123,46.40500009)(231.91671134,46.38500012)(231.80671631,46.365)
\curveto(231.7567115,46.35500015)(231.70671155,46.35000015)(231.65671631,46.35)
\curveto(231.61671164,46.35000015)(231.57171168,46.34500015)(231.52171631,46.335)
\lineto(231.37171631,46.335)
\curveto(231.32171193,46.32500018)(231.26171199,46.32000018)(231.19171631,46.32)
\curveto(231.13171212,46.32000018)(231.08171217,46.32500018)(231.04171631,46.335)
\lineto(230.90671631,46.335)
\curveto(230.8567124,46.34500015)(230.81171244,46.35000015)(230.77171631,46.35)
\curveto(230.73171252,46.35000015)(230.69171256,46.35500015)(230.65171631,46.365)
\curveto(230.60171265,46.37500012)(230.54671271,46.38500012)(230.48671631,46.395)
\curveto(230.42671283,46.39500011)(230.37171288,46.4000001)(230.32171631,46.41)
\curveto(230.23171302,46.43000007)(230.14171311,46.45500005)(230.05171631,46.485)
\curveto(229.96171329,46.50499999)(229.87671338,46.52999997)(229.79671631,46.56)
\curveto(229.7567135,46.57999992)(229.72171353,46.58999991)(229.69171631,46.59)
\curveto(229.66171359,46.5999999)(229.62671363,46.61499988)(229.58671631,46.635)
\curveto(229.43671382,46.70499979)(229.27671398,46.78999971)(229.10671631,46.89)
\curveto(228.81671444,47.07999942)(228.56671469,47.30999919)(228.35671631,47.58)
\curveto(228.1567151,47.85999864)(227.98671527,48.16999833)(227.84671631,48.51)
\curveto(227.79671546,48.61999788)(227.7567155,48.73499777)(227.72671631,48.855)
\curveto(227.70671555,48.97499752)(227.67671558,49.09499741)(227.63671631,49.215)
\curveto(227.62671563,49.25499725)(227.62171563,49.28999721)(227.62171631,49.32)
\curveto(227.62171563,49.34999715)(227.61671564,49.38999711)(227.60671631,49.44)
\curveto(227.58671567,49.51999698)(227.57171568,49.60499689)(227.56171631,49.695)
\curveto(227.5517157,49.78499671)(227.53671572,49.87499662)(227.51671631,49.965)
\lineto(227.51671631,50.175)
\curveto(227.50671575,50.21499628)(227.49671576,50.26999623)(227.48671631,50.34)
\curveto(227.48671577,50.41999608)(227.49171576,50.48499601)(227.50171631,50.535)
\lineto(227.50171631,50.7)
\curveto(227.52171573,50.74999575)(227.52671573,50.7999957)(227.51671631,50.85)
\curveto(227.51671574,50.90999559)(227.52171573,50.96499554)(227.53171631,51.015)
\curveto(227.57171568,51.17499532)(227.60171565,51.33499517)(227.62171631,51.495)
\curveto(227.6517156,51.65499484)(227.69671556,51.8049947)(227.75671631,51.945)
\curveto(227.80671545,52.05499444)(227.8517154,52.16499434)(227.89171631,52.275)
\curveto(227.94171531,52.39499411)(227.99671526,52.50999399)(228.05671631,52.62)
\curveto(228.27671498,52.96999353)(228.52671473,53.26999323)(228.80671631,53.52)
\curveto(229.08671417,53.77999272)(229.43171382,53.99499251)(229.84171631,54.165)
\curveto(229.96171329,54.21499228)(230.08171317,54.24999225)(230.20171631,54.27)
\curveto(230.33171292,54.2999922)(230.46671279,54.32999217)(230.60671631,54.36)
\curveto(230.6567126,54.36999213)(230.70171255,54.37499213)(230.74171631,54.375)
\curveto(230.78171247,54.38499211)(230.82671243,54.38999211)(230.87671631,54.39)
\curveto(230.89671236,54.3999921)(230.92171233,54.3999921)(230.95171631,54.39)
\curveto(230.98171227,54.37999212)(231.00671225,54.38499211)(231.02671631,54.405)
\curveto(231.44671181,54.41499208)(231.81171144,54.36999213)(232.12171631,54.27)
\curveto(232.43171082,54.17999232)(232.71171054,54.05499244)(232.96171631,53.895)
\curveto(233.01171024,53.87499263)(233.0517102,53.84499265)(233.08171631,53.805)
\curveto(233.11171014,53.77499273)(233.14671011,53.74999275)(233.18671631,53.73)
\curveto(233.26670999,53.66999283)(233.34670991,53.5999929)(233.42671631,53.52)
\curveto(233.51670974,53.43999306)(233.59170966,53.35999314)(233.65171631,53.28)
\curveto(233.81170944,53.06999343)(233.94670931,52.86999363)(234.05671631,52.68)
\curveto(234.12670913,52.56999393)(234.18170907,52.44999405)(234.22171631,52.32)
\curveto(234.26170899,52.18999431)(234.30670895,52.05999444)(234.35671631,51.93)
\curveto(234.40670885,51.7999947)(234.44170881,51.66499484)(234.46171631,51.525)
\curveto(234.49170876,51.38499511)(234.52670873,51.24499525)(234.56671631,51.105)
\curveto(234.57670868,51.03499547)(234.58170867,50.96499554)(234.58171631,50.895)
\lineto(234.61171631,50.685)
\moveto(233.15671631,51.195)
\curveto(233.18671007,51.23499527)(233.21171004,51.28499521)(233.23171631,51.345)
\curveto(233.25171,51.41499508)(233.25171,51.48499501)(233.23171631,51.555)
\curveto(233.17171008,51.77499472)(233.08671017,51.97999452)(232.97671631,52.17)
\curveto(232.83671042,52.3999941)(232.68171057,52.59499391)(232.51171631,52.755)
\curveto(232.34171091,52.91499358)(232.12171113,53.04999345)(231.85171631,53.16)
\curveto(231.78171147,53.17999332)(231.71171154,53.19499331)(231.64171631,53.205)
\curveto(231.57171168,53.22499327)(231.49671176,53.24499325)(231.41671631,53.265)
\curveto(231.33671192,53.28499321)(231.251712,53.29499321)(231.16171631,53.295)
\lineto(230.90671631,53.295)
\curveto(230.87671238,53.27499323)(230.84171241,53.26499324)(230.80171631,53.265)
\curveto(230.76171249,53.27499323)(230.72671253,53.27499323)(230.69671631,53.265)
\lineto(230.45671631,53.205)
\curveto(230.38671287,53.19499331)(230.31671294,53.17999332)(230.24671631,53.16)
\curveto(229.9567133,53.03999346)(229.72171353,52.88999361)(229.54171631,52.71)
\curveto(229.37171388,52.52999397)(229.21671404,52.3049942)(229.07671631,52.035)
\curveto(229.04671421,51.98499451)(229.01671424,51.91999458)(228.98671631,51.84)
\curveto(228.9567143,51.76999473)(228.93171432,51.68999481)(228.91171631,51.6)
\curveto(228.89171436,51.50999499)(228.88671437,51.42499508)(228.89671631,51.345)
\curveto(228.90671435,51.26499524)(228.94171431,51.2049953)(229.00171631,51.165)
\curveto(229.08171417,51.1049954)(229.21671404,51.07499542)(229.40671631,51.075)
\curveto(229.60671365,51.08499541)(229.77671348,51.08999541)(229.91671631,51.09)
\lineto(232.19671631,51.09)
\curveto(232.34671091,51.08999541)(232.52671073,51.08499541)(232.73671631,51.075)
\curveto(232.94671031,51.07499542)(233.08671017,51.11499538)(233.15671631,51.195)
}
}
{
\newrgbcolor{curcolor}{0 0 0}
\pscustom[linestyle=none,fillstyle=solid,fillcolor=curcolor]
{
\newpath
\moveto(239.56335693,54.42)
\curveto(239.79335214,54.41999208)(239.92335201,54.35999214)(239.95335693,54.24)
\curveto(239.98335195,54.12999237)(239.99835194,53.96499254)(239.99835693,53.745)
\lineto(239.99835693,53.46)
\curveto(239.99835194,53.36999313)(239.97335196,53.29499321)(239.92335693,53.235)
\curveto(239.86335207,53.15499334)(239.77835216,53.10999339)(239.66835693,53.1)
\curveto(239.55835238,53.0999934)(239.44835249,53.08499341)(239.33835693,53.055)
\curveto(239.19835274,53.02499347)(239.06335287,52.99499351)(238.93335693,52.965)
\curveto(238.81335312,52.93499357)(238.69835324,52.89499361)(238.58835693,52.845)
\curveto(238.29835364,52.71499378)(238.06335387,52.53499397)(237.88335693,52.305)
\curveto(237.70335423,52.08499441)(237.54835439,51.82999467)(237.41835693,51.54)
\curveto(237.37835456,51.42999507)(237.34835459,51.31499518)(237.32835693,51.195)
\curveto(237.30835463,51.08499541)(237.28335465,50.96999553)(237.25335693,50.85)
\curveto(237.24335469,50.7999957)(237.2383547,50.74999575)(237.23835693,50.7)
\curveto(237.24835469,50.64999585)(237.24835469,50.5999959)(237.23835693,50.55)
\curveto(237.20835473,50.42999607)(237.19335474,50.28999621)(237.19335693,50.13)
\curveto(237.20335473,49.97999652)(237.20835473,49.83499667)(237.20835693,49.695)
\lineto(237.20835693,47.85)
\lineto(237.20835693,47.505)
\curveto(237.20835473,47.38499912)(237.20335473,47.26999923)(237.19335693,47.16)
\curveto(237.18335475,47.04999945)(237.17835476,46.95499955)(237.17835693,46.875)
\curveto(237.18835475,46.79499971)(237.16835477,46.72499978)(237.11835693,46.665)
\curveto(237.06835487,46.59499991)(236.98835495,46.55499995)(236.87835693,46.545)
\curveto(236.77835516,46.53499996)(236.66835527,46.52999997)(236.54835693,46.53)
\lineto(236.27835693,46.53)
\curveto(236.22835571,46.54999995)(236.17835576,46.56499994)(236.12835693,46.575)
\curveto(236.08835585,46.59499991)(236.05835588,46.61999988)(236.03835693,46.65)
\curveto(235.98835595,46.71999978)(235.95835598,46.80499969)(235.94835693,46.905)
\lineto(235.94835693,47.235)
\lineto(235.94835693,48.39)
\lineto(235.94835693,52.545)
\lineto(235.94835693,53.58)
\lineto(235.94835693,53.88)
\curveto(235.95835598,53.97999252)(235.98835595,54.06499244)(236.03835693,54.135)
\curveto(236.06835587,54.17499233)(236.11835582,54.2049923)(236.18835693,54.225)
\curveto(236.26835567,54.24499225)(236.35335558,54.25499224)(236.44335693,54.255)
\curveto(236.5333554,54.26499224)(236.62335531,54.26499224)(236.71335693,54.255)
\curveto(236.80335513,54.24499225)(236.87335506,54.22999227)(236.92335693,54.21)
\curveto(237.00335493,54.17999232)(237.05335488,54.11999238)(237.07335693,54.03)
\curveto(237.10335483,53.94999255)(237.11835482,53.85999264)(237.11835693,53.76)
\lineto(237.11835693,53.46)
\curveto(237.11835482,53.35999314)(237.1383548,53.26999323)(237.17835693,53.19)
\curveto(237.18835475,53.16999333)(237.19835474,53.15499334)(237.20835693,53.145)
\lineto(237.25335693,53.1)
\curveto(237.36335457,53.0999934)(237.45335448,53.14499335)(237.52335693,53.235)
\curveto(237.59335434,53.33499317)(237.65335428,53.41499308)(237.70335693,53.475)
\lineto(237.79335693,53.565)
\curveto(237.88335405,53.67499283)(238.00835393,53.78999271)(238.16835693,53.91)
\curveto(238.32835361,54.02999247)(238.47835346,54.11999238)(238.61835693,54.18)
\curveto(238.70835323,54.22999227)(238.80335313,54.26499224)(238.90335693,54.285)
\curveto(239.00335293,54.31499218)(239.10835283,54.34499215)(239.21835693,54.375)
\curveto(239.27835266,54.38499211)(239.3383526,54.38999211)(239.39835693,54.39)
\curveto(239.45835248,54.3999921)(239.51335242,54.40999209)(239.56335693,54.42)
}
}
{
\newrgbcolor{curcolor}{0 0 0}
\pscustom[linestyle=none,fillstyle=solid,fillcolor=curcolor]
{
\newpath
\moveto(247.81312256,47.07)
\curveto(247.84311473,46.90999959)(247.82811474,46.77499972)(247.76812256,46.665)
\curveto(247.70811486,46.56499994)(247.62811494,46.49000001)(247.52812256,46.44)
\curveto(247.47811509,46.42000008)(247.42311515,46.41000009)(247.36312256,46.41)
\curveto(247.31311526,46.41000009)(247.25811531,46.4000001)(247.19812256,46.38)
\curveto(246.97811559,46.33000017)(246.75811581,46.34500015)(246.53812256,46.425)
\curveto(246.32811624,46.49500001)(246.18311639,46.58499992)(246.10312256,46.695)
\curveto(246.05311652,46.76499974)(246.00811656,46.84499965)(245.96812256,46.935)
\curveto(245.92811664,47.03499946)(245.87811669,47.11499938)(245.81812256,47.175)
\curveto(245.79811677,47.19499931)(245.7731168,47.21499928)(245.74312256,47.235)
\curveto(245.72311685,47.25499925)(245.69311688,47.25999924)(245.65312256,47.25)
\curveto(245.54311703,47.21999928)(245.43811713,47.16499934)(245.33812256,47.085)
\curveto(245.24811732,47.00499949)(245.15811741,46.93499956)(245.06812256,46.875)
\curveto(244.93811763,46.79499971)(244.79811777,46.71999978)(244.64812256,46.65)
\curveto(244.49811807,46.58999991)(244.33811823,46.53499996)(244.16812256,46.485)
\curveto(244.0681185,46.45500005)(243.95811861,46.43500006)(243.83812256,46.425)
\curveto(243.72811884,46.41500008)(243.61811895,46.4000001)(243.50812256,46.38)
\curveto(243.45811911,46.37000013)(243.41311916,46.36500014)(243.37312256,46.365)
\lineto(243.26812256,46.365)
\curveto(243.15811941,46.34500015)(243.05311952,46.34500015)(242.95312256,46.365)
\lineto(242.81812256,46.365)
\curveto(242.7681198,46.37500012)(242.71811985,46.38000012)(242.66812256,46.38)
\curveto(242.61811995,46.38000012)(242.57312,46.39000011)(242.53312256,46.41)
\curveto(242.49312008,46.42000008)(242.45812011,46.42500008)(242.42812256,46.425)
\curveto(242.40812016,46.41500008)(242.38312019,46.41500008)(242.35312256,46.425)
\lineto(242.11312256,46.485)
\curveto(242.03312054,46.49500001)(241.95812061,46.51499998)(241.88812256,46.545)
\curveto(241.58812098,46.67499982)(241.34312123,46.81999968)(241.15312256,46.98)
\curveto(240.9731216,47.14999935)(240.82312175,47.38499912)(240.70312256,47.685)
\curveto(240.61312196,47.90499859)(240.568122,48.16999833)(240.56812256,48.48)
\lineto(240.56812256,48.795)
\curveto(240.57812199,48.84499765)(240.58312199,48.89499761)(240.58312256,48.945)
\lineto(240.61312256,49.125)
\lineto(240.73312256,49.455)
\curveto(240.7731218,49.56499694)(240.82312175,49.66499684)(240.88312256,49.755)
\curveto(241.06312151,50.04499645)(241.30812126,50.25999624)(241.61812256,50.4)
\curveto(241.92812064,50.53999596)(242.2681203,50.66499584)(242.63812256,50.775)
\curveto(242.77811979,50.81499568)(242.92311965,50.84499565)(243.07312256,50.865)
\curveto(243.22311935,50.88499561)(243.3731192,50.90999559)(243.52312256,50.94)
\curveto(243.59311898,50.95999554)(243.65811891,50.96999553)(243.71812256,50.97)
\curveto(243.78811878,50.96999553)(243.86311871,50.97999552)(243.94312256,51)
\curveto(244.01311856,51.01999548)(244.08311849,51.02999547)(244.15312256,51.03)
\curveto(244.22311835,51.03999546)(244.29811827,51.05499544)(244.37812256,51.075)
\curveto(244.62811794,51.13499537)(244.86311771,51.18499531)(245.08312256,51.225)
\curveto(245.30311727,51.27499522)(245.47811709,51.38999511)(245.60812256,51.57)
\curveto(245.6681169,51.64999485)(245.71811685,51.74999475)(245.75812256,51.87)
\curveto(245.79811677,51.9999945)(245.79811677,52.13999436)(245.75812256,52.29)
\curveto(245.69811687,52.52999397)(245.60811696,52.71999378)(245.48812256,52.86)
\curveto(245.37811719,52.9999935)(245.21811735,53.10999339)(245.00812256,53.19)
\curveto(244.88811768,53.23999326)(244.74311783,53.27499323)(244.57312256,53.295)
\curveto(244.41311816,53.31499318)(244.24311833,53.32499317)(244.06312256,53.325)
\curveto(243.88311869,53.32499317)(243.70811886,53.31499318)(243.53812256,53.295)
\curveto(243.3681192,53.27499323)(243.22311935,53.24499325)(243.10312256,53.205)
\curveto(242.93311964,53.14499335)(242.7681198,53.05999344)(242.60812256,52.95)
\curveto(242.52812004,52.88999361)(242.45312012,52.80999369)(242.38312256,52.71)
\curveto(242.32312025,52.61999388)(242.2681203,52.51999398)(242.21812256,52.41)
\curveto(242.18812038,52.32999417)(242.15812041,52.24499425)(242.12812256,52.155)
\curveto(242.10812046,52.06499444)(242.06312051,51.99499451)(241.99312256,51.945)
\curveto(241.95312062,51.91499458)(241.88312069,51.88999461)(241.78312256,51.87)
\curveto(241.69312088,51.85999464)(241.59812097,51.85499464)(241.49812256,51.855)
\curveto(241.39812117,51.85499464)(241.29812127,51.85999464)(241.19812256,51.87)
\curveto(241.10812146,51.88999461)(241.04312153,51.91499458)(241.00312256,51.945)
\curveto(240.96312161,51.97499452)(240.93312164,52.02499447)(240.91312256,52.095)
\curveto(240.89312168,52.16499434)(240.89312168,52.23999426)(240.91312256,52.32)
\curveto(240.94312163,52.44999405)(240.9731216,52.56999393)(241.00312256,52.68)
\curveto(241.04312153,52.7999937)(241.08812148,52.91499358)(241.13812256,53.025)
\curveto(241.32812124,53.37499313)(241.568121,53.64499285)(241.85812256,53.835)
\curveto(242.14812042,54.03499247)(242.50812006,54.19499231)(242.93812256,54.315)
\curveto(243.03811953,54.33499217)(243.13811943,54.34999215)(243.23812256,54.36)
\curveto(243.34811922,54.36999213)(243.45811911,54.38499211)(243.56812256,54.405)
\curveto(243.60811896,54.41499208)(243.6731189,54.41499208)(243.76312256,54.405)
\curveto(243.85311872,54.4049921)(243.90811866,54.41499208)(243.92812256,54.435)
\curveto(244.62811794,54.44499206)(245.23811733,54.36499214)(245.75812256,54.195)
\curveto(246.27811629,54.02499247)(246.64311593,53.6999928)(246.85312256,53.22)
\curveto(246.94311563,53.01999348)(246.99311558,52.78499371)(247.00312256,52.515)
\curveto(247.02311555,52.25499424)(247.03311554,51.97999452)(247.03312256,51.69)
\lineto(247.03312256,48.375)
\curveto(247.03311554,48.23499827)(247.03811553,48.0999984)(247.04812256,47.97)
\curveto(247.05811551,47.83999866)(247.08811548,47.73499877)(247.13812256,47.655)
\curveto(247.18811538,47.58499892)(247.25311532,47.53499896)(247.33312256,47.505)
\curveto(247.42311515,47.46499904)(247.50811506,47.43499906)(247.58812256,47.415)
\curveto(247.6681149,47.40499909)(247.72811484,47.35999914)(247.76812256,47.28)
\curveto(247.78811478,47.24999925)(247.79811477,47.21999928)(247.79812256,47.19)
\curveto(247.79811477,47.15999934)(247.80311477,47.11999938)(247.81312256,47.07)
\moveto(245.66812256,48.735)
\curveto(245.72811684,48.87499762)(245.75811681,49.03499747)(245.75812256,49.215)
\curveto(245.7681168,49.40499709)(245.7731168,49.5999969)(245.77312256,49.8)
\curveto(245.7731168,49.90999659)(245.7681168,50.00999649)(245.75812256,50.1)
\curveto(245.74811682,50.18999631)(245.70811686,50.25999624)(245.63812256,50.31)
\curveto(245.60811696,50.32999617)(245.53811703,50.33999616)(245.42812256,50.34)
\curveto(245.40811716,50.31999618)(245.3731172,50.30999619)(245.32312256,50.31)
\curveto(245.2731173,50.30999619)(245.22811734,50.2999962)(245.18812256,50.28)
\curveto(245.10811746,50.25999624)(245.01811755,50.23999626)(244.91812256,50.22)
\lineto(244.61812256,50.16)
\curveto(244.58811798,50.15999634)(244.55311802,50.15499634)(244.51312256,50.145)
\lineto(244.40812256,50.145)
\curveto(244.25811831,50.1049964)(244.09311848,50.07999642)(243.91312256,50.07)
\curveto(243.74311883,50.06999643)(243.58311899,50.04999645)(243.43312256,50.01)
\curveto(243.35311922,49.98999651)(243.27811929,49.96999653)(243.20812256,49.95)
\curveto(243.14811942,49.93999656)(243.07811949,49.92499658)(242.99812256,49.905)
\curveto(242.83811973,49.85499665)(242.68811988,49.78999671)(242.54812256,49.71)
\curveto(242.40812016,49.63999686)(242.28812028,49.54999695)(242.18812256,49.44)
\curveto(242.08812048,49.32999717)(242.01312056,49.19499731)(241.96312256,49.035)
\curveto(241.91312066,48.88499761)(241.89312068,48.6999978)(241.90312256,48.48)
\curveto(241.90312067,48.37999812)(241.91812065,48.28499821)(241.94812256,48.195)
\curveto(241.98812058,48.11499838)(242.03312054,48.03999846)(242.08312256,47.97)
\curveto(242.16312041,47.85999864)(242.2681203,47.76499874)(242.39812256,47.685)
\curveto(242.52812004,47.61499888)(242.6681199,47.55499895)(242.81812256,47.505)
\curveto(242.8681197,47.49499901)(242.91811965,47.48999901)(242.96812256,47.49)
\curveto(243.01811955,47.48999901)(243.0681195,47.48499902)(243.11812256,47.475)
\curveto(243.18811938,47.45499905)(243.2731193,47.43999906)(243.37312256,47.43)
\curveto(243.48311909,47.42999907)(243.573119,47.43999906)(243.64312256,47.46)
\curveto(243.70311887,47.47999902)(243.76311881,47.48499902)(243.82312256,47.475)
\curveto(243.88311869,47.47499902)(243.94311863,47.48499902)(244.00312256,47.505)
\curveto(244.08311849,47.52499898)(244.15811841,47.53999896)(244.22812256,47.55)
\curveto(244.30811826,47.55999894)(244.38311819,47.57999892)(244.45312256,47.61)
\curveto(244.74311783,47.72999877)(244.98811758,47.87499862)(245.18812256,48.045)
\curveto(245.39811717,48.21499828)(245.55811701,48.44499805)(245.66812256,48.735)
}
}
{
\newrgbcolor{curcolor}{0 0 0}
\pscustom[linestyle=none,fillstyle=solid,fillcolor=curcolor]
{
\newpath
\moveto(251.41476318,54.42)
\curveto(252.13475912,54.42999207)(252.73975851,54.34499215)(253.22976318,54.165)
\curveto(253.71975753,53.99499251)(254.09975715,53.68999281)(254.36976318,53.25)
\curveto(254.43975681,53.13999336)(254.49475676,53.02499347)(254.53476318,52.905)
\curveto(254.57475668,52.79499371)(254.61475664,52.66999383)(254.65476318,52.53)
\curveto(254.67475658,52.45999404)(254.67975657,52.38499411)(254.66976318,52.305)
\curveto(254.65975659,52.23499427)(254.64475661,52.17999432)(254.62476318,52.14)
\curveto(254.60475665,52.11999438)(254.57975667,52.0999944)(254.54976318,52.08)
\curveto(254.51975673,52.06999443)(254.49475676,52.05499444)(254.47476318,52.035)
\curveto(254.42475683,52.01499448)(254.37475688,52.00999449)(254.32476318,52.02)
\curveto(254.27475698,52.02999447)(254.22475703,52.02999447)(254.17476318,52.02)
\curveto(254.09475716,51.9999945)(253.98975726,51.99499451)(253.85976318,52.005)
\curveto(253.72975752,52.02499447)(253.63975761,52.04999445)(253.58976318,52.08)
\curveto(253.50975774,52.12999437)(253.4547578,52.19499431)(253.42476318,52.275)
\curveto(253.40475785,52.36499414)(253.36975788,52.44999405)(253.31976318,52.53)
\curveto(253.22975802,52.68999381)(253.10475815,52.83499367)(252.94476318,52.965)
\curveto(252.83475842,53.04499345)(252.71475854,53.1049934)(252.58476318,53.145)
\curveto(252.4547588,53.18499331)(252.31475894,53.22499327)(252.16476318,53.265)
\curveto(252.11475914,53.28499321)(252.06475919,53.28999321)(252.01476318,53.28)
\curveto(251.96475929,53.27999322)(251.91475934,53.28499321)(251.86476318,53.295)
\curveto(251.80475945,53.31499318)(251.72975952,53.32499317)(251.63976318,53.325)
\curveto(251.5497597,53.32499317)(251.47475978,53.31499318)(251.41476318,53.295)
\lineto(251.32476318,53.295)
\lineto(251.17476318,53.265)
\curveto(251.12476013,53.26499324)(251.07476018,53.25999324)(251.02476318,53.25)
\curveto(250.76476049,53.18999331)(250.5497607,53.1049934)(250.37976318,52.995)
\curveto(250.20976104,52.88499361)(250.09476116,52.6999938)(250.03476318,52.44)
\curveto(250.01476124,52.36999413)(250.00976124,52.2999942)(250.01976318,52.23)
\curveto(250.03976121,52.15999434)(250.05976119,52.0999944)(250.07976318,52.05)
\curveto(250.13976111,51.8999946)(250.20976104,51.78999471)(250.28976318,51.72)
\curveto(250.37976087,51.65999484)(250.48976076,51.58999491)(250.61976318,51.51)
\curveto(250.77976047,51.40999509)(250.95976029,51.33499517)(251.15976318,51.285)
\curveto(251.35975989,51.24499525)(251.55975969,51.19499531)(251.75976318,51.135)
\curveto(251.88975936,51.09499541)(252.01975923,51.06499544)(252.14976318,51.045)
\curveto(252.27975897,51.02499548)(252.40975884,50.99499551)(252.53976318,50.955)
\curveto(252.7497585,50.89499561)(252.9547583,50.83499567)(253.15476318,50.775)
\curveto(253.3547579,50.72499578)(253.5547577,50.65999584)(253.75476318,50.58)
\lineto(253.90476318,50.52)
\curveto(253.9547573,50.499996)(254.00475725,50.47499602)(254.05476318,50.445)
\curveto(254.254757,50.32499618)(254.42975682,50.18999631)(254.57976318,50.04)
\curveto(254.72975652,49.88999661)(254.8547564,49.6999968)(254.95476318,49.47)
\curveto(254.97475628,49.3999971)(254.99475626,49.30499719)(255.01476318,49.185)
\curveto(255.03475622,49.11499738)(255.04475621,49.03999746)(255.04476318,48.96)
\curveto(255.0547562,48.88999761)(255.05975619,48.80999769)(255.05976318,48.72)
\lineto(255.05976318,48.57)
\curveto(255.03975621,48.499998)(255.02975622,48.42999807)(255.02976318,48.36)
\curveto(255.02975622,48.28999821)(255.01975623,48.21999828)(254.99976318,48.15)
\curveto(254.96975628,48.03999846)(254.93475632,47.93499857)(254.89476318,47.835)
\curveto(254.8547564,47.73499877)(254.80975644,47.64499885)(254.75976318,47.565)
\curveto(254.59975665,47.30499919)(254.39475686,47.09499941)(254.14476318,46.935)
\curveto(253.89475736,46.78499972)(253.61475764,46.65499985)(253.30476318,46.545)
\curveto(253.21475804,46.51499998)(253.11975813,46.49500001)(253.01976318,46.485)
\curveto(252.92975832,46.46500004)(252.83975841,46.44000006)(252.74976318,46.41)
\curveto(252.6497586,46.39000011)(252.5497587,46.38000012)(252.44976318,46.38)
\curveto(252.3497589,46.38000012)(252.249759,46.37000013)(252.14976318,46.35)
\lineto(251.99976318,46.35)
\curveto(251.9497593,46.34000016)(251.87975937,46.33500016)(251.78976318,46.335)
\curveto(251.69975955,46.33500016)(251.62975962,46.34000016)(251.57976318,46.35)
\lineto(251.41476318,46.35)
\curveto(251.3547599,46.37000013)(251.28975996,46.38000012)(251.21976318,46.38)
\curveto(251.1497601,46.37000013)(251.08976016,46.37500012)(251.03976318,46.395)
\curveto(250.98976026,46.40500009)(250.92476033,46.41000009)(250.84476318,46.41)
\lineto(250.60476318,46.47)
\curveto(250.53476072,46.48000002)(250.45976079,46.5)(250.37976318,46.53)
\curveto(250.06976118,46.62999987)(249.79976145,46.75499975)(249.56976318,46.905)
\curveto(249.33976191,47.05499945)(249.13976211,47.24999925)(248.96976318,47.49)
\curveto(248.87976237,47.61999888)(248.80476245,47.75499875)(248.74476318,47.895)
\curveto(248.68476257,48.03499847)(248.62976262,48.18999831)(248.57976318,48.36)
\curveto(248.55976269,48.41999808)(248.5497627,48.48999801)(248.54976318,48.57)
\curveto(248.55976269,48.65999784)(248.57476268,48.72999777)(248.59476318,48.78)
\curveto(248.62476263,48.81999768)(248.67476258,48.85999764)(248.74476318,48.9)
\curveto(248.79476246,48.91999758)(248.86476239,48.92999757)(248.95476318,48.93)
\curveto(249.04476221,48.93999756)(249.13476212,48.93999756)(249.22476318,48.93)
\curveto(249.31476194,48.91999758)(249.39976185,48.90499759)(249.47976318,48.885)
\curveto(249.56976168,48.87499762)(249.62976162,48.85999764)(249.65976318,48.84)
\curveto(249.72976152,48.78999771)(249.77476148,48.71499778)(249.79476318,48.615)
\curveto(249.82476143,48.52499798)(249.85976139,48.43999806)(249.89976318,48.36)
\curveto(249.99976125,48.13999836)(250.13476112,47.96999853)(250.30476318,47.85)
\curveto(250.42476083,47.75999874)(250.55976069,47.68999881)(250.70976318,47.64)
\curveto(250.85976039,47.58999891)(251.01976023,47.53999896)(251.18976318,47.49)
\lineto(251.50476318,47.445)
\lineto(251.59476318,47.445)
\curveto(251.66475959,47.42499908)(251.7547595,47.41499908)(251.86476318,47.415)
\curveto(251.98475927,47.41499908)(252.08475917,47.42499908)(252.16476318,47.445)
\curveto(252.23475902,47.44499905)(252.28975896,47.44999905)(252.32976318,47.46)
\curveto(252.38975886,47.46999903)(252.4497588,47.47499902)(252.50976318,47.475)
\curveto(252.56975868,47.48499902)(252.62475863,47.49499901)(252.67476318,47.505)
\curveto(252.96475829,47.58499892)(253.19475806,47.68999881)(253.36476318,47.82)
\curveto(253.53475772,47.94999855)(253.6547576,48.16999833)(253.72476318,48.48)
\curveto(253.74475751,48.52999797)(253.7497575,48.58499791)(253.73976318,48.645)
\curveto(253.72975752,48.70499779)(253.71975753,48.74999775)(253.70976318,48.78)
\curveto(253.65975759,48.96999753)(253.58975766,49.10999739)(253.49976318,49.2)
\curveto(253.40975784,49.2999972)(253.29475796,49.38999711)(253.15476318,49.47)
\curveto(253.06475819,49.52999697)(252.96475829,49.57999692)(252.85476318,49.62)
\lineto(252.52476318,49.74)
\curveto(252.49475876,49.74999675)(252.46475879,49.75499675)(252.43476318,49.755)
\curveto(252.41475884,49.75499675)(252.38975886,49.76499674)(252.35976318,49.785)
\curveto(252.01975923,49.89499661)(251.66475959,49.97499652)(251.29476318,50.025)
\curveto(250.93476032,50.08499641)(250.59476066,50.17999632)(250.27476318,50.31)
\curveto(250.17476108,50.34999615)(250.07976117,50.38499611)(249.98976318,50.415)
\curveto(249.89976135,50.44499605)(249.81476144,50.48499601)(249.73476318,50.535)
\curveto(249.54476171,50.64499585)(249.36976188,50.76999573)(249.20976318,50.91)
\curveto(249.0497622,51.04999545)(248.92476233,51.22499528)(248.83476318,51.435)
\curveto(248.80476245,51.504995)(248.77976247,51.57499492)(248.75976318,51.645)
\curveto(248.7497625,51.71499478)(248.73476252,51.78999471)(248.71476318,51.87)
\curveto(248.68476257,51.98999451)(248.67476258,52.12499438)(248.68476318,52.275)
\curveto(248.69476256,52.43499407)(248.70976254,52.56999393)(248.72976318,52.68)
\curveto(248.7497625,52.72999377)(248.75976249,52.76999373)(248.75976318,52.8)
\curveto(248.76976248,52.83999366)(248.78476247,52.87999362)(248.80476318,52.92)
\curveto(248.89476236,53.14999335)(249.01476224,53.34999315)(249.16476318,53.52)
\curveto(249.32476193,53.68999281)(249.50476175,53.83999266)(249.70476318,53.97)
\curveto(249.8547614,54.05999244)(250.01976123,54.12999237)(250.19976318,54.18)
\curveto(250.37976087,54.23999226)(250.56976068,54.29499221)(250.76976318,54.345)
\curveto(250.83976041,54.35499214)(250.90476035,54.36499214)(250.96476318,54.375)
\curveto(251.03476022,54.38499211)(251.10976014,54.39499211)(251.18976318,54.405)
\curveto(251.21976003,54.41499208)(251.25975999,54.41499208)(251.30976318,54.405)
\curveto(251.35975989,54.39499211)(251.39475986,54.3999921)(251.41476318,54.42)
}
}
{
\newrgbcolor{curcolor}{0 0 0}
\pscustom[linestyle=none,fillstyle=solid,fillcolor=curcolor]
{
\newpath
\moveto(382.46584106,57.195)
\lineto(383.38084106,57.195)
\curveto(383.48083841,57.1949893)(383.57583832,57.1949893)(383.66584106,57.195)
\curveto(383.75583814,57.1949893)(383.83083806,57.17498933)(383.89084106,57.135)
\curveto(383.98083791,57.07498943)(384.04083785,56.9949895)(384.07084106,56.895)
\curveto(384.11083778,56.7949897)(384.15583774,56.68998981)(384.20584106,56.58)
\curveto(384.28583761,56.38999011)(384.35583754,56.1999903)(384.41584106,56.01)
\curveto(384.48583741,55.81999068)(384.56083733,55.62999087)(384.64084106,55.44)
\curveto(384.71083718,55.25999124)(384.77583712,55.07499143)(384.83584106,54.885)
\curveto(384.895837,54.7049918)(384.96583693,54.52499197)(385.04584106,54.345)
\curveto(385.10583679,54.2049923)(385.16083673,54.05999244)(385.21084106,53.91)
\curveto(385.26083663,53.75999274)(385.31583658,53.61499288)(385.37584106,53.475)
\curveto(385.55583634,53.02499347)(385.72583617,52.56999393)(385.88584106,52.11)
\curveto(386.04583585,51.65999484)(386.21583568,51.20999529)(386.39584106,50.76)
\curveto(386.41583548,50.70999579)(386.43083546,50.65999584)(386.44084106,50.61)
\lineto(386.50084106,50.46)
\curveto(386.5908353,50.23999626)(386.67583522,50.01499648)(386.75584106,49.785)
\curveto(386.83583506,49.56499694)(386.92083497,49.34499715)(387.01084106,49.125)
\curveto(387.05083484,49.03499747)(387.0908348,48.92499758)(387.13084106,48.795)
\curveto(387.17083472,48.67499782)(387.23583466,48.60499789)(387.32584106,48.585)
\curveto(387.36583453,48.57499792)(387.3958345,48.57499792)(387.41584106,48.585)
\lineto(387.47584106,48.645)
\curveto(387.52583437,48.69499781)(387.56083433,48.74999775)(387.58084106,48.81)
\curveto(387.61083428,48.86999763)(387.64083425,48.93499757)(387.67084106,49.005)
\lineto(387.91084106,49.635)
\curveto(387.9908339,49.85499665)(388.07083382,50.06999643)(388.15084106,50.28)
\lineto(388.21084106,50.43)
\lineto(388.27084106,50.61)
\curveto(388.35083354,50.7999957)(388.42083347,50.98999551)(388.48084106,51.18)
\curveto(388.55083334,51.37999512)(388.62583327,51.57999492)(388.70584106,51.78)
\curveto(388.94583295,52.35999414)(389.16583273,52.94499355)(389.36584106,53.535)
\curveto(389.57583232,54.12499237)(389.80083209,54.70999179)(390.04084106,55.29)
\curveto(390.12083177,55.48999101)(390.1958317,55.6949908)(390.26584106,55.905)
\curveto(390.34583155,56.11499038)(390.42583147,56.31999018)(390.50584106,56.52)
\curveto(390.54583135,56.5999899)(390.58083131,56.6999898)(390.61084106,56.82)
\curveto(390.65083124,56.93998956)(390.70583119,57.02498947)(390.77584106,57.075)
\curveto(390.83583106,57.11498939)(390.91083098,57.14498936)(391.00084106,57.165)
\curveto(391.10083079,57.18498931)(391.21083068,57.1949893)(391.33084106,57.195)
\curveto(391.45083044,57.2049893)(391.57083032,57.2049893)(391.69084106,57.195)
\curveto(391.81083008,57.1949893)(391.92082997,57.1949893)(392.02084106,57.195)
\curveto(392.11082978,57.1949893)(392.20082969,57.1949893)(392.29084106,57.195)
\curveto(392.3908295,57.1949893)(392.46582943,57.17498933)(392.51584106,57.135)
\curveto(392.60582929,57.08498941)(392.65582924,56.9949895)(392.66584106,56.865)
\curveto(392.67582922,56.73498977)(392.68082921,56.5949899)(392.68084106,56.445)
\lineto(392.68084106,54.795)
\lineto(392.68084106,48.525)
\lineto(392.68084106,47.265)
\curveto(392.68082921,47.15499935)(392.68082921,47.04499945)(392.68084106,46.935)
\curveto(392.6908292,46.82499968)(392.67082922,46.73999976)(392.62084106,46.68)
\curveto(392.5908293,46.61999988)(392.54582935,46.57999992)(392.48584106,46.56)
\curveto(392.42582947,46.54999995)(392.35582954,46.53499996)(392.27584106,46.515)
\lineto(392.03584106,46.515)
\lineto(391.67584106,46.515)
\curveto(391.56583033,46.52499998)(391.48583041,46.56999993)(391.43584106,46.65)
\curveto(391.41583048,46.67999982)(391.40083049,46.70999979)(391.39084106,46.74)
\curveto(391.3908305,46.77999972)(391.38083051,46.82499968)(391.36084106,46.875)
\lineto(391.36084106,47.04)
\curveto(391.35083054,47.0999994)(391.34583055,47.16999933)(391.34584106,47.25)
\curveto(391.35583054,47.32999917)(391.36083053,47.40499909)(391.36084106,47.475)
\lineto(391.36084106,48.315)
\lineto(391.36084106,52.74)
\curveto(391.36083053,52.98999351)(391.36083053,53.23999326)(391.36084106,53.49)
\curveto(391.36083053,53.74999275)(391.35583054,53.9999925)(391.34584106,54.24)
\curveto(391.34583055,54.33999216)(391.34083055,54.44999205)(391.33084106,54.57)
\curveto(391.32083057,54.68999181)(391.26583063,54.74999175)(391.16584106,54.75)
\lineto(391.16584106,54.735)
\curveto(391.0958308,54.71499178)(391.03583086,54.64999185)(390.98584106,54.54)
\curveto(390.94583095,54.42999207)(390.91083098,54.33499217)(390.88084106,54.255)
\curveto(390.81083108,54.08499241)(390.74583115,53.90999259)(390.68584106,53.73)
\curveto(390.62583127,53.55999294)(390.55583134,53.38999311)(390.47584106,53.22)
\curveto(390.45583144,53.16999333)(390.44083145,53.12499337)(390.43084106,53.085)
\curveto(390.42083147,53.04499345)(390.40583149,52.9999935)(390.38584106,52.95)
\curveto(390.30583159,52.76999373)(390.23583166,52.58499391)(390.17584106,52.395)
\curveto(390.12583177,52.21499428)(390.06083183,52.03499447)(389.98084106,51.855)
\curveto(389.91083198,51.7049948)(389.85083204,51.54999495)(389.80084106,51.39)
\curveto(389.75083214,51.23999526)(389.6958322,51.08999541)(389.63584106,50.94)
\curveto(389.43583246,50.46999603)(389.25583264,49.99499651)(389.09584106,49.515)
\curveto(388.93583296,49.04499745)(388.76083313,48.57999792)(388.57084106,48.12)
\curveto(388.4908334,47.93999856)(388.42083347,47.75999874)(388.36084106,47.58)
\curveto(388.30083359,47.3999991)(388.23583366,47.21999928)(388.16584106,47.04)
\curveto(388.11583378,46.92999957)(388.06583383,46.82499968)(388.01584106,46.725)
\curveto(387.97583392,46.63499986)(387.890834,46.56999993)(387.76084106,46.53)
\curveto(387.74083415,46.51999998)(387.71583418,46.51499998)(387.68584106,46.515)
\curveto(387.66583423,46.52499998)(387.64083425,46.52499998)(387.61084106,46.515)
\curveto(387.58083431,46.50499999)(387.54583435,46.5)(387.50584106,46.5)
\curveto(387.46583443,46.50999999)(387.42583447,46.51499998)(387.38584106,46.515)
\lineto(387.08584106,46.515)
\curveto(386.98583491,46.51499998)(386.90583499,46.53999996)(386.84584106,46.59)
\curveto(386.76583513,46.63999986)(386.70583519,46.70999979)(386.66584106,46.8)
\curveto(386.63583526,46.8999996)(386.5958353,46.9999995)(386.54584106,47.1)
\curveto(386.46583543,47.2999992)(386.38583551,47.50499899)(386.30584106,47.715)
\curveto(386.23583566,47.93499857)(386.16083573,48.14499835)(386.08084106,48.345)
\curveto(386.00083589,48.52499798)(385.93083596,48.70499779)(385.87084106,48.885)
\curveto(385.82083607,49.07499742)(385.75583614,49.25999724)(385.67584106,49.44)
\curveto(385.44583645,49.9999965)(385.23083666,50.56499594)(385.03084106,51.135)
\curveto(384.83083706,51.7049948)(384.61583728,52.26999423)(384.38584106,52.83)
\lineto(384.14584106,53.46)
\curveto(384.07583782,53.67999282)(384.00083789,53.88999261)(383.92084106,54.09)
\curveto(383.87083802,54.1999923)(383.82583807,54.3049922)(383.78584106,54.405)
\curveto(383.75583814,54.51499198)(383.70583819,54.60999189)(383.63584106,54.69)
\curveto(383.62583827,54.70999179)(383.61583828,54.71999178)(383.60584106,54.72)
\lineto(383.57584106,54.75)
\lineto(383.50084106,54.75)
\lineto(383.47084106,54.72)
\curveto(383.46083843,54.71999178)(383.45083844,54.71499178)(383.44084106,54.705)
\curveto(383.42083847,54.65499184)(383.41083848,54.5999919)(383.41084106,54.54)
\curveto(383.41083848,54.47999202)(383.40083849,54.41999208)(383.38084106,54.36)
\lineto(383.38084106,54.195)
\curveto(383.36083853,54.13499237)(383.35583854,54.06999243)(383.36584106,54)
\curveto(383.37583852,53.92999257)(383.38083851,53.85999264)(383.38084106,53.79)
\lineto(383.38084106,52.98)
\lineto(383.38084106,48.42)
\lineto(383.38084106,47.235)
\curveto(383.38083851,47.12499938)(383.37583852,47.01499948)(383.36584106,46.905)
\curveto(383.36583853,46.79499971)(383.34083855,46.70999979)(383.29084106,46.65)
\curveto(383.24083865,46.56999993)(383.15083874,46.52499998)(383.02084106,46.515)
\lineto(382.63084106,46.515)
\lineto(382.43584106,46.515)
\curveto(382.38583951,46.51499998)(382.33583956,46.52499998)(382.28584106,46.545)
\curveto(382.15583974,46.58499992)(382.08083981,46.66999983)(382.06084106,46.8)
\curveto(382.05083984,46.92999957)(382.04583985,47.07999942)(382.04584106,47.25)
\lineto(382.04584106,48.99)
\lineto(382.04584106,54.99)
\lineto(382.04584106,56.4)
\curveto(382.04583985,56.50998999)(382.04083985,56.62498987)(382.03084106,56.745)
\curveto(382.03083986,56.86498964)(382.05583984,56.95998954)(382.10584106,57.03)
\curveto(382.14583975,57.08998941)(382.22083967,57.13998936)(382.33084106,57.18)
\curveto(382.35083954,57.18998931)(382.37083952,57.18998931)(382.39084106,57.18)
\curveto(382.42083947,57.17998932)(382.44583945,57.18498931)(382.46584106,57.195)
}
}
{
\newrgbcolor{curcolor}{0 0 0}
\pscustom[linestyle=none,fillstyle=solid,fillcolor=curcolor]
{
\newpath
\moveto(401.66795044,47.07)
\curveto(401.69794261,46.90999959)(401.68294262,46.77499972)(401.62295044,46.665)
\curveto(401.56294274,46.56499994)(401.48294282,46.49000001)(401.38295044,46.44)
\curveto(401.33294297,46.42000008)(401.27794303,46.41000009)(401.21795044,46.41)
\curveto(401.16794314,46.41000009)(401.11294319,46.4000001)(401.05295044,46.38)
\curveto(400.83294347,46.33000017)(400.61294369,46.34500015)(400.39295044,46.425)
\curveto(400.18294412,46.49500001)(400.03794427,46.58499992)(399.95795044,46.695)
\curveto(399.9079444,46.76499974)(399.86294444,46.84499965)(399.82295044,46.935)
\curveto(399.78294452,47.03499946)(399.73294457,47.11499938)(399.67295044,47.175)
\curveto(399.65294465,47.19499931)(399.62794468,47.21499928)(399.59795044,47.235)
\curveto(399.57794473,47.25499925)(399.54794476,47.25999924)(399.50795044,47.25)
\curveto(399.39794491,47.21999928)(399.29294501,47.16499934)(399.19295044,47.085)
\curveto(399.1029452,47.00499949)(399.01294529,46.93499956)(398.92295044,46.875)
\curveto(398.79294551,46.79499971)(398.65294565,46.71999978)(398.50295044,46.65)
\curveto(398.35294595,46.58999991)(398.19294611,46.53499996)(398.02295044,46.485)
\curveto(397.92294638,46.45500005)(397.81294649,46.43500006)(397.69295044,46.425)
\curveto(397.58294672,46.41500008)(397.47294683,46.4000001)(397.36295044,46.38)
\curveto(397.31294699,46.37000013)(397.26794704,46.36500014)(397.22795044,46.365)
\lineto(397.12295044,46.365)
\curveto(397.01294729,46.34500015)(396.9079474,46.34500015)(396.80795044,46.365)
\lineto(396.67295044,46.365)
\curveto(396.62294768,46.37500012)(396.57294773,46.38000012)(396.52295044,46.38)
\curveto(396.47294783,46.38000012)(396.42794788,46.39000011)(396.38795044,46.41)
\curveto(396.34794796,46.42000008)(396.31294799,46.42500008)(396.28295044,46.425)
\curveto(396.26294804,46.41500008)(396.23794807,46.41500008)(396.20795044,46.425)
\lineto(395.96795044,46.485)
\curveto(395.88794842,46.49500001)(395.81294849,46.51499998)(395.74295044,46.545)
\curveto(395.44294886,46.67499982)(395.19794911,46.81999968)(395.00795044,46.98)
\curveto(394.82794948,47.14999935)(394.67794963,47.38499912)(394.55795044,47.685)
\curveto(394.46794984,47.90499859)(394.42294988,48.16999833)(394.42295044,48.48)
\lineto(394.42295044,48.795)
\curveto(394.43294987,48.84499765)(394.43794987,48.89499761)(394.43795044,48.945)
\lineto(394.46795044,49.125)
\lineto(394.58795044,49.455)
\curveto(394.62794968,49.56499694)(394.67794963,49.66499684)(394.73795044,49.755)
\curveto(394.91794939,50.04499645)(395.16294914,50.25999624)(395.47295044,50.4)
\curveto(395.78294852,50.53999596)(396.12294818,50.66499584)(396.49295044,50.775)
\curveto(396.63294767,50.81499568)(396.77794753,50.84499565)(396.92795044,50.865)
\curveto(397.07794723,50.88499561)(397.22794708,50.90999559)(397.37795044,50.94)
\curveto(397.44794686,50.95999554)(397.51294679,50.96999553)(397.57295044,50.97)
\curveto(397.64294666,50.96999553)(397.71794659,50.97999552)(397.79795044,51)
\curveto(397.86794644,51.01999548)(397.93794637,51.02999547)(398.00795044,51.03)
\curveto(398.07794623,51.03999546)(398.15294615,51.05499544)(398.23295044,51.075)
\curveto(398.48294582,51.13499537)(398.71794559,51.18499531)(398.93795044,51.225)
\curveto(399.15794515,51.27499522)(399.33294497,51.38999511)(399.46295044,51.57)
\curveto(399.52294478,51.64999485)(399.57294473,51.74999475)(399.61295044,51.87)
\curveto(399.65294465,51.9999945)(399.65294465,52.13999436)(399.61295044,52.29)
\curveto(399.55294475,52.52999397)(399.46294484,52.71999378)(399.34295044,52.86)
\curveto(399.23294507,52.9999935)(399.07294523,53.10999339)(398.86295044,53.19)
\curveto(398.74294556,53.23999326)(398.59794571,53.27499323)(398.42795044,53.295)
\curveto(398.26794604,53.31499318)(398.09794621,53.32499317)(397.91795044,53.325)
\curveto(397.73794657,53.32499317)(397.56294674,53.31499318)(397.39295044,53.295)
\curveto(397.22294708,53.27499323)(397.07794723,53.24499325)(396.95795044,53.205)
\curveto(396.78794752,53.14499335)(396.62294768,53.05999344)(396.46295044,52.95)
\curveto(396.38294792,52.88999361)(396.307948,52.80999369)(396.23795044,52.71)
\curveto(396.17794813,52.61999388)(396.12294818,52.51999398)(396.07295044,52.41)
\curveto(396.04294826,52.32999417)(396.01294829,52.24499425)(395.98295044,52.155)
\curveto(395.96294834,52.06499444)(395.91794839,51.99499451)(395.84795044,51.945)
\curveto(395.8079485,51.91499458)(395.73794857,51.88999461)(395.63795044,51.87)
\curveto(395.54794876,51.85999464)(395.45294885,51.85499464)(395.35295044,51.855)
\curveto(395.25294905,51.85499464)(395.15294915,51.85999464)(395.05295044,51.87)
\curveto(394.96294934,51.88999461)(394.89794941,51.91499458)(394.85795044,51.945)
\curveto(394.81794949,51.97499452)(394.78794952,52.02499447)(394.76795044,52.095)
\curveto(394.74794956,52.16499434)(394.74794956,52.23999426)(394.76795044,52.32)
\curveto(394.79794951,52.44999405)(394.82794948,52.56999393)(394.85795044,52.68)
\curveto(394.89794941,52.7999937)(394.94294936,52.91499358)(394.99295044,53.025)
\curveto(395.18294912,53.37499313)(395.42294888,53.64499285)(395.71295044,53.835)
\curveto(396.0029483,54.03499247)(396.36294794,54.19499231)(396.79295044,54.315)
\curveto(396.89294741,54.33499217)(396.99294731,54.34999215)(397.09295044,54.36)
\curveto(397.2029471,54.36999213)(397.31294699,54.38499211)(397.42295044,54.405)
\curveto(397.46294684,54.41499208)(397.52794678,54.41499208)(397.61795044,54.405)
\curveto(397.7079466,54.4049921)(397.76294654,54.41499208)(397.78295044,54.435)
\curveto(398.48294582,54.44499206)(399.09294521,54.36499214)(399.61295044,54.195)
\curveto(400.13294417,54.02499247)(400.49794381,53.6999928)(400.70795044,53.22)
\curveto(400.79794351,53.01999348)(400.84794346,52.78499371)(400.85795044,52.515)
\curveto(400.87794343,52.25499424)(400.88794342,51.97999452)(400.88795044,51.69)
\lineto(400.88795044,48.375)
\curveto(400.88794342,48.23499827)(400.89294341,48.0999984)(400.90295044,47.97)
\curveto(400.91294339,47.83999866)(400.94294336,47.73499877)(400.99295044,47.655)
\curveto(401.04294326,47.58499892)(401.1079432,47.53499896)(401.18795044,47.505)
\curveto(401.27794303,47.46499904)(401.36294294,47.43499906)(401.44295044,47.415)
\curveto(401.52294278,47.40499909)(401.58294272,47.35999914)(401.62295044,47.28)
\curveto(401.64294266,47.24999925)(401.65294265,47.21999928)(401.65295044,47.19)
\curveto(401.65294265,47.15999934)(401.65794265,47.11999938)(401.66795044,47.07)
\moveto(399.52295044,48.735)
\curveto(399.58294472,48.87499762)(399.61294469,49.03499747)(399.61295044,49.215)
\curveto(399.62294468,49.40499709)(399.62794468,49.5999969)(399.62795044,49.8)
\curveto(399.62794468,49.90999659)(399.62294468,50.00999649)(399.61295044,50.1)
\curveto(399.6029447,50.18999631)(399.56294474,50.25999624)(399.49295044,50.31)
\curveto(399.46294484,50.32999617)(399.39294491,50.33999616)(399.28295044,50.34)
\curveto(399.26294504,50.31999618)(399.22794508,50.30999619)(399.17795044,50.31)
\curveto(399.12794518,50.30999619)(399.08294522,50.2999962)(399.04295044,50.28)
\curveto(398.96294534,50.25999624)(398.87294543,50.23999626)(398.77295044,50.22)
\lineto(398.47295044,50.16)
\curveto(398.44294586,50.15999634)(398.4079459,50.15499634)(398.36795044,50.145)
\lineto(398.26295044,50.145)
\curveto(398.11294619,50.1049964)(397.94794636,50.07999642)(397.76795044,50.07)
\curveto(397.59794671,50.06999643)(397.43794687,50.04999645)(397.28795044,50.01)
\curveto(397.2079471,49.98999651)(397.13294717,49.96999653)(397.06295044,49.95)
\curveto(397.0029473,49.93999656)(396.93294737,49.92499658)(396.85295044,49.905)
\curveto(396.69294761,49.85499665)(396.54294776,49.78999671)(396.40295044,49.71)
\curveto(396.26294804,49.63999686)(396.14294816,49.54999695)(396.04295044,49.44)
\curveto(395.94294836,49.32999717)(395.86794844,49.19499731)(395.81795044,49.035)
\curveto(395.76794854,48.88499761)(395.74794856,48.6999978)(395.75795044,48.48)
\curveto(395.75794855,48.37999812)(395.77294853,48.28499821)(395.80295044,48.195)
\curveto(395.84294846,48.11499838)(395.88794842,48.03999846)(395.93795044,47.97)
\curveto(396.01794829,47.85999864)(396.12294818,47.76499874)(396.25295044,47.685)
\curveto(396.38294792,47.61499888)(396.52294778,47.55499895)(396.67295044,47.505)
\curveto(396.72294758,47.49499901)(396.77294753,47.48999901)(396.82295044,47.49)
\curveto(396.87294743,47.48999901)(396.92294738,47.48499902)(396.97295044,47.475)
\curveto(397.04294726,47.45499905)(397.12794718,47.43999906)(397.22795044,47.43)
\curveto(397.33794697,47.42999907)(397.42794688,47.43999906)(397.49795044,47.46)
\curveto(397.55794675,47.47999902)(397.61794669,47.48499902)(397.67795044,47.475)
\curveto(397.73794657,47.47499902)(397.79794651,47.48499902)(397.85795044,47.505)
\curveto(397.93794637,47.52499898)(398.01294629,47.53999896)(398.08295044,47.55)
\curveto(398.16294614,47.55999894)(398.23794607,47.57999892)(398.30795044,47.61)
\curveto(398.59794571,47.72999877)(398.84294546,47.87499862)(399.04295044,48.045)
\curveto(399.25294505,48.21499828)(399.41294489,48.44499805)(399.52295044,48.735)
}
}
{
\newrgbcolor{curcolor}{0 0 0}
\pscustom[linestyle=none,fillstyle=solid,fillcolor=curcolor]
{
\newpath
\moveto(403.78459106,56.58)
\curveto(403.93458905,56.57998992)(404.0845889,56.57498993)(404.23459106,56.565)
\curveto(404.3845886,56.56498993)(404.4895885,56.52498997)(404.54959106,56.445)
\curveto(404.59958839,56.38499011)(404.62458836,56.2999902)(404.62459106,56.19)
\curveto(404.63458835,56.08999041)(404.63958835,55.98499051)(404.63959106,55.875)
\lineto(404.63959106,55.005)
\curveto(404.63958835,54.92499157)(404.63458835,54.83999166)(404.62459106,54.75)
\curveto(404.62458836,54.66999183)(404.63458835,54.5999919)(404.65459106,54.54)
\curveto(404.69458829,54.3999921)(404.7845882,54.30999219)(404.92459106,54.27)
\curveto(404.97458801,54.25999224)(405.01958797,54.25499224)(405.05959106,54.255)
\lineto(405.20959106,54.255)
\lineto(405.61459106,54.255)
\curveto(405.77458721,54.26499224)(405.8895871,54.25499224)(405.95959106,54.225)
\curveto(406.04958694,54.16499234)(406.10958688,54.1049924)(406.13959106,54.045)
\curveto(406.15958683,54.0049925)(406.16958682,53.95999254)(406.16959106,53.91)
\lineto(406.16959106,53.76)
\curveto(406.16958682,53.64999285)(406.16458682,53.54499295)(406.15459106,53.445)
\curveto(406.14458684,53.35499314)(406.10958688,53.28499321)(406.04959106,53.235)
\curveto(405.989587,53.18499331)(405.90458708,53.15499334)(405.79459106,53.145)
\lineto(405.46459106,53.145)
\curveto(405.35458763,53.15499334)(405.24458774,53.15999334)(405.13459106,53.16)
\curveto(405.02458796,53.15999334)(404.92958806,53.14499335)(404.84959106,53.115)
\curveto(404.77958821,53.08499341)(404.72958826,53.03499347)(404.69959106,52.965)
\curveto(404.66958832,52.89499361)(404.64958834,52.80999369)(404.63959106,52.71)
\curveto(404.62958836,52.61999388)(404.62458836,52.51999398)(404.62459106,52.41)
\curveto(404.63458835,52.30999419)(404.63958835,52.20999429)(404.63959106,52.11)
\lineto(404.63959106,49.14)
\curveto(404.63958835,48.91999758)(404.63458835,48.68499781)(404.62459106,48.435)
\curveto(404.62458836,48.19499831)(404.66958832,48.00999849)(404.75959106,47.88)
\curveto(404.80958818,47.7999987)(404.87458811,47.74499875)(404.95459106,47.715)
\curveto(405.03458795,47.68499881)(405.12958786,47.65999884)(405.23959106,47.64)
\curveto(405.26958772,47.62999887)(405.29958769,47.62499888)(405.32959106,47.625)
\curveto(405.36958762,47.63499887)(405.40458758,47.63499887)(405.43459106,47.625)
\lineto(405.62959106,47.625)
\curveto(405.72958726,47.62499888)(405.81958717,47.61499888)(405.89959106,47.595)
\curveto(405.989587,47.58499892)(406.05458693,47.54999895)(406.09459106,47.49)
\curveto(406.11458687,47.45999904)(406.12958686,47.40499909)(406.13959106,47.325)
\curveto(406.15958683,47.25499925)(406.16958682,47.17999932)(406.16959106,47.1)
\curveto(406.17958681,47.01999948)(406.17958681,46.93999956)(406.16959106,46.86)
\curveto(406.15958683,46.78999971)(406.13958685,46.73499976)(406.10959106,46.695)
\curveto(406.06958692,46.62499988)(405.99458699,46.57499992)(405.88459106,46.545)
\curveto(405.80458718,46.52499998)(405.71458727,46.51499998)(405.61459106,46.515)
\curveto(405.51458747,46.52499998)(405.42458756,46.52999997)(405.34459106,46.53)
\curveto(405.2845877,46.52999997)(405.22458776,46.52499998)(405.16459106,46.515)
\curveto(405.10458788,46.51499998)(405.04958794,46.51999998)(404.99959106,46.53)
\lineto(404.81959106,46.53)
\curveto(404.76958822,46.53999996)(404.71958827,46.54499995)(404.66959106,46.545)
\curveto(404.62958836,46.55499995)(404.5845884,46.55999994)(404.53459106,46.56)
\curveto(404.33458865,46.60999989)(404.15958883,46.66499984)(404.00959106,46.725)
\curveto(403.86958912,46.78499972)(403.74958924,46.88999961)(403.64959106,47.04)
\curveto(403.50958948,47.23999926)(403.42958956,47.48999901)(403.40959106,47.79)
\curveto(403.3895896,48.0999984)(403.37958961,48.42999807)(403.37959106,48.78)
\lineto(403.37959106,52.71)
\curveto(403.34958964,52.83999366)(403.31958967,52.93499357)(403.28959106,52.995)
\curveto(403.26958972,53.05499344)(403.19958979,53.1049934)(403.07959106,53.145)
\curveto(403.03958995,53.15499334)(402.99958999,53.15499334)(402.95959106,53.145)
\curveto(402.91959007,53.13499337)(402.87959011,53.13999336)(402.83959106,53.16)
\lineto(402.59959106,53.16)
\curveto(402.46959052,53.15999334)(402.35959063,53.16999333)(402.26959106,53.19)
\curveto(402.1895908,53.21999328)(402.13459085,53.27999322)(402.10459106,53.37)
\curveto(402.0845909,53.40999309)(402.06959092,53.45499304)(402.05959106,53.505)
\lineto(402.05959106,53.655)
\curveto(402.05959093,53.79499271)(402.06959092,53.90999259)(402.08959106,54)
\curveto(402.10959088,54.0999924)(402.16959082,54.17499233)(402.26959106,54.225)
\curveto(402.37959061,54.26499224)(402.51959047,54.27499223)(402.68959106,54.255)
\curveto(402.86959012,54.23499227)(403.01958997,54.24499225)(403.13959106,54.285)
\curveto(403.22958976,54.33499217)(403.29958969,54.4049921)(403.34959106,54.495)
\curveto(403.36958962,54.55499194)(403.37958961,54.62999187)(403.37959106,54.72)
\lineto(403.37959106,54.975)
\lineto(403.37959106,55.905)
\lineto(403.37959106,56.145)
\curveto(403.37958961,56.23499027)(403.3895896,56.30999019)(403.40959106,56.37)
\curveto(403.44958954,56.44999005)(403.52458946,56.51498998)(403.63459106,56.565)
\curveto(403.66458932,56.56498993)(403.6895893,56.56498993)(403.70959106,56.565)
\curveto(403.73958925,56.57498993)(403.76458922,56.57998992)(403.78459106,56.58)
}
}
{
\newrgbcolor{curcolor}{0 0 0}
\pscustom[linestyle=none,fillstyle=solid,fillcolor=curcolor]
{
\newpath
\moveto(414.30638794,50.685)
\curveto(414.32638025,50.58499591)(414.32638025,50.46999603)(414.30638794,50.34)
\curveto(414.29638028,50.21999628)(414.26638031,50.13499637)(414.21638794,50.085)
\curveto(414.16638041,50.04499645)(414.09138049,50.01499648)(413.99138794,49.995)
\curveto(413.90138068,49.98499651)(413.79638078,49.97999652)(413.67638794,49.98)
\lineto(413.31638794,49.98)
\curveto(413.19638138,49.98999651)(413.09138149,49.99499651)(413.00138794,49.995)
\lineto(409.16138794,49.995)
\curveto(409.0813855,49.99499651)(409.00138558,49.98999651)(408.92138794,49.98)
\curveto(408.84138574,49.97999652)(408.7763858,49.96499654)(408.72638794,49.935)
\curveto(408.68638589,49.91499658)(408.64638593,49.87499662)(408.60638794,49.815)
\curveto(408.58638599,49.78499671)(408.56638601,49.73999676)(408.54638794,49.68)
\curveto(408.52638605,49.62999687)(408.52638605,49.57999692)(408.54638794,49.53)
\curveto(408.55638602,49.47999702)(408.56138602,49.43499707)(408.56138794,49.395)
\curveto(408.56138602,49.35499715)(408.56638601,49.31499718)(408.57638794,49.275)
\curveto(408.59638598,49.19499731)(408.61638596,49.10999739)(408.63638794,49.02)
\curveto(408.65638592,48.93999756)(408.68638589,48.85999764)(408.72638794,48.78)
\curveto(408.95638562,48.23999826)(409.33638524,47.85499865)(409.86638794,47.625)
\curveto(409.92638465,47.59499891)(409.99138459,47.56999893)(410.06138794,47.55)
\lineto(410.27138794,47.49)
\curveto(410.30138428,47.47999902)(410.35138423,47.47499902)(410.42138794,47.475)
\curveto(410.56138402,47.43499906)(410.74638383,47.41499908)(410.97638794,47.415)
\curveto(411.20638337,47.41499908)(411.39138319,47.43499906)(411.53138794,47.475)
\curveto(411.67138291,47.51499898)(411.79638278,47.55499895)(411.90638794,47.595)
\curveto(412.02638255,47.64499885)(412.13638244,47.70499879)(412.23638794,47.775)
\curveto(412.34638223,47.84499865)(412.44138214,47.92499858)(412.52138794,48.015)
\curveto(412.60138198,48.11499838)(412.67138191,48.21999828)(412.73138794,48.33)
\curveto(412.79138179,48.42999807)(412.84138174,48.53499797)(412.88138794,48.645)
\curveto(412.93138165,48.75499775)(413.01138157,48.83499767)(413.12138794,48.885)
\curveto(413.16138142,48.90499759)(413.22638135,48.91999758)(413.31638794,48.93)
\curveto(413.40638117,48.93999756)(413.49638108,48.93999756)(413.58638794,48.93)
\curveto(413.6763809,48.92999757)(413.76138082,48.92499758)(413.84138794,48.915)
\curveto(413.92138066,48.90499759)(413.9763806,48.88499761)(414.00638794,48.855)
\curveto(414.10638047,48.78499771)(414.13138045,48.66999783)(414.08138794,48.51)
\curveto(414.00138058,48.23999826)(413.89638068,47.9999985)(413.76638794,47.79)
\curveto(413.56638101,47.46999903)(413.33638124,47.20499929)(413.07638794,46.995)
\curveto(412.82638175,46.79499971)(412.50638207,46.62999987)(412.11638794,46.5)
\curveto(412.01638256,46.46000004)(411.91638266,46.43500006)(411.81638794,46.425)
\curveto(411.71638286,46.40500009)(411.61138297,46.38500012)(411.50138794,46.365)
\curveto(411.45138313,46.35500015)(411.40138318,46.35000015)(411.35138794,46.35)
\curveto(411.31138327,46.35000015)(411.26638331,46.34500015)(411.21638794,46.335)
\lineto(411.06638794,46.335)
\curveto(411.01638356,46.32500018)(410.95638362,46.32000018)(410.88638794,46.32)
\curveto(410.82638375,46.32000018)(410.7763838,46.32500018)(410.73638794,46.335)
\lineto(410.60138794,46.335)
\curveto(410.55138403,46.34500015)(410.50638407,46.35000015)(410.46638794,46.35)
\curveto(410.42638415,46.35000015)(410.38638419,46.35500015)(410.34638794,46.365)
\curveto(410.29638428,46.37500012)(410.24138434,46.38500012)(410.18138794,46.395)
\curveto(410.12138446,46.39500011)(410.06638451,46.4000001)(410.01638794,46.41)
\curveto(409.92638465,46.43000007)(409.83638474,46.45500005)(409.74638794,46.485)
\curveto(409.65638492,46.50499999)(409.57138501,46.52999997)(409.49138794,46.56)
\curveto(409.45138513,46.57999992)(409.41638516,46.58999991)(409.38638794,46.59)
\curveto(409.35638522,46.5999999)(409.32138526,46.61499988)(409.28138794,46.635)
\curveto(409.13138545,46.70499979)(408.97138561,46.78999971)(408.80138794,46.89)
\curveto(408.51138607,47.07999942)(408.26138632,47.30999919)(408.05138794,47.58)
\curveto(407.85138673,47.85999864)(407.6813869,48.16999833)(407.54138794,48.51)
\curveto(407.49138709,48.61999788)(407.45138713,48.73499777)(407.42138794,48.855)
\curveto(407.40138718,48.97499752)(407.37138721,49.09499741)(407.33138794,49.215)
\curveto(407.32138726,49.25499725)(407.31638726,49.28999721)(407.31638794,49.32)
\curveto(407.31638726,49.34999715)(407.31138727,49.38999711)(407.30138794,49.44)
\curveto(407.2813873,49.51999698)(407.26638731,49.60499689)(407.25638794,49.695)
\curveto(407.24638733,49.78499671)(407.23138735,49.87499662)(407.21138794,49.965)
\lineto(407.21138794,50.175)
\curveto(407.20138738,50.21499628)(407.19138739,50.26999623)(407.18138794,50.34)
\curveto(407.1813874,50.41999608)(407.18638739,50.48499601)(407.19638794,50.535)
\lineto(407.19638794,50.7)
\curveto(407.21638736,50.74999575)(407.22138736,50.7999957)(407.21138794,50.85)
\curveto(407.21138737,50.90999559)(407.21638736,50.96499554)(407.22638794,51.015)
\curveto(407.26638731,51.17499532)(407.29638728,51.33499517)(407.31638794,51.495)
\curveto(407.34638723,51.65499484)(407.39138719,51.8049947)(407.45138794,51.945)
\curveto(407.50138708,52.05499444)(407.54638703,52.16499434)(407.58638794,52.275)
\curveto(407.63638694,52.39499411)(407.69138689,52.50999399)(407.75138794,52.62)
\curveto(407.97138661,52.96999353)(408.22138636,53.26999323)(408.50138794,53.52)
\curveto(408.7813858,53.77999272)(409.12638545,53.99499251)(409.53638794,54.165)
\curveto(409.65638492,54.21499228)(409.7763848,54.24999225)(409.89638794,54.27)
\curveto(410.02638455,54.2999922)(410.16138442,54.32999217)(410.30138794,54.36)
\curveto(410.35138423,54.36999213)(410.39638418,54.37499213)(410.43638794,54.375)
\curveto(410.4763841,54.38499211)(410.52138406,54.38999211)(410.57138794,54.39)
\curveto(410.59138399,54.3999921)(410.61638396,54.3999921)(410.64638794,54.39)
\curveto(410.6763839,54.37999212)(410.70138388,54.38499211)(410.72138794,54.405)
\curveto(411.14138344,54.41499208)(411.50638307,54.36999213)(411.81638794,54.27)
\curveto(412.12638245,54.17999232)(412.40638217,54.05499244)(412.65638794,53.895)
\curveto(412.70638187,53.87499263)(412.74638183,53.84499265)(412.77638794,53.805)
\curveto(412.80638177,53.77499273)(412.84138174,53.74999275)(412.88138794,53.73)
\curveto(412.96138162,53.66999283)(413.04138154,53.5999929)(413.12138794,53.52)
\curveto(413.21138137,53.43999306)(413.28638129,53.35999314)(413.34638794,53.28)
\curveto(413.50638107,53.06999343)(413.64138094,52.86999363)(413.75138794,52.68)
\curveto(413.82138076,52.56999393)(413.8763807,52.44999405)(413.91638794,52.32)
\curveto(413.95638062,52.18999431)(414.00138058,52.05999444)(414.05138794,51.93)
\curveto(414.10138048,51.7999947)(414.13638044,51.66499484)(414.15638794,51.525)
\curveto(414.18638039,51.38499511)(414.22138036,51.24499525)(414.26138794,51.105)
\curveto(414.27138031,51.03499547)(414.2763803,50.96499554)(414.27638794,50.895)
\lineto(414.30638794,50.685)
\moveto(412.85138794,51.195)
\curveto(412.8813817,51.23499527)(412.90638167,51.28499521)(412.92638794,51.345)
\curveto(412.94638163,51.41499508)(412.94638163,51.48499501)(412.92638794,51.555)
\curveto(412.86638171,51.77499472)(412.7813818,51.97999452)(412.67138794,52.17)
\curveto(412.53138205,52.3999941)(412.3763822,52.59499391)(412.20638794,52.755)
\curveto(412.03638254,52.91499358)(411.81638276,53.04999345)(411.54638794,53.16)
\curveto(411.4763831,53.17999332)(411.40638317,53.19499331)(411.33638794,53.205)
\curveto(411.26638331,53.22499327)(411.19138339,53.24499325)(411.11138794,53.265)
\curveto(411.03138355,53.28499321)(410.94638363,53.29499321)(410.85638794,53.295)
\lineto(410.60138794,53.295)
\curveto(410.57138401,53.27499323)(410.53638404,53.26499324)(410.49638794,53.265)
\curveto(410.45638412,53.27499323)(410.42138416,53.27499323)(410.39138794,53.265)
\lineto(410.15138794,53.205)
\curveto(410.0813845,53.19499331)(410.01138457,53.17999332)(409.94138794,53.16)
\curveto(409.65138493,53.03999346)(409.41638516,52.88999361)(409.23638794,52.71)
\curveto(409.06638551,52.52999397)(408.91138567,52.3049942)(408.77138794,52.035)
\curveto(408.74138584,51.98499451)(408.71138587,51.91999458)(408.68138794,51.84)
\curveto(408.65138593,51.76999473)(408.62638595,51.68999481)(408.60638794,51.6)
\curveto(408.58638599,51.50999499)(408.581386,51.42499508)(408.59138794,51.345)
\curveto(408.60138598,51.26499524)(408.63638594,51.2049953)(408.69638794,51.165)
\curveto(408.7763858,51.1049954)(408.91138567,51.07499542)(409.10138794,51.075)
\curveto(409.30138528,51.08499541)(409.47138511,51.08999541)(409.61138794,51.09)
\lineto(411.89138794,51.09)
\curveto(412.04138254,51.08999541)(412.22138236,51.08499541)(412.43138794,51.075)
\curveto(412.64138194,51.07499542)(412.7813818,51.11499538)(412.85138794,51.195)
}
}
{
\newrgbcolor{curcolor}{0 0 0}
\pscustom[linestyle=none,fillstyle=solid,fillcolor=curcolor]
{
\newpath
\moveto(419.25802856,54.42)
\curveto(419.48802377,54.41999208)(419.61802364,54.35999214)(419.64802856,54.24)
\curveto(419.67802358,54.12999237)(419.69302357,53.96499254)(419.69302856,53.745)
\lineto(419.69302856,53.46)
\curveto(419.69302357,53.36999313)(419.66802359,53.29499321)(419.61802856,53.235)
\curveto(419.5580237,53.15499334)(419.47302379,53.10999339)(419.36302856,53.1)
\curveto(419.25302401,53.0999934)(419.14302412,53.08499341)(419.03302856,53.055)
\curveto(418.89302437,53.02499347)(418.7580245,52.99499351)(418.62802856,52.965)
\curveto(418.50802475,52.93499357)(418.39302487,52.89499361)(418.28302856,52.845)
\curveto(417.99302527,52.71499378)(417.7580255,52.53499397)(417.57802856,52.305)
\curveto(417.39802586,52.08499441)(417.24302602,51.82999467)(417.11302856,51.54)
\curveto(417.07302619,51.42999507)(417.04302622,51.31499518)(417.02302856,51.195)
\curveto(417.00302626,51.08499541)(416.97802628,50.96999553)(416.94802856,50.85)
\curveto(416.93802632,50.7999957)(416.93302633,50.74999575)(416.93302856,50.7)
\curveto(416.94302632,50.64999585)(416.94302632,50.5999959)(416.93302856,50.55)
\curveto(416.90302636,50.42999607)(416.88802637,50.28999621)(416.88802856,50.13)
\curveto(416.89802636,49.97999652)(416.90302636,49.83499667)(416.90302856,49.695)
\lineto(416.90302856,47.85)
\lineto(416.90302856,47.505)
\curveto(416.90302636,47.38499912)(416.89802636,47.26999923)(416.88802856,47.16)
\curveto(416.87802638,47.04999945)(416.87302639,46.95499955)(416.87302856,46.875)
\curveto(416.88302638,46.79499971)(416.8630264,46.72499978)(416.81302856,46.665)
\curveto(416.7630265,46.59499991)(416.68302658,46.55499995)(416.57302856,46.545)
\curveto(416.47302679,46.53499996)(416.3630269,46.52999997)(416.24302856,46.53)
\lineto(415.97302856,46.53)
\curveto(415.92302734,46.54999995)(415.87302739,46.56499994)(415.82302856,46.575)
\curveto(415.78302748,46.59499991)(415.75302751,46.61999988)(415.73302856,46.65)
\curveto(415.68302758,46.71999978)(415.65302761,46.80499969)(415.64302856,46.905)
\lineto(415.64302856,47.235)
\lineto(415.64302856,48.39)
\lineto(415.64302856,52.545)
\lineto(415.64302856,53.58)
\lineto(415.64302856,53.88)
\curveto(415.65302761,53.97999252)(415.68302758,54.06499244)(415.73302856,54.135)
\curveto(415.7630275,54.17499233)(415.81302745,54.2049923)(415.88302856,54.225)
\curveto(415.9630273,54.24499225)(416.04802721,54.25499224)(416.13802856,54.255)
\curveto(416.22802703,54.26499224)(416.31802694,54.26499224)(416.40802856,54.255)
\curveto(416.49802676,54.24499225)(416.56802669,54.22999227)(416.61802856,54.21)
\curveto(416.69802656,54.17999232)(416.74802651,54.11999238)(416.76802856,54.03)
\curveto(416.79802646,53.94999255)(416.81302645,53.85999264)(416.81302856,53.76)
\lineto(416.81302856,53.46)
\curveto(416.81302645,53.35999314)(416.83302643,53.26999323)(416.87302856,53.19)
\curveto(416.88302638,53.16999333)(416.89302637,53.15499334)(416.90302856,53.145)
\lineto(416.94802856,53.1)
\curveto(417.0580262,53.0999934)(417.14802611,53.14499335)(417.21802856,53.235)
\curveto(417.28802597,53.33499317)(417.34802591,53.41499308)(417.39802856,53.475)
\lineto(417.48802856,53.565)
\curveto(417.57802568,53.67499283)(417.70302556,53.78999271)(417.86302856,53.91)
\curveto(418.02302524,54.02999247)(418.17302509,54.11999238)(418.31302856,54.18)
\curveto(418.40302486,54.22999227)(418.49802476,54.26499224)(418.59802856,54.285)
\curveto(418.69802456,54.31499218)(418.80302446,54.34499215)(418.91302856,54.375)
\curveto(418.97302429,54.38499211)(419.03302423,54.38999211)(419.09302856,54.39)
\curveto(419.15302411,54.3999921)(419.20802405,54.40999209)(419.25802856,54.42)
}
}
{
\newrgbcolor{curcolor}{0 0 0}
\pscustom[linestyle=none,fillstyle=solid,fillcolor=curcolor]
{
\newpath
\moveto(420.90779419,55.74)
\curveto(420.82779307,55.7999907)(420.78279311,55.9049906)(420.77279419,56.055)
\lineto(420.77279419,56.52)
\lineto(420.77279419,56.775)
\curveto(420.77279312,56.86498964)(420.78779311,56.93998956)(420.81779419,57)
\curveto(420.85779304,57.07998942)(420.93779296,57.13998936)(421.05779419,57.18)
\curveto(421.07779282,57.18998931)(421.0977928,57.18998931)(421.11779419,57.18)
\curveto(421.14779275,57.17998932)(421.17279272,57.18498931)(421.19279419,57.195)
\curveto(421.36279253,57.1949893)(421.52279237,57.18998931)(421.67279419,57.18)
\curveto(421.82279207,57.16998933)(421.92279197,57.10998939)(421.97279419,57)
\curveto(422.00279189,56.93998956)(422.01779188,56.86498964)(422.01779419,56.775)
\lineto(422.01779419,56.52)
\curveto(422.01779188,56.33999016)(422.01279188,56.16999033)(422.00279419,56.01)
\curveto(422.00279189,55.84999065)(421.93779196,55.74499076)(421.80779419,55.695)
\curveto(421.75779214,55.67499083)(421.70279219,55.66499084)(421.64279419,55.665)
\lineto(421.47779419,55.665)
\lineto(421.16279419,55.665)
\curveto(421.06279283,55.66499084)(420.97779292,55.68999081)(420.90779419,55.74)
\moveto(422.01779419,47.235)
\lineto(422.01779419,46.92)
\curveto(422.02779187,46.81999968)(422.00779189,46.73999976)(421.95779419,46.68)
\curveto(421.92779197,46.61999988)(421.88279201,46.57999992)(421.82279419,46.56)
\curveto(421.76279213,46.54999995)(421.6927922,46.53499996)(421.61279419,46.515)
\lineto(421.38779419,46.515)
\curveto(421.25779264,46.51499998)(421.14279275,46.51999998)(421.04279419,46.53)
\curveto(420.95279294,46.54999995)(420.88279301,46.5999999)(420.83279419,46.68)
\curveto(420.7927931,46.73999976)(420.77279312,46.81499968)(420.77279419,46.905)
\lineto(420.77279419,47.19)
\lineto(420.77279419,53.535)
\lineto(420.77279419,53.85)
\curveto(420.77279312,53.95999254)(420.7977931,54.04499245)(420.84779419,54.105)
\curveto(420.87779302,54.15499234)(420.91779298,54.18499231)(420.96779419,54.195)
\curveto(421.01779288,54.2049923)(421.07279282,54.21999228)(421.13279419,54.24)
\curveto(421.15279274,54.23999226)(421.17279272,54.23499227)(421.19279419,54.225)
\curveto(421.22279267,54.22499227)(421.24779265,54.22999227)(421.26779419,54.24)
\curveto(421.3977925,54.23999226)(421.52779237,54.23499227)(421.65779419,54.225)
\curveto(421.7977921,54.22499227)(421.892792,54.18499231)(421.94279419,54.105)
\curveto(421.9927919,54.04499245)(422.01779188,53.96499254)(422.01779419,53.865)
\lineto(422.01779419,53.58)
\lineto(422.01779419,47.235)
}
}
{
\newrgbcolor{curcolor}{0 0 0}
\pscustom[linestyle=none,fillstyle=solid,fillcolor=curcolor]
{
\newpath
\moveto(430.84763794,47.07)
\curveto(430.87763011,46.90999959)(430.86263012,46.77499972)(430.80263794,46.665)
\curveto(430.74263024,46.56499994)(430.66263032,46.49000001)(430.56263794,46.44)
\curveto(430.51263047,46.42000008)(430.45763053,46.41000009)(430.39763794,46.41)
\curveto(430.34763064,46.41000009)(430.29263069,46.4000001)(430.23263794,46.38)
\curveto(430.01263097,46.33000017)(429.79263119,46.34500015)(429.57263794,46.425)
\curveto(429.36263162,46.49500001)(429.21763177,46.58499992)(429.13763794,46.695)
\curveto(429.0876319,46.76499974)(429.04263194,46.84499965)(429.00263794,46.935)
\curveto(428.96263202,47.03499946)(428.91263207,47.11499938)(428.85263794,47.175)
\curveto(428.83263215,47.19499931)(428.80763218,47.21499928)(428.77763794,47.235)
\curveto(428.75763223,47.25499925)(428.72763226,47.25999924)(428.68763794,47.25)
\curveto(428.57763241,47.21999928)(428.47263251,47.16499934)(428.37263794,47.085)
\curveto(428.2826327,47.00499949)(428.19263279,46.93499956)(428.10263794,46.875)
\curveto(427.97263301,46.79499971)(427.83263315,46.71999978)(427.68263794,46.65)
\curveto(427.53263345,46.58999991)(427.37263361,46.53499996)(427.20263794,46.485)
\curveto(427.10263388,46.45500005)(426.99263399,46.43500006)(426.87263794,46.425)
\curveto(426.76263422,46.41500008)(426.65263433,46.4000001)(426.54263794,46.38)
\curveto(426.49263449,46.37000013)(426.44763454,46.36500014)(426.40763794,46.365)
\lineto(426.30263794,46.365)
\curveto(426.19263479,46.34500015)(426.0876349,46.34500015)(425.98763794,46.365)
\lineto(425.85263794,46.365)
\curveto(425.80263518,46.37500012)(425.75263523,46.38000012)(425.70263794,46.38)
\curveto(425.65263533,46.38000012)(425.60763538,46.39000011)(425.56763794,46.41)
\curveto(425.52763546,46.42000008)(425.49263549,46.42500008)(425.46263794,46.425)
\curveto(425.44263554,46.41500008)(425.41763557,46.41500008)(425.38763794,46.425)
\lineto(425.14763794,46.485)
\curveto(425.06763592,46.49500001)(424.99263599,46.51499998)(424.92263794,46.545)
\curveto(424.62263636,46.67499982)(424.37763661,46.81999968)(424.18763794,46.98)
\curveto(424.00763698,47.14999935)(423.85763713,47.38499912)(423.73763794,47.685)
\curveto(423.64763734,47.90499859)(423.60263738,48.16999833)(423.60263794,48.48)
\lineto(423.60263794,48.795)
\curveto(423.61263737,48.84499765)(423.61763737,48.89499761)(423.61763794,48.945)
\lineto(423.64763794,49.125)
\lineto(423.76763794,49.455)
\curveto(423.80763718,49.56499694)(423.85763713,49.66499684)(423.91763794,49.755)
\curveto(424.09763689,50.04499645)(424.34263664,50.25999624)(424.65263794,50.4)
\curveto(424.96263602,50.53999596)(425.30263568,50.66499584)(425.67263794,50.775)
\curveto(425.81263517,50.81499568)(425.95763503,50.84499565)(426.10763794,50.865)
\curveto(426.25763473,50.88499561)(426.40763458,50.90999559)(426.55763794,50.94)
\curveto(426.62763436,50.95999554)(426.69263429,50.96999553)(426.75263794,50.97)
\curveto(426.82263416,50.96999553)(426.89763409,50.97999552)(426.97763794,51)
\curveto(427.04763394,51.01999548)(427.11763387,51.02999547)(427.18763794,51.03)
\curveto(427.25763373,51.03999546)(427.33263365,51.05499544)(427.41263794,51.075)
\curveto(427.66263332,51.13499537)(427.89763309,51.18499531)(428.11763794,51.225)
\curveto(428.33763265,51.27499522)(428.51263247,51.38999511)(428.64263794,51.57)
\curveto(428.70263228,51.64999485)(428.75263223,51.74999475)(428.79263794,51.87)
\curveto(428.83263215,51.9999945)(428.83263215,52.13999436)(428.79263794,52.29)
\curveto(428.73263225,52.52999397)(428.64263234,52.71999378)(428.52263794,52.86)
\curveto(428.41263257,52.9999935)(428.25263273,53.10999339)(428.04263794,53.19)
\curveto(427.92263306,53.23999326)(427.77763321,53.27499323)(427.60763794,53.295)
\curveto(427.44763354,53.31499318)(427.27763371,53.32499317)(427.09763794,53.325)
\curveto(426.91763407,53.32499317)(426.74263424,53.31499318)(426.57263794,53.295)
\curveto(426.40263458,53.27499323)(426.25763473,53.24499325)(426.13763794,53.205)
\curveto(425.96763502,53.14499335)(425.80263518,53.05999344)(425.64263794,52.95)
\curveto(425.56263542,52.88999361)(425.4876355,52.80999369)(425.41763794,52.71)
\curveto(425.35763563,52.61999388)(425.30263568,52.51999398)(425.25263794,52.41)
\curveto(425.22263576,52.32999417)(425.19263579,52.24499425)(425.16263794,52.155)
\curveto(425.14263584,52.06499444)(425.09763589,51.99499451)(425.02763794,51.945)
\curveto(424.987636,51.91499458)(424.91763607,51.88999461)(424.81763794,51.87)
\curveto(424.72763626,51.85999464)(424.63263635,51.85499464)(424.53263794,51.855)
\curveto(424.43263655,51.85499464)(424.33263665,51.85999464)(424.23263794,51.87)
\curveto(424.14263684,51.88999461)(424.07763691,51.91499458)(424.03763794,51.945)
\curveto(423.99763699,51.97499452)(423.96763702,52.02499447)(423.94763794,52.095)
\curveto(423.92763706,52.16499434)(423.92763706,52.23999426)(423.94763794,52.32)
\curveto(423.97763701,52.44999405)(424.00763698,52.56999393)(424.03763794,52.68)
\curveto(424.07763691,52.7999937)(424.12263686,52.91499358)(424.17263794,53.025)
\curveto(424.36263662,53.37499313)(424.60263638,53.64499285)(424.89263794,53.835)
\curveto(425.1826358,54.03499247)(425.54263544,54.19499231)(425.97263794,54.315)
\curveto(426.07263491,54.33499217)(426.17263481,54.34999215)(426.27263794,54.36)
\curveto(426.3826346,54.36999213)(426.49263449,54.38499211)(426.60263794,54.405)
\curveto(426.64263434,54.41499208)(426.70763428,54.41499208)(426.79763794,54.405)
\curveto(426.8876341,54.4049921)(426.94263404,54.41499208)(426.96263794,54.435)
\curveto(427.66263332,54.44499206)(428.27263271,54.36499214)(428.79263794,54.195)
\curveto(429.31263167,54.02499247)(429.67763131,53.6999928)(429.88763794,53.22)
\curveto(429.97763101,53.01999348)(430.02763096,52.78499371)(430.03763794,52.515)
\curveto(430.05763093,52.25499424)(430.06763092,51.97999452)(430.06763794,51.69)
\lineto(430.06763794,48.375)
\curveto(430.06763092,48.23499827)(430.07263091,48.0999984)(430.08263794,47.97)
\curveto(430.09263089,47.83999866)(430.12263086,47.73499877)(430.17263794,47.655)
\curveto(430.22263076,47.58499892)(430.2876307,47.53499896)(430.36763794,47.505)
\curveto(430.45763053,47.46499904)(430.54263044,47.43499906)(430.62263794,47.415)
\curveto(430.70263028,47.40499909)(430.76263022,47.35999914)(430.80263794,47.28)
\curveto(430.82263016,47.24999925)(430.83263015,47.21999928)(430.83263794,47.19)
\curveto(430.83263015,47.15999934)(430.83763015,47.11999938)(430.84763794,47.07)
\moveto(428.70263794,48.735)
\curveto(428.76263222,48.87499762)(428.79263219,49.03499747)(428.79263794,49.215)
\curveto(428.80263218,49.40499709)(428.80763218,49.5999969)(428.80763794,49.8)
\curveto(428.80763218,49.90999659)(428.80263218,50.00999649)(428.79263794,50.1)
\curveto(428.7826322,50.18999631)(428.74263224,50.25999624)(428.67263794,50.31)
\curveto(428.64263234,50.32999617)(428.57263241,50.33999616)(428.46263794,50.34)
\curveto(428.44263254,50.31999618)(428.40763258,50.30999619)(428.35763794,50.31)
\curveto(428.30763268,50.30999619)(428.26263272,50.2999962)(428.22263794,50.28)
\curveto(428.14263284,50.25999624)(428.05263293,50.23999626)(427.95263794,50.22)
\lineto(427.65263794,50.16)
\curveto(427.62263336,50.15999634)(427.5876334,50.15499634)(427.54763794,50.145)
\lineto(427.44263794,50.145)
\curveto(427.29263369,50.1049964)(427.12763386,50.07999642)(426.94763794,50.07)
\curveto(426.77763421,50.06999643)(426.61763437,50.04999645)(426.46763794,50.01)
\curveto(426.3876346,49.98999651)(426.31263467,49.96999653)(426.24263794,49.95)
\curveto(426.1826348,49.93999656)(426.11263487,49.92499658)(426.03263794,49.905)
\curveto(425.87263511,49.85499665)(425.72263526,49.78999671)(425.58263794,49.71)
\curveto(425.44263554,49.63999686)(425.32263566,49.54999695)(425.22263794,49.44)
\curveto(425.12263586,49.32999717)(425.04763594,49.19499731)(424.99763794,49.035)
\curveto(424.94763604,48.88499761)(424.92763606,48.6999978)(424.93763794,48.48)
\curveto(424.93763605,48.37999812)(424.95263603,48.28499821)(424.98263794,48.195)
\curveto(425.02263596,48.11499838)(425.06763592,48.03999846)(425.11763794,47.97)
\curveto(425.19763579,47.85999864)(425.30263568,47.76499874)(425.43263794,47.685)
\curveto(425.56263542,47.61499888)(425.70263528,47.55499895)(425.85263794,47.505)
\curveto(425.90263508,47.49499901)(425.95263503,47.48999901)(426.00263794,47.49)
\curveto(426.05263493,47.48999901)(426.10263488,47.48499902)(426.15263794,47.475)
\curveto(426.22263476,47.45499905)(426.30763468,47.43999906)(426.40763794,47.43)
\curveto(426.51763447,47.42999907)(426.60763438,47.43999906)(426.67763794,47.46)
\curveto(426.73763425,47.47999902)(426.79763419,47.48499902)(426.85763794,47.475)
\curveto(426.91763407,47.47499902)(426.97763401,47.48499902)(427.03763794,47.505)
\curveto(427.11763387,47.52499898)(427.19263379,47.53999896)(427.26263794,47.55)
\curveto(427.34263364,47.55999894)(427.41763357,47.57999892)(427.48763794,47.61)
\curveto(427.77763321,47.72999877)(428.02263296,47.87499862)(428.22263794,48.045)
\curveto(428.43263255,48.21499828)(428.59263239,48.44499805)(428.70263794,48.735)
}
}
{
\newrgbcolor{curcolor}{0 0 0}
\pscustom[linestyle=none,fillstyle=solid,fillcolor=curcolor]
{
\newpath
\moveto(434.44927856,54.42)
\curveto(435.1692745,54.42999207)(435.77427389,54.34499215)(436.26427856,54.165)
\curveto(436.75427291,53.99499251)(437.13427253,53.68999281)(437.40427856,53.25)
\curveto(437.47427219,53.13999336)(437.52927214,53.02499347)(437.56927856,52.905)
\curveto(437.60927206,52.79499371)(437.64927202,52.66999383)(437.68927856,52.53)
\curveto(437.70927196,52.45999404)(437.71427195,52.38499411)(437.70427856,52.305)
\curveto(437.69427197,52.23499427)(437.67927199,52.17999432)(437.65927856,52.14)
\curveto(437.63927203,52.11999438)(437.61427205,52.0999944)(437.58427856,52.08)
\curveto(437.55427211,52.06999443)(437.52927214,52.05499444)(437.50927856,52.035)
\curveto(437.45927221,52.01499448)(437.40927226,52.00999449)(437.35927856,52.02)
\curveto(437.30927236,52.02999447)(437.25927241,52.02999447)(437.20927856,52.02)
\curveto(437.12927254,51.9999945)(437.02427264,51.99499451)(436.89427856,52.005)
\curveto(436.7642729,52.02499447)(436.67427299,52.04999445)(436.62427856,52.08)
\curveto(436.54427312,52.12999437)(436.48927318,52.19499431)(436.45927856,52.275)
\curveto(436.43927323,52.36499414)(436.40427326,52.44999405)(436.35427856,52.53)
\curveto(436.2642734,52.68999381)(436.13927353,52.83499367)(435.97927856,52.965)
\curveto(435.8692738,53.04499345)(435.74927392,53.1049934)(435.61927856,53.145)
\curveto(435.48927418,53.18499331)(435.34927432,53.22499327)(435.19927856,53.265)
\curveto(435.14927452,53.28499321)(435.09927457,53.28999321)(435.04927856,53.28)
\curveto(434.99927467,53.27999322)(434.94927472,53.28499321)(434.89927856,53.295)
\curveto(434.83927483,53.31499318)(434.7642749,53.32499317)(434.67427856,53.325)
\curveto(434.58427508,53.32499317)(434.50927516,53.31499318)(434.44927856,53.295)
\lineto(434.35927856,53.295)
\lineto(434.20927856,53.265)
\curveto(434.15927551,53.26499324)(434.10927556,53.25999324)(434.05927856,53.25)
\curveto(433.79927587,53.18999331)(433.58427608,53.1049934)(433.41427856,52.995)
\curveto(433.24427642,52.88499361)(433.12927654,52.6999938)(433.06927856,52.44)
\curveto(433.04927662,52.36999413)(433.04427662,52.2999942)(433.05427856,52.23)
\curveto(433.07427659,52.15999434)(433.09427657,52.0999944)(433.11427856,52.05)
\curveto(433.17427649,51.8999946)(433.24427642,51.78999471)(433.32427856,51.72)
\curveto(433.41427625,51.65999484)(433.52427614,51.58999491)(433.65427856,51.51)
\curveto(433.81427585,51.40999509)(433.99427567,51.33499517)(434.19427856,51.285)
\curveto(434.39427527,51.24499525)(434.59427507,51.19499531)(434.79427856,51.135)
\curveto(434.92427474,51.09499541)(435.05427461,51.06499544)(435.18427856,51.045)
\curveto(435.31427435,51.02499548)(435.44427422,50.99499551)(435.57427856,50.955)
\curveto(435.78427388,50.89499561)(435.98927368,50.83499567)(436.18927856,50.775)
\curveto(436.38927328,50.72499578)(436.58927308,50.65999584)(436.78927856,50.58)
\lineto(436.93927856,50.52)
\curveto(436.98927268,50.499996)(437.03927263,50.47499602)(437.08927856,50.445)
\curveto(437.28927238,50.32499618)(437.4642722,50.18999631)(437.61427856,50.04)
\curveto(437.7642719,49.88999661)(437.88927178,49.6999968)(437.98927856,49.47)
\curveto(438.00927166,49.3999971)(438.02927164,49.30499719)(438.04927856,49.185)
\curveto(438.0692716,49.11499738)(438.07927159,49.03999746)(438.07927856,48.96)
\curveto(438.08927158,48.88999761)(438.09427157,48.80999769)(438.09427856,48.72)
\lineto(438.09427856,48.57)
\curveto(438.07427159,48.499998)(438.0642716,48.42999807)(438.06427856,48.36)
\curveto(438.0642716,48.28999821)(438.05427161,48.21999828)(438.03427856,48.15)
\curveto(438.00427166,48.03999846)(437.9692717,47.93499857)(437.92927856,47.835)
\curveto(437.88927178,47.73499877)(437.84427182,47.64499885)(437.79427856,47.565)
\curveto(437.63427203,47.30499919)(437.42927224,47.09499941)(437.17927856,46.935)
\curveto(436.92927274,46.78499972)(436.64927302,46.65499985)(436.33927856,46.545)
\curveto(436.24927342,46.51499998)(436.15427351,46.49500001)(436.05427856,46.485)
\curveto(435.9642737,46.46500004)(435.87427379,46.44000006)(435.78427856,46.41)
\curveto(435.68427398,46.39000011)(435.58427408,46.38000012)(435.48427856,46.38)
\curveto(435.38427428,46.38000012)(435.28427438,46.37000013)(435.18427856,46.35)
\lineto(435.03427856,46.35)
\curveto(434.98427468,46.34000016)(434.91427475,46.33500016)(434.82427856,46.335)
\curveto(434.73427493,46.33500016)(434.664275,46.34000016)(434.61427856,46.35)
\lineto(434.44927856,46.35)
\curveto(434.38927528,46.37000013)(434.32427534,46.38000012)(434.25427856,46.38)
\curveto(434.18427548,46.37000013)(434.12427554,46.37500012)(434.07427856,46.395)
\curveto(434.02427564,46.40500009)(433.95927571,46.41000009)(433.87927856,46.41)
\lineto(433.63927856,46.47)
\curveto(433.5692761,46.48000002)(433.49427617,46.5)(433.41427856,46.53)
\curveto(433.10427656,46.62999987)(432.83427683,46.75499975)(432.60427856,46.905)
\curveto(432.37427729,47.05499945)(432.17427749,47.24999925)(432.00427856,47.49)
\curveto(431.91427775,47.61999888)(431.83927783,47.75499875)(431.77927856,47.895)
\curveto(431.71927795,48.03499847)(431.664278,48.18999831)(431.61427856,48.36)
\curveto(431.59427807,48.41999808)(431.58427808,48.48999801)(431.58427856,48.57)
\curveto(431.59427807,48.65999784)(431.60927806,48.72999777)(431.62927856,48.78)
\curveto(431.65927801,48.81999768)(431.70927796,48.85999764)(431.77927856,48.9)
\curveto(431.82927784,48.91999758)(431.89927777,48.92999757)(431.98927856,48.93)
\curveto(432.07927759,48.93999756)(432.1692775,48.93999756)(432.25927856,48.93)
\curveto(432.34927732,48.91999758)(432.43427723,48.90499759)(432.51427856,48.885)
\curveto(432.60427706,48.87499762)(432.664277,48.85999764)(432.69427856,48.84)
\curveto(432.7642769,48.78999771)(432.80927686,48.71499778)(432.82927856,48.615)
\curveto(432.85927681,48.52499798)(432.89427677,48.43999806)(432.93427856,48.36)
\curveto(433.03427663,48.13999836)(433.1692765,47.96999853)(433.33927856,47.85)
\curveto(433.45927621,47.75999874)(433.59427607,47.68999881)(433.74427856,47.64)
\curveto(433.89427577,47.58999891)(434.05427561,47.53999896)(434.22427856,47.49)
\lineto(434.53927856,47.445)
\lineto(434.62927856,47.445)
\curveto(434.69927497,47.42499908)(434.78927488,47.41499908)(434.89927856,47.415)
\curveto(435.01927465,47.41499908)(435.11927455,47.42499908)(435.19927856,47.445)
\curveto(435.2692744,47.44499905)(435.32427434,47.44999905)(435.36427856,47.46)
\curveto(435.42427424,47.46999903)(435.48427418,47.47499902)(435.54427856,47.475)
\curveto(435.60427406,47.48499902)(435.65927401,47.49499901)(435.70927856,47.505)
\curveto(435.99927367,47.58499892)(436.22927344,47.68999881)(436.39927856,47.82)
\curveto(436.5692731,47.94999855)(436.68927298,48.16999833)(436.75927856,48.48)
\curveto(436.77927289,48.52999797)(436.78427288,48.58499791)(436.77427856,48.645)
\curveto(436.7642729,48.70499779)(436.75427291,48.74999775)(436.74427856,48.78)
\curveto(436.69427297,48.96999753)(436.62427304,49.10999739)(436.53427856,49.2)
\curveto(436.44427322,49.2999972)(436.32927334,49.38999711)(436.18927856,49.47)
\curveto(436.09927357,49.52999697)(435.99927367,49.57999692)(435.88927856,49.62)
\lineto(435.55927856,49.74)
\curveto(435.52927414,49.74999675)(435.49927417,49.75499675)(435.46927856,49.755)
\curveto(435.44927422,49.75499675)(435.42427424,49.76499674)(435.39427856,49.785)
\curveto(435.05427461,49.89499661)(434.69927497,49.97499652)(434.32927856,50.025)
\curveto(433.9692757,50.08499641)(433.62927604,50.17999632)(433.30927856,50.31)
\curveto(433.20927646,50.34999615)(433.11427655,50.38499611)(433.02427856,50.415)
\curveto(432.93427673,50.44499605)(432.84927682,50.48499601)(432.76927856,50.535)
\curveto(432.57927709,50.64499585)(432.40427726,50.76999573)(432.24427856,50.91)
\curveto(432.08427758,51.04999545)(431.95927771,51.22499528)(431.86927856,51.435)
\curveto(431.83927783,51.504995)(431.81427785,51.57499492)(431.79427856,51.645)
\curveto(431.78427788,51.71499478)(431.7692779,51.78999471)(431.74927856,51.87)
\curveto(431.71927795,51.98999451)(431.70927796,52.12499438)(431.71927856,52.275)
\curveto(431.72927794,52.43499407)(431.74427792,52.56999393)(431.76427856,52.68)
\curveto(431.78427788,52.72999377)(431.79427787,52.76999373)(431.79427856,52.8)
\curveto(431.80427786,52.83999366)(431.81927785,52.87999362)(431.83927856,52.92)
\curveto(431.92927774,53.14999335)(432.04927762,53.34999315)(432.19927856,53.52)
\curveto(432.35927731,53.68999281)(432.53927713,53.83999266)(432.73927856,53.97)
\curveto(432.88927678,54.05999244)(433.05427661,54.12999237)(433.23427856,54.18)
\curveto(433.41427625,54.23999226)(433.60427606,54.29499221)(433.80427856,54.345)
\curveto(433.87427579,54.35499214)(433.93927573,54.36499214)(433.99927856,54.375)
\curveto(434.0692756,54.38499211)(434.14427552,54.39499211)(434.22427856,54.405)
\curveto(434.25427541,54.41499208)(434.29427537,54.41499208)(434.34427856,54.405)
\curveto(434.39427527,54.39499211)(434.42927524,54.3999921)(434.44927856,54.42)
}
}
{
\newrgbcolor{curcolor}{0 0 0}
\pscustom[linestyle=none,fillstyle=solid,fillcolor=curcolor]
{
\newpath
\moveto(73.44210083,86.16295776)
\lineto(73.44210083,85.90795776)
\curveto(73.45209313,85.827953)(73.44709313,85.75295307)(73.42710083,85.68295776)
\lineto(73.42710083,85.44295776)
\lineto(73.42710083,85.27795776)
\curveto(73.40709317,85.17795365)(73.39709318,85.07295375)(73.39710083,84.96295776)
\curveto(73.39709318,84.86295396)(73.38709319,84.76295406)(73.36710083,84.66295776)
\lineto(73.36710083,84.51295776)
\curveto(73.33709324,84.37295445)(73.31709326,84.23295459)(73.30710083,84.09295776)
\curveto(73.29709328,83.96295486)(73.27209331,83.83295499)(73.23210083,83.70295776)
\curveto(73.21209337,83.6229552)(73.19209339,83.53795529)(73.17210083,83.44795776)
\lineto(73.11210083,83.20795776)
\lineto(72.99210083,82.90795776)
\curveto(72.96209362,82.81795601)(72.92709365,82.7279561)(72.88710083,82.63795776)
\curveto(72.78709379,82.41795641)(72.65209393,82.20295662)(72.48210083,81.99295776)
\curveto(72.32209426,81.78295704)(72.14709443,81.61295721)(71.95710083,81.48295776)
\curveto(71.90709467,81.44295738)(71.84709473,81.40295742)(71.77710083,81.36295776)
\curveto(71.71709486,81.33295749)(71.65709492,81.29795753)(71.59710083,81.25795776)
\curveto(71.51709506,81.20795762)(71.42209516,81.16795766)(71.31210083,81.13795776)
\curveto(71.20209538,81.10795772)(71.09709548,81.07795775)(70.99710083,81.04795776)
\curveto(70.88709569,81.00795782)(70.7770958,80.98295784)(70.66710083,80.97295776)
\curveto(70.55709602,80.96295786)(70.44209614,80.94795788)(70.32210083,80.92795776)
\curveto(70.2820963,80.91795791)(70.23709634,80.91795791)(70.18710083,80.92795776)
\curveto(70.14709643,80.9279579)(70.10709647,80.9229579)(70.06710083,80.91295776)
\curveto(70.02709655,80.90295792)(69.97209661,80.89795793)(69.90210083,80.89795776)
\curveto(69.83209675,80.89795793)(69.7820968,80.90295792)(69.75210083,80.91295776)
\curveto(69.70209688,80.93295789)(69.65709692,80.93795789)(69.61710083,80.92795776)
\curveto(69.577097,80.91795791)(69.54209704,80.91795791)(69.51210083,80.92795776)
\lineto(69.42210083,80.92795776)
\curveto(69.36209722,80.94795788)(69.29709728,80.96295786)(69.22710083,80.97295776)
\curveto(69.16709741,80.97295785)(69.10209748,80.97795785)(69.03210083,80.98795776)
\curveto(68.86209772,81.03795779)(68.70209788,81.08795774)(68.55210083,81.13795776)
\curveto(68.40209818,81.18795764)(68.25709832,81.25295757)(68.11710083,81.33295776)
\curveto(68.06709851,81.37295745)(68.01209857,81.40295742)(67.95210083,81.42295776)
\curveto(67.90209868,81.45295737)(67.85209873,81.48795734)(67.80210083,81.52795776)
\curveto(67.56209902,81.70795712)(67.36209922,81.9279569)(67.20210083,82.18795776)
\curveto(67.04209954,82.44795638)(66.90209968,82.73295609)(66.78210083,83.04295776)
\curveto(66.72209986,83.18295564)(66.6770999,83.3229555)(66.64710083,83.46295776)
\curveto(66.61709996,83.61295521)(66.5821,83.76795506)(66.54210083,83.92795776)
\curveto(66.52210006,84.03795479)(66.50710007,84.14795468)(66.49710083,84.25795776)
\curveto(66.48710009,84.36795446)(66.47210011,84.47795435)(66.45210083,84.58795776)
\curveto(66.44210014,84.6279542)(66.43710014,84.66795416)(66.43710083,84.70795776)
\curveto(66.44710013,84.74795408)(66.44710013,84.78795404)(66.43710083,84.82795776)
\curveto(66.42710015,84.87795395)(66.42210016,84.9279539)(66.42210083,84.97795776)
\lineto(66.42210083,85.14295776)
\curveto(66.40210018,85.19295363)(66.39710018,85.24295358)(66.40710083,85.29295776)
\curveto(66.41710016,85.35295347)(66.41710016,85.40795342)(66.40710083,85.45795776)
\curveto(66.39710018,85.49795333)(66.39710018,85.54295328)(66.40710083,85.59295776)
\curveto(66.41710016,85.64295318)(66.41210017,85.69295313)(66.39210083,85.74295776)
\curveto(66.37210021,85.81295301)(66.36710021,85.88795294)(66.37710083,85.96795776)
\curveto(66.38710019,86.05795277)(66.39210019,86.14295268)(66.39210083,86.22295776)
\curveto(66.39210019,86.31295251)(66.38710019,86.41295241)(66.37710083,86.52295776)
\curveto(66.36710021,86.64295218)(66.37210021,86.74295208)(66.39210083,86.82295776)
\lineto(66.39210083,87.10795776)
\lineto(66.43710083,87.73795776)
\curveto(66.44710013,87.83795099)(66.45710012,87.93295089)(66.46710083,88.02295776)
\lineto(66.49710083,88.32295776)
\curveto(66.51710006,88.37295045)(66.52210006,88.4229504)(66.51210083,88.47295776)
\curveto(66.51210007,88.53295029)(66.52210006,88.58795024)(66.54210083,88.63795776)
\curveto(66.59209999,88.80795002)(66.63209995,88.97294985)(66.66210083,89.13295776)
\curveto(66.69209989,89.30294952)(66.74209984,89.46294936)(66.81210083,89.61295776)
\curveto(67.00209958,90.07294875)(67.22209936,90.44794838)(67.47210083,90.73795776)
\curveto(67.73209885,91.0279478)(68.09209849,91.27294755)(68.55210083,91.47295776)
\curveto(68.6820979,91.5229473)(68.81209777,91.55794727)(68.94210083,91.57795776)
\curveto(69.0820975,91.59794723)(69.22209736,91.6229472)(69.36210083,91.65295776)
\curveto(69.43209715,91.66294716)(69.49709708,91.66794716)(69.55710083,91.66795776)
\curveto(69.61709696,91.66794716)(69.6820969,91.67294715)(69.75210083,91.68295776)
\curveto(70.582096,91.70294712)(71.25209533,91.55294727)(71.76210083,91.23295776)
\curveto(72.27209431,90.9229479)(72.65209393,90.48294834)(72.90210083,89.91295776)
\curveto(72.95209363,89.79294903)(72.99709358,89.66794916)(73.03710083,89.53795776)
\curveto(73.0770935,89.40794942)(73.12209346,89.27294955)(73.17210083,89.13295776)
\curveto(73.19209339,89.05294977)(73.20709337,88.96794986)(73.21710083,88.87795776)
\lineto(73.27710083,88.63795776)
\curveto(73.30709327,88.5279503)(73.32209326,88.41795041)(73.32210083,88.30795776)
\curveto(73.33209325,88.19795063)(73.34709323,88.08795074)(73.36710083,87.97795776)
\curveto(73.38709319,87.9279509)(73.39209319,87.88295094)(73.38210083,87.84295776)
\curveto(73.3820932,87.80295102)(73.38709319,87.76295106)(73.39710083,87.72295776)
\curveto(73.40709317,87.67295115)(73.40709317,87.61795121)(73.39710083,87.55795776)
\curveto(73.39709318,87.50795132)(73.40209318,87.45795137)(73.41210083,87.40795776)
\lineto(73.41210083,87.27295776)
\curveto(73.43209315,87.21295161)(73.43209315,87.14295168)(73.41210083,87.06295776)
\curveto(73.40209318,86.99295183)(73.40709317,86.9279519)(73.42710083,86.86795776)
\curveto(73.43709314,86.83795199)(73.44209314,86.79795203)(73.44210083,86.74795776)
\lineto(73.44210083,86.62795776)
\lineto(73.44210083,86.16295776)
\moveto(71.89710083,83.83795776)
\curveto(71.99709458,84.15795467)(72.05709452,84.5229543)(72.07710083,84.93295776)
\curveto(72.09709448,85.34295348)(72.10709447,85.75295307)(72.10710083,86.16295776)
\curveto(72.10709447,86.59295223)(72.09709448,87.01295181)(72.07710083,87.42295776)
\curveto(72.05709452,87.83295099)(72.01209457,88.21795061)(71.94210083,88.57795776)
\curveto(71.87209471,88.93794989)(71.76209482,89.25794957)(71.61210083,89.53795776)
\curveto(71.47209511,89.827949)(71.2770953,90.06294876)(71.02710083,90.24295776)
\curveto(70.86709571,90.35294847)(70.68709589,90.43294839)(70.48710083,90.48295776)
\curveto(70.28709629,90.54294828)(70.04209654,90.57294825)(69.75210083,90.57295776)
\curveto(69.73209685,90.55294827)(69.69709688,90.54294828)(69.64710083,90.54295776)
\curveto(69.59709698,90.55294827)(69.55709702,90.55294827)(69.52710083,90.54295776)
\curveto(69.44709713,90.5229483)(69.37209721,90.50294832)(69.30210083,90.48295776)
\curveto(69.24209734,90.47294835)(69.1770974,90.45294837)(69.10710083,90.42295776)
\curveto(68.83709774,90.30294852)(68.61709796,90.13294869)(68.44710083,89.91295776)
\curveto(68.28709829,89.70294912)(68.15209843,89.45794937)(68.04210083,89.17795776)
\curveto(67.99209859,89.06794976)(67.95209863,88.94794988)(67.92210083,88.81795776)
\curveto(67.90209868,88.69795013)(67.8770987,88.57295025)(67.84710083,88.44295776)
\curveto(67.82709875,88.39295043)(67.81709876,88.33795049)(67.81710083,88.27795776)
\curveto(67.81709876,88.2279506)(67.81209877,88.17795065)(67.80210083,88.12795776)
\curveto(67.79209879,88.03795079)(67.7820988,87.94295088)(67.77210083,87.84295776)
\curveto(67.76209882,87.75295107)(67.75209883,87.65795117)(67.74210083,87.55795776)
\curveto(67.74209884,87.47795135)(67.73709884,87.39295143)(67.72710083,87.30295776)
\lineto(67.72710083,87.06295776)
\lineto(67.72710083,86.88295776)
\curveto(67.71709886,86.85295197)(67.71209887,86.81795201)(67.71210083,86.77795776)
\lineto(67.71210083,86.64295776)
\lineto(67.71210083,86.19295776)
\curveto(67.71209887,86.11295271)(67.70709887,86.0279528)(67.69710083,85.93795776)
\curveto(67.69709888,85.85795297)(67.70709887,85.78295304)(67.72710083,85.71295776)
\lineto(67.72710083,85.44295776)
\curveto(67.72709885,85.4229534)(67.72209886,85.39295343)(67.71210083,85.35295776)
\curveto(67.71209887,85.3229535)(67.71709886,85.29795353)(67.72710083,85.27795776)
\curveto(67.73709884,85.17795365)(67.74209884,85.07795375)(67.74210083,84.97795776)
\curveto(67.75209883,84.88795394)(67.76209882,84.78795404)(67.77210083,84.67795776)
\curveto(67.80209878,84.55795427)(67.81709876,84.43295439)(67.81710083,84.30295776)
\curveto(67.82709875,84.18295464)(67.85209873,84.06795476)(67.89210083,83.95795776)
\curveto(67.97209861,83.65795517)(68.05709852,83.39295543)(68.14710083,83.16295776)
\curveto(68.24709833,82.93295589)(68.39209819,82.71795611)(68.58210083,82.51795776)
\curveto(68.79209779,82.31795651)(69.05709752,82.16795666)(69.37710083,82.06795776)
\curveto(69.41709716,82.04795678)(69.45209713,82.03795679)(69.48210083,82.03795776)
\curveto(69.52209706,82.04795678)(69.56709701,82.04295678)(69.61710083,82.02295776)
\curveto(69.65709692,82.01295681)(69.72709685,82.00295682)(69.82710083,81.99295776)
\curveto(69.93709664,81.98295684)(70.02209656,81.98795684)(70.08210083,82.00795776)
\curveto(70.15209643,82.0279568)(70.22209636,82.03795679)(70.29210083,82.03795776)
\curveto(70.36209622,82.04795678)(70.42709615,82.06295676)(70.48710083,82.08295776)
\curveto(70.68709589,82.14295668)(70.86709571,82.2279566)(71.02710083,82.33795776)
\curveto(71.05709552,82.35795647)(71.0820955,82.37795645)(71.10210083,82.39795776)
\lineto(71.16210083,82.45795776)
\curveto(71.20209538,82.47795635)(71.25209533,82.51795631)(71.31210083,82.57795776)
\curveto(71.41209517,82.71795611)(71.49709508,82.84795598)(71.56710083,82.96795776)
\curveto(71.63709494,83.08795574)(71.70709487,83.23295559)(71.77710083,83.40295776)
\curveto(71.80709477,83.47295535)(71.82709475,83.54295528)(71.83710083,83.61295776)
\curveto(71.85709472,83.68295514)(71.8770947,83.75795507)(71.89710083,83.83795776)
}
}
{
\newrgbcolor{curcolor}{0 0 0}
\pscustom[linestyle=none,fillstyle=solid,fillcolor=curcolor]
{
\newpath
\moveto(62.1524762,165.96866333)
\curveto(62.25247134,165.96865271)(62.34747125,165.95865272)(62.4374762,165.93866333)
\curveto(62.52747107,165.92865275)(62.592471,165.89865278)(62.6324762,165.84866333)
\curveto(62.6924709,165.76865291)(62.72247087,165.66365302)(62.7224762,165.53366333)
\lineto(62.7224762,165.14366333)
\lineto(62.7224762,163.64366333)
\lineto(62.7224762,157.25366333)
\lineto(62.7224762,156.08366333)
\lineto(62.7224762,155.76866333)
\curveto(62.73247086,155.66866301)(62.71747088,155.58866309)(62.6774762,155.52866333)
\curveto(62.62747097,155.44866323)(62.55247104,155.39866328)(62.4524762,155.37866333)
\curveto(62.36247123,155.36866331)(62.25247134,155.36366332)(62.1224762,155.36366333)
\lineto(61.8974762,155.36366333)
\curveto(61.81747178,155.3836633)(61.74747185,155.39866328)(61.6874762,155.40866333)
\curveto(61.62747197,155.42866325)(61.57747202,155.46866321)(61.5374762,155.52866333)
\curveto(61.4974721,155.58866309)(61.47747212,155.66366302)(61.4774762,155.75366333)
\lineto(61.4774762,156.05366333)
\lineto(61.4774762,157.14866333)
\lineto(61.4774762,162.48866333)
\curveto(61.45747214,162.5786561)(61.44247215,162.65365603)(61.4324762,162.71366333)
\curveto(61.43247216,162.7836559)(61.40247219,162.84365584)(61.3424762,162.89366333)
\curveto(61.27247232,162.94365574)(61.18247241,162.96865571)(61.0724762,162.96866333)
\curveto(60.97247262,162.9786557)(60.86247273,162.9836557)(60.7424762,162.98366333)
\lineto(59.6024762,162.98366333)
\lineto(59.1074762,162.98366333)
\curveto(58.94747465,162.99365569)(58.83747476,163.05365563)(58.7774762,163.16366333)
\curveto(58.75747484,163.19365549)(58.74747485,163.22365546)(58.7474762,163.25366333)
\curveto(58.74747485,163.29365539)(58.74247485,163.33865534)(58.7324762,163.38866333)
\curveto(58.71247488,163.50865517)(58.71747488,163.61865506)(58.7474762,163.71866333)
\curveto(58.78747481,163.81865486)(58.84247475,163.88865479)(58.9124762,163.92866333)
\curveto(58.9924746,163.9786547)(59.11247448,164.00365468)(59.2724762,164.00366333)
\curveto(59.43247416,164.00365468)(59.56747403,164.01865466)(59.6774762,164.04866333)
\curveto(59.72747387,164.05865462)(59.78247381,164.06365462)(59.8424762,164.06366333)
\curveto(59.90247369,164.07365461)(59.96247363,164.08865459)(60.0224762,164.10866333)
\curveto(60.17247342,164.15865452)(60.31747328,164.20865447)(60.4574762,164.25866333)
\curveto(60.597473,164.31865436)(60.73247286,164.38865429)(60.8624762,164.46866333)
\curveto(61.00247259,164.55865412)(61.12247247,164.66365402)(61.2224762,164.78366333)
\curveto(61.32247227,164.90365378)(61.41747218,165.03365365)(61.5074762,165.17366333)
\curveto(61.56747203,165.27365341)(61.61247198,165.3836533)(61.6424762,165.50366333)
\curveto(61.68247191,165.62365306)(61.73247186,165.72865295)(61.7924762,165.81866333)
\curveto(61.84247175,165.8786528)(61.91247168,165.91865276)(62.0024762,165.93866333)
\curveto(62.02247157,165.94865273)(62.04747155,165.95365273)(62.0774762,165.95366333)
\curveto(62.10747149,165.95365273)(62.13247146,165.95865272)(62.1524762,165.96866333)
}
}
{
\newrgbcolor{curcolor}{0 0 0}
\pscustom[linestyle=none,fillstyle=solid,fillcolor=curcolor]
{
\newpath
\moveto(73.44208557,160.44866333)
\lineto(73.44208557,160.19366333)
\curveto(73.45207787,160.11365857)(73.44707787,160.03865864)(73.42708557,159.96866333)
\lineto(73.42708557,159.72866333)
\lineto(73.42708557,159.56366333)
\curveto(73.40707791,159.46365922)(73.39707792,159.35865932)(73.39708557,159.24866333)
\curveto(73.39707792,159.14865953)(73.38707793,159.04865963)(73.36708557,158.94866333)
\lineto(73.36708557,158.79866333)
\curveto(73.33707798,158.65866002)(73.317078,158.51866016)(73.30708557,158.37866333)
\curveto(73.29707802,158.24866043)(73.27207805,158.11866056)(73.23208557,157.98866333)
\curveto(73.21207811,157.90866077)(73.19207813,157.82366086)(73.17208557,157.73366333)
\lineto(73.11208557,157.49366333)
\lineto(72.99208557,157.19366333)
\curveto(72.96207836,157.10366158)(72.92707839,157.01366167)(72.88708557,156.92366333)
\curveto(72.78707853,156.70366198)(72.65207867,156.48866219)(72.48208557,156.27866333)
\curveto(72.322079,156.06866261)(72.14707917,155.89866278)(71.95708557,155.76866333)
\curveto(71.90707941,155.72866295)(71.84707947,155.68866299)(71.77708557,155.64866333)
\curveto(71.7170796,155.61866306)(71.65707966,155.5836631)(71.59708557,155.54366333)
\curveto(71.5170798,155.49366319)(71.4220799,155.45366323)(71.31208557,155.42366333)
\curveto(71.20208012,155.39366329)(71.09708022,155.36366332)(70.99708557,155.33366333)
\curveto(70.88708043,155.29366339)(70.77708054,155.26866341)(70.66708557,155.25866333)
\curveto(70.55708076,155.24866343)(70.44208088,155.23366345)(70.32208557,155.21366333)
\curveto(70.28208104,155.20366348)(70.23708108,155.20366348)(70.18708557,155.21366333)
\curveto(70.14708117,155.21366347)(70.10708121,155.20866347)(70.06708557,155.19866333)
\curveto(70.02708129,155.18866349)(69.97208135,155.1836635)(69.90208557,155.18366333)
\curveto(69.83208149,155.1836635)(69.78208154,155.18866349)(69.75208557,155.19866333)
\curveto(69.70208162,155.21866346)(69.65708166,155.22366346)(69.61708557,155.21366333)
\curveto(69.57708174,155.20366348)(69.54208178,155.20366348)(69.51208557,155.21366333)
\lineto(69.42208557,155.21366333)
\curveto(69.36208196,155.23366345)(69.29708202,155.24866343)(69.22708557,155.25866333)
\curveto(69.16708215,155.25866342)(69.10208222,155.26366342)(69.03208557,155.27366333)
\curveto(68.86208246,155.32366336)(68.70208262,155.37366331)(68.55208557,155.42366333)
\curveto(68.40208292,155.47366321)(68.25708306,155.53866314)(68.11708557,155.61866333)
\curveto(68.06708325,155.65866302)(68.01208331,155.68866299)(67.95208557,155.70866333)
\curveto(67.90208342,155.73866294)(67.85208347,155.77366291)(67.80208557,155.81366333)
\curveto(67.56208376,155.99366269)(67.36208396,156.21366247)(67.20208557,156.47366333)
\curveto(67.04208428,156.73366195)(66.90208442,157.01866166)(66.78208557,157.32866333)
\curveto(66.7220846,157.46866121)(66.67708464,157.60866107)(66.64708557,157.74866333)
\curveto(66.6170847,157.89866078)(66.58208474,158.05366063)(66.54208557,158.21366333)
\curveto(66.5220848,158.32366036)(66.50708481,158.43366025)(66.49708557,158.54366333)
\curveto(66.48708483,158.65366003)(66.47208485,158.76365992)(66.45208557,158.87366333)
\curveto(66.44208488,158.91365977)(66.43708488,158.95365973)(66.43708557,158.99366333)
\curveto(66.44708487,159.03365965)(66.44708487,159.07365961)(66.43708557,159.11366333)
\curveto(66.42708489,159.16365952)(66.4220849,159.21365947)(66.42208557,159.26366333)
\lineto(66.42208557,159.42866333)
\curveto(66.40208492,159.4786592)(66.39708492,159.52865915)(66.40708557,159.57866333)
\curveto(66.4170849,159.63865904)(66.4170849,159.69365899)(66.40708557,159.74366333)
\curveto(66.39708492,159.7836589)(66.39708492,159.82865885)(66.40708557,159.87866333)
\curveto(66.4170849,159.92865875)(66.41208491,159.9786587)(66.39208557,160.02866333)
\curveto(66.37208495,160.09865858)(66.36708495,160.17365851)(66.37708557,160.25366333)
\curveto(66.38708493,160.34365834)(66.39208493,160.42865825)(66.39208557,160.50866333)
\curveto(66.39208493,160.59865808)(66.38708493,160.69865798)(66.37708557,160.80866333)
\curveto(66.36708495,160.92865775)(66.37208495,161.02865765)(66.39208557,161.10866333)
\lineto(66.39208557,161.39366333)
\lineto(66.43708557,162.02366333)
\curveto(66.44708487,162.12365656)(66.45708486,162.21865646)(66.46708557,162.30866333)
\lineto(66.49708557,162.60866333)
\curveto(66.5170848,162.65865602)(66.5220848,162.70865597)(66.51208557,162.75866333)
\curveto(66.51208481,162.81865586)(66.5220848,162.87365581)(66.54208557,162.92366333)
\curveto(66.59208473,163.09365559)(66.63208469,163.25865542)(66.66208557,163.41866333)
\curveto(66.69208463,163.58865509)(66.74208458,163.74865493)(66.81208557,163.89866333)
\curveto(67.00208432,164.35865432)(67.2220841,164.73365395)(67.47208557,165.02366333)
\curveto(67.73208359,165.31365337)(68.09208323,165.55865312)(68.55208557,165.75866333)
\curveto(68.68208264,165.80865287)(68.81208251,165.84365284)(68.94208557,165.86366333)
\curveto(69.08208224,165.8836528)(69.2220821,165.90865277)(69.36208557,165.93866333)
\curveto(69.43208189,165.94865273)(69.49708182,165.95365273)(69.55708557,165.95366333)
\curveto(69.6170817,165.95365273)(69.68208164,165.95865272)(69.75208557,165.96866333)
\curveto(70.58208074,165.98865269)(71.25208007,165.83865284)(71.76208557,165.51866333)
\curveto(72.27207905,165.20865347)(72.65207867,164.76865391)(72.90208557,164.19866333)
\curveto(72.95207837,164.0786546)(72.99707832,163.95365473)(73.03708557,163.82366333)
\curveto(73.07707824,163.69365499)(73.1220782,163.55865512)(73.17208557,163.41866333)
\curveto(73.19207813,163.33865534)(73.20707811,163.25365543)(73.21708557,163.16366333)
\lineto(73.27708557,162.92366333)
\curveto(73.30707801,162.81365587)(73.322078,162.70365598)(73.32208557,162.59366333)
\curveto(73.33207799,162.4836562)(73.34707797,162.37365631)(73.36708557,162.26366333)
\curveto(73.38707793,162.21365647)(73.39207793,162.16865651)(73.38208557,162.12866333)
\curveto(73.38207794,162.08865659)(73.38707793,162.04865663)(73.39708557,162.00866333)
\curveto(73.40707791,161.95865672)(73.40707791,161.90365678)(73.39708557,161.84366333)
\curveto(73.39707792,161.79365689)(73.40207792,161.74365694)(73.41208557,161.69366333)
\lineto(73.41208557,161.55866333)
\curveto(73.43207789,161.49865718)(73.43207789,161.42865725)(73.41208557,161.34866333)
\curveto(73.40207792,161.2786574)(73.40707791,161.21365747)(73.42708557,161.15366333)
\curveto(73.43707788,161.12365756)(73.44207788,161.0836576)(73.44208557,161.03366333)
\lineto(73.44208557,160.91366333)
\lineto(73.44208557,160.44866333)
\moveto(71.89708557,158.12366333)
\curveto(71.99707932,158.44366024)(72.05707926,158.80865987)(72.07708557,159.21866333)
\curveto(72.09707922,159.62865905)(72.10707921,160.03865864)(72.10708557,160.44866333)
\curveto(72.10707921,160.8786578)(72.09707922,161.29865738)(72.07708557,161.70866333)
\curveto(72.05707926,162.11865656)(72.01207931,162.50365618)(71.94208557,162.86366333)
\curveto(71.87207945,163.22365546)(71.76207956,163.54365514)(71.61208557,163.82366333)
\curveto(71.47207985,164.11365457)(71.27708004,164.34865433)(71.02708557,164.52866333)
\curveto(70.86708045,164.63865404)(70.68708063,164.71865396)(70.48708557,164.76866333)
\curveto(70.28708103,164.82865385)(70.04208128,164.85865382)(69.75208557,164.85866333)
\curveto(69.73208159,164.83865384)(69.69708162,164.82865385)(69.64708557,164.82866333)
\curveto(69.59708172,164.83865384)(69.55708176,164.83865384)(69.52708557,164.82866333)
\curveto(69.44708187,164.80865387)(69.37208195,164.78865389)(69.30208557,164.76866333)
\curveto(69.24208208,164.75865392)(69.17708214,164.73865394)(69.10708557,164.70866333)
\curveto(68.83708248,164.58865409)(68.6170827,164.41865426)(68.44708557,164.19866333)
\curveto(68.28708303,163.98865469)(68.15208317,163.74365494)(68.04208557,163.46366333)
\curveto(67.99208333,163.35365533)(67.95208337,163.23365545)(67.92208557,163.10366333)
\curveto(67.90208342,162.9836557)(67.87708344,162.85865582)(67.84708557,162.72866333)
\curveto(67.82708349,162.678656)(67.8170835,162.62365606)(67.81708557,162.56366333)
\curveto(67.8170835,162.51365617)(67.81208351,162.46365622)(67.80208557,162.41366333)
\curveto(67.79208353,162.32365636)(67.78208354,162.22865645)(67.77208557,162.12866333)
\curveto(67.76208356,162.03865664)(67.75208357,161.94365674)(67.74208557,161.84366333)
\curveto(67.74208358,161.76365692)(67.73708358,161.678657)(67.72708557,161.58866333)
\lineto(67.72708557,161.34866333)
\lineto(67.72708557,161.16866333)
\curveto(67.7170836,161.13865754)(67.71208361,161.10365758)(67.71208557,161.06366333)
\lineto(67.71208557,160.92866333)
\lineto(67.71208557,160.47866333)
\curveto(67.71208361,160.39865828)(67.70708361,160.31365837)(67.69708557,160.22366333)
\curveto(67.69708362,160.14365854)(67.70708361,160.06865861)(67.72708557,159.99866333)
\lineto(67.72708557,159.72866333)
\curveto(67.72708359,159.70865897)(67.7220836,159.678659)(67.71208557,159.63866333)
\curveto(67.71208361,159.60865907)(67.7170836,159.5836591)(67.72708557,159.56366333)
\curveto(67.73708358,159.46365922)(67.74208358,159.36365932)(67.74208557,159.26366333)
\curveto(67.75208357,159.17365951)(67.76208356,159.07365961)(67.77208557,158.96366333)
\curveto(67.80208352,158.84365984)(67.8170835,158.71865996)(67.81708557,158.58866333)
\curveto(67.82708349,158.46866021)(67.85208347,158.35366033)(67.89208557,158.24366333)
\curveto(67.97208335,157.94366074)(68.05708326,157.678661)(68.14708557,157.44866333)
\curveto(68.24708307,157.21866146)(68.39208293,157.00366168)(68.58208557,156.80366333)
\curveto(68.79208253,156.60366208)(69.05708226,156.45366223)(69.37708557,156.35366333)
\curveto(69.4170819,156.33366235)(69.45208187,156.32366236)(69.48208557,156.32366333)
\curveto(69.5220818,156.33366235)(69.56708175,156.32866235)(69.61708557,156.30866333)
\curveto(69.65708166,156.29866238)(69.72708159,156.28866239)(69.82708557,156.27866333)
\curveto(69.93708138,156.26866241)(70.0220813,156.27366241)(70.08208557,156.29366333)
\curveto(70.15208117,156.31366237)(70.2220811,156.32366236)(70.29208557,156.32366333)
\curveto(70.36208096,156.33366235)(70.42708089,156.34866233)(70.48708557,156.36866333)
\curveto(70.68708063,156.42866225)(70.86708045,156.51366217)(71.02708557,156.62366333)
\curveto(71.05708026,156.64366204)(71.08208024,156.66366202)(71.10208557,156.68366333)
\lineto(71.16208557,156.74366333)
\curveto(71.20208012,156.76366192)(71.25208007,156.80366188)(71.31208557,156.86366333)
\curveto(71.41207991,157.00366168)(71.49707982,157.13366155)(71.56708557,157.25366333)
\curveto(71.63707968,157.37366131)(71.70707961,157.51866116)(71.77708557,157.68866333)
\curveto(71.80707951,157.75866092)(71.82707949,157.82866085)(71.83708557,157.89866333)
\curveto(71.85707946,157.96866071)(71.87707944,158.04366064)(71.89708557,158.12366333)
}
}
{
\newrgbcolor{curcolor}{0 0 0}
\pscustom[linestyle=none,fillstyle=solid,fillcolor=curcolor]
{
\newpath
\moveto(61.3424762,239.89723694)
\curveto(62.03247156,239.90722631)(62.63247096,239.78722643)(63.1424762,239.53723694)
\curveto(63.66246993,239.28722693)(64.05746954,238.95222726)(64.3274762,238.53223694)
\curveto(64.37746922,238.45222776)(64.42246917,238.36222785)(64.4624762,238.26223694)
\curveto(64.50246909,238.17222804)(64.54746905,238.07722814)(64.5974762,237.97723694)
\curveto(64.63746896,237.87722834)(64.66746893,237.77722844)(64.6874762,237.67723694)
\curveto(64.70746889,237.57722864)(64.72746887,237.47222874)(64.7474762,237.36223694)
\curveto(64.76746883,237.3122289)(64.77246882,237.26722895)(64.7624762,237.22723694)
\curveto(64.75246884,237.18722903)(64.75746884,237.14222907)(64.7774762,237.09223694)
\curveto(64.78746881,237.04222917)(64.7924688,236.95722926)(64.7924762,236.83723694)
\curveto(64.7924688,236.72722949)(64.78746881,236.64222957)(64.7774762,236.58223694)
\curveto(64.75746884,236.52222969)(64.74746885,236.46222975)(64.7474762,236.40223694)
\curveto(64.75746884,236.34222987)(64.75246884,236.28222993)(64.7324762,236.22223694)
\curveto(64.6924689,236.08223013)(64.65746894,235.94723027)(64.6274762,235.81723694)
\curveto(64.597469,235.68723053)(64.55746904,235.56223065)(64.5074762,235.44223694)
\curveto(64.44746915,235.30223091)(64.37746922,235.17723104)(64.2974762,235.06723694)
\curveto(64.22746937,234.95723126)(64.15246944,234.84723137)(64.0724762,234.73723694)
\lineto(64.0124762,234.67723694)
\curveto(64.00246959,234.65723156)(63.98746961,234.63723158)(63.9674762,234.61723694)
\curveto(63.84746975,234.45723176)(63.71246988,234.3122319)(63.5624762,234.18223694)
\curveto(63.41247018,234.05223216)(63.25247034,233.92723229)(63.0824762,233.80723694)
\curveto(62.77247082,233.58723263)(62.47747112,233.38223283)(62.1974762,233.19223694)
\curveto(61.96747163,233.05223316)(61.73747186,232.9172333)(61.5074762,232.78723694)
\curveto(61.28747231,232.65723356)(61.06747253,232.52223369)(60.8474762,232.38223694)
\curveto(60.597473,232.212234)(60.35747324,232.03223418)(60.1274762,231.84223694)
\curveto(59.90747369,231.65223456)(59.71747388,231.42723479)(59.5574762,231.16723694)
\curveto(59.51747408,231.10723511)(59.48247411,231.04723517)(59.4524762,230.98723694)
\curveto(59.42247417,230.93723528)(59.3924742,230.87223534)(59.3624762,230.79223694)
\curveto(59.34247425,230.72223549)(59.33747426,230.66223555)(59.3474762,230.61223694)
\curveto(59.36747423,230.54223567)(59.40247419,230.48723573)(59.4524762,230.44723694)
\curveto(59.50247409,230.4172358)(59.56247403,230.39723582)(59.6324762,230.38723694)
\lineto(59.8724762,230.38723694)
\lineto(60.6224762,230.38723694)
\lineto(63.4274762,230.38723694)
\lineto(64.0874762,230.38723694)
\curveto(64.17746942,230.38723583)(64.26246933,230.38223583)(64.3424762,230.37223694)
\curveto(64.42246917,230.37223584)(64.48746911,230.35223586)(64.5374762,230.31223694)
\curveto(64.58746901,230.27223594)(64.62746897,230.19723602)(64.6574762,230.08723694)
\curveto(64.6974689,229.98723623)(64.70746889,229.88723633)(64.6874762,229.78723694)
\lineto(64.6874762,229.65223694)
\curveto(64.66746893,229.58223663)(64.64746895,229.52223669)(64.6274762,229.47223694)
\curveto(64.60746899,229.42223679)(64.57246902,229.38223683)(64.5224762,229.35223694)
\curveto(64.47246912,229.3122369)(64.40246919,229.29223692)(64.3124762,229.29223694)
\lineto(64.0424762,229.29223694)
\lineto(63.1424762,229.29223694)
\lineto(59.6324762,229.29223694)
\lineto(58.5674762,229.29223694)
\curveto(58.48747511,229.29223692)(58.3974752,229.28723693)(58.2974762,229.27723694)
\curveto(58.1974754,229.27723694)(58.11247548,229.28723693)(58.0424762,229.30723694)
\curveto(57.83247576,229.37723684)(57.76747583,229.55723666)(57.8474762,229.84723694)
\curveto(57.85747574,229.88723633)(57.85747574,229.92223629)(57.8474762,229.95223694)
\curveto(57.84747575,229.99223622)(57.85747574,230.03723618)(57.8774762,230.08723694)
\curveto(57.8974757,230.16723605)(57.91747568,230.25223596)(57.9374762,230.34223694)
\curveto(57.95747564,230.43223578)(57.98247561,230.5172357)(58.0124762,230.59723694)
\curveto(58.17247542,231.08723513)(58.37247522,231.50223471)(58.6124762,231.84223694)
\curveto(58.7924748,232.09223412)(58.9974746,232.3172339)(59.2274762,232.51723694)
\curveto(59.45747414,232.72723349)(59.6974739,232.92223329)(59.9474762,233.10223694)
\curveto(60.20747339,233.28223293)(60.47247312,233.45223276)(60.7424762,233.61223694)
\curveto(61.02247257,233.78223243)(61.2924723,233.95723226)(61.5524762,234.13723694)
\curveto(61.66247193,234.217232)(61.76747183,234.29223192)(61.8674762,234.36223694)
\curveto(61.97747162,234.43223178)(62.08747151,234.50723171)(62.1974762,234.58723694)
\curveto(62.23747136,234.6172316)(62.27247132,234.64723157)(62.3024762,234.67723694)
\curveto(62.34247125,234.7172315)(62.38247121,234.74723147)(62.4224762,234.76723694)
\curveto(62.56247103,234.87723134)(62.68747091,235.00223121)(62.7974762,235.14223694)
\curveto(62.81747078,235.17223104)(62.84247075,235.19723102)(62.8724762,235.21723694)
\curveto(62.90247069,235.24723097)(62.92747067,235.27723094)(62.9474762,235.30723694)
\curveto(63.02747057,235.40723081)(63.0924705,235.50723071)(63.1424762,235.60723694)
\curveto(63.20247039,235.70723051)(63.25747034,235.8172304)(63.3074762,235.93723694)
\curveto(63.33747026,236.00723021)(63.35747024,236.08223013)(63.3674762,236.16223694)
\lineto(63.4274762,236.40223694)
\lineto(63.4274762,236.49223694)
\curveto(63.43747016,236.52222969)(63.44247015,236.55222966)(63.4424762,236.58223694)
\curveto(63.46247013,236.65222956)(63.46747013,236.74722947)(63.4574762,236.86723694)
\curveto(63.45747014,236.99722922)(63.44747015,237.09722912)(63.4274762,237.16723694)
\curveto(63.40747019,237.24722897)(63.38747021,237.32222889)(63.3674762,237.39223694)
\curveto(63.35747024,237.47222874)(63.33747026,237.55222866)(63.3074762,237.63223694)
\curveto(63.1974704,237.87222834)(63.04747055,238.07222814)(62.8574762,238.23223694)
\curveto(62.67747092,238.40222781)(62.45747114,238.54222767)(62.1974762,238.65223694)
\curveto(62.12747147,238.67222754)(62.05747154,238.68722753)(61.9874762,238.69723694)
\curveto(61.91747168,238.7172275)(61.84247175,238.73722748)(61.7624762,238.75723694)
\curveto(61.68247191,238.77722744)(61.57247202,238.78722743)(61.4324762,238.78723694)
\curveto(61.30247229,238.78722743)(61.1974724,238.77722744)(61.1174762,238.75723694)
\curveto(61.05747254,238.74722747)(61.00247259,238.74222747)(60.9524762,238.74223694)
\curveto(60.90247269,238.74222747)(60.85247274,238.73222748)(60.8024762,238.71223694)
\curveto(60.70247289,238.67222754)(60.60747299,238.63222758)(60.5174762,238.59223694)
\curveto(60.43747316,238.55222766)(60.35747324,238.50722771)(60.2774762,238.45723694)
\curveto(60.24747335,238.43722778)(60.21747338,238.4122278)(60.1874762,238.38223694)
\curveto(60.16747343,238.35222786)(60.14247345,238.32722789)(60.1124762,238.30723694)
\lineto(60.0374762,238.23223694)
\curveto(60.00747359,238.212228)(59.98247361,238.19222802)(59.9624762,238.17223694)
\lineto(59.8124762,237.96223694)
\curveto(59.77247382,237.90222831)(59.72747387,237.83722838)(59.6774762,237.76723694)
\curveto(59.61747398,237.67722854)(59.56747403,237.57222864)(59.5274762,237.45223694)
\curveto(59.4974741,237.34222887)(59.46247413,237.23222898)(59.4224762,237.12223694)
\curveto(59.38247421,237.0122292)(59.35747424,236.86722935)(59.3474762,236.68723694)
\curveto(59.33747426,236.5172297)(59.30747429,236.39222982)(59.2574762,236.31223694)
\curveto(59.20747439,236.23222998)(59.13247446,236.18723003)(59.0324762,236.17723694)
\curveto(58.93247466,236.16723005)(58.82247477,236.16223005)(58.7024762,236.16223694)
\curveto(58.66247493,236.16223005)(58.62247497,236.15723006)(58.5824762,236.14723694)
\curveto(58.54247505,236.14723007)(58.50747509,236.15223006)(58.4774762,236.16223694)
\curveto(58.42747517,236.18223003)(58.37747522,236.19223002)(58.3274762,236.19223694)
\curveto(58.28747531,236.19223002)(58.24747535,236.20223001)(58.2074762,236.22223694)
\curveto(58.11747548,236.28222993)(58.07247552,236.4172298)(58.0724762,236.62723694)
\lineto(58.0724762,236.74723694)
\curveto(58.08247551,236.80722941)(58.08747551,236.86722935)(58.0874762,236.92723694)
\curveto(58.0974755,236.99722922)(58.10747549,237.06222915)(58.1174762,237.12223694)
\curveto(58.13747546,237.23222898)(58.15747544,237.33222888)(58.1774762,237.42223694)
\curveto(58.1974754,237.52222869)(58.22747537,237.6172286)(58.2674762,237.70723694)
\curveto(58.28747531,237.77722844)(58.30747529,237.83722838)(58.3274762,237.88723694)
\lineto(58.3874762,238.06723694)
\curveto(58.50747509,238.32722789)(58.66247493,238.57222764)(58.8524762,238.80223694)
\curveto(59.05247454,239.03222718)(59.26747433,239.217227)(59.4974762,239.35723694)
\curveto(59.60747399,239.43722678)(59.72247387,239.50222671)(59.8424762,239.55223694)
\lineto(60.2324762,239.70223694)
\curveto(60.34247325,239.75222646)(60.45747314,239.78222643)(60.5774762,239.79223694)
\curveto(60.6974729,239.8122264)(60.82247277,239.83722638)(60.9524762,239.86723694)
\curveto(61.02247257,239.86722635)(61.08747251,239.86722635)(61.1474762,239.86723694)
\curveto(61.20747239,239.87722634)(61.27247232,239.88722633)(61.3424762,239.89723694)
}
}
{
\newrgbcolor{curcolor}{0 0 0}
\pscustom[linestyle=none,fillstyle=solid,fillcolor=curcolor]
{
\newpath
\moveto(73.44208557,234.37723694)
\lineto(73.44208557,234.12223694)
\curveto(73.45207787,234.04223217)(73.44707787,233.96723225)(73.42708557,233.89723694)
\lineto(73.42708557,233.65723694)
\lineto(73.42708557,233.49223694)
\curveto(73.40707791,233.39223282)(73.39707792,233.28723293)(73.39708557,233.17723694)
\curveto(73.39707792,233.07723314)(73.38707793,232.97723324)(73.36708557,232.87723694)
\lineto(73.36708557,232.72723694)
\curveto(73.33707798,232.58723363)(73.317078,232.44723377)(73.30708557,232.30723694)
\curveto(73.29707802,232.17723404)(73.27207805,232.04723417)(73.23208557,231.91723694)
\curveto(73.21207811,231.83723438)(73.19207813,231.75223446)(73.17208557,231.66223694)
\lineto(73.11208557,231.42223694)
\lineto(72.99208557,231.12223694)
\curveto(72.96207836,231.03223518)(72.92707839,230.94223527)(72.88708557,230.85223694)
\curveto(72.78707853,230.63223558)(72.65207867,230.4172358)(72.48208557,230.20723694)
\curveto(72.322079,229.99723622)(72.14707917,229.82723639)(71.95708557,229.69723694)
\curveto(71.90707941,229.65723656)(71.84707947,229.6172366)(71.77708557,229.57723694)
\curveto(71.7170796,229.54723667)(71.65707966,229.5122367)(71.59708557,229.47223694)
\curveto(71.5170798,229.42223679)(71.4220799,229.38223683)(71.31208557,229.35223694)
\curveto(71.20208012,229.32223689)(71.09708022,229.29223692)(70.99708557,229.26223694)
\curveto(70.88708043,229.22223699)(70.77708054,229.19723702)(70.66708557,229.18723694)
\curveto(70.55708076,229.17723704)(70.44208088,229.16223705)(70.32208557,229.14223694)
\curveto(70.28208104,229.13223708)(70.23708108,229.13223708)(70.18708557,229.14223694)
\curveto(70.14708117,229.14223707)(70.10708121,229.13723708)(70.06708557,229.12723694)
\curveto(70.02708129,229.1172371)(69.97208135,229.1122371)(69.90208557,229.11223694)
\curveto(69.83208149,229.1122371)(69.78208154,229.1172371)(69.75208557,229.12723694)
\curveto(69.70208162,229.14723707)(69.65708166,229.15223706)(69.61708557,229.14223694)
\curveto(69.57708174,229.13223708)(69.54208178,229.13223708)(69.51208557,229.14223694)
\lineto(69.42208557,229.14223694)
\curveto(69.36208196,229.16223705)(69.29708202,229.17723704)(69.22708557,229.18723694)
\curveto(69.16708215,229.18723703)(69.10208222,229.19223702)(69.03208557,229.20223694)
\curveto(68.86208246,229.25223696)(68.70208262,229.30223691)(68.55208557,229.35223694)
\curveto(68.40208292,229.40223681)(68.25708306,229.46723675)(68.11708557,229.54723694)
\curveto(68.06708325,229.58723663)(68.01208331,229.6172366)(67.95208557,229.63723694)
\curveto(67.90208342,229.66723655)(67.85208347,229.70223651)(67.80208557,229.74223694)
\curveto(67.56208376,229.92223629)(67.36208396,230.14223607)(67.20208557,230.40223694)
\curveto(67.04208428,230.66223555)(66.90208442,230.94723527)(66.78208557,231.25723694)
\curveto(66.7220846,231.39723482)(66.67708464,231.53723468)(66.64708557,231.67723694)
\curveto(66.6170847,231.82723439)(66.58208474,231.98223423)(66.54208557,232.14223694)
\curveto(66.5220848,232.25223396)(66.50708481,232.36223385)(66.49708557,232.47223694)
\curveto(66.48708483,232.58223363)(66.47208485,232.69223352)(66.45208557,232.80223694)
\curveto(66.44208488,232.84223337)(66.43708488,232.88223333)(66.43708557,232.92223694)
\curveto(66.44708487,232.96223325)(66.44708487,233.00223321)(66.43708557,233.04223694)
\curveto(66.42708489,233.09223312)(66.4220849,233.14223307)(66.42208557,233.19223694)
\lineto(66.42208557,233.35723694)
\curveto(66.40208492,233.40723281)(66.39708492,233.45723276)(66.40708557,233.50723694)
\curveto(66.4170849,233.56723265)(66.4170849,233.62223259)(66.40708557,233.67223694)
\curveto(66.39708492,233.7122325)(66.39708492,233.75723246)(66.40708557,233.80723694)
\curveto(66.4170849,233.85723236)(66.41208491,233.90723231)(66.39208557,233.95723694)
\curveto(66.37208495,234.02723219)(66.36708495,234.10223211)(66.37708557,234.18223694)
\curveto(66.38708493,234.27223194)(66.39208493,234.35723186)(66.39208557,234.43723694)
\curveto(66.39208493,234.52723169)(66.38708493,234.62723159)(66.37708557,234.73723694)
\curveto(66.36708495,234.85723136)(66.37208495,234.95723126)(66.39208557,235.03723694)
\lineto(66.39208557,235.32223694)
\lineto(66.43708557,235.95223694)
\curveto(66.44708487,236.05223016)(66.45708486,236.14723007)(66.46708557,236.23723694)
\lineto(66.49708557,236.53723694)
\curveto(66.5170848,236.58722963)(66.5220848,236.63722958)(66.51208557,236.68723694)
\curveto(66.51208481,236.74722947)(66.5220848,236.80222941)(66.54208557,236.85223694)
\curveto(66.59208473,237.02222919)(66.63208469,237.18722903)(66.66208557,237.34723694)
\curveto(66.69208463,237.5172287)(66.74208458,237.67722854)(66.81208557,237.82723694)
\curveto(67.00208432,238.28722793)(67.2220841,238.66222755)(67.47208557,238.95223694)
\curveto(67.73208359,239.24222697)(68.09208323,239.48722673)(68.55208557,239.68723694)
\curveto(68.68208264,239.73722648)(68.81208251,239.77222644)(68.94208557,239.79223694)
\curveto(69.08208224,239.8122264)(69.2220821,239.83722638)(69.36208557,239.86723694)
\curveto(69.43208189,239.87722634)(69.49708182,239.88222633)(69.55708557,239.88223694)
\curveto(69.6170817,239.88222633)(69.68208164,239.88722633)(69.75208557,239.89723694)
\curveto(70.58208074,239.9172263)(71.25208007,239.76722645)(71.76208557,239.44723694)
\curveto(72.27207905,239.13722708)(72.65207867,238.69722752)(72.90208557,238.12723694)
\curveto(72.95207837,238.00722821)(72.99707832,237.88222833)(73.03708557,237.75223694)
\curveto(73.07707824,237.62222859)(73.1220782,237.48722873)(73.17208557,237.34723694)
\curveto(73.19207813,237.26722895)(73.20707811,237.18222903)(73.21708557,237.09223694)
\lineto(73.27708557,236.85223694)
\curveto(73.30707801,236.74222947)(73.322078,236.63222958)(73.32208557,236.52223694)
\curveto(73.33207799,236.4122298)(73.34707797,236.30222991)(73.36708557,236.19223694)
\curveto(73.38707793,236.14223007)(73.39207793,236.09723012)(73.38208557,236.05723694)
\curveto(73.38207794,236.0172302)(73.38707793,235.97723024)(73.39708557,235.93723694)
\curveto(73.40707791,235.88723033)(73.40707791,235.83223038)(73.39708557,235.77223694)
\curveto(73.39707792,235.72223049)(73.40207792,235.67223054)(73.41208557,235.62223694)
\lineto(73.41208557,235.48723694)
\curveto(73.43207789,235.42723079)(73.43207789,235.35723086)(73.41208557,235.27723694)
\curveto(73.40207792,235.20723101)(73.40707791,235.14223107)(73.42708557,235.08223694)
\curveto(73.43707788,235.05223116)(73.44207788,235.0122312)(73.44208557,234.96223694)
\lineto(73.44208557,234.84223694)
\lineto(73.44208557,234.37723694)
\moveto(71.89708557,232.05223694)
\curveto(71.99707932,232.37223384)(72.05707926,232.73723348)(72.07708557,233.14723694)
\curveto(72.09707922,233.55723266)(72.10707921,233.96723225)(72.10708557,234.37723694)
\curveto(72.10707921,234.80723141)(72.09707922,235.22723099)(72.07708557,235.63723694)
\curveto(72.05707926,236.04723017)(72.01207931,236.43222978)(71.94208557,236.79223694)
\curveto(71.87207945,237.15222906)(71.76207956,237.47222874)(71.61208557,237.75223694)
\curveto(71.47207985,238.04222817)(71.27708004,238.27722794)(71.02708557,238.45723694)
\curveto(70.86708045,238.56722765)(70.68708063,238.64722757)(70.48708557,238.69723694)
\curveto(70.28708103,238.75722746)(70.04208128,238.78722743)(69.75208557,238.78723694)
\curveto(69.73208159,238.76722745)(69.69708162,238.75722746)(69.64708557,238.75723694)
\curveto(69.59708172,238.76722745)(69.55708176,238.76722745)(69.52708557,238.75723694)
\curveto(69.44708187,238.73722748)(69.37208195,238.7172275)(69.30208557,238.69723694)
\curveto(69.24208208,238.68722753)(69.17708214,238.66722755)(69.10708557,238.63723694)
\curveto(68.83708248,238.5172277)(68.6170827,238.34722787)(68.44708557,238.12723694)
\curveto(68.28708303,237.9172283)(68.15208317,237.67222854)(68.04208557,237.39223694)
\curveto(67.99208333,237.28222893)(67.95208337,237.16222905)(67.92208557,237.03223694)
\curveto(67.90208342,236.9122293)(67.87708344,236.78722943)(67.84708557,236.65723694)
\curveto(67.82708349,236.60722961)(67.8170835,236.55222966)(67.81708557,236.49223694)
\curveto(67.8170835,236.44222977)(67.81208351,236.39222982)(67.80208557,236.34223694)
\curveto(67.79208353,236.25222996)(67.78208354,236.15723006)(67.77208557,236.05723694)
\curveto(67.76208356,235.96723025)(67.75208357,235.87223034)(67.74208557,235.77223694)
\curveto(67.74208358,235.69223052)(67.73708358,235.60723061)(67.72708557,235.51723694)
\lineto(67.72708557,235.27723694)
\lineto(67.72708557,235.09723694)
\curveto(67.7170836,235.06723115)(67.71208361,235.03223118)(67.71208557,234.99223694)
\lineto(67.71208557,234.85723694)
\lineto(67.71208557,234.40723694)
\curveto(67.71208361,234.32723189)(67.70708361,234.24223197)(67.69708557,234.15223694)
\curveto(67.69708362,234.07223214)(67.70708361,233.99723222)(67.72708557,233.92723694)
\lineto(67.72708557,233.65723694)
\curveto(67.72708359,233.63723258)(67.7220836,233.60723261)(67.71208557,233.56723694)
\curveto(67.71208361,233.53723268)(67.7170836,233.5122327)(67.72708557,233.49223694)
\curveto(67.73708358,233.39223282)(67.74208358,233.29223292)(67.74208557,233.19223694)
\curveto(67.75208357,233.10223311)(67.76208356,233.00223321)(67.77208557,232.89223694)
\curveto(67.80208352,232.77223344)(67.8170835,232.64723357)(67.81708557,232.51723694)
\curveto(67.82708349,232.39723382)(67.85208347,232.28223393)(67.89208557,232.17223694)
\curveto(67.97208335,231.87223434)(68.05708326,231.60723461)(68.14708557,231.37723694)
\curveto(68.24708307,231.14723507)(68.39208293,230.93223528)(68.58208557,230.73223694)
\curveto(68.79208253,230.53223568)(69.05708226,230.38223583)(69.37708557,230.28223694)
\curveto(69.4170819,230.26223595)(69.45208187,230.25223596)(69.48208557,230.25223694)
\curveto(69.5220818,230.26223595)(69.56708175,230.25723596)(69.61708557,230.23723694)
\curveto(69.65708166,230.22723599)(69.72708159,230.217236)(69.82708557,230.20723694)
\curveto(69.93708138,230.19723602)(70.0220813,230.20223601)(70.08208557,230.22223694)
\curveto(70.15208117,230.24223597)(70.2220811,230.25223596)(70.29208557,230.25223694)
\curveto(70.36208096,230.26223595)(70.42708089,230.27723594)(70.48708557,230.29723694)
\curveto(70.68708063,230.35723586)(70.86708045,230.44223577)(71.02708557,230.55223694)
\curveto(71.05708026,230.57223564)(71.08208024,230.59223562)(71.10208557,230.61223694)
\lineto(71.16208557,230.67223694)
\curveto(71.20208012,230.69223552)(71.25208007,230.73223548)(71.31208557,230.79223694)
\curveto(71.41207991,230.93223528)(71.49707982,231.06223515)(71.56708557,231.18223694)
\curveto(71.63707968,231.30223491)(71.70707961,231.44723477)(71.77708557,231.61723694)
\curveto(71.80707951,231.68723453)(71.82707949,231.75723446)(71.83708557,231.82723694)
\curveto(71.85707946,231.89723432)(71.87707944,231.97223424)(71.89708557,232.05223694)
}
}
{
\newrgbcolor{curcolor}{0 0 0}
\pscustom[linestyle=none,fillstyle=solid,fillcolor=curcolor]
{
\newpath
\moveto(61.2074762,314.89723694)
\curveto(62.83747076,314.92722629)(63.88746971,314.37222684)(64.3574762,313.23223694)
\curveto(64.45746914,313.00222821)(64.52246907,312.7122285)(64.5524762,312.36223694)
\curveto(64.592469,312.02222919)(64.56746903,311.7122295)(64.4774762,311.43223694)
\curveto(64.38746921,311.17223004)(64.26746933,310.94723027)(64.1174762,310.75723694)
\curveto(64.0974695,310.7172305)(64.07246952,310.68223053)(64.0424762,310.65223694)
\curveto(64.01246958,310.63223058)(63.98746961,310.60723061)(63.9674762,310.57723694)
\lineto(63.8774762,310.45723694)
\curveto(63.84746975,310.42723079)(63.81246978,310.40223081)(63.7724762,310.38223694)
\curveto(63.72246987,310.33223088)(63.66746993,310.28723093)(63.6074762,310.24723694)
\curveto(63.55747004,310.20723101)(63.51247008,310.15723106)(63.4724762,310.09723694)
\curveto(63.43247016,310.05723116)(63.41747018,310.00723121)(63.4274762,309.94723694)
\curveto(63.43747016,309.89723132)(63.46747013,309.85223136)(63.5174762,309.81223694)
\curveto(63.56747003,309.77223144)(63.62246997,309.73223148)(63.6824762,309.69223694)
\curveto(63.75246984,309.66223155)(63.81746978,309.63223158)(63.8774762,309.60223694)
\curveto(63.93746966,309.57223164)(63.98746961,309.54223167)(64.0274762,309.51223694)
\curveto(64.34746925,309.29223192)(64.60246899,308.98223223)(64.7924762,308.58223694)
\curveto(64.83246876,308.49223272)(64.86246873,308.39723282)(64.8824762,308.29723694)
\curveto(64.91246868,308.20723301)(64.93746866,308.1172331)(64.9574762,308.02723694)
\curveto(64.96746863,307.97723324)(64.97246862,307.92723329)(64.9724762,307.87723694)
\curveto(64.98246861,307.83723338)(64.9924686,307.79223342)(65.0024762,307.74223694)
\curveto(65.01246858,307.69223352)(65.01246858,307.64223357)(65.0024762,307.59223694)
\curveto(64.9924686,307.54223367)(64.9974686,307.49223372)(65.0174762,307.44223694)
\curveto(65.02746857,307.39223382)(65.03246856,307.33223388)(65.0324762,307.26223694)
\curveto(65.03246856,307.19223402)(65.02246857,307.13223408)(65.0024762,307.08223694)
\lineto(65.0024762,306.85723694)
\lineto(64.9424762,306.61723694)
\curveto(64.93246866,306.54723467)(64.91746868,306.47723474)(64.8974762,306.40723694)
\curveto(64.86746873,306.3172349)(64.83746876,306.23223498)(64.8074762,306.15223694)
\curveto(64.78746881,306.07223514)(64.75746884,305.99223522)(64.7174762,305.91223694)
\curveto(64.6974689,305.85223536)(64.66746893,305.79223542)(64.6274762,305.73223694)
\curveto(64.597469,305.68223553)(64.56246903,305.63223558)(64.5224762,305.58223694)
\curveto(64.32246927,305.27223594)(64.07246952,305.0122362)(63.7724762,304.80223694)
\curveto(63.47247012,304.60223661)(63.12747047,304.43723678)(62.7374762,304.30723694)
\curveto(62.61747098,304.26723695)(62.48747111,304.24223697)(62.3474762,304.23223694)
\curveto(62.21747138,304.212237)(62.08247151,304.18723703)(61.9424762,304.15723694)
\curveto(61.87247172,304.14723707)(61.80247179,304.14223707)(61.7324762,304.14223694)
\curveto(61.67247192,304.14223707)(61.60747199,304.13723708)(61.5374762,304.12723694)
\curveto(61.4974721,304.1172371)(61.43747216,304.1122371)(61.3574762,304.11223694)
\curveto(61.28747231,304.1122371)(61.23747236,304.1172371)(61.2074762,304.12723694)
\curveto(61.15747244,304.13723708)(61.11247248,304.14223707)(61.0724762,304.14223694)
\lineto(60.9524762,304.14223694)
\curveto(60.85247274,304.16223705)(60.75247284,304.17723704)(60.6524762,304.18723694)
\curveto(60.55247304,304.19723702)(60.45747314,304.212237)(60.3674762,304.23223694)
\curveto(60.25747334,304.26223695)(60.14747345,304.28723693)(60.0374762,304.30723694)
\curveto(59.93747366,304.33723688)(59.83247376,304.37723684)(59.7224762,304.42723694)
\curveto(59.35247424,304.58723663)(59.03747456,304.78723643)(58.7774762,305.02723694)
\curveto(58.51747508,305.27723594)(58.30747529,305.58723563)(58.1474762,305.95723694)
\curveto(58.10747549,306.04723517)(58.07247552,306.14223507)(58.0424762,306.24223694)
\curveto(58.01247558,306.34223487)(57.98247561,306.44723477)(57.9524762,306.55723694)
\curveto(57.93247566,306.60723461)(57.92247567,306.65723456)(57.9224762,306.70723694)
\curveto(57.92247567,306.76723445)(57.91247568,306.82723439)(57.8924762,306.88723694)
\curveto(57.87247572,306.94723427)(57.86247573,307.02723419)(57.8624762,307.12723694)
\curveto(57.86247573,307.22723399)(57.87747572,307.30223391)(57.9074762,307.35223694)
\curveto(57.91747568,307.38223383)(57.93247566,307.40723381)(57.9524762,307.42723694)
\lineto(58.0124762,307.48723694)
\curveto(58.05247554,307.50723371)(58.11247548,307.52223369)(58.1924762,307.53223694)
\curveto(58.28247531,307.54223367)(58.37247522,307.54723367)(58.4624762,307.54723694)
\curveto(58.55247504,307.54723367)(58.63747496,307.54223367)(58.7174762,307.53223694)
\curveto(58.80747479,307.52223369)(58.87247472,307.5122337)(58.9124762,307.50223694)
\curveto(58.93247466,307.48223373)(58.95247464,307.46723375)(58.9724762,307.45723694)
\curveto(58.9924746,307.45723376)(59.01247458,307.44723377)(59.0324762,307.42723694)
\curveto(59.10247449,307.33723388)(59.14247445,307.22223399)(59.1524762,307.08223694)
\curveto(59.17247442,306.94223427)(59.20247439,306.8172344)(59.2424762,306.70723694)
\lineto(59.3924762,306.34723694)
\curveto(59.44247415,306.23723498)(59.50747409,306.13223508)(59.5874762,306.03223694)
\curveto(59.60747399,306.00223521)(59.62747397,305.97723524)(59.6474762,305.95723694)
\curveto(59.67747392,305.93723528)(59.70247389,305.9122353)(59.7224762,305.88223694)
\curveto(59.76247383,305.82223539)(59.7974738,305.77723544)(59.8274762,305.74723694)
\curveto(59.86747373,305.7172355)(59.90247369,305.68723553)(59.9324762,305.65723694)
\curveto(59.97247362,305.62723559)(60.01747358,305.59723562)(60.0674762,305.56723694)
\curveto(60.15747344,305.50723571)(60.25247334,305.45723576)(60.3524762,305.41723694)
\lineto(60.6824762,305.29723694)
\curveto(60.83247276,305.24723597)(61.03247256,305.217236)(61.2824762,305.20723694)
\curveto(61.53247206,305.19723602)(61.74247185,305.217236)(61.9124762,305.26723694)
\curveto(61.9924716,305.28723593)(62.06247153,305.30223591)(62.1224762,305.31223694)
\lineto(62.3324762,305.37223694)
\curveto(62.61247098,305.49223572)(62.85247074,305.64223557)(63.0524762,305.82223694)
\curveto(63.26247033,306.00223521)(63.42747017,306.23223498)(63.5474762,306.51223694)
\curveto(63.57747002,306.58223463)(63.59747,306.65223456)(63.6074762,306.72223694)
\lineto(63.6674762,306.96223694)
\curveto(63.70746989,307.10223411)(63.71746988,307.26223395)(63.6974762,307.44223694)
\curveto(63.67746992,307.63223358)(63.64746995,307.78223343)(63.6074762,307.89223694)
\curveto(63.47747012,308.27223294)(63.2924703,308.56223265)(63.0524762,308.76223694)
\curveto(62.82247077,308.96223225)(62.51247108,309.12223209)(62.1224762,309.24223694)
\curveto(62.01247158,309.27223194)(61.8924717,309.29223192)(61.7624762,309.30223694)
\curveto(61.64247195,309.3122319)(61.51747208,309.3172319)(61.3874762,309.31723694)
\curveto(61.22747237,309.3172319)(61.08747251,309.32223189)(60.9674762,309.33223694)
\curveto(60.84747275,309.34223187)(60.76247283,309.40223181)(60.7124762,309.51223694)
\curveto(60.6924729,309.54223167)(60.68247291,309.57723164)(60.6824762,309.61723694)
\lineto(60.6824762,309.75223694)
\curveto(60.67247292,309.85223136)(60.67247292,309.94723127)(60.6824762,310.03723694)
\curveto(60.70247289,310.12723109)(60.74247285,310.19223102)(60.8024762,310.23223694)
\curveto(60.84247275,310.26223095)(60.88247271,310.28223093)(60.9224762,310.29223694)
\curveto(60.97247262,310.30223091)(61.02747257,310.3122309)(61.0874762,310.32223694)
\curveto(61.10747249,310.33223088)(61.13247246,310.33223088)(61.1624762,310.32223694)
\curveto(61.1924724,310.32223089)(61.21747238,310.32723089)(61.2374762,310.33723694)
\lineto(61.3724762,310.33723694)
\curveto(61.48247211,310.35723086)(61.58247201,310.36723085)(61.6724762,310.36723694)
\curveto(61.77247182,310.37723084)(61.86747173,310.39723082)(61.9574762,310.42723694)
\curveto(62.27747132,310.53723068)(62.53247106,310.68223053)(62.7224762,310.86223694)
\curveto(62.91247068,311.04223017)(63.06247053,311.29222992)(63.1724762,311.61223694)
\curveto(63.20247039,311.7122295)(63.22247037,311.83722938)(63.2324762,311.98723694)
\curveto(63.25247034,312.14722907)(63.24747035,312.29222892)(63.2174762,312.42223694)
\curveto(63.1974704,312.49222872)(63.17747042,312.55722866)(63.1574762,312.61723694)
\curveto(63.14747045,312.68722853)(63.12747047,312.75222846)(63.0974762,312.81223694)
\curveto(62.9974706,313.05222816)(62.85247074,313.24222797)(62.6624762,313.38223694)
\curveto(62.47247112,313.52222769)(62.24747135,313.63222758)(61.9874762,313.71223694)
\curveto(61.92747167,313.73222748)(61.86747173,313.74222747)(61.8074762,313.74223694)
\curveto(61.74747185,313.74222747)(61.68247191,313.75222746)(61.6124762,313.77223694)
\curveto(61.53247206,313.79222742)(61.43747216,313.80222741)(61.3274762,313.80223694)
\curveto(61.21747238,313.80222741)(61.12247247,313.79222742)(61.0424762,313.77223694)
\curveto(60.9924726,313.75222746)(60.94247265,313.74222747)(60.8924762,313.74223694)
\curveto(60.85247274,313.74222747)(60.80747279,313.73222748)(60.7574762,313.71223694)
\curveto(60.57747302,313.66222755)(60.40747319,313.58722763)(60.2474762,313.48723694)
\curveto(60.0974735,313.39722782)(59.96747363,313.28222793)(59.8574762,313.14223694)
\curveto(59.76747383,313.02222819)(59.68747391,312.89222832)(59.6174762,312.75223694)
\curveto(59.54747405,312.6122286)(59.48247411,312.45722876)(59.4224762,312.28723694)
\curveto(59.3924742,312.17722904)(59.37247422,312.05722916)(59.3624762,311.92723694)
\curveto(59.35247424,311.80722941)(59.31747428,311.70722951)(59.2574762,311.62723694)
\curveto(59.23747436,311.58722963)(59.17747442,311.54722967)(59.0774762,311.50723694)
\curveto(59.03747456,311.49722972)(58.97747462,311.48722973)(58.8974762,311.47723694)
\lineto(58.6424762,311.47723694)
\curveto(58.55247504,311.48722973)(58.46747513,311.49722972)(58.3874762,311.50723694)
\curveto(58.31747528,311.5172297)(58.26747533,311.53222968)(58.2374762,311.55223694)
\curveto(58.1974754,311.58222963)(58.16247543,311.63722958)(58.1324762,311.71723694)
\curveto(58.10247549,311.79722942)(58.0974755,311.88222933)(58.1174762,311.97223694)
\curveto(58.12747547,312.02222919)(58.13247546,312.07222914)(58.1324762,312.12223694)
\lineto(58.1624762,312.30223694)
\curveto(58.1924754,312.40222881)(58.21747538,312.50222871)(58.2374762,312.60223694)
\curveto(58.26747533,312.70222851)(58.30247529,312.79222842)(58.3424762,312.87223694)
\curveto(58.3924752,312.98222823)(58.43747516,313.08722813)(58.4774762,313.18723694)
\curveto(58.51747508,313.29722792)(58.56747503,313.40222781)(58.6274762,313.50223694)
\curveto(58.95747464,314.04222717)(59.42747417,314.43722678)(60.0374762,314.68723694)
\curveto(60.15747344,314.73722648)(60.28247331,314.77222644)(60.4124762,314.79223694)
\curveto(60.55247304,314.8122264)(60.6924729,314.83722638)(60.8324762,314.86723694)
\curveto(60.8924727,314.87722634)(60.95247264,314.88222633)(61.0124762,314.88223694)
\curveto(61.08247251,314.88222633)(61.14747245,314.88722633)(61.2074762,314.89723694)
}
}
{
\newrgbcolor{curcolor}{0 0 0}
\pscustom[linestyle=none,fillstyle=solid,fillcolor=curcolor]
{
\newpath
\moveto(73.44208557,309.37723694)
\lineto(73.44208557,309.12223694)
\curveto(73.45207787,309.04223217)(73.44707787,308.96723225)(73.42708557,308.89723694)
\lineto(73.42708557,308.65723694)
\lineto(73.42708557,308.49223694)
\curveto(73.40707791,308.39223282)(73.39707792,308.28723293)(73.39708557,308.17723694)
\curveto(73.39707792,308.07723314)(73.38707793,307.97723324)(73.36708557,307.87723694)
\lineto(73.36708557,307.72723694)
\curveto(73.33707798,307.58723363)(73.317078,307.44723377)(73.30708557,307.30723694)
\curveto(73.29707802,307.17723404)(73.27207805,307.04723417)(73.23208557,306.91723694)
\curveto(73.21207811,306.83723438)(73.19207813,306.75223446)(73.17208557,306.66223694)
\lineto(73.11208557,306.42223694)
\lineto(72.99208557,306.12223694)
\curveto(72.96207836,306.03223518)(72.92707839,305.94223527)(72.88708557,305.85223694)
\curveto(72.78707853,305.63223558)(72.65207867,305.4172358)(72.48208557,305.20723694)
\curveto(72.322079,304.99723622)(72.14707917,304.82723639)(71.95708557,304.69723694)
\curveto(71.90707941,304.65723656)(71.84707947,304.6172366)(71.77708557,304.57723694)
\curveto(71.7170796,304.54723667)(71.65707966,304.5122367)(71.59708557,304.47223694)
\curveto(71.5170798,304.42223679)(71.4220799,304.38223683)(71.31208557,304.35223694)
\curveto(71.20208012,304.32223689)(71.09708022,304.29223692)(70.99708557,304.26223694)
\curveto(70.88708043,304.22223699)(70.77708054,304.19723702)(70.66708557,304.18723694)
\curveto(70.55708076,304.17723704)(70.44208088,304.16223705)(70.32208557,304.14223694)
\curveto(70.28208104,304.13223708)(70.23708108,304.13223708)(70.18708557,304.14223694)
\curveto(70.14708117,304.14223707)(70.10708121,304.13723708)(70.06708557,304.12723694)
\curveto(70.02708129,304.1172371)(69.97208135,304.1122371)(69.90208557,304.11223694)
\curveto(69.83208149,304.1122371)(69.78208154,304.1172371)(69.75208557,304.12723694)
\curveto(69.70208162,304.14723707)(69.65708166,304.15223706)(69.61708557,304.14223694)
\curveto(69.57708174,304.13223708)(69.54208178,304.13223708)(69.51208557,304.14223694)
\lineto(69.42208557,304.14223694)
\curveto(69.36208196,304.16223705)(69.29708202,304.17723704)(69.22708557,304.18723694)
\curveto(69.16708215,304.18723703)(69.10208222,304.19223702)(69.03208557,304.20223694)
\curveto(68.86208246,304.25223696)(68.70208262,304.30223691)(68.55208557,304.35223694)
\curveto(68.40208292,304.40223681)(68.25708306,304.46723675)(68.11708557,304.54723694)
\curveto(68.06708325,304.58723663)(68.01208331,304.6172366)(67.95208557,304.63723694)
\curveto(67.90208342,304.66723655)(67.85208347,304.70223651)(67.80208557,304.74223694)
\curveto(67.56208376,304.92223629)(67.36208396,305.14223607)(67.20208557,305.40223694)
\curveto(67.04208428,305.66223555)(66.90208442,305.94723527)(66.78208557,306.25723694)
\curveto(66.7220846,306.39723482)(66.67708464,306.53723468)(66.64708557,306.67723694)
\curveto(66.6170847,306.82723439)(66.58208474,306.98223423)(66.54208557,307.14223694)
\curveto(66.5220848,307.25223396)(66.50708481,307.36223385)(66.49708557,307.47223694)
\curveto(66.48708483,307.58223363)(66.47208485,307.69223352)(66.45208557,307.80223694)
\curveto(66.44208488,307.84223337)(66.43708488,307.88223333)(66.43708557,307.92223694)
\curveto(66.44708487,307.96223325)(66.44708487,308.00223321)(66.43708557,308.04223694)
\curveto(66.42708489,308.09223312)(66.4220849,308.14223307)(66.42208557,308.19223694)
\lineto(66.42208557,308.35723694)
\curveto(66.40208492,308.40723281)(66.39708492,308.45723276)(66.40708557,308.50723694)
\curveto(66.4170849,308.56723265)(66.4170849,308.62223259)(66.40708557,308.67223694)
\curveto(66.39708492,308.7122325)(66.39708492,308.75723246)(66.40708557,308.80723694)
\curveto(66.4170849,308.85723236)(66.41208491,308.90723231)(66.39208557,308.95723694)
\curveto(66.37208495,309.02723219)(66.36708495,309.10223211)(66.37708557,309.18223694)
\curveto(66.38708493,309.27223194)(66.39208493,309.35723186)(66.39208557,309.43723694)
\curveto(66.39208493,309.52723169)(66.38708493,309.62723159)(66.37708557,309.73723694)
\curveto(66.36708495,309.85723136)(66.37208495,309.95723126)(66.39208557,310.03723694)
\lineto(66.39208557,310.32223694)
\lineto(66.43708557,310.95223694)
\curveto(66.44708487,311.05223016)(66.45708486,311.14723007)(66.46708557,311.23723694)
\lineto(66.49708557,311.53723694)
\curveto(66.5170848,311.58722963)(66.5220848,311.63722958)(66.51208557,311.68723694)
\curveto(66.51208481,311.74722947)(66.5220848,311.80222941)(66.54208557,311.85223694)
\curveto(66.59208473,312.02222919)(66.63208469,312.18722903)(66.66208557,312.34723694)
\curveto(66.69208463,312.5172287)(66.74208458,312.67722854)(66.81208557,312.82723694)
\curveto(67.00208432,313.28722793)(67.2220841,313.66222755)(67.47208557,313.95223694)
\curveto(67.73208359,314.24222697)(68.09208323,314.48722673)(68.55208557,314.68723694)
\curveto(68.68208264,314.73722648)(68.81208251,314.77222644)(68.94208557,314.79223694)
\curveto(69.08208224,314.8122264)(69.2220821,314.83722638)(69.36208557,314.86723694)
\curveto(69.43208189,314.87722634)(69.49708182,314.88222633)(69.55708557,314.88223694)
\curveto(69.6170817,314.88222633)(69.68208164,314.88722633)(69.75208557,314.89723694)
\curveto(70.58208074,314.9172263)(71.25208007,314.76722645)(71.76208557,314.44723694)
\curveto(72.27207905,314.13722708)(72.65207867,313.69722752)(72.90208557,313.12723694)
\curveto(72.95207837,313.00722821)(72.99707832,312.88222833)(73.03708557,312.75223694)
\curveto(73.07707824,312.62222859)(73.1220782,312.48722873)(73.17208557,312.34723694)
\curveto(73.19207813,312.26722895)(73.20707811,312.18222903)(73.21708557,312.09223694)
\lineto(73.27708557,311.85223694)
\curveto(73.30707801,311.74222947)(73.322078,311.63222958)(73.32208557,311.52223694)
\curveto(73.33207799,311.4122298)(73.34707797,311.30222991)(73.36708557,311.19223694)
\curveto(73.38707793,311.14223007)(73.39207793,311.09723012)(73.38208557,311.05723694)
\curveto(73.38207794,311.0172302)(73.38707793,310.97723024)(73.39708557,310.93723694)
\curveto(73.40707791,310.88723033)(73.40707791,310.83223038)(73.39708557,310.77223694)
\curveto(73.39707792,310.72223049)(73.40207792,310.67223054)(73.41208557,310.62223694)
\lineto(73.41208557,310.48723694)
\curveto(73.43207789,310.42723079)(73.43207789,310.35723086)(73.41208557,310.27723694)
\curveto(73.40207792,310.20723101)(73.40707791,310.14223107)(73.42708557,310.08223694)
\curveto(73.43707788,310.05223116)(73.44207788,310.0122312)(73.44208557,309.96223694)
\lineto(73.44208557,309.84223694)
\lineto(73.44208557,309.37723694)
\moveto(71.89708557,307.05223694)
\curveto(71.99707932,307.37223384)(72.05707926,307.73723348)(72.07708557,308.14723694)
\curveto(72.09707922,308.55723266)(72.10707921,308.96723225)(72.10708557,309.37723694)
\curveto(72.10707921,309.80723141)(72.09707922,310.22723099)(72.07708557,310.63723694)
\curveto(72.05707926,311.04723017)(72.01207931,311.43222978)(71.94208557,311.79223694)
\curveto(71.87207945,312.15222906)(71.76207956,312.47222874)(71.61208557,312.75223694)
\curveto(71.47207985,313.04222817)(71.27708004,313.27722794)(71.02708557,313.45723694)
\curveto(70.86708045,313.56722765)(70.68708063,313.64722757)(70.48708557,313.69723694)
\curveto(70.28708103,313.75722746)(70.04208128,313.78722743)(69.75208557,313.78723694)
\curveto(69.73208159,313.76722745)(69.69708162,313.75722746)(69.64708557,313.75723694)
\curveto(69.59708172,313.76722745)(69.55708176,313.76722745)(69.52708557,313.75723694)
\curveto(69.44708187,313.73722748)(69.37208195,313.7172275)(69.30208557,313.69723694)
\curveto(69.24208208,313.68722753)(69.17708214,313.66722755)(69.10708557,313.63723694)
\curveto(68.83708248,313.5172277)(68.6170827,313.34722787)(68.44708557,313.12723694)
\curveto(68.28708303,312.9172283)(68.15208317,312.67222854)(68.04208557,312.39223694)
\curveto(67.99208333,312.28222893)(67.95208337,312.16222905)(67.92208557,312.03223694)
\curveto(67.90208342,311.9122293)(67.87708344,311.78722943)(67.84708557,311.65723694)
\curveto(67.82708349,311.60722961)(67.8170835,311.55222966)(67.81708557,311.49223694)
\curveto(67.8170835,311.44222977)(67.81208351,311.39222982)(67.80208557,311.34223694)
\curveto(67.79208353,311.25222996)(67.78208354,311.15723006)(67.77208557,311.05723694)
\curveto(67.76208356,310.96723025)(67.75208357,310.87223034)(67.74208557,310.77223694)
\curveto(67.74208358,310.69223052)(67.73708358,310.60723061)(67.72708557,310.51723694)
\lineto(67.72708557,310.27723694)
\lineto(67.72708557,310.09723694)
\curveto(67.7170836,310.06723115)(67.71208361,310.03223118)(67.71208557,309.99223694)
\lineto(67.71208557,309.85723694)
\lineto(67.71208557,309.40723694)
\curveto(67.71208361,309.32723189)(67.70708361,309.24223197)(67.69708557,309.15223694)
\curveto(67.69708362,309.07223214)(67.70708361,308.99723222)(67.72708557,308.92723694)
\lineto(67.72708557,308.65723694)
\curveto(67.72708359,308.63723258)(67.7220836,308.60723261)(67.71208557,308.56723694)
\curveto(67.71208361,308.53723268)(67.7170836,308.5122327)(67.72708557,308.49223694)
\curveto(67.73708358,308.39223282)(67.74208358,308.29223292)(67.74208557,308.19223694)
\curveto(67.75208357,308.10223311)(67.76208356,308.00223321)(67.77208557,307.89223694)
\curveto(67.80208352,307.77223344)(67.8170835,307.64723357)(67.81708557,307.51723694)
\curveto(67.82708349,307.39723382)(67.85208347,307.28223393)(67.89208557,307.17223694)
\curveto(67.97208335,306.87223434)(68.05708326,306.60723461)(68.14708557,306.37723694)
\curveto(68.24708307,306.14723507)(68.39208293,305.93223528)(68.58208557,305.73223694)
\curveto(68.79208253,305.53223568)(69.05708226,305.38223583)(69.37708557,305.28223694)
\curveto(69.4170819,305.26223595)(69.45208187,305.25223596)(69.48208557,305.25223694)
\curveto(69.5220818,305.26223595)(69.56708175,305.25723596)(69.61708557,305.23723694)
\curveto(69.65708166,305.22723599)(69.72708159,305.217236)(69.82708557,305.20723694)
\curveto(69.93708138,305.19723602)(70.0220813,305.20223601)(70.08208557,305.22223694)
\curveto(70.15208117,305.24223597)(70.2220811,305.25223596)(70.29208557,305.25223694)
\curveto(70.36208096,305.26223595)(70.42708089,305.27723594)(70.48708557,305.29723694)
\curveto(70.68708063,305.35723586)(70.86708045,305.44223577)(71.02708557,305.55223694)
\curveto(71.05708026,305.57223564)(71.08208024,305.59223562)(71.10208557,305.61223694)
\lineto(71.16208557,305.67223694)
\curveto(71.20208012,305.69223552)(71.25208007,305.73223548)(71.31208557,305.79223694)
\curveto(71.41207991,305.93223528)(71.49707982,306.06223515)(71.56708557,306.18223694)
\curveto(71.63707968,306.30223491)(71.70707961,306.44723477)(71.77708557,306.61723694)
\curveto(71.80707951,306.68723453)(71.82707949,306.75723446)(71.83708557,306.82723694)
\curveto(71.85707946,306.89723432)(71.87707944,306.97223424)(71.89708557,307.05223694)
}
}
{
\newrgbcolor{curcolor}{0 0 0}
\pscustom[linestyle=none,fillstyle=solid,fillcolor=curcolor]
{
\newpath
\moveto(64.9724762,382.07295013)
\curveto(65.04246855,382.02294667)(65.08246851,381.95294674)(65.0924762,381.86295013)
\curveto(65.11246848,381.77294692)(65.12246847,381.66794703)(65.1224762,381.54795013)
\curveto(65.12246847,381.4979472)(65.11746848,381.44794725)(65.1074762,381.39795013)
\curveto(65.10746849,381.34794735)(65.0974685,381.30294739)(65.0774762,381.26295013)
\curveto(65.04746855,381.17294752)(64.98746861,381.11294758)(64.8974762,381.08295013)
\curveto(64.81746878,381.06294763)(64.72246887,381.05294764)(64.6124762,381.05295013)
\lineto(64.2974762,381.05295013)
\curveto(64.18746941,381.06294763)(64.08246951,381.05294764)(63.9824762,381.02295013)
\curveto(63.84246975,380.9929477)(63.75246984,380.91294778)(63.7124762,380.78295013)
\curveto(63.6924699,380.71294798)(63.68246991,380.62794807)(63.6824762,380.52795013)
\lineto(63.6824762,380.25795013)
\lineto(63.6824762,379.31295013)
\lineto(63.6824762,378.98295013)
\curveto(63.68246991,378.87294982)(63.66246993,378.78794991)(63.6224762,378.72795013)
\curveto(63.58247001,378.66795003)(63.53247006,378.62795007)(63.4724762,378.60795013)
\curveto(63.42247017,378.5979501)(63.35747024,378.58295011)(63.2774762,378.56295013)
\lineto(63.0824762,378.56295013)
\curveto(62.96247063,378.56295013)(62.85747074,378.56795013)(62.7674762,378.57795013)
\curveto(62.67747092,378.5979501)(62.60747099,378.64795005)(62.5574762,378.72795013)
\curveto(62.52747107,378.77794992)(62.51247108,378.84794985)(62.5124762,378.93795013)
\lineto(62.5124762,379.23795013)
\lineto(62.5124762,380.27295013)
\curveto(62.51247108,380.43294826)(62.50247109,380.57794812)(62.4824762,380.70795013)
\curveto(62.47247112,380.84794785)(62.41747118,380.94294775)(62.3174762,380.99295013)
\curveto(62.26747133,381.01294768)(62.1974714,381.02794767)(62.1074762,381.03795013)
\curveto(62.02747157,381.04794765)(61.93747166,381.05294764)(61.8374762,381.05295013)
\lineto(61.5524762,381.05295013)
\lineto(61.3124762,381.05295013)
\lineto(59.0474762,381.05295013)
\curveto(58.95747464,381.05294764)(58.85247474,381.04794765)(58.7324762,381.03795013)
\lineto(58.4024762,381.03795013)
\curveto(58.2924753,381.03794766)(58.1924754,381.04794765)(58.1024762,381.06795013)
\curveto(58.01247558,381.08794761)(57.95247564,381.12294757)(57.9224762,381.17295013)
\curveto(57.87247572,381.24294745)(57.84747575,381.33794736)(57.8474762,381.45795013)
\lineto(57.8474762,381.80295013)
\lineto(57.8474762,382.07295013)
\curveto(57.88747571,382.24294645)(57.94247565,382.38294631)(58.0124762,382.49295013)
\curveto(58.08247551,382.60294609)(58.16247543,382.71794598)(58.2524762,382.83795013)
\lineto(58.6124762,383.37795013)
\curveto(59.05247454,384.00794469)(59.48747411,384.62794407)(59.9174762,385.23795013)
\lineto(61.2374762,387.09795013)
\curveto(61.3974722,387.32794137)(61.55247204,387.54794115)(61.7024762,387.75795013)
\curveto(61.85247174,387.97794072)(62.00747159,388.20294049)(62.1674762,388.43295013)
\curveto(62.21747138,388.50294019)(62.26747133,388.56794013)(62.3174762,388.62795013)
\curveto(62.36747123,388.69794)(62.41747118,388.77293992)(62.4674762,388.85295013)
\lineto(62.5274762,388.94295013)
\curveto(62.55747104,388.98293971)(62.58747101,389.01293968)(62.6174762,389.03295013)
\curveto(62.65747094,389.06293963)(62.6974709,389.08293961)(62.7374762,389.09295013)
\curveto(62.77747082,389.11293958)(62.82247077,389.13293956)(62.8724762,389.15295013)
\curveto(62.8924707,389.15293954)(62.91247068,389.14793955)(62.9324762,389.13795013)
\curveto(62.96247063,389.13793956)(62.98747061,389.14793955)(63.0074762,389.16795013)
\curveto(63.13747046,389.16793953)(63.25747034,389.16293953)(63.3674762,389.15295013)
\curveto(63.47747012,389.14293955)(63.55747004,389.0979396)(63.6074762,389.01795013)
\curveto(63.64746995,388.96793973)(63.66746993,388.8979398)(63.6674762,388.80795013)
\curveto(63.67746992,388.71793998)(63.68246991,388.62294007)(63.6824762,388.52295013)
\lineto(63.6824762,383.06295013)
\curveto(63.68246991,382.9929457)(63.67746992,382.91794578)(63.6674762,382.83795013)
\curveto(63.66746993,382.76794593)(63.67246992,382.697946)(63.6824762,382.62795013)
\lineto(63.6824762,382.52295013)
\curveto(63.70246989,382.47294622)(63.71746988,382.41794628)(63.7274762,382.35795013)
\curveto(63.73746986,382.30794639)(63.76246983,382.26794643)(63.8024762,382.23795013)
\curveto(63.87246972,382.18794651)(63.95746964,382.15794654)(64.0574762,382.14795013)
\lineto(64.3874762,382.14795013)
\curveto(64.4974691,382.14794655)(64.60246899,382.14294655)(64.7024762,382.13295013)
\curveto(64.81246878,382.13294656)(64.90246869,382.11294658)(64.9724762,382.07295013)
\moveto(62.4074762,382.26795013)
\curveto(62.48747111,382.37794632)(62.52247107,382.54794615)(62.5124762,382.77795013)
\lineto(62.5124762,383.39295013)
\lineto(62.5124762,385.86795013)
\lineto(62.5124762,386.18295013)
\curveto(62.52247107,386.30294239)(62.51747108,386.40294229)(62.4974762,386.48295013)
\lineto(62.4974762,386.63295013)
\curveto(62.4974711,386.72294197)(62.48247111,386.80794189)(62.4524762,386.88795013)
\curveto(62.44247115,386.90794179)(62.43247116,386.91794178)(62.4224762,386.91795013)
\lineto(62.3774762,386.96295013)
\curveto(62.35747124,386.97294172)(62.32747127,386.97794172)(62.2874762,386.97795013)
\curveto(62.26747133,386.95794174)(62.24747135,386.94294175)(62.2274762,386.93295013)
\curveto(62.21747138,386.93294176)(62.20247139,386.92794177)(62.1824762,386.91795013)
\curveto(62.12247147,386.86794183)(62.06247153,386.7979419)(62.0024762,386.70795013)
\curveto(61.94247165,386.61794208)(61.88747171,386.53794216)(61.8374762,386.46795013)
\curveto(61.73747186,386.32794237)(61.64247195,386.18294251)(61.5524762,386.03295013)
\curveto(61.46247213,385.8929428)(61.36747223,385.75294294)(61.2674762,385.61295013)
\lineto(60.7274762,384.83295013)
\curveto(60.55747304,384.57294412)(60.38247321,384.31294438)(60.2024762,384.05295013)
\curveto(60.12247347,383.94294475)(60.04747355,383.83794486)(59.9774762,383.73795013)
\lineto(59.7674762,383.43795013)
\curveto(59.71747388,383.35794534)(59.66747393,383.28294541)(59.6174762,383.21295013)
\curveto(59.57747402,383.14294555)(59.53247406,383.06794563)(59.4824762,382.98795013)
\curveto(59.43247416,382.92794577)(59.38247421,382.86294583)(59.3324762,382.79295013)
\curveto(59.2924743,382.73294596)(59.25247434,382.66294603)(59.2124762,382.58295013)
\curveto(59.17247442,382.52294617)(59.14747445,382.45294624)(59.1374762,382.37295013)
\curveto(59.12747447,382.30294639)(59.16247443,382.24794645)(59.2424762,382.20795013)
\curveto(59.31247428,382.15794654)(59.42247417,382.13294656)(59.5724762,382.13295013)
\curveto(59.73247386,382.14294655)(59.86747373,382.14794655)(59.9774762,382.14795013)
\lineto(61.6574762,382.14795013)
\lineto(62.0924762,382.14795013)
\curveto(62.24247135,382.14794655)(62.34747125,382.18794651)(62.4074762,382.26795013)
}
}
{
\newrgbcolor{curcolor}{0 0 0}
\pscustom[linestyle=none,fillstyle=solid,fillcolor=curcolor]
{
\newpath
\moveto(73.44208557,383.66295013)
\lineto(73.44208557,383.40795013)
\curveto(73.45207787,383.32794537)(73.44707787,383.25294544)(73.42708557,383.18295013)
\lineto(73.42708557,382.94295013)
\lineto(73.42708557,382.77795013)
\curveto(73.40707791,382.67794602)(73.39707792,382.57294612)(73.39708557,382.46295013)
\curveto(73.39707792,382.36294633)(73.38707793,382.26294643)(73.36708557,382.16295013)
\lineto(73.36708557,382.01295013)
\curveto(73.33707798,381.87294682)(73.317078,381.73294696)(73.30708557,381.59295013)
\curveto(73.29707802,381.46294723)(73.27207805,381.33294736)(73.23208557,381.20295013)
\curveto(73.21207811,381.12294757)(73.19207813,381.03794766)(73.17208557,380.94795013)
\lineto(73.11208557,380.70795013)
\lineto(72.99208557,380.40795013)
\curveto(72.96207836,380.31794838)(72.92707839,380.22794847)(72.88708557,380.13795013)
\curveto(72.78707853,379.91794878)(72.65207867,379.70294899)(72.48208557,379.49295013)
\curveto(72.322079,379.28294941)(72.14707917,379.11294958)(71.95708557,378.98295013)
\curveto(71.90707941,378.94294975)(71.84707947,378.90294979)(71.77708557,378.86295013)
\curveto(71.7170796,378.83294986)(71.65707966,378.7979499)(71.59708557,378.75795013)
\curveto(71.5170798,378.70794999)(71.4220799,378.66795003)(71.31208557,378.63795013)
\curveto(71.20208012,378.60795009)(71.09708022,378.57795012)(70.99708557,378.54795013)
\curveto(70.88708043,378.50795019)(70.77708054,378.48295021)(70.66708557,378.47295013)
\curveto(70.55708076,378.46295023)(70.44208088,378.44795025)(70.32208557,378.42795013)
\curveto(70.28208104,378.41795028)(70.23708108,378.41795028)(70.18708557,378.42795013)
\curveto(70.14708117,378.42795027)(70.10708121,378.42295027)(70.06708557,378.41295013)
\curveto(70.02708129,378.40295029)(69.97208135,378.3979503)(69.90208557,378.39795013)
\curveto(69.83208149,378.3979503)(69.78208154,378.40295029)(69.75208557,378.41295013)
\curveto(69.70208162,378.43295026)(69.65708166,378.43795026)(69.61708557,378.42795013)
\curveto(69.57708174,378.41795028)(69.54208178,378.41795028)(69.51208557,378.42795013)
\lineto(69.42208557,378.42795013)
\curveto(69.36208196,378.44795025)(69.29708202,378.46295023)(69.22708557,378.47295013)
\curveto(69.16708215,378.47295022)(69.10208222,378.47795022)(69.03208557,378.48795013)
\curveto(68.86208246,378.53795016)(68.70208262,378.58795011)(68.55208557,378.63795013)
\curveto(68.40208292,378.68795001)(68.25708306,378.75294994)(68.11708557,378.83295013)
\curveto(68.06708325,378.87294982)(68.01208331,378.90294979)(67.95208557,378.92295013)
\curveto(67.90208342,378.95294974)(67.85208347,378.98794971)(67.80208557,379.02795013)
\curveto(67.56208376,379.20794949)(67.36208396,379.42794927)(67.20208557,379.68795013)
\curveto(67.04208428,379.94794875)(66.90208442,380.23294846)(66.78208557,380.54295013)
\curveto(66.7220846,380.68294801)(66.67708464,380.82294787)(66.64708557,380.96295013)
\curveto(66.6170847,381.11294758)(66.58208474,381.26794743)(66.54208557,381.42795013)
\curveto(66.5220848,381.53794716)(66.50708481,381.64794705)(66.49708557,381.75795013)
\curveto(66.48708483,381.86794683)(66.47208485,381.97794672)(66.45208557,382.08795013)
\curveto(66.44208488,382.12794657)(66.43708488,382.16794653)(66.43708557,382.20795013)
\curveto(66.44708487,382.24794645)(66.44708487,382.28794641)(66.43708557,382.32795013)
\curveto(66.42708489,382.37794632)(66.4220849,382.42794627)(66.42208557,382.47795013)
\lineto(66.42208557,382.64295013)
\curveto(66.40208492,382.692946)(66.39708492,382.74294595)(66.40708557,382.79295013)
\curveto(66.4170849,382.85294584)(66.4170849,382.90794579)(66.40708557,382.95795013)
\curveto(66.39708492,382.9979457)(66.39708492,383.04294565)(66.40708557,383.09295013)
\curveto(66.4170849,383.14294555)(66.41208491,383.1929455)(66.39208557,383.24295013)
\curveto(66.37208495,383.31294538)(66.36708495,383.38794531)(66.37708557,383.46795013)
\curveto(66.38708493,383.55794514)(66.39208493,383.64294505)(66.39208557,383.72295013)
\curveto(66.39208493,383.81294488)(66.38708493,383.91294478)(66.37708557,384.02295013)
\curveto(66.36708495,384.14294455)(66.37208495,384.24294445)(66.39208557,384.32295013)
\lineto(66.39208557,384.60795013)
\lineto(66.43708557,385.23795013)
\curveto(66.44708487,385.33794336)(66.45708486,385.43294326)(66.46708557,385.52295013)
\lineto(66.49708557,385.82295013)
\curveto(66.5170848,385.87294282)(66.5220848,385.92294277)(66.51208557,385.97295013)
\curveto(66.51208481,386.03294266)(66.5220848,386.08794261)(66.54208557,386.13795013)
\curveto(66.59208473,386.30794239)(66.63208469,386.47294222)(66.66208557,386.63295013)
\curveto(66.69208463,386.80294189)(66.74208458,386.96294173)(66.81208557,387.11295013)
\curveto(67.00208432,387.57294112)(67.2220841,387.94794075)(67.47208557,388.23795013)
\curveto(67.73208359,388.52794017)(68.09208323,388.77293992)(68.55208557,388.97295013)
\curveto(68.68208264,389.02293967)(68.81208251,389.05793964)(68.94208557,389.07795013)
\curveto(69.08208224,389.0979396)(69.2220821,389.12293957)(69.36208557,389.15295013)
\curveto(69.43208189,389.16293953)(69.49708182,389.16793953)(69.55708557,389.16795013)
\curveto(69.6170817,389.16793953)(69.68208164,389.17293952)(69.75208557,389.18295013)
\curveto(70.58208074,389.20293949)(71.25208007,389.05293964)(71.76208557,388.73295013)
\curveto(72.27207905,388.42294027)(72.65207867,387.98294071)(72.90208557,387.41295013)
\curveto(72.95207837,387.2929414)(72.99707832,387.16794153)(73.03708557,387.03795013)
\curveto(73.07707824,386.90794179)(73.1220782,386.77294192)(73.17208557,386.63295013)
\curveto(73.19207813,386.55294214)(73.20707811,386.46794223)(73.21708557,386.37795013)
\lineto(73.27708557,386.13795013)
\curveto(73.30707801,386.02794267)(73.322078,385.91794278)(73.32208557,385.80795013)
\curveto(73.33207799,385.697943)(73.34707797,385.58794311)(73.36708557,385.47795013)
\curveto(73.38707793,385.42794327)(73.39207793,385.38294331)(73.38208557,385.34295013)
\curveto(73.38207794,385.30294339)(73.38707793,385.26294343)(73.39708557,385.22295013)
\curveto(73.40707791,385.17294352)(73.40707791,385.11794358)(73.39708557,385.05795013)
\curveto(73.39707792,385.00794369)(73.40207792,384.95794374)(73.41208557,384.90795013)
\lineto(73.41208557,384.77295013)
\curveto(73.43207789,384.71294398)(73.43207789,384.64294405)(73.41208557,384.56295013)
\curveto(73.40207792,384.4929442)(73.40707791,384.42794427)(73.42708557,384.36795013)
\curveto(73.43707788,384.33794436)(73.44207788,384.2979444)(73.44208557,384.24795013)
\lineto(73.44208557,384.12795013)
\lineto(73.44208557,383.66295013)
\moveto(71.89708557,381.33795013)
\curveto(71.99707932,381.65794704)(72.05707926,382.02294667)(72.07708557,382.43295013)
\curveto(72.09707922,382.84294585)(72.10707921,383.25294544)(72.10708557,383.66295013)
\curveto(72.10707921,384.0929446)(72.09707922,384.51294418)(72.07708557,384.92295013)
\curveto(72.05707926,385.33294336)(72.01207931,385.71794298)(71.94208557,386.07795013)
\curveto(71.87207945,386.43794226)(71.76207956,386.75794194)(71.61208557,387.03795013)
\curveto(71.47207985,387.32794137)(71.27708004,387.56294113)(71.02708557,387.74295013)
\curveto(70.86708045,387.85294084)(70.68708063,387.93294076)(70.48708557,387.98295013)
\curveto(70.28708103,388.04294065)(70.04208128,388.07294062)(69.75208557,388.07295013)
\curveto(69.73208159,388.05294064)(69.69708162,388.04294065)(69.64708557,388.04295013)
\curveto(69.59708172,388.05294064)(69.55708176,388.05294064)(69.52708557,388.04295013)
\curveto(69.44708187,388.02294067)(69.37208195,388.00294069)(69.30208557,387.98295013)
\curveto(69.24208208,387.97294072)(69.17708214,387.95294074)(69.10708557,387.92295013)
\curveto(68.83708248,387.80294089)(68.6170827,387.63294106)(68.44708557,387.41295013)
\curveto(68.28708303,387.20294149)(68.15208317,386.95794174)(68.04208557,386.67795013)
\curveto(67.99208333,386.56794213)(67.95208337,386.44794225)(67.92208557,386.31795013)
\curveto(67.90208342,386.1979425)(67.87708344,386.07294262)(67.84708557,385.94295013)
\curveto(67.82708349,385.8929428)(67.8170835,385.83794286)(67.81708557,385.77795013)
\curveto(67.8170835,385.72794297)(67.81208351,385.67794302)(67.80208557,385.62795013)
\curveto(67.79208353,385.53794316)(67.78208354,385.44294325)(67.77208557,385.34295013)
\curveto(67.76208356,385.25294344)(67.75208357,385.15794354)(67.74208557,385.05795013)
\curveto(67.74208358,384.97794372)(67.73708358,384.8929438)(67.72708557,384.80295013)
\lineto(67.72708557,384.56295013)
\lineto(67.72708557,384.38295013)
\curveto(67.7170836,384.35294434)(67.71208361,384.31794438)(67.71208557,384.27795013)
\lineto(67.71208557,384.14295013)
\lineto(67.71208557,383.69295013)
\curveto(67.71208361,383.61294508)(67.70708361,383.52794517)(67.69708557,383.43795013)
\curveto(67.69708362,383.35794534)(67.70708361,383.28294541)(67.72708557,383.21295013)
\lineto(67.72708557,382.94295013)
\curveto(67.72708359,382.92294577)(67.7220836,382.8929458)(67.71208557,382.85295013)
\curveto(67.71208361,382.82294587)(67.7170836,382.7979459)(67.72708557,382.77795013)
\curveto(67.73708358,382.67794602)(67.74208358,382.57794612)(67.74208557,382.47795013)
\curveto(67.75208357,382.38794631)(67.76208356,382.28794641)(67.77208557,382.17795013)
\curveto(67.80208352,382.05794664)(67.8170835,381.93294676)(67.81708557,381.80295013)
\curveto(67.82708349,381.68294701)(67.85208347,381.56794713)(67.89208557,381.45795013)
\curveto(67.97208335,381.15794754)(68.05708326,380.8929478)(68.14708557,380.66295013)
\curveto(68.24708307,380.43294826)(68.39208293,380.21794848)(68.58208557,380.01795013)
\curveto(68.79208253,379.81794888)(69.05708226,379.66794903)(69.37708557,379.56795013)
\curveto(69.4170819,379.54794915)(69.45208187,379.53794916)(69.48208557,379.53795013)
\curveto(69.5220818,379.54794915)(69.56708175,379.54294915)(69.61708557,379.52295013)
\curveto(69.65708166,379.51294918)(69.72708159,379.50294919)(69.82708557,379.49295013)
\curveto(69.93708138,379.48294921)(70.0220813,379.48794921)(70.08208557,379.50795013)
\curveto(70.15208117,379.52794917)(70.2220811,379.53794916)(70.29208557,379.53795013)
\curveto(70.36208096,379.54794915)(70.42708089,379.56294913)(70.48708557,379.58295013)
\curveto(70.68708063,379.64294905)(70.86708045,379.72794897)(71.02708557,379.83795013)
\curveto(71.05708026,379.85794884)(71.08208024,379.87794882)(71.10208557,379.89795013)
\lineto(71.16208557,379.95795013)
\curveto(71.20208012,379.97794872)(71.25208007,380.01794868)(71.31208557,380.07795013)
\curveto(71.41207991,380.21794848)(71.49707982,380.34794835)(71.56708557,380.46795013)
\curveto(71.63707968,380.58794811)(71.70707961,380.73294796)(71.77708557,380.90295013)
\curveto(71.80707951,380.97294772)(71.82707949,381.04294765)(71.83708557,381.11295013)
\curveto(71.85707946,381.18294751)(71.87707944,381.25794744)(71.89708557,381.33795013)
}
}
{
\newrgbcolor{curcolor}{0 0 0}
\pscustom[linestyle=none,fillstyle=solid,fillcolor=curcolor]
{
\newpath
\moveto(765.90647461,382.23397308)
\curveto(766.88646811,382.25396213)(767.70646729,382.09396229)(768.36647461,381.75397308)
\curveto(769.03646596,381.42396296)(769.55646544,380.96396342)(769.92647461,380.37397308)
\curveto(770.02646497,380.21396417)(770.10646489,380.05896432)(770.16647461,379.90897308)
\curveto(770.23646476,379.76896461)(770.30146469,379.59896478)(770.36147461,379.39897308)
\curveto(770.38146461,379.34896503)(770.40146459,379.2789651)(770.42147461,379.18897308)
\curveto(770.44146455,379.10896527)(770.43646456,379.03396535)(770.40647461,378.96397308)
\curveto(770.38646461,378.90396548)(770.34646465,378.86396552)(770.28647461,378.84397308)
\curveto(770.23646476,378.83396555)(770.18146481,378.81896556)(770.12147461,378.79897308)
\lineto(769.97147461,378.79897308)
\curveto(769.94146505,378.78896559)(769.90146509,378.7839656)(769.85147461,378.78397308)
\lineto(769.73147461,378.78397308)
\curveto(769.5914654,378.7839656)(769.46146553,378.78896559)(769.34147461,378.79897308)
\curveto(769.23146576,378.81896556)(769.15146584,378.86896551)(769.10147461,378.94897308)
\curveto(769.03146596,379.04896533)(768.97646602,379.16396522)(768.93647461,379.29397308)
\curveto(768.8964661,379.42396496)(768.84146615,379.54396484)(768.77147461,379.65397308)
\curveto(768.64146635,379.87396451)(768.4914665,380.06396432)(768.32147461,380.22397308)
\curveto(768.16146683,380.383964)(767.97146702,380.53396385)(767.75147461,380.67397308)
\curveto(767.63146736,380.75396363)(767.4964675,380.81396357)(767.34647461,380.85397308)
\curveto(767.20646779,380.89396349)(767.06146793,380.93396345)(766.91147461,380.97397308)
\curveto(766.80146819,381.00396338)(766.67646832,381.02396336)(766.53647461,381.03397308)
\curveto(766.3964686,381.05396333)(766.24646875,381.06396332)(766.08647461,381.06397308)
\curveto(765.93646906,381.06396332)(765.78646921,381.05396333)(765.63647461,381.03397308)
\curveto(765.4964695,381.02396336)(765.37646962,381.00396338)(765.27647461,380.97397308)
\curveto(765.17646982,380.95396343)(765.08146991,380.93396345)(764.99147461,380.91397308)
\curveto(764.90147009,380.89396349)(764.81147018,380.86396352)(764.72147461,380.82397308)
\curveto(763.88147111,380.47396391)(763.27647172,379.87396451)(762.90647461,379.02397308)
\curveto(762.83647216,378.8839655)(762.77647222,378.73396565)(762.72647461,378.57397308)
\curveto(762.68647231,378.42396596)(762.64147235,378.26896611)(762.59147461,378.10897308)
\curveto(762.57147242,378.04896633)(762.56147243,377.9839664)(762.56147461,377.91397308)
\curveto(762.56147243,377.85396653)(762.55147244,377.79396659)(762.53147461,377.73397308)
\curveto(762.52147247,377.69396669)(762.51647248,377.65896672)(762.51647461,377.62897308)
\curveto(762.51647248,377.59896678)(762.51147248,377.56396682)(762.50147461,377.52397308)
\curveto(762.48147251,377.41396697)(762.46647253,377.29896708)(762.45647461,377.17897308)
\lineto(762.45647461,376.83397308)
\curveto(762.45647254,376.76396762)(762.45147254,376.68896769)(762.44147461,376.60897308)
\curveto(762.44147255,376.53896784)(762.44647255,376.47396791)(762.45647461,376.41397308)
\lineto(762.45647461,376.26397308)
\curveto(762.47647252,376.19396819)(762.48147251,376.12396826)(762.47147461,376.05397308)
\curveto(762.47147252,375.9839684)(762.48147251,375.91396847)(762.50147461,375.84397308)
\curveto(762.52147247,375.7839686)(762.52647247,375.72396866)(762.51647461,375.66397308)
\curveto(762.51647248,375.60396878)(762.52647247,375.54896883)(762.54647461,375.49897308)
\curveto(762.57647242,375.36896901)(762.60147239,375.23896914)(762.62147461,375.10897308)
\curveto(762.65147234,374.98896939)(762.68647231,374.86896951)(762.72647461,374.74897308)
\curveto(762.8964721,374.24897013)(763.11647188,373.81897056)(763.38647461,373.45897308)
\curveto(763.65647134,373.10897127)(764.01147098,372.81897156)(764.45147461,372.58897308)
\curveto(764.5914704,372.51897186)(764.73147026,372.46397192)(764.87147461,372.42397308)
\curveto(765.02146997,372.383972)(765.18146981,372.33897204)(765.35147461,372.28897308)
\curveto(765.42146957,372.26897211)(765.48646951,372.25897212)(765.54647461,372.25897308)
\curveto(765.60646939,372.26897211)(765.67646932,372.26397212)(765.75647461,372.24397308)
\curveto(765.80646919,372.23397215)(765.8964691,372.22397216)(766.02647461,372.21397308)
\curveto(766.15646884,372.21397217)(766.25146874,372.22397216)(766.31147461,372.24397308)
\lineto(766.41647461,372.24397308)
\curveto(766.45646854,372.25397213)(766.4964685,372.25397213)(766.53647461,372.24397308)
\curveto(766.57646842,372.24397214)(766.61646838,372.25397213)(766.65647461,372.27397308)
\curveto(766.75646824,372.29397209)(766.85146814,372.30897207)(766.94147461,372.31897308)
\curveto(767.04146795,372.33897204)(767.13646786,372.36897201)(767.22647461,372.40897308)
\curveto(768.00646699,372.72897165)(768.55646644,373.25397113)(768.87647461,373.98397308)
\curveto(768.95646604,374.16397022)(769.03146596,374.37897)(769.10147461,374.62897308)
\curveto(769.12146587,374.71896966)(769.13646586,374.80896957)(769.14647461,374.89897308)
\curveto(769.16646583,374.99896938)(769.20146579,375.08896929)(769.25147461,375.16897308)
\curveto(769.30146569,375.24896913)(769.38146561,375.29396909)(769.49147461,375.30397308)
\curveto(769.60146539,375.31396907)(769.72146527,375.31896906)(769.85147461,375.31897308)
\lineto(770.00147461,375.31897308)
\curveto(770.05146494,375.31896906)(770.0964649,375.31396907)(770.13647461,375.30397308)
\lineto(770.24147461,375.30397308)
\lineto(770.33147461,375.27397308)
\curveto(770.37146462,375.27396911)(770.40146459,375.26396912)(770.42147461,375.24397308)
\curveto(770.4914645,375.20396918)(770.53146446,375.12896925)(770.54147461,375.01897308)
\curveto(770.55146444,374.91896946)(770.54146445,374.81896956)(770.51147461,374.71897308)
\curveto(770.45146454,374.48896989)(770.3964646,374.26897011)(770.34647461,374.05897308)
\curveto(770.2964647,373.84897053)(770.22146477,373.64897073)(770.12147461,373.45897308)
\curveto(770.04146495,373.32897105)(769.96646503,373.20397118)(769.89647461,373.08397308)
\curveto(769.83646516,372.96397142)(769.76646523,372.84397154)(769.68647461,372.72397308)
\curveto(769.50646549,372.46397192)(769.28146571,372.22397216)(769.01147461,372.00397308)
\curveto(768.75146624,371.79397259)(768.46646653,371.61897276)(768.15647461,371.47897308)
\curveto(768.04646695,371.42897295)(767.93646706,371.38897299)(767.82647461,371.35897308)
\curveto(767.72646727,371.32897305)(767.62146737,371.29397309)(767.51147461,371.25397308)
\curveto(767.40146759,371.21397317)(767.28646771,371.18897319)(767.16647461,371.17897308)
\curveto(767.05646794,371.15897322)(766.94146805,371.13897324)(766.82147461,371.11897308)
\curveto(766.77146822,371.09897328)(766.72646827,371.09397329)(766.68647461,371.10397308)
\curveto(766.64646835,371.10397328)(766.60646839,371.09897328)(766.56647461,371.08897308)
\curveto(766.50646849,371.0789733)(766.44646855,371.07397331)(766.38647461,371.07397308)
\curveto(766.32646867,371.07397331)(766.26146873,371.06897331)(766.19147461,371.05897308)
\curveto(766.16146883,371.04897333)(766.0914689,371.04897333)(765.98147461,371.05897308)
\curveto(765.88146911,371.05897332)(765.81646918,371.06397332)(765.78647461,371.07397308)
\curveto(765.73646926,371.0839733)(765.68646931,371.08897329)(765.63647461,371.08897308)
\curveto(765.5964694,371.0789733)(765.55146944,371.0789733)(765.50147461,371.08897308)
\lineto(765.35147461,371.08897308)
\curveto(765.27146972,371.10897327)(765.1964698,371.12397326)(765.12647461,371.13397308)
\curveto(765.05646994,371.13397325)(764.98147001,371.14397324)(764.90147461,371.16397308)
\lineto(764.63147461,371.22397308)
\curveto(764.54147045,371.23397315)(764.45647054,371.25397313)(764.37647461,371.28397308)
\curveto(764.16647083,371.34397304)(763.97647102,371.41897296)(763.80647461,371.50897308)
\curveto(763.17647182,371.7789726)(762.66647233,372.16397222)(762.27647461,372.66397308)
\curveto(761.88647311,373.16397122)(761.57647342,373.75397063)(761.34647461,374.43397308)
\curveto(761.30647369,374.55396983)(761.27147372,374.6789697)(761.24147461,374.80897308)
\curveto(761.22147377,374.93896944)(761.1964738,375.07396931)(761.16647461,375.21397308)
\curveto(761.14647385,375.26396912)(761.13647386,375.30896907)(761.13647461,375.34897308)
\curveto(761.14647385,375.38896899)(761.14647385,375.43396895)(761.13647461,375.48397308)
\curveto(761.11647388,375.57396881)(761.10147389,375.66896871)(761.09147461,375.76897308)
\curveto(761.0914739,375.86896851)(761.08147391,375.96396842)(761.06147461,376.05397308)
\lineto(761.06147461,376.33897308)
\curveto(761.04147395,376.38896799)(761.03147396,376.47396791)(761.03147461,376.59397308)
\curveto(761.03147396,376.71396767)(761.04147395,376.79896758)(761.06147461,376.84897308)
\curveto(761.07147392,376.8789675)(761.07147392,376.90896747)(761.06147461,376.93897308)
\curveto(761.05147394,376.9789674)(761.05147394,377.00896737)(761.06147461,377.02897308)
\lineto(761.06147461,377.16397308)
\curveto(761.07147392,377.24396714)(761.07647392,377.32396706)(761.07647461,377.40397308)
\curveto(761.08647391,377.49396689)(761.10147389,377.5789668)(761.12147461,377.65897308)
\curveto(761.14147385,377.71896666)(761.15147384,377.7789666)(761.15147461,377.83897308)
\curveto(761.15147384,377.90896647)(761.16147383,377.9789664)(761.18147461,378.04897308)
\curveto(761.23147376,378.21896616)(761.27147372,378.383966)(761.30147461,378.54397308)
\curveto(761.33147366,378.70396568)(761.37647362,378.85396553)(761.43647461,378.99397308)
\lineto(761.58647461,379.38397308)
\curveto(761.64647335,379.52396486)(761.71147328,379.64896473)(761.78147461,379.75897308)
\curveto(761.93147306,380.01896436)(762.08147291,380.25396413)(762.23147461,380.46397308)
\curveto(762.26147273,380.51396387)(762.2964727,380.55396383)(762.33647461,380.58397308)
\curveto(762.38647261,380.62396376)(762.42647257,380.66896371)(762.45647461,380.71897308)
\curveto(762.51647248,380.79896358)(762.57647242,380.86896351)(762.63647461,380.92897308)
\lineto(762.84647461,381.10897308)
\curveto(762.90647209,381.15896322)(762.96147203,381.20396318)(763.01147461,381.24397308)
\curveto(763.07147192,381.29396309)(763.13647186,381.34396304)(763.20647461,381.39397308)
\curveto(763.35647164,381.50396288)(763.51147148,381.59896278)(763.67147461,381.67897308)
\curveto(763.84147115,381.75896262)(764.01647098,381.83896254)(764.19647461,381.91897308)
\curveto(764.30647069,381.96896241)(764.42147057,382.00396238)(764.54147461,382.02397308)
\curveto(764.67147032,382.05396233)(764.7964702,382.08896229)(764.91647461,382.12897308)
\curveto(764.98647001,382.13896224)(765.05146994,382.14896223)(765.11147461,382.15897308)
\lineto(765.29147461,382.18897308)
\curveto(765.37146962,382.19896218)(765.44646955,382.20396218)(765.51647461,382.20397308)
\curveto(765.5964694,382.21396217)(765.67646932,382.22396216)(765.75647461,382.23397308)
\curveto(765.77646922,382.24396214)(765.80146919,382.24396214)(765.83147461,382.23397308)
\curveto(765.86146913,382.22396216)(765.88646911,382.22396216)(765.90647461,382.23397308)
}
}
{
\newrgbcolor{curcolor}{0 0 0}
\pscustom[linestyle=none,fillstyle=solid,fillcolor=curcolor]
{
\newpath
\moveto(779.02631836,371.86897308)
\curveto(779.05631053,371.70897267)(779.04131054,371.57397281)(778.98131836,371.46397308)
\curveto(778.92131066,371.36397302)(778.84131074,371.28897309)(778.74131836,371.23897308)
\curveto(778.69131089,371.21897316)(778.63631095,371.20897317)(778.57631836,371.20897308)
\curveto(778.52631106,371.20897317)(778.47131111,371.19897318)(778.41131836,371.17897308)
\curveto(778.19131139,371.12897325)(777.97131161,371.14397324)(777.75131836,371.22397308)
\curveto(777.54131204,371.29397309)(777.39631219,371.383973)(777.31631836,371.49397308)
\curveto(777.26631232,371.56397282)(777.22131236,371.64397274)(777.18131836,371.73397308)
\curveto(777.14131244,371.83397255)(777.09131249,371.91397247)(777.03131836,371.97397308)
\curveto(777.01131257,371.99397239)(776.9863126,372.01397237)(776.95631836,372.03397308)
\curveto(776.93631265,372.05397233)(776.90631268,372.05897232)(776.86631836,372.04897308)
\curveto(776.75631283,372.01897236)(776.65131293,371.96397242)(776.55131836,371.88397308)
\curveto(776.46131312,371.80397258)(776.37131321,371.73397265)(776.28131836,371.67397308)
\curveto(776.15131343,371.59397279)(776.01131357,371.51897286)(775.86131836,371.44897308)
\curveto(775.71131387,371.38897299)(775.55131403,371.33397305)(775.38131836,371.28397308)
\curveto(775.2813143,371.25397313)(775.17131441,371.23397315)(775.05131836,371.22397308)
\curveto(774.94131464,371.21397317)(774.83131475,371.19897318)(774.72131836,371.17897308)
\curveto(774.67131491,371.16897321)(774.62631496,371.16397322)(774.58631836,371.16397308)
\lineto(774.48131836,371.16397308)
\curveto(774.37131521,371.14397324)(774.26631532,371.14397324)(774.16631836,371.16397308)
\lineto(774.03131836,371.16397308)
\curveto(773.9813156,371.17397321)(773.93131565,371.1789732)(773.88131836,371.17897308)
\curveto(773.83131575,371.1789732)(773.7863158,371.18897319)(773.74631836,371.20897308)
\curveto(773.70631588,371.21897316)(773.67131591,371.22397316)(773.64131836,371.22397308)
\curveto(773.62131596,371.21397317)(773.59631599,371.21397317)(773.56631836,371.22397308)
\lineto(773.32631836,371.28397308)
\curveto(773.24631634,371.29397309)(773.17131641,371.31397307)(773.10131836,371.34397308)
\curveto(772.80131678,371.47397291)(772.55631703,371.61897276)(772.36631836,371.77897308)
\curveto(772.1863174,371.94897243)(772.03631755,372.1839722)(771.91631836,372.48397308)
\curveto(771.82631776,372.70397168)(771.7813178,372.96897141)(771.78131836,373.27897308)
\lineto(771.78131836,373.59397308)
\curveto(771.79131779,373.64397074)(771.79631779,373.69397069)(771.79631836,373.74397308)
\lineto(771.82631836,373.92397308)
\lineto(771.94631836,374.25397308)
\curveto(771.9863176,374.36397002)(772.03631755,374.46396992)(772.09631836,374.55397308)
\curveto(772.27631731,374.84396954)(772.52131706,375.05896932)(772.83131836,375.19897308)
\curveto(773.14131644,375.33896904)(773.4813161,375.46396892)(773.85131836,375.57397308)
\curveto(773.99131559,375.61396877)(774.13631545,375.64396874)(774.28631836,375.66397308)
\curveto(774.43631515,375.6839687)(774.586315,375.70896867)(774.73631836,375.73897308)
\curveto(774.80631478,375.75896862)(774.87131471,375.76896861)(774.93131836,375.76897308)
\curveto(775.00131458,375.76896861)(775.07631451,375.7789686)(775.15631836,375.79897308)
\curveto(775.22631436,375.81896856)(775.29631429,375.82896855)(775.36631836,375.82897308)
\curveto(775.43631415,375.83896854)(775.51131407,375.85396853)(775.59131836,375.87397308)
\curveto(775.84131374,375.93396845)(776.07631351,375.9839684)(776.29631836,376.02397308)
\curveto(776.51631307,376.07396831)(776.69131289,376.18896819)(776.82131836,376.36897308)
\curveto(776.8813127,376.44896793)(776.93131265,376.54896783)(776.97131836,376.66897308)
\curveto(777.01131257,376.79896758)(777.01131257,376.93896744)(776.97131836,377.08897308)
\curveto(776.91131267,377.32896705)(776.82131276,377.51896686)(776.70131836,377.65897308)
\curveto(776.59131299,377.79896658)(776.43131315,377.90896647)(776.22131836,377.98897308)
\curveto(776.10131348,378.03896634)(775.95631363,378.07396631)(775.78631836,378.09397308)
\curveto(775.62631396,378.11396627)(775.45631413,378.12396626)(775.27631836,378.12397308)
\curveto(775.09631449,378.12396626)(774.92131466,378.11396627)(774.75131836,378.09397308)
\curveto(774.581315,378.07396631)(774.43631515,378.04396634)(774.31631836,378.00397308)
\curveto(774.14631544,377.94396644)(773.9813156,377.85896652)(773.82131836,377.74897308)
\curveto(773.74131584,377.68896669)(773.66631592,377.60896677)(773.59631836,377.50897308)
\curveto(773.53631605,377.41896696)(773.4813161,377.31896706)(773.43131836,377.20897308)
\curveto(773.40131618,377.12896725)(773.37131621,377.04396734)(773.34131836,376.95397308)
\curveto(773.32131626,376.86396752)(773.27631631,376.79396759)(773.20631836,376.74397308)
\curveto(773.16631642,376.71396767)(773.09631649,376.68896769)(772.99631836,376.66897308)
\curveto(772.90631668,376.65896772)(772.81131677,376.65396773)(772.71131836,376.65397308)
\curveto(772.61131697,376.65396773)(772.51131707,376.65896772)(772.41131836,376.66897308)
\curveto(772.32131726,376.68896769)(772.25631733,376.71396767)(772.21631836,376.74397308)
\curveto(772.17631741,376.77396761)(772.14631744,376.82396756)(772.12631836,376.89397308)
\curveto(772.10631748,376.96396742)(772.10631748,377.03896734)(772.12631836,377.11897308)
\curveto(772.15631743,377.24896713)(772.1863174,377.36896701)(772.21631836,377.47897308)
\curveto(772.25631733,377.59896678)(772.30131728,377.71396667)(772.35131836,377.82397308)
\curveto(772.54131704,378.17396621)(772.7813168,378.44396594)(773.07131836,378.63397308)
\curveto(773.36131622,378.83396555)(773.72131586,378.99396539)(774.15131836,379.11397308)
\curveto(774.25131533,379.13396525)(774.35131523,379.14896523)(774.45131836,379.15897308)
\curveto(774.56131502,379.16896521)(774.67131491,379.1839652)(774.78131836,379.20397308)
\curveto(774.82131476,379.21396517)(774.8863147,379.21396517)(774.97631836,379.20397308)
\curveto(775.06631452,379.20396518)(775.12131446,379.21396517)(775.14131836,379.23397308)
\curveto(775.84131374,379.24396514)(776.45131313,379.16396522)(776.97131836,378.99397308)
\curveto(777.49131209,378.82396556)(777.85631173,378.49896588)(778.06631836,378.01897308)
\curveto(778.15631143,377.81896656)(778.20631138,377.5839668)(778.21631836,377.31397308)
\curveto(778.23631135,377.05396733)(778.24631134,376.7789676)(778.24631836,376.48897308)
\lineto(778.24631836,373.17397308)
\curveto(778.24631134,373.03397135)(778.25131133,372.89897148)(778.26131836,372.76897308)
\curveto(778.27131131,372.63897174)(778.30131128,372.53397185)(778.35131836,372.45397308)
\curveto(778.40131118,372.383972)(778.46631112,372.33397205)(778.54631836,372.30397308)
\curveto(778.63631095,372.26397212)(778.72131086,372.23397215)(778.80131836,372.21397308)
\curveto(778.8813107,372.20397218)(778.94131064,372.15897222)(778.98131836,372.07897308)
\curveto(779.00131058,372.04897233)(779.01131057,372.01897236)(779.01131836,371.98897308)
\curveto(779.01131057,371.95897242)(779.01631057,371.91897246)(779.02631836,371.86897308)
\moveto(776.88131836,373.53397308)
\curveto(776.94131264,373.67397071)(776.97131261,373.83397055)(776.97131836,374.01397308)
\curveto(776.9813126,374.20397018)(776.9863126,374.39896998)(776.98631836,374.59897308)
\curveto(776.9863126,374.70896967)(776.9813126,374.80896957)(776.97131836,374.89897308)
\curveto(776.96131262,374.98896939)(776.92131266,375.05896932)(776.85131836,375.10897308)
\curveto(776.82131276,375.12896925)(776.75131283,375.13896924)(776.64131836,375.13897308)
\curveto(776.62131296,375.11896926)(776.586313,375.10896927)(776.53631836,375.10897308)
\curveto(776.4863131,375.10896927)(776.44131314,375.09896928)(776.40131836,375.07897308)
\curveto(776.32131326,375.05896932)(776.23131335,375.03896934)(776.13131836,375.01897308)
\lineto(775.83131836,374.95897308)
\curveto(775.80131378,374.95896942)(775.76631382,374.95396943)(775.72631836,374.94397308)
\lineto(775.62131836,374.94397308)
\curveto(775.47131411,374.90396948)(775.30631428,374.8789695)(775.12631836,374.86897308)
\curveto(774.95631463,374.86896951)(774.79631479,374.84896953)(774.64631836,374.80897308)
\curveto(774.56631502,374.78896959)(774.49131509,374.76896961)(774.42131836,374.74897308)
\curveto(774.36131522,374.73896964)(774.29131529,374.72396966)(774.21131836,374.70397308)
\curveto(774.05131553,374.65396973)(773.90131568,374.58896979)(773.76131836,374.50897308)
\curveto(773.62131596,374.43896994)(773.50131608,374.34897003)(773.40131836,374.23897308)
\curveto(773.30131628,374.12897025)(773.22631636,373.99397039)(773.17631836,373.83397308)
\curveto(773.12631646,373.6839707)(773.10631648,373.49897088)(773.11631836,373.27897308)
\curveto(773.11631647,373.1789712)(773.13131645,373.0839713)(773.16131836,372.99397308)
\curveto(773.20131638,372.91397147)(773.24631634,372.83897154)(773.29631836,372.76897308)
\curveto(773.37631621,372.65897172)(773.4813161,372.56397182)(773.61131836,372.48397308)
\curveto(773.74131584,372.41397197)(773.8813157,372.35397203)(774.03131836,372.30397308)
\curveto(774.0813155,372.29397209)(774.13131545,372.28897209)(774.18131836,372.28897308)
\curveto(774.23131535,372.28897209)(774.2813153,372.2839721)(774.33131836,372.27397308)
\curveto(774.40131518,372.25397213)(774.4863151,372.23897214)(774.58631836,372.22897308)
\curveto(774.69631489,372.22897215)(774.7863148,372.23897214)(774.85631836,372.25897308)
\curveto(774.91631467,372.2789721)(774.97631461,372.2839721)(775.03631836,372.27397308)
\curveto(775.09631449,372.27397211)(775.15631443,372.2839721)(775.21631836,372.30397308)
\curveto(775.29631429,372.32397206)(775.37131421,372.33897204)(775.44131836,372.34897308)
\curveto(775.52131406,372.35897202)(775.59631399,372.378972)(775.66631836,372.40897308)
\curveto(775.95631363,372.52897185)(776.20131338,372.67397171)(776.40131836,372.84397308)
\curveto(776.61131297,373.01397137)(776.77131281,373.24397114)(776.88131836,373.53397308)
}
}
{
\newrgbcolor{curcolor}{0 0 0}
\pscustom[linestyle=none,fillstyle=solid,fillcolor=curcolor]
{
\newpath
\moveto(783.88795898,379.18897308)
\curveto(784.51795375,379.20896517)(785.02295324,379.12396526)(785.40295898,378.93397308)
\curveto(785.78295248,378.74396564)(786.08795218,378.45896592)(786.31795898,378.07897308)
\curveto(786.37795189,377.9789664)(786.42295184,377.86896651)(786.45295898,377.74897308)
\curveto(786.49295177,377.63896674)(786.52795174,377.52396686)(786.55795898,377.40397308)
\curveto(786.60795166,377.21396717)(786.63795163,377.00896737)(786.64795898,376.78897308)
\curveto(786.65795161,376.56896781)(786.6629516,376.34396804)(786.66295898,376.11397308)
\lineto(786.66295898,374.50897308)
\lineto(786.66295898,372.16897308)
\curveto(786.6629516,371.99897238)(786.65795161,371.82897255)(786.64795898,371.65897308)
\curveto(786.64795162,371.48897289)(786.58295168,371.378973)(786.45295898,371.32897308)
\curveto(786.40295186,371.30897307)(786.34795192,371.29897308)(786.28795898,371.29897308)
\curveto(786.23795203,371.28897309)(786.18295208,371.2839731)(786.12295898,371.28397308)
\curveto(785.99295227,371.2839731)(785.8679524,371.28897309)(785.74795898,371.29897308)
\curveto(785.62795264,371.29897308)(785.54295272,371.33897304)(785.49295898,371.41897308)
\curveto(785.44295282,371.48897289)(785.41795285,371.5789728)(785.41795898,371.68897308)
\lineto(785.41795898,372.01897308)
\lineto(785.41795898,373.30897308)
\lineto(785.41795898,375.75397308)
\curveto(785.41795285,376.02396836)(785.41295285,376.28896809)(785.40295898,376.54897308)
\curveto(785.39295287,376.81896756)(785.34795292,377.04896733)(785.26795898,377.23897308)
\curveto(785.18795308,377.43896694)(785.0679532,377.59896678)(784.90795898,377.71897308)
\curveto(784.74795352,377.84896653)(784.5629537,377.94896643)(784.35295898,378.01897308)
\curveto(784.29295397,378.03896634)(784.22795404,378.04896633)(784.15795898,378.04897308)
\curveto(784.09795417,378.05896632)(784.03795423,378.07396631)(783.97795898,378.09397308)
\curveto(783.92795434,378.10396628)(783.84795442,378.10396628)(783.73795898,378.09397308)
\curveto(783.63795463,378.09396629)(783.5679547,378.08896629)(783.52795898,378.07897308)
\curveto(783.48795478,378.05896632)(783.45295481,378.04896633)(783.42295898,378.04897308)
\curveto(783.39295487,378.05896632)(783.35795491,378.05896632)(783.31795898,378.04897308)
\curveto(783.18795508,378.01896636)(783.0629552,377.9839664)(782.94295898,377.94397308)
\curveto(782.83295543,377.91396647)(782.72795554,377.86896651)(782.62795898,377.80897308)
\curveto(782.58795568,377.78896659)(782.55295571,377.76896661)(782.52295898,377.74897308)
\curveto(782.49295577,377.72896665)(782.45795581,377.70896667)(782.41795898,377.68897308)
\curveto(782.0679562,377.43896694)(781.81295645,377.06396732)(781.65295898,376.56397308)
\curveto(781.62295664,376.4839679)(781.60295666,376.39896798)(781.59295898,376.30897308)
\curveto(781.58295668,376.22896815)(781.5679567,376.14896823)(781.54795898,376.06897308)
\curveto(781.52795674,376.01896836)(781.52295674,375.96896841)(781.53295898,375.91897308)
\curveto(781.54295672,375.8789685)(781.53795673,375.83896854)(781.51795898,375.79897308)
\lineto(781.51795898,375.48397308)
\curveto(781.50795676,375.45396893)(781.50295676,375.41896896)(781.50295898,375.37897308)
\curveto(781.51295675,375.33896904)(781.51795675,375.29396909)(781.51795898,375.24397308)
\lineto(781.51795898,374.79397308)
\lineto(781.51795898,373.35397308)
\lineto(781.51795898,372.03397308)
\lineto(781.51795898,371.68897308)
\curveto(781.51795675,371.5789728)(781.49295677,371.48897289)(781.44295898,371.41897308)
\curveto(781.39295687,371.33897304)(781.30295696,371.29897308)(781.17295898,371.29897308)
\curveto(781.05295721,371.28897309)(780.92795734,371.2839731)(780.79795898,371.28397308)
\curveto(780.71795755,371.2839731)(780.64295762,371.28897309)(780.57295898,371.29897308)
\curveto(780.50295776,371.30897307)(780.44295782,371.33397305)(780.39295898,371.37397308)
\curveto(780.31295795,371.42397296)(780.27295799,371.51897286)(780.27295898,371.65897308)
\lineto(780.27295898,372.06397308)
\lineto(780.27295898,373.83397308)
\lineto(780.27295898,377.46397308)
\lineto(780.27295898,378.37897308)
\lineto(780.27295898,378.64897308)
\curveto(780.27295799,378.73896564)(780.29295797,378.80896557)(780.33295898,378.85897308)
\curveto(780.3629579,378.91896546)(780.41295785,378.95896542)(780.48295898,378.97897308)
\curveto(780.52295774,378.98896539)(780.57795769,378.99896538)(780.64795898,379.00897308)
\curveto(780.72795754,379.01896536)(780.80795746,379.02396536)(780.88795898,379.02397308)
\curveto(780.9679573,379.02396536)(781.04295722,379.01896536)(781.11295898,379.00897308)
\curveto(781.19295707,378.99896538)(781.24795702,378.9839654)(781.27795898,378.96397308)
\curveto(781.38795688,378.89396549)(781.43795683,378.80396558)(781.42795898,378.69397308)
\curveto(781.41795685,378.59396579)(781.43295683,378.4789659)(781.47295898,378.34897308)
\curveto(781.49295677,378.28896609)(781.53295673,378.23896614)(781.59295898,378.19897308)
\curveto(781.71295655,378.18896619)(781.80795646,378.23396615)(781.87795898,378.33397308)
\curveto(781.95795631,378.43396595)(782.03795623,378.51396587)(782.11795898,378.57397308)
\curveto(782.25795601,378.67396571)(782.39795587,378.76396562)(782.53795898,378.84397308)
\curveto(782.68795558,378.93396545)(782.85795541,379.00896537)(783.04795898,379.06897308)
\curveto(783.12795514,379.09896528)(783.21295505,379.11896526)(783.30295898,379.12897308)
\curveto(783.40295486,379.13896524)(783.49795477,379.15396523)(783.58795898,379.17397308)
\curveto(783.63795463,379.1839652)(783.68795458,379.18896519)(783.73795898,379.18897308)
\lineto(783.88795898,379.18897308)
}
}
{
\newrgbcolor{curcolor}{0 0 0}
\pscustom[linestyle=none,fillstyle=solid,fillcolor=curcolor]
{
\newpath
\moveto(789.49256836,381.37897308)
\curveto(789.64256635,381.378963)(789.7925662,381.37396301)(789.94256836,381.36397308)
\curveto(790.0925659,381.36396302)(790.19756579,381.32396306)(790.25756836,381.24397308)
\curveto(790.30756568,381.1839632)(790.33256566,381.09896328)(790.33256836,380.98897308)
\curveto(790.34256565,380.88896349)(790.34756564,380.7839636)(790.34756836,380.67397308)
\lineto(790.34756836,379.80397308)
\curveto(790.34756564,379.72396466)(790.34256565,379.63896474)(790.33256836,379.54897308)
\curveto(790.33256566,379.46896491)(790.34256565,379.39896498)(790.36256836,379.33897308)
\curveto(790.40256559,379.19896518)(790.4925655,379.10896527)(790.63256836,379.06897308)
\curveto(790.68256531,379.05896532)(790.72756526,379.05396533)(790.76756836,379.05397308)
\lineto(790.91756836,379.05397308)
\lineto(791.32256836,379.05397308)
\curveto(791.48256451,379.06396532)(791.59756439,379.05396533)(791.66756836,379.02397308)
\curveto(791.75756423,378.96396542)(791.81756417,378.90396548)(791.84756836,378.84397308)
\curveto(791.86756412,378.80396558)(791.87756411,378.75896562)(791.87756836,378.70897308)
\lineto(791.87756836,378.55897308)
\curveto(791.87756411,378.44896593)(791.87256412,378.34396604)(791.86256836,378.24397308)
\curveto(791.85256414,378.15396623)(791.81756417,378.0839663)(791.75756836,378.03397308)
\curveto(791.69756429,377.9839664)(791.61256438,377.95396643)(791.50256836,377.94397308)
\lineto(791.17256836,377.94397308)
\curveto(791.06256493,377.95396643)(790.95256504,377.95896642)(790.84256836,377.95897308)
\curveto(790.73256526,377.95896642)(790.63756535,377.94396644)(790.55756836,377.91397308)
\curveto(790.4875655,377.8839665)(790.43756555,377.83396655)(790.40756836,377.76397308)
\curveto(790.37756561,377.69396669)(790.35756563,377.60896677)(790.34756836,377.50897308)
\curveto(790.33756565,377.41896696)(790.33256566,377.31896706)(790.33256836,377.20897308)
\curveto(790.34256565,377.10896727)(790.34756564,377.00896737)(790.34756836,376.90897308)
\lineto(790.34756836,373.93897308)
\curveto(790.34756564,373.71897066)(790.34256565,373.4839709)(790.33256836,373.23397308)
\curveto(790.33256566,372.99397139)(790.37756561,372.80897157)(790.46756836,372.67897308)
\curveto(790.51756547,372.59897178)(790.58256541,372.54397184)(790.66256836,372.51397308)
\curveto(790.74256525,372.4839719)(790.83756515,372.45897192)(790.94756836,372.43897308)
\curveto(790.97756501,372.42897195)(791.00756498,372.42397196)(791.03756836,372.42397308)
\curveto(791.07756491,372.43397195)(791.11256488,372.43397195)(791.14256836,372.42397308)
\lineto(791.33756836,372.42397308)
\curveto(791.43756455,372.42397196)(791.52756446,372.41397197)(791.60756836,372.39397308)
\curveto(791.69756429,372.383972)(791.76256423,372.34897203)(791.80256836,372.28897308)
\curveto(791.82256417,372.25897212)(791.83756415,372.20397218)(791.84756836,372.12397308)
\curveto(791.86756412,372.05397233)(791.87756411,371.9789724)(791.87756836,371.89897308)
\curveto(791.8875641,371.81897256)(791.8875641,371.73897264)(791.87756836,371.65897308)
\curveto(791.86756412,371.58897279)(791.84756414,371.53397285)(791.81756836,371.49397308)
\curveto(791.77756421,371.42397296)(791.70256429,371.37397301)(791.59256836,371.34397308)
\curveto(791.51256448,371.32397306)(791.42256457,371.31397307)(791.32256836,371.31397308)
\curveto(791.22256477,371.32397306)(791.13256486,371.32897305)(791.05256836,371.32897308)
\curveto(790.992565,371.32897305)(790.93256506,371.32397306)(790.87256836,371.31397308)
\curveto(790.81256518,371.31397307)(790.75756523,371.31897306)(790.70756836,371.32897308)
\lineto(790.52756836,371.32897308)
\curveto(790.47756551,371.33897304)(790.42756556,371.34397304)(790.37756836,371.34397308)
\curveto(790.33756565,371.35397303)(790.2925657,371.35897302)(790.24256836,371.35897308)
\curveto(790.04256595,371.40897297)(789.86756612,371.46397292)(789.71756836,371.52397308)
\curveto(789.57756641,371.5839728)(789.45756653,371.68897269)(789.35756836,371.83897308)
\curveto(789.21756677,372.03897234)(789.13756685,372.28897209)(789.11756836,372.58897308)
\curveto(789.09756689,372.89897148)(789.0875669,373.22897115)(789.08756836,373.57897308)
\lineto(789.08756836,377.50897308)
\curveto(789.05756693,377.63896674)(789.02756696,377.73396665)(788.99756836,377.79397308)
\curveto(788.97756701,377.85396653)(788.90756708,377.90396648)(788.78756836,377.94397308)
\curveto(788.74756724,377.95396643)(788.70756728,377.95396643)(788.66756836,377.94397308)
\curveto(788.62756736,377.93396645)(788.5875674,377.93896644)(788.54756836,377.95897308)
\lineto(788.30756836,377.95897308)
\curveto(788.17756781,377.95896642)(788.06756792,377.96896641)(787.97756836,377.98897308)
\curveto(787.89756809,378.01896636)(787.84256815,378.0789663)(787.81256836,378.16897308)
\curveto(787.7925682,378.20896617)(787.77756821,378.25396613)(787.76756836,378.30397308)
\lineto(787.76756836,378.45397308)
\curveto(787.76756822,378.59396579)(787.77756821,378.70896567)(787.79756836,378.79897308)
\curveto(787.81756817,378.89896548)(787.87756811,378.97396541)(787.97756836,379.02397308)
\curveto(788.0875679,379.06396532)(788.22756776,379.07396531)(788.39756836,379.05397308)
\curveto(788.57756741,379.03396535)(788.72756726,379.04396534)(788.84756836,379.08397308)
\curveto(788.93756705,379.13396525)(789.00756698,379.20396518)(789.05756836,379.29397308)
\curveto(789.07756691,379.35396503)(789.0875669,379.42896495)(789.08756836,379.51897308)
\lineto(789.08756836,379.77397308)
\lineto(789.08756836,380.70397308)
\lineto(789.08756836,380.94397308)
\curveto(789.0875669,381.03396335)(789.09756689,381.10896327)(789.11756836,381.16897308)
\curveto(789.15756683,381.24896313)(789.23256676,381.31396307)(789.34256836,381.36397308)
\curveto(789.37256662,381.36396302)(789.39756659,381.36396302)(789.41756836,381.36397308)
\curveto(789.44756654,381.37396301)(789.47256652,381.378963)(789.49256836,381.37897308)
}
}
{
\newrgbcolor{curcolor}{0 0 0}
\pscustom[linestyle=none,fillstyle=solid,fillcolor=curcolor]
{
\newpath
\moveto(793.54936523,380.53897308)
\curveto(793.46936411,380.59896378)(793.42436416,380.70396368)(793.41436523,380.85397308)
\lineto(793.41436523,381.31897308)
\lineto(793.41436523,381.57397308)
\curveto(793.41436417,381.66396272)(793.42936415,381.73896264)(793.45936523,381.79897308)
\curveto(793.49936408,381.8789625)(793.579364,381.93896244)(793.69936523,381.97897308)
\curveto(793.71936386,381.98896239)(793.73936384,381.98896239)(793.75936523,381.97897308)
\curveto(793.78936379,381.9789624)(793.81436377,381.9839624)(793.83436523,381.99397308)
\curveto(794.00436358,381.99396239)(794.16436342,381.98896239)(794.31436523,381.97897308)
\curveto(794.46436312,381.96896241)(794.56436302,381.90896247)(794.61436523,381.79897308)
\curveto(794.64436294,381.73896264)(794.65936292,381.66396272)(794.65936523,381.57397308)
\lineto(794.65936523,381.31897308)
\curveto(794.65936292,381.13896324)(794.65436293,380.96896341)(794.64436523,380.80897308)
\curveto(794.64436294,380.64896373)(794.579363,380.54396384)(794.44936523,380.49397308)
\curveto(794.39936318,380.47396391)(794.34436324,380.46396392)(794.28436523,380.46397308)
\lineto(794.11936523,380.46397308)
\lineto(793.80436523,380.46397308)
\curveto(793.70436388,380.46396392)(793.61936396,380.48896389)(793.54936523,380.53897308)
\moveto(794.65936523,372.03397308)
\lineto(794.65936523,371.71897308)
\curveto(794.66936291,371.61897276)(794.64936293,371.53897284)(794.59936523,371.47897308)
\curveto(794.56936301,371.41897296)(794.52436306,371.378973)(794.46436523,371.35897308)
\curveto(794.40436318,371.34897303)(794.33436325,371.33397305)(794.25436523,371.31397308)
\lineto(794.02936523,371.31397308)
\curveto(793.89936368,371.31397307)(793.7843638,371.31897306)(793.68436523,371.32897308)
\curveto(793.59436399,371.34897303)(793.52436406,371.39897298)(793.47436523,371.47897308)
\curveto(793.43436415,371.53897284)(793.41436417,371.61397277)(793.41436523,371.70397308)
\lineto(793.41436523,371.98897308)
\lineto(793.41436523,378.33397308)
\lineto(793.41436523,378.64897308)
\curveto(793.41436417,378.75896562)(793.43936414,378.84396554)(793.48936523,378.90397308)
\curveto(793.51936406,378.95396543)(793.55936402,378.9839654)(793.60936523,378.99397308)
\curveto(793.65936392,379.00396538)(793.71436387,379.01896536)(793.77436523,379.03897308)
\curveto(793.79436379,379.03896534)(793.81436377,379.03396535)(793.83436523,379.02397308)
\curveto(793.86436372,379.02396536)(793.88936369,379.02896535)(793.90936523,379.03897308)
\curveto(794.03936354,379.03896534)(794.16936341,379.03396535)(794.29936523,379.02397308)
\curveto(794.43936314,379.02396536)(794.53436305,378.9839654)(794.58436523,378.90397308)
\curveto(794.63436295,378.84396554)(794.65936292,378.76396562)(794.65936523,378.66397308)
\lineto(794.65936523,378.37897308)
\lineto(794.65936523,372.03397308)
}
}
{
\newrgbcolor{curcolor}{0 0 0}
\pscustom[linestyle=none,fillstyle=solid,fillcolor=curcolor]
{
\newpath
\moveto(803.56420898,372.12397308)
\lineto(803.56420898,371.73397308)
\curveto(803.56420111,371.61397277)(803.53920113,371.51397287)(803.48920898,371.43397308)
\curveto(803.43920123,371.36397302)(803.35420132,371.32397306)(803.23420898,371.31397308)
\lineto(802.88920898,371.31397308)
\curveto(802.82920184,371.31397307)(802.7692019,371.30897307)(802.70920898,371.29897308)
\curveto(802.65920201,371.29897308)(802.61420206,371.30897307)(802.57420898,371.32897308)
\curveto(802.48420219,371.34897303)(802.42420225,371.38897299)(802.39420898,371.44897308)
\curveto(802.35420232,371.49897288)(802.32920234,371.55897282)(802.31920898,371.62897308)
\curveto(802.31920235,371.69897268)(802.30420237,371.76897261)(802.27420898,371.83897308)
\curveto(802.26420241,371.85897252)(802.24920242,371.87397251)(802.22920898,371.88397308)
\curveto(802.21920245,371.90397248)(802.20420247,371.92397246)(802.18420898,371.94397308)
\curveto(802.08420259,371.95397243)(802.00420267,371.93397245)(801.94420898,371.88397308)
\curveto(801.89420278,371.83397255)(801.83920283,371.7839726)(801.77920898,371.73397308)
\curveto(801.57920309,371.5839728)(801.37920329,371.46897291)(801.17920898,371.38897308)
\curveto(800.99920367,371.30897307)(800.78920388,371.24897313)(800.54920898,371.20897308)
\curveto(800.31920435,371.16897321)(800.07920459,371.14897323)(799.82920898,371.14897308)
\curveto(799.58920508,371.13897324)(799.34920532,371.15397323)(799.10920898,371.19397308)
\curveto(798.8692058,371.22397316)(798.65920601,371.2789731)(798.47920898,371.35897308)
\curveto(797.95920671,371.5789728)(797.53920713,371.87397251)(797.21920898,372.24397308)
\curveto(796.89920777,372.62397176)(796.64920802,373.09397129)(796.46920898,373.65397308)
\curveto(796.42920824,373.74397064)(796.39920827,373.83397055)(796.37920898,373.92397308)
\curveto(796.3692083,374.02397036)(796.34920832,374.12397026)(796.31920898,374.22397308)
\curveto(796.30920836,374.27397011)(796.30420837,374.32397006)(796.30420898,374.37397308)
\curveto(796.30420837,374.42396996)(796.29920837,374.47396991)(796.28920898,374.52397308)
\curveto(796.2692084,374.57396981)(796.25920841,374.62396976)(796.25920898,374.67397308)
\curveto(796.2692084,374.73396965)(796.2692084,374.78896959)(796.25920898,374.83897308)
\lineto(796.25920898,374.98897308)
\curveto(796.23920843,375.03896934)(796.22920844,375.10396928)(796.22920898,375.18397308)
\curveto(796.22920844,375.26396912)(796.23920843,375.32896905)(796.25920898,375.37897308)
\lineto(796.25920898,375.54397308)
\curveto(796.27920839,375.61396877)(796.28420839,375.6839687)(796.27420898,375.75397308)
\curveto(796.2742084,375.83396855)(796.28420839,375.90896847)(796.30420898,375.97897308)
\curveto(796.31420836,376.02896835)(796.31920835,376.07396831)(796.31920898,376.11397308)
\curveto(796.31920835,376.15396823)(796.32420835,376.19896818)(796.33420898,376.24897308)
\curveto(796.36420831,376.34896803)(796.38920828,376.44396794)(796.40920898,376.53397308)
\curveto(796.42920824,376.63396775)(796.45420822,376.72896765)(796.48420898,376.81897308)
\curveto(796.61420806,377.19896718)(796.77920789,377.53896684)(796.97920898,377.83897308)
\curveto(797.18920748,378.14896623)(797.43920723,378.40396598)(797.72920898,378.60397308)
\curveto(797.89920677,378.72396566)(798.0742066,378.82396556)(798.25420898,378.90397308)
\curveto(798.44420623,378.9839654)(798.64920602,379.05396533)(798.86920898,379.11397308)
\curveto(798.93920573,379.12396526)(799.00420567,379.13396525)(799.06420898,379.14397308)
\curveto(799.13420554,379.15396523)(799.20420547,379.16896521)(799.27420898,379.18897308)
\lineto(799.42420898,379.18897308)
\curveto(799.50420517,379.20896517)(799.61920505,379.21896516)(799.76920898,379.21897308)
\curveto(799.92920474,379.21896516)(800.04920462,379.20896517)(800.12920898,379.18897308)
\curveto(800.1692045,379.1789652)(800.22420445,379.17396521)(800.29420898,379.17397308)
\curveto(800.40420427,379.14396524)(800.51420416,379.11896526)(800.62420898,379.09897308)
\curveto(800.73420394,379.08896529)(800.83920383,379.05896532)(800.93920898,379.00897308)
\curveto(801.08920358,378.94896543)(801.22920344,378.8839655)(801.35920898,378.81397308)
\curveto(801.49920317,378.74396564)(801.62920304,378.66396572)(801.74920898,378.57397308)
\curveto(801.80920286,378.52396586)(801.8692028,378.46896591)(801.92920898,378.40897308)
\curveto(801.99920267,378.35896602)(802.08920258,378.34396604)(802.19920898,378.36397308)
\curveto(802.21920245,378.39396599)(802.23420244,378.41896596)(802.24420898,378.43897308)
\curveto(802.26420241,378.45896592)(802.27920239,378.48896589)(802.28920898,378.52897308)
\curveto(802.31920235,378.61896576)(802.32920234,378.73396565)(802.31920898,378.87397308)
\lineto(802.31920898,379.24897308)
\lineto(802.31920898,380.97397308)
\lineto(802.31920898,381.43897308)
\curveto(802.31920235,381.61896276)(802.34420233,381.74896263)(802.39420898,381.82897308)
\curveto(802.43420224,381.89896248)(802.49420218,381.94396244)(802.57420898,381.96397308)
\curveto(802.59420208,381.96396242)(802.61920205,381.96396242)(802.64920898,381.96397308)
\curveto(802.67920199,381.97396241)(802.70420197,381.9789624)(802.72420898,381.97897308)
\curveto(802.86420181,381.98896239)(803.00920166,381.98896239)(803.15920898,381.97897308)
\curveto(803.31920135,381.9789624)(803.42920124,381.93896244)(803.48920898,381.85897308)
\curveto(803.53920113,381.7789626)(803.56420111,381.6789627)(803.56420898,381.55897308)
\lineto(803.56420898,381.18397308)
\lineto(803.56420898,372.12397308)
\moveto(802.34920898,374.95897308)
\curveto(802.3692023,375.00896937)(802.37920229,375.07396931)(802.37920898,375.15397308)
\curveto(802.37920229,375.24396914)(802.3692023,375.31396907)(802.34920898,375.36397308)
\lineto(802.34920898,375.58897308)
\curveto(802.32920234,375.6789687)(802.31420236,375.76896861)(802.30420898,375.85897308)
\curveto(802.29420238,375.95896842)(802.2742024,376.04896833)(802.24420898,376.12897308)
\curveto(802.22420245,376.20896817)(802.20420247,376.2839681)(802.18420898,376.35397308)
\curveto(802.1742025,376.42396796)(802.15420252,376.49396789)(802.12420898,376.56397308)
\curveto(802.00420267,376.86396752)(801.84920282,377.12896725)(801.65920898,377.35897308)
\curveto(801.4692032,377.58896679)(801.22920344,377.76896661)(800.93920898,377.89897308)
\curveto(800.83920383,377.94896643)(800.73420394,377.9839664)(800.62420898,378.00397308)
\curveto(800.52420415,378.03396635)(800.41420426,378.05896632)(800.29420898,378.07897308)
\curveto(800.21420446,378.09896628)(800.12420455,378.10896627)(800.02420898,378.10897308)
\lineto(799.75420898,378.10897308)
\curveto(799.70420497,378.09896628)(799.65920501,378.08896629)(799.61920898,378.07897308)
\lineto(799.48420898,378.07897308)
\curveto(799.40420527,378.05896632)(799.31920535,378.03896634)(799.22920898,378.01897308)
\curveto(799.14920552,377.99896638)(799.0692056,377.97396641)(798.98920898,377.94397308)
\curveto(798.669206,377.80396658)(798.40920626,377.59896678)(798.20920898,377.32897308)
\curveto(798.01920665,377.06896731)(797.86420681,376.76396762)(797.74420898,376.41397308)
\curveto(797.70420697,376.30396808)(797.674207,376.18896819)(797.65420898,376.06897308)
\curveto(797.64420703,375.95896842)(797.62920704,375.84896853)(797.60920898,375.73897308)
\curveto(797.60920706,375.69896868)(797.60420707,375.65896872)(797.59420898,375.61897308)
\lineto(797.59420898,375.51397308)
\curveto(797.5742071,375.46396892)(797.56420711,375.40896897)(797.56420898,375.34897308)
\curveto(797.5742071,375.28896909)(797.57920709,375.23396915)(797.57920898,375.18397308)
\lineto(797.57920898,374.85397308)
\curveto(797.57920709,374.75396963)(797.58920708,374.65896972)(797.60920898,374.56897308)
\curveto(797.61920705,374.53896984)(797.62420705,374.48896989)(797.62420898,374.41897308)
\curveto(797.64420703,374.34897003)(797.65920701,374.2789701)(797.66920898,374.20897308)
\lineto(797.72920898,373.99897308)
\curveto(797.83920683,373.64897073)(797.98920668,373.34897103)(798.17920898,373.09897308)
\curveto(798.3692063,372.84897153)(798.60920606,372.64397174)(798.89920898,372.48397308)
\curveto(798.98920568,372.43397195)(799.07920559,372.39397199)(799.16920898,372.36397308)
\curveto(799.25920541,372.33397205)(799.35920531,372.30397208)(799.46920898,372.27397308)
\curveto(799.51920515,372.25397213)(799.5692051,372.24897213)(799.61920898,372.25897308)
\curveto(799.67920499,372.26897211)(799.73420494,372.26397212)(799.78420898,372.24397308)
\curveto(799.82420485,372.23397215)(799.86420481,372.22897215)(799.90420898,372.22897308)
\lineto(800.03920898,372.22897308)
\lineto(800.17420898,372.22897308)
\curveto(800.20420447,372.23897214)(800.25420442,372.24397214)(800.32420898,372.24397308)
\curveto(800.40420427,372.26397212)(800.48420419,372.2789721)(800.56420898,372.28897308)
\curveto(800.64420403,372.30897207)(800.71920395,372.33397205)(800.78920898,372.36397308)
\curveto(801.11920355,372.50397188)(801.38420329,372.6789717)(801.58420898,372.88897308)
\curveto(801.79420288,373.10897127)(801.9692027,373.383971)(802.10920898,373.71397308)
\curveto(802.15920251,373.82397056)(802.19420248,373.93397045)(802.21420898,374.04397308)
\curveto(802.23420244,374.15397023)(802.25920241,374.26397012)(802.28920898,374.37397308)
\curveto(802.30920236,374.41396997)(802.31920235,374.44896993)(802.31920898,374.47897308)
\curveto(802.31920235,374.51896986)(802.32420235,374.55896982)(802.33420898,374.59897308)
\curveto(802.34420233,374.65896972)(802.34420233,374.71896966)(802.33420898,374.77897308)
\curveto(802.33420234,374.83896954)(802.33920233,374.89896948)(802.34920898,374.95897308)
}
}
{
\newrgbcolor{curcolor}{0 0 0}
\pscustom[linestyle=none,fillstyle=solid,fillcolor=curcolor]
{
\newpath
\moveto(812.39545898,371.86897308)
\curveto(812.42545115,371.70897267)(812.41045117,371.57397281)(812.35045898,371.46397308)
\curveto(812.29045129,371.36397302)(812.21045137,371.28897309)(812.11045898,371.23897308)
\curveto(812.06045152,371.21897316)(812.00545157,371.20897317)(811.94545898,371.20897308)
\curveto(811.89545168,371.20897317)(811.84045174,371.19897318)(811.78045898,371.17897308)
\curveto(811.56045202,371.12897325)(811.34045224,371.14397324)(811.12045898,371.22397308)
\curveto(810.91045267,371.29397309)(810.76545281,371.383973)(810.68545898,371.49397308)
\curveto(810.63545294,371.56397282)(810.59045299,371.64397274)(810.55045898,371.73397308)
\curveto(810.51045307,371.83397255)(810.46045312,371.91397247)(810.40045898,371.97397308)
\curveto(810.3804532,371.99397239)(810.35545322,372.01397237)(810.32545898,372.03397308)
\curveto(810.30545327,372.05397233)(810.2754533,372.05897232)(810.23545898,372.04897308)
\curveto(810.12545345,372.01897236)(810.02045356,371.96397242)(809.92045898,371.88397308)
\curveto(809.83045375,371.80397258)(809.74045384,371.73397265)(809.65045898,371.67397308)
\curveto(809.52045406,371.59397279)(809.3804542,371.51897286)(809.23045898,371.44897308)
\curveto(809.0804545,371.38897299)(808.92045466,371.33397305)(808.75045898,371.28397308)
\curveto(808.65045493,371.25397313)(808.54045504,371.23397315)(808.42045898,371.22397308)
\curveto(808.31045527,371.21397317)(808.20045538,371.19897318)(808.09045898,371.17897308)
\curveto(808.04045554,371.16897321)(807.99545558,371.16397322)(807.95545898,371.16397308)
\lineto(807.85045898,371.16397308)
\curveto(807.74045584,371.14397324)(807.63545594,371.14397324)(807.53545898,371.16397308)
\lineto(807.40045898,371.16397308)
\curveto(807.35045623,371.17397321)(807.30045628,371.1789732)(807.25045898,371.17897308)
\curveto(807.20045638,371.1789732)(807.15545642,371.18897319)(807.11545898,371.20897308)
\curveto(807.0754565,371.21897316)(807.04045654,371.22397316)(807.01045898,371.22397308)
\curveto(806.99045659,371.21397317)(806.96545661,371.21397317)(806.93545898,371.22397308)
\lineto(806.69545898,371.28397308)
\curveto(806.61545696,371.29397309)(806.54045704,371.31397307)(806.47045898,371.34397308)
\curveto(806.17045741,371.47397291)(805.92545765,371.61897276)(805.73545898,371.77897308)
\curveto(805.55545802,371.94897243)(805.40545817,372.1839722)(805.28545898,372.48397308)
\curveto(805.19545838,372.70397168)(805.15045843,372.96897141)(805.15045898,373.27897308)
\lineto(805.15045898,373.59397308)
\curveto(805.16045842,373.64397074)(805.16545841,373.69397069)(805.16545898,373.74397308)
\lineto(805.19545898,373.92397308)
\lineto(805.31545898,374.25397308)
\curveto(805.35545822,374.36397002)(805.40545817,374.46396992)(805.46545898,374.55397308)
\curveto(805.64545793,374.84396954)(805.89045769,375.05896932)(806.20045898,375.19897308)
\curveto(806.51045707,375.33896904)(806.85045673,375.46396892)(807.22045898,375.57397308)
\curveto(807.36045622,375.61396877)(807.50545607,375.64396874)(807.65545898,375.66397308)
\curveto(807.80545577,375.6839687)(807.95545562,375.70896867)(808.10545898,375.73897308)
\curveto(808.1754554,375.75896862)(808.24045534,375.76896861)(808.30045898,375.76897308)
\curveto(808.37045521,375.76896861)(808.44545513,375.7789686)(808.52545898,375.79897308)
\curveto(808.59545498,375.81896856)(808.66545491,375.82896855)(808.73545898,375.82897308)
\curveto(808.80545477,375.83896854)(808.8804547,375.85396853)(808.96045898,375.87397308)
\curveto(809.21045437,375.93396845)(809.44545413,375.9839684)(809.66545898,376.02397308)
\curveto(809.88545369,376.07396831)(810.06045352,376.18896819)(810.19045898,376.36897308)
\curveto(810.25045333,376.44896793)(810.30045328,376.54896783)(810.34045898,376.66897308)
\curveto(810.3804532,376.79896758)(810.3804532,376.93896744)(810.34045898,377.08897308)
\curveto(810.2804533,377.32896705)(810.19045339,377.51896686)(810.07045898,377.65897308)
\curveto(809.96045362,377.79896658)(809.80045378,377.90896647)(809.59045898,377.98897308)
\curveto(809.47045411,378.03896634)(809.32545425,378.07396631)(809.15545898,378.09397308)
\curveto(808.99545458,378.11396627)(808.82545475,378.12396626)(808.64545898,378.12397308)
\curveto(808.46545511,378.12396626)(808.29045529,378.11396627)(808.12045898,378.09397308)
\curveto(807.95045563,378.07396631)(807.80545577,378.04396634)(807.68545898,378.00397308)
\curveto(807.51545606,377.94396644)(807.35045623,377.85896652)(807.19045898,377.74897308)
\curveto(807.11045647,377.68896669)(807.03545654,377.60896677)(806.96545898,377.50897308)
\curveto(806.90545667,377.41896696)(806.85045673,377.31896706)(806.80045898,377.20897308)
\curveto(806.77045681,377.12896725)(806.74045684,377.04396734)(806.71045898,376.95397308)
\curveto(806.69045689,376.86396752)(806.64545693,376.79396759)(806.57545898,376.74397308)
\curveto(806.53545704,376.71396767)(806.46545711,376.68896769)(806.36545898,376.66897308)
\curveto(806.2754573,376.65896772)(806.1804574,376.65396773)(806.08045898,376.65397308)
\curveto(805.9804576,376.65396773)(805.8804577,376.65896772)(805.78045898,376.66897308)
\curveto(805.69045789,376.68896769)(805.62545795,376.71396767)(805.58545898,376.74397308)
\curveto(805.54545803,376.77396761)(805.51545806,376.82396756)(805.49545898,376.89397308)
\curveto(805.4754581,376.96396742)(805.4754581,377.03896734)(805.49545898,377.11897308)
\curveto(805.52545805,377.24896713)(805.55545802,377.36896701)(805.58545898,377.47897308)
\curveto(805.62545795,377.59896678)(805.67045791,377.71396667)(805.72045898,377.82397308)
\curveto(805.91045767,378.17396621)(806.15045743,378.44396594)(806.44045898,378.63397308)
\curveto(806.73045685,378.83396555)(807.09045649,378.99396539)(807.52045898,379.11397308)
\curveto(807.62045596,379.13396525)(807.72045586,379.14896523)(807.82045898,379.15897308)
\curveto(807.93045565,379.16896521)(808.04045554,379.1839652)(808.15045898,379.20397308)
\curveto(808.19045539,379.21396517)(808.25545532,379.21396517)(808.34545898,379.20397308)
\curveto(808.43545514,379.20396518)(808.49045509,379.21396517)(808.51045898,379.23397308)
\curveto(809.21045437,379.24396514)(809.82045376,379.16396522)(810.34045898,378.99397308)
\curveto(810.86045272,378.82396556)(811.22545235,378.49896588)(811.43545898,378.01897308)
\curveto(811.52545205,377.81896656)(811.575452,377.5839668)(811.58545898,377.31397308)
\curveto(811.60545197,377.05396733)(811.61545196,376.7789676)(811.61545898,376.48897308)
\lineto(811.61545898,373.17397308)
\curveto(811.61545196,373.03397135)(811.62045196,372.89897148)(811.63045898,372.76897308)
\curveto(811.64045194,372.63897174)(811.67045191,372.53397185)(811.72045898,372.45397308)
\curveto(811.77045181,372.383972)(811.83545174,372.33397205)(811.91545898,372.30397308)
\curveto(812.00545157,372.26397212)(812.09045149,372.23397215)(812.17045898,372.21397308)
\curveto(812.25045133,372.20397218)(812.31045127,372.15897222)(812.35045898,372.07897308)
\curveto(812.37045121,372.04897233)(812.3804512,372.01897236)(812.38045898,371.98897308)
\curveto(812.3804512,371.95897242)(812.38545119,371.91897246)(812.39545898,371.86897308)
\moveto(810.25045898,373.53397308)
\curveto(810.31045327,373.67397071)(810.34045324,373.83397055)(810.34045898,374.01397308)
\curveto(810.35045323,374.20397018)(810.35545322,374.39896998)(810.35545898,374.59897308)
\curveto(810.35545322,374.70896967)(810.35045323,374.80896957)(810.34045898,374.89897308)
\curveto(810.33045325,374.98896939)(810.29045329,375.05896932)(810.22045898,375.10897308)
\curveto(810.19045339,375.12896925)(810.12045346,375.13896924)(810.01045898,375.13897308)
\curveto(809.99045359,375.11896926)(809.95545362,375.10896927)(809.90545898,375.10897308)
\curveto(809.85545372,375.10896927)(809.81045377,375.09896928)(809.77045898,375.07897308)
\curveto(809.69045389,375.05896932)(809.60045398,375.03896934)(809.50045898,375.01897308)
\lineto(809.20045898,374.95897308)
\curveto(809.17045441,374.95896942)(809.13545444,374.95396943)(809.09545898,374.94397308)
\lineto(808.99045898,374.94397308)
\curveto(808.84045474,374.90396948)(808.6754549,374.8789695)(808.49545898,374.86897308)
\curveto(808.32545525,374.86896951)(808.16545541,374.84896953)(808.01545898,374.80897308)
\curveto(807.93545564,374.78896959)(807.86045572,374.76896961)(807.79045898,374.74897308)
\curveto(807.73045585,374.73896964)(807.66045592,374.72396966)(807.58045898,374.70397308)
\curveto(807.42045616,374.65396973)(807.27045631,374.58896979)(807.13045898,374.50897308)
\curveto(806.99045659,374.43896994)(806.87045671,374.34897003)(806.77045898,374.23897308)
\curveto(806.67045691,374.12897025)(806.59545698,373.99397039)(806.54545898,373.83397308)
\curveto(806.49545708,373.6839707)(806.4754571,373.49897088)(806.48545898,373.27897308)
\curveto(806.48545709,373.1789712)(806.50045708,373.0839713)(806.53045898,372.99397308)
\curveto(806.57045701,372.91397147)(806.61545696,372.83897154)(806.66545898,372.76897308)
\curveto(806.74545683,372.65897172)(806.85045673,372.56397182)(806.98045898,372.48397308)
\curveto(807.11045647,372.41397197)(807.25045633,372.35397203)(807.40045898,372.30397308)
\curveto(807.45045613,372.29397209)(807.50045608,372.28897209)(807.55045898,372.28897308)
\curveto(807.60045598,372.28897209)(807.65045593,372.2839721)(807.70045898,372.27397308)
\curveto(807.77045581,372.25397213)(807.85545572,372.23897214)(807.95545898,372.22897308)
\curveto(808.06545551,372.22897215)(808.15545542,372.23897214)(808.22545898,372.25897308)
\curveto(808.28545529,372.2789721)(808.34545523,372.2839721)(808.40545898,372.27397308)
\curveto(808.46545511,372.27397211)(808.52545505,372.2839721)(808.58545898,372.30397308)
\curveto(808.66545491,372.32397206)(808.74045484,372.33897204)(808.81045898,372.34897308)
\curveto(808.89045469,372.35897202)(808.96545461,372.378972)(809.03545898,372.40897308)
\curveto(809.32545425,372.52897185)(809.57045401,372.67397171)(809.77045898,372.84397308)
\curveto(809.9804536,373.01397137)(810.14045344,373.24397114)(810.25045898,373.53397308)
}
}
{
\newrgbcolor{curcolor}{0 0 0}
\pscustom[linestyle=none,fillstyle=solid,fillcolor=curcolor]
{
\newpath
\moveto(820.52709961,372.12397308)
\lineto(820.52709961,371.73397308)
\curveto(820.52709173,371.61397277)(820.50209176,371.51397287)(820.45209961,371.43397308)
\curveto(820.40209186,371.36397302)(820.31709194,371.32397306)(820.19709961,371.31397308)
\lineto(819.85209961,371.31397308)
\curveto(819.79209247,371.31397307)(819.73209253,371.30897307)(819.67209961,371.29897308)
\curveto(819.62209264,371.29897308)(819.57709268,371.30897307)(819.53709961,371.32897308)
\curveto(819.44709281,371.34897303)(819.38709287,371.38897299)(819.35709961,371.44897308)
\curveto(819.31709294,371.49897288)(819.29209297,371.55897282)(819.28209961,371.62897308)
\curveto(819.28209298,371.69897268)(819.26709299,371.76897261)(819.23709961,371.83897308)
\curveto(819.22709303,371.85897252)(819.21209305,371.87397251)(819.19209961,371.88397308)
\curveto(819.18209308,371.90397248)(819.16709309,371.92397246)(819.14709961,371.94397308)
\curveto(819.04709321,371.95397243)(818.96709329,371.93397245)(818.90709961,371.88397308)
\curveto(818.8570934,371.83397255)(818.80209346,371.7839726)(818.74209961,371.73397308)
\curveto(818.54209372,371.5839728)(818.34209392,371.46897291)(818.14209961,371.38897308)
\curveto(817.9620943,371.30897307)(817.75209451,371.24897313)(817.51209961,371.20897308)
\curveto(817.28209498,371.16897321)(817.04209522,371.14897323)(816.79209961,371.14897308)
\curveto(816.55209571,371.13897324)(816.31209595,371.15397323)(816.07209961,371.19397308)
\curveto(815.83209643,371.22397316)(815.62209664,371.2789731)(815.44209961,371.35897308)
\curveto(814.92209734,371.5789728)(814.50209776,371.87397251)(814.18209961,372.24397308)
\curveto(813.8620984,372.62397176)(813.61209865,373.09397129)(813.43209961,373.65397308)
\curveto(813.39209887,373.74397064)(813.3620989,373.83397055)(813.34209961,373.92397308)
\curveto(813.33209893,374.02397036)(813.31209895,374.12397026)(813.28209961,374.22397308)
\curveto(813.27209899,374.27397011)(813.26709899,374.32397006)(813.26709961,374.37397308)
\curveto(813.26709899,374.42396996)(813.262099,374.47396991)(813.25209961,374.52397308)
\curveto(813.23209903,374.57396981)(813.22209904,374.62396976)(813.22209961,374.67397308)
\curveto(813.23209903,374.73396965)(813.23209903,374.78896959)(813.22209961,374.83897308)
\lineto(813.22209961,374.98897308)
\curveto(813.20209906,375.03896934)(813.19209907,375.10396928)(813.19209961,375.18397308)
\curveto(813.19209907,375.26396912)(813.20209906,375.32896905)(813.22209961,375.37897308)
\lineto(813.22209961,375.54397308)
\curveto(813.24209902,375.61396877)(813.24709901,375.6839687)(813.23709961,375.75397308)
\curveto(813.23709902,375.83396855)(813.24709901,375.90896847)(813.26709961,375.97897308)
\curveto(813.27709898,376.02896835)(813.28209898,376.07396831)(813.28209961,376.11397308)
\curveto(813.28209898,376.15396823)(813.28709897,376.19896818)(813.29709961,376.24897308)
\curveto(813.32709893,376.34896803)(813.35209891,376.44396794)(813.37209961,376.53397308)
\curveto(813.39209887,376.63396775)(813.41709884,376.72896765)(813.44709961,376.81897308)
\curveto(813.57709868,377.19896718)(813.74209852,377.53896684)(813.94209961,377.83897308)
\curveto(814.15209811,378.14896623)(814.40209786,378.40396598)(814.69209961,378.60397308)
\curveto(814.8620974,378.72396566)(815.03709722,378.82396556)(815.21709961,378.90397308)
\curveto(815.40709685,378.9839654)(815.61209665,379.05396533)(815.83209961,379.11397308)
\curveto(815.90209636,379.12396526)(815.96709629,379.13396525)(816.02709961,379.14397308)
\curveto(816.09709616,379.15396523)(816.16709609,379.16896521)(816.23709961,379.18897308)
\lineto(816.38709961,379.18897308)
\curveto(816.46709579,379.20896517)(816.58209568,379.21896516)(816.73209961,379.21897308)
\curveto(816.89209537,379.21896516)(817.01209525,379.20896517)(817.09209961,379.18897308)
\curveto(817.13209513,379.1789652)(817.18709507,379.17396521)(817.25709961,379.17397308)
\curveto(817.36709489,379.14396524)(817.47709478,379.11896526)(817.58709961,379.09897308)
\curveto(817.69709456,379.08896529)(817.80209446,379.05896532)(817.90209961,379.00897308)
\curveto(818.05209421,378.94896543)(818.19209407,378.8839655)(818.32209961,378.81397308)
\curveto(818.4620938,378.74396564)(818.59209367,378.66396572)(818.71209961,378.57397308)
\curveto(818.77209349,378.52396586)(818.83209343,378.46896591)(818.89209961,378.40897308)
\curveto(818.9620933,378.35896602)(819.05209321,378.34396604)(819.16209961,378.36397308)
\curveto(819.18209308,378.39396599)(819.19709306,378.41896596)(819.20709961,378.43897308)
\curveto(819.22709303,378.45896592)(819.24209302,378.48896589)(819.25209961,378.52897308)
\curveto(819.28209298,378.61896576)(819.29209297,378.73396565)(819.28209961,378.87397308)
\lineto(819.28209961,379.24897308)
\lineto(819.28209961,380.97397308)
\lineto(819.28209961,381.43897308)
\curveto(819.28209298,381.61896276)(819.30709295,381.74896263)(819.35709961,381.82897308)
\curveto(819.39709286,381.89896248)(819.4570928,381.94396244)(819.53709961,381.96397308)
\curveto(819.5570927,381.96396242)(819.58209268,381.96396242)(819.61209961,381.96397308)
\curveto(819.64209262,381.97396241)(819.66709259,381.9789624)(819.68709961,381.97897308)
\curveto(819.82709243,381.98896239)(819.97209229,381.98896239)(820.12209961,381.97897308)
\curveto(820.28209198,381.9789624)(820.39209187,381.93896244)(820.45209961,381.85897308)
\curveto(820.50209176,381.7789626)(820.52709173,381.6789627)(820.52709961,381.55897308)
\lineto(820.52709961,381.18397308)
\lineto(820.52709961,372.12397308)
\moveto(819.31209961,374.95897308)
\curveto(819.33209293,375.00896937)(819.34209292,375.07396931)(819.34209961,375.15397308)
\curveto(819.34209292,375.24396914)(819.33209293,375.31396907)(819.31209961,375.36397308)
\lineto(819.31209961,375.58897308)
\curveto(819.29209297,375.6789687)(819.27709298,375.76896861)(819.26709961,375.85897308)
\curveto(819.257093,375.95896842)(819.23709302,376.04896833)(819.20709961,376.12897308)
\curveto(819.18709307,376.20896817)(819.16709309,376.2839681)(819.14709961,376.35397308)
\curveto(819.13709312,376.42396796)(819.11709314,376.49396789)(819.08709961,376.56397308)
\curveto(818.96709329,376.86396752)(818.81209345,377.12896725)(818.62209961,377.35897308)
\curveto(818.43209383,377.58896679)(818.19209407,377.76896661)(817.90209961,377.89897308)
\curveto(817.80209446,377.94896643)(817.69709456,377.9839664)(817.58709961,378.00397308)
\curveto(817.48709477,378.03396635)(817.37709488,378.05896632)(817.25709961,378.07897308)
\curveto(817.17709508,378.09896628)(817.08709517,378.10896627)(816.98709961,378.10897308)
\lineto(816.71709961,378.10897308)
\curveto(816.66709559,378.09896628)(816.62209564,378.08896629)(816.58209961,378.07897308)
\lineto(816.44709961,378.07897308)
\curveto(816.36709589,378.05896632)(816.28209598,378.03896634)(816.19209961,378.01897308)
\curveto(816.11209615,377.99896638)(816.03209623,377.97396641)(815.95209961,377.94397308)
\curveto(815.63209663,377.80396658)(815.37209689,377.59896678)(815.17209961,377.32897308)
\curveto(814.98209728,377.06896731)(814.82709743,376.76396762)(814.70709961,376.41397308)
\curveto(814.66709759,376.30396808)(814.63709762,376.18896819)(814.61709961,376.06897308)
\curveto(814.60709765,375.95896842)(814.59209767,375.84896853)(814.57209961,375.73897308)
\curveto(814.57209769,375.69896868)(814.56709769,375.65896872)(814.55709961,375.61897308)
\lineto(814.55709961,375.51397308)
\curveto(814.53709772,375.46396892)(814.52709773,375.40896897)(814.52709961,375.34897308)
\curveto(814.53709772,375.28896909)(814.54209772,375.23396915)(814.54209961,375.18397308)
\lineto(814.54209961,374.85397308)
\curveto(814.54209772,374.75396963)(814.55209771,374.65896972)(814.57209961,374.56897308)
\curveto(814.58209768,374.53896984)(814.58709767,374.48896989)(814.58709961,374.41897308)
\curveto(814.60709765,374.34897003)(814.62209764,374.2789701)(814.63209961,374.20897308)
\lineto(814.69209961,373.99897308)
\curveto(814.80209746,373.64897073)(814.95209731,373.34897103)(815.14209961,373.09897308)
\curveto(815.33209693,372.84897153)(815.57209669,372.64397174)(815.86209961,372.48397308)
\curveto(815.95209631,372.43397195)(816.04209622,372.39397199)(816.13209961,372.36397308)
\curveto(816.22209604,372.33397205)(816.32209594,372.30397208)(816.43209961,372.27397308)
\curveto(816.48209578,372.25397213)(816.53209573,372.24897213)(816.58209961,372.25897308)
\curveto(816.64209562,372.26897211)(816.69709556,372.26397212)(816.74709961,372.24397308)
\curveto(816.78709547,372.23397215)(816.82709543,372.22897215)(816.86709961,372.22897308)
\lineto(817.00209961,372.22897308)
\lineto(817.13709961,372.22897308)
\curveto(817.16709509,372.23897214)(817.21709504,372.24397214)(817.28709961,372.24397308)
\curveto(817.36709489,372.26397212)(817.44709481,372.2789721)(817.52709961,372.28897308)
\curveto(817.60709465,372.30897207)(817.68209458,372.33397205)(817.75209961,372.36397308)
\curveto(818.08209418,372.50397188)(818.34709391,372.6789717)(818.54709961,372.88897308)
\curveto(818.7570935,373.10897127)(818.93209333,373.383971)(819.07209961,373.71397308)
\curveto(819.12209314,373.82397056)(819.1570931,373.93397045)(819.17709961,374.04397308)
\curveto(819.19709306,374.15397023)(819.22209304,374.26397012)(819.25209961,374.37397308)
\curveto(819.27209299,374.41396997)(819.28209298,374.44896993)(819.28209961,374.47897308)
\curveto(819.28209298,374.51896986)(819.28709297,374.55896982)(819.29709961,374.59897308)
\curveto(819.30709295,374.65896972)(819.30709295,374.71896966)(819.29709961,374.77897308)
\curveto(819.29709296,374.83896954)(819.30209296,374.89896948)(819.31209961,374.95897308)
}
}
{
\newrgbcolor{curcolor}{0 0 0}
\pscustom[linestyle=none,fillstyle=solid,fillcolor=curcolor]
{
}
}
{
\newrgbcolor{curcolor}{0 0 0}
\pscustom[linestyle=none,fillstyle=solid,fillcolor=curcolor]
{
\newpath
\moveto(833.59350586,372.12397308)
\lineto(833.59350586,371.73397308)
\curveto(833.59349798,371.61397277)(833.56849801,371.51397287)(833.51850586,371.43397308)
\curveto(833.46849811,371.36397302)(833.38349819,371.32397306)(833.26350586,371.31397308)
\lineto(832.91850586,371.31397308)
\curveto(832.85849872,371.31397307)(832.79849878,371.30897307)(832.73850586,371.29897308)
\curveto(832.68849889,371.29897308)(832.64349893,371.30897307)(832.60350586,371.32897308)
\curveto(832.51349906,371.34897303)(832.45349912,371.38897299)(832.42350586,371.44897308)
\curveto(832.38349919,371.49897288)(832.35849922,371.55897282)(832.34850586,371.62897308)
\curveto(832.34849923,371.69897268)(832.33349924,371.76897261)(832.30350586,371.83897308)
\curveto(832.29349928,371.85897252)(832.2784993,371.87397251)(832.25850586,371.88397308)
\curveto(832.24849933,371.90397248)(832.23349934,371.92397246)(832.21350586,371.94397308)
\curveto(832.11349946,371.95397243)(832.03349954,371.93397245)(831.97350586,371.88397308)
\curveto(831.92349965,371.83397255)(831.86849971,371.7839726)(831.80850586,371.73397308)
\curveto(831.60849997,371.5839728)(831.40850017,371.46897291)(831.20850586,371.38897308)
\curveto(831.02850055,371.30897307)(830.81850076,371.24897313)(830.57850586,371.20897308)
\curveto(830.34850123,371.16897321)(830.10850147,371.14897323)(829.85850586,371.14897308)
\curveto(829.61850196,371.13897324)(829.3785022,371.15397323)(829.13850586,371.19397308)
\curveto(828.89850268,371.22397316)(828.68850289,371.2789731)(828.50850586,371.35897308)
\curveto(827.98850359,371.5789728)(827.56850401,371.87397251)(827.24850586,372.24397308)
\curveto(826.92850465,372.62397176)(826.6785049,373.09397129)(826.49850586,373.65397308)
\curveto(826.45850512,373.74397064)(826.42850515,373.83397055)(826.40850586,373.92397308)
\curveto(826.39850518,374.02397036)(826.3785052,374.12397026)(826.34850586,374.22397308)
\curveto(826.33850524,374.27397011)(826.33350524,374.32397006)(826.33350586,374.37397308)
\curveto(826.33350524,374.42396996)(826.32850525,374.47396991)(826.31850586,374.52397308)
\curveto(826.29850528,374.57396981)(826.28850529,374.62396976)(826.28850586,374.67397308)
\curveto(826.29850528,374.73396965)(826.29850528,374.78896959)(826.28850586,374.83897308)
\lineto(826.28850586,374.98897308)
\curveto(826.26850531,375.03896934)(826.25850532,375.10396928)(826.25850586,375.18397308)
\curveto(826.25850532,375.26396912)(826.26850531,375.32896905)(826.28850586,375.37897308)
\lineto(826.28850586,375.54397308)
\curveto(826.30850527,375.61396877)(826.31350526,375.6839687)(826.30350586,375.75397308)
\curveto(826.30350527,375.83396855)(826.31350526,375.90896847)(826.33350586,375.97897308)
\curveto(826.34350523,376.02896835)(826.34850523,376.07396831)(826.34850586,376.11397308)
\curveto(826.34850523,376.15396823)(826.35350522,376.19896818)(826.36350586,376.24897308)
\curveto(826.39350518,376.34896803)(826.41850516,376.44396794)(826.43850586,376.53397308)
\curveto(826.45850512,376.63396775)(826.48350509,376.72896765)(826.51350586,376.81897308)
\curveto(826.64350493,377.19896718)(826.80850477,377.53896684)(827.00850586,377.83897308)
\curveto(827.21850436,378.14896623)(827.46850411,378.40396598)(827.75850586,378.60397308)
\curveto(827.92850365,378.72396566)(828.10350347,378.82396556)(828.28350586,378.90397308)
\curveto(828.4735031,378.9839654)(828.6785029,379.05396533)(828.89850586,379.11397308)
\curveto(828.96850261,379.12396526)(829.03350254,379.13396525)(829.09350586,379.14397308)
\curveto(829.16350241,379.15396523)(829.23350234,379.16896521)(829.30350586,379.18897308)
\lineto(829.45350586,379.18897308)
\curveto(829.53350204,379.20896517)(829.64850193,379.21896516)(829.79850586,379.21897308)
\curveto(829.95850162,379.21896516)(830.0785015,379.20896517)(830.15850586,379.18897308)
\curveto(830.19850138,379.1789652)(830.25350132,379.17396521)(830.32350586,379.17397308)
\curveto(830.43350114,379.14396524)(830.54350103,379.11896526)(830.65350586,379.09897308)
\curveto(830.76350081,379.08896529)(830.86850071,379.05896532)(830.96850586,379.00897308)
\curveto(831.11850046,378.94896543)(831.25850032,378.8839655)(831.38850586,378.81397308)
\curveto(831.52850005,378.74396564)(831.65849992,378.66396572)(831.77850586,378.57397308)
\curveto(831.83849974,378.52396586)(831.89849968,378.46896591)(831.95850586,378.40897308)
\curveto(832.02849955,378.35896602)(832.11849946,378.34396604)(832.22850586,378.36397308)
\curveto(832.24849933,378.39396599)(832.26349931,378.41896596)(832.27350586,378.43897308)
\curveto(832.29349928,378.45896592)(832.30849927,378.48896589)(832.31850586,378.52897308)
\curveto(832.34849923,378.61896576)(832.35849922,378.73396565)(832.34850586,378.87397308)
\lineto(832.34850586,379.24897308)
\lineto(832.34850586,380.97397308)
\lineto(832.34850586,381.43897308)
\curveto(832.34849923,381.61896276)(832.3734992,381.74896263)(832.42350586,381.82897308)
\curveto(832.46349911,381.89896248)(832.52349905,381.94396244)(832.60350586,381.96397308)
\curveto(832.62349895,381.96396242)(832.64849893,381.96396242)(832.67850586,381.96397308)
\curveto(832.70849887,381.97396241)(832.73349884,381.9789624)(832.75350586,381.97897308)
\curveto(832.89349868,381.98896239)(833.03849854,381.98896239)(833.18850586,381.97897308)
\curveto(833.34849823,381.9789624)(833.45849812,381.93896244)(833.51850586,381.85897308)
\curveto(833.56849801,381.7789626)(833.59349798,381.6789627)(833.59350586,381.55897308)
\lineto(833.59350586,381.18397308)
\lineto(833.59350586,372.12397308)
\moveto(832.37850586,374.95897308)
\curveto(832.39849918,375.00896937)(832.40849917,375.07396931)(832.40850586,375.15397308)
\curveto(832.40849917,375.24396914)(832.39849918,375.31396907)(832.37850586,375.36397308)
\lineto(832.37850586,375.58897308)
\curveto(832.35849922,375.6789687)(832.34349923,375.76896861)(832.33350586,375.85897308)
\curveto(832.32349925,375.95896842)(832.30349927,376.04896833)(832.27350586,376.12897308)
\curveto(832.25349932,376.20896817)(832.23349934,376.2839681)(832.21350586,376.35397308)
\curveto(832.20349937,376.42396796)(832.18349939,376.49396789)(832.15350586,376.56397308)
\curveto(832.03349954,376.86396752)(831.8784997,377.12896725)(831.68850586,377.35897308)
\curveto(831.49850008,377.58896679)(831.25850032,377.76896661)(830.96850586,377.89897308)
\curveto(830.86850071,377.94896643)(830.76350081,377.9839664)(830.65350586,378.00397308)
\curveto(830.55350102,378.03396635)(830.44350113,378.05896632)(830.32350586,378.07897308)
\curveto(830.24350133,378.09896628)(830.15350142,378.10896627)(830.05350586,378.10897308)
\lineto(829.78350586,378.10897308)
\curveto(829.73350184,378.09896628)(829.68850189,378.08896629)(829.64850586,378.07897308)
\lineto(829.51350586,378.07897308)
\curveto(829.43350214,378.05896632)(829.34850223,378.03896634)(829.25850586,378.01897308)
\curveto(829.1785024,377.99896638)(829.09850248,377.97396641)(829.01850586,377.94397308)
\curveto(828.69850288,377.80396658)(828.43850314,377.59896678)(828.23850586,377.32897308)
\curveto(828.04850353,377.06896731)(827.89350368,376.76396762)(827.77350586,376.41397308)
\curveto(827.73350384,376.30396808)(827.70350387,376.18896819)(827.68350586,376.06897308)
\curveto(827.6735039,375.95896842)(827.65850392,375.84896853)(827.63850586,375.73897308)
\curveto(827.63850394,375.69896868)(827.63350394,375.65896872)(827.62350586,375.61897308)
\lineto(827.62350586,375.51397308)
\curveto(827.60350397,375.46396892)(827.59350398,375.40896897)(827.59350586,375.34897308)
\curveto(827.60350397,375.28896909)(827.60850397,375.23396915)(827.60850586,375.18397308)
\lineto(827.60850586,374.85397308)
\curveto(827.60850397,374.75396963)(827.61850396,374.65896972)(827.63850586,374.56897308)
\curveto(827.64850393,374.53896984)(827.65350392,374.48896989)(827.65350586,374.41897308)
\curveto(827.6735039,374.34897003)(827.68850389,374.2789701)(827.69850586,374.20897308)
\lineto(827.75850586,373.99897308)
\curveto(827.86850371,373.64897073)(828.01850356,373.34897103)(828.20850586,373.09897308)
\curveto(828.39850318,372.84897153)(828.63850294,372.64397174)(828.92850586,372.48397308)
\curveto(829.01850256,372.43397195)(829.10850247,372.39397199)(829.19850586,372.36397308)
\curveto(829.28850229,372.33397205)(829.38850219,372.30397208)(829.49850586,372.27397308)
\curveto(829.54850203,372.25397213)(829.59850198,372.24897213)(829.64850586,372.25897308)
\curveto(829.70850187,372.26897211)(829.76350181,372.26397212)(829.81350586,372.24397308)
\curveto(829.85350172,372.23397215)(829.89350168,372.22897215)(829.93350586,372.22897308)
\lineto(830.06850586,372.22897308)
\lineto(830.20350586,372.22897308)
\curveto(830.23350134,372.23897214)(830.28350129,372.24397214)(830.35350586,372.24397308)
\curveto(830.43350114,372.26397212)(830.51350106,372.2789721)(830.59350586,372.28897308)
\curveto(830.6735009,372.30897207)(830.74850083,372.33397205)(830.81850586,372.36397308)
\curveto(831.14850043,372.50397188)(831.41350016,372.6789717)(831.61350586,372.88897308)
\curveto(831.82349975,373.10897127)(831.99849958,373.383971)(832.13850586,373.71397308)
\curveto(832.18849939,373.82397056)(832.22349935,373.93397045)(832.24350586,374.04397308)
\curveto(832.26349931,374.15397023)(832.28849929,374.26397012)(832.31850586,374.37397308)
\curveto(832.33849924,374.41396997)(832.34849923,374.44896993)(832.34850586,374.47897308)
\curveto(832.34849923,374.51896986)(832.35349922,374.55896982)(832.36350586,374.59897308)
\curveto(832.3734992,374.65896972)(832.3734992,374.71896966)(832.36350586,374.77897308)
\curveto(832.36349921,374.83896954)(832.36849921,374.89896948)(832.37850586,374.95897308)
}
}
{
\newrgbcolor{curcolor}{0 0 0}
\pscustom[linestyle=none,fillstyle=solid,fillcolor=curcolor]
{
\newpath
\moveto(842.28975586,375.48397308)
\curveto(842.30974817,375.383969)(842.30974817,375.26896911)(842.28975586,375.13897308)
\curveto(842.2797482,375.01896936)(842.24974823,374.93396945)(842.19975586,374.88397308)
\curveto(842.14974833,374.84396954)(842.07474841,374.81396957)(841.97475586,374.79397308)
\curveto(841.8847486,374.7839696)(841.7797487,374.7789696)(841.65975586,374.77897308)
\lineto(841.29975586,374.77897308)
\curveto(841.1797493,374.78896959)(841.07474941,374.79396959)(840.98475586,374.79397308)
\lineto(837.14475586,374.79397308)
\curveto(837.06475342,374.79396959)(836.9847535,374.78896959)(836.90475586,374.77897308)
\curveto(836.82475366,374.7789696)(836.75975372,374.76396962)(836.70975586,374.73397308)
\curveto(836.66975381,374.71396967)(836.62975385,374.67396971)(836.58975586,374.61397308)
\curveto(836.56975391,374.5839698)(836.54975393,374.53896984)(836.52975586,374.47897308)
\curveto(836.50975397,374.42896995)(836.50975397,374.37897)(836.52975586,374.32897308)
\curveto(836.53975394,374.2789701)(836.54475394,374.23397015)(836.54475586,374.19397308)
\curveto(836.54475394,374.15397023)(836.54975393,374.11397027)(836.55975586,374.07397308)
\curveto(836.5797539,373.99397039)(836.59975388,373.90897047)(836.61975586,373.81897308)
\curveto(836.63975384,373.73897064)(836.66975381,373.65897072)(836.70975586,373.57897308)
\curveto(836.93975354,373.03897134)(837.31975316,372.65397173)(837.84975586,372.42397308)
\curveto(837.90975257,372.39397199)(837.97475251,372.36897201)(838.04475586,372.34897308)
\lineto(838.25475586,372.28897308)
\curveto(838.2847522,372.2789721)(838.33475215,372.27397211)(838.40475586,372.27397308)
\curveto(838.54475194,372.23397215)(838.72975175,372.21397217)(838.95975586,372.21397308)
\curveto(839.18975129,372.21397217)(839.37475111,372.23397215)(839.51475586,372.27397308)
\curveto(839.65475083,372.31397207)(839.7797507,372.35397203)(839.88975586,372.39397308)
\curveto(840.00975047,372.44397194)(840.11975036,372.50397188)(840.21975586,372.57397308)
\curveto(840.32975015,372.64397174)(840.42475006,372.72397166)(840.50475586,372.81397308)
\curveto(840.5847499,372.91397147)(840.65474983,373.01897136)(840.71475586,373.12897308)
\curveto(840.77474971,373.22897115)(840.82474966,373.33397105)(840.86475586,373.44397308)
\curveto(840.91474957,373.55397083)(840.99474949,373.63397075)(841.10475586,373.68397308)
\curveto(841.14474934,373.70397068)(841.20974927,373.71897066)(841.29975586,373.72897308)
\curveto(841.38974909,373.73897064)(841.479749,373.73897064)(841.56975586,373.72897308)
\curveto(841.65974882,373.72897065)(841.74474874,373.72397066)(841.82475586,373.71397308)
\curveto(841.90474858,373.70397068)(841.95974852,373.6839707)(841.98975586,373.65397308)
\curveto(842.08974839,373.5839708)(842.11474837,373.46897091)(842.06475586,373.30897308)
\curveto(841.9847485,373.03897134)(841.8797486,372.79897158)(841.74975586,372.58897308)
\curveto(841.54974893,372.26897211)(841.31974916,372.00397238)(841.05975586,371.79397308)
\curveto(840.80974967,371.59397279)(840.48974999,371.42897295)(840.09975586,371.29897308)
\curveto(839.99975048,371.25897312)(839.89975058,371.23397315)(839.79975586,371.22397308)
\curveto(839.69975078,371.20397318)(839.59475089,371.1839732)(839.48475586,371.16397308)
\curveto(839.43475105,371.15397323)(839.3847511,371.14897323)(839.33475586,371.14897308)
\curveto(839.29475119,371.14897323)(839.24975123,371.14397324)(839.19975586,371.13397308)
\lineto(839.04975586,371.13397308)
\curveto(838.99975148,371.12397326)(838.93975154,371.11897326)(838.86975586,371.11897308)
\curveto(838.80975167,371.11897326)(838.75975172,371.12397326)(838.71975586,371.13397308)
\lineto(838.58475586,371.13397308)
\curveto(838.53475195,371.14397324)(838.48975199,371.14897323)(838.44975586,371.14897308)
\curveto(838.40975207,371.14897323)(838.36975211,371.15397323)(838.32975586,371.16397308)
\curveto(838.2797522,371.17397321)(838.22475226,371.1839732)(838.16475586,371.19397308)
\curveto(838.10475238,371.19397319)(838.04975243,371.19897318)(837.99975586,371.20897308)
\curveto(837.90975257,371.22897315)(837.81975266,371.25397313)(837.72975586,371.28397308)
\curveto(837.63975284,371.30397308)(837.55475293,371.32897305)(837.47475586,371.35897308)
\curveto(837.43475305,371.378973)(837.39975308,371.38897299)(837.36975586,371.38897308)
\curveto(837.33975314,371.39897298)(837.30475318,371.41397297)(837.26475586,371.43397308)
\curveto(837.11475337,371.50397288)(836.95475353,371.58897279)(836.78475586,371.68897308)
\curveto(836.49475399,371.8789725)(836.24475424,372.10897227)(836.03475586,372.37897308)
\curveto(835.83475465,372.65897172)(835.66475482,372.96897141)(835.52475586,373.30897308)
\curveto(835.47475501,373.41897096)(835.43475505,373.53397085)(835.40475586,373.65397308)
\curveto(835.3847551,373.77397061)(835.35475513,373.89397049)(835.31475586,374.01397308)
\curveto(835.30475518,374.05397033)(835.29975518,374.08897029)(835.29975586,374.11897308)
\curveto(835.29975518,374.14897023)(835.29475519,374.18897019)(835.28475586,374.23897308)
\curveto(835.26475522,374.31897006)(835.24975523,374.40396998)(835.23975586,374.49397308)
\curveto(835.22975525,374.5839698)(835.21475527,374.67396971)(835.19475586,374.76397308)
\lineto(835.19475586,374.97397308)
\curveto(835.1847553,375.01396937)(835.17475531,375.06896931)(835.16475586,375.13897308)
\curveto(835.16475532,375.21896916)(835.16975531,375.2839691)(835.17975586,375.33397308)
\lineto(835.17975586,375.49897308)
\curveto(835.19975528,375.54896883)(835.20475528,375.59896878)(835.19475586,375.64897308)
\curveto(835.19475529,375.70896867)(835.19975528,375.76396862)(835.20975586,375.81397308)
\curveto(835.24975523,375.97396841)(835.2797552,376.13396825)(835.29975586,376.29397308)
\curveto(835.32975515,376.45396793)(835.37475511,376.60396778)(835.43475586,376.74397308)
\curveto(835.484755,376.85396753)(835.52975495,376.96396742)(835.56975586,377.07397308)
\curveto(835.61975486,377.19396719)(835.67475481,377.30896707)(835.73475586,377.41897308)
\curveto(835.95475453,377.76896661)(836.20475428,378.06896631)(836.48475586,378.31897308)
\curveto(836.76475372,378.5789658)(837.10975337,378.79396559)(837.51975586,378.96397308)
\curveto(837.63975284,379.01396537)(837.75975272,379.04896533)(837.87975586,379.06897308)
\curveto(838.00975247,379.09896528)(838.14475234,379.12896525)(838.28475586,379.15897308)
\curveto(838.33475215,379.16896521)(838.3797521,379.17396521)(838.41975586,379.17397308)
\curveto(838.45975202,379.1839652)(838.50475198,379.18896519)(838.55475586,379.18897308)
\curveto(838.57475191,379.19896518)(838.59975188,379.19896518)(838.62975586,379.18897308)
\curveto(838.65975182,379.1789652)(838.6847518,379.1839652)(838.70475586,379.20397308)
\curveto(839.12475136,379.21396517)(839.48975099,379.16896521)(839.79975586,379.06897308)
\curveto(840.10975037,378.9789654)(840.38975009,378.85396553)(840.63975586,378.69397308)
\curveto(840.68974979,378.67396571)(840.72974975,378.64396574)(840.75975586,378.60397308)
\curveto(840.78974969,378.57396581)(840.82474966,378.54896583)(840.86475586,378.52897308)
\curveto(840.94474954,378.46896591)(841.02474946,378.39896598)(841.10475586,378.31897308)
\curveto(841.19474929,378.23896614)(841.26974921,378.15896622)(841.32975586,378.07897308)
\curveto(841.48974899,377.86896651)(841.62474886,377.66896671)(841.73475586,377.47897308)
\curveto(841.80474868,377.36896701)(841.85974862,377.24896713)(841.89975586,377.11897308)
\curveto(841.93974854,376.98896739)(841.9847485,376.85896752)(842.03475586,376.72897308)
\curveto(842.0847484,376.59896778)(842.11974836,376.46396792)(842.13975586,376.32397308)
\curveto(842.16974831,376.1839682)(842.20474828,376.04396834)(842.24475586,375.90397308)
\curveto(842.25474823,375.83396855)(842.25974822,375.76396862)(842.25975586,375.69397308)
\lineto(842.28975586,375.48397308)
\moveto(840.83475586,375.99397308)
\curveto(840.86474962,376.03396835)(840.88974959,376.0839683)(840.90975586,376.14397308)
\curveto(840.92974955,376.21396817)(840.92974955,376.2839681)(840.90975586,376.35397308)
\curveto(840.84974963,376.57396781)(840.76474972,376.7789676)(840.65475586,376.96897308)
\curveto(840.51474997,377.19896718)(840.35975012,377.39396699)(840.18975586,377.55397308)
\curveto(840.01975046,377.71396667)(839.79975068,377.84896653)(839.52975586,377.95897308)
\curveto(839.45975102,377.9789664)(839.38975109,377.99396639)(839.31975586,378.00397308)
\curveto(839.24975123,378.02396636)(839.17475131,378.04396634)(839.09475586,378.06397308)
\curveto(839.01475147,378.0839663)(838.92975155,378.09396629)(838.83975586,378.09397308)
\lineto(838.58475586,378.09397308)
\curveto(838.55475193,378.07396631)(838.51975196,378.06396632)(838.47975586,378.06397308)
\curveto(838.43975204,378.07396631)(838.40475208,378.07396631)(838.37475586,378.06397308)
\lineto(838.13475586,378.00397308)
\curveto(838.06475242,377.99396639)(837.99475249,377.9789664)(837.92475586,377.95897308)
\curveto(837.63475285,377.83896654)(837.39975308,377.68896669)(837.21975586,377.50897308)
\curveto(837.04975343,377.32896705)(836.89475359,377.10396728)(836.75475586,376.83397308)
\curveto(836.72475376,376.7839676)(836.69475379,376.71896766)(836.66475586,376.63897308)
\curveto(836.63475385,376.56896781)(836.60975387,376.48896789)(836.58975586,376.39897308)
\curveto(836.56975391,376.30896807)(836.56475392,376.22396816)(836.57475586,376.14397308)
\curveto(836.5847539,376.06396832)(836.61975386,376.00396838)(836.67975586,375.96397308)
\curveto(836.75975372,375.90396848)(836.89475359,375.87396851)(837.08475586,375.87397308)
\curveto(837.2847532,375.8839685)(837.45475303,375.88896849)(837.59475586,375.88897308)
\lineto(839.87475586,375.88897308)
\curveto(840.02475046,375.88896849)(840.20475028,375.8839685)(840.41475586,375.87397308)
\curveto(840.62474986,375.87396851)(840.76474972,375.91396847)(840.83475586,375.99397308)
}
}
{
\newrgbcolor{curcolor}{0 0 0}
\pscustom[linestyle=none,fillstyle=solid,fillcolor=curcolor]
{
\newpath
\moveto(768.02144531,357.41765076)
\curveto(768.04143763,357.31764667)(768.04143763,357.20264679)(768.02144531,357.07265076)
\curveto(768.01143766,356.95264704)(767.98143769,356.86764712)(767.93144531,356.81765076)
\curveto(767.88143779,356.77764721)(767.80643786,356.74764724)(767.70644531,356.72765076)
\curveto(767.61643805,356.71764727)(767.51143816,356.71264728)(767.39144531,356.71265076)
\lineto(767.03144531,356.71265076)
\curveto(766.91143876,356.72264727)(766.80643886,356.72764726)(766.71644531,356.72765076)
\lineto(762.87644531,356.72765076)
\curveto(762.79644287,356.72764726)(762.71644295,356.72264727)(762.63644531,356.71265076)
\curveto(762.55644311,356.71264728)(762.49144318,356.69764729)(762.44144531,356.66765076)
\curveto(762.40144327,356.64764734)(762.36144331,356.60764738)(762.32144531,356.54765076)
\curveto(762.30144337,356.51764747)(762.28144339,356.47264752)(762.26144531,356.41265076)
\curveto(762.24144343,356.36264763)(762.24144343,356.31264768)(762.26144531,356.26265076)
\curveto(762.2714434,356.21264778)(762.27644339,356.16764782)(762.27644531,356.12765076)
\curveto(762.27644339,356.0876479)(762.28144339,356.04764794)(762.29144531,356.00765076)
\curveto(762.31144336,355.92764806)(762.33144334,355.84264815)(762.35144531,355.75265076)
\curveto(762.3714433,355.67264832)(762.40144327,355.5926484)(762.44144531,355.51265076)
\curveto(762.671443,354.97264902)(763.05144262,354.5876494)(763.58144531,354.35765076)
\curveto(763.64144203,354.32764966)(763.70644196,354.30264969)(763.77644531,354.28265076)
\lineto(763.98644531,354.22265076)
\curveto(764.01644165,354.21264978)(764.0664416,354.20764978)(764.13644531,354.20765076)
\curveto(764.27644139,354.16764982)(764.46144121,354.14764984)(764.69144531,354.14765076)
\curveto(764.92144075,354.14764984)(765.10644056,354.16764982)(765.24644531,354.20765076)
\curveto(765.38644028,354.24764974)(765.51144016,354.2876497)(765.62144531,354.32765076)
\curveto(765.74143993,354.37764961)(765.85143982,354.43764955)(765.95144531,354.50765076)
\curveto(766.06143961,354.57764941)(766.15643951,354.65764933)(766.23644531,354.74765076)
\curveto(766.31643935,354.84764914)(766.38643928,354.95264904)(766.44644531,355.06265076)
\curveto(766.50643916,355.16264883)(766.55643911,355.26764872)(766.59644531,355.37765076)
\curveto(766.64643902,355.4876485)(766.72643894,355.56764842)(766.83644531,355.61765076)
\curveto(766.87643879,355.63764835)(766.94143873,355.65264834)(767.03144531,355.66265076)
\curveto(767.12143855,355.67264832)(767.21143846,355.67264832)(767.30144531,355.66265076)
\curveto(767.39143828,355.66264833)(767.47643819,355.65764833)(767.55644531,355.64765076)
\curveto(767.63643803,355.63764835)(767.69143798,355.61764837)(767.72144531,355.58765076)
\curveto(767.82143785,355.51764847)(767.84643782,355.40264859)(767.79644531,355.24265076)
\curveto(767.71643795,354.97264902)(767.61143806,354.73264926)(767.48144531,354.52265076)
\curveto(767.28143839,354.20264979)(767.05143862,353.93765005)(766.79144531,353.72765076)
\curveto(766.54143913,353.52765046)(766.22143945,353.36265063)(765.83144531,353.23265076)
\curveto(765.73143994,353.1926508)(765.63144004,353.16765082)(765.53144531,353.15765076)
\curveto(765.43144024,353.13765085)(765.32644034,353.11765087)(765.21644531,353.09765076)
\curveto(765.1664405,353.0876509)(765.11644055,353.08265091)(765.06644531,353.08265076)
\curveto(765.02644064,353.08265091)(764.98144069,353.07765091)(764.93144531,353.06765076)
\lineto(764.78144531,353.06765076)
\curveto(764.73144094,353.05765093)(764.671441,353.05265094)(764.60144531,353.05265076)
\curveto(764.54144113,353.05265094)(764.49144118,353.05765093)(764.45144531,353.06765076)
\lineto(764.31644531,353.06765076)
\curveto(764.2664414,353.07765091)(764.22144145,353.08265091)(764.18144531,353.08265076)
\curveto(764.14144153,353.08265091)(764.10144157,353.0876509)(764.06144531,353.09765076)
\curveto(764.01144166,353.10765088)(763.95644171,353.11765087)(763.89644531,353.12765076)
\curveto(763.83644183,353.12765086)(763.78144189,353.13265086)(763.73144531,353.14265076)
\curveto(763.64144203,353.16265083)(763.55144212,353.1876508)(763.46144531,353.21765076)
\curveto(763.3714423,353.23765075)(763.28644238,353.26265073)(763.20644531,353.29265076)
\curveto(763.1664425,353.31265068)(763.13144254,353.32265067)(763.10144531,353.32265076)
\curveto(763.0714426,353.33265066)(763.03644263,353.34765064)(762.99644531,353.36765076)
\curveto(762.84644282,353.43765055)(762.68644298,353.52265047)(762.51644531,353.62265076)
\curveto(762.22644344,353.81265018)(761.97644369,354.04264995)(761.76644531,354.31265076)
\curveto(761.5664441,354.5926494)(761.39644427,354.90264909)(761.25644531,355.24265076)
\curveto(761.20644446,355.35264864)(761.1664445,355.46764852)(761.13644531,355.58765076)
\curveto(761.11644455,355.70764828)(761.08644458,355.82764816)(761.04644531,355.94765076)
\curveto(761.03644463,355.987648)(761.03144464,356.02264797)(761.03144531,356.05265076)
\curveto(761.03144464,356.08264791)(761.02644464,356.12264787)(761.01644531,356.17265076)
\curveto(760.99644467,356.25264774)(760.98144469,356.33764765)(760.97144531,356.42765076)
\curveto(760.96144471,356.51764747)(760.94644472,356.60764738)(760.92644531,356.69765076)
\lineto(760.92644531,356.90765076)
\curveto(760.91644475,356.94764704)(760.90644476,357.00264699)(760.89644531,357.07265076)
\curveto(760.89644477,357.15264684)(760.90144477,357.21764677)(760.91144531,357.26765076)
\lineto(760.91144531,357.43265076)
\curveto(760.93144474,357.48264651)(760.93644473,357.53264646)(760.92644531,357.58265076)
\curveto(760.92644474,357.64264635)(760.93144474,357.69764629)(760.94144531,357.74765076)
\curveto(760.98144469,357.90764608)(761.01144466,358.06764592)(761.03144531,358.22765076)
\curveto(761.06144461,358.3876456)(761.10644456,358.53764545)(761.16644531,358.67765076)
\curveto(761.21644445,358.7876452)(761.26144441,358.89764509)(761.30144531,359.00765076)
\curveto(761.35144432,359.12764486)(761.40644426,359.24264475)(761.46644531,359.35265076)
\curveto(761.68644398,359.70264429)(761.93644373,360.00264399)(762.21644531,360.25265076)
\curveto(762.49644317,360.51264348)(762.84144283,360.72764326)(763.25144531,360.89765076)
\curveto(763.3714423,360.94764304)(763.49144218,360.98264301)(763.61144531,361.00265076)
\curveto(763.74144193,361.03264296)(763.87644179,361.06264293)(764.01644531,361.09265076)
\curveto(764.0664416,361.10264289)(764.11144156,361.10764288)(764.15144531,361.10765076)
\curveto(764.19144148,361.11764287)(764.23644143,361.12264287)(764.28644531,361.12265076)
\curveto(764.30644136,361.13264286)(764.33144134,361.13264286)(764.36144531,361.12265076)
\curveto(764.39144128,361.11264288)(764.41644125,361.11764287)(764.43644531,361.13765076)
\curveto(764.85644081,361.14764284)(765.22144045,361.10264289)(765.53144531,361.00265076)
\curveto(765.84143983,360.91264308)(766.12143955,360.7876432)(766.37144531,360.62765076)
\curveto(766.42143925,360.60764338)(766.46143921,360.57764341)(766.49144531,360.53765076)
\curveto(766.52143915,360.50764348)(766.55643911,360.48264351)(766.59644531,360.46265076)
\curveto(766.67643899,360.40264359)(766.75643891,360.33264366)(766.83644531,360.25265076)
\curveto(766.92643874,360.17264382)(767.00143867,360.0926439)(767.06144531,360.01265076)
\curveto(767.22143845,359.80264419)(767.35643831,359.60264439)(767.46644531,359.41265076)
\curveto(767.53643813,359.30264469)(767.59143808,359.18264481)(767.63144531,359.05265076)
\curveto(767.671438,358.92264507)(767.71643795,358.7926452)(767.76644531,358.66265076)
\curveto(767.81643785,358.53264546)(767.85143782,358.39764559)(767.87144531,358.25765076)
\curveto(767.90143777,358.11764587)(767.93643773,357.97764601)(767.97644531,357.83765076)
\curveto(767.98643768,357.76764622)(767.99143768,357.69764629)(767.99144531,357.62765076)
\lineto(768.02144531,357.41765076)
\moveto(766.56644531,357.92765076)
\curveto(766.59643907,357.96764602)(766.62143905,358.01764597)(766.64144531,358.07765076)
\curveto(766.66143901,358.14764584)(766.66143901,358.21764577)(766.64144531,358.28765076)
\curveto(766.58143909,358.50764548)(766.49643917,358.71264528)(766.38644531,358.90265076)
\curveto(766.24643942,359.13264486)(766.09143958,359.32764466)(765.92144531,359.48765076)
\curveto(765.75143992,359.64764434)(765.53144014,359.78264421)(765.26144531,359.89265076)
\curveto(765.19144048,359.91264408)(765.12144055,359.92764406)(765.05144531,359.93765076)
\curveto(764.98144069,359.95764403)(764.90644076,359.97764401)(764.82644531,359.99765076)
\curveto(764.74644092,360.01764397)(764.66144101,360.02764396)(764.57144531,360.02765076)
\lineto(764.31644531,360.02765076)
\curveto(764.28644138,360.00764398)(764.25144142,359.99764399)(764.21144531,359.99765076)
\curveto(764.1714415,360.00764398)(764.13644153,360.00764398)(764.10644531,359.99765076)
\lineto(763.86644531,359.93765076)
\curveto(763.79644187,359.92764406)(763.72644194,359.91264408)(763.65644531,359.89265076)
\curveto(763.3664423,359.77264422)(763.13144254,359.62264437)(762.95144531,359.44265076)
\curveto(762.78144289,359.26264473)(762.62644304,359.03764495)(762.48644531,358.76765076)
\curveto(762.45644321,358.71764527)(762.42644324,358.65264534)(762.39644531,358.57265076)
\curveto(762.3664433,358.50264549)(762.34144333,358.42264557)(762.32144531,358.33265076)
\curveto(762.30144337,358.24264575)(762.29644337,358.15764583)(762.30644531,358.07765076)
\curveto(762.31644335,357.99764599)(762.35144332,357.93764605)(762.41144531,357.89765076)
\curveto(762.49144318,357.83764615)(762.62644304,357.80764618)(762.81644531,357.80765076)
\curveto(763.01644265,357.81764617)(763.18644248,357.82264617)(763.32644531,357.82265076)
\lineto(765.60644531,357.82265076)
\curveto(765.75643991,357.82264617)(765.93643973,357.81764617)(766.14644531,357.80765076)
\curveto(766.35643931,357.80764618)(766.49643917,357.84764614)(766.56644531,357.92765076)
}
}
{
\newrgbcolor{curcolor}{0 0 0}
\pscustom[linestyle=none,fillstyle=solid,fillcolor=curcolor]
{
\newpath
\moveto(771.75808594,361.15265076)
\curveto(772.47808187,361.16264283)(773.08308127,361.07764291)(773.57308594,360.89765076)
\curveto(774.06308029,360.72764326)(774.44307991,360.42264357)(774.71308594,359.98265076)
\curveto(774.78307957,359.87264412)(774.83807951,359.75764423)(774.87808594,359.63765076)
\curveto(774.91807943,359.52764446)(774.95807939,359.40264459)(774.99808594,359.26265076)
\curveto(775.01807933,359.1926448)(775.02307933,359.11764487)(775.01308594,359.03765076)
\curveto(775.00307935,358.96764502)(774.98807936,358.91264508)(774.96808594,358.87265076)
\curveto(774.9480794,358.85264514)(774.92307943,358.83264516)(774.89308594,358.81265076)
\curveto(774.86307949,358.80264519)(774.83807951,358.7876452)(774.81808594,358.76765076)
\curveto(774.76807958,358.74764524)(774.71807963,358.74264525)(774.66808594,358.75265076)
\curveto(774.61807973,358.76264523)(774.56807978,358.76264523)(774.51808594,358.75265076)
\curveto(774.43807991,358.73264526)(774.33308002,358.72764526)(774.20308594,358.73765076)
\curveto(774.07308028,358.75764523)(773.98308037,358.78264521)(773.93308594,358.81265076)
\curveto(773.8530805,358.86264513)(773.79808055,358.92764506)(773.76808594,359.00765076)
\curveto(773.7480806,359.09764489)(773.71308064,359.18264481)(773.66308594,359.26265076)
\curveto(773.57308078,359.42264457)(773.4480809,359.56764442)(773.28808594,359.69765076)
\curveto(773.17808117,359.77764421)(773.05808129,359.83764415)(772.92808594,359.87765076)
\curveto(772.79808155,359.91764407)(772.65808169,359.95764403)(772.50808594,359.99765076)
\curveto(772.45808189,360.01764397)(772.40808194,360.02264397)(772.35808594,360.01265076)
\curveto(772.30808204,360.01264398)(772.25808209,360.01764397)(772.20808594,360.02765076)
\curveto(772.1480822,360.04764394)(772.07308228,360.05764393)(771.98308594,360.05765076)
\curveto(771.89308246,360.05764393)(771.81808253,360.04764394)(771.75808594,360.02765076)
\lineto(771.66808594,360.02765076)
\lineto(771.51808594,359.99765076)
\curveto(771.46808288,359.99764399)(771.41808293,359.992644)(771.36808594,359.98265076)
\curveto(771.10808324,359.92264407)(770.89308346,359.83764415)(770.72308594,359.72765076)
\curveto(770.5530838,359.61764437)(770.43808391,359.43264456)(770.37808594,359.17265076)
\curveto(770.35808399,359.10264489)(770.353084,359.03264496)(770.36308594,358.96265076)
\curveto(770.38308397,358.8926451)(770.40308395,358.83264516)(770.42308594,358.78265076)
\curveto(770.48308387,358.63264536)(770.5530838,358.52264547)(770.63308594,358.45265076)
\curveto(770.72308363,358.3926456)(770.83308352,358.32264567)(770.96308594,358.24265076)
\curveto(771.12308323,358.14264585)(771.30308305,358.06764592)(771.50308594,358.01765076)
\curveto(771.70308265,357.97764601)(771.90308245,357.92764606)(772.10308594,357.86765076)
\curveto(772.23308212,357.82764616)(772.36308199,357.79764619)(772.49308594,357.77765076)
\curveto(772.62308173,357.75764623)(772.7530816,357.72764626)(772.88308594,357.68765076)
\curveto(773.09308126,357.62764636)(773.29808105,357.56764642)(773.49808594,357.50765076)
\curveto(773.69808065,357.45764653)(773.89808045,357.3926466)(774.09808594,357.31265076)
\lineto(774.24808594,357.25265076)
\curveto(774.29808005,357.23264676)(774.34808,357.20764678)(774.39808594,357.17765076)
\curveto(774.59807975,357.05764693)(774.77307958,356.92264707)(774.92308594,356.77265076)
\curveto(775.07307928,356.62264737)(775.19807915,356.43264756)(775.29808594,356.20265076)
\curveto(775.31807903,356.13264786)(775.33807901,356.03764795)(775.35808594,355.91765076)
\curveto(775.37807897,355.84764814)(775.38807896,355.77264822)(775.38808594,355.69265076)
\curveto(775.39807895,355.62264837)(775.40307895,355.54264845)(775.40308594,355.45265076)
\lineto(775.40308594,355.30265076)
\curveto(775.38307897,355.23264876)(775.37307898,355.16264883)(775.37308594,355.09265076)
\curveto(775.37307898,355.02264897)(775.36307899,354.95264904)(775.34308594,354.88265076)
\curveto(775.31307904,354.77264922)(775.27807907,354.66764932)(775.23808594,354.56765076)
\curveto(775.19807915,354.46764952)(775.1530792,354.37764961)(775.10308594,354.29765076)
\curveto(774.94307941,354.03764995)(774.73807961,353.82765016)(774.48808594,353.66765076)
\curveto(774.23808011,353.51765047)(773.95808039,353.3876506)(773.64808594,353.27765076)
\curveto(773.55808079,353.24765074)(773.46308089,353.22765076)(773.36308594,353.21765076)
\curveto(773.27308108,353.19765079)(773.18308117,353.17265082)(773.09308594,353.14265076)
\curveto(772.99308136,353.12265087)(772.89308146,353.11265088)(772.79308594,353.11265076)
\curveto(772.69308166,353.11265088)(772.59308176,353.10265089)(772.49308594,353.08265076)
\lineto(772.34308594,353.08265076)
\curveto(772.29308206,353.07265092)(772.22308213,353.06765092)(772.13308594,353.06765076)
\curveto(772.04308231,353.06765092)(771.97308238,353.07265092)(771.92308594,353.08265076)
\lineto(771.75808594,353.08265076)
\curveto(771.69808265,353.10265089)(771.63308272,353.11265088)(771.56308594,353.11265076)
\curveto(771.49308286,353.10265089)(771.43308292,353.10765088)(771.38308594,353.12765076)
\curveto(771.33308302,353.13765085)(771.26808308,353.14265085)(771.18808594,353.14265076)
\lineto(770.94808594,353.20265076)
\curveto(770.87808347,353.21265078)(770.80308355,353.23265076)(770.72308594,353.26265076)
\curveto(770.41308394,353.36265063)(770.14308421,353.4876505)(769.91308594,353.63765076)
\curveto(769.68308467,353.7876502)(769.48308487,353.98265001)(769.31308594,354.22265076)
\curveto(769.22308513,354.35264964)(769.1480852,354.4876495)(769.08808594,354.62765076)
\curveto(769.02808532,354.76764922)(768.97308538,354.92264907)(768.92308594,355.09265076)
\curveto(768.90308545,355.15264884)(768.89308546,355.22264877)(768.89308594,355.30265076)
\curveto(768.90308545,355.3926486)(768.91808543,355.46264853)(768.93808594,355.51265076)
\curveto(768.96808538,355.55264844)(769.01808533,355.5926484)(769.08808594,355.63265076)
\curveto(769.13808521,355.65264834)(769.20808514,355.66264833)(769.29808594,355.66265076)
\curveto(769.38808496,355.67264832)(769.47808487,355.67264832)(769.56808594,355.66265076)
\curveto(769.65808469,355.65264834)(769.74308461,355.63764835)(769.82308594,355.61765076)
\curveto(769.91308444,355.60764838)(769.97308438,355.5926484)(770.00308594,355.57265076)
\curveto(770.07308428,355.52264847)(770.11808423,355.44764854)(770.13808594,355.34765076)
\curveto(770.16808418,355.25764873)(770.20308415,355.17264882)(770.24308594,355.09265076)
\curveto(770.34308401,354.87264912)(770.47808387,354.70264929)(770.64808594,354.58265076)
\curveto(770.76808358,354.4926495)(770.90308345,354.42264957)(771.05308594,354.37265076)
\curveto(771.20308315,354.32264967)(771.36308299,354.27264972)(771.53308594,354.22265076)
\lineto(771.84808594,354.17765076)
\lineto(771.93808594,354.17765076)
\curveto(772.00808234,354.15764983)(772.09808225,354.14764984)(772.20808594,354.14765076)
\curveto(772.32808202,354.14764984)(772.42808192,354.15764983)(772.50808594,354.17765076)
\curveto(772.57808177,354.17764981)(772.63308172,354.18264981)(772.67308594,354.19265076)
\curveto(772.73308162,354.20264979)(772.79308156,354.20764978)(772.85308594,354.20765076)
\curveto(772.91308144,354.21764977)(772.96808138,354.22764976)(773.01808594,354.23765076)
\curveto(773.30808104,354.31764967)(773.53808081,354.42264957)(773.70808594,354.55265076)
\curveto(773.87808047,354.68264931)(773.99808035,354.90264909)(774.06808594,355.21265076)
\curveto(774.08808026,355.26264873)(774.09308026,355.31764867)(774.08308594,355.37765076)
\curveto(774.07308028,355.43764855)(774.06308029,355.48264851)(774.05308594,355.51265076)
\curveto(774.00308035,355.70264829)(773.93308042,355.84264815)(773.84308594,355.93265076)
\curveto(773.7530806,356.03264796)(773.63808071,356.12264787)(773.49808594,356.20265076)
\curveto(773.40808094,356.26264773)(773.30808104,356.31264768)(773.19808594,356.35265076)
\lineto(772.86808594,356.47265076)
\curveto(772.83808151,356.48264751)(772.80808154,356.4876475)(772.77808594,356.48765076)
\curveto(772.75808159,356.4876475)(772.73308162,356.49764749)(772.70308594,356.51765076)
\curveto(772.36308199,356.62764736)(772.00808234,356.70764728)(771.63808594,356.75765076)
\curveto(771.27808307,356.81764717)(770.93808341,356.91264708)(770.61808594,357.04265076)
\curveto(770.51808383,357.08264691)(770.42308393,357.11764687)(770.33308594,357.14765076)
\curveto(770.24308411,357.17764681)(770.15808419,357.21764677)(770.07808594,357.26765076)
\curveto(769.88808446,357.37764661)(769.71308464,357.50264649)(769.55308594,357.64265076)
\curveto(769.39308496,357.78264621)(769.26808508,357.95764603)(769.17808594,358.16765076)
\curveto(769.1480852,358.23764575)(769.12308523,358.30764568)(769.10308594,358.37765076)
\curveto(769.09308526,358.44764554)(769.07808527,358.52264547)(769.05808594,358.60265076)
\curveto(769.02808532,358.72264527)(769.01808533,358.85764513)(769.02808594,359.00765076)
\curveto(769.03808531,359.16764482)(769.0530853,359.30264469)(769.07308594,359.41265076)
\curveto(769.09308526,359.46264453)(769.10308525,359.50264449)(769.10308594,359.53265076)
\curveto(769.11308524,359.57264442)(769.12808522,359.61264438)(769.14808594,359.65265076)
\curveto(769.23808511,359.88264411)(769.35808499,360.08264391)(769.50808594,360.25265076)
\curveto(769.66808468,360.42264357)(769.8480845,360.57264342)(770.04808594,360.70265076)
\curveto(770.19808415,360.7926432)(770.36308399,360.86264313)(770.54308594,360.91265076)
\curveto(770.72308363,360.97264302)(770.91308344,361.02764296)(771.11308594,361.07765076)
\curveto(771.18308317,361.0876429)(771.2480831,361.09764289)(771.30808594,361.10765076)
\curveto(771.37808297,361.11764287)(771.4530829,361.12764286)(771.53308594,361.13765076)
\curveto(771.56308279,361.14764284)(771.60308275,361.14764284)(771.65308594,361.13765076)
\curveto(771.70308265,361.12764286)(771.73808261,361.13264286)(771.75808594,361.15265076)
}
}
{
\newrgbcolor{curcolor}{0 0 0}
\pscustom[linestyle=none,fillstyle=solid,fillcolor=curcolor]
{
\newpath
\moveto(784.25308594,357.31265076)
\curveto(784.26307759,357.26264673)(784.26807758,357.19764679)(784.26808594,357.11765076)
\curveto(784.26807758,357.03764695)(784.26307759,356.97264702)(784.25308594,356.92265076)
\curveto(784.23307762,356.87264712)(784.22807762,356.82264717)(784.23808594,356.77265076)
\curveto(784.2480776,356.73264726)(784.2480776,356.6926473)(784.23808594,356.65265076)
\curveto(784.23807761,356.58264741)(784.23307762,356.52764746)(784.22308594,356.48765076)
\curveto(784.20307765,356.39764759)(784.18807766,356.30764768)(784.17808594,356.21765076)
\curveto(784.17807767,356.12764786)(784.16807768,356.03764795)(784.14808594,355.94765076)
\lineto(784.08808594,355.70765076)
\curveto(784.06807778,355.63764835)(784.04307781,355.56264843)(784.01308594,355.48265076)
\curveto(783.89307796,355.11264888)(783.72807812,354.77764921)(783.51808594,354.47765076)
\curveto(783.45807839,354.3876496)(783.39307846,354.29764969)(783.32308594,354.20765076)
\curveto(783.2530786,354.12764986)(783.17807867,354.05264994)(783.09808594,353.98265076)
\lineto(783.02308594,353.90765076)
\curveto(782.9530789,353.85765013)(782.88807896,353.80765018)(782.82808594,353.75765076)
\curveto(782.76807908,353.70765028)(782.69807915,353.65765033)(782.61808594,353.60765076)
\curveto(782.50807934,353.52765046)(782.38307947,353.45765053)(782.24308594,353.39765076)
\curveto(782.11307974,353.34765064)(781.97807987,353.29765069)(781.83808594,353.24765076)
\curveto(781.75808009,353.22765076)(781.67808017,353.21265078)(781.59808594,353.20265076)
\curveto(781.52808032,353.1926508)(781.4530804,353.17765081)(781.37308594,353.15765076)
\lineto(781.31308594,353.15765076)
\curveto(781.30308055,353.14765084)(781.28808056,353.14265085)(781.26808594,353.14265076)
\curveto(781.17808067,353.12265087)(781.04308081,353.11265088)(780.86308594,353.11265076)
\curveto(780.69308116,353.10265089)(780.55808129,353.10765088)(780.45808594,353.12765076)
\lineto(780.38308594,353.12765076)
\curveto(780.31308154,353.13765085)(780.2480816,353.14765084)(780.18808594,353.15765076)
\curveto(780.12808172,353.15765083)(780.06808178,353.16765082)(780.00808594,353.18765076)
\curveto(779.83808201,353.23765075)(779.67808217,353.28265071)(779.52808594,353.32265076)
\curveto(779.37808247,353.36265063)(779.23808261,353.42265057)(779.10808594,353.50265076)
\curveto(778.9480829,353.5926504)(778.80808304,353.6876503)(778.68808594,353.78765076)
\curveto(778.6480832,353.81765017)(778.58808326,353.85765013)(778.50808594,353.90765076)
\curveto(778.42808342,353.96765002)(778.3530835,353.97265002)(778.28308594,353.92265076)
\curveto(778.24308361,353.8926501)(778.22308363,353.85265014)(778.22308594,353.80265076)
\curveto(778.22308363,353.75265024)(778.21308364,353.69765029)(778.19308594,353.63765076)
\curveto(778.18308367,353.60765038)(778.18308367,353.57265042)(778.19308594,353.53265076)
\curveto(778.20308365,353.50265049)(778.20308365,353.46765052)(778.19308594,353.42765076)
\curveto(778.17308368,353.36765062)(778.16308369,353.30265069)(778.16308594,353.23265076)
\curveto(778.17308368,353.15265084)(778.17808367,353.08265091)(778.17808594,353.02265076)
\lineto(778.17808594,351.22265076)
\lineto(778.17808594,350.78765076)
\curveto(778.17808367,350.63765335)(778.1480837,350.52265347)(778.08808594,350.44265076)
\curveto(778.03808381,350.37265362)(777.95808389,350.33765365)(777.84808594,350.33765076)
\curveto(777.73808411,350.32765366)(777.62808422,350.32265367)(777.51808594,350.32265076)
\lineto(777.27808594,350.32265076)
\curveto(777.20808464,350.34265365)(777.1480847,350.36265363)(777.09808594,350.38265076)
\curveto(777.05808479,350.40265359)(777.02308483,350.43765355)(776.99308594,350.48765076)
\curveto(776.94308491,350.55765343)(776.91808493,350.66765332)(776.91808594,350.81765076)
\curveto(776.92808492,350.96765302)(776.93308492,351.09765289)(776.93308594,351.20765076)
\lineto(776.93308594,360.20765076)
\lineto(776.93308594,360.56765076)
\curveto(776.94308491,360.69764329)(776.97308488,360.80264319)(777.02308594,360.88265076)
\curveto(777.0530848,360.92264307)(777.11808473,360.95264304)(777.21808594,360.97265076)
\curveto(777.32808452,361.00264299)(777.44308441,361.01264298)(777.56308594,361.00265076)
\curveto(777.68308417,361.00264299)(777.79308406,360.987643)(777.89308594,360.95765076)
\curveto(778.00308385,360.93764305)(778.07308378,360.90764308)(778.10308594,360.86765076)
\curveto(778.14308371,360.81764317)(778.16308369,360.75764323)(778.16308594,360.68765076)
\curveto(778.17308368,360.61764337)(778.19308366,360.54764344)(778.22308594,360.47765076)
\curveto(778.24308361,360.44764354)(778.25808359,360.42264357)(778.26808594,360.40265076)
\curveto(778.28808356,360.3926436)(778.30808354,360.37764361)(778.32808594,360.35765076)
\curveto(778.43808341,360.34764364)(778.52808332,360.38264361)(778.59808594,360.46265076)
\curveto(778.67808317,360.54264345)(778.7530831,360.60764338)(778.82308594,360.65765076)
\curveto(779.08308277,360.83764315)(779.39308246,360.97764301)(779.75308594,361.07765076)
\curveto(779.84308201,361.09764289)(779.93308192,361.11264288)(780.02308594,361.12265076)
\curveto(780.12308173,361.13264286)(780.22308163,361.14764284)(780.32308594,361.16765076)
\curveto(780.36308149,361.17764281)(780.41308144,361.17764281)(780.47308594,361.16765076)
\curveto(780.53308132,361.15764283)(780.57308128,361.16264283)(780.59308594,361.18265076)
\curveto(781.02308083,361.1926428)(781.40308045,361.14764284)(781.73308594,361.04765076)
\curveto(782.06307979,360.95764303)(782.35807949,360.82764316)(782.61808594,360.65765076)
\lineto(782.76808594,360.53765076)
\curveto(782.81807903,360.50764348)(782.86807898,360.47264352)(782.91808594,360.43265076)
\curveto(782.93807891,360.41264358)(782.9530789,360.3926436)(782.96308594,360.37265076)
\curveto(782.98307887,360.36264363)(783.00307885,360.34764364)(783.02308594,360.32765076)
\curveto(783.07307878,360.27764371)(783.12807872,360.22264377)(783.18808594,360.16265076)
\curveto(783.2480786,360.10264389)(783.30307855,360.04264395)(783.35308594,359.98265076)
\curveto(783.47307838,359.81264418)(783.59807825,359.62764436)(783.72808594,359.42765076)
\curveto(783.80807804,359.29764469)(783.87307798,359.15264484)(783.92308594,358.99265076)
\curveto(783.98307787,358.83264516)(784.03807781,358.67264532)(784.08808594,358.51265076)
\curveto(784.10807774,358.43264556)(784.12307773,358.34764564)(784.13308594,358.25765076)
\curveto(784.1530777,358.16764582)(784.17307768,358.08264591)(784.19308594,358.00265076)
\lineto(784.19308594,357.88265076)
\curveto(784.20307765,357.85264614)(784.20807764,357.82264617)(784.20808594,357.79265076)
\curveto(784.22807762,357.74264625)(784.23307762,357.6876463)(784.22308594,357.62765076)
\curveto(784.22307763,357.56764642)(784.23307762,357.51264648)(784.25308594,357.46265076)
\lineto(784.25308594,357.31265076)
\moveto(782.91808594,356.90765076)
\curveto(782.93807891,356.95764703)(782.94307891,357.01764697)(782.93308594,357.08765076)
\curveto(782.92307893,357.16764682)(782.91807893,357.23764675)(782.91808594,357.29765076)
\curveto(782.91807893,357.46764652)(782.90807894,357.62764636)(782.88808594,357.77765076)
\curveto(782.87807897,357.92764606)(782.848079,358.07264592)(782.79808594,358.21265076)
\lineto(782.73808594,358.39265076)
\curveto(782.72807912,358.46264553)(782.70807914,358.52764546)(782.67808594,358.58765076)
\curveto(782.56807928,358.85764513)(782.39307946,359.11764487)(782.15308594,359.36765076)
\curveto(781.92307993,359.61764437)(781.70308015,359.7876442)(781.49308594,359.87765076)
\curveto(781.41308044,359.91764407)(781.32808052,359.94764404)(781.23808594,359.96765076)
\curveto(781.15808069,359.987644)(781.07308078,360.01264398)(780.98308594,360.04265076)
\curveto(780.89308096,360.06264393)(780.78808106,360.07264392)(780.66808594,360.07265076)
\lineto(780.33808594,360.07265076)
\curveto(780.31808153,360.05264394)(780.27808157,360.04264395)(780.21808594,360.04265076)
\curveto(780.16808168,360.05264394)(780.12308173,360.05264394)(780.08308594,360.04265076)
\lineto(779.81308594,359.98265076)
\curveto(779.73308212,359.96264403)(779.6530822,359.93264406)(779.57308594,359.89265076)
\curveto(779.2530826,359.75264424)(778.98808286,359.54764444)(778.77808594,359.27765076)
\curveto(778.57808327,359.01764497)(778.42308343,358.71264528)(778.31308594,358.36265076)
\curveto(778.27308358,358.25264574)(778.24308361,358.14264585)(778.22308594,358.03265076)
\curveto(778.21308364,357.92264607)(778.19808365,357.81264618)(778.17808594,357.70265076)
\curveto(778.16808368,357.66264633)(778.16308369,357.62264637)(778.16308594,357.58265076)
\curveto(778.16308369,357.55264644)(778.15808369,357.51764647)(778.14808594,357.47765076)
\lineto(778.14808594,357.35765076)
\curveto(778.13808371,357.30764668)(778.13308372,357.23264676)(778.13308594,357.13265076)
\curveto(778.13308372,357.04264695)(778.13808371,356.97264702)(778.14808594,356.92265076)
\lineto(778.14808594,356.80265076)
\curveto(778.15808369,356.76264723)(778.16308369,356.72264727)(778.16308594,356.68265076)
\curveto(778.16308369,356.64264735)(778.16808368,356.60764738)(778.17808594,356.57765076)
\curveto(778.18808366,356.54764744)(778.19308366,356.51764747)(778.19308594,356.48765076)
\curveto(778.19308366,356.45764753)(778.19808365,356.42264757)(778.20808594,356.38265076)
\curveto(778.22808362,356.30264769)(778.24308361,356.22264777)(778.25308594,356.14265076)
\lineto(778.31308594,355.90265076)
\curveto(778.42308343,355.56264843)(778.57308328,355.26264873)(778.76308594,355.00265076)
\curveto(778.96308289,354.75264924)(779.22308263,354.55764943)(779.54308594,354.41765076)
\curveto(779.73308212,354.33764965)(779.92808192,354.27764971)(780.12808594,354.23765076)
\curveto(780.16808168,354.21764977)(780.20808164,354.20764978)(780.24808594,354.20765076)
\curveto(780.28808156,354.21764977)(780.32808152,354.21764977)(780.36808594,354.20765076)
\lineto(780.48808594,354.20765076)
\curveto(780.55808129,354.1876498)(780.62808122,354.1876498)(780.69808594,354.20765076)
\lineto(780.81808594,354.20765076)
\curveto(780.92808092,354.22764976)(781.03308082,354.24264975)(781.13308594,354.25265076)
\curveto(781.23308062,354.26264973)(781.33308052,354.2876497)(781.43308594,354.32765076)
\curveto(781.74308011,354.45764953)(781.99307986,354.62764936)(782.18308594,354.83765076)
\curveto(782.38307947,355.05764893)(782.5480793,355.32264867)(782.67808594,355.63265076)
\curveto(782.72807912,355.77264822)(782.76307909,355.91264808)(782.78308594,356.05265076)
\curveto(782.81307904,356.20264779)(782.848079,356.35764763)(782.88808594,356.51765076)
\curveto(782.89807895,356.56764742)(782.90307895,356.61264738)(782.90308594,356.65265076)
\curveto(782.90307895,356.6926473)(782.90807894,356.73764725)(782.91808594,356.78765076)
\lineto(782.91808594,356.90765076)
}
}
{
\newrgbcolor{curcolor}{0 0 0}
\pscustom[linestyle=none,fillstyle=solid,fillcolor=curcolor]
{
\newpath
\moveto(792.61933594,353.80265076)
\curveto(792.64932811,353.64265035)(792.63432812,353.50765048)(792.57433594,353.39765076)
\curveto(792.51432824,353.29765069)(792.43432832,353.22265077)(792.33433594,353.17265076)
\curveto(792.28432847,353.15265084)(792.22932853,353.14265085)(792.16933594,353.14265076)
\curveto(792.11932864,353.14265085)(792.06432869,353.13265086)(792.00433594,353.11265076)
\curveto(791.78432897,353.06265093)(791.56432919,353.07765091)(791.34433594,353.15765076)
\curveto(791.13432962,353.22765076)(790.98932977,353.31765067)(790.90933594,353.42765076)
\curveto(790.8593299,353.49765049)(790.81432994,353.57765041)(790.77433594,353.66765076)
\curveto(790.73433002,353.76765022)(790.68433007,353.84765014)(790.62433594,353.90765076)
\curveto(790.60433015,353.92765006)(790.57933018,353.94765004)(790.54933594,353.96765076)
\curveto(790.52933023,353.98765)(790.49933026,353.99265)(790.45933594,353.98265076)
\curveto(790.34933041,353.95265004)(790.24433051,353.89765009)(790.14433594,353.81765076)
\curveto(790.0543307,353.73765025)(789.96433079,353.66765032)(789.87433594,353.60765076)
\curveto(789.74433101,353.52765046)(789.60433115,353.45265054)(789.45433594,353.38265076)
\curveto(789.30433145,353.32265067)(789.14433161,353.26765072)(788.97433594,353.21765076)
\curveto(788.87433188,353.1876508)(788.76433199,353.16765082)(788.64433594,353.15765076)
\curveto(788.53433222,353.14765084)(788.42433233,353.13265086)(788.31433594,353.11265076)
\curveto(788.26433249,353.10265089)(788.21933254,353.09765089)(788.17933594,353.09765076)
\lineto(788.07433594,353.09765076)
\curveto(787.96433279,353.07765091)(787.8593329,353.07765091)(787.75933594,353.09765076)
\lineto(787.62433594,353.09765076)
\curveto(787.57433318,353.10765088)(787.52433323,353.11265088)(787.47433594,353.11265076)
\curveto(787.42433333,353.11265088)(787.37933338,353.12265087)(787.33933594,353.14265076)
\curveto(787.29933346,353.15265084)(787.26433349,353.15765083)(787.23433594,353.15765076)
\curveto(787.21433354,353.14765084)(787.18933357,353.14765084)(787.15933594,353.15765076)
\lineto(786.91933594,353.21765076)
\curveto(786.83933392,353.22765076)(786.76433399,353.24765074)(786.69433594,353.27765076)
\curveto(786.39433436,353.40765058)(786.14933461,353.55265044)(785.95933594,353.71265076)
\curveto(785.77933498,353.88265011)(785.62933513,354.11764987)(785.50933594,354.41765076)
\curveto(785.41933534,354.63764935)(785.37433538,354.90264909)(785.37433594,355.21265076)
\lineto(785.37433594,355.52765076)
\curveto(785.38433537,355.57764841)(785.38933537,355.62764836)(785.38933594,355.67765076)
\lineto(785.41933594,355.85765076)
\lineto(785.53933594,356.18765076)
\curveto(785.57933518,356.29764769)(785.62933513,356.39764759)(785.68933594,356.48765076)
\curveto(785.86933489,356.77764721)(786.11433464,356.992647)(786.42433594,357.13265076)
\curveto(786.73433402,357.27264672)(787.07433368,357.39764659)(787.44433594,357.50765076)
\curveto(787.58433317,357.54764644)(787.72933303,357.57764641)(787.87933594,357.59765076)
\curveto(788.02933273,357.61764637)(788.17933258,357.64264635)(788.32933594,357.67265076)
\curveto(788.39933236,357.6926463)(788.46433229,357.70264629)(788.52433594,357.70265076)
\curveto(788.59433216,357.70264629)(788.66933209,357.71264628)(788.74933594,357.73265076)
\curveto(788.81933194,357.75264624)(788.88933187,357.76264623)(788.95933594,357.76265076)
\curveto(789.02933173,357.77264622)(789.10433165,357.7876462)(789.18433594,357.80765076)
\curveto(789.43433132,357.86764612)(789.66933109,357.91764607)(789.88933594,357.95765076)
\curveto(790.10933065,358.00764598)(790.28433047,358.12264587)(790.41433594,358.30265076)
\curveto(790.47433028,358.38264561)(790.52433023,358.48264551)(790.56433594,358.60265076)
\curveto(790.60433015,358.73264526)(790.60433015,358.87264512)(790.56433594,359.02265076)
\curveto(790.50433025,359.26264473)(790.41433034,359.45264454)(790.29433594,359.59265076)
\curveto(790.18433057,359.73264426)(790.02433073,359.84264415)(789.81433594,359.92265076)
\curveto(789.69433106,359.97264402)(789.54933121,360.00764398)(789.37933594,360.02765076)
\curveto(789.21933154,360.04764394)(789.04933171,360.05764393)(788.86933594,360.05765076)
\curveto(788.68933207,360.05764393)(788.51433224,360.04764394)(788.34433594,360.02765076)
\curveto(788.17433258,360.00764398)(788.02933273,359.97764401)(787.90933594,359.93765076)
\curveto(787.73933302,359.87764411)(787.57433318,359.7926442)(787.41433594,359.68265076)
\curveto(787.33433342,359.62264437)(787.2593335,359.54264445)(787.18933594,359.44265076)
\curveto(787.12933363,359.35264464)(787.07433368,359.25264474)(787.02433594,359.14265076)
\curveto(786.99433376,359.06264493)(786.96433379,358.97764501)(786.93433594,358.88765076)
\curveto(786.91433384,358.79764519)(786.86933389,358.72764526)(786.79933594,358.67765076)
\curveto(786.759334,358.64764534)(786.68933407,358.62264537)(786.58933594,358.60265076)
\curveto(786.49933426,358.5926454)(786.40433435,358.5876454)(786.30433594,358.58765076)
\curveto(786.20433455,358.5876454)(786.10433465,358.5926454)(786.00433594,358.60265076)
\curveto(785.91433484,358.62264537)(785.84933491,358.64764534)(785.80933594,358.67765076)
\curveto(785.76933499,358.70764528)(785.73933502,358.75764523)(785.71933594,358.82765076)
\curveto(785.69933506,358.89764509)(785.69933506,358.97264502)(785.71933594,359.05265076)
\curveto(785.74933501,359.18264481)(785.77933498,359.30264469)(785.80933594,359.41265076)
\curveto(785.84933491,359.53264446)(785.89433486,359.64764434)(785.94433594,359.75765076)
\curveto(786.13433462,360.10764388)(786.37433438,360.37764361)(786.66433594,360.56765076)
\curveto(786.9543338,360.76764322)(787.31433344,360.92764306)(787.74433594,361.04765076)
\curveto(787.84433291,361.06764292)(787.94433281,361.08264291)(788.04433594,361.09265076)
\curveto(788.1543326,361.10264289)(788.26433249,361.11764287)(788.37433594,361.13765076)
\curveto(788.41433234,361.14764284)(788.47933228,361.14764284)(788.56933594,361.13765076)
\curveto(788.6593321,361.13764285)(788.71433204,361.14764284)(788.73433594,361.16765076)
\curveto(789.43433132,361.17764281)(790.04433071,361.09764289)(790.56433594,360.92765076)
\curveto(791.08432967,360.75764323)(791.44932931,360.43264356)(791.65933594,359.95265076)
\curveto(791.74932901,359.75264424)(791.79932896,359.51764447)(791.80933594,359.24765076)
\curveto(791.82932893,358.987645)(791.83932892,358.71264528)(791.83933594,358.42265076)
\lineto(791.83933594,355.10765076)
\curveto(791.83932892,354.96764902)(791.84432891,354.83264916)(791.85433594,354.70265076)
\curveto(791.86432889,354.57264942)(791.89432886,354.46764952)(791.94433594,354.38765076)
\curveto(791.99432876,354.31764967)(792.0593287,354.26764972)(792.13933594,354.23765076)
\curveto(792.22932853,354.19764979)(792.31432844,354.16764982)(792.39433594,354.14765076)
\curveto(792.47432828,354.13764985)(792.53432822,354.0926499)(792.57433594,354.01265076)
\curveto(792.59432816,353.98265001)(792.60432815,353.95265004)(792.60433594,353.92265076)
\curveto(792.60432815,353.8926501)(792.60932815,353.85265014)(792.61933594,353.80265076)
\moveto(790.47433594,355.46765076)
\curveto(790.53433022,355.60764838)(790.56433019,355.76764822)(790.56433594,355.94765076)
\curveto(790.57433018,356.13764785)(790.57933018,356.33264766)(790.57933594,356.53265076)
\curveto(790.57933018,356.64264735)(790.57433018,356.74264725)(790.56433594,356.83265076)
\curveto(790.5543302,356.92264707)(790.51433024,356.992647)(790.44433594,357.04265076)
\curveto(790.41433034,357.06264693)(790.34433041,357.07264692)(790.23433594,357.07265076)
\curveto(790.21433054,357.05264694)(790.17933058,357.04264695)(790.12933594,357.04265076)
\curveto(790.07933068,357.04264695)(790.03433072,357.03264696)(789.99433594,357.01265076)
\curveto(789.91433084,356.992647)(789.82433093,356.97264702)(789.72433594,356.95265076)
\lineto(789.42433594,356.89265076)
\curveto(789.39433136,356.8926471)(789.3593314,356.8876471)(789.31933594,356.87765076)
\lineto(789.21433594,356.87765076)
\curveto(789.06433169,356.83764715)(788.89933186,356.81264718)(788.71933594,356.80265076)
\curveto(788.54933221,356.80264719)(788.38933237,356.78264721)(788.23933594,356.74265076)
\curveto(788.1593326,356.72264727)(788.08433267,356.70264729)(788.01433594,356.68265076)
\curveto(787.9543328,356.67264732)(787.88433287,356.65764733)(787.80433594,356.63765076)
\curveto(787.64433311,356.5876474)(787.49433326,356.52264747)(787.35433594,356.44265076)
\curveto(787.21433354,356.37264762)(787.09433366,356.28264771)(786.99433594,356.17265076)
\curveto(786.89433386,356.06264793)(786.81933394,355.92764806)(786.76933594,355.76765076)
\curveto(786.71933404,355.61764837)(786.69933406,355.43264856)(786.70933594,355.21265076)
\curveto(786.70933405,355.11264888)(786.72433403,355.01764897)(786.75433594,354.92765076)
\curveto(786.79433396,354.84764914)(786.83933392,354.77264922)(786.88933594,354.70265076)
\curveto(786.96933379,354.5926494)(787.07433368,354.49764949)(787.20433594,354.41765076)
\curveto(787.33433342,354.34764964)(787.47433328,354.2876497)(787.62433594,354.23765076)
\curveto(787.67433308,354.22764976)(787.72433303,354.22264977)(787.77433594,354.22265076)
\curveto(787.82433293,354.22264977)(787.87433288,354.21764977)(787.92433594,354.20765076)
\curveto(787.99433276,354.1876498)(788.07933268,354.17264982)(788.17933594,354.16265076)
\curveto(788.28933247,354.16264983)(788.37933238,354.17264982)(788.44933594,354.19265076)
\curveto(788.50933225,354.21264978)(788.56933219,354.21764977)(788.62933594,354.20765076)
\curveto(788.68933207,354.20764978)(788.74933201,354.21764977)(788.80933594,354.23765076)
\curveto(788.88933187,354.25764973)(788.96433179,354.27264972)(789.03433594,354.28265076)
\curveto(789.11433164,354.2926497)(789.18933157,354.31264968)(789.25933594,354.34265076)
\curveto(789.54933121,354.46264953)(789.79433096,354.60764938)(789.99433594,354.77765076)
\curveto(790.20433055,354.94764904)(790.36433039,355.17764881)(790.47433594,355.46765076)
}
}
{
\newrgbcolor{curcolor}{0 0 0}
\pscustom[linestyle=none,fillstyle=solid,fillcolor=curcolor]
{
\newpath
\moveto(796.92597656,361.15265076)
\curveto(797.66597177,361.16264283)(798.28097116,361.05264294)(798.77097656,360.82265076)
\curveto(799.27097017,360.60264339)(799.66596977,360.26764372)(799.95597656,359.81765076)
\curveto(800.08596935,359.61764437)(800.19596924,359.37264462)(800.28597656,359.08265076)
\curveto(800.30596913,359.03264496)(800.32096912,358.96764502)(800.33097656,358.88765076)
\curveto(800.3409691,358.80764518)(800.3359691,358.73764525)(800.31597656,358.67765076)
\curveto(800.28596915,358.62764536)(800.2359692,358.58264541)(800.16597656,358.54265076)
\curveto(800.1359693,358.52264547)(800.10596933,358.51264548)(800.07597656,358.51265076)
\curveto(800.04596939,358.52264547)(800.01096943,358.52264547)(799.97097656,358.51265076)
\curveto(799.93096951,358.50264549)(799.89096955,358.49764549)(799.85097656,358.49765076)
\curveto(799.81096963,358.50764548)(799.77096967,358.51264548)(799.73097656,358.51265076)
\lineto(799.41597656,358.51265076)
\curveto(799.31597012,358.52264547)(799.23097021,358.55264544)(799.16097656,358.60265076)
\curveto(799.08097036,358.66264533)(799.02597041,358.74764524)(798.99597656,358.85765076)
\curveto(798.96597047,358.96764502)(798.92597051,359.06264493)(798.87597656,359.14265076)
\curveto(798.72597071,359.40264459)(798.53097091,359.60764438)(798.29097656,359.75765076)
\curveto(798.21097123,359.80764418)(798.12597131,359.84764414)(798.03597656,359.87765076)
\curveto(797.94597149,359.91764407)(797.85097159,359.95264404)(797.75097656,359.98265076)
\curveto(797.61097183,360.02264397)(797.42597201,360.04264395)(797.19597656,360.04265076)
\curveto(796.96597247,360.05264394)(796.77597266,360.03264396)(796.62597656,359.98265076)
\curveto(796.55597288,359.96264403)(796.49097295,359.94764404)(796.43097656,359.93765076)
\curveto(796.37097307,359.92764406)(796.30597313,359.91264408)(796.23597656,359.89265076)
\curveto(795.97597346,359.78264421)(795.74597369,359.63264436)(795.54597656,359.44265076)
\curveto(795.34597409,359.25264474)(795.19097425,359.02764496)(795.08097656,358.76765076)
\curveto(795.0409744,358.67764531)(795.00597443,358.58264541)(794.97597656,358.48265076)
\curveto(794.94597449,358.3926456)(794.91597452,358.2926457)(794.88597656,358.18265076)
\lineto(794.79597656,357.77765076)
\curveto(794.78597465,357.72764626)(794.78097466,357.67264632)(794.78097656,357.61265076)
\curveto(794.79097465,357.55264644)(794.78597465,357.49764649)(794.76597656,357.44765076)
\lineto(794.76597656,357.32765076)
\curveto(794.75597468,357.2876467)(794.74597469,357.22264677)(794.73597656,357.13265076)
\curveto(794.7359747,357.04264695)(794.74597469,356.97764701)(794.76597656,356.93765076)
\curveto(794.77597466,356.8876471)(794.77597466,356.83764715)(794.76597656,356.78765076)
\curveto(794.75597468,356.73764725)(794.75597468,356.6876473)(794.76597656,356.63765076)
\curveto(794.77597466,356.59764739)(794.78097466,356.52764746)(794.78097656,356.42765076)
\curveto(794.80097464,356.34764764)(794.81597462,356.26264773)(794.82597656,356.17265076)
\curveto(794.84597459,356.08264791)(794.86597457,355.99764799)(794.88597656,355.91765076)
\curveto(794.99597444,355.59764839)(795.12097432,355.31764867)(795.26097656,355.07765076)
\curveto(795.41097403,354.84764914)(795.61597382,354.64764934)(795.87597656,354.47765076)
\curveto(795.96597347,354.42764956)(796.05597338,354.38264961)(796.14597656,354.34265076)
\curveto(796.24597319,354.30264969)(796.35097309,354.26264973)(796.46097656,354.22265076)
\curveto(796.51097293,354.21264978)(796.55097289,354.20764978)(796.58097656,354.20765076)
\curveto(796.61097283,354.20764978)(796.65097279,354.20264979)(796.70097656,354.19265076)
\curveto(796.73097271,354.18264981)(796.78097266,354.17764981)(796.85097656,354.17765076)
\lineto(797.01597656,354.17765076)
\curveto(797.01597242,354.16764982)(797.0359724,354.16264983)(797.07597656,354.16265076)
\curveto(797.09597234,354.17264982)(797.12097232,354.17264982)(797.15097656,354.16265076)
\curveto(797.18097226,354.16264983)(797.21097223,354.16764982)(797.24097656,354.17765076)
\curveto(797.31097213,354.19764979)(797.37597206,354.20264979)(797.43597656,354.19265076)
\curveto(797.50597193,354.1926498)(797.57597186,354.20264979)(797.64597656,354.22265076)
\curveto(797.90597153,354.30264969)(798.13097131,354.40264959)(798.32097656,354.52265076)
\curveto(798.51097093,354.65264934)(798.67097077,354.81764917)(798.80097656,355.01765076)
\curveto(798.85097059,355.09764889)(798.89597054,355.18264881)(798.93597656,355.27265076)
\lineto(799.05597656,355.54265076)
\curveto(799.07597036,355.62264837)(799.09597034,355.69764829)(799.11597656,355.76765076)
\curveto(799.14597029,355.84764814)(799.19597024,355.91264808)(799.26597656,355.96265076)
\curveto(799.29597014,355.992648)(799.35597008,356.01264798)(799.44597656,356.02265076)
\curveto(799.5359699,356.04264795)(799.63096981,356.05264794)(799.73097656,356.05265076)
\curveto(799.8409696,356.06264793)(799.9409695,356.06264793)(800.03097656,356.05265076)
\curveto(800.13096931,356.04264795)(800.20096924,356.02264797)(800.24097656,355.99265076)
\curveto(800.30096914,355.95264804)(800.3359691,355.8926481)(800.34597656,355.81265076)
\curveto(800.36596907,355.73264826)(800.36596907,355.64764834)(800.34597656,355.55765076)
\curveto(800.29596914,355.40764858)(800.24596919,355.26264873)(800.19597656,355.12265076)
\curveto(800.15596928,354.992649)(800.10096934,354.86264913)(800.03097656,354.73265076)
\curveto(799.88096956,354.43264956)(799.69096975,354.16764982)(799.46097656,353.93765076)
\curveto(799.2409702,353.70765028)(798.97097047,353.52265047)(798.65097656,353.38265076)
\curveto(798.57097087,353.34265065)(798.48597095,353.30765068)(798.39597656,353.27765076)
\curveto(798.30597113,353.25765073)(798.21097123,353.23265076)(798.11097656,353.20265076)
\curveto(798.00097144,353.16265083)(797.89097155,353.14265085)(797.78097656,353.14265076)
\curveto(797.67097177,353.13265086)(797.56097188,353.11765087)(797.45097656,353.09765076)
\curveto(797.41097203,353.07765091)(797.37097207,353.07265092)(797.33097656,353.08265076)
\curveto(797.29097215,353.0926509)(797.25097219,353.0926509)(797.21097656,353.08265076)
\lineto(797.07597656,353.08265076)
\lineto(796.83597656,353.08265076)
\curveto(796.76597267,353.07265092)(796.70097274,353.07765091)(796.64097656,353.09765076)
\lineto(796.56597656,353.09765076)
\lineto(796.20597656,353.14265076)
\curveto(796.07597336,353.18265081)(795.95097349,353.21765077)(795.83097656,353.24765076)
\curveto(795.71097373,353.27765071)(795.59597384,353.31765067)(795.48597656,353.36765076)
\curveto(795.12597431,353.52765046)(794.82597461,353.71765027)(794.58597656,353.93765076)
\curveto(794.35597508,354.15764983)(794.1409753,354.42764956)(793.94097656,354.74765076)
\curveto(793.89097555,354.82764916)(793.84597559,354.91764907)(793.80597656,355.01765076)
\lineto(793.68597656,355.31765076)
\curveto(793.6359758,355.42764856)(793.60097584,355.54264845)(793.58097656,355.66265076)
\curveto(793.56097588,355.78264821)(793.5359759,355.90264809)(793.50597656,356.02265076)
\curveto(793.49597594,356.06264793)(793.49097595,356.10264789)(793.49097656,356.14265076)
\curveto(793.49097595,356.18264781)(793.48597595,356.22264777)(793.47597656,356.26265076)
\curveto(793.45597598,356.32264767)(793.44597599,356.3876476)(793.44597656,356.45765076)
\curveto(793.45597598,356.52764746)(793.45097599,356.5926474)(793.43097656,356.65265076)
\lineto(793.43097656,356.80265076)
\curveto(793.42097602,356.85264714)(793.41597602,356.92264707)(793.41597656,357.01265076)
\curveto(793.41597602,357.10264689)(793.42097602,357.17264682)(793.43097656,357.22265076)
\curveto(793.440976,357.27264672)(793.440976,357.31764667)(793.43097656,357.35765076)
\curveto(793.43097601,357.39764659)(793.435976,357.43764655)(793.44597656,357.47765076)
\curveto(793.46597597,357.54764644)(793.47097597,357.61764637)(793.46097656,357.68765076)
\curveto(793.46097598,357.75764623)(793.47097597,357.82264617)(793.49097656,357.88265076)
\curveto(793.53097591,358.05264594)(793.56597587,358.22264577)(793.59597656,358.39265076)
\curveto(793.62597581,358.56264543)(793.67097577,358.72264527)(793.73097656,358.87265076)
\curveto(793.9409755,359.3926446)(794.19597524,359.81264418)(794.49597656,360.13265076)
\curveto(794.79597464,360.45264354)(795.20597423,360.71764327)(795.72597656,360.92765076)
\curveto(795.8359736,360.97764301)(795.95597348,361.01264298)(796.08597656,361.03265076)
\curveto(796.21597322,361.05264294)(796.35097309,361.07764291)(796.49097656,361.10765076)
\curveto(796.56097288,361.11764287)(796.63097281,361.12264287)(796.70097656,361.12265076)
\curveto(796.77097267,361.13264286)(796.84597259,361.14264285)(796.92597656,361.15265076)
}
}
{
\newrgbcolor{curcolor}{0 0 0}
\pscustom[linestyle=none,fillstyle=solid,fillcolor=curcolor]
{
\newpath
\moveto(802.13261719,362.47265076)
\curveto(802.05261607,362.53264146)(802.00761611,362.63764135)(801.99761719,362.78765076)
\lineto(801.99761719,363.25265076)
\lineto(801.99761719,363.50765076)
\curveto(801.99761612,363.59764039)(802.01261611,363.67264032)(802.04261719,363.73265076)
\curveto(802.08261604,363.81264018)(802.16261596,363.87264012)(802.28261719,363.91265076)
\curveto(802.30261582,363.92264007)(802.3226158,363.92264007)(802.34261719,363.91265076)
\curveto(802.37261575,363.91264008)(802.39761572,363.91764007)(802.41761719,363.92765076)
\curveto(802.58761553,363.92764006)(802.74761537,363.92264007)(802.89761719,363.91265076)
\curveto(803.04761507,363.90264009)(803.14761497,363.84264015)(803.19761719,363.73265076)
\curveto(803.22761489,363.67264032)(803.24261488,363.59764039)(803.24261719,363.50765076)
\lineto(803.24261719,363.25265076)
\curveto(803.24261488,363.07264092)(803.23761488,362.90264109)(803.22761719,362.74265076)
\curveto(803.22761489,362.58264141)(803.16261496,362.47764151)(803.03261719,362.42765076)
\curveto(802.98261514,362.40764158)(802.92761519,362.39764159)(802.86761719,362.39765076)
\lineto(802.70261719,362.39765076)
\lineto(802.38761719,362.39765076)
\curveto(802.28761583,362.39764159)(802.20261592,362.42264157)(802.13261719,362.47265076)
\moveto(803.24261719,353.96765076)
\lineto(803.24261719,353.65265076)
\curveto(803.25261487,353.55265044)(803.23261489,353.47265052)(803.18261719,353.41265076)
\curveto(803.15261497,353.35265064)(803.10761501,353.31265068)(803.04761719,353.29265076)
\curveto(802.98761513,353.28265071)(802.9176152,353.26765072)(802.83761719,353.24765076)
\lineto(802.61261719,353.24765076)
\curveto(802.48261564,353.24765074)(802.36761575,353.25265074)(802.26761719,353.26265076)
\curveto(802.17761594,353.28265071)(802.10761601,353.33265066)(802.05761719,353.41265076)
\curveto(802.0176161,353.47265052)(801.99761612,353.54765044)(801.99761719,353.63765076)
\lineto(801.99761719,353.92265076)
\lineto(801.99761719,360.26765076)
\lineto(801.99761719,360.58265076)
\curveto(801.99761612,360.6926433)(802.0226161,360.77764321)(802.07261719,360.83765076)
\curveto(802.10261602,360.8876431)(802.14261598,360.91764307)(802.19261719,360.92765076)
\curveto(802.24261588,360.93764305)(802.29761582,360.95264304)(802.35761719,360.97265076)
\curveto(802.37761574,360.97264302)(802.39761572,360.96764302)(802.41761719,360.95765076)
\curveto(802.44761567,360.95764303)(802.47261565,360.96264303)(802.49261719,360.97265076)
\curveto(802.6226155,360.97264302)(802.75261537,360.96764302)(802.88261719,360.95765076)
\curveto(803.0226151,360.95764303)(803.117615,360.91764307)(803.16761719,360.83765076)
\curveto(803.2176149,360.77764321)(803.24261488,360.69764329)(803.24261719,360.59765076)
\lineto(803.24261719,360.31265076)
\lineto(803.24261719,353.96765076)
}
}
{
\newrgbcolor{curcolor}{0 0 0}
\pscustom[linestyle=none,fillstyle=solid,fillcolor=curcolor]
{
\newpath
\moveto(812.31246094,357.44765076)
\curveto(812.33245288,357.3876466)(812.34245287,357.2926467)(812.34246094,357.16265076)
\curveto(812.34245287,357.04264695)(812.33745287,356.95764703)(812.32746094,356.90765076)
\lineto(812.32746094,356.75765076)
\curveto(812.31745289,356.67764731)(812.3074529,356.60264739)(812.29746094,356.53265076)
\curveto(812.29745291,356.47264752)(812.29245292,356.40264759)(812.28246094,356.32265076)
\curveto(812.26245295,356.26264773)(812.24745296,356.20264779)(812.23746094,356.14265076)
\curveto(812.23745297,356.08264791)(812.22745298,356.02264797)(812.20746094,355.96265076)
\curveto(812.16745304,355.83264816)(812.13245308,355.70264829)(812.10246094,355.57265076)
\curveto(812.07245314,355.44264855)(812.03245318,355.32264867)(811.98246094,355.21265076)
\curveto(811.77245344,354.73264926)(811.49245372,354.32764966)(811.14246094,353.99765076)
\curveto(810.79245442,353.67765031)(810.36245485,353.43265056)(809.85246094,353.26265076)
\curveto(809.74245547,353.22265077)(809.62245559,353.1926508)(809.49246094,353.17265076)
\curveto(809.37245584,353.15265084)(809.24745596,353.13265086)(809.11746094,353.11265076)
\curveto(809.05745615,353.10265089)(808.99245622,353.09765089)(808.92246094,353.09765076)
\curveto(808.86245635,353.0876509)(808.80245641,353.08265091)(808.74246094,353.08265076)
\curveto(808.70245651,353.07265092)(808.64245657,353.06765092)(808.56246094,353.06765076)
\curveto(808.49245672,353.06765092)(808.44245677,353.07265092)(808.41246094,353.08265076)
\curveto(808.37245684,353.0926509)(808.33245688,353.09765089)(808.29246094,353.09765076)
\curveto(808.25245696,353.0876509)(808.21745699,353.0876509)(808.18746094,353.09765076)
\lineto(808.09746094,353.09765076)
\lineto(807.73746094,353.14265076)
\curveto(807.59745761,353.18265081)(807.46245775,353.22265077)(807.33246094,353.26265076)
\curveto(807.20245801,353.30265069)(807.07745813,353.34765064)(806.95746094,353.39765076)
\curveto(806.5074587,353.59765039)(806.13745907,353.85765013)(805.84746094,354.17765076)
\curveto(805.55745965,354.49764949)(805.31745989,354.8876491)(805.12746094,355.34765076)
\curveto(805.07746013,355.44764854)(805.03746017,355.54764844)(805.00746094,355.64765076)
\curveto(804.98746022,355.74764824)(804.96746024,355.85264814)(804.94746094,355.96265076)
\curveto(804.92746028,356.00264799)(804.91746029,356.03264796)(804.91746094,356.05265076)
\curveto(804.92746028,356.08264791)(804.92746028,356.11764787)(804.91746094,356.15765076)
\curveto(804.89746031,356.23764775)(804.88246033,356.31764767)(804.87246094,356.39765076)
\curveto(804.87246034,356.4876475)(804.86246035,356.57264742)(804.84246094,356.65265076)
\lineto(804.84246094,356.77265076)
\curveto(804.84246037,356.81264718)(804.83746037,356.85764713)(804.82746094,356.90765076)
\curveto(804.81746039,356.95764703)(804.8124604,357.04264695)(804.81246094,357.16265076)
\curveto(804.8124604,357.2926467)(804.82246039,357.3876466)(804.84246094,357.44765076)
\curveto(804.86246035,357.51764647)(804.86746034,357.5876464)(804.85746094,357.65765076)
\curveto(804.84746036,357.72764626)(804.85246036,357.79764619)(804.87246094,357.86765076)
\curveto(804.88246033,357.91764607)(804.88746032,357.95764603)(804.88746094,357.98765076)
\curveto(804.89746031,358.02764596)(804.9074603,358.07264592)(804.91746094,358.12265076)
\curveto(804.94746026,358.24264575)(804.97246024,358.36264563)(804.99246094,358.48265076)
\curveto(805.02246019,358.60264539)(805.06246015,358.71764527)(805.11246094,358.82765076)
\curveto(805.26245995,359.19764479)(805.44245977,359.52764446)(805.65246094,359.81765076)
\curveto(805.87245934,360.11764387)(806.13745907,360.36764362)(806.44746094,360.56765076)
\curveto(806.56745864,360.64764334)(806.69245852,360.71264328)(806.82246094,360.76265076)
\curveto(806.95245826,360.82264317)(807.08745812,360.88264311)(807.22746094,360.94265076)
\curveto(807.34745786,360.992643)(807.47745773,361.02264297)(807.61746094,361.03265076)
\curveto(807.75745745,361.05264294)(807.89745731,361.08264291)(808.03746094,361.12265076)
\lineto(808.23246094,361.12265076)
\curveto(808.30245691,361.13264286)(808.36745684,361.14264285)(808.42746094,361.15265076)
\curveto(809.31745589,361.16264283)(810.05745515,360.97764301)(810.64746094,360.59765076)
\curveto(811.23745397,360.21764377)(811.66245355,359.72264427)(811.92246094,359.11265076)
\curveto(811.97245324,359.01264498)(812.0124532,358.91264508)(812.04246094,358.81265076)
\curveto(812.07245314,358.71264528)(812.1074531,358.60764538)(812.14746094,358.49765076)
\curveto(812.17745303,358.3876456)(812.20245301,358.26764572)(812.22246094,358.13765076)
\curveto(812.24245297,358.01764597)(812.26745294,357.8926461)(812.29746094,357.76265076)
\curveto(812.3074529,357.71264628)(812.3074529,357.65764633)(812.29746094,357.59765076)
\curveto(812.29745291,357.54764644)(812.30245291,357.49764649)(812.31246094,357.44765076)
\moveto(810.97746094,356.59265076)
\curveto(810.99745421,356.66264733)(811.00245421,356.74264725)(810.99246094,356.83265076)
\lineto(810.99246094,357.08765076)
\curveto(810.99245422,357.47764651)(810.95745425,357.80764618)(810.88746094,358.07765076)
\curveto(810.85745435,358.15764583)(810.83245438,358.23764575)(810.81246094,358.31765076)
\curveto(810.79245442,358.39764559)(810.76745444,358.47264552)(810.73746094,358.54265076)
\curveto(810.45745475,359.1926448)(810.0124552,359.64264435)(809.40246094,359.89265076)
\curveto(809.33245588,359.92264407)(809.25745595,359.94264405)(809.17746094,359.95265076)
\lineto(808.93746094,360.01265076)
\curveto(808.85745635,360.03264396)(808.77245644,360.04264395)(808.68246094,360.04265076)
\lineto(808.41246094,360.04265076)
\lineto(808.14246094,359.99765076)
\curveto(808.04245717,359.97764401)(807.94745726,359.95264404)(807.85746094,359.92265076)
\curveto(807.77745743,359.90264409)(807.69745751,359.87264412)(807.61746094,359.83265076)
\curveto(807.54745766,359.81264418)(807.48245773,359.78264421)(807.42246094,359.74265076)
\curveto(807.36245785,359.70264429)(807.3074579,359.66264433)(807.25746094,359.62265076)
\curveto(807.01745819,359.45264454)(806.82245839,359.24764474)(806.67246094,359.00765076)
\curveto(806.52245869,358.76764522)(806.39245882,358.4876455)(806.28246094,358.16765076)
\curveto(806.25245896,358.06764592)(806.23245898,357.96264603)(806.22246094,357.85265076)
\curveto(806.212459,357.75264624)(806.19745901,357.64764634)(806.17746094,357.53765076)
\curveto(806.16745904,357.49764649)(806.16245905,357.43264656)(806.16246094,357.34265076)
\curveto(806.15245906,357.31264668)(806.14745906,357.27764671)(806.14746094,357.23765076)
\curveto(806.15745905,357.19764679)(806.16245905,357.15264684)(806.16246094,357.10265076)
\lineto(806.16246094,356.80265076)
\curveto(806.16245905,356.70264729)(806.17245904,356.61264738)(806.19246094,356.53265076)
\lineto(806.22246094,356.35265076)
\curveto(806.24245897,356.25264774)(806.25745895,356.15264784)(806.26746094,356.05265076)
\curveto(806.28745892,355.96264803)(806.31745889,355.87764811)(806.35746094,355.79765076)
\curveto(806.45745875,355.55764843)(806.57245864,355.33264866)(806.70246094,355.12265076)
\curveto(806.84245837,354.91264908)(807.0124582,354.73764925)(807.21246094,354.59765076)
\curveto(807.26245795,354.56764942)(807.3074579,354.54264945)(807.34746094,354.52265076)
\curveto(807.38745782,354.50264949)(807.43245778,354.47764951)(807.48246094,354.44765076)
\curveto(807.56245765,354.39764959)(807.64745756,354.35264964)(807.73746094,354.31265076)
\curveto(807.83745737,354.28264971)(807.94245727,354.25264974)(808.05246094,354.22265076)
\curveto(808.10245711,354.20264979)(808.14745706,354.1926498)(808.18746094,354.19265076)
\curveto(808.23745697,354.20264979)(808.28745692,354.20264979)(808.33746094,354.19265076)
\curveto(808.36745684,354.18264981)(808.42745678,354.17264982)(808.51746094,354.16265076)
\curveto(808.61745659,354.15264984)(808.69245652,354.15764983)(808.74246094,354.17765076)
\curveto(808.78245643,354.1876498)(808.82245639,354.1876498)(808.86246094,354.17765076)
\curveto(808.90245631,354.17764981)(808.94245627,354.1876498)(808.98246094,354.20765076)
\curveto(809.06245615,354.22764976)(809.14245607,354.24264975)(809.22246094,354.25265076)
\curveto(809.30245591,354.27264972)(809.37745583,354.29764969)(809.44746094,354.32765076)
\curveto(809.78745542,354.46764952)(810.06245515,354.66264933)(810.27246094,354.91265076)
\curveto(810.48245473,355.16264883)(810.65745455,355.45764853)(810.79746094,355.79765076)
\curveto(810.84745436,355.91764807)(810.87745433,356.04264795)(810.88746094,356.17265076)
\curveto(810.9074543,356.31264768)(810.93745427,356.45264754)(810.97746094,356.59265076)
}
}
{
\newrgbcolor{curcolor}{0 0 0}
\pscustom[linestyle=none,fillstyle=solid,fillcolor=curcolor]
{
\newpath
\moveto(816.23074219,361.15265076)
\curveto(816.95073812,361.16264283)(817.55573752,361.07764291)(818.04574219,360.89765076)
\curveto(818.53573654,360.72764326)(818.91573616,360.42264357)(819.18574219,359.98265076)
\curveto(819.25573582,359.87264412)(819.31073576,359.75764423)(819.35074219,359.63765076)
\curveto(819.39073568,359.52764446)(819.43073564,359.40264459)(819.47074219,359.26265076)
\curveto(819.49073558,359.1926448)(819.49573558,359.11764487)(819.48574219,359.03765076)
\curveto(819.4757356,358.96764502)(819.46073561,358.91264508)(819.44074219,358.87265076)
\curveto(819.42073565,358.85264514)(819.39573568,358.83264516)(819.36574219,358.81265076)
\curveto(819.33573574,358.80264519)(819.31073576,358.7876452)(819.29074219,358.76765076)
\curveto(819.24073583,358.74764524)(819.19073588,358.74264525)(819.14074219,358.75265076)
\curveto(819.09073598,358.76264523)(819.04073603,358.76264523)(818.99074219,358.75265076)
\curveto(818.91073616,358.73264526)(818.80573627,358.72764526)(818.67574219,358.73765076)
\curveto(818.54573653,358.75764523)(818.45573662,358.78264521)(818.40574219,358.81265076)
\curveto(818.32573675,358.86264513)(818.2707368,358.92764506)(818.24074219,359.00765076)
\curveto(818.22073685,359.09764489)(818.18573689,359.18264481)(818.13574219,359.26265076)
\curveto(818.04573703,359.42264457)(817.92073715,359.56764442)(817.76074219,359.69765076)
\curveto(817.65073742,359.77764421)(817.53073754,359.83764415)(817.40074219,359.87765076)
\curveto(817.2707378,359.91764407)(817.13073794,359.95764403)(816.98074219,359.99765076)
\curveto(816.93073814,360.01764397)(816.88073819,360.02264397)(816.83074219,360.01265076)
\curveto(816.78073829,360.01264398)(816.73073834,360.01764397)(816.68074219,360.02765076)
\curveto(816.62073845,360.04764394)(816.54573853,360.05764393)(816.45574219,360.05765076)
\curveto(816.36573871,360.05764393)(816.29073878,360.04764394)(816.23074219,360.02765076)
\lineto(816.14074219,360.02765076)
\lineto(815.99074219,359.99765076)
\curveto(815.94073913,359.99764399)(815.89073918,359.992644)(815.84074219,359.98265076)
\curveto(815.58073949,359.92264407)(815.36573971,359.83764415)(815.19574219,359.72765076)
\curveto(815.02574005,359.61764437)(814.91074016,359.43264456)(814.85074219,359.17265076)
\curveto(814.83074024,359.10264489)(814.82574025,359.03264496)(814.83574219,358.96265076)
\curveto(814.85574022,358.8926451)(814.8757402,358.83264516)(814.89574219,358.78265076)
\curveto(814.95574012,358.63264536)(815.02574005,358.52264547)(815.10574219,358.45265076)
\curveto(815.19573988,358.3926456)(815.30573977,358.32264567)(815.43574219,358.24265076)
\curveto(815.59573948,358.14264585)(815.7757393,358.06764592)(815.97574219,358.01765076)
\curveto(816.1757389,357.97764601)(816.3757387,357.92764606)(816.57574219,357.86765076)
\curveto(816.70573837,357.82764616)(816.83573824,357.79764619)(816.96574219,357.77765076)
\curveto(817.09573798,357.75764623)(817.22573785,357.72764626)(817.35574219,357.68765076)
\curveto(817.56573751,357.62764636)(817.7707373,357.56764642)(817.97074219,357.50765076)
\curveto(818.1707369,357.45764653)(818.3707367,357.3926466)(818.57074219,357.31265076)
\lineto(818.72074219,357.25265076)
\curveto(818.7707363,357.23264676)(818.82073625,357.20764678)(818.87074219,357.17765076)
\curveto(819.070736,357.05764693)(819.24573583,356.92264707)(819.39574219,356.77265076)
\curveto(819.54573553,356.62264737)(819.6707354,356.43264756)(819.77074219,356.20265076)
\curveto(819.79073528,356.13264786)(819.81073526,356.03764795)(819.83074219,355.91765076)
\curveto(819.85073522,355.84764814)(819.86073521,355.77264822)(819.86074219,355.69265076)
\curveto(819.8707352,355.62264837)(819.8757352,355.54264845)(819.87574219,355.45265076)
\lineto(819.87574219,355.30265076)
\curveto(819.85573522,355.23264876)(819.84573523,355.16264883)(819.84574219,355.09265076)
\curveto(819.84573523,355.02264897)(819.83573524,354.95264904)(819.81574219,354.88265076)
\curveto(819.78573529,354.77264922)(819.75073532,354.66764932)(819.71074219,354.56765076)
\curveto(819.6707354,354.46764952)(819.62573545,354.37764961)(819.57574219,354.29765076)
\curveto(819.41573566,354.03764995)(819.21073586,353.82765016)(818.96074219,353.66765076)
\curveto(818.71073636,353.51765047)(818.43073664,353.3876506)(818.12074219,353.27765076)
\curveto(818.03073704,353.24765074)(817.93573714,353.22765076)(817.83574219,353.21765076)
\curveto(817.74573733,353.19765079)(817.65573742,353.17265082)(817.56574219,353.14265076)
\curveto(817.46573761,353.12265087)(817.36573771,353.11265088)(817.26574219,353.11265076)
\curveto(817.16573791,353.11265088)(817.06573801,353.10265089)(816.96574219,353.08265076)
\lineto(816.81574219,353.08265076)
\curveto(816.76573831,353.07265092)(816.69573838,353.06765092)(816.60574219,353.06765076)
\curveto(816.51573856,353.06765092)(816.44573863,353.07265092)(816.39574219,353.08265076)
\lineto(816.23074219,353.08265076)
\curveto(816.1707389,353.10265089)(816.10573897,353.11265088)(816.03574219,353.11265076)
\curveto(815.96573911,353.10265089)(815.90573917,353.10765088)(815.85574219,353.12765076)
\curveto(815.80573927,353.13765085)(815.74073933,353.14265085)(815.66074219,353.14265076)
\lineto(815.42074219,353.20265076)
\curveto(815.35073972,353.21265078)(815.2757398,353.23265076)(815.19574219,353.26265076)
\curveto(814.88574019,353.36265063)(814.61574046,353.4876505)(814.38574219,353.63765076)
\curveto(814.15574092,353.7876502)(813.95574112,353.98265001)(813.78574219,354.22265076)
\curveto(813.69574138,354.35264964)(813.62074145,354.4876495)(813.56074219,354.62765076)
\curveto(813.50074157,354.76764922)(813.44574163,354.92264907)(813.39574219,355.09265076)
\curveto(813.3757417,355.15264884)(813.36574171,355.22264877)(813.36574219,355.30265076)
\curveto(813.3757417,355.3926486)(813.39074168,355.46264853)(813.41074219,355.51265076)
\curveto(813.44074163,355.55264844)(813.49074158,355.5926484)(813.56074219,355.63265076)
\curveto(813.61074146,355.65264834)(813.68074139,355.66264833)(813.77074219,355.66265076)
\curveto(813.86074121,355.67264832)(813.95074112,355.67264832)(814.04074219,355.66265076)
\curveto(814.13074094,355.65264834)(814.21574086,355.63764835)(814.29574219,355.61765076)
\curveto(814.38574069,355.60764838)(814.44574063,355.5926484)(814.47574219,355.57265076)
\curveto(814.54574053,355.52264847)(814.59074048,355.44764854)(814.61074219,355.34765076)
\curveto(814.64074043,355.25764873)(814.6757404,355.17264882)(814.71574219,355.09265076)
\curveto(814.81574026,354.87264912)(814.95074012,354.70264929)(815.12074219,354.58265076)
\curveto(815.24073983,354.4926495)(815.3757397,354.42264957)(815.52574219,354.37265076)
\curveto(815.6757394,354.32264967)(815.83573924,354.27264972)(816.00574219,354.22265076)
\lineto(816.32074219,354.17765076)
\lineto(816.41074219,354.17765076)
\curveto(816.48073859,354.15764983)(816.5707385,354.14764984)(816.68074219,354.14765076)
\curveto(816.80073827,354.14764984)(816.90073817,354.15764983)(816.98074219,354.17765076)
\curveto(817.05073802,354.17764981)(817.10573797,354.18264981)(817.14574219,354.19265076)
\curveto(817.20573787,354.20264979)(817.26573781,354.20764978)(817.32574219,354.20765076)
\curveto(817.38573769,354.21764977)(817.44073763,354.22764976)(817.49074219,354.23765076)
\curveto(817.78073729,354.31764967)(818.01073706,354.42264957)(818.18074219,354.55265076)
\curveto(818.35073672,354.68264931)(818.4707366,354.90264909)(818.54074219,355.21265076)
\curveto(818.56073651,355.26264873)(818.56573651,355.31764867)(818.55574219,355.37765076)
\curveto(818.54573653,355.43764855)(818.53573654,355.48264851)(818.52574219,355.51265076)
\curveto(818.4757366,355.70264829)(818.40573667,355.84264815)(818.31574219,355.93265076)
\curveto(818.22573685,356.03264796)(818.11073696,356.12264787)(817.97074219,356.20265076)
\curveto(817.88073719,356.26264773)(817.78073729,356.31264768)(817.67074219,356.35265076)
\lineto(817.34074219,356.47265076)
\curveto(817.31073776,356.48264751)(817.28073779,356.4876475)(817.25074219,356.48765076)
\curveto(817.23073784,356.4876475)(817.20573787,356.49764749)(817.17574219,356.51765076)
\curveto(816.83573824,356.62764736)(816.48073859,356.70764728)(816.11074219,356.75765076)
\curveto(815.75073932,356.81764717)(815.41073966,356.91264708)(815.09074219,357.04265076)
\curveto(814.99074008,357.08264691)(814.89574018,357.11764687)(814.80574219,357.14765076)
\curveto(814.71574036,357.17764681)(814.63074044,357.21764677)(814.55074219,357.26765076)
\curveto(814.36074071,357.37764661)(814.18574089,357.50264649)(814.02574219,357.64265076)
\curveto(813.86574121,357.78264621)(813.74074133,357.95764603)(813.65074219,358.16765076)
\curveto(813.62074145,358.23764575)(813.59574148,358.30764568)(813.57574219,358.37765076)
\curveto(813.56574151,358.44764554)(813.55074152,358.52264547)(813.53074219,358.60265076)
\curveto(813.50074157,358.72264527)(813.49074158,358.85764513)(813.50074219,359.00765076)
\curveto(813.51074156,359.16764482)(813.52574155,359.30264469)(813.54574219,359.41265076)
\curveto(813.56574151,359.46264453)(813.5757415,359.50264449)(813.57574219,359.53265076)
\curveto(813.58574149,359.57264442)(813.60074147,359.61264438)(813.62074219,359.65265076)
\curveto(813.71074136,359.88264411)(813.83074124,360.08264391)(813.98074219,360.25265076)
\curveto(814.14074093,360.42264357)(814.32074075,360.57264342)(814.52074219,360.70265076)
\curveto(814.6707404,360.7926432)(814.83574024,360.86264313)(815.01574219,360.91265076)
\curveto(815.19573988,360.97264302)(815.38573969,361.02764296)(815.58574219,361.07765076)
\curveto(815.65573942,361.0876429)(815.72073935,361.09764289)(815.78074219,361.10765076)
\curveto(815.85073922,361.11764287)(815.92573915,361.12764286)(816.00574219,361.13765076)
\curveto(816.03573904,361.14764284)(816.075739,361.14764284)(816.12574219,361.13765076)
\curveto(816.1757389,361.12764286)(816.21073886,361.13264286)(816.23074219,361.15265076)
}
}
{
\newrgbcolor{curcolor}{0 0 0}
\pscustom[linestyle=none,fillstyle=solid,fillcolor=curcolor]
{
\newpath
\moveto(770.46642334,330.80134644)
\curveto(770.51641372,330.67134618)(770.49641374,330.57134628)(770.40642334,330.50134644)
\curveto(770.35641388,330.47134638)(770.29141394,330.4513464)(770.21142334,330.44134644)
\lineto(769.98642334,330.44134644)
\lineto(769.50642334,330.44134644)
\curveto(769.34641489,330.44134641)(769.22141501,330.47634637)(769.13142334,330.54634644)
\curveto(769.05141518,330.59634625)(768.99641524,330.67134618)(768.96642334,330.77134644)
\lineto(768.90642334,331.10134644)
\curveto(768.89641534,331.14134571)(768.89141534,331.17634567)(768.89142334,331.20634644)
\lineto(768.89142334,331.31134644)
\curveto(768.87141536,331.36134549)(768.86641537,331.40634544)(768.87642334,331.44634644)
\curveto(768.88641535,331.48634536)(768.88641535,331.52634532)(768.87642334,331.56634644)
\curveto(768.86641537,331.62634522)(768.86141537,331.68634516)(768.86142334,331.74634644)
\lineto(768.86142334,331.92634644)
\lineto(768.81642334,332.60134644)
\curveto(768.79641544,332.67134418)(768.78641545,332.74134411)(768.78642334,332.81134644)
\curveto(768.78641545,332.88134397)(768.77641546,332.95634389)(768.75642334,333.03634644)
\curveto(768.70641553,333.21634363)(768.66641557,333.39634345)(768.63642334,333.57634644)
\curveto(768.61641562,333.75634309)(768.57141566,333.92634292)(768.50142334,334.08634644)
\curveto(768.31141592,334.50634234)(767.99641624,334.78634206)(767.55642334,334.92634644)
\curveto(767.42641681,334.97634187)(767.28141695,335.00134185)(767.12142334,335.00134644)
\curveto(766.97141726,335.01134184)(766.81141742,335.01634183)(766.64142334,335.01634644)
\lineto(763.88142334,335.01634644)
\curveto(763.81142042,334.99634185)(763.74642049,334.97634187)(763.68642334,334.95634644)
\curveto(763.6364206,334.9463419)(763.59142064,334.91634193)(763.55142334,334.86634644)
\curveto(763.48142075,334.76634208)(763.44642079,334.60134225)(763.44642334,334.37134644)
\curveto(763.45642078,334.1513427)(763.46142077,333.95634289)(763.46142334,333.78634644)
\lineto(763.46142334,331.61134644)
\curveto(763.46142077,331.47134538)(763.46642077,331.29634555)(763.47642334,331.08634644)
\curveto(763.48642075,330.88634596)(763.46642077,330.73634611)(763.41642334,330.63634644)
\curveto(763.39642084,330.56634628)(763.35642088,330.52134633)(763.29642334,330.50134644)
\curveto(763.25642098,330.48134637)(763.21642102,330.47134638)(763.17642334,330.47134644)
\curveto(763.14642109,330.47134638)(763.10642113,330.46134639)(763.05642334,330.44134644)
\curveto(763.01642122,330.43134642)(762.97142126,330.42634642)(762.92142334,330.42634644)
\curveto(762.87142136,330.43634641)(762.82142141,330.44134641)(762.77142334,330.44134644)
\lineto(762.44142334,330.44134644)
\curveto(762.34142189,330.4513464)(762.25642198,330.48134637)(762.18642334,330.53134644)
\curveto(762.10642213,330.58134627)(762.06642217,330.67134618)(762.06642334,330.80134644)
\lineto(762.06642334,331.20634644)
\lineto(762.06642334,340.32634644)
\curveto(762.06642217,340.43633641)(762.06142217,340.5513363)(762.05142334,340.67134644)
\curveto(762.05142218,340.79133606)(762.07642216,340.88633596)(762.12642334,340.95634644)
\curveto(762.16642207,341.01633583)(762.24142199,341.06633578)(762.35142334,341.10634644)
\curveto(762.37142186,341.11633573)(762.39142184,341.11633573)(762.41142334,341.10634644)
\curveto(762.4314218,341.10633574)(762.45142178,341.11133574)(762.47142334,341.12134644)
\lineto(766.82142334,341.12134644)
\curveto(766.89141734,341.12133573)(766.96641727,341.12133573)(767.04642334,341.12134644)
\curveto(767.12641711,341.13133572)(767.19641704,341.13133572)(767.25642334,341.12134644)
\lineto(767.42142334,341.12134644)
\curveto(767.48141675,341.11133574)(767.54141669,341.10133575)(767.60142334,341.09134644)
\curveto(767.66141657,341.09133576)(767.72641651,341.08633576)(767.79642334,341.07634644)
\curveto(767.87641636,341.05633579)(767.95641628,341.04133581)(768.03642334,341.03134644)
\curveto(768.12641611,341.02133583)(768.21141602,341.00633584)(768.29142334,340.98634644)
\curveto(768.48141575,340.92633592)(768.65641558,340.86133599)(768.81642334,340.79134644)
\curveto(768.97641526,340.72133613)(769.12641511,340.63633621)(769.26642334,340.53634644)
\curveto(769.51641472,340.36633648)(769.71641452,340.15633669)(769.86642334,339.90634644)
\curveto(770.02641421,339.66633718)(770.15641408,339.38133747)(770.25642334,339.05134644)
\curveto(770.27641396,338.97133788)(770.28641395,338.88633796)(770.28642334,338.79634644)
\curveto(770.29641394,338.71633813)(770.31141392,338.63633821)(770.33142334,338.55634644)
\lineto(770.33142334,338.40634644)
\curveto(770.34141389,338.35633849)(770.34141389,338.29633855)(770.33142334,338.22634644)
\curveto(770.3314139,338.16633868)(770.32641391,338.11133874)(770.31642334,338.06134644)
\lineto(770.31642334,337.89634644)
\curveto(770.29641394,337.81633903)(770.28141395,337.74133911)(770.27142334,337.67134644)
\curveto(770.27141396,337.60133925)(770.26141397,337.53133932)(770.24142334,337.46134644)
\curveto(770.19141404,337.31133954)(770.14141409,337.16633968)(770.09142334,337.02634644)
\curveto(770.05141418,336.89633995)(769.99141424,336.77134008)(769.91142334,336.65134644)
\curveto(769.88141435,336.60134025)(769.84641439,336.55634029)(769.80642334,336.51634644)
\curveto(769.77641446,336.47634037)(769.74641449,336.43134042)(769.71642334,336.38134644)
\lineto(769.68642334,336.35134644)
\curveto(769.67641456,336.3513405)(769.66641457,336.3463405)(769.65642334,336.33634644)
\lineto(769.58142334,336.26134644)
\curveto(769.56141467,336.23134062)(769.54141469,336.20634064)(769.52142334,336.18634644)
\curveto(769.44141479,336.12634072)(769.36641487,336.06634078)(769.29642334,336.00634644)
\curveto(769.22641501,335.95634089)(769.15141508,335.90634094)(769.07142334,335.85634644)
\curveto(769.02141521,335.82634102)(768.97641526,335.79134106)(768.93642334,335.75134644)
\curveto(768.89641534,335.72134113)(768.87141536,335.67634117)(768.86142334,335.61634644)
\curveto(768.85141538,335.55634129)(768.87141536,335.50634134)(768.92142334,335.46634644)
\curveto(768.98141525,335.42634142)(769.0314152,335.39634145)(769.07142334,335.37634644)
\curveto(769.18141505,335.30634154)(769.28141495,335.23134162)(769.37142334,335.15134644)
\curveto(769.47141476,335.07134178)(769.55641468,334.97634187)(769.62642334,334.86634644)
\curveto(769.7364145,334.72634212)(769.81641442,334.56634228)(769.86642334,334.38634644)
\curveto(769.91641432,334.21634263)(769.96641427,334.03134282)(770.01642334,333.83134644)
\lineto(770.04642334,333.59134644)
\curveto(770.05641418,333.52134333)(770.06641417,333.4463434)(770.07642334,333.36634644)
\curveto(770.09641414,333.29634355)(770.10141413,333.22634362)(770.09142334,333.15634644)
\curveto(770.08141415,333.08634376)(770.08641415,333.01634383)(770.10642334,332.94634644)
\lineto(770.10642334,332.81134644)
\curveto(770.12641411,332.74134411)(770.1314141,332.66634418)(770.12142334,332.58634644)
\curveto(770.11141412,332.50634434)(770.11641412,332.42634442)(770.13642334,332.34634644)
\curveto(770.14641409,332.30634454)(770.14641409,332.26634458)(770.13642334,332.22634644)
\curveto(770.1364141,332.19634465)(770.14141409,332.15634469)(770.15142334,332.10634644)
\curveto(770.17141406,332.00634484)(770.18641405,331.90134495)(770.19642334,331.79134644)
\curveto(770.20641403,331.69134516)(770.22641401,331.59634525)(770.25642334,331.50634644)
\curveto(770.27641396,331.4463454)(770.28641395,331.38634546)(770.28642334,331.32634644)
\curveto(770.29641394,331.27634557)(770.31141392,331.22134563)(770.33142334,331.16134644)
\lineto(770.46642334,330.80134644)
\moveto(768.65142334,337.02634644)
\curveto(768.72141551,337.13633971)(768.77141546,337.2513396)(768.80142334,337.37134644)
\curveto(768.84141539,337.49133936)(768.87641536,337.62133923)(768.90642334,337.76134644)
\lineto(768.90642334,337.89634644)
\curveto(768.9364153,338.03633881)(768.94141529,338.18633866)(768.92142334,338.34634644)
\curveto(768.90141533,338.51633833)(768.87141536,338.65633819)(768.83142334,338.76634644)
\curveto(768.67141556,339.26633758)(768.35641588,339.61133724)(767.88642334,339.80134644)
\curveto(767.68641655,339.88133697)(767.45141678,339.92633692)(767.18142334,339.93634644)
\curveto(766.92141731,339.9463369)(766.65141758,339.9513369)(766.37142334,339.95134644)
\lineto(763.89642334,339.95134644)
\curveto(763.87642036,339.94133691)(763.85142038,339.93633691)(763.82142334,339.93634644)
\curveto(763.80142043,339.93633691)(763.77642046,339.93133692)(763.74642334,339.92134644)
\curveto(763.62642061,339.89133696)(763.54642069,339.82633702)(763.50642334,339.72634644)
\curveto(763.46642077,339.63633721)(763.44642079,339.51133734)(763.44642334,339.35134644)
\curveto(763.45642078,339.19133766)(763.46142077,339.0463378)(763.46142334,338.91634644)
\lineto(763.46142334,337.19134644)
\curveto(763.46142077,337.04133981)(763.45642078,336.88133997)(763.44642334,336.71134644)
\curveto(763.44642079,336.5513403)(763.48142075,336.42634042)(763.55142334,336.33634644)
\curveto(763.60142063,336.26634058)(763.67642056,336.22134063)(763.77642334,336.20134644)
\curveto(763.87642036,336.19134066)(763.98642025,336.18634066)(764.10642334,336.18634644)
\lineto(765.03642334,336.18634644)
\curveto(765.42641881,336.18634066)(765.80641843,336.18134067)(766.17642334,336.17134644)
\curveto(766.54641769,336.17134068)(766.88641735,336.19134066)(767.19642334,336.23134644)
\curveto(767.51641672,336.28134057)(767.80141643,336.36634048)(768.05142334,336.48634644)
\curveto(768.30141593,336.60634024)(768.50141573,336.78634006)(768.65142334,337.02634644)
}
}
{
\newrgbcolor{curcolor}{0 0 0}
\pscustom[linestyle=none,fillstyle=solid,fillcolor=curcolor]
{
\newpath
\moveto(778.84462646,334.59634644)
\curveto(778.86461878,334.49634235)(778.86461878,334.38134247)(778.84462646,334.25134644)
\curveto(778.83461881,334.13134272)(778.80461884,334.0463428)(778.75462646,333.99634644)
\curveto(778.70461894,333.95634289)(778.62961901,333.92634292)(778.52962646,333.90634644)
\curveto(778.4396192,333.89634295)(778.33461931,333.89134296)(778.21462646,333.89134644)
\lineto(777.85462646,333.89134644)
\curveto(777.73461991,333.90134295)(777.62962001,333.90634294)(777.53962646,333.90634644)
\lineto(773.69962646,333.90634644)
\curveto(773.61962402,333.90634294)(773.5396241,333.90134295)(773.45962646,333.89134644)
\curveto(773.37962426,333.89134296)(773.31462433,333.87634297)(773.26462646,333.84634644)
\curveto(773.22462442,333.82634302)(773.18462446,333.78634306)(773.14462646,333.72634644)
\curveto(773.12462452,333.69634315)(773.10462454,333.6513432)(773.08462646,333.59134644)
\curveto(773.06462458,333.54134331)(773.06462458,333.49134336)(773.08462646,333.44134644)
\curveto(773.09462455,333.39134346)(773.09962454,333.3463435)(773.09962646,333.30634644)
\curveto(773.09962454,333.26634358)(773.10462454,333.22634362)(773.11462646,333.18634644)
\curveto(773.13462451,333.10634374)(773.15462449,333.02134383)(773.17462646,332.93134644)
\curveto(773.19462445,332.851344)(773.22462442,332.77134408)(773.26462646,332.69134644)
\curveto(773.49462415,332.1513447)(773.87462377,331.76634508)(774.40462646,331.53634644)
\curveto(774.46462318,331.50634534)(774.52962311,331.48134537)(774.59962646,331.46134644)
\lineto(774.80962646,331.40134644)
\curveto(774.8396228,331.39134546)(774.88962275,331.38634546)(774.95962646,331.38634644)
\curveto(775.09962254,331.3463455)(775.28462236,331.32634552)(775.51462646,331.32634644)
\curveto(775.7446219,331.32634552)(775.92962171,331.3463455)(776.06962646,331.38634644)
\curveto(776.20962143,331.42634542)(776.33462131,331.46634538)(776.44462646,331.50634644)
\curveto(776.56462108,331.55634529)(776.67462097,331.61634523)(776.77462646,331.68634644)
\curveto(776.88462076,331.75634509)(776.97962066,331.83634501)(777.05962646,331.92634644)
\curveto(777.1396205,332.02634482)(777.20962043,332.13134472)(777.26962646,332.24134644)
\curveto(777.32962031,332.34134451)(777.37962026,332.4463444)(777.41962646,332.55634644)
\curveto(777.46962017,332.66634418)(777.54962009,332.7463441)(777.65962646,332.79634644)
\curveto(777.69961994,332.81634403)(777.76461988,332.83134402)(777.85462646,332.84134644)
\curveto(777.9446197,332.851344)(778.03461961,332.851344)(778.12462646,332.84134644)
\curveto(778.21461943,332.84134401)(778.29961934,332.83634401)(778.37962646,332.82634644)
\curveto(778.45961918,332.81634403)(778.51461913,332.79634405)(778.54462646,332.76634644)
\curveto(778.644619,332.69634415)(778.66961897,332.58134427)(778.61962646,332.42134644)
\curveto(778.5396191,332.1513447)(778.43461921,331.91134494)(778.30462646,331.70134644)
\curveto(778.10461954,331.38134547)(777.87461977,331.11634573)(777.61462646,330.90634644)
\curveto(777.36462028,330.70634614)(777.0446206,330.54134631)(776.65462646,330.41134644)
\curveto(776.55462109,330.37134648)(776.45462119,330.3463465)(776.35462646,330.33634644)
\curveto(776.25462139,330.31634653)(776.14962149,330.29634655)(776.03962646,330.27634644)
\curveto(775.98962165,330.26634658)(775.9396217,330.26134659)(775.88962646,330.26134644)
\curveto(775.84962179,330.26134659)(775.80462184,330.25634659)(775.75462646,330.24634644)
\lineto(775.60462646,330.24634644)
\curveto(775.55462209,330.23634661)(775.49462215,330.23134662)(775.42462646,330.23134644)
\curveto(775.36462228,330.23134662)(775.31462233,330.23634661)(775.27462646,330.24634644)
\lineto(775.13962646,330.24634644)
\curveto(775.08962255,330.25634659)(775.0446226,330.26134659)(775.00462646,330.26134644)
\curveto(774.96462268,330.26134659)(774.92462272,330.26634658)(774.88462646,330.27634644)
\curveto(774.83462281,330.28634656)(774.77962286,330.29634655)(774.71962646,330.30634644)
\curveto(774.65962298,330.30634654)(774.60462304,330.31134654)(774.55462646,330.32134644)
\curveto(774.46462318,330.34134651)(774.37462327,330.36634648)(774.28462646,330.39634644)
\curveto(774.19462345,330.41634643)(774.10962353,330.44134641)(774.02962646,330.47134644)
\curveto(773.98962365,330.49134636)(773.95462369,330.50134635)(773.92462646,330.50134644)
\curveto(773.89462375,330.51134634)(773.85962378,330.52634632)(773.81962646,330.54634644)
\curveto(773.66962397,330.61634623)(773.50962413,330.70134615)(773.33962646,330.80134644)
\curveto(773.04962459,330.99134586)(772.79962484,331.22134563)(772.58962646,331.49134644)
\curveto(772.38962525,331.77134508)(772.21962542,332.08134477)(772.07962646,332.42134644)
\curveto(772.02962561,332.53134432)(771.98962565,332.6463442)(771.95962646,332.76634644)
\curveto(771.9396257,332.88634396)(771.90962573,333.00634384)(771.86962646,333.12634644)
\curveto(771.85962578,333.16634368)(771.85462579,333.20134365)(771.85462646,333.23134644)
\curveto(771.85462579,333.26134359)(771.84962579,333.30134355)(771.83962646,333.35134644)
\curveto(771.81962582,333.43134342)(771.80462584,333.51634333)(771.79462646,333.60634644)
\curveto(771.78462586,333.69634315)(771.76962587,333.78634306)(771.74962646,333.87634644)
\lineto(771.74962646,334.08634644)
\curveto(771.7396259,334.12634272)(771.72962591,334.18134267)(771.71962646,334.25134644)
\curveto(771.71962592,334.33134252)(771.72462592,334.39634245)(771.73462646,334.44634644)
\lineto(771.73462646,334.61134644)
\curveto(771.75462589,334.66134219)(771.75962588,334.71134214)(771.74962646,334.76134644)
\curveto(771.74962589,334.82134203)(771.75462589,334.87634197)(771.76462646,334.92634644)
\curveto(771.80462584,335.08634176)(771.83462581,335.2463416)(771.85462646,335.40634644)
\curveto(771.88462576,335.56634128)(771.92962571,335.71634113)(771.98962646,335.85634644)
\curveto(772.0396256,335.96634088)(772.08462556,336.07634077)(772.12462646,336.18634644)
\curveto(772.17462547,336.30634054)(772.22962541,336.42134043)(772.28962646,336.53134644)
\curveto(772.50962513,336.88133997)(772.75962488,337.18133967)(773.03962646,337.43134644)
\curveto(773.31962432,337.69133916)(773.66462398,337.90633894)(774.07462646,338.07634644)
\curveto(774.19462345,338.12633872)(774.31462333,338.16133869)(774.43462646,338.18134644)
\curveto(774.56462308,338.21133864)(774.69962294,338.24133861)(774.83962646,338.27134644)
\curveto(774.88962275,338.28133857)(774.93462271,338.28633856)(774.97462646,338.28634644)
\curveto(775.01462263,338.29633855)(775.05962258,338.30133855)(775.10962646,338.30134644)
\curveto(775.12962251,338.31133854)(775.15462249,338.31133854)(775.18462646,338.30134644)
\curveto(775.21462243,338.29133856)(775.2396224,338.29633855)(775.25962646,338.31634644)
\curveto(775.67962196,338.32633852)(776.0446216,338.28133857)(776.35462646,338.18134644)
\curveto(776.66462098,338.09133876)(776.9446207,337.96633888)(777.19462646,337.80634644)
\curveto(777.2446204,337.78633906)(777.28462036,337.75633909)(777.31462646,337.71634644)
\curveto(777.3446203,337.68633916)(777.37962026,337.66133919)(777.41962646,337.64134644)
\curveto(777.49962014,337.58133927)(777.57962006,337.51133934)(777.65962646,337.43134644)
\curveto(777.74961989,337.3513395)(777.82461982,337.27133958)(777.88462646,337.19134644)
\curveto(778.0446196,336.98133987)(778.17961946,336.78134007)(778.28962646,336.59134644)
\curveto(778.35961928,336.48134037)(778.41461923,336.36134049)(778.45462646,336.23134644)
\curveto(778.49461915,336.10134075)(778.5396191,335.97134088)(778.58962646,335.84134644)
\curveto(778.639619,335.71134114)(778.67461897,335.57634127)(778.69462646,335.43634644)
\curveto(778.72461892,335.29634155)(778.75961888,335.15634169)(778.79962646,335.01634644)
\curveto(778.80961883,334.9463419)(778.81461883,334.87634197)(778.81462646,334.80634644)
\lineto(778.84462646,334.59634644)
\moveto(777.38962646,335.10634644)
\curveto(777.41962022,335.1463417)(777.4446202,335.19634165)(777.46462646,335.25634644)
\curveto(777.48462016,335.32634152)(777.48462016,335.39634145)(777.46462646,335.46634644)
\curveto(777.40462024,335.68634116)(777.31962032,335.89134096)(777.20962646,336.08134644)
\curveto(777.06962057,336.31134054)(776.91462073,336.50634034)(776.74462646,336.66634644)
\curveto(776.57462107,336.82634002)(776.35462129,336.96133989)(776.08462646,337.07134644)
\curveto(776.01462163,337.09133976)(775.9446217,337.10633974)(775.87462646,337.11634644)
\curveto(775.80462184,337.13633971)(775.72962191,337.15633969)(775.64962646,337.17634644)
\curveto(775.56962207,337.19633965)(775.48462216,337.20633964)(775.39462646,337.20634644)
\lineto(775.13962646,337.20634644)
\curveto(775.10962253,337.18633966)(775.07462257,337.17633967)(775.03462646,337.17634644)
\curveto(774.99462265,337.18633966)(774.95962268,337.18633966)(774.92962646,337.17634644)
\lineto(774.68962646,337.11634644)
\curveto(774.61962302,337.10633974)(774.54962309,337.09133976)(774.47962646,337.07134644)
\curveto(774.18962345,336.9513399)(773.95462369,336.80134005)(773.77462646,336.62134644)
\curveto(773.60462404,336.44134041)(773.44962419,336.21634063)(773.30962646,335.94634644)
\curveto(773.27962436,335.89634095)(773.24962439,335.83134102)(773.21962646,335.75134644)
\curveto(773.18962445,335.68134117)(773.16462448,335.60134125)(773.14462646,335.51134644)
\curveto(773.12462452,335.42134143)(773.11962452,335.33634151)(773.12962646,335.25634644)
\curveto(773.1396245,335.17634167)(773.17462447,335.11634173)(773.23462646,335.07634644)
\curveto(773.31462433,335.01634183)(773.44962419,334.98634186)(773.63962646,334.98634644)
\curveto(773.8396238,334.99634185)(774.00962363,335.00134185)(774.14962646,335.00134644)
\lineto(776.42962646,335.00134644)
\curveto(776.57962106,335.00134185)(776.75962088,334.99634185)(776.96962646,334.98634644)
\curveto(777.17962046,334.98634186)(777.31962032,335.02634182)(777.38962646,335.10634644)
}
}
{
\newrgbcolor{curcolor}{0 0 0}
\pscustom[linestyle=none,fillstyle=solid,fillcolor=curcolor]
{
\newpath
\moveto(783.28626709,338.33134644)
\curveto(784.0262623,338.34133851)(784.64126168,338.23133862)(785.13126709,338.00134644)
\curveto(785.63126069,337.78133907)(786.0262603,337.4463394)(786.31626709,336.99634644)
\curveto(786.44625988,336.79634005)(786.55625977,336.5513403)(786.64626709,336.26134644)
\curveto(786.66625966,336.21134064)(786.68125964,336.1463407)(786.69126709,336.06634644)
\curveto(786.70125962,335.98634086)(786.69625963,335.91634093)(786.67626709,335.85634644)
\curveto(786.64625968,335.80634104)(786.59625973,335.76134109)(786.52626709,335.72134644)
\curveto(786.49625983,335.70134115)(786.46625986,335.69134116)(786.43626709,335.69134644)
\curveto(786.40625992,335.70134115)(786.37125995,335.70134115)(786.33126709,335.69134644)
\curveto(786.29126003,335.68134117)(786.25126007,335.67634117)(786.21126709,335.67634644)
\curveto(786.17126015,335.68634116)(786.13126019,335.69134116)(786.09126709,335.69134644)
\lineto(785.77626709,335.69134644)
\curveto(785.67626065,335.70134115)(785.59126073,335.73134112)(785.52126709,335.78134644)
\curveto(785.44126088,335.84134101)(785.38626094,335.92634092)(785.35626709,336.03634644)
\curveto(785.326261,336.1463407)(785.28626104,336.24134061)(785.23626709,336.32134644)
\curveto(785.08626124,336.58134027)(784.89126143,336.78634006)(784.65126709,336.93634644)
\curveto(784.57126175,336.98633986)(784.48626184,337.02633982)(784.39626709,337.05634644)
\curveto(784.30626202,337.09633975)(784.21126211,337.13133972)(784.11126709,337.16134644)
\curveto(783.97126235,337.20133965)(783.78626254,337.22133963)(783.55626709,337.22134644)
\curveto(783.326263,337.23133962)(783.13626319,337.21133964)(782.98626709,337.16134644)
\curveto(782.91626341,337.14133971)(782.85126347,337.12633972)(782.79126709,337.11634644)
\curveto(782.73126359,337.10633974)(782.66626366,337.09133976)(782.59626709,337.07134644)
\curveto(782.33626399,336.96133989)(782.10626422,336.81134004)(781.90626709,336.62134644)
\curveto(781.70626462,336.43134042)(781.55126477,336.20634064)(781.44126709,335.94634644)
\curveto(781.40126492,335.85634099)(781.36626496,335.76134109)(781.33626709,335.66134644)
\curveto(781.30626502,335.57134128)(781.27626505,335.47134138)(781.24626709,335.36134644)
\lineto(781.15626709,334.95634644)
\curveto(781.14626518,334.90634194)(781.14126518,334.851342)(781.14126709,334.79134644)
\curveto(781.15126517,334.73134212)(781.14626518,334.67634217)(781.12626709,334.62634644)
\lineto(781.12626709,334.50634644)
\curveto(781.11626521,334.46634238)(781.10626522,334.40134245)(781.09626709,334.31134644)
\curveto(781.09626523,334.22134263)(781.10626522,334.15634269)(781.12626709,334.11634644)
\curveto(781.13626519,334.06634278)(781.13626519,334.01634283)(781.12626709,333.96634644)
\curveto(781.11626521,333.91634293)(781.11626521,333.86634298)(781.12626709,333.81634644)
\curveto(781.13626519,333.77634307)(781.14126518,333.70634314)(781.14126709,333.60634644)
\curveto(781.16126516,333.52634332)(781.17626515,333.44134341)(781.18626709,333.35134644)
\curveto(781.20626512,333.26134359)(781.2262651,333.17634367)(781.24626709,333.09634644)
\curveto(781.35626497,332.77634407)(781.48126484,332.49634435)(781.62126709,332.25634644)
\curveto(781.77126455,332.02634482)(781.97626435,331.82634502)(782.23626709,331.65634644)
\curveto(782.326264,331.60634524)(782.41626391,331.56134529)(782.50626709,331.52134644)
\curveto(782.60626372,331.48134537)(782.71126361,331.44134541)(782.82126709,331.40134644)
\curveto(782.87126345,331.39134546)(782.91126341,331.38634546)(782.94126709,331.38634644)
\curveto(782.97126335,331.38634546)(783.01126331,331.38134547)(783.06126709,331.37134644)
\curveto(783.09126323,331.36134549)(783.14126318,331.35634549)(783.21126709,331.35634644)
\lineto(783.37626709,331.35634644)
\curveto(783.37626295,331.3463455)(783.39626293,331.34134551)(783.43626709,331.34134644)
\curveto(783.45626287,331.3513455)(783.48126284,331.3513455)(783.51126709,331.34134644)
\curveto(783.54126278,331.34134551)(783.57126275,331.3463455)(783.60126709,331.35634644)
\curveto(783.67126265,331.37634547)(783.73626259,331.38134547)(783.79626709,331.37134644)
\curveto(783.86626246,331.37134548)(783.93626239,331.38134547)(784.00626709,331.40134644)
\curveto(784.26626206,331.48134537)(784.49126183,331.58134527)(784.68126709,331.70134644)
\curveto(784.87126145,331.83134502)(785.03126129,331.99634485)(785.16126709,332.19634644)
\curveto(785.21126111,332.27634457)(785.25626107,332.36134449)(785.29626709,332.45134644)
\lineto(785.41626709,332.72134644)
\curveto(785.43626089,332.80134405)(785.45626087,332.87634397)(785.47626709,332.94634644)
\curveto(785.50626082,333.02634382)(785.55626077,333.09134376)(785.62626709,333.14134644)
\curveto(785.65626067,333.17134368)(785.71626061,333.19134366)(785.80626709,333.20134644)
\curveto(785.89626043,333.22134363)(785.99126033,333.23134362)(786.09126709,333.23134644)
\curveto(786.20126012,333.24134361)(786.30126002,333.24134361)(786.39126709,333.23134644)
\curveto(786.49125983,333.22134363)(786.56125976,333.20134365)(786.60126709,333.17134644)
\curveto(786.66125966,333.13134372)(786.69625963,333.07134378)(786.70626709,332.99134644)
\curveto(786.7262596,332.91134394)(786.7262596,332.82634402)(786.70626709,332.73634644)
\curveto(786.65625967,332.58634426)(786.60625972,332.44134441)(786.55626709,332.30134644)
\curveto(786.51625981,332.17134468)(786.46125986,332.04134481)(786.39126709,331.91134644)
\curveto(786.24126008,331.61134524)(786.05126027,331.3463455)(785.82126709,331.11634644)
\curveto(785.60126072,330.88634596)(785.33126099,330.70134615)(785.01126709,330.56134644)
\curveto(784.93126139,330.52134633)(784.84626148,330.48634636)(784.75626709,330.45634644)
\curveto(784.66626166,330.43634641)(784.57126175,330.41134644)(784.47126709,330.38134644)
\curveto(784.36126196,330.34134651)(784.25126207,330.32134653)(784.14126709,330.32134644)
\curveto(784.03126229,330.31134654)(783.9212624,330.29634655)(783.81126709,330.27634644)
\curveto(783.77126255,330.25634659)(783.73126259,330.2513466)(783.69126709,330.26134644)
\curveto(783.65126267,330.27134658)(783.61126271,330.27134658)(783.57126709,330.26134644)
\lineto(783.43626709,330.26134644)
\lineto(783.19626709,330.26134644)
\curveto(783.1262632,330.2513466)(783.06126326,330.25634659)(783.00126709,330.27634644)
\lineto(782.92626709,330.27634644)
\lineto(782.56626709,330.32134644)
\curveto(782.43626389,330.36134649)(782.31126401,330.39634645)(782.19126709,330.42634644)
\curveto(782.07126425,330.45634639)(781.95626437,330.49634635)(781.84626709,330.54634644)
\curveto(781.48626484,330.70634614)(781.18626514,330.89634595)(780.94626709,331.11634644)
\curveto(780.71626561,331.33634551)(780.50126582,331.60634524)(780.30126709,331.92634644)
\curveto(780.25126607,332.00634484)(780.20626612,332.09634475)(780.16626709,332.19634644)
\lineto(780.04626709,332.49634644)
\curveto(779.99626633,332.60634424)(779.96126636,332.72134413)(779.94126709,332.84134644)
\curveto(779.9212664,332.96134389)(779.89626643,333.08134377)(779.86626709,333.20134644)
\curveto(779.85626647,333.24134361)(779.85126647,333.28134357)(779.85126709,333.32134644)
\curveto(779.85126647,333.36134349)(779.84626648,333.40134345)(779.83626709,333.44134644)
\curveto(779.81626651,333.50134335)(779.80626652,333.56634328)(779.80626709,333.63634644)
\curveto(779.81626651,333.70634314)(779.81126651,333.77134308)(779.79126709,333.83134644)
\lineto(779.79126709,333.98134644)
\curveto(779.78126654,334.03134282)(779.77626655,334.10134275)(779.77626709,334.19134644)
\curveto(779.77626655,334.28134257)(779.78126654,334.3513425)(779.79126709,334.40134644)
\curveto(779.80126652,334.4513424)(779.80126652,334.49634235)(779.79126709,334.53634644)
\curveto(779.79126653,334.57634227)(779.79626653,334.61634223)(779.80626709,334.65634644)
\curveto(779.8262665,334.72634212)(779.83126649,334.79634205)(779.82126709,334.86634644)
\curveto(779.8212665,334.93634191)(779.83126649,335.00134185)(779.85126709,335.06134644)
\curveto(779.89126643,335.23134162)(779.9262664,335.40134145)(779.95626709,335.57134644)
\curveto(779.98626634,335.74134111)(780.03126629,335.90134095)(780.09126709,336.05134644)
\curveto(780.30126602,336.57134028)(780.55626577,336.99133986)(780.85626709,337.31134644)
\curveto(781.15626517,337.63133922)(781.56626476,337.89633895)(782.08626709,338.10634644)
\curveto(782.19626413,338.15633869)(782.31626401,338.19133866)(782.44626709,338.21134644)
\curveto(782.57626375,338.23133862)(782.71126361,338.25633859)(782.85126709,338.28634644)
\curveto(782.9212634,338.29633855)(782.99126333,338.30133855)(783.06126709,338.30134644)
\curveto(783.13126319,338.31133854)(783.20626312,338.32133853)(783.28626709,338.33134644)
}
}
{
\newrgbcolor{curcolor}{0 0 0}
\pscustom[linestyle=none,fillstyle=solid,fillcolor=curcolor]
{
\newpath
\moveto(788.67290771,338.15134644)
\lineto(789.10790771,338.15134644)
\curveto(789.25790575,338.1513387)(789.36290564,338.11133874)(789.42290771,338.03134644)
\curveto(789.47290553,337.9513389)(789.49790551,337.851339)(789.49790771,337.73134644)
\curveto(789.5079055,337.61133924)(789.51290549,337.49133936)(789.51290771,337.37134644)
\lineto(789.51290771,335.94634644)
\lineto(789.51290771,333.68134644)
\lineto(789.51290771,332.99134644)
\curveto(789.51290549,332.76134409)(789.53790547,332.56134429)(789.58790771,332.39134644)
\curveto(789.74790526,331.94134491)(790.04790496,331.62634522)(790.48790771,331.44634644)
\curveto(790.7079043,331.35634549)(790.97290403,331.32134553)(791.28290771,331.34134644)
\curveto(791.59290341,331.37134548)(791.84290316,331.42634542)(792.03290771,331.50634644)
\curveto(792.36290264,331.6463452)(792.62290238,331.82134503)(792.81290771,332.03134644)
\curveto(793.01290199,332.2513446)(793.16790184,332.53634431)(793.27790771,332.88634644)
\curveto(793.3079017,332.96634388)(793.32790168,333.0463438)(793.33790771,333.12634644)
\curveto(793.34790166,333.20634364)(793.36290164,333.29134356)(793.38290771,333.38134644)
\curveto(793.39290161,333.43134342)(793.39290161,333.47634337)(793.38290771,333.51634644)
\curveto(793.38290162,333.55634329)(793.39290161,333.60134325)(793.41290771,333.65134644)
\lineto(793.41290771,333.96634644)
\curveto(793.43290157,334.0463428)(793.43790157,334.13634271)(793.42790771,334.23634644)
\curveto(793.41790159,334.3463425)(793.41290159,334.4463424)(793.41290771,334.53634644)
\lineto(793.41290771,335.70634644)
\lineto(793.41290771,337.29634644)
\curveto(793.41290159,337.41633943)(793.4079016,337.54133931)(793.39790771,337.67134644)
\curveto(793.39790161,337.81133904)(793.42290158,337.92133893)(793.47290771,338.00134644)
\curveto(793.51290149,338.0513388)(793.55790145,338.08133877)(793.60790771,338.09134644)
\curveto(793.66790134,338.11133874)(793.73790127,338.13133872)(793.81790771,338.15134644)
\lineto(794.04290771,338.15134644)
\curveto(794.16290084,338.1513387)(794.26790074,338.1463387)(794.35790771,338.13634644)
\curveto(794.45790055,338.12633872)(794.53290047,338.08133877)(794.58290771,338.00134644)
\curveto(794.63290037,337.9513389)(794.65790035,337.87633897)(794.65790771,337.77634644)
\lineto(794.65790771,337.49134644)
\lineto(794.65790771,336.47134644)
\lineto(794.65790771,332.43634644)
\lineto(794.65790771,331.08634644)
\curveto(794.65790035,330.96634588)(794.65290035,330.851346)(794.64290771,330.74134644)
\curveto(794.64290036,330.64134621)(794.6079004,330.56634628)(794.53790771,330.51634644)
\curveto(794.49790051,330.48634636)(794.43790057,330.46134639)(794.35790771,330.44134644)
\curveto(794.27790073,330.43134642)(794.18790082,330.42134643)(794.08790771,330.41134644)
\curveto(793.99790101,330.41134644)(793.9079011,330.41634643)(793.81790771,330.42634644)
\curveto(793.73790127,330.43634641)(793.67790133,330.45634639)(793.63790771,330.48634644)
\curveto(793.58790142,330.52634632)(793.54290146,330.59134626)(793.50290771,330.68134644)
\curveto(793.49290151,330.72134613)(793.48290152,330.77634607)(793.47290771,330.84634644)
\curveto(793.47290153,330.91634593)(793.46790154,330.98134587)(793.45790771,331.04134644)
\curveto(793.44790156,331.11134574)(793.42790158,331.16634568)(793.39790771,331.20634644)
\curveto(793.36790164,331.2463456)(793.32290168,331.26134559)(793.26290771,331.25134644)
\curveto(793.18290182,331.23134562)(793.1029019,331.17134568)(793.02290771,331.07134644)
\curveto(792.94290206,330.98134587)(792.86790214,330.91134594)(792.79790771,330.86134644)
\curveto(792.57790243,330.70134615)(792.32790268,330.56134629)(792.04790771,330.44134644)
\curveto(791.93790307,330.39134646)(791.82290318,330.36134649)(791.70290771,330.35134644)
\curveto(791.59290341,330.33134652)(791.47790353,330.30634654)(791.35790771,330.27634644)
\curveto(791.3079037,330.26634658)(791.25290375,330.26634658)(791.19290771,330.27634644)
\curveto(791.14290386,330.28634656)(791.09290391,330.28134657)(791.04290771,330.26134644)
\curveto(790.94290406,330.24134661)(790.85290415,330.24134661)(790.77290771,330.26134644)
\lineto(790.62290771,330.26134644)
\curveto(790.57290443,330.28134657)(790.51290449,330.29134656)(790.44290771,330.29134644)
\curveto(790.38290462,330.29134656)(790.32790468,330.29634655)(790.27790771,330.30634644)
\curveto(790.23790477,330.32634652)(790.19790481,330.33634651)(790.15790771,330.33634644)
\curveto(790.12790488,330.32634652)(790.08790492,330.33134652)(790.03790771,330.35134644)
\lineto(789.79790771,330.41134644)
\curveto(789.72790528,330.43134642)(789.65290535,330.46134639)(789.57290771,330.50134644)
\curveto(789.31290569,330.61134624)(789.09290591,330.75634609)(788.91290771,330.93634644)
\curveto(788.74290626,331.12634572)(788.6029064,331.3513455)(788.49290771,331.61134644)
\curveto(788.45290655,331.70134515)(788.42290658,331.79134506)(788.40290771,331.88134644)
\lineto(788.34290771,332.18134644)
\curveto(788.32290668,332.24134461)(788.31290669,332.29634455)(788.31290771,332.34634644)
\curveto(788.32290668,332.40634444)(788.31790669,332.47134438)(788.29790771,332.54134644)
\curveto(788.28790672,332.56134429)(788.28290672,332.58634426)(788.28290771,332.61634644)
\curveto(788.28290672,332.65634419)(788.27790673,332.69134416)(788.26790771,332.72134644)
\lineto(788.26790771,332.87134644)
\curveto(788.25790675,332.91134394)(788.25290675,332.95634389)(788.25290771,333.00634644)
\curveto(788.26290674,333.06634378)(788.26790674,333.12134373)(788.26790771,333.17134644)
\lineto(788.26790771,333.77134644)
\lineto(788.26790771,336.53134644)
\lineto(788.26790771,337.49134644)
\lineto(788.26790771,337.76134644)
\curveto(788.26790674,337.851339)(788.28790672,337.92633892)(788.32790771,337.98634644)
\curveto(788.36790664,338.05633879)(788.44290656,338.10633874)(788.55290771,338.13634644)
\curveto(788.57290643,338.1463387)(788.59290641,338.1463387)(788.61290771,338.13634644)
\curveto(788.63290637,338.13633871)(788.65290635,338.14133871)(788.67290771,338.15134644)
}
}
{
\newrgbcolor{curcolor}{0 0 0}
\pscustom[linestyle=none,fillstyle=solid,fillcolor=curcolor]
{
\newpath
\moveto(800.20251709,338.33134644)
\curveto(800.4325123,338.33133852)(800.56251217,338.27133858)(800.59251709,338.15134644)
\curveto(800.62251211,338.04133881)(800.63751209,337.87633897)(800.63751709,337.65634644)
\lineto(800.63751709,337.37134644)
\curveto(800.63751209,337.28133957)(800.61251212,337.20633964)(800.56251709,337.14634644)
\curveto(800.50251223,337.06633978)(800.41751231,337.02133983)(800.30751709,337.01134644)
\curveto(800.19751253,337.01133984)(800.08751264,336.99633985)(799.97751709,336.96634644)
\curveto(799.83751289,336.93633991)(799.70251303,336.90633994)(799.57251709,336.87634644)
\curveto(799.45251328,336.84634)(799.33751339,336.80634004)(799.22751709,336.75634644)
\curveto(798.93751379,336.62634022)(798.70251403,336.4463404)(798.52251709,336.21634644)
\curveto(798.34251439,335.99634085)(798.18751454,335.74134111)(798.05751709,335.45134644)
\curveto(798.01751471,335.34134151)(797.98751474,335.22634162)(797.96751709,335.10634644)
\curveto(797.94751478,334.99634185)(797.92251481,334.88134197)(797.89251709,334.76134644)
\curveto(797.88251485,334.71134214)(797.87751485,334.66134219)(797.87751709,334.61134644)
\curveto(797.88751484,334.56134229)(797.88751484,334.51134234)(797.87751709,334.46134644)
\curveto(797.84751488,334.34134251)(797.8325149,334.20134265)(797.83251709,334.04134644)
\curveto(797.84251489,333.89134296)(797.84751488,333.7463431)(797.84751709,333.60634644)
\lineto(797.84751709,331.76134644)
\lineto(797.84751709,331.41634644)
\curveto(797.84751488,331.29634555)(797.84251489,331.18134567)(797.83251709,331.07134644)
\curveto(797.82251491,330.96134589)(797.81751491,330.86634598)(797.81751709,330.78634644)
\curveto(797.8275149,330.70634614)(797.80751492,330.63634621)(797.75751709,330.57634644)
\curveto(797.70751502,330.50634634)(797.6275151,330.46634638)(797.51751709,330.45634644)
\curveto(797.41751531,330.4463464)(797.30751542,330.44134641)(797.18751709,330.44134644)
\lineto(796.91751709,330.44134644)
\curveto(796.86751586,330.46134639)(796.81751591,330.47634637)(796.76751709,330.48634644)
\curveto(796.727516,330.50634634)(796.69751603,330.53134632)(796.67751709,330.56134644)
\curveto(796.6275161,330.63134622)(796.59751613,330.71634613)(796.58751709,330.81634644)
\lineto(796.58751709,331.14634644)
\lineto(796.58751709,332.30134644)
\lineto(796.58751709,336.45634644)
\lineto(796.58751709,337.49134644)
\lineto(796.58751709,337.79134644)
\curveto(796.59751613,337.89133896)(796.6275161,337.97633887)(796.67751709,338.04634644)
\curveto(796.70751602,338.08633876)(796.75751597,338.11633873)(796.82751709,338.13634644)
\curveto(796.90751582,338.15633869)(796.99251574,338.16633868)(797.08251709,338.16634644)
\curveto(797.17251556,338.17633867)(797.26251547,338.17633867)(797.35251709,338.16634644)
\curveto(797.44251529,338.15633869)(797.51251522,338.14133871)(797.56251709,338.12134644)
\curveto(797.64251509,338.09133876)(797.69251504,338.03133882)(797.71251709,337.94134644)
\curveto(797.74251499,337.86133899)(797.75751497,337.77133908)(797.75751709,337.67134644)
\lineto(797.75751709,337.37134644)
\curveto(797.75751497,337.27133958)(797.77751495,337.18133967)(797.81751709,337.10134644)
\curveto(797.8275149,337.08133977)(797.83751489,337.06633978)(797.84751709,337.05634644)
\lineto(797.89251709,337.01134644)
\curveto(798.00251473,337.01133984)(798.09251464,337.05633979)(798.16251709,337.14634644)
\curveto(798.2325145,337.2463396)(798.29251444,337.32633952)(798.34251709,337.38634644)
\lineto(798.43251709,337.47634644)
\curveto(798.52251421,337.58633926)(798.64751408,337.70133915)(798.80751709,337.82134644)
\curveto(798.96751376,337.94133891)(799.11751361,338.03133882)(799.25751709,338.09134644)
\curveto(799.34751338,338.14133871)(799.44251329,338.17633867)(799.54251709,338.19634644)
\curveto(799.64251309,338.22633862)(799.74751298,338.25633859)(799.85751709,338.28634644)
\curveto(799.91751281,338.29633855)(799.97751275,338.30133855)(800.03751709,338.30134644)
\curveto(800.09751263,338.31133854)(800.15251258,338.32133853)(800.20251709,338.33134644)
}
}
{
\newrgbcolor{curcolor}{0 0 0}
\pscustom[linestyle=none,fillstyle=solid,fillcolor=curcolor]
{
\newpath
\moveto(803.99728271,338.33134644)
\curveto(804.71727865,338.34133851)(805.32227804,338.25633859)(805.81228271,338.07634644)
\curveto(806.30227706,337.90633894)(806.68227668,337.60133925)(806.95228271,337.16134644)
\curveto(807.02227634,337.0513398)(807.07727629,336.93633991)(807.11728271,336.81634644)
\curveto(807.15727621,336.70634014)(807.19727617,336.58134027)(807.23728271,336.44134644)
\curveto(807.25727611,336.37134048)(807.2622761,336.29634055)(807.25228271,336.21634644)
\curveto(807.24227612,336.1463407)(807.22727614,336.09134076)(807.20728271,336.05134644)
\curveto(807.18727618,336.03134082)(807.1622762,336.01134084)(807.13228271,335.99134644)
\curveto(807.10227626,335.98134087)(807.07727629,335.96634088)(807.05728271,335.94634644)
\curveto(807.00727636,335.92634092)(806.95727641,335.92134093)(806.90728271,335.93134644)
\curveto(806.85727651,335.94134091)(806.80727656,335.94134091)(806.75728271,335.93134644)
\curveto(806.67727669,335.91134094)(806.57227679,335.90634094)(806.44228271,335.91634644)
\curveto(806.31227705,335.93634091)(806.22227714,335.96134089)(806.17228271,335.99134644)
\curveto(806.09227727,336.04134081)(806.03727733,336.10634074)(806.00728271,336.18634644)
\curveto(805.98727738,336.27634057)(805.95227741,336.36134049)(805.90228271,336.44134644)
\curveto(805.81227755,336.60134025)(805.68727768,336.7463401)(805.52728271,336.87634644)
\curveto(805.41727795,336.95633989)(805.29727807,337.01633983)(805.16728271,337.05634644)
\curveto(805.03727833,337.09633975)(804.89727847,337.13633971)(804.74728271,337.17634644)
\curveto(804.69727867,337.19633965)(804.64727872,337.20133965)(804.59728271,337.19134644)
\curveto(804.54727882,337.19133966)(804.49727887,337.19633965)(804.44728271,337.20634644)
\curveto(804.38727898,337.22633962)(804.31227905,337.23633961)(804.22228271,337.23634644)
\curveto(804.13227923,337.23633961)(804.05727931,337.22633962)(803.99728271,337.20634644)
\lineto(803.90728271,337.20634644)
\lineto(803.75728271,337.17634644)
\curveto(803.70727966,337.17633967)(803.65727971,337.17133968)(803.60728271,337.16134644)
\curveto(803.34728002,337.10133975)(803.13228023,337.01633983)(802.96228271,336.90634644)
\curveto(802.79228057,336.79634005)(802.67728069,336.61134024)(802.61728271,336.35134644)
\curveto(802.59728077,336.28134057)(802.59228077,336.21134064)(802.60228271,336.14134644)
\curveto(802.62228074,336.07134078)(802.64228072,336.01134084)(802.66228271,335.96134644)
\curveto(802.72228064,335.81134104)(802.79228057,335.70134115)(802.87228271,335.63134644)
\curveto(802.9622804,335.57134128)(803.07228029,335.50134135)(803.20228271,335.42134644)
\curveto(803.36228,335.32134153)(803.54227982,335.2463416)(803.74228271,335.19634644)
\curveto(803.94227942,335.15634169)(804.14227922,335.10634174)(804.34228271,335.04634644)
\curveto(804.47227889,335.00634184)(804.60227876,334.97634187)(804.73228271,334.95634644)
\curveto(804.8622785,334.93634191)(804.99227837,334.90634194)(805.12228271,334.86634644)
\curveto(805.33227803,334.80634204)(805.53727783,334.7463421)(805.73728271,334.68634644)
\curveto(805.93727743,334.63634221)(806.13727723,334.57134228)(806.33728271,334.49134644)
\lineto(806.48728271,334.43134644)
\curveto(806.53727683,334.41134244)(806.58727678,334.38634246)(806.63728271,334.35634644)
\curveto(806.83727653,334.23634261)(807.01227635,334.10134275)(807.16228271,333.95134644)
\curveto(807.31227605,333.80134305)(807.43727593,333.61134324)(807.53728271,333.38134644)
\curveto(807.55727581,333.31134354)(807.57727579,333.21634363)(807.59728271,333.09634644)
\curveto(807.61727575,333.02634382)(807.62727574,332.9513439)(807.62728271,332.87134644)
\curveto(807.63727573,332.80134405)(807.64227572,332.72134413)(807.64228271,332.63134644)
\lineto(807.64228271,332.48134644)
\curveto(807.62227574,332.41134444)(807.61227575,332.34134451)(807.61228271,332.27134644)
\curveto(807.61227575,332.20134465)(807.60227576,332.13134472)(807.58228271,332.06134644)
\curveto(807.55227581,331.9513449)(807.51727585,331.846345)(807.47728271,331.74634644)
\curveto(807.43727593,331.6463452)(807.39227597,331.55634529)(807.34228271,331.47634644)
\curveto(807.18227618,331.21634563)(806.97727639,331.00634584)(806.72728271,330.84634644)
\curveto(806.47727689,330.69634615)(806.19727717,330.56634628)(805.88728271,330.45634644)
\curveto(805.79727757,330.42634642)(805.70227766,330.40634644)(805.60228271,330.39634644)
\curveto(805.51227785,330.37634647)(805.42227794,330.3513465)(805.33228271,330.32134644)
\curveto(805.23227813,330.30134655)(805.13227823,330.29134656)(805.03228271,330.29134644)
\curveto(804.93227843,330.29134656)(804.83227853,330.28134657)(804.73228271,330.26134644)
\lineto(804.58228271,330.26134644)
\curveto(804.53227883,330.2513466)(804.4622789,330.2463466)(804.37228271,330.24634644)
\curveto(804.28227908,330.2463466)(804.21227915,330.2513466)(804.16228271,330.26134644)
\lineto(803.99728271,330.26134644)
\curveto(803.93727943,330.28134657)(803.87227949,330.29134656)(803.80228271,330.29134644)
\curveto(803.73227963,330.28134657)(803.67227969,330.28634656)(803.62228271,330.30634644)
\curveto(803.57227979,330.31634653)(803.50727986,330.32134653)(803.42728271,330.32134644)
\lineto(803.18728271,330.38134644)
\curveto(803.11728025,330.39134646)(803.04228032,330.41134644)(802.96228271,330.44134644)
\curveto(802.65228071,330.54134631)(802.38228098,330.66634618)(802.15228271,330.81634644)
\curveto(801.92228144,330.96634588)(801.72228164,331.16134569)(801.55228271,331.40134644)
\curveto(801.4622819,331.53134532)(801.38728198,331.66634518)(801.32728271,331.80634644)
\curveto(801.2672821,331.9463449)(801.21228215,332.10134475)(801.16228271,332.27134644)
\curveto(801.14228222,332.33134452)(801.13228223,332.40134445)(801.13228271,332.48134644)
\curveto(801.14228222,332.57134428)(801.15728221,332.64134421)(801.17728271,332.69134644)
\curveto(801.20728216,332.73134412)(801.25728211,332.77134408)(801.32728271,332.81134644)
\curveto(801.37728199,332.83134402)(801.44728192,332.84134401)(801.53728271,332.84134644)
\curveto(801.62728174,332.851344)(801.71728165,332.851344)(801.80728271,332.84134644)
\curveto(801.89728147,332.83134402)(801.98228138,332.81634403)(802.06228271,332.79634644)
\curveto(802.15228121,332.78634406)(802.21228115,332.77134408)(802.24228271,332.75134644)
\curveto(802.31228105,332.70134415)(802.35728101,332.62634422)(802.37728271,332.52634644)
\curveto(802.40728096,332.43634441)(802.44228092,332.3513445)(802.48228271,332.27134644)
\curveto(802.58228078,332.0513448)(802.71728065,331.88134497)(802.88728271,331.76134644)
\curveto(803.00728036,331.67134518)(803.14228022,331.60134525)(803.29228271,331.55134644)
\curveto(803.44227992,331.50134535)(803.60227976,331.4513454)(803.77228271,331.40134644)
\lineto(804.08728271,331.35634644)
\lineto(804.17728271,331.35634644)
\curveto(804.24727912,331.33634551)(804.33727903,331.32634552)(804.44728271,331.32634644)
\curveto(804.5672788,331.32634552)(804.6672787,331.33634551)(804.74728271,331.35634644)
\curveto(804.81727855,331.35634549)(804.87227849,331.36134549)(804.91228271,331.37134644)
\curveto(804.97227839,331.38134547)(805.03227833,331.38634546)(805.09228271,331.38634644)
\curveto(805.15227821,331.39634545)(805.20727816,331.40634544)(805.25728271,331.41634644)
\curveto(805.54727782,331.49634535)(805.77727759,331.60134525)(805.94728271,331.73134644)
\curveto(806.11727725,331.86134499)(806.23727713,332.08134477)(806.30728271,332.39134644)
\curveto(806.32727704,332.44134441)(806.33227703,332.49634435)(806.32228271,332.55634644)
\curveto(806.31227705,332.61634423)(806.30227706,332.66134419)(806.29228271,332.69134644)
\curveto(806.24227712,332.88134397)(806.17227719,333.02134383)(806.08228271,333.11134644)
\curveto(805.99227737,333.21134364)(805.87727749,333.30134355)(805.73728271,333.38134644)
\curveto(805.64727772,333.44134341)(805.54727782,333.49134336)(805.43728271,333.53134644)
\lineto(805.10728271,333.65134644)
\curveto(805.07727829,333.66134319)(805.04727832,333.66634318)(805.01728271,333.66634644)
\curveto(804.99727837,333.66634318)(804.97227839,333.67634317)(804.94228271,333.69634644)
\curveto(804.60227876,333.80634304)(804.24727912,333.88634296)(803.87728271,333.93634644)
\curveto(803.51727985,333.99634285)(803.17728019,334.09134276)(802.85728271,334.22134644)
\curveto(802.75728061,334.26134259)(802.6622807,334.29634255)(802.57228271,334.32634644)
\curveto(802.48228088,334.35634249)(802.39728097,334.39634245)(802.31728271,334.44634644)
\curveto(802.12728124,334.55634229)(801.95228141,334.68134217)(801.79228271,334.82134644)
\curveto(801.63228173,334.96134189)(801.50728186,335.13634171)(801.41728271,335.34634644)
\curveto(801.38728198,335.41634143)(801.362282,335.48634136)(801.34228271,335.55634644)
\curveto(801.33228203,335.62634122)(801.31728205,335.70134115)(801.29728271,335.78134644)
\curveto(801.2672821,335.90134095)(801.25728211,336.03634081)(801.26728271,336.18634644)
\curveto(801.27728209,336.3463405)(801.29228207,336.48134037)(801.31228271,336.59134644)
\curveto(801.33228203,336.64134021)(801.34228202,336.68134017)(801.34228271,336.71134644)
\curveto(801.35228201,336.7513401)(801.367282,336.79134006)(801.38728271,336.83134644)
\curveto(801.47728189,337.06133979)(801.59728177,337.26133959)(801.74728271,337.43134644)
\curveto(801.90728146,337.60133925)(802.08728128,337.7513391)(802.28728271,337.88134644)
\curveto(802.43728093,337.97133888)(802.60228076,338.04133881)(802.78228271,338.09134644)
\curveto(802.9622804,338.1513387)(803.15228021,338.20633864)(803.35228271,338.25634644)
\curveto(803.42227994,338.26633858)(803.48727988,338.27633857)(803.54728271,338.28634644)
\curveto(803.61727975,338.29633855)(803.69227967,338.30633854)(803.77228271,338.31634644)
\curveto(803.80227956,338.32633852)(803.84227952,338.32633852)(803.89228271,338.31634644)
\curveto(803.94227942,338.30633854)(803.97727939,338.31133854)(803.99728271,338.33134644)
}
}
{
\newrgbcolor{curcolor}{0 0 0}
\pscustom[linestyle=none,fillstyle=solid,fillcolor=curcolor]
{
\newpath
\moveto(816.19228271,334.62634644)
\curveto(816.21227465,334.56634228)(816.22227464,334.47134238)(816.22228271,334.34134644)
\curveto(816.22227464,334.22134263)(816.21727465,334.13634271)(816.20728271,334.08634644)
\lineto(816.20728271,333.93634644)
\curveto(816.19727467,333.85634299)(816.18727468,333.78134307)(816.17728271,333.71134644)
\curveto(816.17727469,333.6513432)(816.17227469,333.58134327)(816.16228271,333.50134644)
\curveto(816.14227472,333.44134341)(816.12727474,333.38134347)(816.11728271,333.32134644)
\curveto(816.11727475,333.26134359)(816.10727476,333.20134365)(816.08728271,333.14134644)
\curveto(816.04727482,333.01134384)(816.01227485,332.88134397)(815.98228271,332.75134644)
\curveto(815.95227491,332.62134423)(815.91227495,332.50134435)(815.86228271,332.39134644)
\curveto(815.65227521,331.91134494)(815.37227549,331.50634534)(815.02228271,331.17634644)
\curveto(814.67227619,330.85634599)(814.24227662,330.61134624)(813.73228271,330.44134644)
\curveto(813.62227724,330.40134645)(813.50227736,330.37134648)(813.37228271,330.35134644)
\curveto(813.25227761,330.33134652)(813.12727774,330.31134654)(812.99728271,330.29134644)
\curveto(812.93727793,330.28134657)(812.87227799,330.27634657)(812.80228271,330.27634644)
\curveto(812.74227812,330.26634658)(812.68227818,330.26134659)(812.62228271,330.26134644)
\curveto(812.58227828,330.2513466)(812.52227834,330.2463466)(812.44228271,330.24634644)
\curveto(812.37227849,330.2463466)(812.32227854,330.2513466)(812.29228271,330.26134644)
\curveto(812.25227861,330.27134658)(812.21227865,330.27634657)(812.17228271,330.27634644)
\curveto(812.13227873,330.26634658)(812.09727877,330.26634658)(812.06728271,330.27634644)
\lineto(811.97728271,330.27634644)
\lineto(811.61728271,330.32134644)
\curveto(811.47727939,330.36134649)(811.34227952,330.40134645)(811.21228271,330.44134644)
\curveto(811.08227978,330.48134637)(810.95727991,330.52634632)(810.83728271,330.57634644)
\curveto(810.38728048,330.77634607)(810.01728085,331.03634581)(809.72728271,331.35634644)
\curveto(809.43728143,331.67634517)(809.19728167,332.06634478)(809.00728271,332.52634644)
\curveto(808.95728191,332.62634422)(808.91728195,332.72634412)(808.88728271,332.82634644)
\curveto(808.867282,332.92634392)(808.84728202,333.03134382)(808.82728271,333.14134644)
\curveto(808.80728206,333.18134367)(808.79728207,333.21134364)(808.79728271,333.23134644)
\curveto(808.80728206,333.26134359)(808.80728206,333.29634355)(808.79728271,333.33634644)
\curveto(808.77728209,333.41634343)(808.7622821,333.49634335)(808.75228271,333.57634644)
\curveto(808.75228211,333.66634318)(808.74228212,333.7513431)(808.72228271,333.83134644)
\lineto(808.72228271,333.95134644)
\curveto(808.72228214,333.99134286)(808.71728215,334.03634281)(808.70728271,334.08634644)
\curveto(808.69728217,334.13634271)(808.69228217,334.22134263)(808.69228271,334.34134644)
\curveto(808.69228217,334.47134238)(808.70228216,334.56634228)(808.72228271,334.62634644)
\curveto(808.74228212,334.69634215)(808.74728212,334.76634208)(808.73728271,334.83634644)
\curveto(808.72728214,334.90634194)(808.73228213,334.97634187)(808.75228271,335.04634644)
\curveto(808.7622821,335.09634175)(808.7672821,335.13634171)(808.76728271,335.16634644)
\curveto(808.77728209,335.20634164)(808.78728208,335.2513416)(808.79728271,335.30134644)
\curveto(808.82728204,335.42134143)(808.85228201,335.54134131)(808.87228271,335.66134644)
\curveto(808.90228196,335.78134107)(808.94228192,335.89634095)(808.99228271,336.00634644)
\curveto(809.14228172,336.37634047)(809.32228154,336.70634014)(809.53228271,336.99634644)
\curveto(809.75228111,337.29633955)(810.01728085,337.5463393)(810.32728271,337.74634644)
\curveto(810.44728042,337.82633902)(810.57228029,337.89133896)(810.70228271,337.94134644)
\curveto(810.83228003,338.00133885)(810.9672799,338.06133879)(811.10728271,338.12134644)
\curveto(811.22727964,338.17133868)(811.35727951,338.20133865)(811.49728271,338.21134644)
\curveto(811.63727923,338.23133862)(811.77727909,338.26133859)(811.91728271,338.30134644)
\lineto(812.11228271,338.30134644)
\curveto(812.18227868,338.31133854)(812.24727862,338.32133853)(812.30728271,338.33134644)
\curveto(813.19727767,338.34133851)(813.93727693,338.15633869)(814.52728271,337.77634644)
\curveto(815.11727575,337.39633945)(815.54227532,336.90133995)(815.80228271,336.29134644)
\curveto(815.85227501,336.19134066)(815.89227497,336.09134076)(815.92228271,335.99134644)
\curveto(815.95227491,335.89134096)(815.98727488,335.78634106)(816.02728271,335.67634644)
\curveto(816.05727481,335.56634128)(816.08227478,335.4463414)(816.10228271,335.31634644)
\curveto(816.12227474,335.19634165)(816.14727472,335.07134178)(816.17728271,334.94134644)
\curveto(816.18727468,334.89134196)(816.18727468,334.83634201)(816.17728271,334.77634644)
\curveto(816.17727469,334.72634212)(816.18227468,334.67634217)(816.19228271,334.62634644)
\moveto(814.85728271,333.77134644)
\curveto(814.87727599,333.84134301)(814.88227598,333.92134293)(814.87228271,334.01134644)
\lineto(814.87228271,334.26634644)
\curveto(814.87227599,334.65634219)(814.83727603,334.98634186)(814.76728271,335.25634644)
\curveto(814.73727613,335.33634151)(814.71227615,335.41634143)(814.69228271,335.49634644)
\curveto(814.67227619,335.57634127)(814.64727622,335.6513412)(814.61728271,335.72134644)
\curveto(814.33727653,336.37134048)(813.89227697,336.82134003)(813.28228271,337.07134644)
\curveto(813.21227765,337.10133975)(813.13727773,337.12133973)(813.05728271,337.13134644)
\lineto(812.81728271,337.19134644)
\curveto(812.73727813,337.21133964)(812.65227821,337.22133963)(812.56228271,337.22134644)
\lineto(812.29228271,337.22134644)
\lineto(812.02228271,337.17634644)
\curveto(811.92227894,337.15633969)(811.82727904,337.13133972)(811.73728271,337.10134644)
\curveto(811.65727921,337.08133977)(811.57727929,337.0513398)(811.49728271,337.01134644)
\curveto(811.42727944,336.99133986)(811.3622795,336.96133989)(811.30228271,336.92134644)
\curveto(811.24227962,336.88133997)(811.18727968,336.84134001)(811.13728271,336.80134644)
\curveto(810.89727997,336.63134022)(810.70228016,336.42634042)(810.55228271,336.18634644)
\curveto(810.40228046,335.9463409)(810.27228059,335.66634118)(810.16228271,335.34634644)
\curveto(810.13228073,335.2463416)(810.11228075,335.14134171)(810.10228271,335.03134644)
\curveto(810.09228077,334.93134192)(810.07728079,334.82634202)(810.05728271,334.71634644)
\curveto(810.04728082,334.67634217)(810.04228082,334.61134224)(810.04228271,334.52134644)
\curveto(810.03228083,334.49134236)(810.02728084,334.45634239)(810.02728271,334.41634644)
\curveto(810.03728083,334.37634247)(810.04228082,334.33134252)(810.04228271,334.28134644)
\lineto(810.04228271,333.98134644)
\curveto(810.04228082,333.88134297)(810.05228081,333.79134306)(810.07228271,333.71134644)
\lineto(810.10228271,333.53134644)
\curveto(810.12228074,333.43134342)(810.13728073,333.33134352)(810.14728271,333.23134644)
\curveto(810.1672807,333.14134371)(810.19728067,333.05634379)(810.23728271,332.97634644)
\curveto(810.33728053,332.73634411)(810.45228041,332.51134434)(810.58228271,332.30134644)
\curveto(810.72228014,332.09134476)(810.89227997,331.91634493)(811.09228271,331.77634644)
\curveto(811.14227972,331.7463451)(811.18727968,331.72134513)(811.22728271,331.70134644)
\curveto(811.2672796,331.68134517)(811.31227955,331.65634519)(811.36228271,331.62634644)
\curveto(811.44227942,331.57634527)(811.52727934,331.53134532)(811.61728271,331.49134644)
\curveto(811.71727915,331.46134539)(811.82227904,331.43134542)(811.93228271,331.40134644)
\curveto(811.98227888,331.38134547)(812.02727884,331.37134548)(812.06728271,331.37134644)
\curveto(812.11727875,331.38134547)(812.1672787,331.38134547)(812.21728271,331.37134644)
\curveto(812.24727862,331.36134549)(812.30727856,331.3513455)(812.39728271,331.34134644)
\curveto(812.49727837,331.33134552)(812.57227829,331.33634551)(812.62228271,331.35634644)
\curveto(812.6622782,331.36634548)(812.70227816,331.36634548)(812.74228271,331.35634644)
\curveto(812.78227808,331.35634549)(812.82227804,331.36634548)(812.86228271,331.38634644)
\curveto(812.94227792,331.40634544)(813.02227784,331.42134543)(813.10228271,331.43134644)
\curveto(813.18227768,331.4513454)(813.25727761,331.47634537)(813.32728271,331.50634644)
\curveto(813.6672772,331.6463452)(813.94227692,331.84134501)(814.15228271,332.09134644)
\curveto(814.3622765,332.34134451)(814.53727633,332.63634421)(814.67728271,332.97634644)
\curveto(814.72727614,333.09634375)(814.75727611,333.22134363)(814.76728271,333.35134644)
\curveto(814.78727608,333.49134336)(814.81727605,333.63134322)(814.85728271,333.77134644)
}
}
{
\newrgbcolor{curcolor}{0 0 0}
\pscustom[linestyle=none,fillstyle=solid,fillcolor=curcolor]
{
\newpath
\moveto(820.11056396,338.33134644)
\curveto(820.8305599,338.34133851)(821.43555929,338.25633859)(821.92556396,338.07634644)
\curveto(822.41555831,337.90633894)(822.79555793,337.60133925)(823.06556396,337.16134644)
\curveto(823.13555759,337.0513398)(823.19055754,336.93633991)(823.23056396,336.81634644)
\curveto(823.27055746,336.70634014)(823.31055742,336.58134027)(823.35056396,336.44134644)
\curveto(823.37055736,336.37134048)(823.37555735,336.29634055)(823.36556396,336.21634644)
\curveto(823.35555737,336.1463407)(823.34055739,336.09134076)(823.32056396,336.05134644)
\curveto(823.30055743,336.03134082)(823.27555745,336.01134084)(823.24556396,335.99134644)
\curveto(823.21555751,335.98134087)(823.19055754,335.96634088)(823.17056396,335.94634644)
\curveto(823.12055761,335.92634092)(823.07055766,335.92134093)(823.02056396,335.93134644)
\curveto(822.97055776,335.94134091)(822.92055781,335.94134091)(822.87056396,335.93134644)
\curveto(822.79055794,335.91134094)(822.68555804,335.90634094)(822.55556396,335.91634644)
\curveto(822.4255583,335.93634091)(822.33555839,335.96134089)(822.28556396,335.99134644)
\curveto(822.20555852,336.04134081)(822.15055858,336.10634074)(822.12056396,336.18634644)
\curveto(822.10055863,336.27634057)(822.06555866,336.36134049)(822.01556396,336.44134644)
\curveto(821.9255588,336.60134025)(821.80055893,336.7463401)(821.64056396,336.87634644)
\curveto(821.5305592,336.95633989)(821.41055932,337.01633983)(821.28056396,337.05634644)
\curveto(821.15055958,337.09633975)(821.01055972,337.13633971)(820.86056396,337.17634644)
\curveto(820.81055992,337.19633965)(820.76055997,337.20133965)(820.71056396,337.19134644)
\curveto(820.66056007,337.19133966)(820.61056012,337.19633965)(820.56056396,337.20634644)
\curveto(820.50056023,337.22633962)(820.4255603,337.23633961)(820.33556396,337.23634644)
\curveto(820.24556048,337.23633961)(820.17056056,337.22633962)(820.11056396,337.20634644)
\lineto(820.02056396,337.20634644)
\lineto(819.87056396,337.17634644)
\curveto(819.82056091,337.17633967)(819.77056096,337.17133968)(819.72056396,337.16134644)
\curveto(819.46056127,337.10133975)(819.24556148,337.01633983)(819.07556396,336.90634644)
\curveto(818.90556182,336.79634005)(818.79056194,336.61134024)(818.73056396,336.35134644)
\curveto(818.71056202,336.28134057)(818.70556202,336.21134064)(818.71556396,336.14134644)
\curveto(818.73556199,336.07134078)(818.75556197,336.01134084)(818.77556396,335.96134644)
\curveto(818.83556189,335.81134104)(818.90556182,335.70134115)(818.98556396,335.63134644)
\curveto(819.07556165,335.57134128)(819.18556154,335.50134135)(819.31556396,335.42134644)
\curveto(819.47556125,335.32134153)(819.65556107,335.2463416)(819.85556396,335.19634644)
\curveto(820.05556067,335.15634169)(820.25556047,335.10634174)(820.45556396,335.04634644)
\curveto(820.58556014,335.00634184)(820.71556001,334.97634187)(820.84556396,334.95634644)
\curveto(820.97555975,334.93634191)(821.10555962,334.90634194)(821.23556396,334.86634644)
\curveto(821.44555928,334.80634204)(821.65055908,334.7463421)(821.85056396,334.68634644)
\curveto(822.05055868,334.63634221)(822.25055848,334.57134228)(822.45056396,334.49134644)
\lineto(822.60056396,334.43134644)
\curveto(822.65055808,334.41134244)(822.70055803,334.38634246)(822.75056396,334.35634644)
\curveto(822.95055778,334.23634261)(823.1255576,334.10134275)(823.27556396,333.95134644)
\curveto(823.4255573,333.80134305)(823.55055718,333.61134324)(823.65056396,333.38134644)
\curveto(823.67055706,333.31134354)(823.69055704,333.21634363)(823.71056396,333.09634644)
\curveto(823.730557,333.02634382)(823.74055699,332.9513439)(823.74056396,332.87134644)
\curveto(823.75055698,332.80134405)(823.75555697,332.72134413)(823.75556396,332.63134644)
\lineto(823.75556396,332.48134644)
\curveto(823.73555699,332.41134444)(823.725557,332.34134451)(823.72556396,332.27134644)
\curveto(823.725557,332.20134465)(823.71555701,332.13134472)(823.69556396,332.06134644)
\curveto(823.66555706,331.9513449)(823.6305571,331.846345)(823.59056396,331.74634644)
\curveto(823.55055718,331.6463452)(823.50555722,331.55634529)(823.45556396,331.47634644)
\curveto(823.29555743,331.21634563)(823.09055764,331.00634584)(822.84056396,330.84634644)
\curveto(822.59055814,330.69634615)(822.31055842,330.56634628)(822.00056396,330.45634644)
\curveto(821.91055882,330.42634642)(821.81555891,330.40634644)(821.71556396,330.39634644)
\curveto(821.6255591,330.37634647)(821.53555919,330.3513465)(821.44556396,330.32134644)
\curveto(821.34555938,330.30134655)(821.24555948,330.29134656)(821.14556396,330.29134644)
\curveto(821.04555968,330.29134656)(820.94555978,330.28134657)(820.84556396,330.26134644)
\lineto(820.69556396,330.26134644)
\curveto(820.64556008,330.2513466)(820.57556015,330.2463466)(820.48556396,330.24634644)
\curveto(820.39556033,330.2463466)(820.3255604,330.2513466)(820.27556396,330.26134644)
\lineto(820.11056396,330.26134644)
\curveto(820.05056068,330.28134657)(819.98556074,330.29134656)(819.91556396,330.29134644)
\curveto(819.84556088,330.28134657)(819.78556094,330.28634656)(819.73556396,330.30634644)
\curveto(819.68556104,330.31634653)(819.62056111,330.32134653)(819.54056396,330.32134644)
\lineto(819.30056396,330.38134644)
\curveto(819.2305615,330.39134646)(819.15556157,330.41134644)(819.07556396,330.44134644)
\curveto(818.76556196,330.54134631)(818.49556223,330.66634618)(818.26556396,330.81634644)
\curveto(818.03556269,330.96634588)(817.83556289,331.16134569)(817.66556396,331.40134644)
\curveto(817.57556315,331.53134532)(817.50056323,331.66634518)(817.44056396,331.80634644)
\curveto(817.38056335,331.9463449)(817.3255634,332.10134475)(817.27556396,332.27134644)
\curveto(817.25556347,332.33134452)(817.24556348,332.40134445)(817.24556396,332.48134644)
\curveto(817.25556347,332.57134428)(817.27056346,332.64134421)(817.29056396,332.69134644)
\curveto(817.32056341,332.73134412)(817.37056336,332.77134408)(817.44056396,332.81134644)
\curveto(817.49056324,332.83134402)(817.56056317,332.84134401)(817.65056396,332.84134644)
\curveto(817.74056299,332.851344)(817.8305629,332.851344)(817.92056396,332.84134644)
\curveto(818.01056272,332.83134402)(818.09556263,332.81634403)(818.17556396,332.79634644)
\curveto(818.26556246,332.78634406)(818.3255624,332.77134408)(818.35556396,332.75134644)
\curveto(818.4255623,332.70134415)(818.47056226,332.62634422)(818.49056396,332.52634644)
\curveto(818.52056221,332.43634441)(818.55556217,332.3513445)(818.59556396,332.27134644)
\curveto(818.69556203,332.0513448)(818.8305619,331.88134497)(819.00056396,331.76134644)
\curveto(819.12056161,331.67134518)(819.25556147,331.60134525)(819.40556396,331.55134644)
\curveto(819.55556117,331.50134535)(819.71556101,331.4513454)(819.88556396,331.40134644)
\lineto(820.20056396,331.35634644)
\lineto(820.29056396,331.35634644)
\curveto(820.36056037,331.33634551)(820.45056028,331.32634552)(820.56056396,331.32634644)
\curveto(820.68056005,331.32634552)(820.78055995,331.33634551)(820.86056396,331.35634644)
\curveto(820.9305598,331.35634549)(820.98555974,331.36134549)(821.02556396,331.37134644)
\curveto(821.08555964,331.38134547)(821.14555958,331.38634546)(821.20556396,331.38634644)
\curveto(821.26555946,331.39634545)(821.32055941,331.40634544)(821.37056396,331.41634644)
\curveto(821.66055907,331.49634535)(821.89055884,331.60134525)(822.06056396,331.73134644)
\curveto(822.2305585,331.86134499)(822.35055838,332.08134477)(822.42056396,332.39134644)
\curveto(822.44055829,332.44134441)(822.44555828,332.49634435)(822.43556396,332.55634644)
\curveto(822.4255583,332.61634423)(822.41555831,332.66134419)(822.40556396,332.69134644)
\curveto(822.35555837,332.88134397)(822.28555844,333.02134383)(822.19556396,333.11134644)
\curveto(822.10555862,333.21134364)(821.99055874,333.30134355)(821.85056396,333.38134644)
\curveto(821.76055897,333.44134341)(821.66055907,333.49134336)(821.55056396,333.53134644)
\lineto(821.22056396,333.65134644)
\curveto(821.19055954,333.66134319)(821.16055957,333.66634318)(821.13056396,333.66634644)
\curveto(821.11055962,333.66634318)(821.08555964,333.67634317)(821.05556396,333.69634644)
\curveto(820.71556001,333.80634304)(820.36056037,333.88634296)(819.99056396,333.93634644)
\curveto(819.6305611,333.99634285)(819.29056144,334.09134276)(818.97056396,334.22134644)
\curveto(818.87056186,334.26134259)(818.77556195,334.29634255)(818.68556396,334.32634644)
\curveto(818.59556213,334.35634249)(818.51056222,334.39634245)(818.43056396,334.44634644)
\curveto(818.24056249,334.55634229)(818.06556266,334.68134217)(817.90556396,334.82134644)
\curveto(817.74556298,334.96134189)(817.62056311,335.13634171)(817.53056396,335.34634644)
\curveto(817.50056323,335.41634143)(817.47556325,335.48634136)(817.45556396,335.55634644)
\curveto(817.44556328,335.62634122)(817.4305633,335.70134115)(817.41056396,335.78134644)
\curveto(817.38056335,335.90134095)(817.37056336,336.03634081)(817.38056396,336.18634644)
\curveto(817.39056334,336.3463405)(817.40556332,336.48134037)(817.42556396,336.59134644)
\curveto(817.44556328,336.64134021)(817.45556327,336.68134017)(817.45556396,336.71134644)
\curveto(817.46556326,336.7513401)(817.48056325,336.79134006)(817.50056396,336.83134644)
\curveto(817.59056314,337.06133979)(817.71056302,337.26133959)(817.86056396,337.43134644)
\curveto(818.02056271,337.60133925)(818.20056253,337.7513391)(818.40056396,337.88134644)
\curveto(818.55056218,337.97133888)(818.71556201,338.04133881)(818.89556396,338.09134644)
\curveto(819.07556165,338.1513387)(819.26556146,338.20633864)(819.46556396,338.25634644)
\curveto(819.53556119,338.26633858)(819.60056113,338.27633857)(819.66056396,338.28634644)
\curveto(819.730561,338.29633855)(819.80556092,338.30633854)(819.88556396,338.31634644)
\curveto(819.91556081,338.32633852)(819.95556077,338.32633852)(820.00556396,338.31634644)
\curveto(820.05556067,338.30633854)(820.09056064,338.31133854)(820.11056396,338.33134644)
}
}
{
\newrgbcolor{curcolor}{0.80000001 0.80000001 0.80000001}
\pscustom[linestyle=none,fillstyle=solid,fillcolor=curcolor]
{
\newpath
\moveto(741.9732666,382.02397156)
\lineto(756.9732666,382.02397156)
\lineto(756.9732666,367.02397156)
\lineto(741.9732666,367.02397156)
\closepath
}
}
{
\newrgbcolor{curcolor}{0.7019608 0.7019608 0.7019608}
\pscustom[linestyle=none,fillstyle=solid,fillcolor=curcolor]
{
\newpath
\moveto(741.9732666,340.94264221)
\lineto(756.9732666,340.94264221)
\lineto(756.9732666,325.94264221)
\lineto(741.9732666,325.94264221)
\closepath
}
}
{
\newrgbcolor{curcolor}{0.80000001 0.80000001 0.80000001}
\pscustom[linestyle=none,fillstyle=solid,fillcolor=curcolor]
{
\newpath
\moveto(106.02189636,92.00222778)
\lineto(133.94314194,92.00222778)
\lineto(133.94314194,86.0827961)
\lineto(106.02189636,86.0827961)
\closepath
}
}
{
\newrgbcolor{curcolor}{0 0 0}
\pscustom[linestyle=none,fillstyle=solid,fillcolor=curcolor]
{
\newpath
\moveto(567.4097168,57.195)
\lineto(572.3147168,57.195)
\lineto(573.6047168,57.195)
\curveto(573.71470892,57.1949893)(573.82470881,57.1949893)(573.9347168,57.195)
\curveto(574.04470859,57.2049893)(574.1347085,57.18498931)(574.2047168,57.135)
\curveto(574.2347084,57.11498939)(574.25970837,57.08998941)(574.2797168,57.06)
\curveto(574.29970833,57.02998947)(574.31970831,56.9999895)(574.3397168,56.97)
\curveto(574.35970827,56.8999896)(574.36970826,56.78498971)(574.3697168,56.625)
\curveto(574.36970826,56.47499003)(574.35970827,56.35999014)(574.3397168,56.28)
\curveto(574.29970833,56.13999036)(574.21470842,56.05999044)(574.0847168,56.04)
\curveto(573.95470868,56.02999047)(573.79970883,56.02499047)(573.6197168,56.025)
\lineto(572.1197168,56.025)
\lineto(569.5997168,56.025)
\lineto(569.0297168,56.025)
\curveto(568.81971381,56.03499047)(568.66471397,56.00999049)(568.5647168,55.95)
\curveto(568.46471417,55.88999061)(568.40971422,55.78499071)(568.3997168,55.635)
\lineto(568.3997168,55.17)
\lineto(568.3997168,53.64)
\curveto(568.39971423,53.52999297)(568.39471424,53.3999931)(568.3847168,53.25)
\curveto(568.38471425,53.0999934)(568.39471424,52.97999352)(568.4147168,52.89)
\curveto(568.44471419,52.76999373)(568.50471413,52.68999381)(568.5947168,52.65)
\curveto(568.634714,52.62999387)(568.70471393,52.60999389)(568.8047168,52.59)
\lineto(568.9547168,52.59)
\curveto(568.99471364,52.57999392)(569.0347136,52.57499393)(569.0747168,52.575)
\curveto(569.12471351,52.58499391)(569.17471346,52.58999391)(569.2247168,52.59)
\lineto(569.7347168,52.59)
\lineto(572.6747168,52.59)
\lineto(572.9747168,52.59)
\curveto(573.08470955,52.5999939)(573.19470944,52.5999939)(573.3047168,52.59)
\curveto(573.42470921,52.58999391)(573.5297091,52.57999392)(573.6197168,52.56)
\curveto(573.71970891,52.54999395)(573.79470884,52.52999397)(573.8447168,52.5)
\curveto(573.87470876,52.47999402)(573.89970873,52.43499407)(573.9197168,52.365)
\curveto(573.93970869,52.29499421)(573.95470868,52.21999428)(573.9647168,52.14)
\curveto(573.97470866,52.05999444)(573.97470866,51.97499452)(573.9647168,51.885)
\curveto(573.96470867,51.8049947)(573.95470868,51.73499477)(573.9347168,51.675)
\curveto(573.91470872,51.58499491)(573.86970876,51.51999498)(573.7997168,51.48)
\curveto(573.77970885,51.45999504)(573.74970888,51.44499505)(573.7097168,51.435)
\curveto(573.67970895,51.43499507)(573.64970898,51.42999507)(573.6197168,51.42)
\lineto(573.5297168,51.42)
\curveto(573.47970915,51.40999509)(573.4297092,51.4049951)(573.3797168,51.405)
\curveto(573.3297093,51.41499508)(573.27970935,51.41999508)(573.2297168,51.42)
\lineto(572.6747168,51.42)
\lineto(569.5097168,51.42)
\lineto(569.1497168,51.42)
\curveto(569.03971359,51.42999507)(568.9347137,51.42499508)(568.8347168,51.405)
\curveto(568.7347139,51.39499511)(568.64471399,51.36999513)(568.5647168,51.33)
\curveto(568.49471414,51.28999521)(568.44471419,51.21999528)(568.4147168,51.12)
\curveto(568.39471424,51.05999544)(568.38471425,50.98999551)(568.3847168,50.91)
\curveto(568.39471424,50.82999567)(568.39971423,50.74999575)(568.3997168,50.67)
\lineto(568.3997168,49.83)
\lineto(568.3997168,48.405)
\curveto(568.39971423,48.26499824)(568.40471423,48.13499837)(568.4147168,48.015)
\curveto(568.42471421,47.90499859)(568.46471417,47.82499868)(568.5347168,47.775)
\curveto(568.60471403,47.72499878)(568.68471395,47.69499881)(568.7747168,47.685)
\lineto(569.0747168,47.685)
\lineto(570.0347168,47.685)
\lineto(572.8097168,47.685)
\lineto(573.6647168,47.685)
\lineto(573.9047168,47.685)
\curveto(573.98470865,47.69499881)(574.05470858,47.68999881)(574.1147168,47.67)
\curveto(574.2347084,47.62999887)(574.31470832,47.57499892)(574.3547168,47.505)
\curveto(574.37470826,47.47499902)(574.38970824,47.42499908)(574.3997168,47.355)
\curveto(574.40970822,47.28499922)(574.41470822,47.20999929)(574.4147168,47.13)
\curveto(574.42470821,47.05999944)(574.42470821,46.98499952)(574.4147168,46.905)
\curveto(574.40470823,46.83499966)(574.39470824,46.77999972)(574.3847168,46.74)
\curveto(574.34470829,46.65999984)(574.29970833,46.60499989)(574.2497168,46.575)
\curveto(574.18970844,46.53499996)(574.10970852,46.51499998)(574.0097168,46.515)
\lineto(573.7397168,46.515)
\lineto(572.6897168,46.515)
\lineto(568.6997168,46.515)
\lineto(567.6497168,46.515)
\curveto(567.50971512,46.51499998)(567.38971524,46.51999998)(567.2897168,46.53)
\curveto(567.18971544,46.54999995)(567.11471552,46.5999999)(567.0647168,46.68)
\curveto(567.02471561,46.73999976)(567.00471563,46.81499968)(567.0047168,46.905)
\lineto(567.0047168,47.19)
\lineto(567.0047168,48.24)
\lineto(567.0047168,52.26)
\lineto(567.0047168,55.62)
\lineto(567.0047168,56.55)
\lineto(567.0047168,56.82)
\curveto(567.00471563,56.90998959)(567.02471561,56.97998952)(567.0647168,57.03)
\curveto(567.10471553,57.0999894)(567.17971545,57.14998935)(567.2897168,57.18)
\curveto(567.30971532,57.18998931)(567.3297153,57.18998931)(567.3497168,57.18)
\curveto(567.36971526,57.17998932)(567.38971524,57.18498931)(567.4097168,57.195)
}
}
{
\newrgbcolor{curcolor}{0 0 0}
\pscustom[linestyle=none,fillstyle=solid,fillcolor=curcolor]
{
\newpath
\moveto(582.87963867,44.31)
\lineto(582.87963867,43.98)
\curveto(582.88963079,43.87000263)(582.86963081,43.78500272)(582.81963867,43.725)
\curveto(582.79963088,43.69500281)(582.7746309,43.67000283)(582.74463867,43.65)
\lineto(582.65463867,43.59)
\curveto(582.62463105,43.58000292)(582.5746311,43.57500292)(582.50463867,43.575)
\curveto(582.43463124,43.56500293)(582.35963132,43.56000294)(582.27963867,43.56)
\curveto(582.20963147,43.56000294)(582.13963154,43.56500293)(582.06963867,43.575)
\curveto(581.99963168,43.57500292)(581.94963173,43.58000292)(581.91963867,43.59)
\curveto(581.81963186,43.61000289)(581.74963193,43.65500285)(581.70963867,43.725)
\curveto(581.65963202,43.80500269)(581.63463204,43.93000257)(581.63463867,44.1)
\lineto(581.63463867,44.52)
\lineto(581.63463867,46.365)
\lineto(581.63463867,46.725)
\curveto(581.64463203,46.86499964)(581.62963205,46.97999952)(581.58963867,47.07)
\curveto(581.56963211,47.08999941)(581.54963213,47.10499939)(581.52963867,47.115)
\curveto(581.51963216,47.13499936)(581.50463217,47.15499935)(581.48463867,47.175)
\curveto(581.38463229,47.17499932)(581.30463237,47.14999935)(581.24463867,47.1)
\lineto(581.09463867,46.95)
\curveto(581.01463266,46.88999961)(580.92963275,46.82999967)(580.83963867,46.77)
\curveto(580.74963293,46.71999978)(580.64963303,46.66999983)(580.53963867,46.62)
\curveto(580.38963329,46.55999994)(580.21463346,46.50999999)(580.01463867,46.47)
\curveto(579.82463385,46.42000008)(579.61963406,46.39000011)(579.39963867,46.38)
\curveto(579.18963449,46.36000014)(578.9746347,46.36000014)(578.75463867,46.38)
\curveto(578.54463513,46.39000011)(578.34963533,46.42000008)(578.16963867,46.47)
\curveto(578.11963556,46.49000001)(578.06963561,46.50499999)(578.01963867,46.515)
\curveto(577.96963571,46.52499998)(577.91963576,46.53999996)(577.86963867,46.56)
\curveto(577.7796359,46.5999999)(577.68963599,46.63499986)(577.59963867,46.665)
\curveto(577.50963617,46.70499979)(577.42463625,46.74999975)(577.34463867,46.8)
\curveto(576.99463668,47.01999948)(576.69963698,47.26999923)(576.45963867,47.55)
\curveto(576.22963745,47.83999866)(576.02963765,48.19499831)(575.85963867,48.615)
\curveto(575.81963786,48.71499778)(575.78463789,48.81999768)(575.75463867,48.93)
\curveto(575.73463794,49.03999746)(575.70963797,49.14999735)(575.67963867,49.26)
\curveto(575.66963801,49.27999722)(575.66463801,49.2999972)(575.66463867,49.32)
\curveto(575.66463801,49.34999715)(575.65963802,49.37999712)(575.64963867,49.41)
\curveto(575.62963805,49.48999701)(575.61463806,49.57999692)(575.60463867,49.68)
\curveto(575.60463807,49.77999672)(575.59463808,49.87499662)(575.57463867,49.965)
\lineto(575.57463867,50.22)
\curveto(575.55463812,50.26999623)(575.54463813,50.33499617)(575.54463867,50.415)
\curveto(575.54463813,50.49499601)(575.55463812,50.55999594)(575.57463867,50.61)
\lineto(575.57463867,50.775)
\curveto(575.5746381,50.83499567)(575.5796381,50.89499561)(575.58963867,50.955)
\curveto(575.59963808,50.99499551)(575.59963808,51.03499547)(575.58963867,51.075)
\curveto(575.58963809,51.11499538)(575.59463808,51.15999534)(575.60463867,51.21)
\curveto(575.63463804,51.31999518)(575.65463802,51.42499508)(575.66463867,51.525)
\curveto(575.68463799,51.63499487)(575.70963797,51.73999476)(575.73963867,51.84)
\curveto(575.7796379,51.96999453)(575.81963786,52.08999441)(575.85963867,52.2)
\curveto(575.89963778,52.31999418)(575.94463773,52.43499407)(575.99463867,52.545)
\curveto(576.06463761,52.68499381)(576.13963754,52.81499368)(576.21963867,52.935)
\curveto(576.30963737,53.06499344)(576.39963728,53.18999331)(576.48963867,53.31)
\curveto(576.49963718,53.30999319)(576.51463716,53.31999318)(576.53463867,53.34)
\curveto(576.58463709,53.41999308)(576.65963702,53.499993)(576.75963867,53.58)
\curveto(576.76963691,53.58999291)(576.7746369,53.5999929)(576.77463867,53.61)
\curveto(576.78463689,53.61999288)(576.79963688,53.62999287)(576.81963867,53.64)
\curveto(576.85963682,53.66999283)(576.89463678,53.6999928)(576.92463867,53.73)
\curveto(576.96463671,53.76999273)(577.00963667,53.8049927)(577.05963867,53.835)
\curveto(577.19963648,53.94499255)(577.35463632,54.03499247)(577.52463867,54.105)
\curveto(577.69463598,54.17499233)(577.8746358,54.23999226)(578.06463867,54.3)
\curveto(578.16463551,54.33999216)(578.26963541,54.36499214)(578.37963867,54.375)
\curveto(578.48963519,54.38499211)(578.59963508,54.3999921)(578.70963867,54.42)
\curveto(578.74963493,54.42999207)(578.80463487,54.42999207)(578.87463867,54.42)
\curveto(578.95463472,54.40999209)(579.00463467,54.41499208)(579.02463867,54.435)
\curveto(579.35463432,54.43499207)(579.674634,54.39499211)(579.98463867,54.315)
\curveto(580.29463338,54.23499227)(580.54963313,54.13499237)(580.74963867,54.015)
\lineto(580.92963867,53.895)
\curveto(580.98963269,53.85499264)(581.04963263,53.80999269)(581.10963867,53.76)
\lineto(581.25963867,53.64)
\curveto(581.30963237,53.5999929)(581.38463229,53.57999292)(581.48463867,53.58)
\curveto(581.50463217,53.5999929)(581.52463215,53.61499288)(581.54463867,53.625)
\curveto(581.56463211,53.64499285)(581.5796321,53.66999283)(581.58963867,53.7)
\curveto(581.61963206,53.76999273)(581.63463204,53.84499265)(581.63463867,53.925)
\curveto(581.64463203,54.0049925)(581.674632,54.06999243)(581.72463867,54.12)
\curveto(581.75463192,54.15999234)(581.81463186,54.18999231)(581.90463867,54.21)
\curveto(582.00463167,54.23999226)(582.10963157,54.25499224)(582.21963867,54.255)
\curveto(582.32963135,54.26499224)(582.43463124,54.25999224)(582.53463867,54.24)
\curveto(582.63463104,54.21999228)(582.70963097,54.19499231)(582.75963867,54.165)
\curveto(582.82963085,54.11499238)(582.86463081,54.02999247)(582.86463867,53.91)
\curveto(582.8746308,53.78999271)(582.8796308,53.66999283)(582.87963867,53.55)
\lineto(582.87963867,44.31)
\moveto(581.66463867,50.175)
\curveto(581.674632,50.22499628)(581.679632,50.29499621)(581.67963867,50.385)
\curveto(581.68963199,50.47499602)(581.68463199,50.54499595)(581.66463867,50.595)
\lineto(581.66463867,50.805)
\lineto(581.60463867,51.105)
\curveto(581.59463208,51.2049953)(581.5796321,51.29499521)(581.55963867,51.375)
\curveto(581.53963214,51.45499504)(581.51963216,51.52499498)(581.49963867,51.585)
\curveto(581.48963219,51.65499484)(581.46963221,51.72499478)(581.43963867,51.795)
\curveto(581.32963235,52.06499444)(581.15463252,52.32999417)(580.91463867,52.59)
\curveto(580.674633,52.84999365)(580.44463323,53.02999347)(580.22463867,53.13)
\curveto(580.14463353,53.16999333)(580.05963362,53.1999933)(579.96963867,53.22)
\curveto(579.88963379,53.23999326)(579.80463387,53.26499324)(579.71463867,53.295)
\curveto(579.61463406,53.31499318)(579.50463417,53.32499317)(579.38463867,53.325)
\lineto(579.03963867,53.325)
\lineto(578.88963867,53.295)
\lineto(578.75463867,53.295)
\lineto(578.51463867,53.235)
\curveto(578.43463524,53.21499328)(578.35963532,53.18499331)(578.28963867,53.145)
\curveto(577.96963571,53.0049935)(577.70963597,52.8049937)(577.50963867,52.545)
\curveto(577.31963636,52.28499421)(577.16963651,51.97999452)(577.05963867,51.63)
\curveto(577.01963666,51.51999498)(576.98963669,51.3999951)(576.96963867,51.27)
\curveto(576.95963672,51.14999535)(576.93963674,51.02499548)(576.90963867,50.895)
\lineto(576.90963867,50.76)
\curveto(576.90963677,50.71999578)(576.90463677,50.67499582)(576.89463867,50.625)
\curveto(576.88463679,50.58499591)(576.8796368,50.53999596)(576.87963867,50.49)
\curveto(576.88963679,50.43999606)(576.89463678,50.38999611)(576.89463867,50.34)
\lineto(576.89463867,50.04)
\curveto(576.89463678,49.94999655)(576.90463677,49.86499664)(576.92463867,49.785)
\curveto(576.93463674,49.75499675)(576.93963674,49.70999679)(576.93963867,49.65)
\curveto(576.95963672,49.57999692)(576.9746367,49.50999699)(576.98463867,49.44)
\lineto(577.04463867,49.23)
\curveto(577.13463654,48.93999756)(577.25463642,48.67499782)(577.40463867,48.435)
\curveto(577.55463612,48.20499829)(577.74463593,48.00999849)(577.97463867,47.85)
\lineto(578.06463867,47.79)
\curveto(578.10463557,47.76999873)(578.13963554,47.74999875)(578.16963867,47.73)
\curveto(578.26963541,47.66999883)(578.3746353,47.61999888)(578.48463867,47.58)
\lineto(578.84463867,47.49)
\curveto(578.89463478,47.46999903)(578.93463474,47.45999904)(578.96463867,47.46)
\curveto(578.99463468,47.46999903)(579.03463464,47.46999903)(579.08463867,47.46)
\curveto(579.12463455,47.44999905)(579.1746345,47.43999906)(579.23463867,47.43)
\curveto(579.29463438,47.42999907)(579.34963433,47.43999906)(579.39963867,47.46)
\lineto(579.51963867,47.46)
\curveto(579.54963413,47.46999903)(579.5796341,47.46999903)(579.60963867,47.46)
\curveto(579.63963404,47.45999904)(579.66963401,47.46499904)(579.69963867,47.475)
\curveto(579.7796339,47.49499901)(579.85963382,47.50999899)(579.93963867,47.52)
\curveto(580.01963366,47.53999896)(580.09463358,47.56499894)(580.16463867,47.595)
\curveto(580.4746332,47.72499878)(580.72963295,47.8999986)(580.92963867,48.12)
\curveto(581.12963255,48.34999815)(581.29463238,48.61499788)(581.42463867,48.915)
\curveto(581.4746322,49.02499748)(581.50963217,49.13499737)(581.52963867,49.245)
\curveto(581.54963213,49.35499715)(581.5746321,49.46999703)(581.60463867,49.59)
\curveto(581.62463205,49.62999687)(581.63463204,49.66999683)(581.63463867,49.71)
\curveto(581.63463204,49.74999675)(581.63963204,49.78999671)(581.64963867,49.83)
\curveto(581.65963202,49.87999662)(581.65963202,49.93499657)(581.64963867,49.995)
\curveto(581.64963203,50.05499645)(581.65463202,50.11499638)(581.66463867,50.175)
}
}
{
\newrgbcolor{curcolor}{0 0 0}
\pscustom[linestyle=none,fillstyle=solid,fillcolor=curcolor]
{
\newpath
\moveto(585.29088867,54.24)
\lineto(585.72588867,54.24)
\curveto(585.87588671,54.23999226)(585.9808866,54.1999923)(586.04088867,54.12)
\curveto(586.09088649,54.03999246)(586.11588647,53.93999256)(586.11588867,53.82)
\curveto(586.12588646,53.6999928)(586.13088645,53.57999292)(586.13088867,53.46)
\lineto(586.13088867,52.035)
\lineto(586.13088867,49.77)
\lineto(586.13088867,49.08)
\curveto(586.13088645,48.84999765)(586.15588643,48.64999785)(586.20588867,48.48)
\curveto(586.36588622,48.02999847)(586.66588592,47.71499878)(587.10588867,47.535)
\curveto(587.32588526,47.44499905)(587.59088499,47.40999909)(587.90088867,47.43)
\curveto(588.21088437,47.45999904)(588.46088412,47.51499898)(588.65088867,47.595)
\curveto(588.9808836,47.73499877)(589.24088334,47.90999859)(589.43088867,48.12)
\curveto(589.63088295,48.33999816)(589.7858828,48.62499788)(589.89588867,48.975)
\curveto(589.92588266,49.05499745)(589.94588264,49.13499737)(589.95588867,49.215)
\curveto(589.96588262,49.29499721)(589.9808826,49.37999712)(590.00088867,49.47)
\curveto(590.01088257,49.51999698)(590.01088257,49.56499694)(590.00088867,49.605)
\curveto(590.00088258,49.64499685)(590.01088257,49.68999681)(590.03088867,49.74)
\lineto(590.03088867,50.055)
\curveto(590.05088253,50.13499637)(590.05588253,50.22499628)(590.04588867,50.325)
\curveto(590.03588255,50.43499607)(590.03088255,50.53499597)(590.03088867,50.625)
\lineto(590.03088867,51.795)
\lineto(590.03088867,53.385)
\curveto(590.03088255,53.504993)(590.02588256,53.62999287)(590.01588867,53.76)
\curveto(590.01588257,53.8999926)(590.04088254,54.00999249)(590.09088867,54.09)
\curveto(590.13088245,54.13999236)(590.17588241,54.16999233)(590.22588867,54.18)
\curveto(590.2858823,54.1999923)(590.35588223,54.21999228)(590.43588867,54.24)
\lineto(590.66088867,54.24)
\curveto(590.7808818,54.23999226)(590.8858817,54.23499227)(590.97588867,54.225)
\curveto(591.07588151,54.21499228)(591.15088143,54.16999233)(591.20088867,54.09)
\curveto(591.25088133,54.03999246)(591.27588131,53.96499254)(591.27588867,53.865)
\lineto(591.27588867,53.58)
\lineto(591.27588867,52.56)
\lineto(591.27588867,48.525)
\lineto(591.27588867,47.175)
\curveto(591.27588131,47.05499945)(591.27088131,46.93999956)(591.26088867,46.83)
\curveto(591.26088132,46.72999977)(591.22588136,46.65499985)(591.15588867,46.605)
\curveto(591.11588147,46.57499992)(591.05588153,46.54999995)(590.97588867,46.53)
\curveto(590.89588169,46.51999998)(590.80588178,46.50999999)(590.70588867,46.5)
\curveto(590.61588197,46.5)(590.52588206,46.50499999)(590.43588867,46.515)
\curveto(590.35588223,46.52499998)(590.29588229,46.54499995)(590.25588867,46.575)
\curveto(590.20588238,46.61499988)(590.16088242,46.67999982)(590.12088867,46.77)
\curveto(590.11088247,46.80999969)(590.10088248,46.86499964)(590.09088867,46.935)
\curveto(590.09088249,47.00499949)(590.0858825,47.06999943)(590.07588867,47.13)
\curveto(590.06588252,47.1999993)(590.04588254,47.25499925)(590.01588867,47.295)
\curveto(589.9858826,47.33499916)(589.94088264,47.34999915)(589.88088867,47.34)
\curveto(589.80088278,47.31999918)(589.72088286,47.25999924)(589.64088867,47.16)
\curveto(589.56088302,47.06999943)(589.4858831,46.9999995)(589.41588867,46.95)
\curveto(589.19588339,46.78999971)(588.94588364,46.64999985)(588.66588867,46.53)
\curveto(588.55588403,46.48000002)(588.44088414,46.45000005)(588.32088867,46.44)
\curveto(588.21088437,46.42000008)(588.09588449,46.39500011)(587.97588867,46.365)
\curveto(587.92588466,46.35500015)(587.87088471,46.35500015)(587.81088867,46.365)
\curveto(587.76088482,46.37500012)(587.71088487,46.37000013)(587.66088867,46.35)
\curveto(587.56088502,46.33000017)(587.47088511,46.33000017)(587.39088867,46.35)
\lineto(587.24088867,46.35)
\curveto(587.19088539,46.37000013)(587.13088545,46.38000012)(587.06088867,46.38)
\curveto(587.00088558,46.38000012)(586.94588564,46.38500012)(586.89588867,46.395)
\curveto(586.85588573,46.41500008)(586.81588577,46.42500008)(586.77588867,46.425)
\curveto(586.74588584,46.41500008)(586.70588588,46.42000008)(586.65588867,46.44)
\lineto(586.41588867,46.5)
\curveto(586.34588624,46.51999998)(586.27088631,46.54999995)(586.19088867,46.59)
\curveto(585.93088665,46.6999998)(585.71088687,46.84499965)(585.53088867,47.025)
\curveto(585.36088722,47.21499928)(585.22088736,47.43999906)(585.11088867,47.7)
\curveto(585.07088751,47.78999871)(585.04088754,47.87999862)(585.02088867,47.97)
\lineto(584.96088867,48.27)
\curveto(584.94088764,48.32999817)(584.93088765,48.38499811)(584.93088867,48.435)
\curveto(584.94088764,48.49499801)(584.93588765,48.55999794)(584.91588867,48.63)
\curveto(584.90588768,48.64999785)(584.90088768,48.67499782)(584.90088867,48.705)
\curveto(584.90088768,48.74499775)(584.89588769,48.77999772)(584.88588867,48.81)
\lineto(584.88588867,48.96)
\curveto(584.87588771,48.9999975)(584.87088771,49.04499745)(584.87088867,49.095)
\curveto(584.8808877,49.15499735)(584.8858877,49.20999729)(584.88588867,49.26)
\lineto(584.88588867,49.86)
\lineto(584.88588867,52.62)
\lineto(584.88588867,53.58)
\lineto(584.88588867,53.85)
\curveto(584.8858877,53.93999256)(584.90588768,54.01499248)(584.94588867,54.075)
\curveto(584.9858876,54.14499235)(585.06088752,54.19499231)(585.17088867,54.225)
\curveto(585.19088739,54.23499227)(585.21088737,54.23499227)(585.23088867,54.225)
\curveto(585.25088733,54.22499227)(585.27088731,54.22999227)(585.29088867,54.24)
}
}
{
\newrgbcolor{curcolor}{0 0 0}
\pscustom[linestyle=none,fillstyle=solid,fillcolor=curcolor]
{
\newpath
\moveto(593.46049805,55.74)
\curveto(593.38049693,55.7999907)(593.33549697,55.9049906)(593.32549805,56.055)
\lineto(593.32549805,56.52)
\lineto(593.32549805,56.775)
\curveto(593.32549698,56.86498964)(593.34049697,56.93998956)(593.37049805,57)
\curveto(593.4104969,57.07998942)(593.49049682,57.13998936)(593.61049805,57.18)
\curveto(593.63049668,57.18998931)(593.65049666,57.18998931)(593.67049805,57.18)
\curveto(593.70049661,57.17998932)(593.72549658,57.18498931)(593.74549805,57.195)
\curveto(593.91549639,57.1949893)(594.07549623,57.18998931)(594.22549805,57.18)
\curveto(594.37549593,57.16998933)(594.47549583,57.10998939)(594.52549805,57)
\curveto(594.55549575,56.93998956)(594.57049574,56.86498964)(594.57049805,56.775)
\lineto(594.57049805,56.52)
\curveto(594.57049574,56.33999016)(594.56549574,56.16999033)(594.55549805,56.01)
\curveto(594.55549575,55.84999065)(594.49049582,55.74499076)(594.36049805,55.695)
\curveto(594.310496,55.67499083)(594.25549605,55.66499084)(594.19549805,55.665)
\lineto(594.03049805,55.665)
\lineto(593.71549805,55.665)
\curveto(593.61549669,55.66499084)(593.53049678,55.68999081)(593.46049805,55.74)
\moveto(594.57049805,47.235)
\lineto(594.57049805,46.92)
\curveto(594.58049573,46.81999968)(594.56049575,46.73999976)(594.51049805,46.68)
\curveto(594.48049583,46.61999988)(594.43549587,46.57999992)(594.37549805,46.56)
\curveto(594.31549599,46.54999995)(594.24549606,46.53499996)(594.16549805,46.515)
\lineto(593.94049805,46.515)
\curveto(593.8104965,46.51499998)(593.69549661,46.51999998)(593.59549805,46.53)
\curveto(593.5054968,46.54999995)(593.43549687,46.5999999)(593.38549805,46.68)
\curveto(593.34549696,46.73999976)(593.32549698,46.81499968)(593.32549805,46.905)
\lineto(593.32549805,47.19)
\lineto(593.32549805,53.535)
\lineto(593.32549805,53.85)
\curveto(593.32549698,53.95999254)(593.35049696,54.04499245)(593.40049805,54.105)
\curveto(593.43049688,54.15499234)(593.47049684,54.18499231)(593.52049805,54.195)
\curveto(593.57049674,54.2049923)(593.62549668,54.21999228)(593.68549805,54.24)
\curveto(593.7054966,54.23999226)(593.72549658,54.23499227)(593.74549805,54.225)
\curveto(593.77549653,54.22499227)(593.80049651,54.22999227)(593.82049805,54.24)
\curveto(593.95049636,54.23999226)(594.08049623,54.23499227)(594.21049805,54.225)
\curveto(594.35049596,54.22499227)(594.44549586,54.18499231)(594.49549805,54.105)
\curveto(594.54549576,54.04499245)(594.57049574,53.96499254)(594.57049805,53.865)
\lineto(594.57049805,53.58)
\lineto(594.57049805,47.235)
}
}
{
\newrgbcolor{curcolor}{0 0 0}
\pscustom[linestyle=none,fillstyle=solid,fillcolor=curcolor]
{
\newpath
\moveto(603.9403418,50.58)
\curveto(603.95033345,50.52999597)(603.95533344,50.46499604)(603.9553418,50.385)
\curveto(603.95533344,50.3049962)(603.95033345,50.23999626)(603.9403418,50.19)
\curveto(603.92033348,50.13999636)(603.91533348,50.08999641)(603.9253418,50.04)
\curveto(603.93533346,49.9999965)(603.93533346,49.95999654)(603.9253418,49.92)
\curveto(603.92533347,49.84999665)(603.92033348,49.79499671)(603.9103418,49.755)
\curveto(603.89033351,49.66499684)(603.87533352,49.57499692)(603.8653418,49.485)
\curveto(603.86533353,49.39499711)(603.85533354,49.30499719)(603.8353418,49.215)
\lineto(603.7753418,48.975)
\curveto(603.75533364,48.90499759)(603.73033367,48.82999767)(603.7003418,48.75)
\curveto(603.58033382,48.37999812)(603.41533398,48.04499845)(603.2053418,47.745)
\curveto(603.14533425,47.65499885)(603.08033432,47.56499894)(603.0103418,47.475)
\curveto(602.94033446,47.39499911)(602.86533453,47.31999918)(602.7853418,47.25)
\lineto(602.7103418,47.175)
\curveto(602.64033476,47.12499938)(602.57533482,47.07499942)(602.5153418,47.025)
\curveto(602.45533494,46.97499952)(602.38533501,46.92499958)(602.3053418,46.875)
\curveto(602.1953352,46.79499971)(602.07033533,46.72499978)(601.9303418,46.665)
\curveto(601.8003356,46.61499988)(601.66533573,46.56499994)(601.5253418,46.515)
\curveto(601.44533595,46.49500001)(601.36533603,46.48000002)(601.2853418,46.47)
\curveto(601.21533618,46.46000004)(601.14033626,46.44500005)(601.0603418,46.425)
\lineto(601.0003418,46.425)
\curveto(600.99033641,46.41500008)(600.97533642,46.41000009)(600.9553418,46.41)
\curveto(600.86533653,46.39000011)(600.73033667,46.38000012)(600.5503418,46.38)
\curveto(600.38033702,46.37000013)(600.24533715,46.37500012)(600.1453418,46.395)
\lineto(600.0703418,46.395)
\curveto(600.0003374,46.40500009)(599.93533746,46.41500008)(599.8753418,46.425)
\curveto(599.81533758,46.42500008)(599.75533764,46.43500006)(599.6953418,46.455)
\curveto(599.52533787,46.50499999)(599.36533803,46.54999995)(599.2153418,46.59)
\curveto(599.06533833,46.62999987)(598.92533847,46.68999981)(598.7953418,46.77)
\curveto(598.63533876,46.85999964)(598.4953389,46.95499955)(598.3753418,47.055)
\curveto(598.33533906,47.08499942)(598.27533912,47.12499938)(598.1953418,47.175)
\curveto(598.11533928,47.23499926)(598.04033936,47.23999926)(597.9703418,47.19)
\curveto(597.93033947,47.15999934)(597.91033949,47.11999938)(597.9103418,47.07)
\curveto(597.91033949,47.01999948)(597.9003395,46.96499954)(597.8803418,46.905)
\curveto(597.87033953,46.87499962)(597.87033953,46.83999966)(597.8803418,46.8)
\curveto(597.89033951,46.76999973)(597.89033951,46.73499976)(597.8803418,46.695)
\curveto(597.86033954,46.63499986)(597.85033955,46.56999993)(597.8503418,46.5)
\curveto(597.86033954,46.42000008)(597.86533953,46.35000015)(597.8653418,46.29)
\lineto(597.8653418,44.49)
\lineto(597.8653418,44.055)
\curveto(597.86533953,43.90500259)(597.83533956,43.79000271)(597.7753418,43.71)
\curveto(597.72533967,43.64000286)(597.64533975,43.60500289)(597.5353418,43.605)
\curveto(597.42533997,43.59500291)(597.31534008,43.59000291)(597.2053418,43.59)
\lineto(596.9653418,43.59)
\curveto(596.8953405,43.61000289)(596.83534056,43.63000287)(596.7853418,43.65)
\curveto(596.74534065,43.67000283)(596.71034069,43.70500279)(596.6803418,43.755)
\curveto(596.63034077,43.82500268)(596.60534079,43.93500256)(596.6053418,44.085)
\curveto(596.61534078,44.23500226)(596.62034078,44.36500213)(596.6203418,44.475)
\lineto(596.6203418,53.475)
\lineto(596.6203418,53.835)
\curveto(596.63034077,53.96499254)(596.66034074,54.06999243)(596.7103418,54.15)
\curveto(596.74034066,54.18999231)(596.80534059,54.21999228)(596.9053418,54.24)
\curveto(597.01534038,54.26999223)(597.13034027,54.27999222)(597.2503418,54.27)
\curveto(597.37034003,54.26999223)(597.48033992,54.25499224)(597.5803418,54.225)
\curveto(597.69033971,54.2049923)(597.76033964,54.17499233)(597.7903418,54.135)
\curveto(597.83033957,54.08499241)(597.85033955,54.02499247)(597.8503418,53.955)
\curveto(597.86033954,53.88499261)(597.88033952,53.81499268)(597.9103418,53.745)
\curveto(597.93033947,53.71499278)(597.94533945,53.68999281)(597.9553418,53.67)
\curveto(597.97533942,53.65999284)(597.9953394,53.64499285)(598.0153418,53.625)
\curveto(598.12533927,53.61499288)(598.21533918,53.64999285)(598.2853418,53.73)
\curveto(598.36533903,53.80999269)(598.44033896,53.87499263)(598.5103418,53.925)
\curveto(598.77033863,54.1049924)(599.08033832,54.24499225)(599.4403418,54.345)
\curveto(599.53033787,54.36499214)(599.62033778,54.37999212)(599.7103418,54.39)
\curveto(599.81033759,54.3999921)(599.91033749,54.41499208)(600.0103418,54.435)
\curveto(600.05033735,54.44499206)(600.1003373,54.44499206)(600.1603418,54.435)
\curveto(600.22033718,54.42499207)(600.26033714,54.42999207)(600.2803418,54.45)
\curveto(600.71033669,54.45999204)(601.09033631,54.41499208)(601.4203418,54.315)
\curveto(601.75033565,54.22499227)(602.04533535,54.09499241)(602.3053418,53.925)
\lineto(602.4553418,53.805)
\curveto(602.50533489,53.77499273)(602.55533484,53.73999276)(602.6053418,53.7)
\curveto(602.62533477,53.67999282)(602.64033476,53.65999284)(602.6503418,53.64)
\curveto(602.67033473,53.62999287)(602.69033471,53.61499288)(602.7103418,53.595)
\curveto(602.76033464,53.54499295)(602.81533458,53.48999301)(602.8753418,53.43)
\curveto(602.93533446,53.36999313)(602.99033441,53.30999319)(603.0403418,53.25)
\curveto(603.16033424,53.07999342)(603.28533411,52.89499361)(603.4153418,52.695)
\curveto(603.4953339,52.56499394)(603.56033384,52.41999408)(603.6103418,52.26)
\curveto(603.67033373,52.0999944)(603.72533367,51.93999456)(603.7753418,51.78)
\curveto(603.7953336,51.6999948)(603.81033359,51.61499488)(603.8203418,51.525)
\curveto(603.84033356,51.43499507)(603.86033354,51.34999515)(603.8803418,51.27)
\lineto(603.8803418,51.15)
\curveto(603.89033351,51.11999538)(603.8953335,51.08999541)(603.8953418,51.06)
\curveto(603.91533348,51.00999549)(603.92033348,50.95499554)(603.9103418,50.895)
\curveto(603.91033349,50.83499567)(603.92033348,50.77999572)(603.9403418,50.73)
\lineto(603.9403418,50.58)
\moveto(602.6053418,50.175)
\curveto(602.62533477,50.22499628)(602.63033477,50.28499621)(602.6203418,50.355)
\curveto(602.61033479,50.43499607)(602.60533479,50.504996)(602.6053418,50.565)
\curveto(602.60533479,50.73499577)(602.5953348,50.89499561)(602.5753418,51.045)
\curveto(602.56533483,51.19499531)(602.53533486,51.33999516)(602.4853418,51.48)
\lineto(602.4253418,51.66)
\curveto(602.41533498,51.72999477)(602.395335,51.79499471)(602.3653418,51.855)
\curveto(602.25533514,52.12499438)(602.08033532,52.38499411)(601.8403418,52.635)
\curveto(601.61033579,52.88499361)(601.39033601,53.05499344)(601.1803418,53.145)
\curveto(601.1003363,53.18499331)(601.01533638,53.21499328)(600.9253418,53.235)
\curveto(600.84533655,53.25499324)(600.76033664,53.27999322)(600.6703418,53.31)
\curveto(600.58033682,53.32999317)(600.47533692,53.33999316)(600.3553418,53.34)
\lineto(600.0253418,53.34)
\curveto(600.00533739,53.31999318)(599.96533743,53.30999319)(599.9053418,53.31)
\curveto(599.85533754,53.31999318)(599.81033759,53.31999318)(599.7703418,53.31)
\lineto(599.5003418,53.25)
\curveto(599.42033798,53.22999327)(599.34033806,53.1999933)(599.2603418,53.16)
\curveto(598.94033846,53.01999348)(598.67533872,52.81499368)(598.4653418,52.545)
\curveto(598.26533913,52.28499421)(598.11033929,51.97999452)(598.0003418,51.63)
\curveto(597.96033944,51.51999498)(597.93033947,51.40999509)(597.9103418,51.3)
\curveto(597.9003395,51.18999531)(597.88533951,51.07999542)(597.8653418,50.97)
\curveto(597.85533954,50.92999557)(597.85033955,50.88999561)(597.8503418,50.85)
\curveto(597.85033955,50.81999568)(597.84533955,50.78499571)(597.8353418,50.745)
\lineto(597.8353418,50.625)
\curveto(597.82533957,50.57499592)(597.82033958,50.499996)(597.8203418,50.4)
\curveto(597.82033958,50.30999619)(597.82533957,50.23999626)(597.8353418,50.19)
\lineto(597.8353418,50.07)
\curveto(597.84533955,50.02999647)(597.85033955,49.98999651)(597.8503418,49.95)
\curveto(597.85033955,49.90999659)(597.85533954,49.87499662)(597.8653418,49.845)
\curveto(597.87533952,49.81499668)(597.88033952,49.78499671)(597.8803418,49.755)
\curveto(597.88033952,49.72499678)(597.88533951,49.68999681)(597.8953418,49.65)
\curveto(597.91533948,49.56999693)(597.93033947,49.48999701)(597.9403418,49.41)
\lineto(598.0003418,49.17)
\curveto(598.11033929,48.82999767)(598.26033914,48.52999797)(598.4503418,48.27)
\curveto(598.65033875,48.01999848)(598.91033849,47.82499868)(599.2303418,47.685)
\curveto(599.42033798,47.60499889)(599.61533778,47.54499895)(599.8153418,47.505)
\curveto(599.85533754,47.48499902)(599.8953375,47.47499902)(599.9353418,47.475)
\curveto(599.97533742,47.48499902)(600.01533738,47.48499902)(600.0553418,47.475)
\lineto(600.1753418,47.475)
\curveto(600.24533715,47.45499905)(600.31533708,47.45499905)(600.3853418,47.475)
\lineto(600.5053418,47.475)
\curveto(600.61533678,47.49499901)(600.72033668,47.50999899)(600.8203418,47.52)
\curveto(600.92033648,47.52999897)(601.02033638,47.55499895)(601.1203418,47.595)
\curveto(601.43033597,47.72499878)(601.68033572,47.89499861)(601.8703418,48.105)
\curveto(602.07033533,48.32499818)(602.23533516,48.58999791)(602.3653418,48.9)
\curveto(602.41533498,49.03999746)(602.45033495,49.17999732)(602.4703418,49.32)
\curveto(602.5003349,49.46999703)(602.53533486,49.62499688)(602.5753418,49.785)
\curveto(602.58533481,49.83499667)(602.59033481,49.87999662)(602.5903418,49.92)
\curveto(602.59033481,49.95999654)(602.5953348,50.0049965)(602.6053418,50.055)
\lineto(602.6053418,50.175)
}
}
{
\newrgbcolor{curcolor}{0 0 0}
\pscustom[linestyle=none,fillstyle=solid,fillcolor=curcolor]
{
\newpath
\moveto(612.5465918,50.715)
\curveto(612.56658374,50.65499584)(612.57658373,50.55999594)(612.5765918,50.43)
\curveto(612.57658373,50.30999619)(612.57158373,50.22499628)(612.5615918,50.175)
\lineto(612.5615918,50.025)
\curveto(612.55158375,49.94499655)(612.54158376,49.86999663)(612.5315918,49.8)
\curveto(612.53158377,49.73999676)(612.52658378,49.66999683)(612.5165918,49.59)
\curveto(612.49658381,49.52999697)(612.48158382,49.46999703)(612.4715918,49.41)
\curveto(612.47158383,49.34999715)(612.46158384,49.28999721)(612.4415918,49.23)
\curveto(612.4015839,49.0999974)(612.36658394,48.96999753)(612.3365918,48.84)
\curveto(612.306584,48.70999779)(612.26658404,48.58999791)(612.2165918,48.48)
\curveto(612.0065843,47.9999985)(611.72658458,47.59499891)(611.3765918,47.265)
\curveto(611.02658528,46.94499955)(610.59658571,46.6999998)(610.0865918,46.53)
\curveto(609.97658633,46.49000001)(609.85658645,46.46000004)(609.7265918,46.44)
\curveto(609.6065867,46.42000008)(609.48158682,46.4000001)(609.3515918,46.38)
\curveto(609.29158701,46.37000013)(609.22658708,46.36500014)(609.1565918,46.365)
\curveto(609.09658721,46.35500015)(609.03658727,46.35000015)(608.9765918,46.35)
\curveto(608.93658737,46.34000016)(608.87658743,46.33500016)(608.7965918,46.335)
\curveto(608.72658758,46.33500016)(608.67658763,46.34000016)(608.6465918,46.35)
\curveto(608.6065877,46.36000014)(608.56658774,46.36500014)(608.5265918,46.365)
\curveto(608.48658782,46.35500015)(608.45158785,46.35500015)(608.4215918,46.365)
\lineto(608.3315918,46.365)
\lineto(607.9715918,46.41)
\curveto(607.83158847,46.45000005)(607.69658861,46.49000001)(607.5665918,46.53)
\curveto(607.43658887,46.56999993)(607.31158899,46.61499988)(607.1915918,46.665)
\curveto(606.74158956,46.86499964)(606.37158993,47.12499938)(606.0815918,47.445)
\curveto(605.79159051,47.76499874)(605.55159075,48.15499835)(605.3615918,48.615)
\curveto(605.31159099,48.71499778)(605.27159103,48.81499768)(605.2415918,48.915)
\curveto(605.22159108,49.01499748)(605.2015911,49.11999738)(605.1815918,49.23)
\curveto(605.16159114,49.26999723)(605.15159115,49.2999972)(605.1515918,49.32)
\curveto(605.16159114,49.34999715)(605.16159114,49.38499711)(605.1515918,49.425)
\curveto(605.13159117,49.50499699)(605.11659119,49.58499691)(605.1065918,49.665)
\curveto(605.1065912,49.75499675)(605.09659121,49.83999666)(605.0765918,49.92)
\lineto(605.0765918,50.04)
\curveto(605.07659123,50.07999642)(605.07159123,50.12499638)(605.0615918,50.175)
\curveto(605.05159125,50.22499628)(605.04659126,50.30999619)(605.0465918,50.43)
\curveto(605.04659126,50.55999594)(605.05659125,50.65499584)(605.0765918,50.715)
\curveto(605.09659121,50.78499571)(605.1015912,50.85499564)(605.0915918,50.925)
\curveto(605.08159122,50.99499551)(605.08659122,51.06499544)(605.1065918,51.135)
\curveto(605.11659119,51.18499531)(605.12159118,51.22499528)(605.1215918,51.255)
\curveto(605.13159117,51.29499521)(605.14159116,51.33999516)(605.1515918,51.39)
\curveto(605.18159112,51.50999499)(605.2065911,51.62999487)(605.2265918,51.75)
\curveto(605.25659105,51.86999463)(605.29659101,51.98499451)(605.3465918,52.095)
\curveto(605.49659081,52.46499404)(605.67659063,52.79499371)(605.8865918,53.085)
\curveto(606.1065902,53.38499311)(606.37158993,53.63499287)(606.6815918,53.835)
\curveto(606.8015895,53.91499258)(606.92658938,53.97999252)(607.0565918,54.03)
\curveto(607.18658912,54.08999241)(607.32158898,54.14999235)(607.4615918,54.21)
\curveto(607.58158872,54.25999224)(607.71158859,54.28999221)(607.8515918,54.3)
\curveto(607.99158831,54.31999218)(608.13158817,54.34999215)(608.2715918,54.39)
\lineto(608.4665918,54.39)
\curveto(608.53658777,54.3999921)(608.6015877,54.40999209)(608.6615918,54.42)
\curveto(609.55158675,54.42999207)(610.29158601,54.24499225)(610.8815918,53.865)
\curveto(611.47158483,53.48499301)(611.89658441,52.98999351)(612.1565918,52.38)
\curveto(612.2065841,52.27999422)(612.24658406,52.17999432)(612.2765918,52.08)
\curveto(612.306584,51.97999452)(612.34158396,51.87499462)(612.3815918,51.765)
\curveto(612.41158389,51.65499484)(612.43658387,51.53499497)(612.4565918,51.405)
\curveto(612.47658383,51.28499521)(612.5015838,51.15999534)(612.5315918,51.03)
\curveto(612.54158376,50.97999552)(612.54158376,50.92499558)(612.5315918,50.865)
\curveto(612.53158377,50.81499568)(612.53658377,50.76499574)(612.5465918,50.715)
\moveto(611.2115918,49.86)
\curveto(611.23158507,49.92999657)(611.23658507,50.00999649)(611.2265918,50.1)
\lineto(611.2265918,50.355)
\curveto(611.22658508,50.74499575)(611.19158511,51.07499542)(611.1215918,51.345)
\curveto(611.09158521,51.42499508)(611.06658524,51.504995)(611.0465918,51.585)
\curveto(611.02658528,51.66499484)(611.0015853,51.73999476)(610.9715918,51.81)
\curveto(610.69158561,52.45999404)(610.24658606,52.90999359)(609.6365918,53.16)
\curveto(609.56658674,53.18999331)(609.49158681,53.20999329)(609.4115918,53.22)
\lineto(609.1715918,53.28)
\curveto(609.09158721,53.2999932)(609.0065873,53.30999319)(608.9165918,53.31)
\lineto(608.6465918,53.31)
\lineto(608.3765918,53.265)
\curveto(608.27658803,53.24499325)(608.18158812,53.21999328)(608.0915918,53.19)
\curveto(608.01158829,53.16999333)(607.93158837,53.13999336)(607.8515918,53.1)
\curveto(607.78158852,53.07999342)(607.71658859,53.04999345)(607.6565918,53.01)
\curveto(607.59658871,52.96999353)(607.54158876,52.92999357)(607.4915918,52.89)
\curveto(607.25158905,52.71999378)(607.05658925,52.51499398)(606.9065918,52.275)
\curveto(606.75658955,52.03499447)(606.62658968,51.75499474)(606.5165918,51.435)
\curveto(606.48658982,51.33499517)(606.46658984,51.22999527)(606.4565918,51.12)
\curveto(606.44658986,51.01999548)(606.43158987,50.91499558)(606.4115918,50.805)
\curveto(606.4015899,50.76499574)(606.39658991,50.6999958)(606.3965918,50.61)
\curveto(606.38658992,50.57999592)(606.38158992,50.54499595)(606.3815918,50.505)
\curveto(606.39158991,50.46499604)(606.39658991,50.41999608)(606.3965918,50.37)
\lineto(606.3965918,50.07)
\curveto(606.39658991,49.96999653)(606.4065899,49.87999662)(606.4265918,49.8)
\lineto(606.4565918,49.62)
\curveto(606.47658983,49.51999698)(606.49158981,49.41999708)(606.5015918,49.32)
\curveto(606.52158978,49.22999727)(606.55158975,49.14499735)(606.5915918,49.065)
\curveto(606.69158961,48.82499768)(606.8065895,48.5999979)(606.9365918,48.39)
\curveto(607.07658923,48.17999832)(607.24658906,48.00499849)(607.4465918,47.865)
\curveto(607.49658881,47.83499867)(607.54158876,47.80999869)(607.5815918,47.79)
\curveto(607.62158868,47.76999873)(607.66658864,47.74499875)(607.7165918,47.715)
\curveto(607.79658851,47.66499884)(607.88158842,47.61999888)(607.9715918,47.58)
\curveto(608.07158823,47.54999895)(608.17658813,47.51999898)(608.2865918,47.49)
\curveto(608.33658797,47.46999903)(608.38158792,47.45999904)(608.4215918,47.46)
\curveto(608.47158783,47.46999903)(608.52158778,47.46999903)(608.5715918,47.46)
\curveto(608.6015877,47.44999905)(608.66158764,47.43999906)(608.7515918,47.43)
\curveto(608.85158745,47.41999908)(608.92658738,47.42499908)(608.9765918,47.445)
\curveto(609.01658729,47.45499905)(609.05658725,47.45499905)(609.0965918,47.445)
\curveto(609.13658717,47.44499905)(609.17658713,47.45499905)(609.2165918,47.475)
\curveto(609.29658701,47.49499901)(609.37658693,47.50999899)(609.4565918,47.52)
\curveto(609.53658677,47.53999896)(609.61158669,47.56499894)(609.6815918,47.595)
\curveto(610.02158628,47.73499877)(610.29658601,47.92999857)(610.5065918,48.18)
\curveto(610.71658559,48.42999807)(610.89158541,48.72499778)(611.0315918,49.065)
\curveto(611.08158522,49.18499731)(611.11158519,49.30999719)(611.1215918,49.44)
\curveto(611.14158516,49.57999692)(611.17158513,49.71999678)(611.2115918,49.86)
}
}
{
\newrgbcolor{curcolor}{0 0 0}
\pscustom[linestyle=none,fillstyle=solid,fillcolor=curcolor]
{
\newpath
\moveto(616.46487305,54.42)
\curveto(617.18486898,54.42999207)(617.78986838,54.34499215)(618.27987305,54.165)
\curveto(618.7698674,53.99499251)(619.14986702,53.68999281)(619.41987305,53.25)
\curveto(619.48986668,53.13999336)(619.54486662,53.02499347)(619.58487305,52.905)
\curveto(619.62486654,52.79499371)(619.6648665,52.66999383)(619.70487305,52.53)
\curveto(619.72486644,52.45999404)(619.72986644,52.38499411)(619.71987305,52.305)
\curveto(619.70986646,52.23499427)(619.69486647,52.17999432)(619.67487305,52.14)
\curveto(619.65486651,52.11999438)(619.62986654,52.0999944)(619.59987305,52.08)
\curveto(619.5698666,52.06999443)(619.54486662,52.05499444)(619.52487305,52.035)
\curveto(619.47486669,52.01499448)(619.42486674,52.00999449)(619.37487305,52.02)
\curveto(619.32486684,52.02999447)(619.27486689,52.02999447)(619.22487305,52.02)
\curveto(619.14486702,51.9999945)(619.03986713,51.99499451)(618.90987305,52.005)
\curveto(618.77986739,52.02499447)(618.68986748,52.04999445)(618.63987305,52.08)
\curveto(618.55986761,52.12999437)(618.50486766,52.19499431)(618.47487305,52.275)
\curveto(618.45486771,52.36499414)(618.41986775,52.44999405)(618.36987305,52.53)
\curveto(618.27986789,52.68999381)(618.15486801,52.83499367)(617.99487305,52.965)
\curveto(617.88486828,53.04499345)(617.7648684,53.1049934)(617.63487305,53.145)
\curveto(617.50486866,53.18499331)(617.3648688,53.22499327)(617.21487305,53.265)
\curveto(617.164869,53.28499321)(617.11486905,53.28999321)(617.06487305,53.28)
\curveto(617.01486915,53.27999322)(616.9648692,53.28499321)(616.91487305,53.295)
\curveto(616.85486931,53.31499318)(616.77986939,53.32499317)(616.68987305,53.325)
\curveto(616.59986957,53.32499317)(616.52486964,53.31499318)(616.46487305,53.295)
\lineto(616.37487305,53.295)
\lineto(616.22487305,53.265)
\curveto(616.17486999,53.26499324)(616.12487004,53.25999324)(616.07487305,53.25)
\curveto(615.81487035,53.18999331)(615.59987057,53.1049934)(615.42987305,52.995)
\curveto(615.25987091,52.88499361)(615.14487102,52.6999938)(615.08487305,52.44)
\curveto(615.0648711,52.36999413)(615.05987111,52.2999942)(615.06987305,52.23)
\curveto(615.08987108,52.15999434)(615.10987106,52.0999944)(615.12987305,52.05)
\curveto(615.18987098,51.8999946)(615.25987091,51.78999471)(615.33987305,51.72)
\curveto(615.42987074,51.65999484)(615.53987063,51.58999491)(615.66987305,51.51)
\curveto(615.82987034,51.40999509)(616.00987016,51.33499517)(616.20987305,51.285)
\curveto(616.40986976,51.24499525)(616.60986956,51.19499531)(616.80987305,51.135)
\curveto(616.93986923,51.09499541)(617.0698691,51.06499544)(617.19987305,51.045)
\curveto(617.32986884,51.02499548)(617.45986871,50.99499551)(617.58987305,50.955)
\curveto(617.79986837,50.89499561)(618.00486816,50.83499567)(618.20487305,50.775)
\curveto(618.40486776,50.72499578)(618.60486756,50.65999584)(618.80487305,50.58)
\lineto(618.95487305,50.52)
\curveto(619.00486716,50.499996)(619.05486711,50.47499602)(619.10487305,50.445)
\curveto(619.30486686,50.32499618)(619.47986669,50.18999631)(619.62987305,50.04)
\curveto(619.77986639,49.88999661)(619.90486626,49.6999968)(620.00487305,49.47)
\curveto(620.02486614,49.3999971)(620.04486612,49.30499719)(620.06487305,49.185)
\curveto(620.08486608,49.11499738)(620.09486607,49.03999746)(620.09487305,48.96)
\curveto(620.10486606,48.88999761)(620.10986606,48.80999769)(620.10987305,48.72)
\lineto(620.10987305,48.57)
\curveto(620.08986608,48.499998)(620.07986609,48.42999807)(620.07987305,48.36)
\curveto(620.07986609,48.28999821)(620.0698661,48.21999828)(620.04987305,48.15)
\curveto(620.01986615,48.03999846)(619.98486618,47.93499857)(619.94487305,47.835)
\curveto(619.90486626,47.73499877)(619.85986631,47.64499885)(619.80987305,47.565)
\curveto(619.64986652,47.30499919)(619.44486672,47.09499941)(619.19487305,46.935)
\curveto(618.94486722,46.78499972)(618.6648675,46.65499985)(618.35487305,46.545)
\curveto(618.2648679,46.51499998)(618.169868,46.49500001)(618.06987305,46.485)
\curveto(617.97986819,46.46500004)(617.88986828,46.44000006)(617.79987305,46.41)
\curveto(617.69986847,46.39000011)(617.59986857,46.38000012)(617.49987305,46.38)
\curveto(617.39986877,46.38000012)(617.29986887,46.37000013)(617.19987305,46.35)
\lineto(617.04987305,46.35)
\curveto(616.99986917,46.34000016)(616.92986924,46.33500016)(616.83987305,46.335)
\curveto(616.74986942,46.33500016)(616.67986949,46.34000016)(616.62987305,46.35)
\lineto(616.46487305,46.35)
\curveto(616.40486976,46.37000013)(616.33986983,46.38000012)(616.26987305,46.38)
\curveto(616.19986997,46.37000013)(616.13987003,46.37500012)(616.08987305,46.395)
\curveto(616.03987013,46.40500009)(615.97487019,46.41000009)(615.89487305,46.41)
\lineto(615.65487305,46.47)
\curveto(615.58487058,46.48000002)(615.50987066,46.5)(615.42987305,46.53)
\curveto(615.11987105,46.62999987)(614.84987132,46.75499975)(614.61987305,46.905)
\curveto(614.38987178,47.05499945)(614.18987198,47.24999925)(614.01987305,47.49)
\curveto(613.92987224,47.61999888)(613.85487231,47.75499875)(613.79487305,47.895)
\curveto(613.73487243,48.03499847)(613.67987249,48.18999831)(613.62987305,48.36)
\curveto(613.60987256,48.41999808)(613.59987257,48.48999801)(613.59987305,48.57)
\curveto(613.60987256,48.65999784)(613.62487254,48.72999777)(613.64487305,48.78)
\curveto(613.67487249,48.81999768)(613.72487244,48.85999764)(613.79487305,48.9)
\curveto(613.84487232,48.91999758)(613.91487225,48.92999757)(614.00487305,48.93)
\curveto(614.09487207,48.93999756)(614.18487198,48.93999756)(614.27487305,48.93)
\curveto(614.3648718,48.91999758)(614.44987172,48.90499759)(614.52987305,48.885)
\curveto(614.61987155,48.87499762)(614.67987149,48.85999764)(614.70987305,48.84)
\curveto(614.77987139,48.78999771)(614.82487134,48.71499778)(614.84487305,48.615)
\curveto(614.87487129,48.52499798)(614.90987126,48.43999806)(614.94987305,48.36)
\curveto(615.04987112,48.13999836)(615.18487098,47.96999853)(615.35487305,47.85)
\curveto(615.47487069,47.75999874)(615.60987056,47.68999881)(615.75987305,47.64)
\curveto(615.90987026,47.58999891)(616.0698701,47.53999896)(616.23987305,47.49)
\lineto(616.55487305,47.445)
\lineto(616.64487305,47.445)
\curveto(616.71486945,47.42499908)(616.80486936,47.41499908)(616.91487305,47.415)
\curveto(617.03486913,47.41499908)(617.13486903,47.42499908)(617.21487305,47.445)
\curveto(617.28486888,47.44499905)(617.33986883,47.44999905)(617.37987305,47.46)
\curveto(617.43986873,47.46999903)(617.49986867,47.47499902)(617.55987305,47.475)
\curveto(617.61986855,47.48499902)(617.67486849,47.49499901)(617.72487305,47.505)
\curveto(618.01486815,47.58499892)(618.24486792,47.68999881)(618.41487305,47.82)
\curveto(618.58486758,47.94999855)(618.70486746,48.16999833)(618.77487305,48.48)
\curveto(618.79486737,48.52999797)(618.79986737,48.58499791)(618.78987305,48.645)
\curveto(618.77986739,48.70499779)(618.7698674,48.74999775)(618.75987305,48.78)
\curveto(618.70986746,48.96999753)(618.63986753,49.10999739)(618.54987305,49.2)
\curveto(618.45986771,49.2999972)(618.34486782,49.38999711)(618.20487305,49.47)
\curveto(618.11486805,49.52999697)(618.01486815,49.57999692)(617.90487305,49.62)
\lineto(617.57487305,49.74)
\curveto(617.54486862,49.74999675)(617.51486865,49.75499675)(617.48487305,49.755)
\curveto(617.4648687,49.75499675)(617.43986873,49.76499674)(617.40987305,49.785)
\curveto(617.0698691,49.89499661)(616.71486945,49.97499652)(616.34487305,50.025)
\curveto(615.98487018,50.08499641)(615.64487052,50.17999632)(615.32487305,50.31)
\curveto(615.22487094,50.34999615)(615.12987104,50.38499611)(615.03987305,50.415)
\curveto(614.94987122,50.44499605)(614.8648713,50.48499601)(614.78487305,50.535)
\curveto(614.59487157,50.64499585)(614.41987175,50.76999573)(614.25987305,50.91)
\curveto(614.09987207,51.04999545)(613.97487219,51.22499528)(613.88487305,51.435)
\curveto(613.85487231,51.504995)(613.82987234,51.57499492)(613.80987305,51.645)
\curveto(613.79987237,51.71499478)(613.78487238,51.78999471)(613.76487305,51.87)
\curveto(613.73487243,51.98999451)(613.72487244,52.12499438)(613.73487305,52.275)
\curveto(613.74487242,52.43499407)(613.75987241,52.56999393)(613.77987305,52.68)
\curveto(613.79987237,52.72999377)(613.80987236,52.76999373)(613.80987305,52.8)
\curveto(613.81987235,52.83999366)(613.83487233,52.87999362)(613.85487305,52.92)
\curveto(613.94487222,53.14999335)(614.0648721,53.34999315)(614.21487305,53.52)
\curveto(614.37487179,53.68999281)(614.55487161,53.83999266)(614.75487305,53.97)
\curveto(614.90487126,54.05999244)(615.0698711,54.12999237)(615.24987305,54.18)
\curveto(615.42987074,54.23999226)(615.61987055,54.29499221)(615.81987305,54.345)
\curveto(615.88987028,54.35499214)(615.95487021,54.36499214)(616.01487305,54.375)
\curveto(616.08487008,54.38499211)(616.15987001,54.39499211)(616.23987305,54.405)
\curveto(616.2698699,54.41499208)(616.30986986,54.41499208)(616.35987305,54.405)
\curveto(616.40986976,54.39499211)(616.44486972,54.3999921)(616.46487305,54.42)
}
}
{
\newrgbcolor{curcolor}{0.7019608 0.7019608 0.7019608}
\pscustom[linestyle=none,fillstyle=solid,fillcolor=curcolor]
{
\newpath
\moveto(134.08439636,211.00222778)
\lineto(162.00564194,211.00222778)
\lineto(162.00564194,86.08279419)
\lineto(134.08439636,86.08279419)
\closepath
}
}
{
\newrgbcolor{curcolor}{0.80000001 0.80000001 0.80000001}
\pscustom[linestyle=none,fillstyle=solid,fillcolor=curcolor]
{
\newpath
\moveto(197.2718811,114.98318481)
\lineto(225.19312668,114.98318481)
\lineto(225.19312668,86.08278275)
\lineto(197.2718811,86.08278275)
\closepath
}
}
{
\newrgbcolor{curcolor}{0.7019608 0.7019608 0.7019608}
\pscustom[linestyle=none,fillstyle=solid,fillcolor=curcolor]
{
\newpath
\moveto(225.3343811,92.00222778)
\lineto(253.25562668,92.00222778)
\lineto(253.25562668,86.0827961)
\lineto(225.3343811,86.0827961)
\closepath
}
}
{
\newrgbcolor{curcolor}{0.80000001 0.80000001 0.80000001}
\pscustom[linestyle=none,fillstyle=solid,fillcolor=curcolor]
{
\newpath
\moveto(288.99200439,121.96588135)
\lineto(316.91324997,121.96588135)
\lineto(316.91324997,86.08280182)
\lineto(288.99200439,86.08280182)
\closepath
}
}
{
\newrgbcolor{curcolor}{0.7019608 0.7019608 0.7019608}
\pscustom[linestyle=none,fillstyle=solid,fillcolor=curcolor]
{
\newpath
\moveto(317.05450439,99.86880493)
\lineto(344.97574997,99.86880493)
\lineto(344.97574997,86.0828104)
\lineto(317.05450439,86.0828104)
\closepath
}
}
{
\newrgbcolor{curcolor}{0.80000001 0.80000001 0.80000001}
\pscustom[linestyle=none,fillstyle=solid,fillcolor=curcolor]
{
\newpath
\moveto(380.11700439,115.51351929)
\lineto(408.03824997,115.51351929)
\lineto(408.03824997,86.08278847)
\lineto(380.11700439,86.08278847)
\closepath
}
}
{
\newrgbcolor{curcolor}{0.7019608 0.7019608 0.7019608}
\pscustom[linestyle=none,fillstyle=solid,fillcolor=curcolor]
{
\newpath
\moveto(408.17950439,181.98156738)
\lineto(436.10074997,181.98156738)
\lineto(436.10074997,86.08280182)
\lineto(408.17950439,86.08280182)
\closepath
}
}
{
\newrgbcolor{curcolor}{0.80000001 0.80000001 0.80000001}
\pscustom[linestyle=none,fillstyle=solid,fillcolor=curcolor]
{
\newpath
\moveto(471.15698242,143.97457886)
\lineto(499.078228,143.97457886)
\lineto(499.078228,86.08280182)
\lineto(471.15698242,86.08280182)
\closepath
}
}
{
\newrgbcolor{curcolor}{0.7019608 0.7019608 0.7019608}
\pscustom[linestyle=none,fillstyle=solid,fillcolor=curcolor]
{
\newpath
\moveto(499.21948242,107.20501709)
\lineto(527.140728,107.20501709)
\lineto(527.140728,86.08279037)
\lineto(499.21948242,86.08279037)
\closepath
}
}
{
\newrgbcolor{curcolor}{0.80000001 0.80000001 0.80000001}
\pscustom[linestyle=none,fillstyle=solid,fillcolor=curcolor]
{
\newpath
\moveto(654.12084961,167.13232422)
\lineto(682.04209518,167.13232422)
\lineto(682.04209518,86.08280182)
\lineto(654.12084961,86.08280182)
\closepath
}
}
{
\newrgbcolor{curcolor}{0.7019608 0.7019608 0.7019608}
\pscustom[linestyle=none,fillstyle=solid,fillcolor=curcolor]
{
\newpath
\moveto(682.18334961,308.02334595)
\lineto(710.10459518,308.02334595)
\lineto(710.10459518,86.08280945)
\lineto(682.18334961,86.08280945)
\closepath
}
}
\end{pspicture}

\caption{Diagrama de barras de los espacios y sus recursos}
\label{espacios_bars_1}
\end{figure}

\begin{figure}
\centering
%LaTeX with PSTricks extensions
%%Creator: inkscape 0.48.5
%%Please note this file requires PSTricks extensions
\psset{xunit=.5pt,yunit=.5pt,runit=.5pt}
\begin{pspicture}(927,369)
{
\newrgbcolor{curcolor}{0 0 0}
\pscustom[linestyle=none,fillstyle=solid,fillcolor=curcolor]
{
\newpath
\moveto(28.17753982,344.30956697)
\curveto(28.17752934,344.2795613)(28.17752934,344.23956134)(28.17753982,344.18956697)
\curveto(28.18752933,344.13956144)(28.19252932,344.0845615)(28.19253982,344.02456697)
\curveto(28.19252932,343.96456162)(28.18752933,343.90956167)(28.17753982,343.85956697)
\curveto(28.17752934,343.80956177)(28.17752934,343.77456181)(28.17753982,343.75456697)
\curveto(28.17752934,343.6845619)(28.17252934,343.61456197)(28.16253982,343.54456697)
\curveto(28.16252935,343.4845621)(28.16252935,343.42456216)(28.16253982,343.36456697)
\curveto(28.14252937,343.31456227)(28.13252938,343.26456232)(28.13253982,343.21456697)
\curveto(28.14252937,343.16456242)(28.14252937,343.11456247)(28.13253982,343.06456697)
\curveto(28.1125294,342.95456263)(28.09752942,342.84456274)(28.08753982,342.73456697)
\curveto(28.07752944,342.62456296)(28.05752946,342.51456307)(28.02753982,342.40456697)
\curveto(27.97752954,342.23456335)(27.93252958,342.06956351)(27.89253982,341.90956697)
\curveto(27.85252966,341.75956382)(27.80252971,341.60956397)(27.74253982,341.45956697)
\curveto(27.57252994,341.03956454)(27.36253015,340.65956492)(27.11253982,340.31956697)
\curveto(26.86253065,339.9795656)(26.56253095,339.68956589)(26.21253982,339.44956697)
\curveto(26.0125315,339.30956627)(25.80253171,339.18956639)(25.58253982,339.08956697)
\curveto(25.37253214,338.98956659)(25.14253237,338.89956668)(24.89253982,338.81956697)
\curveto(24.79253272,338.78956679)(24.68753283,338.76456682)(24.57753982,338.74456697)
\curveto(24.47753304,338.73456685)(24.37253314,338.71456687)(24.26253982,338.68456697)
\curveto(24.2125333,338.67456691)(24.16253335,338.66956691)(24.11253982,338.66956697)
\curveto(24.07253344,338.66956691)(24.02753349,338.66456692)(23.97753982,338.65456697)
\curveto(23.93753358,338.64456694)(23.89753362,338.63956694)(23.85753982,338.63956697)
\curveto(23.8175337,338.64956693)(23.77253374,338.64956693)(23.72253982,338.63956697)
\curveto(23.70253381,338.62956695)(23.67253384,338.62456696)(23.63253982,338.62456697)
\curveto(23.59253392,338.63456695)(23.56253395,338.63456695)(23.54253982,338.62456697)
\curveto(23.46253405,338.60456698)(23.36253415,338.59956698)(23.24253982,338.60956697)
\curveto(23.12253439,338.61956696)(23.0175345,338.62456696)(22.92753982,338.62456697)
\lineto(19.43253982,338.62456697)
\curveto(19.26253825,338.62456696)(19.1175384,338.62956695)(18.99753982,338.63956697)
\curveto(18.88753863,338.65956692)(18.80753871,338.72956685)(18.75753982,338.84956697)
\curveto(18.72753879,338.92956665)(18.7125388,339.04956653)(18.71253982,339.20956697)
\curveto(18.72253879,339.3795662)(18.72753879,339.51956606)(18.72753982,339.62956697)
\lineto(18.72753982,348.43456697)
\curveto(18.72753879,348.55455703)(18.72253879,348.6795569)(18.71253982,348.80956697)
\curveto(18.7125388,348.94955663)(18.73753878,349.05955652)(18.78753982,349.13956697)
\curveto(18.82753869,349.19955638)(18.90253861,349.24955633)(19.01253982,349.28956697)
\curveto(19.03253848,349.29955628)(19.05253846,349.29955628)(19.07253982,349.28956697)
\curveto(19.09253842,349.28955629)(19.1125384,349.29455629)(19.13253982,349.30456697)
\lineto(23.16753982,349.30456697)
\curveto(23.22753429,349.30455628)(23.28753423,349.30455628)(23.34753982,349.30456697)
\curveto(23.4175341,349.31455627)(23.47753404,349.31455627)(23.52753982,349.30456697)
\lineto(23.70753982,349.30456697)
\curveto(23.75753376,349.2845563)(23.8125337,349.27455631)(23.87253982,349.27456697)
\curveto(23.93253358,349.2845563)(23.98753353,349.2795563)(24.03753982,349.25956697)
\curveto(24.09753342,349.23955634)(24.15253336,349.22955635)(24.20253982,349.22956697)
\curveto(24.26253325,349.23955634)(24.32253319,349.23455635)(24.38253982,349.21456697)
\curveto(24.52253299,349.1845564)(24.65753286,349.15455643)(24.78753982,349.12456697)
\curveto(24.9175326,349.10455648)(25.04253247,349.06955651)(25.16253982,349.01956697)
\curveto(25.27253224,348.96955661)(25.38253213,348.92455666)(25.49253982,348.88456697)
\curveto(25.60253191,348.84455674)(25.70753181,348.79455679)(25.80753982,348.73456697)
\curveto(26.05753146,348.57455701)(26.28753123,348.41955716)(26.49753982,348.26956697)
\lineto(26.58753982,348.17956697)
\curveto(26.68753083,348.09955748)(26.77753074,348.00955757)(26.85753982,347.90956697)
\lineto(26.99253982,347.78956697)
\curveto(27.04253047,347.70955787)(27.09753042,347.62955795)(27.15753982,347.54956697)
\curveto(27.22753029,347.4795581)(27.28753023,347.40455818)(27.33753982,347.32456697)
\curveto(27.46753005,347.11455847)(27.58252993,346.88955869)(27.68253982,346.64956697)
\curveto(27.78252973,346.41955916)(27.87252964,346.17455941)(27.95253982,345.91456697)
\curveto(28.00252951,345.7845598)(28.03252948,345.64955993)(28.04253982,345.50956697)
\curveto(28.06252945,345.36956021)(28.08752943,345.22956035)(28.11753982,345.08956697)
\curveto(28.1175294,345.03956054)(28.1175294,344.99456059)(28.11753982,344.95456697)
\curveto(28.12752939,344.92456066)(28.13252938,344.88956069)(28.13253982,344.84956697)
\curveto(28.15252936,344.78956079)(28.15752936,344.72456086)(28.14753982,344.65456697)
\curveto(28.14752937,344.584561)(28.15752936,344.52456106)(28.17753982,344.47456697)
\lineto(28.17753982,344.30956697)
\moveto(25.83753982,343.58956697)
\curveto(25.85753166,343.63956194)(25.86753165,343.71956186)(25.86753982,343.82956697)
\curveto(25.86753165,343.93956164)(25.85753166,344.01956156)(25.83753982,344.06956697)
\lineto(25.83753982,344.35456697)
\curveto(25.8175317,344.44456114)(25.80253171,344.53956104)(25.79253982,344.63956697)
\curveto(25.79253172,344.73956084)(25.78253173,344.82956075)(25.76253982,344.90956697)
\curveto(25.74253177,344.95956062)(25.73253178,345.00456058)(25.73253982,345.04456697)
\curveto(25.74253177,345.09456049)(25.73753178,345.14456044)(25.71753982,345.19456697)
\curveto(25.66753185,345.35456023)(25.6175319,345.50456008)(25.56753982,345.64456697)
\curveto(25.52753199,345.79455979)(25.46753205,345.93455965)(25.38753982,346.06456697)
\curveto(25.23753228,346.30455928)(25.06253245,346.50955907)(24.86253982,346.67956697)
\curveto(24.67253284,346.85955872)(24.43753308,347.00955857)(24.15753982,347.12956697)
\curveto(24.06753345,347.15955842)(23.97753354,347.1845584)(23.88753982,347.20456697)
\curveto(23.79753372,347.23455835)(23.70753381,347.25955832)(23.61753982,347.27956697)
\curveto(23.53753398,347.28955829)(23.46253405,347.29455829)(23.39253982,347.29456697)
\curveto(23.33253418,347.30455828)(23.26253425,347.31955826)(23.18253982,347.33956697)
\curveto(23.14253437,347.34955823)(23.10253441,347.34955823)(23.06253982,347.33956697)
\curveto(23.02253449,347.33955824)(22.98753453,347.34455824)(22.95753982,347.35456697)
\lineto(22.62753982,347.35456697)
\curveto(22.57753494,347.36455822)(22.52253499,347.36455822)(22.46253982,347.35456697)
\lineto(22.28253982,347.35456697)
\lineto(21.60753982,347.35456697)
\curveto(21.58753593,347.33455825)(21.55253596,347.32955825)(21.50253982,347.33956697)
\curveto(21.46253605,347.34955823)(21.42753609,347.34955823)(21.39753982,347.33956697)
\lineto(21.24753982,347.27956697)
\curveto(21.19753632,347.26955831)(21.15753636,347.23955834)(21.12753982,347.18956697)
\curveto(21.08753643,347.13955844)(21.06753645,347.06955851)(21.06753982,346.97956697)
\lineto(21.06753982,346.67956697)
\curveto(21.06753645,346.54955903)(21.06253645,346.41455917)(21.05253982,346.27456697)
\lineto(21.05253982,345.85456697)
\lineto(21.05253982,341.66956697)
\curveto(21.05253646,341.60956397)(21.04753647,341.54456404)(21.03753982,341.47456697)
\curveto(21.03753648,341.40456418)(21.04753647,341.34456424)(21.06753982,341.29456697)
\lineto(21.06753982,341.14456697)
\lineto(21.06753982,340.93456697)
\curveto(21.07753644,340.87456471)(21.09253642,340.81956476)(21.11253982,340.76956697)
\curveto(21.17253634,340.64956493)(21.28753623,340.584565)(21.45753982,340.57456697)
\lineto(21.98253982,340.57456697)
\lineto(23.16753982,340.57456697)
\curveto(23.56753395,340.584565)(23.90753361,340.64456494)(24.18753982,340.75456697)
\curveto(24.55753296,340.90456468)(24.84753267,341.10456448)(25.05753982,341.35456697)
\curveto(25.27753224,341.60456398)(25.46253205,341.91456367)(25.61253982,342.28456697)
\curveto(25.65253186,342.36456322)(25.68253183,342.45456313)(25.70253982,342.55456697)
\curveto(25.72253179,342.65456293)(25.74753177,342.75456283)(25.77753982,342.85456697)
\lineto(25.77753982,342.97456697)
\curveto(25.79753172,343.04456254)(25.80753171,343.11956246)(25.80753982,343.19956697)
\curveto(25.80753171,343.2795623)(25.8175317,343.35956222)(25.83753982,343.43956697)
\lineto(25.83753982,343.58956697)
}
}
{
\newrgbcolor{curcolor}{0 0 0}
\pscustom[linestyle=none,fillstyle=solid,fillcolor=curcolor]
{
\newpath
\moveto(31.67605545,349.19956697)
\curveto(31.7460525,349.11955646)(31.78105246,348.99955658)(31.78105545,348.83956697)
\lineto(31.78105545,348.37456697)
\lineto(31.78105545,347.96956697)
\curveto(31.78105246,347.82955775)(31.7460525,347.73455785)(31.67605545,347.68456697)
\curveto(31.61605263,347.63455795)(31.53605271,347.60455798)(31.43605545,347.59456697)
\curveto(31.3460529,347.584558)(31.246053,347.579558)(31.13605545,347.57956697)
\lineto(30.29605545,347.57956697)
\curveto(30.18605406,347.579558)(30.08605416,347.584558)(29.99605545,347.59456697)
\curveto(29.91605433,347.60455798)(29.8460544,347.63455795)(29.78605545,347.68456697)
\curveto(29.7460545,347.71455787)(29.71605453,347.76955781)(29.69605545,347.84956697)
\curveto(29.68605456,347.93955764)(29.67605457,348.03455755)(29.66605545,348.13456697)
\lineto(29.66605545,348.46456697)
\curveto(29.67605457,348.57455701)(29.68105456,348.66955691)(29.68105545,348.74956697)
\lineto(29.68105545,348.95956697)
\curveto(29.69105455,349.02955655)(29.71105453,349.08955649)(29.74105545,349.13956697)
\curveto(29.76105448,349.1795564)(29.78605446,349.20955637)(29.81605545,349.22956697)
\lineto(29.93605545,349.28956697)
\curveto(29.95605429,349.28955629)(29.98105426,349.28955629)(30.01105545,349.28956697)
\curveto(30.0410542,349.29955628)(30.06605418,349.30455628)(30.08605545,349.30456697)
\lineto(31.18105545,349.30456697)
\curveto(31.28105296,349.30455628)(31.37605287,349.29955628)(31.46605545,349.28956697)
\curveto(31.55605269,349.2795563)(31.62605262,349.24955633)(31.67605545,349.19956697)
\moveto(31.78105545,339.43456697)
\curveto(31.78105246,339.23456635)(31.77605247,339.06456652)(31.76605545,338.92456697)
\curveto(31.75605249,338.7845668)(31.66605258,338.68956689)(31.49605545,338.63956697)
\curveto(31.43605281,338.61956696)(31.37105287,338.60956697)(31.30105545,338.60956697)
\curveto(31.23105301,338.61956696)(31.15605309,338.62456696)(31.07605545,338.62456697)
\lineto(30.23605545,338.62456697)
\curveto(30.1460541,338.62456696)(30.05605419,338.62956695)(29.96605545,338.63956697)
\curveto(29.88605436,338.64956693)(29.82605442,338.6795669)(29.78605545,338.72956697)
\curveto(29.72605452,338.79956678)(29.69105455,338.8845667)(29.68105545,338.98456697)
\lineto(29.68105545,339.32956697)
\lineto(29.68105545,345.65956697)
\lineto(29.68105545,345.95956697)
\curveto(29.68105456,346.05955952)(29.70105454,346.13955944)(29.74105545,346.19956697)
\curveto(29.80105444,346.26955931)(29.88605436,346.31455927)(29.99605545,346.33456697)
\curveto(30.01605423,346.34455924)(30.0410542,346.34455924)(30.07105545,346.33456697)
\curveto(30.11105413,346.33455925)(30.1410541,346.33955924)(30.16105545,346.34956697)
\lineto(30.91105545,346.34956697)
\lineto(31.10605545,346.34956697)
\curveto(31.18605306,346.35955922)(31.25105299,346.35955922)(31.30105545,346.34956697)
\lineto(31.42105545,346.34956697)
\curveto(31.48105276,346.32955925)(31.53605271,346.31455927)(31.58605545,346.30456697)
\curveto(31.63605261,346.29455929)(31.67605257,346.26455932)(31.70605545,346.21456697)
\curveto(31.7460525,346.16455942)(31.76605248,346.09455949)(31.76605545,346.00456697)
\curveto(31.77605247,345.91455967)(31.78105246,345.81955976)(31.78105545,345.71956697)
\lineto(31.78105545,339.43456697)
}
}
{
\newrgbcolor{curcolor}{0 0 0}
\pscustom[linestyle=none,fillstyle=solid,fillcolor=curcolor]
{
\newpath
\moveto(36.41324295,346.55956697)
\curveto(37.16323845,346.579559)(37.8132378,346.49455909)(38.36324295,346.30456697)
\curveto(38.92323669,346.12455946)(39.34823626,345.80955977)(39.63824295,345.35956697)
\curveto(39.7082359,345.24956033)(39.76823584,345.13456045)(39.81824295,345.01456697)
\curveto(39.87823573,344.90456068)(39.92823568,344.7795608)(39.96824295,344.63956697)
\curveto(39.98823562,344.579561)(39.99823561,344.51456107)(39.99824295,344.44456697)
\curveto(39.99823561,344.37456121)(39.98823562,344.31456127)(39.96824295,344.26456697)
\curveto(39.92823568,344.20456138)(39.87323574,344.16456142)(39.80324295,344.14456697)
\curveto(39.75323586,344.12456146)(39.69323592,344.11456147)(39.62324295,344.11456697)
\lineto(39.41324295,344.11456697)
\lineto(38.75324295,344.11456697)
\curveto(38.68323693,344.11456147)(38.613237,344.10956147)(38.54324295,344.09956697)
\curveto(38.47323714,344.09956148)(38.4082372,344.10956147)(38.34824295,344.12956697)
\curveto(38.24823736,344.14956143)(38.17323744,344.18956139)(38.12324295,344.24956697)
\curveto(38.07323754,344.30956127)(38.02823758,344.36956121)(37.98824295,344.42956697)
\lineto(37.86824295,344.63956697)
\curveto(37.83823777,344.71956086)(37.78823782,344.7845608)(37.71824295,344.83456697)
\curveto(37.61823799,344.91456067)(37.51823809,344.97456061)(37.41824295,345.01456697)
\curveto(37.32823828,345.05456053)(37.2132384,345.08956049)(37.07324295,345.11956697)
\curveto(37.00323861,345.13956044)(36.89823871,345.15456043)(36.75824295,345.16456697)
\curveto(36.62823898,345.17456041)(36.52823908,345.16956041)(36.45824295,345.14956697)
\lineto(36.35324295,345.14956697)
\lineto(36.20324295,345.11956697)
\curveto(36.16323945,345.11956046)(36.11823949,345.11456047)(36.06824295,345.10456697)
\curveto(35.89823971,345.05456053)(35.75823985,344.9845606)(35.64824295,344.89456697)
\curveto(35.54824006,344.81456077)(35.47824013,344.68956089)(35.43824295,344.51956697)
\curveto(35.41824019,344.44956113)(35.41824019,344.3845612)(35.43824295,344.32456697)
\curveto(35.45824015,344.26456132)(35.47824013,344.21456137)(35.49824295,344.17456697)
\curveto(35.56824004,344.05456153)(35.64823996,343.95956162)(35.73824295,343.88956697)
\curveto(35.83823977,343.81956176)(35.95323966,343.75956182)(36.08324295,343.70956697)
\curveto(36.27323934,343.62956195)(36.47823913,343.55956202)(36.69824295,343.49956697)
\lineto(37.38824295,343.34956697)
\curveto(37.62823798,343.30956227)(37.85823775,343.25956232)(38.07824295,343.19956697)
\curveto(38.3082373,343.14956243)(38.52323709,343.0845625)(38.72324295,343.00456697)
\curveto(38.8132368,342.96456262)(38.89823671,342.92956265)(38.97824295,342.89956697)
\curveto(39.06823654,342.8795627)(39.15323646,342.84456274)(39.23324295,342.79456697)
\curveto(39.42323619,342.67456291)(39.59323602,342.54456304)(39.74324295,342.40456697)
\curveto(39.90323571,342.26456332)(40.02823558,342.08956349)(40.11824295,341.87956697)
\curveto(40.14823546,341.80956377)(40.17323544,341.73956384)(40.19324295,341.66956697)
\curveto(40.2132354,341.59956398)(40.23323538,341.52456406)(40.25324295,341.44456697)
\curveto(40.26323535,341.3845642)(40.26823534,341.28956429)(40.26824295,341.15956697)
\curveto(40.27823533,341.03956454)(40.27823533,340.94456464)(40.26824295,340.87456697)
\lineto(40.26824295,340.79956697)
\curveto(40.24823536,340.73956484)(40.23323538,340.6795649)(40.22324295,340.61956697)
\curveto(40.22323539,340.56956501)(40.21823539,340.51956506)(40.20824295,340.46956697)
\curveto(40.13823547,340.16956541)(40.02823558,339.90456568)(39.87824295,339.67456697)
\curveto(39.71823589,339.43456615)(39.52323609,339.23956634)(39.29324295,339.08956697)
\curveto(39.06323655,338.93956664)(38.80323681,338.80956677)(38.51324295,338.69956697)
\curveto(38.40323721,338.64956693)(38.28323733,338.61456697)(38.15324295,338.59456697)
\curveto(38.03323758,338.57456701)(37.9132377,338.54956703)(37.79324295,338.51956697)
\curveto(37.70323791,338.49956708)(37.608238,338.48956709)(37.50824295,338.48956697)
\curveto(37.41823819,338.4795671)(37.32823828,338.46456712)(37.23824295,338.44456697)
\lineto(36.96824295,338.44456697)
\curveto(36.9082387,338.42456716)(36.80323881,338.41456717)(36.65324295,338.41456697)
\curveto(36.5132391,338.41456717)(36.4132392,338.42456716)(36.35324295,338.44456697)
\curveto(36.32323929,338.44456714)(36.28823932,338.44956713)(36.24824295,338.45956697)
\lineto(36.14324295,338.45956697)
\curveto(36.02323959,338.4795671)(35.90323971,338.49456709)(35.78324295,338.50456697)
\curveto(35.66323995,338.51456707)(35.54824006,338.53456705)(35.43824295,338.56456697)
\curveto(35.04824056,338.67456691)(34.70324091,338.79956678)(34.40324295,338.93956697)
\curveto(34.10324151,339.08956649)(33.84824176,339.30956627)(33.63824295,339.59956697)
\curveto(33.49824211,339.78956579)(33.37824223,340.00956557)(33.27824295,340.25956697)
\curveto(33.25824235,340.31956526)(33.23824237,340.39956518)(33.21824295,340.49956697)
\curveto(33.19824241,340.54956503)(33.18324243,340.61956496)(33.17324295,340.70956697)
\curveto(33.16324245,340.79956478)(33.16824244,340.87456471)(33.18824295,340.93456697)
\curveto(33.21824239,341.00456458)(33.26824234,341.05456453)(33.33824295,341.08456697)
\curveto(33.38824222,341.10456448)(33.44824216,341.11456447)(33.51824295,341.11456697)
\lineto(33.74324295,341.11456697)
\lineto(34.44824295,341.11456697)
\lineto(34.68824295,341.11456697)
\curveto(34.76824084,341.11456447)(34.83824077,341.10456448)(34.89824295,341.08456697)
\curveto(35.0082406,341.04456454)(35.07824053,340.9795646)(35.10824295,340.88956697)
\curveto(35.14824046,340.79956478)(35.19324042,340.70456488)(35.24324295,340.60456697)
\curveto(35.26324035,340.55456503)(35.29824031,340.48956509)(35.34824295,340.40956697)
\curveto(35.4082402,340.32956525)(35.45824015,340.2795653)(35.49824295,340.25956697)
\curveto(35.61823999,340.15956542)(35.73323988,340.0795655)(35.84324295,340.01956697)
\curveto(35.95323966,339.96956561)(36.09323952,339.91956566)(36.26324295,339.86956697)
\curveto(36.3132393,339.84956573)(36.36323925,339.83956574)(36.41324295,339.83956697)
\curveto(36.46323915,339.84956573)(36.5132391,339.84956573)(36.56324295,339.83956697)
\curveto(36.64323897,339.81956576)(36.72823888,339.80956577)(36.81824295,339.80956697)
\curveto(36.91823869,339.81956576)(37.00323861,339.83456575)(37.07324295,339.85456697)
\curveto(37.12323849,339.86456572)(37.16823844,339.86956571)(37.20824295,339.86956697)
\curveto(37.25823835,339.86956571)(37.3082383,339.8795657)(37.35824295,339.89956697)
\curveto(37.49823811,339.94956563)(37.62323799,340.00956557)(37.73324295,340.07956697)
\curveto(37.85323776,340.14956543)(37.94823766,340.23956534)(38.01824295,340.34956697)
\curveto(38.06823754,340.42956515)(38.1082375,340.55456503)(38.13824295,340.72456697)
\curveto(38.15823745,340.79456479)(38.15823745,340.85956472)(38.13824295,340.91956697)
\curveto(38.11823749,340.9795646)(38.09823751,341.02956455)(38.07824295,341.06956697)
\curveto(38.0082376,341.20956437)(37.91823769,341.31456427)(37.80824295,341.38456697)
\curveto(37.7082379,341.45456413)(37.58823802,341.51956406)(37.44824295,341.57956697)
\curveto(37.25823835,341.65956392)(37.05823855,341.72456386)(36.84824295,341.77456697)
\curveto(36.63823897,341.82456376)(36.42823918,341.8795637)(36.21824295,341.93956697)
\curveto(36.13823947,341.95956362)(36.05323956,341.97456361)(35.96324295,341.98456697)
\curveto(35.88323973,341.99456359)(35.80323981,342.00956357)(35.72324295,342.02956697)
\curveto(35.40324021,342.11956346)(35.09824051,342.20456338)(34.80824295,342.28456697)
\curveto(34.51824109,342.37456321)(34.25324136,342.50456308)(34.01324295,342.67456697)
\curveto(33.73324188,342.87456271)(33.52824208,343.14456244)(33.39824295,343.48456697)
\curveto(33.37824223,343.55456203)(33.35824225,343.64956193)(33.33824295,343.76956697)
\curveto(33.31824229,343.83956174)(33.30324231,343.92456166)(33.29324295,344.02456697)
\curveto(33.28324233,344.12456146)(33.28824232,344.21456137)(33.30824295,344.29456697)
\curveto(33.32824228,344.34456124)(33.33324228,344.3845612)(33.32324295,344.41456697)
\curveto(33.3132423,344.45456113)(33.31824229,344.49956108)(33.33824295,344.54956697)
\curveto(33.35824225,344.65956092)(33.37824223,344.75956082)(33.39824295,344.84956697)
\curveto(33.42824218,344.94956063)(33.46324215,345.04456054)(33.50324295,345.13456697)
\curveto(33.63324198,345.42456016)(33.8132418,345.65955992)(34.04324295,345.83956697)
\curveto(34.27324134,346.01955956)(34.53324108,346.16455942)(34.82324295,346.27456697)
\curveto(34.93324068,346.32455926)(35.04824056,346.35955922)(35.16824295,346.37956697)
\curveto(35.28824032,346.40955917)(35.4132402,346.43955914)(35.54324295,346.46956697)
\curveto(35.60324001,346.48955909)(35.66323995,346.49955908)(35.72324295,346.49956697)
\lineto(35.90324295,346.52956697)
\curveto(35.98323963,346.53955904)(36.06823954,346.54455904)(36.15824295,346.54456697)
\curveto(36.24823936,346.54455904)(36.33323928,346.54955903)(36.41324295,346.55956697)
}
}
{
\newrgbcolor{curcolor}{0 0 0}
\pscustom[linestyle=none,fillstyle=solid,fillcolor=curcolor]
{
\newpath
\moveto(42.54988357,348.65956697)
\lineto(43.55488357,348.65956697)
\curveto(43.70488059,348.65955692)(43.83488046,348.64955693)(43.94488357,348.62956697)
\curveto(44.06488023,348.61955696)(44.14988014,348.55955702)(44.19988357,348.44956697)
\curveto(44.21988007,348.39955718)(44.22988006,348.33955724)(44.22988357,348.26956697)
\lineto(44.22988357,348.05956697)
\lineto(44.22988357,347.38456697)
\curveto(44.22988006,347.33455825)(44.22488007,347.27455831)(44.21488357,347.20456697)
\curveto(44.21488008,347.14455844)(44.21988007,347.08955849)(44.22988357,347.03956697)
\lineto(44.22988357,346.87456697)
\curveto(44.22988006,346.79455879)(44.23488006,346.71955886)(44.24488357,346.64956697)
\curveto(44.25488004,346.58955899)(44.27988001,346.53455905)(44.31988357,346.48456697)
\curveto(44.3898799,346.39455919)(44.51487978,346.34455924)(44.69488357,346.33456697)
\lineto(45.23488357,346.33456697)
\lineto(45.41488357,346.33456697)
\curveto(45.47487882,346.33455925)(45.52987876,346.32455926)(45.57988357,346.30456697)
\curveto(45.6898786,346.25455933)(45.74987854,346.16455942)(45.75988357,346.03456697)
\curveto(45.77987851,345.90455968)(45.7898785,345.75955982)(45.78988357,345.59956697)
\lineto(45.78988357,345.38956697)
\curveto(45.79987849,345.31956026)(45.7948785,345.25956032)(45.77488357,345.20956697)
\curveto(45.72487857,345.04956053)(45.61987867,344.96456062)(45.45988357,344.95456697)
\curveto(45.29987899,344.94456064)(45.11987917,344.93956064)(44.91988357,344.93956697)
\lineto(44.78488357,344.93956697)
\curveto(44.74487955,344.94956063)(44.70987958,344.94956063)(44.67988357,344.93956697)
\curveto(44.63987965,344.92956065)(44.60487969,344.92456066)(44.57488357,344.92456697)
\curveto(44.54487975,344.93456065)(44.51487978,344.92956065)(44.48488357,344.90956697)
\curveto(44.40487989,344.88956069)(44.34487995,344.84456074)(44.30488357,344.77456697)
\curveto(44.27488002,344.71456087)(44.24988004,344.63956094)(44.22988357,344.54956697)
\curveto(44.21988007,344.49956108)(44.21988007,344.44456114)(44.22988357,344.38456697)
\curveto(44.23988005,344.32456126)(44.23988005,344.26956131)(44.22988357,344.21956697)
\lineto(44.22988357,343.28956697)
\lineto(44.22988357,341.53456697)
\curveto(44.22988006,341.2845643)(44.23488006,341.06456452)(44.24488357,340.87456697)
\curveto(44.26488003,340.69456489)(44.32987996,340.53456505)(44.43988357,340.39456697)
\curveto(44.4898798,340.33456525)(44.55487974,340.28956529)(44.63488357,340.25956697)
\lineto(44.90488357,340.19956697)
\curveto(44.93487936,340.18956539)(44.96487933,340.1845654)(44.99488357,340.18456697)
\curveto(45.03487926,340.19456539)(45.06487923,340.19456539)(45.08488357,340.18456697)
\lineto(45.24988357,340.18456697)
\curveto(45.35987893,340.1845654)(45.45487884,340.1795654)(45.53488357,340.16956697)
\curveto(45.61487868,340.15956542)(45.67987861,340.11956546)(45.72988357,340.04956697)
\curveto(45.76987852,339.98956559)(45.7898785,339.90956567)(45.78988357,339.80956697)
\lineto(45.78988357,339.52456697)
\curveto(45.7898785,339.31456627)(45.78487851,339.11956646)(45.77488357,338.93956697)
\curveto(45.77487852,338.76956681)(45.6948786,338.65456693)(45.53488357,338.59456697)
\curveto(45.48487881,338.57456701)(45.43987885,338.56956701)(45.39988357,338.57956697)
\curveto(45.35987893,338.579567)(45.31487898,338.56956701)(45.26488357,338.54956697)
\lineto(45.11488357,338.54956697)
\curveto(45.0948792,338.54956703)(45.06487923,338.55456703)(45.02488357,338.56456697)
\curveto(44.98487931,338.56456702)(44.94987934,338.55956702)(44.91988357,338.54956697)
\curveto(44.86987942,338.53956704)(44.81487948,338.53956704)(44.75488357,338.54956697)
\lineto(44.60488357,338.54956697)
\lineto(44.45488357,338.54956697)
\curveto(44.40487989,338.53956704)(44.35987993,338.53956704)(44.31988357,338.54956697)
\lineto(44.15488357,338.54956697)
\curveto(44.10488019,338.55956702)(44.04988024,338.56456702)(43.98988357,338.56456697)
\curveto(43.92988036,338.56456702)(43.87488042,338.56956701)(43.82488357,338.57956697)
\curveto(43.75488054,338.58956699)(43.6898806,338.59956698)(43.62988357,338.60956697)
\lineto(43.44988357,338.63956697)
\curveto(43.33988095,338.66956691)(43.23488106,338.70456688)(43.13488357,338.74456697)
\curveto(43.03488126,338.7845668)(42.93988135,338.82956675)(42.84988357,338.87956697)
\lineto(42.75988357,338.93956697)
\curveto(42.72988156,338.96956661)(42.6948816,338.99956658)(42.65488357,339.02956697)
\curveto(42.63488166,339.04956653)(42.60988168,339.06956651)(42.57988357,339.08956697)
\lineto(42.50488357,339.16456697)
\curveto(42.36488193,339.35456623)(42.25988203,339.56456602)(42.18988357,339.79456697)
\curveto(42.16988212,339.83456575)(42.15988213,339.86956571)(42.15988357,339.89956697)
\curveto(42.16988212,339.93956564)(42.16988212,339.9845656)(42.15988357,340.03456697)
\curveto(42.14988214,340.05456553)(42.14488215,340.0795655)(42.14488357,340.10956697)
\curveto(42.14488215,340.13956544)(42.13988215,340.16456542)(42.12988357,340.18456697)
\lineto(42.12988357,340.33456697)
\curveto(42.11988217,340.37456521)(42.11488218,340.41956516)(42.11488357,340.46956697)
\curveto(42.12488217,340.51956506)(42.12988216,340.56956501)(42.12988357,340.61956697)
\lineto(42.12988357,341.18956697)
\lineto(42.12988357,343.42456697)
\lineto(42.12988357,344.21956697)
\lineto(42.12988357,344.42956697)
\curveto(42.13988215,344.49956108)(42.13488216,344.56456102)(42.11488357,344.62456697)
\curveto(42.07488222,344.76456082)(42.00488229,344.85456073)(41.90488357,344.89456697)
\curveto(41.7948825,344.94456064)(41.65488264,344.95956062)(41.48488357,344.93956697)
\curveto(41.31488298,344.91956066)(41.16988312,344.93456065)(41.04988357,344.98456697)
\curveto(40.96988332,345.01456057)(40.91988337,345.05956052)(40.89988357,345.11956697)
\curveto(40.87988341,345.1795604)(40.85988343,345.25456033)(40.83988357,345.34456697)
\lineto(40.83988357,345.65956697)
\curveto(40.83988345,345.83955974)(40.84988344,345.9845596)(40.86988357,346.09456697)
\curveto(40.8898834,346.20455938)(40.97488332,346.2795593)(41.12488357,346.31956697)
\curveto(41.16488313,346.33955924)(41.20488309,346.34455924)(41.24488357,346.33456697)
\lineto(41.37988357,346.33456697)
\curveto(41.52988276,346.33455925)(41.66988262,346.33955924)(41.79988357,346.34956697)
\curveto(41.92988236,346.36955921)(42.01988227,346.42955915)(42.06988357,346.52956697)
\curveto(42.09988219,346.59955898)(42.11488218,346.6795589)(42.11488357,346.76956697)
\curveto(42.12488217,346.85955872)(42.12988216,346.94955863)(42.12988357,347.03956697)
\lineto(42.12988357,347.96956697)
\lineto(42.12988357,348.22456697)
\curveto(42.12988216,348.31455727)(42.13988215,348.38955719)(42.15988357,348.44956697)
\curveto(42.20988208,348.54955703)(42.28488201,348.61455697)(42.38488357,348.64456697)
\curveto(42.40488189,348.65455693)(42.42988186,348.65455693)(42.45988357,348.64456697)
\curveto(42.49988179,348.64455694)(42.52988176,348.64955693)(42.54988357,348.65956697)
}
}
{
\newrgbcolor{curcolor}{0 0 0}
\pscustom[linestyle=none,fillstyle=solid,fillcolor=curcolor]
{
\newpath
\moveto(51.19832107,346.54456697)
\curveto(51.30831576,346.54455904)(51.40331566,346.53455905)(51.48332107,346.51456697)
\curveto(51.57331549,346.49455909)(51.64331542,346.44955913)(51.69332107,346.37956697)
\curveto(51.75331531,346.29955928)(51.78331528,346.15955942)(51.78332107,345.95956697)
\lineto(51.78332107,345.44956697)
\lineto(51.78332107,345.07456697)
\curveto(51.79331527,344.93456065)(51.77831529,344.82456076)(51.73832107,344.74456697)
\curveto(51.69831537,344.67456091)(51.63831543,344.62956095)(51.55832107,344.60956697)
\curveto(51.48831558,344.58956099)(51.40331566,344.579561)(51.30332107,344.57956697)
\curveto(51.21331585,344.579561)(51.11331595,344.584561)(51.00332107,344.59456697)
\curveto(50.90331616,344.60456098)(50.80831626,344.59956098)(50.71832107,344.57956697)
\curveto(50.64831642,344.55956102)(50.57831649,344.54456104)(50.50832107,344.53456697)
\curveto(50.43831663,344.53456105)(50.37331669,344.52456106)(50.31332107,344.50456697)
\curveto(50.15331691,344.45456113)(49.99331707,344.3795612)(49.83332107,344.27956697)
\curveto(49.67331739,344.18956139)(49.54831752,344.0845615)(49.45832107,343.96456697)
\curveto(49.40831766,343.8845617)(49.35331771,343.79956178)(49.29332107,343.70956697)
\curveto(49.24331782,343.62956195)(49.19331787,343.54456204)(49.14332107,343.45456697)
\curveto(49.11331795,343.37456221)(49.08331798,343.28956229)(49.05332107,343.19956697)
\lineto(48.99332107,342.95956697)
\curveto(48.97331809,342.88956269)(48.9633181,342.81456277)(48.96332107,342.73456697)
\curveto(48.9633181,342.66456292)(48.95331811,342.59456299)(48.93332107,342.52456697)
\curveto(48.92331814,342.4845631)(48.91831815,342.44456314)(48.91832107,342.40456697)
\curveto(48.92831814,342.37456321)(48.92831814,342.34456324)(48.91832107,342.31456697)
\lineto(48.91832107,342.07456697)
\curveto(48.89831817,342.00456358)(48.89331817,341.92456366)(48.90332107,341.83456697)
\curveto(48.91331815,341.75456383)(48.91831815,341.67456391)(48.91832107,341.59456697)
\lineto(48.91832107,340.63456697)
\lineto(48.91832107,339.35956697)
\curveto(48.91831815,339.22956635)(48.91331815,339.10956647)(48.90332107,338.99956697)
\curveto(48.89331817,338.88956669)(48.8633182,338.79956678)(48.81332107,338.72956697)
\curveto(48.79331827,338.69956688)(48.75831831,338.67456691)(48.70832107,338.65456697)
\curveto(48.6683184,338.64456694)(48.62331844,338.63456695)(48.57332107,338.62456697)
\lineto(48.49832107,338.62456697)
\curveto(48.44831862,338.61456697)(48.39331867,338.60956697)(48.33332107,338.60956697)
\lineto(48.16832107,338.60956697)
\lineto(47.52332107,338.60956697)
\curveto(47.4633196,338.61956696)(47.39831967,338.62456696)(47.32832107,338.62456697)
\lineto(47.13332107,338.62456697)
\curveto(47.08331998,338.64456694)(47.03332003,338.65956692)(46.98332107,338.66956697)
\curveto(46.93332013,338.68956689)(46.89832017,338.72456686)(46.87832107,338.77456697)
\curveto(46.83832023,338.82456676)(46.81332025,338.89456669)(46.80332107,338.98456697)
\lineto(46.80332107,339.28456697)
\lineto(46.80332107,340.30456697)
\lineto(46.80332107,344.53456697)
\lineto(46.80332107,345.64456697)
\lineto(46.80332107,345.92956697)
\curveto(46.80332026,346.02955955)(46.82332024,346.10955947)(46.86332107,346.16956697)
\curveto(46.91332015,346.24955933)(46.98832008,346.29955928)(47.08832107,346.31956697)
\curveto(47.18831988,346.33955924)(47.30831976,346.34955923)(47.44832107,346.34956697)
\lineto(48.21332107,346.34956697)
\curveto(48.33331873,346.34955923)(48.43831863,346.33955924)(48.52832107,346.31956697)
\curveto(48.61831845,346.30955927)(48.68831838,346.26455932)(48.73832107,346.18456697)
\curveto(48.7683183,346.13455945)(48.78331828,346.06455952)(48.78332107,345.97456697)
\lineto(48.81332107,345.70456697)
\curveto(48.82331824,345.62455996)(48.83831823,345.54956003)(48.85832107,345.47956697)
\curveto(48.88831818,345.40956017)(48.93831813,345.37456021)(49.00832107,345.37456697)
\curveto(49.02831804,345.39456019)(49.04831802,345.40456018)(49.06832107,345.40456697)
\curveto(49.08831798,345.40456018)(49.10831796,345.41456017)(49.12832107,345.43456697)
\curveto(49.18831788,345.4845601)(49.23831783,345.53956004)(49.27832107,345.59956697)
\curveto(49.32831774,345.66955991)(49.38831768,345.72955985)(49.45832107,345.77956697)
\curveto(49.49831757,345.80955977)(49.53331753,345.83955974)(49.56332107,345.86956697)
\curveto(49.59331747,345.90955967)(49.62831744,345.94455964)(49.66832107,345.97456697)
\lineto(49.93832107,346.15456697)
\curveto(50.03831703,346.21455937)(50.13831693,346.26955931)(50.23832107,346.31956697)
\curveto(50.33831673,346.35955922)(50.43831663,346.39455919)(50.53832107,346.42456697)
\lineto(50.86832107,346.51456697)
\curveto(50.89831617,346.52455906)(50.95331611,346.52455906)(51.03332107,346.51456697)
\curveto(51.12331594,346.51455907)(51.17831589,346.52455906)(51.19832107,346.54456697)
}
}
{
\newrgbcolor{curcolor}{0 0 0}
\pscustom[linestyle=none,fillstyle=solid,fillcolor=curcolor]
{
\newpath
\moveto(54.7033992,349.19956697)
\curveto(54.77339625,349.11955646)(54.80839621,348.99955658)(54.8083992,348.83956697)
\lineto(54.8083992,348.37456697)
\lineto(54.8083992,347.96956697)
\curveto(54.80839621,347.82955775)(54.77339625,347.73455785)(54.7033992,347.68456697)
\curveto(54.64339638,347.63455795)(54.56339646,347.60455798)(54.4633992,347.59456697)
\curveto(54.37339665,347.584558)(54.27339675,347.579558)(54.1633992,347.57956697)
\lineto(53.3233992,347.57956697)
\curveto(53.21339781,347.579558)(53.11339791,347.584558)(53.0233992,347.59456697)
\curveto(52.94339808,347.60455798)(52.87339815,347.63455795)(52.8133992,347.68456697)
\curveto(52.77339825,347.71455787)(52.74339828,347.76955781)(52.7233992,347.84956697)
\curveto(52.71339831,347.93955764)(52.70339832,348.03455755)(52.6933992,348.13456697)
\lineto(52.6933992,348.46456697)
\curveto(52.70339832,348.57455701)(52.70839831,348.66955691)(52.7083992,348.74956697)
\lineto(52.7083992,348.95956697)
\curveto(52.7183983,349.02955655)(52.73839828,349.08955649)(52.7683992,349.13956697)
\curveto(52.78839823,349.1795564)(52.81339821,349.20955637)(52.8433992,349.22956697)
\lineto(52.9633992,349.28956697)
\curveto(52.98339804,349.28955629)(53.00839801,349.28955629)(53.0383992,349.28956697)
\curveto(53.06839795,349.29955628)(53.09339793,349.30455628)(53.1133992,349.30456697)
\lineto(54.2083992,349.30456697)
\curveto(54.30839671,349.30455628)(54.40339662,349.29955628)(54.4933992,349.28956697)
\curveto(54.58339644,349.2795563)(54.65339637,349.24955633)(54.7033992,349.19956697)
\moveto(54.8083992,339.43456697)
\curveto(54.80839621,339.23456635)(54.80339622,339.06456652)(54.7933992,338.92456697)
\curveto(54.78339624,338.7845668)(54.69339633,338.68956689)(54.5233992,338.63956697)
\curveto(54.46339656,338.61956696)(54.39839662,338.60956697)(54.3283992,338.60956697)
\curveto(54.25839676,338.61956696)(54.18339684,338.62456696)(54.1033992,338.62456697)
\lineto(53.2633992,338.62456697)
\curveto(53.17339785,338.62456696)(53.08339794,338.62956695)(52.9933992,338.63956697)
\curveto(52.91339811,338.64956693)(52.85339817,338.6795669)(52.8133992,338.72956697)
\curveto(52.75339827,338.79956678)(52.7183983,338.8845667)(52.7083992,338.98456697)
\lineto(52.7083992,339.32956697)
\lineto(52.7083992,345.65956697)
\lineto(52.7083992,345.95956697)
\curveto(52.70839831,346.05955952)(52.72839829,346.13955944)(52.7683992,346.19956697)
\curveto(52.82839819,346.26955931)(52.91339811,346.31455927)(53.0233992,346.33456697)
\curveto(53.04339798,346.34455924)(53.06839795,346.34455924)(53.0983992,346.33456697)
\curveto(53.13839788,346.33455925)(53.16839785,346.33955924)(53.1883992,346.34956697)
\lineto(53.9383992,346.34956697)
\lineto(54.1333992,346.34956697)
\curveto(54.21339681,346.35955922)(54.27839674,346.35955922)(54.3283992,346.34956697)
\lineto(54.4483992,346.34956697)
\curveto(54.50839651,346.32955925)(54.56339646,346.31455927)(54.6133992,346.30456697)
\curveto(54.66339636,346.29455929)(54.70339632,346.26455932)(54.7333992,346.21456697)
\curveto(54.77339625,346.16455942)(54.79339623,346.09455949)(54.7933992,346.00456697)
\curveto(54.80339622,345.91455967)(54.80839621,345.81955976)(54.8083992,345.71956697)
\lineto(54.8083992,339.43456697)
}
}
{
\newrgbcolor{curcolor}{0 0 0}
\pscustom[linestyle=none,fillstyle=solid,fillcolor=curcolor]
{
\newpath
\moveto(64.2705867,342.86956697)
\curveto(64.2905781,342.80956277)(64.30057809,342.70456288)(64.3005867,342.55456697)
\curveto(64.30057809,342.41456317)(64.29557809,342.31456327)(64.2855867,342.25456697)
\curveto(64.2855781,342.20456338)(64.28057811,342.15956342)(64.2705867,342.11956697)
\lineto(64.2705867,341.99956697)
\curveto(64.25057814,341.91956366)(64.24057815,341.83956374)(64.2405867,341.75956697)
\curveto(64.24057815,341.68956389)(64.23057816,341.61456397)(64.2105867,341.53456697)
\curveto(64.21057818,341.49456409)(64.20057819,341.42456416)(64.1805867,341.32456697)
\curveto(64.15057824,341.20456438)(64.12057827,341.0795645)(64.0905867,340.94956697)
\curveto(64.07057832,340.82956475)(64.03557835,340.71456487)(63.9855867,340.60456697)
\curveto(63.80557858,340.15456543)(63.58057881,339.76456582)(63.3105867,339.43456697)
\curveto(63.04057935,339.10456648)(62.6855797,338.84456674)(62.2455867,338.65456697)
\curveto(62.15558023,338.61456697)(62.06058033,338.584567)(61.9605867,338.56456697)
\curveto(61.87058052,338.53456705)(61.77058062,338.50456708)(61.6605867,338.47456697)
\curveto(61.60058079,338.45456713)(61.53558085,338.44456714)(61.4655867,338.44456697)
\curveto(61.40558098,338.44456714)(61.34558104,338.43956714)(61.2855867,338.42956697)
\lineto(61.1505867,338.42956697)
\curveto(61.0905813,338.40956717)(61.01058138,338.40456718)(60.9105867,338.41456697)
\curveto(60.81058158,338.41456717)(60.73058166,338.42456716)(60.6705867,338.44456697)
\lineto(60.5805867,338.44456697)
\curveto(60.53058186,338.45456713)(60.47558191,338.46456712)(60.4155867,338.47456697)
\curveto(60.35558203,338.47456711)(60.29558209,338.4795671)(60.2355867,338.48956697)
\curveto(60.04558234,338.53956704)(59.87058252,338.58956699)(59.7105867,338.63956697)
\curveto(59.55058284,338.68956689)(59.40058299,338.75956682)(59.2605867,338.84956697)
\lineto(59.0805867,338.96956697)
\curveto(59.03058336,339.00956657)(58.98058341,339.05456653)(58.9305867,339.10456697)
\lineto(58.8405867,339.16456697)
\curveto(58.81058358,339.1845664)(58.78058361,339.19956638)(58.7505867,339.20956697)
\curveto(58.66058373,339.23956634)(58.60558378,339.21956636)(58.5855867,339.14956697)
\curveto(58.53558385,339.0795665)(58.50058389,338.99456659)(58.4805867,338.89456697)
\curveto(58.47058392,338.80456678)(58.43558395,338.73456685)(58.3755867,338.68456697)
\curveto(58.31558407,338.64456694)(58.24558414,338.61956696)(58.1655867,338.60956697)
\lineto(57.8955867,338.60956697)
\lineto(57.1755867,338.60956697)
\lineto(56.9505867,338.60956697)
\curveto(56.88058551,338.59956698)(56.81558557,338.60456698)(56.7555867,338.62456697)
\curveto(56.61558577,338.67456691)(56.53558585,338.76456682)(56.5155867,338.89456697)
\curveto(56.50558588,339.03456655)(56.50058589,339.18956639)(56.5005867,339.35956697)
\lineto(56.5005867,348.50956697)
\lineto(56.5005867,348.85456697)
\curveto(56.50058589,348.97455661)(56.52558586,349.06955651)(56.5755867,349.13956697)
\curveto(56.61558577,349.20955637)(56.6855857,349.25455633)(56.7855867,349.27456697)
\curveto(56.80558558,349.2845563)(56.82558556,349.2845563)(56.8455867,349.27456697)
\curveto(56.87558551,349.27455631)(56.90058549,349.2795563)(56.9205867,349.28956697)
\lineto(57.8655867,349.28956697)
\curveto(58.04558434,349.28955629)(58.20058419,349.2795563)(58.3305867,349.25956697)
\curveto(58.46058393,349.24955633)(58.54558384,349.17455641)(58.5855867,349.03456697)
\curveto(58.61558377,348.93455665)(58.62558376,348.79955678)(58.6155867,348.62956697)
\curveto(58.60558378,348.46955711)(58.60058379,348.32955725)(58.6005867,348.20956697)
\lineto(58.6005867,346.57456697)
\lineto(58.6005867,346.24456697)
\curveto(58.60058379,346.13455945)(58.61058378,346.03955954)(58.6305867,345.95956697)
\curveto(58.64058375,345.90955967)(58.65058374,345.86455972)(58.6605867,345.82456697)
\curveto(58.67058372,345.79455979)(58.69558369,345.77455981)(58.7355867,345.76456697)
\curveto(58.75558363,345.74455984)(58.78058361,345.73455985)(58.8105867,345.73456697)
\curveto(58.85058354,345.73455985)(58.88058351,345.73955984)(58.9005867,345.74956697)
\curveto(58.97058342,345.78955979)(59.03558335,345.82955975)(59.0955867,345.86956697)
\curveto(59.15558323,345.91955966)(59.22058317,345.96955961)(59.2905867,346.01956697)
\curveto(59.42058297,346.10955947)(59.55558283,346.1845594)(59.6955867,346.24456697)
\curveto(59.83558255,346.31455927)(59.9905824,346.37455921)(60.1605867,346.42456697)
\curveto(60.24058215,346.45455913)(60.32058207,346.46955911)(60.4005867,346.46956697)
\curveto(60.48058191,346.4795591)(60.56058183,346.49455909)(60.6405867,346.51456697)
\curveto(60.71058168,346.53455905)(60.7855816,346.54455904)(60.8655867,346.54456697)
\lineto(61.1055867,346.54456697)
\lineto(61.2555867,346.54456697)
\curveto(61.2855811,346.53455905)(61.32058107,346.52955905)(61.3605867,346.52956697)
\curveto(61.40058099,346.53955904)(61.44058095,346.53955904)(61.4805867,346.52956697)
\curveto(61.5905808,346.49955908)(61.6905807,346.47455911)(61.7805867,346.45456697)
\curveto(61.88058051,346.44455914)(61.97558041,346.41955916)(62.0655867,346.37956697)
\curveto(62.52557986,346.18955939)(62.90057949,345.94455964)(63.1905867,345.64456697)
\curveto(63.48057891,345.34456024)(63.72557866,344.96956061)(63.9255867,344.51956697)
\curveto(63.97557841,344.39956118)(64.01557837,344.27456131)(64.0455867,344.14456697)
\curveto(64.0855783,344.01456157)(64.12557826,343.8795617)(64.1655867,343.73956697)
\curveto(64.1855782,343.66956191)(64.19557819,343.59956198)(64.1955867,343.52956697)
\curveto(64.20557818,343.46956211)(64.22057817,343.39956218)(64.2405867,343.31956697)
\curveto(64.26057813,343.26956231)(64.26557812,343.21456237)(64.2555867,343.15456697)
\curveto(64.25557813,343.09456249)(64.26057813,343.03456255)(64.2705867,342.97456697)
\lineto(64.2705867,342.86956697)
\moveto(62.0505867,341.45956697)
\curveto(62.08058031,341.55956402)(62.10558028,341.6845639)(62.1255867,341.83456697)
\curveto(62.15558023,341.9845636)(62.17058022,342.13456345)(62.1705867,342.28456697)
\curveto(62.18058021,342.44456314)(62.18058021,342.59956298)(62.1705867,342.74956697)
\curveto(62.17058022,342.90956267)(62.15558023,343.04456254)(62.1255867,343.15456697)
\curveto(62.09558029,343.25456233)(62.07558031,343.34956223)(62.0655867,343.43956697)
\curveto(62.05558033,343.52956205)(62.03058036,343.61456197)(61.9905867,343.69456697)
\curveto(61.85058054,344.04456154)(61.65058074,344.33956124)(61.3905867,344.57956697)
\curveto(61.14058125,344.82956075)(60.77058162,344.95456063)(60.2805867,344.95456697)
\curveto(60.24058215,344.95456063)(60.20558218,344.94956063)(60.1755867,344.93956697)
\lineto(60.0705867,344.93956697)
\curveto(60.00058239,344.91956066)(59.93558245,344.89956068)(59.8755867,344.87956697)
\curveto(59.81558257,344.86956071)(59.75558263,344.85456073)(59.6955867,344.83456697)
\curveto(59.40558298,344.70456088)(59.1855832,344.51956106)(59.0355867,344.27956697)
\curveto(58.8855835,344.04956153)(58.76058363,343.7845618)(58.6605867,343.48456697)
\curveto(58.63058376,343.40456218)(58.61058378,343.31956226)(58.6005867,343.22956697)
\curveto(58.60058379,343.14956243)(58.5905838,343.06956251)(58.5705867,342.98956697)
\curveto(58.56058383,342.95956262)(58.55558383,342.90956267)(58.5555867,342.83956697)
\curveto(58.54558384,342.79956278)(58.54058385,342.75956282)(58.5405867,342.71956697)
\curveto(58.55058384,342.6795629)(58.55058384,342.63956294)(58.5405867,342.59956697)
\curveto(58.52058387,342.51956306)(58.51558387,342.40956317)(58.5255867,342.26956697)
\curveto(58.53558385,342.12956345)(58.55058384,342.02956355)(58.5705867,341.96956697)
\curveto(58.5905838,341.8795637)(58.60058379,341.79456379)(58.6005867,341.71456697)
\curveto(58.61058378,341.63456395)(58.63058376,341.55456403)(58.6605867,341.47456697)
\curveto(58.75058364,341.19456439)(58.85558353,340.94956463)(58.9755867,340.73956697)
\curveto(59.10558328,340.53956504)(59.2855831,340.36956521)(59.5155867,340.22956697)
\curveto(59.67558271,340.12956545)(59.84058255,340.05956552)(60.0105867,340.01956697)
\curveto(60.03058236,340.01956556)(60.05058234,340.01456557)(60.0705867,340.00456697)
\lineto(60.1605867,340.00456697)
\curveto(60.1905822,339.99456559)(60.24058215,339.9845656)(60.3105867,339.97456697)
\curveto(60.38058201,339.97456561)(60.44058195,339.9795656)(60.4905867,339.98956697)
\curveto(60.5905818,340.00956557)(60.68058171,340.02456556)(60.7605867,340.03456697)
\curveto(60.85058154,340.05456553)(60.93558145,340.0795655)(61.0155867,340.10956697)
\curveto(61.29558109,340.23956534)(61.51058088,340.41956516)(61.6605867,340.64956697)
\curveto(61.82058057,340.8795647)(61.95058044,341.14956443)(62.0505867,341.45956697)
}
}
{
\newrgbcolor{curcolor}{0 0 0}
\pscustom[linestyle=none,fillstyle=solid,fillcolor=curcolor]
{
\newpath
\moveto(66.06050857,346.33456697)
\lineto(67.18550857,346.33456697)
\curveto(67.29550614,346.33455925)(67.39550604,346.32955925)(67.48550857,346.31956697)
\curveto(67.57550586,346.30955927)(67.64050579,346.27455931)(67.68050857,346.21456697)
\curveto(67.7305057,346.15455943)(67.76050567,346.06955951)(67.77050857,345.95956697)
\curveto(67.78050565,345.85955972)(67.78550565,345.75455983)(67.78550857,345.64456697)
\lineto(67.78550857,344.59456697)
\lineto(67.78550857,342.35956697)
\curveto(67.78550565,341.99956358)(67.80050563,341.65956392)(67.83050857,341.33956697)
\curveto(67.86050557,341.01956456)(67.95050548,340.75456483)(68.10050857,340.54456697)
\curveto(68.24050519,340.33456525)(68.46550497,340.1845654)(68.77550857,340.09456697)
\curveto(68.82550461,340.0845655)(68.86550457,340.0795655)(68.89550857,340.07956697)
\curveto(68.9355045,340.0795655)(68.98050445,340.07456551)(69.03050857,340.06456697)
\curveto(69.08050435,340.05456553)(69.1355043,340.04956553)(69.19550857,340.04956697)
\curveto(69.25550418,340.04956553)(69.30050413,340.05456553)(69.33050857,340.06456697)
\curveto(69.38050405,340.0845655)(69.42050401,340.08956549)(69.45050857,340.07956697)
\curveto(69.49050394,340.06956551)(69.5305039,340.07456551)(69.57050857,340.09456697)
\curveto(69.78050365,340.14456544)(69.94550349,340.20956537)(70.06550857,340.28956697)
\curveto(70.24550319,340.39956518)(70.38550305,340.53956504)(70.48550857,340.70956697)
\curveto(70.59550284,340.88956469)(70.67050276,341.0845645)(70.71050857,341.29456697)
\curveto(70.76050267,341.51456407)(70.79050264,341.75456383)(70.80050857,342.01456697)
\curveto(70.81050262,342.2845633)(70.81550262,342.56456302)(70.81550857,342.85456697)
\lineto(70.81550857,344.66956697)
\lineto(70.81550857,345.64456697)
\lineto(70.81550857,345.91456697)
\curveto(70.81550262,346.01455957)(70.8355026,346.09455949)(70.87550857,346.15456697)
\curveto(70.92550251,346.24455934)(71.00050243,346.29455929)(71.10050857,346.30456697)
\curveto(71.20050223,346.32455926)(71.32050211,346.33455925)(71.46050857,346.33456697)
\lineto(72.25550857,346.33456697)
\lineto(72.54050857,346.33456697)
\curveto(72.6305008,346.33455925)(72.70550073,346.31455927)(72.76550857,346.27456697)
\curveto(72.84550059,346.22455936)(72.89050054,346.14955943)(72.90050857,346.04956697)
\curveto(72.91050052,345.94955963)(72.91550052,345.83455975)(72.91550857,345.70456697)
\lineto(72.91550857,344.56456697)
\lineto(72.91550857,340.34956697)
\lineto(72.91550857,339.28456697)
\lineto(72.91550857,338.98456697)
\curveto(72.91550052,338.8845667)(72.89550054,338.80956677)(72.85550857,338.75956697)
\curveto(72.80550063,338.6795669)(72.7305007,338.63456695)(72.63050857,338.62456697)
\curveto(72.5305009,338.61456697)(72.42550101,338.60956697)(72.31550857,338.60956697)
\lineto(71.50550857,338.60956697)
\curveto(71.39550204,338.60956697)(71.29550214,338.61456697)(71.20550857,338.62456697)
\curveto(71.12550231,338.63456695)(71.06050237,338.67456691)(71.01050857,338.74456697)
\curveto(70.99050244,338.77456681)(70.97050246,338.81956676)(70.95050857,338.87956697)
\curveto(70.94050249,338.93956664)(70.92550251,338.99956658)(70.90550857,339.05956697)
\curveto(70.89550254,339.11956646)(70.88050255,339.17456641)(70.86050857,339.22456697)
\curveto(70.84050259,339.27456631)(70.81050262,339.30456628)(70.77050857,339.31456697)
\curveto(70.75050268,339.33456625)(70.72550271,339.33956624)(70.69550857,339.32956697)
\curveto(70.66550277,339.31956626)(70.64050279,339.30956627)(70.62050857,339.29956697)
\curveto(70.55050288,339.25956632)(70.49050294,339.21456637)(70.44050857,339.16456697)
\curveto(70.39050304,339.11456647)(70.3355031,339.06956651)(70.27550857,339.02956697)
\curveto(70.2355032,338.99956658)(70.19550324,338.96456662)(70.15550857,338.92456697)
\curveto(70.12550331,338.89456669)(70.08550335,338.86456672)(70.03550857,338.83456697)
\curveto(69.80550363,338.69456689)(69.5355039,338.584567)(69.22550857,338.50456697)
\curveto(69.15550428,338.4845671)(69.08550435,338.47456711)(69.01550857,338.47456697)
\curveto(68.94550449,338.46456712)(68.87050456,338.44956713)(68.79050857,338.42956697)
\curveto(68.75050468,338.41956716)(68.70550473,338.41956716)(68.65550857,338.42956697)
\curveto(68.61550482,338.42956715)(68.57550486,338.42456716)(68.53550857,338.41456697)
\curveto(68.50550493,338.40456718)(68.44050499,338.40456718)(68.34050857,338.41456697)
\curveto(68.25050518,338.41456717)(68.19050524,338.41956716)(68.16050857,338.42956697)
\curveto(68.11050532,338.42956715)(68.06050537,338.43456715)(68.01050857,338.44456697)
\lineto(67.86050857,338.44456697)
\curveto(67.74050569,338.47456711)(67.62550581,338.49956708)(67.51550857,338.51956697)
\curveto(67.40550603,338.53956704)(67.29550614,338.56956701)(67.18550857,338.60956697)
\curveto(67.1355063,338.62956695)(67.09050634,338.64456694)(67.05050857,338.65456697)
\curveto(67.02050641,338.67456691)(66.98050645,338.69456689)(66.93050857,338.71456697)
\curveto(66.58050685,338.90456668)(66.30050713,339.16956641)(66.09050857,339.50956697)
\curveto(65.96050747,339.71956586)(65.86550757,339.96956561)(65.80550857,340.25956697)
\curveto(65.74550769,340.55956502)(65.70550773,340.87456471)(65.68550857,341.20456697)
\curveto(65.67550776,341.54456404)(65.67050776,341.88956369)(65.67050857,342.23956697)
\curveto(65.68050775,342.59956298)(65.68550775,342.95456263)(65.68550857,343.30456697)
\lineto(65.68550857,345.34456697)
\curveto(65.68550775,345.47456011)(65.68050775,345.62455996)(65.67050857,345.79456697)
\curveto(65.67050776,345.97455961)(65.69550774,346.10455948)(65.74550857,346.18456697)
\curveto(65.77550766,346.23455935)(65.8355076,346.2795593)(65.92550857,346.31956697)
\curveto(65.98550745,346.31955926)(66.0305074,346.32455926)(66.06050857,346.33456697)
}
}
{
\newrgbcolor{curcolor}{0 0 0}
\pscustom[linestyle=none,fillstyle=solid,fillcolor=curcolor]
{
\newpath
\moveto(78.11675857,346.55956697)
\curveto(78.92675341,346.579559)(79.60175274,346.45955912)(80.14175857,346.19956697)
\curveto(80.69175165,345.93955964)(81.12675121,345.56956001)(81.44675857,345.08956697)
\curveto(81.60675073,344.84956073)(81.72675061,344.57456101)(81.80675857,344.26456697)
\curveto(81.82675051,344.21456137)(81.8417505,344.14956143)(81.85175857,344.06956697)
\curveto(81.87175047,343.98956159)(81.87175047,343.91956166)(81.85175857,343.85956697)
\curveto(81.81175053,343.74956183)(81.7417506,343.6845619)(81.64175857,343.66456697)
\curveto(81.5417508,343.65456193)(81.42175092,343.64956193)(81.28175857,343.64956697)
\lineto(80.50175857,343.64956697)
\lineto(80.21675857,343.64956697)
\curveto(80.12675221,343.64956193)(80.05175229,343.66956191)(79.99175857,343.70956697)
\curveto(79.91175243,343.74956183)(79.85675248,343.80956177)(79.82675857,343.88956697)
\curveto(79.79675254,343.9795616)(79.75675258,344.06956151)(79.70675857,344.15956697)
\curveto(79.64675269,344.26956131)(79.58175276,344.36956121)(79.51175857,344.45956697)
\curveto(79.4417529,344.54956103)(79.36175298,344.62956095)(79.27175857,344.69956697)
\curveto(79.13175321,344.78956079)(78.97675336,344.85956072)(78.80675857,344.90956697)
\curveto(78.74675359,344.92956065)(78.68675365,344.93956064)(78.62675857,344.93956697)
\curveto(78.56675377,344.93956064)(78.51175383,344.94956063)(78.46175857,344.96956697)
\lineto(78.31175857,344.96956697)
\curveto(78.11175423,344.96956061)(77.95175439,344.94956063)(77.83175857,344.90956697)
\curveto(77.5417548,344.81956076)(77.30675503,344.6795609)(77.12675857,344.48956697)
\curveto(76.94675539,344.30956127)(76.80175554,344.08956149)(76.69175857,343.82956697)
\curveto(76.6417557,343.71956186)(76.60175574,343.59956198)(76.57175857,343.46956697)
\curveto(76.55175579,343.34956223)(76.52675581,343.21956236)(76.49675857,343.07956697)
\curveto(76.48675585,343.03956254)(76.48175586,342.99956258)(76.48175857,342.95956697)
\curveto(76.48175586,342.91956266)(76.47675586,342.8795627)(76.46675857,342.83956697)
\curveto(76.44675589,342.73956284)(76.4367559,342.59956298)(76.43675857,342.41956697)
\curveto(76.44675589,342.23956334)(76.46175588,342.09956348)(76.48175857,341.99956697)
\curveto(76.48175586,341.91956366)(76.48675585,341.86456372)(76.49675857,341.83456697)
\curveto(76.51675582,341.76456382)(76.52675581,341.69456389)(76.52675857,341.62456697)
\curveto(76.5367558,341.55456403)(76.55175579,341.4845641)(76.57175857,341.41456697)
\curveto(76.65175569,341.1845644)(76.74675559,340.97456461)(76.85675857,340.78456697)
\curveto(76.96675537,340.59456499)(77.10675523,340.43456515)(77.27675857,340.30456697)
\curveto(77.31675502,340.27456531)(77.37675496,340.23956534)(77.45675857,340.19956697)
\curveto(77.56675477,340.12956545)(77.67675466,340.0845655)(77.78675857,340.06456697)
\curveto(77.90675443,340.04456554)(78.05175429,340.02456556)(78.22175857,340.00456697)
\lineto(78.31175857,340.00456697)
\curveto(78.35175399,340.00456558)(78.38175396,340.00956557)(78.40175857,340.01956697)
\lineto(78.53675857,340.01956697)
\curveto(78.60675373,340.03956554)(78.67175367,340.05456553)(78.73175857,340.06456697)
\curveto(78.80175354,340.0845655)(78.86675347,340.10456548)(78.92675857,340.12456697)
\curveto(79.22675311,340.25456533)(79.45675288,340.44456514)(79.61675857,340.69456697)
\curveto(79.65675268,340.74456484)(79.69175265,340.79956478)(79.72175857,340.85956697)
\curveto(79.75175259,340.92956465)(79.77675256,340.98956459)(79.79675857,341.03956697)
\curveto(79.8367525,341.14956443)(79.87175247,341.24456434)(79.90175857,341.32456697)
\curveto(79.93175241,341.41456417)(80.00175234,341.4845641)(80.11175857,341.53456697)
\curveto(80.20175214,341.57456401)(80.34675199,341.58956399)(80.54675857,341.57956697)
\lineto(81.04175857,341.57956697)
\lineto(81.25175857,341.57956697)
\curveto(81.33175101,341.58956399)(81.39675094,341.584564)(81.44675857,341.56456697)
\lineto(81.56675857,341.56456697)
\lineto(81.68675857,341.53456697)
\curveto(81.72675061,341.53456405)(81.75675058,341.52456406)(81.77675857,341.50456697)
\curveto(81.82675051,341.46456412)(81.85675048,341.40456418)(81.86675857,341.32456697)
\curveto(81.88675045,341.25456433)(81.88675045,341.1795644)(81.86675857,341.09956697)
\curveto(81.77675056,340.76956481)(81.66675067,340.47456511)(81.53675857,340.21456697)
\curveto(81.12675121,339.44456614)(80.47175187,338.90956667)(79.57175857,338.60956697)
\curveto(79.47175287,338.579567)(79.36675297,338.55956702)(79.25675857,338.54956697)
\curveto(79.14675319,338.52956705)(79.0367533,338.50456708)(78.92675857,338.47456697)
\curveto(78.86675347,338.46456712)(78.80675353,338.45956712)(78.74675857,338.45956697)
\curveto(78.68675365,338.45956712)(78.62675371,338.45456713)(78.56675857,338.44456697)
\lineto(78.40175857,338.44456697)
\curveto(78.35175399,338.42456716)(78.27675406,338.41956716)(78.17675857,338.42956697)
\curveto(78.07675426,338.42956715)(78.00175434,338.43456715)(77.95175857,338.44456697)
\curveto(77.87175447,338.46456712)(77.79675454,338.47456711)(77.72675857,338.47456697)
\curveto(77.66675467,338.46456712)(77.60175474,338.46956711)(77.53175857,338.48956697)
\lineto(77.38175857,338.51956697)
\curveto(77.33175501,338.51956706)(77.28175506,338.52456706)(77.23175857,338.53456697)
\curveto(77.12175522,338.56456702)(77.01675532,338.59456699)(76.91675857,338.62456697)
\curveto(76.81675552,338.65456693)(76.72175562,338.68956689)(76.63175857,338.72956697)
\curveto(76.16175618,338.92956665)(75.76675657,339.1845664)(75.44675857,339.49456697)
\curveto(75.12675721,339.81456577)(74.86675747,340.20956537)(74.66675857,340.67956697)
\curveto(74.61675772,340.76956481)(74.57675776,340.86456472)(74.54675857,340.96456697)
\lineto(74.45675857,341.29456697)
\curveto(74.44675789,341.33456425)(74.4417579,341.36956421)(74.44175857,341.39956697)
\curveto(74.4417579,341.43956414)(74.43175791,341.4845641)(74.41175857,341.53456697)
\curveto(74.39175795,341.60456398)(74.38175796,341.67456391)(74.38175857,341.74456697)
\curveto(74.38175796,341.82456376)(74.37175797,341.89956368)(74.35175857,341.96956697)
\lineto(74.35175857,342.22456697)
\curveto(74.33175801,342.27456331)(74.32175802,342.32956325)(74.32175857,342.38956697)
\curveto(74.32175802,342.45956312)(74.33175801,342.51956306)(74.35175857,342.56956697)
\curveto(74.36175798,342.61956296)(74.36175798,342.66456292)(74.35175857,342.70456697)
\curveto(74.341758,342.74456284)(74.341758,342.7845628)(74.35175857,342.82456697)
\curveto(74.37175797,342.89456269)(74.37675796,342.95956262)(74.36675857,343.01956697)
\curveto(74.36675797,343.0795625)(74.37675796,343.13956244)(74.39675857,343.19956697)
\curveto(74.44675789,343.3795622)(74.48675785,343.54956203)(74.51675857,343.70956697)
\curveto(74.54675779,343.8795617)(74.59175775,344.04456154)(74.65175857,344.20456697)
\curveto(74.87175747,344.71456087)(75.14675719,345.13956044)(75.47675857,345.47956697)
\curveto(75.81675652,345.81955976)(76.24675609,346.09455949)(76.76675857,346.30456697)
\curveto(76.90675543,346.36455922)(77.05175529,346.40455918)(77.20175857,346.42456697)
\curveto(77.35175499,346.45455913)(77.50675483,346.48955909)(77.66675857,346.52956697)
\curveto(77.74675459,346.53955904)(77.82175452,346.54455904)(77.89175857,346.54456697)
\curveto(77.96175438,346.54455904)(78.0367543,346.54955903)(78.11675857,346.55956697)
}
}
{
\newrgbcolor{curcolor}{0 0 0}
\pscustom[linestyle=none,fillstyle=solid,fillcolor=curcolor]
{
\newpath
\moveto(85.26003982,349.19956697)
\curveto(85.33003687,349.11955646)(85.36503684,348.99955658)(85.36503982,348.83956697)
\lineto(85.36503982,348.37456697)
\lineto(85.36503982,347.96956697)
\curveto(85.36503684,347.82955775)(85.33003687,347.73455785)(85.26003982,347.68456697)
\curveto(85.200037,347.63455795)(85.12003708,347.60455798)(85.02003982,347.59456697)
\curveto(84.93003727,347.584558)(84.83003737,347.579558)(84.72003982,347.57956697)
\lineto(83.88003982,347.57956697)
\curveto(83.77003843,347.579558)(83.67003853,347.584558)(83.58003982,347.59456697)
\curveto(83.5000387,347.60455798)(83.43003877,347.63455795)(83.37003982,347.68456697)
\curveto(83.33003887,347.71455787)(83.3000389,347.76955781)(83.28003982,347.84956697)
\curveto(83.27003893,347.93955764)(83.26003894,348.03455755)(83.25003982,348.13456697)
\lineto(83.25003982,348.46456697)
\curveto(83.26003894,348.57455701)(83.26503894,348.66955691)(83.26503982,348.74956697)
\lineto(83.26503982,348.95956697)
\curveto(83.27503893,349.02955655)(83.29503891,349.08955649)(83.32503982,349.13956697)
\curveto(83.34503886,349.1795564)(83.37003883,349.20955637)(83.40003982,349.22956697)
\lineto(83.52003982,349.28956697)
\curveto(83.54003866,349.28955629)(83.56503864,349.28955629)(83.59503982,349.28956697)
\curveto(83.62503858,349.29955628)(83.65003855,349.30455628)(83.67003982,349.30456697)
\lineto(84.76503982,349.30456697)
\curveto(84.86503734,349.30455628)(84.96003724,349.29955628)(85.05003982,349.28956697)
\curveto(85.14003706,349.2795563)(85.21003699,349.24955633)(85.26003982,349.19956697)
\moveto(85.36503982,339.43456697)
\curveto(85.36503684,339.23456635)(85.36003684,339.06456652)(85.35003982,338.92456697)
\curveto(85.34003686,338.7845668)(85.25003695,338.68956689)(85.08003982,338.63956697)
\curveto(85.02003718,338.61956696)(84.95503725,338.60956697)(84.88503982,338.60956697)
\curveto(84.81503739,338.61956696)(84.74003746,338.62456696)(84.66003982,338.62456697)
\lineto(83.82003982,338.62456697)
\curveto(83.73003847,338.62456696)(83.64003856,338.62956695)(83.55003982,338.63956697)
\curveto(83.47003873,338.64956693)(83.41003879,338.6795669)(83.37003982,338.72956697)
\curveto(83.31003889,338.79956678)(83.27503893,338.8845667)(83.26503982,338.98456697)
\lineto(83.26503982,339.32956697)
\lineto(83.26503982,345.65956697)
\lineto(83.26503982,345.95956697)
\curveto(83.26503894,346.05955952)(83.28503892,346.13955944)(83.32503982,346.19956697)
\curveto(83.38503882,346.26955931)(83.47003873,346.31455927)(83.58003982,346.33456697)
\curveto(83.6000386,346.34455924)(83.62503858,346.34455924)(83.65503982,346.33456697)
\curveto(83.69503851,346.33455925)(83.72503848,346.33955924)(83.74503982,346.34956697)
\lineto(84.49503982,346.34956697)
\lineto(84.69003982,346.34956697)
\curveto(84.77003743,346.35955922)(84.83503737,346.35955922)(84.88503982,346.34956697)
\lineto(85.00503982,346.34956697)
\curveto(85.06503714,346.32955925)(85.12003708,346.31455927)(85.17003982,346.30456697)
\curveto(85.22003698,346.29455929)(85.26003694,346.26455932)(85.29003982,346.21456697)
\curveto(85.33003687,346.16455942)(85.35003685,346.09455949)(85.35003982,346.00456697)
\curveto(85.36003684,345.91455967)(85.36503684,345.81955976)(85.36503982,345.71956697)
\lineto(85.36503982,339.43456697)
}
}
{
\newrgbcolor{curcolor}{0 0 0}
\pscustom[linestyle=none,fillstyle=solid,fillcolor=curcolor]
{
\newpath
\moveto(94.79722732,342.79456697)
\curveto(94.77721879,342.84456274)(94.7722188,342.89956268)(94.78222732,342.95956697)
\curveto(94.79221878,343.01956256)(94.78721878,343.07456251)(94.76722732,343.12456697)
\curveto(94.75721881,343.16456242)(94.75221882,343.20456238)(94.75222732,343.24456697)
\curveto(94.75221882,343.2845623)(94.74721882,343.32456226)(94.73722732,343.36456697)
\lineto(94.67722732,343.63456697)
\curveto(94.65721891,343.72456186)(94.63221894,343.80956177)(94.60222732,343.88956697)
\curveto(94.55221902,344.02956155)(94.50721906,344.15956142)(94.46722732,344.27956697)
\curveto(94.42721914,344.40956117)(94.3722192,344.52956105)(94.30222732,344.63956697)
\curveto(94.23221934,344.74956083)(94.16221941,344.85456073)(94.09222732,344.95456697)
\curveto(94.03221954,345.05456053)(93.96221961,345.15456043)(93.88222732,345.25456697)
\curveto(93.80221977,345.36456022)(93.70221987,345.46456012)(93.58222732,345.55456697)
\curveto(93.4722201,345.65455993)(93.36222021,345.74455984)(93.25222732,345.82456697)
\curveto(92.92222065,346.05455953)(92.54222103,346.23455935)(92.11222732,346.36456697)
\curveto(91.69222188,346.49455909)(91.19222238,346.55455903)(90.61222732,346.54456697)
\curveto(90.54222303,346.53455905)(90.4722231,346.52955905)(90.40222732,346.52956697)
\curveto(90.33222324,346.52955905)(90.25722331,346.52455906)(90.17722732,346.51456697)
\curveto(90.02722354,346.47455911)(89.88222369,346.44455914)(89.74222732,346.42456697)
\curveto(89.60222397,346.40455918)(89.4672241,346.36955921)(89.33722732,346.31956697)
\curveto(89.22722434,346.26955931)(89.11722445,346.22455936)(89.00722732,346.18456697)
\curveto(88.89722467,346.14455944)(88.79222478,346.09955948)(88.69222732,346.04956697)
\curveto(88.33222524,345.81955976)(88.02722554,345.56456002)(87.77722732,345.28456697)
\curveto(87.52722604,345.01456057)(87.31222626,344.67456091)(87.13222732,344.26456697)
\curveto(87.08222649,344.14456144)(87.04222653,344.01956156)(87.01222732,343.88956697)
\curveto(86.98222659,343.76956181)(86.94722662,343.64456194)(86.90722732,343.51456697)
\curveto(86.88722668,343.46456212)(86.87722669,343.41456217)(86.87722732,343.36456697)
\curveto(86.87722669,343.32456226)(86.8722267,343.2795623)(86.86222732,343.22956697)
\curveto(86.84222673,343.1795624)(86.83222674,343.12456246)(86.83222732,343.06456697)
\curveto(86.84222673,343.01456257)(86.84222673,342.96456262)(86.83222732,342.91456697)
\lineto(86.83222732,342.80956697)
\curveto(86.81222676,342.74956283)(86.79722677,342.66456292)(86.78722732,342.55456697)
\curveto(86.78722678,342.44456314)(86.79722677,342.35956322)(86.81722732,342.29956697)
\lineto(86.81722732,342.16456697)
\curveto(86.81722675,342.12456346)(86.82222675,342.0795635)(86.83222732,342.02956697)
\curveto(86.85222672,341.94956363)(86.86222671,341.86456372)(86.86222732,341.77456697)
\curveto(86.86222671,341.69456389)(86.8722267,341.61456397)(86.89222732,341.53456697)
\curveto(86.91222666,341.4845641)(86.92222665,341.43956414)(86.92222732,341.39956697)
\curveto(86.92222665,341.35956422)(86.93222664,341.31456427)(86.95222732,341.26456697)
\curveto(86.98222659,341.15456443)(87.00722656,341.04956453)(87.02722732,340.94956697)
\curveto(87.05722651,340.84956473)(87.09722647,340.75456483)(87.14722732,340.66456697)
\curveto(87.31722625,340.27456531)(87.52722604,339.93956564)(87.77722732,339.65956697)
\curveto(88.02722554,339.3795662)(88.32722524,339.13456645)(88.67722732,338.92456697)
\curveto(88.78722478,338.86456672)(88.89222468,338.81456677)(88.99222732,338.77456697)
\curveto(89.10222447,338.73456685)(89.21722435,338.69456689)(89.33722732,338.65456697)
\curveto(89.42722414,338.61456697)(89.52222405,338.584567)(89.62222732,338.56456697)
\curveto(89.72222385,338.54456704)(89.82222375,338.51956706)(89.92222732,338.48956697)
\curveto(89.9722236,338.4795671)(90.01222356,338.47456711)(90.04222732,338.47456697)
\curveto(90.08222349,338.47456711)(90.12222345,338.46956711)(90.16222732,338.45956697)
\curveto(90.21222336,338.43956714)(90.26222331,338.43456715)(90.31222732,338.44456697)
\curveto(90.3722232,338.44456714)(90.42722314,338.43956714)(90.47722732,338.42956697)
\lineto(90.62722732,338.42956697)
\curveto(90.68722288,338.40956717)(90.7722228,338.40456718)(90.88222732,338.41456697)
\curveto(90.99222258,338.41456717)(91.0722225,338.41956716)(91.12222732,338.42956697)
\curveto(91.15222242,338.42956715)(91.18222239,338.43456715)(91.21222732,338.44456697)
\lineto(91.31722732,338.44456697)
\curveto(91.3672222,338.45456713)(91.42222215,338.45956712)(91.48222732,338.45956697)
\curveto(91.54222203,338.45956712)(91.59722197,338.46956711)(91.64722732,338.48956697)
\curveto(91.77722179,338.51956706)(91.90222167,338.54956703)(92.02222732,338.57956697)
\curveto(92.15222142,338.59956698)(92.27722129,338.63456695)(92.39722732,338.68456697)
\curveto(92.87722069,338.8845667)(93.28722028,339.13456645)(93.62722732,339.43456697)
\curveto(93.9672196,339.73456585)(94.24221933,340.12456546)(94.45222732,340.60456697)
\curveto(94.50221907,340.70456488)(94.54221903,340.80956477)(94.57222732,340.91956697)
\curveto(94.60221897,341.03956454)(94.63721893,341.15456443)(94.67722732,341.26456697)
\curveto(94.68721888,341.33456425)(94.69721887,341.39956418)(94.70722732,341.45956697)
\curveto(94.71721885,341.51956406)(94.73221884,341.584564)(94.75222732,341.65456697)
\curveto(94.7722188,341.73456385)(94.77721879,341.81456377)(94.76722732,341.89456697)
\curveto(94.7672188,341.97456361)(94.77721879,342.05456353)(94.79722732,342.13456697)
\lineto(94.79722732,342.28456697)
\curveto(94.81721875,342.34456324)(94.82721874,342.42956315)(94.82722732,342.53956697)
\curveto(94.82721874,342.64956293)(94.81721875,342.73456285)(94.79722732,342.79456697)
\moveto(92.69722732,342.25456697)
\curveto(92.68722088,342.20456338)(92.68222089,342.15456343)(92.68222732,342.10456697)
\lineto(92.68222732,341.96956697)
\curveto(92.6722209,341.92956365)(92.6672209,341.88956369)(92.66722732,341.84956697)
\curveto(92.6672209,341.81956376)(92.66222091,341.7845638)(92.65222732,341.74456697)
\curveto(92.62222095,341.63456395)(92.59722097,341.52956405)(92.57722732,341.42956697)
\curveto(92.55722101,341.32956425)(92.52722104,341.22956435)(92.48722732,341.12956697)
\curveto(92.37722119,340.8795647)(92.24222133,340.66956491)(92.08222732,340.49956697)
\curveto(91.92222165,340.32956525)(91.71222186,340.19456539)(91.45222732,340.09456697)
\curveto(91.38222219,340.06456552)(91.30722226,340.04456554)(91.22722732,340.03456697)
\curveto(91.14722242,340.02456556)(91.0672225,340.00956557)(90.98722732,339.98956697)
\lineto(90.86722732,339.98956697)
\curveto(90.82722274,339.9795656)(90.78222279,339.97456561)(90.73222732,339.97456697)
\lineto(90.61222732,340.00456697)
\curveto(90.572223,340.01456557)(90.53722303,340.01456557)(90.50722732,340.00456697)
\curveto(90.47722309,340.00456558)(90.44222313,340.00956557)(90.40222732,340.01956697)
\curveto(90.31222326,340.03956554)(90.22222335,340.06456552)(90.13222732,340.09456697)
\curveto(90.05222352,340.12456546)(89.97722359,340.16456542)(89.90722732,340.21456697)
\curveto(89.65722391,340.36456522)(89.4722241,340.52956505)(89.35222732,340.70956697)
\curveto(89.24222433,340.89956468)(89.13722443,341.14456444)(89.03722732,341.44456697)
\curveto(89.01722455,341.52456406)(89.00222457,341.59956398)(88.99222732,341.66956697)
\curveto(88.98222459,341.74956383)(88.9672246,341.82956375)(88.94722732,341.90956697)
\lineto(88.94722732,342.04456697)
\curveto(88.92722464,342.11456347)(88.91222466,342.21956336)(88.90222732,342.35956697)
\curveto(88.90222467,342.49956308)(88.91222466,342.60456298)(88.93222732,342.67456697)
\lineto(88.93222732,342.82456697)
\curveto(88.93222464,342.87456271)(88.93722463,342.92456266)(88.94722732,342.97456697)
\curveto(88.9672246,343.0845625)(88.98222459,343.19456239)(88.99222732,343.30456697)
\curveto(89.01222456,343.41456217)(89.03722453,343.51956206)(89.06722732,343.61956697)
\curveto(89.15722441,343.88956169)(89.27722429,344.12456146)(89.42722732,344.32456697)
\curveto(89.58722398,344.53456105)(89.79222378,344.69456089)(90.04222732,344.80456697)
\curveto(90.09222348,344.83456075)(90.14722342,344.85456073)(90.20722732,344.86456697)
\lineto(90.41722732,344.92456697)
\curveto(90.44722312,344.93456065)(90.48222309,344.93456065)(90.52222732,344.92456697)
\curveto(90.56222301,344.92456066)(90.59722297,344.93456065)(90.62722732,344.95456697)
\lineto(90.89722732,344.95456697)
\curveto(90.98722258,344.96456062)(91.0722225,344.95956062)(91.15222732,344.93956697)
\curveto(91.22222235,344.91956066)(91.28722228,344.89956068)(91.34722732,344.87956697)
\curveto(91.40722216,344.86956071)(91.4672221,344.85456073)(91.52722732,344.83456697)
\curveto(91.77722179,344.72456086)(91.97722159,344.57456101)(92.12722732,344.38456697)
\curveto(92.27722129,344.20456138)(92.40722116,343.9845616)(92.51722732,343.72456697)
\curveto(92.54722102,343.64456194)(92.567221,343.55956202)(92.57722732,343.46956697)
\lineto(92.63722732,343.22956697)
\curveto(92.64722092,343.20956237)(92.65222092,343.1795624)(92.65222732,343.13956697)
\curveto(92.66222091,343.08956249)(92.6672209,343.03456255)(92.66722732,342.97456697)
\curveto(92.6672209,342.91456267)(92.67722089,342.85956272)(92.69722732,342.80956697)
\lineto(92.69722732,342.68956697)
\curveto(92.70722086,342.63956294)(92.71222086,342.56456302)(92.71222732,342.46456697)
\curveto(92.71222086,342.37456321)(92.70722086,342.30456328)(92.69722732,342.25456697)
\moveto(91.46722732,349.42456697)
\lineto(92.53222732,349.42456697)
\curveto(92.61222096,349.42455616)(92.70722086,349.42455616)(92.81722732,349.42456697)
\curveto(92.92722064,349.42455616)(93.00722056,349.40955617)(93.05722732,349.37956697)
\curveto(93.07722049,349.36955621)(93.08722048,349.35455623)(93.08722732,349.33456697)
\curveto(93.09722047,349.32455626)(93.11222046,349.31455627)(93.13222732,349.30456697)
\curveto(93.14222043,349.1845564)(93.09222048,349.0795565)(92.98222732,348.98956697)
\curveto(92.88222069,348.89955668)(92.79722077,348.81955676)(92.72722732,348.74956697)
\curveto(92.64722092,348.6795569)(92.567221,348.60455698)(92.48722732,348.52456697)
\curveto(92.41722115,348.45455713)(92.34222123,348.38955719)(92.26222732,348.32956697)
\curveto(92.22222135,348.29955728)(92.18722138,348.26455732)(92.15722732,348.22456697)
\curveto(92.13722143,348.19455739)(92.10722146,348.16955741)(92.06722732,348.14956697)
\curveto(92.04722152,348.11955746)(92.02222155,348.09455749)(91.99222732,348.07456697)
\lineto(91.84222732,347.92456697)
\lineto(91.69222732,347.80456697)
\lineto(91.64722732,347.75956697)
\curveto(91.64722192,347.74955783)(91.63722193,347.73455785)(91.61722732,347.71456697)
\curveto(91.53722203,347.65455793)(91.45722211,347.58955799)(91.37722732,347.51956697)
\curveto(91.30722226,347.44955813)(91.21722235,347.39455819)(91.10722732,347.35456697)
\curveto(91.0672225,347.34455824)(91.02722254,347.33955824)(90.98722732,347.33956697)
\curveto(90.95722261,347.33955824)(90.91722265,347.33455825)(90.86722732,347.32456697)
\curveto(90.83722273,347.31455827)(90.79722277,347.30955827)(90.74722732,347.30956697)
\curveto(90.69722287,347.31955826)(90.65222292,347.32455826)(90.61222732,347.32456697)
\lineto(90.26722732,347.32456697)
\curveto(90.14722342,347.32455826)(90.05722351,347.34955823)(89.99722732,347.39956697)
\curveto(89.93722363,347.43955814)(89.92222365,347.50955807)(89.95222732,347.60956697)
\curveto(89.9722236,347.68955789)(90.00722356,347.75955782)(90.05722732,347.81956697)
\curveto(90.10722346,347.88955769)(90.15222342,347.95955762)(90.19222732,348.02956697)
\curveto(90.29222328,348.16955741)(90.38722318,348.30455728)(90.47722732,348.43456697)
\curveto(90.567223,348.56455702)(90.65722291,348.69955688)(90.74722732,348.83956697)
\curveto(90.79722277,348.91955666)(90.84722272,349.00455658)(90.89722732,349.09456697)
\curveto(90.95722261,349.1845564)(91.02222255,349.25455633)(91.09222732,349.30456697)
\curveto(91.13222244,349.33455625)(91.20222237,349.36955621)(91.30222732,349.40956697)
\curveto(91.32222225,349.41955616)(91.34722222,349.41955616)(91.37722732,349.40956697)
\curveto(91.41722215,349.40955617)(91.44722212,349.41455617)(91.46722732,349.42456697)
}
}
{
\newrgbcolor{curcolor}{0 0 0}
\pscustom[linestyle=none,fillstyle=solid,fillcolor=curcolor]
{
\newpath
\moveto(100.6221492,346.54456697)
\curveto(101.22214339,346.56455902)(101.72214289,346.4795591)(102.1221492,346.28956697)
\curveto(102.52214209,346.09955948)(102.83714178,345.81955976)(103.0671492,345.44956697)
\curveto(103.13714148,345.33956024)(103.19214142,345.21956036)(103.2321492,345.08956697)
\curveto(103.27214134,344.96956061)(103.3121413,344.84456074)(103.3521492,344.71456697)
\curveto(103.37214124,344.63456095)(103.38214123,344.55956102)(103.3821492,344.48956697)
\curveto(103.39214122,344.41956116)(103.40714121,344.34956123)(103.4271492,344.27956697)
\curveto(103.42714119,344.21956136)(103.43214118,344.1795614)(103.4421492,344.15956697)
\curveto(103.46214115,344.01956156)(103.47214114,343.87456171)(103.4721492,343.72456697)
\lineto(103.4721492,343.28956697)
\lineto(103.4721492,341.95456697)
\lineto(103.4721492,339.52456697)
\curveto(103.47214114,339.33456625)(103.46714115,339.14956643)(103.4571492,338.96956697)
\curveto(103.45714116,338.79956678)(103.38714123,338.68956689)(103.2471492,338.63956697)
\curveto(103.18714143,338.61956696)(103.1171415,338.60956697)(103.0371492,338.60956697)
\lineto(102.7971492,338.60956697)
\lineto(101.9871492,338.60956697)
\curveto(101.86714275,338.60956697)(101.75714286,338.61456697)(101.6571492,338.62456697)
\curveto(101.56714305,338.64456694)(101.49714312,338.68956689)(101.4471492,338.75956697)
\curveto(101.40714321,338.81956676)(101.38214323,338.89456669)(101.3721492,338.98456697)
\lineto(101.3721492,339.29956697)
\lineto(101.3721492,340.34956697)
\lineto(101.3721492,342.58456697)
\curveto(101.37214324,342.95456263)(101.35714326,343.29456229)(101.3271492,343.60456697)
\curveto(101.29714332,343.92456166)(101.20714341,344.19456139)(101.0571492,344.41456697)
\curveto(100.9171437,344.61456097)(100.7121439,344.75456083)(100.4421492,344.83456697)
\curveto(100.39214422,344.85456073)(100.33714428,344.86456072)(100.2771492,344.86456697)
\curveto(100.22714439,344.86456072)(100.17214444,344.87456071)(100.1121492,344.89456697)
\curveto(100.06214455,344.90456068)(99.99714462,344.90456068)(99.9171492,344.89456697)
\curveto(99.84714477,344.89456069)(99.79214482,344.88956069)(99.7521492,344.87956697)
\curveto(99.7121449,344.86956071)(99.67714494,344.86456072)(99.6471492,344.86456697)
\curveto(99.617145,344.86456072)(99.58714503,344.85956072)(99.5571492,344.84956697)
\curveto(99.32714529,344.78956079)(99.14214547,344.70956087)(99.0021492,344.60956697)
\curveto(98.68214593,344.3795612)(98.49214612,344.04456154)(98.4321492,343.60456697)
\curveto(98.37214624,343.16456242)(98.34214627,342.66956291)(98.3421492,342.11956697)
\lineto(98.3421492,340.24456697)
\lineto(98.3421492,339.32956697)
\lineto(98.3421492,339.05956697)
\curveto(98.34214627,338.96956661)(98.32714629,338.89456669)(98.2971492,338.83456697)
\curveto(98.24714637,338.72456686)(98.16714645,338.65956692)(98.0571492,338.63956697)
\curveto(97.94714667,338.61956696)(97.8121468,338.60956697)(97.6521492,338.60956697)
\lineto(96.9021492,338.60956697)
\curveto(96.79214782,338.60956697)(96.68214793,338.61456697)(96.5721492,338.62456697)
\curveto(96.46214815,338.63456695)(96.38214823,338.66956691)(96.3321492,338.72956697)
\curveto(96.26214835,338.81956676)(96.22714839,338.94956663)(96.2271492,339.11956697)
\curveto(96.23714838,339.28956629)(96.24214837,339.44956613)(96.2421492,339.59956697)
\lineto(96.2421492,341.63956697)
\lineto(96.2421492,344.93956697)
\lineto(96.2421492,345.70456697)
\lineto(96.2421492,346.00456697)
\curveto(96.25214836,346.09455949)(96.28214833,346.16955941)(96.3321492,346.22956697)
\curveto(96.35214826,346.25955932)(96.38214823,346.2795593)(96.4221492,346.28956697)
\curveto(96.47214814,346.30955927)(96.52214809,346.32455926)(96.5721492,346.33456697)
\lineto(96.6471492,346.33456697)
\curveto(96.69714792,346.34455924)(96.74714787,346.34955923)(96.7971492,346.34956697)
\lineto(96.9621492,346.34956697)
\lineto(97.5921492,346.34956697)
\curveto(97.67214694,346.34955923)(97.74714687,346.34455924)(97.8171492,346.33456697)
\curveto(97.89714672,346.33455925)(97.96714665,346.32455926)(98.0271492,346.30456697)
\curveto(98.09714652,346.27455931)(98.14214647,346.22955935)(98.1621492,346.16956697)
\curveto(98.19214642,346.10955947)(98.2171464,346.03955954)(98.2371492,345.95956697)
\curveto(98.24714637,345.91955966)(98.24714637,345.8845597)(98.2371492,345.85456697)
\curveto(98.23714638,345.82455976)(98.24714637,345.79455979)(98.2671492,345.76456697)
\curveto(98.28714633,345.71455987)(98.30214631,345.6845599)(98.3121492,345.67456697)
\curveto(98.33214628,345.66455992)(98.35714626,345.64955993)(98.3871492,345.62956697)
\curveto(98.49714612,345.61955996)(98.58714603,345.65455993)(98.6571492,345.73456697)
\curveto(98.72714589,345.82455976)(98.80214581,345.89455969)(98.8821492,345.94456697)
\curveto(99.15214546,346.14455944)(99.45214516,346.30455928)(99.7821492,346.42456697)
\curveto(99.87214474,346.45455913)(99.96214465,346.47455911)(100.0521492,346.48456697)
\curveto(100.15214446,346.49455909)(100.25714436,346.50955907)(100.3671492,346.52956697)
\curveto(100.39714422,346.53955904)(100.44214417,346.53955904)(100.5021492,346.52956697)
\curveto(100.56214405,346.52955905)(100.60214401,346.53455905)(100.6221492,346.54456697)
}
}
{
\newrgbcolor{curcolor}{0 0 0}
\pscustom[linestyle=none,fillstyle=solid,fillcolor=curcolor]
{
}
}
{
\newrgbcolor{curcolor}{0 0 0}
\pscustom[linestyle=none,fillstyle=solid,fillcolor=curcolor]
{
\newpath
\moveto(116.85355545,339.46456697)
\lineto(116.85355545,339.04456697)
\curveto(116.85354708,338.91456667)(116.82354711,338.80956677)(116.76355545,338.72956697)
\curveto(116.71354722,338.6795669)(116.64854728,338.64456694)(116.56855545,338.62456697)
\curveto(116.48854744,338.61456697)(116.39854753,338.60956697)(116.29855545,338.60956697)
\lineto(115.47355545,338.60956697)
\lineto(115.18855545,338.60956697)
\curveto(115.10854882,338.61956696)(115.04354889,338.64456694)(114.99355545,338.68456697)
\curveto(114.92354901,338.73456685)(114.88354905,338.79956678)(114.87355545,338.87956697)
\curveto(114.86354907,338.95956662)(114.84354909,339.03956654)(114.81355545,339.11956697)
\curveto(114.79354914,339.13956644)(114.77354916,339.15456643)(114.75355545,339.16456697)
\curveto(114.74354919,339.1845664)(114.7285492,339.20456638)(114.70855545,339.22456697)
\curveto(114.59854933,339.22456636)(114.51854941,339.19956638)(114.46855545,339.14956697)
\lineto(114.31855545,338.99956697)
\curveto(114.24854968,338.94956663)(114.18354975,338.90456668)(114.12355545,338.86456697)
\curveto(114.06354987,338.83456675)(113.99854993,338.79456679)(113.92855545,338.74456697)
\curveto(113.88855004,338.72456686)(113.84355009,338.70456688)(113.79355545,338.68456697)
\curveto(113.75355018,338.66456692)(113.70855022,338.64456694)(113.65855545,338.62456697)
\curveto(113.51855041,338.57456701)(113.36855056,338.52956705)(113.20855545,338.48956697)
\curveto(113.15855077,338.46956711)(113.11355082,338.45956712)(113.07355545,338.45956697)
\curveto(113.0335509,338.45956712)(112.99355094,338.45456713)(112.95355545,338.44456697)
\lineto(112.81855545,338.44456697)
\curveto(112.78855114,338.43456715)(112.74855118,338.42956715)(112.69855545,338.42956697)
\lineto(112.56355545,338.42956697)
\curveto(112.50355143,338.40956717)(112.41355152,338.40456718)(112.29355545,338.41456697)
\curveto(112.17355176,338.41456717)(112.08855184,338.42456716)(112.03855545,338.44456697)
\curveto(111.96855196,338.46456712)(111.90355203,338.47456711)(111.84355545,338.47456697)
\curveto(111.79355214,338.46456712)(111.73855219,338.46956711)(111.67855545,338.48956697)
\lineto(111.31855545,338.60956697)
\curveto(111.20855272,338.63956694)(111.09855283,338.6795669)(110.98855545,338.72956697)
\curveto(110.63855329,338.8795667)(110.32355361,339.10956647)(110.04355545,339.41956697)
\curveto(109.77355416,339.73956584)(109.55855437,340.07456551)(109.39855545,340.42456697)
\curveto(109.34855458,340.53456505)(109.30855462,340.63956494)(109.27855545,340.73956697)
\curveto(109.24855468,340.84956473)(109.21355472,340.95956462)(109.17355545,341.06956697)
\curveto(109.16355477,341.10956447)(109.15855477,341.14456444)(109.15855545,341.17456697)
\curveto(109.15855477,341.21456437)(109.14855478,341.25956432)(109.12855545,341.30956697)
\curveto(109.10855482,341.38956419)(109.08855484,341.47456411)(109.06855545,341.56456697)
\curveto(109.05855487,341.66456392)(109.04355489,341.76456382)(109.02355545,341.86456697)
\curveto(109.01355492,341.89456369)(109.00855492,341.92956365)(109.00855545,341.96956697)
\curveto(109.01855491,342.00956357)(109.01855491,342.04456354)(109.00855545,342.07456697)
\lineto(109.00855545,342.20956697)
\curveto(109.00855492,342.25956332)(109.00355493,342.30956327)(108.99355545,342.35956697)
\curveto(108.98355495,342.40956317)(108.97855495,342.46456312)(108.97855545,342.52456697)
\curveto(108.97855495,342.59456299)(108.98355495,342.64956293)(108.99355545,342.68956697)
\curveto(109.00355493,342.73956284)(109.00855492,342.7845628)(109.00855545,342.82456697)
\lineto(109.00855545,342.97456697)
\curveto(109.01855491,343.02456256)(109.01855491,343.06956251)(109.00855545,343.10956697)
\curveto(109.00855492,343.15956242)(109.01855491,343.20956237)(109.03855545,343.25956697)
\curveto(109.05855487,343.36956221)(109.07355486,343.47456211)(109.08355545,343.57456697)
\curveto(109.10355483,343.67456191)(109.1285548,343.77456181)(109.15855545,343.87456697)
\curveto(109.19855473,343.99456159)(109.2335547,344.10956147)(109.26355545,344.21956697)
\curveto(109.29355464,344.32956125)(109.3335546,344.43956114)(109.38355545,344.54956697)
\curveto(109.52355441,344.84956073)(109.69855423,345.13456045)(109.90855545,345.40456697)
\curveto(109.928554,345.43456015)(109.95355398,345.45956012)(109.98355545,345.47956697)
\curveto(110.02355391,345.50956007)(110.05355388,345.53956004)(110.07355545,345.56956697)
\curveto(110.11355382,345.61955996)(110.15355378,345.66455992)(110.19355545,345.70456697)
\curveto(110.2335537,345.74455984)(110.27855365,345.7845598)(110.32855545,345.82456697)
\curveto(110.36855356,345.84455974)(110.40355353,345.86955971)(110.43355545,345.89956697)
\curveto(110.46355347,345.93955964)(110.49855343,345.96955961)(110.53855545,345.98956697)
\curveto(110.78855314,346.15955942)(111.07855285,346.29955928)(111.40855545,346.40956697)
\curveto(111.47855245,346.42955915)(111.54855238,346.44455914)(111.61855545,346.45456697)
\curveto(111.69855223,346.46455912)(111.77855215,346.4795591)(111.85855545,346.49956697)
\curveto(111.928552,346.51955906)(112.01855191,346.52955905)(112.12855545,346.52956697)
\curveto(112.23855169,346.53955904)(112.34855158,346.54455904)(112.45855545,346.54456697)
\curveto(112.56855136,346.54455904)(112.67355126,346.53955904)(112.77355545,346.52956697)
\curveto(112.88355105,346.51955906)(112.97355096,346.50455908)(113.04355545,346.48456697)
\curveto(113.19355074,346.43455915)(113.33855059,346.38955919)(113.47855545,346.34956697)
\curveto(113.61855031,346.30955927)(113.74855018,346.25455933)(113.86855545,346.18456697)
\curveto(113.93854999,346.13455945)(114.00354993,346.0845595)(114.06355545,346.03456697)
\curveto(114.12354981,345.99455959)(114.18854974,345.94955963)(114.25855545,345.89956697)
\curveto(114.29854963,345.86955971)(114.35354958,345.82955975)(114.42355545,345.77956697)
\curveto(114.50354943,345.72955985)(114.57854935,345.72955985)(114.64855545,345.77956697)
\curveto(114.68854924,345.79955978)(114.70854922,345.83455975)(114.70855545,345.88456697)
\curveto(114.70854922,345.93455965)(114.71854921,345.9845596)(114.73855545,346.03456697)
\lineto(114.73855545,346.18456697)
\curveto(114.74854918,346.21455937)(114.75354918,346.24955933)(114.75355545,346.28956697)
\lineto(114.75355545,346.40956697)
\lineto(114.75355545,348.44956697)
\curveto(114.75354918,348.55955702)(114.74854918,348.6795569)(114.73855545,348.80956697)
\curveto(114.73854919,348.94955663)(114.76354917,349.05455653)(114.81355545,349.12456697)
\curveto(114.85354908,349.20455638)(114.928549,349.25455633)(115.03855545,349.27456697)
\curveto(115.05854887,349.2845563)(115.07854885,349.2845563)(115.09855545,349.27456697)
\curveto(115.11854881,349.27455631)(115.13854879,349.2795563)(115.15855545,349.28956697)
\lineto(116.22355545,349.28956697)
\curveto(116.34354759,349.28955629)(116.45354748,349.2845563)(116.55355545,349.27456697)
\curveto(116.65354728,349.26455632)(116.7285472,349.22455636)(116.77855545,349.15456697)
\curveto(116.8285471,349.07455651)(116.85354708,348.96955661)(116.85355545,348.83956697)
\lineto(116.85355545,348.47956697)
\lineto(116.85355545,339.46456697)
\moveto(114.81355545,342.40456697)
\curveto(114.82354911,342.44456314)(114.82354911,342.4845631)(114.81355545,342.52456697)
\lineto(114.81355545,342.65956697)
\curveto(114.81354912,342.75956282)(114.80854912,342.85956272)(114.79855545,342.95956697)
\curveto(114.78854914,343.05956252)(114.77354916,343.14956243)(114.75355545,343.22956697)
\curveto(114.7335492,343.33956224)(114.71354922,343.43956214)(114.69355545,343.52956697)
\curveto(114.68354925,343.61956196)(114.65854927,343.70456188)(114.61855545,343.78456697)
\curveto(114.47854945,344.14456144)(114.27354966,344.42956115)(114.00355545,344.63956697)
\curveto(113.74355019,344.84956073)(113.36355057,344.95456063)(112.86355545,344.95456697)
\curveto(112.80355113,344.95456063)(112.72355121,344.94456064)(112.62355545,344.92456697)
\curveto(112.54355139,344.90456068)(112.46855146,344.8845607)(112.39855545,344.86456697)
\curveto(112.33855159,344.85456073)(112.27855165,344.83456075)(112.21855545,344.80456697)
\curveto(111.94855198,344.69456089)(111.73855219,344.52456106)(111.58855545,344.29456697)
\curveto(111.43855249,344.06456152)(111.31855261,343.80456178)(111.22855545,343.51456697)
\curveto(111.19855273,343.41456217)(111.17855275,343.31456227)(111.16855545,343.21456697)
\curveto(111.15855277,343.11456247)(111.13855279,343.00956257)(111.10855545,342.89956697)
\lineto(111.10855545,342.68956697)
\curveto(111.08855284,342.59956298)(111.08355285,342.47456311)(111.09355545,342.31456697)
\curveto(111.10355283,342.16456342)(111.11855281,342.05456353)(111.13855545,341.98456697)
\lineto(111.13855545,341.89456697)
\curveto(111.14855278,341.87456371)(111.15355278,341.85456373)(111.15355545,341.83456697)
\curveto(111.17355276,341.75456383)(111.18855274,341.6795639)(111.19855545,341.60956697)
\curveto(111.21855271,341.53956404)(111.23855269,341.46456412)(111.25855545,341.38456697)
\curveto(111.4285525,340.86456472)(111.71855221,340.4795651)(112.12855545,340.22956697)
\curveto(112.25855167,340.13956544)(112.43855149,340.06956551)(112.66855545,340.01956697)
\curveto(112.70855122,340.00956557)(112.76855116,340.00456558)(112.84855545,340.00456697)
\curveto(112.87855105,339.99456559)(112.92355101,339.9845656)(112.98355545,339.97456697)
\curveto(113.05355088,339.97456561)(113.10855082,339.9795656)(113.14855545,339.98956697)
\curveto(113.2285507,340.00956557)(113.30855062,340.02456556)(113.38855545,340.03456697)
\curveto(113.46855046,340.04456554)(113.54855038,340.06456552)(113.62855545,340.09456697)
\curveto(113.87855005,340.20456538)(114.07854985,340.34456524)(114.22855545,340.51456697)
\curveto(114.37854955,340.6845649)(114.50854942,340.89956468)(114.61855545,341.15956697)
\curveto(114.65854927,341.24956433)(114.68854924,341.33956424)(114.70855545,341.42956697)
\curveto(114.7285492,341.52956405)(114.74854918,341.63456395)(114.76855545,341.74456697)
\curveto(114.77854915,341.79456379)(114.77854915,341.83956374)(114.76855545,341.87956697)
\curveto(114.76854916,341.92956365)(114.77854915,341.9795636)(114.79855545,342.02956697)
\curveto(114.80854912,342.05956352)(114.81354912,342.09456349)(114.81355545,342.13456697)
\lineto(114.81355545,342.26956697)
\lineto(114.81355545,342.40456697)
}
}
{
\newrgbcolor{curcolor}{0 0 0}
\pscustom[linestyle=none,fillstyle=solid,fillcolor=curcolor]
{
\newpath
\moveto(125.79847732,342.55456697)
\curveto(125.81846916,342.47456311)(125.81846916,342.3845632)(125.79847732,342.28456697)
\curveto(125.7784692,342.1845634)(125.74346923,342.11956346)(125.69347732,342.08956697)
\curveto(125.64346933,342.04956353)(125.56846941,342.01956356)(125.46847732,341.99956697)
\curveto(125.3784696,341.98956359)(125.2734697,341.9795636)(125.15347732,341.96956697)
\lineto(124.80847732,341.96956697)
\curveto(124.69847028,341.9795636)(124.59847038,341.9845636)(124.50847732,341.98456697)
\lineto(120.84847732,341.98456697)
\lineto(120.63847732,341.98456697)
\curveto(120.5784744,341.9845636)(120.52347445,341.97456361)(120.47347732,341.95456697)
\curveto(120.39347458,341.91456367)(120.34347463,341.87456371)(120.32347732,341.83456697)
\curveto(120.30347467,341.81456377)(120.28347469,341.77456381)(120.26347732,341.71456697)
\curveto(120.24347473,341.66456392)(120.23847474,341.61456397)(120.24847732,341.56456697)
\curveto(120.26847471,341.50456408)(120.2784747,341.44456414)(120.27847732,341.38456697)
\curveto(120.28847469,341.33456425)(120.30347467,341.2795643)(120.32347732,341.21956697)
\curveto(120.40347457,340.9795646)(120.49847448,340.7795648)(120.60847732,340.61956697)
\curveto(120.72847425,340.46956511)(120.88847409,340.33456525)(121.08847732,340.21456697)
\curveto(121.16847381,340.16456542)(121.24847373,340.12956545)(121.32847732,340.10956697)
\curveto(121.41847356,340.09956548)(121.50847347,340.0795655)(121.59847732,340.04956697)
\curveto(121.6784733,340.02956555)(121.78847319,340.01456557)(121.92847732,340.00456697)
\curveto(122.06847291,339.99456559)(122.18847279,339.99956558)(122.28847732,340.01956697)
\lineto(122.42347732,340.01956697)
\curveto(122.52347245,340.03956554)(122.61347236,340.05956552)(122.69347732,340.07956697)
\curveto(122.78347219,340.10956547)(122.86847211,340.13956544)(122.94847732,340.16956697)
\curveto(123.04847193,340.21956536)(123.15847182,340.2845653)(123.27847732,340.36456697)
\curveto(123.40847157,340.44456514)(123.50347147,340.52456506)(123.56347732,340.60456697)
\curveto(123.61347136,340.67456491)(123.66347131,340.73956484)(123.71347732,340.79956697)
\curveto(123.7734712,340.86956471)(123.84347113,340.91956466)(123.92347732,340.94956697)
\curveto(124.02347095,340.99956458)(124.14847083,341.01956456)(124.29847732,341.00956697)
\lineto(124.73347732,341.00956697)
\lineto(124.91347732,341.00956697)
\curveto(124.98346999,341.01956456)(125.04346993,341.01456457)(125.09347732,340.99456697)
\lineto(125.24347732,340.99456697)
\curveto(125.34346963,340.97456461)(125.41346956,340.94956463)(125.45347732,340.91956697)
\curveto(125.49346948,340.89956468)(125.51346946,340.85456473)(125.51347732,340.78456697)
\curveto(125.52346945,340.71456487)(125.51846946,340.65456493)(125.49847732,340.60456697)
\curveto(125.44846953,340.46456512)(125.39346958,340.33956524)(125.33347732,340.22956697)
\curveto(125.2734697,340.11956546)(125.20346977,340.00956557)(125.12347732,339.89956697)
\curveto(124.90347007,339.56956601)(124.65347032,339.30456628)(124.37347732,339.10456697)
\curveto(124.09347088,338.90456668)(123.74347123,338.73456685)(123.32347732,338.59456697)
\curveto(123.21347176,338.55456703)(123.10347187,338.52956705)(122.99347732,338.51956697)
\curveto(122.88347209,338.50956707)(122.76847221,338.48956709)(122.64847732,338.45956697)
\curveto(122.60847237,338.44956713)(122.56347241,338.44956713)(122.51347732,338.45956697)
\curveto(122.4734725,338.45956712)(122.43347254,338.45456713)(122.39347732,338.44456697)
\lineto(122.22847732,338.44456697)
\curveto(122.1784728,338.42456716)(122.11847286,338.41956716)(122.04847732,338.42956697)
\curveto(121.98847299,338.42956715)(121.93347304,338.43456715)(121.88347732,338.44456697)
\curveto(121.80347317,338.45456713)(121.73347324,338.45456713)(121.67347732,338.44456697)
\curveto(121.61347336,338.43456715)(121.54847343,338.43956714)(121.47847732,338.45956697)
\curveto(121.42847355,338.4795671)(121.3734736,338.48956709)(121.31347732,338.48956697)
\curveto(121.25347372,338.48956709)(121.19847378,338.49956708)(121.14847732,338.51956697)
\curveto(121.03847394,338.53956704)(120.92847405,338.56456702)(120.81847732,338.59456697)
\curveto(120.70847427,338.61456697)(120.60847437,338.64956693)(120.51847732,338.69956697)
\curveto(120.40847457,338.73956684)(120.30347467,338.77456681)(120.20347732,338.80456697)
\curveto(120.11347486,338.84456674)(120.02847495,338.88956669)(119.94847732,338.93956697)
\curveto(119.62847535,339.13956644)(119.34347563,339.36956621)(119.09347732,339.62956697)
\curveto(118.84347613,339.89956568)(118.63847634,340.20956537)(118.47847732,340.55956697)
\curveto(118.42847655,340.66956491)(118.38847659,340.7795648)(118.35847732,340.88956697)
\curveto(118.32847665,341.00956457)(118.28847669,341.12956445)(118.23847732,341.24956697)
\curveto(118.22847675,341.28956429)(118.22347675,341.32456426)(118.22347732,341.35456697)
\curveto(118.22347675,341.39456419)(118.21847676,341.43456415)(118.20847732,341.47456697)
\curveto(118.16847681,341.59456399)(118.14347683,341.72456386)(118.13347732,341.86456697)
\lineto(118.10347732,342.28456697)
\curveto(118.10347687,342.33456325)(118.09847688,342.38956319)(118.08847732,342.44956697)
\curveto(118.08847689,342.50956307)(118.09347688,342.56456302)(118.10347732,342.61456697)
\lineto(118.10347732,342.79456697)
\lineto(118.14847732,343.15456697)
\curveto(118.18847679,343.32456226)(118.22347675,343.48956209)(118.25347732,343.64956697)
\curveto(118.28347669,343.80956177)(118.32847665,343.95956162)(118.38847732,344.09956697)
\curveto(118.81847616,345.13956044)(119.54847543,345.87455971)(120.57847732,346.30456697)
\curveto(120.71847426,346.36455922)(120.85847412,346.40455918)(120.99847732,346.42456697)
\curveto(121.14847383,346.45455913)(121.30347367,346.48955909)(121.46347732,346.52956697)
\curveto(121.54347343,346.53955904)(121.61847336,346.54455904)(121.68847732,346.54456697)
\curveto(121.75847322,346.54455904)(121.83347314,346.54955903)(121.91347732,346.55956697)
\curveto(122.42347255,346.56955901)(122.85847212,346.50955907)(123.21847732,346.37956697)
\curveto(123.58847139,346.25955932)(123.91847106,346.09955948)(124.20847732,345.89956697)
\curveto(124.29847068,345.83955974)(124.38847059,345.76955981)(124.47847732,345.68956697)
\curveto(124.56847041,345.61955996)(124.64847033,345.54456004)(124.71847732,345.46456697)
\curveto(124.74847023,345.41456017)(124.78847019,345.37456021)(124.83847732,345.34456697)
\curveto(124.91847006,345.23456035)(124.99346998,345.11956046)(125.06347732,344.99956697)
\curveto(125.13346984,344.88956069)(125.20846977,344.77456081)(125.28847732,344.65456697)
\curveto(125.33846964,344.56456102)(125.3784696,344.46956111)(125.40847732,344.36956697)
\curveto(125.44846953,344.2795613)(125.48846949,344.1795614)(125.52847732,344.06956697)
\curveto(125.5784694,343.93956164)(125.61846936,343.80456178)(125.64847732,343.66456697)
\curveto(125.6784693,343.52456206)(125.71346926,343.3845622)(125.75347732,343.24456697)
\curveto(125.7734692,343.16456242)(125.7784692,343.07456251)(125.76847732,342.97456697)
\curveto(125.76846921,342.8845627)(125.7784692,342.79956278)(125.79847732,342.71956697)
\lineto(125.79847732,342.55456697)
\moveto(123.54847732,343.43956697)
\curveto(123.61847136,343.53956204)(123.62347135,343.65956192)(123.56347732,343.79956697)
\curveto(123.51347146,343.94956163)(123.4734715,344.05956152)(123.44347732,344.12956697)
\curveto(123.30347167,344.39956118)(123.11847186,344.60456098)(122.88847732,344.74456697)
\curveto(122.65847232,344.89456069)(122.33847264,344.97456061)(121.92847732,344.98456697)
\curveto(121.89847308,344.96456062)(121.86347311,344.95956062)(121.82347732,344.96956697)
\curveto(121.78347319,344.9795606)(121.74847323,344.9795606)(121.71847732,344.96956697)
\curveto(121.66847331,344.94956063)(121.61347336,344.93456065)(121.55347732,344.92456697)
\curveto(121.49347348,344.92456066)(121.43847354,344.91456067)(121.38847732,344.89456697)
\curveto(120.94847403,344.75456083)(120.62347435,344.4795611)(120.41347732,344.06956697)
\curveto(120.39347458,344.02956155)(120.36847461,343.97456161)(120.33847732,343.90456697)
\curveto(120.31847466,343.84456174)(120.30347467,343.7795618)(120.29347732,343.70956697)
\curveto(120.28347469,343.64956193)(120.28347469,343.58956199)(120.29347732,343.52956697)
\curveto(120.31347466,343.46956211)(120.34847463,343.41956216)(120.39847732,343.37956697)
\curveto(120.4784745,343.32956225)(120.58847439,343.30456228)(120.72847732,343.30456697)
\lineto(121.13347732,343.30456697)
\lineto(122.79847732,343.30456697)
\lineto(123.23347732,343.30456697)
\curveto(123.39347158,343.31456227)(123.49847148,343.35956222)(123.54847732,343.43956697)
}
}
{
\newrgbcolor{curcolor}{0 0 0}
\pscustom[linestyle=none,fillstyle=solid,fillcolor=curcolor]
{
}
}
{
\newrgbcolor{curcolor}{0 0 0}
\pscustom[linestyle=none,fillstyle=solid,fillcolor=curcolor]
{
\newpath
\moveto(138.57191482,342.55456697)
\curveto(138.59190666,342.47456311)(138.59190666,342.3845632)(138.57191482,342.28456697)
\curveto(138.5519067,342.1845634)(138.51690673,342.11956346)(138.46691482,342.08956697)
\curveto(138.41690683,342.04956353)(138.34190691,342.01956356)(138.24191482,341.99956697)
\curveto(138.1519071,341.98956359)(138.0469072,341.9795636)(137.92691482,341.96956697)
\lineto(137.58191482,341.96956697)
\curveto(137.47190778,341.9795636)(137.37190788,341.9845636)(137.28191482,341.98456697)
\lineto(133.62191482,341.98456697)
\lineto(133.41191482,341.98456697)
\curveto(133.3519119,341.9845636)(133.29691195,341.97456361)(133.24691482,341.95456697)
\curveto(133.16691208,341.91456367)(133.11691213,341.87456371)(133.09691482,341.83456697)
\curveto(133.07691217,341.81456377)(133.05691219,341.77456381)(133.03691482,341.71456697)
\curveto(133.01691223,341.66456392)(133.01191224,341.61456397)(133.02191482,341.56456697)
\curveto(133.04191221,341.50456408)(133.0519122,341.44456414)(133.05191482,341.38456697)
\curveto(133.06191219,341.33456425)(133.07691217,341.2795643)(133.09691482,341.21956697)
\curveto(133.17691207,340.9795646)(133.27191198,340.7795648)(133.38191482,340.61956697)
\curveto(133.50191175,340.46956511)(133.66191159,340.33456525)(133.86191482,340.21456697)
\curveto(133.94191131,340.16456542)(134.02191123,340.12956545)(134.10191482,340.10956697)
\curveto(134.19191106,340.09956548)(134.28191097,340.0795655)(134.37191482,340.04956697)
\curveto(134.4519108,340.02956555)(134.56191069,340.01456557)(134.70191482,340.00456697)
\curveto(134.84191041,339.99456559)(134.96191029,339.99956558)(135.06191482,340.01956697)
\lineto(135.19691482,340.01956697)
\curveto(135.29690995,340.03956554)(135.38690986,340.05956552)(135.46691482,340.07956697)
\curveto(135.55690969,340.10956547)(135.64190961,340.13956544)(135.72191482,340.16956697)
\curveto(135.82190943,340.21956536)(135.93190932,340.2845653)(136.05191482,340.36456697)
\curveto(136.18190907,340.44456514)(136.27690897,340.52456506)(136.33691482,340.60456697)
\curveto(136.38690886,340.67456491)(136.43690881,340.73956484)(136.48691482,340.79956697)
\curveto(136.5469087,340.86956471)(136.61690863,340.91956466)(136.69691482,340.94956697)
\curveto(136.79690845,340.99956458)(136.92190833,341.01956456)(137.07191482,341.00956697)
\lineto(137.50691482,341.00956697)
\lineto(137.68691482,341.00956697)
\curveto(137.75690749,341.01956456)(137.81690743,341.01456457)(137.86691482,340.99456697)
\lineto(138.01691482,340.99456697)
\curveto(138.11690713,340.97456461)(138.18690706,340.94956463)(138.22691482,340.91956697)
\curveto(138.26690698,340.89956468)(138.28690696,340.85456473)(138.28691482,340.78456697)
\curveto(138.29690695,340.71456487)(138.29190696,340.65456493)(138.27191482,340.60456697)
\curveto(138.22190703,340.46456512)(138.16690708,340.33956524)(138.10691482,340.22956697)
\curveto(138.0469072,340.11956546)(137.97690727,340.00956557)(137.89691482,339.89956697)
\curveto(137.67690757,339.56956601)(137.42690782,339.30456628)(137.14691482,339.10456697)
\curveto(136.86690838,338.90456668)(136.51690873,338.73456685)(136.09691482,338.59456697)
\curveto(135.98690926,338.55456703)(135.87690937,338.52956705)(135.76691482,338.51956697)
\curveto(135.65690959,338.50956707)(135.54190971,338.48956709)(135.42191482,338.45956697)
\curveto(135.38190987,338.44956713)(135.33690991,338.44956713)(135.28691482,338.45956697)
\curveto(135.24691,338.45956712)(135.20691004,338.45456713)(135.16691482,338.44456697)
\lineto(135.00191482,338.44456697)
\curveto(134.9519103,338.42456716)(134.89191036,338.41956716)(134.82191482,338.42956697)
\curveto(134.76191049,338.42956715)(134.70691054,338.43456715)(134.65691482,338.44456697)
\curveto(134.57691067,338.45456713)(134.50691074,338.45456713)(134.44691482,338.44456697)
\curveto(134.38691086,338.43456715)(134.32191093,338.43956714)(134.25191482,338.45956697)
\curveto(134.20191105,338.4795671)(134.1469111,338.48956709)(134.08691482,338.48956697)
\curveto(134.02691122,338.48956709)(133.97191128,338.49956708)(133.92191482,338.51956697)
\curveto(133.81191144,338.53956704)(133.70191155,338.56456702)(133.59191482,338.59456697)
\curveto(133.48191177,338.61456697)(133.38191187,338.64956693)(133.29191482,338.69956697)
\curveto(133.18191207,338.73956684)(133.07691217,338.77456681)(132.97691482,338.80456697)
\curveto(132.88691236,338.84456674)(132.80191245,338.88956669)(132.72191482,338.93956697)
\curveto(132.40191285,339.13956644)(132.11691313,339.36956621)(131.86691482,339.62956697)
\curveto(131.61691363,339.89956568)(131.41191384,340.20956537)(131.25191482,340.55956697)
\curveto(131.20191405,340.66956491)(131.16191409,340.7795648)(131.13191482,340.88956697)
\curveto(131.10191415,341.00956457)(131.06191419,341.12956445)(131.01191482,341.24956697)
\curveto(131.00191425,341.28956429)(130.99691425,341.32456426)(130.99691482,341.35456697)
\curveto(130.99691425,341.39456419)(130.99191426,341.43456415)(130.98191482,341.47456697)
\curveto(130.94191431,341.59456399)(130.91691433,341.72456386)(130.90691482,341.86456697)
\lineto(130.87691482,342.28456697)
\curveto(130.87691437,342.33456325)(130.87191438,342.38956319)(130.86191482,342.44956697)
\curveto(130.86191439,342.50956307)(130.86691438,342.56456302)(130.87691482,342.61456697)
\lineto(130.87691482,342.79456697)
\lineto(130.92191482,343.15456697)
\curveto(130.96191429,343.32456226)(130.99691425,343.48956209)(131.02691482,343.64956697)
\curveto(131.05691419,343.80956177)(131.10191415,343.95956162)(131.16191482,344.09956697)
\curveto(131.59191366,345.13956044)(132.32191293,345.87455971)(133.35191482,346.30456697)
\curveto(133.49191176,346.36455922)(133.63191162,346.40455918)(133.77191482,346.42456697)
\curveto(133.92191133,346.45455913)(134.07691117,346.48955909)(134.23691482,346.52956697)
\curveto(134.31691093,346.53955904)(134.39191086,346.54455904)(134.46191482,346.54456697)
\curveto(134.53191072,346.54455904)(134.60691064,346.54955903)(134.68691482,346.55956697)
\curveto(135.19691005,346.56955901)(135.63190962,346.50955907)(135.99191482,346.37956697)
\curveto(136.36190889,346.25955932)(136.69190856,346.09955948)(136.98191482,345.89956697)
\curveto(137.07190818,345.83955974)(137.16190809,345.76955981)(137.25191482,345.68956697)
\curveto(137.34190791,345.61955996)(137.42190783,345.54456004)(137.49191482,345.46456697)
\curveto(137.52190773,345.41456017)(137.56190769,345.37456021)(137.61191482,345.34456697)
\curveto(137.69190756,345.23456035)(137.76690748,345.11956046)(137.83691482,344.99956697)
\curveto(137.90690734,344.88956069)(137.98190727,344.77456081)(138.06191482,344.65456697)
\curveto(138.11190714,344.56456102)(138.1519071,344.46956111)(138.18191482,344.36956697)
\curveto(138.22190703,344.2795613)(138.26190699,344.1795614)(138.30191482,344.06956697)
\curveto(138.3519069,343.93956164)(138.39190686,343.80456178)(138.42191482,343.66456697)
\curveto(138.4519068,343.52456206)(138.48690676,343.3845622)(138.52691482,343.24456697)
\curveto(138.5469067,343.16456242)(138.5519067,343.07456251)(138.54191482,342.97456697)
\curveto(138.54190671,342.8845627)(138.5519067,342.79956278)(138.57191482,342.71956697)
\lineto(138.57191482,342.55456697)
\moveto(136.32191482,343.43956697)
\curveto(136.39190886,343.53956204)(136.39690885,343.65956192)(136.33691482,343.79956697)
\curveto(136.28690896,343.94956163)(136.246909,344.05956152)(136.21691482,344.12956697)
\curveto(136.07690917,344.39956118)(135.89190936,344.60456098)(135.66191482,344.74456697)
\curveto(135.43190982,344.89456069)(135.11191014,344.97456061)(134.70191482,344.98456697)
\curveto(134.67191058,344.96456062)(134.63691061,344.95956062)(134.59691482,344.96956697)
\curveto(134.55691069,344.9795606)(134.52191073,344.9795606)(134.49191482,344.96956697)
\curveto(134.44191081,344.94956063)(134.38691086,344.93456065)(134.32691482,344.92456697)
\curveto(134.26691098,344.92456066)(134.21191104,344.91456067)(134.16191482,344.89456697)
\curveto(133.72191153,344.75456083)(133.39691185,344.4795611)(133.18691482,344.06956697)
\curveto(133.16691208,344.02956155)(133.14191211,343.97456161)(133.11191482,343.90456697)
\curveto(133.09191216,343.84456174)(133.07691217,343.7795618)(133.06691482,343.70956697)
\curveto(133.05691219,343.64956193)(133.05691219,343.58956199)(133.06691482,343.52956697)
\curveto(133.08691216,343.46956211)(133.12191213,343.41956216)(133.17191482,343.37956697)
\curveto(133.251912,343.32956225)(133.36191189,343.30456228)(133.50191482,343.30456697)
\lineto(133.90691482,343.30456697)
\lineto(135.57191482,343.30456697)
\lineto(136.00691482,343.30456697)
\curveto(136.16690908,343.31456227)(136.27190898,343.35956222)(136.32191482,343.43956697)
}
}
{
\newrgbcolor{curcolor}{0 0 0}
\pscustom[linestyle=none,fillstyle=solid,fillcolor=curcolor]
{
\newpath
\moveto(142.79019607,346.55956697)
\curveto(143.54019157,346.579559)(144.19019092,346.49455909)(144.74019607,346.30456697)
\curveto(145.30018981,346.12455946)(145.72518939,345.80955977)(146.01519607,345.35956697)
\curveto(146.08518903,345.24956033)(146.14518897,345.13456045)(146.19519607,345.01456697)
\curveto(146.25518886,344.90456068)(146.30518881,344.7795608)(146.34519607,344.63956697)
\curveto(146.36518875,344.579561)(146.37518874,344.51456107)(146.37519607,344.44456697)
\curveto(146.37518874,344.37456121)(146.36518875,344.31456127)(146.34519607,344.26456697)
\curveto(146.30518881,344.20456138)(146.25018886,344.16456142)(146.18019607,344.14456697)
\curveto(146.13018898,344.12456146)(146.07018904,344.11456147)(146.00019607,344.11456697)
\lineto(145.79019607,344.11456697)
\lineto(145.13019607,344.11456697)
\curveto(145.06019005,344.11456147)(144.99019012,344.10956147)(144.92019607,344.09956697)
\curveto(144.85019026,344.09956148)(144.78519033,344.10956147)(144.72519607,344.12956697)
\curveto(144.62519049,344.14956143)(144.55019056,344.18956139)(144.50019607,344.24956697)
\curveto(144.45019066,344.30956127)(144.40519071,344.36956121)(144.36519607,344.42956697)
\lineto(144.24519607,344.63956697)
\curveto(144.2151909,344.71956086)(144.16519095,344.7845608)(144.09519607,344.83456697)
\curveto(143.99519112,344.91456067)(143.89519122,344.97456061)(143.79519607,345.01456697)
\curveto(143.70519141,345.05456053)(143.59019152,345.08956049)(143.45019607,345.11956697)
\curveto(143.38019173,345.13956044)(143.27519184,345.15456043)(143.13519607,345.16456697)
\curveto(143.00519211,345.17456041)(142.90519221,345.16956041)(142.83519607,345.14956697)
\lineto(142.73019607,345.14956697)
\lineto(142.58019607,345.11956697)
\curveto(142.54019257,345.11956046)(142.49519262,345.11456047)(142.44519607,345.10456697)
\curveto(142.27519284,345.05456053)(142.13519298,344.9845606)(142.02519607,344.89456697)
\curveto(141.92519319,344.81456077)(141.85519326,344.68956089)(141.81519607,344.51956697)
\curveto(141.79519332,344.44956113)(141.79519332,344.3845612)(141.81519607,344.32456697)
\curveto(141.83519328,344.26456132)(141.85519326,344.21456137)(141.87519607,344.17456697)
\curveto(141.94519317,344.05456153)(142.02519309,343.95956162)(142.11519607,343.88956697)
\curveto(142.2151929,343.81956176)(142.33019278,343.75956182)(142.46019607,343.70956697)
\curveto(142.65019246,343.62956195)(142.85519226,343.55956202)(143.07519607,343.49956697)
\lineto(143.76519607,343.34956697)
\curveto(144.00519111,343.30956227)(144.23519088,343.25956232)(144.45519607,343.19956697)
\curveto(144.68519043,343.14956243)(144.90019021,343.0845625)(145.10019607,343.00456697)
\curveto(145.19018992,342.96456262)(145.27518984,342.92956265)(145.35519607,342.89956697)
\curveto(145.44518967,342.8795627)(145.53018958,342.84456274)(145.61019607,342.79456697)
\curveto(145.80018931,342.67456291)(145.97018914,342.54456304)(146.12019607,342.40456697)
\curveto(146.28018883,342.26456332)(146.40518871,342.08956349)(146.49519607,341.87956697)
\curveto(146.52518859,341.80956377)(146.55018856,341.73956384)(146.57019607,341.66956697)
\curveto(146.59018852,341.59956398)(146.6101885,341.52456406)(146.63019607,341.44456697)
\curveto(146.64018847,341.3845642)(146.64518847,341.28956429)(146.64519607,341.15956697)
\curveto(146.65518846,341.03956454)(146.65518846,340.94456464)(146.64519607,340.87456697)
\lineto(146.64519607,340.79956697)
\curveto(146.62518849,340.73956484)(146.6101885,340.6795649)(146.60019607,340.61956697)
\curveto(146.60018851,340.56956501)(146.59518852,340.51956506)(146.58519607,340.46956697)
\curveto(146.5151886,340.16956541)(146.40518871,339.90456568)(146.25519607,339.67456697)
\curveto(146.09518902,339.43456615)(145.90018921,339.23956634)(145.67019607,339.08956697)
\curveto(145.44018967,338.93956664)(145.18018993,338.80956677)(144.89019607,338.69956697)
\curveto(144.78019033,338.64956693)(144.66019045,338.61456697)(144.53019607,338.59456697)
\curveto(144.4101907,338.57456701)(144.29019082,338.54956703)(144.17019607,338.51956697)
\curveto(144.08019103,338.49956708)(143.98519113,338.48956709)(143.88519607,338.48956697)
\curveto(143.79519132,338.4795671)(143.70519141,338.46456712)(143.61519607,338.44456697)
\lineto(143.34519607,338.44456697)
\curveto(143.28519183,338.42456716)(143.18019193,338.41456717)(143.03019607,338.41456697)
\curveto(142.89019222,338.41456717)(142.79019232,338.42456716)(142.73019607,338.44456697)
\curveto(142.70019241,338.44456714)(142.66519245,338.44956713)(142.62519607,338.45956697)
\lineto(142.52019607,338.45956697)
\curveto(142.40019271,338.4795671)(142.28019283,338.49456709)(142.16019607,338.50456697)
\curveto(142.04019307,338.51456707)(141.92519319,338.53456705)(141.81519607,338.56456697)
\curveto(141.42519369,338.67456691)(141.08019403,338.79956678)(140.78019607,338.93956697)
\curveto(140.48019463,339.08956649)(140.22519489,339.30956627)(140.01519607,339.59956697)
\curveto(139.87519524,339.78956579)(139.75519536,340.00956557)(139.65519607,340.25956697)
\curveto(139.63519548,340.31956526)(139.6151955,340.39956518)(139.59519607,340.49956697)
\curveto(139.57519554,340.54956503)(139.56019555,340.61956496)(139.55019607,340.70956697)
\curveto(139.54019557,340.79956478)(139.54519557,340.87456471)(139.56519607,340.93456697)
\curveto(139.59519552,341.00456458)(139.64519547,341.05456453)(139.71519607,341.08456697)
\curveto(139.76519535,341.10456448)(139.82519529,341.11456447)(139.89519607,341.11456697)
\lineto(140.12019607,341.11456697)
\lineto(140.82519607,341.11456697)
\lineto(141.06519607,341.11456697)
\curveto(141.14519397,341.11456447)(141.2151939,341.10456448)(141.27519607,341.08456697)
\curveto(141.38519373,341.04456454)(141.45519366,340.9795646)(141.48519607,340.88956697)
\curveto(141.52519359,340.79956478)(141.57019354,340.70456488)(141.62019607,340.60456697)
\curveto(141.64019347,340.55456503)(141.67519344,340.48956509)(141.72519607,340.40956697)
\curveto(141.78519333,340.32956525)(141.83519328,340.2795653)(141.87519607,340.25956697)
\curveto(141.99519312,340.15956542)(142.110193,340.0795655)(142.22019607,340.01956697)
\curveto(142.33019278,339.96956561)(142.47019264,339.91956566)(142.64019607,339.86956697)
\curveto(142.69019242,339.84956573)(142.74019237,339.83956574)(142.79019607,339.83956697)
\curveto(142.84019227,339.84956573)(142.89019222,339.84956573)(142.94019607,339.83956697)
\curveto(143.02019209,339.81956576)(143.10519201,339.80956577)(143.19519607,339.80956697)
\curveto(143.29519182,339.81956576)(143.38019173,339.83456575)(143.45019607,339.85456697)
\curveto(143.50019161,339.86456572)(143.54519157,339.86956571)(143.58519607,339.86956697)
\curveto(143.63519148,339.86956571)(143.68519143,339.8795657)(143.73519607,339.89956697)
\curveto(143.87519124,339.94956563)(144.00019111,340.00956557)(144.11019607,340.07956697)
\curveto(144.23019088,340.14956543)(144.32519079,340.23956534)(144.39519607,340.34956697)
\curveto(144.44519067,340.42956515)(144.48519063,340.55456503)(144.51519607,340.72456697)
\curveto(144.53519058,340.79456479)(144.53519058,340.85956472)(144.51519607,340.91956697)
\curveto(144.49519062,340.9795646)(144.47519064,341.02956455)(144.45519607,341.06956697)
\curveto(144.38519073,341.20956437)(144.29519082,341.31456427)(144.18519607,341.38456697)
\curveto(144.08519103,341.45456413)(143.96519115,341.51956406)(143.82519607,341.57956697)
\curveto(143.63519148,341.65956392)(143.43519168,341.72456386)(143.22519607,341.77456697)
\curveto(143.0151921,341.82456376)(142.80519231,341.8795637)(142.59519607,341.93956697)
\curveto(142.5151926,341.95956362)(142.43019268,341.97456361)(142.34019607,341.98456697)
\curveto(142.26019285,341.99456359)(142.18019293,342.00956357)(142.10019607,342.02956697)
\curveto(141.78019333,342.11956346)(141.47519364,342.20456338)(141.18519607,342.28456697)
\curveto(140.89519422,342.37456321)(140.63019448,342.50456308)(140.39019607,342.67456697)
\curveto(140.110195,342.87456271)(139.90519521,343.14456244)(139.77519607,343.48456697)
\curveto(139.75519536,343.55456203)(139.73519538,343.64956193)(139.71519607,343.76956697)
\curveto(139.69519542,343.83956174)(139.68019543,343.92456166)(139.67019607,344.02456697)
\curveto(139.66019545,344.12456146)(139.66519545,344.21456137)(139.68519607,344.29456697)
\curveto(139.70519541,344.34456124)(139.7101954,344.3845612)(139.70019607,344.41456697)
\curveto(139.69019542,344.45456113)(139.69519542,344.49956108)(139.71519607,344.54956697)
\curveto(139.73519538,344.65956092)(139.75519536,344.75956082)(139.77519607,344.84956697)
\curveto(139.80519531,344.94956063)(139.84019527,345.04456054)(139.88019607,345.13456697)
\curveto(140.0101951,345.42456016)(140.19019492,345.65955992)(140.42019607,345.83956697)
\curveto(140.65019446,346.01955956)(140.9101942,346.16455942)(141.20019607,346.27456697)
\curveto(141.3101938,346.32455926)(141.42519369,346.35955922)(141.54519607,346.37956697)
\curveto(141.66519345,346.40955917)(141.79019332,346.43955914)(141.92019607,346.46956697)
\curveto(141.98019313,346.48955909)(142.04019307,346.49955908)(142.10019607,346.49956697)
\lineto(142.28019607,346.52956697)
\curveto(142.36019275,346.53955904)(142.44519267,346.54455904)(142.53519607,346.54456697)
\curveto(142.62519249,346.54455904)(142.7101924,346.54955903)(142.79019607,346.55956697)
}
}
{
\newrgbcolor{curcolor}{0 0 0}
\pscustom[linestyle=none,fillstyle=solid,fillcolor=curcolor]
{
\newpath
\moveto(155.7668367,342.56956697)
\curveto(155.77682802,342.50956307)(155.78182801,342.41956316)(155.7818367,342.29956697)
\curveto(155.78182801,342.1795634)(155.77182802,342.09456349)(155.7518367,342.04456697)
\lineto(155.7518367,341.84956697)
\curveto(155.72182807,341.73956384)(155.70182809,341.63456395)(155.6918367,341.53456697)
\curveto(155.6918281,341.43456415)(155.67682812,341.33456425)(155.6468367,341.23456697)
\curveto(155.62682817,341.14456444)(155.60682819,341.04956453)(155.5868367,340.94956697)
\curveto(155.56682823,340.85956472)(155.53682826,340.76956481)(155.4968367,340.67956697)
\curveto(155.42682837,340.50956507)(155.35682844,340.34956523)(155.2868367,340.19956697)
\curveto(155.21682858,340.05956552)(155.13682866,339.91956566)(155.0468367,339.77956697)
\curveto(154.98682881,339.68956589)(154.92182887,339.60456598)(154.8518367,339.52456697)
\curveto(154.791829,339.45456613)(154.72182907,339.3795662)(154.6418367,339.29956697)
\lineto(154.5368367,339.19456697)
\curveto(154.48682931,339.14456644)(154.43182936,339.09956648)(154.3718367,339.05956697)
\lineto(154.2218367,338.93956697)
\curveto(154.14182965,338.8795667)(154.05182974,338.82456676)(153.9518367,338.77456697)
\curveto(153.86182993,338.73456685)(153.76683003,338.68956689)(153.6668367,338.63956697)
\curveto(153.56683023,338.58956699)(153.46183033,338.55456703)(153.3518367,338.53456697)
\curveto(153.25183054,338.51456707)(153.14683065,338.49456709)(153.0368367,338.47456697)
\curveto(152.97683082,338.45456713)(152.91183088,338.44456714)(152.8418367,338.44456697)
\curveto(152.78183101,338.44456714)(152.71683108,338.43456715)(152.6468367,338.41456697)
\lineto(152.5118367,338.41456697)
\curveto(152.43183136,338.39456719)(152.35683144,338.39456719)(152.2868367,338.41456697)
\lineto(152.1368367,338.41456697)
\curveto(152.07683172,338.43456715)(152.01183178,338.44456714)(151.9418367,338.44456697)
\curveto(151.88183191,338.43456715)(151.82183197,338.43956714)(151.7618367,338.45956697)
\curveto(151.60183219,338.50956707)(151.44683235,338.55456703)(151.2968367,338.59456697)
\curveto(151.15683264,338.63456695)(151.02683277,338.69456689)(150.9068367,338.77456697)
\curveto(150.83683296,338.81456677)(150.77183302,338.85456673)(150.7118367,338.89456697)
\curveto(150.65183314,338.94456664)(150.58683321,338.99456659)(150.5168367,339.04456697)
\lineto(150.3368367,339.17956697)
\curveto(150.25683354,339.23956634)(150.18683361,339.24456634)(150.1268367,339.19456697)
\curveto(150.07683372,339.16456642)(150.05183374,339.12456646)(150.0518367,339.07456697)
\curveto(150.05183374,339.03456655)(150.04183375,338.9845666)(150.0218367,338.92456697)
\curveto(150.00183379,338.82456676)(149.9918338,338.70956687)(149.9918367,338.57956697)
\curveto(150.00183379,338.44956713)(150.00683379,338.32956725)(150.0068367,338.21956697)
\lineto(150.0068367,336.68956697)
\curveto(150.00683379,336.55956902)(150.00183379,336.43456915)(149.9918367,336.31456697)
\curveto(149.9918338,336.1845694)(149.96683383,336.0795695)(149.9168367,335.99956697)
\curveto(149.88683391,335.95956962)(149.83183396,335.92956965)(149.7518367,335.90956697)
\curveto(149.67183412,335.88956969)(149.58183421,335.8795697)(149.4818367,335.87956697)
\curveto(149.38183441,335.86956971)(149.28183451,335.86956971)(149.1818367,335.87956697)
\lineto(148.9268367,335.87956697)
\lineto(148.5218367,335.87956697)
\lineto(148.4168367,335.87956697)
\curveto(148.37683542,335.8795697)(148.34183545,335.8845697)(148.3118367,335.89456697)
\lineto(148.1918367,335.89456697)
\curveto(148.02183577,335.94456964)(147.93183586,336.04456954)(147.9218367,336.19456697)
\curveto(147.91183588,336.33456925)(147.90683589,336.50456908)(147.9068367,336.70456697)
\lineto(147.9068367,345.50956697)
\curveto(147.90683589,345.61955996)(147.90183589,345.73455985)(147.8918367,345.85456697)
\curveto(147.8918359,345.9845596)(147.91683588,346.0845595)(147.9668367,346.15456697)
\curveto(148.00683579,346.22455936)(148.06183573,346.26955931)(148.1318367,346.28956697)
\curveto(148.18183561,346.30955927)(148.24183555,346.31955926)(148.3118367,346.31956697)
\lineto(148.5368367,346.31956697)
\lineto(149.2568367,346.31956697)
\lineto(149.5418367,346.31956697)
\curveto(149.63183416,346.31955926)(149.70683409,346.29455929)(149.7668367,346.24456697)
\curveto(149.83683396,346.19455939)(149.87183392,346.12955945)(149.8718367,346.04956697)
\curveto(149.88183391,345.9795596)(149.90683389,345.90455968)(149.9468367,345.82456697)
\curveto(149.95683384,345.79455979)(149.96683383,345.76955981)(149.9768367,345.74956697)
\curveto(149.9968338,345.73955984)(150.01683378,345.72455986)(150.0368367,345.70456697)
\curveto(150.14683365,345.69455989)(150.23683356,345.72455986)(150.3068367,345.79456697)
\curveto(150.37683342,345.86455972)(150.44683335,345.92455966)(150.5168367,345.97456697)
\curveto(150.64683315,346.06455952)(150.78183301,346.14455944)(150.9218367,346.21456697)
\curveto(151.06183273,346.29455929)(151.21683258,346.35955922)(151.3868367,346.40956697)
\curveto(151.46683233,346.43955914)(151.55183224,346.45955912)(151.6418367,346.46956697)
\curveto(151.74183205,346.4795591)(151.83683196,346.49455909)(151.9268367,346.51456697)
\curveto(151.96683183,346.52455906)(152.00683179,346.52455906)(152.0468367,346.51456697)
\curveto(152.0968317,346.50455908)(152.13683166,346.50955907)(152.1668367,346.52956697)
\curveto(152.73683106,346.54955903)(153.21683058,346.46955911)(153.6068367,346.28956697)
\curveto(154.00682979,346.11955946)(154.34682945,345.89455969)(154.6268367,345.61456697)
\curveto(154.67682912,345.56456002)(154.72182907,345.51456007)(154.7618367,345.46456697)
\curveto(154.80182899,345.42456016)(154.84182895,345.3795602)(154.8818367,345.32956697)
\curveto(154.95182884,345.23956034)(155.01182878,345.14956043)(155.0618367,345.05956697)
\curveto(155.12182867,344.96956061)(155.17682862,344.8795607)(155.2268367,344.78956697)
\curveto(155.24682855,344.76956081)(155.25682854,344.74456084)(155.2568367,344.71456697)
\curveto(155.26682853,344.6845609)(155.28182851,344.64956093)(155.3018367,344.60956697)
\curveto(155.36182843,344.50956107)(155.41682838,344.38956119)(155.4668367,344.24956697)
\curveto(155.48682831,344.18956139)(155.50682829,344.12456146)(155.5268367,344.05456697)
\curveto(155.54682825,343.99456159)(155.56682823,343.92956165)(155.5868367,343.85956697)
\curveto(155.62682817,343.73956184)(155.65182814,343.61456197)(155.6618367,343.48456697)
\curveto(155.68182811,343.35456223)(155.70682809,343.21956236)(155.7368367,343.07956697)
\lineto(155.7368367,342.91456697)
\lineto(155.7668367,342.73456697)
\lineto(155.7668367,342.56956697)
\moveto(153.6518367,342.22456697)
\curveto(153.66183013,342.27456331)(153.66683013,342.33956324)(153.6668367,342.41956697)
\curveto(153.66683013,342.50956307)(153.66183013,342.579563)(153.6518367,342.62956697)
\lineto(153.6518367,342.76456697)
\curveto(153.63183016,342.82456276)(153.62183017,342.88956269)(153.6218367,342.95956697)
\curveto(153.62183017,343.02956255)(153.61183018,343.09956248)(153.5918367,343.16956697)
\curveto(153.57183022,343.26956231)(153.55183024,343.36456222)(153.5318367,343.45456697)
\curveto(153.51183028,343.55456203)(153.48183031,343.64456194)(153.4418367,343.72456697)
\curveto(153.32183047,344.04456154)(153.16683063,344.29956128)(152.9768367,344.48956697)
\curveto(152.78683101,344.6795609)(152.51683128,344.81956076)(152.1668367,344.90956697)
\curveto(152.08683171,344.92956065)(151.9968318,344.93956064)(151.8968367,344.93956697)
\lineto(151.6268367,344.93956697)
\curveto(151.58683221,344.92956065)(151.55183224,344.92456066)(151.5218367,344.92456697)
\curveto(151.4918323,344.92456066)(151.45683234,344.91956066)(151.4168367,344.90956697)
\lineto(151.2068367,344.84956697)
\curveto(151.14683265,344.83956074)(151.08683271,344.81956076)(151.0268367,344.78956697)
\curveto(150.76683303,344.6795609)(150.56183323,344.50956107)(150.4118367,344.27956697)
\curveto(150.27183352,344.04956153)(150.15683364,343.79456179)(150.0668367,343.51456697)
\curveto(150.04683375,343.43456215)(150.03183376,343.34956223)(150.0218367,343.25956697)
\curveto(150.01183378,343.1795624)(149.9968338,343.09956248)(149.9768367,343.01956697)
\curveto(149.96683383,342.9795626)(149.96183383,342.91456267)(149.9618367,342.82456697)
\curveto(149.94183385,342.7845628)(149.93683386,342.73456285)(149.9468367,342.67456697)
\curveto(149.95683384,342.62456296)(149.95683384,342.57456301)(149.9468367,342.52456697)
\curveto(149.92683387,342.46456312)(149.92683387,342.40956317)(149.9468367,342.35956697)
\lineto(149.9468367,342.17956697)
\lineto(149.9468367,342.04456697)
\curveto(149.94683385,342.00456358)(149.95683384,341.96456362)(149.9768367,341.92456697)
\curveto(149.97683382,341.85456373)(149.98183381,341.79956378)(149.9918367,341.75956697)
\lineto(150.0218367,341.57956697)
\curveto(150.03183376,341.51956406)(150.04683375,341.45956412)(150.0668367,341.39956697)
\curveto(150.15683364,341.10956447)(150.26183353,340.86956471)(150.3818367,340.67956697)
\curveto(150.51183328,340.49956508)(150.6918331,340.33956524)(150.9218367,340.19956697)
\curveto(151.06183273,340.11956546)(151.22683257,340.05456553)(151.4168367,340.00456697)
\curveto(151.45683234,339.99456559)(151.4918323,339.98956559)(151.5218367,339.98956697)
\curveto(151.55183224,339.99956558)(151.58683221,339.99956558)(151.6268367,339.98956697)
\curveto(151.66683213,339.9795656)(151.72683207,339.96956561)(151.8068367,339.95956697)
\curveto(151.88683191,339.95956562)(151.95183184,339.96456562)(152.0018367,339.97456697)
\curveto(152.08183171,339.99456559)(152.16183163,340.00956557)(152.2418367,340.01956697)
\curveto(152.33183146,340.03956554)(152.41683138,340.06456552)(152.4968367,340.09456697)
\curveto(152.73683106,340.19456539)(152.93183086,340.33456525)(153.0818367,340.51456697)
\curveto(153.23183056,340.69456489)(153.35683044,340.90456468)(153.4568367,341.14456697)
\curveto(153.50683029,341.26456432)(153.54183025,341.38956419)(153.5618367,341.51956697)
\curveto(153.58183021,341.64956393)(153.60683019,341.7845638)(153.6368367,341.92456697)
\lineto(153.6368367,342.07456697)
\curveto(153.64683015,342.12456346)(153.65183014,342.17456341)(153.6518367,342.22456697)
}
}
{
\newrgbcolor{curcolor}{0 0 0}
\pscustom[linestyle=none,fillstyle=solid,fillcolor=curcolor]
{
\newpath
\moveto(164.09675857,339.20956697)
\curveto(164.11675072,339.09956648)(164.12675071,338.98956659)(164.12675857,338.87956697)
\curveto(164.1367507,338.76956681)(164.08675075,338.69456689)(163.97675857,338.65456697)
\curveto(163.91675092,338.62456696)(163.84675099,338.60956697)(163.76675857,338.60956697)
\lineto(163.52675857,338.60956697)
\lineto(162.71675857,338.60956697)
\lineto(162.44675857,338.60956697)
\curveto(162.36675247,338.61956696)(162.30175254,338.64456694)(162.25175857,338.68456697)
\curveto(162.18175266,338.72456686)(162.12675271,338.7795668)(162.08675857,338.84956697)
\curveto(162.05675278,338.92956665)(162.01175283,338.99456659)(161.95175857,339.04456697)
\curveto(161.93175291,339.06456652)(161.90675293,339.0795665)(161.87675857,339.08956697)
\curveto(161.84675299,339.10956647)(161.80675303,339.11456647)(161.75675857,339.10456697)
\curveto(161.70675313,339.0845665)(161.65675318,339.05956652)(161.60675857,339.02956697)
\curveto(161.56675327,338.99956658)(161.52175332,338.97456661)(161.47175857,338.95456697)
\curveto(161.42175342,338.91456667)(161.36675347,338.8795667)(161.30675857,338.84956697)
\lineto(161.12675857,338.75956697)
\curveto(160.99675384,338.69956688)(160.86175398,338.64956693)(160.72175857,338.60956697)
\curveto(160.58175426,338.579567)(160.4367544,338.54456704)(160.28675857,338.50456697)
\curveto(160.21675462,338.4845671)(160.14675469,338.47456711)(160.07675857,338.47456697)
\curveto(160.01675482,338.46456712)(159.95175489,338.45456713)(159.88175857,338.44456697)
\lineto(159.79175857,338.44456697)
\curveto(159.76175508,338.43456715)(159.73175511,338.42956715)(159.70175857,338.42956697)
\lineto(159.53675857,338.42956697)
\curveto(159.4367554,338.40956717)(159.3367555,338.40956717)(159.23675857,338.42956697)
\lineto(159.10175857,338.42956697)
\curveto(159.03175581,338.44956713)(158.96175588,338.45956712)(158.89175857,338.45956697)
\curveto(158.83175601,338.44956713)(158.77175607,338.45456713)(158.71175857,338.47456697)
\curveto(158.61175623,338.49456709)(158.51675632,338.51456707)(158.42675857,338.53456697)
\curveto(158.3367565,338.54456704)(158.25175659,338.56956701)(158.17175857,338.60956697)
\curveto(157.88175696,338.71956686)(157.63175721,338.85956672)(157.42175857,339.02956697)
\curveto(157.22175762,339.20956637)(157.06175778,339.44456614)(156.94175857,339.73456697)
\curveto(156.91175793,339.80456578)(156.88175796,339.8795657)(156.85175857,339.95956697)
\curveto(156.83175801,340.03956554)(156.81175803,340.12456546)(156.79175857,340.21456697)
\curveto(156.77175807,340.26456532)(156.76175808,340.31456527)(156.76175857,340.36456697)
\curveto(156.77175807,340.41456517)(156.77175807,340.46456512)(156.76175857,340.51456697)
\curveto(156.75175809,340.54456504)(156.7417581,340.60456498)(156.73175857,340.69456697)
\curveto(156.73175811,340.79456479)(156.7367581,340.86456472)(156.74675857,340.90456697)
\curveto(156.76675807,341.00456458)(156.77675806,341.08956449)(156.77675857,341.15956697)
\lineto(156.86675857,341.48956697)
\curveto(156.89675794,341.60956397)(156.9367579,341.71456387)(156.98675857,341.80456697)
\curveto(157.15675768,342.09456349)(157.35175749,342.31456327)(157.57175857,342.46456697)
\curveto(157.79175705,342.61456297)(158.07175677,342.74456284)(158.41175857,342.85456697)
\curveto(158.5417563,342.90456268)(158.67675616,342.93956264)(158.81675857,342.95956697)
\curveto(158.95675588,342.9795626)(159.09675574,343.00456258)(159.23675857,343.03456697)
\curveto(159.31675552,343.05456253)(159.40175544,343.06456252)(159.49175857,343.06456697)
\curveto(159.58175526,343.07456251)(159.67175517,343.08956249)(159.76175857,343.10956697)
\curveto(159.83175501,343.12956245)(159.90175494,343.13456245)(159.97175857,343.12456697)
\curveto(160.0417548,343.12456246)(160.11675472,343.13456245)(160.19675857,343.15456697)
\curveto(160.26675457,343.17456241)(160.3367545,343.1845624)(160.40675857,343.18456697)
\curveto(160.47675436,343.1845624)(160.55175429,343.19456239)(160.63175857,343.21456697)
\curveto(160.841754,343.26456232)(161.03175381,343.30456228)(161.20175857,343.33456697)
\curveto(161.38175346,343.37456221)(161.5417533,343.46456212)(161.68175857,343.60456697)
\curveto(161.77175307,343.69456189)(161.83175301,343.79456179)(161.86175857,343.90456697)
\curveto(161.87175297,343.93456165)(161.87175297,343.95956162)(161.86175857,343.97956697)
\curveto(161.86175298,343.99956158)(161.86675297,344.01956156)(161.87675857,344.03956697)
\curveto(161.88675295,344.05956152)(161.89175295,344.08956149)(161.89175857,344.12956697)
\lineto(161.89175857,344.21956697)
\lineto(161.86175857,344.33956697)
\curveto(161.86175298,344.3795612)(161.85675298,344.41456117)(161.84675857,344.44456697)
\curveto(161.74675309,344.74456084)(161.5367533,344.94956063)(161.21675857,345.05956697)
\curveto(161.12675371,345.08956049)(161.01675382,345.10956047)(160.88675857,345.11956697)
\curveto(160.76675407,345.13956044)(160.6417542,345.14456044)(160.51175857,345.13456697)
\curveto(160.38175446,345.13456045)(160.25675458,345.12456046)(160.13675857,345.10456697)
\curveto(160.01675482,345.0845605)(159.91175493,345.05956052)(159.82175857,345.02956697)
\curveto(159.76175508,345.00956057)(159.70175514,344.9795606)(159.64175857,344.93956697)
\curveto(159.59175525,344.90956067)(159.5417553,344.87456071)(159.49175857,344.83456697)
\curveto(159.4417554,344.79456079)(159.38675545,344.73956084)(159.32675857,344.66956697)
\curveto(159.27675556,344.59956098)(159.2417556,344.53456105)(159.22175857,344.47456697)
\curveto(159.17175567,344.37456121)(159.12675571,344.2795613)(159.08675857,344.18956697)
\curveto(159.05675578,344.09956148)(158.98675585,344.03956154)(158.87675857,344.00956697)
\curveto(158.79675604,343.98956159)(158.71175613,343.9795616)(158.62175857,343.97956697)
\lineto(158.35175857,343.97956697)
\lineto(157.78175857,343.97956697)
\curveto(157.73175711,343.9795616)(157.68175716,343.97456161)(157.63175857,343.96456697)
\curveto(157.58175726,343.96456162)(157.5367573,343.96956161)(157.49675857,343.97956697)
\lineto(157.36175857,343.97956697)
\curveto(157.3417575,343.98956159)(157.31675752,343.99456159)(157.28675857,343.99456697)
\curveto(157.25675758,343.99456159)(157.23175761,344.00456158)(157.21175857,344.02456697)
\curveto(157.13175771,344.04456154)(157.07675776,344.10956147)(157.04675857,344.21956697)
\curveto(157.0367578,344.26956131)(157.0367578,344.31956126)(157.04675857,344.36956697)
\curveto(157.05675778,344.41956116)(157.06675777,344.46456112)(157.07675857,344.50456697)
\curveto(157.10675773,344.61456097)(157.1367577,344.71456087)(157.16675857,344.80456697)
\curveto(157.20675763,344.90456068)(157.25175759,344.99456059)(157.30175857,345.07456697)
\lineto(157.39175857,345.22456697)
\lineto(157.48175857,345.37456697)
\curveto(157.56175728,345.4845601)(157.66175718,345.58955999)(157.78175857,345.68956697)
\curveto(157.80175704,345.69955988)(157.83175701,345.72455986)(157.87175857,345.76456697)
\curveto(157.92175692,345.80455978)(157.96675687,345.83955974)(158.00675857,345.86956697)
\curveto(158.04675679,345.89955968)(158.09175675,345.92955965)(158.14175857,345.95956697)
\curveto(158.31175653,346.06955951)(158.49175635,346.15455943)(158.68175857,346.21456697)
\curveto(158.87175597,346.2845593)(159.06675577,346.34955923)(159.26675857,346.40956697)
\curveto(159.38675545,346.43955914)(159.51175533,346.45955912)(159.64175857,346.46956697)
\curveto(159.77175507,346.4795591)(159.90175494,346.49955908)(160.03175857,346.52956697)
\curveto(160.07175477,346.53955904)(160.13175471,346.53955904)(160.21175857,346.52956697)
\curveto(160.30175454,346.51955906)(160.35675448,346.52455906)(160.37675857,346.54456697)
\curveto(160.78675405,346.55455903)(161.17675366,346.53955904)(161.54675857,346.49956697)
\curveto(161.92675291,346.45955912)(162.26675257,346.3845592)(162.56675857,346.27456697)
\curveto(162.87675196,346.16455942)(163.1417517,346.01455957)(163.36175857,345.82456697)
\curveto(163.58175126,345.64455994)(163.75175109,345.40956017)(163.87175857,345.11956697)
\curveto(163.9417509,344.94956063)(163.98175086,344.75456083)(163.99175857,344.53456697)
\curveto(164.00175084,344.31456127)(164.00675083,344.08956149)(164.00675857,343.85956697)
\lineto(164.00675857,340.51456697)
\lineto(164.00675857,339.92956697)
\curveto(164.00675083,339.73956584)(164.02675081,339.56456602)(164.06675857,339.40456697)
\curveto(164.07675076,339.37456621)(164.08175076,339.33956624)(164.08175857,339.29956697)
\curveto(164.08175076,339.26956631)(164.08675075,339.23956634)(164.09675857,339.20956697)
\moveto(161.89175857,341.51956697)
\curveto(161.90175294,341.56956401)(161.90675293,341.62456396)(161.90675857,341.68456697)
\curveto(161.90675293,341.75456383)(161.90175294,341.81456377)(161.89175857,341.86456697)
\curveto(161.87175297,341.92456366)(161.86175298,341.9795636)(161.86175857,342.02956697)
\curveto(161.86175298,342.0795635)(161.841753,342.11956346)(161.80175857,342.14956697)
\curveto(161.75175309,342.18956339)(161.67675316,342.20956337)(161.57675857,342.20956697)
\curveto(161.5367533,342.19956338)(161.50175334,342.18956339)(161.47175857,342.17956697)
\curveto(161.4417534,342.1795634)(161.40675343,342.17456341)(161.36675857,342.16456697)
\curveto(161.29675354,342.14456344)(161.22175362,342.12956345)(161.14175857,342.11956697)
\curveto(161.06175378,342.10956347)(160.98175386,342.09456349)(160.90175857,342.07456697)
\curveto(160.87175397,342.06456352)(160.82675401,342.05956352)(160.76675857,342.05956697)
\curveto(160.6367542,342.02956355)(160.50675433,342.00956357)(160.37675857,341.99956697)
\curveto(160.24675459,341.98956359)(160.12175472,341.96456362)(160.00175857,341.92456697)
\curveto(159.92175492,341.90456368)(159.84675499,341.8845637)(159.77675857,341.86456697)
\curveto(159.70675513,341.85456373)(159.6367552,341.83456375)(159.56675857,341.80456697)
\curveto(159.35675548,341.71456387)(159.17675566,341.579564)(159.02675857,341.39956697)
\curveto(158.88675595,341.21956436)(158.836756,340.96956461)(158.87675857,340.64956697)
\curveto(158.89675594,340.4795651)(158.95175589,340.33956524)(159.04175857,340.22956697)
\curveto(159.11175573,340.11956546)(159.21675562,340.02956555)(159.35675857,339.95956697)
\curveto(159.49675534,339.89956568)(159.64675519,339.85456573)(159.80675857,339.82456697)
\curveto(159.97675486,339.79456579)(160.15175469,339.7845658)(160.33175857,339.79456697)
\curveto(160.52175432,339.81456577)(160.69675414,339.84956573)(160.85675857,339.89956697)
\curveto(161.11675372,339.9795656)(161.32175352,340.10456548)(161.47175857,340.27456697)
\curveto(161.62175322,340.45456513)(161.7367531,340.67456491)(161.81675857,340.93456697)
\curveto(161.836753,341.00456458)(161.84675299,341.07456451)(161.84675857,341.14456697)
\curveto(161.85675298,341.22456436)(161.87175297,341.30456428)(161.89175857,341.38456697)
\lineto(161.89175857,341.51956697)
}
}
{
\newrgbcolor{curcolor}{0 0 0}
\pscustom[linestyle=none,fillstyle=solid,fillcolor=curcolor]
{
\newpath
\moveto(169.23003982,346.55956697)
\curveto(170.04003466,346.579559)(170.71503399,346.45955912)(171.25503982,346.19956697)
\curveto(171.8050329,345.93955964)(172.24003246,345.56956001)(172.56003982,345.08956697)
\curveto(172.72003198,344.84956073)(172.84003186,344.57456101)(172.92003982,344.26456697)
\curveto(172.94003176,344.21456137)(172.95503175,344.14956143)(172.96503982,344.06956697)
\curveto(172.98503172,343.98956159)(172.98503172,343.91956166)(172.96503982,343.85956697)
\curveto(172.92503178,343.74956183)(172.85503185,343.6845619)(172.75503982,343.66456697)
\curveto(172.65503205,343.65456193)(172.53503217,343.64956193)(172.39503982,343.64956697)
\lineto(171.61503982,343.64956697)
\lineto(171.33003982,343.64956697)
\curveto(171.24003346,343.64956193)(171.16503354,343.66956191)(171.10503982,343.70956697)
\curveto(171.02503368,343.74956183)(170.97003373,343.80956177)(170.94003982,343.88956697)
\curveto(170.91003379,343.9795616)(170.87003383,344.06956151)(170.82003982,344.15956697)
\curveto(170.76003394,344.26956131)(170.69503401,344.36956121)(170.62503982,344.45956697)
\curveto(170.55503415,344.54956103)(170.47503423,344.62956095)(170.38503982,344.69956697)
\curveto(170.24503446,344.78956079)(170.09003461,344.85956072)(169.92003982,344.90956697)
\curveto(169.86003484,344.92956065)(169.8000349,344.93956064)(169.74003982,344.93956697)
\curveto(169.68003502,344.93956064)(169.62503508,344.94956063)(169.57503982,344.96956697)
\lineto(169.42503982,344.96956697)
\curveto(169.22503548,344.96956061)(169.06503564,344.94956063)(168.94503982,344.90956697)
\curveto(168.65503605,344.81956076)(168.42003628,344.6795609)(168.24003982,344.48956697)
\curveto(168.06003664,344.30956127)(167.91503679,344.08956149)(167.80503982,343.82956697)
\curveto(167.75503695,343.71956186)(167.71503699,343.59956198)(167.68503982,343.46956697)
\curveto(167.66503704,343.34956223)(167.64003706,343.21956236)(167.61003982,343.07956697)
\curveto(167.6000371,343.03956254)(167.59503711,342.99956258)(167.59503982,342.95956697)
\curveto(167.59503711,342.91956266)(167.59003711,342.8795627)(167.58003982,342.83956697)
\curveto(167.56003714,342.73956284)(167.55003715,342.59956298)(167.55003982,342.41956697)
\curveto(167.56003714,342.23956334)(167.57503713,342.09956348)(167.59503982,341.99956697)
\curveto(167.59503711,341.91956366)(167.6000371,341.86456372)(167.61003982,341.83456697)
\curveto(167.63003707,341.76456382)(167.64003706,341.69456389)(167.64003982,341.62456697)
\curveto(167.65003705,341.55456403)(167.66503704,341.4845641)(167.68503982,341.41456697)
\curveto(167.76503694,341.1845644)(167.86003684,340.97456461)(167.97003982,340.78456697)
\curveto(168.08003662,340.59456499)(168.22003648,340.43456515)(168.39003982,340.30456697)
\curveto(168.43003627,340.27456531)(168.49003621,340.23956534)(168.57003982,340.19956697)
\curveto(168.68003602,340.12956545)(168.79003591,340.0845655)(168.90003982,340.06456697)
\curveto(169.02003568,340.04456554)(169.16503554,340.02456556)(169.33503982,340.00456697)
\lineto(169.42503982,340.00456697)
\curveto(169.46503524,340.00456558)(169.49503521,340.00956557)(169.51503982,340.01956697)
\lineto(169.65003982,340.01956697)
\curveto(169.72003498,340.03956554)(169.78503492,340.05456553)(169.84503982,340.06456697)
\curveto(169.91503479,340.0845655)(169.98003472,340.10456548)(170.04003982,340.12456697)
\curveto(170.34003436,340.25456533)(170.57003413,340.44456514)(170.73003982,340.69456697)
\curveto(170.77003393,340.74456484)(170.8050339,340.79956478)(170.83503982,340.85956697)
\curveto(170.86503384,340.92956465)(170.89003381,340.98956459)(170.91003982,341.03956697)
\curveto(170.95003375,341.14956443)(170.98503372,341.24456434)(171.01503982,341.32456697)
\curveto(171.04503366,341.41456417)(171.11503359,341.4845641)(171.22503982,341.53456697)
\curveto(171.31503339,341.57456401)(171.46003324,341.58956399)(171.66003982,341.57956697)
\lineto(172.15503982,341.57956697)
\lineto(172.36503982,341.57956697)
\curveto(172.44503226,341.58956399)(172.51003219,341.584564)(172.56003982,341.56456697)
\lineto(172.68003982,341.56456697)
\lineto(172.80003982,341.53456697)
\curveto(172.84003186,341.53456405)(172.87003183,341.52456406)(172.89003982,341.50456697)
\curveto(172.94003176,341.46456412)(172.97003173,341.40456418)(172.98003982,341.32456697)
\curveto(173.0000317,341.25456433)(173.0000317,341.1795644)(172.98003982,341.09956697)
\curveto(172.89003181,340.76956481)(172.78003192,340.47456511)(172.65003982,340.21456697)
\curveto(172.24003246,339.44456614)(171.58503312,338.90956667)(170.68503982,338.60956697)
\curveto(170.58503412,338.579567)(170.48003422,338.55956702)(170.37003982,338.54956697)
\curveto(170.26003444,338.52956705)(170.15003455,338.50456708)(170.04003982,338.47456697)
\curveto(169.98003472,338.46456712)(169.92003478,338.45956712)(169.86003982,338.45956697)
\curveto(169.8000349,338.45956712)(169.74003496,338.45456713)(169.68003982,338.44456697)
\lineto(169.51503982,338.44456697)
\curveto(169.46503524,338.42456716)(169.39003531,338.41956716)(169.29003982,338.42956697)
\curveto(169.19003551,338.42956715)(169.11503559,338.43456715)(169.06503982,338.44456697)
\curveto(168.98503572,338.46456712)(168.91003579,338.47456711)(168.84003982,338.47456697)
\curveto(168.78003592,338.46456712)(168.71503599,338.46956711)(168.64503982,338.48956697)
\lineto(168.49503982,338.51956697)
\curveto(168.44503626,338.51956706)(168.39503631,338.52456706)(168.34503982,338.53456697)
\curveto(168.23503647,338.56456702)(168.13003657,338.59456699)(168.03003982,338.62456697)
\curveto(167.93003677,338.65456693)(167.83503687,338.68956689)(167.74503982,338.72956697)
\curveto(167.27503743,338.92956665)(166.88003782,339.1845664)(166.56003982,339.49456697)
\curveto(166.24003846,339.81456577)(165.98003872,340.20956537)(165.78003982,340.67956697)
\curveto(165.73003897,340.76956481)(165.69003901,340.86456472)(165.66003982,340.96456697)
\lineto(165.57003982,341.29456697)
\curveto(165.56003914,341.33456425)(165.55503915,341.36956421)(165.55503982,341.39956697)
\curveto(165.55503915,341.43956414)(165.54503916,341.4845641)(165.52503982,341.53456697)
\curveto(165.5050392,341.60456398)(165.49503921,341.67456391)(165.49503982,341.74456697)
\curveto(165.49503921,341.82456376)(165.48503922,341.89956368)(165.46503982,341.96956697)
\lineto(165.46503982,342.22456697)
\curveto(165.44503926,342.27456331)(165.43503927,342.32956325)(165.43503982,342.38956697)
\curveto(165.43503927,342.45956312)(165.44503926,342.51956306)(165.46503982,342.56956697)
\curveto(165.47503923,342.61956296)(165.47503923,342.66456292)(165.46503982,342.70456697)
\curveto(165.45503925,342.74456284)(165.45503925,342.7845628)(165.46503982,342.82456697)
\curveto(165.48503922,342.89456269)(165.49003921,342.95956262)(165.48003982,343.01956697)
\curveto(165.48003922,343.0795625)(165.49003921,343.13956244)(165.51003982,343.19956697)
\curveto(165.56003914,343.3795622)(165.6000391,343.54956203)(165.63003982,343.70956697)
\curveto(165.66003904,343.8795617)(165.705039,344.04456154)(165.76503982,344.20456697)
\curveto(165.98503872,344.71456087)(166.26003844,345.13956044)(166.59003982,345.47956697)
\curveto(166.93003777,345.81955976)(167.36003734,346.09455949)(167.88003982,346.30456697)
\curveto(168.02003668,346.36455922)(168.16503654,346.40455918)(168.31503982,346.42456697)
\curveto(168.46503624,346.45455913)(168.62003608,346.48955909)(168.78003982,346.52956697)
\curveto(168.86003584,346.53955904)(168.93503577,346.54455904)(169.00503982,346.54456697)
\curveto(169.07503563,346.54455904)(169.15003555,346.54955903)(169.23003982,346.55956697)
}
}
{
\newrgbcolor{curcolor}{0 0 0}
\pscustom[linestyle=none,fillstyle=solid,fillcolor=curcolor]
{
\newpath
\moveto(176.37332107,349.19956697)
\curveto(176.44331812,349.11955646)(176.47831809,348.99955658)(176.47832107,348.83956697)
\lineto(176.47832107,348.37456697)
\lineto(176.47832107,347.96956697)
\curveto(176.47831809,347.82955775)(176.44331812,347.73455785)(176.37332107,347.68456697)
\curveto(176.31331825,347.63455795)(176.23331833,347.60455798)(176.13332107,347.59456697)
\curveto(176.04331852,347.584558)(175.94331862,347.579558)(175.83332107,347.57956697)
\lineto(174.99332107,347.57956697)
\curveto(174.88331968,347.579558)(174.78331978,347.584558)(174.69332107,347.59456697)
\curveto(174.61331995,347.60455798)(174.54332002,347.63455795)(174.48332107,347.68456697)
\curveto(174.44332012,347.71455787)(174.41332015,347.76955781)(174.39332107,347.84956697)
\curveto(174.38332018,347.93955764)(174.37332019,348.03455755)(174.36332107,348.13456697)
\lineto(174.36332107,348.46456697)
\curveto(174.37332019,348.57455701)(174.37832019,348.66955691)(174.37832107,348.74956697)
\lineto(174.37832107,348.95956697)
\curveto(174.38832018,349.02955655)(174.40832016,349.08955649)(174.43832107,349.13956697)
\curveto(174.45832011,349.1795564)(174.48332008,349.20955637)(174.51332107,349.22956697)
\lineto(174.63332107,349.28956697)
\curveto(174.65331991,349.28955629)(174.67831989,349.28955629)(174.70832107,349.28956697)
\curveto(174.73831983,349.29955628)(174.7633198,349.30455628)(174.78332107,349.30456697)
\lineto(175.87832107,349.30456697)
\curveto(175.97831859,349.30455628)(176.07331849,349.29955628)(176.16332107,349.28956697)
\curveto(176.25331831,349.2795563)(176.32331824,349.24955633)(176.37332107,349.19956697)
\moveto(176.47832107,339.43456697)
\curveto(176.47831809,339.23456635)(176.47331809,339.06456652)(176.46332107,338.92456697)
\curveto(176.45331811,338.7845668)(176.3633182,338.68956689)(176.19332107,338.63956697)
\curveto(176.13331843,338.61956696)(176.0683185,338.60956697)(175.99832107,338.60956697)
\curveto(175.92831864,338.61956696)(175.85331871,338.62456696)(175.77332107,338.62456697)
\lineto(174.93332107,338.62456697)
\curveto(174.84331972,338.62456696)(174.75331981,338.62956695)(174.66332107,338.63956697)
\curveto(174.58331998,338.64956693)(174.52332004,338.6795669)(174.48332107,338.72956697)
\curveto(174.42332014,338.79956678)(174.38832018,338.8845667)(174.37832107,338.98456697)
\lineto(174.37832107,339.32956697)
\lineto(174.37832107,345.65956697)
\lineto(174.37832107,345.95956697)
\curveto(174.37832019,346.05955952)(174.39832017,346.13955944)(174.43832107,346.19956697)
\curveto(174.49832007,346.26955931)(174.58331998,346.31455927)(174.69332107,346.33456697)
\curveto(174.71331985,346.34455924)(174.73831983,346.34455924)(174.76832107,346.33456697)
\curveto(174.80831976,346.33455925)(174.83831973,346.33955924)(174.85832107,346.34956697)
\lineto(175.60832107,346.34956697)
\lineto(175.80332107,346.34956697)
\curveto(175.88331868,346.35955922)(175.94831862,346.35955922)(175.99832107,346.34956697)
\lineto(176.11832107,346.34956697)
\curveto(176.17831839,346.32955925)(176.23331833,346.31455927)(176.28332107,346.30456697)
\curveto(176.33331823,346.29455929)(176.37331819,346.26455932)(176.40332107,346.21456697)
\curveto(176.44331812,346.16455942)(176.4633181,346.09455949)(176.46332107,346.00456697)
\curveto(176.47331809,345.91455967)(176.47831809,345.81955976)(176.47832107,345.71956697)
\lineto(176.47832107,339.43456697)
}
}
{
\newrgbcolor{curcolor}{0 0 0}
\pscustom[linestyle=none,fillstyle=solid,fillcolor=curcolor]
{
\newpath
\moveto(185.91050857,342.79456697)
\curveto(185.9305,342.73456285)(185.94049999,342.64956293)(185.94050857,342.53956697)
\curveto(185.94049999,342.42956315)(185.9305,342.34456324)(185.91050857,342.28456697)
\lineto(185.91050857,342.13456697)
\curveto(185.89050004,342.05456353)(185.88050005,341.97456361)(185.88050857,341.89456697)
\curveto(185.89050004,341.81456377)(185.88550005,341.73456385)(185.86550857,341.65456697)
\curveto(185.84550009,341.584564)(185.8305001,341.51956406)(185.82050857,341.45956697)
\curveto(185.81050012,341.39956418)(185.80050013,341.33456425)(185.79050857,341.26456697)
\curveto(185.75050018,341.15456443)(185.71550022,341.03956454)(185.68550857,340.91956697)
\curveto(185.65550028,340.80956477)(185.61550032,340.70456488)(185.56550857,340.60456697)
\curveto(185.35550058,340.12456546)(185.08050085,339.73456585)(184.74050857,339.43456697)
\curveto(184.40050153,339.13456645)(183.99050194,338.8845667)(183.51050857,338.68456697)
\curveto(183.39050254,338.63456695)(183.26550267,338.59956698)(183.13550857,338.57956697)
\curveto(183.01550292,338.54956703)(182.89050304,338.51956706)(182.76050857,338.48956697)
\curveto(182.71050322,338.46956711)(182.65550328,338.45956712)(182.59550857,338.45956697)
\curveto(182.5355034,338.45956712)(182.48050345,338.45456713)(182.43050857,338.44456697)
\lineto(182.32550857,338.44456697)
\curveto(182.29550364,338.43456715)(182.26550367,338.42956715)(182.23550857,338.42956697)
\curveto(182.18550375,338.41956716)(182.10550383,338.41456717)(181.99550857,338.41456697)
\curveto(181.88550405,338.40456718)(181.80050413,338.40956717)(181.74050857,338.42956697)
\lineto(181.59050857,338.42956697)
\curveto(181.54050439,338.43956714)(181.48550445,338.44456714)(181.42550857,338.44456697)
\curveto(181.37550456,338.43456715)(181.32550461,338.43956714)(181.27550857,338.45956697)
\curveto(181.2355047,338.46956711)(181.19550474,338.47456711)(181.15550857,338.47456697)
\curveto(181.12550481,338.47456711)(181.08550485,338.4795671)(181.03550857,338.48956697)
\curveto(180.935505,338.51956706)(180.8355051,338.54456704)(180.73550857,338.56456697)
\curveto(180.6355053,338.584567)(180.54050539,338.61456697)(180.45050857,338.65456697)
\curveto(180.3305056,338.69456689)(180.21550572,338.73456685)(180.10550857,338.77456697)
\curveto(180.00550593,338.81456677)(179.90050603,338.86456672)(179.79050857,338.92456697)
\curveto(179.44050649,339.13456645)(179.14050679,339.3795662)(178.89050857,339.65956697)
\curveto(178.64050729,339.93956564)(178.4305075,340.27456531)(178.26050857,340.66456697)
\curveto(178.21050772,340.75456483)(178.17050776,340.84956473)(178.14050857,340.94956697)
\curveto(178.12050781,341.04956453)(178.09550784,341.15456443)(178.06550857,341.26456697)
\curveto(178.04550789,341.31456427)(178.0355079,341.35956422)(178.03550857,341.39956697)
\curveto(178.0355079,341.43956414)(178.02550791,341.4845641)(178.00550857,341.53456697)
\curveto(177.98550795,341.61456397)(177.97550796,341.69456389)(177.97550857,341.77456697)
\curveto(177.97550796,341.86456372)(177.96550797,341.94956363)(177.94550857,342.02956697)
\curveto(177.935508,342.0795635)(177.930508,342.12456346)(177.93050857,342.16456697)
\lineto(177.93050857,342.29956697)
\curveto(177.91050802,342.35956322)(177.90050803,342.44456314)(177.90050857,342.55456697)
\curveto(177.91050802,342.66456292)(177.92550801,342.74956283)(177.94550857,342.80956697)
\lineto(177.94550857,342.91456697)
\curveto(177.95550798,342.96456262)(177.95550798,343.01456257)(177.94550857,343.06456697)
\curveto(177.94550799,343.12456246)(177.95550798,343.1795624)(177.97550857,343.22956697)
\curveto(177.98550795,343.2795623)(177.99050794,343.32456226)(177.99050857,343.36456697)
\curveto(177.99050794,343.41456217)(178.00050793,343.46456212)(178.02050857,343.51456697)
\curveto(178.06050787,343.64456194)(178.09550784,343.76956181)(178.12550857,343.88956697)
\curveto(178.15550778,344.01956156)(178.19550774,344.14456144)(178.24550857,344.26456697)
\curveto(178.42550751,344.67456091)(178.64050729,345.01456057)(178.89050857,345.28456697)
\curveto(179.14050679,345.56456002)(179.44550649,345.81955976)(179.80550857,346.04956697)
\curveto(179.90550603,346.09955948)(180.01050592,346.14455944)(180.12050857,346.18456697)
\curveto(180.2305057,346.22455936)(180.34050559,346.26955931)(180.45050857,346.31956697)
\curveto(180.58050535,346.36955921)(180.71550522,346.40455918)(180.85550857,346.42456697)
\curveto(180.99550494,346.44455914)(181.14050479,346.47455911)(181.29050857,346.51456697)
\curveto(181.37050456,346.52455906)(181.44550449,346.52955905)(181.51550857,346.52956697)
\curveto(181.58550435,346.52955905)(181.65550428,346.53455905)(181.72550857,346.54456697)
\curveto(182.30550363,346.55455903)(182.80550313,346.49455909)(183.22550857,346.36456697)
\curveto(183.65550228,346.23455935)(184.0355019,346.05455953)(184.36550857,345.82456697)
\curveto(184.47550146,345.74455984)(184.58550135,345.65455993)(184.69550857,345.55456697)
\curveto(184.81550112,345.46456012)(184.91550102,345.36456022)(184.99550857,345.25456697)
\curveto(185.07550086,345.15456043)(185.14550079,345.05456053)(185.20550857,344.95456697)
\curveto(185.27550066,344.85456073)(185.34550059,344.74956083)(185.41550857,344.63956697)
\curveto(185.48550045,344.52956105)(185.54050039,344.40956117)(185.58050857,344.27956697)
\curveto(185.62050031,344.15956142)(185.66550027,344.02956155)(185.71550857,343.88956697)
\curveto(185.74550019,343.80956177)(185.77050016,343.72456186)(185.79050857,343.63456697)
\lineto(185.85050857,343.36456697)
\curveto(185.86050007,343.32456226)(185.86550007,343.2845623)(185.86550857,343.24456697)
\curveto(185.86550007,343.20456238)(185.87050006,343.16456242)(185.88050857,343.12456697)
\curveto(185.90050003,343.07456251)(185.90550003,343.01956256)(185.89550857,342.95956697)
\curveto(185.88550005,342.89956268)(185.89050004,342.84456274)(185.91050857,342.79456697)
\moveto(183.81050857,342.25456697)
\curveto(183.82050211,342.30456328)(183.82550211,342.37456321)(183.82550857,342.46456697)
\curveto(183.82550211,342.56456302)(183.82050211,342.63956294)(183.81050857,342.68956697)
\lineto(183.81050857,342.80956697)
\curveto(183.79050214,342.85956272)(183.78050215,342.91456267)(183.78050857,342.97456697)
\curveto(183.78050215,343.03456255)(183.77550216,343.08956249)(183.76550857,343.13956697)
\curveto(183.76550217,343.1795624)(183.76050217,343.20956237)(183.75050857,343.22956697)
\lineto(183.69050857,343.46956697)
\curveto(183.68050225,343.55956202)(183.66050227,343.64456194)(183.63050857,343.72456697)
\curveto(183.52050241,343.9845616)(183.39050254,344.20456138)(183.24050857,344.38456697)
\curveto(183.09050284,344.57456101)(182.89050304,344.72456086)(182.64050857,344.83456697)
\curveto(182.58050335,344.85456073)(182.52050341,344.86956071)(182.46050857,344.87956697)
\curveto(182.40050353,344.89956068)(182.3355036,344.91956066)(182.26550857,344.93956697)
\curveto(182.18550375,344.95956062)(182.10050383,344.96456062)(182.01050857,344.95456697)
\lineto(181.74050857,344.95456697)
\curveto(181.71050422,344.93456065)(181.67550426,344.92456066)(181.63550857,344.92456697)
\curveto(181.59550434,344.93456065)(181.56050437,344.93456065)(181.53050857,344.92456697)
\lineto(181.32050857,344.86456697)
\curveto(181.26050467,344.85456073)(181.20550473,344.83456075)(181.15550857,344.80456697)
\curveto(180.90550503,344.69456089)(180.70050523,344.53456105)(180.54050857,344.32456697)
\curveto(180.39050554,344.12456146)(180.27050566,343.88956169)(180.18050857,343.61956697)
\curveto(180.15050578,343.51956206)(180.12550581,343.41456217)(180.10550857,343.30456697)
\curveto(180.09550584,343.19456239)(180.08050585,343.0845625)(180.06050857,342.97456697)
\curveto(180.05050588,342.92456266)(180.04550589,342.87456271)(180.04550857,342.82456697)
\lineto(180.04550857,342.67456697)
\curveto(180.02550591,342.60456298)(180.01550592,342.49956308)(180.01550857,342.35956697)
\curveto(180.02550591,342.21956336)(180.04050589,342.11456347)(180.06050857,342.04456697)
\lineto(180.06050857,341.90956697)
\curveto(180.08050585,341.82956375)(180.09550584,341.74956383)(180.10550857,341.66956697)
\curveto(180.11550582,341.59956398)(180.1305058,341.52456406)(180.15050857,341.44456697)
\curveto(180.25050568,341.14456444)(180.35550558,340.89956468)(180.46550857,340.70956697)
\curveto(180.58550535,340.52956505)(180.77050516,340.36456522)(181.02050857,340.21456697)
\curveto(181.09050484,340.16456542)(181.16550477,340.12456546)(181.24550857,340.09456697)
\curveto(181.3355046,340.06456552)(181.42550451,340.03956554)(181.51550857,340.01956697)
\curveto(181.55550438,340.00956557)(181.59050434,340.00456558)(181.62050857,340.00456697)
\curveto(181.65050428,340.01456557)(181.68550425,340.01456557)(181.72550857,340.00456697)
\lineto(181.84550857,339.97456697)
\curveto(181.89550404,339.97456561)(181.94050399,339.9795656)(181.98050857,339.98956697)
\lineto(182.10050857,339.98956697)
\curveto(182.18050375,340.00956557)(182.26050367,340.02456556)(182.34050857,340.03456697)
\curveto(182.42050351,340.04456554)(182.49550344,340.06456552)(182.56550857,340.09456697)
\curveto(182.82550311,340.19456539)(183.0355029,340.32956525)(183.19550857,340.49956697)
\curveto(183.35550258,340.66956491)(183.49050244,340.8795647)(183.60050857,341.12956697)
\curveto(183.64050229,341.22956435)(183.67050226,341.32956425)(183.69050857,341.42956697)
\curveto(183.71050222,341.52956405)(183.7355022,341.63456395)(183.76550857,341.74456697)
\curveto(183.77550216,341.7845638)(183.78050215,341.81956376)(183.78050857,341.84956697)
\curveto(183.78050215,341.88956369)(183.78550215,341.92956365)(183.79550857,341.96956697)
\lineto(183.79550857,342.10456697)
\curveto(183.79550214,342.15456343)(183.80050213,342.20456338)(183.81050857,342.25456697)
}
}
{
\newrgbcolor{curcolor}{0 0 0}
\pscustom[linestyle=none,fillstyle=solid,fillcolor=curcolor]
{
\newpath
\moveto(190.28043045,346.55956697)
\curveto(191.03042595,346.579559)(191.6804253,346.49455909)(192.23043045,346.30456697)
\curveto(192.79042419,346.12455946)(193.21542376,345.80955977)(193.50543045,345.35956697)
\curveto(193.5754234,345.24956033)(193.63542334,345.13456045)(193.68543045,345.01456697)
\curveto(193.74542323,344.90456068)(193.79542318,344.7795608)(193.83543045,344.63956697)
\curveto(193.85542312,344.579561)(193.86542311,344.51456107)(193.86543045,344.44456697)
\curveto(193.86542311,344.37456121)(193.85542312,344.31456127)(193.83543045,344.26456697)
\curveto(193.79542318,344.20456138)(193.74042324,344.16456142)(193.67043045,344.14456697)
\curveto(193.62042336,344.12456146)(193.56042342,344.11456147)(193.49043045,344.11456697)
\lineto(193.28043045,344.11456697)
\lineto(192.62043045,344.11456697)
\curveto(192.55042443,344.11456147)(192.4804245,344.10956147)(192.41043045,344.09956697)
\curveto(192.34042464,344.09956148)(192.2754247,344.10956147)(192.21543045,344.12956697)
\curveto(192.11542486,344.14956143)(192.04042494,344.18956139)(191.99043045,344.24956697)
\curveto(191.94042504,344.30956127)(191.89542508,344.36956121)(191.85543045,344.42956697)
\lineto(191.73543045,344.63956697)
\curveto(191.70542527,344.71956086)(191.65542532,344.7845608)(191.58543045,344.83456697)
\curveto(191.48542549,344.91456067)(191.38542559,344.97456061)(191.28543045,345.01456697)
\curveto(191.19542578,345.05456053)(191.0804259,345.08956049)(190.94043045,345.11956697)
\curveto(190.87042611,345.13956044)(190.76542621,345.15456043)(190.62543045,345.16456697)
\curveto(190.49542648,345.17456041)(190.39542658,345.16956041)(190.32543045,345.14956697)
\lineto(190.22043045,345.14956697)
\lineto(190.07043045,345.11956697)
\curveto(190.03042695,345.11956046)(189.98542699,345.11456047)(189.93543045,345.10456697)
\curveto(189.76542721,345.05456053)(189.62542735,344.9845606)(189.51543045,344.89456697)
\curveto(189.41542756,344.81456077)(189.34542763,344.68956089)(189.30543045,344.51956697)
\curveto(189.28542769,344.44956113)(189.28542769,344.3845612)(189.30543045,344.32456697)
\curveto(189.32542765,344.26456132)(189.34542763,344.21456137)(189.36543045,344.17456697)
\curveto(189.43542754,344.05456153)(189.51542746,343.95956162)(189.60543045,343.88956697)
\curveto(189.70542727,343.81956176)(189.82042716,343.75956182)(189.95043045,343.70956697)
\curveto(190.14042684,343.62956195)(190.34542663,343.55956202)(190.56543045,343.49956697)
\lineto(191.25543045,343.34956697)
\curveto(191.49542548,343.30956227)(191.72542525,343.25956232)(191.94543045,343.19956697)
\curveto(192.1754248,343.14956243)(192.39042459,343.0845625)(192.59043045,343.00456697)
\curveto(192.6804243,342.96456262)(192.76542421,342.92956265)(192.84543045,342.89956697)
\curveto(192.93542404,342.8795627)(193.02042396,342.84456274)(193.10043045,342.79456697)
\curveto(193.29042369,342.67456291)(193.46042352,342.54456304)(193.61043045,342.40456697)
\curveto(193.77042321,342.26456332)(193.89542308,342.08956349)(193.98543045,341.87956697)
\curveto(194.01542296,341.80956377)(194.04042294,341.73956384)(194.06043045,341.66956697)
\curveto(194.0804229,341.59956398)(194.10042288,341.52456406)(194.12043045,341.44456697)
\curveto(194.13042285,341.3845642)(194.13542284,341.28956429)(194.13543045,341.15956697)
\curveto(194.14542283,341.03956454)(194.14542283,340.94456464)(194.13543045,340.87456697)
\lineto(194.13543045,340.79956697)
\curveto(194.11542286,340.73956484)(194.10042288,340.6795649)(194.09043045,340.61956697)
\curveto(194.09042289,340.56956501)(194.08542289,340.51956506)(194.07543045,340.46956697)
\curveto(194.00542297,340.16956541)(193.89542308,339.90456568)(193.74543045,339.67456697)
\curveto(193.58542339,339.43456615)(193.39042359,339.23956634)(193.16043045,339.08956697)
\curveto(192.93042405,338.93956664)(192.67042431,338.80956677)(192.38043045,338.69956697)
\curveto(192.27042471,338.64956693)(192.15042483,338.61456697)(192.02043045,338.59456697)
\curveto(191.90042508,338.57456701)(191.7804252,338.54956703)(191.66043045,338.51956697)
\curveto(191.57042541,338.49956708)(191.4754255,338.48956709)(191.37543045,338.48956697)
\curveto(191.28542569,338.4795671)(191.19542578,338.46456712)(191.10543045,338.44456697)
\lineto(190.83543045,338.44456697)
\curveto(190.7754262,338.42456716)(190.67042631,338.41456717)(190.52043045,338.41456697)
\curveto(190.3804266,338.41456717)(190.2804267,338.42456716)(190.22043045,338.44456697)
\curveto(190.19042679,338.44456714)(190.15542682,338.44956713)(190.11543045,338.45956697)
\lineto(190.01043045,338.45956697)
\curveto(189.89042709,338.4795671)(189.77042721,338.49456709)(189.65043045,338.50456697)
\curveto(189.53042745,338.51456707)(189.41542756,338.53456705)(189.30543045,338.56456697)
\curveto(188.91542806,338.67456691)(188.57042841,338.79956678)(188.27043045,338.93956697)
\curveto(187.97042901,339.08956649)(187.71542926,339.30956627)(187.50543045,339.59956697)
\curveto(187.36542961,339.78956579)(187.24542973,340.00956557)(187.14543045,340.25956697)
\curveto(187.12542985,340.31956526)(187.10542987,340.39956518)(187.08543045,340.49956697)
\curveto(187.06542991,340.54956503)(187.05042993,340.61956496)(187.04043045,340.70956697)
\curveto(187.03042995,340.79956478)(187.03542994,340.87456471)(187.05543045,340.93456697)
\curveto(187.08542989,341.00456458)(187.13542984,341.05456453)(187.20543045,341.08456697)
\curveto(187.25542972,341.10456448)(187.31542966,341.11456447)(187.38543045,341.11456697)
\lineto(187.61043045,341.11456697)
\lineto(188.31543045,341.11456697)
\lineto(188.55543045,341.11456697)
\curveto(188.63542834,341.11456447)(188.70542827,341.10456448)(188.76543045,341.08456697)
\curveto(188.8754281,341.04456454)(188.94542803,340.9795646)(188.97543045,340.88956697)
\curveto(189.01542796,340.79956478)(189.06042792,340.70456488)(189.11043045,340.60456697)
\curveto(189.13042785,340.55456503)(189.16542781,340.48956509)(189.21543045,340.40956697)
\curveto(189.2754277,340.32956525)(189.32542765,340.2795653)(189.36543045,340.25956697)
\curveto(189.48542749,340.15956542)(189.60042738,340.0795655)(189.71043045,340.01956697)
\curveto(189.82042716,339.96956561)(189.96042702,339.91956566)(190.13043045,339.86956697)
\curveto(190.1804268,339.84956573)(190.23042675,339.83956574)(190.28043045,339.83956697)
\curveto(190.33042665,339.84956573)(190.3804266,339.84956573)(190.43043045,339.83956697)
\curveto(190.51042647,339.81956576)(190.59542638,339.80956577)(190.68543045,339.80956697)
\curveto(190.78542619,339.81956576)(190.87042611,339.83456575)(190.94043045,339.85456697)
\curveto(190.99042599,339.86456572)(191.03542594,339.86956571)(191.07543045,339.86956697)
\curveto(191.12542585,339.86956571)(191.1754258,339.8795657)(191.22543045,339.89956697)
\curveto(191.36542561,339.94956563)(191.49042549,340.00956557)(191.60043045,340.07956697)
\curveto(191.72042526,340.14956543)(191.81542516,340.23956534)(191.88543045,340.34956697)
\curveto(191.93542504,340.42956515)(191.975425,340.55456503)(192.00543045,340.72456697)
\curveto(192.02542495,340.79456479)(192.02542495,340.85956472)(192.00543045,340.91956697)
\curveto(191.98542499,340.9795646)(191.96542501,341.02956455)(191.94543045,341.06956697)
\curveto(191.8754251,341.20956437)(191.78542519,341.31456427)(191.67543045,341.38456697)
\curveto(191.5754254,341.45456413)(191.45542552,341.51956406)(191.31543045,341.57956697)
\curveto(191.12542585,341.65956392)(190.92542605,341.72456386)(190.71543045,341.77456697)
\curveto(190.50542647,341.82456376)(190.29542668,341.8795637)(190.08543045,341.93956697)
\curveto(190.00542697,341.95956362)(189.92042706,341.97456361)(189.83043045,341.98456697)
\curveto(189.75042723,341.99456359)(189.67042731,342.00956357)(189.59043045,342.02956697)
\curveto(189.27042771,342.11956346)(188.96542801,342.20456338)(188.67543045,342.28456697)
\curveto(188.38542859,342.37456321)(188.12042886,342.50456308)(187.88043045,342.67456697)
\curveto(187.60042938,342.87456271)(187.39542958,343.14456244)(187.26543045,343.48456697)
\curveto(187.24542973,343.55456203)(187.22542975,343.64956193)(187.20543045,343.76956697)
\curveto(187.18542979,343.83956174)(187.17042981,343.92456166)(187.16043045,344.02456697)
\curveto(187.15042983,344.12456146)(187.15542982,344.21456137)(187.17543045,344.29456697)
\curveto(187.19542978,344.34456124)(187.20042978,344.3845612)(187.19043045,344.41456697)
\curveto(187.1804298,344.45456113)(187.18542979,344.49956108)(187.20543045,344.54956697)
\curveto(187.22542975,344.65956092)(187.24542973,344.75956082)(187.26543045,344.84956697)
\curveto(187.29542968,344.94956063)(187.33042965,345.04456054)(187.37043045,345.13456697)
\curveto(187.50042948,345.42456016)(187.6804293,345.65955992)(187.91043045,345.83956697)
\curveto(188.14042884,346.01955956)(188.40042858,346.16455942)(188.69043045,346.27456697)
\curveto(188.80042818,346.32455926)(188.91542806,346.35955922)(189.03543045,346.37956697)
\curveto(189.15542782,346.40955917)(189.2804277,346.43955914)(189.41043045,346.46956697)
\curveto(189.47042751,346.48955909)(189.53042745,346.49955908)(189.59043045,346.49956697)
\lineto(189.77043045,346.52956697)
\curveto(189.85042713,346.53955904)(189.93542704,346.54455904)(190.02543045,346.54456697)
\curveto(190.11542686,346.54455904)(190.20042678,346.54955903)(190.28043045,346.55956697)
}
}
{
\newrgbcolor{curcolor}{0 0 0}
\pscustom[linestyle=none,fillstyle=solid,fillcolor=curcolor]
{
}
}
{
\newrgbcolor{curcolor}{0 0 0}
\pscustom[linestyle=none,fillstyle=solid,fillcolor=curcolor]
{
\newpath
\moveto(202.49722732,346.55956697)
\curveto(203.24722282,346.579559)(203.89722217,346.49455909)(204.44722732,346.30456697)
\curveto(205.00722106,346.12455946)(205.43222064,345.80955977)(205.72222732,345.35956697)
\curveto(205.79222028,345.24956033)(205.85222022,345.13456045)(205.90222732,345.01456697)
\curveto(205.96222011,344.90456068)(206.01222006,344.7795608)(206.05222732,344.63956697)
\curveto(206.07222,344.579561)(206.08221999,344.51456107)(206.08222732,344.44456697)
\curveto(206.08221999,344.37456121)(206.07222,344.31456127)(206.05222732,344.26456697)
\curveto(206.01222006,344.20456138)(205.95722011,344.16456142)(205.88722732,344.14456697)
\curveto(205.83722023,344.12456146)(205.77722029,344.11456147)(205.70722732,344.11456697)
\lineto(205.49722732,344.11456697)
\lineto(204.83722732,344.11456697)
\curveto(204.7672213,344.11456147)(204.69722137,344.10956147)(204.62722732,344.09956697)
\curveto(204.55722151,344.09956148)(204.49222158,344.10956147)(204.43222732,344.12956697)
\curveto(204.33222174,344.14956143)(204.25722181,344.18956139)(204.20722732,344.24956697)
\curveto(204.15722191,344.30956127)(204.11222196,344.36956121)(204.07222732,344.42956697)
\lineto(203.95222732,344.63956697)
\curveto(203.92222215,344.71956086)(203.8722222,344.7845608)(203.80222732,344.83456697)
\curveto(203.70222237,344.91456067)(203.60222247,344.97456061)(203.50222732,345.01456697)
\curveto(203.41222266,345.05456053)(203.29722277,345.08956049)(203.15722732,345.11956697)
\curveto(203.08722298,345.13956044)(202.98222309,345.15456043)(202.84222732,345.16456697)
\curveto(202.71222336,345.17456041)(202.61222346,345.16956041)(202.54222732,345.14956697)
\lineto(202.43722732,345.14956697)
\lineto(202.28722732,345.11956697)
\curveto(202.24722382,345.11956046)(202.20222387,345.11456047)(202.15222732,345.10456697)
\curveto(201.98222409,345.05456053)(201.84222423,344.9845606)(201.73222732,344.89456697)
\curveto(201.63222444,344.81456077)(201.56222451,344.68956089)(201.52222732,344.51956697)
\curveto(201.50222457,344.44956113)(201.50222457,344.3845612)(201.52222732,344.32456697)
\curveto(201.54222453,344.26456132)(201.56222451,344.21456137)(201.58222732,344.17456697)
\curveto(201.65222442,344.05456153)(201.73222434,343.95956162)(201.82222732,343.88956697)
\curveto(201.92222415,343.81956176)(202.03722403,343.75956182)(202.16722732,343.70956697)
\curveto(202.35722371,343.62956195)(202.56222351,343.55956202)(202.78222732,343.49956697)
\lineto(203.47222732,343.34956697)
\curveto(203.71222236,343.30956227)(203.94222213,343.25956232)(204.16222732,343.19956697)
\curveto(204.39222168,343.14956243)(204.60722146,343.0845625)(204.80722732,343.00456697)
\curveto(204.89722117,342.96456262)(204.98222109,342.92956265)(205.06222732,342.89956697)
\curveto(205.15222092,342.8795627)(205.23722083,342.84456274)(205.31722732,342.79456697)
\curveto(205.50722056,342.67456291)(205.67722039,342.54456304)(205.82722732,342.40456697)
\curveto(205.98722008,342.26456332)(206.11221996,342.08956349)(206.20222732,341.87956697)
\curveto(206.23221984,341.80956377)(206.25721981,341.73956384)(206.27722732,341.66956697)
\curveto(206.29721977,341.59956398)(206.31721975,341.52456406)(206.33722732,341.44456697)
\curveto(206.34721972,341.3845642)(206.35221972,341.28956429)(206.35222732,341.15956697)
\curveto(206.36221971,341.03956454)(206.36221971,340.94456464)(206.35222732,340.87456697)
\lineto(206.35222732,340.79956697)
\curveto(206.33221974,340.73956484)(206.31721975,340.6795649)(206.30722732,340.61956697)
\curveto(206.30721976,340.56956501)(206.30221977,340.51956506)(206.29222732,340.46956697)
\curveto(206.22221985,340.16956541)(206.11221996,339.90456568)(205.96222732,339.67456697)
\curveto(205.80222027,339.43456615)(205.60722046,339.23956634)(205.37722732,339.08956697)
\curveto(205.14722092,338.93956664)(204.88722118,338.80956677)(204.59722732,338.69956697)
\curveto(204.48722158,338.64956693)(204.3672217,338.61456697)(204.23722732,338.59456697)
\curveto(204.11722195,338.57456701)(203.99722207,338.54956703)(203.87722732,338.51956697)
\curveto(203.78722228,338.49956708)(203.69222238,338.48956709)(203.59222732,338.48956697)
\curveto(203.50222257,338.4795671)(203.41222266,338.46456712)(203.32222732,338.44456697)
\lineto(203.05222732,338.44456697)
\curveto(202.99222308,338.42456716)(202.88722318,338.41456717)(202.73722732,338.41456697)
\curveto(202.59722347,338.41456717)(202.49722357,338.42456716)(202.43722732,338.44456697)
\curveto(202.40722366,338.44456714)(202.3722237,338.44956713)(202.33222732,338.45956697)
\lineto(202.22722732,338.45956697)
\curveto(202.10722396,338.4795671)(201.98722408,338.49456709)(201.86722732,338.50456697)
\curveto(201.74722432,338.51456707)(201.63222444,338.53456705)(201.52222732,338.56456697)
\curveto(201.13222494,338.67456691)(200.78722528,338.79956678)(200.48722732,338.93956697)
\curveto(200.18722588,339.08956649)(199.93222614,339.30956627)(199.72222732,339.59956697)
\curveto(199.58222649,339.78956579)(199.46222661,340.00956557)(199.36222732,340.25956697)
\curveto(199.34222673,340.31956526)(199.32222675,340.39956518)(199.30222732,340.49956697)
\curveto(199.28222679,340.54956503)(199.2672268,340.61956496)(199.25722732,340.70956697)
\curveto(199.24722682,340.79956478)(199.25222682,340.87456471)(199.27222732,340.93456697)
\curveto(199.30222677,341.00456458)(199.35222672,341.05456453)(199.42222732,341.08456697)
\curveto(199.4722266,341.10456448)(199.53222654,341.11456447)(199.60222732,341.11456697)
\lineto(199.82722732,341.11456697)
\lineto(200.53222732,341.11456697)
\lineto(200.77222732,341.11456697)
\curveto(200.85222522,341.11456447)(200.92222515,341.10456448)(200.98222732,341.08456697)
\curveto(201.09222498,341.04456454)(201.16222491,340.9795646)(201.19222732,340.88956697)
\curveto(201.23222484,340.79956478)(201.27722479,340.70456488)(201.32722732,340.60456697)
\curveto(201.34722472,340.55456503)(201.38222469,340.48956509)(201.43222732,340.40956697)
\curveto(201.49222458,340.32956525)(201.54222453,340.2795653)(201.58222732,340.25956697)
\curveto(201.70222437,340.15956542)(201.81722425,340.0795655)(201.92722732,340.01956697)
\curveto(202.03722403,339.96956561)(202.17722389,339.91956566)(202.34722732,339.86956697)
\curveto(202.39722367,339.84956573)(202.44722362,339.83956574)(202.49722732,339.83956697)
\curveto(202.54722352,339.84956573)(202.59722347,339.84956573)(202.64722732,339.83956697)
\curveto(202.72722334,339.81956576)(202.81222326,339.80956577)(202.90222732,339.80956697)
\curveto(203.00222307,339.81956576)(203.08722298,339.83456575)(203.15722732,339.85456697)
\curveto(203.20722286,339.86456572)(203.25222282,339.86956571)(203.29222732,339.86956697)
\curveto(203.34222273,339.86956571)(203.39222268,339.8795657)(203.44222732,339.89956697)
\curveto(203.58222249,339.94956563)(203.70722236,340.00956557)(203.81722732,340.07956697)
\curveto(203.93722213,340.14956543)(204.03222204,340.23956534)(204.10222732,340.34956697)
\curveto(204.15222192,340.42956515)(204.19222188,340.55456503)(204.22222732,340.72456697)
\curveto(204.24222183,340.79456479)(204.24222183,340.85956472)(204.22222732,340.91956697)
\curveto(204.20222187,340.9795646)(204.18222189,341.02956455)(204.16222732,341.06956697)
\curveto(204.09222198,341.20956437)(204.00222207,341.31456427)(203.89222732,341.38456697)
\curveto(203.79222228,341.45456413)(203.6722224,341.51956406)(203.53222732,341.57956697)
\curveto(203.34222273,341.65956392)(203.14222293,341.72456386)(202.93222732,341.77456697)
\curveto(202.72222335,341.82456376)(202.51222356,341.8795637)(202.30222732,341.93956697)
\curveto(202.22222385,341.95956362)(202.13722393,341.97456361)(202.04722732,341.98456697)
\curveto(201.9672241,341.99456359)(201.88722418,342.00956357)(201.80722732,342.02956697)
\curveto(201.48722458,342.11956346)(201.18222489,342.20456338)(200.89222732,342.28456697)
\curveto(200.60222547,342.37456321)(200.33722573,342.50456308)(200.09722732,342.67456697)
\curveto(199.81722625,342.87456271)(199.61222646,343.14456244)(199.48222732,343.48456697)
\curveto(199.46222661,343.55456203)(199.44222663,343.64956193)(199.42222732,343.76956697)
\curveto(199.40222667,343.83956174)(199.38722668,343.92456166)(199.37722732,344.02456697)
\curveto(199.3672267,344.12456146)(199.3722267,344.21456137)(199.39222732,344.29456697)
\curveto(199.41222666,344.34456124)(199.41722665,344.3845612)(199.40722732,344.41456697)
\curveto(199.39722667,344.45456113)(199.40222667,344.49956108)(199.42222732,344.54956697)
\curveto(199.44222663,344.65956092)(199.46222661,344.75956082)(199.48222732,344.84956697)
\curveto(199.51222656,344.94956063)(199.54722652,345.04456054)(199.58722732,345.13456697)
\curveto(199.71722635,345.42456016)(199.89722617,345.65955992)(200.12722732,345.83956697)
\curveto(200.35722571,346.01955956)(200.61722545,346.16455942)(200.90722732,346.27456697)
\curveto(201.01722505,346.32455926)(201.13222494,346.35955922)(201.25222732,346.37956697)
\curveto(201.3722247,346.40955917)(201.49722457,346.43955914)(201.62722732,346.46956697)
\curveto(201.68722438,346.48955909)(201.74722432,346.49955908)(201.80722732,346.49956697)
\lineto(201.98722732,346.52956697)
\curveto(202.067224,346.53955904)(202.15222392,346.54455904)(202.24222732,346.54456697)
\curveto(202.33222374,346.54455904)(202.41722365,346.54955903)(202.49722732,346.55956697)
}
}
{
\newrgbcolor{curcolor}{0 0 0}
\pscustom[linestyle=none,fillstyle=solid,fillcolor=curcolor]
{
\newpath
\moveto(214.94886795,342.55456697)
\curveto(214.96885978,342.47456311)(214.96885978,342.3845632)(214.94886795,342.28456697)
\curveto(214.92885982,342.1845634)(214.89385986,342.11956346)(214.84386795,342.08956697)
\curveto(214.79385996,342.04956353)(214.71886003,342.01956356)(214.61886795,341.99956697)
\curveto(214.52886022,341.98956359)(214.42386033,341.9795636)(214.30386795,341.96956697)
\lineto(213.95886795,341.96956697)
\curveto(213.8488609,341.9795636)(213.748861,341.9845636)(213.65886795,341.98456697)
\lineto(209.99886795,341.98456697)
\lineto(209.78886795,341.98456697)
\curveto(209.72886502,341.9845636)(209.67386508,341.97456361)(209.62386795,341.95456697)
\curveto(209.54386521,341.91456367)(209.49386526,341.87456371)(209.47386795,341.83456697)
\curveto(209.4538653,341.81456377)(209.43386532,341.77456381)(209.41386795,341.71456697)
\curveto(209.39386536,341.66456392)(209.38886536,341.61456397)(209.39886795,341.56456697)
\curveto(209.41886533,341.50456408)(209.42886532,341.44456414)(209.42886795,341.38456697)
\curveto(209.43886531,341.33456425)(209.4538653,341.2795643)(209.47386795,341.21956697)
\curveto(209.5538652,340.9795646)(209.6488651,340.7795648)(209.75886795,340.61956697)
\curveto(209.87886487,340.46956511)(210.03886471,340.33456525)(210.23886795,340.21456697)
\curveto(210.31886443,340.16456542)(210.39886435,340.12956545)(210.47886795,340.10956697)
\curveto(210.56886418,340.09956548)(210.65886409,340.0795655)(210.74886795,340.04956697)
\curveto(210.82886392,340.02956555)(210.93886381,340.01456557)(211.07886795,340.00456697)
\curveto(211.21886353,339.99456559)(211.33886341,339.99956558)(211.43886795,340.01956697)
\lineto(211.57386795,340.01956697)
\curveto(211.67386308,340.03956554)(211.76386299,340.05956552)(211.84386795,340.07956697)
\curveto(211.93386282,340.10956547)(212.01886273,340.13956544)(212.09886795,340.16956697)
\curveto(212.19886255,340.21956536)(212.30886244,340.2845653)(212.42886795,340.36456697)
\curveto(212.55886219,340.44456514)(212.6538621,340.52456506)(212.71386795,340.60456697)
\curveto(212.76386199,340.67456491)(212.81386194,340.73956484)(212.86386795,340.79956697)
\curveto(212.92386183,340.86956471)(212.99386176,340.91956466)(213.07386795,340.94956697)
\curveto(213.17386158,340.99956458)(213.29886145,341.01956456)(213.44886795,341.00956697)
\lineto(213.88386795,341.00956697)
\lineto(214.06386795,341.00956697)
\curveto(214.13386062,341.01956456)(214.19386056,341.01456457)(214.24386795,340.99456697)
\lineto(214.39386795,340.99456697)
\curveto(214.49386026,340.97456461)(214.56386019,340.94956463)(214.60386795,340.91956697)
\curveto(214.64386011,340.89956468)(214.66386009,340.85456473)(214.66386795,340.78456697)
\curveto(214.67386008,340.71456487)(214.66886008,340.65456493)(214.64886795,340.60456697)
\curveto(214.59886015,340.46456512)(214.54386021,340.33956524)(214.48386795,340.22956697)
\curveto(214.42386033,340.11956546)(214.3538604,340.00956557)(214.27386795,339.89956697)
\curveto(214.0538607,339.56956601)(213.80386095,339.30456628)(213.52386795,339.10456697)
\curveto(213.24386151,338.90456668)(212.89386186,338.73456685)(212.47386795,338.59456697)
\curveto(212.36386239,338.55456703)(212.2538625,338.52956705)(212.14386795,338.51956697)
\curveto(212.03386272,338.50956707)(211.91886283,338.48956709)(211.79886795,338.45956697)
\curveto(211.75886299,338.44956713)(211.71386304,338.44956713)(211.66386795,338.45956697)
\curveto(211.62386313,338.45956712)(211.58386317,338.45456713)(211.54386795,338.44456697)
\lineto(211.37886795,338.44456697)
\curveto(211.32886342,338.42456716)(211.26886348,338.41956716)(211.19886795,338.42956697)
\curveto(211.13886361,338.42956715)(211.08386367,338.43456715)(211.03386795,338.44456697)
\curveto(210.9538638,338.45456713)(210.88386387,338.45456713)(210.82386795,338.44456697)
\curveto(210.76386399,338.43456715)(210.69886405,338.43956714)(210.62886795,338.45956697)
\curveto(210.57886417,338.4795671)(210.52386423,338.48956709)(210.46386795,338.48956697)
\curveto(210.40386435,338.48956709)(210.3488644,338.49956708)(210.29886795,338.51956697)
\curveto(210.18886456,338.53956704)(210.07886467,338.56456702)(209.96886795,338.59456697)
\curveto(209.85886489,338.61456697)(209.75886499,338.64956693)(209.66886795,338.69956697)
\curveto(209.55886519,338.73956684)(209.4538653,338.77456681)(209.35386795,338.80456697)
\curveto(209.26386549,338.84456674)(209.17886557,338.88956669)(209.09886795,338.93956697)
\curveto(208.77886597,339.13956644)(208.49386626,339.36956621)(208.24386795,339.62956697)
\curveto(207.99386676,339.89956568)(207.78886696,340.20956537)(207.62886795,340.55956697)
\curveto(207.57886717,340.66956491)(207.53886721,340.7795648)(207.50886795,340.88956697)
\curveto(207.47886727,341.00956457)(207.43886731,341.12956445)(207.38886795,341.24956697)
\curveto(207.37886737,341.28956429)(207.37386738,341.32456426)(207.37386795,341.35456697)
\curveto(207.37386738,341.39456419)(207.36886738,341.43456415)(207.35886795,341.47456697)
\curveto(207.31886743,341.59456399)(207.29386746,341.72456386)(207.28386795,341.86456697)
\lineto(207.25386795,342.28456697)
\curveto(207.2538675,342.33456325)(207.2488675,342.38956319)(207.23886795,342.44956697)
\curveto(207.23886751,342.50956307)(207.24386751,342.56456302)(207.25386795,342.61456697)
\lineto(207.25386795,342.79456697)
\lineto(207.29886795,343.15456697)
\curveto(207.33886741,343.32456226)(207.37386738,343.48956209)(207.40386795,343.64956697)
\curveto(207.43386732,343.80956177)(207.47886727,343.95956162)(207.53886795,344.09956697)
\curveto(207.96886678,345.13956044)(208.69886605,345.87455971)(209.72886795,346.30456697)
\curveto(209.86886488,346.36455922)(210.00886474,346.40455918)(210.14886795,346.42456697)
\curveto(210.29886445,346.45455913)(210.4538643,346.48955909)(210.61386795,346.52956697)
\curveto(210.69386406,346.53955904)(210.76886398,346.54455904)(210.83886795,346.54456697)
\curveto(210.90886384,346.54455904)(210.98386377,346.54955903)(211.06386795,346.55956697)
\curveto(211.57386318,346.56955901)(212.00886274,346.50955907)(212.36886795,346.37956697)
\curveto(212.73886201,346.25955932)(213.06886168,346.09955948)(213.35886795,345.89956697)
\curveto(213.4488613,345.83955974)(213.53886121,345.76955981)(213.62886795,345.68956697)
\curveto(213.71886103,345.61955996)(213.79886095,345.54456004)(213.86886795,345.46456697)
\curveto(213.89886085,345.41456017)(213.93886081,345.37456021)(213.98886795,345.34456697)
\curveto(214.06886068,345.23456035)(214.14386061,345.11956046)(214.21386795,344.99956697)
\curveto(214.28386047,344.88956069)(214.35886039,344.77456081)(214.43886795,344.65456697)
\curveto(214.48886026,344.56456102)(214.52886022,344.46956111)(214.55886795,344.36956697)
\curveto(214.59886015,344.2795613)(214.63886011,344.1795614)(214.67886795,344.06956697)
\curveto(214.72886002,343.93956164)(214.76885998,343.80456178)(214.79886795,343.66456697)
\curveto(214.82885992,343.52456206)(214.86385989,343.3845622)(214.90386795,343.24456697)
\curveto(214.92385983,343.16456242)(214.92885982,343.07456251)(214.91886795,342.97456697)
\curveto(214.91885983,342.8845627)(214.92885982,342.79956278)(214.94886795,342.71956697)
\lineto(214.94886795,342.55456697)
\moveto(212.69886795,343.43956697)
\curveto(212.76886198,343.53956204)(212.77386198,343.65956192)(212.71386795,343.79956697)
\curveto(212.66386209,343.94956163)(212.62386213,344.05956152)(212.59386795,344.12956697)
\curveto(212.4538623,344.39956118)(212.26886248,344.60456098)(212.03886795,344.74456697)
\curveto(211.80886294,344.89456069)(211.48886326,344.97456061)(211.07886795,344.98456697)
\curveto(211.0488637,344.96456062)(211.01386374,344.95956062)(210.97386795,344.96956697)
\curveto(210.93386382,344.9795606)(210.89886385,344.9795606)(210.86886795,344.96956697)
\curveto(210.81886393,344.94956063)(210.76386399,344.93456065)(210.70386795,344.92456697)
\curveto(210.64386411,344.92456066)(210.58886416,344.91456067)(210.53886795,344.89456697)
\curveto(210.09886465,344.75456083)(209.77386498,344.4795611)(209.56386795,344.06956697)
\curveto(209.54386521,344.02956155)(209.51886523,343.97456161)(209.48886795,343.90456697)
\curveto(209.46886528,343.84456174)(209.4538653,343.7795618)(209.44386795,343.70956697)
\curveto(209.43386532,343.64956193)(209.43386532,343.58956199)(209.44386795,343.52956697)
\curveto(209.46386529,343.46956211)(209.49886525,343.41956216)(209.54886795,343.37956697)
\curveto(209.62886512,343.32956225)(209.73886501,343.30456228)(209.87886795,343.30456697)
\lineto(210.28386795,343.30456697)
\lineto(211.94886795,343.30456697)
\lineto(212.38386795,343.30456697)
\curveto(212.54386221,343.31456227)(212.6488621,343.35956222)(212.69886795,343.43956697)
}
}
{
\newrgbcolor{curcolor}{0 0 0}
\pscustom[linestyle=none,fillstyle=solid,fillcolor=curcolor]
{
\newpath
\moveto(223.5471492,346.25956697)
\curveto(223.617141,346.20955937)(223.65214096,346.12455946)(223.6521492,346.00456697)
\curveto(223.66214095,345.89455969)(223.66714095,345.7795598)(223.6671492,345.65956697)
\lineto(223.6671492,339.25456697)
\curveto(223.66714095,339.17456641)(223.66214095,339.09456649)(223.6521492,339.01456697)
\lineto(223.6521492,338.78956697)
\curveto(223.64214097,338.70956687)(223.63214098,338.63956694)(223.6221492,338.57956697)
\curveto(223.62214099,338.50956707)(223.617141,338.43456715)(223.6071492,338.35456697)
\curveto(223.56714105,338.21456737)(223.53214108,338.0845675)(223.5021492,337.96456697)
\curveto(223.48214113,337.83456775)(223.44714117,337.71456787)(223.3971492,337.60456697)
\curveto(223.22714139,337.22456836)(223.00714161,336.90956867)(222.7371492,336.65956697)
\curveto(222.47714214,336.40956917)(222.15714246,336.20456938)(221.7771492,336.04456697)
\curveto(221.66714295,335.99456959)(221.55714306,335.95456963)(221.4471492,335.92456697)
\curveto(221.33714328,335.89456969)(221.22214339,335.86456972)(221.1021492,335.83456697)
\curveto(220.99214362,335.80456978)(220.88214373,335.7845698)(220.7721492,335.77456697)
\curveto(220.66214395,335.76456982)(220.55214406,335.74956983)(220.4421492,335.72956697)
\lineto(220.3221492,335.72956697)
\curveto(220.28214433,335.71956986)(220.23714438,335.71456987)(220.1871492,335.71456697)
\curveto(220.14714447,335.70456988)(220.10214451,335.70456988)(220.0521492,335.71456697)
\curveto(220.00214461,335.71456987)(219.95214466,335.70956987)(219.9021492,335.69956697)
\curveto(219.85214476,335.68956989)(219.78714483,335.6845699)(219.7071492,335.68456697)
\curveto(219.62714499,335.6845699)(219.56214505,335.68956989)(219.5121492,335.69956697)
\lineto(219.3771492,335.69956697)
\curveto(219.33714528,335.69956988)(219.29714532,335.70456988)(219.2571492,335.71456697)
\curveto(219.17714544,335.73456985)(219.09214552,335.74456984)(219.0021492,335.74456697)
\curveto(218.92214569,335.74456984)(218.84714577,335.75456983)(218.7771492,335.77456697)
\curveto(218.75714586,335.7845698)(218.73214588,335.78956979)(218.7021492,335.78956697)
\curveto(218.67214594,335.78956979)(218.64714597,335.79456979)(218.6271492,335.80456697)
\curveto(218.52714609,335.82456976)(218.42714619,335.84956973)(218.3271492,335.87956697)
\curveto(218.23714638,335.89956968)(218.14714647,335.92956965)(218.0571492,335.96956697)
\curveto(217.67714694,336.12956945)(217.33714728,336.33456925)(217.0371492,336.58456697)
\curveto(216.73714788,336.82456876)(216.5171481,337.14956843)(216.3771492,337.55956697)
\curveto(216.35714826,337.58956799)(216.34714827,337.61956796)(216.3471492,337.64956697)
\curveto(216.34714827,337.6795679)(216.34214827,337.70456788)(216.3321492,337.72456697)
\curveto(216.30214831,337.85456773)(216.3121483,337.95456763)(216.3621492,338.02456697)
\curveto(216.42214819,338.0845675)(216.50214811,338.12456746)(216.6021492,338.14456697)
\curveto(216.70214791,338.16456742)(216.8121478,338.17456741)(216.9321492,338.17456697)
\curveto(217.06214755,338.16456742)(217.18214743,338.15956742)(217.2921492,338.15956697)
\lineto(217.8021492,338.15956697)
\lineto(217.9221492,338.15956697)
\curveto(217.96214665,338.14956743)(218.00714661,338.14456744)(218.0571492,338.14456697)
\curveto(218.2171464,338.10456748)(218.3171463,338.05456753)(218.3571492,337.99456697)
\curveto(218.39714622,337.92456766)(218.45714616,337.83456775)(218.5371492,337.72456697)
\curveto(218.56714605,337.6845679)(218.612146,337.63456795)(218.6721492,337.57456697)
\curveto(218.68214593,337.55456803)(218.69214592,337.53956804)(218.7021492,337.52956697)
\curveto(218.7121459,337.51956806)(218.72214589,337.50456808)(218.7321492,337.48456697)
\curveto(218.8121458,337.42456816)(218.89714572,337.36956821)(218.9871492,337.31956697)
\curveto(219.07714554,337.26956831)(219.17714544,337.22456836)(219.2871492,337.18456697)
\curveto(219.35714526,337.16456842)(219.42714519,337.15456843)(219.4971492,337.15456697)
\curveto(219.56714505,337.14456844)(219.64214497,337.12956845)(219.7221492,337.10956697)
\lineto(219.8871492,337.10956697)
\curveto(219.95714466,337.08956849)(220.04714457,337.08956849)(220.1571492,337.10956697)
\curveto(220.26714435,337.11956846)(220.35214426,337.13456845)(220.4121492,337.15456697)
\curveto(220.46214415,337.17456841)(220.50214411,337.1845684)(220.5321492,337.18456697)
\curveto(220.57214404,337.1845684)(220.612144,337.19456839)(220.6521492,337.21456697)
\curveto(220.86214375,337.30456828)(221.03714358,337.42456816)(221.1771492,337.57456697)
\curveto(221.3171433,337.72456786)(221.43214318,337.89956768)(221.5221492,338.09956697)
\curveto(221.54214307,338.15956742)(221.55714306,338.21956736)(221.5671492,338.27956697)
\curveto(221.57714304,338.33956724)(221.59214302,338.40456718)(221.6121492,338.47456697)
\curveto(221.63214298,338.56456702)(221.64214297,338.65956692)(221.6421492,338.75956697)
\curveto(221.65214296,338.86956671)(221.65714296,338.9795666)(221.6571492,339.08956697)
\lineto(221.6571492,339.20956697)
\curveto(221.66714295,339.24956633)(221.66714295,339.2845663)(221.6571492,339.31456697)
\curveto(221.63714298,339.36456622)(221.62714299,339.40956617)(221.6271492,339.44956697)
\curveto(221.63714298,339.48956609)(221.63214298,339.52956605)(221.6121492,339.56956697)
\curveto(221.60214301,339.58956599)(221.58714303,339.60456598)(221.5671492,339.61456697)
\lineto(221.5221492,339.65956697)
\curveto(221.43214318,339.66956591)(221.35714326,339.64956593)(221.2971492,339.59956697)
\curveto(221.24714337,339.54956603)(221.19714342,339.50456608)(221.1471492,339.46456697)
\curveto(221.05714356,339.39456619)(220.96714365,339.32956625)(220.8771492,339.26956697)
\curveto(220.78714383,339.20956637)(220.68714393,339.15456643)(220.5771492,339.10456697)
\curveto(220.46714415,339.05456653)(220.35714426,339.01456657)(220.2471492,338.98456697)
\curveto(220.13714448,338.95456663)(220.02214459,338.92456666)(219.9021492,338.89456697)
\lineto(219.7221492,338.86456697)
\curveto(219.67214494,338.86456672)(219.62214499,338.85956672)(219.5721492,338.84956697)
\curveto(219.52214509,338.83956674)(219.44214517,338.83456675)(219.3321492,338.83456697)
\curveto(219.22214539,338.83456675)(219.14214547,338.83956674)(219.0921492,338.84956697)
\lineto(218.9721492,338.84956697)
\curveto(218.94214567,338.85956672)(218.90714571,338.86456672)(218.8671492,338.86456697)
\curveto(218.83714578,338.86456672)(218.80214581,338.86956671)(218.7621492,338.87956697)
\curveto(218.62214599,338.90956667)(218.48714613,338.93456665)(218.3571492,338.95456697)
\curveto(218.22714639,338.9845666)(218.10714651,339.02456656)(217.9971492,339.07456697)
\curveto(217.56714705,339.24456634)(217.2171474,339.4795661)(216.9471492,339.77956697)
\curveto(216.68714793,340.08956549)(216.46714815,340.45956512)(216.2871492,340.88956697)
\curveto(216.23714838,340.99956458)(216.20214841,341.11456447)(216.1821492,341.23456697)
\curveto(216.16214845,341.35456423)(216.13214848,341.47456411)(216.0921492,341.59456697)
\curveto(216.09214852,341.64456394)(216.08714853,341.6845639)(216.0771492,341.71456697)
\curveto(216.05714856,341.79456379)(216.04714857,341.8795637)(216.0471492,341.96956697)
\curveto(216.04714857,342.06956351)(216.03714858,342.15956342)(216.0171492,342.23956697)
\curveto(216.00714861,342.28956329)(216.00214861,342.33456325)(216.0021492,342.37456697)
\lineto(216.0021492,342.52456697)
\curveto(215.99214862,342.57456301)(215.98714863,342.63456295)(215.9871492,342.70456697)
\curveto(215.98714863,342.7845628)(215.99214862,342.84956273)(216.0021492,342.89956697)
\lineto(216.0021492,343.04956697)
\curveto(216.0121486,343.08956249)(216.0121486,343.12956245)(216.0021492,343.16956697)
\curveto(216.00214861,343.20956237)(216.0121486,343.24956233)(216.0321492,343.28956697)
\curveto(216.05214856,343.38956219)(216.06714855,343.4845621)(216.0771492,343.57456697)
\curveto(216.08714853,343.67456191)(216.10214851,343.77456181)(216.1221492,343.87456697)
\curveto(216.18214843,344.07456151)(216.24214837,344.26456132)(216.3021492,344.44456697)
\curveto(216.37214824,344.62456096)(216.45714816,344.79456079)(216.5571492,344.95456697)
\curveto(216.60714801,345.05456053)(216.66214795,345.14456044)(216.7221492,345.22456697)
\lineto(216.9321492,345.49456697)
\curveto(216.96214765,345.54456004)(217.00214761,345.59455999)(217.0521492,345.64456697)
\curveto(217.1121475,345.69455989)(217.16714745,345.73955984)(217.2171492,345.77956697)
\lineto(217.3071492,345.86956697)
\curveto(217.35714726,345.90955967)(217.40714721,345.94455964)(217.4571492,345.97456697)
\curveto(217.50714711,346.01455957)(217.55714706,346.04955953)(217.6071492,346.07956697)
\curveto(217.73714688,346.15955942)(217.87214674,346.22955935)(218.0121492,346.28956697)
\curveto(218.15214646,346.34955923)(218.30714631,346.40455918)(218.4771492,346.45456697)
\curveto(218.55714606,346.4845591)(218.63714598,346.49955908)(218.7171492,346.49956697)
\curveto(218.80714581,346.50955907)(218.89214572,346.52455906)(218.9721492,346.54456697)
\curveto(219.0121456,346.55455903)(219.06714555,346.55455903)(219.1371492,346.54456697)
\curveto(219.20714541,346.53455905)(219.25214536,346.53955904)(219.2721492,346.55956697)
\curveto(219.59214502,346.56955901)(219.87714474,346.53955904)(220.1271492,346.46956697)
\curveto(220.38714423,346.39955918)(220.617144,346.29955928)(220.8171492,346.16956697)
\curveto(220.84714377,346.14955943)(220.87714374,346.12455946)(220.9071492,346.09456697)
\curveto(220.93714368,346.07455951)(220.97214364,346.04955953)(221.0121492,346.01956697)
\curveto(221.07214354,345.96955961)(221.12714349,345.91955966)(221.1771492,345.86956697)
\curveto(221.22714339,345.81955976)(221.28714333,345.77455981)(221.3571492,345.73456697)
\curveto(221.37714324,345.72455986)(221.40214321,345.71455987)(221.4321492,345.70456697)
\curveto(221.47214314,345.69455989)(221.50214311,345.69955988)(221.5221492,345.71956697)
\curveto(221.57214304,345.73955984)(221.60214301,345.77455981)(221.6121492,345.82456697)
\curveto(221.62214299,345.87455971)(221.63714298,345.92455966)(221.6571492,345.97456697)
\curveto(221.67714294,346.02455956)(221.69214292,346.07455951)(221.7021492,346.12456697)
\curveto(221.72214289,346.1845594)(221.75214286,346.23455935)(221.7921492,346.27456697)
\curveto(221.85214276,346.31455927)(221.92214269,346.33455925)(222.0021492,346.33456697)
\curveto(222.09214252,346.34455924)(222.18214243,346.34955923)(222.2721492,346.34956697)
\lineto(223.0371492,346.34956697)
\curveto(223.14714147,346.34955923)(223.24214137,346.34455924)(223.3221492,346.33456697)
\curveto(223.4121412,346.33455925)(223.48714113,346.30955927)(223.5471492,346.25956697)
\moveto(221.4921492,341.62456697)
\curveto(221.53214308,341.71456387)(221.56714305,341.82956375)(221.5971492,341.96956697)
\curveto(221.62714299,342.10956347)(221.64714297,342.25456333)(221.6571492,342.40456697)
\curveto(221.66714295,342.56456302)(221.66714295,342.71956286)(221.6571492,342.86956697)
\curveto(221.65714296,343.01956256)(221.64214297,343.15456243)(221.6121492,343.27456697)
\curveto(221.59214302,343.31456227)(221.58214303,343.34456224)(221.5821492,343.36456697)
\curveto(221.59214302,343.39456219)(221.59214302,343.42956215)(221.5821492,343.46956697)
\lineto(221.5221492,343.67956697)
\curveto(221.50214311,343.74956183)(221.47714314,343.81456177)(221.4471492,343.87456697)
\curveto(221.30714331,344.22456136)(221.10714351,344.49456109)(220.8471492,344.68456697)
\curveto(220.58714403,344.87456071)(220.20714441,344.96956061)(219.7071492,344.96956697)
\curveto(219.68714493,344.94956063)(219.65714496,344.93956064)(219.6171492,344.93956697)
\curveto(219.58714503,344.94956063)(219.55714506,344.94956063)(219.5271492,344.93956697)
\curveto(219.45714516,344.91956066)(219.39214522,344.89956068)(219.3321492,344.87956697)
\curveto(219.27214534,344.86956071)(219.2121454,344.85456073)(219.1521492,344.83456697)
\curveto(218.89214572,344.72456086)(218.69214592,344.55956102)(218.5521492,344.33956697)
\curveto(218.4121462,344.11956146)(218.29714632,343.87456171)(218.2071492,343.60456697)
\curveto(218.18714643,343.55456203)(218.17714644,343.51456207)(218.1771492,343.48456697)
\curveto(218.17714644,343.45456213)(218.17214644,343.41456217)(218.1621492,343.36456697)
\curveto(218.13214648,343.25456233)(218.1121465,343.09456249)(218.1021492,342.88456697)
\curveto(218.09214652,342.67456291)(218.10214651,342.50456308)(218.1321492,342.37456697)
\lineto(218.1321492,342.22456697)
\curveto(218.15214646,342.14456344)(218.16714645,342.06456352)(218.1771492,341.98456697)
\curveto(218.18714643,341.91456367)(218.20214641,341.83956374)(218.2221492,341.75956697)
\curveto(218.3121463,341.49956408)(218.42214619,341.26956431)(218.5521492,341.06956697)
\curveto(218.68214593,340.8795647)(218.86214575,340.72456486)(219.0921492,340.60456697)
\curveto(219.19214542,340.55456503)(219.33214528,340.50456508)(219.5121492,340.45456697)
\curveto(219.58214503,340.45456513)(219.63714498,340.44956513)(219.6771492,340.43956697)
\curveto(219.69714492,340.43956514)(219.72714489,340.43456515)(219.7671492,340.42456697)
\curveto(219.80714481,340.42456516)(219.83714478,340.42956515)(219.8571492,340.43956697)
\lineto(220.0071492,340.43956697)
\curveto(220.09714452,340.45956512)(220.18214443,340.47456511)(220.2621492,340.48456697)
\curveto(220.34214427,340.49456509)(220.42214419,340.51956506)(220.5021492,340.55956697)
\curveto(220.75214386,340.65956492)(220.95214366,340.79956478)(221.1021492,340.97956697)
\curveto(221.26214335,341.15956442)(221.39214322,341.37456421)(221.4921492,341.62456697)
}
}
{
\newrgbcolor{curcolor}{0 0 0}
\pscustom[linestyle=none,fillstyle=solid,fillcolor=curcolor]
{
\newpath
\moveto(225.78707107,346.33456697)
\lineto(226.91207107,346.33456697)
\curveto(227.02206864,346.33455925)(227.12206854,346.32955925)(227.21207107,346.31956697)
\curveto(227.30206836,346.30955927)(227.36706829,346.27455931)(227.40707107,346.21456697)
\curveto(227.4570682,346.15455943)(227.48706817,346.06955951)(227.49707107,345.95956697)
\curveto(227.50706815,345.85955972)(227.51206815,345.75455983)(227.51207107,345.64456697)
\lineto(227.51207107,344.59456697)
\lineto(227.51207107,342.35956697)
\curveto(227.51206815,341.99956358)(227.52706813,341.65956392)(227.55707107,341.33956697)
\curveto(227.58706807,341.01956456)(227.67706798,340.75456483)(227.82707107,340.54456697)
\curveto(227.96706769,340.33456525)(228.19206747,340.1845654)(228.50207107,340.09456697)
\curveto(228.55206711,340.0845655)(228.59206707,340.0795655)(228.62207107,340.07956697)
\curveto(228.662067,340.0795655)(228.70706695,340.07456551)(228.75707107,340.06456697)
\curveto(228.80706685,340.05456553)(228.8620668,340.04956553)(228.92207107,340.04956697)
\curveto(228.98206668,340.04956553)(229.02706663,340.05456553)(229.05707107,340.06456697)
\curveto(229.10706655,340.0845655)(229.14706651,340.08956549)(229.17707107,340.07956697)
\curveto(229.21706644,340.06956551)(229.2570664,340.07456551)(229.29707107,340.09456697)
\curveto(229.50706615,340.14456544)(229.67206599,340.20956537)(229.79207107,340.28956697)
\curveto(229.97206569,340.39956518)(230.11206555,340.53956504)(230.21207107,340.70956697)
\curveto(230.32206534,340.88956469)(230.39706526,341.0845645)(230.43707107,341.29456697)
\curveto(230.48706517,341.51456407)(230.51706514,341.75456383)(230.52707107,342.01456697)
\curveto(230.53706512,342.2845633)(230.54206512,342.56456302)(230.54207107,342.85456697)
\lineto(230.54207107,344.66956697)
\lineto(230.54207107,345.64456697)
\lineto(230.54207107,345.91456697)
\curveto(230.54206512,346.01455957)(230.5620651,346.09455949)(230.60207107,346.15456697)
\curveto(230.65206501,346.24455934)(230.72706493,346.29455929)(230.82707107,346.30456697)
\curveto(230.92706473,346.32455926)(231.04706461,346.33455925)(231.18707107,346.33456697)
\lineto(231.98207107,346.33456697)
\lineto(232.26707107,346.33456697)
\curveto(232.3570633,346.33455925)(232.43206323,346.31455927)(232.49207107,346.27456697)
\curveto(232.57206309,346.22455936)(232.61706304,346.14955943)(232.62707107,346.04956697)
\curveto(232.63706302,345.94955963)(232.64206302,345.83455975)(232.64207107,345.70456697)
\lineto(232.64207107,344.56456697)
\lineto(232.64207107,340.34956697)
\lineto(232.64207107,339.28456697)
\lineto(232.64207107,338.98456697)
\curveto(232.64206302,338.8845667)(232.62206304,338.80956677)(232.58207107,338.75956697)
\curveto(232.53206313,338.6795669)(232.4570632,338.63456695)(232.35707107,338.62456697)
\curveto(232.2570634,338.61456697)(232.15206351,338.60956697)(232.04207107,338.60956697)
\lineto(231.23207107,338.60956697)
\curveto(231.12206454,338.60956697)(231.02206464,338.61456697)(230.93207107,338.62456697)
\curveto(230.85206481,338.63456695)(230.78706487,338.67456691)(230.73707107,338.74456697)
\curveto(230.71706494,338.77456681)(230.69706496,338.81956676)(230.67707107,338.87956697)
\curveto(230.66706499,338.93956664)(230.65206501,338.99956658)(230.63207107,339.05956697)
\curveto(230.62206504,339.11956646)(230.60706505,339.17456641)(230.58707107,339.22456697)
\curveto(230.56706509,339.27456631)(230.53706512,339.30456628)(230.49707107,339.31456697)
\curveto(230.47706518,339.33456625)(230.45206521,339.33956624)(230.42207107,339.32956697)
\curveto(230.39206527,339.31956626)(230.36706529,339.30956627)(230.34707107,339.29956697)
\curveto(230.27706538,339.25956632)(230.21706544,339.21456637)(230.16707107,339.16456697)
\curveto(230.11706554,339.11456647)(230.0620656,339.06956651)(230.00207107,339.02956697)
\curveto(229.9620657,338.99956658)(229.92206574,338.96456662)(229.88207107,338.92456697)
\curveto(229.85206581,338.89456669)(229.81206585,338.86456672)(229.76207107,338.83456697)
\curveto(229.53206613,338.69456689)(229.2620664,338.584567)(228.95207107,338.50456697)
\curveto(228.88206678,338.4845671)(228.81206685,338.47456711)(228.74207107,338.47456697)
\curveto(228.67206699,338.46456712)(228.59706706,338.44956713)(228.51707107,338.42956697)
\curveto(228.47706718,338.41956716)(228.43206723,338.41956716)(228.38207107,338.42956697)
\curveto(228.34206732,338.42956715)(228.30206736,338.42456716)(228.26207107,338.41456697)
\curveto(228.23206743,338.40456718)(228.16706749,338.40456718)(228.06707107,338.41456697)
\curveto(227.97706768,338.41456717)(227.91706774,338.41956716)(227.88707107,338.42956697)
\curveto(227.83706782,338.42956715)(227.78706787,338.43456715)(227.73707107,338.44456697)
\lineto(227.58707107,338.44456697)
\curveto(227.46706819,338.47456711)(227.35206831,338.49956708)(227.24207107,338.51956697)
\curveto(227.13206853,338.53956704)(227.02206864,338.56956701)(226.91207107,338.60956697)
\curveto(226.8620688,338.62956695)(226.81706884,338.64456694)(226.77707107,338.65456697)
\curveto(226.74706891,338.67456691)(226.70706895,338.69456689)(226.65707107,338.71456697)
\curveto(226.30706935,338.90456668)(226.02706963,339.16956641)(225.81707107,339.50956697)
\curveto(225.68706997,339.71956586)(225.59207007,339.96956561)(225.53207107,340.25956697)
\curveto(225.47207019,340.55956502)(225.43207023,340.87456471)(225.41207107,341.20456697)
\curveto(225.40207026,341.54456404)(225.39707026,341.88956369)(225.39707107,342.23956697)
\curveto(225.40707025,342.59956298)(225.41207025,342.95456263)(225.41207107,343.30456697)
\lineto(225.41207107,345.34456697)
\curveto(225.41207025,345.47456011)(225.40707025,345.62455996)(225.39707107,345.79456697)
\curveto(225.39707026,345.97455961)(225.42207024,346.10455948)(225.47207107,346.18456697)
\curveto(225.50207016,346.23455935)(225.5620701,346.2795593)(225.65207107,346.31956697)
\curveto(225.71206995,346.31955926)(225.7570699,346.32455926)(225.78707107,346.33456697)
\moveto(229.65707107,349.42456697)
\lineto(230.72207107,349.42456697)
\curveto(230.80206486,349.42455616)(230.89706476,349.42455616)(231.00707107,349.42456697)
\curveto(231.11706454,349.42455616)(231.19706446,349.40955617)(231.24707107,349.37956697)
\curveto(231.26706439,349.36955621)(231.27706438,349.35455623)(231.27707107,349.33456697)
\curveto(231.28706437,349.32455626)(231.30206436,349.31455627)(231.32207107,349.30456697)
\curveto(231.33206433,349.1845564)(231.28206438,349.0795565)(231.17207107,348.98956697)
\curveto(231.07206459,348.89955668)(230.98706467,348.81955676)(230.91707107,348.74956697)
\curveto(230.83706482,348.6795569)(230.7570649,348.60455698)(230.67707107,348.52456697)
\curveto(230.60706505,348.45455713)(230.53206513,348.38955719)(230.45207107,348.32956697)
\curveto(230.41206525,348.29955728)(230.37706528,348.26455732)(230.34707107,348.22456697)
\curveto(230.32706533,348.19455739)(230.29706536,348.16955741)(230.25707107,348.14956697)
\curveto(230.23706542,348.11955746)(230.21206545,348.09455749)(230.18207107,348.07456697)
\lineto(230.03207107,347.92456697)
\lineto(229.88207107,347.80456697)
\lineto(229.83707107,347.75956697)
\curveto(229.83706582,347.74955783)(229.82706583,347.73455785)(229.80707107,347.71456697)
\curveto(229.72706593,347.65455793)(229.64706601,347.58955799)(229.56707107,347.51956697)
\curveto(229.49706616,347.44955813)(229.40706625,347.39455819)(229.29707107,347.35456697)
\curveto(229.2570664,347.34455824)(229.21706644,347.33955824)(229.17707107,347.33956697)
\curveto(229.14706651,347.33955824)(229.10706655,347.33455825)(229.05707107,347.32456697)
\curveto(229.02706663,347.31455827)(228.98706667,347.30955827)(228.93707107,347.30956697)
\curveto(228.88706677,347.31955826)(228.84206682,347.32455826)(228.80207107,347.32456697)
\lineto(228.45707107,347.32456697)
\curveto(228.33706732,347.32455826)(228.24706741,347.34955823)(228.18707107,347.39956697)
\curveto(228.12706753,347.43955814)(228.11206755,347.50955807)(228.14207107,347.60956697)
\curveto(228.1620675,347.68955789)(228.19706746,347.75955782)(228.24707107,347.81956697)
\curveto(228.29706736,347.88955769)(228.34206732,347.95955762)(228.38207107,348.02956697)
\curveto(228.48206718,348.16955741)(228.57706708,348.30455728)(228.66707107,348.43456697)
\curveto(228.7570669,348.56455702)(228.84706681,348.69955688)(228.93707107,348.83956697)
\curveto(228.98706667,348.91955666)(229.03706662,349.00455658)(229.08707107,349.09456697)
\curveto(229.14706651,349.1845564)(229.21206645,349.25455633)(229.28207107,349.30456697)
\curveto(229.32206634,349.33455625)(229.39206627,349.36955621)(229.49207107,349.40956697)
\curveto(229.51206615,349.41955616)(229.53706612,349.41955616)(229.56707107,349.40956697)
\curveto(229.60706605,349.40955617)(229.63706602,349.41455617)(229.65707107,349.42456697)
}
}
{
\newrgbcolor{curcolor}{0 0 0}
\pscustom[linestyle=none,fillstyle=solid,fillcolor=curcolor]
{
\newpath
\moveto(238.69832107,346.54456697)
\curveto(239.29831527,346.56455902)(239.79831477,346.4795591)(240.19832107,346.28956697)
\curveto(240.59831397,346.09955948)(240.91331365,345.81955976)(241.14332107,345.44956697)
\curveto(241.21331335,345.33956024)(241.2683133,345.21956036)(241.30832107,345.08956697)
\curveto(241.34831322,344.96956061)(241.38831318,344.84456074)(241.42832107,344.71456697)
\curveto(241.44831312,344.63456095)(241.45831311,344.55956102)(241.45832107,344.48956697)
\curveto(241.4683131,344.41956116)(241.48331308,344.34956123)(241.50332107,344.27956697)
\curveto(241.50331306,344.21956136)(241.50831306,344.1795614)(241.51832107,344.15956697)
\curveto(241.53831303,344.01956156)(241.54831302,343.87456171)(241.54832107,343.72456697)
\lineto(241.54832107,343.28956697)
\lineto(241.54832107,341.95456697)
\lineto(241.54832107,339.52456697)
\curveto(241.54831302,339.33456625)(241.54331302,339.14956643)(241.53332107,338.96956697)
\curveto(241.53331303,338.79956678)(241.4633131,338.68956689)(241.32332107,338.63956697)
\curveto(241.2633133,338.61956696)(241.19331337,338.60956697)(241.11332107,338.60956697)
\lineto(240.87332107,338.60956697)
\lineto(240.06332107,338.60956697)
\curveto(239.94331462,338.60956697)(239.83331473,338.61456697)(239.73332107,338.62456697)
\curveto(239.64331492,338.64456694)(239.57331499,338.68956689)(239.52332107,338.75956697)
\curveto(239.48331508,338.81956676)(239.45831511,338.89456669)(239.44832107,338.98456697)
\lineto(239.44832107,339.29956697)
\lineto(239.44832107,340.34956697)
\lineto(239.44832107,342.58456697)
\curveto(239.44831512,342.95456263)(239.43331513,343.29456229)(239.40332107,343.60456697)
\curveto(239.37331519,343.92456166)(239.28331528,344.19456139)(239.13332107,344.41456697)
\curveto(238.99331557,344.61456097)(238.78831578,344.75456083)(238.51832107,344.83456697)
\curveto(238.4683161,344.85456073)(238.41331615,344.86456072)(238.35332107,344.86456697)
\curveto(238.30331626,344.86456072)(238.24831632,344.87456071)(238.18832107,344.89456697)
\curveto(238.13831643,344.90456068)(238.07331649,344.90456068)(237.99332107,344.89456697)
\curveto(237.92331664,344.89456069)(237.8683167,344.88956069)(237.82832107,344.87956697)
\curveto(237.78831678,344.86956071)(237.75331681,344.86456072)(237.72332107,344.86456697)
\curveto(237.69331687,344.86456072)(237.6633169,344.85956072)(237.63332107,344.84956697)
\curveto(237.40331716,344.78956079)(237.21831735,344.70956087)(237.07832107,344.60956697)
\curveto(236.75831781,344.3795612)(236.568318,344.04456154)(236.50832107,343.60456697)
\curveto(236.44831812,343.16456242)(236.41831815,342.66956291)(236.41832107,342.11956697)
\lineto(236.41832107,340.24456697)
\lineto(236.41832107,339.32956697)
\lineto(236.41832107,339.05956697)
\curveto(236.41831815,338.96956661)(236.40331816,338.89456669)(236.37332107,338.83456697)
\curveto(236.32331824,338.72456686)(236.24331832,338.65956692)(236.13332107,338.63956697)
\curveto(236.02331854,338.61956696)(235.88831868,338.60956697)(235.72832107,338.60956697)
\lineto(234.97832107,338.60956697)
\curveto(234.8683197,338.60956697)(234.75831981,338.61456697)(234.64832107,338.62456697)
\curveto(234.53832003,338.63456695)(234.45832011,338.66956691)(234.40832107,338.72956697)
\curveto(234.33832023,338.81956676)(234.30332026,338.94956663)(234.30332107,339.11956697)
\curveto(234.31332025,339.28956629)(234.31832025,339.44956613)(234.31832107,339.59956697)
\lineto(234.31832107,341.63956697)
\lineto(234.31832107,344.93956697)
\lineto(234.31832107,345.70456697)
\lineto(234.31832107,346.00456697)
\curveto(234.32832024,346.09455949)(234.35832021,346.16955941)(234.40832107,346.22956697)
\curveto(234.42832014,346.25955932)(234.45832011,346.2795593)(234.49832107,346.28956697)
\curveto(234.54832002,346.30955927)(234.59831997,346.32455926)(234.64832107,346.33456697)
\lineto(234.72332107,346.33456697)
\curveto(234.77331979,346.34455924)(234.82331974,346.34955923)(234.87332107,346.34956697)
\lineto(235.03832107,346.34956697)
\lineto(235.66832107,346.34956697)
\curveto(235.74831882,346.34955923)(235.82331874,346.34455924)(235.89332107,346.33456697)
\curveto(235.97331859,346.33455925)(236.04331852,346.32455926)(236.10332107,346.30456697)
\curveto(236.17331839,346.27455931)(236.21831835,346.22955935)(236.23832107,346.16956697)
\curveto(236.2683183,346.10955947)(236.29331827,346.03955954)(236.31332107,345.95956697)
\curveto(236.32331824,345.91955966)(236.32331824,345.8845597)(236.31332107,345.85456697)
\curveto(236.31331825,345.82455976)(236.32331824,345.79455979)(236.34332107,345.76456697)
\curveto(236.3633182,345.71455987)(236.37831819,345.6845599)(236.38832107,345.67456697)
\curveto(236.40831816,345.66455992)(236.43331813,345.64955993)(236.46332107,345.62956697)
\curveto(236.57331799,345.61955996)(236.6633179,345.65455993)(236.73332107,345.73456697)
\curveto(236.80331776,345.82455976)(236.87831769,345.89455969)(236.95832107,345.94456697)
\curveto(237.22831734,346.14455944)(237.52831704,346.30455928)(237.85832107,346.42456697)
\curveto(237.94831662,346.45455913)(238.03831653,346.47455911)(238.12832107,346.48456697)
\curveto(238.22831634,346.49455909)(238.33331623,346.50955907)(238.44332107,346.52956697)
\curveto(238.47331609,346.53955904)(238.51831605,346.53955904)(238.57832107,346.52956697)
\curveto(238.63831593,346.52955905)(238.67831589,346.53455905)(238.69832107,346.54456697)
}
}
{
\newrgbcolor{curcolor}{0 0 0}
\pscustom[linestyle=none,fillstyle=solid,fillcolor=curcolor]
{
}
}
{
\newrgbcolor{curcolor}{0 0 0}
\pscustom[linestyle=none,fillstyle=solid,fillcolor=curcolor]
{
\newpath
\moveto(250.30972732,346.55956697)
\curveto(251.05972282,346.579559)(251.70972217,346.49455909)(252.25972732,346.30456697)
\curveto(252.81972106,346.12455946)(253.24472064,345.80955977)(253.53472732,345.35956697)
\curveto(253.60472028,345.24956033)(253.66472022,345.13456045)(253.71472732,345.01456697)
\curveto(253.77472011,344.90456068)(253.82472006,344.7795608)(253.86472732,344.63956697)
\curveto(253.88472,344.579561)(253.89471999,344.51456107)(253.89472732,344.44456697)
\curveto(253.89471999,344.37456121)(253.88472,344.31456127)(253.86472732,344.26456697)
\curveto(253.82472006,344.20456138)(253.76972011,344.16456142)(253.69972732,344.14456697)
\curveto(253.64972023,344.12456146)(253.58972029,344.11456147)(253.51972732,344.11456697)
\lineto(253.30972732,344.11456697)
\lineto(252.64972732,344.11456697)
\curveto(252.5797213,344.11456147)(252.50972137,344.10956147)(252.43972732,344.09956697)
\curveto(252.36972151,344.09956148)(252.30472158,344.10956147)(252.24472732,344.12956697)
\curveto(252.14472174,344.14956143)(252.06972181,344.18956139)(252.01972732,344.24956697)
\curveto(251.96972191,344.30956127)(251.92472196,344.36956121)(251.88472732,344.42956697)
\lineto(251.76472732,344.63956697)
\curveto(251.73472215,344.71956086)(251.6847222,344.7845608)(251.61472732,344.83456697)
\curveto(251.51472237,344.91456067)(251.41472247,344.97456061)(251.31472732,345.01456697)
\curveto(251.22472266,345.05456053)(251.10972277,345.08956049)(250.96972732,345.11956697)
\curveto(250.89972298,345.13956044)(250.79472309,345.15456043)(250.65472732,345.16456697)
\curveto(250.52472336,345.17456041)(250.42472346,345.16956041)(250.35472732,345.14956697)
\lineto(250.24972732,345.14956697)
\lineto(250.09972732,345.11956697)
\curveto(250.05972382,345.11956046)(250.01472387,345.11456047)(249.96472732,345.10456697)
\curveto(249.79472409,345.05456053)(249.65472423,344.9845606)(249.54472732,344.89456697)
\curveto(249.44472444,344.81456077)(249.37472451,344.68956089)(249.33472732,344.51956697)
\curveto(249.31472457,344.44956113)(249.31472457,344.3845612)(249.33472732,344.32456697)
\curveto(249.35472453,344.26456132)(249.37472451,344.21456137)(249.39472732,344.17456697)
\curveto(249.46472442,344.05456153)(249.54472434,343.95956162)(249.63472732,343.88956697)
\curveto(249.73472415,343.81956176)(249.84972403,343.75956182)(249.97972732,343.70956697)
\curveto(250.16972371,343.62956195)(250.37472351,343.55956202)(250.59472732,343.49956697)
\lineto(251.28472732,343.34956697)
\curveto(251.52472236,343.30956227)(251.75472213,343.25956232)(251.97472732,343.19956697)
\curveto(252.20472168,343.14956243)(252.41972146,343.0845625)(252.61972732,343.00456697)
\curveto(252.70972117,342.96456262)(252.79472109,342.92956265)(252.87472732,342.89956697)
\curveto(252.96472092,342.8795627)(253.04972083,342.84456274)(253.12972732,342.79456697)
\curveto(253.31972056,342.67456291)(253.48972039,342.54456304)(253.63972732,342.40456697)
\curveto(253.79972008,342.26456332)(253.92471996,342.08956349)(254.01472732,341.87956697)
\curveto(254.04471984,341.80956377)(254.06971981,341.73956384)(254.08972732,341.66956697)
\curveto(254.10971977,341.59956398)(254.12971975,341.52456406)(254.14972732,341.44456697)
\curveto(254.15971972,341.3845642)(254.16471972,341.28956429)(254.16472732,341.15956697)
\curveto(254.17471971,341.03956454)(254.17471971,340.94456464)(254.16472732,340.87456697)
\lineto(254.16472732,340.79956697)
\curveto(254.14471974,340.73956484)(254.12971975,340.6795649)(254.11972732,340.61956697)
\curveto(254.11971976,340.56956501)(254.11471977,340.51956506)(254.10472732,340.46956697)
\curveto(254.03471985,340.16956541)(253.92471996,339.90456568)(253.77472732,339.67456697)
\curveto(253.61472027,339.43456615)(253.41972046,339.23956634)(253.18972732,339.08956697)
\curveto(252.95972092,338.93956664)(252.69972118,338.80956677)(252.40972732,338.69956697)
\curveto(252.29972158,338.64956693)(252.1797217,338.61456697)(252.04972732,338.59456697)
\curveto(251.92972195,338.57456701)(251.80972207,338.54956703)(251.68972732,338.51956697)
\curveto(251.59972228,338.49956708)(251.50472238,338.48956709)(251.40472732,338.48956697)
\curveto(251.31472257,338.4795671)(251.22472266,338.46456712)(251.13472732,338.44456697)
\lineto(250.86472732,338.44456697)
\curveto(250.80472308,338.42456716)(250.69972318,338.41456717)(250.54972732,338.41456697)
\curveto(250.40972347,338.41456717)(250.30972357,338.42456716)(250.24972732,338.44456697)
\curveto(250.21972366,338.44456714)(250.1847237,338.44956713)(250.14472732,338.45956697)
\lineto(250.03972732,338.45956697)
\curveto(249.91972396,338.4795671)(249.79972408,338.49456709)(249.67972732,338.50456697)
\curveto(249.55972432,338.51456707)(249.44472444,338.53456705)(249.33472732,338.56456697)
\curveto(248.94472494,338.67456691)(248.59972528,338.79956678)(248.29972732,338.93956697)
\curveto(247.99972588,339.08956649)(247.74472614,339.30956627)(247.53472732,339.59956697)
\curveto(247.39472649,339.78956579)(247.27472661,340.00956557)(247.17472732,340.25956697)
\curveto(247.15472673,340.31956526)(247.13472675,340.39956518)(247.11472732,340.49956697)
\curveto(247.09472679,340.54956503)(247.0797268,340.61956496)(247.06972732,340.70956697)
\curveto(247.05972682,340.79956478)(247.06472682,340.87456471)(247.08472732,340.93456697)
\curveto(247.11472677,341.00456458)(247.16472672,341.05456453)(247.23472732,341.08456697)
\curveto(247.2847266,341.10456448)(247.34472654,341.11456447)(247.41472732,341.11456697)
\lineto(247.63972732,341.11456697)
\lineto(248.34472732,341.11456697)
\lineto(248.58472732,341.11456697)
\curveto(248.66472522,341.11456447)(248.73472515,341.10456448)(248.79472732,341.08456697)
\curveto(248.90472498,341.04456454)(248.97472491,340.9795646)(249.00472732,340.88956697)
\curveto(249.04472484,340.79956478)(249.08972479,340.70456488)(249.13972732,340.60456697)
\curveto(249.15972472,340.55456503)(249.19472469,340.48956509)(249.24472732,340.40956697)
\curveto(249.30472458,340.32956525)(249.35472453,340.2795653)(249.39472732,340.25956697)
\curveto(249.51472437,340.15956542)(249.62972425,340.0795655)(249.73972732,340.01956697)
\curveto(249.84972403,339.96956561)(249.98972389,339.91956566)(250.15972732,339.86956697)
\curveto(250.20972367,339.84956573)(250.25972362,339.83956574)(250.30972732,339.83956697)
\curveto(250.35972352,339.84956573)(250.40972347,339.84956573)(250.45972732,339.83956697)
\curveto(250.53972334,339.81956576)(250.62472326,339.80956577)(250.71472732,339.80956697)
\curveto(250.81472307,339.81956576)(250.89972298,339.83456575)(250.96972732,339.85456697)
\curveto(251.01972286,339.86456572)(251.06472282,339.86956571)(251.10472732,339.86956697)
\curveto(251.15472273,339.86956571)(251.20472268,339.8795657)(251.25472732,339.89956697)
\curveto(251.39472249,339.94956563)(251.51972236,340.00956557)(251.62972732,340.07956697)
\curveto(251.74972213,340.14956543)(251.84472204,340.23956534)(251.91472732,340.34956697)
\curveto(251.96472192,340.42956515)(252.00472188,340.55456503)(252.03472732,340.72456697)
\curveto(252.05472183,340.79456479)(252.05472183,340.85956472)(252.03472732,340.91956697)
\curveto(252.01472187,340.9795646)(251.99472189,341.02956455)(251.97472732,341.06956697)
\curveto(251.90472198,341.20956437)(251.81472207,341.31456427)(251.70472732,341.38456697)
\curveto(251.60472228,341.45456413)(251.4847224,341.51956406)(251.34472732,341.57956697)
\curveto(251.15472273,341.65956392)(250.95472293,341.72456386)(250.74472732,341.77456697)
\curveto(250.53472335,341.82456376)(250.32472356,341.8795637)(250.11472732,341.93956697)
\curveto(250.03472385,341.95956362)(249.94972393,341.97456361)(249.85972732,341.98456697)
\curveto(249.7797241,341.99456359)(249.69972418,342.00956357)(249.61972732,342.02956697)
\curveto(249.29972458,342.11956346)(248.99472489,342.20456338)(248.70472732,342.28456697)
\curveto(248.41472547,342.37456321)(248.14972573,342.50456308)(247.90972732,342.67456697)
\curveto(247.62972625,342.87456271)(247.42472646,343.14456244)(247.29472732,343.48456697)
\curveto(247.27472661,343.55456203)(247.25472663,343.64956193)(247.23472732,343.76956697)
\curveto(247.21472667,343.83956174)(247.19972668,343.92456166)(247.18972732,344.02456697)
\curveto(247.1797267,344.12456146)(247.1847267,344.21456137)(247.20472732,344.29456697)
\curveto(247.22472666,344.34456124)(247.22972665,344.3845612)(247.21972732,344.41456697)
\curveto(247.20972667,344.45456113)(247.21472667,344.49956108)(247.23472732,344.54956697)
\curveto(247.25472663,344.65956092)(247.27472661,344.75956082)(247.29472732,344.84956697)
\curveto(247.32472656,344.94956063)(247.35972652,345.04456054)(247.39972732,345.13456697)
\curveto(247.52972635,345.42456016)(247.70972617,345.65955992)(247.93972732,345.83956697)
\curveto(248.16972571,346.01955956)(248.42972545,346.16455942)(248.71972732,346.27456697)
\curveto(248.82972505,346.32455926)(248.94472494,346.35955922)(249.06472732,346.37956697)
\curveto(249.1847247,346.40955917)(249.30972457,346.43955914)(249.43972732,346.46956697)
\curveto(249.49972438,346.48955909)(249.55972432,346.49955908)(249.61972732,346.49956697)
\lineto(249.79972732,346.52956697)
\curveto(249.879724,346.53955904)(249.96472392,346.54455904)(250.05472732,346.54456697)
\curveto(250.14472374,346.54455904)(250.22972365,346.54955903)(250.30972732,346.55956697)
}
}
{
\newrgbcolor{curcolor}{0 0 0}
\pscustom[linestyle=none,fillstyle=solid,fillcolor=curcolor]
{
\newpath
\moveto(255.81636795,346.33456697)
\lineto(256.94136795,346.33456697)
\curveto(257.05136551,346.33455925)(257.15136541,346.32955925)(257.24136795,346.31956697)
\curveto(257.33136523,346.30955927)(257.39636517,346.27455931)(257.43636795,346.21456697)
\curveto(257.48636508,346.15455943)(257.51636505,346.06955951)(257.52636795,345.95956697)
\curveto(257.53636503,345.85955972)(257.54136502,345.75455983)(257.54136795,345.64456697)
\lineto(257.54136795,344.59456697)
\lineto(257.54136795,342.35956697)
\curveto(257.54136502,341.99956358)(257.55636501,341.65956392)(257.58636795,341.33956697)
\curveto(257.61636495,341.01956456)(257.70636486,340.75456483)(257.85636795,340.54456697)
\curveto(257.99636457,340.33456525)(258.22136434,340.1845654)(258.53136795,340.09456697)
\curveto(258.58136398,340.0845655)(258.62136394,340.0795655)(258.65136795,340.07956697)
\curveto(258.69136387,340.0795655)(258.73636383,340.07456551)(258.78636795,340.06456697)
\curveto(258.83636373,340.05456553)(258.89136367,340.04956553)(258.95136795,340.04956697)
\curveto(259.01136355,340.04956553)(259.05636351,340.05456553)(259.08636795,340.06456697)
\curveto(259.13636343,340.0845655)(259.17636339,340.08956549)(259.20636795,340.07956697)
\curveto(259.24636332,340.06956551)(259.28636328,340.07456551)(259.32636795,340.09456697)
\curveto(259.53636303,340.14456544)(259.70136286,340.20956537)(259.82136795,340.28956697)
\curveto(260.00136256,340.39956518)(260.14136242,340.53956504)(260.24136795,340.70956697)
\curveto(260.35136221,340.88956469)(260.42636214,341.0845645)(260.46636795,341.29456697)
\curveto(260.51636205,341.51456407)(260.54636202,341.75456383)(260.55636795,342.01456697)
\curveto(260.566362,342.2845633)(260.57136199,342.56456302)(260.57136795,342.85456697)
\lineto(260.57136795,344.66956697)
\lineto(260.57136795,345.64456697)
\lineto(260.57136795,345.91456697)
\curveto(260.57136199,346.01455957)(260.59136197,346.09455949)(260.63136795,346.15456697)
\curveto(260.68136188,346.24455934)(260.75636181,346.29455929)(260.85636795,346.30456697)
\curveto(260.95636161,346.32455926)(261.07636149,346.33455925)(261.21636795,346.33456697)
\lineto(262.01136795,346.33456697)
\lineto(262.29636795,346.33456697)
\curveto(262.38636018,346.33455925)(262.4613601,346.31455927)(262.52136795,346.27456697)
\curveto(262.60135996,346.22455936)(262.64635992,346.14955943)(262.65636795,346.04956697)
\curveto(262.6663599,345.94955963)(262.67135989,345.83455975)(262.67136795,345.70456697)
\lineto(262.67136795,344.56456697)
\lineto(262.67136795,340.34956697)
\lineto(262.67136795,339.28456697)
\lineto(262.67136795,338.98456697)
\curveto(262.67135989,338.8845667)(262.65135991,338.80956677)(262.61136795,338.75956697)
\curveto(262.56136,338.6795669)(262.48636008,338.63456695)(262.38636795,338.62456697)
\curveto(262.28636028,338.61456697)(262.18136038,338.60956697)(262.07136795,338.60956697)
\lineto(261.26136795,338.60956697)
\curveto(261.15136141,338.60956697)(261.05136151,338.61456697)(260.96136795,338.62456697)
\curveto(260.88136168,338.63456695)(260.81636175,338.67456691)(260.76636795,338.74456697)
\curveto(260.74636182,338.77456681)(260.72636184,338.81956676)(260.70636795,338.87956697)
\curveto(260.69636187,338.93956664)(260.68136188,338.99956658)(260.66136795,339.05956697)
\curveto(260.65136191,339.11956646)(260.63636193,339.17456641)(260.61636795,339.22456697)
\curveto(260.59636197,339.27456631)(260.566362,339.30456628)(260.52636795,339.31456697)
\curveto(260.50636206,339.33456625)(260.48136208,339.33956624)(260.45136795,339.32956697)
\curveto(260.42136214,339.31956626)(260.39636217,339.30956627)(260.37636795,339.29956697)
\curveto(260.30636226,339.25956632)(260.24636232,339.21456637)(260.19636795,339.16456697)
\curveto(260.14636242,339.11456647)(260.09136247,339.06956651)(260.03136795,339.02956697)
\curveto(259.99136257,338.99956658)(259.95136261,338.96456662)(259.91136795,338.92456697)
\curveto(259.88136268,338.89456669)(259.84136272,338.86456672)(259.79136795,338.83456697)
\curveto(259.561363,338.69456689)(259.29136327,338.584567)(258.98136795,338.50456697)
\curveto(258.91136365,338.4845671)(258.84136372,338.47456711)(258.77136795,338.47456697)
\curveto(258.70136386,338.46456712)(258.62636394,338.44956713)(258.54636795,338.42956697)
\curveto(258.50636406,338.41956716)(258.4613641,338.41956716)(258.41136795,338.42956697)
\curveto(258.37136419,338.42956715)(258.33136423,338.42456716)(258.29136795,338.41456697)
\curveto(258.2613643,338.40456718)(258.19636437,338.40456718)(258.09636795,338.41456697)
\curveto(258.00636456,338.41456717)(257.94636462,338.41956716)(257.91636795,338.42956697)
\curveto(257.8663647,338.42956715)(257.81636475,338.43456715)(257.76636795,338.44456697)
\lineto(257.61636795,338.44456697)
\curveto(257.49636507,338.47456711)(257.38136518,338.49956708)(257.27136795,338.51956697)
\curveto(257.1613654,338.53956704)(257.05136551,338.56956701)(256.94136795,338.60956697)
\curveto(256.89136567,338.62956695)(256.84636572,338.64456694)(256.80636795,338.65456697)
\curveto(256.77636579,338.67456691)(256.73636583,338.69456689)(256.68636795,338.71456697)
\curveto(256.33636623,338.90456668)(256.05636651,339.16956641)(255.84636795,339.50956697)
\curveto(255.71636685,339.71956586)(255.62136694,339.96956561)(255.56136795,340.25956697)
\curveto(255.50136706,340.55956502)(255.4613671,340.87456471)(255.44136795,341.20456697)
\curveto(255.43136713,341.54456404)(255.42636714,341.88956369)(255.42636795,342.23956697)
\curveto(255.43636713,342.59956298)(255.44136712,342.95456263)(255.44136795,343.30456697)
\lineto(255.44136795,345.34456697)
\curveto(255.44136712,345.47456011)(255.43636713,345.62455996)(255.42636795,345.79456697)
\curveto(255.42636714,345.97455961)(255.45136711,346.10455948)(255.50136795,346.18456697)
\curveto(255.53136703,346.23455935)(255.59136697,346.2795593)(255.68136795,346.31956697)
\curveto(255.74136682,346.31955926)(255.78636678,346.32455926)(255.81636795,346.33456697)
}
}
{
\newrgbcolor{curcolor}{0 0 0}
\pscustom[linestyle=none,fillstyle=solid,fillcolor=curcolor]
{
}
}
{
\newrgbcolor{curcolor}{0 0 0}
\pscustom[linestyle=none,fillstyle=solid,fillcolor=curcolor]
{
\newpath
\moveto(269.5127742,348.65956697)
\lineto(270.5177742,348.65956697)
\curveto(270.66777121,348.65955692)(270.79777108,348.64955693)(270.9077742,348.62956697)
\curveto(271.02777085,348.61955696)(271.11277077,348.55955702)(271.1627742,348.44956697)
\curveto(271.1827707,348.39955718)(271.19277069,348.33955724)(271.1927742,348.26956697)
\lineto(271.1927742,348.05956697)
\lineto(271.1927742,347.38456697)
\curveto(271.19277069,347.33455825)(271.18777069,347.27455831)(271.1777742,347.20456697)
\curveto(271.1777707,347.14455844)(271.1827707,347.08955849)(271.1927742,347.03956697)
\lineto(271.1927742,346.87456697)
\curveto(271.19277069,346.79455879)(271.19777068,346.71955886)(271.2077742,346.64956697)
\curveto(271.21777066,346.58955899)(271.24277064,346.53455905)(271.2827742,346.48456697)
\curveto(271.35277053,346.39455919)(271.4777704,346.34455924)(271.6577742,346.33456697)
\lineto(272.1977742,346.33456697)
\lineto(272.3777742,346.33456697)
\curveto(272.43776944,346.33455925)(272.49276939,346.32455926)(272.5427742,346.30456697)
\curveto(272.65276923,346.25455933)(272.71276917,346.16455942)(272.7227742,346.03456697)
\curveto(272.74276914,345.90455968)(272.75276913,345.75955982)(272.7527742,345.59956697)
\lineto(272.7527742,345.38956697)
\curveto(272.76276912,345.31956026)(272.75776912,345.25956032)(272.7377742,345.20956697)
\curveto(272.68776919,345.04956053)(272.5827693,344.96456062)(272.4227742,344.95456697)
\curveto(272.26276962,344.94456064)(272.0827698,344.93956064)(271.8827742,344.93956697)
\lineto(271.7477742,344.93956697)
\curveto(271.70777017,344.94956063)(271.67277021,344.94956063)(271.6427742,344.93956697)
\curveto(271.60277028,344.92956065)(271.56777031,344.92456066)(271.5377742,344.92456697)
\curveto(271.50777037,344.93456065)(271.4777704,344.92956065)(271.4477742,344.90956697)
\curveto(271.36777051,344.88956069)(271.30777057,344.84456074)(271.2677742,344.77456697)
\curveto(271.23777064,344.71456087)(271.21277067,344.63956094)(271.1927742,344.54956697)
\curveto(271.1827707,344.49956108)(271.1827707,344.44456114)(271.1927742,344.38456697)
\curveto(271.20277068,344.32456126)(271.20277068,344.26956131)(271.1927742,344.21956697)
\lineto(271.1927742,343.28956697)
\lineto(271.1927742,341.53456697)
\curveto(271.19277069,341.2845643)(271.19777068,341.06456452)(271.2077742,340.87456697)
\curveto(271.22777065,340.69456489)(271.29277059,340.53456505)(271.4027742,340.39456697)
\curveto(271.45277043,340.33456525)(271.51777036,340.28956529)(271.5977742,340.25956697)
\lineto(271.8677742,340.19956697)
\curveto(271.89776998,340.18956539)(271.92776995,340.1845654)(271.9577742,340.18456697)
\curveto(271.99776988,340.19456539)(272.02776985,340.19456539)(272.0477742,340.18456697)
\lineto(272.2127742,340.18456697)
\curveto(272.32276956,340.1845654)(272.41776946,340.1795654)(272.4977742,340.16956697)
\curveto(272.5777693,340.15956542)(272.64276924,340.11956546)(272.6927742,340.04956697)
\curveto(272.73276915,339.98956559)(272.75276913,339.90956567)(272.7527742,339.80956697)
\lineto(272.7527742,339.52456697)
\curveto(272.75276913,339.31456627)(272.74776913,339.11956646)(272.7377742,338.93956697)
\curveto(272.73776914,338.76956681)(272.65776922,338.65456693)(272.4977742,338.59456697)
\curveto(272.44776943,338.57456701)(272.40276948,338.56956701)(272.3627742,338.57956697)
\curveto(272.32276956,338.579567)(272.2777696,338.56956701)(272.2277742,338.54956697)
\lineto(272.0777742,338.54956697)
\curveto(272.05776982,338.54956703)(272.02776985,338.55456703)(271.9877742,338.56456697)
\curveto(271.94776993,338.56456702)(271.91276997,338.55956702)(271.8827742,338.54956697)
\curveto(271.83277005,338.53956704)(271.7777701,338.53956704)(271.7177742,338.54956697)
\lineto(271.5677742,338.54956697)
\lineto(271.4177742,338.54956697)
\curveto(271.36777051,338.53956704)(271.32277056,338.53956704)(271.2827742,338.54956697)
\lineto(271.1177742,338.54956697)
\curveto(271.06777081,338.55956702)(271.01277087,338.56456702)(270.9527742,338.56456697)
\curveto(270.89277099,338.56456702)(270.83777104,338.56956701)(270.7877742,338.57956697)
\curveto(270.71777116,338.58956699)(270.65277123,338.59956698)(270.5927742,338.60956697)
\lineto(270.4127742,338.63956697)
\curveto(270.30277158,338.66956691)(270.19777168,338.70456688)(270.0977742,338.74456697)
\curveto(269.99777188,338.7845668)(269.90277198,338.82956675)(269.8127742,338.87956697)
\lineto(269.7227742,338.93956697)
\curveto(269.69277219,338.96956661)(269.65777222,338.99956658)(269.6177742,339.02956697)
\curveto(269.59777228,339.04956653)(269.57277231,339.06956651)(269.5427742,339.08956697)
\lineto(269.4677742,339.16456697)
\curveto(269.32777255,339.35456623)(269.22277266,339.56456602)(269.1527742,339.79456697)
\curveto(269.13277275,339.83456575)(269.12277276,339.86956571)(269.1227742,339.89956697)
\curveto(269.13277275,339.93956564)(269.13277275,339.9845656)(269.1227742,340.03456697)
\curveto(269.11277277,340.05456553)(269.10777277,340.0795655)(269.1077742,340.10956697)
\curveto(269.10777277,340.13956544)(269.10277278,340.16456542)(269.0927742,340.18456697)
\lineto(269.0927742,340.33456697)
\curveto(269.0827728,340.37456521)(269.0777728,340.41956516)(269.0777742,340.46956697)
\curveto(269.08777279,340.51956506)(269.09277279,340.56956501)(269.0927742,340.61956697)
\lineto(269.0927742,341.18956697)
\lineto(269.0927742,343.42456697)
\lineto(269.0927742,344.21956697)
\lineto(269.0927742,344.42956697)
\curveto(269.10277278,344.49956108)(269.09777278,344.56456102)(269.0777742,344.62456697)
\curveto(269.03777284,344.76456082)(268.96777291,344.85456073)(268.8677742,344.89456697)
\curveto(268.75777312,344.94456064)(268.61777326,344.95956062)(268.4477742,344.93956697)
\curveto(268.2777736,344.91956066)(268.13277375,344.93456065)(268.0127742,344.98456697)
\curveto(267.93277395,345.01456057)(267.882774,345.05956052)(267.8627742,345.11956697)
\curveto(267.84277404,345.1795604)(267.82277406,345.25456033)(267.8027742,345.34456697)
\lineto(267.8027742,345.65956697)
\curveto(267.80277408,345.83955974)(267.81277407,345.9845596)(267.8327742,346.09456697)
\curveto(267.85277403,346.20455938)(267.93777394,346.2795593)(268.0877742,346.31956697)
\curveto(268.12777375,346.33955924)(268.16777371,346.34455924)(268.2077742,346.33456697)
\lineto(268.3427742,346.33456697)
\curveto(268.49277339,346.33455925)(268.63277325,346.33955924)(268.7627742,346.34956697)
\curveto(268.89277299,346.36955921)(268.9827729,346.42955915)(269.0327742,346.52956697)
\curveto(269.06277282,346.59955898)(269.0777728,346.6795589)(269.0777742,346.76956697)
\curveto(269.08777279,346.85955872)(269.09277279,346.94955863)(269.0927742,347.03956697)
\lineto(269.0927742,347.96956697)
\lineto(269.0927742,348.22456697)
\curveto(269.09277279,348.31455727)(269.10277278,348.38955719)(269.1227742,348.44956697)
\curveto(269.17277271,348.54955703)(269.24777263,348.61455697)(269.3477742,348.64456697)
\curveto(269.36777251,348.65455693)(269.39277249,348.65455693)(269.4227742,348.64456697)
\curveto(269.46277242,348.64455694)(269.49277239,348.64955693)(269.5127742,348.65956697)
}
}
{
\newrgbcolor{curcolor}{0 0 0}
\pscustom[linestyle=none,fillstyle=solid,fillcolor=curcolor]
{
\newpath
\moveto(275.8362117,349.19956697)
\curveto(275.90620875,349.11955646)(275.94120871,348.99955658)(275.9412117,348.83956697)
\lineto(275.9412117,348.37456697)
\lineto(275.9412117,347.96956697)
\curveto(275.94120871,347.82955775)(275.90620875,347.73455785)(275.8362117,347.68456697)
\curveto(275.77620888,347.63455795)(275.69620896,347.60455798)(275.5962117,347.59456697)
\curveto(275.50620915,347.584558)(275.40620925,347.579558)(275.2962117,347.57956697)
\lineto(274.4562117,347.57956697)
\curveto(274.34621031,347.579558)(274.24621041,347.584558)(274.1562117,347.59456697)
\curveto(274.07621058,347.60455798)(274.00621065,347.63455795)(273.9462117,347.68456697)
\curveto(273.90621075,347.71455787)(273.87621078,347.76955781)(273.8562117,347.84956697)
\curveto(273.84621081,347.93955764)(273.83621082,348.03455755)(273.8262117,348.13456697)
\lineto(273.8262117,348.46456697)
\curveto(273.83621082,348.57455701)(273.84121081,348.66955691)(273.8412117,348.74956697)
\lineto(273.8412117,348.95956697)
\curveto(273.8512108,349.02955655)(273.87121078,349.08955649)(273.9012117,349.13956697)
\curveto(273.92121073,349.1795564)(273.94621071,349.20955637)(273.9762117,349.22956697)
\lineto(274.0962117,349.28956697)
\curveto(274.11621054,349.28955629)(274.14121051,349.28955629)(274.1712117,349.28956697)
\curveto(274.20121045,349.29955628)(274.22621043,349.30455628)(274.2462117,349.30456697)
\lineto(275.3412117,349.30456697)
\curveto(275.44120921,349.30455628)(275.53620912,349.29955628)(275.6262117,349.28956697)
\curveto(275.71620894,349.2795563)(275.78620887,349.24955633)(275.8362117,349.19956697)
\moveto(275.9412117,339.43456697)
\curveto(275.94120871,339.23456635)(275.93620872,339.06456652)(275.9262117,338.92456697)
\curveto(275.91620874,338.7845668)(275.82620883,338.68956689)(275.6562117,338.63956697)
\curveto(275.59620906,338.61956696)(275.53120912,338.60956697)(275.4612117,338.60956697)
\curveto(275.39120926,338.61956696)(275.31620934,338.62456696)(275.2362117,338.62456697)
\lineto(274.3962117,338.62456697)
\curveto(274.30621035,338.62456696)(274.21621044,338.62956695)(274.1262117,338.63956697)
\curveto(274.04621061,338.64956693)(273.98621067,338.6795669)(273.9462117,338.72956697)
\curveto(273.88621077,338.79956678)(273.8512108,338.8845667)(273.8412117,338.98456697)
\lineto(273.8412117,339.32956697)
\lineto(273.8412117,345.65956697)
\lineto(273.8412117,345.95956697)
\curveto(273.84121081,346.05955952)(273.86121079,346.13955944)(273.9012117,346.19956697)
\curveto(273.96121069,346.26955931)(274.04621061,346.31455927)(274.1562117,346.33456697)
\curveto(274.17621048,346.34455924)(274.20121045,346.34455924)(274.2312117,346.33456697)
\curveto(274.27121038,346.33455925)(274.30121035,346.33955924)(274.3212117,346.34956697)
\lineto(275.0712117,346.34956697)
\lineto(275.2662117,346.34956697)
\curveto(275.34620931,346.35955922)(275.41120924,346.35955922)(275.4612117,346.34956697)
\lineto(275.5812117,346.34956697)
\curveto(275.64120901,346.32955925)(275.69620896,346.31455927)(275.7462117,346.30456697)
\curveto(275.79620886,346.29455929)(275.83620882,346.26455932)(275.8662117,346.21456697)
\curveto(275.90620875,346.16455942)(275.92620873,346.09455949)(275.9262117,346.00456697)
\curveto(275.93620872,345.91455967)(275.94120871,345.81955976)(275.9412117,345.71956697)
\lineto(275.9412117,339.43456697)
}
}
{
\newrgbcolor{curcolor}{0 0 0}
\pscustom[linestyle=none,fillstyle=solid,fillcolor=curcolor]
{
\newpath
\moveto(285.4933992,342.56956697)
\curveto(285.50339052,342.50956307)(285.50839051,342.41956316)(285.5083992,342.29956697)
\curveto(285.50839051,342.1795634)(285.49839052,342.09456349)(285.4783992,342.04456697)
\lineto(285.4783992,341.84956697)
\curveto(285.44839057,341.73956384)(285.42839059,341.63456395)(285.4183992,341.53456697)
\curveto(285.4183906,341.43456415)(285.40339062,341.33456425)(285.3733992,341.23456697)
\curveto(285.35339067,341.14456444)(285.33339069,341.04956453)(285.3133992,340.94956697)
\curveto(285.29339073,340.85956472)(285.26339076,340.76956481)(285.2233992,340.67956697)
\curveto(285.15339087,340.50956507)(285.08339094,340.34956523)(285.0133992,340.19956697)
\curveto(284.94339108,340.05956552)(284.86339116,339.91956566)(284.7733992,339.77956697)
\curveto(284.71339131,339.68956589)(284.64839137,339.60456598)(284.5783992,339.52456697)
\curveto(284.5183915,339.45456613)(284.44839157,339.3795662)(284.3683992,339.29956697)
\lineto(284.2633992,339.19456697)
\curveto(284.21339181,339.14456644)(284.15839186,339.09956648)(284.0983992,339.05956697)
\lineto(283.9483992,338.93956697)
\curveto(283.86839215,338.8795667)(283.77839224,338.82456676)(283.6783992,338.77456697)
\curveto(283.58839243,338.73456685)(283.49339253,338.68956689)(283.3933992,338.63956697)
\curveto(283.29339273,338.58956699)(283.18839283,338.55456703)(283.0783992,338.53456697)
\curveto(282.97839304,338.51456707)(282.87339315,338.49456709)(282.7633992,338.47456697)
\curveto(282.70339332,338.45456713)(282.63839338,338.44456714)(282.5683992,338.44456697)
\curveto(282.50839351,338.44456714)(282.44339358,338.43456715)(282.3733992,338.41456697)
\lineto(282.2383992,338.41456697)
\curveto(282.15839386,338.39456719)(282.08339394,338.39456719)(282.0133992,338.41456697)
\lineto(281.8633992,338.41456697)
\curveto(281.80339422,338.43456715)(281.73839428,338.44456714)(281.6683992,338.44456697)
\curveto(281.60839441,338.43456715)(281.54839447,338.43956714)(281.4883992,338.45956697)
\curveto(281.32839469,338.50956707)(281.17339485,338.55456703)(281.0233992,338.59456697)
\curveto(280.88339514,338.63456695)(280.75339527,338.69456689)(280.6333992,338.77456697)
\curveto(280.56339546,338.81456677)(280.49839552,338.85456673)(280.4383992,338.89456697)
\curveto(280.37839564,338.94456664)(280.31339571,338.99456659)(280.2433992,339.04456697)
\lineto(280.0633992,339.17956697)
\curveto(279.98339604,339.23956634)(279.91339611,339.24456634)(279.8533992,339.19456697)
\curveto(279.80339622,339.16456642)(279.77839624,339.12456646)(279.7783992,339.07456697)
\curveto(279.77839624,339.03456655)(279.76839625,338.9845666)(279.7483992,338.92456697)
\curveto(279.72839629,338.82456676)(279.7183963,338.70956687)(279.7183992,338.57956697)
\curveto(279.72839629,338.44956713)(279.73339629,338.32956725)(279.7333992,338.21956697)
\lineto(279.7333992,336.68956697)
\curveto(279.73339629,336.55956902)(279.72839629,336.43456915)(279.7183992,336.31456697)
\curveto(279.7183963,336.1845694)(279.69339633,336.0795695)(279.6433992,335.99956697)
\curveto(279.61339641,335.95956962)(279.55839646,335.92956965)(279.4783992,335.90956697)
\curveto(279.39839662,335.88956969)(279.30839671,335.8795697)(279.2083992,335.87956697)
\curveto(279.10839691,335.86956971)(279.00839701,335.86956971)(278.9083992,335.87956697)
\lineto(278.6533992,335.87956697)
\lineto(278.2483992,335.87956697)
\lineto(278.1433992,335.87956697)
\curveto(278.10339792,335.8795697)(278.06839795,335.8845697)(278.0383992,335.89456697)
\lineto(277.9183992,335.89456697)
\curveto(277.74839827,335.94456964)(277.65839836,336.04456954)(277.6483992,336.19456697)
\curveto(277.63839838,336.33456925)(277.63339839,336.50456908)(277.6333992,336.70456697)
\lineto(277.6333992,345.50956697)
\curveto(277.63339839,345.61955996)(277.62839839,345.73455985)(277.6183992,345.85456697)
\curveto(277.6183984,345.9845596)(277.64339838,346.0845595)(277.6933992,346.15456697)
\curveto(277.73339829,346.22455936)(277.78839823,346.26955931)(277.8583992,346.28956697)
\curveto(277.90839811,346.30955927)(277.96839805,346.31955926)(278.0383992,346.31956697)
\lineto(278.2633992,346.31956697)
\lineto(278.9833992,346.31956697)
\lineto(279.2683992,346.31956697)
\curveto(279.35839666,346.31955926)(279.43339659,346.29455929)(279.4933992,346.24456697)
\curveto(279.56339646,346.19455939)(279.59839642,346.12955945)(279.5983992,346.04956697)
\curveto(279.60839641,345.9795596)(279.63339639,345.90455968)(279.6733992,345.82456697)
\curveto(279.68339634,345.79455979)(279.69339633,345.76955981)(279.7033992,345.74956697)
\curveto(279.7233963,345.73955984)(279.74339628,345.72455986)(279.7633992,345.70456697)
\curveto(279.87339615,345.69455989)(279.96339606,345.72455986)(280.0333992,345.79456697)
\curveto(280.10339592,345.86455972)(280.17339585,345.92455966)(280.2433992,345.97456697)
\curveto(280.37339565,346.06455952)(280.50839551,346.14455944)(280.6483992,346.21456697)
\curveto(280.78839523,346.29455929)(280.94339508,346.35955922)(281.1133992,346.40956697)
\curveto(281.19339483,346.43955914)(281.27839474,346.45955912)(281.3683992,346.46956697)
\curveto(281.46839455,346.4795591)(281.56339446,346.49455909)(281.6533992,346.51456697)
\curveto(281.69339433,346.52455906)(281.73339429,346.52455906)(281.7733992,346.51456697)
\curveto(281.8233942,346.50455908)(281.86339416,346.50955907)(281.8933992,346.52956697)
\curveto(282.46339356,346.54955903)(282.94339308,346.46955911)(283.3333992,346.28956697)
\curveto(283.73339229,346.11955946)(284.07339195,345.89455969)(284.3533992,345.61456697)
\curveto(284.40339162,345.56456002)(284.44839157,345.51456007)(284.4883992,345.46456697)
\curveto(284.52839149,345.42456016)(284.56839145,345.3795602)(284.6083992,345.32956697)
\curveto(284.67839134,345.23956034)(284.73839128,345.14956043)(284.7883992,345.05956697)
\curveto(284.84839117,344.96956061)(284.90339112,344.8795607)(284.9533992,344.78956697)
\curveto(284.97339105,344.76956081)(284.98339104,344.74456084)(284.9833992,344.71456697)
\curveto(284.99339103,344.6845609)(285.00839101,344.64956093)(285.0283992,344.60956697)
\curveto(285.08839093,344.50956107)(285.14339088,344.38956119)(285.1933992,344.24956697)
\curveto(285.21339081,344.18956139)(285.23339079,344.12456146)(285.2533992,344.05456697)
\curveto(285.27339075,343.99456159)(285.29339073,343.92956165)(285.3133992,343.85956697)
\curveto(285.35339067,343.73956184)(285.37839064,343.61456197)(285.3883992,343.48456697)
\curveto(285.40839061,343.35456223)(285.43339059,343.21956236)(285.4633992,343.07956697)
\lineto(285.4633992,342.91456697)
\lineto(285.4933992,342.73456697)
\lineto(285.4933992,342.56956697)
\moveto(283.3783992,342.22456697)
\curveto(283.38839263,342.27456331)(283.39339263,342.33956324)(283.3933992,342.41956697)
\curveto(283.39339263,342.50956307)(283.38839263,342.579563)(283.3783992,342.62956697)
\lineto(283.3783992,342.76456697)
\curveto(283.35839266,342.82456276)(283.34839267,342.88956269)(283.3483992,342.95956697)
\curveto(283.34839267,343.02956255)(283.33839268,343.09956248)(283.3183992,343.16956697)
\curveto(283.29839272,343.26956231)(283.27839274,343.36456222)(283.2583992,343.45456697)
\curveto(283.23839278,343.55456203)(283.20839281,343.64456194)(283.1683992,343.72456697)
\curveto(283.04839297,344.04456154)(282.89339313,344.29956128)(282.7033992,344.48956697)
\curveto(282.51339351,344.6795609)(282.24339378,344.81956076)(281.8933992,344.90956697)
\curveto(281.81339421,344.92956065)(281.7233943,344.93956064)(281.6233992,344.93956697)
\lineto(281.3533992,344.93956697)
\curveto(281.31339471,344.92956065)(281.27839474,344.92456066)(281.2483992,344.92456697)
\curveto(281.2183948,344.92456066)(281.18339484,344.91956066)(281.1433992,344.90956697)
\lineto(280.9333992,344.84956697)
\curveto(280.87339515,344.83956074)(280.81339521,344.81956076)(280.7533992,344.78956697)
\curveto(280.49339553,344.6795609)(280.28839573,344.50956107)(280.1383992,344.27956697)
\curveto(279.99839602,344.04956153)(279.88339614,343.79456179)(279.7933992,343.51456697)
\curveto(279.77339625,343.43456215)(279.75839626,343.34956223)(279.7483992,343.25956697)
\curveto(279.73839628,343.1795624)(279.7233963,343.09956248)(279.7033992,343.01956697)
\curveto(279.69339633,342.9795626)(279.68839633,342.91456267)(279.6883992,342.82456697)
\curveto(279.66839635,342.7845628)(279.66339636,342.73456285)(279.6733992,342.67456697)
\curveto(279.68339634,342.62456296)(279.68339634,342.57456301)(279.6733992,342.52456697)
\curveto(279.65339637,342.46456312)(279.65339637,342.40956317)(279.6733992,342.35956697)
\lineto(279.6733992,342.17956697)
\lineto(279.6733992,342.04456697)
\curveto(279.67339635,342.00456358)(279.68339634,341.96456362)(279.7033992,341.92456697)
\curveto(279.70339632,341.85456373)(279.70839631,341.79956378)(279.7183992,341.75956697)
\lineto(279.7483992,341.57956697)
\curveto(279.75839626,341.51956406)(279.77339625,341.45956412)(279.7933992,341.39956697)
\curveto(279.88339614,341.10956447)(279.98839603,340.86956471)(280.1083992,340.67956697)
\curveto(280.23839578,340.49956508)(280.4183956,340.33956524)(280.6483992,340.19956697)
\curveto(280.78839523,340.11956546)(280.95339507,340.05456553)(281.1433992,340.00456697)
\curveto(281.18339484,339.99456559)(281.2183948,339.98956559)(281.2483992,339.98956697)
\curveto(281.27839474,339.99956558)(281.31339471,339.99956558)(281.3533992,339.98956697)
\curveto(281.39339463,339.9795656)(281.45339457,339.96956561)(281.5333992,339.95956697)
\curveto(281.61339441,339.95956562)(281.67839434,339.96456562)(281.7283992,339.97456697)
\curveto(281.80839421,339.99456559)(281.88839413,340.00956557)(281.9683992,340.01956697)
\curveto(282.05839396,340.03956554)(282.14339388,340.06456552)(282.2233992,340.09456697)
\curveto(282.46339356,340.19456539)(282.65839336,340.33456525)(282.8083992,340.51456697)
\curveto(282.95839306,340.69456489)(283.08339294,340.90456468)(283.1833992,341.14456697)
\curveto(283.23339279,341.26456432)(283.26839275,341.38956419)(283.2883992,341.51956697)
\curveto(283.30839271,341.64956393)(283.33339269,341.7845638)(283.3633992,341.92456697)
\lineto(283.3633992,342.07456697)
\curveto(283.37339265,342.12456346)(283.37839264,342.17456341)(283.3783992,342.22456697)
}
}
{
\newrgbcolor{curcolor}{0 0 0}
\pscustom[linestyle=none,fillstyle=solid,fillcolor=curcolor]
{
\newpath
\moveto(294.54332107,342.79456697)
\curveto(294.5633125,342.73456285)(294.57331249,342.64956293)(294.57332107,342.53956697)
\curveto(294.57331249,342.42956315)(294.5633125,342.34456324)(294.54332107,342.28456697)
\lineto(294.54332107,342.13456697)
\curveto(294.52331254,342.05456353)(294.51331255,341.97456361)(294.51332107,341.89456697)
\curveto(294.52331254,341.81456377)(294.51831255,341.73456385)(294.49832107,341.65456697)
\curveto(294.47831259,341.584564)(294.4633126,341.51956406)(294.45332107,341.45956697)
\curveto(294.44331262,341.39956418)(294.43331263,341.33456425)(294.42332107,341.26456697)
\curveto(294.38331268,341.15456443)(294.34831272,341.03956454)(294.31832107,340.91956697)
\curveto(294.28831278,340.80956477)(294.24831282,340.70456488)(294.19832107,340.60456697)
\curveto(293.98831308,340.12456546)(293.71331335,339.73456585)(293.37332107,339.43456697)
\curveto(293.03331403,339.13456645)(292.62331444,338.8845667)(292.14332107,338.68456697)
\curveto(292.02331504,338.63456695)(291.89831517,338.59956698)(291.76832107,338.57956697)
\curveto(291.64831542,338.54956703)(291.52331554,338.51956706)(291.39332107,338.48956697)
\curveto(291.34331572,338.46956711)(291.28831578,338.45956712)(291.22832107,338.45956697)
\curveto(291.1683159,338.45956712)(291.11331595,338.45456713)(291.06332107,338.44456697)
\lineto(290.95832107,338.44456697)
\curveto(290.92831614,338.43456715)(290.89831617,338.42956715)(290.86832107,338.42956697)
\curveto(290.81831625,338.41956716)(290.73831633,338.41456717)(290.62832107,338.41456697)
\curveto(290.51831655,338.40456718)(290.43331663,338.40956717)(290.37332107,338.42956697)
\lineto(290.22332107,338.42956697)
\curveto(290.17331689,338.43956714)(290.11831695,338.44456714)(290.05832107,338.44456697)
\curveto(290.00831706,338.43456715)(289.95831711,338.43956714)(289.90832107,338.45956697)
\curveto(289.8683172,338.46956711)(289.82831724,338.47456711)(289.78832107,338.47456697)
\curveto(289.75831731,338.47456711)(289.71831735,338.4795671)(289.66832107,338.48956697)
\curveto(289.5683175,338.51956706)(289.4683176,338.54456704)(289.36832107,338.56456697)
\curveto(289.2683178,338.584567)(289.17331789,338.61456697)(289.08332107,338.65456697)
\curveto(288.9633181,338.69456689)(288.84831822,338.73456685)(288.73832107,338.77456697)
\curveto(288.63831843,338.81456677)(288.53331853,338.86456672)(288.42332107,338.92456697)
\curveto(288.07331899,339.13456645)(287.77331929,339.3795662)(287.52332107,339.65956697)
\curveto(287.27331979,339.93956564)(287.06332,340.27456531)(286.89332107,340.66456697)
\curveto(286.84332022,340.75456483)(286.80332026,340.84956473)(286.77332107,340.94956697)
\curveto(286.75332031,341.04956453)(286.72832034,341.15456443)(286.69832107,341.26456697)
\curveto(286.67832039,341.31456427)(286.6683204,341.35956422)(286.66832107,341.39956697)
\curveto(286.6683204,341.43956414)(286.65832041,341.4845641)(286.63832107,341.53456697)
\curveto(286.61832045,341.61456397)(286.60832046,341.69456389)(286.60832107,341.77456697)
\curveto(286.60832046,341.86456372)(286.59832047,341.94956363)(286.57832107,342.02956697)
\curveto(286.5683205,342.0795635)(286.5633205,342.12456346)(286.56332107,342.16456697)
\lineto(286.56332107,342.29956697)
\curveto(286.54332052,342.35956322)(286.53332053,342.44456314)(286.53332107,342.55456697)
\curveto(286.54332052,342.66456292)(286.55832051,342.74956283)(286.57832107,342.80956697)
\lineto(286.57832107,342.91456697)
\curveto(286.58832048,342.96456262)(286.58832048,343.01456257)(286.57832107,343.06456697)
\curveto(286.57832049,343.12456246)(286.58832048,343.1795624)(286.60832107,343.22956697)
\curveto(286.61832045,343.2795623)(286.62332044,343.32456226)(286.62332107,343.36456697)
\curveto(286.62332044,343.41456217)(286.63332043,343.46456212)(286.65332107,343.51456697)
\curveto(286.69332037,343.64456194)(286.72832034,343.76956181)(286.75832107,343.88956697)
\curveto(286.78832028,344.01956156)(286.82832024,344.14456144)(286.87832107,344.26456697)
\curveto(287.05832001,344.67456091)(287.27331979,345.01456057)(287.52332107,345.28456697)
\curveto(287.77331929,345.56456002)(288.07831899,345.81955976)(288.43832107,346.04956697)
\curveto(288.53831853,346.09955948)(288.64331842,346.14455944)(288.75332107,346.18456697)
\curveto(288.8633182,346.22455936)(288.97331809,346.26955931)(289.08332107,346.31956697)
\curveto(289.21331785,346.36955921)(289.34831772,346.40455918)(289.48832107,346.42456697)
\curveto(289.62831744,346.44455914)(289.77331729,346.47455911)(289.92332107,346.51456697)
\curveto(290.00331706,346.52455906)(290.07831699,346.52955905)(290.14832107,346.52956697)
\curveto(290.21831685,346.52955905)(290.28831678,346.53455905)(290.35832107,346.54456697)
\curveto(290.93831613,346.55455903)(291.43831563,346.49455909)(291.85832107,346.36456697)
\curveto(292.28831478,346.23455935)(292.6683144,346.05455953)(292.99832107,345.82456697)
\curveto(293.10831396,345.74455984)(293.21831385,345.65455993)(293.32832107,345.55456697)
\curveto(293.44831362,345.46456012)(293.54831352,345.36456022)(293.62832107,345.25456697)
\curveto(293.70831336,345.15456043)(293.77831329,345.05456053)(293.83832107,344.95456697)
\curveto(293.90831316,344.85456073)(293.97831309,344.74956083)(294.04832107,344.63956697)
\curveto(294.11831295,344.52956105)(294.17331289,344.40956117)(294.21332107,344.27956697)
\curveto(294.25331281,344.15956142)(294.29831277,344.02956155)(294.34832107,343.88956697)
\curveto(294.37831269,343.80956177)(294.40331266,343.72456186)(294.42332107,343.63456697)
\lineto(294.48332107,343.36456697)
\curveto(294.49331257,343.32456226)(294.49831257,343.2845623)(294.49832107,343.24456697)
\curveto(294.49831257,343.20456238)(294.50331256,343.16456242)(294.51332107,343.12456697)
\curveto(294.53331253,343.07456251)(294.53831253,343.01956256)(294.52832107,342.95956697)
\curveto(294.51831255,342.89956268)(294.52331254,342.84456274)(294.54332107,342.79456697)
\moveto(292.44332107,342.25456697)
\curveto(292.45331461,342.30456328)(292.45831461,342.37456321)(292.45832107,342.46456697)
\curveto(292.45831461,342.56456302)(292.45331461,342.63956294)(292.44332107,342.68956697)
\lineto(292.44332107,342.80956697)
\curveto(292.42331464,342.85956272)(292.41331465,342.91456267)(292.41332107,342.97456697)
\curveto(292.41331465,343.03456255)(292.40831466,343.08956249)(292.39832107,343.13956697)
\curveto(292.39831467,343.1795624)(292.39331467,343.20956237)(292.38332107,343.22956697)
\lineto(292.32332107,343.46956697)
\curveto(292.31331475,343.55956202)(292.29331477,343.64456194)(292.26332107,343.72456697)
\curveto(292.15331491,343.9845616)(292.02331504,344.20456138)(291.87332107,344.38456697)
\curveto(291.72331534,344.57456101)(291.52331554,344.72456086)(291.27332107,344.83456697)
\curveto(291.21331585,344.85456073)(291.15331591,344.86956071)(291.09332107,344.87956697)
\curveto(291.03331603,344.89956068)(290.9683161,344.91956066)(290.89832107,344.93956697)
\curveto(290.81831625,344.95956062)(290.73331633,344.96456062)(290.64332107,344.95456697)
\lineto(290.37332107,344.95456697)
\curveto(290.34331672,344.93456065)(290.30831676,344.92456066)(290.26832107,344.92456697)
\curveto(290.22831684,344.93456065)(290.19331687,344.93456065)(290.16332107,344.92456697)
\lineto(289.95332107,344.86456697)
\curveto(289.89331717,344.85456073)(289.83831723,344.83456075)(289.78832107,344.80456697)
\curveto(289.53831753,344.69456089)(289.33331773,344.53456105)(289.17332107,344.32456697)
\curveto(289.02331804,344.12456146)(288.90331816,343.88956169)(288.81332107,343.61956697)
\curveto(288.78331828,343.51956206)(288.75831831,343.41456217)(288.73832107,343.30456697)
\curveto(288.72831834,343.19456239)(288.71331835,343.0845625)(288.69332107,342.97456697)
\curveto(288.68331838,342.92456266)(288.67831839,342.87456271)(288.67832107,342.82456697)
\lineto(288.67832107,342.67456697)
\curveto(288.65831841,342.60456298)(288.64831842,342.49956308)(288.64832107,342.35956697)
\curveto(288.65831841,342.21956336)(288.67331839,342.11456347)(288.69332107,342.04456697)
\lineto(288.69332107,341.90956697)
\curveto(288.71331835,341.82956375)(288.72831834,341.74956383)(288.73832107,341.66956697)
\curveto(288.74831832,341.59956398)(288.7633183,341.52456406)(288.78332107,341.44456697)
\curveto(288.88331818,341.14456444)(288.98831808,340.89956468)(289.09832107,340.70956697)
\curveto(289.21831785,340.52956505)(289.40331766,340.36456522)(289.65332107,340.21456697)
\curveto(289.72331734,340.16456542)(289.79831727,340.12456546)(289.87832107,340.09456697)
\curveto(289.9683171,340.06456552)(290.05831701,340.03956554)(290.14832107,340.01956697)
\curveto(290.18831688,340.00956557)(290.22331684,340.00456558)(290.25332107,340.00456697)
\curveto(290.28331678,340.01456557)(290.31831675,340.01456557)(290.35832107,340.00456697)
\lineto(290.47832107,339.97456697)
\curveto(290.52831654,339.97456561)(290.57331649,339.9795656)(290.61332107,339.98956697)
\lineto(290.73332107,339.98956697)
\curveto(290.81331625,340.00956557)(290.89331617,340.02456556)(290.97332107,340.03456697)
\curveto(291.05331601,340.04456554)(291.12831594,340.06456552)(291.19832107,340.09456697)
\curveto(291.45831561,340.19456539)(291.6683154,340.32956525)(291.82832107,340.49956697)
\curveto(291.98831508,340.66956491)(292.12331494,340.8795647)(292.23332107,341.12956697)
\curveto(292.27331479,341.22956435)(292.30331476,341.32956425)(292.32332107,341.42956697)
\curveto(292.34331472,341.52956405)(292.3683147,341.63456395)(292.39832107,341.74456697)
\curveto(292.40831466,341.7845638)(292.41331465,341.81956376)(292.41332107,341.84956697)
\curveto(292.41331465,341.88956369)(292.41831465,341.92956365)(292.42832107,341.96956697)
\lineto(292.42832107,342.10456697)
\curveto(292.42831464,342.15456343)(292.43331463,342.20456338)(292.44332107,342.25456697)
}
}
{
\newrgbcolor{curcolor}{0 0 0}
\pscustom[linestyle=none,fillstyle=solid,fillcolor=curcolor]
{
\newpath
\moveto(826.0368663,330.52791658)
\curveto(826.05685717,330.4479088)(826.06685716,330.33790891)(826.0668663,330.19791658)
\curveto(826.06685716,330.06790918)(826.05685717,329.96790928)(826.0368663,329.89791658)
\curveto(826.01685721,329.82790942)(826.01185722,329.76290949)(826.0218663,329.70291658)
\curveto(826.0318572,329.64290961)(826.0268572,329.57790967)(826.0068663,329.50791658)
\curveto(825.98685724,329.4479098)(825.97185726,329.38290987)(825.9618663,329.31291658)
\curveto(825.95185728,329.25291)(825.93685729,329.19291006)(825.9168663,329.13291658)
\curveto(825.89685733,329.0529102)(825.87185736,328.97791027)(825.8418663,328.90791658)
\curveto(825.82185741,328.83791041)(825.79685743,328.76791048)(825.7668663,328.69791658)
\curveto(825.74685748,328.66791058)(825.7318575,328.63791061)(825.7218663,328.60791658)
\curveto(825.72185751,328.58791066)(825.71185752,328.56791068)(825.6918663,328.54791658)
\curveto(825.58185765,328.3479109)(825.46185777,328.16791108)(825.3318663,328.00791658)
\curveto(825.31185792,327.96791128)(825.27685795,327.92791132)(825.2268663,327.88791658)
\curveto(825.18685804,327.8479114)(825.15185808,327.81791143)(825.1218663,327.79791658)
\curveto(825.08185815,327.77791147)(825.04685818,327.7479115)(825.0168663,327.70791658)
\curveto(824.98685824,327.67791157)(824.95685827,327.6529116)(824.9268663,327.63291658)
\lineto(824.6118663,327.45291658)
\curveto(824.50185873,327.37291188)(824.37185886,327.31291194)(824.2218663,327.27291658)
\lineto(823.7718663,327.15291658)
\curveto(823.69185954,327.13291212)(823.61185962,327.11791213)(823.5318663,327.10791658)
\curveto(823.45185978,327.10791214)(823.37185986,327.09791215)(823.2918663,327.07791658)
\curveto(823.25185998,327.06791218)(823.21186002,327.06291219)(823.1718663,327.06291658)
\curveto(823.14186009,327.07291218)(823.11186012,327.07291218)(823.0818663,327.06291658)
\curveto(823.0318602,327.0529122)(822.98186025,327.0529122)(822.9318663,327.06291658)
\curveto(822.89186034,327.07291218)(822.84686038,327.07291218)(822.7968663,327.06291658)
\lineto(820.5318663,327.06291658)
\lineto(820.0368663,327.06291658)
\curveto(819.86686336,327.07291218)(819.73686349,327.04291221)(819.6468663,326.97291658)
\curveto(819.53686369,326.89291236)(819.48186375,326.7479125)(819.4818663,326.53791658)
\curveto(819.49186374,326.32791292)(819.49686373,326.13291312)(819.4968663,325.95291658)
\lineto(819.4968663,323.74791658)
\lineto(819.4968663,323.25291658)
\curveto(819.50686372,323.06291619)(819.48686374,322.92791632)(819.4368663,322.84791658)
\curveto(819.39686383,322.78791646)(819.34686388,322.7479165)(819.2868663,322.72791658)
\curveto(819.23686399,322.71791653)(819.17186406,322.70291655)(819.0918663,322.68291658)
\lineto(818.8218663,322.68291658)
\curveto(818.67186456,322.68291657)(818.53686469,322.68791656)(818.4168663,322.69791658)
\curveto(818.29686493,322.70791654)(818.21186502,322.75791649)(818.1618663,322.84791658)
\curveto(818.12186511,322.90791634)(818.10186513,322.98791626)(818.1018663,323.08791658)
\lineto(818.1018663,323.40291658)
\lineto(818.1018663,332.50791658)
\curveto(818.10186513,332.61790663)(818.09686513,332.73790651)(818.0868663,332.86791658)
\curveto(818.08686514,333.00790624)(818.11186512,333.11790613)(818.1618663,333.19791658)
\curveto(818.20186503,333.25790599)(818.27686495,333.30790594)(818.3868663,333.34791658)
\curveto(818.40686482,333.35790589)(818.4268648,333.35790589)(818.4468663,333.34791658)
\curveto(818.46686476,333.3479059)(818.48686474,333.3529059)(818.5068663,333.36291658)
\lineto(821.9118663,333.36291658)
\curveto(822.29186094,333.36290589)(822.66186057,333.35790589)(823.0218663,333.34791658)
\curveto(823.39185984,333.3479059)(823.72185951,333.30290595)(824.0118663,333.21291658)
\curveto(824.46185877,333.06290619)(824.8268584,332.86790638)(825.1068663,332.62791658)
\curveto(825.38685784,332.38790686)(825.61685761,332.05790719)(825.7968663,331.63791658)
\curveto(825.84685738,331.52790772)(825.88185735,331.41290784)(825.9018663,331.29291658)
\curveto(825.9318573,331.17290808)(825.96685726,331.0479082)(826.0068663,330.91791658)
\curveto(826.0268572,330.8479084)(826.0318572,330.78290847)(826.0218663,330.72291658)
\curveto(826.01185722,330.66290859)(826.01685721,330.59790865)(826.0368663,330.52791658)
\moveto(824.6268663,329.98791658)
\curveto(824.66685856,330.12790912)(824.67185856,330.28790896)(824.6418663,330.46791658)
\curveto(824.61185862,330.65790859)(824.58185865,330.80790844)(824.5518663,330.91791658)
\curveto(824.45185878,331.19790805)(824.31685891,331.41790783)(824.1468663,331.57791658)
\curveto(823.98685924,331.7479075)(823.77685945,331.88790736)(823.5168663,331.99791658)
\curveto(823.29685993,332.08790716)(823.04186019,332.14290711)(822.7518663,332.16291658)
\curveto(822.47186076,332.18290707)(822.17686105,332.19290706)(821.8668663,332.19291658)
\lineto(819.9318663,332.19291658)
\curveto(819.91186332,332.18290707)(819.88686334,332.17790707)(819.8568663,332.17791658)
\curveto(819.83686339,332.17790707)(819.81186342,332.17290708)(819.7818663,332.16291658)
\curveto(819.66186357,332.13290712)(819.58186365,332.06790718)(819.5418663,331.96791658)
\curveto(819.50186373,331.86790738)(819.48186375,331.73290752)(819.4818663,331.56291658)
\curveto(819.49186374,331.40290785)(819.49686373,331.252908)(819.4968663,331.11291658)
\lineto(819.4968663,329.31291658)
\curveto(819.49686373,329.16291009)(819.49186374,328.99791025)(819.4818663,328.81791658)
\curveto(819.48186375,328.63791061)(819.51186372,328.49791075)(819.5718663,328.39791658)
\curveto(819.62186361,328.31791093)(819.69686353,328.26791098)(819.7968663,328.24791658)
\curveto(819.90686332,328.23791101)(820.0268632,328.23291102)(820.1568663,328.23291658)
\lineto(822.1818663,328.23291658)
\lineto(822.6468663,328.23291658)
\curveto(822.80686042,328.24291101)(822.94686028,328.26291099)(823.0668663,328.29291658)
\curveto(823.33685989,328.36291089)(823.57185966,328.44291081)(823.7718663,328.53291658)
\curveto(823.98185925,328.63291062)(824.15685907,328.78291047)(824.2968663,328.98291658)
\curveto(824.37685885,329.10291015)(824.43685879,329.22791002)(824.4768663,329.35791658)
\curveto(824.5268587,329.48790976)(824.57185866,329.63290962)(824.6118663,329.79291658)
\curveto(824.62185861,329.83290942)(824.6268586,329.89790935)(824.6268663,329.98791658)
}
}
{
\newrgbcolor{curcolor}{0 0 0}
\pscustom[linestyle=none,fillstyle=solid,fillcolor=curcolor]
{
\newpath
\moveto(834.6984288,326.88291658)
\curveto(834.71842074,326.82291243)(834.72842073,326.72791252)(834.7284288,326.59791658)
\curveto(834.72842073,326.47791277)(834.72342073,326.39291286)(834.7134288,326.34291658)
\lineto(834.7134288,326.19291658)
\curveto(834.70342075,326.11291314)(834.69342076,326.03791321)(834.6834288,325.96791658)
\curveto(834.68342077,325.90791334)(834.67842078,325.83791341)(834.6684288,325.75791658)
\curveto(834.64842081,325.69791355)(834.63342082,325.63791361)(834.6234288,325.57791658)
\curveto(834.62342083,325.51791373)(834.61342084,325.45791379)(834.5934288,325.39791658)
\curveto(834.5534209,325.26791398)(834.51842094,325.13791411)(834.4884288,325.00791658)
\curveto(834.458421,324.87791437)(834.41842104,324.75791449)(834.3684288,324.64791658)
\curveto(834.1584213,324.16791508)(833.87842158,323.76291549)(833.5284288,323.43291658)
\curveto(833.17842228,323.11291614)(832.74842271,322.86791638)(832.2384288,322.69791658)
\curveto(832.12842333,322.65791659)(832.00842345,322.62791662)(831.8784288,322.60791658)
\curveto(831.7584237,322.58791666)(831.63342382,322.56791668)(831.5034288,322.54791658)
\curveto(831.44342401,322.53791671)(831.37842408,322.53291672)(831.3084288,322.53291658)
\curveto(831.24842421,322.52291673)(831.18842427,322.51791673)(831.1284288,322.51791658)
\curveto(831.08842437,322.50791674)(831.02842443,322.50291675)(830.9484288,322.50291658)
\curveto(830.87842458,322.50291675)(830.82842463,322.50791674)(830.7984288,322.51791658)
\curveto(830.7584247,322.52791672)(830.71842474,322.53291672)(830.6784288,322.53291658)
\curveto(830.63842482,322.52291673)(830.60342485,322.52291673)(830.5734288,322.53291658)
\lineto(830.4834288,322.53291658)
\lineto(830.1234288,322.57791658)
\curveto(829.98342547,322.61791663)(829.84842561,322.65791659)(829.7184288,322.69791658)
\curveto(829.58842587,322.73791651)(829.46342599,322.78291647)(829.3434288,322.83291658)
\curveto(828.89342656,323.03291622)(828.52342693,323.29291596)(828.2334288,323.61291658)
\curveto(827.94342751,323.93291532)(827.70342775,324.32291493)(827.5134288,324.78291658)
\curveto(827.46342799,324.88291437)(827.42342803,324.98291427)(827.3934288,325.08291658)
\curveto(827.37342808,325.18291407)(827.3534281,325.28791396)(827.3334288,325.39791658)
\curveto(827.31342814,325.43791381)(827.30342815,325.46791378)(827.3034288,325.48791658)
\curveto(827.31342814,325.51791373)(827.31342814,325.5529137)(827.3034288,325.59291658)
\curveto(827.28342817,325.67291358)(827.26842819,325.7529135)(827.2584288,325.83291658)
\curveto(827.2584282,325.92291333)(827.24842821,326.00791324)(827.2284288,326.08791658)
\lineto(827.2284288,326.20791658)
\curveto(827.22842823,326.247913)(827.22342823,326.29291296)(827.2134288,326.34291658)
\curveto(827.20342825,326.39291286)(827.19842826,326.47791277)(827.1984288,326.59791658)
\curveto(827.19842826,326.72791252)(827.20842825,326.82291243)(827.2284288,326.88291658)
\curveto(827.24842821,326.9529123)(827.2534282,327.02291223)(827.2434288,327.09291658)
\curveto(827.23342822,327.16291209)(827.23842822,327.23291202)(827.2584288,327.30291658)
\curveto(827.26842819,327.3529119)(827.27342818,327.39291186)(827.2734288,327.42291658)
\curveto(827.28342817,327.46291179)(827.29342816,327.50791174)(827.3034288,327.55791658)
\curveto(827.33342812,327.67791157)(827.3584281,327.79791145)(827.3784288,327.91791658)
\curveto(827.40842805,328.03791121)(827.44842801,328.1529111)(827.4984288,328.26291658)
\curveto(827.64842781,328.63291062)(827.82842763,328.96291029)(828.0384288,329.25291658)
\curveto(828.2584272,329.5529097)(828.52342693,329.80290945)(828.8334288,330.00291658)
\curveto(828.9534265,330.08290917)(829.07842638,330.1479091)(829.2084288,330.19791658)
\curveto(829.33842612,330.25790899)(829.47342598,330.31790893)(829.6134288,330.37791658)
\curveto(829.73342572,330.42790882)(829.86342559,330.45790879)(830.0034288,330.46791658)
\curveto(830.14342531,330.48790876)(830.28342517,330.51790873)(830.4234288,330.55791658)
\lineto(830.6184288,330.55791658)
\curveto(830.68842477,330.56790868)(830.7534247,330.57790867)(830.8134288,330.58791658)
\curveto(831.70342375,330.59790865)(832.44342301,330.41290884)(833.0334288,330.03291658)
\curveto(833.62342183,329.6529096)(834.04842141,329.15791009)(834.3084288,328.54791658)
\curveto(834.3584211,328.4479108)(834.39842106,328.3479109)(834.4284288,328.24791658)
\curveto(834.458421,328.1479111)(834.49342096,328.04291121)(834.5334288,327.93291658)
\curveto(834.56342089,327.82291143)(834.58842087,327.70291155)(834.6084288,327.57291658)
\curveto(834.62842083,327.4529118)(834.6534208,327.32791192)(834.6834288,327.19791658)
\curveto(834.69342076,327.1479121)(834.69342076,327.09291216)(834.6834288,327.03291658)
\curveto(834.68342077,326.98291227)(834.68842077,326.93291232)(834.6984288,326.88291658)
\moveto(833.3634288,326.02791658)
\curveto(833.38342207,326.09791315)(833.38842207,326.17791307)(833.3784288,326.26791658)
\lineto(833.3784288,326.52291658)
\curveto(833.37842208,326.91291234)(833.34342211,327.24291201)(833.2734288,327.51291658)
\curveto(833.24342221,327.59291166)(833.21842224,327.67291158)(833.1984288,327.75291658)
\curveto(833.17842228,327.83291142)(833.1534223,327.90791134)(833.1234288,327.97791658)
\curveto(832.84342261,328.62791062)(832.39842306,329.07791017)(831.7884288,329.32791658)
\curveto(831.71842374,329.35790989)(831.64342381,329.37790987)(831.5634288,329.38791658)
\lineto(831.3234288,329.44791658)
\curveto(831.24342421,329.46790978)(831.1584243,329.47790977)(831.0684288,329.47791658)
\lineto(830.7984288,329.47791658)
\lineto(830.5284288,329.43291658)
\curveto(830.42842503,329.41290984)(830.33342512,329.38790986)(830.2434288,329.35791658)
\curveto(830.16342529,329.33790991)(830.08342537,329.30790994)(830.0034288,329.26791658)
\curveto(829.93342552,329.24791)(829.86842559,329.21791003)(829.8084288,329.17791658)
\curveto(829.74842571,329.13791011)(829.69342576,329.09791015)(829.6434288,329.05791658)
\curveto(829.40342605,328.88791036)(829.20842625,328.68291057)(829.0584288,328.44291658)
\curveto(828.90842655,328.20291105)(828.77842668,327.92291133)(828.6684288,327.60291658)
\curveto(828.63842682,327.50291175)(828.61842684,327.39791185)(828.6084288,327.28791658)
\curveto(828.59842686,327.18791206)(828.58342687,327.08291217)(828.5634288,326.97291658)
\curveto(828.5534269,326.93291232)(828.54842691,326.86791238)(828.5484288,326.77791658)
\curveto(828.53842692,326.7479125)(828.53342692,326.71291254)(828.5334288,326.67291658)
\curveto(828.54342691,326.63291262)(828.54842691,326.58791266)(828.5484288,326.53791658)
\lineto(828.5484288,326.23791658)
\curveto(828.54842691,326.13791311)(828.5584269,326.0479132)(828.5784288,325.96791658)
\lineto(828.6084288,325.78791658)
\curveto(828.62842683,325.68791356)(828.64342681,325.58791366)(828.6534288,325.48791658)
\curveto(828.67342678,325.39791385)(828.70342675,325.31291394)(828.7434288,325.23291658)
\curveto(828.84342661,324.99291426)(828.9584265,324.76791448)(829.0884288,324.55791658)
\curveto(829.22842623,324.3479149)(829.39842606,324.17291508)(829.5984288,324.03291658)
\curveto(829.64842581,324.00291525)(829.69342576,323.97791527)(829.7334288,323.95791658)
\curveto(829.77342568,323.93791531)(829.81842564,323.91291534)(829.8684288,323.88291658)
\curveto(829.94842551,323.83291542)(830.03342542,323.78791546)(830.1234288,323.74791658)
\curveto(830.22342523,323.71791553)(830.32842513,323.68791556)(830.4384288,323.65791658)
\curveto(830.48842497,323.63791561)(830.53342492,323.62791562)(830.5734288,323.62791658)
\curveto(830.62342483,323.63791561)(830.67342478,323.63791561)(830.7234288,323.62791658)
\curveto(830.7534247,323.61791563)(830.81342464,323.60791564)(830.9034288,323.59791658)
\curveto(831.00342445,323.58791566)(831.07842438,323.59291566)(831.1284288,323.61291658)
\curveto(831.16842429,323.62291563)(831.20842425,323.62291563)(831.2484288,323.61291658)
\curveto(831.28842417,323.61291564)(831.32842413,323.62291563)(831.3684288,323.64291658)
\curveto(831.44842401,323.66291559)(831.52842393,323.67791557)(831.6084288,323.68791658)
\curveto(831.68842377,323.70791554)(831.76342369,323.73291552)(831.8334288,323.76291658)
\curveto(832.17342328,323.90291535)(832.44842301,324.09791515)(832.6584288,324.34791658)
\curveto(832.86842259,324.59791465)(833.04342241,324.89291436)(833.1834288,325.23291658)
\curveto(833.23342222,325.3529139)(833.26342219,325.47791377)(833.2734288,325.60791658)
\curveto(833.29342216,325.7479135)(833.32342213,325.88791336)(833.3634288,326.02791658)
}
}
{
\newrgbcolor{curcolor}{0 0 0}
\pscustom[linestyle=none,fillstyle=solid,fillcolor=curcolor]
{
\newpath
\moveto(839.83171005,330.58791658)
\curveto(840.06170526,330.58790866)(840.19170513,330.52790872)(840.22171005,330.40791658)
\curveto(840.25170507,330.29790895)(840.26670505,330.13290912)(840.26671005,329.91291658)
\lineto(840.26671005,329.62791658)
\curveto(840.26670505,329.53790971)(840.24170508,329.46290979)(840.19171005,329.40291658)
\curveto(840.13170519,329.32290993)(840.04670527,329.27790997)(839.93671005,329.26791658)
\curveto(839.82670549,329.26790998)(839.7167056,329.25291)(839.60671005,329.22291658)
\curveto(839.46670585,329.19291006)(839.33170599,329.16291009)(839.20171005,329.13291658)
\curveto(839.08170624,329.10291015)(838.96670635,329.06291019)(838.85671005,329.01291658)
\curveto(838.56670675,328.88291037)(838.33170699,328.70291055)(838.15171005,328.47291658)
\curveto(837.97170735,328.252911)(837.8167075,327.99791125)(837.68671005,327.70791658)
\curveto(837.64670767,327.59791165)(837.6167077,327.48291177)(837.59671005,327.36291658)
\curveto(837.57670774,327.252912)(837.55170777,327.13791211)(837.52171005,327.01791658)
\curveto(837.51170781,326.96791228)(837.50670781,326.91791233)(837.50671005,326.86791658)
\curveto(837.5167078,326.81791243)(837.5167078,326.76791248)(837.50671005,326.71791658)
\curveto(837.47670784,326.59791265)(837.46170786,326.45791279)(837.46171005,326.29791658)
\curveto(837.47170785,326.1479131)(837.47670784,326.00291325)(837.47671005,325.86291658)
\lineto(837.47671005,324.01791658)
\lineto(837.47671005,323.67291658)
\curveto(837.47670784,323.5529157)(837.47170785,323.43791581)(837.46171005,323.32791658)
\curveto(837.45170787,323.21791603)(837.44670787,323.12291613)(837.44671005,323.04291658)
\curveto(837.45670786,322.96291629)(837.43670788,322.89291636)(837.38671005,322.83291658)
\curveto(837.33670798,322.76291649)(837.25670806,322.72291653)(837.14671005,322.71291658)
\curveto(837.04670827,322.70291655)(836.93670838,322.69791655)(836.81671005,322.69791658)
\lineto(836.54671005,322.69791658)
\curveto(836.49670882,322.71791653)(836.44670887,322.73291652)(836.39671005,322.74291658)
\curveto(836.35670896,322.76291649)(836.32670899,322.78791646)(836.30671005,322.81791658)
\curveto(836.25670906,322.88791636)(836.22670909,322.97291628)(836.21671005,323.07291658)
\lineto(836.21671005,323.40291658)
\lineto(836.21671005,324.55791658)
\lineto(836.21671005,328.71291658)
\lineto(836.21671005,329.74791658)
\lineto(836.21671005,330.04791658)
\curveto(836.22670909,330.1479091)(836.25670906,330.23290902)(836.30671005,330.30291658)
\curveto(836.33670898,330.34290891)(836.38670893,330.37290888)(836.45671005,330.39291658)
\curveto(836.53670878,330.41290884)(836.6217087,330.42290883)(836.71171005,330.42291658)
\curveto(836.80170852,330.43290882)(836.89170843,330.43290882)(836.98171005,330.42291658)
\curveto(837.07170825,330.41290884)(837.14170818,330.39790885)(837.19171005,330.37791658)
\curveto(837.27170805,330.3479089)(837.321708,330.28790896)(837.34171005,330.19791658)
\curveto(837.37170795,330.11790913)(837.38670793,330.02790922)(837.38671005,329.92791658)
\lineto(837.38671005,329.62791658)
\curveto(837.38670793,329.52790972)(837.40670791,329.43790981)(837.44671005,329.35791658)
\curveto(837.45670786,329.33790991)(837.46670785,329.32290993)(837.47671005,329.31291658)
\lineto(837.52171005,329.26791658)
\curveto(837.63170769,329.26790998)(837.7217076,329.31290994)(837.79171005,329.40291658)
\curveto(837.86170746,329.50290975)(837.9217074,329.58290967)(837.97171005,329.64291658)
\lineto(838.06171005,329.73291658)
\curveto(838.15170717,329.84290941)(838.27670704,329.95790929)(838.43671005,330.07791658)
\curveto(838.59670672,330.19790905)(838.74670657,330.28790896)(838.88671005,330.34791658)
\curveto(838.97670634,330.39790885)(839.07170625,330.43290882)(839.17171005,330.45291658)
\curveto(839.27170605,330.48290877)(839.37670594,330.51290874)(839.48671005,330.54291658)
\curveto(839.54670577,330.5529087)(839.60670571,330.55790869)(839.66671005,330.55791658)
\curveto(839.72670559,330.56790868)(839.78170554,330.57790867)(839.83171005,330.58791658)
}
}
{
\newrgbcolor{curcolor}{0 0 0}
\pscustom[linestyle=none,fillstyle=solid,fillcolor=curcolor]
{
\newpath
\moveto(842.14147567,332.74791658)
\curveto(842.29147366,332.7479065)(842.44147351,332.74290651)(842.59147567,332.73291658)
\curveto(842.74147321,332.73290652)(842.84647311,332.69290656)(842.90647567,332.61291658)
\curveto(842.956473,332.5529067)(842.98147297,332.46790678)(842.98147567,332.35791658)
\curveto(842.99147296,332.25790699)(842.99647296,332.1529071)(842.99647567,332.04291658)
\lineto(842.99647567,331.17291658)
\curveto(842.99647296,331.09290816)(842.99147296,331.00790824)(842.98147567,330.91791658)
\curveto(842.98147297,330.83790841)(842.99147296,330.76790848)(843.01147567,330.70791658)
\curveto(843.0514729,330.56790868)(843.14147281,330.47790877)(843.28147567,330.43791658)
\curveto(843.33147262,330.42790882)(843.37647258,330.42290883)(843.41647567,330.42291658)
\lineto(843.56647567,330.42291658)
\lineto(843.97147567,330.42291658)
\curveto(844.13147182,330.43290882)(844.24647171,330.42290883)(844.31647567,330.39291658)
\curveto(844.40647155,330.33290892)(844.46647149,330.27290898)(844.49647567,330.21291658)
\curveto(844.51647144,330.17290908)(844.52647143,330.12790912)(844.52647567,330.07791658)
\lineto(844.52647567,329.92791658)
\curveto(844.52647143,329.81790943)(844.52147143,329.71290954)(844.51147567,329.61291658)
\curveto(844.50147145,329.52290973)(844.46647149,329.4529098)(844.40647567,329.40291658)
\curveto(844.34647161,329.3529099)(844.26147169,329.32290993)(844.15147567,329.31291658)
\lineto(843.82147567,329.31291658)
\curveto(843.71147224,329.32290993)(843.60147235,329.32790992)(843.49147567,329.32791658)
\curveto(843.38147257,329.32790992)(843.28647267,329.31290994)(843.20647567,329.28291658)
\curveto(843.13647282,329.25291)(843.08647287,329.20291005)(843.05647567,329.13291658)
\curveto(843.02647293,329.06291019)(843.00647295,328.97791027)(842.99647567,328.87791658)
\curveto(842.98647297,328.78791046)(842.98147297,328.68791056)(842.98147567,328.57791658)
\curveto(842.99147296,328.47791077)(842.99647296,328.37791087)(842.99647567,328.27791658)
\lineto(842.99647567,325.30791658)
\curveto(842.99647296,325.08791416)(842.99147296,324.8529144)(842.98147567,324.60291658)
\curveto(842.98147297,324.36291489)(843.02647293,324.17791507)(843.11647567,324.04791658)
\curveto(843.16647279,323.96791528)(843.23147272,323.91291534)(843.31147567,323.88291658)
\curveto(843.39147256,323.8529154)(843.48647247,323.82791542)(843.59647567,323.80791658)
\curveto(843.62647233,323.79791545)(843.6564723,323.79291546)(843.68647567,323.79291658)
\curveto(843.72647223,323.80291545)(843.76147219,323.80291545)(843.79147567,323.79291658)
\lineto(843.98647567,323.79291658)
\curveto(844.08647187,323.79291546)(844.17647178,323.78291547)(844.25647567,323.76291658)
\curveto(844.34647161,323.7529155)(844.41147154,323.71791553)(844.45147567,323.65791658)
\curveto(844.47147148,323.62791562)(844.48647147,323.57291568)(844.49647567,323.49291658)
\curveto(844.51647144,323.42291583)(844.52647143,323.3479159)(844.52647567,323.26791658)
\curveto(844.53647142,323.18791606)(844.53647142,323.10791614)(844.52647567,323.02791658)
\curveto(844.51647144,322.95791629)(844.49647146,322.90291635)(844.46647567,322.86291658)
\curveto(844.42647153,322.79291646)(844.3514716,322.74291651)(844.24147567,322.71291658)
\curveto(844.16147179,322.69291656)(844.07147188,322.68291657)(843.97147567,322.68291658)
\curveto(843.87147208,322.69291656)(843.78147217,322.69791655)(843.70147567,322.69791658)
\curveto(843.64147231,322.69791655)(843.58147237,322.69291656)(843.52147567,322.68291658)
\curveto(843.46147249,322.68291657)(843.40647255,322.68791656)(843.35647567,322.69791658)
\lineto(843.17647567,322.69791658)
\curveto(843.12647283,322.70791654)(843.07647288,322.71291654)(843.02647567,322.71291658)
\curveto(842.98647297,322.72291653)(842.94147301,322.72791652)(842.89147567,322.72791658)
\curveto(842.69147326,322.77791647)(842.51647344,322.83291642)(842.36647567,322.89291658)
\curveto(842.22647373,322.9529163)(842.10647385,323.05791619)(842.00647567,323.20791658)
\curveto(841.86647409,323.40791584)(841.78647417,323.65791559)(841.76647567,323.95791658)
\curveto(841.74647421,324.26791498)(841.73647422,324.59791465)(841.73647567,324.94791658)
\lineto(841.73647567,328.87791658)
\curveto(841.70647425,329.00791024)(841.67647428,329.10291015)(841.64647567,329.16291658)
\curveto(841.62647433,329.22291003)(841.5564744,329.27290998)(841.43647567,329.31291658)
\curveto(841.39647456,329.32290993)(841.3564746,329.32290993)(841.31647567,329.31291658)
\curveto(841.27647468,329.30290995)(841.23647472,329.30790994)(841.19647567,329.32791658)
\lineto(840.95647567,329.32791658)
\curveto(840.82647513,329.32790992)(840.71647524,329.33790991)(840.62647567,329.35791658)
\curveto(840.54647541,329.38790986)(840.49147546,329.4479098)(840.46147567,329.53791658)
\curveto(840.44147551,329.57790967)(840.42647553,329.62290963)(840.41647567,329.67291658)
\lineto(840.41647567,329.82291658)
\curveto(840.41647554,329.96290929)(840.42647553,330.07790917)(840.44647567,330.16791658)
\curveto(840.46647549,330.26790898)(840.52647543,330.34290891)(840.62647567,330.39291658)
\curveto(840.73647522,330.43290882)(840.87647508,330.44290881)(841.04647567,330.42291658)
\curveto(841.22647473,330.40290885)(841.37647458,330.41290884)(841.49647567,330.45291658)
\curveto(841.58647437,330.50290875)(841.6564743,330.57290868)(841.70647567,330.66291658)
\curveto(841.72647423,330.72290853)(841.73647422,330.79790845)(841.73647567,330.88791658)
\lineto(841.73647567,331.14291658)
\lineto(841.73647567,332.07291658)
\lineto(841.73647567,332.31291658)
\curveto(841.73647422,332.40290685)(841.74647421,332.47790677)(841.76647567,332.53791658)
\curveto(841.80647415,332.61790663)(841.88147407,332.68290657)(841.99147567,332.73291658)
\curveto(842.02147393,332.73290652)(842.04647391,332.73290652)(842.06647567,332.73291658)
\curveto(842.09647386,332.74290651)(842.12147383,332.7479065)(842.14147567,332.74791658)
}
}
{
\newrgbcolor{curcolor}{0 0 0}
\pscustom[linestyle=none,fillstyle=solid,fillcolor=curcolor]
{
\newpath
\moveto(852.79827255,323.23791658)
\curveto(852.82826472,323.07791617)(852.81326473,322.94291631)(852.75327255,322.83291658)
\curveto(852.69326485,322.73291652)(852.61326493,322.65791659)(852.51327255,322.60791658)
\curveto(852.46326508,322.58791666)(852.40826514,322.57791667)(852.34827255,322.57791658)
\curveto(852.29826525,322.57791667)(852.2432653,322.56791668)(852.18327255,322.54791658)
\curveto(851.96326558,322.49791675)(851.7432658,322.51291674)(851.52327255,322.59291658)
\curveto(851.31326623,322.66291659)(851.16826638,322.7529165)(851.08827255,322.86291658)
\curveto(851.03826651,322.93291632)(850.99326655,323.01291624)(850.95327255,323.10291658)
\curveto(850.91326663,323.20291605)(850.86326668,323.28291597)(850.80327255,323.34291658)
\curveto(850.78326676,323.36291589)(850.75826679,323.38291587)(850.72827255,323.40291658)
\curveto(850.70826684,323.42291583)(850.67826687,323.42791582)(850.63827255,323.41791658)
\curveto(850.52826702,323.38791586)(850.42326712,323.33291592)(850.32327255,323.25291658)
\curveto(850.23326731,323.17291608)(850.1432674,323.10291615)(850.05327255,323.04291658)
\curveto(849.92326762,322.96291629)(849.78326776,322.88791636)(849.63327255,322.81791658)
\curveto(849.48326806,322.75791649)(849.32326822,322.70291655)(849.15327255,322.65291658)
\curveto(849.05326849,322.62291663)(848.9432686,322.60291665)(848.82327255,322.59291658)
\curveto(848.71326883,322.58291667)(848.60326894,322.56791668)(848.49327255,322.54791658)
\curveto(848.4432691,322.53791671)(848.39826915,322.53291672)(848.35827255,322.53291658)
\lineto(848.25327255,322.53291658)
\curveto(848.1432694,322.51291674)(848.03826951,322.51291674)(847.93827255,322.53291658)
\lineto(847.80327255,322.53291658)
\curveto(847.75326979,322.54291671)(847.70326984,322.5479167)(847.65327255,322.54791658)
\curveto(847.60326994,322.5479167)(847.55826999,322.55791669)(847.51827255,322.57791658)
\curveto(847.47827007,322.58791666)(847.4432701,322.59291666)(847.41327255,322.59291658)
\curveto(847.39327015,322.58291667)(847.36827018,322.58291667)(847.33827255,322.59291658)
\lineto(847.09827255,322.65291658)
\curveto(847.01827053,322.66291659)(846.9432706,322.68291657)(846.87327255,322.71291658)
\curveto(846.57327097,322.84291641)(846.32827122,322.98791626)(846.13827255,323.14791658)
\curveto(845.95827159,323.31791593)(845.80827174,323.5529157)(845.68827255,323.85291658)
\curveto(845.59827195,324.07291518)(845.55327199,324.33791491)(845.55327255,324.64791658)
\lineto(845.55327255,324.96291658)
\curveto(845.56327198,325.01291424)(845.56827198,325.06291419)(845.56827255,325.11291658)
\lineto(845.59827255,325.29291658)
\lineto(845.71827255,325.62291658)
\curveto(845.75827179,325.73291352)(845.80827174,325.83291342)(845.86827255,325.92291658)
\curveto(846.0482715,326.21291304)(846.29327125,326.42791282)(846.60327255,326.56791658)
\curveto(846.91327063,326.70791254)(847.25327029,326.83291242)(847.62327255,326.94291658)
\curveto(847.76326978,326.98291227)(847.90826964,327.01291224)(848.05827255,327.03291658)
\curveto(848.20826934,327.0529122)(848.35826919,327.07791217)(848.50827255,327.10791658)
\curveto(848.57826897,327.12791212)(848.6432689,327.13791211)(848.70327255,327.13791658)
\curveto(848.77326877,327.13791211)(848.8482687,327.1479121)(848.92827255,327.16791658)
\curveto(848.99826855,327.18791206)(849.06826848,327.19791205)(849.13827255,327.19791658)
\curveto(849.20826834,327.20791204)(849.28326826,327.22291203)(849.36327255,327.24291658)
\curveto(849.61326793,327.30291195)(849.8482677,327.3529119)(850.06827255,327.39291658)
\curveto(850.28826726,327.44291181)(850.46326708,327.55791169)(850.59327255,327.73791658)
\curveto(850.65326689,327.81791143)(850.70326684,327.91791133)(850.74327255,328.03791658)
\curveto(850.78326676,328.16791108)(850.78326676,328.30791094)(850.74327255,328.45791658)
\curveto(850.68326686,328.69791055)(850.59326695,328.88791036)(850.47327255,329.02791658)
\curveto(850.36326718,329.16791008)(850.20326734,329.27790997)(849.99327255,329.35791658)
\curveto(849.87326767,329.40790984)(849.72826782,329.44290981)(849.55827255,329.46291658)
\curveto(849.39826815,329.48290977)(849.22826832,329.49290976)(849.04827255,329.49291658)
\curveto(848.86826868,329.49290976)(848.69326885,329.48290977)(848.52327255,329.46291658)
\curveto(848.35326919,329.44290981)(848.20826934,329.41290984)(848.08827255,329.37291658)
\curveto(847.91826963,329.31290994)(847.75326979,329.22791002)(847.59327255,329.11791658)
\curveto(847.51327003,329.05791019)(847.43827011,328.97791027)(847.36827255,328.87791658)
\curveto(847.30827024,328.78791046)(847.25327029,328.68791056)(847.20327255,328.57791658)
\curveto(847.17327037,328.49791075)(847.1432704,328.41291084)(847.11327255,328.32291658)
\curveto(847.09327045,328.23291102)(847.0482705,328.16291109)(846.97827255,328.11291658)
\curveto(846.93827061,328.08291117)(846.86827068,328.05791119)(846.76827255,328.03791658)
\curveto(846.67827087,328.02791122)(846.58327096,328.02291123)(846.48327255,328.02291658)
\curveto(846.38327116,328.02291123)(846.28327126,328.02791122)(846.18327255,328.03791658)
\curveto(846.09327145,328.05791119)(846.02827152,328.08291117)(845.98827255,328.11291658)
\curveto(845.9482716,328.14291111)(845.91827163,328.19291106)(845.89827255,328.26291658)
\curveto(845.87827167,328.33291092)(845.87827167,328.40791084)(845.89827255,328.48791658)
\curveto(845.92827162,328.61791063)(845.95827159,328.73791051)(845.98827255,328.84791658)
\curveto(846.02827152,328.96791028)(846.07327147,329.08291017)(846.12327255,329.19291658)
\curveto(846.31327123,329.54290971)(846.55327099,329.81290944)(846.84327255,330.00291658)
\curveto(847.13327041,330.20290905)(847.49327005,330.36290889)(847.92327255,330.48291658)
\curveto(848.02326952,330.50290875)(848.12326942,330.51790873)(848.22327255,330.52791658)
\curveto(848.33326921,330.53790871)(848.4432691,330.5529087)(848.55327255,330.57291658)
\curveto(848.59326895,330.58290867)(848.65826889,330.58290867)(848.74827255,330.57291658)
\curveto(848.83826871,330.57290868)(848.89326865,330.58290867)(848.91327255,330.60291658)
\curveto(849.61326793,330.61290864)(850.22326732,330.53290872)(850.74327255,330.36291658)
\curveto(851.26326628,330.19290906)(851.62826592,329.86790938)(851.83827255,329.38791658)
\curveto(851.92826562,329.18791006)(851.97826557,328.9529103)(851.98827255,328.68291658)
\curveto(852.00826554,328.42291083)(852.01826553,328.1479111)(852.01827255,327.85791658)
\lineto(852.01827255,324.54291658)
\curveto(852.01826553,324.40291485)(852.02326552,324.26791498)(852.03327255,324.13791658)
\curveto(852.0432655,324.00791524)(852.07326547,323.90291535)(852.12327255,323.82291658)
\curveto(852.17326537,323.7529155)(852.23826531,323.70291555)(852.31827255,323.67291658)
\curveto(852.40826514,323.63291562)(852.49326505,323.60291565)(852.57327255,323.58291658)
\curveto(852.65326489,323.57291568)(852.71326483,323.52791572)(852.75327255,323.44791658)
\curveto(852.77326477,323.41791583)(852.78326476,323.38791586)(852.78327255,323.35791658)
\curveto(852.78326476,323.32791592)(852.78826476,323.28791596)(852.79827255,323.23791658)
\moveto(850.65327255,324.90291658)
\curveto(850.71326683,325.04291421)(850.7432668,325.20291405)(850.74327255,325.38291658)
\curveto(850.75326679,325.57291368)(850.75826679,325.76791348)(850.75827255,325.96791658)
\curveto(850.75826679,326.07791317)(850.75326679,326.17791307)(850.74327255,326.26791658)
\curveto(850.73326681,326.35791289)(850.69326685,326.42791282)(850.62327255,326.47791658)
\curveto(850.59326695,326.49791275)(850.52326702,326.50791274)(850.41327255,326.50791658)
\curveto(850.39326715,326.48791276)(850.35826719,326.47791277)(850.30827255,326.47791658)
\curveto(850.25826729,326.47791277)(850.21326733,326.46791278)(850.17327255,326.44791658)
\curveto(850.09326745,326.42791282)(850.00326754,326.40791284)(849.90327255,326.38791658)
\lineto(849.60327255,326.32791658)
\curveto(849.57326797,326.32791292)(849.53826801,326.32291293)(849.49827255,326.31291658)
\lineto(849.39327255,326.31291658)
\curveto(849.2432683,326.27291298)(849.07826847,326.247913)(848.89827255,326.23791658)
\curveto(848.72826882,326.23791301)(848.56826898,326.21791303)(848.41827255,326.17791658)
\curveto(848.33826921,326.15791309)(848.26326928,326.13791311)(848.19327255,326.11791658)
\curveto(848.13326941,326.10791314)(848.06326948,326.09291316)(847.98327255,326.07291658)
\curveto(847.82326972,326.02291323)(847.67326987,325.95791329)(847.53327255,325.87791658)
\curveto(847.39327015,325.80791344)(847.27327027,325.71791353)(847.17327255,325.60791658)
\curveto(847.07327047,325.49791375)(846.99827055,325.36291389)(846.94827255,325.20291658)
\curveto(846.89827065,325.0529142)(846.87827067,324.86791438)(846.88827255,324.64791658)
\curveto(846.88827066,324.5479147)(846.90327064,324.4529148)(846.93327255,324.36291658)
\curveto(846.97327057,324.28291497)(847.01827053,324.20791504)(847.06827255,324.13791658)
\curveto(847.1482704,324.02791522)(847.25327029,323.93291532)(847.38327255,323.85291658)
\curveto(847.51327003,323.78291547)(847.65326989,323.72291553)(847.80327255,323.67291658)
\curveto(847.85326969,323.66291559)(847.90326964,323.65791559)(847.95327255,323.65791658)
\curveto(848.00326954,323.65791559)(848.05326949,323.6529156)(848.10327255,323.64291658)
\curveto(848.17326937,323.62291563)(848.25826929,323.60791564)(848.35827255,323.59791658)
\curveto(848.46826908,323.59791565)(848.55826899,323.60791564)(848.62827255,323.62791658)
\curveto(848.68826886,323.6479156)(848.7482688,323.6529156)(848.80827255,323.64291658)
\curveto(848.86826868,323.64291561)(848.92826862,323.6529156)(848.98827255,323.67291658)
\curveto(849.06826848,323.69291556)(849.1432684,323.70791554)(849.21327255,323.71791658)
\curveto(849.29326825,323.72791552)(849.36826818,323.7479155)(849.43827255,323.77791658)
\curveto(849.72826782,323.89791535)(849.97326757,324.04291521)(850.17327255,324.21291658)
\curveto(850.38326716,324.38291487)(850.543267,324.61291464)(850.65327255,324.90291658)
}
}
{
\newrgbcolor{curcolor}{0 0 0}
\pscustom[linestyle=none,fillstyle=solid,fillcolor=curcolor]
{
\newpath
\moveto(860.92991317,323.49291658)
\lineto(860.92991317,323.10291658)
\curveto(860.9299053,322.98291627)(860.90490532,322.88291637)(860.85491317,322.80291658)
\curveto(860.80490542,322.73291652)(860.71990551,322.69291656)(860.59991317,322.68291658)
\lineto(860.25491317,322.68291658)
\curveto(860.19490603,322.68291657)(860.13490609,322.67791657)(860.07491317,322.66791658)
\curveto(860.0249062,322.66791658)(859.97990625,322.67791657)(859.93991317,322.69791658)
\curveto(859.84990638,322.71791653)(859.78990644,322.75791649)(859.75991317,322.81791658)
\curveto(859.71990651,322.86791638)(859.69490653,322.92791632)(859.68491317,322.99791658)
\curveto(859.68490654,323.06791618)(859.66990656,323.13791611)(859.63991317,323.20791658)
\curveto(859.6299066,323.22791602)(859.61490661,323.24291601)(859.59491317,323.25291658)
\curveto(859.58490664,323.27291598)(859.56990666,323.29291596)(859.54991317,323.31291658)
\curveto(859.44990678,323.32291593)(859.36990686,323.30291595)(859.30991317,323.25291658)
\curveto(859.25990697,323.20291605)(859.20490702,323.1529161)(859.14491317,323.10291658)
\curveto(858.94490728,322.9529163)(858.74490748,322.83791641)(858.54491317,322.75791658)
\curveto(858.36490786,322.67791657)(858.15490807,322.61791663)(857.91491317,322.57791658)
\curveto(857.68490854,322.53791671)(857.44490878,322.51791673)(857.19491317,322.51791658)
\curveto(856.95490927,322.50791674)(856.71490951,322.52291673)(856.47491317,322.56291658)
\curveto(856.23490999,322.59291666)(856.0249102,322.6479166)(855.84491317,322.72791658)
\curveto(855.3249109,322.9479163)(854.90491132,323.24291601)(854.58491317,323.61291658)
\curveto(854.26491196,323.99291526)(854.01491221,324.46291479)(853.83491317,325.02291658)
\curveto(853.79491243,325.11291414)(853.76491246,325.20291405)(853.74491317,325.29291658)
\curveto(853.73491249,325.39291386)(853.71491251,325.49291376)(853.68491317,325.59291658)
\curveto(853.67491255,325.64291361)(853.66991256,325.69291356)(853.66991317,325.74291658)
\curveto(853.66991256,325.79291346)(853.66491256,325.84291341)(853.65491317,325.89291658)
\curveto(853.63491259,325.94291331)(853.6249126,325.99291326)(853.62491317,326.04291658)
\curveto(853.63491259,326.10291315)(853.63491259,326.15791309)(853.62491317,326.20791658)
\lineto(853.62491317,326.35791658)
\curveto(853.60491262,326.40791284)(853.59491263,326.47291278)(853.59491317,326.55291658)
\curveto(853.59491263,326.63291262)(853.60491262,326.69791255)(853.62491317,326.74791658)
\lineto(853.62491317,326.91291658)
\curveto(853.64491258,326.98291227)(853.64991258,327.0529122)(853.63991317,327.12291658)
\curveto(853.63991259,327.20291205)(853.64991258,327.27791197)(853.66991317,327.34791658)
\curveto(853.67991255,327.39791185)(853.68491254,327.44291181)(853.68491317,327.48291658)
\curveto(853.68491254,327.52291173)(853.68991254,327.56791168)(853.69991317,327.61791658)
\curveto(853.7299125,327.71791153)(853.75491247,327.81291144)(853.77491317,327.90291658)
\curveto(853.79491243,328.00291125)(853.81991241,328.09791115)(853.84991317,328.18791658)
\curveto(853.97991225,328.56791068)(854.14491208,328.90791034)(854.34491317,329.20791658)
\curveto(854.55491167,329.51790973)(854.80491142,329.77290948)(855.09491317,329.97291658)
\curveto(855.26491096,330.09290916)(855.43991079,330.19290906)(855.61991317,330.27291658)
\curveto(855.80991042,330.3529089)(856.01491021,330.42290883)(856.23491317,330.48291658)
\curveto(856.30490992,330.49290876)(856.36990986,330.50290875)(856.42991317,330.51291658)
\curveto(856.49990973,330.52290873)(856.56990966,330.53790871)(856.63991317,330.55791658)
\lineto(856.78991317,330.55791658)
\curveto(856.86990936,330.57790867)(856.98490924,330.58790866)(857.13491317,330.58791658)
\curveto(857.29490893,330.58790866)(857.41490881,330.57790867)(857.49491317,330.55791658)
\curveto(857.53490869,330.5479087)(857.58990864,330.54290871)(857.65991317,330.54291658)
\curveto(857.76990846,330.51290874)(857.87990835,330.48790876)(857.98991317,330.46791658)
\curveto(858.09990813,330.45790879)(858.20490802,330.42790882)(858.30491317,330.37791658)
\curveto(858.45490777,330.31790893)(858.59490763,330.252909)(858.72491317,330.18291658)
\curveto(858.86490736,330.11290914)(858.99490723,330.03290922)(859.11491317,329.94291658)
\curveto(859.17490705,329.89290936)(859.23490699,329.83790941)(859.29491317,329.77791658)
\curveto(859.36490686,329.72790952)(859.45490677,329.71290954)(859.56491317,329.73291658)
\curveto(859.58490664,329.76290949)(859.59990663,329.78790946)(859.60991317,329.80791658)
\curveto(859.6299066,329.82790942)(859.64490658,329.85790939)(859.65491317,329.89791658)
\curveto(859.68490654,329.98790926)(859.69490653,330.10290915)(859.68491317,330.24291658)
\lineto(859.68491317,330.61791658)
\lineto(859.68491317,332.34291658)
\lineto(859.68491317,332.80791658)
\curveto(859.68490654,332.98790626)(859.70990652,333.11790613)(859.75991317,333.19791658)
\curveto(859.79990643,333.26790598)(859.85990637,333.31290594)(859.93991317,333.33291658)
\curveto(859.95990627,333.33290592)(859.98490624,333.33290592)(860.01491317,333.33291658)
\curveto(860.04490618,333.34290591)(860.06990616,333.3479059)(860.08991317,333.34791658)
\curveto(860.229906,333.35790589)(860.37490585,333.35790589)(860.52491317,333.34791658)
\curveto(860.68490554,333.3479059)(860.79490543,333.30790594)(860.85491317,333.22791658)
\curveto(860.90490532,333.1479061)(860.9299053,333.0479062)(860.92991317,332.92791658)
\lineto(860.92991317,332.55291658)
\lineto(860.92991317,323.49291658)
\moveto(859.71491317,326.32791658)
\curveto(859.73490649,326.37791287)(859.74490648,326.44291281)(859.74491317,326.52291658)
\curveto(859.74490648,326.61291264)(859.73490649,326.68291257)(859.71491317,326.73291658)
\lineto(859.71491317,326.95791658)
\curveto(859.69490653,327.0479122)(859.67990655,327.13791211)(859.66991317,327.22791658)
\curveto(859.65990657,327.32791192)(859.63990659,327.41791183)(859.60991317,327.49791658)
\curveto(859.58990664,327.57791167)(859.56990666,327.6529116)(859.54991317,327.72291658)
\curveto(859.53990669,327.79291146)(859.51990671,327.86291139)(859.48991317,327.93291658)
\curveto(859.36990686,328.23291102)(859.21490701,328.49791075)(859.02491317,328.72791658)
\curveto(858.83490739,328.95791029)(858.59490763,329.13791011)(858.30491317,329.26791658)
\curveto(858.20490802,329.31790993)(858.09990813,329.3529099)(857.98991317,329.37291658)
\curveto(857.88990834,329.40290985)(857.77990845,329.42790982)(857.65991317,329.44791658)
\curveto(857.57990865,329.46790978)(857.48990874,329.47790977)(857.38991317,329.47791658)
\lineto(857.11991317,329.47791658)
\curveto(857.06990916,329.46790978)(857.0249092,329.45790979)(856.98491317,329.44791658)
\lineto(856.84991317,329.44791658)
\curveto(856.76990946,329.42790982)(856.68490954,329.40790984)(856.59491317,329.38791658)
\curveto(856.51490971,329.36790988)(856.43490979,329.34290991)(856.35491317,329.31291658)
\curveto(856.03491019,329.17291008)(855.77491045,328.96791028)(855.57491317,328.69791658)
\curveto(855.38491084,328.43791081)(855.229911,328.13291112)(855.10991317,327.78291658)
\curveto(855.06991116,327.67291158)(855.03991119,327.55791169)(855.01991317,327.43791658)
\curveto(855.00991122,327.32791192)(854.99491123,327.21791203)(854.97491317,327.10791658)
\curveto(854.97491125,327.06791218)(854.96991126,327.02791222)(854.95991317,326.98791658)
\lineto(854.95991317,326.88291658)
\curveto(854.93991129,326.83291242)(854.9299113,326.77791247)(854.92991317,326.71791658)
\curveto(854.93991129,326.65791259)(854.94491128,326.60291265)(854.94491317,326.55291658)
\lineto(854.94491317,326.22291658)
\curveto(854.94491128,326.12291313)(854.95491127,326.02791322)(854.97491317,325.93791658)
\curveto(854.98491124,325.90791334)(854.98991124,325.85791339)(854.98991317,325.78791658)
\curveto(855.00991122,325.71791353)(855.0249112,325.6479136)(855.03491317,325.57791658)
\lineto(855.09491317,325.36791658)
\curveto(855.20491102,325.01791423)(855.35491087,324.71791453)(855.54491317,324.46791658)
\curveto(855.73491049,324.21791503)(855.97491025,324.01291524)(856.26491317,323.85291658)
\curveto(856.35490987,323.80291545)(856.44490978,323.76291549)(856.53491317,323.73291658)
\curveto(856.6249096,323.70291555)(856.7249095,323.67291558)(856.83491317,323.64291658)
\curveto(856.88490934,323.62291563)(856.93490929,323.61791563)(856.98491317,323.62791658)
\curveto(857.04490918,323.63791561)(857.09990913,323.63291562)(857.14991317,323.61291658)
\curveto(857.18990904,323.60291565)(857.229909,323.59791565)(857.26991317,323.59791658)
\lineto(857.40491317,323.59791658)
\lineto(857.53991317,323.59791658)
\curveto(857.56990866,323.60791564)(857.61990861,323.61291564)(857.68991317,323.61291658)
\curveto(857.76990846,323.63291562)(857.84990838,323.6479156)(857.92991317,323.65791658)
\curveto(858.00990822,323.67791557)(858.08490814,323.70291555)(858.15491317,323.73291658)
\curveto(858.48490774,323.87291538)(858.74990748,324.0479152)(858.94991317,324.25791658)
\curveto(859.15990707,324.47791477)(859.33490689,324.7529145)(859.47491317,325.08291658)
\curveto(859.5249067,325.19291406)(859.55990667,325.30291395)(859.57991317,325.41291658)
\curveto(859.59990663,325.52291373)(859.6249066,325.63291362)(859.65491317,325.74291658)
\curveto(859.67490655,325.78291347)(859.68490654,325.81791343)(859.68491317,325.84791658)
\curveto(859.68490654,325.88791336)(859.68990654,325.92791332)(859.69991317,325.96791658)
\curveto(859.70990652,326.02791322)(859.70990652,326.08791316)(859.69991317,326.14791658)
\curveto(859.69990653,326.20791304)(859.70490652,326.26791298)(859.71491317,326.32791658)
}
}
{
\newrgbcolor{curcolor}{0 0 0}
\pscustom[linestyle=none,fillstyle=solid,fillcolor=curcolor]
{
\newpath
\moveto(869.76116317,323.23791658)
\curveto(869.79115534,323.07791617)(869.77615536,322.94291631)(869.71616317,322.83291658)
\curveto(869.65615548,322.73291652)(869.57615556,322.65791659)(869.47616317,322.60791658)
\curveto(869.42615571,322.58791666)(869.37115576,322.57791667)(869.31116317,322.57791658)
\curveto(869.26115587,322.57791667)(869.20615593,322.56791668)(869.14616317,322.54791658)
\curveto(868.92615621,322.49791675)(868.70615643,322.51291674)(868.48616317,322.59291658)
\curveto(868.27615686,322.66291659)(868.131157,322.7529165)(868.05116317,322.86291658)
\curveto(868.00115713,322.93291632)(867.95615718,323.01291624)(867.91616317,323.10291658)
\curveto(867.87615726,323.20291605)(867.82615731,323.28291597)(867.76616317,323.34291658)
\curveto(867.74615739,323.36291589)(867.72115741,323.38291587)(867.69116317,323.40291658)
\curveto(867.67115746,323.42291583)(867.64115749,323.42791582)(867.60116317,323.41791658)
\curveto(867.49115764,323.38791586)(867.38615775,323.33291592)(867.28616317,323.25291658)
\curveto(867.19615794,323.17291608)(867.10615803,323.10291615)(867.01616317,323.04291658)
\curveto(866.88615825,322.96291629)(866.74615839,322.88791636)(866.59616317,322.81791658)
\curveto(866.44615869,322.75791649)(866.28615885,322.70291655)(866.11616317,322.65291658)
\curveto(866.01615912,322.62291663)(865.90615923,322.60291665)(865.78616317,322.59291658)
\curveto(865.67615946,322.58291667)(865.56615957,322.56791668)(865.45616317,322.54791658)
\curveto(865.40615973,322.53791671)(865.36115977,322.53291672)(865.32116317,322.53291658)
\lineto(865.21616317,322.53291658)
\curveto(865.10616003,322.51291674)(865.00116013,322.51291674)(864.90116317,322.53291658)
\lineto(864.76616317,322.53291658)
\curveto(864.71616042,322.54291671)(864.66616047,322.5479167)(864.61616317,322.54791658)
\curveto(864.56616057,322.5479167)(864.52116061,322.55791669)(864.48116317,322.57791658)
\curveto(864.44116069,322.58791666)(864.40616073,322.59291666)(864.37616317,322.59291658)
\curveto(864.35616078,322.58291667)(864.3311608,322.58291667)(864.30116317,322.59291658)
\lineto(864.06116317,322.65291658)
\curveto(863.98116115,322.66291659)(863.90616123,322.68291657)(863.83616317,322.71291658)
\curveto(863.5361616,322.84291641)(863.29116184,322.98791626)(863.10116317,323.14791658)
\curveto(862.92116221,323.31791593)(862.77116236,323.5529157)(862.65116317,323.85291658)
\curveto(862.56116257,324.07291518)(862.51616262,324.33791491)(862.51616317,324.64791658)
\lineto(862.51616317,324.96291658)
\curveto(862.52616261,325.01291424)(862.5311626,325.06291419)(862.53116317,325.11291658)
\lineto(862.56116317,325.29291658)
\lineto(862.68116317,325.62291658)
\curveto(862.72116241,325.73291352)(862.77116236,325.83291342)(862.83116317,325.92291658)
\curveto(863.01116212,326.21291304)(863.25616188,326.42791282)(863.56616317,326.56791658)
\curveto(863.87616126,326.70791254)(864.21616092,326.83291242)(864.58616317,326.94291658)
\curveto(864.72616041,326.98291227)(864.87116026,327.01291224)(865.02116317,327.03291658)
\curveto(865.17115996,327.0529122)(865.32115981,327.07791217)(865.47116317,327.10791658)
\curveto(865.54115959,327.12791212)(865.60615953,327.13791211)(865.66616317,327.13791658)
\curveto(865.7361594,327.13791211)(865.81115932,327.1479121)(865.89116317,327.16791658)
\curveto(865.96115917,327.18791206)(866.0311591,327.19791205)(866.10116317,327.19791658)
\curveto(866.17115896,327.20791204)(866.24615889,327.22291203)(866.32616317,327.24291658)
\curveto(866.57615856,327.30291195)(866.81115832,327.3529119)(867.03116317,327.39291658)
\curveto(867.25115788,327.44291181)(867.42615771,327.55791169)(867.55616317,327.73791658)
\curveto(867.61615752,327.81791143)(867.66615747,327.91791133)(867.70616317,328.03791658)
\curveto(867.74615739,328.16791108)(867.74615739,328.30791094)(867.70616317,328.45791658)
\curveto(867.64615749,328.69791055)(867.55615758,328.88791036)(867.43616317,329.02791658)
\curveto(867.32615781,329.16791008)(867.16615797,329.27790997)(866.95616317,329.35791658)
\curveto(866.8361583,329.40790984)(866.69115844,329.44290981)(866.52116317,329.46291658)
\curveto(866.36115877,329.48290977)(866.19115894,329.49290976)(866.01116317,329.49291658)
\curveto(865.8311593,329.49290976)(865.65615948,329.48290977)(865.48616317,329.46291658)
\curveto(865.31615982,329.44290981)(865.17115996,329.41290984)(865.05116317,329.37291658)
\curveto(864.88116025,329.31290994)(864.71616042,329.22791002)(864.55616317,329.11791658)
\curveto(864.47616066,329.05791019)(864.40116073,328.97791027)(864.33116317,328.87791658)
\curveto(864.27116086,328.78791046)(864.21616092,328.68791056)(864.16616317,328.57791658)
\curveto(864.136161,328.49791075)(864.10616103,328.41291084)(864.07616317,328.32291658)
\curveto(864.05616108,328.23291102)(864.01116112,328.16291109)(863.94116317,328.11291658)
\curveto(863.90116123,328.08291117)(863.8311613,328.05791119)(863.73116317,328.03791658)
\curveto(863.64116149,328.02791122)(863.54616159,328.02291123)(863.44616317,328.02291658)
\curveto(863.34616179,328.02291123)(863.24616189,328.02791122)(863.14616317,328.03791658)
\curveto(863.05616208,328.05791119)(862.99116214,328.08291117)(862.95116317,328.11291658)
\curveto(862.91116222,328.14291111)(862.88116225,328.19291106)(862.86116317,328.26291658)
\curveto(862.84116229,328.33291092)(862.84116229,328.40791084)(862.86116317,328.48791658)
\curveto(862.89116224,328.61791063)(862.92116221,328.73791051)(862.95116317,328.84791658)
\curveto(862.99116214,328.96791028)(863.0361621,329.08291017)(863.08616317,329.19291658)
\curveto(863.27616186,329.54290971)(863.51616162,329.81290944)(863.80616317,330.00291658)
\curveto(864.09616104,330.20290905)(864.45616068,330.36290889)(864.88616317,330.48291658)
\curveto(864.98616015,330.50290875)(865.08616005,330.51790873)(865.18616317,330.52791658)
\curveto(865.29615984,330.53790871)(865.40615973,330.5529087)(865.51616317,330.57291658)
\curveto(865.55615958,330.58290867)(865.62115951,330.58290867)(865.71116317,330.57291658)
\curveto(865.80115933,330.57290868)(865.85615928,330.58290867)(865.87616317,330.60291658)
\curveto(866.57615856,330.61290864)(867.18615795,330.53290872)(867.70616317,330.36291658)
\curveto(868.22615691,330.19290906)(868.59115654,329.86790938)(868.80116317,329.38791658)
\curveto(868.89115624,329.18791006)(868.94115619,328.9529103)(868.95116317,328.68291658)
\curveto(868.97115616,328.42291083)(868.98115615,328.1479111)(868.98116317,327.85791658)
\lineto(868.98116317,324.54291658)
\curveto(868.98115615,324.40291485)(868.98615615,324.26791498)(868.99616317,324.13791658)
\curveto(869.00615613,324.00791524)(869.0361561,323.90291535)(869.08616317,323.82291658)
\curveto(869.136156,323.7529155)(869.20115593,323.70291555)(869.28116317,323.67291658)
\curveto(869.37115576,323.63291562)(869.45615568,323.60291565)(869.53616317,323.58291658)
\curveto(869.61615552,323.57291568)(869.67615546,323.52791572)(869.71616317,323.44791658)
\curveto(869.7361554,323.41791583)(869.74615539,323.38791586)(869.74616317,323.35791658)
\curveto(869.74615539,323.32791592)(869.75115538,323.28791596)(869.76116317,323.23791658)
\moveto(867.61616317,324.90291658)
\curveto(867.67615746,325.04291421)(867.70615743,325.20291405)(867.70616317,325.38291658)
\curveto(867.71615742,325.57291368)(867.72115741,325.76791348)(867.72116317,325.96791658)
\curveto(867.72115741,326.07791317)(867.71615742,326.17791307)(867.70616317,326.26791658)
\curveto(867.69615744,326.35791289)(867.65615748,326.42791282)(867.58616317,326.47791658)
\curveto(867.55615758,326.49791275)(867.48615765,326.50791274)(867.37616317,326.50791658)
\curveto(867.35615778,326.48791276)(867.32115781,326.47791277)(867.27116317,326.47791658)
\curveto(867.22115791,326.47791277)(867.17615796,326.46791278)(867.13616317,326.44791658)
\curveto(867.05615808,326.42791282)(866.96615817,326.40791284)(866.86616317,326.38791658)
\lineto(866.56616317,326.32791658)
\curveto(866.5361586,326.32791292)(866.50115863,326.32291293)(866.46116317,326.31291658)
\lineto(866.35616317,326.31291658)
\curveto(866.20615893,326.27291298)(866.04115909,326.247913)(865.86116317,326.23791658)
\curveto(865.69115944,326.23791301)(865.5311596,326.21791303)(865.38116317,326.17791658)
\curveto(865.30115983,326.15791309)(865.22615991,326.13791311)(865.15616317,326.11791658)
\curveto(865.09616004,326.10791314)(865.02616011,326.09291316)(864.94616317,326.07291658)
\curveto(864.78616035,326.02291323)(864.6361605,325.95791329)(864.49616317,325.87791658)
\curveto(864.35616078,325.80791344)(864.2361609,325.71791353)(864.13616317,325.60791658)
\curveto(864.0361611,325.49791375)(863.96116117,325.36291389)(863.91116317,325.20291658)
\curveto(863.86116127,325.0529142)(863.84116129,324.86791438)(863.85116317,324.64791658)
\curveto(863.85116128,324.5479147)(863.86616127,324.4529148)(863.89616317,324.36291658)
\curveto(863.9361612,324.28291497)(863.98116115,324.20791504)(864.03116317,324.13791658)
\curveto(864.11116102,324.02791522)(864.21616092,323.93291532)(864.34616317,323.85291658)
\curveto(864.47616066,323.78291547)(864.61616052,323.72291553)(864.76616317,323.67291658)
\curveto(864.81616032,323.66291559)(864.86616027,323.65791559)(864.91616317,323.65791658)
\curveto(864.96616017,323.65791559)(865.01616012,323.6529156)(865.06616317,323.64291658)
\curveto(865.13616,323.62291563)(865.22115991,323.60791564)(865.32116317,323.59791658)
\curveto(865.4311597,323.59791565)(865.52115961,323.60791564)(865.59116317,323.62791658)
\curveto(865.65115948,323.6479156)(865.71115942,323.6529156)(865.77116317,323.64291658)
\curveto(865.8311593,323.64291561)(865.89115924,323.6529156)(865.95116317,323.67291658)
\curveto(866.0311591,323.69291556)(866.10615903,323.70791554)(866.17616317,323.71791658)
\curveto(866.25615888,323.72791552)(866.3311588,323.7479155)(866.40116317,323.77791658)
\curveto(866.69115844,323.89791535)(866.9361582,324.04291521)(867.13616317,324.21291658)
\curveto(867.34615779,324.38291487)(867.50615763,324.61291464)(867.61616317,324.90291658)
}
}
{
\newrgbcolor{curcolor}{0.90196079 0.90196079 0.90196079}
\pscustom[linestyle=none,fillstyle=solid,fillcolor=curcolor]
{
\newpath
\moveto(798.51865829,333.3929532)
\lineto(813.51865829,333.3929532)
\lineto(813.51865829,318.3929532)
\lineto(798.51865829,318.3929532)
\closepath
}
}
{
\newrgbcolor{curcolor}{0 0 0}
\pscustom[linestyle=none,fillstyle=solid,fillcolor=curcolor]
{
\newpath
\moveto(822.4518663,310.56721102)
\curveto(823.4318598,310.58720006)(824.25185898,310.42720022)(824.9118663,310.08721102)
\curveto(825.58185765,309.75720089)(826.10185713,309.29720135)(826.4718663,308.70721102)
\curveto(826.57185666,308.5472021)(826.65185658,308.39220226)(826.7118663,308.24221102)
\curveto(826.78185645,308.10220255)(826.84685638,307.93220272)(826.9068663,307.73221102)
\curveto(826.9268563,307.68220297)(826.94685628,307.61220304)(826.9668663,307.52221102)
\curveto(826.98685624,307.44220321)(826.98185625,307.36720328)(826.9518663,307.29721102)
\curveto(826.9318563,307.23720341)(826.89185634,307.19720345)(826.8318663,307.17721102)
\curveto(826.78185645,307.16720348)(826.7268565,307.1522035)(826.6668663,307.13221102)
\lineto(826.5168663,307.13221102)
\curveto(826.48685674,307.12220353)(826.44685678,307.11720353)(826.3968663,307.11721102)
\lineto(826.2768663,307.11721102)
\curveto(826.13685709,307.11720353)(826.00685722,307.12220353)(825.8868663,307.13221102)
\curveto(825.77685745,307.1522035)(825.69685753,307.20220345)(825.6468663,307.28221102)
\curveto(825.57685765,307.38220327)(825.52185771,307.49720315)(825.4818663,307.62721102)
\curveto(825.44185779,307.75720289)(825.38685784,307.87720277)(825.3168663,307.98721102)
\curveto(825.18685804,308.20720244)(825.03685819,308.39720225)(824.8668663,308.55721102)
\curveto(824.70685852,308.71720193)(824.51685871,308.86720178)(824.2968663,309.00721102)
\curveto(824.17685905,309.08720156)(824.04185919,309.1472015)(823.8918663,309.18721102)
\curveto(823.75185948,309.22720142)(823.60685962,309.26720138)(823.4568663,309.30721102)
\curveto(823.34685988,309.33720131)(823.22186001,309.35720129)(823.0818663,309.36721102)
\curveto(822.94186029,309.38720126)(822.79186044,309.39720125)(822.6318663,309.39721102)
\curveto(822.48186075,309.39720125)(822.3318609,309.38720126)(822.1818663,309.36721102)
\curveto(822.04186119,309.35720129)(821.92186131,309.33720131)(821.8218663,309.30721102)
\curveto(821.72186151,309.28720136)(821.6268616,309.26720138)(821.5368663,309.24721102)
\curveto(821.44686178,309.22720142)(821.35686187,309.19720145)(821.2668663,309.15721102)
\curveto(820.4268628,308.80720184)(819.82186341,308.20720244)(819.4518663,307.35721102)
\curveto(819.38186385,307.21720343)(819.32186391,307.06720358)(819.2718663,306.90721102)
\curveto(819.231864,306.75720389)(819.18686404,306.60220405)(819.1368663,306.44221102)
\curveto(819.11686411,306.38220427)(819.10686412,306.31720433)(819.1068663,306.24721102)
\curveto(819.10686412,306.18720446)(819.09686413,306.12720452)(819.0768663,306.06721102)
\curveto(819.06686416,306.02720462)(819.06186417,305.99220466)(819.0618663,305.96221102)
\curveto(819.06186417,305.93220472)(819.05686417,305.89720475)(819.0468663,305.85721102)
\curveto(819.0268642,305.7472049)(819.01186422,305.63220502)(819.0018663,305.51221102)
\lineto(819.0018663,305.16721102)
\curveto(819.00186423,305.09720555)(818.99686423,305.02220563)(818.9868663,304.94221102)
\curveto(818.98686424,304.87220578)(818.99186424,304.80720584)(819.0018663,304.74721102)
\lineto(819.0018663,304.59721102)
\curveto(819.02186421,304.52720612)(819.0268642,304.45720619)(819.0168663,304.38721102)
\curveto(819.01686421,304.31720633)(819.0268642,304.2472064)(819.0468663,304.17721102)
\curveto(819.06686416,304.11720653)(819.07186416,304.05720659)(819.0618663,303.99721102)
\curveto(819.06186417,303.93720671)(819.07186416,303.88220677)(819.0918663,303.83221102)
\curveto(819.12186411,303.70220695)(819.14686408,303.57220708)(819.1668663,303.44221102)
\curveto(819.19686403,303.32220733)(819.231864,303.20220745)(819.2718663,303.08221102)
\curveto(819.44186379,302.58220807)(819.66186357,302.1522085)(819.9318663,301.79221102)
\curveto(820.20186303,301.44220921)(820.55686267,301.1522095)(820.9968663,300.92221102)
\curveto(821.13686209,300.8522098)(821.27686195,300.79720985)(821.4168663,300.75721102)
\curveto(821.56686166,300.71720993)(821.7268615,300.67220998)(821.8968663,300.62221102)
\curveto(821.96686126,300.60221005)(822.0318612,300.59221006)(822.0918663,300.59221102)
\curveto(822.15186108,300.60221005)(822.22186101,300.59721005)(822.3018663,300.57721102)
\curveto(822.35186088,300.56721008)(822.44186079,300.55721009)(822.5718663,300.54721102)
\curveto(822.70186053,300.5472101)(822.79686043,300.55721009)(822.8568663,300.57721102)
\lineto(822.9618663,300.57721102)
\curveto(823.00186023,300.58721006)(823.04186019,300.58721006)(823.0818663,300.57721102)
\curveto(823.12186011,300.57721007)(823.16186007,300.58721006)(823.2018663,300.60721102)
\curveto(823.30185993,300.62721002)(823.39685983,300.64221001)(823.4868663,300.65221102)
\curveto(823.58685964,300.67220998)(823.68185955,300.70220995)(823.7718663,300.74221102)
\curveto(824.55185868,301.06220959)(825.10185813,301.58720906)(825.4218663,302.31721102)
\curveto(825.50185773,302.49720815)(825.57685765,302.71220794)(825.6468663,302.96221102)
\curveto(825.66685756,303.0522076)(825.68185755,303.14220751)(825.6918663,303.23221102)
\curveto(825.71185752,303.33220732)(825.74685748,303.42220723)(825.7968663,303.50221102)
\curveto(825.84685738,303.58220707)(825.9268573,303.62720702)(826.0368663,303.63721102)
\curveto(826.14685708,303.647207)(826.26685696,303.652207)(826.3968663,303.65221102)
\lineto(826.5468663,303.65221102)
\curveto(826.59685663,303.652207)(826.64185659,303.647207)(826.6818663,303.63721102)
\lineto(826.7868663,303.63721102)
\lineto(826.8768663,303.60721102)
\curveto(826.91685631,303.60720704)(826.94685628,303.59720705)(826.9668663,303.57721102)
\curveto(827.03685619,303.53720711)(827.07685615,303.46220719)(827.0868663,303.35221102)
\curveto(827.09685613,303.2522074)(827.08685614,303.1522075)(827.0568663,303.05221102)
\curveto(826.99685623,302.82220783)(826.94185629,302.60220805)(826.8918663,302.39221102)
\curveto(826.84185639,302.18220847)(826.76685646,301.98220867)(826.6668663,301.79221102)
\curveto(826.58685664,301.66220899)(826.51185672,301.53720911)(826.4418663,301.41721102)
\curveto(826.38185685,301.29720935)(826.31185692,301.17720947)(826.2318663,301.05721102)
\curveto(826.05185718,300.79720985)(825.8268574,300.55721009)(825.5568663,300.33721102)
\curveto(825.29685793,300.12721052)(825.01185822,299.9522107)(824.7018663,299.81221102)
\curveto(824.59185864,299.76221089)(824.48185875,299.72221093)(824.3718663,299.69221102)
\curveto(824.27185896,299.66221099)(824.16685906,299.62721102)(824.0568663,299.58721102)
\curveto(823.94685928,299.5472111)(823.8318594,299.52221113)(823.7118663,299.51221102)
\curveto(823.60185963,299.49221116)(823.48685974,299.47221118)(823.3668663,299.45221102)
\curveto(823.31685991,299.43221122)(823.27185996,299.42721122)(823.2318663,299.43721102)
\curveto(823.19186004,299.43721121)(823.15186008,299.43221122)(823.1118663,299.42221102)
\curveto(823.05186018,299.41221124)(822.99186024,299.40721124)(822.9318663,299.40721102)
\curveto(822.87186036,299.40721124)(822.80686042,299.40221125)(822.7368663,299.39221102)
\curveto(822.70686052,299.38221127)(822.63686059,299.38221127)(822.5268663,299.39221102)
\curveto(822.4268608,299.39221126)(822.36186087,299.39721125)(822.3318663,299.40721102)
\curveto(822.28186095,299.41721123)(822.231861,299.42221123)(822.1818663,299.42221102)
\curveto(822.14186109,299.41221124)(822.09686113,299.41221124)(822.0468663,299.42221102)
\lineto(821.8968663,299.42221102)
\curveto(821.81686141,299.44221121)(821.74186149,299.45721119)(821.6718663,299.46721102)
\curveto(821.60186163,299.46721118)(821.5268617,299.47721117)(821.4468663,299.49721102)
\lineto(821.1768663,299.55721102)
\curveto(821.08686214,299.56721108)(821.00186223,299.58721106)(820.9218663,299.61721102)
\curveto(820.71186252,299.67721097)(820.52186271,299.7522109)(820.3518663,299.84221102)
\curveto(819.72186351,300.11221054)(819.21186402,300.49721015)(818.8218663,300.99721102)
\curveto(818.4318648,301.49720915)(818.12186511,302.08720856)(817.8918663,302.76721102)
\curveto(817.85186538,302.88720776)(817.81686541,303.01220764)(817.7868663,303.14221102)
\curveto(817.76686546,303.27220738)(817.74186549,303.40720724)(817.7118663,303.54721102)
\curveto(817.69186554,303.59720705)(817.68186555,303.64220701)(817.6818663,303.68221102)
\curveto(817.69186554,303.72220693)(817.69186554,303.76720688)(817.6818663,303.81721102)
\curveto(817.66186557,303.90720674)(817.64686558,304.00220665)(817.6368663,304.10221102)
\curveto(817.63686559,304.20220645)(817.6268656,304.29720635)(817.6068663,304.38721102)
\lineto(817.6068663,304.67221102)
\curveto(817.58686564,304.72220593)(817.57686565,304.80720584)(817.5768663,304.92721102)
\curveto(817.57686565,305.0472056)(817.58686564,305.13220552)(817.6068663,305.18221102)
\curveto(817.61686561,305.21220544)(817.61686561,305.24220541)(817.6068663,305.27221102)
\curveto(817.59686563,305.31220534)(817.59686563,305.34220531)(817.6068663,305.36221102)
\lineto(817.6068663,305.49721102)
\curveto(817.61686561,305.57720507)(817.62186561,305.65720499)(817.6218663,305.73721102)
\curveto(817.6318656,305.82720482)(817.64686558,305.91220474)(817.6668663,305.99221102)
\curveto(817.68686554,306.0522046)(817.69686553,306.11220454)(817.6968663,306.17221102)
\curveto(817.69686553,306.24220441)(817.70686552,306.31220434)(817.7268663,306.38221102)
\curveto(817.77686545,306.5522041)(817.81686541,306.71720393)(817.8468663,306.87721102)
\curveto(817.87686535,307.03720361)(817.92186531,307.18720346)(817.9818663,307.32721102)
\lineto(818.1318663,307.71721102)
\curveto(818.19186504,307.85720279)(818.25686497,307.98220267)(818.3268663,308.09221102)
\curveto(818.47686475,308.3522023)(818.6268646,308.58720206)(818.7768663,308.79721102)
\curveto(818.80686442,308.8472018)(818.84186439,308.88720176)(818.8818663,308.91721102)
\curveto(818.9318643,308.95720169)(818.97186426,309.00220165)(819.0018663,309.05221102)
\curveto(819.06186417,309.13220152)(819.12186411,309.20220145)(819.1818663,309.26221102)
\lineto(819.3918663,309.44221102)
\curveto(819.45186378,309.49220116)(819.50686372,309.53720111)(819.5568663,309.57721102)
\curveto(819.61686361,309.62720102)(819.68186355,309.67720097)(819.7518663,309.72721102)
\curveto(819.90186333,309.83720081)(820.05686317,309.93220072)(820.2168663,310.01221102)
\curveto(820.38686284,310.09220056)(820.56186267,310.17220048)(820.7418663,310.25221102)
\curveto(820.85186238,310.30220035)(820.96686226,310.33720031)(821.0868663,310.35721102)
\curveto(821.21686201,310.38720026)(821.34186189,310.42220023)(821.4618663,310.46221102)
\curveto(821.5318617,310.47220018)(821.59686163,310.48220017)(821.6568663,310.49221102)
\lineto(821.8368663,310.52221102)
\curveto(821.91686131,310.53220012)(821.99186124,310.53720011)(822.0618663,310.53721102)
\curveto(822.14186109,310.5472001)(822.22186101,310.55720009)(822.3018663,310.56721102)
\curveto(822.32186091,310.57720007)(822.34686088,310.57720007)(822.3768663,310.56721102)
\curveto(822.40686082,310.55720009)(822.4318608,310.55720009)(822.4518663,310.56721102)
}
}
{
\newrgbcolor{curcolor}{0 0 0}
\pscustom[linestyle=none,fillstyle=solid,fillcolor=curcolor]
{
\newpath
\moveto(835.57171005,300.20221102)
\curveto(835.60170222,300.04221061)(835.58670223,299.90721074)(835.52671005,299.79721102)
\curveto(835.46670235,299.69721095)(835.38670243,299.62221103)(835.28671005,299.57221102)
\curveto(835.23670258,299.5522111)(835.18170264,299.54221111)(835.12171005,299.54221102)
\curveto(835.07170275,299.54221111)(835.0167028,299.53221112)(834.95671005,299.51221102)
\curveto(834.73670308,299.46221119)(834.5167033,299.47721117)(834.29671005,299.55721102)
\curveto(834.08670373,299.62721102)(833.94170388,299.71721093)(833.86171005,299.82721102)
\curveto(833.81170401,299.89721075)(833.76670405,299.97721067)(833.72671005,300.06721102)
\curveto(833.68670413,300.16721048)(833.63670418,300.2472104)(833.57671005,300.30721102)
\curveto(833.55670426,300.32721032)(833.53170429,300.3472103)(833.50171005,300.36721102)
\curveto(833.48170434,300.38721026)(833.45170437,300.39221026)(833.41171005,300.38221102)
\curveto(833.30170452,300.3522103)(833.19670462,300.29721035)(833.09671005,300.21721102)
\curveto(833.00670481,300.13721051)(832.9167049,300.06721058)(832.82671005,300.00721102)
\curveto(832.69670512,299.92721072)(832.55670526,299.8522108)(832.40671005,299.78221102)
\curveto(832.25670556,299.72221093)(832.09670572,299.66721098)(831.92671005,299.61721102)
\curveto(831.82670599,299.58721106)(831.7167061,299.56721108)(831.59671005,299.55721102)
\curveto(831.48670633,299.5472111)(831.37670644,299.53221112)(831.26671005,299.51221102)
\curveto(831.2167066,299.50221115)(831.17170665,299.49721115)(831.13171005,299.49721102)
\lineto(831.02671005,299.49721102)
\curveto(830.9167069,299.47721117)(830.81170701,299.47721117)(830.71171005,299.49721102)
\lineto(830.57671005,299.49721102)
\curveto(830.52670729,299.50721114)(830.47670734,299.51221114)(830.42671005,299.51221102)
\curveto(830.37670744,299.51221114)(830.33170749,299.52221113)(830.29171005,299.54221102)
\curveto(830.25170757,299.5522111)(830.2167076,299.55721109)(830.18671005,299.55721102)
\curveto(830.16670765,299.5472111)(830.14170768,299.5472111)(830.11171005,299.55721102)
\lineto(829.87171005,299.61721102)
\curveto(829.79170803,299.62721102)(829.7167081,299.647211)(829.64671005,299.67721102)
\curveto(829.34670847,299.80721084)(829.10170872,299.9522107)(828.91171005,300.11221102)
\curveto(828.73170909,300.28221037)(828.58170924,300.51721013)(828.46171005,300.81721102)
\curveto(828.37170945,301.03720961)(828.32670949,301.30220935)(828.32671005,301.61221102)
\lineto(828.32671005,301.92721102)
\curveto(828.33670948,301.97720867)(828.34170948,302.02720862)(828.34171005,302.07721102)
\lineto(828.37171005,302.25721102)
\lineto(828.49171005,302.58721102)
\curveto(828.53170929,302.69720795)(828.58170924,302.79720785)(828.64171005,302.88721102)
\curveto(828.821709,303.17720747)(829.06670875,303.39220726)(829.37671005,303.53221102)
\curveto(829.68670813,303.67220698)(830.02670779,303.79720685)(830.39671005,303.90721102)
\curveto(830.53670728,303.9472067)(830.68170714,303.97720667)(830.83171005,303.99721102)
\curveto(830.98170684,304.01720663)(831.13170669,304.04220661)(831.28171005,304.07221102)
\curveto(831.35170647,304.09220656)(831.4167064,304.10220655)(831.47671005,304.10221102)
\curveto(831.54670627,304.10220655)(831.6217062,304.11220654)(831.70171005,304.13221102)
\curveto(831.77170605,304.1522065)(831.84170598,304.16220649)(831.91171005,304.16221102)
\curveto(831.98170584,304.17220648)(832.05670576,304.18720646)(832.13671005,304.20721102)
\curveto(832.38670543,304.26720638)(832.6217052,304.31720633)(832.84171005,304.35721102)
\curveto(833.06170476,304.40720624)(833.23670458,304.52220613)(833.36671005,304.70221102)
\curveto(833.42670439,304.78220587)(833.47670434,304.88220577)(833.51671005,305.00221102)
\curveto(833.55670426,305.13220552)(833.55670426,305.27220538)(833.51671005,305.42221102)
\curveto(833.45670436,305.66220499)(833.36670445,305.8522048)(833.24671005,305.99221102)
\curveto(833.13670468,306.13220452)(832.97670484,306.24220441)(832.76671005,306.32221102)
\curveto(832.64670517,306.37220428)(832.50170532,306.40720424)(832.33171005,306.42721102)
\curveto(832.17170565,306.4472042)(832.00170582,306.45720419)(831.82171005,306.45721102)
\curveto(831.64170618,306.45720419)(831.46670635,306.4472042)(831.29671005,306.42721102)
\curveto(831.12670669,306.40720424)(830.98170684,306.37720427)(830.86171005,306.33721102)
\curveto(830.69170713,306.27720437)(830.52670729,306.19220446)(830.36671005,306.08221102)
\curveto(830.28670753,306.02220463)(830.21170761,305.94220471)(830.14171005,305.84221102)
\curveto(830.08170774,305.7522049)(830.02670779,305.652205)(829.97671005,305.54221102)
\curveto(829.94670787,305.46220519)(829.9167079,305.37720527)(829.88671005,305.28721102)
\curveto(829.86670795,305.19720545)(829.821708,305.12720552)(829.75171005,305.07721102)
\curveto(829.71170811,305.0472056)(829.64170818,305.02220563)(829.54171005,305.00221102)
\curveto(829.45170837,304.99220566)(829.35670846,304.98720566)(829.25671005,304.98721102)
\curveto(829.15670866,304.98720566)(829.05670876,304.99220566)(828.95671005,305.00221102)
\curveto(828.86670895,305.02220563)(828.80170902,305.0472056)(828.76171005,305.07721102)
\curveto(828.7217091,305.10720554)(828.69170913,305.15720549)(828.67171005,305.22721102)
\curveto(828.65170917,305.29720535)(828.65170917,305.37220528)(828.67171005,305.45221102)
\curveto(828.70170912,305.58220507)(828.73170909,305.70220495)(828.76171005,305.81221102)
\curveto(828.80170902,305.93220472)(828.84670897,306.0472046)(828.89671005,306.15721102)
\curveto(829.08670873,306.50720414)(829.32670849,306.77720387)(829.61671005,306.96721102)
\curveto(829.90670791,307.16720348)(830.26670755,307.32720332)(830.69671005,307.44721102)
\curveto(830.79670702,307.46720318)(830.89670692,307.48220317)(830.99671005,307.49221102)
\curveto(831.10670671,307.50220315)(831.2167066,307.51720313)(831.32671005,307.53721102)
\curveto(831.36670645,307.5472031)(831.43170639,307.5472031)(831.52171005,307.53721102)
\curveto(831.61170621,307.53720311)(831.66670615,307.5472031)(831.68671005,307.56721102)
\curveto(832.38670543,307.57720307)(832.99670482,307.49720315)(833.51671005,307.32721102)
\curveto(834.03670378,307.15720349)(834.40170342,306.83220382)(834.61171005,306.35221102)
\curveto(834.70170312,306.1522045)(834.75170307,305.91720473)(834.76171005,305.64721102)
\curveto(834.78170304,305.38720526)(834.79170303,305.11220554)(834.79171005,304.82221102)
\lineto(834.79171005,301.50721102)
\curveto(834.79170303,301.36720928)(834.79670302,301.23220942)(834.80671005,301.10221102)
\curveto(834.816703,300.97220968)(834.84670297,300.86720978)(834.89671005,300.78721102)
\curveto(834.94670287,300.71720993)(835.01170281,300.66720998)(835.09171005,300.63721102)
\curveto(835.18170264,300.59721005)(835.26670255,300.56721008)(835.34671005,300.54721102)
\curveto(835.42670239,300.53721011)(835.48670233,300.49221016)(835.52671005,300.41221102)
\curveto(835.54670227,300.38221027)(835.55670226,300.3522103)(835.55671005,300.32221102)
\curveto(835.55670226,300.29221036)(835.56170226,300.2522104)(835.57171005,300.20221102)
\moveto(833.42671005,301.86721102)
\curveto(833.48670433,302.00720864)(833.5167043,302.16720848)(833.51671005,302.34721102)
\curveto(833.52670429,302.53720811)(833.53170429,302.73220792)(833.53171005,302.93221102)
\curveto(833.53170429,303.04220761)(833.52670429,303.14220751)(833.51671005,303.23221102)
\curveto(833.50670431,303.32220733)(833.46670435,303.39220726)(833.39671005,303.44221102)
\curveto(833.36670445,303.46220719)(833.29670452,303.47220718)(833.18671005,303.47221102)
\curveto(833.16670465,303.4522072)(833.13170469,303.44220721)(833.08171005,303.44221102)
\curveto(833.03170479,303.44220721)(832.98670483,303.43220722)(832.94671005,303.41221102)
\curveto(832.86670495,303.39220726)(832.77670504,303.37220728)(832.67671005,303.35221102)
\lineto(832.37671005,303.29221102)
\curveto(832.34670547,303.29220736)(832.31170551,303.28720736)(832.27171005,303.27721102)
\lineto(832.16671005,303.27721102)
\curveto(832.0167058,303.23720741)(831.85170597,303.21220744)(831.67171005,303.20221102)
\curveto(831.50170632,303.20220745)(831.34170648,303.18220747)(831.19171005,303.14221102)
\curveto(831.11170671,303.12220753)(831.03670678,303.10220755)(830.96671005,303.08221102)
\curveto(830.90670691,303.07220758)(830.83670698,303.05720759)(830.75671005,303.03721102)
\curveto(830.59670722,302.98720766)(830.44670737,302.92220773)(830.30671005,302.84221102)
\curveto(830.16670765,302.77220788)(830.04670777,302.68220797)(829.94671005,302.57221102)
\curveto(829.84670797,302.46220819)(829.77170805,302.32720832)(829.72171005,302.16721102)
\curveto(829.67170815,302.01720863)(829.65170817,301.83220882)(829.66171005,301.61221102)
\curveto(829.66170816,301.51220914)(829.67670814,301.41720923)(829.70671005,301.32721102)
\curveto(829.74670807,301.2472094)(829.79170803,301.17220948)(829.84171005,301.10221102)
\curveto(829.9217079,300.99220966)(830.02670779,300.89720975)(830.15671005,300.81721102)
\curveto(830.28670753,300.7472099)(830.42670739,300.68720996)(830.57671005,300.63721102)
\curveto(830.62670719,300.62721002)(830.67670714,300.62221003)(830.72671005,300.62221102)
\curveto(830.77670704,300.62221003)(830.82670699,300.61721003)(830.87671005,300.60721102)
\curveto(830.94670687,300.58721006)(831.03170679,300.57221008)(831.13171005,300.56221102)
\curveto(831.24170658,300.56221009)(831.33170649,300.57221008)(831.40171005,300.59221102)
\curveto(831.46170636,300.61221004)(831.5217063,300.61721003)(831.58171005,300.60721102)
\curveto(831.64170618,300.60721004)(831.70170612,300.61721003)(831.76171005,300.63721102)
\curveto(831.84170598,300.65720999)(831.9167059,300.67220998)(831.98671005,300.68221102)
\curveto(832.06670575,300.69220996)(832.14170568,300.71220994)(832.21171005,300.74221102)
\curveto(832.50170532,300.86220979)(832.74670507,301.00720964)(832.94671005,301.17721102)
\curveto(833.15670466,301.3472093)(833.3167045,301.57720907)(833.42671005,301.86721102)
}
}
{
\newrgbcolor{curcolor}{0 0 0}
\pscustom[linestyle=none,fillstyle=solid,fillcolor=curcolor]
{
\newpath
\moveto(840.38835067,307.55221102)
\curveto(840.61834588,307.5522031)(840.74834575,307.49220316)(840.77835067,307.37221102)
\curveto(840.80834569,307.26220339)(840.82334568,307.09720355)(840.82335067,306.87721102)
\lineto(840.82335067,306.59221102)
\curveto(840.82334568,306.50220415)(840.7983457,306.42720422)(840.74835067,306.36721102)
\curveto(840.68834581,306.28720436)(840.6033459,306.24220441)(840.49335067,306.23221102)
\curveto(840.38334612,306.23220442)(840.27334623,306.21720443)(840.16335067,306.18721102)
\curveto(840.02334648,306.15720449)(839.88834661,306.12720452)(839.75835067,306.09721102)
\curveto(839.63834686,306.06720458)(839.52334698,306.02720462)(839.41335067,305.97721102)
\curveto(839.12334738,305.8472048)(838.88834761,305.66720498)(838.70835067,305.43721102)
\curveto(838.52834797,305.21720543)(838.37334813,304.96220569)(838.24335067,304.67221102)
\curveto(838.2033483,304.56220609)(838.17334833,304.4472062)(838.15335067,304.32721102)
\curveto(838.13334837,304.21720643)(838.10834839,304.10220655)(838.07835067,303.98221102)
\curveto(838.06834843,303.93220672)(838.06334844,303.88220677)(838.06335067,303.83221102)
\curveto(838.07334843,303.78220687)(838.07334843,303.73220692)(838.06335067,303.68221102)
\curveto(838.03334847,303.56220709)(838.01834848,303.42220723)(838.01835067,303.26221102)
\curveto(838.02834847,303.11220754)(838.03334847,302.96720768)(838.03335067,302.82721102)
\lineto(838.03335067,300.98221102)
\lineto(838.03335067,300.63721102)
\curveto(838.03334847,300.51721013)(838.02834847,300.40221025)(838.01835067,300.29221102)
\curveto(838.00834849,300.18221047)(838.0033485,300.08721056)(838.00335067,300.00721102)
\curveto(838.01334849,299.92721072)(837.99334851,299.85721079)(837.94335067,299.79721102)
\curveto(837.89334861,299.72721092)(837.81334869,299.68721096)(837.70335067,299.67721102)
\curveto(837.6033489,299.66721098)(837.49334901,299.66221099)(837.37335067,299.66221102)
\lineto(837.10335067,299.66221102)
\curveto(837.05334945,299.68221097)(837.0033495,299.69721095)(836.95335067,299.70721102)
\curveto(836.91334959,299.72721092)(836.88334962,299.7522109)(836.86335067,299.78221102)
\curveto(836.81334969,299.8522108)(836.78334972,299.93721071)(836.77335067,300.03721102)
\lineto(836.77335067,300.36721102)
\lineto(836.77335067,301.52221102)
\lineto(836.77335067,305.67721102)
\lineto(836.77335067,306.71221102)
\lineto(836.77335067,307.01221102)
\curveto(836.78334972,307.11220354)(836.81334969,307.19720345)(836.86335067,307.26721102)
\curveto(836.89334961,307.30720334)(836.94334956,307.33720331)(837.01335067,307.35721102)
\curveto(837.09334941,307.37720327)(837.17834932,307.38720326)(837.26835067,307.38721102)
\curveto(837.35834914,307.39720325)(837.44834905,307.39720325)(837.53835067,307.38721102)
\curveto(837.62834887,307.37720327)(837.6983488,307.36220329)(837.74835067,307.34221102)
\curveto(837.82834867,307.31220334)(837.87834862,307.2522034)(837.89835067,307.16221102)
\curveto(837.92834857,307.08220357)(837.94334856,306.99220366)(837.94335067,306.89221102)
\lineto(837.94335067,306.59221102)
\curveto(837.94334856,306.49220416)(837.96334854,306.40220425)(838.00335067,306.32221102)
\curveto(838.01334849,306.30220435)(838.02334848,306.28720436)(838.03335067,306.27721102)
\lineto(838.07835067,306.23221102)
\curveto(838.18834831,306.23220442)(838.27834822,306.27720437)(838.34835067,306.36721102)
\curveto(838.41834808,306.46720418)(838.47834802,306.5472041)(838.52835067,306.60721102)
\lineto(838.61835067,306.69721102)
\curveto(838.70834779,306.80720384)(838.83334767,306.92220373)(838.99335067,307.04221102)
\curveto(839.15334735,307.16220349)(839.3033472,307.2522034)(839.44335067,307.31221102)
\curveto(839.53334697,307.36220329)(839.62834687,307.39720325)(839.72835067,307.41721102)
\curveto(839.82834667,307.4472032)(839.93334657,307.47720317)(840.04335067,307.50721102)
\curveto(840.1033464,307.51720313)(840.16334634,307.52220313)(840.22335067,307.52221102)
\curveto(840.28334622,307.53220312)(840.33834616,307.54220311)(840.38835067,307.55221102)
}
}
{
\newrgbcolor{curcolor}{0 0 0}
\pscustom[linestyle=none,fillstyle=solid,fillcolor=curcolor]
{
\newpath
\moveto(845.3981163,307.55221102)
\curveto(845.62811151,307.5522031)(845.75811138,307.49220316)(845.7881163,307.37221102)
\curveto(845.81811132,307.26220339)(845.8331113,307.09720355)(845.8331163,306.87721102)
\lineto(845.8331163,306.59221102)
\curveto(845.8331113,306.50220415)(845.80811133,306.42720422)(845.7581163,306.36721102)
\curveto(845.69811144,306.28720436)(845.61311152,306.24220441)(845.5031163,306.23221102)
\curveto(845.39311174,306.23220442)(845.28311185,306.21720443)(845.1731163,306.18721102)
\curveto(845.0331121,306.15720449)(844.89811224,306.12720452)(844.7681163,306.09721102)
\curveto(844.64811249,306.06720458)(844.5331126,306.02720462)(844.4231163,305.97721102)
\curveto(844.133113,305.8472048)(843.89811324,305.66720498)(843.7181163,305.43721102)
\curveto(843.5381136,305.21720543)(843.38311375,304.96220569)(843.2531163,304.67221102)
\curveto(843.21311392,304.56220609)(843.18311395,304.4472062)(843.1631163,304.32721102)
\curveto(843.14311399,304.21720643)(843.11811402,304.10220655)(843.0881163,303.98221102)
\curveto(843.07811406,303.93220672)(843.07311406,303.88220677)(843.0731163,303.83221102)
\curveto(843.08311405,303.78220687)(843.08311405,303.73220692)(843.0731163,303.68221102)
\curveto(843.04311409,303.56220709)(843.02811411,303.42220723)(843.0281163,303.26221102)
\curveto(843.0381141,303.11220754)(843.04311409,302.96720768)(843.0431163,302.82721102)
\lineto(843.0431163,300.98221102)
\lineto(843.0431163,300.63721102)
\curveto(843.04311409,300.51721013)(843.0381141,300.40221025)(843.0281163,300.29221102)
\curveto(843.01811412,300.18221047)(843.01311412,300.08721056)(843.0131163,300.00721102)
\curveto(843.02311411,299.92721072)(843.00311413,299.85721079)(842.9531163,299.79721102)
\curveto(842.90311423,299.72721092)(842.82311431,299.68721096)(842.7131163,299.67721102)
\curveto(842.61311452,299.66721098)(842.50311463,299.66221099)(842.3831163,299.66221102)
\lineto(842.1131163,299.66221102)
\curveto(842.06311507,299.68221097)(842.01311512,299.69721095)(841.9631163,299.70721102)
\curveto(841.92311521,299.72721092)(841.89311524,299.7522109)(841.8731163,299.78221102)
\curveto(841.82311531,299.8522108)(841.79311534,299.93721071)(841.7831163,300.03721102)
\lineto(841.7831163,300.36721102)
\lineto(841.7831163,301.52221102)
\lineto(841.7831163,305.67721102)
\lineto(841.7831163,306.71221102)
\lineto(841.7831163,307.01221102)
\curveto(841.79311534,307.11220354)(841.82311531,307.19720345)(841.8731163,307.26721102)
\curveto(841.90311523,307.30720334)(841.95311518,307.33720331)(842.0231163,307.35721102)
\curveto(842.10311503,307.37720327)(842.18811495,307.38720326)(842.2781163,307.38721102)
\curveto(842.36811477,307.39720325)(842.45811468,307.39720325)(842.5481163,307.38721102)
\curveto(842.6381145,307.37720327)(842.70811443,307.36220329)(842.7581163,307.34221102)
\curveto(842.8381143,307.31220334)(842.88811425,307.2522034)(842.9081163,307.16221102)
\curveto(842.9381142,307.08220357)(842.95311418,306.99220366)(842.9531163,306.89221102)
\lineto(842.9531163,306.59221102)
\curveto(842.95311418,306.49220416)(842.97311416,306.40220425)(843.0131163,306.32221102)
\curveto(843.02311411,306.30220435)(843.0331141,306.28720436)(843.0431163,306.27721102)
\lineto(843.0881163,306.23221102)
\curveto(843.19811394,306.23220442)(843.28811385,306.27720437)(843.3581163,306.36721102)
\curveto(843.42811371,306.46720418)(843.48811365,306.5472041)(843.5381163,306.60721102)
\lineto(843.6281163,306.69721102)
\curveto(843.71811342,306.80720384)(843.84311329,306.92220373)(844.0031163,307.04221102)
\curveto(844.16311297,307.16220349)(844.31311282,307.2522034)(844.4531163,307.31221102)
\curveto(844.54311259,307.36220329)(844.6381125,307.39720325)(844.7381163,307.41721102)
\curveto(844.8381123,307.4472032)(844.94311219,307.47720317)(845.0531163,307.50721102)
\curveto(845.11311202,307.51720313)(845.17311196,307.52220313)(845.2331163,307.52221102)
\curveto(845.29311184,307.53220312)(845.34811179,307.54220311)(845.3981163,307.55221102)
}
}
{
\newrgbcolor{curcolor}{0 0 0}
\pscustom[linestyle=none,fillstyle=solid,fillcolor=curcolor]
{
\newpath
\moveto(853.51288192,303.81721102)
\curveto(853.53287424,303.71720693)(853.53287424,303.60220705)(853.51288192,303.47221102)
\curveto(853.50287427,303.3522073)(853.4728743,303.26720738)(853.42288192,303.21721102)
\curveto(853.3728744,303.17720747)(853.29787447,303.1472075)(853.19788192,303.12721102)
\curveto(853.10787466,303.11720753)(853.00287477,303.11220754)(852.88288192,303.11221102)
\lineto(852.52288192,303.11221102)
\curveto(852.40287537,303.12220753)(852.29787547,303.12720752)(852.20788192,303.12721102)
\lineto(848.36788192,303.12721102)
\curveto(848.28787948,303.12720752)(848.20787956,303.12220753)(848.12788192,303.11221102)
\curveto(848.04787972,303.11220754)(847.98287979,303.09720755)(847.93288192,303.06721102)
\curveto(847.89287988,303.0472076)(847.85287992,303.00720764)(847.81288192,302.94721102)
\curveto(847.79287998,302.91720773)(847.77288,302.87220778)(847.75288192,302.81221102)
\curveto(847.73288004,302.76220789)(847.73288004,302.71220794)(847.75288192,302.66221102)
\curveto(847.76288001,302.61220804)(847.76788,302.56720808)(847.76788192,302.52721102)
\curveto(847.76788,302.48720816)(847.77288,302.4472082)(847.78288192,302.40721102)
\curveto(847.80287997,302.32720832)(847.82287995,302.24220841)(847.84288192,302.15221102)
\curveto(847.86287991,302.07220858)(847.89287988,301.99220866)(847.93288192,301.91221102)
\curveto(848.16287961,301.37220928)(848.54287923,300.98720966)(849.07288192,300.75721102)
\curveto(849.13287864,300.72720992)(849.19787857,300.70220995)(849.26788192,300.68221102)
\lineto(849.47788192,300.62221102)
\curveto(849.50787826,300.61221004)(849.55787821,300.60721004)(849.62788192,300.60721102)
\curveto(849.767878,300.56721008)(849.95287782,300.5472101)(850.18288192,300.54721102)
\curveto(850.41287736,300.5472101)(850.59787717,300.56721008)(850.73788192,300.60721102)
\curveto(850.87787689,300.64721)(851.00287677,300.68720996)(851.11288192,300.72721102)
\curveto(851.23287654,300.77720987)(851.34287643,300.83720981)(851.44288192,300.90721102)
\curveto(851.55287622,300.97720967)(851.64787612,301.05720959)(851.72788192,301.14721102)
\curveto(851.80787596,301.2472094)(851.87787589,301.3522093)(851.93788192,301.46221102)
\curveto(851.99787577,301.56220909)(852.04787572,301.66720898)(852.08788192,301.77721102)
\curveto(852.13787563,301.88720876)(852.21787555,301.96720868)(852.32788192,302.01721102)
\curveto(852.3678754,302.03720861)(852.43287534,302.0522086)(852.52288192,302.06221102)
\curveto(852.61287516,302.07220858)(852.70287507,302.07220858)(852.79288192,302.06221102)
\curveto(852.88287489,302.06220859)(852.9678748,302.05720859)(853.04788192,302.04721102)
\curveto(853.12787464,302.03720861)(853.18287459,302.01720863)(853.21288192,301.98721102)
\curveto(853.31287446,301.91720873)(853.33787443,301.80220885)(853.28788192,301.64221102)
\curveto(853.20787456,301.37220928)(853.10287467,301.13220952)(852.97288192,300.92221102)
\curveto(852.772875,300.60221005)(852.54287523,300.33721031)(852.28288192,300.12721102)
\curveto(852.03287574,299.92721072)(851.71287606,299.76221089)(851.32288192,299.63221102)
\curveto(851.22287655,299.59221106)(851.12287665,299.56721108)(851.02288192,299.55721102)
\curveto(850.92287685,299.53721111)(850.81787695,299.51721113)(850.70788192,299.49721102)
\curveto(850.65787711,299.48721116)(850.60787716,299.48221117)(850.55788192,299.48221102)
\curveto(850.51787725,299.48221117)(850.4728773,299.47721117)(850.42288192,299.46721102)
\lineto(850.27288192,299.46721102)
\curveto(850.22287755,299.45721119)(850.16287761,299.4522112)(850.09288192,299.45221102)
\curveto(850.03287774,299.4522112)(849.98287779,299.45721119)(849.94288192,299.46721102)
\lineto(849.80788192,299.46721102)
\curveto(849.75787801,299.47721117)(849.71287806,299.48221117)(849.67288192,299.48221102)
\curveto(849.63287814,299.48221117)(849.59287818,299.48721116)(849.55288192,299.49721102)
\curveto(849.50287827,299.50721114)(849.44787832,299.51721113)(849.38788192,299.52721102)
\curveto(849.32787844,299.52721112)(849.2728785,299.53221112)(849.22288192,299.54221102)
\curveto(849.13287864,299.56221109)(849.04287873,299.58721106)(848.95288192,299.61721102)
\curveto(848.86287891,299.63721101)(848.77787899,299.66221099)(848.69788192,299.69221102)
\curveto(848.65787911,299.71221094)(848.62287915,299.72221093)(848.59288192,299.72221102)
\curveto(848.56287921,299.73221092)(848.52787924,299.7472109)(848.48788192,299.76721102)
\curveto(848.33787943,299.83721081)(848.17787959,299.92221073)(848.00788192,300.02221102)
\curveto(847.71788005,300.21221044)(847.4678803,300.44221021)(847.25788192,300.71221102)
\curveto(847.05788071,300.99220966)(846.88788088,301.30220935)(846.74788192,301.64221102)
\curveto(846.69788107,301.7522089)(846.65788111,301.86720878)(846.62788192,301.98721102)
\curveto(846.60788116,302.10720854)(846.57788119,302.22720842)(846.53788192,302.34721102)
\curveto(846.52788124,302.38720826)(846.52288125,302.42220823)(846.52288192,302.45221102)
\curveto(846.52288125,302.48220817)(846.51788125,302.52220813)(846.50788192,302.57221102)
\curveto(846.48788128,302.652208)(846.4728813,302.73720791)(846.46288192,302.82721102)
\curveto(846.45288132,302.91720773)(846.43788133,303.00720764)(846.41788192,303.09721102)
\lineto(846.41788192,303.30721102)
\curveto(846.40788136,303.3472073)(846.39788137,303.40220725)(846.38788192,303.47221102)
\curveto(846.38788138,303.5522071)(846.39288138,303.61720703)(846.40288192,303.66721102)
\lineto(846.40288192,303.83221102)
\curveto(846.42288135,303.88220677)(846.42788134,303.93220672)(846.41788192,303.98221102)
\curveto(846.41788135,304.04220661)(846.42288135,304.09720655)(846.43288192,304.14721102)
\curveto(846.4728813,304.30720634)(846.50288127,304.46720618)(846.52288192,304.62721102)
\curveto(846.55288122,304.78720586)(846.59788117,304.93720571)(846.65788192,305.07721102)
\curveto(846.70788106,305.18720546)(846.75288102,305.29720535)(846.79288192,305.40721102)
\curveto(846.84288093,305.52720512)(846.89788087,305.64220501)(846.95788192,305.75221102)
\curveto(847.17788059,306.10220455)(847.42788034,306.40220425)(847.70788192,306.65221102)
\curveto(847.98787978,306.91220374)(848.33287944,307.12720352)(848.74288192,307.29721102)
\curveto(848.86287891,307.3472033)(848.98287879,307.38220327)(849.10288192,307.40221102)
\curveto(849.23287854,307.43220322)(849.3678784,307.46220319)(849.50788192,307.49221102)
\curveto(849.55787821,307.50220315)(849.60287817,307.50720314)(849.64288192,307.50721102)
\curveto(849.68287809,307.51720313)(849.72787804,307.52220313)(849.77788192,307.52221102)
\curveto(849.79787797,307.53220312)(849.82287795,307.53220312)(849.85288192,307.52221102)
\curveto(849.88287789,307.51220314)(849.90787786,307.51720313)(849.92788192,307.53721102)
\curveto(850.34787742,307.5472031)(850.71287706,307.50220315)(851.02288192,307.40221102)
\curveto(851.33287644,307.31220334)(851.61287616,307.18720346)(851.86288192,307.02721102)
\curveto(851.91287586,307.00720364)(851.95287582,306.97720367)(851.98288192,306.93721102)
\curveto(852.01287576,306.90720374)(852.04787572,306.88220377)(852.08788192,306.86221102)
\curveto(852.1678756,306.80220385)(852.24787552,306.73220392)(852.32788192,306.65221102)
\curveto(852.41787535,306.57220408)(852.49287528,306.49220416)(852.55288192,306.41221102)
\curveto(852.71287506,306.20220445)(852.84787492,306.00220465)(852.95788192,305.81221102)
\curveto(853.02787474,305.70220495)(853.08287469,305.58220507)(853.12288192,305.45221102)
\curveto(853.16287461,305.32220533)(853.20787456,305.19220546)(853.25788192,305.06221102)
\curveto(853.30787446,304.93220572)(853.34287443,304.79720585)(853.36288192,304.65721102)
\curveto(853.39287438,304.51720613)(853.42787434,304.37720627)(853.46788192,304.23721102)
\curveto(853.47787429,304.16720648)(853.48287429,304.09720655)(853.48288192,304.02721102)
\lineto(853.51288192,303.81721102)
\moveto(852.05788192,304.32721102)
\curveto(852.08787568,304.36720628)(852.11287566,304.41720623)(852.13288192,304.47721102)
\curveto(852.15287562,304.5472061)(852.15287562,304.61720603)(852.13288192,304.68721102)
\curveto(852.0728757,304.90720574)(851.98787578,305.11220554)(851.87788192,305.30221102)
\curveto(851.73787603,305.53220512)(851.58287619,305.72720492)(851.41288192,305.88721102)
\curveto(851.24287653,306.0472046)(851.02287675,306.18220447)(850.75288192,306.29221102)
\curveto(850.68287709,306.31220434)(850.61287716,306.32720432)(850.54288192,306.33721102)
\curveto(850.4728773,306.35720429)(850.39787737,306.37720427)(850.31788192,306.39721102)
\curveto(850.23787753,306.41720423)(850.15287762,306.42720422)(850.06288192,306.42721102)
\lineto(849.80788192,306.42721102)
\curveto(849.77787799,306.40720424)(849.74287803,306.39720425)(849.70288192,306.39721102)
\curveto(849.66287811,306.40720424)(849.62787814,306.40720424)(849.59788192,306.39721102)
\lineto(849.35788192,306.33721102)
\curveto(849.28787848,306.32720432)(849.21787855,306.31220434)(849.14788192,306.29221102)
\curveto(848.85787891,306.17220448)(848.62287915,306.02220463)(848.44288192,305.84221102)
\curveto(848.2728795,305.66220499)(848.11787965,305.43720521)(847.97788192,305.16721102)
\curveto(847.94787982,305.11720553)(847.91787985,305.0522056)(847.88788192,304.97221102)
\curveto(847.85787991,304.90220575)(847.83287994,304.82220583)(847.81288192,304.73221102)
\curveto(847.79287998,304.64220601)(847.78787998,304.55720609)(847.79788192,304.47721102)
\curveto(847.80787996,304.39720625)(847.84287993,304.33720631)(847.90288192,304.29721102)
\curveto(847.98287979,304.23720641)(848.11787965,304.20720644)(848.30788192,304.20721102)
\curveto(848.50787926,304.21720643)(848.67787909,304.22220643)(848.81788192,304.22221102)
\lineto(851.09788192,304.22221102)
\curveto(851.24787652,304.22220643)(851.42787634,304.21720643)(851.63788192,304.20721102)
\curveto(851.84787592,304.20720644)(851.98787578,304.2472064)(852.05788192,304.32721102)
}
}
{
\newrgbcolor{curcolor}{0 0 0}
\pscustom[linestyle=none,fillstyle=solid,fillcolor=curcolor]
{
\newpath
\moveto(858.46452255,307.55221102)
\curveto(858.69451776,307.5522031)(858.82451763,307.49220316)(858.85452255,307.37221102)
\curveto(858.88451757,307.26220339)(858.89951755,307.09720355)(858.89952255,306.87721102)
\lineto(858.89952255,306.59221102)
\curveto(858.89951755,306.50220415)(858.87451758,306.42720422)(858.82452255,306.36721102)
\curveto(858.76451769,306.28720436)(858.67951777,306.24220441)(858.56952255,306.23221102)
\curveto(858.45951799,306.23220442)(858.3495181,306.21720443)(858.23952255,306.18721102)
\curveto(858.09951835,306.15720449)(857.96451849,306.12720452)(857.83452255,306.09721102)
\curveto(857.71451874,306.06720458)(857.59951885,306.02720462)(857.48952255,305.97721102)
\curveto(857.19951925,305.8472048)(856.96451949,305.66720498)(856.78452255,305.43721102)
\curveto(856.60451985,305.21720543)(856.44952,304.96220569)(856.31952255,304.67221102)
\curveto(856.27952017,304.56220609)(856.2495202,304.4472062)(856.22952255,304.32721102)
\curveto(856.20952024,304.21720643)(856.18452027,304.10220655)(856.15452255,303.98221102)
\curveto(856.14452031,303.93220672)(856.13952031,303.88220677)(856.13952255,303.83221102)
\curveto(856.1495203,303.78220687)(856.1495203,303.73220692)(856.13952255,303.68221102)
\curveto(856.10952034,303.56220709)(856.09452036,303.42220723)(856.09452255,303.26221102)
\curveto(856.10452035,303.11220754)(856.10952034,302.96720768)(856.10952255,302.82721102)
\lineto(856.10952255,300.98221102)
\lineto(856.10952255,300.63721102)
\curveto(856.10952034,300.51721013)(856.10452035,300.40221025)(856.09452255,300.29221102)
\curveto(856.08452037,300.18221047)(856.07952037,300.08721056)(856.07952255,300.00721102)
\curveto(856.08952036,299.92721072)(856.06952038,299.85721079)(856.01952255,299.79721102)
\curveto(855.96952048,299.72721092)(855.88952056,299.68721096)(855.77952255,299.67721102)
\curveto(855.67952077,299.66721098)(855.56952088,299.66221099)(855.44952255,299.66221102)
\lineto(855.17952255,299.66221102)
\curveto(855.12952132,299.68221097)(855.07952137,299.69721095)(855.02952255,299.70721102)
\curveto(854.98952146,299.72721092)(854.95952149,299.7522109)(854.93952255,299.78221102)
\curveto(854.88952156,299.8522108)(854.85952159,299.93721071)(854.84952255,300.03721102)
\lineto(854.84952255,300.36721102)
\lineto(854.84952255,301.52221102)
\lineto(854.84952255,305.67721102)
\lineto(854.84952255,306.71221102)
\lineto(854.84952255,307.01221102)
\curveto(854.85952159,307.11220354)(854.88952156,307.19720345)(854.93952255,307.26721102)
\curveto(854.96952148,307.30720334)(855.01952143,307.33720331)(855.08952255,307.35721102)
\curveto(855.16952128,307.37720327)(855.2545212,307.38720326)(855.34452255,307.38721102)
\curveto(855.43452102,307.39720325)(855.52452093,307.39720325)(855.61452255,307.38721102)
\curveto(855.70452075,307.37720327)(855.77452068,307.36220329)(855.82452255,307.34221102)
\curveto(855.90452055,307.31220334)(855.9545205,307.2522034)(855.97452255,307.16221102)
\curveto(856.00452045,307.08220357)(856.01952043,306.99220366)(856.01952255,306.89221102)
\lineto(856.01952255,306.59221102)
\curveto(856.01952043,306.49220416)(856.03952041,306.40220425)(856.07952255,306.32221102)
\curveto(856.08952036,306.30220435)(856.09952035,306.28720436)(856.10952255,306.27721102)
\lineto(856.15452255,306.23221102)
\curveto(856.26452019,306.23220442)(856.3545201,306.27720437)(856.42452255,306.36721102)
\curveto(856.49451996,306.46720418)(856.5545199,306.5472041)(856.60452255,306.60721102)
\lineto(856.69452255,306.69721102)
\curveto(856.78451967,306.80720384)(856.90951954,306.92220373)(857.06952255,307.04221102)
\curveto(857.22951922,307.16220349)(857.37951907,307.2522034)(857.51952255,307.31221102)
\curveto(857.60951884,307.36220329)(857.70451875,307.39720325)(857.80452255,307.41721102)
\curveto(857.90451855,307.4472032)(858.00951844,307.47720317)(858.11952255,307.50721102)
\curveto(858.17951827,307.51720313)(858.23951821,307.52220313)(858.29952255,307.52221102)
\curveto(858.35951809,307.53220312)(858.41451804,307.54220311)(858.46452255,307.55221102)
}
}
{
\newrgbcolor{curcolor}{0 0 0}
\pscustom[linestyle=none,fillstyle=solid,fillcolor=curcolor]
{
\newpath
\moveto(866.71428817,300.20221102)
\curveto(866.74428034,300.04221061)(866.72928036,299.90721074)(866.66928817,299.79721102)
\curveto(866.60928048,299.69721095)(866.52928056,299.62221103)(866.42928817,299.57221102)
\curveto(866.37928071,299.5522111)(866.32428076,299.54221111)(866.26428817,299.54221102)
\curveto(866.21428087,299.54221111)(866.15928093,299.53221112)(866.09928817,299.51221102)
\curveto(865.87928121,299.46221119)(865.65928143,299.47721117)(865.43928817,299.55721102)
\curveto(865.22928186,299.62721102)(865.084282,299.71721093)(865.00428817,299.82721102)
\curveto(864.95428213,299.89721075)(864.90928218,299.97721067)(864.86928817,300.06721102)
\curveto(864.82928226,300.16721048)(864.77928231,300.2472104)(864.71928817,300.30721102)
\curveto(864.69928239,300.32721032)(864.67428241,300.3472103)(864.64428817,300.36721102)
\curveto(864.62428246,300.38721026)(864.59428249,300.39221026)(864.55428817,300.38221102)
\curveto(864.44428264,300.3522103)(864.33928275,300.29721035)(864.23928817,300.21721102)
\curveto(864.14928294,300.13721051)(864.05928303,300.06721058)(863.96928817,300.00721102)
\curveto(863.83928325,299.92721072)(863.69928339,299.8522108)(863.54928817,299.78221102)
\curveto(863.39928369,299.72221093)(863.23928385,299.66721098)(863.06928817,299.61721102)
\curveto(862.96928412,299.58721106)(862.85928423,299.56721108)(862.73928817,299.55721102)
\curveto(862.62928446,299.5472111)(862.51928457,299.53221112)(862.40928817,299.51221102)
\curveto(862.35928473,299.50221115)(862.31428477,299.49721115)(862.27428817,299.49721102)
\lineto(862.16928817,299.49721102)
\curveto(862.05928503,299.47721117)(861.95428513,299.47721117)(861.85428817,299.49721102)
\lineto(861.71928817,299.49721102)
\curveto(861.66928542,299.50721114)(861.61928547,299.51221114)(861.56928817,299.51221102)
\curveto(861.51928557,299.51221114)(861.47428561,299.52221113)(861.43428817,299.54221102)
\curveto(861.39428569,299.5522111)(861.35928573,299.55721109)(861.32928817,299.55721102)
\curveto(861.30928578,299.5472111)(861.2842858,299.5472111)(861.25428817,299.55721102)
\lineto(861.01428817,299.61721102)
\curveto(860.93428615,299.62721102)(860.85928623,299.647211)(860.78928817,299.67721102)
\curveto(860.4892866,299.80721084)(860.24428684,299.9522107)(860.05428817,300.11221102)
\curveto(859.87428721,300.28221037)(859.72428736,300.51721013)(859.60428817,300.81721102)
\curveto(859.51428757,301.03720961)(859.46928762,301.30220935)(859.46928817,301.61221102)
\lineto(859.46928817,301.92721102)
\curveto(859.47928761,301.97720867)(859.4842876,302.02720862)(859.48428817,302.07721102)
\lineto(859.51428817,302.25721102)
\lineto(859.63428817,302.58721102)
\curveto(859.67428741,302.69720795)(859.72428736,302.79720785)(859.78428817,302.88721102)
\curveto(859.96428712,303.17720747)(860.20928688,303.39220726)(860.51928817,303.53221102)
\curveto(860.82928626,303.67220698)(861.16928592,303.79720685)(861.53928817,303.90721102)
\curveto(861.67928541,303.9472067)(861.82428526,303.97720667)(861.97428817,303.99721102)
\curveto(862.12428496,304.01720663)(862.27428481,304.04220661)(862.42428817,304.07221102)
\curveto(862.49428459,304.09220656)(862.55928453,304.10220655)(862.61928817,304.10221102)
\curveto(862.6892844,304.10220655)(862.76428432,304.11220654)(862.84428817,304.13221102)
\curveto(862.91428417,304.1522065)(862.9842841,304.16220649)(863.05428817,304.16221102)
\curveto(863.12428396,304.17220648)(863.19928389,304.18720646)(863.27928817,304.20721102)
\curveto(863.52928356,304.26720638)(863.76428332,304.31720633)(863.98428817,304.35721102)
\curveto(864.20428288,304.40720624)(864.37928271,304.52220613)(864.50928817,304.70221102)
\curveto(864.56928252,304.78220587)(864.61928247,304.88220577)(864.65928817,305.00221102)
\curveto(864.69928239,305.13220552)(864.69928239,305.27220538)(864.65928817,305.42221102)
\curveto(864.59928249,305.66220499)(864.50928258,305.8522048)(864.38928817,305.99221102)
\curveto(864.27928281,306.13220452)(864.11928297,306.24220441)(863.90928817,306.32221102)
\curveto(863.7892833,306.37220428)(863.64428344,306.40720424)(863.47428817,306.42721102)
\curveto(863.31428377,306.4472042)(863.14428394,306.45720419)(862.96428817,306.45721102)
\curveto(862.7842843,306.45720419)(862.60928448,306.4472042)(862.43928817,306.42721102)
\curveto(862.26928482,306.40720424)(862.12428496,306.37720427)(862.00428817,306.33721102)
\curveto(861.83428525,306.27720437)(861.66928542,306.19220446)(861.50928817,306.08221102)
\curveto(861.42928566,306.02220463)(861.35428573,305.94220471)(861.28428817,305.84221102)
\curveto(861.22428586,305.7522049)(861.16928592,305.652205)(861.11928817,305.54221102)
\curveto(861.089286,305.46220519)(861.05928603,305.37720527)(861.02928817,305.28721102)
\curveto(861.00928608,305.19720545)(860.96428612,305.12720552)(860.89428817,305.07721102)
\curveto(860.85428623,305.0472056)(860.7842863,305.02220563)(860.68428817,305.00221102)
\curveto(860.59428649,304.99220566)(860.49928659,304.98720566)(860.39928817,304.98721102)
\curveto(860.29928679,304.98720566)(860.19928689,304.99220566)(860.09928817,305.00221102)
\curveto(860.00928708,305.02220563)(859.94428714,305.0472056)(859.90428817,305.07721102)
\curveto(859.86428722,305.10720554)(859.83428725,305.15720549)(859.81428817,305.22721102)
\curveto(859.79428729,305.29720535)(859.79428729,305.37220528)(859.81428817,305.45221102)
\curveto(859.84428724,305.58220507)(859.87428721,305.70220495)(859.90428817,305.81221102)
\curveto(859.94428714,305.93220472)(859.9892871,306.0472046)(860.03928817,306.15721102)
\curveto(860.22928686,306.50720414)(860.46928662,306.77720387)(860.75928817,306.96721102)
\curveto(861.04928604,307.16720348)(861.40928568,307.32720332)(861.83928817,307.44721102)
\curveto(861.93928515,307.46720318)(862.03928505,307.48220317)(862.13928817,307.49221102)
\curveto(862.24928484,307.50220315)(862.35928473,307.51720313)(862.46928817,307.53721102)
\curveto(862.50928458,307.5472031)(862.57428451,307.5472031)(862.66428817,307.53721102)
\curveto(862.75428433,307.53720311)(862.80928428,307.5472031)(862.82928817,307.56721102)
\curveto(863.52928356,307.57720307)(864.13928295,307.49720315)(864.65928817,307.32721102)
\curveto(865.17928191,307.15720349)(865.54428154,306.83220382)(865.75428817,306.35221102)
\curveto(865.84428124,306.1522045)(865.89428119,305.91720473)(865.90428817,305.64721102)
\curveto(865.92428116,305.38720526)(865.93428115,305.11220554)(865.93428817,304.82221102)
\lineto(865.93428817,301.50721102)
\curveto(865.93428115,301.36720928)(865.93928115,301.23220942)(865.94928817,301.10221102)
\curveto(865.95928113,300.97220968)(865.9892811,300.86720978)(866.03928817,300.78721102)
\curveto(866.089281,300.71720993)(866.15428093,300.66720998)(866.23428817,300.63721102)
\curveto(866.32428076,300.59721005)(866.40928068,300.56721008)(866.48928817,300.54721102)
\curveto(866.56928052,300.53721011)(866.62928046,300.49221016)(866.66928817,300.41221102)
\curveto(866.6892804,300.38221027)(866.69928039,300.3522103)(866.69928817,300.32221102)
\curveto(866.69928039,300.29221036)(866.70428038,300.2522104)(866.71428817,300.20221102)
\moveto(864.56928817,301.86721102)
\curveto(864.62928246,302.00720864)(864.65928243,302.16720848)(864.65928817,302.34721102)
\curveto(864.66928242,302.53720811)(864.67428241,302.73220792)(864.67428817,302.93221102)
\curveto(864.67428241,303.04220761)(864.66928242,303.14220751)(864.65928817,303.23221102)
\curveto(864.64928244,303.32220733)(864.60928248,303.39220726)(864.53928817,303.44221102)
\curveto(864.50928258,303.46220719)(864.43928265,303.47220718)(864.32928817,303.47221102)
\curveto(864.30928278,303.4522072)(864.27428281,303.44220721)(864.22428817,303.44221102)
\curveto(864.17428291,303.44220721)(864.12928296,303.43220722)(864.08928817,303.41221102)
\curveto(864.00928308,303.39220726)(863.91928317,303.37220728)(863.81928817,303.35221102)
\lineto(863.51928817,303.29221102)
\curveto(863.4892836,303.29220736)(863.45428363,303.28720736)(863.41428817,303.27721102)
\lineto(863.30928817,303.27721102)
\curveto(863.15928393,303.23720741)(862.99428409,303.21220744)(862.81428817,303.20221102)
\curveto(862.64428444,303.20220745)(862.4842846,303.18220747)(862.33428817,303.14221102)
\curveto(862.25428483,303.12220753)(862.17928491,303.10220755)(862.10928817,303.08221102)
\curveto(862.04928504,303.07220758)(861.97928511,303.05720759)(861.89928817,303.03721102)
\curveto(861.73928535,302.98720766)(861.5892855,302.92220773)(861.44928817,302.84221102)
\curveto(861.30928578,302.77220788)(861.1892859,302.68220797)(861.08928817,302.57221102)
\curveto(860.9892861,302.46220819)(860.91428617,302.32720832)(860.86428817,302.16721102)
\curveto(860.81428627,302.01720863)(860.79428629,301.83220882)(860.80428817,301.61221102)
\curveto(860.80428628,301.51220914)(860.81928627,301.41720923)(860.84928817,301.32721102)
\curveto(860.8892862,301.2472094)(860.93428615,301.17220948)(860.98428817,301.10221102)
\curveto(861.06428602,300.99220966)(861.16928592,300.89720975)(861.29928817,300.81721102)
\curveto(861.42928566,300.7472099)(861.56928552,300.68720996)(861.71928817,300.63721102)
\curveto(861.76928532,300.62721002)(861.81928527,300.62221003)(861.86928817,300.62221102)
\curveto(861.91928517,300.62221003)(861.96928512,300.61721003)(862.01928817,300.60721102)
\curveto(862.089285,300.58721006)(862.17428491,300.57221008)(862.27428817,300.56221102)
\curveto(862.3842847,300.56221009)(862.47428461,300.57221008)(862.54428817,300.59221102)
\curveto(862.60428448,300.61221004)(862.66428442,300.61721003)(862.72428817,300.60721102)
\curveto(862.7842843,300.60721004)(862.84428424,300.61721003)(862.90428817,300.63721102)
\curveto(862.9842841,300.65720999)(863.05928403,300.67220998)(863.12928817,300.68221102)
\curveto(863.20928388,300.69220996)(863.2842838,300.71220994)(863.35428817,300.74221102)
\curveto(863.64428344,300.86220979)(863.8892832,301.00720964)(864.08928817,301.17721102)
\curveto(864.29928279,301.3472093)(864.45928263,301.57720907)(864.56928817,301.86721102)
}
}
{
\newrgbcolor{curcolor}{0 0 0}
\pscustom[linestyle=none,fillstyle=solid,fillcolor=curcolor]
{
\newpath
\moveto(870.3159288,307.55221102)
\curveto(871.03592473,307.56220309)(871.64092413,307.47720317)(872.1309288,307.29721102)
\curveto(872.62092315,307.12720352)(873.00092277,306.82220383)(873.2709288,306.38221102)
\curveto(873.34092243,306.27220438)(873.39592237,306.15720449)(873.4359288,306.03721102)
\curveto(873.47592229,305.92720472)(873.51592225,305.80220485)(873.5559288,305.66221102)
\curveto(873.57592219,305.59220506)(873.58092219,305.51720513)(873.5709288,305.43721102)
\curveto(873.56092221,305.36720528)(873.54592222,305.31220534)(873.5259288,305.27221102)
\curveto(873.50592226,305.2522054)(873.48092229,305.23220542)(873.4509288,305.21221102)
\curveto(873.42092235,305.20220545)(873.39592237,305.18720546)(873.3759288,305.16721102)
\curveto(873.32592244,305.1472055)(873.27592249,305.14220551)(873.2259288,305.15221102)
\curveto(873.17592259,305.16220549)(873.12592264,305.16220549)(873.0759288,305.15221102)
\curveto(872.99592277,305.13220552)(872.89092288,305.12720552)(872.7609288,305.13721102)
\curveto(872.63092314,305.15720549)(872.54092323,305.18220547)(872.4909288,305.21221102)
\curveto(872.41092336,305.26220539)(872.35592341,305.32720532)(872.3259288,305.40721102)
\curveto(872.30592346,305.49720515)(872.2709235,305.58220507)(872.2209288,305.66221102)
\curveto(872.13092364,305.82220483)(872.00592376,305.96720468)(871.8459288,306.09721102)
\curveto(871.73592403,306.17720447)(871.61592415,306.23720441)(871.4859288,306.27721102)
\curveto(871.35592441,306.31720433)(871.21592455,306.35720429)(871.0659288,306.39721102)
\curveto(871.01592475,306.41720423)(870.9659248,306.42220423)(870.9159288,306.41221102)
\curveto(870.8659249,306.41220424)(870.81592495,306.41720423)(870.7659288,306.42721102)
\curveto(870.70592506,306.4472042)(870.63092514,306.45720419)(870.5409288,306.45721102)
\curveto(870.45092532,306.45720419)(870.37592539,306.4472042)(870.3159288,306.42721102)
\lineto(870.2259288,306.42721102)
\lineto(870.0759288,306.39721102)
\curveto(870.02592574,306.39720425)(869.97592579,306.39220426)(869.9259288,306.38221102)
\curveto(869.6659261,306.32220433)(869.45092632,306.23720441)(869.2809288,306.12721102)
\curveto(869.11092666,306.01720463)(868.99592677,305.83220482)(868.9359288,305.57221102)
\curveto(868.91592685,305.50220515)(868.91092686,305.43220522)(868.9209288,305.36221102)
\curveto(868.94092683,305.29220536)(868.96092681,305.23220542)(868.9809288,305.18221102)
\curveto(869.04092673,305.03220562)(869.11092666,304.92220573)(869.1909288,304.85221102)
\curveto(869.28092649,304.79220586)(869.39092638,304.72220593)(869.5209288,304.64221102)
\curveto(869.68092609,304.54220611)(869.86092591,304.46720618)(870.0609288,304.41721102)
\curveto(870.26092551,304.37720627)(870.46092531,304.32720632)(870.6609288,304.26721102)
\curveto(870.79092498,304.22720642)(870.92092485,304.19720645)(871.0509288,304.17721102)
\curveto(871.18092459,304.15720649)(871.31092446,304.12720652)(871.4409288,304.08721102)
\curveto(871.65092412,304.02720662)(871.85592391,303.96720668)(872.0559288,303.90721102)
\curveto(872.25592351,303.85720679)(872.45592331,303.79220686)(872.6559288,303.71221102)
\lineto(872.8059288,303.65221102)
\curveto(872.85592291,303.63220702)(872.90592286,303.60720704)(872.9559288,303.57721102)
\curveto(873.15592261,303.45720719)(873.33092244,303.32220733)(873.4809288,303.17221102)
\curveto(873.63092214,303.02220763)(873.75592201,302.83220782)(873.8559288,302.60221102)
\curveto(873.87592189,302.53220812)(873.89592187,302.43720821)(873.9159288,302.31721102)
\curveto(873.93592183,302.2472084)(873.94592182,302.17220848)(873.9459288,302.09221102)
\curveto(873.95592181,302.02220863)(873.96092181,301.94220871)(873.9609288,301.85221102)
\lineto(873.9609288,301.70221102)
\curveto(873.94092183,301.63220902)(873.93092184,301.56220909)(873.9309288,301.49221102)
\curveto(873.93092184,301.42220923)(873.92092185,301.3522093)(873.9009288,301.28221102)
\curveto(873.8709219,301.17220948)(873.83592193,301.06720958)(873.7959288,300.96721102)
\curveto(873.75592201,300.86720978)(873.71092206,300.77720987)(873.6609288,300.69721102)
\curveto(873.50092227,300.43721021)(873.29592247,300.22721042)(873.0459288,300.06721102)
\curveto(872.79592297,299.91721073)(872.51592325,299.78721086)(872.2059288,299.67721102)
\curveto(872.11592365,299.647211)(872.02092375,299.62721102)(871.9209288,299.61721102)
\curveto(871.83092394,299.59721105)(871.74092403,299.57221108)(871.6509288,299.54221102)
\curveto(871.55092422,299.52221113)(871.45092432,299.51221114)(871.3509288,299.51221102)
\curveto(871.25092452,299.51221114)(871.15092462,299.50221115)(871.0509288,299.48221102)
\lineto(870.9009288,299.48221102)
\curveto(870.85092492,299.47221118)(870.78092499,299.46721118)(870.6909288,299.46721102)
\curveto(870.60092517,299.46721118)(870.53092524,299.47221118)(870.4809288,299.48221102)
\lineto(870.3159288,299.48221102)
\curveto(870.25592551,299.50221115)(870.19092558,299.51221114)(870.1209288,299.51221102)
\curveto(870.05092572,299.50221115)(869.99092578,299.50721114)(869.9409288,299.52721102)
\curveto(869.89092588,299.53721111)(869.82592594,299.54221111)(869.7459288,299.54221102)
\lineto(869.5059288,299.60221102)
\curveto(869.43592633,299.61221104)(869.36092641,299.63221102)(869.2809288,299.66221102)
\curveto(868.9709268,299.76221089)(868.70092707,299.88721076)(868.4709288,300.03721102)
\curveto(868.24092753,300.18721046)(868.04092773,300.38221027)(867.8709288,300.62221102)
\curveto(867.78092799,300.7522099)(867.70592806,300.88720976)(867.6459288,301.02721102)
\curveto(867.58592818,301.16720948)(867.53092824,301.32220933)(867.4809288,301.49221102)
\curveto(867.46092831,301.5522091)(867.45092832,301.62220903)(867.4509288,301.70221102)
\curveto(867.46092831,301.79220886)(867.47592829,301.86220879)(867.4959288,301.91221102)
\curveto(867.52592824,301.9522087)(867.57592819,301.99220866)(867.6459288,302.03221102)
\curveto(867.69592807,302.0522086)(867.765928,302.06220859)(867.8559288,302.06221102)
\curveto(867.94592782,302.07220858)(868.03592773,302.07220858)(868.1259288,302.06221102)
\curveto(868.21592755,302.0522086)(868.30092747,302.03720861)(868.3809288,302.01721102)
\curveto(868.4709273,302.00720864)(868.53092724,301.99220866)(868.5609288,301.97221102)
\curveto(868.63092714,301.92220873)(868.67592709,301.8472088)(868.6959288,301.74721102)
\curveto(868.72592704,301.65720899)(868.76092701,301.57220908)(868.8009288,301.49221102)
\curveto(868.90092687,301.27220938)(869.03592673,301.10220955)(869.2059288,300.98221102)
\curveto(869.32592644,300.89220976)(869.46092631,300.82220983)(869.6109288,300.77221102)
\curveto(869.76092601,300.72220993)(869.92092585,300.67220998)(870.0909288,300.62221102)
\lineto(870.4059288,300.57721102)
\lineto(870.4959288,300.57721102)
\curveto(870.5659252,300.55721009)(870.65592511,300.5472101)(870.7659288,300.54721102)
\curveto(870.88592488,300.5472101)(870.98592478,300.55721009)(871.0659288,300.57721102)
\curveto(871.13592463,300.57721007)(871.19092458,300.58221007)(871.2309288,300.59221102)
\curveto(871.29092448,300.60221005)(871.35092442,300.60721004)(871.4109288,300.60721102)
\curveto(871.4709243,300.61721003)(871.52592424,300.62721002)(871.5759288,300.63721102)
\curveto(871.8659239,300.71720993)(872.09592367,300.82220983)(872.2659288,300.95221102)
\curveto(872.43592333,301.08220957)(872.55592321,301.30220935)(872.6259288,301.61221102)
\curveto(872.64592312,301.66220899)(872.65092312,301.71720893)(872.6409288,301.77721102)
\curveto(872.63092314,301.83720881)(872.62092315,301.88220877)(872.6109288,301.91221102)
\curveto(872.56092321,302.10220855)(872.49092328,302.24220841)(872.4009288,302.33221102)
\curveto(872.31092346,302.43220822)(872.19592357,302.52220813)(872.0559288,302.60221102)
\curveto(871.9659238,302.66220799)(871.8659239,302.71220794)(871.7559288,302.75221102)
\lineto(871.4259288,302.87221102)
\curveto(871.39592437,302.88220777)(871.3659244,302.88720776)(871.3359288,302.88721102)
\curveto(871.31592445,302.88720776)(871.29092448,302.89720775)(871.2609288,302.91721102)
\curveto(870.92092485,303.02720762)(870.5659252,303.10720754)(870.1959288,303.15721102)
\curveto(869.83592593,303.21720743)(869.49592627,303.31220734)(869.1759288,303.44221102)
\curveto(869.07592669,303.48220717)(868.98092679,303.51720713)(868.8909288,303.54721102)
\curveto(868.80092697,303.57720707)(868.71592705,303.61720703)(868.6359288,303.66721102)
\curveto(868.44592732,303.77720687)(868.2709275,303.90220675)(868.1109288,304.04221102)
\curveto(867.95092782,304.18220647)(867.82592794,304.35720629)(867.7359288,304.56721102)
\curveto(867.70592806,304.63720601)(867.68092809,304.70720594)(867.6609288,304.77721102)
\curveto(867.65092812,304.8472058)(867.63592813,304.92220573)(867.6159288,305.00221102)
\curveto(867.58592818,305.12220553)(867.57592819,305.25720539)(867.5859288,305.40721102)
\curveto(867.59592817,305.56720508)(867.61092816,305.70220495)(867.6309288,305.81221102)
\curveto(867.65092812,305.86220479)(867.66092811,305.90220475)(867.6609288,305.93221102)
\curveto(867.6709281,305.97220468)(867.68592808,306.01220464)(867.7059288,306.05221102)
\curveto(867.79592797,306.28220437)(867.91592785,306.48220417)(868.0659288,306.65221102)
\curveto(868.22592754,306.82220383)(868.40592736,306.97220368)(868.6059288,307.10221102)
\curveto(868.75592701,307.19220346)(868.92092685,307.26220339)(869.1009288,307.31221102)
\curveto(869.28092649,307.37220328)(869.4709263,307.42720322)(869.6709288,307.47721102)
\curveto(869.74092603,307.48720316)(869.80592596,307.49720315)(869.8659288,307.50721102)
\curveto(869.93592583,307.51720313)(870.01092576,307.52720312)(870.0909288,307.53721102)
\curveto(870.12092565,307.5472031)(870.16092561,307.5472031)(870.2109288,307.53721102)
\curveto(870.26092551,307.52720312)(870.29592547,307.53220312)(870.3159288,307.55221102)
}
}
{
\newrgbcolor{curcolor}{0.80000001 0.80000001 0.80000001}
\pscustom[linestyle=none,fillstyle=solid,fillcolor=curcolor]
{
\newpath
\moveto(798.51865829,310.35724764)
\lineto(813.51865829,310.35724764)
\lineto(813.51865829,295.35724764)
\lineto(798.51865829,295.35724764)
\closepath
}
}
{
\newrgbcolor{curcolor}{0 0 0}
\pscustom[linestyle=none,fillstyle=solid,fillcolor=curcolor]
{
\newpath
\moveto(826.4568663,277.57003328)
\curveto(826.47685675,277.52003254)(826.50185673,277.4600326)(826.5318663,277.39003328)
\curveto(826.56185667,277.32003274)(826.58185665,277.24503281)(826.5918663,277.16503328)
\curveto(826.61185662,277.09503296)(826.61185662,277.02503303)(826.5918663,276.95503328)
\curveto(826.58185665,276.89503316)(826.54185669,276.85003321)(826.4718663,276.82003328)
\curveto(826.42185681,276.80003326)(826.36185687,276.79003327)(826.2918663,276.79003328)
\lineto(826.0818663,276.79003328)
\lineto(825.6318663,276.79003328)
\curveto(825.48185775,276.79003327)(825.36185787,276.81503324)(825.2718663,276.86503328)
\curveto(825.17185806,276.92503313)(825.09685813,277.03003303)(825.0468663,277.18003328)
\curveto(825.00685822,277.33003273)(824.96185827,277.46503259)(824.9118663,277.58503328)
\curveto(824.80185843,277.84503221)(824.70185853,278.11503194)(824.6118663,278.39503328)
\curveto(824.52185871,278.67503138)(824.42185881,278.95003111)(824.3118663,279.22003328)
\curveto(824.28185895,279.31003075)(824.25185898,279.39503066)(824.2218663,279.47503328)
\curveto(824.20185903,279.5550305)(824.17185906,279.63003043)(824.1318663,279.70003328)
\curveto(824.10185913,279.77003029)(824.05685917,279.83003023)(823.9968663,279.88003328)
\curveto(823.93685929,279.93003013)(823.85685937,279.97003009)(823.7568663,280.00003328)
\curveto(823.70685952,280.02003004)(823.64685958,280.02503003)(823.5768663,280.01503328)
\lineto(823.3818663,280.01503328)
\lineto(820.5468663,280.01503328)
\lineto(820.2468663,280.01503328)
\curveto(820.13686309,280.02503003)(820.0318632,280.02503003)(819.9318663,280.01503328)
\curveto(819.8318634,280.00503005)(819.73686349,279.99003007)(819.6468663,279.97003328)
\curveto(819.56686366,279.95003011)(819.50686372,279.91003015)(819.4668663,279.85003328)
\curveto(819.38686384,279.75003031)(819.3268639,279.63503042)(819.2868663,279.50503328)
\curveto(819.25686397,279.38503067)(819.21686401,279.2600308)(819.1668663,279.13003328)
\curveto(819.06686416,278.90003116)(818.97186426,278.6600314)(818.8818663,278.41003328)
\curveto(818.80186443,278.1600319)(818.71186452,277.92003214)(818.6118663,277.69003328)
\curveto(818.59186464,277.63003243)(818.56686466,277.5600325)(818.5368663,277.48003328)
\curveto(818.51686471,277.41003265)(818.49186474,277.33503272)(818.4618663,277.25503328)
\curveto(818.4318648,277.17503288)(818.39686483,277.10003296)(818.3568663,277.03003328)
\curveto(818.3268649,276.97003309)(818.29186494,276.92503313)(818.2518663,276.89503328)
\curveto(818.17186506,276.83503322)(818.06186517,276.80003326)(817.9218663,276.79003328)
\lineto(817.5018663,276.79003328)
\lineto(817.2618663,276.79003328)
\curveto(817.19186604,276.80003326)(817.1318661,276.82503323)(817.0818663,276.86503328)
\curveto(817.0318662,276.89503316)(817.00186623,276.94003312)(816.9918663,277.00003328)
\curveto(816.99186624,277.060033)(816.99686623,277.12003294)(817.0068663,277.18003328)
\curveto(817.0268662,277.25003281)(817.04686618,277.31503274)(817.0668663,277.37503328)
\curveto(817.09686613,277.44503261)(817.12186611,277.49503256)(817.1418663,277.52503328)
\curveto(817.28186595,277.84503221)(817.40686582,278.1600319)(817.5168663,278.47003328)
\curveto(817.6268656,278.79003127)(817.74686548,279.11003095)(817.8768663,279.43003328)
\curveto(817.96686526,279.65003041)(818.05186518,279.86503019)(818.1318663,280.07503328)
\curveto(818.21186502,280.29502976)(818.29686493,280.51502954)(818.3868663,280.73503328)
\curveto(818.68686454,281.4550286)(818.97186426,282.18002788)(819.2418663,282.91003328)
\curveto(819.51186372,283.65002641)(819.79686343,284.38502567)(820.0968663,285.11503328)
\curveto(820.20686302,285.37502468)(820.30686292,285.64002442)(820.3968663,285.91003328)
\curveto(820.49686273,286.18002388)(820.60186263,286.44502361)(820.7118663,286.70503328)
\curveto(820.76186247,286.81502324)(820.80686242,286.93502312)(820.8468663,287.06503328)
\curveto(820.89686233,287.20502285)(820.96686226,287.30502275)(821.0568663,287.36503328)
\curveto(821.09686213,287.40502265)(821.16186207,287.43502262)(821.2518663,287.45503328)
\curveto(821.27186196,287.46502259)(821.29186194,287.46502259)(821.3118663,287.45503328)
\curveto(821.34186189,287.4550226)(821.36686186,287.4600226)(821.3868663,287.47003328)
\curveto(821.56686166,287.47002259)(821.77686145,287.47002259)(822.0168663,287.47003328)
\curveto(822.25686097,287.48002258)(822.4318608,287.44502261)(822.5418663,287.36503328)
\curveto(822.62186061,287.30502275)(822.68186055,287.20502285)(822.7218663,287.06503328)
\curveto(822.77186046,286.93502312)(822.82186041,286.81502324)(822.8718663,286.70503328)
\curveto(822.97186026,286.47502358)(823.06186017,286.24502381)(823.1418663,286.01503328)
\curveto(823.22186001,285.78502427)(823.31185992,285.5550245)(823.4118663,285.32503328)
\curveto(823.49185974,285.12502493)(823.56685966,284.92002514)(823.6368663,284.71003328)
\curveto(823.71685951,284.50002556)(823.80185943,284.29502576)(823.8918663,284.09503328)
\curveto(824.19185904,283.36502669)(824.47685875,282.62502743)(824.7468663,281.87503328)
\curveto(825.0268582,281.13502892)(825.32185791,280.40002966)(825.6318663,279.67003328)
\curveto(825.67185756,279.58003048)(825.70185753,279.49503056)(825.7218663,279.41503328)
\curveto(825.75185748,279.33503072)(825.78185745,279.25003081)(825.8118663,279.16003328)
\curveto(825.92185731,278.90003116)(826.0268572,278.63503142)(826.1268663,278.36503328)
\curveto(826.23685699,278.09503196)(826.34685688,277.83003223)(826.4568663,277.57003328)
\moveto(823.2468663,281.21503328)
\curveto(823.33685989,281.24502881)(823.39185984,281.29502876)(823.4118663,281.36503328)
\curveto(823.44185979,281.43502862)(823.44685978,281.51002855)(823.4268663,281.59003328)
\curveto(823.41685981,281.68002838)(823.39185984,281.76502829)(823.3518663,281.84503328)
\curveto(823.32185991,281.93502812)(823.29185994,282.01002805)(823.2618663,282.07003328)
\curveto(823.24185999,282.11002795)(823.23186,282.14502791)(823.2318663,282.17503328)
\curveto(823.23186,282.20502785)(823.22186001,282.24002782)(823.2018663,282.28003328)
\lineto(823.1118663,282.52003328)
\curveto(823.09186014,282.61002745)(823.06186017,282.70002736)(823.0218663,282.79003328)
\curveto(822.87186036,283.15002691)(822.73686049,283.51502654)(822.6168663,283.88503328)
\curveto(822.50686072,284.26502579)(822.37686085,284.63502542)(822.2268663,284.99503328)
\curveto(822.17686105,285.10502495)(822.1318611,285.21502484)(822.0918663,285.32503328)
\curveto(822.06186117,285.43502462)(822.02186121,285.54002452)(821.9718663,285.64003328)
\curveto(821.95186128,285.69002437)(821.9268613,285.73502432)(821.8968663,285.77503328)
\curveto(821.87686135,285.82502423)(821.8268614,285.85002421)(821.7468663,285.85003328)
\curveto(821.7268615,285.83002423)(821.70686152,285.81502424)(821.6868663,285.80503328)
\curveto(821.66686156,285.79502426)(821.64686158,285.78002428)(821.6268663,285.76003328)
\curveto(821.58686164,285.71002435)(821.55686167,285.6550244)(821.5368663,285.59503328)
\curveto(821.51686171,285.54502451)(821.49686173,285.49002457)(821.4768663,285.43003328)
\curveto(821.4268618,285.32002474)(821.38686184,285.21002485)(821.3568663,285.10003328)
\curveto(821.3268619,284.99002507)(821.28686194,284.88002518)(821.2368663,284.77003328)
\curveto(821.06686216,284.38002568)(820.91686231,283.98502607)(820.7868663,283.58503328)
\curveto(820.66686256,283.18502687)(820.5268627,282.79502726)(820.3668663,282.41503328)
\lineto(820.3068663,282.26503328)
\curveto(820.29686293,282.21502784)(820.28186295,282.16502789)(820.2618663,282.11503328)
\lineto(820.1718663,281.87503328)
\curveto(820.14186309,281.79502826)(820.11686311,281.71502834)(820.0968663,281.63503328)
\curveto(820.07686315,281.58502847)(820.06686316,281.53002853)(820.0668663,281.47003328)
\curveto(820.07686315,281.41002865)(820.09186314,281.3600287)(820.1118663,281.32003328)
\curveto(820.16186307,281.24002882)(820.26686296,281.19502886)(820.4268663,281.18503328)
\lineto(820.8768663,281.18503328)
\lineto(822.4818663,281.18503328)
\curveto(822.59186064,281.18502887)(822.7268605,281.18002888)(822.8868663,281.17003328)
\curveto(823.04686018,281.17002889)(823.16686006,281.18502887)(823.2468663,281.21503328)
}
}
{
\newrgbcolor{curcolor}{0 0 0}
\pscustom[linestyle=none,fillstyle=solid,fillcolor=curcolor]
{
\newpath
\moveto(831.2184288,284.69503328)
\curveto(831.44842401,284.69502536)(831.57842388,284.63502542)(831.6084288,284.51503328)
\curveto(831.63842382,284.40502565)(831.6534238,284.24002582)(831.6534288,284.02003328)
\lineto(831.6534288,283.73503328)
\curveto(831.6534238,283.64502641)(831.62842383,283.57002649)(831.5784288,283.51003328)
\curveto(831.51842394,283.43002663)(831.43342402,283.38502667)(831.3234288,283.37503328)
\curveto(831.21342424,283.37502668)(831.10342435,283.3600267)(830.9934288,283.33003328)
\curveto(830.8534246,283.30002676)(830.71842474,283.27002679)(830.5884288,283.24003328)
\curveto(830.46842499,283.21002685)(830.3534251,283.17002689)(830.2434288,283.12003328)
\curveto(829.9534255,282.99002707)(829.71842574,282.81002725)(829.5384288,282.58003328)
\curveto(829.3584261,282.3600277)(829.20342625,282.10502795)(829.0734288,281.81503328)
\curveto(829.03342642,281.70502835)(829.00342645,281.59002847)(828.9834288,281.47003328)
\curveto(828.96342649,281.3600287)(828.93842652,281.24502881)(828.9084288,281.12503328)
\curveto(828.89842656,281.07502898)(828.89342656,281.02502903)(828.8934288,280.97503328)
\curveto(828.90342655,280.92502913)(828.90342655,280.87502918)(828.8934288,280.82503328)
\curveto(828.86342659,280.70502935)(828.84842661,280.56502949)(828.8484288,280.40503328)
\curveto(828.8584266,280.2550298)(828.86342659,280.11002995)(828.8634288,279.97003328)
\lineto(828.8634288,278.12503328)
\lineto(828.8634288,277.78003328)
\curveto(828.86342659,277.6600324)(828.8584266,277.54503251)(828.8484288,277.43503328)
\curveto(828.83842662,277.32503273)(828.83342662,277.23003283)(828.8334288,277.15003328)
\curveto(828.84342661,277.07003299)(828.82342663,277.00003306)(828.7734288,276.94003328)
\curveto(828.72342673,276.87003319)(828.64342681,276.83003323)(828.5334288,276.82003328)
\curveto(828.43342702,276.81003325)(828.32342713,276.80503325)(828.2034288,276.80503328)
\lineto(827.9334288,276.80503328)
\curveto(827.88342757,276.82503323)(827.83342762,276.84003322)(827.7834288,276.85003328)
\curveto(827.74342771,276.87003319)(827.71342774,276.89503316)(827.6934288,276.92503328)
\curveto(827.64342781,276.99503306)(827.61342784,277.08003298)(827.6034288,277.18003328)
\lineto(827.6034288,277.51003328)
\lineto(827.6034288,278.66503328)
\lineto(827.6034288,282.82003328)
\lineto(827.6034288,283.85503328)
\lineto(827.6034288,284.15503328)
\curveto(827.61342784,284.2550258)(827.64342781,284.34002572)(827.6934288,284.41003328)
\curveto(827.72342773,284.45002561)(827.77342768,284.48002558)(827.8434288,284.50003328)
\curveto(827.92342753,284.52002554)(828.00842745,284.53002553)(828.0984288,284.53003328)
\curveto(828.18842727,284.54002552)(828.27842718,284.54002552)(828.3684288,284.53003328)
\curveto(828.458427,284.52002554)(828.52842693,284.50502555)(828.5784288,284.48503328)
\curveto(828.6584268,284.4550256)(828.70842675,284.39502566)(828.7284288,284.30503328)
\curveto(828.7584267,284.22502583)(828.77342668,284.13502592)(828.7734288,284.03503328)
\lineto(828.7734288,283.73503328)
\curveto(828.77342668,283.63502642)(828.79342666,283.54502651)(828.8334288,283.46503328)
\curveto(828.84342661,283.44502661)(828.8534266,283.43002663)(828.8634288,283.42003328)
\lineto(828.9084288,283.37503328)
\curveto(829.01842644,283.37502668)(829.10842635,283.42002664)(829.1784288,283.51003328)
\curveto(829.24842621,283.61002645)(829.30842615,283.69002637)(829.3584288,283.75003328)
\lineto(829.4484288,283.84003328)
\curveto(829.53842592,283.95002611)(829.66342579,284.06502599)(829.8234288,284.18503328)
\curveto(829.98342547,284.30502575)(830.13342532,284.39502566)(830.2734288,284.45503328)
\curveto(830.36342509,284.50502555)(830.458425,284.54002552)(830.5584288,284.56003328)
\curveto(830.6584248,284.59002547)(830.76342469,284.62002544)(830.8734288,284.65003328)
\curveto(830.93342452,284.6600254)(830.99342446,284.66502539)(831.0534288,284.66503328)
\curveto(831.11342434,284.67502538)(831.16842429,284.68502537)(831.2184288,284.69503328)
}
}
{
\newrgbcolor{curcolor}{0 0 0}
\pscustom[linestyle=none,fillstyle=solid,fillcolor=curcolor]
{
\newpath
\moveto(839.33319442,280.96003328)
\curveto(839.35318674,280.8600292)(839.35318674,280.74502931)(839.33319442,280.61503328)
\curveto(839.32318677,280.49502956)(839.2931868,280.41002965)(839.24319442,280.36003328)
\curveto(839.1931869,280.32002974)(839.11818697,280.29002977)(839.01819442,280.27003328)
\curveto(838.92818716,280.2600298)(838.82318727,280.2550298)(838.70319442,280.25503328)
\lineto(838.34319442,280.25503328)
\curveto(838.22318787,280.26502979)(838.11818797,280.27002979)(838.02819442,280.27003328)
\lineto(834.18819442,280.27003328)
\curveto(834.10819198,280.27002979)(834.02819206,280.26502979)(833.94819442,280.25503328)
\curveto(833.86819222,280.2550298)(833.80319229,280.24002982)(833.75319442,280.21003328)
\curveto(833.71319238,280.19002987)(833.67319242,280.15002991)(833.63319442,280.09003328)
\curveto(833.61319248,280.06003)(833.5931925,280.01503004)(833.57319442,279.95503328)
\curveto(833.55319254,279.90503015)(833.55319254,279.8550302)(833.57319442,279.80503328)
\curveto(833.58319251,279.7550303)(833.5881925,279.71003035)(833.58819442,279.67003328)
\curveto(833.5881925,279.63003043)(833.5931925,279.59003047)(833.60319442,279.55003328)
\curveto(833.62319247,279.47003059)(833.64319245,279.38503067)(833.66319442,279.29503328)
\curveto(833.68319241,279.21503084)(833.71319238,279.13503092)(833.75319442,279.05503328)
\curveto(833.98319211,278.51503154)(834.36319173,278.13003193)(834.89319442,277.90003328)
\curveto(834.95319114,277.87003219)(835.01819107,277.84503221)(835.08819442,277.82503328)
\lineto(835.29819442,277.76503328)
\curveto(835.32819076,277.7550323)(835.37819071,277.75003231)(835.44819442,277.75003328)
\curveto(835.5881905,277.71003235)(835.77319032,277.69003237)(836.00319442,277.69003328)
\curveto(836.23318986,277.69003237)(836.41818967,277.71003235)(836.55819442,277.75003328)
\curveto(836.69818939,277.79003227)(836.82318927,277.83003223)(836.93319442,277.87003328)
\curveto(837.05318904,277.92003214)(837.16318893,277.98003208)(837.26319442,278.05003328)
\curveto(837.37318872,278.12003194)(837.46818862,278.20003186)(837.54819442,278.29003328)
\curveto(837.62818846,278.39003167)(837.69818839,278.49503156)(837.75819442,278.60503328)
\curveto(837.81818827,278.70503135)(837.86818822,278.81003125)(837.90819442,278.92003328)
\curveto(837.95818813,279.03003103)(838.03818805,279.11003095)(838.14819442,279.16003328)
\curveto(838.1881879,279.18003088)(838.25318784,279.19503086)(838.34319442,279.20503328)
\curveto(838.43318766,279.21503084)(838.52318757,279.21503084)(838.61319442,279.20503328)
\curveto(838.70318739,279.20503085)(838.7881873,279.20003086)(838.86819442,279.19003328)
\curveto(838.94818714,279.18003088)(839.00318709,279.1600309)(839.03319442,279.13003328)
\curveto(839.13318696,279.060031)(839.15818693,278.94503111)(839.10819442,278.78503328)
\curveto(839.02818706,278.51503154)(838.92318717,278.27503178)(838.79319442,278.06503328)
\curveto(838.5931875,277.74503231)(838.36318773,277.48003258)(838.10319442,277.27003328)
\curveto(837.85318824,277.07003299)(837.53318856,276.90503315)(837.14319442,276.77503328)
\curveto(837.04318905,276.73503332)(836.94318915,276.71003335)(836.84319442,276.70003328)
\curveto(836.74318935,276.68003338)(836.63818945,276.6600334)(836.52819442,276.64003328)
\curveto(836.47818961,276.63003343)(836.42818966,276.62503343)(836.37819442,276.62503328)
\curveto(836.33818975,276.62503343)(836.2931898,276.62003344)(836.24319442,276.61003328)
\lineto(836.09319442,276.61003328)
\curveto(836.04319005,276.60003346)(835.98319011,276.59503346)(835.91319442,276.59503328)
\curveto(835.85319024,276.59503346)(835.80319029,276.60003346)(835.76319442,276.61003328)
\lineto(835.62819442,276.61003328)
\curveto(835.57819051,276.62003344)(835.53319056,276.62503343)(835.49319442,276.62503328)
\curveto(835.45319064,276.62503343)(835.41319068,276.63003343)(835.37319442,276.64003328)
\curveto(835.32319077,276.65003341)(835.26819082,276.6600334)(835.20819442,276.67003328)
\curveto(835.14819094,276.67003339)(835.093191,276.67503338)(835.04319442,276.68503328)
\curveto(834.95319114,276.70503335)(834.86319123,276.73003333)(834.77319442,276.76003328)
\curveto(834.68319141,276.78003328)(834.59819149,276.80503325)(834.51819442,276.83503328)
\curveto(834.47819161,276.8550332)(834.44319165,276.86503319)(834.41319442,276.86503328)
\curveto(834.38319171,276.87503318)(834.34819174,276.89003317)(834.30819442,276.91003328)
\curveto(834.15819193,276.98003308)(833.99819209,277.06503299)(833.82819442,277.16503328)
\curveto(833.53819255,277.3550327)(833.2881928,277.58503247)(833.07819442,277.85503328)
\curveto(832.87819321,278.13503192)(832.70819338,278.44503161)(832.56819442,278.78503328)
\curveto(832.51819357,278.89503116)(832.47819361,279.01003105)(832.44819442,279.13003328)
\curveto(832.42819366,279.25003081)(832.39819369,279.37003069)(832.35819442,279.49003328)
\curveto(832.34819374,279.53003053)(832.34319375,279.56503049)(832.34319442,279.59503328)
\curveto(832.34319375,279.62503043)(832.33819375,279.66503039)(832.32819442,279.71503328)
\curveto(832.30819378,279.79503026)(832.2931938,279.88003018)(832.28319442,279.97003328)
\curveto(832.27319382,280.06003)(832.25819383,280.15002991)(832.23819442,280.24003328)
\lineto(832.23819442,280.45003328)
\curveto(832.22819386,280.49002957)(832.21819387,280.54502951)(832.20819442,280.61503328)
\curveto(832.20819388,280.69502936)(832.21319388,280.7600293)(832.22319442,280.81003328)
\lineto(832.22319442,280.97503328)
\curveto(832.24319385,281.02502903)(832.24819384,281.07502898)(832.23819442,281.12503328)
\curveto(832.23819385,281.18502887)(832.24319385,281.24002882)(832.25319442,281.29003328)
\curveto(832.2931938,281.45002861)(832.32319377,281.61002845)(832.34319442,281.77003328)
\curveto(832.37319372,281.93002813)(832.41819367,282.08002798)(832.47819442,282.22003328)
\curveto(832.52819356,282.33002773)(832.57319352,282.44002762)(832.61319442,282.55003328)
\curveto(832.66319343,282.67002739)(832.71819337,282.78502727)(832.77819442,282.89503328)
\curveto(832.99819309,283.24502681)(833.24819284,283.54502651)(833.52819442,283.79503328)
\curveto(833.80819228,284.055026)(834.15319194,284.27002579)(834.56319442,284.44003328)
\curveto(834.68319141,284.49002557)(834.80319129,284.52502553)(834.92319442,284.54503328)
\curveto(835.05319104,284.57502548)(835.1881909,284.60502545)(835.32819442,284.63503328)
\curveto(835.37819071,284.64502541)(835.42319067,284.65002541)(835.46319442,284.65003328)
\curveto(835.50319059,284.6600254)(835.54819054,284.66502539)(835.59819442,284.66503328)
\curveto(835.61819047,284.67502538)(835.64319045,284.67502538)(835.67319442,284.66503328)
\curveto(835.70319039,284.6550254)(835.72819036,284.6600254)(835.74819442,284.68003328)
\curveto(836.16818992,284.69002537)(836.53318956,284.64502541)(836.84319442,284.54503328)
\curveto(837.15318894,284.4550256)(837.43318866,284.33002573)(837.68319442,284.17003328)
\curveto(837.73318836,284.15002591)(837.77318832,284.12002594)(837.80319442,284.08003328)
\curveto(837.83318826,284.05002601)(837.86818822,284.02502603)(837.90819442,284.00503328)
\curveto(837.9881881,283.94502611)(838.06818802,283.87502618)(838.14819442,283.79503328)
\curveto(838.23818785,283.71502634)(838.31318778,283.63502642)(838.37319442,283.55503328)
\curveto(838.53318756,283.34502671)(838.66818742,283.14502691)(838.77819442,282.95503328)
\curveto(838.84818724,282.84502721)(838.90318719,282.72502733)(838.94319442,282.59503328)
\curveto(838.98318711,282.46502759)(839.02818706,282.33502772)(839.07819442,282.20503328)
\curveto(839.12818696,282.07502798)(839.16318693,281.94002812)(839.18319442,281.80003328)
\curveto(839.21318688,281.6600284)(839.24818684,281.52002854)(839.28819442,281.38003328)
\curveto(839.29818679,281.31002875)(839.30318679,281.24002882)(839.30319442,281.17003328)
\lineto(839.33319442,280.96003328)
\moveto(837.87819442,281.47003328)
\curveto(837.90818818,281.51002855)(837.93318816,281.5600285)(837.95319442,281.62003328)
\curveto(837.97318812,281.69002837)(837.97318812,281.7600283)(837.95319442,281.83003328)
\curveto(837.8931882,282.05002801)(837.80818828,282.2550278)(837.69819442,282.44503328)
\curveto(837.55818853,282.67502738)(837.40318869,282.87002719)(837.23319442,283.03003328)
\curveto(837.06318903,283.19002687)(836.84318925,283.32502673)(836.57319442,283.43503328)
\curveto(836.50318959,283.4550266)(836.43318966,283.47002659)(836.36319442,283.48003328)
\curveto(836.2931898,283.50002656)(836.21818987,283.52002654)(836.13819442,283.54003328)
\curveto(836.05819003,283.5600265)(835.97319012,283.57002649)(835.88319442,283.57003328)
\lineto(835.62819442,283.57003328)
\curveto(835.59819049,283.55002651)(835.56319053,283.54002652)(835.52319442,283.54003328)
\curveto(835.48319061,283.55002651)(835.44819064,283.55002651)(835.41819442,283.54003328)
\lineto(835.17819442,283.48003328)
\curveto(835.10819098,283.47002659)(835.03819105,283.4550266)(834.96819442,283.43503328)
\curveto(834.67819141,283.31502674)(834.44319165,283.16502689)(834.26319442,282.98503328)
\curveto(834.093192,282.80502725)(833.93819215,282.58002748)(833.79819442,282.31003328)
\curveto(833.76819232,282.2600278)(833.73819235,282.19502786)(833.70819442,282.11503328)
\curveto(833.67819241,282.04502801)(833.65319244,281.96502809)(833.63319442,281.87503328)
\curveto(833.61319248,281.78502827)(833.60819248,281.70002836)(833.61819442,281.62003328)
\curveto(833.62819246,281.54002852)(833.66319243,281.48002858)(833.72319442,281.44003328)
\curveto(833.80319229,281.38002868)(833.93819215,281.35002871)(834.12819442,281.35003328)
\curveto(834.32819176,281.3600287)(834.49819159,281.36502869)(834.63819442,281.36503328)
\lineto(836.91819442,281.36503328)
\curveto(837.06818902,281.36502869)(837.24818884,281.3600287)(837.45819442,281.35003328)
\curveto(837.66818842,281.35002871)(837.80818828,281.39002867)(837.87819442,281.47003328)
}
}
{
\newrgbcolor{curcolor}{0 0 0}
\pscustom[linestyle=none,fillstyle=solid,fillcolor=curcolor]
{
\newpath
\moveto(847.52483505,277.34503328)
\curveto(847.55482722,277.18503287)(847.53982723,277.05003301)(847.47983505,276.94003328)
\curveto(847.41982735,276.84003322)(847.33982743,276.76503329)(847.23983505,276.71503328)
\curveto(847.18982758,276.69503336)(847.13482764,276.68503337)(847.07483505,276.68503328)
\curveto(847.02482775,276.68503337)(846.9698278,276.67503338)(846.90983505,276.65503328)
\curveto(846.68982808,276.60503345)(846.4698283,276.62003344)(846.24983505,276.70003328)
\curveto(846.03982873,276.77003329)(845.89482888,276.8600332)(845.81483505,276.97003328)
\curveto(845.76482901,277.04003302)(845.71982905,277.12003294)(845.67983505,277.21003328)
\curveto(845.63982913,277.31003275)(845.58982918,277.39003267)(845.52983505,277.45003328)
\curveto(845.50982926,277.47003259)(845.48482929,277.49003257)(845.45483505,277.51003328)
\curveto(845.43482934,277.53003253)(845.40482937,277.53503252)(845.36483505,277.52503328)
\curveto(845.25482952,277.49503256)(845.14982962,277.44003262)(845.04983505,277.36003328)
\curveto(844.95982981,277.28003278)(844.8698299,277.21003285)(844.77983505,277.15003328)
\curveto(844.64983012,277.07003299)(844.50983026,276.99503306)(844.35983505,276.92503328)
\curveto(844.20983056,276.86503319)(844.04983072,276.81003325)(843.87983505,276.76003328)
\curveto(843.77983099,276.73003333)(843.6698311,276.71003335)(843.54983505,276.70003328)
\curveto(843.43983133,276.69003337)(843.32983144,276.67503338)(843.21983505,276.65503328)
\curveto(843.1698316,276.64503341)(843.12483165,276.64003342)(843.08483505,276.64003328)
\lineto(842.97983505,276.64003328)
\curveto(842.8698319,276.62003344)(842.76483201,276.62003344)(842.66483505,276.64003328)
\lineto(842.52983505,276.64003328)
\curveto(842.47983229,276.65003341)(842.42983234,276.6550334)(842.37983505,276.65503328)
\curveto(842.32983244,276.6550334)(842.28483249,276.66503339)(842.24483505,276.68503328)
\curveto(842.20483257,276.69503336)(842.1698326,276.70003336)(842.13983505,276.70003328)
\curveto(842.11983265,276.69003337)(842.09483268,276.69003337)(842.06483505,276.70003328)
\lineto(841.82483505,276.76003328)
\curveto(841.74483303,276.77003329)(841.6698331,276.79003327)(841.59983505,276.82003328)
\curveto(841.29983347,276.95003311)(841.05483372,277.09503296)(840.86483505,277.25503328)
\curveto(840.68483409,277.42503263)(840.53483424,277.6600324)(840.41483505,277.96003328)
\curveto(840.32483445,278.18003188)(840.27983449,278.44503161)(840.27983505,278.75503328)
\lineto(840.27983505,279.07003328)
\curveto(840.28983448,279.12003094)(840.29483448,279.17003089)(840.29483505,279.22003328)
\lineto(840.32483505,279.40003328)
\lineto(840.44483505,279.73003328)
\curveto(840.48483429,279.84003022)(840.53483424,279.94003012)(840.59483505,280.03003328)
\curveto(840.774834,280.32002974)(841.01983375,280.53502952)(841.32983505,280.67503328)
\curveto(841.63983313,280.81502924)(841.97983279,280.94002912)(842.34983505,281.05003328)
\curveto(842.48983228,281.09002897)(842.63483214,281.12002894)(842.78483505,281.14003328)
\curveto(842.93483184,281.1600289)(843.08483169,281.18502887)(843.23483505,281.21503328)
\curveto(843.30483147,281.23502882)(843.3698314,281.24502881)(843.42983505,281.24503328)
\curveto(843.49983127,281.24502881)(843.5748312,281.2550288)(843.65483505,281.27503328)
\curveto(843.72483105,281.29502876)(843.79483098,281.30502875)(843.86483505,281.30503328)
\curveto(843.93483084,281.31502874)(844.00983076,281.33002873)(844.08983505,281.35003328)
\curveto(844.33983043,281.41002865)(844.5748302,281.4600286)(844.79483505,281.50003328)
\curveto(845.01482976,281.55002851)(845.18982958,281.66502839)(845.31983505,281.84503328)
\curveto(845.37982939,281.92502813)(845.42982934,282.02502803)(845.46983505,282.14503328)
\curveto(845.50982926,282.27502778)(845.50982926,282.41502764)(845.46983505,282.56503328)
\curveto(845.40982936,282.80502725)(845.31982945,282.99502706)(845.19983505,283.13503328)
\curveto(845.08982968,283.27502678)(844.92982984,283.38502667)(844.71983505,283.46503328)
\curveto(844.59983017,283.51502654)(844.45483032,283.55002651)(844.28483505,283.57003328)
\curveto(844.12483065,283.59002647)(843.95483082,283.60002646)(843.77483505,283.60003328)
\curveto(843.59483118,283.60002646)(843.41983135,283.59002647)(843.24983505,283.57003328)
\curveto(843.07983169,283.55002651)(842.93483184,283.52002654)(842.81483505,283.48003328)
\curveto(842.64483213,283.42002664)(842.47983229,283.33502672)(842.31983505,283.22503328)
\curveto(842.23983253,283.16502689)(842.16483261,283.08502697)(842.09483505,282.98503328)
\curveto(842.03483274,282.89502716)(841.97983279,282.79502726)(841.92983505,282.68503328)
\curveto(841.89983287,282.60502745)(841.8698329,282.52002754)(841.83983505,282.43003328)
\curveto(841.81983295,282.34002772)(841.774833,282.27002779)(841.70483505,282.22003328)
\curveto(841.66483311,282.19002787)(841.59483318,282.16502789)(841.49483505,282.14503328)
\curveto(841.40483337,282.13502792)(841.30983346,282.13002793)(841.20983505,282.13003328)
\curveto(841.10983366,282.13002793)(841.00983376,282.13502792)(840.90983505,282.14503328)
\curveto(840.81983395,282.16502789)(840.75483402,282.19002787)(840.71483505,282.22003328)
\curveto(840.6748341,282.25002781)(840.64483413,282.30002776)(840.62483505,282.37003328)
\curveto(840.60483417,282.44002762)(840.60483417,282.51502754)(840.62483505,282.59503328)
\curveto(840.65483412,282.72502733)(840.68483409,282.84502721)(840.71483505,282.95503328)
\curveto(840.75483402,283.07502698)(840.79983397,283.19002687)(840.84983505,283.30003328)
\curveto(841.03983373,283.65002641)(841.27983349,283.92002614)(841.56983505,284.11003328)
\curveto(841.85983291,284.31002575)(842.21983255,284.47002559)(842.64983505,284.59003328)
\curveto(842.74983202,284.61002545)(842.84983192,284.62502543)(842.94983505,284.63503328)
\curveto(843.05983171,284.64502541)(843.1698316,284.6600254)(843.27983505,284.68003328)
\curveto(843.31983145,284.69002537)(843.38483139,284.69002537)(843.47483505,284.68003328)
\curveto(843.56483121,284.68002538)(843.61983115,284.69002537)(843.63983505,284.71003328)
\curveto(844.33983043,284.72002534)(844.94982982,284.64002542)(845.46983505,284.47003328)
\curveto(845.98982878,284.30002576)(846.35482842,283.97502608)(846.56483505,283.49503328)
\curveto(846.65482812,283.29502676)(846.70482807,283.060027)(846.71483505,282.79003328)
\curveto(846.73482804,282.53002753)(846.74482803,282.2550278)(846.74483505,281.96503328)
\lineto(846.74483505,278.65003328)
\curveto(846.74482803,278.51003155)(846.74982802,278.37503168)(846.75983505,278.24503328)
\curveto(846.769828,278.11503194)(846.79982797,278.01003205)(846.84983505,277.93003328)
\curveto(846.89982787,277.8600322)(846.96482781,277.81003225)(847.04483505,277.78003328)
\curveto(847.13482764,277.74003232)(847.21982755,277.71003235)(847.29983505,277.69003328)
\curveto(847.37982739,277.68003238)(847.43982733,277.63503242)(847.47983505,277.55503328)
\curveto(847.49982727,277.52503253)(847.50982726,277.49503256)(847.50983505,277.46503328)
\curveto(847.50982726,277.43503262)(847.51482726,277.39503266)(847.52483505,277.34503328)
\moveto(845.37983505,279.01003328)
\curveto(845.43982933,279.15003091)(845.4698293,279.31003075)(845.46983505,279.49003328)
\curveto(845.47982929,279.68003038)(845.48482929,279.87503018)(845.48483505,280.07503328)
\curveto(845.48482929,280.18502987)(845.47982929,280.28502977)(845.46983505,280.37503328)
\curveto(845.45982931,280.46502959)(845.41982935,280.53502952)(845.34983505,280.58503328)
\curveto(845.31982945,280.60502945)(845.24982952,280.61502944)(845.13983505,280.61503328)
\curveto(845.11982965,280.59502946)(845.08482969,280.58502947)(845.03483505,280.58503328)
\curveto(844.98482979,280.58502947)(844.93982983,280.57502948)(844.89983505,280.55503328)
\curveto(844.81982995,280.53502952)(844.72983004,280.51502954)(844.62983505,280.49503328)
\lineto(844.32983505,280.43503328)
\curveto(844.29983047,280.43502962)(844.26483051,280.43002963)(844.22483505,280.42003328)
\lineto(844.11983505,280.42003328)
\curveto(843.9698308,280.38002968)(843.80483097,280.3550297)(843.62483505,280.34503328)
\curveto(843.45483132,280.34502971)(843.29483148,280.32502973)(843.14483505,280.28503328)
\curveto(843.06483171,280.26502979)(842.98983178,280.24502981)(842.91983505,280.22503328)
\curveto(842.85983191,280.21502984)(842.78983198,280.20002986)(842.70983505,280.18003328)
\curveto(842.54983222,280.13002993)(842.39983237,280.06502999)(842.25983505,279.98503328)
\curveto(842.11983265,279.91503014)(841.99983277,279.82503023)(841.89983505,279.71503328)
\curveto(841.79983297,279.60503045)(841.72483305,279.47003059)(841.67483505,279.31003328)
\curveto(841.62483315,279.1600309)(841.60483317,278.97503108)(841.61483505,278.75503328)
\curveto(841.61483316,278.6550314)(841.62983314,278.5600315)(841.65983505,278.47003328)
\curveto(841.69983307,278.39003167)(841.74483303,278.31503174)(841.79483505,278.24503328)
\curveto(841.8748329,278.13503192)(841.97983279,278.04003202)(842.10983505,277.96003328)
\curveto(842.23983253,277.89003217)(842.37983239,277.83003223)(842.52983505,277.78003328)
\curveto(842.57983219,277.77003229)(842.62983214,277.76503229)(842.67983505,277.76503328)
\curveto(842.72983204,277.76503229)(842.77983199,277.7600323)(842.82983505,277.75003328)
\curveto(842.89983187,277.73003233)(842.98483179,277.71503234)(843.08483505,277.70503328)
\curveto(843.19483158,277.70503235)(843.28483149,277.71503234)(843.35483505,277.73503328)
\curveto(843.41483136,277.7550323)(843.4748313,277.7600323)(843.53483505,277.75003328)
\curveto(843.59483118,277.75003231)(843.65483112,277.7600323)(843.71483505,277.78003328)
\curveto(843.79483098,277.80003226)(843.8698309,277.81503224)(843.93983505,277.82503328)
\curveto(844.01983075,277.83503222)(844.09483068,277.8550322)(844.16483505,277.88503328)
\curveto(844.45483032,278.00503205)(844.69983007,278.15003191)(844.89983505,278.32003328)
\curveto(845.10982966,278.49003157)(845.2698295,278.72003134)(845.37983505,279.01003328)
}
}
{
\newrgbcolor{curcolor}{0 0 0}
\pscustom[linestyle=none,fillstyle=solid,fillcolor=curcolor]
{
\newpath
\moveto(851.12647567,284.69503328)
\curveto(851.84647161,284.70502535)(852.451471,284.62002544)(852.94147567,284.44003328)
\curveto(853.43147002,284.27002579)(853.81146964,283.96502609)(854.08147567,283.52503328)
\curveto(854.1514693,283.41502664)(854.20646925,283.30002676)(854.24647567,283.18003328)
\curveto(854.28646917,283.07002699)(854.32646913,282.94502711)(854.36647567,282.80503328)
\curveto(854.38646907,282.73502732)(854.39146906,282.6600274)(854.38147567,282.58003328)
\curveto(854.37146908,282.51002755)(854.3564691,282.4550276)(854.33647567,282.41503328)
\curveto(854.31646914,282.39502766)(854.29146916,282.37502768)(854.26147567,282.35503328)
\curveto(854.23146922,282.34502771)(854.20646925,282.33002773)(854.18647567,282.31003328)
\curveto(854.13646932,282.29002777)(854.08646937,282.28502777)(854.03647567,282.29503328)
\curveto(853.98646947,282.30502775)(853.93646952,282.30502775)(853.88647567,282.29503328)
\curveto(853.80646965,282.27502778)(853.70146975,282.27002779)(853.57147567,282.28003328)
\curveto(853.44147001,282.30002776)(853.3514701,282.32502773)(853.30147567,282.35503328)
\curveto(853.22147023,282.40502765)(853.16647029,282.47002759)(853.13647567,282.55003328)
\curveto(853.11647034,282.64002742)(853.08147037,282.72502733)(853.03147567,282.80503328)
\curveto(852.94147051,282.96502709)(852.81647064,283.11002695)(852.65647567,283.24003328)
\curveto(852.54647091,283.32002674)(852.42647103,283.38002668)(852.29647567,283.42003328)
\curveto(852.16647129,283.4600266)(852.02647143,283.50002656)(851.87647567,283.54003328)
\curveto(851.82647163,283.5600265)(851.77647168,283.56502649)(851.72647567,283.55503328)
\curveto(851.67647178,283.5550265)(851.62647183,283.5600265)(851.57647567,283.57003328)
\curveto(851.51647194,283.59002647)(851.44147201,283.60002646)(851.35147567,283.60003328)
\curveto(851.26147219,283.60002646)(851.18647227,283.59002647)(851.12647567,283.57003328)
\lineto(851.03647567,283.57003328)
\lineto(850.88647567,283.54003328)
\curveto(850.83647262,283.54002652)(850.78647267,283.53502652)(850.73647567,283.52503328)
\curveto(850.47647298,283.46502659)(850.26147319,283.38002668)(850.09147567,283.27003328)
\curveto(849.92147353,283.1600269)(849.80647365,282.97502708)(849.74647567,282.71503328)
\curveto(849.72647373,282.64502741)(849.72147373,282.57502748)(849.73147567,282.50503328)
\curveto(849.7514737,282.43502762)(849.77147368,282.37502768)(849.79147567,282.32503328)
\curveto(849.8514736,282.17502788)(849.92147353,282.06502799)(850.00147567,281.99503328)
\curveto(850.09147336,281.93502812)(850.20147325,281.86502819)(850.33147567,281.78503328)
\curveto(850.49147296,281.68502837)(850.67147278,281.61002845)(850.87147567,281.56003328)
\curveto(851.07147238,281.52002854)(851.27147218,281.47002859)(851.47147567,281.41003328)
\curveto(851.60147185,281.37002869)(851.73147172,281.34002872)(851.86147567,281.32003328)
\curveto(851.99147146,281.30002876)(852.12147133,281.27002879)(852.25147567,281.23003328)
\curveto(852.46147099,281.17002889)(852.66647079,281.11002895)(852.86647567,281.05003328)
\curveto(853.06647039,281.00002906)(853.26647019,280.93502912)(853.46647567,280.85503328)
\lineto(853.61647567,280.79503328)
\curveto(853.66646979,280.77502928)(853.71646974,280.75002931)(853.76647567,280.72003328)
\curveto(853.96646949,280.60002946)(854.14146931,280.46502959)(854.29147567,280.31503328)
\curveto(854.44146901,280.16502989)(854.56646889,279.97503008)(854.66647567,279.74503328)
\curveto(854.68646877,279.67503038)(854.70646875,279.58003048)(854.72647567,279.46003328)
\curveto(854.74646871,279.39003067)(854.7564687,279.31503074)(854.75647567,279.23503328)
\curveto(854.76646869,279.16503089)(854.77146868,279.08503097)(854.77147567,278.99503328)
\lineto(854.77147567,278.84503328)
\curveto(854.7514687,278.77503128)(854.74146871,278.70503135)(854.74147567,278.63503328)
\curveto(854.74146871,278.56503149)(854.73146872,278.49503156)(854.71147567,278.42503328)
\curveto(854.68146877,278.31503174)(854.64646881,278.21003185)(854.60647567,278.11003328)
\curveto(854.56646889,278.01003205)(854.52146893,277.92003214)(854.47147567,277.84003328)
\curveto(854.31146914,277.58003248)(854.10646935,277.37003269)(853.85647567,277.21003328)
\curveto(853.60646985,277.060033)(853.32647013,276.93003313)(853.01647567,276.82003328)
\curveto(852.92647053,276.79003327)(852.83147062,276.77003329)(852.73147567,276.76003328)
\curveto(852.64147081,276.74003332)(852.5514709,276.71503334)(852.46147567,276.68503328)
\curveto(852.36147109,276.66503339)(852.26147119,276.6550334)(852.16147567,276.65503328)
\curveto(852.06147139,276.6550334)(851.96147149,276.64503341)(851.86147567,276.62503328)
\lineto(851.71147567,276.62503328)
\curveto(851.66147179,276.61503344)(851.59147186,276.61003345)(851.50147567,276.61003328)
\curveto(851.41147204,276.61003345)(851.34147211,276.61503344)(851.29147567,276.62503328)
\lineto(851.12647567,276.62503328)
\curveto(851.06647239,276.64503341)(851.00147245,276.6550334)(850.93147567,276.65503328)
\curveto(850.86147259,276.64503341)(850.80147265,276.65003341)(850.75147567,276.67003328)
\curveto(850.70147275,276.68003338)(850.63647282,276.68503337)(850.55647567,276.68503328)
\lineto(850.31647567,276.74503328)
\curveto(850.24647321,276.7550333)(850.17147328,276.77503328)(850.09147567,276.80503328)
\curveto(849.78147367,276.90503315)(849.51147394,277.03003303)(849.28147567,277.18003328)
\curveto(849.0514744,277.33003273)(848.8514746,277.52503253)(848.68147567,277.76503328)
\curveto(848.59147486,277.89503216)(848.51647494,278.03003203)(848.45647567,278.17003328)
\curveto(848.39647506,278.31003175)(848.34147511,278.46503159)(848.29147567,278.63503328)
\curveto(848.27147518,278.69503136)(848.26147519,278.76503129)(848.26147567,278.84503328)
\curveto(848.27147518,278.93503112)(848.28647517,279.00503105)(848.30647567,279.05503328)
\curveto(848.33647512,279.09503096)(848.38647507,279.13503092)(848.45647567,279.17503328)
\curveto(848.50647495,279.19503086)(848.57647488,279.20503085)(848.66647567,279.20503328)
\curveto(848.7564747,279.21503084)(848.84647461,279.21503084)(848.93647567,279.20503328)
\curveto(849.02647443,279.19503086)(849.11147434,279.18003088)(849.19147567,279.16003328)
\curveto(849.28147417,279.15003091)(849.34147411,279.13503092)(849.37147567,279.11503328)
\curveto(849.44147401,279.06503099)(849.48647397,278.99003107)(849.50647567,278.89003328)
\curveto(849.53647392,278.80003126)(849.57147388,278.71503134)(849.61147567,278.63503328)
\curveto(849.71147374,278.41503164)(849.84647361,278.24503181)(850.01647567,278.12503328)
\curveto(850.13647332,278.03503202)(850.27147318,277.96503209)(850.42147567,277.91503328)
\curveto(850.57147288,277.86503219)(850.73147272,277.81503224)(850.90147567,277.76503328)
\lineto(851.21647567,277.72003328)
\lineto(851.30647567,277.72003328)
\curveto(851.37647208,277.70003236)(851.46647199,277.69003237)(851.57647567,277.69003328)
\curveto(851.69647176,277.69003237)(851.79647166,277.70003236)(851.87647567,277.72003328)
\curveto(851.94647151,277.72003234)(852.00147145,277.72503233)(852.04147567,277.73503328)
\curveto(852.10147135,277.74503231)(852.16147129,277.75003231)(852.22147567,277.75003328)
\curveto(852.28147117,277.7600323)(852.33647112,277.77003229)(852.38647567,277.78003328)
\curveto(852.67647078,277.8600322)(852.90647055,277.96503209)(853.07647567,278.09503328)
\curveto(853.24647021,278.22503183)(853.36647009,278.44503161)(853.43647567,278.75503328)
\curveto(853.45647,278.80503125)(853.46146999,278.8600312)(853.45147567,278.92003328)
\curveto(853.44147001,278.98003108)(853.43147002,279.02503103)(853.42147567,279.05503328)
\curveto(853.37147008,279.24503081)(853.30147015,279.38503067)(853.21147567,279.47503328)
\curveto(853.12147033,279.57503048)(853.00647045,279.66503039)(852.86647567,279.74503328)
\curveto(852.77647068,279.80503025)(852.67647078,279.8550302)(852.56647567,279.89503328)
\lineto(852.23647567,280.01503328)
\curveto(852.20647125,280.02503003)(852.17647128,280.03003003)(852.14647567,280.03003328)
\curveto(852.12647133,280.03003003)(852.10147135,280.04003002)(852.07147567,280.06003328)
\curveto(851.73147172,280.17002989)(851.37647208,280.25002981)(851.00647567,280.30003328)
\curveto(850.64647281,280.3600297)(850.30647315,280.4550296)(849.98647567,280.58503328)
\curveto(849.88647357,280.62502943)(849.79147366,280.6600294)(849.70147567,280.69003328)
\curveto(849.61147384,280.72002934)(849.52647393,280.7600293)(849.44647567,280.81003328)
\curveto(849.2564742,280.92002914)(849.08147437,281.04502901)(848.92147567,281.18503328)
\curveto(848.76147469,281.32502873)(848.63647482,281.50002856)(848.54647567,281.71003328)
\curveto(848.51647494,281.78002828)(848.49147496,281.85002821)(848.47147567,281.92003328)
\curveto(848.46147499,281.99002807)(848.44647501,282.06502799)(848.42647567,282.14503328)
\curveto(848.39647506,282.26502779)(848.38647507,282.40002766)(848.39647567,282.55003328)
\curveto(848.40647505,282.71002735)(848.42147503,282.84502721)(848.44147567,282.95503328)
\curveto(848.46147499,283.00502705)(848.47147498,283.04502701)(848.47147567,283.07503328)
\curveto(848.48147497,283.11502694)(848.49647496,283.1550269)(848.51647567,283.19503328)
\curveto(848.60647485,283.42502663)(848.72647473,283.62502643)(848.87647567,283.79503328)
\curveto(849.03647442,283.96502609)(849.21647424,284.11502594)(849.41647567,284.24503328)
\curveto(849.56647389,284.33502572)(849.73147372,284.40502565)(849.91147567,284.45503328)
\curveto(850.09147336,284.51502554)(850.28147317,284.57002549)(850.48147567,284.62003328)
\curveto(850.5514729,284.63002543)(850.61647284,284.64002542)(850.67647567,284.65003328)
\curveto(850.74647271,284.6600254)(850.82147263,284.67002539)(850.90147567,284.68003328)
\curveto(850.93147252,284.69002537)(850.97147248,284.69002537)(851.02147567,284.68003328)
\curveto(851.07147238,284.67002539)(851.10647235,284.67502538)(851.12647567,284.69503328)
}
}
{
\newrgbcolor{curcolor}{0.7019608 0.7019608 0.7019608}
\pscustom[linestyle=none,fillstyle=solid,fillcolor=curcolor]
{
\newpath
\moveto(798.51865829,287.5000699)
\lineto(813.51865829,287.5000699)
\lineto(813.51865829,272.5000699)
\lineto(798.51865829,272.5000699)
\closepath
}
}
{
\newrgbcolor{curcolor}{0 0 0}
\pscustom[linestyle=none,fillstyle=solid,fillcolor=curcolor]
{
\newpath
\moveto(818.5668663,264.43426668)
\lineto(819.4818663,264.43426668)
\curveto(819.58186365,264.43425598)(819.67686355,264.43425598)(819.7668663,264.43426668)
\curveto(819.85686337,264.43425598)(819.9318633,264.414256)(819.9918663,264.37426668)
\curveto(820.08186315,264.3142561)(820.14186309,264.23425618)(820.1718663,264.13426668)
\curveto(820.21186302,264.03425638)(820.25686297,263.92925649)(820.3068663,263.81926668)
\curveto(820.38686284,263.62925679)(820.45686277,263.43925698)(820.5168663,263.24926668)
\curveto(820.58686264,263.05925736)(820.66186257,262.86925755)(820.7418663,262.67926668)
\curveto(820.81186242,262.49925792)(820.87686235,262.3142581)(820.9368663,262.12426668)
\curveto(820.99686223,261.94425847)(821.06686216,261.76425865)(821.1468663,261.58426668)
\curveto(821.20686202,261.44425897)(821.26186197,261.29925912)(821.3118663,261.14926668)
\curveto(821.36186187,260.99925942)(821.41686181,260.85425956)(821.4768663,260.71426668)
\curveto(821.65686157,260.26426015)(821.8268614,259.80926061)(821.9868663,259.34926668)
\curveto(822.14686108,258.89926152)(822.31686091,258.44926197)(822.4968663,257.99926668)
\curveto(822.51686071,257.94926247)(822.5318607,257.89926252)(822.5418663,257.84926668)
\lineto(822.6018663,257.69926668)
\curveto(822.69186054,257.47926294)(822.77686045,257.25426316)(822.8568663,257.02426668)
\curveto(822.93686029,256.80426361)(823.02186021,256.58426383)(823.1118663,256.36426668)
\curveto(823.15186008,256.27426414)(823.19186004,256.16426425)(823.2318663,256.03426668)
\curveto(823.27185996,255.9142645)(823.33685989,255.84426457)(823.4268663,255.82426668)
\curveto(823.46685976,255.8142646)(823.49685973,255.8142646)(823.5168663,255.82426668)
\lineto(823.5768663,255.88426668)
\curveto(823.6268596,255.93426448)(823.66185957,255.98926443)(823.6818663,256.04926668)
\curveto(823.71185952,256.10926431)(823.74185949,256.17426424)(823.7718663,256.24426668)
\lineto(824.0118663,256.87426668)
\curveto(824.09185914,257.09426332)(824.17185906,257.30926311)(824.2518663,257.51926668)
\lineto(824.3118663,257.66926668)
\lineto(824.3718663,257.84926668)
\curveto(824.45185878,258.03926238)(824.52185871,258.22926219)(824.5818663,258.41926668)
\curveto(824.65185858,258.6192618)(824.7268585,258.8192616)(824.8068663,259.01926668)
\curveto(825.04685818,259.59926082)(825.26685796,260.18426023)(825.4668663,260.77426668)
\curveto(825.67685755,261.36425905)(825.90185733,261.94925847)(826.1418663,262.52926668)
\curveto(826.22185701,262.72925769)(826.29685693,262.93425748)(826.3668663,263.14426668)
\curveto(826.44685678,263.35425706)(826.5268567,263.55925686)(826.6068663,263.75926668)
\curveto(826.64685658,263.83925658)(826.68185655,263.93925648)(826.7118663,264.05926668)
\curveto(826.75185648,264.17925624)(826.80685642,264.26425615)(826.8768663,264.31426668)
\curveto(826.93685629,264.35425606)(827.01185622,264.38425603)(827.1018663,264.40426668)
\curveto(827.20185603,264.42425599)(827.31185592,264.43425598)(827.4318663,264.43426668)
\curveto(827.55185568,264.44425597)(827.67185556,264.44425597)(827.7918663,264.43426668)
\curveto(827.91185532,264.43425598)(828.02185521,264.43425598)(828.1218663,264.43426668)
\curveto(828.21185502,264.43425598)(828.30185493,264.43425598)(828.3918663,264.43426668)
\curveto(828.49185474,264.43425598)(828.56685466,264.414256)(828.6168663,264.37426668)
\curveto(828.70685452,264.32425609)(828.75685447,264.23425618)(828.7668663,264.10426668)
\curveto(828.77685445,263.97425644)(828.78185445,263.83425658)(828.7818663,263.68426668)
\lineto(828.7818663,262.03426668)
\lineto(828.7818663,255.76426668)
\lineto(828.7818663,254.50426668)
\curveto(828.78185445,254.39426602)(828.78185445,254.28426613)(828.7818663,254.17426668)
\curveto(828.79185444,254.06426635)(828.77185446,253.97926644)(828.7218663,253.91926668)
\curveto(828.69185454,253.85926656)(828.64685458,253.8192666)(828.5868663,253.79926668)
\curveto(828.5268547,253.78926663)(828.45685477,253.77426664)(828.3768663,253.75426668)
\lineto(828.1368663,253.75426668)
\lineto(827.7768663,253.75426668)
\curveto(827.66685556,253.76426665)(827.58685564,253.80926661)(827.5368663,253.88926668)
\curveto(827.51685571,253.9192665)(827.50185573,253.94926647)(827.4918663,253.97926668)
\curveto(827.49185574,254.0192664)(827.48185575,254.06426635)(827.4618663,254.11426668)
\lineto(827.4618663,254.27926668)
\curveto(827.45185578,254.33926608)(827.44685578,254.40926601)(827.4468663,254.48926668)
\curveto(827.45685577,254.56926585)(827.46185577,254.64426577)(827.4618663,254.71426668)
\lineto(827.4618663,255.55426668)
\lineto(827.4618663,259.97926668)
\curveto(827.46185577,260.22926019)(827.46185577,260.47925994)(827.4618663,260.72926668)
\curveto(827.46185577,260.98925943)(827.45685577,261.23925918)(827.4468663,261.47926668)
\curveto(827.44685578,261.57925884)(827.44185579,261.68925873)(827.4318663,261.80926668)
\curveto(827.42185581,261.92925849)(827.36685586,261.98925843)(827.2668663,261.98926668)
\lineto(827.2668663,261.97426668)
\curveto(827.19685603,261.95425846)(827.13685609,261.88925853)(827.0868663,261.77926668)
\curveto(827.04685618,261.66925875)(827.01185622,261.57425884)(826.9818663,261.49426668)
\curveto(826.91185632,261.32425909)(826.84685638,261.14925927)(826.7868663,260.96926668)
\curveto(826.7268565,260.79925962)(826.65685657,260.62925979)(826.5768663,260.45926668)
\curveto(826.55685667,260.40926001)(826.54185669,260.36426005)(826.5318663,260.32426668)
\curveto(826.52185671,260.28426013)(826.50685672,260.23926018)(826.4868663,260.18926668)
\curveto(826.40685682,260.00926041)(826.33685689,259.82426059)(826.2768663,259.63426668)
\curveto(826.226857,259.45426096)(826.16185707,259.27426114)(826.0818663,259.09426668)
\curveto(826.01185722,258.94426147)(825.95185728,258.78926163)(825.9018663,258.62926668)
\curveto(825.85185738,258.47926194)(825.79685743,258.32926209)(825.7368663,258.17926668)
\curveto(825.53685769,257.70926271)(825.35685787,257.23426318)(825.1968663,256.75426668)
\curveto(825.03685819,256.28426413)(824.86185837,255.8192646)(824.6718663,255.35926668)
\curveto(824.59185864,255.17926524)(824.52185871,254.99926542)(824.4618663,254.81926668)
\curveto(824.40185883,254.63926578)(824.33685889,254.45926596)(824.2668663,254.27926668)
\curveto(824.21685901,254.16926625)(824.16685906,254.06426635)(824.1168663,253.96426668)
\curveto(824.07685915,253.87426654)(823.99185924,253.80926661)(823.8618663,253.76926668)
\curveto(823.84185939,253.75926666)(823.81685941,253.75426666)(823.7868663,253.75426668)
\curveto(823.76685946,253.76426665)(823.74185949,253.76426665)(823.7118663,253.75426668)
\curveto(823.68185955,253.74426667)(823.64685958,253.73926668)(823.6068663,253.73926668)
\curveto(823.56685966,253.74926667)(823.5268597,253.75426666)(823.4868663,253.75426668)
\lineto(823.1868663,253.75426668)
\curveto(823.08686014,253.75426666)(823.00686022,253.77926664)(822.9468663,253.82926668)
\curveto(822.86686036,253.87926654)(822.80686042,253.94926647)(822.7668663,254.03926668)
\curveto(822.73686049,254.13926628)(822.69686053,254.23926618)(822.6468663,254.33926668)
\curveto(822.56686066,254.53926588)(822.48686074,254.74426567)(822.4068663,254.95426668)
\curveto(822.33686089,255.17426524)(822.26186097,255.38426503)(822.1818663,255.58426668)
\curveto(822.10186113,255.76426465)(822.0318612,255.94426447)(821.9718663,256.12426668)
\curveto(821.92186131,256.3142641)(821.85686137,256.49926392)(821.7768663,256.67926668)
\curveto(821.54686168,257.23926318)(821.3318619,257.80426261)(821.1318663,258.37426668)
\curveto(820.9318623,258.94426147)(820.71686251,259.50926091)(820.4868663,260.06926668)
\lineto(820.2468663,260.69926668)
\curveto(820.17686305,260.9192595)(820.10186313,261.12925929)(820.0218663,261.32926668)
\curveto(819.97186326,261.43925898)(819.9268633,261.54425887)(819.8868663,261.64426668)
\curveto(819.85686337,261.75425866)(819.80686342,261.84925857)(819.7368663,261.92926668)
\curveto(819.7268635,261.94925847)(819.71686351,261.95925846)(819.7068663,261.95926668)
\lineto(819.6768663,261.98926668)
\lineto(819.6018663,261.98926668)
\lineto(819.5718663,261.95926668)
\curveto(819.56186367,261.95925846)(819.55186368,261.95425846)(819.5418663,261.94426668)
\curveto(819.52186371,261.89425852)(819.51186372,261.83925858)(819.5118663,261.77926668)
\curveto(819.51186372,261.7192587)(819.50186373,261.65925876)(819.4818663,261.59926668)
\lineto(819.4818663,261.43426668)
\curveto(819.46186377,261.37425904)(819.45686377,261.30925911)(819.4668663,261.23926668)
\curveto(819.47686375,261.16925925)(819.48186375,261.09925932)(819.4818663,261.02926668)
\lineto(819.4818663,260.21926668)
\lineto(819.4818663,255.65926668)
\lineto(819.4818663,254.47426668)
\curveto(819.48186375,254.36426605)(819.47686375,254.25426616)(819.4668663,254.14426668)
\curveto(819.46686376,254.03426638)(819.44186379,253.94926647)(819.3918663,253.88926668)
\curveto(819.34186389,253.80926661)(819.25186398,253.76426665)(819.1218663,253.75426668)
\lineto(818.7318663,253.75426668)
\lineto(818.5368663,253.75426668)
\curveto(818.48686474,253.75426666)(818.43686479,253.76426665)(818.3868663,253.78426668)
\curveto(818.25686497,253.82426659)(818.18186505,253.90926651)(818.1618663,254.03926668)
\curveto(818.15186508,254.16926625)(818.14686508,254.3192661)(818.1468663,254.48926668)
\lineto(818.1468663,256.22926668)
\lineto(818.1468663,262.22926668)
\lineto(818.1468663,263.63926668)
\curveto(818.14686508,263.74925667)(818.14186509,263.86425655)(818.1318663,263.98426668)
\curveto(818.1318651,264.10425631)(818.15686507,264.19925622)(818.2068663,264.26926668)
\curveto(818.24686498,264.32925609)(818.32186491,264.37925604)(818.4318663,264.41926668)
\curveto(818.45186478,264.42925599)(818.47186476,264.42925599)(818.4918663,264.41926668)
\curveto(818.52186471,264.419256)(818.54686468,264.42425599)(818.5668663,264.43426668)
}
}
{
\newrgbcolor{curcolor}{0 0 0}
\pscustom[linestyle=none,fillstyle=solid,fillcolor=curcolor]
{
\newpath
\moveto(837.76897567,254.30926668)
\curveto(837.79896784,254.14926627)(837.78396786,254.0142664)(837.72397567,253.90426668)
\curveto(837.66396798,253.80426661)(837.58396806,253.72926669)(837.48397567,253.67926668)
\curveto(837.43396821,253.65926676)(837.37896826,253.64926677)(837.31897567,253.64926668)
\curveto(837.26896837,253.64926677)(837.21396843,253.63926678)(837.15397567,253.61926668)
\curveto(836.93396871,253.56926685)(836.71396893,253.58426683)(836.49397567,253.66426668)
\curveto(836.28396936,253.73426668)(836.1389695,253.82426659)(836.05897567,253.93426668)
\curveto(836.00896963,254.00426641)(835.96396968,254.08426633)(835.92397567,254.17426668)
\curveto(835.88396976,254.27426614)(835.83396981,254.35426606)(835.77397567,254.41426668)
\curveto(835.75396989,254.43426598)(835.72896991,254.45426596)(835.69897567,254.47426668)
\curveto(835.67896996,254.49426592)(835.64896999,254.49926592)(835.60897567,254.48926668)
\curveto(835.49897014,254.45926596)(835.39397025,254.40426601)(835.29397567,254.32426668)
\curveto(835.20397044,254.24426617)(835.11397053,254.17426624)(835.02397567,254.11426668)
\curveto(834.89397075,254.03426638)(834.75397089,253.95926646)(834.60397567,253.88926668)
\curveto(834.45397119,253.82926659)(834.29397135,253.77426664)(834.12397567,253.72426668)
\curveto(834.02397162,253.69426672)(833.91397173,253.67426674)(833.79397567,253.66426668)
\curveto(833.68397196,253.65426676)(833.57397207,253.63926678)(833.46397567,253.61926668)
\curveto(833.41397223,253.60926681)(833.36897227,253.60426681)(833.32897567,253.60426668)
\lineto(833.22397567,253.60426668)
\curveto(833.11397253,253.58426683)(833.00897263,253.58426683)(832.90897567,253.60426668)
\lineto(832.77397567,253.60426668)
\curveto(832.72397292,253.6142668)(832.67397297,253.6192668)(832.62397567,253.61926668)
\curveto(832.57397307,253.6192668)(832.52897311,253.62926679)(832.48897567,253.64926668)
\curveto(832.44897319,253.65926676)(832.41397323,253.66426675)(832.38397567,253.66426668)
\curveto(832.36397328,253.65426676)(832.3389733,253.65426676)(832.30897567,253.66426668)
\lineto(832.06897567,253.72426668)
\curveto(831.98897365,253.73426668)(831.91397373,253.75426666)(831.84397567,253.78426668)
\curveto(831.5439741,253.9142665)(831.29897434,254.05926636)(831.10897567,254.21926668)
\curveto(830.92897471,254.38926603)(830.77897486,254.62426579)(830.65897567,254.92426668)
\curveto(830.56897507,255.14426527)(830.52397512,255.40926501)(830.52397567,255.71926668)
\lineto(830.52397567,256.03426668)
\curveto(830.53397511,256.08426433)(830.5389751,256.13426428)(830.53897567,256.18426668)
\lineto(830.56897567,256.36426668)
\lineto(830.68897567,256.69426668)
\curveto(830.72897491,256.80426361)(830.77897486,256.90426351)(830.83897567,256.99426668)
\curveto(831.01897462,257.28426313)(831.26397438,257.49926292)(831.57397567,257.63926668)
\curveto(831.88397376,257.77926264)(832.22397342,257.90426251)(832.59397567,258.01426668)
\curveto(832.73397291,258.05426236)(832.87897276,258.08426233)(833.02897567,258.10426668)
\curveto(833.17897246,258.12426229)(833.32897231,258.14926227)(833.47897567,258.17926668)
\curveto(833.54897209,258.19926222)(833.61397203,258.20926221)(833.67397567,258.20926668)
\curveto(833.7439719,258.20926221)(833.81897182,258.2192622)(833.89897567,258.23926668)
\curveto(833.96897167,258.25926216)(834.0389716,258.26926215)(834.10897567,258.26926668)
\curveto(834.17897146,258.27926214)(834.25397139,258.29426212)(834.33397567,258.31426668)
\curveto(834.58397106,258.37426204)(834.81897082,258.42426199)(835.03897567,258.46426668)
\curveto(835.25897038,258.5142619)(835.43397021,258.62926179)(835.56397567,258.80926668)
\curveto(835.62397002,258.88926153)(835.67396997,258.98926143)(835.71397567,259.10926668)
\curveto(835.75396989,259.23926118)(835.75396989,259.37926104)(835.71397567,259.52926668)
\curveto(835.65396999,259.76926065)(835.56397008,259.95926046)(835.44397567,260.09926668)
\curveto(835.33397031,260.23926018)(835.17397047,260.34926007)(834.96397567,260.42926668)
\curveto(834.8439708,260.47925994)(834.69897094,260.5142599)(834.52897567,260.53426668)
\curveto(834.36897127,260.55425986)(834.19897144,260.56425985)(834.01897567,260.56426668)
\curveto(833.8389718,260.56425985)(833.66397198,260.55425986)(833.49397567,260.53426668)
\curveto(833.32397232,260.5142599)(833.17897246,260.48425993)(833.05897567,260.44426668)
\curveto(832.88897275,260.38426003)(832.72397292,260.29926012)(832.56397567,260.18926668)
\curveto(832.48397316,260.12926029)(832.40897323,260.04926037)(832.33897567,259.94926668)
\curveto(832.27897336,259.85926056)(832.22397342,259.75926066)(832.17397567,259.64926668)
\curveto(832.1439735,259.56926085)(832.11397353,259.48426093)(832.08397567,259.39426668)
\curveto(832.06397358,259.30426111)(832.01897362,259.23426118)(831.94897567,259.18426668)
\curveto(831.90897373,259.15426126)(831.8389738,259.12926129)(831.73897567,259.10926668)
\curveto(831.64897399,259.09926132)(831.55397409,259.09426132)(831.45397567,259.09426668)
\curveto(831.35397429,259.09426132)(831.25397439,259.09926132)(831.15397567,259.10926668)
\curveto(831.06397458,259.12926129)(830.99897464,259.15426126)(830.95897567,259.18426668)
\curveto(830.91897472,259.2142612)(830.88897475,259.26426115)(830.86897567,259.33426668)
\curveto(830.84897479,259.40426101)(830.84897479,259.47926094)(830.86897567,259.55926668)
\curveto(830.89897474,259.68926073)(830.92897471,259.80926061)(830.95897567,259.91926668)
\curveto(830.99897464,260.03926038)(831.0439746,260.15426026)(831.09397567,260.26426668)
\curveto(831.28397436,260.6142598)(831.52397412,260.88425953)(831.81397567,261.07426668)
\curveto(832.10397354,261.27425914)(832.46397318,261.43425898)(832.89397567,261.55426668)
\curveto(832.99397265,261.57425884)(833.09397255,261.58925883)(833.19397567,261.59926668)
\curveto(833.30397234,261.60925881)(833.41397223,261.62425879)(833.52397567,261.64426668)
\curveto(833.56397208,261.65425876)(833.62897201,261.65425876)(833.71897567,261.64426668)
\curveto(833.80897183,261.64425877)(833.86397178,261.65425876)(833.88397567,261.67426668)
\curveto(834.58397106,261.68425873)(835.19397045,261.60425881)(835.71397567,261.43426668)
\curveto(836.23396941,261.26425915)(836.59896904,260.93925948)(836.80897567,260.45926668)
\curveto(836.89896874,260.25926016)(836.94896869,260.02426039)(836.95897567,259.75426668)
\curveto(836.97896866,259.49426092)(836.98896865,259.2192612)(836.98897567,258.92926668)
\lineto(836.98897567,255.61426668)
\curveto(836.98896865,255.47426494)(836.99396865,255.33926508)(837.00397567,255.20926668)
\curveto(837.01396863,255.07926534)(837.0439686,254.97426544)(837.09397567,254.89426668)
\curveto(837.1439685,254.82426559)(837.20896843,254.77426564)(837.28897567,254.74426668)
\curveto(837.37896826,254.70426571)(837.46396818,254.67426574)(837.54397567,254.65426668)
\curveto(837.62396802,254.64426577)(837.68396796,254.59926582)(837.72397567,254.51926668)
\curveto(837.7439679,254.48926593)(837.75396789,254.45926596)(837.75397567,254.42926668)
\curveto(837.75396789,254.39926602)(837.75896788,254.35926606)(837.76897567,254.30926668)
\moveto(835.62397567,255.97426668)
\curveto(835.68396996,256.1142643)(835.71396993,256.27426414)(835.71397567,256.45426668)
\curveto(835.72396992,256.64426377)(835.72896991,256.83926358)(835.72897567,257.03926668)
\curveto(835.72896991,257.14926327)(835.72396992,257.24926317)(835.71397567,257.33926668)
\curveto(835.70396994,257.42926299)(835.66396998,257.49926292)(835.59397567,257.54926668)
\curveto(835.56397008,257.56926285)(835.49397015,257.57926284)(835.38397567,257.57926668)
\curveto(835.36397028,257.55926286)(835.32897031,257.54926287)(835.27897567,257.54926668)
\curveto(835.22897041,257.54926287)(835.18397046,257.53926288)(835.14397567,257.51926668)
\curveto(835.06397058,257.49926292)(834.97397067,257.47926294)(834.87397567,257.45926668)
\lineto(834.57397567,257.39926668)
\curveto(834.5439711,257.39926302)(834.50897113,257.39426302)(834.46897567,257.38426668)
\lineto(834.36397567,257.38426668)
\curveto(834.21397143,257.34426307)(834.04897159,257.3192631)(833.86897567,257.30926668)
\curveto(833.69897194,257.30926311)(833.5389721,257.28926313)(833.38897567,257.24926668)
\curveto(833.30897233,257.22926319)(833.23397241,257.20926321)(833.16397567,257.18926668)
\curveto(833.10397254,257.17926324)(833.03397261,257.16426325)(832.95397567,257.14426668)
\curveto(832.79397285,257.09426332)(832.643973,257.02926339)(832.50397567,256.94926668)
\curveto(832.36397328,256.87926354)(832.2439734,256.78926363)(832.14397567,256.67926668)
\curveto(832.0439736,256.56926385)(831.96897367,256.43426398)(831.91897567,256.27426668)
\curveto(831.86897377,256.12426429)(831.84897379,255.93926448)(831.85897567,255.71926668)
\curveto(831.85897378,255.6192648)(831.87397377,255.52426489)(831.90397567,255.43426668)
\curveto(831.9439737,255.35426506)(831.98897365,255.27926514)(832.03897567,255.20926668)
\curveto(832.11897352,255.09926532)(832.22397342,255.00426541)(832.35397567,254.92426668)
\curveto(832.48397316,254.85426556)(832.62397302,254.79426562)(832.77397567,254.74426668)
\curveto(832.82397282,254.73426568)(832.87397277,254.72926569)(832.92397567,254.72926668)
\curveto(832.97397267,254.72926569)(833.02397262,254.72426569)(833.07397567,254.71426668)
\curveto(833.1439725,254.69426572)(833.22897241,254.67926574)(833.32897567,254.66926668)
\curveto(833.4389722,254.66926575)(833.52897211,254.67926574)(833.59897567,254.69926668)
\curveto(833.65897198,254.7192657)(833.71897192,254.72426569)(833.77897567,254.71426668)
\curveto(833.8389718,254.7142657)(833.89897174,254.72426569)(833.95897567,254.74426668)
\curveto(834.0389716,254.76426565)(834.11397153,254.77926564)(834.18397567,254.78926668)
\curveto(834.26397138,254.79926562)(834.3389713,254.8192656)(834.40897567,254.84926668)
\curveto(834.69897094,254.96926545)(834.9439707,255.1142653)(835.14397567,255.28426668)
\curveto(835.35397029,255.45426496)(835.51397013,255.68426473)(835.62397567,255.97426668)
}
}
{
\newrgbcolor{curcolor}{0 0 0}
\pscustom[linestyle=none,fillstyle=solid,fillcolor=curcolor]
{
\newpath
\moveto(839.8856163,263.81926668)
\curveto(840.03561429,263.8192566)(840.18561414,263.8142566)(840.3356163,263.80426668)
\curveto(840.48561384,263.80425661)(840.59061373,263.76425665)(840.6506163,263.68426668)
\curveto(840.70061362,263.62425679)(840.7256136,263.53925688)(840.7256163,263.42926668)
\curveto(840.73561359,263.32925709)(840.74061358,263.22425719)(840.7406163,263.11426668)
\lineto(840.7406163,262.24426668)
\curveto(840.74061358,262.16425825)(840.73561359,262.07925834)(840.7256163,261.98926668)
\curveto(840.7256136,261.90925851)(840.73561359,261.83925858)(840.7556163,261.77926668)
\curveto(840.79561353,261.63925878)(840.88561344,261.54925887)(841.0256163,261.50926668)
\curveto(841.07561325,261.49925892)(841.1206132,261.49425892)(841.1606163,261.49426668)
\lineto(841.3106163,261.49426668)
\lineto(841.7156163,261.49426668)
\curveto(841.87561245,261.50425891)(841.99061233,261.49425892)(842.0606163,261.46426668)
\curveto(842.15061217,261.40425901)(842.21061211,261.34425907)(842.2406163,261.28426668)
\curveto(842.26061206,261.24425917)(842.27061205,261.19925922)(842.2706163,261.14926668)
\lineto(842.2706163,260.99926668)
\curveto(842.27061205,260.88925953)(842.26561206,260.78425963)(842.2556163,260.68426668)
\curveto(842.24561208,260.59425982)(842.21061211,260.52425989)(842.1506163,260.47426668)
\curveto(842.09061223,260.42425999)(842.00561232,260.39426002)(841.8956163,260.38426668)
\lineto(841.5656163,260.38426668)
\curveto(841.45561287,260.39426002)(841.34561298,260.39926002)(841.2356163,260.39926668)
\curveto(841.1256132,260.39926002)(841.03061329,260.38426003)(840.9506163,260.35426668)
\curveto(840.88061344,260.32426009)(840.83061349,260.27426014)(840.8006163,260.20426668)
\curveto(840.77061355,260.13426028)(840.75061357,260.04926037)(840.7406163,259.94926668)
\curveto(840.73061359,259.85926056)(840.7256136,259.75926066)(840.7256163,259.64926668)
\curveto(840.73561359,259.54926087)(840.74061358,259.44926097)(840.7406163,259.34926668)
\lineto(840.7406163,256.37926668)
\curveto(840.74061358,256.15926426)(840.73561359,255.92426449)(840.7256163,255.67426668)
\curveto(840.7256136,255.43426498)(840.77061355,255.24926517)(840.8606163,255.11926668)
\curveto(840.91061341,255.03926538)(840.97561335,254.98426543)(841.0556163,254.95426668)
\curveto(841.13561319,254.92426549)(841.23061309,254.89926552)(841.3406163,254.87926668)
\curveto(841.37061295,254.86926555)(841.40061292,254.86426555)(841.4306163,254.86426668)
\curveto(841.47061285,254.87426554)(841.50561282,254.87426554)(841.5356163,254.86426668)
\lineto(841.7306163,254.86426668)
\curveto(841.83061249,254.86426555)(841.9206124,254.85426556)(842.0006163,254.83426668)
\curveto(842.09061223,254.82426559)(842.15561217,254.78926563)(842.1956163,254.72926668)
\curveto(842.21561211,254.69926572)(842.23061209,254.64426577)(842.2406163,254.56426668)
\curveto(842.26061206,254.49426592)(842.27061205,254.419266)(842.2706163,254.33926668)
\curveto(842.28061204,254.25926616)(842.28061204,254.17926624)(842.2706163,254.09926668)
\curveto(842.26061206,254.02926639)(842.24061208,253.97426644)(842.2106163,253.93426668)
\curveto(842.17061215,253.86426655)(842.09561223,253.8142666)(841.9856163,253.78426668)
\curveto(841.90561242,253.76426665)(841.81561251,253.75426666)(841.7156163,253.75426668)
\curveto(841.61561271,253.76426665)(841.5256128,253.76926665)(841.4456163,253.76926668)
\curveto(841.38561294,253.76926665)(841.325613,253.76426665)(841.2656163,253.75426668)
\curveto(841.20561312,253.75426666)(841.15061317,253.75926666)(841.1006163,253.76926668)
\lineto(840.9206163,253.76926668)
\curveto(840.87061345,253.77926664)(840.8206135,253.78426663)(840.7706163,253.78426668)
\curveto(840.73061359,253.79426662)(840.68561364,253.79926662)(840.6356163,253.79926668)
\curveto(840.43561389,253.84926657)(840.26061406,253.90426651)(840.1106163,253.96426668)
\curveto(839.97061435,254.02426639)(839.85061447,254.12926629)(839.7506163,254.27926668)
\curveto(839.61061471,254.47926594)(839.53061479,254.72926569)(839.5106163,255.02926668)
\curveto(839.49061483,255.33926508)(839.48061484,255.66926475)(839.4806163,256.01926668)
\lineto(839.4806163,259.94926668)
\curveto(839.45061487,260.07926034)(839.4206149,260.17426024)(839.3906163,260.23426668)
\curveto(839.37061495,260.29426012)(839.30061502,260.34426007)(839.1806163,260.38426668)
\curveto(839.14061518,260.39426002)(839.10061522,260.39426002)(839.0606163,260.38426668)
\curveto(839.0206153,260.37426004)(838.98061534,260.37926004)(838.9406163,260.39926668)
\lineto(838.7006163,260.39926668)
\curveto(838.57061575,260.39926002)(838.46061586,260.40926001)(838.3706163,260.42926668)
\curveto(838.29061603,260.45925996)(838.23561609,260.5192599)(838.2056163,260.60926668)
\curveto(838.18561614,260.64925977)(838.17061615,260.69425972)(838.1606163,260.74426668)
\lineto(838.1606163,260.89426668)
\curveto(838.16061616,261.03425938)(838.17061615,261.14925927)(838.1906163,261.23926668)
\curveto(838.21061611,261.33925908)(838.27061605,261.414259)(838.3706163,261.46426668)
\curveto(838.48061584,261.50425891)(838.6206157,261.5142589)(838.7906163,261.49426668)
\curveto(838.97061535,261.47425894)(839.1206152,261.48425893)(839.2406163,261.52426668)
\curveto(839.33061499,261.57425884)(839.40061492,261.64425877)(839.4506163,261.73426668)
\curveto(839.47061485,261.79425862)(839.48061484,261.86925855)(839.4806163,261.95926668)
\lineto(839.4806163,262.21426668)
\lineto(839.4806163,263.14426668)
\lineto(839.4806163,263.38426668)
\curveto(839.48061484,263.47425694)(839.49061483,263.54925687)(839.5106163,263.60926668)
\curveto(839.55061477,263.68925673)(839.6256147,263.75425666)(839.7356163,263.80426668)
\curveto(839.76561456,263.80425661)(839.79061453,263.80425661)(839.8106163,263.80426668)
\curveto(839.84061448,263.8142566)(839.86561446,263.8192566)(839.8856163,263.81926668)
}
}
{
\newrgbcolor{curcolor}{0 0 0}
\pscustom[linestyle=none,fillstyle=solid,fillcolor=curcolor]
{
\newpath
\moveto(850.40741317,257.92426668)
\curveto(850.42740549,257.82426259)(850.42740549,257.70926271)(850.40741317,257.57926668)
\curveto(850.39740552,257.45926296)(850.36740555,257.37426304)(850.31741317,257.32426668)
\curveto(850.26740565,257.28426313)(850.19240572,257.25426316)(850.09241317,257.23426668)
\curveto(850.00240591,257.22426319)(849.89740602,257.2192632)(849.77741317,257.21926668)
\lineto(849.41741317,257.21926668)
\curveto(849.29740662,257.22926319)(849.19240672,257.23426318)(849.10241317,257.23426668)
\lineto(845.26241317,257.23426668)
\curveto(845.18241073,257.23426318)(845.10241081,257.22926319)(845.02241317,257.21926668)
\curveto(844.94241097,257.2192632)(844.87741104,257.20426321)(844.82741317,257.17426668)
\curveto(844.78741113,257.15426326)(844.74741117,257.1142633)(844.70741317,257.05426668)
\curveto(844.68741123,257.02426339)(844.66741125,256.97926344)(844.64741317,256.91926668)
\curveto(844.62741129,256.86926355)(844.62741129,256.8192636)(844.64741317,256.76926668)
\curveto(844.65741126,256.7192637)(844.66241125,256.67426374)(844.66241317,256.63426668)
\curveto(844.66241125,256.59426382)(844.66741125,256.55426386)(844.67741317,256.51426668)
\curveto(844.69741122,256.43426398)(844.7174112,256.34926407)(844.73741317,256.25926668)
\curveto(844.75741116,256.17926424)(844.78741113,256.09926432)(844.82741317,256.01926668)
\curveto(845.05741086,255.47926494)(845.43741048,255.09426532)(845.96741317,254.86426668)
\curveto(846.02740989,254.83426558)(846.09240982,254.80926561)(846.16241317,254.78926668)
\lineto(846.37241317,254.72926668)
\curveto(846.40240951,254.7192657)(846.45240946,254.7142657)(846.52241317,254.71426668)
\curveto(846.66240925,254.67426574)(846.84740907,254.65426576)(847.07741317,254.65426668)
\curveto(847.30740861,254.65426576)(847.49240842,254.67426574)(847.63241317,254.71426668)
\curveto(847.77240814,254.75426566)(847.89740802,254.79426562)(848.00741317,254.83426668)
\curveto(848.12740779,254.88426553)(848.23740768,254.94426547)(848.33741317,255.01426668)
\curveto(848.44740747,255.08426533)(848.54240737,255.16426525)(848.62241317,255.25426668)
\curveto(848.70240721,255.35426506)(848.77240714,255.45926496)(848.83241317,255.56926668)
\curveto(848.89240702,255.66926475)(848.94240697,255.77426464)(848.98241317,255.88426668)
\curveto(849.03240688,255.99426442)(849.1124068,256.07426434)(849.22241317,256.12426668)
\curveto(849.26240665,256.14426427)(849.32740659,256.15926426)(849.41741317,256.16926668)
\curveto(849.50740641,256.17926424)(849.59740632,256.17926424)(849.68741317,256.16926668)
\curveto(849.77740614,256.16926425)(849.86240605,256.16426425)(849.94241317,256.15426668)
\curveto(850.02240589,256.14426427)(850.07740584,256.12426429)(850.10741317,256.09426668)
\curveto(850.20740571,256.02426439)(850.23240568,255.90926451)(850.18241317,255.74926668)
\curveto(850.10240581,255.47926494)(849.99740592,255.23926518)(849.86741317,255.02926668)
\curveto(849.66740625,254.70926571)(849.43740648,254.44426597)(849.17741317,254.23426668)
\curveto(848.92740699,254.03426638)(848.60740731,253.86926655)(848.21741317,253.73926668)
\curveto(848.1174078,253.69926672)(848.0174079,253.67426674)(847.91741317,253.66426668)
\curveto(847.8174081,253.64426677)(847.7124082,253.62426679)(847.60241317,253.60426668)
\curveto(847.55240836,253.59426682)(847.50240841,253.58926683)(847.45241317,253.58926668)
\curveto(847.4124085,253.58926683)(847.36740855,253.58426683)(847.31741317,253.57426668)
\lineto(847.16741317,253.57426668)
\curveto(847.1174088,253.56426685)(847.05740886,253.55926686)(846.98741317,253.55926668)
\curveto(846.92740899,253.55926686)(846.87740904,253.56426685)(846.83741317,253.57426668)
\lineto(846.70241317,253.57426668)
\curveto(846.65240926,253.58426683)(846.60740931,253.58926683)(846.56741317,253.58926668)
\curveto(846.52740939,253.58926683)(846.48740943,253.59426682)(846.44741317,253.60426668)
\curveto(846.39740952,253.6142668)(846.34240957,253.62426679)(846.28241317,253.63426668)
\curveto(846.22240969,253.63426678)(846.16740975,253.63926678)(846.11741317,253.64926668)
\curveto(846.02740989,253.66926675)(845.93740998,253.69426672)(845.84741317,253.72426668)
\curveto(845.75741016,253.74426667)(845.67241024,253.76926665)(845.59241317,253.79926668)
\curveto(845.55241036,253.8192666)(845.5174104,253.82926659)(845.48741317,253.82926668)
\curveto(845.45741046,253.83926658)(845.42241049,253.85426656)(845.38241317,253.87426668)
\curveto(845.23241068,253.94426647)(845.07241084,254.02926639)(844.90241317,254.12926668)
\curveto(844.6124113,254.3192661)(844.36241155,254.54926587)(844.15241317,254.81926668)
\curveto(843.95241196,255.09926532)(843.78241213,255.40926501)(843.64241317,255.74926668)
\curveto(843.59241232,255.85926456)(843.55241236,255.97426444)(843.52241317,256.09426668)
\curveto(843.50241241,256.2142642)(843.47241244,256.33426408)(843.43241317,256.45426668)
\curveto(843.42241249,256.49426392)(843.4174125,256.52926389)(843.41741317,256.55926668)
\curveto(843.4174125,256.58926383)(843.4124125,256.62926379)(843.40241317,256.67926668)
\curveto(843.38241253,256.75926366)(843.36741255,256.84426357)(843.35741317,256.93426668)
\curveto(843.34741257,257.02426339)(843.33241258,257.1142633)(843.31241317,257.20426668)
\lineto(843.31241317,257.41426668)
\curveto(843.30241261,257.45426296)(843.29241262,257.50926291)(843.28241317,257.57926668)
\curveto(843.28241263,257.65926276)(843.28741263,257.72426269)(843.29741317,257.77426668)
\lineto(843.29741317,257.93926668)
\curveto(843.3174126,257.98926243)(843.32241259,258.03926238)(843.31241317,258.08926668)
\curveto(843.3124126,258.14926227)(843.3174126,258.20426221)(843.32741317,258.25426668)
\curveto(843.36741255,258.414262)(843.39741252,258.57426184)(843.41741317,258.73426668)
\curveto(843.44741247,258.89426152)(843.49241242,259.04426137)(843.55241317,259.18426668)
\curveto(843.60241231,259.29426112)(843.64741227,259.40426101)(843.68741317,259.51426668)
\curveto(843.73741218,259.63426078)(843.79241212,259.74926067)(843.85241317,259.85926668)
\curveto(844.07241184,260.20926021)(844.32241159,260.50925991)(844.60241317,260.75926668)
\curveto(844.88241103,261.0192594)(845.22741069,261.23425918)(845.63741317,261.40426668)
\curveto(845.75741016,261.45425896)(845.87741004,261.48925893)(845.99741317,261.50926668)
\curveto(846.12740979,261.53925888)(846.26240965,261.56925885)(846.40241317,261.59926668)
\curveto(846.45240946,261.60925881)(846.49740942,261.6142588)(846.53741317,261.61426668)
\curveto(846.57740934,261.62425879)(846.62240929,261.62925879)(846.67241317,261.62926668)
\curveto(846.69240922,261.63925878)(846.7174092,261.63925878)(846.74741317,261.62926668)
\curveto(846.77740914,261.6192588)(846.80240911,261.62425879)(846.82241317,261.64426668)
\curveto(847.24240867,261.65425876)(847.60740831,261.60925881)(847.91741317,261.50926668)
\curveto(848.22740769,261.419259)(848.50740741,261.29425912)(848.75741317,261.13426668)
\curveto(848.80740711,261.1142593)(848.84740707,261.08425933)(848.87741317,261.04426668)
\curveto(848.90740701,261.0142594)(848.94240697,260.98925943)(848.98241317,260.96926668)
\curveto(849.06240685,260.90925951)(849.14240677,260.83925958)(849.22241317,260.75926668)
\curveto(849.3124066,260.67925974)(849.38740653,260.59925982)(849.44741317,260.51926668)
\curveto(849.60740631,260.30926011)(849.74240617,260.10926031)(849.85241317,259.91926668)
\curveto(849.92240599,259.80926061)(849.97740594,259.68926073)(850.01741317,259.55926668)
\curveto(850.05740586,259.42926099)(850.10240581,259.29926112)(850.15241317,259.16926668)
\curveto(850.20240571,259.03926138)(850.23740568,258.90426151)(850.25741317,258.76426668)
\curveto(850.28740563,258.62426179)(850.32240559,258.48426193)(850.36241317,258.34426668)
\curveto(850.37240554,258.27426214)(850.37740554,258.20426221)(850.37741317,258.13426668)
\lineto(850.40741317,257.92426668)
\moveto(848.95241317,258.43426668)
\curveto(848.98240693,258.47426194)(849.00740691,258.52426189)(849.02741317,258.58426668)
\curveto(849.04740687,258.65426176)(849.04740687,258.72426169)(849.02741317,258.79426668)
\curveto(848.96740695,259.0142614)(848.88240703,259.2192612)(848.77241317,259.40926668)
\curveto(848.63240728,259.63926078)(848.47740744,259.83426058)(848.30741317,259.99426668)
\curveto(848.13740778,260.15426026)(847.917408,260.28926013)(847.64741317,260.39926668)
\curveto(847.57740834,260.41926)(847.50740841,260.43425998)(847.43741317,260.44426668)
\curveto(847.36740855,260.46425995)(847.29240862,260.48425993)(847.21241317,260.50426668)
\curveto(847.13240878,260.52425989)(847.04740887,260.53425988)(846.95741317,260.53426668)
\lineto(846.70241317,260.53426668)
\curveto(846.67240924,260.5142599)(846.63740928,260.50425991)(846.59741317,260.50426668)
\curveto(846.55740936,260.5142599)(846.52240939,260.5142599)(846.49241317,260.50426668)
\lineto(846.25241317,260.44426668)
\curveto(846.18240973,260.43425998)(846.1124098,260.41926)(846.04241317,260.39926668)
\curveto(845.75241016,260.27926014)(845.5174104,260.12926029)(845.33741317,259.94926668)
\curveto(845.16741075,259.76926065)(845.0124109,259.54426087)(844.87241317,259.27426668)
\curveto(844.84241107,259.22426119)(844.8124111,259.15926126)(844.78241317,259.07926668)
\curveto(844.75241116,259.00926141)(844.72741119,258.92926149)(844.70741317,258.83926668)
\curveto(844.68741123,258.74926167)(844.68241123,258.66426175)(844.69241317,258.58426668)
\curveto(844.70241121,258.50426191)(844.73741118,258.44426197)(844.79741317,258.40426668)
\curveto(844.87741104,258.34426207)(845.0124109,258.3142621)(845.20241317,258.31426668)
\curveto(845.40241051,258.32426209)(845.57241034,258.32926209)(845.71241317,258.32926668)
\lineto(847.99241317,258.32926668)
\curveto(848.14240777,258.32926209)(848.32240759,258.32426209)(848.53241317,258.31426668)
\curveto(848.74240717,258.3142621)(848.88240703,258.35426206)(848.95241317,258.43426668)
}
}
{
\newrgbcolor{curcolor}{0 0 0}
\pscustom[linestyle=none,fillstyle=solid,fillcolor=curcolor]
{
\newpath
\moveto(855.3590538,261.65926668)
\curveto(855.58904901,261.65925876)(855.71904888,261.59925882)(855.7490538,261.47926668)
\curveto(855.77904882,261.36925905)(855.7940488,261.20425921)(855.7940538,260.98426668)
\lineto(855.7940538,260.69926668)
\curveto(855.7940488,260.60925981)(855.76904883,260.53425988)(855.7190538,260.47426668)
\curveto(855.65904894,260.39426002)(855.57404902,260.34926007)(855.4640538,260.33926668)
\curveto(855.35404924,260.33926008)(855.24404935,260.32426009)(855.1340538,260.29426668)
\curveto(854.9940496,260.26426015)(854.85904974,260.23426018)(854.7290538,260.20426668)
\curveto(854.60904999,260.17426024)(854.4940501,260.13426028)(854.3840538,260.08426668)
\curveto(854.0940505,259.95426046)(853.85905074,259.77426064)(853.6790538,259.54426668)
\curveto(853.4990511,259.32426109)(853.34405125,259.06926135)(853.2140538,258.77926668)
\curveto(853.17405142,258.66926175)(853.14405145,258.55426186)(853.1240538,258.43426668)
\curveto(853.10405149,258.32426209)(853.07905152,258.20926221)(853.0490538,258.08926668)
\curveto(853.03905156,258.03926238)(853.03405156,257.98926243)(853.0340538,257.93926668)
\curveto(853.04405155,257.88926253)(853.04405155,257.83926258)(853.0340538,257.78926668)
\curveto(853.00405159,257.66926275)(852.98905161,257.52926289)(852.9890538,257.36926668)
\curveto(852.9990516,257.2192632)(853.00405159,257.07426334)(853.0040538,256.93426668)
\lineto(853.0040538,255.08926668)
\lineto(853.0040538,254.74426668)
\curveto(853.00405159,254.62426579)(852.9990516,254.50926591)(852.9890538,254.39926668)
\curveto(852.97905162,254.28926613)(852.97405162,254.19426622)(852.9740538,254.11426668)
\curveto(852.98405161,254.03426638)(852.96405163,253.96426645)(852.9140538,253.90426668)
\curveto(852.86405173,253.83426658)(852.78405181,253.79426662)(852.6740538,253.78426668)
\curveto(852.57405202,253.77426664)(852.46405213,253.76926665)(852.3440538,253.76926668)
\lineto(852.0740538,253.76926668)
\curveto(852.02405257,253.78926663)(851.97405262,253.80426661)(851.9240538,253.81426668)
\curveto(851.88405271,253.83426658)(851.85405274,253.85926656)(851.8340538,253.88926668)
\curveto(851.78405281,253.95926646)(851.75405284,254.04426637)(851.7440538,254.14426668)
\lineto(851.7440538,254.47426668)
\lineto(851.7440538,255.62926668)
\lineto(851.7440538,259.78426668)
\lineto(851.7440538,260.81926668)
\lineto(851.7440538,261.11926668)
\curveto(851.75405284,261.2192592)(851.78405281,261.30425911)(851.8340538,261.37426668)
\curveto(851.86405273,261.414259)(851.91405268,261.44425897)(851.9840538,261.46426668)
\curveto(852.06405253,261.48425893)(852.14905245,261.49425892)(852.2390538,261.49426668)
\curveto(852.32905227,261.50425891)(852.41905218,261.50425891)(852.5090538,261.49426668)
\curveto(852.599052,261.48425893)(852.66905193,261.46925895)(852.7190538,261.44926668)
\curveto(852.7990518,261.419259)(852.84905175,261.35925906)(852.8690538,261.26926668)
\curveto(852.8990517,261.18925923)(852.91405168,261.09925932)(852.9140538,260.99926668)
\lineto(852.9140538,260.69926668)
\curveto(852.91405168,260.59925982)(852.93405166,260.50925991)(852.9740538,260.42926668)
\curveto(852.98405161,260.40926001)(852.9940516,260.39426002)(853.0040538,260.38426668)
\lineto(853.0490538,260.33926668)
\curveto(853.15905144,260.33926008)(853.24905135,260.38426003)(853.3190538,260.47426668)
\curveto(853.38905121,260.57425984)(853.44905115,260.65425976)(853.4990538,260.71426668)
\lineto(853.5890538,260.80426668)
\curveto(853.67905092,260.9142595)(853.80405079,261.02925939)(853.9640538,261.14926668)
\curveto(854.12405047,261.26925915)(854.27405032,261.35925906)(854.4140538,261.41926668)
\curveto(854.50405009,261.46925895)(854.59905,261.50425891)(854.6990538,261.52426668)
\curveto(854.7990498,261.55425886)(854.90404969,261.58425883)(855.0140538,261.61426668)
\curveto(855.07404952,261.62425879)(855.13404946,261.62925879)(855.1940538,261.62926668)
\curveto(855.25404934,261.63925878)(855.30904929,261.64925877)(855.3590538,261.65926668)
}
}
{
\newrgbcolor{curcolor}{0 0 0}
\pscustom[linestyle=none,fillstyle=solid,fillcolor=curcolor]
{
\newpath
\moveto(857.00881942,262.97926668)
\curveto(856.9288183,263.03925738)(856.88381835,263.14425727)(856.87381942,263.29426668)
\lineto(856.87381942,263.75926668)
\lineto(856.87381942,264.01426668)
\curveto(856.87381836,264.10425631)(856.88881834,264.17925624)(856.91881942,264.23926668)
\curveto(856.95881827,264.3192561)(857.03881819,264.37925604)(857.15881942,264.41926668)
\curveto(857.17881805,264.42925599)(857.19881803,264.42925599)(857.21881942,264.41926668)
\curveto(857.24881798,264.419256)(857.27381796,264.42425599)(857.29381942,264.43426668)
\curveto(857.46381777,264.43425598)(857.62381761,264.42925599)(857.77381942,264.41926668)
\curveto(857.92381731,264.40925601)(858.02381721,264.34925607)(858.07381942,264.23926668)
\curveto(858.10381713,264.17925624)(858.11881711,264.10425631)(858.11881942,264.01426668)
\lineto(858.11881942,263.75926668)
\curveto(858.11881711,263.57925684)(858.11381712,263.40925701)(858.10381942,263.24926668)
\curveto(858.10381713,263.08925733)(858.03881719,262.98425743)(857.90881942,262.93426668)
\curveto(857.85881737,262.9142575)(857.80381743,262.90425751)(857.74381942,262.90426668)
\lineto(857.57881942,262.90426668)
\lineto(857.26381942,262.90426668)
\curveto(857.16381807,262.90425751)(857.07881815,262.92925749)(857.00881942,262.97926668)
\moveto(858.11881942,254.47426668)
\lineto(858.11881942,254.15926668)
\curveto(858.1288171,254.05926636)(858.10881712,253.97926644)(858.05881942,253.91926668)
\curveto(858.0288172,253.85926656)(857.98381725,253.8192666)(857.92381942,253.79926668)
\curveto(857.86381737,253.78926663)(857.79381744,253.77426664)(857.71381942,253.75426668)
\lineto(857.48881942,253.75426668)
\curveto(857.35881787,253.75426666)(857.24381799,253.75926666)(857.14381942,253.76926668)
\curveto(857.05381818,253.78926663)(856.98381825,253.83926658)(856.93381942,253.91926668)
\curveto(856.89381834,253.97926644)(856.87381836,254.05426636)(856.87381942,254.14426668)
\lineto(856.87381942,254.42926668)
\lineto(856.87381942,260.77426668)
\lineto(856.87381942,261.08926668)
\curveto(856.87381836,261.19925922)(856.89881833,261.28425913)(856.94881942,261.34426668)
\curveto(856.97881825,261.39425902)(857.01881821,261.42425899)(857.06881942,261.43426668)
\curveto(857.11881811,261.44425897)(857.17381806,261.45925896)(857.23381942,261.47926668)
\curveto(857.25381798,261.47925894)(857.27381796,261.47425894)(857.29381942,261.46426668)
\curveto(857.32381791,261.46425895)(857.34881788,261.46925895)(857.36881942,261.47926668)
\curveto(857.49881773,261.47925894)(857.6288176,261.47425894)(857.75881942,261.46426668)
\curveto(857.89881733,261.46425895)(857.99381724,261.42425899)(858.04381942,261.34426668)
\curveto(858.09381714,261.28425913)(858.11881711,261.20425921)(858.11881942,261.10426668)
\lineto(858.11881942,260.81926668)
\lineto(858.11881942,254.47426668)
}
}
{
\newrgbcolor{curcolor}{0 0 0}
\pscustom[linestyle=none,fillstyle=solid,fillcolor=curcolor]
{
\newpath
\moveto(866.94866317,254.30926668)
\curveto(866.97865534,254.14926627)(866.96365536,254.0142664)(866.90366317,253.90426668)
\curveto(866.84365548,253.80426661)(866.76365556,253.72926669)(866.66366317,253.67926668)
\curveto(866.61365571,253.65926676)(866.55865576,253.64926677)(866.49866317,253.64926668)
\curveto(866.44865587,253.64926677)(866.39365593,253.63926678)(866.33366317,253.61926668)
\curveto(866.11365621,253.56926685)(865.89365643,253.58426683)(865.67366317,253.66426668)
\curveto(865.46365686,253.73426668)(865.318657,253.82426659)(865.23866317,253.93426668)
\curveto(865.18865713,254.00426641)(865.14365718,254.08426633)(865.10366317,254.17426668)
\curveto(865.06365726,254.27426614)(865.01365731,254.35426606)(864.95366317,254.41426668)
\curveto(864.93365739,254.43426598)(864.90865741,254.45426596)(864.87866317,254.47426668)
\curveto(864.85865746,254.49426592)(864.82865749,254.49926592)(864.78866317,254.48926668)
\curveto(864.67865764,254.45926596)(864.57365775,254.40426601)(864.47366317,254.32426668)
\curveto(864.38365794,254.24426617)(864.29365803,254.17426624)(864.20366317,254.11426668)
\curveto(864.07365825,254.03426638)(863.93365839,253.95926646)(863.78366317,253.88926668)
\curveto(863.63365869,253.82926659)(863.47365885,253.77426664)(863.30366317,253.72426668)
\curveto(863.20365912,253.69426672)(863.09365923,253.67426674)(862.97366317,253.66426668)
\curveto(862.86365946,253.65426676)(862.75365957,253.63926678)(862.64366317,253.61926668)
\curveto(862.59365973,253.60926681)(862.54865977,253.60426681)(862.50866317,253.60426668)
\lineto(862.40366317,253.60426668)
\curveto(862.29366003,253.58426683)(862.18866013,253.58426683)(862.08866317,253.60426668)
\lineto(861.95366317,253.60426668)
\curveto(861.90366042,253.6142668)(861.85366047,253.6192668)(861.80366317,253.61926668)
\curveto(861.75366057,253.6192668)(861.70866061,253.62926679)(861.66866317,253.64926668)
\curveto(861.62866069,253.65926676)(861.59366073,253.66426675)(861.56366317,253.66426668)
\curveto(861.54366078,253.65426676)(861.5186608,253.65426676)(861.48866317,253.66426668)
\lineto(861.24866317,253.72426668)
\curveto(861.16866115,253.73426668)(861.09366123,253.75426666)(861.02366317,253.78426668)
\curveto(860.7236616,253.9142665)(860.47866184,254.05926636)(860.28866317,254.21926668)
\curveto(860.10866221,254.38926603)(859.95866236,254.62426579)(859.83866317,254.92426668)
\curveto(859.74866257,255.14426527)(859.70366262,255.40926501)(859.70366317,255.71926668)
\lineto(859.70366317,256.03426668)
\curveto(859.71366261,256.08426433)(859.7186626,256.13426428)(859.71866317,256.18426668)
\lineto(859.74866317,256.36426668)
\lineto(859.86866317,256.69426668)
\curveto(859.90866241,256.80426361)(859.95866236,256.90426351)(860.01866317,256.99426668)
\curveto(860.19866212,257.28426313)(860.44366188,257.49926292)(860.75366317,257.63926668)
\curveto(861.06366126,257.77926264)(861.40366092,257.90426251)(861.77366317,258.01426668)
\curveto(861.91366041,258.05426236)(862.05866026,258.08426233)(862.20866317,258.10426668)
\curveto(862.35865996,258.12426229)(862.50865981,258.14926227)(862.65866317,258.17926668)
\curveto(862.72865959,258.19926222)(862.79365953,258.20926221)(862.85366317,258.20926668)
\curveto(862.9236594,258.20926221)(862.99865932,258.2192622)(863.07866317,258.23926668)
\curveto(863.14865917,258.25926216)(863.2186591,258.26926215)(863.28866317,258.26926668)
\curveto(863.35865896,258.27926214)(863.43365889,258.29426212)(863.51366317,258.31426668)
\curveto(863.76365856,258.37426204)(863.99865832,258.42426199)(864.21866317,258.46426668)
\curveto(864.43865788,258.5142619)(864.61365771,258.62926179)(864.74366317,258.80926668)
\curveto(864.80365752,258.88926153)(864.85365747,258.98926143)(864.89366317,259.10926668)
\curveto(864.93365739,259.23926118)(864.93365739,259.37926104)(864.89366317,259.52926668)
\curveto(864.83365749,259.76926065)(864.74365758,259.95926046)(864.62366317,260.09926668)
\curveto(864.51365781,260.23926018)(864.35365797,260.34926007)(864.14366317,260.42926668)
\curveto(864.0236583,260.47925994)(863.87865844,260.5142599)(863.70866317,260.53426668)
\curveto(863.54865877,260.55425986)(863.37865894,260.56425985)(863.19866317,260.56426668)
\curveto(863.0186593,260.56425985)(862.84365948,260.55425986)(862.67366317,260.53426668)
\curveto(862.50365982,260.5142599)(862.35865996,260.48425993)(862.23866317,260.44426668)
\curveto(862.06866025,260.38426003)(861.90366042,260.29926012)(861.74366317,260.18926668)
\curveto(861.66366066,260.12926029)(861.58866073,260.04926037)(861.51866317,259.94926668)
\curveto(861.45866086,259.85926056)(861.40366092,259.75926066)(861.35366317,259.64926668)
\curveto(861.323661,259.56926085)(861.29366103,259.48426093)(861.26366317,259.39426668)
\curveto(861.24366108,259.30426111)(861.19866112,259.23426118)(861.12866317,259.18426668)
\curveto(861.08866123,259.15426126)(861.0186613,259.12926129)(860.91866317,259.10926668)
\curveto(860.82866149,259.09926132)(860.73366159,259.09426132)(860.63366317,259.09426668)
\curveto(860.53366179,259.09426132)(860.43366189,259.09926132)(860.33366317,259.10926668)
\curveto(860.24366208,259.12926129)(860.17866214,259.15426126)(860.13866317,259.18426668)
\curveto(860.09866222,259.2142612)(860.06866225,259.26426115)(860.04866317,259.33426668)
\curveto(860.02866229,259.40426101)(860.02866229,259.47926094)(860.04866317,259.55926668)
\curveto(860.07866224,259.68926073)(860.10866221,259.80926061)(860.13866317,259.91926668)
\curveto(860.17866214,260.03926038)(860.2236621,260.15426026)(860.27366317,260.26426668)
\curveto(860.46366186,260.6142598)(860.70366162,260.88425953)(860.99366317,261.07426668)
\curveto(861.28366104,261.27425914)(861.64366068,261.43425898)(862.07366317,261.55426668)
\curveto(862.17366015,261.57425884)(862.27366005,261.58925883)(862.37366317,261.59926668)
\curveto(862.48365984,261.60925881)(862.59365973,261.62425879)(862.70366317,261.64426668)
\curveto(862.74365958,261.65425876)(862.80865951,261.65425876)(862.89866317,261.64426668)
\curveto(862.98865933,261.64425877)(863.04365928,261.65425876)(863.06366317,261.67426668)
\curveto(863.76365856,261.68425873)(864.37365795,261.60425881)(864.89366317,261.43426668)
\curveto(865.41365691,261.26425915)(865.77865654,260.93925948)(865.98866317,260.45926668)
\curveto(866.07865624,260.25926016)(866.12865619,260.02426039)(866.13866317,259.75426668)
\curveto(866.15865616,259.49426092)(866.16865615,259.2192612)(866.16866317,258.92926668)
\lineto(866.16866317,255.61426668)
\curveto(866.16865615,255.47426494)(866.17365615,255.33926508)(866.18366317,255.20926668)
\curveto(866.19365613,255.07926534)(866.2236561,254.97426544)(866.27366317,254.89426668)
\curveto(866.323656,254.82426559)(866.38865593,254.77426564)(866.46866317,254.74426668)
\curveto(866.55865576,254.70426571)(866.64365568,254.67426574)(866.72366317,254.65426668)
\curveto(866.80365552,254.64426577)(866.86365546,254.59926582)(866.90366317,254.51926668)
\curveto(866.9236554,254.48926593)(866.93365539,254.45926596)(866.93366317,254.42926668)
\curveto(866.93365539,254.39926602)(866.93865538,254.35926606)(866.94866317,254.30926668)
\moveto(864.80366317,255.97426668)
\curveto(864.86365746,256.1142643)(864.89365743,256.27426414)(864.89366317,256.45426668)
\curveto(864.90365742,256.64426377)(864.90865741,256.83926358)(864.90866317,257.03926668)
\curveto(864.90865741,257.14926327)(864.90365742,257.24926317)(864.89366317,257.33926668)
\curveto(864.88365744,257.42926299)(864.84365748,257.49926292)(864.77366317,257.54926668)
\curveto(864.74365758,257.56926285)(864.67365765,257.57926284)(864.56366317,257.57926668)
\curveto(864.54365778,257.55926286)(864.50865781,257.54926287)(864.45866317,257.54926668)
\curveto(864.40865791,257.54926287)(864.36365796,257.53926288)(864.32366317,257.51926668)
\curveto(864.24365808,257.49926292)(864.15365817,257.47926294)(864.05366317,257.45926668)
\lineto(863.75366317,257.39926668)
\curveto(863.7236586,257.39926302)(863.68865863,257.39426302)(863.64866317,257.38426668)
\lineto(863.54366317,257.38426668)
\curveto(863.39365893,257.34426307)(863.22865909,257.3192631)(863.04866317,257.30926668)
\curveto(862.87865944,257.30926311)(862.7186596,257.28926313)(862.56866317,257.24926668)
\curveto(862.48865983,257.22926319)(862.41365991,257.20926321)(862.34366317,257.18926668)
\curveto(862.28366004,257.17926324)(862.21366011,257.16426325)(862.13366317,257.14426668)
\curveto(861.97366035,257.09426332)(861.8236605,257.02926339)(861.68366317,256.94926668)
\curveto(861.54366078,256.87926354)(861.4236609,256.78926363)(861.32366317,256.67926668)
\curveto(861.2236611,256.56926385)(861.14866117,256.43426398)(861.09866317,256.27426668)
\curveto(861.04866127,256.12426429)(861.02866129,255.93926448)(861.03866317,255.71926668)
\curveto(861.03866128,255.6192648)(861.05366127,255.52426489)(861.08366317,255.43426668)
\curveto(861.1236612,255.35426506)(861.16866115,255.27926514)(861.21866317,255.20926668)
\curveto(861.29866102,255.09926532)(861.40366092,255.00426541)(861.53366317,254.92426668)
\curveto(861.66366066,254.85426556)(861.80366052,254.79426562)(861.95366317,254.74426668)
\curveto(862.00366032,254.73426568)(862.05366027,254.72926569)(862.10366317,254.72926668)
\curveto(862.15366017,254.72926569)(862.20366012,254.72426569)(862.25366317,254.71426668)
\curveto(862.32366,254.69426572)(862.40865991,254.67926574)(862.50866317,254.66926668)
\curveto(862.6186597,254.66926575)(862.70865961,254.67926574)(862.77866317,254.69926668)
\curveto(862.83865948,254.7192657)(862.89865942,254.72426569)(862.95866317,254.71426668)
\curveto(863.0186593,254.7142657)(863.07865924,254.72426569)(863.13866317,254.74426668)
\curveto(863.2186591,254.76426565)(863.29365903,254.77926564)(863.36366317,254.78926668)
\curveto(863.44365888,254.79926562)(863.5186588,254.8192656)(863.58866317,254.84926668)
\curveto(863.87865844,254.96926545)(864.1236582,255.1142653)(864.32366317,255.28426668)
\curveto(864.53365779,255.45426496)(864.69365763,255.68426473)(864.80366317,255.97426668)
}
}
{
\newrgbcolor{curcolor}{0 0 0}
\pscustom[linestyle=none,fillstyle=solid,fillcolor=curcolor]
{
\newpath
\moveto(870.5503038,261.65926668)
\curveto(871.27029973,261.66925875)(871.87529913,261.58425883)(872.3653038,261.40426668)
\curveto(872.85529815,261.23425918)(873.23529777,260.92925949)(873.5053038,260.48926668)
\curveto(873.57529743,260.37926004)(873.63029737,260.26426015)(873.6703038,260.14426668)
\curveto(873.71029729,260.03426038)(873.75029725,259.90926051)(873.7903038,259.76926668)
\curveto(873.81029719,259.69926072)(873.81529719,259.62426079)(873.8053038,259.54426668)
\curveto(873.79529721,259.47426094)(873.78029722,259.419261)(873.7603038,259.37926668)
\curveto(873.74029726,259.35926106)(873.71529729,259.33926108)(873.6853038,259.31926668)
\curveto(873.65529735,259.30926111)(873.63029737,259.29426112)(873.6103038,259.27426668)
\curveto(873.56029744,259.25426116)(873.51029749,259.24926117)(873.4603038,259.25926668)
\curveto(873.41029759,259.26926115)(873.36029764,259.26926115)(873.3103038,259.25926668)
\curveto(873.23029777,259.23926118)(873.12529788,259.23426118)(872.9953038,259.24426668)
\curveto(872.86529814,259.26426115)(872.77529823,259.28926113)(872.7253038,259.31926668)
\curveto(872.64529836,259.36926105)(872.59029841,259.43426098)(872.5603038,259.51426668)
\curveto(872.54029846,259.60426081)(872.5052985,259.68926073)(872.4553038,259.76926668)
\curveto(872.36529864,259.92926049)(872.24029876,260.07426034)(872.0803038,260.20426668)
\curveto(871.97029903,260.28426013)(871.85029915,260.34426007)(871.7203038,260.38426668)
\curveto(871.59029941,260.42425999)(871.45029955,260.46425995)(871.3003038,260.50426668)
\curveto(871.25029975,260.52425989)(871.2002998,260.52925989)(871.1503038,260.51926668)
\curveto(871.1002999,260.5192599)(871.05029995,260.52425989)(871.0003038,260.53426668)
\curveto(870.94030006,260.55425986)(870.86530014,260.56425985)(870.7753038,260.56426668)
\curveto(870.68530032,260.56425985)(870.61030039,260.55425986)(870.5503038,260.53426668)
\lineto(870.4603038,260.53426668)
\lineto(870.3103038,260.50426668)
\curveto(870.26030074,260.50425991)(870.21030079,260.49925992)(870.1603038,260.48926668)
\curveto(869.9003011,260.42925999)(869.68530132,260.34426007)(869.5153038,260.23426668)
\curveto(869.34530166,260.12426029)(869.23030177,259.93926048)(869.1703038,259.67926668)
\curveto(869.15030185,259.60926081)(869.14530186,259.53926088)(869.1553038,259.46926668)
\curveto(869.17530183,259.39926102)(869.19530181,259.33926108)(869.2153038,259.28926668)
\curveto(869.27530173,259.13926128)(869.34530166,259.02926139)(869.4253038,258.95926668)
\curveto(869.51530149,258.89926152)(869.62530138,258.82926159)(869.7553038,258.74926668)
\curveto(869.91530109,258.64926177)(870.09530091,258.57426184)(870.2953038,258.52426668)
\curveto(870.49530051,258.48426193)(870.69530031,258.43426198)(870.8953038,258.37426668)
\curveto(871.02529998,258.33426208)(871.15529985,258.30426211)(871.2853038,258.28426668)
\curveto(871.41529959,258.26426215)(871.54529946,258.23426218)(871.6753038,258.19426668)
\curveto(871.88529912,258.13426228)(872.09029891,258.07426234)(872.2903038,258.01426668)
\curveto(872.49029851,257.96426245)(872.69029831,257.89926252)(872.8903038,257.81926668)
\lineto(873.0403038,257.75926668)
\curveto(873.09029791,257.73926268)(873.14029786,257.7142627)(873.1903038,257.68426668)
\curveto(873.39029761,257.56426285)(873.56529744,257.42926299)(873.7153038,257.27926668)
\curveto(873.86529714,257.12926329)(873.99029701,256.93926348)(874.0903038,256.70926668)
\curveto(874.11029689,256.63926378)(874.13029687,256.54426387)(874.1503038,256.42426668)
\curveto(874.17029683,256.35426406)(874.18029682,256.27926414)(874.1803038,256.19926668)
\curveto(874.19029681,256.12926429)(874.19529681,256.04926437)(874.1953038,255.95926668)
\lineto(874.1953038,255.80926668)
\curveto(874.17529683,255.73926468)(874.16529684,255.66926475)(874.1653038,255.59926668)
\curveto(874.16529684,255.52926489)(874.15529685,255.45926496)(874.1353038,255.38926668)
\curveto(874.1052969,255.27926514)(874.07029693,255.17426524)(874.0303038,255.07426668)
\curveto(873.99029701,254.97426544)(873.94529706,254.88426553)(873.8953038,254.80426668)
\curveto(873.73529727,254.54426587)(873.53029747,254.33426608)(873.2803038,254.17426668)
\curveto(873.03029797,254.02426639)(872.75029825,253.89426652)(872.4403038,253.78426668)
\curveto(872.35029865,253.75426666)(872.25529875,253.73426668)(872.1553038,253.72426668)
\curveto(872.06529894,253.70426671)(871.97529903,253.67926674)(871.8853038,253.64926668)
\curveto(871.78529922,253.62926679)(871.68529932,253.6192668)(871.5853038,253.61926668)
\curveto(871.48529952,253.6192668)(871.38529962,253.60926681)(871.2853038,253.58926668)
\lineto(871.1353038,253.58926668)
\curveto(871.08529992,253.57926684)(871.01529999,253.57426684)(870.9253038,253.57426668)
\curveto(870.83530017,253.57426684)(870.76530024,253.57926684)(870.7153038,253.58926668)
\lineto(870.5503038,253.58926668)
\curveto(870.49030051,253.60926681)(870.42530058,253.6192668)(870.3553038,253.61926668)
\curveto(870.28530072,253.60926681)(870.22530078,253.6142668)(870.1753038,253.63426668)
\curveto(870.12530088,253.64426677)(870.06030094,253.64926677)(869.9803038,253.64926668)
\lineto(869.7403038,253.70926668)
\curveto(869.67030133,253.7192667)(869.59530141,253.73926668)(869.5153038,253.76926668)
\curveto(869.2053018,253.86926655)(868.93530207,253.99426642)(868.7053038,254.14426668)
\curveto(868.47530253,254.29426612)(868.27530273,254.48926593)(868.1053038,254.72926668)
\curveto(868.01530299,254.85926556)(867.94030306,254.99426542)(867.8803038,255.13426668)
\curveto(867.82030318,255.27426514)(867.76530324,255.42926499)(867.7153038,255.59926668)
\curveto(867.69530331,255.65926476)(867.68530332,255.72926469)(867.6853038,255.80926668)
\curveto(867.69530331,255.89926452)(867.71030329,255.96926445)(867.7303038,256.01926668)
\curveto(867.76030324,256.05926436)(867.81030319,256.09926432)(867.8803038,256.13926668)
\curveto(867.93030307,256.15926426)(868.000303,256.16926425)(868.0903038,256.16926668)
\curveto(868.18030282,256.17926424)(868.27030273,256.17926424)(868.3603038,256.16926668)
\curveto(868.45030255,256.15926426)(868.53530247,256.14426427)(868.6153038,256.12426668)
\curveto(868.7053023,256.1142643)(868.76530224,256.09926432)(868.7953038,256.07926668)
\curveto(868.86530214,256.02926439)(868.91030209,255.95426446)(868.9303038,255.85426668)
\curveto(868.96030204,255.76426465)(868.99530201,255.67926474)(869.0353038,255.59926668)
\curveto(869.13530187,255.37926504)(869.27030173,255.20926521)(869.4403038,255.08926668)
\curveto(869.56030144,254.99926542)(869.69530131,254.92926549)(869.8453038,254.87926668)
\curveto(869.99530101,254.82926559)(870.15530085,254.77926564)(870.3253038,254.72926668)
\lineto(870.6403038,254.68426668)
\lineto(870.7303038,254.68426668)
\curveto(870.8003002,254.66426575)(870.89030011,254.65426576)(871.0003038,254.65426668)
\curveto(871.12029988,254.65426576)(871.22029978,254.66426575)(871.3003038,254.68426668)
\curveto(871.37029963,254.68426573)(871.42529958,254.68926573)(871.4653038,254.69926668)
\curveto(871.52529948,254.70926571)(871.58529942,254.7142657)(871.6453038,254.71426668)
\curveto(871.7052993,254.72426569)(871.76029924,254.73426568)(871.8103038,254.74426668)
\curveto(872.1002989,254.82426559)(872.33029867,254.92926549)(872.5003038,255.05926668)
\curveto(872.67029833,255.18926523)(872.79029821,255.40926501)(872.8603038,255.71926668)
\curveto(872.88029812,255.76926465)(872.88529812,255.82426459)(872.8753038,255.88426668)
\curveto(872.86529814,255.94426447)(872.85529815,255.98926443)(872.8453038,256.01926668)
\curveto(872.79529821,256.20926421)(872.72529828,256.34926407)(872.6353038,256.43926668)
\curveto(872.54529846,256.53926388)(872.43029857,256.62926379)(872.2903038,256.70926668)
\curveto(872.2002988,256.76926365)(872.1002989,256.8192636)(871.9903038,256.85926668)
\lineto(871.6603038,256.97926668)
\curveto(871.63029937,256.98926343)(871.6002994,256.99426342)(871.5703038,256.99426668)
\curveto(871.55029945,256.99426342)(871.52529948,257.00426341)(871.4953038,257.02426668)
\curveto(871.15529985,257.13426328)(870.8003002,257.2142632)(870.4303038,257.26426668)
\curveto(870.07030093,257.32426309)(869.73030127,257.419263)(869.4103038,257.54926668)
\curveto(869.31030169,257.58926283)(869.21530179,257.62426279)(869.1253038,257.65426668)
\curveto(869.03530197,257.68426273)(868.95030205,257.72426269)(868.8703038,257.77426668)
\curveto(868.68030232,257.88426253)(868.5053025,258.00926241)(868.3453038,258.14926668)
\curveto(868.18530282,258.28926213)(868.06030294,258.46426195)(867.9703038,258.67426668)
\curveto(867.94030306,258.74426167)(867.91530309,258.8142616)(867.8953038,258.88426668)
\curveto(867.88530312,258.95426146)(867.87030313,259.02926139)(867.8503038,259.10926668)
\curveto(867.82030318,259.22926119)(867.81030319,259.36426105)(867.8203038,259.51426668)
\curveto(867.83030317,259.67426074)(867.84530316,259.80926061)(867.8653038,259.91926668)
\curveto(867.88530312,259.96926045)(867.89530311,260.00926041)(867.8953038,260.03926668)
\curveto(867.9053031,260.07926034)(867.92030308,260.1192603)(867.9403038,260.15926668)
\curveto(868.03030297,260.38926003)(868.15030285,260.58925983)(868.3003038,260.75926668)
\curveto(868.46030254,260.92925949)(868.64030236,261.07925934)(868.8403038,261.20926668)
\curveto(868.99030201,261.29925912)(869.15530185,261.36925905)(869.3353038,261.41926668)
\curveto(869.51530149,261.47925894)(869.7053013,261.53425888)(869.9053038,261.58426668)
\curveto(869.97530103,261.59425882)(870.04030096,261.60425881)(870.1003038,261.61426668)
\curveto(870.17030083,261.62425879)(870.24530076,261.63425878)(870.3253038,261.64426668)
\curveto(870.35530065,261.65425876)(870.39530061,261.65425876)(870.4453038,261.64426668)
\curveto(870.49530051,261.63425878)(870.53030047,261.63925878)(870.5503038,261.65926668)
}
}
{
\newrgbcolor{curcolor}{0.60000002 0.60000002 0.60000002}
\pscustom[linestyle=none,fillstyle=solid,fillcolor=curcolor]
{
\newpath
\moveto(798.51865829,264.4643033)
\lineto(813.51865829,264.4643033)
\lineto(813.51865829,249.4643033)
\lineto(798.51865829,249.4643033)
\closepath
}
}
{
\newrgbcolor{curcolor}{0 0 0}
\pscustom[linestyle=none,fillstyle=solid,fillcolor=curcolor]
{
\newpath
\moveto(822.4818663,241.63856111)
\curveto(823.17186006,241.65855016)(823.77685945,241.59355022)(824.2968663,241.44356111)
\curveto(824.81685841,241.30355051)(825.27685795,241.09355072)(825.6768663,240.81356111)
\curveto(825.85685737,240.69355112)(826.02185721,240.55855126)(826.1718663,240.40856111)
\curveto(826.19185704,240.38855143)(826.21185702,240.36855145)(826.2318663,240.34856111)
\curveto(826.25185698,240.32855149)(826.27185696,240.30855151)(826.2918663,240.28856111)
\curveto(826.34185689,240.20855161)(826.39685683,240.13355168)(826.4568663,240.06356111)
\curveto(826.51685671,240.00355181)(826.57185666,239.93355188)(826.6218663,239.85356111)
\curveto(826.7318565,239.68355213)(826.8268564,239.50355231)(826.9068663,239.31356111)
\curveto(826.98685624,239.12355269)(827.06685616,238.92855289)(827.1468663,238.72856111)
\curveto(827.17685605,238.62855319)(827.19185604,238.52355329)(827.1918663,238.41356111)
\curveto(827.19185604,238.30355351)(827.15185608,238.22355359)(827.0718663,238.17356111)
\curveto(827.05185618,238.15355366)(827.02185621,238.14355367)(826.9818663,238.14356111)
\curveto(826.95185628,238.14355367)(826.92185631,238.13855368)(826.8918663,238.12856111)
\lineto(826.7868663,238.12856111)
\curveto(826.73685649,238.10855371)(826.65185658,238.09855372)(826.5318663,238.09856111)
\curveto(826.42185681,238.09855372)(826.34185689,238.10855371)(826.2918663,238.12856111)
\curveto(826.26185697,238.13855368)(826.231857,238.13855368)(826.2018663,238.12856111)
\curveto(826.17185706,238.1185537)(826.13685709,238.12355369)(826.0968663,238.14356111)
\curveto(826.03685719,238.15355366)(825.98185725,238.17855364)(825.9318663,238.21856111)
\curveto(825.87185736,238.26855355)(825.8268574,238.33855348)(825.7968663,238.42856111)
\curveto(825.77685745,238.5185533)(825.74685748,238.60355321)(825.7068663,238.68356111)
\curveto(825.65685757,238.813553)(825.59685763,238.93355288)(825.5268663,239.04356111)
\curveto(825.46685776,239.16355265)(825.39685783,239.27855254)(825.3168663,239.38856111)
\curveto(825.29685793,239.40855241)(825.27185796,239.42855239)(825.2418663,239.44856111)
\curveto(825.21185802,239.47855234)(825.18685804,239.50855231)(825.1668663,239.53856111)
\curveto(825.07685815,239.63855218)(824.98685824,239.72355209)(824.8968663,239.79356111)
\curveto(824.84685838,239.82355199)(824.80185843,239.85355196)(824.7618663,239.88356111)
\curveto(824.72185851,239.92355189)(824.67685855,239.95855186)(824.6268663,239.98856111)
\curveto(824.48685874,240.06855175)(824.33685889,240.13855168)(824.1768663,240.19856111)
\curveto(824.01685921,240.25855156)(823.85185938,240.3135515)(823.6818663,240.36356111)
\curveto(823.59185964,240.38355143)(823.50185973,240.39855142)(823.4118663,240.40856111)
\curveto(823.32185991,240.4185514)(823.23186,240.43355138)(823.1418663,240.45356111)
\curveto(823.10186013,240.46355135)(823.06186017,240.46355135)(823.0218663,240.45356111)
\curveto(822.99186024,240.45355136)(822.96186027,240.45855136)(822.9318663,240.46856111)
\lineto(822.7518663,240.46856111)
\lineto(822.5418663,240.46856111)
\curveto(822.47186076,240.46855135)(822.40686082,240.46355135)(822.3468663,240.45356111)
\curveto(822.3268609,240.45355136)(822.30186093,240.44855137)(822.2718663,240.43856111)
\lineto(822.1968663,240.43856111)
\curveto(822.13686109,240.42855139)(822.07186116,240.4185514)(822.0018663,240.40856111)
\lineto(821.8218663,240.37856111)
\curveto(821.47186176,240.28855153)(821.16686206,240.16355165)(820.9068663,240.00356111)
\curveto(820.48686274,239.74355207)(820.14686308,239.43355238)(819.8868663,239.07356111)
\curveto(819.63686359,238.72355309)(819.4268638,238.29355352)(819.2568663,237.78356111)
\curveto(819.21686401,237.67355414)(819.18686404,237.55855426)(819.1668663,237.43856111)
\curveto(819.14686408,237.3185545)(819.12186411,237.19855462)(819.0918663,237.07856111)
\curveto(819.07186416,237.02855479)(819.06186417,236.97855484)(819.0618663,236.92856111)
\curveto(819.07186416,236.88855493)(819.06686416,236.84355497)(819.0468663,236.79356111)
\curveto(819.0268642,236.72355509)(819.01686421,236.64855517)(819.0168663,236.56856111)
\curveto(819.0268642,236.49855532)(819.02186421,236.42355539)(819.0018663,236.34356111)
\lineto(819.0018663,236.17856111)
\curveto(818.99186424,236.1185557)(818.98686424,236.03355578)(818.9868663,235.92356111)
\curveto(818.98686424,235.813556)(818.99186424,235.73355608)(819.0018663,235.68356111)
\lineto(819.0018663,235.53356111)
\curveto(819.01186422,235.5135563)(819.01686421,235.48355633)(819.0168663,235.44356111)
\curveto(819.01686421,235.4135564)(819.02186421,235.38855643)(819.0318663,235.36856111)
\curveto(819.05186418,235.29855652)(819.05686417,235.23355658)(819.0468663,235.17356111)
\curveto(819.03686419,235.1135567)(819.04186419,235.04855677)(819.0618663,234.97856111)
\curveto(819.08186415,234.89855692)(819.09686413,234.818557)(819.1068663,234.73856111)
\curveto(819.1268641,234.66855715)(819.14686408,234.59355722)(819.1668663,234.51356111)
\lineto(819.3168663,234.06356111)
\lineto(819.4968663,233.64356111)
\curveto(819.54686368,233.52355829)(819.60686362,233.40855841)(819.6768663,233.29856111)
\curveto(819.75686347,233.18855863)(819.83686339,233.08355873)(819.9168663,232.98356111)
\curveto(820.28686294,232.48355933)(820.77186246,232.1135597)(821.3718663,231.87356111)
\curveto(821.44186179,231.84355997)(821.51186172,231.81856)(821.5818663,231.79856111)
\curveto(821.65186158,231.77856004)(821.7268615,231.75856006)(821.8068663,231.73856111)
\curveto(822.04686118,231.66856015)(822.3268609,231.63356018)(822.6468663,231.63356111)
\lineto(822.8418663,231.63356111)
\curveto(822.91186032,231.63356018)(822.97686025,231.63856018)(823.0368663,231.64856111)
\curveto(823.08686014,231.66856015)(823.13686009,231.67356014)(823.1868663,231.66356111)
\curveto(823.24685998,231.65356016)(823.30185993,231.65856016)(823.3518663,231.67856111)
\curveto(823.49185974,231.7185601)(823.62185961,231.74856007)(823.7418663,231.76856111)
\curveto(823.87185936,231.79856002)(823.99185924,231.83855998)(824.1018663,231.88856111)
\curveto(825.0318582,232.25855956)(825.65685757,232.92355889)(825.9768663,233.88356111)
\curveto(825.99685723,233.96355785)(826.01185722,234.04355777)(826.0218663,234.12356111)
\lineto(826.0818663,234.36356111)
\curveto(826.11185712,234.48355733)(826.12185711,234.6135572)(826.1118663,234.75356111)
\curveto(826.10185713,234.90355691)(826.05685717,235.00855681)(825.9768663,235.06856111)
\curveto(825.89685733,235.1185567)(825.78685744,235.14355667)(825.6468663,235.14356111)
\lineto(825.2418663,235.14356111)
\lineto(823.5768663,235.14356111)
\lineto(823.2168663,235.14356111)
\curveto(823.08686014,235.14355667)(822.98186025,235.15855666)(822.9018663,235.18856111)
\curveto(822.82186041,235.22855659)(822.77186046,235.28355653)(822.7518663,235.35356111)
\curveto(822.7318605,235.39355642)(822.71686051,235.44855637)(822.7068663,235.51856111)
\curveto(822.69686053,235.59855622)(822.69186054,235.67855614)(822.6918663,235.75856111)
\curveto(822.69186054,235.83855598)(822.69686053,235.9135559)(822.7068663,235.98356111)
\curveto(822.7268605,236.06355575)(822.74686048,236.1185557)(822.7668663,236.14856111)
\curveto(822.80686042,236.2185556)(822.87686035,236.26855555)(822.9768663,236.29856111)
\curveto(823.0268602,236.3185555)(823.08686014,236.32855549)(823.1568663,236.32856111)
\lineto(823.3668663,236.32856111)
\lineto(824.0418663,236.32856111)
\lineto(826.3218663,236.32856111)
\lineto(826.6518663,236.32856111)
\curveto(826.76185647,236.33855548)(826.86185637,236.33355548)(826.9518663,236.31356111)
\curveto(827.05185618,236.30355551)(827.13685609,236.27855554)(827.2068663,236.23856111)
\curveto(827.27685595,236.20855561)(827.3268559,236.14855567)(827.3568663,236.05856111)
\curveto(827.37685585,235.99855582)(827.38185585,235.92855589)(827.3718663,235.84856111)
\curveto(827.37185586,235.76855605)(827.37185586,235.68855613)(827.3718663,235.60856111)
\lineto(827.3718663,234.78356111)
\lineto(827.3718663,232.00856111)
\lineto(827.3718663,231.31856111)
\curveto(827.37185586,231.24856057)(827.37185586,231.17856064)(827.3718663,231.10856111)
\curveto(827.37185586,231.03856078)(827.36185587,230.97856084)(827.3418663,230.92856111)
\curveto(827.31185592,230.84856097)(827.25185598,230.78856103)(827.1618663,230.74856111)
\curveto(827.1318561,230.72856109)(827.06685616,230.7185611)(826.9668663,230.71856111)
\curveto(826.8268564,230.7185611)(826.72185651,230.73356108)(826.6518663,230.76356111)
\curveto(826.58185665,230.79356102)(826.5268567,230.83856098)(826.4868663,230.89856111)
\curveto(826.44685678,230.95856086)(826.41185682,231.02856079)(826.3818663,231.10856111)
\curveto(826.36185687,231.18856063)(826.33685689,231.27856054)(826.3068663,231.37856111)
\curveto(826.28685694,231.44856037)(826.25185698,231.53856028)(826.2018663,231.64856111)
\curveto(826.16185707,231.75856006)(826.08685714,231.79856002)(825.9768663,231.76856111)
\curveto(825.90685732,231.74856007)(825.85185738,231.7185601)(825.8118663,231.67856111)
\curveto(825.77185746,231.63856018)(825.7268575,231.59856022)(825.6768663,231.55856111)
\curveto(825.59685763,231.49856032)(825.52185771,231.43356038)(825.4518663,231.36356111)
\curveto(825.38185785,231.30356051)(825.30685792,231.24856057)(825.2268663,231.19856111)
\curveto(825.01685821,231.05856076)(824.79185844,230.94356087)(824.5518663,230.85356111)
\curveto(824.32185891,230.76356105)(824.07185916,230.67856114)(823.8018663,230.59856111)
\curveto(823.7318595,230.57856124)(823.66185957,230.56356125)(823.5918663,230.55356111)
\curveto(823.52185971,230.54356127)(823.44685978,230.52856129)(823.3668663,230.50856111)
\curveto(823.28685994,230.50856131)(823.22686,230.50356131)(823.1868663,230.49356111)
\lineto(823.0818663,230.49356111)
\curveto(823.05186018,230.49356132)(823.02186021,230.48856133)(822.9918663,230.47856111)
\lineto(822.8418663,230.47856111)
\curveto(822.80186043,230.46856135)(822.74686048,230.46356135)(822.6768663,230.46356111)
\curveto(822.60686062,230.46356135)(822.54686068,230.46856135)(822.4968663,230.47856111)
\lineto(822.2418663,230.47856111)
\curveto(822.1318611,230.49856132)(822.0268612,230.5135613)(821.9268663,230.52356111)
\curveto(821.83686139,230.52356129)(821.74186149,230.53856128)(821.6418663,230.56856111)
\curveto(821.46186177,230.6185612)(821.28686194,230.66356115)(821.1168663,230.70356111)
\curveto(820.94686228,230.74356107)(820.78186245,230.79856102)(820.6218663,230.86856111)
\curveto(820.05186318,231.12856069)(819.55186368,231.47356034)(819.1218663,231.90356111)
\curveto(818.69186454,232.34355947)(818.34686488,232.85355896)(818.0868663,233.43356111)
\curveto(818.03686519,233.55355826)(817.98686524,233.67855814)(817.9368663,233.80856111)
\curveto(817.89686533,233.93855788)(817.85186538,234.07355774)(817.8018663,234.21356111)
\curveto(817.79186544,234.25355756)(817.78686544,234.28855753)(817.7868663,234.31856111)
\curveto(817.78686544,234.35855746)(817.77686545,234.39855742)(817.7568663,234.43856111)
\curveto(817.7268655,234.54855727)(817.70186553,234.66355715)(817.6818663,234.78356111)
\curveto(817.67186556,234.90355691)(817.65186558,235.02355679)(817.6218663,235.14356111)
\curveto(817.61186562,235.18355663)(817.60686562,235.22355659)(817.6068663,235.26356111)
\curveto(817.61686561,235.30355651)(817.61686561,235.33855648)(817.6068663,235.36856111)
\lineto(817.6068663,235.50356111)
\curveto(817.58686564,235.55355626)(817.57686565,235.64855617)(817.5768663,235.78856111)
\curveto(817.57686565,235.92855589)(817.58686564,236.02855579)(817.6068663,236.08856111)
\lineto(817.6068663,236.22356111)
\lineto(817.6068663,236.40356111)
\curveto(817.60686562,236.46355535)(817.61186562,236.52355529)(817.6218663,236.58356111)
\curveto(817.6318656,236.63355518)(817.63686559,236.68355513)(817.6368663,236.73356111)
\curveto(817.63686559,236.79355502)(817.64186559,236.84855497)(817.6518663,236.89856111)
\curveto(817.68186555,237.0185548)(817.70186553,237.13855468)(817.7118663,237.25856111)
\curveto(817.7318655,237.37855444)(817.75686547,237.49855432)(817.7868663,237.61856111)
\curveto(817.88686534,237.94855387)(817.98686524,238.26355355)(818.0868663,238.56356111)
\curveto(818.19686503,238.87355294)(818.33686489,239.15855266)(818.5068663,239.41856111)
\curveto(818.80686442,239.88855193)(819.16186407,240.28855153)(819.5718663,240.61856111)
\curveto(819.99186324,240.95855086)(820.48686274,241.22355059)(821.0568663,241.41356111)
\curveto(821.16686206,241.45355036)(821.27186196,241.47855034)(821.3718663,241.48856111)
\curveto(821.48186175,241.50855031)(821.59186164,241.53355028)(821.7018663,241.56356111)
\curveto(821.75186148,241.58355023)(821.79686143,241.59355022)(821.8368663,241.59356111)
\curveto(821.88686134,241.59355022)(821.93686129,241.59855022)(821.9868663,241.60856111)
\curveto(822.06686116,241.6185502)(822.14686108,241.62355019)(822.2268663,241.62356111)
\curveto(822.31686091,241.63355018)(822.40186083,241.63855018)(822.4818663,241.63856111)
}
}
{
\newrgbcolor{curcolor}{0 0 0}
\pscustom[linestyle=none,fillstyle=solid,fillcolor=curcolor]
{
\newpath
\moveto(832.88835067,238.62356111)
\curveto(833.11834588,238.62355319)(833.24834575,238.56355325)(833.27835067,238.44356111)
\curveto(833.30834569,238.33355348)(833.32334568,238.16855365)(833.32335067,237.94856111)
\lineto(833.32335067,237.66356111)
\curveto(833.32334568,237.57355424)(833.2983457,237.49855432)(833.24835067,237.43856111)
\curveto(833.18834581,237.35855446)(833.1033459,237.3135545)(832.99335067,237.30356111)
\curveto(832.88334612,237.30355451)(832.77334623,237.28855453)(832.66335067,237.25856111)
\curveto(832.52334648,237.22855459)(832.38834661,237.19855462)(832.25835067,237.16856111)
\curveto(832.13834686,237.13855468)(832.02334698,237.09855472)(831.91335067,237.04856111)
\curveto(831.62334738,236.9185549)(831.38834761,236.73855508)(831.20835067,236.50856111)
\curveto(831.02834797,236.28855553)(830.87334813,236.03355578)(830.74335067,235.74356111)
\curveto(830.7033483,235.63355618)(830.67334833,235.5185563)(830.65335067,235.39856111)
\curveto(830.63334837,235.28855653)(830.60834839,235.17355664)(830.57835067,235.05356111)
\curveto(830.56834843,235.00355681)(830.56334844,234.95355686)(830.56335067,234.90356111)
\curveto(830.57334843,234.85355696)(830.57334843,234.80355701)(830.56335067,234.75356111)
\curveto(830.53334847,234.63355718)(830.51834848,234.49355732)(830.51835067,234.33356111)
\curveto(830.52834847,234.18355763)(830.53334847,234.03855778)(830.53335067,233.89856111)
\lineto(830.53335067,232.05356111)
\lineto(830.53335067,231.70856111)
\curveto(830.53334847,231.58856023)(830.52834847,231.47356034)(830.51835067,231.36356111)
\curveto(830.50834849,231.25356056)(830.5033485,231.15856066)(830.50335067,231.07856111)
\curveto(830.51334849,230.99856082)(830.49334851,230.92856089)(830.44335067,230.86856111)
\curveto(830.39334861,230.79856102)(830.31334869,230.75856106)(830.20335067,230.74856111)
\curveto(830.1033489,230.73856108)(829.99334901,230.73356108)(829.87335067,230.73356111)
\lineto(829.60335067,230.73356111)
\curveto(829.55334945,230.75356106)(829.5033495,230.76856105)(829.45335067,230.77856111)
\curveto(829.41334959,230.79856102)(829.38334962,230.82356099)(829.36335067,230.85356111)
\curveto(829.31334969,230.92356089)(829.28334972,231.00856081)(829.27335067,231.10856111)
\lineto(829.27335067,231.43856111)
\lineto(829.27335067,232.59356111)
\lineto(829.27335067,236.74856111)
\lineto(829.27335067,237.78356111)
\lineto(829.27335067,238.08356111)
\curveto(829.28334972,238.18355363)(829.31334969,238.26855355)(829.36335067,238.33856111)
\curveto(829.39334961,238.37855344)(829.44334956,238.40855341)(829.51335067,238.42856111)
\curveto(829.59334941,238.44855337)(829.67834932,238.45855336)(829.76835067,238.45856111)
\curveto(829.85834914,238.46855335)(829.94834905,238.46855335)(830.03835067,238.45856111)
\curveto(830.12834887,238.44855337)(830.1983488,238.43355338)(830.24835067,238.41356111)
\curveto(830.32834867,238.38355343)(830.37834862,238.32355349)(830.39835067,238.23356111)
\curveto(830.42834857,238.15355366)(830.44334856,238.06355375)(830.44335067,237.96356111)
\lineto(830.44335067,237.66356111)
\curveto(830.44334856,237.56355425)(830.46334854,237.47355434)(830.50335067,237.39356111)
\curveto(830.51334849,237.37355444)(830.52334848,237.35855446)(830.53335067,237.34856111)
\lineto(830.57835067,237.30356111)
\curveto(830.68834831,237.30355451)(830.77834822,237.34855447)(830.84835067,237.43856111)
\curveto(830.91834808,237.53855428)(830.97834802,237.6185542)(831.02835067,237.67856111)
\lineto(831.11835067,237.76856111)
\curveto(831.20834779,237.87855394)(831.33334767,237.99355382)(831.49335067,238.11356111)
\curveto(831.65334735,238.23355358)(831.8033472,238.32355349)(831.94335067,238.38356111)
\curveto(832.03334697,238.43355338)(832.12834687,238.46855335)(832.22835067,238.48856111)
\curveto(832.32834667,238.5185533)(832.43334657,238.54855327)(832.54335067,238.57856111)
\curveto(832.6033464,238.58855323)(832.66334634,238.59355322)(832.72335067,238.59356111)
\curveto(832.78334622,238.60355321)(832.83834616,238.6135532)(832.88835067,238.62356111)
}
}
{
\newrgbcolor{curcolor}{0 0 0}
\pscustom[linestyle=none,fillstyle=solid,fillcolor=curcolor]
{
\newpath
\moveto(834.7181163,238.44356111)
\lineto(835.1531163,238.44356111)
\curveto(835.30311433,238.44355337)(835.40811423,238.40355341)(835.4681163,238.32356111)
\curveto(835.51811412,238.24355357)(835.54311409,238.14355367)(835.5431163,238.02356111)
\curveto(835.55311408,237.90355391)(835.55811408,237.78355403)(835.5581163,237.66356111)
\lineto(835.5581163,236.23856111)
\lineto(835.5581163,233.97356111)
\lineto(835.5581163,233.28356111)
\curveto(835.55811408,233.05355876)(835.58311405,232.85355896)(835.6331163,232.68356111)
\curveto(835.79311384,232.23355958)(836.09311354,231.9185599)(836.5331163,231.73856111)
\curveto(836.75311288,231.64856017)(837.01811262,231.6135602)(837.3281163,231.63356111)
\curveto(837.638112,231.66356015)(837.88811175,231.7185601)(838.0781163,231.79856111)
\curveto(838.40811123,231.93855988)(838.66811097,232.1135597)(838.8581163,232.32356111)
\curveto(839.05811058,232.54355927)(839.21311042,232.82855899)(839.3231163,233.17856111)
\curveto(839.35311028,233.25855856)(839.37311026,233.33855848)(839.3831163,233.41856111)
\curveto(839.39311024,233.49855832)(839.40811023,233.58355823)(839.4281163,233.67356111)
\curveto(839.4381102,233.72355809)(839.4381102,233.76855805)(839.4281163,233.80856111)
\curveto(839.42811021,233.84855797)(839.4381102,233.89355792)(839.4581163,233.94356111)
\lineto(839.4581163,234.25856111)
\curveto(839.47811016,234.33855748)(839.48311015,234.42855739)(839.4731163,234.52856111)
\curveto(839.46311017,234.63855718)(839.45811018,234.73855708)(839.4581163,234.82856111)
\lineto(839.4581163,235.99856111)
\lineto(839.4581163,237.58856111)
\curveto(839.45811018,237.70855411)(839.45311018,237.83355398)(839.4431163,237.96356111)
\curveto(839.44311019,238.10355371)(839.46811017,238.2135536)(839.5181163,238.29356111)
\curveto(839.55811008,238.34355347)(839.60311003,238.37355344)(839.6531163,238.38356111)
\curveto(839.71310992,238.40355341)(839.78310985,238.42355339)(839.8631163,238.44356111)
\lineto(840.0881163,238.44356111)
\curveto(840.20810943,238.44355337)(840.31310932,238.43855338)(840.4031163,238.42856111)
\curveto(840.50310913,238.4185534)(840.57810906,238.37355344)(840.6281163,238.29356111)
\curveto(840.67810896,238.24355357)(840.70310893,238.16855365)(840.7031163,238.06856111)
\lineto(840.7031163,237.78356111)
\lineto(840.7031163,236.76356111)
\lineto(840.7031163,232.72856111)
\lineto(840.7031163,231.37856111)
\curveto(840.70310893,231.25856056)(840.69810894,231.14356067)(840.6881163,231.03356111)
\curveto(840.68810895,230.93356088)(840.65310898,230.85856096)(840.5831163,230.80856111)
\curveto(840.54310909,230.77856104)(840.48310915,230.75356106)(840.4031163,230.73356111)
\curveto(840.32310931,230.72356109)(840.2331094,230.7135611)(840.1331163,230.70356111)
\curveto(840.04310959,230.70356111)(839.95310968,230.70856111)(839.8631163,230.71856111)
\curveto(839.78310985,230.72856109)(839.72310991,230.74856107)(839.6831163,230.77856111)
\curveto(839.63311,230.818561)(839.58811005,230.88356093)(839.5481163,230.97356111)
\curveto(839.5381101,231.0135608)(839.52811011,231.06856075)(839.5181163,231.13856111)
\curveto(839.51811012,231.20856061)(839.51311012,231.27356054)(839.5031163,231.33356111)
\curveto(839.49311014,231.40356041)(839.47311016,231.45856036)(839.4431163,231.49856111)
\curveto(839.41311022,231.53856028)(839.36811027,231.55356026)(839.3081163,231.54356111)
\curveto(839.22811041,231.52356029)(839.14811049,231.46356035)(839.0681163,231.36356111)
\curveto(838.98811065,231.27356054)(838.91311072,231.20356061)(838.8431163,231.15356111)
\curveto(838.62311101,230.99356082)(838.37311126,230.85356096)(838.0931163,230.73356111)
\curveto(837.98311165,230.68356113)(837.86811177,230.65356116)(837.7481163,230.64356111)
\curveto(837.638112,230.62356119)(837.52311211,230.59856122)(837.4031163,230.56856111)
\curveto(837.35311228,230.55856126)(837.29811234,230.55856126)(837.2381163,230.56856111)
\curveto(837.18811245,230.57856124)(837.1381125,230.57356124)(837.0881163,230.55356111)
\curveto(836.98811265,230.53356128)(836.89811274,230.53356128)(836.8181163,230.55356111)
\lineto(836.6681163,230.55356111)
\curveto(836.61811302,230.57356124)(836.55811308,230.58356123)(836.4881163,230.58356111)
\curveto(836.42811321,230.58356123)(836.37311326,230.58856123)(836.3231163,230.59856111)
\curveto(836.28311335,230.6185612)(836.24311339,230.62856119)(836.2031163,230.62856111)
\curveto(836.17311346,230.6185612)(836.1331135,230.62356119)(836.0831163,230.64356111)
\lineto(835.8431163,230.70356111)
\curveto(835.77311386,230.72356109)(835.69811394,230.75356106)(835.6181163,230.79356111)
\curveto(835.35811428,230.90356091)(835.1381145,231.04856077)(834.9581163,231.22856111)
\curveto(834.78811485,231.4185604)(834.64811499,231.64356017)(834.5381163,231.90356111)
\curveto(834.49811514,231.99355982)(834.46811517,232.08355973)(834.4481163,232.17356111)
\lineto(834.3881163,232.47356111)
\curveto(834.36811527,232.53355928)(834.35811528,232.58855923)(834.3581163,232.63856111)
\curveto(834.36811527,232.69855912)(834.36311527,232.76355905)(834.3431163,232.83356111)
\curveto(834.3331153,232.85355896)(834.32811531,232.87855894)(834.3281163,232.90856111)
\curveto(834.32811531,232.94855887)(834.32311531,232.98355883)(834.3131163,233.01356111)
\lineto(834.3131163,233.16356111)
\curveto(834.30311533,233.20355861)(834.29811534,233.24855857)(834.2981163,233.29856111)
\curveto(834.30811533,233.35855846)(834.31311532,233.4135584)(834.3131163,233.46356111)
\lineto(834.3131163,234.06356111)
\lineto(834.3131163,236.82356111)
\lineto(834.3131163,237.78356111)
\lineto(834.3131163,238.05356111)
\curveto(834.31311532,238.14355367)(834.3331153,238.2185536)(834.3731163,238.27856111)
\curveto(834.41311522,238.34855347)(834.48811515,238.39855342)(834.5981163,238.42856111)
\curveto(834.61811502,238.43855338)(834.638115,238.43855338)(834.6581163,238.42856111)
\curveto(834.67811496,238.42855339)(834.69811494,238.43355338)(834.7181163,238.44356111)
}
}
{
\newrgbcolor{curcolor}{0 0 0}
\pscustom[linestyle=none,fillstyle=solid,fillcolor=curcolor]
{
\newpath
\moveto(850.02772567,234.78356111)
\curveto(850.03771732,234.73355708)(850.04271732,234.66855715)(850.04272567,234.58856111)
\curveto(850.04271732,234.50855731)(850.03771732,234.44355737)(850.02772567,234.39356111)
\curveto(850.00771735,234.34355747)(850.00271736,234.29355752)(850.01272567,234.24356111)
\curveto(850.02271734,234.20355761)(850.02271734,234.16355765)(850.01272567,234.12356111)
\curveto(850.01271735,234.05355776)(850.00771735,233.99855782)(849.99772567,233.95856111)
\curveto(849.97771738,233.86855795)(849.9627174,233.77855804)(849.95272567,233.68856111)
\curveto(849.95271741,233.59855822)(849.94271742,233.50855831)(849.92272567,233.41856111)
\lineto(849.86272567,233.17856111)
\curveto(849.84271752,233.10855871)(849.81771754,233.03355878)(849.78772567,232.95356111)
\curveto(849.66771769,232.58355923)(849.50271786,232.24855957)(849.29272567,231.94856111)
\curveto(849.23271813,231.85855996)(849.16771819,231.76856005)(849.09772567,231.67856111)
\curveto(849.02771833,231.59856022)(848.95271841,231.52356029)(848.87272567,231.45356111)
\lineto(848.79772567,231.37856111)
\curveto(848.72771863,231.32856049)(848.6627187,231.27856054)(848.60272567,231.22856111)
\curveto(848.54271882,231.17856064)(848.47271889,231.12856069)(848.39272567,231.07856111)
\curveto(848.28271908,230.99856082)(848.1577192,230.92856089)(848.01772567,230.86856111)
\curveto(847.88771947,230.818561)(847.75271961,230.76856105)(847.61272567,230.71856111)
\curveto(847.53271983,230.69856112)(847.45271991,230.68356113)(847.37272567,230.67356111)
\curveto(847.30272006,230.66356115)(847.22772013,230.64856117)(847.14772567,230.62856111)
\lineto(847.08772567,230.62856111)
\curveto(847.07772028,230.6185612)(847.0627203,230.6135612)(847.04272567,230.61356111)
\curveto(846.95272041,230.59356122)(846.81772054,230.58356123)(846.63772567,230.58356111)
\curveto(846.46772089,230.57356124)(846.33272103,230.57856124)(846.23272567,230.59856111)
\lineto(846.15772567,230.59856111)
\curveto(846.08772127,230.60856121)(846.02272134,230.6185612)(845.96272567,230.62856111)
\curveto(845.90272146,230.62856119)(845.84272152,230.63856118)(845.78272567,230.65856111)
\curveto(845.61272175,230.70856111)(845.45272191,230.75356106)(845.30272567,230.79356111)
\curveto(845.15272221,230.83356098)(845.01272235,230.89356092)(844.88272567,230.97356111)
\curveto(844.72272264,231.06356075)(844.58272278,231.15856066)(844.46272567,231.25856111)
\curveto(844.42272294,231.28856053)(844.362723,231.32856049)(844.28272567,231.37856111)
\curveto(844.20272316,231.43856038)(844.12772323,231.44356037)(844.05772567,231.39356111)
\curveto(844.01772334,231.36356045)(843.99772336,231.32356049)(843.99772567,231.27356111)
\curveto(843.99772336,231.22356059)(843.98772337,231.16856065)(843.96772567,231.10856111)
\curveto(843.9577234,231.07856074)(843.9577234,231.04356077)(843.96772567,231.00356111)
\curveto(843.97772338,230.97356084)(843.97772338,230.93856088)(843.96772567,230.89856111)
\curveto(843.94772341,230.83856098)(843.93772342,230.77356104)(843.93772567,230.70356111)
\curveto(843.94772341,230.62356119)(843.95272341,230.55356126)(843.95272567,230.49356111)
\lineto(843.95272567,228.69356111)
\lineto(843.95272567,228.25856111)
\curveto(843.95272341,228.10856371)(843.92272344,227.99356382)(843.86272567,227.91356111)
\curveto(843.81272355,227.84356397)(843.73272363,227.80856401)(843.62272567,227.80856111)
\curveto(843.51272385,227.79856402)(843.40272396,227.79356402)(843.29272567,227.79356111)
\lineto(843.05272567,227.79356111)
\curveto(842.98272438,227.813564)(842.92272444,227.83356398)(842.87272567,227.85356111)
\curveto(842.83272453,227.87356394)(842.79772456,227.90856391)(842.76772567,227.95856111)
\curveto(842.71772464,228.02856379)(842.69272467,228.13856368)(842.69272567,228.28856111)
\curveto(842.70272466,228.43856338)(842.70772465,228.56856325)(842.70772567,228.67856111)
\lineto(842.70772567,237.67856111)
\lineto(842.70772567,238.03856111)
\curveto(842.71772464,238.16855365)(842.74772461,238.27355354)(842.79772567,238.35356111)
\curveto(842.82772453,238.39355342)(842.89272447,238.42355339)(842.99272567,238.44356111)
\curveto(843.10272426,238.47355334)(843.21772414,238.48355333)(843.33772567,238.47356111)
\curveto(843.4577239,238.47355334)(843.56772379,238.45855336)(843.66772567,238.42856111)
\curveto(843.77772358,238.40855341)(843.84772351,238.37855344)(843.87772567,238.33856111)
\curveto(843.91772344,238.28855353)(843.93772342,238.22855359)(843.93772567,238.15856111)
\curveto(843.94772341,238.08855373)(843.96772339,238.0185538)(843.99772567,237.94856111)
\curveto(844.01772334,237.9185539)(844.03272333,237.89355392)(844.04272567,237.87356111)
\curveto(844.0627233,237.86355395)(844.08272328,237.84855397)(844.10272567,237.82856111)
\curveto(844.21272315,237.818554)(844.30272306,237.85355396)(844.37272567,237.93356111)
\curveto(844.45272291,238.0135538)(844.52772283,238.07855374)(844.59772567,238.12856111)
\curveto(844.8577225,238.30855351)(845.16772219,238.44855337)(845.52772567,238.54856111)
\curveto(845.61772174,238.56855325)(845.70772165,238.58355323)(845.79772567,238.59356111)
\curveto(845.89772146,238.60355321)(845.99772136,238.6185532)(846.09772567,238.63856111)
\curveto(846.13772122,238.64855317)(846.18772117,238.64855317)(846.24772567,238.63856111)
\curveto(846.30772105,238.62855319)(846.34772101,238.63355318)(846.36772567,238.65356111)
\curveto(846.79772056,238.66355315)(847.17772018,238.6185532)(847.50772567,238.51856111)
\curveto(847.83771952,238.42855339)(848.13271923,238.29855352)(848.39272567,238.12856111)
\lineto(848.54272567,238.00856111)
\curveto(848.59271877,237.97855384)(848.64271872,237.94355387)(848.69272567,237.90356111)
\curveto(848.71271865,237.88355393)(848.72771863,237.86355395)(848.73772567,237.84356111)
\curveto(848.7577186,237.83355398)(848.77771858,237.818554)(848.79772567,237.79856111)
\curveto(848.84771851,237.74855407)(848.90271846,237.69355412)(848.96272567,237.63356111)
\curveto(849.02271834,237.57355424)(849.07771828,237.5135543)(849.12772567,237.45356111)
\curveto(849.24771811,237.28355453)(849.37271799,237.09855472)(849.50272567,236.89856111)
\curveto(849.58271778,236.76855505)(849.64771771,236.62355519)(849.69772567,236.46356111)
\curveto(849.7577176,236.30355551)(849.81271755,236.14355567)(849.86272567,235.98356111)
\curveto(849.88271748,235.90355591)(849.89771746,235.818556)(849.90772567,235.72856111)
\curveto(849.92771743,235.63855618)(849.94771741,235.55355626)(849.96772567,235.47356111)
\lineto(849.96772567,235.35356111)
\curveto(849.97771738,235.32355649)(849.98271738,235.29355652)(849.98272567,235.26356111)
\curveto(850.00271736,235.2135566)(850.00771735,235.15855666)(849.99772567,235.09856111)
\curveto(849.99771736,235.03855678)(850.00771735,234.98355683)(850.02772567,234.93356111)
\lineto(850.02772567,234.78356111)
\moveto(848.69272567,234.37856111)
\curveto(848.71271865,234.42855739)(848.71771864,234.48855733)(848.70772567,234.55856111)
\curveto(848.69771866,234.63855718)(848.69271867,234.70855711)(848.69272567,234.76856111)
\curveto(848.69271867,234.93855688)(848.68271868,235.09855672)(848.66272567,235.24856111)
\curveto(848.65271871,235.39855642)(848.62271874,235.54355627)(848.57272567,235.68356111)
\lineto(848.51272567,235.86356111)
\curveto(848.50271886,235.93355588)(848.48271888,235.99855582)(848.45272567,236.05856111)
\curveto(848.34271902,236.32855549)(848.16771919,236.58855523)(847.92772567,236.83856111)
\curveto(847.69771966,237.08855473)(847.47771988,237.25855456)(847.26772567,237.34856111)
\curveto(847.18772017,237.38855443)(847.10272026,237.4185544)(847.01272567,237.43856111)
\curveto(846.93272043,237.45855436)(846.84772051,237.48355433)(846.75772567,237.51356111)
\curveto(846.66772069,237.53355428)(846.5627208,237.54355427)(846.44272567,237.54356111)
\lineto(846.11272567,237.54356111)
\curveto(846.09272127,237.52355429)(846.05272131,237.5135543)(845.99272567,237.51356111)
\curveto(845.94272142,237.52355429)(845.89772146,237.52355429)(845.85772567,237.51356111)
\lineto(845.58772567,237.45356111)
\curveto(845.50772185,237.43355438)(845.42772193,237.40355441)(845.34772567,237.36356111)
\curveto(845.02772233,237.22355459)(844.7627226,237.0185548)(844.55272567,236.74856111)
\curveto(844.35272301,236.48855533)(844.19772316,236.18355563)(844.08772567,235.83356111)
\curveto(844.04772331,235.72355609)(844.01772334,235.6135562)(843.99772567,235.50356111)
\curveto(843.98772337,235.39355642)(843.97272339,235.28355653)(843.95272567,235.17356111)
\curveto(843.94272342,235.13355668)(843.93772342,235.09355672)(843.93772567,235.05356111)
\curveto(843.93772342,235.02355679)(843.93272343,234.98855683)(843.92272567,234.94856111)
\lineto(843.92272567,234.82856111)
\curveto(843.91272345,234.77855704)(843.90772345,234.70355711)(843.90772567,234.60356111)
\curveto(843.90772345,234.5135573)(843.91272345,234.44355737)(843.92272567,234.39356111)
\lineto(843.92272567,234.27356111)
\curveto(843.93272343,234.23355758)(843.93772342,234.19355762)(843.93772567,234.15356111)
\curveto(843.93772342,234.1135577)(843.94272342,234.07855774)(843.95272567,234.04856111)
\curveto(843.9627234,234.0185578)(843.96772339,233.98855783)(843.96772567,233.95856111)
\curveto(843.96772339,233.92855789)(843.97272339,233.89355792)(843.98272567,233.85356111)
\curveto(844.00272336,233.77355804)(844.01772334,233.69355812)(844.02772567,233.61356111)
\lineto(844.08772567,233.37356111)
\curveto(844.19772316,233.03355878)(844.34772301,232.73355908)(844.53772567,232.47356111)
\curveto(844.73772262,232.22355959)(844.99772236,232.02855979)(845.31772567,231.88856111)
\curveto(845.50772185,231.80856001)(845.70272166,231.74856007)(845.90272567,231.70856111)
\curveto(845.94272142,231.68856013)(845.98272138,231.67856014)(846.02272567,231.67856111)
\curveto(846.0627213,231.68856013)(846.10272126,231.68856013)(846.14272567,231.67856111)
\lineto(846.26272567,231.67856111)
\curveto(846.33272103,231.65856016)(846.40272096,231.65856016)(846.47272567,231.67856111)
\lineto(846.59272567,231.67856111)
\curveto(846.70272066,231.69856012)(846.80772055,231.7135601)(846.90772567,231.72356111)
\curveto(847.00772035,231.73356008)(847.10772025,231.75856006)(847.20772567,231.79856111)
\curveto(847.51771984,231.92855989)(847.76771959,232.09855972)(847.95772567,232.30856111)
\curveto(848.1577192,232.52855929)(848.32271904,232.79355902)(848.45272567,233.10356111)
\curveto(848.50271886,233.24355857)(848.53771882,233.38355843)(848.55772567,233.52356111)
\curveto(848.58771877,233.67355814)(848.62271874,233.82855799)(848.66272567,233.98856111)
\curveto(848.67271869,234.03855778)(848.67771868,234.08355773)(848.67772567,234.12356111)
\curveto(848.67771868,234.16355765)(848.68271868,234.20855761)(848.69272567,234.25856111)
\lineto(848.69272567,234.37856111)
}
}
{
\newrgbcolor{curcolor}{0 0 0}
\pscustom[linestyle=none,fillstyle=solid,fillcolor=curcolor]
{
\newpath
\moveto(858.63397567,234.91856111)
\curveto(858.65396761,234.85855696)(858.6639676,234.76355705)(858.66397567,234.63356111)
\curveto(858.6639676,234.5135573)(858.65896761,234.42855739)(858.64897567,234.37856111)
\lineto(858.64897567,234.22856111)
\curveto(858.63896763,234.14855767)(858.62896764,234.07355774)(858.61897567,234.00356111)
\curveto(858.61896765,233.94355787)(858.61396765,233.87355794)(858.60397567,233.79356111)
\curveto(858.58396768,233.73355808)(858.5689677,233.67355814)(858.55897567,233.61356111)
\curveto(858.55896771,233.55355826)(858.54896772,233.49355832)(858.52897567,233.43356111)
\curveto(858.48896778,233.30355851)(858.45396781,233.17355864)(858.42397567,233.04356111)
\curveto(858.39396787,232.9135589)(858.35396791,232.79355902)(858.30397567,232.68356111)
\curveto(858.09396817,232.20355961)(857.81396845,231.79856002)(857.46397567,231.46856111)
\curveto(857.11396915,231.14856067)(856.68396958,230.90356091)(856.17397567,230.73356111)
\curveto(856.0639702,230.69356112)(855.94397032,230.66356115)(855.81397567,230.64356111)
\curveto(855.69397057,230.62356119)(855.5689707,230.60356121)(855.43897567,230.58356111)
\curveto(855.37897089,230.57356124)(855.31397095,230.56856125)(855.24397567,230.56856111)
\curveto(855.18397108,230.55856126)(855.12397114,230.55356126)(855.06397567,230.55356111)
\curveto(855.02397124,230.54356127)(854.9639713,230.53856128)(854.88397567,230.53856111)
\curveto(854.81397145,230.53856128)(854.7639715,230.54356127)(854.73397567,230.55356111)
\curveto(854.69397157,230.56356125)(854.65397161,230.56856125)(854.61397567,230.56856111)
\curveto(854.57397169,230.55856126)(854.53897173,230.55856126)(854.50897567,230.56856111)
\lineto(854.41897567,230.56856111)
\lineto(854.05897567,230.61356111)
\curveto(853.91897235,230.65356116)(853.78397248,230.69356112)(853.65397567,230.73356111)
\curveto(853.52397274,230.77356104)(853.39897287,230.818561)(853.27897567,230.86856111)
\curveto(852.82897344,231.06856075)(852.45897381,231.32856049)(852.16897567,231.64856111)
\curveto(851.87897439,231.96855985)(851.63897463,232.35855946)(851.44897567,232.81856111)
\curveto(851.39897487,232.9185589)(851.35897491,233.0185588)(851.32897567,233.11856111)
\curveto(851.30897496,233.2185586)(851.28897498,233.32355849)(851.26897567,233.43356111)
\curveto(851.24897502,233.47355834)(851.23897503,233.50355831)(851.23897567,233.52356111)
\curveto(851.24897502,233.55355826)(851.24897502,233.58855823)(851.23897567,233.62856111)
\curveto(851.21897505,233.70855811)(851.20397506,233.78855803)(851.19397567,233.86856111)
\curveto(851.19397507,233.95855786)(851.18397508,234.04355777)(851.16397567,234.12356111)
\lineto(851.16397567,234.24356111)
\curveto(851.1639751,234.28355753)(851.15897511,234.32855749)(851.14897567,234.37856111)
\curveto(851.13897513,234.42855739)(851.13397513,234.5135573)(851.13397567,234.63356111)
\curveto(851.13397513,234.76355705)(851.14397512,234.85855696)(851.16397567,234.91856111)
\curveto(851.18397508,234.98855683)(851.18897508,235.05855676)(851.17897567,235.12856111)
\curveto(851.1689751,235.19855662)(851.17397509,235.26855655)(851.19397567,235.33856111)
\curveto(851.20397506,235.38855643)(851.20897506,235.42855639)(851.20897567,235.45856111)
\curveto(851.21897505,235.49855632)(851.22897504,235.54355627)(851.23897567,235.59356111)
\curveto(851.268975,235.7135561)(851.29397497,235.83355598)(851.31397567,235.95356111)
\curveto(851.34397492,236.07355574)(851.38397488,236.18855563)(851.43397567,236.29856111)
\curveto(851.58397468,236.66855515)(851.7639745,236.99855482)(851.97397567,237.28856111)
\curveto(852.19397407,237.58855423)(852.45897381,237.83855398)(852.76897567,238.03856111)
\curveto(852.88897338,238.1185537)(853.01397325,238.18355363)(853.14397567,238.23356111)
\curveto(853.27397299,238.29355352)(853.40897286,238.35355346)(853.54897567,238.41356111)
\curveto(853.6689726,238.46355335)(853.79897247,238.49355332)(853.93897567,238.50356111)
\curveto(854.07897219,238.52355329)(854.21897205,238.55355326)(854.35897567,238.59356111)
\lineto(854.55397567,238.59356111)
\curveto(854.62397164,238.60355321)(854.68897158,238.6135532)(854.74897567,238.62356111)
\curveto(855.63897063,238.63355318)(856.37896989,238.44855337)(856.96897567,238.06856111)
\curveto(857.55896871,237.68855413)(857.98396828,237.19355462)(858.24397567,236.58356111)
\curveto(858.29396797,236.48355533)(858.33396793,236.38355543)(858.36397567,236.28356111)
\curveto(858.39396787,236.18355563)(858.42896784,236.07855574)(858.46897567,235.96856111)
\curveto(858.49896777,235.85855596)(858.52396774,235.73855608)(858.54397567,235.60856111)
\curveto(858.5639677,235.48855633)(858.58896768,235.36355645)(858.61897567,235.23356111)
\curveto(858.62896764,235.18355663)(858.62896764,235.12855669)(858.61897567,235.06856111)
\curveto(858.61896765,235.0185568)(858.62396764,234.96855685)(858.63397567,234.91856111)
\moveto(857.29897567,234.06356111)
\curveto(857.31896895,234.13355768)(857.32396894,234.2135576)(857.31397567,234.30356111)
\lineto(857.31397567,234.55856111)
\curveto(857.31396895,234.94855687)(857.27896899,235.27855654)(857.20897567,235.54856111)
\curveto(857.17896909,235.62855619)(857.15396911,235.70855611)(857.13397567,235.78856111)
\curveto(857.11396915,235.86855595)(857.08896918,235.94355587)(857.05897567,236.01356111)
\curveto(856.77896949,236.66355515)(856.33396993,237.1135547)(855.72397567,237.36356111)
\curveto(855.65397061,237.39355442)(855.57897069,237.4135544)(855.49897567,237.42356111)
\lineto(855.25897567,237.48356111)
\curveto(855.17897109,237.50355431)(855.09397117,237.5135543)(855.00397567,237.51356111)
\lineto(854.73397567,237.51356111)
\lineto(854.46397567,237.46856111)
\curveto(854.3639719,237.44855437)(854.268972,237.42355439)(854.17897567,237.39356111)
\curveto(854.09897217,237.37355444)(854.01897225,237.34355447)(853.93897567,237.30356111)
\curveto(853.8689724,237.28355453)(853.80397246,237.25355456)(853.74397567,237.21356111)
\curveto(853.68397258,237.17355464)(853.62897264,237.13355468)(853.57897567,237.09356111)
\curveto(853.33897293,236.92355489)(853.14397312,236.7185551)(852.99397567,236.47856111)
\curveto(852.84397342,236.23855558)(852.71397355,235.95855586)(852.60397567,235.63856111)
\curveto(852.57397369,235.53855628)(852.55397371,235.43355638)(852.54397567,235.32356111)
\curveto(852.53397373,235.22355659)(852.51897375,235.1185567)(852.49897567,235.00856111)
\curveto(852.48897378,234.96855685)(852.48397378,234.90355691)(852.48397567,234.81356111)
\curveto(852.47397379,234.78355703)(852.4689738,234.74855707)(852.46897567,234.70856111)
\curveto(852.47897379,234.66855715)(852.48397378,234.62355719)(852.48397567,234.57356111)
\lineto(852.48397567,234.27356111)
\curveto(852.48397378,234.17355764)(852.49397377,234.08355773)(852.51397567,234.00356111)
\lineto(852.54397567,233.82356111)
\curveto(852.5639737,233.72355809)(852.57897369,233.62355819)(852.58897567,233.52356111)
\curveto(852.60897366,233.43355838)(852.63897363,233.34855847)(852.67897567,233.26856111)
\curveto(852.77897349,233.02855879)(852.89397337,232.80355901)(853.02397567,232.59356111)
\curveto(853.1639731,232.38355943)(853.33397293,232.20855961)(853.53397567,232.06856111)
\curveto(853.58397268,232.03855978)(853.62897264,232.0135598)(853.66897567,231.99356111)
\curveto(853.70897256,231.97355984)(853.75397251,231.94855987)(853.80397567,231.91856111)
\curveto(853.88397238,231.86855995)(853.9689723,231.82355999)(854.05897567,231.78356111)
\curveto(854.15897211,231.75356006)(854.263972,231.72356009)(854.37397567,231.69356111)
\curveto(854.42397184,231.67356014)(854.4689718,231.66356015)(854.50897567,231.66356111)
\curveto(854.55897171,231.67356014)(854.60897166,231.67356014)(854.65897567,231.66356111)
\curveto(854.68897158,231.65356016)(854.74897152,231.64356017)(854.83897567,231.63356111)
\curveto(854.93897133,231.62356019)(855.01397125,231.62856019)(855.06397567,231.64856111)
\curveto(855.10397116,231.65856016)(855.14397112,231.65856016)(855.18397567,231.64856111)
\curveto(855.22397104,231.64856017)(855.263971,231.65856016)(855.30397567,231.67856111)
\curveto(855.38397088,231.69856012)(855.4639708,231.7135601)(855.54397567,231.72356111)
\curveto(855.62397064,231.74356007)(855.69897057,231.76856005)(855.76897567,231.79856111)
\curveto(856.10897016,231.93855988)(856.38396988,232.13355968)(856.59397567,232.38356111)
\curveto(856.80396946,232.63355918)(856.97896929,232.92855889)(857.11897567,233.26856111)
\curveto(857.1689691,233.38855843)(857.19896907,233.5135583)(857.20897567,233.64356111)
\curveto(857.22896904,233.78355803)(857.25896901,233.92355789)(857.29897567,234.06356111)
}
}
{
\newrgbcolor{curcolor}{0 0 0}
\pscustom[linestyle=none,fillstyle=solid,fillcolor=curcolor]
{
\newpath
\moveto(862.55225692,238.62356111)
\curveto(863.27225286,238.63355318)(863.87725225,238.54855327)(864.36725692,238.36856111)
\curveto(864.85725127,238.19855362)(865.23725089,237.89355392)(865.50725692,237.45356111)
\curveto(865.57725055,237.34355447)(865.6322505,237.22855459)(865.67225692,237.10856111)
\curveto(865.71225042,236.99855482)(865.75225038,236.87355494)(865.79225692,236.73356111)
\curveto(865.81225032,236.66355515)(865.81725031,236.58855523)(865.80725692,236.50856111)
\curveto(865.79725033,236.43855538)(865.78225035,236.38355543)(865.76225692,236.34356111)
\curveto(865.74225039,236.32355549)(865.71725041,236.30355551)(865.68725692,236.28356111)
\curveto(865.65725047,236.27355554)(865.6322505,236.25855556)(865.61225692,236.23856111)
\curveto(865.56225057,236.2185556)(865.51225062,236.2135556)(865.46225692,236.22356111)
\curveto(865.41225072,236.23355558)(865.36225077,236.23355558)(865.31225692,236.22356111)
\curveto(865.2322509,236.20355561)(865.127251,236.19855562)(864.99725692,236.20856111)
\curveto(864.86725126,236.22855559)(864.77725135,236.25355556)(864.72725692,236.28356111)
\curveto(864.64725148,236.33355548)(864.59225154,236.39855542)(864.56225692,236.47856111)
\curveto(864.54225159,236.56855525)(864.50725162,236.65355516)(864.45725692,236.73356111)
\curveto(864.36725176,236.89355492)(864.24225189,237.03855478)(864.08225692,237.16856111)
\curveto(863.97225216,237.24855457)(863.85225228,237.30855451)(863.72225692,237.34856111)
\curveto(863.59225254,237.38855443)(863.45225268,237.42855439)(863.30225692,237.46856111)
\curveto(863.25225288,237.48855433)(863.20225293,237.49355432)(863.15225692,237.48356111)
\curveto(863.10225303,237.48355433)(863.05225308,237.48855433)(863.00225692,237.49856111)
\curveto(862.94225319,237.5185543)(862.86725326,237.52855429)(862.77725692,237.52856111)
\curveto(862.68725344,237.52855429)(862.61225352,237.5185543)(862.55225692,237.49856111)
\lineto(862.46225692,237.49856111)
\lineto(862.31225692,237.46856111)
\curveto(862.26225387,237.46855435)(862.21225392,237.46355435)(862.16225692,237.45356111)
\curveto(861.90225423,237.39355442)(861.68725444,237.30855451)(861.51725692,237.19856111)
\curveto(861.34725478,237.08855473)(861.2322549,236.90355491)(861.17225692,236.64356111)
\curveto(861.15225498,236.57355524)(861.14725498,236.50355531)(861.15725692,236.43356111)
\curveto(861.17725495,236.36355545)(861.19725493,236.30355551)(861.21725692,236.25356111)
\curveto(861.27725485,236.10355571)(861.34725478,235.99355582)(861.42725692,235.92356111)
\curveto(861.51725461,235.86355595)(861.6272545,235.79355602)(861.75725692,235.71356111)
\curveto(861.91725421,235.6135562)(862.09725403,235.53855628)(862.29725692,235.48856111)
\curveto(862.49725363,235.44855637)(862.69725343,235.39855642)(862.89725692,235.33856111)
\curveto(863.0272531,235.29855652)(863.15725297,235.26855655)(863.28725692,235.24856111)
\curveto(863.41725271,235.22855659)(863.54725258,235.19855662)(863.67725692,235.15856111)
\curveto(863.88725224,235.09855672)(864.09225204,235.03855678)(864.29225692,234.97856111)
\curveto(864.49225164,234.92855689)(864.69225144,234.86355695)(864.89225692,234.78356111)
\lineto(865.04225692,234.72356111)
\curveto(865.09225104,234.70355711)(865.14225099,234.67855714)(865.19225692,234.64856111)
\curveto(865.39225074,234.52855729)(865.56725056,234.39355742)(865.71725692,234.24356111)
\curveto(865.86725026,234.09355772)(865.99225014,233.90355791)(866.09225692,233.67356111)
\curveto(866.11225002,233.60355821)(866.13225,233.50855831)(866.15225692,233.38856111)
\curveto(866.17224996,233.3185585)(866.18224995,233.24355857)(866.18225692,233.16356111)
\curveto(866.19224994,233.09355872)(866.19724993,233.0135588)(866.19725692,232.92356111)
\lineto(866.19725692,232.77356111)
\curveto(866.17724995,232.70355911)(866.16724996,232.63355918)(866.16725692,232.56356111)
\curveto(866.16724996,232.49355932)(866.15724997,232.42355939)(866.13725692,232.35356111)
\curveto(866.10725002,232.24355957)(866.07225006,232.13855968)(866.03225692,232.03856111)
\curveto(865.99225014,231.93855988)(865.94725018,231.84855997)(865.89725692,231.76856111)
\curveto(865.73725039,231.50856031)(865.5322506,231.29856052)(865.28225692,231.13856111)
\curveto(865.0322511,230.98856083)(864.75225138,230.85856096)(864.44225692,230.74856111)
\curveto(864.35225178,230.7185611)(864.25725187,230.69856112)(864.15725692,230.68856111)
\curveto(864.06725206,230.66856115)(863.97725215,230.64356117)(863.88725692,230.61356111)
\curveto(863.78725234,230.59356122)(863.68725244,230.58356123)(863.58725692,230.58356111)
\curveto(863.48725264,230.58356123)(863.38725274,230.57356124)(863.28725692,230.55356111)
\lineto(863.13725692,230.55356111)
\curveto(863.08725304,230.54356127)(863.01725311,230.53856128)(862.92725692,230.53856111)
\curveto(862.83725329,230.53856128)(862.76725336,230.54356127)(862.71725692,230.55356111)
\lineto(862.55225692,230.55356111)
\curveto(862.49225364,230.57356124)(862.4272537,230.58356123)(862.35725692,230.58356111)
\curveto(862.28725384,230.57356124)(862.2272539,230.57856124)(862.17725692,230.59856111)
\curveto(862.127254,230.60856121)(862.06225407,230.6135612)(861.98225692,230.61356111)
\lineto(861.74225692,230.67356111)
\curveto(861.67225446,230.68356113)(861.59725453,230.70356111)(861.51725692,230.73356111)
\curveto(861.20725492,230.83356098)(860.93725519,230.95856086)(860.70725692,231.10856111)
\curveto(860.47725565,231.25856056)(860.27725585,231.45356036)(860.10725692,231.69356111)
\curveto(860.01725611,231.82355999)(859.94225619,231.95855986)(859.88225692,232.09856111)
\curveto(859.82225631,232.23855958)(859.76725636,232.39355942)(859.71725692,232.56356111)
\curveto(859.69725643,232.62355919)(859.68725644,232.69355912)(859.68725692,232.77356111)
\curveto(859.69725643,232.86355895)(859.71225642,232.93355888)(859.73225692,232.98356111)
\curveto(859.76225637,233.02355879)(859.81225632,233.06355875)(859.88225692,233.10356111)
\curveto(859.9322562,233.12355869)(860.00225613,233.13355868)(860.09225692,233.13356111)
\curveto(860.18225595,233.14355867)(860.27225586,233.14355867)(860.36225692,233.13356111)
\curveto(860.45225568,233.12355869)(860.53725559,233.10855871)(860.61725692,233.08856111)
\curveto(860.70725542,233.07855874)(860.76725536,233.06355875)(860.79725692,233.04356111)
\curveto(860.86725526,232.99355882)(860.91225522,232.9185589)(860.93225692,232.81856111)
\curveto(860.96225517,232.72855909)(860.99725513,232.64355917)(861.03725692,232.56356111)
\curveto(861.13725499,232.34355947)(861.27225486,232.17355964)(861.44225692,232.05356111)
\curveto(861.56225457,231.96355985)(861.69725443,231.89355992)(861.84725692,231.84356111)
\curveto(861.99725413,231.79356002)(862.15725397,231.74356007)(862.32725692,231.69356111)
\lineto(862.64225692,231.64856111)
\lineto(862.73225692,231.64856111)
\curveto(862.80225333,231.62856019)(862.89225324,231.6185602)(863.00225692,231.61856111)
\curveto(863.12225301,231.6185602)(863.22225291,231.62856019)(863.30225692,231.64856111)
\curveto(863.37225276,231.64856017)(863.4272527,231.65356016)(863.46725692,231.66356111)
\curveto(863.5272526,231.67356014)(863.58725254,231.67856014)(863.64725692,231.67856111)
\curveto(863.70725242,231.68856013)(863.76225237,231.69856012)(863.81225692,231.70856111)
\curveto(864.10225203,231.78856003)(864.3322518,231.89355992)(864.50225692,232.02356111)
\curveto(864.67225146,232.15355966)(864.79225134,232.37355944)(864.86225692,232.68356111)
\curveto(864.88225125,232.73355908)(864.88725124,232.78855903)(864.87725692,232.84856111)
\curveto(864.86725126,232.90855891)(864.85725127,232.95355886)(864.84725692,232.98356111)
\curveto(864.79725133,233.17355864)(864.7272514,233.3135585)(864.63725692,233.40356111)
\curveto(864.54725158,233.50355831)(864.4322517,233.59355822)(864.29225692,233.67356111)
\curveto(864.20225193,233.73355808)(864.10225203,233.78355803)(863.99225692,233.82356111)
\lineto(863.66225692,233.94356111)
\curveto(863.6322525,233.95355786)(863.60225253,233.95855786)(863.57225692,233.95856111)
\curveto(863.55225258,233.95855786)(863.5272526,233.96855785)(863.49725692,233.98856111)
\curveto(863.15725297,234.09855772)(862.80225333,234.17855764)(862.43225692,234.22856111)
\curveto(862.07225406,234.28855753)(861.7322544,234.38355743)(861.41225692,234.51356111)
\curveto(861.31225482,234.55355726)(861.21725491,234.58855723)(861.12725692,234.61856111)
\curveto(861.03725509,234.64855717)(860.95225518,234.68855713)(860.87225692,234.73856111)
\curveto(860.68225545,234.84855697)(860.50725562,234.97355684)(860.34725692,235.11356111)
\curveto(860.18725594,235.25355656)(860.06225607,235.42855639)(859.97225692,235.63856111)
\curveto(859.94225619,235.70855611)(859.91725621,235.77855604)(859.89725692,235.84856111)
\curveto(859.88725624,235.9185559)(859.87225626,235.99355582)(859.85225692,236.07356111)
\curveto(859.82225631,236.19355562)(859.81225632,236.32855549)(859.82225692,236.47856111)
\curveto(859.8322563,236.63855518)(859.84725628,236.77355504)(859.86725692,236.88356111)
\curveto(859.88725624,236.93355488)(859.89725623,236.97355484)(859.89725692,237.00356111)
\curveto(859.90725622,237.04355477)(859.92225621,237.08355473)(859.94225692,237.12356111)
\curveto(860.0322561,237.35355446)(860.15225598,237.55355426)(860.30225692,237.72356111)
\curveto(860.46225567,237.89355392)(860.64225549,238.04355377)(860.84225692,238.17356111)
\curveto(860.99225514,238.26355355)(861.15725497,238.33355348)(861.33725692,238.38356111)
\curveto(861.51725461,238.44355337)(861.70725442,238.49855332)(861.90725692,238.54856111)
\curveto(861.97725415,238.55855326)(862.04225409,238.56855325)(862.10225692,238.57856111)
\curveto(862.17225396,238.58855323)(862.24725388,238.59855322)(862.32725692,238.60856111)
\curveto(862.35725377,238.6185532)(862.39725373,238.6185532)(862.44725692,238.60856111)
\curveto(862.49725363,238.59855322)(862.5322536,238.60355321)(862.55225692,238.62356111)
}
}
{
\newrgbcolor{curcolor}{0.50196081 0.50196081 0.50196081}
\pscustom[linestyle=none,fillstyle=solid,fillcolor=curcolor]
{
\newpath
\moveto(798.51865829,241.42859773)
\lineto(813.51865829,241.42859773)
\lineto(813.51865829,226.42859773)
\lineto(798.51865829,226.42859773)
\closepath
}
}
{
\newrgbcolor{curcolor}{0 0 0}
\pscustom[linestyle=none,fillstyle=solid,fillcolor=curcolor]
{
\newpath
\moveto(822.4518663,218.78138338)
\curveto(823.4318598,218.80137242)(824.25185898,218.64137258)(824.9118663,218.30138338)
\curveto(825.58185765,217.97137325)(826.10185713,217.51137371)(826.4718663,216.92138338)
\curveto(826.57185666,216.76137446)(826.65185658,216.60637462)(826.7118663,216.45638338)
\curveto(826.78185645,216.31637491)(826.84685638,216.14637508)(826.9068663,215.94638338)
\curveto(826.9268563,215.89637533)(826.94685628,215.8263754)(826.9668663,215.73638338)
\curveto(826.98685624,215.65637557)(826.98185625,215.58137564)(826.9518663,215.51138338)
\curveto(826.9318563,215.45137577)(826.89185634,215.41137581)(826.8318663,215.39138338)
\curveto(826.78185645,215.38137584)(826.7268565,215.36637586)(826.6668663,215.34638338)
\lineto(826.5168663,215.34638338)
\curveto(826.48685674,215.33637589)(826.44685678,215.33137589)(826.3968663,215.33138338)
\lineto(826.2768663,215.33138338)
\curveto(826.13685709,215.33137589)(826.00685722,215.33637589)(825.8868663,215.34638338)
\curveto(825.77685745,215.36637586)(825.69685753,215.41637581)(825.6468663,215.49638338)
\curveto(825.57685765,215.59637563)(825.52185771,215.71137551)(825.4818663,215.84138338)
\curveto(825.44185779,215.97137525)(825.38685784,216.09137513)(825.3168663,216.20138338)
\curveto(825.18685804,216.4213748)(825.03685819,216.61137461)(824.8668663,216.77138338)
\curveto(824.70685852,216.93137429)(824.51685871,217.08137414)(824.2968663,217.22138338)
\curveto(824.17685905,217.30137392)(824.04185919,217.36137386)(823.8918663,217.40138338)
\curveto(823.75185948,217.44137378)(823.60685962,217.48137374)(823.4568663,217.52138338)
\curveto(823.34685988,217.55137367)(823.22186001,217.57137365)(823.0818663,217.58138338)
\curveto(822.94186029,217.60137362)(822.79186044,217.61137361)(822.6318663,217.61138338)
\curveto(822.48186075,217.61137361)(822.3318609,217.60137362)(822.1818663,217.58138338)
\curveto(822.04186119,217.57137365)(821.92186131,217.55137367)(821.8218663,217.52138338)
\curveto(821.72186151,217.50137372)(821.6268616,217.48137374)(821.5368663,217.46138338)
\curveto(821.44686178,217.44137378)(821.35686187,217.41137381)(821.2668663,217.37138338)
\curveto(820.4268628,217.0213742)(819.82186341,216.4213748)(819.4518663,215.57138338)
\curveto(819.38186385,215.43137579)(819.32186391,215.28137594)(819.2718663,215.12138338)
\curveto(819.231864,214.97137625)(819.18686404,214.81637641)(819.1368663,214.65638338)
\curveto(819.11686411,214.59637663)(819.10686412,214.53137669)(819.1068663,214.46138338)
\curveto(819.10686412,214.40137682)(819.09686413,214.34137688)(819.0768663,214.28138338)
\curveto(819.06686416,214.24137698)(819.06186417,214.20637702)(819.0618663,214.17638338)
\curveto(819.06186417,214.14637708)(819.05686417,214.11137711)(819.0468663,214.07138338)
\curveto(819.0268642,213.96137726)(819.01186422,213.84637738)(819.0018663,213.72638338)
\lineto(819.0018663,213.38138338)
\curveto(819.00186423,213.31137791)(818.99686423,213.23637799)(818.9868663,213.15638338)
\curveto(818.98686424,213.08637814)(818.99186424,213.0213782)(819.0018663,212.96138338)
\lineto(819.0018663,212.81138338)
\curveto(819.02186421,212.74137848)(819.0268642,212.67137855)(819.0168663,212.60138338)
\curveto(819.01686421,212.53137869)(819.0268642,212.46137876)(819.0468663,212.39138338)
\curveto(819.06686416,212.33137889)(819.07186416,212.27137895)(819.0618663,212.21138338)
\curveto(819.06186417,212.15137907)(819.07186416,212.09637913)(819.0918663,212.04638338)
\curveto(819.12186411,211.91637931)(819.14686408,211.78637944)(819.1668663,211.65638338)
\curveto(819.19686403,211.53637969)(819.231864,211.41637981)(819.2718663,211.29638338)
\curveto(819.44186379,210.79638043)(819.66186357,210.36638086)(819.9318663,210.00638338)
\curveto(820.20186303,209.65638157)(820.55686267,209.36638186)(820.9968663,209.13638338)
\curveto(821.13686209,209.06638216)(821.27686195,209.01138221)(821.4168663,208.97138338)
\curveto(821.56686166,208.93138229)(821.7268615,208.88638234)(821.8968663,208.83638338)
\curveto(821.96686126,208.81638241)(822.0318612,208.80638242)(822.0918663,208.80638338)
\curveto(822.15186108,208.81638241)(822.22186101,208.81138241)(822.3018663,208.79138338)
\curveto(822.35186088,208.78138244)(822.44186079,208.77138245)(822.5718663,208.76138338)
\curveto(822.70186053,208.76138246)(822.79686043,208.77138245)(822.8568663,208.79138338)
\lineto(822.9618663,208.79138338)
\curveto(823.00186023,208.80138242)(823.04186019,208.80138242)(823.0818663,208.79138338)
\curveto(823.12186011,208.79138243)(823.16186007,208.80138242)(823.2018663,208.82138338)
\curveto(823.30185993,208.84138238)(823.39685983,208.85638237)(823.4868663,208.86638338)
\curveto(823.58685964,208.88638234)(823.68185955,208.91638231)(823.7718663,208.95638338)
\curveto(824.55185868,209.27638195)(825.10185813,209.80138142)(825.4218663,210.53138338)
\curveto(825.50185773,210.71138051)(825.57685765,210.9263803)(825.6468663,211.17638338)
\curveto(825.66685756,211.26637996)(825.68185755,211.35637987)(825.6918663,211.44638338)
\curveto(825.71185752,211.54637968)(825.74685748,211.63637959)(825.7968663,211.71638338)
\curveto(825.84685738,211.79637943)(825.9268573,211.84137938)(826.0368663,211.85138338)
\curveto(826.14685708,211.86137936)(826.26685696,211.86637936)(826.3968663,211.86638338)
\lineto(826.5468663,211.86638338)
\curveto(826.59685663,211.86637936)(826.64185659,211.86137936)(826.6818663,211.85138338)
\lineto(826.7868663,211.85138338)
\lineto(826.8768663,211.82138338)
\curveto(826.91685631,211.8213794)(826.94685628,211.81137941)(826.9668663,211.79138338)
\curveto(827.03685619,211.75137947)(827.07685615,211.67637955)(827.0868663,211.56638338)
\curveto(827.09685613,211.46637976)(827.08685614,211.36637986)(827.0568663,211.26638338)
\curveto(826.99685623,211.03638019)(826.94185629,210.81638041)(826.8918663,210.60638338)
\curveto(826.84185639,210.39638083)(826.76685646,210.19638103)(826.6668663,210.00638338)
\curveto(826.58685664,209.87638135)(826.51185672,209.75138147)(826.4418663,209.63138338)
\curveto(826.38185685,209.51138171)(826.31185692,209.39138183)(826.2318663,209.27138338)
\curveto(826.05185718,209.01138221)(825.8268574,208.77138245)(825.5568663,208.55138338)
\curveto(825.29685793,208.34138288)(825.01185822,208.16638306)(824.7018663,208.02638338)
\curveto(824.59185864,207.97638325)(824.48185875,207.93638329)(824.3718663,207.90638338)
\curveto(824.27185896,207.87638335)(824.16685906,207.84138338)(824.0568663,207.80138338)
\curveto(823.94685928,207.76138346)(823.8318594,207.73638349)(823.7118663,207.72638338)
\curveto(823.60185963,207.70638352)(823.48685974,207.68638354)(823.3668663,207.66638338)
\curveto(823.31685991,207.64638358)(823.27185996,207.64138358)(823.2318663,207.65138338)
\curveto(823.19186004,207.65138357)(823.15186008,207.64638358)(823.1118663,207.63638338)
\curveto(823.05186018,207.6263836)(822.99186024,207.6213836)(822.9318663,207.62138338)
\curveto(822.87186036,207.6213836)(822.80686042,207.61638361)(822.7368663,207.60638338)
\curveto(822.70686052,207.59638363)(822.63686059,207.59638363)(822.5268663,207.60638338)
\curveto(822.4268608,207.60638362)(822.36186087,207.61138361)(822.3318663,207.62138338)
\curveto(822.28186095,207.63138359)(822.231861,207.63638359)(822.1818663,207.63638338)
\curveto(822.14186109,207.6263836)(822.09686113,207.6263836)(822.0468663,207.63638338)
\lineto(821.8968663,207.63638338)
\curveto(821.81686141,207.65638357)(821.74186149,207.67138355)(821.6718663,207.68138338)
\curveto(821.60186163,207.68138354)(821.5268617,207.69138353)(821.4468663,207.71138338)
\lineto(821.1768663,207.77138338)
\curveto(821.08686214,207.78138344)(821.00186223,207.80138342)(820.9218663,207.83138338)
\curveto(820.71186252,207.89138333)(820.52186271,207.96638326)(820.3518663,208.05638338)
\curveto(819.72186351,208.3263829)(819.21186402,208.71138251)(818.8218663,209.21138338)
\curveto(818.4318648,209.71138151)(818.12186511,210.30138092)(817.8918663,210.98138338)
\curveto(817.85186538,211.10138012)(817.81686541,211.22638)(817.7868663,211.35638338)
\curveto(817.76686546,211.48637974)(817.74186549,211.6213796)(817.7118663,211.76138338)
\curveto(817.69186554,211.81137941)(817.68186555,211.85637937)(817.6818663,211.89638338)
\curveto(817.69186554,211.93637929)(817.69186554,211.98137924)(817.6818663,212.03138338)
\curveto(817.66186557,212.1213791)(817.64686558,212.21637901)(817.6368663,212.31638338)
\curveto(817.63686559,212.41637881)(817.6268656,212.51137871)(817.6068663,212.60138338)
\lineto(817.6068663,212.88638338)
\curveto(817.58686564,212.93637829)(817.57686565,213.0213782)(817.5768663,213.14138338)
\curveto(817.57686565,213.26137796)(817.58686564,213.34637788)(817.6068663,213.39638338)
\curveto(817.61686561,213.4263778)(817.61686561,213.45637777)(817.6068663,213.48638338)
\curveto(817.59686563,213.5263777)(817.59686563,213.55637767)(817.6068663,213.57638338)
\lineto(817.6068663,213.71138338)
\curveto(817.61686561,213.79137743)(817.62186561,213.87137735)(817.6218663,213.95138338)
\curveto(817.6318656,214.04137718)(817.64686558,214.1263771)(817.6668663,214.20638338)
\curveto(817.68686554,214.26637696)(817.69686553,214.3263769)(817.6968663,214.38638338)
\curveto(817.69686553,214.45637677)(817.70686552,214.5263767)(817.7268663,214.59638338)
\curveto(817.77686545,214.76637646)(817.81686541,214.93137629)(817.8468663,215.09138338)
\curveto(817.87686535,215.25137597)(817.92186531,215.40137582)(817.9818663,215.54138338)
\lineto(818.1318663,215.93138338)
\curveto(818.19186504,216.07137515)(818.25686497,216.19637503)(818.3268663,216.30638338)
\curveto(818.47686475,216.56637466)(818.6268646,216.80137442)(818.7768663,217.01138338)
\curveto(818.80686442,217.06137416)(818.84186439,217.10137412)(818.8818663,217.13138338)
\curveto(818.9318643,217.17137405)(818.97186426,217.21637401)(819.0018663,217.26638338)
\curveto(819.06186417,217.34637388)(819.12186411,217.41637381)(819.1818663,217.47638338)
\lineto(819.3918663,217.65638338)
\curveto(819.45186378,217.70637352)(819.50686372,217.75137347)(819.5568663,217.79138338)
\curveto(819.61686361,217.84137338)(819.68186355,217.89137333)(819.7518663,217.94138338)
\curveto(819.90186333,218.05137317)(820.05686317,218.14637308)(820.2168663,218.22638338)
\curveto(820.38686284,218.30637292)(820.56186267,218.38637284)(820.7418663,218.46638338)
\curveto(820.85186238,218.51637271)(820.96686226,218.55137267)(821.0868663,218.57138338)
\curveto(821.21686201,218.60137262)(821.34186189,218.63637259)(821.4618663,218.67638338)
\curveto(821.5318617,218.68637254)(821.59686163,218.69637253)(821.6568663,218.70638338)
\lineto(821.8368663,218.73638338)
\curveto(821.91686131,218.74637248)(821.99186124,218.75137247)(822.0618663,218.75138338)
\curveto(822.14186109,218.76137246)(822.22186101,218.77137245)(822.3018663,218.78138338)
\curveto(822.32186091,218.79137243)(822.34686088,218.79137243)(822.3768663,218.78138338)
\curveto(822.40686082,218.77137245)(822.4318608,218.77137245)(822.4518663,218.78138338)
}
}
{
\newrgbcolor{curcolor}{0 0 0}
\pscustom[linestyle=none,fillstyle=solid,fillcolor=curcolor]
{
\newpath
\moveto(835.81171005,212.06138338)
\curveto(835.83170199,212.00137922)(835.84170198,211.90637932)(835.84171005,211.77638338)
\curveto(835.84170198,211.65637957)(835.83670198,211.57137965)(835.82671005,211.52138338)
\lineto(835.82671005,211.37138338)
\curveto(835.816702,211.29137993)(835.80670201,211.21638001)(835.79671005,211.14638338)
\curveto(835.79670202,211.08638014)(835.79170203,211.01638021)(835.78171005,210.93638338)
\curveto(835.76170206,210.87638035)(835.74670207,210.81638041)(835.73671005,210.75638338)
\curveto(835.73670208,210.69638053)(835.72670209,210.63638059)(835.70671005,210.57638338)
\curveto(835.66670215,210.44638078)(835.63170219,210.31638091)(835.60171005,210.18638338)
\curveto(835.57170225,210.05638117)(835.53170229,209.93638129)(835.48171005,209.82638338)
\curveto(835.27170255,209.34638188)(834.99170283,208.94138228)(834.64171005,208.61138338)
\curveto(834.29170353,208.29138293)(833.86170396,208.04638318)(833.35171005,207.87638338)
\curveto(833.24170458,207.83638339)(833.1217047,207.80638342)(832.99171005,207.78638338)
\curveto(832.87170495,207.76638346)(832.74670507,207.74638348)(832.61671005,207.72638338)
\curveto(832.55670526,207.71638351)(832.49170533,207.71138351)(832.42171005,207.71138338)
\curveto(832.36170546,207.70138352)(832.30170552,207.69638353)(832.24171005,207.69638338)
\curveto(832.20170562,207.68638354)(832.14170568,207.68138354)(832.06171005,207.68138338)
\curveto(831.99170583,207.68138354)(831.94170588,207.68638354)(831.91171005,207.69638338)
\curveto(831.87170595,207.70638352)(831.83170599,207.71138351)(831.79171005,207.71138338)
\curveto(831.75170607,207.70138352)(831.7167061,207.70138352)(831.68671005,207.71138338)
\lineto(831.59671005,207.71138338)
\lineto(831.23671005,207.75638338)
\curveto(831.09670672,207.79638343)(830.96170686,207.83638339)(830.83171005,207.87638338)
\curveto(830.70170712,207.91638331)(830.57670724,207.96138326)(830.45671005,208.01138338)
\curveto(830.00670781,208.21138301)(829.63670818,208.47138275)(829.34671005,208.79138338)
\curveto(829.05670876,209.11138211)(828.816709,209.50138172)(828.62671005,209.96138338)
\curveto(828.57670924,210.06138116)(828.53670928,210.16138106)(828.50671005,210.26138338)
\curveto(828.48670933,210.36138086)(828.46670935,210.46638076)(828.44671005,210.57638338)
\curveto(828.42670939,210.61638061)(828.4167094,210.64638058)(828.41671005,210.66638338)
\curveto(828.42670939,210.69638053)(828.42670939,210.73138049)(828.41671005,210.77138338)
\curveto(828.39670942,210.85138037)(828.38170944,210.93138029)(828.37171005,211.01138338)
\curveto(828.37170945,211.10138012)(828.36170946,211.18638004)(828.34171005,211.26638338)
\lineto(828.34171005,211.38638338)
\curveto(828.34170948,211.4263798)(828.33670948,211.47137975)(828.32671005,211.52138338)
\curveto(828.3167095,211.57137965)(828.31170951,211.65637957)(828.31171005,211.77638338)
\curveto(828.31170951,211.90637932)(828.3217095,212.00137922)(828.34171005,212.06138338)
\curveto(828.36170946,212.13137909)(828.36670945,212.20137902)(828.35671005,212.27138338)
\curveto(828.34670947,212.34137888)(828.35170947,212.41137881)(828.37171005,212.48138338)
\curveto(828.38170944,212.53137869)(828.38670943,212.57137865)(828.38671005,212.60138338)
\curveto(828.39670942,212.64137858)(828.40670941,212.68637854)(828.41671005,212.73638338)
\curveto(828.44670937,212.85637837)(828.47170935,212.97637825)(828.49171005,213.09638338)
\curveto(828.5217093,213.21637801)(828.56170926,213.33137789)(828.61171005,213.44138338)
\curveto(828.76170906,213.81137741)(828.94170888,214.14137708)(829.15171005,214.43138338)
\curveto(829.37170845,214.73137649)(829.63670818,214.98137624)(829.94671005,215.18138338)
\curveto(830.06670775,215.26137596)(830.19170763,215.3263759)(830.32171005,215.37638338)
\curveto(830.45170737,215.43637579)(830.58670723,215.49637573)(830.72671005,215.55638338)
\curveto(830.84670697,215.60637562)(830.97670684,215.63637559)(831.11671005,215.64638338)
\curveto(831.25670656,215.66637556)(831.39670642,215.69637553)(831.53671005,215.73638338)
\lineto(831.73171005,215.73638338)
\curveto(831.80170602,215.74637548)(831.86670595,215.75637547)(831.92671005,215.76638338)
\curveto(832.816705,215.77637545)(833.55670426,215.59137563)(834.14671005,215.21138338)
\curveto(834.73670308,214.83137639)(835.16170266,214.33637689)(835.42171005,213.72638338)
\curveto(835.47170235,213.6263776)(835.51170231,213.5263777)(835.54171005,213.42638338)
\curveto(835.57170225,213.3263779)(835.60670221,213.221378)(835.64671005,213.11138338)
\curveto(835.67670214,213.00137822)(835.70170212,212.88137834)(835.72171005,212.75138338)
\curveto(835.74170208,212.63137859)(835.76670205,212.50637872)(835.79671005,212.37638338)
\curveto(835.80670201,212.3263789)(835.80670201,212.27137895)(835.79671005,212.21138338)
\curveto(835.79670202,212.16137906)(835.80170202,212.11137911)(835.81171005,212.06138338)
\moveto(834.47671005,211.20638338)
\curveto(834.49670332,211.27637995)(834.50170332,211.35637987)(834.49171005,211.44638338)
\lineto(834.49171005,211.70138338)
\curveto(834.49170333,212.09137913)(834.45670336,212.4213788)(834.38671005,212.69138338)
\curveto(834.35670346,212.77137845)(834.33170349,212.85137837)(834.31171005,212.93138338)
\curveto(834.29170353,213.01137821)(834.26670355,213.08637814)(834.23671005,213.15638338)
\curveto(833.95670386,213.80637742)(833.51170431,214.25637697)(832.90171005,214.50638338)
\curveto(832.83170499,214.53637669)(832.75670506,214.55637667)(832.67671005,214.56638338)
\lineto(832.43671005,214.62638338)
\curveto(832.35670546,214.64637658)(832.27170555,214.65637657)(832.18171005,214.65638338)
\lineto(831.91171005,214.65638338)
\lineto(831.64171005,214.61138338)
\curveto(831.54170628,214.59137663)(831.44670637,214.56637666)(831.35671005,214.53638338)
\curveto(831.27670654,214.51637671)(831.19670662,214.48637674)(831.11671005,214.44638338)
\curveto(831.04670677,214.4263768)(830.98170684,214.39637683)(830.92171005,214.35638338)
\curveto(830.86170696,214.31637691)(830.80670701,214.27637695)(830.75671005,214.23638338)
\curveto(830.5167073,214.06637716)(830.3217075,213.86137736)(830.17171005,213.62138338)
\curveto(830.0217078,213.38137784)(829.89170793,213.10137812)(829.78171005,212.78138338)
\curveto(829.75170807,212.68137854)(829.73170809,212.57637865)(829.72171005,212.46638338)
\curveto(829.71170811,212.36637886)(829.69670812,212.26137896)(829.67671005,212.15138338)
\curveto(829.66670815,212.11137911)(829.66170816,212.04637918)(829.66171005,211.95638338)
\curveto(829.65170817,211.9263793)(829.64670817,211.89137933)(829.64671005,211.85138338)
\curveto(829.65670816,211.81137941)(829.66170816,211.76637946)(829.66171005,211.71638338)
\lineto(829.66171005,211.41638338)
\curveto(829.66170816,211.31637991)(829.67170815,211.22638)(829.69171005,211.14638338)
\lineto(829.72171005,210.96638338)
\curveto(829.74170808,210.86638036)(829.75670806,210.76638046)(829.76671005,210.66638338)
\curveto(829.78670803,210.57638065)(829.816708,210.49138073)(829.85671005,210.41138338)
\curveto(829.95670786,210.17138105)(830.07170775,209.94638128)(830.20171005,209.73638338)
\curveto(830.34170748,209.5263817)(830.51170731,209.35138187)(830.71171005,209.21138338)
\curveto(830.76170706,209.18138204)(830.80670701,209.15638207)(830.84671005,209.13638338)
\curveto(830.88670693,209.11638211)(830.93170689,209.09138213)(830.98171005,209.06138338)
\curveto(831.06170676,209.01138221)(831.14670667,208.96638226)(831.23671005,208.92638338)
\curveto(831.33670648,208.89638233)(831.44170638,208.86638236)(831.55171005,208.83638338)
\curveto(831.60170622,208.81638241)(831.64670617,208.80638242)(831.68671005,208.80638338)
\curveto(831.73670608,208.81638241)(831.78670603,208.81638241)(831.83671005,208.80638338)
\curveto(831.86670595,208.79638243)(831.92670589,208.78638244)(832.01671005,208.77638338)
\curveto(832.1167057,208.76638246)(832.19170563,208.77138245)(832.24171005,208.79138338)
\curveto(832.28170554,208.80138242)(832.3217055,208.80138242)(832.36171005,208.79138338)
\curveto(832.40170542,208.79138243)(832.44170538,208.80138242)(832.48171005,208.82138338)
\curveto(832.56170526,208.84138238)(832.64170518,208.85638237)(832.72171005,208.86638338)
\curveto(832.80170502,208.88638234)(832.87670494,208.91138231)(832.94671005,208.94138338)
\curveto(833.28670453,209.08138214)(833.56170426,209.27638195)(833.77171005,209.52638338)
\curveto(833.98170384,209.77638145)(834.15670366,210.07138115)(834.29671005,210.41138338)
\curveto(834.34670347,210.53138069)(834.37670344,210.65638057)(834.38671005,210.78638338)
\curveto(834.40670341,210.9263803)(834.43670338,211.06638016)(834.47671005,211.20638338)
}
}
{
\newrgbcolor{curcolor}{0 0 0}
\pscustom[linestyle=none,fillstyle=solid,fillcolor=curcolor]
{
\newpath
\moveto(840.9899913,215.76638338)
\curveto(841.36998631,215.77637545)(841.68998599,215.73637549)(841.9499913,215.64638338)
\curveto(842.21998546,215.55637567)(842.46498522,215.4263758)(842.6849913,215.25638338)
\curveto(842.76498492,215.20637602)(842.82998485,215.13637609)(842.8799913,215.04638338)
\curveto(842.93998474,214.96637626)(843.00498468,214.89137633)(843.0749913,214.82138338)
\curveto(843.09498459,214.80137642)(843.12498456,214.77637645)(843.1649913,214.74638338)
\curveto(843.20498448,214.71637651)(843.25498443,214.70637652)(843.3149913,214.71638338)
\curveto(843.41498427,214.74637648)(843.49998418,214.80637642)(843.5699913,214.89638338)
\curveto(843.64998403,214.99637623)(843.72998395,215.07137615)(843.8099913,215.12138338)
\curveto(843.94998373,215.23137599)(844.09498359,215.3263759)(844.2449913,215.40638338)
\curveto(844.39498329,215.49637573)(844.55998312,215.57137565)(844.7399913,215.63138338)
\curveto(844.81998286,215.66137556)(844.90498278,215.68137554)(844.9949913,215.69138338)
\curveto(845.09498259,215.71137551)(845.18998249,215.73137549)(845.2799913,215.75138338)
\curveto(845.32998235,215.76137546)(845.37498231,215.76637546)(845.4149913,215.76638338)
\lineto(845.5649913,215.76638338)
\curveto(845.61498207,215.78637544)(845.684982,215.79137543)(845.7749913,215.78138338)
\curveto(845.86498182,215.78137544)(845.92998175,215.77637545)(845.9699913,215.76638338)
\curveto(846.01998166,215.75637547)(846.09498159,215.75137547)(846.1949913,215.75138338)
\curveto(846.2849814,215.73137549)(846.36998131,215.71137551)(846.4499913,215.69138338)
\curveto(846.53998114,215.68137554)(846.62498106,215.66137556)(846.7049913,215.63138338)
\curveto(846.75498093,215.61137561)(846.79998088,215.59637563)(846.8399913,215.58638338)
\curveto(846.88998079,215.58637564)(846.93998074,215.57637565)(846.9899913,215.55638338)
\curveto(847.48998019,215.33637589)(847.83497985,214.99637623)(848.0249913,214.53638338)
\curveto(848.06497962,214.45637677)(848.09497959,214.36637686)(848.1149913,214.26638338)
\curveto(848.13497955,214.17637705)(848.15497953,214.07637715)(848.1749913,213.96638338)
\curveto(848.19497949,213.93637729)(848.19997948,213.90137732)(848.1899913,213.86138338)
\curveto(848.18997949,213.83137739)(848.19497949,213.80137742)(848.2049913,213.77138338)
\lineto(848.2049913,213.63638338)
\curveto(848.21497947,213.59637763)(848.21497947,213.55137767)(848.2049913,213.50138338)
\curveto(848.20497948,213.45137777)(848.20497948,213.40137782)(848.2049913,213.35138338)
\lineto(848.2049913,212.76638338)
\lineto(848.2049913,211.80638338)
\lineto(848.2049913,208.95638338)
\curveto(848.20497948,208.79638243)(848.20497948,208.60638262)(848.2049913,208.38638338)
\curveto(848.21497947,208.16638306)(848.17497951,208.0213832)(848.0849913,207.95138338)
\curveto(848.04497964,207.9213833)(847.9799797,207.89638333)(847.8899913,207.87638338)
\curveto(847.79997988,207.86638336)(847.70497998,207.86138336)(847.6049913,207.86138338)
\curveto(847.50498018,207.86138336)(847.40498028,207.86638336)(847.3049913,207.87638338)
\curveto(847.21498047,207.88638334)(847.14998053,207.90638332)(847.1099913,207.93638338)
\curveto(847.04998063,207.96638326)(847.00998067,208.0263832)(846.9899913,208.11638338)
\curveto(846.96998071,208.17638305)(846.96498072,208.23638299)(846.9749913,208.29638338)
\curveto(846.9849807,208.36638286)(846.9799807,208.43138279)(846.9599913,208.49138338)
\curveto(846.94998073,208.54138268)(846.94498074,208.59638263)(846.9449913,208.65638338)
\curveto(846.95498073,208.7263825)(846.95998072,208.79138243)(846.9599913,208.85138338)
\lineto(846.9599913,209.52638338)
\lineto(846.9599913,212.39138338)
\curveto(846.95998072,212.7213785)(846.94998073,213.03137819)(846.9299913,213.32138338)
\curveto(846.91998076,213.6213776)(846.84998083,213.87137735)(846.7199913,214.07138338)
\curveto(846.56998111,214.31137691)(846.33998134,214.48637674)(846.0299913,214.59638338)
\curveto(845.96998171,214.61637661)(845.90498178,214.6263766)(845.8349913,214.62638338)
\curveto(845.77498191,214.63637659)(845.70998197,214.65137657)(845.6399913,214.67138338)
\curveto(845.59998208,214.68137654)(845.53498215,214.68137654)(845.4449913,214.67138338)
\curveto(845.35498233,214.67137655)(845.29498239,214.66637656)(845.2649913,214.65638338)
\curveto(845.21498247,214.64637658)(845.16498252,214.64137658)(845.1149913,214.64138338)
\curveto(845.06498262,214.65137657)(845.01498267,214.64637658)(844.9649913,214.62638338)
\curveto(844.82498286,214.59637663)(844.68998299,214.55637667)(844.5599913,214.50638338)
\curveto(844.03998364,214.28637694)(843.68998399,213.90137732)(843.5099913,213.35138338)
\curveto(843.45998422,213.18137804)(843.42998425,212.98637824)(843.4199913,212.76638338)
\lineto(843.4199913,212.09138338)
\lineto(843.4199913,210.12638338)
\lineto(843.4199913,208.67138338)
\lineto(843.4199913,208.29638338)
\curveto(843.41998426,208.17638305)(843.39498429,208.08138314)(843.3449913,208.01138338)
\curveto(843.29498439,207.93138329)(843.20998447,207.88638334)(843.0899913,207.87638338)
\curveto(842.96998471,207.86638336)(842.84498484,207.86138336)(842.7149913,207.86138338)
\curveto(842.54498514,207.86138336)(842.41998526,207.88138334)(842.3399913,207.92138338)
\curveto(842.24998543,207.97138325)(842.19498549,208.05138317)(842.1749913,208.16138338)
\curveto(842.16498552,208.28138294)(842.15998552,208.41138281)(842.1599913,208.55138338)
\lineto(842.1599913,209.97638338)
\lineto(842.1599913,212.45138338)
\curveto(842.15998552,212.77137845)(842.14998553,213.06637816)(842.1299913,213.33638338)
\curveto(842.10998557,213.61637761)(842.03998564,213.85637737)(841.9199913,214.05638338)
\curveto(841.80998587,214.23637699)(841.684986,214.36637686)(841.5449913,214.44638338)
\curveto(841.40498628,214.53637669)(841.21498647,214.60637662)(840.9749913,214.65638338)
\curveto(840.93498675,214.66637656)(840.88998679,214.67137655)(840.8399913,214.67138338)
\lineto(840.7049913,214.67138338)
\curveto(840.4849872,214.67137655)(840.28998739,214.64637658)(840.1199913,214.59638338)
\curveto(839.95998772,214.54637668)(839.81498787,214.48137674)(839.6849913,214.40138338)
\curveto(839.17498851,214.09137713)(838.83498885,213.6263776)(838.6649913,213.00638338)
\curveto(838.62498906,212.87637835)(838.60498908,212.7263785)(838.6049913,212.55638338)
\curveto(838.61498907,212.39637883)(838.61998906,212.23637899)(838.6199913,212.07638338)
\lineto(838.6199913,210.38138338)
\lineto(838.6199913,208.73138338)
\lineto(838.6199913,208.32638338)
\curveto(838.61998906,208.18638304)(838.58998909,208.07638315)(838.5299913,207.99638338)
\curveto(838.4799892,207.9263833)(838.40498928,207.88638334)(838.3049913,207.87638338)
\curveto(838.20498948,207.86638336)(838.09998958,207.86138336)(837.9899913,207.86138338)
\lineto(837.7649913,207.86138338)
\curveto(837.70498998,207.88138334)(837.64499004,207.89638333)(837.5849913,207.90638338)
\curveto(837.53499015,207.91638331)(837.48999019,207.94638328)(837.4499913,207.99638338)
\curveto(837.39999028,208.05638317)(837.37499031,208.13138309)(837.3749913,208.22138338)
\lineto(837.3749913,208.53638338)
\lineto(837.3749913,209.51138338)
\lineto(837.3749913,213.80138338)
\lineto(837.3749913,214.91138338)
\lineto(837.3749913,215.19638338)
\curveto(837.37499031,215.29637593)(837.39499029,215.37637585)(837.4349913,215.43638338)
\curveto(837.46499022,215.49637573)(837.50999017,215.53637569)(837.5699913,215.55638338)
\curveto(837.64999003,215.58637564)(837.77498991,215.60137562)(837.9449913,215.60138338)
\curveto(838.12498956,215.60137562)(838.25498943,215.58637564)(838.3349913,215.55638338)
\curveto(838.41498927,215.51637571)(838.46998921,215.46637576)(838.4999913,215.40638338)
\curveto(838.51998916,215.35637587)(838.52998915,215.29637593)(838.5299913,215.22638338)
\curveto(838.53998914,215.15637607)(838.54998913,215.09137613)(838.5599913,215.03138338)
\curveto(838.56998911,214.97137625)(838.58998909,214.9213763)(838.6199913,214.88138338)
\curveto(838.64998903,214.84137638)(838.69998898,214.8213764)(838.7699913,214.82138338)
\curveto(838.78998889,214.84137638)(838.80998887,214.85137637)(838.8299913,214.85138338)
\curveto(838.85998882,214.85137637)(838.8849888,214.86137636)(838.9049913,214.88138338)
\curveto(838.96498872,214.93137629)(839.01998866,214.98137624)(839.0699913,215.03138338)
\lineto(839.2499913,215.18138338)
\curveto(839.46998821,215.34137588)(839.71998796,215.48137574)(839.9999913,215.60138338)
\curveto(840.09998758,215.64137558)(840.19998748,215.66637556)(840.2999913,215.67638338)
\curveto(840.39998728,215.69637553)(840.50498718,215.7213755)(840.6149913,215.75138338)
\lineto(840.7949913,215.75138338)
\curveto(840.86498682,215.76137546)(840.92998675,215.76637546)(840.9899913,215.76638338)
}
}
{
\newrgbcolor{curcolor}{0 0 0}
\pscustom[linestyle=none,fillstyle=solid,fillcolor=curcolor]
{
\newpath
\moveto(850.56772567,215.58638338)
\lineto(851.00272567,215.58638338)
\curveto(851.15272371,215.58637564)(851.2577236,215.54637568)(851.31772567,215.46638338)
\curveto(851.36772349,215.38637584)(851.39272347,215.28637594)(851.39272567,215.16638338)
\curveto(851.40272346,215.04637618)(851.40772345,214.9263763)(851.40772567,214.80638338)
\lineto(851.40772567,213.38138338)
\lineto(851.40772567,211.11638338)
\lineto(851.40772567,210.42638338)
\curveto(851.40772345,210.19638103)(851.43272343,209.99638123)(851.48272567,209.82638338)
\curveto(851.64272322,209.37638185)(851.94272292,209.06138216)(852.38272567,208.88138338)
\curveto(852.60272226,208.79138243)(852.86772199,208.75638247)(853.17772567,208.77638338)
\curveto(853.48772137,208.80638242)(853.73772112,208.86138236)(853.92772567,208.94138338)
\curveto(854.2577206,209.08138214)(854.51772034,209.25638197)(854.70772567,209.46638338)
\curveto(854.90771995,209.68638154)(855.0627198,209.97138125)(855.17272567,210.32138338)
\curveto(855.20271966,210.40138082)(855.22271964,210.48138074)(855.23272567,210.56138338)
\curveto(855.24271962,210.64138058)(855.2577196,210.7263805)(855.27772567,210.81638338)
\curveto(855.28771957,210.86638036)(855.28771957,210.91138031)(855.27772567,210.95138338)
\curveto(855.27771958,210.99138023)(855.28771957,211.03638019)(855.30772567,211.08638338)
\lineto(855.30772567,211.40138338)
\curveto(855.32771953,211.48137974)(855.33271953,211.57137965)(855.32272567,211.67138338)
\curveto(855.31271955,211.78137944)(855.30771955,211.88137934)(855.30772567,211.97138338)
\lineto(855.30772567,213.14138338)
\lineto(855.30772567,214.73138338)
\curveto(855.30771955,214.85137637)(855.30271956,214.97637625)(855.29272567,215.10638338)
\curveto(855.29271957,215.24637598)(855.31771954,215.35637587)(855.36772567,215.43638338)
\curveto(855.40771945,215.48637574)(855.45271941,215.51637571)(855.50272567,215.52638338)
\curveto(855.5627193,215.54637568)(855.63271923,215.56637566)(855.71272567,215.58638338)
\lineto(855.93772567,215.58638338)
\curveto(856.0577188,215.58637564)(856.1627187,215.58137564)(856.25272567,215.57138338)
\curveto(856.35271851,215.56137566)(856.42771843,215.51637571)(856.47772567,215.43638338)
\curveto(856.52771833,215.38637584)(856.55271831,215.31137591)(856.55272567,215.21138338)
\lineto(856.55272567,214.92638338)
\lineto(856.55272567,213.90638338)
\lineto(856.55272567,209.87138338)
\lineto(856.55272567,208.52138338)
\curveto(856.55271831,208.40138282)(856.54771831,208.28638294)(856.53772567,208.17638338)
\curveto(856.53771832,208.07638315)(856.50271836,208.00138322)(856.43272567,207.95138338)
\curveto(856.39271847,207.9213833)(856.33271853,207.89638333)(856.25272567,207.87638338)
\curveto(856.17271869,207.86638336)(856.08271878,207.85638337)(855.98272567,207.84638338)
\curveto(855.89271897,207.84638338)(855.80271906,207.85138337)(855.71272567,207.86138338)
\curveto(855.63271923,207.87138335)(855.57271929,207.89138333)(855.53272567,207.92138338)
\curveto(855.48271938,207.96138326)(855.43771942,208.0263832)(855.39772567,208.11638338)
\curveto(855.38771947,208.15638307)(855.37771948,208.21138301)(855.36772567,208.28138338)
\curveto(855.36771949,208.35138287)(855.3627195,208.41638281)(855.35272567,208.47638338)
\curveto(855.34271952,208.54638268)(855.32271954,208.60138262)(855.29272567,208.64138338)
\curveto(855.2627196,208.68138254)(855.21771964,208.69638253)(855.15772567,208.68638338)
\curveto(855.07771978,208.66638256)(854.99771986,208.60638262)(854.91772567,208.50638338)
\curveto(854.83772002,208.41638281)(854.7627201,208.34638288)(854.69272567,208.29638338)
\curveto(854.47272039,208.13638309)(854.22272064,207.99638323)(853.94272567,207.87638338)
\curveto(853.83272103,207.8263834)(853.71772114,207.79638343)(853.59772567,207.78638338)
\curveto(853.48772137,207.76638346)(853.37272149,207.74138348)(853.25272567,207.71138338)
\curveto(853.20272166,207.70138352)(853.14772171,207.70138352)(853.08772567,207.71138338)
\curveto(853.03772182,207.7213835)(852.98772187,207.71638351)(852.93772567,207.69638338)
\curveto(852.83772202,207.67638355)(852.74772211,207.67638355)(852.66772567,207.69638338)
\lineto(852.51772567,207.69638338)
\curveto(852.46772239,207.71638351)(852.40772245,207.7263835)(852.33772567,207.72638338)
\curveto(852.27772258,207.7263835)(852.22272264,207.73138349)(852.17272567,207.74138338)
\curveto(852.13272273,207.76138346)(852.09272277,207.77138345)(852.05272567,207.77138338)
\curveto(852.02272284,207.76138346)(851.98272288,207.76638346)(851.93272567,207.78638338)
\lineto(851.69272567,207.84638338)
\curveto(851.62272324,207.86638336)(851.54772331,207.89638333)(851.46772567,207.93638338)
\curveto(851.20772365,208.04638318)(850.98772387,208.19138303)(850.80772567,208.37138338)
\curveto(850.63772422,208.56138266)(850.49772436,208.78638244)(850.38772567,209.04638338)
\curveto(850.34772451,209.13638209)(850.31772454,209.226382)(850.29772567,209.31638338)
\lineto(850.23772567,209.61638338)
\curveto(850.21772464,209.67638155)(850.20772465,209.73138149)(850.20772567,209.78138338)
\curveto(850.21772464,209.84138138)(850.21272465,209.90638132)(850.19272567,209.97638338)
\curveto(850.18272468,209.99638123)(850.17772468,210.0213812)(850.17772567,210.05138338)
\curveto(850.17772468,210.09138113)(850.17272469,210.1263811)(850.16272567,210.15638338)
\lineto(850.16272567,210.30638338)
\curveto(850.15272471,210.34638088)(850.14772471,210.39138083)(850.14772567,210.44138338)
\curveto(850.1577247,210.50138072)(850.1627247,210.55638067)(850.16272567,210.60638338)
\lineto(850.16272567,211.20638338)
\lineto(850.16272567,213.96638338)
\lineto(850.16272567,214.92638338)
\lineto(850.16272567,215.19638338)
\curveto(850.1627247,215.28637594)(850.18272468,215.36137586)(850.22272567,215.42138338)
\curveto(850.2627246,215.49137573)(850.33772452,215.54137568)(850.44772567,215.57138338)
\curveto(850.46772439,215.58137564)(850.48772437,215.58137564)(850.50772567,215.57138338)
\curveto(850.52772433,215.57137565)(850.54772431,215.57637565)(850.56772567,215.58638338)
}
}
{
\newrgbcolor{curcolor}{0 0 0}
\pscustom[linestyle=none,fillstyle=solid,fillcolor=curcolor]
{
\newpath
\moveto(862.14233505,215.73638338)
\curveto(862.77232981,215.75637547)(863.27732931,215.67137555)(863.65733505,215.48138338)
\curveto(864.03732855,215.29137593)(864.34232824,215.00637622)(864.57233505,214.62638338)
\curveto(864.63232795,214.5263767)(864.67732791,214.41637681)(864.70733505,214.29638338)
\curveto(864.74732784,214.18637704)(864.7823278,214.07137715)(864.81233505,213.95138338)
\curveto(864.86232772,213.76137746)(864.89232769,213.55637767)(864.90233505,213.33638338)
\curveto(864.91232767,213.11637811)(864.91732767,212.89137833)(864.91733505,212.66138338)
\lineto(864.91733505,211.05638338)
\lineto(864.91733505,208.71638338)
\curveto(864.91732767,208.54638268)(864.91232767,208.37638285)(864.90233505,208.20638338)
\curveto(864.90232768,208.03638319)(864.83732775,207.9263833)(864.70733505,207.87638338)
\curveto(864.65732793,207.85638337)(864.60232798,207.84638338)(864.54233505,207.84638338)
\curveto(864.49232809,207.83638339)(864.43732815,207.83138339)(864.37733505,207.83138338)
\curveto(864.24732834,207.83138339)(864.12232846,207.83638339)(864.00233505,207.84638338)
\curveto(863.8823287,207.84638338)(863.79732879,207.88638334)(863.74733505,207.96638338)
\curveto(863.69732889,208.03638319)(863.67232891,208.1263831)(863.67233505,208.23638338)
\lineto(863.67233505,208.56638338)
\lineto(863.67233505,209.85638338)
\lineto(863.67233505,212.30138338)
\curveto(863.67232891,212.57137865)(863.66732892,212.83637839)(863.65733505,213.09638338)
\curveto(863.64732894,213.36637786)(863.60232898,213.59637763)(863.52233505,213.78638338)
\curveto(863.44232914,213.98637724)(863.32232926,214.14637708)(863.16233505,214.26638338)
\curveto(863.00232958,214.39637683)(862.81732977,214.49637673)(862.60733505,214.56638338)
\curveto(862.54733004,214.58637664)(862.4823301,214.59637663)(862.41233505,214.59638338)
\curveto(862.35233023,214.60637662)(862.29233029,214.6213766)(862.23233505,214.64138338)
\curveto(862.1823304,214.65137657)(862.10233048,214.65137657)(861.99233505,214.64138338)
\curveto(861.89233069,214.64137658)(861.82233076,214.63637659)(861.78233505,214.62638338)
\curveto(861.74233084,214.60637662)(861.70733088,214.59637663)(861.67733505,214.59638338)
\curveto(861.64733094,214.60637662)(861.61233097,214.60637662)(861.57233505,214.59638338)
\curveto(861.44233114,214.56637666)(861.31733127,214.53137669)(861.19733505,214.49138338)
\curveto(861.0873315,214.46137676)(860.9823316,214.41637681)(860.88233505,214.35638338)
\curveto(860.84233174,214.33637689)(860.80733178,214.31637691)(860.77733505,214.29638338)
\curveto(860.74733184,214.27637695)(860.71233187,214.25637697)(860.67233505,214.23638338)
\curveto(860.32233226,213.98637724)(860.06733252,213.61137761)(859.90733505,213.11138338)
\curveto(859.87733271,213.03137819)(859.85733273,212.94637828)(859.84733505,212.85638338)
\curveto(859.83733275,212.77637845)(859.82233276,212.69637853)(859.80233505,212.61638338)
\curveto(859.7823328,212.56637866)(859.77733281,212.51637871)(859.78733505,212.46638338)
\curveto(859.79733279,212.4263788)(859.79233279,212.38637884)(859.77233505,212.34638338)
\lineto(859.77233505,212.03138338)
\curveto(859.76233282,212.00137922)(859.75733283,211.96637926)(859.75733505,211.92638338)
\curveto(859.76733282,211.88637934)(859.77233281,211.84137938)(859.77233505,211.79138338)
\lineto(859.77233505,211.34138338)
\lineto(859.77233505,209.90138338)
\lineto(859.77233505,208.58138338)
\lineto(859.77233505,208.23638338)
\curveto(859.77233281,208.1263831)(859.74733284,208.03638319)(859.69733505,207.96638338)
\curveto(859.64733294,207.88638334)(859.55733303,207.84638338)(859.42733505,207.84638338)
\curveto(859.30733328,207.83638339)(859.1823334,207.83138339)(859.05233505,207.83138338)
\curveto(858.97233361,207.83138339)(858.89733369,207.83638339)(858.82733505,207.84638338)
\curveto(858.75733383,207.85638337)(858.69733389,207.88138334)(858.64733505,207.92138338)
\curveto(858.56733402,207.97138325)(858.52733406,208.06638316)(858.52733505,208.20638338)
\lineto(858.52733505,208.61138338)
\lineto(858.52733505,210.38138338)
\lineto(858.52733505,214.01138338)
\lineto(858.52733505,214.92638338)
\lineto(858.52733505,215.19638338)
\curveto(858.52733406,215.28637594)(858.54733404,215.35637587)(858.58733505,215.40638338)
\curveto(858.61733397,215.46637576)(858.66733392,215.50637572)(858.73733505,215.52638338)
\curveto(858.77733381,215.53637569)(858.83233375,215.54637568)(858.90233505,215.55638338)
\curveto(858.9823336,215.56637566)(859.06233352,215.57137565)(859.14233505,215.57138338)
\curveto(859.22233336,215.57137565)(859.29733329,215.56637566)(859.36733505,215.55638338)
\curveto(859.44733314,215.54637568)(859.50233308,215.53137569)(859.53233505,215.51138338)
\curveto(859.64233294,215.44137578)(859.69233289,215.35137587)(859.68233505,215.24138338)
\curveto(859.67233291,215.14137608)(859.6873329,215.0263762)(859.72733505,214.89638338)
\curveto(859.74733284,214.83637639)(859.7873328,214.78637644)(859.84733505,214.74638338)
\curveto(859.96733262,214.73637649)(860.06233252,214.78137644)(860.13233505,214.88138338)
\curveto(860.21233237,214.98137624)(860.29233229,215.06137616)(860.37233505,215.12138338)
\curveto(860.51233207,215.221376)(860.65233193,215.31137591)(860.79233505,215.39138338)
\curveto(860.94233164,215.48137574)(861.11233147,215.55637567)(861.30233505,215.61638338)
\curveto(861.3823312,215.64637558)(861.46733112,215.66637556)(861.55733505,215.67638338)
\curveto(861.65733093,215.68637554)(861.75233083,215.70137552)(861.84233505,215.72138338)
\curveto(861.89233069,215.73137549)(861.94233064,215.73637549)(861.99233505,215.73638338)
\lineto(862.14233505,215.73638338)
}
}
{
\newrgbcolor{curcolor}{0 0 0}
\pscustom[linestyle=none,fillstyle=solid,fillcolor=curcolor]
{
\newpath
\moveto(867.08694442,217.08638338)
\curveto(867.0069433,217.14637408)(866.96194335,217.25137397)(866.95194442,217.40138338)
\lineto(866.95194442,217.86638338)
\lineto(866.95194442,218.12138338)
\curveto(866.95194336,218.21137301)(866.96694334,218.28637294)(866.99694442,218.34638338)
\curveto(867.03694327,218.4263728)(867.11694319,218.48637274)(867.23694442,218.52638338)
\curveto(867.25694305,218.53637269)(867.27694303,218.53637269)(867.29694442,218.52638338)
\curveto(867.32694298,218.5263727)(867.35194296,218.53137269)(867.37194442,218.54138338)
\curveto(867.54194277,218.54137268)(867.70194261,218.53637269)(867.85194442,218.52638338)
\curveto(868.00194231,218.51637271)(868.10194221,218.45637277)(868.15194442,218.34638338)
\curveto(868.18194213,218.28637294)(868.19694211,218.21137301)(868.19694442,218.12138338)
\lineto(868.19694442,217.86638338)
\curveto(868.19694211,217.68637354)(868.19194212,217.51637371)(868.18194442,217.35638338)
\curveto(868.18194213,217.19637403)(868.11694219,217.09137413)(867.98694442,217.04138338)
\curveto(867.93694237,217.0213742)(867.88194243,217.01137421)(867.82194442,217.01138338)
\lineto(867.65694442,217.01138338)
\lineto(867.34194442,217.01138338)
\curveto(867.24194307,217.01137421)(867.15694315,217.03637419)(867.08694442,217.08638338)
\moveto(868.19694442,208.58138338)
\lineto(868.19694442,208.26638338)
\curveto(868.2069421,208.16638306)(868.18694212,208.08638314)(868.13694442,208.02638338)
\curveto(868.1069422,207.96638326)(868.06194225,207.9263833)(868.00194442,207.90638338)
\curveto(867.94194237,207.89638333)(867.87194244,207.88138334)(867.79194442,207.86138338)
\lineto(867.56694442,207.86138338)
\curveto(867.43694287,207.86138336)(867.32194299,207.86638336)(867.22194442,207.87638338)
\curveto(867.13194318,207.89638333)(867.06194325,207.94638328)(867.01194442,208.02638338)
\curveto(866.97194334,208.08638314)(866.95194336,208.16138306)(866.95194442,208.25138338)
\lineto(866.95194442,208.53638338)
\lineto(866.95194442,214.88138338)
\lineto(866.95194442,215.19638338)
\curveto(866.95194336,215.30637592)(866.97694333,215.39137583)(867.02694442,215.45138338)
\curveto(867.05694325,215.50137572)(867.09694321,215.53137569)(867.14694442,215.54138338)
\curveto(867.19694311,215.55137567)(867.25194306,215.56637566)(867.31194442,215.58638338)
\curveto(867.33194298,215.58637564)(867.35194296,215.58137564)(867.37194442,215.57138338)
\curveto(867.40194291,215.57137565)(867.42694288,215.57637565)(867.44694442,215.58638338)
\curveto(867.57694273,215.58637564)(867.7069426,215.58137564)(867.83694442,215.57138338)
\curveto(867.97694233,215.57137565)(868.07194224,215.53137569)(868.12194442,215.45138338)
\curveto(868.17194214,215.39137583)(868.19694211,215.31137591)(868.19694442,215.21138338)
\lineto(868.19694442,214.92638338)
\lineto(868.19694442,208.58138338)
}
}
{
\newrgbcolor{curcolor}{0 0 0}
\pscustom[linestyle=none,fillstyle=solid,fillcolor=curcolor]
{
\newpath
\moveto(877.10178817,208.67138338)
\lineto(877.10178817,208.28138338)
\curveto(877.1017803,208.16138306)(877.07678032,208.06138316)(877.02678817,207.98138338)
\curveto(876.97678042,207.91138331)(876.89178051,207.87138335)(876.77178817,207.86138338)
\lineto(876.42678817,207.86138338)
\curveto(876.36678103,207.86138336)(876.30678109,207.85638337)(876.24678817,207.84638338)
\curveto(876.1967812,207.84638338)(876.15178125,207.85638337)(876.11178817,207.87638338)
\curveto(876.02178138,207.89638333)(875.96178144,207.93638329)(875.93178817,207.99638338)
\curveto(875.89178151,208.04638318)(875.86678153,208.10638312)(875.85678817,208.17638338)
\curveto(875.85678154,208.24638298)(875.84178156,208.31638291)(875.81178817,208.38638338)
\curveto(875.8017816,208.40638282)(875.78678161,208.4213828)(875.76678817,208.43138338)
\curveto(875.75678164,208.45138277)(875.74178166,208.47138275)(875.72178817,208.49138338)
\curveto(875.62178178,208.50138272)(875.54178186,208.48138274)(875.48178817,208.43138338)
\curveto(875.43178197,208.38138284)(875.37678202,208.33138289)(875.31678817,208.28138338)
\curveto(875.11678228,208.13138309)(874.91678248,208.01638321)(874.71678817,207.93638338)
\curveto(874.53678286,207.85638337)(874.32678307,207.79638343)(874.08678817,207.75638338)
\curveto(873.85678354,207.71638351)(873.61678378,207.69638353)(873.36678817,207.69638338)
\curveto(873.12678427,207.68638354)(872.88678451,207.70138352)(872.64678817,207.74138338)
\curveto(872.40678499,207.77138345)(872.1967852,207.8263834)(872.01678817,207.90638338)
\curveto(871.4967859,208.1263831)(871.07678632,208.4213828)(870.75678817,208.79138338)
\curveto(870.43678696,209.17138205)(870.18678721,209.64138158)(870.00678817,210.20138338)
\curveto(869.96678743,210.29138093)(869.93678746,210.38138084)(869.91678817,210.47138338)
\curveto(869.90678749,210.57138065)(869.88678751,210.67138055)(869.85678817,210.77138338)
\curveto(869.84678755,210.8213804)(869.84178756,210.87138035)(869.84178817,210.92138338)
\curveto(869.84178756,210.97138025)(869.83678756,211.0213802)(869.82678817,211.07138338)
\curveto(869.80678759,211.1213801)(869.7967876,211.17138005)(869.79678817,211.22138338)
\curveto(869.80678759,211.28137994)(869.80678759,211.33637989)(869.79678817,211.38638338)
\lineto(869.79678817,211.53638338)
\curveto(869.77678762,211.58637964)(869.76678763,211.65137957)(869.76678817,211.73138338)
\curveto(869.76678763,211.81137941)(869.77678762,211.87637935)(869.79678817,211.92638338)
\lineto(869.79678817,212.09138338)
\curveto(869.81678758,212.16137906)(869.82178758,212.23137899)(869.81178817,212.30138338)
\curveto(869.81178759,212.38137884)(869.82178758,212.45637877)(869.84178817,212.52638338)
\curveto(869.85178755,212.57637865)(869.85678754,212.6213786)(869.85678817,212.66138338)
\curveto(869.85678754,212.70137852)(869.86178754,212.74637848)(869.87178817,212.79638338)
\curveto(869.9017875,212.89637833)(869.92678747,212.99137823)(869.94678817,213.08138338)
\curveto(869.96678743,213.18137804)(869.99178741,213.27637795)(870.02178817,213.36638338)
\curveto(870.15178725,213.74637748)(870.31678708,214.08637714)(870.51678817,214.38638338)
\curveto(870.72678667,214.69637653)(870.97678642,214.95137627)(871.26678817,215.15138338)
\curveto(871.43678596,215.27137595)(871.61178579,215.37137585)(871.79178817,215.45138338)
\curveto(871.98178542,215.53137569)(872.18678521,215.60137562)(872.40678817,215.66138338)
\curveto(872.47678492,215.67137555)(872.54178486,215.68137554)(872.60178817,215.69138338)
\curveto(872.67178473,215.70137552)(872.74178466,215.71637551)(872.81178817,215.73638338)
\lineto(872.96178817,215.73638338)
\curveto(873.04178436,215.75637547)(873.15678424,215.76637546)(873.30678817,215.76638338)
\curveto(873.46678393,215.76637546)(873.58678381,215.75637547)(873.66678817,215.73638338)
\curveto(873.70678369,215.7263755)(873.76178364,215.7213755)(873.83178817,215.72138338)
\curveto(873.94178346,215.69137553)(874.05178335,215.66637556)(874.16178817,215.64638338)
\curveto(874.27178313,215.63637559)(874.37678302,215.60637562)(874.47678817,215.55638338)
\curveto(874.62678277,215.49637573)(874.76678263,215.43137579)(874.89678817,215.36138338)
\curveto(875.03678236,215.29137593)(875.16678223,215.21137601)(875.28678817,215.12138338)
\curveto(875.34678205,215.07137615)(875.40678199,215.01637621)(875.46678817,214.95638338)
\curveto(875.53678186,214.90637632)(875.62678177,214.89137633)(875.73678817,214.91138338)
\curveto(875.75678164,214.94137628)(875.77178163,214.96637626)(875.78178817,214.98638338)
\curveto(875.8017816,215.00637622)(875.81678158,215.03637619)(875.82678817,215.07638338)
\curveto(875.85678154,215.16637606)(875.86678153,215.28137594)(875.85678817,215.42138338)
\lineto(875.85678817,215.79638338)
\lineto(875.85678817,217.52138338)
\lineto(875.85678817,217.98638338)
\curveto(875.85678154,218.16637306)(875.88178152,218.29637293)(875.93178817,218.37638338)
\curveto(875.97178143,218.44637278)(876.03178137,218.49137273)(876.11178817,218.51138338)
\curveto(876.13178127,218.51137271)(876.15678124,218.51137271)(876.18678817,218.51138338)
\curveto(876.21678118,218.5213727)(876.24178116,218.5263727)(876.26178817,218.52638338)
\curveto(876.401781,218.53637269)(876.54678085,218.53637269)(876.69678817,218.52638338)
\curveto(876.85678054,218.5263727)(876.96678043,218.48637274)(877.02678817,218.40638338)
\curveto(877.07678032,218.3263729)(877.1017803,218.226373)(877.10178817,218.10638338)
\lineto(877.10178817,217.73138338)
\lineto(877.10178817,208.67138338)
\moveto(875.88678817,211.50638338)
\curveto(875.90678149,211.55637967)(875.91678148,211.6213796)(875.91678817,211.70138338)
\curveto(875.91678148,211.79137943)(875.90678149,211.86137936)(875.88678817,211.91138338)
\lineto(875.88678817,212.13638338)
\curveto(875.86678153,212.226379)(875.85178155,212.31637891)(875.84178817,212.40638338)
\curveto(875.83178157,212.50637872)(875.81178159,212.59637863)(875.78178817,212.67638338)
\curveto(875.76178164,212.75637847)(875.74178166,212.83137839)(875.72178817,212.90138338)
\curveto(875.71178169,212.97137825)(875.69178171,213.04137818)(875.66178817,213.11138338)
\curveto(875.54178186,213.41137781)(875.38678201,213.67637755)(875.19678817,213.90638338)
\curveto(875.00678239,214.13637709)(874.76678263,214.31637691)(874.47678817,214.44638338)
\curveto(874.37678302,214.49637673)(874.27178313,214.53137669)(874.16178817,214.55138338)
\curveto(874.06178334,214.58137664)(873.95178345,214.60637662)(873.83178817,214.62638338)
\curveto(873.75178365,214.64637658)(873.66178374,214.65637657)(873.56178817,214.65638338)
\lineto(873.29178817,214.65638338)
\curveto(873.24178416,214.64637658)(873.1967842,214.63637659)(873.15678817,214.62638338)
\lineto(873.02178817,214.62638338)
\curveto(872.94178446,214.60637662)(872.85678454,214.58637664)(872.76678817,214.56638338)
\curveto(872.68678471,214.54637668)(872.60678479,214.5213767)(872.52678817,214.49138338)
\curveto(872.20678519,214.35137687)(871.94678545,214.14637708)(871.74678817,213.87638338)
\curveto(871.55678584,213.61637761)(871.401786,213.31137791)(871.28178817,212.96138338)
\curveto(871.24178616,212.85137837)(871.21178619,212.73637849)(871.19178817,212.61638338)
\curveto(871.18178622,212.50637872)(871.16678623,212.39637883)(871.14678817,212.28638338)
\curveto(871.14678625,212.24637898)(871.14178626,212.20637902)(871.13178817,212.16638338)
\lineto(871.13178817,212.06138338)
\curveto(871.11178629,212.01137921)(871.1017863,211.95637927)(871.10178817,211.89638338)
\curveto(871.11178629,211.83637939)(871.11678628,211.78137944)(871.11678817,211.73138338)
\lineto(871.11678817,211.40138338)
\curveto(871.11678628,211.30137992)(871.12678627,211.20638002)(871.14678817,211.11638338)
\curveto(871.15678624,211.08638014)(871.16178624,211.03638019)(871.16178817,210.96638338)
\curveto(871.18178622,210.89638033)(871.1967862,210.8263804)(871.20678817,210.75638338)
\lineto(871.26678817,210.54638338)
\curveto(871.37678602,210.19638103)(871.52678587,209.89638133)(871.71678817,209.64638338)
\curveto(871.90678549,209.39638183)(872.14678525,209.19138203)(872.43678817,209.03138338)
\curveto(872.52678487,208.98138224)(872.61678478,208.94138228)(872.70678817,208.91138338)
\curveto(872.7967846,208.88138234)(872.8967845,208.85138237)(873.00678817,208.82138338)
\curveto(873.05678434,208.80138242)(873.10678429,208.79638243)(873.15678817,208.80638338)
\curveto(873.21678418,208.81638241)(873.27178413,208.81138241)(873.32178817,208.79138338)
\curveto(873.36178404,208.78138244)(873.401784,208.77638245)(873.44178817,208.77638338)
\lineto(873.57678817,208.77638338)
\lineto(873.71178817,208.77638338)
\curveto(873.74178366,208.78638244)(873.79178361,208.79138243)(873.86178817,208.79138338)
\curveto(873.94178346,208.81138241)(874.02178338,208.8263824)(874.10178817,208.83638338)
\curveto(874.18178322,208.85638237)(874.25678314,208.88138234)(874.32678817,208.91138338)
\curveto(874.65678274,209.05138217)(874.92178248,209.226382)(875.12178817,209.43638338)
\curveto(875.33178207,209.65638157)(875.50678189,209.93138129)(875.64678817,210.26138338)
\curveto(875.6967817,210.37138085)(875.73178167,210.48138074)(875.75178817,210.59138338)
\curveto(875.77178163,210.70138052)(875.7967816,210.81138041)(875.82678817,210.92138338)
\curveto(875.84678155,210.96138026)(875.85678154,210.99638023)(875.85678817,211.02638338)
\curveto(875.85678154,211.06638016)(875.86178154,211.10638012)(875.87178817,211.14638338)
\curveto(875.88178152,211.20638002)(875.88178152,211.26637996)(875.87178817,211.32638338)
\curveto(875.87178153,211.38637984)(875.87678152,211.44637978)(875.88678817,211.50638338)
}
}
{
\newrgbcolor{curcolor}{0 0 0}
\pscustom[linestyle=none,fillstyle=solid,fillcolor=curcolor]
{
\newpath
\moveto(885.93303817,208.41638338)
\curveto(885.96303034,208.25638297)(885.94803036,208.1213831)(885.88803817,208.01138338)
\curveto(885.82803048,207.91138331)(885.74803056,207.83638339)(885.64803817,207.78638338)
\curveto(885.59803071,207.76638346)(885.54303076,207.75638347)(885.48303817,207.75638338)
\curveto(885.43303087,207.75638347)(885.37803093,207.74638348)(885.31803817,207.72638338)
\curveto(885.09803121,207.67638355)(884.87803143,207.69138353)(884.65803817,207.77138338)
\curveto(884.44803186,207.84138338)(884.303032,207.93138329)(884.22303817,208.04138338)
\curveto(884.17303213,208.11138311)(884.12803218,208.19138303)(884.08803817,208.28138338)
\curveto(884.04803226,208.38138284)(883.99803231,208.46138276)(883.93803817,208.52138338)
\curveto(883.91803239,208.54138268)(883.89303241,208.56138266)(883.86303817,208.58138338)
\curveto(883.84303246,208.60138262)(883.81303249,208.60638262)(883.77303817,208.59638338)
\curveto(883.66303264,208.56638266)(883.55803275,208.51138271)(883.45803817,208.43138338)
\curveto(883.36803294,208.35138287)(883.27803303,208.28138294)(883.18803817,208.22138338)
\curveto(883.05803325,208.14138308)(882.91803339,208.06638316)(882.76803817,207.99638338)
\curveto(882.61803369,207.93638329)(882.45803385,207.88138334)(882.28803817,207.83138338)
\curveto(882.18803412,207.80138342)(882.07803423,207.78138344)(881.95803817,207.77138338)
\curveto(881.84803446,207.76138346)(881.73803457,207.74638348)(881.62803817,207.72638338)
\curveto(881.57803473,207.71638351)(881.53303477,207.71138351)(881.49303817,207.71138338)
\lineto(881.38803817,207.71138338)
\curveto(881.27803503,207.69138353)(881.17303513,207.69138353)(881.07303817,207.71138338)
\lineto(880.93803817,207.71138338)
\curveto(880.88803542,207.7213835)(880.83803547,207.7263835)(880.78803817,207.72638338)
\curveto(880.73803557,207.7263835)(880.69303561,207.73638349)(880.65303817,207.75638338)
\curveto(880.61303569,207.76638346)(880.57803573,207.77138345)(880.54803817,207.77138338)
\curveto(880.52803578,207.76138346)(880.5030358,207.76138346)(880.47303817,207.77138338)
\lineto(880.23303817,207.83138338)
\curveto(880.15303615,207.84138338)(880.07803623,207.86138336)(880.00803817,207.89138338)
\curveto(879.7080366,208.0213832)(879.46303684,208.16638306)(879.27303817,208.32638338)
\curveto(879.09303721,208.49638273)(878.94303736,208.73138249)(878.82303817,209.03138338)
\curveto(878.73303757,209.25138197)(878.68803762,209.51638171)(878.68803817,209.82638338)
\lineto(878.68803817,210.14138338)
\curveto(878.69803761,210.19138103)(878.7030376,210.24138098)(878.70303817,210.29138338)
\lineto(878.73303817,210.47138338)
\lineto(878.85303817,210.80138338)
\curveto(878.89303741,210.91138031)(878.94303736,211.01138021)(879.00303817,211.10138338)
\curveto(879.18303712,211.39137983)(879.42803688,211.60637962)(879.73803817,211.74638338)
\curveto(880.04803626,211.88637934)(880.38803592,212.01137921)(880.75803817,212.12138338)
\curveto(880.89803541,212.16137906)(881.04303526,212.19137903)(881.19303817,212.21138338)
\curveto(881.34303496,212.23137899)(881.49303481,212.25637897)(881.64303817,212.28638338)
\curveto(881.71303459,212.30637892)(881.77803453,212.31637891)(881.83803817,212.31638338)
\curveto(881.9080344,212.31637891)(881.98303432,212.3263789)(882.06303817,212.34638338)
\curveto(882.13303417,212.36637886)(882.2030341,212.37637885)(882.27303817,212.37638338)
\curveto(882.34303396,212.38637884)(882.41803389,212.40137882)(882.49803817,212.42138338)
\curveto(882.74803356,212.48137874)(882.98303332,212.53137869)(883.20303817,212.57138338)
\curveto(883.42303288,212.6213786)(883.59803271,212.73637849)(883.72803817,212.91638338)
\curveto(883.78803252,212.99637823)(883.83803247,213.09637813)(883.87803817,213.21638338)
\curveto(883.91803239,213.34637788)(883.91803239,213.48637774)(883.87803817,213.63638338)
\curveto(883.81803249,213.87637735)(883.72803258,214.06637716)(883.60803817,214.20638338)
\curveto(883.49803281,214.34637688)(883.33803297,214.45637677)(883.12803817,214.53638338)
\curveto(883.0080333,214.58637664)(882.86303344,214.6213766)(882.69303817,214.64138338)
\curveto(882.53303377,214.66137656)(882.36303394,214.67137655)(882.18303817,214.67138338)
\curveto(882.0030343,214.67137655)(881.82803448,214.66137656)(881.65803817,214.64138338)
\curveto(881.48803482,214.6213766)(881.34303496,214.59137663)(881.22303817,214.55138338)
\curveto(881.05303525,214.49137673)(880.88803542,214.40637682)(880.72803817,214.29638338)
\curveto(880.64803566,214.23637699)(880.57303573,214.15637707)(880.50303817,214.05638338)
\curveto(880.44303586,213.96637726)(880.38803592,213.86637736)(880.33803817,213.75638338)
\curveto(880.308036,213.67637755)(880.27803603,213.59137763)(880.24803817,213.50138338)
\curveto(880.22803608,213.41137781)(880.18303612,213.34137788)(880.11303817,213.29138338)
\curveto(880.07303623,213.26137796)(880.0030363,213.23637799)(879.90303817,213.21638338)
\curveto(879.81303649,213.20637802)(879.71803659,213.20137802)(879.61803817,213.20138338)
\curveto(879.51803679,213.20137802)(879.41803689,213.20637802)(879.31803817,213.21638338)
\curveto(879.22803708,213.23637799)(879.16303714,213.26137796)(879.12303817,213.29138338)
\curveto(879.08303722,213.3213779)(879.05303725,213.37137785)(879.03303817,213.44138338)
\curveto(879.01303729,213.51137771)(879.01303729,213.58637764)(879.03303817,213.66638338)
\curveto(879.06303724,213.79637743)(879.09303721,213.91637731)(879.12303817,214.02638338)
\curveto(879.16303714,214.14637708)(879.2080371,214.26137696)(879.25803817,214.37138338)
\curveto(879.44803686,214.7213765)(879.68803662,214.99137623)(879.97803817,215.18138338)
\curveto(880.26803604,215.38137584)(880.62803568,215.54137568)(881.05803817,215.66138338)
\curveto(881.15803515,215.68137554)(881.25803505,215.69637553)(881.35803817,215.70638338)
\curveto(881.46803484,215.71637551)(881.57803473,215.73137549)(881.68803817,215.75138338)
\curveto(881.72803458,215.76137546)(881.79303451,215.76137546)(881.88303817,215.75138338)
\curveto(881.97303433,215.75137547)(882.02803428,215.76137546)(882.04803817,215.78138338)
\curveto(882.74803356,215.79137543)(883.35803295,215.71137551)(883.87803817,215.54138338)
\curveto(884.39803191,215.37137585)(884.76303154,215.04637618)(884.97303817,214.56638338)
\curveto(885.06303124,214.36637686)(885.11303119,214.13137709)(885.12303817,213.86138338)
\curveto(885.14303116,213.60137762)(885.15303115,213.3263779)(885.15303817,213.03638338)
\lineto(885.15303817,209.72138338)
\curveto(885.15303115,209.58138164)(885.15803115,209.44638178)(885.16803817,209.31638338)
\curveto(885.17803113,209.18638204)(885.2080311,209.08138214)(885.25803817,209.00138338)
\curveto(885.308031,208.93138229)(885.37303093,208.88138234)(885.45303817,208.85138338)
\curveto(885.54303076,208.81138241)(885.62803068,208.78138244)(885.70803817,208.76138338)
\curveto(885.78803052,208.75138247)(885.84803046,208.70638252)(885.88803817,208.62638338)
\curveto(885.9080304,208.59638263)(885.91803039,208.56638266)(885.91803817,208.53638338)
\curveto(885.91803039,208.50638272)(885.92303038,208.46638276)(885.93303817,208.41638338)
\moveto(883.78803817,210.08138338)
\curveto(883.84803246,210.221381)(883.87803243,210.38138084)(883.87803817,210.56138338)
\curveto(883.88803242,210.75138047)(883.89303241,210.94638028)(883.89303817,211.14638338)
\curveto(883.89303241,211.25637997)(883.88803242,211.35637987)(883.87803817,211.44638338)
\curveto(883.86803244,211.53637969)(883.82803248,211.60637962)(883.75803817,211.65638338)
\curveto(883.72803258,211.67637955)(883.65803265,211.68637954)(883.54803817,211.68638338)
\curveto(883.52803278,211.66637956)(883.49303281,211.65637957)(883.44303817,211.65638338)
\curveto(883.39303291,211.65637957)(883.34803296,211.64637958)(883.30803817,211.62638338)
\curveto(883.22803308,211.60637962)(883.13803317,211.58637964)(883.03803817,211.56638338)
\lineto(882.73803817,211.50638338)
\curveto(882.7080336,211.50637972)(882.67303363,211.50137972)(882.63303817,211.49138338)
\lineto(882.52803817,211.49138338)
\curveto(882.37803393,211.45137977)(882.21303409,211.4263798)(882.03303817,211.41638338)
\curveto(881.86303444,211.41637981)(881.7030346,211.39637983)(881.55303817,211.35638338)
\curveto(881.47303483,211.33637989)(881.39803491,211.31637991)(881.32803817,211.29638338)
\curveto(881.26803504,211.28637994)(881.19803511,211.27137995)(881.11803817,211.25138338)
\curveto(880.95803535,211.20138002)(880.8080355,211.13638009)(880.66803817,211.05638338)
\curveto(880.52803578,210.98638024)(880.4080359,210.89638033)(880.30803817,210.78638338)
\curveto(880.2080361,210.67638055)(880.13303617,210.54138068)(880.08303817,210.38138338)
\curveto(880.03303627,210.23138099)(880.01303629,210.04638118)(880.02303817,209.82638338)
\curveto(880.02303628,209.7263815)(880.03803627,209.63138159)(880.06803817,209.54138338)
\curveto(880.1080362,209.46138176)(880.15303615,209.38638184)(880.20303817,209.31638338)
\curveto(880.28303602,209.20638202)(880.38803592,209.11138211)(880.51803817,209.03138338)
\curveto(880.64803566,208.96138226)(880.78803552,208.90138232)(880.93803817,208.85138338)
\curveto(880.98803532,208.84138238)(881.03803527,208.83638239)(881.08803817,208.83638338)
\curveto(881.13803517,208.83638239)(881.18803512,208.83138239)(881.23803817,208.82138338)
\curveto(881.308035,208.80138242)(881.39303491,208.78638244)(881.49303817,208.77638338)
\curveto(881.6030347,208.77638245)(881.69303461,208.78638244)(881.76303817,208.80638338)
\curveto(881.82303448,208.8263824)(881.88303442,208.83138239)(881.94303817,208.82138338)
\curveto(882.0030343,208.8213824)(882.06303424,208.83138239)(882.12303817,208.85138338)
\curveto(882.2030341,208.87138235)(882.27803403,208.88638234)(882.34803817,208.89638338)
\curveto(882.42803388,208.90638232)(882.5030338,208.9263823)(882.57303817,208.95638338)
\curveto(882.86303344,209.07638215)(883.1080332,209.221382)(883.30803817,209.39138338)
\curveto(883.51803279,209.56138166)(883.67803263,209.79138143)(883.78803817,210.08138338)
}
}
{
\newrgbcolor{curcolor}{0 0 0}
\pscustom[linestyle=none,fillstyle=solid,fillcolor=curcolor]
{
\newpath
\moveto(894.0646788,208.67138338)
\lineto(894.0646788,208.28138338)
\curveto(894.06467092,208.16138306)(894.03967095,208.06138316)(893.9896788,207.98138338)
\curveto(893.93967105,207.91138331)(893.85467113,207.87138335)(893.7346788,207.86138338)
\lineto(893.3896788,207.86138338)
\curveto(893.32967166,207.86138336)(893.26967172,207.85638337)(893.2096788,207.84638338)
\curveto(893.15967183,207.84638338)(893.11467187,207.85638337)(893.0746788,207.87638338)
\curveto(892.984672,207.89638333)(892.92467206,207.93638329)(892.8946788,207.99638338)
\curveto(892.85467213,208.04638318)(892.82967216,208.10638312)(892.8196788,208.17638338)
\curveto(892.81967217,208.24638298)(892.80467218,208.31638291)(892.7746788,208.38638338)
\curveto(892.76467222,208.40638282)(892.74967224,208.4213828)(892.7296788,208.43138338)
\curveto(892.71967227,208.45138277)(892.70467228,208.47138275)(892.6846788,208.49138338)
\curveto(892.5846724,208.50138272)(892.50467248,208.48138274)(892.4446788,208.43138338)
\curveto(892.39467259,208.38138284)(892.33967265,208.33138289)(892.2796788,208.28138338)
\curveto(892.07967291,208.13138309)(891.87967311,208.01638321)(891.6796788,207.93638338)
\curveto(891.49967349,207.85638337)(891.2896737,207.79638343)(891.0496788,207.75638338)
\curveto(890.81967417,207.71638351)(890.57967441,207.69638353)(890.3296788,207.69638338)
\curveto(890.0896749,207.68638354)(889.84967514,207.70138352)(889.6096788,207.74138338)
\curveto(889.36967562,207.77138345)(889.15967583,207.8263834)(888.9796788,207.90638338)
\curveto(888.45967653,208.1263831)(888.03967695,208.4213828)(887.7196788,208.79138338)
\curveto(887.39967759,209.17138205)(887.14967784,209.64138158)(886.9696788,210.20138338)
\curveto(886.92967806,210.29138093)(886.89967809,210.38138084)(886.8796788,210.47138338)
\curveto(886.86967812,210.57138065)(886.84967814,210.67138055)(886.8196788,210.77138338)
\curveto(886.80967818,210.8213804)(886.80467818,210.87138035)(886.8046788,210.92138338)
\curveto(886.80467818,210.97138025)(886.79967819,211.0213802)(886.7896788,211.07138338)
\curveto(886.76967822,211.1213801)(886.75967823,211.17138005)(886.7596788,211.22138338)
\curveto(886.76967822,211.28137994)(886.76967822,211.33637989)(886.7596788,211.38638338)
\lineto(886.7596788,211.53638338)
\curveto(886.73967825,211.58637964)(886.72967826,211.65137957)(886.7296788,211.73138338)
\curveto(886.72967826,211.81137941)(886.73967825,211.87637935)(886.7596788,211.92638338)
\lineto(886.7596788,212.09138338)
\curveto(886.77967821,212.16137906)(886.7846782,212.23137899)(886.7746788,212.30138338)
\curveto(886.77467821,212.38137884)(886.7846782,212.45637877)(886.8046788,212.52638338)
\curveto(886.81467817,212.57637865)(886.81967817,212.6213786)(886.8196788,212.66138338)
\curveto(886.81967817,212.70137852)(886.82467816,212.74637848)(886.8346788,212.79638338)
\curveto(886.86467812,212.89637833)(886.8896781,212.99137823)(886.9096788,213.08138338)
\curveto(886.92967806,213.18137804)(886.95467803,213.27637795)(886.9846788,213.36638338)
\curveto(887.11467787,213.74637748)(887.27967771,214.08637714)(887.4796788,214.38638338)
\curveto(887.6896773,214.69637653)(887.93967705,214.95137627)(888.2296788,215.15138338)
\curveto(888.39967659,215.27137595)(888.57467641,215.37137585)(888.7546788,215.45138338)
\curveto(888.94467604,215.53137569)(889.14967584,215.60137562)(889.3696788,215.66138338)
\curveto(889.43967555,215.67137555)(889.50467548,215.68137554)(889.5646788,215.69138338)
\curveto(889.63467535,215.70137552)(889.70467528,215.71637551)(889.7746788,215.73638338)
\lineto(889.9246788,215.73638338)
\curveto(890.00467498,215.75637547)(890.11967487,215.76637546)(890.2696788,215.76638338)
\curveto(890.42967456,215.76637546)(890.54967444,215.75637547)(890.6296788,215.73638338)
\curveto(890.66967432,215.7263755)(890.72467426,215.7213755)(890.7946788,215.72138338)
\curveto(890.90467408,215.69137553)(891.01467397,215.66637556)(891.1246788,215.64638338)
\curveto(891.23467375,215.63637559)(891.33967365,215.60637562)(891.4396788,215.55638338)
\curveto(891.5896734,215.49637573)(891.72967326,215.43137579)(891.8596788,215.36138338)
\curveto(891.99967299,215.29137593)(892.12967286,215.21137601)(892.2496788,215.12138338)
\curveto(892.30967268,215.07137615)(892.36967262,215.01637621)(892.4296788,214.95638338)
\curveto(892.49967249,214.90637632)(892.5896724,214.89137633)(892.6996788,214.91138338)
\curveto(892.71967227,214.94137628)(892.73467225,214.96637626)(892.7446788,214.98638338)
\curveto(892.76467222,215.00637622)(892.77967221,215.03637619)(892.7896788,215.07638338)
\curveto(892.81967217,215.16637606)(892.82967216,215.28137594)(892.8196788,215.42138338)
\lineto(892.8196788,215.79638338)
\lineto(892.8196788,217.52138338)
\lineto(892.8196788,217.98638338)
\curveto(892.81967217,218.16637306)(892.84467214,218.29637293)(892.8946788,218.37638338)
\curveto(892.93467205,218.44637278)(892.99467199,218.49137273)(893.0746788,218.51138338)
\curveto(893.09467189,218.51137271)(893.11967187,218.51137271)(893.1496788,218.51138338)
\curveto(893.17967181,218.5213727)(893.20467178,218.5263727)(893.2246788,218.52638338)
\curveto(893.36467162,218.53637269)(893.50967148,218.53637269)(893.6596788,218.52638338)
\curveto(893.81967117,218.5263727)(893.92967106,218.48637274)(893.9896788,218.40638338)
\curveto(894.03967095,218.3263729)(894.06467092,218.226373)(894.0646788,218.10638338)
\lineto(894.0646788,217.73138338)
\lineto(894.0646788,208.67138338)
\moveto(892.8496788,211.50638338)
\curveto(892.86967212,211.55637967)(892.87967211,211.6213796)(892.8796788,211.70138338)
\curveto(892.87967211,211.79137943)(892.86967212,211.86137936)(892.8496788,211.91138338)
\lineto(892.8496788,212.13638338)
\curveto(892.82967216,212.226379)(892.81467217,212.31637891)(892.8046788,212.40638338)
\curveto(892.79467219,212.50637872)(892.77467221,212.59637863)(892.7446788,212.67638338)
\curveto(892.72467226,212.75637847)(892.70467228,212.83137839)(892.6846788,212.90138338)
\curveto(892.67467231,212.97137825)(892.65467233,213.04137818)(892.6246788,213.11138338)
\curveto(892.50467248,213.41137781)(892.34967264,213.67637755)(892.1596788,213.90638338)
\curveto(891.96967302,214.13637709)(891.72967326,214.31637691)(891.4396788,214.44638338)
\curveto(891.33967365,214.49637673)(891.23467375,214.53137669)(891.1246788,214.55138338)
\curveto(891.02467396,214.58137664)(890.91467407,214.60637662)(890.7946788,214.62638338)
\curveto(890.71467427,214.64637658)(890.62467436,214.65637657)(890.5246788,214.65638338)
\lineto(890.2546788,214.65638338)
\curveto(890.20467478,214.64637658)(890.15967483,214.63637659)(890.1196788,214.62638338)
\lineto(889.9846788,214.62638338)
\curveto(889.90467508,214.60637662)(889.81967517,214.58637664)(889.7296788,214.56638338)
\curveto(889.64967534,214.54637668)(889.56967542,214.5213767)(889.4896788,214.49138338)
\curveto(889.16967582,214.35137687)(888.90967608,214.14637708)(888.7096788,213.87638338)
\curveto(888.51967647,213.61637761)(888.36467662,213.31137791)(888.2446788,212.96138338)
\curveto(888.20467678,212.85137837)(888.17467681,212.73637849)(888.1546788,212.61638338)
\curveto(888.14467684,212.50637872)(888.12967686,212.39637883)(888.1096788,212.28638338)
\curveto(888.10967688,212.24637898)(888.10467688,212.20637902)(888.0946788,212.16638338)
\lineto(888.0946788,212.06138338)
\curveto(888.07467691,212.01137921)(888.06467692,211.95637927)(888.0646788,211.89638338)
\curveto(888.07467691,211.83637939)(888.07967691,211.78137944)(888.0796788,211.73138338)
\lineto(888.0796788,211.40138338)
\curveto(888.07967691,211.30137992)(888.0896769,211.20638002)(888.1096788,211.11638338)
\curveto(888.11967687,211.08638014)(888.12467686,211.03638019)(888.1246788,210.96638338)
\curveto(888.14467684,210.89638033)(888.15967683,210.8263804)(888.1696788,210.75638338)
\lineto(888.2296788,210.54638338)
\curveto(888.33967665,210.19638103)(888.4896765,209.89638133)(888.6796788,209.64638338)
\curveto(888.86967612,209.39638183)(889.10967588,209.19138203)(889.3996788,209.03138338)
\curveto(889.4896755,208.98138224)(889.57967541,208.94138228)(889.6696788,208.91138338)
\curveto(889.75967523,208.88138234)(889.85967513,208.85138237)(889.9696788,208.82138338)
\curveto(890.01967497,208.80138242)(890.06967492,208.79638243)(890.1196788,208.80638338)
\curveto(890.17967481,208.81638241)(890.23467475,208.81138241)(890.2846788,208.79138338)
\curveto(890.32467466,208.78138244)(890.36467462,208.77638245)(890.4046788,208.77638338)
\lineto(890.5396788,208.77638338)
\lineto(890.6746788,208.77638338)
\curveto(890.70467428,208.78638244)(890.75467423,208.79138243)(890.8246788,208.79138338)
\curveto(890.90467408,208.81138241)(890.984674,208.8263824)(891.0646788,208.83638338)
\curveto(891.14467384,208.85638237)(891.21967377,208.88138234)(891.2896788,208.91138338)
\curveto(891.61967337,209.05138217)(891.8846731,209.226382)(892.0846788,209.43638338)
\curveto(892.29467269,209.65638157)(892.46967252,209.93138129)(892.6096788,210.26138338)
\curveto(892.65967233,210.37138085)(892.69467229,210.48138074)(892.7146788,210.59138338)
\curveto(892.73467225,210.70138052)(892.75967223,210.81138041)(892.7896788,210.92138338)
\curveto(892.80967218,210.96138026)(892.81967217,210.99638023)(892.8196788,211.02638338)
\curveto(892.81967217,211.06638016)(892.82467216,211.10638012)(892.8346788,211.14638338)
\curveto(892.84467214,211.20638002)(892.84467214,211.26637996)(892.8346788,211.32638338)
\curveto(892.83467215,211.38637984)(892.83967215,211.44637978)(892.8496788,211.50638338)
}
}
{
\newrgbcolor{curcolor}{0 0 0}
\pscustom[linestyle=none,fillstyle=solid,fillcolor=curcolor]
{
\newpath
\moveto(902.7609288,212.03138338)
\curveto(902.78092111,211.93137929)(902.78092111,211.81637941)(902.7609288,211.68638338)
\curveto(902.75092114,211.56637966)(902.72092117,211.48137974)(902.6709288,211.43138338)
\curveto(902.62092127,211.39137983)(902.54592135,211.36137986)(902.4459288,211.34138338)
\curveto(902.35592154,211.33137989)(902.25092164,211.3263799)(902.1309288,211.32638338)
\lineto(901.7709288,211.32638338)
\curveto(901.65092224,211.33637989)(901.54592235,211.34137988)(901.4559288,211.34138338)
\lineto(897.6159288,211.34138338)
\curveto(897.53592636,211.34137988)(897.45592644,211.33637989)(897.3759288,211.32638338)
\curveto(897.2959266,211.3263799)(897.23092666,211.31137991)(897.1809288,211.28138338)
\curveto(897.14092675,211.26137996)(897.10092679,211.22138)(897.0609288,211.16138338)
\curveto(897.04092685,211.13138009)(897.02092687,211.08638014)(897.0009288,211.02638338)
\curveto(896.98092691,210.97638025)(896.98092691,210.9263803)(897.0009288,210.87638338)
\curveto(897.01092688,210.8263804)(897.01592688,210.78138044)(897.0159288,210.74138338)
\curveto(897.01592688,210.70138052)(897.02092687,210.66138056)(897.0309288,210.62138338)
\curveto(897.05092684,210.54138068)(897.07092682,210.45638077)(897.0909288,210.36638338)
\curveto(897.11092678,210.28638094)(897.14092675,210.20638102)(897.1809288,210.12638338)
\curveto(897.41092648,209.58638164)(897.7909261,209.20138202)(898.3209288,208.97138338)
\curveto(898.38092551,208.94138228)(898.44592545,208.91638231)(898.5159288,208.89638338)
\lineto(898.7259288,208.83638338)
\curveto(898.75592514,208.8263824)(898.80592509,208.8213824)(898.8759288,208.82138338)
\curveto(899.01592488,208.78138244)(899.20092469,208.76138246)(899.4309288,208.76138338)
\curveto(899.66092423,208.76138246)(899.84592405,208.78138244)(899.9859288,208.82138338)
\curveto(900.12592377,208.86138236)(900.25092364,208.90138232)(900.3609288,208.94138338)
\curveto(900.48092341,208.99138223)(900.5909233,209.05138217)(900.6909288,209.12138338)
\curveto(900.80092309,209.19138203)(900.895923,209.27138195)(900.9759288,209.36138338)
\curveto(901.05592284,209.46138176)(901.12592277,209.56638166)(901.1859288,209.67638338)
\curveto(901.24592265,209.77638145)(901.2959226,209.88138134)(901.3359288,209.99138338)
\curveto(901.38592251,210.10138112)(901.46592243,210.18138104)(901.5759288,210.23138338)
\curveto(901.61592228,210.25138097)(901.68092221,210.26638096)(901.7709288,210.27638338)
\curveto(901.86092203,210.28638094)(901.95092194,210.28638094)(902.0409288,210.27638338)
\curveto(902.13092176,210.27638095)(902.21592168,210.27138095)(902.2959288,210.26138338)
\curveto(902.37592152,210.25138097)(902.43092146,210.23138099)(902.4609288,210.20138338)
\curveto(902.56092133,210.13138109)(902.58592131,210.01638121)(902.5359288,209.85638338)
\curveto(902.45592144,209.58638164)(902.35092154,209.34638188)(902.2209288,209.13638338)
\curveto(902.02092187,208.81638241)(901.7909221,208.55138267)(901.5309288,208.34138338)
\curveto(901.28092261,208.14138308)(900.96092293,207.97638325)(900.5709288,207.84638338)
\curveto(900.47092342,207.80638342)(900.37092352,207.78138344)(900.2709288,207.77138338)
\curveto(900.17092372,207.75138347)(900.06592383,207.73138349)(899.9559288,207.71138338)
\curveto(899.90592399,207.70138352)(899.85592404,207.69638353)(899.8059288,207.69638338)
\curveto(899.76592413,207.69638353)(899.72092417,207.69138353)(899.6709288,207.68138338)
\lineto(899.5209288,207.68138338)
\curveto(899.47092442,207.67138355)(899.41092448,207.66638356)(899.3409288,207.66638338)
\curveto(899.28092461,207.66638356)(899.23092466,207.67138355)(899.1909288,207.68138338)
\lineto(899.0559288,207.68138338)
\curveto(899.00592489,207.69138353)(898.96092493,207.69638353)(898.9209288,207.69638338)
\curveto(898.88092501,207.69638353)(898.84092505,207.70138352)(898.8009288,207.71138338)
\curveto(898.75092514,207.7213835)(898.6959252,207.73138349)(898.6359288,207.74138338)
\curveto(898.57592532,207.74138348)(898.52092537,207.74638348)(898.4709288,207.75638338)
\curveto(898.38092551,207.77638345)(898.2909256,207.80138342)(898.2009288,207.83138338)
\curveto(898.11092578,207.85138337)(898.02592587,207.87638335)(897.9459288,207.90638338)
\curveto(897.90592599,207.9263833)(897.87092602,207.93638329)(897.8409288,207.93638338)
\curveto(897.81092608,207.94638328)(897.77592612,207.96138326)(897.7359288,207.98138338)
\curveto(897.58592631,208.05138317)(897.42592647,208.13638309)(897.2559288,208.23638338)
\curveto(896.96592693,208.4263828)(896.71592718,208.65638257)(896.5059288,208.92638338)
\curveto(896.30592759,209.20638202)(896.13592776,209.51638171)(895.9959288,209.85638338)
\curveto(895.94592795,209.96638126)(895.90592799,210.08138114)(895.8759288,210.20138338)
\curveto(895.85592804,210.3213809)(895.82592807,210.44138078)(895.7859288,210.56138338)
\curveto(895.77592812,210.60138062)(895.77092812,210.63638059)(895.7709288,210.66638338)
\curveto(895.77092812,210.69638053)(895.76592813,210.73638049)(895.7559288,210.78638338)
\curveto(895.73592816,210.86638036)(895.72092817,210.95138027)(895.7109288,211.04138338)
\curveto(895.70092819,211.13138009)(895.68592821,211.22138)(895.6659288,211.31138338)
\lineto(895.6659288,211.52138338)
\curveto(895.65592824,211.56137966)(895.64592825,211.61637961)(895.6359288,211.68638338)
\curveto(895.63592826,211.76637946)(895.64092825,211.83137939)(895.6509288,211.88138338)
\lineto(895.6509288,212.04638338)
\curveto(895.67092822,212.09637913)(895.67592822,212.14637908)(895.6659288,212.19638338)
\curveto(895.66592823,212.25637897)(895.67092822,212.31137891)(895.6809288,212.36138338)
\curveto(895.72092817,212.5213787)(895.75092814,212.68137854)(895.7709288,212.84138338)
\curveto(895.80092809,213.00137822)(895.84592805,213.15137807)(895.9059288,213.29138338)
\curveto(895.95592794,213.40137782)(896.00092789,213.51137771)(896.0409288,213.62138338)
\curveto(896.0909278,213.74137748)(896.14592775,213.85637737)(896.2059288,213.96638338)
\curveto(896.42592747,214.31637691)(896.67592722,214.61637661)(896.9559288,214.86638338)
\curveto(897.23592666,215.1263761)(897.58092631,215.34137588)(897.9909288,215.51138338)
\curveto(898.11092578,215.56137566)(898.23092566,215.59637563)(898.3509288,215.61638338)
\curveto(898.48092541,215.64637558)(898.61592528,215.67637555)(898.7559288,215.70638338)
\curveto(898.80592509,215.71637551)(898.85092504,215.7213755)(898.8909288,215.72138338)
\curveto(898.93092496,215.73137549)(898.97592492,215.73637549)(899.0259288,215.73638338)
\curveto(899.04592485,215.74637548)(899.07092482,215.74637548)(899.1009288,215.73638338)
\curveto(899.13092476,215.7263755)(899.15592474,215.73137549)(899.1759288,215.75138338)
\curveto(899.5959243,215.76137546)(899.96092393,215.71637551)(900.2709288,215.61638338)
\curveto(900.58092331,215.5263757)(900.86092303,215.40137582)(901.1109288,215.24138338)
\curveto(901.16092273,215.221376)(901.20092269,215.19137603)(901.2309288,215.15138338)
\curveto(901.26092263,215.1213761)(901.2959226,215.09637613)(901.3359288,215.07638338)
\curveto(901.41592248,215.01637621)(901.4959224,214.94637628)(901.5759288,214.86638338)
\curveto(901.66592223,214.78637644)(901.74092215,214.70637652)(901.8009288,214.62638338)
\curveto(901.96092193,214.41637681)(902.0959218,214.21637701)(902.2059288,214.02638338)
\curveto(902.27592162,213.91637731)(902.33092156,213.79637743)(902.3709288,213.66638338)
\curveto(902.41092148,213.53637769)(902.45592144,213.40637782)(902.5059288,213.27638338)
\curveto(902.55592134,213.14637808)(902.5909213,213.01137821)(902.6109288,212.87138338)
\curveto(902.64092125,212.73137849)(902.67592122,212.59137863)(902.7159288,212.45138338)
\curveto(902.72592117,212.38137884)(902.73092116,212.31137891)(902.7309288,212.24138338)
\lineto(902.7609288,212.03138338)
\moveto(901.3059288,212.54138338)
\curveto(901.33592256,212.58137864)(901.36092253,212.63137859)(901.3809288,212.69138338)
\curveto(901.40092249,212.76137846)(901.40092249,212.83137839)(901.3809288,212.90138338)
\curveto(901.32092257,213.1213781)(901.23592266,213.3263779)(901.1259288,213.51638338)
\curveto(900.98592291,213.74637748)(900.83092306,213.94137728)(900.6609288,214.10138338)
\curveto(900.4909234,214.26137696)(900.27092362,214.39637683)(900.0009288,214.50638338)
\curveto(899.93092396,214.5263767)(899.86092403,214.54137668)(899.7909288,214.55138338)
\curveto(899.72092417,214.57137665)(899.64592425,214.59137663)(899.5659288,214.61138338)
\curveto(899.48592441,214.63137659)(899.40092449,214.64137658)(899.3109288,214.64138338)
\lineto(899.0559288,214.64138338)
\curveto(899.02592487,214.6213766)(898.9909249,214.61137661)(898.9509288,214.61138338)
\curveto(898.91092498,214.6213766)(898.87592502,214.6213766)(898.8459288,214.61138338)
\lineto(898.6059288,214.55138338)
\curveto(898.53592536,214.54137668)(898.46592543,214.5263767)(898.3959288,214.50638338)
\curveto(898.10592579,214.38637684)(897.87092602,214.23637699)(897.6909288,214.05638338)
\curveto(897.52092637,213.87637735)(897.36592653,213.65137757)(897.2259288,213.38138338)
\curveto(897.1959267,213.33137789)(897.16592673,213.26637796)(897.1359288,213.18638338)
\curveto(897.10592679,213.11637811)(897.08092681,213.03637819)(897.0609288,212.94638338)
\curveto(897.04092685,212.85637837)(897.03592686,212.77137845)(897.0459288,212.69138338)
\curveto(897.05592684,212.61137861)(897.0909268,212.55137867)(897.1509288,212.51138338)
\curveto(897.23092666,212.45137877)(897.36592653,212.4213788)(897.5559288,212.42138338)
\curveto(897.75592614,212.43137879)(897.92592597,212.43637879)(898.0659288,212.43638338)
\lineto(900.3459288,212.43638338)
\curveto(900.4959234,212.43637879)(900.67592322,212.43137879)(900.8859288,212.42138338)
\curveto(901.0959228,212.4213788)(901.23592266,212.46137876)(901.3059288,212.54138338)
}
}
{
\newrgbcolor{curcolor}{0 0 0}
\pscustom[linestyle=none,fillstyle=solid,fillcolor=curcolor]
{
\newpath
\moveto(906.49756942,215.76638338)
\curveto(907.21756536,215.77637545)(907.82256475,215.69137553)(908.31256942,215.51138338)
\curveto(908.80256377,215.34137588)(909.18256339,215.03637619)(909.45256942,214.59638338)
\curveto(909.52256305,214.48637674)(909.577563,214.37137685)(909.61756942,214.25138338)
\curveto(909.65756292,214.14137708)(909.69756288,214.01637721)(909.73756942,213.87638338)
\curveto(909.75756282,213.80637742)(909.76256281,213.73137749)(909.75256942,213.65138338)
\curveto(909.74256283,213.58137764)(909.72756285,213.5263777)(909.70756942,213.48638338)
\curveto(909.68756289,213.46637776)(909.66256291,213.44637778)(909.63256942,213.42638338)
\curveto(909.60256297,213.41637781)(909.577563,213.40137782)(909.55756942,213.38138338)
\curveto(909.50756307,213.36137786)(909.45756312,213.35637787)(909.40756942,213.36638338)
\curveto(909.35756322,213.37637785)(909.30756327,213.37637785)(909.25756942,213.36638338)
\curveto(909.1775634,213.34637788)(909.0725635,213.34137788)(908.94256942,213.35138338)
\curveto(908.81256376,213.37137785)(908.72256385,213.39637783)(908.67256942,213.42638338)
\curveto(908.59256398,213.47637775)(908.53756404,213.54137768)(908.50756942,213.62138338)
\curveto(908.48756409,213.71137751)(908.45256412,213.79637743)(908.40256942,213.87638338)
\curveto(908.31256426,214.03637719)(908.18756439,214.18137704)(908.02756942,214.31138338)
\curveto(907.91756466,214.39137683)(907.79756478,214.45137677)(907.66756942,214.49138338)
\curveto(907.53756504,214.53137669)(907.39756518,214.57137665)(907.24756942,214.61138338)
\curveto(907.19756538,214.63137659)(907.14756543,214.63637659)(907.09756942,214.62638338)
\curveto(907.04756553,214.6263766)(906.99756558,214.63137659)(906.94756942,214.64138338)
\curveto(906.88756569,214.66137656)(906.81256576,214.67137655)(906.72256942,214.67138338)
\curveto(906.63256594,214.67137655)(906.55756602,214.66137656)(906.49756942,214.64138338)
\lineto(906.40756942,214.64138338)
\lineto(906.25756942,214.61138338)
\curveto(906.20756637,214.61137661)(906.15756642,214.60637662)(906.10756942,214.59638338)
\curveto(905.84756673,214.53637669)(905.63256694,214.45137677)(905.46256942,214.34138338)
\curveto(905.29256728,214.23137699)(905.1775674,214.04637718)(905.11756942,213.78638338)
\curveto(905.09756748,213.71637751)(905.09256748,213.64637758)(905.10256942,213.57638338)
\curveto(905.12256745,213.50637772)(905.14256743,213.44637778)(905.16256942,213.39638338)
\curveto(905.22256735,213.24637798)(905.29256728,213.13637809)(905.37256942,213.06638338)
\curveto(905.46256711,213.00637822)(905.572567,212.93637829)(905.70256942,212.85638338)
\curveto(905.86256671,212.75637847)(906.04256653,212.68137854)(906.24256942,212.63138338)
\curveto(906.44256613,212.59137863)(906.64256593,212.54137868)(906.84256942,212.48138338)
\curveto(906.9725656,212.44137878)(907.10256547,212.41137881)(907.23256942,212.39138338)
\curveto(907.36256521,212.37137885)(907.49256508,212.34137888)(907.62256942,212.30138338)
\curveto(907.83256474,212.24137898)(908.03756454,212.18137904)(908.23756942,212.12138338)
\curveto(908.43756414,212.07137915)(908.63756394,212.00637922)(908.83756942,211.92638338)
\lineto(908.98756942,211.86638338)
\curveto(909.03756354,211.84637938)(909.08756349,211.8213794)(909.13756942,211.79138338)
\curveto(909.33756324,211.67137955)(909.51256306,211.53637969)(909.66256942,211.38638338)
\curveto(909.81256276,211.23637999)(909.93756264,211.04638018)(910.03756942,210.81638338)
\curveto(910.05756252,210.74638048)(910.0775625,210.65138057)(910.09756942,210.53138338)
\curveto(910.11756246,210.46138076)(910.12756245,210.38638084)(910.12756942,210.30638338)
\curveto(910.13756244,210.23638099)(910.14256243,210.15638107)(910.14256942,210.06638338)
\lineto(910.14256942,209.91638338)
\curveto(910.12256245,209.84638138)(910.11256246,209.77638145)(910.11256942,209.70638338)
\curveto(910.11256246,209.63638159)(910.10256247,209.56638166)(910.08256942,209.49638338)
\curveto(910.05256252,209.38638184)(910.01756256,209.28138194)(909.97756942,209.18138338)
\curveto(909.93756264,209.08138214)(909.89256268,208.99138223)(909.84256942,208.91138338)
\curveto(909.68256289,208.65138257)(909.4775631,208.44138278)(909.22756942,208.28138338)
\curveto(908.9775636,208.13138309)(908.69756388,208.00138322)(908.38756942,207.89138338)
\curveto(908.29756428,207.86138336)(908.20256437,207.84138338)(908.10256942,207.83138338)
\curveto(908.01256456,207.81138341)(907.92256465,207.78638344)(907.83256942,207.75638338)
\curveto(907.73256484,207.73638349)(907.63256494,207.7263835)(907.53256942,207.72638338)
\curveto(907.43256514,207.7263835)(907.33256524,207.71638351)(907.23256942,207.69638338)
\lineto(907.08256942,207.69638338)
\curveto(907.03256554,207.68638354)(906.96256561,207.68138354)(906.87256942,207.68138338)
\curveto(906.78256579,207.68138354)(906.71256586,207.68638354)(906.66256942,207.69638338)
\lineto(906.49756942,207.69638338)
\curveto(906.43756614,207.71638351)(906.3725662,207.7263835)(906.30256942,207.72638338)
\curveto(906.23256634,207.71638351)(906.1725664,207.7213835)(906.12256942,207.74138338)
\curveto(906.0725665,207.75138347)(906.00756657,207.75638347)(905.92756942,207.75638338)
\lineto(905.68756942,207.81638338)
\curveto(905.61756696,207.8263834)(905.54256703,207.84638338)(905.46256942,207.87638338)
\curveto(905.15256742,207.97638325)(904.88256769,208.10138312)(904.65256942,208.25138338)
\curveto(904.42256815,208.40138282)(904.22256835,208.59638263)(904.05256942,208.83638338)
\curveto(903.96256861,208.96638226)(903.88756869,209.10138212)(903.82756942,209.24138338)
\curveto(903.76756881,209.38138184)(903.71256886,209.53638169)(903.66256942,209.70638338)
\curveto(903.64256893,209.76638146)(903.63256894,209.83638139)(903.63256942,209.91638338)
\curveto(903.64256893,210.00638122)(903.65756892,210.07638115)(903.67756942,210.12638338)
\curveto(903.70756887,210.16638106)(903.75756882,210.20638102)(903.82756942,210.24638338)
\curveto(903.8775687,210.26638096)(903.94756863,210.27638095)(904.03756942,210.27638338)
\curveto(904.12756845,210.28638094)(904.21756836,210.28638094)(904.30756942,210.27638338)
\curveto(904.39756818,210.26638096)(904.48256809,210.25138097)(904.56256942,210.23138338)
\curveto(904.65256792,210.221381)(904.71256786,210.20638102)(904.74256942,210.18638338)
\curveto(904.81256776,210.13638109)(904.85756772,210.06138116)(904.87756942,209.96138338)
\curveto(904.90756767,209.87138135)(904.94256763,209.78638144)(904.98256942,209.70638338)
\curveto(905.08256749,209.48638174)(905.21756736,209.31638191)(905.38756942,209.19638338)
\curveto(905.50756707,209.10638212)(905.64256693,209.03638219)(905.79256942,208.98638338)
\curveto(905.94256663,208.93638229)(906.10256647,208.88638234)(906.27256942,208.83638338)
\lineto(906.58756942,208.79138338)
\lineto(906.67756942,208.79138338)
\curveto(906.74756583,208.77138245)(906.83756574,208.76138246)(906.94756942,208.76138338)
\curveto(907.06756551,208.76138246)(907.16756541,208.77138245)(907.24756942,208.79138338)
\curveto(907.31756526,208.79138243)(907.3725652,208.79638243)(907.41256942,208.80638338)
\curveto(907.4725651,208.81638241)(907.53256504,208.8213824)(907.59256942,208.82138338)
\curveto(907.65256492,208.83138239)(907.70756487,208.84138238)(907.75756942,208.85138338)
\curveto(908.04756453,208.93138229)(908.2775643,209.03638219)(908.44756942,209.16638338)
\curveto(908.61756396,209.29638193)(908.73756384,209.51638171)(908.80756942,209.82638338)
\curveto(908.82756375,209.87638135)(908.83256374,209.93138129)(908.82256942,209.99138338)
\curveto(908.81256376,210.05138117)(908.80256377,210.09638113)(908.79256942,210.12638338)
\curveto(908.74256383,210.31638091)(908.6725639,210.45638077)(908.58256942,210.54638338)
\curveto(908.49256408,210.64638058)(908.3775642,210.73638049)(908.23756942,210.81638338)
\curveto(908.14756443,210.87638035)(908.04756453,210.9263803)(907.93756942,210.96638338)
\lineto(907.60756942,211.08638338)
\curveto(907.577565,211.09638013)(907.54756503,211.10138012)(907.51756942,211.10138338)
\curveto(907.49756508,211.10138012)(907.4725651,211.11138011)(907.44256942,211.13138338)
\curveto(907.10256547,211.24137998)(906.74756583,211.3213799)(906.37756942,211.37138338)
\curveto(906.01756656,211.43137979)(905.6775669,211.5263797)(905.35756942,211.65638338)
\curveto(905.25756732,211.69637953)(905.16256741,211.73137949)(905.07256942,211.76138338)
\curveto(904.98256759,211.79137943)(904.89756768,211.83137939)(904.81756942,211.88138338)
\curveto(904.62756795,211.99137923)(904.45256812,212.11637911)(904.29256942,212.25638338)
\curveto(904.13256844,212.39637883)(904.00756857,212.57137865)(903.91756942,212.78138338)
\curveto(903.88756869,212.85137837)(903.86256871,212.9213783)(903.84256942,212.99138338)
\curveto(903.83256874,213.06137816)(903.81756876,213.13637809)(903.79756942,213.21638338)
\curveto(903.76756881,213.33637789)(903.75756882,213.47137775)(903.76756942,213.62138338)
\curveto(903.7775688,213.78137744)(903.79256878,213.91637731)(903.81256942,214.02638338)
\curveto(903.83256874,214.07637715)(903.84256873,214.11637711)(903.84256942,214.14638338)
\curveto(903.85256872,214.18637704)(903.86756871,214.226377)(903.88756942,214.26638338)
\curveto(903.9775686,214.49637673)(904.09756848,214.69637653)(904.24756942,214.86638338)
\curveto(904.40756817,215.03637619)(904.58756799,215.18637604)(904.78756942,215.31638338)
\curveto(904.93756764,215.40637582)(905.10256747,215.47637575)(905.28256942,215.52638338)
\curveto(905.46256711,215.58637564)(905.65256692,215.64137558)(905.85256942,215.69138338)
\curveto(905.92256665,215.70137552)(905.98756659,215.71137551)(906.04756942,215.72138338)
\curveto(906.11756646,215.73137549)(906.19256638,215.74137548)(906.27256942,215.75138338)
\curveto(906.30256627,215.76137546)(906.34256623,215.76137546)(906.39256942,215.75138338)
\curveto(906.44256613,215.74137548)(906.4775661,215.74637548)(906.49756942,215.76638338)
}
}
{
\newrgbcolor{curcolor}{0.40000001 0.40000001 0.40000001}
\pscustom[linestyle=none,fillstyle=solid,fillcolor=curcolor]
{
\newpath
\moveto(798.51865829,218.57142)
\lineto(813.51865829,218.57142)
\lineto(813.51865829,203.57142)
\lineto(798.51865829,203.57142)
\closepath
}
}
{
\newrgbcolor{curcolor}{0.80000001 0.80000001 0.80000001}
\pscustom[linestyle=none,fillstyle=solid,fillcolor=curcolor]
{
\newpath
\moveto(244.39332872,308.43968984)
\curveto(281.08391762,301.37128393)(313.70500034,280.59395891)(335.62333163,250.33262335)
\lineto(216.58956691,164.11628684)
\closepath
}
}
{
\newrgbcolor{curcolor}{0.90196079 0.90196079 0.90196079}
\pscustom[linestyle=none,fillstyle=solid,fillcolor=curcolor]
{
\newpath
\moveto(216.58956979,311.09347495)
\curveto(225.99075994,311.09347476)(235.37026491,310.19147838)(244.59916327,308.39988374)
\lineto(216.58956691,164.11628684)
\closepath
}
}
{
\newrgbcolor{curcolor}{0.7019608 0.7019608 0.7019608}
\pscustom[linestyle=none,fillstyle=solid,fillcolor=curcolor]
{
\newpath
\moveto(335.58833524,250.38092016)
\curveto(363.94781985,211.26009382)(371.26767784,160.699381)(355.16737212,115.14194938)
\lineto(216.58956691,164.11628684)
\closepath
}
}
{
\newrgbcolor{curcolor}{0.60000002 0.60000002 0.60000002}
\pscustom[linestyle=none,fillstyle=solid,fillcolor=curcolor]
{
\newpath
\moveto(355.2232877,115.30045656)
\curveto(342.75455511,79.89008286)(317.20846897,50.59318685)(283.82539029,33.41954671)
\lineto(216.58956691,164.11628684)
\closepath
}
}
{
\newrgbcolor{curcolor}{0.50196081 0.50196081 0.50196081}
\pscustom[linestyle=none,fillstyle=solid,fillcolor=curcolor]
{
\newpath
\moveto(284.02254811,33.52116108)
\curveto(213.9783754,-2.64624834)(127.87414614,22.88549849)(88.86711607,91.38892351)
\lineto(216.58956691,164.11628684)
\closepath
}
}
{
\newrgbcolor{curcolor}{0.40000001 0.40000001 0.40000001}
\pscustom[linestyle=none,fillstyle=solid,fillcolor=curcolor]
{
\newpath
\moveto(89.0560104,91.05818812)
\curveto(48.70713663,161.49302647)(73.09662984,251.30096957)(143.53146819,291.64984335)
\curveto(165.77217293,304.3905184)(190.95807066,311.09347545)(216.58956979,311.09347495)
\lineto(216.58956691,164.11628684)
\closepath
}
}
{
\newrgbcolor{curcolor}{0 0 0}
\pscustom[linestyle=none,fillstyle=solid,fillcolor=curcolor]
{
\newpath
\moveto(392.3377037,344.1295743)
\curveto(392.33769322,344.09956863)(392.33769322,344.05956867)(392.3377037,344.0095743)
\curveto(392.34769321,343.95956877)(392.3526932,343.90456882)(392.3527037,343.8445743)
\curveto(392.3526932,343.78456894)(392.34769321,343.729569)(392.3377037,343.6795743)
\curveto(392.33769322,343.6295691)(392.33769322,343.59456913)(392.3377037,343.5745743)
\curveto(392.33769322,343.50456922)(392.33269322,343.43456929)(392.3227037,343.3645743)
\curveto(392.32269323,343.30456942)(392.32269323,343.24456948)(392.3227037,343.1845743)
\curveto(392.30269325,343.13456959)(392.29269326,343.08456964)(392.2927037,343.0345743)
\curveto(392.30269325,342.98456974)(392.30269325,342.93456979)(392.2927037,342.8845743)
\curveto(392.27269328,342.77456995)(392.2576933,342.66457006)(392.2477037,342.5545743)
\curveto(392.23769332,342.44457028)(392.21769334,342.33457039)(392.1877037,342.2245743)
\curveto(392.13769342,342.05457067)(392.09269346,341.88957084)(392.0527037,341.7295743)
\curveto(392.01269354,341.57957115)(391.96269359,341.4295713)(391.9027037,341.2795743)
\curveto(391.73269382,340.85957187)(391.52269403,340.47957225)(391.2727037,340.1395743)
\curveto(391.02269453,339.79957293)(390.72269483,339.50957322)(390.3727037,339.2695743)
\curveto(390.17269538,339.1295736)(389.96269559,339.00957372)(389.7427037,338.9095743)
\curveto(389.53269602,338.80957392)(389.30269625,338.71957401)(389.0527037,338.6395743)
\curveto(388.9526966,338.60957412)(388.84769671,338.58457414)(388.7377037,338.5645743)
\curveto(388.63769692,338.55457417)(388.53269702,338.53457419)(388.4227037,338.5045743)
\curveto(388.37269718,338.49457423)(388.32269723,338.48957424)(388.2727037,338.4895743)
\curveto(388.23269732,338.48957424)(388.18769737,338.48457424)(388.1377037,338.4745743)
\curveto(388.09769746,338.46457426)(388.0576975,338.45957427)(388.0177037,338.4595743)
\curveto(387.97769758,338.46957426)(387.93269762,338.46957426)(387.8827037,338.4595743)
\curveto(387.86269769,338.44957428)(387.83269772,338.44457428)(387.7927037,338.4445743)
\curveto(387.7526978,338.45457427)(387.72269783,338.45457427)(387.7027037,338.4445743)
\curveto(387.62269793,338.4245743)(387.52269803,338.41957431)(387.4027037,338.4295743)
\curveto(387.28269827,338.43957429)(387.17769838,338.44457428)(387.0877037,338.4445743)
\lineto(383.5927037,338.4445743)
\curveto(383.42270213,338.44457428)(383.27770228,338.44957428)(383.1577037,338.4595743)
\curveto(383.04770251,338.47957425)(382.96770259,338.54957418)(382.9177037,338.6695743)
\curveto(382.88770267,338.74957398)(382.87270268,338.86957386)(382.8727037,339.0295743)
\curveto(382.88270267,339.19957353)(382.88770267,339.33957339)(382.8877037,339.4495743)
\lineto(382.8877037,348.2545743)
\curveto(382.88770267,348.37456435)(382.88270267,348.49956423)(382.8727037,348.6295743)
\curveto(382.87270268,348.76956396)(382.89770266,348.87956385)(382.9477037,348.9595743)
\curveto(382.98770257,349.01956371)(383.06270249,349.06956366)(383.1727037,349.1095743)
\curveto(383.19270236,349.11956361)(383.21270234,349.11956361)(383.2327037,349.1095743)
\curveto(383.2527023,349.10956362)(383.27270228,349.11456361)(383.2927037,349.1245743)
\lineto(387.3277037,349.1245743)
\curveto(387.38769817,349.1245636)(387.44769811,349.1245636)(387.5077037,349.1245743)
\curveto(387.57769798,349.13456359)(387.63769792,349.13456359)(387.6877037,349.1245743)
\lineto(387.8677037,349.1245743)
\curveto(387.91769764,349.10456362)(387.97269758,349.09456363)(388.0327037,349.0945743)
\curveto(388.09269746,349.10456362)(388.14769741,349.09956363)(388.1977037,349.0795743)
\curveto(388.2576973,349.05956367)(388.31269724,349.04956368)(388.3627037,349.0495743)
\curveto(388.42269713,349.05956367)(388.48269707,349.05456367)(388.5427037,349.0345743)
\curveto(388.68269687,349.00456372)(388.81769674,348.97456375)(388.9477037,348.9445743)
\curveto(389.07769648,348.9245638)(389.20269635,348.88956384)(389.3227037,348.8395743)
\curveto(389.43269612,348.78956394)(389.54269601,348.74456398)(389.6527037,348.7045743)
\curveto(389.76269579,348.66456406)(389.86769569,348.61456411)(389.9677037,348.5545743)
\curveto(390.21769534,348.39456433)(390.44769511,348.23956449)(390.6577037,348.0895743)
\lineto(390.7477037,347.9995743)
\curveto(390.84769471,347.91956481)(390.93769462,347.8295649)(391.0177037,347.7295743)
\lineto(391.1527037,347.6095743)
\curveto(391.20269435,347.5295652)(391.2576943,347.44956528)(391.3177037,347.3695743)
\curveto(391.38769417,347.29956543)(391.44769411,347.2245655)(391.4977037,347.1445743)
\curveto(391.62769393,346.93456579)(391.74269381,346.70956602)(391.8427037,346.4695743)
\curveto(391.94269361,346.23956649)(392.03269352,345.99456673)(392.1127037,345.7345743)
\curveto(392.16269339,345.60456712)(392.19269336,345.46956726)(392.2027037,345.3295743)
\curveto(392.22269333,345.18956754)(392.24769331,345.04956768)(392.2777037,344.9095743)
\curveto(392.27769328,344.85956787)(392.27769328,344.81456791)(392.2777037,344.7745743)
\curveto(392.28769327,344.74456798)(392.29269326,344.70956802)(392.2927037,344.6695743)
\curveto(392.31269324,344.60956812)(392.31769324,344.54456818)(392.3077037,344.4745743)
\curveto(392.30769325,344.40456832)(392.31769324,344.34456838)(392.3377037,344.2945743)
\lineto(392.3377037,344.1295743)
\moveto(389.9977037,343.4095743)
\curveto(390.01769554,343.45956927)(390.02769553,343.53956919)(390.0277037,343.6495743)
\curveto(390.02769553,343.75956897)(390.01769554,343.83956889)(389.9977037,343.8895743)
\lineto(389.9977037,344.1745743)
\curveto(389.97769558,344.26456846)(389.96269559,344.35956837)(389.9527037,344.4595743)
\curveto(389.9526956,344.55956817)(389.94269561,344.64956808)(389.9227037,344.7295743)
\curveto(389.90269565,344.77956795)(389.89269566,344.8245679)(389.8927037,344.8645743)
\curveto(389.90269565,344.91456781)(389.89769566,344.96456776)(389.8777037,345.0145743)
\curveto(389.82769573,345.17456755)(389.77769578,345.3245674)(389.7277037,345.4645743)
\curveto(389.68769587,345.61456711)(389.62769593,345.75456697)(389.5477037,345.8845743)
\curveto(389.39769616,346.1245666)(389.22269633,346.3295664)(389.0227037,346.4995743)
\curveto(388.83269672,346.67956605)(388.59769696,346.8295659)(388.3177037,346.9495743)
\curveto(388.22769733,346.97956575)(388.13769742,347.00456572)(388.0477037,347.0245743)
\curveto(387.9576976,347.05456567)(387.86769769,347.07956565)(387.7777037,347.0995743)
\curveto(387.69769786,347.10956562)(387.62269793,347.11456561)(387.5527037,347.1145743)
\curveto(387.49269806,347.1245656)(387.42269813,347.13956559)(387.3427037,347.1595743)
\curveto(387.30269825,347.16956556)(387.26269829,347.16956556)(387.2227037,347.1595743)
\curveto(387.18269837,347.15956557)(387.14769841,347.16456556)(387.1177037,347.1745743)
\lineto(386.7877037,347.1745743)
\curveto(386.73769882,347.18456554)(386.68269887,347.18456554)(386.6227037,347.1745743)
\lineto(386.4427037,347.1745743)
\lineto(385.7677037,347.1745743)
\curveto(385.74769981,347.15456557)(385.71269984,347.14956558)(385.6627037,347.1595743)
\curveto(385.62269993,347.16956556)(385.58769997,347.16956556)(385.5577037,347.1595743)
\lineto(385.4077037,347.0995743)
\curveto(385.3577002,347.08956564)(385.31770024,347.05956567)(385.2877037,347.0095743)
\curveto(385.24770031,346.95956577)(385.22770033,346.88956584)(385.2277037,346.7995743)
\lineto(385.2277037,346.4995743)
\curveto(385.22770033,346.36956636)(385.22270033,346.23456649)(385.2127037,346.0945743)
\lineto(385.2127037,345.6745743)
\lineto(385.2127037,341.4895743)
\curveto(385.21270034,341.4295713)(385.20770035,341.36457136)(385.1977037,341.2945743)
\curveto(385.19770036,341.2245715)(385.20770035,341.16457156)(385.2277037,341.1145743)
\lineto(385.2277037,340.9645743)
\lineto(385.2277037,340.7545743)
\curveto(385.23770032,340.69457203)(385.2527003,340.63957209)(385.2727037,340.5895743)
\curveto(385.33270022,340.46957226)(385.44770011,340.40457232)(385.6177037,340.3945743)
\lineto(386.1427037,340.3945743)
\lineto(387.3277037,340.3945743)
\curveto(387.72769783,340.40457232)(388.06769749,340.46457226)(388.3477037,340.5745743)
\curveto(388.71769684,340.724572)(389.00769655,340.9245718)(389.2177037,341.1745743)
\curveto(389.43769612,341.4245713)(389.62269593,341.73457099)(389.7727037,342.1045743)
\curveto(389.81269574,342.18457054)(389.84269571,342.27457045)(389.8627037,342.3745743)
\curveto(389.88269567,342.47457025)(389.90769565,342.57457015)(389.9377037,342.6745743)
\lineto(389.9377037,342.7945743)
\curveto(389.9576956,342.86456986)(389.96769559,342.93956979)(389.9677037,343.0195743)
\curveto(389.96769559,343.09956963)(389.97769558,343.17956955)(389.9977037,343.2595743)
\lineto(389.9977037,343.4095743)
}
}
{
\newrgbcolor{curcolor}{0 0 0}
\pscustom[linestyle=none,fillstyle=solid,fillcolor=curcolor]
{
\newpath
\moveto(395.83621933,349.0195743)
\curveto(395.90621638,348.93956379)(395.94121634,348.81956391)(395.94121933,348.6595743)
\lineto(395.94121933,348.1945743)
\lineto(395.94121933,347.7895743)
\curveto(395.94121634,347.64956508)(395.90621638,347.55456517)(395.83621933,347.5045743)
\curveto(395.77621651,347.45456527)(395.69621659,347.4245653)(395.59621933,347.4145743)
\curveto(395.50621678,347.40456532)(395.40621688,347.39956533)(395.29621933,347.3995743)
\lineto(394.45621933,347.3995743)
\curveto(394.34621794,347.39956533)(394.24621804,347.40456532)(394.15621933,347.4145743)
\curveto(394.07621821,347.4245653)(394.00621828,347.45456527)(393.94621933,347.5045743)
\curveto(393.90621838,347.53456519)(393.87621841,347.58956514)(393.85621933,347.6695743)
\curveto(393.84621844,347.75956497)(393.83621845,347.85456487)(393.82621933,347.9545743)
\lineto(393.82621933,348.2845743)
\curveto(393.83621845,348.39456433)(393.84121844,348.48956424)(393.84121933,348.5695743)
\lineto(393.84121933,348.7795743)
\curveto(393.85121843,348.84956388)(393.87121841,348.90956382)(393.90121933,348.9595743)
\curveto(393.92121836,348.99956373)(393.94621834,349.0295637)(393.97621933,349.0495743)
\lineto(394.09621933,349.1095743)
\curveto(394.11621817,349.10956362)(394.14121814,349.10956362)(394.17121933,349.1095743)
\curveto(394.20121808,349.11956361)(394.22621806,349.1245636)(394.24621933,349.1245743)
\lineto(395.34121933,349.1245743)
\curveto(395.44121684,349.1245636)(395.53621675,349.11956361)(395.62621933,349.1095743)
\curveto(395.71621657,349.09956363)(395.7862165,349.06956366)(395.83621933,349.0195743)
\moveto(395.94121933,339.2545743)
\curveto(395.94121634,339.05457367)(395.93621635,338.88457384)(395.92621933,338.7445743)
\curveto(395.91621637,338.60457412)(395.82621646,338.50957422)(395.65621933,338.4595743)
\curveto(395.59621669,338.43957429)(395.53121675,338.4295743)(395.46121933,338.4295743)
\curveto(395.39121689,338.43957429)(395.31621697,338.44457428)(395.23621933,338.4445743)
\lineto(394.39621933,338.4445743)
\curveto(394.30621798,338.44457428)(394.21621807,338.44957428)(394.12621933,338.4595743)
\curveto(394.04621824,338.46957426)(393.9862183,338.49957423)(393.94621933,338.5495743)
\curveto(393.8862184,338.61957411)(393.85121843,338.70457402)(393.84121933,338.8045743)
\lineto(393.84121933,339.1495743)
\lineto(393.84121933,345.4795743)
\lineto(393.84121933,345.7795743)
\curveto(393.84121844,345.87956685)(393.86121842,345.95956677)(393.90121933,346.0195743)
\curveto(393.96121832,346.08956664)(394.04621824,346.13456659)(394.15621933,346.1545743)
\curveto(394.17621811,346.16456656)(394.20121808,346.16456656)(394.23121933,346.1545743)
\curveto(394.27121801,346.15456657)(394.30121798,346.15956657)(394.32121933,346.1695743)
\lineto(395.07121933,346.1695743)
\lineto(395.26621933,346.1695743)
\curveto(395.34621694,346.17956655)(395.41121687,346.17956655)(395.46121933,346.1695743)
\lineto(395.58121933,346.1695743)
\curveto(395.64121664,346.14956658)(395.69621659,346.13456659)(395.74621933,346.1245743)
\curveto(395.79621649,346.11456661)(395.83621645,346.08456664)(395.86621933,346.0345743)
\curveto(395.90621638,345.98456674)(395.92621636,345.91456681)(395.92621933,345.8245743)
\curveto(395.93621635,345.73456699)(395.94121634,345.63956709)(395.94121933,345.5395743)
\lineto(395.94121933,339.2545743)
}
}
{
\newrgbcolor{curcolor}{0 0 0}
\pscustom[linestyle=none,fillstyle=solid,fillcolor=curcolor]
{
\newpath
\moveto(400.57340683,346.3795743)
\curveto(401.32340233,346.39956633)(401.97340168,346.31456641)(402.52340683,346.1245743)
\curveto(403.08340057,345.94456678)(403.50840014,345.6295671)(403.79840683,345.1795743)
\curveto(403.86839978,345.06956766)(403.92839972,344.95456777)(403.97840683,344.8345743)
\curveto(404.03839961,344.724568)(404.08839956,344.59956813)(404.12840683,344.4595743)
\curveto(404.1483995,344.39956833)(404.15839949,344.33456839)(404.15840683,344.2645743)
\curveto(404.15839949,344.19456853)(404.1483995,344.13456859)(404.12840683,344.0845743)
\curveto(404.08839956,344.0245687)(404.03339962,343.98456874)(403.96340683,343.9645743)
\curveto(403.91339974,343.94456878)(403.8533998,343.93456879)(403.78340683,343.9345743)
\lineto(403.57340683,343.9345743)
\lineto(402.91340683,343.9345743)
\curveto(402.84340081,343.93456879)(402.77340088,343.9295688)(402.70340683,343.9195743)
\curveto(402.63340102,343.91956881)(402.56840108,343.9295688)(402.50840683,343.9495743)
\curveto(402.40840124,343.96956876)(402.33340132,344.00956872)(402.28340683,344.0695743)
\curveto(402.23340142,344.1295686)(402.18840146,344.18956854)(402.14840683,344.2495743)
\lineto(402.02840683,344.4595743)
\curveto(401.99840165,344.53956819)(401.9484017,344.60456812)(401.87840683,344.6545743)
\curveto(401.77840187,344.73456799)(401.67840197,344.79456793)(401.57840683,344.8345743)
\curveto(401.48840216,344.87456785)(401.37340228,344.90956782)(401.23340683,344.9395743)
\curveto(401.16340249,344.95956777)(401.05840259,344.97456775)(400.91840683,344.9845743)
\curveto(400.78840286,344.99456773)(400.68840296,344.98956774)(400.61840683,344.9695743)
\lineto(400.51340683,344.9695743)
\lineto(400.36340683,344.9395743)
\curveto(400.32340333,344.93956779)(400.27840337,344.93456779)(400.22840683,344.9245743)
\curveto(400.05840359,344.87456785)(399.91840373,344.80456792)(399.80840683,344.7145743)
\curveto(399.70840394,344.63456809)(399.63840401,344.50956822)(399.59840683,344.3395743)
\curveto(399.57840407,344.26956846)(399.57840407,344.20456852)(399.59840683,344.1445743)
\curveto(399.61840403,344.08456864)(399.63840401,344.03456869)(399.65840683,343.9945743)
\curveto(399.72840392,343.87456885)(399.80840384,343.77956895)(399.89840683,343.7095743)
\curveto(399.99840365,343.63956909)(400.11340354,343.57956915)(400.24340683,343.5295743)
\curveto(400.43340322,343.44956928)(400.63840301,343.37956935)(400.85840683,343.3195743)
\lineto(401.54840683,343.1695743)
\curveto(401.78840186,343.1295696)(402.01840163,343.07956965)(402.23840683,343.0195743)
\curveto(402.46840118,342.96956976)(402.68340097,342.90456982)(402.88340683,342.8245743)
\curveto(402.97340068,342.78456994)(403.05840059,342.74956998)(403.13840683,342.7195743)
\curveto(403.22840042,342.69957003)(403.31340034,342.66457006)(403.39340683,342.6145743)
\curveto(403.58340007,342.49457023)(403.7533999,342.36457036)(403.90340683,342.2245743)
\curveto(404.06339959,342.08457064)(404.18839946,341.90957082)(404.27840683,341.6995743)
\curveto(404.30839934,341.6295711)(404.33339932,341.55957117)(404.35340683,341.4895743)
\curveto(404.37339928,341.41957131)(404.39339926,341.34457138)(404.41340683,341.2645743)
\curveto(404.42339923,341.20457152)(404.42839922,341.10957162)(404.42840683,340.9795743)
\curveto(404.43839921,340.85957187)(404.43839921,340.76457196)(404.42840683,340.6945743)
\lineto(404.42840683,340.6195743)
\curveto(404.40839924,340.55957217)(404.39339926,340.49957223)(404.38340683,340.4395743)
\curveto(404.38339927,340.38957234)(404.37839927,340.33957239)(404.36840683,340.2895743)
\curveto(404.29839935,339.98957274)(404.18839946,339.724573)(404.03840683,339.4945743)
\curveto(403.87839977,339.25457347)(403.68339997,339.05957367)(403.45340683,338.9095743)
\curveto(403.22340043,338.75957397)(402.96340069,338.6295741)(402.67340683,338.5195743)
\curveto(402.56340109,338.46957426)(402.44340121,338.43457429)(402.31340683,338.4145743)
\curveto(402.19340146,338.39457433)(402.07340158,338.36957436)(401.95340683,338.3395743)
\curveto(401.86340179,338.31957441)(401.76840188,338.30957442)(401.66840683,338.3095743)
\curveto(401.57840207,338.29957443)(401.48840216,338.28457444)(401.39840683,338.2645743)
\lineto(401.12840683,338.2645743)
\curveto(401.06840258,338.24457448)(400.96340269,338.23457449)(400.81340683,338.2345743)
\curveto(400.67340298,338.23457449)(400.57340308,338.24457448)(400.51340683,338.2645743)
\curveto(400.48340317,338.26457446)(400.4484032,338.26957446)(400.40840683,338.2795743)
\lineto(400.30340683,338.2795743)
\curveto(400.18340347,338.29957443)(400.06340359,338.31457441)(399.94340683,338.3245743)
\curveto(399.82340383,338.33457439)(399.70840394,338.35457437)(399.59840683,338.3845743)
\curveto(399.20840444,338.49457423)(398.86340479,338.61957411)(398.56340683,338.7595743)
\curveto(398.26340539,338.90957382)(398.00840564,339.1295736)(397.79840683,339.4195743)
\curveto(397.65840599,339.60957312)(397.53840611,339.8295729)(397.43840683,340.0795743)
\curveto(397.41840623,340.13957259)(397.39840625,340.21957251)(397.37840683,340.3195743)
\curveto(397.35840629,340.36957236)(397.34340631,340.43957229)(397.33340683,340.5295743)
\curveto(397.32340633,340.61957211)(397.32840632,340.69457203)(397.34840683,340.7545743)
\curveto(397.37840627,340.8245719)(397.42840622,340.87457185)(397.49840683,340.9045743)
\curveto(397.5484061,340.9245718)(397.60840604,340.93457179)(397.67840683,340.9345743)
\lineto(397.90340683,340.9345743)
\lineto(398.60840683,340.9345743)
\lineto(398.84840683,340.9345743)
\curveto(398.92840472,340.93457179)(398.99840465,340.9245718)(399.05840683,340.9045743)
\curveto(399.16840448,340.86457186)(399.23840441,340.79957193)(399.26840683,340.7095743)
\curveto(399.30840434,340.61957211)(399.3534043,340.5245722)(399.40340683,340.4245743)
\curveto(399.42340423,340.37457235)(399.45840419,340.30957242)(399.50840683,340.2295743)
\curveto(399.56840408,340.14957258)(399.61840403,340.09957263)(399.65840683,340.0795743)
\curveto(399.77840387,339.97957275)(399.89340376,339.89957283)(400.00340683,339.8395743)
\curveto(400.11340354,339.78957294)(400.2534034,339.73957299)(400.42340683,339.6895743)
\curveto(400.47340318,339.66957306)(400.52340313,339.65957307)(400.57340683,339.6595743)
\curveto(400.62340303,339.66957306)(400.67340298,339.66957306)(400.72340683,339.6595743)
\curveto(400.80340285,339.63957309)(400.88840276,339.6295731)(400.97840683,339.6295743)
\curveto(401.07840257,339.63957309)(401.16340249,339.65457307)(401.23340683,339.6745743)
\curveto(401.28340237,339.68457304)(401.32840232,339.68957304)(401.36840683,339.6895743)
\curveto(401.41840223,339.68957304)(401.46840218,339.69957303)(401.51840683,339.7195743)
\curveto(401.65840199,339.76957296)(401.78340187,339.8295729)(401.89340683,339.8995743)
\curveto(402.01340164,339.96957276)(402.10840154,340.05957267)(402.17840683,340.1695743)
\curveto(402.22840142,340.24957248)(402.26840138,340.37457235)(402.29840683,340.5445743)
\curveto(402.31840133,340.61457211)(402.31840133,340.67957205)(402.29840683,340.7395743)
\curveto(402.27840137,340.79957193)(402.25840139,340.84957188)(402.23840683,340.8895743)
\curveto(402.16840148,341.0295717)(402.07840157,341.13457159)(401.96840683,341.2045743)
\curveto(401.86840178,341.27457145)(401.7484019,341.33957139)(401.60840683,341.3995743)
\curveto(401.41840223,341.47957125)(401.21840243,341.54457118)(401.00840683,341.5945743)
\curveto(400.79840285,341.64457108)(400.58840306,341.69957103)(400.37840683,341.7595743)
\curveto(400.29840335,341.77957095)(400.21340344,341.79457093)(400.12340683,341.8045743)
\curveto(400.04340361,341.81457091)(399.96340369,341.8295709)(399.88340683,341.8495743)
\curveto(399.56340409,341.93957079)(399.25840439,342.0245707)(398.96840683,342.1045743)
\curveto(398.67840497,342.19457053)(398.41340524,342.3245704)(398.17340683,342.4945743)
\curveto(397.89340576,342.69457003)(397.68840596,342.96456976)(397.55840683,343.3045743)
\curveto(397.53840611,343.37456935)(397.51840613,343.46956926)(397.49840683,343.5895743)
\curveto(397.47840617,343.65956907)(397.46340619,343.74456898)(397.45340683,343.8445743)
\curveto(397.44340621,343.94456878)(397.4484062,344.03456869)(397.46840683,344.1145743)
\curveto(397.48840616,344.16456856)(397.49340616,344.20456852)(397.48340683,344.2345743)
\curveto(397.47340618,344.27456845)(397.47840617,344.31956841)(397.49840683,344.3695743)
\curveto(397.51840613,344.47956825)(397.53840611,344.57956815)(397.55840683,344.6695743)
\curveto(397.58840606,344.76956796)(397.62340603,344.86456786)(397.66340683,344.9545743)
\curveto(397.79340586,345.24456748)(397.97340568,345.47956725)(398.20340683,345.6595743)
\curveto(398.43340522,345.83956689)(398.69340496,345.98456674)(398.98340683,346.0945743)
\curveto(399.09340456,346.14456658)(399.20840444,346.17956655)(399.32840683,346.1995743)
\curveto(399.4484042,346.2295665)(399.57340408,346.25956647)(399.70340683,346.2895743)
\curveto(399.76340389,346.30956642)(399.82340383,346.31956641)(399.88340683,346.3195743)
\lineto(400.06340683,346.3495743)
\curveto(400.14340351,346.35956637)(400.22840342,346.36456636)(400.31840683,346.3645743)
\curveto(400.40840324,346.36456636)(400.49340316,346.36956636)(400.57340683,346.3795743)
}
}
{
\newrgbcolor{curcolor}{0 0 0}
\pscustom[linestyle=none,fillstyle=solid,fillcolor=curcolor]
{
\newpath
\moveto(406.71004745,348.4795743)
\lineto(407.71504745,348.4795743)
\curveto(407.86504447,348.47956425)(407.99504434,348.46956426)(408.10504745,348.4495743)
\curveto(408.22504411,348.43956429)(408.31004402,348.37956435)(408.36004745,348.2695743)
\curveto(408.38004395,348.21956451)(408.39004394,348.15956457)(408.39004745,348.0895743)
\lineto(408.39004745,347.8795743)
\lineto(408.39004745,347.2045743)
\curveto(408.39004394,347.15456557)(408.38504395,347.09456563)(408.37504745,347.0245743)
\curveto(408.37504396,346.96456576)(408.38004395,346.90956582)(408.39004745,346.8595743)
\lineto(408.39004745,346.6945743)
\curveto(408.39004394,346.61456611)(408.39504394,346.53956619)(408.40504745,346.4695743)
\curveto(408.41504392,346.40956632)(408.44004389,346.35456637)(408.48004745,346.3045743)
\curveto(408.55004378,346.21456651)(408.67504366,346.16456656)(408.85504745,346.1545743)
\lineto(409.39504745,346.1545743)
\lineto(409.57504745,346.1545743)
\curveto(409.6350427,346.15456657)(409.69004264,346.14456658)(409.74004745,346.1245743)
\curveto(409.85004248,346.07456665)(409.91004242,345.98456674)(409.92004745,345.8545743)
\curveto(409.94004239,345.724567)(409.95004238,345.57956715)(409.95004745,345.4195743)
\lineto(409.95004745,345.2095743)
\curveto(409.96004237,345.13956759)(409.95504238,345.07956765)(409.93504745,345.0295743)
\curveto(409.88504245,344.86956786)(409.78004255,344.78456794)(409.62004745,344.7745743)
\curveto(409.46004287,344.76456796)(409.28004305,344.75956797)(409.08004745,344.7595743)
\lineto(408.94504745,344.7595743)
\curveto(408.90504343,344.76956796)(408.87004346,344.76956796)(408.84004745,344.7595743)
\curveto(408.80004353,344.74956798)(408.76504357,344.74456798)(408.73504745,344.7445743)
\curveto(408.70504363,344.75456797)(408.67504366,344.74956798)(408.64504745,344.7295743)
\curveto(408.56504377,344.70956802)(408.50504383,344.66456806)(408.46504745,344.5945743)
\curveto(408.4350439,344.53456819)(408.41004392,344.45956827)(408.39004745,344.3695743)
\curveto(408.38004395,344.31956841)(408.38004395,344.26456846)(408.39004745,344.2045743)
\curveto(408.40004393,344.14456858)(408.40004393,344.08956864)(408.39004745,344.0395743)
\lineto(408.39004745,343.1095743)
\lineto(408.39004745,341.3545743)
\curveto(408.39004394,341.10457162)(408.39504394,340.88457184)(408.40504745,340.6945743)
\curveto(408.42504391,340.51457221)(408.49004384,340.35457237)(408.60004745,340.2145743)
\curveto(408.65004368,340.15457257)(408.71504362,340.10957262)(408.79504745,340.0795743)
\lineto(409.06504745,340.0195743)
\curveto(409.09504324,340.00957272)(409.12504321,340.00457272)(409.15504745,340.0045743)
\curveto(409.19504314,340.01457271)(409.22504311,340.01457271)(409.24504745,340.0045743)
\lineto(409.41004745,340.0045743)
\curveto(409.52004281,340.00457272)(409.61504272,339.99957273)(409.69504745,339.9895743)
\curveto(409.77504256,339.97957275)(409.84004249,339.93957279)(409.89004745,339.8695743)
\curveto(409.9300424,339.80957292)(409.95004238,339.729573)(409.95004745,339.6295743)
\lineto(409.95004745,339.3445743)
\curveto(409.95004238,339.13457359)(409.94504239,338.93957379)(409.93504745,338.7595743)
\curveto(409.9350424,338.58957414)(409.85504248,338.47457425)(409.69504745,338.4145743)
\curveto(409.64504269,338.39457433)(409.60004273,338.38957434)(409.56004745,338.3995743)
\curveto(409.52004281,338.39957433)(409.47504286,338.38957434)(409.42504745,338.3695743)
\lineto(409.27504745,338.3695743)
\curveto(409.25504308,338.36957436)(409.22504311,338.37457435)(409.18504745,338.3845743)
\curveto(409.14504319,338.38457434)(409.11004322,338.37957435)(409.08004745,338.3695743)
\curveto(409.0300433,338.35957437)(408.97504336,338.35957437)(408.91504745,338.3695743)
\lineto(408.76504745,338.3695743)
\lineto(408.61504745,338.3695743)
\curveto(408.56504377,338.35957437)(408.52004381,338.35957437)(408.48004745,338.3695743)
\lineto(408.31504745,338.3695743)
\curveto(408.26504407,338.37957435)(408.21004412,338.38457434)(408.15004745,338.3845743)
\curveto(408.09004424,338.38457434)(408.0350443,338.38957434)(407.98504745,338.3995743)
\curveto(407.91504442,338.40957432)(407.85004448,338.41957431)(407.79004745,338.4295743)
\lineto(407.61004745,338.4595743)
\curveto(407.50004483,338.48957424)(407.39504494,338.5245742)(407.29504745,338.5645743)
\curveto(407.19504514,338.60457412)(407.10004523,338.64957408)(407.01004745,338.6995743)
\lineto(406.92004745,338.7595743)
\curveto(406.89004544,338.78957394)(406.85504548,338.81957391)(406.81504745,338.8495743)
\curveto(406.79504554,338.86957386)(406.77004556,338.88957384)(406.74004745,338.9095743)
\lineto(406.66504745,338.9845743)
\curveto(406.52504581,339.17457355)(406.42004591,339.38457334)(406.35004745,339.6145743)
\curveto(406.330046,339.65457307)(406.32004601,339.68957304)(406.32004745,339.7195743)
\curveto(406.330046,339.75957297)(406.330046,339.80457292)(406.32004745,339.8545743)
\curveto(406.31004602,339.87457285)(406.30504603,339.89957283)(406.30504745,339.9295743)
\curveto(406.30504603,339.95957277)(406.30004603,339.98457274)(406.29004745,340.0045743)
\lineto(406.29004745,340.1545743)
\curveto(406.28004605,340.19457253)(406.27504606,340.23957249)(406.27504745,340.2895743)
\curveto(406.28504605,340.33957239)(406.29004604,340.38957234)(406.29004745,340.4395743)
\lineto(406.29004745,341.0095743)
\lineto(406.29004745,343.2445743)
\lineto(406.29004745,344.0395743)
\lineto(406.29004745,344.2495743)
\curveto(406.30004603,344.31956841)(406.29504604,344.38456834)(406.27504745,344.4445743)
\curveto(406.2350461,344.58456814)(406.16504617,344.67456805)(406.06504745,344.7145743)
\curveto(405.95504638,344.76456796)(405.81504652,344.77956795)(405.64504745,344.7595743)
\curveto(405.47504686,344.73956799)(405.330047,344.75456797)(405.21004745,344.8045743)
\curveto(405.1300472,344.83456789)(405.08004725,344.87956785)(405.06004745,344.9395743)
\curveto(405.04004729,344.99956773)(405.02004731,345.07456765)(405.00004745,345.1645743)
\lineto(405.00004745,345.4795743)
\curveto(405.00004733,345.65956707)(405.01004732,345.80456692)(405.03004745,345.9145743)
\curveto(405.05004728,346.0245667)(405.1350472,346.09956663)(405.28504745,346.1395743)
\curveto(405.32504701,346.15956657)(405.36504697,346.16456656)(405.40504745,346.1545743)
\lineto(405.54004745,346.1545743)
\curveto(405.69004664,346.15456657)(405.8300465,346.15956657)(405.96004745,346.1695743)
\curveto(406.09004624,346.18956654)(406.18004615,346.24956648)(406.23004745,346.3495743)
\curveto(406.26004607,346.41956631)(406.27504606,346.49956623)(406.27504745,346.5895743)
\curveto(406.28504605,346.67956605)(406.29004604,346.76956596)(406.29004745,346.8595743)
\lineto(406.29004745,347.7895743)
\lineto(406.29004745,348.0445743)
\curveto(406.29004604,348.13456459)(406.30004603,348.20956452)(406.32004745,348.2695743)
\curveto(406.37004596,348.36956436)(406.44504589,348.43456429)(406.54504745,348.4645743)
\curveto(406.56504577,348.47456425)(406.59004574,348.47456425)(406.62004745,348.4645743)
\curveto(406.66004567,348.46456426)(406.69004564,348.46956426)(406.71004745,348.4795743)
}
}
{
\newrgbcolor{curcolor}{0 0 0}
\pscustom[linestyle=none,fillstyle=solid,fillcolor=curcolor]
{
\newpath
\moveto(415.35848495,346.3645743)
\curveto(415.46847964,346.36456636)(415.56347954,346.35456637)(415.64348495,346.3345743)
\curveto(415.73347937,346.31456641)(415.8034793,346.26956646)(415.85348495,346.1995743)
\curveto(415.91347919,346.11956661)(415.94347916,345.97956675)(415.94348495,345.7795743)
\lineto(415.94348495,345.2695743)
\lineto(415.94348495,344.8945743)
\curveto(415.95347915,344.75456797)(415.93847917,344.64456808)(415.89848495,344.5645743)
\curveto(415.85847925,344.49456823)(415.79847931,344.44956828)(415.71848495,344.4295743)
\curveto(415.64847946,344.40956832)(415.56347954,344.39956833)(415.46348495,344.3995743)
\curveto(415.37347973,344.39956833)(415.27347983,344.40456832)(415.16348495,344.4145743)
\curveto(415.06348004,344.4245683)(414.96848014,344.41956831)(414.87848495,344.3995743)
\curveto(414.8084803,344.37956835)(414.73848037,344.36456836)(414.66848495,344.3545743)
\curveto(414.59848051,344.35456837)(414.53348057,344.34456838)(414.47348495,344.3245743)
\curveto(414.31348079,344.27456845)(414.15348095,344.19956853)(413.99348495,344.0995743)
\curveto(413.83348127,344.00956872)(413.7084814,343.90456882)(413.61848495,343.7845743)
\curveto(413.56848154,343.70456902)(413.51348159,343.61956911)(413.45348495,343.5295743)
\curveto(413.4034817,343.44956928)(413.35348175,343.36456936)(413.30348495,343.2745743)
\curveto(413.27348183,343.19456953)(413.24348186,343.10956962)(413.21348495,343.0195743)
\lineto(413.15348495,342.7795743)
\curveto(413.13348197,342.70957002)(413.12348198,342.63457009)(413.12348495,342.5545743)
\curveto(413.12348198,342.48457024)(413.11348199,342.41457031)(413.09348495,342.3445743)
\curveto(413.08348202,342.30457042)(413.07848203,342.26457046)(413.07848495,342.2245743)
\curveto(413.08848202,342.19457053)(413.08848202,342.16457056)(413.07848495,342.1345743)
\lineto(413.07848495,341.8945743)
\curveto(413.05848205,341.8245709)(413.05348205,341.74457098)(413.06348495,341.6545743)
\curveto(413.07348203,341.57457115)(413.07848203,341.49457123)(413.07848495,341.4145743)
\lineto(413.07848495,340.4545743)
\lineto(413.07848495,339.1795743)
\curveto(413.07848203,339.04957368)(413.07348203,338.9295738)(413.06348495,338.8195743)
\curveto(413.05348205,338.70957402)(413.02348208,338.61957411)(412.97348495,338.5495743)
\curveto(412.95348215,338.51957421)(412.91848219,338.49457423)(412.86848495,338.4745743)
\curveto(412.82848228,338.46457426)(412.78348232,338.45457427)(412.73348495,338.4445743)
\lineto(412.65848495,338.4445743)
\curveto(412.6084825,338.43457429)(412.55348255,338.4295743)(412.49348495,338.4295743)
\lineto(412.32848495,338.4295743)
\lineto(411.68348495,338.4295743)
\curveto(411.62348348,338.43957429)(411.55848355,338.44457428)(411.48848495,338.4445743)
\lineto(411.29348495,338.4445743)
\curveto(411.24348386,338.46457426)(411.19348391,338.47957425)(411.14348495,338.4895743)
\curveto(411.09348401,338.50957422)(411.05848405,338.54457418)(411.03848495,338.5945743)
\curveto(410.99848411,338.64457408)(410.97348413,338.71457401)(410.96348495,338.8045743)
\lineto(410.96348495,339.1045743)
\lineto(410.96348495,340.1245743)
\lineto(410.96348495,344.3545743)
\lineto(410.96348495,345.4645743)
\lineto(410.96348495,345.7495743)
\curveto(410.96348414,345.84956688)(410.98348412,345.9295668)(411.02348495,345.9895743)
\curveto(411.07348403,346.06956666)(411.14848396,346.11956661)(411.24848495,346.1395743)
\curveto(411.34848376,346.15956657)(411.46848364,346.16956656)(411.60848495,346.1695743)
\lineto(412.37348495,346.1695743)
\curveto(412.49348261,346.16956656)(412.59848251,346.15956657)(412.68848495,346.1395743)
\curveto(412.77848233,346.1295666)(412.84848226,346.08456664)(412.89848495,346.0045743)
\curveto(412.92848218,345.95456677)(412.94348216,345.88456684)(412.94348495,345.7945743)
\lineto(412.97348495,345.5245743)
\curveto(412.98348212,345.44456728)(412.99848211,345.36956736)(413.01848495,345.2995743)
\curveto(413.04848206,345.2295675)(413.09848201,345.19456753)(413.16848495,345.1945743)
\curveto(413.18848192,345.21456751)(413.2084819,345.2245675)(413.22848495,345.2245743)
\curveto(413.24848186,345.2245675)(413.26848184,345.23456749)(413.28848495,345.2545743)
\curveto(413.34848176,345.30456742)(413.39848171,345.35956737)(413.43848495,345.4195743)
\curveto(413.48848162,345.48956724)(413.54848156,345.54956718)(413.61848495,345.5995743)
\curveto(413.65848145,345.6295671)(413.69348141,345.65956707)(413.72348495,345.6895743)
\curveto(413.75348135,345.729567)(413.78848132,345.76456696)(413.82848495,345.7945743)
\lineto(414.09848495,345.9745743)
\curveto(414.19848091,346.03456669)(414.29848081,346.08956664)(414.39848495,346.1395743)
\curveto(414.49848061,346.17956655)(414.59848051,346.21456651)(414.69848495,346.2445743)
\lineto(415.02848495,346.3345743)
\curveto(415.05848005,346.34456638)(415.11347999,346.34456638)(415.19348495,346.3345743)
\curveto(415.28347982,346.33456639)(415.33847977,346.34456638)(415.35848495,346.3645743)
}
}
{
\newrgbcolor{curcolor}{0 0 0}
\pscustom[linestyle=none,fillstyle=solid,fillcolor=curcolor]
{
\newpath
\moveto(418.86356308,349.0195743)
\curveto(418.93356013,348.93956379)(418.96856009,348.81956391)(418.96856308,348.6595743)
\lineto(418.96856308,348.1945743)
\lineto(418.96856308,347.7895743)
\curveto(418.96856009,347.64956508)(418.93356013,347.55456517)(418.86356308,347.5045743)
\curveto(418.80356026,347.45456527)(418.72356034,347.4245653)(418.62356308,347.4145743)
\curveto(418.53356053,347.40456532)(418.43356063,347.39956533)(418.32356308,347.3995743)
\lineto(417.48356308,347.3995743)
\curveto(417.37356169,347.39956533)(417.27356179,347.40456532)(417.18356308,347.4145743)
\curveto(417.10356196,347.4245653)(417.03356203,347.45456527)(416.97356308,347.5045743)
\curveto(416.93356213,347.53456519)(416.90356216,347.58956514)(416.88356308,347.6695743)
\curveto(416.87356219,347.75956497)(416.8635622,347.85456487)(416.85356308,347.9545743)
\lineto(416.85356308,348.2845743)
\curveto(416.8635622,348.39456433)(416.86856219,348.48956424)(416.86856308,348.5695743)
\lineto(416.86856308,348.7795743)
\curveto(416.87856218,348.84956388)(416.89856216,348.90956382)(416.92856308,348.9595743)
\curveto(416.94856211,348.99956373)(416.97356209,349.0295637)(417.00356308,349.0495743)
\lineto(417.12356308,349.1095743)
\curveto(417.14356192,349.10956362)(417.16856189,349.10956362)(417.19856308,349.1095743)
\curveto(417.22856183,349.11956361)(417.25356181,349.1245636)(417.27356308,349.1245743)
\lineto(418.36856308,349.1245743)
\curveto(418.46856059,349.1245636)(418.5635605,349.11956361)(418.65356308,349.1095743)
\curveto(418.74356032,349.09956363)(418.81356025,349.06956366)(418.86356308,349.0195743)
\moveto(418.96856308,339.2545743)
\curveto(418.96856009,339.05457367)(418.9635601,338.88457384)(418.95356308,338.7445743)
\curveto(418.94356012,338.60457412)(418.85356021,338.50957422)(418.68356308,338.4595743)
\curveto(418.62356044,338.43957429)(418.5585605,338.4295743)(418.48856308,338.4295743)
\curveto(418.41856064,338.43957429)(418.34356072,338.44457428)(418.26356308,338.4445743)
\lineto(417.42356308,338.4445743)
\curveto(417.33356173,338.44457428)(417.24356182,338.44957428)(417.15356308,338.4595743)
\curveto(417.07356199,338.46957426)(417.01356205,338.49957423)(416.97356308,338.5495743)
\curveto(416.91356215,338.61957411)(416.87856218,338.70457402)(416.86856308,338.8045743)
\lineto(416.86856308,339.1495743)
\lineto(416.86856308,345.4795743)
\lineto(416.86856308,345.7795743)
\curveto(416.86856219,345.87956685)(416.88856217,345.95956677)(416.92856308,346.0195743)
\curveto(416.98856207,346.08956664)(417.07356199,346.13456659)(417.18356308,346.1545743)
\curveto(417.20356186,346.16456656)(417.22856183,346.16456656)(417.25856308,346.1545743)
\curveto(417.29856176,346.15456657)(417.32856173,346.15956657)(417.34856308,346.1695743)
\lineto(418.09856308,346.1695743)
\lineto(418.29356308,346.1695743)
\curveto(418.37356069,346.17956655)(418.43856062,346.17956655)(418.48856308,346.1695743)
\lineto(418.60856308,346.1695743)
\curveto(418.66856039,346.14956658)(418.72356034,346.13456659)(418.77356308,346.1245743)
\curveto(418.82356024,346.11456661)(418.8635602,346.08456664)(418.89356308,346.0345743)
\curveto(418.93356013,345.98456674)(418.95356011,345.91456681)(418.95356308,345.8245743)
\curveto(418.9635601,345.73456699)(418.96856009,345.63956709)(418.96856308,345.5395743)
\lineto(418.96856308,339.2545743)
}
}
{
\newrgbcolor{curcolor}{0 0 0}
\pscustom[linestyle=none,fillstyle=solid,fillcolor=curcolor]
{
\newpath
\moveto(428.43075058,342.6895743)
\curveto(428.45074198,342.6295701)(428.46074197,342.5245702)(428.46075058,342.3745743)
\curveto(428.46074197,342.23457049)(428.45574197,342.13457059)(428.44575058,342.0745743)
\curveto(428.44574198,342.0245707)(428.44074199,341.97957075)(428.43075058,341.9395743)
\lineto(428.43075058,341.8195743)
\curveto(428.41074202,341.73957099)(428.40074203,341.65957107)(428.40075058,341.5795743)
\curveto(428.40074203,341.50957122)(428.39074204,341.43457129)(428.37075058,341.3545743)
\curveto(428.37074206,341.31457141)(428.36074207,341.24457148)(428.34075058,341.1445743)
\curveto(428.31074212,341.0245717)(428.28074215,340.89957183)(428.25075058,340.7695743)
\curveto(428.2307422,340.64957208)(428.19574223,340.53457219)(428.14575058,340.4245743)
\curveto(427.96574246,339.97457275)(427.74074269,339.58457314)(427.47075058,339.2545743)
\curveto(427.20074323,338.9245738)(426.84574358,338.66457406)(426.40575058,338.4745743)
\curveto(426.31574411,338.43457429)(426.22074421,338.40457432)(426.12075058,338.3845743)
\curveto(426.0307444,338.35457437)(425.9307445,338.3245744)(425.82075058,338.2945743)
\curveto(425.76074467,338.27457445)(425.69574473,338.26457446)(425.62575058,338.2645743)
\curveto(425.56574486,338.26457446)(425.50574492,338.25957447)(425.44575058,338.2495743)
\lineto(425.31075058,338.2495743)
\curveto(425.25074518,338.2295745)(425.17074526,338.2245745)(425.07075058,338.2345743)
\curveto(424.97074546,338.23457449)(424.89074554,338.24457448)(424.83075058,338.2645743)
\lineto(424.74075058,338.2645743)
\curveto(424.69074574,338.27457445)(424.63574579,338.28457444)(424.57575058,338.2945743)
\curveto(424.51574591,338.29457443)(424.45574597,338.29957443)(424.39575058,338.3095743)
\curveto(424.20574622,338.35957437)(424.0307464,338.40957432)(423.87075058,338.4595743)
\curveto(423.71074672,338.50957422)(423.56074687,338.57957415)(423.42075058,338.6695743)
\lineto(423.24075058,338.7895743)
\curveto(423.19074724,338.8295739)(423.14074729,338.87457385)(423.09075058,338.9245743)
\lineto(423.00075058,338.9845743)
\curveto(422.97074746,339.00457372)(422.94074749,339.01957371)(422.91075058,339.0295743)
\curveto(422.82074761,339.05957367)(422.76574766,339.03957369)(422.74575058,338.9695743)
\curveto(422.69574773,338.89957383)(422.66074777,338.81457391)(422.64075058,338.7145743)
\curveto(422.6307478,338.6245741)(422.59574783,338.55457417)(422.53575058,338.5045743)
\curveto(422.47574795,338.46457426)(422.40574802,338.43957429)(422.32575058,338.4295743)
\lineto(422.05575058,338.4295743)
\lineto(421.33575058,338.4295743)
\lineto(421.11075058,338.4295743)
\curveto(421.04074939,338.41957431)(420.97574945,338.4245743)(420.91575058,338.4445743)
\curveto(420.77574965,338.49457423)(420.69574973,338.58457414)(420.67575058,338.7145743)
\curveto(420.66574976,338.85457387)(420.66074977,339.00957372)(420.66075058,339.1795743)
\lineto(420.66075058,348.3295743)
\lineto(420.66075058,348.6745743)
\curveto(420.66074977,348.79456393)(420.68574974,348.88956384)(420.73575058,348.9595743)
\curveto(420.77574965,349.0295637)(420.84574958,349.07456365)(420.94575058,349.0945743)
\curveto(420.96574946,349.10456362)(420.98574944,349.10456362)(421.00575058,349.0945743)
\curveto(421.03574939,349.09456363)(421.06074937,349.09956363)(421.08075058,349.1095743)
\lineto(422.02575058,349.1095743)
\curveto(422.20574822,349.10956362)(422.36074807,349.09956363)(422.49075058,349.0795743)
\curveto(422.62074781,349.06956366)(422.70574772,348.99456373)(422.74575058,348.8545743)
\curveto(422.77574765,348.75456397)(422.78574764,348.61956411)(422.77575058,348.4495743)
\curveto(422.76574766,348.28956444)(422.76074767,348.14956458)(422.76075058,348.0295743)
\lineto(422.76075058,346.3945743)
\lineto(422.76075058,346.0645743)
\curveto(422.76074767,345.95456677)(422.77074766,345.85956687)(422.79075058,345.7795743)
\curveto(422.80074763,345.729567)(422.81074762,345.68456704)(422.82075058,345.6445743)
\curveto(422.8307476,345.61456711)(422.85574757,345.59456713)(422.89575058,345.5845743)
\curveto(422.91574751,345.56456716)(422.94074749,345.55456717)(422.97075058,345.5545743)
\curveto(423.01074742,345.55456717)(423.04074739,345.55956717)(423.06075058,345.5695743)
\curveto(423.1307473,345.60956712)(423.19574723,345.64956708)(423.25575058,345.6895743)
\curveto(423.31574711,345.73956699)(423.38074705,345.78956694)(423.45075058,345.8395743)
\curveto(423.58074685,345.9295668)(423.71574671,346.00456672)(423.85575058,346.0645743)
\curveto(423.99574643,346.13456659)(424.15074628,346.19456653)(424.32075058,346.2445743)
\curveto(424.40074603,346.27456645)(424.48074595,346.28956644)(424.56075058,346.2895743)
\curveto(424.64074579,346.29956643)(424.72074571,346.31456641)(424.80075058,346.3345743)
\curveto(424.87074556,346.35456637)(424.94574548,346.36456636)(425.02575058,346.3645743)
\lineto(425.26575058,346.3645743)
\lineto(425.41575058,346.3645743)
\curveto(425.44574498,346.35456637)(425.48074495,346.34956638)(425.52075058,346.3495743)
\curveto(425.56074487,346.35956637)(425.60074483,346.35956637)(425.64075058,346.3495743)
\curveto(425.75074468,346.31956641)(425.85074458,346.29456643)(425.94075058,346.2745743)
\curveto(426.04074439,346.26456646)(426.13574429,346.23956649)(426.22575058,346.1995743)
\curveto(426.68574374,346.00956672)(427.06074337,345.76456696)(427.35075058,345.4645743)
\curveto(427.64074279,345.16456756)(427.88574254,344.78956794)(428.08575058,344.3395743)
\curveto(428.13574229,344.21956851)(428.17574225,344.09456863)(428.20575058,343.9645743)
\curveto(428.24574218,343.83456889)(428.28574214,343.69956903)(428.32575058,343.5595743)
\curveto(428.34574208,343.48956924)(428.35574207,343.41956931)(428.35575058,343.3495743)
\curveto(428.36574206,343.28956944)(428.38074205,343.21956951)(428.40075058,343.1395743)
\curveto(428.42074201,343.08956964)(428.425742,343.03456969)(428.41575058,342.9745743)
\curveto(428.41574201,342.91456981)(428.42074201,342.85456987)(428.43075058,342.7945743)
\lineto(428.43075058,342.6895743)
\moveto(426.21075058,341.2795743)
\curveto(426.24074419,341.37957135)(426.26574416,341.50457122)(426.28575058,341.6545743)
\curveto(426.31574411,341.80457092)(426.3307441,341.95457077)(426.33075058,342.1045743)
\curveto(426.34074409,342.26457046)(426.34074409,342.41957031)(426.33075058,342.5695743)
\curveto(426.3307441,342.72957)(426.31574411,342.86456986)(426.28575058,342.9745743)
\curveto(426.25574417,343.07456965)(426.23574419,343.16956956)(426.22575058,343.2595743)
\curveto(426.21574421,343.34956938)(426.19074424,343.43456929)(426.15075058,343.5145743)
\curveto(426.01074442,343.86456886)(425.81074462,344.15956857)(425.55075058,344.3995743)
\curveto(425.30074513,344.64956808)(424.9307455,344.77456795)(424.44075058,344.7745743)
\curveto(424.40074603,344.77456795)(424.36574606,344.76956796)(424.33575058,344.7595743)
\lineto(424.23075058,344.7595743)
\curveto(424.16074627,344.73956799)(424.09574633,344.71956801)(424.03575058,344.6995743)
\curveto(423.97574645,344.68956804)(423.91574651,344.67456805)(423.85575058,344.6545743)
\curveto(423.56574686,344.5245682)(423.34574708,344.33956839)(423.19575058,344.0995743)
\curveto(423.04574738,343.86956886)(422.92074751,343.60456912)(422.82075058,343.3045743)
\curveto(422.79074764,343.2245695)(422.77074766,343.13956959)(422.76075058,343.0495743)
\curveto(422.76074767,342.96956976)(422.75074768,342.88956984)(422.73075058,342.8095743)
\curveto(422.72074771,342.77956995)(422.71574771,342.72957)(422.71575058,342.6595743)
\curveto(422.70574772,342.61957011)(422.70074773,342.57957015)(422.70075058,342.5395743)
\curveto(422.71074772,342.49957023)(422.71074772,342.45957027)(422.70075058,342.4195743)
\curveto(422.68074775,342.33957039)(422.67574775,342.2295705)(422.68575058,342.0895743)
\curveto(422.69574773,341.94957078)(422.71074772,341.84957088)(422.73075058,341.7895743)
\curveto(422.75074768,341.69957103)(422.76074767,341.61457111)(422.76075058,341.5345743)
\curveto(422.77074766,341.45457127)(422.79074764,341.37457135)(422.82075058,341.2945743)
\curveto(422.91074752,341.01457171)(423.01574741,340.76957196)(423.13575058,340.5595743)
\curveto(423.26574716,340.35957237)(423.44574698,340.18957254)(423.67575058,340.0495743)
\curveto(423.83574659,339.94957278)(424.00074643,339.87957285)(424.17075058,339.8395743)
\curveto(424.19074624,339.83957289)(424.21074622,339.83457289)(424.23075058,339.8245743)
\lineto(424.32075058,339.8245743)
\curveto(424.35074608,339.81457291)(424.40074603,339.80457292)(424.47075058,339.7945743)
\curveto(424.54074589,339.79457293)(424.60074583,339.79957293)(424.65075058,339.8095743)
\curveto(424.75074568,339.8295729)(424.84074559,339.84457288)(424.92075058,339.8545743)
\curveto(425.01074542,339.87457285)(425.09574533,339.89957283)(425.17575058,339.9295743)
\curveto(425.45574497,340.05957267)(425.67074476,340.23957249)(425.82075058,340.4695743)
\curveto(425.98074445,340.69957203)(426.11074432,340.96957176)(426.21075058,341.2795743)
}
}
{
\newrgbcolor{curcolor}{0 0 0}
\pscustom[linestyle=none,fillstyle=solid,fillcolor=curcolor]
{
\newpath
\moveto(430.22067245,346.1545743)
\lineto(431.34567245,346.1545743)
\curveto(431.45567002,346.15456657)(431.55566992,346.14956658)(431.64567245,346.1395743)
\curveto(431.73566974,346.1295666)(431.80066967,346.09456663)(431.84067245,346.0345743)
\curveto(431.89066958,345.97456675)(431.92066955,345.88956684)(431.93067245,345.7795743)
\curveto(431.94066953,345.67956705)(431.94566953,345.57456715)(431.94567245,345.4645743)
\lineto(431.94567245,344.4145743)
\lineto(431.94567245,342.1795743)
\curveto(431.94566953,341.81957091)(431.96066951,341.47957125)(431.99067245,341.1595743)
\curveto(432.02066945,340.83957189)(432.11066936,340.57457215)(432.26067245,340.3645743)
\curveto(432.40066907,340.15457257)(432.62566885,340.00457272)(432.93567245,339.9145743)
\curveto(432.98566849,339.90457282)(433.02566845,339.89957283)(433.05567245,339.8995743)
\curveto(433.09566838,339.89957283)(433.14066833,339.89457283)(433.19067245,339.8845743)
\curveto(433.24066823,339.87457285)(433.29566818,339.86957286)(433.35567245,339.8695743)
\curveto(433.41566806,339.86957286)(433.46066801,339.87457285)(433.49067245,339.8845743)
\curveto(433.54066793,339.90457282)(433.58066789,339.90957282)(433.61067245,339.8995743)
\curveto(433.65066782,339.88957284)(433.69066778,339.89457283)(433.73067245,339.9145743)
\curveto(433.94066753,339.96457276)(434.10566737,340.0295727)(434.22567245,340.1095743)
\curveto(434.40566707,340.21957251)(434.54566693,340.35957237)(434.64567245,340.5295743)
\curveto(434.75566672,340.70957202)(434.83066664,340.90457182)(434.87067245,341.1145743)
\curveto(434.92066655,341.33457139)(434.95066652,341.57457115)(434.96067245,341.8345743)
\curveto(434.9706665,342.10457062)(434.9756665,342.38457034)(434.97567245,342.6745743)
\lineto(434.97567245,344.4895743)
\lineto(434.97567245,345.4645743)
\lineto(434.97567245,345.7345743)
\curveto(434.9756665,345.83456689)(434.99566648,345.91456681)(435.03567245,345.9745743)
\curveto(435.08566639,346.06456666)(435.16066631,346.11456661)(435.26067245,346.1245743)
\curveto(435.36066611,346.14456658)(435.48066599,346.15456657)(435.62067245,346.1545743)
\lineto(436.41567245,346.1545743)
\lineto(436.70067245,346.1545743)
\curveto(436.79066468,346.15456657)(436.86566461,346.13456659)(436.92567245,346.0945743)
\curveto(437.00566447,346.04456668)(437.05066442,345.96956676)(437.06067245,345.8695743)
\curveto(437.0706644,345.76956696)(437.0756644,345.65456707)(437.07567245,345.5245743)
\lineto(437.07567245,344.3845743)
\lineto(437.07567245,340.1695743)
\lineto(437.07567245,339.1045743)
\lineto(437.07567245,338.8045743)
\curveto(437.0756644,338.70457402)(437.05566442,338.6295741)(437.01567245,338.5795743)
\curveto(436.96566451,338.49957423)(436.89066458,338.45457427)(436.79067245,338.4445743)
\curveto(436.69066478,338.43457429)(436.58566489,338.4295743)(436.47567245,338.4295743)
\lineto(435.66567245,338.4295743)
\curveto(435.55566592,338.4295743)(435.45566602,338.43457429)(435.36567245,338.4445743)
\curveto(435.28566619,338.45457427)(435.22066625,338.49457423)(435.17067245,338.5645743)
\curveto(435.15066632,338.59457413)(435.13066634,338.63957409)(435.11067245,338.6995743)
\curveto(435.10066637,338.75957397)(435.08566639,338.81957391)(435.06567245,338.8795743)
\curveto(435.05566642,338.93957379)(435.04066643,338.99457373)(435.02067245,339.0445743)
\curveto(435.00066647,339.09457363)(434.9706665,339.1245736)(434.93067245,339.1345743)
\curveto(434.91066656,339.15457357)(434.88566659,339.15957357)(434.85567245,339.1495743)
\curveto(434.82566665,339.13957359)(434.80066667,339.1295736)(434.78067245,339.1195743)
\curveto(434.71066676,339.07957365)(434.65066682,339.03457369)(434.60067245,338.9845743)
\curveto(434.55066692,338.93457379)(434.49566698,338.88957384)(434.43567245,338.8495743)
\curveto(434.39566708,338.81957391)(434.35566712,338.78457394)(434.31567245,338.7445743)
\curveto(434.28566719,338.71457401)(434.24566723,338.68457404)(434.19567245,338.6545743)
\curveto(433.96566751,338.51457421)(433.69566778,338.40457432)(433.38567245,338.3245743)
\curveto(433.31566816,338.30457442)(433.24566823,338.29457443)(433.17567245,338.2945743)
\curveto(433.10566837,338.28457444)(433.03066844,338.26957446)(432.95067245,338.2495743)
\curveto(432.91066856,338.23957449)(432.86566861,338.23957449)(432.81567245,338.2495743)
\curveto(432.7756687,338.24957448)(432.73566874,338.24457448)(432.69567245,338.2345743)
\curveto(432.66566881,338.2245745)(432.60066887,338.2245745)(432.50067245,338.2345743)
\curveto(432.41066906,338.23457449)(432.35066912,338.23957449)(432.32067245,338.2495743)
\curveto(432.2706692,338.24957448)(432.22066925,338.25457447)(432.17067245,338.2645743)
\lineto(432.02067245,338.2645743)
\curveto(431.90066957,338.29457443)(431.78566969,338.31957441)(431.67567245,338.3395743)
\curveto(431.56566991,338.35957437)(431.45567002,338.38957434)(431.34567245,338.4295743)
\curveto(431.29567018,338.44957428)(431.25067022,338.46457426)(431.21067245,338.4745743)
\curveto(431.18067029,338.49457423)(431.14067033,338.51457421)(431.09067245,338.5345743)
\curveto(430.74067073,338.724574)(430.46067101,338.98957374)(430.25067245,339.3295743)
\curveto(430.12067135,339.53957319)(430.02567145,339.78957294)(429.96567245,340.0795743)
\curveto(429.90567157,340.37957235)(429.86567161,340.69457203)(429.84567245,341.0245743)
\curveto(429.83567164,341.36457136)(429.83067164,341.70957102)(429.83067245,342.0595743)
\curveto(429.84067163,342.41957031)(429.84567163,342.77456995)(429.84567245,343.1245743)
\lineto(429.84567245,345.1645743)
\curveto(429.84567163,345.29456743)(429.84067163,345.44456728)(429.83067245,345.6145743)
\curveto(429.83067164,345.79456693)(429.85567162,345.9245668)(429.90567245,346.0045743)
\curveto(429.93567154,346.05456667)(429.99567148,346.09956663)(430.08567245,346.1395743)
\curveto(430.14567133,346.13956659)(430.19067128,346.14456658)(430.22067245,346.1545743)
}
}
{
\newrgbcolor{curcolor}{0 0 0}
\pscustom[linestyle=none,fillstyle=solid,fillcolor=curcolor]
{
\newpath
\moveto(442.27692245,346.3795743)
\curveto(443.08691729,346.39956633)(443.76191662,346.27956645)(444.30192245,346.0195743)
\curveto(444.85191553,345.75956697)(445.28691509,345.38956734)(445.60692245,344.9095743)
\curveto(445.76691461,344.66956806)(445.88691449,344.39456833)(445.96692245,344.0845743)
\curveto(445.98691439,344.03456869)(446.00191438,343.96956876)(446.01192245,343.8895743)
\curveto(446.03191435,343.80956892)(446.03191435,343.73956899)(446.01192245,343.6795743)
\curveto(445.97191441,343.56956916)(445.90191448,343.50456922)(445.80192245,343.4845743)
\curveto(445.70191468,343.47456925)(445.5819148,343.46956926)(445.44192245,343.4695743)
\lineto(444.66192245,343.4695743)
\lineto(444.37692245,343.4695743)
\curveto(444.28691609,343.46956926)(444.21191617,343.48956924)(444.15192245,343.5295743)
\curveto(444.07191631,343.56956916)(444.01691636,343.6295691)(443.98692245,343.7095743)
\curveto(443.95691642,343.79956893)(443.91691646,343.88956884)(443.86692245,343.9795743)
\curveto(443.80691657,344.08956864)(443.74191664,344.18956854)(443.67192245,344.2795743)
\curveto(443.60191678,344.36956836)(443.52191686,344.44956828)(443.43192245,344.5195743)
\curveto(443.29191709,344.60956812)(443.13691724,344.67956805)(442.96692245,344.7295743)
\curveto(442.90691747,344.74956798)(442.84691753,344.75956797)(442.78692245,344.7595743)
\curveto(442.72691765,344.75956797)(442.67191771,344.76956796)(442.62192245,344.7895743)
\lineto(442.47192245,344.7895743)
\curveto(442.27191811,344.78956794)(442.11191827,344.76956796)(441.99192245,344.7295743)
\curveto(441.70191868,344.63956809)(441.46691891,344.49956823)(441.28692245,344.3095743)
\curveto(441.10691927,344.1295686)(440.96191942,343.90956882)(440.85192245,343.6495743)
\curveto(440.80191958,343.53956919)(440.76191962,343.41956931)(440.73192245,343.2895743)
\curveto(440.71191967,343.16956956)(440.68691969,343.03956969)(440.65692245,342.8995743)
\curveto(440.64691973,342.85956987)(440.64191974,342.81956991)(440.64192245,342.7795743)
\curveto(440.64191974,342.73956999)(440.63691974,342.69957003)(440.62692245,342.6595743)
\curveto(440.60691977,342.55957017)(440.59691978,342.41957031)(440.59692245,342.2395743)
\curveto(440.60691977,342.05957067)(440.62191976,341.91957081)(440.64192245,341.8195743)
\curveto(440.64191974,341.73957099)(440.64691973,341.68457104)(440.65692245,341.6545743)
\curveto(440.6769197,341.58457114)(440.68691969,341.51457121)(440.68692245,341.4445743)
\curveto(440.69691968,341.37457135)(440.71191967,341.30457142)(440.73192245,341.2345743)
\curveto(440.81191957,341.00457172)(440.90691947,340.79457193)(441.01692245,340.6045743)
\curveto(441.12691925,340.41457231)(441.26691911,340.25457247)(441.43692245,340.1245743)
\curveto(441.4769189,340.09457263)(441.53691884,340.05957267)(441.61692245,340.0195743)
\curveto(441.72691865,339.94957278)(441.83691854,339.90457282)(441.94692245,339.8845743)
\curveto(442.06691831,339.86457286)(442.21191817,339.84457288)(442.38192245,339.8245743)
\lineto(442.47192245,339.8245743)
\curveto(442.51191787,339.8245729)(442.54191784,339.8295729)(442.56192245,339.8395743)
\lineto(442.69692245,339.8395743)
\curveto(442.76691761,339.85957287)(442.83191755,339.87457285)(442.89192245,339.8845743)
\curveto(442.96191742,339.90457282)(443.02691735,339.9245728)(443.08692245,339.9445743)
\curveto(443.38691699,340.07457265)(443.61691676,340.26457246)(443.77692245,340.5145743)
\curveto(443.81691656,340.56457216)(443.85191653,340.61957211)(443.88192245,340.6795743)
\curveto(443.91191647,340.74957198)(443.93691644,340.80957192)(443.95692245,340.8595743)
\curveto(443.99691638,340.96957176)(444.03191635,341.06457166)(444.06192245,341.1445743)
\curveto(444.09191629,341.23457149)(444.16191622,341.30457142)(444.27192245,341.3545743)
\curveto(444.36191602,341.39457133)(444.50691587,341.40957132)(444.70692245,341.3995743)
\lineto(445.20192245,341.3995743)
\lineto(445.41192245,341.3995743)
\curveto(445.49191489,341.40957132)(445.55691482,341.40457132)(445.60692245,341.3845743)
\lineto(445.72692245,341.3845743)
\lineto(445.84692245,341.3545743)
\curveto(445.88691449,341.35457137)(445.91691446,341.34457138)(445.93692245,341.3245743)
\curveto(445.98691439,341.28457144)(446.01691436,341.2245715)(446.02692245,341.1445743)
\curveto(446.04691433,341.07457165)(446.04691433,340.99957173)(446.02692245,340.9195743)
\curveto(445.93691444,340.58957214)(445.82691455,340.29457243)(445.69692245,340.0345743)
\curveto(445.28691509,339.26457346)(444.63191575,338.729574)(443.73192245,338.4295743)
\curveto(443.63191675,338.39957433)(443.52691685,338.37957435)(443.41692245,338.3695743)
\curveto(443.30691707,338.34957438)(443.19691718,338.3245744)(443.08692245,338.2945743)
\curveto(443.02691735,338.28457444)(442.96691741,338.27957445)(442.90692245,338.2795743)
\curveto(442.84691753,338.27957445)(442.78691759,338.27457445)(442.72692245,338.2645743)
\lineto(442.56192245,338.2645743)
\curveto(442.51191787,338.24457448)(442.43691794,338.23957449)(442.33692245,338.2495743)
\curveto(442.23691814,338.24957448)(442.16191822,338.25457447)(442.11192245,338.2645743)
\curveto(442.03191835,338.28457444)(441.95691842,338.29457443)(441.88692245,338.2945743)
\curveto(441.82691855,338.28457444)(441.76191862,338.28957444)(441.69192245,338.3095743)
\lineto(441.54192245,338.3395743)
\curveto(441.49191889,338.33957439)(441.44191894,338.34457438)(441.39192245,338.3545743)
\curveto(441.2819191,338.38457434)(441.1769192,338.41457431)(441.07692245,338.4445743)
\curveto(440.9769194,338.47457425)(440.8819195,338.50957422)(440.79192245,338.5495743)
\curveto(440.32192006,338.74957398)(439.92692045,339.00457372)(439.60692245,339.3145743)
\curveto(439.28692109,339.63457309)(439.02692135,340.0295727)(438.82692245,340.4995743)
\curveto(438.7769216,340.58957214)(438.73692164,340.68457204)(438.70692245,340.7845743)
\lineto(438.61692245,341.1145743)
\curveto(438.60692177,341.15457157)(438.60192178,341.18957154)(438.60192245,341.2195743)
\curveto(438.60192178,341.25957147)(438.59192179,341.30457142)(438.57192245,341.3545743)
\curveto(438.55192183,341.4245713)(438.54192184,341.49457123)(438.54192245,341.5645743)
\curveto(438.54192184,341.64457108)(438.53192185,341.71957101)(438.51192245,341.7895743)
\lineto(438.51192245,342.0445743)
\curveto(438.49192189,342.09457063)(438.4819219,342.14957058)(438.48192245,342.2095743)
\curveto(438.4819219,342.27957045)(438.49192189,342.33957039)(438.51192245,342.3895743)
\curveto(438.52192186,342.43957029)(438.52192186,342.48457024)(438.51192245,342.5245743)
\curveto(438.50192188,342.56457016)(438.50192188,342.60457012)(438.51192245,342.6445743)
\curveto(438.53192185,342.71457001)(438.53692184,342.77956995)(438.52692245,342.8395743)
\curveto(438.52692185,342.89956983)(438.53692184,342.95956977)(438.55692245,343.0195743)
\curveto(438.60692177,343.19956953)(438.64692173,343.36956936)(438.67692245,343.5295743)
\curveto(438.70692167,343.69956903)(438.75192163,343.86456886)(438.81192245,344.0245743)
\curveto(439.03192135,344.53456819)(439.30692107,344.95956777)(439.63692245,345.2995743)
\curveto(439.9769204,345.63956709)(440.40691997,345.91456681)(440.92692245,346.1245743)
\curveto(441.06691931,346.18456654)(441.21191917,346.2245665)(441.36192245,346.2445743)
\curveto(441.51191887,346.27456645)(441.66691871,346.30956642)(441.82692245,346.3495743)
\curveto(441.90691847,346.35956637)(441.9819184,346.36456636)(442.05192245,346.3645743)
\curveto(442.12191826,346.36456636)(442.19691818,346.36956636)(442.27692245,346.3795743)
}
}
{
\newrgbcolor{curcolor}{0 0 0}
\pscustom[linestyle=none,fillstyle=solid,fillcolor=curcolor]
{
\newpath
\moveto(449.4202037,349.0195743)
\curveto(449.49020075,348.93956379)(449.52520072,348.81956391)(449.5252037,348.6595743)
\lineto(449.5252037,348.1945743)
\lineto(449.5252037,347.7895743)
\curveto(449.52520072,347.64956508)(449.49020075,347.55456517)(449.4202037,347.5045743)
\curveto(449.36020088,347.45456527)(449.28020096,347.4245653)(449.1802037,347.4145743)
\curveto(449.09020115,347.40456532)(448.99020125,347.39956533)(448.8802037,347.3995743)
\lineto(448.0402037,347.3995743)
\curveto(447.93020231,347.39956533)(447.83020241,347.40456532)(447.7402037,347.4145743)
\curveto(447.66020258,347.4245653)(447.59020265,347.45456527)(447.5302037,347.5045743)
\curveto(447.49020275,347.53456519)(447.46020278,347.58956514)(447.4402037,347.6695743)
\curveto(447.43020281,347.75956497)(447.42020282,347.85456487)(447.4102037,347.9545743)
\lineto(447.4102037,348.2845743)
\curveto(447.42020282,348.39456433)(447.42520282,348.48956424)(447.4252037,348.5695743)
\lineto(447.4252037,348.7795743)
\curveto(447.43520281,348.84956388)(447.45520279,348.90956382)(447.4852037,348.9595743)
\curveto(447.50520274,348.99956373)(447.53020271,349.0295637)(447.5602037,349.0495743)
\lineto(447.6802037,349.1095743)
\curveto(447.70020254,349.10956362)(447.72520252,349.10956362)(447.7552037,349.1095743)
\curveto(447.78520246,349.11956361)(447.81020243,349.1245636)(447.8302037,349.1245743)
\lineto(448.9252037,349.1245743)
\curveto(449.02520122,349.1245636)(449.12020112,349.11956361)(449.2102037,349.1095743)
\curveto(449.30020094,349.09956363)(449.37020087,349.06956366)(449.4202037,349.0195743)
\moveto(449.5252037,339.2545743)
\curveto(449.52520072,339.05457367)(449.52020072,338.88457384)(449.5102037,338.7445743)
\curveto(449.50020074,338.60457412)(449.41020083,338.50957422)(449.2402037,338.4595743)
\curveto(449.18020106,338.43957429)(449.11520113,338.4295743)(449.0452037,338.4295743)
\curveto(448.97520127,338.43957429)(448.90020134,338.44457428)(448.8202037,338.4445743)
\lineto(447.9802037,338.4445743)
\curveto(447.89020235,338.44457428)(447.80020244,338.44957428)(447.7102037,338.4595743)
\curveto(447.63020261,338.46957426)(447.57020267,338.49957423)(447.5302037,338.5495743)
\curveto(447.47020277,338.61957411)(447.43520281,338.70457402)(447.4252037,338.8045743)
\lineto(447.4252037,339.1495743)
\lineto(447.4252037,345.4795743)
\lineto(447.4252037,345.7795743)
\curveto(447.42520282,345.87956685)(447.4452028,345.95956677)(447.4852037,346.0195743)
\curveto(447.5452027,346.08956664)(447.63020261,346.13456659)(447.7402037,346.1545743)
\curveto(447.76020248,346.16456656)(447.78520246,346.16456656)(447.8152037,346.1545743)
\curveto(447.85520239,346.15456657)(447.88520236,346.15956657)(447.9052037,346.1695743)
\lineto(448.6552037,346.1695743)
\lineto(448.8502037,346.1695743)
\curveto(448.93020131,346.17956655)(448.99520125,346.17956655)(449.0452037,346.1695743)
\lineto(449.1652037,346.1695743)
\curveto(449.22520102,346.14956658)(449.28020096,346.13456659)(449.3302037,346.1245743)
\curveto(449.38020086,346.11456661)(449.42020082,346.08456664)(449.4502037,346.0345743)
\curveto(449.49020075,345.98456674)(449.51020073,345.91456681)(449.5102037,345.8245743)
\curveto(449.52020072,345.73456699)(449.52520072,345.63956709)(449.5252037,345.5395743)
\lineto(449.5252037,339.2545743)
}
}
{
\newrgbcolor{curcolor}{0 0 0}
\pscustom[linestyle=none,fillstyle=solid,fillcolor=curcolor]
{
\newpath
\moveto(458.9573912,342.6145743)
\curveto(458.93738267,342.66457006)(458.93238268,342.71957001)(458.9423912,342.7795743)
\curveto(458.95238266,342.83956989)(458.94738266,342.89456983)(458.9273912,342.9445743)
\curveto(458.91738269,342.98456974)(458.9123827,343.0245697)(458.9123912,343.0645743)
\curveto(458.9123827,343.10456962)(458.9073827,343.14456958)(458.8973912,343.1845743)
\lineto(458.8373912,343.4545743)
\curveto(458.81738279,343.54456918)(458.79238282,343.6295691)(458.7623912,343.7095743)
\curveto(458.7123829,343.84956888)(458.66738294,343.97956875)(458.6273912,344.0995743)
\curveto(458.58738302,344.2295685)(458.53238308,344.34956838)(458.4623912,344.4595743)
\curveto(458.39238322,344.56956816)(458.32238329,344.67456805)(458.2523912,344.7745743)
\curveto(458.19238342,344.87456785)(458.12238349,344.97456775)(458.0423912,345.0745743)
\curveto(457.96238365,345.18456754)(457.86238375,345.28456744)(457.7423912,345.3745743)
\curveto(457.63238398,345.47456725)(457.52238409,345.56456716)(457.4123912,345.6445743)
\curveto(457.08238453,345.87456685)(456.70238491,346.05456667)(456.2723912,346.1845743)
\curveto(455.85238576,346.31456641)(455.35238626,346.37456635)(454.7723912,346.3645743)
\curveto(454.70238691,346.35456637)(454.63238698,346.34956638)(454.5623912,346.3495743)
\curveto(454.49238712,346.34956638)(454.41738719,346.34456638)(454.3373912,346.3345743)
\curveto(454.18738742,346.29456643)(454.04238757,346.26456646)(453.9023912,346.2445743)
\curveto(453.76238785,346.2245665)(453.62738798,346.18956654)(453.4973912,346.1395743)
\curveto(453.38738822,346.08956664)(453.27738833,346.04456668)(453.1673912,346.0045743)
\curveto(453.05738855,345.96456676)(452.95238866,345.91956681)(452.8523912,345.8695743)
\curveto(452.49238912,345.63956709)(452.18738942,345.38456734)(451.9373912,345.1045743)
\curveto(451.68738992,344.83456789)(451.47239014,344.49456823)(451.2923912,344.0845743)
\curveto(451.24239037,343.96456876)(451.20239041,343.83956889)(451.1723912,343.7095743)
\curveto(451.14239047,343.58956914)(451.1073905,343.46456926)(451.0673912,343.3345743)
\curveto(451.04739056,343.28456944)(451.03739057,343.23456949)(451.0373912,343.1845743)
\curveto(451.03739057,343.14456958)(451.03239058,343.09956963)(451.0223912,343.0495743)
\curveto(451.00239061,342.99956973)(450.99239062,342.94456978)(450.9923912,342.8845743)
\curveto(451.00239061,342.83456989)(451.00239061,342.78456994)(450.9923912,342.7345743)
\lineto(450.9923912,342.6295743)
\curveto(450.97239064,342.56957016)(450.95739065,342.48457024)(450.9473912,342.3745743)
\curveto(450.94739066,342.26457046)(450.95739065,342.17957055)(450.9773912,342.1195743)
\lineto(450.9773912,341.9845743)
\curveto(450.97739063,341.94457078)(450.98239063,341.89957083)(450.9923912,341.8495743)
\curveto(451.0123906,341.76957096)(451.02239059,341.68457104)(451.0223912,341.5945743)
\curveto(451.02239059,341.51457121)(451.03239058,341.43457129)(451.0523912,341.3545743)
\curveto(451.07239054,341.30457142)(451.08239053,341.25957147)(451.0823912,341.2195743)
\curveto(451.08239053,341.17957155)(451.09239052,341.13457159)(451.1123912,341.0845743)
\curveto(451.14239047,340.97457175)(451.16739044,340.86957186)(451.1873912,340.7695743)
\curveto(451.21739039,340.66957206)(451.25739035,340.57457215)(451.3073912,340.4845743)
\curveto(451.47739013,340.09457263)(451.68738992,339.75957297)(451.9373912,339.4795743)
\curveto(452.18738942,339.19957353)(452.48738912,338.95457377)(452.8373912,338.7445743)
\curveto(452.94738866,338.68457404)(453.05238856,338.63457409)(453.1523912,338.5945743)
\curveto(453.26238835,338.55457417)(453.37738823,338.51457421)(453.4973912,338.4745743)
\curveto(453.58738802,338.43457429)(453.68238793,338.40457432)(453.7823912,338.3845743)
\curveto(453.88238773,338.36457436)(453.98238763,338.33957439)(454.0823912,338.3095743)
\curveto(454.13238748,338.29957443)(454.17238744,338.29457443)(454.2023912,338.2945743)
\curveto(454.24238737,338.29457443)(454.28238733,338.28957444)(454.3223912,338.2795743)
\curveto(454.37238724,338.25957447)(454.42238719,338.25457447)(454.4723912,338.2645743)
\curveto(454.53238708,338.26457446)(454.58738702,338.25957447)(454.6373912,338.2495743)
\lineto(454.7873912,338.2495743)
\curveto(454.84738676,338.2295745)(454.93238668,338.2245745)(455.0423912,338.2345743)
\curveto(455.15238646,338.23457449)(455.23238638,338.23957449)(455.2823912,338.2495743)
\curveto(455.3123863,338.24957448)(455.34238627,338.25457447)(455.3723912,338.2645743)
\lineto(455.4773912,338.2645743)
\curveto(455.52738608,338.27457445)(455.58238603,338.27957445)(455.6423912,338.2795743)
\curveto(455.70238591,338.27957445)(455.75738585,338.28957444)(455.8073912,338.3095743)
\curveto(455.93738567,338.33957439)(456.06238555,338.36957436)(456.1823912,338.3995743)
\curveto(456.3123853,338.41957431)(456.43738517,338.45457427)(456.5573912,338.5045743)
\curveto(457.03738457,338.70457402)(457.44738416,338.95457377)(457.7873912,339.2545743)
\curveto(458.12738348,339.55457317)(458.40238321,339.94457278)(458.6123912,340.4245743)
\curveto(458.66238295,340.5245722)(458.70238291,340.6295721)(458.7323912,340.7395743)
\curveto(458.76238285,340.85957187)(458.79738281,340.97457175)(458.8373912,341.0845743)
\curveto(458.84738276,341.15457157)(458.85738275,341.21957151)(458.8673912,341.2795743)
\curveto(458.87738273,341.33957139)(458.89238272,341.40457132)(458.9123912,341.4745743)
\curveto(458.93238268,341.55457117)(458.93738267,341.63457109)(458.9273912,341.7145743)
\curveto(458.92738268,341.79457093)(458.93738267,341.87457085)(458.9573912,341.9545743)
\lineto(458.9573912,342.1045743)
\curveto(458.97738263,342.16457056)(458.98738262,342.24957048)(458.9873912,342.3595743)
\curveto(458.98738262,342.46957026)(458.97738263,342.55457017)(458.9573912,342.6145743)
\moveto(456.8573912,342.0745743)
\curveto(456.84738476,342.0245707)(456.84238477,341.97457075)(456.8423912,341.9245743)
\lineto(456.8423912,341.7895743)
\curveto(456.83238478,341.74957098)(456.82738478,341.70957102)(456.8273912,341.6695743)
\curveto(456.82738478,341.63957109)(456.82238479,341.60457112)(456.8123912,341.5645743)
\curveto(456.78238483,341.45457127)(456.75738485,341.34957138)(456.7373912,341.2495743)
\curveto(456.71738489,341.14957158)(456.68738492,341.04957168)(456.6473912,340.9495743)
\curveto(456.53738507,340.69957203)(456.40238521,340.48957224)(456.2423912,340.3195743)
\curveto(456.08238553,340.14957258)(455.87238574,340.01457271)(455.6123912,339.9145743)
\curveto(455.54238607,339.88457284)(455.46738614,339.86457286)(455.3873912,339.8545743)
\curveto(455.3073863,339.84457288)(455.22738638,339.8295729)(455.1473912,339.8095743)
\lineto(455.0273912,339.8095743)
\curveto(454.98738662,339.79957293)(454.94238667,339.79457293)(454.8923912,339.7945743)
\lineto(454.7723912,339.8245743)
\curveto(454.73238688,339.83457289)(454.69738691,339.83457289)(454.6673912,339.8245743)
\curveto(454.63738697,339.8245729)(454.60238701,339.8295729)(454.5623912,339.8395743)
\curveto(454.47238714,339.85957287)(454.38238723,339.88457284)(454.2923912,339.9145743)
\curveto(454.2123874,339.94457278)(454.13738747,339.98457274)(454.0673912,340.0345743)
\curveto(453.81738779,340.18457254)(453.63238798,340.34957238)(453.5123912,340.5295743)
\curveto(453.40238821,340.71957201)(453.29738831,340.96457176)(453.1973912,341.2645743)
\curveto(453.17738843,341.34457138)(453.16238845,341.41957131)(453.1523912,341.4895743)
\curveto(453.14238847,341.56957116)(453.12738848,341.64957108)(453.1073912,341.7295743)
\lineto(453.1073912,341.8645743)
\curveto(453.08738852,341.93457079)(453.07238854,342.03957069)(453.0623912,342.1795743)
\curveto(453.06238855,342.31957041)(453.07238854,342.4245703)(453.0923912,342.4945743)
\lineto(453.0923912,342.6445743)
\curveto(453.09238852,342.69457003)(453.09738851,342.74456998)(453.1073912,342.7945743)
\curveto(453.12738848,342.90456982)(453.14238847,343.01456971)(453.1523912,343.1245743)
\curveto(453.17238844,343.23456949)(453.19738841,343.33956939)(453.2273912,343.4395743)
\curveto(453.31738829,343.70956902)(453.43738817,343.94456878)(453.5873912,344.1445743)
\curveto(453.74738786,344.35456837)(453.95238766,344.51456821)(454.2023912,344.6245743)
\curveto(454.25238736,344.65456807)(454.3073873,344.67456805)(454.3673912,344.6845743)
\lineto(454.5773912,344.7445743)
\curveto(454.607387,344.75456797)(454.64238697,344.75456797)(454.6823912,344.7445743)
\curveto(454.72238689,344.74456798)(454.75738685,344.75456797)(454.7873912,344.7745743)
\lineto(455.0573912,344.7745743)
\curveto(455.14738646,344.78456794)(455.23238638,344.77956795)(455.3123912,344.7595743)
\curveto(455.38238623,344.73956799)(455.44738616,344.71956801)(455.5073912,344.6995743)
\curveto(455.56738604,344.68956804)(455.62738598,344.67456805)(455.6873912,344.6545743)
\curveto(455.93738567,344.54456818)(456.13738547,344.39456833)(456.2873912,344.2045743)
\curveto(456.43738517,344.0245687)(456.56738504,343.80456892)(456.6773912,343.5445743)
\curveto(456.7073849,343.46456926)(456.72738488,343.37956935)(456.7373912,343.2895743)
\lineto(456.7973912,343.0495743)
\curveto(456.8073848,343.0295697)(456.8123848,342.99956973)(456.8123912,342.9595743)
\curveto(456.82238479,342.90956982)(456.82738478,342.85456987)(456.8273912,342.7945743)
\curveto(456.82738478,342.73456999)(456.83738477,342.67957005)(456.8573912,342.6295743)
\lineto(456.8573912,342.5095743)
\curveto(456.86738474,342.45957027)(456.87238474,342.38457034)(456.8723912,342.2845743)
\curveto(456.87238474,342.19457053)(456.86738474,342.1245706)(456.8573912,342.0745743)
\moveto(455.6273912,349.2445743)
\lineto(456.6923912,349.2445743)
\curveto(456.77238484,349.24456348)(456.86738474,349.24456348)(456.9773912,349.2445743)
\curveto(457.08738452,349.24456348)(457.16738444,349.2295635)(457.2173912,349.1995743)
\curveto(457.23738437,349.18956354)(457.24738436,349.17456355)(457.2473912,349.1545743)
\curveto(457.25738435,349.14456358)(457.27238434,349.13456359)(457.2923912,349.1245743)
\curveto(457.30238431,349.00456372)(457.25238436,348.89956383)(457.1423912,348.8095743)
\curveto(457.04238457,348.71956401)(456.95738465,348.63956409)(456.8873912,348.5695743)
\curveto(456.8073848,348.49956423)(456.72738488,348.4245643)(456.6473912,348.3445743)
\curveto(456.57738503,348.27456445)(456.50238511,348.20956452)(456.4223912,348.1495743)
\curveto(456.38238523,348.11956461)(456.34738526,348.08456464)(456.3173912,348.0445743)
\curveto(456.29738531,348.01456471)(456.26738534,347.98956474)(456.2273912,347.9695743)
\curveto(456.2073854,347.93956479)(456.18238543,347.91456481)(456.1523912,347.8945743)
\lineto(456.0023912,347.7445743)
\lineto(455.8523912,347.6245743)
\lineto(455.8073912,347.5795743)
\curveto(455.8073858,347.56956516)(455.79738581,347.55456517)(455.7773912,347.5345743)
\curveto(455.69738591,347.47456525)(455.61738599,347.40956532)(455.5373912,347.3395743)
\curveto(455.46738614,347.26956546)(455.37738623,347.21456551)(455.2673912,347.1745743)
\curveto(455.22738638,347.16456556)(455.18738642,347.15956557)(455.1473912,347.1595743)
\curveto(455.11738649,347.15956557)(455.07738653,347.15456557)(455.0273912,347.1445743)
\curveto(454.99738661,347.13456559)(454.95738665,347.1295656)(454.9073912,347.1295743)
\curveto(454.85738675,347.13956559)(454.8123868,347.14456558)(454.7723912,347.1445743)
\lineto(454.4273912,347.1445743)
\curveto(454.3073873,347.14456558)(454.21738739,347.16956556)(454.1573912,347.2195743)
\curveto(454.09738751,347.25956547)(454.08238753,347.3295654)(454.1123912,347.4295743)
\curveto(454.13238748,347.50956522)(454.16738744,347.57956515)(454.2173912,347.6395743)
\curveto(454.26738734,347.70956502)(454.3123873,347.77956495)(454.3523912,347.8495743)
\curveto(454.45238716,347.98956474)(454.54738706,348.1245646)(454.6373912,348.2545743)
\curveto(454.72738688,348.38456434)(454.81738679,348.51956421)(454.9073912,348.6595743)
\curveto(454.95738665,348.73956399)(455.0073866,348.8245639)(455.0573912,348.9145743)
\curveto(455.11738649,349.00456372)(455.18238643,349.07456365)(455.2523912,349.1245743)
\curveto(455.29238632,349.15456357)(455.36238625,349.18956354)(455.4623912,349.2295743)
\curveto(455.48238613,349.23956349)(455.5073861,349.23956349)(455.5373912,349.2295743)
\curveto(455.57738603,349.2295635)(455.607386,349.23456349)(455.6273912,349.2445743)
}
}
{
\newrgbcolor{curcolor}{0 0 0}
\pscustom[linestyle=none,fillstyle=solid,fillcolor=curcolor]
{
\newpath
\moveto(464.78231308,346.3645743)
\curveto(465.38230727,346.38456634)(465.88230677,346.29956643)(466.28231308,346.1095743)
\curveto(466.68230597,345.91956681)(466.99730566,345.63956709)(467.22731308,345.2695743)
\curveto(467.29730536,345.15956757)(467.3523053,345.03956769)(467.39231308,344.9095743)
\curveto(467.43230522,344.78956794)(467.47230518,344.66456806)(467.51231308,344.5345743)
\curveto(467.53230512,344.45456827)(467.54230511,344.37956835)(467.54231308,344.3095743)
\curveto(467.5523051,344.23956849)(467.56730509,344.16956856)(467.58731308,344.0995743)
\curveto(467.58730507,344.03956869)(467.59230506,343.99956873)(467.60231308,343.9795743)
\curveto(467.62230503,343.83956889)(467.63230502,343.69456903)(467.63231308,343.5445743)
\lineto(467.63231308,343.1095743)
\lineto(467.63231308,341.7745743)
\lineto(467.63231308,339.3445743)
\curveto(467.63230502,339.15457357)(467.62730503,338.96957376)(467.61731308,338.7895743)
\curveto(467.61730504,338.61957411)(467.54730511,338.50957422)(467.40731308,338.4595743)
\curveto(467.34730531,338.43957429)(467.27730538,338.4295743)(467.19731308,338.4295743)
\lineto(466.95731308,338.4295743)
\lineto(466.14731308,338.4295743)
\curveto(466.02730663,338.4295743)(465.91730674,338.43457429)(465.81731308,338.4445743)
\curveto(465.72730693,338.46457426)(465.657307,338.50957422)(465.60731308,338.5795743)
\curveto(465.56730709,338.63957409)(465.54230711,338.71457401)(465.53231308,338.8045743)
\lineto(465.53231308,339.1195743)
\lineto(465.53231308,340.1695743)
\lineto(465.53231308,342.4045743)
\curveto(465.53230712,342.77456995)(465.51730714,343.11456961)(465.48731308,343.4245743)
\curveto(465.4573072,343.74456898)(465.36730729,344.01456871)(465.21731308,344.2345743)
\curveto(465.07730758,344.43456829)(464.87230778,344.57456815)(464.60231308,344.6545743)
\curveto(464.5523081,344.67456805)(464.49730816,344.68456804)(464.43731308,344.6845743)
\curveto(464.38730827,344.68456804)(464.33230832,344.69456803)(464.27231308,344.7145743)
\curveto(464.22230843,344.724568)(464.1573085,344.724568)(464.07731308,344.7145743)
\curveto(464.00730865,344.71456801)(463.9523087,344.70956802)(463.91231308,344.6995743)
\curveto(463.87230878,344.68956804)(463.83730882,344.68456804)(463.80731308,344.6845743)
\curveto(463.77730888,344.68456804)(463.74730891,344.67956805)(463.71731308,344.6695743)
\curveto(463.48730917,344.60956812)(463.30230935,344.5295682)(463.16231308,344.4295743)
\curveto(462.84230981,344.19956853)(462.65231,343.86456886)(462.59231308,343.4245743)
\curveto(462.53231012,342.98456974)(462.50231015,342.48957024)(462.50231308,341.9395743)
\lineto(462.50231308,340.0645743)
\lineto(462.50231308,339.1495743)
\lineto(462.50231308,338.8795743)
\curveto(462.50231015,338.78957394)(462.48731017,338.71457401)(462.45731308,338.6545743)
\curveto(462.40731025,338.54457418)(462.32731033,338.47957425)(462.21731308,338.4595743)
\curveto(462.10731055,338.43957429)(461.97231068,338.4295743)(461.81231308,338.4295743)
\lineto(461.06231308,338.4295743)
\curveto(460.9523117,338.4295743)(460.84231181,338.43457429)(460.73231308,338.4445743)
\curveto(460.62231203,338.45457427)(460.54231211,338.48957424)(460.49231308,338.5495743)
\curveto(460.42231223,338.63957409)(460.38731227,338.76957396)(460.38731308,338.9395743)
\curveto(460.39731226,339.10957362)(460.40231225,339.26957346)(460.40231308,339.4195743)
\lineto(460.40231308,341.4595743)
\lineto(460.40231308,344.7595743)
\lineto(460.40231308,345.5245743)
\lineto(460.40231308,345.8245743)
\curveto(460.41231224,345.91456681)(460.44231221,345.98956674)(460.49231308,346.0495743)
\curveto(460.51231214,346.07956665)(460.54231211,346.09956663)(460.58231308,346.1095743)
\curveto(460.63231202,346.1295666)(460.68231197,346.14456658)(460.73231308,346.1545743)
\lineto(460.80731308,346.1545743)
\curveto(460.8573118,346.16456656)(460.90731175,346.16956656)(460.95731308,346.1695743)
\lineto(461.12231308,346.1695743)
\lineto(461.75231308,346.1695743)
\curveto(461.83231082,346.16956656)(461.90731075,346.16456656)(461.97731308,346.1545743)
\curveto(462.0573106,346.15456657)(462.12731053,346.14456658)(462.18731308,346.1245743)
\curveto(462.2573104,346.09456663)(462.30231035,346.04956668)(462.32231308,345.9895743)
\curveto(462.3523103,345.9295668)(462.37731028,345.85956687)(462.39731308,345.7795743)
\curveto(462.40731025,345.73956699)(462.40731025,345.70456702)(462.39731308,345.6745743)
\curveto(462.39731026,345.64456708)(462.40731025,345.61456711)(462.42731308,345.5845743)
\curveto(462.44731021,345.53456719)(462.46231019,345.50456722)(462.47231308,345.4945743)
\curveto(462.49231016,345.48456724)(462.51731014,345.46956726)(462.54731308,345.4495743)
\curveto(462.65731,345.43956729)(462.74730991,345.47456725)(462.81731308,345.5545743)
\curveto(462.88730977,345.64456708)(462.96230969,345.71456701)(463.04231308,345.7645743)
\curveto(463.31230934,345.96456676)(463.61230904,346.1245666)(463.94231308,346.2445743)
\curveto(464.03230862,346.27456645)(464.12230853,346.29456643)(464.21231308,346.3045743)
\curveto(464.31230834,346.31456641)(464.41730824,346.3295664)(464.52731308,346.3495743)
\curveto(464.5573081,346.35956637)(464.60230805,346.35956637)(464.66231308,346.3495743)
\curveto(464.72230793,346.34956638)(464.76230789,346.35456637)(464.78231308,346.3645743)
}
}
{
\newrgbcolor{curcolor}{0 0 0}
\pscustom[linestyle=none,fillstyle=solid,fillcolor=curcolor]
{
}
}
{
\newrgbcolor{curcolor}{0 0 0}
\pscustom[linestyle=none,fillstyle=solid,fillcolor=curcolor]
{
\newpath
\moveto(481.01371933,339.2845743)
\lineto(481.01371933,338.8645743)
\curveto(481.01371096,338.73457399)(480.98371099,338.6295741)(480.92371933,338.5495743)
\curveto(480.8737111,338.49957423)(480.80871116,338.46457426)(480.72871933,338.4445743)
\curveto(480.64871132,338.43457429)(480.55871141,338.4295743)(480.45871933,338.4295743)
\lineto(479.63371933,338.4295743)
\lineto(479.34871933,338.4295743)
\curveto(479.2687127,338.43957429)(479.20371277,338.46457426)(479.15371933,338.5045743)
\curveto(479.08371289,338.55457417)(479.04371293,338.61957411)(479.03371933,338.6995743)
\curveto(479.02371295,338.77957395)(479.00371297,338.85957387)(478.97371933,338.9395743)
\curveto(478.95371302,338.95957377)(478.93371304,338.97457375)(478.91371933,338.9845743)
\curveto(478.90371307,339.00457372)(478.88871308,339.0245737)(478.86871933,339.0445743)
\curveto(478.75871321,339.04457368)(478.67871329,339.01957371)(478.62871933,338.9695743)
\lineto(478.47871933,338.8195743)
\curveto(478.40871356,338.76957396)(478.34371363,338.724574)(478.28371933,338.6845743)
\curveto(478.22371375,338.65457407)(478.15871381,338.61457411)(478.08871933,338.5645743)
\curveto(478.04871392,338.54457418)(478.00371397,338.5245742)(477.95371933,338.5045743)
\curveto(477.91371406,338.48457424)(477.8687141,338.46457426)(477.81871933,338.4445743)
\curveto(477.67871429,338.39457433)(477.52871444,338.34957438)(477.36871933,338.3095743)
\curveto(477.31871465,338.28957444)(477.2737147,338.27957445)(477.23371933,338.2795743)
\curveto(477.19371478,338.27957445)(477.15371482,338.27457445)(477.11371933,338.2645743)
\lineto(476.97871933,338.2645743)
\curveto(476.94871502,338.25457447)(476.90871506,338.24957448)(476.85871933,338.2495743)
\lineto(476.72371933,338.2495743)
\curveto(476.66371531,338.2295745)(476.5737154,338.2245745)(476.45371933,338.2345743)
\curveto(476.33371564,338.23457449)(476.24871572,338.24457448)(476.19871933,338.2645743)
\curveto(476.12871584,338.28457444)(476.06371591,338.29457443)(476.00371933,338.2945743)
\curveto(475.95371602,338.28457444)(475.89871607,338.28957444)(475.83871933,338.3095743)
\lineto(475.47871933,338.4295743)
\curveto(475.3687166,338.45957427)(475.25871671,338.49957423)(475.14871933,338.5495743)
\curveto(474.79871717,338.69957403)(474.48371749,338.9295738)(474.20371933,339.2395743)
\curveto(473.93371804,339.55957317)(473.71871825,339.89457283)(473.55871933,340.2445743)
\curveto(473.50871846,340.35457237)(473.4687185,340.45957227)(473.43871933,340.5595743)
\curveto(473.40871856,340.66957206)(473.3737186,340.77957195)(473.33371933,340.8895743)
\curveto(473.32371865,340.9295718)(473.31871865,340.96457176)(473.31871933,340.9945743)
\curveto(473.31871865,341.03457169)(473.30871866,341.07957165)(473.28871933,341.1295743)
\curveto(473.2687187,341.20957152)(473.24871872,341.29457143)(473.22871933,341.3845743)
\curveto(473.21871875,341.48457124)(473.20371877,341.58457114)(473.18371933,341.6845743)
\curveto(473.1737188,341.71457101)(473.1687188,341.74957098)(473.16871933,341.7895743)
\curveto(473.17871879,341.8295709)(473.17871879,341.86457086)(473.16871933,341.8945743)
\lineto(473.16871933,342.0295743)
\curveto(473.1687188,342.07957065)(473.16371881,342.1295706)(473.15371933,342.1795743)
\curveto(473.14371883,342.2295705)(473.13871883,342.28457044)(473.13871933,342.3445743)
\curveto(473.13871883,342.41457031)(473.14371883,342.46957026)(473.15371933,342.5095743)
\curveto(473.16371881,342.55957017)(473.1687188,342.60457012)(473.16871933,342.6445743)
\lineto(473.16871933,342.7945743)
\curveto(473.17871879,342.84456988)(473.17871879,342.88956984)(473.16871933,342.9295743)
\curveto(473.1687188,342.97956975)(473.17871879,343.0295697)(473.19871933,343.0795743)
\curveto(473.21871875,343.18956954)(473.23371874,343.29456943)(473.24371933,343.3945743)
\curveto(473.26371871,343.49456923)(473.28871868,343.59456913)(473.31871933,343.6945743)
\curveto(473.35871861,343.81456891)(473.39371858,343.9295688)(473.42371933,344.0395743)
\curveto(473.45371852,344.14956858)(473.49371848,344.25956847)(473.54371933,344.3695743)
\curveto(473.68371829,344.66956806)(473.85871811,344.95456777)(474.06871933,345.2245743)
\curveto(474.08871788,345.25456747)(474.11371786,345.27956745)(474.14371933,345.2995743)
\curveto(474.18371779,345.3295674)(474.21371776,345.35956737)(474.23371933,345.3895743)
\curveto(474.2737177,345.43956729)(474.31371766,345.48456724)(474.35371933,345.5245743)
\curveto(474.39371758,345.56456716)(474.43871753,345.60456712)(474.48871933,345.6445743)
\curveto(474.52871744,345.66456706)(474.56371741,345.68956704)(474.59371933,345.7195743)
\curveto(474.62371735,345.75956697)(474.65871731,345.78956694)(474.69871933,345.8095743)
\curveto(474.94871702,345.97956675)(475.23871673,346.11956661)(475.56871933,346.2295743)
\curveto(475.63871633,346.24956648)(475.70871626,346.26456646)(475.77871933,346.2745743)
\curveto(475.85871611,346.28456644)(475.93871603,346.29956643)(476.01871933,346.3195743)
\curveto(476.08871588,346.33956639)(476.17871579,346.34956638)(476.28871933,346.3495743)
\curveto(476.39871557,346.35956637)(476.50871546,346.36456636)(476.61871933,346.3645743)
\curveto(476.72871524,346.36456636)(476.83371514,346.35956637)(476.93371933,346.3495743)
\curveto(477.04371493,346.33956639)(477.13371484,346.3245664)(477.20371933,346.3045743)
\curveto(477.35371462,346.25456647)(477.49871447,346.20956652)(477.63871933,346.1695743)
\curveto(477.77871419,346.1295666)(477.90871406,346.07456665)(478.02871933,346.0045743)
\curveto(478.09871387,345.95456677)(478.16371381,345.90456682)(478.22371933,345.8545743)
\curveto(478.28371369,345.81456691)(478.34871362,345.76956696)(478.41871933,345.7195743)
\curveto(478.45871351,345.68956704)(478.51371346,345.64956708)(478.58371933,345.5995743)
\curveto(478.66371331,345.54956718)(478.73871323,345.54956718)(478.80871933,345.5995743)
\curveto(478.84871312,345.61956711)(478.8687131,345.65456707)(478.86871933,345.7045743)
\curveto(478.8687131,345.75456697)(478.87871309,345.80456692)(478.89871933,345.8545743)
\lineto(478.89871933,346.0045743)
\curveto(478.90871306,346.03456669)(478.91371306,346.06956666)(478.91371933,346.1095743)
\lineto(478.91371933,346.2295743)
\lineto(478.91371933,348.2695743)
\curveto(478.91371306,348.37956435)(478.90871306,348.49956423)(478.89871933,348.6295743)
\curveto(478.89871307,348.76956396)(478.92371305,348.87456385)(478.97371933,348.9445743)
\curveto(479.01371296,349.0245637)(479.08871288,349.07456365)(479.19871933,349.0945743)
\curveto(479.21871275,349.10456362)(479.23871273,349.10456362)(479.25871933,349.0945743)
\curveto(479.27871269,349.09456363)(479.29871267,349.09956363)(479.31871933,349.1095743)
\lineto(480.38371933,349.1095743)
\curveto(480.50371147,349.10956362)(480.61371136,349.10456362)(480.71371933,349.0945743)
\curveto(480.81371116,349.08456364)(480.88871108,349.04456368)(480.93871933,348.9745743)
\curveto(480.98871098,348.89456383)(481.01371096,348.78956394)(481.01371933,348.6595743)
\lineto(481.01371933,348.2995743)
\lineto(481.01371933,339.2845743)
\moveto(478.97371933,342.2245743)
\curveto(478.98371299,342.26457046)(478.98371299,342.30457042)(478.97371933,342.3445743)
\lineto(478.97371933,342.4795743)
\curveto(478.973713,342.57957015)(478.968713,342.67957005)(478.95871933,342.7795743)
\curveto(478.94871302,342.87956985)(478.93371304,342.96956976)(478.91371933,343.0495743)
\curveto(478.89371308,343.15956957)(478.8737131,343.25956947)(478.85371933,343.3495743)
\curveto(478.84371313,343.43956929)(478.81871315,343.5245692)(478.77871933,343.6045743)
\curveto(478.63871333,343.96456876)(478.43371354,344.24956848)(478.16371933,344.4595743)
\curveto(477.90371407,344.66956806)(477.52371445,344.77456795)(477.02371933,344.7745743)
\curveto(476.96371501,344.77456795)(476.88371509,344.76456796)(476.78371933,344.7445743)
\curveto(476.70371527,344.724568)(476.62871534,344.70456802)(476.55871933,344.6845743)
\curveto(476.49871547,344.67456805)(476.43871553,344.65456807)(476.37871933,344.6245743)
\curveto(476.10871586,344.51456821)(475.89871607,344.34456838)(475.74871933,344.1145743)
\curveto(475.59871637,343.88456884)(475.47871649,343.6245691)(475.38871933,343.3345743)
\curveto(475.35871661,343.23456949)(475.33871663,343.13456959)(475.32871933,343.0345743)
\curveto(475.31871665,342.93456979)(475.29871667,342.8295699)(475.26871933,342.7195743)
\lineto(475.26871933,342.5095743)
\curveto(475.24871672,342.41957031)(475.24371673,342.29457043)(475.25371933,342.1345743)
\curveto(475.26371671,341.98457074)(475.27871669,341.87457085)(475.29871933,341.8045743)
\lineto(475.29871933,341.7145743)
\curveto(475.30871666,341.69457103)(475.31371666,341.67457105)(475.31371933,341.6545743)
\curveto(475.33371664,341.57457115)(475.34871662,341.49957123)(475.35871933,341.4295743)
\curveto(475.37871659,341.35957137)(475.39871657,341.28457144)(475.41871933,341.2045743)
\curveto(475.58871638,340.68457204)(475.87871609,340.29957243)(476.28871933,340.0495743)
\curveto(476.41871555,339.95957277)(476.59871537,339.88957284)(476.82871933,339.8395743)
\curveto(476.8687151,339.8295729)(476.92871504,339.8245729)(477.00871933,339.8245743)
\curveto(477.03871493,339.81457291)(477.08371489,339.80457292)(477.14371933,339.7945743)
\curveto(477.21371476,339.79457293)(477.2687147,339.79957293)(477.30871933,339.8095743)
\curveto(477.38871458,339.8295729)(477.4687145,339.84457288)(477.54871933,339.8545743)
\curveto(477.62871434,339.86457286)(477.70871426,339.88457284)(477.78871933,339.9145743)
\curveto(478.03871393,340.0245727)(478.23871373,340.16457256)(478.38871933,340.3345743)
\curveto(478.53871343,340.50457222)(478.6687133,340.71957201)(478.77871933,340.9795743)
\curveto(478.81871315,341.06957166)(478.84871312,341.15957157)(478.86871933,341.2495743)
\curveto(478.88871308,341.34957138)(478.90871306,341.45457127)(478.92871933,341.5645743)
\curveto(478.93871303,341.61457111)(478.93871303,341.65957107)(478.92871933,341.6995743)
\curveto(478.92871304,341.74957098)(478.93871303,341.79957093)(478.95871933,341.8495743)
\curveto(478.968713,341.87957085)(478.973713,341.91457081)(478.97371933,341.9545743)
\lineto(478.97371933,342.0895743)
\lineto(478.97371933,342.2245743)
}
}
{
\newrgbcolor{curcolor}{0 0 0}
\pscustom[linestyle=none,fillstyle=solid,fillcolor=curcolor]
{
\newpath
\moveto(489.9586412,342.3745743)
\curveto(489.97863304,342.29457043)(489.97863304,342.20457052)(489.9586412,342.1045743)
\curveto(489.93863308,342.00457072)(489.90363311,341.93957079)(489.8536412,341.9095743)
\curveto(489.80363321,341.86957086)(489.72863329,341.83957089)(489.6286412,341.8195743)
\curveto(489.53863348,341.80957092)(489.43363358,341.79957093)(489.3136412,341.7895743)
\lineto(488.9686412,341.7895743)
\curveto(488.85863416,341.79957093)(488.75863426,341.80457092)(488.6686412,341.8045743)
\lineto(485.0086412,341.8045743)
\lineto(484.7986412,341.8045743)
\curveto(484.73863828,341.80457092)(484.68363833,341.79457093)(484.6336412,341.7745743)
\curveto(484.55363846,341.73457099)(484.50363851,341.69457103)(484.4836412,341.6545743)
\curveto(484.46363855,341.63457109)(484.44363857,341.59457113)(484.4236412,341.5345743)
\curveto(484.40363861,341.48457124)(484.39863862,341.43457129)(484.4086412,341.3845743)
\curveto(484.42863859,341.3245714)(484.43863858,341.26457146)(484.4386412,341.2045743)
\curveto(484.44863857,341.15457157)(484.46363855,341.09957163)(484.4836412,341.0395743)
\curveto(484.56363845,340.79957193)(484.65863836,340.59957213)(484.7686412,340.4395743)
\curveto(484.88863813,340.28957244)(485.04863797,340.15457257)(485.2486412,340.0345743)
\curveto(485.32863769,339.98457274)(485.40863761,339.94957278)(485.4886412,339.9295743)
\curveto(485.57863744,339.91957281)(485.66863735,339.89957283)(485.7586412,339.8695743)
\curveto(485.83863718,339.84957288)(485.94863707,339.83457289)(486.0886412,339.8245743)
\curveto(486.22863679,339.81457291)(486.34863667,339.81957291)(486.4486412,339.8395743)
\lineto(486.5836412,339.8395743)
\curveto(486.68363633,339.85957287)(486.77363624,339.87957285)(486.8536412,339.8995743)
\curveto(486.94363607,339.9295728)(487.02863599,339.95957277)(487.1086412,339.9895743)
\curveto(487.20863581,340.03957269)(487.3186357,340.10457262)(487.4386412,340.1845743)
\curveto(487.56863545,340.26457246)(487.66363535,340.34457238)(487.7236412,340.4245743)
\curveto(487.77363524,340.49457223)(487.82363519,340.55957217)(487.8736412,340.6195743)
\curveto(487.93363508,340.68957204)(488.00363501,340.73957199)(488.0836412,340.7695743)
\curveto(488.18363483,340.81957191)(488.30863471,340.83957189)(488.4586412,340.8295743)
\lineto(488.8936412,340.8295743)
\lineto(489.0736412,340.8295743)
\curveto(489.14363387,340.83957189)(489.20363381,340.83457189)(489.2536412,340.8145743)
\lineto(489.4036412,340.8145743)
\curveto(489.50363351,340.79457193)(489.57363344,340.76957196)(489.6136412,340.7395743)
\curveto(489.65363336,340.71957201)(489.67363334,340.67457205)(489.6736412,340.6045743)
\curveto(489.68363333,340.53457219)(489.67863334,340.47457225)(489.6586412,340.4245743)
\curveto(489.60863341,340.28457244)(489.55363346,340.15957257)(489.4936412,340.0495743)
\curveto(489.43363358,339.93957279)(489.36363365,339.8295729)(489.2836412,339.7195743)
\curveto(489.06363395,339.38957334)(488.8136342,339.1245736)(488.5336412,338.9245743)
\curveto(488.25363476,338.724574)(487.90363511,338.55457417)(487.4836412,338.4145743)
\curveto(487.37363564,338.37457435)(487.26363575,338.34957438)(487.1536412,338.3395743)
\curveto(487.04363597,338.3295744)(486.92863609,338.30957442)(486.8086412,338.2795743)
\curveto(486.76863625,338.26957446)(486.72363629,338.26957446)(486.6736412,338.2795743)
\curveto(486.63363638,338.27957445)(486.59363642,338.27457445)(486.5536412,338.2645743)
\lineto(486.3886412,338.2645743)
\curveto(486.33863668,338.24457448)(486.27863674,338.23957449)(486.2086412,338.2495743)
\curveto(486.14863687,338.24957448)(486.09363692,338.25457447)(486.0436412,338.2645743)
\curveto(485.96363705,338.27457445)(485.89363712,338.27457445)(485.8336412,338.2645743)
\curveto(485.77363724,338.25457447)(485.70863731,338.25957447)(485.6386412,338.2795743)
\curveto(485.58863743,338.29957443)(485.53363748,338.30957442)(485.4736412,338.3095743)
\curveto(485.4136376,338.30957442)(485.35863766,338.31957441)(485.3086412,338.3395743)
\curveto(485.19863782,338.35957437)(485.08863793,338.38457434)(484.9786412,338.4145743)
\curveto(484.86863815,338.43457429)(484.76863825,338.46957426)(484.6786412,338.5195743)
\curveto(484.56863845,338.55957417)(484.46363855,338.59457413)(484.3636412,338.6245743)
\curveto(484.27363874,338.66457406)(484.18863883,338.70957402)(484.1086412,338.7595743)
\curveto(483.78863923,338.95957377)(483.50363951,339.18957354)(483.2536412,339.4495743)
\curveto(483.00364001,339.71957301)(482.79864022,340.0295727)(482.6386412,340.3795743)
\curveto(482.58864043,340.48957224)(482.54864047,340.59957213)(482.5186412,340.7095743)
\curveto(482.48864053,340.8295719)(482.44864057,340.94957178)(482.3986412,341.0695743)
\curveto(482.38864063,341.10957162)(482.38364063,341.14457158)(482.3836412,341.1745743)
\curveto(482.38364063,341.21457151)(482.37864064,341.25457147)(482.3686412,341.2945743)
\curveto(482.32864069,341.41457131)(482.30364071,341.54457118)(482.2936412,341.6845743)
\lineto(482.2636412,342.1045743)
\curveto(482.26364075,342.15457057)(482.25864076,342.20957052)(482.2486412,342.2695743)
\curveto(482.24864077,342.3295704)(482.25364076,342.38457034)(482.2636412,342.4345743)
\lineto(482.2636412,342.6145743)
\lineto(482.3086412,342.9745743)
\curveto(482.34864067,343.14456958)(482.38364063,343.30956942)(482.4136412,343.4695743)
\curveto(482.44364057,343.6295691)(482.48864053,343.77956895)(482.5486412,343.9195743)
\curveto(482.97864004,344.95956777)(483.70863931,345.69456703)(484.7386412,346.1245743)
\curveto(484.87863814,346.18456654)(485.018638,346.2245665)(485.1586412,346.2445743)
\curveto(485.30863771,346.27456645)(485.46363755,346.30956642)(485.6236412,346.3495743)
\curveto(485.70363731,346.35956637)(485.77863724,346.36456636)(485.8486412,346.3645743)
\curveto(485.9186371,346.36456636)(485.99363702,346.36956636)(486.0736412,346.3795743)
\curveto(486.58363643,346.38956634)(487.018636,346.3295664)(487.3786412,346.1995743)
\curveto(487.74863527,346.07956665)(488.07863494,345.91956681)(488.3686412,345.7195743)
\curveto(488.45863456,345.65956707)(488.54863447,345.58956714)(488.6386412,345.5095743)
\curveto(488.72863429,345.43956729)(488.80863421,345.36456736)(488.8786412,345.2845743)
\curveto(488.90863411,345.23456749)(488.94863407,345.19456753)(488.9986412,345.1645743)
\curveto(489.07863394,345.05456767)(489.15363386,344.93956779)(489.2236412,344.8195743)
\curveto(489.29363372,344.70956802)(489.36863365,344.59456813)(489.4486412,344.4745743)
\curveto(489.49863352,344.38456834)(489.53863348,344.28956844)(489.5686412,344.1895743)
\curveto(489.60863341,344.09956863)(489.64863337,343.99956873)(489.6886412,343.8895743)
\curveto(489.73863328,343.75956897)(489.77863324,343.6245691)(489.8086412,343.4845743)
\curveto(489.83863318,343.34456938)(489.87363314,343.20456952)(489.9136412,343.0645743)
\curveto(489.93363308,342.98456974)(489.93863308,342.89456983)(489.9286412,342.7945743)
\curveto(489.92863309,342.70457002)(489.93863308,342.61957011)(489.9586412,342.5395743)
\lineto(489.9586412,342.3745743)
\moveto(487.7086412,343.2595743)
\curveto(487.77863524,343.35956937)(487.78363523,343.47956925)(487.7236412,343.6195743)
\curveto(487.67363534,343.76956896)(487.63363538,343.87956885)(487.6036412,343.9495743)
\curveto(487.46363555,344.21956851)(487.27863574,344.4245683)(487.0486412,344.5645743)
\curveto(486.8186362,344.71456801)(486.49863652,344.79456793)(486.0886412,344.8045743)
\curveto(486.05863696,344.78456794)(486.02363699,344.77956795)(485.9836412,344.7895743)
\curveto(485.94363707,344.79956793)(485.90863711,344.79956793)(485.8786412,344.7895743)
\curveto(485.82863719,344.76956796)(485.77363724,344.75456797)(485.7136412,344.7445743)
\curveto(485.65363736,344.74456798)(485.59863742,344.73456799)(485.5486412,344.7145743)
\curveto(485.10863791,344.57456815)(484.78363823,344.29956843)(484.5736412,343.8895743)
\curveto(484.55363846,343.84956888)(484.52863849,343.79456893)(484.4986412,343.7245743)
\curveto(484.47863854,343.66456906)(484.46363855,343.59956913)(484.4536412,343.5295743)
\curveto(484.44363857,343.46956926)(484.44363857,343.40956932)(484.4536412,343.3495743)
\curveto(484.47363854,343.28956944)(484.50863851,343.23956949)(484.5586412,343.1995743)
\curveto(484.63863838,343.14956958)(484.74863827,343.1245696)(484.8886412,343.1245743)
\lineto(485.2936412,343.1245743)
\lineto(486.9586412,343.1245743)
\lineto(487.3936412,343.1245743)
\curveto(487.55363546,343.13456959)(487.65863536,343.17956955)(487.7086412,343.2595743)
}
}
{
\newrgbcolor{curcolor}{0 0 0}
\pscustom[linestyle=none,fillstyle=solid,fillcolor=curcolor]
{
}
}
{
\newrgbcolor{curcolor}{0 0 0}
\pscustom[linestyle=none,fillstyle=solid,fillcolor=curcolor]
{
\newpath
\moveto(495.8770787,349.1245743)
\lineto(496.9720787,349.1245743)
\curveto(497.07207622,349.1245636)(497.16707612,349.11956361)(497.2570787,349.1095743)
\curveto(497.34707594,349.09956363)(497.41707587,349.06956366)(497.4670787,349.0195743)
\curveto(497.52707576,348.94956378)(497.55707573,348.85456387)(497.5570787,348.7345743)
\curveto(497.56707572,348.6245641)(497.57207572,348.50956422)(497.5720787,348.3895743)
\lineto(497.5720787,347.0545743)
\lineto(497.5720787,341.6695743)
\lineto(497.5720787,339.3745743)
\lineto(497.5720787,338.9545743)
\curveto(497.58207571,338.80457392)(497.56207573,338.68957404)(497.5120787,338.6095743)
\curveto(497.46207583,338.5295742)(497.37207592,338.47457425)(497.2420787,338.4445743)
\curveto(497.18207611,338.4245743)(497.11207618,338.41957431)(497.0320787,338.4295743)
\curveto(496.96207633,338.43957429)(496.8920764,338.44457428)(496.8220787,338.4445743)
\lineto(496.1020787,338.4445743)
\curveto(495.9920773,338.44457428)(495.8920774,338.44957428)(495.8020787,338.4595743)
\curveto(495.71207758,338.46957426)(495.63707765,338.49957423)(495.5770787,338.5495743)
\curveto(495.51707777,338.59957413)(495.48207781,338.67457405)(495.4720787,338.7745743)
\lineto(495.4720787,339.1045743)
\lineto(495.4720787,340.4395743)
\lineto(495.4720787,346.0645743)
\lineto(495.4720787,348.1045743)
\curveto(495.47207782,348.23456449)(495.46707782,348.38956434)(495.4570787,348.5695743)
\curveto(495.45707783,348.74956398)(495.48207781,348.87956385)(495.5320787,348.9595743)
\curveto(495.55207774,348.99956373)(495.57707771,349.0295637)(495.6070787,349.0495743)
\lineto(495.7270787,349.1095743)
\curveto(495.74707754,349.10956362)(495.77207752,349.10956362)(495.8020787,349.1095743)
\curveto(495.83207746,349.11956361)(495.85707743,349.1245636)(495.8770787,349.1245743)
}
}
{
\newrgbcolor{curcolor}{0 0 0}
\pscustom[linestyle=none,fillstyle=solid,fillcolor=curcolor]
{
\newpath
\moveto(507.0042662,342.6145743)
\curveto(507.02425763,342.55457017)(507.03425762,342.46957026)(507.0342662,342.3595743)
\curveto(507.03425762,342.24957048)(507.02425763,342.16457056)(507.0042662,342.1045743)
\lineto(507.0042662,341.9545743)
\curveto(506.98425767,341.87457085)(506.97425768,341.79457093)(506.9742662,341.7145743)
\curveto(506.98425767,341.63457109)(506.97925768,341.55457117)(506.9592662,341.4745743)
\curveto(506.93925772,341.40457132)(506.92425773,341.33957139)(506.9142662,341.2795743)
\curveto(506.90425775,341.21957151)(506.89425776,341.15457157)(506.8842662,341.0845743)
\curveto(506.84425781,340.97457175)(506.80925785,340.85957187)(506.7792662,340.7395743)
\curveto(506.74925791,340.6295721)(506.70925795,340.5245722)(506.6592662,340.4245743)
\curveto(506.44925821,339.94457278)(506.17425848,339.55457317)(505.8342662,339.2545743)
\curveto(505.49425916,338.95457377)(505.08425957,338.70457402)(504.6042662,338.5045743)
\curveto(504.48426017,338.45457427)(504.3592603,338.41957431)(504.2292662,338.3995743)
\curveto(504.10926055,338.36957436)(503.98426067,338.33957439)(503.8542662,338.3095743)
\curveto(503.80426085,338.28957444)(503.74926091,338.27957445)(503.6892662,338.2795743)
\curveto(503.62926103,338.27957445)(503.57426108,338.27457445)(503.5242662,338.2645743)
\lineto(503.4192662,338.2645743)
\curveto(503.38926127,338.25457447)(503.3592613,338.24957448)(503.3292662,338.2495743)
\curveto(503.27926138,338.23957449)(503.19926146,338.23457449)(503.0892662,338.2345743)
\curveto(502.97926168,338.2245745)(502.89426176,338.2295745)(502.8342662,338.2495743)
\lineto(502.6842662,338.2495743)
\curveto(502.63426202,338.25957447)(502.57926208,338.26457446)(502.5192662,338.2645743)
\curveto(502.46926219,338.25457447)(502.41926224,338.25957447)(502.3692662,338.2795743)
\curveto(502.32926233,338.28957444)(502.28926237,338.29457443)(502.2492662,338.2945743)
\curveto(502.21926244,338.29457443)(502.17926248,338.29957443)(502.1292662,338.3095743)
\curveto(502.02926263,338.33957439)(501.92926273,338.36457436)(501.8292662,338.3845743)
\curveto(501.72926293,338.40457432)(501.63426302,338.43457429)(501.5442662,338.4745743)
\curveto(501.42426323,338.51457421)(501.30926335,338.55457417)(501.1992662,338.5945743)
\curveto(501.09926356,338.63457409)(500.99426366,338.68457404)(500.8842662,338.7445743)
\curveto(500.53426412,338.95457377)(500.23426442,339.19957353)(499.9842662,339.4795743)
\curveto(499.73426492,339.75957297)(499.52426513,340.09457263)(499.3542662,340.4845743)
\curveto(499.30426535,340.57457215)(499.26426539,340.66957206)(499.2342662,340.7695743)
\curveto(499.21426544,340.86957186)(499.18926547,340.97457175)(499.1592662,341.0845743)
\curveto(499.13926552,341.13457159)(499.12926553,341.17957155)(499.1292662,341.2195743)
\curveto(499.12926553,341.25957147)(499.11926554,341.30457142)(499.0992662,341.3545743)
\curveto(499.07926558,341.43457129)(499.06926559,341.51457121)(499.0692662,341.5945743)
\curveto(499.06926559,341.68457104)(499.0592656,341.76957096)(499.0392662,341.8495743)
\curveto(499.02926563,341.89957083)(499.02426563,341.94457078)(499.0242662,341.9845743)
\lineto(499.0242662,342.1195743)
\curveto(499.00426565,342.17957055)(498.99426566,342.26457046)(498.9942662,342.3745743)
\curveto(499.00426565,342.48457024)(499.01926564,342.56957016)(499.0392662,342.6295743)
\lineto(499.0392662,342.7345743)
\curveto(499.04926561,342.78456994)(499.04926561,342.83456989)(499.0392662,342.8845743)
\curveto(499.03926562,342.94456978)(499.04926561,342.99956973)(499.0692662,343.0495743)
\curveto(499.07926558,343.09956963)(499.08426557,343.14456958)(499.0842662,343.1845743)
\curveto(499.08426557,343.23456949)(499.09426556,343.28456944)(499.1142662,343.3345743)
\curveto(499.1542655,343.46456926)(499.18926547,343.58956914)(499.2192662,343.7095743)
\curveto(499.24926541,343.83956889)(499.28926537,343.96456876)(499.3392662,344.0845743)
\curveto(499.51926514,344.49456823)(499.73426492,344.83456789)(499.9842662,345.1045743)
\curveto(500.23426442,345.38456734)(500.53926412,345.63956709)(500.8992662,345.8695743)
\curveto(500.99926366,345.91956681)(501.10426355,345.96456676)(501.2142662,346.0045743)
\curveto(501.32426333,346.04456668)(501.43426322,346.08956664)(501.5442662,346.1395743)
\curveto(501.67426298,346.18956654)(501.80926285,346.2245665)(501.9492662,346.2445743)
\curveto(502.08926257,346.26456646)(502.23426242,346.29456643)(502.3842662,346.3345743)
\curveto(502.46426219,346.34456638)(502.53926212,346.34956638)(502.6092662,346.3495743)
\curveto(502.67926198,346.34956638)(502.74926191,346.35456637)(502.8192662,346.3645743)
\curveto(503.39926126,346.37456635)(503.89926076,346.31456641)(504.3192662,346.1845743)
\curveto(504.74925991,346.05456667)(505.12925953,345.87456685)(505.4592662,345.6445743)
\curveto(505.56925909,345.56456716)(505.67925898,345.47456725)(505.7892662,345.3745743)
\curveto(505.90925875,345.28456744)(506.00925865,345.18456754)(506.0892662,345.0745743)
\curveto(506.16925849,344.97456775)(506.23925842,344.87456785)(506.2992662,344.7745743)
\curveto(506.36925829,344.67456805)(506.43925822,344.56956816)(506.5092662,344.4595743)
\curveto(506.57925808,344.34956838)(506.63425802,344.2295685)(506.6742662,344.0995743)
\curveto(506.71425794,343.97956875)(506.7592579,343.84956888)(506.8092662,343.7095743)
\curveto(506.83925782,343.6295691)(506.86425779,343.54456918)(506.8842662,343.4545743)
\lineto(506.9442662,343.1845743)
\curveto(506.9542577,343.14456958)(506.9592577,343.10456962)(506.9592662,343.0645743)
\curveto(506.9592577,343.0245697)(506.96425769,342.98456974)(506.9742662,342.9445743)
\curveto(506.99425766,342.89456983)(506.99925766,342.83956989)(506.9892662,342.7795743)
\curveto(506.97925768,342.71957001)(506.98425767,342.66457006)(507.0042662,342.6145743)
\moveto(504.9042662,342.0745743)
\curveto(504.91425974,342.1245706)(504.91925974,342.19457053)(504.9192662,342.2845743)
\curveto(504.91925974,342.38457034)(504.91425974,342.45957027)(504.9042662,342.5095743)
\lineto(504.9042662,342.6295743)
\curveto(504.88425977,342.67957005)(504.87425978,342.73456999)(504.8742662,342.7945743)
\curveto(504.87425978,342.85456987)(504.86925979,342.90956982)(504.8592662,342.9595743)
\curveto(504.8592598,342.99956973)(504.8542598,343.0295697)(504.8442662,343.0495743)
\lineto(504.7842662,343.2895743)
\curveto(504.77425988,343.37956935)(504.7542599,343.46456926)(504.7242662,343.5445743)
\curveto(504.61426004,343.80456892)(504.48426017,344.0245687)(504.3342662,344.2045743)
\curveto(504.18426047,344.39456833)(503.98426067,344.54456818)(503.7342662,344.6545743)
\curveto(503.67426098,344.67456805)(503.61426104,344.68956804)(503.5542662,344.6995743)
\curveto(503.49426116,344.71956801)(503.42926123,344.73956799)(503.3592662,344.7595743)
\curveto(503.27926138,344.77956795)(503.19426146,344.78456794)(503.1042662,344.7745743)
\lineto(502.8342662,344.7745743)
\curveto(502.80426185,344.75456797)(502.76926189,344.74456798)(502.7292662,344.7445743)
\curveto(502.68926197,344.75456797)(502.654262,344.75456797)(502.6242662,344.7445743)
\lineto(502.4142662,344.6845743)
\curveto(502.3542623,344.67456805)(502.29926236,344.65456807)(502.2492662,344.6245743)
\curveto(501.99926266,344.51456821)(501.79426286,344.35456837)(501.6342662,344.1445743)
\curveto(501.48426317,343.94456878)(501.36426329,343.70956902)(501.2742662,343.4395743)
\curveto(501.24426341,343.33956939)(501.21926344,343.23456949)(501.1992662,343.1245743)
\curveto(501.18926347,343.01456971)(501.17426348,342.90456982)(501.1542662,342.7945743)
\curveto(501.14426351,342.74456998)(501.13926352,342.69457003)(501.1392662,342.6445743)
\lineto(501.1392662,342.4945743)
\curveto(501.11926354,342.4245703)(501.10926355,342.31957041)(501.1092662,342.1795743)
\curveto(501.11926354,342.03957069)(501.13426352,341.93457079)(501.1542662,341.8645743)
\lineto(501.1542662,341.7295743)
\curveto(501.17426348,341.64957108)(501.18926347,341.56957116)(501.1992662,341.4895743)
\curveto(501.20926345,341.41957131)(501.22426343,341.34457138)(501.2442662,341.2645743)
\curveto(501.34426331,340.96457176)(501.44926321,340.71957201)(501.5592662,340.5295743)
\curveto(501.67926298,340.34957238)(501.86426279,340.18457254)(502.1142662,340.0345743)
\curveto(502.18426247,339.98457274)(502.2592624,339.94457278)(502.3392662,339.9145743)
\curveto(502.42926223,339.88457284)(502.51926214,339.85957287)(502.6092662,339.8395743)
\curveto(502.64926201,339.8295729)(502.68426197,339.8245729)(502.7142662,339.8245743)
\curveto(502.74426191,339.83457289)(502.77926188,339.83457289)(502.8192662,339.8245743)
\lineto(502.9392662,339.7945743)
\curveto(502.98926167,339.79457293)(503.03426162,339.79957293)(503.0742662,339.8095743)
\lineto(503.1942662,339.8095743)
\curveto(503.27426138,339.8295729)(503.3542613,339.84457288)(503.4342662,339.8545743)
\curveto(503.51426114,339.86457286)(503.58926107,339.88457284)(503.6592662,339.9145743)
\curveto(503.91926074,340.01457271)(504.12926053,340.14957258)(504.2892662,340.3195743)
\curveto(504.44926021,340.48957224)(504.58426007,340.69957203)(504.6942662,340.9495743)
\curveto(504.73425992,341.04957168)(504.76425989,341.14957158)(504.7842662,341.2495743)
\curveto(504.80425985,341.34957138)(504.82925983,341.45457127)(504.8592662,341.5645743)
\curveto(504.86925979,341.60457112)(504.87425978,341.63957109)(504.8742662,341.6695743)
\curveto(504.87425978,341.70957102)(504.87925978,341.74957098)(504.8892662,341.7895743)
\lineto(504.8892662,341.9245743)
\curveto(504.88925977,341.97457075)(504.89425976,342.0245707)(504.9042662,342.0745743)
}
}
{
\newrgbcolor{curcolor}{0 0 0}
\pscustom[linestyle=none,fillstyle=solid,fillcolor=curcolor]
{
\newpath
\moveto(511.37418808,346.3795743)
\curveto(512.12418358,346.39956633)(512.77418293,346.31456641)(513.32418808,346.1245743)
\curveto(513.88418182,345.94456678)(514.30918139,345.6295671)(514.59918808,345.1795743)
\curveto(514.66918103,345.06956766)(514.72918097,344.95456777)(514.77918808,344.8345743)
\curveto(514.83918086,344.724568)(514.88918081,344.59956813)(514.92918808,344.4595743)
\curveto(514.94918075,344.39956833)(514.95918074,344.33456839)(514.95918808,344.2645743)
\curveto(514.95918074,344.19456853)(514.94918075,344.13456859)(514.92918808,344.0845743)
\curveto(514.88918081,344.0245687)(514.83418087,343.98456874)(514.76418808,343.9645743)
\curveto(514.71418099,343.94456878)(514.65418105,343.93456879)(514.58418808,343.9345743)
\lineto(514.37418808,343.9345743)
\lineto(513.71418808,343.9345743)
\curveto(513.64418206,343.93456879)(513.57418213,343.9295688)(513.50418808,343.9195743)
\curveto(513.43418227,343.91956881)(513.36918233,343.9295688)(513.30918808,343.9495743)
\curveto(513.20918249,343.96956876)(513.13418257,344.00956872)(513.08418808,344.0695743)
\curveto(513.03418267,344.1295686)(512.98918271,344.18956854)(512.94918808,344.2495743)
\lineto(512.82918808,344.4595743)
\curveto(512.7991829,344.53956819)(512.74918295,344.60456812)(512.67918808,344.6545743)
\curveto(512.57918312,344.73456799)(512.47918322,344.79456793)(512.37918808,344.8345743)
\curveto(512.28918341,344.87456785)(512.17418353,344.90956782)(512.03418808,344.9395743)
\curveto(511.96418374,344.95956777)(511.85918384,344.97456775)(511.71918808,344.9845743)
\curveto(511.58918411,344.99456773)(511.48918421,344.98956774)(511.41918808,344.9695743)
\lineto(511.31418808,344.9695743)
\lineto(511.16418808,344.9395743)
\curveto(511.12418458,344.93956779)(511.07918462,344.93456779)(511.02918808,344.9245743)
\curveto(510.85918484,344.87456785)(510.71918498,344.80456792)(510.60918808,344.7145743)
\curveto(510.50918519,344.63456809)(510.43918526,344.50956822)(510.39918808,344.3395743)
\curveto(510.37918532,344.26956846)(510.37918532,344.20456852)(510.39918808,344.1445743)
\curveto(510.41918528,344.08456864)(510.43918526,344.03456869)(510.45918808,343.9945743)
\curveto(510.52918517,343.87456885)(510.60918509,343.77956895)(510.69918808,343.7095743)
\curveto(510.7991849,343.63956909)(510.91418479,343.57956915)(511.04418808,343.5295743)
\curveto(511.23418447,343.44956928)(511.43918426,343.37956935)(511.65918808,343.3195743)
\lineto(512.34918808,343.1695743)
\curveto(512.58918311,343.1295696)(512.81918288,343.07956965)(513.03918808,343.0195743)
\curveto(513.26918243,342.96956976)(513.48418222,342.90456982)(513.68418808,342.8245743)
\curveto(513.77418193,342.78456994)(513.85918184,342.74956998)(513.93918808,342.7195743)
\curveto(514.02918167,342.69957003)(514.11418159,342.66457006)(514.19418808,342.6145743)
\curveto(514.38418132,342.49457023)(514.55418115,342.36457036)(514.70418808,342.2245743)
\curveto(514.86418084,342.08457064)(514.98918071,341.90957082)(515.07918808,341.6995743)
\curveto(515.10918059,341.6295711)(515.13418057,341.55957117)(515.15418808,341.4895743)
\curveto(515.17418053,341.41957131)(515.19418051,341.34457138)(515.21418808,341.2645743)
\curveto(515.22418048,341.20457152)(515.22918047,341.10957162)(515.22918808,340.9795743)
\curveto(515.23918046,340.85957187)(515.23918046,340.76457196)(515.22918808,340.6945743)
\lineto(515.22918808,340.6195743)
\curveto(515.20918049,340.55957217)(515.19418051,340.49957223)(515.18418808,340.4395743)
\curveto(515.18418052,340.38957234)(515.17918052,340.33957239)(515.16918808,340.2895743)
\curveto(515.0991806,339.98957274)(514.98918071,339.724573)(514.83918808,339.4945743)
\curveto(514.67918102,339.25457347)(514.48418122,339.05957367)(514.25418808,338.9095743)
\curveto(514.02418168,338.75957397)(513.76418194,338.6295741)(513.47418808,338.5195743)
\curveto(513.36418234,338.46957426)(513.24418246,338.43457429)(513.11418808,338.4145743)
\curveto(512.99418271,338.39457433)(512.87418283,338.36957436)(512.75418808,338.3395743)
\curveto(512.66418304,338.31957441)(512.56918313,338.30957442)(512.46918808,338.3095743)
\curveto(512.37918332,338.29957443)(512.28918341,338.28457444)(512.19918808,338.2645743)
\lineto(511.92918808,338.2645743)
\curveto(511.86918383,338.24457448)(511.76418394,338.23457449)(511.61418808,338.2345743)
\curveto(511.47418423,338.23457449)(511.37418433,338.24457448)(511.31418808,338.2645743)
\curveto(511.28418442,338.26457446)(511.24918445,338.26957446)(511.20918808,338.2795743)
\lineto(511.10418808,338.2795743)
\curveto(510.98418472,338.29957443)(510.86418484,338.31457441)(510.74418808,338.3245743)
\curveto(510.62418508,338.33457439)(510.50918519,338.35457437)(510.39918808,338.3845743)
\curveto(510.00918569,338.49457423)(509.66418604,338.61957411)(509.36418808,338.7595743)
\curveto(509.06418664,338.90957382)(508.80918689,339.1295736)(508.59918808,339.4195743)
\curveto(508.45918724,339.60957312)(508.33918736,339.8295729)(508.23918808,340.0795743)
\curveto(508.21918748,340.13957259)(508.1991875,340.21957251)(508.17918808,340.3195743)
\curveto(508.15918754,340.36957236)(508.14418756,340.43957229)(508.13418808,340.5295743)
\curveto(508.12418758,340.61957211)(508.12918757,340.69457203)(508.14918808,340.7545743)
\curveto(508.17918752,340.8245719)(508.22918747,340.87457185)(508.29918808,340.9045743)
\curveto(508.34918735,340.9245718)(508.40918729,340.93457179)(508.47918808,340.9345743)
\lineto(508.70418808,340.9345743)
\lineto(509.40918808,340.9345743)
\lineto(509.64918808,340.9345743)
\curveto(509.72918597,340.93457179)(509.7991859,340.9245718)(509.85918808,340.9045743)
\curveto(509.96918573,340.86457186)(510.03918566,340.79957193)(510.06918808,340.7095743)
\curveto(510.10918559,340.61957211)(510.15418555,340.5245722)(510.20418808,340.4245743)
\curveto(510.22418548,340.37457235)(510.25918544,340.30957242)(510.30918808,340.2295743)
\curveto(510.36918533,340.14957258)(510.41918528,340.09957263)(510.45918808,340.0795743)
\curveto(510.57918512,339.97957275)(510.69418501,339.89957283)(510.80418808,339.8395743)
\curveto(510.91418479,339.78957294)(511.05418465,339.73957299)(511.22418808,339.6895743)
\curveto(511.27418443,339.66957306)(511.32418438,339.65957307)(511.37418808,339.6595743)
\curveto(511.42418428,339.66957306)(511.47418423,339.66957306)(511.52418808,339.6595743)
\curveto(511.6041841,339.63957309)(511.68918401,339.6295731)(511.77918808,339.6295743)
\curveto(511.87918382,339.63957309)(511.96418374,339.65457307)(512.03418808,339.6745743)
\curveto(512.08418362,339.68457304)(512.12918357,339.68957304)(512.16918808,339.6895743)
\curveto(512.21918348,339.68957304)(512.26918343,339.69957303)(512.31918808,339.7195743)
\curveto(512.45918324,339.76957296)(512.58418312,339.8295729)(512.69418808,339.8995743)
\curveto(512.81418289,339.96957276)(512.90918279,340.05957267)(512.97918808,340.1695743)
\curveto(513.02918267,340.24957248)(513.06918263,340.37457235)(513.09918808,340.5445743)
\curveto(513.11918258,340.61457211)(513.11918258,340.67957205)(513.09918808,340.7395743)
\curveto(513.07918262,340.79957193)(513.05918264,340.84957188)(513.03918808,340.8895743)
\curveto(512.96918273,341.0295717)(512.87918282,341.13457159)(512.76918808,341.2045743)
\curveto(512.66918303,341.27457145)(512.54918315,341.33957139)(512.40918808,341.3995743)
\curveto(512.21918348,341.47957125)(512.01918368,341.54457118)(511.80918808,341.5945743)
\curveto(511.5991841,341.64457108)(511.38918431,341.69957103)(511.17918808,341.7595743)
\curveto(511.0991846,341.77957095)(511.01418469,341.79457093)(510.92418808,341.8045743)
\curveto(510.84418486,341.81457091)(510.76418494,341.8295709)(510.68418808,341.8495743)
\curveto(510.36418534,341.93957079)(510.05918564,342.0245707)(509.76918808,342.1045743)
\curveto(509.47918622,342.19457053)(509.21418649,342.3245704)(508.97418808,342.4945743)
\curveto(508.69418701,342.69457003)(508.48918721,342.96456976)(508.35918808,343.3045743)
\curveto(508.33918736,343.37456935)(508.31918738,343.46956926)(508.29918808,343.5895743)
\curveto(508.27918742,343.65956907)(508.26418744,343.74456898)(508.25418808,343.8445743)
\curveto(508.24418746,343.94456878)(508.24918745,344.03456869)(508.26918808,344.1145743)
\curveto(508.28918741,344.16456856)(508.29418741,344.20456852)(508.28418808,344.2345743)
\curveto(508.27418743,344.27456845)(508.27918742,344.31956841)(508.29918808,344.3695743)
\curveto(508.31918738,344.47956825)(508.33918736,344.57956815)(508.35918808,344.6695743)
\curveto(508.38918731,344.76956796)(508.42418728,344.86456786)(508.46418808,344.9545743)
\curveto(508.59418711,345.24456748)(508.77418693,345.47956725)(509.00418808,345.6595743)
\curveto(509.23418647,345.83956689)(509.49418621,345.98456674)(509.78418808,346.0945743)
\curveto(509.89418581,346.14456658)(510.00918569,346.17956655)(510.12918808,346.1995743)
\curveto(510.24918545,346.2295665)(510.37418533,346.25956647)(510.50418808,346.2895743)
\curveto(510.56418514,346.30956642)(510.62418508,346.31956641)(510.68418808,346.3195743)
\lineto(510.86418808,346.3495743)
\curveto(510.94418476,346.35956637)(511.02918467,346.36456636)(511.11918808,346.3645743)
\curveto(511.20918449,346.36456636)(511.29418441,346.36956636)(511.37418808,346.3795743)
}
}
{
\newrgbcolor{curcolor}{0 0 0}
\pscustom[linestyle=none,fillstyle=solid,fillcolor=curcolor]
{
}
}
{
\newrgbcolor{curcolor}{0 0 0}
\pscustom[linestyle=none,fillstyle=solid,fillcolor=curcolor]
{
\newpath
\moveto(525.04598495,346.3645743)
\curveto(525.15597964,346.36456636)(525.25097954,346.35456637)(525.33098495,346.3345743)
\curveto(525.42097937,346.31456641)(525.4909793,346.26956646)(525.54098495,346.1995743)
\curveto(525.60097919,346.11956661)(525.63097916,345.97956675)(525.63098495,345.7795743)
\lineto(525.63098495,345.2695743)
\lineto(525.63098495,344.8945743)
\curveto(525.64097915,344.75456797)(525.62597917,344.64456808)(525.58598495,344.5645743)
\curveto(525.54597925,344.49456823)(525.48597931,344.44956828)(525.40598495,344.4295743)
\curveto(525.33597946,344.40956832)(525.25097954,344.39956833)(525.15098495,344.3995743)
\curveto(525.06097973,344.39956833)(524.96097983,344.40456832)(524.85098495,344.4145743)
\curveto(524.75098004,344.4245683)(524.65598014,344.41956831)(524.56598495,344.3995743)
\curveto(524.4959803,344.37956835)(524.42598037,344.36456836)(524.35598495,344.3545743)
\curveto(524.28598051,344.35456837)(524.22098057,344.34456838)(524.16098495,344.3245743)
\curveto(524.00098079,344.27456845)(523.84098095,344.19956853)(523.68098495,344.0995743)
\curveto(523.52098127,344.00956872)(523.3959814,343.90456882)(523.30598495,343.7845743)
\curveto(523.25598154,343.70456902)(523.20098159,343.61956911)(523.14098495,343.5295743)
\curveto(523.0909817,343.44956928)(523.04098175,343.36456936)(522.99098495,343.2745743)
\curveto(522.96098183,343.19456953)(522.93098186,343.10956962)(522.90098495,343.0195743)
\lineto(522.84098495,342.7795743)
\curveto(522.82098197,342.70957002)(522.81098198,342.63457009)(522.81098495,342.5545743)
\curveto(522.81098198,342.48457024)(522.80098199,342.41457031)(522.78098495,342.3445743)
\curveto(522.77098202,342.30457042)(522.76598203,342.26457046)(522.76598495,342.2245743)
\curveto(522.77598202,342.19457053)(522.77598202,342.16457056)(522.76598495,342.1345743)
\lineto(522.76598495,341.8945743)
\curveto(522.74598205,341.8245709)(522.74098205,341.74457098)(522.75098495,341.6545743)
\curveto(522.76098203,341.57457115)(522.76598203,341.49457123)(522.76598495,341.4145743)
\lineto(522.76598495,340.4545743)
\lineto(522.76598495,339.1795743)
\curveto(522.76598203,339.04957368)(522.76098203,338.9295738)(522.75098495,338.8195743)
\curveto(522.74098205,338.70957402)(522.71098208,338.61957411)(522.66098495,338.5495743)
\curveto(522.64098215,338.51957421)(522.60598219,338.49457423)(522.55598495,338.4745743)
\curveto(522.51598228,338.46457426)(522.47098232,338.45457427)(522.42098495,338.4445743)
\lineto(522.34598495,338.4445743)
\curveto(522.2959825,338.43457429)(522.24098255,338.4295743)(522.18098495,338.4295743)
\lineto(522.01598495,338.4295743)
\lineto(521.37098495,338.4295743)
\curveto(521.31098348,338.43957429)(521.24598355,338.44457428)(521.17598495,338.4445743)
\lineto(520.98098495,338.4445743)
\curveto(520.93098386,338.46457426)(520.88098391,338.47957425)(520.83098495,338.4895743)
\curveto(520.78098401,338.50957422)(520.74598405,338.54457418)(520.72598495,338.5945743)
\curveto(520.68598411,338.64457408)(520.66098413,338.71457401)(520.65098495,338.8045743)
\lineto(520.65098495,339.1045743)
\lineto(520.65098495,340.1245743)
\lineto(520.65098495,344.3545743)
\lineto(520.65098495,345.4645743)
\lineto(520.65098495,345.7495743)
\curveto(520.65098414,345.84956688)(520.67098412,345.9295668)(520.71098495,345.9895743)
\curveto(520.76098403,346.06956666)(520.83598396,346.11956661)(520.93598495,346.1395743)
\curveto(521.03598376,346.15956657)(521.15598364,346.16956656)(521.29598495,346.1695743)
\lineto(522.06098495,346.1695743)
\curveto(522.18098261,346.16956656)(522.28598251,346.15956657)(522.37598495,346.1395743)
\curveto(522.46598233,346.1295666)(522.53598226,346.08456664)(522.58598495,346.0045743)
\curveto(522.61598218,345.95456677)(522.63098216,345.88456684)(522.63098495,345.7945743)
\lineto(522.66098495,345.5245743)
\curveto(522.67098212,345.44456728)(522.68598211,345.36956736)(522.70598495,345.2995743)
\curveto(522.73598206,345.2295675)(522.78598201,345.19456753)(522.85598495,345.1945743)
\curveto(522.87598192,345.21456751)(522.8959819,345.2245675)(522.91598495,345.2245743)
\curveto(522.93598186,345.2245675)(522.95598184,345.23456749)(522.97598495,345.2545743)
\curveto(523.03598176,345.30456742)(523.08598171,345.35956737)(523.12598495,345.4195743)
\curveto(523.17598162,345.48956724)(523.23598156,345.54956718)(523.30598495,345.5995743)
\curveto(523.34598145,345.6295671)(523.38098141,345.65956707)(523.41098495,345.6895743)
\curveto(523.44098135,345.729567)(523.47598132,345.76456696)(523.51598495,345.7945743)
\lineto(523.78598495,345.9745743)
\curveto(523.88598091,346.03456669)(523.98598081,346.08956664)(524.08598495,346.1395743)
\curveto(524.18598061,346.17956655)(524.28598051,346.21456651)(524.38598495,346.2445743)
\lineto(524.71598495,346.3345743)
\curveto(524.74598005,346.34456638)(524.80097999,346.34456638)(524.88098495,346.3345743)
\curveto(524.97097982,346.33456639)(525.02597977,346.34456638)(525.04598495,346.3645743)
}
}
{
\newrgbcolor{curcolor}{0 0 0}
\pscustom[linestyle=none,fillstyle=solid,fillcolor=curcolor]
{
\newpath
\moveto(533.5523912,342.3745743)
\curveto(533.57238304,342.29457043)(533.57238304,342.20457052)(533.5523912,342.1045743)
\curveto(533.53238308,342.00457072)(533.49738311,341.93957079)(533.4473912,341.9095743)
\curveto(533.39738321,341.86957086)(533.32238329,341.83957089)(533.2223912,341.8195743)
\curveto(533.13238348,341.80957092)(533.02738358,341.79957093)(532.9073912,341.7895743)
\lineto(532.5623912,341.7895743)
\curveto(532.45238416,341.79957093)(532.35238426,341.80457092)(532.2623912,341.8045743)
\lineto(528.6023912,341.8045743)
\lineto(528.3923912,341.8045743)
\curveto(528.33238828,341.80457092)(528.27738833,341.79457093)(528.2273912,341.7745743)
\curveto(528.14738846,341.73457099)(528.09738851,341.69457103)(528.0773912,341.6545743)
\curveto(528.05738855,341.63457109)(528.03738857,341.59457113)(528.0173912,341.5345743)
\curveto(527.99738861,341.48457124)(527.99238862,341.43457129)(528.0023912,341.3845743)
\curveto(528.02238859,341.3245714)(528.03238858,341.26457146)(528.0323912,341.2045743)
\curveto(528.04238857,341.15457157)(528.05738855,341.09957163)(528.0773912,341.0395743)
\curveto(528.15738845,340.79957193)(528.25238836,340.59957213)(528.3623912,340.4395743)
\curveto(528.48238813,340.28957244)(528.64238797,340.15457257)(528.8423912,340.0345743)
\curveto(528.92238769,339.98457274)(529.00238761,339.94957278)(529.0823912,339.9295743)
\curveto(529.17238744,339.91957281)(529.26238735,339.89957283)(529.3523912,339.8695743)
\curveto(529.43238718,339.84957288)(529.54238707,339.83457289)(529.6823912,339.8245743)
\curveto(529.82238679,339.81457291)(529.94238667,339.81957291)(530.0423912,339.8395743)
\lineto(530.1773912,339.8395743)
\curveto(530.27738633,339.85957287)(530.36738624,339.87957285)(530.4473912,339.8995743)
\curveto(530.53738607,339.9295728)(530.62238599,339.95957277)(530.7023912,339.9895743)
\curveto(530.80238581,340.03957269)(530.9123857,340.10457262)(531.0323912,340.1845743)
\curveto(531.16238545,340.26457246)(531.25738535,340.34457238)(531.3173912,340.4245743)
\curveto(531.36738524,340.49457223)(531.41738519,340.55957217)(531.4673912,340.6195743)
\curveto(531.52738508,340.68957204)(531.59738501,340.73957199)(531.6773912,340.7695743)
\curveto(531.77738483,340.81957191)(531.90238471,340.83957189)(532.0523912,340.8295743)
\lineto(532.4873912,340.8295743)
\lineto(532.6673912,340.8295743)
\curveto(532.73738387,340.83957189)(532.79738381,340.83457189)(532.8473912,340.8145743)
\lineto(532.9973912,340.8145743)
\curveto(533.09738351,340.79457193)(533.16738344,340.76957196)(533.2073912,340.7395743)
\curveto(533.24738336,340.71957201)(533.26738334,340.67457205)(533.2673912,340.6045743)
\curveto(533.27738333,340.53457219)(533.27238334,340.47457225)(533.2523912,340.4245743)
\curveto(533.20238341,340.28457244)(533.14738346,340.15957257)(533.0873912,340.0495743)
\curveto(533.02738358,339.93957279)(532.95738365,339.8295729)(532.8773912,339.7195743)
\curveto(532.65738395,339.38957334)(532.4073842,339.1245736)(532.1273912,338.9245743)
\curveto(531.84738476,338.724574)(531.49738511,338.55457417)(531.0773912,338.4145743)
\curveto(530.96738564,338.37457435)(530.85738575,338.34957438)(530.7473912,338.3395743)
\curveto(530.63738597,338.3295744)(530.52238609,338.30957442)(530.4023912,338.2795743)
\curveto(530.36238625,338.26957446)(530.31738629,338.26957446)(530.2673912,338.2795743)
\curveto(530.22738638,338.27957445)(530.18738642,338.27457445)(530.1473912,338.2645743)
\lineto(529.9823912,338.2645743)
\curveto(529.93238668,338.24457448)(529.87238674,338.23957449)(529.8023912,338.2495743)
\curveto(529.74238687,338.24957448)(529.68738692,338.25457447)(529.6373912,338.2645743)
\curveto(529.55738705,338.27457445)(529.48738712,338.27457445)(529.4273912,338.2645743)
\curveto(529.36738724,338.25457447)(529.30238731,338.25957447)(529.2323912,338.2795743)
\curveto(529.18238743,338.29957443)(529.12738748,338.30957442)(529.0673912,338.3095743)
\curveto(529.0073876,338.30957442)(528.95238766,338.31957441)(528.9023912,338.3395743)
\curveto(528.79238782,338.35957437)(528.68238793,338.38457434)(528.5723912,338.4145743)
\curveto(528.46238815,338.43457429)(528.36238825,338.46957426)(528.2723912,338.5195743)
\curveto(528.16238845,338.55957417)(528.05738855,338.59457413)(527.9573912,338.6245743)
\curveto(527.86738874,338.66457406)(527.78238883,338.70957402)(527.7023912,338.7595743)
\curveto(527.38238923,338.95957377)(527.09738951,339.18957354)(526.8473912,339.4495743)
\curveto(526.59739001,339.71957301)(526.39239022,340.0295727)(526.2323912,340.3795743)
\curveto(526.18239043,340.48957224)(526.14239047,340.59957213)(526.1123912,340.7095743)
\curveto(526.08239053,340.8295719)(526.04239057,340.94957178)(525.9923912,341.0695743)
\curveto(525.98239063,341.10957162)(525.97739063,341.14457158)(525.9773912,341.1745743)
\curveto(525.97739063,341.21457151)(525.97239064,341.25457147)(525.9623912,341.2945743)
\curveto(525.92239069,341.41457131)(525.89739071,341.54457118)(525.8873912,341.6845743)
\lineto(525.8573912,342.1045743)
\curveto(525.85739075,342.15457057)(525.85239076,342.20957052)(525.8423912,342.2695743)
\curveto(525.84239077,342.3295704)(525.84739076,342.38457034)(525.8573912,342.4345743)
\lineto(525.8573912,342.6145743)
\lineto(525.9023912,342.9745743)
\curveto(525.94239067,343.14456958)(525.97739063,343.30956942)(526.0073912,343.4695743)
\curveto(526.03739057,343.6295691)(526.08239053,343.77956895)(526.1423912,343.9195743)
\curveto(526.57239004,344.95956777)(527.30238931,345.69456703)(528.3323912,346.1245743)
\curveto(528.47238814,346.18456654)(528.612388,346.2245665)(528.7523912,346.2445743)
\curveto(528.90238771,346.27456645)(529.05738755,346.30956642)(529.2173912,346.3495743)
\curveto(529.29738731,346.35956637)(529.37238724,346.36456636)(529.4423912,346.3645743)
\curveto(529.5123871,346.36456636)(529.58738702,346.36956636)(529.6673912,346.3795743)
\curveto(530.17738643,346.38956634)(530.612386,346.3295664)(530.9723912,346.1995743)
\curveto(531.34238527,346.07956665)(531.67238494,345.91956681)(531.9623912,345.7195743)
\curveto(532.05238456,345.65956707)(532.14238447,345.58956714)(532.2323912,345.5095743)
\curveto(532.32238429,345.43956729)(532.40238421,345.36456736)(532.4723912,345.2845743)
\curveto(532.50238411,345.23456749)(532.54238407,345.19456753)(532.5923912,345.1645743)
\curveto(532.67238394,345.05456767)(532.74738386,344.93956779)(532.8173912,344.8195743)
\curveto(532.88738372,344.70956802)(532.96238365,344.59456813)(533.0423912,344.4745743)
\curveto(533.09238352,344.38456834)(533.13238348,344.28956844)(533.1623912,344.1895743)
\curveto(533.20238341,344.09956863)(533.24238337,343.99956873)(533.2823912,343.8895743)
\curveto(533.33238328,343.75956897)(533.37238324,343.6245691)(533.4023912,343.4845743)
\curveto(533.43238318,343.34456938)(533.46738314,343.20456952)(533.5073912,343.0645743)
\curveto(533.52738308,342.98456974)(533.53238308,342.89456983)(533.5223912,342.7945743)
\curveto(533.52238309,342.70457002)(533.53238308,342.61957011)(533.5523912,342.5395743)
\lineto(533.5523912,342.3745743)
\moveto(531.3023912,343.2595743)
\curveto(531.37238524,343.35956937)(531.37738523,343.47956925)(531.3173912,343.6195743)
\curveto(531.26738534,343.76956896)(531.22738538,343.87956885)(531.1973912,343.9495743)
\curveto(531.05738555,344.21956851)(530.87238574,344.4245683)(530.6423912,344.5645743)
\curveto(530.4123862,344.71456801)(530.09238652,344.79456793)(529.6823912,344.8045743)
\curveto(529.65238696,344.78456794)(529.61738699,344.77956795)(529.5773912,344.7895743)
\curveto(529.53738707,344.79956793)(529.50238711,344.79956793)(529.4723912,344.7895743)
\curveto(529.42238719,344.76956796)(529.36738724,344.75456797)(529.3073912,344.7445743)
\curveto(529.24738736,344.74456798)(529.19238742,344.73456799)(529.1423912,344.7145743)
\curveto(528.70238791,344.57456815)(528.37738823,344.29956843)(528.1673912,343.8895743)
\curveto(528.14738846,343.84956888)(528.12238849,343.79456893)(528.0923912,343.7245743)
\curveto(528.07238854,343.66456906)(528.05738855,343.59956913)(528.0473912,343.5295743)
\curveto(528.03738857,343.46956926)(528.03738857,343.40956932)(528.0473912,343.3495743)
\curveto(528.06738854,343.28956944)(528.10238851,343.23956949)(528.1523912,343.1995743)
\curveto(528.23238838,343.14956958)(528.34238827,343.1245696)(528.4823912,343.1245743)
\lineto(528.8873912,343.1245743)
\lineto(530.5523912,343.1245743)
\lineto(530.9873912,343.1245743)
\curveto(531.14738546,343.13456959)(531.25238536,343.17956955)(531.3023912,343.2595743)
}
}
{
\newrgbcolor{curcolor}{0 0 0}
\pscustom[linestyle=none,fillstyle=solid,fillcolor=curcolor]
{
\newpath
\moveto(538.37067245,346.3795743)
\curveto(539.18066729,346.39956633)(539.85566662,346.27956645)(540.39567245,346.0195743)
\curveto(540.94566553,345.75956697)(541.38066509,345.38956734)(541.70067245,344.9095743)
\curveto(541.86066461,344.66956806)(541.98066449,344.39456833)(542.06067245,344.0845743)
\curveto(542.08066439,344.03456869)(542.09566438,343.96956876)(542.10567245,343.8895743)
\curveto(542.12566435,343.80956892)(542.12566435,343.73956899)(542.10567245,343.6795743)
\curveto(542.06566441,343.56956916)(541.99566448,343.50456922)(541.89567245,343.4845743)
\curveto(541.79566468,343.47456925)(541.6756648,343.46956926)(541.53567245,343.4695743)
\lineto(540.75567245,343.4695743)
\lineto(540.47067245,343.4695743)
\curveto(540.38066609,343.46956926)(540.30566617,343.48956924)(540.24567245,343.5295743)
\curveto(540.16566631,343.56956916)(540.11066636,343.6295691)(540.08067245,343.7095743)
\curveto(540.05066642,343.79956893)(540.01066646,343.88956884)(539.96067245,343.9795743)
\curveto(539.90066657,344.08956864)(539.83566664,344.18956854)(539.76567245,344.2795743)
\curveto(539.69566678,344.36956836)(539.61566686,344.44956828)(539.52567245,344.5195743)
\curveto(539.38566709,344.60956812)(539.23066724,344.67956805)(539.06067245,344.7295743)
\curveto(539.00066747,344.74956798)(538.94066753,344.75956797)(538.88067245,344.7595743)
\curveto(538.82066765,344.75956797)(538.76566771,344.76956796)(538.71567245,344.7895743)
\lineto(538.56567245,344.7895743)
\curveto(538.36566811,344.78956794)(538.20566827,344.76956796)(538.08567245,344.7295743)
\curveto(537.79566868,344.63956809)(537.56066891,344.49956823)(537.38067245,344.3095743)
\curveto(537.20066927,344.1295686)(537.05566942,343.90956882)(536.94567245,343.6495743)
\curveto(536.89566958,343.53956919)(536.85566962,343.41956931)(536.82567245,343.2895743)
\curveto(536.80566967,343.16956956)(536.78066969,343.03956969)(536.75067245,342.8995743)
\curveto(536.74066973,342.85956987)(536.73566974,342.81956991)(536.73567245,342.7795743)
\curveto(536.73566974,342.73956999)(536.73066974,342.69957003)(536.72067245,342.6595743)
\curveto(536.70066977,342.55957017)(536.69066978,342.41957031)(536.69067245,342.2395743)
\curveto(536.70066977,342.05957067)(536.71566976,341.91957081)(536.73567245,341.8195743)
\curveto(536.73566974,341.73957099)(536.74066973,341.68457104)(536.75067245,341.6545743)
\curveto(536.7706697,341.58457114)(536.78066969,341.51457121)(536.78067245,341.4445743)
\curveto(536.79066968,341.37457135)(536.80566967,341.30457142)(536.82567245,341.2345743)
\curveto(536.90566957,341.00457172)(537.00066947,340.79457193)(537.11067245,340.6045743)
\curveto(537.22066925,340.41457231)(537.36066911,340.25457247)(537.53067245,340.1245743)
\curveto(537.5706689,340.09457263)(537.63066884,340.05957267)(537.71067245,340.0195743)
\curveto(537.82066865,339.94957278)(537.93066854,339.90457282)(538.04067245,339.8845743)
\curveto(538.16066831,339.86457286)(538.30566817,339.84457288)(538.47567245,339.8245743)
\lineto(538.56567245,339.8245743)
\curveto(538.60566787,339.8245729)(538.63566784,339.8295729)(538.65567245,339.8395743)
\lineto(538.79067245,339.8395743)
\curveto(538.86066761,339.85957287)(538.92566755,339.87457285)(538.98567245,339.8845743)
\curveto(539.05566742,339.90457282)(539.12066735,339.9245728)(539.18067245,339.9445743)
\curveto(539.48066699,340.07457265)(539.71066676,340.26457246)(539.87067245,340.5145743)
\curveto(539.91066656,340.56457216)(539.94566653,340.61957211)(539.97567245,340.6795743)
\curveto(540.00566647,340.74957198)(540.03066644,340.80957192)(540.05067245,340.8595743)
\curveto(540.09066638,340.96957176)(540.12566635,341.06457166)(540.15567245,341.1445743)
\curveto(540.18566629,341.23457149)(540.25566622,341.30457142)(540.36567245,341.3545743)
\curveto(540.45566602,341.39457133)(540.60066587,341.40957132)(540.80067245,341.3995743)
\lineto(541.29567245,341.3995743)
\lineto(541.50567245,341.3995743)
\curveto(541.58566489,341.40957132)(541.65066482,341.40457132)(541.70067245,341.3845743)
\lineto(541.82067245,341.3845743)
\lineto(541.94067245,341.3545743)
\curveto(541.98066449,341.35457137)(542.01066446,341.34457138)(542.03067245,341.3245743)
\curveto(542.08066439,341.28457144)(542.11066436,341.2245715)(542.12067245,341.1445743)
\curveto(542.14066433,341.07457165)(542.14066433,340.99957173)(542.12067245,340.9195743)
\curveto(542.03066444,340.58957214)(541.92066455,340.29457243)(541.79067245,340.0345743)
\curveto(541.38066509,339.26457346)(540.72566575,338.729574)(539.82567245,338.4295743)
\curveto(539.72566675,338.39957433)(539.62066685,338.37957435)(539.51067245,338.3695743)
\curveto(539.40066707,338.34957438)(539.29066718,338.3245744)(539.18067245,338.2945743)
\curveto(539.12066735,338.28457444)(539.06066741,338.27957445)(539.00067245,338.2795743)
\curveto(538.94066753,338.27957445)(538.88066759,338.27457445)(538.82067245,338.2645743)
\lineto(538.65567245,338.2645743)
\curveto(538.60566787,338.24457448)(538.53066794,338.23957449)(538.43067245,338.2495743)
\curveto(538.33066814,338.24957448)(538.25566822,338.25457447)(538.20567245,338.2645743)
\curveto(538.12566835,338.28457444)(538.05066842,338.29457443)(537.98067245,338.2945743)
\curveto(537.92066855,338.28457444)(537.85566862,338.28957444)(537.78567245,338.3095743)
\lineto(537.63567245,338.3395743)
\curveto(537.58566889,338.33957439)(537.53566894,338.34457438)(537.48567245,338.3545743)
\curveto(537.3756691,338.38457434)(537.2706692,338.41457431)(537.17067245,338.4445743)
\curveto(537.0706694,338.47457425)(536.9756695,338.50957422)(536.88567245,338.5495743)
\curveto(536.41567006,338.74957398)(536.02067045,339.00457372)(535.70067245,339.3145743)
\curveto(535.38067109,339.63457309)(535.12067135,340.0295727)(534.92067245,340.4995743)
\curveto(534.8706716,340.58957214)(534.83067164,340.68457204)(534.80067245,340.7845743)
\lineto(534.71067245,341.1145743)
\curveto(534.70067177,341.15457157)(534.69567178,341.18957154)(534.69567245,341.2195743)
\curveto(534.69567178,341.25957147)(534.68567179,341.30457142)(534.66567245,341.3545743)
\curveto(534.64567183,341.4245713)(534.63567184,341.49457123)(534.63567245,341.5645743)
\curveto(534.63567184,341.64457108)(534.62567185,341.71957101)(534.60567245,341.7895743)
\lineto(534.60567245,342.0445743)
\curveto(534.58567189,342.09457063)(534.5756719,342.14957058)(534.57567245,342.2095743)
\curveto(534.5756719,342.27957045)(534.58567189,342.33957039)(534.60567245,342.3895743)
\curveto(534.61567186,342.43957029)(534.61567186,342.48457024)(534.60567245,342.5245743)
\curveto(534.59567188,342.56457016)(534.59567188,342.60457012)(534.60567245,342.6445743)
\curveto(534.62567185,342.71457001)(534.63067184,342.77956995)(534.62067245,342.8395743)
\curveto(534.62067185,342.89956983)(534.63067184,342.95956977)(534.65067245,343.0195743)
\curveto(534.70067177,343.19956953)(534.74067173,343.36956936)(534.77067245,343.5295743)
\curveto(534.80067167,343.69956903)(534.84567163,343.86456886)(534.90567245,344.0245743)
\curveto(535.12567135,344.53456819)(535.40067107,344.95956777)(535.73067245,345.2995743)
\curveto(536.0706704,345.63956709)(536.50066997,345.91456681)(537.02067245,346.1245743)
\curveto(537.16066931,346.18456654)(537.30566917,346.2245665)(537.45567245,346.2445743)
\curveto(537.60566887,346.27456645)(537.76066871,346.30956642)(537.92067245,346.3495743)
\curveto(538.00066847,346.35956637)(538.0756684,346.36456636)(538.14567245,346.3645743)
\curveto(538.21566826,346.36456636)(538.29066818,346.36956636)(538.37067245,346.3795743)
}
}
{
\newrgbcolor{curcolor}{0 0 0}
\pscustom[linestyle=none,fillstyle=solid,fillcolor=curcolor]
{
\newpath
\moveto(543.8339537,346.1545743)
\lineto(544.9589537,346.1545743)
\curveto(545.06895127,346.15456657)(545.16895117,346.14956658)(545.2589537,346.1395743)
\curveto(545.34895099,346.1295666)(545.41395092,346.09456663)(545.4539537,346.0345743)
\curveto(545.50395083,345.97456675)(545.5339508,345.88956684)(545.5439537,345.7795743)
\curveto(545.55395078,345.67956705)(545.55895078,345.57456715)(545.5589537,345.4645743)
\lineto(545.5589537,344.4145743)
\lineto(545.5589537,342.1795743)
\curveto(545.55895078,341.81957091)(545.57395076,341.47957125)(545.6039537,341.1595743)
\curveto(545.6339507,340.83957189)(545.72395061,340.57457215)(545.8739537,340.3645743)
\curveto(546.01395032,340.15457257)(546.2389501,340.00457272)(546.5489537,339.9145743)
\curveto(546.59894974,339.90457282)(546.6389497,339.89957283)(546.6689537,339.8995743)
\curveto(546.70894963,339.89957283)(546.75394958,339.89457283)(546.8039537,339.8845743)
\curveto(546.85394948,339.87457285)(546.90894943,339.86957286)(546.9689537,339.8695743)
\curveto(547.02894931,339.86957286)(547.07394926,339.87457285)(547.1039537,339.8845743)
\curveto(547.15394918,339.90457282)(547.19394914,339.90957282)(547.2239537,339.8995743)
\curveto(547.26394907,339.88957284)(547.30394903,339.89457283)(547.3439537,339.9145743)
\curveto(547.55394878,339.96457276)(547.71894862,340.0295727)(547.8389537,340.1095743)
\curveto(548.01894832,340.21957251)(548.15894818,340.35957237)(548.2589537,340.5295743)
\curveto(548.36894797,340.70957202)(548.44394789,340.90457182)(548.4839537,341.1145743)
\curveto(548.5339478,341.33457139)(548.56394777,341.57457115)(548.5739537,341.8345743)
\curveto(548.58394775,342.10457062)(548.58894775,342.38457034)(548.5889537,342.6745743)
\lineto(548.5889537,344.4895743)
\lineto(548.5889537,345.4645743)
\lineto(548.5889537,345.7345743)
\curveto(548.58894775,345.83456689)(548.60894773,345.91456681)(548.6489537,345.9745743)
\curveto(548.69894764,346.06456666)(548.77394756,346.11456661)(548.8739537,346.1245743)
\curveto(548.97394736,346.14456658)(549.09394724,346.15456657)(549.2339537,346.1545743)
\lineto(550.0289537,346.1545743)
\lineto(550.3139537,346.1545743)
\curveto(550.40394593,346.15456657)(550.47894586,346.13456659)(550.5389537,346.0945743)
\curveto(550.61894572,346.04456668)(550.66394567,345.96956676)(550.6739537,345.8695743)
\curveto(550.68394565,345.76956696)(550.68894565,345.65456707)(550.6889537,345.5245743)
\lineto(550.6889537,344.3845743)
\lineto(550.6889537,340.1695743)
\lineto(550.6889537,339.1045743)
\lineto(550.6889537,338.8045743)
\curveto(550.68894565,338.70457402)(550.66894567,338.6295741)(550.6289537,338.5795743)
\curveto(550.57894576,338.49957423)(550.50394583,338.45457427)(550.4039537,338.4445743)
\curveto(550.30394603,338.43457429)(550.19894614,338.4295743)(550.0889537,338.4295743)
\lineto(549.2789537,338.4295743)
\curveto(549.16894717,338.4295743)(549.06894727,338.43457429)(548.9789537,338.4445743)
\curveto(548.89894744,338.45457427)(548.8339475,338.49457423)(548.7839537,338.5645743)
\curveto(548.76394757,338.59457413)(548.74394759,338.63957409)(548.7239537,338.6995743)
\curveto(548.71394762,338.75957397)(548.69894764,338.81957391)(548.6789537,338.8795743)
\curveto(548.66894767,338.93957379)(548.65394768,338.99457373)(548.6339537,339.0445743)
\curveto(548.61394772,339.09457363)(548.58394775,339.1245736)(548.5439537,339.1345743)
\curveto(548.52394781,339.15457357)(548.49894784,339.15957357)(548.4689537,339.1495743)
\curveto(548.4389479,339.13957359)(548.41394792,339.1295736)(548.3939537,339.1195743)
\curveto(548.32394801,339.07957365)(548.26394807,339.03457369)(548.2139537,338.9845743)
\curveto(548.16394817,338.93457379)(548.10894823,338.88957384)(548.0489537,338.8495743)
\curveto(548.00894833,338.81957391)(547.96894837,338.78457394)(547.9289537,338.7445743)
\curveto(547.89894844,338.71457401)(547.85894848,338.68457404)(547.8089537,338.6545743)
\curveto(547.57894876,338.51457421)(547.30894903,338.40457432)(546.9989537,338.3245743)
\curveto(546.92894941,338.30457442)(546.85894948,338.29457443)(546.7889537,338.2945743)
\curveto(546.71894962,338.28457444)(546.64394969,338.26957446)(546.5639537,338.2495743)
\curveto(546.52394981,338.23957449)(546.47894986,338.23957449)(546.4289537,338.2495743)
\curveto(546.38894995,338.24957448)(546.34894999,338.24457448)(546.3089537,338.2345743)
\curveto(546.27895006,338.2245745)(546.21395012,338.2245745)(546.1139537,338.2345743)
\curveto(546.02395031,338.23457449)(545.96395037,338.23957449)(545.9339537,338.2495743)
\curveto(545.88395045,338.24957448)(545.8339505,338.25457447)(545.7839537,338.2645743)
\lineto(545.6339537,338.2645743)
\curveto(545.51395082,338.29457443)(545.39895094,338.31957441)(545.2889537,338.3395743)
\curveto(545.17895116,338.35957437)(545.06895127,338.38957434)(544.9589537,338.4295743)
\curveto(544.90895143,338.44957428)(544.86395147,338.46457426)(544.8239537,338.4745743)
\curveto(544.79395154,338.49457423)(544.75395158,338.51457421)(544.7039537,338.5345743)
\curveto(544.35395198,338.724574)(544.07395226,338.98957374)(543.8639537,339.3295743)
\curveto(543.7339526,339.53957319)(543.6389527,339.78957294)(543.5789537,340.0795743)
\curveto(543.51895282,340.37957235)(543.47895286,340.69457203)(543.4589537,341.0245743)
\curveto(543.44895289,341.36457136)(543.44395289,341.70957102)(543.4439537,342.0595743)
\curveto(543.45395288,342.41957031)(543.45895288,342.77456995)(543.4589537,343.1245743)
\lineto(543.4589537,345.1645743)
\curveto(543.45895288,345.29456743)(543.45395288,345.44456728)(543.4439537,345.6145743)
\curveto(543.44395289,345.79456693)(543.46895287,345.9245668)(543.5189537,346.0045743)
\curveto(543.54895279,346.05456667)(543.60895273,346.09956663)(543.6989537,346.1395743)
\curveto(543.75895258,346.13956659)(543.80395253,346.14456658)(543.8339537,346.1545743)
}
}
{
\newrgbcolor{curcolor}{0 0 0}
\pscustom[linestyle=none,fillstyle=solid,fillcolor=curcolor]
{
\newpath
\moveto(556.7452037,346.3645743)
\curveto(556.85519839,346.36456636)(556.95019829,346.35456637)(557.0302037,346.3345743)
\curveto(557.12019812,346.31456641)(557.19019805,346.26956646)(557.2402037,346.1995743)
\curveto(557.30019794,346.11956661)(557.33019791,345.97956675)(557.3302037,345.7795743)
\lineto(557.3302037,345.2695743)
\lineto(557.3302037,344.8945743)
\curveto(557.3401979,344.75456797)(557.32519792,344.64456808)(557.2852037,344.5645743)
\curveto(557.245198,344.49456823)(557.18519806,344.44956828)(557.1052037,344.4295743)
\curveto(557.03519821,344.40956832)(556.95019829,344.39956833)(556.8502037,344.3995743)
\curveto(556.76019848,344.39956833)(556.66019858,344.40456832)(556.5502037,344.4145743)
\curveto(556.45019879,344.4245683)(556.35519889,344.41956831)(556.2652037,344.3995743)
\curveto(556.19519905,344.37956835)(556.12519912,344.36456836)(556.0552037,344.3545743)
\curveto(555.98519926,344.35456837)(555.92019932,344.34456838)(555.8602037,344.3245743)
\curveto(555.70019954,344.27456845)(555.5401997,344.19956853)(555.3802037,344.0995743)
\curveto(555.22020002,344.00956872)(555.09520015,343.90456882)(555.0052037,343.7845743)
\curveto(554.95520029,343.70456902)(554.90020034,343.61956911)(554.8402037,343.5295743)
\curveto(554.79020045,343.44956928)(554.7402005,343.36456936)(554.6902037,343.2745743)
\curveto(554.66020058,343.19456953)(554.63020061,343.10956962)(554.6002037,343.0195743)
\lineto(554.5402037,342.7795743)
\curveto(554.52020072,342.70957002)(554.51020073,342.63457009)(554.5102037,342.5545743)
\curveto(554.51020073,342.48457024)(554.50020074,342.41457031)(554.4802037,342.3445743)
\curveto(554.47020077,342.30457042)(554.46520078,342.26457046)(554.4652037,342.2245743)
\curveto(554.47520077,342.19457053)(554.47520077,342.16457056)(554.4652037,342.1345743)
\lineto(554.4652037,341.8945743)
\curveto(554.4452008,341.8245709)(554.4402008,341.74457098)(554.4502037,341.6545743)
\curveto(554.46020078,341.57457115)(554.46520078,341.49457123)(554.4652037,341.4145743)
\lineto(554.4652037,340.4545743)
\lineto(554.4652037,339.1795743)
\curveto(554.46520078,339.04957368)(554.46020078,338.9295738)(554.4502037,338.8195743)
\curveto(554.4402008,338.70957402)(554.41020083,338.61957411)(554.3602037,338.5495743)
\curveto(554.3402009,338.51957421)(554.30520094,338.49457423)(554.2552037,338.4745743)
\curveto(554.21520103,338.46457426)(554.17020107,338.45457427)(554.1202037,338.4445743)
\lineto(554.0452037,338.4445743)
\curveto(553.99520125,338.43457429)(553.9402013,338.4295743)(553.8802037,338.4295743)
\lineto(553.7152037,338.4295743)
\lineto(553.0702037,338.4295743)
\curveto(553.01020223,338.43957429)(552.9452023,338.44457428)(552.8752037,338.4445743)
\lineto(552.6802037,338.4445743)
\curveto(552.63020261,338.46457426)(552.58020266,338.47957425)(552.5302037,338.4895743)
\curveto(552.48020276,338.50957422)(552.4452028,338.54457418)(552.4252037,338.5945743)
\curveto(552.38520286,338.64457408)(552.36020288,338.71457401)(552.3502037,338.8045743)
\lineto(552.3502037,339.1045743)
\lineto(552.3502037,340.1245743)
\lineto(552.3502037,344.3545743)
\lineto(552.3502037,345.4645743)
\lineto(552.3502037,345.7495743)
\curveto(552.35020289,345.84956688)(552.37020287,345.9295668)(552.4102037,345.9895743)
\curveto(552.46020278,346.06956666)(552.53520271,346.11956661)(552.6352037,346.1395743)
\curveto(552.73520251,346.15956657)(552.85520239,346.16956656)(552.9952037,346.1695743)
\lineto(553.7602037,346.1695743)
\curveto(553.88020136,346.16956656)(553.98520126,346.15956657)(554.0752037,346.1395743)
\curveto(554.16520108,346.1295666)(554.23520101,346.08456664)(554.2852037,346.0045743)
\curveto(554.31520093,345.95456677)(554.33020091,345.88456684)(554.3302037,345.7945743)
\lineto(554.3602037,345.5245743)
\curveto(554.37020087,345.44456728)(554.38520086,345.36956736)(554.4052037,345.2995743)
\curveto(554.43520081,345.2295675)(554.48520076,345.19456753)(554.5552037,345.1945743)
\curveto(554.57520067,345.21456751)(554.59520065,345.2245675)(554.6152037,345.2245743)
\curveto(554.63520061,345.2245675)(554.65520059,345.23456749)(554.6752037,345.2545743)
\curveto(554.73520051,345.30456742)(554.78520046,345.35956737)(554.8252037,345.4195743)
\curveto(554.87520037,345.48956724)(554.93520031,345.54956718)(555.0052037,345.5995743)
\curveto(555.0452002,345.6295671)(555.08020016,345.65956707)(555.1102037,345.6895743)
\curveto(555.1402001,345.729567)(555.17520007,345.76456696)(555.2152037,345.7945743)
\lineto(555.4852037,345.9745743)
\curveto(555.58519966,346.03456669)(555.68519956,346.08956664)(555.7852037,346.1395743)
\curveto(555.88519936,346.17956655)(555.98519926,346.21456651)(556.0852037,346.2445743)
\lineto(556.4152037,346.3345743)
\curveto(556.4451988,346.34456638)(556.50019874,346.34456638)(556.5802037,346.3345743)
\curveto(556.67019857,346.33456639)(556.72519852,346.34456638)(556.7452037,346.3645743)
}
}
{
\newrgbcolor{curcolor}{0 0 0}
\pscustom[linestyle=none,fillstyle=solid,fillcolor=curcolor]
{
\newpath
\moveto(561.12028183,346.3795743)
\curveto(561.87027733,346.39956633)(562.52027668,346.31456641)(563.07028183,346.1245743)
\curveto(563.63027557,345.94456678)(564.05527514,345.6295671)(564.34528183,345.1795743)
\curveto(564.41527478,345.06956766)(564.47527472,344.95456777)(564.52528183,344.8345743)
\curveto(564.58527461,344.724568)(564.63527456,344.59956813)(564.67528183,344.4595743)
\curveto(564.6952745,344.39956833)(564.70527449,344.33456839)(564.70528183,344.2645743)
\curveto(564.70527449,344.19456853)(564.6952745,344.13456859)(564.67528183,344.0845743)
\curveto(564.63527456,344.0245687)(564.58027462,343.98456874)(564.51028183,343.9645743)
\curveto(564.46027474,343.94456878)(564.4002748,343.93456879)(564.33028183,343.9345743)
\lineto(564.12028183,343.9345743)
\lineto(563.46028183,343.9345743)
\curveto(563.39027581,343.93456879)(563.32027588,343.9295688)(563.25028183,343.9195743)
\curveto(563.18027602,343.91956881)(563.11527608,343.9295688)(563.05528183,343.9495743)
\curveto(562.95527624,343.96956876)(562.88027632,344.00956872)(562.83028183,344.0695743)
\curveto(562.78027642,344.1295686)(562.73527646,344.18956854)(562.69528183,344.2495743)
\lineto(562.57528183,344.4595743)
\curveto(562.54527665,344.53956819)(562.4952767,344.60456812)(562.42528183,344.6545743)
\curveto(562.32527687,344.73456799)(562.22527697,344.79456793)(562.12528183,344.8345743)
\curveto(562.03527716,344.87456785)(561.92027728,344.90956782)(561.78028183,344.9395743)
\curveto(561.71027749,344.95956777)(561.60527759,344.97456775)(561.46528183,344.9845743)
\curveto(561.33527786,344.99456773)(561.23527796,344.98956774)(561.16528183,344.9695743)
\lineto(561.06028183,344.9695743)
\lineto(560.91028183,344.9395743)
\curveto(560.87027833,344.93956779)(560.82527837,344.93456779)(560.77528183,344.9245743)
\curveto(560.60527859,344.87456785)(560.46527873,344.80456792)(560.35528183,344.7145743)
\curveto(560.25527894,344.63456809)(560.18527901,344.50956822)(560.14528183,344.3395743)
\curveto(560.12527907,344.26956846)(560.12527907,344.20456852)(560.14528183,344.1445743)
\curveto(560.16527903,344.08456864)(560.18527901,344.03456869)(560.20528183,343.9945743)
\curveto(560.27527892,343.87456885)(560.35527884,343.77956895)(560.44528183,343.7095743)
\curveto(560.54527865,343.63956909)(560.66027854,343.57956915)(560.79028183,343.5295743)
\curveto(560.98027822,343.44956928)(561.18527801,343.37956935)(561.40528183,343.3195743)
\lineto(562.09528183,343.1695743)
\curveto(562.33527686,343.1295696)(562.56527663,343.07956965)(562.78528183,343.0195743)
\curveto(563.01527618,342.96956976)(563.23027597,342.90456982)(563.43028183,342.8245743)
\curveto(563.52027568,342.78456994)(563.60527559,342.74956998)(563.68528183,342.7195743)
\curveto(563.77527542,342.69957003)(563.86027534,342.66457006)(563.94028183,342.6145743)
\curveto(564.13027507,342.49457023)(564.3002749,342.36457036)(564.45028183,342.2245743)
\curveto(564.61027459,342.08457064)(564.73527446,341.90957082)(564.82528183,341.6995743)
\curveto(564.85527434,341.6295711)(564.88027432,341.55957117)(564.90028183,341.4895743)
\curveto(564.92027428,341.41957131)(564.94027426,341.34457138)(564.96028183,341.2645743)
\curveto(564.97027423,341.20457152)(564.97527422,341.10957162)(564.97528183,340.9795743)
\curveto(564.98527421,340.85957187)(564.98527421,340.76457196)(564.97528183,340.6945743)
\lineto(564.97528183,340.6195743)
\curveto(564.95527424,340.55957217)(564.94027426,340.49957223)(564.93028183,340.4395743)
\curveto(564.93027427,340.38957234)(564.92527427,340.33957239)(564.91528183,340.2895743)
\curveto(564.84527435,339.98957274)(564.73527446,339.724573)(564.58528183,339.4945743)
\curveto(564.42527477,339.25457347)(564.23027497,339.05957367)(564.00028183,338.9095743)
\curveto(563.77027543,338.75957397)(563.51027569,338.6295741)(563.22028183,338.5195743)
\curveto(563.11027609,338.46957426)(562.99027621,338.43457429)(562.86028183,338.4145743)
\curveto(562.74027646,338.39457433)(562.62027658,338.36957436)(562.50028183,338.3395743)
\curveto(562.41027679,338.31957441)(562.31527688,338.30957442)(562.21528183,338.3095743)
\curveto(562.12527707,338.29957443)(562.03527716,338.28457444)(561.94528183,338.2645743)
\lineto(561.67528183,338.2645743)
\curveto(561.61527758,338.24457448)(561.51027769,338.23457449)(561.36028183,338.2345743)
\curveto(561.22027798,338.23457449)(561.12027808,338.24457448)(561.06028183,338.2645743)
\curveto(561.03027817,338.26457446)(560.9952782,338.26957446)(560.95528183,338.2795743)
\lineto(560.85028183,338.2795743)
\curveto(560.73027847,338.29957443)(560.61027859,338.31457441)(560.49028183,338.3245743)
\curveto(560.37027883,338.33457439)(560.25527894,338.35457437)(560.14528183,338.3845743)
\curveto(559.75527944,338.49457423)(559.41027979,338.61957411)(559.11028183,338.7595743)
\curveto(558.81028039,338.90957382)(558.55528064,339.1295736)(558.34528183,339.4195743)
\curveto(558.20528099,339.60957312)(558.08528111,339.8295729)(557.98528183,340.0795743)
\curveto(557.96528123,340.13957259)(557.94528125,340.21957251)(557.92528183,340.3195743)
\curveto(557.90528129,340.36957236)(557.89028131,340.43957229)(557.88028183,340.5295743)
\curveto(557.87028133,340.61957211)(557.87528132,340.69457203)(557.89528183,340.7545743)
\curveto(557.92528127,340.8245719)(557.97528122,340.87457185)(558.04528183,340.9045743)
\curveto(558.0952811,340.9245718)(558.15528104,340.93457179)(558.22528183,340.9345743)
\lineto(558.45028183,340.9345743)
\lineto(559.15528183,340.9345743)
\lineto(559.39528183,340.9345743)
\curveto(559.47527972,340.93457179)(559.54527965,340.9245718)(559.60528183,340.9045743)
\curveto(559.71527948,340.86457186)(559.78527941,340.79957193)(559.81528183,340.7095743)
\curveto(559.85527934,340.61957211)(559.9002793,340.5245722)(559.95028183,340.4245743)
\curveto(559.97027923,340.37457235)(560.00527919,340.30957242)(560.05528183,340.2295743)
\curveto(560.11527908,340.14957258)(560.16527903,340.09957263)(560.20528183,340.0795743)
\curveto(560.32527887,339.97957275)(560.44027876,339.89957283)(560.55028183,339.8395743)
\curveto(560.66027854,339.78957294)(560.8002784,339.73957299)(560.97028183,339.6895743)
\curveto(561.02027818,339.66957306)(561.07027813,339.65957307)(561.12028183,339.6595743)
\curveto(561.17027803,339.66957306)(561.22027798,339.66957306)(561.27028183,339.6595743)
\curveto(561.35027785,339.63957309)(561.43527776,339.6295731)(561.52528183,339.6295743)
\curveto(561.62527757,339.63957309)(561.71027749,339.65457307)(561.78028183,339.6745743)
\curveto(561.83027737,339.68457304)(561.87527732,339.68957304)(561.91528183,339.6895743)
\curveto(561.96527723,339.68957304)(562.01527718,339.69957303)(562.06528183,339.7195743)
\curveto(562.20527699,339.76957296)(562.33027687,339.8295729)(562.44028183,339.8995743)
\curveto(562.56027664,339.96957276)(562.65527654,340.05957267)(562.72528183,340.1695743)
\curveto(562.77527642,340.24957248)(562.81527638,340.37457235)(562.84528183,340.5445743)
\curveto(562.86527633,340.61457211)(562.86527633,340.67957205)(562.84528183,340.7395743)
\curveto(562.82527637,340.79957193)(562.80527639,340.84957188)(562.78528183,340.8895743)
\curveto(562.71527648,341.0295717)(562.62527657,341.13457159)(562.51528183,341.2045743)
\curveto(562.41527678,341.27457145)(562.2952769,341.33957139)(562.15528183,341.3995743)
\curveto(561.96527723,341.47957125)(561.76527743,341.54457118)(561.55528183,341.5945743)
\curveto(561.34527785,341.64457108)(561.13527806,341.69957103)(560.92528183,341.7595743)
\curveto(560.84527835,341.77957095)(560.76027844,341.79457093)(560.67028183,341.8045743)
\curveto(560.59027861,341.81457091)(560.51027869,341.8295709)(560.43028183,341.8495743)
\curveto(560.11027909,341.93957079)(559.80527939,342.0245707)(559.51528183,342.1045743)
\curveto(559.22527997,342.19457053)(558.96028024,342.3245704)(558.72028183,342.4945743)
\curveto(558.44028076,342.69457003)(558.23528096,342.96456976)(558.10528183,343.3045743)
\curveto(558.08528111,343.37456935)(558.06528113,343.46956926)(558.04528183,343.5895743)
\curveto(558.02528117,343.65956907)(558.01028119,343.74456898)(558.00028183,343.8445743)
\curveto(557.99028121,343.94456878)(557.9952812,344.03456869)(558.01528183,344.1145743)
\curveto(558.03528116,344.16456856)(558.04028116,344.20456852)(558.03028183,344.2345743)
\curveto(558.02028118,344.27456845)(558.02528117,344.31956841)(558.04528183,344.3695743)
\curveto(558.06528113,344.47956825)(558.08528111,344.57956815)(558.10528183,344.6695743)
\curveto(558.13528106,344.76956796)(558.17028103,344.86456786)(558.21028183,344.9545743)
\curveto(558.34028086,345.24456748)(558.52028068,345.47956725)(558.75028183,345.6595743)
\curveto(558.98028022,345.83956689)(559.24027996,345.98456674)(559.53028183,346.0945743)
\curveto(559.64027956,346.14456658)(559.75527944,346.17956655)(559.87528183,346.1995743)
\curveto(559.9952792,346.2295665)(560.12027908,346.25956647)(560.25028183,346.2895743)
\curveto(560.31027889,346.30956642)(560.37027883,346.31956641)(560.43028183,346.3195743)
\lineto(560.61028183,346.3495743)
\curveto(560.69027851,346.35956637)(560.77527842,346.36456636)(560.86528183,346.3645743)
\curveto(560.95527824,346.36456636)(561.04027816,346.36956636)(561.12028183,346.3795743)
}
}
{
\newrgbcolor{curcolor}{0 0 0}
\pscustom[linestyle=none,fillstyle=solid,fillcolor=curcolor]
{
\newpath
\moveto(573.97692245,342.6145743)
\curveto(573.99691388,342.55457017)(574.00691387,342.46957026)(574.00692245,342.3595743)
\curveto(574.00691387,342.24957048)(573.99691388,342.16457056)(573.97692245,342.1045743)
\lineto(573.97692245,341.9545743)
\curveto(573.95691392,341.87457085)(573.94691393,341.79457093)(573.94692245,341.7145743)
\curveto(573.95691392,341.63457109)(573.95191393,341.55457117)(573.93192245,341.4745743)
\curveto(573.91191397,341.40457132)(573.89691398,341.33957139)(573.88692245,341.2795743)
\curveto(573.876914,341.21957151)(573.86691401,341.15457157)(573.85692245,341.0845743)
\curveto(573.81691406,340.97457175)(573.7819141,340.85957187)(573.75192245,340.7395743)
\curveto(573.72191416,340.6295721)(573.6819142,340.5245722)(573.63192245,340.4245743)
\curveto(573.42191446,339.94457278)(573.14691473,339.55457317)(572.80692245,339.2545743)
\curveto(572.46691541,338.95457377)(572.05691582,338.70457402)(571.57692245,338.5045743)
\curveto(571.45691642,338.45457427)(571.33191655,338.41957431)(571.20192245,338.3995743)
\curveto(571.0819168,338.36957436)(570.95691692,338.33957439)(570.82692245,338.3095743)
\curveto(570.7769171,338.28957444)(570.72191716,338.27957445)(570.66192245,338.2795743)
\curveto(570.60191728,338.27957445)(570.54691733,338.27457445)(570.49692245,338.2645743)
\lineto(570.39192245,338.2645743)
\curveto(570.36191752,338.25457447)(570.33191755,338.24957448)(570.30192245,338.2495743)
\curveto(570.25191763,338.23957449)(570.17191771,338.23457449)(570.06192245,338.2345743)
\curveto(569.95191793,338.2245745)(569.86691801,338.2295745)(569.80692245,338.2495743)
\lineto(569.65692245,338.2495743)
\curveto(569.60691827,338.25957447)(569.55191833,338.26457446)(569.49192245,338.2645743)
\curveto(569.44191844,338.25457447)(569.39191849,338.25957447)(569.34192245,338.2795743)
\curveto(569.30191858,338.28957444)(569.26191862,338.29457443)(569.22192245,338.2945743)
\curveto(569.19191869,338.29457443)(569.15191873,338.29957443)(569.10192245,338.3095743)
\curveto(569.00191888,338.33957439)(568.90191898,338.36457436)(568.80192245,338.3845743)
\curveto(568.70191918,338.40457432)(568.60691927,338.43457429)(568.51692245,338.4745743)
\curveto(568.39691948,338.51457421)(568.2819196,338.55457417)(568.17192245,338.5945743)
\curveto(568.07191981,338.63457409)(567.96691991,338.68457404)(567.85692245,338.7445743)
\curveto(567.50692037,338.95457377)(567.20692067,339.19957353)(566.95692245,339.4795743)
\curveto(566.70692117,339.75957297)(566.49692138,340.09457263)(566.32692245,340.4845743)
\curveto(566.2769216,340.57457215)(566.23692164,340.66957206)(566.20692245,340.7695743)
\curveto(566.18692169,340.86957186)(566.16192172,340.97457175)(566.13192245,341.0845743)
\curveto(566.11192177,341.13457159)(566.10192178,341.17957155)(566.10192245,341.2195743)
\curveto(566.10192178,341.25957147)(566.09192179,341.30457142)(566.07192245,341.3545743)
\curveto(566.05192183,341.43457129)(566.04192184,341.51457121)(566.04192245,341.5945743)
\curveto(566.04192184,341.68457104)(566.03192185,341.76957096)(566.01192245,341.8495743)
\curveto(566.00192188,341.89957083)(565.99692188,341.94457078)(565.99692245,341.9845743)
\lineto(565.99692245,342.1195743)
\curveto(565.9769219,342.17957055)(565.96692191,342.26457046)(565.96692245,342.3745743)
\curveto(565.9769219,342.48457024)(565.99192189,342.56957016)(566.01192245,342.6295743)
\lineto(566.01192245,342.7345743)
\curveto(566.02192186,342.78456994)(566.02192186,342.83456989)(566.01192245,342.8845743)
\curveto(566.01192187,342.94456978)(566.02192186,342.99956973)(566.04192245,343.0495743)
\curveto(566.05192183,343.09956963)(566.05692182,343.14456958)(566.05692245,343.1845743)
\curveto(566.05692182,343.23456949)(566.06692181,343.28456944)(566.08692245,343.3345743)
\curveto(566.12692175,343.46456926)(566.16192172,343.58956914)(566.19192245,343.7095743)
\curveto(566.22192166,343.83956889)(566.26192162,343.96456876)(566.31192245,344.0845743)
\curveto(566.49192139,344.49456823)(566.70692117,344.83456789)(566.95692245,345.1045743)
\curveto(567.20692067,345.38456734)(567.51192037,345.63956709)(567.87192245,345.8695743)
\curveto(567.97191991,345.91956681)(568.0769198,345.96456676)(568.18692245,346.0045743)
\curveto(568.29691958,346.04456668)(568.40691947,346.08956664)(568.51692245,346.1395743)
\curveto(568.64691923,346.18956654)(568.7819191,346.2245665)(568.92192245,346.2445743)
\curveto(569.06191882,346.26456646)(569.20691867,346.29456643)(569.35692245,346.3345743)
\curveto(569.43691844,346.34456638)(569.51191837,346.34956638)(569.58192245,346.3495743)
\curveto(569.65191823,346.34956638)(569.72191816,346.35456637)(569.79192245,346.3645743)
\curveto(570.37191751,346.37456635)(570.87191701,346.31456641)(571.29192245,346.1845743)
\curveto(571.72191616,346.05456667)(572.10191578,345.87456685)(572.43192245,345.6445743)
\curveto(572.54191534,345.56456716)(572.65191523,345.47456725)(572.76192245,345.3745743)
\curveto(572.881915,345.28456744)(572.9819149,345.18456754)(573.06192245,345.0745743)
\curveto(573.14191474,344.97456775)(573.21191467,344.87456785)(573.27192245,344.7745743)
\curveto(573.34191454,344.67456805)(573.41191447,344.56956816)(573.48192245,344.4595743)
\curveto(573.55191433,344.34956838)(573.60691427,344.2295685)(573.64692245,344.0995743)
\curveto(573.68691419,343.97956875)(573.73191415,343.84956888)(573.78192245,343.7095743)
\curveto(573.81191407,343.6295691)(573.83691404,343.54456918)(573.85692245,343.4545743)
\lineto(573.91692245,343.1845743)
\curveto(573.92691395,343.14456958)(573.93191395,343.10456962)(573.93192245,343.0645743)
\curveto(573.93191395,343.0245697)(573.93691394,342.98456974)(573.94692245,342.9445743)
\curveto(573.96691391,342.89456983)(573.97191391,342.83956989)(573.96192245,342.7795743)
\curveto(573.95191393,342.71957001)(573.95691392,342.66457006)(573.97692245,342.6145743)
\moveto(571.87692245,342.0745743)
\curveto(571.88691599,342.1245706)(571.89191599,342.19457053)(571.89192245,342.2845743)
\curveto(571.89191599,342.38457034)(571.88691599,342.45957027)(571.87692245,342.5095743)
\lineto(571.87692245,342.6295743)
\curveto(571.85691602,342.67957005)(571.84691603,342.73456999)(571.84692245,342.7945743)
\curveto(571.84691603,342.85456987)(571.84191604,342.90956982)(571.83192245,342.9595743)
\curveto(571.83191605,342.99956973)(571.82691605,343.0295697)(571.81692245,343.0495743)
\lineto(571.75692245,343.2895743)
\curveto(571.74691613,343.37956935)(571.72691615,343.46456926)(571.69692245,343.5445743)
\curveto(571.58691629,343.80456892)(571.45691642,344.0245687)(571.30692245,344.2045743)
\curveto(571.15691672,344.39456833)(570.95691692,344.54456818)(570.70692245,344.6545743)
\curveto(570.64691723,344.67456805)(570.58691729,344.68956804)(570.52692245,344.6995743)
\curveto(570.46691741,344.71956801)(570.40191748,344.73956799)(570.33192245,344.7595743)
\curveto(570.25191763,344.77956795)(570.16691771,344.78456794)(570.07692245,344.7745743)
\lineto(569.80692245,344.7745743)
\curveto(569.7769181,344.75456797)(569.74191814,344.74456798)(569.70192245,344.7445743)
\curveto(569.66191822,344.75456797)(569.62691825,344.75456797)(569.59692245,344.7445743)
\lineto(569.38692245,344.6845743)
\curveto(569.32691855,344.67456805)(569.27191861,344.65456807)(569.22192245,344.6245743)
\curveto(568.97191891,344.51456821)(568.76691911,344.35456837)(568.60692245,344.1445743)
\curveto(568.45691942,343.94456878)(568.33691954,343.70956902)(568.24692245,343.4395743)
\curveto(568.21691966,343.33956939)(568.19191969,343.23456949)(568.17192245,343.1245743)
\curveto(568.16191972,343.01456971)(568.14691973,342.90456982)(568.12692245,342.7945743)
\curveto(568.11691976,342.74456998)(568.11191977,342.69457003)(568.11192245,342.6445743)
\lineto(568.11192245,342.4945743)
\curveto(568.09191979,342.4245703)(568.0819198,342.31957041)(568.08192245,342.1795743)
\curveto(568.09191979,342.03957069)(568.10691977,341.93457079)(568.12692245,341.8645743)
\lineto(568.12692245,341.7295743)
\curveto(568.14691973,341.64957108)(568.16191972,341.56957116)(568.17192245,341.4895743)
\curveto(568.1819197,341.41957131)(568.19691968,341.34457138)(568.21692245,341.2645743)
\curveto(568.31691956,340.96457176)(568.42191946,340.71957201)(568.53192245,340.5295743)
\curveto(568.65191923,340.34957238)(568.83691904,340.18457254)(569.08692245,340.0345743)
\curveto(569.15691872,339.98457274)(569.23191865,339.94457278)(569.31192245,339.9145743)
\curveto(569.40191848,339.88457284)(569.49191839,339.85957287)(569.58192245,339.8395743)
\curveto(569.62191826,339.8295729)(569.65691822,339.8245729)(569.68692245,339.8245743)
\curveto(569.71691816,339.83457289)(569.75191813,339.83457289)(569.79192245,339.8245743)
\lineto(569.91192245,339.7945743)
\curveto(569.96191792,339.79457293)(570.00691787,339.79957293)(570.04692245,339.8095743)
\lineto(570.16692245,339.8095743)
\curveto(570.24691763,339.8295729)(570.32691755,339.84457288)(570.40692245,339.8545743)
\curveto(570.48691739,339.86457286)(570.56191732,339.88457284)(570.63192245,339.9145743)
\curveto(570.89191699,340.01457271)(571.10191678,340.14957258)(571.26192245,340.3195743)
\curveto(571.42191646,340.48957224)(571.55691632,340.69957203)(571.66692245,340.9495743)
\curveto(571.70691617,341.04957168)(571.73691614,341.14957158)(571.75692245,341.2495743)
\curveto(571.7769161,341.34957138)(571.80191608,341.45457127)(571.83192245,341.5645743)
\curveto(571.84191604,341.60457112)(571.84691603,341.63957109)(571.84692245,341.6695743)
\curveto(571.84691603,341.70957102)(571.85191603,341.74957098)(571.86192245,341.7895743)
\lineto(571.86192245,341.9245743)
\curveto(571.86191602,341.97457075)(571.86691601,342.0245707)(571.87692245,342.0745743)
}
}
{
\newrgbcolor{curcolor}{0 0 0}
\pscustom[linestyle=none,fillstyle=solid,fillcolor=curcolor]
{
\newpath
\moveto(578.34684433,346.3795743)
\curveto(579.09683983,346.39956633)(579.74683918,346.31456641)(580.29684433,346.1245743)
\curveto(580.85683807,345.94456678)(581.28183764,345.6295671)(581.57184433,345.1795743)
\curveto(581.64183728,345.06956766)(581.70183722,344.95456777)(581.75184433,344.8345743)
\curveto(581.81183711,344.724568)(581.86183706,344.59956813)(581.90184433,344.4595743)
\curveto(581.921837,344.39956833)(581.93183699,344.33456839)(581.93184433,344.2645743)
\curveto(581.93183699,344.19456853)(581.921837,344.13456859)(581.90184433,344.0845743)
\curveto(581.86183706,344.0245687)(581.80683712,343.98456874)(581.73684433,343.9645743)
\curveto(581.68683724,343.94456878)(581.6268373,343.93456879)(581.55684433,343.9345743)
\lineto(581.34684433,343.9345743)
\lineto(580.68684433,343.9345743)
\curveto(580.61683831,343.93456879)(580.54683838,343.9295688)(580.47684433,343.9195743)
\curveto(580.40683852,343.91956881)(580.34183858,343.9295688)(580.28184433,343.9495743)
\curveto(580.18183874,343.96956876)(580.10683882,344.00956872)(580.05684433,344.0695743)
\curveto(580.00683892,344.1295686)(579.96183896,344.18956854)(579.92184433,344.2495743)
\lineto(579.80184433,344.4595743)
\curveto(579.77183915,344.53956819)(579.7218392,344.60456812)(579.65184433,344.6545743)
\curveto(579.55183937,344.73456799)(579.45183947,344.79456793)(579.35184433,344.8345743)
\curveto(579.26183966,344.87456785)(579.14683978,344.90956782)(579.00684433,344.9395743)
\curveto(578.93683999,344.95956777)(578.83184009,344.97456775)(578.69184433,344.9845743)
\curveto(578.56184036,344.99456773)(578.46184046,344.98956774)(578.39184433,344.9695743)
\lineto(578.28684433,344.9695743)
\lineto(578.13684433,344.9395743)
\curveto(578.09684083,344.93956779)(578.05184087,344.93456779)(578.00184433,344.9245743)
\curveto(577.83184109,344.87456785)(577.69184123,344.80456792)(577.58184433,344.7145743)
\curveto(577.48184144,344.63456809)(577.41184151,344.50956822)(577.37184433,344.3395743)
\curveto(577.35184157,344.26956846)(577.35184157,344.20456852)(577.37184433,344.1445743)
\curveto(577.39184153,344.08456864)(577.41184151,344.03456869)(577.43184433,343.9945743)
\curveto(577.50184142,343.87456885)(577.58184134,343.77956895)(577.67184433,343.7095743)
\curveto(577.77184115,343.63956909)(577.88684104,343.57956915)(578.01684433,343.5295743)
\curveto(578.20684072,343.44956928)(578.41184051,343.37956935)(578.63184433,343.3195743)
\lineto(579.32184433,343.1695743)
\curveto(579.56183936,343.1295696)(579.79183913,343.07956965)(580.01184433,343.0195743)
\curveto(580.24183868,342.96956976)(580.45683847,342.90456982)(580.65684433,342.8245743)
\curveto(580.74683818,342.78456994)(580.83183809,342.74956998)(580.91184433,342.7195743)
\curveto(581.00183792,342.69957003)(581.08683784,342.66457006)(581.16684433,342.6145743)
\curveto(581.35683757,342.49457023)(581.5268374,342.36457036)(581.67684433,342.2245743)
\curveto(581.83683709,342.08457064)(581.96183696,341.90957082)(582.05184433,341.6995743)
\curveto(582.08183684,341.6295711)(582.10683682,341.55957117)(582.12684433,341.4895743)
\curveto(582.14683678,341.41957131)(582.16683676,341.34457138)(582.18684433,341.2645743)
\curveto(582.19683673,341.20457152)(582.20183672,341.10957162)(582.20184433,340.9795743)
\curveto(582.21183671,340.85957187)(582.21183671,340.76457196)(582.20184433,340.6945743)
\lineto(582.20184433,340.6195743)
\curveto(582.18183674,340.55957217)(582.16683676,340.49957223)(582.15684433,340.4395743)
\curveto(582.15683677,340.38957234)(582.15183677,340.33957239)(582.14184433,340.2895743)
\curveto(582.07183685,339.98957274)(581.96183696,339.724573)(581.81184433,339.4945743)
\curveto(581.65183727,339.25457347)(581.45683747,339.05957367)(581.22684433,338.9095743)
\curveto(580.99683793,338.75957397)(580.73683819,338.6295741)(580.44684433,338.5195743)
\curveto(580.33683859,338.46957426)(580.21683871,338.43457429)(580.08684433,338.4145743)
\curveto(579.96683896,338.39457433)(579.84683908,338.36957436)(579.72684433,338.3395743)
\curveto(579.63683929,338.31957441)(579.54183938,338.30957442)(579.44184433,338.3095743)
\curveto(579.35183957,338.29957443)(579.26183966,338.28457444)(579.17184433,338.2645743)
\lineto(578.90184433,338.2645743)
\curveto(578.84184008,338.24457448)(578.73684019,338.23457449)(578.58684433,338.2345743)
\curveto(578.44684048,338.23457449)(578.34684058,338.24457448)(578.28684433,338.2645743)
\curveto(578.25684067,338.26457446)(578.2218407,338.26957446)(578.18184433,338.2795743)
\lineto(578.07684433,338.2795743)
\curveto(577.95684097,338.29957443)(577.83684109,338.31457441)(577.71684433,338.3245743)
\curveto(577.59684133,338.33457439)(577.48184144,338.35457437)(577.37184433,338.3845743)
\curveto(576.98184194,338.49457423)(576.63684229,338.61957411)(576.33684433,338.7595743)
\curveto(576.03684289,338.90957382)(575.78184314,339.1295736)(575.57184433,339.4195743)
\curveto(575.43184349,339.60957312)(575.31184361,339.8295729)(575.21184433,340.0795743)
\curveto(575.19184373,340.13957259)(575.17184375,340.21957251)(575.15184433,340.3195743)
\curveto(575.13184379,340.36957236)(575.11684381,340.43957229)(575.10684433,340.5295743)
\curveto(575.09684383,340.61957211)(575.10184382,340.69457203)(575.12184433,340.7545743)
\curveto(575.15184377,340.8245719)(575.20184372,340.87457185)(575.27184433,340.9045743)
\curveto(575.3218436,340.9245718)(575.38184354,340.93457179)(575.45184433,340.9345743)
\lineto(575.67684433,340.9345743)
\lineto(576.38184433,340.9345743)
\lineto(576.62184433,340.9345743)
\curveto(576.70184222,340.93457179)(576.77184215,340.9245718)(576.83184433,340.9045743)
\curveto(576.94184198,340.86457186)(577.01184191,340.79957193)(577.04184433,340.7095743)
\curveto(577.08184184,340.61957211)(577.1268418,340.5245722)(577.17684433,340.4245743)
\curveto(577.19684173,340.37457235)(577.23184169,340.30957242)(577.28184433,340.2295743)
\curveto(577.34184158,340.14957258)(577.39184153,340.09957263)(577.43184433,340.0795743)
\curveto(577.55184137,339.97957275)(577.66684126,339.89957283)(577.77684433,339.8395743)
\curveto(577.88684104,339.78957294)(578.0268409,339.73957299)(578.19684433,339.6895743)
\curveto(578.24684068,339.66957306)(578.29684063,339.65957307)(578.34684433,339.6595743)
\curveto(578.39684053,339.66957306)(578.44684048,339.66957306)(578.49684433,339.6595743)
\curveto(578.57684035,339.63957309)(578.66184026,339.6295731)(578.75184433,339.6295743)
\curveto(578.85184007,339.63957309)(578.93683999,339.65457307)(579.00684433,339.6745743)
\curveto(579.05683987,339.68457304)(579.10183982,339.68957304)(579.14184433,339.6895743)
\curveto(579.19183973,339.68957304)(579.24183968,339.69957303)(579.29184433,339.7195743)
\curveto(579.43183949,339.76957296)(579.55683937,339.8295729)(579.66684433,339.8995743)
\curveto(579.78683914,339.96957276)(579.88183904,340.05957267)(579.95184433,340.1695743)
\curveto(580.00183892,340.24957248)(580.04183888,340.37457235)(580.07184433,340.5445743)
\curveto(580.09183883,340.61457211)(580.09183883,340.67957205)(580.07184433,340.7395743)
\curveto(580.05183887,340.79957193)(580.03183889,340.84957188)(580.01184433,340.8895743)
\curveto(579.94183898,341.0295717)(579.85183907,341.13457159)(579.74184433,341.2045743)
\curveto(579.64183928,341.27457145)(579.5218394,341.33957139)(579.38184433,341.3995743)
\curveto(579.19183973,341.47957125)(578.99183993,341.54457118)(578.78184433,341.5945743)
\curveto(578.57184035,341.64457108)(578.36184056,341.69957103)(578.15184433,341.7595743)
\curveto(578.07184085,341.77957095)(577.98684094,341.79457093)(577.89684433,341.8045743)
\curveto(577.81684111,341.81457091)(577.73684119,341.8295709)(577.65684433,341.8495743)
\curveto(577.33684159,341.93957079)(577.03184189,342.0245707)(576.74184433,342.1045743)
\curveto(576.45184247,342.19457053)(576.18684274,342.3245704)(575.94684433,342.4945743)
\curveto(575.66684326,342.69457003)(575.46184346,342.96456976)(575.33184433,343.3045743)
\curveto(575.31184361,343.37456935)(575.29184363,343.46956926)(575.27184433,343.5895743)
\curveto(575.25184367,343.65956907)(575.23684369,343.74456898)(575.22684433,343.8445743)
\curveto(575.21684371,343.94456878)(575.2218437,344.03456869)(575.24184433,344.1145743)
\curveto(575.26184366,344.16456856)(575.26684366,344.20456852)(575.25684433,344.2345743)
\curveto(575.24684368,344.27456845)(575.25184367,344.31956841)(575.27184433,344.3695743)
\curveto(575.29184363,344.47956825)(575.31184361,344.57956815)(575.33184433,344.6695743)
\curveto(575.36184356,344.76956796)(575.39684353,344.86456786)(575.43684433,344.9545743)
\curveto(575.56684336,345.24456748)(575.74684318,345.47956725)(575.97684433,345.6595743)
\curveto(576.20684272,345.83956689)(576.46684246,345.98456674)(576.75684433,346.0945743)
\curveto(576.86684206,346.14456658)(576.98184194,346.17956655)(577.10184433,346.1995743)
\curveto(577.2218417,346.2295665)(577.34684158,346.25956647)(577.47684433,346.2895743)
\curveto(577.53684139,346.30956642)(577.59684133,346.31956641)(577.65684433,346.3195743)
\lineto(577.83684433,346.3495743)
\curveto(577.91684101,346.35956637)(578.00184092,346.36456636)(578.09184433,346.3645743)
\curveto(578.18184074,346.36456636)(578.26684066,346.36956636)(578.34684433,346.3795743)
}
}
{
\newrgbcolor{curcolor}{0 0 0}
\pscustom[linestyle=none,fillstyle=solid,fillcolor=curcolor]
{
}
}
{
\newrgbcolor{curcolor}{0 0 0}
\pscustom[linestyle=none,fillstyle=solid,fillcolor=curcolor]
{
\newpath
\moveto(595.4836412,342.3895743)
\curveto(595.49363252,342.3295704)(595.49863252,342.23957049)(595.4986412,342.1195743)
\curveto(595.49863252,341.99957073)(595.48863253,341.91457081)(595.4686412,341.8645743)
\lineto(595.4686412,341.6695743)
\curveto(595.43863258,341.55957117)(595.4186326,341.45457127)(595.4086412,341.3545743)
\curveto(595.40863261,341.25457147)(595.39363262,341.15457157)(595.3636412,341.0545743)
\curveto(595.34363267,340.96457176)(595.32363269,340.86957186)(595.3036412,340.7695743)
\curveto(595.28363273,340.67957205)(595.25363276,340.58957214)(595.2136412,340.4995743)
\curveto(595.14363287,340.3295724)(595.07363294,340.16957256)(595.0036412,340.0195743)
\curveto(594.93363308,339.87957285)(594.85363316,339.73957299)(594.7636412,339.5995743)
\curveto(594.70363331,339.50957322)(594.63863338,339.4245733)(594.5686412,339.3445743)
\curveto(594.50863351,339.27457345)(594.43863358,339.19957353)(594.3586412,339.1195743)
\lineto(594.2536412,339.0145743)
\curveto(594.20363381,338.96457376)(594.14863387,338.91957381)(594.0886412,338.8795743)
\lineto(593.9386412,338.7595743)
\curveto(593.85863416,338.69957403)(593.76863425,338.64457408)(593.6686412,338.5945743)
\curveto(593.57863444,338.55457417)(593.48363453,338.50957422)(593.3836412,338.4595743)
\curveto(593.28363473,338.40957432)(593.17863484,338.37457435)(593.0686412,338.3545743)
\curveto(592.96863505,338.33457439)(592.86363515,338.31457441)(592.7536412,338.2945743)
\curveto(592.69363532,338.27457445)(592.62863539,338.26457446)(592.5586412,338.2645743)
\curveto(592.49863552,338.26457446)(592.43363558,338.25457447)(592.3636412,338.2345743)
\lineto(592.2286412,338.2345743)
\curveto(592.14863587,338.21457451)(592.07363594,338.21457451)(592.0036412,338.2345743)
\lineto(591.8536412,338.2345743)
\curveto(591.79363622,338.25457447)(591.72863629,338.26457446)(591.6586412,338.2645743)
\curveto(591.59863642,338.25457447)(591.53863648,338.25957447)(591.4786412,338.2795743)
\curveto(591.3186367,338.3295744)(591.16363685,338.37457435)(591.0136412,338.4145743)
\curveto(590.87363714,338.45457427)(590.74363727,338.51457421)(590.6236412,338.5945743)
\curveto(590.55363746,338.63457409)(590.48863753,338.67457405)(590.4286412,338.7145743)
\curveto(590.36863765,338.76457396)(590.30363771,338.81457391)(590.2336412,338.8645743)
\lineto(590.0536412,338.9995743)
\curveto(589.97363804,339.05957367)(589.90363811,339.06457366)(589.8436412,339.0145743)
\curveto(589.79363822,338.98457374)(589.76863825,338.94457378)(589.7686412,338.8945743)
\curveto(589.76863825,338.85457387)(589.75863826,338.80457392)(589.7386412,338.7445743)
\curveto(589.7186383,338.64457408)(589.70863831,338.5295742)(589.7086412,338.3995743)
\curveto(589.7186383,338.26957446)(589.72363829,338.14957458)(589.7236412,338.0395743)
\lineto(589.7236412,336.5095743)
\curveto(589.72363829,336.37957635)(589.7186383,336.25457647)(589.7086412,336.1345743)
\curveto(589.70863831,336.00457672)(589.68363833,335.89957683)(589.6336412,335.8195743)
\curveto(589.60363841,335.77957695)(589.54863847,335.74957698)(589.4686412,335.7295743)
\curveto(589.38863863,335.70957702)(589.29863872,335.69957703)(589.1986412,335.6995743)
\curveto(589.09863892,335.68957704)(588.99863902,335.68957704)(588.8986412,335.6995743)
\lineto(588.6436412,335.6995743)
\lineto(588.2386412,335.6995743)
\lineto(588.1336412,335.6995743)
\curveto(588.09363992,335.69957703)(588.05863996,335.70457702)(588.0286412,335.7145743)
\lineto(587.9086412,335.7145743)
\curveto(587.73864028,335.76457696)(587.64864037,335.86457686)(587.6386412,336.0145743)
\curveto(587.62864039,336.15457657)(587.62364039,336.3245764)(587.6236412,336.5245743)
\lineto(587.6236412,345.3295743)
\curveto(587.62364039,345.43956729)(587.6186404,345.55456717)(587.6086412,345.6745743)
\curveto(587.60864041,345.80456692)(587.63364038,345.90456682)(587.6836412,345.9745743)
\curveto(587.72364029,346.04456668)(587.77864024,346.08956664)(587.8486412,346.1095743)
\curveto(587.89864012,346.1295666)(587.95864006,346.13956659)(588.0286412,346.1395743)
\lineto(588.2536412,346.1395743)
\lineto(588.9736412,346.1395743)
\lineto(589.2586412,346.1395743)
\curveto(589.34863867,346.13956659)(589.42363859,346.11456661)(589.4836412,346.0645743)
\curveto(589.55363846,346.01456671)(589.58863843,345.94956678)(589.5886412,345.8695743)
\curveto(589.59863842,345.79956693)(589.62363839,345.724567)(589.6636412,345.6445743)
\curveto(589.67363834,345.61456711)(589.68363833,345.58956714)(589.6936412,345.5695743)
\curveto(589.7136383,345.55956717)(589.73363828,345.54456718)(589.7536412,345.5245743)
\curveto(589.86363815,345.51456721)(589.95363806,345.54456718)(590.0236412,345.6145743)
\curveto(590.09363792,345.68456704)(590.16363785,345.74456698)(590.2336412,345.7945743)
\curveto(590.36363765,345.88456684)(590.49863752,345.96456676)(590.6386412,346.0345743)
\curveto(590.77863724,346.11456661)(590.93363708,346.17956655)(591.1036412,346.2295743)
\curveto(591.18363683,346.25956647)(591.26863675,346.27956645)(591.3586412,346.2895743)
\curveto(591.45863656,346.29956643)(591.55363646,346.31456641)(591.6436412,346.3345743)
\curveto(591.68363633,346.34456638)(591.72363629,346.34456638)(591.7636412,346.3345743)
\curveto(591.8136362,346.3245664)(591.85363616,346.3295664)(591.8836412,346.3495743)
\curveto(592.45363556,346.36956636)(592.93363508,346.28956644)(593.3236412,346.1095743)
\curveto(593.72363429,345.93956679)(594.06363395,345.71456701)(594.3436412,345.4345743)
\curveto(594.39363362,345.38456734)(594.43863358,345.33456739)(594.4786412,345.2845743)
\curveto(594.5186335,345.24456748)(594.55863346,345.19956753)(594.5986412,345.1495743)
\curveto(594.66863335,345.05956767)(594.72863329,344.96956776)(594.7786412,344.8795743)
\curveto(594.83863318,344.78956794)(594.89363312,344.69956803)(594.9436412,344.6095743)
\curveto(594.96363305,344.58956814)(594.97363304,344.56456816)(594.9736412,344.5345743)
\curveto(594.98363303,344.50456822)(594.99863302,344.46956826)(595.0186412,344.4295743)
\curveto(595.07863294,344.3295684)(595.13363288,344.20956852)(595.1836412,344.0695743)
\curveto(595.20363281,344.00956872)(595.22363279,343.94456878)(595.2436412,343.8745743)
\curveto(595.26363275,343.81456891)(595.28363273,343.74956898)(595.3036412,343.6795743)
\curveto(595.34363267,343.55956917)(595.36863265,343.43456929)(595.3786412,343.3045743)
\curveto(595.39863262,343.17456955)(595.42363259,343.03956969)(595.4536412,342.8995743)
\lineto(595.4536412,342.7345743)
\lineto(595.4836412,342.5545743)
\lineto(595.4836412,342.3895743)
\moveto(593.3686412,342.0445743)
\curveto(593.37863464,342.09457063)(593.38363463,342.15957057)(593.3836412,342.2395743)
\curveto(593.38363463,342.3295704)(593.37863464,342.39957033)(593.3686412,342.4495743)
\lineto(593.3686412,342.5845743)
\curveto(593.34863467,342.64457008)(593.33863468,342.70957002)(593.3386412,342.7795743)
\curveto(593.33863468,342.84956988)(593.32863469,342.91956981)(593.3086412,342.9895743)
\curveto(593.28863473,343.08956964)(593.26863475,343.18456954)(593.2486412,343.2745743)
\curveto(593.22863479,343.37456935)(593.19863482,343.46456926)(593.1586412,343.5445743)
\curveto(593.03863498,343.86456886)(592.88363513,344.11956861)(592.6936412,344.3095743)
\curveto(592.50363551,344.49956823)(592.23363578,344.63956809)(591.8836412,344.7295743)
\curveto(591.80363621,344.74956798)(591.7136363,344.75956797)(591.6136412,344.7595743)
\lineto(591.3436412,344.7595743)
\curveto(591.30363671,344.74956798)(591.26863675,344.74456798)(591.2386412,344.7445743)
\curveto(591.20863681,344.74456798)(591.17363684,344.73956799)(591.1336412,344.7295743)
\lineto(590.9236412,344.6695743)
\curveto(590.86363715,344.65956807)(590.80363721,344.63956809)(590.7436412,344.6095743)
\curveto(590.48363753,344.49956823)(590.27863774,344.3295684)(590.1286412,344.0995743)
\curveto(589.98863803,343.86956886)(589.87363814,343.61456911)(589.7836412,343.3345743)
\curveto(589.76363825,343.25456947)(589.74863827,343.16956956)(589.7386412,343.0795743)
\curveto(589.72863829,342.99956973)(589.7136383,342.91956981)(589.6936412,342.8395743)
\curveto(589.68363833,342.79956993)(589.67863834,342.73456999)(589.6786412,342.6445743)
\curveto(589.65863836,342.60457012)(589.65363836,342.55457017)(589.6636412,342.4945743)
\curveto(589.67363834,342.44457028)(589.67363834,342.39457033)(589.6636412,342.3445743)
\curveto(589.64363837,342.28457044)(589.64363837,342.2295705)(589.6636412,342.1795743)
\lineto(589.6636412,341.9995743)
\lineto(589.6636412,341.8645743)
\curveto(589.66363835,341.8245709)(589.67363834,341.78457094)(589.6936412,341.7445743)
\curveto(589.69363832,341.67457105)(589.69863832,341.61957111)(589.7086412,341.5795743)
\lineto(589.7386412,341.3995743)
\curveto(589.74863827,341.33957139)(589.76363825,341.27957145)(589.7836412,341.2195743)
\curveto(589.87363814,340.9295718)(589.97863804,340.68957204)(590.0986412,340.4995743)
\curveto(590.22863779,340.31957241)(590.40863761,340.15957257)(590.6386412,340.0195743)
\curveto(590.77863724,339.93957279)(590.94363707,339.87457285)(591.1336412,339.8245743)
\curveto(591.17363684,339.81457291)(591.20863681,339.80957292)(591.2386412,339.8095743)
\curveto(591.26863675,339.81957291)(591.30363671,339.81957291)(591.3436412,339.8095743)
\curveto(591.38363663,339.79957293)(591.44363657,339.78957294)(591.5236412,339.7795743)
\curveto(591.60363641,339.77957295)(591.66863635,339.78457294)(591.7186412,339.7945743)
\curveto(591.79863622,339.81457291)(591.87863614,339.8295729)(591.9586412,339.8395743)
\curveto(592.04863597,339.85957287)(592.13363588,339.88457284)(592.2136412,339.9145743)
\curveto(592.45363556,340.01457271)(592.64863537,340.15457257)(592.7986412,340.3345743)
\curveto(592.94863507,340.51457221)(593.07363494,340.724572)(593.1736412,340.9645743)
\curveto(593.22363479,341.08457164)(593.25863476,341.20957152)(593.2786412,341.3395743)
\curveto(593.29863472,341.46957126)(593.32363469,341.60457112)(593.3536412,341.7445743)
\lineto(593.3536412,341.8945743)
\curveto(593.36363465,341.94457078)(593.36863465,341.99457073)(593.3686412,342.0445743)
}
}
{
\newrgbcolor{curcolor}{0 0 0}
\pscustom[linestyle=none,fillstyle=solid,fillcolor=curcolor]
{
\newpath
\moveto(604.53356308,342.6145743)
\curveto(604.55355451,342.55457017)(604.5635545,342.46957026)(604.56356308,342.3595743)
\curveto(604.5635545,342.24957048)(604.55355451,342.16457056)(604.53356308,342.1045743)
\lineto(604.53356308,341.9545743)
\curveto(604.51355455,341.87457085)(604.50355456,341.79457093)(604.50356308,341.7145743)
\curveto(604.51355455,341.63457109)(604.50855455,341.55457117)(604.48856308,341.4745743)
\curveto(604.46855459,341.40457132)(604.45355461,341.33957139)(604.44356308,341.2795743)
\curveto(604.43355463,341.21957151)(604.42355464,341.15457157)(604.41356308,341.0845743)
\curveto(604.37355469,340.97457175)(604.33855472,340.85957187)(604.30856308,340.7395743)
\curveto(604.27855478,340.6295721)(604.23855482,340.5245722)(604.18856308,340.4245743)
\curveto(603.97855508,339.94457278)(603.70355536,339.55457317)(603.36356308,339.2545743)
\curveto(603.02355604,338.95457377)(602.61355645,338.70457402)(602.13356308,338.5045743)
\curveto(602.01355705,338.45457427)(601.88855717,338.41957431)(601.75856308,338.3995743)
\curveto(601.63855742,338.36957436)(601.51355755,338.33957439)(601.38356308,338.3095743)
\curveto(601.33355773,338.28957444)(601.27855778,338.27957445)(601.21856308,338.2795743)
\curveto(601.1585579,338.27957445)(601.10355796,338.27457445)(601.05356308,338.2645743)
\lineto(600.94856308,338.2645743)
\curveto(600.91855814,338.25457447)(600.88855817,338.24957448)(600.85856308,338.2495743)
\curveto(600.80855825,338.23957449)(600.72855833,338.23457449)(600.61856308,338.2345743)
\curveto(600.50855855,338.2245745)(600.42355864,338.2295745)(600.36356308,338.2495743)
\lineto(600.21356308,338.2495743)
\curveto(600.1635589,338.25957447)(600.10855895,338.26457446)(600.04856308,338.2645743)
\curveto(599.99855906,338.25457447)(599.94855911,338.25957447)(599.89856308,338.2795743)
\curveto(599.8585592,338.28957444)(599.81855924,338.29457443)(599.77856308,338.2945743)
\curveto(599.74855931,338.29457443)(599.70855935,338.29957443)(599.65856308,338.3095743)
\curveto(599.5585595,338.33957439)(599.4585596,338.36457436)(599.35856308,338.3845743)
\curveto(599.2585598,338.40457432)(599.1635599,338.43457429)(599.07356308,338.4745743)
\curveto(598.95356011,338.51457421)(598.83856022,338.55457417)(598.72856308,338.5945743)
\curveto(598.62856043,338.63457409)(598.52356054,338.68457404)(598.41356308,338.7445743)
\curveto(598.063561,338.95457377)(597.7635613,339.19957353)(597.51356308,339.4795743)
\curveto(597.2635618,339.75957297)(597.05356201,340.09457263)(596.88356308,340.4845743)
\curveto(596.83356223,340.57457215)(596.79356227,340.66957206)(596.76356308,340.7695743)
\curveto(596.74356232,340.86957186)(596.71856234,340.97457175)(596.68856308,341.0845743)
\curveto(596.66856239,341.13457159)(596.6585624,341.17957155)(596.65856308,341.2195743)
\curveto(596.6585624,341.25957147)(596.64856241,341.30457142)(596.62856308,341.3545743)
\curveto(596.60856245,341.43457129)(596.59856246,341.51457121)(596.59856308,341.5945743)
\curveto(596.59856246,341.68457104)(596.58856247,341.76957096)(596.56856308,341.8495743)
\curveto(596.5585625,341.89957083)(596.55356251,341.94457078)(596.55356308,341.9845743)
\lineto(596.55356308,342.1195743)
\curveto(596.53356253,342.17957055)(596.52356254,342.26457046)(596.52356308,342.3745743)
\curveto(596.53356253,342.48457024)(596.54856251,342.56957016)(596.56856308,342.6295743)
\lineto(596.56856308,342.7345743)
\curveto(596.57856248,342.78456994)(596.57856248,342.83456989)(596.56856308,342.8845743)
\curveto(596.56856249,342.94456978)(596.57856248,342.99956973)(596.59856308,343.0495743)
\curveto(596.60856245,343.09956963)(596.61356245,343.14456958)(596.61356308,343.1845743)
\curveto(596.61356245,343.23456949)(596.62356244,343.28456944)(596.64356308,343.3345743)
\curveto(596.68356238,343.46456926)(596.71856234,343.58956914)(596.74856308,343.7095743)
\curveto(596.77856228,343.83956889)(596.81856224,343.96456876)(596.86856308,344.0845743)
\curveto(597.04856201,344.49456823)(597.2635618,344.83456789)(597.51356308,345.1045743)
\curveto(597.7635613,345.38456734)(598.06856099,345.63956709)(598.42856308,345.8695743)
\curveto(598.52856053,345.91956681)(598.63356043,345.96456676)(598.74356308,346.0045743)
\curveto(598.85356021,346.04456668)(598.9635601,346.08956664)(599.07356308,346.1395743)
\curveto(599.20355986,346.18956654)(599.33855972,346.2245665)(599.47856308,346.2445743)
\curveto(599.61855944,346.26456646)(599.7635593,346.29456643)(599.91356308,346.3345743)
\curveto(599.99355907,346.34456638)(600.06855899,346.34956638)(600.13856308,346.3495743)
\curveto(600.20855885,346.34956638)(600.27855878,346.35456637)(600.34856308,346.3645743)
\curveto(600.92855813,346.37456635)(601.42855763,346.31456641)(601.84856308,346.1845743)
\curveto(602.27855678,346.05456667)(602.6585564,345.87456685)(602.98856308,345.6445743)
\curveto(603.09855596,345.56456716)(603.20855585,345.47456725)(603.31856308,345.3745743)
\curveto(603.43855562,345.28456744)(603.53855552,345.18456754)(603.61856308,345.0745743)
\curveto(603.69855536,344.97456775)(603.76855529,344.87456785)(603.82856308,344.7745743)
\curveto(603.89855516,344.67456805)(603.96855509,344.56956816)(604.03856308,344.4595743)
\curveto(604.10855495,344.34956838)(604.1635549,344.2295685)(604.20356308,344.0995743)
\curveto(604.24355482,343.97956875)(604.28855477,343.84956888)(604.33856308,343.7095743)
\curveto(604.36855469,343.6295691)(604.39355467,343.54456918)(604.41356308,343.4545743)
\lineto(604.47356308,343.1845743)
\curveto(604.48355458,343.14456958)(604.48855457,343.10456962)(604.48856308,343.0645743)
\curveto(604.48855457,343.0245697)(604.49355457,342.98456974)(604.50356308,342.9445743)
\curveto(604.52355454,342.89456983)(604.52855453,342.83956989)(604.51856308,342.7795743)
\curveto(604.50855455,342.71957001)(604.51355455,342.66457006)(604.53356308,342.6145743)
\moveto(602.43356308,342.0745743)
\curveto(602.44355662,342.1245706)(602.44855661,342.19457053)(602.44856308,342.2845743)
\curveto(602.44855661,342.38457034)(602.44355662,342.45957027)(602.43356308,342.5095743)
\lineto(602.43356308,342.6295743)
\curveto(602.41355665,342.67957005)(602.40355666,342.73456999)(602.40356308,342.7945743)
\curveto(602.40355666,342.85456987)(602.39855666,342.90956982)(602.38856308,342.9595743)
\curveto(602.38855667,342.99956973)(602.38355668,343.0295697)(602.37356308,343.0495743)
\lineto(602.31356308,343.2895743)
\curveto(602.30355676,343.37956935)(602.28355678,343.46456926)(602.25356308,343.5445743)
\curveto(602.14355692,343.80456892)(602.01355705,344.0245687)(601.86356308,344.2045743)
\curveto(601.71355735,344.39456833)(601.51355755,344.54456818)(601.26356308,344.6545743)
\curveto(601.20355786,344.67456805)(601.14355792,344.68956804)(601.08356308,344.6995743)
\curveto(601.02355804,344.71956801)(600.9585581,344.73956799)(600.88856308,344.7595743)
\curveto(600.80855825,344.77956795)(600.72355834,344.78456794)(600.63356308,344.7745743)
\lineto(600.36356308,344.7745743)
\curveto(600.33355873,344.75456797)(600.29855876,344.74456798)(600.25856308,344.7445743)
\curveto(600.21855884,344.75456797)(600.18355888,344.75456797)(600.15356308,344.7445743)
\lineto(599.94356308,344.6845743)
\curveto(599.88355918,344.67456805)(599.82855923,344.65456807)(599.77856308,344.6245743)
\curveto(599.52855953,344.51456821)(599.32355974,344.35456837)(599.16356308,344.1445743)
\curveto(599.01356005,343.94456878)(598.89356017,343.70956902)(598.80356308,343.4395743)
\curveto(598.77356029,343.33956939)(598.74856031,343.23456949)(598.72856308,343.1245743)
\curveto(598.71856034,343.01456971)(598.70356036,342.90456982)(598.68356308,342.7945743)
\curveto(598.67356039,342.74456998)(598.66856039,342.69457003)(598.66856308,342.6445743)
\lineto(598.66856308,342.4945743)
\curveto(598.64856041,342.4245703)(598.63856042,342.31957041)(598.63856308,342.1795743)
\curveto(598.64856041,342.03957069)(598.6635604,341.93457079)(598.68356308,341.8645743)
\lineto(598.68356308,341.7295743)
\curveto(598.70356036,341.64957108)(598.71856034,341.56957116)(598.72856308,341.4895743)
\curveto(598.73856032,341.41957131)(598.75356031,341.34457138)(598.77356308,341.2645743)
\curveto(598.87356019,340.96457176)(598.97856008,340.71957201)(599.08856308,340.5295743)
\curveto(599.20855985,340.34957238)(599.39355967,340.18457254)(599.64356308,340.0345743)
\curveto(599.71355935,339.98457274)(599.78855927,339.94457278)(599.86856308,339.9145743)
\curveto(599.9585591,339.88457284)(600.04855901,339.85957287)(600.13856308,339.8395743)
\curveto(600.17855888,339.8295729)(600.21355885,339.8245729)(600.24356308,339.8245743)
\curveto(600.27355879,339.83457289)(600.30855875,339.83457289)(600.34856308,339.8245743)
\lineto(600.46856308,339.7945743)
\curveto(600.51855854,339.79457293)(600.5635585,339.79957293)(600.60356308,339.8095743)
\lineto(600.72356308,339.8095743)
\curveto(600.80355826,339.8295729)(600.88355818,339.84457288)(600.96356308,339.8545743)
\curveto(601.04355802,339.86457286)(601.11855794,339.88457284)(601.18856308,339.9145743)
\curveto(601.44855761,340.01457271)(601.6585574,340.14957258)(601.81856308,340.3195743)
\curveto(601.97855708,340.48957224)(602.11355695,340.69957203)(602.22356308,340.9495743)
\curveto(602.2635568,341.04957168)(602.29355677,341.14957158)(602.31356308,341.2495743)
\curveto(602.33355673,341.34957138)(602.3585567,341.45457127)(602.38856308,341.5645743)
\curveto(602.39855666,341.60457112)(602.40355666,341.63957109)(602.40356308,341.6695743)
\curveto(602.40355666,341.70957102)(602.40855665,341.74957098)(602.41856308,341.7895743)
\lineto(602.41856308,341.9245743)
\curveto(602.41855664,341.97457075)(602.42355664,342.0245707)(602.43356308,342.0745743)
}
}
{
\newrgbcolor{curcolor}{0 0 0}
\pscustom[linestyle=none,fillstyle=solid,fillcolor=curcolor]
{
\newpath
\moveto(610.35848495,346.3645743)
\curveto(610.46847964,346.36456636)(610.56347954,346.35456637)(610.64348495,346.3345743)
\curveto(610.73347937,346.31456641)(610.8034793,346.26956646)(610.85348495,346.1995743)
\curveto(610.91347919,346.11956661)(610.94347916,345.97956675)(610.94348495,345.7795743)
\lineto(610.94348495,345.2695743)
\lineto(610.94348495,344.8945743)
\curveto(610.95347915,344.75456797)(610.93847917,344.64456808)(610.89848495,344.5645743)
\curveto(610.85847925,344.49456823)(610.79847931,344.44956828)(610.71848495,344.4295743)
\curveto(610.64847946,344.40956832)(610.56347954,344.39956833)(610.46348495,344.3995743)
\curveto(610.37347973,344.39956833)(610.27347983,344.40456832)(610.16348495,344.4145743)
\curveto(610.06348004,344.4245683)(609.96848014,344.41956831)(609.87848495,344.3995743)
\curveto(609.8084803,344.37956835)(609.73848037,344.36456836)(609.66848495,344.3545743)
\curveto(609.59848051,344.35456837)(609.53348057,344.34456838)(609.47348495,344.3245743)
\curveto(609.31348079,344.27456845)(609.15348095,344.19956853)(608.99348495,344.0995743)
\curveto(608.83348127,344.00956872)(608.7084814,343.90456882)(608.61848495,343.7845743)
\curveto(608.56848154,343.70456902)(608.51348159,343.61956911)(608.45348495,343.5295743)
\curveto(608.4034817,343.44956928)(608.35348175,343.36456936)(608.30348495,343.2745743)
\curveto(608.27348183,343.19456953)(608.24348186,343.10956962)(608.21348495,343.0195743)
\lineto(608.15348495,342.7795743)
\curveto(608.13348197,342.70957002)(608.12348198,342.63457009)(608.12348495,342.5545743)
\curveto(608.12348198,342.48457024)(608.11348199,342.41457031)(608.09348495,342.3445743)
\curveto(608.08348202,342.30457042)(608.07848203,342.26457046)(608.07848495,342.2245743)
\curveto(608.08848202,342.19457053)(608.08848202,342.16457056)(608.07848495,342.1345743)
\lineto(608.07848495,341.8945743)
\curveto(608.05848205,341.8245709)(608.05348205,341.74457098)(608.06348495,341.6545743)
\curveto(608.07348203,341.57457115)(608.07848203,341.49457123)(608.07848495,341.4145743)
\lineto(608.07848495,340.4545743)
\lineto(608.07848495,339.1795743)
\curveto(608.07848203,339.04957368)(608.07348203,338.9295738)(608.06348495,338.8195743)
\curveto(608.05348205,338.70957402)(608.02348208,338.61957411)(607.97348495,338.5495743)
\curveto(607.95348215,338.51957421)(607.91848219,338.49457423)(607.86848495,338.4745743)
\curveto(607.82848228,338.46457426)(607.78348232,338.45457427)(607.73348495,338.4445743)
\lineto(607.65848495,338.4445743)
\curveto(607.6084825,338.43457429)(607.55348255,338.4295743)(607.49348495,338.4295743)
\lineto(607.32848495,338.4295743)
\lineto(606.68348495,338.4295743)
\curveto(606.62348348,338.43957429)(606.55848355,338.44457428)(606.48848495,338.4445743)
\lineto(606.29348495,338.4445743)
\curveto(606.24348386,338.46457426)(606.19348391,338.47957425)(606.14348495,338.4895743)
\curveto(606.09348401,338.50957422)(606.05848405,338.54457418)(606.03848495,338.5945743)
\curveto(605.99848411,338.64457408)(605.97348413,338.71457401)(605.96348495,338.8045743)
\lineto(605.96348495,339.1045743)
\lineto(605.96348495,340.1245743)
\lineto(605.96348495,344.3545743)
\lineto(605.96348495,345.4645743)
\lineto(605.96348495,345.7495743)
\curveto(605.96348414,345.84956688)(605.98348412,345.9295668)(606.02348495,345.9895743)
\curveto(606.07348403,346.06956666)(606.14848396,346.11956661)(606.24848495,346.1395743)
\curveto(606.34848376,346.15956657)(606.46848364,346.16956656)(606.60848495,346.1695743)
\lineto(607.37348495,346.1695743)
\curveto(607.49348261,346.16956656)(607.59848251,346.15956657)(607.68848495,346.1395743)
\curveto(607.77848233,346.1295666)(607.84848226,346.08456664)(607.89848495,346.0045743)
\curveto(607.92848218,345.95456677)(607.94348216,345.88456684)(607.94348495,345.7945743)
\lineto(607.97348495,345.5245743)
\curveto(607.98348212,345.44456728)(607.99848211,345.36956736)(608.01848495,345.2995743)
\curveto(608.04848206,345.2295675)(608.09848201,345.19456753)(608.16848495,345.1945743)
\curveto(608.18848192,345.21456751)(608.2084819,345.2245675)(608.22848495,345.2245743)
\curveto(608.24848186,345.2245675)(608.26848184,345.23456749)(608.28848495,345.2545743)
\curveto(608.34848176,345.30456742)(608.39848171,345.35956737)(608.43848495,345.4195743)
\curveto(608.48848162,345.48956724)(608.54848156,345.54956718)(608.61848495,345.5995743)
\curveto(608.65848145,345.6295671)(608.69348141,345.65956707)(608.72348495,345.6895743)
\curveto(608.75348135,345.729567)(608.78848132,345.76456696)(608.82848495,345.7945743)
\lineto(609.09848495,345.9745743)
\curveto(609.19848091,346.03456669)(609.29848081,346.08956664)(609.39848495,346.1395743)
\curveto(609.49848061,346.17956655)(609.59848051,346.21456651)(609.69848495,346.2445743)
\lineto(610.02848495,346.3345743)
\curveto(610.05848005,346.34456638)(610.11347999,346.34456638)(610.19348495,346.3345743)
\curveto(610.28347982,346.33456639)(610.33847977,346.34456638)(610.35848495,346.3645743)
}
}
{
\newrgbcolor{curcolor}{0 0 0}
\pscustom[linestyle=none,fillstyle=solid,fillcolor=curcolor]
{
}
}
{
\newrgbcolor{curcolor}{0 0 0}
\pscustom[linestyle=none,fillstyle=solid,fillcolor=curcolor]
{
\newpath
\moveto(616.97371933,348.4795743)
\lineto(617.97871933,348.4795743)
\curveto(618.12871634,348.47956425)(618.25871621,348.46956426)(618.36871933,348.4495743)
\curveto(618.48871598,348.43956429)(618.5737159,348.37956435)(618.62371933,348.2695743)
\curveto(618.64371583,348.21956451)(618.65371582,348.15956457)(618.65371933,348.0895743)
\lineto(618.65371933,347.8795743)
\lineto(618.65371933,347.2045743)
\curveto(618.65371582,347.15456557)(618.64871582,347.09456563)(618.63871933,347.0245743)
\curveto(618.63871583,346.96456576)(618.64371583,346.90956582)(618.65371933,346.8595743)
\lineto(618.65371933,346.6945743)
\curveto(618.65371582,346.61456611)(618.65871581,346.53956619)(618.66871933,346.4695743)
\curveto(618.67871579,346.40956632)(618.70371577,346.35456637)(618.74371933,346.3045743)
\curveto(618.81371566,346.21456651)(618.93871553,346.16456656)(619.11871933,346.1545743)
\lineto(619.65871933,346.1545743)
\lineto(619.83871933,346.1545743)
\curveto(619.89871457,346.15456657)(619.95371452,346.14456658)(620.00371933,346.1245743)
\curveto(620.11371436,346.07456665)(620.1737143,345.98456674)(620.18371933,345.8545743)
\curveto(620.20371427,345.724567)(620.21371426,345.57956715)(620.21371933,345.4195743)
\lineto(620.21371933,345.2095743)
\curveto(620.22371425,345.13956759)(620.21871425,345.07956765)(620.19871933,345.0295743)
\curveto(620.14871432,344.86956786)(620.04371443,344.78456794)(619.88371933,344.7745743)
\curveto(619.72371475,344.76456796)(619.54371493,344.75956797)(619.34371933,344.7595743)
\lineto(619.20871933,344.7595743)
\curveto(619.1687153,344.76956796)(619.13371534,344.76956796)(619.10371933,344.7595743)
\curveto(619.06371541,344.74956798)(619.02871544,344.74456798)(618.99871933,344.7445743)
\curveto(618.9687155,344.75456797)(618.93871553,344.74956798)(618.90871933,344.7295743)
\curveto(618.82871564,344.70956802)(618.7687157,344.66456806)(618.72871933,344.5945743)
\curveto(618.69871577,344.53456819)(618.6737158,344.45956827)(618.65371933,344.3695743)
\curveto(618.64371583,344.31956841)(618.64371583,344.26456846)(618.65371933,344.2045743)
\curveto(618.66371581,344.14456858)(618.66371581,344.08956864)(618.65371933,344.0395743)
\lineto(618.65371933,343.1095743)
\lineto(618.65371933,341.3545743)
\curveto(618.65371582,341.10457162)(618.65871581,340.88457184)(618.66871933,340.6945743)
\curveto(618.68871578,340.51457221)(618.75371572,340.35457237)(618.86371933,340.2145743)
\curveto(618.91371556,340.15457257)(618.97871549,340.10957262)(619.05871933,340.0795743)
\lineto(619.32871933,340.0195743)
\curveto(619.35871511,340.00957272)(619.38871508,340.00457272)(619.41871933,340.0045743)
\curveto(619.45871501,340.01457271)(619.48871498,340.01457271)(619.50871933,340.0045743)
\lineto(619.67371933,340.0045743)
\curveto(619.78371469,340.00457272)(619.87871459,339.99957273)(619.95871933,339.9895743)
\curveto(620.03871443,339.97957275)(620.10371437,339.93957279)(620.15371933,339.8695743)
\curveto(620.19371428,339.80957292)(620.21371426,339.729573)(620.21371933,339.6295743)
\lineto(620.21371933,339.3445743)
\curveto(620.21371426,339.13457359)(620.20871426,338.93957379)(620.19871933,338.7595743)
\curveto(620.19871427,338.58957414)(620.11871435,338.47457425)(619.95871933,338.4145743)
\curveto(619.90871456,338.39457433)(619.86371461,338.38957434)(619.82371933,338.3995743)
\curveto(619.78371469,338.39957433)(619.73871473,338.38957434)(619.68871933,338.3695743)
\lineto(619.53871933,338.3695743)
\curveto(619.51871495,338.36957436)(619.48871498,338.37457435)(619.44871933,338.3845743)
\curveto(619.40871506,338.38457434)(619.3737151,338.37957435)(619.34371933,338.3695743)
\curveto(619.29371518,338.35957437)(619.23871523,338.35957437)(619.17871933,338.3695743)
\lineto(619.02871933,338.3695743)
\lineto(618.87871933,338.3695743)
\curveto(618.82871564,338.35957437)(618.78371569,338.35957437)(618.74371933,338.3695743)
\lineto(618.57871933,338.3695743)
\curveto(618.52871594,338.37957435)(618.473716,338.38457434)(618.41371933,338.3845743)
\curveto(618.35371612,338.38457434)(618.29871617,338.38957434)(618.24871933,338.3995743)
\curveto(618.17871629,338.40957432)(618.11371636,338.41957431)(618.05371933,338.4295743)
\lineto(617.87371933,338.4595743)
\curveto(617.76371671,338.48957424)(617.65871681,338.5245742)(617.55871933,338.5645743)
\curveto(617.45871701,338.60457412)(617.36371711,338.64957408)(617.27371933,338.6995743)
\lineto(617.18371933,338.7595743)
\curveto(617.15371732,338.78957394)(617.11871735,338.81957391)(617.07871933,338.8495743)
\curveto(617.05871741,338.86957386)(617.03371744,338.88957384)(617.00371933,338.9095743)
\lineto(616.92871933,338.9845743)
\curveto(616.78871768,339.17457355)(616.68371779,339.38457334)(616.61371933,339.6145743)
\curveto(616.59371788,339.65457307)(616.58371789,339.68957304)(616.58371933,339.7195743)
\curveto(616.59371788,339.75957297)(616.59371788,339.80457292)(616.58371933,339.8545743)
\curveto(616.5737179,339.87457285)(616.5687179,339.89957283)(616.56871933,339.9295743)
\curveto(616.5687179,339.95957277)(616.56371791,339.98457274)(616.55371933,340.0045743)
\lineto(616.55371933,340.1545743)
\curveto(616.54371793,340.19457253)(616.53871793,340.23957249)(616.53871933,340.2895743)
\curveto(616.54871792,340.33957239)(616.55371792,340.38957234)(616.55371933,340.4395743)
\lineto(616.55371933,341.0095743)
\lineto(616.55371933,343.2445743)
\lineto(616.55371933,344.0395743)
\lineto(616.55371933,344.2495743)
\curveto(616.56371791,344.31956841)(616.55871791,344.38456834)(616.53871933,344.4445743)
\curveto(616.49871797,344.58456814)(616.42871804,344.67456805)(616.32871933,344.7145743)
\curveto(616.21871825,344.76456796)(616.07871839,344.77956795)(615.90871933,344.7595743)
\curveto(615.73871873,344.73956799)(615.59371888,344.75456797)(615.47371933,344.8045743)
\curveto(615.39371908,344.83456789)(615.34371913,344.87956785)(615.32371933,344.9395743)
\curveto(615.30371917,344.99956773)(615.28371919,345.07456765)(615.26371933,345.1645743)
\lineto(615.26371933,345.4795743)
\curveto(615.26371921,345.65956707)(615.2737192,345.80456692)(615.29371933,345.9145743)
\curveto(615.31371916,346.0245667)(615.39871907,346.09956663)(615.54871933,346.1395743)
\curveto(615.58871888,346.15956657)(615.62871884,346.16456656)(615.66871933,346.1545743)
\lineto(615.80371933,346.1545743)
\curveto(615.95371852,346.15456657)(616.09371838,346.15956657)(616.22371933,346.1695743)
\curveto(616.35371812,346.18956654)(616.44371803,346.24956648)(616.49371933,346.3495743)
\curveto(616.52371795,346.41956631)(616.53871793,346.49956623)(616.53871933,346.5895743)
\curveto(616.54871792,346.67956605)(616.55371792,346.76956596)(616.55371933,346.8595743)
\lineto(616.55371933,347.7895743)
\lineto(616.55371933,348.0445743)
\curveto(616.55371792,348.13456459)(616.56371791,348.20956452)(616.58371933,348.2695743)
\curveto(616.63371784,348.36956436)(616.70871776,348.43456429)(616.80871933,348.4645743)
\curveto(616.82871764,348.47456425)(616.85371762,348.47456425)(616.88371933,348.4645743)
\curveto(616.92371755,348.46456426)(616.95371752,348.46956426)(616.97371933,348.4795743)
}
}
{
\newrgbcolor{curcolor}{0 0 0}
\pscustom[linestyle=none,fillstyle=solid,fillcolor=curcolor]
{
\newpath
\moveto(623.29715683,349.0195743)
\curveto(623.36715388,348.93956379)(623.40215384,348.81956391)(623.40215683,348.6595743)
\lineto(623.40215683,348.1945743)
\lineto(623.40215683,347.7895743)
\curveto(623.40215384,347.64956508)(623.36715388,347.55456517)(623.29715683,347.5045743)
\curveto(623.23715401,347.45456527)(623.15715409,347.4245653)(623.05715683,347.4145743)
\curveto(622.96715428,347.40456532)(622.86715438,347.39956533)(622.75715683,347.3995743)
\lineto(621.91715683,347.3995743)
\curveto(621.80715544,347.39956533)(621.70715554,347.40456532)(621.61715683,347.4145743)
\curveto(621.53715571,347.4245653)(621.46715578,347.45456527)(621.40715683,347.5045743)
\curveto(621.36715588,347.53456519)(621.33715591,347.58956514)(621.31715683,347.6695743)
\curveto(621.30715594,347.75956497)(621.29715595,347.85456487)(621.28715683,347.9545743)
\lineto(621.28715683,348.2845743)
\curveto(621.29715595,348.39456433)(621.30215594,348.48956424)(621.30215683,348.5695743)
\lineto(621.30215683,348.7795743)
\curveto(621.31215593,348.84956388)(621.33215591,348.90956382)(621.36215683,348.9595743)
\curveto(621.38215586,348.99956373)(621.40715584,349.0295637)(621.43715683,349.0495743)
\lineto(621.55715683,349.1095743)
\curveto(621.57715567,349.10956362)(621.60215564,349.10956362)(621.63215683,349.1095743)
\curveto(621.66215558,349.11956361)(621.68715556,349.1245636)(621.70715683,349.1245743)
\lineto(622.80215683,349.1245743)
\curveto(622.90215434,349.1245636)(622.99715425,349.11956361)(623.08715683,349.1095743)
\curveto(623.17715407,349.09956363)(623.247154,349.06956366)(623.29715683,349.0195743)
\moveto(623.40215683,339.2545743)
\curveto(623.40215384,339.05457367)(623.39715385,338.88457384)(623.38715683,338.7445743)
\curveto(623.37715387,338.60457412)(623.28715396,338.50957422)(623.11715683,338.4595743)
\curveto(623.05715419,338.43957429)(622.99215425,338.4295743)(622.92215683,338.4295743)
\curveto(622.85215439,338.43957429)(622.77715447,338.44457428)(622.69715683,338.4445743)
\lineto(621.85715683,338.4445743)
\curveto(621.76715548,338.44457428)(621.67715557,338.44957428)(621.58715683,338.4595743)
\curveto(621.50715574,338.46957426)(621.4471558,338.49957423)(621.40715683,338.5495743)
\curveto(621.3471559,338.61957411)(621.31215593,338.70457402)(621.30215683,338.8045743)
\lineto(621.30215683,339.1495743)
\lineto(621.30215683,345.4795743)
\lineto(621.30215683,345.7795743)
\curveto(621.30215594,345.87956685)(621.32215592,345.95956677)(621.36215683,346.0195743)
\curveto(621.42215582,346.08956664)(621.50715574,346.13456659)(621.61715683,346.1545743)
\curveto(621.63715561,346.16456656)(621.66215558,346.16456656)(621.69215683,346.1545743)
\curveto(621.73215551,346.15456657)(621.76215548,346.15956657)(621.78215683,346.1695743)
\lineto(622.53215683,346.1695743)
\lineto(622.72715683,346.1695743)
\curveto(622.80715444,346.17956655)(622.87215437,346.17956655)(622.92215683,346.1695743)
\lineto(623.04215683,346.1695743)
\curveto(623.10215414,346.14956658)(623.15715409,346.13456659)(623.20715683,346.1245743)
\curveto(623.25715399,346.11456661)(623.29715395,346.08456664)(623.32715683,346.0345743)
\curveto(623.36715388,345.98456674)(623.38715386,345.91456681)(623.38715683,345.8245743)
\curveto(623.39715385,345.73456699)(623.40215384,345.63956709)(623.40215683,345.5395743)
\lineto(623.40215683,339.2545743)
}
}
{
\newrgbcolor{curcolor}{0 0 0}
\pscustom[linestyle=none,fillstyle=solid,fillcolor=curcolor]
{
\newpath
\moveto(632.95434433,342.3895743)
\curveto(632.96433565,342.3295704)(632.96933564,342.23957049)(632.96934433,342.1195743)
\curveto(632.96933564,341.99957073)(632.95933565,341.91457081)(632.93934433,341.8645743)
\lineto(632.93934433,341.6695743)
\curveto(632.9093357,341.55957117)(632.88933572,341.45457127)(632.87934433,341.3545743)
\curveto(632.87933573,341.25457147)(632.86433575,341.15457157)(632.83434433,341.0545743)
\curveto(632.8143358,340.96457176)(632.79433582,340.86957186)(632.77434433,340.7695743)
\curveto(632.75433586,340.67957205)(632.72433589,340.58957214)(632.68434433,340.4995743)
\curveto(632.614336,340.3295724)(632.54433607,340.16957256)(632.47434433,340.0195743)
\curveto(632.40433621,339.87957285)(632.32433629,339.73957299)(632.23434433,339.5995743)
\curveto(632.17433644,339.50957322)(632.1093365,339.4245733)(632.03934433,339.3445743)
\curveto(631.97933663,339.27457345)(631.9093367,339.19957353)(631.82934433,339.1195743)
\lineto(631.72434433,339.0145743)
\curveto(631.67433694,338.96457376)(631.61933699,338.91957381)(631.55934433,338.8795743)
\lineto(631.40934433,338.7595743)
\curveto(631.32933728,338.69957403)(631.23933737,338.64457408)(631.13934433,338.5945743)
\curveto(631.04933756,338.55457417)(630.95433766,338.50957422)(630.85434433,338.4595743)
\curveto(630.75433786,338.40957432)(630.64933796,338.37457435)(630.53934433,338.3545743)
\curveto(630.43933817,338.33457439)(630.33433828,338.31457441)(630.22434433,338.2945743)
\curveto(630.16433845,338.27457445)(630.09933851,338.26457446)(630.02934433,338.2645743)
\curveto(629.96933864,338.26457446)(629.90433871,338.25457447)(629.83434433,338.2345743)
\lineto(629.69934433,338.2345743)
\curveto(629.61933899,338.21457451)(629.54433907,338.21457451)(629.47434433,338.2345743)
\lineto(629.32434433,338.2345743)
\curveto(629.26433935,338.25457447)(629.19933941,338.26457446)(629.12934433,338.2645743)
\curveto(629.06933954,338.25457447)(629.0093396,338.25957447)(628.94934433,338.2795743)
\curveto(628.78933982,338.3295744)(628.63433998,338.37457435)(628.48434433,338.4145743)
\curveto(628.34434027,338.45457427)(628.2143404,338.51457421)(628.09434433,338.5945743)
\curveto(628.02434059,338.63457409)(627.95934065,338.67457405)(627.89934433,338.7145743)
\curveto(627.83934077,338.76457396)(627.77434084,338.81457391)(627.70434433,338.8645743)
\lineto(627.52434433,338.9995743)
\curveto(627.44434117,339.05957367)(627.37434124,339.06457366)(627.31434433,339.0145743)
\curveto(627.26434135,338.98457374)(627.23934137,338.94457378)(627.23934433,338.8945743)
\curveto(627.23934137,338.85457387)(627.22934138,338.80457392)(627.20934433,338.7445743)
\curveto(627.18934142,338.64457408)(627.17934143,338.5295742)(627.17934433,338.3995743)
\curveto(627.18934142,338.26957446)(627.19434142,338.14957458)(627.19434433,338.0395743)
\lineto(627.19434433,336.5095743)
\curveto(627.19434142,336.37957635)(627.18934142,336.25457647)(627.17934433,336.1345743)
\curveto(627.17934143,336.00457672)(627.15434146,335.89957683)(627.10434433,335.8195743)
\curveto(627.07434154,335.77957695)(627.01934159,335.74957698)(626.93934433,335.7295743)
\curveto(626.85934175,335.70957702)(626.76934184,335.69957703)(626.66934433,335.6995743)
\curveto(626.56934204,335.68957704)(626.46934214,335.68957704)(626.36934433,335.6995743)
\lineto(626.11434433,335.6995743)
\lineto(625.70934433,335.6995743)
\lineto(625.60434433,335.6995743)
\curveto(625.56434305,335.69957703)(625.52934308,335.70457702)(625.49934433,335.7145743)
\lineto(625.37934433,335.7145743)
\curveto(625.2093434,335.76457696)(625.11934349,335.86457686)(625.10934433,336.0145743)
\curveto(625.09934351,336.15457657)(625.09434352,336.3245764)(625.09434433,336.5245743)
\lineto(625.09434433,345.3295743)
\curveto(625.09434352,345.43956729)(625.08934352,345.55456717)(625.07934433,345.6745743)
\curveto(625.07934353,345.80456692)(625.10434351,345.90456682)(625.15434433,345.9745743)
\curveto(625.19434342,346.04456668)(625.24934336,346.08956664)(625.31934433,346.1095743)
\curveto(625.36934324,346.1295666)(625.42934318,346.13956659)(625.49934433,346.1395743)
\lineto(625.72434433,346.1395743)
\lineto(626.44434433,346.1395743)
\lineto(626.72934433,346.1395743)
\curveto(626.81934179,346.13956659)(626.89434172,346.11456661)(626.95434433,346.0645743)
\curveto(627.02434159,346.01456671)(627.05934155,345.94956678)(627.05934433,345.8695743)
\curveto(627.06934154,345.79956693)(627.09434152,345.724567)(627.13434433,345.6445743)
\curveto(627.14434147,345.61456711)(627.15434146,345.58956714)(627.16434433,345.5695743)
\curveto(627.18434143,345.55956717)(627.20434141,345.54456718)(627.22434433,345.5245743)
\curveto(627.33434128,345.51456721)(627.42434119,345.54456718)(627.49434433,345.6145743)
\curveto(627.56434105,345.68456704)(627.63434098,345.74456698)(627.70434433,345.7945743)
\curveto(627.83434078,345.88456684)(627.96934064,345.96456676)(628.10934433,346.0345743)
\curveto(628.24934036,346.11456661)(628.40434021,346.17956655)(628.57434433,346.2295743)
\curveto(628.65433996,346.25956647)(628.73933987,346.27956645)(628.82934433,346.2895743)
\curveto(628.92933968,346.29956643)(629.02433959,346.31456641)(629.11434433,346.3345743)
\curveto(629.15433946,346.34456638)(629.19433942,346.34456638)(629.23434433,346.3345743)
\curveto(629.28433933,346.3245664)(629.32433929,346.3295664)(629.35434433,346.3495743)
\curveto(629.92433869,346.36956636)(630.40433821,346.28956644)(630.79434433,346.1095743)
\curveto(631.19433742,345.93956679)(631.53433708,345.71456701)(631.81434433,345.4345743)
\curveto(631.86433675,345.38456734)(631.9093367,345.33456739)(631.94934433,345.2845743)
\curveto(631.98933662,345.24456748)(632.02933658,345.19956753)(632.06934433,345.1495743)
\curveto(632.13933647,345.05956767)(632.19933641,344.96956776)(632.24934433,344.8795743)
\curveto(632.3093363,344.78956794)(632.36433625,344.69956803)(632.41434433,344.6095743)
\curveto(632.43433618,344.58956814)(632.44433617,344.56456816)(632.44434433,344.5345743)
\curveto(632.45433616,344.50456822)(632.46933614,344.46956826)(632.48934433,344.4295743)
\curveto(632.54933606,344.3295684)(632.60433601,344.20956852)(632.65434433,344.0695743)
\curveto(632.67433594,344.00956872)(632.69433592,343.94456878)(632.71434433,343.8745743)
\curveto(632.73433588,343.81456891)(632.75433586,343.74956898)(632.77434433,343.6795743)
\curveto(632.8143358,343.55956917)(632.83933577,343.43456929)(632.84934433,343.3045743)
\curveto(632.86933574,343.17456955)(632.89433572,343.03956969)(632.92434433,342.8995743)
\lineto(632.92434433,342.7345743)
\lineto(632.95434433,342.5545743)
\lineto(632.95434433,342.3895743)
\moveto(630.83934433,342.0445743)
\curveto(630.84933776,342.09457063)(630.85433776,342.15957057)(630.85434433,342.2395743)
\curveto(630.85433776,342.3295704)(630.84933776,342.39957033)(630.83934433,342.4495743)
\lineto(630.83934433,342.5845743)
\curveto(630.81933779,342.64457008)(630.8093378,342.70957002)(630.80934433,342.7795743)
\curveto(630.8093378,342.84956988)(630.79933781,342.91956981)(630.77934433,342.9895743)
\curveto(630.75933785,343.08956964)(630.73933787,343.18456954)(630.71934433,343.2745743)
\curveto(630.69933791,343.37456935)(630.66933794,343.46456926)(630.62934433,343.5445743)
\curveto(630.5093381,343.86456886)(630.35433826,344.11956861)(630.16434433,344.3095743)
\curveto(629.97433864,344.49956823)(629.70433891,344.63956809)(629.35434433,344.7295743)
\curveto(629.27433934,344.74956798)(629.18433943,344.75956797)(629.08434433,344.7595743)
\lineto(628.81434433,344.7595743)
\curveto(628.77433984,344.74956798)(628.73933987,344.74456798)(628.70934433,344.7445743)
\curveto(628.67933993,344.74456798)(628.64433997,344.73956799)(628.60434433,344.7295743)
\lineto(628.39434433,344.6695743)
\curveto(628.33434028,344.65956807)(628.27434034,344.63956809)(628.21434433,344.6095743)
\curveto(627.95434066,344.49956823)(627.74934086,344.3295684)(627.59934433,344.0995743)
\curveto(627.45934115,343.86956886)(627.34434127,343.61456911)(627.25434433,343.3345743)
\curveto(627.23434138,343.25456947)(627.21934139,343.16956956)(627.20934433,343.0795743)
\curveto(627.19934141,342.99956973)(627.18434143,342.91956981)(627.16434433,342.8395743)
\curveto(627.15434146,342.79956993)(627.14934146,342.73456999)(627.14934433,342.6445743)
\curveto(627.12934148,342.60457012)(627.12434149,342.55457017)(627.13434433,342.4945743)
\curveto(627.14434147,342.44457028)(627.14434147,342.39457033)(627.13434433,342.3445743)
\curveto(627.1143415,342.28457044)(627.1143415,342.2295705)(627.13434433,342.1795743)
\lineto(627.13434433,341.9995743)
\lineto(627.13434433,341.8645743)
\curveto(627.13434148,341.8245709)(627.14434147,341.78457094)(627.16434433,341.7445743)
\curveto(627.16434145,341.67457105)(627.16934144,341.61957111)(627.17934433,341.5795743)
\lineto(627.20934433,341.3995743)
\curveto(627.21934139,341.33957139)(627.23434138,341.27957145)(627.25434433,341.2195743)
\curveto(627.34434127,340.9295718)(627.44934116,340.68957204)(627.56934433,340.4995743)
\curveto(627.69934091,340.31957241)(627.87934073,340.15957257)(628.10934433,340.0195743)
\curveto(628.24934036,339.93957279)(628.4143402,339.87457285)(628.60434433,339.8245743)
\curveto(628.64433997,339.81457291)(628.67933993,339.80957292)(628.70934433,339.8095743)
\curveto(628.73933987,339.81957291)(628.77433984,339.81957291)(628.81434433,339.8095743)
\curveto(628.85433976,339.79957293)(628.9143397,339.78957294)(628.99434433,339.7795743)
\curveto(629.07433954,339.77957295)(629.13933947,339.78457294)(629.18934433,339.7945743)
\curveto(629.26933934,339.81457291)(629.34933926,339.8295729)(629.42934433,339.8395743)
\curveto(629.51933909,339.85957287)(629.60433901,339.88457284)(629.68434433,339.9145743)
\curveto(629.92433869,340.01457271)(630.11933849,340.15457257)(630.26934433,340.3345743)
\curveto(630.41933819,340.51457221)(630.54433807,340.724572)(630.64434433,340.9645743)
\curveto(630.69433792,341.08457164)(630.72933788,341.20957152)(630.74934433,341.3395743)
\curveto(630.76933784,341.46957126)(630.79433782,341.60457112)(630.82434433,341.7445743)
\lineto(630.82434433,341.8945743)
\curveto(630.83433778,341.94457078)(630.83933777,341.99457073)(630.83934433,342.0445743)
}
}
{
\newrgbcolor{curcolor}{0 0 0}
\pscustom[linestyle=none,fillstyle=solid,fillcolor=curcolor]
{
\newpath
\moveto(642.0042662,342.6145743)
\curveto(642.02425763,342.55457017)(642.03425762,342.46957026)(642.0342662,342.3595743)
\curveto(642.03425762,342.24957048)(642.02425763,342.16457056)(642.0042662,342.1045743)
\lineto(642.0042662,341.9545743)
\curveto(641.98425767,341.87457085)(641.97425768,341.79457093)(641.9742662,341.7145743)
\curveto(641.98425767,341.63457109)(641.97925768,341.55457117)(641.9592662,341.4745743)
\curveto(641.93925772,341.40457132)(641.92425773,341.33957139)(641.9142662,341.2795743)
\curveto(641.90425775,341.21957151)(641.89425776,341.15457157)(641.8842662,341.0845743)
\curveto(641.84425781,340.97457175)(641.80925785,340.85957187)(641.7792662,340.7395743)
\curveto(641.74925791,340.6295721)(641.70925795,340.5245722)(641.6592662,340.4245743)
\curveto(641.44925821,339.94457278)(641.17425848,339.55457317)(640.8342662,339.2545743)
\curveto(640.49425916,338.95457377)(640.08425957,338.70457402)(639.6042662,338.5045743)
\curveto(639.48426017,338.45457427)(639.3592603,338.41957431)(639.2292662,338.3995743)
\curveto(639.10926055,338.36957436)(638.98426067,338.33957439)(638.8542662,338.3095743)
\curveto(638.80426085,338.28957444)(638.74926091,338.27957445)(638.6892662,338.2795743)
\curveto(638.62926103,338.27957445)(638.57426108,338.27457445)(638.5242662,338.2645743)
\lineto(638.4192662,338.2645743)
\curveto(638.38926127,338.25457447)(638.3592613,338.24957448)(638.3292662,338.2495743)
\curveto(638.27926138,338.23957449)(638.19926146,338.23457449)(638.0892662,338.2345743)
\curveto(637.97926168,338.2245745)(637.89426176,338.2295745)(637.8342662,338.2495743)
\lineto(637.6842662,338.2495743)
\curveto(637.63426202,338.25957447)(637.57926208,338.26457446)(637.5192662,338.2645743)
\curveto(637.46926219,338.25457447)(637.41926224,338.25957447)(637.3692662,338.2795743)
\curveto(637.32926233,338.28957444)(637.28926237,338.29457443)(637.2492662,338.2945743)
\curveto(637.21926244,338.29457443)(637.17926248,338.29957443)(637.1292662,338.3095743)
\curveto(637.02926263,338.33957439)(636.92926273,338.36457436)(636.8292662,338.3845743)
\curveto(636.72926293,338.40457432)(636.63426302,338.43457429)(636.5442662,338.4745743)
\curveto(636.42426323,338.51457421)(636.30926335,338.55457417)(636.1992662,338.5945743)
\curveto(636.09926356,338.63457409)(635.99426366,338.68457404)(635.8842662,338.7445743)
\curveto(635.53426412,338.95457377)(635.23426442,339.19957353)(634.9842662,339.4795743)
\curveto(634.73426492,339.75957297)(634.52426513,340.09457263)(634.3542662,340.4845743)
\curveto(634.30426535,340.57457215)(634.26426539,340.66957206)(634.2342662,340.7695743)
\curveto(634.21426544,340.86957186)(634.18926547,340.97457175)(634.1592662,341.0845743)
\curveto(634.13926552,341.13457159)(634.12926553,341.17957155)(634.1292662,341.2195743)
\curveto(634.12926553,341.25957147)(634.11926554,341.30457142)(634.0992662,341.3545743)
\curveto(634.07926558,341.43457129)(634.06926559,341.51457121)(634.0692662,341.5945743)
\curveto(634.06926559,341.68457104)(634.0592656,341.76957096)(634.0392662,341.8495743)
\curveto(634.02926563,341.89957083)(634.02426563,341.94457078)(634.0242662,341.9845743)
\lineto(634.0242662,342.1195743)
\curveto(634.00426565,342.17957055)(633.99426566,342.26457046)(633.9942662,342.3745743)
\curveto(634.00426565,342.48457024)(634.01926564,342.56957016)(634.0392662,342.6295743)
\lineto(634.0392662,342.7345743)
\curveto(634.04926561,342.78456994)(634.04926561,342.83456989)(634.0392662,342.8845743)
\curveto(634.03926562,342.94456978)(634.04926561,342.99956973)(634.0692662,343.0495743)
\curveto(634.07926558,343.09956963)(634.08426557,343.14456958)(634.0842662,343.1845743)
\curveto(634.08426557,343.23456949)(634.09426556,343.28456944)(634.1142662,343.3345743)
\curveto(634.1542655,343.46456926)(634.18926547,343.58956914)(634.2192662,343.7095743)
\curveto(634.24926541,343.83956889)(634.28926537,343.96456876)(634.3392662,344.0845743)
\curveto(634.51926514,344.49456823)(634.73426492,344.83456789)(634.9842662,345.1045743)
\curveto(635.23426442,345.38456734)(635.53926412,345.63956709)(635.8992662,345.8695743)
\curveto(635.99926366,345.91956681)(636.10426355,345.96456676)(636.2142662,346.0045743)
\curveto(636.32426333,346.04456668)(636.43426322,346.08956664)(636.5442662,346.1395743)
\curveto(636.67426298,346.18956654)(636.80926285,346.2245665)(636.9492662,346.2445743)
\curveto(637.08926257,346.26456646)(637.23426242,346.29456643)(637.3842662,346.3345743)
\curveto(637.46426219,346.34456638)(637.53926212,346.34956638)(637.6092662,346.3495743)
\curveto(637.67926198,346.34956638)(637.74926191,346.35456637)(637.8192662,346.3645743)
\curveto(638.39926126,346.37456635)(638.89926076,346.31456641)(639.3192662,346.1845743)
\curveto(639.74925991,346.05456667)(640.12925953,345.87456685)(640.4592662,345.6445743)
\curveto(640.56925909,345.56456716)(640.67925898,345.47456725)(640.7892662,345.3745743)
\curveto(640.90925875,345.28456744)(641.00925865,345.18456754)(641.0892662,345.0745743)
\curveto(641.16925849,344.97456775)(641.23925842,344.87456785)(641.2992662,344.7745743)
\curveto(641.36925829,344.67456805)(641.43925822,344.56956816)(641.5092662,344.4595743)
\curveto(641.57925808,344.34956838)(641.63425802,344.2295685)(641.6742662,344.0995743)
\curveto(641.71425794,343.97956875)(641.7592579,343.84956888)(641.8092662,343.7095743)
\curveto(641.83925782,343.6295691)(641.86425779,343.54456918)(641.8842662,343.4545743)
\lineto(641.9442662,343.1845743)
\curveto(641.9542577,343.14456958)(641.9592577,343.10456962)(641.9592662,343.0645743)
\curveto(641.9592577,343.0245697)(641.96425769,342.98456974)(641.9742662,342.9445743)
\curveto(641.99425766,342.89456983)(641.99925766,342.83956989)(641.9892662,342.7795743)
\curveto(641.97925768,342.71957001)(641.98425767,342.66457006)(642.0042662,342.6145743)
\moveto(639.9042662,342.0745743)
\curveto(639.91425974,342.1245706)(639.91925974,342.19457053)(639.9192662,342.2845743)
\curveto(639.91925974,342.38457034)(639.91425974,342.45957027)(639.9042662,342.5095743)
\lineto(639.9042662,342.6295743)
\curveto(639.88425977,342.67957005)(639.87425978,342.73456999)(639.8742662,342.7945743)
\curveto(639.87425978,342.85456987)(639.86925979,342.90956982)(639.8592662,342.9595743)
\curveto(639.8592598,342.99956973)(639.8542598,343.0295697)(639.8442662,343.0495743)
\lineto(639.7842662,343.2895743)
\curveto(639.77425988,343.37956935)(639.7542599,343.46456926)(639.7242662,343.5445743)
\curveto(639.61426004,343.80456892)(639.48426017,344.0245687)(639.3342662,344.2045743)
\curveto(639.18426047,344.39456833)(638.98426067,344.54456818)(638.7342662,344.6545743)
\curveto(638.67426098,344.67456805)(638.61426104,344.68956804)(638.5542662,344.6995743)
\curveto(638.49426116,344.71956801)(638.42926123,344.73956799)(638.3592662,344.7595743)
\curveto(638.27926138,344.77956795)(638.19426146,344.78456794)(638.1042662,344.7745743)
\lineto(637.8342662,344.7745743)
\curveto(637.80426185,344.75456797)(637.76926189,344.74456798)(637.7292662,344.7445743)
\curveto(637.68926197,344.75456797)(637.654262,344.75456797)(637.6242662,344.7445743)
\lineto(637.4142662,344.6845743)
\curveto(637.3542623,344.67456805)(637.29926236,344.65456807)(637.2492662,344.6245743)
\curveto(636.99926266,344.51456821)(636.79426286,344.35456837)(636.6342662,344.1445743)
\curveto(636.48426317,343.94456878)(636.36426329,343.70956902)(636.2742662,343.4395743)
\curveto(636.24426341,343.33956939)(636.21926344,343.23456949)(636.1992662,343.1245743)
\curveto(636.18926347,343.01456971)(636.17426348,342.90456982)(636.1542662,342.7945743)
\curveto(636.14426351,342.74456998)(636.13926352,342.69457003)(636.1392662,342.6445743)
\lineto(636.1392662,342.4945743)
\curveto(636.11926354,342.4245703)(636.10926355,342.31957041)(636.1092662,342.1795743)
\curveto(636.11926354,342.03957069)(636.13426352,341.93457079)(636.1542662,341.8645743)
\lineto(636.1542662,341.7295743)
\curveto(636.17426348,341.64957108)(636.18926347,341.56957116)(636.1992662,341.4895743)
\curveto(636.20926345,341.41957131)(636.22426343,341.34457138)(636.2442662,341.2645743)
\curveto(636.34426331,340.96457176)(636.44926321,340.71957201)(636.5592662,340.5295743)
\curveto(636.67926298,340.34957238)(636.86426279,340.18457254)(637.1142662,340.0345743)
\curveto(637.18426247,339.98457274)(637.2592624,339.94457278)(637.3392662,339.9145743)
\curveto(637.42926223,339.88457284)(637.51926214,339.85957287)(637.6092662,339.8395743)
\curveto(637.64926201,339.8295729)(637.68426197,339.8245729)(637.7142662,339.8245743)
\curveto(637.74426191,339.83457289)(637.77926188,339.83457289)(637.8192662,339.8245743)
\lineto(637.9392662,339.7945743)
\curveto(637.98926167,339.79457293)(638.03426162,339.79957293)(638.0742662,339.8095743)
\lineto(638.1942662,339.8095743)
\curveto(638.27426138,339.8295729)(638.3542613,339.84457288)(638.4342662,339.8545743)
\curveto(638.51426114,339.86457286)(638.58926107,339.88457284)(638.6592662,339.9145743)
\curveto(638.91926074,340.01457271)(639.12926053,340.14957258)(639.2892662,340.3195743)
\curveto(639.44926021,340.48957224)(639.58426007,340.69957203)(639.6942662,340.9495743)
\curveto(639.73425992,341.04957168)(639.76425989,341.14957158)(639.7842662,341.2495743)
\curveto(639.80425985,341.34957138)(639.82925983,341.45457127)(639.8592662,341.5645743)
\curveto(639.86925979,341.60457112)(639.87425978,341.63957109)(639.8742662,341.6695743)
\curveto(639.87425978,341.70957102)(639.87925978,341.74957098)(639.8892662,341.7895743)
\lineto(639.8892662,341.9245743)
\curveto(639.88925977,341.97457075)(639.89425976,342.0245707)(639.9042662,342.0745743)
}
}
{
\newrgbcolor{curcolor}{0 0 0}
\pscustom[linestyle=none,fillstyle=solid,fillcolor=curcolor]
{
}
}
{
\newrgbcolor{curcolor}{0 0 0}
\pscustom[linestyle=none,fillstyle=solid,fillcolor=curcolor]
{
\newpath
\moveto(655.15434433,339.2845743)
\lineto(655.15434433,338.8645743)
\curveto(655.15433596,338.73457399)(655.12433599,338.6295741)(655.06434433,338.5495743)
\curveto(655.0143361,338.49957423)(654.94933616,338.46457426)(654.86934433,338.4445743)
\curveto(654.78933632,338.43457429)(654.69933641,338.4295743)(654.59934433,338.4295743)
\lineto(653.77434433,338.4295743)
\lineto(653.48934433,338.4295743)
\curveto(653.4093377,338.43957429)(653.34433777,338.46457426)(653.29434433,338.5045743)
\curveto(653.22433789,338.55457417)(653.18433793,338.61957411)(653.17434433,338.6995743)
\curveto(653.16433795,338.77957395)(653.14433797,338.85957387)(653.11434433,338.9395743)
\curveto(653.09433802,338.95957377)(653.07433804,338.97457375)(653.05434433,338.9845743)
\curveto(653.04433807,339.00457372)(653.02933808,339.0245737)(653.00934433,339.0445743)
\curveto(652.89933821,339.04457368)(652.81933829,339.01957371)(652.76934433,338.9695743)
\lineto(652.61934433,338.8195743)
\curveto(652.54933856,338.76957396)(652.48433863,338.724574)(652.42434433,338.6845743)
\curveto(652.36433875,338.65457407)(652.29933881,338.61457411)(652.22934433,338.5645743)
\curveto(652.18933892,338.54457418)(652.14433897,338.5245742)(652.09434433,338.5045743)
\curveto(652.05433906,338.48457424)(652.0093391,338.46457426)(651.95934433,338.4445743)
\curveto(651.81933929,338.39457433)(651.66933944,338.34957438)(651.50934433,338.3095743)
\curveto(651.45933965,338.28957444)(651.4143397,338.27957445)(651.37434433,338.2795743)
\curveto(651.33433978,338.27957445)(651.29433982,338.27457445)(651.25434433,338.2645743)
\lineto(651.11934433,338.2645743)
\curveto(651.08934002,338.25457447)(651.04934006,338.24957448)(650.99934433,338.2495743)
\lineto(650.86434433,338.2495743)
\curveto(650.80434031,338.2295745)(650.7143404,338.2245745)(650.59434433,338.2345743)
\curveto(650.47434064,338.23457449)(650.38934072,338.24457448)(650.33934433,338.2645743)
\curveto(650.26934084,338.28457444)(650.20434091,338.29457443)(650.14434433,338.2945743)
\curveto(650.09434102,338.28457444)(650.03934107,338.28957444)(649.97934433,338.3095743)
\lineto(649.61934433,338.4295743)
\curveto(649.5093416,338.45957427)(649.39934171,338.49957423)(649.28934433,338.5495743)
\curveto(648.93934217,338.69957403)(648.62434249,338.9295738)(648.34434433,339.2395743)
\curveto(648.07434304,339.55957317)(647.85934325,339.89457283)(647.69934433,340.2445743)
\curveto(647.64934346,340.35457237)(647.6093435,340.45957227)(647.57934433,340.5595743)
\curveto(647.54934356,340.66957206)(647.5143436,340.77957195)(647.47434433,340.8895743)
\curveto(647.46434365,340.9295718)(647.45934365,340.96457176)(647.45934433,340.9945743)
\curveto(647.45934365,341.03457169)(647.44934366,341.07957165)(647.42934433,341.1295743)
\curveto(647.4093437,341.20957152)(647.38934372,341.29457143)(647.36934433,341.3845743)
\curveto(647.35934375,341.48457124)(647.34434377,341.58457114)(647.32434433,341.6845743)
\curveto(647.3143438,341.71457101)(647.3093438,341.74957098)(647.30934433,341.7895743)
\curveto(647.31934379,341.8295709)(647.31934379,341.86457086)(647.30934433,341.8945743)
\lineto(647.30934433,342.0295743)
\curveto(647.3093438,342.07957065)(647.30434381,342.1295706)(647.29434433,342.1795743)
\curveto(647.28434383,342.2295705)(647.27934383,342.28457044)(647.27934433,342.3445743)
\curveto(647.27934383,342.41457031)(647.28434383,342.46957026)(647.29434433,342.5095743)
\curveto(647.30434381,342.55957017)(647.3093438,342.60457012)(647.30934433,342.6445743)
\lineto(647.30934433,342.7945743)
\curveto(647.31934379,342.84456988)(647.31934379,342.88956984)(647.30934433,342.9295743)
\curveto(647.3093438,342.97956975)(647.31934379,343.0295697)(647.33934433,343.0795743)
\curveto(647.35934375,343.18956954)(647.37434374,343.29456943)(647.38434433,343.3945743)
\curveto(647.40434371,343.49456923)(647.42934368,343.59456913)(647.45934433,343.6945743)
\curveto(647.49934361,343.81456891)(647.53434358,343.9295688)(647.56434433,344.0395743)
\curveto(647.59434352,344.14956858)(647.63434348,344.25956847)(647.68434433,344.3695743)
\curveto(647.82434329,344.66956806)(647.99934311,344.95456777)(648.20934433,345.2245743)
\curveto(648.22934288,345.25456747)(648.25434286,345.27956745)(648.28434433,345.2995743)
\curveto(648.32434279,345.3295674)(648.35434276,345.35956737)(648.37434433,345.3895743)
\curveto(648.4143427,345.43956729)(648.45434266,345.48456724)(648.49434433,345.5245743)
\curveto(648.53434258,345.56456716)(648.57934253,345.60456712)(648.62934433,345.6445743)
\curveto(648.66934244,345.66456706)(648.70434241,345.68956704)(648.73434433,345.7195743)
\curveto(648.76434235,345.75956697)(648.79934231,345.78956694)(648.83934433,345.8095743)
\curveto(649.08934202,345.97956675)(649.37934173,346.11956661)(649.70934433,346.2295743)
\curveto(649.77934133,346.24956648)(649.84934126,346.26456646)(649.91934433,346.2745743)
\curveto(649.99934111,346.28456644)(650.07934103,346.29956643)(650.15934433,346.3195743)
\curveto(650.22934088,346.33956639)(650.31934079,346.34956638)(650.42934433,346.3495743)
\curveto(650.53934057,346.35956637)(650.64934046,346.36456636)(650.75934433,346.3645743)
\curveto(650.86934024,346.36456636)(650.97434014,346.35956637)(651.07434433,346.3495743)
\curveto(651.18433993,346.33956639)(651.27433984,346.3245664)(651.34434433,346.3045743)
\curveto(651.49433962,346.25456647)(651.63933947,346.20956652)(651.77934433,346.1695743)
\curveto(651.91933919,346.1295666)(652.04933906,346.07456665)(652.16934433,346.0045743)
\curveto(652.23933887,345.95456677)(652.30433881,345.90456682)(652.36434433,345.8545743)
\curveto(652.42433869,345.81456691)(652.48933862,345.76956696)(652.55934433,345.7195743)
\curveto(652.59933851,345.68956704)(652.65433846,345.64956708)(652.72434433,345.5995743)
\curveto(652.80433831,345.54956718)(652.87933823,345.54956718)(652.94934433,345.5995743)
\curveto(652.98933812,345.61956711)(653.0093381,345.65456707)(653.00934433,345.7045743)
\curveto(653.0093381,345.75456697)(653.01933809,345.80456692)(653.03934433,345.8545743)
\lineto(653.03934433,346.0045743)
\curveto(653.04933806,346.03456669)(653.05433806,346.06956666)(653.05434433,346.1095743)
\lineto(653.05434433,346.2295743)
\lineto(653.05434433,348.2695743)
\curveto(653.05433806,348.37956435)(653.04933806,348.49956423)(653.03934433,348.6295743)
\curveto(653.03933807,348.76956396)(653.06433805,348.87456385)(653.11434433,348.9445743)
\curveto(653.15433796,349.0245637)(653.22933788,349.07456365)(653.33934433,349.0945743)
\curveto(653.35933775,349.10456362)(653.37933773,349.10456362)(653.39934433,349.0945743)
\curveto(653.41933769,349.09456363)(653.43933767,349.09956363)(653.45934433,349.1095743)
\lineto(654.52434433,349.1095743)
\curveto(654.64433647,349.10956362)(654.75433636,349.10456362)(654.85434433,349.0945743)
\curveto(654.95433616,349.08456364)(655.02933608,349.04456368)(655.07934433,348.9745743)
\curveto(655.12933598,348.89456383)(655.15433596,348.78956394)(655.15434433,348.6595743)
\lineto(655.15434433,348.2995743)
\lineto(655.15434433,339.2845743)
\moveto(653.11434433,342.2245743)
\curveto(653.12433799,342.26457046)(653.12433799,342.30457042)(653.11434433,342.3445743)
\lineto(653.11434433,342.4795743)
\curveto(653.114338,342.57957015)(653.109338,342.67957005)(653.09934433,342.7795743)
\curveto(653.08933802,342.87956985)(653.07433804,342.96956976)(653.05434433,343.0495743)
\curveto(653.03433808,343.15956957)(653.0143381,343.25956947)(652.99434433,343.3495743)
\curveto(652.98433813,343.43956929)(652.95933815,343.5245692)(652.91934433,343.6045743)
\curveto(652.77933833,343.96456876)(652.57433854,344.24956848)(652.30434433,344.4595743)
\curveto(652.04433907,344.66956806)(651.66433945,344.77456795)(651.16434433,344.7745743)
\curveto(651.10434001,344.77456795)(651.02434009,344.76456796)(650.92434433,344.7445743)
\curveto(650.84434027,344.724568)(650.76934034,344.70456802)(650.69934433,344.6845743)
\curveto(650.63934047,344.67456805)(650.57934053,344.65456807)(650.51934433,344.6245743)
\curveto(650.24934086,344.51456821)(650.03934107,344.34456838)(649.88934433,344.1145743)
\curveto(649.73934137,343.88456884)(649.61934149,343.6245691)(649.52934433,343.3345743)
\curveto(649.49934161,343.23456949)(649.47934163,343.13456959)(649.46934433,343.0345743)
\curveto(649.45934165,342.93456979)(649.43934167,342.8295699)(649.40934433,342.7195743)
\lineto(649.40934433,342.5095743)
\curveto(649.38934172,342.41957031)(649.38434173,342.29457043)(649.39434433,342.1345743)
\curveto(649.40434171,341.98457074)(649.41934169,341.87457085)(649.43934433,341.8045743)
\lineto(649.43934433,341.7145743)
\curveto(649.44934166,341.69457103)(649.45434166,341.67457105)(649.45434433,341.6545743)
\curveto(649.47434164,341.57457115)(649.48934162,341.49957123)(649.49934433,341.4295743)
\curveto(649.51934159,341.35957137)(649.53934157,341.28457144)(649.55934433,341.2045743)
\curveto(649.72934138,340.68457204)(650.01934109,340.29957243)(650.42934433,340.0495743)
\curveto(650.55934055,339.95957277)(650.73934037,339.88957284)(650.96934433,339.8395743)
\curveto(651.0093401,339.8295729)(651.06934004,339.8245729)(651.14934433,339.8245743)
\curveto(651.17933993,339.81457291)(651.22433989,339.80457292)(651.28434433,339.7945743)
\curveto(651.35433976,339.79457293)(651.4093397,339.79957293)(651.44934433,339.8095743)
\curveto(651.52933958,339.8295729)(651.6093395,339.84457288)(651.68934433,339.8545743)
\curveto(651.76933934,339.86457286)(651.84933926,339.88457284)(651.92934433,339.9145743)
\curveto(652.17933893,340.0245727)(652.37933873,340.16457256)(652.52934433,340.3345743)
\curveto(652.67933843,340.50457222)(652.8093383,340.71957201)(652.91934433,340.9795743)
\curveto(652.95933815,341.06957166)(652.98933812,341.15957157)(653.00934433,341.2495743)
\curveto(653.02933808,341.34957138)(653.04933806,341.45457127)(653.06934433,341.5645743)
\curveto(653.07933803,341.61457111)(653.07933803,341.65957107)(653.06934433,341.6995743)
\curveto(653.06933804,341.74957098)(653.07933803,341.79957093)(653.09934433,341.8495743)
\curveto(653.109338,341.87957085)(653.114338,341.91457081)(653.11434433,341.9545743)
\lineto(653.11434433,342.0895743)
\lineto(653.11434433,342.2245743)
}
}
{
\newrgbcolor{curcolor}{0 0 0}
\pscustom[linestyle=none,fillstyle=solid,fillcolor=curcolor]
{
\newpath
\moveto(664.0992662,342.3745743)
\curveto(664.11925804,342.29457043)(664.11925804,342.20457052)(664.0992662,342.1045743)
\curveto(664.07925808,342.00457072)(664.04425811,341.93957079)(663.9942662,341.9095743)
\curveto(663.94425821,341.86957086)(663.86925829,341.83957089)(663.7692662,341.8195743)
\curveto(663.67925848,341.80957092)(663.57425858,341.79957093)(663.4542662,341.7895743)
\lineto(663.1092662,341.7895743)
\curveto(662.99925916,341.79957093)(662.89925926,341.80457092)(662.8092662,341.8045743)
\lineto(659.1492662,341.8045743)
\lineto(658.9392662,341.8045743)
\curveto(658.87926328,341.80457092)(658.82426333,341.79457093)(658.7742662,341.7745743)
\curveto(658.69426346,341.73457099)(658.64426351,341.69457103)(658.6242662,341.6545743)
\curveto(658.60426355,341.63457109)(658.58426357,341.59457113)(658.5642662,341.5345743)
\curveto(658.54426361,341.48457124)(658.53926362,341.43457129)(658.5492662,341.3845743)
\curveto(658.56926359,341.3245714)(658.57926358,341.26457146)(658.5792662,341.2045743)
\curveto(658.58926357,341.15457157)(658.60426355,341.09957163)(658.6242662,341.0395743)
\curveto(658.70426345,340.79957193)(658.79926336,340.59957213)(658.9092662,340.4395743)
\curveto(659.02926313,340.28957244)(659.18926297,340.15457257)(659.3892662,340.0345743)
\curveto(659.46926269,339.98457274)(659.54926261,339.94957278)(659.6292662,339.9295743)
\curveto(659.71926244,339.91957281)(659.80926235,339.89957283)(659.8992662,339.8695743)
\curveto(659.97926218,339.84957288)(660.08926207,339.83457289)(660.2292662,339.8245743)
\curveto(660.36926179,339.81457291)(660.48926167,339.81957291)(660.5892662,339.8395743)
\lineto(660.7242662,339.8395743)
\curveto(660.82426133,339.85957287)(660.91426124,339.87957285)(660.9942662,339.8995743)
\curveto(661.08426107,339.9295728)(661.16926099,339.95957277)(661.2492662,339.9895743)
\curveto(661.34926081,340.03957269)(661.4592607,340.10457262)(661.5792662,340.1845743)
\curveto(661.70926045,340.26457246)(661.80426035,340.34457238)(661.8642662,340.4245743)
\curveto(661.91426024,340.49457223)(661.96426019,340.55957217)(662.0142662,340.6195743)
\curveto(662.07426008,340.68957204)(662.14426001,340.73957199)(662.2242662,340.7695743)
\curveto(662.32425983,340.81957191)(662.44925971,340.83957189)(662.5992662,340.8295743)
\lineto(663.0342662,340.8295743)
\lineto(663.2142662,340.8295743)
\curveto(663.28425887,340.83957189)(663.34425881,340.83457189)(663.3942662,340.8145743)
\lineto(663.5442662,340.8145743)
\curveto(663.64425851,340.79457193)(663.71425844,340.76957196)(663.7542662,340.7395743)
\curveto(663.79425836,340.71957201)(663.81425834,340.67457205)(663.8142662,340.6045743)
\curveto(663.82425833,340.53457219)(663.81925834,340.47457225)(663.7992662,340.4245743)
\curveto(663.74925841,340.28457244)(663.69425846,340.15957257)(663.6342662,340.0495743)
\curveto(663.57425858,339.93957279)(663.50425865,339.8295729)(663.4242662,339.7195743)
\curveto(663.20425895,339.38957334)(662.9542592,339.1245736)(662.6742662,338.9245743)
\curveto(662.39425976,338.724574)(662.04426011,338.55457417)(661.6242662,338.4145743)
\curveto(661.51426064,338.37457435)(661.40426075,338.34957438)(661.2942662,338.3395743)
\curveto(661.18426097,338.3295744)(661.06926109,338.30957442)(660.9492662,338.2795743)
\curveto(660.90926125,338.26957446)(660.86426129,338.26957446)(660.8142662,338.2795743)
\curveto(660.77426138,338.27957445)(660.73426142,338.27457445)(660.6942662,338.2645743)
\lineto(660.5292662,338.2645743)
\curveto(660.47926168,338.24457448)(660.41926174,338.23957449)(660.3492662,338.2495743)
\curveto(660.28926187,338.24957448)(660.23426192,338.25457447)(660.1842662,338.2645743)
\curveto(660.10426205,338.27457445)(660.03426212,338.27457445)(659.9742662,338.2645743)
\curveto(659.91426224,338.25457447)(659.84926231,338.25957447)(659.7792662,338.2795743)
\curveto(659.72926243,338.29957443)(659.67426248,338.30957442)(659.6142662,338.3095743)
\curveto(659.5542626,338.30957442)(659.49926266,338.31957441)(659.4492662,338.3395743)
\curveto(659.33926282,338.35957437)(659.22926293,338.38457434)(659.1192662,338.4145743)
\curveto(659.00926315,338.43457429)(658.90926325,338.46957426)(658.8192662,338.5195743)
\curveto(658.70926345,338.55957417)(658.60426355,338.59457413)(658.5042662,338.6245743)
\curveto(658.41426374,338.66457406)(658.32926383,338.70957402)(658.2492662,338.7595743)
\curveto(657.92926423,338.95957377)(657.64426451,339.18957354)(657.3942662,339.4495743)
\curveto(657.14426501,339.71957301)(656.93926522,340.0295727)(656.7792662,340.3795743)
\curveto(656.72926543,340.48957224)(656.68926547,340.59957213)(656.6592662,340.7095743)
\curveto(656.62926553,340.8295719)(656.58926557,340.94957178)(656.5392662,341.0695743)
\curveto(656.52926563,341.10957162)(656.52426563,341.14457158)(656.5242662,341.1745743)
\curveto(656.52426563,341.21457151)(656.51926564,341.25457147)(656.5092662,341.2945743)
\curveto(656.46926569,341.41457131)(656.44426571,341.54457118)(656.4342662,341.6845743)
\lineto(656.4042662,342.1045743)
\curveto(656.40426575,342.15457057)(656.39926576,342.20957052)(656.3892662,342.2695743)
\curveto(656.38926577,342.3295704)(656.39426576,342.38457034)(656.4042662,342.4345743)
\lineto(656.4042662,342.6145743)
\lineto(656.4492662,342.9745743)
\curveto(656.48926567,343.14456958)(656.52426563,343.30956942)(656.5542662,343.4695743)
\curveto(656.58426557,343.6295691)(656.62926553,343.77956895)(656.6892662,343.9195743)
\curveto(657.11926504,344.95956777)(657.84926431,345.69456703)(658.8792662,346.1245743)
\curveto(659.01926314,346.18456654)(659.159263,346.2245665)(659.2992662,346.2445743)
\curveto(659.44926271,346.27456645)(659.60426255,346.30956642)(659.7642662,346.3495743)
\curveto(659.84426231,346.35956637)(659.91926224,346.36456636)(659.9892662,346.3645743)
\curveto(660.0592621,346.36456636)(660.13426202,346.36956636)(660.2142662,346.3795743)
\curveto(660.72426143,346.38956634)(661.159261,346.3295664)(661.5192662,346.1995743)
\curveto(661.88926027,346.07956665)(662.21925994,345.91956681)(662.5092662,345.7195743)
\curveto(662.59925956,345.65956707)(662.68925947,345.58956714)(662.7792662,345.5095743)
\curveto(662.86925929,345.43956729)(662.94925921,345.36456736)(663.0192662,345.2845743)
\curveto(663.04925911,345.23456749)(663.08925907,345.19456753)(663.1392662,345.1645743)
\curveto(663.21925894,345.05456767)(663.29425886,344.93956779)(663.3642662,344.8195743)
\curveto(663.43425872,344.70956802)(663.50925865,344.59456813)(663.5892662,344.4745743)
\curveto(663.63925852,344.38456834)(663.67925848,344.28956844)(663.7092662,344.1895743)
\curveto(663.74925841,344.09956863)(663.78925837,343.99956873)(663.8292662,343.8895743)
\curveto(663.87925828,343.75956897)(663.91925824,343.6245691)(663.9492662,343.4845743)
\curveto(663.97925818,343.34456938)(664.01425814,343.20456952)(664.0542662,343.0645743)
\curveto(664.07425808,342.98456974)(664.07925808,342.89456983)(664.0692662,342.7945743)
\curveto(664.06925809,342.70457002)(664.07925808,342.61957011)(664.0992662,342.5395743)
\lineto(664.0992662,342.3745743)
\moveto(661.8492662,343.2595743)
\curveto(661.91926024,343.35956937)(661.92426023,343.47956925)(661.8642662,343.6195743)
\curveto(661.81426034,343.76956896)(661.77426038,343.87956885)(661.7442662,343.9495743)
\curveto(661.60426055,344.21956851)(661.41926074,344.4245683)(661.1892662,344.5645743)
\curveto(660.9592612,344.71456801)(660.63926152,344.79456793)(660.2292662,344.8045743)
\curveto(660.19926196,344.78456794)(660.16426199,344.77956795)(660.1242662,344.7895743)
\curveto(660.08426207,344.79956793)(660.04926211,344.79956793)(660.0192662,344.7895743)
\curveto(659.96926219,344.76956796)(659.91426224,344.75456797)(659.8542662,344.7445743)
\curveto(659.79426236,344.74456798)(659.73926242,344.73456799)(659.6892662,344.7145743)
\curveto(659.24926291,344.57456815)(658.92426323,344.29956843)(658.7142662,343.8895743)
\curveto(658.69426346,343.84956888)(658.66926349,343.79456893)(658.6392662,343.7245743)
\curveto(658.61926354,343.66456906)(658.60426355,343.59956913)(658.5942662,343.5295743)
\curveto(658.58426357,343.46956926)(658.58426357,343.40956932)(658.5942662,343.3495743)
\curveto(658.61426354,343.28956944)(658.64926351,343.23956949)(658.6992662,343.1995743)
\curveto(658.77926338,343.14956958)(658.88926327,343.1245696)(659.0292662,343.1245743)
\lineto(659.4342662,343.1245743)
\lineto(661.0992662,343.1245743)
\lineto(661.5342662,343.1245743)
\curveto(661.69426046,343.13456959)(661.79926036,343.17956955)(661.8492662,343.2595743)
}
}
{
\newrgbcolor{curcolor}{0 0 0}
\pscustom[linestyle=none,fillstyle=solid,fillcolor=curcolor]
{
}
}
{
\newrgbcolor{curcolor}{0 0 0}
\pscustom[linestyle=none,fillstyle=solid,fillcolor=curcolor]
{
\newpath
\moveto(676.8727037,342.3745743)
\curveto(676.89269554,342.29457043)(676.89269554,342.20457052)(676.8727037,342.1045743)
\curveto(676.85269558,342.00457072)(676.81769561,341.93957079)(676.7677037,341.9095743)
\curveto(676.71769571,341.86957086)(676.64269579,341.83957089)(676.5427037,341.8195743)
\curveto(676.45269598,341.80957092)(676.34769608,341.79957093)(676.2277037,341.7895743)
\lineto(675.8827037,341.7895743)
\curveto(675.77269666,341.79957093)(675.67269676,341.80457092)(675.5827037,341.8045743)
\lineto(671.9227037,341.8045743)
\lineto(671.7127037,341.8045743)
\curveto(671.65270078,341.80457092)(671.59770083,341.79457093)(671.5477037,341.7745743)
\curveto(671.46770096,341.73457099)(671.41770101,341.69457103)(671.3977037,341.6545743)
\curveto(671.37770105,341.63457109)(671.35770107,341.59457113)(671.3377037,341.5345743)
\curveto(671.31770111,341.48457124)(671.31270112,341.43457129)(671.3227037,341.3845743)
\curveto(671.34270109,341.3245714)(671.35270108,341.26457146)(671.3527037,341.2045743)
\curveto(671.36270107,341.15457157)(671.37770105,341.09957163)(671.3977037,341.0395743)
\curveto(671.47770095,340.79957193)(671.57270086,340.59957213)(671.6827037,340.4395743)
\curveto(671.80270063,340.28957244)(671.96270047,340.15457257)(672.1627037,340.0345743)
\curveto(672.24270019,339.98457274)(672.32270011,339.94957278)(672.4027037,339.9295743)
\curveto(672.49269994,339.91957281)(672.58269985,339.89957283)(672.6727037,339.8695743)
\curveto(672.75269968,339.84957288)(672.86269957,339.83457289)(673.0027037,339.8245743)
\curveto(673.14269929,339.81457291)(673.26269917,339.81957291)(673.3627037,339.8395743)
\lineto(673.4977037,339.8395743)
\curveto(673.59769883,339.85957287)(673.68769874,339.87957285)(673.7677037,339.8995743)
\curveto(673.85769857,339.9295728)(673.94269849,339.95957277)(674.0227037,339.9895743)
\curveto(674.12269831,340.03957269)(674.2326982,340.10457262)(674.3527037,340.1845743)
\curveto(674.48269795,340.26457246)(674.57769785,340.34457238)(674.6377037,340.4245743)
\curveto(674.68769774,340.49457223)(674.73769769,340.55957217)(674.7877037,340.6195743)
\curveto(674.84769758,340.68957204)(674.91769751,340.73957199)(674.9977037,340.7695743)
\curveto(675.09769733,340.81957191)(675.22269721,340.83957189)(675.3727037,340.8295743)
\lineto(675.8077037,340.8295743)
\lineto(675.9877037,340.8295743)
\curveto(676.05769637,340.83957189)(676.11769631,340.83457189)(676.1677037,340.8145743)
\lineto(676.3177037,340.8145743)
\curveto(676.41769601,340.79457193)(676.48769594,340.76957196)(676.5277037,340.7395743)
\curveto(676.56769586,340.71957201)(676.58769584,340.67457205)(676.5877037,340.6045743)
\curveto(676.59769583,340.53457219)(676.59269584,340.47457225)(676.5727037,340.4245743)
\curveto(676.52269591,340.28457244)(676.46769596,340.15957257)(676.4077037,340.0495743)
\curveto(676.34769608,339.93957279)(676.27769615,339.8295729)(676.1977037,339.7195743)
\curveto(675.97769645,339.38957334)(675.7276967,339.1245736)(675.4477037,338.9245743)
\curveto(675.16769726,338.724574)(674.81769761,338.55457417)(674.3977037,338.4145743)
\curveto(674.28769814,338.37457435)(674.17769825,338.34957438)(674.0677037,338.3395743)
\curveto(673.95769847,338.3295744)(673.84269859,338.30957442)(673.7227037,338.2795743)
\curveto(673.68269875,338.26957446)(673.63769879,338.26957446)(673.5877037,338.2795743)
\curveto(673.54769888,338.27957445)(673.50769892,338.27457445)(673.4677037,338.2645743)
\lineto(673.3027037,338.2645743)
\curveto(673.25269918,338.24457448)(673.19269924,338.23957449)(673.1227037,338.2495743)
\curveto(673.06269937,338.24957448)(673.00769942,338.25457447)(672.9577037,338.2645743)
\curveto(672.87769955,338.27457445)(672.80769962,338.27457445)(672.7477037,338.2645743)
\curveto(672.68769974,338.25457447)(672.62269981,338.25957447)(672.5527037,338.2795743)
\curveto(672.50269993,338.29957443)(672.44769998,338.30957442)(672.3877037,338.3095743)
\curveto(672.3277001,338.30957442)(672.27270016,338.31957441)(672.2227037,338.3395743)
\curveto(672.11270032,338.35957437)(672.00270043,338.38457434)(671.8927037,338.4145743)
\curveto(671.78270065,338.43457429)(671.68270075,338.46957426)(671.5927037,338.5195743)
\curveto(671.48270095,338.55957417)(671.37770105,338.59457413)(671.2777037,338.6245743)
\curveto(671.18770124,338.66457406)(671.10270133,338.70957402)(671.0227037,338.7595743)
\curveto(670.70270173,338.95957377)(670.41770201,339.18957354)(670.1677037,339.4495743)
\curveto(669.91770251,339.71957301)(669.71270272,340.0295727)(669.5527037,340.3795743)
\curveto(669.50270293,340.48957224)(669.46270297,340.59957213)(669.4327037,340.7095743)
\curveto(669.40270303,340.8295719)(669.36270307,340.94957178)(669.3127037,341.0695743)
\curveto(669.30270313,341.10957162)(669.29770313,341.14457158)(669.2977037,341.1745743)
\curveto(669.29770313,341.21457151)(669.29270314,341.25457147)(669.2827037,341.2945743)
\curveto(669.24270319,341.41457131)(669.21770321,341.54457118)(669.2077037,341.6845743)
\lineto(669.1777037,342.1045743)
\curveto(669.17770325,342.15457057)(669.17270326,342.20957052)(669.1627037,342.2695743)
\curveto(669.16270327,342.3295704)(669.16770326,342.38457034)(669.1777037,342.4345743)
\lineto(669.1777037,342.6145743)
\lineto(669.2227037,342.9745743)
\curveto(669.26270317,343.14456958)(669.29770313,343.30956942)(669.3277037,343.4695743)
\curveto(669.35770307,343.6295691)(669.40270303,343.77956895)(669.4627037,343.9195743)
\curveto(669.89270254,344.95956777)(670.62270181,345.69456703)(671.6527037,346.1245743)
\curveto(671.79270064,346.18456654)(671.9327005,346.2245665)(672.0727037,346.2445743)
\curveto(672.22270021,346.27456645)(672.37770005,346.30956642)(672.5377037,346.3495743)
\curveto(672.61769981,346.35956637)(672.69269974,346.36456636)(672.7627037,346.3645743)
\curveto(672.8326996,346.36456636)(672.90769952,346.36956636)(672.9877037,346.3795743)
\curveto(673.49769893,346.38956634)(673.9326985,346.3295664)(674.2927037,346.1995743)
\curveto(674.66269777,346.07956665)(674.99269744,345.91956681)(675.2827037,345.7195743)
\curveto(675.37269706,345.65956707)(675.46269697,345.58956714)(675.5527037,345.5095743)
\curveto(675.64269679,345.43956729)(675.72269671,345.36456736)(675.7927037,345.2845743)
\curveto(675.82269661,345.23456749)(675.86269657,345.19456753)(675.9127037,345.1645743)
\curveto(675.99269644,345.05456767)(676.06769636,344.93956779)(676.1377037,344.8195743)
\curveto(676.20769622,344.70956802)(676.28269615,344.59456813)(676.3627037,344.4745743)
\curveto(676.41269602,344.38456834)(676.45269598,344.28956844)(676.4827037,344.1895743)
\curveto(676.52269591,344.09956863)(676.56269587,343.99956873)(676.6027037,343.8895743)
\curveto(676.65269578,343.75956897)(676.69269574,343.6245691)(676.7227037,343.4845743)
\curveto(676.75269568,343.34456938)(676.78769564,343.20456952)(676.8277037,343.0645743)
\curveto(676.84769558,342.98456974)(676.85269558,342.89456983)(676.8427037,342.7945743)
\curveto(676.84269559,342.70457002)(676.85269558,342.61957011)(676.8727037,342.5395743)
\lineto(676.8727037,342.3745743)
\moveto(674.6227037,343.2595743)
\curveto(674.69269774,343.35956937)(674.69769773,343.47956925)(674.6377037,343.6195743)
\curveto(674.58769784,343.76956896)(674.54769788,343.87956885)(674.5177037,343.9495743)
\curveto(674.37769805,344.21956851)(674.19269824,344.4245683)(673.9627037,344.5645743)
\curveto(673.7326987,344.71456801)(673.41269902,344.79456793)(673.0027037,344.8045743)
\curveto(672.97269946,344.78456794)(672.93769949,344.77956795)(672.8977037,344.7895743)
\curveto(672.85769957,344.79956793)(672.82269961,344.79956793)(672.7927037,344.7895743)
\curveto(672.74269969,344.76956796)(672.68769974,344.75456797)(672.6277037,344.7445743)
\curveto(672.56769986,344.74456798)(672.51269992,344.73456799)(672.4627037,344.7145743)
\curveto(672.02270041,344.57456815)(671.69770073,344.29956843)(671.4877037,343.8895743)
\curveto(671.46770096,343.84956888)(671.44270099,343.79456893)(671.4127037,343.7245743)
\curveto(671.39270104,343.66456906)(671.37770105,343.59956913)(671.3677037,343.5295743)
\curveto(671.35770107,343.46956926)(671.35770107,343.40956932)(671.3677037,343.3495743)
\curveto(671.38770104,343.28956944)(671.42270101,343.23956949)(671.4727037,343.1995743)
\curveto(671.55270088,343.14956958)(671.66270077,343.1245696)(671.8027037,343.1245743)
\lineto(672.2077037,343.1245743)
\lineto(673.8727037,343.1245743)
\lineto(674.3077037,343.1245743)
\curveto(674.46769796,343.13456959)(674.57269786,343.17956955)(674.6227037,343.2595743)
}
}
{
\newrgbcolor{curcolor}{0 0 0}
\pscustom[linestyle=none,fillstyle=solid,fillcolor=curcolor]
{
\newpath
\moveto(681.09098495,346.3795743)
\curveto(681.84098045,346.39956633)(682.4909798,346.31456641)(683.04098495,346.1245743)
\curveto(683.60097869,345.94456678)(684.02597827,345.6295671)(684.31598495,345.1795743)
\curveto(684.38597791,345.06956766)(684.44597785,344.95456777)(684.49598495,344.8345743)
\curveto(684.55597774,344.724568)(684.60597769,344.59956813)(684.64598495,344.4595743)
\curveto(684.66597763,344.39956833)(684.67597762,344.33456839)(684.67598495,344.2645743)
\curveto(684.67597762,344.19456853)(684.66597763,344.13456859)(684.64598495,344.0845743)
\curveto(684.60597769,344.0245687)(684.55097774,343.98456874)(684.48098495,343.9645743)
\curveto(684.43097786,343.94456878)(684.37097792,343.93456879)(684.30098495,343.9345743)
\lineto(684.09098495,343.9345743)
\lineto(683.43098495,343.9345743)
\curveto(683.36097893,343.93456879)(683.290979,343.9295688)(683.22098495,343.9195743)
\curveto(683.15097914,343.91956881)(683.08597921,343.9295688)(683.02598495,343.9495743)
\curveto(682.92597937,343.96956876)(682.85097944,344.00956872)(682.80098495,344.0695743)
\curveto(682.75097954,344.1295686)(682.70597959,344.18956854)(682.66598495,344.2495743)
\lineto(682.54598495,344.4595743)
\curveto(682.51597978,344.53956819)(682.46597983,344.60456812)(682.39598495,344.6545743)
\curveto(682.29598,344.73456799)(682.1959801,344.79456793)(682.09598495,344.8345743)
\curveto(682.00598029,344.87456785)(681.8909804,344.90956782)(681.75098495,344.9395743)
\curveto(681.68098061,344.95956777)(681.57598072,344.97456775)(681.43598495,344.9845743)
\curveto(681.30598099,344.99456773)(681.20598109,344.98956774)(681.13598495,344.9695743)
\lineto(681.03098495,344.9695743)
\lineto(680.88098495,344.9395743)
\curveto(680.84098145,344.93956779)(680.7959815,344.93456779)(680.74598495,344.9245743)
\curveto(680.57598172,344.87456785)(680.43598186,344.80456792)(680.32598495,344.7145743)
\curveto(680.22598207,344.63456809)(680.15598214,344.50956822)(680.11598495,344.3395743)
\curveto(680.0959822,344.26956846)(680.0959822,344.20456852)(680.11598495,344.1445743)
\curveto(680.13598216,344.08456864)(680.15598214,344.03456869)(680.17598495,343.9945743)
\curveto(680.24598205,343.87456885)(680.32598197,343.77956895)(680.41598495,343.7095743)
\curveto(680.51598178,343.63956909)(680.63098166,343.57956915)(680.76098495,343.5295743)
\curveto(680.95098134,343.44956928)(681.15598114,343.37956935)(681.37598495,343.3195743)
\lineto(682.06598495,343.1695743)
\curveto(682.30597999,343.1295696)(682.53597976,343.07956965)(682.75598495,343.0195743)
\curveto(682.98597931,342.96956976)(683.20097909,342.90456982)(683.40098495,342.8245743)
\curveto(683.4909788,342.78456994)(683.57597872,342.74956998)(683.65598495,342.7195743)
\curveto(683.74597855,342.69957003)(683.83097846,342.66457006)(683.91098495,342.6145743)
\curveto(684.10097819,342.49457023)(684.27097802,342.36457036)(684.42098495,342.2245743)
\curveto(684.58097771,342.08457064)(684.70597759,341.90957082)(684.79598495,341.6995743)
\curveto(684.82597747,341.6295711)(684.85097744,341.55957117)(684.87098495,341.4895743)
\curveto(684.8909774,341.41957131)(684.91097738,341.34457138)(684.93098495,341.2645743)
\curveto(684.94097735,341.20457152)(684.94597735,341.10957162)(684.94598495,340.9795743)
\curveto(684.95597734,340.85957187)(684.95597734,340.76457196)(684.94598495,340.6945743)
\lineto(684.94598495,340.6195743)
\curveto(684.92597737,340.55957217)(684.91097738,340.49957223)(684.90098495,340.4395743)
\curveto(684.90097739,340.38957234)(684.8959774,340.33957239)(684.88598495,340.2895743)
\curveto(684.81597748,339.98957274)(684.70597759,339.724573)(684.55598495,339.4945743)
\curveto(684.3959779,339.25457347)(684.20097809,339.05957367)(683.97098495,338.9095743)
\curveto(683.74097855,338.75957397)(683.48097881,338.6295741)(683.19098495,338.5195743)
\curveto(683.08097921,338.46957426)(682.96097933,338.43457429)(682.83098495,338.4145743)
\curveto(682.71097958,338.39457433)(682.5909797,338.36957436)(682.47098495,338.3395743)
\curveto(682.38097991,338.31957441)(682.28598001,338.30957442)(682.18598495,338.3095743)
\curveto(682.0959802,338.29957443)(682.00598029,338.28457444)(681.91598495,338.2645743)
\lineto(681.64598495,338.2645743)
\curveto(681.58598071,338.24457448)(681.48098081,338.23457449)(681.33098495,338.2345743)
\curveto(681.1909811,338.23457449)(681.0909812,338.24457448)(681.03098495,338.2645743)
\curveto(681.00098129,338.26457446)(680.96598133,338.26957446)(680.92598495,338.2795743)
\lineto(680.82098495,338.2795743)
\curveto(680.70098159,338.29957443)(680.58098171,338.31457441)(680.46098495,338.3245743)
\curveto(680.34098195,338.33457439)(680.22598207,338.35457437)(680.11598495,338.3845743)
\curveto(679.72598257,338.49457423)(679.38098291,338.61957411)(679.08098495,338.7595743)
\curveto(678.78098351,338.90957382)(678.52598377,339.1295736)(678.31598495,339.4195743)
\curveto(678.17598412,339.60957312)(678.05598424,339.8295729)(677.95598495,340.0795743)
\curveto(677.93598436,340.13957259)(677.91598438,340.21957251)(677.89598495,340.3195743)
\curveto(677.87598442,340.36957236)(677.86098443,340.43957229)(677.85098495,340.5295743)
\curveto(677.84098445,340.61957211)(677.84598445,340.69457203)(677.86598495,340.7545743)
\curveto(677.8959844,340.8245719)(677.94598435,340.87457185)(678.01598495,340.9045743)
\curveto(678.06598423,340.9245718)(678.12598417,340.93457179)(678.19598495,340.9345743)
\lineto(678.42098495,340.9345743)
\lineto(679.12598495,340.9345743)
\lineto(679.36598495,340.9345743)
\curveto(679.44598285,340.93457179)(679.51598278,340.9245718)(679.57598495,340.9045743)
\curveto(679.68598261,340.86457186)(679.75598254,340.79957193)(679.78598495,340.7095743)
\curveto(679.82598247,340.61957211)(679.87098242,340.5245722)(679.92098495,340.4245743)
\curveto(679.94098235,340.37457235)(679.97598232,340.30957242)(680.02598495,340.2295743)
\curveto(680.08598221,340.14957258)(680.13598216,340.09957263)(680.17598495,340.0795743)
\curveto(680.295982,339.97957275)(680.41098188,339.89957283)(680.52098495,339.8395743)
\curveto(680.63098166,339.78957294)(680.77098152,339.73957299)(680.94098495,339.6895743)
\curveto(680.9909813,339.66957306)(681.04098125,339.65957307)(681.09098495,339.6595743)
\curveto(681.14098115,339.66957306)(681.1909811,339.66957306)(681.24098495,339.6595743)
\curveto(681.32098097,339.63957309)(681.40598089,339.6295731)(681.49598495,339.6295743)
\curveto(681.5959807,339.63957309)(681.68098061,339.65457307)(681.75098495,339.6745743)
\curveto(681.80098049,339.68457304)(681.84598045,339.68957304)(681.88598495,339.6895743)
\curveto(681.93598036,339.68957304)(681.98598031,339.69957303)(682.03598495,339.7195743)
\curveto(682.17598012,339.76957296)(682.30097999,339.8295729)(682.41098495,339.8995743)
\curveto(682.53097976,339.96957276)(682.62597967,340.05957267)(682.69598495,340.1695743)
\curveto(682.74597955,340.24957248)(682.78597951,340.37457235)(682.81598495,340.5445743)
\curveto(682.83597946,340.61457211)(682.83597946,340.67957205)(682.81598495,340.7395743)
\curveto(682.7959795,340.79957193)(682.77597952,340.84957188)(682.75598495,340.8895743)
\curveto(682.68597961,341.0295717)(682.5959797,341.13457159)(682.48598495,341.2045743)
\curveto(682.38597991,341.27457145)(682.26598003,341.33957139)(682.12598495,341.3995743)
\curveto(681.93598036,341.47957125)(681.73598056,341.54457118)(681.52598495,341.5945743)
\curveto(681.31598098,341.64457108)(681.10598119,341.69957103)(680.89598495,341.7595743)
\curveto(680.81598148,341.77957095)(680.73098156,341.79457093)(680.64098495,341.8045743)
\curveto(680.56098173,341.81457091)(680.48098181,341.8295709)(680.40098495,341.8495743)
\curveto(680.08098221,341.93957079)(679.77598252,342.0245707)(679.48598495,342.1045743)
\curveto(679.1959831,342.19457053)(678.93098336,342.3245704)(678.69098495,342.4945743)
\curveto(678.41098388,342.69457003)(678.20598409,342.96456976)(678.07598495,343.3045743)
\curveto(678.05598424,343.37456935)(678.03598426,343.46956926)(678.01598495,343.5895743)
\curveto(677.9959843,343.65956907)(677.98098431,343.74456898)(677.97098495,343.8445743)
\curveto(677.96098433,343.94456878)(677.96598433,344.03456869)(677.98598495,344.1145743)
\curveto(678.00598429,344.16456856)(678.01098428,344.20456852)(678.00098495,344.2345743)
\curveto(677.9909843,344.27456845)(677.9959843,344.31956841)(678.01598495,344.3695743)
\curveto(678.03598426,344.47956825)(678.05598424,344.57956815)(678.07598495,344.6695743)
\curveto(678.10598419,344.76956796)(678.14098415,344.86456786)(678.18098495,344.9545743)
\curveto(678.31098398,345.24456748)(678.4909838,345.47956725)(678.72098495,345.6595743)
\curveto(678.95098334,345.83956689)(679.21098308,345.98456674)(679.50098495,346.0945743)
\curveto(679.61098268,346.14456658)(679.72598257,346.17956655)(679.84598495,346.1995743)
\curveto(679.96598233,346.2295665)(680.0909822,346.25956647)(680.22098495,346.2895743)
\curveto(680.28098201,346.30956642)(680.34098195,346.31956641)(680.40098495,346.3195743)
\lineto(680.58098495,346.3495743)
\curveto(680.66098163,346.35956637)(680.74598155,346.36456636)(680.83598495,346.3645743)
\curveto(680.92598137,346.36456636)(681.01098128,346.36956636)(681.09098495,346.3795743)
}
}
{
\newrgbcolor{curcolor}{0 0 0}
\pscustom[linestyle=none,fillstyle=solid,fillcolor=curcolor]
{
\newpath
\moveto(694.06762558,342.3895743)
\curveto(694.0776169,342.3295704)(694.08261689,342.23957049)(694.08262558,342.1195743)
\curveto(694.08261689,341.99957073)(694.0726169,341.91457081)(694.05262558,341.8645743)
\lineto(694.05262558,341.6695743)
\curveto(694.02261695,341.55957117)(694.00261697,341.45457127)(693.99262558,341.3545743)
\curveto(693.99261698,341.25457147)(693.977617,341.15457157)(693.94762558,341.0545743)
\curveto(693.92761705,340.96457176)(693.90761707,340.86957186)(693.88762558,340.7695743)
\curveto(693.86761711,340.67957205)(693.83761714,340.58957214)(693.79762558,340.4995743)
\curveto(693.72761725,340.3295724)(693.65761732,340.16957256)(693.58762558,340.0195743)
\curveto(693.51761746,339.87957285)(693.43761754,339.73957299)(693.34762558,339.5995743)
\curveto(693.28761769,339.50957322)(693.22261775,339.4245733)(693.15262558,339.3445743)
\curveto(693.09261788,339.27457345)(693.02261795,339.19957353)(692.94262558,339.1195743)
\lineto(692.83762558,339.0145743)
\curveto(692.78761819,338.96457376)(692.73261824,338.91957381)(692.67262558,338.8795743)
\lineto(692.52262558,338.7595743)
\curveto(692.44261853,338.69957403)(692.35261862,338.64457408)(692.25262558,338.5945743)
\curveto(692.16261881,338.55457417)(692.06761891,338.50957422)(691.96762558,338.4595743)
\curveto(691.86761911,338.40957432)(691.76261921,338.37457435)(691.65262558,338.3545743)
\curveto(691.55261942,338.33457439)(691.44761953,338.31457441)(691.33762558,338.2945743)
\curveto(691.2776197,338.27457445)(691.21261976,338.26457446)(691.14262558,338.2645743)
\curveto(691.08261989,338.26457446)(691.01761996,338.25457447)(690.94762558,338.2345743)
\lineto(690.81262558,338.2345743)
\curveto(690.73262024,338.21457451)(690.65762032,338.21457451)(690.58762558,338.2345743)
\lineto(690.43762558,338.2345743)
\curveto(690.3776206,338.25457447)(690.31262066,338.26457446)(690.24262558,338.2645743)
\curveto(690.18262079,338.25457447)(690.12262085,338.25957447)(690.06262558,338.2795743)
\curveto(689.90262107,338.3295744)(689.74762123,338.37457435)(689.59762558,338.4145743)
\curveto(689.45762152,338.45457427)(689.32762165,338.51457421)(689.20762558,338.5945743)
\curveto(689.13762184,338.63457409)(689.0726219,338.67457405)(689.01262558,338.7145743)
\curveto(688.95262202,338.76457396)(688.88762209,338.81457391)(688.81762558,338.8645743)
\lineto(688.63762558,338.9995743)
\curveto(688.55762242,339.05957367)(688.48762249,339.06457366)(688.42762558,339.0145743)
\curveto(688.3776226,338.98457374)(688.35262262,338.94457378)(688.35262558,338.8945743)
\curveto(688.35262262,338.85457387)(688.34262263,338.80457392)(688.32262558,338.7445743)
\curveto(688.30262267,338.64457408)(688.29262268,338.5295742)(688.29262558,338.3995743)
\curveto(688.30262267,338.26957446)(688.30762267,338.14957458)(688.30762558,338.0395743)
\lineto(688.30762558,336.5095743)
\curveto(688.30762267,336.37957635)(688.30262267,336.25457647)(688.29262558,336.1345743)
\curveto(688.29262268,336.00457672)(688.26762271,335.89957683)(688.21762558,335.8195743)
\curveto(688.18762279,335.77957695)(688.13262284,335.74957698)(688.05262558,335.7295743)
\curveto(687.972623,335.70957702)(687.88262309,335.69957703)(687.78262558,335.6995743)
\curveto(687.68262329,335.68957704)(687.58262339,335.68957704)(687.48262558,335.6995743)
\lineto(687.22762558,335.6995743)
\lineto(686.82262558,335.6995743)
\lineto(686.71762558,335.6995743)
\curveto(686.6776243,335.69957703)(686.64262433,335.70457702)(686.61262558,335.7145743)
\lineto(686.49262558,335.7145743)
\curveto(686.32262465,335.76457696)(686.23262474,335.86457686)(686.22262558,336.0145743)
\curveto(686.21262476,336.15457657)(686.20762477,336.3245764)(686.20762558,336.5245743)
\lineto(686.20762558,345.3295743)
\curveto(686.20762477,345.43956729)(686.20262477,345.55456717)(686.19262558,345.6745743)
\curveto(686.19262478,345.80456692)(686.21762476,345.90456682)(686.26762558,345.9745743)
\curveto(686.30762467,346.04456668)(686.36262461,346.08956664)(686.43262558,346.1095743)
\curveto(686.48262449,346.1295666)(686.54262443,346.13956659)(686.61262558,346.1395743)
\lineto(686.83762558,346.1395743)
\lineto(687.55762558,346.1395743)
\lineto(687.84262558,346.1395743)
\curveto(687.93262304,346.13956659)(688.00762297,346.11456661)(688.06762558,346.0645743)
\curveto(688.13762284,346.01456671)(688.1726228,345.94956678)(688.17262558,345.8695743)
\curveto(688.18262279,345.79956693)(688.20762277,345.724567)(688.24762558,345.6445743)
\curveto(688.25762272,345.61456711)(688.26762271,345.58956714)(688.27762558,345.5695743)
\curveto(688.29762268,345.55956717)(688.31762266,345.54456718)(688.33762558,345.5245743)
\curveto(688.44762253,345.51456721)(688.53762244,345.54456718)(688.60762558,345.6145743)
\curveto(688.6776223,345.68456704)(688.74762223,345.74456698)(688.81762558,345.7945743)
\curveto(688.94762203,345.88456684)(689.08262189,345.96456676)(689.22262558,346.0345743)
\curveto(689.36262161,346.11456661)(689.51762146,346.17956655)(689.68762558,346.2295743)
\curveto(689.76762121,346.25956647)(689.85262112,346.27956645)(689.94262558,346.2895743)
\curveto(690.04262093,346.29956643)(690.13762084,346.31456641)(690.22762558,346.3345743)
\curveto(690.26762071,346.34456638)(690.30762067,346.34456638)(690.34762558,346.3345743)
\curveto(690.39762058,346.3245664)(690.43762054,346.3295664)(690.46762558,346.3495743)
\curveto(691.03761994,346.36956636)(691.51761946,346.28956644)(691.90762558,346.1095743)
\curveto(692.30761867,345.93956679)(692.64761833,345.71456701)(692.92762558,345.4345743)
\curveto(692.977618,345.38456734)(693.02261795,345.33456739)(693.06262558,345.2845743)
\curveto(693.10261787,345.24456748)(693.14261783,345.19956753)(693.18262558,345.1495743)
\curveto(693.25261772,345.05956767)(693.31261766,344.96956776)(693.36262558,344.8795743)
\curveto(693.42261755,344.78956794)(693.4776175,344.69956803)(693.52762558,344.6095743)
\curveto(693.54761743,344.58956814)(693.55761742,344.56456816)(693.55762558,344.5345743)
\curveto(693.56761741,344.50456822)(693.58261739,344.46956826)(693.60262558,344.4295743)
\curveto(693.66261731,344.3295684)(693.71761726,344.20956852)(693.76762558,344.0695743)
\curveto(693.78761719,344.00956872)(693.80761717,343.94456878)(693.82762558,343.8745743)
\curveto(693.84761713,343.81456891)(693.86761711,343.74956898)(693.88762558,343.6795743)
\curveto(693.92761705,343.55956917)(693.95261702,343.43456929)(693.96262558,343.3045743)
\curveto(693.98261699,343.17456955)(694.00761697,343.03956969)(694.03762558,342.8995743)
\lineto(694.03762558,342.7345743)
\lineto(694.06762558,342.5545743)
\lineto(694.06762558,342.3895743)
\moveto(691.95262558,342.0445743)
\curveto(691.96261901,342.09457063)(691.96761901,342.15957057)(691.96762558,342.2395743)
\curveto(691.96761901,342.3295704)(691.96261901,342.39957033)(691.95262558,342.4495743)
\lineto(691.95262558,342.5845743)
\curveto(691.93261904,342.64457008)(691.92261905,342.70957002)(691.92262558,342.7795743)
\curveto(691.92261905,342.84956988)(691.91261906,342.91956981)(691.89262558,342.9895743)
\curveto(691.8726191,343.08956964)(691.85261912,343.18456954)(691.83262558,343.2745743)
\curveto(691.81261916,343.37456935)(691.78261919,343.46456926)(691.74262558,343.5445743)
\curveto(691.62261935,343.86456886)(691.46761951,344.11956861)(691.27762558,344.3095743)
\curveto(691.08761989,344.49956823)(690.81762016,344.63956809)(690.46762558,344.7295743)
\curveto(690.38762059,344.74956798)(690.29762068,344.75956797)(690.19762558,344.7595743)
\lineto(689.92762558,344.7595743)
\curveto(689.88762109,344.74956798)(689.85262112,344.74456798)(689.82262558,344.7445743)
\curveto(689.79262118,344.74456798)(689.75762122,344.73956799)(689.71762558,344.7295743)
\lineto(689.50762558,344.6695743)
\curveto(689.44762153,344.65956807)(689.38762159,344.63956809)(689.32762558,344.6095743)
\curveto(689.06762191,344.49956823)(688.86262211,344.3295684)(688.71262558,344.0995743)
\curveto(688.5726224,343.86956886)(688.45762252,343.61456911)(688.36762558,343.3345743)
\curveto(688.34762263,343.25456947)(688.33262264,343.16956956)(688.32262558,343.0795743)
\curveto(688.31262266,342.99956973)(688.29762268,342.91956981)(688.27762558,342.8395743)
\curveto(688.26762271,342.79956993)(688.26262271,342.73456999)(688.26262558,342.6445743)
\curveto(688.24262273,342.60457012)(688.23762274,342.55457017)(688.24762558,342.4945743)
\curveto(688.25762272,342.44457028)(688.25762272,342.39457033)(688.24762558,342.3445743)
\curveto(688.22762275,342.28457044)(688.22762275,342.2295705)(688.24762558,342.1795743)
\lineto(688.24762558,341.9995743)
\lineto(688.24762558,341.8645743)
\curveto(688.24762273,341.8245709)(688.25762272,341.78457094)(688.27762558,341.7445743)
\curveto(688.2776227,341.67457105)(688.28262269,341.61957111)(688.29262558,341.5795743)
\lineto(688.32262558,341.3995743)
\curveto(688.33262264,341.33957139)(688.34762263,341.27957145)(688.36762558,341.2195743)
\curveto(688.45762252,340.9295718)(688.56262241,340.68957204)(688.68262558,340.4995743)
\curveto(688.81262216,340.31957241)(688.99262198,340.15957257)(689.22262558,340.0195743)
\curveto(689.36262161,339.93957279)(689.52762145,339.87457285)(689.71762558,339.8245743)
\curveto(689.75762122,339.81457291)(689.79262118,339.80957292)(689.82262558,339.8095743)
\curveto(689.85262112,339.81957291)(689.88762109,339.81957291)(689.92762558,339.8095743)
\curveto(689.96762101,339.79957293)(690.02762095,339.78957294)(690.10762558,339.7795743)
\curveto(690.18762079,339.77957295)(690.25262072,339.78457294)(690.30262558,339.7945743)
\curveto(690.38262059,339.81457291)(690.46262051,339.8295729)(690.54262558,339.8395743)
\curveto(690.63262034,339.85957287)(690.71762026,339.88457284)(690.79762558,339.9145743)
\curveto(691.03761994,340.01457271)(691.23261974,340.15457257)(691.38262558,340.3345743)
\curveto(691.53261944,340.51457221)(691.65761932,340.724572)(691.75762558,340.9645743)
\curveto(691.80761917,341.08457164)(691.84261913,341.20957152)(691.86262558,341.3395743)
\curveto(691.88261909,341.46957126)(691.90761907,341.60457112)(691.93762558,341.7445743)
\lineto(691.93762558,341.8945743)
\curveto(691.94761903,341.94457078)(691.95261902,341.99457073)(691.95262558,342.0445743)
}
}
{
\newrgbcolor{curcolor}{0 0 0}
\pscustom[linestyle=none,fillstyle=solid,fillcolor=curcolor]
{
\newpath
\moveto(702.39754745,339.0295743)
\curveto(702.4175396,338.91957381)(702.42753959,338.80957392)(702.42754745,338.6995743)
\curveto(702.43753958,338.58957414)(702.38753963,338.51457421)(702.27754745,338.4745743)
\curveto(702.2175398,338.44457428)(702.14753987,338.4295743)(702.06754745,338.4295743)
\lineto(701.82754745,338.4295743)
\lineto(701.01754745,338.4295743)
\lineto(700.74754745,338.4295743)
\curveto(700.66754135,338.43957429)(700.60254142,338.46457426)(700.55254745,338.5045743)
\curveto(700.48254154,338.54457418)(700.42754159,338.59957413)(700.38754745,338.6695743)
\curveto(700.35754166,338.74957398)(700.31254171,338.81457391)(700.25254745,338.8645743)
\curveto(700.23254179,338.88457384)(700.20754181,338.89957383)(700.17754745,338.9095743)
\curveto(700.14754187,338.9295738)(700.10754191,338.93457379)(700.05754745,338.9245743)
\curveto(700.00754201,338.90457382)(699.95754206,338.87957385)(699.90754745,338.8495743)
\curveto(699.86754215,338.81957391)(699.8225422,338.79457393)(699.77254745,338.7745743)
\curveto(699.7225423,338.73457399)(699.66754235,338.69957403)(699.60754745,338.6695743)
\lineto(699.42754745,338.5795743)
\curveto(699.29754272,338.51957421)(699.16254286,338.46957426)(699.02254745,338.4295743)
\curveto(698.88254314,338.39957433)(698.73754328,338.36457436)(698.58754745,338.3245743)
\curveto(698.5175435,338.30457442)(698.44754357,338.29457443)(698.37754745,338.2945743)
\curveto(698.3175437,338.28457444)(698.25254377,338.27457445)(698.18254745,338.2645743)
\lineto(698.09254745,338.2645743)
\curveto(698.06254396,338.25457447)(698.03254399,338.24957448)(698.00254745,338.2495743)
\lineto(697.83754745,338.2495743)
\curveto(697.73754428,338.2295745)(697.63754438,338.2295745)(697.53754745,338.2495743)
\lineto(697.40254745,338.2495743)
\curveto(697.33254469,338.26957446)(697.26254476,338.27957445)(697.19254745,338.2795743)
\curveto(697.13254489,338.26957446)(697.07254495,338.27457445)(697.01254745,338.2945743)
\curveto(696.91254511,338.31457441)(696.8175452,338.33457439)(696.72754745,338.3545743)
\curveto(696.63754538,338.36457436)(696.55254547,338.38957434)(696.47254745,338.4295743)
\curveto(696.18254584,338.53957419)(695.93254609,338.67957405)(695.72254745,338.8495743)
\curveto(695.5225465,339.0295737)(695.36254666,339.26457346)(695.24254745,339.5545743)
\curveto(695.21254681,339.6245731)(695.18254684,339.69957303)(695.15254745,339.7795743)
\curveto(695.13254689,339.85957287)(695.11254691,339.94457278)(695.09254745,340.0345743)
\curveto(695.07254695,340.08457264)(695.06254696,340.13457259)(695.06254745,340.1845743)
\curveto(695.07254695,340.23457249)(695.07254695,340.28457244)(695.06254745,340.3345743)
\curveto(695.05254697,340.36457236)(695.04254698,340.4245723)(695.03254745,340.5145743)
\curveto(695.03254699,340.61457211)(695.03754698,340.68457204)(695.04754745,340.7245743)
\curveto(695.06754695,340.8245719)(695.07754694,340.90957182)(695.07754745,340.9795743)
\lineto(695.16754745,341.3095743)
\curveto(695.19754682,341.4295713)(695.23754678,341.53457119)(695.28754745,341.6245743)
\curveto(695.45754656,341.91457081)(695.65254637,342.13457059)(695.87254745,342.2845743)
\curveto(696.09254593,342.43457029)(696.37254565,342.56457016)(696.71254745,342.6745743)
\curveto(696.84254518,342.72457)(696.97754504,342.75956997)(697.11754745,342.7795743)
\curveto(697.25754476,342.79956993)(697.39754462,342.8245699)(697.53754745,342.8545743)
\curveto(697.6175444,342.87456985)(697.70254432,342.88456984)(697.79254745,342.8845743)
\curveto(697.88254414,342.89456983)(697.97254405,342.90956982)(698.06254745,342.9295743)
\curveto(698.13254389,342.94956978)(698.20254382,342.95456977)(698.27254745,342.9445743)
\curveto(698.34254368,342.94456978)(698.4175436,342.95456977)(698.49754745,342.9745743)
\curveto(698.56754345,342.99456973)(698.63754338,343.00456972)(698.70754745,343.0045743)
\curveto(698.77754324,343.00456972)(698.85254317,343.01456971)(698.93254745,343.0345743)
\curveto(699.14254288,343.08456964)(699.33254269,343.1245696)(699.50254745,343.1545743)
\curveto(699.68254234,343.19456953)(699.84254218,343.28456944)(699.98254745,343.4245743)
\curveto(700.07254195,343.51456921)(700.13254189,343.61456911)(700.16254745,343.7245743)
\curveto(700.17254185,343.75456897)(700.17254185,343.77956895)(700.16254745,343.7995743)
\curveto(700.16254186,343.81956891)(700.16754185,343.83956889)(700.17754745,343.8595743)
\curveto(700.18754183,343.87956885)(700.19254183,343.90956882)(700.19254745,343.9495743)
\lineto(700.19254745,344.0395743)
\lineto(700.16254745,344.1595743)
\curveto(700.16254186,344.19956853)(700.15754186,344.23456849)(700.14754745,344.2645743)
\curveto(700.04754197,344.56456816)(699.83754218,344.76956796)(699.51754745,344.8795743)
\curveto(699.42754259,344.90956782)(699.3175427,344.9295678)(699.18754745,344.9395743)
\curveto(699.06754295,344.95956777)(698.94254308,344.96456776)(698.81254745,344.9545743)
\curveto(698.68254334,344.95456777)(698.55754346,344.94456778)(698.43754745,344.9245743)
\curveto(698.3175437,344.90456782)(698.21254381,344.87956785)(698.12254745,344.8495743)
\curveto(698.06254396,344.8295679)(698.00254402,344.79956793)(697.94254745,344.7595743)
\curveto(697.89254413,344.729568)(697.84254418,344.69456803)(697.79254745,344.6545743)
\curveto(697.74254428,344.61456811)(697.68754433,344.55956817)(697.62754745,344.4895743)
\curveto(697.57754444,344.41956831)(697.54254448,344.35456837)(697.52254745,344.2945743)
\curveto(697.47254455,344.19456853)(697.42754459,344.09956863)(697.38754745,344.0095743)
\curveto(697.35754466,343.91956881)(697.28754473,343.85956887)(697.17754745,343.8295743)
\curveto(697.09754492,343.80956892)(697.01254501,343.79956893)(696.92254745,343.7995743)
\lineto(696.65254745,343.7995743)
\lineto(696.08254745,343.7995743)
\curveto(696.03254599,343.79956893)(695.98254604,343.79456893)(695.93254745,343.7845743)
\curveto(695.88254614,343.78456894)(695.83754618,343.78956894)(695.79754745,343.7995743)
\lineto(695.66254745,343.7995743)
\curveto(695.64254638,343.80956892)(695.6175464,343.81456891)(695.58754745,343.8145743)
\curveto(695.55754646,343.81456891)(695.53254649,343.8245689)(695.51254745,343.8445743)
\curveto(695.43254659,343.86456886)(695.37754664,343.9295688)(695.34754745,344.0395743)
\curveto(695.33754668,344.08956864)(695.33754668,344.13956859)(695.34754745,344.1895743)
\curveto(695.35754666,344.23956849)(695.36754665,344.28456844)(695.37754745,344.3245743)
\curveto(695.40754661,344.43456829)(695.43754658,344.53456819)(695.46754745,344.6245743)
\curveto(695.50754651,344.724568)(695.55254647,344.81456791)(695.60254745,344.8945743)
\lineto(695.69254745,345.0445743)
\lineto(695.78254745,345.1945743)
\curveto(695.86254616,345.30456742)(695.96254606,345.40956732)(696.08254745,345.5095743)
\curveto(696.10254592,345.51956721)(696.13254589,345.54456718)(696.17254745,345.5845743)
\curveto(696.2225458,345.6245671)(696.26754575,345.65956707)(696.30754745,345.6895743)
\curveto(696.34754567,345.71956701)(696.39254563,345.74956698)(696.44254745,345.7795743)
\curveto(696.61254541,345.88956684)(696.79254523,345.97456675)(696.98254745,346.0345743)
\curveto(697.17254485,346.10456662)(697.36754465,346.16956656)(697.56754745,346.2295743)
\curveto(697.68754433,346.25956647)(697.81254421,346.27956645)(697.94254745,346.2895743)
\curveto(698.07254395,346.29956643)(698.20254382,346.31956641)(698.33254745,346.3495743)
\curveto(698.37254365,346.35956637)(698.43254359,346.35956637)(698.51254745,346.3495743)
\curveto(698.60254342,346.33956639)(698.65754336,346.34456638)(698.67754745,346.3645743)
\curveto(699.08754293,346.37456635)(699.47754254,346.35956637)(699.84754745,346.3195743)
\curveto(700.22754179,346.27956645)(700.56754145,346.20456652)(700.86754745,346.0945743)
\curveto(701.17754084,345.98456674)(701.44254058,345.83456689)(701.66254745,345.6445743)
\curveto(701.88254014,345.46456726)(702.05253997,345.2295675)(702.17254745,344.9395743)
\curveto(702.24253978,344.76956796)(702.28253974,344.57456815)(702.29254745,344.3545743)
\curveto(702.30253972,344.13456859)(702.30753971,343.90956882)(702.30754745,343.6795743)
\lineto(702.30754745,340.3345743)
\lineto(702.30754745,339.7495743)
\curveto(702.30753971,339.55957317)(702.32753969,339.38457334)(702.36754745,339.2245743)
\curveto(702.37753964,339.19457353)(702.38253964,339.15957357)(702.38254745,339.1195743)
\curveto(702.38253964,339.08957364)(702.38753963,339.05957367)(702.39754745,339.0295743)
\moveto(700.19254745,341.3395743)
\curveto(700.20254182,341.38957134)(700.20754181,341.44457128)(700.20754745,341.5045743)
\curveto(700.20754181,341.57457115)(700.20254182,341.63457109)(700.19254745,341.6845743)
\curveto(700.17254185,341.74457098)(700.16254186,341.79957093)(700.16254745,341.8495743)
\curveto(700.16254186,341.89957083)(700.14254188,341.93957079)(700.10254745,341.9695743)
\curveto(700.05254197,342.00957072)(699.97754204,342.0295707)(699.87754745,342.0295743)
\curveto(699.83754218,342.01957071)(699.80254222,342.00957072)(699.77254745,341.9995743)
\curveto(699.74254228,341.99957073)(699.70754231,341.99457073)(699.66754745,341.9845743)
\curveto(699.59754242,341.96457076)(699.5225425,341.94957078)(699.44254745,341.9395743)
\curveto(699.36254266,341.9295708)(699.28254274,341.91457081)(699.20254745,341.8945743)
\curveto(699.17254285,341.88457084)(699.12754289,341.87957085)(699.06754745,341.8795743)
\curveto(698.93754308,341.84957088)(698.80754321,341.8295709)(698.67754745,341.8195743)
\curveto(698.54754347,341.80957092)(698.4225436,341.78457094)(698.30254745,341.7445743)
\curveto(698.2225438,341.724571)(698.14754387,341.70457102)(698.07754745,341.6845743)
\curveto(698.00754401,341.67457105)(697.93754408,341.65457107)(697.86754745,341.6245743)
\curveto(697.65754436,341.53457119)(697.47754454,341.39957133)(697.32754745,341.2195743)
\curveto(697.18754483,341.03957169)(697.13754488,340.78957194)(697.17754745,340.4695743)
\curveto(697.19754482,340.29957243)(697.25254477,340.15957257)(697.34254745,340.0495743)
\curveto(697.41254461,339.93957279)(697.5175445,339.84957288)(697.65754745,339.7795743)
\curveto(697.79754422,339.71957301)(697.94754407,339.67457305)(698.10754745,339.6445743)
\curveto(698.27754374,339.61457311)(698.45254357,339.60457312)(698.63254745,339.6145743)
\curveto(698.8225432,339.63457309)(698.99754302,339.66957306)(699.15754745,339.7195743)
\curveto(699.4175426,339.79957293)(699.6225424,339.9245728)(699.77254745,340.0945743)
\curveto(699.9225421,340.27457245)(700.03754198,340.49457223)(700.11754745,340.7545743)
\curveto(700.13754188,340.8245719)(700.14754187,340.89457183)(700.14754745,340.9645743)
\curveto(700.15754186,341.04457168)(700.17254185,341.1245716)(700.19254745,341.2045743)
\lineto(700.19254745,341.3395743)
}
}
{
\newrgbcolor{curcolor}{0 0 0}
\pscustom[linestyle=none,fillstyle=solid,fillcolor=curcolor]
{
\newpath
\moveto(707.5308287,346.3795743)
\curveto(708.34082354,346.39956633)(709.01582287,346.27956645)(709.5558287,346.0195743)
\curveto(710.10582178,345.75956697)(710.54082134,345.38956734)(710.8608287,344.9095743)
\curveto(711.02082086,344.66956806)(711.14082074,344.39456833)(711.2208287,344.0845743)
\curveto(711.24082064,344.03456869)(711.25582063,343.96956876)(711.2658287,343.8895743)
\curveto(711.2858206,343.80956892)(711.2858206,343.73956899)(711.2658287,343.6795743)
\curveto(711.22582066,343.56956916)(711.15582073,343.50456922)(711.0558287,343.4845743)
\curveto(710.95582093,343.47456925)(710.83582105,343.46956926)(710.6958287,343.4695743)
\lineto(709.9158287,343.4695743)
\lineto(709.6308287,343.4695743)
\curveto(709.54082234,343.46956926)(709.46582242,343.48956924)(709.4058287,343.5295743)
\curveto(709.32582256,343.56956916)(709.27082261,343.6295691)(709.2408287,343.7095743)
\curveto(709.21082267,343.79956893)(709.17082271,343.88956884)(709.1208287,343.9795743)
\curveto(709.06082282,344.08956864)(708.99582289,344.18956854)(708.9258287,344.2795743)
\curveto(708.85582303,344.36956836)(708.77582311,344.44956828)(708.6858287,344.5195743)
\curveto(708.54582334,344.60956812)(708.39082349,344.67956805)(708.2208287,344.7295743)
\curveto(708.16082372,344.74956798)(708.10082378,344.75956797)(708.0408287,344.7595743)
\curveto(707.9808239,344.75956797)(707.92582396,344.76956796)(707.8758287,344.7895743)
\lineto(707.7258287,344.7895743)
\curveto(707.52582436,344.78956794)(707.36582452,344.76956796)(707.2458287,344.7295743)
\curveto(706.95582493,344.63956809)(706.72082516,344.49956823)(706.5408287,344.3095743)
\curveto(706.36082552,344.1295686)(706.21582567,343.90956882)(706.1058287,343.6495743)
\curveto(706.05582583,343.53956919)(706.01582587,343.41956931)(705.9858287,343.2895743)
\curveto(705.96582592,343.16956956)(705.94082594,343.03956969)(705.9108287,342.8995743)
\curveto(705.90082598,342.85956987)(705.89582599,342.81956991)(705.8958287,342.7795743)
\curveto(705.89582599,342.73956999)(705.89082599,342.69957003)(705.8808287,342.6595743)
\curveto(705.86082602,342.55957017)(705.85082603,342.41957031)(705.8508287,342.2395743)
\curveto(705.86082602,342.05957067)(705.87582601,341.91957081)(705.8958287,341.8195743)
\curveto(705.89582599,341.73957099)(705.90082598,341.68457104)(705.9108287,341.6545743)
\curveto(705.93082595,341.58457114)(705.94082594,341.51457121)(705.9408287,341.4445743)
\curveto(705.95082593,341.37457135)(705.96582592,341.30457142)(705.9858287,341.2345743)
\curveto(706.06582582,341.00457172)(706.16082572,340.79457193)(706.2708287,340.6045743)
\curveto(706.3808255,340.41457231)(706.52082536,340.25457247)(706.6908287,340.1245743)
\curveto(706.73082515,340.09457263)(706.79082509,340.05957267)(706.8708287,340.0195743)
\curveto(706.9808249,339.94957278)(707.09082479,339.90457282)(707.2008287,339.8845743)
\curveto(707.32082456,339.86457286)(707.46582442,339.84457288)(707.6358287,339.8245743)
\lineto(707.7258287,339.8245743)
\curveto(707.76582412,339.8245729)(707.79582409,339.8295729)(707.8158287,339.8395743)
\lineto(707.9508287,339.8395743)
\curveto(708.02082386,339.85957287)(708.0858238,339.87457285)(708.1458287,339.8845743)
\curveto(708.21582367,339.90457282)(708.2808236,339.9245728)(708.3408287,339.9445743)
\curveto(708.64082324,340.07457265)(708.87082301,340.26457246)(709.0308287,340.5145743)
\curveto(709.07082281,340.56457216)(709.10582278,340.61957211)(709.1358287,340.6795743)
\curveto(709.16582272,340.74957198)(709.19082269,340.80957192)(709.2108287,340.8595743)
\curveto(709.25082263,340.96957176)(709.2858226,341.06457166)(709.3158287,341.1445743)
\curveto(709.34582254,341.23457149)(709.41582247,341.30457142)(709.5258287,341.3545743)
\curveto(709.61582227,341.39457133)(709.76082212,341.40957132)(709.9608287,341.3995743)
\lineto(710.4558287,341.3995743)
\lineto(710.6658287,341.3995743)
\curveto(710.74582114,341.40957132)(710.81082107,341.40457132)(710.8608287,341.3845743)
\lineto(710.9808287,341.3845743)
\lineto(711.1008287,341.3545743)
\curveto(711.14082074,341.35457137)(711.17082071,341.34457138)(711.1908287,341.3245743)
\curveto(711.24082064,341.28457144)(711.27082061,341.2245715)(711.2808287,341.1445743)
\curveto(711.30082058,341.07457165)(711.30082058,340.99957173)(711.2808287,340.9195743)
\curveto(711.19082069,340.58957214)(711.0808208,340.29457243)(710.9508287,340.0345743)
\curveto(710.54082134,339.26457346)(709.885822,338.729574)(708.9858287,338.4295743)
\curveto(708.885823,338.39957433)(708.7808231,338.37957435)(708.6708287,338.3695743)
\curveto(708.56082332,338.34957438)(708.45082343,338.3245744)(708.3408287,338.2945743)
\curveto(708.2808236,338.28457444)(708.22082366,338.27957445)(708.1608287,338.2795743)
\curveto(708.10082378,338.27957445)(708.04082384,338.27457445)(707.9808287,338.2645743)
\lineto(707.8158287,338.2645743)
\curveto(707.76582412,338.24457448)(707.69082419,338.23957449)(707.5908287,338.2495743)
\curveto(707.49082439,338.24957448)(707.41582447,338.25457447)(707.3658287,338.2645743)
\curveto(707.2858246,338.28457444)(707.21082467,338.29457443)(707.1408287,338.2945743)
\curveto(707.0808248,338.28457444)(707.01582487,338.28957444)(706.9458287,338.3095743)
\lineto(706.7958287,338.3395743)
\curveto(706.74582514,338.33957439)(706.69582519,338.34457438)(706.6458287,338.3545743)
\curveto(706.53582535,338.38457434)(706.43082545,338.41457431)(706.3308287,338.4445743)
\curveto(706.23082565,338.47457425)(706.13582575,338.50957422)(706.0458287,338.5495743)
\curveto(705.57582631,338.74957398)(705.1808267,339.00457372)(704.8608287,339.3145743)
\curveto(704.54082734,339.63457309)(704.2808276,340.0295727)(704.0808287,340.4995743)
\curveto(704.03082785,340.58957214)(703.99082789,340.68457204)(703.9608287,340.7845743)
\lineto(703.8708287,341.1145743)
\curveto(703.86082802,341.15457157)(703.85582803,341.18957154)(703.8558287,341.2195743)
\curveto(703.85582803,341.25957147)(703.84582804,341.30457142)(703.8258287,341.3545743)
\curveto(703.80582808,341.4245713)(703.79582809,341.49457123)(703.7958287,341.5645743)
\curveto(703.79582809,341.64457108)(703.7858281,341.71957101)(703.7658287,341.7895743)
\lineto(703.7658287,342.0445743)
\curveto(703.74582814,342.09457063)(703.73582815,342.14957058)(703.7358287,342.2095743)
\curveto(703.73582815,342.27957045)(703.74582814,342.33957039)(703.7658287,342.3895743)
\curveto(703.77582811,342.43957029)(703.77582811,342.48457024)(703.7658287,342.5245743)
\curveto(703.75582813,342.56457016)(703.75582813,342.60457012)(703.7658287,342.6445743)
\curveto(703.7858281,342.71457001)(703.79082809,342.77956995)(703.7808287,342.8395743)
\curveto(703.7808281,342.89956983)(703.79082809,342.95956977)(703.8108287,343.0195743)
\curveto(703.86082802,343.19956953)(703.90082798,343.36956936)(703.9308287,343.5295743)
\curveto(703.96082792,343.69956903)(704.00582788,343.86456886)(704.0658287,344.0245743)
\curveto(704.2858276,344.53456819)(704.56082732,344.95956777)(704.8908287,345.2995743)
\curveto(705.23082665,345.63956709)(705.66082622,345.91456681)(706.1808287,346.1245743)
\curveto(706.32082556,346.18456654)(706.46582542,346.2245665)(706.6158287,346.2445743)
\curveto(706.76582512,346.27456645)(706.92082496,346.30956642)(707.0808287,346.3495743)
\curveto(707.16082472,346.35956637)(707.23582465,346.36456636)(707.3058287,346.3645743)
\curveto(707.37582451,346.36456636)(707.45082443,346.36956636)(707.5308287,346.3795743)
}
}
{
\newrgbcolor{curcolor}{0 0 0}
\pscustom[linestyle=none,fillstyle=solid,fillcolor=curcolor]
{
\newpath
\moveto(714.67410995,349.0195743)
\curveto(714.744107,348.93956379)(714.77910697,348.81956391)(714.77910995,348.6595743)
\lineto(714.77910995,348.1945743)
\lineto(714.77910995,347.7895743)
\curveto(714.77910697,347.64956508)(714.744107,347.55456517)(714.67410995,347.5045743)
\curveto(714.61410713,347.45456527)(714.53410721,347.4245653)(714.43410995,347.4145743)
\curveto(714.3441074,347.40456532)(714.2441075,347.39956533)(714.13410995,347.3995743)
\lineto(713.29410995,347.3995743)
\curveto(713.18410856,347.39956533)(713.08410866,347.40456532)(712.99410995,347.4145743)
\curveto(712.91410883,347.4245653)(712.8441089,347.45456527)(712.78410995,347.5045743)
\curveto(712.744109,347.53456519)(712.71410903,347.58956514)(712.69410995,347.6695743)
\curveto(712.68410906,347.75956497)(712.67410907,347.85456487)(712.66410995,347.9545743)
\lineto(712.66410995,348.2845743)
\curveto(712.67410907,348.39456433)(712.67910907,348.48956424)(712.67910995,348.5695743)
\lineto(712.67910995,348.7795743)
\curveto(712.68910906,348.84956388)(712.70910904,348.90956382)(712.73910995,348.9595743)
\curveto(712.75910899,348.99956373)(712.78410896,349.0295637)(712.81410995,349.0495743)
\lineto(712.93410995,349.1095743)
\curveto(712.95410879,349.10956362)(712.97910877,349.10956362)(713.00910995,349.1095743)
\curveto(713.03910871,349.11956361)(713.06410868,349.1245636)(713.08410995,349.1245743)
\lineto(714.17910995,349.1245743)
\curveto(714.27910747,349.1245636)(714.37410737,349.11956361)(714.46410995,349.1095743)
\curveto(714.55410719,349.09956363)(714.62410712,349.06956366)(714.67410995,349.0195743)
\moveto(714.77910995,339.2545743)
\curveto(714.77910697,339.05457367)(714.77410697,338.88457384)(714.76410995,338.7445743)
\curveto(714.75410699,338.60457412)(714.66410708,338.50957422)(714.49410995,338.4595743)
\curveto(714.43410731,338.43957429)(714.36910738,338.4295743)(714.29910995,338.4295743)
\curveto(714.22910752,338.43957429)(714.15410759,338.44457428)(714.07410995,338.4445743)
\lineto(713.23410995,338.4445743)
\curveto(713.1441086,338.44457428)(713.05410869,338.44957428)(712.96410995,338.4595743)
\curveto(712.88410886,338.46957426)(712.82410892,338.49957423)(712.78410995,338.5495743)
\curveto(712.72410902,338.61957411)(712.68910906,338.70457402)(712.67910995,338.8045743)
\lineto(712.67910995,339.1495743)
\lineto(712.67910995,345.4795743)
\lineto(712.67910995,345.7795743)
\curveto(712.67910907,345.87956685)(712.69910905,345.95956677)(712.73910995,346.0195743)
\curveto(712.79910895,346.08956664)(712.88410886,346.13456659)(712.99410995,346.1545743)
\curveto(713.01410873,346.16456656)(713.03910871,346.16456656)(713.06910995,346.1545743)
\curveto(713.10910864,346.15456657)(713.13910861,346.15956657)(713.15910995,346.1695743)
\lineto(713.90910995,346.1695743)
\lineto(714.10410995,346.1695743)
\curveto(714.18410756,346.17956655)(714.2491075,346.17956655)(714.29910995,346.1695743)
\lineto(714.41910995,346.1695743)
\curveto(714.47910727,346.14956658)(714.53410721,346.13456659)(714.58410995,346.1245743)
\curveto(714.63410711,346.11456661)(714.67410707,346.08456664)(714.70410995,346.0345743)
\curveto(714.744107,345.98456674)(714.76410698,345.91456681)(714.76410995,345.8245743)
\curveto(714.77410697,345.73456699)(714.77910697,345.63956709)(714.77910995,345.5395743)
\lineto(714.77910995,339.2545743)
}
}
{
\newrgbcolor{curcolor}{0 0 0}
\pscustom[linestyle=none,fillstyle=solid,fillcolor=curcolor]
{
\newpath
\moveto(724.21129745,342.6145743)
\curveto(724.23128888,342.55457017)(724.24128887,342.46957026)(724.24129745,342.3595743)
\curveto(724.24128887,342.24957048)(724.23128888,342.16457056)(724.21129745,342.1045743)
\lineto(724.21129745,341.9545743)
\curveto(724.19128892,341.87457085)(724.18128893,341.79457093)(724.18129745,341.7145743)
\curveto(724.19128892,341.63457109)(724.18628893,341.55457117)(724.16629745,341.4745743)
\curveto(724.14628897,341.40457132)(724.13128898,341.33957139)(724.12129745,341.2795743)
\curveto(724.111289,341.21957151)(724.10128901,341.15457157)(724.09129745,341.0845743)
\curveto(724.05128906,340.97457175)(724.0162891,340.85957187)(723.98629745,340.7395743)
\curveto(723.95628916,340.6295721)(723.9162892,340.5245722)(723.86629745,340.4245743)
\curveto(723.65628946,339.94457278)(723.38128973,339.55457317)(723.04129745,339.2545743)
\curveto(722.70129041,338.95457377)(722.29129082,338.70457402)(721.81129745,338.5045743)
\curveto(721.69129142,338.45457427)(721.56629155,338.41957431)(721.43629745,338.3995743)
\curveto(721.3162918,338.36957436)(721.19129192,338.33957439)(721.06129745,338.3095743)
\curveto(721.0112921,338.28957444)(720.95629216,338.27957445)(720.89629745,338.2795743)
\curveto(720.83629228,338.27957445)(720.78129233,338.27457445)(720.73129745,338.2645743)
\lineto(720.62629745,338.2645743)
\curveto(720.59629252,338.25457447)(720.56629255,338.24957448)(720.53629745,338.2495743)
\curveto(720.48629263,338.23957449)(720.40629271,338.23457449)(720.29629745,338.2345743)
\curveto(720.18629293,338.2245745)(720.10129301,338.2295745)(720.04129745,338.2495743)
\lineto(719.89129745,338.2495743)
\curveto(719.84129327,338.25957447)(719.78629333,338.26457446)(719.72629745,338.2645743)
\curveto(719.67629344,338.25457447)(719.62629349,338.25957447)(719.57629745,338.2795743)
\curveto(719.53629358,338.28957444)(719.49629362,338.29457443)(719.45629745,338.2945743)
\curveto(719.42629369,338.29457443)(719.38629373,338.29957443)(719.33629745,338.3095743)
\curveto(719.23629388,338.33957439)(719.13629398,338.36457436)(719.03629745,338.3845743)
\curveto(718.93629418,338.40457432)(718.84129427,338.43457429)(718.75129745,338.4745743)
\curveto(718.63129448,338.51457421)(718.5162946,338.55457417)(718.40629745,338.5945743)
\curveto(718.30629481,338.63457409)(718.20129491,338.68457404)(718.09129745,338.7445743)
\curveto(717.74129537,338.95457377)(717.44129567,339.19957353)(717.19129745,339.4795743)
\curveto(716.94129617,339.75957297)(716.73129638,340.09457263)(716.56129745,340.4845743)
\curveto(716.5112966,340.57457215)(716.47129664,340.66957206)(716.44129745,340.7695743)
\curveto(716.42129669,340.86957186)(716.39629672,340.97457175)(716.36629745,341.0845743)
\curveto(716.34629677,341.13457159)(716.33629678,341.17957155)(716.33629745,341.2195743)
\curveto(716.33629678,341.25957147)(716.32629679,341.30457142)(716.30629745,341.3545743)
\curveto(716.28629683,341.43457129)(716.27629684,341.51457121)(716.27629745,341.5945743)
\curveto(716.27629684,341.68457104)(716.26629685,341.76957096)(716.24629745,341.8495743)
\curveto(716.23629688,341.89957083)(716.23129688,341.94457078)(716.23129745,341.9845743)
\lineto(716.23129745,342.1195743)
\curveto(716.2112969,342.17957055)(716.20129691,342.26457046)(716.20129745,342.3745743)
\curveto(716.2112969,342.48457024)(716.22629689,342.56957016)(716.24629745,342.6295743)
\lineto(716.24629745,342.7345743)
\curveto(716.25629686,342.78456994)(716.25629686,342.83456989)(716.24629745,342.8845743)
\curveto(716.24629687,342.94456978)(716.25629686,342.99956973)(716.27629745,343.0495743)
\curveto(716.28629683,343.09956963)(716.29129682,343.14456958)(716.29129745,343.1845743)
\curveto(716.29129682,343.23456949)(716.30129681,343.28456944)(716.32129745,343.3345743)
\curveto(716.36129675,343.46456926)(716.39629672,343.58956914)(716.42629745,343.7095743)
\curveto(716.45629666,343.83956889)(716.49629662,343.96456876)(716.54629745,344.0845743)
\curveto(716.72629639,344.49456823)(716.94129617,344.83456789)(717.19129745,345.1045743)
\curveto(717.44129567,345.38456734)(717.74629537,345.63956709)(718.10629745,345.8695743)
\curveto(718.20629491,345.91956681)(718.3112948,345.96456676)(718.42129745,346.0045743)
\curveto(718.53129458,346.04456668)(718.64129447,346.08956664)(718.75129745,346.1395743)
\curveto(718.88129423,346.18956654)(719.0162941,346.2245665)(719.15629745,346.2445743)
\curveto(719.29629382,346.26456646)(719.44129367,346.29456643)(719.59129745,346.3345743)
\curveto(719.67129344,346.34456638)(719.74629337,346.34956638)(719.81629745,346.3495743)
\curveto(719.88629323,346.34956638)(719.95629316,346.35456637)(720.02629745,346.3645743)
\curveto(720.60629251,346.37456635)(721.10629201,346.31456641)(721.52629745,346.1845743)
\curveto(721.95629116,346.05456667)(722.33629078,345.87456685)(722.66629745,345.6445743)
\curveto(722.77629034,345.56456716)(722.88629023,345.47456725)(722.99629745,345.3745743)
\curveto(723.11629,345.28456744)(723.2162899,345.18456754)(723.29629745,345.0745743)
\curveto(723.37628974,344.97456775)(723.44628967,344.87456785)(723.50629745,344.7745743)
\curveto(723.57628954,344.67456805)(723.64628947,344.56956816)(723.71629745,344.4595743)
\curveto(723.78628933,344.34956838)(723.84128927,344.2295685)(723.88129745,344.0995743)
\curveto(723.92128919,343.97956875)(723.96628915,343.84956888)(724.01629745,343.7095743)
\curveto(724.04628907,343.6295691)(724.07128904,343.54456918)(724.09129745,343.4545743)
\lineto(724.15129745,343.1845743)
\curveto(724.16128895,343.14456958)(724.16628895,343.10456962)(724.16629745,343.0645743)
\curveto(724.16628895,343.0245697)(724.17128894,342.98456974)(724.18129745,342.9445743)
\curveto(724.20128891,342.89456983)(724.20628891,342.83956989)(724.19629745,342.7795743)
\curveto(724.18628893,342.71957001)(724.19128892,342.66457006)(724.21129745,342.6145743)
\moveto(722.11129745,342.0745743)
\curveto(722.12129099,342.1245706)(722.12629099,342.19457053)(722.12629745,342.2845743)
\curveto(722.12629099,342.38457034)(722.12129099,342.45957027)(722.11129745,342.5095743)
\lineto(722.11129745,342.6295743)
\curveto(722.09129102,342.67957005)(722.08129103,342.73456999)(722.08129745,342.7945743)
\curveto(722.08129103,342.85456987)(722.07629104,342.90956982)(722.06629745,342.9595743)
\curveto(722.06629105,342.99956973)(722.06129105,343.0295697)(722.05129745,343.0495743)
\lineto(721.99129745,343.2895743)
\curveto(721.98129113,343.37956935)(721.96129115,343.46456926)(721.93129745,343.5445743)
\curveto(721.82129129,343.80456892)(721.69129142,344.0245687)(721.54129745,344.2045743)
\curveto(721.39129172,344.39456833)(721.19129192,344.54456818)(720.94129745,344.6545743)
\curveto(720.88129223,344.67456805)(720.82129229,344.68956804)(720.76129745,344.6995743)
\curveto(720.70129241,344.71956801)(720.63629248,344.73956799)(720.56629745,344.7595743)
\curveto(720.48629263,344.77956795)(720.40129271,344.78456794)(720.31129745,344.7745743)
\lineto(720.04129745,344.7745743)
\curveto(720.0112931,344.75456797)(719.97629314,344.74456798)(719.93629745,344.7445743)
\curveto(719.89629322,344.75456797)(719.86129325,344.75456797)(719.83129745,344.7445743)
\lineto(719.62129745,344.6845743)
\curveto(719.56129355,344.67456805)(719.50629361,344.65456807)(719.45629745,344.6245743)
\curveto(719.20629391,344.51456821)(719.00129411,344.35456837)(718.84129745,344.1445743)
\curveto(718.69129442,343.94456878)(718.57129454,343.70956902)(718.48129745,343.4395743)
\curveto(718.45129466,343.33956939)(718.42629469,343.23456949)(718.40629745,343.1245743)
\curveto(718.39629472,343.01456971)(718.38129473,342.90456982)(718.36129745,342.7945743)
\curveto(718.35129476,342.74456998)(718.34629477,342.69457003)(718.34629745,342.6445743)
\lineto(718.34629745,342.4945743)
\curveto(718.32629479,342.4245703)(718.3162948,342.31957041)(718.31629745,342.1795743)
\curveto(718.32629479,342.03957069)(718.34129477,341.93457079)(718.36129745,341.8645743)
\lineto(718.36129745,341.7295743)
\curveto(718.38129473,341.64957108)(718.39629472,341.56957116)(718.40629745,341.4895743)
\curveto(718.4162947,341.41957131)(718.43129468,341.34457138)(718.45129745,341.2645743)
\curveto(718.55129456,340.96457176)(718.65629446,340.71957201)(718.76629745,340.5295743)
\curveto(718.88629423,340.34957238)(719.07129404,340.18457254)(719.32129745,340.0345743)
\curveto(719.39129372,339.98457274)(719.46629365,339.94457278)(719.54629745,339.9145743)
\curveto(719.63629348,339.88457284)(719.72629339,339.85957287)(719.81629745,339.8395743)
\curveto(719.85629326,339.8295729)(719.89129322,339.8245729)(719.92129745,339.8245743)
\curveto(719.95129316,339.83457289)(719.98629313,339.83457289)(720.02629745,339.8245743)
\lineto(720.14629745,339.7945743)
\curveto(720.19629292,339.79457293)(720.24129287,339.79957293)(720.28129745,339.8095743)
\lineto(720.40129745,339.8095743)
\curveto(720.48129263,339.8295729)(720.56129255,339.84457288)(720.64129745,339.8545743)
\curveto(720.72129239,339.86457286)(720.79629232,339.88457284)(720.86629745,339.9145743)
\curveto(721.12629199,340.01457271)(721.33629178,340.14957258)(721.49629745,340.3195743)
\curveto(721.65629146,340.48957224)(721.79129132,340.69957203)(721.90129745,340.9495743)
\curveto(721.94129117,341.04957168)(721.97129114,341.14957158)(721.99129745,341.2495743)
\curveto(722.0112911,341.34957138)(722.03629108,341.45457127)(722.06629745,341.5645743)
\curveto(722.07629104,341.60457112)(722.08129103,341.63957109)(722.08129745,341.6695743)
\curveto(722.08129103,341.70957102)(722.08629103,341.74957098)(722.09629745,341.7895743)
\lineto(722.09629745,341.9245743)
\curveto(722.09629102,341.97457075)(722.10129101,342.0245707)(722.11129745,342.0745743)
}
}
{
\newrgbcolor{curcolor}{0 0 0}
\pscustom[linestyle=none,fillstyle=solid,fillcolor=curcolor]
{
\newpath
\moveto(263.08072616,269.42545931)
\curveto(263.18072131,269.42544869)(263.27572121,269.4154487)(263.36572616,269.39545931)
\curveto(263.45572103,269.38544873)(263.52072097,269.35544876)(263.56072616,269.30545931)
\curveto(263.62072087,269.22544889)(263.65072084,269.12044899)(263.65072616,268.99045931)
\lineto(263.65072616,268.60045931)
\lineto(263.65072616,267.10045931)
\lineto(263.65072616,260.71045931)
\lineto(263.65072616,259.54045931)
\lineto(263.65072616,259.22545931)
\curveto(263.66072083,259.12545899)(263.64572084,259.04545907)(263.60572616,258.98545931)
\curveto(263.55572093,258.90545921)(263.48072101,258.85545926)(263.38072616,258.83545931)
\curveto(263.2907212,258.82545929)(263.18072131,258.82045929)(263.05072616,258.82045931)
\lineto(262.82572616,258.82045931)
\curveto(262.74572174,258.84045927)(262.67572181,258.85545926)(262.61572616,258.86545931)
\curveto(262.55572193,258.88545923)(262.50572198,258.92545919)(262.46572616,258.98545931)
\curveto(262.42572206,259.04545907)(262.40572208,259.12045899)(262.40572616,259.21045931)
\lineto(262.40572616,259.51045931)
\lineto(262.40572616,260.60545931)
\lineto(262.40572616,265.94545931)
\curveto(262.3857221,266.03545208)(262.37072212,266.110452)(262.36072616,266.17045931)
\curveto(262.36072213,266.24045187)(262.33072216,266.30045181)(262.27072616,266.35045931)
\curveto(262.20072229,266.40045171)(262.11072238,266.42545169)(262.00072616,266.42545931)
\curveto(261.90072259,266.43545168)(261.7907227,266.44045167)(261.67072616,266.44045931)
\lineto(260.53072616,266.44045931)
\lineto(260.03572616,266.44045931)
\curveto(259.87572461,266.45045166)(259.76572472,266.5104516)(259.70572616,266.62045931)
\curveto(259.6857248,266.65045146)(259.67572481,266.68045143)(259.67572616,266.71045931)
\curveto(259.67572481,266.75045136)(259.67072482,266.79545132)(259.66072616,266.84545931)
\curveto(259.64072485,266.96545115)(259.64572484,267.07545104)(259.67572616,267.17545931)
\curveto(259.71572477,267.27545084)(259.77072472,267.34545077)(259.84072616,267.38545931)
\curveto(259.92072457,267.43545068)(260.04072445,267.46045065)(260.20072616,267.46045931)
\curveto(260.36072413,267.46045065)(260.49572399,267.47545064)(260.60572616,267.50545931)
\curveto(260.65572383,267.5154506)(260.71072378,267.52045059)(260.77072616,267.52045931)
\curveto(260.83072366,267.53045058)(260.8907236,267.54545057)(260.95072616,267.56545931)
\curveto(261.10072339,267.6154505)(261.24572324,267.66545045)(261.38572616,267.71545931)
\curveto(261.52572296,267.77545034)(261.66072283,267.84545027)(261.79072616,267.92545931)
\curveto(261.93072256,268.0154501)(262.05072244,268.12044999)(262.15072616,268.24045931)
\curveto(262.25072224,268.36044975)(262.34572214,268.49044962)(262.43572616,268.63045931)
\curveto(262.49572199,268.73044938)(262.54072195,268.84044927)(262.57072616,268.96045931)
\curveto(262.61072188,269.08044903)(262.66072183,269.18544893)(262.72072616,269.27545931)
\curveto(262.77072172,269.33544878)(262.84072165,269.37544874)(262.93072616,269.39545931)
\curveto(262.95072154,269.40544871)(262.97572151,269.4104487)(263.00572616,269.41045931)
\curveto(263.03572145,269.4104487)(263.06072143,269.4154487)(263.08072616,269.42545931)
}
}
{
\newrgbcolor{curcolor}{0 0 0}
\pscustom[linestyle=none,fillstyle=solid,fillcolor=curcolor]
{
\newpath
\moveto(270.62033554,269.42545931)
\curveto(271.3103309,269.43544868)(271.9103303,269.3154488)(272.42033554,269.06545931)
\curveto(272.94032927,268.8154493)(273.33532888,268.48044963)(273.60533554,268.06045931)
\curveto(273.65532856,267.98045013)(273.70032851,267.89045022)(273.74033554,267.79045931)
\curveto(273.78032843,267.70045041)(273.82532839,267.60545051)(273.87533554,267.50545931)
\curveto(273.9153283,267.40545071)(273.94532827,267.30545081)(273.96533554,267.20545931)
\curveto(273.98532823,267.10545101)(274.00532821,267.00045111)(274.02533554,266.89045931)
\curveto(274.04532817,266.84045127)(274.05032816,266.79545132)(274.04033554,266.75545931)
\curveto(274.03032818,266.7154514)(274.03532818,266.67045144)(274.05533554,266.62045931)
\curveto(274.06532815,266.57045154)(274.07032814,266.48545163)(274.07033554,266.36545931)
\curveto(274.07032814,266.25545186)(274.06532815,266.17045194)(274.05533554,266.11045931)
\curveto(274.03532818,266.05045206)(274.02532819,265.99045212)(274.02533554,265.93045931)
\curveto(274.03532818,265.87045224)(274.03032818,265.8104523)(274.01033554,265.75045931)
\curveto(273.97032824,265.6104525)(273.93532828,265.47545264)(273.90533554,265.34545931)
\curveto(273.87532834,265.2154529)(273.83532838,265.09045302)(273.78533554,264.97045931)
\curveto(273.72532849,264.83045328)(273.65532856,264.70545341)(273.57533554,264.59545931)
\curveto(273.50532871,264.48545363)(273.43032878,264.37545374)(273.35033554,264.26545931)
\lineto(273.29033554,264.20545931)
\curveto(273.28032893,264.18545393)(273.26532895,264.16545395)(273.24533554,264.14545931)
\curveto(273.12532909,263.98545413)(272.99032922,263.84045427)(272.84033554,263.71045931)
\curveto(272.69032952,263.58045453)(272.53032968,263.45545466)(272.36033554,263.33545931)
\curveto(272.05033016,263.115455)(271.75533046,262.9104552)(271.47533554,262.72045931)
\curveto(271.24533097,262.58045553)(271.0153312,262.44545567)(270.78533554,262.31545931)
\curveto(270.56533165,262.18545593)(270.34533187,262.05045606)(270.12533554,261.91045931)
\curveto(269.87533234,261.74045637)(269.63533258,261.56045655)(269.40533554,261.37045931)
\curveto(269.18533303,261.18045693)(268.99533322,260.95545716)(268.83533554,260.69545931)
\curveto(268.79533342,260.63545748)(268.76033345,260.57545754)(268.73033554,260.51545931)
\curveto(268.70033351,260.46545765)(268.67033354,260.40045771)(268.64033554,260.32045931)
\curveto(268.62033359,260.25045786)(268.6153336,260.19045792)(268.62533554,260.14045931)
\curveto(268.64533357,260.07045804)(268.68033353,260.0154581)(268.73033554,259.97545931)
\curveto(268.78033343,259.94545817)(268.84033337,259.92545819)(268.91033554,259.91545931)
\lineto(269.15033554,259.91545931)
\lineto(269.90033554,259.91545931)
\lineto(272.70533554,259.91545931)
\lineto(273.36533554,259.91545931)
\curveto(273.45532876,259.9154582)(273.54032867,259.9104582)(273.62033554,259.90045931)
\curveto(273.70032851,259.90045821)(273.76532845,259.88045823)(273.81533554,259.84045931)
\curveto(273.86532835,259.80045831)(273.90532831,259.72545839)(273.93533554,259.61545931)
\curveto(273.97532824,259.5154586)(273.98532823,259.4154587)(273.96533554,259.31545931)
\lineto(273.96533554,259.18045931)
\curveto(273.94532827,259.110459)(273.92532829,259.05045906)(273.90533554,259.00045931)
\curveto(273.88532833,258.95045916)(273.85032836,258.9104592)(273.80033554,258.88045931)
\curveto(273.75032846,258.84045927)(273.68032853,258.82045929)(273.59033554,258.82045931)
\lineto(273.32033554,258.82045931)
\lineto(272.42033554,258.82045931)
\lineto(268.91033554,258.82045931)
\lineto(267.84533554,258.82045931)
\curveto(267.76533445,258.82045929)(267.67533454,258.8154593)(267.57533554,258.80545931)
\curveto(267.47533474,258.80545931)(267.39033482,258.8154593)(267.32033554,258.83545931)
\curveto(267.1103351,258.90545921)(267.04533517,259.08545903)(267.12533554,259.37545931)
\curveto(267.13533508,259.4154587)(267.13533508,259.45045866)(267.12533554,259.48045931)
\curveto(267.12533509,259.52045859)(267.13533508,259.56545855)(267.15533554,259.61545931)
\curveto(267.17533504,259.69545842)(267.19533502,259.78045833)(267.21533554,259.87045931)
\curveto(267.23533498,259.96045815)(267.26033495,260.04545807)(267.29033554,260.12545931)
\curveto(267.45033476,260.6154575)(267.65033456,261.03045708)(267.89033554,261.37045931)
\curveto(268.07033414,261.62045649)(268.27533394,261.84545627)(268.50533554,262.04545931)
\curveto(268.73533348,262.25545586)(268.97533324,262.45045566)(269.22533554,262.63045931)
\curveto(269.48533273,262.8104553)(269.75033246,262.98045513)(270.02033554,263.14045931)
\curveto(270.30033191,263.3104548)(270.57033164,263.48545463)(270.83033554,263.66545931)
\curveto(270.94033127,263.74545437)(271.04533117,263.82045429)(271.14533554,263.89045931)
\curveto(271.25533096,263.96045415)(271.36533085,264.03545408)(271.47533554,264.11545931)
\curveto(271.5153307,264.14545397)(271.55033066,264.17545394)(271.58033554,264.20545931)
\curveto(271.62033059,264.24545387)(271.66033055,264.27545384)(271.70033554,264.29545931)
\curveto(271.84033037,264.40545371)(271.96533025,264.53045358)(272.07533554,264.67045931)
\curveto(272.09533012,264.70045341)(272.12033009,264.72545339)(272.15033554,264.74545931)
\curveto(272.18033003,264.77545334)(272.20533001,264.80545331)(272.22533554,264.83545931)
\curveto(272.30532991,264.93545318)(272.37032984,265.03545308)(272.42033554,265.13545931)
\curveto(272.48032973,265.23545288)(272.53532968,265.34545277)(272.58533554,265.46545931)
\curveto(272.6153296,265.53545258)(272.63532958,265.6104525)(272.64533554,265.69045931)
\lineto(272.70533554,265.93045931)
\lineto(272.70533554,266.02045931)
\curveto(272.7153295,266.05045206)(272.72032949,266.08045203)(272.72033554,266.11045931)
\curveto(272.74032947,266.18045193)(272.74532947,266.27545184)(272.73533554,266.39545931)
\curveto(272.73532948,266.52545159)(272.72532949,266.62545149)(272.70533554,266.69545931)
\curveto(272.68532953,266.77545134)(272.66532955,266.85045126)(272.64533554,266.92045931)
\curveto(272.63532958,267.00045111)(272.6153296,267.08045103)(272.58533554,267.16045931)
\curveto(272.47532974,267.40045071)(272.32532989,267.60045051)(272.13533554,267.76045931)
\curveto(271.95533026,267.93045018)(271.73533048,268.07045004)(271.47533554,268.18045931)
\curveto(271.40533081,268.20044991)(271.33533088,268.2154499)(271.26533554,268.22545931)
\curveto(271.19533102,268.24544987)(271.12033109,268.26544985)(271.04033554,268.28545931)
\curveto(270.96033125,268.30544981)(270.85033136,268.3154498)(270.71033554,268.31545931)
\curveto(270.58033163,268.3154498)(270.47533174,268.30544981)(270.39533554,268.28545931)
\curveto(270.33533188,268.27544984)(270.28033193,268.27044984)(270.23033554,268.27045931)
\curveto(270.18033203,268.27044984)(270.13033208,268.26044985)(270.08033554,268.24045931)
\curveto(269.98033223,268.20044991)(269.88533233,268.16044995)(269.79533554,268.12045931)
\curveto(269.7153325,268.08045003)(269.63533258,268.03545008)(269.55533554,267.98545931)
\curveto(269.52533269,267.96545015)(269.49533272,267.94045017)(269.46533554,267.91045931)
\curveto(269.44533277,267.88045023)(269.42033279,267.85545026)(269.39033554,267.83545931)
\lineto(269.31533554,267.76045931)
\curveto(269.28533293,267.74045037)(269.26033295,267.72045039)(269.24033554,267.70045931)
\lineto(269.09033554,267.49045931)
\curveto(269.05033316,267.43045068)(269.00533321,267.36545075)(268.95533554,267.29545931)
\curveto(268.89533332,267.20545091)(268.84533337,267.10045101)(268.80533554,266.98045931)
\curveto(268.77533344,266.87045124)(268.74033347,266.76045135)(268.70033554,266.65045931)
\curveto(268.66033355,266.54045157)(268.63533358,266.39545172)(268.62533554,266.21545931)
\curveto(268.6153336,266.04545207)(268.58533363,265.92045219)(268.53533554,265.84045931)
\curveto(268.48533373,265.76045235)(268.4103338,265.7154524)(268.31033554,265.70545931)
\curveto(268.210334,265.69545242)(268.10033411,265.69045242)(267.98033554,265.69045931)
\curveto(267.94033427,265.69045242)(267.90033431,265.68545243)(267.86033554,265.67545931)
\curveto(267.82033439,265.67545244)(267.78533443,265.68045243)(267.75533554,265.69045931)
\curveto(267.70533451,265.7104524)(267.65533456,265.72045239)(267.60533554,265.72045931)
\curveto(267.56533465,265.72045239)(267.52533469,265.73045238)(267.48533554,265.75045931)
\curveto(267.39533482,265.8104523)(267.35033486,265.94545217)(267.35033554,266.15545931)
\lineto(267.35033554,266.27545931)
\curveto(267.36033485,266.33545178)(267.36533485,266.39545172)(267.36533554,266.45545931)
\curveto(267.37533484,266.52545159)(267.38533483,266.59045152)(267.39533554,266.65045931)
\curveto(267.4153348,266.76045135)(267.43533478,266.86045125)(267.45533554,266.95045931)
\curveto(267.47533474,267.05045106)(267.50533471,267.14545097)(267.54533554,267.23545931)
\curveto(267.56533465,267.30545081)(267.58533463,267.36545075)(267.60533554,267.41545931)
\lineto(267.66533554,267.59545931)
\curveto(267.78533443,267.85545026)(267.94033427,268.10045001)(268.13033554,268.33045931)
\curveto(268.33033388,268.56044955)(268.54533367,268.74544937)(268.77533554,268.88545931)
\curveto(268.88533333,268.96544915)(269.00033321,269.03044908)(269.12033554,269.08045931)
\lineto(269.51033554,269.23045931)
\curveto(269.62033259,269.28044883)(269.73533248,269.3104488)(269.85533554,269.32045931)
\curveto(269.97533224,269.34044877)(270.10033211,269.36544875)(270.23033554,269.39545931)
\curveto(270.30033191,269.39544872)(270.36533185,269.39544872)(270.42533554,269.39545931)
\curveto(270.48533173,269.40544871)(270.55033166,269.4154487)(270.62033554,269.42545931)
}
}
{
\newrgbcolor{curcolor}{0 0 0}
\pscustom[linestyle=none,fillstyle=solid,fillcolor=curcolor]
{
\newpath
\moveto(276.67494491,260.45545931)
\lineto(276.97494491,260.45545931)
\curveto(277.08494285,260.46545765)(277.18994275,260.46545765)(277.28994491,260.45545931)
\curveto(277.39994254,260.45545766)(277.49994244,260.44545767)(277.58994491,260.42545931)
\curveto(277.67994226,260.4154577)(277.74994219,260.39045772)(277.79994491,260.35045931)
\curveto(277.81994212,260.33045778)(277.8349421,260.30045781)(277.84494491,260.26045931)
\curveto(277.86494207,260.22045789)(277.88494205,260.17545794)(277.90494491,260.12545931)
\lineto(277.90494491,260.05045931)
\curveto(277.91494202,260.00045811)(277.91494202,259.94545817)(277.90494491,259.88545931)
\lineto(277.90494491,259.73545931)
\lineto(277.90494491,259.25545931)
\curveto(277.90494203,259.08545903)(277.86494207,258.96545915)(277.78494491,258.89545931)
\curveto(277.71494222,258.84545927)(277.62494231,258.82045929)(277.51494491,258.82045931)
\lineto(277.18494491,258.82045931)
\lineto(276.73494491,258.82045931)
\curveto(276.58494335,258.82045929)(276.46994347,258.85045926)(276.38994491,258.91045931)
\curveto(276.34994359,258.94045917)(276.31994362,258.99045912)(276.29994491,259.06045931)
\curveto(276.27994366,259.14045897)(276.26494367,259.22545889)(276.25494491,259.31545931)
\lineto(276.25494491,259.60045931)
\curveto(276.26494367,259.70045841)(276.26994367,259.78545833)(276.26994491,259.85545931)
\lineto(276.26994491,260.05045931)
\curveto(276.26994367,260.110458)(276.27994366,260.16545795)(276.29994491,260.21545931)
\curveto(276.3399436,260.32545779)(276.40994353,260.39545772)(276.50994491,260.42545931)
\curveto(276.5399434,260.42545769)(276.59494334,260.43545768)(276.67494491,260.45545931)
}
}
{
\newrgbcolor{curcolor}{0 0 0}
\pscustom[linestyle=none,fillstyle=solid,fillcolor=curcolor]
{
\newpath
\moveto(283.94010116,269.42545931)
\curveto(284.04009631,269.42544869)(284.13509621,269.4154487)(284.22510116,269.39545931)
\curveto(284.31509603,269.38544873)(284.38009597,269.35544876)(284.42010116,269.30545931)
\curveto(284.48009587,269.22544889)(284.51009584,269.12044899)(284.51010116,268.99045931)
\lineto(284.51010116,268.60045931)
\lineto(284.51010116,267.10045931)
\lineto(284.51010116,260.71045931)
\lineto(284.51010116,259.54045931)
\lineto(284.51010116,259.22545931)
\curveto(284.52009583,259.12545899)(284.50509584,259.04545907)(284.46510116,258.98545931)
\curveto(284.41509593,258.90545921)(284.34009601,258.85545926)(284.24010116,258.83545931)
\curveto(284.1500962,258.82545929)(284.04009631,258.82045929)(283.91010116,258.82045931)
\lineto(283.68510116,258.82045931)
\curveto(283.60509674,258.84045927)(283.53509681,258.85545926)(283.47510116,258.86545931)
\curveto(283.41509693,258.88545923)(283.36509698,258.92545919)(283.32510116,258.98545931)
\curveto(283.28509706,259.04545907)(283.26509708,259.12045899)(283.26510116,259.21045931)
\lineto(283.26510116,259.51045931)
\lineto(283.26510116,260.60545931)
\lineto(283.26510116,265.94545931)
\curveto(283.2450971,266.03545208)(283.23009712,266.110452)(283.22010116,266.17045931)
\curveto(283.22009713,266.24045187)(283.19009716,266.30045181)(283.13010116,266.35045931)
\curveto(283.06009729,266.40045171)(282.97009738,266.42545169)(282.86010116,266.42545931)
\curveto(282.76009759,266.43545168)(282.6500977,266.44045167)(282.53010116,266.44045931)
\lineto(281.39010116,266.44045931)
\lineto(280.89510116,266.44045931)
\curveto(280.73509961,266.45045166)(280.62509972,266.5104516)(280.56510116,266.62045931)
\curveto(280.5450998,266.65045146)(280.53509981,266.68045143)(280.53510116,266.71045931)
\curveto(280.53509981,266.75045136)(280.53009982,266.79545132)(280.52010116,266.84545931)
\curveto(280.50009985,266.96545115)(280.50509984,267.07545104)(280.53510116,267.17545931)
\curveto(280.57509977,267.27545084)(280.63009972,267.34545077)(280.70010116,267.38545931)
\curveto(280.78009957,267.43545068)(280.90009945,267.46045065)(281.06010116,267.46045931)
\curveto(281.22009913,267.46045065)(281.35509899,267.47545064)(281.46510116,267.50545931)
\curveto(281.51509883,267.5154506)(281.57009878,267.52045059)(281.63010116,267.52045931)
\curveto(281.69009866,267.53045058)(281.7500986,267.54545057)(281.81010116,267.56545931)
\curveto(281.96009839,267.6154505)(282.10509824,267.66545045)(282.24510116,267.71545931)
\curveto(282.38509796,267.77545034)(282.52009783,267.84545027)(282.65010116,267.92545931)
\curveto(282.79009756,268.0154501)(282.91009744,268.12044999)(283.01010116,268.24045931)
\curveto(283.11009724,268.36044975)(283.20509714,268.49044962)(283.29510116,268.63045931)
\curveto(283.35509699,268.73044938)(283.40009695,268.84044927)(283.43010116,268.96045931)
\curveto(283.47009688,269.08044903)(283.52009683,269.18544893)(283.58010116,269.27545931)
\curveto(283.63009672,269.33544878)(283.70009665,269.37544874)(283.79010116,269.39545931)
\curveto(283.81009654,269.40544871)(283.83509651,269.4104487)(283.86510116,269.41045931)
\curveto(283.89509645,269.4104487)(283.92009643,269.4154487)(283.94010116,269.42545931)
}
}
{
\newrgbcolor{curcolor}{0 0 0}
\pscustom[linestyle=none,fillstyle=solid,fillcolor=curcolor]
{
\newpath
\moveto(298.03471054,267.34045931)
\curveto(297.83470024,267.05045106)(297.62470045,266.76545135)(297.40471054,266.48545931)
\curveto(297.19470088,266.20545191)(296.98970108,265.92045219)(296.78971054,265.63045931)
\curveto(296.18970188,264.78045333)(295.58470249,263.94045417)(294.97471054,263.11045931)
\curveto(294.36470371,262.29045582)(293.75970431,261.45545666)(293.15971054,260.60545931)
\lineto(292.64971054,259.88545931)
\lineto(292.13971054,259.19545931)
\curveto(292.05970601,259.08545903)(291.97970609,258.97045914)(291.89971054,258.85045931)
\curveto(291.81970625,258.73045938)(291.72470635,258.63545948)(291.61471054,258.56545931)
\curveto(291.5747065,258.54545957)(291.50970656,258.53045958)(291.41971054,258.52045931)
\curveto(291.33970673,258.50045961)(291.24970682,258.49045962)(291.14971054,258.49045931)
\curveto(291.04970702,258.49045962)(290.95470712,258.49545962)(290.86471054,258.50545931)
\curveto(290.78470729,258.5154596)(290.72470735,258.53545958)(290.68471054,258.56545931)
\curveto(290.65470742,258.58545953)(290.62970744,258.62045949)(290.60971054,258.67045931)
\curveto(290.59970747,258.7104594)(290.60470747,258.75545936)(290.62471054,258.80545931)
\curveto(290.66470741,258.88545923)(290.70970736,258.96045915)(290.75971054,259.03045931)
\curveto(290.81970725,259.110459)(290.8747072,259.19045892)(290.92471054,259.27045931)
\curveto(291.16470691,259.6104585)(291.40970666,259.94545817)(291.65971054,260.27545931)
\curveto(291.90970616,260.60545751)(292.14970592,260.94045717)(292.37971054,261.28045931)
\curveto(292.53970553,261.50045661)(292.69970537,261.7154564)(292.85971054,261.92545931)
\curveto(293.01970505,262.13545598)(293.17970489,262.35045576)(293.33971054,262.57045931)
\curveto(293.69970437,263.09045502)(294.06470401,263.60045451)(294.43471054,264.10045931)
\curveto(294.80470327,264.60045351)(295.1747029,265.110453)(295.54471054,265.63045931)
\curveto(295.68470239,265.83045228)(295.82470225,266.02545209)(295.96471054,266.21545931)
\curveto(296.11470196,266.40545171)(296.25970181,266.60045151)(296.39971054,266.80045931)
\curveto(296.60970146,267.10045101)(296.82470125,267.40045071)(297.04471054,267.70045931)
\lineto(297.70471054,268.60045931)
\lineto(297.88471054,268.87045931)
\lineto(298.09471054,269.14045931)
\lineto(298.21471054,269.32045931)
\curveto(298.26469981,269.38044873)(298.31469976,269.43544868)(298.36471054,269.48545931)
\curveto(298.43469964,269.53544858)(298.50969956,269.57044854)(298.58971054,269.59045931)
\curveto(298.60969946,269.60044851)(298.63469944,269.60044851)(298.66471054,269.59045931)
\curveto(298.70469937,269.59044852)(298.73469934,269.60044851)(298.75471054,269.62045931)
\curveto(298.8746992,269.62044849)(299.00969906,269.6154485)(299.15971054,269.60545931)
\curveto(299.30969876,269.60544851)(299.39969867,269.56044855)(299.42971054,269.47045931)
\curveto(299.44969862,269.44044867)(299.45469862,269.40544871)(299.44471054,269.36545931)
\curveto(299.43469864,269.32544879)(299.41969865,269.29544882)(299.39971054,269.27545931)
\curveto(299.35969871,269.19544892)(299.31969875,269.12544899)(299.27971054,269.06545931)
\curveto(299.23969883,269.00544911)(299.19469888,268.94544917)(299.14471054,268.88545931)
\lineto(298.57471054,268.10545931)
\curveto(298.39469968,267.85545026)(298.21469986,267.60045051)(298.03471054,267.34045931)
\moveto(291.17971054,263.44045931)
\curveto(291.12970694,263.46045465)(291.07970699,263.46545465)(291.02971054,263.45545931)
\curveto(290.97970709,263.44545467)(290.92970714,263.45045466)(290.87971054,263.47045931)
\curveto(290.7697073,263.49045462)(290.66470741,263.5104546)(290.56471054,263.53045931)
\curveto(290.4747076,263.56045455)(290.37970769,263.60045451)(290.27971054,263.65045931)
\curveto(289.94970812,263.79045432)(289.69470838,263.98545413)(289.51471054,264.23545931)
\curveto(289.33470874,264.49545362)(289.18970888,264.80545331)(289.07971054,265.16545931)
\curveto(289.04970902,265.24545287)(289.02970904,265.32545279)(289.01971054,265.40545931)
\curveto(289.00970906,265.49545262)(288.99470908,265.58045253)(288.97471054,265.66045931)
\curveto(288.96470911,265.7104524)(288.95970911,265.77545234)(288.95971054,265.85545931)
\curveto(288.94970912,265.88545223)(288.94470913,265.9154522)(288.94471054,265.94545931)
\curveto(288.94470913,265.98545213)(288.93970913,266.02045209)(288.92971054,266.05045931)
\lineto(288.92971054,266.20045931)
\curveto(288.91970915,266.25045186)(288.91470916,266.3104518)(288.91471054,266.38045931)
\curveto(288.91470916,266.46045165)(288.91970915,266.52545159)(288.92971054,266.57545931)
\lineto(288.92971054,266.74045931)
\curveto(288.94970912,266.79045132)(288.95470912,266.83545128)(288.94471054,266.87545931)
\curveto(288.94470913,266.92545119)(288.94970912,266.97045114)(288.95971054,267.01045931)
\curveto(288.9697091,267.05045106)(288.9747091,267.08545103)(288.97471054,267.11545931)
\curveto(288.9747091,267.15545096)(288.97970909,267.19545092)(288.98971054,267.23545931)
\curveto(289.01970905,267.34545077)(289.03970903,267.45545066)(289.04971054,267.56545931)
\curveto(289.069709,267.68545043)(289.10470897,267.80045031)(289.15471054,267.91045931)
\curveto(289.29470878,268.25044986)(289.45470862,268.52544959)(289.63471054,268.73545931)
\curveto(289.82470825,268.95544916)(290.09470798,269.13544898)(290.44471054,269.27545931)
\curveto(290.52470755,269.30544881)(290.60970746,269.32544879)(290.69971054,269.33545931)
\curveto(290.78970728,269.35544876)(290.88470719,269.37544874)(290.98471054,269.39545931)
\curveto(291.01470706,269.40544871)(291.069707,269.40544871)(291.14971054,269.39545931)
\curveto(291.22970684,269.39544872)(291.27970679,269.40544871)(291.29971054,269.42545931)
\curveto(291.85970621,269.43544868)(292.30970576,269.32544879)(292.64971054,269.09545931)
\curveto(292.99970507,268.86544925)(293.25970481,268.56044955)(293.42971054,268.18045931)
\curveto(293.4697046,268.09045002)(293.50470457,267.99545012)(293.53471054,267.89545931)
\curveto(293.56470451,267.79545032)(293.58970448,267.69545042)(293.60971054,267.59545931)
\curveto(293.62970444,267.56545055)(293.63470444,267.53545058)(293.62471054,267.50545931)
\curveto(293.62470445,267.47545064)(293.62970444,267.44545067)(293.63971054,267.41545931)
\curveto(293.6697044,267.30545081)(293.68970438,267.18045093)(293.69971054,267.04045931)
\curveto(293.70970436,266.9104512)(293.71970435,266.77545134)(293.72971054,266.63545931)
\lineto(293.72971054,266.47045931)
\curveto(293.73970433,266.4104517)(293.73970433,266.35545176)(293.72971054,266.30545931)
\curveto(293.71970435,266.25545186)(293.71470436,266.20545191)(293.71471054,266.15545931)
\lineto(293.71471054,266.02045931)
\curveto(293.70470437,265.98045213)(293.69970437,265.94045217)(293.69971054,265.90045931)
\curveto(293.70970436,265.86045225)(293.70470437,265.8154523)(293.68471054,265.76545931)
\curveto(293.66470441,265.65545246)(293.64470443,265.55045256)(293.62471054,265.45045931)
\curveto(293.61470446,265.35045276)(293.59470448,265.25045286)(293.56471054,265.15045931)
\curveto(293.43470464,264.79045332)(293.2697048,264.47545364)(293.06971054,264.20545931)
\curveto(292.8697052,263.93545418)(292.59470548,263.73045438)(292.24471054,263.59045931)
\curveto(292.16470591,263.56045455)(292.07970599,263.53545458)(291.98971054,263.51545931)
\lineto(291.71971054,263.45545931)
\curveto(291.6697064,263.44545467)(291.62470645,263.44045467)(291.58471054,263.44045931)
\curveto(291.54470653,263.45045466)(291.50470657,263.45045466)(291.46471054,263.44045931)
\curveto(291.36470671,263.42045469)(291.2697068,263.42045469)(291.17971054,263.44045931)
\moveto(290.33971054,264.83545931)
\curveto(290.37970769,264.76545335)(290.41970765,264.70045341)(290.45971054,264.64045931)
\curveto(290.49970757,264.59045352)(290.54970752,264.54045357)(290.60971054,264.49045931)
\lineto(290.75971054,264.37045931)
\curveto(290.81970725,264.34045377)(290.88470719,264.3154538)(290.95471054,264.29545931)
\curveto(290.99470708,264.27545384)(291.02970704,264.26545385)(291.05971054,264.26545931)
\curveto(291.09970697,264.27545384)(291.13970693,264.27045384)(291.17971054,264.25045931)
\curveto(291.20970686,264.25045386)(291.24970682,264.24545387)(291.29971054,264.23545931)
\curveto(291.34970672,264.23545388)(291.38970668,264.24045387)(291.41971054,264.25045931)
\lineto(291.64471054,264.29545931)
\curveto(291.89470618,264.37545374)(292.07970599,264.50045361)(292.19971054,264.67045931)
\curveto(292.27970579,264.77045334)(292.34970572,264.90045321)(292.40971054,265.06045931)
\curveto(292.48970558,265.24045287)(292.54970552,265.46545265)(292.58971054,265.73545931)
\curveto(292.62970544,266.0154521)(292.64470543,266.29545182)(292.63471054,266.57545931)
\curveto(292.62470545,266.86545125)(292.59470548,267.14045097)(292.54471054,267.40045931)
\curveto(292.49470558,267.66045045)(292.41970565,267.87045024)(292.31971054,268.03045931)
\curveto(292.19970587,268.23044988)(292.04970602,268.38044973)(291.86971054,268.48045931)
\curveto(291.78970628,268.53044958)(291.69970637,268.56044955)(291.59971054,268.57045931)
\curveto(291.49970657,268.59044952)(291.39470668,268.60044951)(291.28471054,268.60045931)
\curveto(291.26470681,268.59044952)(291.23970683,268.58544953)(291.20971054,268.58545931)
\curveto(291.18970688,268.59544952)(291.1697069,268.59544952)(291.14971054,268.58545931)
\curveto(291.09970697,268.57544954)(291.05470702,268.56544955)(291.01471054,268.55545931)
\curveto(290.9747071,268.55544956)(290.93470714,268.54544957)(290.89471054,268.52545931)
\curveto(290.71470736,268.44544967)(290.56470751,268.32544979)(290.44471054,268.16545931)
\curveto(290.33470774,268.00545011)(290.24470783,267.82545029)(290.17471054,267.62545931)
\curveto(290.11470796,267.43545068)(290.069708,267.2104509)(290.03971054,266.95045931)
\curveto(290.01970805,266.69045142)(290.01470806,266.42545169)(290.02471054,266.15545931)
\curveto(290.03470804,265.89545222)(290.06470801,265.64545247)(290.11471054,265.40545931)
\curveto(290.1747079,265.17545294)(290.24970782,264.98545313)(290.33971054,264.83545931)
\moveto(301.13971054,261.85045931)
\curveto(301.14969692,261.80045631)(301.15469692,261.7104564)(301.15471054,261.58045931)
\curveto(301.15469692,261.45045666)(301.14469693,261.36045675)(301.12471054,261.31045931)
\curveto(301.10469697,261.26045685)(301.09969697,261.20545691)(301.10971054,261.14545931)
\curveto(301.11969695,261.09545702)(301.11969695,261.04545707)(301.10971054,260.99545931)
\curveto(301.069697,260.85545726)(301.03969703,260.72045739)(301.01971054,260.59045931)
\curveto(301.00969706,260.46045765)(300.97969709,260.34045777)(300.92971054,260.23045931)
\curveto(300.78969728,259.88045823)(300.62469745,259.58545853)(300.43471054,259.34545931)
\curveto(300.24469783,259.115459)(299.9746981,258.93045918)(299.62471054,258.79045931)
\curveto(299.54469853,258.76045935)(299.45969861,258.74045937)(299.36971054,258.73045931)
\curveto(299.27969879,258.7104594)(299.19469888,258.69045942)(299.11471054,258.67045931)
\curveto(299.06469901,258.66045945)(299.01469906,258.65545946)(298.96471054,258.65545931)
\curveto(298.91469916,258.65545946)(298.86469921,258.65045946)(298.81471054,258.64045931)
\curveto(298.78469929,258.63045948)(298.73469934,258.63045948)(298.66471054,258.64045931)
\curveto(298.59469948,258.64045947)(298.54469953,258.64545947)(298.51471054,258.65545931)
\curveto(298.45469962,258.67545944)(298.39469968,258.68545943)(298.33471054,258.68545931)
\curveto(298.28469979,258.67545944)(298.23469984,258.68045943)(298.18471054,258.70045931)
\curveto(298.09469998,258.72045939)(298.00470007,258.74545937)(297.91471054,258.77545931)
\curveto(297.83470024,258.79545932)(297.75470032,258.82545929)(297.67471054,258.86545931)
\curveto(297.35470072,259.00545911)(297.10470097,259.20045891)(296.92471054,259.45045931)
\curveto(296.74470133,259.7104584)(296.59470148,260.0154581)(296.47471054,260.36545931)
\curveto(296.45470162,260.44545767)(296.43970163,260.53045758)(296.42971054,260.62045931)
\curveto(296.41970165,260.7104574)(296.40470167,260.79545732)(296.38471054,260.87545931)
\curveto(296.3747017,260.90545721)(296.3697017,260.93545718)(296.36971054,260.96545931)
\lineto(296.36971054,261.07045931)
\curveto(296.34970172,261.15045696)(296.33970173,261.23045688)(296.33971054,261.31045931)
\lineto(296.33971054,261.44545931)
\curveto(296.31970175,261.54545657)(296.31970175,261.64545647)(296.33971054,261.74545931)
\lineto(296.33971054,261.92545931)
\curveto(296.34970172,261.97545614)(296.35470172,262.02045609)(296.35471054,262.06045931)
\curveto(296.35470172,262.110456)(296.35970171,262.15545596)(296.36971054,262.19545931)
\curveto(296.37970169,262.23545588)(296.38470169,262.27045584)(296.38471054,262.30045931)
\curveto(296.38470169,262.34045577)(296.38970168,262.38045573)(296.39971054,262.42045931)
\lineto(296.45971054,262.75045931)
\curveto(296.47970159,262.87045524)(296.50970156,262.98045513)(296.54971054,263.08045931)
\curveto(296.68970138,263.4104547)(296.84970122,263.68545443)(297.02971054,263.90545931)
\curveto(297.21970085,264.13545398)(297.47970059,264.32045379)(297.80971054,264.46045931)
\curveto(297.88970018,264.50045361)(297.9747001,264.52545359)(298.06471054,264.53545931)
\lineto(298.36471054,264.59545931)
\lineto(298.49971054,264.59545931)
\curveto(298.54969952,264.60545351)(298.59969947,264.6104535)(298.64971054,264.61045931)
\curveto(299.21969885,264.63045348)(299.67969839,264.52545359)(300.02971054,264.29545931)
\curveto(300.38969768,264.07545404)(300.65469742,263.77545434)(300.82471054,263.39545931)
\curveto(300.8746972,263.29545482)(300.91469716,263.19545492)(300.94471054,263.09545931)
\curveto(300.9746971,262.99545512)(301.00469707,262.89045522)(301.03471054,262.78045931)
\curveto(301.04469703,262.74045537)(301.04969702,262.70545541)(301.04971054,262.67545931)
\curveto(301.04969702,262.65545546)(301.05469702,262.62545549)(301.06471054,262.58545931)
\curveto(301.08469699,262.5154556)(301.09469698,262.44045567)(301.09471054,262.36045931)
\curveto(301.09469698,262.28045583)(301.10469697,262.20045591)(301.12471054,262.12045931)
\curveto(301.12469695,262.07045604)(301.12469695,262.02545609)(301.12471054,261.98545931)
\curveto(301.12469695,261.94545617)(301.12969694,261.90045621)(301.13971054,261.85045931)
\moveto(300.02971054,261.41545931)
\curveto(300.03969803,261.46545665)(300.04469803,261.54045657)(300.04471054,261.64045931)
\curveto(300.05469802,261.74045637)(300.04969802,261.8154563)(300.02971054,261.86545931)
\curveto(300.00969806,261.92545619)(300.00469807,261.98045613)(300.01471054,262.03045931)
\curveto(300.03469804,262.09045602)(300.03469804,262.15045596)(300.01471054,262.21045931)
\curveto(300.00469807,262.24045587)(299.99969807,262.27545584)(299.99971054,262.31545931)
\curveto(299.99969807,262.35545576)(299.99469808,262.39545572)(299.98471054,262.43545931)
\curveto(299.96469811,262.5154556)(299.94469813,262.59045552)(299.92471054,262.66045931)
\curveto(299.91469816,262.74045537)(299.89969817,262.82045529)(299.87971054,262.90045931)
\curveto(299.84969822,262.96045515)(299.82469825,263.02045509)(299.80471054,263.08045931)
\curveto(299.78469829,263.14045497)(299.75469832,263.20045491)(299.71471054,263.26045931)
\curveto(299.61469846,263.43045468)(299.48469859,263.56545455)(299.32471054,263.66545931)
\curveto(299.24469883,263.7154544)(299.14969892,263.75045436)(299.03971054,263.77045931)
\curveto(298.92969914,263.79045432)(298.80469927,263.80045431)(298.66471054,263.80045931)
\curveto(298.64469943,263.79045432)(298.61969945,263.78545433)(298.58971054,263.78545931)
\curveto(298.55969951,263.79545432)(298.52969954,263.79545432)(298.49971054,263.78545931)
\lineto(298.34971054,263.72545931)
\curveto(298.29969977,263.7154544)(298.25469982,263.70045441)(298.21471054,263.68045931)
\curveto(298.02470005,263.57045454)(297.87970019,263.42545469)(297.77971054,263.24545931)
\curveto(297.68970038,263.06545505)(297.60970046,262.86045525)(297.53971054,262.63045931)
\curveto(297.49970057,262.50045561)(297.47970059,262.36545575)(297.47971054,262.22545931)
\curveto(297.47970059,262.09545602)(297.4697006,261.95045616)(297.44971054,261.79045931)
\curveto(297.43970063,261.74045637)(297.42970064,261.68045643)(297.41971054,261.61045931)
\curveto(297.41970065,261.54045657)(297.42970064,261.48045663)(297.44971054,261.43045931)
\lineto(297.44971054,261.26545931)
\lineto(297.44971054,261.08545931)
\curveto(297.45970061,261.03545708)(297.4697006,260.98045713)(297.47971054,260.92045931)
\curveto(297.48970058,260.87045724)(297.49470058,260.8154573)(297.49471054,260.75545931)
\curveto(297.50470057,260.69545742)(297.51970055,260.64045747)(297.53971054,260.59045931)
\curveto(297.58970048,260.40045771)(297.64970042,260.22545789)(297.71971054,260.06545931)
\curveto(297.78970028,259.90545821)(297.89470018,259.77545834)(298.03471054,259.67545931)
\curveto(298.16469991,259.57545854)(298.30469977,259.50545861)(298.45471054,259.46545931)
\curveto(298.48469959,259.45545866)(298.50969956,259.45045866)(298.52971054,259.45045931)
\curveto(298.55969951,259.46045865)(298.58969948,259.46045865)(298.61971054,259.45045931)
\curveto(298.63969943,259.45045866)(298.6696994,259.44545867)(298.70971054,259.43545931)
\curveto(298.74969932,259.43545868)(298.78469929,259.44045867)(298.81471054,259.45045931)
\curveto(298.85469922,259.46045865)(298.89469918,259.46545865)(298.93471054,259.46545931)
\curveto(298.9746991,259.46545865)(299.01469906,259.47545864)(299.05471054,259.49545931)
\curveto(299.29469878,259.57545854)(299.48969858,259.7104584)(299.63971054,259.90045931)
\curveto(299.75969831,260.08045803)(299.84969822,260.28545783)(299.90971054,260.51545931)
\curveto(299.92969814,260.58545753)(299.94469813,260.65545746)(299.95471054,260.72545931)
\curveto(299.96469811,260.80545731)(299.97969809,260.88545723)(299.99971054,260.96545931)
\curveto(299.99969807,261.02545709)(300.00469807,261.07045704)(300.01471054,261.10045931)
\curveto(300.01469806,261.12045699)(300.01469806,261.14545697)(300.01471054,261.17545931)
\curveto(300.01469806,261.2154569)(300.01969805,261.24545687)(300.02971054,261.26545931)
\lineto(300.02971054,261.41545931)
}
}
{
\newrgbcolor{curcolor}{0.80000001 0.80000001 0.80000001}
\pscustom[linestyle=none,fillstyle=solid,fillcolor=curcolor]
{
\newpath
\moveto(727.74397797,158.46993191)
\curveto(727.55926691,153.66536842)(727.13890732,148.87276262)(726.48441561,144.10940361)
\lineto(580.87528661,164.11628684)
\closepath
}
}
{
\newrgbcolor{curcolor}{0.90196079 0.90196079 0.90196079}
\pscustom[linestyle=none,fillstyle=solid,fillcolor=curcolor]
{
\newpath
\moveto(580.87528949,311.09347495)
\curveto(662.04854906,311.09347335)(727.85247631,245.28954352)(727.85247472,164.11628395)
\curveto(727.85247468,162.19965142)(727.81498434,160.28320225)(727.74002519,158.3680361)
\lineto(580.87528661,164.11628684)
\closepath
}
}
{
\newrgbcolor{curcolor}{0.7019608 0.7019608 0.7019608}
\pscustom[linestyle=none,fillstyle=solid,fillcolor=curcolor]
{
\newpath
\moveto(726.51285122,144.31745863)
\curveto(725.24067859,134.9595248)(723.0695318,125.7459883)(720.02952451,116.80464028)
\lineto(580.87528661,164.11628684)
\closepath
}
}
{
\newrgbcolor{curcolor}{0.60000002 0.60000002 0.60000002}
\pscustom[linestyle=none,fillstyle=solid,fillcolor=curcolor]
{
\newpath
\moveto(720.08261442,116.96107811)
\curveto(699.86987863,57.29087386)(643.87598866,17.13910041)(580.87529055,17.13909873)
\lineto(580.87528661,164.11628684)
\closepath
}
}
{
\newrgbcolor{curcolor}{0.50196081 0.50196081 0.50196081}
\pscustom[linestyle=none,fillstyle=solid,fillcolor=curcolor]
{
\newpath
\moveto(580.87529055,17.13909873)
\curveto(566.59124939,17.13909834)(552.38349347,19.22134082)(538.7001603,23.32014859)
\lineto(580.87528661,164.11628684)
\closepath
}
}
{
\newrgbcolor{curcolor}{0.40000001 0.40000001 0.40000001}
\pscustom[linestyle=none,fillstyle=solid,fillcolor=curcolor]
{
\newpath
\moveto(538.79940388,23.29045853)
\curveto(461.02344653,46.5283269)(416.81158994,128.41621221)(440.04945831,206.19216957)
\curveto(458.64794775,268.44036287)(515.90805638,311.09347622)(580.87528949,311.09347495)
\lineto(580.87528661,164.11628684)
\closepath
}
}
{
\newrgbcolor{curcolor}{0 0 0}
\pscustom[linestyle=none,fillstyle=solid,fillcolor=curcolor]
{
\newpath
\moveto(312.07312728,186.59297029)
\curveto(312.17312243,186.59295967)(312.26812233,186.58295968)(312.35812728,186.56297029)
\curveto(312.44812215,186.55295971)(312.51312209,186.52295974)(312.55312728,186.47297029)
\curveto(312.61312199,186.39295987)(312.64312196,186.28795998)(312.64312728,186.15797029)
\lineto(312.64312728,185.76797029)
\lineto(312.64312728,184.26797029)
\lineto(312.64312728,177.87797029)
\lineto(312.64312728,176.70797029)
\lineto(312.64312728,176.39297029)
\curveto(312.65312195,176.29296997)(312.63812196,176.21297005)(312.59812728,176.15297029)
\curveto(312.54812205,176.07297019)(312.47312213,176.02297024)(312.37312728,176.00297029)
\curveto(312.28312232,175.99297027)(312.17312243,175.98797028)(312.04312728,175.98797029)
\lineto(311.81812728,175.98797029)
\curveto(311.73812286,176.00797026)(311.66812293,176.02297024)(311.60812728,176.03297029)
\curveto(311.54812305,176.05297021)(311.4981231,176.09297017)(311.45812728,176.15297029)
\curveto(311.41812318,176.21297005)(311.3981232,176.28796998)(311.39812728,176.37797029)
\lineto(311.39812728,176.67797029)
\lineto(311.39812728,177.77297029)
\lineto(311.39812728,183.11297029)
\curveto(311.37812322,183.20296306)(311.36312324,183.27796299)(311.35312728,183.33797029)
\curveto(311.35312325,183.40796286)(311.32312328,183.4679628)(311.26312728,183.51797029)
\curveto(311.19312341,183.5679627)(311.1031235,183.59296267)(310.99312728,183.59297029)
\curveto(310.89312371,183.60296266)(310.78312382,183.60796266)(310.66312728,183.60797029)
\lineto(309.52312728,183.60797029)
\lineto(309.02812728,183.60797029)
\curveto(308.86812573,183.61796265)(308.75812584,183.67796259)(308.69812728,183.78797029)
\curveto(308.67812592,183.81796245)(308.66812593,183.84796242)(308.66812728,183.87797029)
\curveto(308.66812593,183.91796235)(308.66312594,183.9629623)(308.65312728,184.01297029)
\curveto(308.63312597,184.13296213)(308.63812596,184.24296202)(308.66812728,184.34297029)
\curveto(308.70812589,184.44296182)(308.76312584,184.51296175)(308.83312728,184.55297029)
\curveto(308.91312569,184.60296166)(309.03312557,184.62796164)(309.19312728,184.62797029)
\curveto(309.35312525,184.62796164)(309.48812511,184.64296162)(309.59812728,184.67297029)
\curveto(309.64812495,184.68296158)(309.7031249,184.68796158)(309.76312728,184.68797029)
\curveto(309.82312478,184.69796157)(309.88312472,184.71296155)(309.94312728,184.73297029)
\curveto(310.09312451,184.78296148)(310.23812436,184.83296143)(310.37812728,184.88297029)
\curveto(310.51812408,184.94296132)(310.65312395,185.01296125)(310.78312728,185.09297029)
\curveto(310.92312368,185.18296108)(311.04312356,185.28796098)(311.14312728,185.40797029)
\curveto(311.24312336,185.52796074)(311.33812326,185.65796061)(311.42812728,185.79797029)
\curveto(311.48812311,185.89796037)(311.53312307,186.00796026)(311.56312728,186.12797029)
\curveto(311.603123,186.24796002)(311.65312295,186.35295991)(311.71312728,186.44297029)
\curveto(311.76312284,186.50295976)(311.83312277,186.54295972)(311.92312728,186.56297029)
\curveto(311.94312266,186.57295969)(311.96812263,186.57795969)(311.99812728,186.57797029)
\curveto(312.02812257,186.57795969)(312.05312255,186.58295968)(312.07312728,186.59297029)
}
}
{
\newrgbcolor{curcolor}{0 0 0}
\pscustom[linestyle=none,fillstyle=solid,fillcolor=curcolor]
{
\newpath
\moveto(317.87273666,186.39797029)
\lineto(321.47273666,186.39797029)
\lineto(322.11773666,186.39797029)
\curveto(322.19773013,186.39795987)(322.27273005,186.39295987)(322.34273666,186.38297029)
\curveto(322.41272991,186.38295988)(322.47272985,186.37295989)(322.52273666,186.35297029)
\curveto(322.59272973,186.32295994)(322.64772968,186.26296)(322.68773666,186.17297029)
\curveto(322.70772962,186.14296012)(322.71772961,186.10296016)(322.71773666,186.05297029)
\lineto(322.71773666,185.91797029)
\curveto(322.7277296,185.80796046)(322.7227296,185.70296056)(322.70273666,185.60297029)
\curveto(322.69272963,185.50296076)(322.65772967,185.43296083)(322.59773666,185.39297029)
\curveto(322.50772982,185.32296094)(322.37272995,185.28796098)(322.19273666,185.28797029)
\curveto(322.01273031,185.29796097)(321.84773048,185.30296096)(321.69773666,185.30297029)
\lineto(319.70273666,185.30297029)
\lineto(319.20773666,185.30297029)
\lineto(319.07273666,185.30297029)
\curveto(319.03273329,185.30296096)(318.99273333,185.29796097)(318.95273666,185.28797029)
\lineto(318.74273666,185.28797029)
\curveto(318.63273369,185.25796101)(318.55273377,185.21796105)(318.50273666,185.16797029)
\curveto(318.45273387,185.12796114)(318.41773391,185.07296119)(318.39773666,185.00297029)
\curveto(318.37773395,184.94296132)(318.36273396,184.87296139)(318.35273666,184.79297029)
\curveto(318.34273398,184.71296155)(318.322734,184.62296164)(318.29273666,184.52297029)
\curveto(318.24273408,184.32296194)(318.20273412,184.11796215)(318.17273666,183.90797029)
\curveto(318.14273418,183.69796257)(318.10273422,183.49296277)(318.05273666,183.29297029)
\curveto(318.03273429,183.22296304)(318.0227343,183.15296311)(318.02273666,183.08297029)
\curveto(318.0227343,183.02296324)(318.01273431,182.95796331)(317.99273666,182.88797029)
\curveto(317.98273434,182.85796341)(317.97273435,182.81796345)(317.96273666,182.76797029)
\curveto(317.96273436,182.72796354)(317.96773436,182.68796358)(317.97773666,182.64797029)
\curveto(317.99773433,182.59796367)(318.0227343,182.55296371)(318.05273666,182.51297029)
\curveto(318.09273423,182.48296378)(318.15273417,182.47796379)(318.23273666,182.49797029)
\curveto(318.29273403,182.51796375)(318.35273397,182.54296372)(318.41273666,182.57297029)
\curveto(318.47273385,182.61296365)(318.53273379,182.64796362)(318.59273666,182.67797029)
\curveto(318.65273367,182.69796357)(318.70273362,182.71296355)(318.74273666,182.72297029)
\curveto(318.93273339,182.80296346)(319.13773319,182.85796341)(319.35773666,182.88797029)
\curveto(319.58773274,182.91796335)(319.81773251,182.92796334)(320.04773666,182.91797029)
\curveto(320.28773204,182.91796335)(320.51773181,182.89296337)(320.73773666,182.84297029)
\curveto(320.95773137,182.80296346)(321.15773117,182.74296352)(321.33773666,182.66297029)
\curveto(321.38773094,182.64296362)(321.43273089,182.62296364)(321.47273666,182.60297029)
\curveto(321.5227308,182.58296368)(321.57273075,182.55796371)(321.62273666,182.52797029)
\curveto(321.97273035,182.31796395)(322.25273007,182.08796418)(322.46273666,181.83797029)
\curveto(322.68272964,181.58796468)(322.87772945,181.262965)(323.04773666,180.86297029)
\curveto(323.09772923,180.75296551)(323.13272919,180.64296562)(323.15273666,180.53297029)
\curveto(323.17272915,180.42296584)(323.19772913,180.30796596)(323.22773666,180.18797029)
\curveto(323.23772909,180.15796611)(323.24272908,180.11296615)(323.24273666,180.05297029)
\curveto(323.26272906,179.99296627)(323.27272905,179.92296634)(323.27273666,179.84297029)
\curveto(323.27272905,179.77296649)(323.28272904,179.70796656)(323.30273666,179.64797029)
\lineto(323.30273666,179.48297029)
\curveto(323.31272901,179.43296683)(323.31772901,179.3629669)(323.31773666,179.27297029)
\curveto(323.31772901,179.18296708)(323.30772902,179.11296715)(323.28773666,179.06297029)
\curveto(323.26772906,179.00296726)(323.26272906,178.94296732)(323.27273666,178.88297029)
\curveto(323.28272904,178.83296743)(323.27772905,178.78296748)(323.25773666,178.73297029)
\curveto(323.21772911,178.57296769)(323.18272914,178.42296784)(323.15273666,178.28297029)
\curveto(323.1227292,178.14296812)(323.07772925,178.00796826)(323.01773666,177.87797029)
\curveto(322.85772947,177.50796876)(322.63772969,177.17296909)(322.35773666,176.87297029)
\curveto(322.07773025,176.57296969)(321.75773057,176.34296992)(321.39773666,176.18297029)
\curveto(321.2277311,176.10297016)(321.0277313,176.02797024)(320.79773666,175.95797029)
\curveto(320.68773164,175.91797035)(320.57273175,175.89297037)(320.45273666,175.88297029)
\curveto(320.33273199,175.87297039)(320.21273211,175.85297041)(320.09273666,175.82297029)
\curveto(320.04273228,175.80297046)(319.98773234,175.80297046)(319.92773666,175.82297029)
\curveto(319.86773246,175.83297043)(319.80773252,175.82797044)(319.74773666,175.80797029)
\curveto(319.64773268,175.78797048)(319.54773278,175.78797048)(319.44773666,175.80797029)
\lineto(319.31273666,175.80797029)
\curveto(319.26273306,175.82797044)(319.20273312,175.83797043)(319.13273666,175.83797029)
\curveto(319.07273325,175.82797044)(319.01773331,175.83297043)(318.96773666,175.85297029)
\curveto(318.9277334,175.8629704)(318.89273343,175.8679704)(318.86273666,175.86797029)
\curveto(318.83273349,175.8679704)(318.79773353,175.87297039)(318.75773666,175.88297029)
\lineto(318.48773666,175.94297029)
\curveto(318.39773393,175.9629703)(318.31273401,175.99297027)(318.23273666,176.03297029)
\curveto(317.89273443,176.17297009)(317.60273472,176.32796994)(317.36273666,176.49797029)
\curveto(317.1227352,176.67796959)(316.90273542,176.90796936)(316.70273666,177.18797029)
\curveto(316.55273577,177.41796885)(316.43773589,177.65796861)(316.35773666,177.90797029)
\curveto(316.33773599,177.95796831)(316.327736,178.00296826)(316.32773666,178.04297029)
\curveto(316.327736,178.09296817)(316.31773601,178.14296812)(316.29773666,178.19297029)
\curveto(316.27773605,178.25296801)(316.26273606,178.33296793)(316.25273666,178.43297029)
\curveto(316.25273607,178.53296773)(316.27273605,178.60796766)(316.31273666,178.65797029)
\curveto(316.36273596,178.73796753)(316.44273588,178.78296748)(316.55273666,178.79297029)
\curveto(316.66273566,178.80296746)(316.77773555,178.80796746)(316.89773666,178.80797029)
\lineto(317.06273666,178.80797029)
\curveto(317.1227352,178.80796746)(317.17773515,178.79796747)(317.22773666,178.77797029)
\curveto(317.31773501,178.75796751)(317.38773494,178.71796755)(317.43773666,178.65797029)
\curveto(317.50773482,178.5679677)(317.55273477,178.45796781)(317.57273666,178.32797029)
\curveto(317.60273472,178.20796806)(317.64773468,178.10296816)(317.70773666,178.01297029)
\curveto(317.89773443,177.67296859)(318.15773417,177.40296886)(318.48773666,177.20297029)
\curveto(318.58773374,177.14296912)(318.69273363,177.09296917)(318.80273666,177.05297029)
\curveto(318.9227334,177.02296924)(319.04273328,176.98796928)(319.16273666,176.94797029)
\curveto(319.33273299,176.89796937)(319.53773279,176.87796939)(319.77773666,176.88797029)
\curveto(320.0277323,176.90796936)(320.2277321,176.94296932)(320.37773666,176.99297029)
\curveto(320.74773158,177.11296915)(321.03773129,177.27296899)(321.24773666,177.47297029)
\curveto(321.46773086,177.68296858)(321.64773068,177.9629683)(321.78773666,178.31297029)
\curveto(321.83773049,178.41296785)(321.86773046,178.51796775)(321.87773666,178.62797029)
\curveto(321.89773043,178.73796753)(321.9227304,178.85296741)(321.95273666,178.97297029)
\lineto(321.95273666,179.07797029)
\curveto(321.96273036,179.11796715)(321.96773036,179.15796711)(321.96773666,179.19797029)
\curveto(321.97773035,179.22796704)(321.97773035,179.262967)(321.96773666,179.30297029)
\lineto(321.96773666,179.42297029)
\curveto(321.96773036,179.68296658)(321.93773039,179.92796634)(321.87773666,180.15797029)
\curveto(321.76773056,180.50796576)(321.61273071,180.80296546)(321.41273666,181.04297029)
\curveto(321.21273111,181.29296497)(320.95273137,181.48796478)(320.63273666,181.62797029)
\lineto(320.45273666,181.68797029)
\curveto(320.40273192,181.70796456)(320.34273198,181.72796454)(320.27273666,181.74797029)
\curveto(320.2227321,181.7679645)(320.16273216,181.77796449)(320.09273666,181.77797029)
\curveto(320.03273229,181.78796448)(319.96773236,181.80296446)(319.89773666,181.82297029)
\lineto(319.74773666,181.82297029)
\curveto(319.70773262,181.84296442)(319.65273267,181.85296441)(319.58273666,181.85297029)
\curveto(319.5227328,181.85296441)(319.46773286,181.84296442)(319.41773666,181.82297029)
\lineto(319.31273666,181.82297029)
\curveto(319.28273304,181.82296444)(319.24773308,181.81796445)(319.20773666,181.80797029)
\lineto(318.96773666,181.74797029)
\curveto(318.88773344,181.73796453)(318.80773352,181.71796455)(318.72773666,181.68797029)
\curveto(318.48773384,181.58796468)(318.25773407,181.45296481)(318.03773666,181.28297029)
\curveto(317.94773438,181.21296505)(317.86273446,181.13796513)(317.78273666,181.05797029)
\curveto(317.70273462,180.98796528)(317.60273472,180.93296533)(317.48273666,180.89297029)
\curveto(317.39273493,180.8629654)(317.25273507,180.85296541)(317.06273666,180.86297029)
\curveto(316.88273544,180.87296539)(316.76273556,180.89796537)(316.70273666,180.93797029)
\curveto(316.65273567,180.97796529)(316.61273571,181.03796523)(316.58273666,181.11797029)
\curveto(316.56273576,181.19796507)(316.56273576,181.28296498)(316.58273666,181.37297029)
\curveto(316.61273571,181.49296477)(316.63273569,181.61296465)(316.64273666,181.73297029)
\curveto(316.66273566,181.8629644)(316.68773564,181.98796428)(316.71773666,182.10797029)
\curveto(316.73773559,182.14796412)(316.74273558,182.18296408)(316.73273666,182.21297029)
\curveto(316.73273559,182.25296401)(316.74273558,182.29796397)(316.76273666,182.34797029)
\curveto(316.78273554,182.43796383)(316.79773553,182.52796374)(316.80773666,182.61797029)
\curveto(316.81773551,182.71796355)(316.83773549,182.81296345)(316.86773666,182.90297029)
\curveto(316.87773545,182.9629633)(316.88273544,183.02296324)(316.88273666,183.08297029)
\curveto(316.89273543,183.14296312)(316.90773542,183.20296306)(316.92773666,183.26297029)
\curveto(316.97773535,183.4629628)(317.01273531,183.6679626)(317.03273666,183.87797029)
\curveto(317.06273526,184.09796217)(317.10273522,184.30796196)(317.15273666,184.50797029)
\curveto(317.18273514,184.60796166)(317.20273512,184.70796156)(317.21273666,184.80797029)
\curveto(317.2227351,184.90796136)(317.23773509,185.00796126)(317.25773666,185.10797029)
\curveto(317.26773506,185.13796113)(317.27273505,185.17796109)(317.27273666,185.22797029)
\curveto(317.30273502,185.33796093)(317.322735,185.44296082)(317.33273666,185.54297029)
\curveto(317.35273497,185.65296061)(317.37773495,185.7629605)(317.40773666,185.87297029)
\curveto(317.4277349,185.95296031)(317.44273488,186.02296024)(317.45273666,186.08297029)
\curveto(317.46273486,186.15296011)(317.48773484,186.21296005)(317.52773666,186.26297029)
\curveto(317.54773478,186.29295997)(317.57773475,186.31295995)(317.61773666,186.32297029)
\curveto(317.65773467,186.34295992)(317.70273462,186.3629599)(317.75273666,186.38297029)
\curveto(317.81273451,186.38295988)(317.85273447,186.38795988)(317.87273666,186.39797029)
}
}
{
\newrgbcolor{curcolor}{0 0 0}
\pscustom[linestyle=none,fillstyle=solid,fillcolor=curcolor]
{
\newpath
\moveto(325.66734603,177.62297029)
\lineto(325.96734603,177.62297029)
\curveto(326.07734397,177.63296863)(326.18234387,177.63296863)(326.28234603,177.62297029)
\curveto(326.39234366,177.62296864)(326.49234356,177.61296865)(326.58234603,177.59297029)
\curveto(326.67234338,177.58296868)(326.74234331,177.55796871)(326.79234603,177.51797029)
\curveto(326.81234324,177.49796877)(326.82734322,177.4679688)(326.83734603,177.42797029)
\curveto(326.85734319,177.38796888)(326.87734317,177.34296892)(326.89734603,177.29297029)
\lineto(326.89734603,177.21797029)
\curveto(326.90734314,177.1679691)(326.90734314,177.11296915)(326.89734603,177.05297029)
\lineto(326.89734603,176.90297029)
\lineto(326.89734603,176.42297029)
\curveto(326.89734315,176.25297001)(326.85734319,176.13297013)(326.77734603,176.06297029)
\curveto(326.70734334,176.01297025)(326.61734343,175.98797028)(326.50734603,175.98797029)
\lineto(326.17734603,175.98797029)
\lineto(325.72734603,175.98797029)
\curveto(325.57734447,175.98797028)(325.46234459,176.01797025)(325.38234603,176.07797029)
\curveto(325.34234471,176.10797016)(325.31234474,176.15797011)(325.29234603,176.22797029)
\curveto(325.27234478,176.30796996)(325.25734479,176.39296987)(325.24734603,176.48297029)
\lineto(325.24734603,176.76797029)
\curveto(325.25734479,176.8679694)(325.26234479,176.95296931)(325.26234603,177.02297029)
\lineto(325.26234603,177.21797029)
\curveto(325.26234479,177.27796899)(325.27234478,177.33296893)(325.29234603,177.38297029)
\curveto(325.33234472,177.49296877)(325.40234465,177.5629687)(325.50234603,177.59297029)
\curveto(325.53234452,177.59296867)(325.58734446,177.60296866)(325.66734603,177.62297029)
}
}
{
\newrgbcolor{curcolor}{0 0 0}
\pscustom[linestyle=none,fillstyle=solid,fillcolor=curcolor]
{
\newpath
\moveto(332.12250228,186.59297029)
\curveto(332.81249765,186.60295966)(333.41249705,186.48295978)(333.92250228,186.23297029)
\curveto(334.44249602,185.98296028)(334.83749562,185.64796062)(335.10750228,185.22797029)
\curveto(335.1574953,185.14796112)(335.20249526,185.05796121)(335.24250228,184.95797029)
\curveto(335.28249518,184.8679614)(335.32749513,184.77296149)(335.37750228,184.67297029)
\curveto(335.41749504,184.57296169)(335.44749501,184.47296179)(335.46750228,184.37297029)
\curveto(335.48749497,184.27296199)(335.50749495,184.1679621)(335.52750228,184.05797029)
\curveto(335.54749491,184.00796226)(335.55249491,183.9629623)(335.54250228,183.92297029)
\curveto(335.53249493,183.88296238)(335.53749492,183.83796243)(335.55750228,183.78797029)
\curveto(335.56749489,183.73796253)(335.57249489,183.65296261)(335.57250228,183.53297029)
\curveto(335.57249489,183.42296284)(335.56749489,183.33796293)(335.55750228,183.27797029)
\curveto(335.53749492,183.21796305)(335.52749493,183.15796311)(335.52750228,183.09797029)
\curveto(335.53749492,183.03796323)(335.53249493,182.97796329)(335.51250228,182.91797029)
\curveto(335.47249499,182.77796349)(335.43749502,182.64296362)(335.40750228,182.51297029)
\curveto(335.37749508,182.38296388)(335.33749512,182.25796401)(335.28750228,182.13797029)
\curveto(335.22749523,181.99796427)(335.1574953,181.87296439)(335.07750228,181.76297029)
\curveto(335.00749545,181.65296461)(334.93249553,181.54296472)(334.85250228,181.43297029)
\lineto(334.79250228,181.37297029)
\curveto(334.78249568,181.35296491)(334.76749569,181.33296493)(334.74750228,181.31297029)
\curveto(334.62749583,181.15296511)(334.49249597,181.00796526)(334.34250228,180.87797029)
\curveto(334.19249627,180.74796552)(334.03249643,180.62296564)(333.86250228,180.50297029)
\curveto(333.55249691,180.28296598)(333.2574972,180.07796619)(332.97750228,179.88797029)
\curveto(332.74749771,179.74796652)(332.51749794,179.61296665)(332.28750228,179.48297029)
\curveto(332.06749839,179.35296691)(331.84749861,179.21796705)(331.62750228,179.07797029)
\curveto(331.37749908,178.90796736)(331.13749932,178.72796754)(330.90750228,178.53797029)
\curveto(330.68749977,178.34796792)(330.49749996,178.12296814)(330.33750228,177.86297029)
\curveto(330.29750016,177.80296846)(330.2625002,177.74296852)(330.23250228,177.68297029)
\curveto(330.20250026,177.63296863)(330.17250029,177.5679687)(330.14250228,177.48797029)
\curveto(330.12250034,177.41796885)(330.11750034,177.35796891)(330.12750228,177.30797029)
\curveto(330.14750031,177.23796903)(330.18250028,177.18296908)(330.23250228,177.14297029)
\curveto(330.28250018,177.11296915)(330.34250012,177.09296917)(330.41250228,177.08297029)
\lineto(330.65250228,177.08297029)
\lineto(331.40250228,177.08297029)
\lineto(334.20750228,177.08297029)
\lineto(334.86750228,177.08297029)
\curveto(334.9574955,177.08296918)(335.04249542,177.07796919)(335.12250228,177.06797029)
\curveto(335.20249526,177.0679692)(335.26749519,177.04796922)(335.31750228,177.00797029)
\curveto(335.36749509,176.9679693)(335.40749505,176.89296937)(335.43750228,176.78297029)
\curveto(335.47749498,176.68296958)(335.48749497,176.58296968)(335.46750228,176.48297029)
\lineto(335.46750228,176.34797029)
\curveto(335.44749501,176.27796999)(335.42749503,176.21797005)(335.40750228,176.16797029)
\curveto(335.38749507,176.11797015)(335.35249511,176.07797019)(335.30250228,176.04797029)
\curveto(335.25249521,176.00797026)(335.18249528,175.98797028)(335.09250228,175.98797029)
\lineto(334.82250228,175.98797029)
\lineto(333.92250228,175.98797029)
\lineto(330.41250228,175.98797029)
\lineto(329.34750228,175.98797029)
\curveto(329.26750119,175.98797028)(329.17750128,175.98297028)(329.07750228,175.97297029)
\curveto(328.97750148,175.97297029)(328.89250157,175.98297028)(328.82250228,176.00297029)
\curveto(328.61250185,176.07297019)(328.54750191,176.25297001)(328.62750228,176.54297029)
\curveto(328.63750182,176.58296968)(328.63750182,176.61796965)(328.62750228,176.64797029)
\curveto(328.62750183,176.68796958)(328.63750182,176.73296953)(328.65750228,176.78297029)
\curveto(328.67750178,176.8629694)(328.69750176,176.94796932)(328.71750228,177.03797029)
\curveto(328.73750172,177.12796914)(328.7625017,177.21296905)(328.79250228,177.29297029)
\curveto(328.95250151,177.78296848)(329.15250131,178.19796807)(329.39250228,178.53797029)
\curveto(329.57250089,178.78796748)(329.77750068,179.01296725)(330.00750228,179.21297029)
\curveto(330.23750022,179.42296684)(330.47749998,179.61796665)(330.72750228,179.79797029)
\curveto(330.98749947,179.97796629)(331.25249921,180.14796612)(331.52250228,180.30797029)
\curveto(331.80249866,180.47796579)(332.07249839,180.65296561)(332.33250228,180.83297029)
\curveto(332.44249802,180.91296535)(332.54749791,180.98796528)(332.64750228,181.05797029)
\curveto(332.7574977,181.12796514)(332.86749759,181.20296506)(332.97750228,181.28297029)
\curveto(333.01749744,181.31296495)(333.05249741,181.34296492)(333.08250228,181.37297029)
\curveto(333.12249734,181.41296485)(333.1624973,181.44296482)(333.20250228,181.46297029)
\curveto(333.34249712,181.57296469)(333.46749699,181.69796457)(333.57750228,181.83797029)
\curveto(333.59749686,181.8679644)(333.62249684,181.89296437)(333.65250228,181.91297029)
\curveto(333.68249678,181.94296432)(333.70749675,181.97296429)(333.72750228,182.00297029)
\curveto(333.80749665,182.10296416)(333.87249659,182.20296406)(333.92250228,182.30297029)
\curveto(333.98249648,182.40296386)(334.03749642,182.51296375)(334.08750228,182.63297029)
\curveto(334.11749634,182.70296356)(334.13749632,182.77796349)(334.14750228,182.85797029)
\lineto(334.20750228,183.09797029)
\lineto(334.20750228,183.18797029)
\curveto(334.21749624,183.21796305)(334.22249624,183.24796302)(334.22250228,183.27797029)
\curveto(334.24249622,183.34796292)(334.24749621,183.44296282)(334.23750228,183.56297029)
\curveto(334.23749622,183.69296257)(334.22749623,183.79296247)(334.20750228,183.86297029)
\curveto(334.18749627,183.94296232)(334.16749629,184.01796225)(334.14750228,184.08797029)
\curveto(334.13749632,184.1679621)(334.11749634,184.24796202)(334.08750228,184.32797029)
\curveto(333.97749648,184.5679617)(333.82749663,184.7679615)(333.63750228,184.92797029)
\curveto(333.457497,185.09796117)(333.23749722,185.23796103)(332.97750228,185.34797029)
\curveto(332.90749755,185.3679609)(332.83749762,185.38296088)(332.76750228,185.39297029)
\curveto(332.69749776,185.41296085)(332.62249784,185.43296083)(332.54250228,185.45297029)
\curveto(332.462498,185.47296079)(332.35249811,185.48296078)(332.21250228,185.48297029)
\curveto(332.08249838,185.48296078)(331.97749848,185.47296079)(331.89750228,185.45297029)
\curveto(331.83749862,185.44296082)(331.78249868,185.43796083)(331.73250228,185.43797029)
\curveto(331.68249878,185.43796083)(331.63249883,185.42796084)(331.58250228,185.40797029)
\curveto(331.48249898,185.3679609)(331.38749907,185.32796094)(331.29750228,185.28797029)
\curveto(331.21749924,185.24796102)(331.13749932,185.20296106)(331.05750228,185.15297029)
\curveto(331.02749943,185.13296113)(330.99749946,185.10796116)(330.96750228,185.07797029)
\curveto(330.94749951,185.04796122)(330.92249954,185.02296124)(330.89250228,185.00297029)
\lineto(330.81750228,184.92797029)
\curveto(330.78749967,184.90796136)(330.7624997,184.88796138)(330.74250228,184.86797029)
\lineto(330.59250228,184.65797029)
\curveto(330.55249991,184.59796167)(330.50749995,184.53296173)(330.45750228,184.46297029)
\curveto(330.39750006,184.37296189)(330.34750011,184.267962)(330.30750228,184.14797029)
\curveto(330.27750018,184.03796223)(330.24250022,183.92796234)(330.20250228,183.81797029)
\curveto(330.1625003,183.70796256)(330.13750032,183.5629627)(330.12750228,183.38297029)
\curveto(330.11750034,183.21296305)(330.08750037,183.08796318)(330.03750228,183.00797029)
\curveto(329.98750047,182.92796334)(329.91250055,182.88296338)(329.81250228,182.87297029)
\curveto(329.71250075,182.8629634)(329.60250086,182.85796341)(329.48250228,182.85797029)
\curveto(329.44250102,182.85796341)(329.40250106,182.85296341)(329.36250228,182.84297029)
\curveto(329.32250114,182.84296342)(329.28750117,182.84796342)(329.25750228,182.85797029)
\curveto(329.20750125,182.87796339)(329.1575013,182.88796338)(329.10750228,182.88797029)
\curveto(329.06750139,182.88796338)(329.02750143,182.89796337)(328.98750228,182.91797029)
\curveto(328.89750156,182.97796329)(328.85250161,183.11296315)(328.85250228,183.32297029)
\lineto(328.85250228,183.44297029)
\curveto(328.8625016,183.50296276)(328.86750159,183.5629627)(328.86750228,183.62297029)
\curveto(328.87750158,183.69296257)(328.88750157,183.75796251)(328.89750228,183.81797029)
\curveto(328.91750154,183.92796234)(328.93750152,184.02796224)(328.95750228,184.11797029)
\curveto(328.97750148,184.21796205)(329.00750145,184.31296195)(329.04750228,184.40297029)
\curveto(329.06750139,184.47296179)(329.08750137,184.53296173)(329.10750228,184.58297029)
\lineto(329.16750228,184.76297029)
\curveto(329.28750117,185.02296124)(329.44250102,185.267961)(329.63250228,185.49797029)
\curveto(329.83250063,185.72796054)(330.04750041,185.91296035)(330.27750228,186.05297029)
\curveto(330.38750007,186.13296013)(330.50249996,186.19796007)(330.62250228,186.24797029)
\lineto(331.01250228,186.39797029)
\curveto(331.12249934,186.44795982)(331.23749922,186.47795979)(331.35750228,186.48797029)
\curveto(331.47749898,186.50795976)(331.60249886,186.53295973)(331.73250228,186.56297029)
\curveto(331.80249866,186.5629597)(331.86749859,186.5629597)(331.92750228,186.56297029)
\curveto(331.98749847,186.57295969)(332.05249841,186.58295968)(332.12250228,186.59297029)
}
}
{
\newrgbcolor{curcolor}{0 0 0}
\pscustom[linestyle=none,fillstyle=solid,fillcolor=curcolor]
{
\newpath
\moveto(347.02711166,184.50797029)
\curveto(346.82710136,184.21796205)(346.61710157,183.93296233)(346.39711166,183.65297029)
\curveto(346.187102,183.37296289)(345.9821022,183.08796318)(345.78211166,182.79797029)
\curveto(345.182103,181.94796432)(344.57710361,181.10796516)(343.96711166,180.27797029)
\curveto(343.35710483,179.45796681)(342.75210543,178.62296764)(342.15211166,177.77297029)
\lineto(341.64211166,177.05297029)
\lineto(341.13211166,176.36297029)
\curveto(341.05210713,176.25297001)(340.97210721,176.13797013)(340.89211166,176.01797029)
\curveto(340.81210737,175.89797037)(340.71710747,175.80297046)(340.60711166,175.73297029)
\curveto(340.56710762,175.71297055)(340.50210768,175.69797057)(340.41211166,175.68797029)
\curveto(340.33210785,175.6679706)(340.24210794,175.65797061)(340.14211166,175.65797029)
\curveto(340.04210814,175.65797061)(339.94710824,175.6629706)(339.85711166,175.67297029)
\curveto(339.77710841,175.68297058)(339.71710847,175.70297056)(339.67711166,175.73297029)
\curveto(339.64710854,175.75297051)(339.62210856,175.78797048)(339.60211166,175.83797029)
\curveto(339.59210859,175.87797039)(339.59710859,175.92297034)(339.61711166,175.97297029)
\curveto(339.65710853,176.05297021)(339.70210848,176.12797014)(339.75211166,176.19797029)
\curveto(339.81210837,176.27796999)(339.86710832,176.35796991)(339.91711166,176.43797029)
\curveto(340.15710803,176.77796949)(340.40210778,177.11296915)(340.65211166,177.44297029)
\curveto(340.90210728,177.77296849)(341.14210704,178.10796816)(341.37211166,178.44797029)
\curveto(341.53210665,178.6679676)(341.69210649,178.88296738)(341.85211166,179.09297029)
\curveto(342.01210617,179.30296696)(342.17210601,179.51796675)(342.33211166,179.73797029)
\curveto(342.69210549,180.25796601)(343.05710513,180.7679655)(343.42711166,181.26797029)
\curveto(343.79710439,181.7679645)(344.16710402,182.27796399)(344.53711166,182.79797029)
\curveto(344.67710351,182.99796327)(344.81710337,183.19296307)(344.95711166,183.38297029)
\curveto(345.10710308,183.57296269)(345.25210293,183.7679625)(345.39211166,183.96797029)
\curveto(345.60210258,184.267962)(345.81710237,184.5679617)(346.03711166,184.86797029)
\lineto(346.69711166,185.76797029)
\lineto(346.87711166,186.03797029)
\lineto(347.08711166,186.30797029)
\lineto(347.20711166,186.48797029)
\curveto(347.25710093,186.54795972)(347.30710088,186.60295966)(347.35711166,186.65297029)
\curveto(347.42710076,186.70295956)(347.50210068,186.73795953)(347.58211166,186.75797029)
\curveto(347.60210058,186.7679595)(347.62710056,186.7679595)(347.65711166,186.75797029)
\curveto(347.69710049,186.75795951)(347.72710046,186.7679595)(347.74711166,186.78797029)
\curveto(347.86710032,186.78795948)(348.00210018,186.78295948)(348.15211166,186.77297029)
\curveto(348.30209988,186.77295949)(348.39209979,186.72795954)(348.42211166,186.63797029)
\curveto(348.44209974,186.60795966)(348.44709974,186.57295969)(348.43711166,186.53297029)
\curveto(348.42709976,186.49295977)(348.41209977,186.4629598)(348.39211166,186.44297029)
\curveto(348.35209983,186.3629599)(348.31209987,186.29295997)(348.27211166,186.23297029)
\curveto(348.23209995,186.17296009)(348.1871,186.11296015)(348.13711166,186.05297029)
\lineto(347.56711166,185.27297029)
\curveto(347.3871008,185.02296124)(347.20710098,184.7679615)(347.02711166,184.50797029)
\moveto(340.17211166,180.60797029)
\curveto(340.12210806,180.62796564)(340.07210811,180.63296563)(340.02211166,180.62297029)
\curveto(339.97210821,180.61296565)(339.92210826,180.61796565)(339.87211166,180.63797029)
\curveto(339.76210842,180.65796561)(339.65710853,180.67796559)(339.55711166,180.69797029)
\curveto(339.46710872,180.72796554)(339.37210881,180.7679655)(339.27211166,180.81797029)
\curveto(338.94210924,180.95796531)(338.6871095,181.15296511)(338.50711166,181.40297029)
\curveto(338.32710986,181.6629646)(338.18211,181.97296429)(338.07211166,182.33297029)
\curveto(338.04211014,182.41296385)(338.02211016,182.49296377)(338.01211166,182.57297029)
\curveto(338.00211018,182.6629636)(337.9871102,182.74796352)(337.96711166,182.82797029)
\curveto(337.95711023,182.87796339)(337.95211023,182.94296332)(337.95211166,183.02297029)
\curveto(337.94211024,183.05296321)(337.93711025,183.08296318)(337.93711166,183.11297029)
\curveto(337.93711025,183.15296311)(337.93211025,183.18796308)(337.92211166,183.21797029)
\lineto(337.92211166,183.36797029)
\curveto(337.91211027,183.41796285)(337.90711028,183.47796279)(337.90711166,183.54797029)
\curveto(337.90711028,183.62796264)(337.91211027,183.69296257)(337.92211166,183.74297029)
\lineto(337.92211166,183.90797029)
\curveto(337.94211024,183.95796231)(337.94711024,184.00296226)(337.93711166,184.04297029)
\curveto(337.93711025,184.09296217)(337.94211024,184.13796213)(337.95211166,184.17797029)
\curveto(337.96211022,184.21796205)(337.96711022,184.25296201)(337.96711166,184.28297029)
\curveto(337.96711022,184.32296194)(337.97211021,184.3629619)(337.98211166,184.40297029)
\curveto(338.01211017,184.51296175)(338.03211015,184.62296164)(338.04211166,184.73297029)
\curveto(338.06211012,184.85296141)(338.09711009,184.9679613)(338.14711166,185.07797029)
\curveto(338.2871099,185.41796085)(338.44710974,185.69296057)(338.62711166,185.90297029)
\curveto(338.81710937,186.12296014)(339.0871091,186.30295996)(339.43711166,186.44297029)
\curveto(339.51710867,186.47295979)(339.60210858,186.49295977)(339.69211166,186.50297029)
\curveto(339.7821084,186.52295974)(339.87710831,186.54295972)(339.97711166,186.56297029)
\curveto(340.00710818,186.57295969)(340.06210812,186.57295969)(340.14211166,186.56297029)
\curveto(340.22210796,186.5629597)(340.27210791,186.57295969)(340.29211166,186.59297029)
\curveto(340.85210733,186.60295966)(341.30210688,186.49295977)(341.64211166,186.26297029)
\curveto(341.99210619,186.03296023)(342.25210593,185.72796054)(342.42211166,185.34797029)
\curveto(342.46210572,185.25796101)(342.49710569,185.1629611)(342.52711166,185.06297029)
\curveto(342.55710563,184.9629613)(342.5821056,184.8629614)(342.60211166,184.76297029)
\curveto(342.62210556,184.73296153)(342.62710556,184.70296156)(342.61711166,184.67297029)
\curveto(342.61710557,184.64296162)(342.62210556,184.61296165)(342.63211166,184.58297029)
\curveto(342.66210552,184.47296179)(342.6821055,184.34796192)(342.69211166,184.20797029)
\curveto(342.70210548,184.07796219)(342.71210547,183.94296232)(342.72211166,183.80297029)
\lineto(342.72211166,183.63797029)
\curveto(342.73210545,183.57796269)(342.73210545,183.52296274)(342.72211166,183.47297029)
\curveto(342.71210547,183.42296284)(342.70710548,183.37296289)(342.70711166,183.32297029)
\lineto(342.70711166,183.18797029)
\curveto(342.69710549,183.14796312)(342.69210549,183.10796316)(342.69211166,183.06797029)
\curveto(342.70210548,183.02796324)(342.69710549,182.98296328)(342.67711166,182.93297029)
\curveto(342.65710553,182.82296344)(342.63710555,182.71796355)(342.61711166,182.61797029)
\curveto(342.60710558,182.51796375)(342.5871056,182.41796385)(342.55711166,182.31797029)
\curveto(342.42710576,181.95796431)(342.26210592,181.64296462)(342.06211166,181.37297029)
\curveto(341.86210632,181.10296516)(341.5871066,180.89796537)(341.23711166,180.75797029)
\curveto(341.15710703,180.72796554)(341.07210711,180.70296556)(340.98211166,180.68297029)
\lineto(340.71211166,180.62297029)
\curveto(340.66210752,180.61296565)(340.61710757,180.60796566)(340.57711166,180.60797029)
\curveto(340.53710765,180.61796565)(340.49710769,180.61796565)(340.45711166,180.60797029)
\curveto(340.35710783,180.58796568)(340.26210792,180.58796568)(340.17211166,180.60797029)
\moveto(339.33211166,182.00297029)
\curveto(339.37210881,181.93296433)(339.41210877,181.8679644)(339.45211166,181.80797029)
\curveto(339.49210869,181.75796451)(339.54210864,181.70796456)(339.60211166,181.65797029)
\lineto(339.75211166,181.53797029)
\curveto(339.81210837,181.50796476)(339.87710831,181.48296478)(339.94711166,181.46297029)
\curveto(339.9871082,181.44296482)(340.02210816,181.43296483)(340.05211166,181.43297029)
\curveto(340.09210809,181.44296482)(340.13210805,181.43796483)(340.17211166,181.41797029)
\curveto(340.20210798,181.41796485)(340.24210794,181.41296485)(340.29211166,181.40297029)
\curveto(340.34210784,181.40296486)(340.3821078,181.40796486)(340.41211166,181.41797029)
\lineto(340.63711166,181.46297029)
\curveto(340.8871073,181.54296472)(341.07210711,181.6679646)(341.19211166,181.83797029)
\curveto(341.27210691,181.93796433)(341.34210684,182.0679642)(341.40211166,182.22797029)
\curveto(341.4821067,182.40796386)(341.54210664,182.63296363)(341.58211166,182.90297029)
\curveto(341.62210656,183.18296308)(341.63710655,183.4629628)(341.62711166,183.74297029)
\curveto(341.61710657,184.03296223)(341.5871066,184.30796196)(341.53711166,184.56797029)
\curveto(341.4871067,184.82796144)(341.41210677,185.03796123)(341.31211166,185.19797029)
\curveto(341.19210699,185.39796087)(341.04210714,185.54796072)(340.86211166,185.64797029)
\curveto(340.7821074,185.69796057)(340.69210749,185.72796054)(340.59211166,185.73797029)
\curveto(340.49210769,185.75796051)(340.3871078,185.7679605)(340.27711166,185.76797029)
\curveto(340.25710793,185.75796051)(340.23210795,185.75296051)(340.20211166,185.75297029)
\curveto(340.182108,185.7629605)(340.16210802,185.7629605)(340.14211166,185.75297029)
\curveto(340.09210809,185.74296052)(340.04710814,185.73296053)(340.00711166,185.72297029)
\curveto(339.96710822,185.72296054)(339.92710826,185.71296055)(339.88711166,185.69297029)
\curveto(339.70710848,185.61296065)(339.55710863,185.49296077)(339.43711166,185.33297029)
\curveto(339.32710886,185.17296109)(339.23710895,184.99296127)(339.16711166,184.79297029)
\curveto(339.10710908,184.60296166)(339.06210912,184.37796189)(339.03211166,184.11797029)
\curveto(339.01210917,183.85796241)(339.00710918,183.59296267)(339.01711166,183.32297029)
\curveto(339.02710916,183.0629632)(339.05710913,182.81296345)(339.10711166,182.57297029)
\curveto(339.16710902,182.34296392)(339.24210894,182.15296411)(339.33211166,182.00297029)
\moveto(350.13211166,179.01797029)
\curveto(350.14209804,178.9679673)(350.14709804,178.87796739)(350.14711166,178.74797029)
\curveto(350.14709804,178.61796765)(350.13709805,178.52796774)(350.11711166,178.47797029)
\curveto(350.09709809,178.42796784)(350.09209809,178.37296789)(350.10211166,178.31297029)
\curveto(350.11209807,178.262968)(350.11209807,178.21296805)(350.10211166,178.16297029)
\curveto(350.06209812,178.02296824)(350.03209815,177.88796838)(350.01211166,177.75797029)
\curveto(350.00209818,177.62796864)(349.97209821,177.50796876)(349.92211166,177.39797029)
\curveto(349.7820984,177.04796922)(349.61709857,176.75296951)(349.42711166,176.51297029)
\curveto(349.23709895,176.28296998)(348.96709922,176.09797017)(348.61711166,175.95797029)
\curveto(348.53709965,175.92797034)(348.45209973,175.90797036)(348.36211166,175.89797029)
\curveto(348.27209991,175.87797039)(348.1871,175.85797041)(348.10711166,175.83797029)
\curveto(348.05710013,175.82797044)(348.00710018,175.82297044)(347.95711166,175.82297029)
\curveto(347.90710028,175.82297044)(347.85710033,175.81797045)(347.80711166,175.80797029)
\curveto(347.77710041,175.79797047)(347.72710046,175.79797047)(347.65711166,175.80797029)
\curveto(347.5871006,175.80797046)(347.53710065,175.81297045)(347.50711166,175.82297029)
\curveto(347.44710074,175.84297042)(347.3871008,175.85297041)(347.32711166,175.85297029)
\curveto(347.27710091,175.84297042)(347.22710096,175.84797042)(347.17711166,175.86797029)
\curveto(347.0871011,175.88797038)(346.99710119,175.91297035)(346.90711166,175.94297029)
\curveto(346.82710136,175.9629703)(346.74710144,175.99297027)(346.66711166,176.03297029)
\curveto(346.34710184,176.17297009)(346.09710209,176.3679699)(345.91711166,176.61797029)
\curveto(345.73710245,176.87796939)(345.5871026,177.18296908)(345.46711166,177.53297029)
\curveto(345.44710274,177.61296865)(345.43210275,177.69796857)(345.42211166,177.78797029)
\curveto(345.41210277,177.87796839)(345.39710279,177.9629683)(345.37711166,178.04297029)
\curveto(345.36710282,178.07296819)(345.36210282,178.10296816)(345.36211166,178.13297029)
\lineto(345.36211166,178.23797029)
\curveto(345.34210284,178.31796795)(345.33210285,178.39796787)(345.33211166,178.47797029)
\lineto(345.33211166,178.61297029)
\curveto(345.31210287,178.71296755)(345.31210287,178.81296745)(345.33211166,178.91297029)
\lineto(345.33211166,179.09297029)
\curveto(345.34210284,179.14296712)(345.34710284,179.18796708)(345.34711166,179.22797029)
\curveto(345.34710284,179.27796699)(345.35210283,179.32296694)(345.36211166,179.36297029)
\curveto(345.37210281,179.40296686)(345.37710281,179.43796683)(345.37711166,179.46797029)
\curveto(345.37710281,179.50796676)(345.3821028,179.54796672)(345.39211166,179.58797029)
\lineto(345.45211166,179.91797029)
\curveto(345.47210271,180.03796623)(345.50210268,180.14796612)(345.54211166,180.24797029)
\curveto(345.6821025,180.57796569)(345.84210234,180.85296541)(346.02211166,181.07297029)
\curveto(346.21210197,181.30296496)(346.47210171,181.48796478)(346.80211166,181.62797029)
\curveto(346.8821013,181.6679646)(346.96710122,181.69296457)(347.05711166,181.70297029)
\lineto(347.35711166,181.76297029)
\lineto(347.49211166,181.76297029)
\curveto(347.54210064,181.77296449)(347.59210059,181.77796449)(347.64211166,181.77797029)
\curveto(348.21209997,181.79796447)(348.67209951,181.69296457)(349.02211166,181.46297029)
\curveto(349.3820988,181.24296502)(349.64709854,180.94296532)(349.81711166,180.56297029)
\curveto(349.86709832,180.4629658)(349.90709828,180.3629659)(349.93711166,180.26297029)
\curveto(349.96709822,180.1629661)(349.99709819,180.05796621)(350.02711166,179.94797029)
\curveto(350.03709815,179.90796636)(350.04209814,179.87296639)(350.04211166,179.84297029)
\curveto(350.04209814,179.82296644)(350.04709814,179.79296647)(350.05711166,179.75297029)
\curveto(350.07709811,179.68296658)(350.0870981,179.60796666)(350.08711166,179.52797029)
\curveto(350.0870981,179.44796682)(350.09709809,179.3679669)(350.11711166,179.28797029)
\curveto(350.11709807,179.23796703)(350.11709807,179.19296707)(350.11711166,179.15297029)
\curveto(350.11709807,179.11296715)(350.12209806,179.0679672)(350.13211166,179.01797029)
\moveto(349.02211166,178.58297029)
\curveto(349.03209915,178.63296763)(349.03709915,178.70796756)(349.03711166,178.80797029)
\curveto(349.04709914,178.90796736)(349.04209914,178.98296728)(349.02211166,179.03297029)
\curveto(349.00209918,179.09296717)(348.99709919,179.14796712)(349.00711166,179.19797029)
\curveto(349.02709916,179.25796701)(349.02709916,179.31796695)(349.00711166,179.37797029)
\curveto(348.99709919,179.40796686)(348.99209919,179.44296682)(348.99211166,179.48297029)
\curveto(348.99209919,179.52296674)(348.9870992,179.5629667)(348.97711166,179.60297029)
\curveto(348.95709923,179.68296658)(348.93709925,179.75796651)(348.91711166,179.82797029)
\curveto(348.90709928,179.90796636)(348.89209929,179.98796628)(348.87211166,180.06797029)
\curveto(348.84209934,180.12796614)(348.81709937,180.18796608)(348.79711166,180.24797029)
\curveto(348.77709941,180.30796596)(348.74709944,180.3679659)(348.70711166,180.42797029)
\curveto(348.60709958,180.59796567)(348.47709971,180.73296553)(348.31711166,180.83297029)
\curveto(348.23709995,180.88296538)(348.14210004,180.91796535)(348.03211166,180.93797029)
\curveto(347.92210026,180.95796531)(347.79710039,180.9679653)(347.65711166,180.96797029)
\curveto(347.63710055,180.95796531)(347.61210057,180.95296531)(347.58211166,180.95297029)
\curveto(347.55210063,180.9629653)(347.52210066,180.9629653)(347.49211166,180.95297029)
\lineto(347.34211166,180.89297029)
\curveto(347.29210089,180.88296538)(347.24710094,180.8679654)(347.20711166,180.84797029)
\curveto(347.01710117,180.73796553)(346.87210131,180.59296567)(346.77211166,180.41297029)
\curveto(346.6821015,180.23296603)(346.60210158,180.02796624)(346.53211166,179.79797029)
\curveto(346.49210169,179.6679666)(346.47210171,179.53296673)(346.47211166,179.39297029)
\curveto(346.47210171,179.262967)(346.46210172,179.11796715)(346.44211166,178.95797029)
\curveto(346.43210175,178.90796736)(346.42210176,178.84796742)(346.41211166,178.77797029)
\curveto(346.41210177,178.70796756)(346.42210176,178.64796762)(346.44211166,178.59797029)
\lineto(346.44211166,178.43297029)
\lineto(346.44211166,178.25297029)
\curveto(346.45210173,178.20296806)(346.46210172,178.14796812)(346.47211166,178.08797029)
\curveto(346.4821017,178.03796823)(346.4871017,177.98296828)(346.48711166,177.92297029)
\curveto(346.49710169,177.8629684)(346.51210167,177.80796846)(346.53211166,177.75797029)
\curveto(346.5821016,177.5679687)(346.64210154,177.39296887)(346.71211166,177.23297029)
\curveto(346.7821014,177.07296919)(346.8871013,176.94296932)(347.02711166,176.84297029)
\curveto(347.15710103,176.74296952)(347.29710089,176.67296959)(347.44711166,176.63297029)
\curveto(347.47710071,176.62296964)(347.50210068,176.61796965)(347.52211166,176.61797029)
\curveto(347.55210063,176.62796964)(347.5821006,176.62796964)(347.61211166,176.61797029)
\curveto(347.63210055,176.61796965)(347.66210052,176.61296965)(347.70211166,176.60297029)
\curveto(347.74210044,176.60296966)(347.77710041,176.60796966)(347.80711166,176.61797029)
\curveto(347.84710034,176.62796964)(347.8871003,176.63296963)(347.92711166,176.63297029)
\curveto(347.96710022,176.63296963)(348.00710018,176.64296962)(348.04711166,176.66297029)
\curveto(348.2870999,176.74296952)(348.4820997,176.87796939)(348.63211166,177.06797029)
\curveto(348.75209943,177.24796902)(348.84209934,177.45296881)(348.90211166,177.68297029)
\curveto(348.92209926,177.75296851)(348.93709925,177.82296844)(348.94711166,177.89297029)
\curveto(348.95709923,177.97296829)(348.97209921,178.05296821)(348.99211166,178.13297029)
\curveto(348.99209919,178.19296807)(348.99709919,178.23796803)(349.00711166,178.26797029)
\curveto(349.00709918,178.28796798)(349.00709918,178.31296795)(349.00711166,178.34297029)
\curveto(349.00709918,178.38296788)(349.01209917,178.41296785)(349.02211166,178.43297029)
\lineto(349.02211166,178.58297029)
}
}
{
\newrgbcolor{curcolor}{0 0 0}
\pscustom[linestyle=none,fillstyle=solid,fillcolor=curcolor]
{
\newpath
\moveto(289.34470321,95.67921297)
\curveto(289.44469836,95.67920235)(289.53969826,95.66920236)(289.62970321,95.64921297)
\curveto(289.71969808,95.63920239)(289.78469802,95.60920242)(289.82470321,95.55921297)
\curveto(289.88469792,95.47920255)(289.91469789,95.37420265)(289.91470321,95.24421297)
\lineto(289.91470321,94.85421297)
\lineto(289.91470321,93.35421297)
\lineto(289.91470321,86.96421297)
\lineto(289.91470321,85.79421297)
\lineto(289.91470321,85.47921297)
\curveto(289.92469788,85.37921265)(289.90969789,85.29921273)(289.86970321,85.23921297)
\curveto(289.81969798,85.15921287)(289.74469806,85.10921292)(289.64470321,85.08921297)
\curveto(289.55469825,85.07921295)(289.44469836,85.07421295)(289.31470321,85.07421297)
\lineto(289.08970321,85.07421297)
\curveto(289.00969879,85.09421293)(288.93969886,85.10921292)(288.87970321,85.11921297)
\curveto(288.81969898,85.13921289)(288.76969903,85.17921285)(288.72970321,85.23921297)
\curveto(288.68969911,85.29921273)(288.66969913,85.37421265)(288.66970321,85.46421297)
\lineto(288.66970321,85.76421297)
\lineto(288.66970321,86.85921297)
\lineto(288.66970321,92.19921297)
\curveto(288.64969915,92.28920574)(288.63469917,92.36420566)(288.62470321,92.42421297)
\curveto(288.62469918,92.49420553)(288.59469921,92.55420547)(288.53470321,92.60421297)
\curveto(288.46469934,92.65420537)(288.37469943,92.67920535)(288.26470321,92.67921297)
\curveto(288.16469964,92.68920534)(288.05469975,92.69420533)(287.93470321,92.69421297)
\lineto(286.79470321,92.69421297)
\lineto(286.29970321,92.69421297)
\curveto(286.13970166,92.70420532)(286.02970177,92.76420526)(285.96970321,92.87421297)
\curveto(285.94970185,92.90420512)(285.93970186,92.93420509)(285.93970321,92.96421297)
\curveto(285.93970186,93.00420502)(285.93470187,93.04920498)(285.92470321,93.09921297)
\curveto(285.9047019,93.21920481)(285.90970189,93.3292047)(285.93970321,93.42921297)
\curveto(285.97970182,93.5292045)(286.03470177,93.59920443)(286.10470321,93.63921297)
\curveto(286.18470162,93.68920434)(286.3047015,93.71420431)(286.46470321,93.71421297)
\curveto(286.62470118,93.71420431)(286.75970104,93.7292043)(286.86970321,93.75921297)
\curveto(286.91970088,93.76920426)(286.97470083,93.77420425)(287.03470321,93.77421297)
\curveto(287.09470071,93.78420424)(287.15470065,93.79920423)(287.21470321,93.81921297)
\curveto(287.36470044,93.86920416)(287.50970029,93.91920411)(287.64970321,93.96921297)
\curveto(287.78970001,94.029204)(287.92469988,94.09920393)(288.05470321,94.17921297)
\curveto(288.19469961,94.26920376)(288.31469949,94.37420365)(288.41470321,94.49421297)
\curveto(288.51469929,94.61420341)(288.60969919,94.74420328)(288.69970321,94.88421297)
\curveto(288.75969904,94.98420304)(288.804699,95.09420293)(288.83470321,95.21421297)
\curveto(288.87469893,95.33420269)(288.92469888,95.43920259)(288.98470321,95.52921297)
\curveto(289.03469877,95.58920244)(289.1046987,95.6292024)(289.19470321,95.64921297)
\curveto(289.21469859,95.65920237)(289.23969856,95.66420236)(289.26970321,95.66421297)
\curveto(289.2996985,95.66420236)(289.32469848,95.66920236)(289.34470321,95.67921297)
}
}
{
\newrgbcolor{curcolor}{0 0 0}
\pscustom[linestyle=none,fillstyle=solid,fillcolor=curcolor]
{
\newpath
\moveto(296.88431259,95.67921297)
\curveto(297.57430795,95.68920234)(298.17430735,95.56920246)(298.68431259,95.31921297)
\curveto(299.20430632,95.06920296)(299.59930593,94.73420329)(299.86931259,94.31421297)
\curveto(299.91930561,94.23420379)(299.96430556,94.14420388)(300.00431259,94.04421297)
\curveto(300.04430548,93.95420407)(300.08930544,93.85920417)(300.13931259,93.75921297)
\curveto(300.17930535,93.65920437)(300.20930532,93.55920447)(300.22931259,93.45921297)
\curveto(300.24930528,93.35920467)(300.26930526,93.25420477)(300.28931259,93.14421297)
\curveto(300.30930522,93.09420493)(300.31430521,93.04920498)(300.30431259,93.00921297)
\curveto(300.29430523,92.96920506)(300.29930523,92.9242051)(300.31931259,92.87421297)
\curveto(300.3293052,92.8242052)(300.33430519,92.73920529)(300.33431259,92.61921297)
\curveto(300.33430519,92.50920552)(300.3293052,92.4242056)(300.31931259,92.36421297)
\curveto(300.29930523,92.30420572)(300.28930524,92.24420578)(300.28931259,92.18421297)
\curveto(300.29930523,92.1242059)(300.29430523,92.06420596)(300.27431259,92.00421297)
\curveto(300.23430529,91.86420616)(300.19930533,91.7292063)(300.16931259,91.59921297)
\curveto(300.13930539,91.46920656)(300.09930543,91.34420668)(300.04931259,91.22421297)
\curveto(299.98930554,91.08420694)(299.91930561,90.95920707)(299.83931259,90.84921297)
\curveto(299.76930576,90.73920729)(299.69430583,90.6292074)(299.61431259,90.51921297)
\lineto(299.55431259,90.45921297)
\curveto(299.54430598,90.43920759)(299.529306,90.41920761)(299.50931259,90.39921297)
\curveto(299.38930614,90.23920779)(299.25430627,90.09420793)(299.10431259,89.96421297)
\curveto(298.95430657,89.83420819)(298.79430673,89.70920832)(298.62431259,89.58921297)
\curveto(298.31430721,89.36920866)(298.01930751,89.16420886)(297.73931259,88.97421297)
\curveto(297.50930802,88.83420919)(297.27930825,88.69920933)(297.04931259,88.56921297)
\curveto(296.8293087,88.43920959)(296.60930892,88.30420972)(296.38931259,88.16421297)
\curveto(296.13930939,87.99421003)(295.89930963,87.81421021)(295.66931259,87.62421297)
\curveto(295.44931008,87.43421059)(295.25931027,87.20921082)(295.09931259,86.94921297)
\curveto(295.05931047,86.88921114)(295.0243105,86.8292112)(294.99431259,86.76921297)
\curveto(294.96431056,86.71921131)(294.93431059,86.65421137)(294.90431259,86.57421297)
\curveto(294.88431064,86.50421152)(294.87931065,86.44421158)(294.88931259,86.39421297)
\curveto(294.90931062,86.3242117)(294.94431058,86.26921176)(294.99431259,86.22921297)
\curveto(295.04431048,86.19921183)(295.10431042,86.17921185)(295.17431259,86.16921297)
\lineto(295.41431259,86.16921297)
\lineto(296.16431259,86.16921297)
\lineto(298.96931259,86.16921297)
\lineto(299.62931259,86.16921297)
\curveto(299.71930581,86.16921186)(299.80430572,86.16421186)(299.88431259,86.15421297)
\curveto(299.96430556,86.15421187)(300.0293055,86.13421189)(300.07931259,86.09421297)
\curveto(300.1293054,86.05421197)(300.16930536,85.97921205)(300.19931259,85.86921297)
\curveto(300.23930529,85.76921226)(300.24930528,85.66921236)(300.22931259,85.56921297)
\lineto(300.22931259,85.43421297)
\curveto(300.20930532,85.36421266)(300.18930534,85.30421272)(300.16931259,85.25421297)
\curveto(300.14930538,85.20421282)(300.11430541,85.16421286)(300.06431259,85.13421297)
\curveto(300.01430551,85.09421293)(299.94430558,85.07421295)(299.85431259,85.07421297)
\lineto(299.58431259,85.07421297)
\lineto(298.68431259,85.07421297)
\lineto(295.17431259,85.07421297)
\lineto(294.10931259,85.07421297)
\curveto(294.0293115,85.07421295)(293.93931159,85.06921296)(293.83931259,85.05921297)
\curveto(293.73931179,85.05921297)(293.65431187,85.06921296)(293.58431259,85.08921297)
\curveto(293.37431215,85.15921287)(293.30931222,85.33921269)(293.38931259,85.62921297)
\curveto(293.39931213,85.66921236)(293.39931213,85.70421232)(293.38931259,85.73421297)
\curveto(293.38931214,85.77421225)(293.39931213,85.81921221)(293.41931259,85.86921297)
\curveto(293.43931209,85.94921208)(293.45931207,86.03421199)(293.47931259,86.12421297)
\curveto(293.49931203,86.21421181)(293.524312,86.29921173)(293.55431259,86.37921297)
\curveto(293.71431181,86.86921116)(293.91431161,87.28421074)(294.15431259,87.62421297)
\curveto(294.33431119,87.87421015)(294.53931099,88.09920993)(294.76931259,88.29921297)
\curveto(294.99931053,88.50920952)(295.23931029,88.70420932)(295.48931259,88.88421297)
\curveto(295.74930978,89.06420896)(296.01430951,89.23420879)(296.28431259,89.39421297)
\curveto(296.56430896,89.56420846)(296.83430869,89.73920829)(297.09431259,89.91921297)
\curveto(297.20430832,89.99920803)(297.30930822,90.07420795)(297.40931259,90.14421297)
\curveto(297.51930801,90.21420781)(297.6293079,90.28920774)(297.73931259,90.36921297)
\curveto(297.77930775,90.39920763)(297.81430771,90.4292076)(297.84431259,90.45921297)
\curveto(297.88430764,90.49920753)(297.9243076,90.5292075)(297.96431259,90.54921297)
\curveto(298.10430742,90.65920737)(298.2293073,90.78420724)(298.33931259,90.92421297)
\curveto(298.35930717,90.95420707)(298.38430714,90.97920705)(298.41431259,90.99921297)
\curveto(298.44430708,91.029207)(298.46930706,91.05920697)(298.48931259,91.08921297)
\curveto(298.56930696,91.18920684)(298.63430689,91.28920674)(298.68431259,91.38921297)
\curveto(298.74430678,91.48920654)(298.79930673,91.59920643)(298.84931259,91.71921297)
\curveto(298.87930665,91.78920624)(298.89930663,91.86420616)(298.90931259,91.94421297)
\lineto(298.96931259,92.18421297)
\lineto(298.96931259,92.27421297)
\curveto(298.97930655,92.30420572)(298.98430654,92.33420569)(298.98431259,92.36421297)
\curveto(299.00430652,92.43420559)(299.00930652,92.5292055)(298.99931259,92.64921297)
\curveto(298.99930653,92.77920525)(298.98930654,92.87920515)(298.96931259,92.94921297)
\curveto(298.94930658,93.029205)(298.9293066,93.10420492)(298.90931259,93.17421297)
\curveto(298.89930663,93.25420477)(298.87930665,93.33420469)(298.84931259,93.41421297)
\curveto(298.73930679,93.65420437)(298.58930694,93.85420417)(298.39931259,94.01421297)
\curveto(298.21930731,94.18420384)(297.99930753,94.3242037)(297.73931259,94.43421297)
\curveto(297.66930786,94.45420357)(297.59930793,94.46920356)(297.52931259,94.47921297)
\curveto(297.45930807,94.49920353)(297.38430814,94.51920351)(297.30431259,94.53921297)
\curveto(297.2243083,94.55920347)(297.11430841,94.56920346)(296.97431259,94.56921297)
\curveto(296.84430868,94.56920346)(296.73930879,94.55920347)(296.65931259,94.53921297)
\curveto(296.59930893,94.5292035)(296.54430898,94.5242035)(296.49431259,94.52421297)
\curveto(296.44430908,94.5242035)(296.39430913,94.51420351)(296.34431259,94.49421297)
\curveto(296.24430928,94.45420357)(296.14930938,94.41420361)(296.05931259,94.37421297)
\curveto(295.97930955,94.33420369)(295.89930963,94.28920374)(295.81931259,94.23921297)
\curveto(295.78930974,94.21920381)(295.75930977,94.19420383)(295.72931259,94.16421297)
\curveto(295.70930982,94.13420389)(295.68430984,94.10920392)(295.65431259,94.08921297)
\lineto(295.57931259,94.01421297)
\curveto(295.54930998,93.99420403)(295.52431,93.97420405)(295.50431259,93.95421297)
\lineto(295.35431259,93.74421297)
\curveto(295.31431021,93.68420434)(295.26931026,93.61920441)(295.21931259,93.54921297)
\curveto(295.15931037,93.45920457)(295.10931042,93.35420467)(295.06931259,93.23421297)
\curveto(295.03931049,93.1242049)(295.00431052,93.01420501)(294.96431259,92.90421297)
\curveto(294.9243106,92.79420523)(294.89931063,92.64920538)(294.88931259,92.46921297)
\curveto(294.87931065,92.29920573)(294.84931068,92.17420585)(294.79931259,92.09421297)
\curveto(294.74931078,92.01420601)(294.67431085,91.96920606)(294.57431259,91.95921297)
\curveto(294.47431105,91.94920608)(294.36431116,91.94420608)(294.24431259,91.94421297)
\curveto(294.20431132,91.94420608)(294.16431136,91.93920609)(294.12431259,91.92921297)
\curveto(294.08431144,91.9292061)(294.04931148,91.93420609)(294.01931259,91.94421297)
\curveto(293.96931156,91.96420606)(293.91931161,91.97420605)(293.86931259,91.97421297)
\curveto(293.8293117,91.97420605)(293.78931174,91.98420604)(293.74931259,92.00421297)
\curveto(293.65931187,92.06420596)(293.61431191,92.19920583)(293.61431259,92.40921297)
\lineto(293.61431259,92.52921297)
\curveto(293.6243119,92.58920544)(293.6293119,92.64920538)(293.62931259,92.70921297)
\curveto(293.63931189,92.77920525)(293.64931188,92.84420518)(293.65931259,92.90421297)
\curveto(293.67931185,93.01420501)(293.69931183,93.11420491)(293.71931259,93.20421297)
\curveto(293.73931179,93.30420472)(293.76931176,93.39920463)(293.80931259,93.48921297)
\curveto(293.8293117,93.55920447)(293.84931168,93.61920441)(293.86931259,93.66921297)
\lineto(293.92931259,93.84921297)
\curveto(294.04931148,94.10920392)(294.20431132,94.35420367)(294.39431259,94.58421297)
\curveto(294.59431093,94.81420321)(294.80931072,94.99920303)(295.03931259,95.13921297)
\curveto(295.14931038,95.21920281)(295.26431026,95.28420274)(295.38431259,95.33421297)
\lineto(295.77431259,95.48421297)
\curveto(295.88430964,95.53420249)(295.99930953,95.56420246)(296.11931259,95.57421297)
\curveto(296.23930929,95.59420243)(296.36430916,95.61920241)(296.49431259,95.64921297)
\curveto(296.56430896,95.64920238)(296.6293089,95.64920238)(296.68931259,95.64921297)
\curveto(296.74930878,95.65920237)(296.81430871,95.66920236)(296.88431259,95.67921297)
}
}
{
\newrgbcolor{curcolor}{0 0 0}
\pscustom[linestyle=none,fillstyle=solid,fillcolor=curcolor]
{
\newpath
\moveto(302.93892196,86.70921297)
\lineto(303.23892196,86.70921297)
\curveto(303.3489199,86.71921131)(303.4539198,86.71921131)(303.55392196,86.70921297)
\curveto(303.66391959,86.70921132)(303.76391949,86.69921133)(303.85392196,86.67921297)
\curveto(303.94391931,86.66921136)(304.01391924,86.64421138)(304.06392196,86.60421297)
\curveto(304.08391917,86.58421144)(304.09891915,86.55421147)(304.10892196,86.51421297)
\curveto(304.12891912,86.47421155)(304.1489191,86.4292116)(304.16892196,86.37921297)
\lineto(304.16892196,86.30421297)
\curveto(304.17891907,86.25421177)(304.17891907,86.19921183)(304.16892196,86.13921297)
\lineto(304.16892196,85.98921297)
\lineto(304.16892196,85.50921297)
\curveto(304.16891908,85.33921269)(304.12891912,85.21921281)(304.04892196,85.14921297)
\curveto(303.97891927,85.09921293)(303.88891936,85.07421295)(303.77892196,85.07421297)
\lineto(303.44892196,85.07421297)
\lineto(302.99892196,85.07421297)
\curveto(302.8489204,85.07421295)(302.73392052,85.10421292)(302.65392196,85.16421297)
\curveto(302.61392064,85.19421283)(302.58392067,85.24421278)(302.56392196,85.31421297)
\curveto(302.54392071,85.39421263)(302.52892072,85.47921255)(302.51892196,85.56921297)
\lineto(302.51892196,85.85421297)
\curveto(302.52892072,85.95421207)(302.53392072,86.03921199)(302.53392196,86.10921297)
\lineto(302.53392196,86.30421297)
\curveto(302.53392072,86.36421166)(302.54392071,86.41921161)(302.56392196,86.46921297)
\curveto(302.60392065,86.57921145)(302.67392058,86.64921138)(302.77392196,86.67921297)
\curveto(302.80392045,86.67921135)(302.85892039,86.68921134)(302.93892196,86.70921297)
}
}
{
\newrgbcolor{curcolor}{0 0 0}
\pscustom[linestyle=none,fillstyle=solid,fillcolor=curcolor]
{
\newpath
\moveto(310.20407821,95.67921297)
\curveto(310.30407336,95.67920235)(310.39907326,95.66920236)(310.48907821,95.64921297)
\curveto(310.57907308,95.63920239)(310.64407302,95.60920242)(310.68407821,95.55921297)
\curveto(310.74407292,95.47920255)(310.77407289,95.37420265)(310.77407821,95.24421297)
\lineto(310.77407821,94.85421297)
\lineto(310.77407821,93.35421297)
\lineto(310.77407821,86.96421297)
\lineto(310.77407821,85.79421297)
\lineto(310.77407821,85.47921297)
\curveto(310.78407288,85.37921265)(310.76907289,85.29921273)(310.72907821,85.23921297)
\curveto(310.67907298,85.15921287)(310.60407306,85.10921292)(310.50407821,85.08921297)
\curveto(310.41407325,85.07921295)(310.30407336,85.07421295)(310.17407821,85.07421297)
\lineto(309.94907821,85.07421297)
\curveto(309.86907379,85.09421293)(309.79907386,85.10921292)(309.73907821,85.11921297)
\curveto(309.67907398,85.13921289)(309.62907403,85.17921285)(309.58907821,85.23921297)
\curveto(309.54907411,85.29921273)(309.52907413,85.37421265)(309.52907821,85.46421297)
\lineto(309.52907821,85.76421297)
\lineto(309.52907821,86.85921297)
\lineto(309.52907821,92.19921297)
\curveto(309.50907415,92.28920574)(309.49407417,92.36420566)(309.48407821,92.42421297)
\curveto(309.48407418,92.49420553)(309.45407421,92.55420547)(309.39407821,92.60421297)
\curveto(309.32407434,92.65420537)(309.23407443,92.67920535)(309.12407821,92.67921297)
\curveto(309.02407464,92.68920534)(308.91407475,92.69420533)(308.79407821,92.69421297)
\lineto(307.65407821,92.69421297)
\lineto(307.15907821,92.69421297)
\curveto(306.99907666,92.70420532)(306.88907677,92.76420526)(306.82907821,92.87421297)
\curveto(306.80907685,92.90420512)(306.79907686,92.93420509)(306.79907821,92.96421297)
\curveto(306.79907686,93.00420502)(306.79407687,93.04920498)(306.78407821,93.09921297)
\curveto(306.7640769,93.21920481)(306.76907689,93.3292047)(306.79907821,93.42921297)
\curveto(306.83907682,93.5292045)(306.89407677,93.59920443)(306.96407821,93.63921297)
\curveto(307.04407662,93.68920434)(307.1640765,93.71420431)(307.32407821,93.71421297)
\curveto(307.48407618,93.71420431)(307.61907604,93.7292043)(307.72907821,93.75921297)
\curveto(307.77907588,93.76920426)(307.83407583,93.77420425)(307.89407821,93.77421297)
\curveto(307.95407571,93.78420424)(308.01407565,93.79920423)(308.07407821,93.81921297)
\curveto(308.22407544,93.86920416)(308.36907529,93.91920411)(308.50907821,93.96921297)
\curveto(308.64907501,94.029204)(308.78407488,94.09920393)(308.91407821,94.17921297)
\curveto(309.05407461,94.26920376)(309.17407449,94.37420365)(309.27407821,94.49421297)
\curveto(309.37407429,94.61420341)(309.46907419,94.74420328)(309.55907821,94.88421297)
\curveto(309.61907404,94.98420304)(309.664074,95.09420293)(309.69407821,95.21421297)
\curveto(309.73407393,95.33420269)(309.78407388,95.43920259)(309.84407821,95.52921297)
\curveto(309.89407377,95.58920244)(309.9640737,95.6292024)(310.05407821,95.64921297)
\curveto(310.07407359,95.65920237)(310.09907356,95.66420236)(310.12907821,95.66421297)
\curveto(310.1590735,95.66420236)(310.18407348,95.66920236)(310.20407821,95.67921297)
}
}
{
\newrgbcolor{curcolor}{0 0 0}
\pscustom[linestyle=none,fillstyle=solid,fillcolor=curcolor]
{
\newpath
\moveto(324.29868759,93.59421297)
\curveto(324.09867729,93.30420472)(323.8886775,93.01920501)(323.66868759,92.73921297)
\curveto(323.45867793,92.45920557)(323.25367813,92.17420585)(323.05368759,91.88421297)
\curveto(322.45367893,91.03420699)(321.84867954,90.19420783)(321.23868759,89.36421297)
\curveto(320.62868076,88.54420948)(320.02368136,87.70921032)(319.42368759,86.85921297)
\lineto(318.91368759,86.13921297)
\lineto(318.40368759,85.44921297)
\curveto(318.32368306,85.33921269)(318.24368314,85.2242128)(318.16368759,85.10421297)
\curveto(318.0836833,84.98421304)(317.9886834,84.88921314)(317.87868759,84.81921297)
\curveto(317.83868355,84.79921323)(317.77368361,84.78421324)(317.68368759,84.77421297)
\curveto(317.60368378,84.75421327)(317.51368387,84.74421328)(317.41368759,84.74421297)
\curveto(317.31368407,84.74421328)(317.21868417,84.74921328)(317.12868759,84.75921297)
\curveto(317.04868434,84.76921326)(316.9886844,84.78921324)(316.94868759,84.81921297)
\curveto(316.91868447,84.83921319)(316.89368449,84.87421315)(316.87368759,84.92421297)
\curveto(316.86368452,84.96421306)(316.86868452,85.00921302)(316.88868759,85.05921297)
\curveto(316.92868446,85.13921289)(316.97368441,85.21421281)(317.02368759,85.28421297)
\curveto(317.0836843,85.36421266)(317.13868425,85.44421258)(317.18868759,85.52421297)
\curveto(317.42868396,85.86421216)(317.67368371,86.19921183)(317.92368759,86.52921297)
\curveto(318.17368321,86.85921117)(318.41368297,87.19421083)(318.64368759,87.53421297)
\curveto(318.80368258,87.75421027)(318.96368242,87.96921006)(319.12368759,88.17921297)
\curveto(319.2836821,88.38920964)(319.44368194,88.60420942)(319.60368759,88.82421297)
\curveto(319.96368142,89.34420868)(320.32868106,89.85420817)(320.69868759,90.35421297)
\curveto(321.06868032,90.85420717)(321.43867995,91.36420666)(321.80868759,91.88421297)
\curveto(321.94867944,92.08420594)(322.0886793,92.27920575)(322.22868759,92.46921297)
\curveto(322.37867901,92.65920537)(322.52367886,92.85420517)(322.66368759,93.05421297)
\curveto(322.87367851,93.35420467)(323.0886783,93.65420437)(323.30868759,93.95421297)
\lineto(323.96868759,94.85421297)
\lineto(324.14868759,95.12421297)
\lineto(324.35868759,95.39421297)
\lineto(324.47868759,95.57421297)
\curveto(324.52867686,95.63420239)(324.57867681,95.68920234)(324.62868759,95.73921297)
\curveto(324.69867669,95.78920224)(324.77367661,95.8242022)(324.85368759,95.84421297)
\curveto(324.87367651,95.85420217)(324.89867649,95.85420217)(324.92868759,95.84421297)
\curveto(324.96867642,95.84420218)(324.99867639,95.85420217)(325.01868759,95.87421297)
\curveto(325.13867625,95.87420215)(325.27367611,95.86920216)(325.42368759,95.85921297)
\curveto(325.57367581,95.85920217)(325.66367572,95.81420221)(325.69368759,95.72421297)
\curveto(325.71367567,95.69420233)(325.71867567,95.65920237)(325.70868759,95.61921297)
\curveto(325.69867569,95.57920245)(325.6836757,95.54920248)(325.66368759,95.52921297)
\curveto(325.62367576,95.44920258)(325.5836758,95.37920265)(325.54368759,95.31921297)
\curveto(325.50367588,95.25920277)(325.45867593,95.19920283)(325.40868759,95.13921297)
\lineto(324.83868759,94.35921297)
\curveto(324.65867673,94.10920392)(324.47867691,93.85420417)(324.29868759,93.59421297)
\moveto(317.44368759,89.69421297)
\curveto(317.39368399,89.71420831)(317.34368404,89.71920831)(317.29368759,89.70921297)
\curveto(317.24368414,89.69920833)(317.19368419,89.70420832)(317.14368759,89.72421297)
\curveto(317.03368435,89.74420828)(316.92868446,89.76420826)(316.82868759,89.78421297)
\curveto(316.73868465,89.81420821)(316.64368474,89.85420817)(316.54368759,89.90421297)
\curveto(316.21368517,90.04420798)(315.95868543,90.23920779)(315.77868759,90.48921297)
\curveto(315.59868579,90.74920728)(315.45368593,91.05920697)(315.34368759,91.41921297)
\curveto(315.31368607,91.49920653)(315.29368609,91.57920645)(315.28368759,91.65921297)
\curveto(315.27368611,91.74920628)(315.25868613,91.83420619)(315.23868759,91.91421297)
\curveto(315.22868616,91.96420606)(315.22368616,92.029206)(315.22368759,92.10921297)
\curveto(315.21368617,92.13920589)(315.20868618,92.16920586)(315.20868759,92.19921297)
\curveto(315.20868618,92.23920579)(315.20368618,92.27420575)(315.19368759,92.30421297)
\lineto(315.19368759,92.45421297)
\curveto(315.1836862,92.50420552)(315.17868621,92.56420546)(315.17868759,92.63421297)
\curveto(315.17868621,92.71420531)(315.1836862,92.77920525)(315.19368759,92.82921297)
\lineto(315.19368759,92.99421297)
\curveto(315.21368617,93.04420498)(315.21868617,93.08920494)(315.20868759,93.12921297)
\curveto(315.20868618,93.17920485)(315.21368617,93.2242048)(315.22368759,93.26421297)
\curveto(315.23368615,93.30420472)(315.23868615,93.33920469)(315.23868759,93.36921297)
\curveto(315.23868615,93.40920462)(315.24368614,93.44920458)(315.25368759,93.48921297)
\curveto(315.2836861,93.59920443)(315.30368608,93.70920432)(315.31368759,93.81921297)
\curveto(315.33368605,93.93920409)(315.36868602,94.05420397)(315.41868759,94.16421297)
\curveto(315.55868583,94.50420352)(315.71868567,94.77920325)(315.89868759,94.98921297)
\curveto(316.0886853,95.20920282)(316.35868503,95.38920264)(316.70868759,95.52921297)
\curveto(316.7886846,95.55920247)(316.87368451,95.57920245)(316.96368759,95.58921297)
\curveto(317.05368433,95.60920242)(317.14868424,95.6292024)(317.24868759,95.64921297)
\curveto(317.27868411,95.65920237)(317.33368405,95.65920237)(317.41368759,95.64921297)
\curveto(317.49368389,95.64920238)(317.54368384,95.65920237)(317.56368759,95.67921297)
\curveto(318.12368326,95.68920234)(318.57368281,95.57920245)(318.91368759,95.34921297)
\curveto(319.26368212,95.11920291)(319.52368186,94.81420321)(319.69368759,94.43421297)
\curveto(319.73368165,94.34420368)(319.76868162,94.24920378)(319.79868759,94.14921297)
\curveto(319.82868156,94.04920398)(319.85368153,93.94920408)(319.87368759,93.84921297)
\curveto(319.89368149,93.81920421)(319.89868149,93.78920424)(319.88868759,93.75921297)
\curveto(319.8886815,93.7292043)(319.89368149,93.69920433)(319.90368759,93.66921297)
\curveto(319.93368145,93.55920447)(319.95368143,93.43420459)(319.96368759,93.29421297)
\curveto(319.97368141,93.16420486)(319.9836814,93.029205)(319.99368759,92.88921297)
\lineto(319.99368759,92.72421297)
\curveto(320.00368138,92.66420536)(320.00368138,92.60920542)(319.99368759,92.55921297)
\curveto(319.9836814,92.50920552)(319.97868141,92.45920557)(319.97868759,92.40921297)
\lineto(319.97868759,92.27421297)
\curveto(319.96868142,92.23420579)(319.96368142,92.19420583)(319.96368759,92.15421297)
\curveto(319.97368141,92.11420591)(319.96868142,92.06920596)(319.94868759,92.01921297)
\curveto(319.92868146,91.90920612)(319.90868148,91.80420622)(319.88868759,91.70421297)
\curveto(319.87868151,91.60420642)(319.85868153,91.50420652)(319.82868759,91.40421297)
\curveto(319.69868169,91.04420698)(319.53368185,90.7292073)(319.33368759,90.45921297)
\curveto(319.13368225,90.18920784)(318.85868253,89.98420804)(318.50868759,89.84421297)
\curveto(318.42868296,89.81420821)(318.34368304,89.78920824)(318.25368759,89.76921297)
\lineto(317.98368759,89.70921297)
\curveto(317.93368345,89.69920833)(317.8886835,89.69420833)(317.84868759,89.69421297)
\curveto(317.80868358,89.70420832)(317.76868362,89.70420832)(317.72868759,89.69421297)
\curveto(317.62868376,89.67420835)(317.53368385,89.67420835)(317.44368759,89.69421297)
\moveto(316.60368759,91.08921297)
\curveto(316.64368474,91.01920701)(316.6836847,90.95420707)(316.72368759,90.89421297)
\curveto(316.76368462,90.84420718)(316.81368457,90.79420723)(316.87368759,90.74421297)
\lineto(317.02368759,90.62421297)
\curveto(317.0836843,90.59420743)(317.14868424,90.56920746)(317.21868759,90.54921297)
\curveto(317.25868413,90.5292075)(317.29368409,90.51920751)(317.32368759,90.51921297)
\curveto(317.36368402,90.5292075)(317.40368398,90.5242075)(317.44368759,90.50421297)
\curveto(317.47368391,90.50420752)(317.51368387,90.49920753)(317.56368759,90.48921297)
\curveto(317.61368377,90.48920754)(317.65368373,90.49420753)(317.68368759,90.50421297)
\lineto(317.90868759,90.54921297)
\curveto(318.15868323,90.6292074)(318.34368304,90.75420727)(318.46368759,90.92421297)
\curveto(318.54368284,91.024207)(318.61368277,91.15420687)(318.67368759,91.31421297)
\curveto(318.75368263,91.49420653)(318.81368257,91.71920631)(318.85368759,91.98921297)
\curveto(318.89368249,92.26920576)(318.90868248,92.54920548)(318.89868759,92.82921297)
\curveto(318.8886825,93.11920491)(318.85868253,93.39420463)(318.80868759,93.65421297)
\curveto(318.75868263,93.91420411)(318.6836827,94.1242039)(318.58368759,94.28421297)
\curveto(318.46368292,94.48420354)(318.31368307,94.63420339)(318.13368759,94.73421297)
\curveto(318.05368333,94.78420324)(317.96368342,94.81420321)(317.86368759,94.82421297)
\curveto(317.76368362,94.84420318)(317.65868373,94.85420317)(317.54868759,94.85421297)
\curveto(317.52868386,94.84420318)(317.50368388,94.83920319)(317.47368759,94.83921297)
\curveto(317.45368393,94.84920318)(317.43368395,94.84920318)(317.41368759,94.83921297)
\curveto(317.36368402,94.8292032)(317.31868407,94.81920321)(317.27868759,94.80921297)
\curveto(317.23868415,94.80920322)(317.19868419,94.79920323)(317.15868759,94.77921297)
\curveto(316.97868441,94.69920333)(316.82868456,94.57920345)(316.70868759,94.41921297)
\curveto(316.59868479,94.25920377)(316.50868488,94.07920395)(316.43868759,93.87921297)
\curveto(316.37868501,93.68920434)(316.33368505,93.46420456)(316.30368759,93.20421297)
\curveto(316.2836851,92.94420508)(316.27868511,92.67920535)(316.28868759,92.40921297)
\curveto(316.29868509,92.14920588)(316.32868506,91.89920613)(316.37868759,91.65921297)
\curveto(316.43868495,91.4292066)(316.51368487,91.23920679)(316.60368759,91.08921297)
\moveto(327.40368759,88.10421297)
\curveto(327.41367397,88.05420997)(327.41867397,87.96421006)(327.41868759,87.83421297)
\curveto(327.41867397,87.70421032)(327.40867398,87.61421041)(327.38868759,87.56421297)
\curveto(327.36867402,87.51421051)(327.36367402,87.45921057)(327.37368759,87.39921297)
\curveto(327.383674,87.34921068)(327.383674,87.29921073)(327.37368759,87.24921297)
\curveto(327.33367405,87.10921092)(327.30367408,86.97421105)(327.28368759,86.84421297)
\curveto(327.27367411,86.71421131)(327.24367414,86.59421143)(327.19368759,86.48421297)
\curveto(327.05367433,86.13421189)(326.8886745,85.83921219)(326.69868759,85.59921297)
\curveto(326.50867488,85.36921266)(326.23867515,85.18421284)(325.88868759,85.04421297)
\curveto(325.80867558,85.01421301)(325.72367566,84.99421303)(325.63368759,84.98421297)
\curveto(325.54367584,84.96421306)(325.45867593,84.94421308)(325.37868759,84.92421297)
\curveto(325.32867606,84.91421311)(325.27867611,84.90921312)(325.22868759,84.90921297)
\curveto(325.17867621,84.90921312)(325.12867626,84.90421312)(325.07868759,84.89421297)
\curveto(325.04867634,84.88421314)(324.99867639,84.88421314)(324.92868759,84.89421297)
\curveto(324.85867653,84.89421313)(324.80867658,84.89921313)(324.77868759,84.90921297)
\curveto(324.71867667,84.9292131)(324.65867673,84.93921309)(324.59868759,84.93921297)
\curveto(324.54867684,84.9292131)(324.49867689,84.93421309)(324.44868759,84.95421297)
\curveto(324.35867703,84.97421305)(324.26867712,84.99921303)(324.17868759,85.02921297)
\curveto(324.09867729,85.04921298)(324.01867737,85.07921295)(323.93868759,85.11921297)
\curveto(323.61867777,85.25921277)(323.36867802,85.45421257)(323.18868759,85.70421297)
\curveto(323.00867838,85.96421206)(322.85867853,86.26921176)(322.73868759,86.61921297)
\curveto(322.71867867,86.69921133)(322.70367868,86.78421124)(322.69368759,86.87421297)
\curveto(322.6836787,86.96421106)(322.66867872,87.04921098)(322.64868759,87.12921297)
\curveto(322.63867875,87.15921087)(322.63367875,87.18921084)(322.63368759,87.21921297)
\lineto(322.63368759,87.32421297)
\curveto(322.61367877,87.40421062)(322.60367878,87.48421054)(322.60368759,87.56421297)
\lineto(322.60368759,87.69921297)
\curveto(322.5836788,87.79921023)(322.5836788,87.89921013)(322.60368759,87.99921297)
\lineto(322.60368759,88.17921297)
\curveto(322.61367877,88.2292098)(322.61867877,88.27420975)(322.61868759,88.31421297)
\curveto(322.61867877,88.36420966)(322.62367876,88.40920962)(322.63368759,88.44921297)
\curveto(322.64367874,88.48920954)(322.64867874,88.5242095)(322.64868759,88.55421297)
\curveto(322.64867874,88.59420943)(322.65367873,88.63420939)(322.66368759,88.67421297)
\lineto(322.72368759,89.00421297)
\curveto(322.74367864,89.1242089)(322.77367861,89.23420879)(322.81368759,89.33421297)
\curveto(322.95367843,89.66420836)(323.11367827,89.93920809)(323.29368759,90.15921297)
\curveto(323.4836779,90.38920764)(323.74367764,90.57420745)(324.07368759,90.71421297)
\curveto(324.15367723,90.75420727)(324.23867715,90.77920725)(324.32868759,90.78921297)
\lineto(324.62868759,90.84921297)
\lineto(324.76368759,90.84921297)
\curveto(324.81367657,90.85920717)(324.86367652,90.86420716)(324.91368759,90.86421297)
\curveto(325.4836759,90.88420714)(325.94367544,90.77920725)(326.29368759,90.54921297)
\curveto(326.65367473,90.3292077)(326.91867447,90.029208)(327.08868759,89.64921297)
\curveto(327.13867425,89.54920848)(327.17867421,89.44920858)(327.20868759,89.34921297)
\curveto(327.23867415,89.24920878)(327.26867412,89.14420888)(327.29868759,89.03421297)
\curveto(327.30867408,88.99420903)(327.31367407,88.95920907)(327.31368759,88.92921297)
\curveto(327.31367407,88.90920912)(327.31867407,88.87920915)(327.32868759,88.83921297)
\curveto(327.34867404,88.76920926)(327.35867403,88.69420933)(327.35868759,88.61421297)
\curveto(327.35867403,88.53420949)(327.36867402,88.45420957)(327.38868759,88.37421297)
\curveto(327.388674,88.3242097)(327.388674,88.27920975)(327.38868759,88.23921297)
\curveto(327.388674,88.19920983)(327.39367399,88.15420987)(327.40368759,88.10421297)
\moveto(326.29368759,87.66921297)
\curveto(326.30367508,87.71921031)(326.30867508,87.79421023)(326.30868759,87.89421297)
\curveto(326.31867507,87.99421003)(326.31367507,88.06920996)(326.29368759,88.11921297)
\curveto(326.27367511,88.17920985)(326.26867512,88.23420979)(326.27868759,88.28421297)
\curveto(326.29867509,88.34420968)(326.29867509,88.40420962)(326.27868759,88.46421297)
\curveto(326.26867512,88.49420953)(326.26367512,88.5292095)(326.26368759,88.56921297)
\curveto(326.26367512,88.60920942)(326.25867513,88.64920938)(326.24868759,88.68921297)
\curveto(326.22867516,88.76920926)(326.20867518,88.84420918)(326.18868759,88.91421297)
\curveto(326.17867521,88.99420903)(326.16367522,89.07420895)(326.14368759,89.15421297)
\curveto(326.11367527,89.21420881)(326.0886753,89.27420875)(326.06868759,89.33421297)
\curveto(326.04867534,89.39420863)(326.01867537,89.45420857)(325.97868759,89.51421297)
\curveto(325.87867551,89.68420834)(325.74867564,89.81920821)(325.58868759,89.91921297)
\curveto(325.50867588,89.96920806)(325.41367597,90.00420802)(325.30368759,90.02421297)
\curveto(325.19367619,90.04420798)(325.06867632,90.05420797)(324.92868759,90.05421297)
\curveto(324.90867648,90.04420798)(324.8836765,90.03920799)(324.85368759,90.03921297)
\curveto(324.82367656,90.04920798)(324.79367659,90.04920798)(324.76368759,90.03921297)
\lineto(324.61368759,89.97921297)
\curveto(324.56367682,89.96920806)(324.51867687,89.95420807)(324.47868759,89.93421297)
\curveto(324.2886771,89.8242082)(324.14367724,89.67920835)(324.04368759,89.49921297)
\curveto(323.95367743,89.31920871)(323.87367751,89.11420891)(323.80368759,88.88421297)
\curveto(323.76367762,88.75420927)(323.74367764,88.61920941)(323.74368759,88.47921297)
\curveto(323.74367764,88.34920968)(323.73367765,88.20420982)(323.71368759,88.04421297)
\curveto(323.70367768,87.99421003)(323.69367769,87.93421009)(323.68368759,87.86421297)
\curveto(323.6836777,87.79421023)(323.69367769,87.73421029)(323.71368759,87.68421297)
\lineto(323.71368759,87.51921297)
\lineto(323.71368759,87.33921297)
\curveto(323.72367766,87.28921074)(323.73367765,87.23421079)(323.74368759,87.17421297)
\curveto(323.75367763,87.1242109)(323.75867763,87.06921096)(323.75868759,87.00921297)
\curveto(323.76867762,86.94921108)(323.7836776,86.89421113)(323.80368759,86.84421297)
\curveto(323.85367753,86.65421137)(323.91367747,86.47921155)(323.98368759,86.31921297)
\curveto(324.05367733,86.15921187)(324.15867723,86.029212)(324.29868759,85.92921297)
\curveto(324.42867696,85.8292122)(324.56867682,85.75921227)(324.71868759,85.71921297)
\curveto(324.74867664,85.70921232)(324.77367661,85.70421232)(324.79368759,85.70421297)
\curveto(324.82367656,85.71421231)(324.85367653,85.71421231)(324.88368759,85.70421297)
\curveto(324.90367648,85.70421232)(324.93367645,85.69921233)(324.97368759,85.68921297)
\curveto(325.01367637,85.68921234)(325.04867634,85.69421233)(325.07868759,85.70421297)
\curveto(325.11867627,85.71421231)(325.15867623,85.71921231)(325.19868759,85.71921297)
\curveto(325.23867615,85.71921231)(325.27867611,85.7292123)(325.31868759,85.74921297)
\curveto(325.55867583,85.8292122)(325.75367563,85.96421206)(325.90368759,86.15421297)
\curveto(326.02367536,86.33421169)(326.11367527,86.53921149)(326.17368759,86.76921297)
\curveto(326.19367519,86.83921119)(326.20867518,86.90921112)(326.21868759,86.97921297)
\curveto(326.22867516,87.05921097)(326.24367514,87.13921089)(326.26368759,87.21921297)
\curveto(326.26367512,87.27921075)(326.26867512,87.3242107)(326.27868759,87.35421297)
\curveto(326.27867511,87.37421065)(326.27867511,87.39921063)(326.27868759,87.42921297)
\curveto(326.27867511,87.46921056)(326.2836751,87.49921053)(326.29368759,87.51921297)
\lineto(326.29368759,87.66921297)
}
}
{
\newrgbcolor{curcolor}{0 0 0}
\pscustom[linestyle=none,fillstyle=solid,fillcolor=curcolor]
{
\newpath
\moveto(164.79102035,55.27308504)
\curveto(165.48101572,55.28307441)(166.08101512,55.16307453)(166.59102035,54.91308504)
\curveto(167.11101409,54.66307503)(167.50601369,54.32807536)(167.77602035,53.90808504)
\curveto(167.82601337,53.82807586)(167.87101333,53.73807595)(167.91102035,53.63808504)
\curveto(167.95101325,53.54807614)(167.9960132,53.45307624)(168.04602035,53.35308504)
\curveto(168.08601311,53.25307644)(168.11601308,53.15307654)(168.13602035,53.05308504)
\curveto(168.15601304,52.95307674)(168.17601302,52.84807684)(168.19602035,52.73808504)
\curveto(168.21601298,52.688077)(168.22101298,52.64307705)(168.21102035,52.60308504)
\curveto(168.201013,52.56307713)(168.20601299,52.51807717)(168.22602035,52.46808504)
\curveto(168.23601296,52.41807727)(168.24101296,52.33307736)(168.24102035,52.21308504)
\curveto(168.24101296,52.10307759)(168.23601296,52.01807767)(168.22602035,51.95808504)
\curveto(168.20601299,51.89807779)(168.196013,51.83807785)(168.19602035,51.77808504)
\curveto(168.20601299,51.71807797)(168.201013,51.65807803)(168.18102035,51.59808504)
\curveto(168.14101306,51.45807823)(168.10601309,51.32307837)(168.07602035,51.19308504)
\curveto(168.04601315,51.06307863)(168.00601319,50.93807875)(167.95602035,50.81808504)
\curveto(167.8960133,50.67807901)(167.82601337,50.55307914)(167.74602035,50.44308504)
\curveto(167.67601352,50.33307936)(167.6010136,50.22307947)(167.52102035,50.11308504)
\lineto(167.46102035,50.05308504)
\curveto(167.45101375,50.03307966)(167.43601376,50.01307968)(167.41602035,49.99308504)
\curveto(167.2960139,49.83307986)(167.16101404,49.68808)(167.01102035,49.55808504)
\curveto(166.86101434,49.42808026)(166.7010145,49.30308039)(166.53102035,49.18308504)
\curveto(166.22101498,48.96308073)(165.92601527,48.75808093)(165.64602035,48.56808504)
\curveto(165.41601578,48.42808126)(165.18601601,48.2930814)(164.95602035,48.16308504)
\curveto(164.73601646,48.03308166)(164.51601668,47.89808179)(164.29602035,47.75808504)
\curveto(164.04601715,47.5880821)(163.80601739,47.40808228)(163.57602035,47.21808504)
\curveto(163.35601784,47.02808266)(163.16601803,46.80308289)(163.00602035,46.54308504)
\curveto(162.96601823,46.48308321)(162.93101827,46.42308327)(162.90102035,46.36308504)
\curveto(162.87101833,46.31308338)(162.84101836,46.24808344)(162.81102035,46.16808504)
\curveto(162.79101841,46.09808359)(162.78601841,46.03808365)(162.79602035,45.98808504)
\curveto(162.81601838,45.91808377)(162.85101835,45.86308383)(162.90102035,45.82308504)
\curveto(162.95101825,45.7930839)(163.01101819,45.77308392)(163.08102035,45.76308504)
\lineto(163.32102035,45.76308504)
\lineto(164.07102035,45.76308504)
\lineto(166.87602035,45.76308504)
\lineto(167.53602035,45.76308504)
\curveto(167.62601357,45.76308393)(167.71101349,45.75808393)(167.79102035,45.74808504)
\curveto(167.87101333,45.74808394)(167.93601326,45.72808396)(167.98602035,45.68808504)
\curveto(168.03601316,45.64808404)(168.07601312,45.57308412)(168.10602035,45.46308504)
\curveto(168.14601305,45.36308433)(168.15601304,45.26308443)(168.13602035,45.16308504)
\lineto(168.13602035,45.02808504)
\curveto(168.11601308,44.95808473)(168.0960131,44.89808479)(168.07602035,44.84808504)
\curveto(168.05601314,44.79808489)(168.02101318,44.75808493)(167.97102035,44.72808504)
\curveto(167.92101328,44.688085)(167.85101335,44.66808502)(167.76102035,44.66808504)
\lineto(167.49102035,44.66808504)
\lineto(166.59102035,44.66808504)
\lineto(163.08102035,44.66808504)
\lineto(162.01602035,44.66808504)
\curveto(161.93601926,44.66808502)(161.84601935,44.66308503)(161.74602035,44.65308504)
\curveto(161.64601955,44.65308504)(161.56101964,44.66308503)(161.49102035,44.68308504)
\curveto(161.28101992,44.75308494)(161.21601998,44.93308476)(161.29602035,45.22308504)
\curveto(161.30601989,45.26308443)(161.30601989,45.29808439)(161.29602035,45.32808504)
\curveto(161.2960199,45.36808432)(161.30601989,45.41308428)(161.32602035,45.46308504)
\curveto(161.34601985,45.54308415)(161.36601983,45.62808406)(161.38602035,45.71808504)
\curveto(161.40601979,45.80808388)(161.43101977,45.8930838)(161.46102035,45.97308504)
\curveto(161.62101958,46.46308323)(161.82101938,46.87808281)(162.06102035,47.21808504)
\curveto(162.24101896,47.46808222)(162.44601875,47.693082)(162.67602035,47.89308504)
\curveto(162.90601829,48.10308159)(163.14601805,48.29808139)(163.39602035,48.47808504)
\curveto(163.65601754,48.65808103)(163.92101728,48.82808086)(164.19102035,48.98808504)
\curveto(164.47101673,49.15808053)(164.74101646,49.33308036)(165.00102035,49.51308504)
\curveto(165.11101609,49.5930801)(165.21601598,49.66808002)(165.31602035,49.73808504)
\curveto(165.42601577,49.80807988)(165.53601566,49.88307981)(165.64602035,49.96308504)
\curveto(165.68601551,49.9930797)(165.72101548,50.02307967)(165.75102035,50.05308504)
\curveto(165.79101541,50.0930796)(165.83101537,50.12307957)(165.87102035,50.14308504)
\curveto(166.01101519,50.25307944)(166.13601506,50.37807931)(166.24602035,50.51808504)
\curveto(166.26601493,50.54807914)(166.29101491,50.57307912)(166.32102035,50.59308504)
\curveto(166.35101485,50.62307907)(166.37601482,50.65307904)(166.39602035,50.68308504)
\curveto(166.47601472,50.78307891)(166.54101466,50.88307881)(166.59102035,50.98308504)
\curveto(166.65101455,51.08307861)(166.70601449,51.1930785)(166.75602035,51.31308504)
\curveto(166.78601441,51.38307831)(166.80601439,51.45807823)(166.81602035,51.53808504)
\lineto(166.87602035,51.77808504)
\lineto(166.87602035,51.86808504)
\curveto(166.88601431,51.89807779)(166.89101431,51.92807776)(166.89102035,51.95808504)
\curveto(166.91101429,52.02807766)(166.91601428,52.12307757)(166.90602035,52.24308504)
\curveto(166.90601429,52.37307732)(166.8960143,52.47307722)(166.87602035,52.54308504)
\curveto(166.85601434,52.62307707)(166.83601436,52.69807699)(166.81602035,52.76808504)
\curveto(166.80601439,52.84807684)(166.78601441,52.92807676)(166.75602035,53.00808504)
\curveto(166.64601455,53.24807644)(166.4960147,53.44807624)(166.30602035,53.60808504)
\curveto(166.12601507,53.77807591)(165.90601529,53.91807577)(165.64602035,54.02808504)
\curveto(165.57601562,54.04807564)(165.50601569,54.06307563)(165.43602035,54.07308504)
\curveto(165.36601583,54.0930756)(165.29101591,54.11307558)(165.21102035,54.13308504)
\curveto(165.13101607,54.15307554)(165.02101618,54.16307553)(164.88102035,54.16308504)
\curveto(164.75101645,54.16307553)(164.64601655,54.15307554)(164.56602035,54.13308504)
\curveto(164.50601669,54.12307557)(164.45101675,54.11807557)(164.40102035,54.11808504)
\curveto(164.35101685,54.11807557)(164.3010169,54.10807558)(164.25102035,54.08808504)
\curveto(164.15101705,54.04807564)(164.05601714,54.00807568)(163.96602035,53.96808504)
\curveto(163.88601731,53.92807576)(163.80601739,53.88307581)(163.72602035,53.83308504)
\curveto(163.6960175,53.81307588)(163.66601753,53.7880759)(163.63602035,53.75808504)
\curveto(163.61601758,53.72807596)(163.59101761,53.70307599)(163.56102035,53.68308504)
\lineto(163.48602035,53.60808504)
\curveto(163.45601774,53.5880761)(163.43101777,53.56807612)(163.41102035,53.54808504)
\lineto(163.26102035,53.33808504)
\curveto(163.22101798,53.27807641)(163.17601802,53.21307648)(163.12602035,53.14308504)
\curveto(163.06601813,53.05307664)(163.01601818,52.94807674)(162.97602035,52.82808504)
\curveto(162.94601825,52.71807697)(162.91101829,52.60807708)(162.87102035,52.49808504)
\curveto(162.83101837,52.3880773)(162.80601839,52.24307745)(162.79602035,52.06308504)
\curveto(162.78601841,51.8930778)(162.75601844,51.76807792)(162.70602035,51.68808504)
\curveto(162.65601854,51.60807808)(162.58101862,51.56307813)(162.48102035,51.55308504)
\curveto(162.38101882,51.54307815)(162.27101893,51.53807815)(162.15102035,51.53808504)
\curveto(162.11101909,51.53807815)(162.07101913,51.53307816)(162.03102035,51.52308504)
\curveto(161.99101921,51.52307817)(161.95601924,51.52807816)(161.92602035,51.53808504)
\curveto(161.87601932,51.55807813)(161.82601937,51.56807812)(161.77602035,51.56808504)
\curveto(161.73601946,51.56807812)(161.6960195,51.57807811)(161.65602035,51.59808504)
\curveto(161.56601963,51.65807803)(161.52101968,51.7930779)(161.52102035,52.00308504)
\lineto(161.52102035,52.12308504)
\curveto(161.53101967,52.18307751)(161.53601966,52.24307745)(161.53602035,52.30308504)
\curveto(161.54601965,52.37307732)(161.55601964,52.43807725)(161.56602035,52.49808504)
\curveto(161.58601961,52.60807708)(161.60601959,52.70807698)(161.62602035,52.79808504)
\curveto(161.64601955,52.89807679)(161.67601952,52.9930767)(161.71602035,53.08308504)
\curveto(161.73601946,53.15307654)(161.75601944,53.21307648)(161.77602035,53.26308504)
\lineto(161.83602035,53.44308504)
\curveto(161.95601924,53.70307599)(162.11101909,53.94807574)(162.30102035,54.17808504)
\curveto(162.5010187,54.40807528)(162.71601848,54.5930751)(162.94602035,54.73308504)
\curveto(163.05601814,54.81307488)(163.17101803,54.87807481)(163.29102035,54.92808504)
\lineto(163.68102035,55.07808504)
\curveto(163.79101741,55.12807456)(163.90601729,55.15807453)(164.02602035,55.16808504)
\curveto(164.14601705,55.1880745)(164.27101693,55.21307448)(164.40102035,55.24308504)
\curveto(164.47101673,55.24307445)(164.53601666,55.24307445)(164.59602035,55.24308504)
\curveto(164.65601654,55.25307444)(164.72101648,55.26307443)(164.79102035,55.27308504)
}
}
{
\newrgbcolor{curcolor}{0 0 0}
\pscustom[linestyle=none,fillstyle=solid,fillcolor=curcolor]
{
\newpath
\moveto(176.77062973,48.16308504)
\curveto(176.84062208,48.11308158)(176.88062204,48.04308165)(176.89062973,47.95308504)
\curveto(176.91062201,47.86308183)(176.920622,47.75808193)(176.92062973,47.63808504)
\curveto(176.920622,47.5880821)(176.91562201,47.53808215)(176.90562973,47.48808504)
\curveto(176.90562202,47.43808225)(176.89562203,47.3930823)(176.87562973,47.35308504)
\curveto(176.84562208,47.26308243)(176.78562214,47.20308249)(176.69562973,47.17308504)
\curveto(176.61562231,47.15308254)(176.5206224,47.14308255)(176.41062973,47.14308504)
\lineto(176.09562973,47.14308504)
\curveto(175.98562294,47.15308254)(175.88062304,47.14308255)(175.78062973,47.11308504)
\curveto(175.64062328,47.08308261)(175.55062337,47.00308269)(175.51062973,46.87308504)
\curveto(175.49062343,46.80308289)(175.48062344,46.71808297)(175.48062973,46.61808504)
\lineto(175.48062973,46.34808504)
\lineto(175.48062973,45.40308504)
\lineto(175.48062973,45.07308504)
\curveto(175.48062344,44.96308473)(175.46062346,44.87808481)(175.42062973,44.81808504)
\curveto(175.38062354,44.75808493)(175.33062359,44.71808497)(175.27062973,44.69808504)
\curveto(175.2206237,44.688085)(175.15562377,44.67308502)(175.07562973,44.65308504)
\lineto(174.88062973,44.65308504)
\curveto(174.76062416,44.65308504)(174.65562427,44.65808503)(174.56562973,44.66808504)
\curveto(174.47562445,44.688085)(174.40562452,44.73808495)(174.35562973,44.81808504)
\curveto(174.3256246,44.86808482)(174.31062461,44.93808475)(174.31062973,45.02808504)
\lineto(174.31062973,45.32808504)
\lineto(174.31062973,46.36308504)
\curveto(174.31062461,46.52308317)(174.30062462,46.66808302)(174.28062973,46.79808504)
\curveto(174.27062465,46.93808275)(174.21562471,47.03308266)(174.11562973,47.08308504)
\curveto(174.06562486,47.10308259)(173.99562493,47.11808257)(173.90562973,47.12808504)
\curveto(173.8256251,47.13808255)(173.73562519,47.14308255)(173.63562973,47.14308504)
\lineto(173.35062973,47.14308504)
\lineto(173.11062973,47.14308504)
\lineto(170.84562973,47.14308504)
\curveto(170.75562817,47.14308255)(170.65062827,47.13808255)(170.53062973,47.12808504)
\lineto(170.20062973,47.12808504)
\curveto(170.09062883,47.12808256)(169.99062893,47.13808255)(169.90062973,47.15808504)
\curveto(169.81062911,47.17808251)(169.75062917,47.21308248)(169.72062973,47.26308504)
\curveto(169.67062925,47.33308236)(169.64562928,47.42808226)(169.64562973,47.54808504)
\lineto(169.64562973,47.89308504)
\lineto(169.64562973,48.16308504)
\curveto(169.68562924,48.33308136)(169.74062918,48.47308122)(169.81062973,48.58308504)
\curveto(169.88062904,48.693081)(169.96062896,48.80808088)(170.05062973,48.92808504)
\lineto(170.41062973,49.46808504)
\curveto(170.85062807,50.09807959)(171.28562764,50.71807897)(171.71562973,51.32808504)
\lineto(173.03562973,53.18808504)
\curveto(173.19562573,53.41807627)(173.35062557,53.63807605)(173.50062973,53.84808504)
\curveto(173.65062527,54.06807562)(173.80562512,54.2930754)(173.96562973,54.52308504)
\curveto(174.01562491,54.5930751)(174.06562486,54.65807503)(174.11562973,54.71808504)
\curveto(174.16562476,54.7880749)(174.21562471,54.86307483)(174.26562973,54.94308504)
\lineto(174.32562973,55.03308504)
\curveto(174.35562457,55.07307462)(174.38562454,55.10307459)(174.41562973,55.12308504)
\curveto(174.45562447,55.15307454)(174.49562443,55.17307452)(174.53562973,55.18308504)
\curveto(174.57562435,55.20307449)(174.6206243,55.22307447)(174.67062973,55.24308504)
\curveto(174.69062423,55.24307445)(174.71062421,55.23807445)(174.73062973,55.22808504)
\curveto(174.76062416,55.22807446)(174.78562414,55.23807445)(174.80562973,55.25808504)
\curveto(174.93562399,55.25807443)(175.05562387,55.25307444)(175.16562973,55.24308504)
\curveto(175.27562365,55.23307446)(175.35562357,55.1880745)(175.40562973,55.10808504)
\curveto(175.44562348,55.05807463)(175.46562346,54.9880747)(175.46562973,54.89808504)
\curveto(175.47562345,54.80807488)(175.48062344,54.71307498)(175.48062973,54.61308504)
\lineto(175.48062973,49.15308504)
\curveto(175.48062344,49.08308061)(175.47562345,49.00808068)(175.46562973,48.92808504)
\curveto(175.46562346,48.85808083)(175.47062345,48.7880809)(175.48062973,48.71808504)
\lineto(175.48062973,48.61308504)
\curveto(175.50062342,48.56308113)(175.51562341,48.50808118)(175.52562973,48.44808504)
\curveto(175.53562339,48.39808129)(175.56062336,48.35808133)(175.60062973,48.32808504)
\curveto(175.67062325,48.27808141)(175.75562317,48.24808144)(175.85562973,48.23808504)
\lineto(176.18562973,48.23808504)
\curveto(176.29562263,48.23808145)(176.40062252,48.23308146)(176.50062973,48.22308504)
\curveto(176.61062231,48.22308147)(176.70062222,48.20308149)(176.77062973,48.16308504)
\moveto(174.20562973,48.35808504)
\curveto(174.28562464,48.46808122)(174.3206246,48.63808105)(174.31062973,48.86808504)
\lineto(174.31062973,49.48308504)
\lineto(174.31062973,51.95808504)
\lineto(174.31062973,52.27308504)
\curveto(174.3206246,52.3930773)(174.31562461,52.4930772)(174.29562973,52.57308504)
\lineto(174.29562973,52.72308504)
\curveto(174.29562463,52.81307688)(174.28062464,52.89807679)(174.25062973,52.97808504)
\curveto(174.24062468,52.99807669)(174.23062469,53.00807668)(174.22062973,53.00808504)
\lineto(174.17562973,53.05308504)
\curveto(174.15562477,53.06307663)(174.1256248,53.06807662)(174.08562973,53.06808504)
\curveto(174.06562486,53.04807664)(174.04562488,53.03307666)(174.02562973,53.02308504)
\curveto(174.01562491,53.02307667)(174.00062492,53.01807667)(173.98062973,53.00808504)
\curveto(173.920625,52.95807673)(173.86062506,52.8880768)(173.80062973,52.79808504)
\curveto(173.74062518,52.70807698)(173.68562524,52.62807706)(173.63562973,52.55808504)
\curveto(173.53562539,52.41807727)(173.44062548,52.27307742)(173.35062973,52.12308504)
\curveto(173.26062566,51.98307771)(173.16562576,51.84307785)(173.06562973,51.70308504)
\lineto(172.52562973,50.92308504)
\curveto(172.35562657,50.66307903)(172.18062674,50.40307929)(172.00062973,50.14308504)
\curveto(171.920627,50.03307966)(171.84562708,49.92807976)(171.77562973,49.82808504)
\lineto(171.56562973,49.52808504)
\curveto(171.51562741,49.44808024)(171.46562746,49.37308032)(171.41562973,49.30308504)
\curveto(171.37562755,49.23308046)(171.33062759,49.15808053)(171.28062973,49.07808504)
\curveto(171.23062769,49.01808067)(171.18062774,48.95308074)(171.13062973,48.88308504)
\curveto(171.09062783,48.82308087)(171.05062787,48.75308094)(171.01062973,48.67308504)
\curveto(170.97062795,48.61308108)(170.94562798,48.54308115)(170.93562973,48.46308504)
\curveto(170.925628,48.3930813)(170.96062796,48.33808135)(171.04062973,48.29808504)
\curveto(171.11062781,48.24808144)(171.2206277,48.22308147)(171.37062973,48.22308504)
\curveto(171.53062739,48.23308146)(171.66562726,48.23808145)(171.77562973,48.23808504)
\lineto(173.45562973,48.23808504)
\lineto(173.89062973,48.23808504)
\curveto(174.04062488,48.23808145)(174.14562478,48.27808141)(174.20562973,48.35808504)
}
}
{
\newrgbcolor{curcolor}{0 0 0}
\pscustom[linestyle=none,fillstyle=solid,fillcolor=curcolor]
{
\newpath
\moveto(179.1952391,46.30308504)
\lineto(179.4952391,46.30308504)
\curveto(179.60523704,46.31308338)(179.71023694,46.31308338)(179.8102391,46.30308504)
\curveto(179.92023673,46.30308339)(180.02023663,46.2930834)(180.1102391,46.27308504)
\curveto(180.20023645,46.26308343)(180.27023638,46.23808345)(180.3202391,46.19808504)
\curveto(180.34023631,46.17808351)(180.35523629,46.14808354)(180.3652391,46.10808504)
\curveto(180.38523626,46.06808362)(180.40523624,46.02308367)(180.4252391,45.97308504)
\lineto(180.4252391,45.89808504)
\curveto(180.43523621,45.84808384)(180.43523621,45.7930839)(180.4252391,45.73308504)
\lineto(180.4252391,45.58308504)
\lineto(180.4252391,45.10308504)
\curveto(180.42523622,44.93308476)(180.38523626,44.81308488)(180.3052391,44.74308504)
\curveto(180.23523641,44.693085)(180.1452365,44.66808502)(180.0352391,44.66808504)
\lineto(179.7052391,44.66808504)
\lineto(179.2552391,44.66808504)
\curveto(179.10523754,44.66808502)(178.99023766,44.69808499)(178.9102391,44.75808504)
\curveto(178.87023778,44.7880849)(178.84023781,44.83808485)(178.8202391,44.90808504)
\curveto(178.80023785,44.9880847)(178.78523786,45.07308462)(178.7752391,45.16308504)
\lineto(178.7752391,45.44808504)
\curveto(178.78523786,45.54808414)(178.79023786,45.63308406)(178.7902391,45.70308504)
\lineto(178.7902391,45.89808504)
\curveto(178.79023786,45.95808373)(178.80023785,46.01308368)(178.8202391,46.06308504)
\curveto(178.86023779,46.17308352)(178.93023772,46.24308345)(179.0302391,46.27308504)
\curveto(179.06023759,46.27308342)(179.11523753,46.28308341)(179.1952391,46.30308504)
}
}
{
\newrgbcolor{curcolor}{0 0 0}
\pscustom[linestyle=none,fillstyle=solid,fillcolor=curcolor]
{
\newpath
\moveto(185.65039535,55.27308504)
\curveto(186.34039072,55.28307441)(186.94039012,55.16307453)(187.45039535,54.91308504)
\curveto(187.97038909,54.66307503)(188.36538869,54.32807536)(188.63539535,53.90808504)
\curveto(188.68538837,53.82807586)(188.73038833,53.73807595)(188.77039535,53.63808504)
\curveto(188.81038825,53.54807614)(188.8553882,53.45307624)(188.90539535,53.35308504)
\curveto(188.94538811,53.25307644)(188.97538808,53.15307654)(188.99539535,53.05308504)
\curveto(189.01538804,52.95307674)(189.03538802,52.84807684)(189.05539535,52.73808504)
\curveto(189.07538798,52.688077)(189.08038798,52.64307705)(189.07039535,52.60308504)
\curveto(189.060388,52.56307713)(189.06538799,52.51807717)(189.08539535,52.46808504)
\curveto(189.09538796,52.41807727)(189.10038796,52.33307736)(189.10039535,52.21308504)
\curveto(189.10038796,52.10307759)(189.09538796,52.01807767)(189.08539535,51.95808504)
\curveto(189.06538799,51.89807779)(189.055388,51.83807785)(189.05539535,51.77808504)
\curveto(189.06538799,51.71807797)(189.060388,51.65807803)(189.04039535,51.59808504)
\curveto(189.00038806,51.45807823)(188.96538809,51.32307837)(188.93539535,51.19308504)
\curveto(188.90538815,51.06307863)(188.86538819,50.93807875)(188.81539535,50.81808504)
\curveto(188.7553883,50.67807901)(188.68538837,50.55307914)(188.60539535,50.44308504)
\curveto(188.53538852,50.33307936)(188.4603886,50.22307947)(188.38039535,50.11308504)
\lineto(188.32039535,50.05308504)
\curveto(188.31038875,50.03307966)(188.29538876,50.01307968)(188.27539535,49.99308504)
\curveto(188.1553889,49.83307986)(188.02038904,49.68808)(187.87039535,49.55808504)
\curveto(187.72038934,49.42808026)(187.5603895,49.30308039)(187.39039535,49.18308504)
\curveto(187.08038998,48.96308073)(186.78539027,48.75808093)(186.50539535,48.56808504)
\curveto(186.27539078,48.42808126)(186.04539101,48.2930814)(185.81539535,48.16308504)
\curveto(185.59539146,48.03308166)(185.37539168,47.89808179)(185.15539535,47.75808504)
\curveto(184.90539215,47.5880821)(184.66539239,47.40808228)(184.43539535,47.21808504)
\curveto(184.21539284,47.02808266)(184.02539303,46.80308289)(183.86539535,46.54308504)
\curveto(183.82539323,46.48308321)(183.79039327,46.42308327)(183.76039535,46.36308504)
\curveto(183.73039333,46.31308338)(183.70039336,46.24808344)(183.67039535,46.16808504)
\curveto(183.65039341,46.09808359)(183.64539341,46.03808365)(183.65539535,45.98808504)
\curveto(183.67539338,45.91808377)(183.71039335,45.86308383)(183.76039535,45.82308504)
\curveto(183.81039325,45.7930839)(183.87039319,45.77308392)(183.94039535,45.76308504)
\lineto(184.18039535,45.76308504)
\lineto(184.93039535,45.76308504)
\lineto(187.73539535,45.76308504)
\lineto(188.39539535,45.76308504)
\curveto(188.48538857,45.76308393)(188.57038849,45.75808393)(188.65039535,45.74808504)
\curveto(188.73038833,45.74808394)(188.79538826,45.72808396)(188.84539535,45.68808504)
\curveto(188.89538816,45.64808404)(188.93538812,45.57308412)(188.96539535,45.46308504)
\curveto(189.00538805,45.36308433)(189.01538804,45.26308443)(188.99539535,45.16308504)
\lineto(188.99539535,45.02808504)
\curveto(188.97538808,44.95808473)(188.9553881,44.89808479)(188.93539535,44.84808504)
\curveto(188.91538814,44.79808489)(188.88038818,44.75808493)(188.83039535,44.72808504)
\curveto(188.78038828,44.688085)(188.71038835,44.66808502)(188.62039535,44.66808504)
\lineto(188.35039535,44.66808504)
\lineto(187.45039535,44.66808504)
\lineto(183.94039535,44.66808504)
\lineto(182.87539535,44.66808504)
\curveto(182.79539426,44.66808502)(182.70539435,44.66308503)(182.60539535,44.65308504)
\curveto(182.50539455,44.65308504)(182.42039464,44.66308503)(182.35039535,44.68308504)
\curveto(182.14039492,44.75308494)(182.07539498,44.93308476)(182.15539535,45.22308504)
\curveto(182.16539489,45.26308443)(182.16539489,45.29808439)(182.15539535,45.32808504)
\curveto(182.1553949,45.36808432)(182.16539489,45.41308428)(182.18539535,45.46308504)
\curveto(182.20539485,45.54308415)(182.22539483,45.62808406)(182.24539535,45.71808504)
\curveto(182.26539479,45.80808388)(182.29039477,45.8930838)(182.32039535,45.97308504)
\curveto(182.48039458,46.46308323)(182.68039438,46.87808281)(182.92039535,47.21808504)
\curveto(183.10039396,47.46808222)(183.30539375,47.693082)(183.53539535,47.89308504)
\curveto(183.76539329,48.10308159)(184.00539305,48.29808139)(184.25539535,48.47808504)
\curveto(184.51539254,48.65808103)(184.78039228,48.82808086)(185.05039535,48.98808504)
\curveto(185.33039173,49.15808053)(185.60039146,49.33308036)(185.86039535,49.51308504)
\curveto(185.97039109,49.5930801)(186.07539098,49.66808002)(186.17539535,49.73808504)
\curveto(186.28539077,49.80807988)(186.39539066,49.88307981)(186.50539535,49.96308504)
\curveto(186.54539051,49.9930797)(186.58039048,50.02307967)(186.61039535,50.05308504)
\curveto(186.65039041,50.0930796)(186.69039037,50.12307957)(186.73039535,50.14308504)
\curveto(186.87039019,50.25307944)(186.99539006,50.37807931)(187.10539535,50.51808504)
\curveto(187.12538993,50.54807914)(187.15038991,50.57307912)(187.18039535,50.59308504)
\curveto(187.21038985,50.62307907)(187.23538982,50.65307904)(187.25539535,50.68308504)
\curveto(187.33538972,50.78307891)(187.40038966,50.88307881)(187.45039535,50.98308504)
\curveto(187.51038955,51.08307861)(187.56538949,51.1930785)(187.61539535,51.31308504)
\curveto(187.64538941,51.38307831)(187.66538939,51.45807823)(187.67539535,51.53808504)
\lineto(187.73539535,51.77808504)
\lineto(187.73539535,51.86808504)
\curveto(187.74538931,51.89807779)(187.75038931,51.92807776)(187.75039535,51.95808504)
\curveto(187.77038929,52.02807766)(187.77538928,52.12307757)(187.76539535,52.24308504)
\curveto(187.76538929,52.37307732)(187.7553893,52.47307722)(187.73539535,52.54308504)
\curveto(187.71538934,52.62307707)(187.69538936,52.69807699)(187.67539535,52.76808504)
\curveto(187.66538939,52.84807684)(187.64538941,52.92807676)(187.61539535,53.00808504)
\curveto(187.50538955,53.24807644)(187.3553897,53.44807624)(187.16539535,53.60808504)
\curveto(186.98539007,53.77807591)(186.76539029,53.91807577)(186.50539535,54.02808504)
\curveto(186.43539062,54.04807564)(186.36539069,54.06307563)(186.29539535,54.07308504)
\curveto(186.22539083,54.0930756)(186.15039091,54.11307558)(186.07039535,54.13308504)
\curveto(185.99039107,54.15307554)(185.88039118,54.16307553)(185.74039535,54.16308504)
\curveto(185.61039145,54.16307553)(185.50539155,54.15307554)(185.42539535,54.13308504)
\curveto(185.36539169,54.12307557)(185.31039175,54.11807557)(185.26039535,54.11808504)
\curveto(185.21039185,54.11807557)(185.1603919,54.10807558)(185.11039535,54.08808504)
\curveto(185.01039205,54.04807564)(184.91539214,54.00807568)(184.82539535,53.96808504)
\curveto(184.74539231,53.92807576)(184.66539239,53.88307581)(184.58539535,53.83308504)
\curveto(184.5553925,53.81307588)(184.52539253,53.7880759)(184.49539535,53.75808504)
\curveto(184.47539258,53.72807596)(184.45039261,53.70307599)(184.42039535,53.68308504)
\lineto(184.34539535,53.60808504)
\curveto(184.31539274,53.5880761)(184.29039277,53.56807612)(184.27039535,53.54808504)
\lineto(184.12039535,53.33808504)
\curveto(184.08039298,53.27807641)(184.03539302,53.21307648)(183.98539535,53.14308504)
\curveto(183.92539313,53.05307664)(183.87539318,52.94807674)(183.83539535,52.82808504)
\curveto(183.80539325,52.71807697)(183.77039329,52.60807708)(183.73039535,52.49808504)
\curveto(183.69039337,52.3880773)(183.66539339,52.24307745)(183.65539535,52.06308504)
\curveto(183.64539341,51.8930778)(183.61539344,51.76807792)(183.56539535,51.68808504)
\curveto(183.51539354,51.60807808)(183.44039362,51.56307813)(183.34039535,51.55308504)
\curveto(183.24039382,51.54307815)(183.13039393,51.53807815)(183.01039535,51.53808504)
\curveto(182.97039409,51.53807815)(182.93039413,51.53307816)(182.89039535,51.52308504)
\curveto(182.85039421,51.52307817)(182.81539424,51.52807816)(182.78539535,51.53808504)
\curveto(182.73539432,51.55807813)(182.68539437,51.56807812)(182.63539535,51.56808504)
\curveto(182.59539446,51.56807812)(182.5553945,51.57807811)(182.51539535,51.59808504)
\curveto(182.42539463,51.65807803)(182.38039468,51.7930779)(182.38039535,52.00308504)
\lineto(182.38039535,52.12308504)
\curveto(182.39039467,52.18307751)(182.39539466,52.24307745)(182.39539535,52.30308504)
\curveto(182.40539465,52.37307732)(182.41539464,52.43807725)(182.42539535,52.49808504)
\curveto(182.44539461,52.60807708)(182.46539459,52.70807698)(182.48539535,52.79808504)
\curveto(182.50539455,52.89807679)(182.53539452,52.9930767)(182.57539535,53.08308504)
\curveto(182.59539446,53.15307654)(182.61539444,53.21307648)(182.63539535,53.26308504)
\lineto(182.69539535,53.44308504)
\curveto(182.81539424,53.70307599)(182.97039409,53.94807574)(183.16039535,54.17808504)
\curveto(183.3603937,54.40807528)(183.57539348,54.5930751)(183.80539535,54.73308504)
\curveto(183.91539314,54.81307488)(184.03039303,54.87807481)(184.15039535,54.92808504)
\lineto(184.54039535,55.07808504)
\curveto(184.65039241,55.12807456)(184.76539229,55.15807453)(184.88539535,55.16808504)
\curveto(185.00539205,55.1880745)(185.13039193,55.21307448)(185.26039535,55.24308504)
\curveto(185.33039173,55.24307445)(185.39539166,55.24307445)(185.45539535,55.24308504)
\curveto(185.51539154,55.25307444)(185.58039148,55.26307443)(185.65039535,55.27308504)
}
}
{
\newrgbcolor{curcolor}{0 0 0}
\pscustom[linestyle=none,fillstyle=solid,fillcolor=curcolor]
{
\newpath
\moveto(200.55500473,53.18808504)
\curveto(200.35499443,52.89807679)(200.14499464,52.61307708)(199.92500473,52.33308504)
\curveto(199.71499507,52.05307764)(199.50999527,51.76807792)(199.31000473,51.47808504)
\curveto(198.70999607,50.62807906)(198.10499668,49.7880799)(197.49500473,48.95808504)
\curveto(196.8849979,48.13808155)(196.2799985,47.30308239)(195.68000473,46.45308504)
\lineto(195.17000473,45.73308504)
\lineto(194.66000473,45.04308504)
\curveto(194.5800002,44.93308476)(194.50000028,44.81808487)(194.42000473,44.69808504)
\curveto(194.34000044,44.57808511)(194.24500054,44.48308521)(194.13500473,44.41308504)
\curveto(194.09500069,44.3930853)(194.03000075,44.37808531)(193.94000473,44.36808504)
\curveto(193.86000092,44.34808534)(193.77000101,44.33808535)(193.67000473,44.33808504)
\curveto(193.57000121,44.33808535)(193.47500131,44.34308535)(193.38500473,44.35308504)
\curveto(193.30500148,44.36308533)(193.24500154,44.38308531)(193.20500473,44.41308504)
\curveto(193.17500161,44.43308526)(193.15000163,44.46808522)(193.13000473,44.51808504)
\curveto(193.12000166,44.55808513)(193.12500166,44.60308509)(193.14500473,44.65308504)
\curveto(193.1850016,44.73308496)(193.23000155,44.80808488)(193.28000473,44.87808504)
\curveto(193.34000144,44.95808473)(193.39500139,45.03808465)(193.44500473,45.11808504)
\curveto(193.6850011,45.45808423)(193.93000085,45.7930839)(194.18000473,46.12308504)
\curveto(194.43000035,46.45308324)(194.67000011,46.7880829)(194.90000473,47.12808504)
\curveto(195.05999972,47.34808234)(195.21999956,47.56308213)(195.38000473,47.77308504)
\curveto(195.53999924,47.98308171)(195.69999908,48.19808149)(195.86000473,48.41808504)
\curveto(196.21999856,48.93808075)(196.5849982,49.44808024)(196.95500473,49.94808504)
\curveto(197.32499746,50.44807924)(197.69499709,50.95807873)(198.06500473,51.47808504)
\curveto(198.20499658,51.67807801)(198.34499644,51.87307782)(198.48500473,52.06308504)
\curveto(198.63499615,52.25307744)(198.779996,52.44807724)(198.92000473,52.64808504)
\curveto(199.12999565,52.94807674)(199.34499544,53.24807644)(199.56500473,53.54808504)
\lineto(200.22500473,54.44808504)
\lineto(200.40500473,54.71808504)
\lineto(200.61500473,54.98808504)
\lineto(200.73500473,55.16808504)
\curveto(200.784994,55.22807446)(200.83499395,55.28307441)(200.88500473,55.33308504)
\curveto(200.95499383,55.38307431)(201.02999375,55.41807427)(201.11000473,55.43808504)
\curveto(201.12999365,55.44807424)(201.15499363,55.44807424)(201.18500473,55.43808504)
\curveto(201.22499356,55.43807425)(201.25499353,55.44807424)(201.27500473,55.46808504)
\curveto(201.39499339,55.46807422)(201.52999325,55.46307423)(201.68000473,55.45308504)
\curveto(201.82999295,55.45307424)(201.91999286,55.40807428)(201.95000473,55.31808504)
\curveto(201.96999281,55.2880744)(201.97499281,55.25307444)(201.96500473,55.21308504)
\curveto(201.95499283,55.17307452)(201.93999284,55.14307455)(201.92000473,55.12308504)
\curveto(201.8799929,55.04307465)(201.83999294,54.97307472)(201.80000473,54.91308504)
\curveto(201.75999302,54.85307484)(201.71499307,54.7930749)(201.66500473,54.73308504)
\lineto(201.09500473,53.95308504)
\curveto(200.91499387,53.70307599)(200.73499405,53.44807624)(200.55500473,53.18808504)
\moveto(193.70000473,49.28808504)
\curveto(193.65000113,49.30808038)(193.60000118,49.31308038)(193.55000473,49.30308504)
\curveto(193.50000128,49.2930804)(193.45000133,49.29808039)(193.40000473,49.31808504)
\curveto(193.29000149,49.33808035)(193.1850016,49.35808033)(193.08500473,49.37808504)
\curveto(192.99500179,49.40808028)(192.90000188,49.44808024)(192.80000473,49.49808504)
\curveto(192.47000231,49.63808005)(192.21500257,49.83307986)(192.03500473,50.08308504)
\curveto(191.85500293,50.34307935)(191.71000307,50.65307904)(191.60000473,51.01308504)
\curveto(191.57000321,51.0930786)(191.55000323,51.17307852)(191.54000473,51.25308504)
\curveto(191.53000325,51.34307835)(191.51500327,51.42807826)(191.49500473,51.50808504)
\curveto(191.4850033,51.55807813)(191.4800033,51.62307807)(191.48000473,51.70308504)
\curveto(191.47000331,51.73307796)(191.46500332,51.76307793)(191.46500473,51.79308504)
\curveto(191.46500332,51.83307786)(191.46000332,51.86807782)(191.45000473,51.89808504)
\lineto(191.45000473,52.04808504)
\curveto(191.44000334,52.09807759)(191.43500335,52.15807753)(191.43500473,52.22808504)
\curveto(191.43500335,52.30807738)(191.44000334,52.37307732)(191.45000473,52.42308504)
\lineto(191.45000473,52.58808504)
\curveto(191.47000331,52.63807705)(191.47500331,52.68307701)(191.46500473,52.72308504)
\curveto(191.46500332,52.77307692)(191.47000331,52.81807687)(191.48000473,52.85808504)
\curveto(191.49000329,52.89807679)(191.49500329,52.93307676)(191.49500473,52.96308504)
\curveto(191.49500329,53.00307669)(191.50000328,53.04307665)(191.51000473,53.08308504)
\curveto(191.54000324,53.1930765)(191.56000322,53.30307639)(191.57000473,53.41308504)
\curveto(191.59000319,53.53307616)(191.62500316,53.64807604)(191.67500473,53.75808504)
\curveto(191.81500297,54.09807559)(191.97500281,54.37307532)(192.15500473,54.58308504)
\curveto(192.34500244,54.80307489)(192.61500217,54.98307471)(192.96500473,55.12308504)
\curveto(193.04500174,55.15307454)(193.13000165,55.17307452)(193.22000473,55.18308504)
\curveto(193.31000147,55.20307449)(193.40500138,55.22307447)(193.50500473,55.24308504)
\curveto(193.53500125,55.25307444)(193.59000119,55.25307444)(193.67000473,55.24308504)
\curveto(193.75000103,55.24307445)(193.80000098,55.25307444)(193.82000473,55.27308504)
\curveto(194.3800004,55.28307441)(194.82999995,55.17307452)(195.17000473,54.94308504)
\curveto(195.51999926,54.71307498)(195.779999,54.40807528)(195.95000473,54.02808504)
\curveto(195.98999879,53.93807575)(196.02499876,53.84307585)(196.05500473,53.74308504)
\curveto(196.0849987,53.64307605)(196.10999867,53.54307615)(196.13000473,53.44308504)
\curveto(196.14999863,53.41307628)(196.15499863,53.38307631)(196.14500473,53.35308504)
\curveto(196.14499864,53.32307637)(196.14999863,53.2930764)(196.16000473,53.26308504)
\curveto(196.18999859,53.15307654)(196.20999857,53.02807666)(196.22000473,52.88808504)
\curveto(196.22999855,52.75807693)(196.23999854,52.62307707)(196.25000473,52.48308504)
\lineto(196.25000473,52.31808504)
\curveto(196.25999852,52.25807743)(196.25999852,52.20307749)(196.25000473,52.15308504)
\curveto(196.23999854,52.10307759)(196.23499855,52.05307764)(196.23500473,52.00308504)
\lineto(196.23500473,51.86808504)
\curveto(196.22499856,51.82807786)(196.21999856,51.7880779)(196.22000473,51.74808504)
\curveto(196.22999855,51.70807798)(196.22499856,51.66307803)(196.20500473,51.61308504)
\curveto(196.1849986,51.50307819)(196.16499862,51.39807829)(196.14500473,51.29808504)
\curveto(196.13499865,51.19807849)(196.11499867,51.09807859)(196.08500473,50.99808504)
\curveto(195.95499883,50.63807905)(195.78999899,50.32307937)(195.59000473,50.05308504)
\curveto(195.38999939,49.78307991)(195.11499967,49.57808011)(194.76500473,49.43808504)
\curveto(194.6850001,49.40808028)(194.60000018,49.38308031)(194.51000473,49.36308504)
\lineto(194.24000473,49.30308504)
\curveto(194.19000059,49.2930804)(194.14500064,49.2880804)(194.10500473,49.28808504)
\curveto(194.06500072,49.29808039)(194.02500076,49.29808039)(193.98500473,49.28808504)
\curveto(193.8850009,49.26808042)(193.79000099,49.26808042)(193.70000473,49.28808504)
\moveto(192.86000473,50.68308504)
\curveto(192.90000188,50.61307908)(192.94000184,50.54807914)(192.98000473,50.48808504)
\curveto(193.02000176,50.43807925)(193.07000171,50.3880793)(193.13000473,50.33808504)
\lineto(193.28000473,50.21808504)
\curveto(193.34000144,50.1880795)(193.40500138,50.16307953)(193.47500473,50.14308504)
\curveto(193.51500127,50.12307957)(193.55000123,50.11307958)(193.58000473,50.11308504)
\curveto(193.62000116,50.12307957)(193.66000112,50.11807957)(193.70000473,50.09808504)
\curveto(193.73000105,50.09807959)(193.77000101,50.0930796)(193.82000473,50.08308504)
\curveto(193.87000091,50.08307961)(193.91000087,50.0880796)(193.94000473,50.09808504)
\lineto(194.16500473,50.14308504)
\curveto(194.41500037,50.22307947)(194.60000018,50.34807934)(194.72000473,50.51808504)
\curveto(194.79999998,50.61807907)(194.86999991,50.74807894)(194.93000473,50.90808504)
\curveto(195.00999977,51.0880786)(195.06999971,51.31307838)(195.11000473,51.58308504)
\curveto(195.14999963,51.86307783)(195.16499962,52.14307755)(195.15500473,52.42308504)
\curveto(195.14499964,52.71307698)(195.11499967,52.9880767)(195.06500473,53.24808504)
\curveto(195.01499977,53.50807618)(194.93999984,53.71807597)(194.84000473,53.87808504)
\curveto(194.72000006,54.07807561)(194.57000021,54.22807546)(194.39000473,54.32808504)
\curveto(194.31000047,54.37807531)(194.22000056,54.40807528)(194.12000473,54.41808504)
\curveto(194.02000076,54.43807525)(193.91500087,54.44807524)(193.80500473,54.44808504)
\curveto(193.785001,54.43807525)(193.76000102,54.43307526)(193.73000473,54.43308504)
\curveto(193.71000107,54.44307525)(193.69000109,54.44307525)(193.67000473,54.43308504)
\curveto(193.62000116,54.42307527)(193.57500121,54.41307528)(193.53500473,54.40308504)
\curveto(193.49500129,54.40307529)(193.45500133,54.3930753)(193.41500473,54.37308504)
\curveto(193.23500155,54.2930754)(193.0850017,54.17307552)(192.96500473,54.01308504)
\curveto(192.85500193,53.85307584)(192.76500202,53.67307602)(192.69500473,53.47308504)
\curveto(192.63500215,53.28307641)(192.59000219,53.05807663)(192.56000473,52.79808504)
\curveto(192.54000224,52.53807715)(192.53500225,52.27307742)(192.54500473,52.00308504)
\curveto(192.55500223,51.74307795)(192.5850022,51.4930782)(192.63500473,51.25308504)
\curveto(192.69500209,51.02307867)(192.77000201,50.83307886)(192.86000473,50.68308504)
\moveto(203.66000473,47.69808504)
\curveto(203.66999111,47.64808204)(203.67499111,47.55808213)(203.67500473,47.42808504)
\curveto(203.67499111,47.29808239)(203.66499112,47.20808248)(203.64500473,47.15808504)
\curveto(203.62499116,47.10808258)(203.61999116,47.05308264)(203.63000473,46.99308504)
\curveto(203.63999114,46.94308275)(203.63999114,46.8930828)(203.63000473,46.84308504)
\curveto(203.58999119,46.70308299)(203.55999122,46.56808312)(203.54000473,46.43808504)
\curveto(203.52999125,46.30808338)(203.49999128,46.1880835)(203.45000473,46.07808504)
\curveto(203.30999147,45.72808396)(203.14499164,45.43308426)(202.95500473,45.19308504)
\curveto(202.76499202,44.96308473)(202.49499229,44.77808491)(202.14500473,44.63808504)
\curveto(202.06499272,44.60808508)(201.9799928,44.5880851)(201.89000473,44.57808504)
\curveto(201.79999298,44.55808513)(201.71499307,44.53808515)(201.63500473,44.51808504)
\curveto(201.5849932,44.50808518)(201.53499325,44.50308519)(201.48500473,44.50308504)
\curveto(201.43499335,44.50308519)(201.3849934,44.49808519)(201.33500473,44.48808504)
\curveto(201.30499348,44.47808521)(201.25499353,44.47808521)(201.18500473,44.48808504)
\curveto(201.11499367,44.4880852)(201.06499372,44.4930852)(201.03500473,44.50308504)
\curveto(200.97499381,44.52308517)(200.91499387,44.53308516)(200.85500473,44.53308504)
\curveto(200.80499398,44.52308517)(200.75499403,44.52808516)(200.70500473,44.54808504)
\curveto(200.61499417,44.56808512)(200.52499426,44.5930851)(200.43500473,44.62308504)
\curveto(200.35499443,44.64308505)(200.27499451,44.67308502)(200.19500473,44.71308504)
\curveto(199.87499491,44.85308484)(199.62499516,45.04808464)(199.44500473,45.29808504)
\curveto(199.26499552,45.55808413)(199.11499567,45.86308383)(198.99500473,46.21308504)
\curveto(198.97499581,46.2930834)(198.95999582,46.37808331)(198.95000473,46.46808504)
\curveto(198.93999584,46.55808313)(198.92499586,46.64308305)(198.90500473,46.72308504)
\curveto(198.89499589,46.75308294)(198.88999589,46.78308291)(198.89000473,46.81308504)
\lineto(198.89000473,46.91808504)
\curveto(198.86999591,46.99808269)(198.85999592,47.07808261)(198.86000473,47.15808504)
\lineto(198.86000473,47.29308504)
\curveto(198.83999594,47.3930823)(198.83999594,47.4930822)(198.86000473,47.59308504)
\lineto(198.86000473,47.77308504)
\curveto(198.86999591,47.82308187)(198.87499591,47.86808182)(198.87500473,47.90808504)
\curveto(198.87499591,47.95808173)(198.8799959,48.00308169)(198.89000473,48.04308504)
\curveto(198.89999588,48.08308161)(198.90499588,48.11808157)(198.90500473,48.14808504)
\curveto(198.90499588,48.1880815)(198.90999587,48.22808146)(198.92000473,48.26808504)
\lineto(198.98000473,48.59808504)
\curveto(198.99999578,48.71808097)(199.02999575,48.82808086)(199.07000473,48.92808504)
\curveto(199.20999557,49.25808043)(199.36999541,49.53308016)(199.55000473,49.75308504)
\curveto(199.73999504,49.98307971)(199.99999478,50.16807952)(200.33000473,50.30808504)
\curveto(200.40999437,50.34807934)(200.49499429,50.37307932)(200.58500473,50.38308504)
\lineto(200.88500473,50.44308504)
\lineto(201.02000473,50.44308504)
\curveto(201.06999371,50.45307924)(201.11999366,50.45807923)(201.17000473,50.45808504)
\curveto(201.73999304,50.47807921)(202.19999258,50.37307932)(202.55000473,50.14308504)
\curveto(202.90999187,49.92307977)(203.17499161,49.62308007)(203.34500473,49.24308504)
\curveto(203.39499139,49.14308055)(203.43499135,49.04308065)(203.46500473,48.94308504)
\curveto(203.49499129,48.84308085)(203.52499126,48.73808095)(203.55500473,48.62808504)
\curveto(203.56499122,48.5880811)(203.56999121,48.55308114)(203.57000473,48.52308504)
\curveto(203.56999121,48.50308119)(203.57499121,48.47308122)(203.58500473,48.43308504)
\curveto(203.60499118,48.36308133)(203.61499117,48.2880814)(203.61500473,48.20808504)
\curveto(203.61499117,48.12808156)(203.62499116,48.04808164)(203.64500473,47.96808504)
\curveto(203.64499114,47.91808177)(203.64499114,47.87308182)(203.64500473,47.83308504)
\curveto(203.64499114,47.7930819)(203.64999113,47.74808194)(203.66000473,47.69808504)
\moveto(202.55000473,47.26308504)
\curveto(202.55999222,47.31308238)(202.56499222,47.3880823)(202.56500473,47.48808504)
\curveto(202.57499221,47.5880821)(202.56999221,47.66308203)(202.55000473,47.71308504)
\curveto(202.52999225,47.77308192)(202.52499226,47.82808186)(202.53500473,47.87808504)
\curveto(202.55499223,47.93808175)(202.55499223,47.99808169)(202.53500473,48.05808504)
\curveto(202.52499226,48.0880816)(202.51999226,48.12308157)(202.52000473,48.16308504)
\curveto(202.51999226,48.20308149)(202.51499227,48.24308145)(202.50500473,48.28308504)
\curveto(202.4849923,48.36308133)(202.46499232,48.43808125)(202.44500473,48.50808504)
\curveto(202.43499235,48.5880811)(202.41999236,48.66808102)(202.40000473,48.74808504)
\curveto(202.36999241,48.80808088)(202.34499244,48.86808082)(202.32500473,48.92808504)
\curveto(202.30499248,48.9880807)(202.27499251,49.04808064)(202.23500473,49.10808504)
\curveto(202.13499265,49.27808041)(202.00499278,49.41308028)(201.84500473,49.51308504)
\curveto(201.76499302,49.56308013)(201.66999311,49.59808009)(201.56000473,49.61808504)
\curveto(201.44999333,49.63808005)(201.32499346,49.64808004)(201.18500473,49.64808504)
\curveto(201.16499362,49.63808005)(201.13999364,49.63308006)(201.11000473,49.63308504)
\curveto(201.0799937,49.64308005)(201.04999373,49.64308005)(201.02000473,49.63308504)
\lineto(200.87000473,49.57308504)
\curveto(200.81999396,49.56308013)(200.77499401,49.54808014)(200.73500473,49.52808504)
\curveto(200.54499424,49.41808027)(200.39999438,49.27308042)(200.30000473,49.09308504)
\curveto(200.20999457,48.91308078)(200.12999465,48.70808098)(200.06000473,48.47808504)
\curveto(200.01999476,48.34808134)(199.99999478,48.21308148)(200.00000473,48.07308504)
\curveto(199.99999478,47.94308175)(199.98999479,47.79808189)(199.97000473,47.63808504)
\curveto(199.95999482,47.5880821)(199.94999483,47.52808216)(199.94000473,47.45808504)
\curveto(199.93999484,47.3880823)(199.94999483,47.32808236)(199.97000473,47.27808504)
\lineto(199.97000473,47.11308504)
\lineto(199.97000473,46.93308504)
\curveto(199.9799948,46.88308281)(199.98999479,46.82808286)(200.00000473,46.76808504)
\curveto(200.00999477,46.71808297)(200.01499477,46.66308303)(200.01500473,46.60308504)
\curveto(200.02499476,46.54308315)(200.03999474,46.4880832)(200.06000473,46.43808504)
\curveto(200.10999467,46.24808344)(200.16999461,46.07308362)(200.24000473,45.91308504)
\curveto(200.30999447,45.75308394)(200.41499437,45.62308407)(200.55500473,45.52308504)
\curveto(200.6849941,45.42308427)(200.82499396,45.35308434)(200.97500473,45.31308504)
\curveto(201.00499378,45.30308439)(201.02999375,45.29808439)(201.05000473,45.29808504)
\curveto(201.0799937,45.30808438)(201.10999367,45.30808438)(201.14000473,45.29808504)
\curveto(201.15999362,45.29808439)(201.18999359,45.2930844)(201.23000473,45.28308504)
\curveto(201.26999351,45.28308441)(201.30499348,45.2880844)(201.33500473,45.29808504)
\curveto(201.37499341,45.30808438)(201.41499337,45.31308438)(201.45500473,45.31308504)
\curveto(201.49499329,45.31308438)(201.53499325,45.32308437)(201.57500473,45.34308504)
\curveto(201.81499297,45.42308427)(202.00999277,45.55808413)(202.16000473,45.74808504)
\curveto(202.2799925,45.92808376)(202.36999241,46.13308356)(202.43000473,46.36308504)
\curveto(202.44999233,46.43308326)(202.46499232,46.50308319)(202.47500473,46.57308504)
\curveto(202.4849923,46.65308304)(202.49999228,46.73308296)(202.52000473,46.81308504)
\curveto(202.51999226,46.87308282)(202.52499226,46.91808277)(202.53500473,46.94808504)
\curveto(202.53499225,46.96808272)(202.53499225,46.9930827)(202.53500473,47.02308504)
\curveto(202.53499225,47.06308263)(202.53999224,47.0930826)(202.55000473,47.11308504)
\lineto(202.55000473,47.26308504)
}
}
{
\newrgbcolor{curcolor}{0 0 0}
\pscustom[linestyle=none,fillstyle=solid,fillcolor=curcolor]
{
\newpath
\moveto(101.52148361,227.50413362)
\curveto(103.15147817,227.53412297)(104.20147712,226.97912353)(104.67148361,225.83913362)
\curveto(104.77147655,225.6091249)(104.83647648,225.31912519)(104.86648361,224.96913362)
\curveto(104.90647641,224.62912588)(104.88147644,224.31912619)(104.79148361,224.03913362)
\curveto(104.70147662,223.77912673)(104.58147674,223.55412695)(104.43148361,223.36413362)
\curveto(104.41147691,223.32412718)(104.38647693,223.28912722)(104.35648361,223.25913362)
\curveto(104.32647699,223.23912727)(104.30147702,223.21412729)(104.28148361,223.18413362)
\lineto(104.19148361,223.06413362)
\curveto(104.16147716,223.03412747)(104.12647719,223.0091275)(104.08648361,222.98913362)
\curveto(104.03647728,222.93912757)(103.98147734,222.89412761)(103.92148361,222.85413362)
\curveto(103.87147745,222.81412769)(103.82647749,222.76412774)(103.78648361,222.70413362)
\curveto(103.74647757,222.66412784)(103.73147759,222.61412789)(103.74148361,222.55413362)
\curveto(103.75147757,222.504128)(103.78147754,222.45912805)(103.83148361,222.41913362)
\curveto(103.88147744,222.37912813)(103.93647738,222.33912817)(103.99648361,222.29913362)
\curveto(104.06647725,222.26912824)(104.13147719,222.23912827)(104.19148361,222.20913362)
\curveto(104.25147707,222.17912833)(104.30147702,222.14912836)(104.34148361,222.11913362)
\curveto(104.66147666,221.89912861)(104.9164764,221.58912892)(105.10648361,221.18913362)
\curveto(105.14647617,221.09912941)(105.17647614,221.0041295)(105.19648361,220.90413362)
\curveto(105.22647609,220.81412969)(105.25147607,220.72412978)(105.27148361,220.63413362)
\curveto(105.28147604,220.58412992)(105.28647603,220.53412997)(105.28648361,220.48413362)
\curveto(105.29647602,220.44413006)(105.30647601,220.39913011)(105.31648361,220.34913362)
\curveto(105.32647599,220.29913021)(105.32647599,220.24913026)(105.31648361,220.19913362)
\curveto(105.30647601,220.14913036)(105.31147601,220.09913041)(105.33148361,220.04913362)
\curveto(105.34147598,219.99913051)(105.34647597,219.93913057)(105.34648361,219.86913362)
\curveto(105.34647597,219.79913071)(105.33647598,219.73913077)(105.31648361,219.68913362)
\lineto(105.31648361,219.46413362)
\lineto(105.25648361,219.22413362)
\curveto(105.24647607,219.15413135)(105.23147609,219.08413142)(105.21148361,219.01413362)
\curveto(105.18147614,218.92413158)(105.15147617,218.83913167)(105.12148361,218.75913362)
\curveto(105.10147622,218.67913183)(105.07147625,218.59913191)(105.03148361,218.51913362)
\curveto(105.01147631,218.45913205)(104.98147634,218.39913211)(104.94148361,218.33913362)
\curveto(104.91147641,218.28913222)(104.87647644,218.23913227)(104.83648361,218.18913362)
\curveto(104.63647668,217.87913263)(104.38647693,217.61913289)(104.08648361,217.40913362)
\curveto(103.78647753,217.2091333)(103.44147788,217.04413346)(103.05148361,216.91413362)
\curveto(102.93147839,216.87413363)(102.80147852,216.84913366)(102.66148361,216.83913362)
\curveto(102.53147879,216.81913369)(102.39647892,216.79413371)(102.25648361,216.76413362)
\curveto(102.18647913,216.75413375)(102.1164792,216.74913376)(102.04648361,216.74913362)
\curveto(101.98647933,216.74913376)(101.9214794,216.74413376)(101.85148361,216.73413362)
\curveto(101.81147951,216.72413378)(101.75147957,216.71913379)(101.67148361,216.71913362)
\curveto(101.60147972,216.71913379)(101.55147977,216.72413378)(101.52148361,216.73413362)
\curveto(101.47147985,216.74413376)(101.42647989,216.74913376)(101.38648361,216.74913362)
\lineto(101.26648361,216.74913362)
\curveto(101.16648015,216.76913374)(101.06648025,216.78413372)(100.96648361,216.79413362)
\curveto(100.86648045,216.8041337)(100.77148055,216.81913369)(100.68148361,216.83913362)
\curveto(100.57148075,216.86913364)(100.46148086,216.89413361)(100.35148361,216.91413362)
\curveto(100.25148107,216.94413356)(100.14648117,216.98413352)(100.03648361,217.03413362)
\curveto(99.66648165,217.19413331)(99.35148197,217.39413311)(99.09148361,217.63413362)
\curveto(98.83148249,217.88413262)(98.6214827,218.19413231)(98.46148361,218.56413362)
\curveto(98.4214829,218.65413185)(98.38648293,218.74913176)(98.35648361,218.84913362)
\curveto(98.32648299,218.94913156)(98.29648302,219.05413145)(98.26648361,219.16413362)
\curveto(98.24648307,219.21413129)(98.23648308,219.26413124)(98.23648361,219.31413362)
\curveto(98.23648308,219.37413113)(98.22648309,219.43413107)(98.20648361,219.49413362)
\curveto(98.18648313,219.55413095)(98.17648314,219.63413087)(98.17648361,219.73413362)
\curveto(98.17648314,219.83413067)(98.19148313,219.9091306)(98.22148361,219.95913362)
\curveto(98.23148309,219.98913052)(98.24648307,220.01413049)(98.26648361,220.03413362)
\lineto(98.32648361,220.09413362)
\curveto(98.36648295,220.11413039)(98.42648289,220.12913038)(98.50648361,220.13913362)
\curveto(98.59648272,220.14913036)(98.68648263,220.15413035)(98.77648361,220.15413362)
\curveto(98.86648245,220.15413035)(98.95148237,220.14913036)(99.03148361,220.13913362)
\curveto(99.1214822,220.12913038)(99.18648213,220.11913039)(99.22648361,220.10913362)
\curveto(99.24648207,220.08913042)(99.26648205,220.07413043)(99.28648361,220.06413362)
\curveto(99.30648201,220.06413044)(99.32648199,220.05413045)(99.34648361,220.03413362)
\curveto(99.4164819,219.94413056)(99.45648186,219.82913068)(99.46648361,219.68913362)
\curveto(99.48648183,219.54913096)(99.5164818,219.42413108)(99.55648361,219.31413362)
\lineto(99.70648361,218.95413362)
\curveto(99.75648156,218.84413166)(99.8214815,218.73913177)(99.90148361,218.63913362)
\curveto(99.9214814,218.6091319)(99.94148138,218.58413192)(99.96148361,218.56413362)
\curveto(99.99148133,218.54413196)(100.0164813,218.51913199)(100.03648361,218.48913362)
\curveto(100.07648124,218.42913208)(100.11148121,218.38413212)(100.14148361,218.35413362)
\curveto(100.18148114,218.32413218)(100.2164811,218.29413221)(100.24648361,218.26413362)
\curveto(100.28648103,218.23413227)(100.33148099,218.2041323)(100.38148361,218.17413362)
\curveto(100.47148085,218.11413239)(100.56648075,218.06413244)(100.66648361,218.02413362)
\lineto(100.99648361,217.90413362)
\curveto(101.14648017,217.85413265)(101.34647997,217.82413268)(101.59648361,217.81413362)
\curveto(101.84647947,217.8041327)(102.05647926,217.82413268)(102.22648361,217.87413362)
\curveto(102.30647901,217.89413261)(102.37647894,217.9091326)(102.43648361,217.91913362)
\lineto(102.64648361,217.97913362)
\curveto(102.92647839,218.09913241)(103.16647815,218.24913226)(103.36648361,218.42913362)
\curveto(103.57647774,218.6091319)(103.74147758,218.83913167)(103.86148361,219.11913362)
\curveto(103.89147743,219.18913132)(103.91147741,219.25913125)(103.92148361,219.32913362)
\lineto(103.98148361,219.56913362)
\curveto(104.0214773,219.7091308)(104.03147729,219.86913064)(104.01148361,220.04913362)
\curveto(103.99147733,220.23913027)(103.96147736,220.38913012)(103.92148361,220.49913362)
\curveto(103.79147753,220.87912963)(103.60647771,221.16912934)(103.36648361,221.36913362)
\curveto(103.13647818,221.56912894)(102.82647849,221.72912878)(102.43648361,221.84913362)
\curveto(102.32647899,221.87912863)(102.20647911,221.89912861)(102.07648361,221.90913362)
\curveto(101.95647936,221.91912859)(101.83147949,221.92412858)(101.70148361,221.92413362)
\curveto(101.54147978,221.92412858)(101.40147992,221.92912858)(101.28148361,221.93913362)
\curveto(101.16148016,221.94912856)(101.07648024,222.0091285)(101.02648361,222.11913362)
\curveto(101.00648031,222.14912836)(100.99648032,222.18412832)(100.99648361,222.22413362)
\lineto(100.99648361,222.35913362)
\curveto(100.98648033,222.45912805)(100.98648033,222.55412795)(100.99648361,222.64413362)
\curveto(101.0164803,222.73412777)(101.05648026,222.79912771)(101.11648361,222.83913362)
\curveto(101.15648016,222.86912764)(101.19648012,222.88912762)(101.23648361,222.89913362)
\curveto(101.28648003,222.9091276)(101.34147998,222.91912759)(101.40148361,222.92913362)
\curveto(101.4214799,222.93912757)(101.44647987,222.93912757)(101.47648361,222.92913362)
\curveto(101.50647981,222.92912758)(101.53147979,222.93412757)(101.55148361,222.94413362)
\lineto(101.68648361,222.94413362)
\curveto(101.79647952,222.96412754)(101.89647942,222.97412753)(101.98648361,222.97413362)
\curveto(102.08647923,222.98412752)(102.18147914,223.0041275)(102.27148361,223.03413362)
\curveto(102.59147873,223.14412736)(102.84647847,223.28912722)(103.03648361,223.46913362)
\curveto(103.22647809,223.64912686)(103.37647794,223.89912661)(103.48648361,224.21913362)
\curveto(103.5164778,224.31912619)(103.53647778,224.44412606)(103.54648361,224.59413362)
\curveto(103.56647775,224.75412575)(103.56147776,224.89912561)(103.53148361,225.02913362)
\curveto(103.51147781,225.09912541)(103.49147783,225.16412534)(103.47148361,225.22413362)
\curveto(103.46147786,225.29412521)(103.44147788,225.35912515)(103.41148361,225.41913362)
\curveto(103.31147801,225.65912485)(103.16647815,225.84912466)(102.97648361,225.98913362)
\curveto(102.78647853,226.12912438)(102.56147876,226.23912427)(102.30148361,226.31913362)
\curveto(102.24147908,226.33912417)(102.18147914,226.34912416)(102.12148361,226.34913362)
\curveto(102.06147926,226.34912416)(101.99647932,226.35912415)(101.92648361,226.37913362)
\curveto(101.84647947,226.39912411)(101.75147957,226.4091241)(101.64148361,226.40913362)
\curveto(101.53147979,226.4091241)(101.43647988,226.39912411)(101.35648361,226.37913362)
\curveto(101.30648001,226.35912415)(101.25648006,226.34912416)(101.20648361,226.34913362)
\curveto(101.16648015,226.34912416)(101.1214802,226.33912417)(101.07148361,226.31913362)
\curveto(100.89148043,226.26912424)(100.7214806,226.19412431)(100.56148361,226.09413362)
\curveto(100.41148091,226.0041245)(100.28148104,225.88912462)(100.17148361,225.74913362)
\curveto(100.08148124,225.62912488)(100.00148132,225.49912501)(99.93148361,225.35913362)
\curveto(99.86148146,225.21912529)(99.79648152,225.06412544)(99.73648361,224.89413362)
\curveto(99.70648161,224.78412572)(99.68648163,224.66412584)(99.67648361,224.53413362)
\curveto(99.66648165,224.41412609)(99.63148169,224.31412619)(99.57148361,224.23413362)
\curveto(99.55148177,224.19412631)(99.49148183,224.15412635)(99.39148361,224.11413362)
\curveto(99.35148197,224.1041264)(99.29148203,224.09412641)(99.21148361,224.08413362)
\lineto(98.95648361,224.08413362)
\curveto(98.86648245,224.09412641)(98.78148254,224.1041264)(98.70148361,224.11413362)
\curveto(98.63148269,224.12412638)(98.58148274,224.13912637)(98.55148361,224.15913362)
\curveto(98.51148281,224.18912632)(98.47648284,224.24412626)(98.44648361,224.32413362)
\curveto(98.4164829,224.4041261)(98.41148291,224.48912602)(98.43148361,224.57913362)
\curveto(98.44148288,224.62912588)(98.44648287,224.67912583)(98.44648361,224.72913362)
\lineto(98.47648361,224.90913362)
\curveto(98.50648281,225.0091255)(98.53148279,225.1091254)(98.55148361,225.20913362)
\curveto(98.58148274,225.3091252)(98.6164827,225.39912511)(98.65648361,225.47913362)
\curveto(98.70648261,225.58912492)(98.75148257,225.69412481)(98.79148361,225.79413362)
\curveto(98.83148249,225.9041246)(98.88148244,226.0091245)(98.94148361,226.10913362)
\curveto(99.27148205,226.64912386)(99.74148158,227.04412346)(100.35148361,227.29413362)
\curveto(100.47148085,227.34412316)(100.59648072,227.37912313)(100.72648361,227.39913362)
\curveto(100.86648045,227.41912309)(101.00648031,227.44412306)(101.14648361,227.47413362)
\curveto(101.20648011,227.48412302)(101.26648005,227.48912302)(101.32648361,227.48913362)
\curveto(101.39647992,227.48912302)(101.46147986,227.49412301)(101.52148361,227.50413362)
}
}
{
\newrgbcolor{curcolor}{0 0 0}
\pscustom[linestyle=none,fillstyle=solid,fillcolor=curcolor]
{
\newpath
\moveto(109.87109298,227.50413362)
\curveto(111.50108754,227.53412297)(112.55108649,226.97912353)(113.02109298,225.83913362)
\curveto(113.12108592,225.6091249)(113.18608586,225.31912519)(113.21609298,224.96913362)
\curveto(113.25608579,224.62912588)(113.23108581,224.31912619)(113.14109298,224.03913362)
\curveto(113.05108599,223.77912673)(112.93108611,223.55412695)(112.78109298,223.36413362)
\curveto(112.76108628,223.32412718)(112.73608631,223.28912722)(112.70609298,223.25913362)
\curveto(112.67608637,223.23912727)(112.65108639,223.21412729)(112.63109298,223.18413362)
\lineto(112.54109298,223.06413362)
\curveto(112.51108653,223.03412747)(112.47608657,223.0091275)(112.43609298,222.98913362)
\curveto(112.38608666,222.93912757)(112.33108671,222.89412761)(112.27109298,222.85413362)
\curveto(112.22108682,222.81412769)(112.17608687,222.76412774)(112.13609298,222.70413362)
\curveto(112.09608695,222.66412784)(112.08108696,222.61412789)(112.09109298,222.55413362)
\curveto(112.10108694,222.504128)(112.13108691,222.45912805)(112.18109298,222.41913362)
\curveto(112.23108681,222.37912813)(112.28608676,222.33912817)(112.34609298,222.29913362)
\curveto(112.41608663,222.26912824)(112.48108656,222.23912827)(112.54109298,222.20913362)
\curveto(112.60108644,222.17912833)(112.65108639,222.14912836)(112.69109298,222.11913362)
\curveto(113.01108603,221.89912861)(113.26608578,221.58912892)(113.45609298,221.18913362)
\curveto(113.49608555,221.09912941)(113.52608552,221.0041295)(113.54609298,220.90413362)
\curveto(113.57608547,220.81412969)(113.60108544,220.72412978)(113.62109298,220.63413362)
\curveto(113.63108541,220.58412992)(113.63608541,220.53412997)(113.63609298,220.48413362)
\curveto(113.6460854,220.44413006)(113.65608539,220.39913011)(113.66609298,220.34913362)
\curveto(113.67608537,220.29913021)(113.67608537,220.24913026)(113.66609298,220.19913362)
\curveto(113.65608539,220.14913036)(113.66108538,220.09913041)(113.68109298,220.04913362)
\curveto(113.69108535,219.99913051)(113.69608535,219.93913057)(113.69609298,219.86913362)
\curveto(113.69608535,219.79913071)(113.68608536,219.73913077)(113.66609298,219.68913362)
\lineto(113.66609298,219.46413362)
\lineto(113.60609298,219.22413362)
\curveto(113.59608545,219.15413135)(113.58108546,219.08413142)(113.56109298,219.01413362)
\curveto(113.53108551,218.92413158)(113.50108554,218.83913167)(113.47109298,218.75913362)
\curveto(113.45108559,218.67913183)(113.42108562,218.59913191)(113.38109298,218.51913362)
\curveto(113.36108568,218.45913205)(113.33108571,218.39913211)(113.29109298,218.33913362)
\curveto(113.26108578,218.28913222)(113.22608582,218.23913227)(113.18609298,218.18913362)
\curveto(112.98608606,217.87913263)(112.73608631,217.61913289)(112.43609298,217.40913362)
\curveto(112.13608691,217.2091333)(111.79108725,217.04413346)(111.40109298,216.91413362)
\curveto(111.28108776,216.87413363)(111.15108789,216.84913366)(111.01109298,216.83913362)
\curveto(110.88108816,216.81913369)(110.7460883,216.79413371)(110.60609298,216.76413362)
\curveto(110.53608851,216.75413375)(110.46608858,216.74913376)(110.39609298,216.74913362)
\curveto(110.33608871,216.74913376)(110.27108877,216.74413376)(110.20109298,216.73413362)
\curveto(110.16108888,216.72413378)(110.10108894,216.71913379)(110.02109298,216.71913362)
\curveto(109.95108909,216.71913379)(109.90108914,216.72413378)(109.87109298,216.73413362)
\curveto(109.82108922,216.74413376)(109.77608927,216.74913376)(109.73609298,216.74913362)
\lineto(109.61609298,216.74913362)
\curveto(109.51608953,216.76913374)(109.41608963,216.78413372)(109.31609298,216.79413362)
\curveto(109.21608983,216.8041337)(109.12108992,216.81913369)(109.03109298,216.83913362)
\curveto(108.92109012,216.86913364)(108.81109023,216.89413361)(108.70109298,216.91413362)
\curveto(108.60109044,216.94413356)(108.49609055,216.98413352)(108.38609298,217.03413362)
\curveto(108.01609103,217.19413331)(107.70109134,217.39413311)(107.44109298,217.63413362)
\curveto(107.18109186,217.88413262)(106.97109207,218.19413231)(106.81109298,218.56413362)
\curveto(106.77109227,218.65413185)(106.73609231,218.74913176)(106.70609298,218.84913362)
\curveto(106.67609237,218.94913156)(106.6460924,219.05413145)(106.61609298,219.16413362)
\curveto(106.59609245,219.21413129)(106.58609246,219.26413124)(106.58609298,219.31413362)
\curveto(106.58609246,219.37413113)(106.57609247,219.43413107)(106.55609298,219.49413362)
\curveto(106.53609251,219.55413095)(106.52609252,219.63413087)(106.52609298,219.73413362)
\curveto(106.52609252,219.83413067)(106.5410925,219.9091306)(106.57109298,219.95913362)
\curveto(106.58109246,219.98913052)(106.59609245,220.01413049)(106.61609298,220.03413362)
\lineto(106.67609298,220.09413362)
\curveto(106.71609233,220.11413039)(106.77609227,220.12913038)(106.85609298,220.13913362)
\curveto(106.9460921,220.14913036)(107.03609201,220.15413035)(107.12609298,220.15413362)
\curveto(107.21609183,220.15413035)(107.30109174,220.14913036)(107.38109298,220.13913362)
\curveto(107.47109157,220.12913038)(107.53609151,220.11913039)(107.57609298,220.10913362)
\curveto(107.59609145,220.08913042)(107.61609143,220.07413043)(107.63609298,220.06413362)
\curveto(107.65609139,220.06413044)(107.67609137,220.05413045)(107.69609298,220.03413362)
\curveto(107.76609128,219.94413056)(107.80609124,219.82913068)(107.81609298,219.68913362)
\curveto(107.83609121,219.54913096)(107.86609118,219.42413108)(107.90609298,219.31413362)
\lineto(108.05609298,218.95413362)
\curveto(108.10609094,218.84413166)(108.17109087,218.73913177)(108.25109298,218.63913362)
\curveto(108.27109077,218.6091319)(108.29109075,218.58413192)(108.31109298,218.56413362)
\curveto(108.3410907,218.54413196)(108.36609068,218.51913199)(108.38609298,218.48913362)
\curveto(108.42609062,218.42913208)(108.46109058,218.38413212)(108.49109298,218.35413362)
\curveto(108.53109051,218.32413218)(108.56609048,218.29413221)(108.59609298,218.26413362)
\curveto(108.63609041,218.23413227)(108.68109036,218.2041323)(108.73109298,218.17413362)
\curveto(108.82109022,218.11413239)(108.91609013,218.06413244)(109.01609298,218.02413362)
\lineto(109.34609298,217.90413362)
\curveto(109.49608955,217.85413265)(109.69608935,217.82413268)(109.94609298,217.81413362)
\curveto(110.19608885,217.8041327)(110.40608864,217.82413268)(110.57609298,217.87413362)
\curveto(110.65608839,217.89413261)(110.72608832,217.9091326)(110.78609298,217.91913362)
\lineto(110.99609298,217.97913362)
\curveto(111.27608777,218.09913241)(111.51608753,218.24913226)(111.71609298,218.42913362)
\curveto(111.92608712,218.6091319)(112.09108695,218.83913167)(112.21109298,219.11913362)
\curveto(112.2410868,219.18913132)(112.26108678,219.25913125)(112.27109298,219.32913362)
\lineto(112.33109298,219.56913362)
\curveto(112.37108667,219.7091308)(112.38108666,219.86913064)(112.36109298,220.04913362)
\curveto(112.3410867,220.23913027)(112.31108673,220.38913012)(112.27109298,220.49913362)
\curveto(112.1410869,220.87912963)(111.95608709,221.16912934)(111.71609298,221.36913362)
\curveto(111.48608756,221.56912894)(111.17608787,221.72912878)(110.78609298,221.84913362)
\curveto(110.67608837,221.87912863)(110.55608849,221.89912861)(110.42609298,221.90913362)
\curveto(110.30608874,221.91912859)(110.18108886,221.92412858)(110.05109298,221.92413362)
\curveto(109.89108915,221.92412858)(109.75108929,221.92912858)(109.63109298,221.93913362)
\curveto(109.51108953,221.94912856)(109.42608962,222.0091285)(109.37609298,222.11913362)
\curveto(109.35608969,222.14912836)(109.3460897,222.18412832)(109.34609298,222.22413362)
\lineto(109.34609298,222.35913362)
\curveto(109.33608971,222.45912805)(109.33608971,222.55412795)(109.34609298,222.64413362)
\curveto(109.36608968,222.73412777)(109.40608964,222.79912771)(109.46609298,222.83913362)
\curveto(109.50608954,222.86912764)(109.5460895,222.88912762)(109.58609298,222.89913362)
\curveto(109.63608941,222.9091276)(109.69108935,222.91912759)(109.75109298,222.92913362)
\curveto(109.77108927,222.93912757)(109.79608925,222.93912757)(109.82609298,222.92913362)
\curveto(109.85608919,222.92912758)(109.88108916,222.93412757)(109.90109298,222.94413362)
\lineto(110.03609298,222.94413362)
\curveto(110.1460889,222.96412754)(110.2460888,222.97412753)(110.33609298,222.97413362)
\curveto(110.43608861,222.98412752)(110.53108851,223.0041275)(110.62109298,223.03413362)
\curveto(110.9410881,223.14412736)(111.19608785,223.28912722)(111.38609298,223.46913362)
\curveto(111.57608747,223.64912686)(111.72608732,223.89912661)(111.83609298,224.21913362)
\curveto(111.86608718,224.31912619)(111.88608716,224.44412606)(111.89609298,224.59413362)
\curveto(111.91608713,224.75412575)(111.91108713,224.89912561)(111.88109298,225.02913362)
\curveto(111.86108718,225.09912541)(111.8410872,225.16412534)(111.82109298,225.22413362)
\curveto(111.81108723,225.29412521)(111.79108725,225.35912515)(111.76109298,225.41913362)
\curveto(111.66108738,225.65912485)(111.51608753,225.84912466)(111.32609298,225.98913362)
\curveto(111.13608791,226.12912438)(110.91108813,226.23912427)(110.65109298,226.31913362)
\curveto(110.59108845,226.33912417)(110.53108851,226.34912416)(110.47109298,226.34913362)
\curveto(110.41108863,226.34912416)(110.3460887,226.35912415)(110.27609298,226.37913362)
\curveto(110.19608885,226.39912411)(110.10108894,226.4091241)(109.99109298,226.40913362)
\curveto(109.88108916,226.4091241)(109.78608926,226.39912411)(109.70609298,226.37913362)
\curveto(109.65608939,226.35912415)(109.60608944,226.34912416)(109.55609298,226.34913362)
\curveto(109.51608953,226.34912416)(109.47108957,226.33912417)(109.42109298,226.31913362)
\curveto(109.2410898,226.26912424)(109.07108997,226.19412431)(108.91109298,226.09413362)
\curveto(108.76109028,226.0041245)(108.63109041,225.88912462)(108.52109298,225.74913362)
\curveto(108.43109061,225.62912488)(108.35109069,225.49912501)(108.28109298,225.35913362)
\curveto(108.21109083,225.21912529)(108.1460909,225.06412544)(108.08609298,224.89413362)
\curveto(108.05609099,224.78412572)(108.03609101,224.66412584)(108.02609298,224.53413362)
\curveto(108.01609103,224.41412609)(107.98109106,224.31412619)(107.92109298,224.23413362)
\curveto(107.90109114,224.19412631)(107.8410912,224.15412635)(107.74109298,224.11413362)
\curveto(107.70109134,224.1041264)(107.6410914,224.09412641)(107.56109298,224.08413362)
\lineto(107.30609298,224.08413362)
\curveto(107.21609183,224.09412641)(107.13109191,224.1041264)(107.05109298,224.11413362)
\curveto(106.98109206,224.12412638)(106.93109211,224.13912637)(106.90109298,224.15913362)
\curveto(106.86109218,224.18912632)(106.82609222,224.24412626)(106.79609298,224.32413362)
\curveto(106.76609228,224.4041261)(106.76109228,224.48912602)(106.78109298,224.57913362)
\curveto(106.79109225,224.62912588)(106.79609225,224.67912583)(106.79609298,224.72913362)
\lineto(106.82609298,224.90913362)
\curveto(106.85609219,225.0091255)(106.88109216,225.1091254)(106.90109298,225.20913362)
\curveto(106.93109211,225.3091252)(106.96609208,225.39912511)(107.00609298,225.47913362)
\curveto(107.05609199,225.58912492)(107.10109194,225.69412481)(107.14109298,225.79413362)
\curveto(107.18109186,225.9041246)(107.23109181,226.0091245)(107.29109298,226.10913362)
\curveto(107.62109142,226.64912386)(108.09109095,227.04412346)(108.70109298,227.29413362)
\curveto(108.82109022,227.34412316)(108.9460901,227.37912313)(109.07609298,227.39913362)
\curveto(109.21608983,227.41912309)(109.35608969,227.44412306)(109.49609298,227.47413362)
\curveto(109.55608949,227.48412302)(109.61608943,227.48912302)(109.67609298,227.48913362)
\curveto(109.7460893,227.48912302)(109.81108923,227.49412301)(109.87109298,227.50413362)
}
}
{
\newrgbcolor{curcolor}{0 0 0}
\pscustom[linestyle=none,fillstyle=solid,fillcolor=curcolor]
{
\newpath
\moveto(116.06070236,218.53413362)
\lineto(116.36070236,218.53413362)
\curveto(116.4707003,218.54413196)(116.57570019,218.54413196)(116.67570236,218.53413362)
\curveto(116.78569998,218.53413197)(116.88569988,218.52413198)(116.97570236,218.50413362)
\curveto(117.0656997,218.49413201)(117.13569963,218.46913204)(117.18570236,218.42913362)
\curveto(117.20569956,218.4091321)(117.22069955,218.37913213)(117.23070236,218.33913362)
\curveto(117.25069952,218.29913221)(117.2706995,218.25413225)(117.29070236,218.20413362)
\lineto(117.29070236,218.12913362)
\curveto(117.30069947,218.07913243)(117.30069947,218.02413248)(117.29070236,217.96413362)
\lineto(117.29070236,217.81413362)
\lineto(117.29070236,217.33413362)
\curveto(117.29069948,217.16413334)(117.25069952,217.04413346)(117.17070236,216.97413362)
\curveto(117.10069967,216.92413358)(117.01069976,216.89913361)(116.90070236,216.89913362)
\lineto(116.57070236,216.89913362)
\lineto(116.12070236,216.89913362)
\curveto(115.9707008,216.89913361)(115.85570091,216.92913358)(115.77570236,216.98913362)
\curveto(115.73570103,217.01913349)(115.70570106,217.06913344)(115.68570236,217.13913362)
\curveto(115.6657011,217.21913329)(115.65070112,217.3041332)(115.64070236,217.39413362)
\lineto(115.64070236,217.67913362)
\curveto(115.65070112,217.77913273)(115.65570111,217.86413264)(115.65570236,217.93413362)
\lineto(115.65570236,218.12913362)
\curveto(115.65570111,218.18913232)(115.6657011,218.24413226)(115.68570236,218.29413362)
\curveto(115.72570104,218.4041321)(115.79570097,218.47413203)(115.89570236,218.50413362)
\curveto(115.92570084,218.504132)(115.98070079,218.51413199)(116.06070236,218.53413362)
}
}
{
\newrgbcolor{curcolor}{0 0 0}
\pscustom[linestyle=none,fillstyle=solid,fillcolor=curcolor]
{
\newpath
\moveto(122.38085861,227.50413362)
\curveto(124.01085317,227.53412297)(125.06085212,226.97912353)(125.53085861,225.83913362)
\curveto(125.63085155,225.6091249)(125.69585148,225.31912519)(125.72585861,224.96913362)
\curveto(125.76585141,224.62912588)(125.74085144,224.31912619)(125.65085861,224.03913362)
\curveto(125.56085162,223.77912673)(125.44085174,223.55412695)(125.29085861,223.36413362)
\curveto(125.27085191,223.32412718)(125.24585193,223.28912722)(125.21585861,223.25913362)
\curveto(125.18585199,223.23912727)(125.16085202,223.21412729)(125.14085861,223.18413362)
\lineto(125.05085861,223.06413362)
\curveto(125.02085216,223.03412747)(124.98585219,223.0091275)(124.94585861,222.98913362)
\curveto(124.89585228,222.93912757)(124.84085234,222.89412761)(124.78085861,222.85413362)
\curveto(124.73085245,222.81412769)(124.68585249,222.76412774)(124.64585861,222.70413362)
\curveto(124.60585257,222.66412784)(124.59085259,222.61412789)(124.60085861,222.55413362)
\curveto(124.61085257,222.504128)(124.64085254,222.45912805)(124.69085861,222.41913362)
\curveto(124.74085244,222.37912813)(124.79585238,222.33912817)(124.85585861,222.29913362)
\curveto(124.92585225,222.26912824)(124.99085219,222.23912827)(125.05085861,222.20913362)
\curveto(125.11085207,222.17912833)(125.16085202,222.14912836)(125.20085861,222.11913362)
\curveto(125.52085166,221.89912861)(125.7758514,221.58912892)(125.96585861,221.18913362)
\curveto(126.00585117,221.09912941)(126.03585114,221.0041295)(126.05585861,220.90413362)
\curveto(126.08585109,220.81412969)(126.11085107,220.72412978)(126.13085861,220.63413362)
\curveto(126.14085104,220.58412992)(126.14585103,220.53412997)(126.14585861,220.48413362)
\curveto(126.15585102,220.44413006)(126.16585101,220.39913011)(126.17585861,220.34913362)
\curveto(126.18585099,220.29913021)(126.18585099,220.24913026)(126.17585861,220.19913362)
\curveto(126.16585101,220.14913036)(126.17085101,220.09913041)(126.19085861,220.04913362)
\curveto(126.20085098,219.99913051)(126.20585097,219.93913057)(126.20585861,219.86913362)
\curveto(126.20585097,219.79913071)(126.19585098,219.73913077)(126.17585861,219.68913362)
\lineto(126.17585861,219.46413362)
\lineto(126.11585861,219.22413362)
\curveto(126.10585107,219.15413135)(126.09085109,219.08413142)(126.07085861,219.01413362)
\curveto(126.04085114,218.92413158)(126.01085117,218.83913167)(125.98085861,218.75913362)
\curveto(125.96085122,218.67913183)(125.93085125,218.59913191)(125.89085861,218.51913362)
\curveto(125.87085131,218.45913205)(125.84085134,218.39913211)(125.80085861,218.33913362)
\curveto(125.77085141,218.28913222)(125.73585144,218.23913227)(125.69585861,218.18913362)
\curveto(125.49585168,217.87913263)(125.24585193,217.61913289)(124.94585861,217.40913362)
\curveto(124.64585253,217.2091333)(124.30085288,217.04413346)(123.91085861,216.91413362)
\curveto(123.79085339,216.87413363)(123.66085352,216.84913366)(123.52085861,216.83913362)
\curveto(123.39085379,216.81913369)(123.25585392,216.79413371)(123.11585861,216.76413362)
\curveto(123.04585413,216.75413375)(122.9758542,216.74913376)(122.90585861,216.74913362)
\curveto(122.84585433,216.74913376)(122.7808544,216.74413376)(122.71085861,216.73413362)
\curveto(122.67085451,216.72413378)(122.61085457,216.71913379)(122.53085861,216.71913362)
\curveto(122.46085472,216.71913379)(122.41085477,216.72413378)(122.38085861,216.73413362)
\curveto(122.33085485,216.74413376)(122.28585489,216.74913376)(122.24585861,216.74913362)
\lineto(122.12585861,216.74913362)
\curveto(122.02585515,216.76913374)(121.92585525,216.78413372)(121.82585861,216.79413362)
\curveto(121.72585545,216.8041337)(121.63085555,216.81913369)(121.54085861,216.83913362)
\curveto(121.43085575,216.86913364)(121.32085586,216.89413361)(121.21085861,216.91413362)
\curveto(121.11085607,216.94413356)(121.00585617,216.98413352)(120.89585861,217.03413362)
\curveto(120.52585665,217.19413331)(120.21085697,217.39413311)(119.95085861,217.63413362)
\curveto(119.69085749,217.88413262)(119.4808577,218.19413231)(119.32085861,218.56413362)
\curveto(119.2808579,218.65413185)(119.24585793,218.74913176)(119.21585861,218.84913362)
\curveto(119.18585799,218.94913156)(119.15585802,219.05413145)(119.12585861,219.16413362)
\curveto(119.10585807,219.21413129)(119.09585808,219.26413124)(119.09585861,219.31413362)
\curveto(119.09585808,219.37413113)(119.08585809,219.43413107)(119.06585861,219.49413362)
\curveto(119.04585813,219.55413095)(119.03585814,219.63413087)(119.03585861,219.73413362)
\curveto(119.03585814,219.83413067)(119.05085813,219.9091306)(119.08085861,219.95913362)
\curveto(119.09085809,219.98913052)(119.10585807,220.01413049)(119.12585861,220.03413362)
\lineto(119.18585861,220.09413362)
\curveto(119.22585795,220.11413039)(119.28585789,220.12913038)(119.36585861,220.13913362)
\curveto(119.45585772,220.14913036)(119.54585763,220.15413035)(119.63585861,220.15413362)
\curveto(119.72585745,220.15413035)(119.81085737,220.14913036)(119.89085861,220.13913362)
\curveto(119.9808572,220.12913038)(120.04585713,220.11913039)(120.08585861,220.10913362)
\curveto(120.10585707,220.08913042)(120.12585705,220.07413043)(120.14585861,220.06413362)
\curveto(120.16585701,220.06413044)(120.18585699,220.05413045)(120.20585861,220.03413362)
\curveto(120.2758569,219.94413056)(120.31585686,219.82913068)(120.32585861,219.68913362)
\curveto(120.34585683,219.54913096)(120.3758568,219.42413108)(120.41585861,219.31413362)
\lineto(120.56585861,218.95413362)
\curveto(120.61585656,218.84413166)(120.6808565,218.73913177)(120.76085861,218.63913362)
\curveto(120.7808564,218.6091319)(120.80085638,218.58413192)(120.82085861,218.56413362)
\curveto(120.85085633,218.54413196)(120.8758563,218.51913199)(120.89585861,218.48913362)
\curveto(120.93585624,218.42913208)(120.97085621,218.38413212)(121.00085861,218.35413362)
\curveto(121.04085614,218.32413218)(121.0758561,218.29413221)(121.10585861,218.26413362)
\curveto(121.14585603,218.23413227)(121.19085599,218.2041323)(121.24085861,218.17413362)
\curveto(121.33085585,218.11413239)(121.42585575,218.06413244)(121.52585861,218.02413362)
\lineto(121.85585861,217.90413362)
\curveto(122.00585517,217.85413265)(122.20585497,217.82413268)(122.45585861,217.81413362)
\curveto(122.70585447,217.8041327)(122.91585426,217.82413268)(123.08585861,217.87413362)
\curveto(123.16585401,217.89413261)(123.23585394,217.9091326)(123.29585861,217.91913362)
\lineto(123.50585861,217.97913362)
\curveto(123.78585339,218.09913241)(124.02585315,218.24913226)(124.22585861,218.42913362)
\curveto(124.43585274,218.6091319)(124.60085258,218.83913167)(124.72085861,219.11913362)
\curveto(124.75085243,219.18913132)(124.77085241,219.25913125)(124.78085861,219.32913362)
\lineto(124.84085861,219.56913362)
\curveto(124.8808523,219.7091308)(124.89085229,219.86913064)(124.87085861,220.04913362)
\curveto(124.85085233,220.23913027)(124.82085236,220.38913012)(124.78085861,220.49913362)
\curveto(124.65085253,220.87912963)(124.46585271,221.16912934)(124.22585861,221.36913362)
\curveto(123.99585318,221.56912894)(123.68585349,221.72912878)(123.29585861,221.84913362)
\curveto(123.18585399,221.87912863)(123.06585411,221.89912861)(122.93585861,221.90913362)
\curveto(122.81585436,221.91912859)(122.69085449,221.92412858)(122.56085861,221.92413362)
\curveto(122.40085478,221.92412858)(122.26085492,221.92912858)(122.14085861,221.93913362)
\curveto(122.02085516,221.94912856)(121.93585524,222.0091285)(121.88585861,222.11913362)
\curveto(121.86585531,222.14912836)(121.85585532,222.18412832)(121.85585861,222.22413362)
\lineto(121.85585861,222.35913362)
\curveto(121.84585533,222.45912805)(121.84585533,222.55412795)(121.85585861,222.64413362)
\curveto(121.8758553,222.73412777)(121.91585526,222.79912771)(121.97585861,222.83913362)
\curveto(122.01585516,222.86912764)(122.05585512,222.88912762)(122.09585861,222.89913362)
\curveto(122.14585503,222.9091276)(122.20085498,222.91912759)(122.26085861,222.92913362)
\curveto(122.2808549,222.93912757)(122.30585487,222.93912757)(122.33585861,222.92913362)
\curveto(122.36585481,222.92912758)(122.39085479,222.93412757)(122.41085861,222.94413362)
\lineto(122.54585861,222.94413362)
\curveto(122.65585452,222.96412754)(122.75585442,222.97412753)(122.84585861,222.97413362)
\curveto(122.94585423,222.98412752)(123.04085414,223.0041275)(123.13085861,223.03413362)
\curveto(123.45085373,223.14412736)(123.70585347,223.28912722)(123.89585861,223.46913362)
\curveto(124.08585309,223.64912686)(124.23585294,223.89912661)(124.34585861,224.21913362)
\curveto(124.3758528,224.31912619)(124.39585278,224.44412606)(124.40585861,224.59413362)
\curveto(124.42585275,224.75412575)(124.42085276,224.89912561)(124.39085861,225.02913362)
\curveto(124.37085281,225.09912541)(124.35085283,225.16412534)(124.33085861,225.22413362)
\curveto(124.32085286,225.29412521)(124.30085288,225.35912515)(124.27085861,225.41913362)
\curveto(124.17085301,225.65912485)(124.02585315,225.84912466)(123.83585861,225.98913362)
\curveto(123.64585353,226.12912438)(123.42085376,226.23912427)(123.16085861,226.31913362)
\curveto(123.10085408,226.33912417)(123.04085414,226.34912416)(122.98085861,226.34913362)
\curveto(122.92085426,226.34912416)(122.85585432,226.35912415)(122.78585861,226.37913362)
\curveto(122.70585447,226.39912411)(122.61085457,226.4091241)(122.50085861,226.40913362)
\curveto(122.39085479,226.4091241)(122.29585488,226.39912411)(122.21585861,226.37913362)
\curveto(122.16585501,226.35912415)(122.11585506,226.34912416)(122.06585861,226.34913362)
\curveto(122.02585515,226.34912416)(121.9808552,226.33912417)(121.93085861,226.31913362)
\curveto(121.75085543,226.26912424)(121.5808556,226.19412431)(121.42085861,226.09413362)
\curveto(121.27085591,226.0041245)(121.14085604,225.88912462)(121.03085861,225.74913362)
\curveto(120.94085624,225.62912488)(120.86085632,225.49912501)(120.79085861,225.35913362)
\curveto(120.72085646,225.21912529)(120.65585652,225.06412544)(120.59585861,224.89413362)
\curveto(120.56585661,224.78412572)(120.54585663,224.66412584)(120.53585861,224.53413362)
\curveto(120.52585665,224.41412609)(120.49085669,224.31412619)(120.43085861,224.23413362)
\curveto(120.41085677,224.19412631)(120.35085683,224.15412635)(120.25085861,224.11413362)
\curveto(120.21085697,224.1041264)(120.15085703,224.09412641)(120.07085861,224.08413362)
\lineto(119.81585861,224.08413362)
\curveto(119.72585745,224.09412641)(119.64085754,224.1041264)(119.56085861,224.11413362)
\curveto(119.49085769,224.12412638)(119.44085774,224.13912637)(119.41085861,224.15913362)
\curveto(119.37085781,224.18912632)(119.33585784,224.24412626)(119.30585861,224.32413362)
\curveto(119.2758579,224.4041261)(119.27085791,224.48912602)(119.29085861,224.57913362)
\curveto(119.30085788,224.62912588)(119.30585787,224.67912583)(119.30585861,224.72913362)
\lineto(119.33585861,224.90913362)
\curveto(119.36585781,225.0091255)(119.39085779,225.1091254)(119.41085861,225.20913362)
\curveto(119.44085774,225.3091252)(119.4758577,225.39912511)(119.51585861,225.47913362)
\curveto(119.56585761,225.58912492)(119.61085757,225.69412481)(119.65085861,225.79413362)
\curveto(119.69085749,225.9041246)(119.74085744,226.0091245)(119.80085861,226.10913362)
\curveto(120.13085705,226.64912386)(120.60085658,227.04412346)(121.21085861,227.29413362)
\curveto(121.33085585,227.34412316)(121.45585572,227.37912313)(121.58585861,227.39913362)
\curveto(121.72585545,227.41912309)(121.86585531,227.44412306)(122.00585861,227.47413362)
\curveto(122.06585511,227.48412302)(122.12585505,227.48912302)(122.18585861,227.48913362)
\curveto(122.25585492,227.48912302)(122.32085486,227.49412301)(122.38085861,227.50413362)
}
}
{
\newrgbcolor{curcolor}{0 0 0}
\pscustom[linestyle=none,fillstyle=solid,fillcolor=curcolor]
{
\newpath
\moveto(137.42046798,225.41913362)
\curveto(137.22045768,225.12912538)(137.01045789,224.84412566)(136.79046798,224.56413362)
\curveto(136.58045832,224.28412622)(136.37545853,223.99912651)(136.17546798,223.70913362)
\curveto(135.57545933,222.85912765)(134.97045993,222.01912849)(134.36046798,221.18913362)
\curveto(133.75046115,220.36913014)(133.14546176,219.53413097)(132.54546798,218.68413362)
\lineto(132.03546798,217.96413362)
\lineto(131.52546798,217.27413362)
\curveto(131.44546346,217.16413334)(131.36546354,217.04913346)(131.28546798,216.92913362)
\curveto(131.2054637,216.8091337)(131.11046379,216.71413379)(131.00046798,216.64413362)
\curveto(130.96046394,216.62413388)(130.89546401,216.6091339)(130.80546798,216.59913362)
\curveto(130.72546418,216.57913393)(130.63546427,216.56913394)(130.53546798,216.56913362)
\curveto(130.43546447,216.56913394)(130.34046456,216.57413393)(130.25046798,216.58413362)
\curveto(130.17046473,216.59413391)(130.11046479,216.61413389)(130.07046798,216.64413362)
\curveto(130.04046486,216.66413384)(130.01546489,216.69913381)(129.99546798,216.74913362)
\curveto(129.98546492,216.78913372)(129.99046491,216.83413367)(130.01046798,216.88413362)
\curveto(130.05046485,216.96413354)(130.09546481,217.03913347)(130.14546798,217.10913362)
\curveto(130.2054647,217.18913332)(130.26046464,217.26913324)(130.31046798,217.34913362)
\curveto(130.55046435,217.68913282)(130.79546411,218.02413248)(131.04546798,218.35413362)
\curveto(131.29546361,218.68413182)(131.53546337,219.01913149)(131.76546798,219.35913362)
\curveto(131.92546298,219.57913093)(132.08546282,219.79413071)(132.24546798,220.00413362)
\curveto(132.4054625,220.21413029)(132.56546234,220.42913008)(132.72546798,220.64913362)
\curveto(133.08546182,221.16912934)(133.45046145,221.67912883)(133.82046798,222.17913362)
\curveto(134.19046071,222.67912783)(134.56046034,223.18912732)(134.93046798,223.70913362)
\curveto(135.07045983,223.9091266)(135.21045969,224.1041264)(135.35046798,224.29413362)
\curveto(135.5004594,224.48412602)(135.64545926,224.67912583)(135.78546798,224.87913362)
\curveto(135.99545891,225.17912533)(136.21045869,225.47912503)(136.43046798,225.77913362)
\lineto(137.09046798,226.67913362)
\lineto(137.27046798,226.94913362)
\lineto(137.48046798,227.21913362)
\lineto(137.60046798,227.39913362)
\curveto(137.65045725,227.45912305)(137.7004572,227.51412299)(137.75046798,227.56413362)
\curveto(137.82045708,227.61412289)(137.89545701,227.64912286)(137.97546798,227.66913362)
\curveto(137.99545691,227.67912283)(138.02045688,227.67912283)(138.05046798,227.66913362)
\curveto(138.09045681,227.66912284)(138.12045678,227.67912283)(138.14046798,227.69913362)
\curveto(138.26045664,227.69912281)(138.39545651,227.69412281)(138.54546798,227.68413362)
\curveto(138.69545621,227.68412282)(138.78545612,227.63912287)(138.81546798,227.54913362)
\curveto(138.83545607,227.51912299)(138.84045606,227.48412302)(138.83046798,227.44413362)
\curveto(138.82045608,227.4041231)(138.8054561,227.37412313)(138.78546798,227.35413362)
\curveto(138.74545616,227.27412323)(138.7054562,227.2041233)(138.66546798,227.14413362)
\curveto(138.62545628,227.08412342)(138.58045632,227.02412348)(138.53046798,226.96413362)
\lineto(137.96046798,226.18413362)
\curveto(137.78045712,225.93412457)(137.6004573,225.67912483)(137.42046798,225.41913362)
\moveto(130.56546798,221.51913362)
\curveto(130.51546439,221.53912897)(130.46546444,221.54412896)(130.41546798,221.53413362)
\curveto(130.36546454,221.52412898)(130.31546459,221.52912898)(130.26546798,221.54913362)
\curveto(130.15546475,221.56912894)(130.05046485,221.58912892)(129.95046798,221.60913362)
\curveto(129.86046504,221.63912887)(129.76546514,221.67912883)(129.66546798,221.72913362)
\curveto(129.33546557,221.86912864)(129.08046582,222.06412844)(128.90046798,222.31413362)
\curveto(128.72046618,222.57412793)(128.57546633,222.88412762)(128.46546798,223.24413362)
\curveto(128.43546647,223.32412718)(128.41546649,223.4041271)(128.40546798,223.48413362)
\curveto(128.39546651,223.57412693)(128.38046652,223.65912685)(128.36046798,223.73913362)
\curveto(128.35046655,223.78912672)(128.34546656,223.85412665)(128.34546798,223.93413362)
\curveto(128.33546657,223.96412654)(128.33046657,223.99412651)(128.33046798,224.02413362)
\curveto(128.33046657,224.06412644)(128.32546658,224.09912641)(128.31546798,224.12913362)
\lineto(128.31546798,224.27913362)
\curveto(128.3054666,224.32912618)(128.3004666,224.38912612)(128.30046798,224.45913362)
\curveto(128.3004666,224.53912597)(128.3054666,224.6041259)(128.31546798,224.65413362)
\lineto(128.31546798,224.81913362)
\curveto(128.33546657,224.86912564)(128.34046656,224.91412559)(128.33046798,224.95413362)
\curveto(128.33046657,225.0041255)(128.33546657,225.04912546)(128.34546798,225.08913362)
\curveto(128.35546655,225.12912538)(128.36046654,225.16412534)(128.36046798,225.19413362)
\curveto(128.36046654,225.23412527)(128.36546654,225.27412523)(128.37546798,225.31413362)
\curveto(128.4054665,225.42412508)(128.42546648,225.53412497)(128.43546798,225.64413362)
\curveto(128.45546645,225.76412474)(128.49046641,225.87912463)(128.54046798,225.98913362)
\curveto(128.68046622,226.32912418)(128.84046606,226.6041239)(129.02046798,226.81413362)
\curveto(129.21046569,227.03412347)(129.48046542,227.21412329)(129.83046798,227.35413362)
\curveto(129.91046499,227.38412312)(129.99546491,227.4041231)(130.08546798,227.41413362)
\curveto(130.17546473,227.43412307)(130.27046463,227.45412305)(130.37046798,227.47413362)
\curveto(130.4004645,227.48412302)(130.45546445,227.48412302)(130.53546798,227.47413362)
\curveto(130.61546429,227.47412303)(130.66546424,227.48412302)(130.68546798,227.50413362)
\curveto(131.24546366,227.51412299)(131.69546321,227.4041231)(132.03546798,227.17413362)
\curveto(132.38546252,226.94412356)(132.64546226,226.63912387)(132.81546798,226.25913362)
\curveto(132.85546205,226.16912434)(132.89046201,226.07412443)(132.92046798,225.97413362)
\curveto(132.95046195,225.87412463)(132.97546193,225.77412473)(132.99546798,225.67413362)
\curveto(133.01546189,225.64412486)(133.02046188,225.61412489)(133.01046798,225.58413362)
\curveto(133.01046189,225.55412495)(133.01546189,225.52412498)(133.02546798,225.49413362)
\curveto(133.05546185,225.38412512)(133.07546183,225.25912525)(133.08546798,225.11913362)
\curveto(133.09546181,224.98912552)(133.1054618,224.85412565)(133.11546798,224.71413362)
\lineto(133.11546798,224.54913362)
\curveto(133.12546178,224.48912602)(133.12546178,224.43412607)(133.11546798,224.38413362)
\curveto(133.1054618,224.33412617)(133.1004618,224.28412622)(133.10046798,224.23413362)
\lineto(133.10046798,224.09913362)
\curveto(133.09046181,224.05912645)(133.08546182,224.01912649)(133.08546798,223.97913362)
\curveto(133.09546181,223.93912657)(133.09046181,223.89412661)(133.07046798,223.84413362)
\curveto(133.05046185,223.73412677)(133.03046187,223.62912688)(133.01046798,223.52913362)
\curveto(133.0004619,223.42912708)(132.98046192,223.32912718)(132.95046798,223.22913362)
\curveto(132.82046208,222.86912764)(132.65546225,222.55412795)(132.45546798,222.28413362)
\curveto(132.25546265,222.01412849)(131.98046292,221.8091287)(131.63046798,221.66913362)
\curveto(131.55046335,221.63912887)(131.46546344,221.61412889)(131.37546798,221.59413362)
\lineto(131.10546798,221.53413362)
\curveto(131.05546385,221.52412898)(131.01046389,221.51912899)(130.97046798,221.51913362)
\curveto(130.93046397,221.52912898)(130.89046401,221.52912898)(130.85046798,221.51913362)
\curveto(130.75046415,221.49912901)(130.65546425,221.49912901)(130.56546798,221.51913362)
\moveto(129.72546798,222.91413362)
\curveto(129.76546514,222.84412766)(129.8054651,222.77912773)(129.84546798,222.71913362)
\curveto(129.88546502,222.66912784)(129.93546497,222.61912789)(129.99546798,222.56913362)
\lineto(130.14546798,222.44913362)
\curveto(130.2054647,222.41912809)(130.27046463,222.39412811)(130.34046798,222.37413362)
\curveto(130.38046452,222.35412815)(130.41546449,222.34412816)(130.44546798,222.34413362)
\curveto(130.48546442,222.35412815)(130.52546438,222.34912816)(130.56546798,222.32913362)
\curveto(130.59546431,222.32912818)(130.63546427,222.32412818)(130.68546798,222.31413362)
\curveto(130.73546417,222.31412819)(130.77546413,222.31912819)(130.80546798,222.32913362)
\lineto(131.03046798,222.37413362)
\curveto(131.28046362,222.45412805)(131.46546344,222.57912793)(131.58546798,222.74913362)
\curveto(131.66546324,222.84912766)(131.73546317,222.97912753)(131.79546798,223.13913362)
\curveto(131.87546303,223.31912719)(131.93546297,223.54412696)(131.97546798,223.81413362)
\curveto(132.01546289,224.09412641)(132.03046287,224.37412613)(132.02046798,224.65413362)
\curveto(132.01046289,224.94412556)(131.98046292,225.21912529)(131.93046798,225.47913362)
\curveto(131.88046302,225.73912477)(131.8054631,225.94912456)(131.70546798,226.10913362)
\curveto(131.58546332,226.3091242)(131.43546347,226.45912405)(131.25546798,226.55913362)
\curveto(131.17546373,226.6091239)(131.08546382,226.63912387)(130.98546798,226.64913362)
\curveto(130.88546402,226.66912384)(130.78046412,226.67912383)(130.67046798,226.67913362)
\curveto(130.65046425,226.66912384)(130.62546428,226.66412384)(130.59546798,226.66413362)
\curveto(130.57546433,226.67412383)(130.55546435,226.67412383)(130.53546798,226.66413362)
\curveto(130.48546442,226.65412385)(130.44046446,226.64412386)(130.40046798,226.63413362)
\curveto(130.36046454,226.63412387)(130.32046458,226.62412388)(130.28046798,226.60413362)
\curveto(130.1004648,226.52412398)(129.95046495,226.4041241)(129.83046798,226.24413362)
\curveto(129.72046518,226.08412442)(129.63046527,225.9041246)(129.56046798,225.70413362)
\curveto(129.5004654,225.51412499)(129.45546545,225.28912522)(129.42546798,225.02913362)
\curveto(129.4054655,224.76912574)(129.4004655,224.504126)(129.41046798,224.23413362)
\curveto(129.42046548,223.97412653)(129.45046545,223.72412678)(129.50046798,223.48413362)
\curveto(129.56046534,223.25412725)(129.63546527,223.06412744)(129.72546798,222.91413362)
\moveto(140.52546798,219.92913362)
\curveto(140.53545437,219.87913063)(140.54045436,219.78913072)(140.54046798,219.65913362)
\curveto(140.54045436,219.52913098)(140.53045437,219.43913107)(140.51046798,219.38913362)
\curveto(140.49045441,219.33913117)(140.48545442,219.28413122)(140.49546798,219.22413362)
\curveto(140.5054544,219.17413133)(140.5054544,219.12413138)(140.49546798,219.07413362)
\curveto(140.45545445,218.93413157)(140.42545448,218.79913171)(140.40546798,218.66913362)
\curveto(140.39545451,218.53913197)(140.36545454,218.41913209)(140.31546798,218.30913362)
\curveto(140.17545473,217.95913255)(140.01045489,217.66413284)(139.82046798,217.42413362)
\curveto(139.63045527,217.19413331)(139.36045554,217.0091335)(139.01046798,216.86913362)
\curveto(138.93045597,216.83913367)(138.84545606,216.81913369)(138.75546798,216.80913362)
\curveto(138.66545624,216.78913372)(138.58045632,216.76913374)(138.50046798,216.74913362)
\curveto(138.45045645,216.73913377)(138.4004565,216.73413377)(138.35046798,216.73413362)
\curveto(138.3004566,216.73413377)(138.25045665,216.72913378)(138.20046798,216.71913362)
\curveto(138.17045673,216.7091338)(138.12045678,216.7091338)(138.05046798,216.71913362)
\curveto(137.98045692,216.71913379)(137.93045697,216.72413378)(137.90046798,216.73413362)
\curveto(137.84045706,216.75413375)(137.78045712,216.76413374)(137.72046798,216.76413362)
\curveto(137.67045723,216.75413375)(137.62045728,216.75913375)(137.57046798,216.77913362)
\curveto(137.48045742,216.79913371)(137.39045751,216.82413368)(137.30046798,216.85413362)
\curveto(137.22045768,216.87413363)(137.14045776,216.9041336)(137.06046798,216.94413362)
\curveto(136.74045816,217.08413342)(136.49045841,217.27913323)(136.31046798,217.52913362)
\curveto(136.13045877,217.78913272)(135.98045892,218.09413241)(135.86046798,218.44413362)
\curveto(135.84045906,218.52413198)(135.82545908,218.6091319)(135.81546798,218.69913362)
\curveto(135.8054591,218.78913172)(135.79045911,218.87413163)(135.77046798,218.95413362)
\curveto(135.76045914,218.98413152)(135.75545915,219.01413149)(135.75546798,219.04413362)
\lineto(135.75546798,219.14913362)
\curveto(135.73545917,219.22913128)(135.72545918,219.3091312)(135.72546798,219.38913362)
\lineto(135.72546798,219.52413362)
\curveto(135.7054592,219.62413088)(135.7054592,219.72413078)(135.72546798,219.82413362)
\lineto(135.72546798,220.00413362)
\curveto(135.73545917,220.05413045)(135.74045916,220.09913041)(135.74046798,220.13913362)
\curveto(135.74045916,220.18913032)(135.74545916,220.23413027)(135.75546798,220.27413362)
\curveto(135.76545914,220.31413019)(135.77045913,220.34913016)(135.77046798,220.37913362)
\curveto(135.77045913,220.41913009)(135.77545913,220.45913005)(135.78546798,220.49913362)
\lineto(135.84546798,220.82913362)
\curveto(135.86545904,220.94912956)(135.89545901,221.05912945)(135.93546798,221.15913362)
\curveto(136.07545883,221.48912902)(136.23545867,221.76412874)(136.41546798,221.98413362)
\curveto(136.6054583,222.21412829)(136.86545804,222.39912811)(137.19546798,222.53913362)
\curveto(137.27545763,222.57912793)(137.36045754,222.6041279)(137.45046798,222.61413362)
\lineto(137.75046798,222.67413362)
\lineto(137.88546798,222.67413362)
\curveto(137.93545697,222.68412782)(137.98545692,222.68912782)(138.03546798,222.68913362)
\curveto(138.6054563,222.7091278)(139.06545584,222.6041279)(139.41546798,222.37413362)
\curveto(139.77545513,222.15412835)(140.04045486,221.85412865)(140.21046798,221.47413362)
\curveto(140.26045464,221.37412913)(140.3004546,221.27412923)(140.33046798,221.17413362)
\curveto(140.36045454,221.07412943)(140.39045451,220.96912954)(140.42046798,220.85913362)
\curveto(140.43045447,220.81912969)(140.43545447,220.78412972)(140.43546798,220.75413362)
\curveto(140.43545447,220.73412977)(140.44045446,220.7041298)(140.45046798,220.66413362)
\curveto(140.47045443,220.59412991)(140.48045442,220.51912999)(140.48046798,220.43913362)
\curveto(140.48045442,220.35913015)(140.49045441,220.27913023)(140.51046798,220.19913362)
\curveto(140.51045439,220.14913036)(140.51045439,220.1041304)(140.51046798,220.06413362)
\curveto(140.51045439,220.02413048)(140.51545439,219.97913053)(140.52546798,219.92913362)
\moveto(139.41546798,219.49413362)
\curveto(139.42545548,219.54413096)(139.43045547,219.61913089)(139.43046798,219.71913362)
\curveto(139.44045546,219.81913069)(139.43545547,219.89413061)(139.41546798,219.94413362)
\curveto(139.39545551,220.0041305)(139.39045551,220.05913045)(139.40046798,220.10913362)
\curveto(139.42045548,220.16913034)(139.42045548,220.22913028)(139.40046798,220.28913362)
\curveto(139.39045551,220.31913019)(139.38545552,220.35413015)(139.38546798,220.39413362)
\curveto(139.38545552,220.43413007)(139.38045552,220.47413003)(139.37046798,220.51413362)
\curveto(139.35045555,220.59412991)(139.33045557,220.66912984)(139.31046798,220.73913362)
\curveto(139.3004556,220.81912969)(139.28545562,220.89912961)(139.26546798,220.97913362)
\curveto(139.23545567,221.03912947)(139.21045569,221.09912941)(139.19046798,221.15913362)
\curveto(139.17045573,221.21912929)(139.14045576,221.27912923)(139.10046798,221.33913362)
\curveto(139.0004559,221.509129)(138.87045603,221.64412886)(138.71046798,221.74413362)
\curveto(138.63045627,221.79412871)(138.53545637,221.82912868)(138.42546798,221.84913362)
\curveto(138.31545659,221.86912864)(138.19045671,221.87912863)(138.05046798,221.87913362)
\curveto(138.03045687,221.86912864)(138.0054569,221.86412864)(137.97546798,221.86413362)
\curveto(137.94545696,221.87412863)(137.91545699,221.87412863)(137.88546798,221.86413362)
\lineto(137.73546798,221.80413362)
\curveto(137.68545722,221.79412871)(137.64045726,221.77912873)(137.60046798,221.75913362)
\curveto(137.41045749,221.64912886)(137.26545764,221.504129)(137.16546798,221.32413362)
\curveto(137.07545783,221.14412936)(136.99545791,220.93912957)(136.92546798,220.70913362)
\curveto(136.88545802,220.57912993)(136.86545804,220.44413006)(136.86546798,220.30413362)
\curveto(136.86545804,220.17413033)(136.85545805,220.02913048)(136.83546798,219.86913362)
\curveto(136.82545808,219.81913069)(136.81545809,219.75913075)(136.80546798,219.68913362)
\curveto(136.8054581,219.61913089)(136.81545809,219.55913095)(136.83546798,219.50913362)
\lineto(136.83546798,219.34413362)
\lineto(136.83546798,219.16413362)
\curveto(136.84545806,219.11413139)(136.85545805,219.05913145)(136.86546798,218.99913362)
\curveto(136.87545803,218.94913156)(136.88045802,218.89413161)(136.88046798,218.83413362)
\curveto(136.89045801,218.77413173)(136.905458,218.71913179)(136.92546798,218.66913362)
\curveto(136.97545793,218.47913203)(137.03545787,218.3041322)(137.10546798,218.14413362)
\curveto(137.17545773,217.98413252)(137.28045762,217.85413265)(137.42046798,217.75413362)
\curveto(137.55045735,217.65413285)(137.69045721,217.58413292)(137.84046798,217.54413362)
\curveto(137.87045703,217.53413297)(137.89545701,217.52913298)(137.91546798,217.52913362)
\curveto(137.94545696,217.53913297)(137.97545693,217.53913297)(138.00546798,217.52913362)
\curveto(138.02545688,217.52913298)(138.05545685,217.52413298)(138.09546798,217.51413362)
\curveto(138.13545677,217.51413299)(138.17045673,217.51913299)(138.20046798,217.52913362)
\curveto(138.24045666,217.53913297)(138.28045662,217.54413296)(138.32046798,217.54413362)
\curveto(138.36045654,217.54413296)(138.4004565,217.55413295)(138.44046798,217.57413362)
\curveto(138.68045622,217.65413285)(138.87545603,217.78913272)(139.02546798,217.97913362)
\curveto(139.14545576,218.15913235)(139.23545567,218.36413214)(139.29546798,218.59413362)
\curveto(139.31545559,218.66413184)(139.33045557,218.73413177)(139.34046798,218.80413362)
\curveto(139.35045555,218.88413162)(139.36545554,218.96413154)(139.38546798,219.04413362)
\curveto(139.38545552,219.1041314)(139.39045551,219.14913136)(139.40046798,219.17913362)
\curveto(139.4004555,219.19913131)(139.4004555,219.22413128)(139.40046798,219.25413362)
\curveto(139.4004555,219.29413121)(139.4054555,219.32413118)(139.41546798,219.34413362)
\lineto(139.41546798,219.49413362)
}
}
{
\newrgbcolor{curcolor}{0 0 0}
\pscustom[linestyle=none,fillstyle=solid,fillcolor=curcolor]
{
\newpath
\moveto(455.80941757,179.48297029)
\curveto(455.87940992,179.43296683)(455.91940988,179.3629669)(455.92941757,179.27297029)
\curveto(455.94940985,179.18296708)(455.95940984,179.07796719)(455.95941757,178.95797029)
\curveto(455.95940984,178.90796736)(455.95440985,178.85796741)(455.94441757,178.80797029)
\curveto(455.94440986,178.75796751)(455.93440987,178.71296755)(455.91441757,178.67297029)
\curveto(455.88440992,178.58296768)(455.82440998,178.52296774)(455.73441757,178.49297029)
\curveto(455.65441015,178.47296779)(455.55941024,178.4629678)(455.44941757,178.46297029)
\lineto(455.13441757,178.46297029)
\curveto(455.02441078,178.47296779)(454.91941088,178.4629678)(454.81941757,178.43297029)
\curveto(454.67941112,178.40296786)(454.58941121,178.32296794)(454.54941757,178.19297029)
\curveto(454.52941127,178.12296814)(454.51941128,178.03796823)(454.51941757,177.93797029)
\lineto(454.51941757,177.66797029)
\lineto(454.51941757,176.72297029)
\lineto(454.51941757,176.39297029)
\curveto(454.51941128,176.28296998)(454.4994113,176.19797007)(454.45941757,176.13797029)
\curveto(454.41941138,176.07797019)(454.36941143,176.03797023)(454.30941757,176.01797029)
\curveto(454.25941154,176.00797026)(454.19441161,175.99297027)(454.11441757,175.97297029)
\lineto(453.91941757,175.97297029)
\curveto(453.799412,175.97297029)(453.69441211,175.97797029)(453.60441757,175.98797029)
\curveto(453.51441229,176.00797026)(453.44441236,176.05797021)(453.39441757,176.13797029)
\curveto(453.36441244,176.18797008)(453.34941245,176.25797001)(453.34941757,176.34797029)
\lineto(453.34941757,176.64797029)
\lineto(453.34941757,177.68297029)
\curveto(453.34941245,177.84296842)(453.33941246,177.98796828)(453.31941757,178.11797029)
\curveto(453.30941249,178.25796801)(453.25441255,178.35296791)(453.15441757,178.40297029)
\curveto(453.1044127,178.42296784)(453.03441277,178.43796783)(452.94441757,178.44797029)
\curveto(452.86441294,178.45796781)(452.77441303,178.4629678)(452.67441757,178.46297029)
\lineto(452.38941757,178.46297029)
\lineto(452.14941757,178.46297029)
\lineto(449.88441757,178.46297029)
\curveto(449.79441601,178.4629678)(449.68941611,178.45796781)(449.56941757,178.44797029)
\lineto(449.23941757,178.44797029)
\curveto(449.12941667,178.44796782)(449.02941677,178.45796781)(448.93941757,178.47797029)
\curveto(448.84941695,178.49796777)(448.78941701,178.53296773)(448.75941757,178.58297029)
\curveto(448.70941709,178.65296761)(448.68441712,178.74796752)(448.68441757,178.86797029)
\lineto(448.68441757,179.21297029)
\lineto(448.68441757,179.48297029)
\curveto(448.72441708,179.65296661)(448.77941702,179.79296647)(448.84941757,179.90297029)
\curveto(448.91941688,180.01296625)(448.9994168,180.12796614)(449.08941757,180.24797029)
\lineto(449.44941757,180.78797029)
\curveto(449.88941591,181.41796485)(450.32441548,182.03796423)(450.75441757,182.64797029)
\lineto(452.07441757,184.50797029)
\curveto(452.23441357,184.73796153)(452.38941341,184.95796131)(452.53941757,185.16797029)
\curveto(452.68941311,185.38796088)(452.84441296,185.61296065)(453.00441757,185.84297029)
\curveto(453.05441275,185.91296035)(453.1044127,185.97796029)(453.15441757,186.03797029)
\curveto(453.2044126,186.10796016)(453.25441255,186.18296008)(453.30441757,186.26297029)
\lineto(453.36441757,186.35297029)
\curveto(453.39441241,186.39295987)(453.42441238,186.42295984)(453.45441757,186.44297029)
\curveto(453.49441231,186.47295979)(453.53441227,186.49295977)(453.57441757,186.50297029)
\curveto(453.61441219,186.52295974)(453.65941214,186.54295972)(453.70941757,186.56297029)
\curveto(453.72941207,186.5629597)(453.74941205,186.55795971)(453.76941757,186.54797029)
\curveto(453.799412,186.54795972)(453.82441198,186.55795971)(453.84441757,186.57797029)
\curveto(453.97441183,186.57795969)(454.09441171,186.57295969)(454.20441757,186.56297029)
\curveto(454.31441149,186.55295971)(454.39441141,186.50795976)(454.44441757,186.42797029)
\curveto(454.48441132,186.37795989)(454.5044113,186.30795996)(454.50441757,186.21797029)
\curveto(454.51441129,186.12796014)(454.51941128,186.03296023)(454.51941757,185.93297029)
\lineto(454.51941757,180.47297029)
\curveto(454.51941128,180.40296586)(454.51441129,180.32796594)(454.50441757,180.24797029)
\curveto(454.5044113,180.17796609)(454.50941129,180.10796616)(454.51941757,180.03797029)
\lineto(454.51941757,179.93297029)
\curveto(454.53941126,179.88296638)(454.55441125,179.82796644)(454.56441757,179.76797029)
\curveto(454.57441123,179.71796655)(454.5994112,179.67796659)(454.63941757,179.64797029)
\curveto(454.70941109,179.59796667)(454.79441101,179.5679667)(454.89441757,179.55797029)
\lineto(455.22441757,179.55797029)
\curveto(455.33441047,179.55796671)(455.43941036,179.55296671)(455.53941757,179.54297029)
\curveto(455.64941015,179.54296672)(455.73941006,179.52296674)(455.80941757,179.48297029)
\moveto(453.24441757,179.67797029)
\curveto(453.32441248,179.78796648)(453.35941244,179.95796631)(453.34941757,180.18797029)
\lineto(453.34941757,180.80297029)
\lineto(453.34941757,183.27797029)
\lineto(453.34941757,183.59297029)
\curveto(453.35941244,183.71296255)(453.35441245,183.81296245)(453.33441757,183.89297029)
\lineto(453.33441757,184.04297029)
\curveto(453.33441247,184.13296213)(453.31941248,184.21796205)(453.28941757,184.29797029)
\curveto(453.27941252,184.31796195)(453.26941253,184.32796194)(453.25941757,184.32797029)
\lineto(453.21441757,184.37297029)
\curveto(453.19441261,184.38296188)(453.16441264,184.38796188)(453.12441757,184.38797029)
\curveto(453.1044127,184.3679619)(453.08441272,184.35296191)(453.06441757,184.34297029)
\curveto(453.05441275,184.34296192)(453.03941276,184.33796193)(453.01941757,184.32797029)
\curveto(452.95941284,184.27796199)(452.8994129,184.20796206)(452.83941757,184.11797029)
\curveto(452.77941302,184.02796224)(452.72441308,183.94796232)(452.67441757,183.87797029)
\curveto(452.57441323,183.73796253)(452.47941332,183.59296267)(452.38941757,183.44297029)
\curveto(452.2994135,183.30296296)(452.2044136,183.1629631)(452.10441757,183.02297029)
\lineto(451.56441757,182.24297029)
\curveto(451.39441441,181.98296428)(451.21941458,181.72296454)(451.03941757,181.46297029)
\curveto(450.95941484,181.35296491)(450.88441492,181.24796502)(450.81441757,181.14797029)
\lineto(450.60441757,180.84797029)
\curveto(450.55441525,180.7679655)(450.5044153,180.69296557)(450.45441757,180.62297029)
\curveto(450.41441539,180.55296571)(450.36941543,180.47796579)(450.31941757,180.39797029)
\curveto(450.26941553,180.33796593)(450.21941558,180.27296599)(450.16941757,180.20297029)
\curveto(450.12941567,180.14296612)(450.08941571,180.07296619)(450.04941757,179.99297029)
\curveto(450.00941579,179.93296633)(449.98441582,179.8629664)(449.97441757,179.78297029)
\curveto(449.96441584,179.71296655)(449.9994158,179.65796661)(450.07941757,179.61797029)
\curveto(450.14941565,179.5679667)(450.25941554,179.54296672)(450.40941757,179.54297029)
\curveto(450.56941523,179.55296671)(450.7044151,179.55796671)(450.81441757,179.55797029)
\lineto(452.49441757,179.55797029)
\lineto(452.92941757,179.55797029)
\curveto(453.07941272,179.55796671)(453.18441262,179.59796667)(453.24441757,179.67797029)
}
}
{
\newrgbcolor{curcolor}{0 0 0}
\pscustom[linestyle=none,fillstyle=solid,fillcolor=curcolor]
{
\newpath
\moveto(458.78902694,186.39797029)
\lineto(462.38902694,186.39797029)
\lineto(463.03402694,186.39797029)
\curveto(463.11402041,186.39795987)(463.18902034,186.39295987)(463.25902694,186.38297029)
\curveto(463.3290202,186.38295988)(463.38902014,186.37295989)(463.43902694,186.35297029)
\curveto(463.50902002,186.32295994)(463.56401996,186.26296)(463.60402694,186.17297029)
\curveto(463.6240199,186.14296012)(463.63401989,186.10296016)(463.63402694,186.05297029)
\lineto(463.63402694,185.91797029)
\curveto(463.64401988,185.80796046)(463.63901989,185.70296056)(463.61902694,185.60297029)
\curveto(463.60901992,185.50296076)(463.57401995,185.43296083)(463.51402694,185.39297029)
\curveto(463.4240201,185.32296094)(463.28902024,185.28796098)(463.10902694,185.28797029)
\curveto(462.9290206,185.29796097)(462.76402076,185.30296096)(462.61402694,185.30297029)
\lineto(460.61902694,185.30297029)
\lineto(460.12402694,185.30297029)
\lineto(459.98902694,185.30297029)
\curveto(459.94902358,185.30296096)(459.90902362,185.29796097)(459.86902694,185.28797029)
\lineto(459.65902694,185.28797029)
\curveto(459.54902398,185.25796101)(459.46902406,185.21796105)(459.41902694,185.16797029)
\curveto(459.36902416,185.12796114)(459.33402419,185.07296119)(459.31402694,185.00297029)
\curveto(459.29402423,184.94296132)(459.27902425,184.87296139)(459.26902694,184.79297029)
\curveto(459.25902427,184.71296155)(459.23902429,184.62296164)(459.20902694,184.52297029)
\curveto(459.15902437,184.32296194)(459.11902441,184.11796215)(459.08902694,183.90797029)
\curveto(459.05902447,183.69796257)(459.01902451,183.49296277)(458.96902694,183.29297029)
\curveto(458.94902458,183.22296304)(458.93902459,183.15296311)(458.93902694,183.08297029)
\curveto(458.93902459,183.02296324)(458.9290246,182.95796331)(458.90902694,182.88797029)
\curveto(458.89902463,182.85796341)(458.88902464,182.81796345)(458.87902694,182.76797029)
\curveto(458.87902465,182.72796354)(458.88402464,182.68796358)(458.89402694,182.64797029)
\curveto(458.91402461,182.59796367)(458.93902459,182.55296371)(458.96902694,182.51297029)
\curveto(459.00902452,182.48296378)(459.06902446,182.47796379)(459.14902694,182.49797029)
\curveto(459.20902432,182.51796375)(459.26902426,182.54296372)(459.32902694,182.57297029)
\curveto(459.38902414,182.61296365)(459.44902408,182.64796362)(459.50902694,182.67797029)
\curveto(459.56902396,182.69796357)(459.61902391,182.71296355)(459.65902694,182.72297029)
\curveto(459.84902368,182.80296346)(460.05402347,182.85796341)(460.27402694,182.88797029)
\curveto(460.50402302,182.91796335)(460.73402279,182.92796334)(460.96402694,182.91797029)
\curveto(461.20402232,182.91796335)(461.43402209,182.89296337)(461.65402694,182.84297029)
\curveto(461.87402165,182.80296346)(462.07402145,182.74296352)(462.25402694,182.66297029)
\curveto(462.30402122,182.64296362)(462.34902118,182.62296364)(462.38902694,182.60297029)
\curveto(462.43902109,182.58296368)(462.48902104,182.55796371)(462.53902694,182.52797029)
\curveto(462.88902064,182.31796395)(463.16902036,182.08796418)(463.37902694,181.83797029)
\curveto(463.59901993,181.58796468)(463.79401973,181.262965)(463.96402694,180.86297029)
\curveto(464.01401951,180.75296551)(464.04901948,180.64296562)(464.06902694,180.53297029)
\curveto(464.08901944,180.42296584)(464.11401941,180.30796596)(464.14402694,180.18797029)
\curveto(464.15401937,180.15796611)(464.15901937,180.11296615)(464.15902694,180.05297029)
\curveto(464.17901935,179.99296627)(464.18901934,179.92296634)(464.18902694,179.84297029)
\curveto(464.18901934,179.77296649)(464.19901933,179.70796656)(464.21902694,179.64797029)
\lineto(464.21902694,179.48297029)
\curveto(464.2290193,179.43296683)(464.23401929,179.3629669)(464.23402694,179.27297029)
\curveto(464.23401929,179.18296708)(464.2240193,179.11296715)(464.20402694,179.06297029)
\curveto(464.18401934,179.00296726)(464.17901935,178.94296732)(464.18902694,178.88297029)
\curveto(464.19901933,178.83296743)(464.19401933,178.78296748)(464.17402694,178.73297029)
\curveto(464.13401939,178.57296769)(464.09901943,178.42296784)(464.06902694,178.28297029)
\curveto(464.03901949,178.14296812)(463.99401953,178.00796826)(463.93402694,177.87797029)
\curveto(463.77401975,177.50796876)(463.55401997,177.17296909)(463.27402694,176.87297029)
\curveto(462.99402053,176.57296969)(462.67402085,176.34296992)(462.31402694,176.18297029)
\curveto(462.14402138,176.10297016)(461.94402158,176.02797024)(461.71402694,175.95797029)
\curveto(461.60402192,175.91797035)(461.48902204,175.89297037)(461.36902694,175.88297029)
\curveto(461.24902228,175.87297039)(461.1290224,175.85297041)(461.00902694,175.82297029)
\curveto(460.95902257,175.80297046)(460.90402262,175.80297046)(460.84402694,175.82297029)
\curveto(460.78402274,175.83297043)(460.7240228,175.82797044)(460.66402694,175.80797029)
\curveto(460.56402296,175.78797048)(460.46402306,175.78797048)(460.36402694,175.80797029)
\lineto(460.22902694,175.80797029)
\curveto(460.17902335,175.82797044)(460.11902341,175.83797043)(460.04902694,175.83797029)
\curveto(459.98902354,175.82797044)(459.93402359,175.83297043)(459.88402694,175.85297029)
\curveto(459.84402368,175.8629704)(459.80902372,175.8679704)(459.77902694,175.86797029)
\curveto(459.74902378,175.8679704)(459.71402381,175.87297039)(459.67402694,175.88297029)
\lineto(459.40402694,175.94297029)
\curveto(459.31402421,175.9629703)(459.2290243,175.99297027)(459.14902694,176.03297029)
\curveto(458.80902472,176.17297009)(458.51902501,176.32796994)(458.27902694,176.49797029)
\curveto(458.03902549,176.67796959)(457.81902571,176.90796936)(457.61902694,177.18797029)
\curveto(457.46902606,177.41796885)(457.35402617,177.65796861)(457.27402694,177.90797029)
\curveto(457.25402627,177.95796831)(457.24402628,178.00296826)(457.24402694,178.04297029)
\curveto(457.24402628,178.09296817)(457.23402629,178.14296812)(457.21402694,178.19297029)
\curveto(457.19402633,178.25296801)(457.17902635,178.33296793)(457.16902694,178.43297029)
\curveto(457.16902636,178.53296773)(457.18902634,178.60796766)(457.22902694,178.65797029)
\curveto(457.27902625,178.73796753)(457.35902617,178.78296748)(457.46902694,178.79297029)
\curveto(457.57902595,178.80296746)(457.69402583,178.80796746)(457.81402694,178.80797029)
\lineto(457.97902694,178.80797029)
\curveto(458.03902549,178.80796746)(458.09402543,178.79796747)(458.14402694,178.77797029)
\curveto(458.23402529,178.75796751)(458.30402522,178.71796755)(458.35402694,178.65797029)
\curveto(458.4240251,178.5679677)(458.46902506,178.45796781)(458.48902694,178.32797029)
\curveto(458.51902501,178.20796806)(458.56402496,178.10296816)(458.62402694,178.01297029)
\curveto(458.81402471,177.67296859)(459.07402445,177.40296886)(459.40402694,177.20297029)
\curveto(459.50402402,177.14296912)(459.60902392,177.09296917)(459.71902694,177.05297029)
\curveto(459.83902369,177.02296924)(459.95902357,176.98796928)(460.07902694,176.94797029)
\curveto(460.24902328,176.89796937)(460.45402307,176.87796939)(460.69402694,176.88797029)
\curveto(460.94402258,176.90796936)(461.14402238,176.94296932)(461.29402694,176.99297029)
\curveto(461.66402186,177.11296915)(461.95402157,177.27296899)(462.16402694,177.47297029)
\curveto(462.38402114,177.68296858)(462.56402096,177.9629683)(462.70402694,178.31297029)
\curveto(462.75402077,178.41296785)(462.78402074,178.51796775)(462.79402694,178.62797029)
\curveto(462.81402071,178.73796753)(462.83902069,178.85296741)(462.86902694,178.97297029)
\lineto(462.86902694,179.07797029)
\curveto(462.87902065,179.11796715)(462.88402064,179.15796711)(462.88402694,179.19797029)
\curveto(462.89402063,179.22796704)(462.89402063,179.262967)(462.88402694,179.30297029)
\lineto(462.88402694,179.42297029)
\curveto(462.88402064,179.68296658)(462.85402067,179.92796634)(462.79402694,180.15797029)
\curveto(462.68402084,180.50796576)(462.529021,180.80296546)(462.32902694,181.04297029)
\curveto(462.1290214,181.29296497)(461.86902166,181.48796478)(461.54902694,181.62797029)
\lineto(461.36902694,181.68797029)
\curveto(461.31902221,181.70796456)(461.25902227,181.72796454)(461.18902694,181.74797029)
\curveto(461.13902239,181.7679645)(461.07902245,181.77796449)(461.00902694,181.77797029)
\curveto(460.94902258,181.78796448)(460.88402264,181.80296446)(460.81402694,181.82297029)
\lineto(460.66402694,181.82297029)
\curveto(460.6240229,181.84296442)(460.56902296,181.85296441)(460.49902694,181.85297029)
\curveto(460.43902309,181.85296441)(460.38402314,181.84296442)(460.33402694,181.82297029)
\lineto(460.22902694,181.82297029)
\curveto(460.19902333,181.82296444)(460.16402336,181.81796445)(460.12402694,181.80797029)
\lineto(459.88402694,181.74797029)
\curveto(459.80402372,181.73796453)(459.7240238,181.71796455)(459.64402694,181.68797029)
\curveto(459.40402412,181.58796468)(459.17402435,181.45296481)(458.95402694,181.28297029)
\curveto(458.86402466,181.21296505)(458.77902475,181.13796513)(458.69902694,181.05797029)
\curveto(458.61902491,180.98796528)(458.51902501,180.93296533)(458.39902694,180.89297029)
\curveto(458.30902522,180.8629654)(458.16902536,180.85296541)(457.97902694,180.86297029)
\curveto(457.79902573,180.87296539)(457.67902585,180.89796537)(457.61902694,180.93797029)
\curveto(457.56902596,180.97796529)(457.529026,181.03796523)(457.49902694,181.11797029)
\curveto(457.47902605,181.19796507)(457.47902605,181.28296498)(457.49902694,181.37297029)
\curveto(457.529026,181.49296477)(457.54902598,181.61296465)(457.55902694,181.73297029)
\curveto(457.57902595,181.8629644)(457.60402592,181.98796428)(457.63402694,182.10797029)
\curveto(457.65402587,182.14796412)(457.65902587,182.18296408)(457.64902694,182.21297029)
\curveto(457.64902588,182.25296401)(457.65902587,182.29796397)(457.67902694,182.34797029)
\curveto(457.69902583,182.43796383)(457.71402581,182.52796374)(457.72402694,182.61797029)
\curveto(457.73402579,182.71796355)(457.75402577,182.81296345)(457.78402694,182.90297029)
\curveto(457.79402573,182.9629633)(457.79902573,183.02296324)(457.79902694,183.08297029)
\curveto(457.80902572,183.14296312)(457.8240257,183.20296306)(457.84402694,183.26297029)
\curveto(457.89402563,183.4629628)(457.9290256,183.6679626)(457.94902694,183.87797029)
\curveto(457.97902555,184.09796217)(458.01902551,184.30796196)(458.06902694,184.50797029)
\curveto(458.09902543,184.60796166)(458.11902541,184.70796156)(458.12902694,184.80797029)
\curveto(458.13902539,184.90796136)(458.15402537,185.00796126)(458.17402694,185.10797029)
\curveto(458.18402534,185.13796113)(458.18902534,185.17796109)(458.18902694,185.22797029)
\curveto(458.21902531,185.33796093)(458.23902529,185.44296082)(458.24902694,185.54297029)
\curveto(458.26902526,185.65296061)(458.29402523,185.7629605)(458.32402694,185.87297029)
\curveto(458.34402518,185.95296031)(458.35902517,186.02296024)(458.36902694,186.08297029)
\curveto(458.37902515,186.15296011)(458.40402512,186.21296005)(458.44402694,186.26297029)
\curveto(458.46402506,186.29295997)(458.49402503,186.31295995)(458.53402694,186.32297029)
\curveto(458.57402495,186.34295992)(458.61902491,186.3629599)(458.66902694,186.38297029)
\curveto(458.7290248,186.38295988)(458.76902476,186.38795988)(458.78902694,186.39797029)
}
}
{
\newrgbcolor{curcolor}{0 0 0}
\pscustom[linestyle=none,fillstyle=solid,fillcolor=curcolor]
{
\newpath
\moveto(466.58363632,177.62297029)
\lineto(466.88363632,177.62297029)
\curveto(466.99363426,177.63296863)(467.09863415,177.63296863)(467.19863632,177.62297029)
\curveto(467.30863394,177.62296864)(467.40863384,177.61296865)(467.49863632,177.59297029)
\curveto(467.58863366,177.58296868)(467.65863359,177.55796871)(467.70863632,177.51797029)
\curveto(467.72863352,177.49796877)(467.74363351,177.4679688)(467.75363632,177.42797029)
\curveto(467.77363348,177.38796888)(467.79363346,177.34296892)(467.81363632,177.29297029)
\lineto(467.81363632,177.21797029)
\curveto(467.82363343,177.1679691)(467.82363343,177.11296915)(467.81363632,177.05297029)
\lineto(467.81363632,176.90297029)
\lineto(467.81363632,176.42297029)
\curveto(467.81363344,176.25297001)(467.77363348,176.13297013)(467.69363632,176.06297029)
\curveto(467.62363363,176.01297025)(467.53363372,175.98797028)(467.42363632,175.98797029)
\lineto(467.09363632,175.98797029)
\lineto(466.64363632,175.98797029)
\curveto(466.49363476,175.98797028)(466.37863487,176.01797025)(466.29863632,176.07797029)
\curveto(466.25863499,176.10797016)(466.22863502,176.15797011)(466.20863632,176.22797029)
\curveto(466.18863506,176.30796996)(466.17363508,176.39296987)(466.16363632,176.48297029)
\lineto(466.16363632,176.76797029)
\curveto(466.17363508,176.8679694)(466.17863507,176.95296931)(466.17863632,177.02297029)
\lineto(466.17863632,177.21797029)
\curveto(466.17863507,177.27796899)(466.18863506,177.33296893)(466.20863632,177.38297029)
\curveto(466.248635,177.49296877)(466.31863493,177.5629687)(466.41863632,177.59297029)
\curveto(466.4486348,177.59296867)(466.50363475,177.60296866)(466.58363632,177.62297029)
}
}
{
\newrgbcolor{curcolor}{0 0 0}
\pscustom[linestyle=none,fillstyle=solid,fillcolor=curcolor]
{
\newpath
\moveto(471.29879257,186.39797029)
\lineto(474.89879257,186.39797029)
\lineto(475.54379257,186.39797029)
\curveto(475.62378604,186.39795987)(475.69878596,186.39295987)(475.76879257,186.38297029)
\curveto(475.83878582,186.38295988)(475.89878576,186.37295989)(475.94879257,186.35297029)
\curveto(476.01878564,186.32295994)(476.07378559,186.26296)(476.11379257,186.17297029)
\curveto(476.13378553,186.14296012)(476.14378552,186.10296016)(476.14379257,186.05297029)
\lineto(476.14379257,185.91797029)
\curveto(476.15378551,185.80796046)(476.14878551,185.70296056)(476.12879257,185.60297029)
\curveto(476.11878554,185.50296076)(476.08378558,185.43296083)(476.02379257,185.39297029)
\curveto(475.93378573,185.32296094)(475.79878586,185.28796098)(475.61879257,185.28797029)
\curveto(475.43878622,185.29796097)(475.27378639,185.30296096)(475.12379257,185.30297029)
\lineto(473.12879257,185.30297029)
\lineto(472.63379257,185.30297029)
\lineto(472.49879257,185.30297029)
\curveto(472.4587892,185.30296096)(472.41878924,185.29796097)(472.37879257,185.28797029)
\lineto(472.16879257,185.28797029)
\curveto(472.0587896,185.25796101)(471.97878968,185.21796105)(471.92879257,185.16797029)
\curveto(471.87878978,185.12796114)(471.84378982,185.07296119)(471.82379257,185.00297029)
\curveto(471.80378986,184.94296132)(471.78878987,184.87296139)(471.77879257,184.79297029)
\curveto(471.76878989,184.71296155)(471.74878991,184.62296164)(471.71879257,184.52297029)
\curveto(471.66878999,184.32296194)(471.62879003,184.11796215)(471.59879257,183.90797029)
\curveto(471.56879009,183.69796257)(471.52879013,183.49296277)(471.47879257,183.29297029)
\curveto(471.4587902,183.22296304)(471.44879021,183.15296311)(471.44879257,183.08297029)
\curveto(471.44879021,183.02296324)(471.43879022,182.95796331)(471.41879257,182.88797029)
\curveto(471.40879025,182.85796341)(471.39879026,182.81796345)(471.38879257,182.76797029)
\curveto(471.38879027,182.72796354)(471.39379027,182.68796358)(471.40379257,182.64797029)
\curveto(471.42379024,182.59796367)(471.44879021,182.55296371)(471.47879257,182.51297029)
\curveto(471.51879014,182.48296378)(471.57879008,182.47796379)(471.65879257,182.49797029)
\curveto(471.71878994,182.51796375)(471.77878988,182.54296372)(471.83879257,182.57297029)
\curveto(471.89878976,182.61296365)(471.9587897,182.64796362)(472.01879257,182.67797029)
\curveto(472.07878958,182.69796357)(472.12878953,182.71296355)(472.16879257,182.72297029)
\curveto(472.3587893,182.80296346)(472.5637891,182.85796341)(472.78379257,182.88797029)
\curveto(473.01378865,182.91796335)(473.24378842,182.92796334)(473.47379257,182.91797029)
\curveto(473.71378795,182.91796335)(473.94378772,182.89296337)(474.16379257,182.84297029)
\curveto(474.38378728,182.80296346)(474.58378708,182.74296352)(474.76379257,182.66297029)
\curveto(474.81378685,182.64296362)(474.8587868,182.62296364)(474.89879257,182.60297029)
\curveto(474.94878671,182.58296368)(474.99878666,182.55796371)(475.04879257,182.52797029)
\curveto(475.39878626,182.31796395)(475.67878598,182.08796418)(475.88879257,181.83797029)
\curveto(476.10878555,181.58796468)(476.30378536,181.262965)(476.47379257,180.86297029)
\curveto(476.52378514,180.75296551)(476.5587851,180.64296562)(476.57879257,180.53297029)
\curveto(476.59878506,180.42296584)(476.62378504,180.30796596)(476.65379257,180.18797029)
\curveto(476.663785,180.15796611)(476.66878499,180.11296615)(476.66879257,180.05297029)
\curveto(476.68878497,179.99296627)(476.69878496,179.92296634)(476.69879257,179.84297029)
\curveto(476.69878496,179.77296649)(476.70878495,179.70796656)(476.72879257,179.64797029)
\lineto(476.72879257,179.48297029)
\curveto(476.73878492,179.43296683)(476.74378492,179.3629669)(476.74379257,179.27297029)
\curveto(476.74378492,179.18296708)(476.73378493,179.11296715)(476.71379257,179.06297029)
\curveto(476.69378497,179.00296726)(476.68878497,178.94296732)(476.69879257,178.88297029)
\curveto(476.70878495,178.83296743)(476.70378496,178.78296748)(476.68379257,178.73297029)
\curveto(476.64378502,178.57296769)(476.60878505,178.42296784)(476.57879257,178.28297029)
\curveto(476.54878511,178.14296812)(476.50378516,178.00796826)(476.44379257,177.87797029)
\curveto(476.28378538,177.50796876)(476.0637856,177.17296909)(475.78379257,176.87297029)
\curveto(475.50378616,176.57296969)(475.18378648,176.34296992)(474.82379257,176.18297029)
\curveto(474.65378701,176.10297016)(474.45378721,176.02797024)(474.22379257,175.95797029)
\curveto(474.11378755,175.91797035)(473.99878766,175.89297037)(473.87879257,175.88297029)
\curveto(473.7587879,175.87297039)(473.63878802,175.85297041)(473.51879257,175.82297029)
\curveto(473.46878819,175.80297046)(473.41378825,175.80297046)(473.35379257,175.82297029)
\curveto(473.29378837,175.83297043)(473.23378843,175.82797044)(473.17379257,175.80797029)
\curveto(473.07378859,175.78797048)(472.97378869,175.78797048)(472.87379257,175.80797029)
\lineto(472.73879257,175.80797029)
\curveto(472.68878897,175.82797044)(472.62878903,175.83797043)(472.55879257,175.83797029)
\curveto(472.49878916,175.82797044)(472.44378922,175.83297043)(472.39379257,175.85297029)
\curveto(472.35378931,175.8629704)(472.31878934,175.8679704)(472.28879257,175.86797029)
\curveto(472.2587894,175.8679704)(472.22378944,175.87297039)(472.18379257,175.88297029)
\lineto(471.91379257,175.94297029)
\curveto(471.82378984,175.9629703)(471.73878992,175.99297027)(471.65879257,176.03297029)
\curveto(471.31879034,176.17297009)(471.02879063,176.32796994)(470.78879257,176.49797029)
\curveto(470.54879111,176.67796959)(470.32879133,176.90796936)(470.12879257,177.18797029)
\curveto(469.97879168,177.41796885)(469.8637918,177.65796861)(469.78379257,177.90797029)
\curveto(469.7637919,177.95796831)(469.75379191,178.00296826)(469.75379257,178.04297029)
\curveto(469.75379191,178.09296817)(469.74379192,178.14296812)(469.72379257,178.19297029)
\curveto(469.70379196,178.25296801)(469.68879197,178.33296793)(469.67879257,178.43297029)
\curveto(469.67879198,178.53296773)(469.69879196,178.60796766)(469.73879257,178.65797029)
\curveto(469.78879187,178.73796753)(469.86879179,178.78296748)(469.97879257,178.79297029)
\curveto(470.08879157,178.80296746)(470.20379146,178.80796746)(470.32379257,178.80797029)
\lineto(470.48879257,178.80797029)
\curveto(470.54879111,178.80796746)(470.60379106,178.79796747)(470.65379257,178.77797029)
\curveto(470.74379092,178.75796751)(470.81379085,178.71796755)(470.86379257,178.65797029)
\curveto(470.93379073,178.5679677)(470.97879068,178.45796781)(470.99879257,178.32797029)
\curveto(471.02879063,178.20796806)(471.07379059,178.10296816)(471.13379257,178.01297029)
\curveto(471.32379034,177.67296859)(471.58379008,177.40296886)(471.91379257,177.20297029)
\curveto(472.01378965,177.14296912)(472.11878954,177.09296917)(472.22879257,177.05297029)
\curveto(472.34878931,177.02296924)(472.46878919,176.98796928)(472.58879257,176.94797029)
\curveto(472.7587889,176.89796937)(472.9637887,176.87796939)(473.20379257,176.88797029)
\curveto(473.45378821,176.90796936)(473.65378801,176.94296932)(473.80379257,176.99297029)
\curveto(474.17378749,177.11296915)(474.4637872,177.27296899)(474.67379257,177.47297029)
\curveto(474.89378677,177.68296858)(475.07378659,177.9629683)(475.21379257,178.31297029)
\curveto(475.2637864,178.41296785)(475.29378637,178.51796775)(475.30379257,178.62797029)
\curveto(475.32378634,178.73796753)(475.34878631,178.85296741)(475.37879257,178.97297029)
\lineto(475.37879257,179.07797029)
\curveto(475.38878627,179.11796715)(475.39378627,179.15796711)(475.39379257,179.19797029)
\curveto(475.40378626,179.22796704)(475.40378626,179.262967)(475.39379257,179.30297029)
\lineto(475.39379257,179.42297029)
\curveto(475.39378627,179.68296658)(475.3637863,179.92796634)(475.30379257,180.15797029)
\curveto(475.19378647,180.50796576)(475.03878662,180.80296546)(474.83879257,181.04297029)
\curveto(474.63878702,181.29296497)(474.37878728,181.48796478)(474.05879257,181.62797029)
\lineto(473.87879257,181.68797029)
\curveto(473.82878783,181.70796456)(473.76878789,181.72796454)(473.69879257,181.74797029)
\curveto(473.64878801,181.7679645)(473.58878807,181.77796449)(473.51879257,181.77797029)
\curveto(473.4587882,181.78796448)(473.39378827,181.80296446)(473.32379257,181.82297029)
\lineto(473.17379257,181.82297029)
\curveto(473.13378853,181.84296442)(473.07878858,181.85296441)(473.00879257,181.85297029)
\curveto(472.94878871,181.85296441)(472.89378877,181.84296442)(472.84379257,181.82297029)
\lineto(472.73879257,181.82297029)
\curveto(472.70878895,181.82296444)(472.67378899,181.81796445)(472.63379257,181.80797029)
\lineto(472.39379257,181.74797029)
\curveto(472.31378935,181.73796453)(472.23378943,181.71796455)(472.15379257,181.68797029)
\curveto(471.91378975,181.58796468)(471.68378998,181.45296481)(471.46379257,181.28297029)
\curveto(471.37379029,181.21296505)(471.28879037,181.13796513)(471.20879257,181.05797029)
\curveto(471.12879053,180.98796528)(471.02879063,180.93296533)(470.90879257,180.89297029)
\curveto(470.81879084,180.8629654)(470.67879098,180.85296541)(470.48879257,180.86297029)
\curveto(470.30879135,180.87296539)(470.18879147,180.89796537)(470.12879257,180.93797029)
\curveto(470.07879158,180.97796529)(470.03879162,181.03796523)(470.00879257,181.11797029)
\curveto(469.98879167,181.19796507)(469.98879167,181.28296498)(470.00879257,181.37297029)
\curveto(470.03879162,181.49296477)(470.0587916,181.61296465)(470.06879257,181.73297029)
\curveto(470.08879157,181.8629644)(470.11379155,181.98796428)(470.14379257,182.10797029)
\curveto(470.1637915,182.14796412)(470.16879149,182.18296408)(470.15879257,182.21297029)
\curveto(470.1587915,182.25296401)(470.16879149,182.29796397)(470.18879257,182.34797029)
\curveto(470.20879145,182.43796383)(470.22379144,182.52796374)(470.23379257,182.61797029)
\curveto(470.24379142,182.71796355)(470.2637914,182.81296345)(470.29379257,182.90297029)
\curveto(470.30379136,182.9629633)(470.30879135,183.02296324)(470.30879257,183.08297029)
\curveto(470.31879134,183.14296312)(470.33379133,183.20296306)(470.35379257,183.26297029)
\curveto(470.40379126,183.4629628)(470.43879122,183.6679626)(470.45879257,183.87797029)
\curveto(470.48879117,184.09796217)(470.52879113,184.30796196)(470.57879257,184.50797029)
\curveto(470.60879105,184.60796166)(470.62879103,184.70796156)(470.63879257,184.80797029)
\curveto(470.64879101,184.90796136)(470.663791,185.00796126)(470.68379257,185.10797029)
\curveto(470.69379097,185.13796113)(470.69879096,185.17796109)(470.69879257,185.22797029)
\curveto(470.72879093,185.33796093)(470.74879091,185.44296082)(470.75879257,185.54297029)
\curveto(470.77879088,185.65296061)(470.80379086,185.7629605)(470.83379257,185.87297029)
\curveto(470.85379081,185.95296031)(470.86879079,186.02296024)(470.87879257,186.08297029)
\curveto(470.88879077,186.15296011)(470.91379075,186.21296005)(470.95379257,186.26297029)
\curveto(470.97379069,186.29295997)(471.00379066,186.31295995)(471.04379257,186.32297029)
\curveto(471.08379058,186.34295992)(471.12879053,186.3629599)(471.17879257,186.38297029)
\curveto(471.23879042,186.38295988)(471.27879038,186.38795988)(471.29879257,186.39797029)
}
}
{
\newrgbcolor{curcolor}{0 0 0}
\pscustom[linestyle=none,fillstyle=solid,fillcolor=curcolor]
{
\newpath
\moveto(487.94340194,184.50797029)
\curveto(487.74339164,184.21796205)(487.53339185,183.93296233)(487.31340194,183.65297029)
\curveto(487.10339228,183.37296289)(486.89839249,183.08796318)(486.69840194,182.79797029)
\curveto(486.09839329,181.94796432)(485.49339389,181.10796516)(484.88340194,180.27797029)
\curveto(484.27339511,179.45796681)(483.66839572,178.62296764)(483.06840194,177.77297029)
\lineto(482.55840194,177.05297029)
\lineto(482.04840194,176.36297029)
\curveto(481.96839742,176.25297001)(481.8883975,176.13797013)(481.80840194,176.01797029)
\curveto(481.72839766,175.89797037)(481.63339775,175.80297046)(481.52340194,175.73297029)
\curveto(481.4833979,175.71297055)(481.41839797,175.69797057)(481.32840194,175.68797029)
\curveto(481.24839814,175.6679706)(481.15839823,175.65797061)(481.05840194,175.65797029)
\curveto(480.95839843,175.65797061)(480.86339852,175.6629706)(480.77340194,175.67297029)
\curveto(480.69339869,175.68297058)(480.63339875,175.70297056)(480.59340194,175.73297029)
\curveto(480.56339882,175.75297051)(480.53839885,175.78797048)(480.51840194,175.83797029)
\curveto(480.50839888,175.87797039)(480.51339887,175.92297034)(480.53340194,175.97297029)
\curveto(480.57339881,176.05297021)(480.61839877,176.12797014)(480.66840194,176.19797029)
\curveto(480.72839866,176.27796999)(480.7833986,176.35796991)(480.83340194,176.43797029)
\curveto(481.07339831,176.77796949)(481.31839807,177.11296915)(481.56840194,177.44297029)
\curveto(481.81839757,177.77296849)(482.05839733,178.10796816)(482.28840194,178.44797029)
\curveto(482.44839694,178.6679676)(482.60839678,178.88296738)(482.76840194,179.09297029)
\curveto(482.92839646,179.30296696)(483.0883963,179.51796675)(483.24840194,179.73797029)
\curveto(483.60839578,180.25796601)(483.97339541,180.7679655)(484.34340194,181.26797029)
\curveto(484.71339467,181.7679645)(485.0833943,182.27796399)(485.45340194,182.79797029)
\curveto(485.59339379,182.99796327)(485.73339365,183.19296307)(485.87340194,183.38297029)
\curveto(486.02339336,183.57296269)(486.16839322,183.7679625)(486.30840194,183.96797029)
\curveto(486.51839287,184.267962)(486.73339265,184.5679617)(486.95340194,184.86797029)
\lineto(487.61340194,185.76797029)
\lineto(487.79340194,186.03797029)
\lineto(488.00340194,186.30797029)
\lineto(488.12340194,186.48797029)
\curveto(488.17339121,186.54795972)(488.22339116,186.60295966)(488.27340194,186.65297029)
\curveto(488.34339104,186.70295956)(488.41839097,186.73795953)(488.49840194,186.75797029)
\curveto(488.51839087,186.7679595)(488.54339084,186.7679595)(488.57340194,186.75797029)
\curveto(488.61339077,186.75795951)(488.64339074,186.7679595)(488.66340194,186.78797029)
\curveto(488.7833906,186.78795948)(488.91839047,186.78295948)(489.06840194,186.77297029)
\curveto(489.21839017,186.77295949)(489.30839008,186.72795954)(489.33840194,186.63797029)
\curveto(489.35839003,186.60795966)(489.36339002,186.57295969)(489.35340194,186.53297029)
\curveto(489.34339004,186.49295977)(489.32839006,186.4629598)(489.30840194,186.44297029)
\curveto(489.26839012,186.3629599)(489.22839016,186.29295997)(489.18840194,186.23297029)
\curveto(489.14839024,186.17296009)(489.10339028,186.11296015)(489.05340194,186.05297029)
\lineto(488.48340194,185.27297029)
\curveto(488.30339108,185.02296124)(488.12339126,184.7679615)(487.94340194,184.50797029)
\moveto(481.08840194,180.60797029)
\curveto(481.03839835,180.62796564)(480.9883984,180.63296563)(480.93840194,180.62297029)
\curveto(480.8883985,180.61296565)(480.83839855,180.61796565)(480.78840194,180.63797029)
\curveto(480.67839871,180.65796561)(480.57339881,180.67796559)(480.47340194,180.69797029)
\curveto(480.383399,180.72796554)(480.2883991,180.7679655)(480.18840194,180.81797029)
\curveto(479.85839953,180.95796531)(479.60339978,181.15296511)(479.42340194,181.40297029)
\curveto(479.24340014,181.6629646)(479.09840029,181.97296429)(478.98840194,182.33297029)
\curveto(478.95840043,182.41296385)(478.93840045,182.49296377)(478.92840194,182.57297029)
\curveto(478.91840047,182.6629636)(478.90340048,182.74796352)(478.88340194,182.82797029)
\curveto(478.87340051,182.87796339)(478.86840052,182.94296332)(478.86840194,183.02297029)
\curveto(478.85840053,183.05296321)(478.85340053,183.08296318)(478.85340194,183.11297029)
\curveto(478.85340053,183.15296311)(478.84840054,183.18796308)(478.83840194,183.21797029)
\lineto(478.83840194,183.36797029)
\curveto(478.82840056,183.41796285)(478.82340056,183.47796279)(478.82340194,183.54797029)
\curveto(478.82340056,183.62796264)(478.82840056,183.69296257)(478.83840194,183.74297029)
\lineto(478.83840194,183.90797029)
\curveto(478.85840053,183.95796231)(478.86340052,184.00296226)(478.85340194,184.04297029)
\curveto(478.85340053,184.09296217)(478.85840053,184.13796213)(478.86840194,184.17797029)
\curveto(478.87840051,184.21796205)(478.8834005,184.25296201)(478.88340194,184.28297029)
\curveto(478.8834005,184.32296194)(478.8884005,184.3629619)(478.89840194,184.40297029)
\curveto(478.92840046,184.51296175)(478.94840044,184.62296164)(478.95840194,184.73297029)
\curveto(478.97840041,184.85296141)(479.01340037,184.9679613)(479.06340194,185.07797029)
\curveto(479.20340018,185.41796085)(479.36340002,185.69296057)(479.54340194,185.90297029)
\curveto(479.73339965,186.12296014)(480.00339938,186.30295996)(480.35340194,186.44297029)
\curveto(480.43339895,186.47295979)(480.51839887,186.49295977)(480.60840194,186.50297029)
\curveto(480.69839869,186.52295974)(480.79339859,186.54295972)(480.89340194,186.56297029)
\curveto(480.92339846,186.57295969)(480.97839841,186.57295969)(481.05840194,186.56297029)
\curveto(481.13839825,186.5629597)(481.1883982,186.57295969)(481.20840194,186.59297029)
\curveto(481.76839762,186.60295966)(482.21839717,186.49295977)(482.55840194,186.26297029)
\curveto(482.90839648,186.03296023)(483.16839622,185.72796054)(483.33840194,185.34797029)
\curveto(483.37839601,185.25796101)(483.41339597,185.1629611)(483.44340194,185.06297029)
\curveto(483.47339591,184.9629613)(483.49839589,184.8629614)(483.51840194,184.76297029)
\curveto(483.53839585,184.73296153)(483.54339584,184.70296156)(483.53340194,184.67297029)
\curveto(483.53339585,184.64296162)(483.53839585,184.61296165)(483.54840194,184.58297029)
\curveto(483.57839581,184.47296179)(483.59839579,184.34796192)(483.60840194,184.20797029)
\curveto(483.61839577,184.07796219)(483.62839576,183.94296232)(483.63840194,183.80297029)
\lineto(483.63840194,183.63797029)
\curveto(483.64839574,183.57796269)(483.64839574,183.52296274)(483.63840194,183.47297029)
\curveto(483.62839576,183.42296284)(483.62339576,183.37296289)(483.62340194,183.32297029)
\lineto(483.62340194,183.18797029)
\curveto(483.61339577,183.14796312)(483.60839578,183.10796316)(483.60840194,183.06797029)
\curveto(483.61839577,183.02796324)(483.61339577,182.98296328)(483.59340194,182.93297029)
\curveto(483.57339581,182.82296344)(483.55339583,182.71796355)(483.53340194,182.61797029)
\curveto(483.52339586,182.51796375)(483.50339588,182.41796385)(483.47340194,182.31797029)
\curveto(483.34339604,181.95796431)(483.17839621,181.64296462)(482.97840194,181.37297029)
\curveto(482.77839661,181.10296516)(482.50339688,180.89796537)(482.15340194,180.75797029)
\curveto(482.07339731,180.72796554)(481.9883974,180.70296556)(481.89840194,180.68297029)
\lineto(481.62840194,180.62297029)
\curveto(481.57839781,180.61296565)(481.53339785,180.60796566)(481.49340194,180.60797029)
\curveto(481.45339793,180.61796565)(481.41339797,180.61796565)(481.37340194,180.60797029)
\curveto(481.27339811,180.58796568)(481.17839821,180.58796568)(481.08840194,180.60797029)
\moveto(480.24840194,182.00297029)
\curveto(480.2883991,181.93296433)(480.32839906,181.8679644)(480.36840194,181.80797029)
\curveto(480.40839898,181.75796451)(480.45839893,181.70796456)(480.51840194,181.65797029)
\lineto(480.66840194,181.53797029)
\curveto(480.72839866,181.50796476)(480.79339859,181.48296478)(480.86340194,181.46297029)
\curveto(480.90339848,181.44296482)(480.93839845,181.43296483)(480.96840194,181.43297029)
\curveto(481.00839838,181.44296482)(481.04839834,181.43796483)(481.08840194,181.41797029)
\curveto(481.11839827,181.41796485)(481.15839823,181.41296485)(481.20840194,181.40297029)
\curveto(481.25839813,181.40296486)(481.29839809,181.40796486)(481.32840194,181.41797029)
\lineto(481.55340194,181.46297029)
\curveto(481.80339758,181.54296472)(481.9883974,181.6679646)(482.10840194,181.83797029)
\curveto(482.1883972,181.93796433)(482.25839713,182.0679642)(482.31840194,182.22797029)
\curveto(482.39839699,182.40796386)(482.45839693,182.63296363)(482.49840194,182.90297029)
\curveto(482.53839685,183.18296308)(482.55339683,183.4629628)(482.54340194,183.74297029)
\curveto(482.53339685,184.03296223)(482.50339688,184.30796196)(482.45340194,184.56797029)
\curveto(482.40339698,184.82796144)(482.32839706,185.03796123)(482.22840194,185.19797029)
\curveto(482.10839728,185.39796087)(481.95839743,185.54796072)(481.77840194,185.64797029)
\curveto(481.69839769,185.69796057)(481.60839778,185.72796054)(481.50840194,185.73797029)
\curveto(481.40839798,185.75796051)(481.30339808,185.7679605)(481.19340194,185.76797029)
\curveto(481.17339821,185.75796051)(481.14839824,185.75296051)(481.11840194,185.75297029)
\curveto(481.09839829,185.7629605)(481.07839831,185.7629605)(481.05840194,185.75297029)
\curveto(481.00839838,185.74296052)(480.96339842,185.73296053)(480.92340194,185.72297029)
\curveto(480.8833985,185.72296054)(480.84339854,185.71296055)(480.80340194,185.69297029)
\curveto(480.62339876,185.61296065)(480.47339891,185.49296077)(480.35340194,185.33297029)
\curveto(480.24339914,185.17296109)(480.15339923,184.99296127)(480.08340194,184.79297029)
\curveto(480.02339936,184.60296166)(479.97839941,184.37796189)(479.94840194,184.11797029)
\curveto(479.92839946,183.85796241)(479.92339946,183.59296267)(479.93340194,183.32297029)
\curveto(479.94339944,183.0629632)(479.97339941,182.81296345)(480.02340194,182.57297029)
\curveto(480.0833993,182.34296392)(480.15839923,182.15296411)(480.24840194,182.00297029)
\moveto(491.04840194,179.01797029)
\curveto(491.05838833,178.9679673)(491.06338832,178.87796739)(491.06340194,178.74797029)
\curveto(491.06338832,178.61796765)(491.05338833,178.52796774)(491.03340194,178.47797029)
\curveto(491.01338837,178.42796784)(491.00838838,178.37296789)(491.01840194,178.31297029)
\curveto(491.02838836,178.262968)(491.02838836,178.21296805)(491.01840194,178.16297029)
\curveto(490.97838841,178.02296824)(490.94838844,177.88796838)(490.92840194,177.75797029)
\curveto(490.91838847,177.62796864)(490.8883885,177.50796876)(490.83840194,177.39797029)
\curveto(490.69838869,177.04796922)(490.53338885,176.75296951)(490.34340194,176.51297029)
\curveto(490.15338923,176.28296998)(489.8833895,176.09797017)(489.53340194,175.95797029)
\curveto(489.45338993,175.92797034)(489.36839002,175.90797036)(489.27840194,175.89797029)
\curveto(489.1883902,175.87797039)(489.10339028,175.85797041)(489.02340194,175.83797029)
\curveto(488.97339041,175.82797044)(488.92339046,175.82297044)(488.87340194,175.82297029)
\curveto(488.82339056,175.82297044)(488.77339061,175.81797045)(488.72340194,175.80797029)
\curveto(488.69339069,175.79797047)(488.64339074,175.79797047)(488.57340194,175.80797029)
\curveto(488.50339088,175.80797046)(488.45339093,175.81297045)(488.42340194,175.82297029)
\curveto(488.36339102,175.84297042)(488.30339108,175.85297041)(488.24340194,175.85297029)
\curveto(488.19339119,175.84297042)(488.14339124,175.84797042)(488.09340194,175.86797029)
\curveto(488.00339138,175.88797038)(487.91339147,175.91297035)(487.82340194,175.94297029)
\curveto(487.74339164,175.9629703)(487.66339172,175.99297027)(487.58340194,176.03297029)
\curveto(487.26339212,176.17297009)(487.01339237,176.3679699)(486.83340194,176.61797029)
\curveto(486.65339273,176.87796939)(486.50339288,177.18296908)(486.38340194,177.53297029)
\curveto(486.36339302,177.61296865)(486.34839304,177.69796857)(486.33840194,177.78797029)
\curveto(486.32839306,177.87796839)(486.31339307,177.9629683)(486.29340194,178.04297029)
\curveto(486.2833931,178.07296819)(486.27839311,178.10296816)(486.27840194,178.13297029)
\lineto(486.27840194,178.23797029)
\curveto(486.25839313,178.31796795)(486.24839314,178.39796787)(486.24840194,178.47797029)
\lineto(486.24840194,178.61297029)
\curveto(486.22839316,178.71296755)(486.22839316,178.81296745)(486.24840194,178.91297029)
\lineto(486.24840194,179.09297029)
\curveto(486.25839313,179.14296712)(486.26339312,179.18796708)(486.26340194,179.22797029)
\curveto(486.26339312,179.27796699)(486.26839312,179.32296694)(486.27840194,179.36297029)
\curveto(486.2883931,179.40296686)(486.29339309,179.43796683)(486.29340194,179.46797029)
\curveto(486.29339309,179.50796676)(486.29839309,179.54796672)(486.30840194,179.58797029)
\lineto(486.36840194,179.91797029)
\curveto(486.388393,180.03796623)(486.41839297,180.14796612)(486.45840194,180.24797029)
\curveto(486.59839279,180.57796569)(486.75839263,180.85296541)(486.93840194,181.07297029)
\curveto(487.12839226,181.30296496)(487.388392,181.48796478)(487.71840194,181.62797029)
\curveto(487.79839159,181.6679646)(487.8833915,181.69296457)(487.97340194,181.70297029)
\lineto(488.27340194,181.76297029)
\lineto(488.40840194,181.76297029)
\curveto(488.45839093,181.77296449)(488.50839088,181.77796449)(488.55840194,181.77797029)
\curveto(489.12839026,181.79796447)(489.5883898,181.69296457)(489.93840194,181.46297029)
\curveto(490.29838909,181.24296502)(490.56338882,180.94296532)(490.73340194,180.56297029)
\curveto(490.7833886,180.4629658)(490.82338856,180.3629659)(490.85340194,180.26297029)
\curveto(490.8833885,180.1629661)(490.91338847,180.05796621)(490.94340194,179.94797029)
\curveto(490.95338843,179.90796636)(490.95838843,179.87296639)(490.95840194,179.84297029)
\curveto(490.95838843,179.82296644)(490.96338842,179.79296647)(490.97340194,179.75297029)
\curveto(490.99338839,179.68296658)(491.00338838,179.60796666)(491.00340194,179.52797029)
\curveto(491.00338838,179.44796682)(491.01338837,179.3679669)(491.03340194,179.28797029)
\curveto(491.03338835,179.23796703)(491.03338835,179.19296707)(491.03340194,179.15297029)
\curveto(491.03338835,179.11296715)(491.03838835,179.0679672)(491.04840194,179.01797029)
\moveto(489.93840194,178.58297029)
\curveto(489.94838944,178.63296763)(489.95338943,178.70796756)(489.95340194,178.80797029)
\curveto(489.96338942,178.90796736)(489.95838943,178.98296728)(489.93840194,179.03297029)
\curveto(489.91838947,179.09296717)(489.91338947,179.14796712)(489.92340194,179.19797029)
\curveto(489.94338944,179.25796701)(489.94338944,179.31796695)(489.92340194,179.37797029)
\curveto(489.91338947,179.40796686)(489.90838948,179.44296682)(489.90840194,179.48297029)
\curveto(489.90838948,179.52296674)(489.90338948,179.5629667)(489.89340194,179.60297029)
\curveto(489.87338951,179.68296658)(489.85338953,179.75796651)(489.83340194,179.82797029)
\curveto(489.82338956,179.90796636)(489.80838958,179.98796628)(489.78840194,180.06797029)
\curveto(489.75838963,180.12796614)(489.73338965,180.18796608)(489.71340194,180.24797029)
\curveto(489.69338969,180.30796596)(489.66338972,180.3679659)(489.62340194,180.42797029)
\curveto(489.52338986,180.59796567)(489.39338999,180.73296553)(489.23340194,180.83297029)
\curveto(489.15339023,180.88296538)(489.05839033,180.91796535)(488.94840194,180.93797029)
\curveto(488.83839055,180.95796531)(488.71339067,180.9679653)(488.57340194,180.96797029)
\curveto(488.55339083,180.95796531)(488.52839086,180.95296531)(488.49840194,180.95297029)
\curveto(488.46839092,180.9629653)(488.43839095,180.9629653)(488.40840194,180.95297029)
\lineto(488.25840194,180.89297029)
\curveto(488.20839118,180.88296538)(488.16339122,180.8679654)(488.12340194,180.84797029)
\curveto(487.93339145,180.73796553)(487.7883916,180.59296567)(487.68840194,180.41297029)
\curveto(487.59839179,180.23296603)(487.51839187,180.02796624)(487.44840194,179.79797029)
\curveto(487.40839198,179.6679666)(487.388392,179.53296673)(487.38840194,179.39297029)
\curveto(487.388392,179.262967)(487.37839201,179.11796715)(487.35840194,178.95797029)
\curveto(487.34839204,178.90796736)(487.33839205,178.84796742)(487.32840194,178.77797029)
\curveto(487.32839206,178.70796756)(487.33839205,178.64796762)(487.35840194,178.59797029)
\lineto(487.35840194,178.43297029)
\lineto(487.35840194,178.25297029)
\curveto(487.36839202,178.20296806)(487.37839201,178.14796812)(487.38840194,178.08797029)
\curveto(487.39839199,178.03796823)(487.40339198,177.98296828)(487.40340194,177.92297029)
\curveto(487.41339197,177.8629684)(487.42839196,177.80796846)(487.44840194,177.75797029)
\curveto(487.49839189,177.5679687)(487.55839183,177.39296887)(487.62840194,177.23297029)
\curveto(487.69839169,177.07296919)(487.80339158,176.94296932)(487.94340194,176.84297029)
\curveto(488.07339131,176.74296952)(488.21339117,176.67296959)(488.36340194,176.63297029)
\curveto(488.39339099,176.62296964)(488.41839097,176.61796965)(488.43840194,176.61797029)
\curveto(488.46839092,176.62796964)(488.49839089,176.62796964)(488.52840194,176.61797029)
\curveto(488.54839084,176.61796965)(488.57839081,176.61296965)(488.61840194,176.60297029)
\curveto(488.65839073,176.60296966)(488.69339069,176.60796966)(488.72340194,176.61797029)
\curveto(488.76339062,176.62796964)(488.80339058,176.63296963)(488.84340194,176.63297029)
\curveto(488.8833905,176.63296963)(488.92339046,176.64296962)(488.96340194,176.66297029)
\curveto(489.20339018,176.74296952)(489.39838999,176.87796939)(489.54840194,177.06797029)
\curveto(489.66838972,177.24796902)(489.75838963,177.45296881)(489.81840194,177.68297029)
\curveto(489.83838955,177.75296851)(489.85338953,177.82296844)(489.86340194,177.89297029)
\curveto(489.87338951,177.97296829)(489.8883895,178.05296821)(489.90840194,178.13297029)
\curveto(489.90838948,178.19296807)(489.91338947,178.23796803)(489.92340194,178.26797029)
\curveto(489.92338946,178.28796798)(489.92338946,178.31296795)(489.92340194,178.34297029)
\curveto(489.92338946,178.38296788)(489.92838946,178.41296785)(489.93840194,178.43297029)
\lineto(489.93840194,178.58297029)
}
}
{
\newrgbcolor{curcolor}{0 0 0}
\pscustom[linestyle=none,fillstyle=solid,fillcolor=curcolor]
{
\newpath
\moveto(646.12869735,249.72749178)
\curveto(646.81869272,249.73748115)(647.41869212,249.61748127)(647.92869735,249.36749178)
\curveto(648.44869109,249.11748177)(648.84369069,248.7824821)(649.11369735,248.36249178)
\curveto(649.16369037,248.2824826)(649.20869033,248.19248269)(649.24869735,248.09249178)
\curveto(649.28869025,248.00248288)(649.3336902,247.90748298)(649.38369735,247.80749178)
\curveto(649.42369011,247.70748318)(649.45369008,247.60748328)(649.47369735,247.50749178)
\curveto(649.49369004,247.40748348)(649.51369002,247.30248358)(649.53369735,247.19249178)
\curveto(649.55368998,247.14248374)(649.55868998,247.09748379)(649.54869735,247.05749178)
\curveto(649.53869,247.01748387)(649.54368999,246.97248391)(649.56369735,246.92249178)
\curveto(649.57368996,246.87248401)(649.57868996,246.7874841)(649.57869735,246.66749178)
\curveto(649.57868996,246.55748433)(649.57368996,246.47248441)(649.56369735,246.41249178)
\curveto(649.54368999,246.35248453)(649.53369,246.29248459)(649.53369735,246.23249178)
\curveto(649.54368999,246.17248471)(649.53869,246.11248477)(649.51869735,246.05249178)
\curveto(649.47869006,245.91248497)(649.44369009,245.77748511)(649.41369735,245.64749178)
\curveto(649.38369015,245.51748537)(649.34369019,245.39248549)(649.29369735,245.27249178)
\curveto(649.2336903,245.13248575)(649.16369037,245.00748588)(649.08369735,244.89749178)
\curveto(649.01369052,244.7874861)(648.9386906,244.67748621)(648.85869735,244.56749178)
\lineto(648.79869735,244.50749178)
\curveto(648.78869075,244.4874864)(648.77369076,244.46748642)(648.75369735,244.44749178)
\curveto(648.6336909,244.2874866)(648.49869104,244.14248674)(648.34869735,244.01249178)
\curveto(648.19869134,243.882487)(648.0386915,243.75748713)(647.86869735,243.63749178)
\curveto(647.55869198,243.41748747)(647.26369227,243.21248767)(646.98369735,243.02249178)
\curveto(646.75369278,242.882488)(646.52369301,242.74748814)(646.29369735,242.61749178)
\curveto(646.07369346,242.4874884)(645.85369368,242.35248853)(645.63369735,242.21249178)
\curveto(645.38369415,242.04248884)(645.14369439,241.86248902)(644.91369735,241.67249178)
\curveto(644.69369484,241.4824894)(644.50369503,241.25748963)(644.34369735,240.99749178)
\curveto(644.30369523,240.93748995)(644.26869527,240.87749001)(644.23869735,240.81749178)
\curveto(644.20869533,240.76749012)(644.17869536,240.70249018)(644.14869735,240.62249178)
\curveto(644.12869541,240.55249033)(644.12369541,240.49249039)(644.13369735,240.44249178)
\curveto(644.15369538,240.37249051)(644.18869535,240.31749057)(644.23869735,240.27749178)
\curveto(644.28869525,240.24749064)(644.34869519,240.22749066)(644.41869735,240.21749178)
\lineto(644.65869735,240.21749178)
\lineto(645.40869735,240.21749178)
\lineto(648.21369735,240.21749178)
\lineto(648.87369735,240.21749178)
\curveto(648.96369057,240.21749067)(649.04869049,240.21249067)(649.12869735,240.20249178)
\curveto(649.20869033,240.20249068)(649.27369026,240.1824907)(649.32369735,240.14249178)
\curveto(649.37369016,240.10249078)(649.41369012,240.02749086)(649.44369735,239.91749178)
\curveto(649.48369005,239.81749107)(649.49369004,239.71749117)(649.47369735,239.61749178)
\lineto(649.47369735,239.48249178)
\curveto(649.45369008,239.41249147)(649.4336901,239.35249153)(649.41369735,239.30249178)
\curveto(649.39369014,239.25249163)(649.35869018,239.21249167)(649.30869735,239.18249178)
\curveto(649.25869028,239.14249174)(649.18869035,239.12249176)(649.09869735,239.12249178)
\lineto(648.82869735,239.12249178)
\lineto(647.92869735,239.12249178)
\lineto(644.41869735,239.12249178)
\lineto(643.35369735,239.12249178)
\curveto(643.27369626,239.12249176)(643.18369635,239.11749177)(643.08369735,239.10749178)
\curveto(642.98369655,239.10749178)(642.89869664,239.11749177)(642.82869735,239.13749178)
\curveto(642.61869692,239.20749168)(642.55369698,239.3874915)(642.63369735,239.67749178)
\curveto(642.64369689,239.71749117)(642.64369689,239.75249113)(642.63369735,239.78249178)
\curveto(642.6336969,239.82249106)(642.64369689,239.86749102)(642.66369735,239.91749178)
\curveto(642.68369685,239.99749089)(642.70369683,240.0824908)(642.72369735,240.17249178)
\curveto(642.74369679,240.26249062)(642.76869677,240.34749054)(642.79869735,240.42749178)
\curveto(642.95869658,240.91748997)(643.15869638,241.33248955)(643.39869735,241.67249178)
\curveto(643.57869596,241.92248896)(643.78369575,242.14748874)(644.01369735,242.34749178)
\curveto(644.24369529,242.55748833)(644.48369505,242.75248813)(644.73369735,242.93249178)
\curveto(644.99369454,243.11248777)(645.25869428,243.2824876)(645.52869735,243.44249178)
\curveto(645.80869373,243.61248727)(646.07869346,243.7874871)(646.33869735,243.96749178)
\curveto(646.44869309,244.04748684)(646.55369298,244.12248676)(646.65369735,244.19249178)
\curveto(646.76369277,244.26248662)(646.87369266,244.33748655)(646.98369735,244.41749178)
\curveto(647.02369251,244.44748644)(647.05869248,244.47748641)(647.08869735,244.50749178)
\curveto(647.12869241,244.54748634)(647.16869237,244.57748631)(647.20869735,244.59749178)
\curveto(647.34869219,244.70748618)(647.47369206,244.83248605)(647.58369735,244.97249178)
\curveto(647.60369193,245.00248588)(647.62869191,245.02748586)(647.65869735,245.04749178)
\curveto(647.68869185,245.07748581)(647.71369182,245.10748578)(647.73369735,245.13749178)
\curveto(647.81369172,245.23748565)(647.87869166,245.33748555)(647.92869735,245.43749178)
\curveto(647.98869155,245.53748535)(648.04369149,245.64748524)(648.09369735,245.76749178)
\curveto(648.12369141,245.83748505)(648.14369139,245.91248497)(648.15369735,245.99249178)
\lineto(648.21369735,246.23249178)
\lineto(648.21369735,246.32249178)
\curveto(648.22369131,246.35248453)(648.22869131,246.3824845)(648.22869735,246.41249178)
\curveto(648.24869129,246.4824844)(648.25369128,246.57748431)(648.24369735,246.69749178)
\curveto(648.24369129,246.82748406)(648.2336913,246.92748396)(648.21369735,246.99749178)
\curveto(648.19369134,247.07748381)(648.17369136,247.15248373)(648.15369735,247.22249178)
\curveto(648.14369139,247.30248358)(648.12369141,247.3824835)(648.09369735,247.46249178)
\curveto(647.98369155,247.70248318)(647.8336917,247.90248298)(647.64369735,248.06249178)
\curveto(647.46369207,248.23248265)(647.24369229,248.37248251)(646.98369735,248.48249178)
\curveto(646.91369262,248.50248238)(646.84369269,248.51748237)(646.77369735,248.52749178)
\curveto(646.70369283,248.54748234)(646.62869291,248.56748232)(646.54869735,248.58749178)
\curveto(646.46869307,248.60748228)(646.35869318,248.61748227)(646.21869735,248.61749178)
\curveto(646.08869345,248.61748227)(645.98369355,248.60748228)(645.90369735,248.58749178)
\curveto(645.84369369,248.57748231)(645.78869375,248.57248231)(645.73869735,248.57249178)
\curveto(645.68869385,248.57248231)(645.6386939,248.56248232)(645.58869735,248.54249178)
\curveto(645.48869405,248.50248238)(645.39369414,248.46248242)(645.30369735,248.42249178)
\curveto(645.22369431,248.3824825)(645.14369439,248.33748255)(645.06369735,248.28749178)
\curveto(645.0336945,248.26748262)(645.00369453,248.24248264)(644.97369735,248.21249178)
\curveto(644.95369458,248.1824827)(644.92869461,248.15748273)(644.89869735,248.13749178)
\lineto(644.82369735,248.06249178)
\curveto(644.79369474,248.04248284)(644.76869477,248.02248286)(644.74869735,248.00249178)
\lineto(644.59869735,247.79249178)
\curveto(644.55869498,247.73248315)(644.51369502,247.66748322)(644.46369735,247.59749178)
\curveto(644.40369513,247.50748338)(644.35369518,247.40248348)(644.31369735,247.28249178)
\curveto(644.28369525,247.17248371)(644.24869529,247.06248382)(644.20869735,246.95249178)
\curveto(644.16869537,246.84248404)(644.14369539,246.69748419)(644.13369735,246.51749178)
\curveto(644.12369541,246.34748454)(644.09369544,246.22248466)(644.04369735,246.14249178)
\curveto(643.99369554,246.06248482)(643.91869562,246.01748487)(643.81869735,246.00749178)
\curveto(643.71869582,245.99748489)(643.60869593,245.99248489)(643.48869735,245.99249178)
\curveto(643.44869609,245.99248489)(643.40869613,245.9874849)(643.36869735,245.97749178)
\curveto(643.32869621,245.97748491)(643.29369624,245.9824849)(643.26369735,245.99249178)
\curveto(643.21369632,246.01248487)(643.16369637,246.02248486)(643.11369735,246.02249178)
\curveto(643.07369646,246.02248486)(643.0336965,246.03248485)(642.99369735,246.05249178)
\curveto(642.90369663,246.11248477)(642.85869668,246.24748464)(642.85869735,246.45749178)
\lineto(642.85869735,246.57749178)
\curveto(642.86869667,246.63748425)(642.87369666,246.69748419)(642.87369735,246.75749178)
\curveto(642.88369665,246.82748406)(642.89369664,246.89248399)(642.90369735,246.95249178)
\curveto(642.92369661,247.06248382)(642.94369659,247.16248372)(642.96369735,247.25249178)
\curveto(642.98369655,247.35248353)(643.01369652,247.44748344)(643.05369735,247.53749178)
\curveto(643.07369646,247.60748328)(643.09369644,247.66748322)(643.11369735,247.71749178)
\lineto(643.17369735,247.89749178)
\curveto(643.29369624,248.15748273)(643.44869609,248.40248248)(643.63869735,248.63249178)
\curveto(643.8386957,248.86248202)(644.05369548,249.04748184)(644.28369735,249.18749178)
\curveto(644.39369514,249.26748162)(644.50869503,249.33248155)(644.62869735,249.38249178)
\lineto(645.01869735,249.53249178)
\curveto(645.12869441,249.5824813)(645.24369429,249.61248127)(645.36369735,249.62249178)
\curveto(645.48369405,249.64248124)(645.60869393,249.66748122)(645.73869735,249.69749178)
\curveto(645.80869373,249.69748119)(645.87369366,249.69748119)(645.93369735,249.69749178)
\curveto(645.99369354,249.70748118)(646.05869348,249.71748117)(646.12869735,249.72749178)
}
}
{
\newrgbcolor{curcolor}{0 0 0}
\pscustom[linestyle=none,fillstyle=solid,fillcolor=curcolor]
{
\newpath
\moveto(652.73830673,249.53249178)
\lineto(656.33830673,249.53249178)
\lineto(656.98330673,249.53249178)
\curveto(657.0633002,249.53248135)(657.13830012,249.52748136)(657.20830673,249.51749178)
\curveto(657.27829998,249.51748137)(657.33829992,249.50748138)(657.38830673,249.48749178)
\curveto(657.4582998,249.45748143)(657.51329975,249.39748149)(657.55330673,249.30749178)
\curveto(657.57329969,249.27748161)(657.58329968,249.23748165)(657.58330673,249.18749178)
\lineto(657.58330673,249.05249178)
\curveto(657.59329967,248.94248194)(657.58829967,248.83748205)(657.56830673,248.73749178)
\curveto(657.5582997,248.63748225)(657.52329974,248.56748232)(657.46330673,248.52749178)
\curveto(657.37329989,248.45748243)(657.23830002,248.42248246)(657.05830673,248.42249178)
\curveto(656.87830038,248.43248245)(656.71330055,248.43748245)(656.56330673,248.43749178)
\lineto(654.56830673,248.43749178)
\lineto(654.07330673,248.43749178)
\lineto(653.93830673,248.43749178)
\curveto(653.89830336,248.43748245)(653.8583034,248.43248245)(653.81830673,248.42249178)
\lineto(653.60830673,248.42249178)
\curveto(653.49830376,248.39248249)(653.41830384,248.35248253)(653.36830673,248.30249178)
\curveto(653.31830394,248.26248262)(653.28330398,248.20748268)(653.26330673,248.13749178)
\curveto(653.24330402,248.07748281)(653.22830403,248.00748288)(653.21830673,247.92749178)
\curveto(653.20830405,247.84748304)(653.18830407,247.75748313)(653.15830673,247.65749178)
\curveto(653.10830415,247.45748343)(653.06830419,247.25248363)(653.03830673,247.04249178)
\curveto(653.00830425,246.83248405)(652.96830429,246.62748426)(652.91830673,246.42749178)
\curveto(652.89830436,246.35748453)(652.88830437,246.2874846)(652.88830673,246.21749178)
\curveto(652.88830437,246.15748473)(652.87830438,246.09248479)(652.85830673,246.02249178)
\curveto(652.84830441,245.99248489)(652.83830442,245.95248493)(652.82830673,245.90249178)
\curveto(652.82830443,245.86248502)(652.83330443,245.82248506)(652.84330673,245.78249178)
\curveto(652.8633044,245.73248515)(652.88830437,245.6874852)(652.91830673,245.64749178)
\curveto(652.9583043,245.61748527)(653.01830424,245.61248527)(653.09830673,245.63249178)
\curveto(653.1583041,245.65248523)(653.21830404,245.67748521)(653.27830673,245.70749178)
\curveto(653.33830392,245.74748514)(653.39830386,245.7824851)(653.45830673,245.81249178)
\curveto(653.51830374,245.83248505)(653.56830369,245.84748504)(653.60830673,245.85749178)
\curveto(653.79830346,245.93748495)(654.00330326,245.99248489)(654.22330673,246.02249178)
\curveto(654.45330281,246.05248483)(654.68330258,246.06248482)(654.91330673,246.05249178)
\curveto(655.15330211,246.05248483)(655.38330188,246.02748486)(655.60330673,245.97749178)
\curveto(655.82330144,245.93748495)(656.02330124,245.87748501)(656.20330673,245.79749178)
\curveto(656.25330101,245.77748511)(656.29830096,245.75748513)(656.33830673,245.73749178)
\curveto(656.38830087,245.71748517)(656.43830082,245.69248519)(656.48830673,245.66249178)
\curveto(656.83830042,245.45248543)(657.11830014,245.22248566)(657.32830673,244.97249178)
\curveto(657.54829971,244.72248616)(657.74329952,244.39748649)(657.91330673,243.99749178)
\curveto(657.9632993,243.887487)(657.99829926,243.77748711)(658.01830673,243.66749178)
\curveto(658.03829922,243.55748733)(658.0632992,243.44248744)(658.09330673,243.32249178)
\curveto(658.10329916,243.29248759)(658.10829915,243.24748764)(658.10830673,243.18749178)
\curveto(658.12829913,243.12748776)(658.13829912,243.05748783)(658.13830673,242.97749178)
\curveto(658.13829912,242.90748798)(658.14829911,242.84248804)(658.16830673,242.78249178)
\lineto(658.16830673,242.61749178)
\curveto(658.17829908,242.56748832)(658.18329908,242.49748839)(658.18330673,242.40749178)
\curveto(658.18329908,242.31748857)(658.17329909,242.24748864)(658.15330673,242.19749178)
\curveto(658.13329913,242.13748875)(658.12829913,242.07748881)(658.13830673,242.01749178)
\curveto(658.14829911,241.96748892)(658.14329912,241.91748897)(658.12330673,241.86749178)
\curveto(658.08329918,241.70748918)(658.04829921,241.55748933)(658.01830673,241.41749178)
\curveto(657.98829927,241.27748961)(657.94329932,241.14248974)(657.88330673,241.01249178)
\curveto(657.72329954,240.64249024)(657.50329976,240.30749058)(657.22330673,240.00749178)
\curveto(656.94330032,239.70749118)(656.62330064,239.47749141)(656.26330673,239.31749178)
\curveto(656.09330117,239.23749165)(655.89330137,239.16249172)(655.66330673,239.09249178)
\curveto(655.55330171,239.05249183)(655.43830182,239.02749186)(655.31830673,239.01749178)
\curveto(655.19830206,239.00749188)(655.07830218,238.9874919)(654.95830673,238.95749178)
\curveto(654.90830235,238.93749195)(654.85330241,238.93749195)(654.79330673,238.95749178)
\curveto(654.73330253,238.96749192)(654.67330259,238.96249192)(654.61330673,238.94249178)
\curveto(654.51330275,238.92249196)(654.41330285,238.92249196)(654.31330673,238.94249178)
\lineto(654.17830673,238.94249178)
\curveto(654.12830313,238.96249192)(654.06830319,238.97249191)(653.99830673,238.97249178)
\curveto(653.93830332,238.96249192)(653.88330338,238.96749192)(653.83330673,238.98749178)
\curveto(653.79330347,238.99749189)(653.7583035,239.00249188)(653.72830673,239.00249178)
\curveto(653.69830356,239.00249188)(653.6633036,239.00749188)(653.62330673,239.01749178)
\lineto(653.35330673,239.07749178)
\curveto(653.263304,239.09749179)(653.17830408,239.12749176)(653.09830673,239.16749178)
\curveto(652.7583045,239.30749158)(652.46830479,239.46249142)(652.22830673,239.63249178)
\curveto(651.98830527,239.81249107)(651.76830549,240.04249084)(651.56830673,240.32249178)
\curveto(651.41830584,240.55249033)(651.30330596,240.79249009)(651.22330673,241.04249178)
\curveto(651.20330606,241.09248979)(651.19330607,241.13748975)(651.19330673,241.17749178)
\curveto(651.19330607,241.22748966)(651.18330608,241.27748961)(651.16330673,241.32749178)
\curveto(651.14330612,241.3874895)(651.12830613,241.46748942)(651.11830673,241.56749178)
\curveto(651.11830614,241.66748922)(651.13830612,241.74248914)(651.17830673,241.79249178)
\curveto(651.22830603,241.87248901)(651.30830595,241.91748897)(651.41830673,241.92749178)
\curveto(651.52830573,241.93748895)(651.64330562,241.94248894)(651.76330673,241.94249178)
\lineto(651.92830673,241.94249178)
\curveto(651.98830527,241.94248894)(652.04330522,241.93248895)(652.09330673,241.91249178)
\curveto(652.18330508,241.89248899)(652.25330501,241.85248903)(652.30330673,241.79249178)
\curveto(652.37330489,241.70248918)(652.41830484,241.59248929)(652.43830673,241.46249178)
\curveto(652.46830479,241.34248954)(652.51330475,241.23748965)(652.57330673,241.14749178)
\curveto(652.7633045,240.80749008)(653.02330424,240.53749035)(653.35330673,240.33749178)
\curveto(653.45330381,240.27749061)(653.5583037,240.22749066)(653.66830673,240.18749178)
\curveto(653.78830347,240.15749073)(653.90830335,240.12249076)(654.02830673,240.08249178)
\curveto(654.19830306,240.03249085)(654.40330286,240.01249087)(654.64330673,240.02249178)
\curveto(654.89330237,240.04249084)(655.09330217,240.07749081)(655.24330673,240.12749178)
\curveto(655.61330165,240.24749064)(655.90330136,240.40749048)(656.11330673,240.60749178)
\curveto(656.33330093,240.81749007)(656.51330075,241.09748979)(656.65330673,241.44749178)
\curveto(656.70330056,241.54748934)(656.73330053,241.65248923)(656.74330673,241.76249178)
\curveto(656.7633005,241.87248901)(656.78830047,241.9874889)(656.81830673,242.10749178)
\lineto(656.81830673,242.21249178)
\curveto(656.82830043,242.25248863)(656.83330043,242.29248859)(656.83330673,242.33249178)
\curveto(656.84330042,242.36248852)(656.84330042,242.39748849)(656.83330673,242.43749178)
\lineto(656.83330673,242.55749178)
\curveto(656.83330043,242.81748807)(656.80330046,243.06248782)(656.74330673,243.29249178)
\curveto(656.63330063,243.64248724)(656.47830078,243.93748695)(656.27830673,244.17749178)
\curveto(656.07830118,244.42748646)(655.81830144,244.62248626)(655.49830673,244.76249178)
\lineto(655.31830673,244.82249178)
\curveto(655.26830199,244.84248604)(655.20830205,244.86248602)(655.13830673,244.88249178)
\curveto(655.08830217,244.90248598)(655.02830223,244.91248597)(654.95830673,244.91249178)
\curveto(654.89830236,244.92248596)(654.83330243,244.93748595)(654.76330673,244.95749178)
\lineto(654.61330673,244.95749178)
\curveto(654.57330269,244.97748591)(654.51830274,244.9874859)(654.44830673,244.98749178)
\curveto(654.38830287,244.9874859)(654.33330293,244.97748591)(654.28330673,244.95749178)
\lineto(654.17830673,244.95749178)
\curveto(654.14830311,244.95748593)(654.11330315,244.95248593)(654.07330673,244.94249178)
\lineto(653.83330673,244.88249178)
\curveto(653.75330351,244.87248601)(653.67330359,244.85248603)(653.59330673,244.82249178)
\curveto(653.35330391,244.72248616)(653.12330414,244.5874863)(652.90330673,244.41749178)
\curveto(652.81330445,244.34748654)(652.72830453,244.27248661)(652.64830673,244.19249178)
\curveto(652.56830469,244.12248676)(652.46830479,244.06748682)(652.34830673,244.02749178)
\curveto(652.258305,243.99748689)(652.11830514,243.9874869)(651.92830673,243.99749178)
\curveto(651.74830551,244.00748688)(651.62830563,244.03248685)(651.56830673,244.07249178)
\curveto(651.51830574,244.11248677)(651.47830578,244.17248671)(651.44830673,244.25249178)
\curveto(651.42830583,244.33248655)(651.42830583,244.41748647)(651.44830673,244.50749178)
\curveto(651.47830578,244.62748626)(651.49830576,244.74748614)(651.50830673,244.86749178)
\curveto(651.52830573,244.99748589)(651.55330571,245.12248576)(651.58330673,245.24249178)
\curveto(651.60330566,245.2824856)(651.60830565,245.31748557)(651.59830673,245.34749178)
\curveto(651.59830566,245.3874855)(651.60830565,245.43248545)(651.62830673,245.48249178)
\curveto(651.64830561,245.57248531)(651.6633056,245.66248522)(651.67330673,245.75249178)
\curveto(651.68330558,245.85248503)(651.70330556,245.94748494)(651.73330673,246.03749178)
\curveto(651.74330552,246.09748479)(651.74830551,246.15748473)(651.74830673,246.21749178)
\curveto(651.7583055,246.27748461)(651.77330549,246.33748455)(651.79330673,246.39749178)
\curveto(651.84330542,246.59748429)(651.87830538,246.80248408)(651.89830673,247.01249178)
\curveto(651.92830533,247.23248365)(651.96830529,247.44248344)(652.01830673,247.64249178)
\curveto(652.04830521,247.74248314)(652.06830519,247.84248304)(652.07830673,247.94249178)
\curveto(652.08830517,248.04248284)(652.10330516,248.14248274)(652.12330673,248.24249178)
\curveto(652.13330513,248.27248261)(652.13830512,248.31248257)(652.13830673,248.36249178)
\curveto(652.16830509,248.47248241)(652.18830507,248.57748231)(652.19830673,248.67749178)
\curveto(652.21830504,248.7874821)(652.24330502,248.89748199)(652.27330673,249.00749178)
\curveto(652.29330497,249.0874818)(652.30830495,249.15748173)(652.31830673,249.21749178)
\curveto(652.32830493,249.2874816)(652.35330491,249.34748154)(652.39330673,249.39749178)
\curveto(652.41330485,249.42748146)(652.44330482,249.44748144)(652.48330673,249.45749178)
\curveto(652.52330474,249.47748141)(652.56830469,249.49748139)(652.61830673,249.51749178)
\curveto(652.67830458,249.51748137)(652.71830454,249.52248136)(652.73830673,249.53249178)
}
}
{
\newrgbcolor{curcolor}{0 0 0}
\pscustom[linestyle=none,fillstyle=solid,fillcolor=curcolor]
{
\newpath
\moveto(660.5329161,240.75749178)
\lineto(660.8329161,240.75749178)
\curveto(660.94291404,240.76749012)(661.04791394,240.76749012)(661.1479161,240.75749178)
\curveto(661.25791373,240.75749013)(661.35791363,240.74749014)(661.4479161,240.72749178)
\curveto(661.53791345,240.71749017)(661.60791338,240.69249019)(661.6579161,240.65249178)
\curveto(661.67791331,240.63249025)(661.69291329,240.60249028)(661.7029161,240.56249178)
\curveto(661.72291326,240.52249036)(661.74291324,240.47749041)(661.7629161,240.42749178)
\lineto(661.7629161,240.35249178)
\curveto(661.77291321,240.30249058)(661.77291321,240.24749064)(661.7629161,240.18749178)
\lineto(661.7629161,240.03749178)
\lineto(661.7629161,239.55749178)
\curveto(661.76291322,239.3874915)(661.72291326,239.26749162)(661.6429161,239.19749178)
\curveto(661.57291341,239.14749174)(661.4829135,239.12249176)(661.3729161,239.12249178)
\lineto(661.0429161,239.12249178)
\lineto(660.5929161,239.12249178)
\curveto(660.44291454,239.12249176)(660.32791466,239.15249173)(660.2479161,239.21249178)
\curveto(660.20791478,239.24249164)(660.17791481,239.29249159)(660.1579161,239.36249178)
\curveto(660.13791485,239.44249144)(660.12291486,239.52749136)(660.1129161,239.61749178)
\lineto(660.1129161,239.90249178)
\curveto(660.12291486,240.00249088)(660.12791486,240.0874908)(660.1279161,240.15749178)
\lineto(660.1279161,240.35249178)
\curveto(660.12791486,240.41249047)(660.13791485,240.46749042)(660.1579161,240.51749178)
\curveto(660.19791479,240.62749026)(660.26791472,240.69749019)(660.3679161,240.72749178)
\curveto(660.39791459,240.72749016)(660.45291453,240.73749015)(660.5329161,240.75749178)
}
}
{
\newrgbcolor{curcolor}{0 0 0}
\pscustom[linestyle=none,fillstyle=solid,fillcolor=curcolor]
{
\newpath
\moveto(670.75307235,242.19749178)
\curveto(670.76306463,242.15748873)(670.76306463,242.10748878)(670.75307235,242.04749178)
\curveto(670.75306464,241.9874889)(670.74806465,241.93748895)(670.73807235,241.89749178)
\curveto(670.73806466,241.85748903)(670.73306466,241.81748907)(670.72307235,241.77749178)
\lineto(670.72307235,241.67249178)
\curveto(670.70306469,241.59248929)(670.68806471,241.51248937)(670.67807235,241.43249178)
\curveto(670.66806473,241.35248953)(670.64806475,241.27748961)(670.61807235,241.20749178)
\curveto(670.5980648,241.12748976)(670.57806482,241.05248983)(670.55807235,240.98249178)
\curveto(670.53806486,240.91248997)(670.50806489,240.83749005)(670.46807235,240.75749178)
\curveto(670.28806511,240.33749055)(670.03306536,239.99749089)(669.70307235,239.73749178)
\curveto(669.37306602,239.47749141)(668.98306641,239.27249161)(668.53307235,239.12249178)
\curveto(668.41306698,239.0824918)(668.28806711,239.05749183)(668.15807235,239.04749178)
\curveto(668.03806736,239.02749186)(667.91306748,239.00249188)(667.78307235,238.97249178)
\curveto(667.72306767,238.96249192)(667.65806774,238.95749193)(667.58807235,238.95749178)
\curveto(667.52806787,238.95749193)(667.46306793,238.95249193)(667.39307235,238.94249178)
\lineto(667.27307235,238.94249178)
\lineto(667.07807235,238.94249178)
\curveto(667.01806838,238.93249195)(666.96306843,238.93749195)(666.91307235,238.95749178)
\curveto(666.84306855,238.97749191)(666.77806862,238.9824919)(666.71807235,238.97249178)
\curveto(666.65806874,238.96249192)(666.5980688,238.96749192)(666.53807235,238.98749178)
\curveto(666.48806891,238.99749189)(666.44306895,239.00249188)(666.40307235,239.00249178)
\curveto(666.36306903,239.00249188)(666.31806908,239.01249187)(666.26807235,239.03249178)
\curveto(666.18806921,239.05249183)(666.11306928,239.07249181)(666.04307235,239.09249178)
\curveto(665.97306942,239.10249178)(665.90306949,239.11749177)(665.83307235,239.13749178)
\curveto(665.35307004,239.30749158)(664.95307044,239.51749137)(664.63307235,239.76749178)
\curveto(664.32307107,240.02749086)(664.07307132,240.3824905)(663.88307235,240.83249178)
\curveto(663.85307154,240.89248999)(663.82807157,240.95248993)(663.80807235,241.01249178)
\curveto(663.7980716,241.0824898)(663.78307161,241.15748973)(663.76307235,241.23749178)
\curveto(663.74307165,241.29748959)(663.72807167,241.36248952)(663.71807235,241.43249178)
\curveto(663.70807169,241.50248938)(663.6930717,241.57248931)(663.67307235,241.64249178)
\curveto(663.66307173,241.69248919)(663.65807174,241.73248915)(663.65807235,241.76249178)
\lineto(663.65807235,241.88249178)
\curveto(663.64807175,241.92248896)(663.63807176,241.97248891)(663.62807235,242.03249178)
\curveto(663.62807177,242.09248879)(663.63307176,242.14248874)(663.64307235,242.18249178)
\lineto(663.64307235,242.31749178)
\curveto(663.65307174,242.36748852)(663.65807174,242.41748847)(663.65807235,242.46749178)
\curveto(663.67807172,242.56748832)(663.6930717,242.66248822)(663.70307235,242.75249178)
\curveto(663.71307168,242.85248803)(663.73307166,242.94748794)(663.76307235,243.03749178)
\curveto(663.81307158,243.1874877)(663.86807153,243.32748756)(663.92807235,243.45749178)
\curveto(663.98807141,243.5874873)(664.05807134,243.70748718)(664.13807235,243.81749178)
\curveto(664.16807123,243.86748702)(664.1980712,243.90748698)(664.22807235,243.93749178)
\curveto(664.26807113,243.96748692)(664.30307109,244.00248688)(664.33307235,244.04249178)
\curveto(664.393071,244.12248676)(664.46307093,244.19248669)(664.54307235,244.25249178)
\curveto(664.60307079,244.30248658)(664.66307073,244.34748654)(664.72307235,244.38749178)
\lineto(664.93307235,244.53749178)
\curveto(664.98307041,244.57748631)(665.03307036,244.61248627)(665.08307235,244.64249178)
\curveto(665.13307026,244.6824862)(665.16807023,244.73748615)(665.18807235,244.80749178)
\curveto(665.18807021,244.83748605)(665.17807022,244.86248602)(665.15807235,244.88249178)
\curveto(665.14807025,244.91248597)(665.13807026,244.93748595)(665.12807235,244.95749178)
\curveto(665.08807031,245.00748588)(665.03807036,245.05248583)(664.97807235,245.09249178)
\curveto(664.92807047,245.14248574)(664.87807052,245.1874857)(664.82807235,245.22749178)
\curveto(664.78807061,245.25748563)(664.73807066,245.31248557)(664.67807235,245.39249178)
\curveto(664.65807074,245.42248546)(664.62807077,245.44748544)(664.58807235,245.46749178)
\curveto(664.55807084,245.49748539)(664.53307086,245.53248535)(664.51307235,245.57249178)
\curveto(664.34307105,245.7824851)(664.21307118,246.02748486)(664.12307235,246.30749178)
\curveto(664.10307129,246.3874845)(664.08807131,246.46748442)(664.07807235,246.54749178)
\curveto(664.06807133,246.62748426)(664.05307134,246.70748418)(664.03307235,246.78749178)
\curveto(664.01307138,246.83748405)(664.00307139,246.90248398)(664.00307235,246.98249178)
\curveto(664.00307139,247.07248381)(664.01307138,247.14248374)(664.03307235,247.19249178)
\curveto(664.03307136,247.29248359)(664.03807136,247.36248352)(664.04807235,247.40249178)
\curveto(664.06807133,247.4824834)(664.08307131,247.55248333)(664.09307235,247.61249178)
\curveto(664.10307129,247.6824832)(664.11807128,247.75248313)(664.13807235,247.82249178)
\curveto(664.28807111,248.25248263)(664.50307089,248.59748229)(664.78307235,248.85749178)
\curveto(665.07307032,249.11748177)(665.42306997,249.33248155)(665.83307235,249.50249178)
\curveto(665.94306945,249.55248133)(666.05806934,249.5824813)(666.17807235,249.59249178)
\curveto(666.30806909,249.61248127)(666.43806896,249.64248124)(666.56807235,249.68249178)
\curveto(666.64806875,249.6824812)(666.71806868,249.6824812)(666.77807235,249.68249178)
\curveto(666.84806855,249.69248119)(666.92306847,249.70248118)(667.00307235,249.71249178)
\curveto(667.7930676,249.73248115)(668.44806695,249.60248128)(668.96807235,249.32249178)
\curveto(669.4980659,249.04248184)(669.87806552,248.63248225)(670.10807235,248.09249178)
\curveto(670.21806518,247.86248302)(670.28806511,247.57748331)(670.31807235,247.23749178)
\curveto(670.35806504,246.90748398)(670.32806507,246.60248428)(670.22807235,246.32249178)
\curveto(670.18806521,246.19248469)(670.13806526,246.07248481)(670.07807235,245.96249178)
\curveto(670.02806537,245.85248503)(669.96806543,245.74748514)(669.89807235,245.64749178)
\curveto(669.87806552,245.60748528)(669.84806555,245.57248531)(669.80807235,245.54249178)
\lineto(669.71807235,245.45249178)
\curveto(669.66806573,245.36248552)(669.60806579,245.29748559)(669.53807235,245.25749178)
\curveto(669.48806591,245.20748568)(669.43306596,245.15748573)(669.37307235,245.10749178)
\curveto(669.32306607,245.06748582)(669.27806612,245.02248586)(669.23807235,244.97249178)
\curveto(669.21806618,244.95248593)(669.1980662,244.92748596)(669.17807235,244.89749178)
\curveto(669.16806623,244.87748601)(669.16806623,244.85248603)(669.17807235,244.82249178)
\curveto(669.18806621,244.77248611)(669.21806618,244.72248616)(669.26807235,244.67249178)
\curveto(669.31806608,244.63248625)(669.37306602,244.59248629)(669.43307235,244.55249178)
\lineto(669.61307235,244.43249178)
\curveto(669.67306572,244.40248648)(669.72306567,244.37248651)(669.76307235,244.34249178)
\curveto(670.0930653,244.10248678)(670.34306505,243.79248709)(670.51307235,243.41249178)
\curveto(670.55306484,243.33248755)(670.58306481,243.24748764)(670.60307235,243.15749178)
\curveto(670.63306476,243.06748782)(670.65806474,242.97748791)(670.67807235,242.88749178)
\curveto(670.68806471,242.83748805)(670.6980647,242.7824881)(670.70807235,242.72249178)
\lineto(670.73807235,242.57249178)
\curveto(670.74806465,242.51248837)(670.74806465,242.44748844)(670.73807235,242.37749178)
\curveto(670.72806467,242.31748857)(670.73306466,242.25748863)(670.75307235,242.19749178)
\moveto(665.36807235,247.23749178)
\curveto(665.33807006,247.12748376)(665.33307006,246.9874839)(665.35307235,246.81749178)
\curveto(665.37307002,246.65748423)(665.39807,246.53248435)(665.42807235,246.44249178)
\curveto(665.53806986,246.12248476)(665.68806971,245.87748501)(665.87807235,245.70749178)
\curveto(666.06806933,245.54748534)(666.33306906,245.41748547)(666.67307235,245.31749178)
\curveto(666.80306859,245.2874856)(666.96806843,245.26248562)(667.16807235,245.24249178)
\curveto(667.36806803,245.23248565)(667.53806786,245.24748564)(667.67807235,245.28749178)
\curveto(667.96806743,245.36748552)(668.20806719,245.47748541)(668.39807235,245.61749178)
\curveto(668.5980668,245.76748512)(668.75306664,245.96748492)(668.86307235,246.21749178)
\curveto(668.88306651,246.26748462)(668.8930665,246.31248457)(668.89307235,246.35249178)
\curveto(668.90306649,246.39248449)(668.91806648,246.43748445)(668.93807235,246.48749178)
\curveto(668.96806643,246.59748429)(668.98806641,246.73748415)(668.99807235,246.90749178)
\curveto(669.00806639,247.07748381)(668.9980664,247.22248366)(668.96807235,247.34249178)
\curveto(668.94806645,247.43248345)(668.92306647,247.51748337)(668.89307235,247.59749178)
\curveto(668.87306652,247.67748321)(668.83806656,247.75748313)(668.78807235,247.83749178)
\curveto(668.61806678,248.10748278)(668.393067,248.30248258)(668.11307235,248.42249178)
\curveto(667.84306755,248.54248234)(667.48306791,248.60248228)(667.03307235,248.60249178)
\curveto(667.01306838,248.5824823)(666.98306841,248.57748231)(666.94307235,248.58749178)
\curveto(666.90306849,248.59748229)(666.86806853,248.59748229)(666.83807235,248.58749178)
\curveto(666.78806861,248.56748232)(666.73306866,248.55248233)(666.67307235,248.54249178)
\curveto(666.62306877,248.54248234)(666.57306882,248.53248235)(666.52307235,248.51249178)
\curveto(666.28306911,248.42248246)(666.07306932,248.30748258)(665.89307235,248.16749178)
\curveto(665.71306968,248.03748285)(665.57306982,247.85748303)(665.47307235,247.62749178)
\curveto(665.45306994,247.56748332)(665.43306996,247.50248338)(665.41307235,247.43249178)
\curveto(665.40306999,247.37248351)(665.38807001,247.30748358)(665.36807235,247.23749178)
\moveto(669.38807235,241.70249178)
\curveto(669.43806596,241.89248899)(669.44306595,242.09748879)(669.40307235,242.31749178)
\curveto(669.37306602,242.53748835)(669.32806607,242.71748817)(669.26807235,242.85749178)
\curveto(669.0980663,243.22748766)(668.83806656,243.53248735)(668.48807235,243.77249178)
\curveto(668.14806725,244.01248687)(667.71306768,244.13248675)(667.18307235,244.13249178)
\curveto(667.15306824,244.11248677)(667.11306828,244.10748678)(667.06307235,244.11749178)
\curveto(667.01306838,244.13748675)(666.97306842,244.14248674)(666.94307235,244.13249178)
\lineto(666.67307235,244.07249178)
\curveto(666.5930688,244.06248682)(666.51306888,244.04748684)(666.43307235,244.02749178)
\curveto(666.13306926,243.91748697)(665.86806953,243.77248711)(665.63807235,243.59249178)
\curveto(665.41806998,243.41248747)(665.24807015,243.1824877)(665.12807235,242.90249178)
\curveto(665.0980703,242.82248806)(665.07307032,242.74248814)(665.05307235,242.66249178)
\curveto(665.03307036,242.5824883)(665.01307038,242.49748839)(664.99307235,242.40749178)
\curveto(664.96307043,242.2874886)(664.95307044,242.13748875)(664.96307235,241.95749178)
\curveto(664.98307041,241.77748911)(665.00807039,241.63748925)(665.03807235,241.53749178)
\curveto(665.05807034,241.4874894)(665.06807033,241.44248944)(665.06807235,241.40249178)
\curveto(665.07807032,241.37248951)(665.0930703,241.33248955)(665.11307235,241.28249178)
\curveto(665.21307018,241.06248982)(665.34307005,240.86249002)(665.50307235,240.68249178)
\curveto(665.67306972,240.50249038)(665.86806953,240.36749052)(666.08807235,240.27749178)
\curveto(666.15806924,240.23749065)(666.25306914,240.20249068)(666.37307235,240.17249178)
\curveto(666.5930688,240.0824908)(666.84806855,240.03749085)(667.13807235,240.03749178)
\lineto(667.42307235,240.03749178)
\curveto(667.52306787,240.05749083)(667.61806778,240.07249081)(667.70807235,240.08249178)
\curveto(667.7980676,240.09249079)(667.88806751,240.11249077)(667.97807235,240.14249178)
\curveto(668.23806716,240.22249066)(668.47806692,240.35249053)(668.69807235,240.53249178)
\curveto(668.92806647,240.72249016)(669.0980663,240.93748995)(669.20807235,241.17749178)
\curveto(669.24806615,241.25748963)(669.27806612,241.33748955)(669.29807235,241.41749178)
\curveto(669.32806607,241.50748938)(669.35806604,241.60248928)(669.38807235,241.70249178)
}
}
{
\newrgbcolor{curcolor}{0 0 0}
\pscustom[linestyle=none,fillstyle=solid,fillcolor=curcolor]
{
\newpath
\moveto(681.89268173,247.64249178)
\curveto(681.69267143,247.35248353)(681.48267164,247.06748382)(681.26268173,246.78749178)
\curveto(681.05267207,246.50748438)(680.84767227,246.22248466)(680.64768173,245.93249178)
\curveto(680.04767307,245.0824858)(679.44267368,244.24248664)(678.83268173,243.41249178)
\curveto(678.2226749,242.59248829)(677.6176755,241.75748913)(677.01768173,240.90749178)
\lineto(676.50768173,240.18749178)
\lineto(675.99768173,239.49749178)
\curveto(675.9176772,239.3874915)(675.83767728,239.27249161)(675.75768173,239.15249178)
\curveto(675.67767744,239.03249185)(675.58267754,238.93749195)(675.47268173,238.86749178)
\curveto(675.43267769,238.84749204)(675.36767775,238.83249205)(675.27768173,238.82249178)
\curveto(675.19767792,238.80249208)(675.10767801,238.79249209)(675.00768173,238.79249178)
\curveto(674.90767821,238.79249209)(674.81267831,238.79749209)(674.72268173,238.80749178)
\curveto(674.64267848,238.81749207)(674.58267854,238.83749205)(674.54268173,238.86749178)
\curveto(674.51267861,238.887492)(674.48767863,238.92249196)(674.46768173,238.97249178)
\curveto(674.45767866,239.01249187)(674.46267866,239.05749183)(674.48268173,239.10749178)
\curveto(674.5226786,239.1874917)(674.56767855,239.26249162)(674.61768173,239.33249178)
\curveto(674.67767844,239.41249147)(674.73267839,239.49249139)(674.78268173,239.57249178)
\curveto(675.0226781,239.91249097)(675.26767785,240.24749064)(675.51768173,240.57749178)
\curveto(675.76767735,240.90748998)(676.00767711,241.24248964)(676.23768173,241.58249178)
\curveto(676.39767672,241.80248908)(676.55767656,242.01748887)(676.71768173,242.22749178)
\curveto(676.87767624,242.43748845)(677.03767608,242.65248823)(677.19768173,242.87249178)
\curveto(677.55767556,243.39248749)(677.9226752,243.90248698)(678.29268173,244.40249178)
\curveto(678.66267446,244.90248598)(679.03267409,245.41248547)(679.40268173,245.93249178)
\curveto(679.54267358,246.13248475)(679.68267344,246.32748456)(679.82268173,246.51749178)
\curveto(679.97267315,246.70748418)(680.117673,246.90248398)(680.25768173,247.10249178)
\curveto(680.46767265,247.40248348)(680.68267244,247.70248318)(680.90268173,248.00249178)
\lineto(681.56268173,248.90249178)
\lineto(681.74268173,249.17249178)
\lineto(681.95268173,249.44249178)
\lineto(682.07268173,249.62249178)
\curveto(682.122671,249.6824812)(682.17267095,249.73748115)(682.22268173,249.78749178)
\curveto(682.29267083,249.83748105)(682.36767075,249.87248101)(682.44768173,249.89249178)
\curveto(682.46767065,249.90248098)(682.49267063,249.90248098)(682.52268173,249.89249178)
\curveto(682.56267056,249.89248099)(682.59267053,249.90248098)(682.61268173,249.92249178)
\curveto(682.73267039,249.92248096)(682.86767025,249.91748097)(683.01768173,249.90749178)
\curveto(683.16766995,249.90748098)(683.25766986,249.86248102)(683.28768173,249.77249178)
\curveto(683.30766981,249.74248114)(683.31266981,249.70748118)(683.30268173,249.66749178)
\curveto(683.29266983,249.62748126)(683.27766984,249.59748129)(683.25768173,249.57749178)
\curveto(683.2176699,249.49748139)(683.17766994,249.42748146)(683.13768173,249.36749178)
\curveto(683.09767002,249.30748158)(683.05267007,249.24748164)(683.00268173,249.18749178)
\lineto(682.43268173,248.40749178)
\curveto(682.25267087,248.15748273)(682.07267105,247.90248298)(681.89268173,247.64249178)
\moveto(675.03768173,243.74249178)
\curveto(674.98767813,243.76248712)(674.93767818,243.76748712)(674.88768173,243.75749178)
\curveto(674.83767828,243.74748714)(674.78767833,243.75248713)(674.73768173,243.77249178)
\curveto(674.62767849,243.79248709)(674.5226786,243.81248707)(674.42268173,243.83249178)
\curveto(674.33267879,243.86248702)(674.23767888,243.90248698)(674.13768173,243.95249178)
\curveto(673.80767931,244.09248679)(673.55267957,244.2874866)(673.37268173,244.53749178)
\curveto(673.19267993,244.79748609)(673.04768007,245.10748578)(672.93768173,245.46749178)
\curveto(672.90768021,245.54748534)(672.88768023,245.62748526)(672.87768173,245.70749178)
\curveto(672.86768025,245.79748509)(672.85268027,245.882485)(672.83268173,245.96249178)
\curveto(672.8226803,246.01248487)(672.8176803,246.07748481)(672.81768173,246.15749178)
\curveto(672.80768031,246.1874847)(672.80268032,246.21748467)(672.80268173,246.24749178)
\curveto(672.80268032,246.2874846)(672.79768032,246.32248456)(672.78768173,246.35249178)
\lineto(672.78768173,246.50249178)
\curveto(672.77768034,246.55248433)(672.77268035,246.61248427)(672.77268173,246.68249178)
\curveto(672.77268035,246.76248412)(672.77768034,246.82748406)(672.78768173,246.87749178)
\lineto(672.78768173,247.04249178)
\curveto(672.80768031,247.09248379)(672.81268031,247.13748375)(672.80268173,247.17749178)
\curveto(672.80268032,247.22748366)(672.80768031,247.27248361)(672.81768173,247.31249178)
\curveto(672.82768029,247.35248353)(672.83268029,247.3874835)(672.83268173,247.41749178)
\curveto(672.83268029,247.45748343)(672.83768028,247.49748339)(672.84768173,247.53749178)
\curveto(672.87768024,247.64748324)(672.89768022,247.75748313)(672.90768173,247.86749178)
\curveto(672.92768019,247.9874829)(672.96268016,248.10248278)(673.01268173,248.21249178)
\curveto(673.15267997,248.55248233)(673.31267981,248.82748206)(673.49268173,249.03749178)
\curveto(673.68267944,249.25748163)(673.95267917,249.43748145)(674.30268173,249.57749178)
\curveto(674.38267874,249.60748128)(674.46767865,249.62748126)(674.55768173,249.63749178)
\curveto(674.64767847,249.65748123)(674.74267838,249.67748121)(674.84268173,249.69749178)
\curveto(674.87267825,249.70748118)(674.92767819,249.70748118)(675.00768173,249.69749178)
\curveto(675.08767803,249.69748119)(675.13767798,249.70748118)(675.15768173,249.72749178)
\curveto(675.7176774,249.73748115)(676.16767695,249.62748126)(676.50768173,249.39749178)
\curveto(676.85767626,249.16748172)(677.117676,248.86248202)(677.28768173,248.48249178)
\curveto(677.32767579,248.39248249)(677.36267576,248.29748259)(677.39268173,248.19749178)
\curveto(677.4226757,248.09748279)(677.44767567,247.99748289)(677.46768173,247.89749178)
\curveto(677.48767563,247.86748302)(677.49267563,247.83748305)(677.48268173,247.80749178)
\curveto(677.48267564,247.77748311)(677.48767563,247.74748314)(677.49768173,247.71749178)
\curveto(677.52767559,247.60748328)(677.54767557,247.4824834)(677.55768173,247.34249178)
\curveto(677.56767555,247.21248367)(677.57767554,247.07748381)(677.58768173,246.93749178)
\lineto(677.58768173,246.77249178)
\curveto(677.59767552,246.71248417)(677.59767552,246.65748423)(677.58768173,246.60749178)
\curveto(677.57767554,246.55748433)(677.57267555,246.50748438)(677.57268173,246.45749178)
\lineto(677.57268173,246.32249178)
\curveto(677.56267556,246.2824846)(677.55767556,246.24248464)(677.55768173,246.20249178)
\curveto(677.56767555,246.16248472)(677.56267556,246.11748477)(677.54268173,246.06749178)
\curveto(677.5226756,245.95748493)(677.50267562,245.85248503)(677.48268173,245.75249178)
\curveto(677.47267565,245.65248523)(677.45267567,245.55248533)(677.42268173,245.45249178)
\curveto(677.29267583,245.09248579)(677.12767599,244.77748611)(676.92768173,244.50749178)
\curveto(676.72767639,244.23748665)(676.45267667,244.03248685)(676.10268173,243.89249178)
\curveto(676.0226771,243.86248702)(675.93767718,243.83748705)(675.84768173,243.81749178)
\lineto(675.57768173,243.75749178)
\curveto(675.52767759,243.74748714)(675.48267764,243.74248714)(675.44268173,243.74249178)
\curveto(675.40267772,243.75248713)(675.36267776,243.75248713)(675.32268173,243.74249178)
\curveto(675.2226779,243.72248716)(675.12767799,243.72248716)(675.03768173,243.74249178)
\moveto(674.19768173,245.13749178)
\curveto(674.23767888,245.06748582)(674.27767884,245.00248588)(674.31768173,244.94249178)
\curveto(674.35767876,244.89248599)(674.40767871,244.84248604)(674.46768173,244.79249178)
\lineto(674.61768173,244.67249178)
\curveto(674.67767844,244.64248624)(674.74267838,244.61748627)(674.81268173,244.59749178)
\curveto(674.85267827,244.57748631)(674.88767823,244.56748632)(674.91768173,244.56749178)
\curveto(674.95767816,244.57748631)(674.99767812,244.57248631)(675.03768173,244.55249178)
\curveto(675.06767805,244.55248633)(675.10767801,244.54748634)(675.15768173,244.53749178)
\curveto(675.20767791,244.53748635)(675.24767787,244.54248634)(675.27768173,244.55249178)
\lineto(675.50268173,244.59749178)
\curveto(675.75267737,244.67748621)(675.93767718,244.80248608)(676.05768173,244.97249178)
\curveto(676.13767698,245.07248581)(676.20767691,245.20248568)(676.26768173,245.36249178)
\curveto(676.34767677,245.54248534)(676.40767671,245.76748512)(676.44768173,246.03749178)
\curveto(676.48767663,246.31748457)(676.50267662,246.59748429)(676.49268173,246.87749178)
\curveto(676.48267664,247.16748372)(676.45267667,247.44248344)(676.40268173,247.70249178)
\curveto(676.35267677,247.96248292)(676.27767684,248.17248271)(676.17768173,248.33249178)
\curveto(676.05767706,248.53248235)(675.90767721,248.6824822)(675.72768173,248.78249178)
\curveto(675.64767747,248.83248205)(675.55767756,248.86248202)(675.45768173,248.87249178)
\curveto(675.35767776,248.89248199)(675.25267787,248.90248198)(675.14268173,248.90249178)
\curveto(675.122678,248.89248199)(675.09767802,248.887482)(675.06768173,248.88749178)
\curveto(675.04767807,248.89748199)(675.02767809,248.89748199)(675.00768173,248.88749178)
\curveto(674.95767816,248.87748201)(674.91267821,248.86748202)(674.87268173,248.85749178)
\curveto(674.83267829,248.85748203)(674.79267833,248.84748204)(674.75268173,248.82749178)
\curveto(674.57267855,248.74748214)(674.4226787,248.62748226)(674.30268173,248.46749178)
\curveto(674.19267893,248.30748258)(674.10267902,248.12748276)(674.03268173,247.92749178)
\curveto(673.97267915,247.73748315)(673.92767919,247.51248337)(673.89768173,247.25249178)
\curveto(673.87767924,246.99248389)(673.87267925,246.72748416)(673.88268173,246.45749178)
\curveto(673.89267923,246.19748469)(673.9226792,245.94748494)(673.97268173,245.70749178)
\curveto(674.03267909,245.47748541)(674.10767901,245.2874856)(674.19768173,245.13749178)
\moveto(684.99768173,242.15249178)
\curveto(685.00766811,242.10248878)(685.01266811,242.01248887)(685.01268173,241.88249178)
\curveto(685.01266811,241.75248913)(685.00266812,241.66248922)(684.98268173,241.61249178)
\curveto(684.96266816,241.56248932)(684.95766816,241.50748938)(684.96768173,241.44749178)
\curveto(684.97766814,241.39748949)(684.97766814,241.34748954)(684.96768173,241.29749178)
\curveto(684.92766819,241.15748973)(684.89766822,241.02248986)(684.87768173,240.89249178)
\curveto(684.86766825,240.76249012)(684.83766828,240.64249024)(684.78768173,240.53249178)
\curveto(684.64766847,240.1824907)(684.48266864,239.887491)(684.29268173,239.64749178)
\curveto(684.10266902,239.41749147)(683.83266929,239.23249165)(683.48268173,239.09249178)
\curveto(683.40266972,239.06249182)(683.3176698,239.04249184)(683.22768173,239.03249178)
\curveto(683.13766998,239.01249187)(683.05267007,238.99249189)(682.97268173,238.97249178)
\curveto(682.9226702,238.96249192)(682.87267025,238.95749193)(682.82268173,238.95749178)
\curveto(682.77267035,238.95749193)(682.7226704,238.95249193)(682.67268173,238.94249178)
\curveto(682.64267048,238.93249195)(682.59267053,238.93249195)(682.52268173,238.94249178)
\curveto(682.45267067,238.94249194)(682.40267072,238.94749194)(682.37268173,238.95749178)
\curveto(682.31267081,238.97749191)(682.25267087,238.9874919)(682.19268173,238.98749178)
\curveto(682.14267098,238.97749191)(682.09267103,238.9824919)(682.04268173,239.00249178)
\curveto(681.95267117,239.02249186)(681.86267126,239.04749184)(681.77268173,239.07749178)
\curveto(681.69267143,239.09749179)(681.61267151,239.12749176)(681.53268173,239.16749178)
\curveto(681.21267191,239.30749158)(680.96267216,239.50249138)(680.78268173,239.75249178)
\curveto(680.60267252,240.01249087)(680.45267267,240.31749057)(680.33268173,240.66749178)
\curveto(680.31267281,240.74749014)(680.29767282,240.83249005)(680.28768173,240.92249178)
\curveto(680.27767284,241.01248987)(680.26267286,241.09748979)(680.24268173,241.17749178)
\curveto(680.23267289,241.20748968)(680.22767289,241.23748965)(680.22768173,241.26749178)
\lineto(680.22768173,241.37249178)
\curveto(680.20767291,241.45248943)(680.19767292,241.53248935)(680.19768173,241.61249178)
\lineto(680.19768173,241.74749178)
\curveto(680.17767294,241.84748904)(680.17767294,241.94748894)(680.19768173,242.04749178)
\lineto(680.19768173,242.22749178)
\curveto(680.20767291,242.27748861)(680.21267291,242.32248856)(680.21268173,242.36249178)
\curveto(680.21267291,242.41248847)(680.2176729,242.45748843)(680.22768173,242.49749178)
\curveto(680.23767288,242.53748835)(680.24267288,242.57248831)(680.24268173,242.60249178)
\curveto(680.24267288,242.64248824)(680.24767287,242.6824882)(680.25768173,242.72249178)
\lineto(680.31768173,243.05249178)
\curveto(680.33767278,243.17248771)(680.36767275,243.2824876)(680.40768173,243.38249178)
\curveto(680.54767257,243.71248717)(680.70767241,243.9874869)(680.88768173,244.20749178)
\curveto(681.07767204,244.43748645)(681.33767178,244.62248626)(681.66768173,244.76249178)
\curveto(681.74767137,244.80248608)(681.83267129,244.82748606)(681.92268173,244.83749178)
\lineto(682.22268173,244.89749178)
\lineto(682.35768173,244.89749178)
\curveto(682.40767071,244.90748598)(682.45767066,244.91248597)(682.50768173,244.91249178)
\curveto(683.07767004,244.93248595)(683.53766958,244.82748606)(683.88768173,244.59749178)
\curveto(684.24766887,244.37748651)(684.51266861,244.07748681)(684.68268173,243.69749178)
\curveto(684.73266839,243.59748729)(684.77266835,243.49748739)(684.80268173,243.39749178)
\curveto(684.83266829,243.29748759)(684.86266826,243.19248769)(684.89268173,243.08249178)
\curveto(684.90266822,243.04248784)(684.90766821,243.00748788)(684.90768173,242.97749178)
\curveto(684.90766821,242.95748793)(684.91266821,242.92748796)(684.92268173,242.88749178)
\curveto(684.94266818,242.81748807)(684.95266817,242.74248814)(684.95268173,242.66249178)
\curveto(684.95266817,242.5824883)(684.96266816,242.50248838)(684.98268173,242.42249178)
\curveto(684.98266814,242.37248851)(684.98266814,242.32748856)(684.98268173,242.28749178)
\curveto(684.98266814,242.24748864)(684.98766813,242.20248868)(684.99768173,242.15249178)
\moveto(683.88768173,241.71749178)
\curveto(683.89766922,241.76748912)(683.90266922,241.84248904)(683.90268173,241.94249178)
\curveto(683.91266921,242.04248884)(683.90766921,242.11748877)(683.88768173,242.16749178)
\curveto(683.86766925,242.22748866)(683.86266926,242.2824886)(683.87268173,242.33249178)
\curveto(683.89266923,242.39248849)(683.89266923,242.45248843)(683.87268173,242.51249178)
\curveto(683.86266926,242.54248834)(683.85766926,242.57748831)(683.85768173,242.61749178)
\curveto(683.85766926,242.65748823)(683.85266927,242.69748819)(683.84268173,242.73749178)
\curveto(683.8226693,242.81748807)(683.80266932,242.89248799)(683.78268173,242.96249178)
\curveto(683.77266935,243.04248784)(683.75766936,243.12248776)(683.73768173,243.20249178)
\curveto(683.70766941,243.26248762)(683.68266944,243.32248756)(683.66268173,243.38249178)
\curveto(683.64266948,243.44248744)(683.61266951,243.50248738)(683.57268173,243.56249178)
\curveto(683.47266965,243.73248715)(683.34266978,243.86748702)(683.18268173,243.96749178)
\curveto(683.10267002,244.01748687)(683.00767011,244.05248683)(682.89768173,244.07249178)
\curveto(682.78767033,244.09248679)(682.66267046,244.10248678)(682.52268173,244.10249178)
\curveto(682.50267062,244.09248679)(682.47767064,244.0874868)(682.44768173,244.08749178)
\curveto(682.4176707,244.09748679)(682.38767073,244.09748679)(682.35768173,244.08749178)
\lineto(682.20768173,244.02749178)
\curveto(682.15767096,244.01748687)(682.11267101,244.00248688)(682.07268173,243.98249178)
\curveto(681.88267124,243.87248701)(681.73767138,243.72748716)(681.63768173,243.54749178)
\curveto(681.54767157,243.36748752)(681.46767165,243.16248772)(681.39768173,242.93249178)
\curveto(681.35767176,242.80248808)(681.33767178,242.66748822)(681.33768173,242.52749178)
\curveto(681.33767178,242.39748849)(681.32767179,242.25248863)(681.30768173,242.09249178)
\curveto(681.29767182,242.04248884)(681.28767183,241.9824889)(681.27768173,241.91249178)
\curveto(681.27767184,241.84248904)(681.28767183,241.7824891)(681.30768173,241.73249178)
\lineto(681.30768173,241.56749178)
\lineto(681.30768173,241.38749178)
\curveto(681.3176718,241.33748955)(681.32767179,241.2824896)(681.33768173,241.22249178)
\curveto(681.34767177,241.17248971)(681.35267177,241.11748977)(681.35268173,241.05749178)
\curveto(681.36267176,240.99748989)(681.37767174,240.94248994)(681.39768173,240.89249178)
\curveto(681.44767167,240.70249018)(681.50767161,240.52749036)(681.57768173,240.36749178)
\curveto(681.64767147,240.20749068)(681.75267137,240.07749081)(681.89268173,239.97749178)
\curveto(682.0226711,239.87749101)(682.16267096,239.80749108)(682.31268173,239.76749178)
\curveto(682.34267078,239.75749113)(682.36767075,239.75249113)(682.38768173,239.75249178)
\curveto(682.4176707,239.76249112)(682.44767067,239.76249112)(682.47768173,239.75249178)
\curveto(682.49767062,239.75249113)(682.52767059,239.74749114)(682.56768173,239.73749178)
\curveto(682.60767051,239.73749115)(682.64267048,239.74249114)(682.67268173,239.75249178)
\curveto(682.71267041,239.76249112)(682.75267037,239.76749112)(682.79268173,239.76749178)
\curveto(682.83267029,239.76749112)(682.87267025,239.77749111)(682.91268173,239.79749178)
\curveto(683.15266997,239.87749101)(683.34766977,240.01249087)(683.49768173,240.20249178)
\curveto(683.6176695,240.3824905)(683.70766941,240.5874903)(683.76768173,240.81749178)
\curveto(683.78766933,240.88749)(683.80266932,240.95748993)(683.81268173,241.02749178)
\curveto(683.8226693,241.10748978)(683.83766928,241.1874897)(683.85768173,241.26749178)
\curveto(683.85766926,241.32748956)(683.86266926,241.37248951)(683.87268173,241.40249178)
\curveto(683.87266925,241.42248946)(683.87266925,241.44748944)(683.87268173,241.47749178)
\curveto(683.87266925,241.51748937)(683.87766924,241.54748934)(683.88768173,241.56749178)
\lineto(683.88768173,241.71749178)
}
}
{
\newrgbcolor{curcolor}{0 0 0}
\pscustom[linestyle=none,fillstyle=solid,fillcolor=curcolor]
{
\newpath
\moveto(631.78641464,76.48634187)
\curveto(631.88640978,76.48633125)(631.98140969,76.47633126)(632.07141464,76.45634188)
\curveto(632.16140951,76.44633129)(632.22640944,76.41633132)(632.26641464,76.36634187)
\curveto(632.32640934,76.28633145)(632.35640931,76.18133156)(632.35641464,76.05134187)
\lineto(632.35641464,75.66134187)
\lineto(632.35641464,74.16134187)
\lineto(632.35641464,67.77134187)
\lineto(632.35641464,66.60134188)
\lineto(632.35641464,66.28634187)
\curveto(632.3664093,66.18634156)(632.35140932,66.10634164)(632.31141464,66.04634187)
\curveto(632.26140941,65.96634177)(632.18640948,65.91634183)(632.08641464,65.89634187)
\curveto(631.99640967,65.88634185)(631.88640978,65.88134186)(631.75641464,65.88134187)
\lineto(631.53141464,65.88134187)
\curveto(631.45141022,65.90134184)(631.38141029,65.91634183)(631.32141464,65.92634187)
\curveto(631.26141041,65.94634179)(631.21141046,65.98634176)(631.17141464,66.04634187)
\curveto(631.13141054,66.10634164)(631.11141056,66.18134156)(631.11141464,66.27134187)
\lineto(631.11141464,66.57134188)
\lineto(631.11141464,67.66634187)
\lineto(631.11141464,73.00634187)
\curveto(631.09141058,73.09633464)(631.07641059,73.17133457)(631.06641464,73.23134187)
\curveto(631.0664106,73.30133444)(631.03641063,73.36133438)(630.97641464,73.41134187)
\curveto(630.90641076,73.46133428)(630.81641085,73.48633425)(630.70641464,73.48634187)
\curveto(630.60641106,73.49633424)(630.49641117,73.50133424)(630.37641464,73.50134187)
\lineto(629.23641464,73.50134187)
\lineto(628.74141464,73.50134187)
\curveto(628.58141309,73.51133423)(628.4714132,73.57133417)(628.41141464,73.68134188)
\curveto(628.39141328,73.71133403)(628.38141329,73.741334)(628.38141464,73.77134187)
\curveto(628.38141329,73.81133393)(628.37641329,73.85633389)(628.36641464,73.90634187)
\curveto(628.34641332,74.02633371)(628.35141332,74.13633361)(628.38141464,74.23634187)
\curveto(628.42141325,74.33633341)(628.47641319,74.40633333)(628.54641464,74.44634188)
\curveto(628.62641304,74.49633324)(628.74641292,74.52133322)(628.90641464,74.52134187)
\curveto(629.0664126,74.52133322)(629.20141247,74.53633321)(629.31141464,74.56634188)
\curveto(629.36141231,74.57633316)(629.41641225,74.58133316)(629.47641464,74.58134188)
\curveto(629.53641213,74.59133315)(629.59641207,74.60633313)(629.65641464,74.62634187)
\curveto(629.80641186,74.67633306)(629.95141172,74.72633302)(630.09141464,74.77634187)
\curveto(630.23141144,74.8363329)(630.3664113,74.90633283)(630.49641464,74.98634187)
\curveto(630.63641103,75.07633266)(630.75641091,75.18133256)(630.85641464,75.30134187)
\curveto(630.95641071,75.42133232)(631.05141062,75.55133219)(631.14141464,75.69134188)
\curveto(631.20141047,75.79133195)(631.24641042,75.90133184)(631.27641464,76.02134187)
\curveto(631.31641035,76.1413316)(631.3664103,76.24633149)(631.42641464,76.33634188)
\curveto(631.47641019,76.39633135)(631.54641012,76.4363313)(631.63641464,76.45634188)
\curveto(631.65641001,76.46633128)(631.68140999,76.47133127)(631.71141464,76.47134188)
\curveto(631.74140993,76.47133127)(631.7664099,76.47633126)(631.78641464,76.48634187)
}
}
{
\newrgbcolor{curcolor}{0 0 0}
\pscustom[linestyle=none,fillstyle=solid,fillcolor=curcolor]
{
\newpath
\moveto(643.01602401,71.47634188)
\curveto(643.01601638,71.39633635)(643.02101637,71.31633643)(643.03102401,71.23634187)
\curveto(643.04101635,71.15633658)(643.03601636,71.08133666)(643.01602401,71.01134187)
\curveto(642.9960164,70.97133677)(642.9910164,70.92633682)(643.00102401,70.87634187)
\curveto(643.01101638,70.8363369)(643.01101638,70.79633695)(643.00102401,70.75634187)
\lineto(643.00102401,70.60634187)
\curveto(642.9910164,70.51633723)(642.98601641,70.42633731)(642.98602401,70.33634188)
\curveto(642.98601641,70.25633749)(642.98101641,70.17633757)(642.97102401,70.09634188)
\lineto(642.94102401,69.85634187)
\curveto(642.93101646,69.78633796)(642.92101647,69.71133803)(642.91102401,69.63134187)
\curveto(642.90101649,69.59133815)(642.8960165,69.55133819)(642.89602401,69.51134187)
\curveto(642.8960165,69.47133827)(642.8910165,69.42633831)(642.88102401,69.37634187)
\curveto(642.84101655,69.2363385)(642.81101658,69.09633864)(642.79102401,68.95634188)
\curveto(642.78101661,68.81633892)(642.75101664,68.68133906)(642.70102401,68.55134187)
\curveto(642.65101674,68.38133936)(642.5960168,68.21633952)(642.53602401,68.05634188)
\curveto(642.48601691,67.89633984)(642.42601697,67.74134)(642.35602401,67.59134188)
\curveto(642.33601706,67.53134021)(642.30601709,67.47134027)(642.26602401,67.41134187)
\lineto(642.17602401,67.26134187)
\curveto(641.97601742,66.9413408)(641.76101763,66.67634106)(641.53102401,66.46634188)
\curveto(641.30101809,66.25634149)(641.00601839,66.07634166)(640.64602401,65.92634187)
\curveto(640.52601887,65.87634186)(640.396019,65.8413419)(640.25602401,65.82134188)
\curveto(640.12601927,65.80134194)(639.9910194,65.77634197)(639.85102401,65.74634187)
\curveto(639.7910196,65.736342)(639.73101966,65.73134201)(639.67102401,65.73134187)
\curveto(639.61101978,65.73134201)(639.54601985,65.72634202)(639.47602401,65.71634188)
\curveto(639.44601995,65.70634204)(639.39602,65.70634204)(639.32602401,65.71634188)
\lineto(639.17602401,65.71634188)
\lineto(639.02602401,65.71634188)
\curveto(638.94602045,65.736342)(638.86102053,65.75134199)(638.77102401,65.76134187)
\curveto(638.6910207,65.76134198)(638.61602078,65.77134197)(638.54602401,65.79134187)
\curveto(638.50602089,65.80134194)(638.47102092,65.80634193)(638.44102401,65.80634188)
\curveto(638.42102097,65.79634194)(638.396021,65.80134194)(638.36602401,65.82134188)
\lineto(638.09602401,65.88134187)
\curveto(638.00602139,65.91134183)(637.92102147,65.9413418)(637.84102401,65.97134188)
\curveto(637.26102213,66.21134153)(636.82602257,66.58134116)(636.53602401,67.08134188)
\curveto(636.45602294,67.21134053)(636.391023,67.34634039)(636.34102401,67.48634187)
\curveto(636.30102309,67.62634011)(636.25602314,67.77633997)(636.20602401,67.93634188)
\curveto(636.18602321,68.01633972)(636.18102321,68.09633964)(636.19102401,68.17634187)
\curveto(636.21102318,68.25633949)(636.24602315,68.31133943)(636.29602401,68.34134188)
\curveto(636.32602307,68.36133938)(636.38102301,68.37633937)(636.46102401,68.38634187)
\curveto(636.54102285,68.40633933)(636.62602277,68.41633932)(636.71602401,68.41634187)
\curveto(636.80602259,68.42633931)(636.8910225,68.42633931)(636.97102401,68.41634187)
\curveto(637.06102233,68.40633933)(637.13102226,68.39633935)(637.18102401,68.38634187)
\curveto(637.20102219,68.37633937)(637.22602217,68.36133938)(637.25602401,68.34134188)
\curveto(637.2960221,68.32133942)(637.32602207,68.30133944)(637.34602401,68.28134187)
\curveto(637.40602199,68.20133954)(637.45102194,68.10633964)(637.48102401,67.99634187)
\curveto(637.52102187,67.88633985)(637.56602183,67.78633996)(637.61602401,67.69634188)
\curveto(637.86602153,67.30634044)(638.23602116,67.03634071)(638.72602401,66.88634187)
\curveto(638.7960206,66.86634087)(638.86602053,66.85134089)(638.93602401,66.84134188)
\curveto(639.01602038,66.8413409)(639.0960203,66.83134091)(639.17602401,66.81134188)
\curveto(639.21602018,66.80134094)(639.27102012,66.79634094)(639.34102401,66.79634187)
\curveto(639.42101997,66.79634094)(639.47601992,66.80134094)(639.50602401,66.81134188)
\curveto(639.53601986,66.82134092)(639.56601983,66.82634091)(639.59602401,66.82634188)
\lineto(639.70102401,66.82634188)
\curveto(639.78101961,66.8463409)(639.85601954,66.86634087)(639.92602401,66.88634187)
\curveto(640.00601939,66.90634084)(640.08101931,66.93134081)(640.15102401,66.96134188)
\curveto(640.50101889,67.11134063)(640.77101862,67.32634042)(640.96102401,67.60634187)
\curveto(641.15101824,67.88633985)(641.30601809,68.21133953)(641.42602401,68.58134188)
\curveto(641.45601794,68.66133908)(641.47601792,68.736339)(641.48602401,68.80634188)
\curveto(641.50601789,68.87633886)(641.52601787,68.95133879)(641.54602401,69.03134187)
\curveto(641.56601783,69.12133862)(641.58101781,69.21633852)(641.59102401,69.31634188)
\curveto(641.61101778,69.42633831)(641.63101776,69.53133821)(641.65102401,69.63134187)
\curveto(641.66101773,69.68133806)(641.66601773,69.73133801)(641.66602401,69.78134187)
\curveto(641.67601772,69.8413379)(641.68101771,69.89633784)(641.68102401,69.94634188)
\curveto(641.70101769,70.00633773)(641.71101768,70.08133766)(641.71102401,70.17134187)
\curveto(641.71101768,70.27133747)(641.70101769,70.35133739)(641.68102401,70.41134187)
\curveto(641.65101774,70.50133724)(641.60101779,70.5413372)(641.53102401,70.53134187)
\curveto(641.47101792,70.52133722)(641.41601798,70.49133725)(641.36602401,70.44134188)
\curveto(641.28601811,70.39133735)(641.21601818,70.33133741)(641.15602401,70.26134187)
\curveto(641.10601829,70.19133755)(641.04101835,70.13133761)(640.96102401,70.08134188)
\curveto(640.80101859,69.97133777)(640.63601876,69.87133787)(640.46602401,69.78134187)
\curveto(640.2960191,69.70133804)(640.10101929,69.63133811)(639.88102401,69.57134188)
\curveto(639.78101961,69.5413382)(639.68101971,69.52633822)(639.58102401,69.52634187)
\curveto(639.4910199,69.52633822)(639.39102,69.51633823)(639.28102401,69.49634187)
\lineto(639.13102401,69.49634187)
\curveto(639.08102031,69.51633823)(639.03102036,69.52133822)(638.98102401,69.51134187)
\curveto(638.94102045,69.50133824)(638.90102049,69.50133824)(638.86102401,69.51134187)
\curveto(638.83102056,69.52133822)(638.78602061,69.52633822)(638.72602401,69.52634187)
\curveto(638.66602073,69.5363382)(638.60102079,69.54633819)(638.53102401,69.55634188)
\lineto(638.35102401,69.58634188)
\curveto(637.90102149,69.70633804)(637.52102187,69.87133787)(637.21102401,70.08134188)
\curveto(636.94102245,70.27133747)(636.71102268,70.50133724)(636.52102401,70.77134187)
\curveto(636.34102305,71.05133669)(636.1960232,71.36633637)(636.08602401,71.71634188)
\lineto(636.02602401,71.92634187)
\curveto(636.01602338,72.00633573)(636.00102339,72.08633565)(635.98102401,72.16634187)
\curveto(635.97102342,72.19633555)(635.96602343,72.22633551)(635.96602401,72.25634187)
\curveto(635.96602343,72.28633545)(635.96102343,72.31633543)(635.95102401,72.34634188)
\curveto(635.94102345,72.40633533)(635.93602346,72.46633528)(635.93602401,72.52634187)
\curveto(635.93602346,72.59633515)(635.92602347,72.65633509)(635.90602401,72.70634188)
\lineto(635.90602401,72.88634187)
\curveto(635.8960235,72.93633481)(635.8910235,73.00633473)(635.89102401,73.09634188)
\curveto(635.8910235,73.18633456)(635.90102349,73.25633449)(635.92102401,73.30634188)
\lineto(635.92102401,73.47134188)
\curveto(635.94102345,73.55133419)(635.95102344,73.62633411)(635.95102401,73.69634188)
\curveto(635.96102343,73.76633397)(635.97602342,73.8363339)(635.99602401,73.90634187)
\curveto(636.05602334,74.10633363)(636.11602328,74.29633344)(636.17602401,74.47634188)
\curveto(636.24602315,74.65633309)(636.33602306,74.82633291)(636.44602401,74.98634187)
\curveto(636.48602291,75.05633269)(636.52602287,75.12133262)(636.56602401,75.18134188)
\lineto(636.71602401,75.36134187)
\curveto(636.73602266,75.37133237)(636.75602264,75.38633236)(636.77602401,75.40634187)
\curveto(636.86602253,75.53633221)(636.97602242,75.64633209)(637.10602401,75.73634187)
\curveto(637.36602203,75.93633181)(637.63102176,76.09133165)(637.90102401,76.20134188)
\curveto(637.98102141,76.2413315)(638.06102133,76.27133147)(638.14102401,76.29134187)
\curveto(638.23102116,76.32133142)(638.32102107,76.34633139)(638.41102401,76.36634187)
\curveto(638.51102088,76.39633135)(638.61102078,76.41633132)(638.71102401,76.42634187)
\curveto(638.81102058,76.4363313)(638.91602048,76.45133129)(639.02602401,76.47134188)
\curveto(639.05602034,76.48133126)(639.0960203,76.48133126)(639.14602401,76.47134188)
\curveto(639.20602019,76.46133128)(639.24602015,76.46633128)(639.26602401,76.48634187)
\curveto(639.98601941,76.50633123)(640.58601881,76.39133135)(641.06602401,76.14134187)
\curveto(641.54601785,75.89133185)(641.92101747,75.55133219)(642.19102401,75.12134187)
\curveto(642.28101711,74.98133276)(642.36101703,74.8363329)(642.43102401,74.68634188)
\curveto(642.50101689,74.53633321)(642.57101682,74.37633336)(642.64102401,74.20634188)
\curveto(642.6910167,74.06633368)(642.73101666,73.91633383)(642.76102401,73.75634187)
\curveto(642.7910166,73.59633415)(642.82601657,73.4363343)(642.86602401,73.27634187)
\curveto(642.88601651,73.22633451)(642.8960165,73.17133457)(642.89602401,73.11134187)
\curveto(642.8960165,73.06133468)(642.90101649,73.01133473)(642.91102401,72.96134188)
\curveto(642.93101646,72.90133484)(642.94101645,72.8363349)(642.94102401,72.76634187)
\curveto(642.94101645,72.70633503)(642.95101644,72.65133509)(642.97102401,72.60134188)
\lineto(642.97102401,72.43634188)
\curveto(642.9910164,72.38633536)(642.9960164,72.3363354)(642.98602401,72.28634187)
\curveto(642.97601642,72.2363355)(642.98101641,72.18633556)(643.00102401,72.13634187)
\curveto(643.00101639,72.11633563)(642.9960164,72.09133565)(642.98602401,72.06134188)
\curveto(642.98601641,72.03133571)(642.9910164,72.00633573)(643.00102401,71.98634187)
\curveto(643.01101638,71.95633578)(643.01101638,71.92133582)(643.00102401,71.88134187)
\curveto(643.00101639,71.8413359)(643.00601639,71.80133594)(643.01602401,71.76134187)
\curveto(643.02601637,71.72133602)(643.02601637,71.67633606)(643.01602401,71.62634187)
\lineto(643.01602401,71.47634188)
\moveto(641.51602401,72.78134187)
\curveto(641.52601787,72.83133491)(641.53101786,72.89133485)(641.53102401,72.96134188)
\curveto(641.53101786,73.03133471)(641.52601787,73.09133465)(641.51602401,73.14134187)
\curveto(641.50601789,73.19133455)(641.50101789,73.26633448)(641.50102401,73.36634187)
\curveto(641.48101791,73.4463343)(641.46101793,73.52133422)(641.44102401,73.59134188)
\curveto(641.43101796,73.66133408)(641.41601798,73.73133401)(641.39602401,73.80134187)
\curveto(641.25601814,74.23133351)(641.06101833,74.56633317)(640.81102401,74.80634188)
\curveto(640.57101882,75.0463327)(640.22601917,75.22633251)(639.77602401,75.34634188)
\curveto(639.68601971,75.36633237)(639.58601981,75.37633236)(639.47602401,75.37634187)
\lineto(639.14602401,75.37634187)
\curveto(639.12602027,75.35633238)(639.0910203,75.34633239)(639.04102401,75.34634188)
\curveto(638.9910204,75.35633238)(638.94602045,75.35633238)(638.90602401,75.34634188)
\curveto(638.82602057,75.32633242)(638.75102064,75.30633243)(638.68102401,75.28634187)
\lineto(638.47102401,75.22634188)
\curveto(638.18102121,75.09633264)(637.95102144,74.91633283)(637.78102401,74.68634188)
\curveto(637.61102178,74.46633328)(637.47602192,74.20633353)(637.37602401,73.90634187)
\curveto(637.34602205,73.81633392)(637.32102207,73.72133402)(637.30102401,73.62134187)
\curveto(637.2910221,73.53133421)(637.27602212,73.4363343)(637.25602401,73.33634188)
\lineto(637.25602401,73.20134188)
\curveto(637.22602217,73.09133465)(637.21602218,72.95133479)(637.22602401,72.78134187)
\curveto(637.24602215,72.62133512)(637.26602213,72.49133525)(637.28602401,72.39134187)
\curveto(637.30602209,72.33133541)(637.32102207,72.27133547)(637.33102401,72.21134188)
\curveto(637.34102205,72.16133558)(637.35602204,72.11133563)(637.37602401,72.06134188)
\curveto(637.45602194,71.86133588)(637.55102184,71.67133607)(637.66102401,71.49134187)
\curveto(637.78102161,71.31133643)(637.92102147,71.16633657)(638.08102401,71.05634188)
\curveto(638.13102126,71.00633673)(638.18602121,70.96633677)(638.24602401,70.93634188)
\curveto(638.30602109,70.90633683)(638.36602103,70.87133687)(638.42602401,70.83134188)
\curveto(638.57602082,70.75133699)(638.76102063,70.68633705)(638.98102401,70.63634187)
\curveto(639.03102036,70.61633712)(639.07102032,70.61133713)(639.10102401,70.62134187)
\curveto(639.14102025,70.63133711)(639.18602021,70.62633711)(639.23602401,70.60634187)
\curveto(639.27602012,70.59633715)(639.33102006,70.59133715)(639.40102401,70.59134188)
\curveto(639.47101992,70.59133715)(639.53101986,70.59633715)(639.58102401,70.60634187)
\curveto(639.68101971,70.62633711)(639.77601962,70.6413371)(639.86602401,70.65134187)
\curveto(639.95601944,70.67133707)(640.04601935,70.70133704)(640.13602401,70.74134187)
\curveto(640.67601872,70.96133678)(641.07101832,71.35633638)(641.32102401,71.92634187)
\curveto(641.37101802,72.02633571)(641.40601799,72.12633562)(641.42602401,72.22634188)
\curveto(641.44601795,72.3363354)(641.47101792,72.4463353)(641.50102401,72.55634188)
\curveto(641.50101789,72.65633509)(641.50601789,72.73133501)(641.51602401,72.78134187)
}
}
{
\newrgbcolor{curcolor}{0 0 0}
\pscustom[linestyle=none,fillstyle=solid,fillcolor=curcolor]
{
\newpath
\moveto(645.38063339,67.51634187)
\lineto(645.68063339,67.51634187)
\curveto(645.79063133,67.52634022)(645.89563122,67.52634022)(645.99563339,67.51634187)
\curveto(646.10563101,67.51634023)(646.20563091,67.50634024)(646.29563339,67.48634187)
\curveto(646.38563073,67.47634026)(646.45563066,67.45134029)(646.50563339,67.41134187)
\curveto(646.52563059,67.39134035)(646.54063058,67.36134038)(646.55063339,67.32134188)
\curveto(646.57063055,67.28134046)(646.59063053,67.23634051)(646.61063339,67.18634188)
\lineto(646.61063339,67.11134187)
\curveto(646.6206305,67.06134068)(646.6206305,67.00634073)(646.61063339,66.94634188)
\lineto(646.61063339,66.79634187)
\lineto(646.61063339,66.31634188)
\curveto(646.61063051,66.14634159)(646.57063055,66.02634171)(646.49063339,65.95634188)
\curveto(646.4206307,65.90634184)(646.33063079,65.88134186)(646.22063339,65.88134187)
\lineto(645.89063339,65.88134187)
\lineto(645.44063339,65.88134187)
\curveto(645.29063183,65.88134186)(645.17563194,65.91134183)(645.09563339,65.97134188)
\curveto(645.05563206,66.00134174)(645.02563209,66.05134169)(645.00563339,66.12134187)
\curveto(644.98563213,66.20134154)(644.97063215,66.28634145)(644.96063339,66.37634187)
\lineto(644.96063339,66.66134187)
\curveto(644.97063215,66.76134098)(644.97563214,66.8463409)(644.97563339,66.91634187)
\lineto(644.97563339,67.11134187)
\curveto(644.97563214,67.17134057)(644.98563213,67.22634051)(645.00563339,67.27634187)
\curveto(645.04563207,67.38634036)(645.115632,67.45634029)(645.21563339,67.48634187)
\curveto(645.24563187,67.48634025)(645.30063182,67.49634024)(645.38063339,67.51634187)
}
}
{
\newrgbcolor{curcolor}{0 0 0}
\pscustom[linestyle=none,fillstyle=solid,fillcolor=curcolor]
{
\newpath
\moveto(649.04578964,76.29134187)
\lineto(653.84578964,76.29134187)
\lineto(654.85078964,76.29134187)
\curveto(654.99078254,76.29133145)(655.11078242,76.28133146)(655.21078964,76.26134187)
\curveto(655.32078221,76.25133149)(655.40078213,76.20633154)(655.45078964,76.12634187)
\curveto(655.47078206,76.08633165)(655.48078205,76.0363317)(655.48078964,75.97634188)
\curveto(655.49078204,75.91633183)(655.49578203,75.85133189)(655.49578964,75.78134187)
\lineto(655.49578964,75.51134187)
\curveto(655.49578203,75.42133232)(655.48578204,75.3413324)(655.46578964,75.27134187)
\curveto(655.4257821,75.19133255)(655.38078215,75.12133262)(655.33078964,75.06134188)
\lineto(655.18078964,74.88134187)
\curveto(655.15078238,74.83133291)(655.11578241,74.79133295)(655.07578964,74.76134187)
\curveto(655.03578249,74.73133301)(654.99578253,74.69133305)(654.95578964,74.64134187)
\curveto(654.87578265,74.53133321)(654.79078274,74.42133332)(654.70078964,74.31134188)
\curveto(654.61078292,74.21133353)(654.525783,74.10633363)(654.44578964,73.99634187)
\curveto(654.30578322,73.79633395)(654.16578336,73.58633416)(654.02578964,73.36634187)
\curveto(653.88578364,73.15633458)(653.74578378,72.9413348)(653.60578964,72.72134188)
\curveto(653.55578397,72.63133511)(653.50578402,72.53633521)(653.45578964,72.43634188)
\curveto(653.40578412,72.3363354)(653.35078418,72.2413355)(653.29078964,72.15134187)
\curveto(653.27078426,72.13133561)(653.26078427,72.10633563)(653.26078964,72.07634188)
\curveto(653.26078427,72.0463357)(653.25078428,72.02133572)(653.23078964,72.00134187)
\curveto(653.16078437,71.90133584)(653.09578443,71.78633596)(653.03578964,71.65634187)
\curveto(652.97578455,71.5363362)(652.92078461,71.42133632)(652.87078964,71.31134188)
\curveto(652.77078476,71.08133666)(652.67578485,70.8463369)(652.58578964,70.60634187)
\curveto(652.49578503,70.36633737)(652.39578513,70.12633762)(652.28578964,69.88634187)
\curveto(652.26578526,69.8363379)(652.25078528,69.79133795)(652.24078964,69.75134187)
\curveto(652.24078529,69.71133803)(652.2307853,69.66633807)(652.21078964,69.61634187)
\curveto(652.16078537,69.49633824)(652.11578541,69.37133837)(652.07578964,69.24134187)
\curveto(652.04578548,69.12133862)(652.01078552,69.00133874)(651.97078964,68.88134187)
\curveto(651.89078564,68.65133909)(651.8257857,68.41133933)(651.77578964,68.16134187)
\curveto(651.73578579,67.92133982)(651.68578584,67.68134006)(651.62578964,67.44134188)
\curveto(651.58578594,67.29134045)(651.56078597,67.1413406)(651.55078964,66.99134187)
\curveto(651.54078599,66.8413409)(651.52078601,66.69134105)(651.49078964,66.54134187)
\curveto(651.48078605,66.50134124)(651.47578605,66.4413413)(651.47578964,66.36134187)
\curveto(651.44578608,66.2413415)(651.41578611,66.1413416)(651.38578964,66.06134188)
\curveto(651.35578617,65.98134176)(651.28578624,65.92634182)(651.17578964,65.89634187)
\curveto(651.1257864,65.87634186)(651.07078646,65.86634187)(651.01078964,65.86634187)
\lineto(650.81578964,65.86634187)
\curveto(650.67578685,65.86634187)(650.53578699,65.87134187)(650.39578964,65.88134187)
\curveto(650.26578726,65.89134185)(650.17078736,65.9363418)(650.11078964,66.01634187)
\curveto(650.07078746,66.07634166)(650.05078748,66.16134158)(650.05078964,66.27134187)
\curveto(650.06078747,66.38134136)(650.07578745,66.47634126)(650.09578964,66.55634188)
\lineto(650.09578964,66.63134187)
\curveto(650.10578742,66.66134108)(650.11078742,66.69134105)(650.11078964,66.72134188)
\curveto(650.1307874,66.80134094)(650.14078739,66.87634086)(650.14078964,66.94634188)
\curveto(650.14078739,67.01634072)(650.15078738,67.08634065)(650.17078964,67.15634187)
\curveto(650.22078731,67.34634039)(650.26078727,67.53134021)(650.29078964,67.71134188)
\curveto(650.32078721,67.90133984)(650.36078717,68.08133966)(650.41078964,68.25134187)
\curveto(650.4307871,68.30133944)(650.44078709,68.3413394)(650.44078964,68.37134187)
\curveto(650.44078709,68.40133934)(650.44578708,68.4363393)(650.45578964,68.47634188)
\curveto(650.55578697,68.77633897)(650.64578688,69.07133867)(650.72578964,69.36134187)
\curveto(650.81578671,69.65133809)(650.92078661,69.93133781)(651.04078964,70.20134188)
\curveto(651.30078623,70.78133696)(651.57078596,71.33133641)(651.85078964,71.85134188)
\curveto(652.1307854,72.38133536)(652.44078509,72.88633485)(652.78078964,73.36634187)
\curveto(652.92078461,73.56633417)(653.07078446,73.75633398)(653.23078964,73.93634188)
\curveto(653.39078414,74.12633362)(653.54078399,74.31633343)(653.68078964,74.50634187)
\curveto(653.72078381,74.55633318)(653.75578377,74.60133314)(653.78578964,74.64134187)
\curveto(653.8257837,74.69133305)(653.86078367,74.741333)(653.89078964,74.79134187)
\curveto(653.90078363,74.81133293)(653.91078362,74.8363329)(653.92078964,74.86634187)
\curveto(653.94078359,74.89633284)(653.94078359,74.92633282)(653.92078964,74.95634188)
\curveto(653.90078363,75.01633272)(653.86578366,75.05133269)(653.81578964,75.06134188)
\curveto(653.76578376,75.08133266)(653.71578381,75.10133264)(653.66578964,75.12134187)
\lineto(653.56078964,75.12134187)
\curveto(653.52078401,75.13133261)(653.47078406,75.13133261)(653.41078964,75.12134187)
\lineto(653.26078964,75.12134187)
\lineto(652.66078964,75.12134187)
\lineto(650.02078964,75.12134187)
\lineto(649.28578964,75.12134187)
\lineto(649.04578964,75.12134187)
\curveto(648.97578855,75.13133261)(648.91578861,75.14633259)(648.86578964,75.16634187)
\curveto(648.77578875,75.20633254)(648.71578881,75.26633248)(648.68578964,75.34634188)
\curveto(648.63578889,75.44633229)(648.62078891,75.59133215)(648.64078964,75.78134187)
\curveto(648.66078887,75.98133176)(648.69578883,76.11633163)(648.74578964,76.18634188)
\curveto(648.76578876,76.20633154)(648.79078874,76.22133152)(648.82078964,76.23134187)
\lineto(648.94078964,76.29134187)
\curveto(648.96078857,76.29133145)(648.97578855,76.28633145)(648.98578964,76.27634187)
\curveto(649.00578852,76.27633146)(649.0257885,76.28133146)(649.04578964,76.29134187)
}
}
{
\newrgbcolor{curcolor}{0 0 0}
\pscustom[linestyle=none,fillstyle=solid,fillcolor=curcolor]
{
\newpath
\moveto(666.74039901,74.40134187)
\curveto(666.54038871,74.11133363)(666.33038892,73.82633391)(666.11039901,73.54634187)
\curveto(665.90038935,73.26633448)(665.69538956,72.98133476)(665.49539901,72.69134188)
\curveto(664.89539036,71.8413359)(664.29039096,71.00133674)(663.68039901,70.17134187)
\curveto(663.07039218,69.35133839)(662.46539279,68.51633923)(661.86539901,67.66634187)
\lineto(661.35539901,66.94634188)
\lineto(660.84539901,66.25634187)
\curveto(660.76539449,66.14634159)(660.68539457,66.03134171)(660.60539901,65.91134187)
\curveto(660.52539473,65.79134195)(660.43039482,65.69634204)(660.32039901,65.62634187)
\curveto(660.28039497,65.60634213)(660.21539504,65.59134215)(660.12539901,65.58134188)
\curveto(660.04539521,65.56134218)(659.9553953,65.55134219)(659.85539901,65.55134187)
\curveto(659.7553955,65.55134219)(659.66039559,65.55634218)(659.57039901,65.56634188)
\curveto(659.49039576,65.57634217)(659.43039582,65.59634214)(659.39039901,65.62634187)
\curveto(659.36039589,65.6463421)(659.33539592,65.68134206)(659.31539901,65.73134187)
\curveto(659.30539595,65.77134197)(659.31039594,65.81634192)(659.33039901,65.86634187)
\curveto(659.37039588,65.94634179)(659.41539584,66.02134172)(659.46539901,66.09134188)
\curveto(659.52539573,66.17134157)(659.58039567,66.25134149)(659.63039901,66.33134188)
\curveto(659.87039538,66.67134107)(660.11539514,67.00634073)(660.36539901,67.33634188)
\curveto(660.61539464,67.66634007)(660.8553944,68.00133974)(661.08539901,68.34134188)
\curveto(661.24539401,68.56133918)(661.40539385,68.77633897)(661.56539901,68.98634187)
\curveto(661.72539353,69.19633855)(661.88539337,69.41133833)(662.04539901,69.63134187)
\curveto(662.40539285,70.15133759)(662.77039248,70.66133708)(663.14039901,71.16134187)
\curveto(663.51039174,71.66133608)(663.88039137,72.17133557)(664.25039901,72.69134188)
\curveto(664.39039086,72.89133485)(664.53039072,73.08633465)(664.67039901,73.27634187)
\curveto(664.82039043,73.46633428)(664.96539029,73.66133408)(665.10539901,73.86134187)
\curveto(665.31538994,74.16133358)(665.53038972,74.46133328)(665.75039901,74.76134187)
\lineto(666.41039901,75.66134187)
\lineto(666.59039901,75.93134188)
\lineto(666.80039901,76.20134188)
\lineto(666.92039901,76.38134187)
\curveto(666.97038828,76.4413313)(667.02038823,76.49633124)(667.07039901,76.54634187)
\curveto(667.14038811,76.59633115)(667.21538804,76.63133111)(667.29539901,76.65134187)
\curveto(667.31538794,76.66133108)(667.34038791,76.66133108)(667.37039901,76.65134187)
\curveto(667.41038784,76.65133109)(667.44038781,76.66133108)(667.46039901,76.68134188)
\curveto(667.58038767,76.68133106)(667.71538754,76.67633106)(667.86539901,76.66634187)
\curveto(668.01538724,76.66633108)(668.10538715,76.62133112)(668.13539901,76.53134187)
\curveto(668.1553871,76.50133124)(668.16038709,76.46633128)(668.15039901,76.42634187)
\curveto(668.14038711,76.38633136)(668.12538713,76.35633138)(668.10539901,76.33634188)
\curveto(668.06538719,76.25633149)(668.02538723,76.18633156)(667.98539901,76.12634187)
\curveto(667.94538731,76.06633168)(667.90038735,76.00633174)(667.85039901,75.94634188)
\lineto(667.28039901,75.16634187)
\curveto(667.10038815,74.91633283)(666.92038833,74.66133308)(666.74039901,74.40134187)
\moveto(659.88539901,70.50134187)
\curveto(659.83539542,70.52133722)(659.78539547,70.52633722)(659.73539901,70.51634187)
\curveto(659.68539557,70.50633723)(659.63539562,70.51133723)(659.58539901,70.53134187)
\curveto(659.47539578,70.55133719)(659.37039588,70.57133717)(659.27039901,70.59134188)
\curveto(659.18039607,70.62133712)(659.08539617,70.66133708)(658.98539901,70.71134188)
\curveto(658.6553966,70.85133689)(658.40039685,71.0463367)(658.22039901,71.29634187)
\curveto(658.04039721,71.55633618)(657.89539736,71.86633588)(657.78539901,72.22634188)
\curveto(657.7553975,72.30633543)(657.73539752,72.38633536)(657.72539901,72.46634188)
\curveto(657.71539754,72.55633518)(657.70039755,72.6413351)(657.68039901,72.72134188)
\curveto(657.67039758,72.77133497)(657.66539759,72.8363349)(657.66539901,72.91634187)
\curveto(657.6553976,72.94633479)(657.6503976,72.97633477)(657.65039901,73.00634187)
\curveto(657.6503976,73.0463347)(657.64539761,73.08133466)(657.63539901,73.11134187)
\lineto(657.63539901,73.26134187)
\curveto(657.62539763,73.31133443)(657.62039763,73.37133437)(657.62039901,73.44134188)
\curveto(657.62039763,73.52133422)(657.62539763,73.58633416)(657.63539901,73.63634187)
\lineto(657.63539901,73.80134187)
\curveto(657.6553976,73.85133389)(657.66039759,73.89633384)(657.65039901,73.93634188)
\curveto(657.6503976,73.98633376)(657.6553976,74.03133371)(657.66539901,74.07134188)
\curveto(657.67539758,74.11133363)(657.68039757,74.14633359)(657.68039901,74.17634187)
\curveto(657.68039757,74.21633352)(657.68539757,74.25633349)(657.69539901,74.29634187)
\curveto(657.72539753,74.40633333)(657.74539751,74.51633323)(657.75539901,74.62634187)
\curveto(657.77539748,74.74633299)(657.81039744,74.86133288)(657.86039901,74.97134188)
\curveto(658.00039725,75.31133243)(658.16039709,75.58633216)(658.34039901,75.79634187)
\curveto(658.53039672,76.01633172)(658.80039645,76.19633155)(659.15039901,76.33634188)
\curveto(659.23039602,76.36633137)(659.31539594,76.38633136)(659.40539901,76.39634187)
\curveto(659.49539576,76.41633132)(659.59039566,76.4363313)(659.69039901,76.45634188)
\curveto(659.72039553,76.46633128)(659.77539548,76.46633128)(659.85539901,76.45634188)
\curveto(659.93539532,76.45633129)(659.98539527,76.46633128)(660.00539901,76.48634187)
\curveto(660.56539469,76.49633124)(661.01539424,76.38633136)(661.35539901,76.15634187)
\curveto(661.70539355,75.92633182)(661.96539329,75.62133212)(662.13539901,75.24134187)
\curveto(662.17539308,75.15133259)(662.21039304,75.05633269)(662.24039901,74.95634188)
\curveto(662.27039298,74.85633289)(662.29539296,74.75633298)(662.31539901,74.65634187)
\curveto(662.33539292,74.62633311)(662.34039291,74.59633315)(662.33039901,74.56634188)
\curveto(662.33039292,74.53633321)(662.33539292,74.50633323)(662.34539901,74.47634188)
\curveto(662.37539288,74.36633337)(662.39539286,74.2413335)(662.40539901,74.10134188)
\curveto(662.41539284,73.97133377)(662.42539283,73.8363339)(662.43539901,73.69634188)
\lineto(662.43539901,73.53134187)
\curveto(662.44539281,73.47133427)(662.44539281,73.41633432)(662.43539901,73.36634187)
\curveto(662.42539283,73.31633443)(662.42039283,73.26633448)(662.42039901,73.21634188)
\lineto(662.42039901,73.08134188)
\curveto(662.41039284,73.0413347)(662.40539285,73.00133474)(662.40539901,72.96134188)
\curveto(662.41539284,72.92133482)(662.41039284,72.87633486)(662.39039901,72.82634188)
\curveto(662.37039288,72.71633503)(662.3503929,72.61133513)(662.33039901,72.51134187)
\curveto(662.32039293,72.41133533)(662.30039295,72.31133543)(662.27039901,72.21134188)
\curveto(662.14039311,71.85133589)(661.97539328,71.5363362)(661.77539901,71.26634187)
\curveto(661.57539368,70.99633675)(661.30039395,70.79133695)(660.95039901,70.65134187)
\curveto(660.87039438,70.62133712)(660.78539447,70.59633715)(660.69539901,70.57634188)
\lineto(660.42539901,70.51634187)
\curveto(660.37539488,70.50633723)(660.33039492,70.50133724)(660.29039901,70.50134187)
\curveto(660.250395,70.51133723)(660.21039504,70.51133723)(660.17039901,70.50134187)
\curveto(660.07039518,70.48133726)(659.97539528,70.48133726)(659.88539901,70.50134187)
\moveto(659.04539901,71.89634187)
\curveto(659.08539617,71.82633591)(659.12539613,71.76133598)(659.16539901,71.70134188)
\curveto(659.20539605,71.65133609)(659.255396,71.60133614)(659.31539901,71.55134187)
\lineto(659.46539901,71.43134188)
\curveto(659.52539573,71.40133634)(659.59039566,71.37633637)(659.66039901,71.35634187)
\curveto(659.70039555,71.3363364)(659.73539552,71.32633642)(659.76539901,71.32634188)
\curveto(659.80539545,71.3363364)(659.84539541,71.33133641)(659.88539901,71.31134188)
\curveto(659.91539534,71.31133643)(659.9553953,71.30633643)(660.00539901,71.29634187)
\curveto(660.0553952,71.29633644)(660.09539516,71.30133644)(660.12539901,71.31134188)
\lineto(660.35039901,71.35634187)
\curveto(660.60039465,71.4363363)(660.78539447,71.56133618)(660.90539901,71.73134187)
\curveto(660.98539427,71.83133591)(661.0553942,71.96133578)(661.11539901,72.12134187)
\curveto(661.19539406,72.30133544)(661.255394,72.52633522)(661.29539901,72.79634187)
\curveto(661.33539392,73.07633466)(661.3503939,73.35633438)(661.34039901,73.63634187)
\curveto(661.33039392,73.92633382)(661.30039395,74.20133354)(661.25039901,74.46134188)
\curveto(661.20039405,74.72133302)(661.12539413,74.93133281)(661.02539901,75.09134188)
\curveto(660.90539435,75.29133245)(660.7553945,75.4413323)(660.57539901,75.54134187)
\curveto(660.49539476,75.59133215)(660.40539485,75.62133212)(660.30539901,75.63134187)
\curveto(660.20539505,75.65133209)(660.10039515,75.66133208)(659.99039901,75.66134187)
\curveto(659.97039528,75.65133209)(659.94539531,75.64633209)(659.91539901,75.64634187)
\curveto(659.89539536,75.65633209)(659.87539538,75.65633209)(659.85539901,75.64634187)
\curveto(659.80539545,75.6363321)(659.76039549,75.62633211)(659.72039901,75.61634187)
\curveto(659.68039557,75.61633212)(659.64039561,75.60633214)(659.60039901,75.58634188)
\curveto(659.42039583,75.50633223)(659.27039598,75.38633236)(659.15039901,75.22634188)
\curveto(659.04039621,75.06633268)(658.9503963,74.88633285)(658.88039901,74.68634188)
\curveto(658.82039643,74.49633324)(658.77539648,74.27133347)(658.74539901,74.01134187)
\curveto(658.72539653,73.75133399)(658.72039653,73.48633425)(658.73039901,73.21634188)
\curveto(658.74039651,72.95633478)(658.77039648,72.70633503)(658.82039901,72.46634188)
\curveto(658.88039637,72.2363355)(658.9553963,72.0463357)(659.04539901,71.89634187)
\moveto(669.84539901,68.91134187)
\curveto(669.8553854,68.86133888)(669.86038539,68.77133897)(669.86039901,68.64134187)
\curveto(669.86038539,68.51133923)(669.8503854,68.42133932)(669.83039901,68.37134187)
\curveto(669.81038544,68.32133942)(669.80538545,68.26633947)(669.81539901,68.20634188)
\curveto(669.82538543,68.15633958)(669.82538543,68.10633964)(669.81539901,68.05634188)
\curveto(669.77538548,67.91633983)(669.74538551,67.78133996)(669.72539901,67.65134187)
\curveto(669.71538554,67.52134022)(669.68538557,67.40134034)(669.63539901,67.29134187)
\curveto(669.49538576,66.9413408)(669.33038592,66.6463411)(669.14039901,66.40634187)
\curveto(668.9503863,66.17634157)(668.68038657,65.99134175)(668.33039901,65.85134188)
\curveto(668.250387,65.82134192)(668.16538709,65.80134194)(668.07539901,65.79134187)
\curveto(667.98538727,65.77134197)(667.90038735,65.75134199)(667.82039901,65.73134187)
\curveto(667.77038748,65.72134202)(667.72038753,65.71634203)(667.67039901,65.71634188)
\curveto(667.62038763,65.71634203)(667.57038768,65.71134203)(667.52039901,65.70134188)
\curveto(667.49038776,65.69134205)(667.44038781,65.69134205)(667.37039901,65.70134188)
\curveto(667.30038795,65.70134204)(667.250388,65.70634204)(667.22039901,65.71634188)
\curveto(667.16038809,65.736342)(667.10038815,65.74634199)(667.04039901,65.74634187)
\curveto(666.99038826,65.736342)(666.94038831,65.741342)(666.89039901,65.76134187)
\curveto(666.80038845,65.78134196)(666.71038854,65.80634193)(666.62039901,65.83634188)
\curveto(666.54038871,65.85634189)(666.46038879,65.88634185)(666.38039901,65.92634187)
\curveto(666.06038919,66.06634167)(665.81038944,66.26134148)(665.63039901,66.51134187)
\curveto(665.4503898,66.77134097)(665.30038995,67.07634066)(665.18039901,67.42634187)
\curveto(665.16039009,67.50634024)(665.14539011,67.59134015)(665.13539901,67.68134188)
\curveto(665.12539013,67.77133997)(665.11039014,67.85633989)(665.09039901,67.93634188)
\curveto(665.08039017,67.96633977)(665.07539018,67.99633975)(665.07539901,68.02634187)
\lineto(665.07539901,68.13134187)
\curveto(665.0553902,68.21133953)(665.04539021,68.29133945)(665.04539901,68.37134187)
\lineto(665.04539901,68.50634187)
\curveto(665.02539023,68.60633913)(665.02539023,68.70633904)(665.04539901,68.80634188)
\lineto(665.04539901,68.98634187)
\curveto(665.0553902,69.0363387)(665.06039019,69.08133866)(665.06039901,69.12134187)
\curveto(665.06039019,69.17133857)(665.06539019,69.21633852)(665.07539901,69.25634187)
\curveto(665.08539017,69.29633844)(665.09039016,69.33133841)(665.09039901,69.36134187)
\curveto(665.09039016,69.40133834)(665.09539016,69.4413383)(665.10539901,69.48134187)
\lineto(665.16539901,69.81134188)
\curveto(665.18539007,69.93133781)(665.21539004,70.0413377)(665.25539901,70.14134187)
\curveto(665.39538986,70.47133727)(665.5553897,70.74633699)(665.73539901,70.96634188)
\curveto(665.92538933,71.19633655)(666.18538907,71.38133636)(666.51539901,71.52134187)
\curveto(666.59538866,71.56133618)(666.68038857,71.58633616)(666.77039901,71.59634188)
\lineto(667.07039901,71.65634187)
\lineto(667.20539901,71.65634187)
\curveto(667.255388,71.66633608)(667.30538795,71.67133607)(667.35539901,71.67134187)
\curveto(667.92538733,71.69133605)(668.38538687,71.58633616)(668.73539901,71.35634187)
\curveto(669.09538616,71.1363366)(669.36038589,70.8363369)(669.53039901,70.45634188)
\curveto(669.58038567,70.35633738)(669.62038563,70.25633749)(669.65039901,70.15634187)
\curveto(669.68038557,70.05633769)(669.71038554,69.95133779)(669.74039901,69.84134188)
\curveto(669.7503855,69.80133794)(669.7553855,69.76633797)(669.75539901,69.73634187)
\curveto(669.7553855,69.71633803)(669.76038549,69.68633805)(669.77039901,69.64634187)
\curveto(669.79038546,69.57633817)(669.80038545,69.50133824)(669.80039901,69.42134187)
\curveto(669.80038545,69.3413384)(669.81038544,69.26133848)(669.83039901,69.18134188)
\curveto(669.83038542,69.13133861)(669.83038542,69.08633865)(669.83039901,69.04634187)
\curveto(669.83038542,69.00633873)(669.83538542,68.96133878)(669.84539901,68.91134187)
\moveto(668.73539901,68.47634188)
\curveto(668.74538651,68.52633922)(668.7503865,68.60133914)(668.75039901,68.70134188)
\curveto(668.76038649,68.80133894)(668.7553865,68.87633886)(668.73539901,68.92634187)
\curveto(668.71538654,68.98633876)(668.71038654,69.0413387)(668.72039901,69.09134188)
\curveto(668.74038651,69.15133859)(668.74038651,69.21133853)(668.72039901,69.27134187)
\curveto(668.71038654,69.30133844)(668.70538655,69.3363384)(668.70539901,69.37634187)
\curveto(668.70538655,69.41633832)(668.70038655,69.45633829)(668.69039901,69.49634187)
\curveto(668.67038658,69.57633817)(668.6503866,69.65133809)(668.63039901,69.72134188)
\curveto(668.62038663,69.80133794)(668.60538665,69.88133786)(668.58539901,69.96134188)
\curveto(668.5553867,70.02133772)(668.53038672,70.08133766)(668.51039901,70.14134187)
\curveto(668.49038676,70.20133754)(668.46038679,70.26133748)(668.42039901,70.32134188)
\curveto(668.32038693,70.49133725)(668.19038706,70.62633711)(668.03039901,70.72634188)
\curveto(667.9503873,70.77633697)(667.8553874,70.81133693)(667.74539901,70.83134188)
\curveto(667.63538762,70.85133689)(667.51038774,70.86133688)(667.37039901,70.86134187)
\curveto(667.3503879,70.85133689)(667.32538793,70.8463369)(667.29539901,70.84634188)
\curveto(667.26538799,70.85633689)(667.23538802,70.85633689)(667.20539901,70.84634188)
\lineto(667.05539901,70.78634187)
\curveto(667.00538825,70.77633697)(666.96038829,70.76133698)(666.92039901,70.74134187)
\curveto(666.73038852,70.63133711)(666.58538867,70.48633725)(666.48539901,70.30634188)
\curveto(666.39538886,70.12633762)(666.31538894,69.92133782)(666.24539901,69.69134188)
\curveto(666.20538905,69.56133818)(666.18538907,69.42633831)(666.18539901,69.28634187)
\curveto(666.18538907,69.15633858)(666.17538908,69.01133873)(666.15539901,68.85134188)
\curveto(666.14538911,68.80133894)(666.13538912,68.741339)(666.12539901,68.67134187)
\curveto(666.12538913,68.60133914)(666.13538912,68.5413392)(666.15539901,68.49134187)
\lineto(666.15539901,68.32634188)
\lineto(666.15539901,68.14634187)
\curveto(666.16538909,68.09633964)(666.17538908,68.0413397)(666.18539901,67.98134187)
\curveto(666.19538906,67.93133981)(666.20038905,67.87633986)(666.20039901,67.81634188)
\curveto(666.21038904,67.75633998)(666.22538903,67.70134004)(666.24539901,67.65134187)
\curveto(666.29538896,67.46134028)(666.3553889,67.28634045)(666.42539901,67.12634187)
\curveto(666.49538876,66.96634077)(666.60038865,66.83634091)(666.74039901,66.73634187)
\curveto(666.87038838,66.63634111)(667.01038824,66.56634117)(667.16039901,66.52634187)
\curveto(667.19038806,66.51634123)(667.21538804,66.51134123)(667.23539901,66.51134187)
\curveto(667.26538799,66.52134122)(667.29538796,66.52134122)(667.32539901,66.51134187)
\curveto(667.34538791,66.51134123)(667.37538788,66.50634124)(667.41539901,66.49634187)
\curveto(667.4553878,66.49634124)(667.49038776,66.50134124)(667.52039901,66.51134187)
\curveto(667.56038769,66.52134122)(667.60038765,66.52634122)(667.64039901,66.52634187)
\curveto(667.68038757,66.52634122)(667.72038753,66.5363412)(667.76039901,66.55634188)
\curveto(668.00038725,66.63634111)(668.19538706,66.77134097)(668.34539901,66.96134188)
\curveto(668.46538679,67.1413406)(668.5553867,67.34634039)(668.61539901,67.57634188)
\curveto(668.63538662,67.6463401)(668.6503866,67.71634003)(668.66039901,67.78634187)
\curveto(668.67038658,67.86633987)(668.68538657,67.94633979)(668.70539901,68.02634187)
\curveto(668.70538655,68.08633965)(668.71038654,68.13133961)(668.72039901,68.16134187)
\curveto(668.72038653,68.18133956)(668.72038653,68.20633953)(668.72039901,68.23634187)
\curveto(668.72038653,68.27633946)(668.72538653,68.30633944)(668.73539901,68.32634188)
\lineto(668.73539901,68.47634188)
}
}
\end{pspicture}

\caption{Porcentajes de los espacios y sus recursos según su tipo}
\label{espacios_pie_1}
\end{figure}

\subsection{Recursos}
Los recursos pueden ser de varios tipos (figura \ref{recursos_tabla_1}),
destacando la gran cantidad de notas (figura \ref{recursos_bars_1}) por sobre
los otros tipos de recursos, pudiendo esto deberse a la inmensa facilidad de
creación de estas. Aun así son las fotografías la que en proporción reciben
mejor audiencia, y son los archivos los que reciben mayor cantidad de 
comentarios (figura \ref{recursos_pie_1}).

\begin{table}
\centering
\begin{tabular}{l|c c c c c}
$Tipo$ & $Cantidad$ & $Audiencia$ & $Comentarios$ &
$Calificadores$ & $Etiquetas$ \\
\hline
$Notas      $ & $42$ & $811$ & $17$ & $19$ & $61$ \\
$Archivos   $ & $13$ & $ 72$ & $ 9$ & $ 1$ & $13$ \\
$Eventos    $ & $ 4$ & $243$ & $ 2$ & $ 0$ & $ 5$ \\
$Enlaces    $ & $ 3$ & $ 10$ & $ 1$ & $ 0$ & $ 7$ \\
$Fotografias$ & $ 5$ & $394$ & $ 4$ & $ 2$ & $12$ \\
$Videos     $ & $ 1$ & $  1$ & $ 0$ & $ 0$ & $ 2$ \\
\hline
 & & $Aud/Can$ & $Com/Aud$ & $Cal/Aud$ & $Eti/Can$ \\
\hline
$Notas      $ & & $19.31$ & $0.021$ & $0.023$ & $1.452$ \\
$Archivos   $ & & $ 5.53$ & $0.125$ & $0.014$ & $1    $ \\
$Eventos    $ & & $60.75$ & $0.008$ & $0    $ & $1.250$ \\
$Enlaces    $ & & $ 3.33$ & $0.100$ & $0    $ & $2.333$ \\
$Fotografias$ & & $78.80$ & $0.010$ & $0.005$ & $2.400$ \\
$Videos     $ & & $ 1   $ & $0    $ & $0    $ & $2    $ \\
\end{tabular}
\caption{Clasificación de los recursos según su tipo}
\label{recursos_tabla_1}
\end{table}

\begin{figure}
\centering
%LaTeX with PSTricks extensions
%%Creator: inkscape 0.48.5
%%Please note this file requires PSTricks extensions
\psset{xunit=.5pt,yunit=.5pt,runit=.5pt}
\begin{pspicture}(865,422)
{
\newrgbcolor{curcolor}{0 0 0}
\pscustom[linestyle=none,fillstyle=solid,fillcolor=curcolor]
{
\newpath
\moveto(358.52975342,24.63030273)
\lineto(363.43475342,24.63030273)
\lineto(364.72475342,24.63030273)
\curveto(364.83474339,24.63029204)(364.94474328,24.63029204)(365.05475342,24.63030273)
\curveto(365.16474306,24.64029203)(365.24974298,24.62029205)(365.30975342,24.57030273)
\curveto(365.3297429,24.55029212)(365.34474288,24.52529214)(365.35475342,24.49530273)
\curveto(365.37474285,24.4652922)(365.39474283,24.43529223)(365.41475342,24.40530273)
\curveto(365.41474281,24.33529233)(365.39974283,24.22029245)(365.36975342,24.06030273)
\curveto(365.33974289,23.91029276)(365.30474292,23.79529287)(365.26475342,23.71530273)
\curveto(365.20474302,23.57529309)(365.10474312,23.49529317)(364.96475342,23.47530273)
\curveto(364.83474339,23.4652932)(364.67974355,23.46029321)(364.49975342,23.46030273)
\lineto(362.99975342,23.46030273)
\lineto(360.47975342,23.46030273)
\lineto(359.90975342,23.46030273)
\curveto(359.69974853,23.4702932)(359.53974869,23.44529322)(359.42975342,23.38530273)
\curveto(359.31974891,23.32529334)(359.24474898,23.22029345)(359.20475342,23.07030273)
\curveto(359.17474905,22.92029375)(359.14474908,22.7652939)(359.11475342,22.60530273)
\lineto(358.79975342,21.07530273)
\curveto(358.77974945,20.9652957)(358.74974948,20.83529583)(358.70975342,20.68530273)
\curveto(358.67974955,20.53529613)(358.66474956,20.41529625)(358.66475342,20.32530273)
\curveto(358.67474955,20.20529646)(358.71974951,20.12529654)(358.79975342,20.08530273)
\curveto(358.83974939,20.0652966)(358.90474932,20.04529662)(358.99475342,20.02530273)
\lineto(359.14475342,20.02530273)
\curveto(359.18474904,20.01529665)(359.224749,20.01029666)(359.26475342,20.01030273)
\curveto(359.31474891,20.02029665)(359.36474886,20.02529664)(359.41475342,20.02530273)
\lineto(359.92475342,20.02530273)
\lineto(362.86475342,20.02530273)
\lineto(363.16475342,20.02530273)
\curveto(363.27474495,20.03529663)(363.38474484,20.03529663)(363.49475342,20.02530273)
\curveto(363.61474461,20.02529664)(363.71974451,20.01529665)(363.80975342,19.99530273)
\curveto(363.90974432,19.98529668)(363.97974425,19.9652967)(364.01975342,19.93530273)
\curveto(364.04974418,19.91529675)(364.06474416,19.8702968)(364.06475342,19.80030273)
\curveto(364.07474415,19.73029694)(364.07474415,19.65529701)(364.06475342,19.57530273)
\curveto(364.06474416,19.49529717)(364.04974418,19.41029726)(364.01975342,19.32030273)
\curveto(363.99974423,19.24029743)(363.97474425,19.1702975)(363.94475342,19.11030273)
\curveto(363.90474432,19.02029765)(363.84474438,18.95529771)(363.76475342,18.91530273)
\curveto(363.74474448,18.89529777)(363.71474451,18.88029779)(363.67475342,18.87030273)
\curveto(363.64474458,18.8702978)(363.61474461,18.8652978)(363.58475342,18.85530273)
\lineto(363.49475342,18.85530273)
\curveto(363.43474479,18.84529782)(363.37974485,18.84029783)(363.32975342,18.84030273)
\curveto(363.28974494,18.85029782)(363.24474498,18.85529781)(363.19475342,18.85530273)
\lineto(362.63975342,18.85530273)
\lineto(359.47475342,18.85530273)
\lineto(359.11475342,18.85530273)
\curveto(359.00474922,18.8652978)(358.89474933,18.86029781)(358.78475342,18.84030273)
\curveto(358.68474954,18.83029784)(358.59474963,18.80529786)(358.51475342,18.76530273)
\curveto(358.43474979,18.72529794)(358.36974986,18.65529801)(358.31975342,18.55530273)
\curveto(358.27974995,18.49529817)(358.25474997,18.42529824)(358.24475342,18.34530273)
\lineto(358.21475342,18.10530273)
\lineto(358.03475342,17.26530273)
\lineto(357.74975342,15.84030273)
\curveto(357.7297505,15.70030097)(357.70975052,15.5703011)(357.68975342,15.45030273)
\curveto(357.67975055,15.34030133)(357.70475052,15.26030141)(357.76475342,15.21030273)
\curveto(357.8247504,15.16030151)(357.89975033,15.13030154)(357.98975342,15.12030273)
\lineto(358.28975342,15.12030273)
\lineto(359.24975342,15.12030273)
\lineto(362.02475342,15.12030273)
\lineto(362.87975342,15.12030273)
\lineto(363.11975342,15.12030273)
\curveto(363.19974503,15.13030154)(363.26974496,15.12530154)(363.32975342,15.10530273)
\curveto(363.43974479,15.0653016)(363.50474472,15.01030166)(363.52475342,14.94030273)
\curveto(363.54474468,14.91030176)(363.54974468,14.86030181)(363.53975342,14.79030273)
\curveto(363.53974469,14.72030195)(363.53474469,14.64530202)(363.52475342,14.56530273)
\curveto(363.51474471,14.49530217)(363.49474473,14.42030225)(363.46475342,14.34030273)
\curveto(363.44474478,14.2703024)(363.4247448,14.21530245)(363.40475342,14.17530273)
\curveto(363.35474487,14.09530257)(363.29974493,14.04030263)(363.23975342,14.01030273)
\curveto(363.16974506,13.9703027)(363.08474514,13.95030272)(362.98475342,13.95030273)
\lineto(362.71475342,13.95030273)
\lineto(361.66475342,13.95030273)
\lineto(357.67475342,13.95030273)
\lineto(356.62475342,13.95030273)
\curveto(356.48475174,13.95030272)(356.36475186,13.95530271)(356.26475342,13.96530273)
\curveto(356.17475205,13.98530268)(356.10975212,14.03530263)(356.06975342,14.11530273)
\curveto(356.04975218,14.17530249)(356.04475218,14.25030242)(356.05475342,14.34030273)
\curveto(356.07475215,14.44030223)(356.09475213,14.53530213)(356.11475342,14.62530273)
\lineto(356.32475342,15.67530273)
\lineto(357.13475342,19.69530273)
\lineto(357.80975342,23.05530273)
\lineto(357.98975342,23.98530273)
\curveto(358.00975022,24.07529259)(358.0247502,24.1652925)(358.03475342,24.25530273)
\curveto(358.05475017,24.34529232)(358.08975014,24.41529225)(358.13975342,24.46530273)
\curveto(358.19975003,24.53529213)(358.28474994,24.58529208)(358.39475342,24.61530273)
\curveto(358.4247498,24.62529204)(358.44474978,24.62529204)(358.45475342,24.61530273)
\curveto(358.47474975,24.61529205)(358.49974973,24.62029205)(358.52975342,24.63030273)
}
}
{
\newrgbcolor{curcolor}{0 0 0}
\pscustom[linestyle=none,fillstyle=solid,fillcolor=curcolor]
{
\newpath
\moveto(368.91467529,21.85530273)
\curveto(369.63466964,21.8652948)(370.21966905,21.78029489)(370.66967529,21.60030273)
\curveto(371.12966814,21.43029524)(371.44966782,21.12529554)(371.62967529,20.68530273)
\curveto(371.67966759,20.57529609)(371.70966756,20.46029621)(371.71967529,20.34030273)
\curveto(371.73966753,20.23029644)(371.75466752,20.10529656)(371.76467529,19.96530273)
\curveto(371.7746675,19.89529677)(371.76466751,19.82029685)(371.73467529,19.74030273)
\curveto(371.71466756,19.670297)(371.68966758,19.61529705)(371.65967529,19.57530273)
\curveto(371.63966763,19.55529711)(371.60966766,19.53529713)(371.56967529,19.51530273)
\curveto(371.53966773,19.50529716)(371.51466776,19.49029718)(371.49467529,19.47030273)
\curveto(371.43466784,19.45029722)(371.37966789,19.44529722)(371.32967529,19.45530273)
\curveto(371.28966798,19.4652972)(371.24466803,19.4652972)(371.19467529,19.45530273)
\curveto(371.10466817,19.43529723)(370.99466828,19.43029724)(370.86467529,19.44030273)
\curveto(370.74466853,19.46029721)(370.65966861,19.48529718)(370.60967529,19.51530273)
\curveto(370.53966873,19.5652971)(370.49966877,19.63029704)(370.48967529,19.71030273)
\curveto(370.48966878,19.80029687)(370.4696688,19.88529678)(370.42967529,19.96530273)
\curveto(370.37966889,20.12529654)(370.28466899,20.2702964)(370.14467529,20.40030273)
\curveto(370.05466922,20.48029619)(369.94466933,20.54029613)(369.81467529,20.58030273)
\curveto(369.69466958,20.62029605)(369.56466971,20.66029601)(369.42467529,20.70030273)
\curveto(369.38466989,20.72029595)(369.33466994,20.72529594)(369.27467529,20.71530273)
\curveto(369.22467005,20.71529595)(369.17967009,20.72029595)(369.13967529,20.73030273)
\curveto(369.07967019,20.75029592)(369.00467027,20.76029591)(368.91467529,20.76030273)
\curveto(368.82467045,20.76029591)(368.74967052,20.75029592)(368.68967529,20.73030273)
\lineto(368.59967529,20.73030273)
\curveto(368.53967073,20.72029595)(368.48467079,20.71029596)(368.43467529,20.70030273)
\curveto(368.38467089,20.70029597)(368.33467094,20.69529597)(368.28467529,20.68530273)
\curveto(368.01467126,20.62529604)(367.77967149,20.54029613)(367.57967529,20.43030273)
\curveto(367.38967188,20.32029635)(367.23967203,20.13529653)(367.12967529,19.87530273)
\curveto(367.09967217,19.80529686)(367.08467219,19.73529693)(367.08467529,19.66530273)
\curveto(367.08467219,19.59529707)(367.08967218,19.53529713)(367.09967529,19.48530273)
\curveto(367.12967214,19.33529733)(367.17967209,19.22529744)(367.24967529,19.15530273)
\curveto(367.31967195,19.09529757)(367.41467186,19.02529764)(367.53467529,18.94530273)
\curveto(367.6746716,18.84529782)(367.83967143,18.7702979)(368.02967529,18.72030273)
\curveto(368.21967105,18.68029799)(368.40967086,18.63029804)(368.59967529,18.57030273)
\curveto(368.71967055,18.53029814)(368.83967043,18.50029817)(368.95967529,18.48030273)
\curveto(369.08967018,18.46029821)(369.21467006,18.43029824)(369.33467529,18.39030273)
\curveto(369.53466974,18.33029834)(369.72966954,18.2702984)(369.91967529,18.21030273)
\curveto(370.10966916,18.16029851)(370.29466898,18.09529857)(370.47467529,18.01530273)
\curveto(370.52466875,17.99529867)(370.5696687,17.97529869)(370.60967529,17.95530273)
\curveto(370.65966861,17.93529873)(370.70966856,17.91029876)(370.75967529,17.88030273)
\curveto(370.92966834,17.76029891)(371.0746682,17.62529904)(371.19467529,17.47530273)
\curveto(371.31466796,17.32529934)(371.40466787,17.13529953)(371.46467529,16.90530273)
\lineto(371.46467529,16.62030273)
\curveto(371.46466781,16.55030012)(371.45966781,16.47530019)(371.44967529,16.39530273)
\curveto(371.43966783,16.32530034)(371.42966784,16.24530042)(371.41967529,16.15530273)
\lineto(371.38967529,16.00530273)
\curveto(371.34966792,15.93530073)(371.31966795,15.8653008)(371.29967529,15.79530273)
\curveto(371.28966798,15.72530094)(371.269668,15.65530101)(371.23967529,15.58530273)
\curveto(371.18966808,15.47530119)(371.13466814,15.3703013)(371.07467529,15.27030273)
\curveto(371.01466826,15.1703015)(370.94966832,15.08030159)(370.87967529,15.00030273)
\curveto(370.6696686,14.74030193)(370.42466885,14.53030214)(370.14467529,14.37030273)
\curveto(369.86466941,14.22030245)(369.55966971,14.09030258)(369.22967529,13.98030273)
\curveto(369.12967014,13.95030272)(369.02967024,13.93030274)(368.92967529,13.92030273)
\curveto(368.82967044,13.90030277)(368.73467054,13.87530279)(368.64467529,13.84530273)
\curveto(368.53467074,13.82530284)(368.42967084,13.81530285)(368.32967529,13.81530273)
\curveto(368.22967104,13.81530285)(368.12967114,13.80530286)(368.02967529,13.78530273)
\lineto(367.87967529,13.78530273)
\curveto(367.82967144,13.77530289)(367.75967151,13.7703029)(367.66967529,13.77030273)
\curveto(367.57967169,13.7703029)(367.50967176,13.77530289)(367.45967529,13.78530273)
\lineto(367.29467529,13.78530273)
\curveto(367.23467204,13.80530286)(367.1696721,13.81530285)(367.09967529,13.81530273)
\curveto(367.02967224,13.80530286)(366.9746723,13.81030286)(366.93467529,13.83030273)
\curveto(366.88467239,13.84030283)(366.81967245,13.84530282)(366.73967529,13.84530273)
\curveto(366.65967261,13.8653028)(366.58467269,13.88530278)(366.51467529,13.90530273)
\curveto(366.44467283,13.91530275)(366.3696729,13.93530273)(366.28967529,13.96530273)
\curveto(365.99967327,14.0653026)(365.75467352,14.19030248)(365.55467529,14.34030273)
\curveto(365.35467392,14.49030218)(365.19467408,14.68530198)(365.07467529,14.92530273)
\curveto(365.01467426,15.05530161)(364.96467431,15.19030148)(364.92467529,15.33030273)
\curveto(364.89467438,15.4703012)(364.8746744,15.62530104)(364.86467529,15.79530273)
\curveto(364.85467442,15.85530081)(364.85967441,15.92530074)(364.87967529,16.00530273)
\curveto(364.89967437,16.09530057)(364.92467435,16.1653005)(364.95467529,16.21530273)
\curveto(364.99467428,16.25530041)(365.05467422,16.29530037)(365.13467529,16.33530273)
\curveto(365.18467409,16.35530031)(365.25467402,16.3653003)(365.34467529,16.36530273)
\curveto(365.44467383,16.37530029)(365.53467374,16.37530029)(365.61467529,16.36530273)
\curveto(365.70467357,16.35530031)(365.78967348,16.34030033)(365.86967529,16.32030273)
\curveto(365.95967331,16.31030036)(366.01467326,16.29530037)(366.03467529,16.27530273)
\curveto(366.09467318,16.22530044)(366.12467315,16.15030052)(366.12467529,16.05030273)
\curveto(366.13467314,15.96030071)(366.15467312,15.87530079)(366.18467529,15.79530273)
\curveto(366.23467304,15.57530109)(366.33467294,15.40530126)(366.48467529,15.28530273)
\curveto(366.58467269,15.19530147)(366.70467257,15.12530154)(366.84467529,15.07530273)
\curveto(366.98467229,15.02530164)(367.13467214,14.97530169)(367.29467529,14.92530273)
\lineto(367.60967529,14.88030273)
\lineto(367.69967529,14.88030273)
\curveto(367.75967151,14.86030181)(367.84467143,14.85030182)(367.95467529,14.85030273)
\curveto(368.0746712,14.85030182)(368.17967109,14.86030181)(368.26967529,14.88030273)
\curveto(368.33967093,14.88030179)(368.39467088,14.88530178)(368.43467529,14.89530273)
\curveto(368.49467078,14.90530176)(368.55467072,14.91030176)(368.61467529,14.91030273)
\curveto(368.6746706,14.92030175)(368.72967054,14.93030174)(368.77967529,14.94030273)
\curveto(369.08967018,15.02030165)(369.33966993,15.12530154)(369.52967529,15.25530273)
\curveto(369.72966954,15.38530128)(369.89466938,15.60530106)(370.02467529,15.91530273)
\curveto(370.05466922,15.9653007)(370.0696692,16.02030065)(370.06967529,16.08030273)
\curveto(370.07966919,16.14030053)(370.07966919,16.18530048)(370.06967529,16.21530273)
\curveto(370.05966921,16.40530026)(370.01966925,16.54530012)(369.94967529,16.63530273)
\curveto(369.87966939,16.73529993)(369.78466949,16.82529984)(369.66467529,16.90530273)
\curveto(369.58466969,16.9652997)(369.48966978,17.01529965)(369.37967529,17.05530273)
\lineto(369.07967529,17.17530273)
\curveto(369.04967022,17.18529948)(369.01967025,17.19029948)(368.98967529,17.19030273)
\curveto(368.9696703,17.19029948)(368.94967032,17.20029947)(368.92967529,17.22030273)
\curveto(368.60967066,17.33029934)(368.269671,17.41029926)(367.90967529,17.46030273)
\curveto(367.55967171,17.52029915)(367.23967203,17.61529905)(366.94967529,17.74530273)
\curveto(366.85967241,17.78529888)(366.7696725,17.82029885)(366.67967529,17.85030273)
\curveto(366.59967267,17.88029879)(366.52467275,17.92029875)(366.45467529,17.97030273)
\curveto(366.28467299,18.08029859)(366.13467314,18.20529846)(366.00467529,18.34530273)
\curveto(365.8746734,18.48529818)(365.78467349,18.66029801)(365.73467529,18.87030273)
\curveto(365.71467356,18.94029773)(365.70467357,19.01029766)(365.70467529,19.08030273)
\lineto(365.70467529,19.30530273)
\curveto(365.69467358,19.42529724)(365.70967356,19.56029711)(365.74967529,19.71030273)
\curveto(365.78967348,19.8702968)(365.82967344,20.00529666)(365.86967529,20.11530273)
\curveto(365.89967337,20.1652965)(365.91967335,20.20529646)(365.92967529,20.23530273)
\curveto(365.94967332,20.27529639)(365.9746733,20.31529635)(366.00467529,20.35530273)
\curveto(366.13467314,20.58529608)(366.29467298,20.78529588)(366.48467529,20.95530273)
\curveto(366.6746726,21.12529554)(366.88467239,21.27529539)(367.11467529,21.40530273)
\curveto(367.274672,21.49529517)(367.44967182,21.5652951)(367.63967529,21.61530273)
\curveto(367.83967143,21.67529499)(368.04467123,21.73029494)(368.25467529,21.78030273)
\curveto(368.32467095,21.79029488)(368.38967088,21.80029487)(368.44967529,21.81030273)
\curveto(368.51967075,21.82029485)(368.59467068,21.83029484)(368.67467529,21.84030273)
\curveto(368.71467056,21.85029482)(368.75467052,21.85029482)(368.79467529,21.84030273)
\curveto(368.84467043,21.83029484)(368.88467039,21.83529483)(368.91467529,21.85530273)
}
}
{
\newrgbcolor{curcolor}{0 0 0}
\pscustom[linestyle=none,fillstyle=solid,fillcolor=curcolor]
{
\newpath
\moveto(380.62967529,18.01530273)
\curveto(380.62966614,17.9652987)(380.61966615,17.90029877)(380.59967529,17.82030273)
\curveto(380.58966618,17.74029893)(380.5746662,17.67529899)(380.55467529,17.62530273)
\curveto(380.52466625,17.57529909)(380.50966626,17.52529914)(380.50967529,17.47530273)
\curveto(380.50966626,17.43529923)(380.50466627,17.39529927)(380.49467529,17.35530273)
\curveto(380.4746663,17.28529938)(380.45466632,17.23029944)(380.43467529,17.19030273)
\lineto(380.34467529,16.92030273)
\curveto(380.32466645,16.83029984)(380.29466648,16.74029993)(380.25467529,16.65030273)
\curveto(380.22466655,16.5703001)(380.18966658,16.49030018)(380.14967529,16.41030273)
\curveto(380.11966665,16.34030033)(380.07966669,16.2653004)(380.02967529,16.18530273)
\curveto(379.83966693,15.81530085)(379.60966716,15.48030119)(379.33967529,15.18030273)
\curveto(379.25966751,15.09030158)(379.1746676,15.00030167)(379.08467529,14.91030273)
\curveto(378.99466778,14.83030184)(378.90466787,14.75530191)(378.81467529,14.68530273)
\lineto(378.72467529,14.61030273)
\curveto(378.64466813,14.56030211)(378.5696682,14.51030216)(378.49967529,14.46030273)
\curveto(378.42966834,14.41030226)(378.34966842,14.36030231)(378.25967529,14.31030273)
\curveto(378.12966864,14.23030244)(377.98966878,14.16030251)(377.83967529,14.10030273)
\curveto(377.69966907,14.05030262)(377.55466922,14.00030267)(377.40467529,13.95030273)
\curveto(377.32466945,13.93030274)(377.24466953,13.91530275)(377.16467529,13.90530273)
\curveto(377.08466969,13.89530277)(377.00466977,13.88030279)(376.92467529,13.86030273)
\lineto(376.86467529,13.86030273)
\curveto(376.85466992,13.85030282)(376.83966993,13.84530282)(376.81967529,13.84530273)
\curveto(376.71967005,13.82530284)(376.57967019,13.81530285)(376.39967529,13.81530273)
\curveto(376.22967054,13.80530286)(376.09967067,13.81030286)(376.00967529,13.83030273)
\lineto(375.93467529,13.83030273)
\curveto(375.86467091,13.84030283)(375.79967097,13.85030282)(375.73967529,13.86030273)
\curveto(375.67967109,13.86030281)(375.61967115,13.8703028)(375.55967529,13.89030273)
\curveto(375.39967137,13.94030273)(375.24967152,13.98530268)(375.10967529,14.02530273)
\curveto(374.9696718,14.0653026)(374.84467193,14.12530254)(374.73467529,14.20530273)
\curveto(374.58467219,14.29530237)(374.45967231,14.39030228)(374.35967529,14.49030273)
\curveto(374.32967244,14.52030215)(374.27967249,14.56030211)(374.20967529,14.61030273)
\curveto(374.13967263,14.670302)(374.06467271,14.67530199)(373.98467529,14.62530273)
\curveto(373.94467283,14.59530207)(373.91967285,14.55530211)(373.90967529,14.50530273)
\curveto(373.89967287,14.45530221)(373.8746729,14.40030227)(373.83467529,14.34030273)
\curveto(373.82467295,14.31030236)(373.81967295,14.27530239)(373.81967529,14.23530273)
\curveto(373.81967295,14.20530246)(373.81467296,14.1703025)(373.80467529,14.13030273)
\curveto(373.76467301,14.0703026)(373.73967303,14.00530266)(373.72967529,13.93530273)
\curveto(373.71967305,13.85530281)(373.70967306,13.78530288)(373.69967529,13.72530273)
\lineto(373.33967529,11.92530273)
\curveto(373.30967346,11.78530488)(373.27967349,11.64030503)(373.24967529,11.49030273)
\curveto(373.21967355,11.34030533)(373.1696736,11.22530544)(373.09967529,11.14530273)
\curveto(373.02967374,11.07530559)(372.93967383,11.04030563)(372.82967529,11.04030273)
\curveto(372.71967405,11.03030564)(372.60967416,11.02530564)(372.49967529,11.02530273)
\lineto(372.25967529,11.02530273)
\curveto(372.19967457,11.04530562)(372.14467463,11.0653056)(372.09467529,11.08530273)
\curveto(372.05467472,11.10530556)(372.02467475,11.14030553)(372.00467529,11.19030273)
\curveto(371.9746748,11.26030541)(371.9746748,11.3703053)(372.00467529,11.52030273)
\curveto(372.04467473,11.670305)(372.0746747,11.80030487)(372.09467529,11.91030273)
\lineto(373.89467529,20.91030273)
\curveto(373.91467286,21.03029564)(373.93967283,21.15029552)(373.96967529,21.27030273)
\curveto(373.99967277,21.40029527)(374.04967272,21.50529516)(374.11967529,21.58530273)
\curveto(374.15967261,21.62529504)(374.23467254,21.65529501)(374.34467529,21.67530273)
\curveto(374.45467232,21.70529496)(374.5696722,21.71529495)(374.68967529,21.70530273)
\curveto(374.80967196,21.70529496)(374.91967185,21.69029498)(375.01967529,21.66030273)
\curveto(375.11967165,21.64029503)(375.17967159,21.61029506)(375.19967529,21.57030273)
\curveto(375.22967154,21.52029515)(375.23967153,21.46029521)(375.22967529,21.39030273)
\curveto(375.21967155,21.32029535)(375.22467155,21.25029542)(375.24467529,21.18030273)
\curveto(375.25467152,21.15029552)(375.26467151,21.12529554)(375.27467529,21.10530273)
\lineto(375.31967529,21.06030273)
\curveto(375.42967134,21.05029562)(375.52967124,21.08529558)(375.61967529,21.16530273)
\curveto(375.70967106,21.24529542)(375.79467098,21.31029536)(375.87467529,21.36030273)
\curveto(376.1746706,21.54029513)(376.51467026,21.68029499)(376.89467529,21.78030273)
\curveto(376.98466979,21.80029487)(377.0746697,21.81529485)(377.16467529,21.82530273)
\curveto(377.26466951,21.83529483)(377.3696694,21.85029482)(377.47967529,21.87030273)
\curveto(377.51966925,21.88029479)(377.5696692,21.88029479)(377.62967529,21.87030273)
\curveto(377.68966908,21.86029481)(377.72966904,21.8652948)(377.74967529,21.88530273)
\curveto(378.17966859,21.89529477)(378.54966822,21.85029482)(378.85967529,21.75030273)
\curveto(379.1696676,21.66029501)(379.43966733,21.53029514)(379.66967529,21.36030273)
\curveto(379.70966706,21.32029535)(379.74966702,21.28029539)(379.78967529,21.24030273)
\curveto(379.83966693,21.21029546)(379.88466689,21.17529549)(379.92467529,21.13530273)
\curveto(379.94466683,21.11529555)(379.95966681,21.09529557)(379.96967529,21.07530273)
\curveto(379.97966679,21.0652956)(379.99466678,21.05029562)(380.01467529,21.03030273)
\curveto(380.05466672,20.98029569)(380.09466668,20.92529574)(380.13467529,20.86530273)
\curveto(380.18466659,20.80529586)(380.22966654,20.74529592)(380.26967529,20.68530273)
\curveto(380.35966641,20.51529615)(380.44966632,20.33029634)(380.53967529,20.13030273)
\curveto(380.58966618,20.00029667)(380.62466615,19.85529681)(380.64467529,19.69530273)
\curveto(380.6746661,19.53529713)(380.69466608,19.37529729)(380.70467529,19.21530273)
\curveto(380.71466606,19.13529753)(380.71466606,19.05029762)(380.70467529,18.96030273)
\curveto(380.70466607,18.8702978)(380.70966606,18.78529788)(380.71967529,18.70530273)
\lineto(380.68967529,18.58530273)
\lineto(380.68967529,18.49530273)
\curveto(380.69966607,18.44529822)(380.69466608,18.39029828)(380.67467529,18.33030273)
\curveto(380.65466612,18.2702984)(380.64966612,18.21529845)(380.65967529,18.16530273)
\lineto(380.62967529,18.01530273)
\moveto(379.21967529,17.61030273)
\curveto(379.24966752,17.66029901)(379.26466751,17.72029895)(379.26467529,17.79030273)
\curveto(379.2746675,17.8702988)(379.28466749,17.94029873)(379.29467529,18.00030273)
\curveto(379.33466744,18.1702985)(379.35966741,18.33029834)(379.36967529,18.48030273)
\curveto(379.38966738,18.63029804)(379.38466739,18.77529789)(379.35467529,18.91530273)
\curveto(379.35466742,18.97529769)(379.34966742,19.03529763)(379.33967529,19.09530273)
\curveto(379.33966743,19.1652975)(379.32966744,19.23029744)(379.30967529,19.29030273)
\curveto(379.25966751,19.56029711)(379.13966763,19.82029685)(378.94967529,20.07030273)
\curveto(378.769668,20.32029635)(378.58466819,20.49029618)(378.39467529,20.58030273)
\curveto(378.31466846,20.62029605)(378.23466854,20.65029602)(378.15467529,20.67030273)
\curveto(378.0746687,20.69029598)(377.99466878,20.71529595)(377.91467529,20.74530273)
\curveto(377.82466895,20.7652959)(377.71966905,20.77529589)(377.59967529,20.77530273)
\lineto(377.26967529,20.77530273)
\curveto(377.24966952,20.75529591)(377.20966956,20.74529592)(377.14967529,20.74530273)
\curveto(377.09966967,20.75529591)(377.05466972,20.75529591)(377.01467529,20.74530273)
\curveto(376.91466986,20.72529594)(376.81966995,20.70529596)(376.72967529,20.68530273)
\curveto(376.64967012,20.665296)(376.56467021,20.63529603)(376.47467529,20.59530273)
\curveto(376.12467065,20.45529621)(375.81967095,20.25029642)(375.55967529,19.98030273)
\curveto(375.29967147,19.72029695)(375.07967169,19.41529725)(374.89967529,19.06530273)
\curveto(374.83967193,18.95529771)(374.78967198,18.84529782)(374.74967529,18.73530273)
\curveto(374.71967205,18.62529804)(374.68467209,18.51529815)(374.64467529,18.40530273)
\curveto(374.62467215,18.3652983)(374.60967216,18.32529834)(374.59967529,18.28530273)
\curveto(374.58967218,18.25529841)(374.57967219,18.22029845)(374.56967529,18.18030273)
\lineto(374.53967529,18.06030273)
\curveto(374.51967225,18.01029866)(374.49967227,17.93529873)(374.47967529,17.83530273)
\curveto(374.45967231,17.74529892)(374.44967232,17.67529899)(374.44967529,17.62530273)
\lineto(374.43467529,17.50530273)
\curveto(374.43467234,17.4652992)(374.42967234,17.42529924)(374.41967529,17.38530273)
\curveto(374.40967236,17.34529932)(374.40967236,17.31029936)(374.41967529,17.28030273)
\curveto(374.41967235,17.25029942)(374.41467236,17.22029945)(374.40467529,17.19030273)
\lineto(374.40467529,17.08530273)
\lineto(374.40467529,16.84530273)
\curveto(374.40467237,16.7652999)(374.40967236,16.68529998)(374.41967529,16.60530273)
\curveto(374.45967231,16.2653004)(374.54967222,15.9653007)(374.68967529,15.70530273)
\curveto(374.83967193,15.45530121)(375.05967171,15.26030141)(375.34967529,15.12030273)
\curveto(375.51967125,15.04030163)(375.69967107,14.98030169)(375.88967529,14.94030273)
\curveto(375.92967084,14.92030175)(375.9696708,14.91030176)(376.00967529,14.91030273)
\curveto(376.04967072,14.92030175)(376.08967068,14.92030175)(376.12967529,14.91030273)
\lineto(376.24967529,14.91030273)
\curveto(376.31967045,14.89030178)(376.38967038,14.89030178)(376.45967529,14.91030273)
\lineto(376.57967529,14.91030273)
\curveto(376.68967008,14.93030174)(376.79466998,14.94530172)(376.89467529,14.95530273)
\curveto(377.00466977,14.9653017)(377.11466966,14.99030168)(377.22467529,15.03030273)
\curveto(377.55466922,15.16030151)(377.83966893,15.33030134)(378.07967529,15.54030273)
\curveto(378.31966845,15.76030091)(378.53466824,16.02530064)(378.72467529,16.33530273)
\curveto(378.80466797,16.47530019)(378.8696679,16.61530005)(378.91967529,16.75530273)
\curveto(378.97966779,16.90529976)(379.04466773,17.06029961)(379.11467529,17.22030273)
\curveto(379.13466764,17.2702994)(379.14466763,17.31529935)(379.14467529,17.35530273)
\curveto(379.15466762,17.39529927)(379.1696676,17.44029923)(379.18967529,17.49030273)
\lineto(379.21967529,17.61030273)
}
}
{
\newrgbcolor{curcolor}{0 0 0}
\pscustom[linestyle=none,fillstyle=solid,fillcolor=curcolor]
{
\newpath
\moveto(388.30592529,14.50530273)
\curveto(388.29591738,14.34530232)(388.25091743,14.21030246)(388.17092529,14.10030273)
\curveto(388.09091759,14.00030267)(387.99591768,13.92530274)(387.88592529,13.87530273)
\curveto(387.83591784,13.85530281)(387.7809179,13.84530282)(387.72092529,13.84530273)
\curveto(387.67091801,13.84530282)(387.61091807,13.83530283)(387.54092529,13.81530273)
\curveto(387.31091837,13.7653029)(387.09591858,13.78030289)(386.89592529,13.86030273)
\curveto(386.69591898,13.93030274)(386.57091911,14.02030265)(386.52092529,14.13030273)
\curveto(386.4809192,14.20030247)(386.45091923,14.28030239)(386.43092529,14.37030273)
\curveto(386.41091927,14.4703022)(386.3759193,14.55030212)(386.32592529,14.61030273)
\lineto(386.26592529,14.67030273)
\curveto(386.24591943,14.69030198)(386.21591946,14.69530197)(386.17592529,14.68530273)
\curveto(386.05591962,14.65530201)(385.94091974,14.60030207)(385.83092529,14.52030273)
\curveto(385.72091996,14.44030223)(385.61592006,14.3703023)(385.51592529,14.31030273)
\curveto(385.36592031,14.23030244)(385.21092047,14.15530251)(385.05092529,14.08530273)
\curveto(384.89092079,14.02530264)(384.72092096,13.9703027)(384.54092529,13.92030273)
\curveto(384.43092125,13.89030278)(384.31592136,13.8703028)(384.19592529,13.86030273)
\curveto(384.08592159,13.85030282)(383.97092171,13.83530283)(383.85092529,13.81530273)
\curveto(383.80092188,13.80530286)(383.75592192,13.80030287)(383.71592529,13.80030273)
\lineto(383.61092529,13.80030273)
\curveto(383.50092218,13.78030289)(383.39592228,13.78030289)(383.29592529,13.80030273)
\lineto(383.16092529,13.80030273)
\curveto(383.11092257,13.81030286)(383.06092262,13.81530285)(383.01092529,13.81530273)
\curveto(382.96092272,13.81530285)(382.92092276,13.82530284)(382.89092529,13.84530273)
\curveto(382.85092283,13.85530281)(382.81592286,13.86030281)(382.78592529,13.86030273)
\curveto(382.76592291,13.85030282)(382.74092294,13.85030282)(382.71092529,13.86030273)
\lineto(382.47092529,13.92030273)
\curveto(382.40092328,13.93030274)(382.33592334,13.95030272)(382.27592529,13.98030273)
\curveto(381.99592368,14.11030256)(381.7809239,14.25530241)(381.63092529,14.41530273)
\curveto(381.4809242,14.58530208)(381.3759243,14.82030185)(381.31592529,15.12030273)
\curveto(381.26592441,15.34030133)(381.27092441,15.60530106)(381.33092529,15.91530273)
\lineto(381.40592529,16.23030273)
\curveto(381.42592425,16.28030039)(381.44092424,16.33030034)(381.45092529,16.38030273)
\lineto(381.51092529,16.56030273)
\lineto(381.69092529,16.89030273)
\curveto(381.76092392,17.00029967)(381.83092385,17.10029957)(381.90092529,17.19030273)
\curveto(382.14092354,17.48029919)(382.43092325,17.69529897)(382.77092529,17.83530273)
\curveto(383.11092257,17.97529869)(383.4759222,18.10029857)(383.86592529,18.21030273)
\curveto(384.01592166,18.25029842)(384.16592151,18.28029839)(384.31592529,18.30030273)
\curveto(384.4759212,18.32029835)(384.63092105,18.34529832)(384.78092529,18.37530273)
\curveto(384.86092082,18.39529827)(384.93092075,18.40529826)(384.99092529,18.40530273)
\curveto(385.06092062,18.40529826)(385.13592054,18.41529825)(385.21592529,18.43530273)
\curveto(385.28592039,18.45529821)(385.35592032,18.4652982)(385.42592529,18.46530273)
\curveto(385.50592017,18.47529819)(385.58592009,18.49029818)(385.66592529,18.51030273)
\curveto(385.92591975,18.5702981)(386.17091951,18.62029805)(386.40092529,18.66030273)
\curveto(386.63091905,18.71029796)(386.83091885,18.82529784)(387.00092529,19.00530273)
\curveto(387.07091861,19.08529758)(387.13591854,19.18529748)(387.19592529,19.30530273)
\curveto(387.26591841,19.43529723)(387.29591838,19.57529709)(387.28592529,19.72530273)
\curveto(387.2759184,19.9652967)(387.22591845,20.15529651)(387.13592529,20.29530273)
\curveto(387.05591862,20.43529623)(386.91591876,20.54529612)(386.71592529,20.62530273)
\curveto(386.60591907,20.67529599)(386.47091921,20.71029596)(386.31092529,20.73030273)
\curveto(386.15091953,20.75029592)(385.9809197,20.76029591)(385.80092529,20.76030273)
\curveto(385.62092006,20.76029591)(385.44092024,20.75029592)(385.26092529,20.73030273)
\curveto(385.09092059,20.71029596)(384.94092074,20.68029599)(384.81092529,20.64030273)
\curveto(384.63092105,20.58029609)(384.45092123,20.49529617)(384.27092529,20.38530273)
\curveto(384.1809215,20.32529634)(384.09092159,20.24529642)(384.00092529,20.14530273)
\curveto(383.92092176,20.05529661)(383.84592183,19.95529671)(383.77592529,19.84530273)
\curveto(383.72592195,19.7652969)(383.680922,19.68029699)(383.64092529,19.59030273)
\curveto(383.60092208,19.50029717)(383.54092214,19.43029724)(383.46092529,19.38030273)
\curveto(383.41092227,19.35029732)(383.33592234,19.32529734)(383.23592529,19.30530273)
\curveto(383.13592254,19.29529737)(383.03592264,19.29029738)(382.93592529,19.29030273)
\curveto(382.83592284,19.29029738)(382.74092294,19.29529737)(382.65092529,19.30530273)
\curveto(382.56092312,19.32529734)(382.50092318,19.35029732)(382.47092529,19.38030273)
\curveto(382.43092325,19.41029726)(382.40592327,19.46029721)(382.39592529,19.53030273)
\curveto(382.39592328,19.60029707)(382.41592326,19.67529699)(382.45592529,19.75530273)
\curveto(382.50592317,19.88529678)(382.56092312,20.00529666)(382.62092529,20.11530273)
\curveto(382.680923,20.23529643)(382.74592293,20.35029632)(382.81592529,20.46030273)
\curveto(383.0759226,20.81029586)(383.37092231,21.08029559)(383.70092529,21.27030273)
\curveto(384.03092165,21.4702952)(384.42092126,21.63029504)(384.87092529,21.75030273)
\curveto(384.9809207,21.7702949)(385.08592059,21.78529488)(385.18592529,21.79530273)
\curveto(385.29592038,21.80529486)(385.40592027,21.82029485)(385.51592529,21.84030273)
\curveto(385.56592011,21.85029482)(385.63092005,21.85029482)(385.71092529,21.84030273)
\curveto(385.80091988,21.84029483)(385.86091982,21.85029482)(385.89092529,21.87030273)
\curveto(386.59091909,21.88029479)(387.1809185,21.80029487)(387.66092529,21.63030273)
\curveto(388.15091753,21.46029521)(388.45591722,21.13529553)(388.57592529,20.65530273)
\curveto(388.62591705,20.45529621)(388.63091705,20.22029645)(388.59092529,19.95030273)
\curveto(388.55091713,19.69029698)(388.50091718,19.41529725)(388.44092529,19.12530273)
\lineto(387.78092529,15.81030273)
\curveto(387.75091793,15.670301)(387.72591795,15.53530113)(387.70592529,15.40530273)
\curveto(387.69591798,15.27530139)(387.70591797,15.1703015)(387.73592529,15.09030273)
\curveto(387.7759179,15.02030165)(387.83091785,14.9703017)(387.90092529,14.94030273)
\curveto(387.99091769,14.90030177)(388.07091761,14.8703018)(388.14092529,14.85030273)
\curveto(388.22091746,14.84030183)(388.27091741,14.79530187)(388.29092529,14.71530273)
\curveto(388.31091737,14.68530198)(388.31591736,14.65530201)(388.30592529,14.62530273)
\lineto(388.30592529,14.50530273)
\moveto(386.49092529,16.17030273)
\curveto(386.5809191,16.31030036)(386.64591903,16.4703002)(386.68592529,16.65030273)
\curveto(386.72591895,16.84029983)(386.76591891,17.03529963)(386.80592529,17.23530273)
\curveto(386.82591885,17.34529932)(386.84091884,17.44529922)(386.85092529,17.53530273)
\curveto(386.86091882,17.62529904)(386.83591884,17.69529897)(386.77592529,17.74530273)
\curveto(386.74591893,17.7652989)(386.675919,17.77529889)(386.56592529,17.77530273)
\curveto(386.54591913,17.75529891)(386.51091917,17.74529892)(386.46092529,17.74530273)
\curveto(386.41091927,17.74529892)(386.36091932,17.73529893)(386.31092529,17.71530273)
\curveto(386.23091945,17.69529897)(386.13591954,17.67529899)(386.02592529,17.65530273)
\lineto(385.72592529,17.59530273)
\curveto(385.69591998,17.59529907)(385.66092002,17.59029908)(385.62092529,17.58030273)
\lineto(385.51592529,17.58030273)
\curveto(385.35592032,17.54029913)(385.18592049,17.51529915)(385.00592529,17.50530273)
\curveto(384.83592084,17.50529916)(384.67092101,17.48529918)(384.51092529,17.44530273)
\curveto(384.42092126,17.42529924)(384.34092134,17.40529926)(384.27092529,17.38530273)
\curveto(384.21092147,17.37529929)(384.13592154,17.36029931)(384.04592529,17.34030273)
\curveto(383.8759218,17.29029938)(383.71092197,17.22529944)(383.55092529,17.14530273)
\curveto(383.40092228,17.07529959)(383.26592241,16.98529968)(383.14592529,16.87530273)
\curveto(383.02592265,16.7652999)(382.92592275,16.63030004)(382.84592529,16.47030273)
\curveto(382.76592291,16.32030035)(382.70592297,16.13530053)(382.66592529,15.91530273)
\curveto(382.64592303,15.81530085)(382.64592303,15.72030095)(382.66592529,15.63030273)
\curveto(382.68592299,15.55030112)(382.71592296,15.47530119)(382.75592529,15.40530273)
\curveto(382.80592287,15.29530137)(382.88592279,15.20030147)(382.99592529,15.12030273)
\curveto(383.11592256,15.05030162)(383.24592243,14.99030168)(383.38592529,14.94030273)
\curveto(383.43592224,14.93030174)(383.48592219,14.92530174)(383.53592529,14.92530273)
\curveto(383.58592209,14.92530174)(383.63592204,14.92030175)(383.68592529,14.91030273)
\curveto(383.75592192,14.89030178)(383.84092184,14.87530179)(383.94092529,14.86530273)
\curveto(384.04092164,14.8653018)(384.13092155,14.87530179)(384.21092529,14.89530273)
\curveto(384.27092141,14.91530175)(384.33092135,14.92030175)(384.39092529,14.91030273)
\curveto(384.45092123,14.91030176)(384.51092117,14.92030175)(384.57092529,14.94030273)
\curveto(384.66092102,14.96030171)(384.74092094,14.97530169)(384.81092529,14.98530273)
\curveto(384.89092079,14.99530167)(384.97092071,15.01530165)(385.05092529,15.04530273)
\curveto(385.36092032,15.1653015)(385.63592004,15.31030136)(385.87592529,15.48030273)
\curveto(386.11591956,15.65030102)(386.32091936,15.88030079)(386.49092529,16.17030273)
}
}
{
\newrgbcolor{curcolor}{0 0 0}
\pscustom[linestyle=none,fillstyle=solid,fillcolor=curcolor]
{
\newpath
\moveto(394.08256592,21.85530273)
\curveto(394.82255954,21.8652948)(395.41755894,21.75529491)(395.86756592,21.52530273)
\curveto(396.31755804,21.30529536)(396.64255772,20.9702957)(396.84256592,20.52030273)
\curveto(396.93255743,20.32029635)(396.99255737,20.07529659)(397.02256592,19.78530273)
\curveto(397.03255733,19.73529693)(397.03255733,19.670297)(397.02256592,19.59030273)
\curveto(397.02255734,19.51029716)(397.00755735,19.44029723)(396.97756592,19.38030273)
\curveto(396.93755742,19.33029734)(396.87755748,19.28529738)(396.79756592,19.24530273)
\curveto(396.7575576,19.22529744)(396.72255764,19.21529745)(396.69256592,19.21530273)
\curveto(396.67255769,19.22529744)(396.63755772,19.22529744)(396.58756592,19.21530273)
\curveto(396.54755781,19.20529746)(396.50755785,19.20029747)(396.46756592,19.20030273)
\curveto(396.42755793,19.21029746)(396.38755797,19.21529745)(396.34756592,19.21530273)
\lineto(396.03256592,19.21530273)
\curveto(395.94255842,19.22529744)(395.86755849,19.25529741)(395.80756592,19.30530273)
\curveto(395.73755862,19.3652973)(395.69755866,19.45029722)(395.68756592,19.56030273)
\curveto(395.67755868,19.670297)(395.6575587,19.7652969)(395.62756592,19.84530273)
\curveto(395.52755883,20.10529656)(395.37255899,20.31029636)(395.16256592,20.46030273)
\curveto(395.09255927,20.51029616)(395.01255935,20.55029612)(394.92256592,20.58030273)
\curveto(394.84255952,20.62029605)(394.7575596,20.65529601)(394.66756592,20.68530273)
\curveto(394.53755982,20.72529594)(394.35756,20.74529592)(394.12756592,20.74530273)
\curveto(393.89756046,20.75529591)(393.70256066,20.73529593)(393.54256592,20.68530273)
\curveto(393.47256089,20.665296)(393.40256096,20.65029602)(393.33256592,20.64030273)
\curveto(393.27256109,20.63029604)(393.20756115,20.61529605)(393.13756592,20.59530273)
\curveto(392.8575615,20.48529618)(392.59756176,20.33529633)(392.35756592,20.14530273)
\curveto(392.11756224,19.95529671)(391.91756244,19.73029694)(391.75756592,19.47030273)
\curveto(391.69756266,19.38029729)(391.64256272,19.28529738)(391.59256592,19.18530273)
\curveto(391.54256282,19.09529757)(391.49256287,18.99529767)(391.44256592,18.88530273)
\lineto(391.27756592,18.48030273)
\curveto(391.2575631,18.43029824)(391.24256312,18.37529829)(391.23256592,18.31530273)
\curveto(391.22256314,18.25529841)(391.20256316,18.20029847)(391.17256592,18.15030273)
\lineto(391.15756592,18.03030273)
\curveto(391.13756322,17.99029868)(391.11256325,17.92529874)(391.08256592,17.83530273)
\curveto(391.0625633,17.74529892)(391.0575633,17.68029899)(391.06756592,17.64030273)
\curveto(391.06756329,17.59029908)(391.0575633,17.54029913)(391.03756592,17.49030273)
\curveto(391.01756334,17.44029923)(391.00756335,17.39029928)(391.00756592,17.34030273)
\curveto(391.01756334,17.30029937)(391.01256335,17.23029944)(390.99256592,17.13030273)
\curveto(390.99256337,17.05029962)(390.98756337,16.9652997)(390.97756592,16.87530273)
\curveto(390.97756338,16.78529988)(390.98256338,16.70029997)(390.99256592,16.62030273)
\curveto(391.03256333,16.30030037)(391.10256326,16.02030065)(391.20256592,15.78030273)
\curveto(391.30256306,15.55030112)(391.46756289,15.35030132)(391.69756592,15.18030273)
\curveto(391.77756258,15.13030154)(391.8575625,15.08530158)(391.93756592,15.04530273)
\curveto(392.02756233,15.00530166)(392.12256224,14.9653017)(392.22256592,14.92530273)
\curveto(392.27256209,14.91530175)(392.31256205,14.91030176)(392.34256592,14.91030273)
\curveto(392.37256199,14.91030176)(392.41256195,14.90530176)(392.46256592,14.89530273)
\curveto(392.49256187,14.88530178)(392.54256182,14.88030179)(392.61256592,14.88030273)
\lineto(392.77756592,14.88030273)
\curveto(392.76756159,14.8703018)(392.78256158,14.8653018)(392.82256592,14.86530273)
\curveto(392.85256151,14.87530179)(392.87756148,14.87530179)(392.89756592,14.86530273)
\curveto(392.92756143,14.8653018)(392.9625614,14.8703018)(393.00256592,14.88030273)
\curveto(393.07256129,14.90030177)(393.13756122,14.90530176)(393.19756592,14.89530273)
\curveto(393.26756109,14.89530177)(393.33756102,14.90530176)(393.40756592,14.92530273)
\curveto(393.68756067,15.00530166)(393.93256043,15.10530156)(394.14256592,15.22530273)
\curveto(394.36256,15.35530131)(394.5575598,15.52030115)(394.72756592,15.72030273)
\curveto(394.78755957,15.80030087)(394.84755951,15.88530078)(394.90756592,15.97530273)
\lineto(395.08756592,16.24530273)
\curveto(395.11755924,16.32530034)(395.15255921,16.40030027)(395.19256592,16.47030273)
\curveto(395.23255913,16.55030012)(395.29255907,16.61530005)(395.37256592,16.66530273)
\curveto(395.41255895,16.69529997)(395.47755888,16.71529995)(395.56756592,16.72530273)
\curveto(395.66755869,16.74529992)(395.76755859,16.75529991)(395.86756592,16.75530273)
\curveto(395.97755838,16.7652999)(396.07755828,16.7652999)(396.16756592,16.75530273)
\curveto(396.2575581,16.74529992)(396.32255804,16.72529994)(396.36256592,16.69530273)
\curveto(396.41255795,16.65530001)(396.43755792,16.59530007)(396.43756592,16.51530273)
\curveto(396.43755792,16.43530023)(396.41755794,16.35030032)(396.37756592,16.26030273)
\curveto(396.29755806,16.11030056)(396.22255814,15.9653007)(396.15256592,15.82530273)
\curveto(396.08255828,15.69530097)(395.99755836,15.5653011)(395.89756592,15.43530273)
\curveto(395.68755867,15.13530153)(395.44755891,14.8703018)(395.17756592,14.64030273)
\curveto(394.90755945,14.41030226)(394.59755976,14.22530244)(394.24756592,14.08530273)
\curveto(394.1575602,14.04530262)(394.0625603,14.01030266)(393.96256592,13.98030273)
\curveto(393.87256049,13.96030271)(393.77756058,13.93530273)(393.67756592,13.90530273)
\curveto(393.5575608,13.8653028)(393.44256092,13.84530282)(393.33256592,13.84530273)
\curveto(393.22256114,13.83530283)(393.10756125,13.82030285)(392.98756592,13.80030273)
\curveto(392.94756141,13.78030289)(392.90756145,13.77530289)(392.86756592,13.78530273)
\curveto(392.82756153,13.79530287)(392.78756157,13.79530287)(392.74756592,13.78530273)
\lineto(392.61256592,13.78530273)
\lineto(392.37256592,13.78530273)
\curveto(392.30256206,13.77530289)(392.23756212,13.78030289)(392.17756592,13.80030273)
\lineto(392.10256592,13.80030273)
\lineto(391.75756592,13.84530273)
\curveto(391.63756272,13.88530278)(391.51756284,13.92030275)(391.39756592,13.95030273)
\curveto(391.28756307,13.98030269)(391.18256318,14.02030265)(391.08256592,14.07030273)
\curveto(390.75256361,14.23030244)(390.49256387,14.42030225)(390.30256592,14.64030273)
\curveto(390.11256425,14.86030181)(389.94756441,15.13030154)(389.80756592,15.45030273)
\curveto(389.77756458,15.53030114)(389.75256461,15.62030105)(389.73256592,15.72030273)
\lineto(389.67256592,16.02030273)
\curveto(389.64256472,16.13030054)(389.62756473,16.24530042)(389.62756592,16.36530273)
\curveto(389.63756472,16.48530018)(389.63756472,16.60530006)(389.62756592,16.72530273)
\curveto(389.62756473,16.7652999)(389.63256473,16.80529986)(389.64256592,16.84530273)
\curveto(389.65256471,16.88529978)(389.65256471,16.92529974)(389.64256592,16.96530273)
\curveto(389.64256472,17.02529964)(389.64756471,17.09029958)(389.65756592,17.16030273)
\curveto(389.67756468,17.23029944)(389.68756467,17.29529937)(389.68756592,17.35530273)
\lineto(389.71756592,17.50530273)
\curveto(389.71756464,17.55529911)(389.72256464,17.62529904)(389.73256592,17.71530273)
\curveto(389.75256461,17.80529886)(389.77256459,17.87529879)(389.79256592,17.92530273)
\curveto(389.81256455,17.97529869)(389.82256454,18.02029865)(389.82256592,18.06030273)
\curveto(389.83256453,18.10029857)(389.84756451,18.14029853)(389.86756592,18.18030273)
\curveto(389.89756446,18.25029842)(389.91756444,18.32029835)(389.92756592,18.39030273)
\curveto(389.93756442,18.46029821)(389.9575644,18.52529814)(389.98756592,18.58530273)
\curveto(390.0575643,18.75529791)(390.12256424,18.92529774)(390.18256592,19.09530273)
\curveto(390.25256411,19.2652974)(390.33256403,19.42529724)(390.42256592,19.57530273)
\curveto(390.74256362,20.09529657)(391.08256328,20.51529615)(391.44256592,20.83530273)
\curveto(391.80256256,21.15529551)(392.26756209,21.42029525)(392.83756592,21.63030273)
\curveto(392.9575614,21.68029499)(393.08256128,21.71529495)(393.21256592,21.73530273)
\curveto(393.34256102,21.75529491)(393.48256088,21.78029489)(393.63256592,21.81030273)
\curveto(393.70256066,21.82029485)(393.77256059,21.82529484)(393.84256592,21.82530273)
\curveto(393.91256045,21.83529483)(393.99256037,21.84529482)(394.08256592,21.85530273)
}
}
{
\newrgbcolor{curcolor}{0 0 0}
\pscustom[linestyle=none,fillstyle=solid,fillcolor=curcolor]
{
\newpath
\moveto(399.54420654,23.17530273)
\curveto(399.47420357,23.23529343)(399.45420359,23.34029333)(399.48420654,23.49030273)
\curveto(399.51420353,23.65029302)(399.5442035,23.80529286)(399.57420654,23.95530273)
\curveto(399.58420346,24.03529263)(399.59920344,24.12029255)(399.61920654,24.21030273)
\curveto(399.6392034,24.30029237)(399.66920337,24.37529229)(399.70920654,24.43530273)
\curveto(399.76920327,24.51529215)(399.85920318,24.57529209)(399.97920654,24.61530273)
\curveto(400.00920303,24.62529204)(400.03420301,24.62529204)(400.05420654,24.61530273)
\curveto(400.07420297,24.61529205)(400.09920294,24.62029205)(400.12920654,24.63030273)
\curveto(400.29920274,24.63029204)(400.45420259,24.62529204)(400.59420654,24.61530273)
\curveto(400.7442023,24.60529206)(400.83420221,24.54529212)(400.86420654,24.43530273)
\curveto(400.88420216,24.37529229)(400.88420216,24.30029237)(400.86420654,24.21030273)
\curveto(400.8442022,24.13029254)(400.82920221,24.04529262)(400.81920654,23.95530273)
\curveto(400.77920226,23.77529289)(400.7392023,23.60529306)(400.69920654,23.44530273)
\curveto(400.66920237,23.28529338)(400.58420246,23.18029349)(400.44420654,23.13030273)
\curveto(400.38420266,23.11029356)(400.32420272,23.10029357)(400.26420654,23.10030273)
\lineto(400.09920654,23.10030273)
\lineto(399.78420654,23.10030273)
\curveto(399.68420336,23.10029357)(399.60420344,23.12529354)(399.54420654,23.17530273)
\moveto(398.95920654,14.67030273)
\curveto(398.9392041,14.5703021)(398.91920412,14.4653022)(398.89920654,14.35530273)
\curveto(398.88920415,14.25530241)(398.84920419,14.17530249)(398.77920654,14.11530273)
\curveto(398.7392043,14.05530261)(398.68920435,14.01530265)(398.62920654,13.99530273)
\curveto(398.56920447,13.98530268)(398.49420455,13.9703027)(398.40420654,13.95030273)
\lineto(398.17920654,13.95030273)
\curveto(398.04920499,13.95030272)(397.9392051,13.95530271)(397.84920654,13.96530273)
\curveto(397.75920528,13.98530268)(397.69420535,14.03530263)(397.65420654,14.11530273)
\curveto(397.63420541,14.17530249)(397.62920541,14.25030242)(397.63920654,14.34030273)
\curveto(397.65920538,14.44030223)(397.67920536,14.53530213)(397.69920654,14.62530273)
\lineto(398.97420654,20.97030273)
\curveto(398.99420405,21.08029559)(399.01420403,21.18529548)(399.03420654,21.28530273)
\curveto(399.05420399,21.39529527)(399.09420395,21.48029519)(399.15420654,21.54030273)
\curveto(399.19420385,21.59029508)(399.2392038,21.62029505)(399.28920654,21.63030273)
\curveto(399.34920369,21.64029503)(399.40920363,21.65529501)(399.46920654,21.67530273)
\curveto(399.48920355,21.67529499)(399.50920353,21.670295)(399.52920654,21.66030273)
\curveto(399.55920348,21.66029501)(399.58420346,21.665295)(399.60420654,21.67530273)
\curveto(399.73420331,21.67529499)(399.86420318,21.670295)(399.99420654,21.66030273)
\curveto(400.13420291,21.66029501)(400.21920282,21.62029505)(400.24920654,21.54030273)
\curveto(400.28920275,21.48029519)(400.29920274,21.40029527)(400.27920654,21.30030273)
\curveto(400.25920278,21.21029546)(400.2392028,21.11529555)(400.21920654,21.01530273)
\lineto(398.95920654,14.67030273)
}
}
{
\newrgbcolor{curcolor}{0 0 0}
\pscustom[linestyle=none,fillstyle=solid,fillcolor=curcolor]
{
\newpath
\moveto(408.71905029,18.15030273)
\curveto(408.7290414,18.09029858)(408.71904141,17.99529867)(408.68905029,17.86530273)
\curveto(408.66904146,17.74529892)(408.64904148,17.66029901)(408.62905029,17.61030273)
\lineto(408.59905029,17.46030273)
\curveto(408.56904156,17.38029929)(408.54404159,17.30529936)(408.52405029,17.23530273)
\curveto(408.51404162,17.17529949)(408.49404164,17.10529956)(408.46405029,17.02530273)
\curveto(408.4340417,16.9652997)(408.40904172,16.90529976)(408.38905029,16.84530273)
\curveto(408.37904175,16.78529988)(408.35404178,16.72529994)(408.31405029,16.66530273)
\lineto(408.13405029,16.27530273)
\curveto(408.08404205,16.14530052)(408.01904211,16.02530064)(407.93905029,15.91530273)
\curveto(407.63904249,15.43530123)(407.27904285,15.03030164)(406.85905029,14.70030273)
\curveto(406.44904368,14.38030229)(405.96904416,14.13530253)(405.41905029,13.96530273)
\curveto(405.30904482,13.92530274)(405.18904494,13.89530277)(405.05905029,13.87530273)
\curveto(404.9290452,13.85530281)(404.79404534,13.83530283)(404.65405029,13.81530273)
\curveto(404.59404554,13.80530286)(404.5290456,13.80030287)(404.45905029,13.80030273)
\curveto(404.39904573,13.79030288)(404.33904579,13.78530288)(404.27905029,13.78530273)
\curveto(404.23904589,13.77530289)(404.17904595,13.7703029)(404.09905029,13.77030273)
\curveto(404.0290461,13.7703029)(403.97904615,13.77530289)(403.94905029,13.78530273)
\curveto(403.90904622,13.79530287)(403.86904626,13.80030287)(403.82905029,13.80030273)
\curveto(403.78904634,13.79030288)(403.75404638,13.79030288)(403.72405029,13.80030273)
\lineto(403.63405029,13.80030273)
\lineto(403.28905029,13.84530273)
\lineto(402.89905029,13.96530273)
\curveto(402.77904735,14.00530266)(402.66404747,14.05030262)(402.55405029,14.10030273)
\curveto(402.14404799,14.30030237)(401.82404831,14.56030211)(401.59405029,14.88030273)
\curveto(401.37404876,15.20030147)(401.21404892,15.59030108)(401.11405029,16.05030273)
\curveto(401.08404905,16.15030052)(401.06404907,16.25030042)(401.05405029,16.35030273)
\lineto(401.05405029,16.66530273)
\curveto(401.04404909,16.70529996)(401.04404909,16.73529993)(401.05405029,16.75530273)
\curveto(401.06404907,16.78529988)(401.06904906,16.82029985)(401.06905029,16.86030273)
\curveto(401.06904906,16.94029973)(401.07404906,17.02029965)(401.08405029,17.10030273)
\curveto(401.09404904,17.19029948)(401.09904903,17.27529939)(401.09905029,17.35530273)
\curveto(401.10904902,17.40529926)(401.11404902,17.44529922)(401.11405029,17.47530273)
\curveto(401.12404901,17.51529915)(401.129049,17.56029911)(401.12905029,17.61030273)
\curveto(401.129049,17.66029901)(401.13904899,17.74529892)(401.15905029,17.86530273)
\curveto(401.18904894,17.99529867)(401.21904891,18.09029858)(401.24905029,18.15030273)
\curveto(401.28904884,18.22029845)(401.30904882,18.29029838)(401.30905029,18.36030273)
\curveto(401.30904882,18.43029824)(401.3290488,18.50029817)(401.36905029,18.57030273)
\curveto(401.38904874,18.62029805)(401.40404873,18.66029801)(401.41405029,18.69030273)
\curveto(401.42404871,18.73029794)(401.43904869,18.77529789)(401.45905029,18.82530273)
\curveto(401.51904861,18.94529772)(401.56904856,19.0652976)(401.60905029,19.18530273)
\curveto(401.65904847,19.30529736)(401.72404841,19.42029725)(401.80405029,19.53030273)
\curveto(402.02404811,19.90029677)(402.26904786,20.23029644)(402.53905029,20.52030273)
\curveto(402.81904731,20.82029585)(403.134047,21.0702956)(403.48405029,21.27030273)
\curveto(403.61404652,21.35029532)(403.74904638,21.41529525)(403.88905029,21.46530273)
\lineto(404.33905029,21.64530273)
\curveto(404.46904566,21.69529497)(404.60404553,21.72529494)(404.74405029,21.73530273)
\curveto(404.88404525,21.75529491)(405.0290451,21.78529488)(405.17905029,21.82530273)
\lineto(405.37405029,21.82530273)
\lineto(405.58405029,21.85530273)
\curveto(406.47404366,21.8652948)(407.17404296,21.68029499)(407.68405029,21.30030273)
\curveto(408.20404193,20.92029575)(408.5290416,20.42529624)(408.65905029,19.81530273)
\curveto(408.68904144,19.71529695)(408.70904142,19.61529705)(408.71905029,19.51530273)
\curveto(408.7290414,19.41529725)(408.74404139,19.31029736)(408.76405029,19.20030273)
\curveto(408.77404136,19.09029758)(408.77404136,18.9702977)(408.76405029,18.84030273)
\lineto(408.76405029,18.46530273)
\curveto(408.76404137,18.41529825)(408.75404138,18.36029831)(408.73405029,18.30030273)
\curveto(408.72404141,18.25029842)(408.71904141,18.20029847)(408.71905029,18.15030273)
\moveto(407.21905029,17.29530273)
\curveto(407.24904288,17.3652993)(407.26904286,17.44529922)(407.27905029,17.53530273)
\curveto(407.29904283,17.62529904)(407.31404282,17.71029896)(407.32405029,17.79030273)
\curveto(407.40404273,18.18029849)(407.43904269,18.51029816)(407.42905029,18.78030273)
\curveto(407.40904272,18.86029781)(407.39404274,18.94029773)(407.38405029,19.02030273)
\curveto(407.38404275,19.10029757)(407.37904275,19.17529749)(407.36905029,19.24530273)
\curveto(407.21904291,19.89529677)(406.86404327,20.34529632)(406.30405029,20.59530273)
\curveto(406.2340439,20.62529604)(406.15904397,20.64529602)(406.07905029,20.65530273)
\curveto(406.00904412,20.67529599)(405.9340442,20.69529597)(405.85405029,20.71530273)
\curveto(405.78404435,20.73529593)(405.70404443,20.74529592)(405.61405029,20.74530273)
\lineto(405.34405029,20.74530273)
\lineto(405.05905029,20.70030273)
\curveto(404.95904517,20.68029599)(404.86404527,20.65529601)(404.77405029,20.62530273)
\curveto(404.68404545,20.60529606)(404.59404554,20.57529609)(404.50405029,20.53530273)
\curveto(404.4340457,20.51529615)(404.36404577,20.48529618)(404.29405029,20.44530273)
\curveto(404.22404591,20.40529626)(404.15904597,20.3652963)(404.09905029,20.32530273)
\curveto(403.8290463,20.15529651)(403.59404654,19.95029672)(403.39405029,19.71030273)
\curveto(403.19404694,19.4702972)(403.00904712,19.19029748)(402.83905029,18.87030273)
\curveto(402.78904734,18.7702979)(402.74904738,18.665298)(402.71905029,18.55530273)
\curveto(402.68904744,18.45529821)(402.64904748,18.35029832)(402.59905029,18.24030273)
\curveto(402.58904754,18.20029847)(402.57404756,18.13529853)(402.55405029,18.04530273)
\curveto(402.5340476,18.01529865)(402.52404761,17.98029869)(402.52405029,17.94030273)
\curveto(402.52404761,17.90029877)(402.51904761,17.85529881)(402.50905029,17.80530273)
\lineto(402.44905029,17.50530273)
\curveto(402.4290477,17.40529926)(402.41904771,17.31529935)(402.41905029,17.23530273)
\lineto(402.41905029,17.05530273)
\curveto(402.41904771,16.95529971)(402.41404772,16.85529981)(402.40405029,16.75530273)
\curveto(402.40404773,16.6653)(402.41404772,16.58030009)(402.43405029,16.50030273)
\curveto(402.48404765,16.26030041)(402.55404758,16.03530063)(402.64405029,15.82530273)
\curveto(402.74404739,15.61530105)(402.87904725,15.44030123)(403.04905029,15.30030273)
\curveto(403.09904703,15.2703014)(403.13904699,15.24530142)(403.16905029,15.22530273)
\curveto(403.20904692,15.20530146)(403.24904688,15.18030149)(403.28905029,15.15030273)
\curveto(403.35904677,15.10030157)(403.43904669,15.05530161)(403.52905029,15.01530273)
\curveto(403.61904651,14.98530168)(403.71404642,14.95530171)(403.81405029,14.92530273)
\curveto(403.86404627,14.90530176)(403.90904622,14.89530177)(403.94905029,14.89530273)
\curveto(403.99904613,14.90530176)(404.04904608,14.90530176)(404.09905029,14.89530273)
\curveto(404.129046,14.88530178)(404.18904594,14.87530179)(404.27905029,14.86530273)
\curveto(404.36904576,14.85530181)(404.44404569,14.86030181)(404.50405029,14.88030273)
\curveto(404.54404559,14.89030178)(404.58404555,14.89030178)(404.62405029,14.88030273)
\curveto(404.66404547,14.88030179)(404.70404543,14.89030178)(404.74405029,14.91030273)
\curveto(404.82404531,14.93030174)(404.90404523,14.94530172)(404.98405029,14.95530273)
\curveto(405.07404506,14.97530169)(405.15904497,15.00030167)(405.23905029,15.03030273)
\curveto(405.59904453,15.1703015)(405.90904422,15.3653013)(406.16905029,15.61530273)
\curveto(406.4290437,15.8653008)(406.66404347,16.16030051)(406.87405029,16.50030273)
\curveto(406.95404318,16.62030005)(407.01404312,16.74529992)(407.05405029,16.87530273)
\curveto(407.09404304,17.01529965)(407.14904298,17.15529951)(407.21905029,17.29530273)
}
}
{
\newrgbcolor{curcolor}{0 0 0}
\pscustom[linestyle=none,fillstyle=solid,fillcolor=curcolor]
{
\newpath
\moveto(413.38733154,21.85530273)
\curveto(414.10732589,21.8652948)(414.6923253,21.78029489)(415.14233154,21.60030273)
\curveto(415.60232439,21.43029524)(415.92232407,21.12529554)(416.10233154,20.68530273)
\curveto(416.15232384,20.57529609)(416.18232381,20.46029621)(416.19233154,20.34030273)
\curveto(416.21232378,20.23029644)(416.22732377,20.10529656)(416.23733154,19.96530273)
\curveto(416.24732375,19.89529677)(416.23732376,19.82029685)(416.20733154,19.74030273)
\curveto(416.18732381,19.670297)(416.16232383,19.61529705)(416.13233154,19.57530273)
\curveto(416.11232388,19.55529711)(416.08232391,19.53529713)(416.04233154,19.51530273)
\curveto(416.01232398,19.50529716)(415.98732401,19.49029718)(415.96733154,19.47030273)
\curveto(415.90732409,19.45029722)(415.85232414,19.44529722)(415.80233154,19.45530273)
\curveto(415.76232423,19.4652972)(415.71732428,19.4652972)(415.66733154,19.45530273)
\curveto(415.57732442,19.43529723)(415.46732453,19.43029724)(415.33733154,19.44030273)
\curveto(415.21732478,19.46029721)(415.13232486,19.48529718)(415.08233154,19.51530273)
\curveto(415.01232498,19.5652971)(414.97232502,19.63029704)(414.96233154,19.71030273)
\curveto(414.96232503,19.80029687)(414.94232505,19.88529678)(414.90233154,19.96530273)
\curveto(414.85232514,20.12529654)(414.75732524,20.2702964)(414.61733154,20.40030273)
\curveto(414.52732547,20.48029619)(414.41732558,20.54029613)(414.28733154,20.58030273)
\curveto(414.16732583,20.62029605)(414.03732596,20.66029601)(413.89733154,20.70030273)
\curveto(413.85732614,20.72029595)(413.80732619,20.72529594)(413.74733154,20.71530273)
\curveto(413.6973263,20.71529595)(413.65232634,20.72029595)(413.61233154,20.73030273)
\curveto(413.55232644,20.75029592)(413.47732652,20.76029591)(413.38733154,20.76030273)
\curveto(413.2973267,20.76029591)(413.22232677,20.75029592)(413.16233154,20.73030273)
\lineto(413.07233154,20.73030273)
\curveto(413.01232698,20.72029595)(412.95732704,20.71029596)(412.90733154,20.70030273)
\curveto(412.85732714,20.70029597)(412.80732719,20.69529597)(412.75733154,20.68530273)
\curveto(412.48732751,20.62529604)(412.25232774,20.54029613)(412.05233154,20.43030273)
\curveto(411.86232813,20.32029635)(411.71232828,20.13529653)(411.60233154,19.87530273)
\curveto(411.57232842,19.80529686)(411.55732844,19.73529693)(411.55733154,19.66530273)
\curveto(411.55732844,19.59529707)(411.56232843,19.53529713)(411.57233154,19.48530273)
\curveto(411.60232839,19.33529733)(411.65232834,19.22529744)(411.72233154,19.15530273)
\curveto(411.7923282,19.09529757)(411.88732811,19.02529764)(412.00733154,18.94530273)
\curveto(412.14732785,18.84529782)(412.31232768,18.7702979)(412.50233154,18.72030273)
\curveto(412.6923273,18.68029799)(412.88232711,18.63029804)(413.07233154,18.57030273)
\curveto(413.1923268,18.53029814)(413.31232668,18.50029817)(413.43233154,18.48030273)
\curveto(413.56232643,18.46029821)(413.68732631,18.43029824)(413.80733154,18.39030273)
\curveto(414.00732599,18.33029834)(414.20232579,18.2702984)(414.39233154,18.21030273)
\curveto(414.58232541,18.16029851)(414.76732523,18.09529857)(414.94733154,18.01530273)
\curveto(414.997325,17.99529867)(415.04232495,17.97529869)(415.08233154,17.95530273)
\curveto(415.13232486,17.93529873)(415.18232481,17.91029876)(415.23233154,17.88030273)
\curveto(415.40232459,17.76029891)(415.54732445,17.62529904)(415.66733154,17.47530273)
\curveto(415.78732421,17.32529934)(415.87732412,17.13529953)(415.93733154,16.90530273)
\lineto(415.93733154,16.62030273)
\curveto(415.93732406,16.55030012)(415.93232406,16.47530019)(415.92233154,16.39530273)
\curveto(415.91232408,16.32530034)(415.90232409,16.24530042)(415.89233154,16.15530273)
\lineto(415.86233154,16.00530273)
\curveto(415.82232417,15.93530073)(415.7923242,15.8653008)(415.77233154,15.79530273)
\curveto(415.76232423,15.72530094)(415.74232425,15.65530101)(415.71233154,15.58530273)
\curveto(415.66232433,15.47530119)(415.60732439,15.3703013)(415.54733154,15.27030273)
\curveto(415.48732451,15.1703015)(415.42232457,15.08030159)(415.35233154,15.00030273)
\curveto(415.14232485,14.74030193)(414.8973251,14.53030214)(414.61733154,14.37030273)
\curveto(414.33732566,14.22030245)(414.03232596,14.09030258)(413.70233154,13.98030273)
\curveto(413.60232639,13.95030272)(413.50232649,13.93030274)(413.40233154,13.92030273)
\curveto(413.30232669,13.90030277)(413.20732679,13.87530279)(413.11733154,13.84530273)
\curveto(413.00732699,13.82530284)(412.90232709,13.81530285)(412.80233154,13.81530273)
\curveto(412.70232729,13.81530285)(412.60232739,13.80530286)(412.50233154,13.78530273)
\lineto(412.35233154,13.78530273)
\curveto(412.30232769,13.77530289)(412.23232776,13.7703029)(412.14233154,13.77030273)
\curveto(412.05232794,13.7703029)(411.98232801,13.77530289)(411.93233154,13.78530273)
\lineto(411.76733154,13.78530273)
\curveto(411.70732829,13.80530286)(411.64232835,13.81530285)(411.57233154,13.81530273)
\curveto(411.50232849,13.80530286)(411.44732855,13.81030286)(411.40733154,13.83030273)
\curveto(411.35732864,13.84030283)(411.2923287,13.84530282)(411.21233154,13.84530273)
\curveto(411.13232886,13.8653028)(411.05732894,13.88530278)(410.98733154,13.90530273)
\curveto(410.91732908,13.91530275)(410.84232915,13.93530273)(410.76233154,13.96530273)
\curveto(410.47232952,14.0653026)(410.22732977,14.19030248)(410.02733154,14.34030273)
\curveto(409.82733017,14.49030218)(409.66733033,14.68530198)(409.54733154,14.92530273)
\curveto(409.48733051,15.05530161)(409.43733056,15.19030148)(409.39733154,15.33030273)
\curveto(409.36733063,15.4703012)(409.34733065,15.62530104)(409.33733154,15.79530273)
\curveto(409.32733067,15.85530081)(409.33233066,15.92530074)(409.35233154,16.00530273)
\curveto(409.37233062,16.09530057)(409.3973306,16.1653005)(409.42733154,16.21530273)
\curveto(409.46733053,16.25530041)(409.52733047,16.29530037)(409.60733154,16.33530273)
\curveto(409.65733034,16.35530031)(409.72733027,16.3653003)(409.81733154,16.36530273)
\curveto(409.91733008,16.37530029)(410.00732999,16.37530029)(410.08733154,16.36530273)
\curveto(410.17732982,16.35530031)(410.26232973,16.34030033)(410.34233154,16.32030273)
\curveto(410.43232956,16.31030036)(410.48732951,16.29530037)(410.50733154,16.27530273)
\curveto(410.56732943,16.22530044)(410.5973294,16.15030052)(410.59733154,16.05030273)
\curveto(410.60732939,15.96030071)(410.62732937,15.87530079)(410.65733154,15.79530273)
\curveto(410.70732929,15.57530109)(410.80732919,15.40530126)(410.95733154,15.28530273)
\curveto(411.05732894,15.19530147)(411.17732882,15.12530154)(411.31733154,15.07530273)
\curveto(411.45732854,15.02530164)(411.60732839,14.97530169)(411.76733154,14.92530273)
\lineto(412.08233154,14.88030273)
\lineto(412.17233154,14.88030273)
\curveto(412.23232776,14.86030181)(412.31732768,14.85030182)(412.42733154,14.85030273)
\curveto(412.54732745,14.85030182)(412.65232734,14.86030181)(412.74233154,14.88030273)
\curveto(412.81232718,14.88030179)(412.86732713,14.88530178)(412.90733154,14.89530273)
\curveto(412.96732703,14.90530176)(413.02732697,14.91030176)(413.08733154,14.91030273)
\curveto(413.14732685,14.92030175)(413.20232679,14.93030174)(413.25233154,14.94030273)
\curveto(413.56232643,15.02030165)(413.81232618,15.12530154)(414.00233154,15.25530273)
\curveto(414.20232579,15.38530128)(414.36732563,15.60530106)(414.49733154,15.91530273)
\curveto(414.52732547,15.9653007)(414.54232545,16.02030065)(414.54233154,16.08030273)
\curveto(414.55232544,16.14030053)(414.55232544,16.18530048)(414.54233154,16.21530273)
\curveto(414.53232546,16.40530026)(414.4923255,16.54530012)(414.42233154,16.63530273)
\curveto(414.35232564,16.73529993)(414.25732574,16.82529984)(414.13733154,16.90530273)
\curveto(414.05732594,16.9652997)(413.96232603,17.01529965)(413.85233154,17.05530273)
\lineto(413.55233154,17.17530273)
\curveto(413.52232647,17.18529948)(413.4923265,17.19029948)(413.46233154,17.19030273)
\curveto(413.44232655,17.19029948)(413.42232657,17.20029947)(413.40233154,17.22030273)
\curveto(413.08232691,17.33029934)(412.74232725,17.41029926)(412.38233154,17.46030273)
\curveto(412.03232796,17.52029915)(411.71232828,17.61529905)(411.42233154,17.74530273)
\curveto(411.33232866,17.78529888)(411.24232875,17.82029885)(411.15233154,17.85030273)
\curveto(411.07232892,17.88029879)(410.997329,17.92029875)(410.92733154,17.97030273)
\curveto(410.75732924,18.08029859)(410.60732939,18.20529846)(410.47733154,18.34530273)
\curveto(410.34732965,18.48529818)(410.25732974,18.66029801)(410.20733154,18.87030273)
\curveto(410.18732981,18.94029773)(410.17732982,19.01029766)(410.17733154,19.08030273)
\lineto(410.17733154,19.30530273)
\curveto(410.16732983,19.42529724)(410.18232981,19.56029711)(410.22233154,19.71030273)
\curveto(410.26232973,19.8702968)(410.30232969,20.00529666)(410.34233154,20.11530273)
\curveto(410.37232962,20.1652965)(410.3923296,20.20529646)(410.40233154,20.23530273)
\curveto(410.42232957,20.27529639)(410.44732955,20.31529635)(410.47733154,20.35530273)
\curveto(410.60732939,20.58529608)(410.76732923,20.78529588)(410.95733154,20.95530273)
\curveto(411.14732885,21.12529554)(411.35732864,21.27529539)(411.58733154,21.40530273)
\curveto(411.74732825,21.49529517)(411.92232807,21.5652951)(412.11233154,21.61530273)
\curveto(412.31232768,21.67529499)(412.51732748,21.73029494)(412.72733154,21.78030273)
\curveto(412.7973272,21.79029488)(412.86232713,21.80029487)(412.92233154,21.81030273)
\curveto(412.992327,21.82029485)(413.06732693,21.83029484)(413.14733154,21.84030273)
\curveto(413.18732681,21.85029482)(413.22732677,21.85029482)(413.26733154,21.84030273)
\curveto(413.31732668,21.83029484)(413.35732664,21.83529483)(413.38733154,21.85530273)
}
}
{
\newrgbcolor{curcolor}{0 0 0}
\pscustom[linestyle=none,fillstyle=solid,fillcolor=curcolor]
{
}
}
{
\newrgbcolor{curcolor}{0 0 0}
\pscustom[linestyle=none,fillstyle=solid,fillcolor=curcolor]
{
\newpath
\moveto(428.15248779,14.76030273)
\lineto(428.06248779,14.37030273)
\curveto(428.04247986,14.25030242)(428.0024799,14.15030252)(427.94248779,14.07030273)
\curveto(427.87248003,14.00030267)(427.77748013,13.96030271)(427.65748779,13.95030273)
\lineto(427.31248779,13.95030273)
\curveto(427.25248065,13.95030272)(427.19248071,13.94530272)(427.13248779,13.93530273)
\curveto(427.08248082,13.93530273)(427.03748087,13.94530272)(426.99748779,13.96530273)
\curveto(426.91748099,13.98530268)(426.86748104,14.02530264)(426.84748779,14.08530273)
\curveto(426.81748109,14.13530253)(426.8074811,14.19530247)(426.81748779,14.26530273)
\curveto(426.82748108,14.33530233)(426.82248108,14.40530226)(426.80248779,14.47530273)
\curveto(426.8024811,14.49530217)(426.79248111,14.51030216)(426.77248779,14.52030273)
\lineto(426.74248779,14.58030273)
\curveto(426.64248126,14.59030208)(426.55748135,14.5703021)(426.48748779,14.52030273)
\curveto(426.42748148,14.4703022)(426.36248154,14.42030225)(426.29248779,14.37030273)
\curveto(426.06248184,14.22030245)(425.83748207,14.10530256)(425.61748779,14.02530273)
\curveto(425.42748248,13.94530272)(425.2074827,13.88530278)(424.95748779,13.84530273)
\curveto(424.71748319,13.80530286)(424.47248343,13.78530288)(424.22248779,13.78530273)
\curveto(423.98248392,13.77530289)(423.74248416,13.79030288)(423.50248779,13.83030273)
\curveto(423.27248463,13.86030281)(423.07748483,13.91530275)(422.91748779,13.99530273)
\curveto(422.43748547,14.21530245)(422.07248583,14.51030216)(421.82248779,14.88030273)
\curveto(421.58248632,15.26030141)(421.42748648,15.73030094)(421.35748779,16.29030273)
\curveto(421.33748657,16.38030029)(421.32748658,16.4703002)(421.32748779,16.56030273)
\curveto(421.33748657,16.66030001)(421.33748657,16.76029991)(421.32748779,16.86030273)
\curveto(421.32748658,16.91029976)(421.33248657,16.96029971)(421.34248779,17.01030273)
\curveto(421.35248655,17.06029961)(421.35748655,17.11029956)(421.35748779,17.16030273)
\curveto(421.34748656,17.21029946)(421.34748656,17.26029941)(421.35748779,17.31030273)
\curveto(421.37748653,17.3702993)(421.38748652,17.42529924)(421.38748779,17.47530273)
\lineto(421.41748779,17.62530273)
\curveto(421.4074865,17.67529899)(421.4074865,17.74029893)(421.41748779,17.82030273)
\curveto(421.43748647,17.90029877)(421.46248644,17.9652987)(421.49248779,18.01530273)
\lineto(421.53748779,18.18030273)
\curveto(421.56748634,18.25029842)(421.58748632,18.32029835)(421.59748779,18.39030273)
\curveto(421.6074863,18.4702982)(421.62748628,18.54529812)(421.65748779,18.61530273)
\curveto(421.67748623,18.665298)(421.69248621,18.71029796)(421.70248779,18.75030273)
\curveto(421.71248619,18.79029788)(421.72748618,18.83529783)(421.74748779,18.88530273)
\curveto(421.79748611,18.98529768)(421.84248606,19.08029759)(421.88248779,19.17030273)
\curveto(421.92248598,19.2702974)(421.96748594,19.3652973)(422.01748779,19.45530273)
\curveto(422.21748569,19.83529683)(422.44748546,20.17529649)(422.70748779,20.47530273)
\curveto(422.97748493,20.78529588)(423.27748463,21.04029563)(423.60748779,21.24030273)
\curveto(423.8074841,21.36029531)(424.0074839,21.46029521)(424.20748779,21.54030273)
\curveto(424.4074835,21.62029505)(424.62248328,21.69029498)(424.85248779,21.75030273)
\lineto(425.06248779,21.78030273)
\curveto(425.13248277,21.79029488)(425.2024827,21.80529486)(425.27248779,21.82530273)
\lineto(425.42248779,21.82530273)
\curveto(425.51248239,21.84529482)(425.63248227,21.85529481)(425.78248779,21.85530273)
\curveto(425.94248196,21.85529481)(426.05748185,21.84529482)(426.12748779,21.82530273)
\curveto(426.16748174,21.81529485)(426.22248168,21.81029486)(426.29248779,21.81030273)
\curveto(426.39248151,21.78029489)(426.49748141,21.75529491)(426.60748779,21.73530273)
\curveto(426.71748119,21.72529494)(426.81748109,21.69529497)(426.90748779,21.64530273)
\curveto(427.04748086,21.58529508)(427.17748073,21.52029515)(427.29748779,21.45030273)
\curveto(427.41748049,21.38029529)(427.52748038,21.30029537)(427.62748779,21.21030273)
\curveto(427.67748023,21.16029551)(427.72748018,21.10529556)(427.77748779,21.04530273)
\curveto(427.83748007,20.99529567)(427.92247998,20.98029569)(428.03248779,21.00030273)
\lineto(428.10748779,21.07530273)
\curveto(428.12747978,21.09529557)(428.14247976,21.12529554)(428.15248779,21.16530273)
\curveto(428.2024797,21.25529541)(428.23747967,21.3702953)(428.25748779,21.51030273)
\curveto(428.28747962,21.65029502)(428.31247959,21.77529489)(428.33248779,21.88530273)
\lineto(428.67748779,23.61030273)
\curveto(428.7074792,23.75029292)(428.73747917,23.90529276)(428.76748779,24.07530273)
\curveto(428.8074791,24.25529241)(428.85747905,24.38529228)(428.91748779,24.46530273)
\curveto(428.97747893,24.53529213)(429.04747886,24.58029209)(429.12748779,24.60030273)
\curveto(429.14747876,24.60029207)(429.17247873,24.60029207)(429.20248779,24.60030273)
\curveto(429.23247867,24.61029206)(429.25747865,24.61529205)(429.27748779,24.61530273)
\curveto(429.42747848,24.62529204)(429.57747833,24.62529204)(429.72748779,24.61530273)
\curveto(429.87747803,24.61529205)(429.97747793,24.57529209)(430.02748779,24.49530273)
\curveto(430.05747785,24.41529225)(430.05747785,24.31529235)(430.02748779,24.19530273)
\curveto(430.0074779,24.07529259)(429.98747792,23.95029272)(429.96748779,23.82030273)
\lineto(428.15248779,14.76030273)
\moveto(427.50748779,17.59530273)
\curveto(427.53748037,17.64529902)(427.55748035,17.71029896)(427.56748779,17.79030273)
\curveto(427.58748032,17.88029879)(427.59248031,17.95029872)(427.58248779,18.00030273)
\lineto(427.62748779,18.22530273)
\curveto(427.62748028,18.31529835)(427.63248027,18.40529826)(427.64248779,18.49530273)
\curveto(427.65248025,18.59529807)(427.64748026,18.68529798)(427.62748779,18.76530273)
\lineto(427.62748779,18.99030273)
\curveto(427.62748028,19.06029761)(427.61748029,19.13029754)(427.59748779,19.20030273)
\curveto(427.53748037,19.50029717)(427.43248047,19.7652969)(427.28248779,19.99530273)
\curveto(427.14248076,20.22529644)(426.94248096,20.40529626)(426.68248779,20.53530273)
\curveto(426.59248131,20.58529608)(426.49748141,20.62029605)(426.39748779,20.64030273)
\curveto(426.29748161,20.670296)(426.18748172,20.69529597)(426.06748779,20.71530273)
\curveto(425.99748191,20.73529593)(425.91248199,20.74529592)(425.81248779,20.74530273)
\lineto(425.54248779,20.74530273)
\lineto(425.39248779,20.71530273)
\lineto(425.25748779,20.71530273)
\curveto(425.17748273,20.69529597)(425.09248281,20.67529599)(425.00248779,20.65530273)
\curveto(424.91248299,20.63529603)(424.82748308,20.61029606)(424.74748779,20.58030273)
\curveto(424.39748351,20.44029623)(424.09748381,20.23529643)(423.84748779,19.96530273)
\curveto(423.59748431,19.70529696)(423.37748453,19.40029727)(423.18748779,19.05030273)
\curveto(423.12748478,18.94029773)(423.07748483,18.82529784)(423.03748779,18.70530273)
\lineto(422.91748779,18.37530273)
\lineto(422.88748779,18.25530273)
\curveto(422.87748503,18.22529844)(422.86748504,18.19029848)(422.85748779,18.15030273)
\curveto(422.82748508,18.10029857)(422.8074851,18.04529862)(422.79748779,17.98530273)
\curveto(422.79748511,17.92529874)(422.79248511,17.8702988)(422.78248779,17.82030273)
\curveto(422.76248514,17.71029896)(422.73748517,17.60029907)(422.70748779,17.49030273)
\curveto(422.68748522,17.39029928)(422.68248522,17.29529937)(422.69248779,17.20530273)
\curveto(422.69248521,17.17529949)(422.68748522,17.12529954)(422.67748779,17.05530273)
\lineto(422.67748779,16.84530273)
\curveto(422.67748523,16.77529989)(422.68248522,16.70529996)(422.69248779,16.63530273)
\curveto(422.73248517,16.28530038)(422.82248508,15.98530068)(422.96248779,15.73530273)
\curveto(423.1024848,15.48530118)(423.3024846,15.28030139)(423.56248779,15.12030273)
\curveto(423.64248426,15.0703016)(423.72248418,15.03030164)(423.80248779,15.00030273)
\curveto(423.89248401,14.9703017)(423.98748392,14.94030173)(424.08748779,14.91030273)
\curveto(424.13748377,14.89030178)(424.18748372,14.88530178)(424.23748779,14.89530273)
\curveto(424.29748361,14.90530176)(424.35248355,14.90030177)(424.40248779,14.88030273)
\curveto(424.43248347,14.8703018)(424.46748344,14.8653018)(424.50748779,14.86530273)
\lineto(424.64248779,14.86530273)
\lineto(424.77748779,14.86530273)
\curveto(424.81748309,14.87530179)(424.87248303,14.88030179)(424.94248779,14.88030273)
\curveto(425.02248288,14.90030177)(425.1024828,14.91530175)(425.18248779,14.92530273)
\curveto(425.27248263,14.94530172)(425.35248255,14.9703017)(425.42248779,15.00030273)
\curveto(425.78248212,15.14030153)(426.08748182,15.31530135)(426.33748779,15.52530273)
\curveto(426.58748132,15.74530092)(426.81248109,16.02030065)(427.01248779,16.35030273)
\curveto(427.08248082,16.46030021)(427.13748077,16.5703001)(427.17748779,16.68030273)
\lineto(427.32748779,17.01030273)
\curveto(427.35748055,17.05029962)(427.37248053,17.08529958)(427.37248779,17.11530273)
\curveto(427.38248052,17.15529951)(427.39748051,17.19529947)(427.41748779,17.23530273)
\curveto(427.43748047,17.29529937)(427.45248045,17.35529931)(427.46248779,17.41530273)
\curveto(427.47248043,17.47529919)(427.48748042,17.53529913)(427.50748779,17.59530273)
}
}
{
\newrgbcolor{curcolor}{0 0 0}
\pscustom[linestyle=none,fillstyle=solid,fillcolor=curcolor]
{
\newpath
\moveto(437.52373779,18.12030273)
\curveto(437.52372929,18.02029865)(437.50372931,17.90529876)(437.46373779,17.77530273)
\curveto(437.42372939,17.65529901)(437.37372944,17.5702991)(437.31373779,17.52030273)
\curveto(437.25372956,17.48029919)(437.17372964,17.45029922)(437.07373779,17.43030273)
\curveto(436.97372984,17.42029925)(436.86372995,17.41529925)(436.74373779,17.41530273)
\lineto(436.38373779,17.41530273)
\curveto(436.27373054,17.42529924)(436.17373064,17.43029924)(436.08373779,17.43030273)
\lineto(432.24373779,17.43030273)
\curveto(432.16373465,17.43029924)(432.07873473,17.42529924)(431.98873779,17.41530273)
\curveto(431.9087349,17.41529925)(431.84373497,17.40029927)(431.79373779,17.37030273)
\curveto(431.74373507,17.35029932)(431.69373512,17.31029936)(431.64373779,17.25030273)
\lineto(431.55373779,17.11530273)
\curveto(431.52373529,17.0652996)(431.5137353,17.01529965)(431.52373779,16.96530273)
\curveto(431.52373529,16.91529975)(431.51873529,16.8702998)(431.50873779,16.83030273)
\lineto(431.50873779,16.71030273)
\lineto(431.50873779,16.45530273)
\curveto(431.51873529,16.37530029)(431.53373528,16.29530037)(431.55373779,16.21530273)
\curveto(431.68373513,15.67530099)(431.98873482,15.29030138)(432.46873779,15.06030273)
\curveto(432.51873429,15.03030164)(432.57873423,15.00530166)(432.64873779,14.98530273)
\curveto(432.71873409,14.9653017)(432.78373403,14.94530172)(432.84373779,14.92530273)
\curveto(432.87373394,14.91530175)(432.92373389,14.91030176)(432.99373779,14.91030273)
\curveto(433.12373369,14.8703018)(433.30373351,14.85030182)(433.53373779,14.85030273)
\curveto(433.76373305,14.85030182)(433.95373286,14.8703018)(434.10373779,14.91030273)
\curveto(434.25373256,14.95030172)(434.38873242,14.99030168)(434.50873779,15.03030273)
\curveto(434.63873217,15.08030159)(434.75873205,15.14030153)(434.86873779,15.21030273)
\curveto(434.98873182,15.28030139)(435.09873171,15.36030131)(435.19873779,15.45030273)
\curveto(435.29873151,15.55030112)(435.38873142,15.65530101)(435.46873779,15.76530273)
\curveto(435.54873126,15.8653008)(435.62373119,15.9703007)(435.69373779,16.08030273)
\curveto(435.76373105,16.19030048)(435.85873095,16.2703004)(435.97873779,16.32030273)
\curveto(436.01873079,16.34030033)(436.08373073,16.35530031)(436.17373779,16.36530273)
\curveto(436.27373054,16.37530029)(436.36373045,16.37530029)(436.44373779,16.36530273)
\curveto(436.53373028,16.3653003)(436.61873019,16.36030031)(436.69873779,16.35030273)
\curveto(436.77873003,16.34030033)(436.82872998,16.32030035)(436.84873779,16.29030273)
\curveto(436.93872987,16.22030045)(436.94372987,16.10530056)(436.86373779,15.94530273)
\curveto(436.72373009,15.67530099)(436.56873024,15.43530123)(436.39873779,15.22530273)
\curveto(436.13873067,14.90530176)(435.85873095,14.64030203)(435.55873779,14.43030273)
\curveto(435.26873154,14.23030244)(434.9137319,14.0653026)(434.49373779,13.93530273)
\curveto(434.38373243,13.89530277)(434.27873253,13.8703028)(434.17873779,13.86030273)
\curveto(434.07873273,13.84030283)(433.96873284,13.82030285)(433.84873779,13.80030273)
\curveto(433.79873301,13.79030288)(433.74873306,13.78530288)(433.69873779,13.78530273)
\curveto(433.65873315,13.78530288)(433.6137332,13.78030289)(433.56373779,13.77030273)
\lineto(433.41373779,13.77030273)
\curveto(433.36373345,13.76030291)(433.30373351,13.75530291)(433.23373779,13.75530273)
\curveto(433.17373364,13.75530291)(433.12373369,13.76030291)(433.08373779,13.77030273)
\lineto(432.94873779,13.77030273)
\curveto(432.89873391,13.78030289)(432.85373396,13.78530288)(432.81373779,13.78530273)
\curveto(432.77373404,13.78530288)(432.73373408,13.79030288)(432.69373779,13.80030273)
\curveto(432.64373417,13.81030286)(432.58873422,13.82030285)(432.52873779,13.83030273)
\curveto(432.47873433,13.83030284)(432.42873438,13.83530283)(432.37873779,13.84530273)
\curveto(432.28873452,13.8653028)(432.19873461,13.89030278)(432.10873779,13.92030273)
\curveto(432.02873478,13.94030273)(431.95373486,13.9653027)(431.88373779,13.99530273)
\curveto(431.84373497,14.01530265)(431.808735,14.02530264)(431.77873779,14.02530273)
\curveto(431.74873506,14.03530263)(431.71873509,14.05030262)(431.68873779,14.07030273)
\curveto(431.54873526,14.14030253)(431.40373541,14.22530244)(431.25373779,14.32530273)
\curveto(431.00373581,14.51530215)(430.80373601,14.74530192)(430.65373779,15.01530273)
\curveto(430.50373631,15.29530137)(430.39373642,15.60530106)(430.32373779,15.94530273)
\curveto(430.29373652,16.05530061)(430.27873653,16.1703005)(430.27873779,16.29030273)
\curveto(430.27873653,16.41030026)(430.26873654,16.53030014)(430.24873779,16.65030273)
\lineto(430.24873779,16.75530273)
\curveto(430.25873655,16.78529988)(430.26373655,16.82529984)(430.26373779,16.87530273)
\lineto(430.26373779,17.13030273)
\curveto(430.27373654,17.22029945)(430.27873653,17.31029936)(430.27873779,17.40030273)
\lineto(430.32373779,17.61030273)
\curveto(430.32373649,17.65029902)(430.32873648,17.70529896)(430.33873779,17.77530273)
\curveto(430.34873646,17.85529881)(430.36373645,17.92029875)(430.38373779,17.97030273)
\lineto(430.41373779,18.13530273)
\curveto(430.44373637,18.18529848)(430.45873635,18.23529843)(430.45873779,18.28530273)
\curveto(430.46873634,18.34529832)(430.48373633,18.40029827)(430.50373779,18.45030273)
\curveto(430.57373624,18.61029806)(430.63873617,18.7702979)(430.69873779,18.93030273)
\curveto(430.75873605,19.09029758)(430.83373598,19.24029743)(430.92373779,19.38030273)
\curveto(430.99373582,19.49029718)(431.05873575,19.60029707)(431.11873779,19.71030273)
\curveto(431.18873562,19.83029684)(431.26873554,19.94529672)(431.35873779,20.05530273)
\curveto(431.64873516,20.40529626)(431.95873485,20.70529596)(432.28873779,20.95530273)
\curveto(432.61873419,21.21529545)(433.00373381,21.43029524)(433.44373779,21.60030273)
\curveto(433.57373324,21.65029502)(433.70373311,21.68529498)(433.83373779,21.70530273)
\curveto(433.96373285,21.73529493)(434.10373271,21.7652949)(434.25373779,21.79530273)
\curveto(434.30373251,21.80529486)(434.34873246,21.81029486)(434.38873779,21.81030273)
\curveto(434.42873238,21.82029485)(434.47373234,21.82529484)(434.52373779,21.82530273)
\curveto(434.54373227,21.83529483)(434.56873224,21.83529483)(434.59873779,21.82530273)
\curveto(434.62873218,21.81529485)(434.65373216,21.82029485)(434.67373779,21.84030273)
\curveto(435.10373171,21.85029482)(435.46373135,21.80529486)(435.75373779,21.70530273)
\curveto(436.04373077,21.61529505)(436.29873051,21.49029518)(436.51873779,21.33030273)
\curveto(436.55873025,21.31029536)(436.58873022,21.28029539)(436.60873779,21.24030273)
\curveto(436.63873017,21.21029546)(436.66873014,21.18529548)(436.69873779,21.16530273)
\curveto(436.76873004,21.10529556)(436.83872997,21.03529563)(436.90873779,20.95530273)
\curveto(436.97872983,20.87529579)(437.03372978,20.79529587)(437.07373779,20.71530273)
\curveto(437.19372962,20.50529616)(437.28872952,20.30529636)(437.35873779,20.11530273)
\curveto(437.4087294,20.00529666)(437.43872937,19.88529678)(437.44873779,19.75530273)
\lineto(437.50873779,19.36530273)
\curveto(437.53872927,19.23529743)(437.54872926,19.10029757)(437.53873779,18.96030273)
\curveto(437.53872927,18.82029785)(437.54372927,18.68029799)(437.55373779,18.54030273)
\curveto(437.55372926,18.4702982)(437.54872926,18.40029827)(437.53873779,18.33030273)
\curveto(437.52872928,18.26029841)(437.52372929,18.19029848)(437.52373779,18.12030273)
\moveto(436.17373779,18.63030273)
\curveto(436.20373061,18.670298)(436.23373058,18.72029795)(436.26373779,18.78030273)
\curveto(436.30373051,18.85029782)(436.31873049,18.92029775)(436.30873779,18.99030273)
\curveto(436.29873051,19.21029746)(436.25873055,19.41529725)(436.18873779,19.60530273)
\curveto(436.08873072,19.83529683)(435.96873084,20.03029664)(435.82873779,20.19030273)
\curveto(435.69873111,20.35029632)(435.5087313,20.48529618)(435.25873779,20.59530273)
\curveto(435.18873162,20.61529605)(435.11873169,20.63029604)(435.04873779,20.64030273)
\curveto(434.98873182,20.66029601)(434.91873189,20.68029599)(434.83873779,20.70030273)
\curveto(434.76873204,20.72029595)(434.68873212,20.73029594)(434.59873779,20.73030273)
\lineto(434.34373779,20.73030273)
\curveto(434.30373251,20.71029596)(434.26373255,20.70029597)(434.22373779,20.70030273)
\curveto(434.18373263,20.71029596)(434.14873266,20.71029596)(434.11873779,20.70030273)
\lineto(433.87873779,20.64030273)
\curveto(433.79873301,20.63029604)(433.72373309,20.61529605)(433.65373779,20.59530273)
\curveto(433.33373348,20.47529619)(433.06873374,20.32529634)(432.85873779,20.14530273)
\curveto(432.64873416,19.9652967)(432.44873436,19.74029693)(432.25873779,19.47030273)
\curveto(432.21873459,19.42029725)(432.17373464,19.35529731)(432.12373779,19.27530273)
\curveto(432.08373473,19.20529746)(432.04373477,19.12529754)(432.00373779,19.03530273)
\curveto(431.96373485,18.94529772)(431.93873487,18.86029781)(431.92873779,18.78030273)
\curveto(431.92873488,18.70029797)(431.95373486,18.64029803)(432.00373779,18.60030273)
\curveto(432.07373474,18.54029813)(432.20373461,18.51029816)(432.39373779,18.51030273)
\curveto(432.59373422,18.52029815)(432.76373405,18.52529814)(432.90373779,18.52530273)
\lineto(435.18373779,18.52530273)
\curveto(435.33373148,18.52529814)(435.5137313,18.52029815)(435.72373779,18.51030273)
\curveto(435.93373088,18.51029816)(436.08373073,18.55029812)(436.17373779,18.63030273)
}
}
{
\newrgbcolor{curcolor}{0 0 0}
\pscustom[linestyle=none,fillstyle=solid,fillcolor=curcolor]
{
\newpath
\moveto(442.59537842,24.75030273)
\curveto(442.77537272,24.76029191)(442.96537253,24.76029191)(443.16537842,24.75030273)
\curveto(443.36537213,24.74029193)(443.495372,24.68029199)(443.55537842,24.57030273)
\curveto(443.58537191,24.51029216)(443.5953719,24.43529223)(443.58537842,24.34530273)
\curveto(443.57537192,24.2652924)(443.56037193,24.17529249)(443.54037842,24.07530273)
\curveto(443.52037197,23.94529272)(443.47537202,23.84029283)(443.40537842,23.76030273)
\curveto(443.35537214,23.71029296)(443.2903722,23.67529299)(443.21037842,23.65530273)
\curveto(443.13037236,23.64529302)(443.04537245,23.64029303)(442.95537842,23.64030273)
\lineto(442.68537842,23.64030273)
\curveto(442.5953729,23.65029302)(442.51037298,23.65029302)(442.43037842,23.64030273)
\curveto(442.14037335,23.56029311)(441.93537356,23.43029324)(441.81537842,23.25030273)
\curveto(441.6953738,23.08029359)(441.60037389,22.82029385)(441.53037842,22.47030273)
\curveto(441.51037398,22.40029427)(441.48537401,22.32529434)(441.45537842,22.24530273)
\curveto(441.43537406,22.17529449)(441.43037406,22.11029456)(441.44037842,22.05030273)
\curveto(441.44037405,21.90029477)(441.48537401,21.79529487)(441.57537842,21.73530273)
\curveto(441.64537385,21.70529496)(441.74037375,21.69029498)(441.86037842,21.69030273)
\lineto(442.22037842,21.69030273)
\lineto(442.44537842,21.69030273)
\curveto(442.47537302,21.670295)(442.50537299,21.665295)(442.53537842,21.67530273)
\curveto(442.56537293,21.68529498)(442.5953729,21.68029499)(442.62537842,21.66030273)
\curveto(442.71537278,21.63029504)(442.76537273,21.5702951)(442.77537842,21.48030273)
\curveto(442.7953727,21.40029527)(442.7903727,21.29529537)(442.76037842,21.16530273)
\lineto(442.73037842,21.04530273)
\lineto(442.70037842,20.92530273)
\curveto(442.64037285,20.77529589)(442.55537294,20.67529599)(442.44537842,20.62530273)
\curveto(442.30537319,20.57529609)(442.13537336,20.56029611)(441.93537842,20.58030273)
\curveto(441.73537376,20.61029606)(441.56037393,20.60529606)(441.41037842,20.56530273)
\curveto(441.33037416,20.54529612)(441.26537423,20.50529616)(441.21537842,20.44530273)
\curveto(441.16537433,20.39529627)(441.12037437,20.32529634)(441.08037842,20.23530273)
\curveto(441.05037444,20.1652965)(441.03037446,20.08529658)(441.02037842,19.99530273)
\curveto(441.01037448,19.90529676)(440.9953745,19.82029685)(440.97537842,19.74030273)
\lineto(440.78037842,18.75030273)
\lineto(440.15037842,15.57030273)
\lineto(440.00037842,14.82030273)
\curveto(439.9903755,14.76030191)(439.98037551,14.69530197)(439.97037842,14.62530273)
\curveto(439.96037553,14.55530211)(439.94037555,14.49530217)(439.91037842,14.44530273)
\lineto(439.88037842,14.32530273)
\lineto(439.82037842,14.20530273)
\curveto(439.81037568,14.1653025)(439.7903757,14.13030254)(439.76037842,14.10030273)
\curveto(439.70037579,14.03030264)(439.61537588,13.99030268)(439.50537842,13.98030273)
\curveto(439.40537609,13.9703027)(439.2953762,13.9653027)(439.17537842,13.96530273)
\lineto(438.89037842,13.96530273)
\curveto(438.85037664,13.98530268)(438.80537669,14.00030267)(438.75537842,14.01030273)
\curveto(438.71537678,14.03030264)(438.68537681,14.0653026)(438.66537842,14.11530273)
\curveto(438.65537684,14.14530252)(438.65037684,14.21030246)(438.65037842,14.31030273)
\lineto(438.66537842,14.41530273)
\curveto(438.65537684,14.4653022)(438.66037683,14.51530215)(438.68037842,14.56530273)
\curveto(438.70037679,14.62530204)(438.71537678,14.68030199)(438.72537842,14.73030273)
\lineto(438.84537842,15.33030273)
\lineto(439.65537842,19.42530273)
\curveto(439.67537582,19.53529713)(439.70037579,19.65029702)(439.73037842,19.77030273)
\curveto(439.76037573,19.89029678)(439.78037571,20.00029667)(439.79037842,20.10030273)
\curveto(439.81037568,20.21029646)(439.81037568,20.30529636)(439.79037842,20.38530273)
\curveto(439.78037571,20.4652962)(439.73537576,20.52029615)(439.65537842,20.55030273)
\curveto(439.60537589,20.58029609)(439.54037595,20.59529607)(439.46037842,20.59530273)
\lineto(439.23537842,20.59530273)
\lineto(438.99537842,20.59530273)
\curveto(438.92537657,20.59529607)(438.86037663,20.60529606)(438.80037842,20.62530273)
\curveto(438.72037677,20.665296)(438.67537682,20.75029592)(438.66537842,20.88030273)
\lineto(438.66537842,21.01530273)
\curveto(438.67537682,21.05529561)(438.68537681,21.10029557)(438.69537842,21.15030273)
\curveto(438.72537677,21.29029538)(438.76037673,21.40029527)(438.80037842,21.48030273)
\curveto(438.85037664,21.5702951)(438.93037656,21.63029504)(439.04037842,21.66030273)
\curveto(439.12037637,21.69029498)(439.20537629,21.70029497)(439.29537842,21.69030273)
\lineto(439.56537842,21.69030273)
\curveto(439.66537583,21.69029498)(439.75537574,21.70029497)(439.83537842,21.72030273)
\curveto(439.91537558,21.74029493)(439.98537551,21.78029489)(440.04537842,21.84030273)
\curveto(440.13537536,21.92029475)(440.1953753,22.04529462)(440.22537842,22.21530273)
\curveto(440.25537524,22.38529428)(440.28537521,22.54529412)(440.31537842,22.69530273)
\curveto(440.35537514,22.89529377)(440.40537509,23.08029359)(440.46537842,23.25030273)
\curveto(440.52537497,23.43029324)(440.60037489,23.59029308)(440.69037842,23.73030273)
\curveto(440.84037465,23.9702927)(441.02037447,24.1652925)(441.23037842,24.31530273)
\curveto(441.45037404,24.4652922)(441.70037379,24.58029209)(441.98037842,24.66030273)
\curveto(442.04037345,24.68029199)(442.10537339,24.69029198)(442.17537842,24.69030273)
\curveto(442.24537325,24.70029197)(442.31537318,24.71529195)(442.38537842,24.73530273)
\curveto(442.40537309,24.74529192)(442.44037305,24.74529192)(442.49037842,24.73530273)
\curveto(442.54037295,24.73529193)(442.57537292,24.74029193)(442.59537842,24.75030273)
\moveto(444.54537842,23.17530273)
\curveto(444.60537089,23.12529354)(444.68537081,23.10029357)(444.78537842,23.10030273)
\lineto(445.10037842,23.10030273)
\lineto(445.26537842,23.10030273)
\curveto(445.32537017,23.10029357)(445.38537011,23.11029356)(445.44537842,23.13030273)
\curveto(445.58536991,23.18029349)(445.67036982,23.28529338)(445.70037842,23.44530273)
\curveto(445.74036975,23.60529306)(445.78036971,23.77529289)(445.82037842,23.95530273)
\curveto(445.83036966,24.04529262)(445.84536965,24.13029254)(445.86537842,24.21030273)
\curveto(445.88536961,24.30029237)(445.88536961,24.37529229)(445.86537842,24.43530273)
\curveto(445.83536966,24.54529212)(445.74536975,24.60529206)(445.59537842,24.61530273)
\curveto(445.45537004,24.62529204)(445.30037019,24.63029204)(445.13037842,24.63030273)
\curveto(445.10037039,24.62029205)(445.07537042,24.61529205)(445.05537842,24.61530273)
\curveto(445.03537046,24.62529204)(445.01037048,24.62529204)(444.98037842,24.61530273)
\curveto(444.86037063,24.57529209)(444.77037072,24.51529215)(444.71037842,24.43530273)
\curveto(444.67037082,24.37529229)(444.64037085,24.30029237)(444.62037842,24.21030273)
\curveto(444.60037089,24.12029255)(444.58537091,24.03529263)(444.57537842,23.95530273)
\curveto(444.54537095,23.80529286)(444.51537098,23.65029302)(444.48537842,23.49030273)
\curveto(444.45537104,23.34029333)(444.47537102,23.23529343)(444.54537842,23.17530273)
\moveto(445.22037842,21.01530273)
\curveto(445.24037025,21.11529555)(445.26037023,21.21029546)(445.28037842,21.30030273)
\curveto(445.30037019,21.40029527)(445.2903702,21.48029519)(445.25037842,21.54030273)
\curveto(445.22037027,21.62029505)(445.13537036,21.66029501)(444.99537842,21.66030273)
\curveto(444.86537063,21.670295)(444.73537076,21.67529499)(444.60537842,21.67530273)
\curveto(444.58537091,21.665295)(444.56037093,21.66029501)(444.53037842,21.66030273)
\curveto(444.51037098,21.670295)(444.490371,21.67529499)(444.47037842,21.67530273)
\curveto(444.41037108,21.65529501)(444.35037114,21.64029503)(444.29037842,21.63030273)
\curveto(444.24037125,21.62029505)(444.1953713,21.59029508)(444.15537842,21.54030273)
\curveto(444.0953714,21.48029519)(444.05537144,21.39529527)(444.03537842,21.28530273)
\curveto(444.01537148,21.18529548)(443.9953715,21.08029559)(443.97537842,20.97030273)
\lineto(442.70037842,14.62530273)
\curveto(442.68037281,14.53530213)(442.66037283,14.44030223)(442.64037842,14.34030273)
\curveto(442.63037286,14.25030242)(442.63537286,14.17530249)(442.65537842,14.11530273)
\curveto(442.6953728,14.03530263)(442.76037273,13.98530268)(442.85037842,13.96530273)
\curveto(442.94037255,13.95530271)(443.05037244,13.95030272)(443.18037842,13.95030273)
\lineto(443.40537842,13.95030273)
\curveto(443.495372,13.9703027)(443.57037192,13.98530268)(443.63037842,13.99530273)
\curveto(443.6903718,14.01530265)(443.74037175,14.05530261)(443.78037842,14.11530273)
\curveto(443.85037164,14.17530249)(443.8903716,14.25530241)(443.90037842,14.35530273)
\curveto(443.92037157,14.4653022)(443.94037155,14.5703021)(443.96037842,14.67030273)
\lineto(445.22037842,21.01530273)
}
}
{
\newrgbcolor{curcolor}{0 0 0}
\pscustom[linestyle=none,fillstyle=solid,fillcolor=curcolor]
{
\newpath
\moveto(451.01905029,21.82530273)
\curveto(451.65904347,21.84529482)(452.14904298,21.76029491)(452.48905029,21.57030273)
\curveto(452.8290423,21.38029529)(453.07404206,21.09529557)(453.22405029,20.71530273)
\curveto(453.26404187,20.61529605)(453.28904184,20.50529616)(453.29905029,20.38530273)
\curveto(453.31904181,20.27529639)(453.3290418,20.16029651)(453.32905029,20.04030273)
\curveto(453.34904178,19.85029682)(453.33904179,19.64529702)(453.29905029,19.42530273)
\curveto(453.26904186,19.20529746)(453.2290419,18.98029769)(453.17905029,18.75030273)
\lineto(452.86405029,17.14530273)
\lineto(452.39905029,14.80530273)
\lineto(452.27905029,14.29530273)
\curveto(452.23904289,14.12530254)(452.14904298,14.01530265)(452.00905029,13.96530273)
\curveto(451.95904317,13.94530272)(451.90404323,13.93530273)(451.84405029,13.93530273)
\curveto(451.79404334,13.92530274)(451.73904339,13.92030275)(451.67905029,13.92030273)
\curveto(451.54904358,13.92030275)(451.42404371,13.92530274)(451.30405029,13.93530273)
\curveto(451.18404395,13.93530273)(451.10904402,13.97530269)(451.07905029,14.05530273)
\curveto(451.03904409,14.12530254)(451.0290441,14.21530245)(451.04905029,14.32530273)
\curveto(451.06904406,14.43530223)(451.09404404,14.54530212)(451.12405029,14.65530273)
\lineto(451.37905029,15.94530273)
\lineto(451.85905029,18.39030273)
\curveto(451.91904321,18.66029801)(451.96904316,18.92529774)(452.00905029,19.18530273)
\curveto(452.04904308,19.45529721)(452.04904308,19.68529698)(452.00905029,19.87530273)
\curveto(451.96904316,20.07529659)(451.87904325,20.23529643)(451.73905029,20.35530273)
\curveto(451.60904352,20.48529618)(451.44904368,20.58529608)(451.25905029,20.65530273)
\curveto(451.19904393,20.67529599)(451.134044,20.68529598)(451.06405029,20.68530273)
\curveto(451.00404413,20.69529597)(450.94904418,20.71029596)(450.89905029,20.73030273)
\curveto(450.84904428,20.74029593)(450.76904436,20.74029593)(450.65905029,20.73030273)
\curveto(450.55904457,20.73029594)(450.48404465,20.72529594)(450.43405029,20.71530273)
\curveto(450.39404474,20.69529597)(450.35904477,20.68529598)(450.32905029,20.68530273)
\curveto(450.29904483,20.69529597)(450.26404487,20.69529597)(450.22405029,20.68530273)
\curveto(450.08404505,20.65529601)(449.95404518,20.62029605)(449.83405029,20.58030273)
\curveto(449.71404542,20.55029612)(449.59904553,20.50529616)(449.48905029,20.44530273)
\curveto(449.43904569,20.42529624)(449.39904573,20.40529626)(449.36905029,20.38530273)
\curveto(449.33904579,20.3652963)(449.29904583,20.34529632)(449.24905029,20.32530273)
\curveto(448.84904628,20.07529659)(448.51904661,19.70029697)(448.25905029,19.20030273)
\curveto(448.21904691,19.12029755)(448.18404695,19.03529763)(448.15405029,18.94530273)
\lineto(448.06405029,18.70530273)
\curveto(448.0340471,18.65529801)(448.01904711,18.60529806)(448.01905029,18.55530273)
\curveto(448.01904711,18.51529815)(448.00404713,18.47529819)(447.97405029,18.43530273)
\lineto(447.91405029,18.12030273)
\curveto(447.89404724,18.09029858)(447.88404725,18.05529861)(447.88405029,18.01530273)
\curveto(447.88404725,17.97529869)(447.87904725,17.93029874)(447.86905029,17.88030273)
\lineto(447.77905029,17.43030273)
\lineto(447.47905029,15.99030273)
\lineto(447.22405029,14.67030273)
\curveto(447.20404793,14.56030211)(447.17904795,14.44530222)(447.14905029,14.32530273)
\curveto(447.129048,14.21530245)(447.08904804,14.12530254)(447.02905029,14.05530273)
\curveto(446.95904817,13.97530269)(446.85904827,13.93530273)(446.72905029,13.93530273)
\curveto(446.60904852,13.92530274)(446.48404865,13.92030275)(446.35405029,13.92030273)
\curveto(446.27404886,13.92030275)(446.19904893,13.92530274)(446.12905029,13.93530273)
\curveto(446.05904907,13.94530272)(446.00404913,13.9703027)(445.96405029,14.01030273)
\curveto(445.89404924,14.06030261)(445.87404926,14.15530251)(445.90405029,14.29530273)
\curveto(445.9340492,14.43530223)(445.95904917,14.5703021)(445.97905029,14.70030273)
\lineto(446.33905029,16.47030273)
\lineto(447.05905029,20.10030273)
\lineto(447.23905029,21.01530273)
\lineto(447.29905029,21.28530273)
\curveto(447.31904781,21.37529529)(447.35404778,21.44529522)(447.40405029,21.49530273)
\curveto(447.44404769,21.55529511)(447.49904763,21.59529507)(447.56905029,21.61530273)
\curveto(447.61904751,21.62529504)(447.67904745,21.63529503)(447.74905029,21.64530273)
\curveto(447.8290473,21.65529501)(447.90904722,21.66029501)(447.98905029,21.66030273)
\curveto(448.06904706,21.66029501)(448.14404699,21.65529501)(448.21405029,21.64530273)
\curveto(448.29404684,21.63529503)(448.34404679,21.62029505)(448.36405029,21.60030273)
\curveto(448.46404667,21.53029514)(448.49904663,21.44029523)(448.46905029,21.33030273)
\curveto(448.43904669,21.23029544)(448.4290467,21.11529555)(448.43905029,20.98530273)
\curveto(448.44904668,20.92529574)(448.47904665,20.87529579)(448.52905029,20.83530273)
\curveto(448.64904648,20.82529584)(448.75404638,20.8702958)(448.84405029,20.97030273)
\curveto(448.94404619,21.0702956)(449.03904609,21.15029552)(449.12905029,21.21030273)
\curveto(449.28904584,21.31029536)(449.44904568,21.40029527)(449.60905029,21.48030273)
\curveto(449.76904536,21.5702951)(449.95404518,21.64529502)(450.16405029,21.70530273)
\curveto(450.24404489,21.73529493)(450.3340448,21.75529491)(450.43405029,21.76530273)
\curveto(450.5340446,21.77529489)(450.6290445,21.79029488)(450.71905029,21.81030273)
\curveto(450.76904436,21.82029485)(450.81904431,21.82529484)(450.86905029,21.82530273)
\lineto(451.01905029,21.82530273)
}
}
{
\newrgbcolor{curcolor}{0 0 0}
\pscustom[linestyle=none,fillstyle=solid,fillcolor=curcolor]
{
\newpath
\moveto(456.23365967,23.17530273)
\curveto(456.16365669,23.23529343)(456.14365671,23.34029333)(456.17365967,23.49030273)
\curveto(456.20365665,23.65029302)(456.23365662,23.80529286)(456.26365967,23.95530273)
\curveto(456.27365658,24.03529263)(456.28865657,24.12029255)(456.30865967,24.21030273)
\curveto(456.32865653,24.30029237)(456.3586565,24.37529229)(456.39865967,24.43530273)
\curveto(456.4586564,24.51529215)(456.54865631,24.57529209)(456.66865967,24.61530273)
\curveto(456.69865616,24.62529204)(456.72365613,24.62529204)(456.74365967,24.61530273)
\curveto(456.76365609,24.61529205)(456.78865607,24.62029205)(456.81865967,24.63030273)
\curveto(456.98865587,24.63029204)(457.14365571,24.62529204)(457.28365967,24.61530273)
\curveto(457.43365542,24.60529206)(457.52365533,24.54529212)(457.55365967,24.43530273)
\curveto(457.57365528,24.37529229)(457.57365528,24.30029237)(457.55365967,24.21030273)
\curveto(457.53365532,24.13029254)(457.51865534,24.04529262)(457.50865967,23.95530273)
\curveto(457.46865539,23.77529289)(457.42865543,23.60529306)(457.38865967,23.44530273)
\curveto(457.3586555,23.28529338)(457.27365558,23.18029349)(457.13365967,23.13030273)
\curveto(457.07365578,23.11029356)(457.01365584,23.10029357)(456.95365967,23.10030273)
\lineto(456.78865967,23.10030273)
\lineto(456.47365967,23.10030273)
\curveto(456.37365648,23.10029357)(456.29365656,23.12529354)(456.23365967,23.17530273)
\moveto(455.64865967,14.67030273)
\curveto(455.62865723,14.5703021)(455.60865725,14.4653022)(455.58865967,14.35530273)
\curveto(455.57865728,14.25530241)(455.53865732,14.17530249)(455.46865967,14.11530273)
\curveto(455.42865743,14.05530261)(455.37865748,14.01530265)(455.31865967,13.99530273)
\curveto(455.2586576,13.98530268)(455.18365767,13.9703027)(455.09365967,13.95030273)
\lineto(454.86865967,13.95030273)
\curveto(454.73865812,13.95030272)(454.62865823,13.95530271)(454.53865967,13.96530273)
\curveto(454.44865841,13.98530268)(454.38365847,14.03530263)(454.34365967,14.11530273)
\curveto(454.32365853,14.17530249)(454.31865854,14.25030242)(454.32865967,14.34030273)
\curveto(454.34865851,14.44030223)(454.36865849,14.53530213)(454.38865967,14.62530273)
\lineto(455.66365967,20.97030273)
\curveto(455.68365717,21.08029559)(455.70365715,21.18529548)(455.72365967,21.28530273)
\curveto(455.74365711,21.39529527)(455.78365707,21.48029519)(455.84365967,21.54030273)
\curveto(455.88365697,21.59029508)(455.92865693,21.62029505)(455.97865967,21.63030273)
\curveto(456.03865682,21.64029503)(456.09865676,21.65529501)(456.15865967,21.67530273)
\curveto(456.17865668,21.67529499)(456.19865666,21.670295)(456.21865967,21.66030273)
\curveto(456.24865661,21.66029501)(456.27365658,21.665295)(456.29365967,21.67530273)
\curveto(456.42365643,21.67529499)(456.5536563,21.670295)(456.68365967,21.66030273)
\curveto(456.82365603,21.66029501)(456.90865595,21.62029505)(456.93865967,21.54030273)
\curveto(456.97865588,21.48029519)(456.98865587,21.40029527)(456.96865967,21.30030273)
\curveto(456.94865591,21.21029546)(456.92865593,21.11529555)(456.90865967,21.01530273)
\lineto(455.64865967,14.67030273)
}
}
{
\newrgbcolor{curcolor}{0 0 0}
\pscustom[linestyle=none,fillstyle=solid,fillcolor=curcolor]
{
\newpath
\moveto(464.56850342,14.76030273)
\lineto(464.47850342,14.37030273)
\curveto(464.45849549,14.25030242)(464.41849553,14.15030252)(464.35850342,14.07030273)
\curveto(464.28849566,14.00030267)(464.19349575,13.96030271)(464.07350342,13.95030273)
\lineto(463.72850342,13.95030273)
\curveto(463.66849628,13.95030272)(463.60849634,13.94530272)(463.54850342,13.93530273)
\curveto(463.49849645,13.93530273)(463.45349649,13.94530272)(463.41350342,13.96530273)
\curveto(463.33349661,13.98530268)(463.28349666,14.02530264)(463.26350342,14.08530273)
\curveto(463.23349671,14.13530253)(463.22349672,14.19530247)(463.23350342,14.26530273)
\curveto(463.2434967,14.33530233)(463.23849671,14.40530226)(463.21850342,14.47530273)
\curveto(463.21849673,14.49530217)(463.20849674,14.51030216)(463.18850342,14.52030273)
\lineto(463.15850342,14.58030273)
\curveto(463.05849689,14.59030208)(462.97349697,14.5703021)(462.90350342,14.52030273)
\curveto(462.8434971,14.4703022)(462.77849717,14.42030225)(462.70850342,14.37030273)
\curveto(462.47849747,14.22030245)(462.25349769,14.10530256)(462.03350342,14.02530273)
\curveto(461.8434981,13.94530272)(461.62349832,13.88530278)(461.37350342,13.84530273)
\curveto(461.13349881,13.80530286)(460.88849906,13.78530288)(460.63850342,13.78530273)
\curveto(460.39849955,13.77530289)(460.15849979,13.79030288)(459.91850342,13.83030273)
\curveto(459.68850026,13.86030281)(459.49350045,13.91530275)(459.33350342,13.99530273)
\curveto(458.85350109,14.21530245)(458.48850146,14.51030216)(458.23850342,14.88030273)
\curveto(457.99850195,15.26030141)(457.8435021,15.73030094)(457.77350342,16.29030273)
\curveto(457.75350219,16.38030029)(457.7435022,16.4703002)(457.74350342,16.56030273)
\curveto(457.75350219,16.66030001)(457.75350219,16.76029991)(457.74350342,16.86030273)
\curveto(457.7435022,16.91029976)(457.7485022,16.96029971)(457.75850342,17.01030273)
\curveto(457.76850218,17.06029961)(457.77350217,17.11029956)(457.77350342,17.16030273)
\curveto(457.76350218,17.21029946)(457.76350218,17.26029941)(457.77350342,17.31030273)
\curveto(457.79350215,17.3702993)(457.80350214,17.42529924)(457.80350342,17.47530273)
\lineto(457.83350342,17.62530273)
\curveto(457.82350212,17.67529899)(457.82350212,17.74029893)(457.83350342,17.82030273)
\curveto(457.85350209,17.90029877)(457.87850207,17.9652987)(457.90850342,18.01530273)
\lineto(457.95350342,18.18030273)
\curveto(457.98350196,18.25029842)(458.00350194,18.32029835)(458.01350342,18.39030273)
\curveto(458.02350192,18.4702982)(458.0435019,18.54529812)(458.07350342,18.61530273)
\curveto(458.09350185,18.665298)(458.10850184,18.71029796)(458.11850342,18.75030273)
\curveto(458.12850182,18.79029788)(458.1435018,18.83529783)(458.16350342,18.88530273)
\curveto(458.21350173,18.98529768)(458.25850169,19.08029759)(458.29850342,19.17030273)
\curveto(458.33850161,19.2702974)(458.38350156,19.3652973)(458.43350342,19.45530273)
\curveto(458.63350131,19.83529683)(458.86350108,20.17529649)(459.12350342,20.47530273)
\curveto(459.39350055,20.78529588)(459.69350025,21.04029563)(460.02350342,21.24030273)
\curveto(460.22349972,21.36029531)(460.42349952,21.46029521)(460.62350342,21.54030273)
\curveto(460.82349912,21.62029505)(461.03849891,21.69029498)(461.26850342,21.75030273)
\lineto(461.47850342,21.78030273)
\curveto(461.5484984,21.79029488)(461.61849833,21.80529486)(461.68850342,21.82530273)
\lineto(461.83850342,21.82530273)
\curveto(461.92849802,21.84529482)(462.0484979,21.85529481)(462.19850342,21.85530273)
\curveto(462.35849759,21.85529481)(462.47349747,21.84529482)(462.54350342,21.82530273)
\curveto(462.58349736,21.81529485)(462.63849731,21.81029486)(462.70850342,21.81030273)
\curveto(462.80849714,21.78029489)(462.91349703,21.75529491)(463.02350342,21.73530273)
\curveto(463.13349681,21.72529494)(463.23349671,21.69529497)(463.32350342,21.64530273)
\curveto(463.46349648,21.58529508)(463.59349635,21.52029515)(463.71350342,21.45030273)
\curveto(463.83349611,21.38029529)(463.943496,21.30029537)(464.04350342,21.21030273)
\curveto(464.09349585,21.16029551)(464.1434958,21.10529556)(464.19350342,21.04530273)
\curveto(464.25349569,20.99529567)(464.33849561,20.98029569)(464.44850342,21.00030273)
\lineto(464.52350342,21.07530273)
\curveto(464.5434954,21.09529557)(464.55849539,21.12529554)(464.56850342,21.16530273)
\curveto(464.61849533,21.25529541)(464.65349529,21.3702953)(464.67350342,21.51030273)
\curveto(464.70349524,21.65029502)(464.72849522,21.77529489)(464.74850342,21.88530273)
\lineto(465.09350342,23.61030273)
\curveto(465.12349482,23.75029292)(465.15349479,23.90529276)(465.18350342,24.07530273)
\curveto(465.22349472,24.25529241)(465.27349467,24.38529228)(465.33350342,24.46530273)
\curveto(465.39349455,24.53529213)(465.46349448,24.58029209)(465.54350342,24.60030273)
\curveto(465.56349438,24.60029207)(465.58849436,24.60029207)(465.61850342,24.60030273)
\curveto(465.6484943,24.61029206)(465.67349427,24.61529205)(465.69350342,24.61530273)
\curveto(465.8434941,24.62529204)(465.99349395,24.62529204)(466.14350342,24.61530273)
\curveto(466.29349365,24.61529205)(466.39349355,24.57529209)(466.44350342,24.49530273)
\curveto(466.47349347,24.41529225)(466.47349347,24.31529235)(466.44350342,24.19530273)
\curveto(466.42349352,24.07529259)(466.40349354,23.95029272)(466.38350342,23.82030273)
\lineto(464.56850342,14.76030273)
\moveto(463.92350342,17.59530273)
\curveto(463.95349599,17.64529902)(463.97349597,17.71029896)(463.98350342,17.79030273)
\curveto(464.00349594,17.88029879)(464.00849594,17.95029872)(463.99850342,18.00030273)
\lineto(464.04350342,18.22530273)
\curveto(464.0434959,18.31529835)(464.0484959,18.40529826)(464.05850342,18.49530273)
\curveto(464.06849588,18.59529807)(464.06349588,18.68529798)(464.04350342,18.76530273)
\lineto(464.04350342,18.99030273)
\curveto(464.0434959,19.06029761)(464.03349591,19.13029754)(464.01350342,19.20030273)
\curveto(463.95349599,19.50029717)(463.8484961,19.7652969)(463.69850342,19.99530273)
\curveto(463.55849639,20.22529644)(463.35849659,20.40529626)(463.09850342,20.53530273)
\curveto(463.00849694,20.58529608)(462.91349703,20.62029605)(462.81350342,20.64030273)
\curveto(462.71349723,20.670296)(462.60349734,20.69529597)(462.48350342,20.71530273)
\curveto(462.41349753,20.73529593)(462.32849762,20.74529592)(462.22850342,20.74530273)
\lineto(461.95850342,20.74530273)
\lineto(461.80850342,20.71530273)
\lineto(461.67350342,20.71530273)
\curveto(461.59349835,20.69529597)(461.50849844,20.67529599)(461.41850342,20.65530273)
\curveto(461.32849862,20.63529603)(461.2434987,20.61029606)(461.16350342,20.58030273)
\curveto(460.81349913,20.44029623)(460.51349943,20.23529643)(460.26350342,19.96530273)
\curveto(460.01349993,19.70529696)(459.79350015,19.40029727)(459.60350342,19.05030273)
\curveto(459.5435004,18.94029773)(459.49350045,18.82529784)(459.45350342,18.70530273)
\lineto(459.33350342,18.37530273)
\lineto(459.30350342,18.25530273)
\curveto(459.29350065,18.22529844)(459.28350066,18.19029848)(459.27350342,18.15030273)
\curveto(459.2435007,18.10029857)(459.22350072,18.04529862)(459.21350342,17.98530273)
\curveto(459.21350073,17.92529874)(459.20850074,17.8702988)(459.19850342,17.82030273)
\curveto(459.17850077,17.71029896)(459.15350079,17.60029907)(459.12350342,17.49030273)
\curveto(459.10350084,17.39029928)(459.09850085,17.29529937)(459.10850342,17.20530273)
\curveto(459.10850084,17.17529949)(459.10350084,17.12529954)(459.09350342,17.05530273)
\lineto(459.09350342,16.84530273)
\curveto(459.09350085,16.77529989)(459.09850085,16.70529996)(459.10850342,16.63530273)
\curveto(459.1485008,16.28530038)(459.23850071,15.98530068)(459.37850342,15.73530273)
\curveto(459.51850043,15.48530118)(459.71850023,15.28030139)(459.97850342,15.12030273)
\curveto(460.05849989,15.0703016)(460.13849981,15.03030164)(460.21850342,15.00030273)
\curveto(460.30849964,14.9703017)(460.40349954,14.94030173)(460.50350342,14.91030273)
\curveto(460.55349939,14.89030178)(460.60349934,14.88530178)(460.65350342,14.89530273)
\curveto(460.71349923,14.90530176)(460.76849918,14.90030177)(460.81850342,14.88030273)
\curveto(460.8484991,14.8703018)(460.88349906,14.8653018)(460.92350342,14.86530273)
\lineto(461.05850342,14.86530273)
\lineto(461.19350342,14.86530273)
\curveto(461.23349871,14.87530179)(461.28849866,14.88030179)(461.35850342,14.88030273)
\curveto(461.43849851,14.90030177)(461.51849843,14.91530175)(461.59850342,14.92530273)
\curveto(461.68849826,14.94530172)(461.76849818,14.9703017)(461.83850342,15.00030273)
\curveto(462.19849775,15.14030153)(462.50349744,15.31530135)(462.75350342,15.52530273)
\curveto(463.00349694,15.74530092)(463.22849672,16.02030065)(463.42850342,16.35030273)
\curveto(463.49849645,16.46030021)(463.55349639,16.5703001)(463.59350342,16.68030273)
\lineto(463.74350342,17.01030273)
\curveto(463.77349617,17.05029962)(463.78849616,17.08529958)(463.78850342,17.11530273)
\curveto(463.79849615,17.15529951)(463.81349613,17.19529947)(463.83350342,17.23530273)
\curveto(463.85349609,17.29529937)(463.86849608,17.35529931)(463.87850342,17.41530273)
\curveto(463.88849606,17.47529919)(463.90349604,17.53529913)(463.92350342,17.59530273)
}
}
{
\newrgbcolor{curcolor}{0 0 0}
\pscustom[linestyle=none,fillstyle=solid,fillcolor=curcolor]
{
\newpath
\moveto(474.31475342,18.15030273)
\curveto(474.32474453,18.09029858)(474.31474454,17.99529867)(474.28475342,17.86530273)
\curveto(474.26474459,17.74529892)(474.24474461,17.66029901)(474.22475342,17.61030273)
\lineto(474.19475342,17.46030273)
\curveto(474.16474469,17.38029929)(474.13974471,17.30529936)(474.11975342,17.23530273)
\curveto(474.10974474,17.17529949)(474.08974476,17.10529956)(474.05975342,17.02530273)
\curveto(474.02974482,16.9652997)(474.00474485,16.90529976)(473.98475342,16.84530273)
\curveto(473.97474488,16.78529988)(473.9497449,16.72529994)(473.90975342,16.66530273)
\lineto(473.72975342,16.27530273)
\curveto(473.67974517,16.14530052)(473.61474524,16.02530064)(473.53475342,15.91530273)
\curveto(473.23474562,15.43530123)(472.87474598,15.03030164)(472.45475342,14.70030273)
\curveto(472.04474681,14.38030229)(471.56474729,14.13530253)(471.01475342,13.96530273)
\curveto(470.90474795,13.92530274)(470.78474807,13.89530277)(470.65475342,13.87530273)
\curveto(470.52474833,13.85530281)(470.38974846,13.83530283)(470.24975342,13.81530273)
\curveto(470.18974866,13.80530286)(470.12474873,13.80030287)(470.05475342,13.80030273)
\curveto(469.99474886,13.79030288)(469.93474892,13.78530288)(469.87475342,13.78530273)
\curveto(469.83474902,13.77530289)(469.77474908,13.7703029)(469.69475342,13.77030273)
\curveto(469.62474923,13.7703029)(469.57474928,13.77530289)(469.54475342,13.78530273)
\curveto(469.50474935,13.79530287)(469.46474939,13.80030287)(469.42475342,13.80030273)
\curveto(469.38474947,13.79030288)(469.3497495,13.79030288)(469.31975342,13.80030273)
\lineto(469.22975342,13.80030273)
\lineto(468.88475342,13.84530273)
\lineto(468.49475342,13.96530273)
\curveto(468.37475048,14.00530266)(468.25975059,14.05030262)(468.14975342,14.10030273)
\curveto(467.73975111,14.30030237)(467.41975143,14.56030211)(467.18975342,14.88030273)
\curveto(466.96975188,15.20030147)(466.80975204,15.59030108)(466.70975342,16.05030273)
\curveto(466.67975217,16.15030052)(466.65975219,16.25030042)(466.64975342,16.35030273)
\lineto(466.64975342,16.66530273)
\curveto(466.63975221,16.70529996)(466.63975221,16.73529993)(466.64975342,16.75530273)
\curveto(466.65975219,16.78529988)(466.66475219,16.82029985)(466.66475342,16.86030273)
\curveto(466.66475219,16.94029973)(466.66975218,17.02029965)(466.67975342,17.10030273)
\curveto(466.68975216,17.19029948)(466.69475216,17.27529939)(466.69475342,17.35530273)
\curveto(466.70475215,17.40529926)(466.70975214,17.44529922)(466.70975342,17.47530273)
\curveto(466.71975213,17.51529915)(466.72475213,17.56029911)(466.72475342,17.61030273)
\curveto(466.72475213,17.66029901)(466.73475212,17.74529892)(466.75475342,17.86530273)
\curveto(466.78475207,17.99529867)(466.81475204,18.09029858)(466.84475342,18.15030273)
\curveto(466.88475197,18.22029845)(466.90475195,18.29029838)(466.90475342,18.36030273)
\curveto(466.90475195,18.43029824)(466.92475193,18.50029817)(466.96475342,18.57030273)
\curveto(466.98475187,18.62029805)(466.99975185,18.66029801)(467.00975342,18.69030273)
\curveto(467.01975183,18.73029794)(467.03475182,18.77529789)(467.05475342,18.82530273)
\curveto(467.11475174,18.94529772)(467.16475169,19.0652976)(467.20475342,19.18530273)
\curveto(467.2547516,19.30529736)(467.31975153,19.42029725)(467.39975342,19.53030273)
\curveto(467.61975123,19.90029677)(467.86475099,20.23029644)(468.13475342,20.52030273)
\curveto(468.41475044,20.82029585)(468.72975012,21.0702956)(469.07975342,21.27030273)
\curveto(469.20974964,21.35029532)(469.34474951,21.41529525)(469.48475342,21.46530273)
\lineto(469.93475342,21.64530273)
\curveto(470.06474879,21.69529497)(470.19974865,21.72529494)(470.33975342,21.73530273)
\curveto(470.47974837,21.75529491)(470.62474823,21.78529488)(470.77475342,21.82530273)
\lineto(470.96975342,21.82530273)
\lineto(471.17975342,21.85530273)
\curveto(472.06974678,21.8652948)(472.76974608,21.68029499)(473.27975342,21.30030273)
\curveto(473.79974505,20.92029575)(474.12474473,20.42529624)(474.25475342,19.81530273)
\curveto(474.28474457,19.71529695)(474.30474455,19.61529705)(474.31475342,19.51530273)
\curveto(474.32474453,19.41529725)(474.33974451,19.31029736)(474.35975342,19.20030273)
\curveto(474.36974448,19.09029758)(474.36974448,18.9702977)(474.35975342,18.84030273)
\lineto(474.35975342,18.46530273)
\curveto(474.35974449,18.41529825)(474.3497445,18.36029831)(474.32975342,18.30030273)
\curveto(474.31974453,18.25029842)(474.31474454,18.20029847)(474.31475342,18.15030273)
\moveto(472.81475342,17.29530273)
\curveto(472.84474601,17.3652993)(472.86474599,17.44529922)(472.87475342,17.53530273)
\curveto(472.89474596,17.62529904)(472.90974594,17.71029896)(472.91975342,17.79030273)
\curveto(472.99974585,18.18029849)(473.03474582,18.51029816)(473.02475342,18.78030273)
\curveto(473.00474585,18.86029781)(472.98974586,18.94029773)(472.97975342,19.02030273)
\curveto(472.97974587,19.10029757)(472.97474588,19.17529749)(472.96475342,19.24530273)
\curveto(472.81474604,19.89529677)(472.45974639,20.34529632)(471.89975342,20.59530273)
\curveto(471.82974702,20.62529604)(471.7547471,20.64529602)(471.67475342,20.65530273)
\curveto(471.60474725,20.67529599)(471.52974732,20.69529597)(471.44975342,20.71530273)
\curveto(471.37974747,20.73529593)(471.29974755,20.74529592)(471.20975342,20.74530273)
\lineto(470.93975342,20.74530273)
\lineto(470.65475342,20.70030273)
\curveto(470.5547483,20.68029599)(470.45974839,20.65529601)(470.36975342,20.62530273)
\curveto(470.27974857,20.60529606)(470.18974866,20.57529609)(470.09975342,20.53530273)
\curveto(470.02974882,20.51529615)(469.95974889,20.48529618)(469.88975342,20.44530273)
\curveto(469.81974903,20.40529626)(469.7547491,20.3652963)(469.69475342,20.32530273)
\curveto(469.42474943,20.15529651)(469.18974966,19.95029672)(468.98975342,19.71030273)
\curveto(468.78975006,19.4702972)(468.60475025,19.19029748)(468.43475342,18.87030273)
\curveto(468.38475047,18.7702979)(468.34475051,18.665298)(468.31475342,18.55530273)
\curveto(468.28475057,18.45529821)(468.24475061,18.35029832)(468.19475342,18.24030273)
\curveto(468.18475067,18.20029847)(468.16975068,18.13529853)(468.14975342,18.04530273)
\curveto(468.12975072,18.01529865)(468.11975073,17.98029869)(468.11975342,17.94030273)
\curveto(468.11975073,17.90029877)(468.11475074,17.85529881)(468.10475342,17.80530273)
\lineto(468.04475342,17.50530273)
\curveto(468.02475083,17.40529926)(468.01475084,17.31529935)(468.01475342,17.23530273)
\lineto(468.01475342,17.05530273)
\curveto(468.01475084,16.95529971)(468.00975084,16.85529981)(467.99975342,16.75530273)
\curveto(467.99975085,16.6653)(468.00975084,16.58030009)(468.02975342,16.50030273)
\curveto(468.07975077,16.26030041)(468.1497507,16.03530063)(468.23975342,15.82530273)
\curveto(468.33975051,15.61530105)(468.47475038,15.44030123)(468.64475342,15.30030273)
\curveto(468.69475016,15.2703014)(468.73475012,15.24530142)(468.76475342,15.22530273)
\curveto(468.80475005,15.20530146)(468.84475001,15.18030149)(468.88475342,15.15030273)
\curveto(468.9547499,15.10030157)(469.03474982,15.05530161)(469.12475342,15.01530273)
\curveto(469.21474964,14.98530168)(469.30974954,14.95530171)(469.40975342,14.92530273)
\curveto(469.45974939,14.90530176)(469.50474935,14.89530177)(469.54475342,14.89530273)
\curveto(469.59474926,14.90530176)(469.64474921,14.90530176)(469.69475342,14.89530273)
\curveto(469.72474913,14.88530178)(469.78474907,14.87530179)(469.87475342,14.86530273)
\curveto(469.96474889,14.85530181)(470.03974881,14.86030181)(470.09975342,14.88030273)
\curveto(470.13974871,14.89030178)(470.17974867,14.89030178)(470.21975342,14.88030273)
\curveto(470.25974859,14.88030179)(470.29974855,14.89030178)(470.33975342,14.91030273)
\curveto(470.41974843,14.93030174)(470.49974835,14.94530172)(470.57975342,14.95530273)
\curveto(470.66974818,14.97530169)(470.7547481,15.00030167)(470.83475342,15.03030273)
\curveto(471.19474766,15.1703015)(471.50474735,15.3653013)(471.76475342,15.61530273)
\curveto(472.02474683,15.8653008)(472.25974659,16.16030051)(472.46975342,16.50030273)
\curveto(472.5497463,16.62030005)(472.60974624,16.74529992)(472.64975342,16.87530273)
\curveto(472.68974616,17.01529965)(472.74474611,17.15529951)(472.81475342,17.29530273)
}
}
{
\newrgbcolor{curcolor}{0 0 0}
\pscustom[linestyle=none,fillstyle=solid,fillcolor=curcolor]
{
\newpath
\moveto(478.98303467,21.85530273)
\curveto(479.70302901,21.8652948)(480.28802843,21.78029489)(480.73803467,21.60030273)
\curveto(481.19802752,21.43029524)(481.5180272,21.12529554)(481.69803467,20.68530273)
\curveto(481.74802697,20.57529609)(481.77802694,20.46029621)(481.78803467,20.34030273)
\curveto(481.80802691,20.23029644)(481.82302689,20.10529656)(481.83303467,19.96530273)
\curveto(481.84302687,19.89529677)(481.83302688,19.82029685)(481.80303467,19.74030273)
\curveto(481.78302693,19.670297)(481.75802696,19.61529705)(481.72803467,19.57530273)
\curveto(481.70802701,19.55529711)(481.67802704,19.53529713)(481.63803467,19.51530273)
\curveto(481.60802711,19.50529716)(481.58302713,19.49029718)(481.56303467,19.47030273)
\curveto(481.50302721,19.45029722)(481.44802727,19.44529722)(481.39803467,19.45530273)
\curveto(481.35802736,19.4652972)(481.3130274,19.4652972)(481.26303467,19.45530273)
\curveto(481.17302754,19.43529723)(481.06302765,19.43029724)(480.93303467,19.44030273)
\curveto(480.8130279,19.46029721)(480.72802799,19.48529718)(480.67803467,19.51530273)
\curveto(480.60802811,19.5652971)(480.56802815,19.63029704)(480.55803467,19.71030273)
\curveto(480.55802816,19.80029687)(480.53802818,19.88529678)(480.49803467,19.96530273)
\curveto(480.44802827,20.12529654)(480.35302836,20.2702964)(480.21303467,20.40030273)
\curveto(480.12302859,20.48029619)(480.0130287,20.54029613)(479.88303467,20.58030273)
\curveto(479.76302895,20.62029605)(479.63302908,20.66029601)(479.49303467,20.70030273)
\curveto(479.45302926,20.72029595)(479.40302931,20.72529594)(479.34303467,20.71530273)
\curveto(479.29302942,20.71529595)(479.24802947,20.72029595)(479.20803467,20.73030273)
\curveto(479.14802957,20.75029592)(479.07302964,20.76029591)(478.98303467,20.76030273)
\curveto(478.89302982,20.76029591)(478.8180299,20.75029592)(478.75803467,20.73030273)
\lineto(478.66803467,20.73030273)
\curveto(478.60803011,20.72029595)(478.55303016,20.71029596)(478.50303467,20.70030273)
\curveto(478.45303026,20.70029597)(478.40303031,20.69529597)(478.35303467,20.68530273)
\curveto(478.08303063,20.62529604)(477.84803087,20.54029613)(477.64803467,20.43030273)
\curveto(477.45803126,20.32029635)(477.30803141,20.13529653)(477.19803467,19.87530273)
\curveto(477.16803155,19.80529686)(477.15303156,19.73529693)(477.15303467,19.66530273)
\curveto(477.15303156,19.59529707)(477.15803156,19.53529713)(477.16803467,19.48530273)
\curveto(477.19803152,19.33529733)(477.24803147,19.22529744)(477.31803467,19.15530273)
\curveto(477.38803133,19.09529757)(477.48303123,19.02529764)(477.60303467,18.94530273)
\curveto(477.74303097,18.84529782)(477.90803081,18.7702979)(478.09803467,18.72030273)
\curveto(478.28803043,18.68029799)(478.47803024,18.63029804)(478.66803467,18.57030273)
\curveto(478.78802993,18.53029814)(478.90802981,18.50029817)(479.02803467,18.48030273)
\curveto(479.15802956,18.46029821)(479.28302943,18.43029824)(479.40303467,18.39030273)
\curveto(479.60302911,18.33029834)(479.79802892,18.2702984)(479.98803467,18.21030273)
\curveto(480.17802854,18.16029851)(480.36302835,18.09529857)(480.54303467,18.01530273)
\curveto(480.59302812,17.99529867)(480.63802808,17.97529869)(480.67803467,17.95530273)
\curveto(480.72802799,17.93529873)(480.77802794,17.91029876)(480.82803467,17.88030273)
\curveto(480.99802772,17.76029891)(481.14302757,17.62529904)(481.26303467,17.47530273)
\curveto(481.38302733,17.32529934)(481.47302724,17.13529953)(481.53303467,16.90530273)
\lineto(481.53303467,16.62030273)
\curveto(481.53302718,16.55030012)(481.52802719,16.47530019)(481.51803467,16.39530273)
\curveto(481.50802721,16.32530034)(481.49802722,16.24530042)(481.48803467,16.15530273)
\lineto(481.45803467,16.00530273)
\curveto(481.4180273,15.93530073)(481.38802733,15.8653008)(481.36803467,15.79530273)
\curveto(481.35802736,15.72530094)(481.33802738,15.65530101)(481.30803467,15.58530273)
\curveto(481.25802746,15.47530119)(481.20302751,15.3703013)(481.14303467,15.27030273)
\curveto(481.08302763,15.1703015)(481.0180277,15.08030159)(480.94803467,15.00030273)
\curveto(480.73802798,14.74030193)(480.49302822,14.53030214)(480.21303467,14.37030273)
\curveto(479.93302878,14.22030245)(479.62802909,14.09030258)(479.29803467,13.98030273)
\curveto(479.19802952,13.95030272)(479.09802962,13.93030274)(478.99803467,13.92030273)
\curveto(478.89802982,13.90030277)(478.80302991,13.87530279)(478.71303467,13.84530273)
\curveto(478.60303011,13.82530284)(478.49803022,13.81530285)(478.39803467,13.81530273)
\curveto(478.29803042,13.81530285)(478.19803052,13.80530286)(478.09803467,13.78530273)
\lineto(477.94803467,13.78530273)
\curveto(477.89803082,13.77530289)(477.82803089,13.7703029)(477.73803467,13.77030273)
\curveto(477.64803107,13.7703029)(477.57803114,13.77530289)(477.52803467,13.78530273)
\lineto(477.36303467,13.78530273)
\curveto(477.30303141,13.80530286)(477.23803148,13.81530285)(477.16803467,13.81530273)
\curveto(477.09803162,13.80530286)(477.04303167,13.81030286)(477.00303467,13.83030273)
\curveto(476.95303176,13.84030283)(476.88803183,13.84530282)(476.80803467,13.84530273)
\curveto(476.72803199,13.8653028)(476.65303206,13.88530278)(476.58303467,13.90530273)
\curveto(476.5130322,13.91530275)(476.43803228,13.93530273)(476.35803467,13.96530273)
\curveto(476.06803265,14.0653026)(475.82303289,14.19030248)(475.62303467,14.34030273)
\curveto(475.42303329,14.49030218)(475.26303345,14.68530198)(475.14303467,14.92530273)
\curveto(475.08303363,15.05530161)(475.03303368,15.19030148)(474.99303467,15.33030273)
\curveto(474.96303375,15.4703012)(474.94303377,15.62530104)(474.93303467,15.79530273)
\curveto(474.92303379,15.85530081)(474.92803379,15.92530074)(474.94803467,16.00530273)
\curveto(474.96803375,16.09530057)(474.99303372,16.1653005)(475.02303467,16.21530273)
\curveto(475.06303365,16.25530041)(475.12303359,16.29530037)(475.20303467,16.33530273)
\curveto(475.25303346,16.35530031)(475.32303339,16.3653003)(475.41303467,16.36530273)
\curveto(475.5130332,16.37530029)(475.60303311,16.37530029)(475.68303467,16.36530273)
\curveto(475.77303294,16.35530031)(475.85803286,16.34030033)(475.93803467,16.32030273)
\curveto(476.02803269,16.31030036)(476.08303263,16.29530037)(476.10303467,16.27530273)
\curveto(476.16303255,16.22530044)(476.19303252,16.15030052)(476.19303467,16.05030273)
\curveto(476.20303251,15.96030071)(476.22303249,15.87530079)(476.25303467,15.79530273)
\curveto(476.30303241,15.57530109)(476.40303231,15.40530126)(476.55303467,15.28530273)
\curveto(476.65303206,15.19530147)(476.77303194,15.12530154)(476.91303467,15.07530273)
\curveto(477.05303166,15.02530164)(477.20303151,14.97530169)(477.36303467,14.92530273)
\lineto(477.67803467,14.88030273)
\lineto(477.76803467,14.88030273)
\curveto(477.82803089,14.86030181)(477.9130308,14.85030182)(478.02303467,14.85030273)
\curveto(478.14303057,14.85030182)(478.24803047,14.86030181)(478.33803467,14.88030273)
\curveto(478.40803031,14.88030179)(478.46303025,14.88530178)(478.50303467,14.89530273)
\curveto(478.56303015,14.90530176)(478.62303009,14.91030176)(478.68303467,14.91030273)
\curveto(478.74302997,14.92030175)(478.79802992,14.93030174)(478.84803467,14.94030273)
\curveto(479.15802956,15.02030165)(479.40802931,15.12530154)(479.59803467,15.25530273)
\curveto(479.79802892,15.38530128)(479.96302875,15.60530106)(480.09303467,15.91530273)
\curveto(480.12302859,15.9653007)(480.13802858,16.02030065)(480.13803467,16.08030273)
\curveto(480.14802857,16.14030053)(480.14802857,16.18530048)(480.13803467,16.21530273)
\curveto(480.12802859,16.40530026)(480.08802863,16.54530012)(480.01803467,16.63530273)
\curveto(479.94802877,16.73529993)(479.85302886,16.82529984)(479.73303467,16.90530273)
\curveto(479.65302906,16.9652997)(479.55802916,17.01529965)(479.44803467,17.05530273)
\lineto(479.14803467,17.17530273)
\curveto(479.1180296,17.18529948)(479.08802963,17.19029948)(479.05803467,17.19030273)
\curveto(479.03802968,17.19029948)(479.0180297,17.20029947)(478.99803467,17.22030273)
\curveto(478.67803004,17.33029934)(478.33803038,17.41029926)(477.97803467,17.46030273)
\curveto(477.62803109,17.52029915)(477.30803141,17.61529905)(477.01803467,17.74530273)
\curveto(476.92803179,17.78529888)(476.83803188,17.82029885)(476.74803467,17.85030273)
\curveto(476.66803205,17.88029879)(476.59303212,17.92029875)(476.52303467,17.97030273)
\curveto(476.35303236,18.08029859)(476.20303251,18.20529846)(476.07303467,18.34530273)
\curveto(475.94303277,18.48529818)(475.85303286,18.66029801)(475.80303467,18.87030273)
\curveto(475.78303293,18.94029773)(475.77303294,19.01029766)(475.77303467,19.08030273)
\lineto(475.77303467,19.30530273)
\curveto(475.76303295,19.42529724)(475.77803294,19.56029711)(475.81803467,19.71030273)
\curveto(475.85803286,19.8702968)(475.89803282,20.00529666)(475.93803467,20.11530273)
\curveto(475.96803275,20.1652965)(475.98803273,20.20529646)(475.99803467,20.23530273)
\curveto(476.0180327,20.27529639)(476.04303267,20.31529635)(476.07303467,20.35530273)
\curveto(476.20303251,20.58529608)(476.36303235,20.78529588)(476.55303467,20.95530273)
\curveto(476.74303197,21.12529554)(476.95303176,21.27529539)(477.18303467,21.40530273)
\curveto(477.34303137,21.49529517)(477.5180312,21.5652951)(477.70803467,21.61530273)
\curveto(477.90803081,21.67529499)(478.1130306,21.73029494)(478.32303467,21.78030273)
\curveto(478.39303032,21.79029488)(478.45803026,21.80029487)(478.51803467,21.81030273)
\curveto(478.58803013,21.82029485)(478.66303005,21.83029484)(478.74303467,21.84030273)
\curveto(478.78302993,21.85029482)(478.82302989,21.85029482)(478.86303467,21.84030273)
\curveto(478.9130298,21.83029484)(478.95302976,21.83529483)(478.98303467,21.85530273)
}
}
{
\newrgbcolor{curcolor}{0 0 0}
\pscustom[linewidth=1,linecolor=curcolor]
{
\newpath
\moveto(95.01786,75.52252)
\lineto(732.99776,75.52252)
}
}
{
\newrgbcolor{curcolor}{0 0 0}
\pscustom[linewidth=1,linecolor=curcolor]
{
\newpath
\moveto(95.01786,149.51623)
\lineto(732.99776,149.51623)
}
}
{
\newrgbcolor{curcolor}{0 0 0}
\pscustom[linewidth=1,linecolor=curcolor]
{
\newpath
\moveto(95.01786,223.55822)
\lineto(732.99776,223.55822)
}
}
{
\newrgbcolor{curcolor}{0 0 0}
\pscustom[linewidth=1,linecolor=curcolor]
{
\newpath
\moveto(95.01786,298.62335)
\lineto(732.99776,298.62335)
}
}
{
\newrgbcolor{curcolor}{0 0 0}
\pscustom[linewidth=1,linecolor=curcolor]
{
\newpath
\moveto(95.01786,372.589017)
\lineto(732.99776,372.589017)
}
}
{
\newrgbcolor{curcolor}{0 0 0}
\pscustom[linestyle=none,fillstyle=solid,fillcolor=curcolor]
{
\newpath
\moveto(100.64285278,409.08210297)
\curveto(101.69284611,409.102092)(102.55784524,408.92209218)(103.23785278,408.54210297)
\curveto(103.91784388,408.16209294)(104.45784334,407.65709344)(104.85785278,407.02710297)
\curveto(104.96784283,406.85709424)(105.05784274,406.68209442)(105.12785278,406.50210297)
\curveto(105.1978426,406.33209477)(105.26284254,406.14209496)(105.32285278,405.93210297)
\curveto(105.34284246,405.86209524)(105.36284244,405.78209532)(105.38285278,405.69210297)
\curveto(105.4028424,405.6020955)(105.3978424,405.51709558)(105.36785278,405.43710297)
\curveto(105.34784245,405.37709572)(105.30784249,405.33709576)(105.24785278,405.31710297)
\curveto(105.1978426,405.30709579)(105.13784266,405.29209581)(105.06785278,405.27210297)
\lineto(104.94785278,405.27210297)
\curveto(104.88784291,405.25209585)(104.81784298,405.24209586)(104.73785278,405.24210297)
\curveto(104.66784313,405.25209585)(104.5978432,405.25709584)(104.52785278,405.25710297)
\curveto(104.43784336,405.25709584)(104.32784347,405.25209585)(104.19785278,405.24210297)
\lineto(103.83785278,405.24210297)
\curveto(103.71784408,405.25209585)(103.60784419,405.26209584)(103.50785278,405.27210297)
\curveto(103.40784439,405.29209581)(103.33284447,405.31709578)(103.28285278,405.34710297)
\curveto(103.2028446,405.41709568)(103.14284466,405.51209559)(103.10285278,405.63210297)
\curveto(103.06284474,405.75209535)(103.01284479,405.85709524)(102.95285278,405.94710297)
\curveto(102.76284504,406.27709482)(102.51284529,406.53709456)(102.20285278,406.72710297)
\curveto(101.99284581,406.85709424)(101.75784604,406.96209414)(101.49785278,407.04210297)
\curveto(101.32784647,407.102094)(101.11284669,407.13209397)(100.85285278,407.13210297)
\curveto(100.6028472,407.13209397)(100.38284742,407.10709399)(100.19285278,407.05710297)
\curveto(100.11284769,407.03709406)(100.03784776,407.01709408)(99.96785278,406.99710297)
\curveto(99.90784789,406.98709411)(99.84284796,406.96709413)(99.77285278,406.93710297)
\curveto(99.09284871,406.64709445)(98.61284919,406.16709493)(98.33285278,405.49710297)
\curveto(98.28284952,405.37709572)(98.23784956,405.25209585)(98.19785278,405.12210297)
\curveto(98.15784964,404.99209611)(98.11284969,404.85709624)(98.06285278,404.71710297)
\curveto(98.05284975,404.64709645)(98.04284976,404.58209652)(98.03285278,404.52210297)
\curveto(98.02284978,404.46209664)(98.01284979,404.3970967)(98.00285278,404.32710297)
\curveto(97.98284982,404.26709683)(97.97284983,404.2020969)(97.97285278,404.13210297)
\curveto(97.98284982,404.07209703)(97.97784982,404.00709709)(97.95785278,403.93710297)
\curveto(97.93784986,403.85709724)(97.92784987,403.77209733)(97.92785278,403.68210297)
\curveto(97.93784986,403.6020975)(97.94284986,403.52209758)(97.94285278,403.44210297)
\curveto(97.94284986,403.4020977)(97.93784986,403.36209774)(97.92785278,403.32210297)
\curveto(97.92784987,403.28209782)(97.93284987,403.24209786)(97.94285278,403.20210297)
\lineto(97.94285278,403.06710297)
\curveto(97.96284984,403.01709808)(97.96784983,402.96709813)(97.95785278,402.91710297)
\curveto(97.95784984,402.86709823)(97.96784983,402.81709828)(97.98785278,402.76710297)
\curveto(97.98784981,402.70709839)(97.9978498,402.62709847)(98.01785278,402.52710297)
\curveto(98.03784976,402.41709868)(98.05784974,402.31209879)(98.07785278,402.21210297)
\curveto(98.0978497,402.12209898)(98.12284968,402.03209907)(98.15285278,401.94210297)
\curveto(98.29284951,401.52209958)(98.47284933,401.16209994)(98.69285278,400.86210297)
\curveto(98.91284889,400.57210053)(99.2028486,400.33210077)(99.56285278,400.14210297)
\curveto(99.67284813,400.09210101)(99.78784801,400.04710105)(99.90785278,400.00710297)
\curveto(100.02784777,399.97710112)(100.15784764,399.94210116)(100.29785278,399.90210297)
\curveto(100.34784745,399.89210121)(100.39284741,399.88210122)(100.43285278,399.87210297)
\lineto(100.58285278,399.87210297)
\lineto(100.70285278,399.87210297)
\curveto(100.75284705,399.85210125)(100.81784698,399.84210126)(100.89785278,399.84210297)
\curveto(100.97784682,399.85210125)(101.04284676,399.86210124)(101.09285278,399.87210297)
\curveto(101.15284665,399.87210123)(101.1978466,399.87710122)(101.22785278,399.88710297)
\curveto(101.34784645,399.90710119)(101.45784634,399.92710117)(101.55785278,399.94710297)
\curveto(101.66784613,399.96710113)(101.77284603,400.0021011)(101.87285278,400.05210297)
\curveto(102.1028457,400.15210095)(102.2978455,400.28210082)(102.45785278,400.44210297)
\curveto(102.62784517,400.61210049)(102.77784502,400.8021003)(102.90785278,401.01210297)
\curveto(102.96784483,401.11209999)(103.01784478,401.22209988)(103.05785278,401.34210297)
\curveto(103.0978447,401.46209964)(103.14284466,401.58209952)(103.19285278,401.70210297)
\curveto(103.22284458,401.81209929)(103.25284455,401.91209919)(103.28285278,402.00210297)
\curveto(103.31284449,402.09209901)(103.37784442,402.16209894)(103.47785278,402.21210297)
\curveto(103.53784426,402.23209887)(103.61284419,402.24209886)(103.70285278,402.24210297)
\lineto(103.95785278,402.24210297)
\lineto(104.87285278,402.24210297)
\lineto(105.14285278,402.24210297)
\curveto(105.24284256,402.24209886)(105.32284248,402.22209888)(105.38285278,402.18210297)
\curveto(105.45284235,402.13209897)(105.48784231,402.05209905)(105.48785278,401.94210297)
\curveto(105.4978423,401.83209927)(105.48784231,401.72709937)(105.45785278,401.62710297)
\lineto(105.36785278,401.22210297)
\curveto(105.31784248,401.07210003)(105.26784253,400.92710017)(105.21785278,400.78710297)
\curveto(105.17784262,400.64710045)(105.12284268,400.51210059)(105.05285278,400.38210297)
\curveto(105.0028428,400.3021008)(104.96284284,400.22210088)(104.93285278,400.14210297)
\curveto(104.9028429,400.07210103)(104.86284294,400.0021011)(104.81285278,399.93210297)
\curveto(104.26284354,399.07210203)(103.46284434,398.47210263)(102.41285278,398.13210297)
\curveto(102.3028455,398.09210301)(102.19284561,398.06210304)(102.08285278,398.04210297)
\lineto(101.75285278,397.98210297)
\curveto(101.7028461,397.96210314)(101.65284615,397.95710314)(101.60285278,397.96710297)
\curveto(101.56284624,397.96710313)(101.51784628,397.95710314)(101.46785278,397.93710297)
\lineto(101.25785278,397.93710297)
\curveto(101.1978466,397.92710317)(101.13284667,397.91710318)(101.06285278,397.90710297)
\lineto(100.82285278,397.90710297)
\lineto(100.55285278,397.90710297)
\curveto(100.46284734,397.90710319)(100.37784742,397.91710318)(100.29785278,397.93710297)
\curveto(100.26784753,397.94710315)(100.21784758,397.95210315)(100.14785278,397.95210297)
\lineto(99.93785278,397.98210297)
\curveto(99.86784793,397.99210311)(99.79284801,398.0021031)(99.71285278,398.01210297)
\curveto(99.61284819,398.04210306)(99.51284829,398.06710303)(99.41285278,398.08710297)
\curveto(99.32284848,398.10710299)(99.22784857,398.13210297)(99.12785278,398.16210297)
\lineto(98.85785278,398.25210297)
\curveto(98.76784903,398.29210281)(98.68284912,398.33210277)(98.60285278,398.37210297)
\curveto(98.0028498,398.63210247)(97.49285031,398.97710212)(97.07285278,399.40710297)
\curveto(96.66285114,399.84710125)(96.32785147,400.36710073)(96.06785278,400.96710297)
\curveto(96.00785179,401.11709998)(95.95785184,401.26709983)(95.91785278,401.41710297)
\curveto(95.87785192,401.56709953)(95.83285197,401.72209938)(95.78285278,401.88210297)
\curveto(95.76285204,401.93209917)(95.75285205,401.97209913)(95.75285278,402.00210297)
\curveto(95.75285205,402.04209906)(95.74785205,402.08209902)(95.73785278,402.12210297)
\curveto(95.71785208,402.21209889)(95.7028521,402.30709879)(95.69285278,402.40710297)
\curveto(95.68285212,402.50709859)(95.66785213,402.6020985)(95.64785278,402.69210297)
\curveto(95.62785217,402.74209836)(95.61785218,402.78209832)(95.61785278,402.81210297)
\curveto(95.62785217,402.85209825)(95.62785217,402.89209821)(95.61785278,402.93210297)
\lineto(95.61785278,403.21710297)
\curveto(95.5978522,403.26709783)(95.58785221,403.34209776)(95.58785278,403.44210297)
\curveto(95.58785221,403.54209756)(95.5978522,403.61709748)(95.61785278,403.66710297)
\curveto(95.62785217,403.6970974)(95.62785217,403.72709737)(95.61785278,403.75710297)
\lineto(95.61785278,403.84710297)
\lineto(95.61785278,403.98210297)
\curveto(95.63785216,404.06209704)(95.64785215,404.14709695)(95.64785278,404.23710297)
\lineto(95.67785278,404.50710297)
\curveto(95.6978521,404.58709651)(95.71285209,404.66209644)(95.72285278,404.73210297)
\curveto(95.73285207,404.81209629)(95.74785205,404.89209621)(95.76785278,404.97210297)
\curveto(95.80785199,405.11209599)(95.84285196,405.24709585)(95.87285278,405.37710297)
\curveto(95.9028519,405.51709558)(95.94285186,405.64709545)(95.99285278,405.76710297)
\lineto(96.14285278,406.15710297)
\curveto(96.2028516,406.28709481)(96.26785153,406.40709469)(96.33785278,406.51710297)
\curveto(96.4978513,406.7970943)(96.66285114,407.04709405)(96.83285278,407.26710297)
\curveto(96.88285092,407.33709376)(96.93785086,407.4020937)(96.99785278,407.46210297)
\lineto(97.17785278,407.64210297)
\curveto(97.42785037,407.89209321)(97.68785011,408.102093)(97.95785278,408.27210297)
\curveto(98.23784956,408.45209265)(98.55284925,408.61209249)(98.90285278,408.75210297)
\curveto(99.02284878,408.8020923)(99.14784865,408.84209226)(99.27785278,408.87210297)
\curveto(99.40784839,408.91209219)(99.54284826,408.94709215)(99.68285278,408.97710297)
\curveto(99.74284806,408.9970921)(99.802848,409.00709209)(99.86285278,409.00710297)
\curveto(99.92284788,409.00709209)(99.97784782,409.01709208)(100.02785278,409.03710297)
\curveto(100.10784769,409.04709205)(100.18284762,409.05209205)(100.25285278,409.05210297)
\curveto(100.33284747,409.06209204)(100.41284739,409.07209203)(100.49285278,409.08210297)
\curveto(100.51284729,409.09209201)(100.53784726,409.09209201)(100.56785278,409.08210297)
\curveto(100.5978472,409.07209203)(100.62284718,409.07209203)(100.64285278,409.08210297)
}
}
{
\newrgbcolor{curcolor}{0 0 0}
\pscustom[linestyle=none,fillstyle=solid,fillcolor=curcolor]
{
\newpath
\moveto(113.95136841,398.73210297)
\curveto(113.97136056,398.62210248)(113.98136055,398.51210259)(113.98136841,398.40210297)
\curveto(113.99136054,398.29210281)(113.94136059,398.21710288)(113.83136841,398.17710297)
\curveto(113.77136076,398.14710295)(113.70136083,398.13210297)(113.62136841,398.13210297)
\lineto(113.38136841,398.13210297)
\lineto(112.57136841,398.13210297)
\lineto(112.30136841,398.13210297)
\curveto(112.22136231,398.14210296)(112.15636237,398.16710293)(112.10636841,398.20710297)
\curveto(112.03636249,398.24710285)(111.98136255,398.3021028)(111.94136841,398.37210297)
\curveto(111.91136262,398.45210265)(111.86636266,398.51710258)(111.80636841,398.56710297)
\curveto(111.78636274,398.58710251)(111.76136277,398.6021025)(111.73136841,398.61210297)
\curveto(111.70136283,398.63210247)(111.66136287,398.63710246)(111.61136841,398.62710297)
\curveto(111.56136297,398.60710249)(111.51136302,398.58210252)(111.46136841,398.55210297)
\curveto(111.42136311,398.52210258)(111.37636315,398.4971026)(111.32636841,398.47710297)
\curveto(111.27636325,398.43710266)(111.22136331,398.4021027)(111.16136841,398.37210297)
\lineto(110.98136841,398.28210297)
\curveto(110.85136368,398.22210288)(110.71636381,398.17210293)(110.57636841,398.13210297)
\curveto(110.43636409,398.102103)(110.29136424,398.06710303)(110.14136841,398.02710297)
\curveto(110.07136446,398.00710309)(110.00136453,397.9971031)(109.93136841,397.99710297)
\curveto(109.87136466,397.98710311)(109.80636472,397.97710312)(109.73636841,397.96710297)
\lineto(109.64636841,397.96710297)
\curveto(109.61636491,397.95710314)(109.58636494,397.95210315)(109.55636841,397.95210297)
\lineto(109.39136841,397.95210297)
\curveto(109.29136524,397.93210317)(109.19136534,397.93210317)(109.09136841,397.95210297)
\lineto(108.95636841,397.95210297)
\curveto(108.88636564,397.97210313)(108.81636571,397.98210312)(108.74636841,397.98210297)
\curveto(108.68636584,397.97210313)(108.6263659,397.97710312)(108.56636841,397.99710297)
\curveto(108.46636606,398.01710308)(108.37136616,398.03710306)(108.28136841,398.05710297)
\curveto(108.19136634,398.06710303)(108.10636642,398.09210301)(108.02636841,398.13210297)
\curveto(107.73636679,398.24210286)(107.48636704,398.38210272)(107.27636841,398.55210297)
\curveto(107.07636745,398.73210237)(106.91636761,398.96710213)(106.79636841,399.25710297)
\curveto(106.76636776,399.32710177)(106.73636779,399.4021017)(106.70636841,399.48210297)
\curveto(106.68636784,399.56210154)(106.66636786,399.64710145)(106.64636841,399.73710297)
\curveto(106.6263679,399.78710131)(106.61636791,399.83710126)(106.61636841,399.88710297)
\curveto(106.6263679,399.93710116)(106.6263679,399.98710111)(106.61636841,400.03710297)
\curveto(106.60636792,400.06710103)(106.59636793,400.12710097)(106.58636841,400.21710297)
\curveto(106.58636794,400.31710078)(106.59136794,400.38710071)(106.60136841,400.42710297)
\curveto(106.62136791,400.52710057)(106.6313679,400.61210049)(106.63136841,400.68210297)
\lineto(106.72136841,401.01210297)
\curveto(106.75136778,401.13209997)(106.79136774,401.23709986)(106.84136841,401.32710297)
\curveto(107.01136752,401.61709948)(107.20636732,401.83709926)(107.42636841,401.98710297)
\curveto(107.64636688,402.13709896)(107.9263666,402.26709883)(108.26636841,402.37710297)
\curveto(108.39636613,402.42709867)(108.531366,402.46209864)(108.67136841,402.48210297)
\curveto(108.81136572,402.5020986)(108.95136558,402.52709857)(109.09136841,402.55710297)
\curveto(109.17136536,402.57709852)(109.25636527,402.58709851)(109.34636841,402.58710297)
\curveto(109.43636509,402.5970985)(109.526365,402.61209849)(109.61636841,402.63210297)
\curveto(109.68636484,402.65209845)(109.75636477,402.65709844)(109.82636841,402.64710297)
\curveto(109.89636463,402.64709845)(109.97136456,402.65709844)(110.05136841,402.67710297)
\curveto(110.12136441,402.6970984)(110.19136434,402.70709839)(110.26136841,402.70710297)
\curveto(110.3313642,402.70709839)(110.40636412,402.71709838)(110.48636841,402.73710297)
\curveto(110.69636383,402.78709831)(110.88636364,402.82709827)(111.05636841,402.85710297)
\curveto(111.23636329,402.8970982)(111.39636313,402.98709811)(111.53636841,403.12710297)
\curveto(111.6263629,403.21709788)(111.68636284,403.31709778)(111.71636841,403.42710297)
\curveto(111.7263628,403.45709764)(111.7263628,403.48209762)(111.71636841,403.50210297)
\curveto(111.71636281,403.52209758)(111.72136281,403.54209756)(111.73136841,403.56210297)
\curveto(111.74136279,403.58209752)(111.74636278,403.61209749)(111.74636841,403.65210297)
\lineto(111.74636841,403.74210297)
\lineto(111.71636841,403.86210297)
\curveto(111.71636281,403.9020972)(111.71136282,403.93709716)(111.70136841,403.96710297)
\curveto(111.60136293,404.26709683)(111.39136314,404.47209663)(111.07136841,404.58210297)
\curveto(110.98136355,404.61209649)(110.87136366,404.63209647)(110.74136841,404.64210297)
\curveto(110.62136391,404.66209644)(110.49636403,404.66709643)(110.36636841,404.65710297)
\curveto(110.23636429,404.65709644)(110.11136442,404.64709645)(109.99136841,404.62710297)
\curveto(109.87136466,404.60709649)(109.76636476,404.58209652)(109.67636841,404.55210297)
\curveto(109.61636491,404.53209657)(109.55636497,404.5020966)(109.49636841,404.46210297)
\curveto(109.44636508,404.43209667)(109.39636513,404.3970967)(109.34636841,404.35710297)
\curveto(109.29636523,404.31709678)(109.24136529,404.26209684)(109.18136841,404.19210297)
\curveto(109.1313654,404.12209698)(109.09636543,404.05709704)(109.07636841,403.99710297)
\curveto(109.0263655,403.8970972)(108.98136555,403.8020973)(108.94136841,403.71210297)
\curveto(108.91136562,403.62209748)(108.84136569,403.56209754)(108.73136841,403.53210297)
\curveto(108.65136588,403.51209759)(108.56636596,403.5020976)(108.47636841,403.50210297)
\lineto(108.20636841,403.50210297)
\lineto(107.63636841,403.50210297)
\curveto(107.58636694,403.5020976)(107.53636699,403.4970976)(107.48636841,403.48710297)
\curveto(107.43636709,403.48709761)(107.39136714,403.49209761)(107.35136841,403.50210297)
\lineto(107.21636841,403.50210297)
\curveto(107.19636733,403.51209759)(107.17136736,403.51709758)(107.14136841,403.51710297)
\curveto(107.11136742,403.51709758)(107.08636744,403.52709757)(107.06636841,403.54710297)
\curveto(106.98636754,403.56709753)(106.9313676,403.63209747)(106.90136841,403.74210297)
\curveto(106.89136764,403.79209731)(106.89136764,403.84209726)(106.90136841,403.89210297)
\curveto(106.91136762,403.94209716)(106.92136761,403.98709711)(106.93136841,404.02710297)
\curveto(106.96136757,404.13709696)(106.99136754,404.23709686)(107.02136841,404.32710297)
\curveto(107.06136747,404.42709667)(107.10636742,404.51709658)(107.15636841,404.59710297)
\lineto(107.24636841,404.74710297)
\lineto(107.33636841,404.89710297)
\curveto(107.41636711,405.00709609)(107.51636701,405.11209599)(107.63636841,405.21210297)
\curveto(107.65636687,405.22209588)(107.68636684,405.24709585)(107.72636841,405.28710297)
\curveto(107.77636675,405.32709577)(107.82136671,405.36209574)(107.86136841,405.39210297)
\curveto(107.90136663,405.42209568)(107.94636658,405.45209565)(107.99636841,405.48210297)
\curveto(108.16636636,405.59209551)(108.34636618,405.67709542)(108.53636841,405.73710297)
\curveto(108.7263658,405.80709529)(108.92136561,405.87209523)(109.12136841,405.93210297)
\curveto(109.24136529,405.96209514)(109.36636516,405.98209512)(109.49636841,405.99210297)
\curveto(109.6263649,406.0020951)(109.75636477,406.02209508)(109.88636841,406.05210297)
\curveto(109.9263646,406.06209504)(109.98636454,406.06209504)(110.06636841,406.05210297)
\curveto(110.15636437,406.04209506)(110.21136432,406.04709505)(110.23136841,406.06710297)
\curveto(110.64136389,406.07709502)(111.0313635,406.06209504)(111.40136841,406.02210297)
\curveto(111.78136275,405.98209512)(112.12136241,405.90709519)(112.42136841,405.79710297)
\curveto(112.7313618,405.68709541)(112.99636153,405.53709556)(113.21636841,405.34710297)
\curveto(113.43636109,405.16709593)(113.60636092,404.93209617)(113.72636841,404.64210297)
\curveto(113.79636073,404.47209663)(113.83636069,404.27709682)(113.84636841,404.05710297)
\curveto(113.85636067,403.83709726)(113.86136067,403.61209749)(113.86136841,403.38210297)
\lineto(113.86136841,400.03710297)
\lineto(113.86136841,399.45210297)
\curveto(113.86136067,399.26210184)(113.88136065,399.08710201)(113.92136841,398.92710297)
\curveto(113.9313606,398.8971022)(113.93636059,398.86210224)(113.93636841,398.82210297)
\curveto(113.93636059,398.79210231)(113.94136059,398.76210234)(113.95136841,398.73210297)
\moveto(111.74636841,401.04210297)
\curveto(111.75636277,401.09210001)(111.76136277,401.14709995)(111.76136841,401.20710297)
\curveto(111.76136277,401.27709982)(111.75636277,401.33709976)(111.74636841,401.38710297)
\curveto(111.7263628,401.44709965)(111.71636281,401.5020996)(111.71636841,401.55210297)
\curveto(111.71636281,401.6020995)(111.69636283,401.64209946)(111.65636841,401.67210297)
\curveto(111.60636292,401.71209939)(111.531363,401.73209937)(111.43136841,401.73210297)
\curveto(111.39136314,401.72209938)(111.35636317,401.71209939)(111.32636841,401.70210297)
\curveto(111.29636323,401.7020994)(111.26136327,401.6970994)(111.22136841,401.68710297)
\curveto(111.15136338,401.66709943)(111.07636345,401.65209945)(110.99636841,401.64210297)
\curveto(110.91636361,401.63209947)(110.83636369,401.61709948)(110.75636841,401.59710297)
\curveto(110.7263638,401.58709951)(110.68136385,401.58209952)(110.62136841,401.58210297)
\curveto(110.49136404,401.55209955)(110.36136417,401.53209957)(110.23136841,401.52210297)
\curveto(110.10136443,401.51209959)(109.97636455,401.48709961)(109.85636841,401.44710297)
\curveto(109.77636475,401.42709967)(109.70136483,401.40709969)(109.63136841,401.38710297)
\curveto(109.56136497,401.37709972)(109.49136504,401.35709974)(109.42136841,401.32710297)
\curveto(109.21136532,401.23709986)(109.0313655,401.1021)(108.88136841,400.92210297)
\curveto(108.74136579,400.74210036)(108.69136584,400.49210061)(108.73136841,400.17210297)
\curveto(108.75136578,400.0021011)(108.80636572,399.86210124)(108.89636841,399.75210297)
\curveto(108.96636556,399.64210146)(109.07136546,399.55210155)(109.21136841,399.48210297)
\curveto(109.35136518,399.42210168)(109.50136503,399.37710172)(109.66136841,399.34710297)
\curveto(109.8313647,399.31710178)(110.00636452,399.30710179)(110.18636841,399.31710297)
\curveto(110.37636415,399.33710176)(110.55136398,399.37210173)(110.71136841,399.42210297)
\curveto(110.97136356,399.5021016)(111.17636335,399.62710147)(111.32636841,399.79710297)
\curveto(111.47636305,399.97710112)(111.59136294,400.1971009)(111.67136841,400.45710297)
\curveto(111.69136284,400.52710057)(111.70136283,400.5971005)(111.70136841,400.66710297)
\curveto(111.71136282,400.74710035)(111.7263628,400.82710027)(111.74636841,400.90710297)
\lineto(111.74636841,401.04210297)
}
}
{
\newrgbcolor{curcolor}{0 0 0}
\pscustom[linestyle=none,fillstyle=solid,fillcolor=curcolor]
{
\newpath
\moveto(119.93964966,406.06710297)
\curveto(120.53964385,406.08709501)(121.03964335,406.0020951)(121.43964966,405.81210297)
\curveto(121.83964255,405.62209548)(122.15464224,405.34209576)(122.38464966,404.97210297)
\curveto(122.45464194,404.86209624)(122.50964188,404.74209636)(122.54964966,404.61210297)
\curveto(122.5896418,404.49209661)(122.62964176,404.36709673)(122.66964966,404.23710297)
\curveto(122.6896417,404.15709694)(122.69964169,404.08209702)(122.69964966,404.01210297)
\curveto(122.70964168,403.94209716)(122.72464167,403.87209723)(122.74464966,403.80210297)
\curveto(122.74464165,403.74209736)(122.74964164,403.7020974)(122.75964966,403.68210297)
\curveto(122.77964161,403.54209756)(122.7896416,403.3970977)(122.78964966,403.24710297)
\lineto(122.78964966,402.81210297)
\lineto(122.78964966,401.47710297)
\lineto(122.78964966,399.04710297)
\curveto(122.7896416,398.85710224)(122.78464161,398.67210243)(122.77464966,398.49210297)
\curveto(122.77464162,398.32210278)(122.70464169,398.21210289)(122.56464966,398.16210297)
\curveto(122.50464189,398.14210296)(122.43464196,398.13210297)(122.35464966,398.13210297)
\lineto(122.11464966,398.13210297)
\lineto(121.30464966,398.13210297)
\curveto(121.18464321,398.13210297)(121.07464332,398.13710296)(120.97464966,398.14710297)
\curveto(120.88464351,398.16710293)(120.81464358,398.21210289)(120.76464966,398.28210297)
\curveto(120.72464367,398.34210276)(120.69964369,398.41710268)(120.68964966,398.50710297)
\lineto(120.68964966,398.82210297)
\lineto(120.68964966,399.87210297)
\lineto(120.68964966,402.10710297)
\curveto(120.6896437,402.47709862)(120.67464372,402.81709828)(120.64464966,403.12710297)
\curveto(120.61464378,403.44709765)(120.52464387,403.71709738)(120.37464966,403.93710297)
\curveto(120.23464416,404.13709696)(120.02964436,404.27709682)(119.75964966,404.35710297)
\curveto(119.70964468,404.37709672)(119.65464474,404.38709671)(119.59464966,404.38710297)
\curveto(119.54464485,404.38709671)(119.4896449,404.3970967)(119.42964966,404.41710297)
\curveto(119.37964501,404.42709667)(119.31464508,404.42709667)(119.23464966,404.41710297)
\curveto(119.16464523,404.41709668)(119.10964528,404.41209669)(119.06964966,404.40210297)
\curveto(119.02964536,404.39209671)(118.9946454,404.38709671)(118.96464966,404.38710297)
\curveto(118.93464546,404.38709671)(118.90464549,404.38209672)(118.87464966,404.37210297)
\curveto(118.64464575,404.31209679)(118.45964593,404.23209687)(118.31964966,404.13210297)
\curveto(117.99964639,403.9020972)(117.80964658,403.56709753)(117.74964966,403.12710297)
\curveto(117.6896467,402.68709841)(117.65964673,402.19209891)(117.65964966,401.64210297)
\lineto(117.65964966,399.76710297)
\lineto(117.65964966,398.85210297)
\lineto(117.65964966,398.58210297)
\curveto(117.65964673,398.49210261)(117.64464675,398.41710268)(117.61464966,398.35710297)
\curveto(117.56464683,398.24710285)(117.48464691,398.18210292)(117.37464966,398.16210297)
\curveto(117.26464713,398.14210296)(117.12964726,398.13210297)(116.96964966,398.13210297)
\lineto(116.21964966,398.13210297)
\curveto(116.10964828,398.13210297)(115.99964839,398.13710296)(115.88964966,398.14710297)
\curveto(115.77964861,398.15710294)(115.69964869,398.19210291)(115.64964966,398.25210297)
\curveto(115.57964881,398.34210276)(115.54464885,398.47210263)(115.54464966,398.64210297)
\curveto(115.55464884,398.81210229)(115.55964883,398.97210213)(115.55964966,399.12210297)
\lineto(115.55964966,401.16210297)
\lineto(115.55964966,404.46210297)
\lineto(115.55964966,405.22710297)
\lineto(115.55964966,405.52710297)
\curveto(115.56964882,405.61709548)(115.59964879,405.69209541)(115.64964966,405.75210297)
\curveto(115.66964872,405.78209532)(115.69964869,405.8020953)(115.73964966,405.81210297)
\curveto(115.7896486,405.83209527)(115.83964855,405.84709525)(115.88964966,405.85710297)
\lineto(115.96464966,405.85710297)
\curveto(116.01464838,405.86709523)(116.06464833,405.87209523)(116.11464966,405.87210297)
\lineto(116.27964966,405.87210297)
\lineto(116.90964966,405.87210297)
\curveto(116.9896474,405.87209523)(117.06464733,405.86709523)(117.13464966,405.85710297)
\curveto(117.21464718,405.85709524)(117.28464711,405.84709525)(117.34464966,405.82710297)
\curveto(117.41464698,405.7970953)(117.45964693,405.75209535)(117.47964966,405.69210297)
\curveto(117.50964688,405.63209547)(117.53464686,405.56209554)(117.55464966,405.48210297)
\curveto(117.56464683,405.44209566)(117.56464683,405.40709569)(117.55464966,405.37710297)
\curveto(117.55464684,405.34709575)(117.56464683,405.31709578)(117.58464966,405.28710297)
\curveto(117.60464679,405.23709586)(117.61964677,405.20709589)(117.62964966,405.19710297)
\curveto(117.64964674,405.18709591)(117.67464672,405.17209593)(117.70464966,405.15210297)
\curveto(117.81464658,405.14209596)(117.90464649,405.17709592)(117.97464966,405.25710297)
\curveto(118.04464635,405.34709575)(118.11964627,405.41709568)(118.19964966,405.46710297)
\curveto(118.46964592,405.66709543)(118.76964562,405.82709527)(119.09964966,405.94710297)
\curveto(119.1896452,405.97709512)(119.27964511,405.9970951)(119.36964966,406.00710297)
\curveto(119.46964492,406.01709508)(119.57464482,406.03209507)(119.68464966,406.05210297)
\curveto(119.71464468,406.06209504)(119.75964463,406.06209504)(119.81964966,406.05210297)
\curveto(119.87964451,406.05209505)(119.91964447,406.05709504)(119.93964966,406.06710297)
}
}
{
\newrgbcolor{curcolor}{0 0 0}
\pscustom[linestyle=none,fillstyle=solid,fillcolor=curcolor]
{
\newpath
\moveto(125.47089966,408.18210297)
\lineto(126.47589966,408.18210297)
\curveto(126.62589667,408.18209292)(126.75589654,408.17209293)(126.86589966,408.15210297)
\curveto(126.98589631,408.14209296)(127.07089623,408.08209302)(127.12089966,407.97210297)
\curveto(127.14089616,407.92209318)(127.15089615,407.86209324)(127.15089966,407.79210297)
\lineto(127.15089966,407.58210297)
\lineto(127.15089966,406.90710297)
\curveto(127.15089615,406.85709424)(127.14589615,406.7970943)(127.13589966,406.72710297)
\curveto(127.13589616,406.66709443)(127.14089616,406.61209449)(127.15089966,406.56210297)
\lineto(127.15089966,406.39710297)
\curveto(127.15089615,406.31709478)(127.15589614,406.24209486)(127.16589966,406.17210297)
\curveto(127.17589612,406.11209499)(127.2008961,406.05709504)(127.24089966,406.00710297)
\curveto(127.31089599,405.91709518)(127.43589586,405.86709523)(127.61589966,405.85710297)
\lineto(128.15589966,405.85710297)
\lineto(128.33589966,405.85710297)
\curveto(128.3958949,405.85709524)(128.45089485,405.84709525)(128.50089966,405.82710297)
\curveto(128.61089469,405.77709532)(128.67089463,405.68709541)(128.68089966,405.55710297)
\curveto(128.7008946,405.42709567)(128.71089459,405.28209582)(128.71089966,405.12210297)
\lineto(128.71089966,404.91210297)
\curveto(128.72089458,404.84209626)(128.71589458,404.78209632)(128.69589966,404.73210297)
\curveto(128.64589465,404.57209653)(128.54089476,404.48709661)(128.38089966,404.47710297)
\curveto(128.22089508,404.46709663)(128.04089526,404.46209664)(127.84089966,404.46210297)
\lineto(127.70589966,404.46210297)
\curveto(127.66589563,404.47209663)(127.63089567,404.47209663)(127.60089966,404.46210297)
\curveto(127.56089574,404.45209665)(127.52589577,404.44709665)(127.49589966,404.44710297)
\curveto(127.46589583,404.45709664)(127.43589586,404.45209665)(127.40589966,404.43210297)
\curveto(127.32589597,404.41209669)(127.26589603,404.36709673)(127.22589966,404.29710297)
\curveto(127.1958961,404.23709686)(127.17089613,404.16209694)(127.15089966,404.07210297)
\curveto(127.14089616,404.02209708)(127.14089616,403.96709713)(127.15089966,403.90710297)
\curveto(127.16089614,403.84709725)(127.16089614,403.79209731)(127.15089966,403.74210297)
\lineto(127.15089966,402.81210297)
\lineto(127.15089966,401.05710297)
\curveto(127.15089615,400.80710029)(127.15589614,400.58710051)(127.16589966,400.39710297)
\curveto(127.18589611,400.21710088)(127.25089605,400.05710104)(127.36089966,399.91710297)
\curveto(127.41089589,399.85710124)(127.47589582,399.81210129)(127.55589966,399.78210297)
\lineto(127.82589966,399.72210297)
\curveto(127.85589544,399.71210139)(127.88589541,399.70710139)(127.91589966,399.70710297)
\curveto(127.95589534,399.71710138)(127.98589531,399.71710138)(128.00589966,399.70710297)
\lineto(128.17089966,399.70710297)
\curveto(128.28089502,399.70710139)(128.37589492,399.7021014)(128.45589966,399.69210297)
\curveto(128.53589476,399.68210142)(128.6008947,399.64210146)(128.65089966,399.57210297)
\curveto(128.69089461,399.51210159)(128.71089459,399.43210167)(128.71089966,399.33210297)
\lineto(128.71089966,399.04710297)
\curveto(128.71089459,398.83710226)(128.70589459,398.64210246)(128.69589966,398.46210297)
\curveto(128.6958946,398.29210281)(128.61589468,398.17710292)(128.45589966,398.11710297)
\curveto(128.40589489,398.097103)(128.36089494,398.09210301)(128.32089966,398.10210297)
\curveto(128.28089502,398.102103)(128.23589506,398.09210301)(128.18589966,398.07210297)
\lineto(128.03589966,398.07210297)
\curveto(128.01589528,398.07210303)(127.98589531,398.07710302)(127.94589966,398.08710297)
\curveto(127.90589539,398.08710301)(127.87089543,398.08210302)(127.84089966,398.07210297)
\curveto(127.79089551,398.06210304)(127.73589556,398.06210304)(127.67589966,398.07210297)
\lineto(127.52589966,398.07210297)
\lineto(127.37589966,398.07210297)
\curveto(127.32589597,398.06210304)(127.28089602,398.06210304)(127.24089966,398.07210297)
\lineto(127.07589966,398.07210297)
\curveto(127.02589627,398.08210302)(126.97089633,398.08710301)(126.91089966,398.08710297)
\curveto(126.85089645,398.08710301)(126.7958965,398.09210301)(126.74589966,398.10210297)
\curveto(126.67589662,398.11210299)(126.61089669,398.12210298)(126.55089966,398.13210297)
\lineto(126.37089966,398.16210297)
\curveto(126.26089704,398.19210291)(126.15589714,398.22710287)(126.05589966,398.26710297)
\curveto(125.95589734,398.30710279)(125.86089744,398.35210275)(125.77089966,398.40210297)
\lineto(125.68089966,398.46210297)
\curveto(125.65089765,398.49210261)(125.61589768,398.52210258)(125.57589966,398.55210297)
\curveto(125.55589774,398.57210253)(125.53089777,398.59210251)(125.50089966,398.61210297)
\lineto(125.42589966,398.68710297)
\curveto(125.28589801,398.87710222)(125.18089812,399.08710201)(125.11089966,399.31710297)
\curveto(125.09089821,399.35710174)(125.08089822,399.39210171)(125.08089966,399.42210297)
\curveto(125.09089821,399.46210164)(125.09089821,399.50710159)(125.08089966,399.55710297)
\curveto(125.07089823,399.57710152)(125.06589823,399.6021015)(125.06589966,399.63210297)
\curveto(125.06589823,399.66210144)(125.06089824,399.68710141)(125.05089966,399.70710297)
\lineto(125.05089966,399.85710297)
\curveto(125.04089826,399.8971012)(125.03589826,399.94210116)(125.03589966,399.99210297)
\curveto(125.04589825,400.04210106)(125.05089825,400.09210101)(125.05089966,400.14210297)
\lineto(125.05089966,400.71210297)
\lineto(125.05089966,402.94710297)
\lineto(125.05089966,403.74210297)
\lineto(125.05089966,403.95210297)
\curveto(125.06089824,404.02209708)(125.05589824,404.08709701)(125.03589966,404.14710297)
\curveto(124.9958983,404.28709681)(124.92589837,404.37709672)(124.82589966,404.41710297)
\curveto(124.71589858,404.46709663)(124.57589872,404.48209662)(124.40589966,404.46210297)
\curveto(124.23589906,404.44209666)(124.09089921,404.45709664)(123.97089966,404.50710297)
\curveto(123.89089941,404.53709656)(123.84089946,404.58209652)(123.82089966,404.64210297)
\curveto(123.8008995,404.7020964)(123.78089952,404.77709632)(123.76089966,404.86710297)
\lineto(123.76089966,405.18210297)
\curveto(123.76089954,405.36209574)(123.77089953,405.50709559)(123.79089966,405.61710297)
\curveto(123.81089949,405.72709537)(123.8958994,405.8020953)(124.04589966,405.84210297)
\curveto(124.08589921,405.86209524)(124.12589917,405.86709523)(124.16589966,405.85710297)
\lineto(124.30089966,405.85710297)
\curveto(124.45089885,405.85709524)(124.59089871,405.86209524)(124.72089966,405.87210297)
\curveto(124.85089845,405.89209521)(124.94089836,405.95209515)(124.99089966,406.05210297)
\curveto(125.02089828,406.12209498)(125.03589826,406.2020949)(125.03589966,406.29210297)
\curveto(125.04589825,406.38209472)(125.05089825,406.47209463)(125.05089966,406.56210297)
\lineto(125.05089966,407.49210297)
\lineto(125.05089966,407.74710297)
\curveto(125.05089825,407.83709326)(125.06089824,407.91209319)(125.08089966,407.97210297)
\curveto(125.13089817,408.07209303)(125.20589809,408.13709296)(125.30589966,408.16710297)
\curveto(125.32589797,408.17709292)(125.35089795,408.17709292)(125.38089966,408.16710297)
\curveto(125.42089788,408.16709293)(125.45089785,408.17209293)(125.47089966,408.18210297)
}
}
{
\newrgbcolor{curcolor}{0 0 0}
\pscustom[linestyle=none,fillstyle=solid,fillcolor=curcolor]
{
\newpath
\moveto(131.79433716,408.72210297)
\curveto(131.86433421,408.64209246)(131.89933417,408.52209258)(131.89933716,408.36210297)
\lineto(131.89933716,407.89710297)
\lineto(131.89933716,407.49210297)
\curveto(131.89933417,407.35209375)(131.86433421,407.25709384)(131.79433716,407.20710297)
\curveto(131.73433434,407.15709394)(131.65433442,407.12709397)(131.55433716,407.11710297)
\curveto(131.46433461,407.10709399)(131.36433471,407.102094)(131.25433716,407.10210297)
\lineto(130.41433716,407.10210297)
\curveto(130.30433577,407.102094)(130.20433587,407.10709399)(130.11433716,407.11710297)
\curveto(130.03433604,407.12709397)(129.96433611,407.15709394)(129.90433716,407.20710297)
\curveto(129.86433621,407.23709386)(129.83433624,407.29209381)(129.81433716,407.37210297)
\curveto(129.80433627,407.46209364)(129.79433628,407.55709354)(129.78433716,407.65710297)
\lineto(129.78433716,407.98710297)
\curveto(129.79433628,408.097093)(129.79933627,408.19209291)(129.79933716,408.27210297)
\lineto(129.79933716,408.48210297)
\curveto(129.80933626,408.55209255)(129.82933624,408.61209249)(129.85933716,408.66210297)
\curveto(129.87933619,408.7020924)(129.90433617,408.73209237)(129.93433716,408.75210297)
\lineto(130.05433716,408.81210297)
\curveto(130.074336,408.81209229)(130.09933597,408.81209229)(130.12933716,408.81210297)
\curveto(130.15933591,408.82209228)(130.18433589,408.82709227)(130.20433716,408.82710297)
\lineto(131.29933716,408.82710297)
\curveto(131.39933467,408.82709227)(131.49433458,408.82209228)(131.58433716,408.81210297)
\curveto(131.6743344,408.8020923)(131.74433433,408.77209233)(131.79433716,408.72210297)
\moveto(131.89933716,398.95710297)
\curveto(131.89933417,398.75710234)(131.89433418,398.58710251)(131.88433716,398.44710297)
\curveto(131.8743342,398.30710279)(131.78433429,398.21210289)(131.61433716,398.16210297)
\curveto(131.55433452,398.14210296)(131.48933458,398.13210297)(131.41933716,398.13210297)
\curveto(131.34933472,398.14210296)(131.2743348,398.14710295)(131.19433716,398.14710297)
\lineto(130.35433716,398.14710297)
\curveto(130.26433581,398.14710295)(130.1743359,398.15210295)(130.08433716,398.16210297)
\curveto(130.00433607,398.17210293)(129.94433613,398.2021029)(129.90433716,398.25210297)
\curveto(129.84433623,398.32210278)(129.80933626,398.40710269)(129.79933716,398.50710297)
\lineto(129.79933716,398.85210297)
\lineto(129.79933716,405.18210297)
\lineto(129.79933716,405.48210297)
\curveto(129.79933627,405.58209552)(129.81933625,405.66209544)(129.85933716,405.72210297)
\curveto(129.91933615,405.79209531)(130.00433607,405.83709526)(130.11433716,405.85710297)
\curveto(130.13433594,405.86709523)(130.15933591,405.86709523)(130.18933716,405.85710297)
\curveto(130.22933584,405.85709524)(130.25933581,405.86209524)(130.27933716,405.87210297)
\lineto(131.02933716,405.87210297)
\lineto(131.22433716,405.87210297)
\curveto(131.30433477,405.88209522)(131.3693347,405.88209522)(131.41933716,405.87210297)
\lineto(131.53933716,405.87210297)
\curveto(131.59933447,405.85209525)(131.65433442,405.83709526)(131.70433716,405.82710297)
\curveto(131.75433432,405.81709528)(131.79433428,405.78709531)(131.82433716,405.73710297)
\curveto(131.86433421,405.68709541)(131.88433419,405.61709548)(131.88433716,405.52710297)
\curveto(131.89433418,405.43709566)(131.89933417,405.34209576)(131.89933716,405.24210297)
\lineto(131.89933716,398.95710297)
}
}
{
\newrgbcolor{curcolor}{0 0 0}
\pscustom[linestyle=none,fillstyle=solid,fillcolor=curcolor]
{
\newpath
\moveto(141.15152466,398.98710297)
\lineto(141.15152466,398.56710297)
\curveto(141.15151629,398.43710266)(141.12151632,398.33210277)(141.06152466,398.25210297)
\curveto(141.01151643,398.2021029)(140.94651649,398.16710293)(140.86652466,398.14710297)
\curveto(140.78651665,398.13710296)(140.69651674,398.13210297)(140.59652466,398.13210297)
\lineto(139.77152466,398.13210297)
\lineto(139.48652466,398.13210297)
\curveto(139.40651803,398.14210296)(139.3415181,398.16710293)(139.29152466,398.20710297)
\curveto(139.22151822,398.25710284)(139.18151826,398.32210278)(139.17152466,398.40210297)
\curveto(139.16151828,398.48210262)(139.1415183,398.56210254)(139.11152466,398.64210297)
\curveto(139.09151835,398.66210244)(139.07151837,398.67710242)(139.05152466,398.68710297)
\curveto(139.0415184,398.70710239)(139.02651841,398.72710237)(139.00652466,398.74710297)
\curveto(138.89651854,398.74710235)(138.81651862,398.72210238)(138.76652466,398.67210297)
\lineto(138.61652466,398.52210297)
\curveto(138.54651889,398.47210263)(138.48151896,398.42710267)(138.42152466,398.38710297)
\curveto(138.36151908,398.35710274)(138.29651914,398.31710278)(138.22652466,398.26710297)
\curveto(138.18651925,398.24710285)(138.1415193,398.22710287)(138.09152466,398.20710297)
\curveto(138.05151939,398.18710291)(138.00651943,398.16710293)(137.95652466,398.14710297)
\curveto(137.81651962,398.097103)(137.66651977,398.05210305)(137.50652466,398.01210297)
\curveto(137.45651998,397.99210311)(137.41152003,397.98210312)(137.37152466,397.98210297)
\curveto(137.33152011,397.98210312)(137.29152015,397.97710312)(137.25152466,397.96710297)
\lineto(137.11652466,397.96710297)
\curveto(137.08652035,397.95710314)(137.04652039,397.95210315)(136.99652466,397.95210297)
\lineto(136.86152466,397.95210297)
\curveto(136.80152064,397.93210317)(136.71152073,397.92710317)(136.59152466,397.93710297)
\curveto(136.47152097,397.93710316)(136.38652105,397.94710315)(136.33652466,397.96710297)
\curveto(136.26652117,397.98710311)(136.20152124,397.9971031)(136.14152466,397.99710297)
\curveto(136.09152135,397.98710311)(136.0365214,397.99210311)(135.97652466,398.01210297)
\lineto(135.61652466,398.13210297)
\curveto(135.50652193,398.16210294)(135.39652204,398.2021029)(135.28652466,398.25210297)
\curveto(134.9365225,398.4021027)(134.62152282,398.63210247)(134.34152466,398.94210297)
\curveto(134.07152337,399.26210184)(133.85652358,399.5971015)(133.69652466,399.94710297)
\curveto(133.64652379,400.05710104)(133.60652383,400.16210094)(133.57652466,400.26210297)
\curveto(133.54652389,400.37210073)(133.51152393,400.48210062)(133.47152466,400.59210297)
\curveto(133.46152398,400.63210047)(133.45652398,400.66710043)(133.45652466,400.69710297)
\curveto(133.45652398,400.73710036)(133.44652399,400.78210032)(133.42652466,400.83210297)
\curveto(133.40652403,400.91210019)(133.38652405,400.9971001)(133.36652466,401.08710297)
\curveto(133.35652408,401.18709991)(133.3415241,401.28709981)(133.32152466,401.38710297)
\curveto(133.31152413,401.41709968)(133.30652413,401.45209965)(133.30652466,401.49210297)
\curveto(133.31652412,401.53209957)(133.31652412,401.56709953)(133.30652466,401.59710297)
\lineto(133.30652466,401.73210297)
\curveto(133.30652413,401.78209932)(133.30152414,401.83209927)(133.29152466,401.88210297)
\curveto(133.28152416,401.93209917)(133.27652416,401.98709911)(133.27652466,402.04710297)
\curveto(133.27652416,402.11709898)(133.28152416,402.17209893)(133.29152466,402.21210297)
\curveto(133.30152414,402.26209884)(133.30652413,402.30709879)(133.30652466,402.34710297)
\lineto(133.30652466,402.49710297)
\curveto(133.31652412,402.54709855)(133.31652412,402.59209851)(133.30652466,402.63210297)
\curveto(133.30652413,402.68209842)(133.31652412,402.73209837)(133.33652466,402.78210297)
\curveto(133.35652408,402.89209821)(133.37152407,402.9970981)(133.38152466,403.09710297)
\curveto(133.40152404,403.1970979)(133.42652401,403.2970978)(133.45652466,403.39710297)
\curveto(133.49652394,403.51709758)(133.53152391,403.63209747)(133.56152466,403.74210297)
\curveto(133.59152385,403.85209725)(133.63152381,403.96209714)(133.68152466,404.07210297)
\curveto(133.82152362,404.37209673)(133.99652344,404.65709644)(134.20652466,404.92710297)
\curveto(134.22652321,404.95709614)(134.25152319,404.98209612)(134.28152466,405.00210297)
\curveto(134.32152312,405.03209607)(134.35152309,405.06209604)(134.37152466,405.09210297)
\curveto(134.41152303,405.14209596)(134.45152299,405.18709591)(134.49152466,405.22710297)
\curveto(134.53152291,405.26709583)(134.57652286,405.30709579)(134.62652466,405.34710297)
\curveto(134.66652277,405.36709573)(134.70152274,405.39209571)(134.73152466,405.42210297)
\curveto(134.76152268,405.46209564)(134.79652264,405.49209561)(134.83652466,405.51210297)
\curveto(135.08652235,405.68209542)(135.37652206,405.82209528)(135.70652466,405.93210297)
\curveto(135.77652166,405.95209515)(135.84652159,405.96709513)(135.91652466,405.97710297)
\curveto(135.99652144,405.98709511)(136.07652136,406.0020951)(136.15652466,406.02210297)
\curveto(136.22652121,406.04209506)(136.31652112,406.05209505)(136.42652466,406.05210297)
\curveto(136.5365209,406.06209504)(136.64652079,406.06709503)(136.75652466,406.06710297)
\curveto(136.86652057,406.06709503)(136.97152047,406.06209504)(137.07152466,406.05210297)
\curveto(137.18152026,406.04209506)(137.27152017,406.02709507)(137.34152466,406.00710297)
\curveto(137.49151995,405.95709514)(137.6365198,405.91209519)(137.77652466,405.87210297)
\curveto(137.91651952,405.83209527)(138.04651939,405.77709532)(138.16652466,405.70710297)
\curveto(138.2365192,405.65709544)(138.30151914,405.60709549)(138.36152466,405.55710297)
\curveto(138.42151902,405.51709558)(138.48651895,405.47209563)(138.55652466,405.42210297)
\curveto(138.59651884,405.39209571)(138.65151879,405.35209575)(138.72152466,405.30210297)
\curveto(138.80151864,405.25209585)(138.87651856,405.25209585)(138.94652466,405.30210297)
\curveto(138.98651845,405.32209578)(139.00651843,405.35709574)(139.00652466,405.40710297)
\curveto(139.00651843,405.45709564)(139.01651842,405.50709559)(139.03652466,405.55710297)
\lineto(139.03652466,405.70710297)
\curveto(139.04651839,405.73709536)(139.05151839,405.77209533)(139.05152466,405.81210297)
\lineto(139.05152466,405.93210297)
\lineto(139.05152466,407.97210297)
\curveto(139.05151839,408.08209302)(139.04651839,408.2020929)(139.03652466,408.33210297)
\curveto(139.0365184,408.47209263)(139.06151838,408.57709252)(139.11152466,408.64710297)
\curveto(139.15151829,408.72709237)(139.22651821,408.77709232)(139.33652466,408.79710297)
\curveto(139.35651808,408.80709229)(139.37651806,408.80709229)(139.39652466,408.79710297)
\curveto(139.41651802,408.7970923)(139.436518,408.8020923)(139.45652466,408.81210297)
\lineto(140.52152466,408.81210297)
\curveto(140.6415168,408.81209229)(140.75151669,408.80709229)(140.85152466,408.79710297)
\curveto(140.95151649,408.78709231)(141.02651641,408.74709235)(141.07652466,408.67710297)
\curveto(141.12651631,408.5970925)(141.15151629,408.49209261)(141.15152466,408.36210297)
\lineto(141.15152466,408.00210297)
\lineto(141.15152466,398.98710297)
\moveto(139.11152466,401.92710297)
\curveto(139.12151832,401.96709913)(139.12151832,402.00709909)(139.11152466,402.04710297)
\lineto(139.11152466,402.18210297)
\curveto(139.11151833,402.28209882)(139.10651833,402.38209872)(139.09652466,402.48210297)
\curveto(139.08651835,402.58209852)(139.07151837,402.67209843)(139.05152466,402.75210297)
\curveto(139.03151841,402.86209824)(139.01151843,402.96209814)(138.99152466,403.05210297)
\curveto(138.98151846,403.14209796)(138.95651848,403.22709787)(138.91652466,403.30710297)
\curveto(138.77651866,403.66709743)(138.57151887,403.95209715)(138.30152466,404.16210297)
\curveto(138.0415194,404.37209673)(137.66151978,404.47709662)(137.16152466,404.47710297)
\curveto(137.10152034,404.47709662)(137.02152042,404.46709663)(136.92152466,404.44710297)
\curveto(136.8415206,404.42709667)(136.76652067,404.40709669)(136.69652466,404.38710297)
\curveto(136.6365208,404.37709672)(136.57652086,404.35709674)(136.51652466,404.32710297)
\curveto(136.24652119,404.21709688)(136.0365214,404.04709705)(135.88652466,403.81710297)
\curveto(135.7365217,403.58709751)(135.61652182,403.32709777)(135.52652466,403.03710297)
\curveto(135.49652194,402.93709816)(135.47652196,402.83709826)(135.46652466,402.73710297)
\curveto(135.45652198,402.63709846)(135.436522,402.53209857)(135.40652466,402.42210297)
\lineto(135.40652466,402.21210297)
\curveto(135.38652205,402.12209898)(135.38152206,401.9970991)(135.39152466,401.83710297)
\curveto(135.40152204,401.68709941)(135.41652202,401.57709952)(135.43652466,401.50710297)
\lineto(135.43652466,401.41710297)
\curveto(135.44652199,401.3970997)(135.45152199,401.37709972)(135.45152466,401.35710297)
\curveto(135.47152197,401.27709982)(135.48652195,401.2020999)(135.49652466,401.13210297)
\curveto(135.51652192,401.06210004)(135.5365219,400.98710011)(135.55652466,400.90710297)
\curveto(135.72652171,400.38710071)(136.01652142,400.0021011)(136.42652466,399.75210297)
\curveto(136.55652088,399.66210144)(136.7365207,399.59210151)(136.96652466,399.54210297)
\curveto(137.00652043,399.53210157)(137.06652037,399.52710157)(137.14652466,399.52710297)
\curveto(137.17652026,399.51710158)(137.22152022,399.50710159)(137.28152466,399.49710297)
\curveto(137.35152009,399.4971016)(137.40652003,399.5021016)(137.44652466,399.51210297)
\curveto(137.52651991,399.53210157)(137.60651983,399.54710155)(137.68652466,399.55710297)
\curveto(137.76651967,399.56710153)(137.84651959,399.58710151)(137.92652466,399.61710297)
\curveto(138.17651926,399.72710137)(138.37651906,399.86710123)(138.52652466,400.03710297)
\curveto(138.67651876,400.20710089)(138.80651863,400.42210068)(138.91652466,400.68210297)
\curveto(138.95651848,400.77210033)(138.98651845,400.86210024)(139.00652466,400.95210297)
\curveto(139.02651841,401.05210005)(139.04651839,401.15709994)(139.06652466,401.26710297)
\curveto(139.07651836,401.31709978)(139.07651836,401.36209974)(139.06652466,401.40210297)
\curveto(139.06651837,401.45209965)(139.07651836,401.5020996)(139.09652466,401.55210297)
\curveto(139.10651833,401.58209952)(139.11151833,401.61709948)(139.11152466,401.65710297)
\lineto(139.11152466,401.79210297)
\lineto(139.11152466,401.92710297)
}
}
{
\newrgbcolor{curcolor}{0 0 0}
\pscustom[linestyle=none,fillstyle=solid,fillcolor=curcolor]
{
\newpath
\moveto(149.78144653,398.73210297)
\curveto(149.80143868,398.62210248)(149.81143867,398.51210259)(149.81144653,398.40210297)
\curveto(149.82143866,398.29210281)(149.77143871,398.21710288)(149.66144653,398.17710297)
\curveto(149.60143888,398.14710295)(149.53143895,398.13210297)(149.45144653,398.13210297)
\lineto(149.21144653,398.13210297)
\lineto(148.40144653,398.13210297)
\lineto(148.13144653,398.13210297)
\curveto(148.05144043,398.14210296)(147.9864405,398.16710293)(147.93644653,398.20710297)
\curveto(147.86644062,398.24710285)(147.81144067,398.3021028)(147.77144653,398.37210297)
\curveto(147.74144074,398.45210265)(147.69644079,398.51710258)(147.63644653,398.56710297)
\curveto(147.61644087,398.58710251)(147.59144089,398.6021025)(147.56144653,398.61210297)
\curveto(147.53144095,398.63210247)(147.49144099,398.63710246)(147.44144653,398.62710297)
\curveto(147.39144109,398.60710249)(147.34144114,398.58210252)(147.29144653,398.55210297)
\curveto(147.25144123,398.52210258)(147.20644128,398.4971026)(147.15644653,398.47710297)
\curveto(147.10644138,398.43710266)(147.05144143,398.4021027)(146.99144653,398.37210297)
\lineto(146.81144653,398.28210297)
\curveto(146.6814418,398.22210288)(146.54644194,398.17210293)(146.40644653,398.13210297)
\curveto(146.26644222,398.102103)(146.12144236,398.06710303)(145.97144653,398.02710297)
\curveto(145.90144258,398.00710309)(145.83144265,397.9971031)(145.76144653,397.99710297)
\curveto(145.70144278,397.98710311)(145.63644285,397.97710312)(145.56644653,397.96710297)
\lineto(145.47644653,397.96710297)
\curveto(145.44644304,397.95710314)(145.41644307,397.95210315)(145.38644653,397.95210297)
\lineto(145.22144653,397.95210297)
\curveto(145.12144336,397.93210317)(145.02144346,397.93210317)(144.92144653,397.95210297)
\lineto(144.78644653,397.95210297)
\curveto(144.71644377,397.97210313)(144.64644384,397.98210312)(144.57644653,397.98210297)
\curveto(144.51644397,397.97210313)(144.45644403,397.97710312)(144.39644653,397.99710297)
\curveto(144.29644419,398.01710308)(144.20144428,398.03710306)(144.11144653,398.05710297)
\curveto(144.02144446,398.06710303)(143.93644455,398.09210301)(143.85644653,398.13210297)
\curveto(143.56644492,398.24210286)(143.31644517,398.38210272)(143.10644653,398.55210297)
\curveto(142.90644558,398.73210237)(142.74644574,398.96710213)(142.62644653,399.25710297)
\curveto(142.59644589,399.32710177)(142.56644592,399.4021017)(142.53644653,399.48210297)
\curveto(142.51644597,399.56210154)(142.49644599,399.64710145)(142.47644653,399.73710297)
\curveto(142.45644603,399.78710131)(142.44644604,399.83710126)(142.44644653,399.88710297)
\curveto(142.45644603,399.93710116)(142.45644603,399.98710111)(142.44644653,400.03710297)
\curveto(142.43644605,400.06710103)(142.42644606,400.12710097)(142.41644653,400.21710297)
\curveto(142.41644607,400.31710078)(142.42144606,400.38710071)(142.43144653,400.42710297)
\curveto(142.45144603,400.52710057)(142.46144602,400.61210049)(142.46144653,400.68210297)
\lineto(142.55144653,401.01210297)
\curveto(142.5814459,401.13209997)(142.62144586,401.23709986)(142.67144653,401.32710297)
\curveto(142.84144564,401.61709948)(143.03644545,401.83709926)(143.25644653,401.98710297)
\curveto(143.47644501,402.13709896)(143.75644473,402.26709883)(144.09644653,402.37710297)
\curveto(144.22644426,402.42709867)(144.36144412,402.46209864)(144.50144653,402.48210297)
\curveto(144.64144384,402.5020986)(144.7814437,402.52709857)(144.92144653,402.55710297)
\curveto(145.00144348,402.57709852)(145.0864434,402.58709851)(145.17644653,402.58710297)
\curveto(145.26644322,402.5970985)(145.35644313,402.61209849)(145.44644653,402.63210297)
\curveto(145.51644297,402.65209845)(145.5864429,402.65709844)(145.65644653,402.64710297)
\curveto(145.72644276,402.64709845)(145.80144268,402.65709844)(145.88144653,402.67710297)
\curveto(145.95144253,402.6970984)(146.02144246,402.70709839)(146.09144653,402.70710297)
\curveto(146.16144232,402.70709839)(146.23644225,402.71709838)(146.31644653,402.73710297)
\curveto(146.52644196,402.78709831)(146.71644177,402.82709827)(146.88644653,402.85710297)
\curveto(147.06644142,402.8970982)(147.22644126,402.98709811)(147.36644653,403.12710297)
\curveto(147.45644103,403.21709788)(147.51644097,403.31709778)(147.54644653,403.42710297)
\curveto(147.55644093,403.45709764)(147.55644093,403.48209762)(147.54644653,403.50210297)
\curveto(147.54644094,403.52209758)(147.55144093,403.54209756)(147.56144653,403.56210297)
\curveto(147.57144091,403.58209752)(147.57644091,403.61209749)(147.57644653,403.65210297)
\lineto(147.57644653,403.74210297)
\lineto(147.54644653,403.86210297)
\curveto(147.54644094,403.9020972)(147.54144094,403.93709716)(147.53144653,403.96710297)
\curveto(147.43144105,404.26709683)(147.22144126,404.47209663)(146.90144653,404.58210297)
\curveto(146.81144167,404.61209649)(146.70144178,404.63209647)(146.57144653,404.64210297)
\curveto(146.45144203,404.66209644)(146.32644216,404.66709643)(146.19644653,404.65710297)
\curveto(146.06644242,404.65709644)(145.94144254,404.64709645)(145.82144653,404.62710297)
\curveto(145.70144278,404.60709649)(145.59644289,404.58209652)(145.50644653,404.55210297)
\curveto(145.44644304,404.53209657)(145.3864431,404.5020966)(145.32644653,404.46210297)
\curveto(145.27644321,404.43209667)(145.22644326,404.3970967)(145.17644653,404.35710297)
\curveto(145.12644336,404.31709678)(145.07144341,404.26209684)(145.01144653,404.19210297)
\curveto(144.96144352,404.12209698)(144.92644356,404.05709704)(144.90644653,403.99710297)
\curveto(144.85644363,403.8970972)(144.81144367,403.8020973)(144.77144653,403.71210297)
\curveto(144.74144374,403.62209748)(144.67144381,403.56209754)(144.56144653,403.53210297)
\curveto(144.481444,403.51209759)(144.39644409,403.5020976)(144.30644653,403.50210297)
\lineto(144.03644653,403.50210297)
\lineto(143.46644653,403.50210297)
\curveto(143.41644507,403.5020976)(143.36644512,403.4970976)(143.31644653,403.48710297)
\curveto(143.26644522,403.48709761)(143.22144526,403.49209761)(143.18144653,403.50210297)
\lineto(143.04644653,403.50210297)
\curveto(143.02644546,403.51209759)(143.00144548,403.51709758)(142.97144653,403.51710297)
\curveto(142.94144554,403.51709758)(142.91644557,403.52709757)(142.89644653,403.54710297)
\curveto(142.81644567,403.56709753)(142.76144572,403.63209747)(142.73144653,403.74210297)
\curveto(142.72144576,403.79209731)(142.72144576,403.84209726)(142.73144653,403.89210297)
\curveto(142.74144574,403.94209716)(142.75144573,403.98709711)(142.76144653,404.02710297)
\curveto(142.79144569,404.13709696)(142.82144566,404.23709686)(142.85144653,404.32710297)
\curveto(142.89144559,404.42709667)(142.93644555,404.51709658)(142.98644653,404.59710297)
\lineto(143.07644653,404.74710297)
\lineto(143.16644653,404.89710297)
\curveto(143.24644524,405.00709609)(143.34644514,405.11209599)(143.46644653,405.21210297)
\curveto(143.486445,405.22209588)(143.51644497,405.24709585)(143.55644653,405.28710297)
\curveto(143.60644488,405.32709577)(143.65144483,405.36209574)(143.69144653,405.39210297)
\curveto(143.73144475,405.42209568)(143.77644471,405.45209565)(143.82644653,405.48210297)
\curveto(143.99644449,405.59209551)(144.17644431,405.67709542)(144.36644653,405.73710297)
\curveto(144.55644393,405.80709529)(144.75144373,405.87209523)(144.95144653,405.93210297)
\curveto(145.07144341,405.96209514)(145.19644329,405.98209512)(145.32644653,405.99210297)
\curveto(145.45644303,406.0020951)(145.5864429,406.02209508)(145.71644653,406.05210297)
\curveto(145.75644273,406.06209504)(145.81644267,406.06209504)(145.89644653,406.05210297)
\curveto(145.9864425,406.04209506)(146.04144244,406.04709505)(146.06144653,406.06710297)
\curveto(146.47144201,406.07709502)(146.86144162,406.06209504)(147.23144653,406.02210297)
\curveto(147.61144087,405.98209512)(147.95144053,405.90709519)(148.25144653,405.79710297)
\curveto(148.56143992,405.68709541)(148.82643966,405.53709556)(149.04644653,405.34710297)
\curveto(149.26643922,405.16709593)(149.43643905,404.93209617)(149.55644653,404.64210297)
\curveto(149.62643886,404.47209663)(149.66643882,404.27709682)(149.67644653,404.05710297)
\curveto(149.6864388,403.83709726)(149.69143879,403.61209749)(149.69144653,403.38210297)
\lineto(149.69144653,400.03710297)
\lineto(149.69144653,399.45210297)
\curveto(149.69143879,399.26210184)(149.71143877,399.08710201)(149.75144653,398.92710297)
\curveto(149.76143872,398.8971022)(149.76643872,398.86210224)(149.76644653,398.82210297)
\curveto(149.76643872,398.79210231)(149.77143871,398.76210234)(149.78144653,398.73210297)
\moveto(147.57644653,401.04210297)
\curveto(147.5864409,401.09210001)(147.59144089,401.14709995)(147.59144653,401.20710297)
\curveto(147.59144089,401.27709982)(147.5864409,401.33709976)(147.57644653,401.38710297)
\curveto(147.55644093,401.44709965)(147.54644094,401.5020996)(147.54644653,401.55210297)
\curveto(147.54644094,401.6020995)(147.52644096,401.64209946)(147.48644653,401.67210297)
\curveto(147.43644105,401.71209939)(147.36144112,401.73209937)(147.26144653,401.73210297)
\curveto(147.22144126,401.72209938)(147.1864413,401.71209939)(147.15644653,401.70210297)
\curveto(147.12644136,401.7020994)(147.09144139,401.6970994)(147.05144653,401.68710297)
\curveto(146.9814415,401.66709943)(146.90644158,401.65209945)(146.82644653,401.64210297)
\curveto(146.74644174,401.63209947)(146.66644182,401.61709948)(146.58644653,401.59710297)
\curveto(146.55644193,401.58709951)(146.51144197,401.58209952)(146.45144653,401.58210297)
\curveto(146.32144216,401.55209955)(146.19144229,401.53209957)(146.06144653,401.52210297)
\curveto(145.93144255,401.51209959)(145.80644268,401.48709961)(145.68644653,401.44710297)
\curveto(145.60644288,401.42709967)(145.53144295,401.40709969)(145.46144653,401.38710297)
\curveto(145.39144309,401.37709972)(145.32144316,401.35709974)(145.25144653,401.32710297)
\curveto(145.04144344,401.23709986)(144.86144362,401.1021)(144.71144653,400.92210297)
\curveto(144.57144391,400.74210036)(144.52144396,400.49210061)(144.56144653,400.17210297)
\curveto(144.5814439,400.0021011)(144.63644385,399.86210124)(144.72644653,399.75210297)
\curveto(144.79644369,399.64210146)(144.90144358,399.55210155)(145.04144653,399.48210297)
\curveto(145.1814433,399.42210168)(145.33144315,399.37710172)(145.49144653,399.34710297)
\curveto(145.66144282,399.31710178)(145.83644265,399.30710179)(146.01644653,399.31710297)
\curveto(146.20644228,399.33710176)(146.3814421,399.37210173)(146.54144653,399.42210297)
\curveto(146.80144168,399.5021016)(147.00644148,399.62710147)(147.15644653,399.79710297)
\curveto(147.30644118,399.97710112)(147.42144106,400.1971009)(147.50144653,400.45710297)
\curveto(147.52144096,400.52710057)(147.53144095,400.5971005)(147.53144653,400.66710297)
\curveto(147.54144094,400.74710035)(147.55644093,400.82710027)(147.57644653,400.90710297)
\lineto(147.57644653,401.04210297)
}
}
{
\newrgbcolor{curcolor}{0 0 0}
\pscustom[linestyle=none,fillstyle=solid,fillcolor=curcolor]
{
\newpath
\moveto(158.93472778,398.98710297)
\lineto(158.93472778,398.56710297)
\curveto(158.93471941,398.43710266)(158.90471944,398.33210277)(158.84472778,398.25210297)
\curveto(158.79471955,398.2021029)(158.72971962,398.16710293)(158.64972778,398.14710297)
\curveto(158.56971978,398.13710296)(158.47971987,398.13210297)(158.37972778,398.13210297)
\lineto(157.55472778,398.13210297)
\lineto(157.26972778,398.13210297)
\curveto(157.18972116,398.14210296)(157.12472122,398.16710293)(157.07472778,398.20710297)
\curveto(157.00472134,398.25710284)(156.96472138,398.32210278)(156.95472778,398.40210297)
\curveto(156.9447214,398.48210262)(156.92472142,398.56210254)(156.89472778,398.64210297)
\curveto(156.87472147,398.66210244)(156.85472149,398.67710242)(156.83472778,398.68710297)
\curveto(156.82472152,398.70710239)(156.80972154,398.72710237)(156.78972778,398.74710297)
\curveto(156.67972167,398.74710235)(156.59972175,398.72210238)(156.54972778,398.67210297)
\lineto(156.39972778,398.52210297)
\curveto(156.32972202,398.47210263)(156.26472208,398.42710267)(156.20472778,398.38710297)
\curveto(156.1447222,398.35710274)(156.07972227,398.31710278)(156.00972778,398.26710297)
\curveto(155.96972238,398.24710285)(155.92472242,398.22710287)(155.87472778,398.20710297)
\curveto(155.83472251,398.18710291)(155.78972256,398.16710293)(155.73972778,398.14710297)
\curveto(155.59972275,398.097103)(155.4497229,398.05210305)(155.28972778,398.01210297)
\curveto(155.23972311,397.99210311)(155.19472315,397.98210312)(155.15472778,397.98210297)
\curveto(155.11472323,397.98210312)(155.07472327,397.97710312)(155.03472778,397.96710297)
\lineto(154.89972778,397.96710297)
\curveto(154.86972348,397.95710314)(154.82972352,397.95210315)(154.77972778,397.95210297)
\lineto(154.64472778,397.95210297)
\curveto(154.58472376,397.93210317)(154.49472385,397.92710317)(154.37472778,397.93710297)
\curveto(154.25472409,397.93710316)(154.16972418,397.94710315)(154.11972778,397.96710297)
\curveto(154.0497243,397.98710311)(153.98472436,397.9971031)(153.92472778,397.99710297)
\curveto(153.87472447,397.98710311)(153.81972453,397.99210311)(153.75972778,398.01210297)
\lineto(153.39972778,398.13210297)
\curveto(153.28972506,398.16210294)(153.17972517,398.2021029)(153.06972778,398.25210297)
\curveto(152.71972563,398.4021027)(152.40472594,398.63210247)(152.12472778,398.94210297)
\curveto(151.85472649,399.26210184)(151.63972671,399.5971015)(151.47972778,399.94710297)
\curveto(151.42972692,400.05710104)(151.38972696,400.16210094)(151.35972778,400.26210297)
\curveto(151.32972702,400.37210073)(151.29472705,400.48210062)(151.25472778,400.59210297)
\curveto(151.2447271,400.63210047)(151.23972711,400.66710043)(151.23972778,400.69710297)
\curveto(151.23972711,400.73710036)(151.22972712,400.78210032)(151.20972778,400.83210297)
\curveto(151.18972716,400.91210019)(151.16972718,400.9971001)(151.14972778,401.08710297)
\curveto(151.13972721,401.18709991)(151.12472722,401.28709981)(151.10472778,401.38710297)
\curveto(151.09472725,401.41709968)(151.08972726,401.45209965)(151.08972778,401.49210297)
\curveto(151.09972725,401.53209957)(151.09972725,401.56709953)(151.08972778,401.59710297)
\lineto(151.08972778,401.73210297)
\curveto(151.08972726,401.78209932)(151.08472726,401.83209927)(151.07472778,401.88210297)
\curveto(151.06472728,401.93209917)(151.05972729,401.98709911)(151.05972778,402.04710297)
\curveto(151.05972729,402.11709898)(151.06472728,402.17209893)(151.07472778,402.21210297)
\curveto(151.08472726,402.26209884)(151.08972726,402.30709879)(151.08972778,402.34710297)
\lineto(151.08972778,402.49710297)
\curveto(151.09972725,402.54709855)(151.09972725,402.59209851)(151.08972778,402.63210297)
\curveto(151.08972726,402.68209842)(151.09972725,402.73209837)(151.11972778,402.78210297)
\curveto(151.13972721,402.89209821)(151.15472719,402.9970981)(151.16472778,403.09710297)
\curveto(151.18472716,403.1970979)(151.20972714,403.2970978)(151.23972778,403.39710297)
\curveto(151.27972707,403.51709758)(151.31472703,403.63209747)(151.34472778,403.74210297)
\curveto(151.37472697,403.85209725)(151.41472693,403.96209714)(151.46472778,404.07210297)
\curveto(151.60472674,404.37209673)(151.77972657,404.65709644)(151.98972778,404.92710297)
\curveto(152.00972634,404.95709614)(152.03472631,404.98209612)(152.06472778,405.00210297)
\curveto(152.10472624,405.03209607)(152.13472621,405.06209604)(152.15472778,405.09210297)
\curveto(152.19472615,405.14209596)(152.23472611,405.18709591)(152.27472778,405.22710297)
\curveto(152.31472603,405.26709583)(152.35972599,405.30709579)(152.40972778,405.34710297)
\curveto(152.4497259,405.36709573)(152.48472586,405.39209571)(152.51472778,405.42210297)
\curveto(152.5447258,405.46209564)(152.57972577,405.49209561)(152.61972778,405.51210297)
\curveto(152.86972548,405.68209542)(153.15972519,405.82209528)(153.48972778,405.93210297)
\curveto(153.55972479,405.95209515)(153.62972472,405.96709513)(153.69972778,405.97710297)
\curveto(153.77972457,405.98709511)(153.85972449,406.0020951)(153.93972778,406.02210297)
\curveto(154.00972434,406.04209506)(154.09972425,406.05209505)(154.20972778,406.05210297)
\curveto(154.31972403,406.06209504)(154.42972392,406.06709503)(154.53972778,406.06710297)
\curveto(154.6497237,406.06709503)(154.75472359,406.06209504)(154.85472778,406.05210297)
\curveto(154.96472338,406.04209506)(155.05472329,406.02709507)(155.12472778,406.00710297)
\curveto(155.27472307,405.95709514)(155.41972293,405.91209519)(155.55972778,405.87210297)
\curveto(155.69972265,405.83209527)(155.82972252,405.77709532)(155.94972778,405.70710297)
\curveto(156.01972233,405.65709544)(156.08472226,405.60709549)(156.14472778,405.55710297)
\curveto(156.20472214,405.51709558)(156.26972208,405.47209563)(156.33972778,405.42210297)
\curveto(156.37972197,405.39209571)(156.43472191,405.35209575)(156.50472778,405.30210297)
\curveto(156.58472176,405.25209585)(156.65972169,405.25209585)(156.72972778,405.30210297)
\curveto(156.76972158,405.32209578)(156.78972156,405.35709574)(156.78972778,405.40710297)
\curveto(156.78972156,405.45709564)(156.79972155,405.50709559)(156.81972778,405.55710297)
\lineto(156.81972778,405.70710297)
\curveto(156.82972152,405.73709536)(156.83472151,405.77209533)(156.83472778,405.81210297)
\lineto(156.83472778,405.93210297)
\lineto(156.83472778,407.97210297)
\curveto(156.83472151,408.08209302)(156.82972152,408.2020929)(156.81972778,408.33210297)
\curveto(156.81972153,408.47209263)(156.8447215,408.57709252)(156.89472778,408.64710297)
\curveto(156.93472141,408.72709237)(157.00972134,408.77709232)(157.11972778,408.79710297)
\curveto(157.13972121,408.80709229)(157.15972119,408.80709229)(157.17972778,408.79710297)
\curveto(157.19972115,408.7970923)(157.21972113,408.8020923)(157.23972778,408.81210297)
\lineto(158.30472778,408.81210297)
\curveto(158.42471992,408.81209229)(158.53471981,408.80709229)(158.63472778,408.79710297)
\curveto(158.73471961,408.78709231)(158.80971954,408.74709235)(158.85972778,408.67710297)
\curveto(158.90971944,408.5970925)(158.93471941,408.49209261)(158.93472778,408.36210297)
\lineto(158.93472778,408.00210297)
\lineto(158.93472778,398.98710297)
\moveto(156.89472778,401.92710297)
\curveto(156.90472144,401.96709913)(156.90472144,402.00709909)(156.89472778,402.04710297)
\lineto(156.89472778,402.18210297)
\curveto(156.89472145,402.28209882)(156.88972146,402.38209872)(156.87972778,402.48210297)
\curveto(156.86972148,402.58209852)(156.85472149,402.67209843)(156.83472778,402.75210297)
\curveto(156.81472153,402.86209824)(156.79472155,402.96209814)(156.77472778,403.05210297)
\curveto(156.76472158,403.14209796)(156.73972161,403.22709787)(156.69972778,403.30710297)
\curveto(156.55972179,403.66709743)(156.35472199,403.95209715)(156.08472778,404.16210297)
\curveto(155.82472252,404.37209673)(155.4447229,404.47709662)(154.94472778,404.47710297)
\curveto(154.88472346,404.47709662)(154.80472354,404.46709663)(154.70472778,404.44710297)
\curveto(154.62472372,404.42709667)(154.5497238,404.40709669)(154.47972778,404.38710297)
\curveto(154.41972393,404.37709672)(154.35972399,404.35709674)(154.29972778,404.32710297)
\curveto(154.02972432,404.21709688)(153.81972453,404.04709705)(153.66972778,403.81710297)
\curveto(153.51972483,403.58709751)(153.39972495,403.32709777)(153.30972778,403.03710297)
\curveto(153.27972507,402.93709816)(153.25972509,402.83709826)(153.24972778,402.73710297)
\curveto(153.23972511,402.63709846)(153.21972513,402.53209857)(153.18972778,402.42210297)
\lineto(153.18972778,402.21210297)
\curveto(153.16972518,402.12209898)(153.16472518,401.9970991)(153.17472778,401.83710297)
\curveto(153.18472516,401.68709941)(153.19972515,401.57709952)(153.21972778,401.50710297)
\lineto(153.21972778,401.41710297)
\curveto(153.22972512,401.3970997)(153.23472511,401.37709972)(153.23472778,401.35710297)
\curveto(153.25472509,401.27709982)(153.26972508,401.2020999)(153.27972778,401.13210297)
\curveto(153.29972505,401.06210004)(153.31972503,400.98710011)(153.33972778,400.90710297)
\curveto(153.50972484,400.38710071)(153.79972455,400.0021011)(154.20972778,399.75210297)
\curveto(154.33972401,399.66210144)(154.51972383,399.59210151)(154.74972778,399.54210297)
\curveto(154.78972356,399.53210157)(154.8497235,399.52710157)(154.92972778,399.52710297)
\curveto(154.95972339,399.51710158)(155.00472334,399.50710159)(155.06472778,399.49710297)
\curveto(155.13472321,399.4971016)(155.18972316,399.5021016)(155.22972778,399.51210297)
\curveto(155.30972304,399.53210157)(155.38972296,399.54710155)(155.46972778,399.55710297)
\curveto(155.5497228,399.56710153)(155.62972272,399.58710151)(155.70972778,399.61710297)
\curveto(155.95972239,399.72710137)(156.15972219,399.86710123)(156.30972778,400.03710297)
\curveto(156.45972189,400.20710089)(156.58972176,400.42210068)(156.69972778,400.68210297)
\curveto(156.73972161,400.77210033)(156.76972158,400.86210024)(156.78972778,400.95210297)
\curveto(156.80972154,401.05210005)(156.82972152,401.15709994)(156.84972778,401.26710297)
\curveto(156.85972149,401.31709978)(156.85972149,401.36209974)(156.84972778,401.40210297)
\curveto(156.8497215,401.45209965)(156.85972149,401.5020996)(156.87972778,401.55210297)
\curveto(156.88972146,401.58209952)(156.89472145,401.61709948)(156.89472778,401.65710297)
\lineto(156.89472778,401.79210297)
\lineto(156.89472778,401.92710297)
}
}
{
\newrgbcolor{curcolor}{0 0 0}
\pscustom[linestyle=none,fillstyle=solid,fillcolor=curcolor]
{
}
}
{
\newrgbcolor{curcolor}{0 0 0}
\pscustom[linestyle=none,fillstyle=solid,fillcolor=curcolor]
{
\newpath
\moveto(172.26480591,398.98710297)
\lineto(172.26480591,398.56710297)
\curveto(172.26479754,398.43710266)(172.23479757,398.33210277)(172.17480591,398.25210297)
\curveto(172.12479768,398.2021029)(172.05979774,398.16710293)(171.97980591,398.14710297)
\curveto(171.8997979,398.13710296)(171.80979799,398.13210297)(171.70980591,398.13210297)
\lineto(170.88480591,398.13210297)
\lineto(170.59980591,398.13210297)
\curveto(170.51979928,398.14210296)(170.45479935,398.16710293)(170.40480591,398.20710297)
\curveto(170.33479947,398.25710284)(170.29479951,398.32210278)(170.28480591,398.40210297)
\curveto(170.27479953,398.48210262)(170.25479955,398.56210254)(170.22480591,398.64210297)
\curveto(170.2047996,398.66210244)(170.18479962,398.67710242)(170.16480591,398.68710297)
\curveto(170.15479965,398.70710239)(170.13979966,398.72710237)(170.11980591,398.74710297)
\curveto(170.00979979,398.74710235)(169.92979987,398.72210238)(169.87980591,398.67210297)
\lineto(169.72980591,398.52210297)
\curveto(169.65980014,398.47210263)(169.59480021,398.42710267)(169.53480591,398.38710297)
\curveto(169.47480033,398.35710274)(169.40980039,398.31710278)(169.33980591,398.26710297)
\curveto(169.2998005,398.24710285)(169.25480055,398.22710287)(169.20480591,398.20710297)
\curveto(169.16480064,398.18710291)(169.11980068,398.16710293)(169.06980591,398.14710297)
\curveto(168.92980087,398.097103)(168.77980102,398.05210305)(168.61980591,398.01210297)
\curveto(168.56980123,397.99210311)(168.52480128,397.98210312)(168.48480591,397.98210297)
\curveto(168.44480136,397.98210312)(168.4048014,397.97710312)(168.36480591,397.96710297)
\lineto(168.22980591,397.96710297)
\curveto(168.1998016,397.95710314)(168.15980164,397.95210315)(168.10980591,397.95210297)
\lineto(167.97480591,397.95210297)
\curveto(167.91480189,397.93210317)(167.82480198,397.92710317)(167.70480591,397.93710297)
\curveto(167.58480222,397.93710316)(167.4998023,397.94710315)(167.44980591,397.96710297)
\curveto(167.37980242,397.98710311)(167.31480249,397.9971031)(167.25480591,397.99710297)
\curveto(167.2048026,397.98710311)(167.14980265,397.99210311)(167.08980591,398.01210297)
\lineto(166.72980591,398.13210297)
\curveto(166.61980318,398.16210294)(166.50980329,398.2021029)(166.39980591,398.25210297)
\curveto(166.04980375,398.4021027)(165.73480407,398.63210247)(165.45480591,398.94210297)
\curveto(165.18480462,399.26210184)(164.96980483,399.5971015)(164.80980591,399.94710297)
\curveto(164.75980504,400.05710104)(164.71980508,400.16210094)(164.68980591,400.26210297)
\curveto(164.65980514,400.37210073)(164.62480518,400.48210062)(164.58480591,400.59210297)
\curveto(164.57480523,400.63210047)(164.56980523,400.66710043)(164.56980591,400.69710297)
\curveto(164.56980523,400.73710036)(164.55980524,400.78210032)(164.53980591,400.83210297)
\curveto(164.51980528,400.91210019)(164.4998053,400.9971001)(164.47980591,401.08710297)
\curveto(164.46980533,401.18709991)(164.45480535,401.28709981)(164.43480591,401.38710297)
\curveto(164.42480538,401.41709968)(164.41980538,401.45209965)(164.41980591,401.49210297)
\curveto(164.42980537,401.53209957)(164.42980537,401.56709953)(164.41980591,401.59710297)
\lineto(164.41980591,401.73210297)
\curveto(164.41980538,401.78209932)(164.41480539,401.83209927)(164.40480591,401.88210297)
\curveto(164.39480541,401.93209917)(164.38980541,401.98709911)(164.38980591,402.04710297)
\curveto(164.38980541,402.11709898)(164.39480541,402.17209893)(164.40480591,402.21210297)
\curveto(164.41480539,402.26209884)(164.41980538,402.30709879)(164.41980591,402.34710297)
\lineto(164.41980591,402.49710297)
\curveto(164.42980537,402.54709855)(164.42980537,402.59209851)(164.41980591,402.63210297)
\curveto(164.41980538,402.68209842)(164.42980537,402.73209837)(164.44980591,402.78210297)
\curveto(164.46980533,402.89209821)(164.48480532,402.9970981)(164.49480591,403.09710297)
\curveto(164.51480529,403.1970979)(164.53980526,403.2970978)(164.56980591,403.39710297)
\curveto(164.60980519,403.51709758)(164.64480516,403.63209747)(164.67480591,403.74210297)
\curveto(164.7048051,403.85209725)(164.74480506,403.96209714)(164.79480591,404.07210297)
\curveto(164.93480487,404.37209673)(165.10980469,404.65709644)(165.31980591,404.92710297)
\curveto(165.33980446,404.95709614)(165.36480444,404.98209612)(165.39480591,405.00210297)
\curveto(165.43480437,405.03209607)(165.46480434,405.06209604)(165.48480591,405.09210297)
\curveto(165.52480428,405.14209596)(165.56480424,405.18709591)(165.60480591,405.22710297)
\curveto(165.64480416,405.26709583)(165.68980411,405.30709579)(165.73980591,405.34710297)
\curveto(165.77980402,405.36709573)(165.81480399,405.39209571)(165.84480591,405.42210297)
\curveto(165.87480393,405.46209564)(165.90980389,405.49209561)(165.94980591,405.51210297)
\curveto(166.1998036,405.68209542)(166.48980331,405.82209528)(166.81980591,405.93210297)
\curveto(166.88980291,405.95209515)(166.95980284,405.96709513)(167.02980591,405.97710297)
\curveto(167.10980269,405.98709511)(167.18980261,406.0020951)(167.26980591,406.02210297)
\curveto(167.33980246,406.04209506)(167.42980237,406.05209505)(167.53980591,406.05210297)
\curveto(167.64980215,406.06209504)(167.75980204,406.06709503)(167.86980591,406.06710297)
\curveto(167.97980182,406.06709503)(168.08480172,406.06209504)(168.18480591,406.05210297)
\curveto(168.29480151,406.04209506)(168.38480142,406.02709507)(168.45480591,406.00710297)
\curveto(168.6048012,405.95709514)(168.74980105,405.91209519)(168.88980591,405.87210297)
\curveto(169.02980077,405.83209527)(169.15980064,405.77709532)(169.27980591,405.70710297)
\curveto(169.34980045,405.65709544)(169.41480039,405.60709549)(169.47480591,405.55710297)
\curveto(169.53480027,405.51709558)(169.5998002,405.47209563)(169.66980591,405.42210297)
\curveto(169.70980009,405.39209571)(169.76480004,405.35209575)(169.83480591,405.30210297)
\curveto(169.91479989,405.25209585)(169.98979981,405.25209585)(170.05980591,405.30210297)
\curveto(170.0997997,405.32209578)(170.11979968,405.35709574)(170.11980591,405.40710297)
\curveto(170.11979968,405.45709564)(170.12979967,405.50709559)(170.14980591,405.55710297)
\lineto(170.14980591,405.70710297)
\curveto(170.15979964,405.73709536)(170.16479964,405.77209533)(170.16480591,405.81210297)
\lineto(170.16480591,405.93210297)
\lineto(170.16480591,407.97210297)
\curveto(170.16479964,408.08209302)(170.15979964,408.2020929)(170.14980591,408.33210297)
\curveto(170.14979965,408.47209263)(170.17479963,408.57709252)(170.22480591,408.64710297)
\curveto(170.26479954,408.72709237)(170.33979946,408.77709232)(170.44980591,408.79710297)
\curveto(170.46979933,408.80709229)(170.48979931,408.80709229)(170.50980591,408.79710297)
\curveto(170.52979927,408.7970923)(170.54979925,408.8020923)(170.56980591,408.81210297)
\lineto(171.63480591,408.81210297)
\curveto(171.75479805,408.81209229)(171.86479794,408.80709229)(171.96480591,408.79710297)
\curveto(172.06479774,408.78709231)(172.13979766,408.74709235)(172.18980591,408.67710297)
\curveto(172.23979756,408.5970925)(172.26479754,408.49209261)(172.26480591,408.36210297)
\lineto(172.26480591,408.00210297)
\lineto(172.26480591,398.98710297)
\moveto(170.22480591,401.92710297)
\curveto(170.23479957,401.96709913)(170.23479957,402.00709909)(170.22480591,402.04710297)
\lineto(170.22480591,402.18210297)
\curveto(170.22479958,402.28209882)(170.21979958,402.38209872)(170.20980591,402.48210297)
\curveto(170.1997996,402.58209852)(170.18479962,402.67209843)(170.16480591,402.75210297)
\curveto(170.14479966,402.86209824)(170.12479968,402.96209814)(170.10480591,403.05210297)
\curveto(170.09479971,403.14209796)(170.06979973,403.22709787)(170.02980591,403.30710297)
\curveto(169.88979991,403.66709743)(169.68480012,403.95209715)(169.41480591,404.16210297)
\curveto(169.15480065,404.37209673)(168.77480103,404.47709662)(168.27480591,404.47710297)
\curveto(168.21480159,404.47709662)(168.13480167,404.46709663)(168.03480591,404.44710297)
\curveto(167.95480185,404.42709667)(167.87980192,404.40709669)(167.80980591,404.38710297)
\curveto(167.74980205,404.37709672)(167.68980211,404.35709674)(167.62980591,404.32710297)
\curveto(167.35980244,404.21709688)(167.14980265,404.04709705)(166.99980591,403.81710297)
\curveto(166.84980295,403.58709751)(166.72980307,403.32709777)(166.63980591,403.03710297)
\curveto(166.60980319,402.93709816)(166.58980321,402.83709826)(166.57980591,402.73710297)
\curveto(166.56980323,402.63709846)(166.54980325,402.53209857)(166.51980591,402.42210297)
\lineto(166.51980591,402.21210297)
\curveto(166.4998033,402.12209898)(166.49480331,401.9970991)(166.50480591,401.83710297)
\curveto(166.51480329,401.68709941)(166.52980327,401.57709952)(166.54980591,401.50710297)
\lineto(166.54980591,401.41710297)
\curveto(166.55980324,401.3970997)(166.56480324,401.37709972)(166.56480591,401.35710297)
\curveto(166.58480322,401.27709982)(166.5998032,401.2020999)(166.60980591,401.13210297)
\curveto(166.62980317,401.06210004)(166.64980315,400.98710011)(166.66980591,400.90710297)
\curveto(166.83980296,400.38710071)(167.12980267,400.0021011)(167.53980591,399.75210297)
\curveto(167.66980213,399.66210144)(167.84980195,399.59210151)(168.07980591,399.54210297)
\curveto(168.11980168,399.53210157)(168.17980162,399.52710157)(168.25980591,399.52710297)
\curveto(168.28980151,399.51710158)(168.33480147,399.50710159)(168.39480591,399.49710297)
\curveto(168.46480134,399.4971016)(168.51980128,399.5021016)(168.55980591,399.51210297)
\curveto(168.63980116,399.53210157)(168.71980108,399.54710155)(168.79980591,399.55710297)
\curveto(168.87980092,399.56710153)(168.95980084,399.58710151)(169.03980591,399.61710297)
\curveto(169.28980051,399.72710137)(169.48980031,399.86710123)(169.63980591,400.03710297)
\curveto(169.78980001,400.20710089)(169.91979988,400.42210068)(170.02980591,400.68210297)
\curveto(170.06979973,400.77210033)(170.0997997,400.86210024)(170.11980591,400.95210297)
\curveto(170.13979966,401.05210005)(170.15979964,401.15709994)(170.17980591,401.26710297)
\curveto(170.18979961,401.31709978)(170.18979961,401.36209974)(170.17980591,401.40210297)
\curveto(170.17979962,401.45209965)(170.18979961,401.5020996)(170.20980591,401.55210297)
\curveto(170.21979958,401.58209952)(170.22479958,401.61709948)(170.22480591,401.65710297)
\lineto(170.22480591,401.79210297)
\lineto(170.22480591,401.92710297)
}
}
{
\newrgbcolor{curcolor}{0 0 0}
\pscustom[linestyle=none,fillstyle=solid,fillcolor=curcolor]
{
\newpath
\moveto(181.20972778,402.07710297)
\curveto(181.22971962,401.9970991)(181.22971962,401.90709919)(181.20972778,401.80710297)
\curveto(181.18971966,401.70709939)(181.15471969,401.64209946)(181.10472778,401.61210297)
\curveto(181.05471979,401.57209953)(180.97971987,401.54209956)(180.87972778,401.52210297)
\curveto(180.78972006,401.51209959)(180.68472016,401.5020996)(180.56472778,401.49210297)
\lineto(180.21972778,401.49210297)
\curveto(180.10972074,401.5020996)(180.00972084,401.50709959)(179.91972778,401.50710297)
\lineto(176.25972778,401.50710297)
\lineto(176.04972778,401.50710297)
\curveto(175.98972486,401.50709959)(175.93472491,401.4970996)(175.88472778,401.47710297)
\curveto(175.80472504,401.43709966)(175.75472509,401.3970997)(175.73472778,401.35710297)
\curveto(175.71472513,401.33709976)(175.69472515,401.2970998)(175.67472778,401.23710297)
\curveto(175.65472519,401.18709991)(175.6497252,401.13709996)(175.65972778,401.08710297)
\curveto(175.67972517,401.02710007)(175.68972516,400.96710013)(175.68972778,400.90710297)
\curveto(175.69972515,400.85710024)(175.71472513,400.8021003)(175.73472778,400.74210297)
\curveto(175.81472503,400.5021006)(175.90972494,400.3021008)(176.01972778,400.14210297)
\curveto(176.13972471,399.99210111)(176.29972455,399.85710124)(176.49972778,399.73710297)
\curveto(176.57972427,399.68710141)(176.65972419,399.65210145)(176.73972778,399.63210297)
\curveto(176.82972402,399.62210148)(176.91972393,399.6021015)(177.00972778,399.57210297)
\curveto(177.08972376,399.55210155)(177.19972365,399.53710156)(177.33972778,399.52710297)
\curveto(177.47972337,399.51710158)(177.59972325,399.52210158)(177.69972778,399.54210297)
\lineto(177.83472778,399.54210297)
\curveto(177.93472291,399.56210154)(178.02472282,399.58210152)(178.10472778,399.60210297)
\curveto(178.19472265,399.63210147)(178.27972257,399.66210144)(178.35972778,399.69210297)
\curveto(178.45972239,399.74210136)(178.56972228,399.80710129)(178.68972778,399.88710297)
\curveto(178.81972203,399.96710113)(178.91472193,400.04710105)(178.97472778,400.12710297)
\curveto(179.02472182,400.1971009)(179.07472177,400.26210084)(179.12472778,400.32210297)
\curveto(179.18472166,400.39210071)(179.25472159,400.44210066)(179.33472778,400.47210297)
\curveto(179.43472141,400.52210058)(179.55972129,400.54210056)(179.70972778,400.53210297)
\lineto(180.14472778,400.53210297)
\lineto(180.32472778,400.53210297)
\curveto(180.39472045,400.54210056)(180.45472039,400.53710056)(180.50472778,400.51710297)
\lineto(180.65472778,400.51710297)
\curveto(180.75472009,400.4971006)(180.82472002,400.47210063)(180.86472778,400.44210297)
\curveto(180.90471994,400.42210068)(180.92471992,400.37710072)(180.92472778,400.30710297)
\curveto(180.93471991,400.23710086)(180.92971992,400.17710092)(180.90972778,400.12710297)
\curveto(180.85971999,399.98710111)(180.80472004,399.86210124)(180.74472778,399.75210297)
\curveto(180.68472016,399.64210146)(180.61472023,399.53210157)(180.53472778,399.42210297)
\curveto(180.31472053,399.09210201)(180.06472078,398.82710227)(179.78472778,398.62710297)
\curveto(179.50472134,398.42710267)(179.15472169,398.25710284)(178.73472778,398.11710297)
\curveto(178.62472222,398.07710302)(178.51472233,398.05210305)(178.40472778,398.04210297)
\curveto(178.29472255,398.03210307)(178.17972267,398.01210309)(178.05972778,397.98210297)
\curveto(178.01972283,397.97210313)(177.97472287,397.97210313)(177.92472778,397.98210297)
\curveto(177.88472296,397.98210312)(177.844723,397.97710312)(177.80472778,397.96710297)
\lineto(177.63972778,397.96710297)
\curveto(177.58972326,397.94710315)(177.52972332,397.94210316)(177.45972778,397.95210297)
\curveto(177.39972345,397.95210315)(177.3447235,397.95710314)(177.29472778,397.96710297)
\curveto(177.21472363,397.97710312)(177.1447237,397.97710312)(177.08472778,397.96710297)
\curveto(177.02472382,397.95710314)(176.95972389,397.96210314)(176.88972778,397.98210297)
\curveto(176.83972401,398.0021031)(176.78472406,398.01210309)(176.72472778,398.01210297)
\curveto(176.66472418,398.01210309)(176.60972424,398.02210308)(176.55972778,398.04210297)
\curveto(176.4497244,398.06210304)(176.33972451,398.08710301)(176.22972778,398.11710297)
\curveto(176.11972473,398.13710296)(176.01972483,398.17210293)(175.92972778,398.22210297)
\curveto(175.81972503,398.26210284)(175.71472513,398.2971028)(175.61472778,398.32710297)
\curveto(175.52472532,398.36710273)(175.43972541,398.41210269)(175.35972778,398.46210297)
\curveto(175.03972581,398.66210244)(174.75472609,398.89210221)(174.50472778,399.15210297)
\curveto(174.25472659,399.42210168)(174.0497268,399.73210137)(173.88972778,400.08210297)
\curveto(173.83972701,400.19210091)(173.79972705,400.3021008)(173.76972778,400.41210297)
\curveto(173.73972711,400.53210057)(173.69972715,400.65210045)(173.64972778,400.77210297)
\curveto(173.63972721,400.81210029)(173.63472721,400.84710025)(173.63472778,400.87710297)
\curveto(173.63472721,400.91710018)(173.62972722,400.95710014)(173.61972778,400.99710297)
\curveto(173.57972727,401.11709998)(173.55472729,401.24709985)(173.54472778,401.38710297)
\lineto(173.51472778,401.80710297)
\curveto(173.51472733,401.85709924)(173.50972734,401.91209919)(173.49972778,401.97210297)
\curveto(173.49972735,402.03209907)(173.50472734,402.08709901)(173.51472778,402.13710297)
\lineto(173.51472778,402.31710297)
\lineto(173.55972778,402.67710297)
\curveto(173.59972725,402.84709825)(173.63472721,403.01209809)(173.66472778,403.17210297)
\curveto(173.69472715,403.33209777)(173.73972711,403.48209762)(173.79972778,403.62210297)
\curveto(174.22972662,404.66209644)(174.95972589,405.3970957)(175.98972778,405.82710297)
\curveto(176.12972472,405.88709521)(176.26972458,405.92709517)(176.40972778,405.94710297)
\curveto(176.55972429,405.97709512)(176.71472413,406.01209509)(176.87472778,406.05210297)
\curveto(176.95472389,406.06209504)(177.02972382,406.06709503)(177.09972778,406.06710297)
\curveto(177.16972368,406.06709503)(177.2447236,406.07209503)(177.32472778,406.08210297)
\curveto(177.83472301,406.09209501)(178.26972258,406.03209507)(178.62972778,405.90210297)
\curveto(178.99972185,405.78209532)(179.32972152,405.62209548)(179.61972778,405.42210297)
\curveto(179.70972114,405.36209574)(179.79972105,405.29209581)(179.88972778,405.21210297)
\curveto(179.97972087,405.14209596)(180.05972079,405.06709603)(180.12972778,404.98710297)
\curveto(180.15972069,404.93709616)(180.19972065,404.8970962)(180.24972778,404.86710297)
\curveto(180.32972052,404.75709634)(180.40472044,404.64209646)(180.47472778,404.52210297)
\curveto(180.5447203,404.41209669)(180.61972023,404.2970968)(180.69972778,404.17710297)
\curveto(180.7497201,404.08709701)(180.78972006,403.99209711)(180.81972778,403.89210297)
\curveto(180.85971999,403.8020973)(180.89971995,403.7020974)(180.93972778,403.59210297)
\curveto(180.98971986,403.46209764)(181.02971982,403.32709777)(181.05972778,403.18710297)
\curveto(181.08971976,403.04709805)(181.12471972,402.90709819)(181.16472778,402.76710297)
\curveto(181.18471966,402.68709841)(181.18971966,402.5970985)(181.17972778,402.49710297)
\curveto(181.17971967,402.40709869)(181.18971966,402.32209878)(181.20972778,402.24210297)
\lineto(181.20972778,402.07710297)
\moveto(178.95972778,402.96210297)
\curveto(179.02972182,403.06209804)(179.03472181,403.18209792)(178.97472778,403.32210297)
\curveto(178.92472192,403.47209763)(178.88472196,403.58209752)(178.85472778,403.65210297)
\curveto(178.71472213,403.92209718)(178.52972232,404.12709697)(178.29972778,404.26710297)
\curveto(178.06972278,404.41709668)(177.7497231,404.4970966)(177.33972778,404.50710297)
\curveto(177.30972354,404.48709661)(177.27472357,404.48209662)(177.23472778,404.49210297)
\curveto(177.19472365,404.5020966)(177.15972369,404.5020966)(177.12972778,404.49210297)
\curveto(177.07972377,404.47209663)(177.02472382,404.45709664)(176.96472778,404.44710297)
\curveto(176.90472394,404.44709665)(176.849724,404.43709666)(176.79972778,404.41710297)
\curveto(176.35972449,404.27709682)(176.03472481,404.0020971)(175.82472778,403.59210297)
\curveto(175.80472504,403.55209755)(175.77972507,403.4970976)(175.74972778,403.42710297)
\curveto(175.72972512,403.36709773)(175.71472513,403.3020978)(175.70472778,403.23210297)
\curveto(175.69472515,403.17209793)(175.69472515,403.11209799)(175.70472778,403.05210297)
\curveto(175.72472512,402.99209811)(175.75972509,402.94209816)(175.80972778,402.90210297)
\curveto(175.88972496,402.85209825)(175.99972485,402.82709827)(176.13972778,402.82710297)
\lineto(176.54472778,402.82710297)
\lineto(178.20972778,402.82710297)
\lineto(178.64472778,402.82710297)
\curveto(178.80472204,402.83709826)(178.90972194,402.88209822)(178.95972778,402.96210297)
}
}
{
\newrgbcolor{curcolor}{0 0 0}
\pscustom[linestyle=none,fillstyle=solid,fillcolor=curcolor]
{
}
}
{
\newrgbcolor{curcolor}{0 0 0}
\pscustom[linestyle=none,fillstyle=solid,fillcolor=curcolor]
{
\newpath
\moveto(191.04316528,406.06710297)
\curveto(191.15315997,406.06709503)(191.24815987,406.05709504)(191.32816528,406.03710297)
\curveto(191.4181597,406.01709508)(191.48815963,405.97209513)(191.53816528,405.90210297)
\curveto(191.59815952,405.82209528)(191.62815949,405.68209542)(191.62816528,405.48210297)
\lineto(191.62816528,404.97210297)
\lineto(191.62816528,404.59710297)
\curveto(191.63815948,404.45709664)(191.6231595,404.34709675)(191.58316528,404.26710297)
\curveto(191.54315958,404.1970969)(191.48315964,404.15209695)(191.40316528,404.13210297)
\curveto(191.33315979,404.11209699)(191.24815987,404.102097)(191.14816528,404.10210297)
\curveto(191.05816006,404.102097)(190.95816016,404.10709699)(190.84816528,404.11710297)
\curveto(190.74816037,404.12709697)(190.65316047,404.12209698)(190.56316528,404.10210297)
\curveto(190.49316063,404.08209702)(190.4231607,404.06709703)(190.35316528,404.05710297)
\curveto(190.28316084,404.05709704)(190.2181609,404.04709705)(190.15816528,404.02710297)
\curveto(189.99816112,403.97709712)(189.83816128,403.9020972)(189.67816528,403.80210297)
\curveto(189.5181616,403.71209739)(189.39316173,403.60709749)(189.30316528,403.48710297)
\curveto(189.25316187,403.40709769)(189.19816192,403.32209778)(189.13816528,403.23210297)
\curveto(189.08816203,403.15209795)(189.03816208,403.06709803)(188.98816528,402.97710297)
\curveto(188.95816216,402.8970982)(188.92816219,402.81209829)(188.89816528,402.72210297)
\lineto(188.83816528,402.48210297)
\curveto(188.8181623,402.41209869)(188.80816231,402.33709876)(188.80816528,402.25710297)
\curveto(188.80816231,402.18709891)(188.79816232,402.11709898)(188.77816528,402.04710297)
\curveto(188.76816235,402.00709909)(188.76316236,401.96709913)(188.76316528,401.92710297)
\curveto(188.77316235,401.8970992)(188.77316235,401.86709923)(188.76316528,401.83710297)
\lineto(188.76316528,401.59710297)
\curveto(188.74316238,401.52709957)(188.73816238,401.44709965)(188.74816528,401.35710297)
\curveto(188.75816236,401.27709982)(188.76316236,401.1970999)(188.76316528,401.11710297)
\lineto(188.76316528,400.15710297)
\lineto(188.76316528,398.88210297)
\curveto(188.76316236,398.75210235)(188.75816236,398.63210247)(188.74816528,398.52210297)
\curveto(188.73816238,398.41210269)(188.70816241,398.32210278)(188.65816528,398.25210297)
\curveto(188.63816248,398.22210288)(188.60316252,398.1971029)(188.55316528,398.17710297)
\curveto(188.51316261,398.16710293)(188.46816265,398.15710294)(188.41816528,398.14710297)
\lineto(188.34316528,398.14710297)
\curveto(188.29316283,398.13710296)(188.23816288,398.13210297)(188.17816528,398.13210297)
\lineto(188.01316528,398.13210297)
\lineto(187.36816528,398.13210297)
\curveto(187.30816381,398.14210296)(187.24316388,398.14710295)(187.17316528,398.14710297)
\lineto(186.97816528,398.14710297)
\curveto(186.92816419,398.16710293)(186.87816424,398.18210292)(186.82816528,398.19210297)
\curveto(186.77816434,398.21210289)(186.74316438,398.24710285)(186.72316528,398.29710297)
\curveto(186.68316444,398.34710275)(186.65816446,398.41710268)(186.64816528,398.50710297)
\lineto(186.64816528,398.80710297)
\lineto(186.64816528,399.82710297)
\lineto(186.64816528,404.05710297)
\lineto(186.64816528,405.16710297)
\lineto(186.64816528,405.45210297)
\curveto(186.64816447,405.55209555)(186.66816445,405.63209547)(186.70816528,405.69210297)
\curveto(186.75816436,405.77209533)(186.83316429,405.82209528)(186.93316528,405.84210297)
\curveto(187.03316409,405.86209524)(187.15316397,405.87209523)(187.29316528,405.87210297)
\lineto(188.05816528,405.87210297)
\curveto(188.17816294,405.87209523)(188.28316284,405.86209524)(188.37316528,405.84210297)
\curveto(188.46316266,405.83209527)(188.53316259,405.78709531)(188.58316528,405.70710297)
\curveto(188.61316251,405.65709544)(188.62816249,405.58709551)(188.62816528,405.49710297)
\lineto(188.65816528,405.22710297)
\curveto(188.66816245,405.14709595)(188.68316244,405.07209603)(188.70316528,405.00210297)
\curveto(188.73316239,404.93209617)(188.78316234,404.8970962)(188.85316528,404.89710297)
\curveto(188.87316225,404.91709618)(188.89316223,404.92709617)(188.91316528,404.92710297)
\curveto(188.93316219,404.92709617)(188.95316217,404.93709616)(188.97316528,404.95710297)
\curveto(189.03316209,405.00709609)(189.08316204,405.06209604)(189.12316528,405.12210297)
\curveto(189.17316195,405.19209591)(189.23316189,405.25209585)(189.30316528,405.30210297)
\curveto(189.34316178,405.33209577)(189.37816174,405.36209574)(189.40816528,405.39210297)
\curveto(189.43816168,405.43209567)(189.47316165,405.46709563)(189.51316528,405.49710297)
\lineto(189.78316528,405.67710297)
\curveto(189.88316124,405.73709536)(189.98316114,405.79209531)(190.08316528,405.84210297)
\curveto(190.18316094,405.88209522)(190.28316084,405.91709518)(190.38316528,405.94710297)
\lineto(190.71316528,406.03710297)
\curveto(190.74316038,406.04709505)(190.79816032,406.04709505)(190.87816528,406.03710297)
\curveto(190.96816015,406.03709506)(191.0231601,406.04709505)(191.04316528,406.06710297)
}
}
{
\newrgbcolor{curcolor}{0 0 0}
\pscustom[linestyle=none,fillstyle=solid,fillcolor=curcolor]
{
\newpath
\moveto(199.54957153,402.07710297)
\curveto(199.56956337,401.9970991)(199.56956337,401.90709919)(199.54957153,401.80710297)
\curveto(199.52956341,401.70709939)(199.49456344,401.64209946)(199.44457153,401.61210297)
\curveto(199.39456354,401.57209953)(199.31956362,401.54209956)(199.21957153,401.52210297)
\curveto(199.12956381,401.51209959)(199.02456391,401.5020996)(198.90457153,401.49210297)
\lineto(198.55957153,401.49210297)
\curveto(198.44956449,401.5020996)(198.34956459,401.50709959)(198.25957153,401.50710297)
\lineto(194.59957153,401.50710297)
\lineto(194.38957153,401.50710297)
\curveto(194.32956861,401.50709959)(194.27456866,401.4970996)(194.22457153,401.47710297)
\curveto(194.14456879,401.43709966)(194.09456884,401.3970997)(194.07457153,401.35710297)
\curveto(194.05456888,401.33709976)(194.0345689,401.2970998)(194.01457153,401.23710297)
\curveto(193.99456894,401.18709991)(193.98956895,401.13709996)(193.99957153,401.08710297)
\curveto(194.01956892,401.02710007)(194.02956891,400.96710013)(194.02957153,400.90710297)
\curveto(194.0395689,400.85710024)(194.05456888,400.8021003)(194.07457153,400.74210297)
\curveto(194.15456878,400.5021006)(194.24956869,400.3021008)(194.35957153,400.14210297)
\curveto(194.47956846,399.99210111)(194.6395683,399.85710124)(194.83957153,399.73710297)
\curveto(194.91956802,399.68710141)(194.99956794,399.65210145)(195.07957153,399.63210297)
\curveto(195.16956777,399.62210148)(195.25956768,399.6021015)(195.34957153,399.57210297)
\curveto(195.42956751,399.55210155)(195.5395674,399.53710156)(195.67957153,399.52710297)
\curveto(195.81956712,399.51710158)(195.939567,399.52210158)(196.03957153,399.54210297)
\lineto(196.17457153,399.54210297)
\curveto(196.27456666,399.56210154)(196.36456657,399.58210152)(196.44457153,399.60210297)
\curveto(196.5345664,399.63210147)(196.61956632,399.66210144)(196.69957153,399.69210297)
\curveto(196.79956614,399.74210136)(196.90956603,399.80710129)(197.02957153,399.88710297)
\curveto(197.15956578,399.96710113)(197.25456568,400.04710105)(197.31457153,400.12710297)
\curveto(197.36456557,400.1971009)(197.41456552,400.26210084)(197.46457153,400.32210297)
\curveto(197.52456541,400.39210071)(197.59456534,400.44210066)(197.67457153,400.47210297)
\curveto(197.77456516,400.52210058)(197.89956504,400.54210056)(198.04957153,400.53210297)
\lineto(198.48457153,400.53210297)
\lineto(198.66457153,400.53210297)
\curveto(198.7345642,400.54210056)(198.79456414,400.53710056)(198.84457153,400.51710297)
\lineto(198.99457153,400.51710297)
\curveto(199.09456384,400.4971006)(199.16456377,400.47210063)(199.20457153,400.44210297)
\curveto(199.24456369,400.42210068)(199.26456367,400.37710072)(199.26457153,400.30710297)
\curveto(199.27456366,400.23710086)(199.26956367,400.17710092)(199.24957153,400.12710297)
\curveto(199.19956374,399.98710111)(199.14456379,399.86210124)(199.08457153,399.75210297)
\curveto(199.02456391,399.64210146)(198.95456398,399.53210157)(198.87457153,399.42210297)
\curveto(198.65456428,399.09210201)(198.40456453,398.82710227)(198.12457153,398.62710297)
\curveto(197.84456509,398.42710267)(197.49456544,398.25710284)(197.07457153,398.11710297)
\curveto(196.96456597,398.07710302)(196.85456608,398.05210305)(196.74457153,398.04210297)
\curveto(196.6345663,398.03210307)(196.51956642,398.01210309)(196.39957153,397.98210297)
\curveto(196.35956658,397.97210313)(196.31456662,397.97210313)(196.26457153,397.98210297)
\curveto(196.22456671,397.98210312)(196.18456675,397.97710312)(196.14457153,397.96710297)
\lineto(195.97957153,397.96710297)
\curveto(195.92956701,397.94710315)(195.86956707,397.94210316)(195.79957153,397.95210297)
\curveto(195.7395672,397.95210315)(195.68456725,397.95710314)(195.63457153,397.96710297)
\curveto(195.55456738,397.97710312)(195.48456745,397.97710312)(195.42457153,397.96710297)
\curveto(195.36456757,397.95710314)(195.29956764,397.96210314)(195.22957153,397.98210297)
\curveto(195.17956776,398.0021031)(195.12456781,398.01210309)(195.06457153,398.01210297)
\curveto(195.00456793,398.01210309)(194.94956799,398.02210308)(194.89957153,398.04210297)
\curveto(194.78956815,398.06210304)(194.67956826,398.08710301)(194.56957153,398.11710297)
\curveto(194.45956848,398.13710296)(194.35956858,398.17210293)(194.26957153,398.22210297)
\curveto(194.15956878,398.26210284)(194.05456888,398.2971028)(193.95457153,398.32710297)
\curveto(193.86456907,398.36710273)(193.77956916,398.41210269)(193.69957153,398.46210297)
\curveto(193.37956956,398.66210244)(193.09456984,398.89210221)(192.84457153,399.15210297)
\curveto(192.59457034,399.42210168)(192.38957055,399.73210137)(192.22957153,400.08210297)
\curveto(192.17957076,400.19210091)(192.1395708,400.3021008)(192.10957153,400.41210297)
\curveto(192.07957086,400.53210057)(192.0395709,400.65210045)(191.98957153,400.77210297)
\curveto(191.97957096,400.81210029)(191.97457096,400.84710025)(191.97457153,400.87710297)
\curveto(191.97457096,400.91710018)(191.96957097,400.95710014)(191.95957153,400.99710297)
\curveto(191.91957102,401.11709998)(191.89457104,401.24709985)(191.88457153,401.38710297)
\lineto(191.85457153,401.80710297)
\curveto(191.85457108,401.85709924)(191.84957109,401.91209919)(191.83957153,401.97210297)
\curveto(191.8395711,402.03209907)(191.84457109,402.08709901)(191.85457153,402.13710297)
\lineto(191.85457153,402.31710297)
\lineto(191.89957153,402.67710297)
\curveto(191.939571,402.84709825)(191.97457096,403.01209809)(192.00457153,403.17210297)
\curveto(192.0345709,403.33209777)(192.07957086,403.48209762)(192.13957153,403.62210297)
\curveto(192.56957037,404.66209644)(193.29956964,405.3970957)(194.32957153,405.82710297)
\curveto(194.46956847,405.88709521)(194.60956833,405.92709517)(194.74957153,405.94710297)
\curveto(194.89956804,405.97709512)(195.05456788,406.01209509)(195.21457153,406.05210297)
\curveto(195.29456764,406.06209504)(195.36956757,406.06709503)(195.43957153,406.06710297)
\curveto(195.50956743,406.06709503)(195.58456735,406.07209503)(195.66457153,406.08210297)
\curveto(196.17456676,406.09209501)(196.60956633,406.03209507)(196.96957153,405.90210297)
\curveto(197.3395656,405.78209532)(197.66956527,405.62209548)(197.95957153,405.42210297)
\curveto(198.04956489,405.36209574)(198.1395648,405.29209581)(198.22957153,405.21210297)
\curveto(198.31956462,405.14209596)(198.39956454,405.06709603)(198.46957153,404.98710297)
\curveto(198.49956444,404.93709616)(198.5395644,404.8970962)(198.58957153,404.86710297)
\curveto(198.66956427,404.75709634)(198.74456419,404.64209646)(198.81457153,404.52210297)
\curveto(198.88456405,404.41209669)(198.95956398,404.2970968)(199.03957153,404.17710297)
\curveto(199.08956385,404.08709701)(199.12956381,403.99209711)(199.15957153,403.89210297)
\curveto(199.19956374,403.8020973)(199.2395637,403.7020974)(199.27957153,403.59210297)
\curveto(199.32956361,403.46209764)(199.36956357,403.32709777)(199.39957153,403.18710297)
\curveto(199.42956351,403.04709805)(199.46456347,402.90709819)(199.50457153,402.76710297)
\curveto(199.52456341,402.68709841)(199.52956341,402.5970985)(199.51957153,402.49710297)
\curveto(199.51956342,402.40709869)(199.52956341,402.32209878)(199.54957153,402.24210297)
\lineto(199.54957153,402.07710297)
\moveto(197.29957153,402.96210297)
\curveto(197.36956557,403.06209804)(197.37456556,403.18209792)(197.31457153,403.32210297)
\curveto(197.26456567,403.47209763)(197.22456571,403.58209752)(197.19457153,403.65210297)
\curveto(197.05456588,403.92209718)(196.86956607,404.12709697)(196.63957153,404.26710297)
\curveto(196.40956653,404.41709668)(196.08956685,404.4970966)(195.67957153,404.50710297)
\curveto(195.64956729,404.48709661)(195.61456732,404.48209662)(195.57457153,404.49210297)
\curveto(195.5345674,404.5020966)(195.49956744,404.5020966)(195.46957153,404.49210297)
\curveto(195.41956752,404.47209663)(195.36456757,404.45709664)(195.30457153,404.44710297)
\curveto(195.24456769,404.44709665)(195.18956775,404.43709666)(195.13957153,404.41710297)
\curveto(194.69956824,404.27709682)(194.37456856,404.0020971)(194.16457153,403.59210297)
\curveto(194.14456879,403.55209755)(194.11956882,403.4970976)(194.08957153,403.42710297)
\curveto(194.06956887,403.36709773)(194.05456888,403.3020978)(194.04457153,403.23210297)
\curveto(194.0345689,403.17209793)(194.0345689,403.11209799)(194.04457153,403.05210297)
\curveto(194.06456887,402.99209811)(194.09956884,402.94209816)(194.14957153,402.90210297)
\curveto(194.22956871,402.85209825)(194.3395686,402.82709827)(194.47957153,402.82710297)
\lineto(194.88457153,402.82710297)
\lineto(196.54957153,402.82710297)
\lineto(196.98457153,402.82710297)
\curveto(197.14456579,402.83709826)(197.24956569,402.88209822)(197.29957153,402.96210297)
}
}
{
\newrgbcolor{curcolor}{0 0 0}
\pscustom[linestyle=none,fillstyle=solid,fillcolor=curcolor]
{
\newpath
\moveto(204.36785278,406.08210297)
\curveto(205.17784762,406.102095)(205.85284695,405.98209512)(206.39285278,405.72210297)
\curveto(206.94284586,405.46209564)(207.37784542,405.09209601)(207.69785278,404.61210297)
\curveto(207.85784494,404.37209673)(207.97784482,404.097097)(208.05785278,403.78710297)
\curveto(208.07784472,403.73709736)(208.09284471,403.67209743)(208.10285278,403.59210297)
\curveto(208.12284468,403.51209759)(208.12284468,403.44209766)(208.10285278,403.38210297)
\curveto(208.06284474,403.27209783)(207.99284481,403.20709789)(207.89285278,403.18710297)
\curveto(207.79284501,403.17709792)(207.67284513,403.17209793)(207.53285278,403.17210297)
\lineto(206.75285278,403.17210297)
\lineto(206.46785278,403.17210297)
\curveto(206.37784642,403.17209793)(206.3028465,403.19209791)(206.24285278,403.23210297)
\curveto(206.16284664,403.27209783)(206.10784669,403.33209777)(206.07785278,403.41210297)
\curveto(206.04784675,403.5020976)(206.00784679,403.59209751)(205.95785278,403.68210297)
\curveto(205.8978469,403.79209731)(205.83284697,403.89209721)(205.76285278,403.98210297)
\curveto(205.69284711,404.07209703)(205.61284719,404.15209695)(205.52285278,404.22210297)
\curveto(205.38284742,404.31209679)(205.22784757,404.38209672)(205.05785278,404.43210297)
\curveto(204.9978478,404.45209665)(204.93784786,404.46209664)(204.87785278,404.46210297)
\curveto(204.81784798,404.46209664)(204.76284804,404.47209663)(204.71285278,404.49210297)
\lineto(204.56285278,404.49210297)
\curveto(204.36284844,404.49209661)(204.2028486,404.47209663)(204.08285278,404.43210297)
\curveto(203.79284901,404.34209676)(203.55784924,404.2020969)(203.37785278,404.01210297)
\curveto(203.1978496,403.83209727)(203.05284975,403.61209749)(202.94285278,403.35210297)
\curveto(202.89284991,403.24209786)(202.85284995,403.12209798)(202.82285278,402.99210297)
\curveto(202.80285,402.87209823)(202.77785002,402.74209836)(202.74785278,402.60210297)
\curveto(202.73785006,402.56209854)(202.73285007,402.52209858)(202.73285278,402.48210297)
\curveto(202.73285007,402.44209866)(202.72785007,402.4020987)(202.71785278,402.36210297)
\curveto(202.6978501,402.26209884)(202.68785011,402.12209898)(202.68785278,401.94210297)
\curveto(202.6978501,401.76209934)(202.71285009,401.62209948)(202.73285278,401.52210297)
\curveto(202.73285007,401.44209966)(202.73785006,401.38709971)(202.74785278,401.35710297)
\curveto(202.76785003,401.28709981)(202.77785002,401.21709988)(202.77785278,401.14710297)
\curveto(202.78785001,401.07710002)(202.80285,401.00710009)(202.82285278,400.93710297)
\curveto(202.9028499,400.70710039)(202.9978498,400.4971006)(203.10785278,400.30710297)
\curveto(203.21784958,400.11710098)(203.35784944,399.95710114)(203.52785278,399.82710297)
\curveto(203.56784923,399.7971013)(203.62784917,399.76210134)(203.70785278,399.72210297)
\curveto(203.81784898,399.65210145)(203.92784887,399.60710149)(204.03785278,399.58710297)
\curveto(204.15784864,399.56710153)(204.3028485,399.54710155)(204.47285278,399.52710297)
\lineto(204.56285278,399.52710297)
\curveto(204.6028482,399.52710157)(204.63284817,399.53210157)(204.65285278,399.54210297)
\lineto(204.78785278,399.54210297)
\curveto(204.85784794,399.56210154)(204.92284788,399.57710152)(204.98285278,399.58710297)
\curveto(205.05284775,399.60710149)(205.11784768,399.62710147)(205.17785278,399.64710297)
\curveto(205.47784732,399.77710132)(205.70784709,399.96710113)(205.86785278,400.21710297)
\curveto(205.90784689,400.26710083)(205.94284686,400.32210078)(205.97285278,400.38210297)
\curveto(206.0028468,400.45210065)(206.02784677,400.51210059)(206.04785278,400.56210297)
\curveto(206.08784671,400.67210043)(206.12284668,400.76710033)(206.15285278,400.84710297)
\curveto(206.18284662,400.93710016)(206.25284655,401.00710009)(206.36285278,401.05710297)
\curveto(206.45284635,401.0971)(206.5978462,401.11209999)(206.79785278,401.10210297)
\lineto(207.29285278,401.10210297)
\lineto(207.50285278,401.10210297)
\curveto(207.58284522,401.11209999)(207.64784515,401.10709999)(207.69785278,401.08710297)
\lineto(207.81785278,401.08710297)
\lineto(207.93785278,401.05710297)
\curveto(207.97784482,401.05710004)(208.00784479,401.04710005)(208.02785278,401.02710297)
\curveto(208.07784472,400.98710011)(208.10784469,400.92710017)(208.11785278,400.84710297)
\curveto(208.13784466,400.77710032)(208.13784466,400.7021004)(208.11785278,400.62210297)
\curveto(208.02784477,400.29210081)(207.91784488,399.9971011)(207.78785278,399.73710297)
\curveto(207.37784542,398.96710213)(206.72284608,398.43210267)(205.82285278,398.13210297)
\curveto(205.72284708,398.102103)(205.61784718,398.08210302)(205.50785278,398.07210297)
\curveto(205.3978474,398.05210305)(205.28784751,398.02710307)(205.17785278,397.99710297)
\curveto(205.11784768,397.98710311)(205.05784774,397.98210312)(204.99785278,397.98210297)
\curveto(204.93784786,397.98210312)(204.87784792,397.97710312)(204.81785278,397.96710297)
\lineto(204.65285278,397.96710297)
\curveto(204.6028482,397.94710315)(204.52784827,397.94210316)(204.42785278,397.95210297)
\curveto(204.32784847,397.95210315)(204.25284855,397.95710314)(204.20285278,397.96710297)
\curveto(204.12284868,397.98710311)(204.04784875,397.9971031)(203.97785278,397.99710297)
\curveto(203.91784888,397.98710311)(203.85284895,397.99210311)(203.78285278,398.01210297)
\lineto(203.63285278,398.04210297)
\curveto(203.58284922,398.04210306)(203.53284927,398.04710305)(203.48285278,398.05710297)
\curveto(203.37284943,398.08710301)(203.26784953,398.11710298)(203.16785278,398.14710297)
\curveto(203.06784973,398.17710292)(202.97284983,398.21210289)(202.88285278,398.25210297)
\curveto(202.41285039,398.45210265)(202.01785078,398.70710239)(201.69785278,399.01710297)
\curveto(201.37785142,399.33710176)(201.11785168,399.73210137)(200.91785278,400.20210297)
\curveto(200.86785193,400.29210081)(200.82785197,400.38710071)(200.79785278,400.48710297)
\lineto(200.70785278,400.81710297)
\curveto(200.6978521,400.85710024)(200.69285211,400.89210021)(200.69285278,400.92210297)
\curveto(200.69285211,400.96210014)(200.68285212,401.00710009)(200.66285278,401.05710297)
\curveto(200.64285216,401.12709997)(200.63285217,401.1970999)(200.63285278,401.26710297)
\curveto(200.63285217,401.34709975)(200.62285218,401.42209968)(200.60285278,401.49210297)
\lineto(200.60285278,401.74710297)
\curveto(200.58285222,401.7970993)(200.57285223,401.85209925)(200.57285278,401.91210297)
\curveto(200.57285223,401.98209912)(200.58285222,402.04209906)(200.60285278,402.09210297)
\curveto(200.61285219,402.14209896)(200.61285219,402.18709891)(200.60285278,402.22710297)
\curveto(200.59285221,402.26709883)(200.59285221,402.30709879)(200.60285278,402.34710297)
\curveto(200.62285218,402.41709868)(200.62785217,402.48209862)(200.61785278,402.54210297)
\curveto(200.61785218,402.6020985)(200.62785217,402.66209844)(200.64785278,402.72210297)
\curveto(200.6978521,402.9020982)(200.73785206,403.07209803)(200.76785278,403.23210297)
\curveto(200.797852,403.4020977)(200.84285196,403.56709753)(200.90285278,403.72710297)
\curveto(201.12285168,404.23709686)(201.3978514,404.66209644)(201.72785278,405.00210297)
\curveto(202.06785073,405.34209576)(202.4978503,405.61709548)(203.01785278,405.82710297)
\curveto(203.15784964,405.88709521)(203.3028495,405.92709517)(203.45285278,405.94710297)
\curveto(203.6028492,405.97709512)(203.75784904,406.01209509)(203.91785278,406.05210297)
\curveto(203.9978488,406.06209504)(204.07284873,406.06709503)(204.14285278,406.06710297)
\curveto(204.21284859,406.06709503)(204.28784851,406.07209503)(204.36785278,406.08210297)
}
}
{
\newrgbcolor{curcolor}{0 0 0}
\pscustom[linestyle=none,fillstyle=solid,fillcolor=curcolor]
{
\newpath
\moveto(209.83113403,405.85710297)
\lineto(210.95613403,405.85710297)
\curveto(211.0661316,405.85709524)(211.1661315,405.85209525)(211.25613403,405.84210297)
\curveto(211.34613132,405.83209527)(211.41113125,405.7970953)(211.45113403,405.73710297)
\curveto(211.50113116,405.67709542)(211.53113113,405.59209551)(211.54113403,405.48210297)
\curveto(211.55113111,405.38209572)(211.55613111,405.27709582)(211.55613403,405.16710297)
\lineto(211.55613403,404.11710297)
\lineto(211.55613403,401.88210297)
\curveto(211.55613111,401.52209958)(211.57113109,401.18209992)(211.60113403,400.86210297)
\curveto(211.63113103,400.54210056)(211.72113094,400.27710082)(211.87113403,400.06710297)
\curveto(212.01113065,399.85710124)(212.23613043,399.70710139)(212.54613403,399.61710297)
\curveto(212.59613007,399.60710149)(212.63613003,399.6021015)(212.66613403,399.60210297)
\curveto(212.70612996,399.6021015)(212.75112991,399.5971015)(212.80113403,399.58710297)
\curveto(212.85112981,399.57710152)(212.90612976,399.57210153)(212.96613403,399.57210297)
\curveto(213.02612964,399.57210153)(213.07112959,399.57710152)(213.10113403,399.58710297)
\curveto(213.15112951,399.60710149)(213.19112947,399.61210149)(213.22113403,399.60210297)
\curveto(213.2611294,399.59210151)(213.30112936,399.5971015)(213.34113403,399.61710297)
\curveto(213.55112911,399.66710143)(213.71612895,399.73210137)(213.83613403,399.81210297)
\curveto(214.01612865,399.92210118)(214.15612851,400.06210104)(214.25613403,400.23210297)
\curveto(214.3661283,400.41210069)(214.44112822,400.60710049)(214.48113403,400.81710297)
\curveto(214.53112813,401.03710006)(214.5611281,401.27709982)(214.57113403,401.53710297)
\curveto(214.58112808,401.80709929)(214.58612808,402.08709901)(214.58613403,402.37710297)
\lineto(214.58613403,404.19210297)
\lineto(214.58613403,405.16710297)
\lineto(214.58613403,405.43710297)
\curveto(214.58612808,405.53709556)(214.60612806,405.61709548)(214.64613403,405.67710297)
\curveto(214.69612797,405.76709533)(214.77112789,405.81709528)(214.87113403,405.82710297)
\curveto(214.97112769,405.84709525)(215.09112757,405.85709524)(215.23113403,405.85710297)
\lineto(216.02613403,405.85710297)
\lineto(216.31113403,405.85710297)
\curveto(216.40112626,405.85709524)(216.47612619,405.83709526)(216.53613403,405.79710297)
\curveto(216.61612605,405.74709535)(216.661126,405.67209543)(216.67113403,405.57210297)
\curveto(216.68112598,405.47209563)(216.68612598,405.35709574)(216.68613403,405.22710297)
\lineto(216.68613403,404.08710297)
\lineto(216.68613403,399.87210297)
\lineto(216.68613403,398.80710297)
\lineto(216.68613403,398.50710297)
\curveto(216.68612598,398.40710269)(216.666126,398.33210277)(216.62613403,398.28210297)
\curveto(216.57612609,398.2021029)(216.50112616,398.15710294)(216.40113403,398.14710297)
\curveto(216.30112636,398.13710296)(216.19612647,398.13210297)(216.08613403,398.13210297)
\lineto(215.27613403,398.13210297)
\curveto(215.1661275,398.13210297)(215.0661276,398.13710296)(214.97613403,398.14710297)
\curveto(214.89612777,398.15710294)(214.83112783,398.1971029)(214.78113403,398.26710297)
\curveto(214.7611279,398.2971028)(214.74112792,398.34210276)(214.72113403,398.40210297)
\curveto(214.71112795,398.46210264)(214.69612797,398.52210258)(214.67613403,398.58210297)
\curveto(214.666128,398.64210246)(214.65112801,398.6971024)(214.63113403,398.74710297)
\curveto(214.61112805,398.7971023)(214.58112808,398.82710227)(214.54113403,398.83710297)
\curveto(214.52112814,398.85710224)(214.49612817,398.86210224)(214.46613403,398.85210297)
\curveto(214.43612823,398.84210226)(214.41112825,398.83210227)(214.39113403,398.82210297)
\curveto(214.32112834,398.78210232)(214.2611284,398.73710236)(214.21113403,398.68710297)
\curveto(214.1611285,398.63710246)(214.10612856,398.59210251)(214.04613403,398.55210297)
\curveto(214.00612866,398.52210258)(213.9661287,398.48710261)(213.92613403,398.44710297)
\curveto(213.89612877,398.41710268)(213.85612881,398.38710271)(213.80613403,398.35710297)
\curveto(213.57612909,398.21710288)(213.30612936,398.10710299)(212.99613403,398.02710297)
\curveto(212.92612974,398.00710309)(212.85612981,397.9971031)(212.78613403,397.99710297)
\curveto(212.71612995,397.98710311)(212.64113002,397.97210313)(212.56113403,397.95210297)
\curveto(212.52113014,397.94210316)(212.47613019,397.94210316)(212.42613403,397.95210297)
\curveto(212.38613028,397.95210315)(212.34613032,397.94710315)(212.30613403,397.93710297)
\curveto(212.27613039,397.92710317)(212.21113045,397.92710317)(212.11113403,397.93710297)
\curveto(212.02113064,397.93710316)(211.9611307,397.94210316)(211.93113403,397.95210297)
\curveto(211.88113078,397.95210315)(211.83113083,397.95710314)(211.78113403,397.96710297)
\lineto(211.63113403,397.96710297)
\curveto(211.51113115,397.9971031)(211.39613127,398.02210308)(211.28613403,398.04210297)
\curveto(211.17613149,398.06210304)(211.0661316,398.09210301)(210.95613403,398.13210297)
\curveto(210.90613176,398.15210295)(210.8611318,398.16710293)(210.82113403,398.17710297)
\curveto(210.79113187,398.1971029)(210.75113191,398.21710288)(210.70113403,398.23710297)
\curveto(210.35113231,398.42710267)(210.07113259,398.69210241)(209.86113403,399.03210297)
\curveto(209.73113293,399.24210186)(209.63613303,399.49210161)(209.57613403,399.78210297)
\curveto(209.51613315,400.08210102)(209.47613319,400.3971007)(209.45613403,400.72710297)
\curveto(209.44613322,401.06710003)(209.44113322,401.41209969)(209.44113403,401.76210297)
\curveto(209.45113321,402.12209898)(209.45613321,402.47709862)(209.45613403,402.82710297)
\lineto(209.45613403,404.86710297)
\curveto(209.45613321,404.9970961)(209.45113321,405.14709595)(209.44113403,405.31710297)
\curveto(209.44113322,405.4970956)(209.4661332,405.62709547)(209.51613403,405.70710297)
\curveto(209.54613312,405.75709534)(209.60613306,405.8020953)(209.69613403,405.84210297)
\curveto(209.75613291,405.84209526)(209.80113286,405.84709525)(209.83113403,405.85710297)
}
}
{
\newrgbcolor{curcolor}{0 0 0}
\pscustom[linestyle=none,fillstyle=solid,fillcolor=curcolor]
{
\newpath
\moveto(222.74238403,406.06710297)
\curveto(222.85237872,406.06709503)(222.94737862,406.05709504)(223.02738403,406.03710297)
\curveto(223.11737845,406.01709508)(223.18737838,405.97209513)(223.23738403,405.90210297)
\curveto(223.29737827,405.82209528)(223.32737824,405.68209542)(223.32738403,405.48210297)
\lineto(223.32738403,404.97210297)
\lineto(223.32738403,404.59710297)
\curveto(223.33737823,404.45709664)(223.32237825,404.34709675)(223.28238403,404.26710297)
\curveto(223.24237833,404.1970969)(223.18237839,404.15209695)(223.10238403,404.13210297)
\curveto(223.03237854,404.11209699)(222.94737862,404.102097)(222.84738403,404.10210297)
\curveto(222.75737881,404.102097)(222.65737891,404.10709699)(222.54738403,404.11710297)
\curveto(222.44737912,404.12709697)(222.35237922,404.12209698)(222.26238403,404.10210297)
\curveto(222.19237938,404.08209702)(222.12237945,404.06709703)(222.05238403,404.05710297)
\curveto(221.98237959,404.05709704)(221.91737965,404.04709705)(221.85738403,404.02710297)
\curveto(221.69737987,403.97709712)(221.53738003,403.9020972)(221.37738403,403.80210297)
\curveto(221.21738035,403.71209739)(221.09238048,403.60709749)(221.00238403,403.48710297)
\curveto(220.95238062,403.40709769)(220.89738067,403.32209778)(220.83738403,403.23210297)
\curveto(220.78738078,403.15209795)(220.73738083,403.06709803)(220.68738403,402.97710297)
\curveto(220.65738091,402.8970982)(220.62738094,402.81209829)(220.59738403,402.72210297)
\lineto(220.53738403,402.48210297)
\curveto(220.51738105,402.41209869)(220.50738106,402.33709876)(220.50738403,402.25710297)
\curveto(220.50738106,402.18709891)(220.49738107,402.11709898)(220.47738403,402.04710297)
\curveto(220.4673811,402.00709909)(220.46238111,401.96709913)(220.46238403,401.92710297)
\curveto(220.4723811,401.8970992)(220.4723811,401.86709923)(220.46238403,401.83710297)
\lineto(220.46238403,401.59710297)
\curveto(220.44238113,401.52709957)(220.43738113,401.44709965)(220.44738403,401.35710297)
\curveto(220.45738111,401.27709982)(220.46238111,401.1970999)(220.46238403,401.11710297)
\lineto(220.46238403,400.15710297)
\lineto(220.46238403,398.88210297)
\curveto(220.46238111,398.75210235)(220.45738111,398.63210247)(220.44738403,398.52210297)
\curveto(220.43738113,398.41210269)(220.40738116,398.32210278)(220.35738403,398.25210297)
\curveto(220.33738123,398.22210288)(220.30238127,398.1971029)(220.25238403,398.17710297)
\curveto(220.21238136,398.16710293)(220.1673814,398.15710294)(220.11738403,398.14710297)
\lineto(220.04238403,398.14710297)
\curveto(219.99238158,398.13710296)(219.93738163,398.13210297)(219.87738403,398.13210297)
\lineto(219.71238403,398.13210297)
\lineto(219.06738403,398.13210297)
\curveto(219.00738256,398.14210296)(218.94238263,398.14710295)(218.87238403,398.14710297)
\lineto(218.67738403,398.14710297)
\curveto(218.62738294,398.16710293)(218.57738299,398.18210292)(218.52738403,398.19210297)
\curveto(218.47738309,398.21210289)(218.44238313,398.24710285)(218.42238403,398.29710297)
\curveto(218.38238319,398.34710275)(218.35738321,398.41710268)(218.34738403,398.50710297)
\lineto(218.34738403,398.80710297)
\lineto(218.34738403,399.82710297)
\lineto(218.34738403,404.05710297)
\lineto(218.34738403,405.16710297)
\lineto(218.34738403,405.45210297)
\curveto(218.34738322,405.55209555)(218.3673832,405.63209547)(218.40738403,405.69210297)
\curveto(218.45738311,405.77209533)(218.53238304,405.82209528)(218.63238403,405.84210297)
\curveto(218.73238284,405.86209524)(218.85238272,405.87209523)(218.99238403,405.87210297)
\lineto(219.75738403,405.87210297)
\curveto(219.87738169,405.87209523)(219.98238159,405.86209524)(220.07238403,405.84210297)
\curveto(220.16238141,405.83209527)(220.23238134,405.78709531)(220.28238403,405.70710297)
\curveto(220.31238126,405.65709544)(220.32738124,405.58709551)(220.32738403,405.49710297)
\lineto(220.35738403,405.22710297)
\curveto(220.3673812,405.14709595)(220.38238119,405.07209603)(220.40238403,405.00210297)
\curveto(220.43238114,404.93209617)(220.48238109,404.8970962)(220.55238403,404.89710297)
\curveto(220.572381,404.91709618)(220.59238098,404.92709617)(220.61238403,404.92710297)
\curveto(220.63238094,404.92709617)(220.65238092,404.93709616)(220.67238403,404.95710297)
\curveto(220.73238084,405.00709609)(220.78238079,405.06209604)(220.82238403,405.12210297)
\curveto(220.8723807,405.19209591)(220.93238064,405.25209585)(221.00238403,405.30210297)
\curveto(221.04238053,405.33209577)(221.07738049,405.36209574)(221.10738403,405.39210297)
\curveto(221.13738043,405.43209567)(221.1723804,405.46709563)(221.21238403,405.49710297)
\lineto(221.48238403,405.67710297)
\curveto(221.58237999,405.73709536)(221.68237989,405.79209531)(221.78238403,405.84210297)
\curveto(221.88237969,405.88209522)(221.98237959,405.91709518)(222.08238403,405.94710297)
\lineto(222.41238403,406.03710297)
\curveto(222.44237913,406.04709505)(222.49737907,406.04709505)(222.57738403,406.03710297)
\curveto(222.6673789,406.03709506)(222.72237885,406.04709505)(222.74238403,406.06710297)
}
}
{
\newrgbcolor{curcolor}{0 0 0}
\pscustom[linestyle=none,fillstyle=solid,fillcolor=curcolor]
{
\newpath
\moveto(227.11746216,406.08210297)
\curveto(227.86745766,406.102095)(228.51745701,406.01709508)(229.06746216,405.82710297)
\curveto(229.6274559,405.64709545)(230.05245547,405.33209577)(230.34246216,404.88210297)
\curveto(230.41245511,404.77209633)(230.47245505,404.65709644)(230.52246216,404.53710297)
\curveto(230.58245494,404.42709667)(230.63245489,404.3020968)(230.67246216,404.16210297)
\curveto(230.69245483,404.102097)(230.70245482,404.03709706)(230.70246216,403.96710297)
\curveto(230.70245482,403.8970972)(230.69245483,403.83709726)(230.67246216,403.78710297)
\curveto(230.63245489,403.72709737)(230.57745495,403.68709741)(230.50746216,403.66710297)
\curveto(230.45745507,403.64709745)(230.39745513,403.63709746)(230.32746216,403.63710297)
\lineto(230.11746216,403.63710297)
\lineto(229.45746216,403.63710297)
\curveto(229.38745614,403.63709746)(229.31745621,403.63209747)(229.24746216,403.62210297)
\curveto(229.17745635,403.62209748)(229.11245641,403.63209747)(229.05246216,403.65210297)
\curveto(228.95245657,403.67209743)(228.87745665,403.71209739)(228.82746216,403.77210297)
\curveto(228.77745675,403.83209727)(228.73245679,403.89209721)(228.69246216,403.95210297)
\lineto(228.57246216,404.16210297)
\curveto(228.54245698,404.24209686)(228.49245703,404.30709679)(228.42246216,404.35710297)
\curveto(228.3224572,404.43709666)(228.2224573,404.4970966)(228.12246216,404.53710297)
\curveto(228.03245749,404.57709652)(227.91745761,404.61209649)(227.77746216,404.64210297)
\curveto(227.70745782,404.66209644)(227.60245792,404.67709642)(227.46246216,404.68710297)
\curveto(227.33245819,404.6970964)(227.23245829,404.69209641)(227.16246216,404.67210297)
\lineto(227.05746216,404.67210297)
\lineto(226.90746216,404.64210297)
\curveto(226.86745866,404.64209646)(226.8224587,404.63709646)(226.77246216,404.62710297)
\curveto(226.60245892,404.57709652)(226.46245906,404.50709659)(226.35246216,404.41710297)
\curveto(226.25245927,404.33709676)(226.18245934,404.21209689)(226.14246216,404.04210297)
\curveto(226.1224594,403.97209713)(226.1224594,403.90709719)(226.14246216,403.84710297)
\curveto(226.16245936,403.78709731)(226.18245934,403.73709736)(226.20246216,403.69710297)
\curveto(226.27245925,403.57709752)(226.35245917,403.48209762)(226.44246216,403.41210297)
\curveto(226.54245898,403.34209776)(226.65745887,403.28209782)(226.78746216,403.23210297)
\curveto(226.97745855,403.15209795)(227.18245834,403.08209802)(227.40246216,403.02210297)
\lineto(228.09246216,402.87210297)
\curveto(228.33245719,402.83209827)(228.56245696,402.78209832)(228.78246216,402.72210297)
\curveto(229.01245651,402.67209843)(229.2274563,402.60709849)(229.42746216,402.52710297)
\curveto(229.51745601,402.48709861)(229.60245592,402.45209865)(229.68246216,402.42210297)
\curveto(229.77245575,402.4020987)(229.85745567,402.36709873)(229.93746216,402.31710297)
\curveto(230.1274554,402.1970989)(230.29745523,402.06709903)(230.44746216,401.92710297)
\curveto(230.60745492,401.78709931)(230.73245479,401.61209949)(230.82246216,401.40210297)
\curveto(230.85245467,401.33209977)(230.87745465,401.26209984)(230.89746216,401.19210297)
\curveto(230.91745461,401.12209998)(230.93745459,401.04710005)(230.95746216,400.96710297)
\curveto(230.96745456,400.90710019)(230.97245455,400.81210029)(230.97246216,400.68210297)
\curveto(230.98245454,400.56210054)(230.98245454,400.46710063)(230.97246216,400.39710297)
\lineto(230.97246216,400.32210297)
\curveto(230.95245457,400.26210084)(230.93745459,400.2021009)(230.92746216,400.14210297)
\curveto(230.9274546,400.09210101)(230.9224546,400.04210106)(230.91246216,399.99210297)
\curveto(230.84245468,399.69210141)(230.73245479,399.42710167)(230.58246216,399.19710297)
\curveto(230.4224551,398.95710214)(230.2274553,398.76210234)(229.99746216,398.61210297)
\curveto(229.76745576,398.46210264)(229.50745602,398.33210277)(229.21746216,398.22210297)
\curveto(229.10745642,398.17210293)(228.98745654,398.13710296)(228.85746216,398.11710297)
\curveto(228.73745679,398.097103)(228.61745691,398.07210303)(228.49746216,398.04210297)
\curveto(228.40745712,398.02210308)(228.31245721,398.01210309)(228.21246216,398.01210297)
\curveto(228.1224574,398.0021031)(228.03245749,397.98710311)(227.94246216,397.96710297)
\lineto(227.67246216,397.96710297)
\curveto(227.61245791,397.94710315)(227.50745802,397.93710316)(227.35746216,397.93710297)
\curveto(227.21745831,397.93710316)(227.11745841,397.94710315)(227.05746216,397.96710297)
\curveto(227.0274585,397.96710313)(226.99245853,397.97210313)(226.95246216,397.98210297)
\lineto(226.84746216,397.98210297)
\curveto(226.7274588,398.0021031)(226.60745892,398.01710308)(226.48746216,398.02710297)
\curveto(226.36745916,398.03710306)(226.25245927,398.05710304)(226.14246216,398.08710297)
\curveto(225.75245977,398.1971029)(225.40746012,398.32210278)(225.10746216,398.46210297)
\curveto(224.80746072,398.61210249)(224.55246097,398.83210227)(224.34246216,399.12210297)
\curveto(224.20246132,399.31210179)(224.08246144,399.53210157)(223.98246216,399.78210297)
\curveto(223.96246156,399.84210126)(223.94246158,399.92210118)(223.92246216,400.02210297)
\curveto(223.90246162,400.07210103)(223.88746164,400.14210096)(223.87746216,400.23210297)
\curveto(223.86746166,400.32210078)(223.87246165,400.3971007)(223.89246216,400.45710297)
\curveto(223.9224616,400.52710057)(223.97246155,400.57710052)(224.04246216,400.60710297)
\curveto(224.09246143,400.62710047)(224.15246137,400.63710046)(224.22246216,400.63710297)
\lineto(224.44746216,400.63710297)
\lineto(225.15246216,400.63710297)
\lineto(225.39246216,400.63710297)
\curveto(225.47246005,400.63710046)(225.54245998,400.62710047)(225.60246216,400.60710297)
\curveto(225.71245981,400.56710053)(225.78245974,400.5021006)(225.81246216,400.41210297)
\curveto(225.85245967,400.32210078)(225.89745963,400.22710087)(225.94746216,400.12710297)
\curveto(225.96745956,400.07710102)(226.00245952,400.01210109)(226.05246216,399.93210297)
\curveto(226.11245941,399.85210125)(226.16245936,399.8021013)(226.20246216,399.78210297)
\curveto(226.3224592,399.68210142)(226.43745909,399.6021015)(226.54746216,399.54210297)
\curveto(226.65745887,399.49210161)(226.79745873,399.44210166)(226.96746216,399.39210297)
\curveto(227.01745851,399.37210173)(227.06745846,399.36210174)(227.11746216,399.36210297)
\curveto(227.16745836,399.37210173)(227.21745831,399.37210173)(227.26746216,399.36210297)
\curveto(227.34745818,399.34210176)(227.43245809,399.33210177)(227.52246216,399.33210297)
\curveto(227.6224579,399.34210176)(227.70745782,399.35710174)(227.77746216,399.37710297)
\curveto(227.8274577,399.38710171)(227.87245765,399.39210171)(227.91246216,399.39210297)
\curveto(227.96245756,399.39210171)(228.01245751,399.4021017)(228.06246216,399.42210297)
\curveto(228.20245732,399.47210163)(228.3274572,399.53210157)(228.43746216,399.60210297)
\curveto(228.55745697,399.67210143)(228.65245687,399.76210134)(228.72246216,399.87210297)
\curveto(228.77245675,399.95210115)(228.81245671,400.07710102)(228.84246216,400.24710297)
\curveto(228.86245666,400.31710078)(228.86245666,400.38210072)(228.84246216,400.44210297)
\curveto(228.8224567,400.5021006)(228.80245672,400.55210055)(228.78246216,400.59210297)
\curveto(228.71245681,400.73210037)(228.6224569,400.83710026)(228.51246216,400.90710297)
\curveto(228.41245711,400.97710012)(228.29245723,401.04210006)(228.15246216,401.10210297)
\curveto(227.96245756,401.18209992)(227.76245776,401.24709985)(227.55246216,401.29710297)
\curveto(227.34245818,401.34709975)(227.13245839,401.4020997)(226.92246216,401.46210297)
\curveto(226.84245868,401.48209962)(226.75745877,401.4970996)(226.66746216,401.50710297)
\curveto(226.58745894,401.51709958)(226.50745902,401.53209957)(226.42746216,401.55210297)
\curveto(226.10745942,401.64209946)(225.80245972,401.72709937)(225.51246216,401.80710297)
\curveto(225.2224603,401.8970992)(224.95746057,402.02709907)(224.71746216,402.19710297)
\curveto(224.43746109,402.3970987)(224.23246129,402.66709843)(224.10246216,403.00710297)
\curveto(224.08246144,403.07709802)(224.06246146,403.17209793)(224.04246216,403.29210297)
\curveto(224.0224615,403.36209774)(224.00746152,403.44709765)(223.99746216,403.54710297)
\curveto(223.98746154,403.64709745)(223.99246153,403.73709736)(224.01246216,403.81710297)
\curveto(224.03246149,403.86709723)(224.03746149,403.90709719)(224.02746216,403.93710297)
\curveto(224.01746151,403.97709712)(224.0224615,404.02209708)(224.04246216,404.07210297)
\curveto(224.06246146,404.18209692)(224.08246144,404.28209682)(224.10246216,404.37210297)
\curveto(224.13246139,404.47209663)(224.16746136,404.56709653)(224.20746216,404.65710297)
\curveto(224.33746119,404.94709615)(224.51746101,405.18209592)(224.74746216,405.36210297)
\curveto(224.97746055,405.54209556)(225.23746029,405.68709541)(225.52746216,405.79710297)
\curveto(225.63745989,405.84709525)(225.75245977,405.88209522)(225.87246216,405.90210297)
\curveto(225.99245953,405.93209517)(226.11745941,405.96209514)(226.24746216,405.99210297)
\curveto(226.30745922,406.01209509)(226.36745916,406.02209508)(226.42746216,406.02210297)
\lineto(226.60746216,406.05210297)
\curveto(226.68745884,406.06209504)(226.77245875,406.06709503)(226.86246216,406.06710297)
\curveto(226.95245857,406.06709503)(227.03745849,406.07209503)(227.11746216,406.08210297)
}
}
{
\newrgbcolor{curcolor}{0 0 0}
\pscustom[linestyle=none,fillstyle=solid,fillcolor=curcolor]
{
\newpath
\moveto(239.97410278,402.31710297)
\curveto(239.99409421,402.25709884)(240.0040942,402.17209893)(240.00410278,402.06210297)
\curveto(240.0040942,401.95209915)(239.99409421,401.86709923)(239.97410278,401.80710297)
\lineto(239.97410278,401.65710297)
\curveto(239.95409425,401.57709952)(239.94409426,401.4970996)(239.94410278,401.41710297)
\curveto(239.95409425,401.33709976)(239.94909426,401.25709984)(239.92910278,401.17710297)
\curveto(239.9090943,401.10709999)(239.89409431,401.04210006)(239.88410278,400.98210297)
\curveto(239.87409433,400.92210018)(239.86409434,400.85710024)(239.85410278,400.78710297)
\curveto(239.81409439,400.67710042)(239.77909443,400.56210054)(239.74910278,400.44210297)
\curveto(239.71909449,400.33210077)(239.67909453,400.22710087)(239.62910278,400.12710297)
\curveto(239.41909479,399.64710145)(239.14409506,399.25710184)(238.80410278,398.95710297)
\curveto(238.46409574,398.65710244)(238.05409615,398.40710269)(237.57410278,398.20710297)
\curveto(237.45409675,398.15710294)(237.32909688,398.12210298)(237.19910278,398.10210297)
\curveto(237.07909713,398.07210303)(236.95409725,398.04210306)(236.82410278,398.01210297)
\curveto(236.77409743,397.99210311)(236.71909749,397.98210312)(236.65910278,397.98210297)
\curveto(236.59909761,397.98210312)(236.54409766,397.97710312)(236.49410278,397.96710297)
\lineto(236.38910278,397.96710297)
\curveto(236.35909785,397.95710314)(236.32909788,397.95210315)(236.29910278,397.95210297)
\curveto(236.24909796,397.94210316)(236.16909804,397.93710316)(236.05910278,397.93710297)
\curveto(235.94909826,397.92710317)(235.86409834,397.93210317)(235.80410278,397.95210297)
\lineto(235.65410278,397.95210297)
\curveto(235.6040986,397.96210314)(235.54909866,397.96710313)(235.48910278,397.96710297)
\curveto(235.43909877,397.95710314)(235.38909882,397.96210314)(235.33910278,397.98210297)
\curveto(235.29909891,397.99210311)(235.25909895,397.9971031)(235.21910278,397.99710297)
\curveto(235.18909902,397.9971031)(235.14909906,398.0021031)(235.09910278,398.01210297)
\curveto(234.99909921,398.04210306)(234.89909931,398.06710303)(234.79910278,398.08710297)
\curveto(234.69909951,398.10710299)(234.6040996,398.13710296)(234.51410278,398.17710297)
\curveto(234.39409981,398.21710288)(234.27909993,398.25710284)(234.16910278,398.29710297)
\curveto(234.06910014,398.33710276)(233.96410024,398.38710271)(233.85410278,398.44710297)
\curveto(233.5041007,398.65710244)(233.204101,398.9021022)(232.95410278,399.18210297)
\curveto(232.7041015,399.46210164)(232.49410171,399.7971013)(232.32410278,400.18710297)
\curveto(232.27410193,400.27710082)(232.23410197,400.37210073)(232.20410278,400.47210297)
\curveto(232.18410202,400.57210053)(232.15910205,400.67710042)(232.12910278,400.78710297)
\curveto(232.1091021,400.83710026)(232.09910211,400.88210022)(232.09910278,400.92210297)
\curveto(232.09910211,400.96210014)(232.08910212,401.00710009)(232.06910278,401.05710297)
\curveto(232.04910216,401.13709996)(232.03910217,401.21709988)(232.03910278,401.29710297)
\curveto(232.03910217,401.38709971)(232.02910218,401.47209963)(232.00910278,401.55210297)
\curveto(231.99910221,401.6020995)(231.99410221,401.64709945)(231.99410278,401.68710297)
\lineto(231.99410278,401.82210297)
\curveto(231.97410223,401.88209922)(231.96410224,401.96709913)(231.96410278,402.07710297)
\curveto(231.97410223,402.18709891)(231.98910222,402.27209883)(232.00910278,402.33210297)
\lineto(232.00910278,402.43710297)
\curveto(232.01910219,402.48709861)(232.01910219,402.53709856)(232.00910278,402.58710297)
\curveto(232.0091022,402.64709845)(232.01910219,402.7020984)(232.03910278,402.75210297)
\curveto(232.04910216,402.8020983)(232.05410215,402.84709825)(232.05410278,402.88710297)
\curveto(232.05410215,402.93709816)(232.06410214,402.98709811)(232.08410278,403.03710297)
\curveto(232.12410208,403.16709793)(232.15910205,403.29209781)(232.18910278,403.41210297)
\curveto(232.21910199,403.54209756)(232.25910195,403.66709743)(232.30910278,403.78710297)
\curveto(232.48910172,404.1970969)(232.7041015,404.53709656)(232.95410278,404.80710297)
\curveto(233.204101,405.08709601)(233.5091007,405.34209576)(233.86910278,405.57210297)
\curveto(233.96910024,405.62209548)(234.07410013,405.66709543)(234.18410278,405.70710297)
\curveto(234.29409991,405.74709535)(234.4040998,405.79209531)(234.51410278,405.84210297)
\curveto(234.64409956,405.89209521)(234.77909943,405.92709517)(234.91910278,405.94710297)
\curveto(235.05909915,405.96709513)(235.204099,405.9970951)(235.35410278,406.03710297)
\curveto(235.43409877,406.04709505)(235.5090987,406.05209505)(235.57910278,406.05210297)
\curveto(235.64909856,406.05209505)(235.71909849,406.05709504)(235.78910278,406.06710297)
\curveto(236.36909784,406.07709502)(236.86909734,406.01709508)(237.28910278,405.88710297)
\curveto(237.71909649,405.75709534)(238.09909611,405.57709552)(238.42910278,405.34710297)
\curveto(238.53909567,405.26709583)(238.64909556,405.17709592)(238.75910278,405.07710297)
\curveto(238.87909533,404.98709611)(238.97909523,404.88709621)(239.05910278,404.77710297)
\curveto(239.13909507,404.67709642)(239.209095,404.57709652)(239.26910278,404.47710297)
\curveto(239.33909487,404.37709672)(239.4090948,404.27209683)(239.47910278,404.16210297)
\curveto(239.54909466,404.05209705)(239.6040946,403.93209717)(239.64410278,403.80210297)
\curveto(239.68409452,403.68209742)(239.72909448,403.55209755)(239.77910278,403.41210297)
\curveto(239.8090944,403.33209777)(239.83409437,403.24709785)(239.85410278,403.15710297)
\lineto(239.91410278,402.88710297)
\curveto(239.92409428,402.84709825)(239.92909428,402.80709829)(239.92910278,402.76710297)
\curveto(239.92909428,402.72709837)(239.93409427,402.68709841)(239.94410278,402.64710297)
\curveto(239.96409424,402.5970985)(239.96909424,402.54209856)(239.95910278,402.48210297)
\curveto(239.94909426,402.42209868)(239.95409425,402.36709873)(239.97410278,402.31710297)
\moveto(237.87410278,401.77710297)
\curveto(237.88409632,401.82709927)(237.88909632,401.8970992)(237.88910278,401.98710297)
\curveto(237.88909632,402.08709901)(237.88409632,402.16209894)(237.87410278,402.21210297)
\lineto(237.87410278,402.33210297)
\curveto(237.85409635,402.38209872)(237.84409636,402.43709866)(237.84410278,402.49710297)
\curveto(237.84409636,402.55709854)(237.83909637,402.61209849)(237.82910278,402.66210297)
\curveto(237.82909638,402.7020984)(237.82409638,402.73209837)(237.81410278,402.75210297)
\lineto(237.75410278,402.99210297)
\curveto(237.74409646,403.08209802)(237.72409648,403.16709793)(237.69410278,403.24710297)
\curveto(237.58409662,403.50709759)(237.45409675,403.72709737)(237.30410278,403.90710297)
\curveto(237.15409705,404.097097)(236.95409725,404.24709685)(236.70410278,404.35710297)
\curveto(236.64409756,404.37709672)(236.58409762,404.39209671)(236.52410278,404.40210297)
\curveto(236.46409774,404.42209668)(236.39909781,404.44209666)(236.32910278,404.46210297)
\curveto(236.24909796,404.48209662)(236.16409804,404.48709661)(236.07410278,404.47710297)
\lineto(235.80410278,404.47710297)
\curveto(235.77409843,404.45709664)(235.73909847,404.44709665)(235.69910278,404.44710297)
\curveto(235.65909855,404.45709664)(235.62409858,404.45709664)(235.59410278,404.44710297)
\lineto(235.38410278,404.38710297)
\curveto(235.32409888,404.37709672)(235.26909894,404.35709674)(235.21910278,404.32710297)
\curveto(234.96909924,404.21709688)(234.76409944,404.05709704)(234.60410278,403.84710297)
\curveto(234.45409975,403.64709745)(234.33409987,403.41209769)(234.24410278,403.14210297)
\curveto(234.21409999,403.04209806)(234.18910002,402.93709816)(234.16910278,402.82710297)
\curveto(234.15910005,402.71709838)(234.14410006,402.60709849)(234.12410278,402.49710297)
\curveto(234.11410009,402.44709865)(234.1091001,402.3970987)(234.10910278,402.34710297)
\lineto(234.10910278,402.19710297)
\curveto(234.08910012,402.12709897)(234.07910013,402.02209908)(234.07910278,401.88210297)
\curveto(234.08910012,401.74209936)(234.1041001,401.63709946)(234.12410278,401.56710297)
\lineto(234.12410278,401.43210297)
\curveto(234.14410006,401.35209975)(234.15910005,401.27209983)(234.16910278,401.19210297)
\curveto(234.17910003,401.12209998)(234.19410001,401.04710005)(234.21410278,400.96710297)
\curveto(234.31409989,400.66710043)(234.41909979,400.42210068)(234.52910278,400.23210297)
\curveto(234.64909956,400.05210105)(234.83409937,399.88710121)(235.08410278,399.73710297)
\curveto(235.15409905,399.68710141)(235.22909898,399.64710145)(235.30910278,399.61710297)
\curveto(235.39909881,399.58710151)(235.48909872,399.56210154)(235.57910278,399.54210297)
\curveto(235.61909859,399.53210157)(235.65409855,399.52710157)(235.68410278,399.52710297)
\curveto(235.71409849,399.53710156)(235.74909846,399.53710156)(235.78910278,399.52710297)
\lineto(235.90910278,399.49710297)
\curveto(235.95909825,399.4971016)(236.0040982,399.5021016)(236.04410278,399.51210297)
\lineto(236.16410278,399.51210297)
\curveto(236.24409796,399.53210157)(236.32409788,399.54710155)(236.40410278,399.55710297)
\curveto(236.48409772,399.56710153)(236.55909765,399.58710151)(236.62910278,399.61710297)
\curveto(236.88909732,399.71710138)(237.09909711,399.85210125)(237.25910278,400.02210297)
\curveto(237.41909679,400.19210091)(237.55409665,400.4021007)(237.66410278,400.65210297)
\curveto(237.7040965,400.75210035)(237.73409647,400.85210025)(237.75410278,400.95210297)
\curveto(237.77409643,401.05210005)(237.79909641,401.15709994)(237.82910278,401.26710297)
\curveto(237.83909637,401.30709979)(237.84409636,401.34209976)(237.84410278,401.37210297)
\curveto(237.84409636,401.41209969)(237.84909636,401.45209965)(237.85910278,401.49210297)
\lineto(237.85910278,401.62710297)
\curveto(237.85909635,401.67709942)(237.86409634,401.72709937)(237.87410278,401.77710297)
}
}
{
\newrgbcolor{curcolor}{0 0 0}
\pscustom[linestyle=none,fillstyle=solid,fillcolor=curcolor]
{
\newpath
\moveto(244.34402466,406.08210297)
\curveto(245.09402016,406.102095)(245.74401951,406.01709508)(246.29402466,405.82710297)
\curveto(246.8540184,405.64709545)(247.27901797,405.33209577)(247.56902466,404.88210297)
\curveto(247.63901761,404.77209633)(247.69901755,404.65709644)(247.74902466,404.53710297)
\curveto(247.80901744,404.42709667)(247.85901739,404.3020968)(247.89902466,404.16210297)
\curveto(247.91901733,404.102097)(247.92901732,404.03709706)(247.92902466,403.96710297)
\curveto(247.92901732,403.8970972)(247.91901733,403.83709726)(247.89902466,403.78710297)
\curveto(247.85901739,403.72709737)(247.80401745,403.68709741)(247.73402466,403.66710297)
\curveto(247.68401757,403.64709745)(247.62401763,403.63709746)(247.55402466,403.63710297)
\lineto(247.34402466,403.63710297)
\lineto(246.68402466,403.63710297)
\curveto(246.61401864,403.63709746)(246.54401871,403.63209747)(246.47402466,403.62210297)
\curveto(246.40401885,403.62209748)(246.33901891,403.63209747)(246.27902466,403.65210297)
\curveto(246.17901907,403.67209743)(246.10401915,403.71209739)(246.05402466,403.77210297)
\curveto(246.00401925,403.83209727)(245.95901929,403.89209721)(245.91902466,403.95210297)
\lineto(245.79902466,404.16210297)
\curveto(245.76901948,404.24209686)(245.71901953,404.30709679)(245.64902466,404.35710297)
\curveto(245.5490197,404.43709666)(245.4490198,404.4970966)(245.34902466,404.53710297)
\curveto(245.25901999,404.57709652)(245.14402011,404.61209649)(245.00402466,404.64210297)
\curveto(244.93402032,404.66209644)(244.82902042,404.67709642)(244.68902466,404.68710297)
\curveto(244.55902069,404.6970964)(244.45902079,404.69209641)(244.38902466,404.67210297)
\lineto(244.28402466,404.67210297)
\lineto(244.13402466,404.64210297)
\curveto(244.09402116,404.64209646)(244.0490212,404.63709646)(243.99902466,404.62710297)
\curveto(243.82902142,404.57709652)(243.68902156,404.50709659)(243.57902466,404.41710297)
\curveto(243.47902177,404.33709676)(243.40902184,404.21209689)(243.36902466,404.04210297)
\curveto(243.3490219,403.97209713)(243.3490219,403.90709719)(243.36902466,403.84710297)
\curveto(243.38902186,403.78709731)(243.40902184,403.73709736)(243.42902466,403.69710297)
\curveto(243.49902175,403.57709752)(243.57902167,403.48209762)(243.66902466,403.41210297)
\curveto(243.76902148,403.34209776)(243.88402137,403.28209782)(244.01402466,403.23210297)
\curveto(244.20402105,403.15209795)(244.40902084,403.08209802)(244.62902466,403.02210297)
\lineto(245.31902466,402.87210297)
\curveto(245.55901969,402.83209827)(245.78901946,402.78209832)(246.00902466,402.72210297)
\curveto(246.23901901,402.67209843)(246.4540188,402.60709849)(246.65402466,402.52710297)
\curveto(246.74401851,402.48709861)(246.82901842,402.45209865)(246.90902466,402.42210297)
\curveto(246.99901825,402.4020987)(247.08401817,402.36709873)(247.16402466,402.31710297)
\curveto(247.3540179,402.1970989)(247.52401773,402.06709903)(247.67402466,401.92710297)
\curveto(247.83401742,401.78709931)(247.95901729,401.61209949)(248.04902466,401.40210297)
\curveto(248.07901717,401.33209977)(248.10401715,401.26209984)(248.12402466,401.19210297)
\curveto(248.14401711,401.12209998)(248.16401709,401.04710005)(248.18402466,400.96710297)
\curveto(248.19401706,400.90710019)(248.19901705,400.81210029)(248.19902466,400.68210297)
\curveto(248.20901704,400.56210054)(248.20901704,400.46710063)(248.19902466,400.39710297)
\lineto(248.19902466,400.32210297)
\curveto(248.17901707,400.26210084)(248.16401709,400.2021009)(248.15402466,400.14210297)
\curveto(248.1540171,400.09210101)(248.1490171,400.04210106)(248.13902466,399.99210297)
\curveto(248.06901718,399.69210141)(247.95901729,399.42710167)(247.80902466,399.19710297)
\curveto(247.6490176,398.95710214)(247.4540178,398.76210234)(247.22402466,398.61210297)
\curveto(246.99401826,398.46210264)(246.73401852,398.33210277)(246.44402466,398.22210297)
\curveto(246.33401892,398.17210293)(246.21401904,398.13710296)(246.08402466,398.11710297)
\curveto(245.96401929,398.097103)(245.84401941,398.07210303)(245.72402466,398.04210297)
\curveto(245.63401962,398.02210308)(245.53901971,398.01210309)(245.43902466,398.01210297)
\curveto(245.3490199,398.0021031)(245.25901999,397.98710311)(245.16902466,397.96710297)
\lineto(244.89902466,397.96710297)
\curveto(244.83902041,397.94710315)(244.73402052,397.93710316)(244.58402466,397.93710297)
\curveto(244.44402081,397.93710316)(244.34402091,397.94710315)(244.28402466,397.96710297)
\curveto(244.254021,397.96710313)(244.21902103,397.97210313)(244.17902466,397.98210297)
\lineto(244.07402466,397.98210297)
\curveto(243.9540213,398.0021031)(243.83402142,398.01710308)(243.71402466,398.02710297)
\curveto(243.59402166,398.03710306)(243.47902177,398.05710304)(243.36902466,398.08710297)
\curveto(242.97902227,398.1971029)(242.63402262,398.32210278)(242.33402466,398.46210297)
\curveto(242.03402322,398.61210249)(241.77902347,398.83210227)(241.56902466,399.12210297)
\curveto(241.42902382,399.31210179)(241.30902394,399.53210157)(241.20902466,399.78210297)
\curveto(241.18902406,399.84210126)(241.16902408,399.92210118)(241.14902466,400.02210297)
\curveto(241.12902412,400.07210103)(241.11402414,400.14210096)(241.10402466,400.23210297)
\curveto(241.09402416,400.32210078)(241.09902415,400.3971007)(241.11902466,400.45710297)
\curveto(241.1490241,400.52710057)(241.19902405,400.57710052)(241.26902466,400.60710297)
\curveto(241.31902393,400.62710047)(241.37902387,400.63710046)(241.44902466,400.63710297)
\lineto(241.67402466,400.63710297)
\lineto(242.37902466,400.63710297)
\lineto(242.61902466,400.63710297)
\curveto(242.69902255,400.63710046)(242.76902248,400.62710047)(242.82902466,400.60710297)
\curveto(242.93902231,400.56710053)(243.00902224,400.5021006)(243.03902466,400.41210297)
\curveto(243.07902217,400.32210078)(243.12402213,400.22710087)(243.17402466,400.12710297)
\curveto(243.19402206,400.07710102)(243.22902202,400.01210109)(243.27902466,399.93210297)
\curveto(243.33902191,399.85210125)(243.38902186,399.8021013)(243.42902466,399.78210297)
\curveto(243.5490217,399.68210142)(243.66402159,399.6021015)(243.77402466,399.54210297)
\curveto(243.88402137,399.49210161)(244.02402123,399.44210166)(244.19402466,399.39210297)
\curveto(244.24402101,399.37210173)(244.29402096,399.36210174)(244.34402466,399.36210297)
\curveto(244.39402086,399.37210173)(244.44402081,399.37210173)(244.49402466,399.36210297)
\curveto(244.57402068,399.34210176)(244.65902059,399.33210177)(244.74902466,399.33210297)
\curveto(244.8490204,399.34210176)(244.93402032,399.35710174)(245.00402466,399.37710297)
\curveto(245.0540202,399.38710171)(245.09902015,399.39210171)(245.13902466,399.39210297)
\curveto(245.18902006,399.39210171)(245.23902001,399.4021017)(245.28902466,399.42210297)
\curveto(245.42901982,399.47210163)(245.5540197,399.53210157)(245.66402466,399.60210297)
\curveto(245.78401947,399.67210143)(245.87901937,399.76210134)(245.94902466,399.87210297)
\curveto(245.99901925,399.95210115)(246.03901921,400.07710102)(246.06902466,400.24710297)
\curveto(246.08901916,400.31710078)(246.08901916,400.38210072)(246.06902466,400.44210297)
\curveto(246.0490192,400.5021006)(246.02901922,400.55210055)(246.00902466,400.59210297)
\curveto(245.93901931,400.73210037)(245.8490194,400.83710026)(245.73902466,400.90710297)
\curveto(245.63901961,400.97710012)(245.51901973,401.04210006)(245.37902466,401.10210297)
\curveto(245.18902006,401.18209992)(244.98902026,401.24709985)(244.77902466,401.29710297)
\curveto(244.56902068,401.34709975)(244.35902089,401.4020997)(244.14902466,401.46210297)
\curveto(244.06902118,401.48209962)(243.98402127,401.4970996)(243.89402466,401.50710297)
\curveto(243.81402144,401.51709958)(243.73402152,401.53209957)(243.65402466,401.55210297)
\curveto(243.33402192,401.64209946)(243.02902222,401.72709937)(242.73902466,401.80710297)
\curveto(242.4490228,401.8970992)(242.18402307,402.02709907)(241.94402466,402.19710297)
\curveto(241.66402359,402.3970987)(241.45902379,402.66709843)(241.32902466,403.00710297)
\curveto(241.30902394,403.07709802)(241.28902396,403.17209793)(241.26902466,403.29210297)
\curveto(241.249024,403.36209774)(241.23402402,403.44709765)(241.22402466,403.54710297)
\curveto(241.21402404,403.64709745)(241.21902403,403.73709736)(241.23902466,403.81710297)
\curveto(241.25902399,403.86709723)(241.26402399,403.90709719)(241.25402466,403.93710297)
\curveto(241.24402401,403.97709712)(241.249024,404.02209708)(241.26902466,404.07210297)
\curveto(241.28902396,404.18209692)(241.30902394,404.28209682)(241.32902466,404.37210297)
\curveto(241.35902389,404.47209663)(241.39402386,404.56709653)(241.43402466,404.65710297)
\curveto(241.56402369,404.94709615)(241.74402351,405.18209592)(241.97402466,405.36210297)
\curveto(242.20402305,405.54209556)(242.46402279,405.68709541)(242.75402466,405.79710297)
\curveto(242.86402239,405.84709525)(242.97902227,405.88209522)(243.09902466,405.90210297)
\curveto(243.21902203,405.93209517)(243.34402191,405.96209514)(243.47402466,405.99210297)
\curveto(243.53402172,406.01209509)(243.59402166,406.02209508)(243.65402466,406.02210297)
\lineto(243.83402466,406.05210297)
\curveto(243.91402134,406.06209504)(243.99902125,406.06709503)(244.08902466,406.06710297)
\curveto(244.17902107,406.06709503)(244.26402099,406.07209503)(244.34402466,406.08210297)
}
}
{
\newrgbcolor{curcolor}{0 0 0}
\pscustom[linestyle=none,fillstyle=solid,fillcolor=curcolor]
{
}
}
{
\newrgbcolor{curcolor}{0 0 0}
\pscustom[linestyle=none,fillstyle=solid,fillcolor=curcolor]
{
\newpath
\moveto(261.48082153,402.09210297)
\curveto(261.49081285,402.03209907)(261.49581285,401.94209916)(261.49582153,401.82210297)
\curveto(261.49581285,401.7020994)(261.48581286,401.61709948)(261.46582153,401.56710297)
\lineto(261.46582153,401.37210297)
\curveto(261.43581291,401.26209984)(261.41581293,401.15709994)(261.40582153,401.05710297)
\curveto(261.40581294,400.95710014)(261.39081295,400.85710024)(261.36082153,400.75710297)
\curveto(261.340813,400.66710043)(261.32081302,400.57210053)(261.30082153,400.47210297)
\curveto(261.28081306,400.38210072)(261.25081309,400.29210081)(261.21082153,400.20210297)
\curveto(261.1408132,400.03210107)(261.07081327,399.87210123)(261.00082153,399.72210297)
\curveto(260.93081341,399.58210152)(260.85081349,399.44210166)(260.76082153,399.30210297)
\curveto(260.70081364,399.21210189)(260.63581371,399.12710197)(260.56582153,399.04710297)
\curveto(260.50581384,398.97710212)(260.43581391,398.9021022)(260.35582153,398.82210297)
\lineto(260.25082153,398.71710297)
\curveto(260.20081414,398.66710243)(260.1458142,398.62210248)(260.08582153,398.58210297)
\lineto(259.93582153,398.46210297)
\curveto(259.85581449,398.4021027)(259.76581458,398.34710275)(259.66582153,398.29710297)
\curveto(259.57581477,398.25710284)(259.48081486,398.21210289)(259.38082153,398.16210297)
\curveto(259.28081506,398.11210299)(259.17581517,398.07710302)(259.06582153,398.05710297)
\curveto(258.96581538,398.03710306)(258.86081548,398.01710308)(258.75082153,397.99710297)
\curveto(258.69081565,397.97710312)(258.62581572,397.96710313)(258.55582153,397.96710297)
\curveto(258.49581585,397.96710313)(258.43081591,397.95710314)(258.36082153,397.93710297)
\lineto(258.22582153,397.93710297)
\curveto(258.1458162,397.91710318)(258.07081627,397.91710318)(258.00082153,397.93710297)
\lineto(257.85082153,397.93710297)
\curveto(257.79081655,397.95710314)(257.72581662,397.96710313)(257.65582153,397.96710297)
\curveto(257.59581675,397.95710314)(257.53581681,397.96210314)(257.47582153,397.98210297)
\curveto(257.31581703,398.03210307)(257.16081718,398.07710302)(257.01082153,398.11710297)
\curveto(256.87081747,398.15710294)(256.7408176,398.21710288)(256.62082153,398.29710297)
\curveto(256.55081779,398.33710276)(256.48581786,398.37710272)(256.42582153,398.41710297)
\curveto(256.36581798,398.46710263)(256.30081804,398.51710258)(256.23082153,398.56710297)
\lineto(256.05082153,398.70210297)
\curveto(255.97081837,398.76210234)(255.90081844,398.76710233)(255.84082153,398.71710297)
\curveto(255.79081855,398.68710241)(255.76581858,398.64710245)(255.76582153,398.59710297)
\curveto(255.76581858,398.55710254)(255.75581859,398.50710259)(255.73582153,398.44710297)
\curveto(255.71581863,398.34710275)(255.70581864,398.23210287)(255.70582153,398.10210297)
\curveto(255.71581863,397.97210313)(255.72081862,397.85210325)(255.72082153,397.74210297)
\lineto(255.72082153,396.21210297)
\curveto(255.72081862,396.08210502)(255.71581863,395.95710514)(255.70582153,395.83710297)
\curveto(255.70581864,395.70710539)(255.68081866,395.6021055)(255.63082153,395.52210297)
\curveto(255.60081874,395.48210562)(255.5458188,395.45210565)(255.46582153,395.43210297)
\curveto(255.38581896,395.41210569)(255.29581905,395.4021057)(255.19582153,395.40210297)
\curveto(255.09581925,395.39210571)(254.99581935,395.39210571)(254.89582153,395.40210297)
\lineto(254.64082153,395.40210297)
\lineto(254.23582153,395.40210297)
\lineto(254.13082153,395.40210297)
\curveto(254.09082025,395.4021057)(254.05582029,395.40710569)(254.02582153,395.41710297)
\lineto(253.90582153,395.41710297)
\curveto(253.73582061,395.46710563)(253.6458207,395.56710553)(253.63582153,395.71710297)
\curveto(253.62582072,395.85710524)(253.62082072,396.02710507)(253.62082153,396.22710297)
\lineto(253.62082153,405.03210297)
\curveto(253.62082072,405.14209596)(253.61582073,405.25709584)(253.60582153,405.37710297)
\curveto(253.60582074,405.50709559)(253.63082071,405.60709549)(253.68082153,405.67710297)
\curveto(253.72082062,405.74709535)(253.77582057,405.79209531)(253.84582153,405.81210297)
\curveto(253.89582045,405.83209527)(253.95582039,405.84209526)(254.02582153,405.84210297)
\lineto(254.25082153,405.84210297)
\lineto(254.97082153,405.84210297)
\lineto(255.25582153,405.84210297)
\curveto(255.345819,405.84209526)(255.42081892,405.81709528)(255.48082153,405.76710297)
\curveto(255.55081879,405.71709538)(255.58581876,405.65209545)(255.58582153,405.57210297)
\curveto(255.59581875,405.5020956)(255.62081872,405.42709567)(255.66082153,405.34710297)
\curveto(255.67081867,405.31709578)(255.68081866,405.29209581)(255.69082153,405.27210297)
\curveto(255.71081863,405.26209584)(255.73081861,405.24709585)(255.75082153,405.22710297)
\curveto(255.86081848,405.21709588)(255.95081839,405.24709585)(256.02082153,405.31710297)
\curveto(256.09081825,405.38709571)(256.16081818,405.44709565)(256.23082153,405.49710297)
\curveto(256.36081798,405.58709551)(256.49581785,405.66709543)(256.63582153,405.73710297)
\curveto(256.77581757,405.81709528)(256.93081741,405.88209522)(257.10082153,405.93210297)
\curveto(257.18081716,405.96209514)(257.26581708,405.98209512)(257.35582153,405.99210297)
\curveto(257.45581689,406.0020951)(257.55081679,406.01709508)(257.64082153,406.03710297)
\curveto(257.68081666,406.04709505)(257.72081662,406.04709505)(257.76082153,406.03710297)
\curveto(257.81081653,406.02709507)(257.85081649,406.03209507)(257.88082153,406.05210297)
\curveto(258.45081589,406.07209503)(258.93081541,405.99209511)(259.32082153,405.81210297)
\curveto(259.72081462,405.64209546)(260.06081428,405.41709568)(260.34082153,405.13710297)
\curveto(260.39081395,405.08709601)(260.43581391,405.03709606)(260.47582153,404.98710297)
\curveto(260.51581383,404.94709615)(260.55581379,404.9020962)(260.59582153,404.85210297)
\curveto(260.66581368,404.76209634)(260.72581362,404.67209643)(260.77582153,404.58210297)
\curveto(260.83581351,404.49209661)(260.89081345,404.4020967)(260.94082153,404.31210297)
\curveto(260.96081338,404.29209681)(260.97081337,404.26709683)(260.97082153,404.23710297)
\curveto(260.98081336,404.20709689)(260.99581335,404.17209693)(261.01582153,404.13210297)
\curveto(261.07581327,404.03209707)(261.13081321,403.91209719)(261.18082153,403.77210297)
\curveto(261.20081314,403.71209739)(261.22081312,403.64709745)(261.24082153,403.57710297)
\curveto(261.26081308,403.51709758)(261.28081306,403.45209765)(261.30082153,403.38210297)
\curveto(261.340813,403.26209784)(261.36581298,403.13709796)(261.37582153,403.00710297)
\curveto(261.39581295,402.87709822)(261.42081292,402.74209836)(261.45082153,402.60210297)
\lineto(261.45082153,402.43710297)
\lineto(261.48082153,402.25710297)
\lineto(261.48082153,402.09210297)
\moveto(259.36582153,401.74710297)
\curveto(259.37581497,401.7970993)(259.38081496,401.86209924)(259.38082153,401.94210297)
\curveto(259.38081496,402.03209907)(259.37581497,402.102099)(259.36582153,402.15210297)
\lineto(259.36582153,402.28710297)
\curveto(259.345815,402.34709875)(259.33581501,402.41209869)(259.33582153,402.48210297)
\curveto(259.33581501,402.55209855)(259.32581502,402.62209848)(259.30582153,402.69210297)
\curveto(259.28581506,402.79209831)(259.26581508,402.88709821)(259.24582153,402.97710297)
\curveto(259.22581512,403.07709802)(259.19581515,403.16709793)(259.15582153,403.24710297)
\curveto(259.03581531,403.56709753)(258.88081546,403.82209728)(258.69082153,404.01210297)
\curveto(258.50081584,404.2020969)(258.23081611,404.34209676)(257.88082153,404.43210297)
\curveto(257.80081654,404.45209665)(257.71081663,404.46209664)(257.61082153,404.46210297)
\lineto(257.34082153,404.46210297)
\curveto(257.30081704,404.45209665)(257.26581708,404.44709665)(257.23582153,404.44710297)
\curveto(257.20581714,404.44709665)(257.17081717,404.44209666)(257.13082153,404.43210297)
\lineto(256.92082153,404.37210297)
\curveto(256.86081748,404.36209674)(256.80081754,404.34209676)(256.74082153,404.31210297)
\curveto(256.48081786,404.2020969)(256.27581807,404.03209707)(256.12582153,403.80210297)
\curveto(255.98581836,403.57209753)(255.87081847,403.31709778)(255.78082153,403.03710297)
\curveto(255.76081858,402.95709814)(255.7458186,402.87209823)(255.73582153,402.78210297)
\curveto(255.72581862,402.7020984)(255.71081863,402.62209848)(255.69082153,402.54210297)
\curveto(255.68081866,402.5020986)(255.67581867,402.43709866)(255.67582153,402.34710297)
\curveto(255.65581869,402.30709879)(255.65081869,402.25709884)(255.66082153,402.19710297)
\curveto(255.67081867,402.14709895)(255.67081867,402.097099)(255.66082153,402.04710297)
\curveto(255.6408187,401.98709911)(255.6408187,401.93209917)(255.66082153,401.88210297)
\lineto(255.66082153,401.70210297)
\lineto(255.66082153,401.56710297)
\curveto(255.66081868,401.52709957)(255.67081867,401.48709961)(255.69082153,401.44710297)
\curveto(255.69081865,401.37709972)(255.69581865,401.32209978)(255.70582153,401.28210297)
\lineto(255.73582153,401.10210297)
\curveto(255.7458186,401.04210006)(255.76081858,400.98210012)(255.78082153,400.92210297)
\curveto(255.87081847,400.63210047)(255.97581837,400.39210071)(256.09582153,400.20210297)
\curveto(256.22581812,400.02210108)(256.40581794,399.86210124)(256.63582153,399.72210297)
\curveto(256.77581757,399.64210146)(256.9408174,399.57710152)(257.13082153,399.52710297)
\curveto(257.17081717,399.51710158)(257.20581714,399.51210159)(257.23582153,399.51210297)
\curveto(257.26581708,399.52210158)(257.30081704,399.52210158)(257.34082153,399.51210297)
\curveto(257.38081696,399.5021016)(257.4408169,399.49210161)(257.52082153,399.48210297)
\curveto(257.60081674,399.48210162)(257.66581668,399.48710161)(257.71582153,399.49710297)
\curveto(257.79581655,399.51710158)(257.87581647,399.53210157)(257.95582153,399.54210297)
\curveto(258.0458163,399.56210154)(258.13081621,399.58710151)(258.21082153,399.61710297)
\curveto(258.45081589,399.71710138)(258.6458157,399.85710124)(258.79582153,400.03710297)
\curveto(258.9458154,400.21710088)(259.07081527,400.42710067)(259.17082153,400.66710297)
\curveto(259.22081512,400.78710031)(259.25581509,400.91210019)(259.27582153,401.04210297)
\curveto(259.29581505,401.17209993)(259.32081502,401.30709979)(259.35082153,401.44710297)
\lineto(259.35082153,401.59710297)
\curveto(259.36081498,401.64709945)(259.36581498,401.6970994)(259.36582153,401.74710297)
}
}
{
\newrgbcolor{curcolor}{0 0 0}
\pscustom[linestyle=none,fillstyle=solid,fillcolor=curcolor]
{
\newpath
\moveto(263.18074341,405.85710297)
\lineto(264.30574341,405.85710297)
\curveto(264.41574097,405.85709524)(264.51574087,405.85209525)(264.60574341,405.84210297)
\curveto(264.69574069,405.83209527)(264.76074063,405.7970953)(264.80074341,405.73710297)
\curveto(264.85074054,405.67709542)(264.88074051,405.59209551)(264.89074341,405.48210297)
\curveto(264.90074049,405.38209572)(264.90574048,405.27709582)(264.90574341,405.16710297)
\lineto(264.90574341,404.11710297)
\lineto(264.90574341,401.88210297)
\curveto(264.90574048,401.52209958)(264.92074047,401.18209992)(264.95074341,400.86210297)
\curveto(264.98074041,400.54210056)(265.07074032,400.27710082)(265.22074341,400.06710297)
\curveto(265.36074003,399.85710124)(265.5857398,399.70710139)(265.89574341,399.61710297)
\curveto(265.94573944,399.60710149)(265.9857394,399.6021015)(266.01574341,399.60210297)
\curveto(266.05573933,399.6021015)(266.10073929,399.5971015)(266.15074341,399.58710297)
\curveto(266.20073919,399.57710152)(266.25573913,399.57210153)(266.31574341,399.57210297)
\curveto(266.37573901,399.57210153)(266.42073897,399.57710152)(266.45074341,399.58710297)
\curveto(266.50073889,399.60710149)(266.54073885,399.61210149)(266.57074341,399.60210297)
\curveto(266.61073878,399.59210151)(266.65073874,399.5971015)(266.69074341,399.61710297)
\curveto(266.90073849,399.66710143)(267.06573832,399.73210137)(267.18574341,399.81210297)
\curveto(267.36573802,399.92210118)(267.50573788,400.06210104)(267.60574341,400.23210297)
\curveto(267.71573767,400.41210069)(267.7907376,400.60710049)(267.83074341,400.81710297)
\curveto(267.88073751,401.03710006)(267.91073748,401.27709982)(267.92074341,401.53710297)
\curveto(267.93073746,401.80709929)(267.93573745,402.08709901)(267.93574341,402.37710297)
\lineto(267.93574341,404.19210297)
\lineto(267.93574341,405.16710297)
\lineto(267.93574341,405.43710297)
\curveto(267.93573745,405.53709556)(267.95573743,405.61709548)(267.99574341,405.67710297)
\curveto(268.04573734,405.76709533)(268.12073727,405.81709528)(268.22074341,405.82710297)
\curveto(268.32073707,405.84709525)(268.44073695,405.85709524)(268.58074341,405.85710297)
\lineto(269.37574341,405.85710297)
\lineto(269.66074341,405.85710297)
\curveto(269.75073564,405.85709524)(269.82573556,405.83709526)(269.88574341,405.79710297)
\curveto(269.96573542,405.74709535)(270.01073538,405.67209543)(270.02074341,405.57210297)
\curveto(270.03073536,405.47209563)(270.03573535,405.35709574)(270.03574341,405.22710297)
\lineto(270.03574341,404.08710297)
\lineto(270.03574341,399.87210297)
\lineto(270.03574341,398.80710297)
\lineto(270.03574341,398.50710297)
\curveto(270.03573535,398.40710269)(270.01573537,398.33210277)(269.97574341,398.28210297)
\curveto(269.92573546,398.2021029)(269.85073554,398.15710294)(269.75074341,398.14710297)
\curveto(269.65073574,398.13710296)(269.54573584,398.13210297)(269.43574341,398.13210297)
\lineto(268.62574341,398.13210297)
\curveto(268.51573687,398.13210297)(268.41573697,398.13710296)(268.32574341,398.14710297)
\curveto(268.24573714,398.15710294)(268.18073721,398.1971029)(268.13074341,398.26710297)
\curveto(268.11073728,398.2971028)(268.0907373,398.34210276)(268.07074341,398.40210297)
\curveto(268.06073733,398.46210264)(268.04573734,398.52210258)(268.02574341,398.58210297)
\curveto(268.01573737,398.64210246)(268.00073739,398.6971024)(267.98074341,398.74710297)
\curveto(267.96073743,398.7971023)(267.93073746,398.82710227)(267.89074341,398.83710297)
\curveto(267.87073752,398.85710224)(267.84573754,398.86210224)(267.81574341,398.85210297)
\curveto(267.7857376,398.84210226)(267.76073763,398.83210227)(267.74074341,398.82210297)
\curveto(267.67073772,398.78210232)(267.61073778,398.73710236)(267.56074341,398.68710297)
\curveto(267.51073788,398.63710246)(267.45573793,398.59210251)(267.39574341,398.55210297)
\curveto(267.35573803,398.52210258)(267.31573807,398.48710261)(267.27574341,398.44710297)
\curveto(267.24573814,398.41710268)(267.20573818,398.38710271)(267.15574341,398.35710297)
\curveto(266.92573846,398.21710288)(266.65573873,398.10710299)(266.34574341,398.02710297)
\curveto(266.27573911,398.00710309)(266.20573918,397.9971031)(266.13574341,397.99710297)
\curveto(266.06573932,397.98710311)(265.9907394,397.97210313)(265.91074341,397.95210297)
\curveto(265.87073952,397.94210316)(265.82573956,397.94210316)(265.77574341,397.95210297)
\curveto(265.73573965,397.95210315)(265.69573969,397.94710315)(265.65574341,397.93710297)
\curveto(265.62573976,397.92710317)(265.56073983,397.92710317)(265.46074341,397.93710297)
\curveto(265.37074002,397.93710316)(265.31074008,397.94210316)(265.28074341,397.95210297)
\curveto(265.23074016,397.95210315)(265.18074021,397.95710314)(265.13074341,397.96710297)
\lineto(264.98074341,397.96710297)
\curveto(264.86074053,397.9971031)(264.74574064,398.02210308)(264.63574341,398.04210297)
\curveto(264.52574086,398.06210304)(264.41574097,398.09210301)(264.30574341,398.13210297)
\curveto(264.25574113,398.15210295)(264.21074118,398.16710293)(264.17074341,398.17710297)
\curveto(264.14074125,398.1971029)(264.10074129,398.21710288)(264.05074341,398.23710297)
\curveto(263.70074169,398.42710267)(263.42074197,398.69210241)(263.21074341,399.03210297)
\curveto(263.08074231,399.24210186)(262.9857424,399.49210161)(262.92574341,399.78210297)
\curveto(262.86574252,400.08210102)(262.82574256,400.3971007)(262.80574341,400.72710297)
\curveto(262.79574259,401.06710003)(262.7907426,401.41209969)(262.79074341,401.76210297)
\curveto(262.80074259,402.12209898)(262.80574258,402.47709862)(262.80574341,402.82710297)
\lineto(262.80574341,404.86710297)
\curveto(262.80574258,404.9970961)(262.80074259,405.14709595)(262.79074341,405.31710297)
\curveto(262.7907426,405.4970956)(262.81574257,405.62709547)(262.86574341,405.70710297)
\curveto(262.89574249,405.75709534)(262.95574243,405.8020953)(263.04574341,405.84210297)
\curveto(263.10574228,405.84209526)(263.15074224,405.84709525)(263.18074341,405.85710297)
}
}
{
\newrgbcolor{curcolor}{0 0 0}
\pscustom[linestyle=none,fillstyle=solid,fillcolor=curcolor]
{
\newpath
\moveto(279.46699341,402.39210297)
\curveto(279.48698481,402.33209877)(279.4969848,402.22709887)(279.49699341,402.07710297)
\curveto(279.4969848,401.93709916)(279.4919848,401.83709926)(279.48199341,401.77710297)
\curveto(279.48198481,401.72709937)(279.47698482,401.68209942)(279.46699341,401.64210297)
\lineto(279.46699341,401.52210297)
\curveto(279.44698485,401.44209966)(279.43698486,401.36209974)(279.43699341,401.28210297)
\curveto(279.43698486,401.21209989)(279.42698487,401.13709996)(279.40699341,401.05710297)
\curveto(279.40698489,401.01710008)(279.3969849,400.94710015)(279.37699341,400.84710297)
\curveto(279.34698495,400.72710037)(279.31698498,400.6021005)(279.28699341,400.47210297)
\curveto(279.26698503,400.35210075)(279.23198506,400.23710086)(279.18199341,400.12710297)
\curveto(279.00198529,399.67710142)(278.77698552,399.28710181)(278.50699341,398.95710297)
\curveto(278.23698606,398.62710247)(277.88198641,398.36710273)(277.44199341,398.17710297)
\curveto(277.35198694,398.13710296)(277.25698704,398.10710299)(277.15699341,398.08710297)
\curveto(277.06698723,398.05710304)(276.96698733,398.02710307)(276.85699341,397.99710297)
\curveto(276.7969875,397.97710312)(276.73198756,397.96710313)(276.66199341,397.96710297)
\curveto(276.60198769,397.96710313)(276.54198775,397.96210314)(276.48199341,397.95210297)
\lineto(276.34699341,397.95210297)
\curveto(276.28698801,397.93210317)(276.20698809,397.92710317)(276.10699341,397.93710297)
\curveto(276.00698829,397.93710316)(275.92698837,397.94710315)(275.86699341,397.96710297)
\lineto(275.77699341,397.96710297)
\curveto(275.72698857,397.97710312)(275.67198862,397.98710311)(275.61199341,397.99710297)
\curveto(275.55198874,397.9971031)(275.4919888,398.0021031)(275.43199341,398.01210297)
\curveto(275.24198905,398.06210304)(275.06698923,398.11210299)(274.90699341,398.16210297)
\curveto(274.74698955,398.21210289)(274.5969897,398.28210282)(274.45699341,398.37210297)
\lineto(274.27699341,398.49210297)
\curveto(274.22699007,398.53210257)(274.17699012,398.57710252)(274.12699341,398.62710297)
\lineto(274.03699341,398.68710297)
\curveto(274.00699029,398.70710239)(273.97699032,398.72210238)(273.94699341,398.73210297)
\curveto(273.85699044,398.76210234)(273.80199049,398.74210236)(273.78199341,398.67210297)
\curveto(273.73199056,398.6021025)(273.6969906,398.51710258)(273.67699341,398.41710297)
\curveto(273.66699063,398.32710277)(273.63199066,398.25710284)(273.57199341,398.20710297)
\curveto(273.51199078,398.16710293)(273.44199085,398.14210296)(273.36199341,398.13210297)
\lineto(273.09199341,398.13210297)
\lineto(272.37199341,398.13210297)
\lineto(272.14699341,398.13210297)
\curveto(272.07699222,398.12210298)(272.01199228,398.12710297)(271.95199341,398.14710297)
\curveto(271.81199248,398.1971029)(271.73199256,398.28710281)(271.71199341,398.41710297)
\curveto(271.70199259,398.55710254)(271.6969926,398.71210239)(271.69699341,398.88210297)
\lineto(271.69699341,408.03210297)
\lineto(271.69699341,408.37710297)
\curveto(271.6969926,408.4970926)(271.72199257,408.59209251)(271.77199341,408.66210297)
\curveto(271.81199248,408.73209237)(271.88199241,408.77709232)(271.98199341,408.79710297)
\curveto(272.00199229,408.80709229)(272.02199227,408.80709229)(272.04199341,408.79710297)
\curveto(272.07199222,408.7970923)(272.0969922,408.8020923)(272.11699341,408.81210297)
\lineto(273.06199341,408.81210297)
\curveto(273.24199105,408.81209229)(273.3969909,408.8020923)(273.52699341,408.78210297)
\curveto(273.65699064,408.77209233)(273.74199055,408.6970924)(273.78199341,408.55710297)
\curveto(273.81199048,408.45709264)(273.82199047,408.32209278)(273.81199341,408.15210297)
\curveto(273.80199049,407.99209311)(273.7969905,407.85209325)(273.79699341,407.73210297)
\lineto(273.79699341,406.09710297)
\lineto(273.79699341,405.76710297)
\curveto(273.7969905,405.65709544)(273.80699049,405.56209554)(273.82699341,405.48210297)
\curveto(273.83699046,405.43209567)(273.84699045,405.38709571)(273.85699341,405.34710297)
\curveto(273.86699043,405.31709578)(273.8919904,405.2970958)(273.93199341,405.28710297)
\curveto(273.95199034,405.26709583)(273.97699032,405.25709584)(274.00699341,405.25710297)
\curveto(274.04699025,405.25709584)(274.07699022,405.26209584)(274.09699341,405.27210297)
\curveto(274.16699013,405.31209579)(274.23199006,405.35209575)(274.29199341,405.39210297)
\curveto(274.35198994,405.44209566)(274.41698988,405.49209561)(274.48699341,405.54210297)
\curveto(274.61698968,405.63209547)(274.75198954,405.70709539)(274.89199341,405.76710297)
\curveto(275.03198926,405.83709526)(275.18698911,405.8970952)(275.35699341,405.94710297)
\curveto(275.43698886,405.97709512)(275.51698878,405.99209511)(275.59699341,405.99210297)
\curveto(275.67698862,406.0020951)(275.75698854,406.01709508)(275.83699341,406.03710297)
\curveto(275.90698839,406.05709504)(275.98198831,406.06709503)(276.06199341,406.06710297)
\lineto(276.30199341,406.06710297)
\lineto(276.45199341,406.06710297)
\curveto(276.48198781,406.05709504)(276.51698778,406.05209505)(276.55699341,406.05210297)
\curveto(276.5969877,406.06209504)(276.63698766,406.06209504)(276.67699341,406.05210297)
\curveto(276.78698751,406.02209508)(276.88698741,405.9970951)(276.97699341,405.97710297)
\curveto(277.07698722,405.96709513)(277.17198712,405.94209516)(277.26199341,405.90210297)
\curveto(277.72198657,405.71209539)(278.0969862,405.46709563)(278.38699341,405.16710297)
\curveto(278.67698562,404.86709623)(278.92198537,404.49209661)(279.12199341,404.04210297)
\curveto(279.17198512,403.92209718)(279.21198508,403.7970973)(279.24199341,403.66710297)
\curveto(279.28198501,403.53709756)(279.32198497,403.4020977)(279.36199341,403.26210297)
\curveto(279.38198491,403.19209791)(279.3919849,403.12209798)(279.39199341,403.05210297)
\curveto(279.40198489,402.99209811)(279.41698488,402.92209818)(279.43699341,402.84210297)
\curveto(279.45698484,402.79209831)(279.46198483,402.73709836)(279.45199341,402.67710297)
\curveto(279.45198484,402.61709848)(279.45698484,402.55709854)(279.46699341,402.49710297)
\lineto(279.46699341,402.39210297)
\moveto(277.24699341,400.98210297)
\curveto(277.27698702,401.08210002)(277.30198699,401.20709989)(277.32199341,401.35710297)
\curveto(277.35198694,401.50709959)(277.36698693,401.65709944)(277.36699341,401.80710297)
\curveto(277.37698692,401.96709913)(277.37698692,402.12209898)(277.36699341,402.27210297)
\curveto(277.36698693,402.43209867)(277.35198694,402.56709853)(277.32199341,402.67710297)
\curveto(277.291987,402.77709832)(277.27198702,402.87209823)(277.26199341,402.96210297)
\curveto(277.25198704,403.05209805)(277.22698707,403.13709796)(277.18699341,403.21710297)
\curveto(277.04698725,403.56709753)(276.84698745,403.86209724)(276.58699341,404.10210297)
\curveto(276.33698796,404.35209675)(275.96698833,404.47709662)(275.47699341,404.47710297)
\curveto(275.43698886,404.47709662)(275.40198889,404.47209663)(275.37199341,404.46210297)
\lineto(275.26699341,404.46210297)
\curveto(275.1969891,404.44209666)(275.13198916,404.42209668)(275.07199341,404.40210297)
\curveto(275.01198928,404.39209671)(274.95198934,404.37709672)(274.89199341,404.35710297)
\curveto(274.60198969,404.22709687)(274.38198991,404.04209706)(274.23199341,403.80210297)
\curveto(274.08199021,403.57209753)(273.95699034,403.30709779)(273.85699341,403.00710297)
\curveto(273.82699047,402.92709817)(273.80699049,402.84209826)(273.79699341,402.75210297)
\curveto(273.7969905,402.67209843)(273.78699051,402.59209851)(273.76699341,402.51210297)
\curveto(273.75699054,402.48209862)(273.75199054,402.43209867)(273.75199341,402.36210297)
\curveto(273.74199055,402.32209878)(273.73699056,402.28209882)(273.73699341,402.24210297)
\curveto(273.74699055,402.2020989)(273.74699055,402.16209894)(273.73699341,402.12210297)
\curveto(273.71699058,402.04209906)(273.71199058,401.93209917)(273.72199341,401.79210297)
\curveto(273.73199056,401.65209945)(273.74699055,401.55209955)(273.76699341,401.49210297)
\curveto(273.78699051,401.4020997)(273.7969905,401.31709978)(273.79699341,401.23710297)
\curveto(273.80699049,401.15709994)(273.82699047,401.07710002)(273.85699341,400.99710297)
\curveto(273.94699035,400.71710038)(274.05199024,400.47210063)(274.17199341,400.26210297)
\curveto(274.30198999,400.06210104)(274.48198981,399.89210121)(274.71199341,399.75210297)
\curveto(274.87198942,399.65210145)(275.03698926,399.58210152)(275.20699341,399.54210297)
\curveto(275.22698907,399.54210156)(275.24698905,399.53710156)(275.26699341,399.52710297)
\lineto(275.35699341,399.52710297)
\curveto(275.38698891,399.51710158)(275.43698886,399.50710159)(275.50699341,399.49710297)
\curveto(275.57698872,399.4971016)(275.63698866,399.5021016)(275.68699341,399.51210297)
\curveto(275.78698851,399.53210157)(275.87698842,399.54710155)(275.95699341,399.55710297)
\curveto(276.04698825,399.57710152)(276.13198816,399.6021015)(276.21199341,399.63210297)
\curveto(276.4919878,399.76210134)(276.70698759,399.94210116)(276.85699341,400.17210297)
\curveto(277.01698728,400.4021007)(277.14698715,400.67210043)(277.24699341,400.98210297)
}
}
{
\newrgbcolor{curcolor}{0 0 0}
\pscustom[linestyle=none,fillstyle=solid,fillcolor=curcolor]
{
\newpath
\moveto(281.34691528,408.82710297)
\lineto(282.44191528,408.82710297)
\curveto(282.5419128,408.82709227)(282.6369127,408.82209228)(282.72691528,408.81210297)
\curveto(282.81691252,408.8020923)(282.88691245,408.77209233)(282.93691528,408.72210297)
\curveto(282.99691234,408.65209245)(283.02691231,408.55709254)(283.02691528,408.43710297)
\curveto(283.0369123,408.32709277)(283.0419123,408.21209289)(283.04191528,408.09210297)
\lineto(283.04191528,406.75710297)
\lineto(283.04191528,401.37210297)
\lineto(283.04191528,399.07710297)
\lineto(283.04191528,398.65710297)
\curveto(283.05191229,398.50710259)(283.03191231,398.39210271)(282.98191528,398.31210297)
\curveto(282.93191241,398.23210287)(282.8419125,398.17710292)(282.71191528,398.14710297)
\curveto(282.65191269,398.12710297)(282.58191276,398.12210298)(282.50191528,398.13210297)
\curveto(282.43191291,398.14210296)(282.36191298,398.14710295)(282.29191528,398.14710297)
\lineto(281.57191528,398.14710297)
\curveto(281.46191388,398.14710295)(281.36191398,398.15210295)(281.27191528,398.16210297)
\curveto(281.18191416,398.17210293)(281.10691423,398.2021029)(281.04691528,398.25210297)
\curveto(280.98691435,398.3021028)(280.95191439,398.37710272)(280.94191528,398.47710297)
\lineto(280.94191528,398.80710297)
\lineto(280.94191528,400.14210297)
\lineto(280.94191528,405.76710297)
\lineto(280.94191528,407.80710297)
\curveto(280.9419144,407.93709316)(280.9369144,408.09209301)(280.92691528,408.27210297)
\curveto(280.92691441,408.45209265)(280.95191439,408.58209252)(281.00191528,408.66210297)
\curveto(281.02191432,408.7020924)(281.04691429,408.73209237)(281.07691528,408.75210297)
\lineto(281.19691528,408.81210297)
\curveto(281.21691412,408.81209229)(281.2419141,408.81209229)(281.27191528,408.81210297)
\curveto(281.30191404,408.82209228)(281.32691401,408.82709227)(281.34691528,408.82710297)
}
}
{
\newrgbcolor{curcolor}{0 0 0}
\pscustom[linestyle=none,fillstyle=solid,fillcolor=curcolor]
{
\newpath
\moveto(286.80410278,408.72210297)
\curveto(286.87409983,408.64209246)(286.9090998,408.52209258)(286.90910278,408.36210297)
\lineto(286.90910278,407.89710297)
\lineto(286.90910278,407.49210297)
\curveto(286.9090998,407.35209375)(286.87409983,407.25709384)(286.80410278,407.20710297)
\curveto(286.74409996,407.15709394)(286.66410004,407.12709397)(286.56410278,407.11710297)
\curveto(286.47410023,407.10709399)(286.37410033,407.102094)(286.26410278,407.10210297)
\lineto(285.42410278,407.10210297)
\curveto(285.31410139,407.102094)(285.21410149,407.10709399)(285.12410278,407.11710297)
\curveto(285.04410166,407.12709397)(284.97410173,407.15709394)(284.91410278,407.20710297)
\curveto(284.87410183,407.23709386)(284.84410186,407.29209381)(284.82410278,407.37210297)
\curveto(284.81410189,407.46209364)(284.8041019,407.55709354)(284.79410278,407.65710297)
\lineto(284.79410278,407.98710297)
\curveto(284.8041019,408.097093)(284.8091019,408.19209291)(284.80910278,408.27210297)
\lineto(284.80910278,408.48210297)
\curveto(284.81910189,408.55209255)(284.83910187,408.61209249)(284.86910278,408.66210297)
\curveto(284.88910182,408.7020924)(284.91410179,408.73209237)(284.94410278,408.75210297)
\lineto(285.06410278,408.81210297)
\curveto(285.08410162,408.81209229)(285.1091016,408.81209229)(285.13910278,408.81210297)
\curveto(285.16910154,408.82209228)(285.19410151,408.82709227)(285.21410278,408.82710297)
\lineto(286.30910278,408.82710297)
\curveto(286.4091003,408.82709227)(286.5041002,408.82209228)(286.59410278,408.81210297)
\curveto(286.68410002,408.8020923)(286.75409995,408.77209233)(286.80410278,408.72210297)
\moveto(286.90910278,398.95710297)
\curveto(286.9090998,398.75710234)(286.9040998,398.58710251)(286.89410278,398.44710297)
\curveto(286.88409982,398.30710279)(286.79409991,398.21210289)(286.62410278,398.16210297)
\curveto(286.56410014,398.14210296)(286.49910021,398.13210297)(286.42910278,398.13210297)
\curveto(286.35910035,398.14210296)(286.28410042,398.14710295)(286.20410278,398.14710297)
\lineto(285.36410278,398.14710297)
\curveto(285.27410143,398.14710295)(285.18410152,398.15210295)(285.09410278,398.16210297)
\curveto(285.01410169,398.17210293)(284.95410175,398.2021029)(284.91410278,398.25210297)
\curveto(284.85410185,398.32210278)(284.81910189,398.40710269)(284.80910278,398.50710297)
\lineto(284.80910278,398.85210297)
\lineto(284.80910278,405.18210297)
\lineto(284.80910278,405.48210297)
\curveto(284.8091019,405.58209552)(284.82910188,405.66209544)(284.86910278,405.72210297)
\curveto(284.92910178,405.79209531)(285.01410169,405.83709526)(285.12410278,405.85710297)
\curveto(285.14410156,405.86709523)(285.16910154,405.86709523)(285.19910278,405.85710297)
\curveto(285.23910147,405.85709524)(285.26910144,405.86209524)(285.28910278,405.87210297)
\lineto(286.03910278,405.87210297)
\lineto(286.23410278,405.87210297)
\curveto(286.31410039,405.88209522)(286.37910033,405.88209522)(286.42910278,405.87210297)
\lineto(286.54910278,405.87210297)
\curveto(286.6091001,405.85209525)(286.66410004,405.83709526)(286.71410278,405.82710297)
\curveto(286.76409994,405.81709528)(286.8040999,405.78709531)(286.83410278,405.73710297)
\curveto(286.87409983,405.68709541)(286.89409981,405.61709548)(286.89410278,405.52710297)
\curveto(286.9040998,405.43709566)(286.9090998,405.34209576)(286.90910278,405.24210297)
\lineto(286.90910278,398.95710297)
}
}
{
\newrgbcolor{curcolor}{0 0 0}
\pscustom[linestyle=none,fillstyle=solid,fillcolor=curcolor]
{
\newpath
\moveto(292.14129028,406.08210297)
\curveto(292.95128512,406.102095)(293.62628445,405.98209512)(294.16629028,405.72210297)
\curveto(294.71628336,405.46209564)(295.15128292,405.09209601)(295.47129028,404.61210297)
\curveto(295.63128244,404.37209673)(295.75128232,404.097097)(295.83129028,403.78710297)
\curveto(295.85128222,403.73709736)(295.86628221,403.67209743)(295.87629028,403.59210297)
\curveto(295.89628218,403.51209759)(295.89628218,403.44209766)(295.87629028,403.38210297)
\curveto(295.83628224,403.27209783)(295.76628231,403.20709789)(295.66629028,403.18710297)
\curveto(295.56628251,403.17709792)(295.44628263,403.17209793)(295.30629028,403.17210297)
\lineto(294.52629028,403.17210297)
\lineto(294.24129028,403.17210297)
\curveto(294.15128392,403.17209793)(294.076284,403.19209791)(294.01629028,403.23210297)
\curveto(293.93628414,403.27209783)(293.88128419,403.33209777)(293.85129028,403.41210297)
\curveto(293.82128425,403.5020976)(293.78128429,403.59209751)(293.73129028,403.68210297)
\curveto(293.6712844,403.79209731)(293.60628447,403.89209721)(293.53629028,403.98210297)
\curveto(293.46628461,404.07209703)(293.38628469,404.15209695)(293.29629028,404.22210297)
\curveto(293.15628492,404.31209679)(293.00128507,404.38209672)(292.83129028,404.43210297)
\curveto(292.7712853,404.45209665)(292.71128536,404.46209664)(292.65129028,404.46210297)
\curveto(292.59128548,404.46209664)(292.53628554,404.47209663)(292.48629028,404.49210297)
\lineto(292.33629028,404.49210297)
\curveto(292.13628594,404.49209661)(291.9762861,404.47209663)(291.85629028,404.43210297)
\curveto(291.56628651,404.34209676)(291.33128674,404.2020969)(291.15129028,404.01210297)
\curveto(290.9712871,403.83209727)(290.82628725,403.61209749)(290.71629028,403.35210297)
\curveto(290.66628741,403.24209786)(290.62628745,403.12209798)(290.59629028,402.99210297)
\curveto(290.5762875,402.87209823)(290.55128752,402.74209836)(290.52129028,402.60210297)
\curveto(290.51128756,402.56209854)(290.50628757,402.52209858)(290.50629028,402.48210297)
\curveto(290.50628757,402.44209866)(290.50128757,402.4020987)(290.49129028,402.36210297)
\curveto(290.4712876,402.26209884)(290.46128761,402.12209898)(290.46129028,401.94210297)
\curveto(290.4712876,401.76209934)(290.48628759,401.62209948)(290.50629028,401.52210297)
\curveto(290.50628757,401.44209966)(290.51128756,401.38709971)(290.52129028,401.35710297)
\curveto(290.54128753,401.28709981)(290.55128752,401.21709988)(290.55129028,401.14710297)
\curveto(290.56128751,401.07710002)(290.5762875,401.00710009)(290.59629028,400.93710297)
\curveto(290.6762874,400.70710039)(290.7712873,400.4971006)(290.88129028,400.30710297)
\curveto(290.99128708,400.11710098)(291.13128694,399.95710114)(291.30129028,399.82710297)
\curveto(291.34128673,399.7971013)(291.40128667,399.76210134)(291.48129028,399.72210297)
\curveto(291.59128648,399.65210145)(291.70128637,399.60710149)(291.81129028,399.58710297)
\curveto(291.93128614,399.56710153)(292.076286,399.54710155)(292.24629028,399.52710297)
\lineto(292.33629028,399.52710297)
\curveto(292.3762857,399.52710157)(292.40628567,399.53210157)(292.42629028,399.54210297)
\lineto(292.56129028,399.54210297)
\curveto(292.63128544,399.56210154)(292.69628538,399.57710152)(292.75629028,399.58710297)
\curveto(292.82628525,399.60710149)(292.89128518,399.62710147)(292.95129028,399.64710297)
\curveto(293.25128482,399.77710132)(293.48128459,399.96710113)(293.64129028,400.21710297)
\curveto(293.68128439,400.26710083)(293.71628436,400.32210078)(293.74629028,400.38210297)
\curveto(293.7762843,400.45210065)(293.80128427,400.51210059)(293.82129028,400.56210297)
\curveto(293.86128421,400.67210043)(293.89628418,400.76710033)(293.92629028,400.84710297)
\curveto(293.95628412,400.93710016)(294.02628405,401.00710009)(294.13629028,401.05710297)
\curveto(294.22628385,401.0971)(294.3712837,401.11209999)(294.57129028,401.10210297)
\lineto(295.06629028,401.10210297)
\lineto(295.27629028,401.10210297)
\curveto(295.35628272,401.11209999)(295.42128265,401.10709999)(295.47129028,401.08710297)
\lineto(295.59129028,401.08710297)
\lineto(295.71129028,401.05710297)
\curveto(295.75128232,401.05710004)(295.78128229,401.04710005)(295.80129028,401.02710297)
\curveto(295.85128222,400.98710011)(295.88128219,400.92710017)(295.89129028,400.84710297)
\curveto(295.91128216,400.77710032)(295.91128216,400.7021004)(295.89129028,400.62210297)
\curveto(295.80128227,400.29210081)(295.69128238,399.9971011)(295.56129028,399.73710297)
\curveto(295.15128292,398.96710213)(294.49628358,398.43210267)(293.59629028,398.13210297)
\curveto(293.49628458,398.102103)(293.39128468,398.08210302)(293.28129028,398.07210297)
\curveto(293.1712849,398.05210305)(293.06128501,398.02710307)(292.95129028,397.99710297)
\curveto(292.89128518,397.98710311)(292.83128524,397.98210312)(292.77129028,397.98210297)
\curveto(292.71128536,397.98210312)(292.65128542,397.97710312)(292.59129028,397.96710297)
\lineto(292.42629028,397.96710297)
\curveto(292.3762857,397.94710315)(292.30128577,397.94210316)(292.20129028,397.95210297)
\curveto(292.10128597,397.95210315)(292.02628605,397.95710314)(291.97629028,397.96710297)
\curveto(291.89628618,397.98710311)(291.82128625,397.9971031)(291.75129028,397.99710297)
\curveto(291.69128638,397.98710311)(291.62628645,397.99210311)(291.55629028,398.01210297)
\lineto(291.40629028,398.04210297)
\curveto(291.35628672,398.04210306)(291.30628677,398.04710305)(291.25629028,398.05710297)
\curveto(291.14628693,398.08710301)(291.04128703,398.11710298)(290.94129028,398.14710297)
\curveto(290.84128723,398.17710292)(290.74628733,398.21210289)(290.65629028,398.25210297)
\curveto(290.18628789,398.45210265)(289.79128828,398.70710239)(289.47129028,399.01710297)
\curveto(289.15128892,399.33710176)(288.89128918,399.73210137)(288.69129028,400.20210297)
\curveto(288.64128943,400.29210081)(288.60128947,400.38710071)(288.57129028,400.48710297)
\lineto(288.48129028,400.81710297)
\curveto(288.4712896,400.85710024)(288.46628961,400.89210021)(288.46629028,400.92210297)
\curveto(288.46628961,400.96210014)(288.45628962,401.00710009)(288.43629028,401.05710297)
\curveto(288.41628966,401.12709997)(288.40628967,401.1970999)(288.40629028,401.26710297)
\curveto(288.40628967,401.34709975)(288.39628968,401.42209968)(288.37629028,401.49210297)
\lineto(288.37629028,401.74710297)
\curveto(288.35628972,401.7970993)(288.34628973,401.85209925)(288.34629028,401.91210297)
\curveto(288.34628973,401.98209912)(288.35628972,402.04209906)(288.37629028,402.09210297)
\curveto(288.38628969,402.14209896)(288.38628969,402.18709891)(288.37629028,402.22710297)
\curveto(288.36628971,402.26709883)(288.36628971,402.30709879)(288.37629028,402.34710297)
\curveto(288.39628968,402.41709868)(288.40128967,402.48209862)(288.39129028,402.54210297)
\curveto(288.39128968,402.6020985)(288.40128967,402.66209844)(288.42129028,402.72210297)
\curveto(288.4712896,402.9020982)(288.51128956,403.07209803)(288.54129028,403.23210297)
\curveto(288.5712895,403.4020977)(288.61628946,403.56709753)(288.67629028,403.72710297)
\curveto(288.89628918,404.23709686)(289.1712889,404.66209644)(289.50129028,405.00210297)
\curveto(289.84128823,405.34209576)(290.2712878,405.61709548)(290.79129028,405.82710297)
\curveto(290.93128714,405.88709521)(291.076287,405.92709517)(291.22629028,405.94710297)
\curveto(291.3762867,405.97709512)(291.53128654,406.01209509)(291.69129028,406.05210297)
\curveto(291.7712863,406.06209504)(291.84628623,406.06709503)(291.91629028,406.06710297)
\curveto(291.98628609,406.06709503)(292.06128601,406.07209503)(292.14129028,406.08210297)
}
}
{
\newrgbcolor{curcolor}{0 0 0}
\pscustom[linestyle=none,fillstyle=solid,fillcolor=curcolor]
{
\newpath
\moveto(304.23457153,398.73210297)
\curveto(304.25456368,398.62210248)(304.26456367,398.51210259)(304.26457153,398.40210297)
\curveto(304.27456366,398.29210281)(304.22456371,398.21710288)(304.11457153,398.17710297)
\curveto(304.05456388,398.14710295)(303.98456395,398.13210297)(303.90457153,398.13210297)
\lineto(303.66457153,398.13210297)
\lineto(302.85457153,398.13210297)
\lineto(302.58457153,398.13210297)
\curveto(302.50456543,398.14210296)(302.4395655,398.16710293)(302.38957153,398.20710297)
\curveto(302.31956562,398.24710285)(302.26456567,398.3021028)(302.22457153,398.37210297)
\curveto(302.19456574,398.45210265)(302.14956579,398.51710258)(302.08957153,398.56710297)
\curveto(302.06956587,398.58710251)(302.04456589,398.6021025)(302.01457153,398.61210297)
\curveto(301.98456595,398.63210247)(301.94456599,398.63710246)(301.89457153,398.62710297)
\curveto(301.84456609,398.60710249)(301.79456614,398.58210252)(301.74457153,398.55210297)
\curveto(301.70456623,398.52210258)(301.65956628,398.4971026)(301.60957153,398.47710297)
\curveto(301.55956638,398.43710266)(301.50456643,398.4021027)(301.44457153,398.37210297)
\lineto(301.26457153,398.28210297)
\curveto(301.1345668,398.22210288)(300.99956694,398.17210293)(300.85957153,398.13210297)
\curveto(300.71956722,398.102103)(300.57456736,398.06710303)(300.42457153,398.02710297)
\curveto(300.35456758,398.00710309)(300.28456765,397.9971031)(300.21457153,397.99710297)
\curveto(300.15456778,397.98710311)(300.08956785,397.97710312)(300.01957153,397.96710297)
\lineto(299.92957153,397.96710297)
\curveto(299.89956804,397.95710314)(299.86956807,397.95210315)(299.83957153,397.95210297)
\lineto(299.67457153,397.95210297)
\curveto(299.57456836,397.93210317)(299.47456846,397.93210317)(299.37457153,397.95210297)
\lineto(299.23957153,397.95210297)
\curveto(299.16956877,397.97210313)(299.09956884,397.98210312)(299.02957153,397.98210297)
\curveto(298.96956897,397.97210313)(298.90956903,397.97710312)(298.84957153,397.99710297)
\curveto(298.74956919,398.01710308)(298.65456928,398.03710306)(298.56457153,398.05710297)
\curveto(298.47456946,398.06710303)(298.38956955,398.09210301)(298.30957153,398.13210297)
\curveto(298.01956992,398.24210286)(297.76957017,398.38210272)(297.55957153,398.55210297)
\curveto(297.35957058,398.73210237)(297.19957074,398.96710213)(297.07957153,399.25710297)
\curveto(297.04957089,399.32710177)(297.01957092,399.4021017)(296.98957153,399.48210297)
\curveto(296.96957097,399.56210154)(296.94957099,399.64710145)(296.92957153,399.73710297)
\curveto(296.90957103,399.78710131)(296.89957104,399.83710126)(296.89957153,399.88710297)
\curveto(296.90957103,399.93710116)(296.90957103,399.98710111)(296.89957153,400.03710297)
\curveto(296.88957105,400.06710103)(296.87957106,400.12710097)(296.86957153,400.21710297)
\curveto(296.86957107,400.31710078)(296.87457106,400.38710071)(296.88457153,400.42710297)
\curveto(296.90457103,400.52710057)(296.91457102,400.61210049)(296.91457153,400.68210297)
\lineto(297.00457153,401.01210297)
\curveto(297.0345709,401.13209997)(297.07457086,401.23709986)(297.12457153,401.32710297)
\curveto(297.29457064,401.61709948)(297.48957045,401.83709926)(297.70957153,401.98710297)
\curveto(297.92957001,402.13709896)(298.20956973,402.26709883)(298.54957153,402.37710297)
\curveto(298.67956926,402.42709867)(298.81456912,402.46209864)(298.95457153,402.48210297)
\curveto(299.09456884,402.5020986)(299.2345687,402.52709857)(299.37457153,402.55710297)
\curveto(299.45456848,402.57709852)(299.5395684,402.58709851)(299.62957153,402.58710297)
\curveto(299.71956822,402.5970985)(299.80956813,402.61209849)(299.89957153,402.63210297)
\curveto(299.96956797,402.65209845)(300.0395679,402.65709844)(300.10957153,402.64710297)
\curveto(300.17956776,402.64709845)(300.25456768,402.65709844)(300.33457153,402.67710297)
\curveto(300.40456753,402.6970984)(300.47456746,402.70709839)(300.54457153,402.70710297)
\curveto(300.61456732,402.70709839)(300.68956725,402.71709838)(300.76957153,402.73710297)
\curveto(300.97956696,402.78709831)(301.16956677,402.82709827)(301.33957153,402.85710297)
\curveto(301.51956642,402.8970982)(301.67956626,402.98709811)(301.81957153,403.12710297)
\curveto(301.90956603,403.21709788)(301.96956597,403.31709778)(301.99957153,403.42710297)
\curveto(302.00956593,403.45709764)(302.00956593,403.48209762)(301.99957153,403.50210297)
\curveto(301.99956594,403.52209758)(302.00456593,403.54209756)(302.01457153,403.56210297)
\curveto(302.02456591,403.58209752)(302.02956591,403.61209749)(302.02957153,403.65210297)
\lineto(302.02957153,403.74210297)
\lineto(301.99957153,403.86210297)
\curveto(301.99956594,403.9020972)(301.99456594,403.93709716)(301.98457153,403.96710297)
\curveto(301.88456605,404.26709683)(301.67456626,404.47209663)(301.35457153,404.58210297)
\curveto(301.26456667,404.61209649)(301.15456678,404.63209647)(301.02457153,404.64210297)
\curveto(300.90456703,404.66209644)(300.77956716,404.66709643)(300.64957153,404.65710297)
\curveto(300.51956742,404.65709644)(300.39456754,404.64709645)(300.27457153,404.62710297)
\curveto(300.15456778,404.60709649)(300.04956789,404.58209652)(299.95957153,404.55210297)
\curveto(299.89956804,404.53209657)(299.8395681,404.5020966)(299.77957153,404.46210297)
\curveto(299.72956821,404.43209667)(299.67956826,404.3970967)(299.62957153,404.35710297)
\curveto(299.57956836,404.31709678)(299.52456841,404.26209684)(299.46457153,404.19210297)
\curveto(299.41456852,404.12209698)(299.37956856,404.05709704)(299.35957153,403.99710297)
\curveto(299.30956863,403.8970972)(299.26456867,403.8020973)(299.22457153,403.71210297)
\curveto(299.19456874,403.62209748)(299.12456881,403.56209754)(299.01457153,403.53210297)
\curveto(298.934569,403.51209759)(298.84956909,403.5020976)(298.75957153,403.50210297)
\lineto(298.48957153,403.50210297)
\lineto(297.91957153,403.50210297)
\curveto(297.86957007,403.5020976)(297.81957012,403.4970976)(297.76957153,403.48710297)
\curveto(297.71957022,403.48709761)(297.67457026,403.49209761)(297.63457153,403.50210297)
\lineto(297.49957153,403.50210297)
\curveto(297.47957046,403.51209759)(297.45457048,403.51709758)(297.42457153,403.51710297)
\curveto(297.39457054,403.51709758)(297.36957057,403.52709757)(297.34957153,403.54710297)
\curveto(297.26957067,403.56709753)(297.21457072,403.63209747)(297.18457153,403.74210297)
\curveto(297.17457076,403.79209731)(297.17457076,403.84209726)(297.18457153,403.89210297)
\curveto(297.19457074,403.94209716)(297.20457073,403.98709711)(297.21457153,404.02710297)
\curveto(297.24457069,404.13709696)(297.27457066,404.23709686)(297.30457153,404.32710297)
\curveto(297.34457059,404.42709667)(297.38957055,404.51709658)(297.43957153,404.59710297)
\lineto(297.52957153,404.74710297)
\lineto(297.61957153,404.89710297)
\curveto(297.69957024,405.00709609)(297.79957014,405.11209599)(297.91957153,405.21210297)
\curveto(297.93957,405.22209588)(297.96956997,405.24709585)(298.00957153,405.28710297)
\curveto(298.05956988,405.32709577)(298.10456983,405.36209574)(298.14457153,405.39210297)
\curveto(298.18456975,405.42209568)(298.22956971,405.45209565)(298.27957153,405.48210297)
\curveto(298.44956949,405.59209551)(298.62956931,405.67709542)(298.81957153,405.73710297)
\curveto(299.00956893,405.80709529)(299.20456873,405.87209523)(299.40457153,405.93210297)
\curveto(299.52456841,405.96209514)(299.64956829,405.98209512)(299.77957153,405.99210297)
\curveto(299.90956803,406.0020951)(300.0395679,406.02209508)(300.16957153,406.05210297)
\curveto(300.20956773,406.06209504)(300.26956767,406.06209504)(300.34957153,406.05210297)
\curveto(300.4395675,406.04209506)(300.49456744,406.04709505)(300.51457153,406.06710297)
\curveto(300.92456701,406.07709502)(301.31456662,406.06209504)(301.68457153,406.02210297)
\curveto(302.06456587,405.98209512)(302.40456553,405.90709519)(302.70457153,405.79710297)
\curveto(303.01456492,405.68709541)(303.27956466,405.53709556)(303.49957153,405.34710297)
\curveto(303.71956422,405.16709593)(303.88956405,404.93209617)(304.00957153,404.64210297)
\curveto(304.07956386,404.47209663)(304.11956382,404.27709682)(304.12957153,404.05710297)
\curveto(304.1395638,403.83709726)(304.14456379,403.61209749)(304.14457153,403.38210297)
\lineto(304.14457153,400.03710297)
\lineto(304.14457153,399.45210297)
\curveto(304.14456379,399.26210184)(304.16456377,399.08710201)(304.20457153,398.92710297)
\curveto(304.21456372,398.8971022)(304.21956372,398.86210224)(304.21957153,398.82210297)
\curveto(304.21956372,398.79210231)(304.22456371,398.76210234)(304.23457153,398.73210297)
\moveto(302.02957153,401.04210297)
\curveto(302.0395659,401.09210001)(302.04456589,401.14709995)(302.04457153,401.20710297)
\curveto(302.04456589,401.27709982)(302.0395659,401.33709976)(302.02957153,401.38710297)
\curveto(302.00956593,401.44709965)(301.99956594,401.5020996)(301.99957153,401.55210297)
\curveto(301.99956594,401.6020995)(301.97956596,401.64209946)(301.93957153,401.67210297)
\curveto(301.88956605,401.71209939)(301.81456612,401.73209937)(301.71457153,401.73210297)
\curveto(301.67456626,401.72209938)(301.6395663,401.71209939)(301.60957153,401.70210297)
\curveto(301.57956636,401.7020994)(301.54456639,401.6970994)(301.50457153,401.68710297)
\curveto(301.4345665,401.66709943)(301.35956658,401.65209945)(301.27957153,401.64210297)
\curveto(301.19956674,401.63209947)(301.11956682,401.61709948)(301.03957153,401.59710297)
\curveto(301.00956693,401.58709951)(300.96456697,401.58209952)(300.90457153,401.58210297)
\curveto(300.77456716,401.55209955)(300.64456729,401.53209957)(300.51457153,401.52210297)
\curveto(300.38456755,401.51209959)(300.25956768,401.48709961)(300.13957153,401.44710297)
\curveto(300.05956788,401.42709967)(299.98456795,401.40709969)(299.91457153,401.38710297)
\curveto(299.84456809,401.37709972)(299.77456816,401.35709974)(299.70457153,401.32710297)
\curveto(299.49456844,401.23709986)(299.31456862,401.1021)(299.16457153,400.92210297)
\curveto(299.02456891,400.74210036)(298.97456896,400.49210061)(299.01457153,400.17210297)
\curveto(299.0345689,400.0021011)(299.08956885,399.86210124)(299.17957153,399.75210297)
\curveto(299.24956869,399.64210146)(299.35456858,399.55210155)(299.49457153,399.48210297)
\curveto(299.6345683,399.42210168)(299.78456815,399.37710172)(299.94457153,399.34710297)
\curveto(300.11456782,399.31710178)(300.28956765,399.30710179)(300.46957153,399.31710297)
\curveto(300.65956728,399.33710176)(300.8345671,399.37210173)(300.99457153,399.42210297)
\curveto(301.25456668,399.5021016)(301.45956648,399.62710147)(301.60957153,399.79710297)
\curveto(301.75956618,399.97710112)(301.87456606,400.1971009)(301.95457153,400.45710297)
\curveto(301.97456596,400.52710057)(301.98456595,400.5971005)(301.98457153,400.66710297)
\curveto(301.99456594,400.74710035)(302.00956593,400.82710027)(302.02957153,400.90710297)
\lineto(302.02957153,401.04210297)
}
}
{
\newrgbcolor{curcolor}{0 0 0}
\pscustom[linestyle=none,fillstyle=solid,fillcolor=curcolor]
{
\newpath
\moveto(313.38785278,398.98710297)
\lineto(313.38785278,398.56710297)
\curveto(313.38784441,398.43710266)(313.35784444,398.33210277)(313.29785278,398.25210297)
\curveto(313.24784455,398.2021029)(313.18284462,398.16710293)(313.10285278,398.14710297)
\curveto(313.02284478,398.13710296)(312.93284487,398.13210297)(312.83285278,398.13210297)
\lineto(312.00785278,398.13210297)
\lineto(311.72285278,398.13210297)
\curveto(311.64284616,398.14210296)(311.57784622,398.16710293)(311.52785278,398.20710297)
\curveto(311.45784634,398.25710284)(311.41784638,398.32210278)(311.40785278,398.40210297)
\curveto(311.3978464,398.48210262)(311.37784642,398.56210254)(311.34785278,398.64210297)
\curveto(311.32784647,398.66210244)(311.30784649,398.67710242)(311.28785278,398.68710297)
\curveto(311.27784652,398.70710239)(311.26284654,398.72710237)(311.24285278,398.74710297)
\curveto(311.13284667,398.74710235)(311.05284675,398.72210238)(311.00285278,398.67210297)
\lineto(310.85285278,398.52210297)
\curveto(310.78284702,398.47210263)(310.71784708,398.42710267)(310.65785278,398.38710297)
\curveto(310.5978472,398.35710274)(310.53284727,398.31710278)(310.46285278,398.26710297)
\curveto(310.42284738,398.24710285)(310.37784742,398.22710287)(310.32785278,398.20710297)
\curveto(310.28784751,398.18710291)(310.24284756,398.16710293)(310.19285278,398.14710297)
\curveto(310.05284775,398.097103)(309.9028479,398.05210305)(309.74285278,398.01210297)
\curveto(309.69284811,397.99210311)(309.64784815,397.98210312)(309.60785278,397.98210297)
\curveto(309.56784823,397.98210312)(309.52784827,397.97710312)(309.48785278,397.96710297)
\lineto(309.35285278,397.96710297)
\curveto(309.32284848,397.95710314)(309.28284852,397.95210315)(309.23285278,397.95210297)
\lineto(309.09785278,397.95210297)
\curveto(309.03784876,397.93210317)(308.94784885,397.92710317)(308.82785278,397.93710297)
\curveto(308.70784909,397.93710316)(308.62284918,397.94710315)(308.57285278,397.96710297)
\curveto(308.5028493,397.98710311)(308.43784936,397.9971031)(308.37785278,397.99710297)
\curveto(308.32784947,397.98710311)(308.27284953,397.99210311)(308.21285278,398.01210297)
\lineto(307.85285278,398.13210297)
\curveto(307.74285006,398.16210294)(307.63285017,398.2021029)(307.52285278,398.25210297)
\curveto(307.17285063,398.4021027)(306.85785094,398.63210247)(306.57785278,398.94210297)
\curveto(306.30785149,399.26210184)(306.09285171,399.5971015)(305.93285278,399.94710297)
\curveto(305.88285192,400.05710104)(305.84285196,400.16210094)(305.81285278,400.26210297)
\curveto(305.78285202,400.37210073)(305.74785205,400.48210062)(305.70785278,400.59210297)
\curveto(305.6978521,400.63210047)(305.69285211,400.66710043)(305.69285278,400.69710297)
\curveto(305.69285211,400.73710036)(305.68285212,400.78210032)(305.66285278,400.83210297)
\curveto(305.64285216,400.91210019)(305.62285218,400.9971001)(305.60285278,401.08710297)
\curveto(305.59285221,401.18709991)(305.57785222,401.28709981)(305.55785278,401.38710297)
\curveto(305.54785225,401.41709968)(305.54285226,401.45209965)(305.54285278,401.49210297)
\curveto(305.55285225,401.53209957)(305.55285225,401.56709953)(305.54285278,401.59710297)
\lineto(305.54285278,401.73210297)
\curveto(305.54285226,401.78209932)(305.53785226,401.83209927)(305.52785278,401.88210297)
\curveto(305.51785228,401.93209917)(305.51285229,401.98709911)(305.51285278,402.04710297)
\curveto(305.51285229,402.11709898)(305.51785228,402.17209893)(305.52785278,402.21210297)
\curveto(305.53785226,402.26209884)(305.54285226,402.30709879)(305.54285278,402.34710297)
\lineto(305.54285278,402.49710297)
\curveto(305.55285225,402.54709855)(305.55285225,402.59209851)(305.54285278,402.63210297)
\curveto(305.54285226,402.68209842)(305.55285225,402.73209837)(305.57285278,402.78210297)
\curveto(305.59285221,402.89209821)(305.60785219,402.9970981)(305.61785278,403.09710297)
\curveto(305.63785216,403.1970979)(305.66285214,403.2970978)(305.69285278,403.39710297)
\curveto(305.73285207,403.51709758)(305.76785203,403.63209747)(305.79785278,403.74210297)
\curveto(305.82785197,403.85209725)(305.86785193,403.96209714)(305.91785278,404.07210297)
\curveto(306.05785174,404.37209673)(306.23285157,404.65709644)(306.44285278,404.92710297)
\curveto(306.46285134,404.95709614)(306.48785131,404.98209612)(306.51785278,405.00210297)
\curveto(306.55785124,405.03209607)(306.58785121,405.06209604)(306.60785278,405.09210297)
\curveto(306.64785115,405.14209596)(306.68785111,405.18709591)(306.72785278,405.22710297)
\curveto(306.76785103,405.26709583)(306.81285099,405.30709579)(306.86285278,405.34710297)
\curveto(306.9028509,405.36709573)(306.93785086,405.39209571)(306.96785278,405.42210297)
\curveto(306.9978508,405.46209564)(307.03285077,405.49209561)(307.07285278,405.51210297)
\curveto(307.32285048,405.68209542)(307.61285019,405.82209528)(307.94285278,405.93210297)
\curveto(308.01284979,405.95209515)(308.08284972,405.96709513)(308.15285278,405.97710297)
\curveto(308.23284957,405.98709511)(308.31284949,406.0020951)(308.39285278,406.02210297)
\curveto(308.46284934,406.04209506)(308.55284925,406.05209505)(308.66285278,406.05210297)
\curveto(308.77284903,406.06209504)(308.88284892,406.06709503)(308.99285278,406.06710297)
\curveto(309.1028487,406.06709503)(309.20784859,406.06209504)(309.30785278,406.05210297)
\curveto(309.41784838,406.04209506)(309.50784829,406.02709507)(309.57785278,406.00710297)
\curveto(309.72784807,405.95709514)(309.87284793,405.91209519)(310.01285278,405.87210297)
\curveto(310.15284765,405.83209527)(310.28284752,405.77709532)(310.40285278,405.70710297)
\curveto(310.47284733,405.65709544)(310.53784726,405.60709549)(310.59785278,405.55710297)
\curveto(310.65784714,405.51709558)(310.72284708,405.47209563)(310.79285278,405.42210297)
\curveto(310.83284697,405.39209571)(310.88784691,405.35209575)(310.95785278,405.30210297)
\curveto(311.03784676,405.25209585)(311.11284669,405.25209585)(311.18285278,405.30210297)
\curveto(311.22284658,405.32209578)(311.24284656,405.35709574)(311.24285278,405.40710297)
\curveto(311.24284656,405.45709564)(311.25284655,405.50709559)(311.27285278,405.55710297)
\lineto(311.27285278,405.70710297)
\curveto(311.28284652,405.73709536)(311.28784651,405.77209533)(311.28785278,405.81210297)
\lineto(311.28785278,405.93210297)
\lineto(311.28785278,407.97210297)
\curveto(311.28784651,408.08209302)(311.28284652,408.2020929)(311.27285278,408.33210297)
\curveto(311.27284653,408.47209263)(311.2978465,408.57709252)(311.34785278,408.64710297)
\curveto(311.38784641,408.72709237)(311.46284634,408.77709232)(311.57285278,408.79710297)
\curveto(311.59284621,408.80709229)(311.61284619,408.80709229)(311.63285278,408.79710297)
\curveto(311.65284615,408.7970923)(311.67284613,408.8020923)(311.69285278,408.81210297)
\lineto(312.75785278,408.81210297)
\curveto(312.87784492,408.81209229)(312.98784481,408.80709229)(313.08785278,408.79710297)
\curveto(313.18784461,408.78709231)(313.26284454,408.74709235)(313.31285278,408.67710297)
\curveto(313.36284444,408.5970925)(313.38784441,408.49209261)(313.38785278,408.36210297)
\lineto(313.38785278,408.00210297)
\lineto(313.38785278,398.98710297)
\moveto(311.34785278,401.92710297)
\curveto(311.35784644,401.96709913)(311.35784644,402.00709909)(311.34785278,402.04710297)
\lineto(311.34785278,402.18210297)
\curveto(311.34784645,402.28209882)(311.34284646,402.38209872)(311.33285278,402.48210297)
\curveto(311.32284648,402.58209852)(311.30784649,402.67209843)(311.28785278,402.75210297)
\curveto(311.26784653,402.86209824)(311.24784655,402.96209814)(311.22785278,403.05210297)
\curveto(311.21784658,403.14209796)(311.19284661,403.22709787)(311.15285278,403.30710297)
\curveto(311.01284679,403.66709743)(310.80784699,403.95209715)(310.53785278,404.16210297)
\curveto(310.27784752,404.37209673)(309.8978479,404.47709662)(309.39785278,404.47710297)
\curveto(309.33784846,404.47709662)(309.25784854,404.46709663)(309.15785278,404.44710297)
\curveto(309.07784872,404.42709667)(309.0028488,404.40709669)(308.93285278,404.38710297)
\curveto(308.87284893,404.37709672)(308.81284899,404.35709674)(308.75285278,404.32710297)
\curveto(308.48284932,404.21709688)(308.27284953,404.04709705)(308.12285278,403.81710297)
\curveto(307.97284983,403.58709751)(307.85284995,403.32709777)(307.76285278,403.03710297)
\curveto(307.73285007,402.93709816)(307.71285009,402.83709826)(307.70285278,402.73710297)
\curveto(307.69285011,402.63709846)(307.67285013,402.53209857)(307.64285278,402.42210297)
\lineto(307.64285278,402.21210297)
\curveto(307.62285018,402.12209898)(307.61785018,401.9970991)(307.62785278,401.83710297)
\curveto(307.63785016,401.68709941)(307.65285015,401.57709952)(307.67285278,401.50710297)
\lineto(307.67285278,401.41710297)
\curveto(307.68285012,401.3970997)(307.68785011,401.37709972)(307.68785278,401.35710297)
\curveto(307.70785009,401.27709982)(307.72285008,401.2020999)(307.73285278,401.13210297)
\curveto(307.75285005,401.06210004)(307.77285003,400.98710011)(307.79285278,400.90710297)
\curveto(307.96284984,400.38710071)(308.25284955,400.0021011)(308.66285278,399.75210297)
\curveto(308.79284901,399.66210144)(308.97284883,399.59210151)(309.20285278,399.54210297)
\curveto(309.24284856,399.53210157)(309.3028485,399.52710157)(309.38285278,399.52710297)
\curveto(309.41284839,399.51710158)(309.45784834,399.50710159)(309.51785278,399.49710297)
\curveto(309.58784821,399.4971016)(309.64284816,399.5021016)(309.68285278,399.51210297)
\curveto(309.76284804,399.53210157)(309.84284796,399.54710155)(309.92285278,399.55710297)
\curveto(310.0028478,399.56710153)(310.08284772,399.58710151)(310.16285278,399.61710297)
\curveto(310.41284739,399.72710137)(310.61284719,399.86710123)(310.76285278,400.03710297)
\curveto(310.91284689,400.20710089)(311.04284676,400.42210068)(311.15285278,400.68210297)
\curveto(311.19284661,400.77210033)(311.22284658,400.86210024)(311.24285278,400.95210297)
\curveto(311.26284654,401.05210005)(311.28284652,401.15709994)(311.30285278,401.26710297)
\curveto(311.31284649,401.31709978)(311.31284649,401.36209974)(311.30285278,401.40210297)
\curveto(311.3028465,401.45209965)(311.31284649,401.5020996)(311.33285278,401.55210297)
\curveto(311.34284646,401.58209952)(311.34784645,401.61709948)(311.34785278,401.65710297)
\lineto(311.34785278,401.79210297)
\lineto(311.34785278,401.92710297)
}
}
{
\newrgbcolor{curcolor}{0 0 0}
\pscustom[linestyle=none,fillstyle=solid,fillcolor=curcolor]
{
\newpath
\moveto(322.73777466,402.31710297)
\curveto(322.75776609,402.25709884)(322.76776608,402.17209893)(322.76777466,402.06210297)
\curveto(322.76776608,401.95209915)(322.75776609,401.86709923)(322.73777466,401.80710297)
\lineto(322.73777466,401.65710297)
\curveto(322.71776613,401.57709952)(322.70776614,401.4970996)(322.70777466,401.41710297)
\curveto(322.71776613,401.33709976)(322.71276613,401.25709984)(322.69277466,401.17710297)
\curveto(322.67276617,401.10709999)(322.65776619,401.04210006)(322.64777466,400.98210297)
\curveto(322.63776621,400.92210018)(322.62776622,400.85710024)(322.61777466,400.78710297)
\curveto(322.57776627,400.67710042)(322.5427663,400.56210054)(322.51277466,400.44210297)
\curveto(322.48276636,400.33210077)(322.4427664,400.22710087)(322.39277466,400.12710297)
\curveto(322.18276666,399.64710145)(321.90776694,399.25710184)(321.56777466,398.95710297)
\curveto(321.22776762,398.65710244)(320.81776803,398.40710269)(320.33777466,398.20710297)
\curveto(320.21776863,398.15710294)(320.09276875,398.12210298)(319.96277466,398.10210297)
\curveto(319.842769,398.07210303)(319.71776913,398.04210306)(319.58777466,398.01210297)
\curveto(319.53776931,397.99210311)(319.48276936,397.98210312)(319.42277466,397.98210297)
\curveto(319.36276948,397.98210312)(319.30776954,397.97710312)(319.25777466,397.96710297)
\lineto(319.15277466,397.96710297)
\curveto(319.12276972,397.95710314)(319.09276975,397.95210315)(319.06277466,397.95210297)
\curveto(319.01276983,397.94210316)(318.93276991,397.93710316)(318.82277466,397.93710297)
\curveto(318.71277013,397.92710317)(318.62777022,397.93210317)(318.56777466,397.95210297)
\lineto(318.41777466,397.95210297)
\curveto(318.36777048,397.96210314)(318.31277053,397.96710313)(318.25277466,397.96710297)
\curveto(318.20277064,397.95710314)(318.15277069,397.96210314)(318.10277466,397.98210297)
\curveto(318.06277078,397.99210311)(318.02277082,397.9971031)(317.98277466,397.99710297)
\curveto(317.95277089,397.9971031)(317.91277093,398.0021031)(317.86277466,398.01210297)
\curveto(317.76277108,398.04210306)(317.66277118,398.06710303)(317.56277466,398.08710297)
\curveto(317.46277138,398.10710299)(317.36777148,398.13710296)(317.27777466,398.17710297)
\curveto(317.15777169,398.21710288)(317.0427718,398.25710284)(316.93277466,398.29710297)
\curveto(316.83277201,398.33710276)(316.72777212,398.38710271)(316.61777466,398.44710297)
\curveto(316.26777258,398.65710244)(315.96777288,398.9021022)(315.71777466,399.18210297)
\curveto(315.46777338,399.46210164)(315.25777359,399.7971013)(315.08777466,400.18710297)
\curveto(315.03777381,400.27710082)(314.99777385,400.37210073)(314.96777466,400.47210297)
\curveto(314.9477739,400.57210053)(314.92277392,400.67710042)(314.89277466,400.78710297)
\curveto(314.87277397,400.83710026)(314.86277398,400.88210022)(314.86277466,400.92210297)
\curveto(314.86277398,400.96210014)(314.85277399,401.00710009)(314.83277466,401.05710297)
\curveto(314.81277403,401.13709996)(314.80277404,401.21709988)(314.80277466,401.29710297)
\curveto(314.80277404,401.38709971)(314.79277405,401.47209963)(314.77277466,401.55210297)
\curveto(314.76277408,401.6020995)(314.75777409,401.64709945)(314.75777466,401.68710297)
\lineto(314.75777466,401.82210297)
\curveto(314.73777411,401.88209922)(314.72777412,401.96709913)(314.72777466,402.07710297)
\curveto(314.73777411,402.18709891)(314.75277409,402.27209883)(314.77277466,402.33210297)
\lineto(314.77277466,402.43710297)
\curveto(314.78277406,402.48709861)(314.78277406,402.53709856)(314.77277466,402.58710297)
\curveto(314.77277407,402.64709845)(314.78277406,402.7020984)(314.80277466,402.75210297)
\curveto(314.81277403,402.8020983)(314.81777403,402.84709825)(314.81777466,402.88710297)
\curveto(314.81777403,402.93709816)(314.82777402,402.98709811)(314.84777466,403.03710297)
\curveto(314.88777396,403.16709793)(314.92277392,403.29209781)(314.95277466,403.41210297)
\curveto(314.98277386,403.54209756)(315.02277382,403.66709743)(315.07277466,403.78710297)
\curveto(315.25277359,404.1970969)(315.46777338,404.53709656)(315.71777466,404.80710297)
\curveto(315.96777288,405.08709601)(316.27277257,405.34209576)(316.63277466,405.57210297)
\curveto(316.73277211,405.62209548)(316.83777201,405.66709543)(316.94777466,405.70710297)
\curveto(317.05777179,405.74709535)(317.16777168,405.79209531)(317.27777466,405.84210297)
\curveto(317.40777144,405.89209521)(317.5427713,405.92709517)(317.68277466,405.94710297)
\curveto(317.82277102,405.96709513)(317.96777088,405.9970951)(318.11777466,406.03710297)
\curveto(318.19777065,406.04709505)(318.27277057,406.05209505)(318.34277466,406.05210297)
\curveto(318.41277043,406.05209505)(318.48277036,406.05709504)(318.55277466,406.06710297)
\curveto(319.13276971,406.07709502)(319.63276921,406.01709508)(320.05277466,405.88710297)
\curveto(320.48276836,405.75709534)(320.86276798,405.57709552)(321.19277466,405.34710297)
\curveto(321.30276754,405.26709583)(321.41276743,405.17709592)(321.52277466,405.07710297)
\curveto(321.6427672,404.98709611)(321.7427671,404.88709621)(321.82277466,404.77710297)
\curveto(321.90276694,404.67709642)(321.97276687,404.57709652)(322.03277466,404.47710297)
\curveto(322.10276674,404.37709672)(322.17276667,404.27209683)(322.24277466,404.16210297)
\curveto(322.31276653,404.05209705)(322.36776648,403.93209717)(322.40777466,403.80210297)
\curveto(322.4477664,403.68209742)(322.49276635,403.55209755)(322.54277466,403.41210297)
\curveto(322.57276627,403.33209777)(322.59776625,403.24709785)(322.61777466,403.15710297)
\lineto(322.67777466,402.88710297)
\curveto(322.68776616,402.84709825)(322.69276615,402.80709829)(322.69277466,402.76710297)
\curveto(322.69276615,402.72709837)(322.69776615,402.68709841)(322.70777466,402.64710297)
\curveto(322.72776612,402.5970985)(322.73276611,402.54209856)(322.72277466,402.48210297)
\curveto(322.71276613,402.42209868)(322.71776613,402.36709873)(322.73777466,402.31710297)
\moveto(320.63777466,401.77710297)
\curveto(320.6477682,401.82709927)(320.65276819,401.8970992)(320.65277466,401.98710297)
\curveto(320.65276819,402.08709901)(320.6477682,402.16209894)(320.63777466,402.21210297)
\lineto(320.63777466,402.33210297)
\curveto(320.61776823,402.38209872)(320.60776824,402.43709866)(320.60777466,402.49710297)
\curveto(320.60776824,402.55709854)(320.60276824,402.61209849)(320.59277466,402.66210297)
\curveto(320.59276825,402.7020984)(320.58776826,402.73209837)(320.57777466,402.75210297)
\lineto(320.51777466,402.99210297)
\curveto(320.50776834,403.08209802)(320.48776836,403.16709793)(320.45777466,403.24710297)
\curveto(320.3477685,403.50709759)(320.21776863,403.72709737)(320.06777466,403.90710297)
\curveto(319.91776893,404.097097)(319.71776913,404.24709685)(319.46777466,404.35710297)
\curveto(319.40776944,404.37709672)(319.3477695,404.39209671)(319.28777466,404.40210297)
\curveto(319.22776962,404.42209668)(319.16276968,404.44209666)(319.09277466,404.46210297)
\curveto(319.01276983,404.48209662)(318.92776992,404.48709661)(318.83777466,404.47710297)
\lineto(318.56777466,404.47710297)
\curveto(318.53777031,404.45709664)(318.50277034,404.44709665)(318.46277466,404.44710297)
\curveto(318.42277042,404.45709664)(318.38777046,404.45709664)(318.35777466,404.44710297)
\lineto(318.14777466,404.38710297)
\curveto(318.08777076,404.37709672)(318.03277081,404.35709674)(317.98277466,404.32710297)
\curveto(317.73277111,404.21709688)(317.52777132,404.05709704)(317.36777466,403.84710297)
\curveto(317.21777163,403.64709745)(317.09777175,403.41209769)(317.00777466,403.14210297)
\curveto(316.97777187,403.04209806)(316.95277189,402.93709816)(316.93277466,402.82710297)
\curveto(316.92277192,402.71709838)(316.90777194,402.60709849)(316.88777466,402.49710297)
\curveto(316.87777197,402.44709865)(316.87277197,402.3970987)(316.87277466,402.34710297)
\lineto(316.87277466,402.19710297)
\curveto(316.85277199,402.12709897)(316.842772,402.02209908)(316.84277466,401.88210297)
\curveto(316.85277199,401.74209936)(316.86777198,401.63709946)(316.88777466,401.56710297)
\lineto(316.88777466,401.43210297)
\curveto(316.90777194,401.35209975)(316.92277192,401.27209983)(316.93277466,401.19210297)
\curveto(316.9427719,401.12209998)(316.95777189,401.04710005)(316.97777466,400.96710297)
\curveto(317.07777177,400.66710043)(317.18277166,400.42210068)(317.29277466,400.23210297)
\curveto(317.41277143,400.05210105)(317.59777125,399.88710121)(317.84777466,399.73710297)
\curveto(317.91777093,399.68710141)(317.99277085,399.64710145)(318.07277466,399.61710297)
\curveto(318.16277068,399.58710151)(318.25277059,399.56210154)(318.34277466,399.54210297)
\curveto(318.38277046,399.53210157)(318.41777043,399.52710157)(318.44777466,399.52710297)
\curveto(318.47777037,399.53710156)(318.51277033,399.53710156)(318.55277466,399.52710297)
\lineto(318.67277466,399.49710297)
\curveto(318.72277012,399.4971016)(318.76777008,399.5021016)(318.80777466,399.51210297)
\lineto(318.92777466,399.51210297)
\curveto(319.00776984,399.53210157)(319.08776976,399.54710155)(319.16777466,399.55710297)
\curveto(319.2477696,399.56710153)(319.32276952,399.58710151)(319.39277466,399.61710297)
\curveto(319.65276919,399.71710138)(319.86276898,399.85210125)(320.02277466,400.02210297)
\curveto(320.18276866,400.19210091)(320.31776853,400.4021007)(320.42777466,400.65210297)
\curveto(320.46776838,400.75210035)(320.49776835,400.85210025)(320.51777466,400.95210297)
\curveto(320.53776831,401.05210005)(320.56276828,401.15709994)(320.59277466,401.26710297)
\curveto(320.60276824,401.30709979)(320.60776824,401.34209976)(320.60777466,401.37210297)
\curveto(320.60776824,401.41209969)(320.61276823,401.45209965)(320.62277466,401.49210297)
\lineto(320.62277466,401.62710297)
\curveto(320.62276822,401.67709942)(320.62776822,401.72709937)(320.63777466,401.77710297)
}
}
{
\newrgbcolor{curcolor}{0 0 0}
\pscustom[linestyle=none,fillstyle=solid,fillcolor=curcolor]
{
\newpath
\moveto(327.10769653,406.08210297)
\curveto(327.85769203,406.102095)(328.50769138,406.01709508)(329.05769653,405.82710297)
\curveto(329.61769027,405.64709545)(330.04268985,405.33209577)(330.33269653,404.88210297)
\curveto(330.40268949,404.77209633)(330.46268943,404.65709644)(330.51269653,404.53710297)
\curveto(330.57268932,404.42709667)(330.62268927,404.3020968)(330.66269653,404.16210297)
\curveto(330.68268921,404.102097)(330.6926892,404.03709706)(330.69269653,403.96710297)
\curveto(330.6926892,403.8970972)(330.68268921,403.83709726)(330.66269653,403.78710297)
\curveto(330.62268927,403.72709737)(330.56768932,403.68709741)(330.49769653,403.66710297)
\curveto(330.44768944,403.64709745)(330.3876895,403.63709746)(330.31769653,403.63710297)
\lineto(330.10769653,403.63710297)
\lineto(329.44769653,403.63710297)
\curveto(329.37769051,403.63709746)(329.30769058,403.63209747)(329.23769653,403.62210297)
\curveto(329.16769072,403.62209748)(329.10269079,403.63209747)(329.04269653,403.65210297)
\curveto(328.94269095,403.67209743)(328.86769102,403.71209739)(328.81769653,403.77210297)
\curveto(328.76769112,403.83209727)(328.72269117,403.89209721)(328.68269653,403.95210297)
\lineto(328.56269653,404.16210297)
\curveto(328.53269136,404.24209686)(328.48269141,404.30709679)(328.41269653,404.35710297)
\curveto(328.31269158,404.43709666)(328.21269168,404.4970966)(328.11269653,404.53710297)
\curveto(328.02269187,404.57709652)(327.90769198,404.61209649)(327.76769653,404.64210297)
\curveto(327.69769219,404.66209644)(327.5926923,404.67709642)(327.45269653,404.68710297)
\curveto(327.32269257,404.6970964)(327.22269267,404.69209641)(327.15269653,404.67210297)
\lineto(327.04769653,404.67210297)
\lineto(326.89769653,404.64210297)
\curveto(326.85769303,404.64209646)(326.81269308,404.63709646)(326.76269653,404.62710297)
\curveto(326.5926933,404.57709652)(326.45269344,404.50709659)(326.34269653,404.41710297)
\curveto(326.24269365,404.33709676)(326.17269372,404.21209689)(326.13269653,404.04210297)
\curveto(326.11269378,403.97209713)(326.11269378,403.90709719)(326.13269653,403.84710297)
\curveto(326.15269374,403.78709731)(326.17269372,403.73709736)(326.19269653,403.69710297)
\curveto(326.26269363,403.57709752)(326.34269355,403.48209762)(326.43269653,403.41210297)
\curveto(326.53269336,403.34209776)(326.64769324,403.28209782)(326.77769653,403.23210297)
\curveto(326.96769292,403.15209795)(327.17269272,403.08209802)(327.39269653,403.02210297)
\lineto(328.08269653,402.87210297)
\curveto(328.32269157,402.83209827)(328.55269134,402.78209832)(328.77269653,402.72210297)
\curveto(329.00269089,402.67209843)(329.21769067,402.60709849)(329.41769653,402.52710297)
\curveto(329.50769038,402.48709861)(329.5926903,402.45209865)(329.67269653,402.42210297)
\curveto(329.76269013,402.4020987)(329.84769004,402.36709873)(329.92769653,402.31710297)
\curveto(330.11768977,402.1970989)(330.2876896,402.06709903)(330.43769653,401.92710297)
\curveto(330.59768929,401.78709931)(330.72268917,401.61209949)(330.81269653,401.40210297)
\curveto(330.84268905,401.33209977)(330.86768902,401.26209984)(330.88769653,401.19210297)
\curveto(330.90768898,401.12209998)(330.92768896,401.04710005)(330.94769653,400.96710297)
\curveto(330.95768893,400.90710019)(330.96268893,400.81210029)(330.96269653,400.68210297)
\curveto(330.97268892,400.56210054)(330.97268892,400.46710063)(330.96269653,400.39710297)
\lineto(330.96269653,400.32210297)
\curveto(330.94268895,400.26210084)(330.92768896,400.2021009)(330.91769653,400.14210297)
\curveto(330.91768897,400.09210101)(330.91268898,400.04210106)(330.90269653,399.99210297)
\curveto(330.83268906,399.69210141)(330.72268917,399.42710167)(330.57269653,399.19710297)
\curveto(330.41268948,398.95710214)(330.21768967,398.76210234)(329.98769653,398.61210297)
\curveto(329.75769013,398.46210264)(329.49769039,398.33210277)(329.20769653,398.22210297)
\curveto(329.09769079,398.17210293)(328.97769091,398.13710296)(328.84769653,398.11710297)
\curveto(328.72769116,398.097103)(328.60769128,398.07210303)(328.48769653,398.04210297)
\curveto(328.39769149,398.02210308)(328.30269159,398.01210309)(328.20269653,398.01210297)
\curveto(328.11269178,398.0021031)(328.02269187,397.98710311)(327.93269653,397.96710297)
\lineto(327.66269653,397.96710297)
\curveto(327.60269229,397.94710315)(327.49769239,397.93710316)(327.34769653,397.93710297)
\curveto(327.20769268,397.93710316)(327.10769278,397.94710315)(327.04769653,397.96710297)
\curveto(327.01769287,397.96710313)(326.98269291,397.97210313)(326.94269653,397.98210297)
\lineto(326.83769653,397.98210297)
\curveto(326.71769317,398.0021031)(326.59769329,398.01710308)(326.47769653,398.02710297)
\curveto(326.35769353,398.03710306)(326.24269365,398.05710304)(326.13269653,398.08710297)
\curveto(325.74269415,398.1971029)(325.39769449,398.32210278)(325.09769653,398.46210297)
\curveto(324.79769509,398.61210249)(324.54269535,398.83210227)(324.33269653,399.12210297)
\curveto(324.1926957,399.31210179)(324.07269582,399.53210157)(323.97269653,399.78210297)
\curveto(323.95269594,399.84210126)(323.93269596,399.92210118)(323.91269653,400.02210297)
\curveto(323.892696,400.07210103)(323.87769601,400.14210096)(323.86769653,400.23210297)
\curveto(323.85769603,400.32210078)(323.86269603,400.3971007)(323.88269653,400.45710297)
\curveto(323.91269598,400.52710057)(323.96269593,400.57710052)(324.03269653,400.60710297)
\curveto(324.08269581,400.62710047)(324.14269575,400.63710046)(324.21269653,400.63710297)
\lineto(324.43769653,400.63710297)
\lineto(325.14269653,400.63710297)
\lineto(325.38269653,400.63710297)
\curveto(325.46269443,400.63710046)(325.53269436,400.62710047)(325.59269653,400.60710297)
\curveto(325.70269419,400.56710053)(325.77269412,400.5021006)(325.80269653,400.41210297)
\curveto(325.84269405,400.32210078)(325.887694,400.22710087)(325.93769653,400.12710297)
\curveto(325.95769393,400.07710102)(325.9926939,400.01210109)(326.04269653,399.93210297)
\curveto(326.10269379,399.85210125)(326.15269374,399.8021013)(326.19269653,399.78210297)
\curveto(326.31269358,399.68210142)(326.42769346,399.6021015)(326.53769653,399.54210297)
\curveto(326.64769324,399.49210161)(326.7876931,399.44210166)(326.95769653,399.39210297)
\curveto(327.00769288,399.37210173)(327.05769283,399.36210174)(327.10769653,399.36210297)
\curveto(327.15769273,399.37210173)(327.20769268,399.37210173)(327.25769653,399.36210297)
\curveto(327.33769255,399.34210176)(327.42269247,399.33210177)(327.51269653,399.33210297)
\curveto(327.61269228,399.34210176)(327.69769219,399.35710174)(327.76769653,399.37710297)
\curveto(327.81769207,399.38710171)(327.86269203,399.39210171)(327.90269653,399.39210297)
\curveto(327.95269194,399.39210171)(328.00269189,399.4021017)(328.05269653,399.42210297)
\curveto(328.1926917,399.47210163)(328.31769157,399.53210157)(328.42769653,399.60210297)
\curveto(328.54769134,399.67210143)(328.64269125,399.76210134)(328.71269653,399.87210297)
\curveto(328.76269113,399.95210115)(328.80269109,400.07710102)(328.83269653,400.24710297)
\curveto(328.85269104,400.31710078)(328.85269104,400.38210072)(328.83269653,400.44210297)
\curveto(328.81269108,400.5021006)(328.7926911,400.55210055)(328.77269653,400.59210297)
\curveto(328.70269119,400.73210037)(328.61269128,400.83710026)(328.50269653,400.90710297)
\curveto(328.40269149,400.97710012)(328.28269161,401.04210006)(328.14269653,401.10210297)
\curveto(327.95269194,401.18209992)(327.75269214,401.24709985)(327.54269653,401.29710297)
\curveto(327.33269256,401.34709975)(327.12269277,401.4020997)(326.91269653,401.46210297)
\curveto(326.83269306,401.48209962)(326.74769314,401.4970996)(326.65769653,401.50710297)
\curveto(326.57769331,401.51709958)(326.49769339,401.53209957)(326.41769653,401.55210297)
\curveto(326.09769379,401.64209946)(325.7926941,401.72709937)(325.50269653,401.80710297)
\curveto(325.21269468,401.8970992)(324.94769494,402.02709907)(324.70769653,402.19710297)
\curveto(324.42769546,402.3970987)(324.22269567,402.66709843)(324.09269653,403.00710297)
\curveto(324.07269582,403.07709802)(324.05269584,403.17209793)(324.03269653,403.29210297)
\curveto(324.01269588,403.36209774)(323.99769589,403.44709765)(323.98769653,403.54710297)
\curveto(323.97769591,403.64709745)(323.98269591,403.73709736)(324.00269653,403.81710297)
\curveto(324.02269587,403.86709723)(324.02769586,403.90709719)(324.01769653,403.93710297)
\curveto(324.00769588,403.97709712)(324.01269588,404.02209708)(324.03269653,404.07210297)
\curveto(324.05269584,404.18209692)(324.07269582,404.28209682)(324.09269653,404.37210297)
\curveto(324.12269577,404.47209663)(324.15769573,404.56709653)(324.19769653,404.65710297)
\curveto(324.32769556,404.94709615)(324.50769538,405.18209592)(324.73769653,405.36210297)
\curveto(324.96769492,405.54209556)(325.22769466,405.68709541)(325.51769653,405.79710297)
\curveto(325.62769426,405.84709525)(325.74269415,405.88209522)(325.86269653,405.90210297)
\curveto(325.98269391,405.93209517)(326.10769378,405.96209514)(326.23769653,405.99210297)
\curveto(326.29769359,406.01209509)(326.35769353,406.02209508)(326.41769653,406.02210297)
\lineto(326.59769653,406.05210297)
\curveto(326.67769321,406.06209504)(326.76269313,406.06709503)(326.85269653,406.06710297)
\curveto(326.94269295,406.06709503)(327.02769286,406.07209503)(327.10769653,406.08210297)
}
}
{
\newrgbcolor{curcolor}{0 0 0}
\pscustom[linestyle=none,fillstyle=solid,fillcolor=curcolor]
{
}
}
{
\newrgbcolor{curcolor}{0 0 0}
\pscustom[linestyle=none,fillstyle=solid,fillcolor=curcolor]
{
\newpath
\moveto(344.24449341,402.09210297)
\curveto(344.25448473,402.03209907)(344.25948472,401.94209916)(344.25949341,401.82210297)
\curveto(344.25948472,401.7020994)(344.24948473,401.61709948)(344.22949341,401.56710297)
\lineto(344.22949341,401.37210297)
\curveto(344.19948478,401.26209984)(344.1794848,401.15709994)(344.16949341,401.05710297)
\curveto(344.16948481,400.95710014)(344.15448483,400.85710024)(344.12449341,400.75710297)
\curveto(344.10448488,400.66710043)(344.0844849,400.57210053)(344.06449341,400.47210297)
\curveto(344.04448494,400.38210072)(344.01448497,400.29210081)(343.97449341,400.20210297)
\curveto(343.90448508,400.03210107)(343.83448515,399.87210123)(343.76449341,399.72210297)
\curveto(343.69448529,399.58210152)(343.61448537,399.44210166)(343.52449341,399.30210297)
\curveto(343.46448552,399.21210189)(343.39948558,399.12710197)(343.32949341,399.04710297)
\curveto(343.26948571,398.97710212)(343.19948578,398.9021022)(343.11949341,398.82210297)
\lineto(343.01449341,398.71710297)
\curveto(342.96448602,398.66710243)(342.90948607,398.62210248)(342.84949341,398.58210297)
\lineto(342.69949341,398.46210297)
\curveto(342.61948636,398.4021027)(342.52948645,398.34710275)(342.42949341,398.29710297)
\curveto(342.33948664,398.25710284)(342.24448674,398.21210289)(342.14449341,398.16210297)
\curveto(342.04448694,398.11210299)(341.93948704,398.07710302)(341.82949341,398.05710297)
\curveto(341.72948725,398.03710306)(341.62448736,398.01710308)(341.51449341,397.99710297)
\curveto(341.45448753,397.97710312)(341.38948759,397.96710313)(341.31949341,397.96710297)
\curveto(341.25948772,397.96710313)(341.19448779,397.95710314)(341.12449341,397.93710297)
\lineto(340.98949341,397.93710297)
\curveto(340.90948807,397.91710318)(340.83448815,397.91710318)(340.76449341,397.93710297)
\lineto(340.61449341,397.93710297)
\curveto(340.55448843,397.95710314)(340.48948849,397.96710313)(340.41949341,397.96710297)
\curveto(340.35948862,397.95710314)(340.29948868,397.96210314)(340.23949341,397.98210297)
\curveto(340.0794889,398.03210307)(339.92448906,398.07710302)(339.77449341,398.11710297)
\curveto(339.63448935,398.15710294)(339.50448948,398.21710288)(339.38449341,398.29710297)
\curveto(339.31448967,398.33710276)(339.24948973,398.37710272)(339.18949341,398.41710297)
\curveto(339.12948985,398.46710263)(339.06448992,398.51710258)(338.99449341,398.56710297)
\lineto(338.81449341,398.70210297)
\curveto(338.73449025,398.76210234)(338.66449032,398.76710233)(338.60449341,398.71710297)
\curveto(338.55449043,398.68710241)(338.52949045,398.64710245)(338.52949341,398.59710297)
\curveto(338.52949045,398.55710254)(338.51949046,398.50710259)(338.49949341,398.44710297)
\curveto(338.4794905,398.34710275)(338.46949051,398.23210287)(338.46949341,398.10210297)
\curveto(338.4794905,397.97210313)(338.4844905,397.85210325)(338.48449341,397.74210297)
\lineto(338.48449341,396.21210297)
\curveto(338.4844905,396.08210502)(338.4794905,395.95710514)(338.46949341,395.83710297)
\curveto(338.46949051,395.70710539)(338.44449054,395.6021055)(338.39449341,395.52210297)
\curveto(338.36449062,395.48210562)(338.30949067,395.45210565)(338.22949341,395.43210297)
\curveto(338.14949083,395.41210569)(338.05949092,395.4021057)(337.95949341,395.40210297)
\curveto(337.85949112,395.39210571)(337.75949122,395.39210571)(337.65949341,395.40210297)
\lineto(337.40449341,395.40210297)
\lineto(336.99949341,395.40210297)
\lineto(336.89449341,395.40210297)
\curveto(336.85449213,395.4021057)(336.81949216,395.40710569)(336.78949341,395.41710297)
\lineto(336.66949341,395.41710297)
\curveto(336.49949248,395.46710563)(336.40949257,395.56710553)(336.39949341,395.71710297)
\curveto(336.38949259,395.85710524)(336.3844926,396.02710507)(336.38449341,396.22710297)
\lineto(336.38449341,405.03210297)
\curveto(336.3844926,405.14209596)(336.3794926,405.25709584)(336.36949341,405.37710297)
\curveto(336.36949261,405.50709559)(336.39449259,405.60709549)(336.44449341,405.67710297)
\curveto(336.4844925,405.74709535)(336.53949244,405.79209531)(336.60949341,405.81210297)
\curveto(336.65949232,405.83209527)(336.71949226,405.84209526)(336.78949341,405.84210297)
\lineto(337.01449341,405.84210297)
\lineto(337.73449341,405.84210297)
\lineto(338.01949341,405.84210297)
\curveto(338.10949087,405.84209526)(338.1844908,405.81709528)(338.24449341,405.76710297)
\curveto(338.31449067,405.71709538)(338.34949063,405.65209545)(338.34949341,405.57210297)
\curveto(338.35949062,405.5020956)(338.3844906,405.42709567)(338.42449341,405.34710297)
\curveto(338.43449055,405.31709578)(338.44449054,405.29209581)(338.45449341,405.27210297)
\curveto(338.47449051,405.26209584)(338.49449049,405.24709585)(338.51449341,405.22710297)
\curveto(338.62449036,405.21709588)(338.71449027,405.24709585)(338.78449341,405.31710297)
\curveto(338.85449013,405.38709571)(338.92449006,405.44709565)(338.99449341,405.49710297)
\curveto(339.12448986,405.58709551)(339.25948972,405.66709543)(339.39949341,405.73710297)
\curveto(339.53948944,405.81709528)(339.69448929,405.88209522)(339.86449341,405.93210297)
\curveto(339.94448904,405.96209514)(340.02948895,405.98209512)(340.11949341,405.99210297)
\curveto(340.21948876,406.0020951)(340.31448867,406.01709508)(340.40449341,406.03710297)
\curveto(340.44448854,406.04709505)(340.4844885,406.04709505)(340.52449341,406.03710297)
\curveto(340.57448841,406.02709507)(340.61448837,406.03209507)(340.64449341,406.05210297)
\curveto(341.21448777,406.07209503)(341.69448729,405.99209511)(342.08449341,405.81210297)
\curveto(342.4844865,405.64209546)(342.82448616,405.41709568)(343.10449341,405.13710297)
\curveto(343.15448583,405.08709601)(343.19948578,405.03709606)(343.23949341,404.98710297)
\curveto(343.2794857,404.94709615)(343.31948566,404.9020962)(343.35949341,404.85210297)
\curveto(343.42948555,404.76209634)(343.48948549,404.67209643)(343.53949341,404.58210297)
\curveto(343.59948538,404.49209661)(343.65448533,404.4020967)(343.70449341,404.31210297)
\curveto(343.72448526,404.29209681)(343.73448525,404.26709683)(343.73449341,404.23710297)
\curveto(343.74448524,404.20709689)(343.75948522,404.17209693)(343.77949341,404.13210297)
\curveto(343.83948514,404.03209707)(343.89448509,403.91209719)(343.94449341,403.77210297)
\curveto(343.96448502,403.71209739)(343.984485,403.64709745)(344.00449341,403.57710297)
\curveto(344.02448496,403.51709758)(344.04448494,403.45209765)(344.06449341,403.38210297)
\curveto(344.10448488,403.26209784)(344.12948485,403.13709796)(344.13949341,403.00710297)
\curveto(344.15948482,402.87709822)(344.1844848,402.74209836)(344.21449341,402.60210297)
\lineto(344.21449341,402.43710297)
\lineto(344.24449341,402.25710297)
\lineto(344.24449341,402.09210297)
\moveto(342.12949341,401.74710297)
\curveto(342.13948684,401.7970993)(342.14448684,401.86209924)(342.14449341,401.94210297)
\curveto(342.14448684,402.03209907)(342.13948684,402.102099)(342.12949341,402.15210297)
\lineto(342.12949341,402.28710297)
\curveto(342.10948687,402.34709875)(342.09948688,402.41209869)(342.09949341,402.48210297)
\curveto(342.09948688,402.55209855)(342.08948689,402.62209848)(342.06949341,402.69210297)
\curveto(342.04948693,402.79209831)(342.02948695,402.88709821)(342.00949341,402.97710297)
\curveto(341.98948699,403.07709802)(341.95948702,403.16709793)(341.91949341,403.24710297)
\curveto(341.79948718,403.56709753)(341.64448734,403.82209728)(341.45449341,404.01210297)
\curveto(341.26448772,404.2020969)(340.99448799,404.34209676)(340.64449341,404.43210297)
\curveto(340.56448842,404.45209665)(340.47448851,404.46209664)(340.37449341,404.46210297)
\lineto(340.10449341,404.46210297)
\curveto(340.06448892,404.45209665)(340.02948895,404.44709665)(339.99949341,404.44710297)
\curveto(339.96948901,404.44709665)(339.93448905,404.44209666)(339.89449341,404.43210297)
\lineto(339.68449341,404.37210297)
\curveto(339.62448936,404.36209674)(339.56448942,404.34209676)(339.50449341,404.31210297)
\curveto(339.24448974,404.2020969)(339.03948994,404.03209707)(338.88949341,403.80210297)
\curveto(338.74949023,403.57209753)(338.63449035,403.31709778)(338.54449341,403.03710297)
\curveto(338.52449046,402.95709814)(338.50949047,402.87209823)(338.49949341,402.78210297)
\curveto(338.48949049,402.7020984)(338.47449051,402.62209848)(338.45449341,402.54210297)
\curveto(338.44449054,402.5020986)(338.43949054,402.43709866)(338.43949341,402.34710297)
\curveto(338.41949056,402.30709879)(338.41449057,402.25709884)(338.42449341,402.19710297)
\curveto(338.43449055,402.14709895)(338.43449055,402.097099)(338.42449341,402.04710297)
\curveto(338.40449058,401.98709911)(338.40449058,401.93209917)(338.42449341,401.88210297)
\lineto(338.42449341,401.70210297)
\lineto(338.42449341,401.56710297)
\curveto(338.42449056,401.52709957)(338.43449055,401.48709961)(338.45449341,401.44710297)
\curveto(338.45449053,401.37709972)(338.45949052,401.32209978)(338.46949341,401.28210297)
\lineto(338.49949341,401.10210297)
\curveto(338.50949047,401.04210006)(338.52449046,400.98210012)(338.54449341,400.92210297)
\curveto(338.63449035,400.63210047)(338.73949024,400.39210071)(338.85949341,400.20210297)
\curveto(338.98948999,400.02210108)(339.16948981,399.86210124)(339.39949341,399.72210297)
\curveto(339.53948944,399.64210146)(339.70448928,399.57710152)(339.89449341,399.52710297)
\curveto(339.93448905,399.51710158)(339.96948901,399.51210159)(339.99949341,399.51210297)
\curveto(340.02948895,399.52210158)(340.06448892,399.52210158)(340.10449341,399.51210297)
\curveto(340.14448884,399.5021016)(340.20448878,399.49210161)(340.28449341,399.48210297)
\curveto(340.36448862,399.48210162)(340.42948855,399.48710161)(340.47949341,399.49710297)
\curveto(340.55948842,399.51710158)(340.63948834,399.53210157)(340.71949341,399.54210297)
\curveto(340.80948817,399.56210154)(340.89448809,399.58710151)(340.97449341,399.61710297)
\curveto(341.21448777,399.71710138)(341.40948757,399.85710124)(341.55949341,400.03710297)
\curveto(341.70948727,400.21710088)(341.83448715,400.42710067)(341.93449341,400.66710297)
\curveto(341.984487,400.78710031)(342.01948696,400.91210019)(342.03949341,401.04210297)
\curveto(342.05948692,401.17209993)(342.0844869,401.30709979)(342.11449341,401.44710297)
\lineto(342.11449341,401.59710297)
\curveto(342.12448686,401.64709945)(342.12948685,401.6970994)(342.12949341,401.74710297)
}
}
{
\newrgbcolor{curcolor}{0 0 0}
\pscustom[linestyle=none,fillstyle=solid,fillcolor=curcolor]
{
\newpath
\moveto(353.29441528,402.31710297)
\curveto(353.31440671,402.25709884)(353.3244067,402.17209893)(353.32441528,402.06210297)
\curveto(353.3244067,401.95209915)(353.31440671,401.86709923)(353.29441528,401.80710297)
\lineto(353.29441528,401.65710297)
\curveto(353.27440675,401.57709952)(353.26440676,401.4970996)(353.26441528,401.41710297)
\curveto(353.27440675,401.33709976)(353.26940676,401.25709984)(353.24941528,401.17710297)
\curveto(353.2294068,401.10709999)(353.21440681,401.04210006)(353.20441528,400.98210297)
\curveto(353.19440683,400.92210018)(353.18440684,400.85710024)(353.17441528,400.78710297)
\curveto(353.13440689,400.67710042)(353.09940693,400.56210054)(353.06941528,400.44210297)
\curveto(353.03940699,400.33210077)(352.99940703,400.22710087)(352.94941528,400.12710297)
\curveto(352.73940729,399.64710145)(352.46440756,399.25710184)(352.12441528,398.95710297)
\curveto(351.78440824,398.65710244)(351.37440865,398.40710269)(350.89441528,398.20710297)
\curveto(350.77440925,398.15710294)(350.64940938,398.12210298)(350.51941528,398.10210297)
\curveto(350.39940963,398.07210303)(350.27440975,398.04210306)(350.14441528,398.01210297)
\curveto(350.09440993,397.99210311)(350.03940999,397.98210312)(349.97941528,397.98210297)
\curveto(349.91941011,397.98210312)(349.86441016,397.97710312)(349.81441528,397.96710297)
\lineto(349.70941528,397.96710297)
\curveto(349.67941035,397.95710314)(349.64941038,397.95210315)(349.61941528,397.95210297)
\curveto(349.56941046,397.94210316)(349.48941054,397.93710316)(349.37941528,397.93710297)
\curveto(349.26941076,397.92710317)(349.18441084,397.93210317)(349.12441528,397.95210297)
\lineto(348.97441528,397.95210297)
\curveto(348.9244111,397.96210314)(348.86941116,397.96710313)(348.80941528,397.96710297)
\curveto(348.75941127,397.95710314)(348.70941132,397.96210314)(348.65941528,397.98210297)
\curveto(348.61941141,397.99210311)(348.57941145,397.9971031)(348.53941528,397.99710297)
\curveto(348.50941152,397.9971031)(348.46941156,398.0021031)(348.41941528,398.01210297)
\curveto(348.31941171,398.04210306)(348.21941181,398.06710303)(348.11941528,398.08710297)
\curveto(348.01941201,398.10710299)(347.9244121,398.13710296)(347.83441528,398.17710297)
\curveto(347.71441231,398.21710288)(347.59941243,398.25710284)(347.48941528,398.29710297)
\curveto(347.38941264,398.33710276)(347.28441274,398.38710271)(347.17441528,398.44710297)
\curveto(346.8244132,398.65710244)(346.5244135,398.9021022)(346.27441528,399.18210297)
\curveto(346.024414,399.46210164)(345.81441421,399.7971013)(345.64441528,400.18710297)
\curveto(345.59441443,400.27710082)(345.55441447,400.37210073)(345.52441528,400.47210297)
\curveto(345.50441452,400.57210053)(345.47941455,400.67710042)(345.44941528,400.78710297)
\curveto(345.4294146,400.83710026)(345.41941461,400.88210022)(345.41941528,400.92210297)
\curveto(345.41941461,400.96210014)(345.40941462,401.00710009)(345.38941528,401.05710297)
\curveto(345.36941466,401.13709996)(345.35941467,401.21709988)(345.35941528,401.29710297)
\curveto(345.35941467,401.38709971)(345.34941468,401.47209963)(345.32941528,401.55210297)
\curveto(345.31941471,401.6020995)(345.31441471,401.64709945)(345.31441528,401.68710297)
\lineto(345.31441528,401.82210297)
\curveto(345.29441473,401.88209922)(345.28441474,401.96709913)(345.28441528,402.07710297)
\curveto(345.29441473,402.18709891)(345.30941472,402.27209883)(345.32941528,402.33210297)
\lineto(345.32941528,402.43710297)
\curveto(345.33941469,402.48709861)(345.33941469,402.53709856)(345.32941528,402.58710297)
\curveto(345.3294147,402.64709845)(345.33941469,402.7020984)(345.35941528,402.75210297)
\curveto(345.36941466,402.8020983)(345.37441465,402.84709825)(345.37441528,402.88710297)
\curveto(345.37441465,402.93709816)(345.38441464,402.98709811)(345.40441528,403.03710297)
\curveto(345.44441458,403.16709793)(345.47941455,403.29209781)(345.50941528,403.41210297)
\curveto(345.53941449,403.54209756)(345.57941445,403.66709743)(345.62941528,403.78710297)
\curveto(345.80941422,404.1970969)(346.024414,404.53709656)(346.27441528,404.80710297)
\curveto(346.5244135,405.08709601)(346.8294132,405.34209576)(347.18941528,405.57210297)
\curveto(347.28941274,405.62209548)(347.39441263,405.66709543)(347.50441528,405.70710297)
\curveto(347.61441241,405.74709535)(347.7244123,405.79209531)(347.83441528,405.84210297)
\curveto(347.96441206,405.89209521)(348.09941193,405.92709517)(348.23941528,405.94710297)
\curveto(348.37941165,405.96709513)(348.5244115,405.9970951)(348.67441528,406.03710297)
\curveto(348.75441127,406.04709505)(348.8294112,406.05209505)(348.89941528,406.05210297)
\curveto(348.96941106,406.05209505)(349.03941099,406.05709504)(349.10941528,406.06710297)
\curveto(349.68941034,406.07709502)(350.18940984,406.01709508)(350.60941528,405.88710297)
\curveto(351.03940899,405.75709534)(351.41940861,405.57709552)(351.74941528,405.34710297)
\curveto(351.85940817,405.26709583)(351.96940806,405.17709592)(352.07941528,405.07710297)
\curveto(352.19940783,404.98709611)(352.29940773,404.88709621)(352.37941528,404.77710297)
\curveto(352.45940757,404.67709642)(352.5294075,404.57709652)(352.58941528,404.47710297)
\curveto(352.65940737,404.37709672)(352.7294073,404.27209683)(352.79941528,404.16210297)
\curveto(352.86940716,404.05209705)(352.9244071,403.93209717)(352.96441528,403.80210297)
\curveto(353.00440702,403.68209742)(353.04940698,403.55209755)(353.09941528,403.41210297)
\curveto(353.1294069,403.33209777)(353.15440687,403.24709785)(353.17441528,403.15710297)
\lineto(353.23441528,402.88710297)
\curveto(353.24440678,402.84709825)(353.24940678,402.80709829)(353.24941528,402.76710297)
\curveto(353.24940678,402.72709837)(353.25440677,402.68709841)(353.26441528,402.64710297)
\curveto(353.28440674,402.5970985)(353.28940674,402.54209856)(353.27941528,402.48210297)
\curveto(353.26940676,402.42209868)(353.27440675,402.36709873)(353.29441528,402.31710297)
\moveto(351.19441528,401.77710297)
\curveto(351.20440882,401.82709927)(351.20940882,401.8970992)(351.20941528,401.98710297)
\curveto(351.20940882,402.08709901)(351.20440882,402.16209894)(351.19441528,402.21210297)
\lineto(351.19441528,402.33210297)
\curveto(351.17440885,402.38209872)(351.16440886,402.43709866)(351.16441528,402.49710297)
\curveto(351.16440886,402.55709854)(351.15940887,402.61209849)(351.14941528,402.66210297)
\curveto(351.14940888,402.7020984)(351.14440888,402.73209837)(351.13441528,402.75210297)
\lineto(351.07441528,402.99210297)
\curveto(351.06440896,403.08209802)(351.04440898,403.16709793)(351.01441528,403.24710297)
\curveto(350.90440912,403.50709759)(350.77440925,403.72709737)(350.62441528,403.90710297)
\curveto(350.47440955,404.097097)(350.27440975,404.24709685)(350.02441528,404.35710297)
\curveto(349.96441006,404.37709672)(349.90441012,404.39209671)(349.84441528,404.40210297)
\curveto(349.78441024,404.42209668)(349.71941031,404.44209666)(349.64941528,404.46210297)
\curveto(349.56941046,404.48209662)(349.48441054,404.48709661)(349.39441528,404.47710297)
\lineto(349.12441528,404.47710297)
\curveto(349.09441093,404.45709664)(349.05941097,404.44709665)(349.01941528,404.44710297)
\curveto(348.97941105,404.45709664)(348.94441108,404.45709664)(348.91441528,404.44710297)
\lineto(348.70441528,404.38710297)
\curveto(348.64441138,404.37709672)(348.58941144,404.35709674)(348.53941528,404.32710297)
\curveto(348.28941174,404.21709688)(348.08441194,404.05709704)(347.92441528,403.84710297)
\curveto(347.77441225,403.64709745)(347.65441237,403.41209769)(347.56441528,403.14210297)
\curveto(347.53441249,403.04209806)(347.50941252,402.93709816)(347.48941528,402.82710297)
\curveto(347.47941255,402.71709838)(347.46441256,402.60709849)(347.44441528,402.49710297)
\curveto(347.43441259,402.44709865)(347.4294126,402.3970987)(347.42941528,402.34710297)
\lineto(347.42941528,402.19710297)
\curveto(347.40941262,402.12709897)(347.39941263,402.02209908)(347.39941528,401.88210297)
\curveto(347.40941262,401.74209936)(347.4244126,401.63709946)(347.44441528,401.56710297)
\lineto(347.44441528,401.43210297)
\curveto(347.46441256,401.35209975)(347.47941255,401.27209983)(347.48941528,401.19210297)
\curveto(347.49941253,401.12209998)(347.51441251,401.04710005)(347.53441528,400.96710297)
\curveto(347.63441239,400.66710043)(347.73941229,400.42210068)(347.84941528,400.23210297)
\curveto(347.96941206,400.05210105)(348.15441187,399.88710121)(348.40441528,399.73710297)
\curveto(348.47441155,399.68710141)(348.54941148,399.64710145)(348.62941528,399.61710297)
\curveto(348.71941131,399.58710151)(348.80941122,399.56210154)(348.89941528,399.54210297)
\curveto(348.93941109,399.53210157)(348.97441105,399.52710157)(349.00441528,399.52710297)
\curveto(349.03441099,399.53710156)(349.06941096,399.53710156)(349.10941528,399.52710297)
\lineto(349.22941528,399.49710297)
\curveto(349.27941075,399.4971016)(349.3244107,399.5021016)(349.36441528,399.51210297)
\lineto(349.48441528,399.51210297)
\curveto(349.56441046,399.53210157)(349.64441038,399.54710155)(349.72441528,399.55710297)
\curveto(349.80441022,399.56710153)(349.87941015,399.58710151)(349.94941528,399.61710297)
\curveto(350.20940982,399.71710138)(350.41940961,399.85210125)(350.57941528,400.02210297)
\curveto(350.73940929,400.19210091)(350.87440915,400.4021007)(350.98441528,400.65210297)
\curveto(351.024409,400.75210035)(351.05440897,400.85210025)(351.07441528,400.95210297)
\curveto(351.09440893,401.05210005)(351.11940891,401.15709994)(351.14941528,401.26710297)
\curveto(351.15940887,401.30709979)(351.16440886,401.34209976)(351.16441528,401.37210297)
\curveto(351.16440886,401.41209969)(351.16940886,401.45209965)(351.17941528,401.49210297)
\lineto(351.17941528,401.62710297)
\curveto(351.17940885,401.67709942)(351.18440884,401.72709937)(351.19441528,401.77710297)
}
}
{
\newrgbcolor{curcolor}{0 0 0}
\pscustom[linestyle=none,fillstyle=solid,fillcolor=curcolor]
{
\newpath
\moveto(359.11933716,406.06710297)
\curveto(359.22933184,406.06709503)(359.32433175,406.05709504)(359.40433716,406.03710297)
\curveto(359.49433158,406.01709508)(359.56433151,405.97209513)(359.61433716,405.90210297)
\curveto(359.6743314,405.82209528)(359.70433137,405.68209542)(359.70433716,405.48210297)
\lineto(359.70433716,404.97210297)
\lineto(359.70433716,404.59710297)
\curveto(359.71433136,404.45709664)(359.69933137,404.34709675)(359.65933716,404.26710297)
\curveto(359.61933145,404.1970969)(359.55933151,404.15209695)(359.47933716,404.13210297)
\curveto(359.40933166,404.11209699)(359.32433175,404.102097)(359.22433716,404.10210297)
\curveto(359.13433194,404.102097)(359.03433204,404.10709699)(358.92433716,404.11710297)
\curveto(358.82433225,404.12709697)(358.72933234,404.12209698)(358.63933716,404.10210297)
\curveto(358.5693325,404.08209702)(358.49933257,404.06709703)(358.42933716,404.05710297)
\curveto(358.35933271,404.05709704)(358.29433278,404.04709705)(358.23433716,404.02710297)
\curveto(358.074333,403.97709712)(357.91433316,403.9020972)(357.75433716,403.80210297)
\curveto(357.59433348,403.71209739)(357.4693336,403.60709749)(357.37933716,403.48710297)
\curveto(357.32933374,403.40709769)(357.2743338,403.32209778)(357.21433716,403.23210297)
\curveto(357.16433391,403.15209795)(357.11433396,403.06709803)(357.06433716,402.97710297)
\curveto(357.03433404,402.8970982)(357.00433407,402.81209829)(356.97433716,402.72210297)
\lineto(356.91433716,402.48210297)
\curveto(356.89433418,402.41209869)(356.88433419,402.33709876)(356.88433716,402.25710297)
\curveto(356.88433419,402.18709891)(356.8743342,402.11709898)(356.85433716,402.04710297)
\curveto(356.84433423,402.00709909)(356.83933423,401.96709913)(356.83933716,401.92710297)
\curveto(356.84933422,401.8970992)(356.84933422,401.86709923)(356.83933716,401.83710297)
\lineto(356.83933716,401.59710297)
\curveto(356.81933425,401.52709957)(356.81433426,401.44709965)(356.82433716,401.35710297)
\curveto(356.83433424,401.27709982)(356.83933423,401.1970999)(356.83933716,401.11710297)
\lineto(356.83933716,400.15710297)
\lineto(356.83933716,398.88210297)
\curveto(356.83933423,398.75210235)(356.83433424,398.63210247)(356.82433716,398.52210297)
\curveto(356.81433426,398.41210269)(356.78433429,398.32210278)(356.73433716,398.25210297)
\curveto(356.71433436,398.22210288)(356.67933439,398.1971029)(356.62933716,398.17710297)
\curveto(356.58933448,398.16710293)(356.54433453,398.15710294)(356.49433716,398.14710297)
\lineto(356.41933716,398.14710297)
\curveto(356.3693347,398.13710296)(356.31433476,398.13210297)(356.25433716,398.13210297)
\lineto(356.08933716,398.13210297)
\lineto(355.44433716,398.13210297)
\curveto(355.38433569,398.14210296)(355.31933575,398.14710295)(355.24933716,398.14710297)
\lineto(355.05433716,398.14710297)
\curveto(355.00433607,398.16710293)(354.95433612,398.18210292)(354.90433716,398.19210297)
\curveto(354.85433622,398.21210289)(354.81933625,398.24710285)(354.79933716,398.29710297)
\curveto(354.75933631,398.34710275)(354.73433634,398.41710268)(354.72433716,398.50710297)
\lineto(354.72433716,398.80710297)
\lineto(354.72433716,399.82710297)
\lineto(354.72433716,404.05710297)
\lineto(354.72433716,405.16710297)
\lineto(354.72433716,405.45210297)
\curveto(354.72433635,405.55209555)(354.74433633,405.63209547)(354.78433716,405.69210297)
\curveto(354.83433624,405.77209533)(354.90933616,405.82209528)(355.00933716,405.84210297)
\curveto(355.10933596,405.86209524)(355.22933584,405.87209523)(355.36933716,405.87210297)
\lineto(356.13433716,405.87210297)
\curveto(356.25433482,405.87209523)(356.35933471,405.86209524)(356.44933716,405.84210297)
\curveto(356.53933453,405.83209527)(356.60933446,405.78709531)(356.65933716,405.70710297)
\curveto(356.68933438,405.65709544)(356.70433437,405.58709551)(356.70433716,405.49710297)
\lineto(356.73433716,405.22710297)
\curveto(356.74433433,405.14709595)(356.75933431,405.07209603)(356.77933716,405.00210297)
\curveto(356.80933426,404.93209617)(356.85933421,404.8970962)(356.92933716,404.89710297)
\curveto(356.94933412,404.91709618)(356.9693341,404.92709617)(356.98933716,404.92710297)
\curveto(357.00933406,404.92709617)(357.02933404,404.93709616)(357.04933716,404.95710297)
\curveto(357.10933396,405.00709609)(357.15933391,405.06209604)(357.19933716,405.12210297)
\curveto(357.24933382,405.19209591)(357.30933376,405.25209585)(357.37933716,405.30210297)
\curveto(357.41933365,405.33209577)(357.45433362,405.36209574)(357.48433716,405.39210297)
\curveto(357.51433356,405.43209567)(357.54933352,405.46709563)(357.58933716,405.49710297)
\lineto(357.85933716,405.67710297)
\curveto(357.95933311,405.73709536)(358.05933301,405.79209531)(358.15933716,405.84210297)
\curveto(358.25933281,405.88209522)(358.35933271,405.91709518)(358.45933716,405.94710297)
\lineto(358.78933716,406.03710297)
\curveto(358.81933225,406.04709505)(358.8743322,406.04709505)(358.95433716,406.03710297)
\curveto(359.04433203,406.03709506)(359.09933197,406.04709505)(359.11933716,406.06710297)
}
}
{
\newrgbcolor{curcolor}{0 0 0}
\pscustom[linestyle=none,fillstyle=solid,fillcolor=curcolor]
{
}
}
{
\newrgbcolor{curcolor}{0 0 0}
\pscustom[linestyle=none,fillstyle=solid,fillcolor=curcolor]
{
\newpath
\moveto(365.73457153,408.18210297)
\lineto(366.73957153,408.18210297)
\curveto(366.88956855,408.18209292)(367.01956842,408.17209293)(367.12957153,408.15210297)
\curveto(367.24956819,408.14209296)(367.3345681,408.08209302)(367.38457153,407.97210297)
\curveto(367.40456803,407.92209318)(367.41456802,407.86209324)(367.41457153,407.79210297)
\lineto(367.41457153,407.58210297)
\lineto(367.41457153,406.90710297)
\curveto(367.41456802,406.85709424)(367.40956803,406.7970943)(367.39957153,406.72710297)
\curveto(367.39956804,406.66709443)(367.40456803,406.61209449)(367.41457153,406.56210297)
\lineto(367.41457153,406.39710297)
\curveto(367.41456802,406.31709478)(367.41956802,406.24209486)(367.42957153,406.17210297)
\curveto(367.439568,406.11209499)(367.46456797,406.05709504)(367.50457153,406.00710297)
\curveto(367.57456786,405.91709518)(367.69956774,405.86709523)(367.87957153,405.85710297)
\lineto(368.41957153,405.85710297)
\lineto(368.59957153,405.85710297)
\curveto(368.65956678,405.85709524)(368.71456672,405.84709525)(368.76457153,405.82710297)
\curveto(368.87456656,405.77709532)(368.9345665,405.68709541)(368.94457153,405.55710297)
\curveto(368.96456647,405.42709567)(368.97456646,405.28209582)(368.97457153,405.12210297)
\lineto(368.97457153,404.91210297)
\curveto(368.98456645,404.84209626)(368.97956646,404.78209632)(368.95957153,404.73210297)
\curveto(368.90956653,404.57209653)(368.80456663,404.48709661)(368.64457153,404.47710297)
\curveto(368.48456695,404.46709663)(368.30456713,404.46209664)(368.10457153,404.46210297)
\lineto(367.96957153,404.46210297)
\curveto(367.92956751,404.47209663)(367.89456754,404.47209663)(367.86457153,404.46210297)
\curveto(367.82456761,404.45209665)(367.78956765,404.44709665)(367.75957153,404.44710297)
\curveto(367.72956771,404.45709664)(367.69956774,404.45209665)(367.66957153,404.43210297)
\curveto(367.58956785,404.41209669)(367.52956791,404.36709673)(367.48957153,404.29710297)
\curveto(367.45956798,404.23709686)(367.434568,404.16209694)(367.41457153,404.07210297)
\curveto(367.40456803,404.02209708)(367.40456803,403.96709713)(367.41457153,403.90710297)
\curveto(367.42456801,403.84709725)(367.42456801,403.79209731)(367.41457153,403.74210297)
\lineto(367.41457153,402.81210297)
\lineto(367.41457153,401.05710297)
\curveto(367.41456802,400.80710029)(367.41956802,400.58710051)(367.42957153,400.39710297)
\curveto(367.44956799,400.21710088)(367.51456792,400.05710104)(367.62457153,399.91710297)
\curveto(367.67456776,399.85710124)(367.7395677,399.81210129)(367.81957153,399.78210297)
\lineto(368.08957153,399.72210297)
\curveto(368.11956732,399.71210139)(368.14956729,399.70710139)(368.17957153,399.70710297)
\curveto(368.21956722,399.71710138)(368.24956719,399.71710138)(368.26957153,399.70710297)
\lineto(368.43457153,399.70710297)
\curveto(368.54456689,399.70710139)(368.6395668,399.7021014)(368.71957153,399.69210297)
\curveto(368.79956664,399.68210142)(368.86456657,399.64210146)(368.91457153,399.57210297)
\curveto(368.95456648,399.51210159)(368.97456646,399.43210167)(368.97457153,399.33210297)
\lineto(368.97457153,399.04710297)
\curveto(368.97456646,398.83710226)(368.96956647,398.64210246)(368.95957153,398.46210297)
\curveto(368.95956648,398.29210281)(368.87956656,398.17710292)(368.71957153,398.11710297)
\curveto(368.66956677,398.097103)(368.62456681,398.09210301)(368.58457153,398.10210297)
\curveto(368.54456689,398.102103)(368.49956694,398.09210301)(368.44957153,398.07210297)
\lineto(368.29957153,398.07210297)
\curveto(368.27956716,398.07210303)(368.24956719,398.07710302)(368.20957153,398.08710297)
\curveto(368.16956727,398.08710301)(368.1345673,398.08210302)(368.10457153,398.07210297)
\curveto(368.05456738,398.06210304)(367.99956744,398.06210304)(367.93957153,398.07210297)
\lineto(367.78957153,398.07210297)
\lineto(367.63957153,398.07210297)
\curveto(367.58956785,398.06210304)(367.54456789,398.06210304)(367.50457153,398.07210297)
\lineto(367.33957153,398.07210297)
\curveto(367.28956815,398.08210302)(367.2345682,398.08710301)(367.17457153,398.08710297)
\curveto(367.11456832,398.08710301)(367.05956838,398.09210301)(367.00957153,398.10210297)
\curveto(366.9395685,398.11210299)(366.87456856,398.12210298)(366.81457153,398.13210297)
\lineto(366.63457153,398.16210297)
\curveto(366.52456891,398.19210291)(366.41956902,398.22710287)(366.31957153,398.26710297)
\curveto(366.21956922,398.30710279)(366.12456931,398.35210275)(366.03457153,398.40210297)
\lineto(365.94457153,398.46210297)
\curveto(365.91456952,398.49210261)(365.87956956,398.52210258)(365.83957153,398.55210297)
\curveto(365.81956962,398.57210253)(365.79456964,398.59210251)(365.76457153,398.61210297)
\lineto(365.68957153,398.68710297)
\curveto(365.54956989,398.87710222)(365.44456999,399.08710201)(365.37457153,399.31710297)
\curveto(365.35457008,399.35710174)(365.34457009,399.39210171)(365.34457153,399.42210297)
\curveto(365.35457008,399.46210164)(365.35457008,399.50710159)(365.34457153,399.55710297)
\curveto(365.3345701,399.57710152)(365.32957011,399.6021015)(365.32957153,399.63210297)
\curveto(365.32957011,399.66210144)(365.32457011,399.68710141)(365.31457153,399.70710297)
\lineto(365.31457153,399.85710297)
\curveto(365.30457013,399.8971012)(365.29957014,399.94210116)(365.29957153,399.99210297)
\curveto(365.30957013,400.04210106)(365.31457012,400.09210101)(365.31457153,400.14210297)
\lineto(365.31457153,400.71210297)
\lineto(365.31457153,402.94710297)
\lineto(365.31457153,403.74210297)
\lineto(365.31457153,403.95210297)
\curveto(365.32457011,404.02209708)(365.31957012,404.08709701)(365.29957153,404.14710297)
\curveto(365.25957018,404.28709681)(365.18957025,404.37709672)(365.08957153,404.41710297)
\curveto(364.97957046,404.46709663)(364.8395706,404.48209662)(364.66957153,404.46210297)
\curveto(364.49957094,404.44209666)(364.35457108,404.45709664)(364.23457153,404.50710297)
\curveto(364.15457128,404.53709656)(364.10457133,404.58209652)(364.08457153,404.64210297)
\curveto(364.06457137,404.7020964)(364.04457139,404.77709632)(364.02457153,404.86710297)
\lineto(364.02457153,405.18210297)
\curveto(364.02457141,405.36209574)(364.0345714,405.50709559)(364.05457153,405.61710297)
\curveto(364.07457136,405.72709537)(364.15957128,405.8020953)(364.30957153,405.84210297)
\curveto(364.34957109,405.86209524)(364.38957105,405.86709523)(364.42957153,405.85710297)
\lineto(364.56457153,405.85710297)
\curveto(364.71457072,405.85709524)(364.85457058,405.86209524)(364.98457153,405.87210297)
\curveto(365.11457032,405.89209521)(365.20457023,405.95209515)(365.25457153,406.05210297)
\curveto(365.28457015,406.12209498)(365.29957014,406.2020949)(365.29957153,406.29210297)
\curveto(365.30957013,406.38209472)(365.31457012,406.47209463)(365.31457153,406.56210297)
\lineto(365.31457153,407.49210297)
\lineto(365.31457153,407.74710297)
\curveto(365.31457012,407.83709326)(365.32457011,407.91209319)(365.34457153,407.97210297)
\curveto(365.39457004,408.07209303)(365.46956997,408.13709296)(365.56957153,408.16710297)
\curveto(365.58956985,408.17709292)(365.61456982,408.17709292)(365.64457153,408.16710297)
\curveto(365.68456975,408.16709293)(365.71456972,408.17209293)(365.73457153,408.18210297)
}
}
{
\newrgbcolor{curcolor}{0 0 0}
\pscustom[linestyle=none,fillstyle=solid,fillcolor=curcolor]
{
\newpath
\moveto(372.05800903,408.72210297)
\curveto(372.12800608,408.64209246)(372.16300605,408.52209258)(372.16300903,408.36210297)
\lineto(372.16300903,407.89710297)
\lineto(372.16300903,407.49210297)
\curveto(372.16300605,407.35209375)(372.12800608,407.25709384)(372.05800903,407.20710297)
\curveto(371.99800621,407.15709394)(371.91800629,407.12709397)(371.81800903,407.11710297)
\curveto(371.72800648,407.10709399)(371.62800658,407.102094)(371.51800903,407.10210297)
\lineto(370.67800903,407.10210297)
\curveto(370.56800764,407.102094)(370.46800774,407.10709399)(370.37800903,407.11710297)
\curveto(370.29800791,407.12709397)(370.22800798,407.15709394)(370.16800903,407.20710297)
\curveto(370.12800808,407.23709386)(370.09800811,407.29209381)(370.07800903,407.37210297)
\curveto(370.06800814,407.46209364)(370.05800815,407.55709354)(370.04800903,407.65710297)
\lineto(370.04800903,407.98710297)
\curveto(370.05800815,408.097093)(370.06300815,408.19209291)(370.06300903,408.27210297)
\lineto(370.06300903,408.48210297)
\curveto(370.07300814,408.55209255)(370.09300812,408.61209249)(370.12300903,408.66210297)
\curveto(370.14300807,408.7020924)(370.16800804,408.73209237)(370.19800903,408.75210297)
\lineto(370.31800903,408.81210297)
\curveto(370.33800787,408.81209229)(370.36300785,408.81209229)(370.39300903,408.81210297)
\curveto(370.42300779,408.82209228)(370.44800776,408.82709227)(370.46800903,408.82710297)
\lineto(371.56300903,408.82710297)
\curveto(371.66300655,408.82709227)(371.75800645,408.82209228)(371.84800903,408.81210297)
\curveto(371.93800627,408.8020923)(372.0080062,408.77209233)(372.05800903,408.72210297)
\moveto(372.16300903,398.95710297)
\curveto(372.16300605,398.75710234)(372.15800605,398.58710251)(372.14800903,398.44710297)
\curveto(372.13800607,398.30710279)(372.04800616,398.21210289)(371.87800903,398.16210297)
\curveto(371.81800639,398.14210296)(371.75300646,398.13210297)(371.68300903,398.13210297)
\curveto(371.6130066,398.14210296)(371.53800667,398.14710295)(371.45800903,398.14710297)
\lineto(370.61800903,398.14710297)
\curveto(370.52800768,398.14710295)(370.43800777,398.15210295)(370.34800903,398.16210297)
\curveto(370.26800794,398.17210293)(370.208008,398.2021029)(370.16800903,398.25210297)
\curveto(370.1080081,398.32210278)(370.07300814,398.40710269)(370.06300903,398.50710297)
\lineto(370.06300903,398.85210297)
\lineto(370.06300903,405.18210297)
\lineto(370.06300903,405.48210297)
\curveto(370.06300815,405.58209552)(370.08300813,405.66209544)(370.12300903,405.72210297)
\curveto(370.18300803,405.79209531)(370.26800794,405.83709526)(370.37800903,405.85710297)
\curveto(370.39800781,405.86709523)(370.42300779,405.86709523)(370.45300903,405.85710297)
\curveto(370.49300772,405.85709524)(370.52300769,405.86209524)(370.54300903,405.87210297)
\lineto(371.29300903,405.87210297)
\lineto(371.48800903,405.87210297)
\curveto(371.56800664,405.88209522)(371.63300658,405.88209522)(371.68300903,405.87210297)
\lineto(371.80300903,405.87210297)
\curveto(371.86300635,405.85209525)(371.91800629,405.83709526)(371.96800903,405.82710297)
\curveto(372.01800619,405.81709528)(372.05800615,405.78709531)(372.08800903,405.73710297)
\curveto(372.12800608,405.68709541)(372.14800606,405.61709548)(372.14800903,405.52710297)
\curveto(372.15800605,405.43709566)(372.16300605,405.34209576)(372.16300903,405.24210297)
\lineto(372.16300903,398.95710297)
}
}
{
\newrgbcolor{curcolor}{0 0 0}
\pscustom[linestyle=none,fillstyle=solid,fillcolor=curcolor]
{
\newpath
\moveto(381.71519653,402.09210297)
\curveto(381.72518785,402.03209907)(381.73018785,401.94209916)(381.73019653,401.82210297)
\curveto(381.73018785,401.7020994)(381.72018786,401.61709948)(381.70019653,401.56710297)
\lineto(381.70019653,401.37210297)
\curveto(381.67018791,401.26209984)(381.65018793,401.15709994)(381.64019653,401.05710297)
\curveto(381.64018794,400.95710014)(381.62518795,400.85710024)(381.59519653,400.75710297)
\curveto(381.575188,400.66710043)(381.55518802,400.57210053)(381.53519653,400.47210297)
\curveto(381.51518806,400.38210072)(381.48518809,400.29210081)(381.44519653,400.20210297)
\curveto(381.3751882,400.03210107)(381.30518827,399.87210123)(381.23519653,399.72210297)
\curveto(381.16518841,399.58210152)(381.08518849,399.44210166)(380.99519653,399.30210297)
\curveto(380.93518864,399.21210189)(380.87018871,399.12710197)(380.80019653,399.04710297)
\curveto(380.74018884,398.97710212)(380.67018891,398.9021022)(380.59019653,398.82210297)
\lineto(380.48519653,398.71710297)
\curveto(380.43518914,398.66710243)(380.3801892,398.62210248)(380.32019653,398.58210297)
\lineto(380.17019653,398.46210297)
\curveto(380.09018949,398.4021027)(380.00018958,398.34710275)(379.90019653,398.29710297)
\curveto(379.81018977,398.25710284)(379.71518986,398.21210289)(379.61519653,398.16210297)
\curveto(379.51519006,398.11210299)(379.41019017,398.07710302)(379.30019653,398.05710297)
\curveto(379.20019038,398.03710306)(379.09519048,398.01710308)(378.98519653,397.99710297)
\curveto(378.92519065,397.97710312)(378.86019072,397.96710313)(378.79019653,397.96710297)
\curveto(378.73019085,397.96710313)(378.66519091,397.95710314)(378.59519653,397.93710297)
\lineto(378.46019653,397.93710297)
\curveto(378.3801912,397.91710318)(378.30519127,397.91710318)(378.23519653,397.93710297)
\lineto(378.08519653,397.93710297)
\curveto(378.02519155,397.95710314)(377.96019162,397.96710313)(377.89019653,397.96710297)
\curveto(377.83019175,397.95710314)(377.77019181,397.96210314)(377.71019653,397.98210297)
\curveto(377.55019203,398.03210307)(377.39519218,398.07710302)(377.24519653,398.11710297)
\curveto(377.10519247,398.15710294)(376.9751926,398.21710288)(376.85519653,398.29710297)
\curveto(376.78519279,398.33710276)(376.72019286,398.37710272)(376.66019653,398.41710297)
\curveto(376.60019298,398.46710263)(376.53519304,398.51710258)(376.46519653,398.56710297)
\lineto(376.28519653,398.70210297)
\curveto(376.20519337,398.76210234)(376.13519344,398.76710233)(376.07519653,398.71710297)
\curveto(376.02519355,398.68710241)(376.00019358,398.64710245)(376.00019653,398.59710297)
\curveto(376.00019358,398.55710254)(375.99019359,398.50710259)(375.97019653,398.44710297)
\curveto(375.95019363,398.34710275)(375.94019364,398.23210287)(375.94019653,398.10210297)
\curveto(375.95019363,397.97210313)(375.95519362,397.85210325)(375.95519653,397.74210297)
\lineto(375.95519653,396.21210297)
\curveto(375.95519362,396.08210502)(375.95019363,395.95710514)(375.94019653,395.83710297)
\curveto(375.94019364,395.70710539)(375.91519366,395.6021055)(375.86519653,395.52210297)
\curveto(375.83519374,395.48210562)(375.7801938,395.45210565)(375.70019653,395.43210297)
\curveto(375.62019396,395.41210569)(375.53019405,395.4021057)(375.43019653,395.40210297)
\curveto(375.33019425,395.39210571)(375.23019435,395.39210571)(375.13019653,395.40210297)
\lineto(374.87519653,395.40210297)
\lineto(374.47019653,395.40210297)
\lineto(374.36519653,395.40210297)
\curveto(374.32519525,395.4021057)(374.29019529,395.40710569)(374.26019653,395.41710297)
\lineto(374.14019653,395.41710297)
\curveto(373.97019561,395.46710563)(373.8801957,395.56710553)(373.87019653,395.71710297)
\curveto(373.86019572,395.85710524)(373.85519572,396.02710507)(373.85519653,396.22710297)
\lineto(373.85519653,405.03210297)
\curveto(373.85519572,405.14209596)(373.85019573,405.25709584)(373.84019653,405.37710297)
\curveto(373.84019574,405.50709559)(373.86519571,405.60709549)(373.91519653,405.67710297)
\curveto(373.95519562,405.74709535)(374.01019557,405.79209531)(374.08019653,405.81210297)
\curveto(374.13019545,405.83209527)(374.19019539,405.84209526)(374.26019653,405.84210297)
\lineto(374.48519653,405.84210297)
\lineto(375.20519653,405.84210297)
\lineto(375.49019653,405.84210297)
\curveto(375.580194,405.84209526)(375.65519392,405.81709528)(375.71519653,405.76710297)
\curveto(375.78519379,405.71709538)(375.82019376,405.65209545)(375.82019653,405.57210297)
\curveto(375.83019375,405.5020956)(375.85519372,405.42709567)(375.89519653,405.34710297)
\curveto(375.90519367,405.31709578)(375.91519366,405.29209581)(375.92519653,405.27210297)
\curveto(375.94519363,405.26209584)(375.96519361,405.24709585)(375.98519653,405.22710297)
\curveto(376.09519348,405.21709588)(376.18519339,405.24709585)(376.25519653,405.31710297)
\curveto(376.32519325,405.38709571)(376.39519318,405.44709565)(376.46519653,405.49710297)
\curveto(376.59519298,405.58709551)(376.73019285,405.66709543)(376.87019653,405.73710297)
\curveto(377.01019257,405.81709528)(377.16519241,405.88209522)(377.33519653,405.93210297)
\curveto(377.41519216,405.96209514)(377.50019208,405.98209512)(377.59019653,405.99210297)
\curveto(377.69019189,406.0020951)(377.78519179,406.01709508)(377.87519653,406.03710297)
\curveto(377.91519166,406.04709505)(377.95519162,406.04709505)(377.99519653,406.03710297)
\curveto(378.04519153,406.02709507)(378.08519149,406.03209507)(378.11519653,406.05210297)
\curveto(378.68519089,406.07209503)(379.16519041,405.99209511)(379.55519653,405.81210297)
\curveto(379.95518962,405.64209546)(380.29518928,405.41709568)(380.57519653,405.13710297)
\curveto(380.62518895,405.08709601)(380.67018891,405.03709606)(380.71019653,404.98710297)
\curveto(380.75018883,404.94709615)(380.79018879,404.9020962)(380.83019653,404.85210297)
\curveto(380.90018868,404.76209634)(380.96018862,404.67209643)(381.01019653,404.58210297)
\curveto(381.07018851,404.49209661)(381.12518845,404.4020967)(381.17519653,404.31210297)
\curveto(381.19518838,404.29209681)(381.20518837,404.26709683)(381.20519653,404.23710297)
\curveto(381.21518836,404.20709689)(381.23018835,404.17209693)(381.25019653,404.13210297)
\curveto(381.31018827,404.03209707)(381.36518821,403.91209719)(381.41519653,403.77210297)
\curveto(381.43518814,403.71209739)(381.45518812,403.64709745)(381.47519653,403.57710297)
\curveto(381.49518808,403.51709758)(381.51518806,403.45209765)(381.53519653,403.38210297)
\curveto(381.575188,403.26209784)(381.60018798,403.13709796)(381.61019653,403.00710297)
\curveto(381.63018795,402.87709822)(381.65518792,402.74209836)(381.68519653,402.60210297)
\lineto(381.68519653,402.43710297)
\lineto(381.71519653,402.25710297)
\lineto(381.71519653,402.09210297)
\moveto(379.60019653,401.74710297)
\curveto(379.61018997,401.7970993)(379.61518996,401.86209924)(379.61519653,401.94210297)
\curveto(379.61518996,402.03209907)(379.61018997,402.102099)(379.60019653,402.15210297)
\lineto(379.60019653,402.28710297)
\curveto(379.58019,402.34709875)(379.57019001,402.41209869)(379.57019653,402.48210297)
\curveto(379.57019001,402.55209855)(379.56019002,402.62209848)(379.54019653,402.69210297)
\curveto(379.52019006,402.79209831)(379.50019008,402.88709821)(379.48019653,402.97710297)
\curveto(379.46019012,403.07709802)(379.43019015,403.16709793)(379.39019653,403.24710297)
\curveto(379.27019031,403.56709753)(379.11519046,403.82209728)(378.92519653,404.01210297)
\curveto(378.73519084,404.2020969)(378.46519111,404.34209676)(378.11519653,404.43210297)
\curveto(378.03519154,404.45209665)(377.94519163,404.46209664)(377.84519653,404.46210297)
\lineto(377.57519653,404.46210297)
\curveto(377.53519204,404.45209665)(377.50019208,404.44709665)(377.47019653,404.44710297)
\curveto(377.44019214,404.44709665)(377.40519217,404.44209666)(377.36519653,404.43210297)
\lineto(377.15519653,404.37210297)
\curveto(377.09519248,404.36209674)(377.03519254,404.34209676)(376.97519653,404.31210297)
\curveto(376.71519286,404.2020969)(376.51019307,404.03209707)(376.36019653,403.80210297)
\curveto(376.22019336,403.57209753)(376.10519347,403.31709778)(376.01519653,403.03710297)
\curveto(375.99519358,402.95709814)(375.9801936,402.87209823)(375.97019653,402.78210297)
\curveto(375.96019362,402.7020984)(375.94519363,402.62209848)(375.92519653,402.54210297)
\curveto(375.91519366,402.5020986)(375.91019367,402.43709866)(375.91019653,402.34710297)
\curveto(375.89019369,402.30709879)(375.88519369,402.25709884)(375.89519653,402.19710297)
\curveto(375.90519367,402.14709895)(375.90519367,402.097099)(375.89519653,402.04710297)
\curveto(375.8751937,401.98709911)(375.8751937,401.93209917)(375.89519653,401.88210297)
\lineto(375.89519653,401.70210297)
\lineto(375.89519653,401.56710297)
\curveto(375.89519368,401.52709957)(375.90519367,401.48709961)(375.92519653,401.44710297)
\curveto(375.92519365,401.37709972)(375.93019365,401.32209978)(375.94019653,401.28210297)
\lineto(375.97019653,401.10210297)
\curveto(375.9801936,401.04210006)(375.99519358,400.98210012)(376.01519653,400.92210297)
\curveto(376.10519347,400.63210047)(376.21019337,400.39210071)(376.33019653,400.20210297)
\curveto(376.46019312,400.02210108)(376.64019294,399.86210124)(376.87019653,399.72210297)
\curveto(377.01019257,399.64210146)(377.1751924,399.57710152)(377.36519653,399.52710297)
\curveto(377.40519217,399.51710158)(377.44019214,399.51210159)(377.47019653,399.51210297)
\curveto(377.50019208,399.52210158)(377.53519204,399.52210158)(377.57519653,399.51210297)
\curveto(377.61519196,399.5021016)(377.6751919,399.49210161)(377.75519653,399.48210297)
\curveto(377.83519174,399.48210162)(377.90019168,399.48710161)(377.95019653,399.49710297)
\curveto(378.03019155,399.51710158)(378.11019147,399.53210157)(378.19019653,399.54210297)
\curveto(378.2801913,399.56210154)(378.36519121,399.58710151)(378.44519653,399.61710297)
\curveto(378.68519089,399.71710138)(378.8801907,399.85710124)(379.03019653,400.03710297)
\curveto(379.1801904,400.21710088)(379.30519027,400.42710067)(379.40519653,400.66710297)
\curveto(379.45519012,400.78710031)(379.49019009,400.91210019)(379.51019653,401.04210297)
\curveto(379.53019005,401.17209993)(379.55519002,401.30709979)(379.58519653,401.44710297)
\lineto(379.58519653,401.59710297)
\curveto(379.59518998,401.64709945)(379.60018998,401.6970994)(379.60019653,401.74710297)
}
}
{
\newrgbcolor{curcolor}{0 0 0}
\pscustom[linestyle=none,fillstyle=solid,fillcolor=curcolor]
{
\newpath
\moveto(390.76511841,402.31710297)
\curveto(390.78510984,402.25709884)(390.79510983,402.17209893)(390.79511841,402.06210297)
\curveto(390.79510983,401.95209915)(390.78510984,401.86709923)(390.76511841,401.80710297)
\lineto(390.76511841,401.65710297)
\curveto(390.74510988,401.57709952)(390.73510989,401.4970996)(390.73511841,401.41710297)
\curveto(390.74510988,401.33709976)(390.74010988,401.25709984)(390.72011841,401.17710297)
\curveto(390.70010992,401.10709999)(390.68510994,401.04210006)(390.67511841,400.98210297)
\curveto(390.66510996,400.92210018)(390.65510997,400.85710024)(390.64511841,400.78710297)
\curveto(390.60511002,400.67710042)(390.57011005,400.56210054)(390.54011841,400.44210297)
\curveto(390.51011011,400.33210077)(390.47011015,400.22710087)(390.42011841,400.12710297)
\curveto(390.21011041,399.64710145)(389.93511069,399.25710184)(389.59511841,398.95710297)
\curveto(389.25511137,398.65710244)(388.84511178,398.40710269)(388.36511841,398.20710297)
\curveto(388.24511238,398.15710294)(388.1201125,398.12210298)(387.99011841,398.10210297)
\curveto(387.87011275,398.07210303)(387.74511288,398.04210306)(387.61511841,398.01210297)
\curveto(387.56511306,397.99210311)(387.51011311,397.98210312)(387.45011841,397.98210297)
\curveto(387.39011323,397.98210312)(387.33511329,397.97710312)(387.28511841,397.96710297)
\lineto(387.18011841,397.96710297)
\curveto(387.15011347,397.95710314)(387.1201135,397.95210315)(387.09011841,397.95210297)
\curveto(387.04011358,397.94210316)(386.96011366,397.93710316)(386.85011841,397.93710297)
\curveto(386.74011388,397.92710317)(386.65511397,397.93210317)(386.59511841,397.95210297)
\lineto(386.44511841,397.95210297)
\curveto(386.39511423,397.96210314)(386.34011428,397.96710313)(386.28011841,397.96710297)
\curveto(386.23011439,397.95710314)(386.18011444,397.96210314)(386.13011841,397.98210297)
\curveto(386.09011453,397.99210311)(386.05011457,397.9971031)(386.01011841,397.99710297)
\curveto(385.98011464,397.9971031)(385.94011468,398.0021031)(385.89011841,398.01210297)
\curveto(385.79011483,398.04210306)(385.69011493,398.06710303)(385.59011841,398.08710297)
\curveto(385.49011513,398.10710299)(385.39511523,398.13710296)(385.30511841,398.17710297)
\curveto(385.18511544,398.21710288)(385.07011555,398.25710284)(384.96011841,398.29710297)
\curveto(384.86011576,398.33710276)(384.75511587,398.38710271)(384.64511841,398.44710297)
\curveto(384.29511633,398.65710244)(383.99511663,398.9021022)(383.74511841,399.18210297)
\curveto(383.49511713,399.46210164)(383.28511734,399.7971013)(383.11511841,400.18710297)
\curveto(383.06511756,400.27710082)(383.0251176,400.37210073)(382.99511841,400.47210297)
\curveto(382.97511765,400.57210053)(382.95011767,400.67710042)(382.92011841,400.78710297)
\curveto(382.90011772,400.83710026)(382.89011773,400.88210022)(382.89011841,400.92210297)
\curveto(382.89011773,400.96210014)(382.88011774,401.00710009)(382.86011841,401.05710297)
\curveto(382.84011778,401.13709996)(382.83011779,401.21709988)(382.83011841,401.29710297)
\curveto(382.83011779,401.38709971)(382.8201178,401.47209963)(382.80011841,401.55210297)
\curveto(382.79011783,401.6020995)(382.78511784,401.64709945)(382.78511841,401.68710297)
\lineto(382.78511841,401.82210297)
\curveto(382.76511786,401.88209922)(382.75511787,401.96709913)(382.75511841,402.07710297)
\curveto(382.76511786,402.18709891)(382.78011784,402.27209883)(382.80011841,402.33210297)
\lineto(382.80011841,402.43710297)
\curveto(382.81011781,402.48709861)(382.81011781,402.53709856)(382.80011841,402.58710297)
\curveto(382.80011782,402.64709845)(382.81011781,402.7020984)(382.83011841,402.75210297)
\curveto(382.84011778,402.8020983)(382.84511778,402.84709825)(382.84511841,402.88710297)
\curveto(382.84511778,402.93709816)(382.85511777,402.98709811)(382.87511841,403.03710297)
\curveto(382.91511771,403.16709793)(382.95011767,403.29209781)(382.98011841,403.41210297)
\curveto(383.01011761,403.54209756)(383.05011757,403.66709743)(383.10011841,403.78710297)
\curveto(383.28011734,404.1970969)(383.49511713,404.53709656)(383.74511841,404.80710297)
\curveto(383.99511663,405.08709601)(384.30011632,405.34209576)(384.66011841,405.57210297)
\curveto(384.76011586,405.62209548)(384.86511576,405.66709543)(384.97511841,405.70710297)
\curveto(385.08511554,405.74709535)(385.19511543,405.79209531)(385.30511841,405.84210297)
\curveto(385.43511519,405.89209521)(385.57011505,405.92709517)(385.71011841,405.94710297)
\curveto(385.85011477,405.96709513)(385.99511463,405.9970951)(386.14511841,406.03710297)
\curveto(386.2251144,406.04709505)(386.30011432,406.05209505)(386.37011841,406.05210297)
\curveto(386.44011418,406.05209505)(386.51011411,406.05709504)(386.58011841,406.06710297)
\curveto(387.16011346,406.07709502)(387.66011296,406.01709508)(388.08011841,405.88710297)
\curveto(388.51011211,405.75709534)(388.89011173,405.57709552)(389.22011841,405.34710297)
\curveto(389.33011129,405.26709583)(389.44011118,405.17709592)(389.55011841,405.07710297)
\curveto(389.67011095,404.98709611)(389.77011085,404.88709621)(389.85011841,404.77710297)
\curveto(389.93011069,404.67709642)(390.00011062,404.57709652)(390.06011841,404.47710297)
\curveto(390.13011049,404.37709672)(390.20011042,404.27209683)(390.27011841,404.16210297)
\curveto(390.34011028,404.05209705)(390.39511023,403.93209717)(390.43511841,403.80210297)
\curveto(390.47511015,403.68209742)(390.5201101,403.55209755)(390.57011841,403.41210297)
\curveto(390.60011002,403.33209777)(390.62511,403.24709785)(390.64511841,403.15710297)
\lineto(390.70511841,402.88710297)
\curveto(390.71510991,402.84709825)(390.7201099,402.80709829)(390.72011841,402.76710297)
\curveto(390.7201099,402.72709837)(390.7251099,402.68709841)(390.73511841,402.64710297)
\curveto(390.75510987,402.5970985)(390.76010986,402.54209856)(390.75011841,402.48210297)
\curveto(390.74010988,402.42209868)(390.74510988,402.36709873)(390.76511841,402.31710297)
\moveto(388.66511841,401.77710297)
\curveto(388.67511195,401.82709927)(388.68011194,401.8970992)(388.68011841,401.98710297)
\curveto(388.68011194,402.08709901)(388.67511195,402.16209894)(388.66511841,402.21210297)
\lineto(388.66511841,402.33210297)
\curveto(388.64511198,402.38209872)(388.63511199,402.43709866)(388.63511841,402.49710297)
\curveto(388.63511199,402.55709854)(388.63011199,402.61209849)(388.62011841,402.66210297)
\curveto(388.620112,402.7020984)(388.61511201,402.73209837)(388.60511841,402.75210297)
\lineto(388.54511841,402.99210297)
\curveto(388.53511209,403.08209802)(388.51511211,403.16709793)(388.48511841,403.24710297)
\curveto(388.37511225,403.50709759)(388.24511238,403.72709737)(388.09511841,403.90710297)
\curveto(387.94511268,404.097097)(387.74511288,404.24709685)(387.49511841,404.35710297)
\curveto(387.43511319,404.37709672)(387.37511325,404.39209671)(387.31511841,404.40210297)
\curveto(387.25511337,404.42209668)(387.19011343,404.44209666)(387.12011841,404.46210297)
\curveto(387.04011358,404.48209662)(386.95511367,404.48709661)(386.86511841,404.47710297)
\lineto(386.59511841,404.47710297)
\curveto(386.56511406,404.45709664)(386.53011409,404.44709665)(386.49011841,404.44710297)
\curveto(386.45011417,404.45709664)(386.41511421,404.45709664)(386.38511841,404.44710297)
\lineto(386.17511841,404.38710297)
\curveto(386.11511451,404.37709672)(386.06011456,404.35709674)(386.01011841,404.32710297)
\curveto(385.76011486,404.21709688)(385.55511507,404.05709704)(385.39511841,403.84710297)
\curveto(385.24511538,403.64709745)(385.1251155,403.41209769)(385.03511841,403.14210297)
\curveto(385.00511562,403.04209806)(384.98011564,402.93709816)(384.96011841,402.82710297)
\curveto(384.95011567,402.71709838)(384.93511569,402.60709849)(384.91511841,402.49710297)
\curveto(384.90511572,402.44709865)(384.90011572,402.3970987)(384.90011841,402.34710297)
\lineto(384.90011841,402.19710297)
\curveto(384.88011574,402.12709897)(384.87011575,402.02209908)(384.87011841,401.88210297)
\curveto(384.88011574,401.74209936)(384.89511573,401.63709946)(384.91511841,401.56710297)
\lineto(384.91511841,401.43210297)
\curveto(384.93511569,401.35209975)(384.95011567,401.27209983)(384.96011841,401.19210297)
\curveto(384.97011565,401.12209998)(384.98511564,401.04710005)(385.00511841,400.96710297)
\curveto(385.10511552,400.66710043)(385.21011541,400.42210068)(385.32011841,400.23210297)
\curveto(385.44011518,400.05210105)(385.625115,399.88710121)(385.87511841,399.73710297)
\curveto(385.94511468,399.68710141)(386.0201146,399.64710145)(386.10011841,399.61710297)
\curveto(386.19011443,399.58710151)(386.28011434,399.56210154)(386.37011841,399.54210297)
\curveto(386.41011421,399.53210157)(386.44511418,399.52710157)(386.47511841,399.52710297)
\curveto(386.50511412,399.53710156)(386.54011408,399.53710156)(386.58011841,399.52710297)
\lineto(386.70011841,399.49710297)
\curveto(386.75011387,399.4971016)(386.79511383,399.5021016)(386.83511841,399.51210297)
\lineto(386.95511841,399.51210297)
\curveto(387.03511359,399.53210157)(387.11511351,399.54710155)(387.19511841,399.55710297)
\curveto(387.27511335,399.56710153)(387.35011327,399.58710151)(387.42011841,399.61710297)
\curveto(387.68011294,399.71710138)(387.89011273,399.85210125)(388.05011841,400.02210297)
\curveto(388.21011241,400.19210091)(388.34511228,400.4021007)(388.45511841,400.65210297)
\curveto(388.49511213,400.75210035)(388.5251121,400.85210025)(388.54511841,400.95210297)
\curveto(388.56511206,401.05210005)(388.59011203,401.15709994)(388.62011841,401.26710297)
\curveto(388.63011199,401.30709979)(388.63511199,401.34209976)(388.63511841,401.37210297)
\curveto(388.63511199,401.41209969)(388.64011198,401.45209965)(388.65011841,401.49210297)
\lineto(388.65011841,401.62710297)
\curveto(388.65011197,401.67709942)(388.65511197,401.72709937)(388.66511841,401.77710297)
}
}
{
\newrgbcolor{curcolor}{0 0 0}
\pscustom[linestyle=none,fillstyle=solid,fillcolor=curcolor]
{
}
}
{
\newrgbcolor{curcolor}{0 0 0}
\pscustom[linestyle=none,fillstyle=solid,fillcolor=curcolor]
{
\newpath
\moveto(403.91519653,398.98710297)
\lineto(403.91519653,398.56710297)
\curveto(403.91518816,398.43710266)(403.88518819,398.33210277)(403.82519653,398.25210297)
\curveto(403.7751883,398.2021029)(403.71018837,398.16710293)(403.63019653,398.14710297)
\curveto(403.55018853,398.13710296)(403.46018862,398.13210297)(403.36019653,398.13210297)
\lineto(402.53519653,398.13210297)
\lineto(402.25019653,398.13210297)
\curveto(402.17018991,398.14210296)(402.10518997,398.16710293)(402.05519653,398.20710297)
\curveto(401.98519009,398.25710284)(401.94519013,398.32210278)(401.93519653,398.40210297)
\curveto(401.92519015,398.48210262)(401.90519017,398.56210254)(401.87519653,398.64210297)
\curveto(401.85519022,398.66210244)(401.83519024,398.67710242)(401.81519653,398.68710297)
\curveto(401.80519027,398.70710239)(401.79019029,398.72710237)(401.77019653,398.74710297)
\curveto(401.66019042,398.74710235)(401.5801905,398.72210238)(401.53019653,398.67210297)
\lineto(401.38019653,398.52210297)
\curveto(401.31019077,398.47210263)(401.24519083,398.42710267)(401.18519653,398.38710297)
\curveto(401.12519095,398.35710274)(401.06019102,398.31710278)(400.99019653,398.26710297)
\curveto(400.95019113,398.24710285)(400.90519117,398.22710287)(400.85519653,398.20710297)
\curveto(400.81519126,398.18710291)(400.77019131,398.16710293)(400.72019653,398.14710297)
\curveto(400.5801915,398.097103)(400.43019165,398.05210305)(400.27019653,398.01210297)
\curveto(400.22019186,397.99210311)(400.1751919,397.98210312)(400.13519653,397.98210297)
\curveto(400.09519198,397.98210312)(400.05519202,397.97710312)(400.01519653,397.96710297)
\lineto(399.88019653,397.96710297)
\curveto(399.85019223,397.95710314)(399.81019227,397.95210315)(399.76019653,397.95210297)
\lineto(399.62519653,397.95210297)
\curveto(399.56519251,397.93210317)(399.4751926,397.92710317)(399.35519653,397.93710297)
\curveto(399.23519284,397.93710316)(399.15019293,397.94710315)(399.10019653,397.96710297)
\curveto(399.03019305,397.98710311)(398.96519311,397.9971031)(398.90519653,397.99710297)
\curveto(398.85519322,397.98710311)(398.80019328,397.99210311)(398.74019653,398.01210297)
\lineto(398.38019653,398.13210297)
\curveto(398.27019381,398.16210294)(398.16019392,398.2021029)(398.05019653,398.25210297)
\curveto(397.70019438,398.4021027)(397.38519469,398.63210247)(397.10519653,398.94210297)
\curveto(396.83519524,399.26210184)(396.62019546,399.5971015)(396.46019653,399.94710297)
\curveto(396.41019567,400.05710104)(396.37019571,400.16210094)(396.34019653,400.26210297)
\curveto(396.31019577,400.37210073)(396.2751958,400.48210062)(396.23519653,400.59210297)
\curveto(396.22519585,400.63210047)(396.22019586,400.66710043)(396.22019653,400.69710297)
\curveto(396.22019586,400.73710036)(396.21019587,400.78210032)(396.19019653,400.83210297)
\curveto(396.17019591,400.91210019)(396.15019593,400.9971001)(396.13019653,401.08710297)
\curveto(396.12019596,401.18709991)(396.10519597,401.28709981)(396.08519653,401.38710297)
\curveto(396.075196,401.41709968)(396.07019601,401.45209965)(396.07019653,401.49210297)
\curveto(396.080196,401.53209957)(396.080196,401.56709953)(396.07019653,401.59710297)
\lineto(396.07019653,401.73210297)
\curveto(396.07019601,401.78209932)(396.06519601,401.83209927)(396.05519653,401.88210297)
\curveto(396.04519603,401.93209917)(396.04019604,401.98709911)(396.04019653,402.04710297)
\curveto(396.04019604,402.11709898)(396.04519603,402.17209893)(396.05519653,402.21210297)
\curveto(396.06519601,402.26209884)(396.07019601,402.30709879)(396.07019653,402.34710297)
\lineto(396.07019653,402.49710297)
\curveto(396.080196,402.54709855)(396.080196,402.59209851)(396.07019653,402.63210297)
\curveto(396.07019601,402.68209842)(396.080196,402.73209837)(396.10019653,402.78210297)
\curveto(396.12019596,402.89209821)(396.13519594,402.9970981)(396.14519653,403.09710297)
\curveto(396.16519591,403.1970979)(396.19019589,403.2970978)(396.22019653,403.39710297)
\curveto(396.26019582,403.51709758)(396.29519578,403.63209747)(396.32519653,403.74210297)
\curveto(396.35519572,403.85209725)(396.39519568,403.96209714)(396.44519653,404.07210297)
\curveto(396.58519549,404.37209673)(396.76019532,404.65709644)(396.97019653,404.92710297)
\curveto(396.99019509,404.95709614)(397.01519506,404.98209612)(397.04519653,405.00210297)
\curveto(397.08519499,405.03209607)(397.11519496,405.06209604)(397.13519653,405.09210297)
\curveto(397.1751949,405.14209596)(397.21519486,405.18709591)(397.25519653,405.22710297)
\curveto(397.29519478,405.26709583)(397.34019474,405.30709579)(397.39019653,405.34710297)
\curveto(397.43019465,405.36709573)(397.46519461,405.39209571)(397.49519653,405.42210297)
\curveto(397.52519455,405.46209564)(397.56019452,405.49209561)(397.60019653,405.51210297)
\curveto(397.85019423,405.68209542)(398.14019394,405.82209528)(398.47019653,405.93210297)
\curveto(398.54019354,405.95209515)(398.61019347,405.96709513)(398.68019653,405.97710297)
\curveto(398.76019332,405.98709511)(398.84019324,406.0020951)(398.92019653,406.02210297)
\curveto(398.99019309,406.04209506)(399.080193,406.05209505)(399.19019653,406.05210297)
\curveto(399.30019278,406.06209504)(399.41019267,406.06709503)(399.52019653,406.06710297)
\curveto(399.63019245,406.06709503)(399.73519234,406.06209504)(399.83519653,406.05210297)
\curveto(399.94519213,406.04209506)(400.03519204,406.02709507)(400.10519653,406.00710297)
\curveto(400.25519182,405.95709514)(400.40019168,405.91209519)(400.54019653,405.87210297)
\curveto(400.6801914,405.83209527)(400.81019127,405.77709532)(400.93019653,405.70710297)
\curveto(401.00019108,405.65709544)(401.06519101,405.60709549)(401.12519653,405.55710297)
\curveto(401.18519089,405.51709558)(401.25019083,405.47209563)(401.32019653,405.42210297)
\curveto(401.36019072,405.39209571)(401.41519066,405.35209575)(401.48519653,405.30210297)
\curveto(401.56519051,405.25209585)(401.64019044,405.25209585)(401.71019653,405.30210297)
\curveto(401.75019033,405.32209578)(401.77019031,405.35709574)(401.77019653,405.40710297)
\curveto(401.77019031,405.45709564)(401.7801903,405.50709559)(401.80019653,405.55710297)
\lineto(401.80019653,405.70710297)
\curveto(401.81019027,405.73709536)(401.81519026,405.77209533)(401.81519653,405.81210297)
\lineto(401.81519653,405.93210297)
\lineto(401.81519653,407.97210297)
\curveto(401.81519026,408.08209302)(401.81019027,408.2020929)(401.80019653,408.33210297)
\curveto(401.80019028,408.47209263)(401.82519025,408.57709252)(401.87519653,408.64710297)
\curveto(401.91519016,408.72709237)(401.99019009,408.77709232)(402.10019653,408.79710297)
\curveto(402.12018996,408.80709229)(402.14018994,408.80709229)(402.16019653,408.79710297)
\curveto(402.1801899,408.7970923)(402.20018988,408.8020923)(402.22019653,408.81210297)
\lineto(403.28519653,408.81210297)
\curveto(403.40518867,408.81209229)(403.51518856,408.80709229)(403.61519653,408.79710297)
\curveto(403.71518836,408.78709231)(403.79018829,408.74709235)(403.84019653,408.67710297)
\curveto(403.89018819,408.5970925)(403.91518816,408.49209261)(403.91519653,408.36210297)
\lineto(403.91519653,408.00210297)
\lineto(403.91519653,398.98710297)
\moveto(401.87519653,401.92710297)
\curveto(401.88519019,401.96709913)(401.88519019,402.00709909)(401.87519653,402.04710297)
\lineto(401.87519653,402.18210297)
\curveto(401.8751902,402.28209882)(401.87019021,402.38209872)(401.86019653,402.48210297)
\curveto(401.85019023,402.58209852)(401.83519024,402.67209843)(401.81519653,402.75210297)
\curveto(401.79519028,402.86209824)(401.7751903,402.96209814)(401.75519653,403.05210297)
\curveto(401.74519033,403.14209796)(401.72019036,403.22709787)(401.68019653,403.30710297)
\curveto(401.54019054,403.66709743)(401.33519074,403.95209715)(401.06519653,404.16210297)
\curveto(400.80519127,404.37209673)(400.42519165,404.47709662)(399.92519653,404.47710297)
\curveto(399.86519221,404.47709662)(399.78519229,404.46709663)(399.68519653,404.44710297)
\curveto(399.60519247,404.42709667)(399.53019255,404.40709669)(399.46019653,404.38710297)
\curveto(399.40019268,404.37709672)(399.34019274,404.35709674)(399.28019653,404.32710297)
\curveto(399.01019307,404.21709688)(398.80019328,404.04709705)(398.65019653,403.81710297)
\curveto(398.50019358,403.58709751)(398.3801937,403.32709777)(398.29019653,403.03710297)
\curveto(398.26019382,402.93709816)(398.24019384,402.83709826)(398.23019653,402.73710297)
\curveto(398.22019386,402.63709846)(398.20019388,402.53209857)(398.17019653,402.42210297)
\lineto(398.17019653,402.21210297)
\curveto(398.15019393,402.12209898)(398.14519393,401.9970991)(398.15519653,401.83710297)
\curveto(398.16519391,401.68709941)(398.1801939,401.57709952)(398.20019653,401.50710297)
\lineto(398.20019653,401.41710297)
\curveto(398.21019387,401.3970997)(398.21519386,401.37709972)(398.21519653,401.35710297)
\curveto(398.23519384,401.27709982)(398.25019383,401.2020999)(398.26019653,401.13210297)
\curveto(398.2801938,401.06210004)(398.30019378,400.98710011)(398.32019653,400.90710297)
\curveto(398.49019359,400.38710071)(398.7801933,400.0021011)(399.19019653,399.75210297)
\curveto(399.32019276,399.66210144)(399.50019258,399.59210151)(399.73019653,399.54210297)
\curveto(399.77019231,399.53210157)(399.83019225,399.52710157)(399.91019653,399.52710297)
\curveto(399.94019214,399.51710158)(399.98519209,399.50710159)(400.04519653,399.49710297)
\curveto(400.11519196,399.4971016)(400.17019191,399.5021016)(400.21019653,399.51210297)
\curveto(400.29019179,399.53210157)(400.37019171,399.54710155)(400.45019653,399.55710297)
\curveto(400.53019155,399.56710153)(400.61019147,399.58710151)(400.69019653,399.61710297)
\curveto(400.94019114,399.72710137)(401.14019094,399.86710123)(401.29019653,400.03710297)
\curveto(401.44019064,400.20710089)(401.57019051,400.42210068)(401.68019653,400.68210297)
\curveto(401.72019036,400.77210033)(401.75019033,400.86210024)(401.77019653,400.95210297)
\curveto(401.79019029,401.05210005)(401.81019027,401.15709994)(401.83019653,401.26710297)
\curveto(401.84019024,401.31709978)(401.84019024,401.36209974)(401.83019653,401.40210297)
\curveto(401.83019025,401.45209965)(401.84019024,401.5020996)(401.86019653,401.55210297)
\curveto(401.87019021,401.58209952)(401.8751902,401.61709948)(401.87519653,401.65710297)
\lineto(401.87519653,401.79210297)
\lineto(401.87519653,401.92710297)
}
}
{
\newrgbcolor{curcolor}{0 0 0}
\pscustom[linestyle=none,fillstyle=solid,fillcolor=curcolor]
{
\newpath
\moveto(412.86011841,402.07710297)
\curveto(412.88011024,401.9970991)(412.88011024,401.90709919)(412.86011841,401.80710297)
\curveto(412.84011028,401.70709939)(412.80511032,401.64209946)(412.75511841,401.61210297)
\curveto(412.70511042,401.57209953)(412.63011049,401.54209956)(412.53011841,401.52210297)
\curveto(412.44011068,401.51209959)(412.33511079,401.5020996)(412.21511841,401.49210297)
\lineto(411.87011841,401.49210297)
\curveto(411.76011136,401.5020996)(411.66011146,401.50709959)(411.57011841,401.50710297)
\lineto(407.91011841,401.50710297)
\lineto(407.70011841,401.50710297)
\curveto(407.64011548,401.50709959)(407.58511554,401.4970996)(407.53511841,401.47710297)
\curveto(407.45511567,401.43709966)(407.40511572,401.3970997)(407.38511841,401.35710297)
\curveto(407.36511576,401.33709976)(407.34511578,401.2970998)(407.32511841,401.23710297)
\curveto(407.30511582,401.18709991)(407.30011582,401.13709996)(407.31011841,401.08710297)
\curveto(407.33011579,401.02710007)(407.34011578,400.96710013)(407.34011841,400.90710297)
\curveto(407.35011577,400.85710024)(407.36511576,400.8021003)(407.38511841,400.74210297)
\curveto(407.46511566,400.5021006)(407.56011556,400.3021008)(407.67011841,400.14210297)
\curveto(407.79011533,399.99210111)(407.95011517,399.85710124)(408.15011841,399.73710297)
\curveto(408.23011489,399.68710141)(408.31011481,399.65210145)(408.39011841,399.63210297)
\curveto(408.48011464,399.62210148)(408.57011455,399.6021015)(408.66011841,399.57210297)
\curveto(408.74011438,399.55210155)(408.85011427,399.53710156)(408.99011841,399.52710297)
\curveto(409.13011399,399.51710158)(409.25011387,399.52210158)(409.35011841,399.54210297)
\lineto(409.48511841,399.54210297)
\curveto(409.58511354,399.56210154)(409.67511345,399.58210152)(409.75511841,399.60210297)
\curveto(409.84511328,399.63210147)(409.93011319,399.66210144)(410.01011841,399.69210297)
\curveto(410.11011301,399.74210136)(410.2201129,399.80710129)(410.34011841,399.88710297)
\curveto(410.47011265,399.96710113)(410.56511256,400.04710105)(410.62511841,400.12710297)
\curveto(410.67511245,400.1971009)(410.7251124,400.26210084)(410.77511841,400.32210297)
\curveto(410.83511229,400.39210071)(410.90511222,400.44210066)(410.98511841,400.47210297)
\curveto(411.08511204,400.52210058)(411.21011191,400.54210056)(411.36011841,400.53210297)
\lineto(411.79511841,400.53210297)
\lineto(411.97511841,400.53210297)
\curveto(412.04511108,400.54210056)(412.10511102,400.53710056)(412.15511841,400.51710297)
\lineto(412.30511841,400.51710297)
\curveto(412.40511072,400.4971006)(412.47511065,400.47210063)(412.51511841,400.44210297)
\curveto(412.55511057,400.42210068)(412.57511055,400.37710072)(412.57511841,400.30710297)
\curveto(412.58511054,400.23710086)(412.58011054,400.17710092)(412.56011841,400.12710297)
\curveto(412.51011061,399.98710111)(412.45511067,399.86210124)(412.39511841,399.75210297)
\curveto(412.33511079,399.64210146)(412.26511086,399.53210157)(412.18511841,399.42210297)
\curveto(411.96511116,399.09210201)(411.71511141,398.82710227)(411.43511841,398.62710297)
\curveto(411.15511197,398.42710267)(410.80511232,398.25710284)(410.38511841,398.11710297)
\curveto(410.27511285,398.07710302)(410.16511296,398.05210305)(410.05511841,398.04210297)
\curveto(409.94511318,398.03210307)(409.83011329,398.01210309)(409.71011841,397.98210297)
\curveto(409.67011345,397.97210313)(409.6251135,397.97210313)(409.57511841,397.98210297)
\curveto(409.53511359,397.98210312)(409.49511363,397.97710312)(409.45511841,397.96710297)
\lineto(409.29011841,397.96710297)
\curveto(409.24011388,397.94710315)(409.18011394,397.94210316)(409.11011841,397.95210297)
\curveto(409.05011407,397.95210315)(408.99511413,397.95710314)(408.94511841,397.96710297)
\curveto(408.86511426,397.97710312)(408.79511433,397.97710312)(408.73511841,397.96710297)
\curveto(408.67511445,397.95710314)(408.61011451,397.96210314)(408.54011841,397.98210297)
\curveto(408.49011463,398.0021031)(408.43511469,398.01210309)(408.37511841,398.01210297)
\curveto(408.31511481,398.01210309)(408.26011486,398.02210308)(408.21011841,398.04210297)
\curveto(408.10011502,398.06210304)(407.99011513,398.08710301)(407.88011841,398.11710297)
\curveto(407.77011535,398.13710296)(407.67011545,398.17210293)(407.58011841,398.22210297)
\curveto(407.47011565,398.26210284)(407.36511576,398.2971028)(407.26511841,398.32710297)
\curveto(407.17511595,398.36710273)(407.09011603,398.41210269)(407.01011841,398.46210297)
\curveto(406.69011643,398.66210244)(406.40511672,398.89210221)(406.15511841,399.15210297)
\curveto(405.90511722,399.42210168)(405.70011742,399.73210137)(405.54011841,400.08210297)
\curveto(405.49011763,400.19210091)(405.45011767,400.3021008)(405.42011841,400.41210297)
\curveto(405.39011773,400.53210057)(405.35011777,400.65210045)(405.30011841,400.77210297)
\curveto(405.29011783,400.81210029)(405.28511784,400.84710025)(405.28511841,400.87710297)
\curveto(405.28511784,400.91710018)(405.28011784,400.95710014)(405.27011841,400.99710297)
\curveto(405.23011789,401.11709998)(405.20511792,401.24709985)(405.19511841,401.38710297)
\lineto(405.16511841,401.80710297)
\curveto(405.16511796,401.85709924)(405.16011796,401.91209919)(405.15011841,401.97210297)
\curveto(405.15011797,402.03209907)(405.15511797,402.08709901)(405.16511841,402.13710297)
\lineto(405.16511841,402.31710297)
\lineto(405.21011841,402.67710297)
\curveto(405.25011787,402.84709825)(405.28511784,403.01209809)(405.31511841,403.17210297)
\curveto(405.34511778,403.33209777)(405.39011773,403.48209762)(405.45011841,403.62210297)
\curveto(405.88011724,404.66209644)(406.61011651,405.3970957)(407.64011841,405.82710297)
\curveto(407.78011534,405.88709521)(407.9201152,405.92709517)(408.06011841,405.94710297)
\curveto(408.21011491,405.97709512)(408.36511476,406.01209509)(408.52511841,406.05210297)
\curveto(408.60511452,406.06209504)(408.68011444,406.06709503)(408.75011841,406.06710297)
\curveto(408.8201143,406.06709503)(408.89511423,406.07209503)(408.97511841,406.08210297)
\curveto(409.48511364,406.09209501)(409.9201132,406.03209507)(410.28011841,405.90210297)
\curveto(410.65011247,405.78209532)(410.98011214,405.62209548)(411.27011841,405.42210297)
\curveto(411.36011176,405.36209574)(411.45011167,405.29209581)(411.54011841,405.21210297)
\curveto(411.63011149,405.14209596)(411.71011141,405.06709603)(411.78011841,404.98710297)
\curveto(411.81011131,404.93709616)(411.85011127,404.8970962)(411.90011841,404.86710297)
\curveto(411.98011114,404.75709634)(412.05511107,404.64209646)(412.12511841,404.52210297)
\curveto(412.19511093,404.41209669)(412.27011085,404.2970968)(412.35011841,404.17710297)
\curveto(412.40011072,404.08709701)(412.44011068,403.99209711)(412.47011841,403.89210297)
\curveto(412.51011061,403.8020973)(412.55011057,403.7020974)(412.59011841,403.59210297)
\curveto(412.64011048,403.46209764)(412.68011044,403.32709777)(412.71011841,403.18710297)
\curveto(412.74011038,403.04709805)(412.77511035,402.90709819)(412.81511841,402.76710297)
\curveto(412.83511029,402.68709841)(412.84011028,402.5970985)(412.83011841,402.49710297)
\curveto(412.83011029,402.40709869)(412.84011028,402.32209878)(412.86011841,402.24210297)
\lineto(412.86011841,402.07710297)
\moveto(410.61011841,402.96210297)
\curveto(410.68011244,403.06209804)(410.68511244,403.18209792)(410.62511841,403.32210297)
\curveto(410.57511255,403.47209763)(410.53511259,403.58209752)(410.50511841,403.65210297)
\curveto(410.36511276,403.92209718)(410.18011294,404.12709697)(409.95011841,404.26710297)
\curveto(409.7201134,404.41709668)(409.40011372,404.4970966)(408.99011841,404.50710297)
\curveto(408.96011416,404.48709661)(408.9251142,404.48209662)(408.88511841,404.49210297)
\curveto(408.84511428,404.5020966)(408.81011431,404.5020966)(408.78011841,404.49210297)
\curveto(408.73011439,404.47209663)(408.67511445,404.45709664)(408.61511841,404.44710297)
\curveto(408.55511457,404.44709665)(408.50011462,404.43709666)(408.45011841,404.41710297)
\curveto(408.01011511,404.27709682)(407.68511544,404.0020971)(407.47511841,403.59210297)
\curveto(407.45511567,403.55209755)(407.43011569,403.4970976)(407.40011841,403.42710297)
\curveto(407.38011574,403.36709773)(407.36511576,403.3020978)(407.35511841,403.23210297)
\curveto(407.34511578,403.17209793)(407.34511578,403.11209799)(407.35511841,403.05210297)
\curveto(407.37511575,402.99209811)(407.41011571,402.94209816)(407.46011841,402.90210297)
\curveto(407.54011558,402.85209825)(407.65011547,402.82709827)(407.79011841,402.82710297)
\lineto(408.19511841,402.82710297)
\lineto(409.86011841,402.82710297)
\lineto(410.29511841,402.82710297)
\curveto(410.45511267,402.83709826)(410.56011256,402.88209822)(410.61011841,402.96210297)
}
}
{
\newrgbcolor{curcolor}{0 0 0}
\pscustom[linestyle=none,fillstyle=solid,fillcolor=curcolor]
{
}
}
{
\newrgbcolor{curcolor}{0 0 0}
\pscustom[linestyle=none,fillstyle=solid,fillcolor=curcolor]
{
\newpath
\moveto(422.69355591,406.06710297)
\curveto(422.80355059,406.06709503)(422.8985505,406.05709504)(422.97855591,406.03710297)
\curveto(423.06855033,406.01709508)(423.13855026,405.97209513)(423.18855591,405.90210297)
\curveto(423.24855015,405.82209528)(423.27855012,405.68209542)(423.27855591,405.48210297)
\lineto(423.27855591,404.97210297)
\lineto(423.27855591,404.59710297)
\curveto(423.28855011,404.45709664)(423.27355012,404.34709675)(423.23355591,404.26710297)
\curveto(423.1935502,404.1970969)(423.13355026,404.15209695)(423.05355591,404.13210297)
\curveto(422.98355041,404.11209699)(422.8985505,404.102097)(422.79855591,404.10210297)
\curveto(422.70855069,404.102097)(422.60855079,404.10709699)(422.49855591,404.11710297)
\curveto(422.398551,404.12709697)(422.30355109,404.12209698)(422.21355591,404.10210297)
\curveto(422.14355125,404.08209702)(422.07355132,404.06709703)(422.00355591,404.05710297)
\curveto(421.93355146,404.05709704)(421.86855153,404.04709705)(421.80855591,404.02710297)
\curveto(421.64855175,403.97709712)(421.48855191,403.9020972)(421.32855591,403.80210297)
\curveto(421.16855223,403.71209739)(421.04355235,403.60709749)(420.95355591,403.48710297)
\curveto(420.90355249,403.40709769)(420.84855255,403.32209778)(420.78855591,403.23210297)
\curveto(420.73855266,403.15209795)(420.68855271,403.06709803)(420.63855591,402.97710297)
\curveto(420.60855279,402.8970982)(420.57855282,402.81209829)(420.54855591,402.72210297)
\lineto(420.48855591,402.48210297)
\curveto(420.46855293,402.41209869)(420.45855294,402.33709876)(420.45855591,402.25710297)
\curveto(420.45855294,402.18709891)(420.44855295,402.11709898)(420.42855591,402.04710297)
\curveto(420.41855298,402.00709909)(420.41355298,401.96709913)(420.41355591,401.92710297)
\curveto(420.42355297,401.8970992)(420.42355297,401.86709923)(420.41355591,401.83710297)
\lineto(420.41355591,401.59710297)
\curveto(420.393553,401.52709957)(420.38855301,401.44709965)(420.39855591,401.35710297)
\curveto(420.40855299,401.27709982)(420.41355298,401.1970999)(420.41355591,401.11710297)
\lineto(420.41355591,400.15710297)
\lineto(420.41355591,398.88210297)
\curveto(420.41355298,398.75210235)(420.40855299,398.63210247)(420.39855591,398.52210297)
\curveto(420.38855301,398.41210269)(420.35855304,398.32210278)(420.30855591,398.25210297)
\curveto(420.28855311,398.22210288)(420.25355314,398.1971029)(420.20355591,398.17710297)
\curveto(420.16355323,398.16710293)(420.11855328,398.15710294)(420.06855591,398.14710297)
\lineto(419.99355591,398.14710297)
\curveto(419.94355345,398.13710296)(419.88855351,398.13210297)(419.82855591,398.13210297)
\lineto(419.66355591,398.13210297)
\lineto(419.01855591,398.13210297)
\curveto(418.95855444,398.14210296)(418.8935545,398.14710295)(418.82355591,398.14710297)
\lineto(418.62855591,398.14710297)
\curveto(418.57855482,398.16710293)(418.52855487,398.18210292)(418.47855591,398.19210297)
\curveto(418.42855497,398.21210289)(418.393555,398.24710285)(418.37355591,398.29710297)
\curveto(418.33355506,398.34710275)(418.30855509,398.41710268)(418.29855591,398.50710297)
\lineto(418.29855591,398.80710297)
\lineto(418.29855591,399.82710297)
\lineto(418.29855591,404.05710297)
\lineto(418.29855591,405.16710297)
\lineto(418.29855591,405.45210297)
\curveto(418.2985551,405.55209555)(418.31855508,405.63209547)(418.35855591,405.69210297)
\curveto(418.40855499,405.77209533)(418.48355491,405.82209528)(418.58355591,405.84210297)
\curveto(418.68355471,405.86209524)(418.80355459,405.87209523)(418.94355591,405.87210297)
\lineto(419.70855591,405.87210297)
\curveto(419.82855357,405.87209523)(419.93355346,405.86209524)(420.02355591,405.84210297)
\curveto(420.11355328,405.83209527)(420.18355321,405.78709531)(420.23355591,405.70710297)
\curveto(420.26355313,405.65709544)(420.27855312,405.58709551)(420.27855591,405.49710297)
\lineto(420.30855591,405.22710297)
\curveto(420.31855308,405.14709595)(420.33355306,405.07209603)(420.35355591,405.00210297)
\curveto(420.38355301,404.93209617)(420.43355296,404.8970962)(420.50355591,404.89710297)
\curveto(420.52355287,404.91709618)(420.54355285,404.92709617)(420.56355591,404.92710297)
\curveto(420.58355281,404.92709617)(420.60355279,404.93709616)(420.62355591,404.95710297)
\curveto(420.68355271,405.00709609)(420.73355266,405.06209604)(420.77355591,405.12210297)
\curveto(420.82355257,405.19209591)(420.88355251,405.25209585)(420.95355591,405.30210297)
\curveto(420.9935524,405.33209577)(421.02855237,405.36209574)(421.05855591,405.39210297)
\curveto(421.08855231,405.43209567)(421.12355227,405.46709563)(421.16355591,405.49710297)
\lineto(421.43355591,405.67710297)
\curveto(421.53355186,405.73709536)(421.63355176,405.79209531)(421.73355591,405.84210297)
\curveto(421.83355156,405.88209522)(421.93355146,405.91709518)(422.03355591,405.94710297)
\lineto(422.36355591,406.03710297)
\curveto(422.393551,406.04709505)(422.44855095,406.04709505)(422.52855591,406.03710297)
\curveto(422.61855078,406.03709506)(422.67355072,406.04709505)(422.69355591,406.06710297)
}
}
{
\newrgbcolor{curcolor}{0 0 0}
\pscustom[linestyle=none,fillstyle=solid,fillcolor=curcolor]
{
\newpath
\moveto(431.19996216,402.07710297)
\curveto(431.21995399,401.9970991)(431.21995399,401.90709919)(431.19996216,401.80710297)
\curveto(431.17995403,401.70709939)(431.14495407,401.64209946)(431.09496216,401.61210297)
\curveto(431.04495417,401.57209953)(430.96995424,401.54209956)(430.86996216,401.52210297)
\curveto(430.77995443,401.51209959)(430.67495454,401.5020996)(430.55496216,401.49210297)
\lineto(430.20996216,401.49210297)
\curveto(430.09995511,401.5020996)(429.99995521,401.50709959)(429.90996216,401.50710297)
\lineto(426.24996216,401.50710297)
\lineto(426.03996216,401.50710297)
\curveto(425.97995923,401.50709959)(425.92495929,401.4970996)(425.87496216,401.47710297)
\curveto(425.79495942,401.43709966)(425.74495947,401.3970997)(425.72496216,401.35710297)
\curveto(425.70495951,401.33709976)(425.68495953,401.2970998)(425.66496216,401.23710297)
\curveto(425.64495957,401.18709991)(425.63995957,401.13709996)(425.64996216,401.08710297)
\curveto(425.66995954,401.02710007)(425.67995953,400.96710013)(425.67996216,400.90710297)
\curveto(425.68995952,400.85710024)(425.70495951,400.8021003)(425.72496216,400.74210297)
\curveto(425.80495941,400.5021006)(425.89995931,400.3021008)(426.00996216,400.14210297)
\curveto(426.12995908,399.99210111)(426.28995892,399.85710124)(426.48996216,399.73710297)
\curveto(426.56995864,399.68710141)(426.64995856,399.65210145)(426.72996216,399.63210297)
\curveto(426.81995839,399.62210148)(426.9099583,399.6021015)(426.99996216,399.57210297)
\curveto(427.07995813,399.55210155)(427.18995802,399.53710156)(427.32996216,399.52710297)
\curveto(427.46995774,399.51710158)(427.58995762,399.52210158)(427.68996216,399.54210297)
\lineto(427.82496216,399.54210297)
\curveto(427.92495729,399.56210154)(428.0149572,399.58210152)(428.09496216,399.60210297)
\curveto(428.18495703,399.63210147)(428.26995694,399.66210144)(428.34996216,399.69210297)
\curveto(428.44995676,399.74210136)(428.55995665,399.80710129)(428.67996216,399.88710297)
\curveto(428.8099564,399.96710113)(428.90495631,400.04710105)(428.96496216,400.12710297)
\curveto(429.0149562,400.1971009)(429.06495615,400.26210084)(429.11496216,400.32210297)
\curveto(429.17495604,400.39210071)(429.24495597,400.44210066)(429.32496216,400.47210297)
\curveto(429.42495579,400.52210058)(429.54995566,400.54210056)(429.69996216,400.53210297)
\lineto(430.13496216,400.53210297)
\lineto(430.31496216,400.53210297)
\curveto(430.38495483,400.54210056)(430.44495477,400.53710056)(430.49496216,400.51710297)
\lineto(430.64496216,400.51710297)
\curveto(430.74495447,400.4971006)(430.8149544,400.47210063)(430.85496216,400.44210297)
\curveto(430.89495432,400.42210068)(430.9149543,400.37710072)(430.91496216,400.30710297)
\curveto(430.92495429,400.23710086)(430.91995429,400.17710092)(430.89996216,400.12710297)
\curveto(430.84995436,399.98710111)(430.79495442,399.86210124)(430.73496216,399.75210297)
\curveto(430.67495454,399.64210146)(430.60495461,399.53210157)(430.52496216,399.42210297)
\curveto(430.30495491,399.09210201)(430.05495516,398.82710227)(429.77496216,398.62710297)
\curveto(429.49495572,398.42710267)(429.14495607,398.25710284)(428.72496216,398.11710297)
\curveto(428.6149566,398.07710302)(428.50495671,398.05210305)(428.39496216,398.04210297)
\curveto(428.28495693,398.03210307)(428.16995704,398.01210309)(428.04996216,397.98210297)
\curveto(428.0099572,397.97210313)(427.96495725,397.97210313)(427.91496216,397.98210297)
\curveto(427.87495734,397.98210312)(427.83495738,397.97710312)(427.79496216,397.96710297)
\lineto(427.62996216,397.96710297)
\curveto(427.57995763,397.94710315)(427.51995769,397.94210316)(427.44996216,397.95210297)
\curveto(427.38995782,397.95210315)(427.33495788,397.95710314)(427.28496216,397.96710297)
\curveto(427.20495801,397.97710312)(427.13495808,397.97710312)(427.07496216,397.96710297)
\curveto(427.0149582,397.95710314)(426.94995826,397.96210314)(426.87996216,397.98210297)
\curveto(426.82995838,398.0021031)(426.77495844,398.01210309)(426.71496216,398.01210297)
\curveto(426.65495856,398.01210309)(426.59995861,398.02210308)(426.54996216,398.04210297)
\curveto(426.43995877,398.06210304)(426.32995888,398.08710301)(426.21996216,398.11710297)
\curveto(426.1099591,398.13710296)(426.0099592,398.17210293)(425.91996216,398.22210297)
\curveto(425.8099594,398.26210284)(425.70495951,398.2971028)(425.60496216,398.32710297)
\curveto(425.5149597,398.36710273)(425.42995978,398.41210269)(425.34996216,398.46210297)
\curveto(425.02996018,398.66210244)(424.74496047,398.89210221)(424.49496216,399.15210297)
\curveto(424.24496097,399.42210168)(424.03996117,399.73210137)(423.87996216,400.08210297)
\curveto(423.82996138,400.19210091)(423.78996142,400.3021008)(423.75996216,400.41210297)
\curveto(423.72996148,400.53210057)(423.68996152,400.65210045)(423.63996216,400.77210297)
\curveto(423.62996158,400.81210029)(423.62496159,400.84710025)(423.62496216,400.87710297)
\curveto(423.62496159,400.91710018)(423.61996159,400.95710014)(423.60996216,400.99710297)
\curveto(423.56996164,401.11709998)(423.54496167,401.24709985)(423.53496216,401.38710297)
\lineto(423.50496216,401.80710297)
\curveto(423.50496171,401.85709924)(423.49996171,401.91209919)(423.48996216,401.97210297)
\curveto(423.48996172,402.03209907)(423.49496172,402.08709901)(423.50496216,402.13710297)
\lineto(423.50496216,402.31710297)
\lineto(423.54996216,402.67710297)
\curveto(423.58996162,402.84709825)(423.62496159,403.01209809)(423.65496216,403.17210297)
\curveto(423.68496153,403.33209777)(423.72996148,403.48209762)(423.78996216,403.62210297)
\curveto(424.21996099,404.66209644)(424.94996026,405.3970957)(425.97996216,405.82710297)
\curveto(426.11995909,405.88709521)(426.25995895,405.92709517)(426.39996216,405.94710297)
\curveto(426.54995866,405.97709512)(426.70495851,406.01209509)(426.86496216,406.05210297)
\curveto(426.94495827,406.06209504)(427.01995819,406.06709503)(427.08996216,406.06710297)
\curveto(427.15995805,406.06709503)(427.23495798,406.07209503)(427.31496216,406.08210297)
\curveto(427.82495739,406.09209501)(428.25995695,406.03209507)(428.61996216,405.90210297)
\curveto(428.98995622,405.78209532)(429.31995589,405.62209548)(429.60996216,405.42210297)
\curveto(429.69995551,405.36209574)(429.78995542,405.29209581)(429.87996216,405.21210297)
\curveto(429.96995524,405.14209596)(430.04995516,405.06709603)(430.11996216,404.98710297)
\curveto(430.14995506,404.93709616)(430.18995502,404.8970962)(430.23996216,404.86710297)
\curveto(430.31995489,404.75709634)(430.39495482,404.64209646)(430.46496216,404.52210297)
\curveto(430.53495468,404.41209669)(430.6099546,404.2970968)(430.68996216,404.17710297)
\curveto(430.73995447,404.08709701)(430.77995443,403.99209711)(430.80996216,403.89210297)
\curveto(430.84995436,403.8020973)(430.88995432,403.7020974)(430.92996216,403.59210297)
\curveto(430.97995423,403.46209764)(431.01995419,403.32709777)(431.04996216,403.18710297)
\curveto(431.07995413,403.04709805)(431.1149541,402.90709819)(431.15496216,402.76710297)
\curveto(431.17495404,402.68709841)(431.17995403,402.5970985)(431.16996216,402.49710297)
\curveto(431.16995404,402.40709869)(431.17995403,402.32209878)(431.19996216,402.24210297)
\lineto(431.19996216,402.07710297)
\moveto(428.94996216,402.96210297)
\curveto(429.01995619,403.06209804)(429.02495619,403.18209792)(428.96496216,403.32210297)
\curveto(428.9149563,403.47209763)(428.87495634,403.58209752)(428.84496216,403.65210297)
\curveto(428.70495651,403.92209718)(428.51995669,404.12709697)(428.28996216,404.26710297)
\curveto(428.05995715,404.41709668)(427.73995747,404.4970966)(427.32996216,404.50710297)
\curveto(427.29995791,404.48709661)(427.26495795,404.48209662)(427.22496216,404.49210297)
\curveto(427.18495803,404.5020966)(427.14995806,404.5020966)(427.11996216,404.49210297)
\curveto(427.06995814,404.47209663)(427.0149582,404.45709664)(426.95496216,404.44710297)
\curveto(426.89495832,404.44709665)(426.83995837,404.43709666)(426.78996216,404.41710297)
\curveto(426.34995886,404.27709682)(426.02495919,404.0020971)(425.81496216,403.59210297)
\curveto(425.79495942,403.55209755)(425.76995944,403.4970976)(425.73996216,403.42710297)
\curveto(425.71995949,403.36709773)(425.70495951,403.3020978)(425.69496216,403.23210297)
\curveto(425.68495953,403.17209793)(425.68495953,403.11209799)(425.69496216,403.05210297)
\curveto(425.7149595,402.99209811)(425.74995946,402.94209816)(425.79996216,402.90210297)
\curveto(425.87995933,402.85209825)(425.98995922,402.82709827)(426.12996216,402.82710297)
\lineto(426.53496216,402.82710297)
\lineto(428.19996216,402.82710297)
\lineto(428.63496216,402.82710297)
\curveto(428.79495642,402.83709826)(428.89995631,402.88209822)(428.94996216,402.96210297)
}
}
{
\newrgbcolor{curcolor}{0 0 0}
\pscustom[linestyle=none,fillstyle=solid,fillcolor=curcolor]
{
\newpath
\moveto(436.01824341,406.08210297)
\curveto(436.82823825,406.102095)(437.50323757,405.98209512)(438.04324341,405.72210297)
\curveto(438.59323648,405.46209564)(439.02823605,405.09209601)(439.34824341,404.61210297)
\curveto(439.50823557,404.37209673)(439.62823545,404.097097)(439.70824341,403.78710297)
\curveto(439.72823535,403.73709736)(439.74323533,403.67209743)(439.75324341,403.59210297)
\curveto(439.7732353,403.51209759)(439.7732353,403.44209766)(439.75324341,403.38210297)
\curveto(439.71323536,403.27209783)(439.64323543,403.20709789)(439.54324341,403.18710297)
\curveto(439.44323563,403.17709792)(439.32323575,403.17209793)(439.18324341,403.17210297)
\lineto(438.40324341,403.17210297)
\lineto(438.11824341,403.17210297)
\curveto(438.02823705,403.17209793)(437.95323712,403.19209791)(437.89324341,403.23210297)
\curveto(437.81323726,403.27209783)(437.75823732,403.33209777)(437.72824341,403.41210297)
\curveto(437.69823738,403.5020976)(437.65823742,403.59209751)(437.60824341,403.68210297)
\curveto(437.54823753,403.79209731)(437.48323759,403.89209721)(437.41324341,403.98210297)
\curveto(437.34323773,404.07209703)(437.26323781,404.15209695)(437.17324341,404.22210297)
\curveto(437.03323804,404.31209679)(436.8782382,404.38209672)(436.70824341,404.43210297)
\curveto(436.64823843,404.45209665)(436.58823849,404.46209664)(436.52824341,404.46210297)
\curveto(436.46823861,404.46209664)(436.41323866,404.47209663)(436.36324341,404.49210297)
\lineto(436.21324341,404.49210297)
\curveto(436.01323906,404.49209661)(435.85323922,404.47209663)(435.73324341,404.43210297)
\curveto(435.44323963,404.34209676)(435.20823987,404.2020969)(435.02824341,404.01210297)
\curveto(434.84824023,403.83209727)(434.70324037,403.61209749)(434.59324341,403.35210297)
\curveto(434.54324053,403.24209786)(434.50324057,403.12209798)(434.47324341,402.99210297)
\curveto(434.45324062,402.87209823)(434.42824065,402.74209836)(434.39824341,402.60210297)
\curveto(434.38824069,402.56209854)(434.38324069,402.52209858)(434.38324341,402.48210297)
\curveto(434.38324069,402.44209866)(434.3782407,402.4020987)(434.36824341,402.36210297)
\curveto(434.34824073,402.26209884)(434.33824074,402.12209898)(434.33824341,401.94210297)
\curveto(434.34824073,401.76209934)(434.36324071,401.62209948)(434.38324341,401.52210297)
\curveto(434.38324069,401.44209966)(434.38824069,401.38709971)(434.39824341,401.35710297)
\curveto(434.41824066,401.28709981)(434.42824065,401.21709988)(434.42824341,401.14710297)
\curveto(434.43824064,401.07710002)(434.45324062,401.00710009)(434.47324341,400.93710297)
\curveto(434.55324052,400.70710039)(434.64824043,400.4971006)(434.75824341,400.30710297)
\curveto(434.86824021,400.11710098)(435.00824007,399.95710114)(435.17824341,399.82710297)
\curveto(435.21823986,399.7971013)(435.2782398,399.76210134)(435.35824341,399.72210297)
\curveto(435.46823961,399.65210145)(435.5782395,399.60710149)(435.68824341,399.58710297)
\curveto(435.80823927,399.56710153)(435.95323912,399.54710155)(436.12324341,399.52710297)
\lineto(436.21324341,399.52710297)
\curveto(436.25323882,399.52710157)(436.28323879,399.53210157)(436.30324341,399.54210297)
\lineto(436.43824341,399.54210297)
\curveto(436.50823857,399.56210154)(436.5732385,399.57710152)(436.63324341,399.58710297)
\curveto(436.70323837,399.60710149)(436.76823831,399.62710147)(436.82824341,399.64710297)
\curveto(437.12823795,399.77710132)(437.35823772,399.96710113)(437.51824341,400.21710297)
\curveto(437.55823752,400.26710083)(437.59323748,400.32210078)(437.62324341,400.38210297)
\curveto(437.65323742,400.45210065)(437.6782374,400.51210059)(437.69824341,400.56210297)
\curveto(437.73823734,400.67210043)(437.7732373,400.76710033)(437.80324341,400.84710297)
\curveto(437.83323724,400.93710016)(437.90323717,401.00710009)(438.01324341,401.05710297)
\curveto(438.10323697,401.0971)(438.24823683,401.11209999)(438.44824341,401.10210297)
\lineto(438.94324341,401.10210297)
\lineto(439.15324341,401.10210297)
\curveto(439.23323584,401.11209999)(439.29823578,401.10709999)(439.34824341,401.08710297)
\lineto(439.46824341,401.08710297)
\lineto(439.58824341,401.05710297)
\curveto(439.62823545,401.05710004)(439.65823542,401.04710005)(439.67824341,401.02710297)
\curveto(439.72823535,400.98710011)(439.75823532,400.92710017)(439.76824341,400.84710297)
\curveto(439.78823529,400.77710032)(439.78823529,400.7021004)(439.76824341,400.62210297)
\curveto(439.6782354,400.29210081)(439.56823551,399.9971011)(439.43824341,399.73710297)
\curveto(439.02823605,398.96710213)(438.3732367,398.43210267)(437.47324341,398.13210297)
\curveto(437.3732377,398.102103)(437.26823781,398.08210302)(437.15824341,398.07210297)
\curveto(437.04823803,398.05210305)(436.93823814,398.02710307)(436.82824341,397.99710297)
\curveto(436.76823831,397.98710311)(436.70823837,397.98210312)(436.64824341,397.98210297)
\curveto(436.58823849,397.98210312)(436.52823855,397.97710312)(436.46824341,397.96710297)
\lineto(436.30324341,397.96710297)
\curveto(436.25323882,397.94710315)(436.1782389,397.94210316)(436.07824341,397.95210297)
\curveto(435.9782391,397.95210315)(435.90323917,397.95710314)(435.85324341,397.96710297)
\curveto(435.7732393,397.98710311)(435.69823938,397.9971031)(435.62824341,397.99710297)
\curveto(435.56823951,397.98710311)(435.50323957,397.99210311)(435.43324341,398.01210297)
\lineto(435.28324341,398.04210297)
\curveto(435.23323984,398.04210306)(435.18323989,398.04710305)(435.13324341,398.05710297)
\curveto(435.02324005,398.08710301)(434.91824016,398.11710298)(434.81824341,398.14710297)
\curveto(434.71824036,398.17710292)(434.62324045,398.21210289)(434.53324341,398.25210297)
\curveto(434.06324101,398.45210265)(433.66824141,398.70710239)(433.34824341,399.01710297)
\curveto(433.02824205,399.33710176)(432.76824231,399.73210137)(432.56824341,400.20210297)
\curveto(432.51824256,400.29210081)(432.4782426,400.38710071)(432.44824341,400.48710297)
\lineto(432.35824341,400.81710297)
\curveto(432.34824273,400.85710024)(432.34324273,400.89210021)(432.34324341,400.92210297)
\curveto(432.34324273,400.96210014)(432.33324274,401.00710009)(432.31324341,401.05710297)
\curveto(432.29324278,401.12709997)(432.28324279,401.1970999)(432.28324341,401.26710297)
\curveto(432.28324279,401.34709975)(432.2732428,401.42209968)(432.25324341,401.49210297)
\lineto(432.25324341,401.74710297)
\curveto(432.23324284,401.7970993)(432.22324285,401.85209925)(432.22324341,401.91210297)
\curveto(432.22324285,401.98209912)(432.23324284,402.04209906)(432.25324341,402.09210297)
\curveto(432.26324281,402.14209896)(432.26324281,402.18709891)(432.25324341,402.22710297)
\curveto(432.24324283,402.26709883)(432.24324283,402.30709879)(432.25324341,402.34710297)
\curveto(432.2732428,402.41709868)(432.2782428,402.48209862)(432.26824341,402.54210297)
\curveto(432.26824281,402.6020985)(432.2782428,402.66209844)(432.29824341,402.72210297)
\curveto(432.34824273,402.9020982)(432.38824269,403.07209803)(432.41824341,403.23210297)
\curveto(432.44824263,403.4020977)(432.49324258,403.56709753)(432.55324341,403.72710297)
\curveto(432.7732423,404.23709686)(433.04824203,404.66209644)(433.37824341,405.00210297)
\curveto(433.71824136,405.34209576)(434.14824093,405.61709548)(434.66824341,405.82710297)
\curveto(434.80824027,405.88709521)(434.95324012,405.92709517)(435.10324341,405.94710297)
\curveto(435.25323982,405.97709512)(435.40823967,406.01209509)(435.56824341,406.05210297)
\curveto(435.64823943,406.06209504)(435.72323935,406.06709503)(435.79324341,406.06710297)
\curveto(435.86323921,406.06709503)(435.93823914,406.07209503)(436.01824341,406.08210297)
}
}
{
\newrgbcolor{curcolor}{0 0 0}
\pscustom[linestyle=none,fillstyle=solid,fillcolor=curcolor]
{
\newpath
\moveto(441.48152466,405.85710297)
\lineto(442.60652466,405.85710297)
\curveto(442.71652222,405.85709524)(442.81652212,405.85209525)(442.90652466,405.84210297)
\curveto(442.99652194,405.83209527)(443.06152188,405.7970953)(443.10152466,405.73710297)
\curveto(443.15152179,405.67709542)(443.18152176,405.59209551)(443.19152466,405.48210297)
\curveto(443.20152174,405.38209572)(443.20652173,405.27709582)(443.20652466,405.16710297)
\lineto(443.20652466,404.11710297)
\lineto(443.20652466,401.88210297)
\curveto(443.20652173,401.52209958)(443.22152172,401.18209992)(443.25152466,400.86210297)
\curveto(443.28152166,400.54210056)(443.37152157,400.27710082)(443.52152466,400.06710297)
\curveto(443.66152128,399.85710124)(443.88652105,399.70710139)(444.19652466,399.61710297)
\curveto(444.24652069,399.60710149)(444.28652065,399.6021015)(444.31652466,399.60210297)
\curveto(444.35652058,399.6021015)(444.40152054,399.5971015)(444.45152466,399.58710297)
\curveto(444.50152044,399.57710152)(444.55652038,399.57210153)(444.61652466,399.57210297)
\curveto(444.67652026,399.57210153)(444.72152022,399.57710152)(444.75152466,399.58710297)
\curveto(444.80152014,399.60710149)(444.8415201,399.61210149)(444.87152466,399.60210297)
\curveto(444.91152003,399.59210151)(444.95151999,399.5971015)(444.99152466,399.61710297)
\curveto(445.20151974,399.66710143)(445.36651957,399.73210137)(445.48652466,399.81210297)
\curveto(445.66651927,399.92210118)(445.80651913,400.06210104)(445.90652466,400.23210297)
\curveto(446.01651892,400.41210069)(446.09151885,400.60710049)(446.13152466,400.81710297)
\curveto(446.18151876,401.03710006)(446.21151873,401.27709982)(446.22152466,401.53710297)
\curveto(446.23151871,401.80709929)(446.2365187,402.08709901)(446.23652466,402.37710297)
\lineto(446.23652466,404.19210297)
\lineto(446.23652466,405.16710297)
\lineto(446.23652466,405.43710297)
\curveto(446.2365187,405.53709556)(446.25651868,405.61709548)(446.29652466,405.67710297)
\curveto(446.34651859,405.76709533)(446.42151852,405.81709528)(446.52152466,405.82710297)
\curveto(446.62151832,405.84709525)(446.7415182,405.85709524)(446.88152466,405.85710297)
\lineto(447.67652466,405.85710297)
\lineto(447.96152466,405.85710297)
\curveto(448.05151689,405.85709524)(448.12651681,405.83709526)(448.18652466,405.79710297)
\curveto(448.26651667,405.74709535)(448.31151663,405.67209543)(448.32152466,405.57210297)
\curveto(448.33151661,405.47209563)(448.3365166,405.35709574)(448.33652466,405.22710297)
\lineto(448.33652466,404.08710297)
\lineto(448.33652466,399.87210297)
\lineto(448.33652466,398.80710297)
\lineto(448.33652466,398.50710297)
\curveto(448.3365166,398.40710269)(448.31651662,398.33210277)(448.27652466,398.28210297)
\curveto(448.22651671,398.2021029)(448.15151679,398.15710294)(448.05152466,398.14710297)
\curveto(447.95151699,398.13710296)(447.84651709,398.13210297)(447.73652466,398.13210297)
\lineto(446.92652466,398.13210297)
\curveto(446.81651812,398.13210297)(446.71651822,398.13710296)(446.62652466,398.14710297)
\curveto(446.54651839,398.15710294)(446.48151846,398.1971029)(446.43152466,398.26710297)
\curveto(446.41151853,398.2971028)(446.39151855,398.34210276)(446.37152466,398.40210297)
\curveto(446.36151858,398.46210264)(446.34651859,398.52210258)(446.32652466,398.58210297)
\curveto(446.31651862,398.64210246)(446.30151864,398.6971024)(446.28152466,398.74710297)
\curveto(446.26151868,398.7971023)(446.23151871,398.82710227)(446.19152466,398.83710297)
\curveto(446.17151877,398.85710224)(446.14651879,398.86210224)(446.11652466,398.85210297)
\curveto(446.08651885,398.84210226)(446.06151888,398.83210227)(446.04152466,398.82210297)
\curveto(445.97151897,398.78210232)(445.91151903,398.73710236)(445.86152466,398.68710297)
\curveto(445.81151913,398.63710246)(445.75651918,398.59210251)(445.69652466,398.55210297)
\curveto(445.65651928,398.52210258)(445.61651932,398.48710261)(445.57652466,398.44710297)
\curveto(445.54651939,398.41710268)(445.50651943,398.38710271)(445.45652466,398.35710297)
\curveto(445.22651971,398.21710288)(444.95651998,398.10710299)(444.64652466,398.02710297)
\curveto(444.57652036,398.00710309)(444.50652043,397.9971031)(444.43652466,397.99710297)
\curveto(444.36652057,397.98710311)(444.29152065,397.97210313)(444.21152466,397.95210297)
\curveto(444.17152077,397.94210316)(444.12652081,397.94210316)(444.07652466,397.95210297)
\curveto(444.0365209,397.95210315)(443.99652094,397.94710315)(443.95652466,397.93710297)
\curveto(443.92652101,397.92710317)(443.86152108,397.92710317)(443.76152466,397.93710297)
\curveto(443.67152127,397.93710316)(443.61152133,397.94210316)(443.58152466,397.95210297)
\curveto(443.53152141,397.95210315)(443.48152146,397.95710314)(443.43152466,397.96710297)
\lineto(443.28152466,397.96710297)
\curveto(443.16152178,397.9971031)(443.04652189,398.02210308)(442.93652466,398.04210297)
\curveto(442.82652211,398.06210304)(442.71652222,398.09210301)(442.60652466,398.13210297)
\curveto(442.55652238,398.15210295)(442.51152243,398.16710293)(442.47152466,398.17710297)
\curveto(442.4415225,398.1971029)(442.40152254,398.21710288)(442.35152466,398.23710297)
\curveto(442.00152294,398.42710267)(441.72152322,398.69210241)(441.51152466,399.03210297)
\curveto(441.38152356,399.24210186)(441.28652365,399.49210161)(441.22652466,399.78210297)
\curveto(441.16652377,400.08210102)(441.12652381,400.3971007)(441.10652466,400.72710297)
\curveto(441.09652384,401.06710003)(441.09152385,401.41209969)(441.09152466,401.76210297)
\curveto(441.10152384,402.12209898)(441.10652383,402.47709862)(441.10652466,402.82710297)
\lineto(441.10652466,404.86710297)
\curveto(441.10652383,404.9970961)(441.10152384,405.14709595)(441.09152466,405.31710297)
\curveto(441.09152385,405.4970956)(441.11652382,405.62709547)(441.16652466,405.70710297)
\curveto(441.19652374,405.75709534)(441.25652368,405.8020953)(441.34652466,405.84210297)
\curveto(441.40652353,405.84209526)(441.45152349,405.84709525)(441.48152466,405.85710297)
}
}
{
\newrgbcolor{curcolor}{0 0 0}
\pscustom[linestyle=none,fillstyle=solid,fillcolor=curcolor]
{
\newpath
\moveto(454.39277466,406.06710297)
\curveto(454.50276934,406.06709503)(454.59776925,406.05709504)(454.67777466,406.03710297)
\curveto(454.76776908,406.01709508)(454.83776901,405.97209513)(454.88777466,405.90210297)
\curveto(454.9477689,405.82209528)(454.97776887,405.68209542)(454.97777466,405.48210297)
\lineto(454.97777466,404.97210297)
\lineto(454.97777466,404.59710297)
\curveto(454.98776886,404.45709664)(454.97276887,404.34709675)(454.93277466,404.26710297)
\curveto(454.89276895,404.1970969)(454.83276901,404.15209695)(454.75277466,404.13210297)
\curveto(454.68276916,404.11209699)(454.59776925,404.102097)(454.49777466,404.10210297)
\curveto(454.40776944,404.102097)(454.30776954,404.10709699)(454.19777466,404.11710297)
\curveto(454.09776975,404.12709697)(454.00276984,404.12209698)(453.91277466,404.10210297)
\curveto(453.84277,404.08209702)(453.77277007,404.06709703)(453.70277466,404.05710297)
\curveto(453.63277021,404.05709704)(453.56777028,404.04709705)(453.50777466,404.02710297)
\curveto(453.3477705,403.97709712)(453.18777066,403.9020972)(453.02777466,403.80210297)
\curveto(452.86777098,403.71209739)(452.7427711,403.60709749)(452.65277466,403.48710297)
\curveto(452.60277124,403.40709769)(452.5477713,403.32209778)(452.48777466,403.23210297)
\curveto(452.43777141,403.15209795)(452.38777146,403.06709803)(452.33777466,402.97710297)
\curveto(452.30777154,402.8970982)(452.27777157,402.81209829)(452.24777466,402.72210297)
\lineto(452.18777466,402.48210297)
\curveto(452.16777168,402.41209869)(452.15777169,402.33709876)(452.15777466,402.25710297)
\curveto(452.15777169,402.18709891)(452.1477717,402.11709898)(452.12777466,402.04710297)
\curveto(452.11777173,402.00709909)(452.11277173,401.96709913)(452.11277466,401.92710297)
\curveto(452.12277172,401.8970992)(452.12277172,401.86709923)(452.11277466,401.83710297)
\lineto(452.11277466,401.59710297)
\curveto(452.09277175,401.52709957)(452.08777176,401.44709965)(452.09777466,401.35710297)
\curveto(452.10777174,401.27709982)(452.11277173,401.1970999)(452.11277466,401.11710297)
\lineto(452.11277466,400.15710297)
\lineto(452.11277466,398.88210297)
\curveto(452.11277173,398.75210235)(452.10777174,398.63210247)(452.09777466,398.52210297)
\curveto(452.08777176,398.41210269)(452.05777179,398.32210278)(452.00777466,398.25210297)
\curveto(451.98777186,398.22210288)(451.95277189,398.1971029)(451.90277466,398.17710297)
\curveto(451.86277198,398.16710293)(451.81777203,398.15710294)(451.76777466,398.14710297)
\lineto(451.69277466,398.14710297)
\curveto(451.6427722,398.13710296)(451.58777226,398.13210297)(451.52777466,398.13210297)
\lineto(451.36277466,398.13210297)
\lineto(450.71777466,398.13210297)
\curveto(450.65777319,398.14210296)(450.59277325,398.14710295)(450.52277466,398.14710297)
\lineto(450.32777466,398.14710297)
\curveto(450.27777357,398.16710293)(450.22777362,398.18210292)(450.17777466,398.19210297)
\curveto(450.12777372,398.21210289)(450.09277375,398.24710285)(450.07277466,398.29710297)
\curveto(450.03277381,398.34710275)(450.00777384,398.41710268)(449.99777466,398.50710297)
\lineto(449.99777466,398.80710297)
\lineto(449.99777466,399.82710297)
\lineto(449.99777466,404.05710297)
\lineto(449.99777466,405.16710297)
\lineto(449.99777466,405.45210297)
\curveto(449.99777385,405.55209555)(450.01777383,405.63209547)(450.05777466,405.69210297)
\curveto(450.10777374,405.77209533)(450.18277366,405.82209528)(450.28277466,405.84210297)
\curveto(450.38277346,405.86209524)(450.50277334,405.87209523)(450.64277466,405.87210297)
\lineto(451.40777466,405.87210297)
\curveto(451.52777232,405.87209523)(451.63277221,405.86209524)(451.72277466,405.84210297)
\curveto(451.81277203,405.83209527)(451.88277196,405.78709531)(451.93277466,405.70710297)
\curveto(451.96277188,405.65709544)(451.97777187,405.58709551)(451.97777466,405.49710297)
\lineto(452.00777466,405.22710297)
\curveto(452.01777183,405.14709595)(452.03277181,405.07209603)(452.05277466,405.00210297)
\curveto(452.08277176,404.93209617)(452.13277171,404.8970962)(452.20277466,404.89710297)
\curveto(452.22277162,404.91709618)(452.2427716,404.92709617)(452.26277466,404.92710297)
\curveto(452.28277156,404.92709617)(452.30277154,404.93709616)(452.32277466,404.95710297)
\curveto(452.38277146,405.00709609)(452.43277141,405.06209604)(452.47277466,405.12210297)
\curveto(452.52277132,405.19209591)(452.58277126,405.25209585)(452.65277466,405.30210297)
\curveto(452.69277115,405.33209577)(452.72777112,405.36209574)(452.75777466,405.39210297)
\curveto(452.78777106,405.43209567)(452.82277102,405.46709563)(452.86277466,405.49710297)
\lineto(453.13277466,405.67710297)
\curveto(453.23277061,405.73709536)(453.33277051,405.79209531)(453.43277466,405.84210297)
\curveto(453.53277031,405.88209522)(453.63277021,405.91709518)(453.73277466,405.94710297)
\lineto(454.06277466,406.03710297)
\curveto(454.09276975,406.04709505)(454.1477697,406.04709505)(454.22777466,406.03710297)
\curveto(454.31776953,406.03709506)(454.37276947,406.04709505)(454.39277466,406.06710297)
}
}
{
\newrgbcolor{curcolor}{0 0 0}
\pscustom[linestyle=none,fillstyle=solid,fillcolor=curcolor]
{
\newpath
\moveto(458.76785278,406.08210297)
\curveto(459.51784828,406.102095)(460.16784763,406.01709508)(460.71785278,405.82710297)
\curveto(461.27784652,405.64709545)(461.7028461,405.33209577)(461.99285278,404.88210297)
\curveto(462.06284574,404.77209633)(462.12284568,404.65709644)(462.17285278,404.53710297)
\curveto(462.23284557,404.42709667)(462.28284552,404.3020968)(462.32285278,404.16210297)
\curveto(462.34284546,404.102097)(462.35284545,404.03709706)(462.35285278,403.96710297)
\curveto(462.35284545,403.8970972)(462.34284546,403.83709726)(462.32285278,403.78710297)
\curveto(462.28284552,403.72709737)(462.22784557,403.68709741)(462.15785278,403.66710297)
\curveto(462.10784569,403.64709745)(462.04784575,403.63709746)(461.97785278,403.63710297)
\lineto(461.76785278,403.63710297)
\lineto(461.10785278,403.63710297)
\curveto(461.03784676,403.63709746)(460.96784683,403.63209747)(460.89785278,403.62210297)
\curveto(460.82784697,403.62209748)(460.76284704,403.63209747)(460.70285278,403.65210297)
\curveto(460.6028472,403.67209743)(460.52784727,403.71209739)(460.47785278,403.77210297)
\curveto(460.42784737,403.83209727)(460.38284742,403.89209721)(460.34285278,403.95210297)
\lineto(460.22285278,404.16210297)
\curveto(460.19284761,404.24209686)(460.14284766,404.30709679)(460.07285278,404.35710297)
\curveto(459.97284783,404.43709666)(459.87284793,404.4970966)(459.77285278,404.53710297)
\curveto(459.68284812,404.57709652)(459.56784823,404.61209649)(459.42785278,404.64210297)
\curveto(459.35784844,404.66209644)(459.25284855,404.67709642)(459.11285278,404.68710297)
\curveto(458.98284882,404.6970964)(458.88284892,404.69209641)(458.81285278,404.67210297)
\lineto(458.70785278,404.67210297)
\lineto(458.55785278,404.64210297)
\curveto(458.51784928,404.64209646)(458.47284933,404.63709646)(458.42285278,404.62710297)
\curveto(458.25284955,404.57709652)(458.11284969,404.50709659)(458.00285278,404.41710297)
\curveto(457.9028499,404.33709676)(457.83284997,404.21209689)(457.79285278,404.04210297)
\curveto(457.77285003,403.97209713)(457.77285003,403.90709719)(457.79285278,403.84710297)
\curveto(457.81284999,403.78709731)(457.83284997,403.73709736)(457.85285278,403.69710297)
\curveto(457.92284988,403.57709752)(458.0028498,403.48209762)(458.09285278,403.41210297)
\curveto(458.19284961,403.34209776)(458.30784949,403.28209782)(458.43785278,403.23210297)
\curveto(458.62784917,403.15209795)(458.83284897,403.08209802)(459.05285278,403.02210297)
\lineto(459.74285278,402.87210297)
\curveto(459.98284782,402.83209827)(460.21284759,402.78209832)(460.43285278,402.72210297)
\curveto(460.66284714,402.67209843)(460.87784692,402.60709849)(461.07785278,402.52710297)
\curveto(461.16784663,402.48709861)(461.25284655,402.45209865)(461.33285278,402.42210297)
\curveto(461.42284638,402.4020987)(461.50784629,402.36709873)(461.58785278,402.31710297)
\curveto(461.77784602,402.1970989)(461.94784585,402.06709903)(462.09785278,401.92710297)
\curveto(462.25784554,401.78709931)(462.38284542,401.61209949)(462.47285278,401.40210297)
\curveto(462.5028453,401.33209977)(462.52784527,401.26209984)(462.54785278,401.19210297)
\curveto(462.56784523,401.12209998)(462.58784521,401.04710005)(462.60785278,400.96710297)
\curveto(462.61784518,400.90710019)(462.62284518,400.81210029)(462.62285278,400.68210297)
\curveto(462.63284517,400.56210054)(462.63284517,400.46710063)(462.62285278,400.39710297)
\lineto(462.62285278,400.32210297)
\curveto(462.6028452,400.26210084)(462.58784521,400.2021009)(462.57785278,400.14210297)
\curveto(462.57784522,400.09210101)(462.57284523,400.04210106)(462.56285278,399.99210297)
\curveto(462.49284531,399.69210141)(462.38284542,399.42710167)(462.23285278,399.19710297)
\curveto(462.07284573,398.95710214)(461.87784592,398.76210234)(461.64785278,398.61210297)
\curveto(461.41784638,398.46210264)(461.15784664,398.33210277)(460.86785278,398.22210297)
\curveto(460.75784704,398.17210293)(460.63784716,398.13710296)(460.50785278,398.11710297)
\curveto(460.38784741,398.097103)(460.26784753,398.07210303)(460.14785278,398.04210297)
\curveto(460.05784774,398.02210308)(459.96284784,398.01210309)(459.86285278,398.01210297)
\curveto(459.77284803,398.0021031)(459.68284812,397.98710311)(459.59285278,397.96710297)
\lineto(459.32285278,397.96710297)
\curveto(459.26284854,397.94710315)(459.15784864,397.93710316)(459.00785278,397.93710297)
\curveto(458.86784893,397.93710316)(458.76784903,397.94710315)(458.70785278,397.96710297)
\curveto(458.67784912,397.96710313)(458.64284916,397.97210313)(458.60285278,397.98210297)
\lineto(458.49785278,397.98210297)
\curveto(458.37784942,398.0021031)(458.25784954,398.01710308)(458.13785278,398.02710297)
\curveto(458.01784978,398.03710306)(457.9028499,398.05710304)(457.79285278,398.08710297)
\curveto(457.4028504,398.1971029)(457.05785074,398.32210278)(456.75785278,398.46210297)
\curveto(456.45785134,398.61210249)(456.2028516,398.83210227)(455.99285278,399.12210297)
\curveto(455.85285195,399.31210179)(455.73285207,399.53210157)(455.63285278,399.78210297)
\curveto(455.61285219,399.84210126)(455.59285221,399.92210118)(455.57285278,400.02210297)
\curveto(455.55285225,400.07210103)(455.53785226,400.14210096)(455.52785278,400.23210297)
\curveto(455.51785228,400.32210078)(455.52285228,400.3971007)(455.54285278,400.45710297)
\curveto(455.57285223,400.52710057)(455.62285218,400.57710052)(455.69285278,400.60710297)
\curveto(455.74285206,400.62710047)(455.802852,400.63710046)(455.87285278,400.63710297)
\lineto(456.09785278,400.63710297)
\lineto(456.80285278,400.63710297)
\lineto(457.04285278,400.63710297)
\curveto(457.12285068,400.63710046)(457.19285061,400.62710047)(457.25285278,400.60710297)
\curveto(457.36285044,400.56710053)(457.43285037,400.5021006)(457.46285278,400.41210297)
\curveto(457.5028503,400.32210078)(457.54785025,400.22710087)(457.59785278,400.12710297)
\curveto(457.61785018,400.07710102)(457.65285015,400.01210109)(457.70285278,399.93210297)
\curveto(457.76285004,399.85210125)(457.81284999,399.8021013)(457.85285278,399.78210297)
\curveto(457.97284983,399.68210142)(458.08784971,399.6021015)(458.19785278,399.54210297)
\curveto(458.30784949,399.49210161)(458.44784935,399.44210166)(458.61785278,399.39210297)
\curveto(458.66784913,399.37210173)(458.71784908,399.36210174)(458.76785278,399.36210297)
\curveto(458.81784898,399.37210173)(458.86784893,399.37210173)(458.91785278,399.36210297)
\curveto(458.9978488,399.34210176)(459.08284872,399.33210177)(459.17285278,399.33210297)
\curveto(459.27284853,399.34210176)(459.35784844,399.35710174)(459.42785278,399.37710297)
\curveto(459.47784832,399.38710171)(459.52284828,399.39210171)(459.56285278,399.39210297)
\curveto(459.61284819,399.39210171)(459.66284814,399.4021017)(459.71285278,399.42210297)
\curveto(459.85284795,399.47210163)(459.97784782,399.53210157)(460.08785278,399.60210297)
\curveto(460.20784759,399.67210143)(460.3028475,399.76210134)(460.37285278,399.87210297)
\curveto(460.42284738,399.95210115)(460.46284734,400.07710102)(460.49285278,400.24710297)
\curveto(460.51284729,400.31710078)(460.51284729,400.38210072)(460.49285278,400.44210297)
\curveto(460.47284733,400.5021006)(460.45284735,400.55210055)(460.43285278,400.59210297)
\curveto(460.36284744,400.73210037)(460.27284753,400.83710026)(460.16285278,400.90710297)
\curveto(460.06284774,400.97710012)(459.94284786,401.04210006)(459.80285278,401.10210297)
\curveto(459.61284819,401.18209992)(459.41284839,401.24709985)(459.20285278,401.29710297)
\curveto(458.99284881,401.34709975)(458.78284902,401.4020997)(458.57285278,401.46210297)
\curveto(458.49284931,401.48209962)(458.40784939,401.4970996)(458.31785278,401.50710297)
\curveto(458.23784956,401.51709958)(458.15784964,401.53209957)(458.07785278,401.55210297)
\curveto(457.75785004,401.64209946)(457.45285035,401.72709937)(457.16285278,401.80710297)
\curveto(456.87285093,401.8970992)(456.60785119,402.02709907)(456.36785278,402.19710297)
\curveto(456.08785171,402.3970987)(455.88285192,402.66709843)(455.75285278,403.00710297)
\curveto(455.73285207,403.07709802)(455.71285209,403.17209793)(455.69285278,403.29210297)
\curveto(455.67285213,403.36209774)(455.65785214,403.44709765)(455.64785278,403.54710297)
\curveto(455.63785216,403.64709745)(455.64285216,403.73709736)(455.66285278,403.81710297)
\curveto(455.68285212,403.86709723)(455.68785211,403.90709719)(455.67785278,403.93710297)
\curveto(455.66785213,403.97709712)(455.67285213,404.02209708)(455.69285278,404.07210297)
\curveto(455.71285209,404.18209692)(455.73285207,404.28209682)(455.75285278,404.37210297)
\curveto(455.78285202,404.47209663)(455.81785198,404.56709653)(455.85785278,404.65710297)
\curveto(455.98785181,404.94709615)(456.16785163,405.18209592)(456.39785278,405.36210297)
\curveto(456.62785117,405.54209556)(456.88785091,405.68709541)(457.17785278,405.79710297)
\curveto(457.28785051,405.84709525)(457.4028504,405.88209522)(457.52285278,405.90210297)
\curveto(457.64285016,405.93209517)(457.76785003,405.96209514)(457.89785278,405.99210297)
\curveto(457.95784984,406.01209509)(458.01784978,406.02209508)(458.07785278,406.02210297)
\lineto(458.25785278,406.05210297)
\curveto(458.33784946,406.06209504)(458.42284938,406.06709503)(458.51285278,406.06710297)
\curveto(458.6028492,406.06709503)(458.68784911,406.07209503)(458.76785278,406.08210297)
}
}
{
\newrgbcolor{curcolor}{0 0 0}
\pscustom[linestyle=none,fillstyle=solid,fillcolor=curcolor]
{
\newpath
\moveto(471.62449341,402.31710297)
\curveto(471.64448484,402.25709884)(471.65448483,402.17209893)(471.65449341,402.06210297)
\curveto(471.65448483,401.95209915)(471.64448484,401.86709923)(471.62449341,401.80710297)
\lineto(471.62449341,401.65710297)
\curveto(471.60448488,401.57709952)(471.59448489,401.4970996)(471.59449341,401.41710297)
\curveto(471.60448488,401.33709976)(471.59948488,401.25709984)(471.57949341,401.17710297)
\curveto(471.55948492,401.10709999)(471.54448494,401.04210006)(471.53449341,400.98210297)
\curveto(471.52448496,400.92210018)(471.51448497,400.85710024)(471.50449341,400.78710297)
\curveto(471.46448502,400.67710042)(471.42948505,400.56210054)(471.39949341,400.44210297)
\curveto(471.36948511,400.33210077)(471.32948515,400.22710087)(471.27949341,400.12710297)
\curveto(471.06948541,399.64710145)(470.79448569,399.25710184)(470.45449341,398.95710297)
\curveto(470.11448637,398.65710244)(469.70448678,398.40710269)(469.22449341,398.20710297)
\curveto(469.10448738,398.15710294)(468.9794875,398.12210298)(468.84949341,398.10210297)
\curveto(468.72948775,398.07210303)(468.60448788,398.04210306)(468.47449341,398.01210297)
\curveto(468.42448806,397.99210311)(468.36948811,397.98210312)(468.30949341,397.98210297)
\curveto(468.24948823,397.98210312)(468.19448829,397.97710312)(468.14449341,397.96710297)
\lineto(468.03949341,397.96710297)
\curveto(468.00948847,397.95710314)(467.9794885,397.95210315)(467.94949341,397.95210297)
\curveto(467.89948858,397.94210316)(467.81948866,397.93710316)(467.70949341,397.93710297)
\curveto(467.59948888,397.92710317)(467.51448897,397.93210317)(467.45449341,397.95210297)
\lineto(467.30449341,397.95210297)
\curveto(467.25448923,397.96210314)(467.19948928,397.96710313)(467.13949341,397.96710297)
\curveto(467.08948939,397.95710314)(467.03948944,397.96210314)(466.98949341,397.98210297)
\curveto(466.94948953,397.99210311)(466.90948957,397.9971031)(466.86949341,397.99710297)
\curveto(466.83948964,397.9971031)(466.79948968,398.0021031)(466.74949341,398.01210297)
\curveto(466.64948983,398.04210306)(466.54948993,398.06710303)(466.44949341,398.08710297)
\curveto(466.34949013,398.10710299)(466.25449023,398.13710296)(466.16449341,398.17710297)
\curveto(466.04449044,398.21710288)(465.92949055,398.25710284)(465.81949341,398.29710297)
\curveto(465.71949076,398.33710276)(465.61449087,398.38710271)(465.50449341,398.44710297)
\curveto(465.15449133,398.65710244)(464.85449163,398.9021022)(464.60449341,399.18210297)
\curveto(464.35449213,399.46210164)(464.14449234,399.7971013)(463.97449341,400.18710297)
\curveto(463.92449256,400.27710082)(463.8844926,400.37210073)(463.85449341,400.47210297)
\curveto(463.83449265,400.57210053)(463.80949267,400.67710042)(463.77949341,400.78710297)
\curveto(463.75949272,400.83710026)(463.74949273,400.88210022)(463.74949341,400.92210297)
\curveto(463.74949273,400.96210014)(463.73949274,401.00710009)(463.71949341,401.05710297)
\curveto(463.69949278,401.13709996)(463.68949279,401.21709988)(463.68949341,401.29710297)
\curveto(463.68949279,401.38709971)(463.6794928,401.47209963)(463.65949341,401.55210297)
\curveto(463.64949283,401.6020995)(463.64449284,401.64709945)(463.64449341,401.68710297)
\lineto(463.64449341,401.82210297)
\curveto(463.62449286,401.88209922)(463.61449287,401.96709913)(463.61449341,402.07710297)
\curveto(463.62449286,402.18709891)(463.63949284,402.27209883)(463.65949341,402.33210297)
\lineto(463.65949341,402.43710297)
\curveto(463.66949281,402.48709861)(463.66949281,402.53709856)(463.65949341,402.58710297)
\curveto(463.65949282,402.64709845)(463.66949281,402.7020984)(463.68949341,402.75210297)
\curveto(463.69949278,402.8020983)(463.70449278,402.84709825)(463.70449341,402.88710297)
\curveto(463.70449278,402.93709816)(463.71449277,402.98709811)(463.73449341,403.03710297)
\curveto(463.77449271,403.16709793)(463.80949267,403.29209781)(463.83949341,403.41210297)
\curveto(463.86949261,403.54209756)(463.90949257,403.66709743)(463.95949341,403.78710297)
\curveto(464.13949234,404.1970969)(464.35449213,404.53709656)(464.60449341,404.80710297)
\curveto(464.85449163,405.08709601)(465.15949132,405.34209576)(465.51949341,405.57210297)
\curveto(465.61949086,405.62209548)(465.72449076,405.66709543)(465.83449341,405.70710297)
\curveto(465.94449054,405.74709535)(466.05449043,405.79209531)(466.16449341,405.84210297)
\curveto(466.29449019,405.89209521)(466.42949005,405.92709517)(466.56949341,405.94710297)
\curveto(466.70948977,405.96709513)(466.85448963,405.9970951)(467.00449341,406.03710297)
\curveto(467.0844894,406.04709505)(467.15948932,406.05209505)(467.22949341,406.05210297)
\curveto(467.29948918,406.05209505)(467.36948911,406.05709504)(467.43949341,406.06710297)
\curveto(468.01948846,406.07709502)(468.51948796,406.01709508)(468.93949341,405.88710297)
\curveto(469.36948711,405.75709534)(469.74948673,405.57709552)(470.07949341,405.34710297)
\curveto(470.18948629,405.26709583)(470.29948618,405.17709592)(470.40949341,405.07710297)
\curveto(470.52948595,404.98709611)(470.62948585,404.88709621)(470.70949341,404.77710297)
\curveto(470.78948569,404.67709642)(470.85948562,404.57709652)(470.91949341,404.47710297)
\curveto(470.98948549,404.37709672)(471.05948542,404.27209683)(471.12949341,404.16210297)
\curveto(471.19948528,404.05209705)(471.25448523,403.93209717)(471.29449341,403.80210297)
\curveto(471.33448515,403.68209742)(471.3794851,403.55209755)(471.42949341,403.41210297)
\curveto(471.45948502,403.33209777)(471.484485,403.24709785)(471.50449341,403.15710297)
\lineto(471.56449341,402.88710297)
\curveto(471.57448491,402.84709825)(471.5794849,402.80709829)(471.57949341,402.76710297)
\curveto(471.5794849,402.72709837)(471.5844849,402.68709841)(471.59449341,402.64710297)
\curveto(471.61448487,402.5970985)(471.61948486,402.54209856)(471.60949341,402.48210297)
\curveto(471.59948488,402.42209868)(471.60448488,402.36709873)(471.62449341,402.31710297)
\moveto(469.52449341,401.77710297)
\curveto(469.53448695,401.82709927)(469.53948694,401.8970992)(469.53949341,401.98710297)
\curveto(469.53948694,402.08709901)(469.53448695,402.16209894)(469.52449341,402.21210297)
\lineto(469.52449341,402.33210297)
\curveto(469.50448698,402.38209872)(469.49448699,402.43709866)(469.49449341,402.49710297)
\curveto(469.49448699,402.55709854)(469.48948699,402.61209849)(469.47949341,402.66210297)
\curveto(469.479487,402.7020984)(469.47448701,402.73209837)(469.46449341,402.75210297)
\lineto(469.40449341,402.99210297)
\curveto(469.39448709,403.08209802)(469.37448711,403.16709793)(469.34449341,403.24710297)
\curveto(469.23448725,403.50709759)(469.10448738,403.72709737)(468.95449341,403.90710297)
\curveto(468.80448768,404.097097)(468.60448788,404.24709685)(468.35449341,404.35710297)
\curveto(468.29448819,404.37709672)(468.23448825,404.39209671)(468.17449341,404.40210297)
\curveto(468.11448837,404.42209668)(468.04948843,404.44209666)(467.97949341,404.46210297)
\curveto(467.89948858,404.48209662)(467.81448867,404.48709661)(467.72449341,404.47710297)
\lineto(467.45449341,404.47710297)
\curveto(467.42448906,404.45709664)(467.38948909,404.44709665)(467.34949341,404.44710297)
\curveto(467.30948917,404.45709664)(467.27448921,404.45709664)(467.24449341,404.44710297)
\lineto(467.03449341,404.38710297)
\curveto(466.97448951,404.37709672)(466.91948956,404.35709674)(466.86949341,404.32710297)
\curveto(466.61948986,404.21709688)(466.41449007,404.05709704)(466.25449341,403.84710297)
\curveto(466.10449038,403.64709745)(465.9844905,403.41209769)(465.89449341,403.14210297)
\curveto(465.86449062,403.04209806)(465.83949064,402.93709816)(465.81949341,402.82710297)
\curveto(465.80949067,402.71709838)(465.79449069,402.60709849)(465.77449341,402.49710297)
\curveto(465.76449072,402.44709865)(465.75949072,402.3970987)(465.75949341,402.34710297)
\lineto(465.75949341,402.19710297)
\curveto(465.73949074,402.12709897)(465.72949075,402.02209908)(465.72949341,401.88210297)
\curveto(465.73949074,401.74209936)(465.75449073,401.63709946)(465.77449341,401.56710297)
\lineto(465.77449341,401.43210297)
\curveto(465.79449069,401.35209975)(465.80949067,401.27209983)(465.81949341,401.19210297)
\curveto(465.82949065,401.12209998)(465.84449064,401.04710005)(465.86449341,400.96710297)
\curveto(465.96449052,400.66710043)(466.06949041,400.42210068)(466.17949341,400.23210297)
\curveto(466.29949018,400.05210105)(466.48449,399.88710121)(466.73449341,399.73710297)
\curveto(466.80448968,399.68710141)(466.8794896,399.64710145)(466.95949341,399.61710297)
\curveto(467.04948943,399.58710151)(467.13948934,399.56210154)(467.22949341,399.54210297)
\curveto(467.26948921,399.53210157)(467.30448918,399.52710157)(467.33449341,399.52710297)
\curveto(467.36448912,399.53710156)(467.39948908,399.53710156)(467.43949341,399.52710297)
\lineto(467.55949341,399.49710297)
\curveto(467.60948887,399.4971016)(467.65448883,399.5021016)(467.69449341,399.51210297)
\lineto(467.81449341,399.51210297)
\curveto(467.89448859,399.53210157)(467.97448851,399.54710155)(468.05449341,399.55710297)
\curveto(468.13448835,399.56710153)(468.20948827,399.58710151)(468.27949341,399.61710297)
\curveto(468.53948794,399.71710138)(468.74948773,399.85210125)(468.90949341,400.02210297)
\curveto(469.06948741,400.19210091)(469.20448728,400.4021007)(469.31449341,400.65210297)
\curveto(469.35448713,400.75210035)(469.3844871,400.85210025)(469.40449341,400.95210297)
\curveto(469.42448706,401.05210005)(469.44948703,401.15709994)(469.47949341,401.26710297)
\curveto(469.48948699,401.30709979)(469.49448699,401.34209976)(469.49449341,401.37210297)
\curveto(469.49448699,401.41209969)(469.49948698,401.45209965)(469.50949341,401.49210297)
\lineto(469.50949341,401.62710297)
\curveto(469.50948697,401.67709942)(469.51448697,401.72709937)(469.52449341,401.77710297)
}
}
{
\newrgbcolor{curcolor}{0 0 0}
\pscustom[linestyle=none,fillstyle=solid,fillcolor=curcolor]
{
\newpath
\moveto(12.08081787,196.61524658)
\curveto(12.08082857,196.75524277)(12.08082857,196.9202426)(12.08081787,197.11024658)
\curveto(12.07082858,197.31024221)(12.10582854,197.43024209)(12.18581787,197.47024658)
\curveto(12.27582837,197.53024199)(12.41082824,197.54024198)(12.59081787,197.50024658)
\curveto(12.77082788,197.46024206)(12.94082771,197.4252421)(13.10081787,197.39524658)
\lineto(15.35081787,196.94524658)
\lineto(18.14081787,196.39024658)
\curveto(18.49082216,196.3202432)(18.83582181,196.25524327)(19.17581787,196.19524658)
\curveto(19.50582114,196.13524339)(19.80082085,196.1252434)(20.06081787,196.16524658)
\curveto(20.4508202,196.21524331)(20.77081988,196.33524319)(21.02081787,196.52524658)
\curveto(21.27081938,196.71524281)(21.47581917,196.98024254)(21.63581787,197.32024658)
\curveto(21.68581896,197.4202421)(21.72081893,197.53024199)(21.74081787,197.65024658)
\curveto(21.7508189,197.77024175)(21.77081888,197.88524164)(21.80081787,197.99524658)
\lineto(21.83081787,198.17524658)
\curveto(21.83081882,198.24524128)(21.83581881,198.30524122)(21.84581787,198.35524658)
\lineto(21.84581787,198.50524658)
\curveto(21.8458188,198.56524096)(21.8508188,198.6252409)(21.86081787,198.68524658)
\curveto(21.86081879,198.75524077)(21.8508188,198.8202407)(21.83081787,198.88024658)
\curveto(21.82081883,198.93024059)(21.82081883,198.98024054)(21.83081787,199.03024658)
\curveto(21.83081882,199.09024043)(21.82581882,199.14524038)(21.81581787,199.19524658)
\curveto(21.78581886,199.31524021)(21.76081889,199.43024009)(21.74081787,199.54024658)
\curveto(21.72081893,199.66023986)(21.68581896,199.78023974)(21.63581787,199.90024658)
\curveto(21.48581916,200.31023921)(21.29081936,200.64523888)(21.05081787,200.90524658)
\curveto(20.81081984,201.17523835)(20.49082016,201.41523811)(20.09081787,201.62524658)
\curveto(19.84082081,201.75523777)(19.56082109,201.85523767)(19.25081787,201.92524658)
\curveto(18.93082172,202.00523752)(18.60582204,202.08023744)(18.27581787,202.15024658)
\lineto(15.81581787,202.64524658)
\lineto(13.22081787,203.15524658)
\curveto(13.0508276,203.19523633)(12.86082779,203.23023629)(12.65081787,203.26024658)
\curveto(12.43082822,203.30023622)(12.27582837,203.37023615)(12.18581787,203.47024658)
\curveto(12.11582853,203.54023598)(12.08082857,203.64023588)(12.08081787,203.77024658)
\curveto(12.08082857,203.91023561)(12.08082857,204.04523548)(12.08081787,204.17524658)
\curveto(12.08082857,204.2252353)(12.08582856,204.27023525)(12.09581787,204.31024658)
\curveto(12.09582855,204.35023517)(12.09582855,204.39523513)(12.09581787,204.44524658)
\curveto(12.12582852,204.58523494)(12.17582847,204.66523486)(12.24581787,204.68524658)
\curveto(12.32582832,204.7252348)(12.44082821,204.7252348)(12.59081787,204.68524658)
\curveto(12.74082791,204.65523487)(12.87582777,204.6252349)(12.99581787,204.59524658)
\lineto(15.06581787,204.19024658)
\lineto(18.18581787,203.56024658)
\curveto(18.55582209,203.49023603)(18.92082173,203.41023611)(19.28081787,203.32024658)
\curveto(19.64082101,203.24023628)(19.96582068,203.13523639)(20.25581787,203.00524658)
\curveto(20.7458199,202.76523676)(21.16581948,202.49523703)(21.51581787,202.19524658)
\curveto(21.85581879,201.90523762)(22.1458185,201.54523798)(22.38581787,201.11524658)
\curveto(22.47581817,200.94523858)(22.55581809,200.77023875)(22.62581787,200.59024658)
\curveto(22.69581795,200.41023911)(22.76081789,200.21523931)(22.82081787,200.00524658)
\curveto(22.8508178,199.91523961)(22.87081778,199.8202397)(22.88081787,199.72024658)
\curveto(22.90081775,199.63023989)(22.92081773,199.53523999)(22.94081787,199.43524658)
\curveto(22.96081769,199.3252402)(22.97081768,199.21524031)(22.97081787,199.10524658)
\curveto(22.97081768,199.00524052)(22.98081767,198.90524062)(23.00081787,198.80524658)
\lineto(23.00081787,198.62524658)
\curveto(23.01081764,198.57524095)(23.01581763,198.49524103)(23.01581787,198.38524658)
\curveto(23.01581763,198.27524125)(23.01081764,198.19524133)(23.00081787,198.14524658)
\lineto(23.00081787,197.96524658)
\curveto(22.98081767,197.89524163)(22.97081768,197.8202417)(22.97081787,197.74024658)
\curveto(22.98081767,197.67024185)(22.97581767,197.60524192)(22.95581787,197.54524658)
\lineto(22.95581787,197.42524658)
\curveto(22.93581771,197.34524218)(22.92081773,197.26524226)(22.91081787,197.18524658)
\curveto(22.90081775,197.11524241)(22.88581776,197.04524248)(22.86581787,196.97524658)
\curveto(22.81581783,196.81524271)(22.77081788,196.65524287)(22.73081787,196.49524658)
\curveto(22.68081797,196.34524318)(22.62081803,196.20024332)(22.55081787,196.06024658)
\curveto(22.52081813,196.00024352)(22.48581816,195.94024358)(22.44581787,195.88024658)
\curveto(22.40581824,195.8202437)(22.36081829,195.76024376)(22.31081787,195.70024658)
\curveto(22.12081853,195.44024408)(21.89581875,195.23524429)(21.63581787,195.08524658)
\curveto(21.37581927,194.93524459)(21.07081958,194.8252447)(20.72081787,194.75524658)
\curveto(20.57082008,194.7252448)(20.41582023,194.71024481)(20.25581787,194.71024658)
\curveto(20.09582055,194.7202448)(19.93082072,194.7202448)(19.76081787,194.71024658)
\curveto(19.68082097,194.7202448)(19.60582104,194.73024479)(19.53581787,194.74024658)
\curveto(19.45582119,194.75024477)(19.37582127,194.75524477)(19.29581787,194.75524658)
\lineto(19.14581787,194.78524658)
\curveto(19.06582158,194.78524474)(18.98582166,194.79524473)(18.90581787,194.81524658)
\curveto(18.81582183,194.84524468)(18.73082192,194.87024465)(18.65081787,194.89024658)
\lineto(17.67581787,195.08524658)
\lineto(13.76081787,195.86524658)
\lineto(12.74081787,196.07524658)
\curveto(12.650828,196.09524343)(12.56082809,196.11024341)(12.47081787,196.12024658)
\curveto(12.38082827,196.14024338)(12.30582834,196.17524335)(12.24581787,196.22524658)
\curveto(12.17582847,196.28524324)(12.12582852,196.37024315)(12.09581787,196.48024658)
\curveto(12.09582855,196.54024298)(12.09082856,196.58524294)(12.08081787,196.61524658)
}
}
{
\newrgbcolor{curcolor}{0 0 0}
\pscustom[linestyle=none,fillstyle=solid,fillcolor=curcolor]
{
\newpath
\moveto(14.88581787,209.96009033)
\curveto(14.86582578,210.60008351)(14.9508257,211.09008302)(15.14081787,211.43009033)
\curveto(15.33082532,211.77008234)(15.61582503,212.0150821)(15.99581787,212.16509033)
\curveto(16.09582455,212.20508191)(16.20582444,212.23008188)(16.32581787,212.24009033)
\curveto(16.43582421,212.26008185)(16.5508241,212.27008184)(16.67081787,212.27009033)
\curveto(16.86082379,212.29008182)(17.06582358,212.28008183)(17.28581787,212.24009033)
\curveto(17.50582314,212.2100819)(17.73082292,212.17008194)(17.96081787,212.12009033)
\lineto(19.56581787,211.80509033)
\lineto(21.90581787,211.34009033)
\lineto(22.41581787,211.22009033)
\curveto(22.58581806,211.18008293)(22.69581795,211.09008302)(22.74581787,210.95009033)
\curveto(22.76581788,210.90008321)(22.77581787,210.84508327)(22.77581787,210.78509033)
\curveto(22.78581786,210.73508338)(22.79081786,210.68008343)(22.79081787,210.62009033)
\curveto(22.79081786,210.49008362)(22.78581786,210.36508375)(22.77581787,210.24509033)
\curveto(22.77581787,210.12508399)(22.73581791,210.05008406)(22.65581787,210.02009033)
\curveto(22.58581806,209.98008413)(22.49581815,209.97008414)(22.38581787,209.99009033)
\curveto(22.27581837,210.0100841)(22.16581848,210.03508408)(22.05581787,210.06509033)
\lineto(20.76581787,210.32009033)
\lineto(18.32081787,210.80009033)
\curveto(18.0508226,210.86008325)(17.78582286,210.9100832)(17.52581787,210.95009033)
\curveto(17.25582339,210.99008312)(17.02582362,210.99008312)(16.83581787,210.95009033)
\curveto(16.63582401,210.9100832)(16.47582417,210.82008329)(16.35581787,210.68009033)
\curveto(16.22582442,210.55008356)(16.12582452,210.39008372)(16.05581787,210.20009033)
\curveto(16.03582461,210.14008397)(16.02582462,210.07508404)(16.02581787,210.00509033)
\curveto(16.01582463,209.94508417)(16.00082465,209.89008422)(15.98081787,209.84009033)
\curveto(15.97082468,209.79008432)(15.97082468,209.7100844)(15.98081787,209.60009033)
\curveto(15.98082467,209.50008461)(15.98582466,209.42508469)(15.99581787,209.37509033)
\curveto(16.01582463,209.33508478)(16.02582462,209.30008481)(16.02581787,209.27009033)
\curveto(16.01582463,209.24008487)(16.01582463,209.20508491)(16.02581787,209.16509033)
\curveto(16.05582459,209.02508509)(16.09082456,208.89508522)(16.13081787,208.77509033)
\curveto(16.16082449,208.65508546)(16.20582444,208.54008557)(16.26581787,208.43009033)
\curveto(16.28582436,208.38008573)(16.30582434,208.34008577)(16.32581787,208.31009033)
\curveto(16.3458243,208.28008583)(16.36582428,208.24008587)(16.38581787,208.19009033)
\curveto(16.63582401,207.79008632)(17.01082364,207.46008665)(17.51081787,207.20009033)
\curveto(17.59082306,207.16008695)(17.67582297,207.12508699)(17.76581787,207.09509033)
\lineto(18.00581787,207.00509033)
\curveto(18.05582259,206.97508714)(18.10582254,206.96008715)(18.15581787,206.96009033)
\curveto(18.19582245,206.96008715)(18.23582241,206.94508717)(18.27581787,206.91509033)
\lineto(18.59081787,206.85509033)
\curveto(18.62082203,206.83508728)(18.65582199,206.82508729)(18.69581787,206.82509033)
\curveto(18.73582191,206.82508729)(18.78082187,206.82008729)(18.83081787,206.81009033)
\lineto(19.28081787,206.72009033)
\lineto(20.72081787,206.42009033)
\lineto(22.04081787,206.16509033)
\curveto(22.1508185,206.14508797)(22.26581838,206.12008799)(22.38581787,206.09009033)
\curveto(22.49581815,206.07008804)(22.58581806,206.03008808)(22.65581787,205.97009033)
\curveto(22.73581791,205.90008821)(22.77581787,205.80008831)(22.77581787,205.67009033)
\curveto(22.78581786,205.55008856)(22.79081786,205.42508869)(22.79081787,205.29509033)
\curveto(22.79081786,205.2150889)(22.78581786,205.14008897)(22.77581787,205.07009033)
\curveto(22.76581788,205.00008911)(22.74081791,204.94508917)(22.70081787,204.90509033)
\curveto(22.650818,204.83508928)(22.55581809,204.8150893)(22.41581787,204.84509033)
\curveto(22.27581837,204.87508924)(22.14081851,204.90008921)(22.01081787,204.92009033)
\lineto(20.24081787,205.28009033)
\lineto(16.61081787,206.00009033)
\lineto(15.69581787,206.18009033)
\lineto(15.42581787,206.24009033)
\curveto(15.33582531,206.26008785)(15.26582538,206.29508782)(15.21581787,206.34509033)
\curveto(15.15582549,206.38508773)(15.11582553,206.44008767)(15.09581787,206.51009033)
\curveto(15.08582556,206.56008755)(15.07582557,206.62008749)(15.06581787,206.69009033)
\curveto(15.05582559,206.77008734)(15.0508256,206.85008726)(15.05081787,206.93009033)
\curveto(15.0508256,207.0100871)(15.05582559,207.08508703)(15.06581787,207.15509033)
\curveto(15.07582557,207.23508688)(15.09082556,207.28508683)(15.11081787,207.30509033)
\curveto(15.18082547,207.40508671)(15.27082538,207.44008667)(15.38081787,207.41009033)
\curveto(15.48082517,207.38008673)(15.59582505,207.37008674)(15.72581787,207.38009033)
\curveto(15.78582486,207.39008672)(15.83582481,207.42008669)(15.87581787,207.47009033)
\curveto(15.88582476,207.59008652)(15.84082481,207.69508642)(15.74081787,207.78509033)
\curveto(15.64082501,207.88508623)(15.56082509,207.98008613)(15.50081787,208.07009033)
\curveto(15.40082525,208.23008588)(15.31082534,208.39008572)(15.23081787,208.55009033)
\curveto(15.14082551,208.7100854)(15.06582558,208.89508522)(15.00581787,209.10509033)
\curveto(14.97582567,209.18508493)(14.95582569,209.27508484)(14.94581787,209.37509033)
\curveto(14.93582571,209.47508464)(14.92082573,209.57008454)(14.90081787,209.66009033)
\curveto(14.89082576,209.7100844)(14.88582576,209.76008435)(14.88581787,209.81009033)
\lineto(14.88581787,209.96009033)
}
}
{
\newrgbcolor{curcolor}{0 0 0}
\pscustom[linestyle=none,fillstyle=solid,fillcolor=curcolor]
{
\newpath
\moveto(13.53581787,215.17469971)
\curveto(13.47582717,215.10469673)(13.37082728,215.08469675)(13.22081787,215.11469971)
\curveto(13.06082759,215.14469669)(12.90582774,215.17469666)(12.75581787,215.20469971)
\curveto(12.67582797,215.21469662)(12.59082806,215.22969661)(12.50081787,215.24969971)
\curveto(12.41082824,215.26969657)(12.33582831,215.29969654)(12.27581787,215.33969971)
\curveto(12.19582845,215.39969644)(12.13582851,215.48969635)(12.09581787,215.60969971)
\curveto(12.08582856,215.6396962)(12.08582856,215.66469617)(12.09581787,215.68469971)
\curveto(12.09582855,215.70469613)(12.09082856,215.72969611)(12.08081787,215.75969971)
\curveto(12.08082857,215.92969591)(12.08582856,216.08469575)(12.09581787,216.22469971)
\curveto(12.10582854,216.37469546)(12.16582848,216.46469537)(12.27581787,216.49469971)
\curveto(12.33582831,216.51469532)(12.41082824,216.51469532)(12.50081787,216.49469971)
\curveto(12.58082807,216.47469536)(12.66582798,216.45969538)(12.75581787,216.44969971)
\curveto(12.93582771,216.40969543)(13.10582754,216.36969547)(13.26581787,216.32969971)
\curveto(13.42582722,216.29969554)(13.53082712,216.21469562)(13.58081787,216.07469971)
\curveto(13.60082705,216.01469582)(13.61082704,215.95469588)(13.61081787,215.89469971)
\lineto(13.61081787,215.72969971)
\lineto(13.61081787,215.41469971)
\curveto(13.61082704,215.31469652)(13.58582706,215.2346966)(13.53581787,215.17469971)
\moveto(22.04081787,214.58969971)
\curveto(22.14081851,214.56969727)(22.2458184,214.54969729)(22.35581787,214.52969971)
\curveto(22.45581819,214.51969732)(22.53581811,214.47969736)(22.59581787,214.40969971)
\curveto(22.65581799,214.36969747)(22.69581795,214.31969752)(22.71581787,214.25969971)
\curveto(22.72581792,214.19969764)(22.74081791,214.12469771)(22.76081787,214.03469971)
\lineto(22.76081787,213.80969971)
\curveto(22.76081789,213.67969816)(22.75581789,213.56969827)(22.74581787,213.47969971)
\curveto(22.72581792,213.38969845)(22.67581797,213.32469851)(22.59581787,213.28469971)
\curveto(22.53581811,213.26469857)(22.46081819,213.25969858)(22.37081787,213.26969971)
\curveto(22.27081838,213.28969855)(22.17581847,213.30969853)(22.08581787,213.32969971)
\lineto(15.74081787,214.60469971)
\curveto(15.63082502,214.62469721)(15.52582512,214.64469719)(15.42581787,214.66469971)
\curveto(15.31582533,214.68469715)(15.23082542,214.72469711)(15.17081787,214.78469971)
\curveto(15.12082553,214.82469701)(15.09082556,214.86969697)(15.08081787,214.91969971)
\curveto(15.07082558,214.97969686)(15.05582559,215.0396968)(15.03581787,215.09969971)
\curveto(15.03582561,215.11969672)(15.04082561,215.1396967)(15.05081787,215.15969971)
\curveto(15.0508256,215.18969665)(15.0458256,215.21469662)(15.03581787,215.23469971)
\curveto(15.03582561,215.36469647)(15.04082561,215.49469634)(15.05081787,215.62469971)
\curveto(15.0508256,215.76469607)(15.09082556,215.84969599)(15.17081787,215.87969971)
\curveto(15.23082542,215.91969592)(15.31082534,215.92969591)(15.41081787,215.90969971)
\curveto(15.50082515,215.88969595)(15.59582505,215.86969597)(15.69581787,215.84969971)
\lineto(22.04081787,214.58969971)
}
}
{
\newrgbcolor{curcolor}{0 0 0}
\pscustom[linestyle=none,fillstyle=solid,fillcolor=curcolor]
{
\newpath
\moveto(21.95081787,223.50954346)
\lineto(22.34081787,223.41954346)
\curveto(22.46081819,223.39953553)(22.56081809,223.35953557)(22.64081787,223.29954346)
\curveto(22.71081794,223.2295357)(22.7508179,223.13453579)(22.76081787,223.01454346)
\lineto(22.76081787,222.66954346)
\curveto(22.76081789,222.60953632)(22.76581788,222.54953638)(22.77581787,222.48954346)
\curveto(22.77581787,222.43953649)(22.76581788,222.39453653)(22.74581787,222.35454346)
\curveto(22.72581792,222.27453665)(22.68581796,222.2245367)(22.62581787,222.20454346)
\curveto(22.57581807,222.17453675)(22.51581813,222.16453676)(22.44581787,222.17454346)
\curveto(22.37581827,222.18453674)(22.30581834,222.17953675)(22.23581787,222.15954346)
\curveto(22.21581843,222.15953677)(22.20081845,222.14953678)(22.19081787,222.12954346)
\lineto(22.13081787,222.09954346)
\curveto(22.12081853,221.99953693)(22.14081851,221.91453701)(22.19081787,221.84454346)
\curveto(22.24081841,221.78453714)(22.29081836,221.71953721)(22.34081787,221.64954346)
\curveto(22.49081816,221.41953751)(22.60581804,221.19453773)(22.68581787,220.97454346)
\curveto(22.76581788,220.78453814)(22.82581782,220.56453836)(22.86581787,220.31454346)
\curveto(22.90581774,220.07453885)(22.92581772,219.8295391)(22.92581787,219.57954346)
\curveto(22.93581771,219.33953959)(22.92081773,219.09953983)(22.88081787,218.85954346)
\curveto(22.8508178,218.6295403)(22.79581785,218.43454049)(22.71581787,218.27454346)
\curveto(22.49581815,217.79454113)(22.20081845,217.4295415)(21.83081787,217.17954346)
\curveto(21.4508192,216.93954199)(20.98081967,216.78454214)(20.42081787,216.71454346)
\curveto(20.33082032,216.69454223)(20.24082041,216.68454224)(20.15081787,216.68454346)
\curveto(20.0508206,216.69454223)(19.9508207,216.69454223)(19.85081787,216.68454346)
\curveto(19.80082085,216.68454224)(19.7508209,216.68954224)(19.70081787,216.69954346)
\curveto(19.650821,216.70954222)(19.60082105,216.71454221)(19.55081787,216.71454346)
\curveto(19.50082115,216.70454222)(19.4508212,216.70454222)(19.40081787,216.71454346)
\curveto(19.34082131,216.73454219)(19.28582136,216.74454218)(19.23581787,216.74454346)
\lineto(19.08581787,216.77454346)
\curveto(19.03582161,216.76454216)(18.97082168,216.76454216)(18.89081787,216.77454346)
\curveto(18.81082184,216.79454213)(18.7458219,216.81954211)(18.69581787,216.84954346)
\lineto(18.53081787,216.89454346)
\curveto(18.46082219,216.924542)(18.39082226,216.94454198)(18.32081787,216.95454346)
\curveto(18.24082241,216.96454196)(18.16582248,216.98454194)(18.09581787,217.01454346)
\curveto(18.0458226,217.03454189)(18.00082265,217.04954188)(17.96081787,217.05954346)
\curveto(17.92082273,217.06954186)(17.87582277,217.08454184)(17.82581787,217.10454346)
\curveto(17.72582292,217.15454177)(17.63082302,217.19954173)(17.54081787,217.23954346)
\curveto(17.44082321,217.27954165)(17.3458233,217.3245416)(17.25581787,217.37454346)
\curveto(16.87582377,217.57454135)(16.53582411,217.80454112)(16.23581787,218.06454346)
\curveto(15.92582472,218.33454059)(15.67082498,218.63454029)(15.47081787,218.96454346)
\curveto(15.3508253,219.16453976)(15.2508254,219.36453956)(15.17081787,219.56454346)
\curveto(15.09082556,219.76453916)(15.02082563,219.97953895)(14.96081787,220.20954346)
\lineto(14.93081787,220.41954346)
\curveto(14.92082573,220.48953844)(14.90582574,220.55953837)(14.88581787,220.62954346)
\lineto(14.88581787,220.77954346)
\curveto(14.86582578,220.86953806)(14.85582579,220.98953794)(14.85581787,221.13954346)
\curveto(14.85582579,221.29953763)(14.86582578,221.41453751)(14.88581787,221.48454346)
\curveto(14.89582575,221.5245374)(14.90082575,221.57953735)(14.90081787,221.64954346)
\curveto(14.93082572,221.74953718)(14.95582569,221.85453707)(14.97581787,221.96454346)
\curveto(14.98582566,222.07453685)(15.01582563,222.17453675)(15.06581787,222.26454346)
\curveto(15.12582552,222.40453652)(15.19082546,222.53453639)(15.26081787,222.65454346)
\curveto(15.33082532,222.77453615)(15.41082524,222.88453604)(15.50081787,222.98454346)
\curveto(15.5508251,223.03453589)(15.60582504,223.08453584)(15.66581787,223.13454346)
\curveto(15.71582493,223.19453573)(15.73082492,223.27953565)(15.71081787,223.38954346)
\lineto(15.63581787,223.46454346)
\curveto(15.61582503,223.48453544)(15.58582506,223.49953543)(15.54581787,223.50954346)
\curveto(15.45582519,223.55953537)(15.34082531,223.59453533)(15.20081787,223.61454346)
\curveto(15.06082559,223.64453528)(14.93582571,223.66953526)(14.82581787,223.68954346)
\lineto(13.10081787,224.03454346)
\curveto(12.96082769,224.06453486)(12.80582784,224.09453483)(12.63581787,224.12454346)
\curveto(12.45582819,224.16453476)(12.32582832,224.21453471)(12.24581787,224.27454346)
\curveto(12.17582847,224.33453459)(12.13082852,224.40453452)(12.11081787,224.48454346)
\curveto(12.11082854,224.50453442)(12.11082854,224.5295344)(12.11081787,224.55954346)
\curveto(12.10082855,224.58953434)(12.09582855,224.61453431)(12.09581787,224.63454346)
\curveto(12.08582856,224.78453414)(12.08582856,224.93453399)(12.09581787,225.08454346)
\curveto(12.09582855,225.23453369)(12.13582851,225.33453359)(12.21581787,225.38454346)
\curveto(12.29582835,225.41453351)(12.39582825,225.41453351)(12.51581787,225.38454346)
\curveto(12.63582801,225.36453356)(12.76082789,225.34453358)(12.89081787,225.32454346)
\lineto(21.95081787,223.50954346)
\moveto(19.11581787,222.86454346)
\curveto(19.06582158,222.89453603)(19.00082165,222.91453601)(18.92081787,222.92454346)
\curveto(18.83082182,222.94453598)(18.76082189,222.94953598)(18.71081787,222.93954346)
\lineto(18.48581787,222.98454346)
\curveto(18.39582225,222.98453594)(18.30582234,222.98953594)(18.21581787,222.99954346)
\curveto(18.11582253,223.00953592)(18.02582262,223.00453592)(17.94581787,222.98454346)
\lineto(17.72081787,222.98454346)
\curveto(17.650823,222.98453594)(17.58082307,222.97453595)(17.51081787,222.95454346)
\curveto(17.21082344,222.89453603)(16.9458237,222.78953614)(16.71581787,222.63954346)
\curveto(16.48582416,222.49953643)(16.30582434,222.29953663)(16.17581787,222.03954346)
\curveto(16.12582452,221.94953698)(16.09082456,221.85453707)(16.07081787,221.75454346)
\curveto(16.04082461,221.65453727)(16.01582463,221.54453738)(15.99581787,221.42454346)
\curveto(15.97582467,221.35453757)(15.96582468,221.26953766)(15.96581787,221.16954346)
\lineto(15.96581787,220.89954346)
\lineto(15.99581787,220.74954346)
\lineto(15.99581787,220.61454346)
\curveto(16.01582463,220.53453839)(16.03582461,220.44953848)(16.05581787,220.35954346)
\curveto(16.07582457,220.26953866)(16.10082455,220.18453874)(16.13081787,220.10454346)
\curveto(16.27082438,219.75453917)(16.47582417,219.45453947)(16.74581787,219.20454346)
\curveto(17.00582364,218.95453997)(17.31082334,218.73454019)(17.66081787,218.54454346)
\curveto(17.77082288,218.48454044)(17.88582276,218.43454049)(18.00581787,218.39454346)
\lineto(18.33581787,218.27454346)
\lineto(18.45581787,218.24454346)
\curveto(18.48582216,218.23454069)(18.52082213,218.2245407)(18.56081787,218.21454346)
\curveto(18.61082204,218.18454074)(18.66582198,218.16454076)(18.72581787,218.15454346)
\curveto(18.78582186,218.15454077)(18.84082181,218.14954078)(18.89081787,218.13954346)
\curveto(19.00082165,218.11954081)(19.11082154,218.09454083)(19.22081787,218.06454346)
\curveto(19.32082133,218.04454088)(19.41582123,218.03954089)(19.50581787,218.04954346)
\curveto(19.53582111,218.04954088)(19.58582106,218.04454088)(19.65581787,218.03454346)
\lineto(19.86581787,218.03454346)
\curveto(19.93582071,218.03454089)(20.00582064,218.03954089)(20.07581787,218.04954346)
\curveto(20.42582022,218.08954084)(20.72581992,218.17954075)(20.97581787,218.31954346)
\curveto(21.22581942,218.45954047)(21.43081922,218.65954027)(21.59081787,218.91954346)
\curveto(21.64081901,218.99953993)(21.68081897,219.07953985)(21.71081787,219.15954346)
\curveto(21.74081891,219.24953968)(21.77081888,219.34453958)(21.80081787,219.44454346)
\curveto(21.82081883,219.49453943)(21.82581882,219.54453938)(21.81581787,219.59454346)
\curveto(21.80581884,219.65453927)(21.81081884,219.70953922)(21.83081787,219.75954346)
\curveto(21.84081881,219.78953914)(21.8458188,219.8245391)(21.84581787,219.86454346)
\lineto(21.84581787,219.99954346)
\lineto(21.84581787,220.13454346)
\curveto(21.83581881,220.17453875)(21.83081882,220.2295387)(21.83081787,220.29954346)
\curveto(21.81081884,220.37953855)(21.79581885,220.45953847)(21.78581787,220.53954346)
\curveto(21.76581888,220.6295383)(21.74081891,220.70953822)(21.71081787,220.77954346)
\curveto(21.57081908,221.13953779)(21.39581925,221.44453748)(21.18581787,221.69454346)
\curveto(20.96581968,221.94453698)(20.69081996,222.16953676)(20.36081787,222.36954346)
\curveto(20.2508204,222.43953649)(20.14082051,222.49453643)(20.03081787,222.53454346)
\lineto(19.70081787,222.68454346)
\curveto(19.66082099,222.71453621)(19.62582102,222.7295362)(19.59581787,222.72954346)
\curveto(19.55582109,222.73953619)(19.51582113,222.75453617)(19.47581787,222.77454346)
\curveto(19.41582123,222.79453613)(19.35582129,222.80953612)(19.29581787,222.81954346)
\curveto(19.23582141,222.8295361)(19.17582147,222.84453608)(19.11581787,222.86454346)
}
}
{
\newrgbcolor{curcolor}{0 0 0}
\pscustom[linestyle=none,fillstyle=solid,fillcolor=curcolor]
{
\newpath
\moveto(22.20581787,232.29579346)
\curveto(22.36581828,232.28578555)(22.50081815,232.24078559)(22.61081787,232.16079346)
\curveto(22.71081794,232.08078575)(22.78581786,231.98578585)(22.83581787,231.87579346)
\curveto(22.85581779,231.82578601)(22.86581778,231.77078606)(22.86581787,231.71079346)
\curveto(22.86581778,231.66078617)(22.87581777,231.60078623)(22.89581787,231.53079346)
\curveto(22.9458177,231.30078653)(22.93081772,231.08578675)(22.85081787,230.88579346)
\curveto(22.78081787,230.68578715)(22.69081796,230.56078727)(22.58081787,230.51079346)
\curveto(22.51081814,230.47078736)(22.43081822,230.44078739)(22.34081787,230.42079346)
\curveto(22.24081841,230.40078743)(22.16081849,230.36578747)(22.10081787,230.31579346)
\lineto(22.04081787,230.25579346)
\curveto(22.02081863,230.2357876)(22.01581863,230.20578763)(22.02581787,230.16579346)
\curveto(22.05581859,230.04578779)(22.11081854,229.9307879)(22.19081787,229.82079346)
\curveto(22.27081838,229.71078812)(22.34081831,229.60578823)(22.40081787,229.50579346)
\curveto(22.48081817,229.35578848)(22.55581809,229.20078863)(22.62581787,229.04079346)
\curveto(22.68581796,228.88078895)(22.74081791,228.71078912)(22.79081787,228.53079346)
\curveto(22.82081783,228.42078941)(22.84081781,228.30578953)(22.85081787,228.18579346)
\curveto(22.86081779,228.07578976)(22.87581777,227.96078987)(22.89581787,227.84079346)
\curveto(22.90581774,227.79079004)(22.91081774,227.74579009)(22.91081787,227.70579346)
\lineto(22.91081787,227.60079346)
\curveto(22.93081772,227.49079034)(22.93081772,227.38579045)(22.91081787,227.28579346)
\lineto(22.91081787,227.15079346)
\curveto(22.90081775,227.10079073)(22.89581775,227.05079078)(22.89581787,227.00079346)
\curveto(22.89581775,226.95079088)(22.88581776,226.91079092)(22.86581787,226.88079346)
\curveto(22.85581779,226.84079099)(22.8508178,226.80579103)(22.85081787,226.77579346)
\curveto(22.86081779,226.75579108)(22.86081779,226.7307911)(22.85081787,226.70079346)
\lineto(22.79081787,226.46079346)
\curveto(22.78081787,226.39079144)(22.76081789,226.32579151)(22.73081787,226.26579346)
\curveto(22.60081805,225.98579185)(22.45581819,225.77079206)(22.29581787,225.62079346)
\curveto(22.12581852,225.47079236)(21.89081876,225.36579247)(21.59081787,225.30579346)
\curveto(21.37081928,225.25579258)(21.10581954,225.26079257)(20.79581787,225.32079346)
\lineto(20.48081787,225.39579346)
\curveto(20.43082022,225.41579242)(20.38082027,225.4307924)(20.33081787,225.44079346)
\lineto(20.15081787,225.50079346)
\lineto(19.82081787,225.68079346)
\curveto(19.71082094,225.75079208)(19.61082104,225.82079201)(19.52081787,225.89079346)
\curveto(19.23082142,226.1307917)(19.01582163,226.42079141)(18.87581787,226.76079346)
\curveto(18.73582191,227.10079073)(18.61082204,227.46579037)(18.50081787,227.85579346)
\curveto(18.46082219,228.00578983)(18.43082222,228.15578968)(18.41081787,228.30579346)
\curveto(18.39082226,228.46578937)(18.36582228,228.62078921)(18.33581787,228.77079346)
\curveto(18.31582233,228.85078898)(18.30582234,228.92078891)(18.30581787,228.98079346)
\curveto(18.30582234,229.05078878)(18.29582235,229.12578871)(18.27581787,229.20579346)
\curveto(18.25582239,229.27578856)(18.2458224,229.34578849)(18.24581787,229.41579346)
\curveto(18.23582241,229.49578834)(18.22082243,229.57578826)(18.20081787,229.65579346)
\curveto(18.14082251,229.91578792)(18.09082256,230.16078767)(18.05081787,230.39079346)
\curveto(18.00082265,230.62078721)(17.88582276,230.82078701)(17.70581787,230.99079346)
\curveto(17.62582302,231.06078677)(17.52582312,231.12578671)(17.40581787,231.18579346)
\curveto(17.27582337,231.25578658)(17.13582351,231.28578655)(16.98581787,231.27579346)
\curveto(16.7458239,231.26578657)(16.55582409,231.21578662)(16.41581787,231.12579346)
\curveto(16.27582437,231.04578679)(16.16582448,230.90578693)(16.08581787,230.70579346)
\curveto(16.03582461,230.59578724)(16.00082465,230.46078737)(15.98081787,230.30079346)
\curveto(15.96082469,230.14078769)(15.9508247,229.97078786)(15.95081787,229.79079346)
\curveto(15.9508247,229.61078822)(15.96082469,229.4307884)(15.98081787,229.25079346)
\curveto(16.00082465,229.08078875)(16.03082462,228.9307889)(16.07081787,228.80079346)
\curveto(16.13082452,228.62078921)(16.21582443,228.44078939)(16.32581787,228.26079346)
\curveto(16.38582426,228.17078966)(16.46582418,228.08078975)(16.56581787,227.99079346)
\curveto(16.65582399,227.91078992)(16.75582389,227.83579)(16.86581787,227.76579346)
\curveto(16.9458237,227.71579012)(17.03082362,227.67079016)(17.12081787,227.63079346)
\curveto(17.21082344,227.59079024)(17.28082337,227.5307903)(17.33081787,227.45079346)
\curveto(17.36082329,227.40079043)(17.38582326,227.32579051)(17.40581787,227.22579346)
\curveto(17.41582323,227.12579071)(17.42082323,227.02579081)(17.42081787,226.92579346)
\curveto(17.42082323,226.82579101)(17.41582323,226.7307911)(17.40581787,226.64079346)
\curveto(17.38582326,226.55079128)(17.36082329,226.49079134)(17.33081787,226.46079346)
\curveto(17.30082335,226.42079141)(17.2508234,226.39579144)(17.18081787,226.38579346)
\curveto(17.11082354,226.38579145)(17.03582361,226.40579143)(16.95581787,226.44579346)
\curveto(16.82582382,226.49579134)(16.70582394,226.55079128)(16.59581787,226.61079346)
\curveto(16.47582417,226.67079116)(16.36082429,226.7357911)(16.25081787,226.80579346)
\curveto(15.90082475,227.06579077)(15.63082502,227.36079047)(15.44081787,227.69079346)
\curveto(15.24082541,228.02078981)(15.08082557,228.41078942)(14.96081787,228.86079346)
\curveto(14.94082571,228.97078886)(14.92582572,229.07578876)(14.91581787,229.17579346)
\curveto(14.90582574,229.28578855)(14.89082576,229.39578844)(14.87081787,229.50579346)
\curveto(14.86082579,229.55578828)(14.86082579,229.62078821)(14.87081787,229.70079346)
\curveto(14.87082578,229.79078804)(14.86082579,229.85078798)(14.84081787,229.88079346)
\curveto(14.83082582,230.58078725)(14.91082574,231.17078666)(15.08081787,231.65079346)
\curveto(15.2508254,232.14078569)(15.57582507,232.44578539)(16.05581787,232.56579346)
\curveto(16.25582439,232.61578522)(16.49082416,232.62078521)(16.76081787,232.58079346)
\curveto(17.02082363,232.54078529)(17.29582335,232.49078534)(17.58581787,232.43079346)
\lineto(20.90081787,231.77079346)
\curveto(21.04081961,231.74078609)(21.17581947,231.71578612)(21.30581787,231.69579346)
\curveto(21.43581921,231.68578615)(21.54081911,231.69578614)(21.62081787,231.72579346)
\curveto(21.69081896,231.76578607)(21.74081891,231.82078601)(21.77081787,231.89079346)
\curveto(21.81081884,231.98078585)(21.84081881,232.06078577)(21.86081787,232.13079346)
\curveto(21.87081878,232.21078562)(21.91581873,232.26078557)(21.99581787,232.28079346)
\curveto(22.02581862,232.30078553)(22.05581859,232.30578553)(22.08581787,232.29579346)
\lineto(22.20581787,232.29579346)
\moveto(20.54081787,230.48079346)
\curveto(20.40082025,230.57078726)(20.24082041,230.6357872)(20.06081787,230.67579346)
\curveto(19.87082078,230.71578712)(19.67582097,230.75578708)(19.47581787,230.79579346)
\curveto(19.36582128,230.81578702)(19.26582138,230.830787)(19.17581787,230.84079346)
\curveto(19.08582156,230.85078698)(19.01582163,230.82578701)(18.96581787,230.76579346)
\curveto(18.9458217,230.7357871)(18.93582171,230.66578717)(18.93581787,230.55579346)
\curveto(18.95582169,230.5357873)(18.96582168,230.50078733)(18.96581787,230.45079346)
\curveto(18.96582168,230.40078743)(18.97582167,230.35078748)(18.99581787,230.30079346)
\curveto(19.01582163,230.22078761)(19.03582161,230.12578771)(19.05581787,230.01579346)
\lineto(19.11581787,229.71579346)
\curveto(19.11582153,229.68578815)(19.12082153,229.65078818)(19.13081787,229.61079346)
\lineto(19.13081787,229.50579346)
\curveto(19.17082148,229.34578849)(19.19582145,229.17578866)(19.20581787,228.99579346)
\curveto(19.20582144,228.82578901)(19.22582142,228.66078917)(19.26581787,228.50079346)
\curveto(19.28582136,228.41078942)(19.30582134,228.3307895)(19.32581787,228.26079346)
\curveto(19.33582131,228.20078963)(19.3508213,228.12578971)(19.37081787,228.03579346)
\curveto(19.42082123,227.86578997)(19.48582116,227.70079013)(19.56581787,227.54079346)
\curveto(19.63582101,227.39079044)(19.72582092,227.25579058)(19.83581787,227.13579346)
\curveto(19.9458207,227.01579082)(20.08082057,226.91579092)(20.24081787,226.83579346)
\curveto(20.39082026,226.75579108)(20.57582007,226.69579114)(20.79581787,226.65579346)
\curveto(20.89581975,226.6357912)(20.99081966,226.6357912)(21.08081787,226.65579346)
\curveto(21.16081949,226.67579116)(21.23581941,226.70579113)(21.30581787,226.74579346)
\curveto(21.41581923,226.79579104)(21.51081914,226.87579096)(21.59081787,226.98579346)
\curveto(21.66081899,227.10579073)(21.72081893,227.2357906)(21.77081787,227.37579346)
\curveto(21.78081887,227.42579041)(21.78581886,227.47579036)(21.78581787,227.52579346)
\curveto(21.78581886,227.57579026)(21.79081886,227.62579021)(21.80081787,227.67579346)
\curveto(21.82081883,227.74579009)(21.83581881,227.83079)(21.84581787,227.93079346)
\curveto(21.8458188,228.0307898)(21.83581881,228.12078971)(21.81581787,228.20079346)
\curveto(21.79581885,228.26078957)(21.79081886,228.32078951)(21.80081787,228.38079346)
\curveto(21.80081885,228.44078939)(21.79081886,228.50078933)(21.77081787,228.56079346)
\curveto(21.7508189,228.65078918)(21.73581891,228.7307891)(21.72581787,228.80079346)
\curveto(21.71581893,228.88078895)(21.69581895,228.96078887)(21.66581787,229.04079346)
\curveto(21.5458191,229.35078848)(21.40081925,229.62578821)(21.23081787,229.86579346)
\curveto(21.06081959,230.10578773)(20.83081982,230.31078752)(20.54081787,230.48079346)
}
}
{
\newrgbcolor{curcolor}{0 0 0}
\pscustom[linestyle=none,fillstyle=solid,fillcolor=curcolor]
{
\newpath
\moveto(21.95081787,240.47243408)
\lineto(22.34081787,240.38243408)
\curveto(22.46081819,240.36242615)(22.56081809,240.32242619)(22.64081787,240.26243408)
\curveto(22.71081794,240.19242632)(22.7508179,240.09742642)(22.76081787,239.97743408)
\lineto(22.76081787,239.63243408)
\curveto(22.76081789,239.57242694)(22.76581788,239.512427)(22.77581787,239.45243408)
\curveto(22.77581787,239.40242711)(22.76581788,239.35742716)(22.74581787,239.31743408)
\curveto(22.72581792,239.23742728)(22.68581796,239.18742733)(22.62581787,239.16743408)
\curveto(22.57581807,239.13742738)(22.51581813,239.12742739)(22.44581787,239.13743408)
\curveto(22.37581827,239.14742737)(22.30581834,239.14242737)(22.23581787,239.12243408)
\curveto(22.21581843,239.12242739)(22.20081845,239.1124274)(22.19081787,239.09243408)
\lineto(22.13081787,239.06243408)
\curveto(22.12081853,238.96242755)(22.14081851,238.87742764)(22.19081787,238.80743408)
\curveto(22.24081841,238.74742777)(22.29081836,238.68242783)(22.34081787,238.61243408)
\curveto(22.49081816,238.38242813)(22.60581804,238.15742836)(22.68581787,237.93743408)
\curveto(22.76581788,237.74742877)(22.82581782,237.52742899)(22.86581787,237.27743408)
\curveto(22.90581774,237.03742948)(22.92581772,236.79242972)(22.92581787,236.54243408)
\curveto(22.93581771,236.30243021)(22.92081773,236.06243045)(22.88081787,235.82243408)
\curveto(22.8508178,235.59243092)(22.79581785,235.39743112)(22.71581787,235.23743408)
\curveto(22.49581815,234.75743176)(22.20081845,234.39243212)(21.83081787,234.14243408)
\curveto(21.4508192,233.90243261)(20.98081967,233.74743277)(20.42081787,233.67743408)
\curveto(20.33082032,233.65743286)(20.24082041,233.64743287)(20.15081787,233.64743408)
\curveto(20.0508206,233.65743286)(19.9508207,233.65743286)(19.85081787,233.64743408)
\curveto(19.80082085,233.64743287)(19.7508209,233.65243286)(19.70081787,233.66243408)
\curveto(19.650821,233.67243284)(19.60082105,233.67743284)(19.55081787,233.67743408)
\curveto(19.50082115,233.66743285)(19.4508212,233.66743285)(19.40081787,233.67743408)
\curveto(19.34082131,233.69743282)(19.28582136,233.70743281)(19.23581787,233.70743408)
\lineto(19.08581787,233.73743408)
\curveto(19.03582161,233.72743279)(18.97082168,233.72743279)(18.89081787,233.73743408)
\curveto(18.81082184,233.75743276)(18.7458219,233.78243273)(18.69581787,233.81243408)
\lineto(18.53081787,233.85743408)
\curveto(18.46082219,233.88743263)(18.39082226,233.90743261)(18.32081787,233.91743408)
\curveto(18.24082241,233.92743259)(18.16582248,233.94743257)(18.09581787,233.97743408)
\curveto(18.0458226,233.99743252)(18.00082265,234.0124325)(17.96081787,234.02243408)
\curveto(17.92082273,234.03243248)(17.87582277,234.04743247)(17.82581787,234.06743408)
\curveto(17.72582292,234.1174324)(17.63082302,234.16243235)(17.54081787,234.20243408)
\curveto(17.44082321,234.24243227)(17.3458233,234.28743223)(17.25581787,234.33743408)
\curveto(16.87582377,234.53743198)(16.53582411,234.76743175)(16.23581787,235.02743408)
\curveto(15.92582472,235.29743122)(15.67082498,235.59743092)(15.47081787,235.92743408)
\curveto(15.3508253,236.12743039)(15.2508254,236.32743019)(15.17081787,236.52743408)
\curveto(15.09082556,236.72742979)(15.02082563,236.94242957)(14.96081787,237.17243408)
\lineto(14.93081787,237.38243408)
\curveto(14.92082573,237.45242906)(14.90582574,237.52242899)(14.88581787,237.59243408)
\lineto(14.88581787,237.74243408)
\curveto(14.86582578,237.83242868)(14.85582579,237.95242856)(14.85581787,238.10243408)
\curveto(14.85582579,238.26242825)(14.86582578,238.37742814)(14.88581787,238.44743408)
\curveto(14.89582575,238.48742803)(14.90082575,238.54242797)(14.90081787,238.61243408)
\curveto(14.93082572,238.7124278)(14.95582569,238.8174277)(14.97581787,238.92743408)
\curveto(14.98582566,239.03742748)(15.01582563,239.13742738)(15.06581787,239.22743408)
\curveto(15.12582552,239.36742715)(15.19082546,239.49742702)(15.26081787,239.61743408)
\curveto(15.33082532,239.73742678)(15.41082524,239.84742667)(15.50081787,239.94743408)
\curveto(15.5508251,239.99742652)(15.60582504,240.04742647)(15.66581787,240.09743408)
\curveto(15.71582493,240.15742636)(15.73082492,240.24242627)(15.71081787,240.35243408)
\lineto(15.63581787,240.42743408)
\curveto(15.61582503,240.44742607)(15.58582506,240.46242605)(15.54581787,240.47243408)
\curveto(15.45582519,240.52242599)(15.34082531,240.55742596)(15.20081787,240.57743408)
\curveto(15.06082559,240.60742591)(14.93582571,240.63242588)(14.82581787,240.65243408)
\lineto(13.10081787,240.99743408)
\curveto(12.96082769,241.02742549)(12.80582784,241.05742546)(12.63581787,241.08743408)
\curveto(12.45582819,241.12742539)(12.32582832,241.17742534)(12.24581787,241.23743408)
\curveto(12.17582847,241.29742522)(12.13082852,241.36742515)(12.11081787,241.44743408)
\curveto(12.11082854,241.46742505)(12.11082854,241.49242502)(12.11081787,241.52243408)
\curveto(12.10082855,241.55242496)(12.09582855,241.57742494)(12.09581787,241.59743408)
\curveto(12.08582856,241.74742477)(12.08582856,241.89742462)(12.09581787,242.04743408)
\curveto(12.09582855,242.19742432)(12.13582851,242.29742422)(12.21581787,242.34743408)
\curveto(12.29582835,242.37742414)(12.39582825,242.37742414)(12.51581787,242.34743408)
\curveto(12.63582801,242.32742419)(12.76082789,242.30742421)(12.89081787,242.28743408)
\lineto(21.95081787,240.47243408)
\moveto(19.11581787,239.82743408)
\curveto(19.06582158,239.85742666)(19.00082165,239.87742664)(18.92081787,239.88743408)
\curveto(18.83082182,239.90742661)(18.76082189,239.9124266)(18.71081787,239.90243408)
\lineto(18.48581787,239.94743408)
\curveto(18.39582225,239.94742657)(18.30582234,239.95242656)(18.21581787,239.96243408)
\curveto(18.11582253,239.97242654)(18.02582262,239.96742655)(17.94581787,239.94743408)
\lineto(17.72081787,239.94743408)
\curveto(17.650823,239.94742657)(17.58082307,239.93742658)(17.51081787,239.91743408)
\curveto(17.21082344,239.85742666)(16.9458237,239.75242676)(16.71581787,239.60243408)
\curveto(16.48582416,239.46242705)(16.30582434,239.26242725)(16.17581787,239.00243408)
\curveto(16.12582452,238.9124276)(16.09082456,238.8174277)(16.07081787,238.71743408)
\curveto(16.04082461,238.6174279)(16.01582463,238.50742801)(15.99581787,238.38743408)
\curveto(15.97582467,238.3174282)(15.96582468,238.23242828)(15.96581787,238.13243408)
\lineto(15.96581787,237.86243408)
\lineto(15.99581787,237.71243408)
\lineto(15.99581787,237.57743408)
\curveto(16.01582463,237.49742902)(16.03582461,237.4124291)(16.05581787,237.32243408)
\curveto(16.07582457,237.23242928)(16.10082455,237.14742937)(16.13081787,237.06743408)
\curveto(16.27082438,236.7174298)(16.47582417,236.4174301)(16.74581787,236.16743408)
\curveto(17.00582364,235.9174306)(17.31082334,235.69743082)(17.66081787,235.50743408)
\curveto(17.77082288,235.44743107)(17.88582276,235.39743112)(18.00581787,235.35743408)
\lineto(18.33581787,235.23743408)
\lineto(18.45581787,235.20743408)
\curveto(18.48582216,235.19743132)(18.52082213,235.18743133)(18.56081787,235.17743408)
\curveto(18.61082204,235.14743137)(18.66582198,235.12743139)(18.72581787,235.11743408)
\curveto(18.78582186,235.1174314)(18.84082181,235.1124314)(18.89081787,235.10243408)
\curveto(19.00082165,235.08243143)(19.11082154,235.05743146)(19.22081787,235.02743408)
\curveto(19.32082133,235.00743151)(19.41582123,235.00243151)(19.50581787,235.01243408)
\curveto(19.53582111,235.0124315)(19.58582106,235.00743151)(19.65581787,234.99743408)
\lineto(19.86581787,234.99743408)
\curveto(19.93582071,234.99743152)(20.00582064,235.00243151)(20.07581787,235.01243408)
\curveto(20.42582022,235.05243146)(20.72581992,235.14243137)(20.97581787,235.28243408)
\curveto(21.22581942,235.42243109)(21.43081922,235.62243089)(21.59081787,235.88243408)
\curveto(21.64081901,235.96243055)(21.68081897,236.04243047)(21.71081787,236.12243408)
\curveto(21.74081891,236.2124303)(21.77081888,236.30743021)(21.80081787,236.40743408)
\curveto(21.82081883,236.45743006)(21.82581882,236.50743001)(21.81581787,236.55743408)
\curveto(21.80581884,236.6174299)(21.81081884,236.67242984)(21.83081787,236.72243408)
\curveto(21.84081881,236.75242976)(21.8458188,236.78742973)(21.84581787,236.82743408)
\lineto(21.84581787,236.96243408)
\lineto(21.84581787,237.09743408)
\curveto(21.83581881,237.13742938)(21.83081882,237.19242932)(21.83081787,237.26243408)
\curveto(21.81081884,237.34242917)(21.79581885,237.42242909)(21.78581787,237.50243408)
\curveto(21.76581888,237.59242892)(21.74081891,237.67242884)(21.71081787,237.74243408)
\curveto(21.57081908,238.10242841)(21.39581925,238.40742811)(21.18581787,238.65743408)
\curveto(20.96581968,238.90742761)(20.69081996,239.13242738)(20.36081787,239.33243408)
\curveto(20.2508204,239.40242711)(20.14082051,239.45742706)(20.03081787,239.49743408)
\lineto(19.70081787,239.64743408)
\curveto(19.66082099,239.67742684)(19.62582102,239.69242682)(19.59581787,239.69243408)
\curveto(19.55582109,239.70242681)(19.51582113,239.7174268)(19.47581787,239.73743408)
\curveto(19.41582123,239.75742676)(19.35582129,239.77242674)(19.29581787,239.78243408)
\curveto(19.23582141,239.79242672)(19.17582147,239.80742671)(19.11581787,239.82743408)
}
}
{
\newrgbcolor{curcolor}{0 0 0}
\pscustom[linestyle=none,fillstyle=solid,fillcolor=curcolor]
{
\newpath
\moveto(18.59081787,249.84368408)
\curveto(18.69082196,249.84367558)(18.80582184,249.8236756)(18.93581787,249.78368408)
\curveto(19.05582159,249.74367568)(19.14082151,249.69367573)(19.19081787,249.63368408)
\curveto(19.23082142,249.57367585)(19.26082139,249.49367593)(19.28081787,249.39368408)
\curveto(19.29082136,249.29367613)(19.29582135,249.18367624)(19.29581787,249.06368408)
\lineto(19.29581787,248.70368408)
\curveto(19.28582136,248.59367683)(19.28082137,248.49367693)(19.28081787,248.40368408)
\lineto(19.28081787,244.56368408)
\curveto(19.28082137,244.48368094)(19.28582136,244.39868102)(19.29581787,244.30868408)
\curveto(19.29582135,244.22868119)(19.31082134,244.16368126)(19.34081787,244.11368408)
\curveto(19.36082129,244.06368136)(19.40082125,244.01368141)(19.46081787,243.96368408)
\lineto(19.59581787,243.87368408)
\curveto(19.645821,243.84368158)(19.69582095,243.83368159)(19.74581787,243.84368408)
\curveto(19.79582085,243.84368158)(19.84082081,243.83868158)(19.88081787,243.82868408)
\lineto(20.00081787,243.82868408)
\lineto(20.25581787,243.82868408)
\curveto(20.33582031,243.83868158)(20.41582023,243.85368157)(20.49581787,243.87368408)
\curveto(21.03581961,244.00368142)(21.42081923,244.30868111)(21.65081787,244.78868408)
\curveto(21.68081897,244.83868058)(21.70581894,244.89868052)(21.72581787,244.96868408)
\curveto(21.7458189,245.03868038)(21.76581888,245.10368032)(21.78581787,245.16368408)
\curveto(21.79581885,245.19368023)(21.80081885,245.24368018)(21.80081787,245.31368408)
\curveto(21.84081881,245.44367998)(21.86081879,245.6236798)(21.86081787,245.85368408)
\curveto(21.86081879,246.08367934)(21.84081881,246.27367915)(21.80081787,246.42368408)
\curveto(21.76081889,246.57367885)(21.72081893,246.70867871)(21.68081787,246.82868408)
\curveto(21.63081902,246.95867846)(21.57081908,247.07867834)(21.50081787,247.18868408)
\curveto(21.43081922,247.30867811)(21.3508193,247.418678)(21.26081787,247.51868408)
\curveto(21.16081949,247.6186778)(21.05581959,247.70867771)(20.94581787,247.78868408)
\curveto(20.8458198,247.86867755)(20.74081991,247.94367748)(20.63081787,248.01368408)
\curveto(20.52082013,248.08367734)(20.44082021,248.17867724)(20.39081787,248.29868408)
\curveto(20.37082028,248.33867708)(20.35582029,248.40367702)(20.34581787,248.49368408)
\curveto(20.33582031,248.59367683)(20.33582031,248.68367674)(20.34581787,248.76368408)
\curveto(20.3458203,248.85367657)(20.3508203,248.93867648)(20.36081787,249.01868408)
\curveto(20.37082028,249.09867632)(20.39082026,249.14867627)(20.42081787,249.16868408)
\curveto(20.49082016,249.25867616)(20.60582004,249.26367616)(20.76581787,249.18368408)
\curveto(21.03581961,249.04367638)(21.27581937,248.88867653)(21.48581787,248.71868408)
\curveto(21.80581884,248.45867696)(22.07081858,248.17867724)(22.28081787,247.87868408)
\curveto(22.48081817,247.58867783)(22.645818,247.23367819)(22.77581787,246.81368408)
\curveto(22.81581783,246.70367872)(22.84081781,246.59867882)(22.85081787,246.49868408)
\curveto(22.87081778,246.39867902)(22.89081776,246.28867913)(22.91081787,246.16868408)
\curveto(22.92081773,246.1186793)(22.92581772,246.06867935)(22.92581787,246.01868408)
\curveto(22.92581772,245.97867944)(22.93081772,245.93367949)(22.94081787,245.88368408)
\lineto(22.94081787,245.73368408)
\curveto(22.9508177,245.68367974)(22.95581769,245.6236798)(22.95581787,245.55368408)
\curveto(22.95581769,245.49367993)(22.9508177,245.44367998)(22.94081787,245.40368408)
\lineto(22.94081787,245.26868408)
\curveto(22.93081772,245.2186802)(22.92581772,245.17368025)(22.92581787,245.13368408)
\curveto(22.92581772,245.09368033)(22.92081773,245.05368037)(22.91081787,245.01368408)
\curveto(22.90081775,244.96368046)(22.89081776,244.90868051)(22.88081787,244.84868408)
\curveto(22.88081777,244.79868062)(22.87581777,244.74868067)(22.86581787,244.69868408)
\curveto(22.8458178,244.60868081)(22.82081783,244.5186809)(22.79081787,244.42868408)
\curveto(22.77081788,244.34868107)(22.7458179,244.27368115)(22.71581787,244.20368408)
\curveto(22.69581795,244.16368126)(22.68581796,244.12868129)(22.68581787,244.09868408)
\curveto(22.67581797,244.06868135)(22.66081799,244.03868138)(22.64081787,244.00868408)
\curveto(22.57081808,243.86868155)(22.48581816,243.7236817)(22.38581787,243.57368408)
\curveto(22.19581845,243.3236821)(21.96581868,243.1236823)(21.69581787,242.97368408)
\curveto(21.41581923,242.8236826)(21.10581954,242.71368271)(20.76581787,242.64368408)
\curveto(20.65581999,242.61368281)(20.54082011,242.59868282)(20.42081787,242.59868408)
\curveto(20.30082035,242.59868282)(20.18082047,242.58868283)(20.06081787,242.56868408)
\lineto(19.95581787,242.56868408)
\curveto(19.92582072,242.57868284)(19.88582076,242.58368284)(19.83581787,242.58368408)
\lineto(19.58081787,242.58368408)
\curveto(19.49082116,242.59368283)(19.40082125,242.59868282)(19.31081787,242.59868408)
\lineto(19.10081787,242.64368408)
\curveto(19.06082159,242.64368278)(19.00582164,242.64868277)(18.93581787,242.65868408)
\curveto(18.85582179,242.66868275)(18.79082186,242.68368274)(18.74081787,242.70368408)
\lineto(18.57581787,242.73368408)
\curveto(18.52582212,242.76368266)(18.47582217,242.77868264)(18.42581787,242.77868408)
\curveto(18.36582228,242.78868263)(18.31082234,242.80368262)(18.26081787,242.82368408)
\curveto(18.10082255,242.89368253)(17.94082271,242.95868246)(17.78081787,243.01868408)
\curveto(17.62082303,243.07868234)(17.47082318,243.15368227)(17.33081787,243.24368408)
\curveto(17.22082343,243.31368211)(17.11082354,243.37868204)(17.00081787,243.43868408)
\curveto(16.88082377,243.50868191)(16.76582388,243.58868183)(16.65581787,243.67868408)
\curveto(16.30582434,243.96868145)(16.00582464,244.27868114)(15.75581787,244.60868408)
\curveto(15.49582515,244.93868048)(15.28082537,245.3236801)(15.11081787,245.76368408)
\curveto(15.06082559,245.89367953)(15.02582562,246.0236794)(15.00581787,246.15368408)
\curveto(14.97582567,246.28367914)(14.9458257,246.423679)(14.91581787,246.57368408)
\curveto(14.90582574,246.6236788)(14.90082575,246.66867875)(14.90081787,246.70868408)
\curveto(14.89082576,246.74867867)(14.88582576,246.79367863)(14.88581787,246.84368408)
\curveto(14.87582577,246.86367856)(14.87582577,246.88867853)(14.88581787,246.91868408)
\curveto(14.89582575,246.94867847)(14.89082576,246.97367845)(14.87081787,246.99368408)
\curveto(14.86082579,247.423678)(14.90582574,247.78367764)(15.00581787,248.07368408)
\curveto(15.09582555,248.36367706)(15.22082543,248.6186768)(15.38081787,248.83868408)
\curveto(15.40082525,248.87867654)(15.43082522,248.90867651)(15.47081787,248.92868408)
\curveto(15.50082515,248.95867646)(15.52582512,248.98867643)(15.54581787,249.01868408)
\curveto(15.60582504,249.08867633)(15.67582497,249.15867626)(15.75581787,249.22868408)
\curveto(15.83582481,249.29867612)(15.91582473,249.35367607)(15.99581787,249.39368408)
\curveto(16.20582444,249.51367591)(16.40582424,249.60867581)(16.59581787,249.67868408)
\curveto(16.70582394,249.72867569)(16.82582382,249.75867566)(16.95581787,249.76868408)
\lineto(17.34581787,249.82868408)
\curveto(17.47582317,249.85867556)(17.61082304,249.86867555)(17.75081787,249.85868408)
\curveto(17.89082276,249.85867556)(18.03082262,249.86367556)(18.17081787,249.87368408)
\curveto(18.24082241,249.87367555)(18.31082234,249.86867555)(18.38081787,249.85868408)
\curveto(18.4508222,249.84867557)(18.52082213,249.84367558)(18.59081787,249.84368408)
\moveto(18.08081787,248.49368408)
\curveto(18.04082261,248.5236769)(17.99082266,248.55367687)(17.93081787,248.58368408)
\curveto(17.86082279,248.6236768)(17.79082286,248.63867678)(17.72081787,248.62868408)
\curveto(17.50082315,248.6186768)(17.29582335,248.57867684)(17.10581787,248.50868408)
\curveto(16.87582377,248.40867701)(16.68082397,248.28867713)(16.52081787,248.14868408)
\curveto(16.36082429,248.0186774)(16.22582442,247.82867759)(16.11581787,247.57868408)
\curveto(16.09582455,247.50867791)(16.08082457,247.43867798)(16.07081787,247.36868408)
\curveto(16.0508246,247.30867811)(16.03082462,247.23867818)(16.01081787,247.15868408)
\curveto(15.99082466,247.08867833)(15.98082467,247.00867841)(15.98081787,246.91868408)
\lineto(15.98081787,246.66368408)
\curveto(16.00082465,246.6236788)(16.01082464,246.58367884)(16.01081787,246.54368408)
\curveto(16.00082465,246.50367892)(16.00082465,246.46867895)(16.01081787,246.43868408)
\lineto(16.07081787,246.19868408)
\curveto(16.08082457,246.1186793)(16.09582455,246.04367938)(16.11581787,245.97368408)
\curveto(16.23582441,245.65367977)(16.38582426,245.38868003)(16.56581787,245.17868408)
\curveto(16.7458239,244.96868045)(16.97082368,244.76868065)(17.24081787,244.57868408)
\curveto(17.29082336,244.53868088)(17.35582329,244.49368093)(17.43581787,244.44368408)
\curveto(17.50582314,244.40368102)(17.58582306,244.36368106)(17.67581787,244.32368408)
\curveto(17.76582288,244.28368114)(17.8508228,244.25868116)(17.93081787,244.24868408)
\curveto(18.01082264,244.24868117)(18.07082258,244.27368115)(18.11081787,244.32368408)
\curveto(18.17082248,244.39368103)(18.20082245,244.5236809)(18.20081787,244.71368408)
\curveto(18.19082246,244.91368051)(18.18582246,245.08368034)(18.18581787,245.22368408)
\lineto(18.18581787,247.50368408)
\curveto(18.18582246,247.65367777)(18.19082246,247.83367759)(18.20081787,248.04368408)
\curveto(18.20082245,248.25367717)(18.16082249,248.40367702)(18.08081787,248.49368408)
}
}
{
\newrgbcolor{curcolor}{0 0 0}
\pscustom[linestyle=none,fillstyle=solid,fillcolor=curcolor]
{
\newpath
\moveto(14.85581787,254.33032471)
\curveto(14.8458258,255.05031905)(14.93082572,255.63531847)(15.11081787,256.08532471)
\curveto(15.28082537,256.54531756)(15.58582506,256.86531724)(16.02581787,257.04532471)
\curveto(16.13582451,257.09531701)(16.2508244,257.12531698)(16.37081787,257.13532471)
\curveto(16.48082417,257.15531695)(16.60582404,257.17031693)(16.74581787,257.18032471)
\curveto(16.81582383,257.19031691)(16.89082376,257.18031692)(16.97081787,257.15032471)
\curveto(17.04082361,257.13031697)(17.09582355,257.105317)(17.13581787,257.07532471)
\curveto(17.15582349,257.05531705)(17.17582347,257.02531708)(17.19581787,256.98532471)
\curveto(17.20582344,256.95531715)(17.22082343,256.93031717)(17.24081787,256.91032471)
\curveto(17.26082339,256.85031725)(17.26582338,256.79531731)(17.25581787,256.74532471)
\curveto(17.2458234,256.7053174)(17.2458234,256.66031744)(17.25581787,256.61032471)
\curveto(17.27582337,256.52031758)(17.28082337,256.41031769)(17.27081787,256.28032471)
\curveto(17.2508234,256.16031794)(17.22582342,256.07531803)(17.19581787,256.02532471)
\curveto(17.1458235,255.95531815)(17.08082357,255.91531819)(17.00081787,255.90532471)
\curveto(16.91082374,255.9053182)(16.82582382,255.88531822)(16.74581787,255.84532471)
\curveto(16.58582406,255.79531831)(16.44082421,255.7003184)(16.31081787,255.56032471)
\curveto(16.23082442,255.47031863)(16.17082448,255.36031874)(16.13081787,255.23032471)
\curveto(16.09082456,255.11031899)(16.0508246,254.98031912)(16.01081787,254.84032471)
\curveto(15.99082466,254.8003193)(15.98582466,254.75031935)(15.99581787,254.69032471)
\curveto(15.99582465,254.64031946)(15.99082466,254.59531951)(15.98081787,254.55532471)
\curveto(15.96082469,254.49531961)(15.9508247,254.42031968)(15.95081787,254.33032471)
\curveto(15.9508247,254.24031986)(15.96082469,254.16531994)(15.98081787,254.10532471)
\lineto(15.98081787,254.01532471)
\curveto(15.99082466,253.95532015)(16.00082465,253.9003202)(16.01081787,253.85032471)
\curveto(16.01082464,253.8003203)(16.01582463,253.75032035)(16.02581787,253.70032471)
\curveto(16.08582456,253.43032067)(16.17082448,253.19532091)(16.28081787,252.99532471)
\curveto(16.39082426,252.8053213)(16.57582407,252.65532145)(16.83581787,252.54532471)
\curveto(16.90582374,252.51532159)(16.97582367,252.5003216)(17.04581787,252.50032471)
\curveto(17.11582353,252.5003216)(17.17582347,252.5053216)(17.22581787,252.51532471)
\curveto(17.37582327,252.54532156)(17.48582316,252.59532151)(17.55581787,252.66532471)
\curveto(17.61582303,252.73532137)(17.68582296,252.83032127)(17.76581787,252.95032471)
\curveto(17.86582278,253.09032101)(17.94082271,253.25532085)(17.99081787,253.44532471)
\curveto(18.03082262,253.63532047)(18.08082257,253.82532028)(18.14081787,254.01532471)
\curveto(18.18082247,254.13531997)(18.21082244,254.25531985)(18.23081787,254.37532471)
\curveto(18.2508224,254.5053196)(18.28082237,254.63031947)(18.32081787,254.75032471)
\curveto(18.38082227,254.95031915)(18.44082221,255.14531896)(18.50081787,255.33532471)
\curveto(18.5508221,255.52531858)(18.61582203,255.71031839)(18.69581787,255.89032471)
\curveto(18.71582193,255.94031816)(18.73582191,255.98531812)(18.75581787,256.02532471)
\curveto(18.77582187,256.07531803)(18.80082185,256.12531798)(18.83081787,256.17532471)
\curveto(18.9508217,256.34531776)(19.08582156,256.49031761)(19.23581787,256.61032471)
\curveto(19.38582126,256.73031737)(19.57582107,256.82031728)(19.80581787,256.88032471)
\lineto(20.09081787,256.88032471)
\curveto(20.16082049,256.88031722)(20.23582041,256.87531723)(20.31581787,256.86532471)
\curveto(20.38582026,256.85531725)(20.46582018,256.84531726)(20.55581787,256.83532471)
\lineto(20.70581787,256.80532471)
\curveto(20.77581987,256.76531734)(20.8458198,256.73531737)(20.91581787,256.71532471)
\curveto(20.98581966,256.7053174)(21.05581959,256.68531742)(21.12581787,256.65532471)
\curveto(21.23581941,256.6053175)(21.34081931,256.55031755)(21.44081787,256.49032471)
\curveto(21.54081911,256.43031767)(21.63081902,256.36531774)(21.71081787,256.29532471)
\curveto(21.97081868,256.08531802)(22.18081847,255.84031826)(22.34081787,255.56032471)
\curveto(22.49081816,255.28031882)(22.62081803,254.97531913)(22.73081787,254.64532471)
\curveto(22.76081789,254.54531956)(22.78081787,254.44531966)(22.79081787,254.34532471)
\curveto(22.81081784,254.24531986)(22.83581781,254.15031995)(22.86581787,254.06032471)
\curveto(22.88581776,253.95032015)(22.89581775,253.84532026)(22.89581787,253.74532471)
\curveto(22.89581775,253.64532046)(22.90581774,253.54532056)(22.92581787,253.44532471)
\lineto(22.92581787,253.29532471)
\curveto(22.93581771,253.24532086)(22.94081771,253.17532093)(22.94081787,253.08532471)
\curveto(22.94081771,252.99532111)(22.93581771,252.92532118)(22.92581787,252.87532471)
\lineto(22.92581787,252.71032471)
\curveto(22.90581774,252.65032145)(22.89581775,252.58532152)(22.89581787,252.51532471)
\curveto(22.90581774,252.44532166)(22.90081775,252.39032171)(22.88081787,252.35032471)
\curveto(22.87081778,252.3003218)(22.86581778,252.23532187)(22.86581787,252.15532471)
\curveto(22.8458178,252.07532203)(22.82581782,252.0003221)(22.80581787,251.93032471)
\curveto(22.79581785,251.86032224)(22.77581787,251.78532232)(22.74581787,251.70532471)
\curveto(22.645818,251.41532269)(22.52081813,251.17032293)(22.37081787,250.97032471)
\curveto(22.22081843,250.77032333)(22.02581862,250.61032349)(21.78581787,250.49032471)
\curveto(21.65581899,250.43032367)(21.52081913,250.38032372)(21.38081787,250.34032471)
\curveto(21.24081941,250.31032379)(21.08581956,250.29032381)(20.91581787,250.28032471)
\curveto(20.85581979,250.27032383)(20.78581986,250.27532383)(20.70581787,250.29532471)
\curveto(20.61582003,250.31532379)(20.5458201,250.34032376)(20.49581787,250.37032471)
\curveto(20.45582019,250.41032369)(20.41582023,250.47032363)(20.37581787,250.55032471)
\curveto(20.35582029,250.6003235)(20.3458203,250.67032343)(20.34581787,250.76032471)
\curveto(20.33582031,250.86032324)(20.33582031,250.95032315)(20.34581787,251.03032471)
\curveto(20.35582029,251.12032298)(20.37082028,251.2053229)(20.39081787,251.28532471)
\curveto(20.40082025,251.37532273)(20.41582023,251.43032267)(20.43581787,251.45032471)
\curveto(20.48582016,251.51032259)(20.56082009,251.54032256)(20.66081787,251.54032471)
\curveto(20.7508199,251.55032255)(20.83581981,251.57032253)(20.91581787,251.60032471)
\curveto(21.13581951,251.65032245)(21.30581934,251.75032235)(21.42581787,251.90032471)
\curveto(21.51581913,252.0003221)(21.58581906,252.12032198)(21.63581787,252.26032471)
\curveto(21.68581896,252.4003217)(21.73581891,252.55032155)(21.78581787,252.71032471)
\lineto(21.83081787,253.02532471)
\lineto(21.83081787,253.11532471)
\curveto(21.8508188,253.17532093)(21.86081879,253.26032084)(21.86081787,253.37032471)
\curveto(21.86081879,253.49032061)(21.8508188,253.59532051)(21.83081787,253.68532471)
\curveto(21.83081882,253.75532035)(21.82581882,253.81032029)(21.81581787,253.85032471)
\curveto(21.80581884,253.91032019)(21.80081885,253.97032013)(21.80081787,254.03032471)
\curveto(21.79081886,254.09032001)(21.78081887,254.14531996)(21.77081787,254.19532471)
\curveto(21.69081896,254.5053196)(21.58581906,254.75531935)(21.45581787,254.94532471)
\curveto(21.32581932,255.14531896)(21.10581954,255.31031879)(20.79581787,255.44032471)
\curveto(20.7458199,255.47031863)(20.69081996,255.48531862)(20.63081787,255.48532471)
\curveto(20.57082008,255.49531861)(20.52582012,255.49531861)(20.49581787,255.48532471)
\curveto(20.30582034,255.47531863)(20.16582048,255.43531867)(20.07581787,255.36532471)
\curveto(19.97582067,255.29531881)(19.88582076,255.2003189)(19.80581787,255.08032471)
\curveto(19.7458209,255.0003191)(19.69582095,254.9053192)(19.65581787,254.79532471)
\lineto(19.53581787,254.49532471)
\curveto(19.52582112,254.46531964)(19.52082113,254.43531967)(19.52081787,254.40532471)
\curveto(19.52082113,254.38531972)(19.51082114,254.36531974)(19.49081787,254.34532471)
\curveto(19.38082127,254.02532008)(19.30082135,253.68532042)(19.25081787,253.32532471)
\curveto(19.19082146,252.97532113)(19.09582155,252.65532145)(18.96581787,252.36532471)
\curveto(18.92582172,252.27532183)(18.89082176,252.18532192)(18.86081787,252.09532471)
\curveto(18.83082182,252.01532209)(18.79082186,251.94032216)(18.74081787,251.87032471)
\curveto(18.63082202,251.7003224)(18.50582214,251.55032255)(18.36581787,251.42032471)
\curveto(18.22582242,251.29032281)(18.0508226,251.2003229)(17.84081787,251.15032471)
\curveto(17.77082288,251.13032297)(17.70082295,251.12032298)(17.63081787,251.12032471)
\lineto(17.40581787,251.12032471)
\curveto(17.28582336,251.11032299)(17.1508235,251.12532298)(17.00081787,251.16532471)
\curveto(16.84082381,251.2053229)(16.70582394,251.24532286)(16.59581787,251.28532471)
\curveto(16.5458241,251.31532279)(16.50582414,251.33532277)(16.47581787,251.34532471)
\curveto(16.43582421,251.36532274)(16.39582425,251.39032271)(16.35581787,251.42032471)
\curveto(16.12582452,251.55032255)(15.92582472,251.71032239)(15.75581787,251.90032471)
\curveto(15.58582506,252.09032201)(15.43582521,252.3003218)(15.30581787,252.53032471)
\curveto(15.21582543,252.69032141)(15.1458255,252.86532124)(15.09581787,253.05532471)
\curveto(15.03582561,253.25532085)(14.98082567,253.46032064)(14.93081787,253.67032471)
\curveto(14.92082573,253.74032036)(14.91082574,253.8053203)(14.90081787,253.86532471)
\curveto(14.89082576,253.93532017)(14.88082577,254.01032009)(14.87081787,254.09032471)
\curveto(14.86082579,254.13031997)(14.86082579,254.17031993)(14.87081787,254.21032471)
\curveto(14.88082577,254.26031984)(14.87582577,254.3003198)(14.85581787,254.33032471)
}
}
{
\newrgbcolor{curcolor}{0 0 0}
\pscustom[linestyle=none,fillstyle=solid,fillcolor=curcolor]
{
\newpath
\moveto(131.34959778,65.43651611)
\curveto(131.52959601,65.43650542)(131.72959581,65.43650542)(131.94959778,65.43651611)
\curveto(132.16959537,65.44650541)(132.3345952,65.41150544)(132.44459778,65.33151611)
\curveto(132.52459501,65.27150558)(132.59959494,65.18150567)(132.66959778,65.06151611)
\curveto(132.7395948,64.9515059)(132.80459473,64.851506)(132.86459778,64.76151611)
\curveto(132.99459454,64.56150629)(133.12459441,64.3565065)(133.25459778,64.14651611)
\curveto(133.39459414,63.94650691)(133.52959401,63.74150711)(133.65959778,63.53151611)
\lineto(133.86959778,63.20151611)
\curveto(133.94959359,63.10150775)(134.02459351,62.99650786)(134.09459778,62.88651611)
\curveto(134.39459314,62.40650845)(134.69959284,61.92650893)(135.00959778,61.44651611)
\curveto(135.31959222,60.97650988)(135.62959191,60.50151035)(135.93959778,60.02151611)
\curveto(136.01959152,59.88151097)(136.10459143,59.74651111)(136.19459778,59.61651611)
\curveto(136.29459124,59.49651136)(136.38459115,59.36651149)(136.46459778,59.22651611)
\lineto(136.97459778,58.41651611)
\curveto(137.15459038,58.1565127)(137.32959021,57.89651296)(137.49959778,57.63651611)
\curveto(137.54958999,57.5565133)(137.60958993,57.4565134)(137.67959778,57.33651611)
\curveto(137.75958978,57.22651363)(137.85458968,57.17151368)(137.96459778,57.17151611)
\curveto(138.01458952,57.19151366)(138.0395895,57.20651365)(138.03959778,57.21651611)
\curveto(138.08958945,57.27651358)(138.11458942,57.36151349)(138.11459778,57.47151611)
\lineto(138.11459778,57.78651611)
\lineto(138.11459778,58.97151611)
\lineto(138.11459778,63.56151611)
\lineto(138.11459778,64.46151611)
\curveto(138.11458942,64.53150632)(138.10958943,64.60650625)(138.09959778,64.68651611)
\curveto(138.08958945,64.76650609)(138.09458944,64.84150601)(138.11459778,64.91151611)
\lineto(138.11459778,65.07651611)
\curveto(138.1345894,65.11650574)(138.14458939,65.1565057)(138.14459778,65.19651611)
\curveto(138.15458938,65.23650562)(138.16958937,65.27150558)(138.18959778,65.30151611)
\curveto(138.24958929,65.38150547)(138.3395892,65.42150543)(138.45959778,65.42151611)
\curveto(138.57958896,65.43150542)(138.70958883,65.43650542)(138.84959778,65.43651611)
\curveto(138.90958863,65.43650542)(138.96958857,65.43650542)(139.02959778,65.43651611)
\curveto(139.09958844,65.43650542)(139.15958838,65.42650543)(139.20959778,65.40651611)
\curveto(139.32958821,65.3565055)(139.39458814,65.26650559)(139.40459778,65.13651611)
\curveto(139.42458811,65.01650584)(139.4345881,64.87150598)(139.43459778,64.70151611)
\lineto(139.43459778,63.05151611)
\lineto(139.43459778,56.76651611)
\lineto(139.43459778,55.50651611)
\lineto(139.43459778,55.17651611)
\curveto(139.44458809,55.06651579)(139.42458811,54.98151587)(139.37459778,54.92151611)
\curveto(139.3345882,54.86151599)(139.28458825,54.82151603)(139.22459778,54.80151611)
\curveto(139.17458836,54.79151606)(139.10958843,54.77651608)(139.02959778,54.75651611)
\lineto(138.63959778,54.75651611)
\lineto(138.26459778,54.75651611)
\curveto(138.14458939,54.7565161)(138.04458949,54.77651608)(137.96459778,54.81651611)
\curveto(137.88458965,54.84651601)(137.81958972,54.89651596)(137.76959778,54.96651611)
\curveto(137.72958981,55.03651582)(137.68458985,55.10651575)(137.63459778,55.17651611)
\curveto(137.55458998,55.29651556)(137.46959007,55.42151543)(137.37959778,55.55151611)
\lineto(137.13959778,55.94151611)
\curveto(136.77959076,56.48151437)(136.42459111,57.01651384)(136.07459778,57.54651611)
\curveto(135.72459181,58.07651278)(135.37459216,58.61651224)(135.02459778,59.16651611)
\curveto(134.8345927,59.46651139)(134.6395929,59.76151109)(134.43959778,60.05151611)
\curveto(134.24959329,60.34151051)(134.05959348,60.63651022)(133.86959778,60.93651611)
\curveto(133.539594,61.4565094)(133.19459434,61.98150887)(132.83459778,62.51151611)
\curveto(132.79459474,62.58150827)(132.75459478,62.64650821)(132.71459778,62.70651611)
\curveto(132.67459486,62.77650808)(132.61959492,62.83650802)(132.54959778,62.88651611)
\curveto(132.52959501,62.89650796)(132.50959503,62.91150794)(132.48959778,62.93151611)
\curveto(132.46959507,62.9515079)(132.44459509,62.9565079)(132.41459778,62.94651611)
\curveto(132.35459518,62.92650793)(132.31959522,62.88650797)(132.30959778,62.82651611)
\curveto(132.29959524,62.76650809)(132.28459525,62.70650815)(132.26459778,62.64651611)
\lineto(132.26459778,62.54151611)
\curveto(132.24459529,62.47150838)(132.2395953,62.39150846)(132.24959778,62.30151611)
\curveto(132.25959528,62.22150863)(132.26459527,62.14150871)(132.26459778,62.06151611)
\lineto(132.26459778,61.07151611)
\lineto(132.26459778,56.30151611)
\lineto(132.26459778,55.59651611)
\lineto(132.26459778,55.41651611)
\curveto(132.27459526,55.34651551)(132.26959527,55.28651557)(132.24959778,55.23651611)
\lineto(132.24959778,55.11651611)
\curveto(132.22959531,55.01651584)(132.20959533,54.94651591)(132.18959778,54.90651611)
\curveto(132.16959537,54.856516)(132.1345954,54.82151603)(132.08459778,54.80151611)
\curveto(132.0345955,54.79151606)(131.97959556,54.77651608)(131.91959778,54.75651611)
\lineto(131.61959778,54.75651611)
\curveto(131.47959606,54.7565161)(131.35459618,54.76151609)(131.24459778,54.77151611)
\curveto(131.1345964,54.78151607)(131.05459648,54.82651603)(131.00459778,54.90651611)
\curveto(130.95459658,54.96651589)(130.92959661,55.04651581)(130.92959778,55.14651611)
\lineto(130.92959778,55.47651611)
\lineto(130.92959778,56.69151611)
\lineto(130.92959778,62.96151611)
\lineto(130.92959778,64.58151611)
\lineto(130.92959778,64.95651611)
\curveto(130.92959661,65.09650576)(130.95459658,65.20650565)(131.00459778,65.28651611)
\curveto(131.0345965,65.33650552)(131.09459644,65.38150547)(131.18459778,65.42151611)
\curveto(131.20459633,65.43150542)(131.22959631,65.43150542)(131.25959778,65.42151611)
\curveto(131.29959624,65.42150543)(131.32959621,65.42650543)(131.34959778,65.43651611)
}
}
{
\newrgbcolor{curcolor}{0 0 0}
\pscustom[linestyle=none,fillstyle=solid,fillcolor=curcolor]
{
\newpath
\moveto(148.61014465,58.95651611)
\curveto(148.63013659,58.89651196)(148.64013658,58.80151205)(148.64014465,58.67151611)
\curveto(148.64013658,58.5515123)(148.63513659,58.46651239)(148.62514465,58.41651611)
\lineto(148.62514465,58.26651611)
\curveto(148.61513661,58.18651267)(148.60513662,58.11151274)(148.59514465,58.04151611)
\curveto(148.59513663,57.98151287)(148.59013663,57.91151294)(148.58014465,57.83151611)
\curveto(148.56013666,57.77151308)(148.54513668,57.71151314)(148.53514465,57.65151611)
\curveto(148.53513669,57.59151326)(148.5251367,57.53151332)(148.50514465,57.47151611)
\curveto(148.46513676,57.34151351)(148.43013679,57.21151364)(148.40014465,57.08151611)
\curveto(148.37013685,56.9515139)(148.33013689,56.83151402)(148.28014465,56.72151611)
\curveto(148.07013715,56.24151461)(147.79013743,55.83651502)(147.44014465,55.50651611)
\curveto(147.09013813,55.18651567)(146.66013856,54.94151591)(146.15014465,54.77151611)
\curveto(146.04013918,54.73151612)(145.9201393,54.70151615)(145.79014465,54.68151611)
\curveto(145.67013955,54.66151619)(145.54513968,54.64151621)(145.41514465,54.62151611)
\curveto(145.35513987,54.61151624)(145.29013993,54.60651625)(145.22014465,54.60651611)
\curveto(145.16014006,54.59651626)(145.10014012,54.59151626)(145.04014465,54.59151611)
\curveto(145.00014022,54.58151627)(144.94014028,54.57651628)(144.86014465,54.57651611)
\curveto(144.79014043,54.57651628)(144.74014048,54.58151627)(144.71014465,54.59151611)
\curveto(144.67014055,54.60151625)(144.63014059,54.60651625)(144.59014465,54.60651611)
\curveto(144.55014067,54.59651626)(144.51514071,54.59651626)(144.48514465,54.60651611)
\lineto(144.39514465,54.60651611)
\lineto(144.03514465,54.65151611)
\curveto(143.89514133,54.69151616)(143.76014146,54.73151612)(143.63014465,54.77151611)
\curveto(143.50014172,54.81151604)(143.37514185,54.856516)(143.25514465,54.90651611)
\curveto(142.80514242,55.10651575)(142.43514279,55.36651549)(142.14514465,55.68651611)
\curveto(141.85514337,56.00651485)(141.61514361,56.39651446)(141.42514465,56.85651611)
\curveto(141.37514385,56.9565139)(141.33514389,57.0565138)(141.30514465,57.15651611)
\curveto(141.28514394,57.2565136)(141.26514396,57.36151349)(141.24514465,57.47151611)
\curveto(141.225144,57.51151334)(141.21514401,57.54151331)(141.21514465,57.56151611)
\curveto(141.225144,57.59151326)(141.225144,57.62651323)(141.21514465,57.66651611)
\curveto(141.19514403,57.74651311)(141.18014404,57.82651303)(141.17014465,57.90651611)
\curveto(141.17014405,57.99651286)(141.16014406,58.08151277)(141.14014465,58.16151611)
\lineto(141.14014465,58.28151611)
\curveto(141.14014408,58.32151253)(141.13514409,58.36651249)(141.12514465,58.41651611)
\curveto(141.11514411,58.46651239)(141.11014411,58.5515123)(141.11014465,58.67151611)
\curveto(141.11014411,58.80151205)(141.1201441,58.89651196)(141.14014465,58.95651611)
\curveto(141.16014406,59.02651183)(141.16514406,59.09651176)(141.15514465,59.16651611)
\curveto(141.14514408,59.23651162)(141.15014407,59.30651155)(141.17014465,59.37651611)
\curveto(141.18014404,59.42651143)(141.18514404,59.46651139)(141.18514465,59.49651611)
\curveto(141.19514403,59.53651132)(141.20514402,59.58151127)(141.21514465,59.63151611)
\curveto(141.24514398,59.7515111)(141.27014395,59.87151098)(141.29014465,59.99151611)
\curveto(141.3201439,60.11151074)(141.36014386,60.22651063)(141.41014465,60.33651611)
\curveto(141.56014366,60.70651015)(141.74014348,61.03650982)(141.95014465,61.32651611)
\curveto(142.17014305,61.62650923)(142.43514279,61.87650898)(142.74514465,62.07651611)
\curveto(142.86514236,62.1565087)(142.99014223,62.22150863)(143.12014465,62.27151611)
\curveto(143.25014197,62.33150852)(143.38514184,62.39150846)(143.52514465,62.45151611)
\curveto(143.64514158,62.50150835)(143.77514145,62.53150832)(143.91514465,62.54151611)
\curveto(144.05514117,62.56150829)(144.19514103,62.59150826)(144.33514465,62.63151611)
\lineto(144.53014465,62.63151611)
\curveto(144.60014062,62.64150821)(144.66514056,62.6515082)(144.72514465,62.66151611)
\curveto(145.61513961,62.67150818)(146.35513887,62.48650837)(146.94514465,62.10651611)
\curveto(147.53513769,61.72650913)(147.96013726,61.23150962)(148.22014465,60.62151611)
\curveto(148.27013695,60.52151033)(148.31013691,60.42151043)(148.34014465,60.32151611)
\curveto(148.37013685,60.22151063)(148.40513682,60.11651074)(148.44514465,60.00651611)
\curveto(148.47513675,59.89651096)(148.50013672,59.77651108)(148.52014465,59.64651611)
\curveto(148.54013668,59.52651133)(148.56513666,59.40151145)(148.59514465,59.27151611)
\curveto(148.60513662,59.22151163)(148.60513662,59.16651169)(148.59514465,59.10651611)
\curveto(148.59513663,59.0565118)(148.60013662,59.00651185)(148.61014465,58.95651611)
\moveto(147.27514465,58.10151611)
\curveto(147.29513793,58.17151268)(147.30013792,58.2515126)(147.29014465,58.34151611)
\lineto(147.29014465,58.59651611)
\curveto(147.29013793,58.98651187)(147.25513797,59.31651154)(147.18514465,59.58651611)
\curveto(147.15513807,59.66651119)(147.13013809,59.74651111)(147.11014465,59.82651611)
\curveto(147.09013813,59.90651095)(147.06513816,59.98151087)(147.03514465,60.05151611)
\curveto(146.75513847,60.70151015)(146.31013891,61.1515097)(145.70014465,61.40151611)
\curveto(145.63013959,61.43150942)(145.55513967,61.4515094)(145.47514465,61.46151611)
\lineto(145.23514465,61.52151611)
\curveto(145.15514007,61.54150931)(145.07014015,61.5515093)(144.98014465,61.55151611)
\lineto(144.71014465,61.55151611)
\lineto(144.44014465,61.50651611)
\curveto(144.34014088,61.48650937)(144.24514098,61.46150939)(144.15514465,61.43151611)
\curveto(144.07514115,61.41150944)(143.99514123,61.38150947)(143.91514465,61.34151611)
\curveto(143.84514138,61.32150953)(143.78014144,61.29150956)(143.72014465,61.25151611)
\curveto(143.66014156,61.21150964)(143.60514162,61.17150968)(143.55514465,61.13151611)
\curveto(143.31514191,60.96150989)(143.1201421,60.7565101)(142.97014465,60.51651611)
\curveto(142.8201424,60.27651058)(142.69014253,59.99651086)(142.58014465,59.67651611)
\curveto(142.55014267,59.57651128)(142.53014269,59.47151138)(142.52014465,59.36151611)
\curveto(142.51014271,59.26151159)(142.49514273,59.1565117)(142.47514465,59.04651611)
\curveto(142.46514276,59.00651185)(142.46014276,58.94151191)(142.46014465,58.85151611)
\curveto(142.45014277,58.82151203)(142.44514278,58.78651207)(142.44514465,58.74651611)
\curveto(142.45514277,58.70651215)(142.46014276,58.66151219)(142.46014465,58.61151611)
\lineto(142.46014465,58.31151611)
\curveto(142.46014276,58.21151264)(142.47014275,58.12151273)(142.49014465,58.04151611)
\lineto(142.52014465,57.86151611)
\curveto(142.54014268,57.76151309)(142.55514267,57.66151319)(142.56514465,57.56151611)
\curveto(142.58514264,57.47151338)(142.61514261,57.38651347)(142.65514465,57.30651611)
\curveto(142.75514247,57.06651379)(142.87014235,56.84151401)(143.00014465,56.63151611)
\curveto(143.14014208,56.42151443)(143.31014191,56.24651461)(143.51014465,56.10651611)
\curveto(143.56014166,56.07651478)(143.60514162,56.0515148)(143.64514465,56.03151611)
\curveto(143.68514154,56.01151484)(143.73014149,55.98651487)(143.78014465,55.95651611)
\curveto(143.86014136,55.90651495)(143.94514128,55.86151499)(144.03514465,55.82151611)
\curveto(144.13514109,55.79151506)(144.24014098,55.76151509)(144.35014465,55.73151611)
\curveto(144.40014082,55.71151514)(144.44514078,55.70151515)(144.48514465,55.70151611)
\curveto(144.53514069,55.71151514)(144.58514064,55.71151514)(144.63514465,55.70151611)
\curveto(144.66514056,55.69151516)(144.7251405,55.68151517)(144.81514465,55.67151611)
\curveto(144.91514031,55.66151519)(144.99014023,55.66651519)(145.04014465,55.68651611)
\curveto(145.08014014,55.69651516)(145.1201401,55.69651516)(145.16014465,55.68651611)
\curveto(145.20014002,55.68651517)(145.24013998,55.69651516)(145.28014465,55.71651611)
\curveto(145.36013986,55.73651512)(145.44013978,55.7515151)(145.52014465,55.76151611)
\curveto(145.60013962,55.78151507)(145.67513955,55.80651505)(145.74514465,55.83651611)
\curveto(146.08513914,55.97651488)(146.36013886,56.17151468)(146.57014465,56.42151611)
\curveto(146.78013844,56.67151418)(146.95513827,56.96651389)(147.09514465,57.30651611)
\curveto(147.14513808,57.42651343)(147.17513805,57.5515133)(147.18514465,57.68151611)
\curveto(147.20513802,57.82151303)(147.23513799,57.96151289)(147.27514465,58.10151611)
}
}
{
\newrgbcolor{curcolor}{0 0 0}
\pscustom[linestyle=none,fillstyle=solid,fillcolor=curcolor]
{
\newpath
\moveto(151.0434259,64.82151611)
\curveto(151.19342389,64.82150603)(151.34342374,64.81650604)(151.4934259,64.80651611)
\curveto(151.64342344,64.80650605)(151.74842334,64.76650609)(151.8084259,64.68651611)
\curveto(151.85842323,64.62650623)(151.8834232,64.54150631)(151.8834259,64.43151611)
\curveto(151.89342319,64.33150652)(151.89842319,64.22650663)(151.8984259,64.11651611)
\lineto(151.8984259,63.24651611)
\curveto(151.89842319,63.16650769)(151.89342319,63.08150777)(151.8834259,62.99151611)
\curveto(151.8834232,62.91150794)(151.89342319,62.84150801)(151.9134259,62.78151611)
\curveto(151.95342313,62.64150821)(152.04342304,62.5515083)(152.1834259,62.51151611)
\curveto(152.23342285,62.50150835)(152.27842281,62.49650836)(152.3184259,62.49651611)
\lineto(152.4684259,62.49651611)
\lineto(152.8734259,62.49651611)
\curveto(153.03342205,62.50650835)(153.14842194,62.49650836)(153.2184259,62.46651611)
\curveto(153.30842178,62.40650845)(153.36842172,62.34650851)(153.3984259,62.28651611)
\curveto(153.41842167,62.24650861)(153.42842166,62.20150865)(153.4284259,62.15151611)
\lineto(153.4284259,62.00151611)
\curveto(153.42842166,61.89150896)(153.42342166,61.78650907)(153.4134259,61.68651611)
\curveto(153.40342168,61.59650926)(153.36842172,61.52650933)(153.3084259,61.47651611)
\curveto(153.24842184,61.42650943)(153.16342192,61.39650946)(153.0534259,61.38651611)
\lineto(152.7234259,61.38651611)
\curveto(152.61342247,61.39650946)(152.50342258,61.40150945)(152.3934259,61.40151611)
\curveto(152.2834228,61.40150945)(152.1884229,61.38650947)(152.1084259,61.35651611)
\curveto(152.03842305,61.32650953)(151.9884231,61.27650958)(151.9584259,61.20651611)
\curveto(151.92842316,61.13650972)(151.90842318,61.0515098)(151.8984259,60.95151611)
\curveto(151.8884232,60.86150999)(151.8834232,60.76151009)(151.8834259,60.65151611)
\curveto(151.89342319,60.5515103)(151.89842319,60.4515104)(151.8984259,60.35151611)
\lineto(151.8984259,57.38151611)
\curveto(151.89842319,57.16151369)(151.89342319,56.92651393)(151.8834259,56.67651611)
\curveto(151.8834232,56.43651442)(151.92842316,56.2515146)(152.0184259,56.12151611)
\curveto(152.06842302,56.04151481)(152.13342295,55.98651487)(152.2134259,55.95651611)
\curveto(152.29342279,55.92651493)(152.3884227,55.90151495)(152.4984259,55.88151611)
\curveto(152.52842256,55.87151498)(152.55842253,55.86651499)(152.5884259,55.86651611)
\curveto(152.62842246,55.87651498)(152.66342242,55.87651498)(152.6934259,55.86651611)
\lineto(152.8884259,55.86651611)
\curveto(152.9884221,55.86651499)(153.07842201,55.856515)(153.1584259,55.83651611)
\curveto(153.24842184,55.82651503)(153.31342177,55.79151506)(153.3534259,55.73151611)
\curveto(153.37342171,55.70151515)(153.3884217,55.64651521)(153.3984259,55.56651611)
\curveto(153.41842167,55.49651536)(153.42842166,55.42151543)(153.4284259,55.34151611)
\curveto(153.43842165,55.26151559)(153.43842165,55.18151567)(153.4284259,55.10151611)
\curveto(153.41842167,55.03151582)(153.39842169,54.97651588)(153.3684259,54.93651611)
\curveto(153.32842176,54.86651599)(153.25342183,54.81651604)(153.1434259,54.78651611)
\curveto(153.06342202,54.76651609)(152.97342211,54.7565161)(152.8734259,54.75651611)
\curveto(152.77342231,54.76651609)(152.6834224,54.77151608)(152.6034259,54.77151611)
\curveto(152.54342254,54.77151608)(152.4834226,54.76651609)(152.4234259,54.75651611)
\curveto(152.36342272,54.7565161)(152.30842278,54.76151609)(152.2584259,54.77151611)
\lineto(152.0784259,54.77151611)
\curveto(152.02842306,54.78151607)(151.97842311,54.78651607)(151.9284259,54.78651611)
\curveto(151.8884232,54.79651606)(151.84342324,54.80151605)(151.7934259,54.80151611)
\curveto(151.59342349,54.851516)(151.41842367,54.90651595)(151.2684259,54.96651611)
\curveto(151.12842396,55.02651583)(151.00842408,55.13151572)(150.9084259,55.28151611)
\curveto(150.76842432,55.48151537)(150.6884244,55.73151512)(150.6684259,56.03151611)
\curveto(150.64842444,56.34151451)(150.63842445,56.67151418)(150.6384259,57.02151611)
\lineto(150.6384259,60.95151611)
\curveto(150.60842448,61.08150977)(150.57842451,61.17650968)(150.5484259,61.23651611)
\curveto(150.52842456,61.29650956)(150.45842463,61.34650951)(150.3384259,61.38651611)
\curveto(150.29842479,61.39650946)(150.25842483,61.39650946)(150.2184259,61.38651611)
\curveto(150.17842491,61.37650948)(150.13842495,61.38150947)(150.0984259,61.40151611)
\lineto(149.8584259,61.40151611)
\curveto(149.72842536,61.40150945)(149.61842547,61.41150944)(149.5284259,61.43151611)
\curveto(149.44842564,61.46150939)(149.39342569,61.52150933)(149.3634259,61.61151611)
\curveto(149.34342574,61.6515092)(149.32842576,61.69650916)(149.3184259,61.74651611)
\lineto(149.3184259,61.89651611)
\curveto(149.31842577,62.03650882)(149.32842576,62.1515087)(149.3484259,62.24151611)
\curveto(149.36842572,62.34150851)(149.42842566,62.41650844)(149.5284259,62.46651611)
\curveto(149.63842545,62.50650835)(149.77842531,62.51650834)(149.9484259,62.49651611)
\curveto(150.12842496,62.47650838)(150.27842481,62.48650837)(150.3984259,62.52651611)
\curveto(150.4884246,62.57650828)(150.55842453,62.64650821)(150.6084259,62.73651611)
\curveto(150.62842446,62.79650806)(150.63842445,62.87150798)(150.6384259,62.96151611)
\lineto(150.6384259,63.21651611)
\lineto(150.6384259,64.14651611)
\lineto(150.6384259,64.38651611)
\curveto(150.63842445,64.47650638)(150.64842444,64.5515063)(150.6684259,64.61151611)
\curveto(150.70842438,64.69150616)(150.7834243,64.7565061)(150.8934259,64.80651611)
\curveto(150.92342416,64.80650605)(150.94842414,64.80650605)(150.9684259,64.80651611)
\curveto(150.99842409,64.81650604)(151.02342406,64.82150603)(151.0434259,64.82151611)
}
}
{
\newrgbcolor{curcolor}{0 0 0}
\pscustom[linestyle=none,fillstyle=solid,fillcolor=curcolor]
{
\newpath
\moveto(161.70022278,55.31151611)
\curveto(161.73021495,55.1515157)(161.71521496,55.01651584)(161.65522278,54.90651611)
\curveto(161.59521508,54.80651605)(161.51521516,54.73151612)(161.41522278,54.68151611)
\curveto(161.36521531,54.66151619)(161.31021537,54.6515162)(161.25022278,54.65151611)
\curveto(161.20021548,54.6515162)(161.14521553,54.64151621)(161.08522278,54.62151611)
\curveto(160.86521581,54.57151628)(160.64521603,54.58651627)(160.42522278,54.66651611)
\curveto(160.21521646,54.73651612)(160.07021661,54.82651603)(159.99022278,54.93651611)
\curveto(159.94021674,55.00651585)(159.89521678,55.08651577)(159.85522278,55.17651611)
\curveto(159.81521686,55.27651558)(159.76521691,55.3565155)(159.70522278,55.41651611)
\curveto(159.68521699,55.43651542)(159.66021702,55.4565154)(159.63022278,55.47651611)
\curveto(159.61021707,55.49651536)(159.5802171,55.50151535)(159.54022278,55.49151611)
\curveto(159.43021725,55.46151539)(159.32521735,55.40651545)(159.22522278,55.32651611)
\curveto(159.13521754,55.24651561)(159.04521763,55.17651568)(158.95522278,55.11651611)
\curveto(158.82521785,55.03651582)(158.68521799,54.96151589)(158.53522278,54.89151611)
\curveto(158.38521829,54.83151602)(158.22521845,54.77651608)(158.05522278,54.72651611)
\curveto(157.95521872,54.69651616)(157.84521883,54.67651618)(157.72522278,54.66651611)
\curveto(157.61521906,54.6565162)(157.50521917,54.64151621)(157.39522278,54.62151611)
\curveto(157.34521933,54.61151624)(157.30021938,54.60651625)(157.26022278,54.60651611)
\lineto(157.15522278,54.60651611)
\curveto(157.04521963,54.58651627)(156.94021974,54.58651627)(156.84022278,54.60651611)
\lineto(156.70522278,54.60651611)
\curveto(156.65522002,54.61651624)(156.60522007,54.62151623)(156.55522278,54.62151611)
\curveto(156.50522017,54.62151623)(156.46022022,54.63151622)(156.42022278,54.65151611)
\curveto(156.3802203,54.66151619)(156.34522033,54.66651619)(156.31522278,54.66651611)
\curveto(156.29522038,54.6565162)(156.27022041,54.6565162)(156.24022278,54.66651611)
\lineto(156.00022278,54.72651611)
\curveto(155.92022076,54.73651612)(155.84522083,54.7565161)(155.77522278,54.78651611)
\curveto(155.4752212,54.91651594)(155.23022145,55.06151579)(155.04022278,55.22151611)
\curveto(154.86022182,55.39151546)(154.71022197,55.62651523)(154.59022278,55.92651611)
\curveto(154.50022218,56.14651471)(154.45522222,56.41151444)(154.45522278,56.72151611)
\lineto(154.45522278,57.03651611)
\curveto(154.46522221,57.08651377)(154.47022221,57.13651372)(154.47022278,57.18651611)
\lineto(154.50022278,57.36651611)
\lineto(154.62022278,57.69651611)
\curveto(154.66022202,57.80651305)(154.71022197,57.90651295)(154.77022278,57.99651611)
\curveto(154.95022173,58.28651257)(155.19522148,58.50151235)(155.50522278,58.64151611)
\curveto(155.81522086,58.78151207)(156.15522052,58.90651195)(156.52522278,59.01651611)
\curveto(156.66522001,59.0565118)(156.81021987,59.08651177)(156.96022278,59.10651611)
\curveto(157.11021957,59.12651173)(157.26021942,59.1515117)(157.41022278,59.18151611)
\curveto(157.4802192,59.20151165)(157.54521913,59.21151164)(157.60522278,59.21151611)
\curveto(157.675219,59.21151164)(157.75021893,59.22151163)(157.83022278,59.24151611)
\curveto(157.90021878,59.26151159)(157.97021871,59.27151158)(158.04022278,59.27151611)
\curveto(158.11021857,59.28151157)(158.18521849,59.29651156)(158.26522278,59.31651611)
\curveto(158.51521816,59.37651148)(158.75021793,59.42651143)(158.97022278,59.46651611)
\curveto(159.19021749,59.51651134)(159.36521731,59.63151122)(159.49522278,59.81151611)
\curveto(159.55521712,59.89151096)(159.60521707,59.99151086)(159.64522278,60.11151611)
\curveto(159.68521699,60.24151061)(159.68521699,60.38151047)(159.64522278,60.53151611)
\curveto(159.58521709,60.77151008)(159.49521718,60.96150989)(159.37522278,61.10151611)
\curveto(159.26521741,61.24150961)(159.10521757,61.3515095)(158.89522278,61.43151611)
\curveto(158.7752179,61.48150937)(158.63021805,61.51650934)(158.46022278,61.53651611)
\curveto(158.30021838,61.5565093)(158.13021855,61.56650929)(157.95022278,61.56651611)
\curveto(157.77021891,61.56650929)(157.59521908,61.5565093)(157.42522278,61.53651611)
\curveto(157.25521942,61.51650934)(157.11021957,61.48650937)(156.99022278,61.44651611)
\curveto(156.82021986,61.38650947)(156.65522002,61.30150955)(156.49522278,61.19151611)
\curveto(156.41522026,61.13150972)(156.34022034,61.0515098)(156.27022278,60.95151611)
\curveto(156.21022047,60.86150999)(156.15522052,60.76151009)(156.10522278,60.65151611)
\curveto(156.0752206,60.57151028)(156.04522063,60.48651037)(156.01522278,60.39651611)
\curveto(155.99522068,60.30651055)(155.95022073,60.23651062)(155.88022278,60.18651611)
\curveto(155.84022084,60.1565107)(155.77022091,60.13151072)(155.67022278,60.11151611)
\curveto(155.5802211,60.10151075)(155.48522119,60.09651076)(155.38522278,60.09651611)
\curveto(155.28522139,60.09651076)(155.18522149,60.10151075)(155.08522278,60.11151611)
\curveto(154.99522168,60.13151072)(154.93022175,60.1565107)(154.89022278,60.18651611)
\curveto(154.85022183,60.21651064)(154.82022186,60.26651059)(154.80022278,60.33651611)
\curveto(154.7802219,60.40651045)(154.7802219,60.48151037)(154.80022278,60.56151611)
\curveto(154.83022185,60.69151016)(154.86022182,60.81151004)(154.89022278,60.92151611)
\curveto(154.93022175,61.04150981)(154.9752217,61.1565097)(155.02522278,61.26651611)
\curveto(155.21522146,61.61650924)(155.45522122,61.88650897)(155.74522278,62.07651611)
\curveto(156.03522064,62.27650858)(156.39522028,62.43650842)(156.82522278,62.55651611)
\curveto(156.92521975,62.57650828)(157.02521965,62.59150826)(157.12522278,62.60151611)
\curveto(157.23521944,62.61150824)(157.34521933,62.62650823)(157.45522278,62.64651611)
\curveto(157.49521918,62.6565082)(157.56021912,62.6565082)(157.65022278,62.64651611)
\curveto(157.74021894,62.64650821)(157.79521888,62.6565082)(157.81522278,62.67651611)
\curveto(158.51521816,62.68650817)(159.12521755,62.60650825)(159.64522278,62.43651611)
\curveto(160.16521651,62.26650859)(160.53021615,61.94150891)(160.74022278,61.46151611)
\curveto(160.83021585,61.26150959)(160.8802158,61.02650983)(160.89022278,60.75651611)
\curveto(160.91021577,60.49651036)(160.92021576,60.22151063)(160.92022278,59.93151611)
\lineto(160.92022278,56.61651611)
\curveto(160.92021576,56.47651438)(160.92521575,56.34151451)(160.93522278,56.21151611)
\curveto(160.94521573,56.08151477)(160.9752157,55.97651488)(161.02522278,55.89651611)
\curveto(161.0752156,55.82651503)(161.14021554,55.77651508)(161.22022278,55.74651611)
\curveto(161.31021537,55.70651515)(161.39521528,55.67651518)(161.47522278,55.65651611)
\curveto(161.55521512,55.64651521)(161.61521506,55.60151525)(161.65522278,55.52151611)
\curveto(161.675215,55.49151536)(161.68521499,55.46151539)(161.68522278,55.43151611)
\curveto(161.68521499,55.40151545)(161.69021499,55.36151549)(161.70022278,55.31151611)
\moveto(159.55522278,56.97651611)
\curveto(159.61521706,57.11651374)(159.64521703,57.27651358)(159.64522278,57.45651611)
\curveto(159.65521702,57.64651321)(159.66021702,57.84151301)(159.66022278,58.04151611)
\curveto(159.66021702,58.1515127)(159.65521702,58.2515126)(159.64522278,58.34151611)
\curveto(159.63521704,58.43151242)(159.59521708,58.50151235)(159.52522278,58.55151611)
\curveto(159.49521718,58.57151228)(159.42521725,58.58151227)(159.31522278,58.58151611)
\curveto(159.29521738,58.56151229)(159.26021742,58.5515123)(159.21022278,58.55151611)
\curveto(159.16021752,58.5515123)(159.11521756,58.54151231)(159.07522278,58.52151611)
\curveto(158.99521768,58.50151235)(158.90521777,58.48151237)(158.80522278,58.46151611)
\lineto(158.50522278,58.40151611)
\curveto(158.4752182,58.40151245)(158.44021824,58.39651246)(158.40022278,58.38651611)
\lineto(158.29522278,58.38651611)
\curveto(158.14521853,58.34651251)(157.9802187,58.32151253)(157.80022278,58.31151611)
\curveto(157.63021905,58.31151254)(157.47021921,58.29151256)(157.32022278,58.25151611)
\curveto(157.24021944,58.23151262)(157.16521951,58.21151264)(157.09522278,58.19151611)
\curveto(157.03521964,58.18151267)(156.96521971,58.16651269)(156.88522278,58.14651611)
\curveto(156.72521995,58.09651276)(156.5752201,58.03151282)(156.43522278,57.95151611)
\curveto(156.29522038,57.88151297)(156.1752205,57.79151306)(156.07522278,57.68151611)
\curveto(155.9752207,57.57151328)(155.90022078,57.43651342)(155.85022278,57.27651611)
\curveto(155.80022088,57.12651373)(155.7802209,56.94151391)(155.79022278,56.72151611)
\curveto(155.79022089,56.62151423)(155.80522087,56.52651433)(155.83522278,56.43651611)
\curveto(155.8752208,56.3565145)(155.92022076,56.28151457)(155.97022278,56.21151611)
\curveto(156.05022063,56.10151475)(156.15522052,56.00651485)(156.28522278,55.92651611)
\curveto(156.41522026,55.856515)(156.55522012,55.79651506)(156.70522278,55.74651611)
\curveto(156.75521992,55.73651512)(156.80521987,55.73151512)(156.85522278,55.73151611)
\curveto(156.90521977,55.73151512)(156.95521972,55.72651513)(157.00522278,55.71651611)
\curveto(157.0752196,55.69651516)(157.16021952,55.68151517)(157.26022278,55.67151611)
\curveto(157.37021931,55.67151518)(157.46021922,55.68151517)(157.53022278,55.70151611)
\curveto(157.59021909,55.72151513)(157.65021903,55.72651513)(157.71022278,55.71651611)
\curveto(157.77021891,55.71651514)(157.83021885,55.72651513)(157.89022278,55.74651611)
\curveto(157.97021871,55.76651509)(158.04521863,55.78151507)(158.11522278,55.79151611)
\curveto(158.19521848,55.80151505)(158.27021841,55.82151503)(158.34022278,55.85151611)
\curveto(158.63021805,55.97151488)(158.8752178,56.11651474)(159.07522278,56.28651611)
\curveto(159.28521739,56.4565144)(159.44521723,56.68651417)(159.55522278,56.97651611)
}
}
{
\newrgbcolor{curcolor}{0 0 0}
\pscustom[linestyle=none,fillstyle=solid,fillcolor=curcolor]
{
\newpath
\moveto(165.3018634,62.66151611)
\curveto(166.02185934,62.67150818)(166.62685873,62.58650827)(167.1168634,62.40651611)
\curveto(167.60685775,62.23650862)(167.98685737,61.93150892)(168.2568634,61.49151611)
\curveto(168.32685703,61.38150947)(168.38185698,61.26650959)(168.4218634,61.14651611)
\curveto(168.4618569,61.03650982)(168.50185686,60.91150994)(168.5418634,60.77151611)
\curveto(168.5618568,60.70151015)(168.56685679,60.62651023)(168.5568634,60.54651611)
\curveto(168.54685681,60.47651038)(168.53185683,60.42151043)(168.5118634,60.38151611)
\curveto(168.49185687,60.36151049)(168.46685689,60.34151051)(168.4368634,60.32151611)
\curveto(168.40685695,60.31151054)(168.38185698,60.29651056)(168.3618634,60.27651611)
\curveto(168.31185705,60.2565106)(168.2618571,60.2515106)(168.2118634,60.26151611)
\curveto(168.1618572,60.27151058)(168.11185725,60.27151058)(168.0618634,60.26151611)
\curveto(167.98185738,60.24151061)(167.87685748,60.23651062)(167.7468634,60.24651611)
\curveto(167.61685774,60.26651059)(167.52685783,60.29151056)(167.4768634,60.32151611)
\curveto(167.39685796,60.37151048)(167.34185802,60.43651042)(167.3118634,60.51651611)
\curveto(167.29185807,60.60651025)(167.2568581,60.69151016)(167.2068634,60.77151611)
\curveto(167.11685824,60.93150992)(166.99185837,61.07650978)(166.8318634,61.20651611)
\curveto(166.72185864,61.28650957)(166.60185876,61.34650951)(166.4718634,61.38651611)
\curveto(166.34185902,61.42650943)(166.20185916,61.46650939)(166.0518634,61.50651611)
\curveto(166.00185936,61.52650933)(165.95185941,61.53150932)(165.9018634,61.52151611)
\curveto(165.85185951,61.52150933)(165.80185956,61.52650933)(165.7518634,61.53651611)
\curveto(165.69185967,61.5565093)(165.61685974,61.56650929)(165.5268634,61.56651611)
\curveto(165.43685992,61.56650929)(165.36186,61.5565093)(165.3018634,61.53651611)
\lineto(165.2118634,61.53651611)
\lineto(165.0618634,61.50651611)
\curveto(165.01186035,61.50650935)(164.9618604,61.50150935)(164.9118634,61.49151611)
\curveto(164.65186071,61.43150942)(164.43686092,61.34650951)(164.2668634,61.23651611)
\curveto(164.09686126,61.12650973)(163.98186138,60.94150991)(163.9218634,60.68151611)
\curveto(163.90186146,60.61151024)(163.89686146,60.54151031)(163.9068634,60.47151611)
\curveto(163.92686143,60.40151045)(163.94686141,60.34151051)(163.9668634,60.29151611)
\curveto(164.02686133,60.14151071)(164.09686126,60.03151082)(164.1768634,59.96151611)
\curveto(164.26686109,59.90151095)(164.37686098,59.83151102)(164.5068634,59.75151611)
\curveto(164.66686069,59.6515112)(164.84686051,59.57651128)(165.0468634,59.52651611)
\curveto(165.24686011,59.48651137)(165.44685991,59.43651142)(165.6468634,59.37651611)
\curveto(165.77685958,59.33651152)(165.90685945,59.30651155)(166.0368634,59.28651611)
\curveto(166.16685919,59.26651159)(166.29685906,59.23651162)(166.4268634,59.19651611)
\curveto(166.63685872,59.13651172)(166.84185852,59.07651178)(167.0418634,59.01651611)
\curveto(167.24185812,58.96651189)(167.44185792,58.90151195)(167.6418634,58.82151611)
\lineto(167.7918634,58.76151611)
\curveto(167.84185752,58.74151211)(167.89185747,58.71651214)(167.9418634,58.68651611)
\curveto(168.14185722,58.56651229)(168.31685704,58.43151242)(168.4668634,58.28151611)
\curveto(168.61685674,58.13151272)(168.74185662,57.94151291)(168.8418634,57.71151611)
\curveto(168.8618565,57.64151321)(168.88185648,57.54651331)(168.9018634,57.42651611)
\curveto(168.92185644,57.3565135)(168.93185643,57.28151357)(168.9318634,57.20151611)
\curveto(168.94185642,57.13151372)(168.94685641,57.0515138)(168.9468634,56.96151611)
\lineto(168.9468634,56.81151611)
\curveto(168.92685643,56.74151411)(168.91685644,56.67151418)(168.9168634,56.60151611)
\curveto(168.91685644,56.53151432)(168.90685645,56.46151439)(168.8868634,56.39151611)
\curveto(168.8568565,56.28151457)(168.82185654,56.17651468)(168.7818634,56.07651611)
\curveto(168.74185662,55.97651488)(168.69685666,55.88651497)(168.6468634,55.80651611)
\curveto(168.48685687,55.54651531)(168.28185708,55.33651552)(168.0318634,55.17651611)
\curveto(167.78185758,55.02651583)(167.50185786,54.89651596)(167.1918634,54.78651611)
\curveto(167.10185826,54.7565161)(167.00685835,54.73651612)(166.9068634,54.72651611)
\curveto(166.81685854,54.70651615)(166.72685863,54.68151617)(166.6368634,54.65151611)
\curveto(166.53685882,54.63151622)(166.43685892,54.62151623)(166.3368634,54.62151611)
\curveto(166.23685912,54.62151623)(166.13685922,54.61151624)(166.0368634,54.59151611)
\lineto(165.8868634,54.59151611)
\curveto(165.83685952,54.58151627)(165.76685959,54.57651628)(165.6768634,54.57651611)
\curveto(165.58685977,54.57651628)(165.51685984,54.58151627)(165.4668634,54.59151611)
\lineto(165.3018634,54.59151611)
\curveto(165.24186012,54.61151624)(165.17686018,54.62151623)(165.1068634,54.62151611)
\curveto(165.03686032,54.61151624)(164.97686038,54.61651624)(164.9268634,54.63651611)
\curveto(164.87686048,54.64651621)(164.81186055,54.6515162)(164.7318634,54.65151611)
\lineto(164.4918634,54.71151611)
\curveto(164.42186094,54.72151613)(164.34686101,54.74151611)(164.2668634,54.77151611)
\curveto(163.9568614,54.87151598)(163.68686167,54.99651586)(163.4568634,55.14651611)
\curveto(163.22686213,55.29651556)(163.02686233,55.49151536)(162.8568634,55.73151611)
\curveto(162.76686259,55.86151499)(162.69186267,55.99651486)(162.6318634,56.13651611)
\curveto(162.57186279,56.27651458)(162.51686284,56.43151442)(162.4668634,56.60151611)
\curveto(162.44686291,56.66151419)(162.43686292,56.73151412)(162.4368634,56.81151611)
\curveto(162.44686291,56.90151395)(162.4618629,56.97151388)(162.4818634,57.02151611)
\curveto(162.51186285,57.06151379)(162.5618628,57.10151375)(162.6318634,57.14151611)
\curveto(162.68186268,57.16151369)(162.75186261,57.17151368)(162.8418634,57.17151611)
\curveto(162.93186243,57.18151367)(163.02186234,57.18151367)(163.1118634,57.17151611)
\curveto(163.20186216,57.16151369)(163.28686207,57.14651371)(163.3668634,57.12651611)
\curveto(163.4568619,57.11651374)(163.51686184,57.10151375)(163.5468634,57.08151611)
\curveto(163.61686174,57.03151382)(163.6618617,56.9565139)(163.6818634,56.85651611)
\curveto(163.71186165,56.76651409)(163.74686161,56.68151417)(163.7868634,56.60151611)
\curveto(163.88686147,56.38151447)(164.02186134,56.21151464)(164.1918634,56.09151611)
\curveto(164.31186105,56.00151485)(164.44686091,55.93151492)(164.5968634,55.88151611)
\curveto(164.74686061,55.83151502)(164.90686045,55.78151507)(165.0768634,55.73151611)
\lineto(165.3918634,55.68651611)
\lineto(165.4818634,55.68651611)
\curveto(165.55185981,55.66651519)(165.64185972,55.6565152)(165.7518634,55.65651611)
\curveto(165.87185949,55.6565152)(165.97185939,55.66651519)(166.0518634,55.68651611)
\curveto(166.12185924,55.68651517)(166.17685918,55.69151516)(166.2168634,55.70151611)
\curveto(166.27685908,55.71151514)(166.33685902,55.71651514)(166.3968634,55.71651611)
\curveto(166.4568589,55.72651513)(166.51185885,55.73651512)(166.5618634,55.74651611)
\curveto(166.85185851,55.82651503)(167.08185828,55.93151492)(167.2518634,56.06151611)
\curveto(167.42185794,56.19151466)(167.54185782,56.41151444)(167.6118634,56.72151611)
\curveto(167.63185773,56.77151408)(167.63685772,56.82651403)(167.6268634,56.88651611)
\curveto(167.61685774,56.94651391)(167.60685775,56.99151386)(167.5968634,57.02151611)
\curveto(167.54685781,57.21151364)(167.47685788,57.3515135)(167.3868634,57.44151611)
\curveto(167.29685806,57.54151331)(167.18185818,57.63151322)(167.0418634,57.71151611)
\curveto(166.95185841,57.77151308)(166.85185851,57.82151303)(166.7418634,57.86151611)
\lineto(166.4118634,57.98151611)
\curveto(166.38185898,57.99151286)(166.35185901,57.99651286)(166.3218634,57.99651611)
\curveto(166.30185906,57.99651286)(166.27685908,58.00651285)(166.2468634,58.02651611)
\curveto(165.90685945,58.13651272)(165.55185981,58.21651264)(165.1818634,58.26651611)
\curveto(164.82186054,58.32651253)(164.48186088,58.42151243)(164.1618634,58.55151611)
\curveto(164.0618613,58.59151226)(163.96686139,58.62651223)(163.8768634,58.65651611)
\curveto(163.78686157,58.68651217)(163.70186166,58.72651213)(163.6218634,58.77651611)
\curveto(163.43186193,58.88651197)(163.2568621,59.01151184)(163.0968634,59.15151611)
\curveto(162.93686242,59.29151156)(162.81186255,59.46651139)(162.7218634,59.67651611)
\curveto(162.69186267,59.74651111)(162.66686269,59.81651104)(162.6468634,59.88651611)
\curveto(162.63686272,59.9565109)(162.62186274,60.03151082)(162.6018634,60.11151611)
\curveto(162.57186279,60.23151062)(162.5618628,60.36651049)(162.5718634,60.51651611)
\curveto(162.58186278,60.67651018)(162.59686276,60.81151004)(162.6168634,60.92151611)
\curveto(162.63686272,60.97150988)(162.64686271,61.01150984)(162.6468634,61.04151611)
\curveto(162.6568627,61.08150977)(162.67186269,61.12150973)(162.6918634,61.16151611)
\curveto(162.78186258,61.39150946)(162.90186246,61.59150926)(163.0518634,61.76151611)
\curveto(163.21186215,61.93150892)(163.39186197,62.08150877)(163.5918634,62.21151611)
\curveto(163.74186162,62.30150855)(163.90686145,62.37150848)(164.0868634,62.42151611)
\curveto(164.26686109,62.48150837)(164.4568609,62.53650832)(164.6568634,62.58651611)
\curveto(164.72686063,62.59650826)(164.79186057,62.60650825)(164.8518634,62.61651611)
\curveto(164.92186044,62.62650823)(164.99686036,62.63650822)(165.0768634,62.64651611)
\curveto(165.10686025,62.6565082)(165.14686021,62.6565082)(165.1968634,62.64651611)
\curveto(165.24686011,62.63650822)(165.28186008,62.64150821)(165.3018634,62.66151611)
}
}
{
\newrgbcolor{curcolor}{0 0 0}
\pscustom[linestyle=none,fillstyle=solid,fillcolor=curcolor]
{
\newpath
\moveto(337.40385742,65.42900879)
\lineto(342.30885742,65.42900879)
\lineto(343.59885742,65.42900879)
\curveto(343.70884954,65.42899809)(343.81884943,65.42899809)(343.92885742,65.42900879)
\curveto(344.03884921,65.43899808)(344.12884912,65.4189981)(344.19885742,65.36900879)
\curveto(344.22884902,65.34899817)(344.253849,65.3239982)(344.27385742,65.29400879)
\curveto(344.29384896,65.26399826)(344.31384894,65.23399829)(344.33385742,65.20400879)
\curveto(344.3538489,65.13399839)(344.36384889,65.0189985)(344.36385742,64.85900879)
\curveto(344.36384889,64.70899881)(344.3538489,64.59399893)(344.33385742,64.51400879)
\curveto(344.29384896,64.37399915)(344.20884904,64.29399923)(344.07885742,64.27400879)
\curveto(343.9488493,64.26399926)(343.79384946,64.25899926)(343.61385742,64.25900879)
\lineto(342.11385742,64.25900879)
\lineto(339.59385742,64.25900879)
\lineto(339.02385742,64.25900879)
\curveto(338.81385444,64.26899925)(338.65885459,64.24399928)(338.55885742,64.18400879)
\curveto(338.45885479,64.1239994)(338.40385485,64.0189995)(338.39385742,63.86900879)
\lineto(338.39385742,63.40400879)
\lineto(338.39385742,61.87400879)
\curveto(338.39385486,61.76400176)(338.38885486,61.63400189)(338.37885742,61.48400879)
\curveto(338.37885487,61.33400219)(338.38885486,61.21400231)(338.40885742,61.12400879)
\curveto(338.43885481,61.00400252)(338.49885475,60.9240026)(338.58885742,60.88400879)
\curveto(338.62885462,60.86400266)(338.69885455,60.84400268)(338.79885742,60.82400879)
\lineto(338.94885742,60.82400879)
\curveto(338.98885426,60.81400271)(339.02885422,60.80900271)(339.06885742,60.80900879)
\curveto(339.11885413,60.8190027)(339.16885408,60.8240027)(339.21885742,60.82400879)
\lineto(339.72885742,60.82400879)
\lineto(342.66885742,60.82400879)
\lineto(342.96885742,60.82400879)
\curveto(343.07885017,60.83400269)(343.18885006,60.83400269)(343.29885742,60.82400879)
\curveto(343.41884983,60.8240027)(343.52384973,60.81400271)(343.61385742,60.79400879)
\curveto(343.71384954,60.78400274)(343.78884946,60.76400276)(343.83885742,60.73400879)
\curveto(343.86884938,60.71400281)(343.89384936,60.66900285)(343.91385742,60.59900879)
\curveto(343.93384932,60.52900299)(343.9488493,60.45400307)(343.95885742,60.37400879)
\curveto(343.96884928,60.29400323)(343.96884928,60.20900331)(343.95885742,60.11900879)
\curveto(343.95884929,60.03900348)(343.9488493,59.96900355)(343.92885742,59.90900879)
\curveto(343.90884934,59.8190037)(343.86384939,59.75400377)(343.79385742,59.71400879)
\curveto(343.77384948,59.69400383)(343.74384951,59.67900384)(343.70385742,59.66900879)
\curveto(343.67384958,59.66900385)(343.64384961,59.66400386)(343.61385742,59.65400879)
\lineto(343.52385742,59.65400879)
\curveto(343.47384978,59.64400388)(343.42384983,59.63900388)(343.37385742,59.63900879)
\curveto(343.32384993,59.64900387)(343.27384998,59.65400387)(343.22385742,59.65400879)
\lineto(342.66885742,59.65400879)
\lineto(339.50385742,59.65400879)
\lineto(339.14385742,59.65400879)
\curveto(339.03385422,59.66400386)(338.92885432,59.65900386)(338.82885742,59.63900879)
\curveto(338.72885452,59.62900389)(338.63885461,59.60400392)(338.55885742,59.56400879)
\curveto(338.48885476,59.524004)(338.43885481,59.45400407)(338.40885742,59.35400879)
\curveto(338.38885486,59.29400423)(338.37885487,59.2240043)(338.37885742,59.14400879)
\curveto(338.38885486,59.06400446)(338.39385486,58.98400454)(338.39385742,58.90400879)
\lineto(338.39385742,58.06400879)
\lineto(338.39385742,56.63900879)
\curveto(338.39385486,56.49900702)(338.39885485,56.36900715)(338.40885742,56.24900879)
\curveto(338.41885483,56.13900738)(338.45885479,56.05900746)(338.52885742,56.00900879)
\curveto(338.59885465,55.95900756)(338.67885457,55.92900759)(338.76885742,55.91900879)
\lineto(339.06885742,55.91900879)
\lineto(340.02885742,55.91900879)
\lineto(342.80385742,55.91900879)
\lineto(343.65885742,55.91900879)
\lineto(343.89885742,55.91900879)
\curveto(343.97884927,55.92900759)(344.0488492,55.9240076)(344.10885742,55.90400879)
\curveto(344.22884902,55.86400766)(344.30884894,55.80900771)(344.34885742,55.73900879)
\curveto(344.36884888,55.70900781)(344.38384887,55.65900786)(344.39385742,55.58900879)
\curveto(344.40384885,55.519008)(344.40884884,55.44400808)(344.40885742,55.36400879)
\curveto(344.41884883,55.29400823)(344.41884883,55.2190083)(344.40885742,55.13900879)
\curveto(344.39884885,55.06900845)(344.38884886,55.01400851)(344.37885742,54.97400879)
\curveto(344.33884891,54.89400863)(344.29384896,54.83900868)(344.24385742,54.80900879)
\curveto(344.18384907,54.76900875)(344.10384915,54.74900877)(344.00385742,54.74900879)
\lineto(343.73385742,54.74900879)
\lineto(342.68385742,54.74900879)
\lineto(338.69385742,54.74900879)
\lineto(337.64385742,54.74900879)
\curveto(337.50385575,54.74900877)(337.38385587,54.75400877)(337.28385742,54.76400879)
\curveto(337.18385607,54.78400874)(337.10885614,54.83400869)(337.05885742,54.91400879)
\curveto(337.01885623,54.97400855)(336.99885625,55.04900847)(336.99885742,55.13900879)
\lineto(336.99885742,55.42400879)
\lineto(336.99885742,56.47400879)
\lineto(336.99885742,60.49400879)
\lineto(336.99885742,63.85400879)
\lineto(336.99885742,64.78400879)
\lineto(336.99885742,65.05400879)
\curveto(336.99885625,65.14399838)(337.01885623,65.21399831)(337.05885742,65.26400879)
\curveto(337.09885615,65.33399819)(337.17385608,65.38399814)(337.28385742,65.41400879)
\curveto(337.30385595,65.4239981)(337.32385593,65.4239981)(337.34385742,65.41400879)
\curveto(337.36385589,65.41399811)(337.38385587,65.4189981)(337.40385742,65.42900879)
}
}
{
\newrgbcolor{curcolor}{0 0 0}
\pscustom[linestyle=none,fillstyle=solid,fillcolor=curcolor]
{
\newpath
\moveto(345.6887793,62.47400879)
\lineto(346.1687793,62.47400879)
\curveto(346.33877796,62.47400105)(346.46877783,62.44400108)(346.5587793,62.38400879)
\curveto(346.62877767,62.33400119)(346.67377762,62.26900125)(346.6937793,62.18900879)
\curveto(346.72377757,62.1190014)(346.75377754,62.04400148)(346.7837793,61.96400879)
\curveto(346.84377745,61.8240017)(346.8937774,61.68400184)(346.9337793,61.54400879)
\curveto(346.97377732,61.40400212)(347.01877728,61.26400226)(347.0687793,61.12400879)
\curveto(347.26877703,60.58400294)(347.45377684,60.03900348)(347.6237793,59.48900879)
\curveto(347.7937765,58.94900457)(347.97877632,58.40900511)(348.1787793,57.86900879)
\curveto(348.24877605,57.68900583)(348.30877599,57.50400602)(348.3587793,57.31400879)
\curveto(348.40877589,57.13400639)(348.47377582,56.95400657)(348.5537793,56.77400879)
\curveto(348.57377572,56.70400682)(348.5987757,56.62900689)(348.6287793,56.54900879)
\curveto(348.65877564,56.46900705)(348.70877559,56.4190071)(348.7787793,56.39900879)
\curveto(348.85877544,56.37900714)(348.91877538,56.41400711)(348.9587793,56.50400879)
\curveto(349.00877529,56.59400693)(349.04377525,56.66400686)(349.0637793,56.71400879)
\curveto(349.14377515,56.90400662)(349.20877509,57.09400643)(349.2587793,57.28400879)
\curveto(349.31877498,57.48400604)(349.38377491,57.68400584)(349.4537793,57.88400879)
\curveto(349.58377471,58.26400526)(349.70877459,58.63900488)(349.8287793,59.00900879)
\curveto(349.94877435,59.38900413)(350.07377422,59.76900375)(350.2037793,60.14900879)
\curveto(350.25377404,60.3190032)(350.30377399,60.48400304)(350.3537793,60.64400879)
\curveto(350.40377389,60.81400271)(350.46377383,60.97900254)(350.5337793,61.13900879)
\curveto(350.58377371,61.27900224)(350.62877367,61.4190021)(350.6687793,61.55900879)
\curveto(350.70877359,61.69900182)(350.75377354,61.83900168)(350.8037793,61.97900879)
\curveto(350.82377347,62.04900147)(350.84877345,62.1190014)(350.8787793,62.18900879)
\curveto(350.90877339,62.25900126)(350.94877335,62.3190012)(350.9987793,62.36900879)
\curveto(351.07877322,62.4190011)(351.16877313,62.44900107)(351.2687793,62.45900879)
\curveto(351.36877293,62.46900105)(351.48877281,62.47400105)(351.6287793,62.47400879)
\curveto(351.6987726,62.47400105)(351.76377253,62.46900105)(351.8237793,62.45900879)
\curveto(351.88377241,62.45900106)(351.93877236,62.44900107)(351.9887793,62.42900879)
\curveto(352.07877222,62.38900113)(352.12377217,62.3240012)(352.1237793,62.23400879)
\curveto(352.13377216,62.14400138)(352.11877218,62.05400147)(352.0787793,61.96400879)
\curveto(352.01877228,61.79400173)(351.95877234,61.6190019)(351.8987793,61.43900879)
\curveto(351.83877246,61.25900226)(351.76877253,61.08400244)(351.6887793,60.91400879)
\curveto(351.66877263,60.86400266)(351.65377264,60.81400271)(351.6437793,60.76400879)
\curveto(351.63377266,60.7240028)(351.61877268,60.67900284)(351.5987793,60.62900879)
\curveto(351.51877278,60.45900306)(351.45377284,60.28400324)(351.4037793,60.10400879)
\curveto(351.35377294,59.9240036)(351.28877301,59.74400378)(351.2087793,59.56400879)
\curveto(351.15877314,59.43400409)(351.10877319,59.29900422)(351.0587793,59.15900879)
\curveto(351.01877328,59.02900449)(350.96877333,58.89900462)(350.9087793,58.76900879)
\curveto(350.73877356,58.35900516)(350.58377371,57.94400558)(350.4437793,57.52400879)
\curveto(350.31377398,57.10400642)(350.16377413,56.68900683)(349.9937793,56.27900879)
\curveto(349.93377436,56.1190074)(349.87877442,55.95900756)(349.8287793,55.79900879)
\curveto(349.77877452,55.63900788)(349.71877458,55.47900804)(349.6487793,55.31900879)
\curveto(349.5987747,55.20900831)(349.55377474,55.10400842)(349.5137793,55.00400879)
\curveto(349.48377481,54.91400861)(349.41377488,54.84400868)(349.3037793,54.79400879)
\curveto(349.24377505,54.76400876)(349.17377512,54.74900877)(349.0937793,54.74900879)
\lineto(348.8687793,54.74900879)
\lineto(348.4037793,54.74900879)
\curveto(348.25377604,54.75900876)(348.14377615,54.80900871)(348.0737793,54.89900879)
\curveto(348.00377629,54.97900854)(347.95377634,55.07400845)(347.9237793,55.18400879)
\curveto(347.8937764,55.30400822)(347.85377644,55.4190081)(347.8037793,55.52900879)
\curveto(347.74377655,55.66900785)(347.68377661,55.81400771)(347.6237793,55.96400879)
\curveto(347.57377672,56.1240074)(347.52377677,56.27400725)(347.4737793,56.41400879)
\curveto(347.45377684,56.46400706)(347.43877686,56.50400702)(347.4287793,56.53400879)
\curveto(347.41877688,56.57400695)(347.40377689,56.6190069)(347.3837793,56.66900879)
\curveto(347.18377711,57.14900637)(346.9987773,57.63400589)(346.8287793,58.12400879)
\curveto(346.66877763,58.61400491)(346.48877781,59.09900442)(346.2887793,59.57900879)
\curveto(346.22877807,59.73900378)(346.16877813,59.89400363)(346.1087793,60.04400879)
\curveto(346.05877824,60.20400332)(346.00377829,60.36400316)(345.9437793,60.52400879)
\lineto(345.8837793,60.67400879)
\curveto(345.87377842,60.73400279)(345.85877844,60.78900273)(345.8387793,60.83900879)
\curveto(345.75877854,61.00900251)(345.68877861,61.17900234)(345.6287793,61.34900879)
\curveto(345.57877872,61.519002)(345.51877878,61.68900183)(345.4487793,61.85900879)
\curveto(345.42877887,61.9190016)(345.40377889,61.99900152)(345.3737793,62.09900879)
\curveto(345.34377895,62.19900132)(345.34877895,62.28400124)(345.3887793,62.35400879)
\curveto(345.43877886,62.40400112)(345.4987788,62.43900108)(345.5687793,62.45900879)
\curveto(345.63877866,62.45900106)(345.67877862,62.46400106)(345.6887793,62.47400879)
}
}
{
\newrgbcolor{curcolor}{0 0 0}
\pscustom[linestyle=none,fillstyle=solid,fillcolor=curcolor]
{
\newpath
\moveto(360.1637793,58.91900879)
\curveto(360.18377161,58.8190047)(360.18377161,58.70400482)(360.1637793,58.57400879)
\curveto(360.15377164,58.45400507)(360.12377167,58.36900515)(360.0737793,58.31900879)
\curveto(360.02377177,58.27900524)(359.94877185,58.24900527)(359.8487793,58.22900879)
\curveto(359.75877204,58.2190053)(359.65377214,58.21400531)(359.5337793,58.21400879)
\lineto(359.1737793,58.21400879)
\curveto(359.05377274,58.2240053)(358.94877285,58.22900529)(358.8587793,58.22900879)
\lineto(355.0187793,58.22900879)
\curveto(354.93877686,58.22900529)(354.85877694,58.2240053)(354.7787793,58.21400879)
\curveto(354.6987771,58.21400531)(354.63377716,58.19900532)(354.5837793,58.16900879)
\curveto(354.54377725,58.14900537)(354.50377729,58.10900541)(354.4637793,58.04900879)
\curveto(354.44377735,58.0190055)(354.42377737,57.97400555)(354.4037793,57.91400879)
\curveto(354.38377741,57.86400566)(354.38377741,57.81400571)(354.4037793,57.76400879)
\curveto(354.41377738,57.71400581)(354.41877738,57.66900585)(354.4187793,57.62900879)
\curveto(354.41877738,57.58900593)(354.42377737,57.54900597)(354.4337793,57.50900879)
\curveto(354.45377734,57.42900609)(354.47377732,57.34400618)(354.4937793,57.25400879)
\curveto(354.51377728,57.17400635)(354.54377725,57.09400643)(354.5837793,57.01400879)
\curveto(354.81377698,56.47400705)(355.1937766,56.08900743)(355.7237793,55.85900879)
\curveto(355.78377601,55.82900769)(355.84877595,55.80400772)(355.9187793,55.78400879)
\lineto(356.1287793,55.72400879)
\curveto(356.15877564,55.71400781)(356.20877559,55.70900781)(356.2787793,55.70900879)
\curveto(356.41877538,55.66900785)(356.60377519,55.64900787)(356.8337793,55.64900879)
\curveto(357.06377473,55.64900787)(357.24877455,55.66900785)(357.3887793,55.70900879)
\curveto(357.52877427,55.74900777)(357.65377414,55.78900773)(357.7637793,55.82900879)
\curveto(357.88377391,55.87900764)(357.9937738,55.93900758)(358.0937793,56.00900879)
\curveto(358.20377359,56.07900744)(358.2987735,56.15900736)(358.3787793,56.24900879)
\curveto(358.45877334,56.34900717)(358.52877327,56.45400707)(358.5887793,56.56400879)
\curveto(358.64877315,56.66400686)(358.6987731,56.76900675)(358.7387793,56.87900879)
\curveto(358.78877301,56.98900653)(358.86877293,57.06900645)(358.9787793,57.11900879)
\curveto(359.01877278,57.13900638)(359.08377271,57.15400637)(359.1737793,57.16400879)
\curveto(359.26377253,57.17400635)(359.35377244,57.17400635)(359.4437793,57.16400879)
\curveto(359.53377226,57.16400636)(359.61877218,57.15900636)(359.6987793,57.14900879)
\curveto(359.77877202,57.13900638)(359.83377196,57.1190064)(359.8637793,57.08900879)
\curveto(359.96377183,57.0190065)(359.98877181,56.90400662)(359.9387793,56.74400879)
\curveto(359.85877194,56.47400705)(359.75377204,56.23400729)(359.6237793,56.02400879)
\curveto(359.42377237,55.70400782)(359.1937726,55.43900808)(358.9337793,55.22900879)
\curveto(358.68377311,55.02900849)(358.36377343,54.86400866)(357.9737793,54.73400879)
\curveto(357.87377392,54.69400883)(357.77377402,54.66900885)(357.6737793,54.65900879)
\curveto(357.57377422,54.63900888)(357.46877433,54.6190089)(357.3587793,54.59900879)
\curveto(357.30877449,54.58900893)(357.25877454,54.58400894)(357.2087793,54.58400879)
\curveto(357.16877463,54.58400894)(357.12377467,54.57900894)(357.0737793,54.56900879)
\lineto(356.9237793,54.56900879)
\curveto(356.87377492,54.55900896)(356.81377498,54.55400897)(356.7437793,54.55400879)
\curveto(356.68377511,54.55400897)(356.63377516,54.55900896)(356.5937793,54.56900879)
\lineto(356.4587793,54.56900879)
\curveto(356.40877539,54.57900894)(356.36377543,54.58400894)(356.3237793,54.58400879)
\curveto(356.28377551,54.58400894)(356.24377555,54.58900893)(356.2037793,54.59900879)
\curveto(356.15377564,54.60900891)(356.0987757,54.6190089)(356.0387793,54.62900879)
\curveto(355.97877582,54.62900889)(355.92377587,54.63400889)(355.8737793,54.64400879)
\curveto(355.78377601,54.66400886)(355.6937761,54.68900883)(355.6037793,54.71900879)
\curveto(355.51377628,54.73900878)(355.42877637,54.76400876)(355.3487793,54.79400879)
\curveto(355.30877649,54.81400871)(355.27377652,54.8240087)(355.2437793,54.82400879)
\curveto(355.21377658,54.83400869)(355.17877662,54.84900867)(355.1387793,54.86900879)
\curveto(354.98877681,54.93900858)(354.82877697,55.0240085)(354.6587793,55.12400879)
\curveto(354.36877743,55.31400821)(354.11877768,55.54400798)(353.9087793,55.81400879)
\curveto(353.70877809,56.09400743)(353.53877826,56.40400712)(353.3987793,56.74400879)
\curveto(353.34877845,56.85400667)(353.30877849,56.96900655)(353.2787793,57.08900879)
\curveto(353.25877854,57.20900631)(353.22877857,57.32900619)(353.1887793,57.44900879)
\curveto(353.17877862,57.48900603)(353.17377862,57.524006)(353.1737793,57.55400879)
\curveto(353.17377862,57.58400594)(353.16877863,57.6240059)(353.1587793,57.67400879)
\curveto(353.13877866,57.75400577)(353.12377867,57.83900568)(353.1137793,57.92900879)
\curveto(353.10377869,58.0190055)(353.08877871,58.10900541)(353.0687793,58.19900879)
\lineto(353.0687793,58.40900879)
\curveto(353.05877874,58.44900507)(353.04877875,58.50400502)(353.0387793,58.57400879)
\curveto(353.03877876,58.65400487)(353.04377875,58.7190048)(353.0537793,58.76900879)
\lineto(353.0537793,58.93400879)
\curveto(353.07377872,58.98400454)(353.07877872,59.03400449)(353.0687793,59.08400879)
\curveto(353.06877873,59.14400438)(353.07377872,59.19900432)(353.0837793,59.24900879)
\curveto(353.12377867,59.40900411)(353.15377864,59.56900395)(353.1737793,59.72900879)
\curveto(353.20377859,59.88900363)(353.24877855,60.03900348)(353.3087793,60.17900879)
\curveto(353.35877844,60.28900323)(353.40377839,60.39900312)(353.4437793,60.50900879)
\curveto(353.4937783,60.62900289)(353.54877825,60.74400278)(353.6087793,60.85400879)
\curveto(353.82877797,61.20400232)(354.07877772,61.50400202)(354.3587793,61.75400879)
\curveto(354.63877716,62.01400151)(354.98377681,62.22900129)(355.3937793,62.39900879)
\curveto(355.51377628,62.44900107)(355.63377616,62.48400104)(355.7537793,62.50400879)
\curveto(355.88377591,62.53400099)(356.01877578,62.56400096)(356.1587793,62.59400879)
\curveto(356.20877559,62.60400092)(356.25377554,62.60900091)(356.2937793,62.60900879)
\curveto(356.33377546,62.6190009)(356.37877542,62.6240009)(356.4287793,62.62400879)
\curveto(356.44877535,62.63400089)(356.47377532,62.63400089)(356.5037793,62.62400879)
\curveto(356.53377526,62.61400091)(356.55877524,62.6190009)(356.5787793,62.63900879)
\curveto(356.9987748,62.64900087)(357.36377443,62.60400092)(357.6737793,62.50400879)
\curveto(357.98377381,62.41400111)(358.26377353,62.28900123)(358.5137793,62.12900879)
\curveto(358.56377323,62.10900141)(358.60377319,62.07900144)(358.6337793,62.03900879)
\curveto(358.66377313,62.00900151)(358.6987731,61.98400154)(358.7387793,61.96400879)
\curveto(358.81877298,61.90400162)(358.8987729,61.83400169)(358.9787793,61.75400879)
\curveto(359.06877273,61.67400185)(359.14377265,61.59400193)(359.2037793,61.51400879)
\curveto(359.36377243,61.30400222)(359.4987723,61.10400242)(359.6087793,60.91400879)
\curveto(359.67877212,60.80400272)(359.73377206,60.68400284)(359.7737793,60.55400879)
\curveto(359.81377198,60.4240031)(359.85877194,60.29400323)(359.9087793,60.16400879)
\curveto(359.95877184,60.03400349)(359.9937718,59.89900362)(360.0137793,59.75900879)
\curveto(360.04377175,59.6190039)(360.07877172,59.47900404)(360.1187793,59.33900879)
\curveto(360.12877167,59.26900425)(360.13377166,59.19900432)(360.1337793,59.12900879)
\lineto(360.1637793,58.91900879)
\moveto(358.7087793,59.42900879)
\curveto(358.73877306,59.46900405)(358.76377303,59.519004)(358.7837793,59.57900879)
\curveto(358.80377299,59.64900387)(358.80377299,59.7190038)(358.7837793,59.78900879)
\curveto(358.72377307,60.00900351)(358.63877316,60.21400331)(358.5287793,60.40400879)
\curveto(358.38877341,60.63400289)(358.23377356,60.82900269)(358.0637793,60.98900879)
\curveto(357.8937739,61.14900237)(357.67377412,61.28400224)(357.4037793,61.39400879)
\curveto(357.33377446,61.41400211)(357.26377453,61.42900209)(357.1937793,61.43900879)
\curveto(357.12377467,61.45900206)(357.04877475,61.47900204)(356.9687793,61.49900879)
\curveto(356.88877491,61.519002)(356.80377499,61.52900199)(356.7137793,61.52900879)
\lineto(356.4587793,61.52900879)
\curveto(356.42877537,61.50900201)(356.3937754,61.49900202)(356.3537793,61.49900879)
\curveto(356.31377548,61.50900201)(356.27877552,61.50900201)(356.2487793,61.49900879)
\lineto(356.0087793,61.43900879)
\curveto(355.93877586,61.42900209)(355.86877593,61.41400211)(355.7987793,61.39400879)
\curveto(355.50877629,61.27400225)(355.27377652,61.1240024)(355.0937793,60.94400879)
\curveto(354.92377687,60.76400276)(354.76877703,60.53900298)(354.6287793,60.26900879)
\curveto(354.5987772,60.2190033)(354.56877723,60.15400337)(354.5387793,60.07400879)
\curveto(354.50877729,60.00400352)(354.48377731,59.9240036)(354.4637793,59.83400879)
\curveto(354.44377735,59.74400378)(354.43877736,59.65900386)(354.4487793,59.57900879)
\curveto(354.45877734,59.49900402)(354.4937773,59.43900408)(354.5537793,59.39900879)
\curveto(354.63377716,59.33900418)(354.76877703,59.30900421)(354.9587793,59.30900879)
\curveto(355.15877664,59.3190042)(355.32877647,59.3240042)(355.4687793,59.32400879)
\lineto(357.7487793,59.32400879)
\curveto(357.8987739,59.3240042)(358.07877372,59.3190042)(358.2887793,59.30900879)
\curveto(358.4987733,59.30900421)(358.63877316,59.34900417)(358.7087793,59.42900879)
}
}
{
\newrgbcolor{curcolor}{0 0 0}
\pscustom[linestyle=none,fillstyle=solid,fillcolor=curcolor]
{
\newpath
\moveto(365.16041992,62.62400879)
\curveto(365.79041469,62.64400088)(366.29541418,62.55900096)(366.67541992,62.36900879)
\curveto(367.05541342,62.17900134)(367.36041312,61.89400163)(367.59041992,61.51400879)
\curveto(367.65041283,61.41400211)(367.69541278,61.30400222)(367.72541992,61.18400879)
\curveto(367.76541271,61.07400245)(367.80041268,60.95900256)(367.83041992,60.83900879)
\curveto(367.8804126,60.64900287)(367.91041257,60.44400308)(367.92041992,60.22400879)
\curveto(367.93041255,60.00400352)(367.93541254,59.77900374)(367.93541992,59.54900879)
\lineto(367.93541992,57.94400879)
\lineto(367.93541992,55.60400879)
\curveto(367.93541254,55.43400809)(367.93041255,55.26400826)(367.92041992,55.09400879)
\curveto(367.92041256,54.9240086)(367.85541262,54.81400871)(367.72541992,54.76400879)
\curveto(367.6754128,54.74400878)(367.62041286,54.73400879)(367.56041992,54.73400879)
\curveto(367.51041297,54.7240088)(367.45541302,54.7190088)(367.39541992,54.71900879)
\curveto(367.26541321,54.7190088)(367.14041334,54.7240088)(367.02041992,54.73400879)
\curveto(366.90041358,54.73400879)(366.81541366,54.77400875)(366.76541992,54.85400879)
\curveto(366.71541376,54.9240086)(366.69041379,55.01400851)(366.69041992,55.12400879)
\lineto(366.69041992,55.45400879)
\lineto(366.69041992,56.74400879)
\lineto(366.69041992,59.18900879)
\curveto(366.69041379,59.45900406)(366.68541379,59.7240038)(366.67541992,59.98400879)
\curveto(366.66541381,60.25400327)(366.62041386,60.48400304)(366.54041992,60.67400879)
\curveto(366.46041402,60.87400265)(366.34041414,61.03400249)(366.18041992,61.15400879)
\curveto(366.02041446,61.28400224)(365.83541464,61.38400214)(365.62541992,61.45400879)
\curveto(365.56541491,61.47400205)(365.50041498,61.48400204)(365.43041992,61.48400879)
\curveto(365.37041511,61.49400203)(365.31041517,61.50900201)(365.25041992,61.52900879)
\curveto(365.20041528,61.53900198)(365.12041536,61.53900198)(365.01041992,61.52900879)
\curveto(364.91041557,61.52900199)(364.84041564,61.524002)(364.80041992,61.51400879)
\curveto(364.76041572,61.49400203)(364.72541575,61.48400204)(364.69541992,61.48400879)
\curveto(364.66541581,61.49400203)(364.63041585,61.49400203)(364.59041992,61.48400879)
\curveto(364.46041602,61.45400207)(364.33541614,61.4190021)(364.21541992,61.37900879)
\curveto(364.10541637,61.34900217)(364.00041648,61.30400222)(363.90041992,61.24400879)
\curveto(363.86041662,61.2240023)(363.82541665,61.20400232)(363.79541992,61.18400879)
\curveto(363.76541671,61.16400236)(363.73041675,61.14400238)(363.69041992,61.12400879)
\curveto(363.34041714,60.87400265)(363.08541739,60.49900302)(362.92541992,59.99900879)
\curveto(362.89541758,59.9190036)(362.8754176,59.83400369)(362.86541992,59.74400879)
\curveto(362.85541762,59.66400386)(362.84041764,59.58400394)(362.82041992,59.50400879)
\curveto(362.80041768,59.45400407)(362.79541768,59.40400412)(362.80541992,59.35400879)
\curveto(362.81541766,59.31400421)(362.81041767,59.27400425)(362.79041992,59.23400879)
\lineto(362.79041992,58.91900879)
\curveto(362.7804177,58.88900463)(362.7754177,58.85400467)(362.77541992,58.81400879)
\curveto(362.78541769,58.77400475)(362.79041769,58.72900479)(362.79041992,58.67900879)
\lineto(362.79041992,58.22900879)
\lineto(362.79041992,56.78900879)
\lineto(362.79041992,55.46900879)
\lineto(362.79041992,55.12400879)
\curveto(362.79041769,55.01400851)(362.76541771,54.9240086)(362.71541992,54.85400879)
\curveto(362.66541781,54.77400875)(362.5754179,54.73400879)(362.44541992,54.73400879)
\curveto(362.32541815,54.7240088)(362.20041828,54.7190088)(362.07041992,54.71900879)
\curveto(361.99041849,54.7190088)(361.91541856,54.7240088)(361.84541992,54.73400879)
\curveto(361.7754187,54.74400878)(361.71541876,54.76900875)(361.66541992,54.80900879)
\curveto(361.58541889,54.85900866)(361.54541893,54.95400857)(361.54541992,55.09400879)
\lineto(361.54541992,55.49900879)
\lineto(361.54541992,57.26900879)
\lineto(361.54541992,60.89900879)
\lineto(361.54541992,61.81400879)
\lineto(361.54541992,62.08400879)
\curveto(361.54541893,62.17400135)(361.56541891,62.24400128)(361.60541992,62.29400879)
\curveto(361.63541884,62.35400117)(361.68541879,62.39400113)(361.75541992,62.41400879)
\curveto(361.79541868,62.4240011)(361.85041863,62.43400109)(361.92041992,62.44400879)
\curveto(362.00041848,62.45400107)(362.0804184,62.45900106)(362.16041992,62.45900879)
\curveto(362.24041824,62.45900106)(362.31541816,62.45400107)(362.38541992,62.44400879)
\curveto(362.46541801,62.43400109)(362.52041796,62.4190011)(362.55041992,62.39900879)
\curveto(362.66041782,62.32900119)(362.71041777,62.23900128)(362.70041992,62.12900879)
\curveto(362.69041779,62.02900149)(362.70541777,61.91400161)(362.74541992,61.78400879)
\curveto(362.76541771,61.7240018)(362.80541767,61.67400185)(362.86541992,61.63400879)
\curveto(362.98541749,61.6240019)(363.0804174,61.66900185)(363.15041992,61.76900879)
\curveto(363.23041725,61.86900165)(363.31041717,61.94900157)(363.39041992,62.00900879)
\curveto(363.53041695,62.10900141)(363.67041681,62.19900132)(363.81041992,62.27900879)
\curveto(363.96041652,62.36900115)(364.13041635,62.44400108)(364.32041992,62.50400879)
\curveto(364.40041608,62.53400099)(364.48541599,62.55400097)(364.57541992,62.56400879)
\curveto(364.6754158,62.57400095)(364.77041571,62.58900093)(364.86041992,62.60900879)
\curveto(364.91041557,62.6190009)(364.96041552,62.6240009)(365.01041992,62.62400879)
\lineto(365.16041992,62.62400879)
}
}
{
\newrgbcolor{curcolor}{0 0 0}
\pscustom[linestyle=none,fillstyle=solid,fillcolor=curcolor]
{
\newpath
\moveto(370.7650293,64.81400879)
\curveto(370.91502729,64.81399871)(371.06502714,64.80899871)(371.2150293,64.79900879)
\curveto(371.36502684,64.79899872)(371.47002673,64.75899876)(371.5300293,64.67900879)
\curveto(371.58002662,64.6189989)(371.6050266,64.53399899)(371.6050293,64.42400879)
\curveto(371.61502659,64.3239992)(371.62002658,64.2189993)(371.6200293,64.10900879)
\lineto(371.6200293,63.23900879)
\curveto(371.62002658,63.15900036)(371.61502659,63.07400045)(371.6050293,62.98400879)
\curveto(371.6050266,62.90400062)(371.61502659,62.83400069)(371.6350293,62.77400879)
\curveto(371.67502653,62.63400089)(371.76502644,62.54400098)(371.9050293,62.50400879)
\curveto(371.95502625,62.49400103)(372.0000262,62.48900103)(372.0400293,62.48900879)
\lineto(372.1900293,62.48900879)
\lineto(372.5950293,62.48900879)
\curveto(372.75502545,62.49900102)(372.87002533,62.48900103)(372.9400293,62.45900879)
\curveto(373.03002517,62.39900112)(373.09002511,62.33900118)(373.1200293,62.27900879)
\curveto(373.14002506,62.23900128)(373.15002505,62.19400133)(373.1500293,62.14400879)
\lineto(373.1500293,61.99400879)
\curveto(373.15002505,61.88400164)(373.14502506,61.77900174)(373.1350293,61.67900879)
\curveto(373.12502508,61.58900193)(373.09002511,61.519002)(373.0300293,61.46900879)
\curveto(372.97002523,61.4190021)(372.88502532,61.38900213)(372.7750293,61.37900879)
\lineto(372.4450293,61.37900879)
\curveto(372.33502587,61.38900213)(372.22502598,61.39400213)(372.1150293,61.39400879)
\curveto(372.0050262,61.39400213)(371.91002629,61.37900214)(371.8300293,61.34900879)
\curveto(371.76002644,61.3190022)(371.71002649,61.26900225)(371.6800293,61.19900879)
\curveto(371.65002655,61.12900239)(371.63002657,61.04400248)(371.6200293,60.94400879)
\curveto(371.61002659,60.85400267)(371.6050266,60.75400277)(371.6050293,60.64400879)
\curveto(371.61502659,60.54400298)(371.62002658,60.44400308)(371.6200293,60.34400879)
\lineto(371.6200293,57.37400879)
\curveto(371.62002658,57.15400637)(371.61502659,56.9190066)(371.6050293,56.66900879)
\curveto(371.6050266,56.42900709)(371.65002655,56.24400728)(371.7400293,56.11400879)
\curveto(371.79002641,56.03400749)(371.85502635,55.97900754)(371.9350293,55.94900879)
\curveto(372.01502619,55.9190076)(372.11002609,55.89400763)(372.2200293,55.87400879)
\curveto(372.25002595,55.86400766)(372.28002592,55.85900766)(372.3100293,55.85900879)
\curveto(372.35002585,55.86900765)(372.38502582,55.86900765)(372.4150293,55.85900879)
\lineto(372.6100293,55.85900879)
\curveto(372.71002549,55.85900766)(372.8000254,55.84900767)(372.8800293,55.82900879)
\curveto(372.97002523,55.8190077)(373.03502517,55.78400774)(373.0750293,55.72400879)
\curveto(373.09502511,55.69400783)(373.11002509,55.63900788)(373.1200293,55.55900879)
\curveto(373.14002506,55.48900803)(373.15002505,55.41400811)(373.1500293,55.33400879)
\curveto(373.16002504,55.25400827)(373.16002504,55.17400835)(373.1500293,55.09400879)
\curveto(373.14002506,55.0240085)(373.12002508,54.96900855)(373.0900293,54.92900879)
\curveto(373.05002515,54.85900866)(372.97502523,54.80900871)(372.8650293,54.77900879)
\curveto(372.78502542,54.75900876)(372.69502551,54.74900877)(372.5950293,54.74900879)
\curveto(372.49502571,54.75900876)(372.4050258,54.76400876)(372.3250293,54.76400879)
\curveto(372.26502594,54.76400876)(372.205026,54.75900876)(372.1450293,54.74900879)
\curveto(372.08502612,54.74900877)(372.03002617,54.75400877)(371.9800293,54.76400879)
\lineto(371.8000293,54.76400879)
\curveto(371.75002645,54.77400875)(371.7000265,54.77900874)(371.6500293,54.77900879)
\curveto(371.61002659,54.78900873)(371.56502664,54.79400873)(371.5150293,54.79400879)
\curveto(371.31502689,54.84400868)(371.14002706,54.89900862)(370.9900293,54.95900879)
\curveto(370.85002735,55.0190085)(370.73002747,55.1240084)(370.6300293,55.27400879)
\curveto(370.49002771,55.47400805)(370.41002779,55.7240078)(370.3900293,56.02400879)
\curveto(370.37002783,56.33400719)(370.36002784,56.66400686)(370.3600293,57.01400879)
\lineto(370.3600293,60.94400879)
\curveto(370.33002787,61.07400245)(370.3000279,61.16900235)(370.2700293,61.22900879)
\curveto(370.25002795,61.28900223)(370.18002802,61.33900218)(370.0600293,61.37900879)
\curveto(370.02002818,61.38900213)(369.98002822,61.38900213)(369.9400293,61.37900879)
\curveto(369.9000283,61.36900215)(369.86002834,61.37400215)(369.8200293,61.39400879)
\lineto(369.5800293,61.39400879)
\curveto(369.45002875,61.39400213)(369.34002886,61.40400212)(369.2500293,61.42400879)
\curveto(369.17002903,61.45400207)(369.11502909,61.51400201)(369.0850293,61.60400879)
\curveto(369.06502914,61.64400188)(369.05002915,61.68900183)(369.0400293,61.73900879)
\lineto(369.0400293,61.88900879)
\curveto(369.04002916,62.02900149)(369.05002915,62.14400138)(369.0700293,62.23400879)
\curveto(369.09002911,62.33400119)(369.15002905,62.40900111)(369.2500293,62.45900879)
\curveto(369.36002884,62.49900102)(369.5000287,62.50900101)(369.6700293,62.48900879)
\curveto(369.85002835,62.46900105)(370.0000282,62.47900104)(370.1200293,62.51900879)
\curveto(370.21002799,62.56900095)(370.28002792,62.63900088)(370.3300293,62.72900879)
\curveto(370.35002785,62.78900073)(370.36002784,62.86400066)(370.3600293,62.95400879)
\lineto(370.3600293,63.20900879)
\lineto(370.3600293,64.13900879)
\lineto(370.3600293,64.37900879)
\curveto(370.36002784,64.46899905)(370.37002783,64.54399898)(370.3900293,64.60400879)
\curveto(370.43002777,64.68399884)(370.5050277,64.74899877)(370.6150293,64.79900879)
\curveto(370.64502756,64.79899872)(370.67002753,64.79899872)(370.6900293,64.79900879)
\curveto(370.72002748,64.80899871)(370.74502746,64.81399871)(370.7650293,64.81400879)
}
}
{
\newrgbcolor{curcolor}{0 0 0}
\pscustom[linestyle=none,fillstyle=solid,fillcolor=curcolor]
{
\newpath
\moveto(381.66182617,58.94900879)
\curveto(381.68181811,58.88900463)(381.6918181,58.79400473)(381.69182617,58.66400879)
\curveto(381.6918181,58.54400498)(381.68681811,58.45900506)(381.67682617,58.40900879)
\lineto(381.67682617,58.25900879)
\curveto(381.66681813,58.17900534)(381.65681814,58.10400542)(381.64682617,58.03400879)
\curveto(381.64681815,57.97400555)(381.64181815,57.90400562)(381.63182617,57.82400879)
\curveto(381.61181818,57.76400576)(381.5968182,57.70400582)(381.58682617,57.64400879)
\curveto(381.58681821,57.58400594)(381.57681822,57.524006)(381.55682617,57.46400879)
\curveto(381.51681828,57.33400619)(381.48181831,57.20400632)(381.45182617,57.07400879)
\curveto(381.42181837,56.94400658)(381.38181841,56.8240067)(381.33182617,56.71400879)
\curveto(381.12181867,56.23400729)(380.84181895,55.82900769)(380.49182617,55.49900879)
\curveto(380.14181965,55.17900834)(379.71182008,54.93400859)(379.20182617,54.76400879)
\curveto(379.0918207,54.7240088)(378.97182082,54.69400883)(378.84182617,54.67400879)
\curveto(378.72182107,54.65400887)(378.5968212,54.63400889)(378.46682617,54.61400879)
\curveto(378.40682139,54.60400892)(378.34182145,54.59900892)(378.27182617,54.59900879)
\curveto(378.21182158,54.58900893)(378.15182164,54.58400894)(378.09182617,54.58400879)
\curveto(378.05182174,54.57400895)(377.9918218,54.56900895)(377.91182617,54.56900879)
\curveto(377.84182195,54.56900895)(377.791822,54.57400895)(377.76182617,54.58400879)
\curveto(377.72182207,54.59400893)(377.68182211,54.59900892)(377.64182617,54.59900879)
\curveto(377.60182219,54.58900893)(377.56682223,54.58900893)(377.53682617,54.59900879)
\lineto(377.44682617,54.59900879)
\lineto(377.08682617,54.64400879)
\curveto(376.94682285,54.68400884)(376.81182298,54.7240088)(376.68182617,54.76400879)
\curveto(376.55182324,54.80400872)(376.42682337,54.84900867)(376.30682617,54.89900879)
\curveto(375.85682394,55.09900842)(375.48682431,55.35900816)(375.19682617,55.67900879)
\curveto(374.90682489,55.99900752)(374.66682513,56.38900713)(374.47682617,56.84900879)
\curveto(374.42682537,56.94900657)(374.38682541,57.04900647)(374.35682617,57.14900879)
\curveto(374.33682546,57.24900627)(374.31682548,57.35400617)(374.29682617,57.46400879)
\curveto(374.27682552,57.50400602)(374.26682553,57.53400599)(374.26682617,57.55400879)
\curveto(374.27682552,57.58400594)(374.27682552,57.6190059)(374.26682617,57.65900879)
\curveto(374.24682555,57.73900578)(374.23182556,57.8190057)(374.22182617,57.89900879)
\curveto(374.22182557,57.98900553)(374.21182558,58.07400545)(374.19182617,58.15400879)
\lineto(374.19182617,58.27400879)
\curveto(374.1918256,58.31400521)(374.18682561,58.35900516)(374.17682617,58.40900879)
\curveto(374.16682563,58.45900506)(374.16182563,58.54400498)(374.16182617,58.66400879)
\curveto(374.16182563,58.79400473)(374.17182562,58.88900463)(374.19182617,58.94900879)
\curveto(374.21182558,59.0190045)(374.21682558,59.08900443)(374.20682617,59.15900879)
\curveto(374.1968256,59.22900429)(374.20182559,59.29900422)(374.22182617,59.36900879)
\curveto(374.23182556,59.4190041)(374.23682556,59.45900406)(374.23682617,59.48900879)
\curveto(374.24682555,59.52900399)(374.25682554,59.57400395)(374.26682617,59.62400879)
\curveto(374.2968255,59.74400378)(374.32182547,59.86400366)(374.34182617,59.98400879)
\curveto(374.37182542,60.10400342)(374.41182538,60.2190033)(374.46182617,60.32900879)
\curveto(374.61182518,60.69900282)(374.791825,61.02900249)(375.00182617,61.31900879)
\curveto(375.22182457,61.6190019)(375.48682431,61.86900165)(375.79682617,62.06900879)
\curveto(375.91682388,62.14900137)(376.04182375,62.21400131)(376.17182617,62.26400879)
\curveto(376.30182349,62.3240012)(376.43682336,62.38400114)(376.57682617,62.44400879)
\curveto(376.6968231,62.49400103)(376.82682297,62.524001)(376.96682617,62.53400879)
\curveto(377.10682269,62.55400097)(377.24682255,62.58400094)(377.38682617,62.62400879)
\lineto(377.58182617,62.62400879)
\curveto(377.65182214,62.63400089)(377.71682208,62.64400088)(377.77682617,62.65400879)
\curveto(378.66682113,62.66400086)(379.40682039,62.47900104)(379.99682617,62.09900879)
\curveto(380.58681921,61.7190018)(381.01181878,61.2240023)(381.27182617,60.61400879)
\curveto(381.32181847,60.51400301)(381.36181843,60.41400311)(381.39182617,60.31400879)
\curveto(381.42181837,60.21400331)(381.45681834,60.10900341)(381.49682617,59.99900879)
\curveto(381.52681827,59.88900363)(381.55181824,59.76900375)(381.57182617,59.63900879)
\curveto(381.5918182,59.519004)(381.61681818,59.39400413)(381.64682617,59.26400879)
\curveto(381.65681814,59.21400431)(381.65681814,59.15900436)(381.64682617,59.09900879)
\curveto(381.64681815,59.04900447)(381.65181814,58.99900452)(381.66182617,58.94900879)
\moveto(380.32682617,58.09400879)
\curveto(380.34681945,58.16400536)(380.35181944,58.24400528)(380.34182617,58.33400879)
\lineto(380.34182617,58.58900879)
\curveto(380.34181945,58.97900454)(380.30681949,59.30900421)(380.23682617,59.57900879)
\curveto(380.20681959,59.65900386)(380.18181961,59.73900378)(380.16182617,59.81900879)
\curveto(380.14181965,59.89900362)(380.11681968,59.97400355)(380.08682617,60.04400879)
\curveto(379.80681999,60.69400283)(379.36182043,61.14400238)(378.75182617,61.39400879)
\curveto(378.68182111,61.4240021)(378.60682119,61.44400208)(378.52682617,61.45400879)
\lineto(378.28682617,61.51400879)
\curveto(378.20682159,61.53400199)(378.12182167,61.54400198)(378.03182617,61.54400879)
\lineto(377.76182617,61.54400879)
\lineto(377.49182617,61.49900879)
\curveto(377.3918224,61.47900204)(377.2968225,61.45400207)(377.20682617,61.42400879)
\curveto(377.12682267,61.40400212)(377.04682275,61.37400215)(376.96682617,61.33400879)
\curveto(376.8968229,61.31400221)(376.83182296,61.28400224)(376.77182617,61.24400879)
\curveto(376.71182308,61.20400232)(376.65682314,61.16400236)(376.60682617,61.12400879)
\curveto(376.36682343,60.95400257)(376.17182362,60.74900277)(376.02182617,60.50900879)
\curveto(375.87182392,60.26900325)(375.74182405,59.98900353)(375.63182617,59.66900879)
\curveto(375.60182419,59.56900395)(375.58182421,59.46400406)(375.57182617,59.35400879)
\curveto(375.56182423,59.25400427)(375.54682425,59.14900437)(375.52682617,59.03900879)
\curveto(375.51682428,58.99900452)(375.51182428,58.93400459)(375.51182617,58.84400879)
\curveto(375.50182429,58.81400471)(375.4968243,58.77900474)(375.49682617,58.73900879)
\curveto(375.50682429,58.69900482)(375.51182428,58.65400487)(375.51182617,58.60400879)
\lineto(375.51182617,58.30400879)
\curveto(375.51182428,58.20400532)(375.52182427,58.11400541)(375.54182617,58.03400879)
\lineto(375.57182617,57.85400879)
\curveto(375.5918242,57.75400577)(375.60682419,57.65400587)(375.61682617,57.55400879)
\curveto(375.63682416,57.46400606)(375.66682413,57.37900614)(375.70682617,57.29900879)
\curveto(375.80682399,57.05900646)(375.92182387,56.83400669)(376.05182617,56.62400879)
\curveto(376.1918236,56.41400711)(376.36182343,56.23900728)(376.56182617,56.09900879)
\curveto(376.61182318,56.06900745)(376.65682314,56.04400748)(376.69682617,56.02400879)
\curveto(376.73682306,56.00400752)(376.78182301,55.97900754)(376.83182617,55.94900879)
\curveto(376.91182288,55.89900762)(376.9968228,55.85400767)(377.08682617,55.81400879)
\curveto(377.18682261,55.78400774)(377.2918225,55.75400777)(377.40182617,55.72400879)
\curveto(377.45182234,55.70400782)(377.4968223,55.69400783)(377.53682617,55.69400879)
\curveto(377.58682221,55.70400782)(377.63682216,55.70400782)(377.68682617,55.69400879)
\curveto(377.71682208,55.68400784)(377.77682202,55.67400785)(377.86682617,55.66400879)
\curveto(377.96682183,55.65400787)(378.04182175,55.65900786)(378.09182617,55.67900879)
\curveto(378.13182166,55.68900783)(378.17182162,55.68900783)(378.21182617,55.67900879)
\curveto(378.25182154,55.67900784)(378.2918215,55.68900783)(378.33182617,55.70900879)
\curveto(378.41182138,55.72900779)(378.4918213,55.74400778)(378.57182617,55.75400879)
\curveto(378.65182114,55.77400775)(378.72682107,55.79900772)(378.79682617,55.82900879)
\curveto(379.13682066,55.96900755)(379.41182038,56.16400736)(379.62182617,56.41400879)
\curveto(379.83181996,56.66400686)(380.00681979,56.95900656)(380.14682617,57.29900879)
\curveto(380.1968196,57.4190061)(380.22681957,57.54400598)(380.23682617,57.67400879)
\curveto(380.25681954,57.81400571)(380.28681951,57.95400557)(380.32682617,58.09400879)
}
}
{
\newrgbcolor{curcolor}{0 0 0}
\pscustom[linestyle=none,fillstyle=solid,fillcolor=curcolor]
{
\newpath
\moveto(385.58010742,62.65400879)
\curveto(386.30010336,62.66400086)(386.90510275,62.57900094)(387.39510742,62.39900879)
\curveto(387.88510177,62.22900129)(388.26510139,61.9240016)(388.53510742,61.48400879)
\curveto(388.60510105,61.37400215)(388.660101,61.25900226)(388.70010742,61.13900879)
\curveto(388.74010092,61.02900249)(388.78010088,60.90400262)(388.82010742,60.76400879)
\curveto(388.84010082,60.69400283)(388.84510081,60.6190029)(388.83510742,60.53900879)
\curveto(388.82510083,60.46900305)(388.81010085,60.41400311)(388.79010742,60.37400879)
\curveto(388.77010089,60.35400317)(388.74510091,60.33400319)(388.71510742,60.31400879)
\curveto(388.68510097,60.30400322)(388.660101,60.28900323)(388.64010742,60.26900879)
\curveto(388.59010107,60.24900327)(388.54010112,60.24400328)(388.49010742,60.25400879)
\curveto(388.44010122,60.26400326)(388.39010127,60.26400326)(388.34010742,60.25400879)
\curveto(388.2601014,60.23400329)(388.1551015,60.22900329)(388.02510742,60.23900879)
\curveto(387.89510176,60.25900326)(387.80510185,60.28400324)(387.75510742,60.31400879)
\curveto(387.67510198,60.36400316)(387.62010204,60.42900309)(387.59010742,60.50900879)
\curveto(387.57010209,60.59900292)(387.53510212,60.68400284)(387.48510742,60.76400879)
\curveto(387.39510226,60.9240026)(387.27010239,61.06900245)(387.11010742,61.19900879)
\curveto(387.00010266,61.27900224)(386.88010278,61.33900218)(386.75010742,61.37900879)
\curveto(386.62010304,61.4190021)(386.48010318,61.45900206)(386.33010742,61.49900879)
\curveto(386.28010338,61.519002)(386.23010343,61.524002)(386.18010742,61.51400879)
\curveto(386.13010353,61.51400201)(386.08010358,61.519002)(386.03010742,61.52900879)
\curveto(385.97010369,61.54900197)(385.89510376,61.55900196)(385.80510742,61.55900879)
\curveto(385.71510394,61.55900196)(385.64010402,61.54900197)(385.58010742,61.52900879)
\lineto(385.49010742,61.52900879)
\lineto(385.34010742,61.49900879)
\curveto(385.29010437,61.49900202)(385.24010442,61.49400203)(385.19010742,61.48400879)
\curveto(384.93010473,61.4240021)(384.71510494,61.33900218)(384.54510742,61.22900879)
\curveto(384.37510528,61.1190024)(384.2601054,60.93400259)(384.20010742,60.67400879)
\curveto(384.18010548,60.60400292)(384.17510548,60.53400299)(384.18510742,60.46400879)
\curveto(384.20510545,60.39400313)(384.22510543,60.33400319)(384.24510742,60.28400879)
\curveto(384.30510535,60.13400339)(384.37510528,60.0240035)(384.45510742,59.95400879)
\curveto(384.54510511,59.89400363)(384.655105,59.8240037)(384.78510742,59.74400879)
\curveto(384.94510471,59.64400388)(385.12510453,59.56900395)(385.32510742,59.51900879)
\curveto(385.52510413,59.47900404)(385.72510393,59.42900409)(385.92510742,59.36900879)
\curveto(386.0551036,59.32900419)(386.18510347,59.29900422)(386.31510742,59.27900879)
\curveto(386.44510321,59.25900426)(386.57510308,59.22900429)(386.70510742,59.18900879)
\curveto(386.91510274,59.12900439)(387.12010254,59.06900445)(387.32010742,59.00900879)
\curveto(387.52010214,58.95900456)(387.72010194,58.89400463)(387.92010742,58.81400879)
\lineto(388.07010742,58.75400879)
\curveto(388.12010154,58.73400479)(388.17010149,58.70900481)(388.22010742,58.67900879)
\curveto(388.42010124,58.55900496)(388.59510106,58.4240051)(388.74510742,58.27400879)
\curveto(388.89510076,58.1240054)(389.02010064,57.93400559)(389.12010742,57.70400879)
\curveto(389.14010052,57.63400589)(389.1601005,57.53900598)(389.18010742,57.41900879)
\curveto(389.20010046,57.34900617)(389.21010045,57.27400625)(389.21010742,57.19400879)
\curveto(389.22010044,57.1240064)(389.22510043,57.04400648)(389.22510742,56.95400879)
\lineto(389.22510742,56.80400879)
\curveto(389.20510045,56.73400679)(389.19510046,56.66400686)(389.19510742,56.59400879)
\curveto(389.19510046,56.524007)(389.18510047,56.45400707)(389.16510742,56.38400879)
\curveto(389.13510052,56.27400725)(389.10010056,56.16900735)(389.06010742,56.06900879)
\curveto(389.02010064,55.96900755)(388.97510068,55.87900764)(388.92510742,55.79900879)
\curveto(388.76510089,55.53900798)(388.5601011,55.32900819)(388.31010742,55.16900879)
\curveto(388.0601016,55.0190085)(387.78010188,54.88900863)(387.47010742,54.77900879)
\curveto(387.38010228,54.74900877)(387.28510237,54.72900879)(387.18510742,54.71900879)
\curveto(387.09510256,54.69900882)(387.00510265,54.67400885)(386.91510742,54.64400879)
\curveto(386.81510284,54.6240089)(386.71510294,54.61400891)(386.61510742,54.61400879)
\curveto(386.51510314,54.61400891)(386.41510324,54.60400892)(386.31510742,54.58400879)
\lineto(386.16510742,54.58400879)
\curveto(386.11510354,54.57400895)(386.04510361,54.56900895)(385.95510742,54.56900879)
\curveto(385.86510379,54.56900895)(385.79510386,54.57400895)(385.74510742,54.58400879)
\lineto(385.58010742,54.58400879)
\curveto(385.52010414,54.60400892)(385.4551042,54.61400891)(385.38510742,54.61400879)
\curveto(385.31510434,54.60400892)(385.2551044,54.60900891)(385.20510742,54.62900879)
\curveto(385.1551045,54.63900888)(385.09010457,54.64400888)(385.01010742,54.64400879)
\lineto(384.77010742,54.70400879)
\curveto(384.70010496,54.71400881)(384.62510503,54.73400879)(384.54510742,54.76400879)
\curveto(384.23510542,54.86400866)(383.96510569,54.98900853)(383.73510742,55.13900879)
\curveto(383.50510615,55.28900823)(383.30510635,55.48400804)(383.13510742,55.72400879)
\curveto(383.04510661,55.85400767)(382.97010669,55.98900753)(382.91010742,56.12900879)
\curveto(382.85010681,56.26900725)(382.79510686,56.4240071)(382.74510742,56.59400879)
\curveto(382.72510693,56.65400687)(382.71510694,56.7240068)(382.71510742,56.80400879)
\curveto(382.72510693,56.89400663)(382.74010692,56.96400656)(382.76010742,57.01400879)
\curveto(382.79010687,57.05400647)(382.84010682,57.09400643)(382.91010742,57.13400879)
\curveto(382.9601067,57.15400637)(383.03010663,57.16400636)(383.12010742,57.16400879)
\curveto(383.21010645,57.17400635)(383.30010636,57.17400635)(383.39010742,57.16400879)
\curveto(383.48010618,57.15400637)(383.56510609,57.13900638)(383.64510742,57.11900879)
\curveto(383.73510592,57.10900641)(383.79510586,57.09400643)(383.82510742,57.07400879)
\curveto(383.89510576,57.0240065)(383.94010572,56.94900657)(383.96010742,56.84900879)
\curveto(383.99010567,56.75900676)(384.02510563,56.67400685)(384.06510742,56.59400879)
\curveto(384.16510549,56.37400715)(384.30010536,56.20400732)(384.47010742,56.08400879)
\curveto(384.59010507,55.99400753)(384.72510493,55.9240076)(384.87510742,55.87400879)
\curveto(385.02510463,55.8240077)(385.18510447,55.77400775)(385.35510742,55.72400879)
\lineto(385.67010742,55.67900879)
\lineto(385.76010742,55.67900879)
\curveto(385.83010383,55.65900786)(385.92010374,55.64900787)(386.03010742,55.64900879)
\curveto(386.15010351,55.64900787)(386.25010341,55.65900786)(386.33010742,55.67900879)
\curveto(386.40010326,55.67900784)(386.4551032,55.68400784)(386.49510742,55.69400879)
\curveto(386.5551031,55.70400782)(386.61510304,55.70900781)(386.67510742,55.70900879)
\curveto(386.73510292,55.7190078)(386.79010287,55.72900779)(386.84010742,55.73900879)
\curveto(387.13010253,55.8190077)(387.3601023,55.9240076)(387.53010742,56.05400879)
\curveto(387.70010196,56.18400734)(387.82010184,56.40400712)(387.89010742,56.71400879)
\curveto(387.91010175,56.76400676)(387.91510174,56.8190067)(387.90510742,56.87900879)
\curveto(387.89510176,56.93900658)(387.88510177,56.98400654)(387.87510742,57.01400879)
\curveto(387.82510183,57.20400632)(387.7551019,57.34400618)(387.66510742,57.43400879)
\curveto(387.57510208,57.53400599)(387.4601022,57.6240059)(387.32010742,57.70400879)
\curveto(387.23010243,57.76400576)(387.13010253,57.81400571)(387.02010742,57.85400879)
\lineto(386.69010742,57.97400879)
\curveto(386.660103,57.98400554)(386.63010303,57.98900553)(386.60010742,57.98900879)
\curveto(386.58010308,57.98900553)(386.5551031,57.99900552)(386.52510742,58.01900879)
\curveto(386.18510347,58.12900539)(385.83010383,58.20900531)(385.46010742,58.25900879)
\curveto(385.10010456,58.3190052)(384.7601049,58.41400511)(384.44010742,58.54400879)
\curveto(384.34010532,58.58400494)(384.24510541,58.6190049)(384.15510742,58.64900879)
\curveto(384.06510559,58.67900484)(383.98010568,58.7190048)(383.90010742,58.76900879)
\curveto(383.71010595,58.87900464)(383.53510612,59.00400452)(383.37510742,59.14400879)
\curveto(383.21510644,59.28400424)(383.09010657,59.45900406)(383.00010742,59.66900879)
\curveto(382.97010669,59.73900378)(382.94510671,59.80900371)(382.92510742,59.87900879)
\curveto(382.91510674,59.94900357)(382.90010676,60.0240035)(382.88010742,60.10400879)
\curveto(382.85010681,60.2240033)(382.84010682,60.35900316)(382.85010742,60.50900879)
\curveto(382.8601068,60.66900285)(382.87510678,60.80400272)(382.89510742,60.91400879)
\curveto(382.91510674,60.96400256)(382.92510673,61.00400252)(382.92510742,61.03400879)
\curveto(382.93510672,61.07400245)(382.95010671,61.11400241)(382.97010742,61.15400879)
\curveto(383.0601066,61.38400214)(383.18010648,61.58400194)(383.33010742,61.75400879)
\curveto(383.49010617,61.9240016)(383.67010599,62.07400145)(383.87010742,62.20400879)
\curveto(384.02010564,62.29400123)(384.18510547,62.36400116)(384.36510742,62.41400879)
\curveto(384.54510511,62.47400105)(384.73510492,62.52900099)(384.93510742,62.57900879)
\curveto(385.00510465,62.58900093)(385.07010459,62.59900092)(385.13010742,62.60900879)
\curveto(385.20010446,62.6190009)(385.27510438,62.62900089)(385.35510742,62.63900879)
\curveto(385.38510427,62.64900087)(385.42510423,62.64900087)(385.47510742,62.63900879)
\curveto(385.52510413,62.62900089)(385.5601041,62.63400089)(385.58010742,62.65400879)
}
}
{
\newrgbcolor{curcolor}{0 0 0}
\pscustom[linestyle=none,fillstyle=solid,fillcolor=curcolor]
{
\newpath
\moveto(540.82596191,65.61150391)
\lineto(545.46096191,65.61150391)
\lineto(546.67596191,65.61150391)
\curveto(546.78595436,65.61149321)(546.89095426,65.61149321)(546.99096191,65.61150391)
\curveto(547.10095405,65.61149321)(547.18595396,65.59149323)(547.24596191,65.55150391)
\curveto(547.32595382,65.50149332)(547.37095378,65.4264934)(547.38096191,65.32650391)
\curveto(547.40095375,65.23649359)(547.41095374,65.1264937)(547.41096191,64.99650391)
\lineto(547.41096191,64.84650391)
\curveto(547.42095373,64.80649402)(547.41595373,64.76649406)(547.39596191,64.72650391)
\curveto(547.35595379,64.56649426)(547.26595388,64.47649435)(547.12596191,64.45650391)
\curveto(546.99595415,64.44649438)(546.83095432,64.44149438)(546.63096191,64.44150391)
\lineto(545.07096191,64.44150391)
\lineto(542.86596191,64.44150391)
\lineto(542.35596191,64.44150391)
\curveto(542.17595897,64.45149437)(542.04095911,64.4214944)(541.95096191,64.35150391)
\curveto(541.86095929,64.29149453)(541.81095934,64.18649464)(541.80096191,64.03650391)
\lineto(541.80096191,63.58650391)
\lineto(541.80096191,62.10150391)
\curveto(541.80095935,62.0214968)(541.79595935,61.9214969)(541.78596191,61.80150391)
\curveto(541.78595936,61.68149714)(541.79595935,61.58149724)(541.81596191,61.50150391)
\lineto(541.81596191,61.38150391)
\curveto(541.83595931,61.3214975)(541.8509593,61.26149756)(541.86096191,61.20150391)
\curveto(541.88095927,61.15149767)(541.91595923,61.11149771)(541.96596191,61.08150391)
\curveto(542.05595909,61.0214978)(542.19595895,60.99149783)(542.38596191,60.99150391)
\curveto(542.57595857,61.00149782)(542.74095841,61.00649782)(542.88096191,61.00650391)
\lineto(545.58096191,61.00650391)
\lineto(545.86596191,61.00650391)
\curveto(545.97595517,61.01649781)(546.08095507,61.01649781)(546.18096191,61.00650391)
\curveto(546.29095486,61.00649782)(546.38595476,60.99649783)(546.46596191,60.97650391)
\curveto(546.55595459,60.95649787)(546.61595453,60.9214979)(546.64596191,60.87150391)
\curveto(546.69595445,60.81149801)(546.72095443,60.73649809)(546.72096191,60.64650391)
\lineto(546.72096191,60.34650391)
\lineto(546.72096191,60.18150391)
\curveto(546.72095443,60.13149869)(546.71095444,60.08649874)(546.69096191,60.04650391)
\curveto(546.6509545,59.94649888)(546.59595455,59.89149893)(546.52596191,59.88150391)
\curveto(546.48595466,59.86149896)(546.4459547,59.85149897)(546.40596191,59.85150391)
\curveto(546.37595477,59.85149897)(546.33595481,59.84649898)(546.28596191,59.83650391)
\curveto(546.2459549,59.826499)(546.20095495,59.821499)(546.15096191,59.82150391)
\curveto(546.11095504,59.83149899)(546.07095508,59.83649899)(546.03096191,59.83650391)
\lineto(545.50596191,59.83650391)
\lineto(542.97096191,59.83650391)
\lineto(542.40096191,59.83650391)
\curveto(542.19095896,59.84649898)(542.04095911,59.81649901)(541.95096191,59.74650391)
\curveto(541.90095925,59.70649912)(541.86095929,59.64149918)(541.83096191,59.55150391)
\curveto(541.81095934,59.47149935)(541.79595935,59.37649945)(541.78596191,59.26650391)
\lineto(541.78596191,58.92150391)
\curveto(541.79595935,58.81150001)(541.80095935,58.71150011)(541.80096191,58.62150391)
\lineto(541.80096191,56.04150391)
\curveto(541.80095935,55.87150295)(541.80595934,55.68650314)(541.81596191,55.48650391)
\curveto(541.82595932,55.28650354)(541.79095936,55.13650369)(541.71096191,55.03650391)
\curveto(541.68095947,54.99650383)(541.63595951,54.97150385)(541.57596191,54.96150391)
\curveto(541.51595963,54.96150386)(541.45595969,54.95150387)(541.39596191,54.93150391)
\lineto(541.11096191,54.93150391)
\curveto(540.97096018,54.93150389)(540.84096031,54.93650389)(540.72096191,54.94650391)
\curveto(540.60096055,54.95650387)(540.51596063,55.00650382)(540.46596191,55.09650391)
\curveto(540.42596072,55.15650367)(540.40596074,55.23650359)(540.40596191,55.33650391)
\lineto(540.40596191,55.66650391)
\lineto(540.40596191,56.86650391)
\lineto(540.40596191,63.13650391)
\lineto(540.40596191,64.75650391)
\curveto(540.40596074,64.86649396)(540.40096075,64.98649384)(540.39096191,65.11650391)
\curveto(540.39096076,65.25649357)(540.41596073,65.36649346)(540.46596191,65.44650391)
\curveto(540.50596064,65.50649332)(540.58096057,65.55649327)(540.69096191,65.59650391)
\curveto(540.71096044,65.60649322)(540.73096042,65.60649322)(540.75096191,65.59650391)
\curveto(540.78096037,65.59649323)(540.80596034,65.60149322)(540.82596191,65.61150391)
}
}
{
\newrgbcolor{curcolor}{0 0 0}
\pscustom[linestyle=none,fillstyle=solid,fillcolor=curcolor]
{
\newpath
\moveto(555.88924316,59.13150391)
\curveto(555.9092351,59.07149975)(555.91923509,58.97649985)(555.91924316,58.84650391)
\curveto(555.91923509,58.7265001)(555.9142351,58.64150018)(555.90424316,58.59150391)
\lineto(555.90424316,58.44150391)
\curveto(555.89423512,58.36150046)(555.88423513,58.28650054)(555.87424316,58.21650391)
\curveto(555.87423514,58.15650067)(555.86923514,58.08650074)(555.85924316,58.00650391)
\curveto(555.83923517,57.94650088)(555.82423519,57.88650094)(555.81424316,57.82650391)
\curveto(555.8142352,57.76650106)(555.80423521,57.70650112)(555.78424316,57.64650391)
\curveto(555.74423527,57.51650131)(555.7092353,57.38650144)(555.67924316,57.25650391)
\curveto(555.64923536,57.1265017)(555.6092354,57.00650182)(555.55924316,56.89650391)
\curveto(555.34923566,56.41650241)(555.06923594,56.01150281)(554.71924316,55.68150391)
\curveto(554.36923664,55.36150346)(553.93923707,55.11650371)(553.42924316,54.94650391)
\curveto(553.31923769,54.90650392)(553.19923781,54.87650395)(553.06924316,54.85650391)
\curveto(552.94923806,54.83650399)(552.82423819,54.81650401)(552.69424316,54.79650391)
\curveto(552.63423838,54.78650404)(552.56923844,54.78150404)(552.49924316,54.78150391)
\curveto(552.43923857,54.77150405)(552.37923863,54.76650406)(552.31924316,54.76650391)
\curveto(552.27923873,54.75650407)(552.21923879,54.75150407)(552.13924316,54.75150391)
\curveto(552.06923894,54.75150407)(552.01923899,54.75650407)(551.98924316,54.76650391)
\curveto(551.94923906,54.77650405)(551.9092391,54.78150404)(551.86924316,54.78150391)
\curveto(551.82923918,54.77150405)(551.79423922,54.77150405)(551.76424316,54.78150391)
\lineto(551.67424316,54.78150391)
\lineto(551.31424316,54.82650391)
\curveto(551.17423984,54.86650396)(551.03923997,54.90650392)(550.90924316,54.94650391)
\curveto(550.77924023,54.98650384)(550.65424036,55.03150379)(550.53424316,55.08150391)
\curveto(550.08424093,55.28150354)(549.7142413,55.54150328)(549.42424316,55.86150391)
\curveto(549.13424188,56.18150264)(548.89424212,56.57150225)(548.70424316,57.03150391)
\curveto(548.65424236,57.13150169)(548.6142424,57.23150159)(548.58424316,57.33150391)
\curveto(548.56424245,57.43150139)(548.54424247,57.53650129)(548.52424316,57.64650391)
\curveto(548.50424251,57.68650114)(548.49424252,57.71650111)(548.49424316,57.73650391)
\curveto(548.50424251,57.76650106)(548.50424251,57.80150102)(548.49424316,57.84150391)
\curveto(548.47424254,57.9215009)(548.45924255,58.00150082)(548.44924316,58.08150391)
\curveto(548.44924256,58.17150065)(548.43924257,58.25650057)(548.41924316,58.33650391)
\lineto(548.41924316,58.45650391)
\curveto(548.41924259,58.49650033)(548.4142426,58.54150028)(548.40424316,58.59150391)
\curveto(548.39424262,58.64150018)(548.38924262,58.7265001)(548.38924316,58.84650391)
\curveto(548.38924262,58.97649985)(548.39924261,59.07149975)(548.41924316,59.13150391)
\curveto(548.43924257,59.20149962)(548.44424257,59.27149955)(548.43424316,59.34150391)
\curveto(548.42424259,59.41149941)(548.42924258,59.48149934)(548.44924316,59.55150391)
\curveto(548.45924255,59.60149922)(548.46424255,59.64149918)(548.46424316,59.67150391)
\curveto(548.47424254,59.71149911)(548.48424253,59.75649907)(548.49424316,59.80650391)
\curveto(548.52424249,59.9264989)(548.54924246,60.04649878)(548.56924316,60.16650391)
\curveto(548.59924241,60.28649854)(548.63924237,60.40149842)(548.68924316,60.51150391)
\curveto(548.83924217,60.88149794)(549.01924199,61.21149761)(549.22924316,61.50150391)
\curveto(549.44924156,61.80149702)(549.7142413,62.05149677)(550.02424316,62.25150391)
\curveto(550.14424087,62.33149649)(550.26924074,62.39649643)(550.39924316,62.44650391)
\curveto(550.52924048,62.50649632)(550.66424035,62.56649626)(550.80424316,62.62650391)
\curveto(550.92424009,62.67649615)(551.05423996,62.70649612)(551.19424316,62.71650391)
\curveto(551.33423968,62.73649609)(551.47423954,62.76649606)(551.61424316,62.80650391)
\lineto(551.80924316,62.80650391)
\curveto(551.87923913,62.81649601)(551.94423907,62.826496)(552.00424316,62.83650391)
\curveto(552.89423812,62.84649598)(553.63423738,62.66149616)(554.22424316,62.28150391)
\curveto(554.8142362,61.90149692)(555.23923577,61.40649742)(555.49924316,60.79650391)
\curveto(555.54923546,60.69649813)(555.58923542,60.59649823)(555.61924316,60.49650391)
\curveto(555.64923536,60.39649843)(555.68423533,60.29149853)(555.72424316,60.18150391)
\curveto(555.75423526,60.07149875)(555.77923523,59.95149887)(555.79924316,59.82150391)
\curveto(555.81923519,59.70149912)(555.84423517,59.57649925)(555.87424316,59.44650391)
\curveto(555.88423513,59.39649943)(555.88423513,59.34149948)(555.87424316,59.28150391)
\curveto(555.87423514,59.23149959)(555.87923513,59.18149964)(555.88924316,59.13150391)
\moveto(554.55424316,58.27650391)
\curveto(554.57423644,58.34650048)(554.57923643,58.4265004)(554.56924316,58.51650391)
\lineto(554.56924316,58.77150391)
\curveto(554.56923644,59.16149966)(554.53423648,59.49149933)(554.46424316,59.76150391)
\curveto(554.43423658,59.84149898)(554.4092366,59.9214989)(554.38924316,60.00150391)
\curveto(554.36923664,60.08149874)(554.34423667,60.15649867)(554.31424316,60.22650391)
\curveto(554.03423698,60.87649795)(553.58923742,61.3264975)(552.97924316,61.57650391)
\curveto(552.9092381,61.60649722)(552.83423818,61.6264972)(552.75424316,61.63650391)
\lineto(552.51424316,61.69650391)
\curveto(552.43423858,61.71649711)(552.34923866,61.7264971)(552.25924316,61.72650391)
\lineto(551.98924316,61.72650391)
\lineto(551.71924316,61.68150391)
\curveto(551.61923939,61.66149716)(551.52423949,61.63649719)(551.43424316,61.60650391)
\curveto(551.35423966,61.58649724)(551.27423974,61.55649727)(551.19424316,61.51650391)
\curveto(551.12423989,61.49649733)(551.05923995,61.46649736)(550.99924316,61.42650391)
\curveto(550.93924007,61.38649744)(550.88424013,61.34649748)(550.83424316,61.30650391)
\curveto(550.59424042,61.13649769)(550.39924061,60.93149789)(550.24924316,60.69150391)
\curveto(550.09924091,60.45149837)(549.96924104,60.17149865)(549.85924316,59.85150391)
\curveto(549.82924118,59.75149907)(549.8092412,59.64649918)(549.79924316,59.53650391)
\curveto(549.78924122,59.43649939)(549.77424124,59.33149949)(549.75424316,59.22150391)
\curveto(549.74424127,59.18149964)(549.73924127,59.11649971)(549.73924316,59.02650391)
\curveto(549.72924128,58.99649983)(549.72424129,58.96149986)(549.72424316,58.92150391)
\curveto(549.73424128,58.88149994)(549.73924127,58.83649999)(549.73924316,58.78650391)
\lineto(549.73924316,58.48650391)
\curveto(549.73924127,58.38650044)(549.74924126,58.29650053)(549.76924316,58.21650391)
\lineto(549.79924316,58.03650391)
\curveto(549.81924119,57.93650089)(549.83424118,57.83650099)(549.84424316,57.73650391)
\curveto(549.86424115,57.64650118)(549.89424112,57.56150126)(549.93424316,57.48150391)
\curveto(550.03424098,57.24150158)(550.14924086,57.01650181)(550.27924316,56.80650391)
\curveto(550.41924059,56.59650223)(550.58924042,56.4215024)(550.78924316,56.28150391)
\curveto(550.83924017,56.25150257)(550.88424013,56.2265026)(550.92424316,56.20650391)
\curveto(550.96424005,56.18650264)(551.00924,56.16150266)(551.05924316,56.13150391)
\curveto(551.13923987,56.08150274)(551.22423979,56.03650279)(551.31424316,55.99650391)
\curveto(551.4142396,55.96650286)(551.51923949,55.93650289)(551.62924316,55.90650391)
\curveto(551.67923933,55.88650294)(551.72423929,55.87650295)(551.76424316,55.87650391)
\curveto(551.8142392,55.88650294)(551.86423915,55.88650294)(551.91424316,55.87650391)
\curveto(551.94423907,55.86650296)(552.00423901,55.85650297)(552.09424316,55.84650391)
\curveto(552.19423882,55.83650299)(552.26923874,55.84150298)(552.31924316,55.86150391)
\curveto(552.35923865,55.87150295)(552.39923861,55.87150295)(552.43924316,55.86150391)
\curveto(552.47923853,55.86150296)(552.51923849,55.87150295)(552.55924316,55.89150391)
\curveto(552.63923837,55.91150291)(552.71923829,55.9265029)(552.79924316,55.93650391)
\curveto(552.87923813,55.95650287)(552.95423806,55.98150284)(553.02424316,56.01150391)
\curveto(553.36423765,56.15150267)(553.63923737,56.34650248)(553.84924316,56.59650391)
\curveto(554.05923695,56.84650198)(554.23423678,57.14150168)(554.37424316,57.48150391)
\curveto(554.42423659,57.60150122)(554.45423656,57.7265011)(554.46424316,57.85650391)
\curveto(554.48423653,57.99650083)(554.5142365,58.13650069)(554.55424316,58.27650391)
}
}
{
\newrgbcolor{curcolor}{0 0 0}
\pscustom[linestyle=none,fillstyle=solid,fillcolor=curcolor]
{
\newpath
\moveto(558.32252441,64.99650391)
\curveto(558.4725224,64.99649383)(558.62252225,64.99149383)(558.77252441,64.98150391)
\curveto(558.92252195,64.98149384)(559.02752185,64.94149388)(559.08752441,64.86150391)
\curveto(559.13752174,64.80149402)(559.16252171,64.71649411)(559.16252441,64.60650391)
\curveto(559.1725217,64.50649432)(559.1775217,64.40149442)(559.17752441,64.29150391)
\lineto(559.17752441,63.42150391)
\curveto(559.1775217,63.34149548)(559.1725217,63.25649557)(559.16252441,63.16650391)
\curveto(559.16252171,63.08649574)(559.1725217,63.01649581)(559.19252441,62.95650391)
\curveto(559.23252164,62.81649601)(559.32252155,62.7264961)(559.46252441,62.68650391)
\curveto(559.51252136,62.67649615)(559.55752132,62.67149615)(559.59752441,62.67150391)
\lineto(559.74752441,62.67150391)
\lineto(560.15252441,62.67150391)
\curveto(560.31252056,62.68149614)(560.42752045,62.67149615)(560.49752441,62.64150391)
\curveto(560.58752029,62.58149624)(560.64752023,62.5214963)(560.67752441,62.46150391)
\curveto(560.69752018,62.4214964)(560.70752017,62.37649645)(560.70752441,62.32650391)
\lineto(560.70752441,62.17650391)
\curveto(560.70752017,62.06649676)(560.70252017,61.96149686)(560.69252441,61.86150391)
\curveto(560.68252019,61.77149705)(560.64752023,61.70149712)(560.58752441,61.65150391)
\curveto(560.52752035,61.60149722)(560.44252043,61.57149725)(560.33252441,61.56150391)
\lineto(560.00252441,61.56150391)
\curveto(559.89252098,61.57149725)(559.78252109,61.57649725)(559.67252441,61.57650391)
\curveto(559.56252131,61.57649725)(559.46752141,61.56149726)(559.38752441,61.53150391)
\curveto(559.31752156,61.50149732)(559.26752161,61.45149737)(559.23752441,61.38150391)
\curveto(559.20752167,61.31149751)(559.18752169,61.2264976)(559.17752441,61.12650391)
\curveto(559.16752171,61.03649779)(559.16252171,60.93649789)(559.16252441,60.82650391)
\curveto(559.1725217,60.7264981)(559.1775217,60.6264982)(559.17752441,60.52650391)
\lineto(559.17752441,57.55650391)
\curveto(559.1775217,57.33650149)(559.1725217,57.10150172)(559.16252441,56.85150391)
\curveto(559.16252171,56.61150221)(559.20752167,56.4265024)(559.29752441,56.29650391)
\curveto(559.34752153,56.21650261)(559.41252146,56.16150266)(559.49252441,56.13150391)
\curveto(559.5725213,56.10150272)(559.66752121,56.07650275)(559.77752441,56.05650391)
\curveto(559.80752107,56.04650278)(559.83752104,56.04150278)(559.86752441,56.04150391)
\curveto(559.90752097,56.05150277)(559.94252093,56.05150277)(559.97252441,56.04150391)
\lineto(560.16752441,56.04150391)
\curveto(560.26752061,56.04150278)(560.35752052,56.03150279)(560.43752441,56.01150391)
\curveto(560.52752035,56.00150282)(560.59252028,55.96650286)(560.63252441,55.90650391)
\curveto(560.65252022,55.87650295)(560.66752021,55.821503)(560.67752441,55.74150391)
\curveto(560.69752018,55.67150315)(560.70752017,55.59650323)(560.70752441,55.51650391)
\curveto(560.71752016,55.43650339)(560.71752016,55.35650347)(560.70752441,55.27650391)
\curveto(560.69752018,55.20650362)(560.6775202,55.15150367)(560.64752441,55.11150391)
\curveto(560.60752027,55.04150378)(560.53252034,54.99150383)(560.42252441,54.96150391)
\curveto(560.34252053,54.94150388)(560.25252062,54.93150389)(560.15252441,54.93150391)
\curveto(560.05252082,54.94150388)(559.96252091,54.94650388)(559.88252441,54.94650391)
\curveto(559.82252105,54.94650388)(559.76252111,54.94150388)(559.70252441,54.93150391)
\curveto(559.64252123,54.93150389)(559.58752129,54.93650389)(559.53752441,54.94650391)
\lineto(559.35752441,54.94650391)
\curveto(559.30752157,54.95650387)(559.25752162,54.96150386)(559.20752441,54.96150391)
\curveto(559.16752171,54.97150385)(559.12252175,54.97650385)(559.07252441,54.97650391)
\curveto(558.872522,55.0265038)(558.69752218,55.08150374)(558.54752441,55.14150391)
\curveto(558.40752247,55.20150362)(558.28752259,55.30650352)(558.18752441,55.45650391)
\curveto(558.04752283,55.65650317)(557.96752291,55.90650292)(557.94752441,56.20650391)
\curveto(557.92752295,56.51650231)(557.91752296,56.84650198)(557.91752441,57.19650391)
\lineto(557.91752441,61.12650391)
\curveto(557.88752299,61.25649757)(557.85752302,61.35149747)(557.82752441,61.41150391)
\curveto(557.80752307,61.47149735)(557.73752314,61.5214973)(557.61752441,61.56150391)
\curveto(557.5775233,61.57149725)(557.53752334,61.57149725)(557.49752441,61.56150391)
\curveto(557.45752342,61.55149727)(557.41752346,61.55649727)(557.37752441,61.57650391)
\lineto(557.13752441,61.57650391)
\curveto(557.00752387,61.57649725)(556.89752398,61.58649724)(556.80752441,61.60650391)
\curveto(556.72752415,61.63649719)(556.6725242,61.69649713)(556.64252441,61.78650391)
\curveto(556.62252425,61.826497)(556.60752427,61.87149695)(556.59752441,61.92150391)
\lineto(556.59752441,62.07150391)
\curveto(556.59752428,62.21149661)(556.60752427,62.3264965)(556.62752441,62.41650391)
\curveto(556.64752423,62.51649631)(556.70752417,62.59149623)(556.80752441,62.64150391)
\curveto(556.91752396,62.68149614)(557.05752382,62.69149613)(557.22752441,62.67150391)
\curveto(557.40752347,62.65149617)(557.55752332,62.66149616)(557.67752441,62.70150391)
\curveto(557.76752311,62.75149607)(557.83752304,62.821496)(557.88752441,62.91150391)
\curveto(557.90752297,62.97149585)(557.91752296,63.04649578)(557.91752441,63.13650391)
\lineto(557.91752441,63.39150391)
\lineto(557.91752441,64.32150391)
\lineto(557.91752441,64.56150391)
\curveto(557.91752296,64.65149417)(557.92752295,64.7264941)(557.94752441,64.78650391)
\curveto(557.98752289,64.86649396)(558.06252281,64.93149389)(558.17252441,64.98150391)
\curveto(558.20252267,64.98149384)(558.22752265,64.98149384)(558.24752441,64.98150391)
\curveto(558.2775226,64.99149383)(558.30252257,64.99649383)(558.32252441,64.99650391)
}
}
{
\newrgbcolor{curcolor}{0 0 0}
\pscustom[linestyle=none,fillstyle=solid,fillcolor=curcolor]
{
\newpath
\moveto(569.21932129,59.13150391)
\curveto(569.23931323,59.07149975)(569.24931322,58.97649985)(569.24932129,58.84650391)
\curveto(569.24931322,58.7265001)(569.24431322,58.64150018)(569.23432129,58.59150391)
\lineto(569.23432129,58.44150391)
\curveto(569.22431324,58.36150046)(569.21431325,58.28650054)(569.20432129,58.21650391)
\curveto(569.20431326,58.15650067)(569.19931327,58.08650074)(569.18932129,58.00650391)
\curveto(569.1693133,57.94650088)(569.15431331,57.88650094)(569.14432129,57.82650391)
\curveto(569.14431332,57.76650106)(569.13431333,57.70650112)(569.11432129,57.64650391)
\curveto(569.07431339,57.51650131)(569.03931343,57.38650144)(569.00932129,57.25650391)
\curveto(568.97931349,57.1265017)(568.93931353,57.00650182)(568.88932129,56.89650391)
\curveto(568.67931379,56.41650241)(568.39931407,56.01150281)(568.04932129,55.68150391)
\curveto(567.69931477,55.36150346)(567.2693152,55.11650371)(566.75932129,54.94650391)
\curveto(566.64931582,54.90650392)(566.52931594,54.87650395)(566.39932129,54.85650391)
\curveto(566.27931619,54.83650399)(566.15431631,54.81650401)(566.02432129,54.79650391)
\curveto(565.9643165,54.78650404)(565.89931657,54.78150404)(565.82932129,54.78150391)
\curveto(565.7693167,54.77150405)(565.70931676,54.76650406)(565.64932129,54.76650391)
\curveto(565.60931686,54.75650407)(565.54931692,54.75150407)(565.46932129,54.75150391)
\curveto(565.39931707,54.75150407)(565.34931712,54.75650407)(565.31932129,54.76650391)
\curveto(565.27931719,54.77650405)(565.23931723,54.78150404)(565.19932129,54.78150391)
\curveto(565.15931731,54.77150405)(565.12431734,54.77150405)(565.09432129,54.78150391)
\lineto(565.00432129,54.78150391)
\lineto(564.64432129,54.82650391)
\curveto(564.50431796,54.86650396)(564.3693181,54.90650392)(564.23932129,54.94650391)
\curveto(564.10931836,54.98650384)(563.98431848,55.03150379)(563.86432129,55.08150391)
\curveto(563.41431905,55.28150354)(563.04431942,55.54150328)(562.75432129,55.86150391)
\curveto(562.46432,56.18150264)(562.22432024,56.57150225)(562.03432129,57.03150391)
\curveto(561.98432048,57.13150169)(561.94432052,57.23150159)(561.91432129,57.33150391)
\curveto(561.89432057,57.43150139)(561.87432059,57.53650129)(561.85432129,57.64650391)
\curveto(561.83432063,57.68650114)(561.82432064,57.71650111)(561.82432129,57.73650391)
\curveto(561.83432063,57.76650106)(561.83432063,57.80150102)(561.82432129,57.84150391)
\curveto(561.80432066,57.9215009)(561.78932068,58.00150082)(561.77932129,58.08150391)
\curveto(561.77932069,58.17150065)(561.7693207,58.25650057)(561.74932129,58.33650391)
\lineto(561.74932129,58.45650391)
\curveto(561.74932072,58.49650033)(561.74432072,58.54150028)(561.73432129,58.59150391)
\curveto(561.72432074,58.64150018)(561.71932075,58.7265001)(561.71932129,58.84650391)
\curveto(561.71932075,58.97649985)(561.72932074,59.07149975)(561.74932129,59.13150391)
\curveto(561.7693207,59.20149962)(561.77432069,59.27149955)(561.76432129,59.34150391)
\curveto(561.75432071,59.41149941)(561.75932071,59.48149934)(561.77932129,59.55150391)
\curveto(561.78932068,59.60149922)(561.79432067,59.64149918)(561.79432129,59.67150391)
\curveto(561.80432066,59.71149911)(561.81432065,59.75649907)(561.82432129,59.80650391)
\curveto(561.85432061,59.9264989)(561.87932059,60.04649878)(561.89932129,60.16650391)
\curveto(561.92932054,60.28649854)(561.9693205,60.40149842)(562.01932129,60.51150391)
\curveto(562.1693203,60.88149794)(562.34932012,61.21149761)(562.55932129,61.50150391)
\curveto(562.77931969,61.80149702)(563.04431942,62.05149677)(563.35432129,62.25150391)
\curveto(563.47431899,62.33149649)(563.59931887,62.39649643)(563.72932129,62.44650391)
\curveto(563.85931861,62.50649632)(563.99431847,62.56649626)(564.13432129,62.62650391)
\curveto(564.25431821,62.67649615)(564.38431808,62.70649612)(564.52432129,62.71650391)
\curveto(564.6643178,62.73649609)(564.80431766,62.76649606)(564.94432129,62.80650391)
\lineto(565.13932129,62.80650391)
\curveto(565.20931726,62.81649601)(565.27431719,62.826496)(565.33432129,62.83650391)
\curveto(566.22431624,62.84649598)(566.9643155,62.66149616)(567.55432129,62.28150391)
\curveto(568.14431432,61.90149692)(568.5693139,61.40649742)(568.82932129,60.79650391)
\curveto(568.87931359,60.69649813)(568.91931355,60.59649823)(568.94932129,60.49650391)
\curveto(568.97931349,60.39649843)(569.01431345,60.29149853)(569.05432129,60.18150391)
\curveto(569.08431338,60.07149875)(569.10931336,59.95149887)(569.12932129,59.82150391)
\curveto(569.14931332,59.70149912)(569.17431329,59.57649925)(569.20432129,59.44650391)
\curveto(569.21431325,59.39649943)(569.21431325,59.34149948)(569.20432129,59.28150391)
\curveto(569.20431326,59.23149959)(569.20931326,59.18149964)(569.21932129,59.13150391)
\moveto(567.88432129,58.27650391)
\curveto(567.90431456,58.34650048)(567.90931456,58.4265004)(567.89932129,58.51650391)
\lineto(567.89932129,58.77150391)
\curveto(567.89931457,59.16149966)(567.8643146,59.49149933)(567.79432129,59.76150391)
\curveto(567.7643147,59.84149898)(567.73931473,59.9214989)(567.71932129,60.00150391)
\curveto(567.69931477,60.08149874)(567.67431479,60.15649867)(567.64432129,60.22650391)
\curveto(567.3643151,60.87649795)(566.91931555,61.3264975)(566.30932129,61.57650391)
\curveto(566.23931623,61.60649722)(566.1643163,61.6264972)(566.08432129,61.63650391)
\lineto(565.84432129,61.69650391)
\curveto(565.7643167,61.71649711)(565.67931679,61.7264971)(565.58932129,61.72650391)
\lineto(565.31932129,61.72650391)
\lineto(565.04932129,61.68150391)
\curveto(564.94931752,61.66149716)(564.85431761,61.63649719)(564.76432129,61.60650391)
\curveto(564.68431778,61.58649724)(564.60431786,61.55649727)(564.52432129,61.51650391)
\curveto(564.45431801,61.49649733)(564.38931808,61.46649736)(564.32932129,61.42650391)
\curveto(564.2693182,61.38649744)(564.21431825,61.34649748)(564.16432129,61.30650391)
\curveto(563.92431854,61.13649769)(563.72931874,60.93149789)(563.57932129,60.69150391)
\curveto(563.42931904,60.45149837)(563.29931917,60.17149865)(563.18932129,59.85150391)
\curveto(563.15931931,59.75149907)(563.13931933,59.64649918)(563.12932129,59.53650391)
\curveto(563.11931935,59.43649939)(563.10431936,59.33149949)(563.08432129,59.22150391)
\curveto(563.07431939,59.18149964)(563.0693194,59.11649971)(563.06932129,59.02650391)
\curveto(563.05931941,58.99649983)(563.05431941,58.96149986)(563.05432129,58.92150391)
\curveto(563.0643194,58.88149994)(563.0693194,58.83649999)(563.06932129,58.78650391)
\lineto(563.06932129,58.48650391)
\curveto(563.0693194,58.38650044)(563.07931939,58.29650053)(563.09932129,58.21650391)
\lineto(563.12932129,58.03650391)
\curveto(563.14931932,57.93650089)(563.1643193,57.83650099)(563.17432129,57.73650391)
\curveto(563.19431927,57.64650118)(563.22431924,57.56150126)(563.26432129,57.48150391)
\curveto(563.3643191,57.24150158)(563.47931899,57.01650181)(563.60932129,56.80650391)
\curveto(563.74931872,56.59650223)(563.91931855,56.4215024)(564.11932129,56.28150391)
\curveto(564.1693183,56.25150257)(564.21431825,56.2265026)(564.25432129,56.20650391)
\curveto(564.29431817,56.18650264)(564.33931813,56.16150266)(564.38932129,56.13150391)
\curveto(564.469318,56.08150274)(564.55431791,56.03650279)(564.64432129,55.99650391)
\curveto(564.74431772,55.96650286)(564.84931762,55.93650289)(564.95932129,55.90650391)
\curveto(565.00931746,55.88650294)(565.05431741,55.87650295)(565.09432129,55.87650391)
\curveto(565.14431732,55.88650294)(565.19431727,55.88650294)(565.24432129,55.87650391)
\curveto(565.27431719,55.86650296)(565.33431713,55.85650297)(565.42432129,55.84650391)
\curveto(565.52431694,55.83650299)(565.59931687,55.84150298)(565.64932129,55.86150391)
\curveto(565.68931678,55.87150295)(565.72931674,55.87150295)(565.76932129,55.86150391)
\curveto(565.80931666,55.86150296)(565.84931662,55.87150295)(565.88932129,55.89150391)
\curveto(565.9693165,55.91150291)(566.04931642,55.9265029)(566.12932129,55.93650391)
\curveto(566.20931626,55.95650287)(566.28431618,55.98150284)(566.35432129,56.01150391)
\curveto(566.69431577,56.15150267)(566.9693155,56.34650248)(567.17932129,56.59650391)
\curveto(567.38931508,56.84650198)(567.5643149,57.14150168)(567.70432129,57.48150391)
\curveto(567.75431471,57.60150122)(567.78431468,57.7265011)(567.79432129,57.85650391)
\curveto(567.81431465,57.99650083)(567.84431462,58.13650069)(567.88432129,58.27650391)
}
}
{
\newrgbcolor{curcolor}{0 0 0}
\pscustom[linestyle=none,fillstyle=solid,fillcolor=curcolor]
{
\newpath
\moveto(577.32260254,62.55150391)
\curveto(577.39259494,62.50149632)(577.4275949,62.4264964)(577.42760254,62.32650391)
\curveto(577.43759489,62.2264966)(577.44259489,62.1214967)(577.44260254,62.01150391)
\lineto(577.44260254,55.74150391)
\lineto(577.44260254,55.14150391)
\curveto(577.42259491,55.09150373)(577.41759491,55.04150378)(577.42760254,54.99150391)
\curveto(577.43759489,54.95150387)(577.4325949,54.90650392)(577.41260254,54.85650391)
\curveto(577.39259494,54.75650407)(577.37759495,54.65650417)(577.36760254,54.55650391)
\curveto(577.36759496,54.44650438)(577.35259498,54.34150448)(577.32260254,54.24150391)
\curveto(577.29259504,54.13150469)(577.26259507,54.0265048)(577.23260254,53.92650391)
\curveto(577.21259512,53.826505)(577.17759515,53.7265051)(577.12760254,53.62650391)
\curveto(577.0275953,53.36650546)(576.89759543,53.13150569)(576.73760254,52.92150391)
\curveto(576.58759574,52.71150611)(576.40759592,52.53650629)(576.19760254,52.39650391)
\curveto(576.0275963,52.27650655)(575.84759648,52.18150664)(575.65760254,52.11150391)
\curveto(575.46759686,52.03150679)(575.26259707,51.95650687)(575.04260254,51.88650391)
\curveto(574.95259738,51.86650696)(574.86259747,51.85650697)(574.77260254,51.85650391)
\curveto(574.68259765,51.84650698)(574.59259774,51.83150699)(574.50260254,51.81150391)
\lineto(574.41260254,51.81150391)
\curveto(574.39259794,51.80150702)(574.37259796,51.79650703)(574.35260254,51.79650391)
\curveto(574.30259803,51.78650704)(574.25259808,51.78650704)(574.20260254,51.79650391)
\curveto(574.16259817,51.80650702)(574.11759821,51.80150702)(574.06760254,51.78150391)
\curveto(573.99759833,51.76150706)(573.88759844,51.75650707)(573.73760254,51.76650391)
\curveto(573.59759873,51.76650706)(573.49759883,51.77650705)(573.43760254,51.79650391)
\curveto(573.40759892,51.79650703)(573.37759895,51.80150702)(573.34760254,51.81150391)
\lineto(573.28760254,51.81150391)
\curveto(573.19759913,51.83150699)(573.10759922,51.84650698)(573.01760254,51.85650391)
\curveto(572.9275994,51.85650697)(572.84259949,51.86650696)(572.76260254,51.88650391)
\curveto(572.68259965,51.90650692)(572.60259973,51.93150689)(572.52260254,51.96150391)
\curveto(572.44259989,51.98150684)(572.36259997,52.00650682)(572.28260254,52.03650391)
\curveto(571.96260037,52.16650666)(571.69260064,52.31150651)(571.47260254,52.47150391)
\curveto(571.26260107,52.63150619)(571.07260126,52.85650597)(570.90260254,53.14650391)
\curveto(570.88260145,53.16650566)(570.86760146,53.19150563)(570.85760254,53.22150391)
\curveto(570.85760147,53.24150558)(570.84760148,53.26650556)(570.82760254,53.29650391)
\curveto(570.79760153,53.37650545)(570.76260157,53.49150533)(570.72260254,53.64150391)
\curveto(570.69260164,53.78150504)(570.72260161,53.88650494)(570.81260254,53.95650391)
\curveto(570.87260146,54.00650482)(570.95260138,54.03150479)(571.05260254,54.03150391)
\lineto(571.38260254,54.03150391)
\lineto(571.54760254,54.03150391)
\curveto(571.60760072,54.03150479)(571.66260067,54.0215048)(571.71260254,54.00150391)
\curveto(571.80260053,53.97150485)(571.86760046,53.9215049)(571.90760254,53.85150391)
\curveto(571.94760038,53.78150504)(571.99260034,53.70650512)(572.04260254,53.62650391)
\lineto(572.16260254,53.44650391)
\curveto(572.21260012,53.37650545)(572.26260007,53.3215055)(572.31260254,53.28150391)
\curveto(572.56259977,53.09150573)(572.86259947,52.95150587)(573.21260254,52.86150391)
\curveto(573.27259906,52.84150598)(573.332599,52.83150599)(573.39260254,52.83150391)
\curveto(573.46259887,52.821506)(573.5275988,52.80650602)(573.58760254,52.78650391)
\lineto(573.67760254,52.78650391)
\curveto(573.74759858,52.76650606)(573.8325985,52.75650607)(573.93260254,52.75650391)
\curveto(574.0325983,52.75650607)(574.12259821,52.76650606)(574.20260254,52.78650391)
\curveto(574.2325981,52.79650603)(574.27259806,52.80150602)(574.32260254,52.80150391)
\curveto(574.42259791,52.821506)(574.51759781,52.84150598)(574.60760254,52.86150391)
\curveto(574.69759763,52.87150595)(574.78259755,52.89650593)(574.86260254,52.93650391)
\curveto(575.15259718,53.05650577)(575.38759694,53.2215056)(575.56760254,53.43150391)
\curveto(575.75759657,53.63150519)(575.91259642,53.87650495)(576.03260254,54.16650391)
\curveto(576.07259626,54.25650457)(576.09759623,54.35150447)(576.10760254,54.45150391)
\curveto(576.1275962,54.55150427)(576.15259618,54.65650417)(576.18260254,54.76650391)
\curveto(576.20259613,54.81650401)(576.21259612,54.86650396)(576.21260254,54.91650391)
\curveto(576.21259612,54.96650386)(576.21759611,55.01650381)(576.22760254,55.06650391)
\curveto(576.23759609,55.09650373)(576.24259609,55.14650368)(576.24260254,55.21650391)
\curveto(576.26259607,55.29650353)(576.26259607,55.38150344)(576.24260254,55.47150391)
\curveto(576.2325961,55.5215033)(576.2275961,55.56650326)(576.22760254,55.60650391)
\curveto(576.23759609,55.64650318)(576.2325961,55.68150314)(576.21260254,55.71150391)
\curveto(576.19259614,55.73150309)(576.17759615,55.74150308)(576.16760254,55.74150391)
\lineto(576.12260254,55.78650391)
\curveto(576.02259631,55.78650304)(575.94759638,55.75650307)(575.89760254,55.69650391)
\curveto(575.85759647,55.64650318)(575.80759652,55.60150322)(575.74760254,55.56150391)
\lineto(575.50760254,55.35150391)
\curveto(575.4275969,55.29150353)(575.33759699,55.23650359)(575.23760254,55.18650391)
\curveto(575.09759723,55.09650373)(574.92259741,55.0215038)(574.71260254,54.96150391)
\curveto(574.50259783,54.91150391)(574.28259805,54.87650395)(574.05260254,54.85650391)
\curveto(573.82259851,54.83650399)(573.59259874,54.84150398)(573.36260254,54.87150391)
\curveto(573.1325992,54.89150393)(572.92259941,54.93150389)(572.73260254,54.99150391)
\curveto(571.79260054,55.30150352)(571.1326012,55.89650293)(570.75260254,56.77650391)
\curveto(570.70260163,56.87650195)(570.66260167,56.97150185)(570.63260254,57.06150391)
\curveto(570.60260173,57.16150166)(570.56760176,57.26650156)(570.52760254,57.37650391)
\curveto(570.50760182,57.4265014)(570.49760183,57.47150135)(570.49760254,57.51150391)
\curveto(570.49760183,57.55150127)(570.48760184,57.59650123)(570.46760254,57.64650391)
\curveto(570.44760188,57.71650111)(570.4326019,57.78650104)(570.42260254,57.85650391)
\curveto(570.42260191,57.93650089)(570.41260192,58.01150081)(570.39260254,58.08150391)
\curveto(570.38260195,58.1215007)(570.37760195,58.15650067)(570.37760254,58.18650391)
\curveto(570.38760194,58.2265006)(570.38760194,58.26650056)(570.37760254,58.30650391)
\curveto(570.37760195,58.34650048)(570.37260196,58.38650044)(570.36260254,58.42650391)
\lineto(570.36260254,58.54650391)
\curveto(570.34260199,58.66650016)(570.34260199,58.79150003)(570.36260254,58.92150391)
\curveto(570.37260196,58.98149984)(570.37760195,59.04149978)(570.37760254,59.10150391)
\lineto(570.37760254,59.26650391)
\curveto(570.38760194,59.31649951)(570.39260194,59.35649947)(570.39260254,59.38650391)
\curveto(570.39260194,59.4264994)(570.39760193,59.47149935)(570.40760254,59.52150391)
\curveto(570.43760189,59.63149919)(570.45760187,59.73649909)(570.46760254,59.83650391)
\curveto(570.47760185,59.94649888)(570.50260183,60.05649877)(570.54260254,60.16650391)
\curveto(570.58260175,60.28649854)(570.61760171,60.40149842)(570.64760254,60.51150391)
\curveto(570.68760164,60.63149819)(570.7326016,60.74649808)(570.78260254,60.85650391)
\curveto(570.85260148,61.01649781)(570.9326014,61.16149766)(571.02260254,61.29150391)
\curveto(571.11260122,61.43149739)(571.20760112,61.56649726)(571.30760254,61.69650391)
\curveto(571.37760095,61.80649702)(571.46760086,61.89649693)(571.57760254,61.96650391)
\lineto(571.63760254,62.02650391)
\lineto(571.69760254,62.08650391)
\lineto(571.84760254,62.20650391)
\lineto(572.02760254,62.32650391)
\curveto(572.15760017,62.40649642)(572.29260004,62.47649635)(572.43260254,62.53650391)
\curveto(572.58259975,62.59649623)(572.74259959,62.65149617)(572.91260254,62.70150391)
\curveto(573.01259932,62.73149609)(573.11259922,62.75149607)(573.21260254,62.76150391)
\curveto(573.32259901,62.77149605)(573.4325989,62.78649604)(573.54260254,62.80650391)
\curveto(573.58259875,62.81649601)(573.6325987,62.81649601)(573.69260254,62.80650391)
\curveto(573.76259857,62.79649603)(573.81259852,62.80149602)(573.84260254,62.82150391)
\curveto(574.16259817,62.83149599)(574.44759788,62.80149602)(574.69760254,62.73150391)
\curveto(574.95759737,62.66149616)(575.18759714,62.56149626)(575.38760254,62.43150391)
\curveto(575.45759687,62.39149643)(575.52259681,62.34649648)(575.58260254,62.29650391)
\lineto(575.76260254,62.14650391)
\curveto(575.81259652,62.10649672)(575.85759647,62.06149676)(575.89760254,62.01150391)
\curveto(575.94759638,61.97149685)(576.02259631,61.95149687)(576.12260254,61.95150391)
\lineto(576.16760254,61.99650391)
\curveto(576.18759614,62.01649681)(576.20759612,62.04149678)(576.22760254,62.07150391)
\curveto(576.25759607,62.15149667)(576.27259606,62.23149659)(576.27260254,62.31150391)
\curveto(576.28259605,62.39149643)(576.31259602,62.46149636)(576.36260254,62.52150391)
\curveto(576.39259594,62.56149626)(576.45259588,62.59149623)(576.54260254,62.61150391)
\curveto(576.6325957,62.64149618)(576.7275956,62.65649617)(576.82760254,62.65650391)
\curveto(576.9275954,62.65649617)(577.02259531,62.64649618)(577.11260254,62.62650391)
\curveto(577.21259512,62.60649622)(577.28259505,62.58149624)(577.32260254,62.55150391)
\moveto(576.19760254,58.77150391)
\curveto(576.20759612,58.81150001)(576.21259612,58.86149996)(576.21260254,58.92150391)
\curveto(576.21259612,58.99149983)(576.20759612,59.04649978)(576.19760254,59.08650391)
\lineto(576.19760254,59.32650391)
\curveto(576.17759615,59.41649941)(576.16259617,59.50149932)(576.15260254,59.58150391)
\curveto(576.14259619,59.67149915)(576.1275962,59.75649907)(576.10760254,59.83650391)
\curveto(576.08759624,59.91649891)(576.06759626,59.99149883)(576.04760254,60.06150391)
\curveto(576.03759629,60.14149868)(576.01759631,60.21649861)(575.98760254,60.28650391)
\curveto(575.87759645,60.56649826)(575.7325966,60.81649801)(575.55260254,61.03650391)
\curveto(575.38259695,61.25649757)(575.16259717,61.4214974)(574.89260254,61.53150391)
\curveto(574.81259752,61.57149725)(574.7275976,61.60149722)(574.63760254,61.62150391)
\curveto(574.54759778,61.65149717)(574.45259788,61.67649715)(574.35260254,61.69650391)
\curveto(574.27259806,61.71649711)(574.18259815,61.7214971)(574.08260254,61.71150391)
\lineto(573.81260254,61.71150391)
\curveto(573.76259857,61.70149712)(573.71259862,61.69649713)(573.66260254,61.69650391)
\curveto(573.62259871,61.69649713)(573.57759875,61.69149713)(573.52760254,61.68150391)
\curveto(573.33759899,61.63149719)(573.17759915,61.58149724)(573.04760254,61.53150391)
\curveto(572.70759962,61.39149743)(572.44259989,61.18149764)(572.25260254,60.90150391)
\curveto(572.06260027,60.6214982)(571.91260042,60.29649853)(571.80260254,59.92650391)
\curveto(571.78260055,59.84649898)(571.76760056,59.76649906)(571.75760254,59.68650391)
\curveto(571.75760057,59.61649921)(571.74760058,59.54149928)(571.72760254,59.46150391)
\curveto(571.70760062,59.43149939)(571.69760063,59.39649943)(571.69760254,59.35650391)
\curveto(571.70760062,59.31649951)(571.70760062,59.28149954)(571.69760254,59.25150391)
\lineto(571.69760254,58.92150391)
\lineto(571.69760254,58.57650391)
\curveto(571.69760063,58.46650036)(571.70760062,58.36150046)(571.72760254,58.26150391)
\lineto(571.72760254,58.18650391)
\curveto(571.73760059,58.15650067)(571.74260059,58.13150069)(571.74260254,58.11150391)
\curveto(571.76260057,58.0215008)(571.77760055,57.93150089)(571.78760254,57.84150391)
\curveto(571.80760052,57.75150107)(571.8326005,57.66650116)(571.86260254,57.58650391)
\curveto(571.94260039,57.3265015)(572.04260029,57.08650174)(572.16260254,56.86650391)
\curveto(572.28260005,56.64650218)(572.44259989,56.46650236)(572.64260254,56.32650391)
\lineto(572.76260254,56.23650391)
\curveto(572.80259953,56.21650261)(572.84759948,56.19650263)(572.89760254,56.17650391)
\curveto(572.97759935,56.1265027)(573.06259927,56.08650274)(573.15260254,56.05650391)
\curveto(573.24259909,56.0265028)(573.34259899,55.99650283)(573.45260254,55.96650391)
\curveto(573.50259883,55.95650287)(573.54759878,55.95150287)(573.58760254,55.95150391)
\curveto(573.63759869,55.96150286)(573.68759864,55.95650287)(573.73760254,55.93650391)
\curveto(573.76759856,55.9265029)(573.81759851,55.9215029)(573.88760254,55.92150391)
\curveto(573.95759837,55.9215029)(574.00759832,55.9265029)(574.03760254,55.93650391)
\curveto(574.06759826,55.94650288)(574.09759823,55.94650288)(574.12760254,55.93650391)
\curveto(574.16759816,55.93650289)(574.20759812,55.94150288)(574.24760254,55.95150391)
\curveto(574.33759799,55.97150285)(574.42259791,55.99150283)(574.50260254,56.01150391)
\curveto(574.58259775,56.03150279)(574.66259767,56.05650277)(574.74260254,56.08650391)
\curveto(575.08259725,56.23650259)(575.35259698,56.44650238)(575.55260254,56.71650391)
\curveto(575.75259658,56.98650184)(575.91259642,57.30150152)(576.03260254,57.66150391)
\curveto(576.06259627,57.75150107)(576.08259625,57.84150098)(576.09260254,57.93150391)
\curveto(576.11259622,58.03150079)(576.1325962,58.1265007)(576.15260254,58.21650391)
\curveto(576.16259617,58.25650057)(576.16759616,58.29150053)(576.16760254,58.32150391)
\curveto(576.16759616,58.36150046)(576.17259616,58.40150042)(576.18260254,58.44150391)
\curveto(576.20259613,58.49150033)(576.20259613,58.54150028)(576.18260254,58.59150391)
\curveto(576.17259616,58.65150017)(576.17759615,58.71150011)(576.19760254,58.77150391)
}
}
{
\newrgbcolor{curcolor}{0 0 0}
\pscustom[linestyle=none,fillstyle=solid,fillcolor=curcolor]
{
\newpath
\moveto(582.96588379,62.83650391)
\curveto(583.195879,62.83649599)(583.32587887,62.77649605)(583.35588379,62.65650391)
\curveto(583.38587881,62.54649628)(583.40087879,62.38149644)(583.40088379,62.16150391)
\lineto(583.40088379,61.87650391)
\curveto(583.40087879,61.78649704)(583.37587882,61.71149711)(583.32588379,61.65150391)
\curveto(583.26587893,61.57149725)(583.18087901,61.5264973)(583.07088379,61.51650391)
\curveto(582.96087923,61.51649731)(582.85087934,61.50149732)(582.74088379,61.47150391)
\curveto(582.60087959,61.44149738)(582.46587973,61.41149741)(582.33588379,61.38150391)
\curveto(582.21587998,61.35149747)(582.10088009,61.31149751)(581.99088379,61.26150391)
\curveto(581.70088049,61.13149769)(581.46588073,60.95149787)(581.28588379,60.72150391)
\curveto(581.10588109,60.50149832)(580.95088124,60.24649858)(580.82088379,59.95650391)
\curveto(580.78088141,59.84649898)(580.75088144,59.73149909)(580.73088379,59.61150391)
\curveto(580.71088148,59.50149932)(580.68588151,59.38649944)(580.65588379,59.26650391)
\curveto(580.64588155,59.21649961)(580.64088155,59.16649966)(580.64088379,59.11650391)
\curveto(580.65088154,59.06649976)(580.65088154,59.01649981)(580.64088379,58.96650391)
\curveto(580.61088158,58.84649998)(580.5958816,58.70650012)(580.59588379,58.54650391)
\curveto(580.60588159,58.39650043)(580.61088158,58.25150057)(580.61088379,58.11150391)
\lineto(580.61088379,56.26650391)
\lineto(580.61088379,55.92150391)
\curveto(580.61088158,55.80150302)(580.60588159,55.68650314)(580.59588379,55.57650391)
\curveto(580.58588161,55.46650336)(580.58088161,55.37150345)(580.58088379,55.29150391)
\curveto(580.5908816,55.21150361)(580.57088162,55.14150368)(580.52088379,55.08150391)
\curveto(580.47088172,55.01150381)(580.3908818,54.97150385)(580.28088379,54.96150391)
\curveto(580.18088201,54.95150387)(580.07088212,54.94650388)(579.95088379,54.94650391)
\lineto(579.68088379,54.94650391)
\curveto(579.63088256,54.96650386)(579.58088261,54.98150384)(579.53088379,54.99150391)
\curveto(579.4908827,55.01150381)(579.46088273,55.03650379)(579.44088379,55.06650391)
\curveto(579.3908828,55.13650369)(579.36088283,55.2215036)(579.35088379,55.32150391)
\lineto(579.35088379,55.65150391)
\lineto(579.35088379,56.80650391)
\lineto(579.35088379,60.96150391)
\lineto(579.35088379,61.99650391)
\lineto(579.35088379,62.29650391)
\curveto(579.36088283,62.39649643)(579.3908828,62.48149634)(579.44088379,62.55150391)
\curveto(579.47088272,62.59149623)(579.52088267,62.6214962)(579.59088379,62.64150391)
\curveto(579.67088252,62.66149616)(579.75588244,62.67149615)(579.84588379,62.67150391)
\curveto(579.93588226,62.68149614)(580.02588217,62.68149614)(580.11588379,62.67150391)
\curveto(580.20588199,62.66149616)(580.27588192,62.64649618)(580.32588379,62.62650391)
\curveto(580.40588179,62.59649623)(580.45588174,62.53649629)(580.47588379,62.44650391)
\curveto(580.50588169,62.36649646)(580.52088167,62.27649655)(580.52088379,62.17650391)
\lineto(580.52088379,61.87650391)
\curveto(580.52088167,61.77649705)(580.54088165,61.68649714)(580.58088379,61.60650391)
\curveto(580.5908816,61.58649724)(580.60088159,61.57149725)(580.61088379,61.56150391)
\lineto(580.65588379,61.51650391)
\curveto(580.76588143,61.51649731)(580.85588134,61.56149726)(580.92588379,61.65150391)
\curveto(580.9958812,61.75149707)(581.05588114,61.83149699)(581.10588379,61.89150391)
\lineto(581.19588379,61.98150391)
\curveto(581.28588091,62.09149673)(581.41088078,62.20649662)(581.57088379,62.32650391)
\curveto(581.73088046,62.44649638)(581.88088031,62.53649629)(582.02088379,62.59650391)
\curveto(582.11088008,62.64649618)(582.20587999,62.68149614)(582.30588379,62.70150391)
\curveto(582.40587979,62.73149609)(582.51087968,62.76149606)(582.62088379,62.79150391)
\curveto(582.68087951,62.80149602)(582.74087945,62.80649602)(582.80088379,62.80650391)
\curveto(582.86087933,62.81649601)(582.91587928,62.826496)(582.96588379,62.83650391)
}
}
{
\newrgbcolor{curcolor}{0 0 0}
\pscustom[linestyle=none,fillstyle=solid,fillcolor=curcolor]
{
\newpath
\moveto(591.21564941,55.48650391)
\curveto(591.24564158,55.3265035)(591.2306416,55.19150363)(591.17064941,55.08150391)
\curveto(591.11064172,54.98150384)(591.0306418,54.90650392)(590.93064941,54.85650391)
\curveto(590.88064195,54.83650399)(590.825642,54.826504)(590.76564941,54.82650391)
\curveto(590.71564211,54.826504)(590.66064217,54.81650401)(590.60064941,54.79650391)
\curveto(590.38064245,54.74650408)(590.16064267,54.76150406)(589.94064941,54.84150391)
\curveto(589.7306431,54.91150391)(589.58564324,55.00150382)(589.50564941,55.11150391)
\curveto(589.45564337,55.18150364)(589.41064342,55.26150356)(589.37064941,55.35150391)
\curveto(589.3306435,55.45150337)(589.28064355,55.53150329)(589.22064941,55.59150391)
\curveto(589.20064363,55.61150321)(589.17564365,55.63150319)(589.14564941,55.65150391)
\curveto(589.1256437,55.67150315)(589.09564373,55.67650315)(589.05564941,55.66650391)
\curveto(588.94564388,55.63650319)(588.84064399,55.58150324)(588.74064941,55.50150391)
\curveto(588.65064418,55.4215034)(588.56064427,55.35150347)(588.47064941,55.29150391)
\curveto(588.34064449,55.21150361)(588.20064463,55.13650369)(588.05064941,55.06650391)
\curveto(587.90064493,55.00650382)(587.74064509,54.95150387)(587.57064941,54.90150391)
\curveto(587.47064536,54.87150395)(587.36064547,54.85150397)(587.24064941,54.84150391)
\curveto(587.1306457,54.83150399)(587.02064581,54.81650401)(586.91064941,54.79650391)
\curveto(586.86064597,54.78650404)(586.81564601,54.78150404)(586.77564941,54.78150391)
\lineto(586.67064941,54.78150391)
\curveto(586.56064627,54.76150406)(586.45564637,54.76150406)(586.35564941,54.78150391)
\lineto(586.22064941,54.78150391)
\curveto(586.17064666,54.79150403)(586.12064671,54.79650403)(586.07064941,54.79650391)
\curveto(586.02064681,54.79650403)(585.97564685,54.80650402)(585.93564941,54.82650391)
\curveto(585.89564693,54.83650399)(585.86064697,54.84150398)(585.83064941,54.84150391)
\curveto(585.81064702,54.83150399)(585.78564704,54.83150399)(585.75564941,54.84150391)
\lineto(585.51564941,54.90150391)
\curveto(585.43564739,54.91150391)(585.36064747,54.93150389)(585.29064941,54.96150391)
\curveto(584.99064784,55.09150373)(584.74564808,55.23650359)(584.55564941,55.39650391)
\curveto(584.37564845,55.56650326)(584.2256486,55.80150302)(584.10564941,56.10150391)
\curveto(584.01564881,56.3215025)(583.97064886,56.58650224)(583.97064941,56.89650391)
\lineto(583.97064941,57.21150391)
\curveto(583.98064885,57.26150156)(583.98564884,57.31150151)(583.98564941,57.36150391)
\lineto(584.01564941,57.54150391)
\lineto(584.13564941,57.87150391)
\curveto(584.17564865,57.98150084)(584.2256486,58.08150074)(584.28564941,58.17150391)
\curveto(584.46564836,58.46150036)(584.71064812,58.67650015)(585.02064941,58.81650391)
\curveto(585.3306475,58.95649987)(585.67064716,59.08149974)(586.04064941,59.19150391)
\curveto(586.18064665,59.23149959)(586.3256465,59.26149956)(586.47564941,59.28150391)
\curveto(586.6256462,59.30149952)(586.77564605,59.3264995)(586.92564941,59.35650391)
\curveto(586.99564583,59.37649945)(587.06064577,59.38649944)(587.12064941,59.38650391)
\curveto(587.19064564,59.38649944)(587.26564556,59.39649943)(587.34564941,59.41650391)
\curveto(587.41564541,59.43649939)(587.48564534,59.44649938)(587.55564941,59.44650391)
\curveto(587.6256452,59.45649937)(587.70064513,59.47149935)(587.78064941,59.49150391)
\curveto(588.0306448,59.55149927)(588.26564456,59.60149922)(588.48564941,59.64150391)
\curveto(588.70564412,59.69149913)(588.88064395,59.80649902)(589.01064941,59.98650391)
\curveto(589.07064376,60.06649876)(589.12064371,60.16649866)(589.16064941,60.28650391)
\curveto(589.20064363,60.41649841)(589.20064363,60.55649827)(589.16064941,60.70650391)
\curveto(589.10064373,60.94649788)(589.01064382,61.13649769)(588.89064941,61.27650391)
\curveto(588.78064405,61.41649741)(588.62064421,61.5264973)(588.41064941,61.60650391)
\curveto(588.29064454,61.65649717)(588.14564468,61.69149713)(587.97564941,61.71150391)
\curveto(587.81564501,61.73149709)(587.64564518,61.74149708)(587.46564941,61.74150391)
\curveto(587.28564554,61.74149708)(587.11064572,61.73149709)(586.94064941,61.71150391)
\curveto(586.77064606,61.69149713)(586.6256462,61.66149716)(586.50564941,61.62150391)
\curveto(586.33564649,61.56149726)(586.17064666,61.47649735)(586.01064941,61.36650391)
\curveto(585.9306469,61.30649752)(585.85564697,61.2264976)(585.78564941,61.12650391)
\curveto(585.7256471,61.03649779)(585.67064716,60.93649789)(585.62064941,60.82650391)
\curveto(585.59064724,60.74649808)(585.56064727,60.66149816)(585.53064941,60.57150391)
\curveto(585.51064732,60.48149834)(585.46564736,60.41149841)(585.39564941,60.36150391)
\curveto(585.35564747,60.33149849)(585.28564754,60.30649852)(585.18564941,60.28650391)
\curveto(585.09564773,60.27649855)(585.00064783,60.27149855)(584.90064941,60.27150391)
\curveto(584.80064803,60.27149855)(584.70064813,60.27649855)(584.60064941,60.28650391)
\curveto(584.51064832,60.30649852)(584.44564838,60.33149849)(584.40564941,60.36150391)
\curveto(584.36564846,60.39149843)(584.33564849,60.44149838)(584.31564941,60.51150391)
\curveto(584.29564853,60.58149824)(584.29564853,60.65649817)(584.31564941,60.73650391)
\curveto(584.34564848,60.86649796)(584.37564845,60.98649784)(584.40564941,61.09650391)
\curveto(584.44564838,61.21649761)(584.49064834,61.33149749)(584.54064941,61.44150391)
\curveto(584.7306481,61.79149703)(584.97064786,62.06149676)(585.26064941,62.25150391)
\curveto(585.55064728,62.45149637)(585.91064692,62.61149621)(586.34064941,62.73150391)
\curveto(586.44064639,62.75149607)(586.54064629,62.76649606)(586.64064941,62.77650391)
\curveto(586.75064608,62.78649604)(586.86064597,62.80149602)(586.97064941,62.82150391)
\curveto(587.01064582,62.83149599)(587.07564575,62.83149599)(587.16564941,62.82150391)
\curveto(587.25564557,62.821496)(587.31064552,62.83149599)(587.33064941,62.85150391)
\curveto(588.0306448,62.86149596)(588.64064419,62.78149604)(589.16064941,62.61150391)
\curveto(589.68064315,62.44149638)(590.04564278,62.11649671)(590.25564941,61.63650391)
\curveto(590.34564248,61.43649739)(590.39564243,61.20149762)(590.40564941,60.93150391)
\curveto(590.4256424,60.67149815)(590.43564239,60.39649843)(590.43564941,60.10650391)
\lineto(590.43564941,56.79150391)
\curveto(590.43564239,56.65150217)(590.44064239,56.51650231)(590.45064941,56.38650391)
\curveto(590.46064237,56.25650257)(590.49064234,56.15150267)(590.54064941,56.07150391)
\curveto(590.59064224,56.00150282)(590.65564217,55.95150287)(590.73564941,55.92150391)
\curveto(590.825642,55.88150294)(590.91064192,55.85150297)(590.99064941,55.83150391)
\curveto(591.07064176,55.821503)(591.1306417,55.77650305)(591.17064941,55.69650391)
\curveto(591.19064164,55.66650316)(591.20064163,55.63650319)(591.20064941,55.60650391)
\curveto(591.20064163,55.57650325)(591.20564162,55.53650329)(591.21564941,55.48650391)
\moveto(589.07064941,57.15150391)
\curveto(589.1306437,57.29150153)(589.16064367,57.45150137)(589.16064941,57.63150391)
\curveto(589.17064366,57.821501)(589.17564365,58.01650081)(589.17564941,58.21650391)
\curveto(589.17564365,58.3265005)(589.17064366,58.4265004)(589.16064941,58.51650391)
\curveto(589.15064368,58.60650022)(589.11064372,58.67650015)(589.04064941,58.72650391)
\curveto(589.01064382,58.74650008)(588.94064389,58.75650007)(588.83064941,58.75650391)
\curveto(588.81064402,58.73650009)(588.77564405,58.7265001)(588.72564941,58.72650391)
\curveto(588.67564415,58.7265001)(588.6306442,58.71650011)(588.59064941,58.69650391)
\curveto(588.51064432,58.67650015)(588.42064441,58.65650017)(588.32064941,58.63650391)
\lineto(588.02064941,58.57650391)
\curveto(587.99064484,58.57650025)(587.95564487,58.57150025)(587.91564941,58.56150391)
\lineto(587.81064941,58.56150391)
\curveto(587.66064517,58.5215003)(587.49564533,58.49650033)(587.31564941,58.48650391)
\curveto(587.14564568,58.48650034)(586.98564584,58.46650036)(586.83564941,58.42650391)
\curveto(586.75564607,58.40650042)(586.68064615,58.38650044)(586.61064941,58.36650391)
\curveto(586.55064628,58.35650047)(586.48064635,58.34150048)(586.40064941,58.32150391)
\curveto(586.24064659,58.27150055)(586.09064674,58.20650062)(585.95064941,58.12650391)
\curveto(585.81064702,58.05650077)(585.69064714,57.96650086)(585.59064941,57.85650391)
\curveto(585.49064734,57.74650108)(585.41564741,57.61150121)(585.36564941,57.45150391)
\curveto(585.31564751,57.30150152)(585.29564753,57.11650171)(585.30564941,56.89650391)
\curveto(585.30564752,56.79650203)(585.32064751,56.70150212)(585.35064941,56.61150391)
\curveto(585.39064744,56.53150229)(585.43564739,56.45650237)(585.48564941,56.38650391)
\curveto(585.56564726,56.27650255)(585.67064716,56.18150264)(585.80064941,56.10150391)
\curveto(585.9306469,56.03150279)(586.07064676,55.97150285)(586.22064941,55.92150391)
\curveto(586.27064656,55.91150291)(586.32064651,55.90650292)(586.37064941,55.90650391)
\curveto(586.42064641,55.90650292)(586.47064636,55.90150292)(586.52064941,55.89150391)
\curveto(586.59064624,55.87150295)(586.67564615,55.85650297)(586.77564941,55.84650391)
\curveto(586.88564594,55.84650298)(586.97564585,55.85650297)(587.04564941,55.87650391)
\curveto(587.10564572,55.89650293)(587.16564566,55.90150292)(587.22564941,55.89150391)
\curveto(587.28564554,55.89150293)(587.34564548,55.90150292)(587.40564941,55.92150391)
\curveto(587.48564534,55.94150288)(587.56064527,55.95650287)(587.63064941,55.96650391)
\curveto(587.71064512,55.97650285)(587.78564504,55.99650283)(587.85564941,56.02650391)
\curveto(588.14564468,56.14650268)(588.39064444,56.29150253)(588.59064941,56.46150391)
\curveto(588.80064403,56.63150219)(588.96064387,56.86150196)(589.07064941,57.15150391)
}
}
{
\newrgbcolor{curcolor}{0 0 0}
\pscustom[linestyle=none,fillstyle=solid,fillcolor=curcolor]
{
\newpath
\moveto(594.83229004,65.73150391)
\curveto(595.0122865,65.74149308)(595.20228631,65.74149308)(595.40229004,65.73150391)
\curveto(595.60228591,65.7214931)(595.74228577,65.66149316)(595.82229004,65.55150391)
\curveto(595.86228565,65.49149333)(595.88728562,65.41649341)(595.89729004,65.32650391)
\curveto(595.9072856,65.24649358)(595.9122856,65.15649367)(595.91229004,65.05650391)
\curveto(595.9122856,64.9264939)(595.88728562,64.821494)(595.83729004,64.74150391)
\curveto(595.79728571,64.69149413)(595.73728577,64.65649417)(595.65729004,64.63650391)
\curveto(595.58728592,64.6264942)(595.507286,64.6214942)(595.41729004,64.62150391)
\lineto(595.13229004,64.62150391)
\curveto(595.04228647,64.63149419)(594.96228655,64.63149419)(594.89229004,64.62150391)
\curveto(594.6122869,64.54149428)(594.42728708,64.41149441)(594.33729004,64.23150391)
\curveto(594.25728725,64.06149476)(594.21728729,63.80149502)(594.21729004,63.45150391)
\curveto(594.21728729,63.38149544)(594.2122873,63.30649552)(594.20229004,63.22650391)
\curveto(594.19228732,63.15649567)(594.19728731,63.09149573)(594.21729004,63.03150391)
\curveto(594.24728726,62.88149594)(594.3122872,62.77649605)(594.41229004,62.71650391)
\curveto(594.49228702,62.68649614)(594.59228692,62.67149615)(594.71229004,62.67150391)
\lineto(595.07229004,62.67150391)
\lineto(595.29729004,62.67150391)
\curveto(595.32728618,62.65149617)(595.35728615,62.64649618)(595.38729004,62.65650391)
\curveto(595.41728609,62.66649616)(595.44728606,62.66149616)(595.47729004,62.64150391)
\curveto(595.57728593,62.61149621)(595.64228587,62.55149627)(595.67229004,62.46150391)
\curveto(595.70228581,62.38149644)(595.71728579,62.27649655)(595.71729004,62.14650391)
\curveto(595.7072858,62.10649672)(595.70228581,62.06649676)(595.70229004,62.02650391)
\lineto(595.70229004,61.90650391)
\curveto(595.67228584,61.75649707)(595.6072859,61.65649717)(595.50729004,61.60650391)
\curveto(595.37728613,61.55649727)(595.2072863,61.54149728)(594.99729004,61.56150391)
\curveto(594.79728671,61.59149723)(594.62728688,61.58649724)(594.48729004,61.54650391)
\curveto(594.4072871,61.5264973)(594.34728716,61.48649734)(594.30729004,61.42650391)
\curveto(594.26728724,61.37649745)(594.23728727,61.30649752)(594.21729004,61.21650391)
\curveto(594.19728731,61.14649768)(594.19228732,61.06649776)(594.20229004,60.97650391)
\curveto(594.2122873,60.88649794)(594.21728729,60.80149802)(594.21729004,60.72150391)
\lineto(594.21729004,59.73150391)
\lineto(594.21729004,56.55150391)
\lineto(594.21729004,55.80150391)
\lineto(594.21729004,55.60650391)
\curveto(594.22728728,55.53650329)(594.22228729,55.47650335)(594.20229004,55.42650391)
\lineto(594.20229004,55.30650391)
\lineto(594.17229004,55.18650391)
\curveto(594.16228735,55.14650368)(594.14728736,55.11150371)(594.12729004,55.08150391)
\curveto(594.07728743,55.01150381)(594.00228751,54.97150385)(593.90229004,54.96150391)
\curveto(593.80228771,54.95150387)(593.69228782,54.94650388)(593.57229004,54.94650391)
\lineto(593.28729004,54.94650391)
\curveto(593.23728827,54.96650386)(593.18728832,54.98150384)(593.13729004,54.99150391)
\curveto(593.09728841,55.01150381)(593.06228845,55.04650378)(593.03229004,55.09650391)
\curveto(593.0122885,55.1265037)(592.99228852,55.19150363)(592.97229004,55.29150391)
\lineto(592.97229004,55.39650391)
\curveto(592.95228856,55.44650338)(592.94228857,55.49650333)(592.94229004,55.54650391)
\curveto(592.95228856,55.60650322)(592.95728855,55.66150316)(592.95729004,55.71150391)
\lineto(592.95729004,56.31150391)
\lineto(592.95729004,60.40650391)
\lineto(592.95729004,60.75150391)
\curveto(592.96728854,60.87149795)(592.96728854,60.98149784)(592.95729004,61.08150391)
\curveto(592.95728855,61.19149763)(592.93728857,61.28649754)(592.89729004,61.36650391)
\curveto(592.86728864,61.44649738)(592.8122887,61.50149732)(592.73229004,61.53150391)
\curveto(592.67228884,61.56149726)(592.60228891,61.57649725)(592.52229004,61.57650391)
\lineto(592.29729004,61.57650391)
\lineto(592.05729004,61.57650391)
\curveto(591.98728952,61.57649725)(591.92228959,61.58649724)(591.86229004,61.60650391)
\curveto(591.77228974,61.64649718)(591.7072898,61.73149709)(591.66729004,61.86150391)
\curveto(591.65728985,61.91149691)(591.65228986,61.95649687)(591.65229004,61.99650391)
\lineto(591.65229004,62.13150391)
\curveto(591.65228986,62.27149655)(591.66728984,62.38149644)(591.69729004,62.46150391)
\curveto(591.72728978,62.55149627)(591.79228972,62.61149621)(591.89229004,62.64150391)
\curveto(591.96228955,62.67149615)(592.04228947,62.68149614)(592.13229004,62.67150391)
\lineto(592.41729004,62.67150391)
\curveto(592.51728899,62.67149615)(592.60228891,62.68149614)(592.67229004,62.70150391)
\curveto(592.75228876,62.7214961)(592.81728869,62.76149606)(592.86729004,62.82150391)
\curveto(592.93728857,62.90149592)(592.96728854,63.0264958)(592.95729004,63.19650391)
\lineto(592.95729004,63.67650391)
\curveto(592.95728855,63.87649495)(592.96728854,64.06149476)(592.98729004,64.23150391)
\curveto(593.01728849,64.41149441)(593.06228845,64.57149425)(593.12229004,64.71150391)
\curveto(593.23228828,64.95149387)(593.37728813,65.14649368)(593.55729004,65.29650391)
\curveto(593.74728776,65.44649338)(593.97228754,65.56149326)(594.23229004,65.64150391)
\curveto(594.29228722,65.66149316)(594.35228716,65.67149315)(594.41229004,65.67150391)
\curveto(594.48228703,65.68149314)(594.55228696,65.69649313)(594.62229004,65.71650391)
\curveto(594.64228687,65.7264931)(594.67728683,65.7264931)(594.72729004,65.71650391)
\curveto(594.77728673,65.71649311)(594.8122867,65.7214931)(594.83229004,65.73150391)
\moveto(597.09729004,64.15650391)
\curveto(597.16728434,64.10649472)(597.25228426,64.08149474)(597.35229004,64.08150391)
\lineto(597.66729004,64.08150391)
\lineto(597.83229004,64.08150391)
\curveto(597.89228362,64.08149474)(597.94728356,64.09149473)(597.99729004,64.11150391)
\curveto(598.12728338,64.16149466)(598.19228332,64.26649456)(598.19229004,64.42650391)
\curveto(598.20228331,64.58649424)(598.2072833,64.75649407)(598.20729004,64.93650391)
\lineto(598.20729004,65.19150391)
\curveto(598.2072833,65.28149354)(598.19228332,65.35649347)(598.16229004,65.41650391)
\curveto(598.1122834,65.5264933)(598.0122835,65.58649324)(597.86229004,65.59650391)
\curveto(597.7122838,65.60649322)(597.55228396,65.61149321)(597.38229004,65.61150391)
\curveto(597.36228415,65.60149322)(597.33728417,65.59649323)(597.30729004,65.59650391)
\curveto(597.28728422,65.60649322)(597.26728424,65.60649322)(597.24729004,65.59650391)
\curveto(597.12728438,65.55649327)(597.04728446,65.49649333)(597.00729004,65.41650391)
\curveto(596.97728453,65.35649347)(596.96228455,65.28149354)(596.96229004,65.19150391)
\lineto(596.96229004,64.93650391)
\lineto(596.96229004,64.47150391)
\curveto(596.97228454,64.3214945)(597.01728449,64.21649461)(597.09729004,64.15650391)
\moveto(598.20729004,61.99650391)
\lineto(598.20729004,62.28150391)
\curveto(598.2072833,62.38149644)(598.18228333,62.46149636)(598.13229004,62.52150391)
\curveto(598.08228343,62.60149622)(597.98728352,62.64149618)(597.84729004,62.64150391)
\curveto(597.71728379,62.65149617)(597.58728392,62.65649617)(597.45729004,62.65650391)
\curveto(597.43728407,62.64649618)(597.4122841,62.64149618)(597.38229004,62.64150391)
\curveto(597.36228415,62.65149617)(597.34228417,62.65649617)(597.32229004,62.65650391)
\curveto(597.26228425,62.63649619)(597.2072843,62.6214962)(597.15729004,62.61150391)
\curveto(597.1072844,62.60149622)(597.06728444,62.57149625)(597.03729004,62.52150391)
\curveto(596.98728452,62.46149636)(596.96228455,62.37649645)(596.96229004,62.26650391)
\lineto(596.96229004,61.95150391)
\lineto(596.96229004,55.60650391)
\lineto(596.96229004,55.32150391)
\curveto(596.96228455,55.23150359)(596.98228453,55.15650367)(597.02229004,55.09650391)
\curveto(597.07228444,55.01650381)(597.14228437,54.96650386)(597.23229004,54.94650391)
\curveto(597.33228418,54.93650389)(597.44728406,54.93150389)(597.57729004,54.93150391)
\lineto(597.80229004,54.93150391)
\curveto(597.88228363,54.95150387)(597.95228356,54.96650386)(598.01229004,54.97650391)
\curveto(598.07228344,54.99650383)(598.11728339,55.03650379)(598.14729004,55.09650391)
\curveto(598.19728331,55.15650367)(598.21728329,55.23650359)(598.20729004,55.33650391)
\lineto(598.20729004,55.65150391)
\lineto(598.20729004,61.99650391)
}
}
{
\newrgbcolor{curcolor}{0 0 0}
\pscustom[linestyle=none,fillstyle=solid,fillcolor=curcolor]
{
\newpath
\moveto(607.03596191,55.48650391)
\curveto(607.06595408,55.3265035)(607.0509541,55.19150363)(606.99096191,55.08150391)
\curveto(606.93095422,54.98150384)(606.8509543,54.90650392)(606.75096191,54.85650391)
\curveto(606.70095445,54.83650399)(606.6459545,54.826504)(606.58596191,54.82650391)
\curveto(606.53595461,54.826504)(606.48095467,54.81650401)(606.42096191,54.79650391)
\curveto(606.20095495,54.74650408)(605.98095517,54.76150406)(605.76096191,54.84150391)
\curveto(605.5509556,54.91150391)(605.40595574,55.00150382)(605.32596191,55.11150391)
\curveto(605.27595587,55.18150364)(605.23095592,55.26150356)(605.19096191,55.35150391)
\curveto(605.150956,55.45150337)(605.10095605,55.53150329)(605.04096191,55.59150391)
\curveto(605.02095613,55.61150321)(604.99595615,55.63150319)(604.96596191,55.65150391)
\curveto(604.9459562,55.67150315)(604.91595623,55.67650315)(604.87596191,55.66650391)
\curveto(604.76595638,55.63650319)(604.66095649,55.58150324)(604.56096191,55.50150391)
\curveto(604.47095668,55.4215034)(604.38095677,55.35150347)(604.29096191,55.29150391)
\curveto(604.16095699,55.21150361)(604.02095713,55.13650369)(603.87096191,55.06650391)
\curveto(603.72095743,55.00650382)(603.56095759,54.95150387)(603.39096191,54.90150391)
\curveto(603.29095786,54.87150395)(603.18095797,54.85150397)(603.06096191,54.84150391)
\curveto(602.9509582,54.83150399)(602.84095831,54.81650401)(602.73096191,54.79650391)
\curveto(602.68095847,54.78650404)(602.63595851,54.78150404)(602.59596191,54.78150391)
\lineto(602.49096191,54.78150391)
\curveto(602.38095877,54.76150406)(602.27595887,54.76150406)(602.17596191,54.78150391)
\lineto(602.04096191,54.78150391)
\curveto(601.99095916,54.79150403)(601.94095921,54.79650403)(601.89096191,54.79650391)
\curveto(601.84095931,54.79650403)(601.79595935,54.80650402)(601.75596191,54.82650391)
\curveto(601.71595943,54.83650399)(601.68095947,54.84150398)(601.65096191,54.84150391)
\curveto(601.63095952,54.83150399)(601.60595954,54.83150399)(601.57596191,54.84150391)
\lineto(601.33596191,54.90150391)
\curveto(601.25595989,54.91150391)(601.18095997,54.93150389)(601.11096191,54.96150391)
\curveto(600.81096034,55.09150373)(600.56596058,55.23650359)(600.37596191,55.39650391)
\curveto(600.19596095,55.56650326)(600.0459611,55.80150302)(599.92596191,56.10150391)
\curveto(599.83596131,56.3215025)(599.79096136,56.58650224)(599.79096191,56.89650391)
\lineto(599.79096191,57.21150391)
\curveto(599.80096135,57.26150156)(599.80596134,57.31150151)(599.80596191,57.36150391)
\lineto(599.83596191,57.54150391)
\lineto(599.95596191,57.87150391)
\curveto(599.99596115,57.98150084)(600.0459611,58.08150074)(600.10596191,58.17150391)
\curveto(600.28596086,58.46150036)(600.53096062,58.67650015)(600.84096191,58.81650391)
\curveto(601.15096,58.95649987)(601.49095966,59.08149974)(601.86096191,59.19150391)
\curveto(602.00095915,59.23149959)(602.145959,59.26149956)(602.29596191,59.28150391)
\curveto(602.4459587,59.30149952)(602.59595855,59.3264995)(602.74596191,59.35650391)
\curveto(602.81595833,59.37649945)(602.88095827,59.38649944)(602.94096191,59.38650391)
\curveto(603.01095814,59.38649944)(603.08595806,59.39649943)(603.16596191,59.41650391)
\curveto(603.23595791,59.43649939)(603.30595784,59.44649938)(603.37596191,59.44650391)
\curveto(603.4459577,59.45649937)(603.52095763,59.47149935)(603.60096191,59.49150391)
\curveto(603.8509573,59.55149927)(604.08595706,59.60149922)(604.30596191,59.64150391)
\curveto(604.52595662,59.69149913)(604.70095645,59.80649902)(604.83096191,59.98650391)
\curveto(604.89095626,60.06649876)(604.94095621,60.16649866)(604.98096191,60.28650391)
\curveto(605.02095613,60.41649841)(605.02095613,60.55649827)(604.98096191,60.70650391)
\curveto(604.92095623,60.94649788)(604.83095632,61.13649769)(604.71096191,61.27650391)
\curveto(604.60095655,61.41649741)(604.44095671,61.5264973)(604.23096191,61.60650391)
\curveto(604.11095704,61.65649717)(603.96595718,61.69149713)(603.79596191,61.71150391)
\curveto(603.63595751,61.73149709)(603.46595768,61.74149708)(603.28596191,61.74150391)
\curveto(603.10595804,61.74149708)(602.93095822,61.73149709)(602.76096191,61.71150391)
\curveto(602.59095856,61.69149713)(602.4459587,61.66149716)(602.32596191,61.62150391)
\curveto(602.15595899,61.56149726)(601.99095916,61.47649735)(601.83096191,61.36650391)
\curveto(601.7509594,61.30649752)(601.67595947,61.2264976)(601.60596191,61.12650391)
\curveto(601.5459596,61.03649779)(601.49095966,60.93649789)(601.44096191,60.82650391)
\curveto(601.41095974,60.74649808)(601.38095977,60.66149816)(601.35096191,60.57150391)
\curveto(601.33095982,60.48149834)(601.28595986,60.41149841)(601.21596191,60.36150391)
\curveto(601.17595997,60.33149849)(601.10596004,60.30649852)(601.00596191,60.28650391)
\curveto(600.91596023,60.27649855)(600.82096033,60.27149855)(600.72096191,60.27150391)
\curveto(600.62096053,60.27149855)(600.52096063,60.27649855)(600.42096191,60.28650391)
\curveto(600.33096082,60.30649852)(600.26596088,60.33149849)(600.22596191,60.36150391)
\curveto(600.18596096,60.39149843)(600.15596099,60.44149838)(600.13596191,60.51150391)
\curveto(600.11596103,60.58149824)(600.11596103,60.65649817)(600.13596191,60.73650391)
\curveto(600.16596098,60.86649796)(600.19596095,60.98649784)(600.22596191,61.09650391)
\curveto(600.26596088,61.21649761)(600.31096084,61.33149749)(600.36096191,61.44150391)
\curveto(600.5509606,61.79149703)(600.79096036,62.06149676)(601.08096191,62.25150391)
\curveto(601.37095978,62.45149637)(601.73095942,62.61149621)(602.16096191,62.73150391)
\curveto(602.26095889,62.75149607)(602.36095879,62.76649606)(602.46096191,62.77650391)
\curveto(602.57095858,62.78649604)(602.68095847,62.80149602)(602.79096191,62.82150391)
\curveto(602.83095832,62.83149599)(602.89595825,62.83149599)(602.98596191,62.82150391)
\curveto(603.07595807,62.821496)(603.13095802,62.83149599)(603.15096191,62.85150391)
\curveto(603.8509573,62.86149596)(604.46095669,62.78149604)(604.98096191,62.61150391)
\curveto(605.50095565,62.44149638)(605.86595528,62.11649671)(606.07596191,61.63650391)
\curveto(606.16595498,61.43649739)(606.21595493,61.20149762)(606.22596191,60.93150391)
\curveto(606.2459549,60.67149815)(606.25595489,60.39649843)(606.25596191,60.10650391)
\lineto(606.25596191,56.79150391)
\curveto(606.25595489,56.65150217)(606.26095489,56.51650231)(606.27096191,56.38650391)
\curveto(606.28095487,56.25650257)(606.31095484,56.15150267)(606.36096191,56.07150391)
\curveto(606.41095474,56.00150282)(606.47595467,55.95150287)(606.55596191,55.92150391)
\curveto(606.6459545,55.88150294)(606.73095442,55.85150297)(606.81096191,55.83150391)
\curveto(606.89095426,55.821503)(606.9509542,55.77650305)(606.99096191,55.69650391)
\curveto(607.01095414,55.66650316)(607.02095413,55.63650319)(607.02096191,55.60650391)
\curveto(607.02095413,55.57650325)(607.02595412,55.53650329)(607.03596191,55.48650391)
\moveto(604.89096191,57.15150391)
\curveto(604.9509562,57.29150153)(604.98095617,57.45150137)(604.98096191,57.63150391)
\curveto(604.99095616,57.821501)(604.99595615,58.01650081)(604.99596191,58.21650391)
\curveto(604.99595615,58.3265005)(604.99095616,58.4265004)(604.98096191,58.51650391)
\curveto(604.97095618,58.60650022)(604.93095622,58.67650015)(604.86096191,58.72650391)
\curveto(604.83095632,58.74650008)(604.76095639,58.75650007)(604.65096191,58.75650391)
\curveto(604.63095652,58.73650009)(604.59595655,58.7265001)(604.54596191,58.72650391)
\curveto(604.49595665,58.7265001)(604.4509567,58.71650011)(604.41096191,58.69650391)
\curveto(604.33095682,58.67650015)(604.24095691,58.65650017)(604.14096191,58.63650391)
\lineto(603.84096191,58.57650391)
\curveto(603.81095734,58.57650025)(603.77595737,58.57150025)(603.73596191,58.56150391)
\lineto(603.63096191,58.56150391)
\curveto(603.48095767,58.5215003)(603.31595783,58.49650033)(603.13596191,58.48650391)
\curveto(602.96595818,58.48650034)(602.80595834,58.46650036)(602.65596191,58.42650391)
\curveto(602.57595857,58.40650042)(602.50095865,58.38650044)(602.43096191,58.36650391)
\curveto(602.37095878,58.35650047)(602.30095885,58.34150048)(602.22096191,58.32150391)
\curveto(602.06095909,58.27150055)(601.91095924,58.20650062)(601.77096191,58.12650391)
\curveto(601.63095952,58.05650077)(601.51095964,57.96650086)(601.41096191,57.85650391)
\curveto(601.31095984,57.74650108)(601.23595991,57.61150121)(601.18596191,57.45150391)
\curveto(601.13596001,57.30150152)(601.11596003,57.11650171)(601.12596191,56.89650391)
\curveto(601.12596002,56.79650203)(601.14096001,56.70150212)(601.17096191,56.61150391)
\curveto(601.21095994,56.53150229)(601.25595989,56.45650237)(601.30596191,56.38650391)
\curveto(601.38595976,56.27650255)(601.49095966,56.18150264)(601.62096191,56.10150391)
\curveto(601.7509594,56.03150279)(601.89095926,55.97150285)(602.04096191,55.92150391)
\curveto(602.09095906,55.91150291)(602.14095901,55.90650292)(602.19096191,55.90650391)
\curveto(602.24095891,55.90650292)(602.29095886,55.90150292)(602.34096191,55.89150391)
\curveto(602.41095874,55.87150295)(602.49595865,55.85650297)(602.59596191,55.84650391)
\curveto(602.70595844,55.84650298)(602.79595835,55.85650297)(602.86596191,55.87650391)
\curveto(602.92595822,55.89650293)(602.98595816,55.90150292)(603.04596191,55.89150391)
\curveto(603.10595804,55.89150293)(603.16595798,55.90150292)(603.22596191,55.92150391)
\curveto(603.30595784,55.94150288)(603.38095777,55.95650287)(603.45096191,55.96650391)
\curveto(603.53095762,55.97650285)(603.60595754,55.99650283)(603.67596191,56.02650391)
\curveto(603.96595718,56.14650268)(604.21095694,56.29150253)(604.41096191,56.46150391)
\curveto(604.62095653,56.63150219)(604.78095637,56.86150196)(604.89096191,57.15150391)
}
}
{
\newrgbcolor{curcolor}{0 0 0}
\pscustom[linestyle=none,fillstyle=solid,fillcolor=curcolor]
{
\newpath
\moveto(610.63760254,62.83650391)
\curveto(611.35759847,62.84649598)(611.96259787,62.76149606)(612.45260254,62.58150391)
\curveto(612.94259689,62.41149641)(613.32259651,62.10649672)(613.59260254,61.66650391)
\curveto(613.66259617,61.55649727)(613.71759611,61.44149738)(613.75760254,61.32150391)
\curveto(613.79759603,61.21149761)(613.83759599,61.08649774)(613.87760254,60.94650391)
\curveto(613.89759593,60.87649795)(613.90259593,60.80149802)(613.89260254,60.72150391)
\curveto(613.88259595,60.65149817)(613.86759596,60.59649823)(613.84760254,60.55650391)
\curveto(613.827596,60.53649829)(613.80259603,60.51649831)(613.77260254,60.49650391)
\curveto(613.74259609,60.48649834)(613.71759611,60.47149835)(613.69760254,60.45150391)
\curveto(613.64759618,60.43149839)(613.59759623,60.4264984)(613.54760254,60.43650391)
\curveto(613.49759633,60.44649838)(613.44759638,60.44649838)(613.39760254,60.43650391)
\curveto(613.31759651,60.41649841)(613.21259662,60.41149841)(613.08260254,60.42150391)
\curveto(612.95259688,60.44149838)(612.86259697,60.46649836)(612.81260254,60.49650391)
\curveto(612.7325971,60.54649828)(612.67759715,60.61149821)(612.64760254,60.69150391)
\curveto(612.6275972,60.78149804)(612.59259724,60.86649796)(612.54260254,60.94650391)
\curveto(612.45259738,61.10649772)(612.3275975,61.25149757)(612.16760254,61.38150391)
\curveto(612.05759777,61.46149736)(611.93759789,61.5214973)(611.80760254,61.56150391)
\curveto(611.67759815,61.60149722)(611.53759829,61.64149718)(611.38760254,61.68150391)
\curveto(611.33759849,61.70149712)(611.28759854,61.70649712)(611.23760254,61.69650391)
\curveto(611.18759864,61.69649713)(611.13759869,61.70149712)(611.08760254,61.71150391)
\curveto(611.0275988,61.73149709)(610.95259888,61.74149708)(610.86260254,61.74150391)
\curveto(610.77259906,61.74149708)(610.69759913,61.73149709)(610.63760254,61.71150391)
\lineto(610.54760254,61.71150391)
\lineto(610.39760254,61.68150391)
\curveto(610.34759948,61.68149714)(610.29759953,61.67649715)(610.24760254,61.66650391)
\curveto(609.98759984,61.60649722)(609.77260006,61.5214973)(609.60260254,61.41150391)
\curveto(609.4326004,61.30149752)(609.31760051,61.11649771)(609.25760254,60.85650391)
\curveto(609.23760059,60.78649804)(609.2326006,60.71649811)(609.24260254,60.64650391)
\curveto(609.26260057,60.57649825)(609.28260055,60.51649831)(609.30260254,60.46650391)
\curveto(609.36260047,60.31649851)(609.4326004,60.20649862)(609.51260254,60.13650391)
\curveto(609.60260023,60.07649875)(609.71260012,60.00649882)(609.84260254,59.92650391)
\curveto(610.00259983,59.826499)(610.18259965,59.75149907)(610.38260254,59.70150391)
\curveto(610.58259925,59.66149916)(610.78259905,59.61149921)(610.98260254,59.55150391)
\curveto(611.11259872,59.51149931)(611.24259859,59.48149934)(611.37260254,59.46150391)
\curveto(611.50259833,59.44149938)(611.6325982,59.41149941)(611.76260254,59.37150391)
\curveto(611.97259786,59.31149951)(612.17759765,59.25149957)(612.37760254,59.19150391)
\curveto(612.57759725,59.14149968)(612.77759705,59.07649975)(612.97760254,58.99650391)
\lineto(613.12760254,58.93650391)
\curveto(613.17759665,58.91649991)(613.2275966,58.89149993)(613.27760254,58.86150391)
\curveto(613.47759635,58.74150008)(613.65259618,58.60650022)(613.80260254,58.45650391)
\curveto(613.95259588,58.30650052)(614.07759575,58.11650071)(614.17760254,57.88650391)
\curveto(614.19759563,57.81650101)(614.21759561,57.7215011)(614.23760254,57.60150391)
\curveto(614.25759557,57.53150129)(614.26759556,57.45650137)(614.26760254,57.37650391)
\curveto(614.27759555,57.30650152)(614.28259555,57.2265016)(614.28260254,57.13650391)
\lineto(614.28260254,56.98650391)
\curveto(614.26259557,56.91650191)(614.25259558,56.84650198)(614.25260254,56.77650391)
\curveto(614.25259558,56.70650212)(614.24259559,56.63650219)(614.22260254,56.56650391)
\curveto(614.19259564,56.45650237)(614.15759567,56.35150247)(614.11760254,56.25150391)
\curveto(614.07759575,56.15150267)(614.0325958,56.06150276)(613.98260254,55.98150391)
\curveto(613.82259601,55.7215031)(613.61759621,55.51150331)(613.36760254,55.35150391)
\curveto(613.11759671,55.20150362)(612.83759699,55.07150375)(612.52760254,54.96150391)
\curveto(612.43759739,54.93150389)(612.34259749,54.91150391)(612.24260254,54.90150391)
\curveto(612.15259768,54.88150394)(612.06259777,54.85650397)(611.97260254,54.82650391)
\curveto(611.87259796,54.80650402)(611.77259806,54.79650403)(611.67260254,54.79650391)
\curveto(611.57259826,54.79650403)(611.47259836,54.78650404)(611.37260254,54.76650391)
\lineto(611.22260254,54.76650391)
\curveto(611.17259866,54.75650407)(611.10259873,54.75150407)(611.01260254,54.75150391)
\curveto(610.92259891,54.75150407)(610.85259898,54.75650407)(610.80260254,54.76650391)
\lineto(610.63760254,54.76650391)
\curveto(610.57759925,54.78650404)(610.51259932,54.79650403)(610.44260254,54.79650391)
\curveto(610.37259946,54.78650404)(610.31259952,54.79150403)(610.26260254,54.81150391)
\curveto(610.21259962,54.821504)(610.14759968,54.826504)(610.06760254,54.82650391)
\lineto(609.82760254,54.88650391)
\curveto(609.75760007,54.89650393)(609.68260015,54.91650391)(609.60260254,54.94650391)
\curveto(609.29260054,55.04650378)(609.02260081,55.17150365)(608.79260254,55.32150391)
\curveto(608.56260127,55.47150335)(608.36260147,55.66650316)(608.19260254,55.90650391)
\curveto(608.10260173,56.03650279)(608.0276018,56.17150265)(607.96760254,56.31150391)
\curveto(607.90760192,56.45150237)(607.85260198,56.60650222)(607.80260254,56.77650391)
\curveto(607.78260205,56.83650199)(607.77260206,56.90650192)(607.77260254,56.98650391)
\curveto(607.78260205,57.07650175)(607.79760203,57.14650168)(607.81760254,57.19650391)
\curveto(607.84760198,57.23650159)(607.89760193,57.27650155)(607.96760254,57.31650391)
\curveto(608.01760181,57.33650149)(608.08760174,57.34650148)(608.17760254,57.34650391)
\curveto(608.26760156,57.35650147)(608.35760147,57.35650147)(608.44760254,57.34650391)
\curveto(608.53760129,57.33650149)(608.62260121,57.3215015)(608.70260254,57.30150391)
\curveto(608.79260104,57.29150153)(608.85260098,57.27650155)(608.88260254,57.25650391)
\curveto(608.95260088,57.20650162)(608.99760083,57.13150169)(609.01760254,57.03150391)
\curveto(609.04760078,56.94150188)(609.08260075,56.85650197)(609.12260254,56.77650391)
\curveto(609.22260061,56.55650227)(609.35760047,56.38650244)(609.52760254,56.26650391)
\curveto(609.64760018,56.17650265)(609.78260005,56.10650272)(609.93260254,56.05650391)
\curveto(610.08259975,56.00650282)(610.24259959,55.95650287)(610.41260254,55.90650391)
\lineto(610.72760254,55.86150391)
\lineto(610.81760254,55.86150391)
\curveto(610.88759894,55.84150298)(610.97759885,55.83150299)(611.08760254,55.83150391)
\curveto(611.20759862,55.83150299)(611.30759852,55.84150298)(611.38760254,55.86150391)
\curveto(611.45759837,55.86150296)(611.51259832,55.86650296)(611.55260254,55.87650391)
\curveto(611.61259822,55.88650294)(611.67259816,55.89150293)(611.73260254,55.89150391)
\curveto(611.79259804,55.90150292)(611.84759798,55.91150291)(611.89760254,55.92150391)
\curveto(612.18759764,56.00150282)(612.41759741,56.10650272)(612.58760254,56.23650391)
\curveto(612.75759707,56.36650246)(612.87759695,56.58650224)(612.94760254,56.89650391)
\curveto(612.96759686,56.94650188)(612.97259686,57.00150182)(612.96260254,57.06150391)
\curveto(612.95259688,57.1215017)(612.94259689,57.16650166)(612.93260254,57.19650391)
\curveto(612.88259695,57.38650144)(612.81259702,57.5265013)(612.72260254,57.61650391)
\curveto(612.6325972,57.71650111)(612.51759731,57.80650102)(612.37760254,57.88650391)
\curveto(612.28759754,57.94650088)(612.18759764,57.99650083)(612.07760254,58.03650391)
\lineto(611.74760254,58.15650391)
\curveto(611.71759811,58.16650066)(611.68759814,58.17150065)(611.65760254,58.17150391)
\curveto(611.63759819,58.17150065)(611.61259822,58.18150064)(611.58260254,58.20150391)
\curveto(611.24259859,58.31150051)(610.88759894,58.39150043)(610.51760254,58.44150391)
\curveto(610.15759967,58.50150032)(609.81760001,58.59650023)(609.49760254,58.72650391)
\curveto(609.39760043,58.76650006)(609.30260053,58.80150002)(609.21260254,58.83150391)
\curveto(609.12260071,58.86149996)(609.03760079,58.90149992)(608.95760254,58.95150391)
\curveto(608.76760106,59.06149976)(608.59260124,59.18649964)(608.43260254,59.32650391)
\curveto(608.27260156,59.46649936)(608.14760168,59.64149918)(608.05760254,59.85150391)
\curveto(608.0276018,59.9214989)(608.00260183,59.99149883)(607.98260254,60.06150391)
\curveto(607.97260186,60.13149869)(607.95760187,60.20649862)(607.93760254,60.28650391)
\curveto(607.90760192,60.40649842)(607.89760193,60.54149828)(607.90760254,60.69150391)
\curveto(607.91760191,60.85149797)(607.9326019,60.98649784)(607.95260254,61.09650391)
\curveto(607.97260186,61.14649768)(607.98260185,61.18649764)(607.98260254,61.21650391)
\curveto(607.99260184,61.25649757)(608.00760182,61.29649753)(608.02760254,61.33650391)
\curveto(608.11760171,61.56649726)(608.23760159,61.76649706)(608.38760254,61.93650391)
\curveto(608.54760128,62.10649672)(608.7276011,62.25649657)(608.92760254,62.38650391)
\curveto(609.07760075,62.47649635)(609.24260059,62.54649628)(609.42260254,62.59650391)
\curveto(609.60260023,62.65649617)(609.79260004,62.71149611)(609.99260254,62.76150391)
\curveto(610.06259977,62.77149605)(610.1275997,62.78149604)(610.18760254,62.79150391)
\curveto(610.25759957,62.80149602)(610.3325995,62.81149601)(610.41260254,62.82150391)
\curveto(610.44259939,62.83149599)(610.48259935,62.83149599)(610.53260254,62.82150391)
\curveto(610.58259925,62.81149601)(610.61759921,62.81649601)(610.63760254,62.83650391)
}
}
{
\newrgbcolor{curcolor}{0 0 0}
\pscustom[linestyle=none,fillstyle=solid,fillcolor=curcolor]
{
\newpath
\moveto(660.7180835,65.67151611)
\curveto(660.87808284,65.67150543)(661.05308267,65.67150543)(661.2430835,65.67151611)
\curveto(661.43308229,65.68150542)(661.57808214,65.65650545)(661.6780835,65.59651611)
\curveto(661.76808195,65.53650557)(661.82808189,65.44150566)(661.8580835,65.31151611)
\curveto(661.89808182,65.18150592)(661.93808178,65.06150604)(661.9780835,64.95151611)
\curveto(662.05808166,64.75150635)(662.12808159,64.54650656)(662.1880835,64.33651611)
\curveto(662.24808147,64.13650697)(662.3180814,63.93650717)(662.3980835,63.73651611)
\curveto(662.4180813,63.68650742)(662.43308129,63.63650747)(662.4430835,63.58651611)
\lineto(662.4730835,63.43651611)
\curveto(662.54308118,63.26650784)(662.60308112,63.08650802)(662.6530835,62.89651611)
\curveto(662.71308101,62.71650839)(662.77308095,62.53150857)(662.8330835,62.34151611)
\curveto(662.97308075,61.93150917)(663.10808061,61.52650958)(663.2380835,61.12651611)
\curveto(663.37808034,60.72651038)(663.5180802,60.32151078)(663.6580835,59.91151611)
\curveto(663.72807999,59.71151139)(663.78807993,59.5065116)(663.8380835,59.29651611)
\curveto(663.89807982,59.09651201)(663.96807975,58.89651221)(664.0480835,58.69651611)
\curveto(664.06807965,58.64651246)(664.08307964,58.59151251)(664.0930835,58.53151611)
\lineto(664.1530835,58.35151611)
\curveto(664.26307946,58.06151304)(664.36307936,57.76151334)(664.4530835,57.45151611)
\curveto(664.49307923,57.35151375)(664.52807919,57.24651386)(664.5580835,57.13651611)
\curveto(664.58807913,57.03651407)(664.63307909,56.94651416)(664.6930835,56.86651611)
\curveto(664.71307901,56.84651426)(664.74807897,56.81651429)(664.7980835,56.77651611)
\curveto(664.9180788,56.78651432)(664.99307873,56.83651427)(665.0230835,56.92651611)
\curveto(665.05307867,57.02651408)(665.08807863,57.12151398)(665.1280835,57.21151611)
\curveto(665.23807848,57.47151363)(665.32807839,57.73651337)(665.3980835,58.00651611)
\curveto(665.46807825,58.27651283)(665.55307817,58.54151256)(665.6530835,58.80151611)
\curveto(665.71307801,58.96151214)(665.76307796,59.12151198)(665.8030835,59.28151611)
\curveto(665.85307787,59.44151166)(665.90807781,59.6015115)(665.9680835,59.76151611)
\curveto(666.0180777,59.88151122)(666.05807766,60.0015111)(666.0880835,60.12151611)
\curveto(666.12807759,60.25151085)(666.17307755,60.37651073)(666.2230835,60.49651611)
\curveto(666.37307735,60.91651019)(666.51307721,61.34150976)(666.6430835,61.77151611)
\curveto(666.77307695,62.2015089)(666.9180768,62.62650848)(667.0780835,63.04651611)
\curveto(667.09807662,63.08650802)(667.10807661,63.12150798)(667.1080835,63.15151611)
\curveto(667.10807661,63.19150791)(667.1180766,63.23150787)(667.1380835,63.27151611)
\curveto(667.19807652,63.42150768)(667.25307647,63.57650753)(667.3030835,63.73651611)
\curveto(667.35307637,63.89650721)(667.40307632,64.05150705)(667.4530835,64.20151611)
\curveto(667.51307621,64.35150675)(667.56307616,64.5015066)(667.6030835,64.65151611)
\curveto(667.65307607,64.81150629)(667.70807601,64.97150613)(667.7680835,65.13151611)
\curveto(667.79807592,65.22150588)(667.82807589,65.3065058)(667.8580835,65.38651611)
\curveto(667.89807582,65.47650563)(667.95807576,65.54650556)(668.0380835,65.59651611)
\curveto(668.09807562,65.64650546)(668.17807554,65.67150543)(668.2780835,65.67151611)
\curveto(668.38807533,65.67150543)(668.49807522,65.67150543)(668.6080835,65.67151611)
\lineto(668.9380835,65.67151611)
\curveto(669.0180747,65.65150545)(669.08807463,65.63150547)(669.1480835,65.61151611)
\curveto(669.20807451,65.6015055)(669.24807447,65.55650555)(669.2680835,65.47651611)
\lineto(669.2680835,65.40151611)
\curveto(669.27807444,65.38150572)(669.27807444,65.36150574)(669.2680835,65.34151611)
\curveto(669.24807447,65.24150586)(669.2180745,65.14150596)(669.1780835,65.04151611)
\curveto(669.14807457,64.95150615)(669.1180746,64.86650624)(669.0880835,64.78651611)
\curveto(669.04807467,64.7065064)(669.01307471,64.62150648)(668.9830835,64.53151611)
\curveto(668.96307476,64.45150665)(668.93807478,64.37150673)(668.9080835,64.29151611)
\curveto(668.84807487,64.15150695)(668.79307493,64.0015071)(668.7430835,63.84151611)
\curveto(668.70307502,63.69150741)(668.65307507,63.54650756)(668.5930835,63.40651611)
\curveto(668.57307515,63.36650774)(668.56307516,63.33150777)(668.5630835,63.30151611)
\curveto(668.56307516,63.27150783)(668.55307517,63.23650787)(668.5330835,63.19651611)
\curveto(668.45307527,63.02650808)(668.38307534,62.84650826)(668.3230835,62.65651611)
\curveto(668.27307545,62.46650864)(668.20807551,62.28650882)(668.1280835,62.11651611)
\curveto(668.10807561,62.07650903)(668.09807562,62.03650907)(668.0980835,61.99651611)
\curveto(668.09807562,61.96650914)(668.08807563,61.93650917)(668.0680835,61.90651611)
\curveto(668.0180757,61.77650933)(667.96807575,61.64150946)(667.9180835,61.50151611)
\curveto(667.87807584,61.37150973)(667.83307589,61.24150986)(667.7830835,61.11151611)
\curveto(667.76307596,61.08151002)(667.74807597,61.04651006)(667.7380835,61.00651611)
\curveto(667.73807598,60.97651013)(667.72807599,60.94151016)(667.7080835,60.90151611)
\curveto(667.55807616,60.52151058)(667.4180763,60.13651097)(667.2880835,59.74651611)
\curveto(667.16807655,59.35651175)(667.03307669,58.97151213)(666.8830835,58.59151611)
\lineto(666.8380835,58.45651611)
\lineto(666.7180835,58.12651611)
\curveto(666.68807703,58.02651308)(666.65307707,57.92151318)(666.6130835,57.81151611)
\curveto(666.56307716,57.69151341)(666.5180772,57.56651354)(666.4780835,57.43651611)
\curveto(666.43807728,57.31651379)(666.39307733,57.19651391)(666.3430835,57.07651611)
\lineto(666.2830835,56.89651611)
\lineto(666.2230835,56.71651611)
\curveto(666.16307756,56.56651454)(666.10807761,56.41151469)(666.0580835,56.25151611)
\curveto(666.00807771,56.09151501)(665.95307777,55.94151516)(665.8930835,55.80151611)
\curveto(665.84307788,55.67151543)(665.79307793,55.53151557)(665.7430835,55.38151611)
\curveto(665.70307802,55.24151586)(665.63307809,55.13651597)(665.5330835,55.06651611)
\curveto(665.49307823,55.04651606)(665.44807827,55.03151607)(665.3980835,55.02151611)
\curveto(665.34807837,55.01151609)(665.29307843,55.0015161)(665.2330835,54.99151611)
\lineto(664.8130835,54.99151611)
\lineto(664.4230835,54.99151611)
\curveto(664.28307944,54.98151612)(664.17307955,55.0015161)(664.0930835,55.05151611)
\curveto(664.00307972,55.101516)(663.94307978,55.17151593)(663.9130835,55.26151611)
\curveto(663.88307984,55.36151574)(663.84307988,55.46651564)(663.7930835,55.57651611)
\curveto(663.73307999,55.72651538)(663.67808004,55.88151522)(663.6280835,56.04151611)
\curveto(663.57808014,56.21151489)(663.5180802,56.37651473)(663.4480835,56.53651611)
\curveto(663.42808029,56.57651453)(663.41308031,56.61651449)(663.4030835,56.65651611)
\curveto(663.40308032,56.69651441)(663.39308033,56.73651437)(663.3730835,56.77651611)
\curveto(663.29308043,56.97651413)(663.2180805,57.17651393)(663.1480835,57.37651611)
\curveto(663.08808063,57.58651352)(663.0180807,57.78651332)(662.9380835,57.97651611)
\curveto(662.9180808,58.02651308)(662.90308082,58.07151303)(662.8930835,58.11151611)
\curveto(662.89308083,58.15151295)(662.88308084,58.19151291)(662.8630835,58.23151611)
\curveto(662.81308091,58.37151273)(662.76308096,58.5065126)(662.7130835,58.63651611)
\lineto(662.5630835,59.05651611)
\curveto(662.54308118,59.09651201)(662.52808119,59.13651197)(662.5180835,59.17651611)
\curveto(662.5180812,59.21651189)(662.50808121,59.25651185)(662.4880835,59.29651611)
\lineto(662.3380835,59.68651611)
\curveto(662.29808142,59.82651128)(662.25308147,59.96651114)(662.2030835,60.10651611)
\curveto(662.15308157,60.21651089)(662.11308161,60.32651078)(662.0830835,60.43651611)
\curveto(662.05308167,60.55651055)(662.01308171,60.67151043)(661.9630835,60.78151611)
\curveto(661.85308187,61.06151004)(661.75308197,61.34650976)(661.6630835,61.63651611)
\curveto(661.57308215,61.93650917)(661.46808225,62.22650888)(661.3480835,62.50651611)
\curveto(661.30808241,62.59650851)(661.27308245,62.68650842)(661.2430835,62.77651611)
\curveto(661.2230825,62.87650823)(661.19808252,62.96650814)(661.1680835,63.04651611)
\curveto(661.13808258,63.106508)(661.11308261,63.16650794)(661.0930835,63.22651611)
\curveto(661.08308264,63.29650781)(661.06308266,63.36150774)(661.0330835,63.42151611)
\curveto(660.94308278,63.65150745)(660.85808286,63.88650722)(660.7780835,64.12651611)
\curveto(660.70808301,64.36650674)(660.62808309,64.6015065)(660.5380835,64.83151611)
\curveto(660.5180832,64.9015062)(660.49308323,64.97150613)(660.4630835,65.04151611)
\curveto(660.44308328,65.11150599)(660.4230833,65.18650592)(660.4030835,65.26651611)
\curveto(660.36308336,65.36650574)(660.35808336,65.45150565)(660.3880835,65.52151611)
\curveto(660.4180833,65.59150551)(660.48808323,65.63650547)(660.5980835,65.65651611)
\curveto(660.6180831,65.66650544)(660.63808308,65.66650544)(660.6580835,65.65651611)
\curveto(660.67808304,65.65650545)(660.69808302,65.66150544)(660.7180835,65.67151611)
}
}
{
\newrgbcolor{curcolor}{0 0 0}
\pscustom[linestyle=none,fillstyle=solid,fillcolor=curcolor]
{
\newpath
\moveto(670.59300537,64.23151611)
\curveto(670.51300425,64.29150681)(670.4680043,64.39650671)(670.45800537,64.54651611)
\lineto(670.45800537,65.01151611)
\lineto(670.45800537,65.26651611)
\curveto(670.45800431,65.35650575)(670.47300429,65.43150567)(670.50300537,65.49151611)
\curveto(670.54300422,65.57150553)(670.62300414,65.63150547)(670.74300537,65.67151611)
\curveto(670.763004,65.68150542)(670.78300398,65.68150542)(670.80300537,65.67151611)
\curveto(670.83300393,65.67150543)(670.85800391,65.67650543)(670.87800537,65.68651611)
\curveto(671.04800372,65.68650542)(671.20800356,65.68150542)(671.35800537,65.67151611)
\curveto(671.50800326,65.66150544)(671.60800316,65.6015055)(671.65800537,65.49151611)
\curveto(671.68800308,65.43150567)(671.70300306,65.35650575)(671.70300537,65.26651611)
\lineto(671.70300537,65.01151611)
\curveto(671.70300306,64.83150627)(671.69800307,64.66150644)(671.68800537,64.50151611)
\curveto(671.68800308,64.34150676)(671.62300314,64.23650687)(671.49300537,64.18651611)
\curveto(671.44300332,64.16650694)(671.38800338,64.15650695)(671.32800537,64.15651611)
\lineto(671.16300537,64.15651611)
\lineto(670.84800537,64.15651611)
\curveto(670.74800402,64.15650695)(670.6630041,64.18150692)(670.59300537,64.23151611)
\moveto(671.70300537,55.72651611)
\lineto(671.70300537,55.41151611)
\curveto(671.71300305,55.31151579)(671.69300307,55.23151587)(671.64300537,55.17151611)
\curveto(671.61300315,55.11151599)(671.5680032,55.07151603)(671.50800537,55.05151611)
\curveto(671.44800332,55.04151606)(671.37800339,55.02651608)(671.29800537,55.00651611)
\lineto(671.07300537,55.00651611)
\curveto(670.94300382,55.0065161)(670.82800394,55.01151609)(670.72800537,55.02151611)
\curveto(670.63800413,55.04151606)(670.5680042,55.09151601)(670.51800537,55.17151611)
\curveto(670.47800429,55.23151587)(670.45800431,55.3065158)(670.45800537,55.39651611)
\lineto(670.45800537,55.68151611)
\lineto(670.45800537,62.02651611)
\lineto(670.45800537,62.34151611)
\curveto(670.45800431,62.45150865)(670.48300428,62.53650857)(670.53300537,62.59651611)
\curveto(670.5630042,62.64650846)(670.60300416,62.67650843)(670.65300537,62.68651611)
\curveto(670.70300406,62.69650841)(670.75800401,62.71150839)(670.81800537,62.73151611)
\curveto(670.83800393,62.73150837)(670.85800391,62.72650838)(670.87800537,62.71651611)
\curveto(670.90800386,62.71650839)(670.93300383,62.72150838)(670.95300537,62.73151611)
\curveto(671.08300368,62.73150837)(671.21300355,62.72650838)(671.34300537,62.71651611)
\curveto(671.48300328,62.71650839)(671.57800319,62.67650843)(671.62800537,62.59651611)
\curveto(671.67800309,62.53650857)(671.70300306,62.45650865)(671.70300537,62.35651611)
\lineto(671.70300537,62.07151611)
\lineto(671.70300537,55.72651611)
}
}
{
\newrgbcolor{curcolor}{0 0 0}
\pscustom[linestyle=none,fillstyle=solid,fillcolor=curcolor]
{
\newpath
\moveto(680.60784912,55.81651611)
\lineto(680.60784912,55.42651611)
\curveto(680.60784125,55.3065158)(680.58284127,55.2065159)(680.53284912,55.12651611)
\curveto(680.48284137,55.05651605)(680.39784146,55.01651609)(680.27784912,55.00651611)
\lineto(679.93284912,55.00651611)
\curveto(679.87284198,55.0065161)(679.81284204,55.0015161)(679.75284912,54.99151611)
\curveto(679.70284215,54.99151611)(679.6578422,55.0015161)(679.61784912,55.02151611)
\curveto(679.52784233,55.04151606)(679.46784239,55.08151602)(679.43784912,55.14151611)
\curveto(679.39784246,55.19151591)(679.37284248,55.25151585)(679.36284912,55.32151611)
\curveto(679.36284249,55.39151571)(679.34784251,55.46151564)(679.31784912,55.53151611)
\curveto(679.30784255,55.55151555)(679.29284256,55.56651554)(679.27284912,55.57651611)
\curveto(679.26284259,55.59651551)(679.24784261,55.61651549)(679.22784912,55.63651611)
\curveto(679.12784273,55.64651546)(679.04784281,55.62651548)(678.98784912,55.57651611)
\curveto(678.93784292,55.52651558)(678.88284297,55.47651563)(678.82284912,55.42651611)
\curveto(678.62284323,55.27651583)(678.42284343,55.16151594)(678.22284912,55.08151611)
\curveto(678.04284381,55.0015161)(677.83284402,54.94151616)(677.59284912,54.90151611)
\curveto(677.36284449,54.86151624)(677.12284473,54.84151626)(676.87284912,54.84151611)
\curveto(676.63284522,54.83151627)(676.39284546,54.84651626)(676.15284912,54.88651611)
\curveto(675.91284594,54.91651619)(675.70284615,54.97151613)(675.52284912,55.05151611)
\curveto(675.00284685,55.27151583)(674.58284727,55.56651554)(674.26284912,55.93651611)
\curveto(673.94284791,56.31651479)(673.69284816,56.78651432)(673.51284912,57.34651611)
\curveto(673.47284838,57.43651367)(673.44284841,57.52651358)(673.42284912,57.61651611)
\curveto(673.41284844,57.71651339)(673.39284846,57.81651329)(673.36284912,57.91651611)
\curveto(673.3528485,57.96651314)(673.34784851,58.01651309)(673.34784912,58.06651611)
\curveto(673.34784851,58.11651299)(673.34284851,58.16651294)(673.33284912,58.21651611)
\curveto(673.31284854,58.26651284)(673.30284855,58.31651279)(673.30284912,58.36651611)
\curveto(673.31284854,58.42651268)(673.31284854,58.48151262)(673.30284912,58.53151611)
\lineto(673.30284912,58.68151611)
\curveto(673.28284857,58.73151237)(673.27284858,58.79651231)(673.27284912,58.87651611)
\curveto(673.27284858,58.95651215)(673.28284857,59.02151208)(673.30284912,59.07151611)
\lineto(673.30284912,59.23651611)
\curveto(673.32284853,59.3065118)(673.32784853,59.37651173)(673.31784912,59.44651611)
\curveto(673.31784854,59.52651158)(673.32784853,59.6015115)(673.34784912,59.67151611)
\curveto(673.3578485,59.72151138)(673.36284849,59.76651134)(673.36284912,59.80651611)
\curveto(673.36284849,59.84651126)(673.36784849,59.89151121)(673.37784912,59.94151611)
\curveto(673.40784845,60.04151106)(673.43284842,60.13651097)(673.45284912,60.22651611)
\curveto(673.47284838,60.32651078)(673.49784836,60.42151068)(673.52784912,60.51151611)
\curveto(673.6578482,60.89151021)(673.82284803,61.23150987)(674.02284912,61.53151611)
\curveto(674.23284762,61.84150926)(674.48284737,62.09650901)(674.77284912,62.29651611)
\curveto(674.94284691,62.41650869)(675.11784674,62.51650859)(675.29784912,62.59651611)
\curveto(675.48784637,62.67650843)(675.69284616,62.74650836)(675.91284912,62.80651611)
\curveto(675.98284587,62.81650829)(676.04784581,62.82650828)(676.10784912,62.83651611)
\curveto(676.17784568,62.84650826)(676.24784561,62.86150824)(676.31784912,62.88151611)
\lineto(676.46784912,62.88151611)
\curveto(676.54784531,62.9015082)(676.66284519,62.91150819)(676.81284912,62.91151611)
\curveto(676.97284488,62.91150819)(677.09284476,62.9015082)(677.17284912,62.88151611)
\curveto(677.21284464,62.87150823)(677.26784459,62.86650824)(677.33784912,62.86651611)
\curveto(677.44784441,62.83650827)(677.5578443,62.81150829)(677.66784912,62.79151611)
\curveto(677.77784408,62.78150832)(677.88284397,62.75150835)(677.98284912,62.70151611)
\curveto(678.13284372,62.64150846)(678.27284358,62.57650853)(678.40284912,62.50651611)
\curveto(678.54284331,62.43650867)(678.67284318,62.35650875)(678.79284912,62.26651611)
\curveto(678.852843,62.21650889)(678.91284294,62.16150894)(678.97284912,62.10151611)
\curveto(679.04284281,62.05150905)(679.13284272,62.03650907)(679.24284912,62.05651611)
\curveto(679.26284259,62.08650902)(679.27784258,62.11150899)(679.28784912,62.13151611)
\curveto(679.30784255,62.15150895)(679.32284253,62.18150892)(679.33284912,62.22151611)
\curveto(679.36284249,62.31150879)(679.37284248,62.42650868)(679.36284912,62.56651611)
\lineto(679.36284912,62.94151611)
\lineto(679.36284912,64.66651611)
\lineto(679.36284912,65.13151611)
\curveto(679.36284249,65.31150579)(679.38784247,65.44150566)(679.43784912,65.52151611)
\curveto(679.47784238,65.59150551)(679.53784232,65.63650547)(679.61784912,65.65651611)
\curveto(679.63784222,65.65650545)(679.66284219,65.65650545)(679.69284912,65.65651611)
\curveto(679.72284213,65.66650544)(679.74784211,65.67150543)(679.76784912,65.67151611)
\curveto(679.90784195,65.68150542)(680.0528418,65.68150542)(680.20284912,65.67151611)
\curveto(680.36284149,65.67150543)(680.47284138,65.63150547)(680.53284912,65.55151611)
\curveto(680.58284127,65.47150563)(680.60784125,65.37150573)(680.60784912,65.25151611)
\lineto(680.60784912,64.87651611)
\lineto(680.60784912,55.81651611)
\moveto(679.39284912,58.65151611)
\curveto(679.41284244,58.7015124)(679.42284243,58.76651234)(679.42284912,58.84651611)
\curveto(679.42284243,58.93651217)(679.41284244,59.0065121)(679.39284912,59.05651611)
\lineto(679.39284912,59.28151611)
\curveto(679.37284248,59.37151173)(679.3578425,59.46151164)(679.34784912,59.55151611)
\curveto(679.33784252,59.65151145)(679.31784254,59.74151136)(679.28784912,59.82151611)
\curveto(679.26784259,59.9015112)(679.24784261,59.97651113)(679.22784912,60.04651611)
\curveto(679.21784264,60.11651099)(679.19784266,60.18651092)(679.16784912,60.25651611)
\curveto(679.04784281,60.55651055)(678.89284296,60.82151028)(678.70284912,61.05151611)
\curveto(678.51284334,61.28150982)(678.27284358,61.46150964)(677.98284912,61.59151611)
\curveto(677.88284397,61.64150946)(677.77784408,61.67650943)(677.66784912,61.69651611)
\curveto(677.56784429,61.72650938)(677.4578444,61.75150935)(677.33784912,61.77151611)
\curveto(677.2578446,61.79150931)(677.16784469,61.8015093)(677.06784912,61.80151611)
\lineto(676.79784912,61.80151611)
\curveto(676.74784511,61.79150931)(676.70284515,61.78150932)(676.66284912,61.77151611)
\lineto(676.52784912,61.77151611)
\curveto(676.44784541,61.75150935)(676.36284549,61.73150937)(676.27284912,61.71151611)
\curveto(676.19284566,61.69150941)(676.11284574,61.66650944)(676.03284912,61.63651611)
\curveto(675.71284614,61.49650961)(675.4528464,61.29150981)(675.25284912,61.02151611)
\curveto(675.06284679,60.76151034)(674.90784695,60.45651065)(674.78784912,60.10651611)
\curveto(674.74784711,59.99651111)(674.71784714,59.88151122)(674.69784912,59.76151611)
\curveto(674.68784717,59.65151145)(674.67284718,59.54151156)(674.65284912,59.43151611)
\curveto(674.6528472,59.39151171)(674.64784721,59.35151175)(674.63784912,59.31151611)
\lineto(674.63784912,59.20651611)
\curveto(674.61784724,59.15651195)(674.60784725,59.101512)(674.60784912,59.04151611)
\curveto(674.61784724,58.98151212)(674.62284723,58.92651218)(674.62284912,58.87651611)
\lineto(674.62284912,58.54651611)
\curveto(674.62284723,58.44651266)(674.63284722,58.35151275)(674.65284912,58.26151611)
\curveto(674.66284719,58.23151287)(674.66784719,58.18151292)(674.66784912,58.11151611)
\curveto(674.68784717,58.04151306)(674.70284715,57.97151313)(674.71284912,57.90151611)
\lineto(674.77284912,57.69151611)
\curveto(674.88284697,57.34151376)(675.03284682,57.04151406)(675.22284912,56.79151611)
\curveto(675.41284644,56.54151456)(675.6528462,56.33651477)(675.94284912,56.17651611)
\curveto(676.03284582,56.12651498)(676.12284573,56.08651502)(676.21284912,56.05651611)
\curveto(676.30284555,56.02651508)(676.40284545,55.99651511)(676.51284912,55.96651611)
\curveto(676.56284529,55.94651516)(676.61284524,55.94151516)(676.66284912,55.95151611)
\curveto(676.72284513,55.96151514)(676.77784508,55.95651515)(676.82784912,55.93651611)
\curveto(676.86784499,55.92651518)(676.90784495,55.92151518)(676.94784912,55.92151611)
\lineto(677.08284912,55.92151611)
\lineto(677.21784912,55.92151611)
\curveto(677.24784461,55.93151517)(677.29784456,55.93651517)(677.36784912,55.93651611)
\curveto(677.44784441,55.95651515)(677.52784433,55.97151513)(677.60784912,55.98151611)
\curveto(677.68784417,56.0015151)(677.76284409,56.02651508)(677.83284912,56.05651611)
\curveto(678.16284369,56.19651491)(678.42784343,56.37151473)(678.62784912,56.58151611)
\curveto(678.83784302,56.8015143)(679.01284284,57.07651403)(679.15284912,57.40651611)
\curveto(679.20284265,57.51651359)(679.23784262,57.62651348)(679.25784912,57.73651611)
\curveto(679.27784258,57.84651326)(679.30284255,57.95651315)(679.33284912,58.06651611)
\curveto(679.3528425,58.106513)(679.36284249,58.14151296)(679.36284912,58.17151611)
\curveto(679.36284249,58.21151289)(679.36784249,58.25151285)(679.37784912,58.29151611)
\curveto(679.38784247,58.35151275)(679.38784247,58.41151269)(679.37784912,58.47151611)
\curveto(679.37784248,58.53151257)(679.38284247,58.59151251)(679.39284912,58.65151611)
}
}
{
\newrgbcolor{curcolor}{0 0 0}
\pscustom[linestyle=none,fillstyle=solid,fillcolor=curcolor]
{
\newpath
\moveto(689.30409912,59.17651611)
\curveto(689.32409144,59.07651203)(689.32409144,58.96151214)(689.30409912,58.83151611)
\curveto(689.29409147,58.71151239)(689.2640915,58.62651248)(689.21409912,58.57651611)
\curveto(689.1640916,58.53651257)(689.08909167,58.5065126)(688.98909912,58.48651611)
\curveto(688.89909186,58.47651263)(688.79409197,58.47151263)(688.67409912,58.47151611)
\lineto(688.31409912,58.47151611)
\curveto(688.19409257,58.48151262)(688.08909267,58.48651262)(687.99909912,58.48651611)
\lineto(684.15909912,58.48651611)
\curveto(684.07909668,58.48651262)(683.99909676,58.48151262)(683.91909912,58.47151611)
\curveto(683.83909692,58.47151263)(683.77409699,58.45651265)(683.72409912,58.42651611)
\curveto(683.68409708,58.4065127)(683.64409712,58.36651274)(683.60409912,58.30651611)
\curveto(683.58409718,58.27651283)(683.5640972,58.23151287)(683.54409912,58.17151611)
\curveto(683.52409724,58.12151298)(683.52409724,58.07151303)(683.54409912,58.02151611)
\curveto(683.55409721,57.97151313)(683.5590972,57.92651318)(683.55909912,57.88651611)
\curveto(683.5590972,57.84651326)(683.5640972,57.8065133)(683.57409912,57.76651611)
\curveto(683.59409717,57.68651342)(683.61409715,57.6015135)(683.63409912,57.51151611)
\curveto(683.65409711,57.43151367)(683.68409708,57.35151375)(683.72409912,57.27151611)
\curveto(683.95409681,56.73151437)(684.33409643,56.34651476)(684.86409912,56.11651611)
\curveto(684.92409584,56.08651502)(684.98909577,56.06151504)(685.05909912,56.04151611)
\lineto(685.26909912,55.98151611)
\curveto(685.29909546,55.97151513)(685.34909541,55.96651514)(685.41909912,55.96651611)
\curveto(685.5590952,55.92651518)(685.74409502,55.9065152)(685.97409912,55.90651611)
\curveto(686.20409456,55.9065152)(686.38909437,55.92651518)(686.52909912,55.96651611)
\curveto(686.66909409,56.0065151)(686.79409397,56.04651506)(686.90409912,56.08651611)
\curveto(687.02409374,56.13651497)(687.13409363,56.19651491)(687.23409912,56.26651611)
\curveto(687.34409342,56.33651477)(687.43909332,56.41651469)(687.51909912,56.50651611)
\curveto(687.59909316,56.6065145)(687.66909309,56.71151439)(687.72909912,56.82151611)
\curveto(687.78909297,56.92151418)(687.83909292,57.02651408)(687.87909912,57.13651611)
\curveto(687.92909283,57.24651386)(688.00909275,57.32651378)(688.11909912,57.37651611)
\curveto(688.1590926,57.39651371)(688.22409254,57.41151369)(688.31409912,57.42151611)
\curveto(688.40409236,57.43151367)(688.49409227,57.43151367)(688.58409912,57.42151611)
\curveto(688.67409209,57.42151368)(688.759092,57.41651369)(688.83909912,57.40651611)
\curveto(688.91909184,57.39651371)(688.97409179,57.37651373)(689.00409912,57.34651611)
\curveto(689.10409166,57.27651383)(689.12909163,57.16151394)(689.07909912,57.00151611)
\curveto(688.99909176,56.73151437)(688.89409187,56.49151461)(688.76409912,56.28151611)
\curveto(688.5640922,55.96151514)(688.33409243,55.69651541)(688.07409912,55.48651611)
\curveto(687.82409294,55.28651582)(687.50409326,55.12151598)(687.11409912,54.99151611)
\curveto(687.01409375,54.95151615)(686.91409385,54.92651618)(686.81409912,54.91651611)
\curveto(686.71409405,54.89651621)(686.60909415,54.87651623)(686.49909912,54.85651611)
\curveto(686.44909431,54.84651626)(686.39909436,54.84151626)(686.34909912,54.84151611)
\curveto(686.30909445,54.84151626)(686.2640945,54.83651627)(686.21409912,54.82651611)
\lineto(686.06409912,54.82651611)
\curveto(686.01409475,54.81651629)(685.95409481,54.81151629)(685.88409912,54.81151611)
\curveto(685.82409494,54.81151629)(685.77409499,54.81651629)(685.73409912,54.82651611)
\lineto(685.59909912,54.82651611)
\curveto(685.54909521,54.83651627)(685.50409526,54.84151626)(685.46409912,54.84151611)
\curveto(685.42409534,54.84151626)(685.38409538,54.84651626)(685.34409912,54.85651611)
\curveto(685.29409547,54.86651624)(685.23909552,54.87651623)(685.17909912,54.88651611)
\curveto(685.11909564,54.88651622)(685.0640957,54.89151621)(685.01409912,54.90151611)
\curveto(684.92409584,54.92151618)(684.83409593,54.94651616)(684.74409912,54.97651611)
\curveto(684.65409611,54.99651611)(684.56909619,55.02151608)(684.48909912,55.05151611)
\curveto(684.44909631,55.07151603)(684.41409635,55.08151602)(684.38409912,55.08151611)
\curveto(684.35409641,55.09151601)(684.31909644,55.106516)(684.27909912,55.12651611)
\curveto(684.12909663,55.19651591)(683.96909679,55.28151582)(683.79909912,55.38151611)
\curveto(683.50909725,55.57151553)(683.2590975,55.8015153)(683.04909912,56.07151611)
\curveto(682.84909791,56.35151475)(682.67909808,56.66151444)(682.53909912,57.00151611)
\curveto(682.48909827,57.11151399)(682.44909831,57.22651388)(682.41909912,57.34651611)
\curveto(682.39909836,57.46651364)(682.36909839,57.58651352)(682.32909912,57.70651611)
\curveto(682.31909844,57.74651336)(682.31409845,57.78151332)(682.31409912,57.81151611)
\curveto(682.31409845,57.84151326)(682.30909845,57.88151322)(682.29909912,57.93151611)
\curveto(682.27909848,58.01151309)(682.2640985,58.09651301)(682.25409912,58.18651611)
\curveto(682.24409852,58.27651283)(682.22909853,58.36651274)(682.20909912,58.45651611)
\lineto(682.20909912,58.66651611)
\curveto(682.19909856,58.7065124)(682.18909857,58.76151234)(682.17909912,58.83151611)
\curveto(682.17909858,58.91151219)(682.18409858,58.97651213)(682.19409912,59.02651611)
\lineto(682.19409912,59.19151611)
\curveto(682.21409855,59.24151186)(682.21909854,59.29151181)(682.20909912,59.34151611)
\curveto(682.20909855,59.4015117)(682.21409855,59.45651165)(682.22409912,59.50651611)
\curveto(682.2640985,59.66651144)(682.29409847,59.82651128)(682.31409912,59.98651611)
\curveto(682.34409842,60.14651096)(682.38909837,60.29651081)(682.44909912,60.43651611)
\curveto(682.49909826,60.54651056)(682.54409822,60.65651045)(682.58409912,60.76651611)
\curveto(682.63409813,60.88651022)(682.68909807,61.0015101)(682.74909912,61.11151611)
\curveto(682.96909779,61.46150964)(683.21909754,61.76150934)(683.49909912,62.01151611)
\curveto(683.77909698,62.27150883)(684.12409664,62.48650862)(684.53409912,62.65651611)
\curveto(684.65409611,62.7065084)(684.77409599,62.74150836)(684.89409912,62.76151611)
\curveto(685.02409574,62.79150831)(685.1590956,62.82150828)(685.29909912,62.85151611)
\curveto(685.34909541,62.86150824)(685.39409537,62.86650824)(685.43409912,62.86651611)
\curveto(685.47409529,62.87650823)(685.51909524,62.88150822)(685.56909912,62.88151611)
\curveto(685.58909517,62.89150821)(685.61409515,62.89150821)(685.64409912,62.88151611)
\curveto(685.67409509,62.87150823)(685.69909506,62.87650823)(685.71909912,62.89651611)
\curveto(686.13909462,62.9065082)(686.50409426,62.86150824)(686.81409912,62.76151611)
\curveto(687.12409364,62.67150843)(687.40409336,62.54650856)(687.65409912,62.38651611)
\curveto(687.70409306,62.36650874)(687.74409302,62.33650877)(687.77409912,62.29651611)
\curveto(687.80409296,62.26650884)(687.83909292,62.24150886)(687.87909912,62.22151611)
\curveto(687.9590928,62.16150894)(688.03909272,62.09150901)(688.11909912,62.01151611)
\curveto(688.20909255,61.93150917)(688.28409248,61.85150925)(688.34409912,61.77151611)
\curveto(688.50409226,61.56150954)(688.63909212,61.36150974)(688.74909912,61.17151611)
\curveto(688.81909194,61.06151004)(688.87409189,60.94151016)(688.91409912,60.81151611)
\curveto(688.95409181,60.68151042)(688.99909176,60.55151055)(689.04909912,60.42151611)
\curveto(689.09909166,60.29151081)(689.13409163,60.15651095)(689.15409912,60.01651611)
\curveto(689.18409158,59.87651123)(689.21909154,59.73651137)(689.25909912,59.59651611)
\curveto(689.26909149,59.52651158)(689.27409149,59.45651165)(689.27409912,59.38651611)
\lineto(689.30409912,59.17651611)
\moveto(687.84909912,59.68651611)
\curveto(687.87909288,59.72651138)(687.90409286,59.77651133)(687.92409912,59.83651611)
\curveto(687.94409282,59.9065112)(687.94409282,59.97651113)(687.92409912,60.04651611)
\curveto(687.8640929,60.26651084)(687.77909298,60.47151063)(687.66909912,60.66151611)
\curveto(687.52909323,60.89151021)(687.37409339,61.08651002)(687.20409912,61.24651611)
\curveto(687.03409373,61.4065097)(686.81409395,61.54150956)(686.54409912,61.65151611)
\curveto(686.47409429,61.67150943)(686.40409436,61.68650942)(686.33409912,61.69651611)
\curveto(686.2640945,61.71650939)(686.18909457,61.73650937)(686.10909912,61.75651611)
\curveto(686.02909473,61.77650933)(685.94409482,61.78650932)(685.85409912,61.78651611)
\lineto(685.59909912,61.78651611)
\curveto(685.56909519,61.76650934)(685.53409523,61.75650935)(685.49409912,61.75651611)
\curveto(685.45409531,61.76650934)(685.41909534,61.76650934)(685.38909912,61.75651611)
\lineto(685.14909912,61.69651611)
\curveto(685.07909568,61.68650942)(685.00909575,61.67150943)(684.93909912,61.65151611)
\curveto(684.64909611,61.53150957)(684.41409635,61.38150972)(684.23409912,61.20151611)
\curveto(684.0640967,61.02151008)(683.90909685,60.79651031)(683.76909912,60.52651611)
\curveto(683.73909702,60.47651063)(683.70909705,60.41151069)(683.67909912,60.33151611)
\curveto(683.64909711,60.26151084)(683.62409714,60.18151092)(683.60409912,60.09151611)
\curveto(683.58409718,60.0015111)(683.57909718,59.91651119)(683.58909912,59.83651611)
\curveto(683.59909716,59.75651135)(683.63409713,59.69651141)(683.69409912,59.65651611)
\curveto(683.77409699,59.59651151)(683.90909685,59.56651154)(684.09909912,59.56651611)
\curveto(684.29909646,59.57651153)(684.46909629,59.58151152)(684.60909912,59.58151611)
\lineto(686.88909912,59.58151611)
\curveto(687.03909372,59.58151152)(687.21909354,59.57651153)(687.42909912,59.56651611)
\curveto(687.63909312,59.56651154)(687.77909298,59.6065115)(687.84909912,59.68651611)
}
}
{
\newrgbcolor{curcolor}{0 0 0}
\pscustom[linestyle=none,fillstyle=solid,fillcolor=curcolor]
{
\newpath
\moveto(697.73573975,59.20651611)
\curveto(697.75573169,59.14651196)(697.76573168,59.05151205)(697.76573975,58.92151611)
\curveto(697.76573168,58.8015123)(697.76073168,58.71651239)(697.75073975,58.66651611)
\lineto(697.75073975,58.51651611)
\curveto(697.7407317,58.43651267)(697.73073171,58.36151274)(697.72073975,58.29151611)
\curveto(697.72073172,58.23151287)(697.71573173,58.16151294)(697.70573975,58.08151611)
\curveto(697.68573176,58.02151308)(697.67073177,57.96151314)(697.66073975,57.90151611)
\curveto(697.66073178,57.84151326)(697.65073179,57.78151332)(697.63073975,57.72151611)
\curveto(697.59073185,57.59151351)(697.55573189,57.46151364)(697.52573975,57.33151611)
\curveto(697.49573195,57.2015139)(697.45573199,57.08151402)(697.40573975,56.97151611)
\curveto(697.19573225,56.49151461)(696.91573253,56.08651502)(696.56573975,55.75651611)
\curveto(696.21573323,55.43651567)(695.78573366,55.19151591)(695.27573975,55.02151611)
\curveto(695.16573428,54.98151612)(695.0457344,54.95151615)(694.91573975,54.93151611)
\curveto(694.79573465,54.91151619)(694.67073477,54.89151621)(694.54073975,54.87151611)
\curveto(694.48073496,54.86151624)(694.41573503,54.85651625)(694.34573975,54.85651611)
\curveto(694.28573516,54.84651626)(694.22573522,54.84151626)(694.16573975,54.84151611)
\curveto(694.12573532,54.83151627)(694.06573538,54.82651628)(693.98573975,54.82651611)
\curveto(693.91573553,54.82651628)(693.86573558,54.83151627)(693.83573975,54.84151611)
\curveto(693.79573565,54.85151625)(693.75573569,54.85651625)(693.71573975,54.85651611)
\curveto(693.67573577,54.84651626)(693.6407358,54.84651626)(693.61073975,54.85651611)
\lineto(693.52073975,54.85651611)
\lineto(693.16073975,54.90151611)
\curveto(693.02073642,54.94151616)(692.88573656,54.98151612)(692.75573975,55.02151611)
\curveto(692.62573682,55.06151604)(692.50073694,55.106516)(692.38073975,55.15651611)
\curveto(691.93073751,55.35651575)(691.56073788,55.61651549)(691.27073975,55.93651611)
\curveto(690.98073846,56.25651485)(690.7407387,56.64651446)(690.55073975,57.10651611)
\curveto(690.50073894,57.2065139)(690.46073898,57.3065138)(690.43073975,57.40651611)
\curveto(690.41073903,57.5065136)(690.39073905,57.61151349)(690.37073975,57.72151611)
\curveto(690.35073909,57.76151334)(690.3407391,57.79151331)(690.34073975,57.81151611)
\curveto(690.35073909,57.84151326)(690.35073909,57.87651323)(690.34073975,57.91651611)
\curveto(690.32073912,57.99651311)(690.30573914,58.07651303)(690.29573975,58.15651611)
\curveto(690.29573915,58.24651286)(690.28573916,58.33151277)(690.26573975,58.41151611)
\lineto(690.26573975,58.53151611)
\curveto(690.26573918,58.57151253)(690.26073918,58.61651249)(690.25073975,58.66651611)
\curveto(690.2407392,58.71651239)(690.23573921,58.8015123)(690.23573975,58.92151611)
\curveto(690.23573921,59.05151205)(690.2457392,59.14651196)(690.26573975,59.20651611)
\curveto(690.28573916,59.27651183)(690.29073915,59.34651176)(690.28073975,59.41651611)
\curveto(690.27073917,59.48651162)(690.27573917,59.55651155)(690.29573975,59.62651611)
\curveto(690.30573914,59.67651143)(690.31073913,59.71651139)(690.31073975,59.74651611)
\curveto(690.32073912,59.78651132)(690.33073911,59.83151127)(690.34073975,59.88151611)
\curveto(690.37073907,60.0015111)(690.39573905,60.12151098)(690.41573975,60.24151611)
\curveto(690.445739,60.36151074)(690.48573896,60.47651063)(690.53573975,60.58651611)
\curveto(690.68573876,60.95651015)(690.86573858,61.28650982)(691.07573975,61.57651611)
\curveto(691.29573815,61.87650923)(691.56073788,62.12650898)(691.87073975,62.32651611)
\curveto(691.99073745,62.4065087)(692.11573733,62.47150863)(692.24573975,62.52151611)
\curveto(692.37573707,62.58150852)(692.51073693,62.64150846)(692.65073975,62.70151611)
\curveto(692.77073667,62.75150835)(692.90073654,62.78150832)(693.04073975,62.79151611)
\curveto(693.18073626,62.81150829)(693.32073612,62.84150826)(693.46073975,62.88151611)
\lineto(693.65573975,62.88151611)
\curveto(693.72573572,62.89150821)(693.79073565,62.9015082)(693.85073975,62.91151611)
\curveto(694.7407347,62.92150818)(695.48073396,62.73650837)(696.07073975,62.35651611)
\curveto(696.66073278,61.97650913)(697.08573236,61.48150962)(697.34573975,60.87151611)
\curveto(697.39573205,60.77151033)(697.43573201,60.67151043)(697.46573975,60.57151611)
\curveto(697.49573195,60.47151063)(697.53073191,60.36651074)(697.57073975,60.25651611)
\curveto(697.60073184,60.14651096)(697.62573182,60.02651108)(697.64573975,59.89651611)
\curveto(697.66573178,59.77651133)(697.69073175,59.65151145)(697.72073975,59.52151611)
\curveto(697.73073171,59.47151163)(697.73073171,59.41651169)(697.72073975,59.35651611)
\curveto(697.72073172,59.3065118)(697.72573172,59.25651185)(697.73573975,59.20651611)
\moveto(696.40073975,58.35151611)
\curveto(696.42073302,58.42151268)(696.42573302,58.5015126)(696.41573975,58.59151611)
\lineto(696.41573975,58.84651611)
\curveto(696.41573303,59.23651187)(696.38073306,59.56651154)(696.31073975,59.83651611)
\curveto(696.28073316,59.91651119)(696.25573319,59.99651111)(696.23573975,60.07651611)
\curveto(696.21573323,60.15651095)(696.19073325,60.23151087)(696.16073975,60.30151611)
\curveto(695.88073356,60.95151015)(695.43573401,61.4015097)(694.82573975,61.65151611)
\curveto(694.75573469,61.68150942)(694.68073476,61.7015094)(694.60073975,61.71151611)
\lineto(694.36073975,61.77151611)
\curveto(694.28073516,61.79150931)(694.19573525,61.8015093)(694.10573975,61.80151611)
\lineto(693.83573975,61.80151611)
\lineto(693.56573975,61.75651611)
\curveto(693.46573598,61.73650937)(693.37073607,61.71150939)(693.28073975,61.68151611)
\curveto(693.20073624,61.66150944)(693.12073632,61.63150947)(693.04073975,61.59151611)
\curveto(692.97073647,61.57150953)(692.90573654,61.54150956)(692.84573975,61.50151611)
\curveto(692.78573666,61.46150964)(692.73073671,61.42150968)(692.68073975,61.38151611)
\curveto(692.440737,61.21150989)(692.2457372,61.0065101)(692.09573975,60.76651611)
\curveto(691.9457375,60.52651058)(691.81573763,60.24651086)(691.70573975,59.92651611)
\curveto(691.67573777,59.82651128)(691.65573779,59.72151138)(691.64573975,59.61151611)
\curveto(691.63573781,59.51151159)(691.62073782,59.4065117)(691.60073975,59.29651611)
\curveto(691.59073785,59.25651185)(691.58573786,59.19151191)(691.58573975,59.10151611)
\curveto(691.57573787,59.07151203)(691.57073787,59.03651207)(691.57073975,58.99651611)
\curveto(691.58073786,58.95651215)(691.58573786,58.91151219)(691.58573975,58.86151611)
\lineto(691.58573975,58.56151611)
\curveto(691.58573786,58.46151264)(691.59573785,58.37151273)(691.61573975,58.29151611)
\lineto(691.64573975,58.11151611)
\curveto(691.66573778,58.01151309)(691.68073776,57.91151319)(691.69073975,57.81151611)
\curveto(691.71073773,57.72151338)(691.7407377,57.63651347)(691.78073975,57.55651611)
\curveto(691.88073756,57.31651379)(691.99573745,57.09151401)(692.12573975,56.88151611)
\curveto(692.26573718,56.67151443)(692.43573701,56.49651461)(692.63573975,56.35651611)
\curveto(692.68573676,56.32651478)(692.73073671,56.3015148)(692.77073975,56.28151611)
\curveto(692.81073663,56.26151484)(692.85573659,56.23651487)(692.90573975,56.20651611)
\curveto(692.98573646,56.15651495)(693.07073637,56.11151499)(693.16073975,56.07151611)
\curveto(693.26073618,56.04151506)(693.36573608,56.01151509)(693.47573975,55.98151611)
\curveto(693.52573592,55.96151514)(693.57073587,55.95151515)(693.61073975,55.95151611)
\curveto(693.66073578,55.96151514)(693.71073573,55.96151514)(693.76073975,55.95151611)
\curveto(693.79073565,55.94151516)(693.85073559,55.93151517)(693.94073975,55.92151611)
\curveto(694.0407354,55.91151519)(694.11573533,55.91651519)(694.16573975,55.93651611)
\curveto(694.20573524,55.94651516)(694.2457352,55.94651516)(694.28573975,55.93651611)
\curveto(694.32573512,55.93651517)(694.36573508,55.94651516)(694.40573975,55.96651611)
\curveto(694.48573496,55.98651512)(694.56573488,56.0015151)(694.64573975,56.01151611)
\curveto(694.72573472,56.03151507)(694.80073464,56.05651505)(694.87073975,56.08651611)
\curveto(695.21073423,56.22651488)(695.48573396,56.42151468)(695.69573975,56.67151611)
\curveto(695.90573354,56.92151418)(696.08073336,57.21651389)(696.22073975,57.55651611)
\curveto(696.27073317,57.67651343)(696.30073314,57.8015133)(696.31073975,57.93151611)
\curveto(696.33073311,58.07151303)(696.36073308,58.21151289)(696.40073975,58.35151611)
}
}
{
\newrgbcolor{curcolor}{0 0 0}
\pscustom[linestyle=none,fillstyle=solid,fillcolor=curcolor]
{
\newpath
\moveto(701.654021,62.91151611)
\curveto(702.37401693,62.92150818)(702.97901633,62.83650827)(703.469021,62.65651611)
\curveto(703.95901535,62.48650862)(704.33901497,62.18150892)(704.609021,61.74151611)
\curveto(704.67901463,61.63150947)(704.73401457,61.51650959)(704.774021,61.39651611)
\curveto(704.81401449,61.28650982)(704.85401445,61.16150994)(704.894021,61.02151611)
\curveto(704.91401439,60.95151015)(704.91901439,60.87651023)(704.909021,60.79651611)
\curveto(704.89901441,60.72651038)(704.88401442,60.67151043)(704.864021,60.63151611)
\curveto(704.84401446,60.61151049)(704.81901449,60.59151051)(704.789021,60.57151611)
\curveto(704.75901455,60.56151054)(704.73401457,60.54651056)(704.714021,60.52651611)
\curveto(704.66401464,60.5065106)(704.61401469,60.5015106)(704.564021,60.51151611)
\curveto(704.51401479,60.52151058)(704.46401484,60.52151058)(704.414021,60.51151611)
\curveto(704.33401497,60.49151061)(704.22901508,60.48651062)(704.099021,60.49651611)
\curveto(703.96901534,60.51651059)(703.87901543,60.54151056)(703.829021,60.57151611)
\curveto(703.74901556,60.62151048)(703.69401561,60.68651042)(703.664021,60.76651611)
\curveto(703.64401566,60.85651025)(703.6090157,60.94151016)(703.559021,61.02151611)
\curveto(703.46901584,61.18150992)(703.34401596,61.32650978)(703.184021,61.45651611)
\curveto(703.07401623,61.53650957)(702.95401635,61.59650951)(702.824021,61.63651611)
\curveto(702.69401661,61.67650943)(702.55401675,61.71650939)(702.404021,61.75651611)
\curveto(702.35401695,61.77650933)(702.304017,61.78150932)(702.254021,61.77151611)
\curveto(702.2040171,61.77150933)(702.15401715,61.77650933)(702.104021,61.78651611)
\curveto(702.04401726,61.8065093)(701.96901734,61.81650929)(701.879021,61.81651611)
\curveto(701.78901752,61.81650929)(701.71401759,61.8065093)(701.654021,61.78651611)
\lineto(701.564021,61.78651611)
\lineto(701.414021,61.75651611)
\curveto(701.36401794,61.75650935)(701.31401799,61.75150935)(701.264021,61.74151611)
\curveto(701.0040183,61.68150942)(700.78901852,61.59650951)(700.619021,61.48651611)
\curveto(700.44901886,61.37650973)(700.33401897,61.19150991)(700.274021,60.93151611)
\curveto(700.25401905,60.86151024)(700.24901906,60.79151031)(700.259021,60.72151611)
\curveto(700.27901903,60.65151045)(700.29901901,60.59151051)(700.319021,60.54151611)
\curveto(700.37901893,60.39151071)(700.44901886,60.28151082)(700.529021,60.21151611)
\curveto(700.61901869,60.15151095)(700.72901858,60.08151102)(700.859021,60.00151611)
\curveto(701.01901829,59.9015112)(701.19901811,59.82651128)(701.399021,59.77651611)
\curveto(701.59901771,59.73651137)(701.79901751,59.68651142)(701.999021,59.62651611)
\curveto(702.12901718,59.58651152)(702.25901705,59.55651155)(702.389021,59.53651611)
\curveto(702.51901679,59.51651159)(702.64901666,59.48651162)(702.779021,59.44651611)
\curveto(702.98901632,59.38651172)(703.19401611,59.32651178)(703.394021,59.26651611)
\curveto(703.59401571,59.21651189)(703.79401551,59.15151195)(703.994021,59.07151611)
\lineto(704.144021,59.01151611)
\curveto(704.19401511,58.99151211)(704.24401506,58.96651214)(704.294021,58.93651611)
\curveto(704.49401481,58.81651229)(704.66901464,58.68151242)(704.819021,58.53151611)
\curveto(704.96901434,58.38151272)(705.09401421,58.19151291)(705.194021,57.96151611)
\curveto(705.21401409,57.89151321)(705.23401407,57.79651331)(705.254021,57.67651611)
\curveto(705.27401403,57.6065135)(705.28401402,57.53151357)(705.284021,57.45151611)
\curveto(705.29401401,57.38151372)(705.29901401,57.3015138)(705.299021,57.21151611)
\lineto(705.299021,57.06151611)
\curveto(705.27901403,56.99151411)(705.26901404,56.92151418)(705.269021,56.85151611)
\curveto(705.26901404,56.78151432)(705.25901405,56.71151439)(705.239021,56.64151611)
\curveto(705.2090141,56.53151457)(705.17401413,56.42651468)(705.134021,56.32651611)
\curveto(705.09401421,56.22651488)(705.04901426,56.13651497)(704.999021,56.05651611)
\curveto(704.83901447,55.79651531)(704.63401467,55.58651552)(704.384021,55.42651611)
\curveto(704.13401517,55.27651583)(703.85401545,55.14651596)(703.544021,55.03651611)
\curveto(703.45401585,55.0065161)(703.35901595,54.98651612)(703.259021,54.97651611)
\curveto(703.16901614,54.95651615)(703.07901623,54.93151617)(702.989021,54.90151611)
\curveto(702.88901642,54.88151622)(702.78901652,54.87151623)(702.689021,54.87151611)
\curveto(702.58901672,54.87151623)(702.48901682,54.86151624)(702.389021,54.84151611)
\lineto(702.239021,54.84151611)
\curveto(702.18901712,54.83151627)(702.11901719,54.82651628)(702.029021,54.82651611)
\curveto(701.93901737,54.82651628)(701.86901744,54.83151627)(701.819021,54.84151611)
\lineto(701.654021,54.84151611)
\curveto(701.59401771,54.86151624)(701.52901778,54.87151623)(701.459021,54.87151611)
\curveto(701.38901792,54.86151624)(701.32901798,54.86651624)(701.279021,54.88651611)
\curveto(701.22901808,54.89651621)(701.16401814,54.9015162)(701.084021,54.90151611)
\lineto(700.844021,54.96151611)
\curveto(700.77401853,54.97151613)(700.69901861,54.99151611)(700.619021,55.02151611)
\curveto(700.309019,55.12151598)(700.03901927,55.24651586)(699.809021,55.39651611)
\curveto(699.57901973,55.54651556)(699.37901993,55.74151536)(699.209021,55.98151611)
\curveto(699.11902019,56.11151499)(699.04402026,56.24651486)(698.984021,56.38651611)
\curveto(698.92402038,56.52651458)(698.86902044,56.68151442)(698.819021,56.85151611)
\curveto(698.79902051,56.91151419)(698.78902052,56.98151412)(698.789021,57.06151611)
\curveto(698.79902051,57.15151395)(698.81402049,57.22151388)(698.834021,57.27151611)
\curveto(698.86402044,57.31151379)(698.91402039,57.35151375)(698.984021,57.39151611)
\curveto(699.03402027,57.41151369)(699.1040202,57.42151368)(699.194021,57.42151611)
\curveto(699.28402002,57.43151367)(699.37401993,57.43151367)(699.464021,57.42151611)
\curveto(699.55401975,57.41151369)(699.63901967,57.39651371)(699.719021,57.37651611)
\curveto(699.8090195,57.36651374)(699.86901944,57.35151375)(699.899021,57.33151611)
\curveto(699.96901934,57.28151382)(700.01401929,57.2065139)(700.034021,57.10651611)
\curveto(700.06401924,57.01651409)(700.09901921,56.93151417)(700.139021,56.85151611)
\curveto(700.23901907,56.63151447)(700.37401893,56.46151464)(700.544021,56.34151611)
\curveto(700.66401864,56.25151485)(700.79901851,56.18151492)(700.949021,56.13151611)
\curveto(701.09901821,56.08151502)(701.25901805,56.03151507)(701.429021,55.98151611)
\lineto(701.744021,55.93651611)
\lineto(701.834021,55.93651611)
\curveto(701.9040174,55.91651519)(701.99401731,55.9065152)(702.104021,55.90651611)
\curveto(702.22401708,55.9065152)(702.32401698,55.91651519)(702.404021,55.93651611)
\curveto(702.47401683,55.93651517)(702.52901678,55.94151516)(702.569021,55.95151611)
\curveto(702.62901668,55.96151514)(702.68901662,55.96651514)(702.749021,55.96651611)
\curveto(702.8090165,55.97651513)(702.86401644,55.98651512)(702.914021,55.99651611)
\curveto(703.2040161,56.07651503)(703.43401587,56.18151492)(703.604021,56.31151611)
\curveto(703.77401553,56.44151466)(703.89401541,56.66151444)(703.964021,56.97151611)
\curveto(703.98401532,57.02151408)(703.98901532,57.07651403)(703.979021,57.13651611)
\curveto(703.96901534,57.19651391)(703.95901535,57.24151386)(703.949021,57.27151611)
\curveto(703.89901541,57.46151364)(703.82901548,57.6015135)(703.739021,57.69151611)
\curveto(703.64901566,57.79151331)(703.53401577,57.88151322)(703.394021,57.96151611)
\curveto(703.304016,58.02151308)(703.2040161,58.07151303)(703.094021,58.11151611)
\lineto(702.764021,58.23151611)
\curveto(702.73401657,58.24151286)(702.7040166,58.24651286)(702.674021,58.24651611)
\curveto(702.65401665,58.24651286)(702.62901668,58.25651285)(702.599021,58.27651611)
\curveto(702.25901705,58.38651272)(701.9040174,58.46651264)(701.534021,58.51651611)
\curveto(701.17401813,58.57651253)(700.83401847,58.67151243)(700.514021,58.80151611)
\curveto(700.41401889,58.84151226)(700.31901899,58.87651223)(700.229021,58.90651611)
\curveto(700.13901917,58.93651217)(700.05401925,58.97651213)(699.974021,59.02651611)
\curveto(699.78401952,59.13651197)(699.6090197,59.26151184)(699.449021,59.40151611)
\curveto(699.28902002,59.54151156)(699.16402014,59.71651139)(699.074021,59.92651611)
\curveto(699.04402026,59.99651111)(699.01902029,60.06651104)(698.999021,60.13651611)
\curveto(698.98902032,60.2065109)(698.97402033,60.28151082)(698.954021,60.36151611)
\curveto(698.92402038,60.48151062)(698.91402039,60.61651049)(698.924021,60.76651611)
\curveto(698.93402037,60.92651018)(698.94902036,61.06151004)(698.969021,61.17151611)
\curveto(698.98902032,61.22150988)(698.99902031,61.26150984)(698.999021,61.29151611)
\curveto(699.0090203,61.33150977)(699.02402028,61.37150973)(699.044021,61.41151611)
\curveto(699.13402017,61.64150946)(699.25402005,61.84150926)(699.404021,62.01151611)
\curveto(699.56401974,62.18150892)(699.74401956,62.33150877)(699.944021,62.46151611)
\curveto(700.09401921,62.55150855)(700.25901905,62.62150848)(700.439021,62.67151611)
\curveto(700.61901869,62.73150837)(700.8090185,62.78650832)(701.009021,62.83651611)
\curveto(701.07901823,62.84650826)(701.14401816,62.85650825)(701.204021,62.86651611)
\curveto(701.27401803,62.87650823)(701.34901796,62.88650822)(701.429021,62.89651611)
\curveto(701.45901785,62.9065082)(701.49901781,62.9065082)(701.549021,62.89651611)
\curveto(701.59901771,62.88650822)(701.63401767,62.89150821)(701.654021,62.91151611)
}
}
{
\newrgbcolor{curcolor}{0 0 0}
\pscustom[linestyle=none,fillstyle=solid,fillcolor=curcolor]
{
\newpath
\moveto(237.42056885,55.52150146)
\curveto(237.4405593,55.47150072)(237.46555928,55.41150078)(237.49556885,55.34150146)
\curveto(237.52555922,55.27150092)(237.5455592,55.19650099)(237.55556885,55.11650146)
\curveto(237.57555917,55.04650114)(237.57555917,54.97650121)(237.55556885,54.90650146)
\curveto(237.5455592,54.84650134)(237.50555924,54.80150139)(237.43556885,54.77150146)
\curveto(237.38555936,54.75150144)(237.32555942,54.74150145)(237.25556885,54.74150146)
\lineto(237.04556885,54.74150146)
\lineto(236.59556885,54.74150146)
\curveto(236.4455603,54.74150145)(236.32556042,54.76650142)(236.23556885,54.81650146)
\curveto(236.13556061,54.87650131)(236.06056068,54.98150121)(236.01056885,55.13150146)
\curveto(235.97056077,55.28150091)(235.92556082,55.41650077)(235.87556885,55.53650146)
\curveto(235.76556098,55.79650039)(235.66556108,56.06650012)(235.57556885,56.34650146)
\curveto(235.48556126,56.62649956)(235.38556136,56.90149929)(235.27556885,57.17150146)
\curveto(235.2455615,57.26149893)(235.21556153,57.34649884)(235.18556885,57.42650146)
\curveto(235.16556158,57.50649868)(235.13556161,57.58149861)(235.09556885,57.65150146)
\curveto(235.06556168,57.72149847)(235.02056172,57.78149841)(234.96056885,57.83150146)
\curveto(234.90056184,57.88149831)(234.82056192,57.92149827)(234.72056885,57.95150146)
\curveto(234.67056207,57.97149822)(234.61056213,57.97649821)(234.54056885,57.96650146)
\lineto(234.34556885,57.96650146)
\lineto(231.51056885,57.96650146)
\lineto(231.21056885,57.96650146)
\curveto(231.10056564,57.97649821)(230.99556575,57.97649821)(230.89556885,57.96650146)
\curveto(230.79556595,57.95649823)(230.70056604,57.94149825)(230.61056885,57.92150146)
\curveto(230.53056621,57.90149829)(230.47056627,57.86149833)(230.43056885,57.80150146)
\curveto(230.35056639,57.70149849)(230.29056645,57.5864986)(230.25056885,57.45650146)
\curveto(230.22056652,57.33649885)(230.18056656,57.21149898)(230.13056885,57.08150146)
\curveto(230.03056671,56.85149934)(229.93556681,56.61149958)(229.84556885,56.36150146)
\curveto(229.76556698,56.11150008)(229.67556707,55.87150032)(229.57556885,55.64150146)
\curveto(229.55556719,55.58150061)(229.53056721,55.51150068)(229.50056885,55.43150146)
\curveto(229.48056726,55.36150083)(229.45556729,55.2865009)(229.42556885,55.20650146)
\curveto(229.39556735,55.12650106)(229.36056738,55.05150114)(229.32056885,54.98150146)
\curveto(229.29056745,54.92150127)(229.25556749,54.87650131)(229.21556885,54.84650146)
\curveto(229.13556761,54.7865014)(229.02556772,54.75150144)(228.88556885,54.74150146)
\lineto(228.46556885,54.74150146)
\lineto(228.22556885,54.74150146)
\curveto(228.15556859,54.75150144)(228.09556865,54.77650141)(228.04556885,54.81650146)
\curveto(227.99556875,54.84650134)(227.96556878,54.8915013)(227.95556885,54.95150146)
\curveto(227.95556879,55.01150118)(227.96056878,55.07150112)(227.97056885,55.13150146)
\curveto(227.99056875,55.20150099)(228.01056873,55.26650092)(228.03056885,55.32650146)
\curveto(228.06056868,55.39650079)(228.08556866,55.44650074)(228.10556885,55.47650146)
\curveto(228.2455685,55.79650039)(228.37056837,56.11150008)(228.48056885,56.42150146)
\curveto(228.59056815,56.74149945)(228.71056803,57.06149913)(228.84056885,57.38150146)
\curveto(228.93056781,57.60149859)(229.01556773,57.81649837)(229.09556885,58.02650146)
\curveto(229.17556757,58.24649794)(229.26056748,58.46649772)(229.35056885,58.68650146)
\curveto(229.65056709,59.40649678)(229.93556681,60.13149606)(230.20556885,60.86150146)
\curveto(230.47556627,61.60149459)(230.76056598,62.33649385)(231.06056885,63.06650146)
\curveto(231.17056557,63.32649286)(231.27056547,63.5914926)(231.36056885,63.86150146)
\curveto(231.46056528,64.13149206)(231.56556518,64.39649179)(231.67556885,64.65650146)
\curveto(231.72556502,64.76649142)(231.77056497,64.8864913)(231.81056885,65.01650146)
\curveto(231.86056488,65.15649103)(231.93056481,65.25649093)(232.02056885,65.31650146)
\curveto(232.06056468,65.35649083)(232.12556462,65.3864908)(232.21556885,65.40650146)
\curveto(232.23556451,65.41649077)(232.25556449,65.41649077)(232.27556885,65.40650146)
\curveto(232.30556444,65.40649078)(232.33056441,65.41149078)(232.35056885,65.42150146)
\curveto(232.53056421,65.42149077)(232.740564,65.42149077)(232.98056885,65.42150146)
\curveto(233.22056352,65.43149076)(233.39556335,65.39649079)(233.50556885,65.31650146)
\curveto(233.58556316,65.25649093)(233.6455631,65.15649103)(233.68556885,65.01650146)
\curveto(233.73556301,64.8864913)(233.78556296,64.76649142)(233.83556885,64.65650146)
\curveto(233.93556281,64.42649176)(234.02556272,64.19649199)(234.10556885,63.96650146)
\curveto(234.18556256,63.73649245)(234.27556247,63.50649268)(234.37556885,63.27650146)
\curveto(234.45556229,63.07649311)(234.53056221,62.87149332)(234.60056885,62.66150146)
\curveto(234.68056206,62.45149374)(234.76556198,62.24649394)(234.85556885,62.04650146)
\curveto(235.15556159,61.31649487)(235.4405613,60.57649561)(235.71056885,59.82650146)
\curveto(235.99056075,59.0864971)(236.28556046,58.35149784)(236.59556885,57.62150146)
\curveto(236.63556011,57.53149866)(236.66556008,57.44649874)(236.68556885,57.36650146)
\curveto(236.71556003,57.2864989)(236.74556,57.20149899)(236.77556885,57.11150146)
\curveto(236.88555986,56.85149934)(236.99055975,56.5864996)(237.09056885,56.31650146)
\curveto(237.20055954,56.04650014)(237.31055943,55.78150041)(237.42056885,55.52150146)
\moveto(234.21056885,59.16650146)
\curveto(234.30056244,59.19649699)(234.35556239,59.24649694)(234.37556885,59.31650146)
\curveto(234.40556234,59.3864968)(234.41056233,59.46149673)(234.39056885,59.54150146)
\curveto(234.38056236,59.63149656)(234.35556239,59.71649647)(234.31556885,59.79650146)
\curveto(234.28556246,59.8864963)(234.25556249,59.96149623)(234.22556885,60.02150146)
\curveto(234.20556254,60.06149613)(234.19556255,60.09649609)(234.19556885,60.12650146)
\curveto(234.19556255,60.15649603)(234.18556256,60.191496)(234.16556885,60.23150146)
\lineto(234.07556885,60.47150146)
\curveto(234.05556269,60.56149563)(234.02556272,60.65149554)(233.98556885,60.74150146)
\curveto(233.83556291,61.10149509)(233.70056304,61.46649472)(233.58056885,61.83650146)
\curveto(233.47056327,62.21649397)(233.3405634,62.5864936)(233.19056885,62.94650146)
\curveto(233.1405636,63.05649313)(233.09556365,63.16649302)(233.05556885,63.27650146)
\curveto(233.02556372,63.3864928)(232.98556376,63.4914927)(232.93556885,63.59150146)
\curveto(232.91556383,63.64149255)(232.89056385,63.6864925)(232.86056885,63.72650146)
\curveto(232.8405639,63.77649241)(232.79056395,63.80149239)(232.71056885,63.80150146)
\curveto(232.69056405,63.78149241)(232.67056407,63.76649242)(232.65056885,63.75650146)
\curveto(232.63056411,63.74649244)(232.61056413,63.73149246)(232.59056885,63.71150146)
\curveto(232.55056419,63.66149253)(232.52056422,63.60649258)(232.50056885,63.54650146)
\curveto(232.48056426,63.49649269)(232.46056428,63.44149275)(232.44056885,63.38150146)
\curveto(232.39056435,63.27149292)(232.35056439,63.16149303)(232.32056885,63.05150146)
\curveto(232.29056445,62.94149325)(232.25056449,62.83149336)(232.20056885,62.72150146)
\curveto(232.03056471,62.33149386)(231.88056486,61.93649425)(231.75056885,61.53650146)
\curveto(231.63056511,61.13649505)(231.49056525,60.74649544)(231.33056885,60.36650146)
\lineto(231.27056885,60.21650146)
\curveto(231.26056548,60.16649602)(231.2455655,60.11649607)(231.22556885,60.06650146)
\lineto(231.13556885,59.82650146)
\curveto(231.10556564,59.74649644)(231.08056566,59.66649652)(231.06056885,59.58650146)
\curveto(231.0405657,59.53649665)(231.03056571,59.48149671)(231.03056885,59.42150146)
\curveto(231.0405657,59.36149683)(231.05556569,59.31149688)(231.07556885,59.27150146)
\curveto(231.12556562,59.191497)(231.23056551,59.14649704)(231.39056885,59.13650146)
\lineto(231.84056885,59.13650146)
\lineto(233.44556885,59.13650146)
\curveto(233.55556319,59.13649705)(233.69056305,59.13149706)(233.85056885,59.12150146)
\curveto(234.01056273,59.12149707)(234.13056261,59.13649705)(234.21056885,59.16650146)
}
}
{
\newrgbcolor{curcolor}{0 0 0}
\pscustom[linestyle=none,fillstyle=solid,fillcolor=curcolor]
{
\newpath
\moveto(242.18213135,62.64650146)
\curveto(242.41212656,62.64649354)(242.54212643,62.5864936)(242.57213135,62.46650146)
\curveto(242.60212637,62.35649383)(242.61712635,62.191494)(242.61713135,61.97150146)
\lineto(242.61713135,61.68650146)
\curveto(242.61712635,61.59649459)(242.59212638,61.52149467)(242.54213135,61.46150146)
\curveto(242.48212649,61.38149481)(242.39712657,61.33649485)(242.28713135,61.32650146)
\curveto(242.17712679,61.32649486)(242.0671269,61.31149488)(241.95713135,61.28150146)
\curveto(241.81712715,61.25149494)(241.68212729,61.22149497)(241.55213135,61.19150146)
\curveto(241.43212754,61.16149503)(241.31712765,61.12149507)(241.20713135,61.07150146)
\curveto(240.91712805,60.94149525)(240.68212829,60.76149543)(240.50213135,60.53150146)
\curveto(240.32212865,60.31149588)(240.1671288,60.05649613)(240.03713135,59.76650146)
\curveto(239.99712897,59.65649653)(239.967129,59.54149665)(239.94713135,59.42150146)
\curveto(239.92712904,59.31149688)(239.90212907,59.19649699)(239.87213135,59.07650146)
\curveto(239.86212911,59.02649716)(239.85712911,58.97649721)(239.85713135,58.92650146)
\curveto(239.8671291,58.87649731)(239.8671291,58.82649736)(239.85713135,58.77650146)
\curveto(239.82712914,58.65649753)(239.81212916,58.51649767)(239.81213135,58.35650146)
\curveto(239.82212915,58.20649798)(239.82712914,58.06149813)(239.82713135,57.92150146)
\lineto(239.82713135,56.07650146)
\lineto(239.82713135,55.73150146)
\curveto(239.82712914,55.61150058)(239.82212915,55.49650069)(239.81213135,55.38650146)
\curveto(239.80212917,55.27650091)(239.79712917,55.18150101)(239.79713135,55.10150146)
\curveto(239.80712916,55.02150117)(239.78712918,54.95150124)(239.73713135,54.89150146)
\curveto(239.68712928,54.82150137)(239.60712936,54.78150141)(239.49713135,54.77150146)
\curveto(239.39712957,54.76150143)(239.28712968,54.75650143)(239.16713135,54.75650146)
\lineto(238.89713135,54.75650146)
\curveto(238.84713012,54.77650141)(238.79713017,54.7915014)(238.74713135,54.80150146)
\curveto(238.70713026,54.82150137)(238.67713029,54.84650134)(238.65713135,54.87650146)
\curveto(238.60713036,54.94650124)(238.57713039,55.03150116)(238.56713135,55.13150146)
\lineto(238.56713135,55.46150146)
\lineto(238.56713135,56.61650146)
\lineto(238.56713135,60.77150146)
\lineto(238.56713135,61.80650146)
\lineto(238.56713135,62.10650146)
\curveto(238.57713039,62.20649398)(238.60713036,62.2914939)(238.65713135,62.36150146)
\curveto(238.68713028,62.40149379)(238.73713023,62.43149376)(238.80713135,62.45150146)
\curveto(238.88713008,62.47149372)(238.97213,62.48149371)(239.06213135,62.48150146)
\curveto(239.15212982,62.4914937)(239.24212973,62.4914937)(239.33213135,62.48150146)
\curveto(239.42212955,62.47149372)(239.49212948,62.45649373)(239.54213135,62.43650146)
\curveto(239.62212935,62.40649378)(239.6721293,62.34649384)(239.69213135,62.25650146)
\curveto(239.72212925,62.17649401)(239.73712923,62.0864941)(239.73713135,61.98650146)
\lineto(239.73713135,61.68650146)
\curveto(239.73712923,61.5864946)(239.75712921,61.49649469)(239.79713135,61.41650146)
\curveto(239.80712916,61.39649479)(239.81712915,61.38149481)(239.82713135,61.37150146)
\lineto(239.87213135,61.32650146)
\curveto(239.98212899,61.32649486)(240.0721289,61.37149482)(240.14213135,61.46150146)
\curveto(240.21212876,61.56149463)(240.2721287,61.64149455)(240.32213135,61.70150146)
\lineto(240.41213135,61.79150146)
\curveto(240.50212847,61.90149429)(240.62712834,62.01649417)(240.78713135,62.13650146)
\curveto(240.94712802,62.25649393)(241.09712787,62.34649384)(241.23713135,62.40650146)
\curveto(241.32712764,62.45649373)(241.42212755,62.4914937)(241.52213135,62.51150146)
\curveto(241.62212735,62.54149365)(241.72712724,62.57149362)(241.83713135,62.60150146)
\curveto(241.89712707,62.61149358)(241.95712701,62.61649357)(242.01713135,62.61650146)
\curveto(242.07712689,62.62649356)(242.13212684,62.63649355)(242.18213135,62.64650146)
}
}
{
\newrgbcolor{curcolor}{0 0 0}
\pscustom[linestyle=none,fillstyle=solid,fillcolor=curcolor]
{
\newpath
\moveto(246.68189697,62.64650146)
\curveto(247.42189218,62.65649353)(248.03689157,62.54649364)(248.52689697,62.31650146)
\curveto(249.02689058,62.09649409)(249.42189018,61.76149443)(249.71189697,61.31150146)
\curveto(249.84188976,61.11149508)(249.95188965,60.86649532)(250.04189697,60.57650146)
\curveto(250.06188954,60.52649566)(250.07688953,60.46149573)(250.08689697,60.38150146)
\curveto(250.09688951,60.30149589)(250.09188951,60.23149596)(250.07189697,60.17150146)
\curveto(250.04188956,60.12149607)(249.99188961,60.07649611)(249.92189697,60.03650146)
\curveto(249.89188971,60.01649617)(249.86188974,60.00649618)(249.83189697,60.00650146)
\curveto(249.8018898,60.01649617)(249.76688984,60.01649617)(249.72689697,60.00650146)
\curveto(249.68688992,59.99649619)(249.64688996,59.9914962)(249.60689697,59.99150146)
\curveto(249.56689004,60.00149619)(249.52689008,60.00649618)(249.48689697,60.00650146)
\lineto(249.17189697,60.00650146)
\curveto(249.07189053,60.01649617)(248.98689062,60.04649614)(248.91689697,60.09650146)
\curveto(248.83689077,60.15649603)(248.78189082,60.24149595)(248.75189697,60.35150146)
\curveto(248.72189088,60.46149573)(248.68189092,60.55649563)(248.63189697,60.63650146)
\curveto(248.48189112,60.89649529)(248.28689132,61.10149509)(248.04689697,61.25150146)
\curveto(247.96689164,61.30149489)(247.88189172,61.34149485)(247.79189697,61.37150146)
\curveto(247.7018919,61.41149478)(247.606892,61.44649474)(247.50689697,61.47650146)
\curveto(247.36689224,61.51649467)(247.18189242,61.53649465)(246.95189697,61.53650146)
\curveto(246.72189288,61.54649464)(246.53189307,61.52649466)(246.38189697,61.47650146)
\curveto(246.31189329,61.45649473)(246.24689336,61.44149475)(246.18689697,61.43150146)
\curveto(246.12689348,61.42149477)(246.06189354,61.40649478)(245.99189697,61.38650146)
\curveto(245.73189387,61.27649491)(245.5018941,61.12649506)(245.30189697,60.93650146)
\curveto(245.1018945,60.74649544)(244.94689466,60.52149567)(244.83689697,60.26150146)
\curveto(244.79689481,60.17149602)(244.76189484,60.07649611)(244.73189697,59.97650146)
\curveto(244.7018949,59.8864963)(244.67189493,59.7864964)(244.64189697,59.67650146)
\lineto(244.55189697,59.27150146)
\curveto(244.54189506,59.22149697)(244.53689507,59.16649702)(244.53689697,59.10650146)
\curveto(244.54689506,59.04649714)(244.54189506,58.9914972)(244.52189697,58.94150146)
\lineto(244.52189697,58.82150146)
\curveto(244.51189509,58.78149741)(244.5018951,58.71649747)(244.49189697,58.62650146)
\curveto(244.49189511,58.53649765)(244.5018951,58.47149772)(244.52189697,58.43150146)
\curveto(244.53189507,58.38149781)(244.53189507,58.33149786)(244.52189697,58.28150146)
\curveto(244.51189509,58.23149796)(244.51189509,58.18149801)(244.52189697,58.13150146)
\curveto(244.53189507,58.0914981)(244.53689507,58.02149817)(244.53689697,57.92150146)
\curveto(244.55689505,57.84149835)(244.57189503,57.75649843)(244.58189697,57.66650146)
\curveto(244.601895,57.57649861)(244.62189498,57.4914987)(244.64189697,57.41150146)
\curveto(244.75189485,57.0914991)(244.87689473,56.81149938)(245.01689697,56.57150146)
\curveto(245.16689444,56.34149985)(245.37189423,56.14150005)(245.63189697,55.97150146)
\curveto(245.72189388,55.92150027)(245.81189379,55.87650031)(245.90189697,55.83650146)
\curveto(246.0018936,55.79650039)(246.1068935,55.75650043)(246.21689697,55.71650146)
\curveto(246.26689334,55.70650048)(246.3068933,55.70150049)(246.33689697,55.70150146)
\curveto(246.36689324,55.70150049)(246.4068932,55.69650049)(246.45689697,55.68650146)
\curveto(246.48689312,55.67650051)(246.53689307,55.67150052)(246.60689697,55.67150146)
\lineto(246.77189697,55.67150146)
\curveto(246.77189283,55.66150053)(246.79189281,55.65650053)(246.83189697,55.65650146)
\curveto(246.85189275,55.66650052)(246.87689273,55.66650052)(246.90689697,55.65650146)
\curveto(246.93689267,55.65650053)(246.96689264,55.66150053)(246.99689697,55.67150146)
\curveto(247.06689254,55.6915005)(247.13189247,55.69650049)(247.19189697,55.68650146)
\curveto(247.26189234,55.6865005)(247.33189227,55.69650049)(247.40189697,55.71650146)
\curveto(247.66189194,55.79650039)(247.88689172,55.89650029)(248.07689697,56.01650146)
\curveto(248.26689134,56.14650004)(248.42689118,56.31149988)(248.55689697,56.51150146)
\curveto(248.606891,56.5914996)(248.65189095,56.67649951)(248.69189697,56.76650146)
\lineto(248.81189697,57.03650146)
\curveto(248.83189077,57.11649907)(248.85189075,57.191499)(248.87189697,57.26150146)
\curveto(248.9018907,57.34149885)(248.95189065,57.40649878)(249.02189697,57.45650146)
\curveto(249.05189055,57.4864987)(249.11189049,57.50649868)(249.20189697,57.51650146)
\curveto(249.29189031,57.53649865)(249.38689022,57.54649864)(249.48689697,57.54650146)
\curveto(249.59689001,57.55649863)(249.69688991,57.55649863)(249.78689697,57.54650146)
\curveto(249.88688972,57.53649865)(249.95688965,57.51649867)(249.99689697,57.48650146)
\curveto(250.05688955,57.44649874)(250.09188951,57.3864988)(250.10189697,57.30650146)
\curveto(250.12188948,57.22649896)(250.12188948,57.14149905)(250.10189697,57.05150146)
\curveto(250.05188955,56.90149929)(250.0018896,56.75649943)(249.95189697,56.61650146)
\curveto(249.91188969,56.4864997)(249.85688975,56.35649983)(249.78689697,56.22650146)
\curveto(249.63688997,55.92650026)(249.44689016,55.66150053)(249.21689697,55.43150146)
\curveto(248.99689061,55.20150099)(248.72689088,55.01650117)(248.40689697,54.87650146)
\curveto(248.32689128,54.83650135)(248.24189136,54.80150139)(248.15189697,54.77150146)
\curveto(248.06189154,54.75150144)(247.96689164,54.72650146)(247.86689697,54.69650146)
\curveto(247.75689185,54.65650153)(247.64689196,54.63650155)(247.53689697,54.63650146)
\curveto(247.42689218,54.62650156)(247.31689229,54.61150158)(247.20689697,54.59150146)
\curveto(247.16689244,54.57150162)(247.12689248,54.56650162)(247.08689697,54.57650146)
\curveto(247.04689256,54.5865016)(247.0068926,54.5865016)(246.96689697,54.57650146)
\lineto(246.83189697,54.57650146)
\lineto(246.59189697,54.57650146)
\curveto(246.52189308,54.56650162)(246.45689315,54.57150162)(246.39689697,54.59150146)
\lineto(246.32189697,54.59150146)
\lineto(245.96189697,54.63650146)
\curveto(245.83189377,54.67650151)(245.7068939,54.71150148)(245.58689697,54.74150146)
\curveto(245.46689414,54.77150142)(245.35189425,54.81150138)(245.24189697,54.86150146)
\curveto(244.88189472,55.02150117)(244.58189502,55.21150098)(244.34189697,55.43150146)
\curveto(244.11189549,55.65150054)(243.89689571,55.92150027)(243.69689697,56.24150146)
\curveto(243.64689596,56.32149987)(243.601896,56.41149978)(243.56189697,56.51150146)
\lineto(243.44189697,56.81150146)
\curveto(243.39189621,56.92149927)(243.35689625,57.03649915)(243.33689697,57.15650146)
\curveto(243.31689629,57.27649891)(243.29189631,57.39649879)(243.26189697,57.51650146)
\curveto(243.25189635,57.55649863)(243.24689636,57.59649859)(243.24689697,57.63650146)
\curveto(243.24689636,57.67649851)(243.24189636,57.71649847)(243.23189697,57.75650146)
\curveto(243.21189639,57.81649837)(243.2018964,57.88149831)(243.20189697,57.95150146)
\curveto(243.21189639,58.02149817)(243.2068964,58.0864981)(243.18689697,58.14650146)
\lineto(243.18689697,58.29650146)
\curveto(243.17689643,58.34649784)(243.17189643,58.41649777)(243.17189697,58.50650146)
\curveto(243.17189643,58.59649759)(243.17689643,58.66649752)(243.18689697,58.71650146)
\curveto(243.19689641,58.76649742)(243.19689641,58.81149738)(243.18689697,58.85150146)
\curveto(243.18689642,58.8914973)(243.19189641,58.93149726)(243.20189697,58.97150146)
\curveto(243.22189638,59.04149715)(243.22689638,59.11149708)(243.21689697,59.18150146)
\curveto(243.21689639,59.25149694)(243.22689638,59.31649687)(243.24689697,59.37650146)
\curveto(243.28689632,59.54649664)(243.32189628,59.71649647)(243.35189697,59.88650146)
\curveto(243.38189622,60.05649613)(243.42689618,60.21649597)(243.48689697,60.36650146)
\curveto(243.69689591,60.8864953)(243.95189565,61.30649488)(244.25189697,61.62650146)
\curveto(244.55189505,61.94649424)(244.96189464,62.21149398)(245.48189697,62.42150146)
\curveto(245.59189401,62.47149372)(245.71189389,62.50649368)(245.84189697,62.52650146)
\curveto(245.97189363,62.54649364)(246.1068935,62.57149362)(246.24689697,62.60150146)
\curveto(246.31689329,62.61149358)(246.38689322,62.61649357)(246.45689697,62.61650146)
\curveto(246.52689308,62.62649356)(246.601893,62.63649355)(246.68189697,62.64650146)
}
}
{
\newrgbcolor{curcolor}{0 0 0}
\pscustom[linestyle=none,fillstyle=solid,fillcolor=curcolor]
{
\newpath
\moveto(252.0685376,65.40650146)
\curveto(252.20853608,65.40649078)(252.35353593,65.40149079)(252.5035376,65.39150146)
\curveto(252.66353562,65.3914908)(252.77353551,65.35149084)(252.8335376,65.27150146)
\curveto(252.8835354,65.20149099)(252.90853538,65.09649109)(252.9085376,64.95650146)
\lineto(252.9085376,64.56650146)
\lineto(252.9085376,62.97650146)
\lineto(252.9085376,62.52650146)
\curveto(252.90853538,62.4864937)(252.90353538,62.44649374)(252.8935376,62.40650146)
\curveto(252.89353539,62.36649382)(252.89853539,62.32649386)(252.9085376,62.28650146)
\curveto(252.91853537,62.25649393)(252.91853537,62.22149397)(252.9085376,62.18150146)
\curveto(252.90853538,62.14149405)(252.91353537,62.10649408)(252.9235376,62.07650146)
\curveto(252.94353534,61.99649419)(252.95353533,61.92149427)(252.9535376,61.85150146)
\curveto(252.96353532,61.78149441)(253.01853527,61.74649444)(253.1185376,61.74650146)
\curveto(253.13853515,61.76649442)(253.15853513,61.77649441)(253.1785376,61.77650146)
\curveto(253.20853508,61.7864944)(253.23353505,61.80149439)(253.2535376,61.82150146)
\curveto(253.31353497,61.86149433)(253.36853492,61.90149429)(253.4185376,61.94150146)
\curveto(253.46853482,61.9914942)(253.52353476,62.04149415)(253.5835376,62.09150146)
\curveto(253.69353459,62.17149402)(253.81353447,62.24149395)(253.9435376,62.30150146)
\curveto(254.0835342,62.37149382)(254.22353406,62.43149376)(254.3635376,62.48150146)
\curveto(254.44353384,62.50149369)(254.52853376,62.51649367)(254.6185376,62.52650146)
\curveto(254.70853358,62.54649364)(254.79353349,62.56649362)(254.8735376,62.58650146)
\curveto(254.91353337,62.60649358)(254.95353333,62.61149358)(254.9935376,62.60150146)
\curveto(255.03353325,62.60149359)(255.07853321,62.60649358)(255.1285376,62.61650146)
\curveto(255.17853311,62.63649355)(255.25853303,62.64649354)(255.3685376,62.64650146)
\curveto(255.4885328,62.64649354)(255.57353271,62.63649355)(255.6235376,62.61650146)
\lineto(255.7585376,62.61650146)
\curveto(255.80853248,62.61649357)(255.85853243,62.61149358)(255.9085376,62.60150146)
\curveto(255.9885323,62.58149361)(256.06853222,62.56649362)(256.1485376,62.55650146)
\curveto(256.22853206,62.54649364)(256.30853198,62.53149366)(256.3885376,62.51150146)
\curveto(256.43853185,62.4914937)(256.47853181,62.47649371)(256.5085376,62.46650146)
\curveto(256.53853175,62.46649372)(256.57853171,62.45649373)(256.6285376,62.43650146)
\curveto(256.73853155,62.3864938)(256.84353144,62.33149386)(256.9435376,62.27150146)
\curveto(257.04353124,62.22149397)(257.14353114,62.16149403)(257.2435376,62.09150146)
\curveto(257.34353094,62.00149419)(257.43853085,61.89649429)(257.5285376,61.77650146)
\curveto(257.5885307,61.6864945)(257.64353064,61.59649459)(257.6935376,61.50650146)
\curveto(257.74353054,61.41649477)(257.79353049,61.31649487)(257.8435376,61.20650146)
\curveto(257.87353041,61.13649505)(257.89353039,61.06649512)(257.9035376,60.99650146)
\curveto(257.92353036,60.92649526)(257.94353034,60.85149534)(257.9635376,60.77150146)
\curveto(257.9835303,60.72149547)(257.99353029,60.67149552)(257.9935376,60.62150146)
\curveto(257.99353029,60.57149562)(257.99853029,60.51649567)(258.0085376,60.45650146)
\curveto(258.02853026,60.40649578)(258.03353025,60.35649583)(258.0235376,60.30650146)
\curveto(258.02353026,60.25649593)(258.03353025,60.20649598)(258.0535376,60.15650146)
\lineto(258.0535376,60.00650146)
\curveto(258.06353022,59.95649623)(258.06353022,59.90149629)(258.0535376,59.84150146)
\lineto(258.0535376,59.67650146)
\lineto(258.0535376,59.03150146)
\lineto(258.0535376,55.91150146)
\lineto(258.0535376,55.61150146)
\curveto(258.06353022,55.50150069)(258.06353022,55.3915008)(258.0535376,55.28150146)
\curveto(258.05353023,55.18150101)(258.04353024,55.0865011)(258.0235376,54.99650146)
\curveto(258.00353028,54.90650128)(257.97353031,54.84650134)(257.9335376,54.81650146)
\curveto(257.86353042,54.75650143)(257.73353055,54.72650146)(257.5435376,54.72650146)
\lineto(257.1535376,54.72650146)
\curveto(257.03353125,54.72650146)(256.94353134,54.76650142)(256.8835376,54.84650146)
\curveto(256.83353145,54.91650127)(256.80853148,54.99650119)(256.8085376,55.08650146)
\lineto(256.8085376,55.40150146)
\lineto(256.8085376,56.48150146)
\lineto(256.8085376,58.88150146)
\lineto(256.8085376,59.73650146)
\curveto(256.81853147,60.04649614)(256.7885315,60.31149588)(256.7185376,60.53150146)
\curveto(256.59853169,60.87149532)(256.39853189,61.12149507)(256.1185376,61.28150146)
\curveto(256.03853225,61.33149486)(255.95353233,61.37149482)(255.8635376,61.40150146)
\curveto(255.77353251,61.44149475)(255.67353261,61.47149472)(255.5635376,61.49150146)
\curveto(255.52353276,61.50149469)(255.46353282,61.50649468)(255.3835376,61.50650146)
\curveto(255.34353294,61.51649467)(255.29353299,61.52649466)(255.2335376,61.53650146)
\curveto(255.1835331,61.54649464)(255.13353315,61.54149465)(255.0835376,61.52150146)
\curveto(255.04353324,61.51149468)(255.00353328,61.50649468)(254.9635376,61.50650146)
\curveto(254.92353336,61.51649467)(254.87853341,61.51649467)(254.8285376,61.50650146)
\curveto(254.73853355,61.4864947)(254.64353364,61.46649472)(254.5435376,61.44650146)
\curveto(254.45353383,61.43649475)(254.36853392,61.41149478)(254.2885376,61.37150146)
\curveto(253.94853434,61.23149496)(253.67853461,61.04149515)(253.4785376,60.80150146)
\curveto(253.27853501,60.56149563)(253.12353516,60.25649593)(253.0135376,59.88650146)
\curveto(252.99353529,59.81649637)(252.97853531,59.74149645)(252.9685376,59.66150146)
\curveto(252.95853533,59.58149661)(252.94353534,59.50149669)(252.9235376,59.42150146)
\curveto(252.91353537,59.3914968)(252.90853538,59.35649683)(252.9085376,59.31650146)
\curveto(252.91853537,59.2864969)(252.91853537,59.25649693)(252.9085376,59.22650146)
\curveto(252.89853539,59.17649701)(252.89853539,59.12649706)(252.9085376,59.07650146)
\curveto(252.91853537,59.02649716)(252.91853537,58.97649721)(252.9085376,58.92650146)
\lineto(252.9085376,55.91150146)
\lineto(252.9085376,55.62650146)
\curveto(252.91853537,55.52650066)(252.91853537,55.42650076)(252.9085376,55.32650146)
\curveto(252.90853538,55.22650096)(252.90353538,55.13150106)(252.8935376,55.04150146)
\curveto(252.8835354,54.95150124)(252.86353542,54.8865013)(252.8335376,54.84650146)
\curveto(252.79353549,54.79650139)(252.74353554,54.76650142)(252.6835376,54.75650146)
\curveto(252.63353565,54.75650143)(252.57353571,54.74650144)(252.5035376,54.72650146)
\lineto(252.2935376,54.72650146)
\lineto(251.9785376,54.72650146)
\curveto(251.87853641,54.73650145)(251.80353648,54.77150142)(251.7535376,54.83150146)
\curveto(251.70353658,54.91150128)(251.67353661,55.00650118)(251.6635376,55.11650146)
\lineto(251.6635376,55.49150146)
\lineto(251.6635376,56.87150146)
\lineto(251.6635376,63.12650146)
\lineto(251.6635376,64.59650146)
\curveto(251.66353662,64.70649148)(251.65853663,64.82149137)(251.6485376,64.94150146)
\curveto(251.64853664,65.07149112)(251.67353661,65.17149102)(251.7235376,65.24150146)
\curveto(251.76353652,65.31149088)(251.83853645,65.36149083)(251.9485376,65.39150146)
\curveto(251.96853632,65.40149079)(251.9885363,65.40149079)(252.0085376,65.39150146)
\curveto(252.02853626,65.3914908)(252.04853624,65.39649079)(252.0685376,65.40650146)
}
}
{
\newrgbcolor{curcolor}{0 0 0}
\pscustom[linestyle=none,fillstyle=solid,fillcolor=curcolor]
{
\newpath
\moveto(260.23814697,63.96650146)
\curveto(260.15814585,64.02649216)(260.1131459,64.13149206)(260.10314697,64.28150146)
\lineto(260.10314697,64.74650146)
\lineto(260.10314697,65.00150146)
\curveto(260.10314591,65.0914911)(260.11814589,65.16649102)(260.14814697,65.22650146)
\curveto(260.18814582,65.30649088)(260.26814574,65.36649082)(260.38814697,65.40650146)
\curveto(260.4081456,65.41649077)(260.42814558,65.41649077)(260.44814697,65.40650146)
\curveto(260.47814553,65.40649078)(260.50314551,65.41149078)(260.52314697,65.42150146)
\curveto(260.69314532,65.42149077)(260.85314516,65.41649077)(261.00314697,65.40650146)
\curveto(261.15314486,65.39649079)(261.25314476,65.33649085)(261.30314697,65.22650146)
\curveto(261.33314468,65.16649102)(261.34814466,65.0914911)(261.34814697,65.00150146)
\lineto(261.34814697,64.74650146)
\curveto(261.34814466,64.56649162)(261.34314467,64.39649179)(261.33314697,64.23650146)
\curveto(261.33314468,64.07649211)(261.26814474,63.97149222)(261.13814697,63.92150146)
\curveto(261.08814492,63.90149229)(261.03314498,63.8914923)(260.97314697,63.89150146)
\lineto(260.80814697,63.89150146)
\lineto(260.49314697,63.89150146)
\curveto(260.39314562,63.8914923)(260.3081457,63.91649227)(260.23814697,63.96650146)
\moveto(261.34814697,55.46150146)
\lineto(261.34814697,55.14650146)
\curveto(261.35814465,55.04650114)(261.33814467,54.96650122)(261.28814697,54.90650146)
\curveto(261.25814475,54.84650134)(261.2131448,54.80650138)(261.15314697,54.78650146)
\curveto(261.09314492,54.77650141)(261.02314499,54.76150143)(260.94314697,54.74150146)
\lineto(260.71814697,54.74150146)
\curveto(260.58814542,54.74150145)(260.47314554,54.74650144)(260.37314697,54.75650146)
\curveto(260.28314573,54.77650141)(260.2131458,54.82650136)(260.16314697,54.90650146)
\curveto(260.12314589,54.96650122)(260.10314591,55.04150115)(260.10314697,55.13150146)
\lineto(260.10314697,55.41650146)
\lineto(260.10314697,61.76150146)
\lineto(260.10314697,62.07650146)
\curveto(260.10314591,62.186494)(260.12814588,62.27149392)(260.17814697,62.33150146)
\curveto(260.2081458,62.38149381)(260.24814576,62.41149378)(260.29814697,62.42150146)
\curveto(260.34814566,62.43149376)(260.40314561,62.44649374)(260.46314697,62.46650146)
\curveto(260.48314553,62.46649372)(260.50314551,62.46149373)(260.52314697,62.45150146)
\curveto(260.55314546,62.45149374)(260.57814543,62.45649373)(260.59814697,62.46650146)
\curveto(260.72814528,62.46649372)(260.85814515,62.46149373)(260.98814697,62.45150146)
\curveto(261.12814488,62.45149374)(261.22314479,62.41149378)(261.27314697,62.33150146)
\curveto(261.32314469,62.27149392)(261.34814466,62.191494)(261.34814697,62.09150146)
\lineto(261.34814697,61.80650146)
\lineto(261.34814697,55.46150146)
}
}
{
\newrgbcolor{curcolor}{0 0 0}
\pscustom[linestyle=none,fillstyle=solid,fillcolor=curcolor]
{
\newpath
\moveto(263.06799072,62.46650146)
\lineto(263.54799072,62.46650146)
\curveto(263.71798938,62.46649372)(263.84798925,62.43649375)(263.93799072,62.37650146)
\curveto(264.00798909,62.32649386)(264.05298905,62.26149393)(264.07299072,62.18150146)
\curveto(264.102989,62.11149408)(264.13298897,62.03649415)(264.16299072,61.95650146)
\curveto(264.22298888,61.81649437)(264.27298883,61.67649451)(264.31299072,61.53650146)
\curveto(264.35298875,61.39649479)(264.3979887,61.25649493)(264.44799072,61.11650146)
\curveto(264.64798845,60.57649561)(264.83298827,60.03149616)(265.00299072,59.48150146)
\curveto(265.17298793,58.94149725)(265.35798774,58.40149779)(265.55799072,57.86150146)
\curveto(265.62798747,57.68149851)(265.68798741,57.49649869)(265.73799072,57.30650146)
\curveto(265.78798731,57.12649906)(265.85298725,56.94649924)(265.93299072,56.76650146)
\curveto(265.95298715,56.69649949)(265.97798712,56.62149957)(266.00799072,56.54150146)
\curveto(266.03798706,56.46149973)(266.08798701,56.41149978)(266.15799072,56.39150146)
\curveto(266.23798686,56.37149982)(266.2979868,56.40649978)(266.33799072,56.49650146)
\curveto(266.38798671,56.5864996)(266.42298668,56.65649953)(266.44299072,56.70650146)
\curveto(266.52298658,56.89649929)(266.58798651,57.0864991)(266.63799072,57.27650146)
\curveto(266.6979864,57.47649871)(266.76298634,57.67649851)(266.83299072,57.87650146)
\curveto(266.96298614,58.25649793)(267.08798601,58.63149756)(267.20799072,59.00150146)
\curveto(267.32798577,59.38149681)(267.45298565,59.76149643)(267.58299072,60.14150146)
\curveto(267.63298547,60.31149588)(267.68298542,60.47649571)(267.73299072,60.63650146)
\curveto(267.78298532,60.80649538)(267.84298526,60.97149522)(267.91299072,61.13150146)
\curveto(267.96298514,61.27149492)(268.00798509,61.41149478)(268.04799072,61.55150146)
\curveto(268.08798501,61.6914945)(268.13298497,61.83149436)(268.18299072,61.97150146)
\curveto(268.2029849,62.04149415)(268.22798487,62.11149408)(268.25799072,62.18150146)
\curveto(268.28798481,62.25149394)(268.32798477,62.31149388)(268.37799072,62.36150146)
\curveto(268.45798464,62.41149378)(268.54798455,62.44149375)(268.64799072,62.45150146)
\curveto(268.74798435,62.46149373)(268.86798423,62.46649372)(269.00799072,62.46650146)
\curveto(269.07798402,62.46649372)(269.14298396,62.46149373)(269.20299072,62.45150146)
\curveto(269.26298384,62.45149374)(269.31798378,62.44149375)(269.36799072,62.42150146)
\curveto(269.45798364,62.38149381)(269.5029836,62.31649387)(269.50299072,62.22650146)
\curveto(269.51298359,62.13649405)(269.4979836,62.04649414)(269.45799072,61.95650146)
\curveto(269.3979837,61.7864944)(269.33798376,61.61149458)(269.27799072,61.43150146)
\curveto(269.21798388,61.25149494)(269.14798395,61.07649511)(269.06799072,60.90650146)
\curveto(269.04798405,60.85649533)(269.03298407,60.80649538)(269.02299072,60.75650146)
\curveto(269.01298409,60.71649547)(268.9979841,60.67149552)(268.97799072,60.62150146)
\curveto(268.8979842,60.45149574)(268.83298427,60.27649591)(268.78299072,60.09650146)
\curveto(268.73298437,59.91649627)(268.66798443,59.73649645)(268.58799072,59.55650146)
\curveto(268.53798456,59.42649676)(268.48798461,59.2914969)(268.43799072,59.15150146)
\curveto(268.3979847,59.02149717)(268.34798475,58.8914973)(268.28799072,58.76150146)
\curveto(268.11798498,58.35149784)(267.96298514,57.93649825)(267.82299072,57.51650146)
\curveto(267.69298541,57.09649909)(267.54298556,56.68149951)(267.37299072,56.27150146)
\curveto(267.31298579,56.11150008)(267.25798584,55.95150024)(267.20799072,55.79150146)
\curveto(267.15798594,55.63150056)(267.097986,55.47150072)(267.02799072,55.31150146)
\curveto(266.97798612,55.20150099)(266.93298617,55.09650109)(266.89299072,54.99650146)
\curveto(266.86298624,54.90650128)(266.79298631,54.83650135)(266.68299072,54.78650146)
\curveto(266.62298648,54.75650143)(266.55298655,54.74150145)(266.47299072,54.74150146)
\lineto(266.24799072,54.74150146)
\lineto(265.78299072,54.74150146)
\curveto(265.63298747,54.75150144)(265.52298758,54.80150139)(265.45299072,54.89150146)
\curveto(265.38298772,54.97150122)(265.33298777,55.06650112)(265.30299072,55.17650146)
\curveto(265.27298783,55.29650089)(265.23298787,55.41150078)(265.18299072,55.52150146)
\curveto(265.12298798,55.66150053)(265.06298804,55.80650038)(265.00299072,55.95650146)
\curveto(264.95298815,56.11650007)(264.9029882,56.26649992)(264.85299072,56.40650146)
\curveto(264.83298827,56.45649973)(264.81798828,56.49649969)(264.80799072,56.52650146)
\curveto(264.7979883,56.56649962)(264.78298832,56.61149958)(264.76299072,56.66150146)
\curveto(264.56298854,57.14149905)(264.37798872,57.62649856)(264.20799072,58.11650146)
\curveto(264.04798905,58.60649758)(263.86798923,59.0914971)(263.66799072,59.57150146)
\curveto(263.60798949,59.73149646)(263.54798955,59.8864963)(263.48799072,60.03650146)
\curveto(263.43798966,60.19649599)(263.38298972,60.35649583)(263.32299072,60.51650146)
\lineto(263.26299072,60.66650146)
\curveto(263.25298985,60.72649546)(263.23798986,60.78149541)(263.21799072,60.83150146)
\curveto(263.13798996,61.00149519)(263.06799003,61.17149502)(263.00799072,61.34150146)
\curveto(262.95799014,61.51149468)(262.8979902,61.68149451)(262.82799072,61.85150146)
\curveto(262.80799029,61.91149428)(262.78299032,61.9914942)(262.75299072,62.09150146)
\curveto(262.72299038,62.191494)(262.72799037,62.27649391)(262.76799072,62.34650146)
\curveto(262.81799028,62.39649379)(262.87799022,62.43149376)(262.94799072,62.45150146)
\curveto(263.01799008,62.45149374)(263.05799004,62.45649373)(263.06799072,62.46650146)
}
}
{
\newrgbcolor{curcolor}{0 0 0}
\pscustom[linestyle=none,fillstyle=solid,fillcolor=curcolor]
{
\newpath
\moveto(277.91799072,58.94150146)
\curveto(277.93798266,58.88149731)(277.94798265,58.7864974)(277.94799072,58.65650146)
\curveto(277.94798265,58.53649765)(277.94298266,58.45149774)(277.93299072,58.40150146)
\lineto(277.93299072,58.25150146)
\curveto(277.92298268,58.17149802)(277.91298269,58.09649809)(277.90299072,58.02650146)
\curveto(277.9029827,57.96649822)(277.8979827,57.89649829)(277.88799072,57.81650146)
\curveto(277.86798273,57.75649843)(277.85298275,57.69649849)(277.84299072,57.63650146)
\curveto(277.84298276,57.57649861)(277.83298277,57.51649867)(277.81299072,57.45650146)
\curveto(277.77298283,57.32649886)(277.73798286,57.19649899)(277.70799072,57.06650146)
\curveto(277.67798292,56.93649925)(277.63798296,56.81649937)(277.58799072,56.70650146)
\curveto(277.37798322,56.22649996)(277.0979835,55.82150037)(276.74799072,55.49150146)
\curveto(276.3979842,55.17150102)(275.96798463,54.92650126)(275.45799072,54.75650146)
\curveto(275.34798525,54.71650147)(275.22798537,54.6865015)(275.09799072,54.66650146)
\curveto(274.97798562,54.64650154)(274.85298575,54.62650156)(274.72299072,54.60650146)
\curveto(274.66298594,54.59650159)(274.597986,54.5915016)(274.52799072,54.59150146)
\curveto(274.46798613,54.58150161)(274.40798619,54.57650161)(274.34799072,54.57650146)
\curveto(274.30798629,54.56650162)(274.24798635,54.56150163)(274.16799072,54.56150146)
\curveto(274.0979865,54.56150163)(274.04798655,54.56650162)(274.01799072,54.57650146)
\curveto(273.97798662,54.5865016)(273.93798666,54.5915016)(273.89799072,54.59150146)
\curveto(273.85798674,54.58150161)(273.82298678,54.58150161)(273.79299072,54.59150146)
\lineto(273.70299072,54.59150146)
\lineto(273.34299072,54.63650146)
\curveto(273.2029874,54.67650151)(273.06798753,54.71650147)(272.93799072,54.75650146)
\curveto(272.80798779,54.79650139)(272.68298792,54.84150135)(272.56299072,54.89150146)
\curveto(272.11298849,55.0915011)(271.74298886,55.35150084)(271.45299072,55.67150146)
\curveto(271.16298944,55.9915002)(270.92298968,56.38149981)(270.73299072,56.84150146)
\curveto(270.68298992,56.94149925)(270.64298996,57.04149915)(270.61299072,57.14150146)
\curveto(270.59299001,57.24149895)(270.57299003,57.34649884)(270.55299072,57.45650146)
\curveto(270.53299007,57.49649869)(270.52299008,57.52649866)(270.52299072,57.54650146)
\curveto(270.53299007,57.57649861)(270.53299007,57.61149858)(270.52299072,57.65150146)
\curveto(270.5029901,57.73149846)(270.48799011,57.81149838)(270.47799072,57.89150146)
\curveto(270.47799012,57.98149821)(270.46799013,58.06649812)(270.44799072,58.14650146)
\lineto(270.44799072,58.26650146)
\curveto(270.44799015,58.30649788)(270.44299016,58.35149784)(270.43299072,58.40150146)
\curveto(270.42299018,58.45149774)(270.41799018,58.53649765)(270.41799072,58.65650146)
\curveto(270.41799018,58.7864974)(270.42799017,58.88149731)(270.44799072,58.94150146)
\curveto(270.46799013,59.01149718)(270.47299013,59.08149711)(270.46299072,59.15150146)
\curveto(270.45299015,59.22149697)(270.45799014,59.2914969)(270.47799072,59.36150146)
\curveto(270.48799011,59.41149678)(270.49299011,59.45149674)(270.49299072,59.48150146)
\curveto(270.5029901,59.52149667)(270.51299009,59.56649662)(270.52299072,59.61650146)
\curveto(270.55299005,59.73649645)(270.57799002,59.85649633)(270.59799072,59.97650146)
\curveto(270.62798997,60.09649609)(270.66798993,60.21149598)(270.71799072,60.32150146)
\curveto(270.86798973,60.6914955)(271.04798955,61.02149517)(271.25799072,61.31150146)
\curveto(271.47798912,61.61149458)(271.74298886,61.86149433)(272.05299072,62.06150146)
\curveto(272.17298843,62.14149405)(272.2979883,62.20649398)(272.42799072,62.25650146)
\curveto(272.55798804,62.31649387)(272.69298791,62.37649381)(272.83299072,62.43650146)
\curveto(272.95298765,62.4864937)(273.08298752,62.51649367)(273.22299072,62.52650146)
\curveto(273.36298724,62.54649364)(273.5029871,62.57649361)(273.64299072,62.61650146)
\lineto(273.83799072,62.61650146)
\curveto(273.90798669,62.62649356)(273.97298663,62.63649355)(274.03299072,62.64650146)
\curveto(274.92298568,62.65649353)(275.66298494,62.47149372)(276.25299072,62.09150146)
\curveto(276.84298376,61.71149448)(277.26798333,61.21649497)(277.52799072,60.60650146)
\curveto(277.57798302,60.50649568)(277.61798298,60.40649578)(277.64799072,60.30650146)
\curveto(277.67798292,60.20649598)(277.71298289,60.10149609)(277.75299072,59.99150146)
\curveto(277.78298282,59.88149631)(277.80798279,59.76149643)(277.82799072,59.63150146)
\curveto(277.84798275,59.51149668)(277.87298273,59.3864968)(277.90299072,59.25650146)
\curveto(277.91298269,59.20649698)(277.91298269,59.15149704)(277.90299072,59.09150146)
\curveto(277.9029827,59.04149715)(277.90798269,58.9914972)(277.91799072,58.94150146)
\moveto(276.58299072,58.08650146)
\curveto(276.602984,58.15649803)(276.60798399,58.23649795)(276.59799072,58.32650146)
\lineto(276.59799072,58.58150146)
\curveto(276.597984,58.97149722)(276.56298404,59.30149689)(276.49299072,59.57150146)
\curveto(276.46298414,59.65149654)(276.43798416,59.73149646)(276.41799072,59.81150146)
\curveto(276.3979842,59.8914963)(276.37298423,59.96649622)(276.34299072,60.03650146)
\curveto(276.06298454,60.6864955)(275.61798498,61.13649505)(275.00799072,61.38650146)
\curveto(274.93798566,61.41649477)(274.86298574,61.43649475)(274.78299072,61.44650146)
\lineto(274.54299072,61.50650146)
\curveto(274.46298614,61.52649466)(274.37798622,61.53649465)(274.28799072,61.53650146)
\lineto(274.01799072,61.53650146)
\lineto(273.74799072,61.49150146)
\curveto(273.64798695,61.47149472)(273.55298705,61.44649474)(273.46299072,61.41650146)
\curveto(273.38298722,61.39649479)(273.3029873,61.36649482)(273.22299072,61.32650146)
\curveto(273.15298745,61.30649488)(273.08798751,61.27649491)(273.02799072,61.23650146)
\curveto(272.96798763,61.19649499)(272.91298769,61.15649503)(272.86299072,61.11650146)
\curveto(272.62298798,60.94649524)(272.42798817,60.74149545)(272.27799072,60.50150146)
\curveto(272.12798847,60.26149593)(271.9979886,59.98149621)(271.88799072,59.66150146)
\curveto(271.85798874,59.56149663)(271.83798876,59.45649673)(271.82799072,59.34650146)
\curveto(271.81798878,59.24649694)(271.8029888,59.14149705)(271.78299072,59.03150146)
\curveto(271.77298883,58.9914972)(271.76798883,58.92649726)(271.76799072,58.83650146)
\curveto(271.75798884,58.80649738)(271.75298885,58.77149742)(271.75299072,58.73150146)
\curveto(271.76298884,58.6914975)(271.76798883,58.64649754)(271.76799072,58.59650146)
\lineto(271.76799072,58.29650146)
\curveto(271.76798883,58.19649799)(271.77798882,58.10649808)(271.79799072,58.02650146)
\lineto(271.82799072,57.84650146)
\curveto(271.84798875,57.74649844)(271.86298874,57.64649854)(271.87299072,57.54650146)
\curveto(271.89298871,57.45649873)(271.92298868,57.37149882)(271.96299072,57.29150146)
\curveto(272.06298854,57.05149914)(272.17798842,56.82649936)(272.30799072,56.61650146)
\curveto(272.44798815,56.40649978)(272.61798798,56.23149996)(272.81799072,56.09150146)
\curveto(272.86798773,56.06150013)(272.91298769,56.03650015)(272.95299072,56.01650146)
\curveto(272.99298761,55.99650019)(273.03798756,55.97150022)(273.08799072,55.94150146)
\curveto(273.16798743,55.8915003)(273.25298735,55.84650034)(273.34299072,55.80650146)
\curveto(273.44298716,55.77650041)(273.54798705,55.74650044)(273.65799072,55.71650146)
\curveto(273.70798689,55.69650049)(273.75298685,55.6865005)(273.79299072,55.68650146)
\curveto(273.84298676,55.69650049)(273.89298671,55.69650049)(273.94299072,55.68650146)
\curveto(273.97298663,55.67650051)(274.03298657,55.66650052)(274.12299072,55.65650146)
\curveto(274.22298638,55.64650054)(274.2979863,55.65150054)(274.34799072,55.67150146)
\curveto(274.38798621,55.68150051)(274.42798617,55.68150051)(274.46799072,55.67150146)
\curveto(274.50798609,55.67150052)(274.54798605,55.68150051)(274.58799072,55.70150146)
\curveto(274.66798593,55.72150047)(274.74798585,55.73650045)(274.82799072,55.74650146)
\curveto(274.90798569,55.76650042)(274.98298562,55.7915004)(275.05299072,55.82150146)
\curveto(275.39298521,55.96150023)(275.66798493,56.15650003)(275.87799072,56.40650146)
\curveto(276.08798451,56.65649953)(276.26298434,56.95149924)(276.40299072,57.29150146)
\curveto(276.45298415,57.41149878)(276.48298412,57.53649865)(276.49299072,57.66650146)
\curveto(276.51298409,57.80649838)(276.54298406,57.94649824)(276.58299072,58.08650146)
}
}
{
\newrgbcolor{curcolor}{0 0 0}
\pscustom[linestyle=none,fillstyle=solid,fillcolor=curcolor]
{
\newpath
\moveto(281.83627197,62.64650146)
\curveto(282.55626791,62.65649353)(283.1612673,62.57149362)(283.65127197,62.39150146)
\curveto(284.14126632,62.22149397)(284.52126594,61.91649427)(284.79127197,61.47650146)
\curveto(284.8612656,61.36649482)(284.91626555,61.25149494)(284.95627197,61.13150146)
\curveto(284.99626547,61.02149517)(285.03626543,60.89649529)(285.07627197,60.75650146)
\curveto(285.09626537,60.6864955)(285.10126536,60.61149558)(285.09127197,60.53150146)
\curveto(285.08126538,60.46149573)(285.0662654,60.40649578)(285.04627197,60.36650146)
\curveto(285.02626544,60.34649584)(285.00126546,60.32649586)(284.97127197,60.30650146)
\curveto(284.94126552,60.29649589)(284.91626555,60.28149591)(284.89627197,60.26150146)
\curveto(284.84626562,60.24149595)(284.79626567,60.23649595)(284.74627197,60.24650146)
\curveto(284.69626577,60.25649593)(284.64626582,60.25649593)(284.59627197,60.24650146)
\curveto(284.51626595,60.22649596)(284.41126605,60.22149597)(284.28127197,60.23150146)
\curveto(284.15126631,60.25149594)(284.0612664,60.27649591)(284.01127197,60.30650146)
\curveto(283.93126653,60.35649583)(283.87626659,60.42149577)(283.84627197,60.50150146)
\curveto(283.82626664,60.5914956)(283.79126667,60.67649551)(283.74127197,60.75650146)
\curveto(283.65126681,60.91649527)(283.52626694,61.06149513)(283.36627197,61.19150146)
\curveto(283.25626721,61.27149492)(283.13626733,61.33149486)(283.00627197,61.37150146)
\curveto(282.87626759,61.41149478)(282.73626773,61.45149474)(282.58627197,61.49150146)
\curveto(282.53626793,61.51149468)(282.48626798,61.51649467)(282.43627197,61.50650146)
\curveto(282.38626808,61.50649468)(282.33626813,61.51149468)(282.28627197,61.52150146)
\curveto(282.22626824,61.54149465)(282.15126831,61.55149464)(282.06127197,61.55150146)
\curveto(281.97126849,61.55149464)(281.89626857,61.54149465)(281.83627197,61.52150146)
\lineto(281.74627197,61.52150146)
\lineto(281.59627197,61.49150146)
\curveto(281.54626892,61.4914947)(281.49626897,61.4864947)(281.44627197,61.47650146)
\curveto(281.18626928,61.41649477)(280.97126949,61.33149486)(280.80127197,61.22150146)
\curveto(280.63126983,61.11149508)(280.51626995,60.92649526)(280.45627197,60.66650146)
\curveto(280.43627003,60.59649559)(280.43127003,60.52649566)(280.44127197,60.45650146)
\curveto(280.46127,60.3864958)(280.48126998,60.32649586)(280.50127197,60.27650146)
\curveto(280.5612699,60.12649606)(280.63126983,60.01649617)(280.71127197,59.94650146)
\curveto(280.80126966,59.8864963)(280.91126955,59.81649637)(281.04127197,59.73650146)
\curveto(281.20126926,59.63649655)(281.38126908,59.56149663)(281.58127197,59.51150146)
\curveto(281.78126868,59.47149672)(281.98126848,59.42149677)(282.18127197,59.36150146)
\curveto(282.31126815,59.32149687)(282.44126802,59.2914969)(282.57127197,59.27150146)
\curveto(282.70126776,59.25149694)(282.83126763,59.22149697)(282.96127197,59.18150146)
\curveto(283.17126729,59.12149707)(283.37626709,59.06149713)(283.57627197,59.00150146)
\curveto(283.77626669,58.95149724)(283.97626649,58.8864973)(284.17627197,58.80650146)
\lineto(284.32627197,58.74650146)
\curveto(284.37626609,58.72649746)(284.42626604,58.70149749)(284.47627197,58.67150146)
\curveto(284.67626579,58.55149764)(284.85126561,58.41649777)(285.00127197,58.26650146)
\curveto(285.15126531,58.11649807)(285.27626519,57.92649826)(285.37627197,57.69650146)
\curveto(285.39626507,57.62649856)(285.41626505,57.53149866)(285.43627197,57.41150146)
\curveto(285.45626501,57.34149885)(285.466265,57.26649892)(285.46627197,57.18650146)
\curveto(285.47626499,57.11649907)(285.48126498,57.03649915)(285.48127197,56.94650146)
\lineto(285.48127197,56.79650146)
\curveto(285.461265,56.72649946)(285.45126501,56.65649953)(285.45127197,56.58650146)
\curveto(285.45126501,56.51649967)(285.44126502,56.44649974)(285.42127197,56.37650146)
\curveto(285.39126507,56.26649992)(285.35626511,56.16150003)(285.31627197,56.06150146)
\curveto(285.27626519,55.96150023)(285.23126523,55.87150032)(285.18127197,55.79150146)
\curveto(285.02126544,55.53150066)(284.81626565,55.32150087)(284.56627197,55.16150146)
\curveto(284.31626615,55.01150118)(284.03626643,54.88150131)(283.72627197,54.77150146)
\curveto(283.63626683,54.74150145)(283.54126692,54.72150147)(283.44127197,54.71150146)
\curveto(283.35126711,54.6915015)(283.2612672,54.66650152)(283.17127197,54.63650146)
\curveto(283.07126739,54.61650157)(282.97126749,54.60650158)(282.87127197,54.60650146)
\curveto(282.77126769,54.60650158)(282.67126779,54.59650159)(282.57127197,54.57650146)
\lineto(282.42127197,54.57650146)
\curveto(282.37126809,54.56650162)(282.30126816,54.56150163)(282.21127197,54.56150146)
\curveto(282.12126834,54.56150163)(282.05126841,54.56650162)(282.00127197,54.57650146)
\lineto(281.83627197,54.57650146)
\curveto(281.77626869,54.59650159)(281.71126875,54.60650158)(281.64127197,54.60650146)
\curveto(281.57126889,54.59650159)(281.51126895,54.60150159)(281.46127197,54.62150146)
\curveto(281.41126905,54.63150156)(281.34626912,54.63650155)(281.26627197,54.63650146)
\lineto(281.02627197,54.69650146)
\curveto(280.95626951,54.70650148)(280.88126958,54.72650146)(280.80127197,54.75650146)
\curveto(280.49126997,54.85650133)(280.22127024,54.98150121)(279.99127197,55.13150146)
\curveto(279.7612707,55.28150091)(279.5612709,55.47650071)(279.39127197,55.71650146)
\curveto(279.30127116,55.84650034)(279.22627124,55.98150021)(279.16627197,56.12150146)
\curveto(279.10627136,56.26149993)(279.05127141,56.41649977)(279.00127197,56.58650146)
\curveto(278.98127148,56.64649954)(278.97127149,56.71649947)(278.97127197,56.79650146)
\curveto(278.98127148,56.8864993)(278.99627147,56.95649923)(279.01627197,57.00650146)
\curveto(279.04627142,57.04649914)(279.09627137,57.0864991)(279.16627197,57.12650146)
\curveto(279.21627125,57.14649904)(279.28627118,57.15649903)(279.37627197,57.15650146)
\curveto(279.466271,57.16649902)(279.55627091,57.16649902)(279.64627197,57.15650146)
\curveto(279.73627073,57.14649904)(279.82127064,57.13149906)(279.90127197,57.11150146)
\curveto(279.99127047,57.10149909)(280.05127041,57.0864991)(280.08127197,57.06650146)
\curveto(280.15127031,57.01649917)(280.19627027,56.94149925)(280.21627197,56.84150146)
\curveto(280.24627022,56.75149944)(280.28127018,56.66649952)(280.32127197,56.58650146)
\curveto(280.42127004,56.36649982)(280.55626991,56.19649999)(280.72627197,56.07650146)
\curveto(280.84626962,55.9865002)(280.98126948,55.91650027)(281.13127197,55.86650146)
\curveto(281.28126918,55.81650037)(281.44126902,55.76650042)(281.61127197,55.71650146)
\lineto(281.92627197,55.67150146)
\lineto(282.01627197,55.67150146)
\curveto(282.08626838,55.65150054)(282.17626829,55.64150055)(282.28627197,55.64150146)
\curveto(282.40626806,55.64150055)(282.50626796,55.65150054)(282.58627197,55.67150146)
\curveto(282.65626781,55.67150052)(282.71126775,55.67650051)(282.75127197,55.68650146)
\curveto(282.81126765,55.69650049)(282.87126759,55.70150049)(282.93127197,55.70150146)
\curveto(282.99126747,55.71150048)(283.04626742,55.72150047)(283.09627197,55.73150146)
\curveto(283.38626708,55.81150038)(283.61626685,55.91650027)(283.78627197,56.04650146)
\curveto(283.95626651,56.17650001)(284.07626639,56.39649979)(284.14627197,56.70650146)
\curveto(284.1662663,56.75649943)(284.17126629,56.81149938)(284.16127197,56.87150146)
\curveto(284.15126631,56.93149926)(284.14126632,56.97649921)(284.13127197,57.00650146)
\curveto(284.08126638,57.19649899)(284.01126645,57.33649885)(283.92127197,57.42650146)
\curveto(283.83126663,57.52649866)(283.71626675,57.61649857)(283.57627197,57.69650146)
\curveto(283.48626698,57.75649843)(283.38626708,57.80649838)(283.27627197,57.84650146)
\lineto(282.94627197,57.96650146)
\curveto(282.91626755,57.97649821)(282.88626758,57.98149821)(282.85627197,57.98150146)
\curveto(282.83626763,57.98149821)(282.81126765,57.9914982)(282.78127197,58.01150146)
\curveto(282.44126802,58.12149807)(282.08626838,58.20149799)(281.71627197,58.25150146)
\curveto(281.35626911,58.31149788)(281.01626945,58.40649778)(280.69627197,58.53650146)
\curveto(280.59626987,58.57649761)(280.50126996,58.61149758)(280.41127197,58.64150146)
\curveto(280.32127014,58.67149752)(280.23627023,58.71149748)(280.15627197,58.76150146)
\curveto(279.9662705,58.87149732)(279.79127067,58.99649719)(279.63127197,59.13650146)
\curveto(279.47127099,59.27649691)(279.34627112,59.45149674)(279.25627197,59.66150146)
\curveto(279.22627124,59.73149646)(279.20127126,59.80149639)(279.18127197,59.87150146)
\curveto(279.17127129,59.94149625)(279.15627131,60.01649617)(279.13627197,60.09650146)
\curveto(279.10627136,60.21649597)(279.09627137,60.35149584)(279.10627197,60.50150146)
\curveto(279.11627135,60.66149553)(279.13127133,60.79649539)(279.15127197,60.90650146)
\curveto(279.17127129,60.95649523)(279.18127128,60.99649519)(279.18127197,61.02650146)
\curveto(279.19127127,61.06649512)(279.20627126,61.10649508)(279.22627197,61.14650146)
\curveto(279.31627115,61.37649481)(279.43627103,61.57649461)(279.58627197,61.74650146)
\curveto(279.74627072,61.91649427)(279.92627054,62.06649412)(280.12627197,62.19650146)
\curveto(280.27627019,62.2864939)(280.44127002,62.35649383)(280.62127197,62.40650146)
\curveto(280.80126966,62.46649372)(280.99126947,62.52149367)(281.19127197,62.57150146)
\curveto(281.2612692,62.58149361)(281.32626914,62.5914936)(281.38627197,62.60150146)
\curveto(281.45626901,62.61149358)(281.53126893,62.62149357)(281.61127197,62.63150146)
\curveto(281.64126882,62.64149355)(281.68126878,62.64149355)(281.73127197,62.63150146)
\curveto(281.78126868,62.62149357)(281.81626865,62.62649356)(281.83627197,62.64650146)
}
}
{
\newrgbcolor{curcolor}{0 0 0}
\pscustom[linestyle=none,fillstyle=solid,fillcolor=curcolor]
{
\newpath
\moveto(444.44404907,65.43651611)
\lineto(449.34904907,65.43651611)
\lineto(450.63904907,65.43651611)
\curveto(450.74904119,65.43650542)(450.85904108,65.43650542)(450.96904907,65.43651611)
\curveto(451.07904086,65.44650541)(451.16904077,65.42650543)(451.23904907,65.37651611)
\curveto(451.26904067,65.3565055)(451.29404065,65.33150552)(451.31404907,65.30151611)
\curveto(451.33404061,65.27150558)(451.35404059,65.24150561)(451.37404907,65.21151611)
\curveto(451.39404055,65.14150571)(451.40404054,65.02650583)(451.40404907,64.86651611)
\curveto(451.40404054,64.71650614)(451.39404055,64.60150625)(451.37404907,64.52151611)
\curveto(451.33404061,64.38150647)(451.24904069,64.30150655)(451.11904907,64.28151611)
\curveto(450.98904095,64.27150658)(450.83404111,64.26650659)(450.65404907,64.26651611)
\lineto(449.15404907,64.26651611)
\lineto(446.63404907,64.26651611)
\lineto(446.06404907,64.26651611)
\curveto(445.85404609,64.27650658)(445.69904624,64.2515066)(445.59904907,64.19151611)
\curveto(445.49904644,64.13150672)(445.4440465,64.02650683)(445.43404907,63.87651611)
\lineto(445.43404907,63.41151611)
\lineto(445.43404907,61.88151611)
\curveto(445.43404651,61.77150908)(445.42904651,61.64150921)(445.41904907,61.49151611)
\curveto(445.41904652,61.34150951)(445.42904651,61.22150963)(445.44904907,61.13151611)
\curveto(445.47904646,61.01150984)(445.5390464,60.93150992)(445.62904907,60.89151611)
\curveto(445.66904627,60.87150998)(445.7390462,60.85151)(445.83904907,60.83151611)
\lineto(445.98904907,60.83151611)
\curveto(446.02904591,60.82151003)(446.06904587,60.81651004)(446.10904907,60.81651611)
\curveto(446.15904578,60.82651003)(446.20904573,60.83151002)(446.25904907,60.83151611)
\lineto(446.76904907,60.83151611)
\lineto(449.70904907,60.83151611)
\lineto(450.00904907,60.83151611)
\curveto(450.11904182,60.84151001)(450.22904171,60.84151001)(450.33904907,60.83151611)
\curveto(450.45904148,60.83151002)(450.56404138,60.82151003)(450.65404907,60.80151611)
\curveto(450.75404119,60.79151006)(450.82904111,60.77151008)(450.87904907,60.74151611)
\curveto(450.90904103,60.72151013)(450.93404101,60.67651018)(450.95404907,60.60651611)
\curveto(450.97404097,60.53651032)(450.98904095,60.46151039)(450.99904907,60.38151611)
\curveto(451.00904093,60.30151055)(451.00904093,60.21651064)(450.99904907,60.12651611)
\curveto(450.99904094,60.04651081)(450.98904095,59.97651088)(450.96904907,59.91651611)
\curveto(450.94904099,59.82651103)(450.90404104,59.76151109)(450.83404907,59.72151611)
\curveto(450.81404113,59.70151115)(450.78404116,59.68651117)(450.74404907,59.67651611)
\curveto(450.71404123,59.67651118)(450.68404126,59.67151118)(450.65404907,59.66151611)
\lineto(450.56404907,59.66151611)
\curveto(450.51404143,59.6515112)(450.46404148,59.64651121)(450.41404907,59.64651611)
\curveto(450.36404158,59.6565112)(450.31404163,59.66151119)(450.26404907,59.66151611)
\lineto(449.70904907,59.66151611)
\lineto(446.54404907,59.66151611)
\lineto(446.18404907,59.66151611)
\curveto(446.07404587,59.67151118)(445.96904597,59.66651119)(445.86904907,59.64651611)
\curveto(445.76904617,59.63651122)(445.67904626,59.61151124)(445.59904907,59.57151611)
\curveto(445.52904641,59.53151132)(445.47904646,59.46151139)(445.44904907,59.36151611)
\curveto(445.42904651,59.30151155)(445.41904652,59.23151162)(445.41904907,59.15151611)
\curveto(445.42904651,59.07151178)(445.43404651,58.99151186)(445.43404907,58.91151611)
\lineto(445.43404907,58.07151611)
\lineto(445.43404907,56.64651611)
\curveto(445.43404651,56.50651435)(445.4390465,56.37651448)(445.44904907,56.25651611)
\curveto(445.45904648,56.14651471)(445.49904644,56.06651479)(445.56904907,56.01651611)
\curveto(445.6390463,55.96651489)(445.71904622,55.93651492)(445.80904907,55.92651611)
\lineto(446.10904907,55.92651611)
\lineto(447.06904907,55.92651611)
\lineto(449.84404907,55.92651611)
\lineto(450.69904907,55.92651611)
\lineto(450.93904907,55.92651611)
\curveto(451.01904092,55.93651492)(451.08904085,55.93151492)(451.14904907,55.91151611)
\curveto(451.26904067,55.87151498)(451.34904059,55.81651504)(451.38904907,55.74651611)
\curveto(451.40904053,55.71651514)(451.42404052,55.66651519)(451.43404907,55.59651611)
\curveto(451.4440405,55.52651533)(451.44904049,55.4515154)(451.44904907,55.37151611)
\curveto(451.45904048,55.30151555)(451.45904048,55.22651563)(451.44904907,55.14651611)
\curveto(451.4390405,55.07651578)(451.42904051,55.02151583)(451.41904907,54.98151611)
\curveto(451.37904056,54.90151595)(451.33404061,54.84651601)(451.28404907,54.81651611)
\curveto(451.22404072,54.77651608)(451.1440408,54.7565161)(451.04404907,54.75651611)
\lineto(450.77404907,54.75651611)
\lineto(449.72404907,54.75651611)
\lineto(445.73404907,54.75651611)
\lineto(444.68404907,54.75651611)
\curveto(444.5440474,54.7565161)(444.42404752,54.76151609)(444.32404907,54.77151611)
\curveto(444.22404772,54.79151606)(444.14904779,54.84151601)(444.09904907,54.92151611)
\curveto(444.05904788,54.98151587)(444.0390479,55.0565158)(444.03904907,55.14651611)
\lineto(444.03904907,55.43151611)
\lineto(444.03904907,56.48151611)
\lineto(444.03904907,60.50151611)
\lineto(444.03904907,63.86151611)
\lineto(444.03904907,64.79151611)
\lineto(444.03904907,65.06151611)
\curveto(444.0390479,65.1515057)(444.05904788,65.22150563)(444.09904907,65.27151611)
\curveto(444.1390478,65.34150551)(444.21404773,65.39150546)(444.32404907,65.42151611)
\curveto(444.3440476,65.43150542)(444.36404758,65.43150542)(444.38404907,65.42151611)
\curveto(444.40404754,65.42150543)(444.42404752,65.42650543)(444.44404907,65.43651611)
}
}
{
\newrgbcolor{curcolor}{0 0 0}
\pscustom[linestyle=none,fillstyle=solid,fillcolor=curcolor]
{
\newpath
\moveto(456.64397095,62.63151611)
\curveto(457.27396571,62.6515082)(457.77896521,62.56650829)(458.15897095,62.37651611)
\curveto(458.53896445,62.18650867)(458.84396414,61.90150895)(459.07397095,61.52151611)
\curveto(459.13396385,61.42150943)(459.17896381,61.31150954)(459.20897095,61.19151611)
\curveto(459.24896374,61.08150977)(459.2839637,60.96650989)(459.31397095,60.84651611)
\curveto(459.36396362,60.6565102)(459.39396359,60.4515104)(459.40397095,60.23151611)
\curveto(459.41396357,60.01151084)(459.41896357,59.78651107)(459.41897095,59.55651611)
\lineto(459.41897095,57.95151611)
\lineto(459.41897095,55.61151611)
\curveto(459.41896357,55.44151541)(459.41396357,55.27151558)(459.40397095,55.10151611)
\curveto(459.40396358,54.93151592)(459.33896365,54.82151603)(459.20897095,54.77151611)
\curveto(459.15896383,54.7515161)(459.10396388,54.74151611)(459.04397095,54.74151611)
\curveto(458.99396399,54.73151612)(458.93896405,54.72651613)(458.87897095,54.72651611)
\curveto(458.74896424,54.72651613)(458.62396436,54.73151612)(458.50397095,54.74151611)
\curveto(458.3839646,54.74151611)(458.29896469,54.78151607)(458.24897095,54.86151611)
\curveto(458.19896479,54.93151592)(458.17396481,55.02151583)(458.17397095,55.13151611)
\lineto(458.17397095,55.46151611)
\lineto(458.17397095,56.75151611)
\lineto(458.17397095,59.19651611)
\curveto(458.17396481,59.46651139)(458.16896482,59.73151112)(458.15897095,59.99151611)
\curveto(458.14896484,60.26151059)(458.10396488,60.49151036)(458.02397095,60.68151611)
\curveto(457.94396504,60.88150997)(457.82396516,61.04150981)(457.66397095,61.16151611)
\curveto(457.50396548,61.29150956)(457.31896567,61.39150946)(457.10897095,61.46151611)
\curveto(457.04896594,61.48150937)(456.983966,61.49150936)(456.91397095,61.49151611)
\curveto(456.85396613,61.50150935)(456.79396619,61.51650934)(456.73397095,61.53651611)
\curveto(456.6839663,61.54650931)(456.60396638,61.54650931)(456.49397095,61.53651611)
\curveto(456.39396659,61.53650932)(456.32396666,61.53150932)(456.28397095,61.52151611)
\curveto(456.24396674,61.50150935)(456.20896678,61.49150936)(456.17897095,61.49151611)
\curveto(456.14896684,61.50150935)(456.11396687,61.50150935)(456.07397095,61.49151611)
\curveto(455.94396704,61.46150939)(455.81896717,61.42650943)(455.69897095,61.38651611)
\curveto(455.5889674,61.3565095)(455.4839675,61.31150954)(455.38397095,61.25151611)
\curveto(455.34396764,61.23150962)(455.30896768,61.21150964)(455.27897095,61.19151611)
\curveto(455.24896774,61.17150968)(455.21396777,61.1515097)(455.17397095,61.13151611)
\curveto(454.82396816,60.88150997)(454.56896842,60.50651035)(454.40897095,60.00651611)
\curveto(454.37896861,59.92651093)(454.35896863,59.84151101)(454.34897095,59.75151611)
\curveto(454.33896865,59.67151118)(454.32396866,59.59151126)(454.30397095,59.51151611)
\curveto(454.2839687,59.46151139)(454.27896871,59.41151144)(454.28897095,59.36151611)
\curveto(454.29896869,59.32151153)(454.29396869,59.28151157)(454.27397095,59.24151611)
\lineto(454.27397095,58.92651611)
\curveto(454.26396872,58.89651196)(454.25896873,58.86151199)(454.25897095,58.82151611)
\curveto(454.26896872,58.78151207)(454.27396871,58.73651212)(454.27397095,58.68651611)
\lineto(454.27397095,58.23651611)
\lineto(454.27397095,56.79651611)
\lineto(454.27397095,55.47651611)
\lineto(454.27397095,55.13151611)
\curveto(454.27396871,55.02151583)(454.24896874,54.93151592)(454.19897095,54.86151611)
\curveto(454.14896884,54.78151607)(454.05896893,54.74151611)(453.92897095,54.74151611)
\curveto(453.80896918,54.73151612)(453.6839693,54.72651613)(453.55397095,54.72651611)
\curveto(453.47396951,54.72651613)(453.39896959,54.73151612)(453.32897095,54.74151611)
\curveto(453.25896973,54.7515161)(453.19896979,54.77651608)(453.14897095,54.81651611)
\curveto(453.06896992,54.86651599)(453.02896996,54.96151589)(453.02897095,55.10151611)
\lineto(453.02897095,55.50651611)
\lineto(453.02897095,57.27651611)
\lineto(453.02897095,60.90651611)
\lineto(453.02897095,61.82151611)
\lineto(453.02897095,62.09151611)
\curveto(453.02896996,62.18150867)(453.04896994,62.2515086)(453.08897095,62.30151611)
\curveto(453.11896987,62.36150849)(453.16896982,62.40150845)(453.23897095,62.42151611)
\curveto(453.27896971,62.43150842)(453.33396965,62.44150841)(453.40397095,62.45151611)
\curveto(453.4839695,62.46150839)(453.56396942,62.46650839)(453.64397095,62.46651611)
\curveto(453.72396926,62.46650839)(453.79896919,62.46150839)(453.86897095,62.45151611)
\curveto(453.94896904,62.44150841)(454.00396898,62.42650843)(454.03397095,62.40651611)
\curveto(454.14396884,62.33650852)(454.19396879,62.24650861)(454.18397095,62.13651611)
\curveto(454.17396881,62.03650882)(454.1889688,61.92150893)(454.22897095,61.79151611)
\curveto(454.24896874,61.73150912)(454.2889687,61.68150917)(454.34897095,61.64151611)
\curveto(454.46896852,61.63150922)(454.56396842,61.67650918)(454.63397095,61.77651611)
\curveto(454.71396827,61.87650898)(454.79396819,61.9565089)(454.87397095,62.01651611)
\curveto(455.01396797,62.11650874)(455.15396783,62.20650865)(455.29397095,62.28651611)
\curveto(455.44396754,62.37650848)(455.61396737,62.4515084)(455.80397095,62.51151611)
\curveto(455.8839671,62.54150831)(455.96896702,62.56150829)(456.05897095,62.57151611)
\curveto(456.15896683,62.58150827)(456.25396673,62.59650826)(456.34397095,62.61651611)
\curveto(456.39396659,62.62650823)(456.44396654,62.63150822)(456.49397095,62.63151611)
\lineto(456.64397095,62.63151611)
}
}
{
\newrgbcolor{curcolor}{0 0 0}
\pscustom[linestyle=none,fillstyle=solid,fillcolor=curcolor]
{
\newpath
\moveto(461.87358032,65.43651611)
\curveto(462.00357871,65.43650542)(462.13857857,65.43650542)(462.27858032,65.43651611)
\curveto(462.42857828,65.43650542)(462.53857817,65.40150545)(462.60858032,65.33151611)
\curveto(462.65857805,65.26150559)(462.68357803,65.16650569)(462.68358032,65.04651611)
\curveto(462.69357802,64.93650592)(462.69857801,64.82150603)(462.69858032,64.70151611)
\lineto(462.69858032,63.36651611)
\lineto(462.69858032,57.29151611)
\lineto(462.69858032,55.61151611)
\lineto(462.69858032,55.22151611)
\curveto(462.69857801,55.08151577)(462.67357804,54.97151588)(462.62358032,54.89151611)
\curveto(462.59357812,54.84151601)(462.54857816,54.81151604)(462.48858032,54.80151611)
\curveto(462.43857827,54.79151606)(462.37357834,54.77651608)(462.29358032,54.75651611)
\lineto(462.08358032,54.75651611)
\lineto(461.76858032,54.75651611)
\curveto(461.66857904,54.76651609)(461.59357912,54.80151605)(461.54358032,54.86151611)
\curveto(461.49357922,54.94151591)(461.46357925,55.04151581)(461.45358032,55.16151611)
\lineto(461.45358032,55.53651611)
\lineto(461.45358032,56.91651611)
\lineto(461.45358032,63.15651611)
\lineto(461.45358032,64.62651611)
\curveto(461.45357926,64.73650612)(461.44857926,64.851506)(461.43858032,64.97151611)
\curveto(461.43857927,65.10150575)(461.46357925,65.20150565)(461.51358032,65.27151611)
\curveto(461.55357916,65.33150552)(461.62857908,65.38150547)(461.73858032,65.42151611)
\curveto(461.75857895,65.43150542)(461.77857893,65.43150542)(461.79858032,65.42151611)
\curveto(461.82857888,65.42150543)(461.85357886,65.42650543)(461.87358032,65.43651611)
}
}
{
\newrgbcolor{curcolor}{0 0 0}
\pscustom[linestyle=none,fillstyle=solid,fillcolor=curcolor]
{
\newpath
\moveto(471.52842407,55.31151611)
\curveto(471.55841624,55.1515157)(471.54341626,55.01651584)(471.48342407,54.90651611)
\curveto(471.42341638,54.80651605)(471.34341646,54.73151612)(471.24342407,54.68151611)
\curveto(471.19341661,54.66151619)(471.13841666,54.6515162)(471.07842407,54.65151611)
\curveto(471.02841677,54.6515162)(470.97341683,54.64151621)(470.91342407,54.62151611)
\curveto(470.69341711,54.57151628)(470.47341733,54.58651627)(470.25342407,54.66651611)
\curveto(470.04341776,54.73651612)(469.8984179,54.82651603)(469.81842407,54.93651611)
\curveto(469.76841803,55.00651585)(469.72341808,55.08651577)(469.68342407,55.17651611)
\curveto(469.64341816,55.27651558)(469.59341821,55.3565155)(469.53342407,55.41651611)
\curveto(469.51341829,55.43651542)(469.48841831,55.4565154)(469.45842407,55.47651611)
\curveto(469.43841836,55.49651536)(469.40841839,55.50151535)(469.36842407,55.49151611)
\curveto(469.25841854,55.46151539)(469.15341865,55.40651545)(469.05342407,55.32651611)
\curveto(468.96341884,55.24651561)(468.87341893,55.17651568)(468.78342407,55.11651611)
\curveto(468.65341915,55.03651582)(468.51341929,54.96151589)(468.36342407,54.89151611)
\curveto(468.21341959,54.83151602)(468.05341975,54.77651608)(467.88342407,54.72651611)
\curveto(467.78342002,54.69651616)(467.67342013,54.67651618)(467.55342407,54.66651611)
\curveto(467.44342036,54.6565162)(467.33342047,54.64151621)(467.22342407,54.62151611)
\curveto(467.17342063,54.61151624)(467.12842067,54.60651625)(467.08842407,54.60651611)
\lineto(466.98342407,54.60651611)
\curveto(466.87342093,54.58651627)(466.76842103,54.58651627)(466.66842407,54.60651611)
\lineto(466.53342407,54.60651611)
\curveto(466.48342132,54.61651624)(466.43342137,54.62151623)(466.38342407,54.62151611)
\curveto(466.33342147,54.62151623)(466.28842151,54.63151622)(466.24842407,54.65151611)
\curveto(466.20842159,54.66151619)(466.17342163,54.66651619)(466.14342407,54.66651611)
\curveto(466.12342168,54.6565162)(466.0984217,54.6565162)(466.06842407,54.66651611)
\lineto(465.82842407,54.72651611)
\curveto(465.74842205,54.73651612)(465.67342213,54.7565161)(465.60342407,54.78651611)
\curveto(465.3034225,54.91651594)(465.05842274,55.06151579)(464.86842407,55.22151611)
\curveto(464.68842311,55.39151546)(464.53842326,55.62651523)(464.41842407,55.92651611)
\curveto(464.32842347,56.14651471)(464.28342352,56.41151444)(464.28342407,56.72151611)
\lineto(464.28342407,57.03651611)
\curveto(464.29342351,57.08651377)(464.2984235,57.13651372)(464.29842407,57.18651611)
\lineto(464.32842407,57.36651611)
\lineto(464.44842407,57.69651611)
\curveto(464.48842331,57.80651305)(464.53842326,57.90651295)(464.59842407,57.99651611)
\curveto(464.77842302,58.28651257)(465.02342278,58.50151235)(465.33342407,58.64151611)
\curveto(465.64342216,58.78151207)(465.98342182,58.90651195)(466.35342407,59.01651611)
\curveto(466.49342131,59.0565118)(466.63842116,59.08651177)(466.78842407,59.10651611)
\curveto(466.93842086,59.12651173)(467.08842071,59.1515117)(467.23842407,59.18151611)
\curveto(467.30842049,59.20151165)(467.37342043,59.21151164)(467.43342407,59.21151611)
\curveto(467.5034203,59.21151164)(467.57842022,59.22151163)(467.65842407,59.24151611)
\curveto(467.72842007,59.26151159)(467.79842,59.27151158)(467.86842407,59.27151611)
\curveto(467.93841986,59.28151157)(468.01341979,59.29651156)(468.09342407,59.31651611)
\curveto(468.34341946,59.37651148)(468.57841922,59.42651143)(468.79842407,59.46651611)
\curveto(469.01841878,59.51651134)(469.19341861,59.63151122)(469.32342407,59.81151611)
\curveto(469.38341842,59.89151096)(469.43341837,59.99151086)(469.47342407,60.11151611)
\curveto(469.51341829,60.24151061)(469.51341829,60.38151047)(469.47342407,60.53151611)
\curveto(469.41341839,60.77151008)(469.32341848,60.96150989)(469.20342407,61.10151611)
\curveto(469.09341871,61.24150961)(468.93341887,61.3515095)(468.72342407,61.43151611)
\curveto(468.6034192,61.48150937)(468.45841934,61.51650934)(468.28842407,61.53651611)
\curveto(468.12841967,61.5565093)(467.95841984,61.56650929)(467.77842407,61.56651611)
\curveto(467.5984202,61.56650929)(467.42342038,61.5565093)(467.25342407,61.53651611)
\curveto(467.08342072,61.51650934)(466.93842086,61.48650937)(466.81842407,61.44651611)
\curveto(466.64842115,61.38650947)(466.48342132,61.30150955)(466.32342407,61.19151611)
\curveto(466.24342156,61.13150972)(466.16842163,61.0515098)(466.09842407,60.95151611)
\curveto(466.03842176,60.86150999)(465.98342182,60.76151009)(465.93342407,60.65151611)
\curveto(465.9034219,60.57151028)(465.87342193,60.48651037)(465.84342407,60.39651611)
\curveto(465.82342198,60.30651055)(465.77842202,60.23651062)(465.70842407,60.18651611)
\curveto(465.66842213,60.1565107)(465.5984222,60.13151072)(465.49842407,60.11151611)
\curveto(465.40842239,60.10151075)(465.31342249,60.09651076)(465.21342407,60.09651611)
\curveto(465.11342269,60.09651076)(465.01342279,60.10151075)(464.91342407,60.11151611)
\curveto(464.82342298,60.13151072)(464.75842304,60.1565107)(464.71842407,60.18651611)
\curveto(464.67842312,60.21651064)(464.64842315,60.26651059)(464.62842407,60.33651611)
\curveto(464.60842319,60.40651045)(464.60842319,60.48151037)(464.62842407,60.56151611)
\curveto(464.65842314,60.69151016)(464.68842311,60.81151004)(464.71842407,60.92151611)
\curveto(464.75842304,61.04150981)(464.803423,61.1565097)(464.85342407,61.26651611)
\curveto(465.04342276,61.61650924)(465.28342252,61.88650897)(465.57342407,62.07651611)
\curveto(465.86342194,62.27650858)(466.22342158,62.43650842)(466.65342407,62.55651611)
\curveto(466.75342105,62.57650828)(466.85342095,62.59150826)(466.95342407,62.60151611)
\curveto(467.06342074,62.61150824)(467.17342063,62.62650823)(467.28342407,62.64651611)
\curveto(467.32342048,62.6565082)(467.38842041,62.6565082)(467.47842407,62.64651611)
\curveto(467.56842023,62.64650821)(467.62342018,62.6565082)(467.64342407,62.67651611)
\curveto(468.34341946,62.68650817)(468.95341885,62.60650825)(469.47342407,62.43651611)
\curveto(469.99341781,62.26650859)(470.35841744,61.94150891)(470.56842407,61.46151611)
\curveto(470.65841714,61.26150959)(470.70841709,61.02650983)(470.71842407,60.75651611)
\curveto(470.73841706,60.49651036)(470.74841705,60.22151063)(470.74842407,59.93151611)
\lineto(470.74842407,56.61651611)
\curveto(470.74841705,56.47651438)(470.75341705,56.34151451)(470.76342407,56.21151611)
\curveto(470.77341703,56.08151477)(470.803417,55.97651488)(470.85342407,55.89651611)
\curveto(470.9034169,55.82651503)(470.96841683,55.77651508)(471.04842407,55.74651611)
\curveto(471.13841666,55.70651515)(471.22341658,55.67651518)(471.30342407,55.65651611)
\curveto(471.38341642,55.64651521)(471.44341636,55.60151525)(471.48342407,55.52151611)
\curveto(471.5034163,55.49151536)(471.51341629,55.46151539)(471.51342407,55.43151611)
\curveto(471.51341629,55.40151545)(471.51841628,55.36151549)(471.52842407,55.31151611)
\moveto(469.38342407,56.97651611)
\curveto(469.44341836,57.11651374)(469.47341833,57.27651358)(469.47342407,57.45651611)
\curveto(469.48341832,57.64651321)(469.48841831,57.84151301)(469.48842407,58.04151611)
\curveto(469.48841831,58.1515127)(469.48341832,58.2515126)(469.47342407,58.34151611)
\curveto(469.46341834,58.43151242)(469.42341838,58.50151235)(469.35342407,58.55151611)
\curveto(469.32341848,58.57151228)(469.25341855,58.58151227)(469.14342407,58.58151611)
\curveto(469.12341868,58.56151229)(469.08841871,58.5515123)(469.03842407,58.55151611)
\curveto(468.98841881,58.5515123)(468.94341886,58.54151231)(468.90342407,58.52151611)
\curveto(468.82341898,58.50151235)(468.73341907,58.48151237)(468.63342407,58.46151611)
\lineto(468.33342407,58.40151611)
\curveto(468.3034195,58.40151245)(468.26841953,58.39651246)(468.22842407,58.38651611)
\lineto(468.12342407,58.38651611)
\curveto(467.97341983,58.34651251)(467.80841999,58.32151253)(467.62842407,58.31151611)
\curveto(467.45842034,58.31151254)(467.2984205,58.29151256)(467.14842407,58.25151611)
\curveto(467.06842073,58.23151262)(466.99342081,58.21151264)(466.92342407,58.19151611)
\curveto(466.86342094,58.18151267)(466.79342101,58.16651269)(466.71342407,58.14651611)
\curveto(466.55342125,58.09651276)(466.4034214,58.03151282)(466.26342407,57.95151611)
\curveto(466.12342168,57.88151297)(466.0034218,57.79151306)(465.90342407,57.68151611)
\curveto(465.803422,57.57151328)(465.72842207,57.43651342)(465.67842407,57.27651611)
\curveto(465.62842217,57.12651373)(465.60842219,56.94151391)(465.61842407,56.72151611)
\curveto(465.61842218,56.62151423)(465.63342217,56.52651433)(465.66342407,56.43651611)
\curveto(465.7034221,56.3565145)(465.74842205,56.28151457)(465.79842407,56.21151611)
\curveto(465.87842192,56.10151475)(465.98342182,56.00651485)(466.11342407,55.92651611)
\curveto(466.24342156,55.856515)(466.38342142,55.79651506)(466.53342407,55.74651611)
\curveto(466.58342122,55.73651512)(466.63342117,55.73151512)(466.68342407,55.73151611)
\curveto(466.73342107,55.73151512)(466.78342102,55.72651513)(466.83342407,55.71651611)
\curveto(466.9034209,55.69651516)(466.98842081,55.68151517)(467.08842407,55.67151611)
\curveto(467.1984206,55.67151518)(467.28842051,55.68151517)(467.35842407,55.70151611)
\curveto(467.41842038,55.72151513)(467.47842032,55.72651513)(467.53842407,55.71651611)
\curveto(467.5984202,55.71651514)(467.65842014,55.72651513)(467.71842407,55.74651611)
\curveto(467.79842,55.76651509)(467.87341993,55.78151507)(467.94342407,55.79151611)
\curveto(468.02341978,55.80151505)(468.0984197,55.82151503)(468.16842407,55.85151611)
\curveto(468.45841934,55.97151488)(468.7034191,56.11651474)(468.90342407,56.28651611)
\curveto(469.11341869,56.4565144)(469.27341853,56.68651417)(469.38342407,56.97651611)
}
}
{
\newrgbcolor{curcolor}{0 0 0}
\pscustom[linestyle=none,fillstyle=solid,fillcolor=curcolor]
{
\newpath
\moveto(475.8350647,62.66151611)
\curveto(476.57505991,62.67150818)(477.19005929,62.56150829)(477.6800647,62.33151611)
\curveto(478.1800583,62.11150874)(478.57505791,61.77650908)(478.8650647,61.32651611)
\curveto(478.99505749,61.12650973)(479.10505738,60.88150997)(479.1950647,60.59151611)
\curveto(479.21505727,60.54151031)(479.23005725,60.47651038)(479.2400647,60.39651611)
\curveto(479.25005723,60.31651054)(479.24505724,60.24651061)(479.2250647,60.18651611)
\curveto(479.19505729,60.13651072)(479.14505734,60.09151076)(479.0750647,60.05151611)
\curveto(479.04505744,60.03151082)(479.01505747,60.02151083)(478.9850647,60.02151611)
\curveto(478.95505753,60.03151082)(478.92005756,60.03151082)(478.8800647,60.02151611)
\curveto(478.84005764,60.01151084)(478.80005768,60.00651085)(478.7600647,60.00651611)
\curveto(478.72005776,60.01651084)(478.6800578,60.02151083)(478.6400647,60.02151611)
\lineto(478.3250647,60.02151611)
\curveto(478.22505826,60.03151082)(478.14005834,60.06151079)(478.0700647,60.11151611)
\curveto(477.99005849,60.17151068)(477.93505855,60.2565106)(477.9050647,60.36651611)
\curveto(477.87505861,60.47651038)(477.83505865,60.57151028)(477.7850647,60.65151611)
\curveto(477.63505885,60.91150994)(477.44005904,61.11650974)(477.2000647,61.26651611)
\curveto(477.12005936,61.31650954)(477.03505945,61.3565095)(476.9450647,61.38651611)
\curveto(476.85505963,61.42650943)(476.76005972,61.46150939)(476.6600647,61.49151611)
\curveto(476.52005996,61.53150932)(476.33506015,61.5515093)(476.1050647,61.55151611)
\curveto(475.87506061,61.56150929)(475.6850608,61.54150931)(475.5350647,61.49151611)
\curveto(475.46506102,61.47150938)(475.40006108,61.4565094)(475.3400647,61.44651611)
\curveto(475.2800612,61.43650942)(475.21506127,61.42150943)(475.1450647,61.40151611)
\curveto(474.8850616,61.29150956)(474.65506183,61.14150971)(474.4550647,60.95151611)
\curveto(474.25506223,60.76151009)(474.10006238,60.53651032)(473.9900647,60.27651611)
\curveto(473.95006253,60.18651067)(473.91506257,60.09151076)(473.8850647,59.99151611)
\curveto(473.85506263,59.90151095)(473.82506266,59.80151105)(473.7950647,59.69151611)
\lineto(473.7050647,59.28651611)
\curveto(473.69506279,59.23651162)(473.69006279,59.18151167)(473.6900647,59.12151611)
\curveto(473.70006278,59.06151179)(473.69506279,59.00651185)(473.6750647,58.95651611)
\lineto(473.6750647,58.83651611)
\curveto(473.66506282,58.79651206)(473.65506283,58.73151212)(473.6450647,58.64151611)
\curveto(473.64506284,58.5515123)(473.65506283,58.48651237)(473.6750647,58.44651611)
\curveto(473.6850628,58.39651246)(473.6850628,58.34651251)(473.6750647,58.29651611)
\curveto(473.66506282,58.24651261)(473.66506282,58.19651266)(473.6750647,58.14651611)
\curveto(473.6850628,58.10651275)(473.69006279,58.03651282)(473.6900647,57.93651611)
\curveto(473.71006277,57.856513)(473.72506276,57.77151308)(473.7350647,57.68151611)
\curveto(473.75506273,57.59151326)(473.77506271,57.50651335)(473.7950647,57.42651611)
\curveto(473.90506258,57.10651375)(474.03006245,56.82651403)(474.1700647,56.58651611)
\curveto(474.32006216,56.3565145)(474.52506196,56.1565147)(474.7850647,55.98651611)
\curveto(474.87506161,55.93651492)(474.96506152,55.89151496)(475.0550647,55.85151611)
\curveto(475.15506133,55.81151504)(475.26006122,55.77151508)(475.3700647,55.73151611)
\curveto(475.42006106,55.72151513)(475.46006102,55.71651514)(475.4900647,55.71651611)
\curveto(475.52006096,55.71651514)(475.56006092,55.71151514)(475.6100647,55.70151611)
\curveto(475.64006084,55.69151516)(475.69006079,55.68651517)(475.7600647,55.68651611)
\lineto(475.9250647,55.68651611)
\curveto(475.92506056,55.67651518)(475.94506054,55.67151518)(475.9850647,55.67151611)
\curveto(476.00506048,55.68151517)(476.03006045,55.68151517)(476.0600647,55.67151611)
\curveto(476.09006039,55.67151518)(476.12006036,55.67651518)(476.1500647,55.68651611)
\curveto(476.22006026,55.70651515)(476.2850602,55.71151514)(476.3450647,55.70151611)
\curveto(476.41506007,55.70151515)(476.48506,55.71151514)(476.5550647,55.73151611)
\curveto(476.81505967,55.81151504)(477.04005944,55.91151494)(477.2300647,56.03151611)
\curveto(477.42005906,56.16151469)(477.5800589,56.32651453)(477.7100647,56.52651611)
\curveto(477.76005872,56.60651425)(477.80505868,56.69151416)(477.8450647,56.78151611)
\lineto(477.9650647,57.05151611)
\curveto(477.9850585,57.13151372)(478.00505848,57.20651365)(478.0250647,57.27651611)
\curveto(478.05505843,57.3565135)(478.10505838,57.42151343)(478.1750647,57.47151611)
\curveto(478.20505828,57.50151335)(478.26505822,57.52151333)(478.3550647,57.53151611)
\curveto(478.44505804,57.5515133)(478.54005794,57.56151329)(478.6400647,57.56151611)
\curveto(478.75005773,57.57151328)(478.85005763,57.57151328)(478.9400647,57.56151611)
\curveto(479.04005744,57.5515133)(479.11005737,57.53151332)(479.1500647,57.50151611)
\curveto(479.21005727,57.46151339)(479.24505724,57.40151345)(479.2550647,57.32151611)
\curveto(479.27505721,57.24151361)(479.27505721,57.1565137)(479.2550647,57.06651611)
\curveto(479.20505728,56.91651394)(479.15505733,56.77151408)(479.1050647,56.63151611)
\curveto(479.06505742,56.50151435)(479.01005747,56.37151448)(478.9400647,56.24151611)
\curveto(478.79005769,55.94151491)(478.60005788,55.67651518)(478.3700647,55.44651611)
\curveto(478.15005833,55.21651564)(477.8800586,55.03151582)(477.5600647,54.89151611)
\curveto(477.480059,54.851516)(477.39505909,54.81651604)(477.3050647,54.78651611)
\curveto(477.21505927,54.76651609)(477.12005936,54.74151611)(477.0200647,54.71151611)
\curveto(476.91005957,54.67151618)(476.80005968,54.6515162)(476.6900647,54.65151611)
\curveto(476.5800599,54.64151621)(476.47006001,54.62651623)(476.3600647,54.60651611)
\curveto(476.32006016,54.58651627)(476.2800602,54.58151627)(476.2400647,54.59151611)
\curveto(476.20006028,54.60151625)(476.16006032,54.60151625)(476.1200647,54.59151611)
\lineto(475.9850647,54.59151611)
\lineto(475.7450647,54.59151611)
\curveto(475.67506081,54.58151627)(475.61006087,54.58651627)(475.5500647,54.60651611)
\lineto(475.4750647,54.60651611)
\lineto(475.1150647,54.65151611)
\curveto(474.9850615,54.69151616)(474.86006162,54.72651613)(474.7400647,54.75651611)
\curveto(474.62006186,54.78651607)(474.50506198,54.82651603)(474.3950647,54.87651611)
\curveto(474.03506245,55.03651582)(473.73506275,55.22651563)(473.4950647,55.44651611)
\curveto(473.26506322,55.66651519)(473.05006343,55.93651492)(472.8500647,56.25651611)
\curveto(472.80006368,56.33651452)(472.75506373,56.42651443)(472.7150647,56.52651611)
\lineto(472.5950647,56.82651611)
\curveto(472.54506394,56.93651392)(472.51006397,57.0515138)(472.4900647,57.17151611)
\curveto(472.47006401,57.29151356)(472.44506404,57.41151344)(472.4150647,57.53151611)
\curveto(472.40506408,57.57151328)(472.40006408,57.61151324)(472.4000647,57.65151611)
\curveto(472.40006408,57.69151316)(472.39506409,57.73151312)(472.3850647,57.77151611)
\curveto(472.36506412,57.83151302)(472.35506413,57.89651296)(472.3550647,57.96651611)
\curveto(472.36506412,58.03651282)(472.36006412,58.10151275)(472.3400647,58.16151611)
\lineto(472.3400647,58.31151611)
\curveto(472.33006415,58.36151249)(472.32506416,58.43151242)(472.3250647,58.52151611)
\curveto(472.32506416,58.61151224)(472.33006415,58.68151217)(472.3400647,58.73151611)
\curveto(472.35006413,58.78151207)(472.35006413,58.82651203)(472.3400647,58.86651611)
\curveto(472.34006414,58.90651195)(472.34506414,58.94651191)(472.3550647,58.98651611)
\curveto(472.37506411,59.0565118)(472.3800641,59.12651173)(472.3700647,59.19651611)
\curveto(472.37006411,59.26651159)(472.3800641,59.33151152)(472.4000647,59.39151611)
\curveto(472.44006404,59.56151129)(472.47506401,59.73151112)(472.5050647,59.90151611)
\curveto(472.53506395,60.07151078)(472.5800639,60.23151062)(472.6400647,60.38151611)
\curveto(472.85006363,60.90150995)(473.10506338,61.32150953)(473.4050647,61.64151611)
\curveto(473.70506278,61.96150889)(474.11506237,62.22650863)(474.6350647,62.43651611)
\curveto(474.74506174,62.48650837)(474.86506162,62.52150833)(474.9950647,62.54151611)
\curveto(475.12506136,62.56150829)(475.26006122,62.58650827)(475.4000647,62.61651611)
\curveto(475.47006101,62.62650823)(475.54006094,62.63150822)(475.6100647,62.63151611)
\curveto(475.6800608,62.64150821)(475.75506073,62.6515082)(475.8350647,62.66151611)
}
}
{
\newrgbcolor{curcolor}{0 0 0}
\pscustom[linestyle=none,fillstyle=solid,fillcolor=curcolor]
{
\newpath
\moveto(487.50670532,58.92651611)
\curveto(487.52669764,58.82651203)(487.52669764,58.71151214)(487.50670532,58.58151611)
\curveto(487.49669767,58.46151239)(487.4666977,58.37651248)(487.41670532,58.32651611)
\curveto(487.3666978,58.28651257)(487.29169787,58.2565126)(487.19170532,58.23651611)
\curveto(487.10169806,58.22651263)(486.99669817,58.22151263)(486.87670532,58.22151611)
\lineto(486.51670532,58.22151611)
\curveto(486.39669877,58.23151262)(486.29169887,58.23651262)(486.20170532,58.23651611)
\lineto(482.36170532,58.23651611)
\curveto(482.28170288,58.23651262)(482.20170296,58.23151262)(482.12170532,58.22151611)
\curveto(482.04170312,58.22151263)(481.97670319,58.20651265)(481.92670532,58.17651611)
\curveto(481.88670328,58.1565127)(481.84670332,58.11651274)(481.80670532,58.05651611)
\curveto(481.78670338,58.02651283)(481.7667034,57.98151287)(481.74670532,57.92151611)
\curveto(481.72670344,57.87151298)(481.72670344,57.82151303)(481.74670532,57.77151611)
\curveto(481.75670341,57.72151313)(481.7617034,57.67651318)(481.76170532,57.63651611)
\curveto(481.7617034,57.59651326)(481.7667034,57.5565133)(481.77670532,57.51651611)
\curveto(481.79670337,57.43651342)(481.81670335,57.3515135)(481.83670532,57.26151611)
\curveto(481.85670331,57.18151367)(481.88670328,57.10151375)(481.92670532,57.02151611)
\curveto(482.15670301,56.48151437)(482.53670263,56.09651476)(483.06670532,55.86651611)
\curveto(483.12670204,55.83651502)(483.19170197,55.81151504)(483.26170532,55.79151611)
\lineto(483.47170532,55.73151611)
\curveto(483.50170166,55.72151513)(483.55170161,55.71651514)(483.62170532,55.71651611)
\curveto(483.7617014,55.67651518)(483.94670122,55.6565152)(484.17670532,55.65651611)
\curveto(484.40670076,55.6565152)(484.59170057,55.67651518)(484.73170532,55.71651611)
\curveto(484.87170029,55.7565151)(484.99670017,55.79651506)(485.10670532,55.83651611)
\curveto(485.22669994,55.88651497)(485.33669983,55.94651491)(485.43670532,56.01651611)
\curveto(485.54669962,56.08651477)(485.64169952,56.16651469)(485.72170532,56.25651611)
\curveto(485.80169936,56.3565145)(485.87169929,56.46151439)(485.93170532,56.57151611)
\curveto(485.99169917,56.67151418)(486.04169912,56.77651408)(486.08170532,56.88651611)
\curveto(486.13169903,56.99651386)(486.21169895,57.07651378)(486.32170532,57.12651611)
\curveto(486.3616988,57.14651371)(486.42669874,57.16151369)(486.51670532,57.17151611)
\curveto(486.60669856,57.18151367)(486.69669847,57.18151367)(486.78670532,57.17151611)
\curveto(486.87669829,57.17151368)(486.9616982,57.16651369)(487.04170532,57.15651611)
\curveto(487.12169804,57.14651371)(487.17669799,57.12651373)(487.20670532,57.09651611)
\curveto(487.30669786,57.02651383)(487.33169783,56.91151394)(487.28170532,56.75151611)
\curveto(487.20169796,56.48151437)(487.09669807,56.24151461)(486.96670532,56.03151611)
\curveto(486.7666984,55.71151514)(486.53669863,55.44651541)(486.27670532,55.23651611)
\curveto(486.02669914,55.03651582)(485.70669946,54.87151598)(485.31670532,54.74151611)
\curveto(485.21669995,54.70151615)(485.11670005,54.67651618)(485.01670532,54.66651611)
\curveto(484.91670025,54.64651621)(484.81170035,54.62651623)(484.70170532,54.60651611)
\curveto(484.65170051,54.59651626)(484.60170056,54.59151626)(484.55170532,54.59151611)
\curveto(484.51170065,54.59151626)(484.4667007,54.58651627)(484.41670532,54.57651611)
\lineto(484.26670532,54.57651611)
\curveto(484.21670095,54.56651629)(484.15670101,54.56151629)(484.08670532,54.56151611)
\curveto(484.02670114,54.56151629)(483.97670119,54.56651629)(483.93670532,54.57651611)
\lineto(483.80170532,54.57651611)
\curveto(483.75170141,54.58651627)(483.70670146,54.59151626)(483.66670532,54.59151611)
\curveto(483.62670154,54.59151626)(483.58670158,54.59651626)(483.54670532,54.60651611)
\curveto(483.49670167,54.61651624)(483.44170172,54.62651623)(483.38170532,54.63651611)
\curveto(483.32170184,54.63651622)(483.2667019,54.64151621)(483.21670532,54.65151611)
\curveto(483.12670204,54.67151618)(483.03670213,54.69651616)(482.94670532,54.72651611)
\curveto(482.85670231,54.74651611)(482.77170239,54.77151608)(482.69170532,54.80151611)
\curveto(482.65170251,54.82151603)(482.61670255,54.83151602)(482.58670532,54.83151611)
\curveto(482.55670261,54.84151601)(482.52170264,54.856516)(482.48170532,54.87651611)
\curveto(482.33170283,54.94651591)(482.17170299,55.03151582)(482.00170532,55.13151611)
\curveto(481.71170345,55.32151553)(481.4617037,55.5515153)(481.25170532,55.82151611)
\curveto(481.05170411,56.10151475)(480.88170428,56.41151444)(480.74170532,56.75151611)
\curveto(480.69170447,56.86151399)(480.65170451,56.97651388)(480.62170532,57.09651611)
\curveto(480.60170456,57.21651364)(480.57170459,57.33651352)(480.53170532,57.45651611)
\curveto(480.52170464,57.49651336)(480.51670465,57.53151332)(480.51670532,57.56151611)
\curveto(480.51670465,57.59151326)(480.51170465,57.63151322)(480.50170532,57.68151611)
\curveto(480.48170468,57.76151309)(480.4667047,57.84651301)(480.45670532,57.93651611)
\curveto(480.44670472,58.02651283)(480.43170473,58.11651274)(480.41170532,58.20651611)
\lineto(480.41170532,58.41651611)
\curveto(480.40170476,58.4565124)(480.39170477,58.51151234)(480.38170532,58.58151611)
\curveto(480.38170478,58.66151219)(480.38670478,58.72651213)(480.39670532,58.77651611)
\lineto(480.39670532,58.94151611)
\curveto(480.41670475,58.99151186)(480.42170474,59.04151181)(480.41170532,59.09151611)
\curveto(480.41170475,59.1515117)(480.41670475,59.20651165)(480.42670532,59.25651611)
\curveto(480.4667047,59.41651144)(480.49670467,59.57651128)(480.51670532,59.73651611)
\curveto(480.54670462,59.89651096)(480.59170457,60.04651081)(480.65170532,60.18651611)
\curveto(480.70170446,60.29651056)(480.74670442,60.40651045)(480.78670532,60.51651611)
\curveto(480.83670433,60.63651022)(480.89170427,60.7515101)(480.95170532,60.86151611)
\curveto(481.17170399,61.21150964)(481.42170374,61.51150934)(481.70170532,61.76151611)
\curveto(481.98170318,62.02150883)(482.32670284,62.23650862)(482.73670532,62.40651611)
\curveto(482.85670231,62.4565084)(482.97670219,62.49150836)(483.09670532,62.51151611)
\curveto(483.22670194,62.54150831)(483.3617018,62.57150828)(483.50170532,62.60151611)
\curveto(483.55170161,62.61150824)(483.59670157,62.61650824)(483.63670532,62.61651611)
\curveto(483.67670149,62.62650823)(483.72170144,62.63150822)(483.77170532,62.63151611)
\curveto(483.79170137,62.64150821)(483.81670135,62.64150821)(483.84670532,62.63151611)
\curveto(483.87670129,62.62150823)(483.90170126,62.62650823)(483.92170532,62.64651611)
\curveto(484.34170082,62.6565082)(484.70670046,62.61150824)(485.01670532,62.51151611)
\curveto(485.32669984,62.42150843)(485.60669956,62.29650856)(485.85670532,62.13651611)
\curveto(485.90669926,62.11650874)(485.94669922,62.08650877)(485.97670532,62.04651611)
\curveto(486.00669916,62.01650884)(486.04169912,61.99150886)(486.08170532,61.97151611)
\curveto(486.161699,61.91150894)(486.24169892,61.84150901)(486.32170532,61.76151611)
\curveto(486.41169875,61.68150917)(486.48669868,61.60150925)(486.54670532,61.52151611)
\curveto(486.70669846,61.31150954)(486.84169832,61.11150974)(486.95170532,60.92151611)
\curveto(487.02169814,60.81151004)(487.07669809,60.69151016)(487.11670532,60.56151611)
\curveto(487.15669801,60.43151042)(487.20169796,60.30151055)(487.25170532,60.17151611)
\curveto(487.30169786,60.04151081)(487.33669783,59.90651095)(487.35670532,59.76651611)
\curveto(487.38669778,59.62651123)(487.42169774,59.48651137)(487.46170532,59.34651611)
\curveto(487.47169769,59.27651158)(487.47669769,59.20651165)(487.47670532,59.13651611)
\lineto(487.50670532,58.92651611)
\moveto(486.05170532,59.43651611)
\curveto(486.08169908,59.47651138)(486.10669906,59.52651133)(486.12670532,59.58651611)
\curveto(486.14669902,59.6565112)(486.14669902,59.72651113)(486.12670532,59.79651611)
\curveto(486.0666991,60.01651084)(485.98169918,60.22151063)(485.87170532,60.41151611)
\curveto(485.73169943,60.64151021)(485.57669959,60.83651002)(485.40670532,60.99651611)
\curveto(485.23669993,61.1565097)(485.01670015,61.29150956)(484.74670532,61.40151611)
\curveto(484.67670049,61.42150943)(484.60670056,61.43650942)(484.53670532,61.44651611)
\curveto(484.4667007,61.46650939)(484.39170077,61.48650937)(484.31170532,61.50651611)
\curveto(484.23170093,61.52650933)(484.14670102,61.53650932)(484.05670532,61.53651611)
\lineto(483.80170532,61.53651611)
\curveto(483.77170139,61.51650934)(483.73670143,61.50650935)(483.69670532,61.50651611)
\curveto(483.65670151,61.51650934)(483.62170154,61.51650934)(483.59170532,61.50651611)
\lineto(483.35170532,61.44651611)
\curveto(483.28170188,61.43650942)(483.21170195,61.42150943)(483.14170532,61.40151611)
\curveto(482.85170231,61.28150957)(482.61670255,61.13150972)(482.43670532,60.95151611)
\curveto(482.2667029,60.77151008)(482.11170305,60.54651031)(481.97170532,60.27651611)
\curveto(481.94170322,60.22651063)(481.91170325,60.16151069)(481.88170532,60.08151611)
\curveto(481.85170331,60.01151084)(481.82670334,59.93151092)(481.80670532,59.84151611)
\curveto(481.78670338,59.7515111)(481.78170338,59.66651119)(481.79170532,59.58651611)
\curveto(481.80170336,59.50651135)(481.83670333,59.44651141)(481.89670532,59.40651611)
\curveto(481.97670319,59.34651151)(482.11170305,59.31651154)(482.30170532,59.31651611)
\curveto(482.50170266,59.32651153)(482.67170249,59.33151152)(482.81170532,59.33151611)
\lineto(485.09170532,59.33151611)
\curveto(485.24169992,59.33151152)(485.42169974,59.32651153)(485.63170532,59.31651611)
\curveto(485.84169932,59.31651154)(485.98169918,59.3565115)(486.05170532,59.43651611)
}
}
{
\newrgbcolor{curcolor}{0 0 0}
\pscustom[linestyle=none,fillstyle=solid,fillcolor=curcolor]
{
\newpath
\moveto(491.24334595,62.66151611)
\curveto(491.96334188,62.67150818)(492.56834128,62.58650827)(493.05834595,62.40651611)
\curveto(493.5483403,62.23650862)(493.92833992,61.93150892)(494.19834595,61.49151611)
\curveto(494.26833958,61.38150947)(494.32333952,61.26650959)(494.36334595,61.14651611)
\curveto(494.40333944,61.03650982)(494.4433394,60.91150994)(494.48334595,60.77151611)
\curveto(494.50333934,60.70151015)(494.50833934,60.62651023)(494.49834595,60.54651611)
\curveto(494.48833936,60.47651038)(494.47333937,60.42151043)(494.45334595,60.38151611)
\curveto(494.43333941,60.36151049)(494.40833944,60.34151051)(494.37834595,60.32151611)
\curveto(494.3483395,60.31151054)(494.32333952,60.29651056)(494.30334595,60.27651611)
\curveto(494.25333959,60.2565106)(494.20333964,60.2515106)(494.15334595,60.26151611)
\curveto(494.10333974,60.27151058)(494.05333979,60.27151058)(494.00334595,60.26151611)
\curveto(493.92333992,60.24151061)(493.81834003,60.23651062)(493.68834595,60.24651611)
\curveto(493.55834029,60.26651059)(493.46834038,60.29151056)(493.41834595,60.32151611)
\curveto(493.33834051,60.37151048)(493.28334056,60.43651042)(493.25334595,60.51651611)
\curveto(493.23334061,60.60651025)(493.19834065,60.69151016)(493.14834595,60.77151611)
\curveto(493.05834079,60.93150992)(492.93334091,61.07650978)(492.77334595,61.20651611)
\curveto(492.66334118,61.28650957)(492.5433413,61.34650951)(492.41334595,61.38651611)
\curveto(492.28334156,61.42650943)(492.1433417,61.46650939)(491.99334595,61.50651611)
\curveto(491.9433419,61.52650933)(491.89334195,61.53150932)(491.84334595,61.52151611)
\curveto(491.79334205,61.52150933)(491.7433421,61.52650933)(491.69334595,61.53651611)
\curveto(491.63334221,61.5565093)(491.55834229,61.56650929)(491.46834595,61.56651611)
\curveto(491.37834247,61.56650929)(491.30334254,61.5565093)(491.24334595,61.53651611)
\lineto(491.15334595,61.53651611)
\lineto(491.00334595,61.50651611)
\curveto(490.95334289,61.50650935)(490.90334294,61.50150935)(490.85334595,61.49151611)
\curveto(490.59334325,61.43150942)(490.37834347,61.34650951)(490.20834595,61.23651611)
\curveto(490.03834381,61.12650973)(489.92334392,60.94150991)(489.86334595,60.68151611)
\curveto(489.843344,60.61151024)(489.83834401,60.54151031)(489.84834595,60.47151611)
\curveto(489.86834398,60.40151045)(489.88834396,60.34151051)(489.90834595,60.29151611)
\curveto(489.96834388,60.14151071)(490.03834381,60.03151082)(490.11834595,59.96151611)
\curveto(490.20834364,59.90151095)(490.31834353,59.83151102)(490.44834595,59.75151611)
\curveto(490.60834324,59.6515112)(490.78834306,59.57651128)(490.98834595,59.52651611)
\curveto(491.18834266,59.48651137)(491.38834246,59.43651142)(491.58834595,59.37651611)
\curveto(491.71834213,59.33651152)(491.848342,59.30651155)(491.97834595,59.28651611)
\curveto(492.10834174,59.26651159)(492.23834161,59.23651162)(492.36834595,59.19651611)
\curveto(492.57834127,59.13651172)(492.78334106,59.07651178)(492.98334595,59.01651611)
\curveto(493.18334066,58.96651189)(493.38334046,58.90151195)(493.58334595,58.82151611)
\lineto(493.73334595,58.76151611)
\curveto(493.78334006,58.74151211)(493.83334001,58.71651214)(493.88334595,58.68651611)
\curveto(494.08333976,58.56651229)(494.25833959,58.43151242)(494.40834595,58.28151611)
\curveto(494.55833929,58.13151272)(494.68333916,57.94151291)(494.78334595,57.71151611)
\curveto(494.80333904,57.64151321)(494.82333902,57.54651331)(494.84334595,57.42651611)
\curveto(494.86333898,57.3565135)(494.87333897,57.28151357)(494.87334595,57.20151611)
\curveto(494.88333896,57.13151372)(494.88833896,57.0515138)(494.88834595,56.96151611)
\lineto(494.88834595,56.81151611)
\curveto(494.86833898,56.74151411)(494.85833899,56.67151418)(494.85834595,56.60151611)
\curveto(494.85833899,56.53151432)(494.848339,56.46151439)(494.82834595,56.39151611)
\curveto(494.79833905,56.28151457)(494.76333908,56.17651468)(494.72334595,56.07651611)
\curveto(494.68333916,55.97651488)(494.63833921,55.88651497)(494.58834595,55.80651611)
\curveto(494.42833942,55.54651531)(494.22333962,55.33651552)(493.97334595,55.17651611)
\curveto(493.72334012,55.02651583)(493.4433404,54.89651596)(493.13334595,54.78651611)
\curveto(493.0433408,54.7565161)(492.9483409,54.73651612)(492.84834595,54.72651611)
\curveto(492.75834109,54.70651615)(492.66834118,54.68151617)(492.57834595,54.65151611)
\curveto(492.47834137,54.63151622)(492.37834147,54.62151623)(492.27834595,54.62151611)
\curveto(492.17834167,54.62151623)(492.07834177,54.61151624)(491.97834595,54.59151611)
\lineto(491.82834595,54.59151611)
\curveto(491.77834207,54.58151627)(491.70834214,54.57651628)(491.61834595,54.57651611)
\curveto(491.52834232,54.57651628)(491.45834239,54.58151627)(491.40834595,54.59151611)
\lineto(491.24334595,54.59151611)
\curveto(491.18334266,54.61151624)(491.11834273,54.62151623)(491.04834595,54.62151611)
\curveto(490.97834287,54.61151624)(490.91834293,54.61651624)(490.86834595,54.63651611)
\curveto(490.81834303,54.64651621)(490.75334309,54.6515162)(490.67334595,54.65151611)
\lineto(490.43334595,54.71151611)
\curveto(490.36334348,54.72151613)(490.28834356,54.74151611)(490.20834595,54.77151611)
\curveto(489.89834395,54.87151598)(489.62834422,54.99651586)(489.39834595,55.14651611)
\curveto(489.16834468,55.29651556)(488.96834488,55.49151536)(488.79834595,55.73151611)
\curveto(488.70834514,55.86151499)(488.63334521,55.99651486)(488.57334595,56.13651611)
\curveto(488.51334533,56.27651458)(488.45834539,56.43151442)(488.40834595,56.60151611)
\curveto(488.38834546,56.66151419)(488.37834547,56.73151412)(488.37834595,56.81151611)
\curveto(488.38834546,56.90151395)(488.40334544,56.97151388)(488.42334595,57.02151611)
\curveto(488.45334539,57.06151379)(488.50334534,57.10151375)(488.57334595,57.14151611)
\curveto(488.62334522,57.16151369)(488.69334515,57.17151368)(488.78334595,57.17151611)
\curveto(488.87334497,57.18151367)(488.96334488,57.18151367)(489.05334595,57.17151611)
\curveto(489.1433447,57.16151369)(489.22834462,57.14651371)(489.30834595,57.12651611)
\curveto(489.39834445,57.11651374)(489.45834439,57.10151375)(489.48834595,57.08151611)
\curveto(489.55834429,57.03151382)(489.60334424,56.9565139)(489.62334595,56.85651611)
\curveto(489.65334419,56.76651409)(489.68834416,56.68151417)(489.72834595,56.60151611)
\curveto(489.82834402,56.38151447)(489.96334388,56.21151464)(490.13334595,56.09151611)
\curveto(490.25334359,56.00151485)(490.38834346,55.93151492)(490.53834595,55.88151611)
\curveto(490.68834316,55.83151502)(490.848343,55.78151507)(491.01834595,55.73151611)
\lineto(491.33334595,55.68651611)
\lineto(491.42334595,55.68651611)
\curveto(491.49334235,55.66651519)(491.58334226,55.6565152)(491.69334595,55.65651611)
\curveto(491.81334203,55.6565152)(491.91334193,55.66651519)(491.99334595,55.68651611)
\curveto(492.06334178,55.68651517)(492.11834173,55.69151516)(492.15834595,55.70151611)
\curveto(492.21834163,55.71151514)(492.27834157,55.71651514)(492.33834595,55.71651611)
\curveto(492.39834145,55.72651513)(492.45334139,55.73651512)(492.50334595,55.74651611)
\curveto(492.79334105,55.82651503)(493.02334082,55.93151492)(493.19334595,56.06151611)
\curveto(493.36334048,56.19151466)(493.48334036,56.41151444)(493.55334595,56.72151611)
\curveto(493.57334027,56.77151408)(493.57834027,56.82651403)(493.56834595,56.88651611)
\curveto(493.55834029,56.94651391)(493.5483403,56.99151386)(493.53834595,57.02151611)
\curveto(493.48834036,57.21151364)(493.41834043,57.3515135)(493.32834595,57.44151611)
\curveto(493.23834061,57.54151331)(493.12334072,57.63151322)(492.98334595,57.71151611)
\curveto(492.89334095,57.77151308)(492.79334105,57.82151303)(492.68334595,57.86151611)
\lineto(492.35334595,57.98151611)
\curveto(492.32334152,57.99151286)(492.29334155,57.99651286)(492.26334595,57.99651611)
\curveto(492.2433416,57.99651286)(492.21834163,58.00651285)(492.18834595,58.02651611)
\curveto(491.848342,58.13651272)(491.49334235,58.21651264)(491.12334595,58.26651611)
\curveto(490.76334308,58.32651253)(490.42334342,58.42151243)(490.10334595,58.55151611)
\curveto(490.00334384,58.59151226)(489.90834394,58.62651223)(489.81834595,58.65651611)
\curveto(489.72834412,58.68651217)(489.6433442,58.72651213)(489.56334595,58.77651611)
\curveto(489.37334447,58.88651197)(489.19834465,59.01151184)(489.03834595,59.15151611)
\curveto(488.87834497,59.29151156)(488.75334509,59.46651139)(488.66334595,59.67651611)
\curveto(488.63334521,59.74651111)(488.60834524,59.81651104)(488.58834595,59.88651611)
\curveto(488.57834527,59.9565109)(488.56334528,60.03151082)(488.54334595,60.11151611)
\curveto(488.51334533,60.23151062)(488.50334534,60.36651049)(488.51334595,60.51651611)
\curveto(488.52334532,60.67651018)(488.53834531,60.81151004)(488.55834595,60.92151611)
\curveto(488.57834527,60.97150988)(488.58834526,61.01150984)(488.58834595,61.04151611)
\curveto(488.59834525,61.08150977)(488.61334523,61.12150973)(488.63334595,61.16151611)
\curveto(488.72334512,61.39150946)(488.843345,61.59150926)(488.99334595,61.76151611)
\curveto(489.15334469,61.93150892)(489.33334451,62.08150877)(489.53334595,62.21151611)
\curveto(489.68334416,62.30150855)(489.848344,62.37150848)(490.02834595,62.42151611)
\curveto(490.20834364,62.48150837)(490.39834345,62.53650832)(490.59834595,62.58651611)
\curveto(490.66834318,62.59650826)(490.73334311,62.60650825)(490.79334595,62.61651611)
\curveto(490.86334298,62.62650823)(490.93834291,62.63650822)(491.01834595,62.64651611)
\curveto(491.0483428,62.6565082)(491.08834276,62.6565082)(491.13834595,62.64651611)
\curveto(491.18834266,62.63650822)(491.22334262,62.64150821)(491.24334595,62.66151611)
}
}
{
\newrgbcolor{curcolor}{0 0 0}
\pscustom[linestyle=none,fillstyle=solid,fillcolor=curcolor]
{
\newpath
\moveto(79.44210083,76.16295776)
\lineto(79.44210083,75.90795776)
\curveto(79.45209313,75.827953)(79.44709313,75.75295307)(79.42710083,75.68295776)
\lineto(79.42710083,75.44295776)
\lineto(79.42710083,75.27795776)
\curveto(79.40709317,75.17795365)(79.39709318,75.07295375)(79.39710083,74.96295776)
\curveto(79.39709318,74.86295396)(79.38709319,74.76295406)(79.36710083,74.66295776)
\lineto(79.36710083,74.51295776)
\curveto(79.33709324,74.37295445)(79.31709326,74.23295459)(79.30710083,74.09295776)
\curveto(79.29709328,73.96295486)(79.27209331,73.83295499)(79.23210083,73.70295776)
\curveto(79.21209337,73.6229552)(79.19209339,73.53795529)(79.17210083,73.44795776)
\lineto(79.11210083,73.20795776)
\lineto(78.99210083,72.90795776)
\curveto(78.96209362,72.81795601)(78.92709365,72.7279561)(78.88710083,72.63795776)
\curveto(78.78709379,72.41795641)(78.65209393,72.20295662)(78.48210083,71.99295776)
\curveto(78.32209426,71.78295704)(78.14709443,71.61295721)(77.95710083,71.48295776)
\curveto(77.90709467,71.44295738)(77.84709473,71.40295742)(77.77710083,71.36295776)
\curveto(77.71709486,71.33295749)(77.65709492,71.29795753)(77.59710083,71.25795776)
\curveto(77.51709506,71.20795762)(77.42209516,71.16795766)(77.31210083,71.13795776)
\curveto(77.20209538,71.10795772)(77.09709548,71.07795775)(76.99710083,71.04795776)
\curveto(76.88709569,71.00795782)(76.7770958,70.98295784)(76.66710083,70.97295776)
\curveto(76.55709602,70.96295786)(76.44209614,70.94795788)(76.32210083,70.92795776)
\curveto(76.2820963,70.91795791)(76.23709634,70.91795791)(76.18710083,70.92795776)
\curveto(76.14709643,70.9279579)(76.10709647,70.9229579)(76.06710083,70.91295776)
\curveto(76.02709655,70.90295792)(75.97209661,70.89795793)(75.90210083,70.89795776)
\curveto(75.83209675,70.89795793)(75.7820968,70.90295792)(75.75210083,70.91295776)
\curveto(75.70209688,70.93295789)(75.65709692,70.93795789)(75.61710083,70.92795776)
\curveto(75.577097,70.91795791)(75.54209704,70.91795791)(75.51210083,70.92795776)
\lineto(75.42210083,70.92795776)
\curveto(75.36209722,70.94795788)(75.29709728,70.96295786)(75.22710083,70.97295776)
\curveto(75.16709741,70.97295785)(75.10209748,70.97795785)(75.03210083,70.98795776)
\curveto(74.86209772,71.03795779)(74.70209788,71.08795774)(74.55210083,71.13795776)
\curveto(74.40209818,71.18795764)(74.25709832,71.25295757)(74.11710083,71.33295776)
\curveto(74.06709851,71.37295745)(74.01209857,71.40295742)(73.95210083,71.42295776)
\curveto(73.90209868,71.45295737)(73.85209873,71.48795734)(73.80210083,71.52795776)
\curveto(73.56209902,71.70795712)(73.36209922,71.9279569)(73.20210083,72.18795776)
\curveto(73.04209954,72.44795638)(72.90209968,72.73295609)(72.78210083,73.04295776)
\curveto(72.72209986,73.18295564)(72.6770999,73.3229555)(72.64710083,73.46295776)
\curveto(72.61709996,73.61295521)(72.5821,73.76795506)(72.54210083,73.92795776)
\curveto(72.52210006,74.03795479)(72.50710007,74.14795468)(72.49710083,74.25795776)
\curveto(72.48710009,74.36795446)(72.47210011,74.47795435)(72.45210083,74.58795776)
\curveto(72.44210014,74.6279542)(72.43710014,74.66795416)(72.43710083,74.70795776)
\curveto(72.44710013,74.74795408)(72.44710013,74.78795404)(72.43710083,74.82795776)
\curveto(72.42710015,74.87795395)(72.42210016,74.9279539)(72.42210083,74.97795776)
\lineto(72.42210083,75.14295776)
\curveto(72.40210018,75.19295363)(72.39710018,75.24295358)(72.40710083,75.29295776)
\curveto(72.41710016,75.35295347)(72.41710016,75.40795342)(72.40710083,75.45795776)
\curveto(72.39710018,75.49795333)(72.39710018,75.54295328)(72.40710083,75.59295776)
\curveto(72.41710016,75.64295318)(72.41210017,75.69295313)(72.39210083,75.74295776)
\curveto(72.37210021,75.81295301)(72.36710021,75.88795294)(72.37710083,75.96795776)
\curveto(72.38710019,76.05795277)(72.39210019,76.14295268)(72.39210083,76.22295776)
\curveto(72.39210019,76.31295251)(72.38710019,76.41295241)(72.37710083,76.52295776)
\curveto(72.36710021,76.64295218)(72.37210021,76.74295208)(72.39210083,76.82295776)
\lineto(72.39210083,77.10795776)
\lineto(72.43710083,77.73795776)
\curveto(72.44710013,77.83795099)(72.45710012,77.93295089)(72.46710083,78.02295776)
\lineto(72.49710083,78.32295776)
\curveto(72.51710006,78.37295045)(72.52210006,78.4229504)(72.51210083,78.47295776)
\curveto(72.51210007,78.53295029)(72.52210006,78.58795024)(72.54210083,78.63795776)
\curveto(72.59209999,78.80795002)(72.63209995,78.97294985)(72.66210083,79.13295776)
\curveto(72.69209989,79.30294952)(72.74209984,79.46294936)(72.81210083,79.61295776)
\curveto(73.00209958,80.07294875)(73.22209936,80.44794838)(73.47210083,80.73795776)
\curveto(73.73209885,81.0279478)(74.09209849,81.27294755)(74.55210083,81.47295776)
\curveto(74.6820979,81.5229473)(74.81209777,81.55794727)(74.94210083,81.57795776)
\curveto(75.0820975,81.59794723)(75.22209736,81.6229472)(75.36210083,81.65295776)
\curveto(75.43209715,81.66294716)(75.49709708,81.66794716)(75.55710083,81.66795776)
\curveto(75.61709696,81.66794716)(75.6820969,81.67294715)(75.75210083,81.68295776)
\curveto(76.582096,81.70294712)(77.25209533,81.55294727)(77.76210083,81.23295776)
\curveto(78.27209431,80.9229479)(78.65209393,80.48294834)(78.90210083,79.91295776)
\curveto(78.95209363,79.79294903)(78.99709358,79.66794916)(79.03710083,79.53795776)
\curveto(79.0770935,79.40794942)(79.12209346,79.27294955)(79.17210083,79.13295776)
\curveto(79.19209339,79.05294977)(79.20709337,78.96794986)(79.21710083,78.87795776)
\lineto(79.27710083,78.63795776)
\curveto(79.30709327,78.5279503)(79.32209326,78.41795041)(79.32210083,78.30795776)
\curveto(79.33209325,78.19795063)(79.34709323,78.08795074)(79.36710083,77.97795776)
\curveto(79.38709319,77.9279509)(79.39209319,77.88295094)(79.38210083,77.84295776)
\curveto(79.3820932,77.80295102)(79.38709319,77.76295106)(79.39710083,77.72295776)
\curveto(79.40709317,77.67295115)(79.40709317,77.61795121)(79.39710083,77.55795776)
\curveto(79.39709318,77.50795132)(79.40209318,77.45795137)(79.41210083,77.40795776)
\lineto(79.41210083,77.27295776)
\curveto(79.43209315,77.21295161)(79.43209315,77.14295168)(79.41210083,77.06295776)
\curveto(79.40209318,76.99295183)(79.40709317,76.9279519)(79.42710083,76.86795776)
\curveto(79.43709314,76.83795199)(79.44209314,76.79795203)(79.44210083,76.74795776)
\lineto(79.44210083,76.62795776)
\lineto(79.44210083,76.16295776)
\moveto(77.89710083,73.83795776)
\curveto(77.99709458,74.15795467)(78.05709452,74.5229543)(78.07710083,74.93295776)
\curveto(78.09709448,75.34295348)(78.10709447,75.75295307)(78.10710083,76.16295776)
\curveto(78.10709447,76.59295223)(78.09709448,77.01295181)(78.07710083,77.42295776)
\curveto(78.05709452,77.83295099)(78.01209457,78.21795061)(77.94210083,78.57795776)
\curveto(77.87209471,78.93794989)(77.76209482,79.25794957)(77.61210083,79.53795776)
\curveto(77.47209511,79.827949)(77.2770953,80.06294876)(77.02710083,80.24295776)
\curveto(76.86709571,80.35294847)(76.68709589,80.43294839)(76.48710083,80.48295776)
\curveto(76.28709629,80.54294828)(76.04209654,80.57294825)(75.75210083,80.57295776)
\curveto(75.73209685,80.55294827)(75.69709688,80.54294828)(75.64710083,80.54295776)
\curveto(75.59709698,80.55294827)(75.55709702,80.55294827)(75.52710083,80.54295776)
\curveto(75.44709713,80.5229483)(75.37209721,80.50294832)(75.30210083,80.48295776)
\curveto(75.24209734,80.47294835)(75.1770974,80.45294837)(75.10710083,80.42295776)
\curveto(74.83709774,80.30294852)(74.61709796,80.13294869)(74.44710083,79.91295776)
\curveto(74.28709829,79.70294912)(74.15209843,79.45794937)(74.04210083,79.17795776)
\curveto(73.99209859,79.06794976)(73.95209863,78.94794988)(73.92210083,78.81795776)
\curveto(73.90209868,78.69795013)(73.8770987,78.57295025)(73.84710083,78.44295776)
\curveto(73.82709875,78.39295043)(73.81709876,78.33795049)(73.81710083,78.27795776)
\curveto(73.81709876,78.2279506)(73.81209877,78.17795065)(73.80210083,78.12795776)
\curveto(73.79209879,78.03795079)(73.7820988,77.94295088)(73.77210083,77.84295776)
\curveto(73.76209882,77.75295107)(73.75209883,77.65795117)(73.74210083,77.55795776)
\curveto(73.74209884,77.47795135)(73.73709884,77.39295143)(73.72710083,77.30295776)
\lineto(73.72710083,77.06295776)
\lineto(73.72710083,76.88295776)
\curveto(73.71709886,76.85295197)(73.71209887,76.81795201)(73.71210083,76.77795776)
\lineto(73.71210083,76.64295776)
\lineto(73.71210083,76.19295776)
\curveto(73.71209887,76.11295271)(73.70709887,76.0279528)(73.69710083,75.93795776)
\curveto(73.69709888,75.85795297)(73.70709887,75.78295304)(73.72710083,75.71295776)
\lineto(73.72710083,75.44295776)
\curveto(73.72709885,75.4229534)(73.72209886,75.39295343)(73.71210083,75.35295776)
\curveto(73.71209887,75.3229535)(73.71709886,75.29795353)(73.72710083,75.27795776)
\curveto(73.73709884,75.17795365)(73.74209884,75.07795375)(73.74210083,74.97795776)
\curveto(73.75209883,74.88795394)(73.76209882,74.78795404)(73.77210083,74.67795776)
\curveto(73.80209878,74.55795427)(73.81709876,74.43295439)(73.81710083,74.30295776)
\curveto(73.82709875,74.18295464)(73.85209873,74.06795476)(73.89210083,73.95795776)
\curveto(73.97209861,73.65795517)(74.05709852,73.39295543)(74.14710083,73.16295776)
\curveto(74.24709833,72.93295589)(74.39209819,72.71795611)(74.58210083,72.51795776)
\curveto(74.79209779,72.31795651)(75.05709752,72.16795666)(75.37710083,72.06795776)
\curveto(75.41709716,72.04795678)(75.45209713,72.03795679)(75.48210083,72.03795776)
\curveto(75.52209706,72.04795678)(75.56709701,72.04295678)(75.61710083,72.02295776)
\curveto(75.65709692,72.01295681)(75.72709685,72.00295682)(75.82710083,71.99295776)
\curveto(75.93709664,71.98295684)(76.02209656,71.98795684)(76.08210083,72.00795776)
\curveto(76.15209643,72.0279568)(76.22209636,72.03795679)(76.29210083,72.03795776)
\curveto(76.36209622,72.04795678)(76.42709615,72.06295676)(76.48710083,72.08295776)
\curveto(76.68709589,72.14295668)(76.86709571,72.2279566)(77.02710083,72.33795776)
\curveto(77.05709552,72.35795647)(77.0820955,72.37795645)(77.10210083,72.39795776)
\lineto(77.16210083,72.45795776)
\curveto(77.20209538,72.47795635)(77.25209533,72.51795631)(77.31210083,72.57795776)
\curveto(77.41209517,72.71795611)(77.49709508,72.84795598)(77.56710083,72.96795776)
\curveto(77.63709494,73.08795574)(77.70709487,73.23295559)(77.77710083,73.40295776)
\curveto(77.80709477,73.47295535)(77.82709475,73.54295528)(77.83710083,73.61295776)
\curveto(77.85709472,73.68295514)(77.8770947,73.75795507)(77.89710083,73.83795776)
}
}
{
\newrgbcolor{curcolor}{0 0 0}
\pscustom[linestyle=none,fillstyle=solid,fillcolor=curcolor]
{
\newpath
\moveto(58.99288208,155.96866333)
\curveto(59.68287745,155.9786527)(60.28287685,155.85865282)(60.79288208,155.60866333)
\curveto(61.31287582,155.35865332)(61.70787542,155.02365366)(61.97788208,154.60366333)
\curveto(62.0278751,154.52365416)(62.07287506,154.43365425)(62.11288208,154.33366333)
\curveto(62.15287498,154.24365444)(62.19787493,154.14865453)(62.24788208,154.04866333)
\curveto(62.28787484,153.94865473)(62.31787481,153.84865483)(62.33788208,153.74866333)
\curveto(62.35787477,153.64865503)(62.37787475,153.54365514)(62.39788208,153.43366333)
\curveto(62.41787471,153.3836553)(62.42287471,153.33865534)(62.41288208,153.29866333)
\curveto(62.40287473,153.25865542)(62.40787472,153.21365547)(62.42788208,153.16366333)
\curveto(62.43787469,153.11365557)(62.44287469,153.02865565)(62.44288208,152.90866333)
\curveto(62.44287469,152.79865588)(62.43787469,152.71365597)(62.42788208,152.65366333)
\curveto(62.40787472,152.59365609)(62.39787473,152.53365615)(62.39788208,152.47366333)
\curveto(62.40787472,152.41365627)(62.40287473,152.35365633)(62.38288208,152.29366333)
\curveto(62.34287479,152.15365653)(62.30787482,152.01865666)(62.27788208,151.88866333)
\curveto(62.24787488,151.75865692)(62.20787492,151.63365705)(62.15788208,151.51366333)
\curveto(62.09787503,151.37365731)(62.0278751,151.24865743)(61.94788208,151.13866333)
\curveto(61.87787525,151.02865765)(61.80287533,150.91865776)(61.72288208,150.80866333)
\lineto(61.66288208,150.74866333)
\curveto(61.65287548,150.72865795)(61.63787549,150.70865797)(61.61788208,150.68866333)
\curveto(61.49787563,150.52865815)(61.36287577,150.3836583)(61.21288208,150.25366333)
\curveto(61.06287607,150.12365856)(60.90287623,149.99865868)(60.73288208,149.87866333)
\curveto(60.42287671,149.65865902)(60.127877,149.45365923)(59.84788208,149.26366333)
\curveto(59.61787751,149.12365956)(59.38787774,148.98865969)(59.15788208,148.85866333)
\curveto(58.93787819,148.72865995)(58.71787841,148.59366009)(58.49788208,148.45366333)
\curveto(58.24787888,148.2836604)(58.00787912,148.10366058)(57.77788208,147.91366333)
\curveto(57.55787957,147.72366096)(57.36787976,147.49866118)(57.20788208,147.23866333)
\curveto(57.16787996,147.1786615)(57.13288,147.11866156)(57.10288208,147.05866333)
\curveto(57.07288006,147.00866167)(57.04288009,146.94366174)(57.01288208,146.86366333)
\curveto(56.99288014,146.79366189)(56.98788014,146.73366195)(56.99788208,146.68366333)
\curveto(57.01788011,146.61366207)(57.05288008,146.55866212)(57.10288208,146.51866333)
\curveto(57.15287998,146.48866219)(57.21287992,146.46866221)(57.28288208,146.45866333)
\lineto(57.52288208,146.45866333)
\lineto(58.27288208,146.45866333)
\lineto(61.07788208,146.45866333)
\lineto(61.73788208,146.45866333)
\curveto(61.8278753,146.45866222)(61.91287522,146.45366223)(61.99288208,146.44366333)
\curveto(62.07287506,146.44366224)(62.13787499,146.42366226)(62.18788208,146.38366333)
\curveto(62.23787489,146.34366234)(62.27787485,146.26866241)(62.30788208,146.15866333)
\curveto(62.34787478,146.05866262)(62.35787477,145.95866272)(62.33788208,145.85866333)
\lineto(62.33788208,145.72366333)
\curveto(62.31787481,145.65366303)(62.29787483,145.59366309)(62.27788208,145.54366333)
\curveto(62.25787487,145.49366319)(62.22287491,145.45366323)(62.17288208,145.42366333)
\curveto(62.12287501,145.3836633)(62.05287508,145.36366332)(61.96288208,145.36366333)
\lineto(61.69288208,145.36366333)
\lineto(60.79288208,145.36366333)
\lineto(57.28288208,145.36366333)
\lineto(56.21788208,145.36366333)
\curveto(56.13788099,145.36366332)(56.04788108,145.35866332)(55.94788208,145.34866333)
\curveto(55.84788128,145.34866333)(55.76288137,145.35866332)(55.69288208,145.37866333)
\curveto(55.48288165,145.44866323)(55.41788171,145.62866305)(55.49788208,145.91866333)
\curveto(55.50788162,145.95866272)(55.50788162,145.99366269)(55.49788208,146.02366333)
\curveto(55.49788163,146.06366262)(55.50788162,146.10866257)(55.52788208,146.15866333)
\curveto(55.54788158,146.23866244)(55.56788156,146.32366236)(55.58788208,146.41366333)
\curveto(55.60788152,146.50366218)(55.6328815,146.58866209)(55.66288208,146.66866333)
\curveto(55.82288131,147.15866152)(56.02288111,147.57366111)(56.26288208,147.91366333)
\curveto(56.44288069,148.16366052)(56.64788048,148.38866029)(56.87788208,148.58866333)
\curveto(57.10788002,148.79865988)(57.34787978,148.99365969)(57.59788208,149.17366333)
\curveto(57.85787927,149.35365933)(58.12287901,149.52365916)(58.39288208,149.68366333)
\curveto(58.67287846,149.85365883)(58.94287819,150.02865865)(59.20288208,150.20866333)
\curveto(59.31287782,150.28865839)(59.41787771,150.36365832)(59.51788208,150.43366333)
\curveto(59.6278775,150.50365818)(59.73787739,150.5786581)(59.84788208,150.65866333)
\curveto(59.88787724,150.68865799)(59.92287721,150.71865796)(59.95288208,150.74866333)
\curveto(59.99287714,150.78865789)(60.0328771,150.81865786)(60.07288208,150.83866333)
\curveto(60.21287692,150.94865773)(60.33787679,151.07365761)(60.44788208,151.21366333)
\curveto(60.46787666,151.24365744)(60.49287664,151.26865741)(60.52288208,151.28866333)
\curveto(60.55287658,151.31865736)(60.57787655,151.34865733)(60.59788208,151.37866333)
\curveto(60.67787645,151.4786572)(60.74287639,151.5786571)(60.79288208,151.67866333)
\curveto(60.85287628,151.7786569)(60.90787622,151.88865679)(60.95788208,152.00866333)
\curveto(60.98787614,152.0786566)(61.00787612,152.15365653)(61.01788208,152.23366333)
\lineto(61.07788208,152.47366333)
\lineto(61.07788208,152.56366333)
\curveto(61.08787604,152.59365609)(61.09287604,152.62365606)(61.09288208,152.65366333)
\curveto(61.11287602,152.72365596)(61.11787601,152.81865586)(61.10788208,152.93866333)
\curveto(61.10787602,153.06865561)(61.09787603,153.16865551)(61.07788208,153.23866333)
\curveto(61.05787607,153.31865536)(61.03787609,153.39365529)(61.01788208,153.46366333)
\curveto(61.00787612,153.54365514)(60.98787614,153.62365506)(60.95788208,153.70366333)
\curveto(60.84787628,153.94365474)(60.69787643,154.14365454)(60.50788208,154.30366333)
\curveto(60.3278768,154.47365421)(60.10787702,154.61365407)(59.84788208,154.72366333)
\curveto(59.77787735,154.74365394)(59.70787742,154.75865392)(59.63788208,154.76866333)
\curveto(59.56787756,154.78865389)(59.49287764,154.80865387)(59.41288208,154.82866333)
\curveto(59.3328778,154.84865383)(59.22287791,154.85865382)(59.08288208,154.85866333)
\curveto(58.95287818,154.85865382)(58.84787828,154.84865383)(58.76788208,154.82866333)
\curveto(58.70787842,154.81865386)(58.65287848,154.81365387)(58.60288208,154.81366333)
\curveto(58.55287858,154.81365387)(58.50287863,154.80365388)(58.45288208,154.78366333)
\curveto(58.35287878,154.74365394)(58.25787887,154.70365398)(58.16788208,154.66366333)
\curveto(58.08787904,154.62365406)(58.00787912,154.5786541)(57.92788208,154.52866333)
\curveto(57.89787923,154.50865417)(57.86787926,154.4836542)(57.83788208,154.45366333)
\curveto(57.81787931,154.42365426)(57.79287934,154.39865428)(57.76288208,154.37866333)
\lineto(57.68788208,154.30366333)
\curveto(57.65787947,154.2836544)(57.6328795,154.26365442)(57.61288208,154.24366333)
\lineto(57.46288208,154.03366333)
\curveto(57.42287971,153.97365471)(57.37787975,153.90865477)(57.32788208,153.83866333)
\curveto(57.26787986,153.74865493)(57.21787991,153.64365504)(57.17788208,153.52366333)
\curveto(57.14787998,153.41365527)(57.11288002,153.30365538)(57.07288208,153.19366333)
\curveto(57.0328801,153.0836556)(57.00788012,152.93865574)(56.99788208,152.75866333)
\curveto(56.98788014,152.58865609)(56.95788017,152.46365622)(56.90788208,152.38366333)
\curveto(56.85788027,152.30365638)(56.78288035,152.25865642)(56.68288208,152.24866333)
\curveto(56.58288055,152.23865644)(56.47288066,152.23365645)(56.35288208,152.23366333)
\curveto(56.31288082,152.23365645)(56.27288086,152.22865645)(56.23288208,152.21866333)
\curveto(56.19288094,152.21865646)(56.15788097,152.22365646)(56.12788208,152.23366333)
\curveto(56.07788105,152.25365643)(56.0278811,152.26365642)(55.97788208,152.26366333)
\curveto(55.93788119,152.26365642)(55.89788123,152.27365641)(55.85788208,152.29366333)
\curveto(55.76788136,152.35365633)(55.72288141,152.48865619)(55.72288208,152.69866333)
\lineto(55.72288208,152.81866333)
\curveto(55.7328814,152.8786558)(55.73788139,152.93865574)(55.73788208,152.99866333)
\curveto(55.74788138,153.06865561)(55.75788137,153.13365555)(55.76788208,153.19366333)
\curveto(55.78788134,153.30365538)(55.80788132,153.40365528)(55.82788208,153.49366333)
\curveto(55.84788128,153.59365509)(55.87788125,153.68865499)(55.91788208,153.77866333)
\curveto(55.93788119,153.84865483)(55.95788117,153.90865477)(55.97788208,153.95866333)
\lineto(56.03788208,154.13866333)
\curveto(56.15788097,154.39865428)(56.31288082,154.64365404)(56.50288208,154.87366333)
\curveto(56.70288043,155.10365358)(56.91788021,155.28865339)(57.14788208,155.42866333)
\curveto(57.25787987,155.50865317)(57.37287976,155.57365311)(57.49288208,155.62366333)
\lineto(57.88288208,155.77366333)
\curveto(57.99287914,155.82365286)(58.10787902,155.85365283)(58.22788208,155.86366333)
\curveto(58.34787878,155.8836528)(58.47287866,155.90865277)(58.60288208,155.93866333)
\curveto(58.67287846,155.93865274)(58.73787839,155.93865274)(58.79788208,155.93866333)
\curveto(58.85787827,155.94865273)(58.92287821,155.95865272)(58.99288208,155.96866333)
}
}
{
\newrgbcolor{curcolor}{0 0 0}
\pscustom[linestyle=none,fillstyle=solid,fillcolor=curcolor]
{
\newpath
\moveto(65.60249146,155.77366333)
\lineto(69.20249146,155.77366333)
\lineto(69.84749146,155.77366333)
\curveto(69.92748493,155.77365291)(70.00248485,155.76865291)(70.07249146,155.75866333)
\curveto(70.14248471,155.75865292)(70.20248465,155.74865293)(70.25249146,155.72866333)
\curveto(70.32248453,155.69865298)(70.37748448,155.63865304)(70.41749146,155.54866333)
\curveto(70.43748442,155.51865316)(70.44748441,155.4786532)(70.44749146,155.42866333)
\lineto(70.44749146,155.29366333)
\curveto(70.4574844,155.1836535)(70.4524844,155.0786536)(70.43249146,154.97866333)
\curveto(70.42248443,154.8786538)(70.38748447,154.80865387)(70.32749146,154.76866333)
\curveto(70.23748462,154.69865398)(70.10248475,154.66365402)(69.92249146,154.66366333)
\curveto(69.74248511,154.67365401)(69.57748528,154.678654)(69.42749146,154.67866333)
\lineto(67.43249146,154.67866333)
\lineto(66.93749146,154.67866333)
\lineto(66.80249146,154.67866333)
\curveto(66.76248809,154.678654)(66.72248813,154.67365401)(66.68249146,154.66366333)
\lineto(66.47249146,154.66366333)
\curveto(66.36248849,154.63365405)(66.28248857,154.59365409)(66.23249146,154.54366333)
\curveto(66.18248867,154.50365418)(66.14748871,154.44865423)(66.12749146,154.37866333)
\curveto(66.10748875,154.31865436)(66.09248876,154.24865443)(66.08249146,154.16866333)
\curveto(66.07248878,154.08865459)(66.0524888,153.99865468)(66.02249146,153.89866333)
\curveto(65.97248888,153.69865498)(65.93248892,153.49365519)(65.90249146,153.28366333)
\curveto(65.87248898,153.07365561)(65.83248902,152.86865581)(65.78249146,152.66866333)
\curveto(65.76248909,152.59865608)(65.7524891,152.52865615)(65.75249146,152.45866333)
\curveto(65.7524891,152.39865628)(65.74248911,152.33365635)(65.72249146,152.26366333)
\curveto(65.71248914,152.23365645)(65.70248915,152.19365649)(65.69249146,152.14366333)
\curveto(65.69248916,152.10365658)(65.69748916,152.06365662)(65.70749146,152.02366333)
\curveto(65.72748913,151.97365671)(65.7524891,151.92865675)(65.78249146,151.88866333)
\curveto(65.82248903,151.85865682)(65.88248897,151.85365683)(65.96249146,151.87366333)
\curveto(66.02248883,151.89365679)(66.08248877,151.91865676)(66.14249146,151.94866333)
\curveto(66.20248865,151.98865669)(66.26248859,152.02365666)(66.32249146,152.05366333)
\curveto(66.38248847,152.07365661)(66.43248842,152.08865659)(66.47249146,152.09866333)
\curveto(66.66248819,152.1786565)(66.86748799,152.23365645)(67.08749146,152.26366333)
\curveto(67.31748754,152.29365639)(67.54748731,152.30365638)(67.77749146,152.29366333)
\curveto(68.01748684,152.29365639)(68.24748661,152.26865641)(68.46749146,152.21866333)
\curveto(68.68748617,152.1786565)(68.88748597,152.11865656)(69.06749146,152.03866333)
\curveto(69.11748574,152.01865666)(69.16248569,151.99865668)(69.20249146,151.97866333)
\curveto(69.2524856,151.95865672)(69.30248555,151.93365675)(69.35249146,151.90366333)
\curveto(69.70248515,151.69365699)(69.98248487,151.46365722)(70.19249146,151.21366333)
\curveto(70.41248444,150.96365772)(70.60748425,150.63865804)(70.77749146,150.23866333)
\curveto(70.82748403,150.12865855)(70.86248399,150.01865866)(70.88249146,149.90866333)
\curveto(70.90248395,149.79865888)(70.92748393,149.683659)(70.95749146,149.56366333)
\curveto(70.96748389,149.53365915)(70.97248388,149.48865919)(70.97249146,149.42866333)
\curveto(70.99248386,149.36865931)(71.00248385,149.29865938)(71.00249146,149.21866333)
\curveto(71.00248385,149.14865953)(71.01248384,149.0836596)(71.03249146,149.02366333)
\lineto(71.03249146,148.85866333)
\curveto(71.04248381,148.80865987)(71.04748381,148.73865994)(71.04749146,148.64866333)
\curveto(71.04748381,148.55866012)(71.03748382,148.48866019)(71.01749146,148.43866333)
\curveto(70.99748386,148.3786603)(70.99248386,148.31866036)(71.00249146,148.25866333)
\curveto(71.01248384,148.20866047)(71.00748385,148.15866052)(70.98749146,148.10866333)
\curveto(70.94748391,147.94866073)(70.91248394,147.79866088)(70.88249146,147.65866333)
\curveto(70.852484,147.51866116)(70.80748405,147.3836613)(70.74749146,147.25366333)
\curveto(70.58748427,146.8836618)(70.36748449,146.54866213)(70.08749146,146.24866333)
\curveto(69.80748505,145.94866273)(69.48748537,145.71866296)(69.12749146,145.55866333)
\curveto(68.9574859,145.4786632)(68.7574861,145.40366328)(68.52749146,145.33366333)
\curveto(68.41748644,145.29366339)(68.30248655,145.26866341)(68.18249146,145.25866333)
\curveto(68.06248679,145.24866343)(67.94248691,145.22866345)(67.82249146,145.19866333)
\curveto(67.77248708,145.1786635)(67.71748714,145.1786635)(67.65749146,145.19866333)
\curveto(67.59748726,145.20866347)(67.53748732,145.20366348)(67.47749146,145.18366333)
\curveto(67.37748748,145.16366352)(67.27748758,145.16366352)(67.17749146,145.18366333)
\lineto(67.04249146,145.18366333)
\curveto(66.99248786,145.20366348)(66.93248792,145.21366347)(66.86249146,145.21366333)
\curveto(66.80248805,145.20366348)(66.74748811,145.20866347)(66.69749146,145.22866333)
\curveto(66.6574882,145.23866344)(66.62248823,145.24366344)(66.59249146,145.24366333)
\curveto(66.56248829,145.24366344)(66.52748833,145.24866343)(66.48749146,145.25866333)
\lineto(66.21749146,145.31866333)
\curveto(66.12748873,145.33866334)(66.04248881,145.36866331)(65.96249146,145.40866333)
\curveto(65.62248923,145.54866313)(65.33248952,145.70366298)(65.09249146,145.87366333)
\curveto(64.85249,146.05366263)(64.63249022,146.2836624)(64.43249146,146.56366333)
\curveto(64.28249057,146.79366189)(64.16749069,147.03366165)(64.08749146,147.28366333)
\curveto(64.06749079,147.33366135)(64.0574908,147.3786613)(64.05749146,147.41866333)
\curveto(64.0574908,147.46866121)(64.04749081,147.51866116)(64.02749146,147.56866333)
\curveto(64.00749085,147.62866105)(63.99249086,147.70866097)(63.98249146,147.80866333)
\curveto(63.98249087,147.90866077)(64.00249085,147.9836607)(64.04249146,148.03366333)
\curveto(64.09249076,148.11366057)(64.17249068,148.15866052)(64.28249146,148.16866333)
\curveto(64.39249046,148.1786605)(64.50749035,148.1836605)(64.62749146,148.18366333)
\lineto(64.79249146,148.18366333)
\curveto(64.85249,148.1836605)(64.90748995,148.17366051)(64.95749146,148.15366333)
\curveto(65.04748981,148.13366055)(65.11748974,148.09366059)(65.16749146,148.03366333)
\curveto(65.23748962,147.94366074)(65.28248957,147.83366085)(65.30249146,147.70366333)
\curveto(65.33248952,147.5836611)(65.37748948,147.4786612)(65.43749146,147.38866333)
\curveto(65.62748923,147.04866163)(65.88748897,146.7786619)(66.21749146,146.57866333)
\curveto(66.31748854,146.51866216)(66.42248843,146.46866221)(66.53249146,146.42866333)
\curveto(66.6524882,146.39866228)(66.77248808,146.36366232)(66.89249146,146.32366333)
\curveto(67.06248779,146.27366241)(67.26748759,146.25366243)(67.50749146,146.26366333)
\curveto(67.7574871,146.2836624)(67.9574869,146.31866236)(68.10749146,146.36866333)
\curveto(68.47748638,146.48866219)(68.76748609,146.64866203)(68.97749146,146.84866333)
\curveto(69.19748566,147.05866162)(69.37748548,147.33866134)(69.51749146,147.68866333)
\curveto(69.56748529,147.78866089)(69.59748526,147.89366079)(69.60749146,148.00366333)
\curveto(69.62748523,148.11366057)(69.6524852,148.22866045)(69.68249146,148.34866333)
\lineto(69.68249146,148.45366333)
\curveto(69.69248516,148.49366019)(69.69748516,148.53366015)(69.69749146,148.57366333)
\curveto(69.70748515,148.60366008)(69.70748515,148.63866004)(69.69749146,148.67866333)
\lineto(69.69749146,148.79866333)
\curveto(69.69748516,149.05865962)(69.66748519,149.30365938)(69.60749146,149.53366333)
\curveto(69.49748536,149.8836588)(69.34248551,150.1786585)(69.14249146,150.41866333)
\curveto(68.94248591,150.66865801)(68.68248617,150.86365782)(68.36249146,151.00366333)
\lineto(68.18249146,151.06366333)
\curveto(68.13248672,151.0836576)(68.07248678,151.10365758)(68.00249146,151.12366333)
\curveto(67.9524869,151.14365754)(67.89248696,151.15365753)(67.82249146,151.15366333)
\curveto(67.76248709,151.16365752)(67.69748716,151.1786575)(67.62749146,151.19866333)
\lineto(67.47749146,151.19866333)
\curveto(67.43748742,151.21865746)(67.38248747,151.22865745)(67.31249146,151.22866333)
\curveto(67.2524876,151.22865745)(67.19748766,151.21865746)(67.14749146,151.19866333)
\lineto(67.04249146,151.19866333)
\curveto(67.01248784,151.19865748)(66.97748788,151.19365749)(66.93749146,151.18366333)
\lineto(66.69749146,151.12366333)
\curveto(66.61748824,151.11365757)(66.53748832,151.09365759)(66.45749146,151.06366333)
\curveto(66.21748864,150.96365772)(65.98748887,150.82865785)(65.76749146,150.65866333)
\curveto(65.67748918,150.58865809)(65.59248926,150.51365817)(65.51249146,150.43366333)
\curveto(65.43248942,150.36365832)(65.33248952,150.30865837)(65.21249146,150.26866333)
\curveto(65.12248973,150.23865844)(64.98248987,150.22865845)(64.79249146,150.23866333)
\curveto(64.61249024,150.24865843)(64.49249036,150.27365841)(64.43249146,150.31366333)
\curveto(64.38249047,150.35365833)(64.34249051,150.41365827)(64.31249146,150.49366333)
\curveto(64.29249056,150.57365811)(64.29249056,150.65865802)(64.31249146,150.74866333)
\curveto(64.34249051,150.86865781)(64.36249049,150.98865769)(64.37249146,151.10866333)
\curveto(64.39249046,151.23865744)(64.41749044,151.36365732)(64.44749146,151.48366333)
\curveto(64.46749039,151.52365716)(64.47249038,151.55865712)(64.46249146,151.58866333)
\curveto(64.46249039,151.62865705)(64.47249038,151.67365701)(64.49249146,151.72366333)
\curveto(64.51249034,151.81365687)(64.52749033,151.90365678)(64.53749146,151.99366333)
\curveto(64.54749031,152.09365659)(64.56749029,152.18865649)(64.59749146,152.27866333)
\curveto(64.60749025,152.33865634)(64.61249024,152.39865628)(64.61249146,152.45866333)
\curveto(64.62249023,152.51865616)(64.63749022,152.5786561)(64.65749146,152.63866333)
\curveto(64.70749015,152.83865584)(64.74249011,153.04365564)(64.76249146,153.25366333)
\curveto(64.79249006,153.47365521)(64.83249002,153.683655)(64.88249146,153.88366333)
\curveto(64.91248994,153.9836547)(64.93248992,154.0836546)(64.94249146,154.18366333)
\curveto(64.9524899,154.2836544)(64.96748989,154.3836543)(64.98749146,154.48366333)
\curveto(64.99748986,154.51365417)(65.00248985,154.55365413)(65.00249146,154.60366333)
\curveto(65.03248982,154.71365397)(65.0524898,154.81865386)(65.06249146,154.91866333)
\curveto(65.08248977,155.02865365)(65.10748975,155.13865354)(65.13749146,155.24866333)
\curveto(65.1574897,155.32865335)(65.17248968,155.39865328)(65.18249146,155.45866333)
\curveto(65.19248966,155.52865315)(65.21748964,155.58865309)(65.25749146,155.63866333)
\curveto(65.27748958,155.66865301)(65.30748955,155.68865299)(65.34749146,155.69866333)
\curveto(65.38748947,155.71865296)(65.43248942,155.73865294)(65.48249146,155.75866333)
\curveto(65.54248931,155.75865292)(65.58248927,155.76365292)(65.60249146,155.77366333)
}
}
{
\newrgbcolor{curcolor}{0 0 0}
\pscustom[linestyle=none,fillstyle=solid,fillcolor=curcolor]
{
\newpath
\moveto(79.44210083,150.44866333)
\lineto(79.44210083,150.19366333)
\curveto(79.45209313,150.11365857)(79.44709313,150.03865864)(79.42710083,149.96866333)
\lineto(79.42710083,149.72866333)
\lineto(79.42710083,149.56366333)
\curveto(79.40709317,149.46365922)(79.39709318,149.35865932)(79.39710083,149.24866333)
\curveto(79.39709318,149.14865953)(79.38709319,149.04865963)(79.36710083,148.94866333)
\lineto(79.36710083,148.79866333)
\curveto(79.33709324,148.65866002)(79.31709326,148.51866016)(79.30710083,148.37866333)
\curveto(79.29709328,148.24866043)(79.27209331,148.11866056)(79.23210083,147.98866333)
\curveto(79.21209337,147.90866077)(79.19209339,147.82366086)(79.17210083,147.73366333)
\lineto(79.11210083,147.49366333)
\lineto(78.99210083,147.19366333)
\curveto(78.96209362,147.10366158)(78.92709365,147.01366167)(78.88710083,146.92366333)
\curveto(78.78709379,146.70366198)(78.65209393,146.48866219)(78.48210083,146.27866333)
\curveto(78.32209426,146.06866261)(78.14709443,145.89866278)(77.95710083,145.76866333)
\curveto(77.90709467,145.72866295)(77.84709473,145.68866299)(77.77710083,145.64866333)
\curveto(77.71709486,145.61866306)(77.65709492,145.5836631)(77.59710083,145.54366333)
\curveto(77.51709506,145.49366319)(77.42209516,145.45366323)(77.31210083,145.42366333)
\curveto(77.20209538,145.39366329)(77.09709548,145.36366332)(76.99710083,145.33366333)
\curveto(76.88709569,145.29366339)(76.7770958,145.26866341)(76.66710083,145.25866333)
\curveto(76.55709602,145.24866343)(76.44209614,145.23366345)(76.32210083,145.21366333)
\curveto(76.2820963,145.20366348)(76.23709634,145.20366348)(76.18710083,145.21366333)
\curveto(76.14709643,145.21366347)(76.10709647,145.20866347)(76.06710083,145.19866333)
\curveto(76.02709655,145.18866349)(75.97209661,145.1836635)(75.90210083,145.18366333)
\curveto(75.83209675,145.1836635)(75.7820968,145.18866349)(75.75210083,145.19866333)
\curveto(75.70209688,145.21866346)(75.65709692,145.22366346)(75.61710083,145.21366333)
\curveto(75.577097,145.20366348)(75.54209704,145.20366348)(75.51210083,145.21366333)
\lineto(75.42210083,145.21366333)
\curveto(75.36209722,145.23366345)(75.29709728,145.24866343)(75.22710083,145.25866333)
\curveto(75.16709741,145.25866342)(75.10209748,145.26366342)(75.03210083,145.27366333)
\curveto(74.86209772,145.32366336)(74.70209788,145.37366331)(74.55210083,145.42366333)
\curveto(74.40209818,145.47366321)(74.25709832,145.53866314)(74.11710083,145.61866333)
\curveto(74.06709851,145.65866302)(74.01209857,145.68866299)(73.95210083,145.70866333)
\curveto(73.90209868,145.73866294)(73.85209873,145.77366291)(73.80210083,145.81366333)
\curveto(73.56209902,145.99366269)(73.36209922,146.21366247)(73.20210083,146.47366333)
\curveto(73.04209954,146.73366195)(72.90209968,147.01866166)(72.78210083,147.32866333)
\curveto(72.72209986,147.46866121)(72.6770999,147.60866107)(72.64710083,147.74866333)
\curveto(72.61709996,147.89866078)(72.5821,148.05366063)(72.54210083,148.21366333)
\curveto(72.52210006,148.32366036)(72.50710007,148.43366025)(72.49710083,148.54366333)
\curveto(72.48710009,148.65366003)(72.47210011,148.76365992)(72.45210083,148.87366333)
\curveto(72.44210014,148.91365977)(72.43710014,148.95365973)(72.43710083,148.99366333)
\curveto(72.44710013,149.03365965)(72.44710013,149.07365961)(72.43710083,149.11366333)
\curveto(72.42710015,149.16365952)(72.42210016,149.21365947)(72.42210083,149.26366333)
\lineto(72.42210083,149.42866333)
\curveto(72.40210018,149.4786592)(72.39710018,149.52865915)(72.40710083,149.57866333)
\curveto(72.41710016,149.63865904)(72.41710016,149.69365899)(72.40710083,149.74366333)
\curveto(72.39710018,149.7836589)(72.39710018,149.82865885)(72.40710083,149.87866333)
\curveto(72.41710016,149.92865875)(72.41210017,149.9786587)(72.39210083,150.02866333)
\curveto(72.37210021,150.09865858)(72.36710021,150.17365851)(72.37710083,150.25366333)
\curveto(72.38710019,150.34365834)(72.39210019,150.42865825)(72.39210083,150.50866333)
\curveto(72.39210019,150.59865808)(72.38710019,150.69865798)(72.37710083,150.80866333)
\curveto(72.36710021,150.92865775)(72.37210021,151.02865765)(72.39210083,151.10866333)
\lineto(72.39210083,151.39366333)
\lineto(72.43710083,152.02366333)
\curveto(72.44710013,152.12365656)(72.45710012,152.21865646)(72.46710083,152.30866333)
\lineto(72.49710083,152.60866333)
\curveto(72.51710006,152.65865602)(72.52210006,152.70865597)(72.51210083,152.75866333)
\curveto(72.51210007,152.81865586)(72.52210006,152.87365581)(72.54210083,152.92366333)
\curveto(72.59209999,153.09365559)(72.63209995,153.25865542)(72.66210083,153.41866333)
\curveto(72.69209989,153.58865509)(72.74209984,153.74865493)(72.81210083,153.89866333)
\curveto(73.00209958,154.35865432)(73.22209936,154.73365395)(73.47210083,155.02366333)
\curveto(73.73209885,155.31365337)(74.09209849,155.55865312)(74.55210083,155.75866333)
\curveto(74.6820979,155.80865287)(74.81209777,155.84365284)(74.94210083,155.86366333)
\curveto(75.0820975,155.8836528)(75.22209736,155.90865277)(75.36210083,155.93866333)
\curveto(75.43209715,155.94865273)(75.49709708,155.95365273)(75.55710083,155.95366333)
\curveto(75.61709696,155.95365273)(75.6820969,155.95865272)(75.75210083,155.96866333)
\curveto(76.582096,155.98865269)(77.25209533,155.83865284)(77.76210083,155.51866333)
\curveto(78.27209431,155.20865347)(78.65209393,154.76865391)(78.90210083,154.19866333)
\curveto(78.95209363,154.0786546)(78.99709358,153.95365473)(79.03710083,153.82366333)
\curveto(79.0770935,153.69365499)(79.12209346,153.55865512)(79.17210083,153.41866333)
\curveto(79.19209339,153.33865534)(79.20709337,153.25365543)(79.21710083,153.16366333)
\lineto(79.27710083,152.92366333)
\curveto(79.30709327,152.81365587)(79.32209326,152.70365598)(79.32210083,152.59366333)
\curveto(79.33209325,152.4836562)(79.34709323,152.37365631)(79.36710083,152.26366333)
\curveto(79.38709319,152.21365647)(79.39209319,152.16865651)(79.38210083,152.12866333)
\curveto(79.3820932,152.08865659)(79.38709319,152.04865663)(79.39710083,152.00866333)
\curveto(79.40709317,151.95865672)(79.40709317,151.90365678)(79.39710083,151.84366333)
\curveto(79.39709318,151.79365689)(79.40209318,151.74365694)(79.41210083,151.69366333)
\lineto(79.41210083,151.55866333)
\curveto(79.43209315,151.49865718)(79.43209315,151.42865725)(79.41210083,151.34866333)
\curveto(79.40209318,151.2786574)(79.40709317,151.21365747)(79.42710083,151.15366333)
\curveto(79.43709314,151.12365756)(79.44209314,151.0836576)(79.44210083,151.03366333)
\lineto(79.44210083,150.91366333)
\lineto(79.44210083,150.44866333)
\moveto(77.89710083,148.12366333)
\curveto(77.99709458,148.44366024)(78.05709452,148.80865987)(78.07710083,149.21866333)
\curveto(78.09709448,149.62865905)(78.10709447,150.03865864)(78.10710083,150.44866333)
\curveto(78.10709447,150.8786578)(78.09709448,151.29865738)(78.07710083,151.70866333)
\curveto(78.05709452,152.11865656)(78.01209457,152.50365618)(77.94210083,152.86366333)
\curveto(77.87209471,153.22365546)(77.76209482,153.54365514)(77.61210083,153.82366333)
\curveto(77.47209511,154.11365457)(77.2770953,154.34865433)(77.02710083,154.52866333)
\curveto(76.86709571,154.63865404)(76.68709589,154.71865396)(76.48710083,154.76866333)
\curveto(76.28709629,154.82865385)(76.04209654,154.85865382)(75.75210083,154.85866333)
\curveto(75.73209685,154.83865384)(75.69709688,154.82865385)(75.64710083,154.82866333)
\curveto(75.59709698,154.83865384)(75.55709702,154.83865384)(75.52710083,154.82866333)
\curveto(75.44709713,154.80865387)(75.37209721,154.78865389)(75.30210083,154.76866333)
\curveto(75.24209734,154.75865392)(75.1770974,154.73865394)(75.10710083,154.70866333)
\curveto(74.83709774,154.58865409)(74.61709796,154.41865426)(74.44710083,154.19866333)
\curveto(74.28709829,153.98865469)(74.15209843,153.74365494)(74.04210083,153.46366333)
\curveto(73.99209859,153.35365533)(73.95209863,153.23365545)(73.92210083,153.10366333)
\curveto(73.90209868,152.9836557)(73.8770987,152.85865582)(73.84710083,152.72866333)
\curveto(73.82709875,152.678656)(73.81709876,152.62365606)(73.81710083,152.56366333)
\curveto(73.81709876,152.51365617)(73.81209877,152.46365622)(73.80210083,152.41366333)
\curveto(73.79209879,152.32365636)(73.7820988,152.22865645)(73.77210083,152.12866333)
\curveto(73.76209882,152.03865664)(73.75209883,151.94365674)(73.74210083,151.84366333)
\curveto(73.74209884,151.76365692)(73.73709884,151.678657)(73.72710083,151.58866333)
\lineto(73.72710083,151.34866333)
\lineto(73.72710083,151.16866333)
\curveto(73.71709886,151.13865754)(73.71209887,151.10365758)(73.71210083,151.06366333)
\lineto(73.71210083,150.92866333)
\lineto(73.71210083,150.47866333)
\curveto(73.71209887,150.39865828)(73.70709887,150.31365837)(73.69710083,150.22366333)
\curveto(73.69709888,150.14365854)(73.70709887,150.06865861)(73.72710083,149.99866333)
\lineto(73.72710083,149.72866333)
\curveto(73.72709885,149.70865897)(73.72209886,149.678659)(73.71210083,149.63866333)
\curveto(73.71209887,149.60865907)(73.71709886,149.5836591)(73.72710083,149.56366333)
\curveto(73.73709884,149.46365922)(73.74209884,149.36365932)(73.74210083,149.26366333)
\curveto(73.75209883,149.17365951)(73.76209882,149.07365961)(73.77210083,148.96366333)
\curveto(73.80209878,148.84365984)(73.81709876,148.71865996)(73.81710083,148.58866333)
\curveto(73.82709875,148.46866021)(73.85209873,148.35366033)(73.89210083,148.24366333)
\curveto(73.97209861,147.94366074)(74.05709852,147.678661)(74.14710083,147.44866333)
\curveto(74.24709833,147.21866146)(74.39209819,147.00366168)(74.58210083,146.80366333)
\curveto(74.79209779,146.60366208)(75.05709752,146.45366223)(75.37710083,146.35366333)
\curveto(75.41709716,146.33366235)(75.45209713,146.32366236)(75.48210083,146.32366333)
\curveto(75.52209706,146.33366235)(75.56709701,146.32866235)(75.61710083,146.30866333)
\curveto(75.65709692,146.29866238)(75.72709685,146.28866239)(75.82710083,146.27866333)
\curveto(75.93709664,146.26866241)(76.02209656,146.27366241)(76.08210083,146.29366333)
\curveto(76.15209643,146.31366237)(76.22209636,146.32366236)(76.29210083,146.32366333)
\curveto(76.36209622,146.33366235)(76.42709615,146.34866233)(76.48710083,146.36866333)
\curveto(76.68709589,146.42866225)(76.86709571,146.51366217)(77.02710083,146.62366333)
\curveto(77.05709552,146.64366204)(77.0820955,146.66366202)(77.10210083,146.68366333)
\lineto(77.16210083,146.74366333)
\curveto(77.20209538,146.76366192)(77.25209533,146.80366188)(77.31210083,146.86366333)
\curveto(77.41209517,147.00366168)(77.49709508,147.13366155)(77.56710083,147.25366333)
\curveto(77.63709494,147.37366131)(77.70709487,147.51866116)(77.77710083,147.68866333)
\curveto(77.80709477,147.75866092)(77.82709475,147.82866085)(77.83710083,147.89866333)
\curveto(77.85709472,147.96866071)(77.8770947,148.04366064)(77.89710083,148.12366333)
}
}
{
\newrgbcolor{curcolor}{0 0 0}
\pscustom[linestyle=none,fillstyle=solid,fillcolor=curcolor]
{
\newpath
\moveto(57.25288208,229.70223694)
\lineto(60.85288208,229.70223694)
\lineto(61.49788208,229.70223694)
\curveto(61.57787555,229.70222651)(61.65287548,229.69722652)(61.72288208,229.68723694)
\curveto(61.79287534,229.68722653)(61.85287528,229.67722654)(61.90288208,229.65723694)
\curveto(61.97287516,229.62722659)(62.0278751,229.56722665)(62.06788208,229.47723694)
\curveto(62.08787504,229.44722677)(62.09787503,229.40722681)(62.09788208,229.35723694)
\lineto(62.09788208,229.22223694)
\curveto(62.10787502,229.1122271)(62.10287503,229.00722721)(62.08288208,228.90723694)
\curveto(62.07287506,228.80722741)(62.03787509,228.73722748)(61.97788208,228.69723694)
\curveto(61.88787524,228.62722759)(61.75287538,228.59222762)(61.57288208,228.59223694)
\curveto(61.39287574,228.60222761)(61.2278759,228.60722761)(61.07788208,228.60723694)
\lineto(59.08288208,228.60723694)
\lineto(58.58788208,228.60723694)
\lineto(58.45288208,228.60723694)
\curveto(58.41287872,228.60722761)(58.37287876,228.60222761)(58.33288208,228.59223694)
\lineto(58.12288208,228.59223694)
\curveto(58.01287912,228.56222765)(57.9328792,228.52222769)(57.88288208,228.47223694)
\curveto(57.8328793,228.43222778)(57.79787933,228.37722784)(57.77788208,228.30723694)
\curveto(57.75787937,228.24722797)(57.74287939,228.17722804)(57.73288208,228.09723694)
\curveto(57.72287941,228.0172282)(57.70287943,227.92722829)(57.67288208,227.82723694)
\curveto(57.62287951,227.62722859)(57.58287955,227.42222879)(57.55288208,227.21223694)
\curveto(57.52287961,227.00222921)(57.48287965,226.79722942)(57.43288208,226.59723694)
\curveto(57.41287972,226.52722969)(57.40287973,226.45722976)(57.40288208,226.38723694)
\curveto(57.40287973,226.32722989)(57.39287974,226.26222995)(57.37288208,226.19223694)
\curveto(57.36287977,226.16223005)(57.35287978,226.12223009)(57.34288208,226.07223694)
\curveto(57.34287979,226.03223018)(57.34787978,225.99223022)(57.35788208,225.95223694)
\curveto(57.37787975,225.90223031)(57.40287973,225.85723036)(57.43288208,225.81723694)
\curveto(57.47287966,225.78723043)(57.5328796,225.78223043)(57.61288208,225.80223694)
\curveto(57.67287946,225.82223039)(57.7328794,225.84723037)(57.79288208,225.87723694)
\curveto(57.85287928,225.9172303)(57.91287922,225.95223026)(57.97288208,225.98223694)
\curveto(58.0328791,226.00223021)(58.08287905,226.0172302)(58.12288208,226.02723694)
\curveto(58.31287882,226.10723011)(58.51787861,226.16223005)(58.73788208,226.19223694)
\curveto(58.96787816,226.22222999)(59.19787793,226.23222998)(59.42788208,226.22223694)
\curveto(59.66787746,226.22222999)(59.89787723,226.19723002)(60.11788208,226.14723694)
\curveto(60.33787679,226.10723011)(60.53787659,226.04723017)(60.71788208,225.96723694)
\curveto(60.76787636,225.94723027)(60.81287632,225.92723029)(60.85288208,225.90723694)
\curveto(60.90287623,225.88723033)(60.95287618,225.86223035)(61.00288208,225.83223694)
\curveto(61.35287578,225.62223059)(61.6328755,225.39223082)(61.84288208,225.14223694)
\curveto(62.06287507,224.89223132)(62.25787487,224.56723165)(62.42788208,224.16723694)
\curveto(62.47787465,224.05723216)(62.51287462,223.94723227)(62.53288208,223.83723694)
\curveto(62.55287458,223.72723249)(62.57787455,223.6122326)(62.60788208,223.49223694)
\curveto(62.61787451,223.46223275)(62.62287451,223.4172328)(62.62288208,223.35723694)
\curveto(62.64287449,223.29723292)(62.65287448,223.22723299)(62.65288208,223.14723694)
\curveto(62.65287448,223.07723314)(62.66287447,223.0122332)(62.68288208,222.95223694)
\lineto(62.68288208,222.78723694)
\curveto(62.69287444,222.73723348)(62.69787443,222.66723355)(62.69788208,222.57723694)
\curveto(62.69787443,222.48723373)(62.68787444,222.4172338)(62.66788208,222.36723694)
\curveto(62.64787448,222.30723391)(62.64287449,222.24723397)(62.65288208,222.18723694)
\curveto(62.66287447,222.13723408)(62.65787447,222.08723413)(62.63788208,222.03723694)
\curveto(62.59787453,221.87723434)(62.56287457,221.72723449)(62.53288208,221.58723694)
\curveto(62.50287463,221.44723477)(62.45787467,221.3122349)(62.39788208,221.18223694)
\curveto(62.23787489,220.8122354)(62.01787511,220.47723574)(61.73788208,220.17723694)
\curveto(61.45787567,219.87723634)(61.13787599,219.64723657)(60.77788208,219.48723694)
\curveto(60.60787652,219.40723681)(60.40787672,219.33223688)(60.17788208,219.26223694)
\curveto(60.06787706,219.22223699)(59.95287718,219.19723702)(59.83288208,219.18723694)
\curveto(59.71287742,219.17723704)(59.59287754,219.15723706)(59.47288208,219.12723694)
\curveto(59.42287771,219.10723711)(59.36787776,219.10723711)(59.30788208,219.12723694)
\curveto(59.24787788,219.13723708)(59.18787794,219.13223708)(59.12788208,219.11223694)
\curveto(59.0278781,219.09223712)(58.9278782,219.09223712)(58.82788208,219.11223694)
\lineto(58.69288208,219.11223694)
\curveto(58.64287849,219.13223708)(58.58287855,219.14223707)(58.51288208,219.14223694)
\curveto(58.45287868,219.13223708)(58.39787873,219.13723708)(58.34788208,219.15723694)
\curveto(58.30787882,219.16723705)(58.27287886,219.17223704)(58.24288208,219.17223694)
\curveto(58.21287892,219.17223704)(58.17787895,219.17723704)(58.13788208,219.18723694)
\lineto(57.86788208,219.24723694)
\curveto(57.77787935,219.26723695)(57.69287944,219.29723692)(57.61288208,219.33723694)
\curveto(57.27287986,219.47723674)(56.98288015,219.63223658)(56.74288208,219.80223694)
\curveto(56.50288063,219.98223623)(56.28288085,220.212236)(56.08288208,220.49223694)
\curveto(55.9328812,220.72223549)(55.81788131,220.96223525)(55.73788208,221.21223694)
\curveto(55.71788141,221.26223495)(55.70788142,221.30723491)(55.70788208,221.34723694)
\curveto(55.70788142,221.39723482)(55.69788143,221.44723477)(55.67788208,221.49723694)
\curveto(55.65788147,221.55723466)(55.64288149,221.63723458)(55.63288208,221.73723694)
\curveto(55.6328815,221.83723438)(55.65288148,221.9122343)(55.69288208,221.96223694)
\curveto(55.74288139,222.04223417)(55.82288131,222.08723413)(55.93288208,222.09723694)
\curveto(56.04288109,222.10723411)(56.15788097,222.1122341)(56.27788208,222.11223694)
\lineto(56.44288208,222.11223694)
\curveto(56.50288063,222.1122341)(56.55788057,222.10223411)(56.60788208,222.08223694)
\curveto(56.69788043,222.06223415)(56.76788036,222.02223419)(56.81788208,221.96223694)
\curveto(56.88788024,221.87223434)(56.9328802,221.76223445)(56.95288208,221.63223694)
\curveto(56.98288015,221.5122347)(57.0278801,221.40723481)(57.08788208,221.31723694)
\curveto(57.27787985,220.97723524)(57.53787959,220.70723551)(57.86788208,220.50723694)
\curveto(57.96787916,220.44723577)(58.07287906,220.39723582)(58.18288208,220.35723694)
\curveto(58.30287883,220.32723589)(58.42287871,220.29223592)(58.54288208,220.25223694)
\curveto(58.71287842,220.20223601)(58.91787821,220.18223603)(59.15788208,220.19223694)
\curveto(59.40787772,220.212236)(59.60787752,220.24723597)(59.75788208,220.29723694)
\curveto(60.127877,220.4172358)(60.41787671,220.57723564)(60.62788208,220.77723694)
\curveto(60.84787628,220.98723523)(61.0278761,221.26723495)(61.16788208,221.61723694)
\curveto(61.21787591,221.7172345)(61.24787588,221.82223439)(61.25788208,221.93223694)
\curveto(61.27787585,222.04223417)(61.30287583,222.15723406)(61.33288208,222.27723694)
\lineto(61.33288208,222.38223694)
\curveto(61.34287579,222.42223379)(61.34787578,222.46223375)(61.34788208,222.50223694)
\curveto(61.35787577,222.53223368)(61.35787577,222.56723365)(61.34788208,222.60723694)
\lineto(61.34788208,222.72723694)
\curveto(61.34787578,222.98723323)(61.31787581,223.23223298)(61.25788208,223.46223694)
\curveto(61.14787598,223.8122324)(60.99287614,224.10723211)(60.79288208,224.34723694)
\curveto(60.59287654,224.59723162)(60.3328768,224.79223142)(60.01288208,224.93223694)
\lineto(59.83288208,224.99223694)
\curveto(59.78287735,225.0122312)(59.72287741,225.03223118)(59.65288208,225.05223694)
\curveto(59.60287753,225.07223114)(59.54287759,225.08223113)(59.47288208,225.08223694)
\curveto(59.41287772,225.09223112)(59.34787778,225.10723111)(59.27788208,225.12723694)
\lineto(59.12788208,225.12723694)
\curveto(59.08787804,225.14723107)(59.0328781,225.15723106)(58.96288208,225.15723694)
\curveto(58.90287823,225.15723106)(58.84787828,225.14723107)(58.79788208,225.12723694)
\lineto(58.69288208,225.12723694)
\curveto(58.66287847,225.12723109)(58.6278785,225.12223109)(58.58788208,225.11223694)
\lineto(58.34788208,225.05223694)
\curveto(58.26787886,225.04223117)(58.18787894,225.02223119)(58.10788208,224.99223694)
\curveto(57.86787926,224.89223132)(57.63787949,224.75723146)(57.41788208,224.58723694)
\curveto(57.3278798,224.5172317)(57.24287989,224.44223177)(57.16288208,224.36223694)
\curveto(57.08288005,224.29223192)(56.98288015,224.23723198)(56.86288208,224.19723694)
\curveto(56.77288036,224.16723205)(56.6328805,224.15723206)(56.44288208,224.16723694)
\curveto(56.26288087,224.17723204)(56.14288099,224.20223201)(56.08288208,224.24223694)
\curveto(56.0328811,224.28223193)(55.99288114,224.34223187)(55.96288208,224.42223694)
\curveto(55.94288119,224.50223171)(55.94288119,224.58723163)(55.96288208,224.67723694)
\curveto(55.99288114,224.79723142)(56.01288112,224.9172313)(56.02288208,225.03723694)
\curveto(56.04288109,225.16723105)(56.06788106,225.29223092)(56.09788208,225.41223694)
\curveto(56.11788101,225.45223076)(56.12288101,225.48723073)(56.11288208,225.51723694)
\curveto(56.11288102,225.55723066)(56.12288101,225.60223061)(56.14288208,225.65223694)
\curveto(56.16288097,225.74223047)(56.17788095,225.83223038)(56.18788208,225.92223694)
\curveto(56.19788093,226.02223019)(56.21788091,226.1172301)(56.24788208,226.20723694)
\curveto(56.25788087,226.26722995)(56.26288087,226.32722989)(56.26288208,226.38723694)
\curveto(56.27288086,226.44722977)(56.28788084,226.50722971)(56.30788208,226.56723694)
\curveto(56.35788077,226.76722945)(56.39288074,226.97222924)(56.41288208,227.18223694)
\curveto(56.44288069,227.40222881)(56.48288065,227.6122286)(56.53288208,227.81223694)
\curveto(56.56288057,227.9122283)(56.58288055,228.0122282)(56.59288208,228.11223694)
\curveto(56.60288053,228.212228)(56.61788051,228.3122279)(56.63788208,228.41223694)
\curveto(56.64788048,228.44222777)(56.65288048,228.48222773)(56.65288208,228.53223694)
\curveto(56.68288045,228.64222757)(56.70288043,228.74722747)(56.71288208,228.84723694)
\curveto(56.7328804,228.95722726)(56.75788037,229.06722715)(56.78788208,229.17723694)
\curveto(56.80788032,229.25722696)(56.82288031,229.32722689)(56.83288208,229.38723694)
\curveto(56.84288029,229.45722676)(56.86788026,229.5172267)(56.90788208,229.56723694)
\curveto(56.9278802,229.59722662)(56.95788017,229.6172266)(56.99788208,229.62723694)
\curveto(57.03788009,229.64722657)(57.08288005,229.66722655)(57.13288208,229.68723694)
\curveto(57.19287994,229.68722653)(57.2328799,229.69222652)(57.25288208,229.70223694)
}
}
{
\newrgbcolor{curcolor}{0 0 0}
\pscustom[linestyle=none,fillstyle=solid,fillcolor=curcolor]
{
\newpath
\moveto(71.09249146,224.37723694)
\lineto(71.09249146,224.12223694)
\curveto(71.10248375,224.04223217)(71.09748376,223.96723225)(71.07749146,223.89723694)
\lineto(71.07749146,223.65723694)
\lineto(71.07749146,223.49223694)
\curveto(71.0574838,223.39223282)(71.04748381,223.28723293)(71.04749146,223.17723694)
\curveto(71.04748381,223.07723314)(71.03748382,222.97723324)(71.01749146,222.87723694)
\lineto(71.01749146,222.72723694)
\curveto(70.98748387,222.58723363)(70.96748389,222.44723377)(70.95749146,222.30723694)
\curveto(70.94748391,222.17723404)(70.92248393,222.04723417)(70.88249146,221.91723694)
\curveto(70.86248399,221.83723438)(70.84248401,221.75223446)(70.82249146,221.66223694)
\lineto(70.76249146,221.42223694)
\lineto(70.64249146,221.12223694)
\curveto(70.61248424,221.03223518)(70.57748428,220.94223527)(70.53749146,220.85223694)
\curveto(70.43748442,220.63223558)(70.30248455,220.4172358)(70.13249146,220.20723694)
\curveto(69.97248488,219.99723622)(69.79748506,219.82723639)(69.60749146,219.69723694)
\curveto(69.5574853,219.65723656)(69.49748536,219.6172366)(69.42749146,219.57723694)
\curveto(69.36748549,219.54723667)(69.30748555,219.5122367)(69.24749146,219.47223694)
\curveto(69.16748569,219.42223679)(69.07248578,219.38223683)(68.96249146,219.35223694)
\curveto(68.852486,219.32223689)(68.74748611,219.29223692)(68.64749146,219.26223694)
\curveto(68.53748632,219.22223699)(68.42748643,219.19723702)(68.31749146,219.18723694)
\curveto(68.20748665,219.17723704)(68.09248676,219.16223705)(67.97249146,219.14223694)
\curveto(67.93248692,219.13223708)(67.88748697,219.13223708)(67.83749146,219.14223694)
\curveto(67.79748706,219.14223707)(67.7574871,219.13723708)(67.71749146,219.12723694)
\curveto(67.67748718,219.1172371)(67.62248723,219.1122371)(67.55249146,219.11223694)
\curveto(67.48248737,219.1122371)(67.43248742,219.1172371)(67.40249146,219.12723694)
\curveto(67.3524875,219.14723707)(67.30748755,219.15223706)(67.26749146,219.14223694)
\curveto(67.22748763,219.13223708)(67.19248766,219.13223708)(67.16249146,219.14223694)
\lineto(67.07249146,219.14223694)
\curveto(67.01248784,219.16223705)(66.94748791,219.17723704)(66.87749146,219.18723694)
\curveto(66.81748804,219.18723703)(66.7524881,219.19223702)(66.68249146,219.20223694)
\curveto(66.51248834,219.25223696)(66.3524885,219.30223691)(66.20249146,219.35223694)
\curveto(66.0524888,219.40223681)(65.90748895,219.46723675)(65.76749146,219.54723694)
\curveto(65.71748914,219.58723663)(65.66248919,219.6172366)(65.60249146,219.63723694)
\curveto(65.5524893,219.66723655)(65.50248935,219.70223651)(65.45249146,219.74223694)
\curveto(65.21248964,219.92223629)(65.01248984,220.14223607)(64.85249146,220.40223694)
\curveto(64.69249016,220.66223555)(64.5524903,220.94723527)(64.43249146,221.25723694)
\curveto(64.37249048,221.39723482)(64.32749053,221.53723468)(64.29749146,221.67723694)
\curveto(64.26749059,221.82723439)(64.23249062,221.98223423)(64.19249146,222.14223694)
\curveto(64.17249068,222.25223396)(64.1574907,222.36223385)(64.14749146,222.47223694)
\curveto(64.13749072,222.58223363)(64.12249073,222.69223352)(64.10249146,222.80223694)
\curveto(64.09249076,222.84223337)(64.08749077,222.88223333)(64.08749146,222.92223694)
\curveto(64.09749076,222.96223325)(64.09749076,223.00223321)(64.08749146,223.04223694)
\curveto(64.07749078,223.09223312)(64.07249078,223.14223307)(64.07249146,223.19223694)
\lineto(64.07249146,223.35723694)
\curveto(64.0524908,223.40723281)(64.04749081,223.45723276)(64.05749146,223.50723694)
\curveto(64.06749079,223.56723265)(64.06749079,223.62223259)(64.05749146,223.67223694)
\curveto(64.04749081,223.7122325)(64.04749081,223.75723246)(64.05749146,223.80723694)
\curveto(64.06749079,223.85723236)(64.06249079,223.90723231)(64.04249146,223.95723694)
\curveto(64.02249083,224.02723219)(64.01749084,224.10223211)(64.02749146,224.18223694)
\curveto(64.03749082,224.27223194)(64.04249081,224.35723186)(64.04249146,224.43723694)
\curveto(64.04249081,224.52723169)(64.03749082,224.62723159)(64.02749146,224.73723694)
\curveto(64.01749084,224.85723136)(64.02249083,224.95723126)(64.04249146,225.03723694)
\lineto(64.04249146,225.32223694)
\lineto(64.08749146,225.95223694)
\curveto(64.09749076,226.05223016)(64.10749075,226.14723007)(64.11749146,226.23723694)
\lineto(64.14749146,226.53723694)
\curveto(64.16749069,226.58722963)(64.17249068,226.63722958)(64.16249146,226.68723694)
\curveto(64.16249069,226.74722947)(64.17249068,226.80222941)(64.19249146,226.85223694)
\curveto(64.24249061,227.02222919)(64.28249057,227.18722903)(64.31249146,227.34723694)
\curveto(64.34249051,227.5172287)(64.39249046,227.67722854)(64.46249146,227.82723694)
\curveto(64.6524902,228.28722793)(64.87248998,228.66222755)(65.12249146,228.95223694)
\curveto(65.38248947,229.24222697)(65.74248911,229.48722673)(66.20249146,229.68723694)
\curveto(66.33248852,229.73722648)(66.46248839,229.77222644)(66.59249146,229.79223694)
\curveto(66.73248812,229.8122264)(66.87248798,229.83722638)(67.01249146,229.86723694)
\curveto(67.08248777,229.87722634)(67.14748771,229.88222633)(67.20749146,229.88223694)
\curveto(67.26748759,229.88222633)(67.33248752,229.88722633)(67.40249146,229.89723694)
\curveto(68.23248662,229.9172263)(68.90248595,229.76722645)(69.41249146,229.44723694)
\curveto(69.92248493,229.13722708)(70.30248455,228.69722752)(70.55249146,228.12723694)
\curveto(70.60248425,228.00722821)(70.64748421,227.88222833)(70.68749146,227.75223694)
\curveto(70.72748413,227.62222859)(70.77248408,227.48722873)(70.82249146,227.34723694)
\curveto(70.84248401,227.26722895)(70.857484,227.18222903)(70.86749146,227.09223694)
\lineto(70.92749146,226.85223694)
\curveto(70.9574839,226.74222947)(70.97248388,226.63222958)(70.97249146,226.52223694)
\curveto(70.98248387,226.4122298)(70.99748386,226.30222991)(71.01749146,226.19223694)
\curveto(71.03748382,226.14223007)(71.04248381,226.09723012)(71.03249146,226.05723694)
\curveto(71.03248382,226.0172302)(71.03748382,225.97723024)(71.04749146,225.93723694)
\curveto(71.0574838,225.88723033)(71.0574838,225.83223038)(71.04749146,225.77223694)
\curveto(71.04748381,225.72223049)(71.0524838,225.67223054)(71.06249146,225.62223694)
\lineto(71.06249146,225.48723694)
\curveto(71.08248377,225.42723079)(71.08248377,225.35723086)(71.06249146,225.27723694)
\curveto(71.0524838,225.20723101)(71.0574838,225.14223107)(71.07749146,225.08223694)
\curveto(71.08748377,225.05223116)(71.09248376,225.0122312)(71.09249146,224.96223694)
\lineto(71.09249146,224.84223694)
\lineto(71.09249146,224.37723694)
\moveto(69.54749146,222.05223694)
\curveto(69.64748521,222.37223384)(69.70748515,222.73723348)(69.72749146,223.14723694)
\curveto(69.74748511,223.55723266)(69.7574851,223.96723225)(69.75749146,224.37723694)
\curveto(69.7574851,224.80723141)(69.74748511,225.22723099)(69.72749146,225.63723694)
\curveto(69.70748515,226.04723017)(69.66248519,226.43222978)(69.59249146,226.79223694)
\curveto(69.52248533,227.15222906)(69.41248544,227.47222874)(69.26249146,227.75223694)
\curveto(69.12248573,228.04222817)(68.92748593,228.27722794)(68.67749146,228.45723694)
\curveto(68.51748634,228.56722765)(68.33748652,228.64722757)(68.13749146,228.69723694)
\curveto(67.93748692,228.75722746)(67.69248716,228.78722743)(67.40249146,228.78723694)
\curveto(67.38248747,228.76722745)(67.34748751,228.75722746)(67.29749146,228.75723694)
\curveto(67.24748761,228.76722745)(67.20748765,228.76722745)(67.17749146,228.75723694)
\curveto(67.09748776,228.73722748)(67.02248783,228.7172275)(66.95249146,228.69723694)
\curveto(66.89248796,228.68722753)(66.82748803,228.66722755)(66.75749146,228.63723694)
\curveto(66.48748837,228.5172277)(66.26748859,228.34722787)(66.09749146,228.12723694)
\curveto(65.93748892,227.9172283)(65.80248905,227.67222854)(65.69249146,227.39223694)
\curveto(65.64248921,227.28222893)(65.60248925,227.16222905)(65.57249146,227.03223694)
\curveto(65.5524893,226.9122293)(65.52748933,226.78722943)(65.49749146,226.65723694)
\curveto(65.47748938,226.60722961)(65.46748939,226.55222966)(65.46749146,226.49223694)
\curveto(65.46748939,226.44222977)(65.46248939,226.39222982)(65.45249146,226.34223694)
\curveto(65.44248941,226.25222996)(65.43248942,226.15723006)(65.42249146,226.05723694)
\curveto(65.41248944,225.96723025)(65.40248945,225.87223034)(65.39249146,225.77223694)
\curveto(65.39248946,225.69223052)(65.38748947,225.60723061)(65.37749146,225.51723694)
\lineto(65.37749146,225.27723694)
\lineto(65.37749146,225.09723694)
\curveto(65.36748949,225.06723115)(65.36248949,225.03223118)(65.36249146,224.99223694)
\lineto(65.36249146,224.85723694)
\lineto(65.36249146,224.40723694)
\curveto(65.36248949,224.32723189)(65.3574895,224.24223197)(65.34749146,224.15223694)
\curveto(65.34748951,224.07223214)(65.3574895,223.99723222)(65.37749146,223.92723694)
\lineto(65.37749146,223.65723694)
\curveto(65.37748948,223.63723258)(65.37248948,223.60723261)(65.36249146,223.56723694)
\curveto(65.36248949,223.53723268)(65.36748949,223.5122327)(65.37749146,223.49223694)
\curveto(65.38748947,223.39223282)(65.39248946,223.29223292)(65.39249146,223.19223694)
\curveto(65.40248945,223.10223311)(65.41248944,223.00223321)(65.42249146,222.89223694)
\curveto(65.4524894,222.77223344)(65.46748939,222.64723357)(65.46749146,222.51723694)
\curveto(65.47748938,222.39723382)(65.50248935,222.28223393)(65.54249146,222.17223694)
\curveto(65.62248923,221.87223434)(65.70748915,221.60723461)(65.79749146,221.37723694)
\curveto(65.89748896,221.14723507)(66.04248881,220.93223528)(66.23249146,220.73223694)
\curveto(66.44248841,220.53223568)(66.70748815,220.38223583)(67.02749146,220.28223694)
\curveto(67.06748779,220.26223595)(67.10248775,220.25223596)(67.13249146,220.25223694)
\curveto(67.17248768,220.26223595)(67.21748764,220.25723596)(67.26749146,220.23723694)
\curveto(67.30748755,220.22723599)(67.37748748,220.217236)(67.47749146,220.20723694)
\curveto(67.58748727,220.19723602)(67.67248718,220.20223601)(67.73249146,220.22223694)
\curveto(67.80248705,220.24223597)(67.87248698,220.25223596)(67.94249146,220.25223694)
\curveto(68.01248684,220.26223595)(68.07748678,220.27723594)(68.13749146,220.29723694)
\curveto(68.33748652,220.35723586)(68.51748634,220.44223577)(68.67749146,220.55223694)
\curveto(68.70748615,220.57223564)(68.73248612,220.59223562)(68.75249146,220.61223694)
\lineto(68.81249146,220.67223694)
\curveto(68.852486,220.69223552)(68.90248595,220.73223548)(68.96249146,220.79223694)
\curveto(69.06248579,220.93223528)(69.14748571,221.06223515)(69.21749146,221.18223694)
\curveto(69.28748557,221.30223491)(69.3574855,221.44723477)(69.42749146,221.61723694)
\curveto(69.4574854,221.68723453)(69.47748538,221.75723446)(69.48749146,221.82723694)
\curveto(69.50748535,221.89723432)(69.52748533,221.97223424)(69.54749146,222.05223694)
}
}
{
\newrgbcolor{curcolor}{0 0 0}
\pscustom[linestyle=none,fillstyle=solid,fillcolor=curcolor]
{
\newpath
\moveto(79.44210083,224.37723694)
\lineto(79.44210083,224.12223694)
\curveto(79.45209313,224.04223217)(79.44709313,223.96723225)(79.42710083,223.89723694)
\lineto(79.42710083,223.65723694)
\lineto(79.42710083,223.49223694)
\curveto(79.40709317,223.39223282)(79.39709318,223.28723293)(79.39710083,223.17723694)
\curveto(79.39709318,223.07723314)(79.38709319,222.97723324)(79.36710083,222.87723694)
\lineto(79.36710083,222.72723694)
\curveto(79.33709324,222.58723363)(79.31709326,222.44723377)(79.30710083,222.30723694)
\curveto(79.29709328,222.17723404)(79.27209331,222.04723417)(79.23210083,221.91723694)
\curveto(79.21209337,221.83723438)(79.19209339,221.75223446)(79.17210083,221.66223694)
\lineto(79.11210083,221.42223694)
\lineto(78.99210083,221.12223694)
\curveto(78.96209362,221.03223518)(78.92709365,220.94223527)(78.88710083,220.85223694)
\curveto(78.78709379,220.63223558)(78.65209393,220.4172358)(78.48210083,220.20723694)
\curveto(78.32209426,219.99723622)(78.14709443,219.82723639)(77.95710083,219.69723694)
\curveto(77.90709467,219.65723656)(77.84709473,219.6172366)(77.77710083,219.57723694)
\curveto(77.71709486,219.54723667)(77.65709492,219.5122367)(77.59710083,219.47223694)
\curveto(77.51709506,219.42223679)(77.42209516,219.38223683)(77.31210083,219.35223694)
\curveto(77.20209538,219.32223689)(77.09709548,219.29223692)(76.99710083,219.26223694)
\curveto(76.88709569,219.22223699)(76.7770958,219.19723702)(76.66710083,219.18723694)
\curveto(76.55709602,219.17723704)(76.44209614,219.16223705)(76.32210083,219.14223694)
\curveto(76.2820963,219.13223708)(76.23709634,219.13223708)(76.18710083,219.14223694)
\curveto(76.14709643,219.14223707)(76.10709647,219.13723708)(76.06710083,219.12723694)
\curveto(76.02709655,219.1172371)(75.97209661,219.1122371)(75.90210083,219.11223694)
\curveto(75.83209675,219.1122371)(75.7820968,219.1172371)(75.75210083,219.12723694)
\curveto(75.70209688,219.14723707)(75.65709692,219.15223706)(75.61710083,219.14223694)
\curveto(75.577097,219.13223708)(75.54209704,219.13223708)(75.51210083,219.14223694)
\lineto(75.42210083,219.14223694)
\curveto(75.36209722,219.16223705)(75.29709728,219.17723704)(75.22710083,219.18723694)
\curveto(75.16709741,219.18723703)(75.10209748,219.19223702)(75.03210083,219.20223694)
\curveto(74.86209772,219.25223696)(74.70209788,219.30223691)(74.55210083,219.35223694)
\curveto(74.40209818,219.40223681)(74.25709832,219.46723675)(74.11710083,219.54723694)
\curveto(74.06709851,219.58723663)(74.01209857,219.6172366)(73.95210083,219.63723694)
\curveto(73.90209868,219.66723655)(73.85209873,219.70223651)(73.80210083,219.74223694)
\curveto(73.56209902,219.92223629)(73.36209922,220.14223607)(73.20210083,220.40223694)
\curveto(73.04209954,220.66223555)(72.90209968,220.94723527)(72.78210083,221.25723694)
\curveto(72.72209986,221.39723482)(72.6770999,221.53723468)(72.64710083,221.67723694)
\curveto(72.61709996,221.82723439)(72.5821,221.98223423)(72.54210083,222.14223694)
\curveto(72.52210006,222.25223396)(72.50710007,222.36223385)(72.49710083,222.47223694)
\curveto(72.48710009,222.58223363)(72.47210011,222.69223352)(72.45210083,222.80223694)
\curveto(72.44210014,222.84223337)(72.43710014,222.88223333)(72.43710083,222.92223694)
\curveto(72.44710013,222.96223325)(72.44710013,223.00223321)(72.43710083,223.04223694)
\curveto(72.42710015,223.09223312)(72.42210016,223.14223307)(72.42210083,223.19223694)
\lineto(72.42210083,223.35723694)
\curveto(72.40210018,223.40723281)(72.39710018,223.45723276)(72.40710083,223.50723694)
\curveto(72.41710016,223.56723265)(72.41710016,223.62223259)(72.40710083,223.67223694)
\curveto(72.39710018,223.7122325)(72.39710018,223.75723246)(72.40710083,223.80723694)
\curveto(72.41710016,223.85723236)(72.41210017,223.90723231)(72.39210083,223.95723694)
\curveto(72.37210021,224.02723219)(72.36710021,224.10223211)(72.37710083,224.18223694)
\curveto(72.38710019,224.27223194)(72.39210019,224.35723186)(72.39210083,224.43723694)
\curveto(72.39210019,224.52723169)(72.38710019,224.62723159)(72.37710083,224.73723694)
\curveto(72.36710021,224.85723136)(72.37210021,224.95723126)(72.39210083,225.03723694)
\lineto(72.39210083,225.32223694)
\lineto(72.43710083,225.95223694)
\curveto(72.44710013,226.05223016)(72.45710012,226.14723007)(72.46710083,226.23723694)
\lineto(72.49710083,226.53723694)
\curveto(72.51710006,226.58722963)(72.52210006,226.63722958)(72.51210083,226.68723694)
\curveto(72.51210007,226.74722947)(72.52210006,226.80222941)(72.54210083,226.85223694)
\curveto(72.59209999,227.02222919)(72.63209995,227.18722903)(72.66210083,227.34723694)
\curveto(72.69209989,227.5172287)(72.74209984,227.67722854)(72.81210083,227.82723694)
\curveto(73.00209958,228.28722793)(73.22209936,228.66222755)(73.47210083,228.95223694)
\curveto(73.73209885,229.24222697)(74.09209849,229.48722673)(74.55210083,229.68723694)
\curveto(74.6820979,229.73722648)(74.81209777,229.77222644)(74.94210083,229.79223694)
\curveto(75.0820975,229.8122264)(75.22209736,229.83722638)(75.36210083,229.86723694)
\curveto(75.43209715,229.87722634)(75.49709708,229.88222633)(75.55710083,229.88223694)
\curveto(75.61709696,229.88222633)(75.6820969,229.88722633)(75.75210083,229.89723694)
\curveto(76.582096,229.9172263)(77.25209533,229.76722645)(77.76210083,229.44723694)
\curveto(78.27209431,229.13722708)(78.65209393,228.69722752)(78.90210083,228.12723694)
\curveto(78.95209363,228.00722821)(78.99709358,227.88222833)(79.03710083,227.75223694)
\curveto(79.0770935,227.62222859)(79.12209346,227.48722873)(79.17210083,227.34723694)
\curveto(79.19209339,227.26722895)(79.20709337,227.18222903)(79.21710083,227.09223694)
\lineto(79.27710083,226.85223694)
\curveto(79.30709327,226.74222947)(79.32209326,226.63222958)(79.32210083,226.52223694)
\curveto(79.33209325,226.4122298)(79.34709323,226.30222991)(79.36710083,226.19223694)
\curveto(79.38709319,226.14223007)(79.39209319,226.09723012)(79.38210083,226.05723694)
\curveto(79.3820932,226.0172302)(79.38709319,225.97723024)(79.39710083,225.93723694)
\curveto(79.40709317,225.88723033)(79.40709317,225.83223038)(79.39710083,225.77223694)
\curveto(79.39709318,225.72223049)(79.40209318,225.67223054)(79.41210083,225.62223694)
\lineto(79.41210083,225.48723694)
\curveto(79.43209315,225.42723079)(79.43209315,225.35723086)(79.41210083,225.27723694)
\curveto(79.40209318,225.20723101)(79.40709317,225.14223107)(79.42710083,225.08223694)
\curveto(79.43709314,225.05223116)(79.44209314,225.0122312)(79.44210083,224.96223694)
\lineto(79.44210083,224.84223694)
\lineto(79.44210083,224.37723694)
\moveto(77.89710083,222.05223694)
\curveto(77.99709458,222.37223384)(78.05709452,222.73723348)(78.07710083,223.14723694)
\curveto(78.09709448,223.55723266)(78.10709447,223.96723225)(78.10710083,224.37723694)
\curveto(78.10709447,224.80723141)(78.09709448,225.22723099)(78.07710083,225.63723694)
\curveto(78.05709452,226.04723017)(78.01209457,226.43222978)(77.94210083,226.79223694)
\curveto(77.87209471,227.15222906)(77.76209482,227.47222874)(77.61210083,227.75223694)
\curveto(77.47209511,228.04222817)(77.2770953,228.27722794)(77.02710083,228.45723694)
\curveto(76.86709571,228.56722765)(76.68709589,228.64722757)(76.48710083,228.69723694)
\curveto(76.28709629,228.75722746)(76.04209654,228.78722743)(75.75210083,228.78723694)
\curveto(75.73209685,228.76722745)(75.69709688,228.75722746)(75.64710083,228.75723694)
\curveto(75.59709698,228.76722745)(75.55709702,228.76722745)(75.52710083,228.75723694)
\curveto(75.44709713,228.73722748)(75.37209721,228.7172275)(75.30210083,228.69723694)
\curveto(75.24209734,228.68722753)(75.1770974,228.66722755)(75.10710083,228.63723694)
\curveto(74.83709774,228.5172277)(74.61709796,228.34722787)(74.44710083,228.12723694)
\curveto(74.28709829,227.9172283)(74.15209843,227.67222854)(74.04210083,227.39223694)
\curveto(73.99209859,227.28222893)(73.95209863,227.16222905)(73.92210083,227.03223694)
\curveto(73.90209868,226.9122293)(73.8770987,226.78722943)(73.84710083,226.65723694)
\curveto(73.82709875,226.60722961)(73.81709876,226.55222966)(73.81710083,226.49223694)
\curveto(73.81709876,226.44222977)(73.81209877,226.39222982)(73.80210083,226.34223694)
\curveto(73.79209879,226.25222996)(73.7820988,226.15723006)(73.77210083,226.05723694)
\curveto(73.76209882,225.96723025)(73.75209883,225.87223034)(73.74210083,225.77223694)
\curveto(73.74209884,225.69223052)(73.73709884,225.60723061)(73.72710083,225.51723694)
\lineto(73.72710083,225.27723694)
\lineto(73.72710083,225.09723694)
\curveto(73.71709886,225.06723115)(73.71209887,225.03223118)(73.71210083,224.99223694)
\lineto(73.71210083,224.85723694)
\lineto(73.71210083,224.40723694)
\curveto(73.71209887,224.32723189)(73.70709887,224.24223197)(73.69710083,224.15223694)
\curveto(73.69709888,224.07223214)(73.70709887,223.99723222)(73.72710083,223.92723694)
\lineto(73.72710083,223.65723694)
\curveto(73.72709885,223.63723258)(73.72209886,223.60723261)(73.71210083,223.56723694)
\curveto(73.71209887,223.53723268)(73.71709886,223.5122327)(73.72710083,223.49223694)
\curveto(73.73709884,223.39223282)(73.74209884,223.29223292)(73.74210083,223.19223694)
\curveto(73.75209883,223.10223311)(73.76209882,223.00223321)(73.77210083,222.89223694)
\curveto(73.80209878,222.77223344)(73.81709876,222.64723357)(73.81710083,222.51723694)
\curveto(73.82709875,222.39723382)(73.85209873,222.28223393)(73.89210083,222.17223694)
\curveto(73.97209861,221.87223434)(74.05709852,221.60723461)(74.14710083,221.37723694)
\curveto(74.24709833,221.14723507)(74.39209819,220.93223528)(74.58210083,220.73223694)
\curveto(74.79209779,220.53223568)(75.05709752,220.38223583)(75.37710083,220.28223694)
\curveto(75.41709716,220.26223595)(75.45209713,220.25223596)(75.48210083,220.25223694)
\curveto(75.52209706,220.26223595)(75.56709701,220.25723596)(75.61710083,220.23723694)
\curveto(75.65709692,220.22723599)(75.72709685,220.217236)(75.82710083,220.20723694)
\curveto(75.93709664,220.19723602)(76.02209656,220.20223601)(76.08210083,220.22223694)
\curveto(76.15209643,220.24223597)(76.22209636,220.25223596)(76.29210083,220.25223694)
\curveto(76.36209622,220.26223595)(76.42709615,220.27723594)(76.48710083,220.29723694)
\curveto(76.68709589,220.35723586)(76.86709571,220.44223577)(77.02710083,220.55223694)
\curveto(77.05709552,220.57223564)(77.0820955,220.59223562)(77.10210083,220.61223694)
\lineto(77.16210083,220.67223694)
\curveto(77.20209538,220.69223552)(77.25209533,220.73223548)(77.31210083,220.79223694)
\curveto(77.41209517,220.93223528)(77.49709508,221.06223515)(77.56710083,221.18223694)
\curveto(77.63709494,221.30223491)(77.70709487,221.44723477)(77.77710083,221.61723694)
\curveto(77.80709477,221.68723453)(77.82709475,221.75723446)(77.83710083,221.82723694)
\curveto(77.85709472,221.89723432)(77.8770947,221.97223424)(77.89710083,222.05223694)
}
}
{
\newrgbcolor{curcolor}{0 0 0}
\pscustom[linestyle=none,fillstyle=solid,fillcolor=curcolor]
{
\newpath
\moveto(56.20288208,304.70223694)
\lineto(61.00288208,304.70223694)
\lineto(62.00788208,304.70223694)
\curveto(62.14787498,304.70222651)(62.26787486,304.69222652)(62.36788208,304.67223694)
\curveto(62.47787465,304.66222655)(62.55787457,304.6172266)(62.60788208,304.53723694)
\curveto(62.6278745,304.49722672)(62.63787449,304.44722677)(62.63788208,304.38723694)
\curveto(62.64787448,304.32722689)(62.65287448,304.26222695)(62.65288208,304.19223694)
\lineto(62.65288208,303.92223694)
\curveto(62.65287448,303.83222738)(62.64287449,303.75222746)(62.62288208,303.68223694)
\curveto(62.58287455,303.60222761)(62.53787459,303.53222768)(62.48788208,303.47223694)
\lineto(62.33788208,303.29223694)
\curveto(62.30787482,303.24222797)(62.27287486,303.20222801)(62.23288208,303.17223694)
\curveto(62.19287494,303.14222807)(62.15287498,303.10222811)(62.11288208,303.05223694)
\curveto(62.0328751,302.94222827)(61.94787518,302.83222838)(61.85788208,302.72223694)
\curveto(61.76787536,302.62222859)(61.68287545,302.5172287)(61.60288208,302.40723694)
\curveto(61.46287567,302.20722901)(61.32287581,301.99722922)(61.18288208,301.77723694)
\curveto(61.04287609,301.56722965)(60.90287623,301.35222986)(60.76288208,301.13223694)
\curveto(60.71287642,301.04223017)(60.66287647,300.94723027)(60.61288208,300.84723694)
\curveto(60.56287657,300.74723047)(60.50787662,300.65223056)(60.44788208,300.56223694)
\curveto(60.4278767,300.54223067)(60.41787671,300.5172307)(60.41788208,300.48723694)
\curveto(60.41787671,300.45723076)(60.40787672,300.43223078)(60.38788208,300.41223694)
\curveto(60.31787681,300.3122309)(60.25287688,300.19723102)(60.19288208,300.06723694)
\curveto(60.132877,299.94723127)(60.07787705,299.83223138)(60.02788208,299.72223694)
\curveto(59.9278772,299.49223172)(59.8328773,299.25723196)(59.74288208,299.01723694)
\curveto(59.65287748,298.77723244)(59.55287758,298.53723268)(59.44288208,298.29723694)
\curveto(59.42287771,298.24723297)(59.40787772,298.20223301)(59.39788208,298.16223694)
\curveto(59.39787773,298.12223309)(59.38787774,298.07723314)(59.36788208,298.02723694)
\curveto(59.31787781,297.90723331)(59.27287786,297.78223343)(59.23288208,297.65223694)
\curveto(59.20287793,297.53223368)(59.16787796,297.4122338)(59.12788208,297.29223694)
\curveto(59.04787808,297.06223415)(58.98287815,296.82223439)(58.93288208,296.57223694)
\curveto(58.89287824,296.33223488)(58.84287829,296.09223512)(58.78288208,295.85223694)
\curveto(58.74287839,295.70223551)(58.71787841,295.55223566)(58.70788208,295.40223694)
\curveto(58.69787843,295.25223596)(58.67787845,295.10223611)(58.64788208,294.95223694)
\curveto(58.63787849,294.9122363)(58.6328785,294.85223636)(58.63288208,294.77223694)
\curveto(58.60287853,294.65223656)(58.57287856,294.55223666)(58.54288208,294.47223694)
\curveto(58.51287862,294.39223682)(58.44287869,294.33723688)(58.33288208,294.30723694)
\curveto(58.28287885,294.28723693)(58.2278789,294.27723694)(58.16788208,294.27723694)
\lineto(57.97288208,294.27723694)
\curveto(57.8328793,294.27723694)(57.69287944,294.28223693)(57.55288208,294.29223694)
\curveto(57.42287971,294.30223691)(57.3278798,294.34723687)(57.26788208,294.42723694)
\curveto(57.2278799,294.48723673)(57.20787992,294.57223664)(57.20788208,294.68223694)
\curveto(57.21787991,294.79223642)(57.2328799,294.88723633)(57.25288208,294.96723694)
\lineto(57.25288208,295.04223694)
\curveto(57.26287987,295.07223614)(57.26787986,295.10223611)(57.26788208,295.13223694)
\curveto(57.28787984,295.212236)(57.29787983,295.28723593)(57.29788208,295.35723694)
\curveto(57.29787983,295.42723579)(57.30787982,295.49723572)(57.32788208,295.56723694)
\curveto(57.37787975,295.75723546)(57.41787971,295.94223527)(57.44788208,296.12223694)
\curveto(57.47787965,296.3122349)(57.51787961,296.49223472)(57.56788208,296.66223694)
\curveto(57.58787954,296.7122345)(57.59787953,296.75223446)(57.59788208,296.78223694)
\curveto(57.59787953,296.8122344)(57.60287953,296.84723437)(57.61288208,296.88723694)
\curveto(57.71287942,297.18723403)(57.80287933,297.48223373)(57.88288208,297.77223694)
\curveto(57.97287916,298.06223315)(58.07787905,298.34223287)(58.19788208,298.61223694)
\curveto(58.45787867,299.19223202)(58.7278784,299.74223147)(59.00788208,300.26223694)
\curveto(59.28787784,300.79223042)(59.59787753,301.29722992)(59.93788208,301.77723694)
\curveto(60.07787705,301.97722924)(60.2278769,302.16722905)(60.38788208,302.34723694)
\curveto(60.54787658,302.53722868)(60.69787643,302.72722849)(60.83788208,302.91723694)
\curveto(60.87787625,302.96722825)(60.91287622,303.0122282)(60.94288208,303.05223694)
\curveto(60.98287615,303.10222811)(61.01787611,303.15222806)(61.04788208,303.20223694)
\curveto(61.05787607,303.22222799)(61.06787606,303.24722797)(61.07788208,303.27723694)
\curveto(61.09787603,303.30722791)(61.09787603,303.33722788)(61.07788208,303.36723694)
\curveto(61.05787607,303.42722779)(61.02287611,303.46222775)(60.97288208,303.47223694)
\curveto(60.92287621,303.49222772)(60.87287626,303.5122277)(60.82288208,303.53223694)
\lineto(60.71788208,303.53223694)
\curveto(60.67787645,303.54222767)(60.6278765,303.54222767)(60.56788208,303.53223694)
\lineto(60.41788208,303.53223694)
\lineto(59.81788208,303.53223694)
\lineto(57.17788208,303.53223694)
\lineto(56.44288208,303.53223694)
\lineto(56.20288208,303.53223694)
\curveto(56.132881,303.54222767)(56.07288106,303.55722766)(56.02288208,303.57723694)
\curveto(55.9328812,303.6172276)(55.87288126,303.67722754)(55.84288208,303.75723694)
\curveto(55.79288134,303.85722736)(55.77788135,304.00222721)(55.79788208,304.19223694)
\curveto(55.81788131,304.39222682)(55.85288128,304.52722669)(55.90288208,304.59723694)
\curveto(55.92288121,304.6172266)(55.94788118,304.63222658)(55.97788208,304.64223694)
\lineto(56.09788208,304.70223694)
\curveto(56.11788101,304.70222651)(56.132881,304.69722652)(56.14288208,304.68723694)
\curveto(56.16288097,304.68722653)(56.18288095,304.69222652)(56.20288208,304.70223694)
}
}
{
\newrgbcolor{curcolor}{0 0 0}
\pscustom[linestyle=none,fillstyle=solid,fillcolor=curcolor]
{
\newpath
\moveto(65.60249146,304.70223694)
\lineto(69.20249146,304.70223694)
\lineto(69.84749146,304.70223694)
\curveto(69.92748493,304.70222651)(70.00248485,304.69722652)(70.07249146,304.68723694)
\curveto(70.14248471,304.68722653)(70.20248465,304.67722654)(70.25249146,304.65723694)
\curveto(70.32248453,304.62722659)(70.37748448,304.56722665)(70.41749146,304.47723694)
\curveto(70.43748442,304.44722677)(70.44748441,304.40722681)(70.44749146,304.35723694)
\lineto(70.44749146,304.22223694)
\curveto(70.4574844,304.1122271)(70.4524844,304.00722721)(70.43249146,303.90723694)
\curveto(70.42248443,303.80722741)(70.38748447,303.73722748)(70.32749146,303.69723694)
\curveto(70.23748462,303.62722759)(70.10248475,303.59222762)(69.92249146,303.59223694)
\curveto(69.74248511,303.60222761)(69.57748528,303.60722761)(69.42749146,303.60723694)
\lineto(67.43249146,303.60723694)
\lineto(66.93749146,303.60723694)
\lineto(66.80249146,303.60723694)
\curveto(66.76248809,303.60722761)(66.72248813,303.60222761)(66.68249146,303.59223694)
\lineto(66.47249146,303.59223694)
\curveto(66.36248849,303.56222765)(66.28248857,303.52222769)(66.23249146,303.47223694)
\curveto(66.18248867,303.43222778)(66.14748871,303.37722784)(66.12749146,303.30723694)
\curveto(66.10748875,303.24722797)(66.09248876,303.17722804)(66.08249146,303.09723694)
\curveto(66.07248878,303.0172282)(66.0524888,302.92722829)(66.02249146,302.82723694)
\curveto(65.97248888,302.62722859)(65.93248892,302.42222879)(65.90249146,302.21223694)
\curveto(65.87248898,302.00222921)(65.83248902,301.79722942)(65.78249146,301.59723694)
\curveto(65.76248909,301.52722969)(65.7524891,301.45722976)(65.75249146,301.38723694)
\curveto(65.7524891,301.32722989)(65.74248911,301.26222995)(65.72249146,301.19223694)
\curveto(65.71248914,301.16223005)(65.70248915,301.12223009)(65.69249146,301.07223694)
\curveto(65.69248916,301.03223018)(65.69748916,300.99223022)(65.70749146,300.95223694)
\curveto(65.72748913,300.90223031)(65.7524891,300.85723036)(65.78249146,300.81723694)
\curveto(65.82248903,300.78723043)(65.88248897,300.78223043)(65.96249146,300.80223694)
\curveto(66.02248883,300.82223039)(66.08248877,300.84723037)(66.14249146,300.87723694)
\curveto(66.20248865,300.9172303)(66.26248859,300.95223026)(66.32249146,300.98223694)
\curveto(66.38248847,301.00223021)(66.43248842,301.0172302)(66.47249146,301.02723694)
\curveto(66.66248819,301.10723011)(66.86748799,301.16223005)(67.08749146,301.19223694)
\curveto(67.31748754,301.22222999)(67.54748731,301.23222998)(67.77749146,301.22223694)
\curveto(68.01748684,301.22222999)(68.24748661,301.19723002)(68.46749146,301.14723694)
\curveto(68.68748617,301.10723011)(68.88748597,301.04723017)(69.06749146,300.96723694)
\curveto(69.11748574,300.94723027)(69.16248569,300.92723029)(69.20249146,300.90723694)
\curveto(69.2524856,300.88723033)(69.30248555,300.86223035)(69.35249146,300.83223694)
\curveto(69.70248515,300.62223059)(69.98248487,300.39223082)(70.19249146,300.14223694)
\curveto(70.41248444,299.89223132)(70.60748425,299.56723165)(70.77749146,299.16723694)
\curveto(70.82748403,299.05723216)(70.86248399,298.94723227)(70.88249146,298.83723694)
\curveto(70.90248395,298.72723249)(70.92748393,298.6122326)(70.95749146,298.49223694)
\curveto(70.96748389,298.46223275)(70.97248388,298.4172328)(70.97249146,298.35723694)
\curveto(70.99248386,298.29723292)(71.00248385,298.22723299)(71.00249146,298.14723694)
\curveto(71.00248385,298.07723314)(71.01248384,298.0122332)(71.03249146,297.95223694)
\lineto(71.03249146,297.78723694)
\curveto(71.04248381,297.73723348)(71.04748381,297.66723355)(71.04749146,297.57723694)
\curveto(71.04748381,297.48723373)(71.03748382,297.4172338)(71.01749146,297.36723694)
\curveto(70.99748386,297.30723391)(70.99248386,297.24723397)(71.00249146,297.18723694)
\curveto(71.01248384,297.13723408)(71.00748385,297.08723413)(70.98749146,297.03723694)
\curveto(70.94748391,296.87723434)(70.91248394,296.72723449)(70.88249146,296.58723694)
\curveto(70.852484,296.44723477)(70.80748405,296.3122349)(70.74749146,296.18223694)
\curveto(70.58748427,295.8122354)(70.36748449,295.47723574)(70.08749146,295.17723694)
\curveto(69.80748505,294.87723634)(69.48748537,294.64723657)(69.12749146,294.48723694)
\curveto(68.9574859,294.40723681)(68.7574861,294.33223688)(68.52749146,294.26223694)
\curveto(68.41748644,294.22223699)(68.30248655,294.19723702)(68.18249146,294.18723694)
\curveto(68.06248679,294.17723704)(67.94248691,294.15723706)(67.82249146,294.12723694)
\curveto(67.77248708,294.10723711)(67.71748714,294.10723711)(67.65749146,294.12723694)
\curveto(67.59748726,294.13723708)(67.53748732,294.13223708)(67.47749146,294.11223694)
\curveto(67.37748748,294.09223712)(67.27748758,294.09223712)(67.17749146,294.11223694)
\lineto(67.04249146,294.11223694)
\curveto(66.99248786,294.13223708)(66.93248792,294.14223707)(66.86249146,294.14223694)
\curveto(66.80248805,294.13223708)(66.74748811,294.13723708)(66.69749146,294.15723694)
\curveto(66.6574882,294.16723705)(66.62248823,294.17223704)(66.59249146,294.17223694)
\curveto(66.56248829,294.17223704)(66.52748833,294.17723704)(66.48749146,294.18723694)
\lineto(66.21749146,294.24723694)
\curveto(66.12748873,294.26723695)(66.04248881,294.29723692)(65.96249146,294.33723694)
\curveto(65.62248923,294.47723674)(65.33248952,294.63223658)(65.09249146,294.80223694)
\curveto(64.85249,294.98223623)(64.63249022,295.212236)(64.43249146,295.49223694)
\curveto(64.28249057,295.72223549)(64.16749069,295.96223525)(64.08749146,296.21223694)
\curveto(64.06749079,296.26223495)(64.0574908,296.30723491)(64.05749146,296.34723694)
\curveto(64.0574908,296.39723482)(64.04749081,296.44723477)(64.02749146,296.49723694)
\curveto(64.00749085,296.55723466)(63.99249086,296.63723458)(63.98249146,296.73723694)
\curveto(63.98249087,296.83723438)(64.00249085,296.9122343)(64.04249146,296.96223694)
\curveto(64.09249076,297.04223417)(64.17249068,297.08723413)(64.28249146,297.09723694)
\curveto(64.39249046,297.10723411)(64.50749035,297.1122341)(64.62749146,297.11223694)
\lineto(64.79249146,297.11223694)
\curveto(64.85249,297.1122341)(64.90748995,297.10223411)(64.95749146,297.08223694)
\curveto(65.04748981,297.06223415)(65.11748974,297.02223419)(65.16749146,296.96223694)
\curveto(65.23748962,296.87223434)(65.28248957,296.76223445)(65.30249146,296.63223694)
\curveto(65.33248952,296.5122347)(65.37748948,296.40723481)(65.43749146,296.31723694)
\curveto(65.62748923,295.97723524)(65.88748897,295.70723551)(66.21749146,295.50723694)
\curveto(66.31748854,295.44723577)(66.42248843,295.39723582)(66.53249146,295.35723694)
\curveto(66.6524882,295.32723589)(66.77248808,295.29223592)(66.89249146,295.25223694)
\curveto(67.06248779,295.20223601)(67.26748759,295.18223603)(67.50749146,295.19223694)
\curveto(67.7574871,295.212236)(67.9574869,295.24723597)(68.10749146,295.29723694)
\curveto(68.47748638,295.4172358)(68.76748609,295.57723564)(68.97749146,295.77723694)
\curveto(69.19748566,295.98723523)(69.37748548,296.26723495)(69.51749146,296.61723694)
\curveto(69.56748529,296.7172345)(69.59748526,296.82223439)(69.60749146,296.93223694)
\curveto(69.62748523,297.04223417)(69.6524852,297.15723406)(69.68249146,297.27723694)
\lineto(69.68249146,297.38223694)
\curveto(69.69248516,297.42223379)(69.69748516,297.46223375)(69.69749146,297.50223694)
\curveto(69.70748515,297.53223368)(69.70748515,297.56723365)(69.69749146,297.60723694)
\lineto(69.69749146,297.72723694)
\curveto(69.69748516,297.98723323)(69.66748519,298.23223298)(69.60749146,298.46223694)
\curveto(69.49748536,298.8122324)(69.34248551,299.10723211)(69.14249146,299.34723694)
\curveto(68.94248591,299.59723162)(68.68248617,299.79223142)(68.36249146,299.93223694)
\lineto(68.18249146,299.99223694)
\curveto(68.13248672,300.0122312)(68.07248678,300.03223118)(68.00249146,300.05223694)
\curveto(67.9524869,300.07223114)(67.89248696,300.08223113)(67.82249146,300.08223694)
\curveto(67.76248709,300.09223112)(67.69748716,300.10723111)(67.62749146,300.12723694)
\lineto(67.47749146,300.12723694)
\curveto(67.43748742,300.14723107)(67.38248747,300.15723106)(67.31249146,300.15723694)
\curveto(67.2524876,300.15723106)(67.19748766,300.14723107)(67.14749146,300.12723694)
\lineto(67.04249146,300.12723694)
\curveto(67.01248784,300.12723109)(66.97748788,300.12223109)(66.93749146,300.11223694)
\lineto(66.69749146,300.05223694)
\curveto(66.61748824,300.04223117)(66.53748832,300.02223119)(66.45749146,299.99223694)
\curveto(66.21748864,299.89223132)(65.98748887,299.75723146)(65.76749146,299.58723694)
\curveto(65.67748918,299.5172317)(65.59248926,299.44223177)(65.51249146,299.36223694)
\curveto(65.43248942,299.29223192)(65.33248952,299.23723198)(65.21249146,299.19723694)
\curveto(65.12248973,299.16723205)(64.98248987,299.15723206)(64.79249146,299.16723694)
\curveto(64.61249024,299.17723204)(64.49249036,299.20223201)(64.43249146,299.24223694)
\curveto(64.38249047,299.28223193)(64.34249051,299.34223187)(64.31249146,299.42223694)
\curveto(64.29249056,299.50223171)(64.29249056,299.58723163)(64.31249146,299.67723694)
\curveto(64.34249051,299.79723142)(64.36249049,299.9172313)(64.37249146,300.03723694)
\curveto(64.39249046,300.16723105)(64.41749044,300.29223092)(64.44749146,300.41223694)
\curveto(64.46749039,300.45223076)(64.47249038,300.48723073)(64.46249146,300.51723694)
\curveto(64.46249039,300.55723066)(64.47249038,300.60223061)(64.49249146,300.65223694)
\curveto(64.51249034,300.74223047)(64.52749033,300.83223038)(64.53749146,300.92223694)
\curveto(64.54749031,301.02223019)(64.56749029,301.1172301)(64.59749146,301.20723694)
\curveto(64.60749025,301.26722995)(64.61249024,301.32722989)(64.61249146,301.38723694)
\curveto(64.62249023,301.44722977)(64.63749022,301.50722971)(64.65749146,301.56723694)
\curveto(64.70749015,301.76722945)(64.74249011,301.97222924)(64.76249146,302.18223694)
\curveto(64.79249006,302.40222881)(64.83249002,302.6122286)(64.88249146,302.81223694)
\curveto(64.91248994,302.9122283)(64.93248992,303.0122282)(64.94249146,303.11223694)
\curveto(64.9524899,303.212228)(64.96748989,303.3122279)(64.98749146,303.41223694)
\curveto(64.99748986,303.44222777)(65.00248985,303.48222773)(65.00249146,303.53223694)
\curveto(65.03248982,303.64222757)(65.0524898,303.74722747)(65.06249146,303.84723694)
\curveto(65.08248977,303.95722726)(65.10748975,304.06722715)(65.13749146,304.17723694)
\curveto(65.1574897,304.25722696)(65.17248968,304.32722689)(65.18249146,304.38723694)
\curveto(65.19248966,304.45722676)(65.21748964,304.5172267)(65.25749146,304.56723694)
\curveto(65.27748958,304.59722662)(65.30748955,304.6172266)(65.34749146,304.62723694)
\curveto(65.38748947,304.64722657)(65.43248942,304.66722655)(65.48249146,304.68723694)
\curveto(65.54248931,304.68722653)(65.58248927,304.69222652)(65.60249146,304.70223694)
}
}
{
\newrgbcolor{curcolor}{0 0 0}
\pscustom[linestyle=none,fillstyle=solid,fillcolor=curcolor]
{
\newpath
\moveto(79.44210083,299.37723694)
\lineto(79.44210083,299.12223694)
\curveto(79.45209313,299.04223217)(79.44709313,298.96723225)(79.42710083,298.89723694)
\lineto(79.42710083,298.65723694)
\lineto(79.42710083,298.49223694)
\curveto(79.40709317,298.39223282)(79.39709318,298.28723293)(79.39710083,298.17723694)
\curveto(79.39709318,298.07723314)(79.38709319,297.97723324)(79.36710083,297.87723694)
\lineto(79.36710083,297.72723694)
\curveto(79.33709324,297.58723363)(79.31709326,297.44723377)(79.30710083,297.30723694)
\curveto(79.29709328,297.17723404)(79.27209331,297.04723417)(79.23210083,296.91723694)
\curveto(79.21209337,296.83723438)(79.19209339,296.75223446)(79.17210083,296.66223694)
\lineto(79.11210083,296.42223694)
\lineto(78.99210083,296.12223694)
\curveto(78.96209362,296.03223518)(78.92709365,295.94223527)(78.88710083,295.85223694)
\curveto(78.78709379,295.63223558)(78.65209393,295.4172358)(78.48210083,295.20723694)
\curveto(78.32209426,294.99723622)(78.14709443,294.82723639)(77.95710083,294.69723694)
\curveto(77.90709467,294.65723656)(77.84709473,294.6172366)(77.77710083,294.57723694)
\curveto(77.71709486,294.54723667)(77.65709492,294.5122367)(77.59710083,294.47223694)
\curveto(77.51709506,294.42223679)(77.42209516,294.38223683)(77.31210083,294.35223694)
\curveto(77.20209538,294.32223689)(77.09709548,294.29223692)(76.99710083,294.26223694)
\curveto(76.88709569,294.22223699)(76.7770958,294.19723702)(76.66710083,294.18723694)
\curveto(76.55709602,294.17723704)(76.44209614,294.16223705)(76.32210083,294.14223694)
\curveto(76.2820963,294.13223708)(76.23709634,294.13223708)(76.18710083,294.14223694)
\curveto(76.14709643,294.14223707)(76.10709647,294.13723708)(76.06710083,294.12723694)
\curveto(76.02709655,294.1172371)(75.97209661,294.1122371)(75.90210083,294.11223694)
\curveto(75.83209675,294.1122371)(75.7820968,294.1172371)(75.75210083,294.12723694)
\curveto(75.70209688,294.14723707)(75.65709692,294.15223706)(75.61710083,294.14223694)
\curveto(75.577097,294.13223708)(75.54209704,294.13223708)(75.51210083,294.14223694)
\lineto(75.42210083,294.14223694)
\curveto(75.36209722,294.16223705)(75.29709728,294.17723704)(75.22710083,294.18723694)
\curveto(75.16709741,294.18723703)(75.10209748,294.19223702)(75.03210083,294.20223694)
\curveto(74.86209772,294.25223696)(74.70209788,294.30223691)(74.55210083,294.35223694)
\curveto(74.40209818,294.40223681)(74.25709832,294.46723675)(74.11710083,294.54723694)
\curveto(74.06709851,294.58723663)(74.01209857,294.6172366)(73.95210083,294.63723694)
\curveto(73.90209868,294.66723655)(73.85209873,294.70223651)(73.80210083,294.74223694)
\curveto(73.56209902,294.92223629)(73.36209922,295.14223607)(73.20210083,295.40223694)
\curveto(73.04209954,295.66223555)(72.90209968,295.94723527)(72.78210083,296.25723694)
\curveto(72.72209986,296.39723482)(72.6770999,296.53723468)(72.64710083,296.67723694)
\curveto(72.61709996,296.82723439)(72.5821,296.98223423)(72.54210083,297.14223694)
\curveto(72.52210006,297.25223396)(72.50710007,297.36223385)(72.49710083,297.47223694)
\curveto(72.48710009,297.58223363)(72.47210011,297.69223352)(72.45210083,297.80223694)
\curveto(72.44210014,297.84223337)(72.43710014,297.88223333)(72.43710083,297.92223694)
\curveto(72.44710013,297.96223325)(72.44710013,298.00223321)(72.43710083,298.04223694)
\curveto(72.42710015,298.09223312)(72.42210016,298.14223307)(72.42210083,298.19223694)
\lineto(72.42210083,298.35723694)
\curveto(72.40210018,298.40723281)(72.39710018,298.45723276)(72.40710083,298.50723694)
\curveto(72.41710016,298.56723265)(72.41710016,298.62223259)(72.40710083,298.67223694)
\curveto(72.39710018,298.7122325)(72.39710018,298.75723246)(72.40710083,298.80723694)
\curveto(72.41710016,298.85723236)(72.41210017,298.90723231)(72.39210083,298.95723694)
\curveto(72.37210021,299.02723219)(72.36710021,299.10223211)(72.37710083,299.18223694)
\curveto(72.38710019,299.27223194)(72.39210019,299.35723186)(72.39210083,299.43723694)
\curveto(72.39210019,299.52723169)(72.38710019,299.62723159)(72.37710083,299.73723694)
\curveto(72.36710021,299.85723136)(72.37210021,299.95723126)(72.39210083,300.03723694)
\lineto(72.39210083,300.32223694)
\lineto(72.43710083,300.95223694)
\curveto(72.44710013,301.05223016)(72.45710012,301.14723007)(72.46710083,301.23723694)
\lineto(72.49710083,301.53723694)
\curveto(72.51710006,301.58722963)(72.52210006,301.63722958)(72.51210083,301.68723694)
\curveto(72.51210007,301.74722947)(72.52210006,301.80222941)(72.54210083,301.85223694)
\curveto(72.59209999,302.02222919)(72.63209995,302.18722903)(72.66210083,302.34723694)
\curveto(72.69209989,302.5172287)(72.74209984,302.67722854)(72.81210083,302.82723694)
\curveto(73.00209958,303.28722793)(73.22209936,303.66222755)(73.47210083,303.95223694)
\curveto(73.73209885,304.24222697)(74.09209849,304.48722673)(74.55210083,304.68723694)
\curveto(74.6820979,304.73722648)(74.81209777,304.77222644)(74.94210083,304.79223694)
\curveto(75.0820975,304.8122264)(75.22209736,304.83722638)(75.36210083,304.86723694)
\curveto(75.43209715,304.87722634)(75.49709708,304.88222633)(75.55710083,304.88223694)
\curveto(75.61709696,304.88222633)(75.6820969,304.88722633)(75.75210083,304.89723694)
\curveto(76.582096,304.9172263)(77.25209533,304.76722645)(77.76210083,304.44723694)
\curveto(78.27209431,304.13722708)(78.65209393,303.69722752)(78.90210083,303.12723694)
\curveto(78.95209363,303.00722821)(78.99709358,302.88222833)(79.03710083,302.75223694)
\curveto(79.0770935,302.62222859)(79.12209346,302.48722873)(79.17210083,302.34723694)
\curveto(79.19209339,302.26722895)(79.20709337,302.18222903)(79.21710083,302.09223694)
\lineto(79.27710083,301.85223694)
\curveto(79.30709327,301.74222947)(79.32209326,301.63222958)(79.32210083,301.52223694)
\curveto(79.33209325,301.4122298)(79.34709323,301.30222991)(79.36710083,301.19223694)
\curveto(79.38709319,301.14223007)(79.39209319,301.09723012)(79.38210083,301.05723694)
\curveto(79.3820932,301.0172302)(79.38709319,300.97723024)(79.39710083,300.93723694)
\curveto(79.40709317,300.88723033)(79.40709317,300.83223038)(79.39710083,300.77223694)
\curveto(79.39709318,300.72223049)(79.40209318,300.67223054)(79.41210083,300.62223694)
\lineto(79.41210083,300.48723694)
\curveto(79.43209315,300.42723079)(79.43209315,300.35723086)(79.41210083,300.27723694)
\curveto(79.40209318,300.20723101)(79.40709317,300.14223107)(79.42710083,300.08223694)
\curveto(79.43709314,300.05223116)(79.44209314,300.0122312)(79.44210083,299.96223694)
\lineto(79.44210083,299.84223694)
\lineto(79.44210083,299.37723694)
\moveto(77.89710083,297.05223694)
\curveto(77.99709458,297.37223384)(78.05709452,297.73723348)(78.07710083,298.14723694)
\curveto(78.09709448,298.55723266)(78.10709447,298.96723225)(78.10710083,299.37723694)
\curveto(78.10709447,299.80723141)(78.09709448,300.22723099)(78.07710083,300.63723694)
\curveto(78.05709452,301.04723017)(78.01209457,301.43222978)(77.94210083,301.79223694)
\curveto(77.87209471,302.15222906)(77.76209482,302.47222874)(77.61210083,302.75223694)
\curveto(77.47209511,303.04222817)(77.2770953,303.27722794)(77.02710083,303.45723694)
\curveto(76.86709571,303.56722765)(76.68709589,303.64722757)(76.48710083,303.69723694)
\curveto(76.28709629,303.75722746)(76.04209654,303.78722743)(75.75210083,303.78723694)
\curveto(75.73209685,303.76722745)(75.69709688,303.75722746)(75.64710083,303.75723694)
\curveto(75.59709698,303.76722745)(75.55709702,303.76722745)(75.52710083,303.75723694)
\curveto(75.44709713,303.73722748)(75.37209721,303.7172275)(75.30210083,303.69723694)
\curveto(75.24209734,303.68722753)(75.1770974,303.66722755)(75.10710083,303.63723694)
\curveto(74.83709774,303.5172277)(74.61709796,303.34722787)(74.44710083,303.12723694)
\curveto(74.28709829,302.9172283)(74.15209843,302.67222854)(74.04210083,302.39223694)
\curveto(73.99209859,302.28222893)(73.95209863,302.16222905)(73.92210083,302.03223694)
\curveto(73.90209868,301.9122293)(73.8770987,301.78722943)(73.84710083,301.65723694)
\curveto(73.82709875,301.60722961)(73.81709876,301.55222966)(73.81710083,301.49223694)
\curveto(73.81709876,301.44222977)(73.81209877,301.39222982)(73.80210083,301.34223694)
\curveto(73.79209879,301.25222996)(73.7820988,301.15723006)(73.77210083,301.05723694)
\curveto(73.76209882,300.96723025)(73.75209883,300.87223034)(73.74210083,300.77223694)
\curveto(73.74209884,300.69223052)(73.73709884,300.60723061)(73.72710083,300.51723694)
\lineto(73.72710083,300.27723694)
\lineto(73.72710083,300.09723694)
\curveto(73.71709886,300.06723115)(73.71209887,300.03223118)(73.71210083,299.99223694)
\lineto(73.71210083,299.85723694)
\lineto(73.71210083,299.40723694)
\curveto(73.71209887,299.32723189)(73.70709887,299.24223197)(73.69710083,299.15223694)
\curveto(73.69709888,299.07223214)(73.70709887,298.99723222)(73.72710083,298.92723694)
\lineto(73.72710083,298.65723694)
\curveto(73.72709885,298.63723258)(73.72209886,298.60723261)(73.71210083,298.56723694)
\curveto(73.71209887,298.53723268)(73.71709886,298.5122327)(73.72710083,298.49223694)
\curveto(73.73709884,298.39223282)(73.74209884,298.29223292)(73.74210083,298.19223694)
\curveto(73.75209883,298.10223311)(73.76209882,298.00223321)(73.77210083,297.89223694)
\curveto(73.80209878,297.77223344)(73.81709876,297.64723357)(73.81710083,297.51723694)
\curveto(73.82709875,297.39723382)(73.85209873,297.28223393)(73.89210083,297.17223694)
\curveto(73.97209861,296.87223434)(74.05709852,296.60723461)(74.14710083,296.37723694)
\curveto(74.24709833,296.14723507)(74.39209819,295.93223528)(74.58210083,295.73223694)
\curveto(74.79209779,295.53223568)(75.05709752,295.38223583)(75.37710083,295.28223694)
\curveto(75.41709716,295.26223595)(75.45209713,295.25223596)(75.48210083,295.25223694)
\curveto(75.52209706,295.26223595)(75.56709701,295.25723596)(75.61710083,295.23723694)
\curveto(75.65709692,295.22723599)(75.72709685,295.217236)(75.82710083,295.20723694)
\curveto(75.93709664,295.19723602)(76.02209656,295.20223601)(76.08210083,295.22223694)
\curveto(76.15209643,295.24223597)(76.22209636,295.25223596)(76.29210083,295.25223694)
\curveto(76.36209622,295.26223595)(76.42709615,295.27723594)(76.48710083,295.29723694)
\curveto(76.68709589,295.35723586)(76.86709571,295.44223577)(77.02710083,295.55223694)
\curveto(77.05709552,295.57223564)(77.0820955,295.59223562)(77.10210083,295.61223694)
\lineto(77.16210083,295.67223694)
\curveto(77.20209538,295.69223552)(77.25209533,295.73223548)(77.31210083,295.79223694)
\curveto(77.41209517,295.93223528)(77.49709508,296.06223515)(77.56710083,296.18223694)
\curveto(77.63709494,296.30223491)(77.70709487,296.44723477)(77.77710083,296.61723694)
\curveto(77.80709477,296.68723453)(77.82709475,296.75723446)(77.83710083,296.82723694)
\curveto(77.85709472,296.89723432)(77.8770947,296.97223424)(77.89710083,297.05223694)
}
}
{
\newrgbcolor{curcolor}{0 0 0}
\pscustom[linestyle=none,fillstyle=solid,fillcolor=curcolor]
{
\newpath
\moveto(51.45327271,379.18295013)
\curveto(51.55326785,379.18293951)(51.64826776,379.17293952)(51.73827271,379.15295013)
\curveto(51.82826758,379.14293955)(51.89326751,379.11293958)(51.93327271,379.06295013)
\curveto(51.99326741,378.98293971)(52.02326738,378.87793982)(52.02327271,378.74795013)
\lineto(52.02327271,378.35795013)
\lineto(52.02327271,376.85795013)
\lineto(52.02327271,370.46795013)
\lineto(52.02327271,369.29795013)
\lineto(52.02327271,368.98295013)
\curveto(52.03326737,368.88294981)(52.01826739,368.80294989)(51.97827271,368.74295013)
\curveto(51.92826748,368.66295003)(51.85326755,368.61295008)(51.75327271,368.59295013)
\curveto(51.66326774,368.58295011)(51.55326785,368.57795012)(51.42327271,368.57795013)
\lineto(51.19827271,368.57795013)
\curveto(51.11826829,368.5979501)(51.04826836,368.61295008)(50.98827271,368.62295013)
\curveto(50.92826848,368.64295005)(50.87826853,368.68295001)(50.83827271,368.74295013)
\curveto(50.79826861,368.80294989)(50.77826863,368.87794982)(50.77827271,368.96795013)
\lineto(50.77827271,369.26795013)
\lineto(50.77827271,370.36295013)
\lineto(50.77827271,375.70295013)
\curveto(50.75826865,375.7929429)(50.74326866,375.86794283)(50.73327271,375.92795013)
\curveto(50.73326867,375.9979427)(50.7032687,376.05794264)(50.64327271,376.10795013)
\curveto(50.57326883,376.15794254)(50.48326892,376.18294251)(50.37327271,376.18295013)
\curveto(50.27326913,376.1929425)(50.16326924,376.1979425)(50.04327271,376.19795013)
\lineto(48.90327271,376.19795013)
\lineto(48.40827271,376.19795013)
\curveto(48.24827116,376.20794249)(48.13827127,376.26794243)(48.07827271,376.37795013)
\curveto(48.05827135,376.40794229)(48.04827136,376.43794226)(48.04827271,376.46795013)
\curveto(48.04827136,376.50794219)(48.04327136,376.55294214)(48.03327271,376.60295013)
\curveto(48.01327139,376.72294197)(48.01827139,376.83294186)(48.04827271,376.93295013)
\curveto(48.08827132,377.03294166)(48.14327126,377.10294159)(48.21327271,377.14295013)
\curveto(48.29327111,377.1929415)(48.41327099,377.21794148)(48.57327271,377.21795013)
\curveto(48.73327067,377.21794148)(48.86827054,377.23294146)(48.97827271,377.26295013)
\curveto(49.02827038,377.27294142)(49.08327032,377.27794142)(49.14327271,377.27795013)
\curveto(49.2032702,377.28794141)(49.26327014,377.30294139)(49.32327271,377.32295013)
\curveto(49.47326993,377.37294132)(49.61826979,377.42294127)(49.75827271,377.47295013)
\curveto(49.89826951,377.53294116)(50.03326937,377.60294109)(50.16327271,377.68295013)
\curveto(50.3032691,377.77294092)(50.42326898,377.87794082)(50.52327271,377.99795013)
\curveto(50.62326878,378.11794058)(50.71826869,378.24794045)(50.80827271,378.38795013)
\curveto(50.86826854,378.48794021)(50.91326849,378.5979401)(50.94327271,378.71795013)
\curveto(50.98326842,378.83793986)(51.03326837,378.94293975)(51.09327271,379.03295013)
\curveto(51.14326826,379.0929396)(51.21326819,379.13293956)(51.30327271,379.15295013)
\curveto(51.32326808,379.16293953)(51.34826806,379.16793953)(51.37827271,379.16795013)
\curveto(51.408268,379.16793953)(51.43326797,379.17293952)(51.45327271,379.18295013)
}
}
{
\newrgbcolor{curcolor}{0 0 0}
\pscustom[linestyle=none,fillstyle=solid,fillcolor=curcolor]
{
\newpath
\moveto(62.74288208,373.66295013)
\lineto(62.74288208,373.40795013)
\curveto(62.75287438,373.32794537)(62.74787438,373.25294544)(62.72788208,373.18295013)
\lineto(62.72788208,372.94295013)
\lineto(62.72788208,372.77795013)
\curveto(62.70787442,372.67794602)(62.69787443,372.57294612)(62.69788208,372.46295013)
\curveto(62.69787443,372.36294633)(62.68787444,372.26294643)(62.66788208,372.16295013)
\lineto(62.66788208,372.01295013)
\curveto(62.63787449,371.87294682)(62.61787451,371.73294696)(62.60788208,371.59295013)
\curveto(62.59787453,371.46294723)(62.57287456,371.33294736)(62.53288208,371.20295013)
\curveto(62.51287462,371.12294757)(62.49287464,371.03794766)(62.47288208,370.94795013)
\lineto(62.41288208,370.70795013)
\lineto(62.29288208,370.40795013)
\curveto(62.26287487,370.31794838)(62.2278749,370.22794847)(62.18788208,370.13795013)
\curveto(62.08787504,369.91794878)(61.95287518,369.70294899)(61.78288208,369.49295013)
\curveto(61.62287551,369.28294941)(61.44787568,369.11294958)(61.25788208,368.98295013)
\curveto(61.20787592,368.94294975)(61.14787598,368.90294979)(61.07788208,368.86295013)
\curveto(61.01787611,368.83294986)(60.95787617,368.7979499)(60.89788208,368.75795013)
\curveto(60.81787631,368.70794999)(60.72287641,368.66795003)(60.61288208,368.63795013)
\curveto(60.50287663,368.60795009)(60.39787673,368.57795012)(60.29788208,368.54795013)
\curveto(60.18787694,368.50795019)(60.07787705,368.48295021)(59.96788208,368.47295013)
\curveto(59.85787727,368.46295023)(59.74287739,368.44795025)(59.62288208,368.42795013)
\curveto(59.58287755,368.41795028)(59.53787759,368.41795028)(59.48788208,368.42795013)
\curveto(59.44787768,368.42795027)(59.40787772,368.42295027)(59.36788208,368.41295013)
\curveto(59.3278778,368.40295029)(59.27287786,368.3979503)(59.20288208,368.39795013)
\curveto(59.132878,368.3979503)(59.08287805,368.40295029)(59.05288208,368.41295013)
\curveto(59.00287813,368.43295026)(58.95787817,368.43795026)(58.91788208,368.42795013)
\curveto(58.87787825,368.41795028)(58.84287829,368.41795028)(58.81288208,368.42795013)
\lineto(58.72288208,368.42795013)
\curveto(58.66287847,368.44795025)(58.59787853,368.46295023)(58.52788208,368.47295013)
\curveto(58.46787866,368.47295022)(58.40287873,368.47795022)(58.33288208,368.48795013)
\curveto(58.16287897,368.53795016)(58.00287913,368.58795011)(57.85288208,368.63795013)
\curveto(57.70287943,368.68795001)(57.55787957,368.75294994)(57.41788208,368.83295013)
\curveto(57.36787976,368.87294982)(57.31287982,368.90294979)(57.25288208,368.92295013)
\curveto(57.20287993,368.95294974)(57.15287998,368.98794971)(57.10288208,369.02795013)
\curveto(56.86288027,369.20794949)(56.66288047,369.42794927)(56.50288208,369.68795013)
\curveto(56.34288079,369.94794875)(56.20288093,370.23294846)(56.08288208,370.54295013)
\curveto(56.02288111,370.68294801)(55.97788115,370.82294787)(55.94788208,370.96295013)
\curveto(55.91788121,371.11294758)(55.88288125,371.26794743)(55.84288208,371.42795013)
\curveto(55.82288131,371.53794716)(55.80788132,371.64794705)(55.79788208,371.75795013)
\curveto(55.78788134,371.86794683)(55.77288136,371.97794672)(55.75288208,372.08795013)
\curveto(55.74288139,372.12794657)(55.73788139,372.16794653)(55.73788208,372.20795013)
\curveto(55.74788138,372.24794645)(55.74788138,372.28794641)(55.73788208,372.32795013)
\curveto(55.7278814,372.37794632)(55.72288141,372.42794627)(55.72288208,372.47795013)
\lineto(55.72288208,372.64295013)
\curveto(55.70288143,372.692946)(55.69788143,372.74294595)(55.70788208,372.79295013)
\curveto(55.71788141,372.85294584)(55.71788141,372.90794579)(55.70788208,372.95795013)
\curveto(55.69788143,372.9979457)(55.69788143,373.04294565)(55.70788208,373.09295013)
\curveto(55.71788141,373.14294555)(55.71288142,373.1929455)(55.69288208,373.24295013)
\curveto(55.67288146,373.31294538)(55.66788146,373.38794531)(55.67788208,373.46795013)
\curveto(55.68788144,373.55794514)(55.69288144,373.64294505)(55.69288208,373.72295013)
\curveto(55.69288144,373.81294488)(55.68788144,373.91294478)(55.67788208,374.02295013)
\curveto(55.66788146,374.14294455)(55.67288146,374.24294445)(55.69288208,374.32295013)
\lineto(55.69288208,374.60795013)
\lineto(55.73788208,375.23795013)
\curveto(55.74788138,375.33794336)(55.75788137,375.43294326)(55.76788208,375.52295013)
\lineto(55.79788208,375.82295013)
\curveto(55.81788131,375.87294282)(55.82288131,375.92294277)(55.81288208,375.97295013)
\curveto(55.81288132,376.03294266)(55.82288131,376.08794261)(55.84288208,376.13795013)
\curveto(55.89288124,376.30794239)(55.9328812,376.47294222)(55.96288208,376.63295013)
\curveto(55.99288114,376.80294189)(56.04288109,376.96294173)(56.11288208,377.11295013)
\curveto(56.30288083,377.57294112)(56.52288061,377.94794075)(56.77288208,378.23795013)
\curveto(57.0328801,378.52794017)(57.39287974,378.77293992)(57.85288208,378.97295013)
\curveto(57.98287915,379.02293967)(58.11287902,379.05793964)(58.24288208,379.07795013)
\curveto(58.38287875,379.0979396)(58.52287861,379.12293957)(58.66288208,379.15295013)
\curveto(58.7328784,379.16293953)(58.79787833,379.16793953)(58.85788208,379.16795013)
\curveto(58.91787821,379.16793953)(58.98287815,379.17293952)(59.05288208,379.18295013)
\curveto(59.88287725,379.20293949)(60.55287658,379.05293964)(61.06288208,378.73295013)
\curveto(61.57287556,378.42294027)(61.95287518,377.98294071)(62.20288208,377.41295013)
\curveto(62.25287488,377.2929414)(62.29787483,377.16794153)(62.33788208,377.03795013)
\curveto(62.37787475,376.90794179)(62.42287471,376.77294192)(62.47288208,376.63295013)
\curveto(62.49287464,376.55294214)(62.50787462,376.46794223)(62.51788208,376.37795013)
\lineto(62.57788208,376.13795013)
\curveto(62.60787452,376.02794267)(62.62287451,375.91794278)(62.62288208,375.80795013)
\curveto(62.6328745,375.697943)(62.64787448,375.58794311)(62.66788208,375.47795013)
\curveto(62.68787444,375.42794327)(62.69287444,375.38294331)(62.68288208,375.34295013)
\curveto(62.68287445,375.30294339)(62.68787444,375.26294343)(62.69788208,375.22295013)
\curveto(62.70787442,375.17294352)(62.70787442,375.11794358)(62.69788208,375.05795013)
\curveto(62.69787443,375.00794369)(62.70287443,374.95794374)(62.71288208,374.90795013)
\lineto(62.71288208,374.77295013)
\curveto(62.7328744,374.71294398)(62.7328744,374.64294405)(62.71288208,374.56295013)
\curveto(62.70287443,374.4929442)(62.70787442,374.42794427)(62.72788208,374.36795013)
\curveto(62.73787439,374.33794436)(62.74287439,374.2979444)(62.74288208,374.24795013)
\lineto(62.74288208,374.12795013)
\lineto(62.74288208,373.66295013)
\moveto(61.19788208,371.33795013)
\curveto(61.29787583,371.65794704)(61.35787577,372.02294667)(61.37788208,372.43295013)
\curveto(61.39787573,372.84294585)(61.40787572,373.25294544)(61.40788208,373.66295013)
\curveto(61.40787572,374.0929446)(61.39787573,374.51294418)(61.37788208,374.92295013)
\curveto(61.35787577,375.33294336)(61.31287582,375.71794298)(61.24288208,376.07795013)
\curveto(61.17287596,376.43794226)(61.06287607,376.75794194)(60.91288208,377.03795013)
\curveto(60.77287636,377.32794137)(60.57787655,377.56294113)(60.32788208,377.74295013)
\curveto(60.16787696,377.85294084)(59.98787714,377.93294076)(59.78788208,377.98295013)
\curveto(59.58787754,378.04294065)(59.34287779,378.07294062)(59.05288208,378.07295013)
\curveto(59.0328781,378.05294064)(58.99787813,378.04294065)(58.94788208,378.04295013)
\curveto(58.89787823,378.05294064)(58.85787827,378.05294064)(58.82788208,378.04295013)
\curveto(58.74787838,378.02294067)(58.67287846,378.00294069)(58.60288208,377.98295013)
\curveto(58.54287859,377.97294072)(58.47787865,377.95294074)(58.40788208,377.92295013)
\curveto(58.13787899,377.80294089)(57.91787921,377.63294106)(57.74788208,377.41295013)
\curveto(57.58787954,377.20294149)(57.45287968,376.95794174)(57.34288208,376.67795013)
\curveto(57.29287984,376.56794213)(57.25287988,376.44794225)(57.22288208,376.31795013)
\curveto(57.20287993,376.1979425)(57.17787995,376.07294262)(57.14788208,375.94295013)
\curveto(57.12788,375.8929428)(57.11788001,375.83794286)(57.11788208,375.77795013)
\curveto(57.11788001,375.72794297)(57.11288002,375.67794302)(57.10288208,375.62795013)
\curveto(57.09288004,375.53794316)(57.08288005,375.44294325)(57.07288208,375.34295013)
\curveto(57.06288007,375.25294344)(57.05288008,375.15794354)(57.04288208,375.05795013)
\curveto(57.04288009,374.97794372)(57.03788009,374.8929438)(57.02788208,374.80295013)
\lineto(57.02788208,374.56295013)
\lineto(57.02788208,374.38295013)
\curveto(57.01788011,374.35294434)(57.01288012,374.31794438)(57.01288208,374.27795013)
\lineto(57.01288208,374.14295013)
\lineto(57.01288208,373.69295013)
\curveto(57.01288012,373.61294508)(57.00788012,373.52794517)(56.99788208,373.43795013)
\curveto(56.99788013,373.35794534)(57.00788012,373.28294541)(57.02788208,373.21295013)
\lineto(57.02788208,372.94295013)
\curveto(57.0278801,372.92294577)(57.02288011,372.8929458)(57.01288208,372.85295013)
\curveto(57.01288012,372.82294587)(57.01788011,372.7979459)(57.02788208,372.77795013)
\curveto(57.03788009,372.67794602)(57.04288009,372.57794612)(57.04288208,372.47795013)
\curveto(57.05288008,372.38794631)(57.06288007,372.28794641)(57.07288208,372.17795013)
\curveto(57.10288003,372.05794664)(57.11788001,371.93294676)(57.11788208,371.80295013)
\curveto(57.12788,371.68294701)(57.15287998,371.56794713)(57.19288208,371.45795013)
\curveto(57.27287986,371.15794754)(57.35787977,370.8929478)(57.44788208,370.66295013)
\curveto(57.54787958,370.43294826)(57.69287944,370.21794848)(57.88288208,370.01795013)
\curveto(58.09287904,369.81794888)(58.35787877,369.66794903)(58.67788208,369.56795013)
\curveto(58.71787841,369.54794915)(58.75287838,369.53794916)(58.78288208,369.53795013)
\curveto(58.82287831,369.54794915)(58.86787826,369.54294915)(58.91788208,369.52295013)
\curveto(58.95787817,369.51294918)(59.0278781,369.50294919)(59.12788208,369.49295013)
\curveto(59.23787789,369.48294921)(59.32287781,369.48794921)(59.38288208,369.50795013)
\curveto(59.45287768,369.52794917)(59.52287761,369.53794916)(59.59288208,369.53795013)
\curveto(59.66287747,369.54794915)(59.7278774,369.56294913)(59.78788208,369.58295013)
\curveto(59.98787714,369.64294905)(60.16787696,369.72794897)(60.32788208,369.83795013)
\curveto(60.35787677,369.85794884)(60.38287675,369.87794882)(60.40288208,369.89795013)
\lineto(60.46288208,369.95795013)
\curveto(60.50287663,369.97794872)(60.55287658,370.01794868)(60.61288208,370.07795013)
\curveto(60.71287642,370.21794848)(60.79787633,370.34794835)(60.86788208,370.46795013)
\curveto(60.93787619,370.58794811)(61.00787612,370.73294796)(61.07788208,370.90295013)
\curveto(61.10787602,370.97294772)(61.127876,371.04294765)(61.13788208,371.11295013)
\curveto(61.15787597,371.18294751)(61.17787595,371.25794744)(61.19788208,371.33795013)
}
}
{
\newrgbcolor{curcolor}{0 0 0}
\pscustom[linestyle=none,fillstyle=solid,fillcolor=curcolor]
{
\newpath
\moveto(71.09249146,373.66295013)
\lineto(71.09249146,373.40795013)
\curveto(71.10248375,373.32794537)(71.09748376,373.25294544)(71.07749146,373.18295013)
\lineto(71.07749146,372.94295013)
\lineto(71.07749146,372.77795013)
\curveto(71.0574838,372.67794602)(71.04748381,372.57294612)(71.04749146,372.46295013)
\curveto(71.04748381,372.36294633)(71.03748382,372.26294643)(71.01749146,372.16295013)
\lineto(71.01749146,372.01295013)
\curveto(70.98748387,371.87294682)(70.96748389,371.73294696)(70.95749146,371.59295013)
\curveto(70.94748391,371.46294723)(70.92248393,371.33294736)(70.88249146,371.20295013)
\curveto(70.86248399,371.12294757)(70.84248401,371.03794766)(70.82249146,370.94795013)
\lineto(70.76249146,370.70795013)
\lineto(70.64249146,370.40795013)
\curveto(70.61248424,370.31794838)(70.57748428,370.22794847)(70.53749146,370.13795013)
\curveto(70.43748442,369.91794878)(70.30248455,369.70294899)(70.13249146,369.49295013)
\curveto(69.97248488,369.28294941)(69.79748506,369.11294958)(69.60749146,368.98295013)
\curveto(69.5574853,368.94294975)(69.49748536,368.90294979)(69.42749146,368.86295013)
\curveto(69.36748549,368.83294986)(69.30748555,368.7979499)(69.24749146,368.75795013)
\curveto(69.16748569,368.70794999)(69.07248578,368.66795003)(68.96249146,368.63795013)
\curveto(68.852486,368.60795009)(68.74748611,368.57795012)(68.64749146,368.54795013)
\curveto(68.53748632,368.50795019)(68.42748643,368.48295021)(68.31749146,368.47295013)
\curveto(68.20748665,368.46295023)(68.09248676,368.44795025)(67.97249146,368.42795013)
\curveto(67.93248692,368.41795028)(67.88748697,368.41795028)(67.83749146,368.42795013)
\curveto(67.79748706,368.42795027)(67.7574871,368.42295027)(67.71749146,368.41295013)
\curveto(67.67748718,368.40295029)(67.62248723,368.3979503)(67.55249146,368.39795013)
\curveto(67.48248737,368.3979503)(67.43248742,368.40295029)(67.40249146,368.41295013)
\curveto(67.3524875,368.43295026)(67.30748755,368.43795026)(67.26749146,368.42795013)
\curveto(67.22748763,368.41795028)(67.19248766,368.41795028)(67.16249146,368.42795013)
\lineto(67.07249146,368.42795013)
\curveto(67.01248784,368.44795025)(66.94748791,368.46295023)(66.87749146,368.47295013)
\curveto(66.81748804,368.47295022)(66.7524881,368.47795022)(66.68249146,368.48795013)
\curveto(66.51248834,368.53795016)(66.3524885,368.58795011)(66.20249146,368.63795013)
\curveto(66.0524888,368.68795001)(65.90748895,368.75294994)(65.76749146,368.83295013)
\curveto(65.71748914,368.87294982)(65.66248919,368.90294979)(65.60249146,368.92295013)
\curveto(65.5524893,368.95294974)(65.50248935,368.98794971)(65.45249146,369.02795013)
\curveto(65.21248964,369.20794949)(65.01248984,369.42794927)(64.85249146,369.68795013)
\curveto(64.69249016,369.94794875)(64.5524903,370.23294846)(64.43249146,370.54295013)
\curveto(64.37249048,370.68294801)(64.32749053,370.82294787)(64.29749146,370.96295013)
\curveto(64.26749059,371.11294758)(64.23249062,371.26794743)(64.19249146,371.42795013)
\curveto(64.17249068,371.53794716)(64.1574907,371.64794705)(64.14749146,371.75795013)
\curveto(64.13749072,371.86794683)(64.12249073,371.97794672)(64.10249146,372.08795013)
\curveto(64.09249076,372.12794657)(64.08749077,372.16794653)(64.08749146,372.20795013)
\curveto(64.09749076,372.24794645)(64.09749076,372.28794641)(64.08749146,372.32795013)
\curveto(64.07749078,372.37794632)(64.07249078,372.42794627)(64.07249146,372.47795013)
\lineto(64.07249146,372.64295013)
\curveto(64.0524908,372.692946)(64.04749081,372.74294595)(64.05749146,372.79295013)
\curveto(64.06749079,372.85294584)(64.06749079,372.90794579)(64.05749146,372.95795013)
\curveto(64.04749081,372.9979457)(64.04749081,373.04294565)(64.05749146,373.09295013)
\curveto(64.06749079,373.14294555)(64.06249079,373.1929455)(64.04249146,373.24295013)
\curveto(64.02249083,373.31294538)(64.01749084,373.38794531)(64.02749146,373.46795013)
\curveto(64.03749082,373.55794514)(64.04249081,373.64294505)(64.04249146,373.72295013)
\curveto(64.04249081,373.81294488)(64.03749082,373.91294478)(64.02749146,374.02295013)
\curveto(64.01749084,374.14294455)(64.02249083,374.24294445)(64.04249146,374.32295013)
\lineto(64.04249146,374.60795013)
\lineto(64.08749146,375.23795013)
\curveto(64.09749076,375.33794336)(64.10749075,375.43294326)(64.11749146,375.52295013)
\lineto(64.14749146,375.82295013)
\curveto(64.16749069,375.87294282)(64.17249068,375.92294277)(64.16249146,375.97295013)
\curveto(64.16249069,376.03294266)(64.17249068,376.08794261)(64.19249146,376.13795013)
\curveto(64.24249061,376.30794239)(64.28249057,376.47294222)(64.31249146,376.63295013)
\curveto(64.34249051,376.80294189)(64.39249046,376.96294173)(64.46249146,377.11295013)
\curveto(64.6524902,377.57294112)(64.87248998,377.94794075)(65.12249146,378.23795013)
\curveto(65.38248947,378.52794017)(65.74248911,378.77293992)(66.20249146,378.97295013)
\curveto(66.33248852,379.02293967)(66.46248839,379.05793964)(66.59249146,379.07795013)
\curveto(66.73248812,379.0979396)(66.87248798,379.12293957)(67.01249146,379.15295013)
\curveto(67.08248777,379.16293953)(67.14748771,379.16793953)(67.20749146,379.16795013)
\curveto(67.26748759,379.16793953)(67.33248752,379.17293952)(67.40249146,379.18295013)
\curveto(68.23248662,379.20293949)(68.90248595,379.05293964)(69.41249146,378.73295013)
\curveto(69.92248493,378.42294027)(70.30248455,377.98294071)(70.55249146,377.41295013)
\curveto(70.60248425,377.2929414)(70.64748421,377.16794153)(70.68749146,377.03795013)
\curveto(70.72748413,376.90794179)(70.77248408,376.77294192)(70.82249146,376.63295013)
\curveto(70.84248401,376.55294214)(70.857484,376.46794223)(70.86749146,376.37795013)
\lineto(70.92749146,376.13795013)
\curveto(70.9574839,376.02794267)(70.97248388,375.91794278)(70.97249146,375.80795013)
\curveto(70.98248387,375.697943)(70.99748386,375.58794311)(71.01749146,375.47795013)
\curveto(71.03748382,375.42794327)(71.04248381,375.38294331)(71.03249146,375.34295013)
\curveto(71.03248382,375.30294339)(71.03748382,375.26294343)(71.04749146,375.22295013)
\curveto(71.0574838,375.17294352)(71.0574838,375.11794358)(71.04749146,375.05795013)
\curveto(71.04748381,375.00794369)(71.0524838,374.95794374)(71.06249146,374.90795013)
\lineto(71.06249146,374.77295013)
\curveto(71.08248377,374.71294398)(71.08248377,374.64294405)(71.06249146,374.56295013)
\curveto(71.0524838,374.4929442)(71.0574838,374.42794427)(71.07749146,374.36795013)
\curveto(71.08748377,374.33794436)(71.09248376,374.2979444)(71.09249146,374.24795013)
\lineto(71.09249146,374.12795013)
\lineto(71.09249146,373.66295013)
\moveto(69.54749146,371.33795013)
\curveto(69.64748521,371.65794704)(69.70748515,372.02294667)(69.72749146,372.43295013)
\curveto(69.74748511,372.84294585)(69.7574851,373.25294544)(69.75749146,373.66295013)
\curveto(69.7574851,374.0929446)(69.74748511,374.51294418)(69.72749146,374.92295013)
\curveto(69.70748515,375.33294336)(69.66248519,375.71794298)(69.59249146,376.07795013)
\curveto(69.52248533,376.43794226)(69.41248544,376.75794194)(69.26249146,377.03795013)
\curveto(69.12248573,377.32794137)(68.92748593,377.56294113)(68.67749146,377.74295013)
\curveto(68.51748634,377.85294084)(68.33748652,377.93294076)(68.13749146,377.98295013)
\curveto(67.93748692,378.04294065)(67.69248716,378.07294062)(67.40249146,378.07295013)
\curveto(67.38248747,378.05294064)(67.34748751,378.04294065)(67.29749146,378.04295013)
\curveto(67.24748761,378.05294064)(67.20748765,378.05294064)(67.17749146,378.04295013)
\curveto(67.09748776,378.02294067)(67.02248783,378.00294069)(66.95249146,377.98295013)
\curveto(66.89248796,377.97294072)(66.82748803,377.95294074)(66.75749146,377.92295013)
\curveto(66.48748837,377.80294089)(66.26748859,377.63294106)(66.09749146,377.41295013)
\curveto(65.93748892,377.20294149)(65.80248905,376.95794174)(65.69249146,376.67795013)
\curveto(65.64248921,376.56794213)(65.60248925,376.44794225)(65.57249146,376.31795013)
\curveto(65.5524893,376.1979425)(65.52748933,376.07294262)(65.49749146,375.94295013)
\curveto(65.47748938,375.8929428)(65.46748939,375.83794286)(65.46749146,375.77795013)
\curveto(65.46748939,375.72794297)(65.46248939,375.67794302)(65.45249146,375.62795013)
\curveto(65.44248941,375.53794316)(65.43248942,375.44294325)(65.42249146,375.34295013)
\curveto(65.41248944,375.25294344)(65.40248945,375.15794354)(65.39249146,375.05795013)
\curveto(65.39248946,374.97794372)(65.38748947,374.8929438)(65.37749146,374.80295013)
\lineto(65.37749146,374.56295013)
\lineto(65.37749146,374.38295013)
\curveto(65.36748949,374.35294434)(65.36248949,374.31794438)(65.36249146,374.27795013)
\lineto(65.36249146,374.14295013)
\lineto(65.36249146,373.69295013)
\curveto(65.36248949,373.61294508)(65.3574895,373.52794517)(65.34749146,373.43795013)
\curveto(65.34748951,373.35794534)(65.3574895,373.28294541)(65.37749146,373.21295013)
\lineto(65.37749146,372.94295013)
\curveto(65.37748948,372.92294577)(65.37248948,372.8929458)(65.36249146,372.85295013)
\curveto(65.36248949,372.82294587)(65.36748949,372.7979459)(65.37749146,372.77795013)
\curveto(65.38748947,372.67794602)(65.39248946,372.57794612)(65.39249146,372.47795013)
\curveto(65.40248945,372.38794631)(65.41248944,372.28794641)(65.42249146,372.17795013)
\curveto(65.4524894,372.05794664)(65.46748939,371.93294676)(65.46749146,371.80295013)
\curveto(65.47748938,371.68294701)(65.50248935,371.56794713)(65.54249146,371.45795013)
\curveto(65.62248923,371.15794754)(65.70748915,370.8929478)(65.79749146,370.66295013)
\curveto(65.89748896,370.43294826)(66.04248881,370.21794848)(66.23249146,370.01795013)
\curveto(66.44248841,369.81794888)(66.70748815,369.66794903)(67.02749146,369.56795013)
\curveto(67.06748779,369.54794915)(67.10248775,369.53794916)(67.13249146,369.53795013)
\curveto(67.17248768,369.54794915)(67.21748764,369.54294915)(67.26749146,369.52295013)
\curveto(67.30748755,369.51294918)(67.37748748,369.50294919)(67.47749146,369.49295013)
\curveto(67.58748727,369.48294921)(67.67248718,369.48794921)(67.73249146,369.50795013)
\curveto(67.80248705,369.52794917)(67.87248698,369.53794916)(67.94249146,369.53795013)
\curveto(68.01248684,369.54794915)(68.07748678,369.56294913)(68.13749146,369.58295013)
\curveto(68.33748652,369.64294905)(68.51748634,369.72794897)(68.67749146,369.83795013)
\curveto(68.70748615,369.85794884)(68.73248612,369.87794882)(68.75249146,369.89795013)
\lineto(68.81249146,369.95795013)
\curveto(68.852486,369.97794872)(68.90248595,370.01794868)(68.96249146,370.07795013)
\curveto(69.06248579,370.21794848)(69.14748571,370.34794835)(69.21749146,370.46795013)
\curveto(69.28748557,370.58794811)(69.3574855,370.73294796)(69.42749146,370.90295013)
\curveto(69.4574854,370.97294772)(69.47748538,371.04294765)(69.48749146,371.11295013)
\curveto(69.50748535,371.18294751)(69.52748533,371.25794744)(69.54749146,371.33795013)
}
}
{
\newrgbcolor{curcolor}{0 0 0}
\pscustom[linestyle=none,fillstyle=solid,fillcolor=curcolor]
{
\newpath
\moveto(79.44210083,373.66295013)
\lineto(79.44210083,373.40795013)
\curveto(79.45209313,373.32794537)(79.44709313,373.25294544)(79.42710083,373.18295013)
\lineto(79.42710083,372.94295013)
\lineto(79.42710083,372.77795013)
\curveto(79.40709317,372.67794602)(79.39709318,372.57294612)(79.39710083,372.46295013)
\curveto(79.39709318,372.36294633)(79.38709319,372.26294643)(79.36710083,372.16295013)
\lineto(79.36710083,372.01295013)
\curveto(79.33709324,371.87294682)(79.31709326,371.73294696)(79.30710083,371.59295013)
\curveto(79.29709328,371.46294723)(79.27209331,371.33294736)(79.23210083,371.20295013)
\curveto(79.21209337,371.12294757)(79.19209339,371.03794766)(79.17210083,370.94795013)
\lineto(79.11210083,370.70795013)
\lineto(78.99210083,370.40795013)
\curveto(78.96209362,370.31794838)(78.92709365,370.22794847)(78.88710083,370.13795013)
\curveto(78.78709379,369.91794878)(78.65209393,369.70294899)(78.48210083,369.49295013)
\curveto(78.32209426,369.28294941)(78.14709443,369.11294958)(77.95710083,368.98295013)
\curveto(77.90709467,368.94294975)(77.84709473,368.90294979)(77.77710083,368.86295013)
\curveto(77.71709486,368.83294986)(77.65709492,368.7979499)(77.59710083,368.75795013)
\curveto(77.51709506,368.70794999)(77.42209516,368.66795003)(77.31210083,368.63795013)
\curveto(77.20209538,368.60795009)(77.09709548,368.57795012)(76.99710083,368.54795013)
\curveto(76.88709569,368.50795019)(76.7770958,368.48295021)(76.66710083,368.47295013)
\curveto(76.55709602,368.46295023)(76.44209614,368.44795025)(76.32210083,368.42795013)
\curveto(76.2820963,368.41795028)(76.23709634,368.41795028)(76.18710083,368.42795013)
\curveto(76.14709643,368.42795027)(76.10709647,368.42295027)(76.06710083,368.41295013)
\curveto(76.02709655,368.40295029)(75.97209661,368.3979503)(75.90210083,368.39795013)
\curveto(75.83209675,368.3979503)(75.7820968,368.40295029)(75.75210083,368.41295013)
\curveto(75.70209688,368.43295026)(75.65709692,368.43795026)(75.61710083,368.42795013)
\curveto(75.577097,368.41795028)(75.54209704,368.41795028)(75.51210083,368.42795013)
\lineto(75.42210083,368.42795013)
\curveto(75.36209722,368.44795025)(75.29709728,368.46295023)(75.22710083,368.47295013)
\curveto(75.16709741,368.47295022)(75.10209748,368.47795022)(75.03210083,368.48795013)
\curveto(74.86209772,368.53795016)(74.70209788,368.58795011)(74.55210083,368.63795013)
\curveto(74.40209818,368.68795001)(74.25709832,368.75294994)(74.11710083,368.83295013)
\curveto(74.06709851,368.87294982)(74.01209857,368.90294979)(73.95210083,368.92295013)
\curveto(73.90209868,368.95294974)(73.85209873,368.98794971)(73.80210083,369.02795013)
\curveto(73.56209902,369.20794949)(73.36209922,369.42794927)(73.20210083,369.68795013)
\curveto(73.04209954,369.94794875)(72.90209968,370.23294846)(72.78210083,370.54295013)
\curveto(72.72209986,370.68294801)(72.6770999,370.82294787)(72.64710083,370.96295013)
\curveto(72.61709996,371.11294758)(72.5821,371.26794743)(72.54210083,371.42795013)
\curveto(72.52210006,371.53794716)(72.50710007,371.64794705)(72.49710083,371.75795013)
\curveto(72.48710009,371.86794683)(72.47210011,371.97794672)(72.45210083,372.08795013)
\curveto(72.44210014,372.12794657)(72.43710014,372.16794653)(72.43710083,372.20795013)
\curveto(72.44710013,372.24794645)(72.44710013,372.28794641)(72.43710083,372.32795013)
\curveto(72.42710015,372.37794632)(72.42210016,372.42794627)(72.42210083,372.47795013)
\lineto(72.42210083,372.64295013)
\curveto(72.40210018,372.692946)(72.39710018,372.74294595)(72.40710083,372.79295013)
\curveto(72.41710016,372.85294584)(72.41710016,372.90794579)(72.40710083,372.95795013)
\curveto(72.39710018,372.9979457)(72.39710018,373.04294565)(72.40710083,373.09295013)
\curveto(72.41710016,373.14294555)(72.41210017,373.1929455)(72.39210083,373.24295013)
\curveto(72.37210021,373.31294538)(72.36710021,373.38794531)(72.37710083,373.46795013)
\curveto(72.38710019,373.55794514)(72.39210019,373.64294505)(72.39210083,373.72295013)
\curveto(72.39210019,373.81294488)(72.38710019,373.91294478)(72.37710083,374.02295013)
\curveto(72.36710021,374.14294455)(72.37210021,374.24294445)(72.39210083,374.32295013)
\lineto(72.39210083,374.60795013)
\lineto(72.43710083,375.23795013)
\curveto(72.44710013,375.33794336)(72.45710012,375.43294326)(72.46710083,375.52295013)
\lineto(72.49710083,375.82295013)
\curveto(72.51710006,375.87294282)(72.52210006,375.92294277)(72.51210083,375.97295013)
\curveto(72.51210007,376.03294266)(72.52210006,376.08794261)(72.54210083,376.13795013)
\curveto(72.59209999,376.30794239)(72.63209995,376.47294222)(72.66210083,376.63295013)
\curveto(72.69209989,376.80294189)(72.74209984,376.96294173)(72.81210083,377.11295013)
\curveto(73.00209958,377.57294112)(73.22209936,377.94794075)(73.47210083,378.23795013)
\curveto(73.73209885,378.52794017)(74.09209849,378.77293992)(74.55210083,378.97295013)
\curveto(74.6820979,379.02293967)(74.81209777,379.05793964)(74.94210083,379.07795013)
\curveto(75.0820975,379.0979396)(75.22209736,379.12293957)(75.36210083,379.15295013)
\curveto(75.43209715,379.16293953)(75.49709708,379.16793953)(75.55710083,379.16795013)
\curveto(75.61709696,379.16793953)(75.6820969,379.17293952)(75.75210083,379.18295013)
\curveto(76.582096,379.20293949)(77.25209533,379.05293964)(77.76210083,378.73295013)
\curveto(78.27209431,378.42294027)(78.65209393,377.98294071)(78.90210083,377.41295013)
\curveto(78.95209363,377.2929414)(78.99709358,377.16794153)(79.03710083,377.03795013)
\curveto(79.0770935,376.90794179)(79.12209346,376.77294192)(79.17210083,376.63295013)
\curveto(79.19209339,376.55294214)(79.20709337,376.46794223)(79.21710083,376.37795013)
\lineto(79.27710083,376.13795013)
\curveto(79.30709327,376.02794267)(79.32209326,375.91794278)(79.32210083,375.80795013)
\curveto(79.33209325,375.697943)(79.34709323,375.58794311)(79.36710083,375.47795013)
\curveto(79.38709319,375.42794327)(79.39209319,375.38294331)(79.38210083,375.34295013)
\curveto(79.3820932,375.30294339)(79.38709319,375.26294343)(79.39710083,375.22295013)
\curveto(79.40709317,375.17294352)(79.40709317,375.11794358)(79.39710083,375.05795013)
\curveto(79.39709318,375.00794369)(79.40209318,374.95794374)(79.41210083,374.90795013)
\lineto(79.41210083,374.77295013)
\curveto(79.43209315,374.71294398)(79.43209315,374.64294405)(79.41210083,374.56295013)
\curveto(79.40209318,374.4929442)(79.40709317,374.42794427)(79.42710083,374.36795013)
\curveto(79.43709314,374.33794436)(79.44209314,374.2979444)(79.44210083,374.24795013)
\lineto(79.44210083,374.12795013)
\lineto(79.44210083,373.66295013)
\moveto(77.89710083,371.33795013)
\curveto(77.99709458,371.65794704)(78.05709452,372.02294667)(78.07710083,372.43295013)
\curveto(78.09709448,372.84294585)(78.10709447,373.25294544)(78.10710083,373.66295013)
\curveto(78.10709447,374.0929446)(78.09709448,374.51294418)(78.07710083,374.92295013)
\curveto(78.05709452,375.33294336)(78.01209457,375.71794298)(77.94210083,376.07795013)
\curveto(77.87209471,376.43794226)(77.76209482,376.75794194)(77.61210083,377.03795013)
\curveto(77.47209511,377.32794137)(77.2770953,377.56294113)(77.02710083,377.74295013)
\curveto(76.86709571,377.85294084)(76.68709589,377.93294076)(76.48710083,377.98295013)
\curveto(76.28709629,378.04294065)(76.04209654,378.07294062)(75.75210083,378.07295013)
\curveto(75.73209685,378.05294064)(75.69709688,378.04294065)(75.64710083,378.04295013)
\curveto(75.59709698,378.05294064)(75.55709702,378.05294064)(75.52710083,378.04295013)
\curveto(75.44709713,378.02294067)(75.37209721,378.00294069)(75.30210083,377.98295013)
\curveto(75.24209734,377.97294072)(75.1770974,377.95294074)(75.10710083,377.92295013)
\curveto(74.83709774,377.80294089)(74.61709796,377.63294106)(74.44710083,377.41295013)
\curveto(74.28709829,377.20294149)(74.15209843,376.95794174)(74.04210083,376.67795013)
\curveto(73.99209859,376.56794213)(73.95209863,376.44794225)(73.92210083,376.31795013)
\curveto(73.90209868,376.1979425)(73.8770987,376.07294262)(73.84710083,375.94295013)
\curveto(73.82709875,375.8929428)(73.81709876,375.83794286)(73.81710083,375.77795013)
\curveto(73.81709876,375.72794297)(73.81209877,375.67794302)(73.80210083,375.62795013)
\curveto(73.79209879,375.53794316)(73.7820988,375.44294325)(73.77210083,375.34295013)
\curveto(73.76209882,375.25294344)(73.75209883,375.15794354)(73.74210083,375.05795013)
\curveto(73.74209884,374.97794372)(73.73709884,374.8929438)(73.72710083,374.80295013)
\lineto(73.72710083,374.56295013)
\lineto(73.72710083,374.38295013)
\curveto(73.71709886,374.35294434)(73.71209887,374.31794438)(73.71210083,374.27795013)
\lineto(73.71210083,374.14295013)
\lineto(73.71210083,373.69295013)
\curveto(73.71209887,373.61294508)(73.70709887,373.52794517)(73.69710083,373.43795013)
\curveto(73.69709888,373.35794534)(73.70709887,373.28294541)(73.72710083,373.21295013)
\lineto(73.72710083,372.94295013)
\curveto(73.72709885,372.92294577)(73.72209886,372.8929458)(73.71210083,372.85295013)
\curveto(73.71209887,372.82294587)(73.71709886,372.7979459)(73.72710083,372.77795013)
\curveto(73.73709884,372.67794602)(73.74209884,372.57794612)(73.74210083,372.47795013)
\curveto(73.75209883,372.38794631)(73.76209882,372.28794641)(73.77210083,372.17795013)
\curveto(73.80209878,372.05794664)(73.81709876,371.93294676)(73.81710083,371.80295013)
\curveto(73.82709875,371.68294701)(73.85209873,371.56794713)(73.89210083,371.45795013)
\curveto(73.97209861,371.15794754)(74.05709852,370.8929478)(74.14710083,370.66295013)
\curveto(74.24709833,370.43294826)(74.39209819,370.21794848)(74.58210083,370.01795013)
\curveto(74.79209779,369.81794888)(75.05709752,369.66794903)(75.37710083,369.56795013)
\curveto(75.41709716,369.54794915)(75.45209713,369.53794916)(75.48210083,369.53795013)
\curveto(75.52209706,369.54794915)(75.56709701,369.54294915)(75.61710083,369.52295013)
\curveto(75.65709692,369.51294918)(75.72709685,369.50294919)(75.82710083,369.49295013)
\curveto(75.93709664,369.48294921)(76.02209656,369.48794921)(76.08210083,369.50795013)
\curveto(76.15209643,369.52794917)(76.22209636,369.53794916)(76.29210083,369.53795013)
\curveto(76.36209622,369.54794915)(76.42709615,369.56294913)(76.48710083,369.58295013)
\curveto(76.68709589,369.64294905)(76.86709571,369.72794897)(77.02710083,369.83795013)
\curveto(77.05709552,369.85794884)(77.0820955,369.87794882)(77.10210083,369.89795013)
\lineto(77.16210083,369.95795013)
\curveto(77.20209538,369.97794872)(77.25209533,370.01794868)(77.31210083,370.07795013)
\curveto(77.41209517,370.21794848)(77.49709508,370.34794835)(77.56710083,370.46795013)
\curveto(77.63709494,370.58794811)(77.70709487,370.73294796)(77.77710083,370.90295013)
\curveto(77.80709477,370.97294772)(77.82709475,371.04294765)(77.83710083,371.11295013)
\curveto(77.85709472,371.18294751)(77.8770947,371.25794744)(77.89710083,371.33795013)
}
}
{
\newrgbcolor{curcolor}{0 0 0}
\pscustom[linestyle=none,fillstyle=solid,fillcolor=curcolor]
{
\newpath
\moveto(771.90647461,372.23397308)
\curveto(772.88646811,372.25396213)(773.70646729,372.09396229)(774.36647461,371.75397308)
\curveto(775.03646596,371.42396296)(775.55646544,370.96396342)(775.92647461,370.37397308)
\curveto(776.02646497,370.21396417)(776.10646489,370.05896432)(776.16647461,369.90897308)
\curveto(776.23646476,369.76896461)(776.30146469,369.59896478)(776.36147461,369.39897308)
\curveto(776.38146461,369.34896503)(776.40146459,369.2789651)(776.42147461,369.18897308)
\curveto(776.44146455,369.10896527)(776.43646456,369.03396535)(776.40647461,368.96397308)
\curveto(776.38646461,368.90396548)(776.34646465,368.86396552)(776.28647461,368.84397308)
\curveto(776.23646476,368.83396555)(776.18146481,368.81896556)(776.12147461,368.79897308)
\lineto(775.97147461,368.79897308)
\curveto(775.94146505,368.78896559)(775.90146509,368.7839656)(775.85147461,368.78397308)
\lineto(775.73147461,368.78397308)
\curveto(775.5914654,368.7839656)(775.46146553,368.78896559)(775.34147461,368.79897308)
\curveto(775.23146576,368.81896556)(775.15146584,368.86896551)(775.10147461,368.94897308)
\curveto(775.03146596,369.04896533)(774.97646602,369.16396522)(774.93647461,369.29397308)
\curveto(774.8964661,369.42396496)(774.84146615,369.54396484)(774.77147461,369.65397308)
\curveto(774.64146635,369.87396451)(774.4914665,370.06396432)(774.32147461,370.22397308)
\curveto(774.16146683,370.383964)(773.97146702,370.53396385)(773.75147461,370.67397308)
\curveto(773.63146736,370.75396363)(773.4964675,370.81396357)(773.34647461,370.85397308)
\curveto(773.20646779,370.89396349)(773.06146793,370.93396345)(772.91147461,370.97397308)
\curveto(772.80146819,371.00396338)(772.67646832,371.02396336)(772.53647461,371.03397308)
\curveto(772.3964686,371.05396333)(772.24646875,371.06396332)(772.08647461,371.06397308)
\curveto(771.93646906,371.06396332)(771.78646921,371.05396333)(771.63647461,371.03397308)
\curveto(771.4964695,371.02396336)(771.37646962,371.00396338)(771.27647461,370.97397308)
\curveto(771.17646982,370.95396343)(771.08146991,370.93396345)(770.99147461,370.91397308)
\curveto(770.90147009,370.89396349)(770.81147018,370.86396352)(770.72147461,370.82397308)
\curveto(769.88147111,370.47396391)(769.27647172,369.87396451)(768.90647461,369.02397308)
\curveto(768.83647216,368.8839655)(768.77647222,368.73396565)(768.72647461,368.57397308)
\curveto(768.68647231,368.42396596)(768.64147235,368.26896611)(768.59147461,368.10897308)
\curveto(768.57147242,368.04896633)(768.56147243,367.9839664)(768.56147461,367.91397308)
\curveto(768.56147243,367.85396653)(768.55147244,367.79396659)(768.53147461,367.73397308)
\curveto(768.52147247,367.69396669)(768.51647248,367.65896672)(768.51647461,367.62897308)
\curveto(768.51647248,367.59896678)(768.51147248,367.56396682)(768.50147461,367.52397308)
\curveto(768.48147251,367.41396697)(768.46647253,367.29896708)(768.45647461,367.17897308)
\lineto(768.45647461,366.83397308)
\curveto(768.45647254,366.76396762)(768.45147254,366.68896769)(768.44147461,366.60897308)
\curveto(768.44147255,366.53896784)(768.44647255,366.47396791)(768.45647461,366.41397308)
\lineto(768.45647461,366.26397308)
\curveto(768.47647252,366.19396819)(768.48147251,366.12396826)(768.47147461,366.05397308)
\curveto(768.47147252,365.9839684)(768.48147251,365.91396847)(768.50147461,365.84397308)
\curveto(768.52147247,365.7839686)(768.52647247,365.72396866)(768.51647461,365.66397308)
\curveto(768.51647248,365.60396878)(768.52647247,365.54896883)(768.54647461,365.49897308)
\curveto(768.57647242,365.36896901)(768.60147239,365.23896914)(768.62147461,365.10897308)
\curveto(768.65147234,364.98896939)(768.68647231,364.86896951)(768.72647461,364.74897308)
\curveto(768.8964721,364.24897013)(769.11647188,363.81897056)(769.38647461,363.45897308)
\curveto(769.65647134,363.10897127)(770.01147098,362.81897156)(770.45147461,362.58897308)
\curveto(770.5914704,362.51897186)(770.73147026,362.46397192)(770.87147461,362.42397308)
\curveto(771.02146997,362.383972)(771.18146981,362.33897204)(771.35147461,362.28897308)
\curveto(771.42146957,362.26897211)(771.48646951,362.25897212)(771.54647461,362.25897308)
\curveto(771.60646939,362.26897211)(771.67646932,362.26397212)(771.75647461,362.24397308)
\curveto(771.80646919,362.23397215)(771.8964691,362.22397216)(772.02647461,362.21397308)
\curveto(772.15646884,362.21397217)(772.25146874,362.22397216)(772.31147461,362.24397308)
\lineto(772.41647461,362.24397308)
\curveto(772.45646854,362.25397213)(772.4964685,362.25397213)(772.53647461,362.24397308)
\curveto(772.57646842,362.24397214)(772.61646838,362.25397213)(772.65647461,362.27397308)
\curveto(772.75646824,362.29397209)(772.85146814,362.30897207)(772.94147461,362.31897308)
\curveto(773.04146795,362.33897204)(773.13646786,362.36897201)(773.22647461,362.40897308)
\curveto(774.00646699,362.72897165)(774.55646644,363.25397113)(774.87647461,363.98397308)
\curveto(774.95646604,364.16397022)(775.03146596,364.37897)(775.10147461,364.62897308)
\curveto(775.12146587,364.71896966)(775.13646586,364.80896957)(775.14647461,364.89897308)
\curveto(775.16646583,364.99896938)(775.20146579,365.08896929)(775.25147461,365.16897308)
\curveto(775.30146569,365.24896913)(775.38146561,365.29396909)(775.49147461,365.30397308)
\curveto(775.60146539,365.31396907)(775.72146527,365.31896906)(775.85147461,365.31897308)
\lineto(776.00147461,365.31897308)
\curveto(776.05146494,365.31896906)(776.0964649,365.31396907)(776.13647461,365.30397308)
\lineto(776.24147461,365.30397308)
\lineto(776.33147461,365.27397308)
\curveto(776.37146462,365.27396911)(776.40146459,365.26396912)(776.42147461,365.24397308)
\curveto(776.4914645,365.20396918)(776.53146446,365.12896925)(776.54147461,365.01897308)
\curveto(776.55146444,364.91896946)(776.54146445,364.81896956)(776.51147461,364.71897308)
\curveto(776.45146454,364.48896989)(776.3964646,364.26897011)(776.34647461,364.05897308)
\curveto(776.2964647,363.84897053)(776.22146477,363.64897073)(776.12147461,363.45897308)
\curveto(776.04146495,363.32897105)(775.96646503,363.20397118)(775.89647461,363.08397308)
\curveto(775.83646516,362.96397142)(775.76646523,362.84397154)(775.68647461,362.72397308)
\curveto(775.50646549,362.46397192)(775.28146571,362.22397216)(775.01147461,362.00397308)
\curveto(774.75146624,361.79397259)(774.46646653,361.61897276)(774.15647461,361.47897308)
\curveto(774.04646695,361.42897295)(773.93646706,361.38897299)(773.82647461,361.35897308)
\curveto(773.72646727,361.32897305)(773.62146737,361.29397309)(773.51147461,361.25397308)
\curveto(773.40146759,361.21397317)(773.28646771,361.18897319)(773.16647461,361.17897308)
\curveto(773.05646794,361.15897322)(772.94146805,361.13897324)(772.82147461,361.11897308)
\curveto(772.77146822,361.09897328)(772.72646827,361.09397329)(772.68647461,361.10397308)
\curveto(772.64646835,361.10397328)(772.60646839,361.09897328)(772.56647461,361.08897308)
\curveto(772.50646849,361.0789733)(772.44646855,361.07397331)(772.38647461,361.07397308)
\curveto(772.32646867,361.07397331)(772.26146873,361.06897331)(772.19147461,361.05897308)
\curveto(772.16146883,361.04897333)(772.0914689,361.04897333)(771.98147461,361.05897308)
\curveto(771.88146911,361.05897332)(771.81646918,361.06397332)(771.78647461,361.07397308)
\curveto(771.73646926,361.0839733)(771.68646931,361.08897329)(771.63647461,361.08897308)
\curveto(771.5964694,361.0789733)(771.55146944,361.0789733)(771.50147461,361.08897308)
\lineto(771.35147461,361.08897308)
\curveto(771.27146972,361.10897327)(771.1964698,361.12397326)(771.12647461,361.13397308)
\curveto(771.05646994,361.13397325)(770.98147001,361.14397324)(770.90147461,361.16397308)
\lineto(770.63147461,361.22397308)
\curveto(770.54147045,361.23397315)(770.45647054,361.25397313)(770.37647461,361.28397308)
\curveto(770.16647083,361.34397304)(769.97647102,361.41897296)(769.80647461,361.50897308)
\curveto(769.17647182,361.7789726)(768.66647233,362.16397222)(768.27647461,362.66397308)
\curveto(767.88647311,363.16397122)(767.57647342,363.75397063)(767.34647461,364.43397308)
\curveto(767.30647369,364.55396983)(767.27147372,364.6789697)(767.24147461,364.80897308)
\curveto(767.22147377,364.93896944)(767.1964738,365.07396931)(767.16647461,365.21397308)
\curveto(767.14647385,365.26396912)(767.13647386,365.30896907)(767.13647461,365.34897308)
\curveto(767.14647385,365.38896899)(767.14647385,365.43396895)(767.13647461,365.48397308)
\curveto(767.11647388,365.57396881)(767.10147389,365.66896871)(767.09147461,365.76897308)
\curveto(767.0914739,365.86896851)(767.08147391,365.96396842)(767.06147461,366.05397308)
\lineto(767.06147461,366.33897308)
\curveto(767.04147395,366.38896799)(767.03147396,366.47396791)(767.03147461,366.59397308)
\curveto(767.03147396,366.71396767)(767.04147395,366.79896758)(767.06147461,366.84897308)
\curveto(767.07147392,366.8789675)(767.07147392,366.90896747)(767.06147461,366.93897308)
\curveto(767.05147394,366.9789674)(767.05147394,367.00896737)(767.06147461,367.02897308)
\lineto(767.06147461,367.16397308)
\curveto(767.07147392,367.24396714)(767.07647392,367.32396706)(767.07647461,367.40397308)
\curveto(767.08647391,367.49396689)(767.10147389,367.5789668)(767.12147461,367.65897308)
\curveto(767.14147385,367.71896666)(767.15147384,367.7789666)(767.15147461,367.83897308)
\curveto(767.15147384,367.90896647)(767.16147383,367.9789664)(767.18147461,368.04897308)
\curveto(767.23147376,368.21896616)(767.27147372,368.383966)(767.30147461,368.54397308)
\curveto(767.33147366,368.70396568)(767.37647362,368.85396553)(767.43647461,368.99397308)
\lineto(767.58647461,369.38397308)
\curveto(767.64647335,369.52396486)(767.71147328,369.64896473)(767.78147461,369.75897308)
\curveto(767.93147306,370.01896436)(768.08147291,370.25396413)(768.23147461,370.46397308)
\curveto(768.26147273,370.51396387)(768.2964727,370.55396383)(768.33647461,370.58397308)
\curveto(768.38647261,370.62396376)(768.42647257,370.66896371)(768.45647461,370.71897308)
\curveto(768.51647248,370.79896358)(768.57647242,370.86896351)(768.63647461,370.92897308)
\lineto(768.84647461,371.10897308)
\curveto(768.90647209,371.15896322)(768.96147203,371.20396318)(769.01147461,371.24397308)
\curveto(769.07147192,371.29396309)(769.13647186,371.34396304)(769.20647461,371.39397308)
\curveto(769.35647164,371.50396288)(769.51147148,371.59896278)(769.67147461,371.67897308)
\curveto(769.84147115,371.75896262)(770.01647098,371.83896254)(770.19647461,371.91897308)
\curveto(770.30647069,371.96896241)(770.42147057,372.00396238)(770.54147461,372.02397308)
\curveto(770.67147032,372.05396233)(770.7964702,372.08896229)(770.91647461,372.12897308)
\curveto(770.98647001,372.13896224)(771.05146994,372.14896223)(771.11147461,372.15897308)
\lineto(771.29147461,372.18897308)
\curveto(771.37146962,372.19896218)(771.44646955,372.20396218)(771.51647461,372.20397308)
\curveto(771.5964694,372.21396217)(771.67646932,372.22396216)(771.75647461,372.23397308)
\curveto(771.77646922,372.24396214)(771.80146919,372.24396214)(771.83147461,372.23397308)
\curveto(771.86146913,372.22396216)(771.88646911,372.22396216)(771.90647461,372.23397308)
}
}
{
\newrgbcolor{curcolor}{0 0 0}
\pscustom[linestyle=none,fillstyle=solid,fillcolor=curcolor]
{
\newpath
\moveto(785.02631836,361.86897308)
\curveto(785.05631053,361.70897267)(785.04131054,361.57397281)(784.98131836,361.46397308)
\curveto(784.92131066,361.36397302)(784.84131074,361.28897309)(784.74131836,361.23897308)
\curveto(784.69131089,361.21897316)(784.63631095,361.20897317)(784.57631836,361.20897308)
\curveto(784.52631106,361.20897317)(784.47131111,361.19897318)(784.41131836,361.17897308)
\curveto(784.19131139,361.12897325)(783.97131161,361.14397324)(783.75131836,361.22397308)
\curveto(783.54131204,361.29397309)(783.39631219,361.383973)(783.31631836,361.49397308)
\curveto(783.26631232,361.56397282)(783.22131236,361.64397274)(783.18131836,361.73397308)
\curveto(783.14131244,361.83397255)(783.09131249,361.91397247)(783.03131836,361.97397308)
\curveto(783.01131257,361.99397239)(782.9863126,362.01397237)(782.95631836,362.03397308)
\curveto(782.93631265,362.05397233)(782.90631268,362.05897232)(782.86631836,362.04897308)
\curveto(782.75631283,362.01897236)(782.65131293,361.96397242)(782.55131836,361.88397308)
\curveto(782.46131312,361.80397258)(782.37131321,361.73397265)(782.28131836,361.67397308)
\curveto(782.15131343,361.59397279)(782.01131357,361.51897286)(781.86131836,361.44897308)
\curveto(781.71131387,361.38897299)(781.55131403,361.33397305)(781.38131836,361.28397308)
\curveto(781.2813143,361.25397313)(781.17131441,361.23397315)(781.05131836,361.22397308)
\curveto(780.94131464,361.21397317)(780.83131475,361.19897318)(780.72131836,361.17897308)
\curveto(780.67131491,361.16897321)(780.62631496,361.16397322)(780.58631836,361.16397308)
\lineto(780.48131836,361.16397308)
\curveto(780.37131521,361.14397324)(780.26631532,361.14397324)(780.16631836,361.16397308)
\lineto(780.03131836,361.16397308)
\curveto(779.9813156,361.17397321)(779.93131565,361.1789732)(779.88131836,361.17897308)
\curveto(779.83131575,361.1789732)(779.7863158,361.18897319)(779.74631836,361.20897308)
\curveto(779.70631588,361.21897316)(779.67131591,361.22397316)(779.64131836,361.22397308)
\curveto(779.62131596,361.21397317)(779.59631599,361.21397317)(779.56631836,361.22397308)
\lineto(779.32631836,361.28397308)
\curveto(779.24631634,361.29397309)(779.17131641,361.31397307)(779.10131836,361.34397308)
\curveto(778.80131678,361.47397291)(778.55631703,361.61897276)(778.36631836,361.77897308)
\curveto(778.1863174,361.94897243)(778.03631755,362.1839722)(777.91631836,362.48397308)
\curveto(777.82631776,362.70397168)(777.7813178,362.96897141)(777.78131836,363.27897308)
\lineto(777.78131836,363.59397308)
\curveto(777.79131779,363.64397074)(777.79631779,363.69397069)(777.79631836,363.74397308)
\lineto(777.82631836,363.92397308)
\lineto(777.94631836,364.25397308)
\curveto(777.9863176,364.36397002)(778.03631755,364.46396992)(778.09631836,364.55397308)
\curveto(778.27631731,364.84396954)(778.52131706,365.05896932)(778.83131836,365.19897308)
\curveto(779.14131644,365.33896904)(779.4813161,365.46396892)(779.85131836,365.57397308)
\curveto(779.99131559,365.61396877)(780.13631545,365.64396874)(780.28631836,365.66397308)
\curveto(780.43631515,365.6839687)(780.586315,365.70896867)(780.73631836,365.73897308)
\curveto(780.80631478,365.75896862)(780.87131471,365.76896861)(780.93131836,365.76897308)
\curveto(781.00131458,365.76896861)(781.07631451,365.7789686)(781.15631836,365.79897308)
\curveto(781.22631436,365.81896856)(781.29631429,365.82896855)(781.36631836,365.82897308)
\curveto(781.43631415,365.83896854)(781.51131407,365.85396853)(781.59131836,365.87397308)
\curveto(781.84131374,365.93396845)(782.07631351,365.9839684)(782.29631836,366.02397308)
\curveto(782.51631307,366.07396831)(782.69131289,366.18896819)(782.82131836,366.36897308)
\curveto(782.8813127,366.44896793)(782.93131265,366.54896783)(782.97131836,366.66897308)
\curveto(783.01131257,366.79896758)(783.01131257,366.93896744)(782.97131836,367.08897308)
\curveto(782.91131267,367.32896705)(782.82131276,367.51896686)(782.70131836,367.65897308)
\curveto(782.59131299,367.79896658)(782.43131315,367.90896647)(782.22131836,367.98897308)
\curveto(782.10131348,368.03896634)(781.95631363,368.07396631)(781.78631836,368.09397308)
\curveto(781.62631396,368.11396627)(781.45631413,368.12396626)(781.27631836,368.12397308)
\curveto(781.09631449,368.12396626)(780.92131466,368.11396627)(780.75131836,368.09397308)
\curveto(780.581315,368.07396631)(780.43631515,368.04396634)(780.31631836,368.00397308)
\curveto(780.14631544,367.94396644)(779.9813156,367.85896652)(779.82131836,367.74897308)
\curveto(779.74131584,367.68896669)(779.66631592,367.60896677)(779.59631836,367.50897308)
\curveto(779.53631605,367.41896696)(779.4813161,367.31896706)(779.43131836,367.20897308)
\curveto(779.40131618,367.12896725)(779.37131621,367.04396734)(779.34131836,366.95397308)
\curveto(779.32131626,366.86396752)(779.27631631,366.79396759)(779.20631836,366.74397308)
\curveto(779.16631642,366.71396767)(779.09631649,366.68896769)(778.99631836,366.66897308)
\curveto(778.90631668,366.65896772)(778.81131677,366.65396773)(778.71131836,366.65397308)
\curveto(778.61131697,366.65396773)(778.51131707,366.65896772)(778.41131836,366.66897308)
\curveto(778.32131726,366.68896769)(778.25631733,366.71396767)(778.21631836,366.74397308)
\curveto(778.17631741,366.77396761)(778.14631744,366.82396756)(778.12631836,366.89397308)
\curveto(778.10631748,366.96396742)(778.10631748,367.03896734)(778.12631836,367.11897308)
\curveto(778.15631743,367.24896713)(778.1863174,367.36896701)(778.21631836,367.47897308)
\curveto(778.25631733,367.59896678)(778.30131728,367.71396667)(778.35131836,367.82397308)
\curveto(778.54131704,368.17396621)(778.7813168,368.44396594)(779.07131836,368.63397308)
\curveto(779.36131622,368.83396555)(779.72131586,368.99396539)(780.15131836,369.11397308)
\curveto(780.25131533,369.13396525)(780.35131523,369.14896523)(780.45131836,369.15897308)
\curveto(780.56131502,369.16896521)(780.67131491,369.1839652)(780.78131836,369.20397308)
\curveto(780.82131476,369.21396517)(780.8863147,369.21396517)(780.97631836,369.20397308)
\curveto(781.06631452,369.20396518)(781.12131446,369.21396517)(781.14131836,369.23397308)
\curveto(781.84131374,369.24396514)(782.45131313,369.16396522)(782.97131836,368.99397308)
\curveto(783.49131209,368.82396556)(783.85631173,368.49896588)(784.06631836,368.01897308)
\curveto(784.15631143,367.81896656)(784.20631138,367.5839668)(784.21631836,367.31397308)
\curveto(784.23631135,367.05396733)(784.24631134,366.7789676)(784.24631836,366.48897308)
\lineto(784.24631836,363.17397308)
\curveto(784.24631134,363.03397135)(784.25131133,362.89897148)(784.26131836,362.76897308)
\curveto(784.27131131,362.63897174)(784.30131128,362.53397185)(784.35131836,362.45397308)
\curveto(784.40131118,362.383972)(784.46631112,362.33397205)(784.54631836,362.30397308)
\curveto(784.63631095,362.26397212)(784.72131086,362.23397215)(784.80131836,362.21397308)
\curveto(784.8813107,362.20397218)(784.94131064,362.15897222)(784.98131836,362.07897308)
\curveto(785.00131058,362.04897233)(785.01131057,362.01897236)(785.01131836,361.98897308)
\curveto(785.01131057,361.95897242)(785.01631057,361.91897246)(785.02631836,361.86897308)
\moveto(782.88131836,363.53397308)
\curveto(782.94131264,363.67397071)(782.97131261,363.83397055)(782.97131836,364.01397308)
\curveto(782.9813126,364.20397018)(782.9863126,364.39896998)(782.98631836,364.59897308)
\curveto(782.9863126,364.70896967)(782.9813126,364.80896957)(782.97131836,364.89897308)
\curveto(782.96131262,364.98896939)(782.92131266,365.05896932)(782.85131836,365.10897308)
\curveto(782.82131276,365.12896925)(782.75131283,365.13896924)(782.64131836,365.13897308)
\curveto(782.62131296,365.11896926)(782.586313,365.10896927)(782.53631836,365.10897308)
\curveto(782.4863131,365.10896927)(782.44131314,365.09896928)(782.40131836,365.07897308)
\curveto(782.32131326,365.05896932)(782.23131335,365.03896934)(782.13131836,365.01897308)
\lineto(781.83131836,364.95897308)
\curveto(781.80131378,364.95896942)(781.76631382,364.95396943)(781.72631836,364.94397308)
\lineto(781.62131836,364.94397308)
\curveto(781.47131411,364.90396948)(781.30631428,364.8789695)(781.12631836,364.86897308)
\curveto(780.95631463,364.86896951)(780.79631479,364.84896953)(780.64631836,364.80897308)
\curveto(780.56631502,364.78896959)(780.49131509,364.76896961)(780.42131836,364.74897308)
\curveto(780.36131522,364.73896964)(780.29131529,364.72396966)(780.21131836,364.70397308)
\curveto(780.05131553,364.65396973)(779.90131568,364.58896979)(779.76131836,364.50897308)
\curveto(779.62131596,364.43896994)(779.50131608,364.34897003)(779.40131836,364.23897308)
\curveto(779.30131628,364.12897025)(779.22631636,363.99397039)(779.17631836,363.83397308)
\curveto(779.12631646,363.6839707)(779.10631648,363.49897088)(779.11631836,363.27897308)
\curveto(779.11631647,363.1789712)(779.13131645,363.0839713)(779.16131836,362.99397308)
\curveto(779.20131638,362.91397147)(779.24631634,362.83897154)(779.29631836,362.76897308)
\curveto(779.37631621,362.65897172)(779.4813161,362.56397182)(779.61131836,362.48397308)
\curveto(779.74131584,362.41397197)(779.8813157,362.35397203)(780.03131836,362.30397308)
\curveto(780.0813155,362.29397209)(780.13131545,362.28897209)(780.18131836,362.28897308)
\curveto(780.23131535,362.28897209)(780.2813153,362.2839721)(780.33131836,362.27397308)
\curveto(780.40131518,362.25397213)(780.4863151,362.23897214)(780.58631836,362.22897308)
\curveto(780.69631489,362.22897215)(780.7863148,362.23897214)(780.85631836,362.25897308)
\curveto(780.91631467,362.2789721)(780.97631461,362.2839721)(781.03631836,362.27397308)
\curveto(781.09631449,362.27397211)(781.15631443,362.2839721)(781.21631836,362.30397308)
\curveto(781.29631429,362.32397206)(781.37131421,362.33897204)(781.44131836,362.34897308)
\curveto(781.52131406,362.35897202)(781.59631399,362.378972)(781.66631836,362.40897308)
\curveto(781.95631363,362.52897185)(782.20131338,362.67397171)(782.40131836,362.84397308)
\curveto(782.61131297,363.01397137)(782.77131281,363.24397114)(782.88131836,363.53397308)
}
}
{
\newrgbcolor{curcolor}{0 0 0}
\pscustom[linestyle=none,fillstyle=solid,fillcolor=curcolor]
{
\newpath
\moveto(789.88795898,369.18897308)
\curveto(790.51795375,369.20896517)(791.02295324,369.12396526)(791.40295898,368.93397308)
\curveto(791.78295248,368.74396564)(792.08795218,368.45896592)(792.31795898,368.07897308)
\curveto(792.37795189,367.9789664)(792.42295184,367.86896651)(792.45295898,367.74897308)
\curveto(792.49295177,367.63896674)(792.52795174,367.52396686)(792.55795898,367.40397308)
\curveto(792.60795166,367.21396717)(792.63795163,367.00896737)(792.64795898,366.78897308)
\curveto(792.65795161,366.56896781)(792.6629516,366.34396804)(792.66295898,366.11397308)
\lineto(792.66295898,364.50897308)
\lineto(792.66295898,362.16897308)
\curveto(792.6629516,361.99897238)(792.65795161,361.82897255)(792.64795898,361.65897308)
\curveto(792.64795162,361.48897289)(792.58295168,361.378973)(792.45295898,361.32897308)
\curveto(792.40295186,361.30897307)(792.34795192,361.29897308)(792.28795898,361.29897308)
\curveto(792.23795203,361.28897309)(792.18295208,361.2839731)(792.12295898,361.28397308)
\curveto(791.99295227,361.2839731)(791.8679524,361.28897309)(791.74795898,361.29897308)
\curveto(791.62795264,361.29897308)(791.54295272,361.33897304)(791.49295898,361.41897308)
\curveto(791.44295282,361.48897289)(791.41795285,361.5789728)(791.41795898,361.68897308)
\lineto(791.41795898,362.01897308)
\lineto(791.41795898,363.30897308)
\lineto(791.41795898,365.75397308)
\curveto(791.41795285,366.02396836)(791.41295285,366.28896809)(791.40295898,366.54897308)
\curveto(791.39295287,366.81896756)(791.34795292,367.04896733)(791.26795898,367.23897308)
\curveto(791.18795308,367.43896694)(791.0679532,367.59896678)(790.90795898,367.71897308)
\curveto(790.74795352,367.84896653)(790.5629537,367.94896643)(790.35295898,368.01897308)
\curveto(790.29295397,368.03896634)(790.22795404,368.04896633)(790.15795898,368.04897308)
\curveto(790.09795417,368.05896632)(790.03795423,368.07396631)(789.97795898,368.09397308)
\curveto(789.92795434,368.10396628)(789.84795442,368.10396628)(789.73795898,368.09397308)
\curveto(789.63795463,368.09396629)(789.5679547,368.08896629)(789.52795898,368.07897308)
\curveto(789.48795478,368.05896632)(789.45295481,368.04896633)(789.42295898,368.04897308)
\curveto(789.39295487,368.05896632)(789.35795491,368.05896632)(789.31795898,368.04897308)
\curveto(789.18795508,368.01896636)(789.0629552,367.9839664)(788.94295898,367.94397308)
\curveto(788.83295543,367.91396647)(788.72795554,367.86896651)(788.62795898,367.80897308)
\curveto(788.58795568,367.78896659)(788.55295571,367.76896661)(788.52295898,367.74897308)
\curveto(788.49295577,367.72896665)(788.45795581,367.70896667)(788.41795898,367.68897308)
\curveto(788.0679562,367.43896694)(787.81295645,367.06396732)(787.65295898,366.56397308)
\curveto(787.62295664,366.4839679)(787.60295666,366.39896798)(787.59295898,366.30897308)
\curveto(787.58295668,366.22896815)(787.5679567,366.14896823)(787.54795898,366.06897308)
\curveto(787.52795674,366.01896836)(787.52295674,365.96896841)(787.53295898,365.91897308)
\curveto(787.54295672,365.8789685)(787.53795673,365.83896854)(787.51795898,365.79897308)
\lineto(787.51795898,365.48397308)
\curveto(787.50795676,365.45396893)(787.50295676,365.41896896)(787.50295898,365.37897308)
\curveto(787.51295675,365.33896904)(787.51795675,365.29396909)(787.51795898,365.24397308)
\lineto(787.51795898,364.79397308)
\lineto(787.51795898,363.35397308)
\lineto(787.51795898,362.03397308)
\lineto(787.51795898,361.68897308)
\curveto(787.51795675,361.5789728)(787.49295677,361.48897289)(787.44295898,361.41897308)
\curveto(787.39295687,361.33897304)(787.30295696,361.29897308)(787.17295898,361.29897308)
\curveto(787.05295721,361.28897309)(786.92795734,361.2839731)(786.79795898,361.28397308)
\curveto(786.71795755,361.2839731)(786.64295762,361.28897309)(786.57295898,361.29897308)
\curveto(786.50295776,361.30897307)(786.44295782,361.33397305)(786.39295898,361.37397308)
\curveto(786.31295795,361.42397296)(786.27295799,361.51897286)(786.27295898,361.65897308)
\lineto(786.27295898,362.06397308)
\lineto(786.27295898,363.83397308)
\lineto(786.27295898,367.46397308)
\lineto(786.27295898,368.37897308)
\lineto(786.27295898,368.64897308)
\curveto(786.27295799,368.73896564)(786.29295797,368.80896557)(786.33295898,368.85897308)
\curveto(786.3629579,368.91896546)(786.41295785,368.95896542)(786.48295898,368.97897308)
\curveto(786.52295774,368.98896539)(786.57795769,368.99896538)(786.64795898,369.00897308)
\curveto(786.72795754,369.01896536)(786.80795746,369.02396536)(786.88795898,369.02397308)
\curveto(786.9679573,369.02396536)(787.04295722,369.01896536)(787.11295898,369.00897308)
\curveto(787.19295707,368.99896538)(787.24795702,368.9839654)(787.27795898,368.96397308)
\curveto(787.38795688,368.89396549)(787.43795683,368.80396558)(787.42795898,368.69397308)
\curveto(787.41795685,368.59396579)(787.43295683,368.4789659)(787.47295898,368.34897308)
\curveto(787.49295677,368.28896609)(787.53295673,368.23896614)(787.59295898,368.19897308)
\curveto(787.71295655,368.18896619)(787.80795646,368.23396615)(787.87795898,368.33397308)
\curveto(787.95795631,368.43396595)(788.03795623,368.51396587)(788.11795898,368.57397308)
\curveto(788.25795601,368.67396571)(788.39795587,368.76396562)(788.53795898,368.84397308)
\curveto(788.68795558,368.93396545)(788.85795541,369.00896537)(789.04795898,369.06897308)
\curveto(789.12795514,369.09896528)(789.21295505,369.11896526)(789.30295898,369.12897308)
\curveto(789.40295486,369.13896524)(789.49795477,369.15396523)(789.58795898,369.17397308)
\curveto(789.63795463,369.1839652)(789.68795458,369.18896519)(789.73795898,369.18897308)
\lineto(789.88795898,369.18897308)
}
}
{
\newrgbcolor{curcolor}{0 0 0}
\pscustom[linestyle=none,fillstyle=solid,fillcolor=curcolor]
{
\newpath
\moveto(795.49256836,371.37897308)
\curveto(795.64256635,371.378963)(795.7925662,371.37396301)(795.94256836,371.36397308)
\curveto(796.0925659,371.36396302)(796.19756579,371.32396306)(796.25756836,371.24397308)
\curveto(796.30756568,371.1839632)(796.33256566,371.09896328)(796.33256836,370.98897308)
\curveto(796.34256565,370.88896349)(796.34756564,370.7839636)(796.34756836,370.67397308)
\lineto(796.34756836,369.80397308)
\curveto(796.34756564,369.72396466)(796.34256565,369.63896474)(796.33256836,369.54897308)
\curveto(796.33256566,369.46896491)(796.34256565,369.39896498)(796.36256836,369.33897308)
\curveto(796.40256559,369.19896518)(796.4925655,369.10896527)(796.63256836,369.06897308)
\curveto(796.68256531,369.05896532)(796.72756526,369.05396533)(796.76756836,369.05397308)
\lineto(796.91756836,369.05397308)
\lineto(797.32256836,369.05397308)
\curveto(797.48256451,369.06396532)(797.59756439,369.05396533)(797.66756836,369.02397308)
\curveto(797.75756423,368.96396542)(797.81756417,368.90396548)(797.84756836,368.84397308)
\curveto(797.86756412,368.80396558)(797.87756411,368.75896562)(797.87756836,368.70897308)
\lineto(797.87756836,368.55897308)
\curveto(797.87756411,368.44896593)(797.87256412,368.34396604)(797.86256836,368.24397308)
\curveto(797.85256414,368.15396623)(797.81756417,368.0839663)(797.75756836,368.03397308)
\curveto(797.69756429,367.9839664)(797.61256438,367.95396643)(797.50256836,367.94397308)
\lineto(797.17256836,367.94397308)
\curveto(797.06256493,367.95396643)(796.95256504,367.95896642)(796.84256836,367.95897308)
\curveto(796.73256526,367.95896642)(796.63756535,367.94396644)(796.55756836,367.91397308)
\curveto(796.4875655,367.8839665)(796.43756555,367.83396655)(796.40756836,367.76397308)
\curveto(796.37756561,367.69396669)(796.35756563,367.60896677)(796.34756836,367.50897308)
\curveto(796.33756565,367.41896696)(796.33256566,367.31896706)(796.33256836,367.20897308)
\curveto(796.34256565,367.10896727)(796.34756564,367.00896737)(796.34756836,366.90897308)
\lineto(796.34756836,363.93897308)
\curveto(796.34756564,363.71897066)(796.34256565,363.4839709)(796.33256836,363.23397308)
\curveto(796.33256566,362.99397139)(796.37756561,362.80897157)(796.46756836,362.67897308)
\curveto(796.51756547,362.59897178)(796.58256541,362.54397184)(796.66256836,362.51397308)
\curveto(796.74256525,362.4839719)(796.83756515,362.45897192)(796.94756836,362.43897308)
\curveto(796.97756501,362.42897195)(797.00756498,362.42397196)(797.03756836,362.42397308)
\curveto(797.07756491,362.43397195)(797.11256488,362.43397195)(797.14256836,362.42397308)
\lineto(797.33756836,362.42397308)
\curveto(797.43756455,362.42397196)(797.52756446,362.41397197)(797.60756836,362.39397308)
\curveto(797.69756429,362.383972)(797.76256423,362.34897203)(797.80256836,362.28897308)
\curveto(797.82256417,362.25897212)(797.83756415,362.20397218)(797.84756836,362.12397308)
\curveto(797.86756412,362.05397233)(797.87756411,361.9789724)(797.87756836,361.89897308)
\curveto(797.8875641,361.81897256)(797.8875641,361.73897264)(797.87756836,361.65897308)
\curveto(797.86756412,361.58897279)(797.84756414,361.53397285)(797.81756836,361.49397308)
\curveto(797.77756421,361.42397296)(797.70256429,361.37397301)(797.59256836,361.34397308)
\curveto(797.51256448,361.32397306)(797.42256457,361.31397307)(797.32256836,361.31397308)
\curveto(797.22256477,361.32397306)(797.13256486,361.32897305)(797.05256836,361.32897308)
\curveto(796.992565,361.32897305)(796.93256506,361.32397306)(796.87256836,361.31397308)
\curveto(796.81256518,361.31397307)(796.75756523,361.31897306)(796.70756836,361.32897308)
\lineto(796.52756836,361.32897308)
\curveto(796.47756551,361.33897304)(796.42756556,361.34397304)(796.37756836,361.34397308)
\curveto(796.33756565,361.35397303)(796.2925657,361.35897302)(796.24256836,361.35897308)
\curveto(796.04256595,361.40897297)(795.86756612,361.46397292)(795.71756836,361.52397308)
\curveto(795.57756641,361.5839728)(795.45756653,361.68897269)(795.35756836,361.83897308)
\curveto(795.21756677,362.03897234)(795.13756685,362.28897209)(795.11756836,362.58897308)
\curveto(795.09756689,362.89897148)(795.0875669,363.22897115)(795.08756836,363.57897308)
\lineto(795.08756836,367.50897308)
\curveto(795.05756693,367.63896674)(795.02756696,367.73396665)(794.99756836,367.79397308)
\curveto(794.97756701,367.85396653)(794.90756708,367.90396648)(794.78756836,367.94397308)
\curveto(794.74756724,367.95396643)(794.70756728,367.95396643)(794.66756836,367.94397308)
\curveto(794.62756736,367.93396645)(794.5875674,367.93896644)(794.54756836,367.95897308)
\lineto(794.30756836,367.95897308)
\curveto(794.17756781,367.95896642)(794.06756792,367.96896641)(793.97756836,367.98897308)
\curveto(793.89756809,368.01896636)(793.84256815,368.0789663)(793.81256836,368.16897308)
\curveto(793.7925682,368.20896617)(793.77756821,368.25396613)(793.76756836,368.30397308)
\lineto(793.76756836,368.45397308)
\curveto(793.76756822,368.59396579)(793.77756821,368.70896567)(793.79756836,368.79897308)
\curveto(793.81756817,368.89896548)(793.87756811,368.97396541)(793.97756836,369.02397308)
\curveto(794.0875679,369.06396532)(794.22756776,369.07396531)(794.39756836,369.05397308)
\curveto(794.57756741,369.03396535)(794.72756726,369.04396534)(794.84756836,369.08397308)
\curveto(794.93756705,369.13396525)(795.00756698,369.20396518)(795.05756836,369.29397308)
\curveto(795.07756691,369.35396503)(795.0875669,369.42896495)(795.08756836,369.51897308)
\lineto(795.08756836,369.77397308)
\lineto(795.08756836,370.70397308)
\lineto(795.08756836,370.94397308)
\curveto(795.0875669,371.03396335)(795.09756689,371.10896327)(795.11756836,371.16897308)
\curveto(795.15756683,371.24896313)(795.23256676,371.31396307)(795.34256836,371.36397308)
\curveto(795.37256662,371.36396302)(795.39756659,371.36396302)(795.41756836,371.36397308)
\curveto(795.44756654,371.37396301)(795.47256652,371.378963)(795.49256836,371.37897308)
}
}
{
\newrgbcolor{curcolor}{0 0 0}
\pscustom[linestyle=none,fillstyle=solid,fillcolor=curcolor]
{
\newpath
\moveto(799.54936523,370.53897308)
\curveto(799.46936411,370.59896378)(799.42436416,370.70396368)(799.41436523,370.85397308)
\lineto(799.41436523,371.31897308)
\lineto(799.41436523,371.57397308)
\curveto(799.41436417,371.66396272)(799.42936415,371.73896264)(799.45936523,371.79897308)
\curveto(799.49936408,371.8789625)(799.579364,371.93896244)(799.69936523,371.97897308)
\curveto(799.71936386,371.98896239)(799.73936384,371.98896239)(799.75936523,371.97897308)
\curveto(799.78936379,371.9789624)(799.81436377,371.9839624)(799.83436523,371.99397308)
\curveto(800.00436358,371.99396239)(800.16436342,371.98896239)(800.31436523,371.97897308)
\curveto(800.46436312,371.96896241)(800.56436302,371.90896247)(800.61436523,371.79897308)
\curveto(800.64436294,371.73896264)(800.65936292,371.66396272)(800.65936523,371.57397308)
\lineto(800.65936523,371.31897308)
\curveto(800.65936292,371.13896324)(800.65436293,370.96896341)(800.64436523,370.80897308)
\curveto(800.64436294,370.64896373)(800.579363,370.54396384)(800.44936523,370.49397308)
\curveto(800.39936318,370.47396391)(800.34436324,370.46396392)(800.28436523,370.46397308)
\lineto(800.11936523,370.46397308)
\lineto(799.80436523,370.46397308)
\curveto(799.70436388,370.46396392)(799.61936396,370.48896389)(799.54936523,370.53897308)
\moveto(800.65936523,362.03397308)
\lineto(800.65936523,361.71897308)
\curveto(800.66936291,361.61897276)(800.64936293,361.53897284)(800.59936523,361.47897308)
\curveto(800.56936301,361.41897296)(800.52436306,361.378973)(800.46436523,361.35897308)
\curveto(800.40436318,361.34897303)(800.33436325,361.33397305)(800.25436523,361.31397308)
\lineto(800.02936523,361.31397308)
\curveto(799.89936368,361.31397307)(799.7843638,361.31897306)(799.68436523,361.32897308)
\curveto(799.59436399,361.34897303)(799.52436406,361.39897298)(799.47436523,361.47897308)
\curveto(799.43436415,361.53897284)(799.41436417,361.61397277)(799.41436523,361.70397308)
\lineto(799.41436523,361.98897308)
\lineto(799.41436523,368.33397308)
\lineto(799.41436523,368.64897308)
\curveto(799.41436417,368.75896562)(799.43936414,368.84396554)(799.48936523,368.90397308)
\curveto(799.51936406,368.95396543)(799.55936402,368.9839654)(799.60936523,368.99397308)
\curveto(799.65936392,369.00396538)(799.71436387,369.01896536)(799.77436523,369.03897308)
\curveto(799.79436379,369.03896534)(799.81436377,369.03396535)(799.83436523,369.02397308)
\curveto(799.86436372,369.02396536)(799.88936369,369.02896535)(799.90936523,369.03897308)
\curveto(800.03936354,369.03896534)(800.16936341,369.03396535)(800.29936523,369.02397308)
\curveto(800.43936314,369.02396536)(800.53436305,368.9839654)(800.58436523,368.90397308)
\curveto(800.63436295,368.84396554)(800.65936292,368.76396562)(800.65936523,368.66397308)
\lineto(800.65936523,368.37897308)
\lineto(800.65936523,362.03397308)
}
}
{
\newrgbcolor{curcolor}{0 0 0}
\pscustom[linestyle=none,fillstyle=solid,fillcolor=curcolor]
{
\newpath
\moveto(809.56420898,362.12397308)
\lineto(809.56420898,361.73397308)
\curveto(809.56420111,361.61397277)(809.53920113,361.51397287)(809.48920898,361.43397308)
\curveto(809.43920123,361.36397302)(809.35420132,361.32397306)(809.23420898,361.31397308)
\lineto(808.88920898,361.31397308)
\curveto(808.82920184,361.31397307)(808.7692019,361.30897307)(808.70920898,361.29897308)
\curveto(808.65920201,361.29897308)(808.61420206,361.30897307)(808.57420898,361.32897308)
\curveto(808.48420219,361.34897303)(808.42420225,361.38897299)(808.39420898,361.44897308)
\curveto(808.35420232,361.49897288)(808.32920234,361.55897282)(808.31920898,361.62897308)
\curveto(808.31920235,361.69897268)(808.30420237,361.76897261)(808.27420898,361.83897308)
\curveto(808.26420241,361.85897252)(808.24920242,361.87397251)(808.22920898,361.88397308)
\curveto(808.21920245,361.90397248)(808.20420247,361.92397246)(808.18420898,361.94397308)
\curveto(808.08420259,361.95397243)(808.00420267,361.93397245)(807.94420898,361.88397308)
\curveto(807.89420278,361.83397255)(807.83920283,361.7839726)(807.77920898,361.73397308)
\curveto(807.57920309,361.5839728)(807.37920329,361.46897291)(807.17920898,361.38897308)
\curveto(806.99920367,361.30897307)(806.78920388,361.24897313)(806.54920898,361.20897308)
\curveto(806.31920435,361.16897321)(806.07920459,361.14897323)(805.82920898,361.14897308)
\curveto(805.58920508,361.13897324)(805.34920532,361.15397323)(805.10920898,361.19397308)
\curveto(804.8692058,361.22397316)(804.65920601,361.2789731)(804.47920898,361.35897308)
\curveto(803.95920671,361.5789728)(803.53920713,361.87397251)(803.21920898,362.24397308)
\curveto(802.89920777,362.62397176)(802.64920802,363.09397129)(802.46920898,363.65397308)
\curveto(802.42920824,363.74397064)(802.39920827,363.83397055)(802.37920898,363.92397308)
\curveto(802.3692083,364.02397036)(802.34920832,364.12397026)(802.31920898,364.22397308)
\curveto(802.30920836,364.27397011)(802.30420837,364.32397006)(802.30420898,364.37397308)
\curveto(802.30420837,364.42396996)(802.29920837,364.47396991)(802.28920898,364.52397308)
\curveto(802.2692084,364.57396981)(802.25920841,364.62396976)(802.25920898,364.67397308)
\curveto(802.2692084,364.73396965)(802.2692084,364.78896959)(802.25920898,364.83897308)
\lineto(802.25920898,364.98897308)
\curveto(802.23920843,365.03896934)(802.22920844,365.10396928)(802.22920898,365.18397308)
\curveto(802.22920844,365.26396912)(802.23920843,365.32896905)(802.25920898,365.37897308)
\lineto(802.25920898,365.54397308)
\curveto(802.27920839,365.61396877)(802.28420839,365.6839687)(802.27420898,365.75397308)
\curveto(802.2742084,365.83396855)(802.28420839,365.90896847)(802.30420898,365.97897308)
\curveto(802.31420836,366.02896835)(802.31920835,366.07396831)(802.31920898,366.11397308)
\curveto(802.31920835,366.15396823)(802.32420835,366.19896818)(802.33420898,366.24897308)
\curveto(802.36420831,366.34896803)(802.38920828,366.44396794)(802.40920898,366.53397308)
\curveto(802.42920824,366.63396775)(802.45420822,366.72896765)(802.48420898,366.81897308)
\curveto(802.61420806,367.19896718)(802.77920789,367.53896684)(802.97920898,367.83897308)
\curveto(803.18920748,368.14896623)(803.43920723,368.40396598)(803.72920898,368.60397308)
\curveto(803.89920677,368.72396566)(804.0742066,368.82396556)(804.25420898,368.90397308)
\curveto(804.44420623,368.9839654)(804.64920602,369.05396533)(804.86920898,369.11397308)
\curveto(804.93920573,369.12396526)(805.00420567,369.13396525)(805.06420898,369.14397308)
\curveto(805.13420554,369.15396523)(805.20420547,369.16896521)(805.27420898,369.18897308)
\lineto(805.42420898,369.18897308)
\curveto(805.50420517,369.20896517)(805.61920505,369.21896516)(805.76920898,369.21897308)
\curveto(805.92920474,369.21896516)(806.04920462,369.20896517)(806.12920898,369.18897308)
\curveto(806.1692045,369.1789652)(806.22420445,369.17396521)(806.29420898,369.17397308)
\curveto(806.40420427,369.14396524)(806.51420416,369.11896526)(806.62420898,369.09897308)
\curveto(806.73420394,369.08896529)(806.83920383,369.05896532)(806.93920898,369.00897308)
\curveto(807.08920358,368.94896543)(807.22920344,368.8839655)(807.35920898,368.81397308)
\curveto(807.49920317,368.74396564)(807.62920304,368.66396572)(807.74920898,368.57397308)
\curveto(807.80920286,368.52396586)(807.8692028,368.46896591)(807.92920898,368.40897308)
\curveto(807.99920267,368.35896602)(808.08920258,368.34396604)(808.19920898,368.36397308)
\curveto(808.21920245,368.39396599)(808.23420244,368.41896596)(808.24420898,368.43897308)
\curveto(808.26420241,368.45896592)(808.27920239,368.48896589)(808.28920898,368.52897308)
\curveto(808.31920235,368.61896576)(808.32920234,368.73396565)(808.31920898,368.87397308)
\lineto(808.31920898,369.24897308)
\lineto(808.31920898,370.97397308)
\lineto(808.31920898,371.43897308)
\curveto(808.31920235,371.61896276)(808.34420233,371.74896263)(808.39420898,371.82897308)
\curveto(808.43420224,371.89896248)(808.49420218,371.94396244)(808.57420898,371.96397308)
\curveto(808.59420208,371.96396242)(808.61920205,371.96396242)(808.64920898,371.96397308)
\curveto(808.67920199,371.97396241)(808.70420197,371.9789624)(808.72420898,371.97897308)
\curveto(808.86420181,371.98896239)(809.00920166,371.98896239)(809.15920898,371.97897308)
\curveto(809.31920135,371.9789624)(809.42920124,371.93896244)(809.48920898,371.85897308)
\curveto(809.53920113,371.7789626)(809.56420111,371.6789627)(809.56420898,371.55897308)
\lineto(809.56420898,371.18397308)
\lineto(809.56420898,362.12397308)
\moveto(808.34920898,364.95897308)
\curveto(808.3692023,365.00896937)(808.37920229,365.07396931)(808.37920898,365.15397308)
\curveto(808.37920229,365.24396914)(808.3692023,365.31396907)(808.34920898,365.36397308)
\lineto(808.34920898,365.58897308)
\curveto(808.32920234,365.6789687)(808.31420236,365.76896861)(808.30420898,365.85897308)
\curveto(808.29420238,365.95896842)(808.2742024,366.04896833)(808.24420898,366.12897308)
\curveto(808.22420245,366.20896817)(808.20420247,366.2839681)(808.18420898,366.35397308)
\curveto(808.1742025,366.42396796)(808.15420252,366.49396789)(808.12420898,366.56397308)
\curveto(808.00420267,366.86396752)(807.84920282,367.12896725)(807.65920898,367.35897308)
\curveto(807.4692032,367.58896679)(807.22920344,367.76896661)(806.93920898,367.89897308)
\curveto(806.83920383,367.94896643)(806.73420394,367.9839664)(806.62420898,368.00397308)
\curveto(806.52420415,368.03396635)(806.41420426,368.05896632)(806.29420898,368.07897308)
\curveto(806.21420446,368.09896628)(806.12420455,368.10896627)(806.02420898,368.10897308)
\lineto(805.75420898,368.10897308)
\curveto(805.70420497,368.09896628)(805.65920501,368.08896629)(805.61920898,368.07897308)
\lineto(805.48420898,368.07897308)
\curveto(805.40420527,368.05896632)(805.31920535,368.03896634)(805.22920898,368.01897308)
\curveto(805.14920552,367.99896638)(805.0692056,367.97396641)(804.98920898,367.94397308)
\curveto(804.669206,367.80396658)(804.40920626,367.59896678)(804.20920898,367.32897308)
\curveto(804.01920665,367.06896731)(803.86420681,366.76396762)(803.74420898,366.41397308)
\curveto(803.70420697,366.30396808)(803.674207,366.18896819)(803.65420898,366.06897308)
\curveto(803.64420703,365.95896842)(803.62920704,365.84896853)(803.60920898,365.73897308)
\curveto(803.60920706,365.69896868)(803.60420707,365.65896872)(803.59420898,365.61897308)
\lineto(803.59420898,365.51397308)
\curveto(803.5742071,365.46396892)(803.56420711,365.40896897)(803.56420898,365.34897308)
\curveto(803.5742071,365.28896909)(803.57920709,365.23396915)(803.57920898,365.18397308)
\lineto(803.57920898,364.85397308)
\curveto(803.57920709,364.75396963)(803.58920708,364.65896972)(803.60920898,364.56897308)
\curveto(803.61920705,364.53896984)(803.62420705,364.48896989)(803.62420898,364.41897308)
\curveto(803.64420703,364.34897003)(803.65920701,364.2789701)(803.66920898,364.20897308)
\lineto(803.72920898,363.99897308)
\curveto(803.83920683,363.64897073)(803.98920668,363.34897103)(804.17920898,363.09897308)
\curveto(804.3692063,362.84897153)(804.60920606,362.64397174)(804.89920898,362.48397308)
\curveto(804.98920568,362.43397195)(805.07920559,362.39397199)(805.16920898,362.36397308)
\curveto(805.25920541,362.33397205)(805.35920531,362.30397208)(805.46920898,362.27397308)
\curveto(805.51920515,362.25397213)(805.5692051,362.24897213)(805.61920898,362.25897308)
\curveto(805.67920499,362.26897211)(805.73420494,362.26397212)(805.78420898,362.24397308)
\curveto(805.82420485,362.23397215)(805.86420481,362.22897215)(805.90420898,362.22897308)
\lineto(806.03920898,362.22897308)
\lineto(806.17420898,362.22897308)
\curveto(806.20420447,362.23897214)(806.25420442,362.24397214)(806.32420898,362.24397308)
\curveto(806.40420427,362.26397212)(806.48420419,362.2789721)(806.56420898,362.28897308)
\curveto(806.64420403,362.30897207)(806.71920395,362.33397205)(806.78920898,362.36397308)
\curveto(807.11920355,362.50397188)(807.38420329,362.6789717)(807.58420898,362.88897308)
\curveto(807.79420288,363.10897127)(807.9692027,363.383971)(808.10920898,363.71397308)
\curveto(808.15920251,363.82397056)(808.19420248,363.93397045)(808.21420898,364.04397308)
\curveto(808.23420244,364.15397023)(808.25920241,364.26397012)(808.28920898,364.37397308)
\curveto(808.30920236,364.41396997)(808.31920235,364.44896993)(808.31920898,364.47897308)
\curveto(808.31920235,364.51896986)(808.32420235,364.55896982)(808.33420898,364.59897308)
\curveto(808.34420233,364.65896972)(808.34420233,364.71896966)(808.33420898,364.77897308)
\curveto(808.33420234,364.83896954)(808.33920233,364.89896948)(808.34920898,364.95897308)
}
}
{
\newrgbcolor{curcolor}{0 0 0}
\pscustom[linestyle=none,fillstyle=solid,fillcolor=curcolor]
{
\newpath
\moveto(818.39545898,361.86897308)
\curveto(818.42545115,361.70897267)(818.41045117,361.57397281)(818.35045898,361.46397308)
\curveto(818.29045129,361.36397302)(818.21045137,361.28897309)(818.11045898,361.23897308)
\curveto(818.06045152,361.21897316)(818.00545157,361.20897317)(817.94545898,361.20897308)
\curveto(817.89545168,361.20897317)(817.84045174,361.19897318)(817.78045898,361.17897308)
\curveto(817.56045202,361.12897325)(817.34045224,361.14397324)(817.12045898,361.22397308)
\curveto(816.91045267,361.29397309)(816.76545281,361.383973)(816.68545898,361.49397308)
\curveto(816.63545294,361.56397282)(816.59045299,361.64397274)(816.55045898,361.73397308)
\curveto(816.51045307,361.83397255)(816.46045312,361.91397247)(816.40045898,361.97397308)
\curveto(816.3804532,361.99397239)(816.35545322,362.01397237)(816.32545898,362.03397308)
\curveto(816.30545327,362.05397233)(816.2754533,362.05897232)(816.23545898,362.04897308)
\curveto(816.12545345,362.01897236)(816.02045356,361.96397242)(815.92045898,361.88397308)
\curveto(815.83045375,361.80397258)(815.74045384,361.73397265)(815.65045898,361.67397308)
\curveto(815.52045406,361.59397279)(815.3804542,361.51897286)(815.23045898,361.44897308)
\curveto(815.0804545,361.38897299)(814.92045466,361.33397305)(814.75045898,361.28397308)
\curveto(814.65045493,361.25397313)(814.54045504,361.23397315)(814.42045898,361.22397308)
\curveto(814.31045527,361.21397317)(814.20045538,361.19897318)(814.09045898,361.17897308)
\curveto(814.04045554,361.16897321)(813.99545558,361.16397322)(813.95545898,361.16397308)
\lineto(813.85045898,361.16397308)
\curveto(813.74045584,361.14397324)(813.63545594,361.14397324)(813.53545898,361.16397308)
\lineto(813.40045898,361.16397308)
\curveto(813.35045623,361.17397321)(813.30045628,361.1789732)(813.25045898,361.17897308)
\curveto(813.20045638,361.1789732)(813.15545642,361.18897319)(813.11545898,361.20897308)
\curveto(813.0754565,361.21897316)(813.04045654,361.22397316)(813.01045898,361.22397308)
\curveto(812.99045659,361.21397317)(812.96545661,361.21397317)(812.93545898,361.22397308)
\lineto(812.69545898,361.28397308)
\curveto(812.61545696,361.29397309)(812.54045704,361.31397307)(812.47045898,361.34397308)
\curveto(812.17045741,361.47397291)(811.92545765,361.61897276)(811.73545898,361.77897308)
\curveto(811.55545802,361.94897243)(811.40545817,362.1839722)(811.28545898,362.48397308)
\curveto(811.19545838,362.70397168)(811.15045843,362.96897141)(811.15045898,363.27897308)
\lineto(811.15045898,363.59397308)
\curveto(811.16045842,363.64397074)(811.16545841,363.69397069)(811.16545898,363.74397308)
\lineto(811.19545898,363.92397308)
\lineto(811.31545898,364.25397308)
\curveto(811.35545822,364.36397002)(811.40545817,364.46396992)(811.46545898,364.55397308)
\curveto(811.64545793,364.84396954)(811.89045769,365.05896932)(812.20045898,365.19897308)
\curveto(812.51045707,365.33896904)(812.85045673,365.46396892)(813.22045898,365.57397308)
\curveto(813.36045622,365.61396877)(813.50545607,365.64396874)(813.65545898,365.66397308)
\curveto(813.80545577,365.6839687)(813.95545562,365.70896867)(814.10545898,365.73897308)
\curveto(814.1754554,365.75896862)(814.24045534,365.76896861)(814.30045898,365.76897308)
\curveto(814.37045521,365.76896861)(814.44545513,365.7789686)(814.52545898,365.79897308)
\curveto(814.59545498,365.81896856)(814.66545491,365.82896855)(814.73545898,365.82897308)
\curveto(814.80545477,365.83896854)(814.8804547,365.85396853)(814.96045898,365.87397308)
\curveto(815.21045437,365.93396845)(815.44545413,365.9839684)(815.66545898,366.02397308)
\curveto(815.88545369,366.07396831)(816.06045352,366.18896819)(816.19045898,366.36897308)
\curveto(816.25045333,366.44896793)(816.30045328,366.54896783)(816.34045898,366.66897308)
\curveto(816.3804532,366.79896758)(816.3804532,366.93896744)(816.34045898,367.08897308)
\curveto(816.2804533,367.32896705)(816.19045339,367.51896686)(816.07045898,367.65897308)
\curveto(815.96045362,367.79896658)(815.80045378,367.90896647)(815.59045898,367.98897308)
\curveto(815.47045411,368.03896634)(815.32545425,368.07396631)(815.15545898,368.09397308)
\curveto(814.99545458,368.11396627)(814.82545475,368.12396626)(814.64545898,368.12397308)
\curveto(814.46545511,368.12396626)(814.29045529,368.11396627)(814.12045898,368.09397308)
\curveto(813.95045563,368.07396631)(813.80545577,368.04396634)(813.68545898,368.00397308)
\curveto(813.51545606,367.94396644)(813.35045623,367.85896652)(813.19045898,367.74897308)
\curveto(813.11045647,367.68896669)(813.03545654,367.60896677)(812.96545898,367.50897308)
\curveto(812.90545667,367.41896696)(812.85045673,367.31896706)(812.80045898,367.20897308)
\curveto(812.77045681,367.12896725)(812.74045684,367.04396734)(812.71045898,366.95397308)
\curveto(812.69045689,366.86396752)(812.64545693,366.79396759)(812.57545898,366.74397308)
\curveto(812.53545704,366.71396767)(812.46545711,366.68896769)(812.36545898,366.66897308)
\curveto(812.2754573,366.65896772)(812.1804574,366.65396773)(812.08045898,366.65397308)
\curveto(811.9804576,366.65396773)(811.8804577,366.65896772)(811.78045898,366.66897308)
\curveto(811.69045789,366.68896769)(811.62545795,366.71396767)(811.58545898,366.74397308)
\curveto(811.54545803,366.77396761)(811.51545806,366.82396756)(811.49545898,366.89397308)
\curveto(811.4754581,366.96396742)(811.4754581,367.03896734)(811.49545898,367.11897308)
\curveto(811.52545805,367.24896713)(811.55545802,367.36896701)(811.58545898,367.47897308)
\curveto(811.62545795,367.59896678)(811.67045791,367.71396667)(811.72045898,367.82397308)
\curveto(811.91045767,368.17396621)(812.15045743,368.44396594)(812.44045898,368.63397308)
\curveto(812.73045685,368.83396555)(813.09045649,368.99396539)(813.52045898,369.11397308)
\curveto(813.62045596,369.13396525)(813.72045586,369.14896523)(813.82045898,369.15897308)
\curveto(813.93045565,369.16896521)(814.04045554,369.1839652)(814.15045898,369.20397308)
\curveto(814.19045539,369.21396517)(814.25545532,369.21396517)(814.34545898,369.20397308)
\curveto(814.43545514,369.20396518)(814.49045509,369.21396517)(814.51045898,369.23397308)
\curveto(815.21045437,369.24396514)(815.82045376,369.16396522)(816.34045898,368.99397308)
\curveto(816.86045272,368.82396556)(817.22545235,368.49896588)(817.43545898,368.01897308)
\curveto(817.52545205,367.81896656)(817.575452,367.5839668)(817.58545898,367.31397308)
\curveto(817.60545197,367.05396733)(817.61545196,366.7789676)(817.61545898,366.48897308)
\lineto(817.61545898,363.17397308)
\curveto(817.61545196,363.03397135)(817.62045196,362.89897148)(817.63045898,362.76897308)
\curveto(817.64045194,362.63897174)(817.67045191,362.53397185)(817.72045898,362.45397308)
\curveto(817.77045181,362.383972)(817.83545174,362.33397205)(817.91545898,362.30397308)
\curveto(818.00545157,362.26397212)(818.09045149,362.23397215)(818.17045898,362.21397308)
\curveto(818.25045133,362.20397218)(818.31045127,362.15897222)(818.35045898,362.07897308)
\curveto(818.37045121,362.04897233)(818.3804512,362.01897236)(818.38045898,361.98897308)
\curveto(818.3804512,361.95897242)(818.38545119,361.91897246)(818.39545898,361.86897308)
\moveto(816.25045898,363.53397308)
\curveto(816.31045327,363.67397071)(816.34045324,363.83397055)(816.34045898,364.01397308)
\curveto(816.35045323,364.20397018)(816.35545322,364.39896998)(816.35545898,364.59897308)
\curveto(816.35545322,364.70896967)(816.35045323,364.80896957)(816.34045898,364.89897308)
\curveto(816.33045325,364.98896939)(816.29045329,365.05896932)(816.22045898,365.10897308)
\curveto(816.19045339,365.12896925)(816.12045346,365.13896924)(816.01045898,365.13897308)
\curveto(815.99045359,365.11896926)(815.95545362,365.10896927)(815.90545898,365.10897308)
\curveto(815.85545372,365.10896927)(815.81045377,365.09896928)(815.77045898,365.07897308)
\curveto(815.69045389,365.05896932)(815.60045398,365.03896934)(815.50045898,365.01897308)
\lineto(815.20045898,364.95897308)
\curveto(815.17045441,364.95896942)(815.13545444,364.95396943)(815.09545898,364.94397308)
\lineto(814.99045898,364.94397308)
\curveto(814.84045474,364.90396948)(814.6754549,364.8789695)(814.49545898,364.86897308)
\curveto(814.32545525,364.86896951)(814.16545541,364.84896953)(814.01545898,364.80897308)
\curveto(813.93545564,364.78896959)(813.86045572,364.76896961)(813.79045898,364.74897308)
\curveto(813.73045585,364.73896964)(813.66045592,364.72396966)(813.58045898,364.70397308)
\curveto(813.42045616,364.65396973)(813.27045631,364.58896979)(813.13045898,364.50897308)
\curveto(812.99045659,364.43896994)(812.87045671,364.34897003)(812.77045898,364.23897308)
\curveto(812.67045691,364.12897025)(812.59545698,363.99397039)(812.54545898,363.83397308)
\curveto(812.49545708,363.6839707)(812.4754571,363.49897088)(812.48545898,363.27897308)
\curveto(812.48545709,363.1789712)(812.50045708,363.0839713)(812.53045898,362.99397308)
\curveto(812.57045701,362.91397147)(812.61545696,362.83897154)(812.66545898,362.76897308)
\curveto(812.74545683,362.65897172)(812.85045673,362.56397182)(812.98045898,362.48397308)
\curveto(813.11045647,362.41397197)(813.25045633,362.35397203)(813.40045898,362.30397308)
\curveto(813.45045613,362.29397209)(813.50045608,362.28897209)(813.55045898,362.28897308)
\curveto(813.60045598,362.28897209)(813.65045593,362.2839721)(813.70045898,362.27397308)
\curveto(813.77045581,362.25397213)(813.85545572,362.23897214)(813.95545898,362.22897308)
\curveto(814.06545551,362.22897215)(814.15545542,362.23897214)(814.22545898,362.25897308)
\curveto(814.28545529,362.2789721)(814.34545523,362.2839721)(814.40545898,362.27397308)
\curveto(814.46545511,362.27397211)(814.52545505,362.2839721)(814.58545898,362.30397308)
\curveto(814.66545491,362.32397206)(814.74045484,362.33897204)(814.81045898,362.34897308)
\curveto(814.89045469,362.35897202)(814.96545461,362.378972)(815.03545898,362.40897308)
\curveto(815.32545425,362.52897185)(815.57045401,362.67397171)(815.77045898,362.84397308)
\curveto(815.9804536,363.01397137)(816.14045344,363.24397114)(816.25045898,363.53397308)
}
}
{
\newrgbcolor{curcolor}{0 0 0}
\pscustom[linestyle=none,fillstyle=solid,fillcolor=curcolor]
{
\newpath
\moveto(826.52709961,362.12397308)
\lineto(826.52709961,361.73397308)
\curveto(826.52709173,361.61397277)(826.50209176,361.51397287)(826.45209961,361.43397308)
\curveto(826.40209186,361.36397302)(826.31709194,361.32397306)(826.19709961,361.31397308)
\lineto(825.85209961,361.31397308)
\curveto(825.79209247,361.31397307)(825.73209253,361.30897307)(825.67209961,361.29897308)
\curveto(825.62209264,361.29897308)(825.57709268,361.30897307)(825.53709961,361.32897308)
\curveto(825.44709281,361.34897303)(825.38709287,361.38897299)(825.35709961,361.44897308)
\curveto(825.31709294,361.49897288)(825.29209297,361.55897282)(825.28209961,361.62897308)
\curveto(825.28209298,361.69897268)(825.26709299,361.76897261)(825.23709961,361.83897308)
\curveto(825.22709303,361.85897252)(825.21209305,361.87397251)(825.19209961,361.88397308)
\curveto(825.18209308,361.90397248)(825.16709309,361.92397246)(825.14709961,361.94397308)
\curveto(825.04709321,361.95397243)(824.96709329,361.93397245)(824.90709961,361.88397308)
\curveto(824.8570934,361.83397255)(824.80209346,361.7839726)(824.74209961,361.73397308)
\curveto(824.54209372,361.5839728)(824.34209392,361.46897291)(824.14209961,361.38897308)
\curveto(823.9620943,361.30897307)(823.75209451,361.24897313)(823.51209961,361.20897308)
\curveto(823.28209498,361.16897321)(823.04209522,361.14897323)(822.79209961,361.14897308)
\curveto(822.55209571,361.13897324)(822.31209595,361.15397323)(822.07209961,361.19397308)
\curveto(821.83209643,361.22397316)(821.62209664,361.2789731)(821.44209961,361.35897308)
\curveto(820.92209734,361.5789728)(820.50209776,361.87397251)(820.18209961,362.24397308)
\curveto(819.8620984,362.62397176)(819.61209865,363.09397129)(819.43209961,363.65397308)
\curveto(819.39209887,363.74397064)(819.3620989,363.83397055)(819.34209961,363.92397308)
\curveto(819.33209893,364.02397036)(819.31209895,364.12397026)(819.28209961,364.22397308)
\curveto(819.27209899,364.27397011)(819.26709899,364.32397006)(819.26709961,364.37397308)
\curveto(819.26709899,364.42396996)(819.262099,364.47396991)(819.25209961,364.52397308)
\curveto(819.23209903,364.57396981)(819.22209904,364.62396976)(819.22209961,364.67397308)
\curveto(819.23209903,364.73396965)(819.23209903,364.78896959)(819.22209961,364.83897308)
\lineto(819.22209961,364.98897308)
\curveto(819.20209906,365.03896934)(819.19209907,365.10396928)(819.19209961,365.18397308)
\curveto(819.19209907,365.26396912)(819.20209906,365.32896905)(819.22209961,365.37897308)
\lineto(819.22209961,365.54397308)
\curveto(819.24209902,365.61396877)(819.24709901,365.6839687)(819.23709961,365.75397308)
\curveto(819.23709902,365.83396855)(819.24709901,365.90896847)(819.26709961,365.97897308)
\curveto(819.27709898,366.02896835)(819.28209898,366.07396831)(819.28209961,366.11397308)
\curveto(819.28209898,366.15396823)(819.28709897,366.19896818)(819.29709961,366.24897308)
\curveto(819.32709893,366.34896803)(819.35209891,366.44396794)(819.37209961,366.53397308)
\curveto(819.39209887,366.63396775)(819.41709884,366.72896765)(819.44709961,366.81897308)
\curveto(819.57709868,367.19896718)(819.74209852,367.53896684)(819.94209961,367.83897308)
\curveto(820.15209811,368.14896623)(820.40209786,368.40396598)(820.69209961,368.60397308)
\curveto(820.8620974,368.72396566)(821.03709722,368.82396556)(821.21709961,368.90397308)
\curveto(821.40709685,368.9839654)(821.61209665,369.05396533)(821.83209961,369.11397308)
\curveto(821.90209636,369.12396526)(821.96709629,369.13396525)(822.02709961,369.14397308)
\curveto(822.09709616,369.15396523)(822.16709609,369.16896521)(822.23709961,369.18897308)
\lineto(822.38709961,369.18897308)
\curveto(822.46709579,369.20896517)(822.58209568,369.21896516)(822.73209961,369.21897308)
\curveto(822.89209537,369.21896516)(823.01209525,369.20896517)(823.09209961,369.18897308)
\curveto(823.13209513,369.1789652)(823.18709507,369.17396521)(823.25709961,369.17397308)
\curveto(823.36709489,369.14396524)(823.47709478,369.11896526)(823.58709961,369.09897308)
\curveto(823.69709456,369.08896529)(823.80209446,369.05896532)(823.90209961,369.00897308)
\curveto(824.05209421,368.94896543)(824.19209407,368.8839655)(824.32209961,368.81397308)
\curveto(824.4620938,368.74396564)(824.59209367,368.66396572)(824.71209961,368.57397308)
\curveto(824.77209349,368.52396586)(824.83209343,368.46896591)(824.89209961,368.40897308)
\curveto(824.9620933,368.35896602)(825.05209321,368.34396604)(825.16209961,368.36397308)
\curveto(825.18209308,368.39396599)(825.19709306,368.41896596)(825.20709961,368.43897308)
\curveto(825.22709303,368.45896592)(825.24209302,368.48896589)(825.25209961,368.52897308)
\curveto(825.28209298,368.61896576)(825.29209297,368.73396565)(825.28209961,368.87397308)
\lineto(825.28209961,369.24897308)
\lineto(825.28209961,370.97397308)
\lineto(825.28209961,371.43897308)
\curveto(825.28209298,371.61896276)(825.30709295,371.74896263)(825.35709961,371.82897308)
\curveto(825.39709286,371.89896248)(825.4570928,371.94396244)(825.53709961,371.96397308)
\curveto(825.5570927,371.96396242)(825.58209268,371.96396242)(825.61209961,371.96397308)
\curveto(825.64209262,371.97396241)(825.66709259,371.9789624)(825.68709961,371.97897308)
\curveto(825.82709243,371.98896239)(825.97209229,371.98896239)(826.12209961,371.97897308)
\curveto(826.28209198,371.9789624)(826.39209187,371.93896244)(826.45209961,371.85897308)
\curveto(826.50209176,371.7789626)(826.52709173,371.6789627)(826.52709961,371.55897308)
\lineto(826.52709961,371.18397308)
\lineto(826.52709961,362.12397308)
\moveto(825.31209961,364.95897308)
\curveto(825.33209293,365.00896937)(825.34209292,365.07396931)(825.34209961,365.15397308)
\curveto(825.34209292,365.24396914)(825.33209293,365.31396907)(825.31209961,365.36397308)
\lineto(825.31209961,365.58897308)
\curveto(825.29209297,365.6789687)(825.27709298,365.76896861)(825.26709961,365.85897308)
\curveto(825.257093,365.95896842)(825.23709302,366.04896833)(825.20709961,366.12897308)
\curveto(825.18709307,366.20896817)(825.16709309,366.2839681)(825.14709961,366.35397308)
\curveto(825.13709312,366.42396796)(825.11709314,366.49396789)(825.08709961,366.56397308)
\curveto(824.96709329,366.86396752)(824.81209345,367.12896725)(824.62209961,367.35897308)
\curveto(824.43209383,367.58896679)(824.19209407,367.76896661)(823.90209961,367.89897308)
\curveto(823.80209446,367.94896643)(823.69709456,367.9839664)(823.58709961,368.00397308)
\curveto(823.48709477,368.03396635)(823.37709488,368.05896632)(823.25709961,368.07897308)
\curveto(823.17709508,368.09896628)(823.08709517,368.10896627)(822.98709961,368.10897308)
\lineto(822.71709961,368.10897308)
\curveto(822.66709559,368.09896628)(822.62209564,368.08896629)(822.58209961,368.07897308)
\lineto(822.44709961,368.07897308)
\curveto(822.36709589,368.05896632)(822.28209598,368.03896634)(822.19209961,368.01897308)
\curveto(822.11209615,367.99896638)(822.03209623,367.97396641)(821.95209961,367.94397308)
\curveto(821.63209663,367.80396658)(821.37209689,367.59896678)(821.17209961,367.32897308)
\curveto(820.98209728,367.06896731)(820.82709743,366.76396762)(820.70709961,366.41397308)
\curveto(820.66709759,366.30396808)(820.63709762,366.18896819)(820.61709961,366.06897308)
\curveto(820.60709765,365.95896842)(820.59209767,365.84896853)(820.57209961,365.73897308)
\curveto(820.57209769,365.69896868)(820.56709769,365.65896872)(820.55709961,365.61897308)
\lineto(820.55709961,365.51397308)
\curveto(820.53709772,365.46396892)(820.52709773,365.40896897)(820.52709961,365.34897308)
\curveto(820.53709772,365.28896909)(820.54209772,365.23396915)(820.54209961,365.18397308)
\lineto(820.54209961,364.85397308)
\curveto(820.54209772,364.75396963)(820.55209771,364.65896972)(820.57209961,364.56897308)
\curveto(820.58209768,364.53896984)(820.58709767,364.48896989)(820.58709961,364.41897308)
\curveto(820.60709765,364.34897003)(820.62209764,364.2789701)(820.63209961,364.20897308)
\lineto(820.69209961,363.99897308)
\curveto(820.80209746,363.64897073)(820.95209731,363.34897103)(821.14209961,363.09897308)
\curveto(821.33209693,362.84897153)(821.57209669,362.64397174)(821.86209961,362.48397308)
\curveto(821.95209631,362.43397195)(822.04209622,362.39397199)(822.13209961,362.36397308)
\curveto(822.22209604,362.33397205)(822.32209594,362.30397208)(822.43209961,362.27397308)
\curveto(822.48209578,362.25397213)(822.53209573,362.24897213)(822.58209961,362.25897308)
\curveto(822.64209562,362.26897211)(822.69709556,362.26397212)(822.74709961,362.24397308)
\curveto(822.78709547,362.23397215)(822.82709543,362.22897215)(822.86709961,362.22897308)
\lineto(823.00209961,362.22897308)
\lineto(823.13709961,362.22897308)
\curveto(823.16709509,362.23897214)(823.21709504,362.24397214)(823.28709961,362.24397308)
\curveto(823.36709489,362.26397212)(823.44709481,362.2789721)(823.52709961,362.28897308)
\curveto(823.60709465,362.30897207)(823.68209458,362.33397205)(823.75209961,362.36397308)
\curveto(824.08209418,362.50397188)(824.34709391,362.6789717)(824.54709961,362.88897308)
\curveto(824.7570935,363.10897127)(824.93209333,363.383971)(825.07209961,363.71397308)
\curveto(825.12209314,363.82397056)(825.1570931,363.93397045)(825.17709961,364.04397308)
\curveto(825.19709306,364.15397023)(825.22209304,364.26397012)(825.25209961,364.37397308)
\curveto(825.27209299,364.41396997)(825.28209298,364.44896993)(825.28209961,364.47897308)
\curveto(825.28209298,364.51896986)(825.28709297,364.55896982)(825.29709961,364.59897308)
\curveto(825.30709295,364.65896972)(825.30709295,364.71896966)(825.29709961,364.77897308)
\curveto(825.29709296,364.83896954)(825.30209296,364.89896948)(825.31209961,364.95897308)
}
}
{
\newrgbcolor{curcolor}{0 0 0}
\pscustom[linestyle=none,fillstyle=solid,fillcolor=curcolor]
{
\newpath
\moveto(775.88144531,338.77765076)
\curveto(775.90143577,338.72765001)(775.92643574,338.66765007)(775.95644531,338.59765076)
\curveto(775.98643568,338.52765021)(776.00643566,338.45265029)(776.01644531,338.37265076)
\curveto(776.03643563,338.30265044)(776.03643563,338.23265051)(776.01644531,338.16265076)
\curveto(776.00643566,338.10265064)(775.9664357,338.05765068)(775.89644531,338.02765076)
\curveto(775.84643582,338.00765073)(775.78643588,337.99765074)(775.71644531,337.99765076)
\lineto(775.50644531,337.99765076)
\lineto(775.05644531,337.99765076)
\curveto(774.90643676,337.99765074)(774.78643688,338.02265072)(774.69644531,338.07265076)
\curveto(774.59643707,338.13265061)(774.52143715,338.2376505)(774.47144531,338.38765076)
\curveto(774.43143724,338.5376502)(774.38643728,338.67265007)(774.33644531,338.79265076)
\curveto(774.22643744,339.05264969)(774.12643754,339.32264942)(774.03644531,339.60265076)
\curveto(773.94643772,339.88264886)(773.84643782,340.15764858)(773.73644531,340.42765076)
\curveto(773.70643796,340.51764822)(773.67643799,340.60264814)(773.64644531,340.68265076)
\curveto(773.62643804,340.76264798)(773.59643807,340.8376479)(773.55644531,340.90765076)
\curveto(773.52643814,340.97764776)(773.48143819,341.0376477)(773.42144531,341.08765076)
\curveto(773.36143831,341.1376476)(773.28143839,341.17764756)(773.18144531,341.20765076)
\curveto(773.13143854,341.22764751)(773.0714386,341.23264751)(773.00144531,341.22265076)
\lineto(772.80644531,341.22265076)
\lineto(769.97144531,341.22265076)
\lineto(769.67144531,341.22265076)
\curveto(769.56144211,341.23264751)(769.45644221,341.23264751)(769.35644531,341.22265076)
\curveto(769.25644241,341.21264753)(769.16144251,341.19764754)(769.07144531,341.17765076)
\curveto(768.99144268,341.15764758)(768.93144274,341.11764762)(768.89144531,341.05765076)
\curveto(768.81144286,340.95764778)(768.75144292,340.8426479)(768.71144531,340.71265076)
\curveto(768.68144299,340.59264815)(768.64144303,340.46764827)(768.59144531,340.33765076)
\curveto(768.49144318,340.10764863)(768.39644327,339.86764887)(768.30644531,339.61765076)
\curveto(768.22644344,339.36764937)(768.13644353,339.12764961)(768.03644531,338.89765076)
\curveto(768.01644365,338.8376499)(767.99144368,338.76764997)(767.96144531,338.68765076)
\curveto(767.94144373,338.61765012)(767.91644375,338.5426502)(767.88644531,338.46265076)
\curveto(767.85644381,338.38265036)(767.82144385,338.30765043)(767.78144531,338.23765076)
\curveto(767.75144392,338.17765056)(767.71644395,338.13265061)(767.67644531,338.10265076)
\curveto(767.59644407,338.0426507)(767.48644418,338.00765073)(767.34644531,337.99765076)
\lineto(766.92644531,337.99765076)
\lineto(766.68644531,337.99765076)
\curveto(766.61644505,338.00765073)(766.55644511,338.03265071)(766.50644531,338.07265076)
\curveto(766.45644521,338.10265064)(766.42644524,338.14765059)(766.41644531,338.20765076)
\curveto(766.41644525,338.26765047)(766.42144525,338.32765041)(766.43144531,338.38765076)
\curveto(766.45144522,338.45765028)(766.4714452,338.52265022)(766.49144531,338.58265076)
\curveto(766.52144515,338.65265009)(766.54644512,338.70265004)(766.56644531,338.73265076)
\curveto(766.70644496,339.05264969)(766.83144484,339.36764937)(766.94144531,339.67765076)
\curveto(767.05144462,339.99764874)(767.1714445,340.31764842)(767.30144531,340.63765076)
\curveto(767.39144428,340.85764788)(767.47644419,341.07264767)(767.55644531,341.28265076)
\curveto(767.63644403,341.50264724)(767.72144395,341.72264702)(767.81144531,341.94265076)
\curveto(768.11144356,342.66264608)(768.39644327,343.38764535)(768.66644531,344.11765076)
\curveto(768.93644273,344.85764388)(769.22144245,345.59264315)(769.52144531,346.32265076)
\curveto(769.63144204,346.58264216)(769.73144194,346.84764189)(769.82144531,347.11765076)
\curveto(769.92144175,347.38764135)(770.02644164,347.65264109)(770.13644531,347.91265076)
\curveto(770.18644148,348.02264072)(770.23144144,348.1426406)(770.27144531,348.27265076)
\curveto(770.32144135,348.41264033)(770.39144128,348.51264023)(770.48144531,348.57265076)
\curveto(770.52144115,348.61264013)(770.58644108,348.6426401)(770.67644531,348.66265076)
\curveto(770.69644097,348.67264007)(770.71644095,348.67264007)(770.73644531,348.66265076)
\curveto(770.7664409,348.66264008)(770.79144088,348.66764007)(770.81144531,348.67765076)
\curveto(770.99144068,348.67764006)(771.20144047,348.67764006)(771.44144531,348.67765076)
\curveto(771.68143999,348.68764005)(771.85643981,348.65264009)(771.96644531,348.57265076)
\curveto(772.04643962,348.51264023)(772.10643956,348.41264033)(772.14644531,348.27265076)
\curveto(772.19643947,348.1426406)(772.24643942,348.02264072)(772.29644531,347.91265076)
\curveto(772.39643927,347.68264106)(772.48643918,347.45264129)(772.56644531,347.22265076)
\curveto(772.64643902,346.99264175)(772.73643893,346.76264198)(772.83644531,346.53265076)
\curveto(772.91643875,346.33264241)(772.99143868,346.12764261)(773.06144531,345.91765076)
\curveto(773.14143853,345.70764303)(773.22643844,345.50264324)(773.31644531,345.30265076)
\curveto(773.61643805,344.57264417)(773.90143777,343.83264491)(774.17144531,343.08265076)
\curveto(774.45143722,342.3426464)(774.74643692,341.60764713)(775.05644531,340.87765076)
\curveto(775.09643657,340.78764795)(775.12643654,340.70264804)(775.14644531,340.62265076)
\curveto(775.17643649,340.5426482)(775.20643646,340.45764828)(775.23644531,340.36765076)
\curveto(775.34643632,340.10764863)(775.45143622,339.8426489)(775.55144531,339.57265076)
\curveto(775.66143601,339.30264944)(775.7714359,339.0376497)(775.88144531,338.77765076)
\moveto(772.67144531,342.42265076)
\curveto(772.76143891,342.45264629)(772.81643885,342.50264624)(772.83644531,342.57265076)
\curveto(772.8664388,342.6426461)(772.8714388,342.71764602)(772.85144531,342.79765076)
\curveto(772.84143883,342.88764585)(772.81643885,342.97264577)(772.77644531,343.05265076)
\curveto(772.74643892,343.1426456)(772.71643895,343.21764552)(772.68644531,343.27765076)
\curveto(772.666439,343.31764542)(772.65643901,343.35264539)(772.65644531,343.38265076)
\curveto(772.65643901,343.41264533)(772.64643902,343.44764529)(772.62644531,343.48765076)
\lineto(772.53644531,343.72765076)
\curveto(772.51643915,343.81764492)(772.48643918,343.90764483)(772.44644531,343.99765076)
\curveto(772.29643937,344.35764438)(772.16143951,344.72264402)(772.04144531,345.09265076)
\curveto(771.93143974,345.47264327)(771.80143987,345.8426429)(771.65144531,346.20265076)
\curveto(771.60144007,346.31264243)(771.55644011,346.42264232)(771.51644531,346.53265076)
\curveto(771.48644018,346.6426421)(771.44644022,346.74764199)(771.39644531,346.84765076)
\curveto(771.37644029,346.89764184)(771.35144032,346.9426418)(771.32144531,346.98265076)
\curveto(771.30144037,347.03264171)(771.25144042,347.05764168)(771.17144531,347.05765076)
\curveto(771.15144052,347.0376417)(771.13144054,347.02264172)(771.11144531,347.01265076)
\curveto(771.09144058,347.00264174)(771.0714406,346.98764175)(771.05144531,346.96765076)
\curveto(771.01144066,346.91764182)(770.98144069,346.86264188)(770.96144531,346.80265076)
\curveto(770.94144073,346.75264199)(770.92144075,346.69764204)(770.90144531,346.63765076)
\curveto(770.85144082,346.52764221)(770.81144086,346.41764232)(770.78144531,346.30765076)
\curveto(770.75144092,346.19764254)(770.71144096,346.08764265)(770.66144531,345.97765076)
\curveto(770.49144118,345.58764315)(770.34144133,345.19264355)(770.21144531,344.79265076)
\curveto(770.09144158,344.39264435)(769.95144172,344.00264474)(769.79144531,343.62265076)
\lineto(769.73144531,343.47265076)
\curveto(769.72144195,343.42264532)(769.70644196,343.37264537)(769.68644531,343.32265076)
\lineto(769.59644531,343.08265076)
\curveto(769.5664421,343.00264574)(769.54144213,342.92264582)(769.52144531,342.84265076)
\curveto(769.50144217,342.79264595)(769.49144218,342.737646)(769.49144531,342.67765076)
\curveto(769.50144217,342.61764612)(769.51644215,342.56764617)(769.53644531,342.52765076)
\curveto(769.58644208,342.44764629)(769.69144198,342.40264634)(769.85144531,342.39265076)
\lineto(770.30144531,342.39265076)
\lineto(771.90644531,342.39265076)
\curveto(772.01643965,342.39264635)(772.15143952,342.38764635)(772.31144531,342.37765076)
\curveto(772.4714392,342.37764636)(772.59143908,342.39264635)(772.67144531,342.42265076)
}
}
{
\newrgbcolor{curcolor}{0 0 0}
\pscustom[linestyle=none,fillstyle=solid,fillcolor=curcolor]
{
\newpath
\moveto(777.46300781,345.72265076)
\lineto(777.89800781,345.72265076)
\curveto(778.04800585,345.72264302)(778.15300574,345.68264306)(778.21300781,345.60265076)
\curveto(778.26300563,345.52264322)(778.28800561,345.42264332)(778.28800781,345.30265076)
\curveto(778.2980056,345.18264356)(778.30300559,345.06264368)(778.30300781,344.94265076)
\lineto(778.30300781,343.51765076)
\lineto(778.30300781,341.25265076)
\lineto(778.30300781,340.56265076)
\curveto(778.30300559,340.33264841)(778.32800557,340.13264861)(778.37800781,339.96265076)
\curveto(778.53800536,339.51264923)(778.83800506,339.19764954)(779.27800781,339.01765076)
\curveto(779.4980044,338.92764981)(779.76300413,338.89264985)(780.07300781,338.91265076)
\curveto(780.38300351,338.9426498)(780.63300326,338.99764974)(780.82300781,339.07765076)
\curveto(781.15300274,339.21764952)(781.41300248,339.39264935)(781.60300781,339.60265076)
\curveto(781.80300209,339.82264892)(781.95800194,340.10764863)(782.06800781,340.45765076)
\curveto(782.0980018,340.5376482)(782.11800178,340.61764812)(782.12800781,340.69765076)
\curveto(782.13800176,340.77764796)(782.15300174,340.86264788)(782.17300781,340.95265076)
\curveto(782.18300171,341.00264774)(782.18300171,341.04764769)(782.17300781,341.08765076)
\curveto(782.17300172,341.12764761)(782.18300171,341.17264757)(782.20300781,341.22265076)
\lineto(782.20300781,341.53765076)
\curveto(782.22300167,341.61764712)(782.22800167,341.70764703)(782.21800781,341.80765076)
\curveto(782.20800169,341.91764682)(782.20300169,342.01764672)(782.20300781,342.10765076)
\lineto(782.20300781,343.27765076)
\lineto(782.20300781,344.86765076)
\curveto(782.20300169,344.98764375)(782.1980017,345.11264363)(782.18800781,345.24265076)
\curveto(782.18800171,345.38264336)(782.21300168,345.49264325)(782.26300781,345.57265076)
\curveto(782.30300159,345.62264312)(782.34800155,345.65264309)(782.39800781,345.66265076)
\curveto(782.45800144,345.68264306)(782.52800137,345.70264304)(782.60800781,345.72265076)
\lineto(782.83300781,345.72265076)
\curveto(782.95300094,345.72264302)(783.05800084,345.71764302)(783.14800781,345.70765076)
\curveto(783.24800065,345.69764304)(783.32300057,345.65264309)(783.37300781,345.57265076)
\curveto(783.42300047,345.52264322)(783.44800045,345.44764329)(783.44800781,345.34765076)
\lineto(783.44800781,345.06265076)
\lineto(783.44800781,344.04265076)
\lineto(783.44800781,340.00765076)
\lineto(783.44800781,338.65765076)
\curveto(783.44800045,338.5376502)(783.44300045,338.42265032)(783.43300781,338.31265076)
\curveto(783.43300046,338.21265053)(783.3980005,338.1376506)(783.32800781,338.08765076)
\curveto(783.28800061,338.05765068)(783.22800067,338.03265071)(783.14800781,338.01265076)
\curveto(783.06800083,338.00265074)(782.97800092,337.99265075)(782.87800781,337.98265076)
\curveto(782.78800111,337.98265076)(782.6980012,337.98765075)(782.60800781,337.99765076)
\curveto(782.52800137,338.00765073)(782.46800143,338.02765071)(782.42800781,338.05765076)
\curveto(782.37800152,338.09765064)(782.33300156,338.16265058)(782.29300781,338.25265076)
\curveto(782.28300161,338.29265045)(782.27300162,338.34765039)(782.26300781,338.41765076)
\curveto(782.26300163,338.48765025)(782.25800164,338.55265019)(782.24800781,338.61265076)
\curveto(782.23800166,338.68265006)(782.21800168,338.73765)(782.18800781,338.77765076)
\curveto(782.15800174,338.81764992)(782.11300178,338.83264991)(782.05300781,338.82265076)
\curveto(781.97300192,338.80264994)(781.893002,338.74265)(781.81300781,338.64265076)
\curveto(781.73300216,338.55265019)(781.65800224,338.48265026)(781.58800781,338.43265076)
\curveto(781.36800253,338.27265047)(781.11800278,338.13265061)(780.83800781,338.01265076)
\curveto(780.72800317,337.96265078)(780.61300328,337.93265081)(780.49300781,337.92265076)
\curveto(780.38300351,337.90265084)(780.26800363,337.87765086)(780.14800781,337.84765076)
\curveto(780.0980038,337.8376509)(780.04300385,337.8376509)(779.98300781,337.84765076)
\curveto(779.93300396,337.85765088)(779.88300401,337.85265089)(779.83300781,337.83265076)
\curveto(779.73300416,337.81265093)(779.64300425,337.81265093)(779.56300781,337.83265076)
\lineto(779.41300781,337.83265076)
\curveto(779.36300453,337.85265089)(779.30300459,337.86265088)(779.23300781,337.86265076)
\curveto(779.17300472,337.86265088)(779.11800478,337.86765087)(779.06800781,337.87765076)
\curveto(779.02800487,337.89765084)(778.98800491,337.90765083)(778.94800781,337.90765076)
\curveto(778.91800498,337.89765084)(778.87800502,337.90265084)(778.82800781,337.92265076)
\lineto(778.58800781,337.98265076)
\curveto(778.51800538,338.00265074)(778.44300545,338.03265071)(778.36300781,338.07265076)
\curveto(778.10300579,338.18265056)(777.88300601,338.32765041)(777.70300781,338.50765076)
\curveto(777.53300636,338.69765004)(777.3930065,338.92264982)(777.28300781,339.18265076)
\curveto(777.24300665,339.27264947)(777.21300668,339.36264938)(777.19300781,339.45265076)
\lineto(777.13300781,339.75265076)
\curveto(777.11300678,339.81264893)(777.10300679,339.86764887)(777.10300781,339.91765076)
\curveto(777.11300678,339.97764876)(777.10800679,340.0426487)(777.08800781,340.11265076)
\curveto(777.07800682,340.13264861)(777.07300682,340.15764858)(777.07300781,340.18765076)
\curveto(777.07300682,340.22764851)(777.06800683,340.26264848)(777.05800781,340.29265076)
\lineto(777.05800781,340.44265076)
\curveto(777.04800685,340.48264826)(777.04300685,340.52764821)(777.04300781,340.57765076)
\curveto(777.05300684,340.6376481)(777.05800684,340.69264805)(777.05800781,340.74265076)
\lineto(777.05800781,341.34265076)
\lineto(777.05800781,344.10265076)
\lineto(777.05800781,345.06265076)
\lineto(777.05800781,345.33265076)
\curveto(777.05800684,345.42264332)(777.07800682,345.49764324)(777.11800781,345.55765076)
\curveto(777.15800674,345.62764311)(777.23300666,345.67764306)(777.34300781,345.70765076)
\curveto(777.36300653,345.71764302)(777.38300651,345.71764302)(777.40300781,345.70765076)
\curveto(777.42300647,345.70764303)(777.44300645,345.71264303)(777.46300781,345.72265076)
}
}
{
\newrgbcolor{curcolor}{0 0 0}
\pscustom[linestyle=none,fillstyle=solid,fillcolor=curcolor]
{
\newpath
\moveto(792.30761719,338.80765076)
\lineto(792.30761719,338.41765076)
\curveto(792.30760931,338.29765044)(792.28260934,338.19765054)(792.23261719,338.11765076)
\curveto(792.18260944,338.04765069)(792.09760952,338.00765073)(791.97761719,337.99765076)
\lineto(791.63261719,337.99765076)
\curveto(791.57261005,337.99765074)(791.51261011,337.99265075)(791.45261719,337.98265076)
\curveto(791.40261022,337.98265076)(791.35761026,337.99265075)(791.31761719,338.01265076)
\curveto(791.22761039,338.03265071)(791.16761045,338.07265067)(791.13761719,338.13265076)
\curveto(791.09761052,338.18265056)(791.07261055,338.2426505)(791.06261719,338.31265076)
\curveto(791.06261056,338.38265036)(791.04761057,338.45265029)(791.01761719,338.52265076)
\curveto(791.00761061,338.5426502)(790.99261063,338.55765018)(790.97261719,338.56765076)
\curveto(790.96261066,338.58765015)(790.94761067,338.60765013)(790.92761719,338.62765076)
\curveto(790.82761079,338.6376501)(790.74761087,338.61765012)(790.68761719,338.56765076)
\curveto(790.63761098,338.51765022)(790.58261104,338.46765027)(790.52261719,338.41765076)
\curveto(790.3226113,338.26765047)(790.1226115,338.15265059)(789.92261719,338.07265076)
\curveto(789.74261188,337.99265075)(789.53261209,337.93265081)(789.29261719,337.89265076)
\curveto(789.06261256,337.85265089)(788.8226128,337.83265091)(788.57261719,337.83265076)
\curveto(788.33261329,337.82265092)(788.09261353,337.8376509)(787.85261719,337.87765076)
\curveto(787.61261401,337.90765083)(787.40261422,337.96265078)(787.22261719,338.04265076)
\curveto(786.70261492,338.26265048)(786.28261534,338.55765018)(785.96261719,338.92765076)
\curveto(785.64261598,339.30764943)(785.39261623,339.77764896)(785.21261719,340.33765076)
\curveto(785.17261645,340.42764831)(785.14261648,340.51764822)(785.12261719,340.60765076)
\curveto(785.11261651,340.70764803)(785.09261653,340.80764793)(785.06261719,340.90765076)
\curveto(785.05261657,340.95764778)(785.04761657,341.00764773)(785.04761719,341.05765076)
\curveto(785.04761657,341.10764763)(785.04261658,341.15764758)(785.03261719,341.20765076)
\curveto(785.01261661,341.25764748)(785.00261662,341.30764743)(785.00261719,341.35765076)
\curveto(785.01261661,341.41764732)(785.01261661,341.47264727)(785.00261719,341.52265076)
\lineto(785.00261719,341.67265076)
\curveto(784.98261664,341.72264702)(784.97261665,341.78764695)(784.97261719,341.86765076)
\curveto(784.97261665,341.94764679)(784.98261664,342.01264673)(785.00261719,342.06265076)
\lineto(785.00261719,342.22765076)
\curveto(785.0226166,342.29764644)(785.02761659,342.36764637)(785.01761719,342.43765076)
\curveto(785.0176166,342.51764622)(785.02761659,342.59264615)(785.04761719,342.66265076)
\curveto(785.05761656,342.71264603)(785.06261656,342.75764598)(785.06261719,342.79765076)
\curveto(785.06261656,342.8376459)(785.06761655,342.88264586)(785.07761719,342.93265076)
\curveto(785.10761651,343.03264571)(785.13261649,343.12764561)(785.15261719,343.21765076)
\curveto(785.17261645,343.31764542)(785.19761642,343.41264533)(785.22761719,343.50265076)
\curveto(785.35761626,343.88264486)(785.5226161,344.22264452)(785.72261719,344.52265076)
\curveto(785.93261569,344.83264391)(786.18261544,345.08764365)(786.47261719,345.28765076)
\curveto(786.64261498,345.40764333)(786.8176148,345.50764323)(786.99761719,345.58765076)
\curveto(787.18761443,345.66764307)(787.39261423,345.737643)(787.61261719,345.79765076)
\curveto(787.68261394,345.80764293)(787.74761387,345.81764292)(787.80761719,345.82765076)
\curveto(787.87761374,345.8376429)(787.94761367,345.85264289)(788.01761719,345.87265076)
\lineto(788.16761719,345.87265076)
\curveto(788.24761337,345.89264285)(788.36261326,345.90264284)(788.51261719,345.90265076)
\curveto(788.67261295,345.90264284)(788.79261283,345.89264285)(788.87261719,345.87265076)
\curveto(788.91261271,345.86264288)(788.96761265,345.85764288)(789.03761719,345.85765076)
\curveto(789.14761247,345.82764291)(789.25761236,345.80264294)(789.36761719,345.78265076)
\curveto(789.47761214,345.77264297)(789.58261204,345.742643)(789.68261719,345.69265076)
\curveto(789.83261179,345.63264311)(789.97261165,345.56764317)(790.10261719,345.49765076)
\curveto(790.24261138,345.42764331)(790.37261125,345.34764339)(790.49261719,345.25765076)
\curveto(790.55261107,345.20764353)(790.61261101,345.15264359)(790.67261719,345.09265076)
\curveto(790.74261088,345.0426437)(790.83261079,345.02764371)(790.94261719,345.04765076)
\curveto(790.96261066,345.07764366)(790.97761064,345.10264364)(790.98761719,345.12265076)
\curveto(791.00761061,345.1426436)(791.0226106,345.17264357)(791.03261719,345.21265076)
\curveto(791.06261056,345.30264344)(791.07261055,345.41764332)(791.06261719,345.55765076)
\lineto(791.06261719,345.93265076)
\lineto(791.06261719,347.65765076)
\lineto(791.06261719,348.12265076)
\curveto(791.06261056,348.30264044)(791.08761053,348.43264031)(791.13761719,348.51265076)
\curveto(791.17761044,348.58264016)(791.23761038,348.62764011)(791.31761719,348.64765076)
\curveto(791.33761028,348.64764009)(791.36261026,348.64764009)(791.39261719,348.64765076)
\curveto(791.4226102,348.65764008)(791.44761017,348.66264008)(791.46761719,348.66265076)
\curveto(791.60761001,348.67264007)(791.75260987,348.67264007)(791.90261719,348.66265076)
\curveto(792.06260956,348.66264008)(792.17260945,348.62264012)(792.23261719,348.54265076)
\curveto(792.28260934,348.46264028)(792.30760931,348.36264038)(792.30761719,348.24265076)
\lineto(792.30761719,347.86765076)
\lineto(792.30761719,338.80765076)
\moveto(791.09261719,341.64265076)
\curveto(791.11261051,341.69264705)(791.1226105,341.75764698)(791.12261719,341.83765076)
\curveto(791.1226105,341.92764681)(791.11261051,341.99764674)(791.09261719,342.04765076)
\lineto(791.09261719,342.27265076)
\curveto(791.07261055,342.36264638)(791.05761056,342.45264629)(791.04761719,342.54265076)
\curveto(791.03761058,342.6426461)(791.0176106,342.73264601)(790.98761719,342.81265076)
\curveto(790.96761065,342.89264585)(790.94761067,342.96764577)(790.92761719,343.03765076)
\curveto(790.9176107,343.10764563)(790.89761072,343.17764556)(790.86761719,343.24765076)
\curveto(790.74761087,343.54764519)(790.59261103,343.81264493)(790.40261719,344.04265076)
\curveto(790.21261141,344.27264447)(789.97261165,344.45264429)(789.68261719,344.58265076)
\curveto(789.58261204,344.63264411)(789.47761214,344.66764407)(789.36761719,344.68765076)
\curveto(789.26761235,344.71764402)(789.15761246,344.742644)(789.03761719,344.76265076)
\curveto(788.95761266,344.78264396)(788.86761275,344.79264395)(788.76761719,344.79265076)
\lineto(788.49761719,344.79265076)
\curveto(788.44761317,344.78264396)(788.40261322,344.77264397)(788.36261719,344.76265076)
\lineto(788.22761719,344.76265076)
\curveto(788.14761347,344.742644)(788.06261356,344.72264402)(787.97261719,344.70265076)
\curveto(787.89261373,344.68264406)(787.81261381,344.65764408)(787.73261719,344.62765076)
\curveto(787.41261421,344.48764425)(787.15261447,344.28264446)(786.95261719,344.01265076)
\curveto(786.76261486,343.75264499)(786.60761501,343.44764529)(786.48761719,343.09765076)
\curveto(786.44761517,342.98764575)(786.4176152,342.87264587)(786.39761719,342.75265076)
\curveto(786.38761523,342.6426461)(786.37261525,342.53264621)(786.35261719,342.42265076)
\curveto(786.35261527,342.38264636)(786.34761527,342.3426464)(786.33761719,342.30265076)
\lineto(786.33761719,342.19765076)
\curveto(786.3176153,342.14764659)(786.30761531,342.09264665)(786.30761719,342.03265076)
\curveto(786.3176153,341.97264677)(786.3226153,341.91764682)(786.32261719,341.86765076)
\lineto(786.32261719,341.53765076)
\curveto(786.3226153,341.4376473)(786.33261529,341.3426474)(786.35261719,341.25265076)
\curveto(786.36261526,341.22264752)(786.36761525,341.17264757)(786.36761719,341.10265076)
\curveto(786.38761523,341.03264771)(786.40261522,340.96264778)(786.41261719,340.89265076)
\lineto(786.47261719,340.68265076)
\curveto(786.58261504,340.33264841)(786.73261489,340.03264871)(786.92261719,339.78265076)
\curveto(787.11261451,339.53264921)(787.35261427,339.32764941)(787.64261719,339.16765076)
\curveto(787.73261389,339.11764962)(787.8226138,339.07764966)(787.91261719,339.04765076)
\curveto(788.00261362,339.01764972)(788.10261352,338.98764975)(788.21261719,338.95765076)
\curveto(788.26261336,338.9376498)(788.31261331,338.93264981)(788.36261719,338.94265076)
\curveto(788.4226132,338.95264979)(788.47761314,338.94764979)(788.52761719,338.92765076)
\curveto(788.56761305,338.91764982)(788.60761301,338.91264983)(788.64761719,338.91265076)
\lineto(788.78261719,338.91265076)
\lineto(788.91761719,338.91265076)
\curveto(788.94761267,338.92264982)(788.99761262,338.92764981)(789.06761719,338.92765076)
\curveto(789.14761247,338.94764979)(789.22761239,338.96264978)(789.30761719,338.97265076)
\curveto(789.38761223,338.99264975)(789.46261216,339.01764972)(789.53261719,339.04765076)
\curveto(789.86261176,339.18764955)(790.12761149,339.36264938)(790.32761719,339.57265076)
\curveto(790.53761108,339.79264895)(790.71261091,340.06764867)(790.85261719,340.39765076)
\curveto(790.90261072,340.50764823)(790.93761068,340.61764812)(790.95761719,340.72765076)
\curveto(790.97761064,340.8376479)(791.00261062,340.94764779)(791.03261719,341.05765076)
\curveto(791.05261057,341.09764764)(791.06261056,341.13264761)(791.06261719,341.16265076)
\curveto(791.06261056,341.20264754)(791.06761055,341.2426475)(791.07761719,341.28265076)
\curveto(791.08761053,341.3426474)(791.08761053,341.40264734)(791.07761719,341.46265076)
\curveto(791.07761054,341.52264722)(791.08261054,341.58264716)(791.09261719,341.64265076)
}
}
{
\newrgbcolor{curcolor}{0 0 0}
\pscustom[linestyle=none,fillstyle=solid,fillcolor=curcolor]
{
\newpath
\moveto(794.53886719,347.22265076)
\curveto(794.45886607,347.28264146)(794.41386611,347.38764135)(794.40386719,347.53765076)
\lineto(794.40386719,348.00265076)
\lineto(794.40386719,348.25765076)
\curveto(794.40386612,348.34764039)(794.41886611,348.42264032)(794.44886719,348.48265076)
\curveto(794.48886604,348.56264018)(794.56886596,348.62264012)(794.68886719,348.66265076)
\curveto(794.70886582,348.67264007)(794.7288658,348.67264007)(794.74886719,348.66265076)
\curveto(794.77886575,348.66264008)(794.80386572,348.66764007)(794.82386719,348.67765076)
\curveto(794.99386553,348.67764006)(795.15386537,348.67264007)(795.30386719,348.66265076)
\curveto(795.45386507,348.65264009)(795.55386497,348.59264015)(795.60386719,348.48265076)
\curveto(795.63386489,348.42264032)(795.64886488,348.34764039)(795.64886719,348.25765076)
\lineto(795.64886719,348.00265076)
\curveto(795.64886488,347.82264092)(795.64386488,347.65264109)(795.63386719,347.49265076)
\curveto(795.63386489,347.33264141)(795.56886496,347.22764151)(795.43886719,347.17765076)
\curveto(795.38886514,347.15764158)(795.33386519,347.14764159)(795.27386719,347.14765076)
\lineto(795.10886719,347.14765076)
\lineto(794.79386719,347.14765076)
\curveto(794.69386583,347.14764159)(794.60886592,347.17264157)(794.53886719,347.22265076)
\moveto(795.64886719,338.71765076)
\lineto(795.64886719,338.40265076)
\curveto(795.65886487,338.30265044)(795.63886489,338.22265052)(795.58886719,338.16265076)
\curveto(795.55886497,338.10265064)(795.51386501,338.06265068)(795.45386719,338.04265076)
\curveto(795.39386513,338.03265071)(795.3238652,338.01765072)(795.24386719,337.99765076)
\lineto(795.01886719,337.99765076)
\curveto(794.88886564,337.99765074)(794.77386575,338.00265074)(794.67386719,338.01265076)
\curveto(794.58386594,338.03265071)(794.51386601,338.08265066)(794.46386719,338.16265076)
\curveto(794.4238661,338.22265052)(794.40386612,338.29765044)(794.40386719,338.38765076)
\lineto(794.40386719,338.67265076)
\lineto(794.40386719,345.01765076)
\lineto(794.40386719,345.33265076)
\curveto(794.40386612,345.4426433)(794.4288661,345.52764321)(794.47886719,345.58765076)
\curveto(794.50886602,345.6376431)(794.54886598,345.66764307)(794.59886719,345.67765076)
\curveto(794.64886588,345.68764305)(794.70386582,345.70264304)(794.76386719,345.72265076)
\curveto(794.78386574,345.72264302)(794.80386572,345.71764302)(794.82386719,345.70765076)
\curveto(794.85386567,345.70764303)(794.87886565,345.71264303)(794.89886719,345.72265076)
\curveto(795.0288655,345.72264302)(795.15886537,345.71764302)(795.28886719,345.70765076)
\curveto(795.4288651,345.70764303)(795.523865,345.66764307)(795.57386719,345.58765076)
\curveto(795.6238649,345.52764321)(795.64886488,345.44764329)(795.64886719,345.34765076)
\lineto(795.64886719,345.06265076)
\lineto(795.64886719,338.71765076)
}
}
{
\newrgbcolor{curcolor}{0 0 0}
\pscustom[linestyle=none,fillstyle=solid,fillcolor=curcolor]
{
\newpath
\moveto(804.34371094,342.16765076)
\curveto(804.36370325,342.06764667)(804.36370325,341.95264679)(804.34371094,341.82265076)
\curveto(804.33370328,341.70264704)(804.30370331,341.61764712)(804.25371094,341.56765076)
\curveto(804.20370341,341.52764721)(804.12870349,341.49764724)(804.02871094,341.47765076)
\curveto(803.93870368,341.46764727)(803.83370378,341.46264728)(803.71371094,341.46265076)
\lineto(803.35371094,341.46265076)
\curveto(803.23370438,341.47264727)(803.12870449,341.47764726)(803.03871094,341.47765076)
\lineto(799.19871094,341.47765076)
\curveto(799.1187085,341.47764726)(799.03870858,341.47264727)(798.95871094,341.46265076)
\curveto(798.87870874,341.46264728)(798.8137088,341.44764729)(798.76371094,341.41765076)
\curveto(798.72370889,341.39764734)(798.68370893,341.35764738)(798.64371094,341.29765076)
\curveto(798.62370899,341.26764747)(798.60370901,341.22264752)(798.58371094,341.16265076)
\curveto(798.56370905,341.11264763)(798.56370905,341.06264768)(798.58371094,341.01265076)
\curveto(798.59370902,340.96264778)(798.59870902,340.91764782)(798.59871094,340.87765076)
\curveto(798.59870902,340.8376479)(798.60370901,340.79764794)(798.61371094,340.75765076)
\curveto(798.63370898,340.67764806)(798.65370896,340.59264815)(798.67371094,340.50265076)
\curveto(798.69370892,340.42264832)(798.72370889,340.3426484)(798.76371094,340.26265076)
\curveto(798.99370862,339.72264902)(799.37370824,339.3376494)(799.90371094,339.10765076)
\curveto(799.96370765,339.07764966)(800.02870759,339.05264969)(800.09871094,339.03265076)
\lineto(800.30871094,338.97265076)
\curveto(800.33870728,338.96264978)(800.38870723,338.95764978)(800.45871094,338.95765076)
\curveto(800.59870702,338.91764982)(800.78370683,338.89764984)(801.01371094,338.89765076)
\curveto(801.24370637,338.89764984)(801.42870619,338.91764982)(801.56871094,338.95765076)
\curveto(801.70870591,338.99764974)(801.83370578,339.0376497)(801.94371094,339.07765076)
\curveto(802.06370555,339.12764961)(802.17370544,339.18764955)(802.27371094,339.25765076)
\curveto(802.38370523,339.32764941)(802.47870514,339.40764933)(802.55871094,339.49765076)
\curveto(802.63870498,339.59764914)(802.70870491,339.70264904)(802.76871094,339.81265076)
\curveto(802.82870479,339.91264883)(802.87870474,340.01764872)(802.91871094,340.12765076)
\curveto(802.96870465,340.2376485)(803.04870457,340.31764842)(803.15871094,340.36765076)
\curveto(803.19870442,340.38764835)(803.26370435,340.40264834)(803.35371094,340.41265076)
\curveto(803.44370417,340.42264832)(803.53370408,340.42264832)(803.62371094,340.41265076)
\curveto(803.7137039,340.41264833)(803.79870382,340.40764833)(803.87871094,340.39765076)
\curveto(803.95870366,340.38764835)(804.0137036,340.36764837)(804.04371094,340.33765076)
\curveto(804.14370347,340.26764847)(804.16870345,340.15264859)(804.11871094,339.99265076)
\curveto(804.03870358,339.72264902)(803.93370368,339.48264926)(803.80371094,339.27265076)
\curveto(803.60370401,338.95264979)(803.37370424,338.68765005)(803.11371094,338.47765076)
\curveto(802.86370475,338.27765046)(802.54370507,338.11265063)(802.15371094,337.98265076)
\curveto(802.05370556,337.9426508)(801.95370566,337.91765082)(801.85371094,337.90765076)
\curveto(801.75370586,337.88765085)(801.64870597,337.86765087)(801.53871094,337.84765076)
\curveto(801.48870613,337.8376509)(801.43870618,337.83265091)(801.38871094,337.83265076)
\curveto(801.34870627,337.83265091)(801.30370631,337.82765091)(801.25371094,337.81765076)
\lineto(801.10371094,337.81765076)
\curveto(801.05370656,337.80765093)(800.99370662,337.80265094)(800.92371094,337.80265076)
\curveto(800.86370675,337.80265094)(800.8137068,337.80765093)(800.77371094,337.81765076)
\lineto(800.63871094,337.81765076)
\curveto(800.58870703,337.82765091)(800.54370707,337.83265091)(800.50371094,337.83265076)
\curveto(800.46370715,337.83265091)(800.42370719,337.8376509)(800.38371094,337.84765076)
\curveto(800.33370728,337.85765088)(800.27870734,337.86765087)(800.21871094,337.87765076)
\curveto(800.15870746,337.87765086)(800.10370751,337.88265086)(800.05371094,337.89265076)
\curveto(799.96370765,337.91265083)(799.87370774,337.9376508)(799.78371094,337.96765076)
\curveto(799.69370792,337.98765075)(799.60870801,338.01265073)(799.52871094,338.04265076)
\curveto(799.48870813,338.06265068)(799.45370816,338.07265067)(799.42371094,338.07265076)
\curveto(799.39370822,338.08265066)(799.35870826,338.09765064)(799.31871094,338.11765076)
\curveto(799.16870845,338.18765055)(799.00870861,338.27265047)(798.83871094,338.37265076)
\curveto(798.54870907,338.56265018)(798.29870932,338.79264995)(798.08871094,339.06265076)
\curveto(797.88870973,339.3426494)(797.7187099,339.65264909)(797.57871094,339.99265076)
\curveto(797.52871009,340.10264864)(797.48871013,340.21764852)(797.45871094,340.33765076)
\curveto(797.43871018,340.45764828)(797.40871021,340.57764816)(797.36871094,340.69765076)
\curveto(797.35871026,340.737648)(797.35371026,340.77264797)(797.35371094,340.80265076)
\curveto(797.35371026,340.83264791)(797.34871027,340.87264787)(797.33871094,340.92265076)
\curveto(797.3187103,341.00264774)(797.30371031,341.08764765)(797.29371094,341.17765076)
\curveto(797.28371033,341.26764747)(797.26871035,341.35764738)(797.24871094,341.44765076)
\lineto(797.24871094,341.65765076)
\curveto(797.23871038,341.69764704)(797.22871039,341.75264699)(797.21871094,341.82265076)
\curveto(797.2187104,341.90264684)(797.22371039,341.96764677)(797.23371094,342.01765076)
\lineto(797.23371094,342.18265076)
\curveto(797.25371036,342.23264651)(797.25871036,342.28264646)(797.24871094,342.33265076)
\curveto(797.24871037,342.39264635)(797.25371036,342.44764629)(797.26371094,342.49765076)
\curveto(797.30371031,342.65764608)(797.33371028,342.81764592)(797.35371094,342.97765076)
\curveto(797.38371023,343.1376456)(797.42871019,343.28764545)(797.48871094,343.42765076)
\curveto(797.53871008,343.5376452)(797.58371003,343.64764509)(797.62371094,343.75765076)
\curveto(797.67370994,343.87764486)(797.72870989,343.99264475)(797.78871094,344.10265076)
\curveto(798.00870961,344.45264429)(798.25870936,344.75264399)(798.53871094,345.00265076)
\curveto(798.8187088,345.26264348)(799.16370845,345.47764326)(799.57371094,345.64765076)
\curveto(799.69370792,345.69764304)(799.8137078,345.73264301)(799.93371094,345.75265076)
\curveto(800.06370755,345.78264296)(800.19870742,345.81264293)(800.33871094,345.84265076)
\curveto(800.38870723,345.85264289)(800.43370718,345.85764288)(800.47371094,345.85765076)
\curveto(800.5137071,345.86764287)(800.55870706,345.87264287)(800.60871094,345.87265076)
\curveto(800.62870699,345.88264286)(800.65370696,345.88264286)(800.68371094,345.87265076)
\curveto(800.7137069,345.86264288)(800.73870688,345.86764287)(800.75871094,345.88765076)
\curveto(801.17870644,345.89764284)(801.54370607,345.85264289)(801.85371094,345.75265076)
\curveto(802.16370545,345.66264308)(802.44370517,345.5376432)(802.69371094,345.37765076)
\curveto(802.74370487,345.35764338)(802.78370483,345.32764341)(802.81371094,345.28765076)
\curveto(802.84370477,345.25764348)(802.87870474,345.23264351)(802.91871094,345.21265076)
\curveto(802.99870462,345.15264359)(803.07870454,345.08264366)(803.15871094,345.00265076)
\curveto(803.24870437,344.92264382)(803.32370429,344.8426439)(803.38371094,344.76265076)
\curveto(803.54370407,344.55264419)(803.67870394,344.35264439)(803.78871094,344.16265076)
\curveto(803.85870376,344.05264469)(803.9137037,343.93264481)(803.95371094,343.80265076)
\curveto(803.99370362,343.67264507)(804.03870358,343.5426452)(804.08871094,343.41265076)
\curveto(804.13870348,343.28264546)(804.17370344,343.14764559)(804.19371094,343.00765076)
\curveto(804.22370339,342.86764587)(804.25870336,342.72764601)(804.29871094,342.58765076)
\curveto(804.30870331,342.51764622)(804.3137033,342.44764629)(804.31371094,342.37765076)
\lineto(804.34371094,342.16765076)
\moveto(802.88871094,342.67765076)
\curveto(802.9187047,342.71764602)(802.94370467,342.76764597)(802.96371094,342.82765076)
\curveto(802.98370463,342.89764584)(802.98370463,342.96764577)(802.96371094,343.03765076)
\curveto(802.90370471,343.25764548)(802.8187048,343.46264528)(802.70871094,343.65265076)
\curveto(802.56870505,343.88264486)(802.4137052,344.07764466)(802.24371094,344.23765076)
\curveto(802.07370554,344.39764434)(801.85370576,344.53264421)(801.58371094,344.64265076)
\curveto(801.5137061,344.66264408)(801.44370617,344.67764406)(801.37371094,344.68765076)
\curveto(801.30370631,344.70764403)(801.22870639,344.72764401)(801.14871094,344.74765076)
\curveto(801.06870655,344.76764397)(800.98370663,344.77764396)(800.89371094,344.77765076)
\lineto(800.63871094,344.77765076)
\curveto(800.60870701,344.75764398)(800.57370704,344.74764399)(800.53371094,344.74765076)
\curveto(800.49370712,344.75764398)(800.45870716,344.75764398)(800.42871094,344.74765076)
\lineto(800.18871094,344.68765076)
\curveto(800.1187075,344.67764406)(800.04870757,344.66264408)(799.97871094,344.64265076)
\curveto(799.68870793,344.52264422)(799.45370816,344.37264437)(799.27371094,344.19265076)
\curveto(799.10370851,344.01264473)(798.94870867,343.78764495)(798.80871094,343.51765076)
\curveto(798.77870884,343.46764527)(798.74870887,343.40264534)(798.71871094,343.32265076)
\curveto(798.68870893,343.25264549)(798.66370895,343.17264557)(798.64371094,343.08265076)
\curveto(798.62370899,342.99264575)(798.618709,342.90764583)(798.62871094,342.82765076)
\curveto(798.63870898,342.74764599)(798.67370894,342.68764605)(798.73371094,342.64765076)
\curveto(798.8137088,342.58764615)(798.94870867,342.55764618)(799.13871094,342.55765076)
\curveto(799.33870828,342.56764617)(799.50870811,342.57264617)(799.64871094,342.57265076)
\lineto(801.92871094,342.57265076)
\curveto(802.07870554,342.57264617)(802.25870536,342.56764617)(802.46871094,342.55765076)
\curveto(802.67870494,342.55764618)(802.8187048,342.59764614)(802.88871094,342.67765076)
}
}
{
\newrgbcolor{curcolor}{0 0 0}
\pscustom[linestyle=none,fillstyle=solid,fillcolor=curcolor]
{
\newpath
\moveto(809.34035156,345.87265076)
\curveto(809.97034633,345.89264285)(810.47534582,345.80764293)(810.85535156,345.61765076)
\curveto(811.23534506,345.42764331)(811.54034476,345.1426436)(811.77035156,344.76265076)
\curveto(811.83034447,344.66264408)(811.87534442,344.55264419)(811.90535156,344.43265076)
\curveto(811.94534435,344.32264442)(811.98034432,344.20764453)(812.01035156,344.08765076)
\curveto(812.06034424,343.89764484)(812.09034421,343.69264505)(812.10035156,343.47265076)
\curveto(812.11034419,343.25264549)(812.11534418,343.02764571)(812.11535156,342.79765076)
\lineto(812.11535156,341.19265076)
\lineto(812.11535156,338.85265076)
\curveto(812.11534418,338.68265006)(812.11034419,338.51265023)(812.10035156,338.34265076)
\curveto(812.1003442,338.17265057)(812.03534426,338.06265068)(811.90535156,338.01265076)
\curveto(811.85534444,337.99265075)(811.8003445,337.98265076)(811.74035156,337.98265076)
\curveto(811.69034461,337.97265077)(811.63534466,337.96765077)(811.57535156,337.96765076)
\curveto(811.44534485,337.96765077)(811.32034498,337.97265077)(811.20035156,337.98265076)
\curveto(811.08034522,337.98265076)(810.9953453,338.02265072)(810.94535156,338.10265076)
\curveto(810.8953454,338.17265057)(810.87034543,338.26265048)(810.87035156,338.37265076)
\lineto(810.87035156,338.70265076)
\lineto(810.87035156,339.99265076)
\lineto(810.87035156,342.43765076)
\curveto(810.87034543,342.70764603)(810.86534543,342.97264577)(810.85535156,343.23265076)
\curveto(810.84534545,343.50264524)(810.8003455,343.73264501)(810.72035156,343.92265076)
\curveto(810.64034566,344.12264462)(810.52034578,344.28264446)(810.36035156,344.40265076)
\curveto(810.2003461,344.53264421)(810.01534628,344.63264411)(809.80535156,344.70265076)
\curveto(809.74534655,344.72264402)(809.68034662,344.73264401)(809.61035156,344.73265076)
\curveto(809.55034675,344.742644)(809.49034681,344.75764398)(809.43035156,344.77765076)
\curveto(809.38034692,344.78764395)(809.300347,344.78764395)(809.19035156,344.77765076)
\curveto(809.09034721,344.77764396)(809.02034728,344.77264397)(808.98035156,344.76265076)
\curveto(808.94034736,344.742644)(808.90534739,344.73264401)(808.87535156,344.73265076)
\curveto(808.84534745,344.742644)(808.81034749,344.742644)(808.77035156,344.73265076)
\curveto(808.64034766,344.70264404)(808.51534778,344.66764407)(808.39535156,344.62765076)
\curveto(808.28534801,344.59764414)(808.18034812,344.55264419)(808.08035156,344.49265076)
\curveto(808.04034826,344.47264427)(808.00534829,344.45264429)(807.97535156,344.43265076)
\curveto(807.94534835,344.41264433)(807.91034839,344.39264435)(807.87035156,344.37265076)
\curveto(807.52034878,344.12264462)(807.26534903,343.74764499)(807.10535156,343.24765076)
\curveto(807.07534922,343.16764557)(807.05534924,343.08264566)(807.04535156,342.99265076)
\curveto(807.03534926,342.91264583)(807.02034928,342.83264591)(807.00035156,342.75265076)
\curveto(806.98034932,342.70264604)(806.97534932,342.65264609)(806.98535156,342.60265076)
\curveto(806.9953493,342.56264618)(806.99034931,342.52264622)(806.97035156,342.48265076)
\lineto(806.97035156,342.16765076)
\curveto(806.96034934,342.1376466)(806.95534934,342.10264664)(806.95535156,342.06265076)
\curveto(806.96534933,342.02264672)(806.97034933,341.97764676)(806.97035156,341.92765076)
\lineto(806.97035156,341.47765076)
\lineto(806.97035156,340.03765076)
\lineto(806.97035156,338.71765076)
\lineto(806.97035156,338.37265076)
\curveto(806.97034933,338.26265048)(806.94534935,338.17265057)(806.89535156,338.10265076)
\curveto(806.84534945,338.02265072)(806.75534954,337.98265076)(806.62535156,337.98265076)
\curveto(806.50534979,337.97265077)(806.38034992,337.96765077)(806.25035156,337.96765076)
\curveto(806.17035013,337.96765077)(806.0953502,337.97265077)(806.02535156,337.98265076)
\curveto(805.95535034,337.99265075)(805.8953504,338.01765072)(805.84535156,338.05765076)
\curveto(805.76535053,338.10765063)(805.72535057,338.20265054)(805.72535156,338.34265076)
\lineto(805.72535156,338.74765076)
\lineto(805.72535156,340.51765076)
\lineto(805.72535156,344.14765076)
\lineto(805.72535156,345.06265076)
\lineto(805.72535156,345.33265076)
\curveto(805.72535057,345.42264332)(805.74535055,345.49264325)(805.78535156,345.54265076)
\curveto(805.81535048,345.60264314)(805.86535043,345.6426431)(805.93535156,345.66265076)
\curveto(805.97535032,345.67264307)(806.03035027,345.68264306)(806.10035156,345.69265076)
\curveto(806.18035012,345.70264304)(806.26035004,345.70764303)(806.34035156,345.70765076)
\curveto(806.42034988,345.70764303)(806.4953498,345.70264304)(806.56535156,345.69265076)
\curveto(806.64534965,345.68264306)(806.7003496,345.66764307)(806.73035156,345.64765076)
\curveto(806.84034946,345.57764316)(806.89034941,345.48764325)(806.88035156,345.37765076)
\curveto(806.87034943,345.27764346)(806.88534941,345.16264358)(806.92535156,345.03265076)
\curveto(806.94534935,344.97264377)(806.98534931,344.92264382)(807.04535156,344.88265076)
\curveto(807.16534913,344.87264387)(807.26034904,344.91764382)(807.33035156,345.01765076)
\curveto(807.41034889,345.11764362)(807.49034881,345.19764354)(807.57035156,345.25765076)
\curveto(807.71034859,345.35764338)(807.85034845,345.44764329)(807.99035156,345.52765076)
\curveto(808.14034816,345.61764312)(808.31034799,345.69264305)(808.50035156,345.75265076)
\curveto(808.58034772,345.78264296)(808.66534763,345.80264294)(808.75535156,345.81265076)
\curveto(808.85534744,345.82264292)(808.95034735,345.8376429)(809.04035156,345.85765076)
\curveto(809.09034721,345.86764287)(809.14034716,345.87264287)(809.19035156,345.87265076)
\lineto(809.34035156,345.87265076)
}
}
{
\newrgbcolor{curcolor}{0 0 0}
\pscustom[linestyle=none,fillstyle=solid,fillcolor=curcolor]
{
\newpath
\moveto(817.13496094,345.90265076)
\curveto(817.87495615,345.91264283)(818.48995553,345.80264294)(818.97996094,345.57265076)
\curveto(819.47995454,345.35264339)(819.87495415,345.01764372)(820.16496094,344.56765076)
\curveto(820.29495373,344.36764437)(820.40495362,344.12264462)(820.49496094,343.83265076)
\curveto(820.51495351,343.78264496)(820.52995349,343.71764502)(820.53996094,343.63765076)
\curveto(820.54995347,343.55764518)(820.54495348,343.48764525)(820.52496094,343.42765076)
\curveto(820.49495353,343.37764536)(820.44495358,343.33264541)(820.37496094,343.29265076)
\curveto(820.34495368,343.27264547)(820.31495371,343.26264548)(820.28496094,343.26265076)
\curveto(820.25495377,343.27264547)(820.2199538,343.27264547)(820.17996094,343.26265076)
\curveto(820.13995388,343.25264549)(820.09995392,343.24764549)(820.05996094,343.24765076)
\curveto(820.019954,343.25764548)(819.97995404,343.26264548)(819.93996094,343.26265076)
\lineto(819.62496094,343.26265076)
\curveto(819.5249545,343.27264547)(819.43995458,343.30264544)(819.36996094,343.35265076)
\curveto(819.28995473,343.41264533)(819.23495479,343.49764524)(819.20496094,343.60765076)
\curveto(819.17495485,343.71764502)(819.13495489,343.81264493)(819.08496094,343.89265076)
\curveto(818.93495509,344.15264459)(818.73995528,344.35764438)(818.49996094,344.50765076)
\curveto(818.4199556,344.55764418)(818.33495569,344.59764414)(818.24496094,344.62765076)
\curveto(818.15495587,344.66764407)(818.05995596,344.70264404)(817.95996094,344.73265076)
\curveto(817.8199562,344.77264397)(817.63495639,344.79264395)(817.40496094,344.79265076)
\curveto(817.17495685,344.80264394)(816.98495704,344.78264396)(816.83496094,344.73265076)
\curveto(816.76495726,344.71264403)(816.69995732,344.69764404)(816.63996094,344.68765076)
\curveto(816.57995744,344.67764406)(816.51495751,344.66264408)(816.44496094,344.64265076)
\curveto(816.18495784,344.53264421)(815.95495807,344.38264436)(815.75496094,344.19265076)
\curveto(815.55495847,344.00264474)(815.39995862,343.77764496)(815.28996094,343.51765076)
\curveto(815.24995877,343.42764531)(815.21495881,343.33264541)(815.18496094,343.23265076)
\curveto(815.15495887,343.1426456)(815.1249589,343.0426457)(815.09496094,342.93265076)
\lineto(815.00496094,342.52765076)
\curveto(814.99495903,342.47764626)(814.98995903,342.42264632)(814.98996094,342.36265076)
\curveto(814.99995902,342.30264644)(814.99495903,342.24764649)(814.97496094,342.19765076)
\lineto(814.97496094,342.07765076)
\curveto(814.96495906,342.0376467)(814.95495907,341.97264677)(814.94496094,341.88265076)
\curveto(814.94495908,341.79264695)(814.95495907,341.72764701)(814.97496094,341.68765076)
\curveto(814.98495904,341.6376471)(814.98495904,341.58764715)(814.97496094,341.53765076)
\curveto(814.96495906,341.48764725)(814.96495906,341.4376473)(814.97496094,341.38765076)
\curveto(814.98495904,341.34764739)(814.98995903,341.27764746)(814.98996094,341.17765076)
\curveto(815.00995901,341.09764764)(815.024959,341.01264773)(815.03496094,340.92265076)
\curveto(815.05495897,340.83264791)(815.07495895,340.74764799)(815.09496094,340.66765076)
\curveto(815.20495882,340.34764839)(815.32995869,340.06764867)(815.46996094,339.82765076)
\curveto(815.6199584,339.59764914)(815.8249582,339.39764934)(816.08496094,339.22765076)
\curveto(816.17495785,339.17764956)(816.26495776,339.13264961)(816.35496094,339.09265076)
\curveto(816.45495757,339.05264969)(816.55995746,339.01264973)(816.66996094,338.97265076)
\curveto(816.7199573,338.96264978)(816.75995726,338.95764978)(816.78996094,338.95765076)
\curveto(816.8199572,338.95764978)(816.85995716,338.95264979)(816.90996094,338.94265076)
\curveto(816.93995708,338.93264981)(816.98995703,338.92764981)(817.05996094,338.92765076)
\lineto(817.22496094,338.92765076)
\curveto(817.2249568,338.91764982)(817.24495678,338.91264983)(817.28496094,338.91265076)
\curveto(817.30495672,338.92264982)(817.32995669,338.92264982)(817.35996094,338.91265076)
\curveto(817.38995663,338.91264983)(817.4199566,338.91764982)(817.44996094,338.92765076)
\curveto(817.5199565,338.94764979)(817.58495644,338.95264979)(817.64496094,338.94265076)
\curveto(817.71495631,338.9426498)(817.78495624,338.95264979)(817.85496094,338.97265076)
\curveto(818.11495591,339.05264969)(818.33995568,339.15264959)(818.52996094,339.27265076)
\curveto(818.7199553,339.40264934)(818.87995514,339.56764917)(819.00996094,339.76765076)
\curveto(819.05995496,339.84764889)(819.10495492,339.93264881)(819.14496094,340.02265076)
\lineto(819.26496094,340.29265076)
\curveto(819.28495474,340.37264837)(819.30495472,340.44764829)(819.32496094,340.51765076)
\curveto(819.35495467,340.59764814)(819.40495462,340.66264808)(819.47496094,340.71265076)
\curveto(819.50495452,340.742648)(819.56495446,340.76264798)(819.65496094,340.77265076)
\curveto(819.74495428,340.79264795)(819.83995418,340.80264794)(819.93996094,340.80265076)
\curveto(820.04995397,340.81264793)(820.14995387,340.81264793)(820.23996094,340.80265076)
\curveto(820.33995368,340.79264795)(820.40995361,340.77264797)(820.44996094,340.74265076)
\curveto(820.50995351,340.70264804)(820.54495348,340.6426481)(820.55496094,340.56265076)
\curveto(820.57495345,340.48264826)(820.57495345,340.39764834)(820.55496094,340.30765076)
\curveto(820.50495352,340.15764858)(820.45495357,340.01264873)(820.40496094,339.87265076)
\curveto(820.36495366,339.742649)(820.30995371,339.61264913)(820.23996094,339.48265076)
\curveto(820.08995393,339.18264956)(819.89995412,338.91764982)(819.66996094,338.68765076)
\curveto(819.44995457,338.45765028)(819.17995484,338.27265047)(818.85996094,338.13265076)
\curveto(818.77995524,338.09265065)(818.69495533,338.05765068)(818.60496094,338.02765076)
\curveto(818.51495551,338.00765073)(818.4199556,337.98265076)(818.31996094,337.95265076)
\curveto(818.20995581,337.91265083)(818.09995592,337.89265085)(817.98996094,337.89265076)
\curveto(817.87995614,337.88265086)(817.76995625,337.86765087)(817.65996094,337.84765076)
\curveto(817.6199564,337.82765091)(817.57995644,337.82265092)(817.53996094,337.83265076)
\curveto(817.49995652,337.8426509)(817.45995656,337.8426509)(817.41996094,337.83265076)
\lineto(817.28496094,337.83265076)
\lineto(817.04496094,337.83265076)
\curveto(816.97495705,337.82265092)(816.90995711,337.82765091)(816.84996094,337.84765076)
\lineto(816.77496094,337.84765076)
\lineto(816.41496094,337.89265076)
\curveto(816.28495774,337.93265081)(816.15995786,337.96765077)(816.03996094,337.99765076)
\curveto(815.9199581,338.02765071)(815.80495822,338.06765067)(815.69496094,338.11765076)
\curveto(815.33495869,338.27765046)(815.03495899,338.46765027)(814.79496094,338.68765076)
\curveto(814.56495946,338.90764983)(814.34995967,339.17764956)(814.14996094,339.49765076)
\curveto(814.09995992,339.57764916)(814.05495997,339.66764907)(814.01496094,339.76765076)
\lineto(813.89496094,340.06765076)
\curveto(813.84496018,340.17764856)(813.80996021,340.29264845)(813.78996094,340.41265076)
\curveto(813.76996025,340.53264821)(813.74496028,340.65264809)(813.71496094,340.77265076)
\curveto(813.70496032,340.81264793)(813.69996032,340.85264789)(813.69996094,340.89265076)
\curveto(813.69996032,340.93264781)(813.69496033,340.97264777)(813.68496094,341.01265076)
\curveto(813.66496036,341.07264767)(813.65496037,341.1376476)(813.65496094,341.20765076)
\curveto(813.66496036,341.27764746)(813.65996036,341.3426474)(813.63996094,341.40265076)
\lineto(813.63996094,341.55265076)
\curveto(813.62996039,341.60264714)(813.6249604,341.67264707)(813.62496094,341.76265076)
\curveto(813.6249604,341.85264689)(813.62996039,341.92264682)(813.63996094,341.97265076)
\curveto(813.64996037,342.02264672)(813.64996037,342.06764667)(813.63996094,342.10765076)
\curveto(813.63996038,342.14764659)(813.64496038,342.18764655)(813.65496094,342.22765076)
\curveto(813.67496035,342.29764644)(813.67996034,342.36764637)(813.66996094,342.43765076)
\curveto(813.66996035,342.50764623)(813.67996034,342.57264617)(813.69996094,342.63265076)
\curveto(813.73996028,342.80264594)(813.77496025,342.97264577)(813.80496094,343.14265076)
\curveto(813.83496019,343.31264543)(813.87996014,343.47264527)(813.93996094,343.62265076)
\curveto(814.14995987,344.1426446)(814.40495962,344.56264418)(814.70496094,344.88265076)
\curveto(815.00495902,345.20264354)(815.41495861,345.46764327)(815.93496094,345.67765076)
\curveto(816.04495798,345.72764301)(816.16495786,345.76264298)(816.29496094,345.78265076)
\curveto(816.4249576,345.80264294)(816.55995746,345.82764291)(816.69996094,345.85765076)
\curveto(816.76995725,345.86764287)(816.83995718,345.87264287)(816.90996094,345.87265076)
\curveto(816.97995704,345.88264286)(817.05495697,345.89264285)(817.13496094,345.90265076)
}
}
{
\newrgbcolor{curcolor}{0 0 0}
\pscustom[linestyle=none,fillstyle=solid,fillcolor=curcolor]
{
\newpath
\moveto(822.34160156,347.22265076)
\curveto(822.26160044,347.28264146)(822.21660049,347.38764135)(822.20660156,347.53765076)
\lineto(822.20660156,348.00265076)
\lineto(822.20660156,348.25765076)
\curveto(822.2066005,348.34764039)(822.22160048,348.42264032)(822.25160156,348.48265076)
\curveto(822.29160041,348.56264018)(822.37160033,348.62264012)(822.49160156,348.66265076)
\curveto(822.51160019,348.67264007)(822.53160017,348.67264007)(822.55160156,348.66265076)
\curveto(822.58160012,348.66264008)(822.6066001,348.66764007)(822.62660156,348.67765076)
\curveto(822.79659991,348.67764006)(822.95659975,348.67264007)(823.10660156,348.66265076)
\curveto(823.25659945,348.65264009)(823.35659935,348.59264015)(823.40660156,348.48265076)
\curveto(823.43659927,348.42264032)(823.45159925,348.34764039)(823.45160156,348.25765076)
\lineto(823.45160156,348.00265076)
\curveto(823.45159925,347.82264092)(823.44659926,347.65264109)(823.43660156,347.49265076)
\curveto(823.43659927,347.33264141)(823.37159933,347.22764151)(823.24160156,347.17765076)
\curveto(823.19159951,347.15764158)(823.13659957,347.14764159)(823.07660156,347.14765076)
\lineto(822.91160156,347.14765076)
\lineto(822.59660156,347.14765076)
\curveto(822.49660021,347.14764159)(822.41160029,347.17264157)(822.34160156,347.22265076)
\moveto(823.45160156,338.71765076)
\lineto(823.45160156,338.40265076)
\curveto(823.46159924,338.30265044)(823.44159926,338.22265052)(823.39160156,338.16265076)
\curveto(823.36159934,338.10265064)(823.31659939,338.06265068)(823.25660156,338.04265076)
\curveto(823.19659951,338.03265071)(823.12659958,338.01765072)(823.04660156,337.99765076)
\lineto(822.82160156,337.99765076)
\curveto(822.69160001,337.99765074)(822.57660013,338.00265074)(822.47660156,338.01265076)
\curveto(822.38660032,338.03265071)(822.31660039,338.08265066)(822.26660156,338.16265076)
\curveto(822.22660048,338.22265052)(822.2066005,338.29765044)(822.20660156,338.38765076)
\lineto(822.20660156,338.67265076)
\lineto(822.20660156,345.01765076)
\lineto(822.20660156,345.33265076)
\curveto(822.2066005,345.4426433)(822.23160047,345.52764321)(822.28160156,345.58765076)
\curveto(822.31160039,345.6376431)(822.35160035,345.66764307)(822.40160156,345.67765076)
\curveto(822.45160025,345.68764305)(822.5066002,345.70264304)(822.56660156,345.72265076)
\curveto(822.58660012,345.72264302)(822.6066001,345.71764302)(822.62660156,345.70765076)
\curveto(822.65660005,345.70764303)(822.68160002,345.71264303)(822.70160156,345.72265076)
\curveto(822.83159987,345.72264302)(822.96159974,345.71764302)(823.09160156,345.70765076)
\curveto(823.23159947,345.70764303)(823.32659938,345.66764307)(823.37660156,345.58765076)
\curveto(823.42659928,345.52764321)(823.45159925,345.44764329)(823.45160156,345.34765076)
\lineto(823.45160156,345.06265076)
\lineto(823.45160156,338.71765076)
}
}
{
\newrgbcolor{curcolor}{0 0 0}
\pscustom[linestyle=none,fillstyle=solid,fillcolor=curcolor]
{
\newpath
\moveto(832.28144531,338.55265076)
\curveto(832.31143748,338.39265035)(832.2964375,338.25765048)(832.23644531,338.14765076)
\curveto(832.17643762,338.04765069)(832.0964377,337.97265077)(831.99644531,337.92265076)
\curveto(831.94643785,337.90265084)(831.8914379,337.89265085)(831.83144531,337.89265076)
\curveto(831.78143801,337.89265085)(831.72643807,337.88265086)(831.66644531,337.86265076)
\curveto(831.44643835,337.81265093)(831.22643857,337.82765091)(831.00644531,337.90765076)
\curveto(830.796439,337.97765076)(830.65143914,338.06765067)(830.57144531,338.17765076)
\curveto(830.52143927,338.24765049)(830.47643932,338.32765041)(830.43644531,338.41765076)
\curveto(830.3964394,338.51765022)(830.34643945,338.59765014)(830.28644531,338.65765076)
\curveto(830.26643953,338.67765006)(830.24143955,338.69765004)(830.21144531,338.71765076)
\curveto(830.1914396,338.73765)(830.16143963,338.74265)(830.12144531,338.73265076)
\curveto(830.01143978,338.70265004)(829.90643989,338.64765009)(829.80644531,338.56765076)
\curveto(829.71644008,338.48765025)(829.62644017,338.41765032)(829.53644531,338.35765076)
\curveto(829.40644039,338.27765046)(829.26644053,338.20265054)(829.11644531,338.13265076)
\curveto(828.96644083,338.07265067)(828.80644099,338.01765072)(828.63644531,337.96765076)
\curveto(828.53644126,337.9376508)(828.42644137,337.91765082)(828.30644531,337.90765076)
\curveto(828.1964416,337.89765084)(828.08644171,337.88265086)(827.97644531,337.86265076)
\curveto(827.92644187,337.85265089)(827.88144191,337.84765089)(827.84144531,337.84765076)
\lineto(827.73644531,337.84765076)
\curveto(827.62644217,337.82765091)(827.52144227,337.82765091)(827.42144531,337.84765076)
\lineto(827.28644531,337.84765076)
\curveto(827.23644256,337.85765088)(827.18644261,337.86265088)(827.13644531,337.86265076)
\curveto(827.08644271,337.86265088)(827.04144275,337.87265087)(827.00144531,337.89265076)
\curveto(826.96144283,337.90265084)(826.92644287,337.90765083)(826.89644531,337.90765076)
\curveto(826.87644292,337.89765084)(826.85144294,337.89765084)(826.82144531,337.90765076)
\lineto(826.58144531,337.96765076)
\curveto(826.50144329,337.97765076)(826.42644337,337.99765074)(826.35644531,338.02765076)
\curveto(826.05644374,338.15765058)(825.81144398,338.30265044)(825.62144531,338.46265076)
\curveto(825.44144435,338.63265011)(825.2914445,338.86764987)(825.17144531,339.16765076)
\curveto(825.08144471,339.38764935)(825.03644476,339.65264909)(825.03644531,339.96265076)
\lineto(825.03644531,340.27765076)
\curveto(825.04644475,340.32764841)(825.05144474,340.37764836)(825.05144531,340.42765076)
\lineto(825.08144531,340.60765076)
\lineto(825.20144531,340.93765076)
\curveto(825.24144455,341.04764769)(825.2914445,341.14764759)(825.35144531,341.23765076)
\curveto(825.53144426,341.52764721)(825.77644402,341.742647)(826.08644531,341.88265076)
\curveto(826.3964434,342.02264672)(826.73644306,342.14764659)(827.10644531,342.25765076)
\curveto(827.24644255,342.29764644)(827.3914424,342.32764641)(827.54144531,342.34765076)
\curveto(827.6914421,342.36764637)(827.84144195,342.39264635)(827.99144531,342.42265076)
\curveto(828.06144173,342.4426463)(828.12644167,342.45264629)(828.18644531,342.45265076)
\curveto(828.25644154,342.45264629)(828.33144146,342.46264628)(828.41144531,342.48265076)
\curveto(828.48144131,342.50264624)(828.55144124,342.51264623)(828.62144531,342.51265076)
\curveto(828.6914411,342.52264622)(828.76644103,342.5376462)(828.84644531,342.55765076)
\curveto(829.0964407,342.61764612)(829.33144046,342.66764607)(829.55144531,342.70765076)
\curveto(829.77144002,342.75764598)(829.94643985,342.87264587)(830.07644531,343.05265076)
\curveto(830.13643966,343.13264561)(830.18643961,343.23264551)(830.22644531,343.35265076)
\curveto(830.26643953,343.48264526)(830.26643953,343.62264512)(830.22644531,343.77265076)
\curveto(830.16643963,344.01264473)(830.07643972,344.20264454)(829.95644531,344.34265076)
\curveto(829.84643995,344.48264426)(829.68644011,344.59264415)(829.47644531,344.67265076)
\curveto(829.35644044,344.72264402)(829.21144058,344.75764398)(829.04144531,344.77765076)
\curveto(828.88144091,344.79764394)(828.71144108,344.80764393)(828.53144531,344.80765076)
\curveto(828.35144144,344.80764393)(828.17644162,344.79764394)(828.00644531,344.77765076)
\curveto(827.83644196,344.75764398)(827.6914421,344.72764401)(827.57144531,344.68765076)
\curveto(827.40144239,344.62764411)(827.23644256,344.5426442)(827.07644531,344.43265076)
\curveto(826.9964428,344.37264437)(826.92144287,344.29264445)(826.85144531,344.19265076)
\curveto(826.791443,344.10264464)(826.73644306,344.00264474)(826.68644531,343.89265076)
\curveto(826.65644314,343.81264493)(826.62644317,343.72764501)(826.59644531,343.63765076)
\curveto(826.57644322,343.54764519)(826.53144326,343.47764526)(826.46144531,343.42765076)
\curveto(826.42144337,343.39764534)(826.35144344,343.37264537)(826.25144531,343.35265076)
\curveto(826.16144363,343.3426454)(826.06644373,343.3376454)(825.96644531,343.33765076)
\curveto(825.86644393,343.3376454)(825.76644403,343.3426454)(825.66644531,343.35265076)
\curveto(825.57644422,343.37264537)(825.51144428,343.39764534)(825.47144531,343.42765076)
\curveto(825.43144436,343.45764528)(825.40144439,343.50764523)(825.38144531,343.57765076)
\curveto(825.36144443,343.64764509)(825.36144443,343.72264502)(825.38144531,343.80265076)
\curveto(825.41144438,343.93264481)(825.44144435,344.05264469)(825.47144531,344.16265076)
\curveto(825.51144428,344.28264446)(825.55644424,344.39764434)(825.60644531,344.50765076)
\curveto(825.796444,344.85764388)(826.03644376,345.12764361)(826.32644531,345.31765076)
\curveto(826.61644318,345.51764322)(826.97644282,345.67764306)(827.40644531,345.79765076)
\curveto(827.50644229,345.81764292)(827.60644219,345.83264291)(827.70644531,345.84265076)
\curveto(827.81644198,345.85264289)(827.92644187,345.86764287)(828.03644531,345.88765076)
\curveto(828.07644172,345.89764284)(828.14144165,345.89764284)(828.23144531,345.88765076)
\curveto(828.32144147,345.88764285)(828.37644142,345.89764284)(828.39644531,345.91765076)
\curveto(829.0964407,345.92764281)(829.70644009,345.84764289)(830.22644531,345.67765076)
\curveto(830.74643905,345.50764323)(831.11143868,345.18264356)(831.32144531,344.70265076)
\curveto(831.41143838,344.50264424)(831.46143833,344.26764447)(831.47144531,343.99765076)
\curveto(831.4914383,343.737645)(831.50143829,343.46264528)(831.50144531,343.17265076)
\lineto(831.50144531,339.85765076)
\curveto(831.50143829,339.71764902)(831.50643829,339.58264916)(831.51644531,339.45265076)
\curveto(831.52643827,339.32264942)(831.55643824,339.21764952)(831.60644531,339.13765076)
\curveto(831.65643814,339.06764967)(831.72143807,339.01764972)(831.80144531,338.98765076)
\curveto(831.8914379,338.94764979)(831.97643782,338.91764982)(832.05644531,338.89765076)
\curveto(832.13643766,338.88764985)(832.1964376,338.8426499)(832.23644531,338.76265076)
\curveto(832.25643754,338.73265001)(832.26643753,338.70265004)(832.26644531,338.67265076)
\curveto(832.26643753,338.6426501)(832.27143752,338.60265014)(832.28144531,338.55265076)
\moveto(830.13644531,340.21765076)
\curveto(830.1964396,340.35764838)(830.22643957,340.51764822)(830.22644531,340.69765076)
\curveto(830.23643956,340.88764785)(830.24143955,341.08264766)(830.24144531,341.28265076)
\curveto(830.24143955,341.39264735)(830.23643956,341.49264725)(830.22644531,341.58265076)
\curveto(830.21643958,341.67264707)(830.17643962,341.742647)(830.10644531,341.79265076)
\curveto(830.07643972,341.81264693)(830.00643979,341.82264692)(829.89644531,341.82265076)
\curveto(829.87643992,341.80264694)(829.84143995,341.79264695)(829.79144531,341.79265076)
\curveto(829.74144005,341.79264695)(829.6964401,341.78264696)(829.65644531,341.76265076)
\curveto(829.57644022,341.742647)(829.48644031,341.72264702)(829.38644531,341.70265076)
\lineto(829.08644531,341.64265076)
\curveto(829.05644074,341.6426471)(829.02144077,341.6376471)(828.98144531,341.62765076)
\lineto(828.87644531,341.62765076)
\curveto(828.72644107,341.58764715)(828.56144123,341.56264718)(828.38144531,341.55265076)
\curveto(828.21144158,341.55264719)(828.05144174,341.53264721)(827.90144531,341.49265076)
\curveto(827.82144197,341.47264727)(827.74644205,341.45264729)(827.67644531,341.43265076)
\curveto(827.61644218,341.42264732)(827.54644225,341.40764733)(827.46644531,341.38765076)
\curveto(827.30644249,341.3376474)(827.15644264,341.27264747)(827.01644531,341.19265076)
\curveto(826.87644292,341.12264762)(826.75644304,341.03264771)(826.65644531,340.92265076)
\curveto(826.55644324,340.81264793)(826.48144331,340.67764806)(826.43144531,340.51765076)
\curveto(826.38144341,340.36764837)(826.36144343,340.18264856)(826.37144531,339.96265076)
\curveto(826.37144342,339.86264888)(826.38644341,339.76764897)(826.41644531,339.67765076)
\curveto(826.45644334,339.59764914)(826.50144329,339.52264922)(826.55144531,339.45265076)
\curveto(826.63144316,339.3426494)(826.73644306,339.24764949)(826.86644531,339.16765076)
\curveto(826.9964428,339.09764964)(827.13644266,339.0376497)(827.28644531,338.98765076)
\curveto(827.33644246,338.97764976)(827.38644241,338.97264977)(827.43644531,338.97265076)
\curveto(827.48644231,338.97264977)(827.53644226,338.96764977)(827.58644531,338.95765076)
\curveto(827.65644214,338.9376498)(827.74144205,338.92264982)(827.84144531,338.91265076)
\curveto(827.95144184,338.91264983)(828.04144175,338.92264982)(828.11144531,338.94265076)
\curveto(828.17144162,338.96264978)(828.23144156,338.96764977)(828.29144531,338.95765076)
\curveto(828.35144144,338.95764978)(828.41144138,338.96764977)(828.47144531,338.98765076)
\curveto(828.55144124,339.00764973)(828.62644117,339.02264972)(828.69644531,339.03265076)
\curveto(828.77644102,339.0426497)(828.85144094,339.06264968)(828.92144531,339.09265076)
\curveto(829.21144058,339.21264953)(829.45644034,339.35764938)(829.65644531,339.52765076)
\curveto(829.86643993,339.69764904)(830.02643977,339.92764881)(830.13644531,340.21765076)
}
}
{
\newrgbcolor{curcolor}{0 0 0}
\pscustom[linestyle=none,fillstyle=solid,fillcolor=curcolor]
{
\newpath
\moveto(772.41642334,325.34634644)
\curveto(773.39641684,325.36633548)(774.21641602,325.20633564)(774.87642334,324.86634644)
\curveto(775.54641469,324.53633631)(776.06641417,324.07633677)(776.43642334,323.48634644)
\curveto(776.5364137,323.32633752)(776.61641362,323.17133768)(776.67642334,323.02134644)
\curveto(776.74641349,322.88133797)(776.81141342,322.71133814)(776.87142334,322.51134644)
\curveto(776.89141334,322.46133839)(776.91141332,322.39133846)(776.93142334,322.30134644)
\curveto(776.95141328,322.22133863)(776.94641329,322.1463387)(776.91642334,322.07634644)
\curveto(776.89641334,322.01633883)(776.85641338,321.97633887)(776.79642334,321.95634644)
\curveto(776.74641349,321.9463389)(776.69141354,321.93133892)(776.63142334,321.91134644)
\lineto(776.48142334,321.91134644)
\curveto(776.45141378,321.90133895)(776.41141382,321.89633895)(776.36142334,321.89634644)
\lineto(776.24142334,321.89634644)
\curveto(776.10141413,321.89633895)(775.97141426,321.90133895)(775.85142334,321.91134644)
\curveto(775.74141449,321.93133892)(775.66141457,321.98133887)(775.61142334,322.06134644)
\curveto(775.54141469,322.16133869)(775.48641475,322.27633857)(775.44642334,322.40634644)
\curveto(775.40641483,322.53633831)(775.35141488,322.65633819)(775.28142334,322.76634644)
\curveto(775.15141508,322.98633786)(775.00141523,323.17633767)(774.83142334,323.33634644)
\curveto(774.67141556,323.49633735)(774.48141575,323.6463372)(774.26142334,323.78634644)
\curveto(774.14141609,323.86633698)(774.00641623,323.92633692)(773.85642334,323.96634644)
\curveto(773.71641652,324.00633684)(773.57141666,324.0463368)(773.42142334,324.08634644)
\curveto(773.31141692,324.11633673)(773.18641705,324.13633671)(773.04642334,324.14634644)
\curveto(772.90641733,324.16633668)(772.75641748,324.17633667)(772.59642334,324.17634644)
\curveto(772.44641779,324.17633667)(772.29641794,324.16633668)(772.14642334,324.14634644)
\curveto(772.00641823,324.13633671)(771.88641835,324.11633673)(771.78642334,324.08634644)
\curveto(771.68641855,324.06633678)(771.59141864,324.0463368)(771.50142334,324.02634644)
\curveto(771.41141882,324.00633684)(771.32141891,323.97633687)(771.23142334,323.93634644)
\curveto(770.39141984,323.58633726)(769.78642045,322.98633786)(769.41642334,322.13634644)
\curveto(769.34642089,321.99633885)(769.28642095,321.846339)(769.23642334,321.68634644)
\curveto(769.19642104,321.53633931)(769.15142108,321.38133947)(769.10142334,321.22134644)
\curveto(769.08142115,321.16133969)(769.07142116,321.09633975)(769.07142334,321.02634644)
\curveto(769.07142116,320.96633988)(769.06142117,320.90633994)(769.04142334,320.84634644)
\curveto(769.0314212,320.80634004)(769.02642121,320.77134008)(769.02642334,320.74134644)
\curveto(769.02642121,320.71134014)(769.02142121,320.67634017)(769.01142334,320.63634644)
\curveto(768.99142124,320.52634032)(768.97642126,320.41134044)(768.96642334,320.29134644)
\lineto(768.96642334,319.94634644)
\curveto(768.96642127,319.87634097)(768.96142127,319.80134105)(768.95142334,319.72134644)
\curveto(768.95142128,319.6513412)(768.95642128,319.58634126)(768.96642334,319.52634644)
\lineto(768.96642334,319.37634644)
\curveto(768.98642125,319.30634154)(768.99142124,319.23634161)(768.98142334,319.16634644)
\curveto(768.98142125,319.09634175)(768.99142124,319.02634182)(769.01142334,318.95634644)
\curveto(769.0314212,318.89634195)(769.0364212,318.83634201)(769.02642334,318.77634644)
\curveto(769.02642121,318.71634213)(769.0364212,318.66134219)(769.05642334,318.61134644)
\curveto(769.08642115,318.48134237)(769.11142112,318.3513425)(769.13142334,318.22134644)
\curveto(769.16142107,318.10134275)(769.19642104,317.98134287)(769.23642334,317.86134644)
\curveto(769.40642083,317.36134349)(769.62642061,316.93134392)(769.89642334,316.57134644)
\curveto(770.16642007,316.22134463)(770.52141971,315.93134492)(770.96142334,315.70134644)
\curveto(771.10141913,315.63134522)(771.24141899,315.57634527)(771.38142334,315.53634644)
\curveto(771.5314187,315.49634535)(771.69141854,315.4513454)(771.86142334,315.40134644)
\curveto(771.9314183,315.38134547)(771.99641824,315.37134548)(772.05642334,315.37134644)
\curveto(772.11641812,315.38134547)(772.18641805,315.37634547)(772.26642334,315.35634644)
\curveto(772.31641792,315.3463455)(772.40641783,315.33634551)(772.53642334,315.32634644)
\curveto(772.66641757,315.32634552)(772.76141747,315.33634551)(772.82142334,315.35634644)
\lineto(772.92642334,315.35634644)
\curveto(772.96641727,315.36634548)(773.00641723,315.36634548)(773.04642334,315.35634644)
\curveto(773.08641715,315.35634549)(773.12641711,315.36634548)(773.16642334,315.38634644)
\curveto(773.26641697,315.40634544)(773.36141687,315.42134543)(773.45142334,315.43134644)
\curveto(773.55141668,315.4513454)(773.64641659,315.48134537)(773.73642334,315.52134644)
\curveto(774.51641572,315.84134501)(775.06641517,316.36634448)(775.38642334,317.09634644)
\curveto(775.46641477,317.27634357)(775.54141469,317.49134336)(775.61142334,317.74134644)
\curveto(775.6314146,317.83134302)(775.64641459,317.92134293)(775.65642334,318.01134644)
\curveto(775.67641456,318.11134274)(775.71141452,318.20134265)(775.76142334,318.28134644)
\curveto(775.81141442,318.36134249)(775.89141434,318.40634244)(776.00142334,318.41634644)
\curveto(776.11141412,318.42634242)(776.231414,318.43134242)(776.36142334,318.43134644)
\lineto(776.51142334,318.43134644)
\curveto(776.56141367,318.43134242)(776.60641363,318.42634242)(776.64642334,318.41634644)
\lineto(776.75142334,318.41634644)
\lineto(776.84142334,318.38634644)
\curveto(776.88141335,318.38634246)(776.91141332,318.37634247)(776.93142334,318.35634644)
\curveto(777.00141323,318.31634253)(777.04141319,318.24134261)(777.05142334,318.13134644)
\curveto(777.06141317,318.03134282)(777.05141318,317.93134292)(777.02142334,317.83134644)
\curveto(776.96141327,317.60134325)(776.90641333,317.38134347)(776.85642334,317.17134644)
\curveto(776.80641343,316.96134389)(776.7314135,316.76134409)(776.63142334,316.57134644)
\curveto(776.55141368,316.44134441)(776.47641376,316.31634453)(776.40642334,316.19634644)
\curveto(776.34641389,316.07634477)(776.27641396,315.95634489)(776.19642334,315.83634644)
\curveto(776.01641422,315.57634527)(775.79141444,315.33634551)(775.52142334,315.11634644)
\curveto(775.26141497,314.90634594)(774.97641526,314.73134612)(774.66642334,314.59134644)
\curveto(774.55641568,314.54134631)(774.44641579,314.50134635)(774.33642334,314.47134644)
\curveto(774.236416,314.44134641)(774.1314161,314.40634644)(774.02142334,314.36634644)
\curveto(773.91141632,314.32634652)(773.79641644,314.30134655)(773.67642334,314.29134644)
\curveto(773.56641667,314.27134658)(773.45141678,314.2513466)(773.33142334,314.23134644)
\curveto(773.28141695,314.21134664)(773.236417,314.20634664)(773.19642334,314.21634644)
\curveto(773.15641708,314.21634663)(773.11641712,314.21134664)(773.07642334,314.20134644)
\curveto(773.01641722,314.19134666)(772.95641728,314.18634666)(772.89642334,314.18634644)
\curveto(772.8364174,314.18634666)(772.77141746,314.18134667)(772.70142334,314.17134644)
\curveto(772.67141756,314.16134669)(772.60141763,314.16134669)(772.49142334,314.17134644)
\curveto(772.39141784,314.17134668)(772.32641791,314.17634667)(772.29642334,314.18634644)
\curveto(772.24641799,314.19634665)(772.19641804,314.20134665)(772.14642334,314.20134644)
\curveto(772.10641813,314.19134666)(772.06141817,314.19134666)(772.01142334,314.20134644)
\lineto(771.86142334,314.20134644)
\curveto(771.78141845,314.22134663)(771.70641853,314.23634661)(771.63642334,314.24634644)
\curveto(771.56641867,314.2463466)(771.49141874,314.25634659)(771.41142334,314.27634644)
\lineto(771.14142334,314.33634644)
\curveto(771.05141918,314.3463465)(770.96641927,314.36634648)(770.88642334,314.39634644)
\curveto(770.67641956,314.45634639)(770.48641975,314.53134632)(770.31642334,314.62134644)
\curveto(769.68642055,314.89134596)(769.17642106,315.27634557)(768.78642334,315.77634644)
\curveto(768.39642184,316.27634457)(768.08642215,316.86634398)(767.85642334,317.54634644)
\curveto(767.81642242,317.66634318)(767.78142245,317.79134306)(767.75142334,317.92134644)
\curveto(767.7314225,318.0513428)(767.70642253,318.18634266)(767.67642334,318.32634644)
\curveto(767.65642258,318.37634247)(767.64642259,318.42134243)(767.64642334,318.46134644)
\curveto(767.65642258,318.50134235)(767.65642258,318.5463423)(767.64642334,318.59634644)
\curveto(767.62642261,318.68634216)(767.61142262,318.78134207)(767.60142334,318.88134644)
\curveto(767.60142263,318.98134187)(767.59142264,319.07634177)(767.57142334,319.16634644)
\lineto(767.57142334,319.45134644)
\curveto(767.55142268,319.50134135)(767.54142269,319.58634126)(767.54142334,319.70634644)
\curveto(767.54142269,319.82634102)(767.55142268,319.91134094)(767.57142334,319.96134644)
\curveto(767.58142265,319.99134086)(767.58142265,320.02134083)(767.57142334,320.05134644)
\curveto(767.56142267,320.09134076)(767.56142267,320.12134073)(767.57142334,320.14134644)
\lineto(767.57142334,320.27634644)
\curveto(767.58142265,320.35634049)(767.58642265,320.43634041)(767.58642334,320.51634644)
\curveto(767.59642264,320.60634024)(767.61142262,320.69134016)(767.63142334,320.77134644)
\curveto(767.65142258,320.83134002)(767.66142257,320.89133996)(767.66142334,320.95134644)
\curveto(767.66142257,321.02133983)(767.67142256,321.09133976)(767.69142334,321.16134644)
\curveto(767.74142249,321.33133952)(767.78142245,321.49633935)(767.81142334,321.65634644)
\curveto(767.84142239,321.81633903)(767.88642235,321.96633888)(767.94642334,322.10634644)
\lineto(768.09642334,322.49634644)
\curveto(768.15642208,322.63633821)(768.22142201,322.76133809)(768.29142334,322.87134644)
\curveto(768.44142179,323.13133772)(768.59142164,323.36633748)(768.74142334,323.57634644)
\curveto(768.77142146,323.62633722)(768.80642143,323.66633718)(768.84642334,323.69634644)
\curveto(768.89642134,323.73633711)(768.9364213,323.78133707)(768.96642334,323.83134644)
\curveto(769.02642121,323.91133694)(769.08642115,323.98133687)(769.14642334,324.04134644)
\lineto(769.35642334,324.22134644)
\curveto(769.41642082,324.27133658)(769.47142076,324.31633653)(769.52142334,324.35634644)
\curveto(769.58142065,324.40633644)(769.64642059,324.45633639)(769.71642334,324.50634644)
\curveto(769.86642037,324.61633623)(770.02142021,324.71133614)(770.18142334,324.79134644)
\curveto(770.35141988,324.87133598)(770.52641971,324.9513359)(770.70642334,325.03134644)
\curveto(770.81641942,325.08133577)(770.9314193,325.11633573)(771.05142334,325.13634644)
\curveto(771.18141905,325.16633568)(771.30641893,325.20133565)(771.42642334,325.24134644)
\curveto(771.49641874,325.2513356)(771.56141867,325.26133559)(771.62142334,325.27134644)
\lineto(771.80142334,325.30134644)
\curveto(771.88141835,325.31133554)(771.95641828,325.31633553)(772.02642334,325.31634644)
\curveto(772.10641813,325.32633552)(772.18641805,325.33633551)(772.26642334,325.34634644)
\curveto(772.28641795,325.35633549)(772.31141792,325.35633549)(772.34142334,325.34634644)
\curveto(772.37141786,325.33633551)(772.39641784,325.33633551)(772.41642334,325.34634644)
}
}
{
\newrgbcolor{curcolor}{0 0 0}
\pscustom[linestyle=none,fillstyle=solid,fillcolor=curcolor]
{
\newpath
\moveto(785.77626709,318.62634644)
\curveto(785.79625903,318.56634228)(785.80625902,318.47134238)(785.80626709,318.34134644)
\curveto(785.80625902,318.22134263)(785.80125902,318.13634271)(785.79126709,318.08634644)
\lineto(785.79126709,317.93634644)
\curveto(785.78125904,317.85634299)(785.77125905,317.78134307)(785.76126709,317.71134644)
\curveto(785.76125906,317.6513432)(785.75625907,317.58134327)(785.74626709,317.50134644)
\curveto(785.7262591,317.44134341)(785.71125911,317.38134347)(785.70126709,317.32134644)
\curveto(785.70125912,317.26134359)(785.69125913,317.20134365)(785.67126709,317.14134644)
\curveto(785.63125919,317.01134384)(785.59625923,316.88134397)(785.56626709,316.75134644)
\curveto(785.53625929,316.62134423)(785.49625933,316.50134435)(785.44626709,316.39134644)
\curveto(785.23625959,315.91134494)(784.95625987,315.50634534)(784.60626709,315.17634644)
\curveto(784.25626057,314.85634599)(783.826261,314.61134624)(783.31626709,314.44134644)
\curveto(783.20626162,314.40134645)(783.08626174,314.37134648)(782.95626709,314.35134644)
\curveto(782.83626199,314.33134652)(782.71126211,314.31134654)(782.58126709,314.29134644)
\curveto(782.5212623,314.28134657)(782.45626237,314.27634657)(782.38626709,314.27634644)
\curveto(782.3262625,314.26634658)(782.26626256,314.26134659)(782.20626709,314.26134644)
\curveto(782.16626266,314.2513466)(782.10626272,314.2463466)(782.02626709,314.24634644)
\curveto(781.95626287,314.2463466)(781.90626292,314.2513466)(781.87626709,314.26134644)
\curveto(781.83626299,314.27134658)(781.79626303,314.27634657)(781.75626709,314.27634644)
\curveto(781.71626311,314.26634658)(781.68126314,314.26634658)(781.65126709,314.27634644)
\lineto(781.56126709,314.27634644)
\lineto(781.20126709,314.32134644)
\curveto(781.06126376,314.36134649)(780.9262639,314.40134645)(780.79626709,314.44134644)
\curveto(780.66626416,314.48134637)(780.54126428,314.52634632)(780.42126709,314.57634644)
\curveto(779.97126485,314.77634607)(779.60126522,315.03634581)(779.31126709,315.35634644)
\curveto(779.0212658,315.67634517)(778.78126604,316.06634478)(778.59126709,316.52634644)
\curveto(778.54126628,316.62634422)(778.50126632,316.72634412)(778.47126709,316.82634644)
\curveto(778.45126637,316.92634392)(778.43126639,317.03134382)(778.41126709,317.14134644)
\curveto(778.39126643,317.18134367)(778.38126644,317.21134364)(778.38126709,317.23134644)
\curveto(778.39126643,317.26134359)(778.39126643,317.29634355)(778.38126709,317.33634644)
\curveto(778.36126646,317.41634343)(778.34626648,317.49634335)(778.33626709,317.57634644)
\curveto(778.33626649,317.66634318)(778.3262665,317.7513431)(778.30626709,317.83134644)
\lineto(778.30626709,317.95134644)
\curveto(778.30626652,317.99134286)(778.30126652,318.03634281)(778.29126709,318.08634644)
\curveto(778.28126654,318.13634271)(778.27626655,318.22134263)(778.27626709,318.34134644)
\curveto(778.27626655,318.47134238)(778.28626654,318.56634228)(778.30626709,318.62634644)
\curveto(778.3262665,318.69634215)(778.33126649,318.76634208)(778.32126709,318.83634644)
\curveto(778.31126651,318.90634194)(778.31626651,318.97634187)(778.33626709,319.04634644)
\curveto(778.34626648,319.09634175)(778.35126647,319.13634171)(778.35126709,319.16634644)
\curveto(778.36126646,319.20634164)(778.37126645,319.2513416)(778.38126709,319.30134644)
\curveto(778.41126641,319.42134143)(778.43626639,319.54134131)(778.45626709,319.66134644)
\curveto(778.48626634,319.78134107)(778.5262663,319.89634095)(778.57626709,320.00634644)
\curveto(778.7262661,320.37634047)(778.90626592,320.70634014)(779.11626709,320.99634644)
\curveto(779.33626549,321.29633955)(779.60126522,321.5463393)(779.91126709,321.74634644)
\curveto(780.03126479,321.82633902)(780.15626467,321.89133896)(780.28626709,321.94134644)
\curveto(780.41626441,322.00133885)(780.55126427,322.06133879)(780.69126709,322.12134644)
\curveto(780.81126401,322.17133868)(780.94126388,322.20133865)(781.08126709,322.21134644)
\curveto(781.2212636,322.23133862)(781.36126346,322.26133859)(781.50126709,322.30134644)
\lineto(781.69626709,322.30134644)
\curveto(781.76626306,322.31133854)(781.83126299,322.32133853)(781.89126709,322.33134644)
\curveto(782.78126204,322.34133851)(783.5212613,322.15633869)(784.11126709,321.77634644)
\curveto(784.70126012,321.39633945)(785.1262597,320.90133995)(785.38626709,320.29134644)
\curveto(785.43625939,320.19134066)(785.47625935,320.09134076)(785.50626709,319.99134644)
\curveto(785.53625929,319.89134096)(785.57125925,319.78634106)(785.61126709,319.67634644)
\curveto(785.64125918,319.56634128)(785.66625916,319.4463414)(785.68626709,319.31634644)
\curveto(785.70625912,319.19634165)(785.73125909,319.07134178)(785.76126709,318.94134644)
\curveto(785.77125905,318.89134196)(785.77125905,318.83634201)(785.76126709,318.77634644)
\curveto(785.76125906,318.72634212)(785.76625906,318.67634217)(785.77626709,318.62634644)
\moveto(784.44126709,317.77134644)
\curveto(784.46126036,317.84134301)(784.46626036,317.92134293)(784.45626709,318.01134644)
\lineto(784.45626709,318.26634644)
\curveto(784.45626037,318.65634219)(784.4212604,318.98634186)(784.35126709,319.25634644)
\curveto(784.3212605,319.33634151)(784.29626053,319.41634143)(784.27626709,319.49634644)
\curveto(784.25626057,319.57634127)(784.23126059,319.6513412)(784.20126709,319.72134644)
\curveto(783.9212609,320.37134048)(783.47626135,320.82134003)(782.86626709,321.07134644)
\curveto(782.79626203,321.10133975)(782.7212621,321.12133973)(782.64126709,321.13134644)
\lineto(782.40126709,321.19134644)
\curveto(782.3212625,321.21133964)(782.23626259,321.22133963)(782.14626709,321.22134644)
\lineto(781.87626709,321.22134644)
\lineto(781.60626709,321.17634644)
\curveto(781.50626332,321.15633969)(781.41126341,321.13133972)(781.32126709,321.10134644)
\curveto(781.24126358,321.08133977)(781.16126366,321.0513398)(781.08126709,321.01134644)
\curveto(781.01126381,320.99133986)(780.94626388,320.96133989)(780.88626709,320.92134644)
\curveto(780.826264,320.88133997)(780.77126405,320.84134001)(780.72126709,320.80134644)
\curveto(780.48126434,320.63134022)(780.28626454,320.42634042)(780.13626709,320.18634644)
\curveto(779.98626484,319.9463409)(779.85626497,319.66634118)(779.74626709,319.34634644)
\curveto(779.71626511,319.2463416)(779.69626513,319.14134171)(779.68626709,319.03134644)
\curveto(779.67626515,318.93134192)(779.66126516,318.82634202)(779.64126709,318.71634644)
\curveto(779.63126519,318.67634217)(779.6262652,318.61134224)(779.62626709,318.52134644)
\curveto(779.61626521,318.49134236)(779.61126521,318.45634239)(779.61126709,318.41634644)
\curveto(779.6212652,318.37634247)(779.6262652,318.33134252)(779.62626709,318.28134644)
\lineto(779.62626709,317.98134644)
\curveto(779.6262652,317.88134297)(779.63626519,317.79134306)(779.65626709,317.71134644)
\lineto(779.68626709,317.53134644)
\curveto(779.70626512,317.43134342)(779.7212651,317.33134352)(779.73126709,317.23134644)
\curveto(779.75126507,317.14134371)(779.78126504,317.05634379)(779.82126709,316.97634644)
\curveto(779.9212649,316.73634411)(780.03626479,316.51134434)(780.16626709,316.30134644)
\curveto(780.30626452,316.09134476)(780.47626435,315.91634493)(780.67626709,315.77634644)
\curveto(780.7262641,315.7463451)(780.77126405,315.72134513)(780.81126709,315.70134644)
\curveto(780.85126397,315.68134517)(780.89626393,315.65634519)(780.94626709,315.62634644)
\curveto(781.0262638,315.57634527)(781.11126371,315.53134532)(781.20126709,315.49134644)
\curveto(781.30126352,315.46134539)(781.40626342,315.43134542)(781.51626709,315.40134644)
\curveto(781.56626326,315.38134547)(781.61126321,315.37134548)(781.65126709,315.37134644)
\curveto(781.70126312,315.38134547)(781.75126307,315.38134547)(781.80126709,315.37134644)
\curveto(781.83126299,315.36134549)(781.89126293,315.3513455)(781.98126709,315.34134644)
\curveto(782.08126274,315.33134552)(782.15626267,315.33634551)(782.20626709,315.35634644)
\curveto(782.24626258,315.36634548)(782.28626254,315.36634548)(782.32626709,315.35634644)
\curveto(782.36626246,315.35634549)(782.40626242,315.36634548)(782.44626709,315.38634644)
\curveto(782.5262623,315.40634544)(782.60626222,315.42134543)(782.68626709,315.43134644)
\curveto(782.76626206,315.4513454)(782.84126198,315.47634537)(782.91126709,315.50634644)
\curveto(783.25126157,315.6463452)(783.5262613,315.84134501)(783.73626709,316.09134644)
\curveto(783.94626088,316.34134451)(784.1212607,316.63634421)(784.26126709,316.97634644)
\curveto(784.31126051,317.09634375)(784.34126048,317.22134363)(784.35126709,317.35134644)
\curveto(784.37126045,317.49134336)(784.40126042,317.63134322)(784.44126709,317.77134644)
}
}
{
\newrgbcolor{curcolor}{0 0 0}
\pscustom[linestyle=none,fillstyle=solid,fillcolor=curcolor]
{
\newpath
\moveto(790.95454834,322.33134644)
\curveto(791.33454335,322.34133851)(791.65454303,322.30133855)(791.91454834,322.21134644)
\curveto(792.1845425,322.12133873)(792.42954226,321.99133886)(792.64954834,321.82134644)
\curveto(792.72954196,321.77133908)(792.79454189,321.70133915)(792.84454834,321.61134644)
\curveto(792.90454178,321.53133932)(792.96954172,321.45633939)(793.03954834,321.38634644)
\curveto(793.05954163,321.36633948)(793.0895416,321.34133951)(793.12954834,321.31134644)
\curveto(793.16954152,321.28133957)(793.21954147,321.27133958)(793.27954834,321.28134644)
\curveto(793.37954131,321.31133954)(793.46454122,321.37133948)(793.53454834,321.46134644)
\curveto(793.61454107,321.56133929)(793.69454099,321.63633921)(793.77454834,321.68634644)
\curveto(793.91454077,321.79633905)(794.05954063,321.89133896)(794.20954834,321.97134644)
\curveto(794.35954033,322.06133879)(794.52454016,322.13633871)(794.70454834,322.19634644)
\curveto(794.7845399,322.22633862)(794.86953982,322.2463386)(794.95954834,322.25634644)
\curveto(795.05953963,322.27633857)(795.15453953,322.29633855)(795.24454834,322.31634644)
\curveto(795.29453939,322.32633852)(795.33953935,322.33133852)(795.37954834,322.33134644)
\lineto(795.52954834,322.33134644)
\curveto(795.57953911,322.3513385)(795.64953904,322.35633849)(795.73954834,322.34634644)
\curveto(795.82953886,322.3463385)(795.89453879,322.34133851)(795.93454834,322.33134644)
\curveto(795.9845387,322.32133853)(796.05953863,322.31633853)(796.15954834,322.31634644)
\curveto(796.24953844,322.29633855)(796.33453835,322.27633857)(796.41454834,322.25634644)
\curveto(796.50453818,322.2463386)(796.5895381,322.22633862)(796.66954834,322.19634644)
\curveto(796.71953797,322.17633867)(796.76453792,322.16133869)(796.80454834,322.15134644)
\curveto(796.85453783,322.1513387)(796.90453778,322.14133871)(796.95454834,322.12134644)
\curveto(797.45453723,321.90133895)(797.79953689,321.56133929)(797.98954834,321.10134644)
\curveto(798.02953666,321.02133983)(798.05953663,320.93133992)(798.07954834,320.83134644)
\curveto(798.09953659,320.74134011)(798.11953657,320.64134021)(798.13954834,320.53134644)
\curveto(798.15953653,320.50134035)(798.16453652,320.46634038)(798.15454834,320.42634644)
\curveto(798.15453653,320.39634045)(798.15953653,320.36634048)(798.16954834,320.33634644)
\lineto(798.16954834,320.20134644)
\curveto(798.17953651,320.16134069)(798.17953651,320.11634073)(798.16954834,320.06634644)
\curveto(798.16953652,320.01634083)(798.16953652,319.96634088)(798.16954834,319.91634644)
\lineto(798.16954834,319.33134644)
\lineto(798.16954834,318.37134644)
\lineto(798.16954834,315.52134644)
\curveto(798.16953652,315.36134549)(798.16953652,315.17134568)(798.16954834,314.95134644)
\curveto(798.17953651,314.73134612)(798.13953655,314.58634626)(798.04954834,314.51634644)
\curveto(798.00953668,314.48634636)(797.94453674,314.46134639)(797.85454834,314.44134644)
\curveto(797.76453692,314.43134642)(797.66953702,314.42634642)(797.56954834,314.42634644)
\curveto(797.46953722,314.42634642)(797.36953732,314.43134642)(797.26954834,314.44134644)
\curveto(797.17953751,314.4513464)(797.11453757,314.47134638)(797.07454834,314.50134644)
\curveto(797.01453767,314.53134632)(796.97453771,314.59134626)(796.95454834,314.68134644)
\curveto(796.93453775,314.74134611)(796.92953776,314.80134605)(796.93954834,314.86134644)
\curveto(796.94953774,314.93134592)(796.94453774,314.99634585)(796.92454834,315.05634644)
\curveto(796.91453777,315.10634574)(796.90953778,315.16134569)(796.90954834,315.22134644)
\curveto(796.91953777,315.29134556)(796.92453776,315.35634549)(796.92454834,315.41634644)
\lineto(796.92454834,316.09134644)
\lineto(796.92454834,318.95634644)
\curveto(796.92453776,319.28634156)(796.91453777,319.59634125)(796.89454834,319.88634644)
\curveto(796.8845378,320.18634066)(796.81453787,320.43634041)(796.68454834,320.63634644)
\curveto(796.53453815,320.87633997)(796.30453838,321.0513398)(795.99454834,321.16134644)
\curveto(795.93453875,321.18133967)(795.86953882,321.19133966)(795.79954834,321.19134644)
\curveto(795.73953895,321.20133965)(795.67453901,321.21633963)(795.60454834,321.23634644)
\curveto(795.56453912,321.2463396)(795.49953919,321.2463396)(795.40954834,321.23634644)
\curveto(795.31953937,321.23633961)(795.25953943,321.23133962)(795.22954834,321.22134644)
\curveto(795.17953951,321.21133964)(795.12953956,321.20633964)(795.07954834,321.20634644)
\curveto(795.02953966,321.21633963)(794.97953971,321.21133964)(794.92954834,321.19134644)
\curveto(794.7895399,321.16133969)(794.65454003,321.12133973)(794.52454834,321.07134644)
\curveto(794.00454068,320.85134)(793.65454103,320.46634038)(793.47454834,319.91634644)
\curveto(793.42454126,319.7463411)(793.39454129,319.5513413)(793.38454834,319.33134644)
\lineto(793.38454834,318.65634644)
\lineto(793.38454834,316.69134644)
\lineto(793.38454834,315.23634644)
\lineto(793.38454834,314.86134644)
\curveto(793.3845413,314.74134611)(793.35954133,314.6463462)(793.30954834,314.57634644)
\curveto(793.25954143,314.49634635)(793.17454151,314.4513464)(793.05454834,314.44134644)
\curveto(792.93454175,314.43134642)(792.80954188,314.42634642)(792.67954834,314.42634644)
\curveto(792.50954218,314.42634642)(792.3845423,314.4463464)(792.30454834,314.48634644)
\curveto(792.21454247,314.53634631)(792.15954253,314.61634623)(792.13954834,314.72634644)
\curveto(792.12954256,314.846346)(792.12454256,314.97634587)(792.12454834,315.11634644)
\lineto(792.12454834,316.54134644)
\lineto(792.12454834,319.01634644)
\curveto(792.12454256,319.33634151)(792.11454257,319.63134122)(792.09454834,319.90134644)
\curveto(792.07454261,320.18134067)(792.00454268,320.42134043)(791.88454834,320.62134644)
\curveto(791.77454291,320.80134005)(791.64954304,320.93133992)(791.50954834,321.01134644)
\curveto(791.36954332,321.10133975)(791.17954351,321.17133968)(790.93954834,321.22134644)
\curveto(790.89954379,321.23133962)(790.85454383,321.23633961)(790.80454834,321.23634644)
\lineto(790.66954834,321.23634644)
\curveto(790.44954424,321.23633961)(790.25454443,321.21133964)(790.08454834,321.16134644)
\curveto(789.92454476,321.11133974)(789.77954491,321.0463398)(789.64954834,320.96634644)
\curveto(789.13954555,320.65634019)(788.79954589,320.19134066)(788.62954834,319.57134644)
\curveto(788.5895461,319.44134141)(788.56954612,319.29134156)(788.56954834,319.12134644)
\curveto(788.57954611,318.96134189)(788.5845461,318.80134205)(788.58454834,318.64134644)
\lineto(788.58454834,316.94634644)
\lineto(788.58454834,315.29634644)
\lineto(788.58454834,314.89134644)
\curveto(788.5845461,314.7513461)(788.55454613,314.64134621)(788.49454834,314.56134644)
\curveto(788.44454624,314.49134636)(788.36954632,314.4513464)(788.26954834,314.44134644)
\curveto(788.16954652,314.43134642)(788.06454662,314.42634642)(787.95454834,314.42634644)
\lineto(787.72954834,314.42634644)
\curveto(787.66954702,314.4463464)(787.60954708,314.46134639)(787.54954834,314.47134644)
\curveto(787.49954719,314.48134637)(787.45454723,314.51134634)(787.41454834,314.56134644)
\curveto(787.36454732,314.62134623)(787.33954735,314.69634615)(787.33954834,314.78634644)
\lineto(787.33954834,315.10134644)
\lineto(787.33954834,316.07634644)
\lineto(787.33954834,320.36634644)
\lineto(787.33954834,321.47634644)
\lineto(787.33954834,321.76134644)
\curveto(787.33954735,321.86133899)(787.35954733,321.94133891)(787.39954834,322.00134644)
\curveto(787.42954726,322.06133879)(787.47454721,322.10133875)(787.53454834,322.12134644)
\curveto(787.61454707,322.1513387)(787.73954695,322.16633868)(787.90954834,322.16634644)
\curveto(788.0895466,322.16633868)(788.21954647,322.1513387)(788.29954834,322.12134644)
\curveto(788.37954631,322.08133877)(788.43454625,322.03133882)(788.46454834,321.97134644)
\curveto(788.4845462,321.92133893)(788.49454619,321.86133899)(788.49454834,321.79134644)
\curveto(788.50454618,321.72133913)(788.51454617,321.65633919)(788.52454834,321.59634644)
\curveto(788.53454615,321.53633931)(788.55454613,321.48633936)(788.58454834,321.44634644)
\curveto(788.61454607,321.40633944)(788.66454602,321.38633946)(788.73454834,321.38634644)
\curveto(788.75454593,321.40633944)(788.77454591,321.41633943)(788.79454834,321.41634644)
\curveto(788.82454586,321.41633943)(788.84954584,321.42633942)(788.86954834,321.44634644)
\curveto(788.92954576,321.49633935)(788.9845457,321.5463393)(789.03454834,321.59634644)
\lineto(789.21454834,321.74634644)
\curveto(789.43454525,321.90633894)(789.684545,322.0463388)(789.96454834,322.16634644)
\curveto(790.06454462,322.20633864)(790.16454452,322.23133862)(790.26454834,322.24134644)
\curveto(790.36454432,322.26133859)(790.46954422,322.28633856)(790.57954834,322.31634644)
\lineto(790.75954834,322.31634644)
\curveto(790.82954386,322.32633852)(790.89454379,322.33133852)(790.95454834,322.33134644)
}
}
{
\newrgbcolor{curcolor}{0 0 0}
\pscustom[linestyle=none,fillstyle=solid,fillcolor=curcolor]
{
\newpath
\moveto(806.81728271,318.59634644)
\curveto(806.83727503,318.49634235)(806.83727503,318.38134247)(806.81728271,318.25134644)
\curveto(806.80727506,318.13134272)(806.77727509,318.0463428)(806.72728271,317.99634644)
\curveto(806.67727519,317.95634289)(806.60227526,317.92634292)(806.50228271,317.90634644)
\curveto(806.41227545,317.89634295)(806.30727556,317.89134296)(806.18728271,317.89134644)
\lineto(805.82728271,317.89134644)
\curveto(805.70727616,317.90134295)(805.60227626,317.90634294)(805.51228271,317.90634644)
\lineto(801.67228271,317.90634644)
\curveto(801.59228027,317.90634294)(801.51228035,317.90134295)(801.43228271,317.89134644)
\curveto(801.35228051,317.89134296)(801.28728058,317.87634297)(801.23728271,317.84634644)
\curveto(801.19728067,317.82634302)(801.15728071,317.78634306)(801.11728271,317.72634644)
\curveto(801.09728077,317.69634315)(801.07728079,317.6513432)(801.05728271,317.59134644)
\curveto(801.03728083,317.54134331)(801.03728083,317.49134336)(801.05728271,317.44134644)
\curveto(801.0672808,317.39134346)(801.07228079,317.3463435)(801.07228271,317.30634644)
\curveto(801.07228079,317.26634358)(801.07728079,317.22634362)(801.08728271,317.18634644)
\curveto(801.10728076,317.10634374)(801.12728074,317.02134383)(801.14728271,316.93134644)
\curveto(801.1672807,316.851344)(801.19728067,316.77134408)(801.23728271,316.69134644)
\curveto(801.4672804,316.1513447)(801.84728002,315.76634508)(802.37728271,315.53634644)
\curveto(802.43727943,315.50634534)(802.50227936,315.48134537)(802.57228271,315.46134644)
\lineto(802.78228271,315.40134644)
\curveto(802.81227905,315.39134546)(802.862279,315.38634546)(802.93228271,315.38634644)
\curveto(803.07227879,315.3463455)(803.25727861,315.32634552)(803.48728271,315.32634644)
\curveto(803.71727815,315.32634552)(803.90227796,315.3463455)(804.04228271,315.38634644)
\curveto(804.18227768,315.42634542)(804.30727756,315.46634538)(804.41728271,315.50634644)
\curveto(804.53727733,315.55634529)(804.64727722,315.61634523)(804.74728271,315.68634644)
\curveto(804.85727701,315.75634509)(804.95227691,315.83634501)(805.03228271,315.92634644)
\curveto(805.11227675,316.02634482)(805.18227668,316.13134472)(805.24228271,316.24134644)
\curveto(805.30227656,316.34134451)(805.35227651,316.4463444)(805.39228271,316.55634644)
\curveto(805.44227642,316.66634418)(805.52227634,316.7463441)(805.63228271,316.79634644)
\curveto(805.67227619,316.81634403)(805.73727613,316.83134402)(805.82728271,316.84134644)
\curveto(805.91727595,316.851344)(806.00727586,316.851344)(806.09728271,316.84134644)
\curveto(806.18727568,316.84134401)(806.27227559,316.83634401)(806.35228271,316.82634644)
\curveto(806.43227543,316.81634403)(806.48727538,316.79634405)(806.51728271,316.76634644)
\curveto(806.61727525,316.69634415)(806.64227522,316.58134427)(806.59228271,316.42134644)
\curveto(806.51227535,316.1513447)(806.40727546,315.91134494)(806.27728271,315.70134644)
\curveto(806.07727579,315.38134547)(805.84727602,315.11634573)(805.58728271,314.90634644)
\curveto(805.33727653,314.70634614)(805.01727685,314.54134631)(804.62728271,314.41134644)
\curveto(804.52727734,314.37134648)(804.42727744,314.3463465)(804.32728271,314.33634644)
\curveto(804.22727764,314.31634653)(804.12227774,314.29634655)(804.01228271,314.27634644)
\curveto(803.9622779,314.26634658)(803.91227795,314.26134659)(803.86228271,314.26134644)
\curveto(803.82227804,314.26134659)(803.77727809,314.25634659)(803.72728271,314.24634644)
\lineto(803.57728271,314.24634644)
\curveto(803.52727834,314.23634661)(803.4672784,314.23134662)(803.39728271,314.23134644)
\curveto(803.33727853,314.23134662)(803.28727858,314.23634661)(803.24728271,314.24634644)
\lineto(803.11228271,314.24634644)
\curveto(803.0622788,314.25634659)(803.01727885,314.26134659)(802.97728271,314.26134644)
\curveto(802.93727893,314.26134659)(802.89727897,314.26634658)(802.85728271,314.27634644)
\curveto(802.80727906,314.28634656)(802.75227911,314.29634655)(802.69228271,314.30634644)
\curveto(802.63227923,314.30634654)(802.57727929,314.31134654)(802.52728271,314.32134644)
\curveto(802.43727943,314.34134651)(802.34727952,314.36634648)(802.25728271,314.39634644)
\curveto(802.1672797,314.41634643)(802.08227978,314.44134641)(802.00228271,314.47134644)
\curveto(801.9622799,314.49134636)(801.92727994,314.50134635)(801.89728271,314.50134644)
\curveto(801.86728,314.51134634)(801.83228003,314.52634632)(801.79228271,314.54634644)
\curveto(801.64228022,314.61634623)(801.48228038,314.70134615)(801.31228271,314.80134644)
\curveto(801.02228084,314.99134586)(800.77228109,315.22134563)(800.56228271,315.49134644)
\curveto(800.3622815,315.77134508)(800.19228167,316.08134477)(800.05228271,316.42134644)
\curveto(800.00228186,316.53134432)(799.9622819,316.6463442)(799.93228271,316.76634644)
\curveto(799.91228195,316.88634396)(799.88228198,317.00634384)(799.84228271,317.12634644)
\curveto(799.83228203,317.16634368)(799.82728204,317.20134365)(799.82728271,317.23134644)
\curveto(799.82728204,317.26134359)(799.82228204,317.30134355)(799.81228271,317.35134644)
\curveto(799.79228207,317.43134342)(799.77728209,317.51634333)(799.76728271,317.60634644)
\curveto(799.75728211,317.69634315)(799.74228212,317.78634306)(799.72228271,317.87634644)
\lineto(799.72228271,318.08634644)
\curveto(799.71228215,318.12634272)(799.70228216,318.18134267)(799.69228271,318.25134644)
\curveto(799.69228217,318.33134252)(799.69728217,318.39634245)(799.70728271,318.44634644)
\lineto(799.70728271,318.61134644)
\curveto(799.72728214,318.66134219)(799.73228213,318.71134214)(799.72228271,318.76134644)
\curveto(799.72228214,318.82134203)(799.72728214,318.87634197)(799.73728271,318.92634644)
\curveto(799.77728209,319.08634176)(799.80728206,319.2463416)(799.82728271,319.40634644)
\curveto(799.85728201,319.56634128)(799.90228196,319.71634113)(799.96228271,319.85634644)
\curveto(800.01228185,319.96634088)(800.05728181,320.07634077)(800.09728271,320.18634644)
\curveto(800.14728172,320.30634054)(800.20228166,320.42134043)(800.26228271,320.53134644)
\curveto(800.48228138,320.88133997)(800.73228113,321.18133967)(801.01228271,321.43134644)
\curveto(801.29228057,321.69133916)(801.63728023,321.90633894)(802.04728271,322.07634644)
\curveto(802.1672797,322.12633872)(802.28727958,322.16133869)(802.40728271,322.18134644)
\curveto(802.53727933,322.21133864)(802.67227919,322.24133861)(802.81228271,322.27134644)
\curveto(802.862279,322.28133857)(802.90727896,322.28633856)(802.94728271,322.28634644)
\curveto(802.98727888,322.29633855)(803.03227883,322.30133855)(803.08228271,322.30134644)
\curveto(803.10227876,322.31133854)(803.12727874,322.31133854)(803.15728271,322.30134644)
\curveto(803.18727868,322.29133856)(803.21227865,322.29633855)(803.23228271,322.31634644)
\curveto(803.65227821,322.32633852)(804.01727785,322.28133857)(804.32728271,322.18134644)
\curveto(804.63727723,322.09133876)(804.91727695,321.96633888)(805.16728271,321.80634644)
\curveto(805.21727665,321.78633906)(805.25727661,321.75633909)(805.28728271,321.71634644)
\curveto(805.31727655,321.68633916)(805.35227651,321.66133919)(805.39228271,321.64134644)
\curveto(805.47227639,321.58133927)(805.55227631,321.51133934)(805.63228271,321.43134644)
\curveto(805.72227614,321.3513395)(805.79727607,321.27133958)(805.85728271,321.19134644)
\curveto(806.01727585,320.98133987)(806.15227571,320.78134007)(806.26228271,320.59134644)
\curveto(806.33227553,320.48134037)(806.38727548,320.36134049)(806.42728271,320.23134644)
\curveto(806.4672754,320.10134075)(806.51227535,319.97134088)(806.56228271,319.84134644)
\curveto(806.61227525,319.71134114)(806.64727522,319.57634127)(806.66728271,319.43634644)
\curveto(806.69727517,319.29634155)(806.73227513,319.15634169)(806.77228271,319.01634644)
\curveto(806.78227508,318.9463419)(806.78727508,318.87634197)(806.78728271,318.80634644)
\lineto(806.81728271,318.59634644)
\moveto(805.36228271,319.10634644)
\curveto(805.39227647,319.1463417)(805.41727645,319.19634165)(805.43728271,319.25634644)
\curveto(805.45727641,319.32634152)(805.45727641,319.39634145)(805.43728271,319.46634644)
\curveto(805.37727649,319.68634116)(805.29227657,319.89134096)(805.18228271,320.08134644)
\curveto(805.04227682,320.31134054)(804.88727698,320.50634034)(804.71728271,320.66634644)
\curveto(804.54727732,320.82634002)(804.32727754,320.96133989)(804.05728271,321.07134644)
\curveto(803.98727788,321.09133976)(803.91727795,321.10633974)(803.84728271,321.11634644)
\curveto(803.77727809,321.13633971)(803.70227816,321.15633969)(803.62228271,321.17634644)
\curveto(803.54227832,321.19633965)(803.45727841,321.20633964)(803.36728271,321.20634644)
\lineto(803.11228271,321.20634644)
\curveto(803.08227878,321.18633966)(803.04727882,321.17633967)(803.00728271,321.17634644)
\curveto(802.9672789,321.18633966)(802.93227893,321.18633966)(802.90228271,321.17634644)
\lineto(802.66228271,321.11634644)
\curveto(802.59227927,321.10633974)(802.52227934,321.09133976)(802.45228271,321.07134644)
\curveto(802.1622797,320.9513399)(801.92727994,320.80134005)(801.74728271,320.62134644)
\curveto(801.57728029,320.44134041)(801.42228044,320.21634063)(801.28228271,319.94634644)
\curveto(801.25228061,319.89634095)(801.22228064,319.83134102)(801.19228271,319.75134644)
\curveto(801.1622807,319.68134117)(801.13728073,319.60134125)(801.11728271,319.51134644)
\curveto(801.09728077,319.42134143)(801.09228077,319.33634151)(801.10228271,319.25634644)
\curveto(801.11228075,319.17634167)(801.14728072,319.11634173)(801.20728271,319.07634644)
\curveto(801.28728058,319.01634183)(801.42228044,318.98634186)(801.61228271,318.98634644)
\curveto(801.81228005,318.99634185)(801.98227988,319.00134185)(802.12228271,319.00134644)
\lineto(804.40228271,319.00134644)
\curveto(804.55227731,319.00134185)(804.73227713,318.99634185)(804.94228271,318.98634644)
\curveto(805.15227671,318.98634186)(805.29227657,319.02634182)(805.36228271,319.10634644)
}
}
{
\newrgbcolor{curcolor}{0 0 0}
\pscustom[linestyle=none,fillstyle=solid,fillcolor=curcolor]
{
\newpath
\moveto(811.81392334,322.30134644)
\curveto(812.4439181,322.32133853)(812.9489176,322.23633861)(813.32892334,322.04634644)
\curveto(813.70891684,321.85633899)(814.01391653,321.57133928)(814.24392334,321.19134644)
\curveto(814.30391624,321.09133976)(814.3489162,320.98133987)(814.37892334,320.86134644)
\curveto(814.41891613,320.7513401)(814.45391609,320.63634021)(814.48392334,320.51634644)
\curveto(814.53391601,320.32634052)(814.56391598,320.12134073)(814.57392334,319.90134644)
\curveto(814.58391596,319.68134117)(814.58891596,319.45634139)(814.58892334,319.22634644)
\lineto(814.58892334,317.62134644)
\lineto(814.58892334,315.28134644)
\curveto(814.58891596,315.11134574)(814.58391596,314.94134591)(814.57392334,314.77134644)
\curveto(814.57391597,314.60134625)(814.50891604,314.49134636)(814.37892334,314.44134644)
\curveto(814.32891622,314.42134643)(814.27391627,314.41134644)(814.21392334,314.41134644)
\curveto(814.16391638,314.40134645)(814.10891644,314.39634645)(814.04892334,314.39634644)
\curveto(813.91891663,314.39634645)(813.79391675,314.40134645)(813.67392334,314.41134644)
\curveto(813.55391699,314.41134644)(813.46891708,314.4513464)(813.41892334,314.53134644)
\curveto(813.36891718,314.60134625)(813.3439172,314.69134616)(813.34392334,314.80134644)
\lineto(813.34392334,315.13134644)
\lineto(813.34392334,316.42134644)
\lineto(813.34392334,318.86634644)
\curveto(813.3439172,319.13634171)(813.33891721,319.40134145)(813.32892334,319.66134644)
\curveto(813.31891723,319.93134092)(813.27391727,320.16134069)(813.19392334,320.35134644)
\curveto(813.11391743,320.5513403)(812.99391755,320.71134014)(812.83392334,320.83134644)
\curveto(812.67391787,320.96133989)(812.48891806,321.06133979)(812.27892334,321.13134644)
\curveto(812.21891833,321.1513397)(812.15391839,321.16133969)(812.08392334,321.16134644)
\curveto(812.02391852,321.17133968)(811.96391858,321.18633966)(811.90392334,321.20634644)
\curveto(811.85391869,321.21633963)(811.77391877,321.21633963)(811.66392334,321.20634644)
\curveto(811.56391898,321.20633964)(811.49391905,321.20133965)(811.45392334,321.19134644)
\curveto(811.41391913,321.17133968)(811.37891917,321.16133969)(811.34892334,321.16134644)
\curveto(811.31891923,321.17133968)(811.28391926,321.17133968)(811.24392334,321.16134644)
\curveto(811.11391943,321.13133972)(810.98891956,321.09633975)(810.86892334,321.05634644)
\curveto(810.75891979,321.02633982)(810.65391989,320.98133987)(810.55392334,320.92134644)
\curveto(810.51392003,320.90133995)(810.47892007,320.88133997)(810.44892334,320.86134644)
\curveto(810.41892013,320.84134001)(810.38392016,320.82134003)(810.34392334,320.80134644)
\curveto(809.99392055,320.5513403)(809.73892081,320.17634067)(809.57892334,319.67634644)
\curveto(809.548921,319.59634125)(809.52892102,319.51134134)(809.51892334,319.42134644)
\curveto(809.50892104,319.34134151)(809.49392105,319.26134159)(809.47392334,319.18134644)
\curveto(809.45392109,319.13134172)(809.4489211,319.08134177)(809.45892334,319.03134644)
\curveto(809.46892108,318.99134186)(809.46392108,318.9513419)(809.44392334,318.91134644)
\lineto(809.44392334,318.59634644)
\curveto(809.43392111,318.56634228)(809.42892112,318.53134232)(809.42892334,318.49134644)
\curveto(809.43892111,318.4513424)(809.4439211,318.40634244)(809.44392334,318.35634644)
\lineto(809.44392334,317.90634644)
\lineto(809.44392334,316.46634644)
\lineto(809.44392334,315.14634644)
\lineto(809.44392334,314.80134644)
\curveto(809.4439211,314.69134616)(809.41892113,314.60134625)(809.36892334,314.53134644)
\curveto(809.31892123,314.4513464)(809.22892132,314.41134644)(809.09892334,314.41134644)
\curveto(808.97892157,314.40134645)(808.85392169,314.39634645)(808.72392334,314.39634644)
\curveto(808.6439219,314.39634645)(808.56892198,314.40134645)(808.49892334,314.41134644)
\curveto(808.42892212,314.42134643)(808.36892218,314.4463464)(808.31892334,314.48634644)
\curveto(808.23892231,314.53634631)(808.19892235,314.63134622)(808.19892334,314.77134644)
\lineto(808.19892334,315.17634644)
\lineto(808.19892334,316.94634644)
\lineto(808.19892334,320.57634644)
\lineto(808.19892334,321.49134644)
\lineto(808.19892334,321.76134644)
\curveto(808.19892235,321.851339)(808.21892233,321.92133893)(808.25892334,321.97134644)
\curveto(808.28892226,322.03133882)(808.33892221,322.07133878)(808.40892334,322.09134644)
\curveto(808.4489221,322.10133875)(808.50392204,322.11133874)(808.57392334,322.12134644)
\curveto(808.65392189,322.13133872)(808.73392181,322.13633871)(808.81392334,322.13634644)
\curveto(808.89392165,322.13633871)(808.96892158,322.13133872)(809.03892334,322.12134644)
\curveto(809.11892143,322.11133874)(809.17392137,322.09633875)(809.20392334,322.07634644)
\curveto(809.31392123,322.00633884)(809.36392118,321.91633893)(809.35392334,321.80634644)
\curveto(809.3439212,321.70633914)(809.35892119,321.59133926)(809.39892334,321.46134644)
\curveto(809.41892113,321.40133945)(809.45892109,321.3513395)(809.51892334,321.31134644)
\curveto(809.63892091,321.30133955)(809.73392081,321.3463395)(809.80392334,321.44634644)
\curveto(809.88392066,321.5463393)(809.96392058,321.62633922)(810.04392334,321.68634644)
\curveto(810.18392036,321.78633906)(810.32392022,321.87633897)(810.46392334,321.95634644)
\curveto(810.61391993,322.0463388)(810.78391976,322.12133873)(810.97392334,322.18134644)
\curveto(811.05391949,322.21133864)(811.13891941,322.23133862)(811.22892334,322.24134644)
\curveto(811.32891922,322.2513386)(811.42391912,322.26633858)(811.51392334,322.28634644)
\curveto(811.56391898,322.29633855)(811.61391893,322.30133855)(811.66392334,322.30134644)
\lineto(811.81392334,322.30134644)
}
}
{
\newrgbcolor{curcolor}{0 0 0}
\pscustom[linestyle=none,fillstyle=solid,fillcolor=curcolor]
{
\newpath
\moveto(817.41853271,324.49134644)
\curveto(817.5685307,324.49133636)(817.71853055,324.48633636)(817.86853271,324.47634644)
\curveto(818.01853025,324.47633637)(818.12353015,324.43633641)(818.18353271,324.35634644)
\curveto(818.23353004,324.29633655)(818.25853001,324.21133664)(818.25853271,324.10134644)
\curveto(818.26853,324.00133685)(818.27353,323.89633695)(818.27353271,323.78634644)
\lineto(818.27353271,322.91634644)
\curveto(818.27353,322.83633801)(818.26853,322.7513381)(818.25853271,322.66134644)
\curveto(818.25853001,322.58133827)(818.26853,322.51133834)(818.28853271,322.45134644)
\curveto(818.32852994,322.31133854)(818.41852985,322.22133863)(818.55853271,322.18134644)
\curveto(818.60852966,322.17133868)(818.65352962,322.16633868)(818.69353271,322.16634644)
\lineto(818.84353271,322.16634644)
\lineto(819.24853271,322.16634644)
\curveto(819.40852886,322.17633867)(819.52352875,322.16633868)(819.59353271,322.13634644)
\curveto(819.68352859,322.07633877)(819.74352853,322.01633883)(819.77353271,321.95634644)
\curveto(819.79352848,321.91633893)(819.80352847,321.87133898)(819.80353271,321.82134644)
\lineto(819.80353271,321.67134644)
\curveto(819.80352847,321.56133929)(819.79852847,321.45633939)(819.78853271,321.35634644)
\curveto(819.77852849,321.26633958)(819.74352853,321.19633965)(819.68353271,321.14634644)
\curveto(819.62352865,321.09633975)(819.53852873,321.06633978)(819.42853271,321.05634644)
\lineto(819.09853271,321.05634644)
\curveto(818.98852928,321.06633978)(818.87852939,321.07133978)(818.76853271,321.07134644)
\curveto(818.65852961,321.07133978)(818.56352971,321.05633979)(818.48353271,321.02634644)
\curveto(818.41352986,320.99633985)(818.36352991,320.9463399)(818.33353271,320.87634644)
\curveto(818.30352997,320.80634004)(818.28352999,320.72134013)(818.27353271,320.62134644)
\curveto(818.26353001,320.53134032)(818.25853001,320.43134042)(818.25853271,320.32134644)
\curveto(818.26853,320.22134063)(818.27353,320.12134073)(818.27353271,320.02134644)
\lineto(818.27353271,317.05134644)
\curveto(818.27353,316.83134402)(818.26853,316.59634425)(818.25853271,316.34634644)
\curveto(818.25853001,316.10634474)(818.30352997,315.92134493)(818.39353271,315.79134644)
\curveto(818.44352983,315.71134514)(818.50852976,315.65634519)(818.58853271,315.62634644)
\curveto(818.6685296,315.59634525)(818.76352951,315.57134528)(818.87353271,315.55134644)
\curveto(818.90352937,315.54134531)(818.93352934,315.53634531)(818.96353271,315.53634644)
\curveto(819.00352927,315.5463453)(819.03852923,315.5463453)(819.06853271,315.53634644)
\lineto(819.26353271,315.53634644)
\curveto(819.36352891,315.53634531)(819.45352882,315.52634532)(819.53353271,315.50634644)
\curveto(819.62352865,315.49634535)(819.68852858,315.46134539)(819.72853271,315.40134644)
\curveto(819.74852852,315.37134548)(819.76352851,315.31634553)(819.77353271,315.23634644)
\curveto(819.79352848,315.16634568)(819.80352847,315.09134576)(819.80353271,315.01134644)
\curveto(819.81352846,314.93134592)(819.81352846,314.851346)(819.80353271,314.77134644)
\curveto(819.79352848,314.70134615)(819.7735285,314.6463462)(819.74353271,314.60634644)
\curveto(819.70352857,314.53634631)(819.62852864,314.48634636)(819.51853271,314.45634644)
\curveto(819.43852883,314.43634641)(819.34852892,314.42634642)(819.24853271,314.42634644)
\curveto(819.14852912,314.43634641)(819.05852921,314.44134641)(818.97853271,314.44134644)
\curveto(818.91852935,314.44134641)(818.85852941,314.43634641)(818.79853271,314.42634644)
\curveto(818.73852953,314.42634642)(818.68352959,314.43134642)(818.63353271,314.44134644)
\lineto(818.45353271,314.44134644)
\curveto(818.40352987,314.4513464)(818.35352992,314.45634639)(818.30353271,314.45634644)
\curveto(818.26353001,314.46634638)(818.21853005,314.47134638)(818.16853271,314.47134644)
\curveto(817.9685303,314.52134633)(817.79353048,314.57634627)(817.64353271,314.63634644)
\curveto(817.50353077,314.69634615)(817.38353089,314.80134605)(817.28353271,314.95134644)
\curveto(817.14353113,315.1513457)(817.06353121,315.40134545)(817.04353271,315.70134644)
\curveto(817.02353125,316.01134484)(817.01353126,316.34134451)(817.01353271,316.69134644)
\lineto(817.01353271,320.62134644)
\curveto(816.98353129,320.7513401)(816.95353132,320.84634)(816.92353271,320.90634644)
\curveto(816.90353137,320.96633988)(816.83353144,321.01633983)(816.71353271,321.05634644)
\curveto(816.6735316,321.06633978)(816.63353164,321.06633978)(816.59353271,321.05634644)
\curveto(816.55353172,321.0463398)(816.51353176,321.0513398)(816.47353271,321.07134644)
\lineto(816.23353271,321.07134644)
\curveto(816.10353217,321.07133978)(815.99353228,321.08133977)(815.90353271,321.10134644)
\curveto(815.82353245,321.13133972)(815.7685325,321.19133966)(815.73853271,321.28134644)
\curveto(815.71853255,321.32133953)(815.70353257,321.36633948)(815.69353271,321.41634644)
\lineto(815.69353271,321.56634644)
\curveto(815.69353258,321.70633914)(815.70353257,321.82133903)(815.72353271,321.91134644)
\curveto(815.74353253,322.01133884)(815.80353247,322.08633876)(815.90353271,322.13634644)
\curveto(816.01353226,322.17633867)(816.15353212,322.18633866)(816.32353271,322.16634644)
\curveto(816.50353177,322.1463387)(816.65353162,322.15633869)(816.77353271,322.19634644)
\curveto(816.86353141,322.2463386)(816.93353134,322.31633853)(816.98353271,322.40634644)
\curveto(817.00353127,322.46633838)(817.01353126,322.54133831)(817.01353271,322.63134644)
\lineto(817.01353271,322.88634644)
\lineto(817.01353271,323.81634644)
\lineto(817.01353271,324.05634644)
\curveto(817.01353126,324.1463367)(817.02353125,324.22133663)(817.04353271,324.28134644)
\curveto(817.08353119,324.36133649)(817.15853111,324.42633642)(817.26853271,324.47634644)
\curveto(817.29853097,324.47633637)(817.32353095,324.47633637)(817.34353271,324.47634644)
\curveto(817.3735309,324.48633636)(817.39853087,324.49133636)(817.41853271,324.49134644)
}
}
{
\newrgbcolor{curcolor}{0 0 0}
\pscustom[linestyle=none,fillstyle=solid,fillcolor=curcolor]
{
\newpath
\moveto(828.07532959,314.98134644)
\curveto(828.10532176,314.82134603)(828.09032177,314.68634616)(828.03032959,314.57634644)
\curveto(827.97032189,314.47634637)(827.89032197,314.40134645)(827.79032959,314.35134644)
\curveto(827.74032212,314.33134652)(827.68532218,314.32134653)(827.62532959,314.32134644)
\curveto(827.57532229,314.32134653)(827.52032234,314.31134654)(827.46032959,314.29134644)
\curveto(827.24032262,314.24134661)(827.02032284,314.25634659)(826.80032959,314.33634644)
\curveto(826.59032327,314.40634644)(826.44532342,314.49634635)(826.36532959,314.60634644)
\curveto(826.31532355,314.67634617)(826.27032359,314.75634609)(826.23032959,314.84634644)
\curveto(826.19032367,314.9463459)(826.14032372,315.02634582)(826.08032959,315.08634644)
\curveto(826.0603238,315.10634574)(826.03532383,315.12634572)(826.00532959,315.14634644)
\curveto(825.98532388,315.16634568)(825.95532391,315.17134568)(825.91532959,315.16134644)
\curveto(825.80532406,315.13134572)(825.70032416,315.07634577)(825.60032959,314.99634644)
\curveto(825.51032435,314.91634593)(825.42032444,314.846346)(825.33032959,314.78634644)
\curveto(825.20032466,314.70634614)(825.0603248,314.63134622)(824.91032959,314.56134644)
\curveto(824.7603251,314.50134635)(824.60032526,314.4463464)(824.43032959,314.39634644)
\curveto(824.33032553,314.36634648)(824.22032564,314.3463465)(824.10032959,314.33634644)
\curveto(823.99032587,314.32634652)(823.88032598,314.31134654)(823.77032959,314.29134644)
\curveto(823.72032614,314.28134657)(823.67532619,314.27634657)(823.63532959,314.27634644)
\lineto(823.53032959,314.27634644)
\curveto(823.42032644,314.25634659)(823.31532655,314.25634659)(823.21532959,314.27634644)
\lineto(823.08032959,314.27634644)
\curveto(823.03032683,314.28634656)(822.98032688,314.29134656)(822.93032959,314.29134644)
\curveto(822.88032698,314.29134656)(822.83532703,314.30134655)(822.79532959,314.32134644)
\curveto(822.75532711,314.33134652)(822.72032714,314.33634651)(822.69032959,314.33634644)
\curveto(822.67032719,314.32634652)(822.64532722,314.32634652)(822.61532959,314.33634644)
\lineto(822.37532959,314.39634644)
\curveto(822.29532757,314.40634644)(822.22032764,314.42634642)(822.15032959,314.45634644)
\curveto(821.85032801,314.58634626)(821.60532826,314.73134612)(821.41532959,314.89134644)
\curveto(821.23532863,315.06134579)(821.08532878,315.29634555)(820.96532959,315.59634644)
\curveto(820.87532899,315.81634503)(820.83032903,316.08134477)(820.83032959,316.39134644)
\lineto(820.83032959,316.70634644)
\curveto(820.84032902,316.75634409)(820.84532902,316.80634404)(820.84532959,316.85634644)
\lineto(820.87532959,317.03634644)
\lineto(820.99532959,317.36634644)
\curveto(821.03532883,317.47634337)(821.08532878,317.57634327)(821.14532959,317.66634644)
\curveto(821.32532854,317.95634289)(821.57032829,318.17134268)(821.88032959,318.31134644)
\curveto(822.19032767,318.4513424)(822.53032733,318.57634227)(822.90032959,318.68634644)
\curveto(823.04032682,318.72634212)(823.18532668,318.75634209)(823.33532959,318.77634644)
\curveto(823.48532638,318.79634205)(823.63532623,318.82134203)(823.78532959,318.85134644)
\curveto(823.85532601,318.87134198)(823.92032594,318.88134197)(823.98032959,318.88134644)
\curveto(824.05032581,318.88134197)(824.12532574,318.89134196)(824.20532959,318.91134644)
\curveto(824.27532559,318.93134192)(824.34532552,318.94134191)(824.41532959,318.94134644)
\curveto(824.48532538,318.9513419)(824.5603253,318.96634188)(824.64032959,318.98634644)
\curveto(824.89032497,319.0463418)(825.12532474,319.09634175)(825.34532959,319.13634644)
\curveto(825.5653243,319.18634166)(825.74032412,319.30134155)(825.87032959,319.48134644)
\curveto(825.93032393,319.56134129)(825.98032388,319.66134119)(826.02032959,319.78134644)
\curveto(826.0603238,319.91134094)(826.0603238,320.0513408)(826.02032959,320.20134644)
\curveto(825.9603239,320.44134041)(825.87032399,320.63134022)(825.75032959,320.77134644)
\curveto(825.64032422,320.91133994)(825.48032438,321.02133983)(825.27032959,321.10134644)
\curveto(825.15032471,321.1513397)(825.00532486,321.18633966)(824.83532959,321.20634644)
\curveto(824.67532519,321.22633962)(824.50532536,321.23633961)(824.32532959,321.23634644)
\curveto(824.14532572,321.23633961)(823.97032589,321.22633962)(823.80032959,321.20634644)
\curveto(823.63032623,321.18633966)(823.48532638,321.15633969)(823.36532959,321.11634644)
\curveto(823.19532667,321.05633979)(823.03032683,320.97133988)(822.87032959,320.86134644)
\curveto(822.79032707,320.80134005)(822.71532715,320.72134013)(822.64532959,320.62134644)
\curveto(822.58532728,320.53134032)(822.53032733,320.43134042)(822.48032959,320.32134644)
\curveto(822.45032741,320.24134061)(822.42032744,320.15634069)(822.39032959,320.06634644)
\curveto(822.37032749,319.97634087)(822.32532754,319.90634094)(822.25532959,319.85634644)
\curveto(822.21532765,319.82634102)(822.14532772,319.80134105)(822.04532959,319.78134644)
\curveto(821.95532791,319.77134108)(821.860328,319.76634108)(821.76032959,319.76634644)
\curveto(821.6603282,319.76634108)(821.5603283,319.77134108)(821.46032959,319.78134644)
\curveto(821.37032849,319.80134105)(821.30532856,319.82634102)(821.26532959,319.85634644)
\curveto(821.22532864,319.88634096)(821.19532867,319.93634091)(821.17532959,320.00634644)
\curveto(821.15532871,320.07634077)(821.15532871,320.1513407)(821.17532959,320.23134644)
\curveto(821.20532866,320.36134049)(821.23532863,320.48134037)(821.26532959,320.59134644)
\curveto(821.30532856,320.71134014)(821.35032851,320.82634002)(821.40032959,320.93634644)
\curveto(821.59032827,321.28633956)(821.83032803,321.55633929)(822.12032959,321.74634644)
\curveto(822.41032745,321.9463389)(822.77032709,322.10633874)(823.20032959,322.22634644)
\curveto(823.30032656,322.2463386)(823.40032646,322.26133859)(823.50032959,322.27134644)
\curveto(823.61032625,322.28133857)(823.72032614,322.29633855)(823.83032959,322.31634644)
\curveto(823.87032599,322.32633852)(823.93532593,322.32633852)(824.02532959,322.31634644)
\curveto(824.11532575,322.31633853)(824.17032569,322.32633852)(824.19032959,322.34634644)
\curveto(824.89032497,322.35633849)(825.50032436,322.27633857)(826.02032959,322.10634644)
\curveto(826.54032332,321.93633891)(826.90532296,321.61133924)(827.11532959,321.13134644)
\curveto(827.20532266,320.93133992)(827.25532261,320.69634015)(827.26532959,320.42634644)
\curveto(827.28532258,320.16634068)(827.29532257,319.89134096)(827.29532959,319.60134644)
\lineto(827.29532959,316.28634644)
\curveto(827.29532257,316.1463447)(827.30032256,316.01134484)(827.31032959,315.88134644)
\curveto(827.32032254,315.7513451)(827.35032251,315.6463452)(827.40032959,315.56634644)
\curveto(827.45032241,315.49634535)(827.51532235,315.4463454)(827.59532959,315.41634644)
\curveto(827.68532218,315.37634547)(827.77032209,315.3463455)(827.85032959,315.32634644)
\curveto(827.93032193,315.31634553)(827.99032187,315.27134558)(828.03032959,315.19134644)
\curveto(828.05032181,315.16134569)(828.0603218,315.13134572)(828.06032959,315.10134644)
\curveto(828.0603218,315.07134578)(828.0653218,315.03134582)(828.07532959,314.98134644)
\moveto(825.93032959,316.64634644)
\curveto(825.99032387,316.78634406)(826.02032384,316.9463439)(826.02032959,317.12634644)
\curveto(826.03032383,317.31634353)(826.03532383,317.51134334)(826.03532959,317.71134644)
\curveto(826.03532383,317.82134303)(826.03032383,317.92134293)(826.02032959,318.01134644)
\curveto(826.01032385,318.10134275)(825.97032389,318.17134268)(825.90032959,318.22134644)
\curveto(825.87032399,318.24134261)(825.80032406,318.2513426)(825.69032959,318.25134644)
\curveto(825.67032419,318.23134262)(825.63532423,318.22134263)(825.58532959,318.22134644)
\curveto(825.53532433,318.22134263)(825.49032437,318.21134264)(825.45032959,318.19134644)
\curveto(825.37032449,318.17134268)(825.28032458,318.1513427)(825.18032959,318.13134644)
\lineto(824.88032959,318.07134644)
\curveto(824.85032501,318.07134278)(824.81532505,318.06634278)(824.77532959,318.05634644)
\lineto(824.67032959,318.05634644)
\curveto(824.52032534,318.01634283)(824.35532551,317.99134286)(824.17532959,317.98134644)
\curveto(824.00532586,317.98134287)(823.84532602,317.96134289)(823.69532959,317.92134644)
\curveto(823.61532625,317.90134295)(823.54032632,317.88134297)(823.47032959,317.86134644)
\curveto(823.41032645,317.851343)(823.34032652,317.83634301)(823.26032959,317.81634644)
\curveto(823.10032676,317.76634308)(822.95032691,317.70134315)(822.81032959,317.62134644)
\curveto(822.67032719,317.5513433)(822.55032731,317.46134339)(822.45032959,317.35134644)
\curveto(822.35032751,317.24134361)(822.27532759,317.10634374)(822.22532959,316.94634644)
\curveto(822.17532769,316.79634405)(822.15532771,316.61134424)(822.16532959,316.39134644)
\curveto(822.1653277,316.29134456)(822.18032768,316.19634465)(822.21032959,316.10634644)
\curveto(822.25032761,316.02634482)(822.29532757,315.9513449)(822.34532959,315.88134644)
\curveto(822.42532744,315.77134508)(822.53032733,315.67634517)(822.66032959,315.59634644)
\curveto(822.79032707,315.52634532)(822.93032693,315.46634538)(823.08032959,315.41634644)
\curveto(823.13032673,315.40634544)(823.18032668,315.40134545)(823.23032959,315.40134644)
\curveto(823.28032658,315.40134545)(823.33032653,315.39634545)(823.38032959,315.38634644)
\curveto(823.45032641,315.36634548)(823.53532633,315.3513455)(823.63532959,315.34134644)
\curveto(823.74532612,315.34134551)(823.83532603,315.3513455)(823.90532959,315.37134644)
\curveto(823.9653259,315.39134546)(824.02532584,315.39634545)(824.08532959,315.38634644)
\curveto(824.14532572,315.38634546)(824.20532566,315.39634545)(824.26532959,315.41634644)
\curveto(824.34532552,315.43634541)(824.42032544,315.4513454)(824.49032959,315.46134644)
\curveto(824.57032529,315.47134538)(824.64532522,315.49134536)(824.71532959,315.52134644)
\curveto(825.00532486,315.64134521)(825.25032461,315.78634506)(825.45032959,315.95634644)
\curveto(825.6603242,316.12634472)(825.82032404,316.35634449)(825.93032959,316.64634644)
}
}
{
\newrgbcolor{curcolor}{0 0 0}
\pscustom[linestyle=none,fillstyle=solid,fillcolor=curcolor]
{
\newpath
\moveto(832.89197021,322.33134644)
\curveto(833.12196542,322.33133852)(833.25196529,322.27133858)(833.28197021,322.15134644)
\curveto(833.31196523,322.04133881)(833.32696522,321.87633897)(833.32697021,321.65634644)
\lineto(833.32697021,321.37134644)
\curveto(833.32696522,321.28133957)(833.30196524,321.20633964)(833.25197021,321.14634644)
\curveto(833.19196535,321.06633978)(833.10696544,321.02133983)(832.99697021,321.01134644)
\curveto(832.88696566,321.01133984)(832.77696577,320.99633985)(832.66697021,320.96634644)
\curveto(832.52696602,320.93633991)(832.39196615,320.90633994)(832.26197021,320.87634644)
\curveto(832.1419664,320.84634)(832.02696652,320.80634004)(831.91697021,320.75634644)
\curveto(831.62696692,320.62634022)(831.39196715,320.4463404)(831.21197021,320.21634644)
\curveto(831.03196751,319.99634085)(830.87696767,319.74134111)(830.74697021,319.45134644)
\curveto(830.70696784,319.34134151)(830.67696787,319.22634162)(830.65697021,319.10634644)
\curveto(830.63696791,318.99634185)(830.61196793,318.88134197)(830.58197021,318.76134644)
\curveto(830.57196797,318.71134214)(830.56696798,318.66134219)(830.56697021,318.61134644)
\curveto(830.57696797,318.56134229)(830.57696797,318.51134234)(830.56697021,318.46134644)
\curveto(830.53696801,318.34134251)(830.52196802,318.20134265)(830.52197021,318.04134644)
\curveto(830.53196801,317.89134296)(830.53696801,317.7463431)(830.53697021,317.60634644)
\lineto(830.53697021,315.76134644)
\lineto(830.53697021,315.41634644)
\curveto(830.53696801,315.29634555)(830.53196801,315.18134567)(830.52197021,315.07134644)
\curveto(830.51196803,314.96134589)(830.50696804,314.86634598)(830.50697021,314.78634644)
\curveto(830.51696803,314.70634614)(830.49696805,314.63634621)(830.44697021,314.57634644)
\curveto(830.39696815,314.50634634)(830.31696823,314.46634638)(830.20697021,314.45634644)
\curveto(830.10696844,314.4463464)(829.99696855,314.44134641)(829.87697021,314.44134644)
\lineto(829.60697021,314.44134644)
\curveto(829.55696899,314.46134639)(829.50696904,314.47634637)(829.45697021,314.48634644)
\curveto(829.41696913,314.50634634)(829.38696916,314.53134632)(829.36697021,314.56134644)
\curveto(829.31696923,314.63134622)(829.28696926,314.71634613)(829.27697021,314.81634644)
\lineto(829.27697021,315.14634644)
\lineto(829.27697021,316.30134644)
\lineto(829.27697021,320.45634644)
\lineto(829.27697021,321.49134644)
\lineto(829.27697021,321.79134644)
\curveto(829.28696926,321.89133896)(829.31696923,321.97633887)(829.36697021,322.04634644)
\curveto(829.39696915,322.08633876)(829.4469691,322.11633873)(829.51697021,322.13634644)
\curveto(829.59696895,322.15633869)(829.68196886,322.16633868)(829.77197021,322.16634644)
\curveto(829.86196868,322.17633867)(829.95196859,322.17633867)(830.04197021,322.16634644)
\curveto(830.13196841,322.15633869)(830.20196834,322.14133871)(830.25197021,322.12134644)
\curveto(830.33196821,322.09133876)(830.38196816,322.03133882)(830.40197021,321.94134644)
\curveto(830.43196811,321.86133899)(830.4469681,321.77133908)(830.44697021,321.67134644)
\lineto(830.44697021,321.37134644)
\curveto(830.4469681,321.27133958)(830.46696808,321.18133967)(830.50697021,321.10134644)
\curveto(830.51696803,321.08133977)(830.52696802,321.06633978)(830.53697021,321.05634644)
\lineto(830.58197021,321.01134644)
\curveto(830.69196785,321.01133984)(830.78196776,321.05633979)(830.85197021,321.14634644)
\curveto(830.92196762,321.2463396)(830.98196756,321.32633952)(831.03197021,321.38634644)
\lineto(831.12197021,321.47634644)
\curveto(831.21196733,321.58633926)(831.33696721,321.70133915)(831.49697021,321.82134644)
\curveto(831.65696689,321.94133891)(831.80696674,322.03133882)(831.94697021,322.09134644)
\curveto(832.03696651,322.14133871)(832.13196641,322.17633867)(832.23197021,322.19634644)
\curveto(832.33196621,322.22633862)(832.43696611,322.25633859)(832.54697021,322.28634644)
\curveto(832.60696594,322.29633855)(832.66696588,322.30133855)(832.72697021,322.30134644)
\curveto(832.78696576,322.31133854)(832.8419657,322.32133853)(832.89197021,322.33134644)
}
}
{
\newrgbcolor{curcolor}{0 0 0}
\pscustom[linestyle=none,fillstyle=solid,fillcolor=curcolor]
{
\newpath
\moveto(834.54173584,323.65134644)
\curveto(834.46173472,323.71133714)(834.41673476,323.81633703)(834.40673584,323.96634644)
\lineto(834.40673584,324.43134644)
\lineto(834.40673584,324.68634644)
\curveto(834.40673477,324.77633607)(834.42173476,324.851336)(834.45173584,324.91134644)
\curveto(834.49173469,324.99133586)(834.57173461,325.0513358)(834.69173584,325.09134644)
\curveto(834.71173447,325.10133575)(834.73173445,325.10133575)(834.75173584,325.09134644)
\curveto(834.7817344,325.09133576)(834.80673437,325.09633575)(834.82673584,325.10634644)
\curveto(834.99673418,325.10633574)(835.15673402,325.10133575)(835.30673584,325.09134644)
\curveto(835.45673372,325.08133577)(835.55673362,325.02133583)(835.60673584,324.91134644)
\curveto(835.63673354,324.851336)(835.65173353,324.77633607)(835.65173584,324.68634644)
\lineto(835.65173584,324.43134644)
\curveto(835.65173353,324.2513366)(835.64673353,324.08133677)(835.63673584,323.92134644)
\curveto(835.63673354,323.76133709)(835.57173361,323.65633719)(835.44173584,323.60634644)
\curveto(835.39173379,323.58633726)(835.33673384,323.57633727)(835.27673584,323.57634644)
\lineto(835.11173584,323.57634644)
\lineto(834.79673584,323.57634644)
\curveto(834.69673448,323.57633727)(834.61173457,323.60133725)(834.54173584,323.65134644)
\moveto(835.65173584,315.14634644)
\lineto(835.65173584,314.83134644)
\curveto(835.66173352,314.73134612)(835.64173354,314.6513462)(835.59173584,314.59134644)
\curveto(835.56173362,314.53134632)(835.51673366,314.49134636)(835.45673584,314.47134644)
\curveto(835.39673378,314.46134639)(835.32673385,314.4463464)(835.24673584,314.42634644)
\lineto(835.02173584,314.42634644)
\curveto(834.89173429,314.42634642)(834.7767344,314.43134642)(834.67673584,314.44134644)
\curveto(834.58673459,314.46134639)(834.51673466,314.51134634)(834.46673584,314.59134644)
\curveto(834.42673475,314.6513462)(834.40673477,314.72634612)(834.40673584,314.81634644)
\lineto(834.40673584,315.10134644)
\lineto(834.40673584,321.44634644)
\lineto(834.40673584,321.76134644)
\curveto(834.40673477,321.87133898)(834.43173475,321.95633889)(834.48173584,322.01634644)
\curveto(834.51173467,322.06633878)(834.55173463,322.09633875)(834.60173584,322.10634644)
\curveto(834.65173453,322.11633873)(834.70673447,322.13133872)(834.76673584,322.15134644)
\curveto(834.78673439,322.1513387)(834.80673437,322.1463387)(834.82673584,322.13634644)
\curveto(834.85673432,322.13633871)(834.8817343,322.14133871)(834.90173584,322.15134644)
\curveto(835.03173415,322.1513387)(835.16173402,322.1463387)(835.29173584,322.13634644)
\curveto(835.43173375,322.13633871)(835.52673365,322.09633875)(835.57673584,322.01634644)
\curveto(835.62673355,321.95633889)(835.65173353,321.87633897)(835.65173584,321.77634644)
\lineto(835.65173584,321.49134644)
\lineto(835.65173584,315.14634644)
}
}
{
\newrgbcolor{curcolor}{0 0 0}
\pscustom[linestyle=none,fillstyle=solid,fillcolor=curcolor]
{
\newpath
\moveto(844.72157959,318.62634644)
\curveto(844.74157153,318.56634228)(844.75157152,318.47134238)(844.75157959,318.34134644)
\curveto(844.75157152,318.22134263)(844.74657152,318.13634271)(844.73657959,318.08634644)
\lineto(844.73657959,317.93634644)
\curveto(844.72657154,317.85634299)(844.71657155,317.78134307)(844.70657959,317.71134644)
\curveto(844.70657156,317.6513432)(844.70157157,317.58134327)(844.69157959,317.50134644)
\curveto(844.6715716,317.44134341)(844.65657161,317.38134347)(844.64657959,317.32134644)
\curveto(844.64657162,317.26134359)(844.63657163,317.20134365)(844.61657959,317.14134644)
\curveto(844.57657169,317.01134384)(844.54157173,316.88134397)(844.51157959,316.75134644)
\curveto(844.48157179,316.62134423)(844.44157183,316.50134435)(844.39157959,316.39134644)
\curveto(844.18157209,315.91134494)(843.90157237,315.50634534)(843.55157959,315.17634644)
\curveto(843.20157307,314.85634599)(842.7715735,314.61134624)(842.26157959,314.44134644)
\curveto(842.15157412,314.40134645)(842.03157424,314.37134648)(841.90157959,314.35134644)
\curveto(841.78157449,314.33134652)(841.65657461,314.31134654)(841.52657959,314.29134644)
\curveto(841.4665748,314.28134657)(841.40157487,314.27634657)(841.33157959,314.27634644)
\curveto(841.271575,314.26634658)(841.21157506,314.26134659)(841.15157959,314.26134644)
\curveto(841.11157516,314.2513466)(841.05157522,314.2463466)(840.97157959,314.24634644)
\curveto(840.90157537,314.2463466)(840.85157542,314.2513466)(840.82157959,314.26134644)
\curveto(840.78157549,314.27134658)(840.74157553,314.27634657)(840.70157959,314.27634644)
\curveto(840.66157561,314.26634658)(840.62657564,314.26634658)(840.59657959,314.27634644)
\lineto(840.50657959,314.27634644)
\lineto(840.14657959,314.32134644)
\curveto(840.00657626,314.36134649)(839.8715764,314.40134645)(839.74157959,314.44134644)
\curveto(839.61157666,314.48134637)(839.48657678,314.52634632)(839.36657959,314.57634644)
\curveto(838.91657735,314.77634607)(838.54657772,315.03634581)(838.25657959,315.35634644)
\curveto(837.9665783,315.67634517)(837.72657854,316.06634478)(837.53657959,316.52634644)
\curveto(837.48657878,316.62634422)(837.44657882,316.72634412)(837.41657959,316.82634644)
\curveto(837.39657887,316.92634392)(837.37657889,317.03134382)(837.35657959,317.14134644)
\curveto(837.33657893,317.18134367)(837.32657894,317.21134364)(837.32657959,317.23134644)
\curveto(837.33657893,317.26134359)(837.33657893,317.29634355)(837.32657959,317.33634644)
\curveto(837.30657896,317.41634343)(837.29157898,317.49634335)(837.28157959,317.57634644)
\curveto(837.28157899,317.66634318)(837.271579,317.7513431)(837.25157959,317.83134644)
\lineto(837.25157959,317.95134644)
\curveto(837.25157902,317.99134286)(837.24657902,318.03634281)(837.23657959,318.08634644)
\curveto(837.22657904,318.13634271)(837.22157905,318.22134263)(837.22157959,318.34134644)
\curveto(837.22157905,318.47134238)(837.23157904,318.56634228)(837.25157959,318.62634644)
\curveto(837.271579,318.69634215)(837.27657899,318.76634208)(837.26657959,318.83634644)
\curveto(837.25657901,318.90634194)(837.26157901,318.97634187)(837.28157959,319.04634644)
\curveto(837.29157898,319.09634175)(837.29657897,319.13634171)(837.29657959,319.16634644)
\curveto(837.30657896,319.20634164)(837.31657895,319.2513416)(837.32657959,319.30134644)
\curveto(837.35657891,319.42134143)(837.38157889,319.54134131)(837.40157959,319.66134644)
\curveto(837.43157884,319.78134107)(837.4715788,319.89634095)(837.52157959,320.00634644)
\curveto(837.6715786,320.37634047)(837.85157842,320.70634014)(838.06157959,320.99634644)
\curveto(838.28157799,321.29633955)(838.54657772,321.5463393)(838.85657959,321.74634644)
\curveto(838.97657729,321.82633902)(839.10157717,321.89133896)(839.23157959,321.94134644)
\curveto(839.36157691,322.00133885)(839.49657677,322.06133879)(839.63657959,322.12134644)
\curveto(839.75657651,322.17133868)(839.88657638,322.20133865)(840.02657959,322.21134644)
\curveto(840.1665761,322.23133862)(840.30657596,322.26133859)(840.44657959,322.30134644)
\lineto(840.64157959,322.30134644)
\curveto(840.71157556,322.31133854)(840.77657549,322.32133853)(840.83657959,322.33134644)
\curveto(841.72657454,322.34133851)(842.4665738,322.15633869)(843.05657959,321.77634644)
\curveto(843.64657262,321.39633945)(844.0715722,320.90133995)(844.33157959,320.29134644)
\curveto(844.38157189,320.19134066)(844.42157185,320.09134076)(844.45157959,319.99134644)
\curveto(844.48157179,319.89134096)(844.51657175,319.78634106)(844.55657959,319.67634644)
\curveto(844.58657168,319.56634128)(844.61157166,319.4463414)(844.63157959,319.31634644)
\curveto(844.65157162,319.19634165)(844.67657159,319.07134178)(844.70657959,318.94134644)
\curveto(844.71657155,318.89134196)(844.71657155,318.83634201)(844.70657959,318.77634644)
\curveto(844.70657156,318.72634212)(844.71157156,318.67634217)(844.72157959,318.62634644)
\moveto(843.38657959,317.77134644)
\curveto(843.40657286,317.84134301)(843.41157286,317.92134293)(843.40157959,318.01134644)
\lineto(843.40157959,318.26634644)
\curveto(843.40157287,318.65634219)(843.3665729,318.98634186)(843.29657959,319.25634644)
\curveto(843.266573,319.33634151)(843.24157303,319.41634143)(843.22157959,319.49634644)
\curveto(843.20157307,319.57634127)(843.17657309,319.6513412)(843.14657959,319.72134644)
\curveto(842.8665734,320.37134048)(842.42157385,320.82134003)(841.81157959,321.07134644)
\curveto(841.74157453,321.10133975)(841.6665746,321.12133973)(841.58657959,321.13134644)
\lineto(841.34657959,321.19134644)
\curveto(841.266575,321.21133964)(841.18157509,321.22133963)(841.09157959,321.22134644)
\lineto(840.82157959,321.22134644)
\lineto(840.55157959,321.17634644)
\curveto(840.45157582,321.15633969)(840.35657591,321.13133972)(840.26657959,321.10134644)
\curveto(840.18657608,321.08133977)(840.10657616,321.0513398)(840.02657959,321.01134644)
\curveto(839.95657631,320.99133986)(839.89157638,320.96133989)(839.83157959,320.92134644)
\curveto(839.7715765,320.88133997)(839.71657655,320.84134001)(839.66657959,320.80134644)
\curveto(839.42657684,320.63134022)(839.23157704,320.42634042)(839.08157959,320.18634644)
\curveto(838.93157734,319.9463409)(838.80157747,319.66634118)(838.69157959,319.34634644)
\curveto(838.66157761,319.2463416)(838.64157763,319.14134171)(838.63157959,319.03134644)
\curveto(838.62157765,318.93134192)(838.60657766,318.82634202)(838.58657959,318.71634644)
\curveto(838.57657769,318.67634217)(838.5715777,318.61134224)(838.57157959,318.52134644)
\curveto(838.56157771,318.49134236)(838.55657771,318.45634239)(838.55657959,318.41634644)
\curveto(838.5665777,318.37634247)(838.5715777,318.33134252)(838.57157959,318.28134644)
\lineto(838.57157959,317.98134644)
\curveto(838.5715777,317.88134297)(838.58157769,317.79134306)(838.60157959,317.71134644)
\lineto(838.63157959,317.53134644)
\curveto(838.65157762,317.43134342)(838.6665776,317.33134352)(838.67657959,317.23134644)
\curveto(838.69657757,317.14134371)(838.72657754,317.05634379)(838.76657959,316.97634644)
\curveto(838.8665774,316.73634411)(838.98157729,316.51134434)(839.11157959,316.30134644)
\curveto(839.25157702,316.09134476)(839.42157685,315.91634493)(839.62157959,315.77634644)
\curveto(839.6715766,315.7463451)(839.71657655,315.72134513)(839.75657959,315.70134644)
\curveto(839.79657647,315.68134517)(839.84157643,315.65634519)(839.89157959,315.62634644)
\curveto(839.9715763,315.57634527)(840.05657621,315.53134532)(840.14657959,315.49134644)
\curveto(840.24657602,315.46134539)(840.35157592,315.43134542)(840.46157959,315.40134644)
\curveto(840.51157576,315.38134547)(840.55657571,315.37134548)(840.59657959,315.37134644)
\curveto(840.64657562,315.38134547)(840.69657557,315.38134547)(840.74657959,315.37134644)
\curveto(840.77657549,315.36134549)(840.83657543,315.3513455)(840.92657959,315.34134644)
\curveto(841.02657524,315.33134552)(841.10157517,315.33634551)(841.15157959,315.35634644)
\curveto(841.19157508,315.36634548)(841.23157504,315.36634548)(841.27157959,315.35634644)
\curveto(841.31157496,315.35634549)(841.35157492,315.36634548)(841.39157959,315.38634644)
\curveto(841.4715748,315.40634544)(841.55157472,315.42134543)(841.63157959,315.43134644)
\curveto(841.71157456,315.4513454)(841.78657448,315.47634537)(841.85657959,315.50634644)
\curveto(842.19657407,315.6463452)(842.4715738,315.84134501)(842.68157959,316.09134644)
\curveto(842.89157338,316.34134451)(843.0665732,316.63634421)(843.20657959,316.97634644)
\curveto(843.25657301,317.09634375)(843.28657298,317.22134363)(843.29657959,317.35134644)
\curveto(843.31657295,317.49134336)(843.34657292,317.63134322)(843.38657959,317.77134644)
}
}
{
\newrgbcolor{curcolor}{0 0 0}
\pscustom[linestyle=none,fillstyle=solid,fillcolor=curcolor]
{
\newpath
\moveto(848.63986084,322.33134644)
\curveto(849.35985677,322.34133851)(849.96485617,322.25633859)(850.45486084,322.07634644)
\curveto(850.94485519,321.90633894)(851.32485481,321.60133925)(851.59486084,321.16134644)
\curveto(851.66485447,321.0513398)(851.71985441,320.93633991)(851.75986084,320.81634644)
\curveto(851.79985433,320.70634014)(851.83985429,320.58134027)(851.87986084,320.44134644)
\curveto(851.89985423,320.37134048)(851.90485423,320.29634055)(851.89486084,320.21634644)
\curveto(851.88485425,320.1463407)(851.86985426,320.09134076)(851.84986084,320.05134644)
\curveto(851.8298543,320.03134082)(851.80485433,320.01134084)(851.77486084,319.99134644)
\curveto(851.74485439,319.98134087)(851.71985441,319.96634088)(851.69986084,319.94634644)
\curveto(851.64985448,319.92634092)(851.59985453,319.92134093)(851.54986084,319.93134644)
\curveto(851.49985463,319.94134091)(851.44985468,319.94134091)(851.39986084,319.93134644)
\curveto(851.31985481,319.91134094)(851.21485492,319.90634094)(851.08486084,319.91634644)
\curveto(850.95485518,319.93634091)(850.86485527,319.96134089)(850.81486084,319.99134644)
\curveto(850.7348554,320.04134081)(850.67985545,320.10634074)(850.64986084,320.18634644)
\curveto(850.6298555,320.27634057)(850.59485554,320.36134049)(850.54486084,320.44134644)
\curveto(850.45485568,320.60134025)(850.3298558,320.7463401)(850.16986084,320.87634644)
\curveto(850.05985607,320.95633989)(849.93985619,321.01633983)(849.80986084,321.05634644)
\curveto(849.67985645,321.09633975)(849.53985659,321.13633971)(849.38986084,321.17634644)
\curveto(849.33985679,321.19633965)(849.28985684,321.20133965)(849.23986084,321.19134644)
\curveto(849.18985694,321.19133966)(849.13985699,321.19633965)(849.08986084,321.20634644)
\curveto(849.0298571,321.22633962)(848.95485718,321.23633961)(848.86486084,321.23634644)
\curveto(848.77485736,321.23633961)(848.69985743,321.22633962)(848.63986084,321.20634644)
\lineto(848.54986084,321.20634644)
\lineto(848.39986084,321.17634644)
\curveto(848.34985778,321.17633967)(848.29985783,321.17133968)(848.24986084,321.16134644)
\curveto(847.98985814,321.10133975)(847.77485836,321.01633983)(847.60486084,320.90634644)
\curveto(847.4348587,320.79634005)(847.31985881,320.61134024)(847.25986084,320.35134644)
\curveto(847.23985889,320.28134057)(847.2348589,320.21134064)(847.24486084,320.14134644)
\curveto(847.26485887,320.07134078)(847.28485885,320.01134084)(847.30486084,319.96134644)
\curveto(847.36485877,319.81134104)(847.4348587,319.70134115)(847.51486084,319.63134644)
\curveto(847.60485853,319.57134128)(847.71485842,319.50134135)(847.84486084,319.42134644)
\curveto(848.00485813,319.32134153)(848.18485795,319.2463416)(848.38486084,319.19634644)
\curveto(848.58485755,319.15634169)(848.78485735,319.10634174)(848.98486084,319.04634644)
\curveto(849.11485702,319.00634184)(849.24485689,318.97634187)(849.37486084,318.95634644)
\curveto(849.50485663,318.93634191)(849.6348565,318.90634194)(849.76486084,318.86634644)
\curveto(849.97485616,318.80634204)(850.17985595,318.7463421)(850.37986084,318.68634644)
\curveto(850.57985555,318.63634221)(850.77985535,318.57134228)(850.97986084,318.49134644)
\lineto(851.12986084,318.43134644)
\curveto(851.17985495,318.41134244)(851.2298549,318.38634246)(851.27986084,318.35634644)
\curveto(851.47985465,318.23634261)(851.65485448,318.10134275)(851.80486084,317.95134644)
\curveto(851.95485418,317.80134305)(852.07985405,317.61134324)(852.17986084,317.38134644)
\curveto(852.19985393,317.31134354)(852.21985391,317.21634363)(852.23986084,317.09634644)
\curveto(852.25985387,317.02634382)(852.26985386,316.9513439)(852.26986084,316.87134644)
\curveto(852.27985385,316.80134405)(852.28485385,316.72134413)(852.28486084,316.63134644)
\lineto(852.28486084,316.48134644)
\curveto(852.26485387,316.41134444)(852.25485388,316.34134451)(852.25486084,316.27134644)
\curveto(852.25485388,316.20134465)(852.24485389,316.13134472)(852.22486084,316.06134644)
\curveto(852.19485394,315.9513449)(852.15985397,315.846345)(852.11986084,315.74634644)
\curveto(852.07985405,315.6463452)(852.0348541,315.55634529)(851.98486084,315.47634644)
\curveto(851.82485431,315.21634563)(851.61985451,315.00634584)(851.36986084,314.84634644)
\curveto(851.11985501,314.69634615)(850.83985529,314.56634628)(850.52986084,314.45634644)
\curveto(850.43985569,314.42634642)(850.34485579,314.40634644)(850.24486084,314.39634644)
\curveto(850.15485598,314.37634647)(850.06485607,314.3513465)(849.97486084,314.32134644)
\curveto(849.87485626,314.30134655)(849.77485636,314.29134656)(849.67486084,314.29134644)
\curveto(849.57485656,314.29134656)(849.47485666,314.28134657)(849.37486084,314.26134644)
\lineto(849.22486084,314.26134644)
\curveto(849.17485696,314.2513466)(849.10485703,314.2463466)(849.01486084,314.24634644)
\curveto(848.92485721,314.2463466)(848.85485728,314.2513466)(848.80486084,314.26134644)
\lineto(848.63986084,314.26134644)
\curveto(848.57985755,314.28134657)(848.51485762,314.29134656)(848.44486084,314.29134644)
\curveto(848.37485776,314.28134657)(848.31485782,314.28634656)(848.26486084,314.30634644)
\curveto(848.21485792,314.31634653)(848.14985798,314.32134653)(848.06986084,314.32134644)
\lineto(847.82986084,314.38134644)
\curveto(847.75985837,314.39134646)(847.68485845,314.41134644)(847.60486084,314.44134644)
\curveto(847.29485884,314.54134631)(847.02485911,314.66634618)(846.79486084,314.81634644)
\curveto(846.56485957,314.96634588)(846.36485977,315.16134569)(846.19486084,315.40134644)
\curveto(846.10486003,315.53134532)(846.0298601,315.66634518)(845.96986084,315.80634644)
\curveto(845.90986022,315.9463449)(845.85486028,316.10134475)(845.80486084,316.27134644)
\curveto(845.78486035,316.33134452)(845.77486036,316.40134445)(845.77486084,316.48134644)
\curveto(845.78486035,316.57134428)(845.79986033,316.64134421)(845.81986084,316.69134644)
\curveto(845.84986028,316.73134412)(845.89986023,316.77134408)(845.96986084,316.81134644)
\curveto(846.01986011,316.83134402)(846.08986004,316.84134401)(846.17986084,316.84134644)
\curveto(846.26985986,316.851344)(846.35985977,316.851344)(846.44986084,316.84134644)
\curveto(846.53985959,316.83134402)(846.62485951,316.81634403)(846.70486084,316.79634644)
\curveto(846.79485934,316.78634406)(846.85485928,316.77134408)(846.88486084,316.75134644)
\curveto(846.95485918,316.70134415)(846.99985913,316.62634422)(847.01986084,316.52634644)
\curveto(847.04985908,316.43634441)(847.08485905,316.3513445)(847.12486084,316.27134644)
\curveto(847.22485891,316.0513448)(847.35985877,315.88134497)(847.52986084,315.76134644)
\curveto(847.64985848,315.67134518)(847.78485835,315.60134525)(847.93486084,315.55134644)
\curveto(848.08485805,315.50134535)(848.24485789,315.4513454)(848.41486084,315.40134644)
\lineto(848.72986084,315.35634644)
\lineto(848.81986084,315.35634644)
\curveto(848.88985724,315.33634551)(848.97985715,315.32634552)(849.08986084,315.32634644)
\curveto(849.20985692,315.32634552)(849.30985682,315.33634551)(849.38986084,315.35634644)
\curveto(849.45985667,315.35634549)(849.51485662,315.36134549)(849.55486084,315.37134644)
\curveto(849.61485652,315.38134547)(849.67485646,315.38634546)(849.73486084,315.38634644)
\curveto(849.79485634,315.39634545)(849.84985628,315.40634544)(849.89986084,315.41634644)
\curveto(850.18985594,315.49634535)(850.41985571,315.60134525)(850.58986084,315.73134644)
\curveto(850.75985537,315.86134499)(850.87985525,316.08134477)(850.94986084,316.39134644)
\curveto(850.96985516,316.44134441)(850.97485516,316.49634435)(850.96486084,316.55634644)
\curveto(850.95485518,316.61634423)(850.94485519,316.66134419)(850.93486084,316.69134644)
\curveto(850.88485525,316.88134397)(850.81485532,317.02134383)(850.72486084,317.11134644)
\curveto(850.6348555,317.21134364)(850.51985561,317.30134355)(850.37986084,317.38134644)
\curveto(850.28985584,317.44134341)(850.18985594,317.49134336)(850.07986084,317.53134644)
\lineto(849.74986084,317.65134644)
\curveto(849.71985641,317.66134319)(849.68985644,317.66634318)(849.65986084,317.66634644)
\curveto(849.63985649,317.66634318)(849.61485652,317.67634317)(849.58486084,317.69634644)
\curveto(849.24485689,317.80634304)(848.88985724,317.88634296)(848.51986084,317.93634644)
\curveto(848.15985797,317.99634285)(847.81985831,318.09134276)(847.49986084,318.22134644)
\curveto(847.39985873,318.26134259)(847.30485883,318.29634255)(847.21486084,318.32634644)
\curveto(847.12485901,318.35634249)(847.03985909,318.39634245)(846.95986084,318.44634644)
\curveto(846.76985936,318.55634229)(846.59485954,318.68134217)(846.43486084,318.82134644)
\curveto(846.27485986,318.96134189)(846.14985998,319.13634171)(846.05986084,319.34634644)
\curveto(846.0298601,319.41634143)(846.00486013,319.48634136)(845.98486084,319.55634644)
\curveto(845.97486016,319.62634122)(845.95986017,319.70134115)(845.93986084,319.78134644)
\curveto(845.90986022,319.90134095)(845.89986023,320.03634081)(845.90986084,320.18634644)
\curveto(845.91986021,320.3463405)(845.9348602,320.48134037)(845.95486084,320.59134644)
\curveto(845.97486016,320.64134021)(845.98486015,320.68134017)(845.98486084,320.71134644)
\curveto(845.99486014,320.7513401)(846.00986012,320.79134006)(846.02986084,320.83134644)
\curveto(846.11986001,321.06133979)(846.23985989,321.26133959)(846.38986084,321.43134644)
\curveto(846.54985958,321.60133925)(846.7298594,321.7513391)(846.92986084,321.88134644)
\curveto(847.07985905,321.97133888)(847.24485889,322.04133881)(847.42486084,322.09134644)
\curveto(847.60485853,322.1513387)(847.79485834,322.20633864)(847.99486084,322.25634644)
\curveto(848.06485807,322.26633858)(848.129858,322.27633857)(848.18986084,322.28634644)
\curveto(848.25985787,322.29633855)(848.3348578,322.30633854)(848.41486084,322.31634644)
\curveto(848.44485769,322.32633852)(848.48485765,322.32633852)(848.53486084,322.31634644)
\curveto(848.58485755,322.30633854)(848.61985751,322.31133854)(848.63986084,322.33134644)
}
}
{
\newrgbcolor{curcolor}{0 0 0}
\pscustom[linestyle=none,fillstyle=solid,fillcolor=curcolor]
{
\newpath
\moveto(771.89145996,303.0080896)
\curveto(772.87145346,303.02807864)(773.69145264,302.8680788)(774.35145996,302.5280896)
\curveto(775.02145131,302.19807947)(775.54145079,301.73807993)(775.91145996,301.1480896)
\curveto(776.01145032,300.98808068)(776.09145024,300.83308084)(776.15145996,300.6830896)
\curveto(776.22145011,300.54308113)(776.28645005,300.3730813)(776.34645996,300.1730896)
\curveto(776.36644997,300.12308155)(776.38644995,300.05308162)(776.40645996,299.9630896)
\curveto(776.42644991,299.88308179)(776.42144991,299.80808186)(776.39145996,299.7380896)
\curveto(776.37144996,299.67808199)(776.33145,299.63808203)(776.27145996,299.6180896)
\curveto(776.22145011,299.60808206)(776.16645017,299.59308208)(776.10645996,299.5730896)
\lineto(775.95645996,299.5730896)
\curveto(775.92645041,299.56308211)(775.88645045,299.55808211)(775.83645996,299.5580896)
\lineto(775.71645996,299.5580896)
\curveto(775.57645076,299.55808211)(775.44645089,299.56308211)(775.32645996,299.5730896)
\curveto(775.21645112,299.59308208)(775.1364512,299.64308203)(775.08645996,299.7230896)
\curveto(775.01645132,299.82308185)(774.96145137,299.93808173)(774.92145996,300.0680896)
\curveto(774.88145145,300.19808147)(774.82645151,300.31808135)(774.75645996,300.4280896)
\curveto(774.62645171,300.64808102)(774.47645186,300.83808083)(774.30645996,300.9980896)
\curveto(774.14645219,301.15808051)(773.95645238,301.30808036)(773.73645996,301.4480896)
\curveto(773.61645272,301.52808014)(773.48145285,301.58808008)(773.33145996,301.6280896)
\curveto(773.19145314,301.66808)(773.04645329,301.70807996)(772.89645996,301.7480896)
\curveto(772.78645355,301.77807989)(772.66145367,301.79807987)(772.52145996,301.8080896)
\curveto(772.38145395,301.82807984)(772.2314541,301.83807983)(772.07145996,301.8380896)
\curveto(771.92145441,301.83807983)(771.77145456,301.82807984)(771.62145996,301.8080896)
\curveto(771.48145485,301.79807987)(771.36145497,301.77807989)(771.26145996,301.7480896)
\curveto(771.16145517,301.72807994)(771.06645527,301.70807996)(770.97645996,301.6880896)
\curveto(770.88645545,301.66808)(770.79645554,301.63808003)(770.70645996,301.5980896)
\curveto(769.86645647,301.24808042)(769.26145707,300.64808102)(768.89145996,299.7980896)
\curveto(768.82145751,299.65808201)(768.76145757,299.50808216)(768.71145996,299.3480896)
\curveto(768.67145766,299.19808247)(768.62645771,299.04308263)(768.57645996,298.8830896)
\curveto(768.55645778,298.82308285)(768.54645779,298.75808291)(768.54645996,298.6880896)
\curveto(768.54645779,298.62808304)(768.5364578,298.5680831)(768.51645996,298.5080896)
\curveto(768.50645783,298.4680832)(768.50145783,298.43308324)(768.50145996,298.4030896)
\curveto(768.50145783,298.3730833)(768.49645784,298.33808333)(768.48645996,298.2980896)
\curveto(768.46645787,298.18808348)(768.45145788,298.0730836)(768.44145996,297.9530896)
\lineto(768.44145996,297.6080896)
\curveto(768.44145789,297.53808413)(768.4364579,297.46308421)(768.42645996,297.3830896)
\curveto(768.42645791,297.31308436)(768.4314579,297.24808442)(768.44145996,297.1880896)
\lineto(768.44145996,297.0380896)
\curveto(768.46145787,296.9680847)(768.46645787,296.89808477)(768.45645996,296.8280896)
\curveto(768.45645788,296.75808491)(768.46645787,296.68808498)(768.48645996,296.6180896)
\curveto(768.50645783,296.55808511)(768.51145782,296.49808517)(768.50145996,296.4380896)
\curveto(768.50145783,296.37808529)(768.51145782,296.32308535)(768.53145996,296.2730896)
\curveto(768.56145777,296.14308553)(768.58645775,296.01308566)(768.60645996,295.8830896)
\curveto(768.6364577,295.76308591)(768.67145766,295.64308603)(768.71145996,295.5230896)
\curveto(768.88145745,295.02308665)(769.10145723,294.59308708)(769.37145996,294.2330896)
\curveto(769.64145669,293.88308779)(769.99645634,293.59308808)(770.43645996,293.3630896)
\curveto(770.57645576,293.29308838)(770.71645562,293.23808843)(770.85645996,293.1980896)
\curveto(771.00645533,293.15808851)(771.16645517,293.11308856)(771.33645996,293.0630896)
\curveto(771.40645493,293.04308863)(771.47145486,293.03308864)(771.53145996,293.0330896)
\curveto(771.59145474,293.04308863)(771.66145467,293.03808863)(771.74145996,293.0180896)
\curveto(771.79145454,293.00808866)(771.88145445,292.99808867)(772.01145996,292.9880896)
\curveto(772.14145419,292.98808868)(772.2364541,292.99808867)(772.29645996,293.0180896)
\lineto(772.40145996,293.0180896)
\curveto(772.44145389,293.02808864)(772.48145385,293.02808864)(772.52145996,293.0180896)
\curveto(772.56145377,293.01808865)(772.60145373,293.02808864)(772.64145996,293.0480896)
\curveto(772.74145359,293.0680886)(772.8364535,293.08308859)(772.92645996,293.0930896)
\curveto(773.02645331,293.11308856)(773.12145321,293.14308853)(773.21145996,293.1830896)
\curveto(773.99145234,293.50308817)(774.54145179,294.02808764)(774.86145996,294.7580896)
\curveto(774.94145139,294.93808673)(775.01645132,295.15308652)(775.08645996,295.4030896)
\curveto(775.10645123,295.49308618)(775.12145121,295.58308609)(775.13145996,295.6730896)
\curveto(775.15145118,295.7730859)(775.18645115,295.86308581)(775.23645996,295.9430896)
\curveto(775.28645105,296.02308565)(775.36645097,296.0680856)(775.47645996,296.0780896)
\curveto(775.58645075,296.08808558)(775.70645063,296.09308558)(775.83645996,296.0930896)
\lineto(775.98645996,296.0930896)
\curveto(776.0364503,296.09308558)(776.08145025,296.08808558)(776.12145996,296.0780896)
\lineto(776.22645996,296.0780896)
\lineto(776.31645996,296.0480896)
\curveto(776.35644998,296.04808562)(776.38644995,296.03808563)(776.40645996,296.0180896)
\curveto(776.47644986,295.97808569)(776.51644982,295.90308577)(776.52645996,295.7930896)
\curveto(776.5364498,295.69308598)(776.52644981,295.59308608)(776.49645996,295.4930896)
\curveto(776.4364499,295.26308641)(776.38144995,295.04308663)(776.33145996,294.8330896)
\curveto(776.28145005,294.62308705)(776.20645013,294.42308725)(776.10645996,294.2330896)
\curveto(776.02645031,294.10308757)(775.95145038,293.97808769)(775.88145996,293.8580896)
\curveto(775.82145051,293.73808793)(775.75145058,293.61808805)(775.67145996,293.4980896)
\curveto(775.49145084,293.23808843)(775.26645107,292.99808867)(774.99645996,292.7780896)
\curveto(774.7364516,292.5680891)(774.45145188,292.39308928)(774.14145996,292.2530896)
\curveto(774.0314523,292.20308947)(773.92145241,292.16308951)(773.81145996,292.1330896)
\curveto(773.71145262,292.10308957)(773.60645273,292.0680896)(773.49645996,292.0280896)
\curveto(773.38645295,291.98808968)(773.27145306,291.96308971)(773.15145996,291.9530896)
\curveto(773.04145329,291.93308974)(772.92645341,291.91308976)(772.80645996,291.8930896)
\curveto(772.75645358,291.8730898)(772.71145362,291.8680898)(772.67145996,291.8780896)
\curveto(772.6314537,291.87808979)(772.59145374,291.8730898)(772.55145996,291.8630896)
\curveto(772.49145384,291.85308982)(772.4314539,291.84808982)(772.37145996,291.8480896)
\curveto(772.31145402,291.84808982)(772.24645409,291.84308983)(772.17645996,291.8330896)
\curveto(772.14645419,291.82308985)(772.07645426,291.82308985)(771.96645996,291.8330896)
\curveto(771.86645447,291.83308984)(771.80145453,291.83808983)(771.77145996,291.8480896)
\curveto(771.72145461,291.85808981)(771.67145466,291.86308981)(771.62145996,291.8630896)
\curveto(771.58145475,291.85308982)(771.5364548,291.85308982)(771.48645996,291.8630896)
\lineto(771.33645996,291.8630896)
\curveto(771.25645508,291.88308979)(771.18145515,291.89808977)(771.11145996,291.9080896)
\curveto(771.04145529,291.90808976)(770.96645537,291.91808975)(770.88645996,291.9380896)
\lineto(770.61645996,291.9980896)
\curveto(770.52645581,292.00808966)(770.44145589,292.02808964)(770.36145996,292.0580896)
\curveto(770.15145618,292.11808955)(769.96145637,292.19308948)(769.79145996,292.2830896)
\curveto(769.16145717,292.55308912)(768.65145768,292.93808873)(768.26145996,293.4380896)
\curveto(767.87145846,293.93808773)(767.56145877,294.52808714)(767.33145996,295.2080896)
\curveto(767.29145904,295.32808634)(767.25645908,295.45308622)(767.22645996,295.5830896)
\curveto(767.20645913,295.71308596)(767.18145915,295.84808582)(767.15145996,295.9880896)
\curveto(767.1314592,296.03808563)(767.12145921,296.08308559)(767.12145996,296.1230896)
\curveto(767.1314592,296.16308551)(767.1314592,296.20808546)(767.12145996,296.2580896)
\curveto(767.10145923,296.34808532)(767.08645925,296.44308523)(767.07645996,296.5430896)
\curveto(767.07645926,296.64308503)(767.06645927,296.73808493)(767.04645996,296.8280896)
\lineto(767.04645996,297.1130896)
\curveto(767.02645931,297.16308451)(767.01645932,297.24808442)(767.01645996,297.3680896)
\curveto(767.01645932,297.48808418)(767.02645931,297.5730841)(767.04645996,297.6230896)
\curveto(767.05645928,297.65308402)(767.05645928,297.68308399)(767.04645996,297.7130896)
\curveto(767.0364593,297.75308392)(767.0364593,297.78308389)(767.04645996,297.8030896)
\lineto(767.04645996,297.9380896)
\curveto(767.05645928,298.01808365)(767.06145927,298.09808357)(767.06145996,298.1780896)
\curveto(767.07145926,298.2680834)(767.08645925,298.35308332)(767.10645996,298.4330896)
\curveto(767.12645921,298.49308318)(767.1364592,298.55308312)(767.13645996,298.6130896)
\curveto(767.1364592,298.68308299)(767.14645919,298.75308292)(767.16645996,298.8230896)
\curveto(767.21645912,298.99308268)(767.25645908,299.15808251)(767.28645996,299.3180896)
\curveto(767.31645902,299.47808219)(767.36145897,299.62808204)(767.42145996,299.7680896)
\lineto(767.57145996,300.1580896)
\curveto(767.6314587,300.29808137)(767.69645864,300.42308125)(767.76645996,300.5330896)
\curveto(767.91645842,300.79308088)(768.06645827,301.02808064)(768.21645996,301.2380896)
\curveto(768.24645809,301.28808038)(768.28145805,301.32808034)(768.32145996,301.3580896)
\curveto(768.37145796,301.39808027)(768.41145792,301.44308023)(768.44145996,301.4930896)
\curveto(768.50145783,301.5730801)(768.56145777,301.64308003)(768.62145996,301.7030896)
\lineto(768.83145996,301.8830896)
\curveto(768.89145744,301.93307974)(768.94645739,301.97807969)(768.99645996,302.0180896)
\curveto(769.05645728,302.0680796)(769.12145721,302.11807955)(769.19145996,302.1680896)
\curveto(769.34145699,302.27807939)(769.49645684,302.3730793)(769.65645996,302.4530896)
\curveto(769.82645651,302.53307914)(770.00145633,302.61307906)(770.18145996,302.6930896)
\curveto(770.29145604,302.74307893)(770.40645593,302.77807889)(770.52645996,302.7980896)
\curveto(770.65645568,302.82807884)(770.78145555,302.86307881)(770.90145996,302.9030896)
\curveto(770.97145536,302.91307876)(771.0364553,302.92307875)(771.09645996,302.9330896)
\lineto(771.27645996,302.9630896)
\curveto(771.35645498,302.9730787)(771.4314549,302.97807869)(771.50145996,302.9780896)
\curveto(771.58145475,302.98807868)(771.66145467,302.99807867)(771.74145996,303.0080896)
\curveto(771.76145457,303.01807865)(771.78645455,303.01807865)(771.81645996,303.0080896)
\curveto(771.84645449,302.99807867)(771.87145446,302.99807867)(771.89145996,303.0080896)
}
}
{
\newrgbcolor{curcolor}{0 0 0}
\pscustom[linestyle=none,fillstyle=solid,fillcolor=curcolor]
{
\newpath
\moveto(785.01130371,292.6430896)
\curveto(785.04129588,292.48308919)(785.0262959,292.34808932)(784.96630371,292.2380896)
\curveto(784.90629602,292.13808953)(784.8262961,292.06308961)(784.72630371,292.0130896)
\curveto(784.67629625,291.99308968)(784.6212963,291.98308969)(784.56130371,291.9830896)
\curveto(784.51129641,291.98308969)(784.45629647,291.9730897)(784.39630371,291.9530896)
\curveto(784.17629675,291.90308977)(783.95629697,291.91808975)(783.73630371,291.9980896)
\curveto(783.5262974,292.0680896)(783.38129754,292.15808951)(783.30130371,292.2680896)
\curveto(783.25129767,292.33808933)(783.20629772,292.41808925)(783.16630371,292.5080896)
\curveto(783.1262978,292.60808906)(783.07629785,292.68808898)(783.01630371,292.7480896)
\curveto(782.99629793,292.7680889)(782.97129795,292.78808888)(782.94130371,292.8080896)
\curveto(782.921298,292.82808884)(782.89129803,292.83308884)(782.85130371,292.8230896)
\curveto(782.74129818,292.79308888)(782.63629829,292.73808893)(782.53630371,292.6580896)
\curveto(782.44629848,292.57808909)(782.35629857,292.50808916)(782.26630371,292.4480896)
\curveto(782.13629879,292.3680893)(781.99629893,292.29308938)(781.84630371,292.2230896)
\curveto(781.69629923,292.16308951)(781.53629939,292.10808956)(781.36630371,292.0580896)
\curveto(781.26629966,292.02808964)(781.15629977,292.00808966)(781.03630371,291.9980896)
\curveto(780.9263,291.98808968)(780.81630011,291.9730897)(780.70630371,291.9530896)
\curveto(780.65630027,291.94308973)(780.61130031,291.93808973)(780.57130371,291.9380896)
\lineto(780.46630371,291.9380896)
\curveto(780.35630057,291.91808975)(780.25130067,291.91808975)(780.15130371,291.9380896)
\lineto(780.01630371,291.9380896)
\curveto(779.96630096,291.94808972)(779.91630101,291.95308972)(779.86630371,291.9530896)
\curveto(779.81630111,291.95308972)(779.77130115,291.96308971)(779.73130371,291.9830896)
\curveto(779.69130123,291.99308968)(779.65630127,291.99808967)(779.62630371,291.9980896)
\curveto(779.60630132,291.98808968)(779.58130134,291.98808968)(779.55130371,291.9980896)
\lineto(779.31130371,292.0580896)
\curveto(779.23130169,292.0680896)(779.15630177,292.08808958)(779.08630371,292.1180896)
\curveto(778.78630214,292.24808942)(778.54130238,292.39308928)(778.35130371,292.5530896)
\curveto(778.17130275,292.72308895)(778.0213029,292.95808871)(777.90130371,293.2580896)
\curveto(777.81130311,293.47808819)(777.76630316,293.74308793)(777.76630371,294.0530896)
\lineto(777.76630371,294.3680896)
\curveto(777.77630315,294.41808725)(777.78130314,294.4680872)(777.78130371,294.5180896)
\lineto(777.81130371,294.6980896)
\lineto(777.93130371,295.0280896)
\curveto(777.97130295,295.13808653)(778.0213029,295.23808643)(778.08130371,295.3280896)
\curveto(778.26130266,295.61808605)(778.50630242,295.83308584)(778.81630371,295.9730896)
\curveto(779.1263018,296.11308556)(779.46630146,296.23808543)(779.83630371,296.3480896)
\curveto(779.97630095,296.38808528)(780.1213008,296.41808525)(780.27130371,296.4380896)
\curveto(780.4213005,296.45808521)(780.57130035,296.48308519)(780.72130371,296.5130896)
\curveto(780.79130013,296.53308514)(780.85630007,296.54308513)(780.91630371,296.5430896)
\curveto(780.98629994,296.54308513)(781.06129986,296.55308512)(781.14130371,296.5730896)
\curveto(781.21129971,296.59308508)(781.28129964,296.60308507)(781.35130371,296.6030896)
\curveto(781.4212995,296.61308506)(781.49629943,296.62808504)(781.57630371,296.6480896)
\curveto(781.8262991,296.70808496)(782.06129886,296.75808491)(782.28130371,296.7980896)
\curveto(782.50129842,296.84808482)(782.67629825,296.96308471)(782.80630371,297.1430896)
\curveto(782.86629806,297.22308445)(782.91629801,297.32308435)(782.95630371,297.4430896)
\curveto(782.99629793,297.5730841)(782.99629793,297.71308396)(782.95630371,297.8630896)
\curveto(782.89629803,298.10308357)(782.80629812,298.29308338)(782.68630371,298.4330896)
\curveto(782.57629835,298.5730831)(782.41629851,298.68308299)(782.20630371,298.7630896)
\curveto(782.08629884,298.81308286)(781.94129898,298.84808282)(781.77130371,298.8680896)
\curveto(781.61129931,298.88808278)(781.44129948,298.89808277)(781.26130371,298.8980896)
\curveto(781.08129984,298.89808277)(780.90630002,298.88808278)(780.73630371,298.8680896)
\curveto(780.56630036,298.84808282)(780.4213005,298.81808285)(780.30130371,298.7780896)
\curveto(780.13130079,298.71808295)(779.96630096,298.63308304)(779.80630371,298.5230896)
\curveto(779.7263012,298.46308321)(779.65130127,298.38308329)(779.58130371,298.2830896)
\curveto(779.5213014,298.19308348)(779.46630146,298.09308358)(779.41630371,297.9830896)
\curveto(779.38630154,297.90308377)(779.35630157,297.81808385)(779.32630371,297.7280896)
\curveto(779.30630162,297.63808403)(779.26130166,297.5680841)(779.19130371,297.5180896)
\curveto(779.15130177,297.48808418)(779.08130184,297.46308421)(778.98130371,297.4430896)
\curveto(778.89130203,297.43308424)(778.79630213,297.42808424)(778.69630371,297.4280896)
\curveto(778.59630233,297.42808424)(778.49630243,297.43308424)(778.39630371,297.4430896)
\curveto(778.30630262,297.46308421)(778.24130268,297.48808418)(778.20130371,297.5180896)
\curveto(778.16130276,297.54808412)(778.13130279,297.59808407)(778.11130371,297.6680896)
\curveto(778.09130283,297.73808393)(778.09130283,297.81308386)(778.11130371,297.8930896)
\curveto(778.14130278,298.02308365)(778.17130275,298.14308353)(778.20130371,298.2530896)
\curveto(778.24130268,298.3730833)(778.28630264,298.48808318)(778.33630371,298.5980896)
\curveto(778.5263024,298.94808272)(778.76630216,299.21808245)(779.05630371,299.4080896)
\curveto(779.34630158,299.60808206)(779.70630122,299.7680819)(780.13630371,299.8880896)
\curveto(780.23630069,299.90808176)(780.33630059,299.92308175)(780.43630371,299.9330896)
\curveto(780.54630038,299.94308173)(780.65630027,299.95808171)(780.76630371,299.9780896)
\curveto(780.80630012,299.98808168)(780.87130005,299.98808168)(780.96130371,299.9780896)
\curveto(781.05129987,299.97808169)(781.10629982,299.98808168)(781.12630371,300.0080896)
\curveto(781.8262991,300.01808165)(782.43629849,299.93808173)(782.95630371,299.7680896)
\curveto(783.47629745,299.59808207)(783.84129708,299.2730824)(784.05130371,298.7930896)
\curveto(784.14129678,298.59308308)(784.19129673,298.35808331)(784.20130371,298.0880896)
\curveto(784.2212967,297.82808384)(784.23129669,297.55308412)(784.23130371,297.2630896)
\lineto(784.23130371,293.9480896)
\curveto(784.23129669,293.80808786)(784.23629669,293.673088)(784.24630371,293.5430896)
\curveto(784.25629667,293.41308826)(784.28629664,293.30808836)(784.33630371,293.2280896)
\curveto(784.38629654,293.15808851)(784.45129647,293.10808856)(784.53130371,293.0780896)
\curveto(784.6212963,293.03808863)(784.70629622,293.00808866)(784.78630371,292.9880896)
\curveto(784.86629606,292.97808869)(784.926296,292.93308874)(784.96630371,292.8530896)
\curveto(784.98629594,292.82308885)(784.99629593,292.79308888)(784.99630371,292.7630896)
\curveto(784.99629593,292.73308894)(785.00129592,292.69308898)(785.01130371,292.6430896)
\moveto(782.86630371,294.3080896)
\curveto(782.926298,294.44808722)(782.95629797,294.60808706)(782.95630371,294.7880896)
\curveto(782.96629796,294.97808669)(782.97129795,295.1730865)(782.97130371,295.3730896)
\curveto(782.97129795,295.48308619)(782.96629796,295.58308609)(782.95630371,295.6730896)
\curveto(782.94629798,295.76308591)(782.90629802,295.83308584)(782.83630371,295.8830896)
\curveto(782.80629812,295.90308577)(782.73629819,295.91308576)(782.62630371,295.9130896)
\curveto(782.60629832,295.89308578)(782.57129835,295.88308579)(782.52130371,295.8830896)
\curveto(782.47129845,295.88308579)(782.4262985,295.8730858)(782.38630371,295.8530896)
\curveto(782.30629862,295.83308584)(782.21629871,295.81308586)(782.11630371,295.7930896)
\lineto(781.81630371,295.7330896)
\curveto(781.78629914,295.73308594)(781.75129917,295.72808594)(781.71130371,295.7180896)
\lineto(781.60630371,295.7180896)
\curveto(781.45629947,295.67808599)(781.29129963,295.65308602)(781.11130371,295.6430896)
\curveto(780.94129998,295.64308603)(780.78130014,295.62308605)(780.63130371,295.5830896)
\curveto(780.55130037,295.56308611)(780.47630045,295.54308613)(780.40630371,295.5230896)
\curveto(780.34630058,295.51308616)(780.27630065,295.49808617)(780.19630371,295.4780896)
\curveto(780.03630089,295.42808624)(779.88630104,295.36308631)(779.74630371,295.2830896)
\curveto(779.60630132,295.21308646)(779.48630144,295.12308655)(779.38630371,295.0130896)
\curveto(779.28630164,294.90308677)(779.21130171,294.7680869)(779.16130371,294.6080896)
\curveto(779.11130181,294.45808721)(779.09130183,294.2730874)(779.10130371,294.0530896)
\curveto(779.10130182,293.95308772)(779.11630181,293.85808781)(779.14630371,293.7680896)
\curveto(779.18630174,293.68808798)(779.23130169,293.61308806)(779.28130371,293.5430896)
\curveto(779.36130156,293.43308824)(779.46630146,293.33808833)(779.59630371,293.2580896)
\curveto(779.7263012,293.18808848)(779.86630106,293.12808854)(780.01630371,293.0780896)
\curveto(780.06630086,293.0680886)(780.11630081,293.06308861)(780.16630371,293.0630896)
\curveto(780.21630071,293.06308861)(780.26630066,293.05808861)(780.31630371,293.0480896)
\curveto(780.38630054,293.02808864)(780.47130045,293.01308866)(780.57130371,293.0030896)
\curveto(780.68130024,293.00308867)(780.77130015,293.01308866)(780.84130371,293.0330896)
\curveto(780.90130002,293.05308862)(780.96129996,293.05808861)(781.02130371,293.0480896)
\curveto(781.08129984,293.04808862)(781.14129978,293.05808861)(781.20130371,293.0780896)
\curveto(781.28129964,293.09808857)(781.35629957,293.11308856)(781.42630371,293.1230896)
\curveto(781.50629942,293.13308854)(781.58129934,293.15308852)(781.65130371,293.1830896)
\curveto(781.94129898,293.30308837)(782.18629874,293.44808822)(782.38630371,293.6180896)
\curveto(782.59629833,293.78808788)(782.75629817,294.01808765)(782.86630371,294.3080896)
}
}
{
\newrgbcolor{curcolor}{0 0 0}
\pscustom[linestyle=none,fillstyle=solid,fillcolor=curcolor]
{
\newpath
\moveto(786.75294434,302.7680896)
\curveto(786.88294272,302.7680789)(787.01794259,302.7680789)(787.15794434,302.7680896)
\curveto(787.3079423,302.7680789)(787.41794219,302.73307894)(787.48794434,302.6630896)
\curveto(787.53794207,302.59307908)(787.56294204,302.49807917)(787.56294434,302.3780896)
\curveto(787.57294203,302.2680794)(787.57794203,302.15307952)(787.57794434,302.0330896)
\lineto(787.57794434,300.6980896)
\lineto(787.57794434,294.6230896)
\lineto(787.57794434,292.9430896)
\lineto(787.57794434,292.5530896)
\curveto(787.57794203,292.41308926)(787.55294205,292.30308937)(787.50294434,292.2230896)
\curveto(787.47294213,292.1730895)(787.42794218,292.14308953)(787.36794434,292.1330896)
\curveto(787.31794229,292.12308955)(787.25294235,292.10808956)(787.17294434,292.0880896)
\lineto(786.96294434,292.0880896)
\lineto(786.64794434,292.0880896)
\curveto(786.54794306,292.09808957)(786.47294313,292.13308954)(786.42294434,292.1930896)
\curveto(786.37294323,292.2730894)(786.34294326,292.3730893)(786.33294434,292.4930896)
\lineto(786.33294434,292.8680896)
\lineto(786.33294434,294.2480896)
\lineto(786.33294434,300.4880896)
\lineto(786.33294434,301.9580896)
\curveto(786.33294327,302.0680796)(786.32794328,302.18307949)(786.31794434,302.3030896)
\curveto(786.31794329,302.43307924)(786.34294326,302.53307914)(786.39294434,302.6030896)
\curveto(786.43294317,302.66307901)(786.5079431,302.71307896)(786.61794434,302.7530896)
\curveto(786.63794297,302.76307891)(786.65794295,302.76307891)(786.67794434,302.7530896)
\curveto(786.7079429,302.75307892)(786.73294287,302.75807891)(786.75294434,302.7680896)
}
}
{
\newrgbcolor{curcolor}{0 0 0}
\pscustom[linestyle=none,fillstyle=solid,fillcolor=curcolor]
{
\newpath
\moveto(789.80778809,301.3130896)
\curveto(789.72778697,301.3730803)(789.68278701,301.47808019)(789.67278809,301.6280896)
\lineto(789.67278809,302.0930896)
\lineto(789.67278809,302.3480896)
\curveto(789.67278702,302.43807923)(789.68778701,302.51307916)(789.71778809,302.5730896)
\curveto(789.75778694,302.65307902)(789.83778686,302.71307896)(789.95778809,302.7530896)
\curveto(789.97778672,302.76307891)(789.9977867,302.76307891)(790.01778809,302.7530896)
\curveto(790.04778665,302.75307892)(790.07278662,302.75807891)(790.09278809,302.7680896)
\curveto(790.26278643,302.7680789)(790.42278627,302.76307891)(790.57278809,302.7530896)
\curveto(790.72278597,302.74307893)(790.82278587,302.68307899)(790.87278809,302.5730896)
\curveto(790.90278579,302.51307916)(790.91778578,302.43807923)(790.91778809,302.3480896)
\lineto(790.91778809,302.0930896)
\curveto(790.91778578,301.91307976)(790.91278578,301.74307993)(790.90278809,301.5830896)
\curveto(790.90278579,301.42308025)(790.83778586,301.31808035)(790.70778809,301.2680896)
\curveto(790.65778604,301.24808042)(790.60278609,301.23808043)(790.54278809,301.2380896)
\lineto(790.37778809,301.2380896)
\lineto(790.06278809,301.2380896)
\curveto(789.96278673,301.23808043)(789.87778682,301.26308041)(789.80778809,301.3130896)
\moveto(790.91778809,292.8080896)
\lineto(790.91778809,292.4930896)
\curveto(790.92778577,292.39308928)(790.90778579,292.31308936)(790.85778809,292.2530896)
\curveto(790.82778587,292.19308948)(790.78278591,292.15308952)(790.72278809,292.1330896)
\curveto(790.66278603,292.12308955)(790.5927861,292.10808956)(790.51278809,292.0880896)
\lineto(790.28778809,292.0880896)
\curveto(790.15778654,292.08808958)(790.04278665,292.09308958)(789.94278809,292.1030896)
\curveto(789.85278684,292.12308955)(789.78278691,292.1730895)(789.73278809,292.2530896)
\curveto(789.692787,292.31308936)(789.67278702,292.38808928)(789.67278809,292.4780896)
\lineto(789.67278809,292.7630896)
\lineto(789.67278809,299.1080896)
\lineto(789.67278809,299.4230896)
\curveto(789.67278702,299.53308214)(789.697787,299.61808205)(789.74778809,299.6780896)
\curveto(789.77778692,299.72808194)(789.81778688,299.75808191)(789.86778809,299.7680896)
\curveto(789.91778678,299.77808189)(789.97278672,299.79308188)(790.03278809,299.8130896)
\curveto(790.05278664,299.81308186)(790.07278662,299.80808186)(790.09278809,299.7980896)
\curveto(790.12278657,299.79808187)(790.14778655,299.80308187)(790.16778809,299.8130896)
\curveto(790.2977864,299.81308186)(790.42778627,299.80808186)(790.55778809,299.7980896)
\curveto(790.697786,299.79808187)(790.7927859,299.75808191)(790.84278809,299.6780896)
\curveto(790.8927858,299.61808205)(790.91778578,299.53808213)(790.91778809,299.4380896)
\lineto(790.91778809,299.1530896)
\lineto(790.91778809,292.8080896)
}
}
{
\newrgbcolor{curcolor}{0 0 0}
\pscustom[linestyle=none,fillstyle=solid,fillcolor=curcolor]
{
\newpath
\moveto(795.30763184,302.8880896)
\curveto(795.4876283,302.89807877)(795.67762811,302.89807877)(795.87763184,302.8880896)
\curveto(796.07762771,302.87807879)(796.21762757,302.81807885)(796.29763184,302.7080896)
\curveto(796.33762745,302.64807902)(796.36262742,302.5730791)(796.37263184,302.4830896)
\curveto(796.3826274,302.40307927)(796.3876274,302.31307936)(796.38763184,302.2130896)
\curveto(796.3876274,302.08307959)(796.36262742,301.97807969)(796.31263184,301.8980896)
\curveto(796.27262751,301.84807982)(796.21262757,301.81307986)(796.13263184,301.7930896)
\curveto(796.06262772,301.78307989)(795.9826278,301.77807989)(795.89263184,301.7780896)
\lineto(795.60763184,301.7780896)
\curveto(795.51762827,301.78807988)(795.43762835,301.78807988)(795.36763184,301.7780896)
\curveto(795.0876287,301.69807997)(794.90262888,301.5680801)(794.81263184,301.3880896)
\curveto(794.73262905,301.21808045)(794.69262909,300.95808071)(794.69263184,300.6080896)
\curveto(794.69262909,300.53808113)(794.6876291,300.46308121)(794.67763184,300.3830896)
\curveto(794.66762912,300.31308136)(794.67262911,300.24808142)(794.69263184,300.1880896)
\curveto(794.72262906,300.03808163)(794.787629,299.93308174)(794.88763184,299.8730896)
\curveto(794.96762882,299.84308183)(795.06762872,299.82808184)(795.18763184,299.8280896)
\lineto(795.54763184,299.8280896)
\lineto(795.77263184,299.8280896)
\curveto(795.80262798,299.80808186)(795.83262795,299.80308187)(795.86263184,299.8130896)
\curveto(795.89262789,299.82308185)(795.92262786,299.81808185)(795.95263184,299.7980896)
\curveto(796.05262773,299.7680819)(796.11762767,299.70808196)(796.14763184,299.6180896)
\curveto(796.17762761,299.53808213)(796.19262759,299.43308224)(796.19263184,299.3030896)
\curveto(796.1826276,299.26308241)(796.17762761,299.22308245)(796.17763184,299.1830896)
\lineto(796.17763184,299.0630896)
\curveto(796.14762764,298.91308276)(796.0826277,298.81308286)(795.98263184,298.7630896)
\curveto(795.85262793,298.71308296)(795.6826281,298.69808297)(795.47263184,298.7180896)
\curveto(795.27262851,298.74808292)(795.10262868,298.74308293)(794.96263184,298.7030896)
\curveto(794.8826289,298.68308299)(794.82262896,298.64308303)(794.78263184,298.5830896)
\curveto(794.74262904,298.53308314)(794.71262907,298.46308321)(794.69263184,298.3730896)
\curveto(794.67262911,298.30308337)(794.66762912,298.22308345)(794.67763184,298.1330896)
\curveto(794.6876291,298.04308363)(794.69262909,297.95808371)(794.69263184,297.8780896)
\lineto(794.69263184,296.8880896)
\lineto(794.69263184,293.7080896)
\lineto(794.69263184,292.9580896)
\lineto(794.69263184,292.7630896)
\curveto(794.70262908,292.69308898)(794.69762909,292.63308904)(794.67763184,292.5830896)
\lineto(794.67763184,292.4630896)
\lineto(794.64763184,292.3430896)
\curveto(794.63762915,292.30308937)(794.62262916,292.2680894)(794.60263184,292.2380896)
\curveto(794.55262923,292.1680895)(794.47762931,292.12808954)(794.37763184,292.1180896)
\curveto(794.27762951,292.10808956)(794.16762962,292.10308957)(794.04763184,292.1030896)
\lineto(793.76263184,292.1030896)
\curveto(793.71263007,292.12308955)(793.66263012,292.13808953)(793.61263184,292.1480896)
\curveto(793.57263021,292.1680895)(793.53763025,292.20308947)(793.50763184,292.2530896)
\curveto(793.4876303,292.28308939)(793.46763032,292.34808932)(793.44763184,292.4480896)
\lineto(793.44763184,292.5530896)
\curveto(793.42763036,292.60308907)(793.41763037,292.65308902)(793.41763184,292.7030896)
\curveto(793.42763036,292.76308891)(793.43263035,292.81808885)(793.43263184,292.8680896)
\lineto(793.43263184,293.4680896)
\lineto(793.43263184,297.5630896)
\lineto(793.43263184,297.9080896)
\curveto(793.44263034,298.02808364)(793.44263034,298.13808353)(793.43263184,298.2380896)
\curveto(793.43263035,298.34808332)(793.41263037,298.44308323)(793.37263184,298.5230896)
\curveto(793.34263044,298.60308307)(793.2876305,298.65808301)(793.20763184,298.6880896)
\curveto(793.14763064,298.71808295)(793.07763071,298.73308294)(792.99763184,298.7330896)
\lineto(792.77263184,298.7330896)
\lineto(792.53263184,298.7330896)
\curveto(792.46263132,298.73308294)(792.39763139,298.74308293)(792.33763184,298.7630896)
\curveto(792.24763154,298.80308287)(792.1826316,298.88808278)(792.14263184,299.0180896)
\curveto(792.13263165,299.0680826)(792.12763166,299.11308256)(792.12763184,299.1530896)
\lineto(792.12763184,299.2880896)
\curveto(792.12763166,299.42808224)(792.14263164,299.53808213)(792.17263184,299.6180896)
\curveto(792.20263158,299.70808196)(792.26763152,299.7680819)(792.36763184,299.7980896)
\curveto(792.43763135,299.82808184)(792.51763127,299.83808183)(792.60763184,299.8280896)
\lineto(792.89263184,299.8280896)
\curveto(792.99263079,299.82808184)(793.07763071,299.83808183)(793.14763184,299.8580896)
\curveto(793.22763056,299.87808179)(793.29263049,299.91808175)(793.34263184,299.9780896)
\curveto(793.41263037,300.05808161)(793.44263034,300.18308149)(793.43263184,300.3530896)
\lineto(793.43263184,300.8330896)
\curveto(793.43263035,301.03308064)(793.44263034,301.21808045)(793.46263184,301.3880896)
\curveto(793.49263029,301.5680801)(793.53763025,301.72807994)(793.59763184,301.8680896)
\curveto(793.70763008,302.10807956)(793.85262993,302.30307937)(794.03263184,302.4530896)
\curveto(794.22262956,302.60307907)(794.44762934,302.71807895)(794.70763184,302.7980896)
\curveto(794.76762902,302.81807885)(794.82762896,302.82807884)(794.88763184,302.8280896)
\curveto(794.95762883,302.83807883)(795.02762876,302.85307882)(795.09763184,302.8730896)
\curveto(795.11762867,302.88307879)(795.15262863,302.88307879)(795.20263184,302.8730896)
\curveto(795.25262853,302.8730788)(795.2876285,302.87807879)(795.30763184,302.8880896)
\moveto(797.57263184,301.3130896)
\curveto(797.64262614,301.26308041)(797.72762606,301.23808043)(797.82763184,301.2380896)
\lineto(798.14263184,301.2380896)
\lineto(798.30763184,301.2380896)
\curveto(798.36762542,301.23808043)(798.42262536,301.24808042)(798.47263184,301.2680896)
\curveto(798.60262518,301.31808035)(798.66762512,301.42308025)(798.66763184,301.5830896)
\curveto(798.67762511,301.74307993)(798.6826251,301.91307976)(798.68263184,302.0930896)
\lineto(798.68263184,302.3480896)
\curveto(798.6826251,302.43807923)(798.66762512,302.51307916)(798.63763184,302.5730896)
\curveto(798.5876252,302.68307899)(798.4876253,302.74307893)(798.33763184,302.7530896)
\curveto(798.1876256,302.76307891)(798.02762576,302.7680789)(797.85763184,302.7680896)
\curveto(797.83762595,302.75807891)(797.81262597,302.75307892)(797.78263184,302.7530896)
\curveto(797.76262602,302.76307891)(797.74262604,302.76307891)(797.72263184,302.7530896)
\curveto(797.60262618,302.71307896)(797.52262626,302.65307902)(797.48263184,302.5730896)
\curveto(797.45262633,302.51307916)(797.43762635,302.43807923)(797.43763184,302.3480896)
\lineto(797.43763184,302.0930896)
\lineto(797.43763184,301.6280896)
\curveto(797.44762634,301.47808019)(797.49262629,301.3730803)(797.57263184,301.3130896)
\moveto(798.68263184,299.1530896)
\lineto(798.68263184,299.4380896)
\curveto(798.6826251,299.53808213)(798.65762513,299.61808205)(798.60763184,299.6780896)
\curveto(798.55762523,299.75808191)(798.46262532,299.79808187)(798.32263184,299.7980896)
\curveto(798.19262559,299.80808186)(798.06262572,299.81308186)(797.93263184,299.8130896)
\curveto(797.91262587,299.80308187)(797.8876259,299.79808187)(797.85763184,299.7980896)
\curveto(797.83762595,299.80808186)(797.81762597,299.81308186)(797.79763184,299.8130896)
\curveto(797.73762605,299.79308188)(797.6826261,299.77808189)(797.63263184,299.7680896)
\curveto(797.5826262,299.75808191)(797.54262624,299.72808194)(797.51263184,299.6780896)
\curveto(797.46262632,299.61808205)(797.43762635,299.53308214)(797.43763184,299.4230896)
\lineto(797.43763184,299.1080896)
\lineto(797.43763184,292.7630896)
\lineto(797.43763184,292.4780896)
\curveto(797.43762635,292.38808928)(797.45762633,292.31308936)(797.49763184,292.2530896)
\curveto(797.54762624,292.1730895)(797.61762617,292.12308955)(797.70763184,292.1030896)
\curveto(797.80762598,292.09308958)(797.92262586,292.08808958)(798.05263184,292.0880896)
\lineto(798.27763184,292.0880896)
\curveto(798.35762543,292.10808956)(798.42762536,292.12308955)(798.48763184,292.1330896)
\curveto(798.54762524,292.15308952)(798.59262519,292.19308948)(798.62263184,292.2530896)
\curveto(798.67262511,292.31308936)(798.69262509,292.39308928)(798.68263184,292.4930896)
\lineto(798.68263184,292.8080896)
\lineto(798.68263184,299.1530896)
}
}
{
\newrgbcolor{curcolor}{0 0 0}
\pscustom[linestyle=none,fillstyle=solid,fillcolor=curcolor]
{
\newpath
\moveto(803.76130371,299.9930896)
\curveto(804.50129892,300.00308167)(805.11629831,299.89308178)(805.60630371,299.6630896)
\curveto(806.10629732,299.44308223)(806.50129692,299.10808256)(806.79130371,298.6580896)
\curveto(806.9212965,298.45808321)(807.03129639,298.21308346)(807.12130371,297.9230896)
\curveto(807.14129628,297.8730838)(807.15629627,297.80808386)(807.16630371,297.7280896)
\curveto(807.17629625,297.64808402)(807.17129625,297.57808409)(807.15130371,297.5180896)
\curveto(807.1212963,297.4680842)(807.07129635,297.42308425)(807.00130371,297.3830896)
\curveto(806.97129645,297.36308431)(806.94129648,297.35308432)(806.91130371,297.3530896)
\curveto(806.88129654,297.36308431)(806.84629658,297.36308431)(806.80630371,297.3530896)
\curveto(806.76629666,297.34308433)(806.7262967,297.33808433)(806.68630371,297.3380896)
\curveto(806.64629678,297.34808432)(806.60629682,297.35308432)(806.56630371,297.3530896)
\lineto(806.25130371,297.3530896)
\curveto(806.15129727,297.36308431)(806.06629736,297.39308428)(805.99630371,297.4430896)
\curveto(805.91629751,297.50308417)(805.86129756,297.58808408)(805.83130371,297.6980896)
\curveto(805.80129762,297.80808386)(805.76129766,297.90308377)(805.71130371,297.9830896)
\curveto(805.56129786,298.24308343)(805.36629806,298.44808322)(805.12630371,298.5980896)
\curveto(805.04629838,298.64808302)(804.96129846,298.68808298)(804.87130371,298.7180896)
\curveto(804.78129864,298.75808291)(804.68629874,298.79308288)(804.58630371,298.8230896)
\curveto(804.44629898,298.86308281)(804.26129916,298.88308279)(804.03130371,298.8830896)
\curveto(803.80129962,298.89308278)(803.61129981,298.8730828)(803.46130371,298.8230896)
\curveto(803.39130003,298.80308287)(803.3263001,298.78808288)(803.26630371,298.7780896)
\curveto(803.20630022,298.7680829)(803.14130028,298.75308292)(803.07130371,298.7330896)
\curveto(802.81130061,298.62308305)(802.58130084,298.4730832)(802.38130371,298.2830896)
\curveto(802.18130124,298.09308358)(802.0263014,297.8680838)(801.91630371,297.6080896)
\curveto(801.87630155,297.51808415)(801.84130158,297.42308425)(801.81130371,297.3230896)
\curveto(801.78130164,297.23308444)(801.75130167,297.13308454)(801.72130371,297.0230896)
\lineto(801.63130371,296.6180896)
\curveto(801.6213018,296.5680851)(801.61630181,296.51308516)(801.61630371,296.4530896)
\curveto(801.6263018,296.39308528)(801.6213018,296.33808533)(801.60130371,296.2880896)
\lineto(801.60130371,296.1680896)
\curveto(801.59130183,296.12808554)(801.58130184,296.06308561)(801.57130371,295.9730896)
\curveto(801.57130185,295.88308579)(801.58130184,295.81808585)(801.60130371,295.7780896)
\curveto(801.61130181,295.72808594)(801.61130181,295.67808599)(801.60130371,295.6280896)
\curveto(801.59130183,295.57808609)(801.59130183,295.52808614)(801.60130371,295.4780896)
\curveto(801.61130181,295.43808623)(801.61630181,295.3680863)(801.61630371,295.2680896)
\curveto(801.63630179,295.18808648)(801.65130177,295.10308657)(801.66130371,295.0130896)
\curveto(801.68130174,294.92308675)(801.70130172,294.83808683)(801.72130371,294.7580896)
\curveto(801.83130159,294.43808723)(801.95630147,294.15808751)(802.09630371,293.9180896)
\curveto(802.24630118,293.68808798)(802.45130097,293.48808818)(802.71130371,293.3180896)
\curveto(802.80130062,293.2680884)(802.89130053,293.22308845)(802.98130371,293.1830896)
\curveto(803.08130034,293.14308853)(803.18630024,293.10308857)(803.29630371,293.0630896)
\curveto(803.34630008,293.05308862)(803.38630004,293.04808862)(803.41630371,293.0480896)
\curveto(803.44629998,293.04808862)(803.48629994,293.04308863)(803.53630371,293.0330896)
\curveto(803.56629986,293.02308865)(803.61629981,293.01808865)(803.68630371,293.0180896)
\lineto(803.85130371,293.0180896)
\curveto(803.85129957,293.00808866)(803.87129955,293.00308867)(803.91130371,293.0030896)
\curveto(803.93129949,293.01308866)(803.95629947,293.01308866)(803.98630371,293.0030896)
\curveto(804.01629941,293.00308867)(804.04629938,293.00808866)(804.07630371,293.0180896)
\curveto(804.14629928,293.03808863)(804.21129921,293.04308863)(804.27130371,293.0330896)
\curveto(804.34129908,293.03308864)(804.41129901,293.04308863)(804.48130371,293.0630896)
\curveto(804.74129868,293.14308853)(804.96629846,293.24308843)(805.15630371,293.3630896)
\curveto(805.34629808,293.49308818)(805.50629792,293.65808801)(805.63630371,293.8580896)
\curveto(805.68629774,293.93808773)(805.73129769,294.02308765)(805.77130371,294.1130896)
\lineto(805.89130371,294.3830896)
\curveto(805.91129751,294.46308721)(805.93129749,294.53808713)(805.95130371,294.6080896)
\curveto(805.98129744,294.68808698)(806.03129739,294.75308692)(806.10130371,294.8030896)
\curveto(806.13129729,294.83308684)(806.19129723,294.85308682)(806.28130371,294.8630896)
\curveto(806.37129705,294.88308679)(806.46629696,294.89308678)(806.56630371,294.8930896)
\curveto(806.67629675,294.90308677)(806.77629665,294.90308677)(806.86630371,294.8930896)
\curveto(806.96629646,294.88308679)(807.03629639,294.86308681)(807.07630371,294.8330896)
\curveto(807.13629629,294.79308688)(807.17129625,294.73308694)(807.18130371,294.6530896)
\curveto(807.20129622,294.5730871)(807.20129622,294.48808718)(807.18130371,294.3980896)
\curveto(807.13129629,294.24808742)(807.08129634,294.10308757)(807.03130371,293.9630896)
\curveto(806.99129643,293.83308784)(806.93629649,293.70308797)(806.86630371,293.5730896)
\curveto(806.71629671,293.2730884)(806.5262969,293.00808866)(806.29630371,292.7780896)
\curveto(806.07629735,292.54808912)(805.80629762,292.36308931)(805.48630371,292.2230896)
\curveto(805.40629802,292.18308949)(805.3212981,292.14808952)(805.23130371,292.1180896)
\curveto(805.14129828,292.09808957)(805.04629838,292.0730896)(804.94630371,292.0430896)
\curveto(804.83629859,292.00308967)(804.7262987,291.98308969)(804.61630371,291.9830896)
\curveto(804.50629892,291.9730897)(804.39629903,291.95808971)(804.28630371,291.9380896)
\curveto(804.24629918,291.91808975)(804.20629922,291.91308976)(804.16630371,291.9230896)
\curveto(804.1262993,291.93308974)(804.08629934,291.93308974)(804.04630371,291.9230896)
\lineto(803.91130371,291.9230896)
\lineto(803.67130371,291.9230896)
\curveto(803.60129982,291.91308976)(803.53629989,291.91808975)(803.47630371,291.9380896)
\lineto(803.40130371,291.9380896)
\lineto(803.04130371,291.9830896)
\curveto(802.91130051,292.02308965)(802.78630064,292.05808961)(802.66630371,292.0880896)
\curveto(802.54630088,292.11808955)(802.43130099,292.15808951)(802.32130371,292.2080896)
\curveto(801.96130146,292.3680893)(801.66130176,292.55808911)(801.42130371,292.7780896)
\curveto(801.19130223,292.99808867)(800.97630245,293.2680884)(800.77630371,293.5880896)
\curveto(800.7263027,293.668088)(800.68130274,293.75808791)(800.64130371,293.8580896)
\lineto(800.52130371,294.1580896)
\curveto(800.47130295,294.2680874)(800.43630299,294.38308729)(800.41630371,294.5030896)
\curveto(800.39630303,294.62308705)(800.37130305,294.74308693)(800.34130371,294.8630896)
\curveto(800.33130309,294.90308677)(800.3263031,294.94308673)(800.32630371,294.9830896)
\curveto(800.3263031,295.02308665)(800.3213031,295.06308661)(800.31130371,295.1030896)
\curveto(800.29130313,295.16308651)(800.28130314,295.22808644)(800.28130371,295.2980896)
\curveto(800.29130313,295.3680863)(800.28630314,295.43308624)(800.26630371,295.4930896)
\lineto(800.26630371,295.6430896)
\curveto(800.25630317,295.69308598)(800.25130317,295.76308591)(800.25130371,295.8530896)
\curveto(800.25130317,295.94308573)(800.25630317,296.01308566)(800.26630371,296.0630896)
\curveto(800.27630315,296.11308556)(800.27630315,296.15808551)(800.26630371,296.1980896)
\curveto(800.26630316,296.23808543)(800.27130315,296.27808539)(800.28130371,296.3180896)
\curveto(800.30130312,296.38808528)(800.30630312,296.45808521)(800.29630371,296.5280896)
\curveto(800.29630313,296.59808507)(800.30630312,296.66308501)(800.32630371,296.7230896)
\curveto(800.36630306,296.89308478)(800.40130302,297.06308461)(800.43130371,297.2330896)
\curveto(800.46130296,297.40308427)(800.50630292,297.56308411)(800.56630371,297.7130896)
\curveto(800.77630265,298.23308344)(801.03130239,298.65308302)(801.33130371,298.9730896)
\curveto(801.63130179,299.29308238)(802.04130138,299.55808211)(802.56130371,299.7680896)
\curveto(802.67130075,299.81808185)(802.79130063,299.85308182)(802.92130371,299.8730896)
\curveto(803.05130037,299.89308178)(803.18630024,299.91808175)(803.32630371,299.9480896)
\curveto(803.39630003,299.95808171)(803.46629996,299.96308171)(803.53630371,299.9630896)
\curveto(803.60629982,299.9730817)(803.68129974,299.98308169)(803.76130371,299.9930896)
}
}
{
\newrgbcolor{curcolor}{0 0 0}
\pscustom[linestyle=none,fillstyle=solid,fillcolor=curcolor]
{
\newpath
\moveto(815.56794434,292.6430896)
\curveto(815.59793651,292.48308919)(815.58293652,292.34808932)(815.52294434,292.2380896)
\curveto(815.46293664,292.13808953)(815.38293672,292.06308961)(815.28294434,292.0130896)
\curveto(815.23293687,291.99308968)(815.17793693,291.98308969)(815.11794434,291.9830896)
\curveto(815.06793704,291.98308969)(815.01293709,291.9730897)(814.95294434,291.9530896)
\curveto(814.73293737,291.90308977)(814.51293759,291.91808975)(814.29294434,291.9980896)
\curveto(814.08293802,292.0680896)(813.93793817,292.15808951)(813.85794434,292.2680896)
\curveto(813.8079383,292.33808933)(813.76293834,292.41808925)(813.72294434,292.5080896)
\curveto(813.68293842,292.60808906)(813.63293847,292.68808898)(813.57294434,292.7480896)
\curveto(813.55293855,292.7680889)(813.52793858,292.78808888)(813.49794434,292.8080896)
\curveto(813.47793863,292.82808884)(813.44793866,292.83308884)(813.40794434,292.8230896)
\curveto(813.29793881,292.79308888)(813.19293891,292.73808893)(813.09294434,292.6580896)
\curveto(813.0029391,292.57808909)(812.91293919,292.50808916)(812.82294434,292.4480896)
\curveto(812.69293941,292.3680893)(812.55293955,292.29308938)(812.40294434,292.2230896)
\curveto(812.25293985,292.16308951)(812.09294001,292.10808956)(811.92294434,292.0580896)
\curveto(811.82294028,292.02808964)(811.71294039,292.00808966)(811.59294434,291.9980896)
\curveto(811.48294062,291.98808968)(811.37294073,291.9730897)(811.26294434,291.9530896)
\curveto(811.21294089,291.94308973)(811.16794094,291.93808973)(811.12794434,291.9380896)
\lineto(811.02294434,291.9380896)
\curveto(810.91294119,291.91808975)(810.8079413,291.91808975)(810.70794434,291.9380896)
\lineto(810.57294434,291.9380896)
\curveto(810.52294158,291.94808972)(810.47294163,291.95308972)(810.42294434,291.9530896)
\curveto(810.37294173,291.95308972)(810.32794178,291.96308971)(810.28794434,291.9830896)
\curveto(810.24794186,291.99308968)(810.21294189,291.99808967)(810.18294434,291.9980896)
\curveto(810.16294194,291.98808968)(810.13794197,291.98808968)(810.10794434,291.9980896)
\lineto(809.86794434,292.0580896)
\curveto(809.78794232,292.0680896)(809.71294239,292.08808958)(809.64294434,292.1180896)
\curveto(809.34294276,292.24808942)(809.09794301,292.39308928)(808.90794434,292.5530896)
\curveto(808.72794338,292.72308895)(808.57794353,292.95808871)(808.45794434,293.2580896)
\curveto(808.36794374,293.47808819)(808.32294378,293.74308793)(808.32294434,294.0530896)
\lineto(808.32294434,294.3680896)
\curveto(808.33294377,294.41808725)(808.33794377,294.4680872)(808.33794434,294.5180896)
\lineto(808.36794434,294.6980896)
\lineto(808.48794434,295.0280896)
\curveto(808.52794358,295.13808653)(808.57794353,295.23808643)(808.63794434,295.3280896)
\curveto(808.81794329,295.61808605)(809.06294304,295.83308584)(809.37294434,295.9730896)
\curveto(809.68294242,296.11308556)(810.02294208,296.23808543)(810.39294434,296.3480896)
\curveto(810.53294157,296.38808528)(810.67794143,296.41808525)(810.82794434,296.4380896)
\curveto(810.97794113,296.45808521)(811.12794098,296.48308519)(811.27794434,296.5130896)
\curveto(811.34794076,296.53308514)(811.41294069,296.54308513)(811.47294434,296.5430896)
\curveto(811.54294056,296.54308513)(811.61794049,296.55308512)(811.69794434,296.5730896)
\curveto(811.76794034,296.59308508)(811.83794027,296.60308507)(811.90794434,296.6030896)
\curveto(811.97794013,296.61308506)(812.05294005,296.62808504)(812.13294434,296.6480896)
\curveto(812.38293972,296.70808496)(812.61793949,296.75808491)(812.83794434,296.7980896)
\curveto(813.05793905,296.84808482)(813.23293887,296.96308471)(813.36294434,297.1430896)
\curveto(813.42293868,297.22308445)(813.47293863,297.32308435)(813.51294434,297.4430896)
\curveto(813.55293855,297.5730841)(813.55293855,297.71308396)(813.51294434,297.8630896)
\curveto(813.45293865,298.10308357)(813.36293874,298.29308338)(813.24294434,298.4330896)
\curveto(813.13293897,298.5730831)(812.97293913,298.68308299)(812.76294434,298.7630896)
\curveto(812.64293946,298.81308286)(812.49793961,298.84808282)(812.32794434,298.8680896)
\curveto(812.16793994,298.88808278)(811.99794011,298.89808277)(811.81794434,298.8980896)
\curveto(811.63794047,298.89808277)(811.46294064,298.88808278)(811.29294434,298.8680896)
\curveto(811.12294098,298.84808282)(810.97794113,298.81808285)(810.85794434,298.7780896)
\curveto(810.68794142,298.71808295)(810.52294158,298.63308304)(810.36294434,298.5230896)
\curveto(810.28294182,298.46308321)(810.2079419,298.38308329)(810.13794434,298.2830896)
\curveto(810.07794203,298.19308348)(810.02294208,298.09308358)(809.97294434,297.9830896)
\curveto(809.94294216,297.90308377)(809.91294219,297.81808385)(809.88294434,297.7280896)
\curveto(809.86294224,297.63808403)(809.81794229,297.5680841)(809.74794434,297.5180896)
\curveto(809.7079424,297.48808418)(809.63794247,297.46308421)(809.53794434,297.4430896)
\curveto(809.44794266,297.43308424)(809.35294275,297.42808424)(809.25294434,297.4280896)
\curveto(809.15294295,297.42808424)(809.05294305,297.43308424)(808.95294434,297.4430896)
\curveto(808.86294324,297.46308421)(808.79794331,297.48808418)(808.75794434,297.5180896)
\curveto(808.71794339,297.54808412)(808.68794342,297.59808407)(808.66794434,297.6680896)
\curveto(808.64794346,297.73808393)(808.64794346,297.81308386)(808.66794434,297.8930896)
\curveto(808.69794341,298.02308365)(808.72794338,298.14308353)(808.75794434,298.2530896)
\curveto(808.79794331,298.3730833)(808.84294326,298.48808318)(808.89294434,298.5980896)
\curveto(809.08294302,298.94808272)(809.32294278,299.21808245)(809.61294434,299.4080896)
\curveto(809.9029422,299.60808206)(810.26294184,299.7680819)(810.69294434,299.8880896)
\curveto(810.79294131,299.90808176)(810.89294121,299.92308175)(810.99294434,299.9330896)
\curveto(811.102941,299.94308173)(811.21294089,299.95808171)(811.32294434,299.9780896)
\curveto(811.36294074,299.98808168)(811.42794068,299.98808168)(811.51794434,299.9780896)
\curveto(811.6079405,299.97808169)(811.66294044,299.98808168)(811.68294434,300.0080896)
\curveto(812.38293972,300.01808165)(812.99293911,299.93808173)(813.51294434,299.7680896)
\curveto(814.03293807,299.59808207)(814.39793771,299.2730824)(814.60794434,298.7930896)
\curveto(814.69793741,298.59308308)(814.74793736,298.35808331)(814.75794434,298.0880896)
\curveto(814.77793733,297.82808384)(814.78793732,297.55308412)(814.78794434,297.2630896)
\lineto(814.78794434,293.9480896)
\curveto(814.78793732,293.80808786)(814.79293731,293.673088)(814.80294434,293.5430896)
\curveto(814.81293729,293.41308826)(814.84293726,293.30808836)(814.89294434,293.2280896)
\curveto(814.94293716,293.15808851)(815.0079371,293.10808856)(815.08794434,293.0780896)
\curveto(815.17793693,293.03808863)(815.26293684,293.00808866)(815.34294434,292.9880896)
\curveto(815.42293668,292.97808869)(815.48293662,292.93308874)(815.52294434,292.8530896)
\curveto(815.54293656,292.82308885)(815.55293655,292.79308888)(815.55294434,292.7630896)
\curveto(815.55293655,292.73308894)(815.55793655,292.69308898)(815.56794434,292.6430896)
\moveto(813.42294434,294.3080896)
\curveto(813.48293862,294.44808722)(813.51293859,294.60808706)(813.51294434,294.7880896)
\curveto(813.52293858,294.97808669)(813.52793858,295.1730865)(813.52794434,295.3730896)
\curveto(813.52793858,295.48308619)(813.52293858,295.58308609)(813.51294434,295.6730896)
\curveto(813.5029386,295.76308591)(813.46293864,295.83308584)(813.39294434,295.8830896)
\curveto(813.36293874,295.90308577)(813.29293881,295.91308576)(813.18294434,295.9130896)
\curveto(813.16293894,295.89308578)(813.12793898,295.88308579)(813.07794434,295.8830896)
\curveto(813.02793908,295.88308579)(812.98293912,295.8730858)(812.94294434,295.8530896)
\curveto(812.86293924,295.83308584)(812.77293933,295.81308586)(812.67294434,295.7930896)
\lineto(812.37294434,295.7330896)
\curveto(812.34293976,295.73308594)(812.3079398,295.72808594)(812.26794434,295.7180896)
\lineto(812.16294434,295.7180896)
\curveto(812.01294009,295.67808599)(811.84794026,295.65308602)(811.66794434,295.6430896)
\curveto(811.49794061,295.64308603)(811.33794077,295.62308605)(811.18794434,295.5830896)
\curveto(811.107941,295.56308611)(811.03294107,295.54308613)(810.96294434,295.5230896)
\curveto(810.9029412,295.51308616)(810.83294127,295.49808617)(810.75294434,295.4780896)
\curveto(810.59294151,295.42808624)(810.44294166,295.36308631)(810.30294434,295.2830896)
\curveto(810.16294194,295.21308646)(810.04294206,295.12308655)(809.94294434,295.0130896)
\curveto(809.84294226,294.90308677)(809.76794234,294.7680869)(809.71794434,294.6080896)
\curveto(809.66794244,294.45808721)(809.64794246,294.2730874)(809.65794434,294.0530896)
\curveto(809.65794245,293.95308772)(809.67294243,293.85808781)(809.70294434,293.7680896)
\curveto(809.74294236,293.68808798)(809.78794232,293.61308806)(809.83794434,293.5430896)
\curveto(809.91794219,293.43308824)(810.02294208,293.33808833)(810.15294434,293.2580896)
\curveto(810.28294182,293.18808848)(810.42294168,293.12808854)(810.57294434,293.0780896)
\curveto(810.62294148,293.0680886)(810.67294143,293.06308861)(810.72294434,293.0630896)
\curveto(810.77294133,293.06308861)(810.82294128,293.05808861)(810.87294434,293.0480896)
\curveto(810.94294116,293.02808864)(811.02794108,293.01308866)(811.12794434,293.0030896)
\curveto(811.23794087,293.00308867)(811.32794078,293.01308866)(811.39794434,293.0330896)
\curveto(811.45794065,293.05308862)(811.51794059,293.05808861)(811.57794434,293.0480896)
\curveto(811.63794047,293.04808862)(811.69794041,293.05808861)(811.75794434,293.0780896)
\curveto(811.83794027,293.09808857)(811.91294019,293.11308856)(811.98294434,293.1230896)
\curveto(812.06294004,293.13308854)(812.13793997,293.15308852)(812.20794434,293.1830896)
\curveto(812.49793961,293.30308837)(812.74293936,293.44808822)(812.94294434,293.6180896)
\curveto(813.15293895,293.78808788)(813.31293879,294.01808765)(813.42294434,294.3080896)
}
}
{
\newrgbcolor{curcolor}{0 0 0}
\pscustom[linestyle=none,fillstyle=solid,fillcolor=curcolor]
{
\newpath
\moveto(823.69958496,292.8980896)
\lineto(823.69958496,292.5080896)
\curveto(823.69957709,292.38808928)(823.67457711,292.28808938)(823.62458496,292.2080896)
\curveto(823.57457721,292.13808953)(823.4895773,292.09808957)(823.36958496,292.0880896)
\lineto(823.02458496,292.0880896)
\curveto(822.96457782,292.08808958)(822.90457788,292.08308959)(822.84458496,292.0730896)
\curveto(822.79457799,292.0730896)(822.74957804,292.08308959)(822.70958496,292.1030896)
\curveto(822.61957817,292.12308955)(822.55957823,292.16308951)(822.52958496,292.2230896)
\curveto(822.4895783,292.2730894)(822.46457832,292.33308934)(822.45458496,292.4030896)
\curveto(822.45457833,292.4730892)(822.43957835,292.54308913)(822.40958496,292.6130896)
\curveto(822.39957839,292.63308904)(822.3845784,292.64808902)(822.36458496,292.6580896)
\curveto(822.35457843,292.67808899)(822.33957845,292.69808897)(822.31958496,292.7180896)
\curveto(822.21957857,292.72808894)(822.13957865,292.70808896)(822.07958496,292.6580896)
\curveto(822.02957876,292.60808906)(821.97457881,292.55808911)(821.91458496,292.5080896)
\curveto(821.71457907,292.35808931)(821.51457927,292.24308943)(821.31458496,292.1630896)
\curveto(821.13457965,292.08308959)(820.92457986,292.02308965)(820.68458496,291.9830896)
\curveto(820.45458033,291.94308973)(820.21458057,291.92308975)(819.96458496,291.9230896)
\curveto(819.72458106,291.91308976)(819.4845813,291.92808974)(819.24458496,291.9680896)
\curveto(819.00458178,291.99808967)(818.79458199,292.05308962)(818.61458496,292.1330896)
\curveto(818.09458269,292.35308932)(817.67458311,292.64808902)(817.35458496,293.0180896)
\curveto(817.03458375,293.39808827)(816.784584,293.8680878)(816.60458496,294.4280896)
\curveto(816.56458422,294.51808715)(816.53458425,294.60808706)(816.51458496,294.6980896)
\curveto(816.50458428,294.79808687)(816.4845843,294.89808677)(816.45458496,294.9980896)
\curveto(816.44458434,295.04808662)(816.43958435,295.09808657)(816.43958496,295.1480896)
\curveto(816.43958435,295.19808647)(816.43458435,295.24808642)(816.42458496,295.2980896)
\curveto(816.40458438,295.34808632)(816.39458439,295.39808627)(816.39458496,295.4480896)
\curveto(816.40458438,295.50808616)(816.40458438,295.56308611)(816.39458496,295.6130896)
\lineto(816.39458496,295.7630896)
\curveto(816.37458441,295.81308586)(816.36458442,295.87808579)(816.36458496,295.9580896)
\curveto(816.36458442,296.03808563)(816.37458441,296.10308557)(816.39458496,296.1530896)
\lineto(816.39458496,296.3180896)
\curveto(816.41458437,296.38808528)(816.41958437,296.45808521)(816.40958496,296.5280896)
\curveto(816.40958438,296.60808506)(816.41958437,296.68308499)(816.43958496,296.7530896)
\curveto(816.44958434,296.80308487)(816.45458433,296.84808482)(816.45458496,296.8880896)
\curveto(816.45458433,296.92808474)(816.45958433,296.9730847)(816.46958496,297.0230896)
\curveto(816.49958429,297.12308455)(816.52458426,297.21808445)(816.54458496,297.3080896)
\curveto(816.56458422,297.40808426)(816.5895842,297.50308417)(816.61958496,297.5930896)
\curveto(816.74958404,297.9730837)(816.91458387,298.31308336)(817.11458496,298.6130896)
\curveto(817.32458346,298.92308275)(817.57458321,299.17808249)(817.86458496,299.3780896)
\curveto(818.03458275,299.49808217)(818.20958258,299.59808207)(818.38958496,299.6780896)
\curveto(818.57958221,299.75808191)(818.784582,299.82808184)(819.00458496,299.8880896)
\curveto(819.07458171,299.89808177)(819.13958165,299.90808176)(819.19958496,299.9180896)
\curveto(819.26958152,299.92808174)(819.33958145,299.94308173)(819.40958496,299.9630896)
\lineto(819.55958496,299.9630896)
\curveto(819.63958115,299.98308169)(819.75458103,299.99308168)(819.90458496,299.9930896)
\curveto(820.06458072,299.99308168)(820.1845806,299.98308169)(820.26458496,299.9630896)
\curveto(820.30458048,299.95308172)(820.35958043,299.94808172)(820.42958496,299.9480896)
\curveto(820.53958025,299.91808175)(820.64958014,299.89308178)(820.75958496,299.8730896)
\curveto(820.86957992,299.86308181)(820.97457981,299.83308184)(821.07458496,299.7830896)
\curveto(821.22457956,299.72308195)(821.36457942,299.65808201)(821.49458496,299.5880896)
\curveto(821.63457915,299.51808215)(821.76457902,299.43808223)(821.88458496,299.3480896)
\curveto(821.94457884,299.29808237)(822.00457878,299.24308243)(822.06458496,299.1830896)
\curveto(822.13457865,299.13308254)(822.22457856,299.11808255)(822.33458496,299.1380896)
\curveto(822.35457843,299.1680825)(822.36957842,299.19308248)(822.37958496,299.2130896)
\curveto(822.39957839,299.23308244)(822.41457837,299.26308241)(822.42458496,299.3030896)
\curveto(822.45457833,299.39308228)(822.46457832,299.50808216)(822.45458496,299.6480896)
\lineto(822.45458496,300.0230896)
\lineto(822.45458496,301.7480896)
\lineto(822.45458496,302.2130896)
\curveto(822.45457833,302.39307928)(822.47957831,302.52307915)(822.52958496,302.6030896)
\curveto(822.56957822,302.673079)(822.62957816,302.71807895)(822.70958496,302.7380896)
\curveto(822.72957806,302.73807893)(822.75457803,302.73807893)(822.78458496,302.7380896)
\curveto(822.81457797,302.74807892)(822.83957795,302.75307892)(822.85958496,302.7530896)
\curveto(822.99957779,302.76307891)(823.14457764,302.76307891)(823.29458496,302.7530896)
\curveto(823.45457733,302.75307892)(823.56457722,302.71307896)(823.62458496,302.6330896)
\curveto(823.67457711,302.55307912)(823.69957709,302.45307922)(823.69958496,302.3330896)
\lineto(823.69958496,301.9580896)
\lineto(823.69958496,292.8980896)
\moveto(822.48458496,295.7330896)
\curveto(822.50457828,295.78308589)(822.51457827,295.84808582)(822.51458496,295.9280896)
\curveto(822.51457827,296.01808565)(822.50457828,296.08808558)(822.48458496,296.1380896)
\lineto(822.48458496,296.3630896)
\curveto(822.46457832,296.45308522)(822.44957834,296.54308513)(822.43958496,296.6330896)
\curveto(822.42957836,296.73308494)(822.40957838,296.82308485)(822.37958496,296.9030896)
\curveto(822.35957843,296.98308469)(822.33957845,297.05808461)(822.31958496,297.1280896)
\curveto(822.30957848,297.19808447)(822.2895785,297.2680844)(822.25958496,297.3380896)
\curveto(822.13957865,297.63808403)(821.9845788,297.90308377)(821.79458496,298.1330896)
\curveto(821.60457918,298.36308331)(821.36457942,298.54308313)(821.07458496,298.6730896)
\curveto(820.97457981,298.72308295)(820.86957992,298.75808291)(820.75958496,298.7780896)
\curveto(820.65958013,298.80808286)(820.54958024,298.83308284)(820.42958496,298.8530896)
\curveto(820.34958044,298.8730828)(820.25958053,298.88308279)(820.15958496,298.8830896)
\lineto(819.88958496,298.8830896)
\curveto(819.83958095,298.8730828)(819.79458099,298.86308281)(819.75458496,298.8530896)
\lineto(819.61958496,298.8530896)
\curveto(819.53958125,298.83308284)(819.45458133,298.81308286)(819.36458496,298.7930896)
\curveto(819.2845815,298.7730829)(819.20458158,298.74808292)(819.12458496,298.7180896)
\curveto(818.80458198,298.57808309)(818.54458224,298.3730833)(818.34458496,298.1030896)
\curveto(818.15458263,297.84308383)(817.99958279,297.53808413)(817.87958496,297.1880896)
\curveto(817.83958295,297.07808459)(817.80958298,296.96308471)(817.78958496,296.8430896)
\curveto(817.77958301,296.73308494)(817.76458302,296.62308505)(817.74458496,296.5130896)
\curveto(817.74458304,296.4730852)(817.73958305,296.43308524)(817.72958496,296.3930896)
\lineto(817.72958496,296.2880896)
\curveto(817.70958308,296.23808543)(817.69958309,296.18308549)(817.69958496,296.1230896)
\curveto(817.70958308,296.06308561)(817.71458307,296.00808566)(817.71458496,295.9580896)
\lineto(817.71458496,295.6280896)
\curveto(817.71458307,295.52808614)(817.72458306,295.43308624)(817.74458496,295.3430896)
\curveto(817.75458303,295.31308636)(817.75958303,295.26308641)(817.75958496,295.1930896)
\curveto(817.77958301,295.12308655)(817.79458299,295.05308662)(817.80458496,294.9830896)
\lineto(817.86458496,294.7730896)
\curveto(817.97458281,294.42308725)(818.12458266,294.12308755)(818.31458496,293.8730896)
\curveto(818.50458228,293.62308805)(818.74458204,293.41808825)(819.03458496,293.2580896)
\curveto(819.12458166,293.20808846)(819.21458157,293.1680885)(819.30458496,293.1380896)
\curveto(819.39458139,293.10808856)(819.49458129,293.07808859)(819.60458496,293.0480896)
\curveto(819.65458113,293.02808864)(819.70458108,293.02308865)(819.75458496,293.0330896)
\curveto(819.81458097,293.04308863)(819.86958092,293.03808863)(819.91958496,293.0180896)
\curveto(819.95958083,293.00808866)(819.99958079,293.00308867)(820.03958496,293.0030896)
\lineto(820.17458496,293.0030896)
\lineto(820.30958496,293.0030896)
\curveto(820.33958045,293.01308866)(820.3895804,293.01808865)(820.45958496,293.0180896)
\curveto(820.53958025,293.03808863)(820.61958017,293.05308862)(820.69958496,293.0630896)
\curveto(820.77958001,293.08308859)(820.85457993,293.10808856)(820.92458496,293.1380896)
\curveto(821.25457953,293.27808839)(821.51957927,293.45308822)(821.71958496,293.6630896)
\curveto(821.92957886,293.88308779)(822.10457868,294.15808751)(822.24458496,294.4880896)
\curveto(822.29457849,294.59808707)(822.32957846,294.70808696)(822.34958496,294.8180896)
\curveto(822.36957842,294.92808674)(822.39457839,295.03808663)(822.42458496,295.1480896)
\curveto(822.44457834,295.18808648)(822.45457833,295.22308645)(822.45458496,295.2530896)
\curveto(822.45457833,295.29308638)(822.45957833,295.33308634)(822.46958496,295.3730896)
\curveto(822.47957831,295.43308624)(822.47957831,295.49308618)(822.46958496,295.5530896)
\curveto(822.46957832,295.61308606)(822.47457831,295.673086)(822.48458496,295.7330896)
}
}
{
\newrgbcolor{curcolor}{0 0 0}
\pscustom[linestyle=none,fillstyle=solid,fillcolor=curcolor]
{
\newpath
\moveto(832.77083496,296.2880896)
\curveto(832.7908269,296.22808544)(832.80082689,296.13308554)(832.80083496,296.0030896)
\curveto(832.80082689,295.88308579)(832.7958269,295.79808587)(832.78583496,295.7480896)
\lineto(832.78583496,295.5980896)
\curveto(832.77582692,295.51808615)(832.76582693,295.44308623)(832.75583496,295.3730896)
\curveto(832.75582694,295.31308636)(832.75082694,295.24308643)(832.74083496,295.1630896)
\curveto(832.72082697,295.10308657)(832.70582699,295.04308663)(832.69583496,294.9830896)
\curveto(832.695827,294.92308675)(832.68582701,294.86308681)(832.66583496,294.8030896)
\curveto(832.62582707,294.673087)(832.5908271,294.54308713)(832.56083496,294.4130896)
\curveto(832.53082716,294.28308739)(832.4908272,294.16308751)(832.44083496,294.0530896)
\curveto(832.23082746,293.5730881)(831.95082774,293.1680885)(831.60083496,292.8380896)
\curveto(831.25082844,292.51808915)(830.82082887,292.2730894)(830.31083496,292.1030896)
\curveto(830.20082949,292.06308961)(830.08082961,292.03308964)(829.95083496,292.0130896)
\curveto(829.83082986,291.99308968)(829.70582999,291.9730897)(829.57583496,291.9530896)
\curveto(829.51583018,291.94308973)(829.45083024,291.93808973)(829.38083496,291.9380896)
\curveto(829.32083037,291.92808974)(829.26083043,291.92308975)(829.20083496,291.9230896)
\curveto(829.16083053,291.91308976)(829.10083059,291.90808976)(829.02083496,291.9080896)
\curveto(828.95083074,291.90808976)(828.90083079,291.91308976)(828.87083496,291.9230896)
\curveto(828.83083086,291.93308974)(828.7908309,291.93808973)(828.75083496,291.9380896)
\curveto(828.71083098,291.92808974)(828.67583102,291.92808974)(828.64583496,291.9380896)
\lineto(828.55583496,291.9380896)
\lineto(828.19583496,291.9830896)
\curveto(828.05583164,292.02308965)(827.92083177,292.06308961)(827.79083496,292.1030896)
\curveto(827.66083203,292.14308953)(827.53583216,292.18808948)(827.41583496,292.2380896)
\curveto(826.96583273,292.43808923)(826.5958331,292.69808897)(826.30583496,293.0180896)
\curveto(826.01583368,293.33808833)(825.77583392,293.72808794)(825.58583496,294.1880896)
\curveto(825.53583416,294.28808738)(825.4958342,294.38808728)(825.46583496,294.4880896)
\curveto(825.44583425,294.58808708)(825.42583427,294.69308698)(825.40583496,294.8030896)
\curveto(825.38583431,294.84308683)(825.37583432,294.8730868)(825.37583496,294.8930896)
\curveto(825.38583431,294.92308675)(825.38583431,294.95808671)(825.37583496,294.9980896)
\curveto(825.35583434,295.07808659)(825.34083435,295.15808651)(825.33083496,295.2380896)
\curveto(825.33083436,295.32808634)(825.32083437,295.41308626)(825.30083496,295.4930896)
\lineto(825.30083496,295.6130896)
\curveto(825.30083439,295.65308602)(825.2958344,295.69808597)(825.28583496,295.7480896)
\curveto(825.27583442,295.79808587)(825.27083442,295.88308579)(825.27083496,296.0030896)
\curveto(825.27083442,296.13308554)(825.28083441,296.22808544)(825.30083496,296.2880896)
\curveto(825.32083437,296.35808531)(825.32583437,296.42808524)(825.31583496,296.4980896)
\curveto(825.30583439,296.5680851)(825.31083438,296.63808503)(825.33083496,296.7080896)
\curveto(825.34083435,296.75808491)(825.34583435,296.79808487)(825.34583496,296.8280896)
\curveto(825.35583434,296.8680848)(825.36583433,296.91308476)(825.37583496,296.9630896)
\curveto(825.40583429,297.08308459)(825.43083426,297.20308447)(825.45083496,297.3230896)
\curveto(825.48083421,297.44308423)(825.52083417,297.55808411)(825.57083496,297.6680896)
\curveto(825.72083397,298.03808363)(825.90083379,298.3680833)(826.11083496,298.6580896)
\curveto(826.33083336,298.95808271)(826.5958331,299.20808246)(826.90583496,299.4080896)
\curveto(827.02583267,299.48808218)(827.15083254,299.55308212)(827.28083496,299.6030896)
\curveto(827.41083228,299.66308201)(827.54583215,299.72308195)(827.68583496,299.7830896)
\curveto(827.80583189,299.83308184)(827.93583176,299.86308181)(828.07583496,299.8730896)
\curveto(828.21583148,299.89308178)(828.35583134,299.92308175)(828.49583496,299.9630896)
\lineto(828.69083496,299.9630896)
\curveto(828.76083093,299.9730817)(828.82583087,299.98308169)(828.88583496,299.9930896)
\curveto(829.77582992,300.00308167)(830.51582918,299.81808185)(831.10583496,299.4380896)
\curveto(831.695828,299.05808261)(832.12082757,298.56308311)(832.38083496,297.9530896)
\curveto(832.43082726,297.85308382)(832.47082722,297.75308392)(832.50083496,297.6530896)
\curveto(832.53082716,297.55308412)(832.56582713,297.44808422)(832.60583496,297.3380896)
\curveto(832.63582706,297.22808444)(832.66082703,297.10808456)(832.68083496,296.9780896)
\curveto(832.70082699,296.85808481)(832.72582697,296.73308494)(832.75583496,296.6030896)
\curveto(832.76582693,296.55308512)(832.76582693,296.49808517)(832.75583496,296.4380896)
\curveto(832.75582694,296.38808528)(832.76082693,296.33808533)(832.77083496,296.2880896)
\moveto(831.43583496,295.4330896)
\curveto(831.45582824,295.50308617)(831.46082823,295.58308609)(831.45083496,295.6730896)
\lineto(831.45083496,295.9280896)
\curveto(831.45082824,296.31808535)(831.41582828,296.64808502)(831.34583496,296.9180896)
\curveto(831.31582838,296.99808467)(831.2908284,297.07808459)(831.27083496,297.1580896)
\curveto(831.25082844,297.23808443)(831.22582847,297.31308436)(831.19583496,297.3830896)
\curveto(830.91582878,298.03308364)(830.47082922,298.48308319)(829.86083496,298.7330896)
\curveto(829.7908299,298.76308291)(829.71582998,298.78308289)(829.63583496,298.7930896)
\lineto(829.39583496,298.8530896)
\curveto(829.31583038,298.8730828)(829.23083046,298.88308279)(829.14083496,298.8830896)
\lineto(828.87083496,298.8830896)
\lineto(828.60083496,298.8380896)
\curveto(828.50083119,298.81808285)(828.40583129,298.79308288)(828.31583496,298.7630896)
\curveto(828.23583146,298.74308293)(828.15583154,298.71308296)(828.07583496,298.6730896)
\curveto(828.00583169,298.65308302)(827.94083175,298.62308305)(827.88083496,298.5830896)
\curveto(827.82083187,298.54308313)(827.76583193,298.50308317)(827.71583496,298.4630896)
\curveto(827.47583222,298.29308338)(827.28083241,298.08808358)(827.13083496,297.8480896)
\curveto(826.98083271,297.60808406)(826.85083284,297.32808434)(826.74083496,297.0080896)
\curveto(826.71083298,296.90808476)(826.690833,296.80308487)(826.68083496,296.6930896)
\curveto(826.67083302,296.59308508)(826.65583304,296.48808518)(826.63583496,296.3780896)
\curveto(826.62583307,296.33808533)(826.62083307,296.2730854)(826.62083496,296.1830896)
\curveto(826.61083308,296.15308552)(826.60583309,296.11808555)(826.60583496,296.0780896)
\curveto(826.61583308,296.03808563)(826.62083307,295.99308568)(826.62083496,295.9430896)
\lineto(826.62083496,295.6430896)
\curveto(826.62083307,295.54308613)(826.63083306,295.45308622)(826.65083496,295.3730896)
\lineto(826.68083496,295.1930896)
\curveto(826.70083299,295.09308658)(826.71583298,294.99308668)(826.72583496,294.8930896)
\curveto(826.74583295,294.80308687)(826.77583292,294.71808695)(826.81583496,294.6380896)
\curveto(826.91583278,294.39808727)(827.03083266,294.1730875)(827.16083496,293.9630896)
\curveto(827.30083239,293.75308792)(827.47083222,293.57808809)(827.67083496,293.4380896)
\curveto(827.72083197,293.40808826)(827.76583193,293.38308829)(827.80583496,293.3630896)
\curveto(827.84583185,293.34308833)(827.8908318,293.31808835)(827.94083496,293.2880896)
\curveto(828.02083167,293.23808843)(828.10583159,293.19308848)(828.19583496,293.1530896)
\curveto(828.2958314,293.12308855)(828.40083129,293.09308858)(828.51083496,293.0630896)
\curveto(828.56083113,293.04308863)(828.60583109,293.03308864)(828.64583496,293.0330896)
\curveto(828.695831,293.04308863)(828.74583095,293.04308863)(828.79583496,293.0330896)
\curveto(828.82583087,293.02308865)(828.88583081,293.01308866)(828.97583496,293.0030896)
\curveto(829.07583062,292.99308868)(829.15083054,292.99808867)(829.20083496,293.0180896)
\curveto(829.24083045,293.02808864)(829.28083041,293.02808864)(829.32083496,293.0180896)
\curveto(829.36083033,293.01808865)(829.40083029,293.02808864)(829.44083496,293.0480896)
\curveto(829.52083017,293.0680886)(829.60083009,293.08308859)(829.68083496,293.0930896)
\curveto(829.76082993,293.11308856)(829.83582986,293.13808853)(829.90583496,293.1680896)
\curveto(830.24582945,293.30808836)(830.52082917,293.50308817)(830.73083496,293.7530896)
\curveto(830.94082875,294.00308767)(831.11582858,294.29808737)(831.25583496,294.6380896)
\curveto(831.30582839,294.75808691)(831.33582836,294.88308679)(831.34583496,295.0130896)
\curveto(831.36582833,295.15308652)(831.3958283,295.29308638)(831.43583496,295.4330896)
}
}
{
\newrgbcolor{curcolor}{0 0 0}
\pscustom[linestyle=none,fillstyle=solid,fillcolor=curcolor]
{
\newpath
\moveto(837.90411621,299.9930896)
\curveto(838.13411142,299.99308168)(838.26411129,299.93308174)(838.29411621,299.8130896)
\curveto(838.32411123,299.70308197)(838.33911122,299.53808213)(838.33911621,299.3180896)
\lineto(838.33911621,299.0330896)
\curveto(838.33911122,298.94308273)(838.31411124,298.8680828)(838.26411621,298.8080896)
\curveto(838.20411135,298.72808294)(838.11911144,298.68308299)(838.00911621,298.6730896)
\curveto(837.89911166,298.673083)(837.78911177,298.65808301)(837.67911621,298.6280896)
\curveto(837.53911202,298.59808307)(837.40411215,298.5680831)(837.27411621,298.5380896)
\curveto(837.1541124,298.50808316)(837.03911252,298.4680832)(836.92911621,298.4180896)
\curveto(836.63911292,298.28808338)(836.40411315,298.10808356)(836.22411621,297.8780896)
\curveto(836.04411351,297.65808401)(835.88911367,297.40308427)(835.75911621,297.1130896)
\curveto(835.71911384,297.00308467)(835.68911387,296.88808478)(835.66911621,296.7680896)
\curveto(835.64911391,296.65808501)(835.62411393,296.54308513)(835.59411621,296.4230896)
\curveto(835.58411397,296.3730853)(835.57911398,296.32308535)(835.57911621,296.2730896)
\curveto(835.58911397,296.22308545)(835.58911397,296.1730855)(835.57911621,296.1230896)
\curveto(835.54911401,296.00308567)(835.53411402,295.86308581)(835.53411621,295.7030896)
\curveto(835.54411401,295.55308612)(835.54911401,295.40808626)(835.54911621,295.2680896)
\lineto(835.54911621,293.4230896)
\lineto(835.54911621,293.0780896)
\curveto(835.54911401,292.95808871)(835.54411401,292.84308883)(835.53411621,292.7330896)
\curveto(835.52411403,292.62308905)(835.51911404,292.52808914)(835.51911621,292.4480896)
\curveto(835.52911403,292.3680893)(835.50911405,292.29808937)(835.45911621,292.2380896)
\curveto(835.40911415,292.1680895)(835.32911423,292.12808954)(835.21911621,292.1180896)
\curveto(835.11911444,292.10808956)(835.00911455,292.10308957)(834.88911621,292.1030896)
\lineto(834.61911621,292.1030896)
\curveto(834.56911499,292.12308955)(834.51911504,292.13808953)(834.46911621,292.1480896)
\curveto(834.42911513,292.1680895)(834.39911516,292.19308948)(834.37911621,292.2230896)
\curveto(834.32911523,292.29308938)(834.29911526,292.37808929)(834.28911621,292.4780896)
\lineto(834.28911621,292.8080896)
\lineto(834.28911621,293.9630896)
\lineto(834.28911621,298.1180896)
\lineto(834.28911621,299.1530896)
\lineto(834.28911621,299.4530896)
\curveto(834.29911526,299.55308212)(834.32911523,299.63808203)(834.37911621,299.7080896)
\curveto(834.40911515,299.74808192)(834.4591151,299.77808189)(834.52911621,299.7980896)
\curveto(834.60911495,299.81808185)(834.69411486,299.82808184)(834.78411621,299.8280896)
\curveto(834.87411468,299.83808183)(834.96411459,299.83808183)(835.05411621,299.8280896)
\curveto(835.14411441,299.81808185)(835.21411434,299.80308187)(835.26411621,299.7830896)
\curveto(835.34411421,299.75308192)(835.39411416,299.69308198)(835.41411621,299.6030896)
\curveto(835.44411411,299.52308215)(835.4591141,299.43308224)(835.45911621,299.3330896)
\lineto(835.45911621,299.0330896)
\curveto(835.4591141,298.93308274)(835.47911408,298.84308283)(835.51911621,298.7630896)
\curveto(835.52911403,298.74308293)(835.53911402,298.72808294)(835.54911621,298.7180896)
\lineto(835.59411621,298.6730896)
\curveto(835.70411385,298.673083)(835.79411376,298.71808295)(835.86411621,298.8080896)
\curveto(835.93411362,298.90808276)(835.99411356,298.98808268)(836.04411621,299.0480896)
\lineto(836.13411621,299.1380896)
\curveto(836.22411333,299.24808242)(836.34911321,299.36308231)(836.50911621,299.4830896)
\curveto(836.66911289,299.60308207)(836.81911274,299.69308198)(836.95911621,299.7530896)
\curveto(837.04911251,299.80308187)(837.14411241,299.83808183)(837.24411621,299.8580896)
\curveto(837.34411221,299.88808178)(837.44911211,299.91808175)(837.55911621,299.9480896)
\curveto(837.61911194,299.95808171)(837.67911188,299.96308171)(837.73911621,299.9630896)
\curveto(837.79911176,299.9730817)(837.8541117,299.98308169)(837.90411621,299.9930896)
}
}
{
\newrgbcolor{curcolor}{0 0 0}
\pscustom[linestyle=none,fillstyle=solid,fillcolor=curcolor]
{
\newpath
\moveto(846.01888184,296.2580896)
\curveto(846.03887415,296.15808551)(846.03887415,296.04308563)(846.01888184,295.9130896)
\curveto(846.00887418,295.79308588)(845.97887421,295.70808596)(845.92888184,295.6580896)
\curveto(845.87887431,295.61808605)(845.80387439,295.58808608)(845.70388184,295.5680896)
\curveto(845.61387458,295.55808611)(845.50887468,295.55308612)(845.38888184,295.5530896)
\lineto(845.02888184,295.5530896)
\curveto(844.90887528,295.56308611)(844.80387539,295.5680861)(844.71388184,295.5680896)
\lineto(840.87388184,295.5680896)
\curveto(840.7938794,295.5680861)(840.71387948,295.56308611)(840.63388184,295.5530896)
\curveto(840.55387964,295.55308612)(840.4888797,295.53808613)(840.43888184,295.5080896)
\curveto(840.39887979,295.48808618)(840.35887983,295.44808622)(840.31888184,295.3880896)
\curveto(840.29887989,295.35808631)(840.27887991,295.31308636)(840.25888184,295.2530896)
\curveto(840.23887995,295.20308647)(840.23887995,295.15308652)(840.25888184,295.1030896)
\curveto(840.26887992,295.05308662)(840.27387992,295.00808666)(840.27388184,294.9680896)
\curveto(840.27387992,294.92808674)(840.27887991,294.88808678)(840.28888184,294.8480896)
\curveto(840.30887988,294.7680869)(840.32887986,294.68308699)(840.34888184,294.5930896)
\curveto(840.36887982,294.51308716)(840.39887979,294.43308724)(840.43888184,294.3530896)
\curveto(840.66887952,293.81308786)(841.04887914,293.42808824)(841.57888184,293.1980896)
\curveto(841.63887855,293.1680885)(841.70387849,293.14308853)(841.77388184,293.1230896)
\lineto(841.98388184,293.0630896)
\curveto(842.01387818,293.05308862)(842.06387813,293.04808862)(842.13388184,293.0480896)
\curveto(842.27387792,293.00808866)(842.45887773,292.98808868)(842.68888184,292.9880896)
\curveto(842.91887727,292.98808868)(843.10387709,293.00808866)(843.24388184,293.0480896)
\curveto(843.38387681,293.08808858)(843.50887668,293.12808854)(843.61888184,293.1680896)
\curveto(843.73887645,293.21808845)(843.84887634,293.27808839)(843.94888184,293.3480896)
\curveto(844.05887613,293.41808825)(844.15387604,293.49808817)(844.23388184,293.5880896)
\curveto(844.31387588,293.68808798)(844.38387581,293.79308788)(844.44388184,293.9030896)
\curveto(844.50387569,294.00308767)(844.55387564,294.10808756)(844.59388184,294.2180896)
\curveto(844.64387555,294.32808734)(844.72387547,294.40808726)(844.83388184,294.4580896)
\curveto(844.87387532,294.47808719)(844.93887525,294.49308718)(845.02888184,294.5030896)
\curveto(845.11887507,294.51308716)(845.20887498,294.51308716)(845.29888184,294.5030896)
\curveto(845.3888748,294.50308717)(845.47387472,294.49808717)(845.55388184,294.4880896)
\curveto(845.63387456,294.47808719)(845.6888745,294.45808721)(845.71888184,294.4280896)
\curveto(845.81887437,294.35808731)(845.84387435,294.24308743)(845.79388184,294.0830896)
\curveto(845.71387448,293.81308786)(845.60887458,293.5730881)(845.47888184,293.3630896)
\curveto(845.27887491,293.04308863)(845.04887514,292.77808889)(844.78888184,292.5680896)
\curveto(844.53887565,292.3680893)(844.21887597,292.20308947)(843.82888184,292.0730896)
\curveto(843.72887646,292.03308964)(843.62887656,292.00808966)(843.52888184,291.9980896)
\curveto(843.42887676,291.97808969)(843.32387687,291.95808971)(843.21388184,291.9380896)
\curveto(843.16387703,291.92808974)(843.11387708,291.92308975)(843.06388184,291.9230896)
\curveto(843.02387717,291.92308975)(842.97887721,291.91808975)(842.92888184,291.9080896)
\lineto(842.77888184,291.9080896)
\curveto(842.72887746,291.89808977)(842.66887752,291.89308978)(842.59888184,291.8930896)
\curveto(842.53887765,291.89308978)(842.4888777,291.89808977)(842.44888184,291.9080896)
\lineto(842.31388184,291.9080896)
\curveto(842.26387793,291.91808975)(842.21887797,291.92308975)(842.17888184,291.9230896)
\curveto(842.13887805,291.92308975)(842.09887809,291.92808974)(842.05888184,291.9380896)
\curveto(842.00887818,291.94808972)(841.95387824,291.95808971)(841.89388184,291.9680896)
\curveto(841.83387836,291.9680897)(841.77887841,291.9730897)(841.72888184,291.9830896)
\curveto(841.63887855,292.00308967)(841.54887864,292.02808964)(841.45888184,292.0580896)
\curveto(841.36887882,292.07808959)(841.28387891,292.10308957)(841.20388184,292.1330896)
\curveto(841.16387903,292.15308952)(841.12887906,292.16308951)(841.09888184,292.1630896)
\curveto(841.06887912,292.1730895)(841.03387916,292.18808948)(840.99388184,292.2080896)
\curveto(840.84387935,292.27808939)(840.68387951,292.36308931)(840.51388184,292.4630896)
\curveto(840.22387997,292.65308902)(839.97388022,292.88308879)(839.76388184,293.1530896)
\curveto(839.56388063,293.43308824)(839.3938808,293.74308793)(839.25388184,294.0830896)
\curveto(839.20388099,294.19308748)(839.16388103,294.30808736)(839.13388184,294.4280896)
\curveto(839.11388108,294.54808712)(839.08388111,294.668087)(839.04388184,294.7880896)
\curveto(839.03388116,294.82808684)(839.02888116,294.86308681)(839.02888184,294.8930896)
\curveto(839.02888116,294.92308675)(839.02388117,294.96308671)(839.01388184,295.0130896)
\curveto(838.9938812,295.09308658)(838.97888121,295.17808649)(838.96888184,295.2680896)
\curveto(838.95888123,295.35808631)(838.94388125,295.44808622)(838.92388184,295.5380896)
\lineto(838.92388184,295.7480896)
\curveto(838.91388128,295.78808588)(838.90388129,295.84308583)(838.89388184,295.9130896)
\curveto(838.8938813,295.99308568)(838.89888129,296.05808561)(838.90888184,296.1080896)
\lineto(838.90888184,296.2730896)
\curveto(838.92888126,296.32308535)(838.93388126,296.3730853)(838.92388184,296.4230896)
\curveto(838.92388127,296.48308519)(838.92888126,296.53808513)(838.93888184,296.5880896)
\curveto(838.97888121,296.74808492)(839.00888118,296.90808476)(839.02888184,297.0680896)
\curveto(839.05888113,297.22808444)(839.10388109,297.37808429)(839.16388184,297.5180896)
\curveto(839.21388098,297.62808404)(839.25888093,297.73808393)(839.29888184,297.8480896)
\curveto(839.34888084,297.9680837)(839.40388079,298.08308359)(839.46388184,298.1930896)
\curveto(839.68388051,298.54308313)(839.93388026,298.84308283)(840.21388184,299.0930896)
\curveto(840.4938797,299.35308232)(840.83887935,299.5680821)(841.24888184,299.7380896)
\curveto(841.36887882,299.78808188)(841.4888787,299.82308185)(841.60888184,299.8430896)
\curveto(841.73887845,299.8730818)(841.87387832,299.90308177)(842.01388184,299.9330896)
\curveto(842.06387813,299.94308173)(842.10887808,299.94808172)(842.14888184,299.9480896)
\curveto(842.188878,299.95808171)(842.23387796,299.96308171)(842.28388184,299.9630896)
\curveto(842.30387789,299.9730817)(842.32887786,299.9730817)(842.35888184,299.9630896)
\curveto(842.3888778,299.95308172)(842.41387778,299.95808171)(842.43388184,299.9780896)
\curveto(842.85387734,299.98808168)(843.21887697,299.94308173)(843.52888184,299.8430896)
\curveto(843.83887635,299.75308192)(844.11887607,299.62808204)(844.36888184,299.4680896)
\curveto(844.41887577,299.44808222)(844.45887573,299.41808225)(844.48888184,299.3780896)
\curveto(844.51887567,299.34808232)(844.55387564,299.32308235)(844.59388184,299.3030896)
\curveto(844.67387552,299.24308243)(844.75387544,299.1730825)(844.83388184,299.0930896)
\curveto(844.92387527,299.01308266)(844.99887519,298.93308274)(845.05888184,298.8530896)
\curveto(845.21887497,298.64308303)(845.35387484,298.44308323)(845.46388184,298.2530896)
\curveto(845.53387466,298.14308353)(845.5888746,298.02308365)(845.62888184,297.8930896)
\curveto(845.66887452,297.76308391)(845.71387448,297.63308404)(845.76388184,297.5030896)
\curveto(845.81387438,297.3730843)(845.84887434,297.23808443)(845.86888184,297.0980896)
\curveto(845.89887429,296.95808471)(845.93387426,296.81808485)(845.97388184,296.6780896)
\curveto(845.98387421,296.60808506)(845.9888742,296.53808513)(845.98888184,296.4680896)
\lineto(846.01888184,296.2580896)
\moveto(844.56388184,296.7680896)
\curveto(844.5938756,296.80808486)(844.61887557,296.85808481)(844.63888184,296.9180896)
\curveto(844.65887553,296.98808468)(844.65887553,297.05808461)(844.63888184,297.1280896)
\curveto(844.57887561,297.34808432)(844.4938757,297.55308412)(844.38388184,297.7430896)
\curveto(844.24387595,297.9730837)(844.0888761,298.1680835)(843.91888184,298.3280896)
\curveto(843.74887644,298.48808318)(843.52887666,298.62308305)(843.25888184,298.7330896)
\curveto(843.188877,298.75308292)(843.11887707,298.7680829)(843.04888184,298.7780896)
\curveto(842.97887721,298.79808287)(842.90387729,298.81808285)(842.82388184,298.8380896)
\curveto(842.74387745,298.85808281)(842.65887753,298.8680828)(842.56888184,298.8680896)
\lineto(842.31388184,298.8680896)
\curveto(842.28387791,298.84808282)(842.24887794,298.83808283)(842.20888184,298.8380896)
\curveto(842.16887802,298.84808282)(842.13387806,298.84808282)(842.10388184,298.8380896)
\lineto(841.86388184,298.7780896)
\curveto(841.7938784,298.7680829)(841.72387847,298.75308292)(841.65388184,298.7330896)
\curveto(841.36387883,298.61308306)(841.12887906,298.46308321)(840.94888184,298.2830896)
\curveto(840.77887941,298.10308357)(840.62387957,297.87808379)(840.48388184,297.6080896)
\curveto(840.45387974,297.55808411)(840.42387977,297.49308418)(840.39388184,297.4130896)
\curveto(840.36387983,297.34308433)(840.33887985,297.26308441)(840.31888184,297.1730896)
\curveto(840.29887989,297.08308459)(840.2938799,296.99808467)(840.30388184,296.9180896)
\curveto(840.31387988,296.83808483)(840.34887984,296.77808489)(840.40888184,296.7380896)
\curveto(840.4888797,296.67808499)(840.62387957,296.64808502)(840.81388184,296.6480896)
\curveto(841.01387918,296.65808501)(841.18387901,296.66308501)(841.32388184,296.6630896)
\lineto(843.60388184,296.6630896)
\curveto(843.75387644,296.66308501)(843.93387626,296.65808501)(844.14388184,296.6480896)
\curveto(844.35387584,296.64808502)(844.4938757,296.68808498)(844.56388184,296.7680896)
}
}
{
\newrgbcolor{curcolor}{0 0 0}
\pscustom[linestyle=none,fillstyle=solid,fillcolor=curcolor]
{
\newpath
\moveto(849.75552246,299.9930896)
\curveto(850.4755184,300.00308167)(851.08051779,299.91808175)(851.57052246,299.7380896)
\curveto(852.06051681,299.5680821)(852.44051643,299.26308241)(852.71052246,298.8230896)
\curveto(852.78051609,298.71308296)(852.83551604,298.59808307)(852.87552246,298.4780896)
\curveto(852.91551596,298.3680833)(852.95551592,298.24308343)(852.99552246,298.1030896)
\curveto(853.01551586,298.03308364)(853.02051585,297.95808371)(853.01052246,297.8780896)
\curveto(853.00051587,297.80808386)(852.98551589,297.75308392)(852.96552246,297.7130896)
\curveto(852.94551593,297.69308398)(852.92051595,297.673084)(852.89052246,297.6530896)
\curveto(852.86051601,297.64308403)(852.83551604,297.62808404)(852.81552246,297.6080896)
\curveto(852.76551611,297.58808408)(852.71551616,297.58308409)(852.66552246,297.5930896)
\curveto(852.61551626,297.60308407)(852.56551631,297.60308407)(852.51552246,297.5930896)
\curveto(852.43551644,297.5730841)(852.33051654,297.5680841)(852.20052246,297.5780896)
\curveto(852.0705168,297.59808407)(851.98051689,297.62308405)(851.93052246,297.6530896)
\curveto(851.85051702,297.70308397)(851.79551708,297.7680839)(851.76552246,297.8480896)
\curveto(851.74551713,297.93808373)(851.71051716,298.02308365)(851.66052246,298.1030896)
\curveto(851.5705173,298.26308341)(851.44551743,298.40808326)(851.28552246,298.5380896)
\curveto(851.1755177,298.61808305)(851.05551782,298.67808299)(850.92552246,298.7180896)
\curveto(850.79551808,298.75808291)(850.65551822,298.79808287)(850.50552246,298.8380896)
\curveto(850.45551842,298.85808281)(850.40551847,298.86308281)(850.35552246,298.8530896)
\curveto(850.30551857,298.85308282)(850.25551862,298.85808281)(850.20552246,298.8680896)
\curveto(850.14551873,298.88808278)(850.0705188,298.89808277)(849.98052246,298.8980896)
\curveto(849.89051898,298.89808277)(849.81551906,298.88808278)(849.75552246,298.8680896)
\lineto(849.66552246,298.8680896)
\lineto(849.51552246,298.8380896)
\curveto(849.46551941,298.83808283)(849.41551946,298.83308284)(849.36552246,298.8230896)
\curveto(849.10551977,298.76308291)(848.89051998,298.67808299)(848.72052246,298.5680896)
\curveto(848.55052032,298.45808321)(848.43552044,298.2730834)(848.37552246,298.0130896)
\curveto(848.35552052,297.94308373)(848.35052052,297.8730838)(848.36052246,297.8030896)
\curveto(848.38052049,297.73308394)(848.40052047,297.673084)(848.42052246,297.6230896)
\curveto(848.48052039,297.4730842)(848.55052032,297.36308431)(848.63052246,297.2930896)
\curveto(848.72052015,297.23308444)(848.83052004,297.16308451)(848.96052246,297.0830896)
\curveto(849.12051975,296.98308469)(849.30051957,296.90808476)(849.50052246,296.8580896)
\curveto(849.70051917,296.81808485)(849.90051897,296.7680849)(850.10052246,296.7080896)
\curveto(850.23051864,296.668085)(850.36051851,296.63808503)(850.49052246,296.6180896)
\curveto(850.62051825,296.59808507)(850.75051812,296.5680851)(850.88052246,296.5280896)
\curveto(851.09051778,296.4680852)(851.29551758,296.40808526)(851.49552246,296.3480896)
\curveto(851.69551718,296.29808537)(851.89551698,296.23308544)(852.09552246,296.1530896)
\lineto(852.24552246,296.0930896)
\curveto(852.29551658,296.0730856)(852.34551653,296.04808562)(852.39552246,296.0180896)
\curveto(852.59551628,295.89808577)(852.7705161,295.76308591)(852.92052246,295.6130896)
\curveto(853.0705158,295.46308621)(853.19551568,295.2730864)(853.29552246,295.0430896)
\curveto(853.31551556,294.9730867)(853.33551554,294.87808679)(853.35552246,294.7580896)
\curveto(853.3755155,294.68808698)(853.38551549,294.61308706)(853.38552246,294.5330896)
\curveto(853.39551548,294.46308721)(853.40051547,294.38308729)(853.40052246,294.2930896)
\lineto(853.40052246,294.1430896)
\curveto(853.38051549,294.0730876)(853.3705155,294.00308767)(853.37052246,293.9330896)
\curveto(853.3705155,293.86308781)(853.36051551,293.79308788)(853.34052246,293.7230896)
\curveto(853.31051556,293.61308806)(853.2755156,293.50808816)(853.23552246,293.4080896)
\curveto(853.19551568,293.30808836)(853.15051572,293.21808845)(853.10052246,293.1380896)
\curveto(852.94051593,292.87808879)(852.73551614,292.668089)(852.48552246,292.5080896)
\curveto(852.23551664,292.35808931)(851.95551692,292.22808944)(851.64552246,292.1180896)
\curveto(851.55551732,292.08808958)(851.46051741,292.0680896)(851.36052246,292.0580896)
\curveto(851.2705176,292.03808963)(851.18051769,292.01308966)(851.09052246,291.9830896)
\curveto(850.99051788,291.96308971)(850.89051798,291.95308972)(850.79052246,291.9530896)
\curveto(850.69051818,291.95308972)(850.59051828,291.94308973)(850.49052246,291.9230896)
\lineto(850.34052246,291.9230896)
\curveto(850.29051858,291.91308976)(850.22051865,291.90808976)(850.13052246,291.9080896)
\curveto(850.04051883,291.90808976)(849.9705189,291.91308976)(849.92052246,291.9230896)
\lineto(849.75552246,291.9230896)
\curveto(849.69551918,291.94308973)(849.63051924,291.95308972)(849.56052246,291.9530896)
\curveto(849.49051938,291.94308973)(849.43051944,291.94808972)(849.38052246,291.9680896)
\curveto(849.33051954,291.97808969)(849.26551961,291.98308969)(849.18552246,291.9830896)
\lineto(848.94552246,292.0430896)
\curveto(848.87552,292.05308962)(848.80052007,292.0730896)(848.72052246,292.1030896)
\curveto(848.41052046,292.20308947)(848.14052073,292.32808934)(847.91052246,292.4780896)
\curveto(847.68052119,292.62808904)(847.48052139,292.82308885)(847.31052246,293.0630896)
\curveto(847.22052165,293.19308848)(847.14552173,293.32808834)(847.08552246,293.4680896)
\curveto(847.02552185,293.60808806)(846.9705219,293.76308791)(846.92052246,293.9330896)
\curveto(846.90052197,293.99308768)(846.89052198,294.06308761)(846.89052246,294.1430896)
\curveto(846.90052197,294.23308744)(846.91552196,294.30308737)(846.93552246,294.3530896)
\curveto(846.96552191,294.39308728)(847.01552186,294.43308724)(847.08552246,294.4730896)
\curveto(847.13552174,294.49308718)(847.20552167,294.50308717)(847.29552246,294.5030896)
\curveto(847.38552149,294.51308716)(847.4755214,294.51308716)(847.56552246,294.5030896)
\curveto(847.65552122,294.49308718)(847.74052113,294.47808719)(847.82052246,294.4580896)
\curveto(847.91052096,294.44808722)(847.9705209,294.43308724)(848.00052246,294.4130896)
\curveto(848.0705208,294.36308731)(848.11552076,294.28808738)(848.13552246,294.1880896)
\curveto(848.16552071,294.09808757)(848.20052067,294.01308766)(848.24052246,293.9330896)
\curveto(848.34052053,293.71308796)(848.4755204,293.54308813)(848.64552246,293.4230896)
\curveto(848.76552011,293.33308834)(848.90051997,293.26308841)(849.05052246,293.2130896)
\curveto(849.20051967,293.16308851)(849.36051951,293.11308856)(849.53052246,293.0630896)
\lineto(849.84552246,293.0180896)
\lineto(849.93552246,293.0180896)
\curveto(850.00551887,292.99808867)(850.09551878,292.98808868)(850.20552246,292.9880896)
\curveto(850.32551855,292.98808868)(850.42551845,292.99808867)(850.50552246,293.0180896)
\curveto(850.5755183,293.01808865)(850.63051824,293.02308865)(850.67052246,293.0330896)
\curveto(850.73051814,293.04308863)(850.79051808,293.04808862)(850.85052246,293.0480896)
\curveto(850.91051796,293.05808861)(850.96551791,293.0680886)(851.01552246,293.0780896)
\curveto(851.30551757,293.15808851)(851.53551734,293.26308841)(851.70552246,293.3930896)
\curveto(851.875517,293.52308815)(851.99551688,293.74308793)(852.06552246,294.0530896)
\curveto(852.08551679,294.10308757)(852.09051678,294.15808751)(852.08052246,294.2180896)
\curveto(852.0705168,294.27808739)(852.06051681,294.32308735)(852.05052246,294.3530896)
\curveto(852.00051687,294.54308713)(851.93051694,294.68308699)(851.84052246,294.7730896)
\curveto(851.75051712,294.8730868)(851.63551724,294.96308671)(851.49552246,295.0430896)
\curveto(851.40551747,295.10308657)(851.30551757,295.15308652)(851.19552246,295.1930896)
\lineto(850.86552246,295.3130896)
\curveto(850.83551804,295.32308635)(850.80551807,295.32808634)(850.77552246,295.3280896)
\curveto(850.75551812,295.32808634)(850.73051814,295.33808633)(850.70052246,295.3580896)
\curveto(850.36051851,295.4680862)(850.00551887,295.54808612)(849.63552246,295.5980896)
\curveto(849.2755196,295.65808601)(848.93551994,295.75308592)(848.61552246,295.8830896)
\curveto(848.51552036,295.92308575)(848.42052045,295.95808571)(848.33052246,295.9880896)
\curveto(848.24052063,296.01808565)(848.15552072,296.05808561)(848.07552246,296.1080896)
\curveto(847.88552099,296.21808545)(847.71052116,296.34308533)(847.55052246,296.4830896)
\curveto(847.39052148,296.62308505)(847.26552161,296.79808487)(847.17552246,297.0080896)
\curveto(847.14552173,297.07808459)(847.12052175,297.14808452)(847.10052246,297.2180896)
\curveto(847.09052178,297.28808438)(847.0755218,297.36308431)(847.05552246,297.4430896)
\curveto(847.02552185,297.56308411)(847.01552186,297.69808397)(847.02552246,297.8480896)
\curveto(847.03552184,298.00808366)(847.05052182,298.14308353)(847.07052246,298.2530896)
\curveto(847.09052178,298.30308337)(847.10052177,298.34308333)(847.10052246,298.3730896)
\curveto(847.11052176,298.41308326)(847.12552175,298.45308322)(847.14552246,298.4930896)
\curveto(847.23552164,298.72308295)(847.35552152,298.92308275)(847.50552246,299.0930896)
\curveto(847.66552121,299.26308241)(847.84552103,299.41308226)(848.04552246,299.5430896)
\curveto(848.19552068,299.63308204)(848.36052051,299.70308197)(848.54052246,299.7530896)
\curveto(848.72052015,299.81308186)(848.91051996,299.8680818)(849.11052246,299.9180896)
\curveto(849.18051969,299.92808174)(849.24551963,299.93808173)(849.30552246,299.9480896)
\curveto(849.3755195,299.95808171)(849.45051942,299.9680817)(849.53052246,299.9780896)
\curveto(849.56051931,299.98808168)(849.60051927,299.98808168)(849.65052246,299.9780896)
\curveto(849.70051917,299.9680817)(849.73551914,299.9730817)(849.75552246,299.9930896)
}
}
{
\newrgbcolor{curcolor}{0.80000001 0.80000001 0.80000001}
\pscustom[linestyle=none,fillstyle=solid,fillcolor=curcolor]
{
\newpath
\moveto(747.9732666,372.02397156)
\lineto(762.9732666,372.02397156)
\lineto(762.9732666,357.02397156)
\lineto(747.9732666,357.02397156)
\closepath
}
}
{
\newrgbcolor{curcolor}{0.7019608 0.7019608 0.7019608}
\pscustom[linestyle=none,fillstyle=solid,fillcolor=curcolor]
{
\newpath
\moveto(747.9732666,348.69264984)
\lineto(762.9732666,348.69264984)
\lineto(762.9732666,333.69264984)
\lineto(747.9732666,333.69264984)
\closepath
}
}
{
\newrgbcolor{curcolor}{0.60000002 0.60000002 0.60000002}
\pscustom[linestyle=none,fillstyle=solid,fillcolor=curcolor]
{
\newpath
\moveto(747.9732666,325.37634277)
\lineto(762.9732666,325.37634277)
\lineto(762.9732666,310.37634277)
\lineto(747.9732666,310.37634277)
\closepath
}
}
{
\newrgbcolor{curcolor}{0.50196081 0.50196081 0.50196081}
\pscustom[linestyle=none,fillstyle=solid,fillcolor=curcolor]
{
\newpath
\moveto(747.9732666,302.78309631)
\lineto(762.9732666,302.78309631)
\lineto(762.9732666,287.78309631)
\lineto(747.9732666,287.78309631)
\closepath
}
}
{
\newrgbcolor{curcolor}{0.80000001 0.80000001 0.80000001}
\pscustom[linestyle=none,fillstyle=solid,fillcolor=curcolor]
{
\newpath
\moveto(541.98950195,77.19110107)
\lineto(555.03574848,77.19110107)
\lineto(555.03574848,76.08277607)
\lineto(541.98950195,76.08277607)
\closepath
}
}
{
\newrgbcolor{curcolor}{0 0 0}
\pscustom[linestyle=none,fillstyle=solid,fillcolor=curcolor]
{
\newpath
\moveto(767.94645996,280.0180896)
\lineto(772.85145996,280.0180896)
\lineto(774.14145996,280.0180896)
\curveto(774.25145208,280.0180789)(774.36145197,280.0180789)(774.47145996,280.0180896)
\curveto(774.58145175,280.02807889)(774.67145166,280.00807891)(774.74145996,279.9580896)
\curveto(774.77145156,279.93807898)(774.79645154,279.91307901)(774.81645996,279.8830896)
\curveto(774.8364515,279.85307907)(774.85645148,279.8230791)(774.87645996,279.7930896)
\curveto(774.89645144,279.7230792)(774.90645143,279.60807931)(774.90645996,279.4480896)
\curveto(774.90645143,279.29807962)(774.89645144,279.18307974)(774.87645996,279.1030896)
\curveto(774.8364515,278.96307996)(774.75145158,278.88308004)(774.62145996,278.8630896)
\curveto(774.49145184,278.85308007)(774.336452,278.84808007)(774.15645996,278.8480896)
\lineto(772.65645996,278.8480896)
\lineto(770.13645996,278.8480896)
\lineto(769.56645996,278.8480896)
\curveto(769.35645698,278.85808006)(769.20145713,278.83308009)(769.10145996,278.7730896)
\curveto(769.00145733,278.71308021)(768.94645739,278.60808031)(768.93645996,278.4580896)
\lineto(768.93645996,277.9930896)
\lineto(768.93645996,276.4630896)
\curveto(768.9364574,276.35308257)(768.9314574,276.2230827)(768.92145996,276.0730896)
\curveto(768.92145741,275.923083)(768.9314574,275.80308312)(768.95145996,275.7130896)
\curveto(768.98145735,275.59308333)(769.04145729,275.51308341)(769.13145996,275.4730896)
\curveto(769.17145716,275.45308347)(769.24145709,275.43308349)(769.34145996,275.4130896)
\lineto(769.49145996,275.4130896)
\curveto(769.5314568,275.40308352)(769.57145676,275.39808352)(769.61145996,275.3980896)
\curveto(769.66145667,275.40808351)(769.71145662,275.41308351)(769.76145996,275.4130896)
\lineto(770.27145996,275.4130896)
\lineto(773.21145996,275.4130896)
\lineto(773.51145996,275.4130896)
\curveto(773.62145271,275.4230835)(773.7314526,275.4230835)(773.84145996,275.4130896)
\curveto(773.96145237,275.41308351)(774.06645227,275.40308352)(774.15645996,275.3830896)
\curveto(774.25645208,275.37308355)(774.331452,275.35308357)(774.38145996,275.3230896)
\curveto(774.41145192,275.30308362)(774.4364519,275.25808366)(774.45645996,275.1880896)
\curveto(774.47645186,275.1180838)(774.49145184,275.04308388)(774.50145996,274.9630896)
\curveto(774.51145182,274.88308404)(774.51145182,274.79808412)(774.50145996,274.7080896)
\curveto(774.50145183,274.62808429)(774.49145184,274.55808436)(774.47145996,274.4980896)
\curveto(774.45145188,274.40808451)(774.40645193,274.34308458)(774.33645996,274.3030896)
\curveto(774.31645202,274.28308464)(774.28645205,274.26808465)(774.24645996,274.2580896)
\curveto(774.21645212,274.25808466)(774.18645215,274.25308467)(774.15645996,274.2430896)
\lineto(774.06645996,274.2430896)
\curveto(774.01645232,274.23308469)(773.96645237,274.22808469)(773.91645996,274.2280896)
\curveto(773.86645247,274.23808468)(773.81645252,274.24308468)(773.76645996,274.2430896)
\lineto(773.21145996,274.2430896)
\lineto(770.04645996,274.2430896)
\lineto(769.68645996,274.2430896)
\curveto(769.57645676,274.25308467)(769.47145686,274.24808467)(769.37145996,274.2280896)
\curveto(769.27145706,274.2180847)(769.18145715,274.19308473)(769.10145996,274.1530896)
\curveto(769.0314573,274.11308481)(768.98145735,274.04308488)(768.95145996,273.9430896)
\curveto(768.9314574,273.88308504)(768.92145741,273.81308511)(768.92145996,273.7330896)
\curveto(768.9314574,273.65308527)(768.9364574,273.57308535)(768.93645996,273.4930896)
\lineto(768.93645996,272.6530896)
\lineto(768.93645996,271.2280896)
\curveto(768.9364574,271.08808783)(768.94145739,270.95808796)(768.95145996,270.8380896)
\curveto(768.96145737,270.72808819)(769.00145733,270.64808827)(769.07145996,270.5980896)
\curveto(769.14145719,270.54808837)(769.22145711,270.5180884)(769.31145996,270.5080896)
\lineto(769.61145996,270.5080896)
\lineto(770.57145996,270.5080896)
\lineto(773.34645996,270.5080896)
\lineto(774.20145996,270.5080896)
\lineto(774.44145996,270.5080896)
\curveto(774.52145181,270.5180884)(774.59145174,270.51308841)(774.65145996,270.4930896)
\curveto(774.77145156,270.45308847)(774.85145148,270.39808852)(774.89145996,270.3280896)
\curveto(774.91145142,270.29808862)(774.92645141,270.24808867)(774.93645996,270.1780896)
\curveto(774.94645139,270.10808881)(774.95145138,270.03308889)(774.95145996,269.9530896)
\curveto(774.96145137,269.88308904)(774.96145137,269.80808911)(774.95145996,269.7280896)
\curveto(774.94145139,269.65808926)(774.9314514,269.60308932)(774.92145996,269.5630896)
\curveto(774.88145145,269.48308944)(774.8364515,269.42808949)(774.78645996,269.3980896)
\curveto(774.72645161,269.35808956)(774.64645169,269.33808958)(774.54645996,269.3380896)
\lineto(774.27645996,269.3380896)
\lineto(773.22645996,269.3380896)
\lineto(769.23645996,269.3380896)
\lineto(768.18645996,269.3380896)
\curveto(768.04645829,269.33808958)(767.92645841,269.34308958)(767.82645996,269.3530896)
\curveto(767.72645861,269.37308955)(767.65145868,269.4230895)(767.60145996,269.5030896)
\curveto(767.56145877,269.56308936)(767.54145879,269.63808928)(767.54145996,269.7280896)
\lineto(767.54145996,270.0130896)
\lineto(767.54145996,271.0630896)
\lineto(767.54145996,275.0830896)
\lineto(767.54145996,278.4430896)
\lineto(767.54145996,279.3730896)
\lineto(767.54145996,279.6430896)
\curveto(767.54145879,279.73307919)(767.56145877,279.80307912)(767.60145996,279.8530896)
\curveto(767.64145869,279.923079)(767.71645862,279.97307895)(767.82645996,280.0030896)
\curveto(767.84645849,280.01307891)(767.86645847,280.01307891)(767.88645996,280.0030896)
\curveto(767.90645843,280.00307892)(767.92645841,280.00807891)(767.94645996,280.0180896)
}
}
{
\newrgbcolor{curcolor}{0 0 0}
\pscustom[linestyle=none,fillstyle=solid,fillcolor=curcolor]
{
\newpath
\moveto(777.40138184,279.4030896)
\curveto(777.55137983,279.40307952)(777.70137968,279.39807952)(777.85138184,279.3880896)
\curveto(778.00137938,279.38807953)(778.10637927,279.34807957)(778.16638184,279.2680896)
\curveto(778.21637916,279.20807971)(778.24137914,279.1230798)(778.24138184,279.0130896)
\curveto(778.25137913,278.91308001)(778.25637912,278.80808011)(778.25638184,278.6980896)
\lineto(778.25638184,277.8280896)
\curveto(778.25637912,277.74808117)(778.25137913,277.66308126)(778.24138184,277.5730896)
\curveto(778.24137914,277.49308143)(778.25137913,277.4230815)(778.27138184,277.3630896)
\curveto(778.31137907,277.2230817)(778.40137898,277.13308179)(778.54138184,277.0930896)
\curveto(778.59137879,277.08308184)(778.63637874,277.07808184)(778.67638184,277.0780896)
\lineto(778.82638184,277.0780896)
\lineto(779.23138184,277.0780896)
\curveto(779.39137799,277.08808183)(779.50637787,277.07808184)(779.57638184,277.0480896)
\curveto(779.66637771,276.98808193)(779.72637765,276.92808199)(779.75638184,276.8680896)
\curveto(779.7763776,276.82808209)(779.78637759,276.78308214)(779.78638184,276.7330896)
\lineto(779.78638184,276.5830896)
\curveto(779.78637759,276.47308245)(779.7813776,276.36808255)(779.77138184,276.2680896)
\curveto(779.76137762,276.17808274)(779.72637765,276.10808281)(779.66638184,276.0580896)
\curveto(779.60637777,276.00808291)(779.52137786,275.97808294)(779.41138184,275.9680896)
\lineto(779.08138184,275.9680896)
\curveto(778.97137841,275.97808294)(778.86137852,275.98308294)(778.75138184,275.9830896)
\curveto(778.64137874,275.98308294)(778.54637883,275.96808295)(778.46638184,275.9380896)
\curveto(778.39637898,275.90808301)(778.34637903,275.85808306)(778.31638184,275.7880896)
\curveto(778.28637909,275.7180832)(778.26637911,275.63308329)(778.25638184,275.5330896)
\curveto(778.24637913,275.44308348)(778.24137914,275.34308358)(778.24138184,275.2330896)
\curveto(778.25137913,275.13308379)(778.25637912,275.03308389)(778.25638184,274.9330896)
\lineto(778.25638184,271.9630896)
\curveto(778.25637912,271.74308718)(778.25137913,271.50808741)(778.24138184,271.2580896)
\curveto(778.24137914,271.0180879)(778.28637909,270.83308809)(778.37638184,270.7030896)
\curveto(778.42637895,270.6230883)(778.49137889,270.56808835)(778.57138184,270.5380896)
\curveto(778.65137873,270.50808841)(778.74637863,270.48308844)(778.85638184,270.4630896)
\curveto(778.88637849,270.45308847)(778.91637846,270.44808847)(778.94638184,270.4480896)
\curveto(778.98637839,270.45808846)(779.02137836,270.45808846)(779.05138184,270.4480896)
\lineto(779.24638184,270.4480896)
\curveto(779.34637803,270.44808847)(779.43637794,270.43808848)(779.51638184,270.4180896)
\curveto(779.60637777,270.40808851)(779.67137771,270.37308855)(779.71138184,270.3130896)
\curveto(779.73137765,270.28308864)(779.74637763,270.22808869)(779.75638184,270.1480896)
\curveto(779.7763776,270.07808884)(779.78637759,270.00308892)(779.78638184,269.9230896)
\curveto(779.79637758,269.84308908)(779.79637758,269.76308916)(779.78638184,269.6830896)
\curveto(779.7763776,269.61308931)(779.75637762,269.55808936)(779.72638184,269.5180896)
\curveto(779.68637769,269.44808947)(779.61137777,269.39808952)(779.50138184,269.3680896)
\curveto(779.42137796,269.34808957)(779.33137805,269.33808958)(779.23138184,269.3380896)
\curveto(779.13137825,269.34808957)(779.04137834,269.35308957)(778.96138184,269.3530896)
\curveto(778.90137848,269.35308957)(778.84137854,269.34808957)(778.78138184,269.3380896)
\curveto(778.72137866,269.33808958)(778.66637871,269.34308958)(778.61638184,269.3530896)
\lineto(778.43638184,269.3530896)
\curveto(778.38637899,269.36308956)(778.33637904,269.36808955)(778.28638184,269.3680896)
\curveto(778.24637913,269.37808954)(778.20137918,269.38308954)(778.15138184,269.3830896)
\curveto(777.95137943,269.43308949)(777.7763796,269.48808943)(777.62638184,269.5480896)
\curveto(777.48637989,269.60808931)(777.36638001,269.71308921)(777.26638184,269.8630896)
\curveto(777.12638025,270.06308886)(777.04638033,270.31308861)(777.02638184,270.6130896)
\curveto(777.00638037,270.923088)(776.99638038,271.25308767)(776.99638184,271.6030896)
\lineto(776.99638184,275.5330896)
\curveto(776.96638041,275.66308326)(776.93638044,275.75808316)(776.90638184,275.8180896)
\curveto(776.88638049,275.87808304)(776.81638056,275.92808299)(776.69638184,275.9680896)
\curveto(776.65638072,275.97808294)(776.61638076,275.97808294)(776.57638184,275.9680896)
\curveto(776.53638084,275.95808296)(776.49638088,275.96308296)(776.45638184,275.9830896)
\lineto(776.21638184,275.9830896)
\curveto(776.08638129,275.98308294)(775.9763814,275.99308293)(775.88638184,276.0130896)
\curveto(775.80638157,276.04308288)(775.75138163,276.10308282)(775.72138184,276.1930896)
\curveto(775.70138168,276.23308269)(775.68638169,276.27808264)(775.67638184,276.3280896)
\lineto(775.67638184,276.4780896)
\curveto(775.6763817,276.6180823)(775.68638169,276.73308219)(775.70638184,276.8230896)
\curveto(775.72638165,276.923082)(775.78638159,276.99808192)(775.88638184,277.0480896)
\curveto(775.99638138,277.08808183)(776.13638124,277.09808182)(776.30638184,277.0780896)
\curveto(776.48638089,277.05808186)(776.63638074,277.06808185)(776.75638184,277.1080896)
\curveto(776.84638053,277.15808176)(776.91638046,277.22808169)(776.96638184,277.3180896)
\curveto(776.98638039,277.37808154)(776.99638038,277.45308147)(776.99638184,277.5430896)
\lineto(776.99638184,277.7980896)
\lineto(776.99638184,278.7280896)
\lineto(776.99638184,278.9680896)
\curveto(776.99638038,279.05807986)(777.00638037,279.13307979)(777.02638184,279.1930896)
\curveto(777.06638031,279.27307965)(777.14138024,279.33807958)(777.25138184,279.3880896)
\curveto(777.2813801,279.38807953)(777.30638007,279.38807953)(777.32638184,279.3880896)
\curveto(777.35638002,279.39807952)(777.38138,279.40307952)(777.40138184,279.4030896)
}
}
{
\newrgbcolor{curcolor}{0 0 0}
\pscustom[linestyle=none,fillstyle=solid,fillcolor=curcolor]
{
\newpath
\moveto(781.45817871,278.5630896)
\curveto(781.37817759,278.6230803)(781.33317764,278.72808019)(781.32317871,278.8780896)
\lineto(781.32317871,279.3430896)
\lineto(781.32317871,279.5980896)
\curveto(781.32317765,279.68807923)(781.33817763,279.76307916)(781.36817871,279.8230896)
\curveto(781.40817756,279.90307902)(781.48817748,279.96307896)(781.60817871,280.0030896)
\curveto(781.62817734,280.01307891)(781.64817732,280.01307891)(781.66817871,280.0030896)
\curveto(781.69817727,280.00307892)(781.72317725,280.00807891)(781.74317871,280.0180896)
\curveto(781.91317706,280.0180789)(782.0731769,280.01307891)(782.22317871,280.0030896)
\curveto(782.3731766,279.99307893)(782.4731765,279.93307899)(782.52317871,279.8230896)
\curveto(782.55317642,279.76307916)(782.5681764,279.68807923)(782.56817871,279.5980896)
\lineto(782.56817871,279.3430896)
\curveto(782.5681764,279.16307976)(782.56317641,278.99307993)(782.55317871,278.8330896)
\curveto(782.55317642,278.67308025)(782.48817648,278.56808035)(782.35817871,278.5180896)
\curveto(782.30817666,278.49808042)(782.25317672,278.48808043)(782.19317871,278.4880896)
\lineto(782.02817871,278.4880896)
\lineto(781.71317871,278.4880896)
\curveto(781.61317736,278.48808043)(781.52817744,278.51308041)(781.45817871,278.5630896)
\moveto(782.56817871,270.0580896)
\lineto(782.56817871,269.7430896)
\curveto(782.57817639,269.64308928)(782.55817641,269.56308936)(782.50817871,269.5030896)
\curveto(782.47817649,269.44308948)(782.43317654,269.40308952)(782.37317871,269.3830896)
\curveto(782.31317666,269.37308955)(782.24317673,269.35808956)(782.16317871,269.3380896)
\lineto(781.93817871,269.3380896)
\curveto(781.80817716,269.33808958)(781.69317728,269.34308958)(781.59317871,269.3530896)
\curveto(781.50317747,269.37308955)(781.43317754,269.4230895)(781.38317871,269.5030896)
\curveto(781.34317763,269.56308936)(781.32317765,269.63808928)(781.32317871,269.7280896)
\lineto(781.32317871,270.0130896)
\lineto(781.32317871,276.3580896)
\lineto(781.32317871,276.6730896)
\curveto(781.32317765,276.78308214)(781.34817762,276.86808205)(781.39817871,276.9280896)
\curveto(781.42817754,276.97808194)(781.4681775,277.00808191)(781.51817871,277.0180896)
\curveto(781.5681774,277.02808189)(781.62317735,277.04308188)(781.68317871,277.0630896)
\curveto(781.70317727,277.06308186)(781.72317725,277.05808186)(781.74317871,277.0480896)
\curveto(781.7731772,277.04808187)(781.79817717,277.05308187)(781.81817871,277.0630896)
\curveto(781.94817702,277.06308186)(782.07817689,277.05808186)(782.20817871,277.0480896)
\curveto(782.34817662,277.04808187)(782.44317653,277.00808191)(782.49317871,276.9280896)
\curveto(782.54317643,276.86808205)(782.5681764,276.78808213)(782.56817871,276.6880896)
\lineto(782.56817871,276.4030896)
\lineto(782.56817871,270.0580896)
}
}
{
\newrgbcolor{curcolor}{0 0 0}
\pscustom[linestyle=none,fillstyle=solid,fillcolor=curcolor]
{
\newpath
\moveto(791.47302246,267.1330896)
\lineto(791.47302246,266.8030896)
\curveto(791.48301458,266.69309223)(791.4630146,266.60809231)(791.41302246,266.5480896)
\curveto(791.39301467,266.5180924)(791.36801469,266.49309243)(791.33802246,266.4730896)
\lineto(791.24802246,266.4130896)
\curveto(791.21801484,266.40309252)(791.16801489,266.39809252)(791.09802246,266.3980896)
\curveto(791.02801503,266.38809253)(790.95301511,266.38309254)(790.87302246,266.3830896)
\curveto(790.80301526,266.38309254)(790.73301533,266.38809253)(790.66302246,266.3980896)
\curveto(790.59301547,266.39809252)(790.54301552,266.40309252)(790.51302246,266.4130896)
\curveto(790.41301565,266.43309249)(790.34301572,266.47809244)(790.30302246,266.5480896)
\curveto(790.25301581,266.62809229)(790.22801583,266.75309217)(790.22802246,266.9230896)
\lineto(790.22802246,267.3430896)
\lineto(790.22802246,269.1880896)
\lineto(790.22802246,269.5480896)
\curveto(790.23801582,269.68808923)(790.22301584,269.80308912)(790.18302246,269.8930896)
\curveto(790.1630159,269.91308901)(790.14301592,269.92808899)(790.12302246,269.9380896)
\curveto(790.11301595,269.95808896)(790.09801596,269.97808894)(790.07802246,269.9980896)
\curveto(789.97801608,269.99808892)(789.89801616,269.97308895)(789.83802246,269.9230896)
\lineto(789.68802246,269.7730896)
\curveto(789.60801645,269.71308921)(789.52301654,269.65308927)(789.43302246,269.5930896)
\curveto(789.34301672,269.54308938)(789.24301682,269.49308943)(789.13302246,269.4430896)
\curveto(788.98301708,269.38308954)(788.80801725,269.33308959)(788.60802246,269.2930896)
\curveto(788.41801764,269.24308968)(788.21301785,269.21308971)(787.99302246,269.2030896)
\curveto(787.78301828,269.18308974)(787.56801849,269.18308974)(787.34802246,269.2030896)
\curveto(787.13801892,269.21308971)(786.94301912,269.24308968)(786.76302246,269.2930896)
\curveto(786.71301935,269.31308961)(786.6630194,269.32808959)(786.61302246,269.3380896)
\curveto(786.5630195,269.34808957)(786.51301955,269.36308956)(786.46302246,269.3830896)
\curveto(786.37301969,269.4230895)(786.28301978,269.45808946)(786.19302246,269.4880896)
\curveto(786.10301996,269.52808939)(786.01802004,269.57308935)(785.93802246,269.6230896)
\curveto(785.58802047,269.84308908)(785.29302077,270.09308883)(785.05302246,270.3730896)
\curveto(784.82302124,270.66308826)(784.62302144,271.0180879)(784.45302246,271.4380896)
\curveto(784.41302165,271.53808738)(784.37802168,271.64308728)(784.34802246,271.7530896)
\curveto(784.32802173,271.86308706)(784.30302176,271.97308695)(784.27302246,272.0830896)
\curveto(784.2630218,272.10308682)(784.2580218,272.1230868)(784.25802246,272.1430896)
\curveto(784.2580218,272.17308675)(784.25302181,272.20308672)(784.24302246,272.2330896)
\curveto(784.22302184,272.31308661)(784.20802185,272.40308652)(784.19802246,272.5030896)
\curveto(784.19802186,272.60308632)(784.18802187,272.69808622)(784.16802246,272.7880896)
\lineto(784.16802246,273.0430896)
\curveto(784.14802191,273.09308583)(784.13802192,273.15808576)(784.13802246,273.2380896)
\curveto(784.13802192,273.3180856)(784.14802191,273.38308554)(784.16802246,273.4330896)
\lineto(784.16802246,273.5980896)
\curveto(784.16802189,273.65808526)(784.17302189,273.7180852)(784.18302246,273.7780896)
\curveto(784.19302187,273.8180851)(784.19302187,273.85808506)(784.18302246,273.8980896)
\curveto(784.18302188,273.93808498)(784.18802187,273.98308494)(784.19802246,274.0330896)
\curveto(784.22802183,274.14308478)(784.24802181,274.24808467)(784.25802246,274.3480896)
\curveto(784.27802178,274.45808446)(784.30302176,274.56308436)(784.33302246,274.6630896)
\curveto(784.37302169,274.79308413)(784.41302165,274.91308401)(784.45302246,275.0230896)
\curveto(784.49302157,275.14308378)(784.53802152,275.25808366)(784.58802246,275.3680896)
\curveto(784.6580214,275.50808341)(784.73302133,275.63808328)(784.81302246,275.7580896)
\curveto(784.90302116,275.88808303)(784.99302107,276.01308291)(785.08302246,276.1330896)
\curveto(785.09302097,276.13308279)(785.10802095,276.14308278)(785.12802246,276.1630896)
\curveto(785.17802088,276.24308268)(785.25302081,276.3230826)(785.35302246,276.4030896)
\curveto(785.3630207,276.41308251)(785.36802069,276.4230825)(785.36802246,276.4330896)
\curveto(785.37802068,276.44308248)(785.39302067,276.45308247)(785.41302246,276.4630896)
\curveto(785.45302061,276.49308243)(785.48802057,276.5230824)(785.51802246,276.5530896)
\curveto(785.5580205,276.59308233)(785.60302046,276.62808229)(785.65302246,276.6580896)
\curveto(785.79302027,276.76808215)(785.94802011,276.85808206)(786.11802246,276.9280896)
\curveto(786.28801977,276.99808192)(786.46801959,277.06308186)(786.65802246,277.1230896)
\curveto(786.7580193,277.16308176)(786.8630192,277.18808173)(786.97302246,277.1980896)
\curveto(787.08301898,277.20808171)(787.19301887,277.2230817)(787.30302246,277.2430896)
\curveto(787.34301872,277.25308167)(787.39801866,277.25308167)(787.46802246,277.2430896)
\curveto(787.54801851,277.23308169)(787.59801846,277.23808168)(787.61802246,277.2580896)
\curveto(787.94801811,277.25808166)(788.26801779,277.2180817)(788.57802246,277.1380896)
\curveto(788.88801717,277.05808186)(789.14301692,276.95808196)(789.34302246,276.8380896)
\lineto(789.52302246,276.7180896)
\curveto(789.58301648,276.67808224)(789.64301642,276.63308229)(789.70302246,276.5830896)
\lineto(789.85302246,276.4630896)
\curveto(789.90301616,276.4230825)(789.97801608,276.40308252)(790.07802246,276.4030896)
\curveto(790.09801596,276.4230825)(790.11801594,276.43808248)(790.13802246,276.4480896)
\curveto(790.1580159,276.46808245)(790.17301589,276.49308243)(790.18302246,276.5230896)
\curveto(790.21301585,276.59308233)(790.22801583,276.66808225)(790.22802246,276.7480896)
\curveto(790.23801582,276.82808209)(790.26801579,276.89308203)(790.31802246,276.9430896)
\curveto(790.34801571,276.98308194)(790.40801565,277.01308191)(790.49802246,277.0330896)
\curveto(790.59801546,277.06308186)(790.70301536,277.07808184)(790.81302246,277.0780896)
\curveto(790.92301514,277.08808183)(791.02801503,277.08308184)(791.12802246,277.0630896)
\curveto(791.22801483,277.04308188)(791.30301476,277.0180819)(791.35302246,276.9880896)
\curveto(791.42301464,276.93808198)(791.4580146,276.85308207)(791.45802246,276.7330896)
\curveto(791.46801459,276.61308231)(791.47301459,276.49308243)(791.47302246,276.3730896)
\lineto(791.47302246,267.1330896)
\moveto(790.25802246,272.9980896)
\curveto(790.26801579,273.04808587)(790.27301579,273.1180858)(790.27302246,273.2080896)
\curveto(790.28301578,273.29808562)(790.27801578,273.36808555)(790.25802246,273.4180896)
\lineto(790.25802246,273.6280896)
\lineto(790.19802246,273.9280896)
\curveto(790.18801587,274.02808489)(790.17301589,274.1180848)(790.15302246,274.1980896)
\curveto(790.13301593,274.27808464)(790.11301595,274.34808457)(790.09302246,274.4080896)
\curveto(790.08301598,274.47808444)(790.063016,274.54808437)(790.03302246,274.6180896)
\curveto(789.92301614,274.88808403)(789.74801631,275.15308377)(789.50802246,275.4130896)
\curveto(789.26801679,275.67308325)(789.03801702,275.85308307)(788.81802246,275.9530896)
\curveto(788.73801732,275.99308293)(788.65301741,276.0230829)(788.56302246,276.0430896)
\curveto(788.48301758,276.06308286)(788.39801766,276.08808283)(788.30802246,276.1180896)
\curveto(788.20801785,276.13808278)(788.09801796,276.14808277)(787.97802246,276.1480896)
\lineto(787.63302246,276.1480896)
\lineto(787.48302246,276.1180896)
\lineto(787.34802246,276.1180896)
\lineto(787.10802246,276.0580896)
\curveto(787.02801903,276.03808288)(786.95301911,276.00808291)(786.88302246,275.9680896)
\curveto(786.5630195,275.82808309)(786.30301976,275.62808329)(786.10302246,275.3680896)
\curveto(785.91302015,275.10808381)(785.7630203,274.80308412)(785.65302246,274.4530896)
\curveto(785.61302045,274.34308458)(785.58302048,274.2230847)(785.56302246,274.0930896)
\curveto(785.55302051,273.97308495)(785.53302053,273.84808507)(785.50302246,273.7180896)
\lineto(785.50302246,273.5830896)
\curveto(785.50302056,273.54308538)(785.49802056,273.49808542)(785.48802246,273.4480896)
\curveto(785.47802058,273.40808551)(785.47302059,273.36308556)(785.47302246,273.3130896)
\curveto(785.48302058,273.26308566)(785.48802057,273.21308571)(785.48802246,273.1630896)
\lineto(785.48802246,272.8630896)
\curveto(785.48802057,272.77308615)(785.49802056,272.68808623)(785.51802246,272.6080896)
\curveto(785.52802053,272.57808634)(785.53302053,272.53308639)(785.53302246,272.4730896)
\curveto(785.55302051,272.40308652)(785.56802049,272.33308659)(785.57802246,272.2630896)
\lineto(785.63802246,272.0530896)
\curveto(785.72802033,271.76308716)(785.84802021,271.49808742)(785.99802246,271.2580896)
\curveto(786.14801991,271.02808789)(786.33801972,270.83308809)(786.56802246,270.6730896)
\lineto(786.65802246,270.6130896)
\curveto(786.69801936,270.59308833)(786.73301933,270.57308835)(786.76302246,270.5530896)
\curveto(786.8630192,270.49308843)(786.96801909,270.44308848)(787.07802246,270.4030896)
\lineto(787.43802246,270.3130896)
\curveto(787.48801857,270.29308863)(787.52801853,270.28308864)(787.55802246,270.2830896)
\curveto(787.58801847,270.29308863)(787.62801843,270.29308863)(787.67802246,270.2830896)
\curveto(787.71801834,270.27308865)(787.76801829,270.26308866)(787.82802246,270.2530896)
\curveto(787.88801817,270.25308867)(787.94301812,270.26308866)(787.99302246,270.2830896)
\lineto(788.11302246,270.2830896)
\curveto(788.14301792,270.29308863)(788.17301789,270.29308863)(788.20302246,270.2830896)
\curveto(788.23301783,270.28308864)(788.2630178,270.28808863)(788.29302246,270.2980896)
\curveto(788.37301769,270.3180886)(788.45301761,270.33308859)(788.53302246,270.3430896)
\curveto(788.61301745,270.36308856)(788.68801737,270.38808853)(788.75802246,270.4180896)
\curveto(789.06801699,270.54808837)(789.32301674,270.7230882)(789.52302246,270.9430896)
\curveto(789.72301634,271.17308775)(789.88801617,271.43808748)(790.01802246,271.7380896)
\curveto(790.06801599,271.84808707)(790.10301596,271.95808696)(790.12302246,272.0680896)
\curveto(790.14301592,272.17808674)(790.16801589,272.29308663)(790.19802246,272.4130896)
\curveto(790.21801584,272.45308647)(790.22801583,272.49308643)(790.22802246,272.5330896)
\curveto(790.22801583,272.57308635)(790.23301583,272.61308631)(790.24302246,272.6530896)
\curveto(790.25301581,272.70308622)(790.25301581,272.75808616)(790.24302246,272.8180896)
\curveto(790.24301582,272.87808604)(790.24801581,272.93808598)(790.25802246,272.9980896)
}
}
{
\newrgbcolor{curcolor}{0 0 0}
\pscustom[linestyle=none,fillstyle=solid,fillcolor=curcolor]
{
\newpath
\moveto(793.88427246,277.0630896)
\lineto(794.31927246,277.0630896)
\curveto(794.4692705,277.06308186)(794.57427039,277.0230819)(794.63427246,276.9430896)
\curveto(794.68427028,276.86308206)(794.70927026,276.76308216)(794.70927246,276.6430896)
\curveto(794.71927025,276.5230824)(794.72427024,276.40308252)(794.72427246,276.2830896)
\lineto(794.72427246,274.8580896)
\lineto(794.72427246,272.5930896)
\lineto(794.72427246,271.9030896)
\curveto(794.72427024,271.67308725)(794.74927022,271.47308745)(794.79927246,271.3030896)
\curveto(794.95927001,270.85308807)(795.25926971,270.53808838)(795.69927246,270.3580896)
\curveto(795.91926905,270.26808865)(796.18426878,270.23308869)(796.49427246,270.2530896)
\curveto(796.80426816,270.28308864)(797.05426791,270.33808858)(797.24427246,270.4180896)
\curveto(797.57426739,270.55808836)(797.83426713,270.73308819)(798.02427246,270.9430896)
\curveto(798.22426674,271.16308776)(798.37926659,271.44808747)(798.48927246,271.7980896)
\curveto(798.51926645,271.87808704)(798.53926643,271.95808696)(798.54927246,272.0380896)
\curveto(798.55926641,272.1180868)(798.57426639,272.20308672)(798.59427246,272.2930896)
\curveto(798.60426636,272.34308658)(798.60426636,272.38808653)(798.59427246,272.4280896)
\curveto(798.59426637,272.46808645)(798.60426636,272.51308641)(798.62427246,272.5630896)
\lineto(798.62427246,272.8780896)
\curveto(798.64426632,272.95808596)(798.64926632,273.04808587)(798.63927246,273.1480896)
\curveto(798.62926634,273.25808566)(798.62426634,273.35808556)(798.62427246,273.4480896)
\lineto(798.62427246,274.6180896)
\lineto(798.62427246,276.2080896)
\curveto(798.62426634,276.32808259)(798.61926635,276.45308247)(798.60927246,276.5830896)
\curveto(798.60926636,276.7230822)(798.63426633,276.83308209)(798.68427246,276.9130896)
\curveto(798.72426624,276.96308196)(798.7692662,276.99308193)(798.81927246,277.0030896)
\curveto(798.87926609,277.0230819)(798.94926602,277.04308188)(799.02927246,277.0630896)
\lineto(799.25427246,277.0630896)
\curveto(799.37426559,277.06308186)(799.47926549,277.05808186)(799.56927246,277.0480896)
\curveto(799.6692653,277.03808188)(799.74426522,276.99308193)(799.79427246,276.9130896)
\curveto(799.84426512,276.86308206)(799.8692651,276.78808213)(799.86927246,276.6880896)
\lineto(799.86927246,276.4030896)
\lineto(799.86927246,275.3830896)
\lineto(799.86927246,271.3480896)
\lineto(799.86927246,269.9980896)
\curveto(799.8692651,269.87808904)(799.8642651,269.76308916)(799.85427246,269.6530896)
\curveto(799.85426511,269.55308937)(799.81926515,269.47808944)(799.74927246,269.4280896)
\curveto(799.70926526,269.39808952)(799.64926532,269.37308955)(799.56927246,269.3530896)
\curveto(799.48926548,269.34308958)(799.39926557,269.33308959)(799.29927246,269.3230896)
\curveto(799.20926576,269.3230896)(799.11926585,269.32808959)(799.02927246,269.3380896)
\curveto(798.94926602,269.34808957)(798.88926608,269.36808955)(798.84927246,269.3980896)
\curveto(798.79926617,269.43808948)(798.75426621,269.50308942)(798.71427246,269.5930896)
\curveto(798.70426626,269.63308929)(798.69426627,269.68808923)(798.68427246,269.7580896)
\curveto(798.68426628,269.82808909)(798.67926629,269.89308903)(798.66927246,269.9530896)
\curveto(798.65926631,270.0230889)(798.63926633,270.07808884)(798.60927246,270.1180896)
\curveto(798.57926639,270.15808876)(798.53426643,270.17308875)(798.47427246,270.1630896)
\curveto(798.39426657,270.14308878)(798.31426665,270.08308884)(798.23427246,269.9830896)
\curveto(798.15426681,269.89308903)(798.07926689,269.8230891)(798.00927246,269.7730896)
\curveto(797.78926718,269.61308931)(797.53926743,269.47308945)(797.25927246,269.3530896)
\curveto(797.14926782,269.30308962)(797.03426793,269.27308965)(796.91427246,269.2630896)
\curveto(796.80426816,269.24308968)(796.68926828,269.2180897)(796.56927246,269.1880896)
\curveto(796.51926845,269.17808974)(796.4642685,269.17808974)(796.40427246,269.1880896)
\curveto(796.35426861,269.19808972)(796.30426866,269.19308973)(796.25427246,269.1730896)
\curveto(796.15426881,269.15308977)(796.0642689,269.15308977)(795.98427246,269.1730896)
\lineto(795.83427246,269.1730896)
\curveto(795.78426918,269.19308973)(795.72426924,269.20308972)(795.65427246,269.2030896)
\curveto(795.59426937,269.20308972)(795.53926943,269.20808971)(795.48927246,269.2180896)
\curveto(795.44926952,269.23808968)(795.40926956,269.24808967)(795.36927246,269.2480896)
\curveto(795.33926963,269.23808968)(795.29926967,269.24308968)(795.24927246,269.2630896)
\lineto(795.00927246,269.3230896)
\curveto(794.93927003,269.34308958)(794.8642701,269.37308955)(794.78427246,269.4130896)
\curveto(794.52427044,269.5230894)(794.30427066,269.66808925)(794.12427246,269.8480896)
\curveto(793.95427101,270.03808888)(793.81427115,270.26308866)(793.70427246,270.5230896)
\curveto(793.6642713,270.61308831)(793.63427133,270.70308822)(793.61427246,270.7930896)
\lineto(793.55427246,271.0930896)
\curveto(793.53427143,271.15308777)(793.52427144,271.20808771)(793.52427246,271.2580896)
\curveto(793.53427143,271.3180876)(793.52927144,271.38308754)(793.50927246,271.4530896)
\curveto(793.49927147,271.47308745)(793.49427147,271.49808742)(793.49427246,271.5280896)
\curveto(793.49427147,271.56808735)(793.48927148,271.60308732)(793.47927246,271.6330896)
\lineto(793.47927246,271.7830896)
\curveto(793.4692715,271.8230871)(793.4642715,271.86808705)(793.46427246,271.9180896)
\curveto(793.47427149,271.97808694)(793.47927149,272.03308689)(793.47927246,272.0830896)
\lineto(793.47927246,272.6830896)
\lineto(793.47927246,275.4430896)
\lineto(793.47927246,276.4030896)
\lineto(793.47927246,276.6730896)
\curveto(793.47927149,276.76308216)(793.49927147,276.83808208)(793.53927246,276.8980896)
\curveto(793.57927139,276.96808195)(793.65427131,277.0180819)(793.76427246,277.0480896)
\curveto(793.78427118,277.05808186)(793.80427116,277.05808186)(793.82427246,277.0480896)
\curveto(793.84427112,277.04808187)(793.8642711,277.05308187)(793.88427246,277.0630896)
}
}
{
\newrgbcolor{curcolor}{0 0 0}
\pscustom[linestyle=none,fillstyle=solid,fillcolor=curcolor]
{
\newpath
\moveto(808.51888184,273.5080896)
\curveto(808.53887415,273.40808551)(808.53887415,273.29308563)(808.51888184,273.1630896)
\curveto(808.50887418,273.04308588)(808.47887421,272.95808596)(808.42888184,272.9080896)
\curveto(808.37887431,272.86808605)(808.30387439,272.83808608)(808.20388184,272.8180896)
\curveto(808.11387458,272.80808611)(808.00887468,272.80308612)(807.88888184,272.8030896)
\lineto(807.52888184,272.8030896)
\curveto(807.40887528,272.81308611)(807.30387539,272.8180861)(807.21388184,272.8180896)
\lineto(803.37388184,272.8180896)
\curveto(803.2938794,272.8180861)(803.21387948,272.81308611)(803.13388184,272.8030896)
\curveto(803.05387964,272.80308612)(802.9888797,272.78808613)(802.93888184,272.7580896)
\curveto(802.89887979,272.73808618)(802.85887983,272.69808622)(802.81888184,272.6380896)
\curveto(802.79887989,272.60808631)(802.77887991,272.56308636)(802.75888184,272.5030896)
\curveto(802.73887995,272.45308647)(802.73887995,272.40308652)(802.75888184,272.3530896)
\curveto(802.76887992,272.30308662)(802.77387992,272.25808666)(802.77388184,272.2180896)
\curveto(802.77387992,272.17808674)(802.77887991,272.13808678)(802.78888184,272.0980896)
\curveto(802.80887988,272.0180869)(802.82887986,271.93308699)(802.84888184,271.8430896)
\curveto(802.86887982,271.76308716)(802.89887979,271.68308724)(802.93888184,271.6030896)
\curveto(803.16887952,271.06308786)(803.54887914,270.67808824)(804.07888184,270.4480896)
\curveto(804.13887855,270.4180885)(804.20387849,270.39308853)(804.27388184,270.3730896)
\lineto(804.48388184,270.3130896)
\curveto(804.51387818,270.30308862)(804.56387813,270.29808862)(804.63388184,270.2980896)
\curveto(804.77387792,270.25808866)(804.95887773,270.23808868)(805.18888184,270.2380896)
\curveto(805.41887727,270.23808868)(805.60387709,270.25808866)(805.74388184,270.2980896)
\curveto(805.88387681,270.33808858)(806.00887668,270.37808854)(806.11888184,270.4180896)
\curveto(806.23887645,270.46808845)(806.34887634,270.52808839)(806.44888184,270.5980896)
\curveto(806.55887613,270.66808825)(806.65387604,270.74808817)(806.73388184,270.8380896)
\curveto(806.81387588,270.93808798)(806.88387581,271.04308788)(806.94388184,271.1530896)
\curveto(807.00387569,271.25308767)(807.05387564,271.35808756)(807.09388184,271.4680896)
\curveto(807.14387555,271.57808734)(807.22387547,271.65808726)(807.33388184,271.7080896)
\curveto(807.37387532,271.72808719)(807.43887525,271.74308718)(807.52888184,271.7530896)
\curveto(807.61887507,271.76308716)(807.70887498,271.76308716)(807.79888184,271.7530896)
\curveto(807.8888748,271.75308717)(807.97387472,271.74808717)(808.05388184,271.7380896)
\curveto(808.13387456,271.72808719)(808.1888745,271.70808721)(808.21888184,271.6780896)
\curveto(808.31887437,271.60808731)(808.34387435,271.49308743)(808.29388184,271.3330896)
\curveto(808.21387448,271.06308786)(808.10887458,270.8230881)(807.97888184,270.6130896)
\curveto(807.77887491,270.29308863)(807.54887514,270.02808889)(807.28888184,269.8180896)
\curveto(807.03887565,269.6180893)(806.71887597,269.45308947)(806.32888184,269.3230896)
\curveto(806.22887646,269.28308964)(806.12887656,269.25808966)(806.02888184,269.2480896)
\curveto(805.92887676,269.22808969)(805.82387687,269.20808971)(805.71388184,269.1880896)
\curveto(805.66387703,269.17808974)(805.61387708,269.17308975)(805.56388184,269.1730896)
\curveto(805.52387717,269.17308975)(805.47887721,269.16808975)(805.42888184,269.1580896)
\lineto(805.27888184,269.1580896)
\curveto(805.22887746,269.14808977)(805.16887752,269.14308978)(805.09888184,269.1430896)
\curveto(805.03887765,269.14308978)(804.9888777,269.14808977)(804.94888184,269.1580896)
\lineto(804.81388184,269.1580896)
\curveto(804.76387793,269.16808975)(804.71887797,269.17308975)(804.67888184,269.1730896)
\curveto(804.63887805,269.17308975)(804.59887809,269.17808974)(804.55888184,269.1880896)
\curveto(804.50887818,269.19808972)(804.45387824,269.20808971)(804.39388184,269.2180896)
\curveto(804.33387836,269.2180897)(804.27887841,269.2230897)(804.22888184,269.2330896)
\curveto(804.13887855,269.25308967)(804.04887864,269.27808964)(803.95888184,269.3080896)
\curveto(803.86887882,269.32808959)(803.78387891,269.35308957)(803.70388184,269.3830896)
\curveto(803.66387903,269.40308952)(803.62887906,269.41308951)(803.59888184,269.4130896)
\curveto(803.56887912,269.4230895)(803.53387916,269.43808948)(803.49388184,269.4580896)
\curveto(803.34387935,269.52808939)(803.18387951,269.61308931)(803.01388184,269.7130896)
\curveto(802.72387997,269.90308902)(802.47388022,270.13308879)(802.26388184,270.4030896)
\curveto(802.06388063,270.68308824)(801.8938808,270.99308793)(801.75388184,271.3330896)
\curveto(801.70388099,271.44308748)(801.66388103,271.55808736)(801.63388184,271.6780896)
\curveto(801.61388108,271.79808712)(801.58388111,271.918087)(801.54388184,272.0380896)
\curveto(801.53388116,272.07808684)(801.52888116,272.11308681)(801.52888184,272.1430896)
\curveto(801.52888116,272.17308675)(801.52388117,272.21308671)(801.51388184,272.2630896)
\curveto(801.4938812,272.34308658)(801.47888121,272.42808649)(801.46888184,272.5180896)
\curveto(801.45888123,272.60808631)(801.44388125,272.69808622)(801.42388184,272.7880896)
\lineto(801.42388184,272.9980896)
\curveto(801.41388128,273.03808588)(801.40388129,273.09308583)(801.39388184,273.1630896)
\curveto(801.3938813,273.24308568)(801.39888129,273.30808561)(801.40888184,273.3580896)
\lineto(801.40888184,273.5230896)
\curveto(801.42888126,273.57308535)(801.43388126,273.6230853)(801.42388184,273.6730896)
\curveto(801.42388127,273.73308519)(801.42888126,273.78808513)(801.43888184,273.8380896)
\curveto(801.47888121,273.99808492)(801.50888118,274.15808476)(801.52888184,274.3180896)
\curveto(801.55888113,274.47808444)(801.60388109,274.62808429)(801.66388184,274.7680896)
\curveto(801.71388098,274.87808404)(801.75888093,274.98808393)(801.79888184,275.0980896)
\curveto(801.84888084,275.2180837)(801.90388079,275.33308359)(801.96388184,275.4430896)
\curveto(802.18388051,275.79308313)(802.43388026,276.09308283)(802.71388184,276.3430896)
\curveto(802.9938797,276.60308232)(803.33887935,276.8180821)(803.74888184,276.9880896)
\curveto(803.86887882,277.03808188)(803.9888787,277.07308185)(804.10888184,277.0930896)
\curveto(804.23887845,277.1230818)(804.37387832,277.15308177)(804.51388184,277.1830896)
\curveto(804.56387813,277.19308173)(804.60887808,277.19808172)(804.64888184,277.1980896)
\curveto(804.688878,277.20808171)(804.73387796,277.21308171)(804.78388184,277.2130896)
\curveto(804.80387789,277.2230817)(804.82887786,277.2230817)(804.85888184,277.2130896)
\curveto(804.8888778,277.20308172)(804.91387778,277.20808171)(804.93388184,277.2280896)
\curveto(805.35387734,277.23808168)(805.71887697,277.19308173)(806.02888184,277.0930896)
\curveto(806.33887635,277.00308192)(806.61887607,276.87808204)(806.86888184,276.7180896)
\curveto(806.91887577,276.69808222)(806.95887573,276.66808225)(806.98888184,276.6280896)
\curveto(807.01887567,276.59808232)(807.05387564,276.57308235)(807.09388184,276.5530896)
\curveto(807.17387552,276.49308243)(807.25387544,276.4230825)(807.33388184,276.3430896)
\curveto(807.42387527,276.26308266)(807.49887519,276.18308274)(807.55888184,276.1030896)
\curveto(807.71887497,275.89308303)(807.85387484,275.69308323)(807.96388184,275.5030896)
\curveto(808.03387466,275.39308353)(808.0888746,275.27308365)(808.12888184,275.1430896)
\curveto(808.16887452,275.01308391)(808.21387448,274.88308404)(808.26388184,274.7530896)
\curveto(808.31387438,274.6230843)(808.34887434,274.48808443)(808.36888184,274.3480896)
\curveto(808.39887429,274.20808471)(808.43387426,274.06808485)(808.47388184,273.9280896)
\curveto(808.48387421,273.85808506)(808.4888742,273.78808513)(808.48888184,273.7180896)
\lineto(808.51888184,273.5080896)
\moveto(807.06388184,274.0180896)
\curveto(807.0938756,274.05808486)(807.11887557,274.10808481)(807.13888184,274.1680896)
\curveto(807.15887553,274.23808468)(807.15887553,274.30808461)(807.13888184,274.3780896)
\curveto(807.07887561,274.59808432)(806.9938757,274.80308412)(806.88388184,274.9930896)
\curveto(806.74387595,275.2230837)(806.5888761,275.4180835)(806.41888184,275.5780896)
\curveto(806.24887644,275.73808318)(806.02887666,275.87308305)(805.75888184,275.9830896)
\curveto(805.688877,276.00308292)(805.61887707,276.0180829)(805.54888184,276.0280896)
\curveto(805.47887721,276.04808287)(805.40387729,276.06808285)(805.32388184,276.0880896)
\curveto(805.24387745,276.10808281)(805.15887753,276.1180828)(805.06888184,276.1180896)
\lineto(804.81388184,276.1180896)
\curveto(804.78387791,276.09808282)(804.74887794,276.08808283)(804.70888184,276.0880896)
\curveto(804.66887802,276.09808282)(804.63387806,276.09808282)(804.60388184,276.0880896)
\lineto(804.36388184,276.0280896)
\curveto(804.2938784,276.0180829)(804.22387847,276.00308292)(804.15388184,275.9830896)
\curveto(803.86387883,275.86308306)(803.62887906,275.71308321)(803.44888184,275.5330896)
\curveto(803.27887941,275.35308357)(803.12387957,275.12808379)(802.98388184,274.8580896)
\curveto(802.95387974,274.80808411)(802.92387977,274.74308418)(802.89388184,274.6630896)
\curveto(802.86387983,274.59308433)(802.83887985,274.51308441)(802.81888184,274.4230896)
\curveto(802.79887989,274.33308459)(802.7938799,274.24808467)(802.80388184,274.1680896)
\curveto(802.81387988,274.08808483)(802.84887984,274.02808489)(802.90888184,273.9880896)
\curveto(802.9888797,273.92808499)(803.12387957,273.89808502)(803.31388184,273.8980896)
\curveto(803.51387918,273.90808501)(803.68387901,273.91308501)(803.82388184,273.9130896)
\lineto(806.10388184,273.9130896)
\curveto(806.25387644,273.91308501)(806.43387626,273.90808501)(806.64388184,273.8980896)
\curveto(806.85387584,273.89808502)(806.9938757,273.93808498)(807.06388184,274.0180896)
}
}
{
\newrgbcolor{curcolor}{0 0 0}
\pscustom[linestyle=none,fillstyle=solid,fillcolor=curcolor]
{
\newpath
\moveto(810.77052246,279.4030896)
\curveto(810.92052045,279.40307952)(811.0705203,279.39807952)(811.22052246,279.3880896)
\curveto(811.37052,279.38807953)(811.4755199,279.34807957)(811.53552246,279.2680896)
\curveto(811.58551979,279.20807971)(811.61051976,279.1230798)(811.61052246,279.0130896)
\curveto(811.62051975,278.91308001)(811.62551975,278.80808011)(811.62552246,278.6980896)
\lineto(811.62552246,277.8280896)
\curveto(811.62551975,277.74808117)(811.62051975,277.66308126)(811.61052246,277.5730896)
\curveto(811.61051976,277.49308143)(811.62051975,277.4230815)(811.64052246,277.3630896)
\curveto(811.68051969,277.2230817)(811.7705196,277.13308179)(811.91052246,277.0930896)
\curveto(811.96051941,277.08308184)(812.00551937,277.07808184)(812.04552246,277.0780896)
\lineto(812.19552246,277.0780896)
\lineto(812.60052246,277.0780896)
\curveto(812.76051861,277.08808183)(812.8755185,277.07808184)(812.94552246,277.0480896)
\curveto(813.03551834,276.98808193)(813.09551828,276.92808199)(813.12552246,276.8680896)
\curveto(813.14551823,276.82808209)(813.15551822,276.78308214)(813.15552246,276.7330896)
\lineto(813.15552246,276.5830896)
\curveto(813.15551822,276.47308245)(813.15051822,276.36808255)(813.14052246,276.2680896)
\curveto(813.13051824,276.17808274)(813.09551828,276.10808281)(813.03552246,276.0580896)
\curveto(812.9755184,276.00808291)(812.89051848,275.97808294)(812.78052246,275.9680896)
\lineto(812.45052246,275.9680896)
\curveto(812.34051903,275.97808294)(812.23051914,275.98308294)(812.12052246,275.9830896)
\curveto(812.01051936,275.98308294)(811.91551946,275.96808295)(811.83552246,275.9380896)
\curveto(811.76551961,275.90808301)(811.71551966,275.85808306)(811.68552246,275.7880896)
\curveto(811.65551972,275.7180832)(811.63551974,275.63308329)(811.62552246,275.5330896)
\curveto(811.61551976,275.44308348)(811.61051976,275.34308358)(811.61052246,275.2330896)
\curveto(811.62051975,275.13308379)(811.62551975,275.03308389)(811.62552246,274.9330896)
\lineto(811.62552246,271.9630896)
\curveto(811.62551975,271.74308718)(811.62051975,271.50808741)(811.61052246,271.2580896)
\curveto(811.61051976,271.0180879)(811.65551972,270.83308809)(811.74552246,270.7030896)
\curveto(811.79551958,270.6230883)(811.86051951,270.56808835)(811.94052246,270.5380896)
\curveto(812.02051935,270.50808841)(812.11551926,270.48308844)(812.22552246,270.4630896)
\curveto(812.25551912,270.45308847)(812.28551909,270.44808847)(812.31552246,270.4480896)
\curveto(812.35551902,270.45808846)(812.39051898,270.45808846)(812.42052246,270.4480896)
\lineto(812.61552246,270.4480896)
\curveto(812.71551866,270.44808847)(812.80551857,270.43808848)(812.88552246,270.4180896)
\curveto(812.9755184,270.40808851)(813.04051833,270.37308855)(813.08052246,270.3130896)
\curveto(813.10051827,270.28308864)(813.11551826,270.22808869)(813.12552246,270.1480896)
\curveto(813.14551823,270.07808884)(813.15551822,270.00308892)(813.15552246,269.9230896)
\curveto(813.16551821,269.84308908)(813.16551821,269.76308916)(813.15552246,269.6830896)
\curveto(813.14551823,269.61308931)(813.12551825,269.55808936)(813.09552246,269.5180896)
\curveto(813.05551832,269.44808947)(812.98051839,269.39808952)(812.87052246,269.3680896)
\curveto(812.79051858,269.34808957)(812.70051867,269.33808958)(812.60052246,269.3380896)
\curveto(812.50051887,269.34808957)(812.41051896,269.35308957)(812.33052246,269.3530896)
\curveto(812.2705191,269.35308957)(812.21051916,269.34808957)(812.15052246,269.3380896)
\curveto(812.09051928,269.33808958)(812.03551934,269.34308958)(811.98552246,269.3530896)
\lineto(811.80552246,269.3530896)
\curveto(811.75551962,269.36308956)(811.70551967,269.36808955)(811.65552246,269.3680896)
\curveto(811.61551976,269.37808954)(811.5705198,269.38308954)(811.52052246,269.3830896)
\curveto(811.32052005,269.43308949)(811.14552023,269.48808943)(810.99552246,269.5480896)
\curveto(810.85552052,269.60808931)(810.73552064,269.71308921)(810.63552246,269.8630896)
\curveto(810.49552088,270.06308886)(810.41552096,270.31308861)(810.39552246,270.6130896)
\curveto(810.375521,270.923088)(810.36552101,271.25308767)(810.36552246,271.6030896)
\lineto(810.36552246,275.5330896)
\curveto(810.33552104,275.66308326)(810.30552107,275.75808316)(810.27552246,275.8180896)
\curveto(810.25552112,275.87808304)(810.18552119,275.92808299)(810.06552246,275.9680896)
\curveto(810.02552135,275.97808294)(809.98552139,275.97808294)(809.94552246,275.9680896)
\curveto(809.90552147,275.95808296)(809.86552151,275.96308296)(809.82552246,275.9830896)
\lineto(809.58552246,275.9830896)
\curveto(809.45552192,275.98308294)(809.34552203,275.99308293)(809.25552246,276.0130896)
\curveto(809.1755222,276.04308288)(809.12052225,276.10308282)(809.09052246,276.1930896)
\curveto(809.0705223,276.23308269)(809.05552232,276.27808264)(809.04552246,276.3280896)
\lineto(809.04552246,276.4780896)
\curveto(809.04552233,276.6180823)(809.05552232,276.73308219)(809.07552246,276.8230896)
\curveto(809.09552228,276.923082)(809.15552222,276.99808192)(809.25552246,277.0480896)
\curveto(809.36552201,277.08808183)(809.50552187,277.09808182)(809.67552246,277.0780896)
\curveto(809.85552152,277.05808186)(810.00552137,277.06808185)(810.12552246,277.1080896)
\curveto(810.21552116,277.15808176)(810.28552109,277.22808169)(810.33552246,277.3180896)
\curveto(810.35552102,277.37808154)(810.36552101,277.45308147)(810.36552246,277.5430896)
\lineto(810.36552246,277.7980896)
\lineto(810.36552246,278.7280896)
\lineto(810.36552246,278.9680896)
\curveto(810.36552101,279.05807986)(810.375521,279.13307979)(810.39552246,279.1930896)
\curveto(810.43552094,279.27307965)(810.51052086,279.33807958)(810.62052246,279.3880896)
\curveto(810.65052072,279.38807953)(810.6755207,279.38807953)(810.69552246,279.3880896)
\curveto(810.72552065,279.39807952)(810.75052062,279.40307952)(810.77052246,279.4030896)
}
}
{
\newrgbcolor{curcolor}{0 0 0}
\pscustom[linestyle=none,fillstyle=solid,fillcolor=curcolor]
{
\newpath
\moveto(821.42731934,269.8930896)
\curveto(821.45731151,269.73308919)(821.44231152,269.59808932)(821.38231934,269.4880896)
\curveto(821.32231164,269.38808953)(821.24231172,269.31308961)(821.14231934,269.2630896)
\curveto(821.09231187,269.24308968)(821.03731193,269.23308969)(820.97731934,269.2330896)
\curveto(820.92731204,269.23308969)(820.87231209,269.2230897)(820.81231934,269.2030896)
\curveto(820.59231237,269.15308977)(820.37231259,269.16808975)(820.15231934,269.2480896)
\curveto(819.94231302,269.3180896)(819.79731317,269.40808951)(819.71731934,269.5180896)
\curveto(819.6673133,269.58808933)(819.62231334,269.66808925)(819.58231934,269.7580896)
\curveto(819.54231342,269.85808906)(819.49231347,269.93808898)(819.43231934,269.9980896)
\curveto(819.41231355,270.0180889)(819.38731358,270.03808888)(819.35731934,270.0580896)
\curveto(819.33731363,270.07808884)(819.30731366,270.08308884)(819.26731934,270.0730896)
\curveto(819.15731381,270.04308888)(819.05231391,269.98808893)(818.95231934,269.9080896)
\curveto(818.8623141,269.82808909)(818.77231419,269.75808916)(818.68231934,269.6980896)
\curveto(818.55231441,269.6180893)(818.41231455,269.54308938)(818.26231934,269.4730896)
\curveto(818.11231485,269.41308951)(817.95231501,269.35808956)(817.78231934,269.3080896)
\curveto(817.68231528,269.27808964)(817.57231539,269.25808966)(817.45231934,269.2480896)
\curveto(817.34231562,269.23808968)(817.23231573,269.2230897)(817.12231934,269.2030896)
\curveto(817.07231589,269.19308973)(817.02731594,269.18808973)(816.98731934,269.1880896)
\lineto(816.88231934,269.1880896)
\curveto(816.77231619,269.16808975)(816.6673163,269.16808975)(816.56731934,269.1880896)
\lineto(816.43231934,269.1880896)
\curveto(816.38231658,269.19808972)(816.33231663,269.20308972)(816.28231934,269.2030896)
\curveto(816.23231673,269.20308972)(816.18731678,269.21308971)(816.14731934,269.2330896)
\curveto(816.10731686,269.24308968)(816.07231689,269.24808967)(816.04231934,269.2480896)
\curveto(816.02231694,269.23808968)(815.99731697,269.23808968)(815.96731934,269.2480896)
\lineto(815.72731934,269.3080896)
\curveto(815.64731732,269.3180896)(815.57231739,269.33808958)(815.50231934,269.3680896)
\curveto(815.20231776,269.49808942)(814.95731801,269.64308928)(814.76731934,269.8030896)
\curveto(814.58731838,269.97308895)(814.43731853,270.20808871)(814.31731934,270.5080896)
\curveto(814.22731874,270.72808819)(814.18231878,270.99308793)(814.18231934,271.3030896)
\lineto(814.18231934,271.6180896)
\curveto(814.19231877,271.66808725)(814.19731877,271.7180872)(814.19731934,271.7680896)
\lineto(814.22731934,271.9480896)
\lineto(814.34731934,272.2780896)
\curveto(814.38731858,272.38808653)(814.43731853,272.48808643)(814.49731934,272.5780896)
\curveto(814.67731829,272.86808605)(814.92231804,273.08308584)(815.23231934,273.2230896)
\curveto(815.54231742,273.36308556)(815.88231708,273.48808543)(816.25231934,273.5980896)
\curveto(816.39231657,273.63808528)(816.53731643,273.66808525)(816.68731934,273.6880896)
\curveto(816.83731613,273.70808521)(816.98731598,273.73308519)(817.13731934,273.7630896)
\curveto(817.20731576,273.78308514)(817.27231569,273.79308513)(817.33231934,273.7930896)
\curveto(817.40231556,273.79308513)(817.47731549,273.80308512)(817.55731934,273.8230896)
\curveto(817.62731534,273.84308508)(817.69731527,273.85308507)(817.76731934,273.8530896)
\curveto(817.83731513,273.86308506)(817.91231505,273.87808504)(817.99231934,273.8980896)
\curveto(818.24231472,273.95808496)(818.47731449,274.00808491)(818.69731934,274.0480896)
\curveto(818.91731405,274.09808482)(819.09231387,274.21308471)(819.22231934,274.3930896)
\curveto(819.28231368,274.47308445)(819.33231363,274.57308435)(819.37231934,274.6930896)
\curveto(819.41231355,274.8230841)(819.41231355,274.96308396)(819.37231934,275.1130896)
\curveto(819.31231365,275.35308357)(819.22231374,275.54308338)(819.10231934,275.6830896)
\curveto(818.99231397,275.8230831)(818.83231413,275.93308299)(818.62231934,276.0130896)
\curveto(818.50231446,276.06308286)(818.35731461,276.09808282)(818.18731934,276.1180896)
\curveto(818.02731494,276.13808278)(817.85731511,276.14808277)(817.67731934,276.1480896)
\curveto(817.49731547,276.14808277)(817.32231564,276.13808278)(817.15231934,276.1180896)
\curveto(816.98231598,276.09808282)(816.83731613,276.06808285)(816.71731934,276.0280896)
\curveto(816.54731642,275.96808295)(816.38231658,275.88308304)(816.22231934,275.7730896)
\curveto(816.14231682,275.71308321)(816.0673169,275.63308329)(815.99731934,275.5330896)
\curveto(815.93731703,275.44308348)(815.88231708,275.34308358)(815.83231934,275.2330896)
\curveto(815.80231716,275.15308377)(815.77231719,275.06808385)(815.74231934,274.9780896)
\curveto(815.72231724,274.88808403)(815.67731729,274.8180841)(815.60731934,274.7680896)
\curveto(815.5673174,274.73808418)(815.49731747,274.71308421)(815.39731934,274.6930896)
\curveto(815.30731766,274.68308424)(815.21231775,274.67808424)(815.11231934,274.6780896)
\curveto(815.01231795,274.67808424)(814.91231805,274.68308424)(814.81231934,274.6930896)
\curveto(814.72231824,274.71308421)(814.65731831,274.73808418)(814.61731934,274.7680896)
\curveto(814.57731839,274.79808412)(814.54731842,274.84808407)(814.52731934,274.9180896)
\curveto(814.50731846,274.98808393)(814.50731846,275.06308386)(814.52731934,275.1430896)
\curveto(814.55731841,275.27308365)(814.58731838,275.39308353)(814.61731934,275.5030896)
\curveto(814.65731831,275.6230833)(814.70231826,275.73808318)(814.75231934,275.8480896)
\curveto(814.94231802,276.19808272)(815.18231778,276.46808245)(815.47231934,276.6580896)
\curveto(815.7623172,276.85808206)(816.12231684,277.0180819)(816.55231934,277.1380896)
\curveto(816.65231631,277.15808176)(816.75231621,277.17308175)(816.85231934,277.1830896)
\curveto(816.962316,277.19308173)(817.07231589,277.20808171)(817.18231934,277.2280896)
\curveto(817.22231574,277.23808168)(817.28731568,277.23808168)(817.37731934,277.2280896)
\curveto(817.4673155,277.22808169)(817.52231544,277.23808168)(817.54231934,277.2580896)
\curveto(818.24231472,277.26808165)(818.85231411,277.18808173)(819.37231934,277.0180896)
\curveto(819.89231307,276.84808207)(820.25731271,276.5230824)(820.46731934,276.0430896)
\curveto(820.55731241,275.84308308)(820.60731236,275.60808331)(820.61731934,275.3380896)
\curveto(820.63731233,275.07808384)(820.64731232,274.80308412)(820.64731934,274.5130896)
\lineto(820.64731934,271.1980896)
\curveto(820.64731232,271.05808786)(820.65231231,270.923088)(820.66231934,270.7930896)
\curveto(820.67231229,270.66308826)(820.70231226,270.55808836)(820.75231934,270.4780896)
\curveto(820.80231216,270.40808851)(820.8673121,270.35808856)(820.94731934,270.3280896)
\curveto(821.03731193,270.28808863)(821.12231184,270.25808866)(821.20231934,270.2380896)
\curveto(821.28231168,270.22808869)(821.34231162,270.18308874)(821.38231934,270.1030896)
\curveto(821.40231156,270.07308885)(821.41231155,270.04308888)(821.41231934,270.0130896)
\curveto(821.41231155,269.98308894)(821.41731155,269.94308898)(821.42731934,269.8930896)
\moveto(819.28231934,271.5580896)
\curveto(819.34231362,271.69808722)(819.37231359,271.85808706)(819.37231934,272.0380896)
\curveto(819.38231358,272.22808669)(819.38731358,272.4230865)(819.38731934,272.6230896)
\curveto(819.38731358,272.73308619)(819.38231358,272.83308609)(819.37231934,272.9230896)
\curveto(819.3623136,273.01308591)(819.32231364,273.08308584)(819.25231934,273.1330896)
\curveto(819.22231374,273.15308577)(819.15231381,273.16308576)(819.04231934,273.1630896)
\curveto(819.02231394,273.14308578)(818.98731398,273.13308579)(818.93731934,273.1330896)
\curveto(818.88731408,273.13308579)(818.84231412,273.1230858)(818.80231934,273.1030896)
\curveto(818.72231424,273.08308584)(818.63231433,273.06308586)(818.53231934,273.0430896)
\lineto(818.23231934,272.9830896)
\curveto(818.20231476,272.98308594)(818.1673148,272.97808594)(818.12731934,272.9680896)
\lineto(818.02231934,272.9680896)
\curveto(817.87231509,272.92808599)(817.70731526,272.90308602)(817.52731934,272.8930896)
\curveto(817.35731561,272.89308603)(817.19731577,272.87308605)(817.04731934,272.8330896)
\curveto(816.967316,272.81308611)(816.89231607,272.79308613)(816.82231934,272.7730896)
\curveto(816.7623162,272.76308616)(816.69231627,272.74808617)(816.61231934,272.7280896)
\curveto(816.45231651,272.67808624)(816.30231666,272.61308631)(816.16231934,272.5330896)
\curveto(816.02231694,272.46308646)(815.90231706,272.37308655)(815.80231934,272.2630896)
\curveto(815.70231726,272.15308677)(815.62731734,272.0180869)(815.57731934,271.8580896)
\curveto(815.52731744,271.70808721)(815.50731746,271.5230874)(815.51731934,271.3030896)
\curveto(815.51731745,271.20308772)(815.53231743,271.10808781)(815.56231934,271.0180896)
\curveto(815.60231736,270.93808798)(815.64731732,270.86308806)(815.69731934,270.7930896)
\curveto(815.77731719,270.68308824)(815.88231708,270.58808833)(816.01231934,270.5080896)
\curveto(816.14231682,270.43808848)(816.28231668,270.37808854)(816.43231934,270.3280896)
\curveto(816.48231648,270.3180886)(816.53231643,270.31308861)(816.58231934,270.3130896)
\curveto(816.63231633,270.31308861)(816.68231628,270.30808861)(816.73231934,270.2980896)
\curveto(816.80231616,270.27808864)(816.88731608,270.26308866)(816.98731934,270.2530896)
\curveto(817.09731587,270.25308867)(817.18731578,270.26308866)(817.25731934,270.2830896)
\curveto(817.31731565,270.30308862)(817.37731559,270.30808861)(817.43731934,270.2980896)
\curveto(817.49731547,270.29808862)(817.55731541,270.30808861)(817.61731934,270.3280896)
\curveto(817.69731527,270.34808857)(817.77231519,270.36308856)(817.84231934,270.3730896)
\curveto(817.92231504,270.38308854)(817.99731497,270.40308852)(818.06731934,270.4330896)
\curveto(818.35731461,270.55308837)(818.60231436,270.69808822)(818.80231934,270.8680896)
\curveto(819.01231395,271.03808788)(819.17231379,271.26808765)(819.28231934,271.5580896)
}
}
{
\newrgbcolor{curcolor}{0 0 0}
\pscustom[linestyle=none,fillstyle=solid,fillcolor=curcolor]
{
\newpath
\moveto(825.02895996,277.2430896)
\curveto(825.7489559,277.25308167)(826.35395529,277.16808175)(826.84395996,276.9880896)
\curveto(827.33395431,276.8180821)(827.71395393,276.51308241)(827.98395996,276.0730896)
\curveto(828.05395359,275.96308296)(828.10895354,275.84808307)(828.14895996,275.7280896)
\curveto(828.18895346,275.6180833)(828.22895342,275.49308343)(828.26895996,275.3530896)
\curveto(828.28895336,275.28308364)(828.29395335,275.20808371)(828.28395996,275.1280896)
\curveto(828.27395337,275.05808386)(828.25895339,275.00308392)(828.23895996,274.9630896)
\curveto(828.21895343,274.94308398)(828.19395345,274.923084)(828.16395996,274.9030896)
\curveto(828.13395351,274.89308403)(828.10895354,274.87808404)(828.08895996,274.8580896)
\curveto(828.03895361,274.83808408)(827.98895366,274.83308409)(827.93895996,274.8430896)
\curveto(827.88895376,274.85308407)(827.83895381,274.85308407)(827.78895996,274.8430896)
\curveto(827.70895394,274.8230841)(827.60395404,274.8180841)(827.47395996,274.8280896)
\curveto(827.3439543,274.84808407)(827.25395439,274.87308405)(827.20395996,274.9030896)
\curveto(827.12395452,274.95308397)(827.06895458,275.0180839)(827.03895996,275.0980896)
\curveto(827.01895463,275.18808373)(826.98395466,275.27308365)(826.93395996,275.3530896)
\curveto(826.8439548,275.51308341)(826.71895493,275.65808326)(826.55895996,275.7880896)
\curveto(826.4489552,275.86808305)(826.32895532,275.92808299)(826.19895996,275.9680896)
\curveto(826.06895558,276.00808291)(825.92895572,276.04808287)(825.77895996,276.0880896)
\curveto(825.72895592,276.10808281)(825.67895597,276.11308281)(825.62895996,276.1030896)
\curveto(825.57895607,276.10308282)(825.52895612,276.10808281)(825.47895996,276.1180896)
\curveto(825.41895623,276.13808278)(825.3439563,276.14808277)(825.25395996,276.1480896)
\curveto(825.16395648,276.14808277)(825.08895656,276.13808278)(825.02895996,276.1180896)
\lineto(824.93895996,276.1180896)
\lineto(824.78895996,276.0880896)
\curveto(824.73895691,276.08808283)(824.68895696,276.08308284)(824.63895996,276.0730896)
\curveto(824.37895727,276.01308291)(824.16395748,275.92808299)(823.99395996,275.8180896)
\curveto(823.82395782,275.70808321)(823.70895794,275.5230834)(823.64895996,275.2630896)
\curveto(823.62895802,275.19308373)(823.62395802,275.1230838)(823.63395996,275.0530896)
\curveto(823.65395799,274.98308394)(823.67395797,274.923084)(823.69395996,274.8730896)
\curveto(823.75395789,274.7230842)(823.82395782,274.61308431)(823.90395996,274.5430896)
\curveto(823.99395765,274.48308444)(824.10395754,274.41308451)(824.23395996,274.3330896)
\curveto(824.39395725,274.23308469)(824.57395707,274.15808476)(824.77395996,274.1080896)
\curveto(824.97395667,274.06808485)(825.17395647,274.0180849)(825.37395996,273.9580896)
\curveto(825.50395614,273.918085)(825.63395601,273.88808503)(825.76395996,273.8680896)
\curveto(825.89395575,273.84808507)(826.02395562,273.8180851)(826.15395996,273.7780896)
\curveto(826.36395528,273.7180852)(826.56895508,273.65808526)(826.76895996,273.5980896)
\curveto(826.96895468,273.54808537)(827.16895448,273.48308544)(827.36895996,273.4030896)
\lineto(827.51895996,273.3430896)
\curveto(827.56895408,273.3230856)(827.61895403,273.29808562)(827.66895996,273.2680896)
\curveto(827.86895378,273.14808577)(828.0439536,273.01308591)(828.19395996,272.8630896)
\curveto(828.3439533,272.71308621)(828.46895318,272.5230864)(828.56895996,272.2930896)
\curveto(828.58895306,272.2230867)(828.60895304,272.12808679)(828.62895996,272.0080896)
\curveto(828.648953,271.93808698)(828.65895299,271.86308706)(828.65895996,271.7830896)
\curveto(828.66895298,271.71308721)(828.67395297,271.63308729)(828.67395996,271.5430896)
\lineto(828.67395996,271.3930896)
\curveto(828.65395299,271.3230876)(828.643953,271.25308767)(828.64395996,271.1830896)
\curveto(828.643953,271.11308781)(828.63395301,271.04308788)(828.61395996,270.9730896)
\curveto(828.58395306,270.86308806)(828.5489531,270.75808816)(828.50895996,270.6580896)
\curveto(828.46895318,270.55808836)(828.42395322,270.46808845)(828.37395996,270.3880896)
\curveto(828.21395343,270.12808879)(828.00895364,269.918089)(827.75895996,269.7580896)
\curveto(827.50895414,269.60808931)(827.22895442,269.47808944)(826.91895996,269.3680896)
\curveto(826.82895482,269.33808958)(826.73395491,269.3180896)(826.63395996,269.3080896)
\curveto(826.5439551,269.28808963)(826.45395519,269.26308966)(826.36395996,269.2330896)
\curveto(826.26395538,269.21308971)(826.16395548,269.20308972)(826.06395996,269.2030896)
\curveto(825.96395568,269.20308972)(825.86395578,269.19308973)(825.76395996,269.1730896)
\lineto(825.61395996,269.1730896)
\curveto(825.56395608,269.16308976)(825.49395615,269.15808976)(825.40395996,269.1580896)
\curveto(825.31395633,269.15808976)(825.2439564,269.16308976)(825.19395996,269.1730896)
\lineto(825.02895996,269.1730896)
\curveto(824.96895668,269.19308973)(824.90395674,269.20308972)(824.83395996,269.2030896)
\curveto(824.76395688,269.19308973)(824.70395694,269.19808972)(824.65395996,269.2180896)
\curveto(824.60395704,269.22808969)(824.53895711,269.23308969)(824.45895996,269.2330896)
\lineto(824.21895996,269.2930896)
\curveto(824.1489575,269.30308962)(824.07395757,269.3230896)(823.99395996,269.3530896)
\curveto(823.68395796,269.45308947)(823.41395823,269.57808934)(823.18395996,269.7280896)
\curveto(822.95395869,269.87808904)(822.75395889,270.07308885)(822.58395996,270.3130896)
\curveto(822.49395915,270.44308848)(822.41895923,270.57808834)(822.35895996,270.7180896)
\curveto(822.29895935,270.85808806)(822.2439594,271.01308791)(822.19395996,271.1830896)
\curveto(822.17395947,271.24308768)(822.16395948,271.31308761)(822.16395996,271.3930896)
\curveto(822.17395947,271.48308744)(822.18895946,271.55308737)(822.20895996,271.6030896)
\curveto(822.23895941,271.64308728)(822.28895936,271.68308724)(822.35895996,271.7230896)
\curveto(822.40895924,271.74308718)(822.47895917,271.75308717)(822.56895996,271.7530896)
\curveto(822.65895899,271.76308716)(822.7489589,271.76308716)(822.83895996,271.7530896)
\curveto(822.92895872,271.74308718)(823.01395863,271.72808719)(823.09395996,271.7080896)
\curveto(823.18395846,271.69808722)(823.2439584,271.68308724)(823.27395996,271.6630896)
\curveto(823.3439583,271.61308731)(823.38895826,271.53808738)(823.40895996,271.4380896)
\curveto(823.43895821,271.34808757)(823.47395817,271.26308766)(823.51395996,271.1830896)
\curveto(823.61395803,270.96308796)(823.7489579,270.79308813)(823.91895996,270.6730896)
\curveto(824.03895761,270.58308834)(824.17395747,270.51308841)(824.32395996,270.4630896)
\curveto(824.47395717,270.41308851)(824.63395701,270.36308856)(824.80395996,270.3130896)
\lineto(825.11895996,270.2680896)
\lineto(825.20895996,270.2680896)
\curveto(825.27895637,270.24808867)(825.36895628,270.23808868)(825.47895996,270.2380896)
\curveto(825.59895605,270.23808868)(825.69895595,270.24808867)(825.77895996,270.2680896)
\curveto(825.8489558,270.26808865)(825.90395574,270.27308865)(825.94395996,270.2830896)
\curveto(826.00395564,270.29308863)(826.06395558,270.29808862)(826.12395996,270.2980896)
\curveto(826.18395546,270.30808861)(826.23895541,270.3180886)(826.28895996,270.3280896)
\curveto(826.57895507,270.40808851)(826.80895484,270.51308841)(826.97895996,270.6430896)
\curveto(827.1489545,270.77308815)(827.26895438,270.99308793)(827.33895996,271.3030896)
\curveto(827.35895429,271.35308757)(827.36395428,271.40808751)(827.35395996,271.4680896)
\curveto(827.3439543,271.52808739)(827.33395431,271.57308735)(827.32395996,271.6030896)
\curveto(827.27395437,271.79308713)(827.20395444,271.93308699)(827.11395996,272.0230896)
\curveto(827.02395462,272.1230868)(826.90895474,272.21308671)(826.76895996,272.2930896)
\curveto(826.67895497,272.35308657)(826.57895507,272.40308652)(826.46895996,272.4430896)
\lineto(826.13895996,272.5630896)
\curveto(826.10895554,272.57308635)(826.07895557,272.57808634)(826.04895996,272.5780896)
\curveto(826.02895562,272.57808634)(826.00395564,272.58808633)(825.97395996,272.6080896)
\curveto(825.63395601,272.7180862)(825.27895637,272.79808612)(824.90895996,272.8480896)
\curveto(824.5489571,272.90808601)(824.20895744,273.00308592)(823.88895996,273.1330896)
\curveto(823.78895786,273.17308575)(823.69395795,273.20808571)(823.60395996,273.2380896)
\curveto(823.51395813,273.26808565)(823.42895822,273.30808561)(823.34895996,273.3580896)
\curveto(823.15895849,273.46808545)(822.98395866,273.59308533)(822.82395996,273.7330896)
\curveto(822.66395898,273.87308505)(822.53895911,274.04808487)(822.44895996,274.2580896)
\curveto(822.41895923,274.32808459)(822.39395925,274.39808452)(822.37395996,274.4680896)
\curveto(822.36395928,274.53808438)(822.3489593,274.61308431)(822.32895996,274.6930896)
\curveto(822.29895935,274.81308411)(822.28895936,274.94808397)(822.29895996,275.0980896)
\curveto(822.30895934,275.25808366)(822.32395932,275.39308353)(822.34395996,275.5030896)
\curveto(822.36395928,275.55308337)(822.37395927,275.59308333)(822.37395996,275.6230896)
\curveto(822.38395926,275.66308326)(822.39895925,275.70308322)(822.41895996,275.7430896)
\curveto(822.50895914,275.97308295)(822.62895902,276.17308275)(822.77895996,276.3430896)
\curveto(822.93895871,276.51308241)(823.11895853,276.66308226)(823.31895996,276.7930896)
\curveto(823.46895818,276.88308204)(823.63395801,276.95308197)(823.81395996,277.0030896)
\curveto(823.99395765,277.06308186)(824.18395746,277.1180818)(824.38395996,277.1680896)
\curveto(824.45395719,277.17808174)(824.51895713,277.18808173)(824.57895996,277.1980896)
\curveto(824.648957,277.20808171)(824.72395692,277.2180817)(824.80395996,277.2280896)
\curveto(824.83395681,277.23808168)(824.87395677,277.23808168)(824.92395996,277.2280896)
\curveto(824.97395667,277.2180817)(825.00895664,277.2230817)(825.02895996,277.2430896)
}
}
{
\newrgbcolor{curcolor}{0.40000001 0.40000001 0.40000001}
\pscustom[linestyle=none,fillstyle=solid,fillcolor=curcolor]
{
\newpath
\moveto(747.9732666,280.03309631)
\lineto(762.9732666,280.03309631)
\lineto(762.9732666,265.03309631)
\lineto(747.9732666,265.03309631)
\closepath
}
}
{
\newrgbcolor{curcolor}{0.7019608 0.7019608 0.7019608}
\pscustom[linestyle=none,fillstyle=solid,fillcolor=curcolor]
{
\newpath
\moveto(554.98260498,192.00756836)
\lineto(568.02885151,192.00756836)
\lineto(568.02885151,76.08277893)
\lineto(554.98260498,76.08277893)
\closepath
}
}
{
\newrgbcolor{curcolor}{0.60000002 0.60000002 0.60000002}
\pscustom[linestyle=none,fillstyle=solid,fillcolor=curcolor]
{
\newpath
\moveto(567.97570801,77.18908691)
\lineto(581.02195454,77.18908691)
\lineto(581.02195454,76.08278656)
\lineto(567.97570801,76.08278656)
\closepath
}
}
{
\newrgbcolor{curcolor}{0.50196081 0.50196081 0.50196081}
\pscustom[linestyle=none,fillstyle=solid,fillcolor=curcolor]
{
\newpath
\moveto(580.96881104,76.50158691)
\lineto(594.01505756,76.50158691)
\lineto(594.01505756,76.08278662)
\lineto(580.96881104,76.08278662)
\closepath
}
}
{
\newrgbcolor{curcolor}{0.40000001 0.40000001 0.40000001}
\pscustom[linestyle=none,fillstyle=solid,fillcolor=curcolor]
{
\newpath
\moveto(593.96191406,78.93908691)
\lineto(607.00816059,78.93908691)
\lineto(607.00816059,76.08278656)
\lineto(593.96191406,76.08278656)
\closepath
}
}
{
\newrgbcolor{curcolor}{0.80000001 0.80000001 0.80000001}
\pscustom[linestyle=none,fillstyle=solid,fillcolor=curcolor]
{
\newpath
\moveto(114.89689636,87.00222778)
\lineto(127.94314289,87.00222778)
\lineto(127.94314289,76.0827961)
\lineto(114.89689636,76.0827961)
\closepath
}
}
{
\newrgbcolor{curcolor}{0.7019608 0.7019608 0.7019608}
\pscustom[linestyle=none,fillstyle=solid,fillcolor=curcolor]
{
\newpath
\moveto(127.88999939,315.98626709)
\lineto(140.93624592,315.98626709)
\lineto(140.93624592,76.08279419)
\lineto(127.88999939,76.08279419)
\closepath
}
}
{
\newrgbcolor{curcolor}{0.60000002 0.60000002 0.60000002}
\pscustom[linestyle=none,fillstyle=solid,fillcolor=curcolor]
{
\newpath
\moveto(140.88310242,80.00158691)
\lineto(153.92934895,80.00158691)
\lineto(153.92934895,76.08278656)
\lineto(140.88310242,76.08278656)
\closepath
}
}
{
\newrgbcolor{curcolor}{0.50196081 0.50196081 0.50196081}
\pscustom[linestyle=none,fillstyle=solid,fillcolor=curcolor]
{
\newpath
\moveto(153.87620544,80.93908691)
\lineto(166.92245197,80.93908691)
\lineto(166.92245197,76.08278656)
\lineto(153.87620544,76.08278656)
\closepath
}
}
{
\newrgbcolor{curcolor}{0.40000001 0.40000001 0.40000001}
\pscustom[linestyle=none,fillstyle=solid,fillcolor=curcolor]
{
\newpath
\moveto(166.86930847,93.00158691)
\lineto(179.915555,93.00158691)
\lineto(179.915555,76.08278656)
\lineto(166.86930847,76.08278656)
\closepath
}
}
{
\newrgbcolor{curcolor}{0.80000001 0.80000001 0.80000001}
\pscustom[linestyle=none,fillstyle=solid,fillcolor=curcolor]
{
\newpath
\moveto(222.05667114,79.00308228)
\lineto(235.10291767,79.00308228)
\lineto(235.10291767,76.0827961)
\lineto(222.05667114,76.0827961)
\closepath
}
}
{
\newrgbcolor{curcolor}{0.7019608 0.7019608 0.7019608}
\pscustom[linestyle=none,fillstyle=solid,fillcolor=curcolor]
{
\newpath
\moveto(235.04977417,96.01782227)
\lineto(248.0960207,96.01782227)
\lineto(248.0960207,76.08277893)
\lineto(235.04977417,76.08277893)
\closepath
}
}
{
\newrgbcolor{curcolor}{0.60000002 0.60000002 0.60000002}
\pscustom[linestyle=none,fillstyle=solid,fillcolor=curcolor]
{
\newpath
\moveto(248.0428772,78.03079224)
\lineto(261.08912373,78.03079224)
\lineto(261.08912373,76.08277786)
\lineto(248.0428772,76.08277786)
\closepath
}
}
{
\newrgbcolor{curcolor}{0.50196081 0.50196081 0.50196081}
\pscustom[linestyle=none,fillstyle=solid,fillcolor=curcolor]
{
\newpath
\moveto(261.03598022,76.8817749)
\lineto(274.08222675,76.8817749)
\lineto(274.08222675,76.08280909)
\lineto(261.03598022,76.08280909)
\closepath
}
}
{
\newrgbcolor{curcolor}{0.40000001 0.40000001 0.40000001}
\pscustom[linestyle=none,fillstyle=solid,fillcolor=curcolor]
{
\newpath
\moveto(274.02908325,79.04727173)
\lineto(287.07532978,79.04727173)
\lineto(287.07532978,76.08279133)
\lineto(274.02908325,76.08279133)
\closepath
}
}
{
\newrgbcolor{curcolor}{0.80000001 0.80000001 0.80000001}
\pscustom[linestyle=none,fillstyle=solid,fillcolor=curcolor]
{
\newpath
\moveto(328.92700195,77.00158691)
\lineto(341.97324848,77.00158691)
\lineto(341.97324848,76.08278662)
\lineto(328.92700195,76.08278662)
\closepath
}
}
{
\newrgbcolor{curcolor}{0.7019608 0.7019608 0.7019608}
\pscustom[linestyle=none,fillstyle=solid,fillcolor=curcolor]
{
\newpath
\moveto(341.92010498,147.06210327)
\lineto(354.96635151,147.06210327)
\lineto(354.96635151,76.08279419)
\lineto(341.92010498,76.08279419)
\closepath
}
}
{
\newrgbcolor{curcolor}{0.60000002 0.60000002 0.60000002}
\pscustom[linestyle=none,fillstyle=solid,fillcolor=curcolor]
{
\newpath
\moveto(354.91320801,77.01434326)
\lineto(367.95945454,77.01434326)
\lineto(367.95945454,76.0827949)
\lineto(354.91320801,76.0827949)
\closepath
}
}
{
\newrgbcolor{curcolor}{0.40000001 0.40000001 0.40000001}
\pscustom[linestyle=none,fillstyle=solid,fillcolor=curcolor]
{
\newpath
\moveto(380.89941406,77.10272217)
\lineto(393.94566059,77.10272217)
\lineto(393.94566059,76.08278549)
\lineto(380.89941406,76.08278549)
\closepath
}
}
{
\newrgbcolor{curcolor}{0.80000001 0.80000001 0.80000001}
\pscustom[linestyle=none,fillstyle=solid,fillcolor=curcolor]
{
\newpath
\moveto(434.98950195,77.00158691)
\lineto(448.03574848,77.00158691)
\lineto(448.03574848,76.08278662)
\lineto(434.98950195,76.08278662)
\closepath
}
}
{
\newrgbcolor{curcolor}{0.7019608 0.7019608 0.7019608}
\pscustom[linestyle=none,fillstyle=solid,fillcolor=curcolor]
{
\newpath
\moveto(447.98260498,78.06408691)
\lineto(461.02885151,78.06408691)
\lineto(461.02885151,76.08278656)
\lineto(447.98260498,76.08278656)
\closepath
}
}
{
\newrgbcolor{curcolor}{0.60000002 0.60000002 0.60000002}
\pscustom[linestyle=none,fillstyle=solid,fillcolor=curcolor]
{
\newpath
\moveto(460.97570801,76.87658691)
\lineto(474.02195454,76.87658691)
\lineto(474.02195454,76.08278662)
\lineto(460.97570801,76.08278662)
\closepath
}
}
{
\newrgbcolor{curcolor}{0.40000001 0.40000001 0.40000001}
\pscustom[linestyle=none,fillstyle=solid,fillcolor=curcolor]
{
\newpath
\moveto(486.96191406,78.00158691)
\lineto(500.00816059,78.00158691)
\lineto(500.00816059,76.08278656)
\lineto(486.96191406,76.08278656)
\closepath
}
}
{
\newrgbcolor{curcolor}{0.80000001 0.80000001 0.80000001}
\pscustom[linestyle=none,fillstyle=solid,fillcolor=curcolor]
{
\newpath
\moveto(648.92700195,77.06408691)
\lineto(661.97324848,77.06408691)
\lineto(661.97324848,76.08278662)
\lineto(648.92700195,76.08278662)
\closepath
}
}
{
\newrgbcolor{curcolor}{0.7019608 0.7019608 0.7019608}
\pscustom[linestyle=none,fillstyle=solid,fillcolor=curcolor]
{
\newpath
\moveto(661.92010498,77.06408691)
\lineto(674.96635151,77.06408691)
\lineto(674.96635151,76.08278662)
\lineto(661.92010498,76.08278662)
\closepath
}
}
{
\newrgbcolor{curcolor}{0.40000001 0.40000001 0.40000001}
\pscustom[linestyle=none,fillstyle=solid,fillcolor=curcolor]
{
\newpath
\moveto(700.89941406,77.06408691)
\lineto(713.94566059,77.06408691)
\lineto(713.94566059,76.08278662)
\lineto(700.89941406,76.08278662)
\closepath
}
}
\end{pspicture}

\caption{Diagrama de barras de los recursos y sus niveles de repercusión}
\label{recursos_bars_1}
\end{figure}

%\begin{figure}
%\centering
%%LaTeX with PSTricks extensions
%%Creator: inkscape 0.48.5
%%Please note this file requires PSTricks extensions
\psset{xunit=.5pt,yunit=.5pt,runit=.5pt}
\begin{pspicture}(923,727)
{
\newrgbcolor{curcolor}{0 0 0}
\pscustom[linestyle=none,fillstyle=solid,fillcolor=curcolor]
{
\newpath
\moveto(28.32111654,703.11942034)
\curveto(28.32110606,703.08941467)(28.32110606,703.04941471)(28.32111654,702.99942034)
\curveto(28.33110605,702.94941481)(28.33610604,702.89441487)(28.33611654,702.83442034)
\curveto(28.33610604,702.77441499)(28.33110605,702.71941504)(28.32111654,702.66942034)
\curveto(28.32110606,702.61941514)(28.32110606,702.58441518)(28.32111654,702.56442034)
\curveto(28.32110606,702.49441527)(28.31610606,702.42441534)(28.30611654,702.35442034)
\curveto(28.30610607,702.29441547)(28.30610607,702.23441553)(28.30611654,702.17442034)
\curveto(28.28610609,702.12441564)(28.2761061,702.07441569)(28.27611654,702.02442034)
\curveto(28.28610609,701.97441579)(28.28610609,701.92441584)(28.27611654,701.87442034)
\curveto(28.25610612,701.764416)(28.24110614,701.65441611)(28.23111654,701.54442034)
\curveto(28.22110616,701.43441633)(28.20110618,701.32441644)(28.17111654,701.21442034)
\curveto(28.12110626,701.04441672)(28.0761063,700.87941688)(28.03611654,700.71942034)
\curveto(27.99610638,700.56941719)(27.94610643,700.41941734)(27.88611654,700.26942034)
\curveto(27.71610666,699.84941791)(27.50610687,699.46941829)(27.25611654,699.12942034)
\curveto(27.00610737,698.78941897)(26.70610767,698.49941926)(26.35611654,698.25942034)
\curveto(26.15610822,698.11941964)(25.94610843,697.99941976)(25.72611654,697.89942034)
\curveto(25.51610886,697.79941996)(25.28610909,697.70942005)(25.03611654,697.62942034)
\curveto(24.93610944,697.59942016)(24.83110955,697.57442019)(24.72111654,697.55442034)
\curveto(24.62110976,697.54442022)(24.51610986,697.52442024)(24.40611654,697.49442034)
\curveto(24.35611002,697.48442028)(24.30611007,697.47942028)(24.25611654,697.47942034)
\curveto(24.21611016,697.47942028)(24.17111021,697.47442029)(24.12111654,697.46442034)
\curveto(24.0811103,697.45442031)(24.04111034,697.44942031)(24.00111654,697.44942034)
\curveto(23.96111042,697.4594203)(23.91611046,697.4594203)(23.86611654,697.44942034)
\curveto(23.84611053,697.43942032)(23.81611056,697.43442033)(23.77611654,697.43442034)
\curveto(23.73611064,697.44442032)(23.70611067,697.44442032)(23.68611654,697.43442034)
\curveto(23.60611077,697.41442035)(23.50611087,697.40942035)(23.38611654,697.41942034)
\curveto(23.26611111,697.42942033)(23.16111122,697.43442033)(23.07111654,697.43442034)
\lineto(19.57611654,697.43442034)
\curveto(19.40611497,697.43442033)(19.26111512,697.43942032)(19.14111654,697.44942034)
\curveto(19.03111535,697.46942029)(18.95111543,697.53942022)(18.90111654,697.65942034)
\curveto(18.87111551,697.73942002)(18.85611552,697.8594199)(18.85611654,698.01942034)
\curveto(18.86611551,698.18941957)(18.87111551,698.32941943)(18.87111654,698.43942034)
\lineto(18.87111654,707.24442034)
\curveto(18.87111551,707.3644104)(18.86611551,707.48941027)(18.85611654,707.61942034)
\curveto(18.85611552,707.75941)(18.8811155,707.86940989)(18.93111654,707.94942034)
\curveto(18.97111541,708.00940975)(19.04611533,708.0594097)(19.15611654,708.09942034)
\curveto(19.1761152,708.10940965)(19.19611518,708.10940965)(19.21611654,708.09942034)
\curveto(19.23611514,708.09940966)(19.25611512,708.10440966)(19.27611654,708.11442034)
\lineto(23.31111654,708.11442034)
\curveto(23.37111101,708.11440965)(23.43111095,708.11440965)(23.49111654,708.11442034)
\curveto(23.56111082,708.12440964)(23.62111076,708.12440964)(23.67111654,708.11442034)
\lineto(23.85111654,708.11442034)
\curveto(23.90111048,708.09440967)(23.95611042,708.08440968)(24.01611654,708.08442034)
\curveto(24.0761103,708.09440967)(24.13111025,708.08940967)(24.18111654,708.06942034)
\curveto(24.24111014,708.04940971)(24.29611008,708.03940972)(24.34611654,708.03942034)
\curveto(24.40610997,708.04940971)(24.46610991,708.04440972)(24.52611654,708.02442034)
\curveto(24.66610971,707.99440977)(24.80110958,707.9644098)(24.93111654,707.93442034)
\curveto(25.06110932,707.91440985)(25.18610919,707.87940988)(25.30611654,707.82942034)
\curveto(25.41610896,707.77940998)(25.52610885,707.73441003)(25.63611654,707.69442034)
\curveto(25.74610863,707.65441011)(25.85110853,707.60441016)(25.95111654,707.54442034)
\curveto(26.20110818,707.38441038)(26.43110795,707.22941053)(26.64111654,707.07942034)
\lineto(26.73111654,706.98942034)
\curveto(26.83110755,706.90941085)(26.92110746,706.81941094)(27.00111654,706.71942034)
\lineto(27.13611654,706.59942034)
\curveto(27.18610719,706.51941124)(27.24110714,706.43941132)(27.30111654,706.35942034)
\curveto(27.37110701,706.28941147)(27.43110695,706.21441155)(27.48111654,706.13442034)
\curveto(27.61110677,705.92441184)(27.72610665,705.69941206)(27.82611654,705.45942034)
\curveto(27.92610645,705.22941253)(28.01610636,704.98441278)(28.09611654,704.72442034)
\curveto(28.14610623,704.59441317)(28.1761062,704.4594133)(28.18611654,704.31942034)
\curveto(28.20610617,704.17941358)(28.23110615,704.03941372)(28.26111654,703.89942034)
\curveto(28.26110612,703.84941391)(28.26110612,703.80441396)(28.26111654,703.76442034)
\curveto(28.27110611,703.73441403)(28.2761061,703.69941406)(28.27611654,703.65942034)
\curveto(28.29610608,703.59941416)(28.30110608,703.53441423)(28.29111654,703.46442034)
\curveto(28.29110609,703.39441437)(28.30110608,703.33441443)(28.32111654,703.28442034)
\lineto(28.32111654,703.11942034)
\moveto(25.98111654,702.39942034)
\curveto(26.00110838,702.44941531)(26.01110837,702.52941523)(26.01111654,702.63942034)
\curveto(26.01110837,702.74941501)(26.00110838,702.82941493)(25.98111654,702.87942034)
\lineto(25.98111654,703.16442034)
\curveto(25.96110842,703.25441451)(25.94610843,703.34941441)(25.93611654,703.44942034)
\curveto(25.93610844,703.54941421)(25.92610845,703.63941412)(25.90611654,703.71942034)
\curveto(25.88610849,703.76941399)(25.8761085,703.81441395)(25.87611654,703.85442034)
\curveto(25.88610849,703.90441386)(25.8811085,703.95441381)(25.86111654,704.00442034)
\curveto(25.81110857,704.1644136)(25.76110862,704.31441345)(25.71111654,704.45442034)
\curveto(25.67110871,704.60441316)(25.61110877,704.74441302)(25.53111654,704.87442034)
\curveto(25.381109,705.11441265)(25.20610917,705.31941244)(25.00611654,705.48942034)
\curveto(24.81610956,705.66941209)(24.5811098,705.81941194)(24.30111654,705.93942034)
\curveto(24.21111017,705.96941179)(24.12111026,705.99441177)(24.03111654,706.01442034)
\curveto(23.94111044,706.04441172)(23.85111053,706.06941169)(23.76111654,706.08942034)
\curveto(23.6811107,706.09941166)(23.60611077,706.10441166)(23.53611654,706.10442034)
\curveto(23.4761109,706.11441165)(23.40611097,706.12941163)(23.32611654,706.14942034)
\curveto(23.28611109,706.1594116)(23.24611113,706.1594116)(23.20611654,706.14942034)
\curveto(23.16611121,706.14941161)(23.13111125,706.15441161)(23.10111654,706.16442034)
\lineto(22.77111654,706.16442034)
\curveto(22.72111166,706.17441159)(22.66611171,706.17441159)(22.60611654,706.16442034)
\lineto(22.42611654,706.16442034)
\lineto(21.75111654,706.16442034)
\curveto(21.73111265,706.14441162)(21.69611268,706.13941162)(21.64611654,706.14942034)
\curveto(21.60611277,706.1594116)(21.57111281,706.1594116)(21.54111654,706.14942034)
\lineto(21.39111654,706.08942034)
\curveto(21.34111304,706.07941168)(21.30111308,706.04941171)(21.27111654,705.99942034)
\curveto(21.23111315,705.94941181)(21.21111317,705.87941188)(21.21111654,705.78942034)
\lineto(21.21111654,705.48942034)
\curveto(21.21111317,705.3594124)(21.20611317,705.22441254)(21.19611654,705.08442034)
\lineto(21.19611654,704.66442034)
\lineto(21.19611654,700.47942034)
\curveto(21.19611318,700.41941734)(21.19111319,700.35441741)(21.18111654,700.28442034)
\curveto(21.1811132,700.21441755)(21.19111319,700.15441761)(21.21111654,700.10442034)
\lineto(21.21111654,699.95442034)
\lineto(21.21111654,699.74442034)
\curveto(21.22111316,699.68441808)(21.23611314,699.62941813)(21.25611654,699.57942034)
\curveto(21.31611306,699.4594183)(21.43111295,699.39441837)(21.60111654,699.38442034)
\lineto(22.12611654,699.38442034)
\lineto(23.31111654,699.38442034)
\curveto(23.71111067,699.39441837)(24.05111033,699.45441831)(24.33111654,699.56442034)
\curveto(24.70110968,699.71441805)(24.99110939,699.91441785)(25.20111654,700.16442034)
\curveto(25.42110896,700.41441735)(25.60610877,700.72441704)(25.75611654,701.09442034)
\curveto(25.79610858,701.17441659)(25.82610855,701.2644165)(25.84611654,701.36442034)
\curveto(25.86610851,701.4644163)(25.89110849,701.5644162)(25.92111654,701.66442034)
\lineto(25.92111654,701.78442034)
\curveto(25.94110844,701.85441591)(25.95110843,701.92941583)(25.95111654,702.00942034)
\curveto(25.95110843,702.08941567)(25.96110842,702.16941559)(25.98111654,702.24942034)
\lineto(25.98111654,702.39942034)
}
}
{
\newrgbcolor{curcolor}{0 0 0}
\pscustom[linestyle=none,fillstyle=solid,fillcolor=curcolor]
{
\newpath
\moveto(31.81963217,708.00942034)
\curveto(31.88962922,707.92940983)(31.92462918,707.80940995)(31.92463217,707.64942034)
\lineto(31.92463217,707.18442034)
\lineto(31.92463217,706.77942034)
\curveto(31.92462918,706.63941112)(31.88962922,706.54441122)(31.81963217,706.49442034)
\curveto(31.75962935,706.44441132)(31.67962943,706.41441135)(31.57963217,706.40442034)
\curveto(31.48962962,706.39441137)(31.38962972,706.38941137)(31.27963217,706.38942034)
\lineto(30.43963217,706.38942034)
\curveto(30.32963078,706.38941137)(30.22963088,706.39441137)(30.13963217,706.40442034)
\curveto(30.05963105,706.41441135)(29.98963112,706.44441132)(29.92963217,706.49442034)
\curveto(29.88963122,706.52441124)(29.85963125,706.57941118)(29.83963217,706.65942034)
\curveto(29.82963128,706.74941101)(29.81963129,706.84441092)(29.80963217,706.94442034)
\lineto(29.80963217,707.27442034)
\curveto(29.81963129,707.38441038)(29.82463128,707.47941028)(29.82463217,707.55942034)
\lineto(29.82463217,707.76942034)
\curveto(29.83463127,707.83940992)(29.85463125,707.89940986)(29.88463217,707.94942034)
\curveto(29.9046312,707.98940977)(29.92963118,708.01940974)(29.95963217,708.03942034)
\lineto(30.07963217,708.09942034)
\curveto(30.09963101,708.09940966)(30.12463098,708.09940966)(30.15463217,708.09942034)
\curveto(30.18463092,708.10940965)(30.2096309,708.11440965)(30.22963217,708.11442034)
\lineto(31.32463217,708.11442034)
\curveto(31.42462968,708.11440965)(31.51962959,708.10940965)(31.60963217,708.09942034)
\curveto(31.69962941,708.08940967)(31.76962934,708.0594097)(31.81963217,708.00942034)
\moveto(31.92463217,698.24442034)
\curveto(31.92462918,698.04441972)(31.91962919,697.87441989)(31.90963217,697.73442034)
\curveto(31.89962921,697.59442017)(31.8096293,697.49942026)(31.63963217,697.44942034)
\curveto(31.57962953,697.42942033)(31.51462959,697.41942034)(31.44463217,697.41942034)
\curveto(31.37462973,697.42942033)(31.29962981,697.43442033)(31.21963217,697.43442034)
\lineto(30.37963217,697.43442034)
\curveto(30.28963082,697.43442033)(30.19963091,697.43942032)(30.10963217,697.44942034)
\curveto(30.02963108,697.4594203)(29.96963114,697.48942027)(29.92963217,697.53942034)
\curveto(29.86963124,697.60942015)(29.83463127,697.69442007)(29.82463217,697.79442034)
\lineto(29.82463217,698.13942034)
\lineto(29.82463217,704.46942034)
\lineto(29.82463217,704.76942034)
\curveto(29.82463128,704.86941289)(29.84463126,704.94941281)(29.88463217,705.00942034)
\curveto(29.94463116,705.07941268)(30.02963108,705.12441264)(30.13963217,705.14442034)
\curveto(30.15963095,705.15441261)(30.18463092,705.15441261)(30.21463217,705.14442034)
\curveto(30.25463085,705.14441262)(30.28463082,705.14941261)(30.30463217,705.15942034)
\lineto(31.05463217,705.15942034)
\lineto(31.24963217,705.15942034)
\curveto(31.32962978,705.16941259)(31.39462971,705.16941259)(31.44463217,705.15942034)
\lineto(31.56463217,705.15942034)
\curveto(31.62462948,705.13941262)(31.67962943,705.12441264)(31.72963217,705.11442034)
\curveto(31.77962933,705.10441266)(31.81962929,705.07441269)(31.84963217,705.02442034)
\curveto(31.88962922,704.97441279)(31.9096292,704.90441286)(31.90963217,704.81442034)
\curveto(31.91962919,704.72441304)(31.92462918,704.62941313)(31.92463217,704.52942034)
\lineto(31.92463217,698.24442034)
}
}
{
\newrgbcolor{curcolor}{0 0 0}
\pscustom[linestyle=none,fillstyle=solid,fillcolor=curcolor]
{
\newpath
\moveto(36.55681967,705.36942034)
\curveto(37.30681517,705.38941237)(37.95681452,705.30441246)(38.50681967,705.11442034)
\curveto(39.06681341,704.93441283)(39.49181298,704.61941314)(39.78181967,704.16942034)
\curveto(39.85181262,704.0594137)(39.91181256,703.94441382)(39.96181967,703.82442034)
\curveto(40.02181245,703.71441405)(40.0718124,703.58941417)(40.11181967,703.44942034)
\curveto(40.13181234,703.38941437)(40.14181233,703.32441444)(40.14181967,703.25442034)
\curveto(40.14181233,703.18441458)(40.13181234,703.12441464)(40.11181967,703.07442034)
\curveto(40.0718124,703.01441475)(40.01681246,702.97441479)(39.94681967,702.95442034)
\curveto(39.89681258,702.93441483)(39.83681264,702.92441484)(39.76681967,702.92442034)
\lineto(39.55681967,702.92442034)
\lineto(38.89681967,702.92442034)
\curveto(38.82681365,702.92441484)(38.75681372,702.91941484)(38.68681967,702.90942034)
\curveto(38.61681386,702.90941485)(38.55181392,702.91941484)(38.49181967,702.93942034)
\curveto(38.39181408,702.9594148)(38.31681416,702.99941476)(38.26681967,703.05942034)
\curveto(38.21681426,703.11941464)(38.1718143,703.17941458)(38.13181967,703.23942034)
\lineto(38.01181967,703.44942034)
\curveto(37.98181449,703.52941423)(37.93181454,703.59441417)(37.86181967,703.64442034)
\curveto(37.76181471,703.72441404)(37.66181481,703.78441398)(37.56181967,703.82442034)
\curveto(37.471815,703.8644139)(37.35681512,703.89941386)(37.21681967,703.92942034)
\curveto(37.14681533,703.94941381)(37.04181543,703.9644138)(36.90181967,703.97442034)
\curveto(36.7718157,703.98441378)(36.6718158,703.97941378)(36.60181967,703.95942034)
\lineto(36.49681967,703.95942034)
\lineto(36.34681967,703.92942034)
\curveto(36.30681617,703.92941383)(36.26181621,703.92441384)(36.21181967,703.91442034)
\curveto(36.04181643,703.8644139)(35.90181657,703.79441397)(35.79181967,703.70442034)
\curveto(35.69181678,703.62441414)(35.62181685,703.49941426)(35.58181967,703.32942034)
\curveto(35.56181691,703.2594145)(35.56181691,703.19441457)(35.58181967,703.13442034)
\curveto(35.60181687,703.07441469)(35.62181685,703.02441474)(35.64181967,702.98442034)
\curveto(35.71181676,702.8644149)(35.79181668,702.76941499)(35.88181967,702.69942034)
\curveto(35.98181649,702.62941513)(36.09681638,702.56941519)(36.22681967,702.51942034)
\curveto(36.41681606,702.43941532)(36.62181585,702.36941539)(36.84181967,702.30942034)
\lineto(37.53181967,702.15942034)
\curveto(37.7718147,702.11941564)(38.00181447,702.06941569)(38.22181967,702.00942034)
\curveto(38.45181402,701.9594158)(38.66681381,701.89441587)(38.86681967,701.81442034)
\curveto(38.95681352,701.77441599)(39.04181343,701.73941602)(39.12181967,701.70942034)
\curveto(39.21181326,701.68941607)(39.29681318,701.65441611)(39.37681967,701.60442034)
\curveto(39.56681291,701.48441628)(39.73681274,701.35441641)(39.88681967,701.21442034)
\curveto(40.04681243,701.07441669)(40.1718123,700.89941686)(40.26181967,700.68942034)
\curveto(40.29181218,700.61941714)(40.31681216,700.54941721)(40.33681967,700.47942034)
\curveto(40.35681212,700.40941735)(40.3768121,700.33441743)(40.39681967,700.25442034)
\curveto(40.40681207,700.19441757)(40.41181206,700.09941766)(40.41181967,699.96942034)
\curveto(40.42181205,699.84941791)(40.42181205,699.75441801)(40.41181967,699.68442034)
\lineto(40.41181967,699.60942034)
\curveto(40.39181208,699.54941821)(40.3768121,699.48941827)(40.36681967,699.42942034)
\curveto(40.36681211,699.37941838)(40.36181211,699.32941843)(40.35181967,699.27942034)
\curveto(40.28181219,698.97941878)(40.1718123,698.71441905)(40.02181967,698.48442034)
\curveto(39.86181261,698.24441952)(39.66681281,698.04941971)(39.43681967,697.89942034)
\curveto(39.20681327,697.74942001)(38.94681353,697.61942014)(38.65681967,697.50942034)
\curveto(38.54681393,697.4594203)(38.42681405,697.42442034)(38.29681967,697.40442034)
\curveto(38.1768143,697.38442038)(38.05681442,697.3594204)(37.93681967,697.32942034)
\curveto(37.84681463,697.30942045)(37.75181472,697.29942046)(37.65181967,697.29942034)
\curveto(37.56181491,697.28942047)(37.471815,697.27442049)(37.38181967,697.25442034)
\lineto(37.11181967,697.25442034)
\curveto(37.05181542,697.23442053)(36.94681553,697.22442054)(36.79681967,697.22442034)
\curveto(36.65681582,697.22442054)(36.55681592,697.23442053)(36.49681967,697.25442034)
\curveto(36.46681601,697.25442051)(36.43181604,697.2594205)(36.39181967,697.26942034)
\lineto(36.28681967,697.26942034)
\curveto(36.16681631,697.28942047)(36.04681643,697.30442046)(35.92681967,697.31442034)
\curveto(35.80681667,697.32442044)(35.69181678,697.34442042)(35.58181967,697.37442034)
\curveto(35.19181728,697.48442028)(34.84681763,697.60942015)(34.54681967,697.74942034)
\curveto(34.24681823,697.89941986)(33.99181848,698.11941964)(33.78181967,698.40942034)
\curveto(33.64181883,698.59941916)(33.52181895,698.81941894)(33.42181967,699.06942034)
\curveto(33.40181907,699.12941863)(33.38181909,699.20941855)(33.36181967,699.30942034)
\curveto(33.34181913,699.3594184)(33.32681915,699.42941833)(33.31681967,699.51942034)
\curveto(33.30681917,699.60941815)(33.31181916,699.68441808)(33.33181967,699.74442034)
\curveto(33.36181911,699.81441795)(33.41181906,699.8644179)(33.48181967,699.89442034)
\curveto(33.53181894,699.91441785)(33.59181888,699.92441784)(33.66181967,699.92442034)
\lineto(33.88681967,699.92442034)
\lineto(34.59181967,699.92442034)
\lineto(34.83181967,699.92442034)
\curveto(34.91181756,699.92441784)(34.98181749,699.91441785)(35.04181967,699.89442034)
\curveto(35.15181732,699.85441791)(35.22181725,699.78941797)(35.25181967,699.69942034)
\curveto(35.29181718,699.60941815)(35.33681714,699.51441825)(35.38681967,699.41442034)
\curveto(35.40681707,699.3644184)(35.44181703,699.29941846)(35.49181967,699.21942034)
\curveto(35.55181692,699.13941862)(35.60181687,699.08941867)(35.64181967,699.06942034)
\curveto(35.76181671,698.96941879)(35.8768166,698.88941887)(35.98681967,698.82942034)
\curveto(36.09681638,698.77941898)(36.23681624,698.72941903)(36.40681967,698.67942034)
\curveto(36.45681602,698.6594191)(36.50681597,698.64941911)(36.55681967,698.64942034)
\curveto(36.60681587,698.6594191)(36.65681582,698.6594191)(36.70681967,698.64942034)
\curveto(36.78681569,698.62941913)(36.8718156,698.61941914)(36.96181967,698.61942034)
\curveto(37.06181541,698.62941913)(37.14681533,698.64441912)(37.21681967,698.66442034)
\curveto(37.26681521,698.67441909)(37.31181516,698.67941908)(37.35181967,698.67942034)
\curveto(37.40181507,698.67941908)(37.45181502,698.68941907)(37.50181967,698.70942034)
\curveto(37.64181483,698.759419)(37.76681471,698.81941894)(37.87681967,698.88942034)
\curveto(37.99681448,698.9594188)(38.09181438,699.04941871)(38.16181967,699.15942034)
\curveto(38.21181426,699.23941852)(38.25181422,699.3644184)(38.28181967,699.53442034)
\curveto(38.30181417,699.60441816)(38.30181417,699.66941809)(38.28181967,699.72942034)
\curveto(38.26181421,699.78941797)(38.24181423,699.83941792)(38.22181967,699.87942034)
\curveto(38.15181432,700.01941774)(38.06181441,700.12441764)(37.95181967,700.19442034)
\curveto(37.85181462,700.2644175)(37.73181474,700.32941743)(37.59181967,700.38942034)
\curveto(37.40181507,700.46941729)(37.20181527,700.53441723)(36.99181967,700.58442034)
\curveto(36.78181569,700.63441713)(36.5718159,700.68941707)(36.36181967,700.74942034)
\curveto(36.28181619,700.76941699)(36.19681628,700.78441698)(36.10681967,700.79442034)
\curveto(36.02681645,700.80441696)(35.94681653,700.81941694)(35.86681967,700.83942034)
\curveto(35.54681693,700.92941683)(35.24181723,701.01441675)(34.95181967,701.09442034)
\curveto(34.66181781,701.18441658)(34.39681808,701.31441645)(34.15681967,701.48442034)
\curveto(33.8768186,701.68441608)(33.6718188,701.95441581)(33.54181967,702.29442034)
\curveto(33.52181895,702.3644154)(33.50181897,702.4594153)(33.48181967,702.57942034)
\curveto(33.46181901,702.64941511)(33.44681903,702.73441503)(33.43681967,702.83442034)
\curveto(33.42681905,702.93441483)(33.43181904,703.02441474)(33.45181967,703.10442034)
\curveto(33.471819,703.15441461)(33.476819,703.19441457)(33.46681967,703.22442034)
\curveto(33.45681902,703.2644145)(33.46181901,703.30941445)(33.48181967,703.35942034)
\curveto(33.50181897,703.46941429)(33.52181895,703.56941419)(33.54181967,703.65942034)
\curveto(33.5718189,703.759414)(33.60681887,703.85441391)(33.64681967,703.94442034)
\curveto(33.7768187,704.23441353)(33.95681852,704.46941329)(34.18681967,704.64942034)
\curveto(34.41681806,704.82941293)(34.6768178,704.97441279)(34.96681967,705.08442034)
\curveto(35.0768174,705.13441263)(35.19181728,705.16941259)(35.31181967,705.18942034)
\curveto(35.43181704,705.21941254)(35.55681692,705.24941251)(35.68681967,705.27942034)
\curveto(35.74681673,705.29941246)(35.80681667,705.30941245)(35.86681967,705.30942034)
\lineto(36.04681967,705.33942034)
\curveto(36.12681635,705.34941241)(36.21181626,705.35441241)(36.30181967,705.35442034)
\curveto(36.39181608,705.35441241)(36.476816,705.3594124)(36.55681967,705.36942034)
}
}
{
\newrgbcolor{curcolor}{0 0 0}
\pscustom[linestyle=none,fillstyle=solid,fillcolor=curcolor]
{
\newpath
\moveto(42.69346029,707.46942034)
\lineto(43.69846029,707.46942034)
\curveto(43.84845731,707.46941029)(43.97845718,707.4594103)(44.08846029,707.43942034)
\curveto(44.20845695,707.42941033)(44.29345686,707.36941039)(44.34346029,707.25942034)
\curveto(44.36345679,707.20941055)(44.37345678,707.14941061)(44.37346029,707.07942034)
\lineto(44.37346029,706.86942034)
\lineto(44.37346029,706.19442034)
\curveto(44.37345678,706.14441162)(44.36845679,706.08441168)(44.35846029,706.01442034)
\curveto(44.3584568,705.95441181)(44.36345679,705.89941186)(44.37346029,705.84942034)
\lineto(44.37346029,705.68442034)
\curveto(44.37345678,705.60441216)(44.37845678,705.52941223)(44.38846029,705.45942034)
\curveto(44.39845676,705.39941236)(44.42345673,705.34441242)(44.46346029,705.29442034)
\curveto(44.53345662,705.20441256)(44.6584565,705.15441261)(44.83846029,705.14442034)
\lineto(45.37846029,705.14442034)
\lineto(45.55846029,705.14442034)
\curveto(45.61845554,705.14441262)(45.67345548,705.13441263)(45.72346029,705.11442034)
\curveto(45.83345532,705.0644127)(45.89345526,704.97441279)(45.90346029,704.84442034)
\curveto(45.92345523,704.71441305)(45.93345522,704.56941319)(45.93346029,704.40942034)
\lineto(45.93346029,704.19942034)
\curveto(45.94345521,704.12941363)(45.93845522,704.06941369)(45.91846029,704.01942034)
\curveto(45.86845529,703.8594139)(45.76345539,703.77441399)(45.60346029,703.76442034)
\curveto(45.44345571,703.75441401)(45.26345589,703.74941401)(45.06346029,703.74942034)
\lineto(44.92846029,703.74942034)
\curveto(44.88845627,703.759414)(44.8534563,703.759414)(44.82346029,703.74942034)
\curveto(44.78345637,703.73941402)(44.74845641,703.73441403)(44.71846029,703.73442034)
\curveto(44.68845647,703.74441402)(44.6584565,703.73941402)(44.62846029,703.71942034)
\curveto(44.54845661,703.69941406)(44.48845667,703.65441411)(44.44846029,703.58442034)
\curveto(44.41845674,703.52441424)(44.39345676,703.44941431)(44.37346029,703.35942034)
\curveto(44.36345679,703.30941445)(44.36345679,703.25441451)(44.37346029,703.19442034)
\curveto(44.38345677,703.13441463)(44.38345677,703.07941468)(44.37346029,703.02942034)
\lineto(44.37346029,702.09942034)
\lineto(44.37346029,700.34442034)
\curveto(44.37345678,700.09441767)(44.37845678,699.87441789)(44.38846029,699.68442034)
\curveto(44.40845675,699.50441826)(44.47345668,699.34441842)(44.58346029,699.20442034)
\curveto(44.63345652,699.14441862)(44.69845646,699.09941866)(44.77846029,699.06942034)
\lineto(45.04846029,699.00942034)
\curveto(45.07845608,698.99941876)(45.10845605,698.99441877)(45.13846029,698.99442034)
\curveto(45.17845598,699.00441876)(45.20845595,699.00441876)(45.22846029,698.99442034)
\lineto(45.39346029,698.99442034)
\curveto(45.50345565,698.99441877)(45.59845556,698.98941877)(45.67846029,698.97942034)
\curveto(45.7584554,698.96941879)(45.82345533,698.92941883)(45.87346029,698.85942034)
\curveto(45.91345524,698.79941896)(45.93345522,698.71941904)(45.93346029,698.61942034)
\lineto(45.93346029,698.33442034)
\curveto(45.93345522,698.12441964)(45.92845523,697.92941983)(45.91846029,697.74942034)
\curveto(45.91845524,697.57942018)(45.83845532,697.4644203)(45.67846029,697.40442034)
\curveto(45.62845553,697.38442038)(45.58345557,697.37942038)(45.54346029,697.38942034)
\curveto(45.50345565,697.38942037)(45.4584557,697.37942038)(45.40846029,697.35942034)
\lineto(45.25846029,697.35942034)
\curveto(45.23845592,697.3594204)(45.20845595,697.3644204)(45.16846029,697.37442034)
\curveto(45.12845603,697.37442039)(45.09345606,697.36942039)(45.06346029,697.35942034)
\curveto(45.01345614,697.34942041)(44.9584562,697.34942041)(44.89846029,697.35942034)
\lineto(44.74846029,697.35942034)
\lineto(44.59846029,697.35942034)
\curveto(44.54845661,697.34942041)(44.50345665,697.34942041)(44.46346029,697.35942034)
\lineto(44.29846029,697.35942034)
\curveto(44.24845691,697.36942039)(44.19345696,697.37442039)(44.13346029,697.37442034)
\curveto(44.07345708,697.37442039)(44.01845714,697.37942038)(43.96846029,697.38942034)
\curveto(43.89845726,697.39942036)(43.83345732,697.40942035)(43.77346029,697.41942034)
\lineto(43.59346029,697.44942034)
\curveto(43.48345767,697.47942028)(43.37845778,697.51442025)(43.27846029,697.55442034)
\curveto(43.17845798,697.59442017)(43.08345807,697.63942012)(42.99346029,697.68942034)
\lineto(42.90346029,697.74942034)
\curveto(42.87345828,697.77941998)(42.83845832,697.80941995)(42.79846029,697.83942034)
\curveto(42.77845838,697.8594199)(42.7534584,697.87941988)(42.72346029,697.89942034)
\lineto(42.64846029,697.97442034)
\curveto(42.50845865,698.1644196)(42.40345875,698.37441939)(42.33346029,698.60442034)
\curveto(42.31345884,698.64441912)(42.30345885,698.67941908)(42.30346029,698.70942034)
\curveto(42.31345884,698.74941901)(42.31345884,698.79441897)(42.30346029,698.84442034)
\curveto(42.29345886,698.8644189)(42.28845887,698.88941887)(42.28846029,698.91942034)
\curveto(42.28845887,698.94941881)(42.28345887,698.97441879)(42.27346029,698.99442034)
\lineto(42.27346029,699.14442034)
\curveto(42.26345889,699.18441858)(42.2584589,699.22941853)(42.25846029,699.27942034)
\curveto(42.26845889,699.32941843)(42.27345888,699.37941838)(42.27346029,699.42942034)
\lineto(42.27346029,699.99942034)
\lineto(42.27346029,702.23442034)
\lineto(42.27346029,703.02942034)
\lineto(42.27346029,703.23942034)
\curveto(42.28345887,703.30941445)(42.27845888,703.37441439)(42.25846029,703.43442034)
\curveto(42.21845894,703.57441419)(42.14845901,703.6644141)(42.04846029,703.70442034)
\curveto(41.93845922,703.75441401)(41.79845936,703.76941399)(41.62846029,703.74942034)
\curveto(41.4584597,703.72941403)(41.31345984,703.74441402)(41.19346029,703.79442034)
\curveto(41.11346004,703.82441394)(41.06346009,703.86941389)(41.04346029,703.92942034)
\curveto(41.02346013,703.98941377)(41.00346015,704.0644137)(40.98346029,704.15442034)
\lineto(40.98346029,704.46942034)
\curveto(40.98346017,704.64941311)(40.99346016,704.79441297)(41.01346029,704.90442034)
\curveto(41.03346012,705.01441275)(41.11846004,705.08941267)(41.26846029,705.12942034)
\curveto(41.30845985,705.14941261)(41.34845981,705.15441261)(41.38846029,705.14442034)
\lineto(41.52346029,705.14442034)
\curveto(41.67345948,705.14441262)(41.81345934,705.14941261)(41.94346029,705.15942034)
\curveto(42.07345908,705.17941258)(42.16345899,705.23941252)(42.21346029,705.33942034)
\curveto(42.24345891,705.40941235)(42.2584589,705.48941227)(42.25846029,705.57942034)
\curveto(42.26845889,705.66941209)(42.27345888,705.759412)(42.27346029,705.84942034)
\lineto(42.27346029,706.77942034)
\lineto(42.27346029,707.03442034)
\curveto(42.27345888,707.12441064)(42.28345887,707.19941056)(42.30346029,707.25942034)
\curveto(42.3534588,707.3594104)(42.42845873,707.42441034)(42.52846029,707.45442034)
\curveto(42.54845861,707.4644103)(42.57345858,707.4644103)(42.60346029,707.45442034)
\curveto(42.64345851,707.45441031)(42.67345848,707.4594103)(42.69346029,707.46942034)
}
}
{
\newrgbcolor{curcolor}{0 0 0}
\pscustom[linestyle=none,fillstyle=solid,fillcolor=curcolor]
{
\newpath
\moveto(51.34189779,705.35442034)
\curveto(51.45189248,705.35441241)(51.54689238,705.34441242)(51.62689779,705.32442034)
\curveto(51.71689221,705.30441246)(51.78689214,705.2594125)(51.83689779,705.18942034)
\curveto(51.89689203,705.10941265)(51.926892,704.96941279)(51.92689779,704.76942034)
\lineto(51.92689779,704.25942034)
\lineto(51.92689779,703.88442034)
\curveto(51.93689199,703.74441402)(51.92189201,703.63441413)(51.88189779,703.55442034)
\curveto(51.84189209,703.48441428)(51.78189215,703.43941432)(51.70189779,703.41942034)
\curveto(51.6318923,703.39941436)(51.54689238,703.38941437)(51.44689779,703.38942034)
\curveto(51.35689257,703.38941437)(51.25689267,703.39441437)(51.14689779,703.40442034)
\curveto(51.04689288,703.41441435)(50.95189298,703.40941435)(50.86189779,703.38942034)
\curveto(50.79189314,703.36941439)(50.72189321,703.35441441)(50.65189779,703.34442034)
\curveto(50.58189335,703.34441442)(50.51689341,703.33441443)(50.45689779,703.31442034)
\curveto(50.29689363,703.2644145)(50.13689379,703.18941457)(49.97689779,703.08942034)
\curveto(49.81689411,702.99941476)(49.69189424,702.89441487)(49.60189779,702.77442034)
\curveto(49.55189438,702.69441507)(49.49689443,702.60941515)(49.43689779,702.51942034)
\curveto(49.38689454,702.43941532)(49.33689459,702.35441541)(49.28689779,702.26442034)
\curveto(49.25689467,702.18441558)(49.2268947,702.09941566)(49.19689779,702.00942034)
\lineto(49.13689779,701.76942034)
\curveto(49.11689481,701.69941606)(49.10689482,701.62441614)(49.10689779,701.54442034)
\curveto(49.10689482,701.47441629)(49.09689483,701.40441636)(49.07689779,701.33442034)
\curveto(49.06689486,701.29441647)(49.06189487,701.25441651)(49.06189779,701.21442034)
\curveto(49.07189486,701.18441658)(49.07189486,701.15441661)(49.06189779,701.12442034)
\lineto(49.06189779,700.88442034)
\curveto(49.04189489,700.81441695)(49.03689489,700.73441703)(49.04689779,700.64442034)
\curveto(49.05689487,700.5644172)(49.06189487,700.48441728)(49.06189779,700.40442034)
\lineto(49.06189779,699.44442034)
\lineto(49.06189779,698.16942034)
\curveto(49.06189487,698.03941972)(49.05689487,697.91941984)(49.04689779,697.80942034)
\curveto(49.03689489,697.69942006)(49.00689492,697.60942015)(48.95689779,697.53942034)
\curveto(48.93689499,697.50942025)(48.90189503,697.48442028)(48.85189779,697.46442034)
\curveto(48.81189512,697.45442031)(48.76689516,697.44442032)(48.71689779,697.43442034)
\lineto(48.64189779,697.43442034)
\curveto(48.59189534,697.42442034)(48.53689539,697.41942034)(48.47689779,697.41942034)
\lineto(48.31189779,697.41942034)
\lineto(47.66689779,697.41942034)
\curveto(47.60689632,697.42942033)(47.54189639,697.43442033)(47.47189779,697.43442034)
\lineto(47.27689779,697.43442034)
\curveto(47.2268967,697.45442031)(47.17689675,697.46942029)(47.12689779,697.47942034)
\curveto(47.07689685,697.49942026)(47.04189689,697.53442023)(47.02189779,697.58442034)
\curveto(46.98189695,697.63442013)(46.95689697,697.70442006)(46.94689779,697.79442034)
\lineto(46.94689779,698.09442034)
\lineto(46.94689779,699.11442034)
\lineto(46.94689779,703.34442034)
\lineto(46.94689779,704.45442034)
\lineto(46.94689779,704.73942034)
\curveto(46.94689698,704.83941292)(46.96689696,704.91941284)(47.00689779,704.97942034)
\curveto(47.05689687,705.0594127)(47.1318968,705.10941265)(47.23189779,705.12942034)
\curveto(47.3318966,705.14941261)(47.45189648,705.1594126)(47.59189779,705.15942034)
\lineto(48.35689779,705.15942034)
\curveto(48.47689545,705.1594126)(48.58189535,705.14941261)(48.67189779,705.12942034)
\curveto(48.76189517,705.11941264)(48.8318951,705.07441269)(48.88189779,704.99442034)
\curveto(48.91189502,704.94441282)(48.926895,704.87441289)(48.92689779,704.78442034)
\lineto(48.95689779,704.51442034)
\curveto(48.96689496,704.43441333)(48.98189495,704.3594134)(49.00189779,704.28942034)
\curveto(49.0318949,704.21941354)(49.08189485,704.18441358)(49.15189779,704.18442034)
\curveto(49.17189476,704.20441356)(49.19189474,704.21441355)(49.21189779,704.21442034)
\curveto(49.2318947,704.21441355)(49.25189468,704.22441354)(49.27189779,704.24442034)
\curveto(49.3318946,704.29441347)(49.38189455,704.34941341)(49.42189779,704.40942034)
\curveto(49.47189446,704.47941328)(49.5318944,704.53941322)(49.60189779,704.58942034)
\curveto(49.64189429,704.61941314)(49.67689425,704.64941311)(49.70689779,704.67942034)
\curveto(49.73689419,704.71941304)(49.77189416,704.75441301)(49.81189779,704.78442034)
\lineto(50.08189779,704.96442034)
\curveto(50.18189375,705.02441274)(50.28189365,705.07941268)(50.38189779,705.12942034)
\curveto(50.48189345,705.16941259)(50.58189335,705.20441256)(50.68189779,705.23442034)
\lineto(51.01189779,705.32442034)
\curveto(51.04189289,705.33441243)(51.09689283,705.33441243)(51.17689779,705.32442034)
\curveto(51.26689266,705.32441244)(51.32189261,705.33441243)(51.34189779,705.35442034)
}
}
{
\newrgbcolor{curcolor}{0 0 0}
\pscustom[linestyle=none,fillstyle=solid,fillcolor=curcolor]
{
\newpath
\moveto(54.84697592,708.00942034)
\curveto(54.91697297,707.92940983)(54.95197293,707.80940995)(54.95197592,707.64942034)
\lineto(54.95197592,707.18442034)
\lineto(54.95197592,706.77942034)
\curveto(54.95197293,706.63941112)(54.91697297,706.54441122)(54.84697592,706.49442034)
\curveto(54.7869731,706.44441132)(54.70697318,706.41441135)(54.60697592,706.40442034)
\curveto(54.51697337,706.39441137)(54.41697347,706.38941137)(54.30697592,706.38942034)
\lineto(53.46697592,706.38942034)
\curveto(53.35697453,706.38941137)(53.25697463,706.39441137)(53.16697592,706.40442034)
\curveto(53.0869748,706.41441135)(53.01697487,706.44441132)(52.95697592,706.49442034)
\curveto(52.91697497,706.52441124)(52.886975,706.57941118)(52.86697592,706.65942034)
\curveto(52.85697503,706.74941101)(52.84697504,706.84441092)(52.83697592,706.94442034)
\lineto(52.83697592,707.27442034)
\curveto(52.84697504,707.38441038)(52.85197503,707.47941028)(52.85197592,707.55942034)
\lineto(52.85197592,707.76942034)
\curveto(52.86197502,707.83940992)(52.881975,707.89940986)(52.91197592,707.94942034)
\curveto(52.93197495,707.98940977)(52.95697493,708.01940974)(52.98697592,708.03942034)
\lineto(53.10697592,708.09942034)
\curveto(53.12697476,708.09940966)(53.15197473,708.09940966)(53.18197592,708.09942034)
\curveto(53.21197467,708.10940965)(53.23697465,708.11440965)(53.25697592,708.11442034)
\lineto(54.35197592,708.11442034)
\curveto(54.45197343,708.11440965)(54.54697334,708.10940965)(54.63697592,708.09942034)
\curveto(54.72697316,708.08940967)(54.79697309,708.0594097)(54.84697592,708.00942034)
\moveto(54.95197592,698.24442034)
\curveto(54.95197293,698.04441972)(54.94697294,697.87441989)(54.93697592,697.73442034)
\curveto(54.92697296,697.59442017)(54.83697305,697.49942026)(54.66697592,697.44942034)
\curveto(54.60697328,697.42942033)(54.54197334,697.41942034)(54.47197592,697.41942034)
\curveto(54.40197348,697.42942033)(54.32697356,697.43442033)(54.24697592,697.43442034)
\lineto(53.40697592,697.43442034)
\curveto(53.31697457,697.43442033)(53.22697466,697.43942032)(53.13697592,697.44942034)
\curveto(53.05697483,697.4594203)(52.99697489,697.48942027)(52.95697592,697.53942034)
\curveto(52.89697499,697.60942015)(52.86197502,697.69442007)(52.85197592,697.79442034)
\lineto(52.85197592,698.13942034)
\lineto(52.85197592,704.46942034)
\lineto(52.85197592,704.76942034)
\curveto(52.85197503,704.86941289)(52.87197501,704.94941281)(52.91197592,705.00942034)
\curveto(52.97197491,705.07941268)(53.05697483,705.12441264)(53.16697592,705.14442034)
\curveto(53.1869747,705.15441261)(53.21197467,705.15441261)(53.24197592,705.14442034)
\curveto(53.2819746,705.14441262)(53.31197457,705.14941261)(53.33197592,705.15942034)
\lineto(54.08197592,705.15942034)
\lineto(54.27697592,705.15942034)
\curveto(54.35697353,705.16941259)(54.42197346,705.16941259)(54.47197592,705.15942034)
\lineto(54.59197592,705.15942034)
\curveto(54.65197323,705.13941262)(54.70697318,705.12441264)(54.75697592,705.11442034)
\curveto(54.80697308,705.10441266)(54.84697304,705.07441269)(54.87697592,705.02442034)
\curveto(54.91697297,704.97441279)(54.93697295,704.90441286)(54.93697592,704.81442034)
\curveto(54.94697294,704.72441304)(54.95197293,704.62941313)(54.95197592,704.52942034)
\lineto(54.95197592,698.24442034)
}
}
{
\newrgbcolor{curcolor}{0 0 0}
\pscustom[linestyle=none,fillstyle=solid,fillcolor=curcolor]
{
\newpath
\moveto(64.41416342,701.67942034)
\curveto(64.43415482,701.61941614)(64.44415481,701.51441625)(64.44416342,701.36442034)
\curveto(64.44415481,701.22441654)(64.43915481,701.12441664)(64.42916342,701.06442034)
\curveto(64.42915482,701.01441675)(64.42415483,700.96941679)(64.41416342,700.92942034)
\lineto(64.41416342,700.80942034)
\curveto(64.39415486,700.72941703)(64.38415487,700.64941711)(64.38416342,700.56942034)
\curveto(64.38415487,700.49941726)(64.37415488,700.42441734)(64.35416342,700.34442034)
\curveto(64.3541549,700.30441746)(64.34415491,700.23441753)(64.32416342,700.13442034)
\curveto(64.29415496,700.01441775)(64.26415499,699.88941787)(64.23416342,699.75942034)
\curveto(64.21415504,699.63941812)(64.17915507,699.52441824)(64.12916342,699.41442034)
\curveto(63.9491553,698.9644188)(63.72415553,698.57441919)(63.45416342,698.24442034)
\curveto(63.18415607,697.91441985)(62.82915642,697.65442011)(62.38916342,697.46442034)
\curveto(62.29915695,697.42442034)(62.20415705,697.39442037)(62.10416342,697.37442034)
\curveto(62.01415724,697.34442042)(61.91415734,697.31442045)(61.80416342,697.28442034)
\curveto(61.74415751,697.2644205)(61.67915757,697.25442051)(61.60916342,697.25442034)
\curveto(61.5491577,697.25442051)(61.48915776,697.24942051)(61.42916342,697.23942034)
\lineto(61.29416342,697.23942034)
\curveto(61.23415802,697.21942054)(61.1541581,697.21442055)(61.05416342,697.22442034)
\curveto(60.9541583,697.22442054)(60.87415838,697.23442053)(60.81416342,697.25442034)
\lineto(60.72416342,697.25442034)
\curveto(60.67415858,697.2644205)(60.61915863,697.27442049)(60.55916342,697.28442034)
\curveto(60.49915875,697.28442048)(60.43915881,697.28942047)(60.37916342,697.29942034)
\curveto(60.18915906,697.34942041)(60.01415924,697.39942036)(59.85416342,697.44942034)
\curveto(59.69415956,697.49942026)(59.54415971,697.56942019)(59.40416342,697.65942034)
\lineto(59.22416342,697.77942034)
\curveto(59.17416008,697.81941994)(59.12416013,697.8644199)(59.07416342,697.91442034)
\lineto(58.98416342,697.97442034)
\curveto(58.9541603,697.99441977)(58.92416033,698.00941975)(58.89416342,698.01942034)
\curveto(58.80416045,698.04941971)(58.7491605,698.02941973)(58.72916342,697.95942034)
\curveto(58.67916057,697.88941987)(58.64416061,697.80441996)(58.62416342,697.70442034)
\curveto(58.61416064,697.61442015)(58.57916067,697.54442022)(58.51916342,697.49442034)
\curveto(58.45916079,697.45442031)(58.38916086,697.42942033)(58.30916342,697.41942034)
\lineto(58.03916342,697.41942034)
\lineto(57.31916342,697.41942034)
\lineto(57.09416342,697.41942034)
\curveto(57.02416223,697.40942035)(56.95916229,697.41442035)(56.89916342,697.43442034)
\curveto(56.75916249,697.48442028)(56.67916257,697.57442019)(56.65916342,697.70442034)
\curveto(56.6491626,697.84441992)(56.64416261,697.99941976)(56.64416342,698.16942034)
\lineto(56.64416342,707.31942034)
\lineto(56.64416342,707.66442034)
\curveto(56.64416261,707.78440998)(56.66916258,707.87940988)(56.71916342,707.94942034)
\curveto(56.75916249,708.01940974)(56.82916242,708.0644097)(56.92916342,708.08442034)
\curveto(56.9491623,708.09440967)(56.96916228,708.09440967)(56.98916342,708.08442034)
\curveto(57.01916223,708.08440968)(57.04416221,708.08940967)(57.06416342,708.09942034)
\lineto(58.00916342,708.09942034)
\curveto(58.18916106,708.09940966)(58.34416091,708.08940967)(58.47416342,708.06942034)
\curveto(58.60416065,708.0594097)(58.68916056,707.98440978)(58.72916342,707.84442034)
\curveto(58.75916049,707.74441002)(58.76916048,707.60941015)(58.75916342,707.43942034)
\curveto(58.7491605,707.27941048)(58.74416051,707.13941062)(58.74416342,707.01942034)
\lineto(58.74416342,705.38442034)
\lineto(58.74416342,705.05442034)
\curveto(58.74416051,704.94441282)(58.7541605,704.84941291)(58.77416342,704.76942034)
\curveto(58.78416047,704.71941304)(58.79416046,704.67441309)(58.80416342,704.63442034)
\curveto(58.81416044,704.60441316)(58.83916041,704.58441318)(58.87916342,704.57442034)
\curveto(58.89916035,704.55441321)(58.92416033,704.54441322)(58.95416342,704.54442034)
\curveto(58.99416026,704.54441322)(59.02416023,704.54941321)(59.04416342,704.55942034)
\curveto(59.11416014,704.59941316)(59.17916007,704.63941312)(59.23916342,704.67942034)
\curveto(59.29915995,704.72941303)(59.36415989,704.77941298)(59.43416342,704.82942034)
\curveto(59.56415969,704.91941284)(59.69915955,704.99441277)(59.83916342,705.05442034)
\curveto(59.97915927,705.12441264)(60.13415912,705.18441258)(60.30416342,705.23442034)
\curveto(60.38415887,705.2644125)(60.46415879,705.27941248)(60.54416342,705.27942034)
\curveto(60.62415863,705.28941247)(60.70415855,705.30441246)(60.78416342,705.32442034)
\curveto(60.8541584,705.34441242)(60.92915832,705.35441241)(61.00916342,705.35442034)
\lineto(61.24916342,705.35442034)
\lineto(61.39916342,705.35442034)
\curveto(61.42915782,705.34441242)(61.46415779,705.33941242)(61.50416342,705.33942034)
\curveto(61.54415771,705.34941241)(61.58415767,705.34941241)(61.62416342,705.33942034)
\curveto(61.73415752,705.30941245)(61.83415742,705.28441248)(61.92416342,705.26442034)
\curveto(62.02415723,705.25441251)(62.11915713,705.22941253)(62.20916342,705.18942034)
\curveto(62.66915658,704.99941276)(63.04415621,704.75441301)(63.33416342,704.45442034)
\curveto(63.62415563,704.15441361)(63.86915538,703.77941398)(64.06916342,703.32942034)
\curveto(64.11915513,703.20941455)(64.15915509,703.08441468)(64.18916342,702.95442034)
\curveto(64.22915502,702.82441494)(64.26915498,702.68941507)(64.30916342,702.54942034)
\curveto(64.32915492,702.47941528)(64.33915491,702.40941535)(64.33916342,702.33942034)
\curveto(64.3491549,702.27941548)(64.36415489,702.20941555)(64.38416342,702.12942034)
\curveto(64.40415485,702.07941568)(64.40915484,702.02441574)(64.39916342,701.96442034)
\curveto(64.39915485,701.90441586)(64.40415485,701.84441592)(64.41416342,701.78442034)
\lineto(64.41416342,701.67942034)
\moveto(62.19416342,700.26942034)
\curveto(62.22415703,700.36941739)(62.249157,700.49441727)(62.26916342,700.64442034)
\curveto(62.29915695,700.79441697)(62.31415694,700.94441682)(62.31416342,701.09442034)
\curveto(62.32415693,701.25441651)(62.32415693,701.40941635)(62.31416342,701.55942034)
\curveto(62.31415694,701.71941604)(62.29915695,701.85441591)(62.26916342,701.96442034)
\curveto(62.23915701,702.0644157)(62.21915703,702.1594156)(62.20916342,702.24942034)
\curveto(62.19915705,702.33941542)(62.17415708,702.42441534)(62.13416342,702.50442034)
\curveto(61.99415726,702.85441491)(61.79415746,703.14941461)(61.53416342,703.38942034)
\curveto(61.28415797,703.63941412)(60.91415834,703.764414)(60.42416342,703.76442034)
\curveto(60.38415887,703.764414)(60.3491589,703.759414)(60.31916342,703.74942034)
\lineto(60.21416342,703.74942034)
\curveto(60.14415911,703.72941403)(60.07915917,703.70941405)(60.01916342,703.68942034)
\curveto(59.95915929,703.67941408)(59.89915935,703.6644141)(59.83916342,703.64442034)
\curveto(59.5491597,703.51441425)(59.32915992,703.32941443)(59.17916342,703.08942034)
\curveto(59.02916022,702.8594149)(58.90416035,702.59441517)(58.80416342,702.29442034)
\curveto(58.77416048,702.21441555)(58.7541605,702.12941563)(58.74416342,702.03942034)
\curveto(58.74416051,701.9594158)(58.73416052,701.87941588)(58.71416342,701.79942034)
\curveto(58.70416055,701.76941599)(58.69916055,701.71941604)(58.69916342,701.64942034)
\curveto(58.68916056,701.60941615)(58.68416057,701.56941619)(58.68416342,701.52942034)
\curveto(58.69416056,701.48941627)(58.69416056,701.44941631)(58.68416342,701.40942034)
\curveto(58.66416059,701.32941643)(58.65916059,701.21941654)(58.66916342,701.07942034)
\curveto(58.67916057,700.93941682)(58.69416056,700.83941692)(58.71416342,700.77942034)
\curveto(58.73416052,700.68941707)(58.74416051,700.60441716)(58.74416342,700.52442034)
\curveto(58.7541605,700.44441732)(58.77416048,700.3644174)(58.80416342,700.28442034)
\curveto(58.89416036,700.00441776)(58.99916025,699.759418)(59.11916342,699.54942034)
\curveto(59.24916,699.34941841)(59.42915982,699.17941858)(59.65916342,699.03942034)
\curveto(59.81915943,698.93941882)(59.98415927,698.86941889)(60.15416342,698.82942034)
\curveto(60.17415908,698.82941893)(60.19415906,698.82441894)(60.21416342,698.81442034)
\lineto(60.30416342,698.81442034)
\curveto(60.33415892,698.80441896)(60.38415887,698.79441897)(60.45416342,698.78442034)
\curveto(60.52415873,698.78441898)(60.58415867,698.78941897)(60.63416342,698.79942034)
\curveto(60.73415852,698.81941894)(60.82415843,698.83441893)(60.90416342,698.84442034)
\curveto(60.99415826,698.8644189)(61.07915817,698.88941887)(61.15916342,698.91942034)
\curveto(61.43915781,699.04941871)(61.6541576,699.22941853)(61.80416342,699.45942034)
\curveto(61.96415729,699.68941807)(62.09415716,699.9594178)(62.19416342,700.26942034)
}
}
{
\newrgbcolor{curcolor}{0 0 0}
\pscustom[linestyle=none,fillstyle=solid,fillcolor=curcolor]
{
\newpath
\moveto(66.20408529,705.14442034)
\lineto(67.32908529,705.14442034)
\curveto(67.43908286,705.14441262)(67.53908276,705.13941262)(67.62908529,705.12942034)
\curveto(67.71908258,705.11941264)(67.78408251,705.08441268)(67.82408529,705.02442034)
\curveto(67.87408242,704.9644128)(67.90408239,704.87941288)(67.91408529,704.76942034)
\curveto(67.92408237,704.66941309)(67.92908237,704.5644132)(67.92908529,704.45442034)
\lineto(67.92908529,703.40442034)
\lineto(67.92908529,701.16942034)
\curveto(67.92908237,700.80941695)(67.94408235,700.46941729)(67.97408529,700.14942034)
\curveto(68.00408229,699.82941793)(68.0940822,699.5644182)(68.24408529,699.35442034)
\curveto(68.38408191,699.14441862)(68.60908169,698.99441877)(68.91908529,698.90442034)
\curveto(68.96908133,698.89441887)(69.00908129,698.88941887)(69.03908529,698.88942034)
\curveto(69.07908122,698.88941887)(69.12408117,698.88441888)(69.17408529,698.87442034)
\curveto(69.22408107,698.8644189)(69.27908102,698.8594189)(69.33908529,698.85942034)
\curveto(69.3990809,698.8594189)(69.44408085,698.8644189)(69.47408529,698.87442034)
\curveto(69.52408077,698.89441887)(69.56408073,698.89941886)(69.59408529,698.88942034)
\curveto(69.63408066,698.87941888)(69.67408062,698.88441888)(69.71408529,698.90442034)
\curveto(69.92408037,698.95441881)(70.08908021,699.01941874)(70.20908529,699.09942034)
\curveto(70.38907991,699.20941855)(70.52907977,699.34941841)(70.62908529,699.51942034)
\curveto(70.73907956,699.69941806)(70.81407948,699.89441787)(70.85408529,700.10442034)
\curveto(70.90407939,700.32441744)(70.93407936,700.5644172)(70.94408529,700.82442034)
\curveto(70.95407934,701.09441667)(70.95907934,701.37441639)(70.95908529,701.66442034)
\lineto(70.95908529,703.47942034)
\lineto(70.95908529,704.45442034)
\lineto(70.95908529,704.72442034)
\curveto(70.95907934,704.82441294)(70.97907932,704.90441286)(71.01908529,704.96442034)
\curveto(71.06907923,705.05441271)(71.14407915,705.10441266)(71.24408529,705.11442034)
\curveto(71.34407895,705.13441263)(71.46407883,705.14441262)(71.60408529,705.14442034)
\lineto(72.39908529,705.14442034)
\lineto(72.68408529,705.14442034)
\curveto(72.77407752,705.14441262)(72.84907745,705.12441264)(72.90908529,705.08442034)
\curveto(72.98907731,705.03441273)(73.03407726,704.9594128)(73.04408529,704.85942034)
\curveto(73.05407724,704.759413)(73.05907724,704.64441312)(73.05908529,704.51442034)
\lineto(73.05908529,703.37442034)
\lineto(73.05908529,699.15942034)
\lineto(73.05908529,698.09442034)
\lineto(73.05908529,697.79442034)
\curveto(73.05907724,697.69442007)(73.03907726,697.61942014)(72.99908529,697.56942034)
\curveto(72.94907735,697.48942027)(72.87407742,697.44442032)(72.77408529,697.43442034)
\curveto(72.67407762,697.42442034)(72.56907773,697.41942034)(72.45908529,697.41942034)
\lineto(71.64908529,697.41942034)
\curveto(71.53907876,697.41942034)(71.43907886,697.42442034)(71.34908529,697.43442034)
\curveto(71.26907903,697.44442032)(71.20407909,697.48442028)(71.15408529,697.55442034)
\curveto(71.13407916,697.58442018)(71.11407918,697.62942013)(71.09408529,697.68942034)
\curveto(71.08407921,697.74942001)(71.06907923,697.80941995)(71.04908529,697.86942034)
\curveto(71.03907926,697.92941983)(71.02407927,697.98441978)(71.00408529,698.03442034)
\curveto(70.98407931,698.08441968)(70.95407934,698.11441965)(70.91408529,698.12442034)
\curveto(70.8940794,698.14441962)(70.86907943,698.14941961)(70.83908529,698.13942034)
\curveto(70.80907949,698.12941963)(70.78407951,698.11941964)(70.76408529,698.10942034)
\curveto(70.6940796,698.06941969)(70.63407966,698.02441974)(70.58408529,697.97442034)
\curveto(70.53407976,697.92441984)(70.47907982,697.87941988)(70.41908529,697.83942034)
\curveto(70.37907992,697.80941995)(70.33907996,697.77441999)(70.29908529,697.73442034)
\curveto(70.26908003,697.70442006)(70.22908007,697.67442009)(70.17908529,697.64442034)
\curveto(69.94908035,697.50442026)(69.67908062,697.39442037)(69.36908529,697.31442034)
\curveto(69.299081,697.29442047)(69.22908107,697.28442048)(69.15908529,697.28442034)
\curveto(69.08908121,697.27442049)(69.01408128,697.2594205)(68.93408529,697.23942034)
\curveto(68.8940814,697.22942053)(68.84908145,697.22942053)(68.79908529,697.23942034)
\curveto(68.75908154,697.23942052)(68.71908158,697.23442053)(68.67908529,697.22442034)
\curveto(68.64908165,697.21442055)(68.58408171,697.21442055)(68.48408529,697.22442034)
\curveto(68.3940819,697.22442054)(68.33408196,697.22942053)(68.30408529,697.23942034)
\curveto(68.25408204,697.23942052)(68.20408209,697.24442052)(68.15408529,697.25442034)
\lineto(68.00408529,697.25442034)
\curveto(67.88408241,697.28442048)(67.76908253,697.30942045)(67.65908529,697.32942034)
\curveto(67.54908275,697.34942041)(67.43908286,697.37942038)(67.32908529,697.41942034)
\curveto(67.27908302,697.43942032)(67.23408306,697.45442031)(67.19408529,697.46442034)
\curveto(67.16408313,697.48442028)(67.12408317,697.50442026)(67.07408529,697.52442034)
\curveto(66.72408357,697.71442005)(66.44408385,697.97941978)(66.23408529,698.31942034)
\curveto(66.10408419,698.52941923)(66.00908429,698.77941898)(65.94908529,699.06942034)
\curveto(65.88908441,699.36941839)(65.84908445,699.68441808)(65.82908529,700.01442034)
\curveto(65.81908448,700.35441741)(65.81408448,700.69941706)(65.81408529,701.04942034)
\curveto(65.82408447,701.40941635)(65.82908447,701.764416)(65.82908529,702.11442034)
\lineto(65.82908529,704.15442034)
\curveto(65.82908447,704.28441348)(65.82408447,704.43441333)(65.81408529,704.60442034)
\curveto(65.81408448,704.78441298)(65.83908446,704.91441285)(65.88908529,704.99442034)
\curveto(65.91908438,705.04441272)(65.97908432,705.08941267)(66.06908529,705.12942034)
\curveto(66.12908417,705.12941263)(66.17408412,705.13441263)(66.20408529,705.14442034)
}
}
{
\newrgbcolor{curcolor}{0 0 0}
\pscustom[linestyle=none,fillstyle=solid,fillcolor=curcolor]
{
\newpath
\moveto(78.26033529,705.36942034)
\curveto(79.07033013,705.38941237)(79.74532946,705.26941249)(80.28533529,705.00942034)
\curveto(80.83532837,704.74941301)(81.27032793,704.37941338)(81.59033529,703.89942034)
\curveto(81.75032745,703.6594141)(81.87032733,703.38441438)(81.95033529,703.07442034)
\curveto(81.97032723,703.02441474)(81.98532722,702.9594148)(81.99533529,702.87942034)
\curveto(82.01532719,702.79941496)(82.01532719,702.72941503)(81.99533529,702.66942034)
\curveto(81.95532725,702.5594152)(81.88532732,702.49441527)(81.78533529,702.47442034)
\curveto(81.68532752,702.4644153)(81.56532764,702.4594153)(81.42533529,702.45942034)
\lineto(80.64533529,702.45942034)
\lineto(80.36033529,702.45942034)
\curveto(80.27032893,702.4594153)(80.19532901,702.47941528)(80.13533529,702.51942034)
\curveto(80.05532915,702.5594152)(80.0003292,702.61941514)(79.97033529,702.69942034)
\curveto(79.94032926,702.78941497)(79.9003293,702.87941488)(79.85033529,702.96942034)
\curveto(79.79032941,703.07941468)(79.72532948,703.17941458)(79.65533529,703.26942034)
\curveto(79.58532962,703.3594144)(79.5053297,703.43941432)(79.41533529,703.50942034)
\curveto(79.27532993,703.59941416)(79.12033008,703.66941409)(78.95033529,703.71942034)
\curveto(78.89033031,703.73941402)(78.83033037,703.74941401)(78.77033529,703.74942034)
\curveto(78.71033049,703.74941401)(78.65533055,703.759414)(78.60533529,703.77942034)
\lineto(78.45533529,703.77942034)
\curveto(78.25533095,703.77941398)(78.09533111,703.759414)(77.97533529,703.71942034)
\curveto(77.68533152,703.62941413)(77.45033175,703.48941427)(77.27033529,703.29942034)
\curveto(77.09033211,703.11941464)(76.94533226,702.89941486)(76.83533529,702.63942034)
\curveto(76.78533242,702.52941523)(76.74533246,702.40941535)(76.71533529,702.27942034)
\curveto(76.69533251,702.1594156)(76.67033253,702.02941573)(76.64033529,701.88942034)
\curveto(76.63033257,701.84941591)(76.62533258,701.80941595)(76.62533529,701.76942034)
\curveto(76.62533258,701.72941603)(76.62033258,701.68941607)(76.61033529,701.64942034)
\curveto(76.59033261,701.54941621)(76.58033262,701.40941635)(76.58033529,701.22942034)
\curveto(76.59033261,701.04941671)(76.6053326,700.90941685)(76.62533529,700.80942034)
\curveto(76.62533258,700.72941703)(76.63033257,700.67441709)(76.64033529,700.64442034)
\curveto(76.66033254,700.57441719)(76.67033253,700.50441726)(76.67033529,700.43442034)
\curveto(76.68033252,700.3644174)(76.69533251,700.29441747)(76.71533529,700.22442034)
\curveto(76.79533241,699.99441777)(76.89033231,699.78441798)(77.00033529,699.59442034)
\curveto(77.11033209,699.40441836)(77.25033195,699.24441852)(77.42033529,699.11442034)
\curveto(77.46033174,699.08441868)(77.52033168,699.04941871)(77.60033529,699.00942034)
\curveto(77.71033149,698.93941882)(77.82033138,698.89441887)(77.93033529,698.87442034)
\curveto(78.05033115,698.85441891)(78.19533101,698.83441893)(78.36533529,698.81442034)
\lineto(78.45533529,698.81442034)
\curveto(78.49533071,698.81441895)(78.52533068,698.81941894)(78.54533529,698.82942034)
\lineto(78.68033529,698.82942034)
\curveto(78.75033045,698.84941891)(78.81533039,698.8644189)(78.87533529,698.87442034)
\curveto(78.94533026,698.89441887)(79.01033019,698.91441885)(79.07033529,698.93442034)
\curveto(79.37032983,699.0644187)(79.6003296,699.25441851)(79.76033529,699.50442034)
\curveto(79.8003294,699.55441821)(79.83532937,699.60941815)(79.86533529,699.66942034)
\curveto(79.89532931,699.73941802)(79.92032928,699.79941796)(79.94033529,699.84942034)
\curveto(79.98032922,699.9594178)(80.01532919,700.05441771)(80.04533529,700.13442034)
\curveto(80.07532913,700.22441754)(80.14532906,700.29441747)(80.25533529,700.34442034)
\curveto(80.34532886,700.38441738)(80.49032871,700.39941736)(80.69033529,700.38942034)
\lineto(81.18533529,700.38942034)
\lineto(81.39533529,700.38942034)
\curveto(81.47532773,700.39941736)(81.54032766,700.39441737)(81.59033529,700.37442034)
\lineto(81.71033529,700.37442034)
\lineto(81.83033529,700.34442034)
\curveto(81.87032733,700.34441742)(81.9003273,700.33441743)(81.92033529,700.31442034)
\curveto(81.97032723,700.27441749)(82.0003272,700.21441755)(82.01033529,700.13442034)
\curveto(82.03032717,700.0644177)(82.03032717,699.98941777)(82.01033529,699.90942034)
\curveto(81.92032728,699.57941818)(81.81032739,699.28441848)(81.68033529,699.02442034)
\curveto(81.27032793,698.25441951)(80.61532859,697.71942004)(79.71533529,697.41942034)
\curveto(79.61532959,697.38942037)(79.51032969,697.36942039)(79.40033529,697.35942034)
\curveto(79.29032991,697.33942042)(79.18033002,697.31442045)(79.07033529,697.28442034)
\curveto(79.01033019,697.27442049)(78.95033025,697.26942049)(78.89033529,697.26942034)
\curveto(78.83033037,697.26942049)(78.77033043,697.2644205)(78.71033529,697.25442034)
\lineto(78.54533529,697.25442034)
\curveto(78.49533071,697.23442053)(78.42033078,697.22942053)(78.32033529,697.23942034)
\curveto(78.22033098,697.23942052)(78.14533106,697.24442052)(78.09533529,697.25442034)
\curveto(78.01533119,697.27442049)(77.94033126,697.28442048)(77.87033529,697.28442034)
\curveto(77.81033139,697.27442049)(77.74533146,697.27942048)(77.67533529,697.29942034)
\lineto(77.52533529,697.32942034)
\curveto(77.47533173,697.32942043)(77.42533178,697.33442043)(77.37533529,697.34442034)
\curveto(77.26533194,697.37442039)(77.16033204,697.40442036)(77.06033529,697.43442034)
\curveto(76.96033224,697.4644203)(76.86533234,697.49942026)(76.77533529,697.53942034)
\curveto(76.3053329,697.73942002)(75.91033329,697.99441977)(75.59033529,698.30442034)
\curveto(75.27033393,698.62441914)(75.01033419,699.01941874)(74.81033529,699.48942034)
\curveto(74.76033444,699.57941818)(74.72033448,699.67441809)(74.69033529,699.77442034)
\lineto(74.60033529,700.10442034)
\curveto(74.59033461,700.14441762)(74.58533462,700.17941758)(74.58533529,700.20942034)
\curveto(74.58533462,700.24941751)(74.57533463,700.29441747)(74.55533529,700.34442034)
\curveto(74.53533467,700.41441735)(74.52533468,700.48441728)(74.52533529,700.55442034)
\curveto(74.52533468,700.63441713)(74.51533469,700.70941705)(74.49533529,700.77942034)
\lineto(74.49533529,701.03442034)
\curveto(74.47533473,701.08441668)(74.46533474,701.13941662)(74.46533529,701.19942034)
\curveto(74.46533474,701.26941649)(74.47533473,701.32941643)(74.49533529,701.37942034)
\curveto(74.5053347,701.42941633)(74.5053347,701.47441629)(74.49533529,701.51442034)
\curveto(74.48533472,701.55441621)(74.48533472,701.59441617)(74.49533529,701.63442034)
\curveto(74.51533469,701.70441606)(74.52033468,701.76941599)(74.51033529,701.82942034)
\curveto(74.51033469,701.88941587)(74.52033468,701.94941581)(74.54033529,702.00942034)
\curveto(74.59033461,702.18941557)(74.63033457,702.3594154)(74.66033529,702.51942034)
\curveto(74.69033451,702.68941507)(74.73533447,702.85441491)(74.79533529,703.01442034)
\curveto(75.01533419,703.52441424)(75.29033391,703.94941381)(75.62033529,704.28942034)
\curveto(75.96033324,704.62941313)(76.39033281,704.90441286)(76.91033529,705.11442034)
\curveto(77.05033215,705.17441259)(77.19533201,705.21441255)(77.34533529,705.23442034)
\curveto(77.49533171,705.2644125)(77.65033155,705.29941246)(77.81033529,705.33942034)
\curveto(77.89033131,705.34941241)(77.96533124,705.35441241)(78.03533529,705.35442034)
\curveto(78.1053311,705.35441241)(78.18033102,705.3594124)(78.26033529,705.36942034)
}
}
{
\newrgbcolor{curcolor}{0 0 0}
\pscustom[linestyle=none,fillstyle=solid,fillcolor=curcolor]
{
\newpath
\moveto(85.40361654,708.00942034)
\curveto(85.47361359,707.92940983)(85.50861356,707.80940995)(85.50861654,707.64942034)
\lineto(85.50861654,707.18442034)
\lineto(85.50861654,706.77942034)
\curveto(85.50861356,706.63941112)(85.47361359,706.54441122)(85.40361654,706.49442034)
\curveto(85.34361372,706.44441132)(85.2636138,706.41441135)(85.16361654,706.40442034)
\curveto(85.07361399,706.39441137)(84.97361409,706.38941137)(84.86361654,706.38942034)
\lineto(84.02361654,706.38942034)
\curveto(83.91361515,706.38941137)(83.81361525,706.39441137)(83.72361654,706.40442034)
\curveto(83.64361542,706.41441135)(83.57361549,706.44441132)(83.51361654,706.49442034)
\curveto(83.47361559,706.52441124)(83.44361562,706.57941118)(83.42361654,706.65942034)
\curveto(83.41361565,706.74941101)(83.40361566,706.84441092)(83.39361654,706.94442034)
\lineto(83.39361654,707.27442034)
\curveto(83.40361566,707.38441038)(83.40861566,707.47941028)(83.40861654,707.55942034)
\lineto(83.40861654,707.76942034)
\curveto(83.41861565,707.83940992)(83.43861563,707.89940986)(83.46861654,707.94942034)
\curveto(83.48861558,707.98940977)(83.51361555,708.01940974)(83.54361654,708.03942034)
\lineto(83.66361654,708.09942034)
\curveto(83.68361538,708.09940966)(83.70861536,708.09940966)(83.73861654,708.09942034)
\curveto(83.7686153,708.10940965)(83.79361527,708.11440965)(83.81361654,708.11442034)
\lineto(84.90861654,708.11442034)
\curveto(85.00861406,708.11440965)(85.10361396,708.10940965)(85.19361654,708.09942034)
\curveto(85.28361378,708.08940967)(85.35361371,708.0594097)(85.40361654,708.00942034)
\moveto(85.50861654,698.24442034)
\curveto(85.50861356,698.04441972)(85.50361356,697.87441989)(85.49361654,697.73442034)
\curveto(85.48361358,697.59442017)(85.39361367,697.49942026)(85.22361654,697.44942034)
\curveto(85.1636139,697.42942033)(85.09861397,697.41942034)(85.02861654,697.41942034)
\curveto(84.95861411,697.42942033)(84.88361418,697.43442033)(84.80361654,697.43442034)
\lineto(83.96361654,697.43442034)
\curveto(83.87361519,697.43442033)(83.78361528,697.43942032)(83.69361654,697.44942034)
\curveto(83.61361545,697.4594203)(83.55361551,697.48942027)(83.51361654,697.53942034)
\curveto(83.45361561,697.60942015)(83.41861565,697.69442007)(83.40861654,697.79442034)
\lineto(83.40861654,698.13942034)
\lineto(83.40861654,704.46942034)
\lineto(83.40861654,704.76942034)
\curveto(83.40861566,704.86941289)(83.42861564,704.94941281)(83.46861654,705.00942034)
\curveto(83.52861554,705.07941268)(83.61361545,705.12441264)(83.72361654,705.14442034)
\curveto(83.74361532,705.15441261)(83.7686153,705.15441261)(83.79861654,705.14442034)
\curveto(83.83861523,705.14441262)(83.8686152,705.14941261)(83.88861654,705.15942034)
\lineto(84.63861654,705.15942034)
\lineto(84.83361654,705.15942034)
\curveto(84.91361415,705.16941259)(84.97861409,705.16941259)(85.02861654,705.15942034)
\lineto(85.14861654,705.15942034)
\curveto(85.20861386,705.13941262)(85.2636138,705.12441264)(85.31361654,705.11442034)
\curveto(85.3636137,705.10441266)(85.40361366,705.07441269)(85.43361654,705.02442034)
\curveto(85.47361359,704.97441279)(85.49361357,704.90441286)(85.49361654,704.81442034)
\curveto(85.50361356,704.72441304)(85.50861356,704.62941313)(85.50861654,704.52942034)
\lineto(85.50861654,698.24442034)
}
}
{
\newrgbcolor{curcolor}{0 0 0}
\pscustom[linestyle=none,fillstyle=solid,fillcolor=curcolor]
{
\newpath
\moveto(94.94080404,701.60442034)
\curveto(94.92079551,701.65441611)(94.91579552,701.70941605)(94.92580404,701.76942034)
\curveto(94.9357955,701.82941593)(94.9307955,701.88441588)(94.91080404,701.93442034)
\curveto(94.90079553,701.97441579)(94.89579554,702.01441575)(94.89580404,702.05442034)
\curveto(94.89579554,702.09441567)(94.89079554,702.13441563)(94.88080404,702.17442034)
\lineto(94.82080404,702.44442034)
\curveto(94.80079563,702.53441523)(94.77579566,702.61941514)(94.74580404,702.69942034)
\curveto(94.69579574,702.83941492)(94.65079578,702.96941479)(94.61080404,703.08942034)
\curveto(94.57079586,703.21941454)(94.51579592,703.33941442)(94.44580404,703.44942034)
\curveto(94.37579606,703.5594142)(94.30579613,703.6644141)(94.23580404,703.76442034)
\curveto(94.17579626,703.8644139)(94.10579633,703.9644138)(94.02580404,704.06442034)
\curveto(93.94579649,704.17441359)(93.84579659,704.27441349)(93.72580404,704.36442034)
\curveto(93.61579682,704.4644133)(93.50579693,704.55441321)(93.39580404,704.63442034)
\curveto(93.06579737,704.8644129)(92.68579775,705.04441272)(92.25580404,705.17442034)
\curveto(91.8357986,705.30441246)(91.3357991,705.3644124)(90.75580404,705.35442034)
\curveto(90.68579975,705.34441242)(90.61579982,705.33941242)(90.54580404,705.33942034)
\curveto(90.47579996,705.33941242)(90.40080003,705.33441243)(90.32080404,705.32442034)
\curveto(90.17080026,705.28441248)(90.02580041,705.25441251)(89.88580404,705.23442034)
\curveto(89.74580069,705.21441255)(89.61080082,705.17941258)(89.48080404,705.12942034)
\curveto(89.37080106,705.07941268)(89.26080117,705.03441273)(89.15080404,704.99442034)
\curveto(89.04080139,704.95441281)(88.9358015,704.90941285)(88.83580404,704.85942034)
\curveto(88.47580196,704.62941313)(88.17080226,704.37441339)(87.92080404,704.09442034)
\curveto(87.67080276,703.82441394)(87.45580298,703.48441428)(87.27580404,703.07442034)
\curveto(87.22580321,702.95441481)(87.18580325,702.82941493)(87.15580404,702.69942034)
\curveto(87.12580331,702.57941518)(87.09080334,702.45441531)(87.05080404,702.32442034)
\curveto(87.0308034,702.27441549)(87.02080341,702.22441554)(87.02080404,702.17442034)
\curveto(87.02080341,702.13441563)(87.01580342,702.08941567)(87.00580404,702.03942034)
\curveto(86.98580345,701.98941577)(86.97580346,701.93441583)(86.97580404,701.87442034)
\curveto(86.98580345,701.82441594)(86.98580345,701.77441599)(86.97580404,701.72442034)
\lineto(86.97580404,701.61942034)
\curveto(86.95580348,701.5594162)(86.94080349,701.47441629)(86.93080404,701.36442034)
\curveto(86.9308035,701.25441651)(86.94080349,701.16941659)(86.96080404,701.10942034)
\lineto(86.96080404,700.97442034)
\curveto(86.96080347,700.93441683)(86.96580347,700.88941687)(86.97580404,700.83942034)
\curveto(86.99580344,700.759417)(87.00580343,700.67441709)(87.00580404,700.58442034)
\curveto(87.00580343,700.50441726)(87.01580342,700.42441734)(87.03580404,700.34442034)
\curveto(87.05580338,700.29441747)(87.06580337,700.24941751)(87.06580404,700.20942034)
\curveto(87.06580337,700.16941759)(87.07580336,700.12441764)(87.09580404,700.07442034)
\curveto(87.12580331,699.9644178)(87.15080328,699.8594179)(87.17080404,699.75942034)
\curveto(87.20080323,699.6594181)(87.24080319,699.5644182)(87.29080404,699.47442034)
\curveto(87.46080297,699.08441868)(87.67080276,698.74941901)(87.92080404,698.46942034)
\curveto(88.17080226,698.18941957)(88.47080196,697.94441982)(88.82080404,697.73442034)
\curveto(88.9308015,697.67442009)(89.0358014,697.62442014)(89.13580404,697.58442034)
\curveto(89.24580119,697.54442022)(89.36080107,697.50442026)(89.48080404,697.46442034)
\curveto(89.57080086,697.42442034)(89.66580077,697.39442037)(89.76580404,697.37442034)
\curveto(89.86580057,697.35442041)(89.96580047,697.32942043)(90.06580404,697.29942034)
\curveto(90.11580032,697.28942047)(90.15580028,697.28442048)(90.18580404,697.28442034)
\curveto(90.22580021,697.28442048)(90.26580017,697.27942048)(90.30580404,697.26942034)
\curveto(90.35580008,697.24942051)(90.40580003,697.24442052)(90.45580404,697.25442034)
\curveto(90.51579992,697.25442051)(90.57079986,697.24942051)(90.62080404,697.23942034)
\lineto(90.77080404,697.23942034)
\curveto(90.8307996,697.21942054)(90.91579952,697.21442055)(91.02580404,697.22442034)
\curveto(91.1357993,697.22442054)(91.21579922,697.22942053)(91.26580404,697.23942034)
\curveto(91.29579914,697.23942052)(91.32579911,697.24442052)(91.35580404,697.25442034)
\lineto(91.46080404,697.25442034)
\curveto(91.51079892,697.2644205)(91.56579887,697.26942049)(91.62580404,697.26942034)
\curveto(91.68579875,697.26942049)(91.74079869,697.27942048)(91.79080404,697.29942034)
\curveto(91.92079851,697.32942043)(92.04579839,697.3594204)(92.16580404,697.38942034)
\curveto(92.29579814,697.40942035)(92.42079801,697.44442032)(92.54080404,697.49442034)
\curveto(93.02079741,697.69442007)(93.430797,697.94441982)(93.77080404,698.24442034)
\curveto(94.11079632,698.54441922)(94.38579605,698.93441883)(94.59580404,699.41442034)
\curveto(94.64579579,699.51441825)(94.68579575,699.61941814)(94.71580404,699.72942034)
\curveto(94.74579569,699.84941791)(94.78079565,699.9644178)(94.82080404,700.07442034)
\curveto(94.8307956,700.14441762)(94.84079559,700.20941755)(94.85080404,700.26942034)
\curveto(94.86079557,700.32941743)(94.87579556,700.39441737)(94.89580404,700.46442034)
\curveto(94.91579552,700.54441722)(94.92079551,700.62441714)(94.91080404,700.70442034)
\curveto(94.91079552,700.78441698)(94.92079551,700.8644169)(94.94080404,700.94442034)
\lineto(94.94080404,701.09442034)
\curveto(94.96079547,701.15441661)(94.97079546,701.23941652)(94.97080404,701.34942034)
\curveto(94.97079546,701.4594163)(94.96079547,701.54441622)(94.94080404,701.60442034)
\moveto(92.84080404,701.06442034)
\curveto(92.8307976,701.01441675)(92.82579761,700.9644168)(92.82580404,700.91442034)
\lineto(92.82580404,700.77942034)
\curveto(92.81579762,700.73941702)(92.81079762,700.69941706)(92.81080404,700.65942034)
\curveto(92.81079762,700.62941713)(92.80579763,700.59441717)(92.79580404,700.55442034)
\curveto(92.76579767,700.44441732)(92.74079769,700.33941742)(92.72080404,700.23942034)
\curveto(92.70079773,700.13941762)(92.67079776,700.03941772)(92.63080404,699.93942034)
\curveto(92.52079791,699.68941807)(92.38579805,699.47941828)(92.22580404,699.30942034)
\curveto(92.06579837,699.13941862)(91.85579858,699.00441876)(91.59580404,698.90442034)
\curveto(91.52579891,698.87441889)(91.45079898,698.85441891)(91.37080404,698.84442034)
\curveto(91.29079914,698.83441893)(91.21079922,698.81941894)(91.13080404,698.79942034)
\lineto(91.01080404,698.79942034)
\curveto(90.97079946,698.78941897)(90.92579951,698.78441898)(90.87580404,698.78442034)
\lineto(90.75580404,698.81442034)
\curveto(90.71579972,698.82441894)(90.68079975,698.82441894)(90.65080404,698.81442034)
\curveto(90.62079981,698.81441895)(90.58579985,698.81941894)(90.54580404,698.82942034)
\curveto(90.45579998,698.84941891)(90.36580007,698.87441889)(90.27580404,698.90442034)
\curveto(90.19580024,698.93441883)(90.12080031,698.97441879)(90.05080404,699.02442034)
\curveto(89.80080063,699.17441859)(89.61580082,699.33941842)(89.49580404,699.51942034)
\curveto(89.38580105,699.70941805)(89.28080115,699.95441781)(89.18080404,700.25442034)
\curveto(89.16080127,700.33441743)(89.14580129,700.40941735)(89.13580404,700.47942034)
\curveto(89.12580131,700.5594172)(89.11080132,700.63941712)(89.09080404,700.71942034)
\lineto(89.09080404,700.85442034)
\curveto(89.07080136,700.92441684)(89.05580138,701.02941673)(89.04580404,701.16942034)
\curveto(89.04580139,701.30941645)(89.05580138,701.41441635)(89.07580404,701.48442034)
\lineto(89.07580404,701.63442034)
\curveto(89.07580136,701.68441608)(89.08080135,701.73441603)(89.09080404,701.78442034)
\curveto(89.11080132,701.89441587)(89.12580131,702.00441576)(89.13580404,702.11442034)
\curveto(89.15580128,702.22441554)(89.18080125,702.32941543)(89.21080404,702.42942034)
\curveto(89.30080113,702.69941506)(89.42080101,702.93441483)(89.57080404,703.13442034)
\curveto(89.7308007,703.34441442)(89.9358005,703.50441426)(90.18580404,703.61442034)
\curveto(90.2358002,703.64441412)(90.29080014,703.6644141)(90.35080404,703.67442034)
\lineto(90.56080404,703.73442034)
\curveto(90.59079984,703.74441402)(90.62579981,703.74441402)(90.66580404,703.73442034)
\curveto(90.70579973,703.73441403)(90.74079969,703.74441402)(90.77080404,703.76442034)
\lineto(91.04080404,703.76442034)
\curveto(91.1307993,703.77441399)(91.21579922,703.76941399)(91.29580404,703.74942034)
\curveto(91.36579907,703.72941403)(91.430799,703.70941405)(91.49080404,703.68942034)
\curveto(91.55079888,703.67941408)(91.61079882,703.6644141)(91.67080404,703.64442034)
\curveto(91.92079851,703.53441423)(92.12079831,703.38441438)(92.27080404,703.19442034)
\curveto(92.42079801,703.01441475)(92.55079788,702.79441497)(92.66080404,702.53442034)
\curveto(92.69079774,702.45441531)(92.71079772,702.36941539)(92.72080404,702.27942034)
\lineto(92.78080404,702.03942034)
\curveto(92.79079764,702.01941574)(92.79579764,701.98941577)(92.79580404,701.94942034)
\curveto(92.80579763,701.89941586)(92.81079762,701.84441592)(92.81080404,701.78442034)
\curveto(92.81079762,701.72441604)(92.82079761,701.66941609)(92.84080404,701.61942034)
\lineto(92.84080404,701.49942034)
\curveto(92.85079758,701.44941631)(92.85579758,701.37441639)(92.85580404,701.27442034)
\curveto(92.85579758,701.18441658)(92.85079758,701.11441665)(92.84080404,701.06442034)
\moveto(91.61080404,708.23442034)
\lineto(92.67580404,708.23442034)
\curveto(92.75579768,708.23440953)(92.85079758,708.23440953)(92.96080404,708.23442034)
\curveto(93.07079736,708.23440953)(93.15079728,708.21940954)(93.20080404,708.18942034)
\curveto(93.22079721,708.17940958)(93.2307972,708.1644096)(93.23080404,708.14442034)
\curveto(93.24079719,708.13440963)(93.25579718,708.12440964)(93.27580404,708.11442034)
\curveto(93.28579715,707.99440977)(93.2357972,707.88940987)(93.12580404,707.79942034)
\curveto(93.02579741,707.70941005)(92.94079749,707.62941013)(92.87080404,707.55942034)
\curveto(92.79079764,707.48941027)(92.71079772,707.41441035)(92.63080404,707.33442034)
\curveto(92.56079787,707.2644105)(92.48579795,707.19941056)(92.40580404,707.13942034)
\curveto(92.36579807,707.10941065)(92.3307981,707.07441069)(92.30080404,707.03442034)
\curveto(92.28079815,707.00441076)(92.25079818,706.97941078)(92.21080404,706.95942034)
\curveto(92.19079824,706.92941083)(92.16579827,706.90441086)(92.13580404,706.88442034)
\lineto(91.98580404,706.73442034)
\lineto(91.83580404,706.61442034)
\lineto(91.79080404,706.56942034)
\curveto(91.79079864,706.5594112)(91.78079865,706.54441122)(91.76080404,706.52442034)
\curveto(91.68079875,706.4644113)(91.60079883,706.39941136)(91.52080404,706.32942034)
\curveto(91.45079898,706.2594115)(91.36079907,706.20441156)(91.25080404,706.16442034)
\curveto(91.21079922,706.15441161)(91.17079926,706.14941161)(91.13080404,706.14942034)
\curveto(91.10079933,706.14941161)(91.06079937,706.14441162)(91.01080404,706.13442034)
\curveto(90.98079945,706.12441164)(90.94079949,706.11941164)(90.89080404,706.11942034)
\curveto(90.84079959,706.12941163)(90.79579964,706.13441163)(90.75580404,706.13442034)
\lineto(90.41080404,706.13442034)
\curveto(90.29080014,706.13441163)(90.20080023,706.1594116)(90.14080404,706.20942034)
\curveto(90.08080035,706.24941151)(90.06580037,706.31941144)(90.09580404,706.41942034)
\curveto(90.11580032,706.49941126)(90.15080028,706.56941119)(90.20080404,706.62942034)
\curveto(90.25080018,706.69941106)(90.29580014,706.76941099)(90.33580404,706.83942034)
\curveto(90.4358,706.97941078)(90.5307999,707.11441065)(90.62080404,707.24442034)
\curveto(90.71079972,707.37441039)(90.80079963,707.50941025)(90.89080404,707.64942034)
\curveto(90.94079949,707.72941003)(90.99079944,707.81440995)(91.04080404,707.90442034)
\curveto(91.10079933,707.99440977)(91.16579927,708.0644097)(91.23580404,708.11442034)
\curveto(91.27579916,708.14440962)(91.34579909,708.17940958)(91.44580404,708.21942034)
\curveto(91.46579897,708.22940953)(91.49079894,708.22940953)(91.52080404,708.21942034)
\curveto(91.56079887,708.21940954)(91.59079884,708.22440954)(91.61080404,708.23442034)
}
}
{
\newrgbcolor{curcolor}{0 0 0}
\pscustom[linestyle=none,fillstyle=solid,fillcolor=curcolor]
{
\newpath
\moveto(100.76572592,705.35442034)
\curveto(101.36572011,705.37441239)(101.86571961,705.28941247)(102.26572592,705.09942034)
\curveto(102.66571881,704.90941285)(102.9807185,704.62941313)(103.21072592,704.25942034)
\curveto(103.2807182,704.14941361)(103.33571814,704.02941373)(103.37572592,703.89942034)
\curveto(103.41571806,703.77941398)(103.45571802,703.65441411)(103.49572592,703.52442034)
\curveto(103.51571796,703.44441432)(103.52571795,703.36941439)(103.52572592,703.29942034)
\curveto(103.53571794,703.22941453)(103.55071793,703.1594146)(103.57072592,703.08942034)
\curveto(103.57071791,703.02941473)(103.5757179,702.98941477)(103.58572592,702.96942034)
\curveto(103.60571787,702.82941493)(103.61571786,702.68441508)(103.61572592,702.53442034)
\lineto(103.61572592,702.09942034)
\lineto(103.61572592,700.76442034)
\lineto(103.61572592,698.33442034)
\curveto(103.61571786,698.14441962)(103.61071787,697.9594198)(103.60072592,697.77942034)
\curveto(103.60071788,697.60942015)(103.53071795,697.49942026)(103.39072592,697.44942034)
\curveto(103.33071815,697.42942033)(103.26071822,697.41942034)(103.18072592,697.41942034)
\lineto(102.94072592,697.41942034)
\lineto(102.13072592,697.41942034)
\curveto(102.01071947,697.41942034)(101.90071958,697.42442034)(101.80072592,697.43442034)
\curveto(101.71071977,697.45442031)(101.64071984,697.49942026)(101.59072592,697.56942034)
\curveto(101.55071993,697.62942013)(101.52571995,697.70442006)(101.51572592,697.79442034)
\lineto(101.51572592,698.10942034)
\lineto(101.51572592,699.15942034)
\lineto(101.51572592,701.39442034)
\curveto(101.51571996,701.764416)(101.50071998,702.10441566)(101.47072592,702.41442034)
\curveto(101.44072004,702.73441503)(101.35072013,703.00441476)(101.20072592,703.22442034)
\curveto(101.06072042,703.42441434)(100.85572062,703.5644142)(100.58572592,703.64442034)
\curveto(100.53572094,703.6644141)(100.480721,703.67441409)(100.42072592,703.67442034)
\curveto(100.37072111,703.67441409)(100.31572116,703.68441408)(100.25572592,703.70442034)
\curveto(100.20572127,703.71441405)(100.14072134,703.71441405)(100.06072592,703.70442034)
\curveto(99.99072149,703.70441406)(99.93572154,703.69941406)(99.89572592,703.68942034)
\curveto(99.85572162,703.67941408)(99.82072166,703.67441409)(99.79072592,703.67442034)
\curveto(99.76072172,703.67441409)(99.73072175,703.66941409)(99.70072592,703.65942034)
\curveto(99.47072201,703.59941416)(99.28572219,703.51941424)(99.14572592,703.41942034)
\curveto(98.82572265,703.18941457)(98.63572284,702.85441491)(98.57572592,702.41442034)
\curveto(98.51572296,701.97441579)(98.48572299,701.47941628)(98.48572592,700.92942034)
\lineto(98.48572592,699.05442034)
\lineto(98.48572592,698.13942034)
\lineto(98.48572592,697.86942034)
\curveto(98.48572299,697.77941998)(98.47072301,697.70442006)(98.44072592,697.64442034)
\curveto(98.39072309,697.53442023)(98.31072317,697.46942029)(98.20072592,697.44942034)
\curveto(98.09072339,697.42942033)(97.95572352,697.41942034)(97.79572592,697.41942034)
\lineto(97.04572592,697.41942034)
\curveto(96.93572454,697.41942034)(96.82572465,697.42442034)(96.71572592,697.43442034)
\curveto(96.60572487,697.44442032)(96.52572495,697.47942028)(96.47572592,697.53942034)
\curveto(96.40572507,697.62942013)(96.37072511,697.75942)(96.37072592,697.92942034)
\curveto(96.3807251,698.09941966)(96.38572509,698.2594195)(96.38572592,698.40942034)
\lineto(96.38572592,700.44942034)
\lineto(96.38572592,703.74942034)
\lineto(96.38572592,704.51442034)
\lineto(96.38572592,704.81442034)
\curveto(96.39572508,704.90441286)(96.42572505,704.97941278)(96.47572592,705.03942034)
\curveto(96.49572498,705.06941269)(96.52572495,705.08941267)(96.56572592,705.09942034)
\curveto(96.61572486,705.11941264)(96.66572481,705.13441263)(96.71572592,705.14442034)
\lineto(96.79072592,705.14442034)
\curveto(96.84072464,705.15441261)(96.89072459,705.1594126)(96.94072592,705.15942034)
\lineto(97.10572592,705.15942034)
\lineto(97.73572592,705.15942034)
\curveto(97.81572366,705.1594126)(97.89072359,705.15441261)(97.96072592,705.14442034)
\curveto(98.04072344,705.14441262)(98.11072337,705.13441263)(98.17072592,705.11442034)
\curveto(98.24072324,705.08441268)(98.28572319,705.03941272)(98.30572592,704.97942034)
\curveto(98.33572314,704.91941284)(98.36072312,704.84941291)(98.38072592,704.76942034)
\curveto(98.39072309,704.72941303)(98.39072309,704.69441307)(98.38072592,704.66442034)
\curveto(98.3807231,704.63441313)(98.39072309,704.60441316)(98.41072592,704.57442034)
\curveto(98.43072305,704.52441324)(98.44572303,704.49441327)(98.45572592,704.48442034)
\curveto(98.475723,704.47441329)(98.50072298,704.4594133)(98.53072592,704.43942034)
\curveto(98.64072284,704.42941333)(98.73072275,704.4644133)(98.80072592,704.54442034)
\curveto(98.87072261,704.63441313)(98.94572253,704.70441306)(99.02572592,704.75442034)
\curveto(99.29572218,704.95441281)(99.59572188,705.11441265)(99.92572592,705.23442034)
\curveto(100.01572146,705.2644125)(100.10572137,705.28441248)(100.19572592,705.29442034)
\curveto(100.29572118,705.30441246)(100.40072108,705.31941244)(100.51072592,705.33942034)
\curveto(100.54072094,705.34941241)(100.58572089,705.34941241)(100.64572592,705.33942034)
\curveto(100.70572077,705.33941242)(100.74572073,705.34441242)(100.76572592,705.35442034)
}
}
{
\newrgbcolor{curcolor}{0 0 0}
\pscustom[linestyle=none,fillstyle=solid,fillcolor=curcolor]
{
}
}
{
\newrgbcolor{curcolor}{0 0 0}
\pscustom[linestyle=none,fillstyle=solid,fillcolor=curcolor]
{
\newpath
\moveto(116.99713217,698.27442034)
\lineto(116.99713217,697.85442034)
\curveto(116.9971238,697.72442004)(116.96712383,697.61942014)(116.90713217,697.53942034)
\curveto(116.85712394,697.48942027)(116.792124,697.45442031)(116.71213217,697.43442034)
\curveto(116.63212416,697.42442034)(116.54212425,697.41942034)(116.44213217,697.41942034)
\lineto(115.61713217,697.41942034)
\lineto(115.33213217,697.41942034)
\curveto(115.25212554,697.42942033)(115.18712561,697.45442031)(115.13713217,697.49442034)
\curveto(115.06712573,697.54442022)(115.02712577,697.60942015)(115.01713217,697.68942034)
\curveto(115.00712579,697.76941999)(114.98712581,697.84941991)(114.95713217,697.92942034)
\curveto(114.93712586,697.94941981)(114.91712588,697.9644198)(114.89713217,697.97442034)
\curveto(114.88712591,697.99441977)(114.87212592,698.01441975)(114.85213217,698.03442034)
\curveto(114.74212605,698.03441973)(114.66212613,698.00941975)(114.61213217,697.95942034)
\lineto(114.46213217,697.80942034)
\curveto(114.3921264,697.75942)(114.32712647,697.71442005)(114.26713217,697.67442034)
\curveto(114.20712659,697.64442012)(114.14212665,697.60442016)(114.07213217,697.55442034)
\curveto(114.03212676,697.53442023)(113.98712681,697.51442025)(113.93713217,697.49442034)
\curveto(113.8971269,697.47442029)(113.85212694,697.45442031)(113.80213217,697.43442034)
\curveto(113.66212713,697.38442038)(113.51212728,697.33942042)(113.35213217,697.29942034)
\curveto(113.30212749,697.27942048)(113.25712754,697.26942049)(113.21713217,697.26942034)
\curveto(113.17712762,697.26942049)(113.13712766,697.2644205)(113.09713217,697.25442034)
\lineto(112.96213217,697.25442034)
\curveto(112.93212786,697.24442052)(112.8921279,697.23942052)(112.84213217,697.23942034)
\lineto(112.70713217,697.23942034)
\curveto(112.64712815,697.21942054)(112.55712824,697.21442055)(112.43713217,697.22442034)
\curveto(112.31712848,697.22442054)(112.23212856,697.23442053)(112.18213217,697.25442034)
\curveto(112.11212868,697.27442049)(112.04712875,697.28442048)(111.98713217,697.28442034)
\curveto(111.93712886,697.27442049)(111.88212891,697.27942048)(111.82213217,697.29942034)
\lineto(111.46213217,697.41942034)
\curveto(111.35212944,697.44942031)(111.24212955,697.48942027)(111.13213217,697.53942034)
\curveto(110.78213001,697.68942007)(110.46713033,697.91941984)(110.18713217,698.22942034)
\curveto(109.91713088,698.54941921)(109.70213109,698.88441888)(109.54213217,699.23442034)
\curveto(109.4921313,699.34441842)(109.45213134,699.44941831)(109.42213217,699.54942034)
\curveto(109.3921314,699.6594181)(109.35713144,699.76941799)(109.31713217,699.87942034)
\curveto(109.30713149,699.91941784)(109.30213149,699.95441781)(109.30213217,699.98442034)
\curveto(109.30213149,700.02441774)(109.2921315,700.06941769)(109.27213217,700.11942034)
\curveto(109.25213154,700.19941756)(109.23213156,700.28441748)(109.21213217,700.37442034)
\curveto(109.20213159,700.47441729)(109.18713161,700.57441719)(109.16713217,700.67442034)
\curveto(109.15713164,700.70441706)(109.15213164,700.73941702)(109.15213217,700.77942034)
\curveto(109.16213163,700.81941694)(109.16213163,700.85441691)(109.15213217,700.88442034)
\lineto(109.15213217,701.01942034)
\curveto(109.15213164,701.06941669)(109.14713165,701.11941664)(109.13713217,701.16942034)
\curveto(109.12713167,701.21941654)(109.12213167,701.27441649)(109.12213217,701.33442034)
\curveto(109.12213167,701.40441636)(109.12713167,701.4594163)(109.13713217,701.49942034)
\curveto(109.14713165,701.54941621)(109.15213164,701.59441617)(109.15213217,701.63442034)
\lineto(109.15213217,701.78442034)
\curveto(109.16213163,701.83441593)(109.16213163,701.87941588)(109.15213217,701.91942034)
\curveto(109.15213164,701.96941579)(109.16213163,702.01941574)(109.18213217,702.06942034)
\curveto(109.20213159,702.17941558)(109.21713158,702.28441548)(109.22713217,702.38442034)
\curveto(109.24713155,702.48441528)(109.27213152,702.58441518)(109.30213217,702.68442034)
\curveto(109.34213145,702.80441496)(109.37713142,702.91941484)(109.40713217,703.02942034)
\curveto(109.43713136,703.13941462)(109.47713132,703.24941451)(109.52713217,703.35942034)
\curveto(109.66713113,703.6594141)(109.84213095,703.94441382)(110.05213217,704.21442034)
\curveto(110.07213072,704.24441352)(110.0971307,704.26941349)(110.12713217,704.28942034)
\curveto(110.16713063,704.31941344)(110.1971306,704.34941341)(110.21713217,704.37942034)
\curveto(110.25713054,704.42941333)(110.2971305,704.47441329)(110.33713217,704.51442034)
\curveto(110.37713042,704.55441321)(110.42213037,704.59441317)(110.47213217,704.63442034)
\curveto(110.51213028,704.65441311)(110.54713025,704.67941308)(110.57713217,704.70942034)
\curveto(110.60713019,704.74941301)(110.64213015,704.77941298)(110.68213217,704.79942034)
\curveto(110.93212986,704.96941279)(111.22212957,705.10941265)(111.55213217,705.21942034)
\curveto(111.62212917,705.23941252)(111.6921291,705.25441251)(111.76213217,705.26442034)
\curveto(111.84212895,705.27441249)(111.92212887,705.28941247)(112.00213217,705.30942034)
\curveto(112.07212872,705.32941243)(112.16212863,705.33941242)(112.27213217,705.33942034)
\curveto(112.38212841,705.34941241)(112.4921283,705.35441241)(112.60213217,705.35442034)
\curveto(112.71212808,705.35441241)(112.81712798,705.34941241)(112.91713217,705.33942034)
\curveto(113.02712777,705.32941243)(113.11712768,705.31441245)(113.18713217,705.29442034)
\curveto(113.33712746,705.24441252)(113.48212731,705.19941256)(113.62213217,705.15942034)
\curveto(113.76212703,705.11941264)(113.8921269,705.0644127)(114.01213217,704.99442034)
\curveto(114.08212671,704.94441282)(114.14712665,704.89441287)(114.20713217,704.84442034)
\curveto(114.26712653,704.80441296)(114.33212646,704.759413)(114.40213217,704.70942034)
\curveto(114.44212635,704.67941308)(114.4971263,704.63941312)(114.56713217,704.58942034)
\curveto(114.64712615,704.53941322)(114.72212607,704.53941322)(114.79213217,704.58942034)
\curveto(114.83212596,704.60941315)(114.85212594,704.64441312)(114.85213217,704.69442034)
\curveto(114.85212594,704.74441302)(114.86212593,704.79441297)(114.88213217,704.84442034)
\lineto(114.88213217,704.99442034)
\curveto(114.8921259,705.02441274)(114.8971259,705.0594127)(114.89713217,705.09942034)
\lineto(114.89713217,705.21942034)
\lineto(114.89713217,707.25942034)
\curveto(114.8971259,707.36941039)(114.8921259,707.48941027)(114.88213217,707.61942034)
\curveto(114.88212591,707.75941)(114.90712589,707.8644099)(114.95713217,707.93442034)
\curveto(114.9971258,708.01440975)(115.07212572,708.0644097)(115.18213217,708.08442034)
\curveto(115.20212559,708.09440967)(115.22212557,708.09440967)(115.24213217,708.08442034)
\curveto(115.26212553,708.08440968)(115.28212551,708.08940967)(115.30213217,708.09942034)
\lineto(116.36713217,708.09942034)
\curveto(116.48712431,708.09940966)(116.5971242,708.09440967)(116.69713217,708.08442034)
\curveto(116.797124,708.07440969)(116.87212392,708.03440973)(116.92213217,707.96442034)
\curveto(116.97212382,707.88440988)(116.9971238,707.77940998)(116.99713217,707.64942034)
\lineto(116.99713217,707.28942034)
\lineto(116.99713217,698.27442034)
\moveto(114.95713217,701.21442034)
\curveto(114.96712583,701.25441651)(114.96712583,701.29441647)(114.95713217,701.33442034)
\lineto(114.95713217,701.46942034)
\curveto(114.95712584,701.56941619)(114.95212584,701.66941609)(114.94213217,701.76942034)
\curveto(114.93212586,701.86941589)(114.91712588,701.9594158)(114.89713217,702.03942034)
\curveto(114.87712592,702.14941561)(114.85712594,702.24941551)(114.83713217,702.33942034)
\curveto(114.82712597,702.42941533)(114.80212599,702.51441525)(114.76213217,702.59442034)
\curveto(114.62212617,702.95441481)(114.41712638,703.23941452)(114.14713217,703.44942034)
\curveto(113.88712691,703.6594141)(113.50712729,703.764414)(113.00713217,703.76442034)
\curveto(112.94712785,703.764414)(112.86712793,703.75441401)(112.76713217,703.73442034)
\curveto(112.68712811,703.71441405)(112.61212818,703.69441407)(112.54213217,703.67442034)
\curveto(112.48212831,703.6644141)(112.42212837,703.64441412)(112.36213217,703.61442034)
\curveto(112.0921287,703.50441426)(111.88212891,703.33441443)(111.73213217,703.10442034)
\curveto(111.58212921,702.87441489)(111.46212933,702.61441515)(111.37213217,702.32442034)
\curveto(111.34212945,702.22441554)(111.32212947,702.12441564)(111.31213217,702.02442034)
\curveto(111.30212949,701.92441584)(111.28212951,701.81941594)(111.25213217,701.70942034)
\lineto(111.25213217,701.49942034)
\curveto(111.23212956,701.40941635)(111.22712957,701.28441648)(111.23713217,701.12442034)
\curveto(111.24712955,700.97441679)(111.26212953,700.8644169)(111.28213217,700.79442034)
\lineto(111.28213217,700.70442034)
\curveto(111.2921295,700.68441708)(111.2971295,700.6644171)(111.29713217,700.64442034)
\curveto(111.31712948,700.5644172)(111.33212946,700.48941727)(111.34213217,700.41942034)
\curveto(111.36212943,700.34941741)(111.38212941,700.27441749)(111.40213217,700.19442034)
\curveto(111.57212922,699.67441809)(111.86212893,699.28941847)(112.27213217,699.03942034)
\curveto(112.40212839,698.94941881)(112.58212821,698.87941888)(112.81213217,698.82942034)
\curveto(112.85212794,698.81941894)(112.91212788,698.81441895)(112.99213217,698.81442034)
\curveto(113.02212777,698.80441896)(113.06712773,698.79441897)(113.12713217,698.78442034)
\curveto(113.1971276,698.78441898)(113.25212754,698.78941897)(113.29213217,698.79942034)
\curveto(113.37212742,698.81941894)(113.45212734,698.83441893)(113.53213217,698.84442034)
\curveto(113.61212718,698.85441891)(113.6921271,698.87441889)(113.77213217,698.90442034)
\curveto(114.02212677,699.01441875)(114.22212657,699.15441861)(114.37213217,699.32442034)
\curveto(114.52212627,699.49441827)(114.65212614,699.70941805)(114.76213217,699.96942034)
\curveto(114.80212599,700.0594177)(114.83212596,700.14941761)(114.85213217,700.23942034)
\curveto(114.87212592,700.33941742)(114.8921259,700.44441732)(114.91213217,700.55442034)
\curveto(114.92212587,700.60441716)(114.92212587,700.64941711)(114.91213217,700.68942034)
\curveto(114.91212588,700.73941702)(114.92212587,700.78941697)(114.94213217,700.83942034)
\curveto(114.95212584,700.86941689)(114.95712584,700.90441686)(114.95713217,700.94442034)
\lineto(114.95713217,701.07942034)
\lineto(114.95713217,701.21442034)
}
}
{
\newrgbcolor{curcolor}{0 0 0}
\pscustom[linestyle=none,fillstyle=solid,fillcolor=curcolor]
{
\newpath
\moveto(125.94205404,701.36442034)
\curveto(125.96204588,701.28441648)(125.96204588,701.19441657)(125.94205404,701.09442034)
\curveto(125.92204592,700.99441677)(125.88704595,700.92941683)(125.83705404,700.89942034)
\curveto(125.78704605,700.8594169)(125.71204613,700.82941693)(125.61205404,700.80942034)
\curveto(125.52204632,700.79941696)(125.41704642,700.78941697)(125.29705404,700.77942034)
\lineto(124.95205404,700.77942034)
\curveto(124.842047,700.78941697)(124.7420471,700.79441697)(124.65205404,700.79442034)
\lineto(120.99205404,700.79442034)
\lineto(120.78205404,700.79442034)
\curveto(120.72205112,700.79441697)(120.66705117,700.78441698)(120.61705404,700.76442034)
\curveto(120.5370513,700.72441704)(120.48705135,700.68441708)(120.46705404,700.64442034)
\curveto(120.44705139,700.62441714)(120.42705141,700.58441718)(120.40705404,700.52442034)
\curveto(120.38705145,700.47441729)(120.38205146,700.42441734)(120.39205404,700.37442034)
\curveto(120.41205143,700.31441745)(120.42205142,700.25441751)(120.42205404,700.19442034)
\curveto(120.43205141,700.14441762)(120.44705139,700.08941767)(120.46705404,700.02942034)
\curveto(120.54705129,699.78941797)(120.6420512,699.58941817)(120.75205404,699.42942034)
\curveto(120.87205097,699.27941848)(121.03205081,699.14441862)(121.23205404,699.02442034)
\curveto(121.31205053,698.97441879)(121.39205045,698.93941882)(121.47205404,698.91942034)
\curveto(121.56205028,698.90941885)(121.65205019,698.88941887)(121.74205404,698.85942034)
\curveto(121.82205002,698.83941892)(121.93204991,698.82441894)(122.07205404,698.81442034)
\curveto(122.21204963,698.80441896)(122.33204951,698.80941895)(122.43205404,698.82942034)
\lineto(122.56705404,698.82942034)
\curveto(122.66704917,698.84941891)(122.75704908,698.86941889)(122.83705404,698.88942034)
\curveto(122.92704891,698.91941884)(123.01204883,698.94941881)(123.09205404,698.97942034)
\curveto(123.19204865,699.02941873)(123.30204854,699.09441867)(123.42205404,699.17442034)
\curveto(123.55204829,699.25441851)(123.64704819,699.33441843)(123.70705404,699.41442034)
\curveto(123.75704808,699.48441828)(123.80704803,699.54941821)(123.85705404,699.60942034)
\curveto(123.91704792,699.67941808)(123.98704785,699.72941803)(124.06705404,699.75942034)
\curveto(124.16704767,699.80941795)(124.29204755,699.82941793)(124.44205404,699.81942034)
\lineto(124.87705404,699.81942034)
\lineto(125.05705404,699.81942034)
\curveto(125.12704671,699.82941793)(125.18704665,699.82441794)(125.23705404,699.80442034)
\lineto(125.38705404,699.80442034)
\curveto(125.48704635,699.78441798)(125.55704628,699.759418)(125.59705404,699.72942034)
\curveto(125.6370462,699.70941805)(125.65704618,699.6644181)(125.65705404,699.59442034)
\curveto(125.66704617,699.52441824)(125.66204618,699.4644183)(125.64205404,699.41442034)
\curveto(125.59204625,699.27441849)(125.5370463,699.14941861)(125.47705404,699.03942034)
\curveto(125.41704642,698.92941883)(125.34704649,698.81941894)(125.26705404,698.70942034)
\curveto(125.04704679,698.37941938)(124.79704704,698.11441965)(124.51705404,697.91442034)
\curveto(124.2370476,697.71442005)(123.88704795,697.54442022)(123.46705404,697.40442034)
\curveto(123.35704848,697.3644204)(123.24704859,697.33942042)(123.13705404,697.32942034)
\curveto(123.02704881,697.31942044)(122.91204893,697.29942046)(122.79205404,697.26942034)
\curveto(122.75204909,697.2594205)(122.70704913,697.2594205)(122.65705404,697.26942034)
\curveto(122.61704922,697.26942049)(122.57704926,697.2644205)(122.53705404,697.25442034)
\lineto(122.37205404,697.25442034)
\curveto(122.32204952,697.23442053)(122.26204958,697.22942053)(122.19205404,697.23942034)
\curveto(122.13204971,697.23942052)(122.07704976,697.24442052)(122.02705404,697.25442034)
\curveto(121.94704989,697.2644205)(121.87704996,697.2644205)(121.81705404,697.25442034)
\curveto(121.75705008,697.24442052)(121.69205015,697.24942051)(121.62205404,697.26942034)
\curveto(121.57205027,697.28942047)(121.51705032,697.29942046)(121.45705404,697.29942034)
\curveto(121.39705044,697.29942046)(121.3420505,697.30942045)(121.29205404,697.32942034)
\curveto(121.18205066,697.34942041)(121.07205077,697.37442039)(120.96205404,697.40442034)
\curveto(120.85205099,697.42442034)(120.75205109,697.4594203)(120.66205404,697.50942034)
\curveto(120.55205129,697.54942021)(120.44705139,697.58442018)(120.34705404,697.61442034)
\curveto(120.25705158,697.65442011)(120.17205167,697.69942006)(120.09205404,697.74942034)
\curveto(119.77205207,697.94941981)(119.48705235,698.17941958)(119.23705404,698.43942034)
\curveto(118.98705285,698.70941905)(118.78205306,699.01941874)(118.62205404,699.36942034)
\curveto(118.57205327,699.47941828)(118.53205331,699.58941817)(118.50205404,699.69942034)
\curveto(118.47205337,699.81941794)(118.43205341,699.93941782)(118.38205404,700.05942034)
\curveto(118.37205347,700.09941766)(118.36705347,700.13441763)(118.36705404,700.16442034)
\curveto(118.36705347,700.20441756)(118.36205348,700.24441752)(118.35205404,700.28442034)
\curveto(118.31205353,700.40441736)(118.28705355,700.53441723)(118.27705404,700.67442034)
\lineto(118.24705404,701.09442034)
\curveto(118.24705359,701.14441662)(118.2420536,701.19941656)(118.23205404,701.25942034)
\curveto(118.23205361,701.31941644)(118.2370536,701.37441639)(118.24705404,701.42442034)
\lineto(118.24705404,701.60442034)
\lineto(118.29205404,701.96442034)
\curveto(118.33205351,702.13441563)(118.36705347,702.29941546)(118.39705404,702.45942034)
\curveto(118.42705341,702.61941514)(118.47205337,702.76941499)(118.53205404,702.90942034)
\curveto(118.96205288,703.94941381)(119.69205215,704.68441308)(120.72205404,705.11442034)
\curveto(120.86205098,705.17441259)(121.00205084,705.21441255)(121.14205404,705.23442034)
\curveto(121.29205055,705.2644125)(121.44705039,705.29941246)(121.60705404,705.33942034)
\curveto(121.68705015,705.34941241)(121.76205008,705.35441241)(121.83205404,705.35442034)
\curveto(121.90204994,705.35441241)(121.97704986,705.3594124)(122.05705404,705.36942034)
\curveto(122.56704927,705.37941238)(123.00204884,705.31941244)(123.36205404,705.18942034)
\curveto(123.73204811,705.06941269)(124.06204778,704.90941285)(124.35205404,704.70942034)
\curveto(124.4420474,704.64941311)(124.53204731,704.57941318)(124.62205404,704.49942034)
\curveto(124.71204713,704.42941333)(124.79204705,704.35441341)(124.86205404,704.27442034)
\curveto(124.89204695,704.22441354)(124.93204691,704.18441358)(124.98205404,704.15442034)
\curveto(125.06204678,704.04441372)(125.1370467,703.92941383)(125.20705404,703.80942034)
\curveto(125.27704656,703.69941406)(125.35204649,703.58441418)(125.43205404,703.46442034)
\curveto(125.48204636,703.37441439)(125.52204632,703.27941448)(125.55205404,703.17942034)
\curveto(125.59204625,703.08941467)(125.63204621,702.98941477)(125.67205404,702.87942034)
\curveto(125.72204612,702.74941501)(125.76204608,702.61441515)(125.79205404,702.47442034)
\curveto(125.82204602,702.33441543)(125.85704598,702.19441557)(125.89705404,702.05442034)
\curveto(125.91704592,701.97441579)(125.92204592,701.88441588)(125.91205404,701.78442034)
\curveto(125.91204593,701.69441607)(125.92204592,701.60941615)(125.94205404,701.52942034)
\lineto(125.94205404,701.36442034)
\moveto(123.69205404,702.24942034)
\curveto(123.76204808,702.34941541)(123.76704807,702.46941529)(123.70705404,702.60942034)
\curveto(123.65704818,702.759415)(123.61704822,702.86941489)(123.58705404,702.93942034)
\curveto(123.44704839,703.20941455)(123.26204858,703.41441435)(123.03205404,703.55442034)
\curveto(122.80204904,703.70441406)(122.48204936,703.78441398)(122.07205404,703.79442034)
\curveto(122.0420498,703.77441399)(122.00704983,703.76941399)(121.96705404,703.77942034)
\curveto(121.92704991,703.78941397)(121.89204995,703.78941397)(121.86205404,703.77942034)
\curveto(121.81205003,703.759414)(121.75705008,703.74441402)(121.69705404,703.73442034)
\curveto(121.6370502,703.73441403)(121.58205026,703.72441404)(121.53205404,703.70442034)
\curveto(121.09205075,703.5644142)(120.76705107,703.28941447)(120.55705404,702.87942034)
\curveto(120.5370513,702.83941492)(120.51205133,702.78441498)(120.48205404,702.71442034)
\curveto(120.46205138,702.65441511)(120.44705139,702.58941517)(120.43705404,702.51942034)
\curveto(120.42705141,702.4594153)(120.42705141,702.39941536)(120.43705404,702.33942034)
\curveto(120.45705138,702.27941548)(120.49205135,702.22941553)(120.54205404,702.18942034)
\curveto(120.62205122,702.13941562)(120.73205111,702.11441565)(120.87205404,702.11442034)
\lineto(121.27705404,702.11442034)
\lineto(122.94205404,702.11442034)
\lineto(123.37705404,702.11442034)
\curveto(123.5370483,702.12441564)(123.6420482,702.16941559)(123.69205404,702.24942034)
}
}
{
\newrgbcolor{curcolor}{0 0 0}
\pscustom[linestyle=none,fillstyle=solid,fillcolor=curcolor]
{
}
}
{
\newrgbcolor{curcolor}{0 0 0}
\pscustom[linestyle=none,fillstyle=solid,fillcolor=curcolor]
{
\newpath
\moveto(131.86049154,708.11442034)
\lineto(132.95549154,708.11442034)
\curveto(133.05548906,708.11440965)(133.15048896,708.10940965)(133.24049154,708.09942034)
\curveto(133.33048878,708.08940967)(133.40048871,708.0594097)(133.45049154,708.00942034)
\curveto(133.5104886,707.93940982)(133.54048857,707.84440992)(133.54049154,707.72442034)
\curveto(133.55048856,707.61441015)(133.55548856,707.49941026)(133.55549154,707.37942034)
\lineto(133.55549154,706.04442034)
\lineto(133.55549154,700.65942034)
\lineto(133.55549154,698.36442034)
\lineto(133.55549154,697.94442034)
\curveto(133.56548855,697.79441997)(133.54548857,697.67942008)(133.49549154,697.59942034)
\curveto(133.44548867,697.51942024)(133.35548876,697.4644203)(133.22549154,697.43442034)
\curveto(133.16548895,697.41442035)(133.09548902,697.40942035)(133.01549154,697.41942034)
\curveto(132.94548917,697.42942033)(132.87548924,697.43442033)(132.80549154,697.43442034)
\lineto(132.08549154,697.43442034)
\curveto(131.97549014,697.43442033)(131.87549024,697.43942032)(131.78549154,697.44942034)
\curveto(131.69549042,697.4594203)(131.62049049,697.48942027)(131.56049154,697.53942034)
\curveto(131.50049061,697.58942017)(131.46549065,697.6644201)(131.45549154,697.76442034)
\lineto(131.45549154,698.09442034)
\lineto(131.45549154,699.42942034)
\lineto(131.45549154,705.05442034)
\lineto(131.45549154,707.09442034)
\curveto(131.45549066,707.22441054)(131.45049066,707.37941038)(131.44049154,707.55942034)
\curveto(131.44049067,707.73941002)(131.46549065,707.86940989)(131.51549154,707.94942034)
\curveto(131.53549058,707.98940977)(131.56049055,708.01940974)(131.59049154,708.03942034)
\lineto(131.71049154,708.09942034)
\curveto(131.73049038,708.09940966)(131.75549036,708.09940966)(131.78549154,708.09942034)
\curveto(131.8154903,708.10940965)(131.84049027,708.11440965)(131.86049154,708.11442034)
}
}
{
\newrgbcolor{curcolor}{0 0 0}
\pscustom[linestyle=none,fillstyle=solid,fillcolor=curcolor]
{
\newpath
\moveto(142.98767904,701.60442034)
\curveto(143.00767047,701.54441622)(143.01767046,701.4594163)(143.01767904,701.34942034)
\curveto(143.01767046,701.23941652)(143.00767047,701.15441661)(142.98767904,701.09442034)
\lineto(142.98767904,700.94442034)
\curveto(142.96767051,700.8644169)(142.95767052,700.78441698)(142.95767904,700.70442034)
\curveto(142.96767051,700.62441714)(142.96267052,700.54441722)(142.94267904,700.46442034)
\curveto(142.92267056,700.39441737)(142.90767057,700.32941743)(142.89767904,700.26942034)
\curveto(142.88767059,700.20941755)(142.8776706,700.14441762)(142.86767904,700.07442034)
\curveto(142.82767065,699.9644178)(142.79267069,699.84941791)(142.76267904,699.72942034)
\curveto(142.73267075,699.61941814)(142.69267079,699.51441825)(142.64267904,699.41442034)
\curveto(142.43267105,698.93441883)(142.15767132,698.54441922)(141.81767904,698.24442034)
\curveto(141.477672,697.94441982)(141.06767241,697.69442007)(140.58767904,697.49442034)
\curveto(140.46767301,697.44442032)(140.34267314,697.40942035)(140.21267904,697.38942034)
\curveto(140.09267339,697.3594204)(139.96767351,697.32942043)(139.83767904,697.29942034)
\curveto(139.78767369,697.27942048)(139.73267375,697.26942049)(139.67267904,697.26942034)
\curveto(139.61267387,697.26942049)(139.55767392,697.2644205)(139.50767904,697.25442034)
\lineto(139.40267904,697.25442034)
\curveto(139.37267411,697.24442052)(139.34267414,697.23942052)(139.31267904,697.23942034)
\curveto(139.26267422,697.22942053)(139.1826743,697.22442054)(139.07267904,697.22442034)
\curveto(138.96267452,697.21442055)(138.8776746,697.21942054)(138.81767904,697.23942034)
\lineto(138.66767904,697.23942034)
\curveto(138.61767486,697.24942051)(138.56267492,697.25442051)(138.50267904,697.25442034)
\curveto(138.45267503,697.24442052)(138.40267508,697.24942051)(138.35267904,697.26942034)
\curveto(138.31267517,697.27942048)(138.27267521,697.28442048)(138.23267904,697.28442034)
\curveto(138.20267528,697.28442048)(138.16267532,697.28942047)(138.11267904,697.29942034)
\curveto(138.01267547,697.32942043)(137.91267557,697.35442041)(137.81267904,697.37442034)
\curveto(137.71267577,697.39442037)(137.61767586,697.42442034)(137.52767904,697.46442034)
\curveto(137.40767607,697.50442026)(137.29267619,697.54442022)(137.18267904,697.58442034)
\curveto(137.0826764,697.62442014)(136.9776765,697.67442009)(136.86767904,697.73442034)
\curveto(136.51767696,697.94441982)(136.21767726,698.18941957)(135.96767904,698.46942034)
\curveto(135.71767776,698.74941901)(135.50767797,699.08441868)(135.33767904,699.47442034)
\curveto(135.28767819,699.5644182)(135.24767823,699.6594181)(135.21767904,699.75942034)
\curveto(135.19767828,699.8594179)(135.17267831,699.9644178)(135.14267904,700.07442034)
\curveto(135.12267836,700.12441764)(135.11267837,700.16941759)(135.11267904,700.20942034)
\curveto(135.11267837,700.24941751)(135.10267838,700.29441747)(135.08267904,700.34442034)
\curveto(135.06267842,700.42441734)(135.05267843,700.50441726)(135.05267904,700.58442034)
\curveto(135.05267843,700.67441709)(135.04267844,700.759417)(135.02267904,700.83942034)
\curveto(135.01267847,700.88941687)(135.00767847,700.93441683)(135.00767904,700.97442034)
\lineto(135.00767904,701.10942034)
\curveto(134.98767849,701.16941659)(134.9776785,701.25441651)(134.97767904,701.36442034)
\curveto(134.98767849,701.47441629)(135.00267848,701.5594162)(135.02267904,701.61942034)
\lineto(135.02267904,701.72442034)
\curveto(135.03267845,701.77441599)(135.03267845,701.82441594)(135.02267904,701.87442034)
\curveto(135.02267846,701.93441583)(135.03267845,701.98941577)(135.05267904,702.03942034)
\curveto(135.06267842,702.08941567)(135.06767841,702.13441563)(135.06767904,702.17442034)
\curveto(135.06767841,702.22441554)(135.0776784,702.27441549)(135.09767904,702.32442034)
\curveto(135.13767834,702.45441531)(135.17267831,702.57941518)(135.20267904,702.69942034)
\curveto(135.23267825,702.82941493)(135.27267821,702.95441481)(135.32267904,703.07442034)
\curveto(135.50267798,703.48441428)(135.71767776,703.82441394)(135.96767904,704.09442034)
\curveto(136.21767726,704.37441339)(136.52267696,704.62941313)(136.88267904,704.85942034)
\curveto(136.9826765,704.90941285)(137.08767639,704.95441281)(137.19767904,704.99442034)
\curveto(137.30767617,705.03441273)(137.41767606,705.07941268)(137.52767904,705.12942034)
\curveto(137.65767582,705.17941258)(137.79267569,705.21441255)(137.93267904,705.23442034)
\curveto(138.07267541,705.25441251)(138.21767526,705.28441248)(138.36767904,705.32442034)
\curveto(138.44767503,705.33441243)(138.52267496,705.33941242)(138.59267904,705.33942034)
\curveto(138.66267482,705.33941242)(138.73267475,705.34441242)(138.80267904,705.35442034)
\curveto(139.3826741,705.3644124)(139.8826736,705.30441246)(140.30267904,705.17442034)
\curveto(140.73267275,705.04441272)(141.11267237,704.8644129)(141.44267904,704.63442034)
\curveto(141.55267193,704.55441321)(141.66267182,704.4644133)(141.77267904,704.36442034)
\curveto(141.89267159,704.27441349)(141.99267149,704.17441359)(142.07267904,704.06442034)
\curveto(142.15267133,703.9644138)(142.22267126,703.8644139)(142.28267904,703.76442034)
\curveto(142.35267113,703.6644141)(142.42267106,703.5594142)(142.49267904,703.44942034)
\curveto(142.56267092,703.33941442)(142.61767086,703.21941454)(142.65767904,703.08942034)
\curveto(142.69767078,702.96941479)(142.74267074,702.83941492)(142.79267904,702.69942034)
\curveto(142.82267066,702.61941514)(142.84767063,702.53441523)(142.86767904,702.44442034)
\lineto(142.92767904,702.17442034)
\curveto(142.93767054,702.13441563)(142.94267054,702.09441567)(142.94267904,702.05442034)
\curveto(142.94267054,702.01441575)(142.94767053,701.97441579)(142.95767904,701.93442034)
\curveto(142.9776705,701.88441588)(142.9826705,701.82941593)(142.97267904,701.76942034)
\curveto(142.96267052,701.70941605)(142.96767051,701.65441611)(142.98767904,701.60442034)
\moveto(140.88767904,701.06442034)
\curveto(140.89767258,701.11441665)(140.90267258,701.18441658)(140.90267904,701.27442034)
\curveto(140.90267258,701.37441639)(140.89767258,701.44941631)(140.88767904,701.49942034)
\lineto(140.88767904,701.61942034)
\curveto(140.86767261,701.66941609)(140.85767262,701.72441604)(140.85767904,701.78442034)
\curveto(140.85767262,701.84441592)(140.85267263,701.89941586)(140.84267904,701.94942034)
\curveto(140.84267264,701.98941577)(140.83767264,702.01941574)(140.82767904,702.03942034)
\lineto(140.76767904,702.27942034)
\curveto(140.75767272,702.36941539)(140.73767274,702.45441531)(140.70767904,702.53442034)
\curveto(140.59767288,702.79441497)(140.46767301,703.01441475)(140.31767904,703.19442034)
\curveto(140.16767331,703.38441438)(139.96767351,703.53441423)(139.71767904,703.64442034)
\curveto(139.65767382,703.6644141)(139.59767388,703.67941408)(139.53767904,703.68942034)
\curveto(139.477674,703.70941405)(139.41267407,703.72941403)(139.34267904,703.74942034)
\curveto(139.26267422,703.76941399)(139.1776743,703.77441399)(139.08767904,703.76442034)
\lineto(138.81767904,703.76442034)
\curveto(138.78767469,703.74441402)(138.75267473,703.73441403)(138.71267904,703.73442034)
\curveto(138.67267481,703.74441402)(138.63767484,703.74441402)(138.60767904,703.73442034)
\lineto(138.39767904,703.67442034)
\curveto(138.33767514,703.6644141)(138.2826752,703.64441412)(138.23267904,703.61442034)
\curveto(137.9826755,703.50441426)(137.7776757,703.34441442)(137.61767904,703.13442034)
\curveto(137.46767601,702.93441483)(137.34767613,702.69941506)(137.25767904,702.42942034)
\curveto(137.22767625,702.32941543)(137.20267628,702.22441554)(137.18267904,702.11442034)
\curveto(137.17267631,702.00441576)(137.15767632,701.89441587)(137.13767904,701.78442034)
\curveto(137.12767635,701.73441603)(137.12267636,701.68441608)(137.12267904,701.63442034)
\lineto(137.12267904,701.48442034)
\curveto(137.10267638,701.41441635)(137.09267639,701.30941645)(137.09267904,701.16942034)
\curveto(137.10267638,701.02941673)(137.11767636,700.92441684)(137.13767904,700.85442034)
\lineto(137.13767904,700.71942034)
\curveto(137.15767632,700.63941712)(137.17267631,700.5594172)(137.18267904,700.47942034)
\curveto(137.19267629,700.40941735)(137.20767627,700.33441743)(137.22767904,700.25442034)
\curveto(137.32767615,699.95441781)(137.43267605,699.70941805)(137.54267904,699.51942034)
\curveto(137.66267582,699.33941842)(137.84767563,699.17441859)(138.09767904,699.02442034)
\curveto(138.16767531,698.97441879)(138.24267524,698.93441883)(138.32267904,698.90442034)
\curveto(138.41267507,698.87441889)(138.50267498,698.84941891)(138.59267904,698.82942034)
\curveto(138.63267485,698.81941894)(138.66767481,698.81441895)(138.69767904,698.81442034)
\curveto(138.72767475,698.82441894)(138.76267472,698.82441894)(138.80267904,698.81442034)
\lineto(138.92267904,698.78442034)
\curveto(138.97267451,698.78441898)(139.01767446,698.78941897)(139.05767904,698.79942034)
\lineto(139.17767904,698.79942034)
\curveto(139.25767422,698.81941894)(139.33767414,698.83441893)(139.41767904,698.84442034)
\curveto(139.49767398,698.85441891)(139.57267391,698.87441889)(139.64267904,698.90442034)
\curveto(139.90267358,699.00441876)(140.11267337,699.13941862)(140.27267904,699.30942034)
\curveto(140.43267305,699.47941828)(140.56767291,699.68941807)(140.67767904,699.93942034)
\curveto(140.71767276,700.03941772)(140.74767273,700.13941762)(140.76767904,700.23942034)
\curveto(140.78767269,700.33941742)(140.81267267,700.44441732)(140.84267904,700.55442034)
\curveto(140.85267263,700.59441717)(140.85767262,700.62941713)(140.85767904,700.65942034)
\curveto(140.85767262,700.69941706)(140.86267262,700.73941702)(140.87267904,700.77942034)
\lineto(140.87267904,700.91442034)
\curveto(140.87267261,700.9644168)(140.8776726,701.01441675)(140.88767904,701.06442034)
}
}
{
\newrgbcolor{curcolor}{0 0 0}
\pscustom[linestyle=none,fillstyle=solid,fillcolor=curcolor]
{
\newpath
\moveto(147.35760092,705.36942034)
\curveto(148.10759642,705.38941237)(148.75759577,705.30441246)(149.30760092,705.11442034)
\curveto(149.86759466,704.93441283)(150.29259423,704.61941314)(150.58260092,704.16942034)
\curveto(150.65259387,704.0594137)(150.71259381,703.94441382)(150.76260092,703.82442034)
\curveto(150.8225937,703.71441405)(150.87259365,703.58941417)(150.91260092,703.44942034)
\curveto(150.93259359,703.38941437)(150.94259358,703.32441444)(150.94260092,703.25442034)
\curveto(150.94259358,703.18441458)(150.93259359,703.12441464)(150.91260092,703.07442034)
\curveto(150.87259365,703.01441475)(150.81759371,702.97441479)(150.74760092,702.95442034)
\curveto(150.69759383,702.93441483)(150.63759389,702.92441484)(150.56760092,702.92442034)
\lineto(150.35760092,702.92442034)
\lineto(149.69760092,702.92442034)
\curveto(149.6275949,702.92441484)(149.55759497,702.91941484)(149.48760092,702.90942034)
\curveto(149.41759511,702.90941485)(149.35259517,702.91941484)(149.29260092,702.93942034)
\curveto(149.19259533,702.9594148)(149.11759541,702.99941476)(149.06760092,703.05942034)
\curveto(149.01759551,703.11941464)(148.97259555,703.17941458)(148.93260092,703.23942034)
\lineto(148.81260092,703.44942034)
\curveto(148.78259574,703.52941423)(148.73259579,703.59441417)(148.66260092,703.64442034)
\curveto(148.56259596,703.72441404)(148.46259606,703.78441398)(148.36260092,703.82442034)
\curveto(148.27259625,703.8644139)(148.15759637,703.89941386)(148.01760092,703.92942034)
\curveto(147.94759658,703.94941381)(147.84259668,703.9644138)(147.70260092,703.97442034)
\curveto(147.57259695,703.98441378)(147.47259705,703.97941378)(147.40260092,703.95942034)
\lineto(147.29760092,703.95942034)
\lineto(147.14760092,703.92942034)
\curveto(147.10759742,703.92941383)(147.06259746,703.92441384)(147.01260092,703.91442034)
\curveto(146.84259768,703.8644139)(146.70259782,703.79441397)(146.59260092,703.70442034)
\curveto(146.49259803,703.62441414)(146.4225981,703.49941426)(146.38260092,703.32942034)
\curveto(146.36259816,703.2594145)(146.36259816,703.19441457)(146.38260092,703.13442034)
\curveto(146.40259812,703.07441469)(146.4225981,703.02441474)(146.44260092,702.98442034)
\curveto(146.51259801,702.8644149)(146.59259793,702.76941499)(146.68260092,702.69942034)
\curveto(146.78259774,702.62941513)(146.89759763,702.56941519)(147.02760092,702.51942034)
\curveto(147.21759731,702.43941532)(147.4225971,702.36941539)(147.64260092,702.30942034)
\lineto(148.33260092,702.15942034)
\curveto(148.57259595,702.11941564)(148.80259572,702.06941569)(149.02260092,702.00942034)
\curveto(149.25259527,701.9594158)(149.46759506,701.89441587)(149.66760092,701.81442034)
\curveto(149.75759477,701.77441599)(149.84259468,701.73941602)(149.92260092,701.70942034)
\curveto(150.01259451,701.68941607)(150.09759443,701.65441611)(150.17760092,701.60442034)
\curveto(150.36759416,701.48441628)(150.53759399,701.35441641)(150.68760092,701.21442034)
\curveto(150.84759368,701.07441669)(150.97259355,700.89941686)(151.06260092,700.68942034)
\curveto(151.09259343,700.61941714)(151.11759341,700.54941721)(151.13760092,700.47942034)
\curveto(151.15759337,700.40941735)(151.17759335,700.33441743)(151.19760092,700.25442034)
\curveto(151.20759332,700.19441757)(151.21259331,700.09941766)(151.21260092,699.96942034)
\curveto(151.2225933,699.84941791)(151.2225933,699.75441801)(151.21260092,699.68442034)
\lineto(151.21260092,699.60942034)
\curveto(151.19259333,699.54941821)(151.17759335,699.48941827)(151.16760092,699.42942034)
\curveto(151.16759336,699.37941838)(151.16259336,699.32941843)(151.15260092,699.27942034)
\curveto(151.08259344,698.97941878)(150.97259355,698.71441905)(150.82260092,698.48442034)
\curveto(150.66259386,698.24441952)(150.46759406,698.04941971)(150.23760092,697.89942034)
\curveto(150.00759452,697.74942001)(149.74759478,697.61942014)(149.45760092,697.50942034)
\curveto(149.34759518,697.4594203)(149.2275953,697.42442034)(149.09760092,697.40442034)
\curveto(148.97759555,697.38442038)(148.85759567,697.3594204)(148.73760092,697.32942034)
\curveto(148.64759588,697.30942045)(148.55259597,697.29942046)(148.45260092,697.29942034)
\curveto(148.36259616,697.28942047)(148.27259625,697.27442049)(148.18260092,697.25442034)
\lineto(147.91260092,697.25442034)
\curveto(147.85259667,697.23442053)(147.74759678,697.22442054)(147.59760092,697.22442034)
\curveto(147.45759707,697.22442054)(147.35759717,697.23442053)(147.29760092,697.25442034)
\curveto(147.26759726,697.25442051)(147.23259729,697.2594205)(147.19260092,697.26942034)
\lineto(147.08760092,697.26942034)
\curveto(146.96759756,697.28942047)(146.84759768,697.30442046)(146.72760092,697.31442034)
\curveto(146.60759792,697.32442044)(146.49259803,697.34442042)(146.38260092,697.37442034)
\curveto(145.99259853,697.48442028)(145.64759888,697.60942015)(145.34760092,697.74942034)
\curveto(145.04759948,697.89941986)(144.79259973,698.11941964)(144.58260092,698.40942034)
\curveto(144.44260008,698.59941916)(144.3226002,698.81941894)(144.22260092,699.06942034)
\curveto(144.20260032,699.12941863)(144.18260034,699.20941855)(144.16260092,699.30942034)
\curveto(144.14260038,699.3594184)(144.1276004,699.42941833)(144.11760092,699.51942034)
\curveto(144.10760042,699.60941815)(144.11260041,699.68441808)(144.13260092,699.74442034)
\curveto(144.16260036,699.81441795)(144.21260031,699.8644179)(144.28260092,699.89442034)
\curveto(144.33260019,699.91441785)(144.39260013,699.92441784)(144.46260092,699.92442034)
\lineto(144.68760092,699.92442034)
\lineto(145.39260092,699.92442034)
\lineto(145.63260092,699.92442034)
\curveto(145.71259881,699.92441784)(145.78259874,699.91441785)(145.84260092,699.89442034)
\curveto(145.95259857,699.85441791)(146.0225985,699.78941797)(146.05260092,699.69942034)
\curveto(146.09259843,699.60941815)(146.13759839,699.51441825)(146.18760092,699.41442034)
\curveto(146.20759832,699.3644184)(146.24259828,699.29941846)(146.29260092,699.21942034)
\curveto(146.35259817,699.13941862)(146.40259812,699.08941867)(146.44260092,699.06942034)
\curveto(146.56259796,698.96941879)(146.67759785,698.88941887)(146.78760092,698.82942034)
\curveto(146.89759763,698.77941898)(147.03759749,698.72941903)(147.20760092,698.67942034)
\curveto(147.25759727,698.6594191)(147.30759722,698.64941911)(147.35760092,698.64942034)
\curveto(147.40759712,698.6594191)(147.45759707,698.6594191)(147.50760092,698.64942034)
\curveto(147.58759694,698.62941913)(147.67259685,698.61941914)(147.76260092,698.61942034)
\curveto(147.86259666,698.62941913)(147.94759658,698.64441912)(148.01760092,698.66442034)
\curveto(148.06759646,698.67441909)(148.11259641,698.67941908)(148.15260092,698.67942034)
\curveto(148.20259632,698.67941908)(148.25259627,698.68941907)(148.30260092,698.70942034)
\curveto(148.44259608,698.759419)(148.56759596,698.81941894)(148.67760092,698.88942034)
\curveto(148.79759573,698.9594188)(148.89259563,699.04941871)(148.96260092,699.15942034)
\curveto(149.01259551,699.23941852)(149.05259547,699.3644184)(149.08260092,699.53442034)
\curveto(149.10259542,699.60441816)(149.10259542,699.66941809)(149.08260092,699.72942034)
\curveto(149.06259546,699.78941797)(149.04259548,699.83941792)(149.02260092,699.87942034)
\curveto(148.95259557,700.01941774)(148.86259566,700.12441764)(148.75260092,700.19442034)
\curveto(148.65259587,700.2644175)(148.53259599,700.32941743)(148.39260092,700.38942034)
\curveto(148.20259632,700.46941729)(148.00259652,700.53441723)(147.79260092,700.58442034)
\curveto(147.58259694,700.63441713)(147.37259715,700.68941707)(147.16260092,700.74942034)
\curveto(147.08259744,700.76941699)(146.99759753,700.78441698)(146.90760092,700.79442034)
\curveto(146.8275977,700.80441696)(146.74759778,700.81941694)(146.66760092,700.83942034)
\curveto(146.34759818,700.92941683)(146.04259848,701.01441675)(145.75260092,701.09442034)
\curveto(145.46259906,701.18441658)(145.19759933,701.31441645)(144.95760092,701.48442034)
\curveto(144.67759985,701.68441608)(144.47260005,701.95441581)(144.34260092,702.29442034)
\curveto(144.3226002,702.3644154)(144.30260022,702.4594153)(144.28260092,702.57942034)
\curveto(144.26260026,702.64941511)(144.24760028,702.73441503)(144.23760092,702.83442034)
\curveto(144.2276003,702.93441483)(144.23260029,703.02441474)(144.25260092,703.10442034)
\curveto(144.27260025,703.15441461)(144.27760025,703.19441457)(144.26760092,703.22442034)
\curveto(144.25760027,703.2644145)(144.26260026,703.30941445)(144.28260092,703.35942034)
\curveto(144.30260022,703.46941429)(144.3226002,703.56941419)(144.34260092,703.65942034)
\curveto(144.37260015,703.759414)(144.40760012,703.85441391)(144.44760092,703.94442034)
\curveto(144.57759995,704.23441353)(144.75759977,704.46941329)(144.98760092,704.64942034)
\curveto(145.21759931,704.82941293)(145.47759905,704.97441279)(145.76760092,705.08442034)
\curveto(145.87759865,705.13441263)(145.99259853,705.16941259)(146.11260092,705.18942034)
\curveto(146.23259829,705.21941254)(146.35759817,705.24941251)(146.48760092,705.27942034)
\curveto(146.54759798,705.29941246)(146.60759792,705.30941245)(146.66760092,705.30942034)
\lineto(146.84760092,705.33942034)
\curveto(146.9275976,705.34941241)(147.01259751,705.35441241)(147.10260092,705.35442034)
\curveto(147.19259733,705.35441241)(147.27759725,705.3594124)(147.35760092,705.36942034)
}
}
{
\newrgbcolor{curcolor}{0 0 0}
\pscustom[linestyle=none,fillstyle=solid,fillcolor=curcolor]
{
}
}
{
\newrgbcolor{curcolor}{0 0 0}
\pscustom[linestyle=none,fillstyle=solid,fillcolor=curcolor]
{
\newpath
\moveto(161.02939779,705.35442034)
\curveto(161.13939248,705.35441241)(161.23439238,705.34441242)(161.31439779,705.32442034)
\curveto(161.40439221,705.30441246)(161.47439214,705.2594125)(161.52439779,705.18942034)
\curveto(161.58439203,705.10941265)(161.614392,704.96941279)(161.61439779,704.76942034)
\lineto(161.61439779,704.25942034)
\lineto(161.61439779,703.88442034)
\curveto(161.62439199,703.74441402)(161.60939201,703.63441413)(161.56939779,703.55442034)
\curveto(161.52939209,703.48441428)(161.46939215,703.43941432)(161.38939779,703.41942034)
\curveto(161.3193923,703.39941436)(161.23439238,703.38941437)(161.13439779,703.38942034)
\curveto(161.04439257,703.38941437)(160.94439267,703.39441437)(160.83439779,703.40442034)
\curveto(160.73439288,703.41441435)(160.63939298,703.40941435)(160.54939779,703.38942034)
\curveto(160.47939314,703.36941439)(160.40939321,703.35441441)(160.33939779,703.34442034)
\curveto(160.26939335,703.34441442)(160.20439341,703.33441443)(160.14439779,703.31442034)
\curveto(159.98439363,703.2644145)(159.82439379,703.18941457)(159.66439779,703.08942034)
\curveto(159.50439411,702.99941476)(159.37939424,702.89441487)(159.28939779,702.77442034)
\curveto(159.23939438,702.69441507)(159.18439443,702.60941515)(159.12439779,702.51942034)
\curveto(159.07439454,702.43941532)(159.02439459,702.35441541)(158.97439779,702.26442034)
\curveto(158.94439467,702.18441558)(158.9143947,702.09941566)(158.88439779,702.00942034)
\lineto(158.82439779,701.76942034)
\curveto(158.80439481,701.69941606)(158.79439482,701.62441614)(158.79439779,701.54442034)
\curveto(158.79439482,701.47441629)(158.78439483,701.40441636)(158.76439779,701.33442034)
\curveto(158.75439486,701.29441647)(158.74939487,701.25441651)(158.74939779,701.21442034)
\curveto(158.75939486,701.18441658)(158.75939486,701.15441661)(158.74939779,701.12442034)
\lineto(158.74939779,700.88442034)
\curveto(158.72939489,700.81441695)(158.72439489,700.73441703)(158.73439779,700.64442034)
\curveto(158.74439487,700.5644172)(158.74939487,700.48441728)(158.74939779,700.40442034)
\lineto(158.74939779,699.44442034)
\lineto(158.74939779,698.16942034)
\curveto(158.74939487,698.03941972)(158.74439487,697.91941984)(158.73439779,697.80942034)
\curveto(158.72439489,697.69942006)(158.69439492,697.60942015)(158.64439779,697.53942034)
\curveto(158.62439499,697.50942025)(158.58939503,697.48442028)(158.53939779,697.46442034)
\curveto(158.49939512,697.45442031)(158.45439516,697.44442032)(158.40439779,697.43442034)
\lineto(158.32939779,697.43442034)
\curveto(158.27939534,697.42442034)(158.22439539,697.41942034)(158.16439779,697.41942034)
\lineto(157.99939779,697.41942034)
\lineto(157.35439779,697.41942034)
\curveto(157.29439632,697.42942033)(157.22939639,697.43442033)(157.15939779,697.43442034)
\lineto(156.96439779,697.43442034)
\curveto(156.9143967,697.45442031)(156.86439675,697.46942029)(156.81439779,697.47942034)
\curveto(156.76439685,697.49942026)(156.72939689,697.53442023)(156.70939779,697.58442034)
\curveto(156.66939695,697.63442013)(156.64439697,697.70442006)(156.63439779,697.79442034)
\lineto(156.63439779,698.09442034)
\lineto(156.63439779,699.11442034)
\lineto(156.63439779,703.34442034)
\lineto(156.63439779,704.45442034)
\lineto(156.63439779,704.73942034)
\curveto(156.63439698,704.83941292)(156.65439696,704.91941284)(156.69439779,704.97942034)
\curveto(156.74439687,705.0594127)(156.8193968,705.10941265)(156.91939779,705.12942034)
\curveto(157.0193966,705.14941261)(157.13939648,705.1594126)(157.27939779,705.15942034)
\lineto(158.04439779,705.15942034)
\curveto(158.16439545,705.1594126)(158.26939535,705.14941261)(158.35939779,705.12942034)
\curveto(158.44939517,705.11941264)(158.5193951,705.07441269)(158.56939779,704.99442034)
\curveto(158.59939502,704.94441282)(158.614395,704.87441289)(158.61439779,704.78442034)
\lineto(158.64439779,704.51442034)
\curveto(158.65439496,704.43441333)(158.66939495,704.3594134)(158.68939779,704.28942034)
\curveto(158.7193949,704.21941354)(158.76939485,704.18441358)(158.83939779,704.18442034)
\curveto(158.85939476,704.20441356)(158.87939474,704.21441355)(158.89939779,704.21442034)
\curveto(158.9193947,704.21441355)(158.93939468,704.22441354)(158.95939779,704.24442034)
\curveto(159.0193946,704.29441347)(159.06939455,704.34941341)(159.10939779,704.40942034)
\curveto(159.15939446,704.47941328)(159.2193944,704.53941322)(159.28939779,704.58942034)
\curveto(159.32939429,704.61941314)(159.36439425,704.64941311)(159.39439779,704.67942034)
\curveto(159.42439419,704.71941304)(159.45939416,704.75441301)(159.49939779,704.78442034)
\lineto(159.76939779,704.96442034)
\curveto(159.86939375,705.02441274)(159.96939365,705.07941268)(160.06939779,705.12942034)
\curveto(160.16939345,705.16941259)(160.26939335,705.20441256)(160.36939779,705.23442034)
\lineto(160.69939779,705.32442034)
\curveto(160.72939289,705.33441243)(160.78439283,705.33441243)(160.86439779,705.32442034)
\curveto(160.95439266,705.32441244)(161.00939261,705.33441243)(161.02939779,705.35442034)
}
}
{
\newrgbcolor{curcolor}{0 0 0}
\pscustom[linestyle=none,fillstyle=solid,fillcolor=curcolor]
{
\newpath
\moveto(169.53580404,701.36442034)
\curveto(169.55579588,701.28441648)(169.55579588,701.19441657)(169.53580404,701.09442034)
\curveto(169.51579592,700.99441677)(169.48079595,700.92941683)(169.43080404,700.89942034)
\curveto(169.38079605,700.8594169)(169.30579613,700.82941693)(169.20580404,700.80942034)
\curveto(169.11579632,700.79941696)(169.01079642,700.78941697)(168.89080404,700.77942034)
\lineto(168.54580404,700.77942034)
\curveto(168.435797,700.78941697)(168.3357971,700.79441697)(168.24580404,700.79442034)
\lineto(164.58580404,700.79442034)
\lineto(164.37580404,700.79442034)
\curveto(164.31580112,700.79441697)(164.26080117,700.78441698)(164.21080404,700.76442034)
\curveto(164.1308013,700.72441704)(164.08080135,700.68441708)(164.06080404,700.64442034)
\curveto(164.04080139,700.62441714)(164.02080141,700.58441718)(164.00080404,700.52442034)
\curveto(163.98080145,700.47441729)(163.97580146,700.42441734)(163.98580404,700.37442034)
\curveto(164.00580143,700.31441745)(164.01580142,700.25441751)(164.01580404,700.19442034)
\curveto(164.02580141,700.14441762)(164.04080139,700.08941767)(164.06080404,700.02942034)
\curveto(164.14080129,699.78941797)(164.2358012,699.58941817)(164.34580404,699.42942034)
\curveto(164.46580097,699.27941848)(164.62580081,699.14441862)(164.82580404,699.02442034)
\curveto(164.90580053,698.97441879)(164.98580045,698.93941882)(165.06580404,698.91942034)
\curveto(165.15580028,698.90941885)(165.24580019,698.88941887)(165.33580404,698.85942034)
\curveto(165.41580002,698.83941892)(165.52579991,698.82441894)(165.66580404,698.81442034)
\curveto(165.80579963,698.80441896)(165.92579951,698.80941895)(166.02580404,698.82942034)
\lineto(166.16080404,698.82942034)
\curveto(166.26079917,698.84941891)(166.35079908,698.86941889)(166.43080404,698.88942034)
\curveto(166.52079891,698.91941884)(166.60579883,698.94941881)(166.68580404,698.97942034)
\curveto(166.78579865,699.02941873)(166.89579854,699.09441867)(167.01580404,699.17442034)
\curveto(167.14579829,699.25441851)(167.24079819,699.33441843)(167.30080404,699.41442034)
\curveto(167.35079808,699.48441828)(167.40079803,699.54941821)(167.45080404,699.60942034)
\curveto(167.51079792,699.67941808)(167.58079785,699.72941803)(167.66080404,699.75942034)
\curveto(167.76079767,699.80941795)(167.88579755,699.82941793)(168.03580404,699.81942034)
\lineto(168.47080404,699.81942034)
\lineto(168.65080404,699.81942034)
\curveto(168.72079671,699.82941793)(168.78079665,699.82441794)(168.83080404,699.80442034)
\lineto(168.98080404,699.80442034)
\curveto(169.08079635,699.78441798)(169.15079628,699.759418)(169.19080404,699.72942034)
\curveto(169.2307962,699.70941805)(169.25079618,699.6644181)(169.25080404,699.59442034)
\curveto(169.26079617,699.52441824)(169.25579618,699.4644183)(169.23580404,699.41442034)
\curveto(169.18579625,699.27441849)(169.1307963,699.14941861)(169.07080404,699.03942034)
\curveto(169.01079642,698.92941883)(168.94079649,698.81941894)(168.86080404,698.70942034)
\curveto(168.64079679,698.37941938)(168.39079704,698.11441965)(168.11080404,697.91442034)
\curveto(167.8307976,697.71442005)(167.48079795,697.54442022)(167.06080404,697.40442034)
\curveto(166.95079848,697.3644204)(166.84079859,697.33942042)(166.73080404,697.32942034)
\curveto(166.62079881,697.31942044)(166.50579893,697.29942046)(166.38580404,697.26942034)
\curveto(166.34579909,697.2594205)(166.30079913,697.2594205)(166.25080404,697.26942034)
\curveto(166.21079922,697.26942049)(166.17079926,697.2644205)(166.13080404,697.25442034)
\lineto(165.96580404,697.25442034)
\curveto(165.91579952,697.23442053)(165.85579958,697.22942053)(165.78580404,697.23942034)
\curveto(165.72579971,697.23942052)(165.67079976,697.24442052)(165.62080404,697.25442034)
\curveto(165.54079989,697.2644205)(165.47079996,697.2644205)(165.41080404,697.25442034)
\curveto(165.35080008,697.24442052)(165.28580015,697.24942051)(165.21580404,697.26942034)
\curveto(165.16580027,697.28942047)(165.11080032,697.29942046)(165.05080404,697.29942034)
\curveto(164.99080044,697.29942046)(164.9358005,697.30942045)(164.88580404,697.32942034)
\curveto(164.77580066,697.34942041)(164.66580077,697.37442039)(164.55580404,697.40442034)
\curveto(164.44580099,697.42442034)(164.34580109,697.4594203)(164.25580404,697.50942034)
\curveto(164.14580129,697.54942021)(164.04080139,697.58442018)(163.94080404,697.61442034)
\curveto(163.85080158,697.65442011)(163.76580167,697.69942006)(163.68580404,697.74942034)
\curveto(163.36580207,697.94941981)(163.08080235,698.17941958)(162.83080404,698.43942034)
\curveto(162.58080285,698.70941905)(162.37580306,699.01941874)(162.21580404,699.36942034)
\curveto(162.16580327,699.47941828)(162.12580331,699.58941817)(162.09580404,699.69942034)
\curveto(162.06580337,699.81941794)(162.02580341,699.93941782)(161.97580404,700.05942034)
\curveto(161.96580347,700.09941766)(161.96080347,700.13441763)(161.96080404,700.16442034)
\curveto(161.96080347,700.20441756)(161.95580348,700.24441752)(161.94580404,700.28442034)
\curveto(161.90580353,700.40441736)(161.88080355,700.53441723)(161.87080404,700.67442034)
\lineto(161.84080404,701.09442034)
\curveto(161.84080359,701.14441662)(161.8358036,701.19941656)(161.82580404,701.25942034)
\curveto(161.82580361,701.31941644)(161.8308036,701.37441639)(161.84080404,701.42442034)
\lineto(161.84080404,701.60442034)
\lineto(161.88580404,701.96442034)
\curveto(161.92580351,702.13441563)(161.96080347,702.29941546)(161.99080404,702.45942034)
\curveto(162.02080341,702.61941514)(162.06580337,702.76941499)(162.12580404,702.90942034)
\curveto(162.55580288,703.94941381)(163.28580215,704.68441308)(164.31580404,705.11442034)
\curveto(164.45580098,705.17441259)(164.59580084,705.21441255)(164.73580404,705.23442034)
\curveto(164.88580055,705.2644125)(165.04080039,705.29941246)(165.20080404,705.33942034)
\curveto(165.28080015,705.34941241)(165.35580008,705.35441241)(165.42580404,705.35442034)
\curveto(165.49579994,705.35441241)(165.57079986,705.3594124)(165.65080404,705.36942034)
\curveto(166.16079927,705.37941238)(166.59579884,705.31941244)(166.95580404,705.18942034)
\curveto(167.32579811,705.06941269)(167.65579778,704.90941285)(167.94580404,704.70942034)
\curveto(168.0357974,704.64941311)(168.12579731,704.57941318)(168.21580404,704.49942034)
\curveto(168.30579713,704.42941333)(168.38579705,704.35441341)(168.45580404,704.27442034)
\curveto(168.48579695,704.22441354)(168.52579691,704.18441358)(168.57580404,704.15442034)
\curveto(168.65579678,704.04441372)(168.7307967,703.92941383)(168.80080404,703.80942034)
\curveto(168.87079656,703.69941406)(168.94579649,703.58441418)(169.02580404,703.46442034)
\curveto(169.07579636,703.37441439)(169.11579632,703.27941448)(169.14580404,703.17942034)
\curveto(169.18579625,703.08941467)(169.22579621,702.98941477)(169.26580404,702.87942034)
\curveto(169.31579612,702.74941501)(169.35579608,702.61441515)(169.38580404,702.47442034)
\curveto(169.41579602,702.33441543)(169.45079598,702.19441557)(169.49080404,702.05442034)
\curveto(169.51079592,701.97441579)(169.51579592,701.88441588)(169.50580404,701.78442034)
\curveto(169.50579593,701.69441607)(169.51579592,701.60941615)(169.53580404,701.52942034)
\lineto(169.53580404,701.36442034)
\moveto(167.28580404,702.24942034)
\curveto(167.35579808,702.34941541)(167.36079807,702.46941529)(167.30080404,702.60942034)
\curveto(167.25079818,702.759415)(167.21079822,702.86941489)(167.18080404,702.93942034)
\curveto(167.04079839,703.20941455)(166.85579858,703.41441435)(166.62580404,703.55442034)
\curveto(166.39579904,703.70441406)(166.07579936,703.78441398)(165.66580404,703.79442034)
\curveto(165.6357998,703.77441399)(165.60079983,703.76941399)(165.56080404,703.77942034)
\curveto(165.52079991,703.78941397)(165.48579995,703.78941397)(165.45580404,703.77942034)
\curveto(165.40580003,703.759414)(165.35080008,703.74441402)(165.29080404,703.73442034)
\curveto(165.2308002,703.73441403)(165.17580026,703.72441404)(165.12580404,703.70442034)
\curveto(164.68580075,703.5644142)(164.36080107,703.28941447)(164.15080404,702.87942034)
\curveto(164.1308013,702.83941492)(164.10580133,702.78441498)(164.07580404,702.71442034)
\curveto(164.05580138,702.65441511)(164.04080139,702.58941517)(164.03080404,702.51942034)
\curveto(164.02080141,702.4594153)(164.02080141,702.39941536)(164.03080404,702.33942034)
\curveto(164.05080138,702.27941548)(164.08580135,702.22941553)(164.13580404,702.18942034)
\curveto(164.21580122,702.13941562)(164.32580111,702.11441565)(164.46580404,702.11442034)
\lineto(164.87080404,702.11442034)
\lineto(166.53580404,702.11442034)
\lineto(166.97080404,702.11442034)
\curveto(167.1307983,702.12441564)(167.2357982,702.16941559)(167.28580404,702.24942034)
}
}
{
\newrgbcolor{curcolor}{0 0 0}
\pscustom[linestyle=none,fillstyle=solid,fillcolor=curcolor]
{
\newpath
\moveto(174.35408529,705.36942034)
\curveto(175.16408013,705.38941237)(175.83907946,705.26941249)(176.37908529,705.00942034)
\curveto(176.92907837,704.74941301)(177.36407793,704.37941338)(177.68408529,703.89942034)
\curveto(177.84407745,703.6594141)(177.96407733,703.38441438)(178.04408529,703.07442034)
\curveto(178.06407723,703.02441474)(178.07907722,702.9594148)(178.08908529,702.87942034)
\curveto(178.10907719,702.79941496)(178.10907719,702.72941503)(178.08908529,702.66942034)
\curveto(178.04907725,702.5594152)(177.97907732,702.49441527)(177.87908529,702.47442034)
\curveto(177.77907752,702.4644153)(177.65907764,702.4594153)(177.51908529,702.45942034)
\lineto(176.73908529,702.45942034)
\lineto(176.45408529,702.45942034)
\curveto(176.36407893,702.4594153)(176.28907901,702.47941528)(176.22908529,702.51942034)
\curveto(176.14907915,702.5594152)(176.0940792,702.61941514)(176.06408529,702.69942034)
\curveto(176.03407926,702.78941497)(175.9940793,702.87941488)(175.94408529,702.96942034)
\curveto(175.88407941,703.07941468)(175.81907948,703.17941458)(175.74908529,703.26942034)
\curveto(175.67907962,703.3594144)(175.5990797,703.43941432)(175.50908529,703.50942034)
\curveto(175.36907993,703.59941416)(175.21408008,703.66941409)(175.04408529,703.71942034)
\curveto(174.98408031,703.73941402)(174.92408037,703.74941401)(174.86408529,703.74942034)
\curveto(174.80408049,703.74941401)(174.74908055,703.759414)(174.69908529,703.77942034)
\lineto(174.54908529,703.77942034)
\curveto(174.34908095,703.77941398)(174.18908111,703.759414)(174.06908529,703.71942034)
\curveto(173.77908152,703.62941413)(173.54408175,703.48941427)(173.36408529,703.29942034)
\curveto(173.18408211,703.11941464)(173.03908226,702.89941486)(172.92908529,702.63942034)
\curveto(172.87908242,702.52941523)(172.83908246,702.40941535)(172.80908529,702.27942034)
\curveto(172.78908251,702.1594156)(172.76408253,702.02941573)(172.73408529,701.88942034)
\curveto(172.72408257,701.84941591)(172.71908258,701.80941595)(172.71908529,701.76942034)
\curveto(172.71908258,701.72941603)(172.71408258,701.68941607)(172.70408529,701.64942034)
\curveto(172.68408261,701.54941621)(172.67408262,701.40941635)(172.67408529,701.22942034)
\curveto(172.68408261,701.04941671)(172.6990826,700.90941685)(172.71908529,700.80942034)
\curveto(172.71908258,700.72941703)(172.72408257,700.67441709)(172.73408529,700.64442034)
\curveto(172.75408254,700.57441719)(172.76408253,700.50441726)(172.76408529,700.43442034)
\curveto(172.77408252,700.3644174)(172.78908251,700.29441747)(172.80908529,700.22442034)
\curveto(172.88908241,699.99441777)(172.98408231,699.78441798)(173.09408529,699.59442034)
\curveto(173.20408209,699.40441836)(173.34408195,699.24441852)(173.51408529,699.11442034)
\curveto(173.55408174,699.08441868)(173.61408168,699.04941871)(173.69408529,699.00942034)
\curveto(173.80408149,698.93941882)(173.91408138,698.89441887)(174.02408529,698.87442034)
\curveto(174.14408115,698.85441891)(174.28908101,698.83441893)(174.45908529,698.81442034)
\lineto(174.54908529,698.81442034)
\curveto(174.58908071,698.81441895)(174.61908068,698.81941894)(174.63908529,698.82942034)
\lineto(174.77408529,698.82942034)
\curveto(174.84408045,698.84941891)(174.90908039,698.8644189)(174.96908529,698.87442034)
\curveto(175.03908026,698.89441887)(175.10408019,698.91441885)(175.16408529,698.93442034)
\curveto(175.46407983,699.0644187)(175.6940796,699.25441851)(175.85408529,699.50442034)
\curveto(175.8940794,699.55441821)(175.92907937,699.60941815)(175.95908529,699.66942034)
\curveto(175.98907931,699.73941802)(176.01407928,699.79941796)(176.03408529,699.84942034)
\curveto(176.07407922,699.9594178)(176.10907919,700.05441771)(176.13908529,700.13442034)
\curveto(176.16907913,700.22441754)(176.23907906,700.29441747)(176.34908529,700.34442034)
\curveto(176.43907886,700.38441738)(176.58407871,700.39941736)(176.78408529,700.38942034)
\lineto(177.27908529,700.38942034)
\lineto(177.48908529,700.38942034)
\curveto(177.56907773,700.39941736)(177.63407766,700.39441737)(177.68408529,700.37442034)
\lineto(177.80408529,700.37442034)
\lineto(177.92408529,700.34442034)
\curveto(177.96407733,700.34441742)(177.9940773,700.33441743)(178.01408529,700.31442034)
\curveto(178.06407723,700.27441749)(178.0940772,700.21441755)(178.10408529,700.13442034)
\curveto(178.12407717,700.0644177)(178.12407717,699.98941777)(178.10408529,699.90942034)
\curveto(178.01407728,699.57941818)(177.90407739,699.28441848)(177.77408529,699.02442034)
\curveto(177.36407793,698.25441951)(176.70907859,697.71942004)(175.80908529,697.41942034)
\curveto(175.70907959,697.38942037)(175.60407969,697.36942039)(175.49408529,697.35942034)
\curveto(175.38407991,697.33942042)(175.27408002,697.31442045)(175.16408529,697.28442034)
\curveto(175.10408019,697.27442049)(175.04408025,697.26942049)(174.98408529,697.26942034)
\curveto(174.92408037,697.26942049)(174.86408043,697.2644205)(174.80408529,697.25442034)
\lineto(174.63908529,697.25442034)
\curveto(174.58908071,697.23442053)(174.51408078,697.22942053)(174.41408529,697.23942034)
\curveto(174.31408098,697.23942052)(174.23908106,697.24442052)(174.18908529,697.25442034)
\curveto(174.10908119,697.27442049)(174.03408126,697.28442048)(173.96408529,697.28442034)
\curveto(173.90408139,697.27442049)(173.83908146,697.27942048)(173.76908529,697.29942034)
\lineto(173.61908529,697.32942034)
\curveto(173.56908173,697.32942043)(173.51908178,697.33442043)(173.46908529,697.34442034)
\curveto(173.35908194,697.37442039)(173.25408204,697.40442036)(173.15408529,697.43442034)
\curveto(173.05408224,697.4644203)(172.95908234,697.49942026)(172.86908529,697.53942034)
\curveto(172.3990829,697.73942002)(172.00408329,697.99441977)(171.68408529,698.30442034)
\curveto(171.36408393,698.62441914)(171.10408419,699.01941874)(170.90408529,699.48942034)
\curveto(170.85408444,699.57941818)(170.81408448,699.67441809)(170.78408529,699.77442034)
\lineto(170.69408529,700.10442034)
\curveto(170.68408461,700.14441762)(170.67908462,700.17941758)(170.67908529,700.20942034)
\curveto(170.67908462,700.24941751)(170.66908463,700.29441747)(170.64908529,700.34442034)
\curveto(170.62908467,700.41441735)(170.61908468,700.48441728)(170.61908529,700.55442034)
\curveto(170.61908468,700.63441713)(170.60908469,700.70941705)(170.58908529,700.77942034)
\lineto(170.58908529,701.03442034)
\curveto(170.56908473,701.08441668)(170.55908474,701.13941662)(170.55908529,701.19942034)
\curveto(170.55908474,701.26941649)(170.56908473,701.32941643)(170.58908529,701.37942034)
\curveto(170.5990847,701.42941633)(170.5990847,701.47441629)(170.58908529,701.51442034)
\curveto(170.57908472,701.55441621)(170.57908472,701.59441617)(170.58908529,701.63442034)
\curveto(170.60908469,701.70441606)(170.61408468,701.76941599)(170.60408529,701.82942034)
\curveto(170.60408469,701.88941587)(170.61408468,701.94941581)(170.63408529,702.00942034)
\curveto(170.68408461,702.18941557)(170.72408457,702.3594154)(170.75408529,702.51942034)
\curveto(170.78408451,702.68941507)(170.82908447,702.85441491)(170.88908529,703.01442034)
\curveto(171.10908419,703.52441424)(171.38408391,703.94941381)(171.71408529,704.28942034)
\curveto(172.05408324,704.62941313)(172.48408281,704.90441286)(173.00408529,705.11442034)
\curveto(173.14408215,705.17441259)(173.28908201,705.21441255)(173.43908529,705.23442034)
\curveto(173.58908171,705.2644125)(173.74408155,705.29941246)(173.90408529,705.33942034)
\curveto(173.98408131,705.34941241)(174.05908124,705.35441241)(174.12908529,705.35442034)
\curveto(174.1990811,705.35441241)(174.27408102,705.3594124)(174.35408529,705.36942034)
}
}
{
\newrgbcolor{curcolor}{0 0 0}
\pscustom[linestyle=none,fillstyle=solid,fillcolor=curcolor]
{
\newpath
\moveto(179.81736654,705.14442034)
\lineto(180.94236654,705.14442034)
\curveto(181.05236411,705.14441262)(181.15236401,705.13941262)(181.24236654,705.12942034)
\curveto(181.33236383,705.11941264)(181.39736376,705.08441268)(181.43736654,705.02442034)
\curveto(181.48736367,704.9644128)(181.51736364,704.87941288)(181.52736654,704.76942034)
\curveto(181.53736362,704.66941309)(181.54236362,704.5644132)(181.54236654,704.45442034)
\lineto(181.54236654,703.40442034)
\lineto(181.54236654,701.16942034)
\curveto(181.54236362,700.80941695)(181.5573636,700.46941729)(181.58736654,700.14942034)
\curveto(181.61736354,699.82941793)(181.70736345,699.5644182)(181.85736654,699.35442034)
\curveto(181.99736316,699.14441862)(182.22236294,698.99441877)(182.53236654,698.90442034)
\curveto(182.58236258,698.89441887)(182.62236254,698.88941887)(182.65236654,698.88942034)
\curveto(182.69236247,698.88941887)(182.73736242,698.88441888)(182.78736654,698.87442034)
\curveto(182.83736232,698.8644189)(182.89236227,698.8594189)(182.95236654,698.85942034)
\curveto(183.01236215,698.8594189)(183.0573621,698.8644189)(183.08736654,698.87442034)
\curveto(183.13736202,698.89441887)(183.17736198,698.89941886)(183.20736654,698.88942034)
\curveto(183.24736191,698.87941888)(183.28736187,698.88441888)(183.32736654,698.90442034)
\curveto(183.53736162,698.95441881)(183.70236146,699.01941874)(183.82236654,699.09942034)
\curveto(184.00236116,699.20941855)(184.14236102,699.34941841)(184.24236654,699.51942034)
\curveto(184.35236081,699.69941806)(184.42736073,699.89441787)(184.46736654,700.10442034)
\curveto(184.51736064,700.32441744)(184.54736061,700.5644172)(184.55736654,700.82442034)
\curveto(184.56736059,701.09441667)(184.57236059,701.37441639)(184.57236654,701.66442034)
\lineto(184.57236654,703.47942034)
\lineto(184.57236654,704.45442034)
\lineto(184.57236654,704.72442034)
\curveto(184.57236059,704.82441294)(184.59236057,704.90441286)(184.63236654,704.96442034)
\curveto(184.68236048,705.05441271)(184.7573604,705.10441266)(184.85736654,705.11442034)
\curveto(184.9573602,705.13441263)(185.07736008,705.14441262)(185.21736654,705.14442034)
\lineto(186.01236654,705.14442034)
\lineto(186.29736654,705.14442034)
\curveto(186.38735877,705.14441262)(186.4623587,705.12441264)(186.52236654,705.08442034)
\curveto(186.60235856,705.03441273)(186.64735851,704.9594128)(186.65736654,704.85942034)
\curveto(186.66735849,704.759413)(186.67235849,704.64441312)(186.67236654,704.51442034)
\lineto(186.67236654,703.37442034)
\lineto(186.67236654,699.15942034)
\lineto(186.67236654,698.09442034)
\lineto(186.67236654,697.79442034)
\curveto(186.67235849,697.69442007)(186.65235851,697.61942014)(186.61236654,697.56942034)
\curveto(186.5623586,697.48942027)(186.48735867,697.44442032)(186.38736654,697.43442034)
\curveto(186.28735887,697.42442034)(186.18235898,697.41942034)(186.07236654,697.41942034)
\lineto(185.26236654,697.41942034)
\curveto(185.15236001,697.41942034)(185.05236011,697.42442034)(184.96236654,697.43442034)
\curveto(184.88236028,697.44442032)(184.81736034,697.48442028)(184.76736654,697.55442034)
\curveto(184.74736041,697.58442018)(184.72736043,697.62942013)(184.70736654,697.68942034)
\curveto(184.69736046,697.74942001)(184.68236048,697.80941995)(184.66236654,697.86942034)
\curveto(184.65236051,697.92941983)(184.63736052,697.98441978)(184.61736654,698.03442034)
\curveto(184.59736056,698.08441968)(184.56736059,698.11441965)(184.52736654,698.12442034)
\curveto(184.50736065,698.14441962)(184.48236068,698.14941961)(184.45236654,698.13942034)
\curveto(184.42236074,698.12941963)(184.39736076,698.11941964)(184.37736654,698.10942034)
\curveto(184.30736085,698.06941969)(184.24736091,698.02441974)(184.19736654,697.97442034)
\curveto(184.14736101,697.92441984)(184.09236107,697.87941988)(184.03236654,697.83942034)
\curveto(183.99236117,697.80941995)(183.95236121,697.77441999)(183.91236654,697.73442034)
\curveto(183.88236128,697.70442006)(183.84236132,697.67442009)(183.79236654,697.64442034)
\curveto(183.5623616,697.50442026)(183.29236187,697.39442037)(182.98236654,697.31442034)
\curveto(182.91236225,697.29442047)(182.84236232,697.28442048)(182.77236654,697.28442034)
\curveto(182.70236246,697.27442049)(182.62736253,697.2594205)(182.54736654,697.23942034)
\curveto(182.50736265,697.22942053)(182.4623627,697.22942053)(182.41236654,697.23942034)
\curveto(182.37236279,697.23942052)(182.33236283,697.23442053)(182.29236654,697.22442034)
\curveto(182.2623629,697.21442055)(182.19736296,697.21442055)(182.09736654,697.22442034)
\curveto(182.00736315,697.22442054)(181.94736321,697.22942053)(181.91736654,697.23942034)
\curveto(181.86736329,697.23942052)(181.81736334,697.24442052)(181.76736654,697.25442034)
\lineto(181.61736654,697.25442034)
\curveto(181.49736366,697.28442048)(181.38236378,697.30942045)(181.27236654,697.32942034)
\curveto(181.162364,697.34942041)(181.05236411,697.37942038)(180.94236654,697.41942034)
\curveto(180.89236427,697.43942032)(180.84736431,697.45442031)(180.80736654,697.46442034)
\curveto(180.77736438,697.48442028)(180.73736442,697.50442026)(180.68736654,697.52442034)
\curveto(180.33736482,697.71442005)(180.0573651,697.97941978)(179.84736654,698.31942034)
\curveto(179.71736544,698.52941923)(179.62236554,698.77941898)(179.56236654,699.06942034)
\curveto(179.50236566,699.36941839)(179.4623657,699.68441808)(179.44236654,700.01442034)
\curveto(179.43236573,700.35441741)(179.42736573,700.69941706)(179.42736654,701.04942034)
\curveto(179.43736572,701.40941635)(179.44236572,701.764416)(179.44236654,702.11442034)
\lineto(179.44236654,704.15442034)
\curveto(179.44236572,704.28441348)(179.43736572,704.43441333)(179.42736654,704.60442034)
\curveto(179.42736573,704.78441298)(179.45236571,704.91441285)(179.50236654,704.99442034)
\curveto(179.53236563,705.04441272)(179.59236557,705.08941267)(179.68236654,705.12942034)
\curveto(179.74236542,705.12941263)(179.78736537,705.13441263)(179.81736654,705.14442034)
}
}
{
\newrgbcolor{curcolor}{0 0 0}
\pscustom[linestyle=none,fillstyle=solid,fillcolor=curcolor]
{
\newpath
\moveto(192.72861654,705.35442034)
\curveto(192.83861123,705.35441241)(192.93361113,705.34441242)(193.01361654,705.32442034)
\curveto(193.10361096,705.30441246)(193.17361089,705.2594125)(193.22361654,705.18942034)
\curveto(193.28361078,705.10941265)(193.31361075,704.96941279)(193.31361654,704.76942034)
\lineto(193.31361654,704.25942034)
\lineto(193.31361654,703.88442034)
\curveto(193.32361074,703.74441402)(193.30861076,703.63441413)(193.26861654,703.55442034)
\curveto(193.22861084,703.48441428)(193.1686109,703.43941432)(193.08861654,703.41942034)
\curveto(193.01861105,703.39941436)(192.93361113,703.38941437)(192.83361654,703.38942034)
\curveto(192.74361132,703.38941437)(192.64361142,703.39441437)(192.53361654,703.40442034)
\curveto(192.43361163,703.41441435)(192.33861173,703.40941435)(192.24861654,703.38942034)
\curveto(192.17861189,703.36941439)(192.10861196,703.35441441)(192.03861654,703.34442034)
\curveto(191.9686121,703.34441442)(191.90361216,703.33441443)(191.84361654,703.31442034)
\curveto(191.68361238,703.2644145)(191.52361254,703.18941457)(191.36361654,703.08942034)
\curveto(191.20361286,702.99941476)(191.07861299,702.89441487)(190.98861654,702.77442034)
\curveto(190.93861313,702.69441507)(190.88361318,702.60941515)(190.82361654,702.51942034)
\curveto(190.77361329,702.43941532)(190.72361334,702.35441541)(190.67361654,702.26442034)
\curveto(190.64361342,702.18441558)(190.61361345,702.09941566)(190.58361654,702.00942034)
\lineto(190.52361654,701.76942034)
\curveto(190.50361356,701.69941606)(190.49361357,701.62441614)(190.49361654,701.54442034)
\curveto(190.49361357,701.47441629)(190.48361358,701.40441636)(190.46361654,701.33442034)
\curveto(190.45361361,701.29441647)(190.44861362,701.25441651)(190.44861654,701.21442034)
\curveto(190.45861361,701.18441658)(190.45861361,701.15441661)(190.44861654,701.12442034)
\lineto(190.44861654,700.88442034)
\curveto(190.42861364,700.81441695)(190.42361364,700.73441703)(190.43361654,700.64442034)
\curveto(190.44361362,700.5644172)(190.44861362,700.48441728)(190.44861654,700.40442034)
\lineto(190.44861654,699.44442034)
\lineto(190.44861654,698.16942034)
\curveto(190.44861362,698.03941972)(190.44361362,697.91941984)(190.43361654,697.80942034)
\curveto(190.42361364,697.69942006)(190.39361367,697.60942015)(190.34361654,697.53942034)
\curveto(190.32361374,697.50942025)(190.28861378,697.48442028)(190.23861654,697.46442034)
\curveto(190.19861387,697.45442031)(190.15361391,697.44442032)(190.10361654,697.43442034)
\lineto(190.02861654,697.43442034)
\curveto(189.97861409,697.42442034)(189.92361414,697.41942034)(189.86361654,697.41942034)
\lineto(189.69861654,697.41942034)
\lineto(189.05361654,697.41942034)
\curveto(188.99361507,697.42942033)(188.92861514,697.43442033)(188.85861654,697.43442034)
\lineto(188.66361654,697.43442034)
\curveto(188.61361545,697.45442031)(188.5636155,697.46942029)(188.51361654,697.47942034)
\curveto(188.4636156,697.49942026)(188.42861564,697.53442023)(188.40861654,697.58442034)
\curveto(188.3686157,697.63442013)(188.34361572,697.70442006)(188.33361654,697.79442034)
\lineto(188.33361654,698.09442034)
\lineto(188.33361654,699.11442034)
\lineto(188.33361654,703.34442034)
\lineto(188.33361654,704.45442034)
\lineto(188.33361654,704.73942034)
\curveto(188.33361573,704.83941292)(188.35361571,704.91941284)(188.39361654,704.97942034)
\curveto(188.44361562,705.0594127)(188.51861555,705.10941265)(188.61861654,705.12942034)
\curveto(188.71861535,705.14941261)(188.83861523,705.1594126)(188.97861654,705.15942034)
\lineto(189.74361654,705.15942034)
\curveto(189.8636142,705.1594126)(189.9686141,705.14941261)(190.05861654,705.12942034)
\curveto(190.14861392,705.11941264)(190.21861385,705.07441269)(190.26861654,704.99442034)
\curveto(190.29861377,704.94441282)(190.31361375,704.87441289)(190.31361654,704.78442034)
\lineto(190.34361654,704.51442034)
\curveto(190.35361371,704.43441333)(190.3686137,704.3594134)(190.38861654,704.28942034)
\curveto(190.41861365,704.21941354)(190.4686136,704.18441358)(190.53861654,704.18442034)
\curveto(190.55861351,704.20441356)(190.57861349,704.21441355)(190.59861654,704.21442034)
\curveto(190.61861345,704.21441355)(190.63861343,704.22441354)(190.65861654,704.24442034)
\curveto(190.71861335,704.29441347)(190.7686133,704.34941341)(190.80861654,704.40942034)
\curveto(190.85861321,704.47941328)(190.91861315,704.53941322)(190.98861654,704.58942034)
\curveto(191.02861304,704.61941314)(191.063613,704.64941311)(191.09361654,704.67942034)
\curveto(191.12361294,704.71941304)(191.15861291,704.75441301)(191.19861654,704.78442034)
\lineto(191.46861654,704.96442034)
\curveto(191.5686125,705.02441274)(191.6686124,705.07941268)(191.76861654,705.12942034)
\curveto(191.8686122,705.16941259)(191.9686121,705.20441256)(192.06861654,705.23442034)
\lineto(192.39861654,705.32442034)
\curveto(192.42861164,705.33441243)(192.48361158,705.33441243)(192.56361654,705.32442034)
\curveto(192.65361141,705.32441244)(192.70861136,705.33441243)(192.72861654,705.35442034)
}
}
{
\newrgbcolor{curcolor}{0 0 0}
\pscustom[linestyle=none,fillstyle=solid,fillcolor=curcolor]
{
\newpath
\moveto(197.10369467,705.36942034)
\curveto(197.85369017,705.38941237)(198.50368952,705.30441246)(199.05369467,705.11442034)
\curveto(199.61368841,704.93441283)(200.03868798,704.61941314)(200.32869467,704.16942034)
\curveto(200.39868762,704.0594137)(200.45868756,703.94441382)(200.50869467,703.82442034)
\curveto(200.56868745,703.71441405)(200.6186874,703.58941417)(200.65869467,703.44942034)
\curveto(200.67868734,703.38941437)(200.68868733,703.32441444)(200.68869467,703.25442034)
\curveto(200.68868733,703.18441458)(200.67868734,703.12441464)(200.65869467,703.07442034)
\curveto(200.6186874,703.01441475)(200.56368746,702.97441479)(200.49369467,702.95442034)
\curveto(200.44368758,702.93441483)(200.38368764,702.92441484)(200.31369467,702.92442034)
\lineto(200.10369467,702.92442034)
\lineto(199.44369467,702.92442034)
\curveto(199.37368865,702.92441484)(199.30368872,702.91941484)(199.23369467,702.90942034)
\curveto(199.16368886,702.90941485)(199.09868892,702.91941484)(199.03869467,702.93942034)
\curveto(198.93868908,702.9594148)(198.86368916,702.99941476)(198.81369467,703.05942034)
\curveto(198.76368926,703.11941464)(198.7186893,703.17941458)(198.67869467,703.23942034)
\lineto(198.55869467,703.44942034)
\curveto(198.52868949,703.52941423)(198.47868954,703.59441417)(198.40869467,703.64442034)
\curveto(198.30868971,703.72441404)(198.20868981,703.78441398)(198.10869467,703.82442034)
\curveto(198.01869,703.8644139)(197.90369012,703.89941386)(197.76369467,703.92942034)
\curveto(197.69369033,703.94941381)(197.58869043,703.9644138)(197.44869467,703.97442034)
\curveto(197.3186907,703.98441378)(197.2186908,703.97941378)(197.14869467,703.95942034)
\lineto(197.04369467,703.95942034)
\lineto(196.89369467,703.92942034)
\curveto(196.85369117,703.92941383)(196.80869121,703.92441384)(196.75869467,703.91442034)
\curveto(196.58869143,703.8644139)(196.44869157,703.79441397)(196.33869467,703.70442034)
\curveto(196.23869178,703.62441414)(196.16869185,703.49941426)(196.12869467,703.32942034)
\curveto(196.10869191,703.2594145)(196.10869191,703.19441457)(196.12869467,703.13442034)
\curveto(196.14869187,703.07441469)(196.16869185,703.02441474)(196.18869467,702.98442034)
\curveto(196.25869176,702.8644149)(196.33869168,702.76941499)(196.42869467,702.69942034)
\curveto(196.52869149,702.62941513)(196.64369138,702.56941519)(196.77369467,702.51942034)
\curveto(196.96369106,702.43941532)(197.16869085,702.36941539)(197.38869467,702.30942034)
\lineto(198.07869467,702.15942034)
\curveto(198.3186897,702.11941564)(198.54868947,702.06941569)(198.76869467,702.00942034)
\curveto(198.99868902,701.9594158)(199.21368881,701.89441587)(199.41369467,701.81442034)
\curveto(199.50368852,701.77441599)(199.58868843,701.73941602)(199.66869467,701.70942034)
\curveto(199.75868826,701.68941607)(199.84368818,701.65441611)(199.92369467,701.60442034)
\curveto(200.11368791,701.48441628)(200.28368774,701.35441641)(200.43369467,701.21442034)
\curveto(200.59368743,701.07441669)(200.7186873,700.89941686)(200.80869467,700.68942034)
\curveto(200.83868718,700.61941714)(200.86368716,700.54941721)(200.88369467,700.47942034)
\curveto(200.90368712,700.40941735)(200.9236871,700.33441743)(200.94369467,700.25442034)
\curveto(200.95368707,700.19441757)(200.95868706,700.09941766)(200.95869467,699.96942034)
\curveto(200.96868705,699.84941791)(200.96868705,699.75441801)(200.95869467,699.68442034)
\lineto(200.95869467,699.60942034)
\curveto(200.93868708,699.54941821)(200.9236871,699.48941827)(200.91369467,699.42942034)
\curveto(200.91368711,699.37941838)(200.90868711,699.32941843)(200.89869467,699.27942034)
\curveto(200.82868719,698.97941878)(200.7186873,698.71441905)(200.56869467,698.48442034)
\curveto(200.40868761,698.24441952)(200.21368781,698.04941971)(199.98369467,697.89942034)
\curveto(199.75368827,697.74942001)(199.49368853,697.61942014)(199.20369467,697.50942034)
\curveto(199.09368893,697.4594203)(198.97368905,697.42442034)(198.84369467,697.40442034)
\curveto(198.7236893,697.38442038)(198.60368942,697.3594204)(198.48369467,697.32942034)
\curveto(198.39368963,697.30942045)(198.29868972,697.29942046)(198.19869467,697.29942034)
\curveto(198.10868991,697.28942047)(198.01869,697.27442049)(197.92869467,697.25442034)
\lineto(197.65869467,697.25442034)
\curveto(197.59869042,697.23442053)(197.49369053,697.22442054)(197.34369467,697.22442034)
\curveto(197.20369082,697.22442054)(197.10369092,697.23442053)(197.04369467,697.25442034)
\curveto(197.01369101,697.25442051)(196.97869104,697.2594205)(196.93869467,697.26942034)
\lineto(196.83369467,697.26942034)
\curveto(196.71369131,697.28942047)(196.59369143,697.30442046)(196.47369467,697.31442034)
\curveto(196.35369167,697.32442044)(196.23869178,697.34442042)(196.12869467,697.37442034)
\curveto(195.73869228,697.48442028)(195.39369263,697.60942015)(195.09369467,697.74942034)
\curveto(194.79369323,697.89941986)(194.53869348,698.11941964)(194.32869467,698.40942034)
\curveto(194.18869383,698.59941916)(194.06869395,698.81941894)(193.96869467,699.06942034)
\curveto(193.94869407,699.12941863)(193.92869409,699.20941855)(193.90869467,699.30942034)
\curveto(193.88869413,699.3594184)(193.87369415,699.42941833)(193.86369467,699.51942034)
\curveto(193.85369417,699.60941815)(193.85869416,699.68441808)(193.87869467,699.74442034)
\curveto(193.90869411,699.81441795)(193.95869406,699.8644179)(194.02869467,699.89442034)
\curveto(194.07869394,699.91441785)(194.13869388,699.92441784)(194.20869467,699.92442034)
\lineto(194.43369467,699.92442034)
\lineto(195.13869467,699.92442034)
\lineto(195.37869467,699.92442034)
\curveto(195.45869256,699.92441784)(195.52869249,699.91441785)(195.58869467,699.89442034)
\curveto(195.69869232,699.85441791)(195.76869225,699.78941797)(195.79869467,699.69942034)
\curveto(195.83869218,699.60941815)(195.88369214,699.51441825)(195.93369467,699.41442034)
\curveto(195.95369207,699.3644184)(195.98869203,699.29941846)(196.03869467,699.21942034)
\curveto(196.09869192,699.13941862)(196.14869187,699.08941867)(196.18869467,699.06942034)
\curveto(196.30869171,698.96941879)(196.4236916,698.88941887)(196.53369467,698.82942034)
\curveto(196.64369138,698.77941898)(196.78369124,698.72941903)(196.95369467,698.67942034)
\curveto(197.00369102,698.6594191)(197.05369097,698.64941911)(197.10369467,698.64942034)
\curveto(197.15369087,698.6594191)(197.20369082,698.6594191)(197.25369467,698.64942034)
\curveto(197.33369069,698.62941913)(197.4186906,698.61941914)(197.50869467,698.61942034)
\curveto(197.60869041,698.62941913)(197.69369033,698.64441912)(197.76369467,698.66442034)
\curveto(197.81369021,698.67441909)(197.85869016,698.67941908)(197.89869467,698.67942034)
\curveto(197.94869007,698.67941908)(197.99869002,698.68941907)(198.04869467,698.70942034)
\curveto(198.18868983,698.759419)(198.31368971,698.81941894)(198.42369467,698.88942034)
\curveto(198.54368948,698.9594188)(198.63868938,699.04941871)(198.70869467,699.15942034)
\curveto(198.75868926,699.23941852)(198.79868922,699.3644184)(198.82869467,699.53442034)
\curveto(198.84868917,699.60441816)(198.84868917,699.66941809)(198.82869467,699.72942034)
\curveto(198.80868921,699.78941797)(198.78868923,699.83941792)(198.76869467,699.87942034)
\curveto(198.69868932,700.01941774)(198.60868941,700.12441764)(198.49869467,700.19442034)
\curveto(198.39868962,700.2644175)(198.27868974,700.32941743)(198.13869467,700.38942034)
\curveto(197.94869007,700.46941729)(197.74869027,700.53441723)(197.53869467,700.58442034)
\curveto(197.32869069,700.63441713)(197.1186909,700.68941707)(196.90869467,700.74942034)
\curveto(196.82869119,700.76941699)(196.74369128,700.78441698)(196.65369467,700.79442034)
\curveto(196.57369145,700.80441696)(196.49369153,700.81941694)(196.41369467,700.83942034)
\curveto(196.09369193,700.92941683)(195.78869223,701.01441675)(195.49869467,701.09442034)
\curveto(195.20869281,701.18441658)(194.94369308,701.31441645)(194.70369467,701.48442034)
\curveto(194.4236936,701.68441608)(194.2186938,701.95441581)(194.08869467,702.29442034)
\curveto(194.06869395,702.3644154)(194.04869397,702.4594153)(194.02869467,702.57942034)
\curveto(194.00869401,702.64941511)(193.99369403,702.73441503)(193.98369467,702.83442034)
\curveto(193.97369405,702.93441483)(193.97869404,703.02441474)(193.99869467,703.10442034)
\curveto(194.018694,703.15441461)(194.023694,703.19441457)(194.01369467,703.22442034)
\curveto(194.00369402,703.2644145)(194.00869401,703.30941445)(194.02869467,703.35942034)
\curveto(194.04869397,703.46941429)(194.06869395,703.56941419)(194.08869467,703.65942034)
\curveto(194.1186939,703.759414)(194.15369387,703.85441391)(194.19369467,703.94442034)
\curveto(194.3236937,704.23441353)(194.50369352,704.46941329)(194.73369467,704.64942034)
\curveto(194.96369306,704.82941293)(195.2236928,704.97441279)(195.51369467,705.08442034)
\curveto(195.6236924,705.13441263)(195.73869228,705.16941259)(195.85869467,705.18942034)
\curveto(195.97869204,705.21941254)(196.10369192,705.24941251)(196.23369467,705.27942034)
\curveto(196.29369173,705.29941246)(196.35369167,705.30941245)(196.41369467,705.30942034)
\lineto(196.59369467,705.33942034)
\curveto(196.67369135,705.34941241)(196.75869126,705.35441241)(196.84869467,705.35442034)
\curveto(196.93869108,705.35441241)(197.023691,705.3594124)(197.10369467,705.36942034)
}
}
{
\newrgbcolor{curcolor}{0 0 0}
\pscustom[linestyle=none,fillstyle=solid,fillcolor=curcolor]
{
\newpath
\moveto(209.96033529,701.60442034)
\curveto(209.98032672,701.54441622)(209.99032671,701.4594163)(209.99033529,701.34942034)
\curveto(209.99032671,701.23941652)(209.98032672,701.15441661)(209.96033529,701.09442034)
\lineto(209.96033529,700.94442034)
\curveto(209.94032676,700.8644169)(209.93032677,700.78441698)(209.93033529,700.70442034)
\curveto(209.94032676,700.62441714)(209.93532677,700.54441722)(209.91533529,700.46442034)
\curveto(209.89532681,700.39441737)(209.88032682,700.32941743)(209.87033529,700.26942034)
\curveto(209.86032684,700.20941755)(209.85032685,700.14441762)(209.84033529,700.07442034)
\curveto(209.8003269,699.9644178)(209.76532694,699.84941791)(209.73533529,699.72942034)
\curveto(209.705327,699.61941814)(209.66532704,699.51441825)(209.61533529,699.41442034)
\curveto(209.4053273,698.93441883)(209.13032757,698.54441922)(208.79033529,698.24442034)
\curveto(208.45032825,697.94441982)(208.04032866,697.69442007)(207.56033529,697.49442034)
\curveto(207.44032926,697.44442032)(207.31532939,697.40942035)(207.18533529,697.38942034)
\curveto(207.06532964,697.3594204)(206.94032976,697.32942043)(206.81033529,697.29942034)
\curveto(206.76032994,697.27942048)(206.70533,697.26942049)(206.64533529,697.26942034)
\curveto(206.58533012,697.26942049)(206.53033017,697.2644205)(206.48033529,697.25442034)
\lineto(206.37533529,697.25442034)
\curveto(206.34533036,697.24442052)(206.31533039,697.23942052)(206.28533529,697.23942034)
\curveto(206.23533047,697.22942053)(206.15533055,697.22442054)(206.04533529,697.22442034)
\curveto(205.93533077,697.21442055)(205.85033085,697.21942054)(205.79033529,697.23942034)
\lineto(205.64033529,697.23942034)
\curveto(205.59033111,697.24942051)(205.53533117,697.25442051)(205.47533529,697.25442034)
\curveto(205.42533128,697.24442052)(205.37533133,697.24942051)(205.32533529,697.26942034)
\curveto(205.28533142,697.27942048)(205.24533146,697.28442048)(205.20533529,697.28442034)
\curveto(205.17533153,697.28442048)(205.13533157,697.28942047)(205.08533529,697.29942034)
\curveto(204.98533172,697.32942043)(204.88533182,697.35442041)(204.78533529,697.37442034)
\curveto(204.68533202,697.39442037)(204.59033211,697.42442034)(204.50033529,697.46442034)
\curveto(204.38033232,697.50442026)(204.26533244,697.54442022)(204.15533529,697.58442034)
\curveto(204.05533265,697.62442014)(203.95033275,697.67442009)(203.84033529,697.73442034)
\curveto(203.49033321,697.94441982)(203.19033351,698.18941957)(202.94033529,698.46942034)
\curveto(202.69033401,698.74941901)(202.48033422,699.08441868)(202.31033529,699.47442034)
\curveto(202.26033444,699.5644182)(202.22033448,699.6594181)(202.19033529,699.75942034)
\curveto(202.17033453,699.8594179)(202.14533456,699.9644178)(202.11533529,700.07442034)
\curveto(202.09533461,700.12441764)(202.08533462,700.16941759)(202.08533529,700.20942034)
\curveto(202.08533462,700.24941751)(202.07533463,700.29441747)(202.05533529,700.34442034)
\curveto(202.03533467,700.42441734)(202.02533468,700.50441726)(202.02533529,700.58442034)
\curveto(202.02533468,700.67441709)(202.01533469,700.759417)(201.99533529,700.83942034)
\curveto(201.98533472,700.88941687)(201.98033472,700.93441683)(201.98033529,700.97442034)
\lineto(201.98033529,701.10942034)
\curveto(201.96033474,701.16941659)(201.95033475,701.25441651)(201.95033529,701.36442034)
\curveto(201.96033474,701.47441629)(201.97533473,701.5594162)(201.99533529,701.61942034)
\lineto(201.99533529,701.72442034)
\curveto(202.0053347,701.77441599)(202.0053347,701.82441594)(201.99533529,701.87442034)
\curveto(201.99533471,701.93441583)(202.0053347,701.98941577)(202.02533529,702.03942034)
\curveto(202.03533467,702.08941567)(202.04033466,702.13441563)(202.04033529,702.17442034)
\curveto(202.04033466,702.22441554)(202.05033465,702.27441549)(202.07033529,702.32442034)
\curveto(202.11033459,702.45441531)(202.14533456,702.57941518)(202.17533529,702.69942034)
\curveto(202.2053345,702.82941493)(202.24533446,702.95441481)(202.29533529,703.07442034)
\curveto(202.47533423,703.48441428)(202.69033401,703.82441394)(202.94033529,704.09442034)
\curveto(203.19033351,704.37441339)(203.49533321,704.62941313)(203.85533529,704.85942034)
\curveto(203.95533275,704.90941285)(204.06033264,704.95441281)(204.17033529,704.99442034)
\curveto(204.28033242,705.03441273)(204.39033231,705.07941268)(204.50033529,705.12942034)
\curveto(204.63033207,705.17941258)(204.76533194,705.21441255)(204.90533529,705.23442034)
\curveto(205.04533166,705.25441251)(205.19033151,705.28441248)(205.34033529,705.32442034)
\curveto(205.42033128,705.33441243)(205.49533121,705.33941242)(205.56533529,705.33942034)
\curveto(205.63533107,705.33941242)(205.705331,705.34441242)(205.77533529,705.35442034)
\curveto(206.35533035,705.3644124)(206.85532985,705.30441246)(207.27533529,705.17442034)
\curveto(207.705329,705.04441272)(208.08532862,704.8644129)(208.41533529,704.63442034)
\curveto(208.52532818,704.55441321)(208.63532807,704.4644133)(208.74533529,704.36442034)
\curveto(208.86532784,704.27441349)(208.96532774,704.17441359)(209.04533529,704.06442034)
\curveto(209.12532758,703.9644138)(209.19532751,703.8644139)(209.25533529,703.76442034)
\curveto(209.32532738,703.6644141)(209.39532731,703.5594142)(209.46533529,703.44942034)
\curveto(209.53532717,703.33941442)(209.59032711,703.21941454)(209.63033529,703.08942034)
\curveto(209.67032703,702.96941479)(209.71532699,702.83941492)(209.76533529,702.69942034)
\curveto(209.79532691,702.61941514)(209.82032688,702.53441523)(209.84033529,702.44442034)
\lineto(209.90033529,702.17442034)
\curveto(209.91032679,702.13441563)(209.91532679,702.09441567)(209.91533529,702.05442034)
\curveto(209.91532679,702.01441575)(209.92032678,701.97441579)(209.93033529,701.93442034)
\curveto(209.95032675,701.88441588)(209.95532675,701.82941593)(209.94533529,701.76942034)
\curveto(209.93532677,701.70941605)(209.94032676,701.65441611)(209.96033529,701.60442034)
\moveto(207.86033529,701.06442034)
\curveto(207.87032883,701.11441665)(207.87532883,701.18441658)(207.87533529,701.27442034)
\curveto(207.87532883,701.37441639)(207.87032883,701.44941631)(207.86033529,701.49942034)
\lineto(207.86033529,701.61942034)
\curveto(207.84032886,701.66941609)(207.83032887,701.72441604)(207.83033529,701.78442034)
\curveto(207.83032887,701.84441592)(207.82532888,701.89941586)(207.81533529,701.94942034)
\curveto(207.81532889,701.98941577)(207.81032889,702.01941574)(207.80033529,702.03942034)
\lineto(207.74033529,702.27942034)
\curveto(207.73032897,702.36941539)(207.71032899,702.45441531)(207.68033529,702.53442034)
\curveto(207.57032913,702.79441497)(207.44032926,703.01441475)(207.29033529,703.19442034)
\curveto(207.14032956,703.38441438)(206.94032976,703.53441423)(206.69033529,703.64442034)
\curveto(206.63033007,703.6644141)(206.57033013,703.67941408)(206.51033529,703.68942034)
\curveto(206.45033025,703.70941405)(206.38533032,703.72941403)(206.31533529,703.74942034)
\curveto(206.23533047,703.76941399)(206.15033055,703.77441399)(206.06033529,703.76442034)
\lineto(205.79033529,703.76442034)
\curveto(205.76033094,703.74441402)(205.72533098,703.73441403)(205.68533529,703.73442034)
\curveto(205.64533106,703.74441402)(205.61033109,703.74441402)(205.58033529,703.73442034)
\lineto(205.37033529,703.67442034)
\curveto(205.31033139,703.6644141)(205.25533145,703.64441412)(205.20533529,703.61442034)
\curveto(204.95533175,703.50441426)(204.75033195,703.34441442)(204.59033529,703.13442034)
\curveto(204.44033226,702.93441483)(204.32033238,702.69941506)(204.23033529,702.42942034)
\curveto(204.2003325,702.32941543)(204.17533253,702.22441554)(204.15533529,702.11442034)
\curveto(204.14533256,702.00441576)(204.13033257,701.89441587)(204.11033529,701.78442034)
\curveto(204.1003326,701.73441603)(204.09533261,701.68441608)(204.09533529,701.63442034)
\lineto(204.09533529,701.48442034)
\curveto(204.07533263,701.41441635)(204.06533264,701.30941645)(204.06533529,701.16942034)
\curveto(204.07533263,701.02941673)(204.09033261,700.92441684)(204.11033529,700.85442034)
\lineto(204.11033529,700.71942034)
\curveto(204.13033257,700.63941712)(204.14533256,700.5594172)(204.15533529,700.47942034)
\curveto(204.16533254,700.40941735)(204.18033252,700.33441743)(204.20033529,700.25442034)
\curveto(204.3003324,699.95441781)(204.4053323,699.70941805)(204.51533529,699.51942034)
\curveto(204.63533207,699.33941842)(204.82033188,699.17441859)(205.07033529,699.02442034)
\curveto(205.14033156,698.97441879)(205.21533149,698.93441883)(205.29533529,698.90442034)
\curveto(205.38533132,698.87441889)(205.47533123,698.84941891)(205.56533529,698.82942034)
\curveto(205.6053311,698.81941894)(205.64033106,698.81441895)(205.67033529,698.81442034)
\curveto(205.700331,698.82441894)(205.73533097,698.82441894)(205.77533529,698.81442034)
\lineto(205.89533529,698.78442034)
\curveto(205.94533076,698.78441898)(205.99033071,698.78941897)(206.03033529,698.79942034)
\lineto(206.15033529,698.79942034)
\curveto(206.23033047,698.81941894)(206.31033039,698.83441893)(206.39033529,698.84442034)
\curveto(206.47033023,698.85441891)(206.54533016,698.87441889)(206.61533529,698.90442034)
\curveto(206.87532983,699.00441876)(207.08532962,699.13941862)(207.24533529,699.30942034)
\curveto(207.4053293,699.47941828)(207.54032916,699.68941807)(207.65033529,699.93942034)
\curveto(207.69032901,700.03941772)(207.72032898,700.13941762)(207.74033529,700.23942034)
\curveto(207.76032894,700.33941742)(207.78532892,700.44441732)(207.81533529,700.55442034)
\curveto(207.82532888,700.59441717)(207.83032887,700.62941713)(207.83033529,700.65942034)
\curveto(207.83032887,700.69941706)(207.83532887,700.73941702)(207.84533529,700.77942034)
\lineto(207.84533529,700.91442034)
\curveto(207.84532886,700.9644168)(207.85032885,701.01441675)(207.86033529,701.06442034)
}
}
{
\newrgbcolor{curcolor}{0 0 0}
\pscustom[linestyle=none,fillstyle=solid,fillcolor=curcolor]
{
\newpath
\moveto(214.33025717,705.36942034)
\curveto(215.08025267,705.38941237)(215.73025202,705.30441246)(216.28025717,705.11442034)
\curveto(216.84025091,704.93441283)(217.26525048,704.61941314)(217.55525717,704.16942034)
\curveto(217.62525012,704.0594137)(217.68525006,703.94441382)(217.73525717,703.82442034)
\curveto(217.79524995,703.71441405)(217.8452499,703.58941417)(217.88525717,703.44942034)
\curveto(217.90524984,703.38941437)(217.91524983,703.32441444)(217.91525717,703.25442034)
\curveto(217.91524983,703.18441458)(217.90524984,703.12441464)(217.88525717,703.07442034)
\curveto(217.8452499,703.01441475)(217.79024996,702.97441479)(217.72025717,702.95442034)
\curveto(217.67025008,702.93441483)(217.61025014,702.92441484)(217.54025717,702.92442034)
\lineto(217.33025717,702.92442034)
\lineto(216.67025717,702.92442034)
\curveto(216.60025115,702.92441484)(216.53025122,702.91941484)(216.46025717,702.90942034)
\curveto(216.39025136,702.90941485)(216.32525142,702.91941484)(216.26525717,702.93942034)
\curveto(216.16525158,702.9594148)(216.09025166,702.99941476)(216.04025717,703.05942034)
\curveto(215.99025176,703.11941464)(215.9452518,703.17941458)(215.90525717,703.23942034)
\lineto(215.78525717,703.44942034)
\curveto(215.75525199,703.52941423)(215.70525204,703.59441417)(215.63525717,703.64442034)
\curveto(215.53525221,703.72441404)(215.43525231,703.78441398)(215.33525717,703.82442034)
\curveto(215.2452525,703.8644139)(215.13025262,703.89941386)(214.99025717,703.92942034)
\curveto(214.92025283,703.94941381)(214.81525293,703.9644138)(214.67525717,703.97442034)
\curveto(214.5452532,703.98441378)(214.4452533,703.97941378)(214.37525717,703.95942034)
\lineto(214.27025717,703.95942034)
\lineto(214.12025717,703.92942034)
\curveto(214.08025367,703.92941383)(214.03525371,703.92441384)(213.98525717,703.91442034)
\curveto(213.81525393,703.8644139)(213.67525407,703.79441397)(213.56525717,703.70442034)
\curveto(213.46525428,703.62441414)(213.39525435,703.49941426)(213.35525717,703.32942034)
\curveto(213.33525441,703.2594145)(213.33525441,703.19441457)(213.35525717,703.13442034)
\curveto(213.37525437,703.07441469)(213.39525435,703.02441474)(213.41525717,702.98442034)
\curveto(213.48525426,702.8644149)(213.56525418,702.76941499)(213.65525717,702.69942034)
\curveto(213.75525399,702.62941513)(213.87025388,702.56941519)(214.00025717,702.51942034)
\curveto(214.19025356,702.43941532)(214.39525335,702.36941539)(214.61525717,702.30942034)
\lineto(215.30525717,702.15942034)
\curveto(215.5452522,702.11941564)(215.77525197,702.06941569)(215.99525717,702.00942034)
\curveto(216.22525152,701.9594158)(216.44025131,701.89441587)(216.64025717,701.81442034)
\curveto(216.73025102,701.77441599)(216.81525093,701.73941602)(216.89525717,701.70942034)
\curveto(216.98525076,701.68941607)(217.07025068,701.65441611)(217.15025717,701.60442034)
\curveto(217.34025041,701.48441628)(217.51025024,701.35441641)(217.66025717,701.21442034)
\curveto(217.82024993,701.07441669)(217.9452498,700.89941686)(218.03525717,700.68942034)
\curveto(218.06524968,700.61941714)(218.09024966,700.54941721)(218.11025717,700.47942034)
\curveto(218.13024962,700.40941735)(218.1502496,700.33441743)(218.17025717,700.25442034)
\curveto(218.18024957,700.19441757)(218.18524956,700.09941766)(218.18525717,699.96942034)
\curveto(218.19524955,699.84941791)(218.19524955,699.75441801)(218.18525717,699.68442034)
\lineto(218.18525717,699.60942034)
\curveto(218.16524958,699.54941821)(218.1502496,699.48941827)(218.14025717,699.42942034)
\curveto(218.14024961,699.37941838)(218.13524961,699.32941843)(218.12525717,699.27942034)
\curveto(218.05524969,698.97941878)(217.9452498,698.71441905)(217.79525717,698.48442034)
\curveto(217.63525011,698.24441952)(217.44025031,698.04941971)(217.21025717,697.89942034)
\curveto(216.98025077,697.74942001)(216.72025103,697.61942014)(216.43025717,697.50942034)
\curveto(216.32025143,697.4594203)(216.20025155,697.42442034)(216.07025717,697.40442034)
\curveto(215.9502518,697.38442038)(215.83025192,697.3594204)(215.71025717,697.32942034)
\curveto(215.62025213,697.30942045)(215.52525222,697.29942046)(215.42525717,697.29942034)
\curveto(215.33525241,697.28942047)(215.2452525,697.27442049)(215.15525717,697.25442034)
\lineto(214.88525717,697.25442034)
\curveto(214.82525292,697.23442053)(214.72025303,697.22442054)(214.57025717,697.22442034)
\curveto(214.43025332,697.22442054)(214.33025342,697.23442053)(214.27025717,697.25442034)
\curveto(214.24025351,697.25442051)(214.20525354,697.2594205)(214.16525717,697.26942034)
\lineto(214.06025717,697.26942034)
\curveto(213.94025381,697.28942047)(213.82025393,697.30442046)(213.70025717,697.31442034)
\curveto(213.58025417,697.32442044)(213.46525428,697.34442042)(213.35525717,697.37442034)
\curveto(212.96525478,697.48442028)(212.62025513,697.60942015)(212.32025717,697.74942034)
\curveto(212.02025573,697.89941986)(211.76525598,698.11941964)(211.55525717,698.40942034)
\curveto(211.41525633,698.59941916)(211.29525645,698.81941894)(211.19525717,699.06942034)
\curveto(211.17525657,699.12941863)(211.15525659,699.20941855)(211.13525717,699.30942034)
\curveto(211.11525663,699.3594184)(211.10025665,699.42941833)(211.09025717,699.51942034)
\curveto(211.08025667,699.60941815)(211.08525666,699.68441808)(211.10525717,699.74442034)
\curveto(211.13525661,699.81441795)(211.18525656,699.8644179)(211.25525717,699.89442034)
\curveto(211.30525644,699.91441785)(211.36525638,699.92441784)(211.43525717,699.92442034)
\lineto(211.66025717,699.92442034)
\lineto(212.36525717,699.92442034)
\lineto(212.60525717,699.92442034)
\curveto(212.68525506,699.92441784)(212.75525499,699.91441785)(212.81525717,699.89442034)
\curveto(212.92525482,699.85441791)(212.99525475,699.78941797)(213.02525717,699.69942034)
\curveto(213.06525468,699.60941815)(213.11025464,699.51441825)(213.16025717,699.41442034)
\curveto(213.18025457,699.3644184)(213.21525453,699.29941846)(213.26525717,699.21942034)
\curveto(213.32525442,699.13941862)(213.37525437,699.08941867)(213.41525717,699.06942034)
\curveto(213.53525421,698.96941879)(213.6502541,698.88941887)(213.76025717,698.82942034)
\curveto(213.87025388,698.77941898)(214.01025374,698.72941903)(214.18025717,698.67942034)
\curveto(214.23025352,698.6594191)(214.28025347,698.64941911)(214.33025717,698.64942034)
\curveto(214.38025337,698.6594191)(214.43025332,698.6594191)(214.48025717,698.64942034)
\curveto(214.56025319,698.62941913)(214.6452531,698.61941914)(214.73525717,698.61942034)
\curveto(214.83525291,698.62941913)(214.92025283,698.64441912)(214.99025717,698.66442034)
\curveto(215.04025271,698.67441909)(215.08525266,698.67941908)(215.12525717,698.67942034)
\curveto(215.17525257,698.67941908)(215.22525252,698.68941907)(215.27525717,698.70942034)
\curveto(215.41525233,698.759419)(215.54025221,698.81941894)(215.65025717,698.88942034)
\curveto(215.77025198,698.9594188)(215.86525188,699.04941871)(215.93525717,699.15942034)
\curveto(215.98525176,699.23941852)(216.02525172,699.3644184)(216.05525717,699.53442034)
\curveto(216.07525167,699.60441816)(216.07525167,699.66941809)(216.05525717,699.72942034)
\curveto(216.03525171,699.78941797)(216.01525173,699.83941792)(215.99525717,699.87942034)
\curveto(215.92525182,700.01941774)(215.83525191,700.12441764)(215.72525717,700.19442034)
\curveto(215.62525212,700.2644175)(215.50525224,700.32941743)(215.36525717,700.38942034)
\curveto(215.17525257,700.46941729)(214.97525277,700.53441723)(214.76525717,700.58442034)
\curveto(214.55525319,700.63441713)(214.3452534,700.68941707)(214.13525717,700.74942034)
\curveto(214.05525369,700.76941699)(213.97025378,700.78441698)(213.88025717,700.79442034)
\curveto(213.80025395,700.80441696)(213.72025403,700.81941694)(213.64025717,700.83942034)
\curveto(213.32025443,700.92941683)(213.01525473,701.01441675)(212.72525717,701.09442034)
\curveto(212.43525531,701.18441658)(212.17025558,701.31441645)(211.93025717,701.48442034)
\curveto(211.6502561,701.68441608)(211.4452563,701.95441581)(211.31525717,702.29442034)
\curveto(211.29525645,702.3644154)(211.27525647,702.4594153)(211.25525717,702.57942034)
\curveto(211.23525651,702.64941511)(211.22025653,702.73441503)(211.21025717,702.83442034)
\curveto(211.20025655,702.93441483)(211.20525654,703.02441474)(211.22525717,703.10442034)
\curveto(211.2452565,703.15441461)(211.2502565,703.19441457)(211.24025717,703.22442034)
\curveto(211.23025652,703.2644145)(211.23525651,703.30941445)(211.25525717,703.35942034)
\curveto(211.27525647,703.46941429)(211.29525645,703.56941419)(211.31525717,703.65942034)
\curveto(211.3452564,703.759414)(211.38025637,703.85441391)(211.42025717,703.94442034)
\curveto(211.5502562,704.23441353)(211.73025602,704.46941329)(211.96025717,704.64942034)
\curveto(212.19025556,704.82941293)(212.4502553,704.97441279)(212.74025717,705.08442034)
\curveto(212.8502549,705.13441263)(212.96525478,705.16941259)(213.08525717,705.18942034)
\curveto(213.20525454,705.21941254)(213.33025442,705.24941251)(213.46025717,705.27942034)
\curveto(213.52025423,705.29941246)(213.58025417,705.30941245)(213.64025717,705.30942034)
\lineto(213.82025717,705.33942034)
\curveto(213.90025385,705.34941241)(213.98525376,705.35441241)(214.07525717,705.35442034)
\curveto(214.16525358,705.35441241)(214.2502535,705.3594124)(214.33025717,705.36942034)
}
}
{
\newrgbcolor{curcolor}{0 0 0}
\pscustom[linestyle=none,fillstyle=solid,fillcolor=curcolor]
{
\newpath
\moveto(21.58611654,686.61942034)
\curveto(22.33611204,686.63941237)(22.98611139,686.55441246)(23.53611654,686.36442034)
\curveto(24.09611028,686.18441283)(24.52110986,685.86941314)(24.81111654,685.41942034)
\curveto(24.8811095,685.3094137)(24.94110944,685.19441382)(24.99111654,685.07442034)
\curveto(25.05110933,684.96441405)(25.10110928,684.83941417)(25.14111654,684.69942034)
\curveto(25.16110922,684.63941437)(25.17110921,684.57441444)(25.17111654,684.50442034)
\curveto(25.17110921,684.43441458)(25.16110922,684.37441464)(25.14111654,684.32442034)
\curveto(25.10110928,684.26441475)(25.04610933,684.22441479)(24.97611654,684.20442034)
\curveto(24.92610945,684.18441483)(24.86610951,684.17441484)(24.79611654,684.17442034)
\lineto(24.58611654,684.17442034)
\lineto(23.92611654,684.17442034)
\curveto(23.85611052,684.17441484)(23.78611059,684.16941484)(23.71611654,684.15942034)
\curveto(23.64611073,684.15941485)(23.5811108,684.16941484)(23.52111654,684.18942034)
\curveto(23.42111096,684.2094148)(23.34611103,684.24941476)(23.29611654,684.30942034)
\curveto(23.24611113,684.36941464)(23.20111118,684.42941458)(23.16111654,684.48942034)
\lineto(23.04111654,684.69942034)
\curveto(23.01111137,684.77941423)(22.96111142,684.84441417)(22.89111654,684.89442034)
\curveto(22.79111159,684.97441404)(22.69111169,685.03441398)(22.59111654,685.07442034)
\curveto(22.50111188,685.1144139)(22.38611199,685.14941386)(22.24611654,685.17942034)
\curveto(22.1761122,685.19941381)(22.07111231,685.2144138)(21.93111654,685.22442034)
\curveto(21.80111258,685.23441378)(21.70111268,685.22941378)(21.63111654,685.20942034)
\lineto(21.52611654,685.20942034)
\lineto(21.37611654,685.17942034)
\curveto(21.33611304,685.17941383)(21.29111309,685.17441384)(21.24111654,685.16442034)
\curveto(21.07111331,685.1144139)(20.93111345,685.04441397)(20.82111654,684.95442034)
\curveto(20.72111366,684.87441414)(20.65111373,684.74941426)(20.61111654,684.57942034)
\curveto(20.59111379,684.5094145)(20.59111379,684.44441457)(20.61111654,684.38442034)
\curveto(20.63111375,684.32441469)(20.65111373,684.27441474)(20.67111654,684.23442034)
\curveto(20.74111364,684.1144149)(20.82111356,684.01941499)(20.91111654,683.94942034)
\curveto(21.01111337,683.87941513)(21.12611325,683.81941519)(21.25611654,683.76942034)
\curveto(21.44611293,683.68941532)(21.65111273,683.61941539)(21.87111654,683.55942034)
\lineto(22.56111654,683.40942034)
\curveto(22.80111158,683.36941564)(23.03111135,683.31941569)(23.25111654,683.25942034)
\curveto(23.4811109,683.2094158)(23.69611068,683.14441587)(23.89611654,683.06442034)
\curveto(23.98611039,683.02441599)(24.07111031,682.98941602)(24.15111654,682.95942034)
\curveto(24.24111014,682.93941607)(24.32611005,682.90441611)(24.40611654,682.85442034)
\curveto(24.59610978,682.73441628)(24.76610961,682.60441641)(24.91611654,682.46442034)
\curveto(25.0761093,682.32441669)(25.20110918,682.14941686)(25.29111654,681.93942034)
\curveto(25.32110906,681.86941714)(25.34610903,681.79941721)(25.36611654,681.72942034)
\curveto(25.38610899,681.65941735)(25.40610897,681.58441743)(25.42611654,681.50442034)
\curveto(25.43610894,681.44441757)(25.44110894,681.34941766)(25.44111654,681.21942034)
\curveto(25.45110893,681.09941791)(25.45110893,681.00441801)(25.44111654,680.93442034)
\lineto(25.44111654,680.85942034)
\curveto(25.42110896,680.79941821)(25.40610897,680.73941827)(25.39611654,680.67942034)
\curveto(25.39610898,680.62941838)(25.39110899,680.57941843)(25.38111654,680.52942034)
\curveto(25.31110907,680.22941878)(25.20110918,679.96441905)(25.05111654,679.73442034)
\curveto(24.89110949,679.49441952)(24.69610968,679.29941971)(24.46611654,679.14942034)
\curveto(24.23611014,678.99942001)(23.9761104,678.86942014)(23.68611654,678.75942034)
\curveto(23.5761108,678.7094203)(23.45611092,678.67442034)(23.32611654,678.65442034)
\curveto(23.20611117,678.63442038)(23.08611129,678.6094204)(22.96611654,678.57942034)
\curveto(22.8761115,678.55942045)(22.7811116,678.54942046)(22.68111654,678.54942034)
\curveto(22.59111179,678.53942047)(22.50111188,678.52442049)(22.41111654,678.50442034)
\lineto(22.14111654,678.50442034)
\curveto(22.0811123,678.48442053)(21.9761124,678.47442054)(21.82611654,678.47442034)
\curveto(21.68611269,678.47442054)(21.58611279,678.48442053)(21.52611654,678.50442034)
\curveto(21.49611288,678.50442051)(21.46111292,678.5094205)(21.42111654,678.51942034)
\lineto(21.31611654,678.51942034)
\curveto(21.19611318,678.53942047)(21.0761133,678.55442046)(20.95611654,678.56442034)
\curveto(20.83611354,678.57442044)(20.72111366,678.59442042)(20.61111654,678.62442034)
\curveto(20.22111416,678.73442028)(19.8761145,678.85942015)(19.57611654,678.99942034)
\curveto(19.2761151,679.14941986)(19.02111536,679.36941964)(18.81111654,679.65942034)
\curveto(18.67111571,679.84941916)(18.55111583,680.06941894)(18.45111654,680.31942034)
\curveto(18.43111595,680.37941863)(18.41111597,680.45941855)(18.39111654,680.55942034)
\curveto(18.37111601,680.6094184)(18.35611602,680.67941833)(18.34611654,680.76942034)
\curveto(18.33611604,680.85941815)(18.34111604,680.93441808)(18.36111654,680.99442034)
\curveto(18.39111599,681.06441795)(18.44111594,681.1144179)(18.51111654,681.14442034)
\curveto(18.56111582,681.16441785)(18.62111576,681.17441784)(18.69111654,681.17442034)
\lineto(18.91611654,681.17442034)
\lineto(19.62111654,681.17442034)
\lineto(19.86111654,681.17442034)
\curveto(19.94111444,681.17441784)(20.01111437,681.16441785)(20.07111654,681.14442034)
\curveto(20.1811142,681.10441791)(20.25111413,681.03941797)(20.28111654,680.94942034)
\curveto(20.32111406,680.85941815)(20.36611401,680.76441825)(20.41611654,680.66442034)
\curveto(20.43611394,680.6144184)(20.47111391,680.54941846)(20.52111654,680.46942034)
\curveto(20.5811138,680.38941862)(20.63111375,680.33941867)(20.67111654,680.31942034)
\curveto(20.79111359,680.21941879)(20.90611347,680.13941887)(21.01611654,680.07942034)
\curveto(21.12611325,680.02941898)(21.26611311,679.97941903)(21.43611654,679.92942034)
\curveto(21.48611289,679.9094191)(21.53611284,679.89941911)(21.58611654,679.89942034)
\curveto(21.63611274,679.9094191)(21.68611269,679.9094191)(21.73611654,679.89942034)
\curveto(21.81611256,679.87941913)(21.90111248,679.86941914)(21.99111654,679.86942034)
\curveto(22.09111229,679.87941913)(22.1761122,679.89441912)(22.24611654,679.91442034)
\curveto(22.29611208,679.92441909)(22.34111204,679.92941908)(22.38111654,679.92942034)
\curveto(22.43111195,679.92941908)(22.4811119,679.93941907)(22.53111654,679.95942034)
\curveto(22.67111171,680.009419)(22.79611158,680.06941894)(22.90611654,680.13942034)
\curveto(23.02611135,680.2094188)(23.12111126,680.29941871)(23.19111654,680.40942034)
\curveto(23.24111114,680.48941852)(23.2811111,680.6144184)(23.31111654,680.78442034)
\curveto(23.33111105,680.85441816)(23.33111105,680.91941809)(23.31111654,680.97942034)
\curveto(23.29111109,681.03941797)(23.27111111,681.08941792)(23.25111654,681.12942034)
\curveto(23.1811112,681.26941774)(23.09111129,681.37441764)(22.98111654,681.44442034)
\curveto(22.8811115,681.5144175)(22.76111162,681.57941743)(22.62111654,681.63942034)
\curveto(22.43111195,681.71941729)(22.23111215,681.78441723)(22.02111654,681.83442034)
\curveto(21.81111257,681.88441713)(21.60111278,681.93941707)(21.39111654,681.99942034)
\curveto(21.31111307,682.01941699)(21.22611315,682.03441698)(21.13611654,682.04442034)
\curveto(21.05611332,682.05441696)(20.9761134,682.06941694)(20.89611654,682.08942034)
\curveto(20.5761138,682.17941683)(20.27111411,682.26441675)(19.98111654,682.34442034)
\curveto(19.69111469,682.43441658)(19.42611495,682.56441645)(19.18611654,682.73442034)
\curveto(18.90611547,682.93441608)(18.70111568,683.20441581)(18.57111654,683.54442034)
\curveto(18.55111583,683.6144154)(18.53111585,683.7094153)(18.51111654,683.82942034)
\curveto(18.49111589,683.89941511)(18.4761159,683.98441503)(18.46611654,684.08442034)
\curveto(18.45611592,684.18441483)(18.46111592,684.27441474)(18.48111654,684.35442034)
\curveto(18.50111588,684.40441461)(18.50611587,684.44441457)(18.49611654,684.47442034)
\curveto(18.48611589,684.5144145)(18.49111589,684.55941445)(18.51111654,684.60942034)
\curveto(18.53111585,684.71941429)(18.55111583,684.81941419)(18.57111654,684.90942034)
\curveto(18.60111578,685.009414)(18.63611574,685.10441391)(18.67611654,685.19442034)
\curveto(18.80611557,685.48441353)(18.98611539,685.71941329)(19.21611654,685.89942034)
\curveto(19.44611493,686.07941293)(19.70611467,686.22441279)(19.99611654,686.33442034)
\curveto(20.10611427,686.38441263)(20.22111416,686.41941259)(20.34111654,686.43942034)
\curveto(20.46111392,686.46941254)(20.58611379,686.49941251)(20.71611654,686.52942034)
\curveto(20.7761136,686.54941246)(20.83611354,686.55941245)(20.89611654,686.55942034)
\lineto(21.07611654,686.58942034)
\curveto(21.15611322,686.59941241)(21.24111314,686.60441241)(21.33111654,686.60442034)
\curveto(21.42111296,686.60441241)(21.50611287,686.6094124)(21.58611654,686.61942034)
}
}
{
\newrgbcolor{curcolor}{0 0 0}
\pscustom[linestyle=none,fillstyle=solid,fillcolor=curcolor]
{
\newpath
\moveto(34.03775717,682.61442034)
\curveto(34.057749,682.53441648)(34.057749,682.44441657)(34.03775717,682.34442034)
\curveto(34.01774904,682.24441677)(33.98274908,682.17941683)(33.93275717,682.14942034)
\curveto(33.88274918,682.1094169)(33.80774925,682.07941693)(33.70775717,682.05942034)
\curveto(33.61774944,682.04941696)(33.51274955,682.03941697)(33.39275717,682.02942034)
\lineto(33.04775717,682.02942034)
\curveto(32.93775012,682.03941697)(32.83775022,682.04441697)(32.74775717,682.04442034)
\lineto(29.08775717,682.04442034)
\lineto(28.87775717,682.04442034)
\curveto(28.81775424,682.04441697)(28.7627543,682.03441698)(28.71275717,682.01442034)
\curveto(28.63275443,681.97441704)(28.58275448,681.93441708)(28.56275717,681.89442034)
\curveto(28.54275452,681.87441714)(28.52275454,681.83441718)(28.50275717,681.77442034)
\curveto(28.48275458,681.72441729)(28.47775458,681.67441734)(28.48775717,681.62442034)
\curveto(28.50775455,681.56441745)(28.51775454,681.50441751)(28.51775717,681.44442034)
\curveto(28.52775453,681.39441762)(28.54275452,681.33941767)(28.56275717,681.27942034)
\curveto(28.64275442,681.03941797)(28.73775432,680.83941817)(28.84775717,680.67942034)
\curveto(28.96775409,680.52941848)(29.12775393,680.39441862)(29.32775717,680.27442034)
\curveto(29.40775365,680.22441879)(29.48775357,680.18941882)(29.56775717,680.16942034)
\curveto(29.6577534,680.15941885)(29.74775331,680.13941887)(29.83775717,680.10942034)
\curveto(29.91775314,680.08941892)(30.02775303,680.07441894)(30.16775717,680.06442034)
\curveto(30.30775275,680.05441896)(30.42775263,680.05941895)(30.52775717,680.07942034)
\lineto(30.66275717,680.07942034)
\curveto(30.7627523,680.09941891)(30.85275221,680.11941889)(30.93275717,680.13942034)
\curveto(31.02275204,680.16941884)(31.10775195,680.19941881)(31.18775717,680.22942034)
\curveto(31.28775177,680.27941873)(31.39775166,680.34441867)(31.51775717,680.42442034)
\curveto(31.64775141,680.50441851)(31.74275132,680.58441843)(31.80275717,680.66442034)
\curveto(31.85275121,680.73441828)(31.90275116,680.79941821)(31.95275717,680.85942034)
\curveto(32.01275105,680.92941808)(32.08275098,680.97941803)(32.16275717,681.00942034)
\curveto(32.2627508,681.05941795)(32.38775067,681.07941793)(32.53775717,681.06942034)
\lineto(32.97275717,681.06942034)
\lineto(33.15275717,681.06942034)
\curveto(33.22274984,681.07941793)(33.28274978,681.07441794)(33.33275717,681.05442034)
\lineto(33.48275717,681.05442034)
\curveto(33.58274948,681.03441798)(33.65274941,681.009418)(33.69275717,680.97942034)
\curveto(33.73274933,680.95941805)(33.75274931,680.9144181)(33.75275717,680.84442034)
\curveto(33.7627493,680.77441824)(33.7577493,680.7144183)(33.73775717,680.66442034)
\curveto(33.68774937,680.52441849)(33.63274943,680.39941861)(33.57275717,680.28942034)
\curveto(33.51274955,680.17941883)(33.44274962,680.06941894)(33.36275717,679.95942034)
\curveto(33.14274992,679.62941938)(32.89275017,679.36441965)(32.61275717,679.16442034)
\curveto(32.33275073,678.96442005)(31.98275108,678.79442022)(31.56275717,678.65442034)
\curveto(31.45275161,678.6144204)(31.34275172,678.58942042)(31.23275717,678.57942034)
\curveto(31.12275194,678.56942044)(31.00775205,678.54942046)(30.88775717,678.51942034)
\curveto(30.84775221,678.5094205)(30.80275226,678.5094205)(30.75275717,678.51942034)
\curveto(30.71275235,678.51942049)(30.67275239,678.5144205)(30.63275717,678.50442034)
\lineto(30.46775717,678.50442034)
\curveto(30.41775264,678.48442053)(30.3577527,678.47942053)(30.28775717,678.48942034)
\curveto(30.22775283,678.48942052)(30.17275289,678.49442052)(30.12275717,678.50442034)
\curveto(30.04275302,678.5144205)(29.97275309,678.5144205)(29.91275717,678.50442034)
\curveto(29.85275321,678.49442052)(29.78775327,678.49942051)(29.71775717,678.51942034)
\curveto(29.66775339,678.53942047)(29.61275345,678.54942046)(29.55275717,678.54942034)
\curveto(29.49275357,678.54942046)(29.43775362,678.55942045)(29.38775717,678.57942034)
\curveto(29.27775378,678.59942041)(29.16775389,678.62442039)(29.05775717,678.65442034)
\curveto(28.94775411,678.67442034)(28.84775421,678.7094203)(28.75775717,678.75942034)
\curveto(28.64775441,678.79942021)(28.54275452,678.83442018)(28.44275717,678.86442034)
\curveto(28.35275471,678.90442011)(28.26775479,678.94942006)(28.18775717,678.99942034)
\curveto(27.86775519,679.19941981)(27.58275548,679.42941958)(27.33275717,679.68942034)
\curveto(27.08275598,679.95941905)(26.87775618,680.26941874)(26.71775717,680.61942034)
\curveto(26.66775639,680.72941828)(26.62775643,680.83941817)(26.59775717,680.94942034)
\curveto(26.56775649,681.06941794)(26.52775653,681.18941782)(26.47775717,681.30942034)
\curveto(26.46775659,681.34941766)(26.4627566,681.38441763)(26.46275717,681.41442034)
\curveto(26.4627566,681.45441756)(26.4577566,681.49441752)(26.44775717,681.53442034)
\curveto(26.40775665,681.65441736)(26.38275668,681.78441723)(26.37275717,681.92442034)
\lineto(26.34275717,682.34442034)
\curveto(26.34275672,682.39441662)(26.33775672,682.44941656)(26.32775717,682.50942034)
\curveto(26.32775673,682.56941644)(26.33275673,682.62441639)(26.34275717,682.67442034)
\lineto(26.34275717,682.85442034)
\lineto(26.38775717,683.21442034)
\curveto(26.42775663,683.38441563)(26.4627566,683.54941546)(26.49275717,683.70942034)
\curveto(26.52275654,683.86941514)(26.56775649,684.01941499)(26.62775717,684.15942034)
\curveto(27.057756,685.19941381)(27.78775527,685.93441308)(28.81775717,686.36442034)
\curveto(28.9577541,686.42441259)(29.09775396,686.46441255)(29.23775717,686.48442034)
\curveto(29.38775367,686.5144125)(29.54275352,686.54941246)(29.70275717,686.58942034)
\curveto(29.78275328,686.59941241)(29.8577532,686.60441241)(29.92775717,686.60442034)
\curveto(29.99775306,686.60441241)(30.07275299,686.6094124)(30.15275717,686.61942034)
\curveto(30.6627524,686.62941238)(31.09775196,686.56941244)(31.45775717,686.43942034)
\curveto(31.82775123,686.31941269)(32.1577509,686.15941285)(32.44775717,685.95942034)
\curveto(32.53775052,685.89941311)(32.62775043,685.82941318)(32.71775717,685.74942034)
\curveto(32.80775025,685.67941333)(32.88775017,685.60441341)(32.95775717,685.52442034)
\curveto(32.98775007,685.47441354)(33.02775003,685.43441358)(33.07775717,685.40442034)
\curveto(33.1577499,685.29441372)(33.23274983,685.17941383)(33.30275717,685.05942034)
\curveto(33.37274969,684.94941406)(33.44774961,684.83441418)(33.52775717,684.71442034)
\curveto(33.57774948,684.62441439)(33.61774944,684.52941448)(33.64775717,684.42942034)
\curveto(33.68774937,684.33941467)(33.72774933,684.23941477)(33.76775717,684.12942034)
\curveto(33.81774924,683.99941501)(33.8577492,683.86441515)(33.88775717,683.72442034)
\curveto(33.91774914,683.58441543)(33.95274911,683.44441557)(33.99275717,683.30442034)
\curveto(34.01274905,683.22441579)(34.01774904,683.13441588)(34.00775717,683.03442034)
\curveto(34.00774905,682.94441607)(34.01774904,682.85941615)(34.03775717,682.77942034)
\lineto(34.03775717,682.61442034)
\moveto(31.78775717,683.49942034)
\curveto(31.8577512,683.59941541)(31.8627512,683.71941529)(31.80275717,683.85942034)
\curveto(31.75275131,684.009415)(31.71275135,684.11941489)(31.68275717,684.18942034)
\curveto(31.54275152,684.45941455)(31.3577517,684.66441435)(31.12775717,684.80442034)
\curveto(30.89775216,684.95441406)(30.57775248,685.03441398)(30.16775717,685.04442034)
\curveto(30.13775292,685.02441399)(30.10275296,685.01941399)(30.06275717,685.02942034)
\curveto(30.02275304,685.03941397)(29.98775307,685.03941397)(29.95775717,685.02942034)
\curveto(29.90775315,685.009414)(29.85275321,684.99441402)(29.79275717,684.98442034)
\curveto(29.73275333,684.98441403)(29.67775338,684.97441404)(29.62775717,684.95442034)
\curveto(29.18775387,684.8144142)(28.8627542,684.53941447)(28.65275717,684.12942034)
\curveto(28.63275443,684.08941492)(28.60775445,684.03441498)(28.57775717,683.96442034)
\curveto(28.5577545,683.90441511)(28.54275452,683.83941517)(28.53275717,683.76942034)
\curveto(28.52275454,683.7094153)(28.52275454,683.64941536)(28.53275717,683.58942034)
\curveto(28.55275451,683.52941548)(28.58775447,683.47941553)(28.63775717,683.43942034)
\curveto(28.71775434,683.38941562)(28.82775423,683.36441565)(28.96775717,683.36442034)
\lineto(29.37275717,683.36442034)
\lineto(31.03775717,683.36442034)
\lineto(31.47275717,683.36442034)
\curveto(31.63275143,683.37441564)(31.73775132,683.41941559)(31.78775717,683.49942034)
}
}
{
\newrgbcolor{curcolor}{0 0 0}
\pscustom[linestyle=none,fillstyle=solid,fillcolor=curcolor]
{
\newpath
\moveto(42.63603842,686.31942034)
\curveto(42.70603022,686.26941274)(42.74103018,686.18441283)(42.74103842,686.06442034)
\curveto(42.75103017,685.95441306)(42.75603017,685.83941317)(42.75603842,685.71942034)
\lineto(42.75603842,679.31442034)
\curveto(42.75603017,679.23441978)(42.75103017,679.15441986)(42.74103842,679.07442034)
\lineto(42.74103842,678.84942034)
\curveto(42.73103019,678.76942024)(42.7210302,678.69942031)(42.71103842,678.63942034)
\curveto(42.71103021,678.56942044)(42.70603022,678.49442052)(42.69603842,678.41442034)
\curveto(42.65603027,678.27442074)(42.6210303,678.14442087)(42.59103842,678.02442034)
\curveto(42.57103035,677.89442112)(42.53603039,677.77442124)(42.48603842,677.66442034)
\curveto(42.31603061,677.28442173)(42.09603083,676.96942204)(41.82603842,676.71942034)
\curveto(41.56603136,676.46942254)(41.24603168,676.26442275)(40.86603842,676.10442034)
\curveto(40.75603217,676.05442296)(40.64603228,676.014423)(40.53603842,675.98442034)
\curveto(40.4260325,675.95442306)(40.31103261,675.92442309)(40.19103842,675.89442034)
\curveto(40.08103284,675.86442315)(39.97103295,675.84442317)(39.86103842,675.83442034)
\curveto(39.75103317,675.82442319)(39.64103328,675.8094232)(39.53103842,675.78942034)
\lineto(39.41103842,675.78942034)
\curveto(39.37103355,675.77942323)(39.3260336,675.77442324)(39.27603842,675.77442034)
\curveto(39.23603369,675.76442325)(39.19103373,675.76442325)(39.14103842,675.77442034)
\curveto(39.09103383,675.77442324)(39.04103388,675.76942324)(38.99103842,675.75942034)
\curveto(38.94103398,675.74942326)(38.87603405,675.74442327)(38.79603842,675.74442034)
\curveto(38.71603421,675.74442327)(38.65103427,675.74942326)(38.60103842,675.75942034)
\lineto(38.46603842,675.75942034)
\curveto(38.4260345,675.75942325)(38.38603454,675.76442325)(38.34603842,675.77442034)
\curveto(38.26603466,675.79442322)(38.18103474,675.80442321)(38.09103842,675.80442034)
\curveto(38.01103491,675.80442321)(37.93603499,675.8144232)(37.86603842,675.83442034)
\curveto(37.84603508,675.84442317)(37.8210351,675.84942316)(37.79103842,675.84942034)
\curveto(37.76103516,675.84942316)(37.73603519,675.85442316)(37.71603842,675.86442034)
\curveto(37.61603531,675.88442313)(37.51603541,675.9094231)(37.41603842,675.93942034)
\curveto(37.3260356,675.95942305)(37.23603569,675.98942302)(37.14603842,676.02942034)
\curveto(36.76603616,676.18942282)(36.4260365,676.39442262)(36.12603842,676.64442034)
\curveto(35.8260371,676.88442213)(35.60603732,677.2094218)(35.46603842,677.61942034)
\curveto(35.44603748,677.64942136)(35.43603749,677.67942133)(35.43603842,677.70942034)
\curveto(35.43603749,677.73942127)(35.43103749,677.76442125)(35.42103842,677.78442034)
\curveto(35.39103753,677.9144211)(35.40103752,678.014421)(35.45103842,678.08442034)
\curveto(35.51103741,678.14442087)(35.59103733,678.18442083)(35.69103842,678.20442034)
\curveto(35.79103713,678.22442079)(35.90103702,678.23442078)(36.02103842,678.23442034)
\curveto(36.15103677,678.22442079)(36.27103665,678.21942079)(36.38103842,678.21942034)
\lineto(36.89103842,678.21942034)
\lineto(37.01103842,678.21942034)
\curveto(37.05103587,678.2094208)(37.09603583,678.20442081)(37.14603842,678.20442034)
\curveto(37.30603562,678.16442085)(37.40603552,678.1144209)(37.44603842,678.05442034)
\curveto(37.48603544,677.98442103)(37.54603538,677.89442112)(37.62603842,677.78442034)
\curveto(37.65603527,677.74442127)(37.70103522,677.69442132)(37.76103842,677.63442034)
\curveto(37.77103515,677.6144214)(37.78103514,677.59942141)(37.79103842,677.58942034)
\curveto(37.80103512,677.57942143)(37.81103511,677.56442145)(37.82103842,677.54442034)
\curveto(37.90103502,677.48442153)(37.98603494,677.42942158)(38.07603842,677.37942034)
\curveto(38.16603476,677.32942168)(38.26603466,677.28442173)(38.37603842,677.24442034)
\curveto(38.44603448,677.22442179)(38.51603441,677.2144218)(38.58603842,677.21442034)
\curveto(38.65603427,677.20442181)(38.73103419,677.18942182)(38.81103842,677.16942034)
\lineto(38.97603842,677.16942034)
\curveto(39.04603388,677.14942186)(39.13603379,677.14942186)(39.24603842,677.16942034)
\curveto(39.35603357,677.17942183)(39.44103348,677.19442182)(39.50103842,677.21442034)
\curveto(39.55103337,677.23442178)(39.59103333,677.24442177)(39.62103842,677.24442034)
\curveto(39.66103326,677.24442177)(39.70103322,677.25442176)(39.74103842,677.27442034)
\curveto(39.95103297,677.36442165)(40.1260328,677.48442153)(40.26603842,677.63442034)
\curveto(40.40603252,677.78442123)(40.5210324,677.95942105)(40.61103842,678.15942034)
\curveto(40.63103229,678.21942079)(40.64603228,678.27942073)(40.65603842,678.33942034)
\curveto(40.66603226,678.39942061)(40.68103224,678.46442055)(40.70103842,678.53442034)
\curveto(40.7210322,678.62442039)(40.73103219,678.71942029)(40.73103842,678.81942034)
\curveto(40.74103218,678.92942008)(40.74603218,679.03941997)(40.74603842,679.14942034)
\lineto(40.74603842,679.26942034)
\curveto(40.75603217,679.3094197)(40.75603217,679.34441967)(40.74603842,679.37442034)
\curveto(40.7260322,679.42441959)(40.71603221,679.46941954)(40.71603842,679.50942034)
\curveto(40.7260322,679.54941946)(40.7210322,679.58941942)(40.70103842,679.62942034)
\curveto(40.69103223,679.64941936)(40.67603225,679.66441935)(40.65603842,679.67442034)
\lineto(40.61103842,679.71942034)
\curveto(40.5210324,679.72941928)(40.44603248,679.7094193)(40.38603842,679.65942034)
\curveto(40.33603259,679.6094194)(40.28603264,679.56441945)(40.23603842,679.52442034)
\curveto(40.14603278,679.45441956)(40.05603287,679.38941962)(39.96603842,679.32942034)
\curveto(39.87603305,679.26941974)(39.77603315,679.2144198)(39.66603842,679.16442034)
\curveto(39.55603337,679.1144199)(39.44603348,679.07441994)(39.33603842,679.04442034)
\curveto(39.2260337,679.01442)(39.11103381,678.98442003)(38.99103842,678.95442034)
\lineto(38.81103842,678.92442034)
\curveto(38.76103416,678.92442009)(38.71103421,678.91942009)(38.66103842,678.90942034)
\curveto(38.61103431,678.89942011)(38.53103439,678.89442012)(38.42103842,678.89442034)
\curveto(38.31103461,678.89442012)(38.23103469,678.89942011)(38.18103842,678.90942034)
\lineto(38.06103842,678.90942034)
\curveto(38.03103489,678.91942009)(37.99603493,678.92442009)(37.95603842,678.92442034)
\curveto(37.926035,678.92442009)(37.89103503,678.92942008)(37.85103842,678.93942034)
\curveto(37.71103521,678.96942004)(37.57603535,678.99442002)(37.44603842,679.01442034)
\curveto(37.31603561,679.04441997)(37.19603573,679.08441993)(37.08603842,679.13442034)
\curveto(36.65603627,679.30441971)(36.30603662,679.53941947)(36.03603842,679.83942034)
\curveto(35.77603715,680.14941886)(35.55603737,680.51941849)(35.37603842,680.94942034)
\curveto(35.3260376,681.05941795)(35.29103763,681.17441784)(35.27103842,681.29442034)
\curveto(35.25103767,681.4144176)(35.2210377,681.53441748)(35.18103842,681.65442034)
\curveto(35.18103774,681.70441731)(35.17603775,681.74441727)(35.16603842,681.77442034)
\curveto(35.14603778,681.85441716)(35.13603779,681.93941707)(35.13603842,682.02942034)
\curveto(35.13603779,682.12941688)(35.1260378,682.21941679)(35.10603842,682.29942034)
\curveto(35.09603783,682.34941666)(35.09103783,682.39441662)(35.09103842,682.43442034)
\lineto(35.09103842,682.58442034)
\curveto(35.08103784,682.63441638)(35.07603785,682.69441632)(35.07603842,682.76442034)
\curveto(35.07603785,682.84441617)(35.08103784,682.9094161)(35.09103842,682.95942034)
\lineto(35.09103842,683.10942034)
\curveto(35.10103782,683.14941586)(35.10103782,683.18941582)(35.09103842,683.22942034)
\curveto(35.09103783,683.26941574)(35.10103782,683.3094157)(35.12103842,683.34942034)
\curveto(35.14103778,683.44941556)(35.15603777,683.54441547)(35.16603842,683.63442034)
\curveto(35.17603775,683.73441528)(35.19103773,683.83441518)(35.21103842,683.93442034)
\curveto(35.27103765,684.13441488)(35.33103759,684.32441469)(35.39103842,684.50442034)
\curveto(35.46103746,684.68441433)(35.54603738,684.85441416)(35.64603842,685.01442034)
\curveto(35.69603723,685.1144139)(35.75103717,685.20441381)(35.81103842,685.28442034)
\lineto(36.02103842,685.55442034)
\curveto(36.05103687,685.60441341)(36.09103683,685.65441336)(36.14103842,685.70442034)
\curveto(36.20103672,685.75441326)(36.25603667,685.79941321)(36.30603842,685.83942034)
\lineto(36.39603842,685.92942034)
\curveto(36.44603648,685.96941304)(36.49603643,686.00441301)(36.54603842,686.03442034)
\curveto(36.59603633,686.07441294)(36.64603628,686.1094129)(36.69603842,686.13942034)
\curveto(36.8260361,686.21941279)(36.96103596,686.28941272)(37.10103842,686.34942034)
\curveto(37.24103568,686.4094126)(37.39603553,686.46441255)(37.56603842,686.51442034)
\curveto(37.64603528,686.54441247)(37.7260352,686.55941245)(37.80603842,686.55942034)
\curveto(37.89603503,686.56941244)(37.98103494,686.58441243)(38.06103842,686.60442034)
\curveto(38.10103482,686.6144124)(38.15603477,686.6144124)(38.22603842,686.60442034)
\curveto(38.29603463,686.59441242)(38.34103458,686.59941241)(38.36103842,686.61942034)
\curveto(38.68103424,686.62941238)(38.96603396,686.59941241)(39.21603842,686.52942034)
\curveto(39.47603345,686.45941255)(39.70603322,686.35941265)(39.90603842,686.22942034)
\curveto(39.93603299,686.2094128)(39.96603296,686.18441283)(39.99603842,686.15442034)
\curveto(40.0260329,686.13441288)(40.06103286,686.1094129)(40.10103842,686.07942034)
\curveto(40.16103276,686.02941298)(40.21603271,685.97941303)(40.26603842,685.92942034)
\curveto(40.31603261,685.87941313)(40.37603255,685.83441318)(40.44603842,685.79442034)
\curveto(40.46603246,685.78441323)(40.49103243,685.77441324)(40.52103842,685.76442034)
\curveto(40.56103236,685.75441326)(40.59103233,685.75941325)(40.61103842,685.77942034)
\curveto(40.66103226,685.79941321)(40.69103223,685.83441318)(40.70103842,685.88442034)
\curveto(40.71103221,685.93441308)(40.7260322,685.98441303)(40.74603842,686.03442034)
\curveto(40.76603216,686.08441293)(40.78103214,686.13441288)(40.79103842,686.18442034)
\curveto(40.81103211,686.24441277)(40.84103208,686.29441272)(40.88103842,686.33442034)
\curveto(40.94103198,686.37441264)(41.01103191,686.39441262)(41.09103842,686.39442034)
\curveto(41.18103174,686.40441261)(41.27103165,686.4094126)(41.36103842,686.40942034)
\lineto(42.12603842,686.40942034)
\curveto(42.23603069,686.4094126)(42.33103059,686.40441261)(42.41103842,686.39442034)
\curveto(42.50103042,686.39441262)(42.57603035,686.36941264)(42.63603842,686.31942034)
\moveto(40.58103842,681.68442034)
\curveto(40.6210323,681.77441724)(40.65603227,681.88941712)(40.68603842,682.02942034)
\curveto(40.71603221,682.16941684)(40.73603219,682.3144167)(40.74603842,682.46442034)
\curveto(40.75603217,682.62441639)(40.75603217,682.77941623)(40.74603842,682.92942034)
\curveto(40.74603218,683.07941593)(40.73103219,683.2144158)(40.70103842,683.33442034)
\curveto(40.68103224,683.37441564)(40.67103225,683.40441561)(40.67103842,683.42442034)
\curveto(40.68103224,683.45441556)(40.68103224,683.48941552)(40.67103842,683.52942034)
\lineto(40.61103842,683.73942034)
\curveto(40.59103233,683.8094152)(40.56603236,683.87441514)(40.53603842,683.93442034)
\curveto(40.39603253,684.28441473)(40.19603273,684.55441446)(39.93603842,684.74442034)
\curveto(39.67603325,684.93441408)(39.29603363,685.02941398)(38.79603842,685.02942034)
\curveto(38.77603415,685.009414)(38.74603418,684.99941401)(38.70603842,684.99942034)
\curveto(38.67603425,685.009414)(38.64603428,685.009414)(38.61603842,684.99942034)
\curveto(38.54603438,684.97941403)(38.48103444,684.95941405)(38.42103842,684.93942034)
\curveto(38.36103456,684.92941408)(38.30103462,684.9144141)(38.24103842,684.89442034)
\curveto(37.98103494,684.78441423)(37.78103514,684.61941439)(37.64103842,684.39942034)
\curveto(37.50103542,684.17941483)(37.38603554,683.93441508)(37.29603842,683.66442034)
\curveto(37.27603565,683.6144154)(37.26603566,683.57441544)(37.26603842,683.54442034)
\curveto(37.26603566,683.5144155)(37.26103566,683.47441554)(37.25103842,683.42442034)
\curveto(37.2210357,683.3144157)(37.20103572,683.15441586)(37.19103842,682.94442034)
\curveto(37.18103574,682.73441628)(37.19103573,682.56441645)(37.22103842,682.43442034)
\lineto(37.22103842,682.28442034)
\curveto(37.24103568,682.20441681)(37.25603567,682.12441689)(37.26603842,682.04442034)
\curveto(37.27603565,681.97441704)(37.29103563,681.89941711)(37.31103842,681.81942034)
\curveto(37.40103552,681.55941745)(37.51103541,681.32941768)(37.64103842,681.12942034)
\curveto(37.77103515,680.93941807)(37.95103497,680.78441823)(38.18103842,680.66442034)
\curveto(38.28103464,680.6144184)(38.4210345,680.56441845)(38.60103842,680.51442034)
\curveto(38.67103425,680.5144185)(38.7260342,680.5094185)(38.76603842,680.49942034)
\curveto(38.78603414,680.49941851)(38.81603411,680.49441852)(38.85603842,680.48442034)
\curveto(38.89603403,680.48441853)(38.926034,680.48941852)(38.94603842,680.49942034)
\lineto(39.09603842,680.49942034)
\curveto(39.18603374,680.51941849)(39.27103365,680.53441848)(39.35103842,680.54442034)
\curveto(39.43103349,680.55441846)(39.51103341,680.57941843)(39.59103842,680.61942034)
\curveto(39.84103308,680.71941829)(40.04103288,680.85941815)(40.19103842,681.03942034)
\curveto(40.35103257,681.21941779)(40.48103244,681.43441758)(40.58103842,681.68442034)
}
}
{
\newrgbcolor{curcolor}{0 0 0}
\pscustom[linestyle=none,fillstyle=solid,fillcolor=curcolor]
{
\newpath
\moveto(44.87596029,686.39442034)
\lineto(46.00096029,686.39442034)
\curveto(46.11095786,686.39441262)(46.21095776,686.38941262)(46.30096029,686.37942034)
\curveto(46.39095758,686.36941264)(46.45595751,686.33441268)(46.49596029,686.27442034)
\curveto(46.54595742,686.2144128)(46.57595739,686.12941288)(46.58596029,686.01942034)
\curveto(46.59595737,685.91941309)(46.60095737,685.8144132)(46.60096029,685.70442034)
\lineto(46.60096029,684.65442034)
\lineto(46.60096029,682.41942034)
\curveto(46.60095737,682.05941695)(46.61595735,681.71941729)(46.64596029,681.39942034)
\curveto(46.67595729,681.07941793)(46.7659572,680.8144182)(46.91596029,680.60442034)
\curveto(47.05595691,680.39441862)(47.28095669,680.24441877)(47.59096029,680.15442034)
\curveto(47.64095633,680.14441887)(47.68095629,680.13941887)(47.71096029,680.13942034)
\curveto(47.75095622,680.13941887)(47.79595617,680.13441888)(47.84596029,680.12442034)
\curveto(47.89595607,680.1144189)(47.95095602,680.1094189)(48.01096029,680.10942034)
\curveto(48.0709559,680.1094189)(48.11595585,680.1144189)(48.14596029,680.12442034)
\curveto(48.19595577,680.14441887)(48.23595573,680.14941886)(48.26596029,680.13942034)
\curveto(48.30595566,680.12941888)(48.34595562,680.13441888)(48.38596029,680.15442034)
\curveto(48.59595537,680.20441881)(48.76095521,680.26941874)(48.88096029,680.34942034)
\curveto(49.06095491,680.45941855)(49.20095477,680.59941841)(49.30096029,680.76942034)
\curveto(49.41095456,680.94941806)(49.48595448,681.14441787)(49.52596029,681.35442034)
\curveto(49.57595439,681.57441744)(49.60595436,681.8144172)(49.61596029,682.07442034)
\curveto(49.62595434,682.34441667)(49.63095434,682.62441639)(49.63096029,682.91442034)
\lineto(49.63096029,684.72942034)
\lineto(49.63096029,685.70442034)
\lineto(49.63096029,685.97442034)
\curveto(49.63095434,686.07441294)(49.65095432,686.15441286)(49.69096029,686.21442034)
\curveto(49.74095423,686.30441271)(49.81595415,686.35441266)(49.91596029,686.36442034)
\curveto(50.01595395,686.38441263)(50.13595383,686.39441262)(50.27596029,686.39442034)
\lineto(51.07096029,686.39442034)
\lineto(51.35596029,686.39442034)
\curveto(51.44595252,686.39441262)(51.52095245,686.37441264)(51.58096029,686.33442034)
\curveto(51.66095231,686.28441273)(51.70595226,686.2094128)(51.71596029,686.10942034)
\curveto(51.72595224,686.009413)(51.73095224,685.89441312)(51.73096029,685.76442034)
\lineto(51.73096029,684.62442034)
\lineto(51.73096029,680.40942034)
\lineto(51.73096029,679.34442034)
\lineto(51.73096029,679.04442034)
\curveto(51.73095224,678.94442007)(51.71095226,678.86942014)(51.67096029,678.81942034)
\curveto(51.62095235,678.73942027)(51.54595242,678.69442032)(51.44596029,678.68442034)
\curveto(51.34595262,678.67442034)(51.24095273,678.66942034)(51.13096029,678.66942034)
\lineto(50.32096029,678.66942034)
\curveto(50.21095376,678.66942034)(50.11095386,678.67442034)(50.02096029,678.68442034)
\curveto(49.94095403,678.69442032)(49.87595409,678.73442028)(49.82596029,678.80442034)
\curveto(49.80595416,678.83442018)(49.78595418,678.87942013)(49.76596029,678.93942034)
\curveto(49.75595421,678.99942001)(49.74095423,679.05941995)(49.72096029,679.11942034)
\curveto(49.71095426,679.17941983)(49.69595427,679.23441978)(49.67596029,679.28442034)
\curveto(49.65595431,679.33441968)(49.62595434,679.36441965)(49.58596029,679.37442034)
\curveto(49.5659544,679.39441962)(49.54095443,679.39941961)(49.51096029,679.38942034)
\curveto(49.48095449,679.37941963)(49.45595451,679.36941964)(49.43596029,679.35942034)
\curveto(49.3659546,679.31941969)(49.30595466,679.27441974)(49.25596029,679.22442034)
\curveto(49.20595476,679.17441984)(49.15095482,679.12941988)(49.09096029,679.08942034)
\curveto(49.05095492,679.05941995)(49.01095496,679.02441999)(48.97096029,678.98442034)
\curveto(48.94095503,678.95442006)(48.90095507,678.92442009)(48.85096029,678.89442034)
\curveto(48.62095535,678.75442026)(48.35095562,678.64442037)(48.04096029,678.56442034)
\curveto(47.970956,678.54442047)(47.90095607,678.53442048)(47.83096029,678.53442034)
\curveto(47.76095621,678.52442049)(47.68595628,678.5094205)(47.60596029,678.48942034)
\curveto(47.5659564,678.47942053)(47.52095645,678.47942053)(47.47096029,678.48942034)
\curveto(47.43095654,678.48942052)(47.39095658,678.48442053)(47.35096029,678.47442034)
\curveto(47.32095665,678.46442055)(47.25595671,678.46442055)(47.15596029,678.47442034)
\curveto(47.0659569,678.47442054)(47.00595696,678.47942053)(46.97596029,678.48942034)
\curveto(46.92595704,678.48942052)(46.87595709,678.49442052)(46.82596029,678.50442034)
\lineto(46.67596029,678.50442034)
\curveto(46.55595741,678.53442048)(46.44095753,678.55942045)(46.33096029,678.57942034)
\curveto(46.22095775,678.59942041)(46.11095786,678.62942038)(46.00096029,678.66942034)
\curveto(45.95095802,678.68942032)(45.90595806,678.70442031)(45.86596029,678.71442034)
\curveto(45.83595813,678.73442028)(45.79595817,678.75442026)(45.74596029,678.77442034)
\curveto(45.39595857,678.96442005)(45.11595885,679.22941978)(44.90596029,679.56942034)
\curveto(44.77595919,679.77941923)(44.68095929,680.02941898)(44.62096029,680.31942034)
\curveto(44.56095941,680.61941839)(44.52095945,680.93441808)(44.50096029,681.26442034)
\curveto(44.49095948,681.60441741)(44.48595948,681.94941706)(44.48596029,682.29942034)
\curveto(44.49595947,682.65941635)(44.50095947,683.014416)(44.50096029,683.36442034)
\lineto(44.50096029,685.40442034)
\curveto(44.50095947,685.53441348)(44.49595947,685.68441333)(44.48596029,685.85442034)
\curveto(44.48595948,686.03441298)(44.51095946,686.16441285)(44.56096029,686.24442034)
\curveto(44.59095938,686.29441272)(44.65095932,686.33941267)(44.74096029,686.37942034)
\curveto(44.80095917,686.37941263)(44.84595912,686.38441263)(44.87596029,686.39442034)
\moveto(48.74596029,689.48442034)
\lineto(49.81096029,689.48442034)
\curveto(49.89095408,689.48440953)(49.98595398,689.48440953)(50.09596029,689.48442034)
\curveto(50.20595376,689.48440953)(50.28595368,689.46940954)(50.33596029,689.43942034)
\curveto(50.35595361,689.42940958)(50.3659536,689.4144096)(50.36596029,689.39442034)
\curveto(50.37595359,689.38440963)(50.39095358,689.37440964)(50.41096029,689.36442034)
\curveto(50.42095355,689.24440977)(50.3709536,689.13940987)(50.26096029,689.04942034)
\curveto(50.16095381,688.95941005)(50.07595389,688.87941013)(50.00596029,688.80942034)
\curveto(49.92595404,688.73941027)(49.84595412,688.66441035)(49.76596029,688.58442034)
\curveto(49.69595427,688.5144105)(49.62095435,688.44941056)(49.54096029,688.38942034)
\curveto(49.50095447,688.35941065)(49.4659545,688.32441069)(49.43596029,688.28442034)
\curveto(49.41595455,688.25441076)(49.38595458,688.22941078)(49.34596029,688.20942034)
\curveto(49.32595464,688.17941083)(49.30095467,688.15441086)(49.27096029,688.13442034)
\lineto(49.12096029,687.98442034)
\lineto(48.97096029,687.86442034)
\lineto(48.92596029,687.81942034)
\curveto(48.92595504,687.8094112)(48.91595505,687.79441122)(48.89596029,687.77442034)
\curveto(48.81595515,687.7144113)(48.73595523,687.64941136)(48.65596029,687.57942034)
\curveto(48.58595538,687.5094115)(48.49595547,687.45441156)(48.38596029,687.41442034)
\curveto(48.34595562,687.40441161)(48.30595566,687.39941161)(48.26596029,687.39942034)
\curveto(48.23595573,687.39941161)(48.19595577,687.39441162)(48.14596029,687.38442034)
\curveto(48.11595585,687.37441164)(48.07595589,687.36941164)(48.02596029,687.36942034)
\curveto(47.97595599,687.37941163)(47.93095604,687.38441163)(47.89096029,687.38442034)
\lineto(47.54596029,687.38442034)
\curveto(47.42595654,687.38441163)(47.33595663,687.4094116)(47.27596029,687.45942034)
\curveto(47.21595675,687.49941151)(47.20095677,687.56941144)(47.23096029,687.66942034)
\curveto(47.25095672,687.74941126)(47.28595668,687.81941119)(47.33596029,687.87942034)
\curveto(47.38595658,687.94941106)(47.43095654,688.01941099)(47.47096029,688.08942034)
\curveto(47.5709564,688.22941078)(47.6659563,688.36441065)(47.75596029,688.49442034)
\curveto(47.84595612,688.62441039)(47.93595603,688.75941025)(48.02596029,688.89942034)
\curveto(48.07595589,688.97941003)(48.12595584,689.06440995)(48.17596029,689.15442034)
\curveto(48.23595573,689.24440977)(48.30095567,689.3144097)(48.37096029,689.36442034)
\curveto(48.41095556,689.39440962)(48.48095549,689.42940958)(48.58096029,689.46942034)
\curveto(48.60095537,689.47940953)(48.62595534,689.47940953)(48.65596029,689.46942034)
\curveto(48.69595527,689.46940954)(48.72595524,689.47440954)(48.74596029,689.48442034)
}
}
{
\newrgbcolor{curcolor}{0 0 0}
\pscustom[linestyle=none,fillstyle=solid,fillcolor=curcolor]
{
\newpath
\moveto(57.78721029,686.60442034)
\curveto(58.38720449,686.62441239)(58.88720399,686.53941247)(59.28721029,686.34942034)
\curveto(59.68720319,686.15941285)(60.00220287,685.87941313)(60.23221029,685.50942034)
\curveto(60.30220257,685.39941361)(60.35720252,685.27941373)(60.39721029,685.14942034)
\curveto(60.43720244,685.02941398)(60.4772024,684.90441411)(60.51721029,684.77442034)
\curveto(60.53720234,684.69441432)(60.54720233,684.61941439)(60.54721029,684.54942034)
\curveto(60.55720232,684.47941453)(60.5722023,684.4094146)(60.59221029,684.33942034)
\curveto(60.59220228,684.27941473)(60.59720228,684.23941477)(60.60721029,684.21942034)
\curveto(60.62720225,684.07941493)(60.63720224,683.93441508)(60.63721029,683.78442034)
\lineto(60.63721029,683.34942034)
\lineto(60.63721029,682.01442034)
\lineto(60.63721029,679.58442034)
\curveto(60.63720224,679.39441962)(60.63220224,679.2094198)(60.62221029,679.02942034)
\curveto(60.62220225,678.85942015)(60.55220232,678.74942026)(60.41221029,678.69942034)
\curveto(60.35220252,678.67942033)(60.28220259,678.66942034)(60.20221029,678.66942034)
\lineto(59.96221029,678.66942034)
\lineto(59.15221029,678.66942034)
\curveto(59.03220384,678.66942034)(58.92220395,678.67442034)(58.82221029,678.68442034)
\curveto(58.73220414,678.70442031)(58.66220421,678.74942026)(58.61221029,678.81942034)
\curveto(58.5722043,678.87942013)(58.54720433,678.95442006)(58.53721029,679.04442034)
\lineto(58.53721029,679.35942034)
\lineto(58.53721029,680.40942034)
\lineto(58.53721029,682.64442034)
\curveto(58.53720434,683.014416)(58.52220435,683.35441566)(58.49221029,683.66442034)
\curveto(58.46220441,683.98441503)(58.3722045,684.25441476)(58.22221029,684.47442034)
\curveto(58.08220479,684.67441434)(57.877205,684.8144142)(57.60721029,684.89442034)
\curveto(57.55720532,684.9144141)(57.50220537,684.92441409)(57.44221029,684.92442034)
\curveto(57.39220548,684.92441409)(57.33720554,684.93441408)(57.27721029,684.95442034)
\curveto(57.22720565,684.96441405)(57.16220571,684.96441405)(57.08221029,684.95442034)
\curveto(57.01220586,684.95441406)(56.95720592,684.94941406)(56.91721029,684.93942034)
\curveto(56.877206,684.92941408)(56.84220603,684.92441409)(56.81221029,684.92442034)
\curveto(56.78220609,684.92441409)(56.75220612,684.91941409)(56.72221029,684.90942034)
\curveto(56.49220638,684.84941416)(56.30720657,684.76941424)(56.16721029,684.66942034)
\curveto(55.84720703,684.43941457)(55.65720722,684.10441491)(55.59721029,683.66442034)
\curveto(55.53720734,683.22441579)(55.50720737,682.72941628)(55.50721029,682.17942034)
\lineto(55.50721029,680.30442034)
\lineto(55.50721029,679.38942034)
\lineto(55.50721029,679.11942034)
\curveto(55.50720737,679.02941998)(55.49220738,678.95442006)(55.46221029,678.89442034)
\curveto(55.41220746,678.78442023)(55.33220754,678.71942029)(55.22221029,678.69942034)
\curveto(55.11220776,678.67942033)(54.9772079,678.66942034)(54.81721029,678.66942034)
\lineto(54.06721029,678.66942034)
\curveto(53.95720892,678.66942034)(53.84720903,678.67442034)(53.73721029,678.68442034)
\curveto(53.62720925,678.69442032)(53.54720933,678.72942028)(53.49721029,678.78942034)
\curveto(53.42720945,678.87942013)(53.39220948,679.00942)(53.39221029,679.17942034)
\curveto(53.40220947,679.34941966)(53.40720947,679.5094195)(53.40721029,679.65942034)
\lineto(53.40721029,681.69942034)
\lineto(53.40721029,684.99942034)
\lineto(53.40721029,685.76442034)
\lineto(53.40721029,686.06442034)
\curveto(53.41720946,686.15441286)(53.44720943,686.22941278)(53.49721029,686.28942034)
\curveto(53.51720936,686.31941269)(53.54720933,686.33941267)(53.58721029,686.34942034)
\curveto(53.63720924,686.36941264)(53.68720919,686.38441263)(53.73721029,686.39442034)
\lineto(53.81221029,686.39442034)
\curveto(53.86220901,686.40441261)(53.91220896,686.4094126)(53.96221029,686.40942034)
\lineto(54.12721029,686.40942034)
\lineto(54.75721029,686.40942034)
\curveto(54.83720804,686.4094126)(54.91220796,686.40441261)(54.98221029,686.39442034)
\curveto(55.06220781,686.39441262)(55.13220774,686.38441263)(55.19221029,686.36442034)
\curveto(55.26220761,686.33441268)(55.30720757,686.28941272)(55.32721029,686.22942034)
\curveto(55.35720752,686.16941284)(55.38220749,686.09941291)(55.40221029,686.01942034)
\curveto(55.41220746,685.97941303)(55.41220746,685.94441307)(55.40221029,685.91442034)
\curveto(55.40220747,685.88441313)(55.41220746,685.85441316)(55.43221029,685.82442034)
\curveto(55.45220742,685.77441324)(55.46720741,685.74441327)(55.47721029,685.73442034)
\curveto(55.49720738,685.72441329)(55.52220735,685.7094133)(55.55221029,685.68942034)
\curveto(55.66220721,685.67941333)(55.75220712,685.7144133)(55.82221029,685.79442034)
\curveto(55.89220698,685.88441313)(55.96720691,685.95441306)(56.04721029,686.00442034)
\curveto(56.31720656,686.20441281)(56.61720626,686.36441265)(56.94721029,686.48442034)
\curveto(57.03720584,686.5144125)(57.12720575,686.53441248)(57.21721029,686.54442034)
\curveto(57.31720556,686.55441246)(57.42220545,686.56941244)(57.53221029,686.58942034)
\curveto(57.56220531,686.59941241)(57.60720527,686.59941241)(57.66721029,686.58942034)
\curveto(57.72720515,686.58941242)(57.76720511,686.59441242)(57.78721029,686.60442034)
}
}
{
\newrgbcolor{curcolor}{0 0 0}
\pscustom[linestyle=none,fillstyle=solid,fillcolor=curcolor]
{
}
}
{
\newrgbcolor{curcolor}{0 0 0}
\pscustom[linestyle=none,fillstyle=solid,fillcolor=curcolor]
{
\newpath
\moveto(69.39861654,686.61942034)
\curveto(70.14861204,686.63941237)(70.79861139,686.55441246)(71.34861654,686.36442034)
\curveto(71.90861028,686.18441283)(72.33360986,685.86941314)(72.62361654,685.41942034)
\curveto(72.6936095,685.3094137)(72.75360944,685.19441382)(72.80361654,685.07442034)
\curveto(72.86360933,684.96441405)(72.91360928,684.83941417)(72.95361654,684.69942034)
\curveto(72.97360922,684.63941437)(72.98360921,684.57441444)(72.98361654,684.50442034)
\curveto(72.98360921,684.43441458)(72.97360922,684.37441464)(72.95361654,684.32442034)
\curveto(72.91360928,684.26441475)(72.85860933,684.22441479)(72.78861654,684.20442034)
\curveto(72.73860945,684.18441483)(72.67860951,684.17441484)(72.60861654,684.17442034)
\lineto(72.39861654,684.17442034)
\lineto(71.73861654,684.17442034)
\curveto(71.66861052,684.17441484)(71.59861059,684.16941484)(71.52861654,684.15942034)
\curveto(71.45861073,684.15941485)(71.3936108,684.16941484)(71.33361654,684.18942034)
\curveto(71.23361096,684.2094148)(71.15861103,684.24941476)(71.10861654,684.30942034)
\curveto(71.05861113,684.36941464)(71.01361118,684.42941458)(70.97361654,684.48942034)
\lineto(70.85361654,684.69942034)
\curveto(70.82361137,684.77941423)(70.77361142,684.84441417)(70.70361654,684.89442034)
\curveto(70.60361159,684.97441404)(70.50361169,685.03441398)(70.40361654,685.07442034)
\curveto(70.31361188,685.1144139)(70.19861199,685.14941386)(70.05861654,685.17942034)
\curveto(69.9886122,685.19941381)(69.88361231,685.2144138)(69.74361654,685.22442034)
\curveto(69.61361258,685.23441378)(69.51361268,685.22941378)(69.44361654,685.20942034)
\lineto(69.33861654,685.20942034)
\lineto(69.18861654,685.17942034)
\curveto(69.14861304,685.17941383)(69.10361309,685.17441384)(69.05361654,685.16442034)
\curveto(68.88361331,685.1144139)(68.74361345,685.04441397)(68.63361654,684.95442034)
\curveto(68.53361366,684.87441414)(68.46361373,684.74941426)(68.42361654,684.57942034)
\curveto(68.40361379,684.5094145)(68.40361379,684.44441457)(68.42361654,684.38442034)
\curveto(68.44361375,684.32441469)(68.46361373,684.27441474)(68.48361654,684.23442034)
\curveto(68.55361364,684.1144149)(68.63361356,684.01941499)(68.72361654,683.94942034)
\curveto(68.82361337,683.87941513)(68.93861325,683.81941519)(69.06861654,683.76942034)
\curveto(69.25861293,683.68941532)(69.46361273,683.61941539)(69.68361654,683.55942034)
\lineto(70.37361654,683.40942034)
\curveto(70.61361158,683.36941564)(70.84361135,683.31941569)(71.06361654,683.25942034)
\curveto(71.2936109,683.2094158)(71.50861068,683.14441587)(71.70861654,683.06442034)
\curveto(71.79861039,683.02441599)(71.88361031,682.98941602)(71.96361654,682.95942034)
\curveto(72.05361014,682.93941607)(72.13861005,682.90441611)(72.21861654,682.85442034)
\curveto(72.40860978,682.73441628)(72.57860961,682.60441641)(72.72861654,682.46442034)
\curveto(72.8886093,682.32441669)(73.01360918,682.14941686)(73.10361654,681.93942034)
\curveto(73.13360906,681.86941714)(73.15860903,681.79941721)(73.17861654,681.72942034)
\curveto(73.19860899,681.65941735)(73.21860897,681.58441743)(73.23861654,681.50442034)
\curveto(73.24860894,681.44441757)(73.25360894,681.34941766)(73.25361654,681.21942034)
\curveto(73.26360893,681.09941791)(73.26360893,681.00441801)(73.25361654,680.93442034)
\lineto(73.25361654,680.85942034)
\curveto(73.23360896,680.79941821)(73.21860897,680.73941827)(73.20861654,680.67942034)
\curveto(73.20860898,680.62941838)(73.20360899,680.57941843)(73.19361654,680.52942034)
\curveto(73.12360907,680.22941878)(73.01360918,679.96441905)(72.86361654,679.73442034)
\curveto(72.70360949,679.49441952)(72.50860968,679.29941971)(72.27861654,679.14942034)
\curveto(72.04861014,678.99942001)(71.7886104,678.86942014)(71.49861654,678.75942034)
\curveto(71.3886108,678.7094203)(71.26861092,678.67442034)(71.13861654,678.65442034)
\curveto(71.01861117,678.63442038)(70.89861129,678.6094204)(70.77861654,678.57942034)
\curveto(70.6886115,678.55942045)(70.5936116,678.54942046)(70.49361654,678.54942034)
\curveto(70.40361179,678.53942047)(70.31361188,678.52442049)(70.22361654,678.50442034)
\lineto(69.95361654,678.50442034)
\curveto(69.8936123,678.48442053)(69.7886124,678.47442054)(69.63861654,678.47442034)
\curveto(69.49861269,678.47442054)(69.39861279,678.48442053)(69.33861654,678.50442034)
\curveto(69.30861288,678.50442051)(69.27361292,678.5094205)(69.23361654,678.51942034)
\lineto(69.12861654,678.51942034)
\curveto(69.00861318,678.53942047)(68.8886133,678.55442046)(68.76861654,678.56442034)
\curveto(68.64861354,678.57442044)(68.53361366,678.59442042)(68.42361654,678.62442034)
\curveto(68.03361416,678.73442028)(67.6886145,678.85942015)(67.38861654,678.99942034)
\curveto(67.0886151,679.14941986)(66.83361536,679.36941964)(66.62361654,679.65942034)
\curveto(66.48361571,679.84941916)(66.36361583,680.06941894)(66.26361654,680.31942034)
\curveto(66.24361595,680.37941863)(66.22361597,680.45941855)(66.20361654,680.55942034)
\curveto(66.18361601,680.6094184)(66.16861602,680.67941833)(66.15861654,680.76942034)
\curveto(66.14861604,680.85941815)(66.15361604,680.93441808)(66.17361654,680.99442034)
\curveto(66.20361599,681.06441795)(66.25361594,681.1144179)(66.32361654,681.14442034)
\curveto(66.37361582,681.16441785)(66.43361576,681.17441784)(66.50361654,681.17442034)
\lineto(66.72861654,681.17442034)
\lineto(67.43361654,681.17442034)
\lineto(67.67361654,681.17442034)
\curveto(67.75361444,681.17441784)(67.82361437,681.16441785)(67.88361654,681.14442034)
\curveto(67.9936142,681.10441791)(68.06361413,681.03941797)(68.09361654,680.94942034)
\curveto(68.13361406,680.85941815)(68.17861401,680.76441825)(68.22861654,680.66442034)
\curveto(68.24861394,680.6144184)(68.28361391,680.54941846)(68.33361654,680.46942034)
\curveto(68.3936138,680.38941862)(68.44361375,680.33941867)(68.48361654,680.31942034)
\curveto(68.60361359,680.21941879)(68.71861347,680.13941887)(68.82861654,680.07942034)
\curveto(68.93861325,680.02941898)(69.07861311,679.97941903)(69.24861654,679.92942034)
\curveto(69.29861289,679.9094191)(69.34861284,679.89941911)(69.39861654,679.89942034)
\curveto(69.44861274,679.9094191)(69.49861269,679.9094191)(69.54861654,679.89942034)
\curveto(69.62861256,679.87941913)(69.71361248,679.86941914)(69.80361654,679.86942034)
\curveto(69.90361229,679.87941913)(69.9886122,679.89441912)(70.05861654,679.91442034)
\curveto(70.10861208,679.92441909)(70.15361204,679.92941908)(70.19361654,679.92942034)
\curveto(70.24361195,679.92941908)(70.2936119,679.93941907)(70.34361654,679.95942034)
\curveto(70.48361171,680.009419)(70.60861158,680.06941894)(70.71861654,680.13942034)
\curveto(70.83861135,680.2094188)(70.93361126,680.29941871)(71.00361654,680.40942034)
\curveto(71.05361114,680.48941852)(71.0936111,680.6144184)(71.12361654,680.78442034)
\curveto(71.14361105,680.85441816)(71.14361105,680.91941809)(71.12361654,680.97942034)
\curveto(71.10361109,681.03941797)(71.08361111,681.08941792)(71.06361654,681.12942034)
\curveto(70.9936112,681.26941774)(70.90361129,681.37441764)(70.79361654,681.44442034)
\curveto(70.6936115,681.5144175)(70.57361162,681.57941743)(70.43361654,681.63942034)
\curveto(70.24361195,681.71941729)(70.04361215,681.78441723)(69.83361654,681.83442034)
\curveto(69.62361257,681.88441713)(69.41361278,681.93941707)(69.20361654,681.99942034)
\curveto(69.12361307,682.01941699)(69.03861315,682.03441698)(68.94861654,682.04442034)
\curveto(68.86861332,682.05441696)(68.7886134,682.06941694)(68.70861654,682.08942034)
\curveto(68.3886138,682.17941683)(68.08361411,682.26441675)(67.79361654,682.34442034)
\curveto(67.50361469,682.43441658)(67.23861495,682.56441645)(66.99861654,682.73442034)
\curveto(66.71861547,682.93441608)(66.51361568,683.20441581)(66.38361654,683.54442034)
\curveto(66.36361583,683.6144154)(66.34361585,683.7094153)(66.32361654,683.82942034)
\curveto(66.30361589,683.89941511)(66.2886159,683.98441503)(66.27861654,684.08442034)
\curveto(66.26861592,684.18441483)(66.27361592,684.27441474)(66.29361654,684.35442034)
\curveto(66.31361588,684.40441461)(66.31861587,684.44441457)(66.30861654,684.47442034)
\curveto(66.29861589,684.5144145)(66.30361589,684.55941445)(66.32361654,684.60942034)
\curveto(66.34361585,684.71941429)(66.36361583,684.81941419)(66.38361654,684.90942034)
\curveto(66.41361578,685.009414)(66.44861574,685.10441391)(66.48861654,685.19442034)
\curveto(66.61861557,685.48441353)(66.79861539,685.71941329)(67.02861654,685.89942034)
\curveto(67.25861493,686.07941293)(67.51861467,686.22441279)(67.80861654,686.33442034)
\curveto(67.91861427,686.38441263)(68.03361416,686.41941259)(68.15361654,686.43942034)
\curveto(68.27361392,686.46941254)(68.39861379,686.49941251)(68.52861654,686.52942034)
\curveto(68.5886136,686.54941246)(68.64861354,686.55941245)(68.70861654,686.55942034)
\lineto(68.88861654,686.58942034)
\curveto(68.96861322,686.59941241)(69.05361314,686.60441241)(69.14361654,686.60442034)
\curveto(69.23361296,686.60441241)(69.31861287,686.6094124)(69.39861654,686.61942034)
}
}
{
\newrgbcolor{curcolor}{0 0 0}
\pscustom[linestyle=none,fillstyle=solid,fillcolor=curcolor]
{
\newpath
\moveto(74.90525717,686.39442034)
\lineto(76.03025717,686.39442034)
\curveto(76.14025473,686.39441262)(76.24025463,686.38941262)(76.33025717,686.37942034)
\curveto(76.42025445,686.36941264)(76.48525439,686.33441268)(76.52525717,686.27442034)
\curveto(76.5752543,686.2144128)(76.60525427,686.12941288)(76.61525717,686.01942034)
\curveto(76.62525425,685.91941309)(76.63025424,685.8144132)(76.63025717,685.70442034)
\lineto(76.63025717,684.65442034)
\lineto(76.63025717,682.41942034)
\curveto(76.63025424,682.05941695)(76.64525423,681.71941729)(76.67525717,681.39942034)
\curveto(76.70525417,681.07941793)(76.79525408,680.8144182)(76.94525717,680.60442034)
\curveto(77.08525379,680.39441862)(77.31025356,680.24441877)(77.62025717,680.15442034)
\curveto(77.6702532,680.14441887)(77.71025316,680.13941887)(77.74025717,680.13942034)
\curveto(77.78025309,680.13941887)(77.82525305,680.13441888)(77.87525717,680.12442034)
\curveto(77.92525295,680.1144189)(77.98025289,680.1094189)(78.04025717,680.10942034)
\curveto(78.10025277,680.1094189)(78.14525273,680.1144189)(78.17525717,680.12442034)
\curveto(78.22525265,680.14441887)(78.26525261,680.14941886)(78.29525717,680.13942034)
\curveto(78.33525254,680.12941888)(78.3752525,680.13441888)(78.41525717,680.15442034)
\curveto(78.62525225,680.20441881)(78.79025208,680.26941874)(78.91025717,680.34942034)
\curveto(79.09025178,680.45941855)(79.23025164,680.59941841)(79.33025717,680.76942034)
\curveto(79.44025143,680.94941806)(79.51525136,681.14441787)(79.55525717,681.35442034)
\curveto(79.60525127,681.57441744)(79.63525124,681.8144172)(79.64525717,682.07442034)
\curveto(79.65525122,682.34441667)(79.66025121,682.62441639)(79.66025717,682.91442034)
\lineto(79.66025717,684.72942034)
\lineto(79.66025717,685.70442034)
\lineto(79.66025717,685.97442034)
\curveto(79.66025121,686.07441294)(79.68025119,686.15441286)(79.72025717,686.21442034)
\curveto(79.7702511,686.30441271)(79.84525103,686.35441266)(79.94525717,686.36442034)
\curveto(80.04525083,686.38441263)(80.16525071,686.39441262)(80.30525717,686.39442034)
\lineto(81.10025717,686.39442034)
\lineto(81.38525717,686.39442034)
\curveto(81.4752494,686.39441262)(81.55024932,686.37441264)(81.61025717,686.33442034)
\curveto(81.69024918,686.28441273)(81.73524914,686.2094128)(81.74525717,686.10942034)
\curveto(81.75524912,686.009413)(81.76024911,685.89441312)(81.76025717,685.76442034)
\lineto(81.76025717,684.62442034)
\lineto(81.76025717,680.40942034)
\lineto(81.76025717,679.34442034)
\lineto(81.76025717,679.04442034)
\curveto(81.76024911,678.94442007)(81.74024913,678.86942014)(81.70025717,678.81942034)
\curveto(81.65024922,678.73942027)(81.5752493,678.69442032)(81.47525717,678.68442034)
\curveto(81.3752495,678.67442034)(81.2702496,678.66942034)(81.16025717,678.66942034)
\lineto(80.35025717,678.66942034)
\curveto(80.24025063,678.66942034)(80.14025073,678.67442034)(80.05025717,678.68442034)
\curveto(79.9702509,678.69442032)(79.90525097,678.73442028)(79.85525717,678.80442034)
\curveto(79.83525104,678.83442018)(79.81525106,678.87942013)(79.79525717,678.93942034)
\curveto(79.78525109,678.99942001)(79.7702511,679.05941995)(79.75025717,679.11942034)
\curveto(79.74025113,679.17941983)(79.72525115,679.23441978)(79.70525717,679.28442034)
\curveto(79.68525119,679.33441968)(79.65525122,679.36441965)(79.61525717,679.37442034)
\curveto(79.59525128,679.39441962)(79.5702513,679.39941961)(79.54025717,679.38942034)
\curveto(79.51025136,679.37941963)(79.48525139,679.36941964)(79.46525717,679.35942034)
\curveto(79.39525148,679.31941969)(79.33525154,679.27441974)(79.28525717,679.22442034)
\curveto(79.23525164,679.17441984)(79.18025169,679.12941988)(79.12025717,679.08942034)
\curveto(79.08025179,679.05941995)(79.04025183,679.02441999)(79.00025717,678.98442034)
\curveto(78.9702519,678.95442006)(78.93025194,678.92442009)(78.88025717,678.89442034)
\curveto(78.65025222,678.75442026)(78.38025249,678.64442037)(78.07025717,678.56442034)
\curveto(78.00025287,678.54442047)(77.93025294,678.53442048)(77.86025717,678.53442034)
\curveto(77.79025308,678.52442049)(77.71525316,678.5094205)(77.63525717,678.48942034)
\curveto(77.59525328,678.47942053)(77.55025332,678.47942053)(77.50025717,678.48942034)
\curveto(77.46025341,678.48942052)(77.42025345,678.48442053)(77.38025717,678.47442034)
\curveto(77.35025352,678.46442055)(77.28525359,678.46442055)(77.18525717,678.47442034)
\curveto(77.09525378,678.47442054)(77.03525384,678.47942053)(77.00525717,678.48942034)
\curveto(76.95525392,678.48942052)(76.90525397,678.49442052)(76.85525717,678.50442034)
\lineto(76.70525717,678.50442034)
\curveto(76.58525429,678.53442048)(76.4702544,678.55942045)(76.36025717,678.57942034)
\curveto(76.25025462,678.59942041)(76.14025473,678.62942038)(76.03025717,678.66942034)
\curveto(75.98025489,678.68942032)(75.93525494,678.70442031)(75.89525717,678.71442034)
\curveto(75.86525501,678.73442028)(75.82525505,678.75442026)(75.77525717,678.77442034)
\curveto(75.42525545,678.96442005)(75.14525573,679.22941978)(74.93525717,679.56942034)
\curveto(74.80525607,679.77941923)(74.71025616,680.02941898)(74.65025717,680.31942034)
\curveto(74.59025628,680.61941839)(74.55025632,680.93441808)(74.53025717,681.26442034)
\curveto(74.52025635,681.60441741)(74.51525636,681.94941706)(74.51525717,682.29942034)
\curveto(74.52525635,682.65941635)(74.53025634,683.014416)(74.53025717,683.36442034)
\lineto(74.53025717,685.40442034)
\curveto(74.53025634,685.53441348)(74.52525635,685.68441333)(74.51525717,685.85442034)
\curveto(74.51525636,686.03441298)(74.54025633,686.16441285)(74.59025717,686.24442034)
\curveto(74.62025625,686.29441272)(74.68025619,686.33941267)(74.77025717,686.37942034)
\curveto(74.83025604,686.37941263)(74.875256,686.38441263)(74.90525717,686.39442034)
}
}
{
\newrgbcolor{curcolor}{0 0 0}
\pscustom[linestyle=none,fillstyle=solid,fillcolor=curcolor]
{
}
}
{
\newrgbcolor{curcolor}{0 0 0}
\pscustom[linestyle=none,fillstyle=solid,fillcolor=curcolor]
{
\newpath
\moveto(88.60166342,688.71942034)
\lineto(89.60666342,688.71942034)
\curveto(89.75666043,688.71941029)(89.8866603,688.7094103)(89.99666342,688.68942034)
\curveto(90.11666007,688.67941033)(90.20165999,688.61941039)(90.25166342,688.50942034)
\curveto(90.27165992,688.45941055)(90.28165991,688.39941061)(90.28166342,688.32942034)
\lineto(90.28166342,688.11942034)
\lineto(90.28166342,687.44442034)
\curveto(90.28165991,687.39441162)(90.27665991,687.33441168)(90.26666342,687.26442034)
\curveto(90.26665992,687.20441181)(90.27165992,687.14941186)(90.28166342,687.09942034)
\lineto(90.28166342,686.93442034)
\curveto(90.28165991,686.85441216)(90.2866599,686.77941223)(90.29666342,686.70942034)
\curveto(90.30665988,686.64941236)(90.33165986,686.59441242)(90.37166342,686.54442034)
\curveto(90.44165975,686.45441256)(90.56665962,686.40441261)(90.74666342,686.39442034)
\lineto(91.28666342,686.39442034)
\lineto(91.46666342,686.39442034)
\curveto(91.52665866,686.39441262)(91.58165861,686.38441263)(91.63166342,686.36442034)
\curveto(91.74165845,686.3144127)(91.80165839,686.22441279)(91.81166342,686.09442034)
\curveto(91.83165836,685.96441305)(91.84165835,685.81941319)(91.84166342,685.65942034)
\lineto(91.84166342,685.44942034)
\curveto(91.85165834,685.37941363)(91.84665834,685.31941369)(91.82666342,685.26942034)
\curveto(91.77665841,685.1094139)(91.67165852,685.02441399)(91.51166342,685.01442034)
\curveto(91.35165884,685.00441401)(91.17165902,684.99941401)(90.97166342,684.99942034)
\lineto(90.83666342,684.99942034)
\curveto(90.79665939,685.009414)(90.76165943,685.009414)(90.73166342,684.99942034)
\curveto(90.6916595,684.98941402)(90.65665953,684.98441403)(90.62666342,684.98442034)
\curveto(90.59665959,684.99441402)(90.56665962,684.98941402)(90.53666342,684.96942034)
\curveto(90.45665973,684.94941406)(90.39665979,684.90441411)(90.35666342,684.83442034)
\curveto(90.32665986,684.77441424)(90.30165989,684.69941431)(90.28166342,684.60942034)
\curveto(90.27165992,684.55941445)(90.27165992,684.50441451)(90.28166342,684.44442034)
\curveto(90.2916599,684.38441463)(90.2916599,684.32941468)(90.28166342,684.27942034)
\lineto(90.28166342,683.34942034)
\lineto(90.28166342,681.59442034)
\curveto(90.28165991,681.34441767)(90.2866599,681.12441789)(90.29666342,680.93442034)
\curveto(90.31665987,680.75441826)(90.38165981,680.59441842)(90.49166342,680.45442034)
\curveto(90.54165965,680.39441862)(90.60665958,680.34941866)(90.68666342,680.31942034)
\lineto(90.95666342,680.25942034)
\curveto(90.9866592,680.24941876)(91.01665917,680.24441877)(91.04666342,680.24442034)
\curveto(91.0866591,680.25441876)(91.11665907,680.25441876)(91.13666342,680.24442034)
\lineto(91.30166342,680.24442034)
\curveto(91.41165878,680.24441877)(91.50665868,680.23941877)(91.58666342,680.22942034)
\curveto(91.66665852,680.21941879)(91.73165846,680.17941883)(91.78166342,680.10942034)
\curveto(91.82165837,680.04941896)(91.84165835,679.96941904)(91.84166342,679.86942034)
\lineto(91.84166342,679.58442034)
\curveto(91.84165835,679.37441964)(91.83665835,679.17941983)(91.82666342,678.99942034)
\curveto(91.82665836,678.82942018)(91.74665844,678.7144203)(91.58666342,678.65442034)
\curveto(91.53665865,678.63442038)(91.4916587,678.62942038)(91.45166342,678.63942034)
\curveto(91.41165878,678.63942037)(91.36665882,678.62942038)(91.31666342,678.60942034)
\lineto(91.16666342,678.60942034)
\curveto(91.14665904,678.6094204)(91.11665907,678.6144204)(91.07666342,678.62442034)
\curveto(91.03665915,678.62442039)(91.00165919,678.61942039)(90.97166342,678.60942034)
\curveto(90.92165927,678.59942041)(90.86665932,678.59942041)(90.80666342,678.60942034)
\lineto(90.65666342,678.60942034)
\lineto(90.50666342,678.60942034)
\curveto(90.45665973,678.59942041)(90.41165978,678.59942041)(90.37166342,678.60942034)
\lineto(90.20666342,678.60942034)
\curveto(90.15666003,678.61942039)(90.10166009,678.62442039)(90.04166342,678.62442034)
\curveto(89.98166021,678.62442039)(89.92666026,678.62942038)(89.87666342,678.63942034)
\curveto(89.80666038,678.64942036)(89.74166045,678.65942035)(89.68166342,678.66942034)
\lineto(89.50166342,678.69942034)
\curveto(89.3916608,678.72942028)(89.2866609,678.76442025)(89.18666342,678.80442034)
\curveto(89.0866611,678.84442017)(88.9916612,678.88942012)(88.90166342,678.93942034)
\lineto(88.81166342,678.99942034)
\curveto(88.78166141,679.02941998)(88.74666144,679.05941995)(88.70666342,679.08942034)
\curveto(88.6866615,679.1094199)(88.66166153,679.12941988)(88.63166342,679.14942034)
\lineto(88.55666342,679.22442034)
\curveto(88.41666177,679.4144196)(88.31166188,679.62441939)(88.24166342,679.85442034)
\curveto(88.22166197,679.89441912)(88.21166198,679.92941908)(88.21166342,679.95942034)
\curveto(88.22166197,679.99941901)(88.22166197,680.04441897)(88.21166342,680.09442034)
\curveto(88.20166199,680.1144189)(88.19666199,680.13941887)(88.19666342,680.16942034)
\curveto(88.19666199,680.19941881)(88.191662,680.22441879)(88.18166342,680.24442034)
\lineto(88.18166342,680.39442034)
\curveto(88.17166202,680.43441858)(88.16666202,680.47941853)(88.16666342,680.52942034)
\curveto(88.17666201,680.57941843)(88.18166201,680.62941838)(88.18166342,680.67942034)
\lineto(88.18166342,681.24942034)
\lineto(88.18166342,683.48442034)
\lineto(88.18166342,684.27942034)
\lineto(88.18166342,684.48942034)
\curveto(88.191662,684.55941445)(88.186662,684.62441439)(88.16666342,684.68442034)
\curveto(88.12666206,684.82441419)(88.05666213,684.9144141)(87.95666342,684.95442034)
\curveto(87.84666234,685.00441401)(87.70666248,685.01941399)(87.53666342,684.99942034)
\curveto(87.36666282,684.97941403)(87.22166297,684.99441402)(87.10166342,685.04442034)
\curveto(87.02166317,685.07441394)(86.97166322,685.11941389)(86.95166342,685.17942034)
\curveto(86.93166326,685.23941377)(86.91166328,685.3144137)(86.89166342,685.40442034)
\lineto(86.89166342,685.71942034)
\curveto(86.8916633,685.89941311)(86.90166329,686.04441297)(86.92166342,686.15442034)
\curveto(86.94166325,686.26441275)(87.02666316,686.33941267)(87.17666342,686.37942034)
\curveto(87.21666297,686.39941261)(87.25666293,686.40441261)(87.29666342,686.39442034)
\lineto(87.43166342,686.39442034)
\curveto(87.58166261,686.39441262)(87.72166247,686.39941261)(87.85166342,686.40942034)
\curveto(87.98166221,686.42941258)(88.07166212,686.48941252)(88.12166342,686.58942034)
\curveto(88.15166204,686.65941235)(88.16666202,686.73941227)(88.16666342,686.82942034)
\curveto(88.17666201,686.91941209)(88.18166201,687.009412)(88.18166342,687.09942034)
\lineto(88.18166342,688.02942034)
\lineto(88.18166342,688.28442034)
\curveto(88.18166201,688.37441064)(88.191662,688.44941056)(88.21166342,688.50942034)
\curveto(88.26166193,688.6094104)(88.33666185,688.67441034)(88.43666342,688.70442034)
\curveto(88.45666173,688.7144103)(88.48166171,688.7144103)(88.51166342,688.70442034)
\curveto(88.55166164,688.70441031)(88.58166161,688.7094103)(88.60166342,688.71942034)
}
}
{
\newrgbcolor{curcolor}{0 0 0}
\pscustom[linestyle=none,fillstyle=solid,fillcolor=curcolor]
{
\newpath
\moveto(94.92510092,689.25942034)
\curveto(94.99509797,689.17940983)(95.03009793,689.05940995)(95.03010092,688.89942034)
\lineto(95.03010092,688.43442034)
\lineto(95.03010092,688.02942034)
\curveto(95.03009793,687.88941112)(94.99509797,687.79441122)(94.92510092,687.74442034)
\curveto(94.8650981,687.69441132)(94.78509818,687.66441135)(94.68510092,687.65442034)
\curveto(94.59509837,687.64441137)(94.49509847,687.63941137)(94.38510092,687.63942034)
\lineto(93.54510092,687.63942034)
\curveto(93.43509953,687.63941137)(93.33509963,687.64441137)(93.24510092,687.65442034)
\curveto(93.1650998,687.66441135)(93.09509987,687.69441132)(93.03510092,687.74442034)
\curveto(92.99509997,687.77441124)(92.9651,687.82941118)(92.94510092,687.90942034)
\curveto(92.93510003,687.99941101)(92.92510004,688.09441092)(92.91510092,688.19442034)
\lineto(92.91510092,688.52442034)
\curveto(92.92510004,688.63441038)(92.93010003,688.72941028)(92.93010092,688.80942034)
\lineto(92.93010092,689.01942034)
\curveto(92.94010002,689.08940992)(92.9601,689.14940986)(92.99010092,689.19942034)
\curveto(93.01009995,689.23940977)(93.03509993,689.26940974)(93.06510092,689.28942034)
\lineto(93.18510092,689.34942034)
\curveto(93.20509976,689.34940966)(93.23009973,689.34940966)(93.26010092,689.34942034)
\curveto(93.29009967,689.35940965)(93.31509965,689.36440965)(93.33510092,689.36442034)
\lineto(94.43010092,689.36442034)
\curveto(94.53009843,689.36440965)(94.62509834,689.35940965)(94.71510092,689.34942034)
\curveto(94.80509816,689.33940967)(94.87509809,689.3094097)(94.92510092,689.25942034)
\moveto(95.03010092,679.49442034)
\curveto(95.03009793,679.29441972)(95.02509794,679.12441989)(95.01510092,678.98442034)
\curveto(95.00509796,678.84442017)(94.91509805,678.74942026)(94.74510092,678.69942034)
\curveto(94.68509828,678.67942033)(94.62009834,678.66942034)(94.55010092,678.66942034)
\curveto(94.48009848,678.67942033)(94.40509856,678.68442033)(94.32510092,678.68442034)
\lineto(93.48510092,678.68442034)
\curveto(93.39509957,678.68442033)(93.30509966,678.68942032)(93.21510092,678.69942034)
\curveto(93.13509983,678.7094203)(93.07509989,678.73942027)(93.03510092,678.78942034)
\curveto(92.97509999,678.85942015)(92.94010002,678.94442007)(92.93010092,679.04442034)
\lineto(92.93010092,679.38942034)
\lineto(92.93010092,685.71942034)
\lineto(92.93010092,686.01942034)
\curveto(92.93010003,686.11941289)(92.95010001,686.19941281)(92.99010092,686.25942034)
\curveto(93.05009991,686.32941268)(93.13509983,686.37441264)(93.24510092,686.39442034)
\curveto(93.2650997,686.40441261)(93.29009967,686.40441261)(93.32010092,686.39442034)
\curveto(93.3600996,686.39441262)(93.39009957,686.39941261)(93.41010092,686.40942034)
\lineto(94.16010092,686.40942034)
\lineto(94.35510092,686.40942034)
\curveto(94.43509853,686.41941259)(94.50009846,686.41941259)(94.55010092,686.40942034)
\lineto(94.67010092,686.40942034)
\curveto(94.73009823,686.38941262)(94.78509818,686.37441264)(94.83510092,686.36442034)
\curveto(94.88509808,686.35441266)(94.92509804,686.32441269)(94.95510092,686.27442034)
\curveto(94.99509797,686.22441279)(95.01509795,686.15441286)(95.01510092,686.06442034)
\curveto(95.02509794,685.97441304)(95.03009793,685.87941313)(95.03010092,685.77942034)
\lineto(95.03010092,679.49442034)
}
}
{
\newrgbcolor{curcolor}{0 0 0}
\pscustom[linestyle=none,fillstyle=solid,fillcolor=curcolor]
{
\newpath
\moveto(104.58228842,682.62942034)
\curveto(104.59227974,682.56941644)(104.59727973,682.47941653)(104.59728842,682.35942034)
\curveto(104.59727973,682.23941677)(104.58727974,682.15441686)(104.56728842,682.10442034)
\lineto(104.56728842,681.90942034)
\curveto(104.53727979,681.79941721)(104.51727981,681.69441732)(104.50728842,681.59442034)
\curveto(104.50727982,681.49441752)(104.49227984,681.39441762)(104.46228842,681.29442034)
\curveto(104.44227989,681.20441781)(104.42227991,681.1094179)(104.40228842,681.00942034)
\curveto(104.38227995,680.91941809)(104.35227998,680.82941818)(104.31228842,680.73942034)
\curveto(104.24228009,680.56941844)(104.17228016,680.4094186)(104.10228842,680.25942034)
\curveto(104.0322803,680.11941889)(103.95228038,679.97941903)(103.86228842,679.83942034)
\curveto(103.80228053,679.74941926)(103.73728059,679.66441935)(103.66728842,679.58442034)
\curveto(103.60728072,679.5144195)(103.53728079,679.43941957)(103.45728842,679.35942034)
\lineto(103.35228842,679.25442034)
\curveto(103.30228103,679.20441981)(103.24728108,679.15941985)(103.18728842,679.11942034)
\lineto(103.03728842,678.99942034)
\curveto(102.95728137,678.93942007)(102.86728146,678.88442013)(102.76728842,678.83442034)
\curveto(102.67728165,678.79442022)(102.58228175,678.74942026)(102.48228842,678.69942034)
\curveto(102.38228195,678.64942036)(102.27728205,678.6144204)(102.16728842,678.59442034)
\curveto(102.06728226,678.57442044)(101.96228237,678.55442046)(101.85228842,678.53442034)
\curveto(101.79228254,678.5144205)(101.7272826,678.50442051)(101.65728842,678.50442034)
\curveto(101.59728273,678.50442051)(101.5322828,678.49442052)(101.46228842,678.47442034)
\lineto(101.32728842,678.47442034)
\curveto(101.24728308,678.45442056)(101.17228316,678.45442056)(101.10228842,678.47442034)
\lineto(100.95228842,678.47442034)
\curveto(100.89228344,678.49442052)(100.8272835,678.50442051)(100.75728842,678.50442034)
\curveto(100.69728363,678.49442052)(100.63728369,678.49942051)(100.57728842,678.51942034)
\curveto(100.41728391,678.56942044)(100.26228407,678.6144204)(100.11228842,678.65442034)
\curveto(99.97228436,678.69442032)(99.84228449,678.75442026)(99.72228842,678.83442034)
\curveto(99.65228468,678.87442014)(99.58728474,678.9144201)(99.52728842,678.95442034)
\curveto(99.46728486,679.00442001)(99.40228493,679.05441996)(99.33228842,679.10442034)
\lineto(99.15228842,679.23942034)
\curveto(99.07228526,679.29941971)(99.00228533,679.30441971)(98.94228842,679.25442034)
\curveto(98.89228544,679.22441979)(98.86728546,679.18441983)(98.86728842,679.13442034)
\curveto(98.86728546,679.09441992)(98.85728547,679.04441997)(98.83728842,678.98442034)
\curveto(98.81728551,678.88442013)(98.80728552,678.76942024)(98.80728842,678.63942034)
\curveto(98.81728551,678.5094205)(98.82228551,678.38942062)(98.82228842,678.27942034)
\lineto(98.82228842,676.74942034)
\curveto(98.82228551,676.61942239)(98.81728551,676.49442252)(98.80728842,676.37442034)
\curveto(98.80728552,676.24442277)(98.78228555,676.13942287)(98.73228842,676.05942034)
\curveto(98.70228563,676.01942299)(98.64728568,675.98942302)(98.56728842,675.96942034)
\curveto(98.48728584,675.94942306)(98.39728593,675.93942307)(98.29728842,675.93942034)
\curveto(98.19728613,675.92942308)(98.09728623,675.92942308)(97.99728842,675.93942034)
\lineto(97.74228842,675.93942034)
\lineto(97.33728842,675.93942034)
\lineto(97.23228842,675.93942034)
\curveto(97.19228714,675.93942307)(97.15728717,675.94442307)(97.12728842,675.95442034)
\lineto(97.00728842,675.95442034)
\curveto(96.83728749,676.00442301)(96.74728758,676.10442291)(96.73728842,676.25442034)
\curveto(96.7272876,676.39442262)(96.72228761,676.56442245)(96.72228842,676.76442034)
\lineto(96.72228842,685.56942034)
\curveto(96.72228761,685.67941333)(96.71728761,685.79441322)(96.70728842,685.91442034)
\curveto(96.70728762,686.04441297)(96.7322876,686.14441287)(96.78228842,686.21442034)
\curveto(96.82228751,686.28441273)(96.87728745,686.32941268)(96.94728842,686.34942034)
\curveto(96.99728733,686.36941264)(97.05728727,686.37941263)(97.12728842,686.37942034)
\lineto(97.35228842,686.37942034)
\lineto(98.07228842,686.37942034)
\lineto(98.35728842,686.37942034)
\curveto(98.44728588,686.37941263)(98.52228581,686.35441266)(98.58228842,686.30442034)
\curveto(98.65228568,686.25441276)(98.68728564,686.18941282)(98.68728842,686.10942034)
\curveto(98.69728563,686.03941297)(98.72228561,685.96441305)(98.76228842,685.88442034)
\curveto(98.77228556,685.85441316)(98.78228555,685.82941318)(98.79228842,685.80942034)
\curveto(98.81228552,685.79941321)(98.8322855,685.78441323)(98.85228842,685.76442034)
\curveto(98.96228537,685.75441326)(99.05228528,685.78441323)(99.12228842,685.85442034)
\curveto(99.19228514,685.92441309)(99.26228507,685.98441303)(99.33228842,686.03442034)
\curveto(99.46228487,686.12441289)(99.59728473,686.20441281)(99.73728842,686.27442034)
\curveto(99.87728445,686.35441266)(100.0322843,686.41941259)(100.20228842,686.46942034)
\curveto(100.28228405,686.49941251)(100.36728396,686.51941249)(100.45728842,686.52942034)
\curveto(100.55728377,686.53941247)(100.65228368,686.55441246)(100.74228842,686.57442034)
\curveto(100.78228355,686.58441243)(100.82228351,686.58441243)(100.86228842,686.57442034)
\curveto(100.91228342,686.56441245)(100.95228338,686.56941244)(100.98228842,686.58942034)
\curveto(101.55228278,686.6094124)(102.0322823,686.52941248)(102.42228842,686.34942034)
\curveto(102.82228151,686.17941283)(103.16228117,685.95441306)(103.44228842,685.67442034)
\curveto(103.49228084,685.62441339)(103.53728079,685.57441344)(103.57728842,685.52442034)
\curveto(103.61728071,685.48441353)(103.65728067,685.43941357)(103.69728842,685.38942034)
\curveto(103.76728056,685.29941371)(103.8272805,685.2094138)(103.87728842,685.11942034)
\curveto(103.93728039,685.02941398)(103.99228034,684.93941407)(104.04228842,684.84942034)
\curveto(104.06228027,684.82941418)(104.07228026,684.80441421)(104.07228842,684.77442034)
\curveto(104.08228025,684.74441427)(104.09728023,684.7094143)(104.11728842,684.66942034)
\curveto(104.17728015,684.56941444)(104.2322801,684.44941456)(104.28228842,684.30942034)
\curveto(104.30228003,684.24941476)(104.32228001,684.18441483)(104.34228842,684.11442034)
\curveto(104.36227997,684.05441496)(104.38227995,683.98941502)(104.40228842,683.91942034)
\curveto(104.44227989,683.79941521)(104.46727986,683.67441534)(104.47728842,683.54442034)
\curveto(104.49727983,683.4144156)(104.52227981,683.27941573)(104.55228842,683.13942034)
\lineto(104.55228842,682.97442034)
\lineto(104.58228842,682.79442034)
\lineto(104.58228842,682.62942034)
\moveto(102.46728842,682.28442034)
\curveto(102.47728185,682.33441668)(102.48228185,682.39941661)(102.48228842,682.47942034)
\curveto(102.48228185,682.56941644)(102.47728185,682.63941637)(102.46728842,682.68942034)
\lineto(102.46728842,682.82442034)
\curveto(102.44728188,682.88441613)(102.43728189,682.94941606)(102.43728842,683.01942034)
\curveto(102.43728189,683.08941592)(102.4272819,683.15941585)(102.40728842,683.22942034)
\curveto(102.38728194,683.32941568)(102.36728196,683.42441559)(102.34728842,683.51442034)
\curveto(102.327282,683.6144154)(102.29728203,683.70441531)(102.25728842,683.78442034)
\curveto(102.13728219,684.10441491)(101.98228235,684.35941465)(101.79228842,684.54942034)
\curveto(101.60228273,684.73941427)(101.332283,684.87941413)(100.98228842,684.96942034)
\curveto(100.90228343,684.98941402)(100.81228352,684.99941401)(100.71228842,684.99942034)
\lineto(100.44228842,684.99942034)
\curveto(100.40228393,684.98941402)(100.36728396,684.98441403)(100.33728842,684.98442034)
\curveto(100.30728402,684.98441403)(100.27228406,684.97941403)(100.23228842,684.96942034)
\lineto(100.02228842,684.90942034)
\curveto(99.96228437,684.89941411)(99.90228443,684.87941413)(99.84228842,684.84942034)
\curveto(99.58228475,684.73941427)(99.37728495,684.56941444)(99.22728842,684.33942034)
\curveto(99.08728524,684.1094149)(98.97228536,683.85441516)(98.88228842,683.57442034)
\curveto(98.86228547,683.49441552)(98.84728548,683.4094156)(98.83728842,683.31942034)
\curveto(98.8272855,683.23941577)(98.81228552,683.15941585)(98.79228842,683.07942034)
\curveto(98.78228555,683.03941597)(98.77728555,682.97441604)(98.77728842,682.88442034)
\curveto(98.75728557,682.84441617)(98.75228558,682.79441622)(98.76228842,682.73442034)
\curveto(98.77228556,682.68441633)(98.77228556,682.63441638)(98.76228842,682.58442034)
\curveto(98.74228559,682.52441649)(98.74228559,682.46941654)(98.76228842,682.41942034)
\lineto(98.76228842,682.23942034)
\lineto(98.76228842,682.10442034)
\curveto(98.76228557,682.06441695)(98.77228556,682.02441699)(98.79228842,681.98442034)
\curveto(98.79228554,681.9144171)(98.79728553,681.85941715)(98.80728842,681.81942034)
\lineto(98.83728842,681.63942034)
\curveto(98.84728548,681.57941743)(98.86228547,681.51941749)(98.88228842,681.45942034)
\curveto(98.97228536,681.16941784)(99.07728525,680.92941808)(99.19728842,680.73942034)
\curveto(99.327285,680.55941845)(99.50728482,680.39941861)(99.73728842,680.25942034)
\curveto(99.87728445,680.17941883)(100.04228429,680.1144189)(100.23228842,680.06442034)
\curveto(100.27228406,680.05441896)(100.30728402,680.04941896)(100.33728842,680.04942034)
\curveto(100.36728396,680.05941895)(100.40228393,680.05941895)(100.44228842,680.04942034)
\curveto(100.48228385,680.03941897)(100.54228379,680.02941898)(100.62228842,680.01942034)
\curveto(100.70228363,680.01941899)(100.76728356,680.02441899)(100.81728842,680.03442034)
\curveto(100.89728343,680.05441896)(100.97728335,680.06941894)(101.05728842,680.07942034)
\curveto(101.14728318,680.09941891)(101.2322831,680.12441889)(101.31228842,680.15442034)
\curveto(101.55228278,680.25441876)(101.74728258,680.39441862)(101.89728842,680.57442034)
\curveto(102.04728228,680.75441826)(102.17228216,680.96441805)(102.27228842,681.20442034)
\curveto(102.32228201,681.32441769)(102.35728197,681.44941756)(102.37728842,681.57942034)
\curveto(102.39728193,681.7094173)(102.42228191,681.84441717)(102.45228842,681.98442034)
\lineto(102.45228842,682.13442034)
\curveto(102.46228187,682.18441683)(102.46728186,682.23441678)(102.46728842,682.28442034)
}
}
{
\newrgbcolor{curcolor}{0 0 0}
\pscustom[linestyle=none,fillstyle=solid,fillcolor=curcolor]
{
\newpath
\moveto(113.63221029,682.85442034)
\curveto(113.65220172,682.79441622)(113.66220171,682.7094163)(113.66221029,682.59942034)
\curveto(113.66220171,682.48941652)(113.65220172,682.40441661)(113.63221029,682.34442034)
\lineto(113.63221029,682.19442034)
\curveto(113.61220176,682.1144169)(113.60220177,682.03441698)(113.60221029,681.95442034)
\curveto(113.61220176,681.87441714)(113.60720177,681.79441722)(113.58721029,681.71442034)
\curveto(113.56720181,681.64441737)(113.55220182,681.57941743)(113.54221029,681.51942034)
\curveto(113.53220184,681.45941755)(113.52220185,681.39441762)(113.51221029,681.32442034)
\curveto(113.4722019,681.2144178)(113.43720194,681.09941791)(113.40721029,680.97942034)
\curveto(113.377202,680.86941814)(113.33720204,680.76441825)(113.28721029,680.66442034)
\curveto(113.0772023,680.18441883)(112.80220257,679.79441922)(112.46221029,679.49442034)
\curveto(112.12220325,679.19441982)(111.71220366,678.94442007)(111.23221029,678.74442034)
\curveto(111.11220426,678.69442032)(110.98720439,678.65942035)(110.85721029,678.63942034)
\curveto(110.73720464,678.6094204)(110.61220476,678.57942043)(110.48221029,678.54942034)
\curveto(110.43220494,678.52942048)(110.377205,678.51942049)(110.31721029,678.51942034)
\curveto(110.25720512,678.51942049)(110.20220517,678.5144205)(110.15221029,678.50442034)
\lineto(110.04721029,678.50442034)
\curveto(110.01720536,678.49442052)(109.98720539,678.48942052)(109.95721029,678.48942034)
\curveto(109.90720547,678.47942053)(109.82720555,678.47442054)(109.71721029,678.47442034)
\curveto(109.60720577,678.46442055)(109.52220585,678.46942054)(109.46221029,678.48942034)
\lineto(109.31221029,678.48942034)
\curveto(109.26220611,678.49942051)(109.20720617,678.50442051)(109.14721029,678.50442034)
\curveto(109.09720628,678.49442052)(109.04720633,678.49942051)(108.99721029,678.51942034)
\curveto(108.95720642,678.52942048)(108.91720646,678.53442048)(108.87721029,678.53442034)
\curveto(108.84720653,678.53442048)(108.80720657,678.53942047)(108.75721029,678.54942034)
\curveto(108.65720672,678.57942043)(108.55720682,678.60442041)(108.45721029,678.62442034)
\curveto(108.35720702,678.64442037)(108.26220711,678.67442034)(108.17221029,678.71442034)
\curveto(108.05220732,678.75442026)(107.93720744,678.79442022)(107.82721029,678.83442034)
\curveto(107.72720765,678.87442014)(107.62220775,678.92442009)(107.51221029,678.98442034)
\curveto(107.16220821,679.19441982)(106.86220851,679.43941957)(106.61221029,679.71942034)
\curveto(106.36220901,679.99941901)(106.15220922,680.33441868)(105.98221029,680.72442034)
\curveto(105.93220944,680.8144182)(105.89220948,680.9094181)(105.86221029,681.00942034)
\curveto(105.84220953,681.1094179)(105.81720956,681.2144178)(105.78721029,681.32442034)
\curveto(105.76720961,681.37441764)(105.75720962,681.41941759)(105.75721029,681.45942034)
\curveto(105.75720962,681.49941751)(105.74720963,681.54441747)(105.72721029,681.59442034)
\curveto(105.70720967,681.67441734)(105.69720968,681.75441726)(105.69721029,681.83442034)
\curveto(105.69720968,681.92441709)(105.68720969,682.009417)(105.66721029,682.08942034)
\curveto(105.65720972,682.13941687)(105.65220972,682.18441683)(105.65221029,682.22442034)
\lineto(105.65221029,682.35942034)
\curveto(105.63220974,682.41941659)(105.62220975,682.50441651)(105.62221029,682.61442034)
\curveto(105.63220974,682.72441629)(105.64720973,682.8094162)(105.66721029,682.86942034)
\lineto(105.66721029,682.97442034)
\curveto(105.6772097,683.02441599)(105.6772097,683.07441594)(105.66721029,683.12442034)
\curveto(105.66720971,683.18441583)(105.6772097,683.23941577)(105.69721029,683.28942034)
\curveto(105.70720967,683.33941567)(105.71220966,683.38441563)(105.71221029,683.42442034)
\curveto(105.71220966,683.47441554)(105.72220965,683.52441549)(105.74221029,683.57442034)
\curveto(105.78220959,683.70441531)(105.81720956,683.82941518)(105.84721029,683.94942034)
\curveto(105.8772095,684.07941493)(105.91720946,684.20441481)(105.96721029,684.32442034)
\curveto(106.14720923,684.73441428)(106.36220901,685.07441394)(106.61221029,685.34442034)
\curveto(106.86220851,685.62441339)(107.16720821,685.87941313)(107.52721029,686.10942034)
\curveto(107.62720775,686.15941285)(107.73220764,686.20441281)(107.84221029,686.24442034)
\curveto(107.95220742,686.28441273)(108.06220731,686.32941268)(108.17221029,686.37942034)
\curveto(108.30220707,686.42941258)(108.43720694,686.46441255)(108.57721029,686.48442034)
\curveto(108.71720666,686.50441251)(108.86220651,686.53441248)(109.01221029,686.57442034)
\curveto(109.09220628,686.58441243)(109.16720621,686.58941242)(109.23721029,686.58942034)
\curveto(109.30720607,686.58941242)(109.377206,686.59441242)(109.44721029,686.60442034)
\curveto(110.02720535,686.6144124)(110.52720485,686.55441246)(110.94721029,686.42442034)
\curveto(111.377204,686.29441272)(111.75720362,686.1144129)(112.08721029,685.88442034)
\curveto(112.19720318,685.80441321)(112.30720307,685.7144133)(112.41721029,685.61442034)
\curveto(112.53720284,685.52441349)(112.63720274,685.42441359)(112.71721029,685.31442034)
\curveto(112.79720258,685.2144138)(112.86720251,685.1144139)(112.92721029,685.01442034)
\curveto(112.99720238,684.9144141)(113.06720231,684.8094142)(113.13721029,684.69942034)
\curveto(113.20720217,684.58941442)(113.26220211,684.46941454)(113.30221029,684.33942034)
\curveto(113.34220203,684.21941479)(113.38720199,684.08941492)(113.43721029,683.94942034)
\curveto(113.46720191,683.86941514)(113.49220188,683.78441523)(113.51221029,683.69442034)
\lineto(113.57221029,683.42442034)
\curveto(113.58220179,683.38441563)(113.58720179,683.34441567)(113.58721029,683.30442034)
\curveto(113.58720179,683.26441575)(113.59220178,683.22441579)(113.60221029,683.18442034)
\curveto(113.62220175,683.13441588)(113.62720175,683.07941593)(113.61721029,683.01942034)
\curveto(113.60720177,682.95941605)(113.61220176,682.90441611)(113.63221029,682.85442034)
\moveto(111.53221029,682.31442034)
\curveto(111.54220383,682.36441665)(111.54720383,682.43441658)(111.54721029,682.52442034)
\curveto(111.54720383,682.62441639)(111.54220383,682.69941631)(111.53221029,682.74942034)
\lineto(111.53221029,682.86942034)
\curveto(111.51220386,682.91941609)(111.50220387,682.97441604)(111.50221029,683.03442034)
\curveto(111.50220387,683.09441592)(111.49720388,683.14941586)(111.48721029,683.19942034)
\curveto(111.48720389,683.23941577)(111.48220389,683.26941574)(111.47221029,683.28942034)
\lineto(111.41221029,683.52942034)
\curveto(111.40220397,683.61941539)(111.38220399,683.70441531)(111.35221029,683.78442034)
\curveto(111.24220413,684.04441497)(111.11220426,684.26441475)(110.96221029,684.44442034)
\curveto(110.81220456,684.63441438)(110.61220476,684.78441423)(110.36221029,684.89442034)
\curveto(110.30220507,684.9144141)(110.24220513,684.92941408)(110.18221029,684.93942034)
\curveto(110.12220525,684.95941405)(110.05720532,684.97941403)(109.98721029,684.99942034)
\curveto(109.90720547,685.01941399)(109.82220555,685.02441399)(109.73221029,685.01442034)
\lineto(109.46221029,685.01442034)
\curveto(109.43220594,684.99441402)(109.39720598,684.98441403)(109.35721029,684.98442034)
\curveto(109.31720606,684.99441402)(109.28220609,684.99441402)(109.25221029,684.98442034)
\lineto(109.04221029,684.92442034)
\curveto(108.98220639,684.9144141)(108.92720645,684.89441412)(108.87721029,684.86442034)
\curveto(108.62720675,684.75441426)(108.42220695,684.59441442)(108.26221029,684.38442034)
\curveto(108.11220726,684.18441483)(107.99220738,683.94941506)(107.90221029,683.67942034)
\curveto(107.8722075,683.57941543)(107.84720753,683.47441554)(107.82721029,683.36442034)
\curveto(107.81720756,683.25441576)(107.80220757,683.14441587)(107.78221029,683.03442034)
\curveto(107.7722076,682.98441603)(107.76720761,682.93441608)(107.76721029,682.88442034)
\lineto(107.76721029,682.73442034)
\curveto(107.74720763,682.66441635)(107.73720764,682.55941645)(107.73721029,682.41942034)
\curveto(107.74720763,682.27941673)(107.76220761,682.17441684)(107.78221029,682.10442034)
\lineto(107.78221029,681.96942034)
\curveto(107.80220757,681.88941712)(107.81720756,681.8094172)(107.82721029,681.72942034)
\curveto(107.83720754,681.65941735)(107.85220752,681.58441743)(107.87221029,681.50442034)
\curveto(107.9722074,681.20441781)(108.0772073,680.95941805)(108.18721029,680.76942034)
\curveto(108.30720707,680.58941842)(108.49220688,680.42441859)(108.74221029,680.27442034)
\curveto(108.81220656,680.22441879)(108.88720649,680.18441883)(108.96721029,680.15442034)
\curveto(109.05720632,680.12441889)(109.14720623,680.09941891)(109.23721029,680.07942034)
\curveto(109.2772061,680.06941894)(109.31220606,680.06441895)(109.34221029,680.06442034)
\curveto(109.372206,680.07441894)(109.40720597,680.07441894)(109.44721029,680.06442034)
\lineto(109.56721029,680.03442034)
\curveto(109.61720576,680.03441898)(109.66220571,680.03941897)(109.70221029,680.04942034)
\lineto(109.82221029,680.04942034)
\curveto(109.90220547,680.06941894)(109.98220539,680.08441893)(110.06221029,680.09442034)
\curveto(110.14220523,680.10441891)(110.21720516,680.12441889)(110.28721029,680.15442034)
\curveto(110.54720483,680.25441876)(110.75720462,680.38941862)(110.91721029,680.55942034)
\curveto(111.0772043,680.72941828)(111.21220416,680.93941807)(111.32221029,681.18942034)
\curveto(111.36220401,681.28941772)(111.39220398,681.38941762)(111.41221029,681.48942034)
\curveto(111.43220394,681.58941742)(111.45720392,681.69441732)(111.48721029,681.80442034)
\curveto(111.49720388,681.84441717)(111.50220387,681.87941713)(111.50221029,681.90942034)
\curveto(111.50220387,681.94941706)(111.50720387,681.98941702)(111.51721029,682.02942034)
\lineto(111.51721029,682.16442034)
\curveto(111.51720386,682.2144168)(111.52220385,682.26441675)(111.53221029,682.31442034)
}
}
{
\newrgbcolor{curcolor}{0 0 0}
\pscustom[linestyle=none,fillstyle=solid,fillcolor=curcolor]
{
\newpath
\moveto(832.8075815,693.36292498)
\curveto(832.98757973,693.36291429)(833.18757953,693.36291429)(833.4075815,693.36292498)
\curveto(833.62757909,693.37291428)(833.79257892,693.33791431)(833.9025815,693.25792498)
\curveto(833.98257873,693.19791445)(834.05757866,693.10791454)(834.1275815,692.98792498)
\curveto(834.19757852,692.87791477)(834.26257845,692.77791487)(834.3225815,692.68792498)
\curveto(834.45257826,692.48791516)(834.58257813,692.28291537)(834.7125815,692.07292498)
\curveto(834.85257786,691.87291578)(834.98757773,691.66791598)(835.1175815,691.45792498)
\lineto(835.3275815,691.12792498)
\curveto(835.40757731,691.02791662)(835.48257723,690.92291673)(835.5525815,690.81292498)
\curveto(835.85257686,690.33291732)(836.15757656,689.8529178)(836.4675815,689.37292498)
\curveto(836.77757594,688.90291875)(837.08757563,688.42791922)(837.3975815,687.94792498)
\curveto(837.47757524,687.80791984)(837.56257515,687.67291998)(837.6525815,687.54292498)
\curveto(837.75257496,687.42292023)(837.84257487,687.29292036)(837.9225815,687.15292498)
\lineto(838.4325815,686.34292498)
\curveto(838.6125741,686.08292157)(838.78757393,685.82292183)(838.9575815,685.56292498)
\curveto(839.00757371,685.48292217)(839.06757365,685.38292227)(839.1375815,685.26292498)
\curveto(839.2175735,685.1529225)(839.3125734,685.09792255)(839.4225815,685.09792498)
\curveto(839.47257324,685.11792253)(839.49757322,685.13292252)(839.4975815,685.14292498)
\curveto(839.54757317,685.20292245)(839.57257314,685.28792236)(839.5725815,685.39792498)
\lineto(839.5725815,685.71292498)
\lineto(839.5725815,686.89792498)
\lineto(839.5725815,691.48792498)
\lineto(839.5725815,692.38792498)
\curveto(839.57257314,692.45791519)(839.56757315,692.53291512)(839.5575815,692.61292498)
\curveto(839.54757317,692.69291496)(839.55257316,692.76791488)(839.5725815,692.83792498)
\lineto(839.5725815,693.00292498)
\curveto(839.59257312,693.04291461)(839.60257311,693.08291457)(839.6025815,693.12292498)
\curveto(839.6125731,693.16291449)(839.62757309,693.19791445)(839.6475815,693.22792498)
\curveto(839.70757301,693.30791434)(839.79757292,693.3479143)(839.9175815,693.34792498)
\curveto(840.03757268,693.35791429)(840.16757255,693.36291429)(840.3075815,693.36292498)
\curveto(840.36757235,693.36291429)(840.42757229,693.36291429)(840.4875815,693.36292498)
\curveto(840.55757216,693.36291429)(840.6175721,693.3529143)(840.6675815,693.33292498)
\curveto(840.78757193,693.28291437)(840.85257186,693.19291446)(840.8625815,693.06292498)
\curveto(840.88257183,692.94291471)(840.89257182,692.79791485)(840.8925815,692.62792498)
\lineto(840.8925815,690.97792498)
\lineto(840.8925815,684.69292498)
\lineto(840.8925815,683.43292498)
\lineto(840.8925815,683.10292498)
\curveto(840.90257181,682.99292466)(840.88257183,682.90792474)(840.8325815,682.84792498)
\curveto(840.79257192,682.78792486)(840.74257197,682.7479249)(840.6825815,682.72792498)
\curveto(840.63257208,682.71792493)(840.56757215,682.70292495)(840.4875815,682.68292498)
\lineto(840.0975815,682.68292498)
\lineto(839.7225815,682.68292498)
\curveto(839.60257311,682.68292497)(839.50257321,682.70292495)(839.4225815,682.74292498)
\curveto(839.34257337,682.77292488)(839.27757344,682.82292483)(839.2275815,682.89292498)
\curveto(839.18757353,682.96292469)(839.14257357,683.03292462)(839.0925815,683.10292498)
\curveto(839.0125737,683.22292443)(838.92757379,683.3479243)(838.8375815,683.47792498)
\lineto(838.5975815,683.86792498)
\curveto(838.23757448,684.40792324)(837.88257483,684.94292271)(837.5325815,685.47292498)
\curveto(837.18257553,686.00292165)(836.83257588,686.54292111)(836.4825815,687.09292498)
\curveto(836.29257642,687.39292026)(836.09757662,687.68791996)(835.8975815,687.97792498)
\curveto(835.70757701,688.26791938)(835.5175772,688.56291909)(835.3275815,688.86292498)
\curveto(834.99757772,689.38291827)(834.65257806,689.90791774)(834.2925815,690.43792498)
\curveto(834.25257846,690.50791714)(834.2125785,690.57291708)(834.1725815,690.63292498)
\curveto(834.13257858,690.70291695)(834.07757864,690.76291689)(834.0075815,690.81292498)
\curveto(833.98757873,690.82291683)(833.96757875,690.83791681)(833.9475815,690.85792498)
\curveto(833.92757879,690.87791677)(833.90257881,690.88291677)(833.8725815,690.87292498)
\curveto(833.8125789,690.8529168)(833.77757894,690.81291684)(833.7675815,690.75292498)
\curveto(833.75757896,690.69291696)(833.74257897,690.63291702)(833.7225815,690.57292498)
\lineto(833.7225815,690.46792498)
\curveto(833.70257901,690.39791725)(833.69757902,690.31791733)(833.7075815,690.22792498)
\curveto(833.717579,690.1479175)(833.72257899,690.06791758)(833.7225815,689.98792498)
\lineto(833.7225815,688.99792498)
\lineto(833.7225815,684.22792498)
\lineto(833.7225815,683.52292498)
\lineto(833.7225815,683.34292498)
\curveto(833.73257898,683.27292438)(833.72757899,683.21292444)(833.7075815,683.16292498)
\lineto(833.7075815,683.04292498)
\curveto(833.68757903,682.94292471)(833.66757905,682.87292478)(833.6475815,682.83292498)
\curveto(833.62757909,682.78292487)(833.59257912,682.7479249)(833.5425815,682.72792498)
\curveto(833.49257922,682.71792493)(833.43757928,682.70292495)(833.3775815,682.68292498)
\lineto(833.0775815,682.68292498)
\curveto(832.93757978,682.68292497)(832.8125799,682.68792496)(832.7025815,682.69792498)
\curveto(832.59258012,682.70792494)(832.5125802,682.7529249)(832.4625815,682.83292498)
\curveto(832.4125803,682.89292476)(832.38758033,682.97292468)(832.3875815,683.07292498)
\lineto(832.3875815,683.40292498)
\lineto(832.3875815,684.61792498)
\lineto(832.3875815,690.88792498)
\lineto(832.3875815,692.50792498)
\lineto(832.3875815,692.88292498)
\curveto(832.38758033,693.02291463)(832.4125803,693.13291452)(832.4625815,693.21292498)
\curveto(832.49258022,693.26291439)(832.55258016,693.30791434)(832.6425815,693.34792498)
\curveto(832.66258005,693.35791429)(832.68758003,693.35791429)(832.7175815,693.34792498)
\curveto(832.75757996,693.3479143)(832.78757993,693.3529143)(832.8075815,693.36292498)
}
}
{
\newrgbcolor{curcolor}{0 0 0}
\pscustom[linestyle=none,fillstyle=solid,fillcolor=curcolor]
{
\newpath
\moveto(850.06812837,686.88292498)
\curveto(850.08812031,686.82292083)(850.0981203,686.72792092)(850.09812837,686.59792498)
\curveto(850.0981203,686.47792117)(850.09312031,686.39292126)(850.08312837,686.34292498)
\lineto(850.08312837,686.19292498)
\curveto(850.07312033,686.11292154)(850.06312034,686.03792161)(850.05312837,685.96792498)
\curveto(850.05312035,685.90792174)(850.04812035,685.83792181)(850.03812837,685.75792498)
\curveto(850.01812038,685.69792195)(850.0031204,685.63792201)(849.99312837,685.57792498)
\curveto(849.99312041,685.51792213)(849.98312042,685.45792219)(849.96312837,685.39792498)
\curveto(849.92312048,685.26792238)(849.88812051,685.13792251)(849.85812837,685.00792498)
\curveto(849.82812057,684.87792277)(849.78812061,684.75792289)(849.73812837,684.64792498)
\curveto(849.52812087,684.16792348)(849.24812115,683.76292389)(848.89812837,683.43292498)
\curveto(848.54812185,683.11292454)(848.11812228,682.86792478)(847.60812837,682.69792498)
\curveto(847.4981229,682.65792499)(847.37812302,682.62792502)(847.24812837,682.60792498)
\curveto(847.12812327,682.58792506)(847.0031234,682.56792508)(846.87312837,682.54792498)
\curveto(846.81312359,682.53792511)(846.74812365,682.53292512)(846.67812837,682.53292498)
\curveto(846.61812378,682.52292513)(846.55812384,682.51792513)(846.49812837,682.51792498)
\curveto(846.45812394,682.50792514)(846.398124,682.50292515)(846.31812837,682.50292498)
\curveto(846.24812415,682.50292515)(846.1981242,682.50792514)(846.16812837,682.51792498)
\curveto(846.12812427,682.52792512)(846.08812431,682.53292512)(846.04812837,682.53292498)
\curveto(846.00812439,682.52292513)(845.97312443,682.52292513)(845.94312837,682.53292498)
\lineto(845.85312837,682.53292498)
\lineto(845.49312837,682.57792498)
\curveto(845.35312505,682.61792503)(845.21812518,682.65792499)(845.08812837,682.69792498)
\curveto(844.95812544,682.73792491)(844.83312557,682.78292487)(844.71312837,682.83292498)
\curveto(844.26312614,683.03292462)(843.89312651,683.29292436)(843.60312837,683.61292498)
\curveto(843.31312709,683.93292372)(843.07312733,684.32292333)(842.88312837,684.78292498)
\curveto(842.83312757,684.88292277)(842.79312761,684.98292267)(842.76312837,685.08292498)
\curveto(842.74312766,685.18292247)(842.72312768,685.28792236)(842.70312837,685.39792498)
\curveto(842.68312772,685.43792221)(842.67312773,685.46792218)(842.67312837,685.48792498)
\curveto(842.68312772,685.51792213)(842.68312772,685.5529221)(842.67312837,685.59292498)
\curveto(842.65312775,685.67292198)(842.63812776,685.7529219)(842.62812837,685.83292498)
\curveto(842.62812777,685.92292173)(842.61812778,686.00792164)(842.59812837,686.08792498)
\lineto(842.59812837,686.20792498)
\curveto(842.5981278,686.2479214)(842.59312781,686.29292136)(842.58312837,686.34292498)
\curveto(842.57312783,686.39292126)(842.56812783,686.47792117)(842.56812837,686.59792498)
\curveto(842.56812783,686.72792092)(842.57812782,686.82292083)(842.59812837,686.88292498)
\curveto(842.61812778,686.9529207)(842.62312778,687.02292063)(842.61312837,687.09292498)
\curveto(842.6031278,687.16292049)(842.60812779,687.23292042)(842.62812837,687.30292498)
\curveto(842.63812776,687.3529203)(842.64312776,687.39292026)(842.64312837,687.42292498)
\curveto(842.65312775,687.46292019)(842.66312774,687.50792014)(842.67312837,687.55792498)
\curveto(842.7031277,687.67791997)(842.72812767,687.79791985)(842.74812837,687.91792498)
\curveto(842.77812762,688.03791961)(842.81812758,688.1529195)(842.86812837,688.26292498)
\curveto(843.01812738,688.63291902)(843.1981272,688.96291869)(843.40812837,689.25292498)
\curveto(843.62812677,689.5529181)(843.89312651,689.80291785)(844.20312837,690.00292498)
\curveto(844.32312608,690.08291757)(844.44812595,690.1479175)(844.57812837,690.19792498)
\curveto(844.70812569,690.25791739)(844.84312556,690.31791733)(844.98312837,690.37792498)
\curveto(845.1031253,690.42791722)(845.23312517,690.45791719)(845.37312837,690.46792498)
\curveto(845.51312489,690.48791716)(845.65312475,690.51791713)(845.79312837,690.55792498)
\lineto(845.98812837,690.55792498)
\curveto(846.05812434,690.56791708)(846.12312428,690.57791707)(846.18312837,690.58792498)
\curveto(847.07312333,690.59791705)(847.81312259,690.41291724)(848.40312837,690.03292498)
\curveto(848.99312141,689.652918)(849.41812098,689.15791849)(849.67812837,688.54792498)
\curveto(849.72812067,688.4479192)(849.76812063,688.3479193)(849.79812837,688.24792498)
\curveto(849.82812057,688.1479195)(849.86312054,688.04291961)(849.90312837,687.93292498)
\curveto(849.93312047,687.82291983)(849.95812044,687.70291995)(849.97812837,687.57292498)
\curveto(849.9981204,687.4529202)(850.02312038,687.32792032)(850.05312837,687.19792498)
\curveto(850.06312034,687.1479205)(850.06312034,687.09292056)(850.05312837,687.03292498)
\curveto(850.05312035,686.98292067)(850.05812034,686.93292072)(850.06812837,686.88292498)
\moveto(848.73312837,686.02792498)
\curveto(848.75312165,686.09792155)(848.75812164,686.17792147)(848.74812837,686.26792498)
\lineto(848.74812837,686.52292498)
\curveto(848.74812165,686.91292074)(848.71312169,687.24292041)(848.64312837,687.51292498)
\curveto(848.61312179,687.59292006)(848.58812181,687.67291998)(848.56812837,687.75292498)
\curveto(848.54812185,687.83291982)(848.52312188,687.90791974)(848.49312837,687.97792498)
\curveto(848.21312219,688.62791902)(847.76812263,689.07791857)(847.15812837,689.32792498)
\curveto(847.08812331,689.35791829)(847.01312339,689.37791827)(846.93312837,689.38792498)
\lineto(846.69312837,689.44792498)
\curveto(846.61312379,689.46791818)(846.52812387,689.47791817)(846.43812837,689.47792498)
\lineto(846.16812837,689.47792498)
\lineto(845.89812837,689.43292498)
\curveto(845.7981246,689.41291824)(845.7031247,689.38791826)(845.61312837,689.35792498)
\curveto(845.53312487,689.33791831)(845.45312495,689.30791834)(845.37312837,689.26792498)
\curveto(845.3031251,689.2479184)(845.23812516,689.21791843)(845.17812837,689.17792498)
\curveto(845.11812528,689.13791851)(845.06312534,689.09791855)(845.01312837,689.05792498)
\curveto(844.77312563,688.88791876)(844.57812582,688.68291897)(844.42812837,688.44292498)
\curveto(844.27812612,688.20291945)(844.14812625,687.92291973)(844.03812837,687.60292498)
\curveto(844.00812639,687.50292015)(843.98812641,687.39792025)(843.97812837,687.28792498)
\curveto(843.96812643,687.18792046)(843.95312645,687.08292057)(843.93312837,686.97292498)
\curveto(843.92312648,686.93292072)(843.91812648,686.86792078)(843.91812837,686.77792498)
\curveto(843.90812649,686.7479209)(843.9031265,686.71292094)(843.90312837,686.67292498)
\curveto(843.91312649,686.63292102)(843.91812648,686.58792106)(843.91812837,686.53792498)
\lineto(843.91812837,686.23792498)
\curveto(843.91812648,686.13792151)(843.92812647,686.0479216)(843.94812837,685.96792498)
\lineto(843.97812837,685.78792498)
\curveto(843.9981264,685.68792196)(844.01312639,685.58792206)(844.02312837,685.48792498)
\curveto(844.04312636,685.39792225)(844.07312633,685.31292234)(844.11312837,685.23292498)
\curveto(844.21312619,684.99292266)(844.32812607,684.76792288)(844.45812837,684.55792498)
\curveto(844.5981258,684.3479233)(844.76812563,684.17292348)(844.96812837,684.03292498)
\curveto(845.01812538,684.00292365)(845.06312534,683.97792367)(845.10312837,683.95792498)
\curveto(845.14312526,683.93792371)(845.18812521,683.91292374)(845.23812837,683.88292498)
\curveto(845.31812508,683.83292382)(845.403125,683.78792386)(845.49312837,683.74792498)
\curveto(845.59312481,683.71792393)(845.6981247,683.68792396)(845.80812837,683.65792498)
\curveto(845.85812454,683.63792401)(845.9031245,683.62792402)(845.94312837,683.62792498)
\curveto(845.99312441,683.63792401)(846.04312436,683.63792401)(846.09312837,683.62792498)
\curveto(846.12312428,683.61792403)(846.18312422,683.60792404)(846.27312837,683.59792498)
\curveto(846.37312403,683.58792406)(846.44812395,683.59292406)(846.49812837,683.61292498)
\curveto(846.53812386,683.62292403)(846.57812382,683.62292403)(846.61812837,683.61292498)
\curveto(846.65812374,683.61292404)(846.6981237,683.62292403)(846.73812837,683.64292498)
\curveto(846.81812358,683.66292399)(846.8981235,683.67792397)(846.97812837,683.68792498)
\curveto(847.05812334,683.70792394)(847.13312327,683.73292392)(847.20312837,683.76292498)
\curveto(847.54312286,683.90292375)(847.81812258,684.09792355)(848.02812837,684.34792498)
\curveto(848.23812216,684.59792305)(848.41312199,684.89292276)(848.55312837,685.23292498)
\curveto(848.6031218,685.3529223)(848.63312177,685.47792217)(848.64312837,685.60792498)
\curveto(848.66312174,685.7479219)(848.69312171,685.88792176)(848.73312837,686.02792498)
}
}
{
\newrgbcolor{curcolor}{0 0 0}
\pscustom[linestyle=none,fillstyle=solid,fillcolor=curcolor]
{
\newpath
\moveto(852.50140962,692.74792498)
\curveto(852.65140761,692.7479149)(852.80140746,692.74291491)(852.95140962,692.73292498)
\curveto(853.10140716,692.73291492)(853.20640706,692.69291496)(853.26640962,692.61292498)
\curveto(853.31640695,692.5529151)(853.34140692,692.46791518)(853.34140962,692.35792498)
\curveto(853.35140691,692.25791539)(853.35640691,692.1529155)(853.35640962,692.04292498)
\lineto(853.35640962,691.17292498)
\curveto(853.35640691,691.09291656)(853.35140691,691.00791664)(853.34140962,690.91792498)
\curveto(853.34140692,690.83791681)(853.35140691,690.76791688)(853.37140962,690.70792498)
\curveto(853.41140685,690.56791708)(853.50140676,690.47791717)(853.64140962,690.43792498)
\curveto(853.69140657,690.42791722)(853.73640653,690.42291723)(853.77640962,690.42292498)
\lineto(853.92640962,690.42292498)
\lineto(854.33140962,690.42292498)
\curveto(854.49140577,690.43291722)(854.60640566,690.42291723)(854.67640962,690.39292498)
\curveto(854.7664055,690.33291732)(854.82640544,690.27291738)(854.85640962,690.21292498)
\curveto(854.87640539,690.17291748)(854.88640538,690.12791752)(854.88640962,690.07792498)
\lineto(854.88640962,689.92792498)
\curveto(854.88640538,689.81791783)(854.88140538,689.71291794)(854.87140962,689.61292498)
\curveto(854.8614054,689.52291813)(854.82640544,689.4529182)(854.76640962,689.40292498)
\curveto(854.70640556,689.3529183)(854.62140564,689.32291833)(854.51140962,689.31292498)
\lineto(854.18140962,689.31292498)
\curveto(854.07140619,689.32291833)(853.9614063,689.32791832)(853.85140962,689.32792498)
\curveto(853.74140652,689.32791832)(853.64640662,689.31291834)(853.56640962,689.28292498)
\curveto(853.49640677,689.2529184)(853.44640682,689.20291845)(853.41640962,689.13292498)
\curveto(853.38640688,689.06291859)(853.3664069,688.97791867)(853.35640962,688.87792498)
\curveto(853.34640692,688.78791886)(853.34140692,688.68791896)(853.34140962,688.57792498)
\curveto(853.35140691,688.47791917)(853.35640691,688.37791927)(853.35640962,688.27792498)
\lineto(853.35640962,685.30792498)
\curveto(853.35640691,685.08792256)(853.35140691,684.8529228)(853.34140962,684.60292498)
\curveto(853.34140692,684.36292329)(853.38640688,684.17792347)(853.47640962,684.04792498)
\curveto(853.52640674,683.96792368)(853.59140667,683.91292374)(853.67140962,683.88292498)
\curveto(853.75140651,683.8529238)(853.84640642,683.82792382)(853.95640962,683.80792498)
\curveto(853.98640628,683.79792385)(854.01640625,683.79292386)(854.04640962,683.79292498)
\curveto(854.08640618,683.80292385)(854.12140614,683.80292385)(854.15140962,683.79292498)
\lineto(854.34640962,683.79292498)
\curveto(854.44640582,683.79292386)(854.53640573,683.78292387)(854.61640962,683.76292498)
\curveto(854.70640556,683.7529239)(854.77140549,683.71792393)(854.81140962,683.65792498)
\curveto(854.83140543,683.62792402)(854.84640542,683.57292408)(854.85640962,683.49292498)
\curveto(854.87640539,683.42292423)(854.88640538,683.3479243)(854.88640962,683.26792498)
\curveto(854.89640537,683.18792446)(854.89640537,683.10792454)(854.88640962,683.02792498)
\curveto(854.87640539,682.95792469)(854.85640541,682.90292475)(854.82640962,682.86292498)
\curveto(854.78640548,682.79292486)(854.71140555,682.74292491)(854.60140962,682.71292498)
\curveto(854.52140574,682.69292496)(854.43140583,682.68292497)(854.33140962,682.68292498)
\curveto(854.23140603,682.69292496)(854.14140612,682.69792495)(854.06140962,682.69792498)
\curveto(854.00140626,682.69792495)(853.94140632,682.69292496)(853.88140962,682.68292498)
\curveto(853.82140644,682.68292497)(853.7664065,682.68792496)(853.71640962,682.69792498)
\lineto(853.53640962,682.69792498)
\curveto(853.48640678,682.70792494)(853.43640683,682.71292494)(853.38640962,682.71292498)
\curveto(853.34640692,682.72292493)(853.30140696,682.72792492)(853.25140962,682.72792498)
\curveto(853.05140721,682.77792487)(852.87640739,682.83292482)(852.72640962,682.89292498)
\curveto(852.58640768,682.9529247)(852.4664078,683.05792459)(852.36640962,683.20792498)
\curveto(852.22640804,683.40792424)(852.14640812,683.65792399)(852.12640962,683.95792498)
\curveto(852.10640816,684.26792338)(852.09640817,684.59792305)(852.09640962,684.94792498)
\lineto(852.09640962,688.87792498)
\curveto(852.0664082,689.00791864)(852.03640823,689.10291855)(852.00640962,689.16292498)
\curveto(851.98640828,689.22291843)(851.91640835,689.27291838)(851.79640962,689.31292498)
\curveto(851.75640851,689.32291833)(851.71640855,689.32291833)(851.67640962,689.31292498)
\curveto(851.63640863,689.30291835)(851.59640867,689.30791834)(851.55640962,689.32792498)
\lineto(851.31640962,689.32792498)
\curveto(851.18640908,689.32791832)(851.07640919,689.33791831)(850.98640962,689.35792498)
\curveto(850.90640936,689.38791826)(850.85140941,689.4479182)(850.82140962,689.53792498)
\curveto(850.80140946,689.57791807)(850.78640948,689.62291803)(850.77640962,689.67292498)
\lineto(850.77640962,689.82292498)
\curveto(850.77640949,689.96291769)(850.78640948,690.07791757)(850.80640962,690.16792498)
\curveto(850.82640944,690.26791738)(850.88640938,690.34291731)(850.98640962,690.39292498)
\curveto(851.09640917,690.43291722)(851.23640903,690.44291721)(851.40640962,690.42292498)
\curveto(851.58640868,690.40291725)(851.73640853,690.41291724)(851.85640962,690.45292498)
\curveto(851.94640832,690.50291715)(852.01640825,690.57291708)(852.06640962,690.66292498)
\curveto(852.08640818,690.72291693)(852.09640817,690.79791685)(852.09640962,690.88792498)
\lineto(852.09640962,691.14292498)
\lineto(852.09640962,692.07292498)
\lineto(852.09640962,692.31292498)
\curveto(852.09640817,692.40291525)(852.10640816,692.47791517)(852.12640962,692.53792498)
\curveto(852.1664081,692.61791503)(852.24140802,692.68291497)(852.35140962,692.73292498)
\curveto(852.38140788,692.73291492)(852.40640786,692.73291492)(852.42640962,692.73292498)
\curveto(852.45640781,692.74291491)(852.48140778,692.7479149)(852.50140962,692.74792498)
}
}
{
\newrgbcolor{curcolor}{0 0 0}
\pscustom[linestyle=none,fillstyle=solid,fillcolor=curcolor]
{
\newpath
\moveto(863.1582065,683.23792498)
\curveto(863.18819867,683.07792457)(863.17319868,682.94292471)(863.1132065,682.83292498)
\curveto(863.0531988,682.73292492)(862.97319888,682.65792499)(862.8732065,682.60792498)
\curveto(862.82319903,682.58792506)(862.76819909,682.57792507)(862.7082065,682.57792498)
\curveto(862.6581992,682.57792507)(862.60319925,682.56792508)(862.5432065,682.54792498)
\curveto(862.32319953,682.49792515)(862.10319975,682.51292514)(861.8832065,682.59292498)
\curveto(861.67320018,682.66292499)(861.52820033,682.7529249)(861.4482065,682.86292498)
\curveto(861.39820046,682.93292472)(861.3532005,683.01292464)(861.3132065,683.10292498)
\curveto(861.27320058,683.20292445)(861.22320063,683.28292437)(861.1632065,683.34292498)
\curveto(861.14320071,683.36292429)(861.11820074,683.38292427)(861.0882065,683.40292498)
\curveto(861.06820079,683.42292423)(861.03820082,683.42792422)(860.9982065,683.41792498)
\curveto(860.88820097,683.38792426)(860.78320107,683.33292432)(860.6832065,683.25292498)
\curveto(860.59320126,683.17292448)(860.50320135,683.10292455)(860.4132065,683.04292498)
\curveto(860.28320157,682.96292469)(860.14320171,682.88792476)(859.9932065,682.81792498)
\curveto(859.84320201,682.75792489)(859.68320217,682.70292495)(859.5132065,682.65292498)
\curveto(859.41320244,682.62292503)(859.30320255,682.60292505)(859.1832065,682.59292498)
\curveto(859.07320278,682.58292507)(858.96320289,682.56792508)(858.8532065,682.54792498)
\curveto(858.80320305,682.53792511)(858.7582031,682.53292512)(858.7182065,682.53292498)
\lineto(858.6132065,682.53292498)
\curveto(858.50320335,682.51292514)(858.39820346,682.51292514)(858.2982065,682.53292498)
\lineto(858.1632065,682.53292498)
\curveto(858.11320374,682.54292511)(858.06320379,682.5479251)(858.0132065,682.54792498)
\curveto(857.96320389,682.5479251)(857.91820394,682.55792509)(857.8782065,682.57792498)
\curveto(857.83820402,682.58792506)(857.80320405,682.59292506)(857.7732065,682.59292498)
\curveto(857.7532041,682.58292507)(857.72820413,682.58292507)(857.6982065,682.59292498)
\lineto(857.4582065,682.65292498)
\curveto(857.37820448,682.66292499)(857.30320455,682.68292497)(857.2332065,682.71292498)
\curveto(856.93320492,682.84292481)(856.68820517,682.98792466)(856.4982065,683.14792498)
\curveto(856.31820554,683.31792433)(856.16820569,683.5529241)(856.0482065,683.85292498)
\curveto(855.9582059,684.07292358)(855.91320594,684.33792331)(855.9132065,684.64792498)
\lineto(855.9132065,684.96292498)
\curveto(855.92320593,685.01292264)(855.92820593,685.06292259)(855.9282065,685.11292498)
\lineto(855.9582065,685.29292498)
\lineto(856.0782065,685.62292498)
\curveto(856.11820574,685.73292192)(856.16820569,685.83292182)(856.2282065,685.92292498)
\curveto(856.40820545,686.21292144)(856.6532052,686.42792122)(856.9632065,686.56792498)
\curveto(857.27320458,686.70792094)(857.61320424,686.83292082)(857.9832065,686.94292498)
\curveto(858.12320373,686.98292067)(858.26820359,687.01292064)(858.4182065,687.03292498)
\curveto(858.56820329,687.0529206)(858.71820314,687.07792057)(858.8682065,687.10792498)
\curveto(858.93820292,687.12792052)(859.00320285,687.13792051)(859.0632065,687.13792498)
\curveto(859.13320272,687.13792051)(859.20820265,687.1479205)(859.2882065,687.16792498)
\curveto(859.3582025,687.18792046)(859.42820243,687.19792045)(859.4982065,687.19792498)
\curveto(859.56820229,687.20792044)(859.64320221,687.22292043)(859.7232065,687.24292498)
\curveto(859.97320188,687.30292035)(860.20820165,687.3529203)(860.4282065,687.39292498)
\curveto(860.64820121,687.44292021)(860.82320103,687.55792009)(860.9532065,687.73792498)
\curveto(861.01320084,687.81791983)(861.06320079,687.91791973)(861.1032065,688.03792498)
\curveto(861.14320071,688.16791948)(861.14320071,688.30791934)(861.1032065,688.45792498)
\curveto(861.04320081,688.69791895)(860.9532009,688.88791876)(860.8332065,689.02792498)
\curveto(860.72320113,689.16791848)(860.56320129,689.27791837)(860.3532065,689.35792498)
\curveto(860.23320162,689.40791824)(860.08820177,689.44291821)(859.9182065,689.46292498)
\curveto(859.7582021,689.48291817)(859.58820227,689.49291816)(859.4082065,689.49292498)
\curveto(859.22820263,689.49291816)(859.0532028,689.48291817)(858.8832065,689.46292498)
\curveto(858.71320314,689.44291821)(858.56820329,689.41291824)(858.4482065,689.37292498)
\curveto(858.27820358,689.31291834)(858.11320374,689.22791842)(857.9532065,689.11792498)
\curveto(857.87320398,689.05791859)(857.79820406,688.97791867)(857.7282065,688.87792498)
\curveto(857.66820419,688.78791886)(857.61320424,688.68791896)(857.5632065,688.57792498)
\curveto(857.53320432,688.49791915)(857.50320435,688.41291924)(857.4732065,688.32292498)
\curveto(857.4532044,688.23291942)(857.40820445,688.16291949)(857.3382065,688.11292498)
\curveto(857.29820456,688.08291957)(857.22820463,688.05791959)(857.1282065,688.03792498)
\curveto(857.03820482,688.02791962)(856.94320491,688.02291963)(856.8432065,688.02292498)
\curveto(856.74320511,688.02291963)(856.64320521,688.02791962)(856.5432065,688.03792498)
\curveto(856.4532054,688.05791959)(856.38820547,688.08291957)(856.3482065,688.11292498)
\curveto(856.30820555,688.14291951)(856.27820558,688.19291946)(856.2582065,688.26292498)
\curveto(856.23820562,688.33291932)(856.23820562,688.40791924)(856.2582065,688.48792498)
\curveto(856.28820557,688.61791903)(856.31820554,688.73791891)(856.3482065,688.84792498)
\curveto(856.38820547,688.96791868)(856.43320542,689.08291857)(856.4832065,689.19292498)
\curveto(856.67320518,689.54291811)(856.91320494,689.81291784)(857.2032065,690.00292498)
\curveto(857.49320436,690.20291745)(857.853204,690.36291729)(858.2832065,690.48292498)
\curveto(858.38320347,690.50291715)(858.48320337,690.51791713)(858.5832065,690.52792498)
\curveto(858.69320316,690.53791711)(858.80320305,690.5529171)(858.9132065,690.57292498)
\curveto(858.9532029,690.58291707)(859.01820284,690.58291707)(859.1082065,690.57292498)
\curveto(859.19820266,690.57291708)(859.2532026,690.58291707)(859.2732065,690.60292498)
\curveto(859.97320188,690.61291704)(860.58320127,690.53291712)(861.1032065,690.36292498)
\curveto(861.62320023,690.19291746)(861.98819987,689.86791778)(862.1982065,689.38792498)
\curveto(862.28819957,689.18791846)(862.33819952,688.9529187)(862.3482065,688.68292498)
\curveto(862.36819949,688.42291923)(862.37819948,688.1479195)(862.3782065,687.85792498)
\lineto(862.3782065,684.54292498)
\curveto(862.37819948,684.40292325)(862.38319947,684.26792338)(862.3932065,684.13792498)
\curveto(862.40319945,684.00792364)(862.43319942,683.90292375)(862.4832065,683.82292498)
\curveto(862.53319932,683.7529239)(862.59819926,683.70292395)(862.6782065,683.67292498)
\curveto(862.76819909,683.63292402)(862.853199,683.60292405)(862.9332065,683.58292498)
\curveto(863.01319884,683.57292408)(863.07319878,683.52792412)(863.1132065,683.44792498)
\curveto(863.13319872,683.41792423)(863.14319871,683.38792426)(863.1432065,683.35792498)
\curveto(863.14319871,683.32792432)(863.14819871,683.28792436)(863.1582065,683.23792498)
\moveto(861.0132065,684.90292498)
\curveto(861.07320078,685.04292261)(861.10320075,685.20292245)(861.1032065,685.38292498)
\curveto(861.11320074,685.57292208)(861.11820074,685.76792188)(861.1182065,685.96792498)
\curveto(861.11820074,686.07792157)(861.11320074,686.17792147)(861.1032065,686.26792498)
\curveto(861.09320076,686.35792129)(861.0532008,686.42792122)(860.9832065,686.47792498)
\curveto(860.9532009,686.49792115)(860.88320097,686.50792114)(860.7732065,686.50792498)
\curveto(860.7532011,686.48792116)(860.71820114,686.47792117)(860.6682065,686.47792498)
\curveto(860.61820124,686.47792117)(860.57320128,686.46792118)(860.5332065,686.44792498)
\curveto(860.4532014,686.42792122)(860.36320149,686.40792124)(860.2632065,686.38792498)
\lineto(859.9632065,686.32792498)
\curveto(859.93320192,686.32792132)(859.89820196,686.32292133)(859.8582065,686.31292498)
\lineto(859.7532065,686.31292498)
\curveto(859.60320225,686.27292138)(859.43820242,686.2479214)(859.2582065,686.23792498)
\curveto(859.08820277,686.23792141)(858.92820293,686.21792143)(858.7782065,686.17792498)
\curveto(858.69820316,686.15792149)(858.62320323,686.13792151)(858.5532065,686.11792498)
\curveto(858.49320336,686.10792154)(858.42320343,686.09292156)(858.3432065,686.07292498)
\curveto(858.18320367,686.02292163)(858.03320382,685.95792169)(857.8932065,685.87792498)
\curveto(857.7532041,685.80792184)(857.63320422,685.71792193)(857.5332065,685.60792498)
\curveto(857.43320442,685.49792215)(857.3582045,685.36292229)(857.3082065,685.20292498)
\curveto(857.2582046,685.0529226)(857.23820462,684.86792278)(857.2482065,684.64792498)
\curveto(857.24820461,684.5479231)(857.26320459,684.4529232)(857.2932065,684.36292498)
\curveto(857.33320452,684.28292337)(857.37820448,684.20792344)(857.4282065,684.13792498)
\curveto(857.50820435,684.02792362)(857.61320424,683.93292372)(857.7432065,683.85292498)
\curveto(857.87320398,683.78292387)(858.01320384,683.72292393)(858.1632065,683.67292498)
\curveto(858.21320364,683.66292399)(858.26320359,683.65792399)(858.3132065,683.65792498)
\curveto(858.36320349,683.65792399)(858.41320344,683.652924)(858.4632065,683.64292498)
\curveto(858.53320332,683.62292403)(858.61820324,683.60792404)(858.7182065,683.59792498)
\curveto(858.82820303,683.59792405)(858.91820294,683.60792404)(858.9882065,683.62792498)
\curveto(859.04820281,683.647924)(859.10820275,683.652924)(859.1682065,683.64292498)
\curveto(859.22820263,683.64292401)(859.28820257,683.652924)(859.3482065,683.67292498)
\curveto(859.42820243,683.69292396)(859.50320235,683.70792394)(859.5732065,683.71792498)
\curveto(859.6532022,683.72792392)(859.72820213,683.7479239)(859.7982065,683.77792498)
\curveto(860.08820177,683.89792375)(860.33320152,684.04292361)(860.5332065,684.21292498)
\curveto(860.74320111,684.38292327)(860.90320095,684.61292304)(861.0132065,684.90292498)
}
}
{
\newrgbcolor{curcolor}{0 0 0}
\pscustom[linestyle=none,fillstyle=solid,fillcolor=curcolor]
{
\newpath
\moveto(866.75984712,690.58792498)
\curveto(867.47984306,690.59791705)(868.08484245,690.51291714)(868.57484712,690.33292498)
\curveto(869.06484147,690.16291749)(869.44484109,689.85791779)(869.71484712,689.41792498)
\curveto(869.78484075,689.30791834)(869.8398407,689.19291846)(869.87984712,689.07292498)
\curveto(869.91984062,688.96291869)(869.95984058,688.83791881)(869.99984712,688.69792498)
\curveto(870.01984052,688.62791902)(870.02484051,688.5529191)(870.01484712,688.47292498)
\curveto(870.00484053,688.40291925)(869.98984055,688.3479193)(869.96984712,688.30792498)
\curveto(869.94984059,688.28791936)(869.92484061,688.26791938)(869.89484712,688.24792498)
\curveto(869.86484067,688.23791941)(869.8398407,688.22291943)(869.81984712,688.20292498)
\curveto(869.76984077,688.18291947)(869.71984082,688.17791947)(869.66984712,688.18792498)
\curveto(869.61984092,688.19791945)(869.56984097,688.19791945)(869.51984712,688.18792498)
\curveto(869.4398411,688.16791948)(869.3348412,688.16291949)(869.20484712,688.17292498)
\curveto(869.07484146,688.19291946)(868.98484155,688.21791943)(868.93484712,688.24792498)
\curveto(868.85484168,688.29791935)(868.79984174,688.36291929)(868.76984712,688.44292498)
\curveto(868.74984179,688.53291912)(868.71484182,688.61791903)(868.66484712,688.69792498)
\curveto(868.57484196,688.85791879)(868.44984209,689.00291865)(868.28984712,689.13292498)
\curveto(868.17984236,689.21291844)(868.05984248,689.27291838)(867.92984712,689.31292498)
\curveto(867.79984274,689.3529183)(867.65984288,689.39291826)(867.50984712,689.43292498)
\curveto(867.45984308,689.4529182)(867.40984313,689.45791819)(867.35984712,689.44792498)
\curveto(867.30984323,689.4479182)(867.25984328,689.4529182)(867.20984712,689.46292498)
\curveto(867.14984339,689.48291817)(867.07484346,689.49291816)(866.98484712,689.49292498)
\curveto(866.89484364,689.49291816)(866.81984372,689.48291817)(866.75984712,689.46292498)
\lineto(866.66984712,689.46292498)
\lineto(866.51984712,689.43292498)
\curveto(866.46984407,689.43291822)(866.41984412,689.42791822)(866.36984712,689.41792498)
\curveto(866.10984443,689.35791829)(865.89484464,689.27291838)(865.72484712,689.16292498)
\curveto(865.55484498,689.0529186)(865.4398451,688.86791878)(865.37984712,688.60792498)
\curveto(865.35984518,688.53791911)(865.35484518,688.46791918)(865.36484712,688.39792498)
\curveto(865.38484515,688.32791932)(865.40484513,688.26791938)(865.42484712,688.21792498)
\curveto(865.48484505,688.06791958)(865.55484498,687.95791969)(865.63484712,687.88792498)
\curveto(865.72484481,687.82791982)(865.8348447,687.75791989)(865.96484712,687.67792498)
\curveto(866.12484441,687.57792007)(866.30484423,687.50292015)(866.50484712,687.45292498)
\curveto(866.70484383,687.41292024)(866.90484363,687.36292029)(867.10484712,687.30292498)
\curveto(867.2348433,687.26292039)(867.36484317,687.23292042)(867.49484712,687.21292498)
\curveto(867.62484291,687.19292046)(867.75484278,687.16292049)(867.88484712,687.12292498)
\curveto(868.09484244,687.06292059)(868.29984224,687.00292065)(868.49984712,686.94292498)
\curveto(868.69984184,686.89292076)(868.89984164,686.82792082)(869.09984712,686.74792498)
\lineto(869.24984712,686.68792498)
\curveto(869.29984124,686.66792098)(869.34984119,686.64292101)(869.39984712,686.61292498)
\curveto(869.59984094,686.49292116)(869.77484076,686.35792129)(869.92484712,686.20792498)
\curveto(870.07484046,686.05792159)(870.19984034,685.86792178)(870.29984712,685.63792498)
\curveto(870.31984022,685.56792208)(870.3398402,685.47292218)(870.35984712,685.35292498)
\curveto(870.37984016,685.28292237)(870.38984015,685.20792244)(870.38984712,685.12792498)
\curveto(870.39984014,685.05792259)(870.40484013,684.97792267)(870.40484712,684.88792498)
\lineto(870.40484712,684.73792498)
\curveto(870.38484015,684.66792298)(870.37484016,684.59792305)(870.37484712,684.52792498)
\curveto(870.37484016,684.45792319)(870.36484017,684.38792326)(870.34484712,684.31792498)
\curveto(870.31484022,684.20792344)(870.27984026,684.10292355)(870.23984712,684.00292498)
\curveto(870.19984034,683.90292375)(870.15484038,683.81292384)(870.10484712,683.73292498)
\curveto(869.94484059,683.47292418)(869.7398408,683.26292439)(869.48984712,683.10292498)
\curveto(869.2398413,682.9529247)(868.95984158,682.82292483)(868.64984712,682.71292498)
\curveto(868.55984198,682.68292497)(868.46484207,682.66292499)(868.36484712,682.65292498)
\curveto(868.27484226,682.63292502)(868.18484235,682.60792504)(868.09484712,682.57792498)
\curveto(867.99484254,682.55792509)(867.89484264,682.5479251)(867.79484712,682.54792498)
\curveto(867.69484284,682.5479251)(867.59484294,682.53792511)(867.49484712,682.51792498)
\lineto(867.34484712,682.51792498)
\curveto(867.29484324,682.50792514)(867.22484331,682.50292515)(867.13484712,682.50292498)
\curveto(867.04484349,682.50292515)(866.97484356,682.50792514)(866.92484712,682.51792498)
\lineto(866.75984712,682.51792498)
\curveto(866.69984384,682.53792511)(866.6348439,682.5479251)(866.56484712,682.54792498)
\curveto(866.49484404,682.53792511)(866.4348441,682.54292511)(866.38484712,682.56292498)
\curveto(866.3348442,682.57292508)(866.26984427,682.57792507)(866.18984712,682.57792498)
\lineto(865.94984712,682.63792498)
\curveto(865.87984466,682.647925)(865.80484473,682.66792498)(865.72484712,682.69792498)
\curveto(865.41484512,682.79792485)(865.14484539,682.92292473)(864.91484712,683.07292498)
\curveto(864.68484585,683.22292443)(864.48484605,683.41792423)(864.31484712,683.65792498)
\curveto(864.22484631,683.78792386)(864.14984639,683.92292373)(864.08984712,684.06292498)
\curveto(864.02984651,684.20292345)(863.97484656,684.35792329)(863.92484712,684.52792498)
\curveto(863.90484663,684.58792306)(863.89484664,684.65792299)(863.89484712,684.73792498)
\curveto(863.90484663,684.82792282)(863.91984662,684.89792275)(863.93984712,684.94792498)
\curveto(863.96984657,684.98792266)(864.01984652,685.02792262)(864.08984712,685.06792498)
\curveto(864.1398464,685.08792256)(864.20984633,685.09792255)(864.29984712,685.09792498)
\curveto(864.38984615,685.10792254)(864.47984606,685.10792254)(864.56984712,685.09792498)
\curveto(864.65984588,685.08792256)(864.74484579,685.07292258)(864.82484712,685.05292498)
\curveto(864.91484562,685.04292261)(864.97484556,685.02792262)(865.00484712,685.00792498)
\curveto(865.07484546,684.95792269)(865.11984542,684.88292277)(865.13984712,684.78292498)
\curveto(865.16984537,684.69292296)(865.20484533,684.60792304)(865.24484712,684.52792498)
\curveto(865.34484519,684.30792334)(865.47984506,684.13792351)(865.64984712,684.01792498)
\curveto(865.76984477,683.92792372)(865.90484463,683.85792379)(866.05484712,683.80792498)
\curveto(866.20484433,683.75792389)(866.36484417,683.70792394)(866.53484712,683.65792498)
\lineto(866.84984712,683.61292498)
\lineto(866.93984712,683.61292498)
\curveto(867.00984353,683.59292406)(867.09984344,683.58292407)(867.20984712,683.58292498)
\curveto(867.32984321,683.58292407)(867.42984311,683.59292406)(867.50984712,683.61292498)
\curveto(867.57984296,683.61292404)(867.6348429,683.61792403)(867.67484712,683.62792498)
\curveto(867.7348428,683.63792401)(867.79484274,683.64292401)(867.85484712,683.64292498)
\curveto(867.91484262,683.652924)(867.96984257,683.66292399)(868.01984712,683.67292498)
\curveto(868.30984223,683.7529239)(868.539842,683.85792379)(868.70984712,683.98792498)
\curveto(868.87984166,684.11792353)(868.99984154,684.33792331)(869.06984712,684.64792498)
\curveto(869.08984145,684.69792295)(869.09484144,684.7529229)(869.08484712,684.81292498)
\curveto(869.07484146,684.87292278)(869.06484147,684.91792273)(869.05484712,684.94792498)
\curveto(869.00484153,685.13792251)(868.9348416,685.27792237)(868.84484712,685.36792498)
\curveto(868.75484178,685.46792218)(868.6398419,685.55792209)(868.49984712,685.63792498)
\curveto(868.40984213,685.69792195)(868.30984223,685.7479219)(868.19984712,685.78792498)
\lineto(867.86984712,685.90792498)
\curveto(867.8398427,685.91792173)(867.80984273,685.92292173)(867.77984712,685.92292498)
\curveto(867.75984278,685.92292173)(867.7348428,685.93292172)(867.70484712,685.95292498)
\curveto(867.36484317,686.06292159)(867.00984353,686.14292151)(866.63984712,686.19292498)
\curveto(866.27984426,686.2529214)(865.9398446,686.3479213)(865.61984712,686.47792498)
\curveto(865.51984502,686.51792113)(865.42484511,686.5529211)(865.33484712,686.58292498)
\curveto(865.24484529,686.61292104)(865.15984538,686.652921)(865.07984712,686.70292498)
\curveto(864.88984565,686.81292084)(864.71484582,686.93792071)(864.55484712,687.07792498)
\curveto(864.39484614,687.21792043)(864.26984627,687.39292026)(864.17984712,687.60292498)
\curveto(864.14984639,687.67291998)(864.12484641,687.74291991)(864.10484712,687.81292498)
\curveto(864.09484644,687.88291977)(864.07984646,687.95791969)(864.05984712,688.03792498)
\curveto(864.02984651,688.15791949)(864.01984652,688.29291936)(864.02984712,688.44292498)
\curveto(864.0398465,688.60291905)(864.05484648,688.73791891)(864.07484712,688.84792498)
\curveto(864.09484644,688.89791875)(864.10484643,688.93791871)(864.10484712,688.96792498)
\curveto(864.11484642,689.00791864)(864.12984641,689.0479186)(864.14984712,689.08792498)
\curveto(864.2398463,689.31791833)(864.35984618,689.51791813)(864.50984712,689.68792498)
\curveto(864.66984587,689.85791779)(864.84984569,690.00791764)(865.04984712,690.13792498)
\curveto(865.19984534,690.22791742)(865.36484517,690.29791735)(865.54484712,690.34792498)
\curveto(865.72484481,690.40791724)(865.91484462,690.46291719)(866.11484712,690.51292498)
\curveto(866.18484435,690.52291713)(866.24984429,690.53291712)(866.30984712,690.54292498)
\curveto(866.37984416,690.5529171)(866.45484408,690.56291709)(866.53484712,690.57292498)
\curveto(866.56484397,690.58291707)(866.60484393,690.58291707)(866.65484712,690.57292498)
\curveto(866.70484383,690.56291709)(866.7398438,690.56791708)(866.75984712,690.58792498)
}
}
{
\newrgbcolor{curcolor}{0.90196079 0.90196079 0.90196079}
\pscustom[linestyle=none,fillstyle=solid,fillcolor=curcolor]
{
\newpath
\moveto(812.80437349,693.3929616)
\lineto(827.80437349,693.3929616)
\lineto(827.80437349,678.3929616)
\lineto(812.80437349,678.3929616)
\closepath
}
}
{
\newrgbcolor{curcolor}{0 0 0}
\pscustom[linestyle=none,fillstyle=solid,fillcolor=curcolor]
{
\newpath
\moveto(840.7425815,660.42721942)
\curveto(840.76257195,660.37721867)(840.78757193,660.31721873)(840.8175815,660.24721942)
\curveto(840.84757187,660.17721887)(840.86757185,660.10221895)(840.8775815,660.02221942)
\curveto(840.89757182,659.9522191)(840.89757182,659.88221917)(840.8775815,659.81221942)
\curveto(840.86757185,659.7522193)(840.82757189,659.70721934)(840.7575815,659.67721942)
\curveto(840.70757201,659.65721939)(840.64757207,659.6472194)(840.5775815,659.64721942)
\lineto(840.3675815,659.64721942)
\lineto(839.9175815,659.64721942)
\curveto(839.76757295,659.6472194)(839.64757307,659.67221938)(839.5575815,659.72221942)
\curveto(839.45757326,659.78221927)(839.38257333,659.88721916)(839.3325815,660.03721942)
\curveto(839.29257342,660.18721886)(839.24757347,660.32221873)(839.1975815,660.44221942)
\curveto(839.08757363,660.70221835)(838.98757373,660.97221808)(838.8975815,661.25221942)
\curveto(838.80757391,661.53221752)(838.70757401,661.80721724)(838.5975815,662.07721942)
\curveto(838.56757415,662.16721688)(838.53757418,662.2522168)(838.5075815,662.33221942)
\curveto(838.48757423,662.41221664)(838.45757426,662.48721656)(838.4175815,662.55721942)
\curveto(838.38757433,662.62721642)(838.34257437,662.68721636)(838.2825815,662.73721942)
\curveto(838.22257449,662.78721626)(838.14257457,662.82721622)(838.0425815,662.85721942)
\curveto(837.99257472,662.87721617)(837.93257478,662.88221617)(837.8625815,662.87221942)
\lineto(837.6675815,662.87221942)
\lineto(834.8325815,662.87221942)
\lineto(834.5325815,662.87221942)
\curveto(834.42257829,662.88221617)(834.3175784,662.88221617)(834.2175815,662.87221942)
\curveto(834.1175786,662.86221619)(834.02257869,662.8472162)(833.9325815,662.82721942)
\curveto(833.85257886,662.80721624)(833.79257892,662.76721628)(833.7525815,662.70721942)
\curveto(833.67257904,662.60721644)(833.6125791,662.49221656)(833.5725815,662.36221942)
\curveto(833.54257917,662.24221681)(833.50257921,662.11721693)(833.4525815,661.98721942)
\curveto(833.35257936,661.75721729)(833.25757946,661.51721753)(833.1675815,661.26721942)
\curveto(833.08757963,661.01721803)(832.99757972,660.77721827)(832.8975815,660.54721942)
\curveto(832.87757984,660.48721856)(832.85257986,660.41721863)(832.8225815,660.33721942)
\curveto(832.80257991,660.26721878)(832.77757994,660.19221886)(832.7475815,660.11221942)
\curveto(832.71758,660.03221902)(832.68258003,659.95721909)(832.6425815,659.88721942)
\curveto(832.6125801,659.82721922)(832.57758014,659.78221927)(832.5375815,659.75221942)
\curveto(832.45758026,659.69221936)(832.34758037,659.65721939)(832.2075815,659.64721942)
\lineto(831.7875815,659.64721942)
\lineto(831.5475815,659.64721942)
\curveto(831.47758124,659.65721939)(831.4175813,659.68221937)(831.3675815,659.72221942)
\curveto(831.3175814,659.7522193)(831.28758143,659.79721925)(831.2775815,659.85721942)
\curveto(831.27758144,659.91721913)(831.28258143,659.97721907)(831.2925815,660.03721942)
\curveto(831.3125814,660.10721894)(831.33258138,660.17221888)(831.3525815,660.23221942)
\curveto(831.38258133,660.30221875)(831.40758131,660.3522187)(831.4275815,660.38221942)
\curveto(831.56758115,660.70221835)(831.69258102,661.01721803)(831.8025815,661.32721942)
\curveto(831.9125808,661.6472174)(832.03258068,661.96721708)(832.1625815,662.28721942)
\curveto(832.25258046,662.50721654)(832.33758038,662.72221633)(832.4175815,662.93221942)
\curveto(832.49758022,663.1522159)(832.58258013,663.37221568)(832.6725815,663.59221942)
\curveto(832.97257974,664.31221474)(833.25757946,665.03721401)(833.5275815,665.76721942)
\curveto(833.79757892,666.50721254)(834.08257863,667.24221181)(834.3825815,667.97221942)
\curveto(834.49257822,668.23221082)(834.59257812,668.49721055)(834.6825815,668.76721942)
\curveto(834.78257793,669.03721001)(834.88757783,669.30220975)(834.9975815,669.56221942)
\curveto(835.04757767,669.67220938)(835.09257762,669.79220926)(835.1325815,669.92221942)
\curveto(835.18257753,670.06220899)(835.25257746,670.16220889)(835.3425815,670.22221942)
\curveto(835.38257733,670.26220879)(835.44757727,670.29220876)(835.5375815,670.31221942)
\curveto(835.55757716,670.32220873)(835.57757714,670.32220873)(835.5975815,670.31221942)
\curveto(835.62757709,670.31220874)(835.65257706,670.31720873)(835.6725815,670.32721942)
\curveto(835.85257686,670.32720872)(836.06257665,670.32720872)(836.3025815,670.32721942)
\curveto(836.54257617,670.33720871)(836.717576,670.30220875)(836.8275815,670.22221942)
\curveto(836.90757581,670.16220889)(836.96757575,670.06220899)(837.0075815,669.92221942)
\curveto(837.05757566,669.79220926)(837.10757561,669.67220938)(837.1575815,669.56221942)
\curveto(837.25757546,669.33220972)(837.34757537,669.10220995)(837.4275815,668.87221942)
\curveto(837.50757521,668.64221041)(837.59757512,668.41221064)(837.6975815,668.18221942)
\curveto(837.77757494,667.98221107)(837.85257486,667.77721127)(837.9225815,667.56721942)
\curveto(838.00257471,667.35721169)(838.08757463,667.1522119)(838.1775815,666.95221942)
\curveto(838.47757424,666.22221283)(838.76257395,665.48221357)(839.0325815,664.73221942)
\curveto(839.3125734,663.99221506)(839.60757311,663.25721579)(839.9175815,662.52721942)
\curveto(839.95757276,662.43721661)(839.98757273,662.3522167)(840.0075815,662.27221942)
\curveto(840.03757268,662.19221686)(840.06757265,662.10721694)(840.0975815,662.01721942)
\curveto(840.20757251,661.75721729)(840.3125724,661.49221756)(840.4125815,661.22221942)
\curveto(840.52257219,660.9522181)(840.63257208,660.68721836)(840.7425815,660.42721942)
\moveto(837.5325815,664.07221942)
\curveto(837.62257509,664.10221495)(837.67757504,664.1522149)(837.6975815,664.22221942)
\curveto(837.72757499,664.29221476)(837.73257498,664.36721468)(837.7125815,664.44721942)
\curveto(837.70257501,664.53721451)(837.67757504,664.62221443)(837.6375815,664.70221942)
\curveto(837.60757511,664.79221426)(837.57757514,664.86721418)(837.5475815,664.92721942)
\curveto(837.52757519,664.96721408)(837.5175752,665.00221405)(837.5175815,665.03221942)
\curveto(837.5175752,665.06221399)(837.50757521,665.09721395)(837.4875815,665.13721942)
\lineto(837.3975815,665.37721942)
\curveto(837.37757534,665.46721358)(837.34757537,665.55721349)(837.3075815,665.64721942)
\curveto(837.15757556,666.00721304)(837.02257569,666.37221268)(836.9025815,666.74221942)
\curveto(836.79257592,667.12221193)(836.66257605,667.49221156)(836.5125815,667.85221942)
\curveto(836.46257625,667.96221109)(836.4175763,668.07221098)(836.3775815,668.18221942)
\curveto(836.34757637,668.29221076)(836.30757641,668.39721065)(836.2575815,668.49721942)
\curveto(836.23757648,668.5472105)(836.2125765,668.59221046)(836.1825815,668.63221942)
\curveto(836.16257655,668.68221037)(836.1125766,668.70721034)(836.0325815,668.70721942)
\curveto(836.0125767,668.68721036)(835.99257672,668.67221038)(835.9725815,668.66221942)
\curveto(835.95257676,668.6522104)(835.93257678,668.63721041)(835.9125815,668.61721942)
\curveto(835.87257684,668.56721048)(835.84257687,668.51221054)(835.8225815,668.45221942)
\curveto(835.80257691,668.40221065)(835.78257693,668.3472107)(835.7625815,668.28721942)
\curveto(835.712577,668.17721087)(835.67257704,668.06721098)(835.6425815,667.95721942)
\curveto(835.6125771,667.8472112)(835.57257714,667.73721131)(835.5225815,667.62721942)
\curveto(835.35257736,667.23721181)(835.20257751,666.84221221)(835.0725815,666.44221942)
\curveto(834.95257776,666.04221301)(834.8125779,665.6522134)(834.6525815,665.27221942)
\lineto(834.5925815,665.12221942)
\curveto(834.58257813,665.07221398)(834.56757815,665.02221403)(834.5475815,664.97221942)
\lineto(834.4575815,664.73221942)
\curveto(834.42757829,664.6522144)(834.40257831,664.57221448)(834.3825815,664.49221942)
\curveto(834.36257835,664.44221461)(834.35257836,664.38721466)(834.3525815,664.32721942)
\curveto(834.36257835,664.26721478)(834.37757834,664.21721483)(834.3975815,664.17721942)
\curveto(834.44757827,664.09721495)(834.55257816,664.052215)(834.7125815,664.04221942)
\lineto(835.1625815,664.04221942)
\lineto(836.7675815,664.04221942)
\curveto(836.87757584,664.04221501)(837.0125757,664.03721501)(837.1725815,664.02721942)
\curveto(837.33257538,664.02721502)(837.45257526,664.04221501)(837.5325815,664.07221942)
}
}
{
\newrgbcolor{curcolor}{0 0 0}
\pscustom[linestyle=none,fillstyle=solid,fillcolor=curcolor]
{
\newpath
\moveto(845.504144,667.55221942)
\curveto(845.73413921,667.5522115)(845.86413908,667.49221156)(845.894144,667.37221942)
\curveto(845.92413902,667.26221179)(845.939139,667.09721195)(845.939144,666.87721942)
\lineto(845.939144,666.59221942)
\curveto(845.939139,666.50221255)(845.91413903,666.42721262)(845.864144,666.36721942)
\curveto(845.80413914,666.28721276)(845.71913922,666.24221281)(845.609144,666.23221942)
\curveto(845.49913944,666.23221282)(845.38913955,666.21721283)(845.279144,666.18721942)
\curveto(845.1391398,666.15721289)(845.00413994,666.12721292)(844.874144,666.09721942)
\curveto(844.75414019,666.06721298)(844.6391403,666.02721302)(844.529144,665.97721942)
\curveto(844.2391407,665.8472132)(844.00414094,665.66721338)(843.824144,665.43721942)
\curveto(843.6441413,665.21721383)(843.48914145,664.96221409)(843.359144,664.67221942)
\curveto(843.31914162,664.56221449)(843.28914165,664.4472146)(843.269144,664.32721942)
\curveto(843.24914169,664.21721483)(843.22414172,664.10221495)(843.194144,663.98221942)
\curveto(843.18414176,663.93221512)(843.17914176,663.88221517)(843.179144,663.83221942)
\curveto(843.18914175,663.78221527)(843.18914175,663.73221532)(843.179144,663.68221942)
\curveto(843.14914179,663.56221549)(843.13414181,663.42221563)(843.134144,663.26221942)
\curveto(843.1441418,663.11221594)(843.14914179,662.96721608)(843.149144,662.82721942)
\lineto(843.149144,660.98221942)
\lineto(843.149144,660.63721942)
\curveto(843.14914179,660.51721853)(843.1441418,660.40221865)(843.134144,660.29221942)
\curveto(843.12414182,660.18221887)(843.11914182,660.08721896)(843.119144,660.00721942)
\curveto(843.12914181,659.92721912)(843.10914183,659.85721919)(843.059144,659.79721942)
\curveto(843.00914193,659.72721932)(842.92914201,659.68721936)(842.819144,659.67721942)
\curveto(842.71914222,659.66721938)(842.60914233,659.66221939)(842.489144,659.66221942)
\lineto(842.219144,659.66221942)
\curveto(842.16914277,659.68221937)(842.11914282,659.69721935)(842.069144,659.70721942)
\curveto(842.02914291,659.72721932)(841.99914294,659.7522193)(841.979144,659.78221942)
\curveto(841.92914301,659.8522192)(841.89914304,659.93721911)(841.889144,660.03721942)
\lineto(841.889144,660.36721942)
\lineto(841.889144,661.52221942)
\lineto(841.889144,665.67721942)
\lineto(841.889144,666.71221942)
\lineto(841.889144,667.01221942)
\curveto(841.89914304,667.11221194)(841.92914301,667.19721185)(841.979144,667.26721942)
\curveto(842.00914293,667.30721174)(842.05914288,667.33721171)(842.129144,667.35721942)
\curveto(842.20914273,667.37721167)(842.29414265,667.38721166)(842.384144,667.38721942)
\curveto(842.47414247,667.39721165)(842.56414238,667.39721165)(842.654144,667.38721942)
\curveto(842.7441422,667.37721167)(842.81414213,667.36221169)(842.864144,667.34221942)
\curveto(842.944142,667.31221174)(842.99414195,667.2522118)(843.014144,667.16221942)
\curveto(843.0441419,667.08221197)(843.05914188,666.99221206)(843.059144,666.89221942)
\lineto(843.059144,666.59221942)
\curveto(843.05914188,666.49221256)(843.07914186,666.40221265)(843.119144,666.32221942)
\curveto(843.12914181,666.30221275)(843.1391418,666.28721276)(843.149144,666.27721942)
\lineto(843.194144,666.23221942)
\curveto(843.30414164,666.23221282)(843.39414155,666.27721277)(843.464144,666.36721942)
\curveto(843.53414141,666.46721258)(843.59414135,666.5472125)(843.644144,666.60721942)
\lineto(843.734144,666.69721942)
\curveto(843.82414112,666.80721224)(843.94914099,666.92221213)(844.109144,667.04221942)
\curveto(844.26914067,667.16221189)(844.41914052,667.2522118)(844.559144,667.31221942)
\curveto(844.64914029,667.36221169)(844.7441402,667.39721165)(844.844144,667.41721942)
\curveto(844.94414,667.4472116)(845.04913989,667.47721157)(845.159144,667.50721942)
\curveto(845.21913972,667.51721153)(845.27913966,667.52221153)(845.339144,667.52221942)
\curveto(845.39913954,667.53221152)(845.45413949,667.54221151)(845.504144,667.55221942)
}
}
{
\newrgbcolor{curcolor}{0 0 0}
\pscustom[linestyle=none,fillstyle=solid,fillcolor=curcolor]
{
\newpath
\moveto(850.00390962,667.55221942)
\curveto(850.74390483,667.56221149)(851.35890422,667.4522116)(851.84890962,667.22221942)
\curveto(852.34890323,667.00221205)(852.74390283,666.66721238)(853.03390962,666.21721942)
\curveto(853.16390241,666.01721303)(853.2739023,665.77221328)(853.36390962,665.48221942)
\curveto(853.38390219,665.43221362)(853.39890218,665.36721368)(853.40890962,665.28721942)
\curveto(853.41890216,665.20721384)(853.41390216,665.13721391)(853.39390962,665.07721942)
\curveto(853.36390221,665.02721402)(853.31390226,664.98221407)(853.24390962,664.94221942)
\curveto(853.21390236,664.92221413)(853.18390239,664.91221414)(853.15390962,664.91221942)
\curveto(853.12390245,664.92221413)(853.08890249,664.92221413)(853.04890962,664.91221942)
\curveto(853.00890257,664.90221415)(852.96890261,664.89721415)(852.92890962,664.89721942)
\curveto(852.88890269,664.90721414)(852.84890273,664.91221414)(852.80890962,664.91221942)
\lineto(852.49390962,664.91221942)
\curveto(852.39390318,664.92221413)(852.30890327,664.9522141)(852.23890962,665.00221942)
\curveto(852.15890342,665.06221399)(852.10390347,665.1472139)(852.07390962,665.25721942)
\curveto(852.04390353,665.36721368)(852.00390357,665.46221359)(851.95390962,665.54221942)
\curveto(851.80390377,665.80221325)(851.60890397,666.00721304)(851.36890962,666.15721942)
\curveto(851.28890429,666.20721284)(851.20390437,666.2472128)(851.11390962,666.27721942)
\curveto(851.02390455,666.31721273)(850.92890465,666.3522127)(850.82890962,666.38221942)
\curveto(850.68890489,666.42221263)(850.50390507,666.44221261)(850.27390962,666.44221942)
\curveto(850.04390553,666.4522126)(849.85390572,666.43221262)(849.70390962,666.38221942)
\curveto(849.63390594,666.36221269)(849.56890601,666.3472127)(849.50890962,666.33721942)
\curveto(849.44890613,666.32721272)(849.38390619,666.31221274)(849.31390962,666.29221942)
\curveto(849.05390652,666.18221287)(848.82390675,666.03221302)(848.62390962,665.84221942)
\curveto(848.42390715,665.6522134)(848.26890731,665.42721362)(848.15890962,665.16721942)
\curveto(848.11890746,665.07721397)(848.08390749,664.98221407)(848.05390962,664.88221942)
\curveto(848.02390755,664.79221426)(847.99390758,664.69221436)(847.96390962,664.58221942)
\lineto(847.87390962,664.17721942)
\curveto(847.86390771,664.12721492)(847.85890772,664.07221498)(847.85890962,664.01221942)
\curveto(847.86890771,663.9522151)(847.86390771,663.89721515)(847.84390962,663.84721942)
\lineto(847.84390962,663.72721942)
\curveto(847.83390774,663.68721536)(847.82390775,663.62221543)(847.81390962,663.53221942)
\curveto(847.81390776,663.44221561)(847.82390775,663.37721567)(847.84390962,663.33721942)
\curveto(847.85390772,663.28721576)(847.85390772,663.23721581)(847.84390962,663.18721942)
\curveto(847.83390774,663.13721591)(847.83390774,663.08721596)(847.84390962,663.03721942)
\curveto(847.85390772,662.99721605)(847.85890772,662.92721612)(847.85890962,662.82721942)
\curveto(847.8789077,662.7472163)(847.89390768,662.66221639)(847.90390962,662.57221942)
\curveto(847.92390765,662.48221657)(847.94390763,662.39721665)(847.96390962,662.31721942)
\curveto(848.0739075,661.99721705)(848.19890738,661.71721733)(848.33890962,661.47721942)
\curveto(848.48890709,661.2472178)(848.69390688,661.047218)(848.95390962,660.87721942)
\curveto(849.04390653,660.82721822)(849.13390644,660.78221827)(849.22390962,660.74221942)
\curveto(849.32390625,660.70221835)(849.42890615,660.66221839)(849.53890962,660.62221942)
\curveto(849.58890599,660.61221844)(849.62890595,660.60721844)(849.65890962,660.60721942)
\curveto(849.68890589,660.60721844)(849.72890585,660.60221845)(849.77890962,660.59221942)
\curveto(849.80890577,660.58221847)(849.85890572,660.57721847)(849.92890962,660.57721942)
\lineto(850.09390962,660.57721942)
\curveto(850.09390548,660.56721848)(850.11390546,660.56221849)(850.15390962,660.56221942)
\curveto(850.1739054,660.57221848)(850.19890538,660.57221848)(850.22890962,660.56221942)
\curveto(850.25890532,660.56221849)(850.28890529,660.56721848)(850.31890962,660.57721942)
\curveto(850.38890519,660.59721845)(850.45390512,660.60221845)(850.51390962,660.59221942)
\curveto(850.58390499,660.59221846)(850.65390492,660.60221845)(850.72390962,660.62221942)
\curveto(850.98390459,660.70221835)(851.20890437,660.80221825)(851.39890962,660.92221942)
\curveto(851.58890399,661.052218)(851.74890383,661.21721783)(851.87890962,661.41721942)
\curveto(851.92890365,661.49721755)(851.9739036,661.58221747)(852.01390962,661.67221942)
\lineto(852.13390962,661.94221942)
\curveto(852.15390342,662.02221703)(852.1739034,662.09721695)(852.19390962,662.16721942)
\curveto(852.22390335,662.2472168)(852.2739033,662.31221674)(852.34390962,662.36221942)
\curveto(852.3739032,662.39221666)(852.43390314,662.41221664)(852.52390962,662.42221942)
\curveto(852.61390296,662.44221661)(852.70890287,662.4522166)(852.80890962,662.45221942)
\curveto(852.91890266,662.46221659)(853.01890256,662.46221659)(853.10890962,662.45221942)
\curveto(853.20890237,662.44221661)(853.2789023,662.42221663)(853.31890962,662.39221942)
\curveto(853.3789022,662.3522167)(853.41390216,662.29221676)(853.42390962,662.21221942)
\curveto(853.44390213,662.13221692)(853.44390213,662.047217)(853.42390962,661.95721942)
\curveto(853.3739022,661.80721724)(853.32390225,661.66221739)(853.27390962,661.52221942)
\curveto(853.23390234,661.39221766)(853.1789024,661.26221779)(853.10890962,661.13221942)
\curveto(852.95890262,660.83221822)(852.76890281,660.56721848)(852.53890962,660.33721942)
\curveto(852.31890326,660.10721894)(852.04890353,659.92221913)(851.72890962,659.78221942)
\curveto(851.64890393,659.74221931)(851.56390401,659.70721934)(851.47390962,659.67721942)
\curveto(851.38390419,659.65721939)(851.28890429,659.63221942)(851.18890962,659.60221942)
\curveto(851.0789045,659.56221949)(850.96890461,659.54221951)(850.85890962,659.54221942)
\curveto(850.74890483,659.53221952)(850.63890494,659.51721953)(850.52890962,659.49721942)
\curveto(850.48890509,659.47721957)(850.44890513,659.47221958)(850.40890962,659.48221942)
\curveto(850.36890521,659.49221956)(850.32890525,659.49221956)(850.28890962,659.48221942)
\lineto(850.15390962,659.48221942)
\lineto(849.91390962,659.48221942)
\curveto(849.84390573,659.47221958)(849.7789058,659.47721957)(849.71890962,659.49721942)
\lineto(849.64390962,659.49721942)
\lineto(849.28390962,659.54221942)
\curveto(849.15390642,659.58221947)(849.02890655,659.61721943)(848.90890962,659.64721942)
\curveto(848.78890679,659.67721937)(848.6739069,659.71721933)(848.56390962,659.76721942)
\curveto(848.20390737,659.92721912)(847.90390767,660.11721893)(847.66390962,660.33721942)
\curveto(847.43390814,660.55721849)(847.21890836,660.82721822)(847.01890962,661.14721942)
\curveto(846.96890861,661.22721782)(846.92390865,661.31721773)(846.88390962,661.41721942)
\lineto(846.76390962,661.71721942)
\curveto(846.71390886,661.82721722)(846.6789089,661.94221711)(846.65890962,662.06221942)
\curveto(846.63890894,662.18221687)(846.61390896,662.30221675)(846.58390962,662.42221942)
\curveto(846.573909,662.46221659)(846.56890901,662.50221655)(846.56890962,662.54221942)
\curveto(846.56890901,662.58221647)(846.56390901,662.62221643)(846.55390962,662.66221942)
\curveto(846.53390904,662.72221633)(846.52390905,662.78721626)(846.52390962,662.85721942)
\curveto(846.53390904,662.92721612)(846.52890905,662.99221606)(846.50890962,663.05221942)
\lineto(846.50890962,663.20221942)
\curveto(846.49890908,663.2522158)(846.49390908,663.32221573)(846.49390962,663.41221942)
\curveto(846.49390908,663.50221555)(846.49890908,663.57221548)(846.50890962,663.62221942)
\curveto(846.51890906,663.67221538)(846.51890906,663.71721533)(846.50890962,663.75721942)
\curveto(846.50890907,663.79721525)(846.51390906,663.83721521)(846.52390962,663.87721942)
\curveto(846.54390903,663.9472151)(846.54890903,664.01721503)(846.53890962,664.08721942)
\curveto(846.53890904,664.15721489)(846.54890903,664.22221483)(846.56890962,664.28221942)
\curveto(846.60890897,664.4522146)(846.64390893,664.62221443)(846.67390962,664.79221942)
\curveto(846.70390887,664.96221409)(846.74890883,665.12221393)(846.80890962,665.27221942)
\curveto(847.01890856,665.79221326)(847.2739083,666.21221284)(847.57390962,666.53221942)
\curveto(847.8739077,666.8522122)(848.28390729,667.11721193)(848.80390962,667.32721942)
\curveto(848.91390666,667.37721167)(849.03390654,667.41221164)(849.16390962,667.43221942)
\curveto(849.29390628,667.4522116)(849.42890615,667.47721157)(849.56890962,667.50721942)
\curveto(849.63890594,667.51721153)(849.70890587,667.52221153)(849.77890962,667.52221942)
\curveto(849.84890573,667.53221152)(849.92390565,667.54221151)(850.00390962,667.55221942)
}
}
{
\newrgbcolor{curcolor}{0 0 0}
\pscustom[linestyle=none,fillstyle=solid,fillcolor=curcolor]
{
\newpath
\moveto(855.39055025,670.31221942)
\curveto(855.53054873,670.31220874)(855.67554858,670.30720874)(855.82555025,670.29721942)
\curveto(855.98554827,670.29720875)(856.09554816,670.25720879)(856.15555025,670.17721942)
\curveto(856.20554805,670.10720894)(856.23054803,670.00220905)(856.23055025,669.86221942)
\lineto(856.23055025,669.47221942)
\lineto(856.23055025,667.88221942)
\lineto(856.23055025,667.43221942)
\curveto(856.23054803,667.39221166)(856.22554803,667.3522117)(856.21555025,667.31221942)
\curveto(856.21554804,667.27221178)(856.22054804,667.23221182)(856.23055025,667.19221942)
\curveto(856.24054802,667.16221189)(856.24054802,667.12721192)(856.23055025,667.08721942)
\curveto(856.23054803,667.047212)(856.23554802,667.01221204)(856.24555025,666.98221942)
\curveto(856.26554799,666.90221215)(856.27554798,666.82721222)(856.27555025,666.75721942)
\curveto(856.28554797,666.68721236)(856.34054792,666.6522124)(856.44055025,666.65221942)
\curveto(856.4605478,666.67221238)(856.48054778,666.68221237)(856.50055025,666.68221942)
\curveto(856.53054773,666.69221236)(856.5555477,666.70721234)(856.57555025,666.72721942)
\curveto(856.63554762,666.76721228)(856.69054757,666.80721224)(856.74055025,666.84721942)
\curveto(856.79054747,666.89721215)(856.84554741,666.9472121)(856.90555025,666.99721942)
\curveto(857.01554724,667.07721197)(857.13554712,667.1472119)(857.26555025,667.20721942)
\curveto(857.40554685,667.27721177)(857.54554671,667.33721171)(857.68555025,667.38721942)
\curveto(857.76554649,667.40721164)(857.85054641,667.42221163)(857.94055025,667.43221942)
\curveto(858.03054623,667.4522116)(858.11554614,667.47221158)(858.19555025,667.49221942)
\curveto(858.23554602,667.51221154)(858.27554598,667.51721153)(858.31555025,667.50721942)
\curveto(858.3555459,667.50721154)(858.40054586,667.51221154)(858.45055025,667.52221942)
\curveto(858.50054576,667.54221151)(858.58054568,667.5522115)(858.69055025,667.55221942)
\curveto(858.81054545,667.5522115)(858.89554536,667.54221151)(858.94555025,667.52221942)
\lineto(859.08055025,667.52221942)
\curveto(859.13054513,667.52221153)(859.18054508,667.51721153)(859.23055025,667.50721942)
\curveto(859.31054495,667.48721156)(859.39054487,667.47221158)(859.47055025,667.46221942)
\curveto(859.55054471,667.4522116)(859.63054463,667.43721161)(859.71055025,667.41721942)
\curveto(859.7605445,667.39721165)(859.80054446,667.38221167)(859.83055025,667.37221942)
\curveto(859.8605444,667.37221168)(859.90054436,667.36221169)(859.95055025,667.34221942)
\curveto(860.0605442,667.29221176)(860.16554409,667.23721181)(860.26555025,667.17721942)
\curveto(860.36554389,667.12721192)(860.46554379,667.06721198)(860.56555025,666.99721942)
\curveto(860.66554359,666.90721214)(860.7605435,666.80221225)(860.85055025,666.68221942)
\curveto(860.91054335,666.59221246)(860.96554329,666.50221255)(861.01555025,666.41221942)
\curveto(861.06554319,666.32221273)(861.11554314,666.22221283)(861.16555025,666.11221942)
\curveto(861.19554306,666.04221301)(861.21554304,665.97221308)(861.22555025,665.90221942)
\curveto(861.24554301,665.83221322)(861.26554299,665.75721329)(861.28555025,665.67721942)
\curveto(861.30554295,665.62721342)(861.31554294,665.57721347)(861.31555025,665.52721942)
\curveto(861.31554294,665.47721357)(861.32054294,665.42221363)(861.33055025,665.36221942)
\curveto(861.35054291,665.31221374)(861.3555429,665.26221379)(861.34555025,665.21221942)
\curveto(861.34554291,665.16221389)(861.3555429,665.11221394)(861.37555025,665.06221942)
\lineto(861.37555025,664.91221942)
\curveto(861.38554287,664.86221419)(861.38554287,664.80721424)(861.37555025,664.74721942)
\lineto(861.37555025,664.58221942)
\lineto(861.37555025,663.93721942)
\lineto(861.37555025,660.81721942)
\lineto(861.37555025,660.51721942)
\curveto(861.38554287,660.40721864)(861.38554287,660.29721875)(861.37555025,660.18721942)
\curveto(861.37554288,660.08721896)(861.36554289,659.99221906)(861.34555025,659.90221942)
\curveto(861.32554293,659.81221924)(861.29554296,659.7522193)(861.25555025,659.72221942)
\curveto(861.18554307,659.66221939)(861.0555432,659.63221942)(860.86555025,659.63221942)
\lineto(860.47555025,659.63221942)
\curveto(860.3555439,659.63221942)(860.26554399,659.67221938)(860.20555025,659.75221942)
\curveto(860.1555441,659.82221923)(860.13054413,659.90221915)(860.13055025,659.99221942)
\lineto(860.13055025,660.30721942)
\lineto(860.13055025,661.38721942)
\lineto(860.13055025,663.78721942)
\lineto(860.13055025,664.64221942)
\curveto(860.14054412,664.9522141)(860.11054415,665.21721383)(860.04055025,665.43721942)
\curveto(859.92054434,665.77721327)(859.72054454,666.02721302)(859.44055025,666.18721942)
\curveto(859.3605449,666.23721281)(859.27554498,666.27721277)(859.18555025,666.30721942)
\curveto(859.09554516,666.3472127)(858.99554526,666.37721267)(858.88555025,666.39721942)
\curveto(858.84554541,666.40721264)(858.78554547,666.41221264)(858.70555025,666.41221942)
\curveto(858.66554559,666.42221263)(858.61554564,666.43221262)(858.55555025,666.44221942)
\curveto(858.50554575,666.4522126)(858.4555458,666.4472126)(858.40555025,666.42721942)
\curveto(858.36554589,666.41721263)(858.32554593,666.41221264)(858.28555025,666.41221942)
\curveto(858.24554601,666.42221263)(858.20054606,666.42221263)(858.15055025,666.41221942)
\curveto(858.0605462,666.39221266)(857.96554629,666.37221268)(857.86555025,666.35221942)
\curveto(857.77554648,666.34221271)(857.69054657,666.31721273)(857.61055025,666.27721942)
\curveto(857.27054699,666.13721291)(857.00054726,665.9472131)(856.80055025,665.70721942)
\curveto(856.60054766,665.46721358)(856.44554781,665.16221389)(856.33555025,664.79221942)
\curveto(856.31554794,664.72221433)(856.30054796,664.6472144)(856.29055025,664.56721942)
\curveto(856.28054798,664.48721456)(856.26554799,664.40721464)(856.24555025,664.32721942)
\curveto(856.23554802,664.29721475)(856.23054803,664.26221479)(856.23055025,664.22221942)
\curveto(856.24054802,664.19221486)(856.24054802,664.16221489)(856.23055025,664.13221942)
\curveto(856.22054804,664.08221497)(856.22054804,664.03221502)(856.23055025,663.98221942)
\curveto(856.24054802,663.93221512)(856.24054802,663.88221517)(856.23055025,663.83221942)
\lineto(856.23055025,660.81721942)
\lineto(856.23055025,660.53221942)
\curveto(856.24054802,660.43221862)(856.24054802,660.33221872)(856.23055025,660.23221942)
\curveto(856.23054803,660.13221892)(856.22554803,660.03721901)(856.21555025,659.94721942)
\curveto(856.20554805,659.85721919)(856.18554807,659.79221926)(856.15555025,659.75221942)
\curveto(856.11554814,659.70221935)(856.06554819,659.67221938)(856.00555025,659.66221942)
\curveto(855.9555483,659.66221939)(855.89554836,659.6522194)(855.82555025,659.63221942)
\lineto(855.61555025,659.63221942)
\lineto(855.30055025,659.63221942)
\curveto(855.20054906,659.64221941)(855.12554913,659.67721937)(855.07555025,659.73721942)
\curveto(855.02554923,659.81721923)(854.99554926,659.91221914)(854.98555025,660.02221942)
\lineto(854.98555025,660.39721942)
\lineto(854.98555025,661.77721942)
\lineto(854.98555025,668.03221942)
\lineto(854.98555025,669.50221942)
\curveto(854.98554927,669.61220944)(854.98054928,669.72720932)(854.97055025,669.84721942)
\curveto(854.97054929,669.97720907)(854.99554926,670.07720897)(855.04555025,670.14721942)
\curveto(855.08554917,670.21720883)(855.1605491,670.26720878)(855.27055025,670.29721942)
\curveto(855.29054897,670.30720874)(855.31054895,670.30720874)(855.33055025,670.29721942)
\curveto(855.35054891,670.29720875)(855.37054889,670.30220875)(855.39055025,670.31221942)
}
}
{
\newrgbcolor{curcolor}{0 0 0}
\pscustom[linestyle=none,fillstyle=solid,fillcolor=curcolor]
{
\newpath
\moveto(863.56015962,668.87221942)
\curveto(863.4801585,668.93221012)(863.43515855,669.03721001)(863.42515962,669.18721942)
\lineto(863.42515962,669.65221942)
\lineto(863.42515962,669.90721942)
\curveto(863.42515856,669.99720905)(863.44015854,670.07220898)(863.47015962,670.13221942)
\curveto(863.51015847,670.21220884)(863.59015839,670.27220878)(863.71015962,670.31221942)
\curveto(863.73015825,670.32220873)(863.75015823,670.32220873)(863.77015962,670.31221942)
\curveto(863.80015818,670.31220874)(863.82515816,670.31720873)(863.84515962,670.32721942)
\curveto(864.01515797,670.32720872)(864.17515781,670.32220873)(864.32515962,670.31221942)
\curveto(864.47515751,670.30220875)(864.57515741,670.24220881)(864.62515962,670.13221942)
\curveto(864.65515733,670.07220898)(864.67015731,669.99720905)(864.67015962,669.90721942)
\lineto(864.67015962,669.65221942)
\curveto(864.67015731,669.47220958)(864.66515732,669.30220975)(864.65515962,669.14221942)
\curveto(864.65515733,668.98221007)(864.59015739,668.87721017)(864.46015962,668.82721942)
\curveto(864.41015757,668.80721024)(864.35515763,668.79721025)(864.29515962,668.79721942)
\lineto(864.13015962,668.79721942)
\lineto(863.81515962,668.79721942)
\curveto(863.71515827,668.79721025)(863.63015835,668.82221023)(863.56015962,668.87221942)
\moveto(864.67015962,660.36721942)
\lineto(864.67015962,660.05221942)
\curveto(864.6801573,659.9522191)(864.66015732,659.87221918)(864.61015962,659.81221942)
\curveto(864.5801574,659.7522193)(864.53515745,659.71221934)(864.47515962,659.69221942)
\curveto(864.41515757,659.68221937)(864.34515764,659.66721938)(864.26515962,659.64721942)
\lineto(864.04015962,659.64721942)
\curveto(863.91015807,659.6472194)(863.79515819,659.6522194)(863.69515962,659.66221942)
\curveto(863.60515838,659.68221937)(863.53515845,659.73221932)(863.48515962,659.81221942)
\curveto(863.44515854,659.87221918)(863.42515856,659.9472191)(863.42515962,660.03721942)
\lineto(863.42515962,660.32221942)
\lineto(863.42515962,666.66721942)
\lineto(863.42515962,666.98221942)
\curveto(863.42515856,667.09221196)(863.45015853,667.17721187)(863.50015962,667.23721942)
\curveto(863.53015845,667.28721176)(863.57015841,667.31721173)(863.62015962,667.32721942)
\curveto(863.67015831,667.33721171)(863.72515826,667.3522117)(863.78515962,667.37221942)
\curveto(863.80515818,667.37221168)(863.82515816,667.36721168)(863.84515962,667.35721942)
\curveto(863.87515811,667.35721169)(863.90015808,667.36221169)(863.92015962,667.37221942)
\curveto(864.05015793,667.37221168)(864.1801578,667.36721168)(864.31015962,667.35721942)
\curveto(864.45015753,667.35721169)(864.54515744,667.31721173)(864.59515962,667.23721942)
\curveto(864.64515734,667.17721187)(864.67015731,667.09721195)(864.67015962,666.99721942)
\lineto(864.67015962,666.71221942)
\lineto(864.67015962,660.36721942)
}
}
{
\newrgbcolor{curcolor}{0 0 0}
\pscustom[linestyle=none,fillstyle=solid,fillcolor=curcolor]
{
\newpath
\moveto(866.39000337,667.37221942)
\lineto(866.87000337,667.37221942)
\curveto(867.04000203,667.37221168)(867.1700019,667.34221171)(867.26000337,667.28221942)
\curveto(867.33000174,667.23221182)(867.3750017,667.16721188)(867.39500337,667.08721942)
\curveto(867.42500165,667.01721203)(867.45500162,666.94221211)(867.48500337,666.86221942)
\curveto(867.54500153,666.72221233)(867.59500148,666.58221247)(867.63500337,666.44221942)
\curveto(867.6750014,666.30221275)(867.72000135,666.16221289)(867.77000337,666.02221942)
\curveto(867.9700011,665.48221357)(868.15500092,664.93721411)(868.32500337,664.38721942)
\curveto(868.49500058,663.8472152)(868.68000039,663.30721574)(868.88000337,662.76721942)
\curveto(868.95000012,662.58721646)(869.01000006,662.40221665)(869.06000337,662.21221942)
\curveto(869.10999996,662.03221702)(869.1749999,661.8522172)(869.25500337,661.67221942)
\curveto(869.2749998,661.60221745)(869.29999977,661.52721752)(869.33000337,661.44721942)
\curveto(869.35999971,661.36721768)(869.40999966,661.31721773)(869.48000337,661.29721942)
\curveto(869.55999951,661.27721777)(869.61999945,661.31221774)(869.66000337,661.40221942)
\curveto(869.70999936,661.49221756)(869.74499933,661.56221749)(869.76500337,661.61221942)
\curveto(869.84499923,661.80221725)(869.90999916,661.99221706)(869.96000337,662.18221942)
\curveto(870.01999905,662.38221667)(870.08499899,662.58221647)(870.15500337,662.78221942)
\curveto(870.28499879,663.16221589)(870.40999866,663.53721551)(870.53000337,663.90721942)
\curveto(870.64999842,664.28721476)(870.7749983,664.66721438)(870.90500337,665.04721942)
\curveto(870.95499812,665.21721383)(871.00499807,665.38221367)(871.05500337,665.54221942)
\curveto(871.10499797,665.71221334)(871.16499791,665.87721317)(871.23500337,666.03721942)
\curveto(871.28499779,666.17721287)(871.32999774,666.31721273)(871.37000337,666.45721942)
\curveto(871.40999766,666.59721245)(871.45499762,666.73721231)(871.50500337,666.87721942)
\curveto(871.52499755,666.9472121)(871.54999752,667.01721203)(871.58000337,667.08721942)
\curveto(871.60999746,667.15721189)(871.64999742,667.21721183)(871.70000337,667.26721942)
\curveto(871.77999729,667.31721173)(871.8699972,667.3472117)(871.97000337,667.35721942)
\curveto(872.069997,667.36721168)(872.18999688,667.37221168)(872.33000337,667.37221942)
\curveto(872.39999667,667.37221168)(872.46499661,667.36721168)(872.52500337,667.35721942)
\curveto(872.58499649,667.35721169)(872.63999643,667.3472117)(872.69000337,667.32721942)
\curveto(872.77999629,667.28721176)(872.82499625,667.22221183)(872.82500337,667.13221942)
\curveto(872.83499624,667.04221201)(872.81999625,666.9522121)(872.78000337,666.86221942)
\curveto(872.71999635,666.69221236)(872.65999641,666.51721253)(872.60000337,666.33721942)
\curveto(872.53999653,666.15721289)(872.4699966,665.98221307)(872.39000337,665.81221942)
\curveto(872.3699967,665.76221329)(872.35499672,665.71221334)(872.34500337,665.66221942)
\curveto(872.33499674,665.62221343)(872.31999675,665.57721347)(872.30000337,665.52721942)
\curveto(872.21999685,665.35721369)(872.15499692,665.18221387)(872.10500337,665.00221942)
\curveto(872.05499702,664.82221423)(871.98999708,664.64221441)(871.91000337,664.46221942)
\curveto(871.85999721,664.33221472)(871.80999726,664.19721485)(871.76000337,664.05721942)
\curveto(871.71999735,663.92721512)(871.6699974,663.79721525)(871.61000337,663.66721942)
\curveto(871.43999763,663.25721579)(871.28499779,662.84221621)(871.14500337,662.42221942)
\curveto(871.01499806,662.00221705)(870.86499821,661.58721746)(870.69500337,661.17721942)
\curveto(870.63499844,661.01721803)(870.57999849,660.85721819)(870.53000337,660.69721942)
\curveto(870.47999859,660.53721851)(870.41999865,660.37721867)(870.35000337,660.21721942)
\curveto(870.29999877,660.10721894)(870.25499882,660.00221905)(870.21500337,659.90221942)
\curveto(870.18499889,659.81221924)(870.11499896,659.74221931)(870.00500337,659.69221942)
\curveto(869.94499913,659.66221939)(869.8749992,659.6472194)(869.79500337,659.64721942)
\lineto(869.57000337,659.64721942)
\lineto(869.10500337,659.64721942)
\curveto(868.95500012,659.65721939)(868.84500023,659.70721934)(868.77500337,659.79721942)
\curveto(868.70500037,659.87721917)(868.65500042,659.97221908)(868.62500337,660.08221942)
\curveto(868.59500048,660.20221885)(868.55500052,660.31721873)(868.50500337,660.42721942)
\curveto(868.44500063,660.56721848)(868.38500069,660.71221834)(868.32500337,660.86221942)
\curveto(868.2750008,661.02221803)(868.22500085,661.17221788)(868.17500337,661.31221942)
\curveto(868.15500092,661.36221769)(868.14000093,661.40221765)(868.13000337,661.43221942)
\curveto(868.12000095,661.47221758)(868.10500097,661.51721753)(868.08500337,661.56721942)
\curveto(867.88500119,662.047217)(867.70000137,662.53221652)(867.53000337,663.02221942)
\curveto(867.3700017,663.51221554)(867.19000188,663.99721505)(866.99000337,664.47721942)
\curveto(866.93000214,664.63721441)(866.8700022,664.79221426)(866.81000337,664.94221942)
\curveto(866.76000231,665.10221395)(866.70500237,665.26221379)(866.64500337,665.42221942)
\lineto(866.58500337,665.57221942)
\curveto(866.5750025,665.63221342)(866.56000251,665.68721336)(866.54000337,665.73721942)
\curveto(866.46000261,665.90721314)(866.39000268,666.07721297)(866.33000337,666.24721942)
\curveto(866.28000279,666.41721263)(866.22000285,666.58721246)(866.15000337,666.75721942)
\curveto(866.13000294,666.81721223)(866.10500297,666.89721215)(866.07500337,666.99721942)
\curveto(866.04500303,667.09721195)(866.05000302,667.18221187)(866.09000337,667.25221942)
\curveto(866.14000293,667.30221175)(866.20000287,667.33721171)(866.27000337,667.35721942)
\curveto(866.34000273,667.35721169)(866.38000269,667.36221169)(866.39000337,667.37221942)
}
}
{
\newrgbcolor{curcolor}{0 0 0}
\pscustom[linestyle=none,fillstyle=solid,fillcolor=curcolor]
{
\newpath
\moveto(881.24000337,663.84721942)
\curveto(881.25999531,663.78721526)(881.2699953,663.69221536)(881.27000337,663.56221942)
\curveto(881.2699953,663.44221561)(881.26499531,663.35721569)(881.25500337,663.30721942)
\lineto(881.25500337,663.15721942)
\curveto(881.24499533,663.07721597)(881.23499534,663.00221605)(881.22500337,662.93221942)
\curveto(881.22499535,662.87221618)(881.21999535,662.80221625)(881.21000337,662.72221942)
\curveto(881.18999538,662.66221639)(881.1749954,662.60221645)(881.16500337,662.54221942)
\curveto(881.16499541,662.48221657)(881.15499542,662.42221663)(881.13500337,662.36221942)
\curveto(881.09499548,662.23221682)(881.05999551,662.10221695)(881.03000337,661.97221942)
\curveto(880.99999557,661.84221721)(880.95999561,661.72221733)(880.91000337,661.61221942)
\curveto(880.69999587,661.13221792)(880.41999615,660.72721832)(880.07000337,660.39721942)
\curveto(879.71999685,660.07721897)(879.28999728,659.83221922)(878.78000337,659.66221942)
\curveto(878.6699979,659.62221943)(878.54999802,659.59221946)(878.42000337,659.57221942)
\curveto(878.29999827,659.5522195)(878.1749984,659.53221952)(878.04500337,659.51221942)
\curveto(877.98499859,659.50221955)(877.91999865,659.49721955)(877.85000337,659.49721942)
\curveto(877.78999878,659.48721956)(877.72999884,659.48221957)(877.67000337,659.48221942)
\curveto(877.62999894,659.47221958)(877.569999,659.46721958)(877.49000337,659.46721942)
\curveto(877.41999915,659.46721958)(877.3699992,659.47221958)(877.34000337,659.48221942)
\curveto(877.29999927,659.49221956)(877.25999931,659.49721955)(877.22000337,659.49721942)
\curveto(877.17999939,659.48721956)(877.14499943,659.48721956)(877.11500337,659.49721942)
\lineto(877.02500337,659.49721942)
\lineto(876.66500337,659.54221942)
\curveto(876.52500005,659.58221947)(876.39000018,659.62221943)(876.26000337,659.66221942)
\curveto(876.13000044,659.70221935)(876.00500057,659.7472193)(875.88500337,659.79721942)
\curveto(875.43500114,659.99721905)(875.06500151,660.25721879)(874.77500337,660.57721942)
\curveto(874.48500209,660.89721815)(874.24500233,661.28721776)(874.05500337,661.74721942)
\curveto(874.00500257,661.8472172)(873.96500261,661.9472171)(873.93500337,662.04721942)
\curveto(873.91500266,662.1472169)(873.89500268,662.2522168)(873.87500337,662.36221942)
\curveto(873.85500272,662.40221665)(873.84500273,662.43221662)(873.84500337,662.45221942)
\curveto(873.85500272,662.48221657)(873.85500272,662.51721653)(873.84500337,662.55721942)
\curveto(873.82500275,662.63721641)(873.81000276,662.71721633)(873.80000337,662.79721942)
\curveto(873.80000277,662.88721616)(873.79000278,662.97221608)(873.77000337,663.05221942)
\lineto(873.77000337,663.17221942)
\curveto(873.7700028,663.21221584)(873.76500281,663.25721579)(873.75500337,663.30721942)
\curveto(873.74500283,663.35721569)(873.74000283,663.44221561)(873.74000337,663.56221942)
\curveto(873.74000283,663.69221536)(873.75000282,663.78721526)(873.77000337,663.84721942)
\curveto(873.79000278,663.91721513)(873.79500278,663.98721506)(873.78500337,664.05721942)
\curveto(873.7750028,664.12721492)(873.78000279,664.19721485)(873.80000337,664.26721942)
\curveto(873.81000276,664.31721473)(873.81500276,664.35721469)(873.81500337,664.38721942)
\curveto(873.82500275,664.42721462)(873.83500274,664.47221458)(873.84500337,664.52221942)
\curveto(873.8750027,664.64221441)(873.90000267,664.76221429)(873.92000337,664.88221942)
\curveto(873.95000262,665.00221405)(873.99000258,665.11721393)(874.04000337,665.22721942)
\curveto(874.19000238,665.59721345)(874.3700022,665.92721312)(874.58000337,666.21721942)
\curveto(874.80000177,666.51721253)(875.06500151,666.76721228)(875.37500337,666.96721942)
\curveto(875.49500108,667.047212)(875.62000095,667.11221194)(875.75000337,667.16221942)
\curveto(875.88000069,667.22221183)(876.01500056,667.28221177)(876.15500337,667.34221942)
\curveto(876.2750003,667.39221166)(876.40500017,667.42221163)(876.54500337,667.43221942)
\curveto(876.68499989,667.4522116)(876.82499975,667.48221157)(876.96500337,667.52221942)
\lineto(877.16000337,667.52221942)
\curveto(877.22999934,667.53221152)(877.29499928,667.54221151)(877.35500337,667.55221942)
\curveto(878.24499833,667.56221149)(878.98499759,667.37721167)(879.57500337,666.99721942)
\curveto(880.16499641,666.61721243)(880.58999598,666.12221293)(880.85000337,665.51221942)
\curveto(880.89999567,665.41221364)(880.93999563,665.31221374)(880.97000337,665.21221942)
\curveto(880.99999557,665.11221394)(881.03499554,665.00721404)(881.07500337,664.89721942)
\curveto(881.10499547,664.78721426)(881.12999544,664.66721438)(881.15000337,664.53721942)
\curveto(881.1699954,664.41721463)(881.19499538,664.29221476)(881.22500337,664.16221942)
\curveto(881.23499534,664.11221494)(881.23499534,664.05721499)(881.22500337,663.99721942)
\curveto(881.22499535,663.9472151)(881.22999534,663.89721515)(881.24000337,663.84721942)
\moveto(879.90500337,662.99221942)
\curveto(879.92499665,663.06221599)(879.92999664,663.14221591)(879.92000337,663.23221942)
\lineto(879.92000337,663.48721942)
\curveto(879.91999665,663.87721517)(879.88499669,664.20721484)(879.81500337,664.47721942)
\curveto(879.78499679,664.55721449)(879.75999681,664.63721441)(879.74000337,664.71721942)
\curveto(879.71999685,664.79721425)(879.69499688,664.87221418)(879.66500337,664.94221942)
\curveto(879.38499719,665.59221346)(878.93999763,666.04221301)(878.33000337,666.29221942)
\curveto(878.25999831,666.32221273)(878.18499839,666.34221271)(878.10500337,666.35221942)
\lineto(877.86500337,666.41221942)
\curveto(877.78499879,666.43221262)(877.69999887,666.44221261)(877.61000337,666.44221942)
\lineto(877.34000337,666.44221942)
\lineto(877.07000337,666.39721942)
\curveto(876.9699996,666.37721267)(876.8749997,666.3522127)(876.78500337,666.32221942)
\curveto(876.70499987,666.30221275)(876.62499995,666.27221278)(876.54500337,666.23221942)
\curveto(876.4750001,666.21221284)(876.41000016,666.18221287)(876.35000337,666.14221942)
\curveto(876.29000028,666.10221295)(876.23500034,666.06221299)(876.18500337,666.02221942)
\curveto(875.94500063,665.8522132)(875.75000082,665.6472134)(875.60000337,665.40721942)
\curveto(875.45000112,665.16721388)(875.32000125,664.88721416)(875.21000337,664.56721942)
\curveto(875.18000139,664.46721458)(875.16000141,664.36221469)(875.15000337,664.25221942)
\curveto(875.14000143,664.1522149)(875.12500145,664.047215)(875.10500337,663.93721942)
\curveto(875.09500148,663.89721515)(875.09000148,663.83221522)(875.09000337,663.74221942)
\curveto(875.08000149,663.71221534)(875.0750015,663.67721537)(875.07500337,663.63721942)
\curveto(875.08500149,663.59721545)(875.09000148,663.5522155)(875.09000337,663.50221942)
\lineto(875.09000337,663.20221942)
\curveto(875.09000148,663.10221595)(875.10000147,663.01221604)(875.12000337,662.93221942)
\lineto(875.15000337,662.75221942)
\curveto(875.1700014,662.6522164)(875.18500139,662.5522165)(875.19500337,662.45221942)
\curveto(875.21500136,662.36221669)(875.24500133,662.27721677)(875.28500337,662.19721942)
\curveto(875.38500119,661.95721709)(875.50000107,661.73221732)(875.63000337,661.52221942)
\curveto(875.7700008,661.31221774)(875.94000063,661.13721791)(876.14000337,660.99721942)
\curveto(876.19000038,660.96721808)(876.23500034,660.94221811)(876.27500337,660.92221942)
\curveto(876.31500026,660.90221815)(876.36000021,660.87721817)(876.41000337,660.84721942)
\curveto(876.49000008,660.79721825)(876.575,660.7522183)(876.66500337,660.71221942)
\curveto(876.76499981,660.68221837)(876.8699997,660.6522184)(876.98000337,660.62221942)
\curveto(877.02999954,660.60221845)(877.0749995,660.59221846)(877.11500337,660.59221942)
\curveto(877.16499941,660.60221845)(877.21499936,660.60221845)(877.26500337,660.59221942)
\curveto(877.29499928,660.58221847)(877.35499922,660.57221848)(877.44500337,660.56221942)
\curveto(877.54499903,660.5522185)(877.61999895,660.55721849)(877.67000337,660.57721942)
\curveto(877.70999886,660.58721846)(877.74999882,660.58721846)(877.79000337,660.57721942)
\curveto(877.82999874,660.57721847)(877.8699987,660.58721846)(877.91000337,660.60721942)
\curveto(877.98999858,660.62721842)(878.0699985,660.64221841)(878.15000337,660.65221942)
\curveto(878.22999834,660.67221838)(878.30499827,660.69721835)(878.37500337,660.72721942)
\curveto(878.71499786,660.86721818)(878.98999758,661.06221799)(879.20000337,661.31221942)
\curveto(879.40999716,661.56221749)(879.58499699,661.85721719)(879.72500337,662.19721942)
\curveto(879.7749968,662.31721673)(879.80499677,662.44221661)(879.81500337,662.57221942)
\curveto(879.83499674,662.71221634)(879.86499671,662.8522162)(879.90500337,662.99221942)
}
}
{
\newrgbcolor{curcolor}{0 0 0}
\pscustom[linestyle=none,fillstyle=solid,fillcolor=curcolor]
{
\newpath
\moveto(885.15828462,667.55221942)
\curveto(885.87828056,667.56221149)(886.48327995,667.47721157)(886.97328462,667.29721942)
\curveto(887.46327897,667.12721192)(887.84327859,666.82221223)(888.11328462,666.38221942)
\curveto(888.18327825,666.27221278)(888.2382782,666.15721289)(888.27828462,666.03721942)
\curveto(888.31827812,665.92721312)(888.35827808,665.80221325)(888.39828462,665.66221942)
\curveto(888.41827802,665.59221346)(888.42327801,665.51721353)(888.41328462,665.43721942)
\curveto(888.40327803,665.36721368)(888.38827805,665.31221374)(888.36828462,665.27221942)
\curveto(888.34827809,665.2522138)(888.32327811,665.23221382)(888.29328462,665.21221942)
\curveto(888.26327817,665.20221385)(888.2382782,665.18721386)(888.21828462,665.16721942)
\curveto(888.16827827,665.1472139)(888.11827832,665.14221391)(888.06828462,665.15221942)
\curveto(888.01827842,665.16221389)(887.96827847,665.16221389)(887.91828462,665.15221942)
\curveto(887.8382786,665.13221392)(887.7332787,665.12721392)(887.60328462,665.13721942)
\curveto(887.47327896,665.15721389)(887.38327905,665.18221387)(887.33328462,665.21221942)
\curveto(887.25327918,665.26221379)(887.19827924,665.32721372)(887.16828462,665.40721942)
\curveto(887.14827929,665.49721355)(887.11327932,665.58221347)(887.06328462,665.66221942)
\curveto(886.97327946,665.82221323)(886.84827959,665.96721308)(886.68828462,666.09721942)
\curveto(886.57827986,666.17721287)(886.45827998,666.23721281)(886.32828462,666.27721942)
\curveto(886.19828024,666.31721273)(886.05828038,666.35721269)(885.90828462,666.39721942)
\curveto(885.85828058,666.41721263)(885.80828063,666.42221263)(885.75828462,666.41221942)
\curveto(885.70828073,666.41221264)(885.65828078,666.41721263)(885.60828462,666.42721942)
\curveto(885.54828089,666.4472126)(885.47328096,666.45721259)(885.38328462,666.45721942)
\curveto(885.29328114,666.45721259)(885.21828122,666.4472126)(885.15828462,666.42721942)
\lineto(885.06828462,666.42721942)
\lineto(884.91828462,666.39721942)
\curveto(884.86828157,666.39721265)(884.81828162,666.39221266)(884.76828462,666.38221942)
\curveto(884.50828193,666.32221273)(884.29328214,666.23721281)(884.12328462,666.12721942)
\curveto(883.95328248,666.01721303)(883.8382826,665.83221322)(883.77828462,665.57221942)
\curveto(883.75828268,665.50221355)(883.75328268,665.43221362)(883.76328462,665.36221942)
\curveto(883.78328265,665.29221376)(883.80328263,665.23221382)(883.82328462,665.18221942)
\curveto(883.88328255,665.03221402)(883.95328248,664.92221413)(884.03328462,664.85221942)
\curveto(884.12328231,664.79221426)(884.2332822,664.72221433)(884.36328462,664.64221942)
\curveto(884.52328191,664.54221451)(884.70328173,664.46721458)(884.90328462,664.41721942)
\curveto(885.10328133,664.37721467)(885.30328113,664.32721472)(885.50328462,664.26721942)
\curveto(885.6332808,664.22721482)(885.76328067,664.19721485)(885.89328462,664.17721942)
\curveto(886.02328041,664.15721489)(886.15328028,664.12721492)(886.28328462,664.08721942)
\curveto(886.49327994,664.02721502)(886.69827974,663.96721508)(886.89828462,663.90721942)
\curveto(887.09827934,663.85721519)(887.29827914,663.79221526)(887.49828462,663.71221942)
\lineto(887.64828462,663.65221942)
\curveto(887.69827874,663.63221542)(887.74827869,663.60721544)(887.79828462,663.57721942)
\curveto(887.99827844,663.45721559)(888.17327826,663.32221573)(888.32328462,663.17221942)
\curveto(888.47327796,663.02221603)(888.59827784,662.83221622)(888.69828462,662.60221942)
\curveto(888.71827772,662.53221652)(888.7382777,662.43721661)(888.75828462,662.31721942)
\curveto(888.77827766,662.2472168)(888.78827765,662.17221688)(888.78828462,662.09221942)
\curveto(888.79827764,662.02221703)(888.80327763,661.94221711)(888.80328462,661.85221942)
\lineto(888.80328462,661.70221942)
\curveto(888.78327765,661.63221742)(888.77327766,661.56221749)(888.77328462,661.49221942)
\curveto(888.77327766,661.42221763)(888.76327767,661.3522177)(888.74328462,661.28221942)
\curveto(888.71327772,661.17221788)(888.67827776,661.06721798)(888.63828462,660.96721942)
\curveto(888.59827784,660.86721818)(888.55327788,660.77721827)(888.50328462,660.69721942)
\curveto(888.34327809,660.43721861)(888.1382783,660.22721882)(887.88828462,660.06721942)
\curveto(887.6382788,659.91721913)(887.35827908,659.78721926)(887.04828462,659.67721942)
\curveto(886.95827948,659.6472194)(886.86327957,659.62721942)(886.76328462,659.61721942)
\curveto(886.67327976,659.59721945)(886.58327985,659.57221948)(886.49328462,659.54221942)
\curveto(886.39328004,659.52221953)(886.29328014,659.51221954)(886.19328462,659.51221942)
\curveto(886.09328034,659.51221954)(885.99328044,659.50221955)(885.89328462,659.48221942)
\lineto(885.74328462,659.48221942)
\curveto(885.69328074,659.47221958)(885.62328081,659.46721958)(885.53328462,659.46721942)
\curveto(885.44328099,659.46721958)(885.37328106,659.47221958)(885.32328462,659.48221942)
\lineto(885.15828462,659.48221942)
\curveto(885.09828134,659.50221955)(885.0332814,659.51221954)(884.96328462,659.51221942)
\curveto(884.89328154,659.50221955)(884.8332816,659.50721954)(884.78328462,659.52721942)
\curveto(884.7332817,659.53721951)(884.66828177,659.54221951)(884.58828462,659.54221942)
\lineto(884.34828462,659.60221942)
\curveto(884.27828216,659.61221944)(884.20328223,659.63221942)(884.12328462,659.66221942)
\curveto(883.81328262,659.76221929)(883.54328289,659.88721916)(883.31328462,660.03721942)
\curveto(883.08328335,660.18721886)(882.88328355,660.38221867)(882.71328462,660.62221942)
\curveto(882.62328381,660.7522183)(882.54828389,660.88721816)(882.48828462,661.02721942)
\curveto(882.42828401,661.16721788)(882.37328406,661.32221773)(882.32328462,661.49221942)
\curveto(882.30328413,661.5522175)(882.29328414,661.62221743)(882.29328462,661.70221942)
\curveto(882.30328413,661.79221726)(882.31828412,661.86221719)(882.33828462,661.91221942)
\curveto(882.36828407,661.9522171)(882.41828402,661.99221706)(882.48828462,662.03221942)
\curveto(882.5382839,662.052217)(882.60828383,662.06221699)(882.69828462,662.06221942)
\curveto(882.78828365,662.07221698)(882.87828356,662.07221698)(882.96828462,662.06221942)
\curveto(883.05828338,662.052217)(883.14328329,662.03721701)(883.22328462,662.01721942)
\curveto(883.31328312,662.00721704)(883.37328306,661.99221706)(883.40328462,661.97221942)
\curveto(883.47328296,661.92221713)(883.51828292,661.8472172)(883.53828462,661.74721942)
\curveto(883.56828287,661.65721739)(883.60328283,661.57221748)(883.64328462,661.49221942)
\curveto(883.74328269,661.27221778)(883.87828256,661.10221795)(884.04828462,660.98221942)
\curveto(884.16828227,660.89221816)(884.30328213,660.82221823)(884.45328462,660.77221942)
\curveto(884.60328183,660.72221833)(884.76328167,660.67221838)(884.93328462,660.62221942)
\lineto(885.24828462,660.57721942)
\lineto(885.33828462,660.57721942)
\curveto(885.40828103,660.55721849)(885.49828094,660.5472185)(885.60828462,660.54721942)
\curveto(885.72828071,660.5472185)(885.82828061,660.55721849)(885.90828462,660.57721942)
\curveto(885.97828046,660.57721847)(886.0332804,660.58221847)(886.07328462,660.59221942)
\curveto(886.1332803,660.60221845)(886.19328024,660.60721844)(886.25328462,660.60721942)
\curveto(886.31328012,660.61721843)(886.36828007,660.62721842)(886.41828462,660.63721942)
\curveto(886.70827973,660.71721833)(886.9382795,660.82221823)(887.10828462,660.95221942)
\curveto(887.27827916,661.08221797)(887.39827904,661.30221775)(887.46828462,661.61221942)
\curveto(887.48827895,661.66221739)(887.49327894,661.71721733)(887.48328462,661.77721942)
\curveto(887.47327896,661.83721721)(887.46327897,661.88221717)(887.45328462,661.91221942)
\curveto(887.40327903,662.10221695)(887.3332791,662.24221681)(887.24328462,662.33221942)
\curveto(887.15327928,662.43221662)(887.0382794,662.52221653)(886.89828462,662.60221942)
\curveto(886.80827963,662.66221639)(886.70827973,662.71221634)(886.59828462,662.75221942)
\lineto(886.26828462,662.87221942)
\curveto(886.2382802,662.88221617)(886.20828023,662.88721616)(886.17828462,662.88721942)
\curveto(886.15828028,662.88721616)(886.1332803,662.89721615)(886.10328462,662.91721942)
\curveto(885.76328067,663.02721602)(885.40828103,663.10721594)(885.03828462,663.15721942)
\curveto(884.67828176,663.21721583)(884.3382821,663.31221574)(884.01828462,663.44221942)
\curveto(883.91828252,663.48221557)(883.82328261,663.51721553)(883.73328462,663.54721942)
\curveto(883.64328279,663.57721547)(883.55828288,663.61721543)(883.47828462,663.66721942)
\curveto(883.28828315,663.77721527)(883.11328332,663.90221515)(882.95328462,664.04221942)
\curveto(882.79328364,664.18221487)(882.66828377,664.35721469)(882.57828462,664.56721942)
\curveto(882.54828389,664.63721441)(882.52328391,664.70721434)(882.50328462,664.77721942)
\curveto(882.49328394,664.8472142)(882.47828396,664.92221413)(882.45828462,665.00221942)
\curveto(882.42828401,665.12221393)(882.41828402,665.25721379)(882.42828462,665.40721942)
\curveto(882.438284,665.56721348)(882.45328398,665.70221335)(882.47328462,665.81221942)
\curveto(882.49328394,665.86221319)(882.50328393,665.90221315)(882.50328462,665.93221942)
\curveto(882.51328392,665.97221308)(882.52828391,666.01221304)(882.54828462,666.05221942)
\curveto(882.6382838,666.28221277)(882.75828368,666.48221257)(882.90828462,666.65221942)
\curveto(883.06828337,666.82221223)(883.24828319,666.97221208)(883.44828462,667.10221942)
\curveto(883.59828284,667.19221186)(883.76328267,667.26221179)(883.94328462,667.31221942)
\curveto(884.12328231,667.37221168)(884.31328212,667.42721162)(884.51328462,667.47721942)
\curveto(884.58328185,667.48721156)(884.64828179,667.49721155)(884.70828462,667.50721942)
\curveto(884.77828166,667.51721153)(884.85328158,667.52721152)(884.93328462,667.53721942)
\curveto(884.96328147,667.5472115)(885.00328143,667.5472115)(885.05328462,667.53721942)
\curveto(885.10328133,667.52721152)(885.1382813,667.53221152)(885.15828462,667.55221942)
}
}
{
\newrgbcolor{curcolor}{0.80000001 0.80000001 0.80000001}
\pscustom[linestyle=none,fillstyle=solid,fillcolor=curcolor]
{
\newpath
\moveto(812.80437349,670.35725604)
\lineto(827.80437349,670.35725604)
\lineto(827.80437349,655.35725604)
\lineto(812.80437349,655.35725604)
\closepath
}
}
{
\newrgbcolor{curcolor}{0 0 0}
\pscustom[linestyle=none,fillstyle=solid,fillcolor=curcolor]
{
\newpath
\moveto(832.7925815,647.47004168)
\lineto(837.6975815,647.47004168)
\lineto(838.9875815,647.47004168)
\curveto(839.09757362,647.47003099)(839.20757351,647.47003099)(839.3175815,647.47004168)
\curveto(839.42757329,647.48003098)(839.5175732,647.460031)(839.5875815,647.41004168)
\curveto(839.6175731,647.39003107)(839.64257307,647.36503109)(839.6625815,647.33504168)
\curveto(839.68257303,647.30503115)(839.70257301,647.27503118)(839.7225815,647.24504168)
\curveto(839.74257297,647.17503128)(839.75257296,647.0600314)(839.7525815,646.90004168)
\curveto(839.75257296,646.75003171)(839.74257297,646.63503182)(839.7225815,646.55504168)
\curveto(839.68257303,646.41503204)(839.59757312,646.33503212)(839.4675815,646.31504168)
\curveto(839.33757338,646.30503215)(839.18257353,646.30003216)(839.0025815,646.30004168)
\lineto(837.5025815,646.30004168)
\lineto(834.9825815,646.30004168)
\lineto(834.4125815,646.30004168)
\curveto(834.20257851,646.31003215)(834.04757867,646.28503217)(833.9475815,646.22504168)
\curveto(833.84757887,646.16503229)(833.79257892,646.0600324)(833.7825815,645.91004168)
\lineto(833.7825815,645.44504168)
\lineto(833.7825815,643.91504168)
\curveto(833.78257893,643.80503465)(833.77757894,643.67503478)(833.7675815,643.52504168)
\curveto(833.76757895,643.37503508)(833.77757894,643.2550352)(833.7975815,643.16504168)
\curveto(833.82757889,643.04503541)(833.88757883,642.96503549)(833.9775815,642.92504168)
\curveto(834.0175787,642.90503555)(834.08757863,642.88503557)(834.1875815,642.86504168)
\lineto(834.3375815,642.86504168)
\curveto(834.37757834,642.8550356)(834.4175783,642.85003561)(834.4575815,642.85004168)
\curveto(834.50757821,642.8600356)(834.55757816,642.86503559)(834.6075815,642.86504168)
\lineto(835.1175815,642.86504168)
\lineto(838.0575815,642.86504168)
\lineto(838.3575815,642.86504168)
\curveto(838.46757425,642.87503558)(838.57757414,642.87503558)(838.6875815,642.86504168)
\curveto(838.80757391,642.86503559)(838.9125738,642.8550356)(839.0025815,642.83504168)
\curveto(839.10257361,642.82503563)(839.17757354,642.80503565)(839.2275815,642.77504168)
\curveto(839.25757346,642.7550357)(839.28257343,642.71003575)(839.3025815,642.64004168)
\curveto(839.32257339,642.57003589)(839.33757338,642.49503596)(839.3475815,642.41504168)
\curveto(839.35757336,642.33503612)(839.35757336,642.25003621)(839.3475815,642.16004168)
\curveto(839.34757337,642.08003638)(839.33757338,642.01003645)(839.3175815,641.95004168)
\curveto(839.29757342,641.8600366)(839.25257346,641.79503666)(839.1825815,641.75504168)
\curveto(839.16257355,641.73503672)(839.13257358,641.72003674)(839.0925815,641.71004168)
\curveto(839.06257365,641.71003675)(839.03257368,641.70503675)(839.0025815,641.69504168)
\lineto(838.9125815,641.69504168)
\curveto(838.86257385,641.68503677)(838.8125739,641.68003678)(838.7625815,641.68004168)
\curveto(838.712574,641.69003677)(838.66257405,641.69503676)(838.6125815,641.69504168)
\lineto(838.0575815,641.69504168)
\lineto(834.8925815,641.69504168)
\lineto(834.5325815,641.69504168)
\curveto(834.42257829,641.70503675)(834.3175784,641.70003676)(834.2175815,641.68004168)
\curveto(834.1175786,641.67003679)(834.02757869,641.64503681)(833.9475815,641.60504168)
\curveto(833.87757884,641.56503689)(833.82757889,641.49503696)(833.7975815,641.39504168)
\curveto(833.77757894,641.33503712)(833.76757895,641.26503719)(833.7675815,641.18504168)
\curveto(833.77757894,641.10503735)(833.78257893,641.02503743)(833.7825815,640.94504168)
\lineto(833.7825815,640.10504168)
\lineto(833.7825815,638.68004168)
\curveto(833.78257893,638.54003992)(833.78757893,638.41004005)(833.7975815,638.29004168)
\curveto(833.80757891,638.18004028)(833.84757887,638.10004036)(833.9175815,638.05004168)
\curveto(833.98757873,638.00004046)(834.06757865,637.97004049)(834.1575815,637.96004168)
\lineto(834.4575815,637.96004168)
\lineto(835.4175815,637.96004168)
\lineto(838.1925815,637.96004168)
\lineto(839.0475815,637.96004168)
\lineto(839.2875815,637.96004168)
\curveto(839.36757335,637.97004049)(839.43757328,637.96504049)(839.4975815,637.94504168)
\curveto(839.6175731,637.90504055)(839.69757302,637.85004061)(839.7375815,637.78004168)
\curveto(839.75757296,637.75004071)(839.77257294,637.70004076)(839.7825815,637.63004168)
\curveto(839.79257292,637.5600409)(839.79757292,637.48504097)(839.7975815,637.40504168)
\curveto(839.80757291,637.33504112)(839.80757291,637.2600412)(839.7975815,637.18004168)
\curveto(839.78757293,637.11004135)(839.77757294,637.0550414)(839.7675815,637.01504168)
\curveto(839.72757299,636.93504152)(839.68257303,636.88004158)(839.6325815,636.85004168)
\curveto(839.57257314,636.81004165)(839.49257322,636.79004167)(839.3925815,636.79004168)
\lineto(839.1225815,636.79004168)
\lineto(838.0725815,636.79004168)
\lineto(834.0825815,636.79004168)
\lineto(833.0325815,636.79004168)
\curveto(832.89257982,636.79004167)(832.77257994,636.79504166)(832.6725815,636.80504168)
\curveto(832.57258014,636.82504163)(832.49758022,636.87504158)(832.4475815,636.95504168)
\curveto(832.40758031,637.01504144)(832.38758033,637.09004137)(832.3875815,637.18004168)
\lineto(832.3875815,637.46504168)
\lineto(832.3875815,638.51504168)
\lineto(832.3875815,642.53504168)
\lineto(832.3875815,645.89504168)
\lineto(832.3875815,646.82504168)
\lineto(832.3875815,647.09504168)
\curveto(832.38758033,647.18503127)(832.40758031,647.2550312)(832.4475815,647.30504168)
\curveto(832.48758023,647.37503108)(832.56258015,647.42503103)(832.6725815,647.45504168)
\curveto(832.69258002,647.46503099)(832.71258,647.46503099)(832.7325815,647.45504168)
\curveto(832.75257996,647.455031)(832.77257994,647.460031)(832.7925815,647.47004168)
}
}
{
\newrgbcolor{curcolor}{0 0 0}
\pscustom[linestyle=none,fillstyle=solid,fillcolor=curcolor]
{
\newpath
\moveto(841.07750337,644.51504168)
\lineto(841.55750337,644.51504168)
\curveto(841.72750203,644.51503394)(841.8575019,644.48503397)(841.94750337,644.42504168)
\curveto(842.01750174,644.37503408)(842.0625017,644.31003415)(842.08250337,644.23004168)
\curveto(842.11250165,644.1600343)(842.14250162,644.08503437)(842.17250337,644.00504168)
\curveto(842.23250153,643.86503459)(842.28250148,643.72503473)(842.32250337,643.58504168)
\curveto(842.3625014,643.44503501)(842.40750135,643.30503515)(842.45750337,643.16504168)
\curveto(842.6575011,642.62503583)(842.84250092,642.08003638)(843.01250337,641.53004168)
\curveto(843.18250058,640.99003747)(843.36750039,640.45003801)(843.56750337,639.91004168)
\curveto(843.63750012,639.73003873)(843.69750006,639.54503891)(843.74750337,639.35504168)
\curveto(843.79749996,639.17503928)(843.8624999,638.99503946)(843.94250337,638.81504168)
\curveto(843.9624998,638.74503971)(843.98749977,638.67003979)(844.01750337,638.59004168)
\curveto(844.04749971,638.51003995)(844.09749966,638.46004)(844.16750337,638.44004168)
\curveto(844.24749951,638.42004004)(844.30749945,638.45504)(844.34750337,638.54504168)
\curveto(844.39749936,638.63503982)(844.43249933,638.70503975)(844.45250337,638.75504168)
\curveto(844.53249923,638.94503951)(844.59749916,639.13503932)(844.64750337,639.32504168)
\curveto(844.70749905,639.52503893)(844.77249899,639.72503873)(844.84250337,639.92504168)
\curveto(844.97249879,640.30503815)(845.09749866,640.68003778)(845.21750337,641.05004168)
\curveto(845.33749842,641.43003703)(845.4624983,641.81003665)(845.59250337,642.19004168)
\curveto(845.64249812,642.3600361)(845.69249807,642.52503593)(845.74250337,642.68504168)
\curveto(845.79249797,642.8550356)(845.85249791,643.02003544)(845.92250337,643.18004168)
\curveto(845.97249779,643.32003514)(846.01749774,643.460035)(846.05750337,643.60004168)
\curveto(846.09749766,643.74003472)(846.14249762,643.88003458)(846.19250337,644.02004168)
\curveto(846.21249755,644.09003437)(846.23749752,644.1600343)(846.26750337,644.23004168)
\curveto(846.29749746,644.30003416)(846.33749742,644.3600341)(846.38750337,644.41004168)
\curveto(846.46749729,644.460034)(846.5574972,644.49003397)(846.65750337,644.50004168)
\curveto(846.757497,644.51003395)(846.87749688,644.51503394)(847.01750337,644.51504168)
\curveto(847.08749667,644.51503394)(847.15249661,644.51003395)(847.21250337,644.50004168)
\curveto(847.27249649,644.50003396)(847.32749643,644.49003397)(847.37750337,644.47004168)
\curveto(847.46749629,644.43003403)(847.51249625,644.36503409)(847.51250337,644.27504168)
\curveto(847.52249624,644.18503427)(847.50749625,644.09503436)(847.46750337,644.00504168)
\curveto(847.40749635,643.83503462)(847.34749641,643.6600348)(847.28750337,643.48004168)
\curveto(847.22749653,643.30003516)(847.1574966,643.12503533)(847.07750337,642.95504168)
\curveto(847.0574967,642.90503555)(847.04249672,642.8550356)(847.03250337,642.80504168)
\curveto(847.02249674,642.76503569)(847.00749675,642.72003574)(846.98750337,642.67004168)
\curveto(846.90749685,642.50003596)(846.84249692,642.32503613)(846.79250337,642.14504168)
\curveto(846.74249702,641.96503649)(846.67749708,641.78503667)(846.59750337,641.60504168)
\curveto(846.54749721,641.47503698)(846.49749726,641.34003712)(846.44750337,641.20004168)
\curveto(846.40749735,641.07003739)(846.3574974,640.94003752)(846.29750337,640.81004168)
\curveto(846.12749763,640.40003806)(845.97249779,639.98503847)(845.83250337,639.56504168)
\curveto(845.70249806,639.14503931)(845.55249821,638.73003973)(845.38250337,638.32004168)
\curveto(845.32249844,638.1600403)(845.26749849,638.00004046)(845.21750337,637.84004168)
\curveto(845.16749859,637.68004078)(845.10749865,637.52004094)(845.03750337,637.36004168)
\curveto(844.98749877,637.25004121)(844.94249882,637.14504131)(844.90250337,637.04504168)
\curveto(844.87249889,636.9550415)(844.80249896,636.88504157)(844.69250337,636.83504168)
\curveto(844.63249913,636.80504165)(844.5624992,636.79004167)(844.48250337,636.79004168)
\lineto(844.25750337,636.79004168)
\lineto(843.79250337,636.79004168)
\curveto(843.64250012,636.80004166)(843.53250023,636.85004161)(843.46250337,636.94004168)
\curveto(843.39250037,637.02004144)(843.34250042,637.11504134)(843.31250337,637.22504168)
\curveto(843.28250048,637.34504111)(843.24250052,637.460041)(843.19250337,637.57004168)
\curveto(843.13250063,637.71004075)(843.07250069,637.8550406)(843.01250337,638.00504168)
\curveto(842.9625008,638.16504029)(842.91250085,638.31504014)(842.86250337,638.45504168)
\curveto(842.84250092,638.50503995)(842.82750093,638.54503991)(842.81750337,638.57504168)
\curveto(842.80750095,638.61503984)(842.79250097,638.6600398)(842.77250337,638.71004168)
\curveto(842.57250119,639.19003927)(842.38750137,639.67503878)(842.21750337,640.16504168)
\curveto(842.0575017,640.6550378)(841.87750188,641.14003732)(841.67750337,641.62004168)
\curveto(841.61750214,641.78003668)(841.5575022,641.93503652)(841.49750337,642.08504168)
\curveto(841.44750231,642.24503621)(841.39250237,642.40503605)(841.33250337,642.56504168)
\lineto(841.27250337,642.71504168)
\curveto(841.2625025,642.77503568)(841.24750251,642.83003563)(841.22750337,642.88004168)
\curveto(841.14750261,643.05003541)(841.07750268,643.22003524)(841.01750337,643.39004168)
\curveto(840.96750279,643.5600349)(840.90750285,643.73003473)(840.83750337,643.90004168)
\curveto(840.81750294,643.9600345)(840.79250297,644.04003442)(840.76250337,644.14004168)
\curveto(840.73250303,644.24003422)(840.73750302,644.32503413)(840.77750337,644.39504168)
\curveto(840.82750293,644.44503401)(840.88750287,644.48003398)(840.95750337,644.50004168)
\curveto(841.02750273,644.50003396)(841.06750269,644.50503395)(841.07750337,644.51504168)
}
}
{
\newrgbcolor{curcolor}{0 0 0}
\pscustom[linestyle=none,fillstyle=solid,fillcolor=curcolor]
{
\newpath
\moveto(855.55250337,640.96004168)
\curveto(855.57249569,640.8600376)(855.57249569,640.74503771)(855.55250337,640.61504168)
\curveto(855.54249572,640.49503796)(855.51249575,640.41003805)(855.46250337,640.36004168)
\curveto(855.41249585,640.32003814)(855.33749592,640.29003817)(855.23750337,640.27004168)
\curveto(855.14749611,640.2600382)(855.04249622,640.2550382)(854.92250337,640.25504168)
\lineto(854.56250337,640.25504168)
\curveto(854.44249682,640.26503819)(854.33749692,640.27003819)(854.24750337,640.27004168)
\lineto(850.40750337,640.27004168)
\curveto(850.32750093,640.27003819)(850.24750101,640.26503819)(850.16750337,640.25504168)
\curveto(850.08750117,640.2550382)(850.02250124,640.24003822)(849.97250337,640.21004168)
\curveto(849.93250133,640.19003827)(849.89250137,640.15003831)(849.85250337,640.09004168)
\curveto(849.83250143,640.0600384)(849.81250145,640.01503844)(849.79250337,639.95504168)
\curveto(849.77250149,639.90503855)(849.77250149,639.8550386)(849.79250337,639.80504168)
\curveto(849.80250146,639.7550387)(849.80750145,639.71003875)(849.80750337,639.67004168)
\curveto(849.80750145,639.63003883)(849.81250145,639.59003887)(849.82250337,639.55004168)
\curveto(849.84250142,639.47003899)(849.8625014,639.38503907)(849.88250337,639.29504168)
\curveto(849.90250136,639.21503924)(849.93250133,639.13503932)(849.97250337,639.05504168)
\curveto(850.20250106,638.51503994)(850.58250068,638.13004033)(851.11250337,637.90004168)
\curveto(851.17250009,637.87004059)(851.23750002,637.84504061)(851.30750337,637.82504168)
\lineto(851.51750337,637.76504168)
\curveto(851.54749971,637.7550407)(851.59749966,637.75004071)(851.66750337,637.75004168)
\curveto(851.80749945,637.71004075)(851.99249927,637.69004077)(852.22250337,637.69004168)
\curveto(852.45249881,637.69004077)(852.63749862,637.71004075)(852.77750337,637.75004168)
\curveto(852.91749834,637.79004067)(853.04249822,637.83004063)(853.15250337,637.87004168)
\curveto(853.27249799,637.92004054)(853.38249788,637.98004048)(853.48250337,638.05004168)
\curveto(853.59249767,638.12004034)(853.68749757,638.20004026)(853.76750337,638.29004168)
\curveto(853.84749741,638.39004007)(853.91749734,638.49503996)(853.97750337,638.60504168)
\curveto(854.03749722,638.70503975)(854.08749717,638.81003965)(854.12750337,638.92004168)
\curveto(854.17749708,639.03003943)(854.257497,639.11003935)(854.36750337,639.16004168)
\curveto(854.40749685,639.18003928)(854.47249679,639.19503926)(854.56250337,639.20504168)
\curveto(854.65249661,639.21503924)(854.74249652,639.21503924)(854.83250337,639.20504168)
\curveto(854.92249634,639.20503925)(855.00749625,639.20003926)(855.08750337,639.19004168)
\curveto(855.16749609,639.18003928)(855.22249604,639.1600393)(855.25250337,639.13004168)
\curveto(855.35249591,639.0600394)(855.37749588,638.94503951)(855.32750337,638.78504168)
\curveto(855.24749601,638.51503994)(855.14249612,638.27504018)(855.01250337,638.06504168)
\curveto(854.81249645,637.74504071)(854.58249668,637.48004098)(854.32250337,637.27004168)
\curveto(854.07249719,637.07004139)(853.75249751,636.90504155)(853.36250337,636.77504168)
\curveto(853.262498,636.73504172)(853.1624981,636.71004175)(853.06250337,636.70004168)
\curveto(852.9624983,636.68004178)(852.8574984,636.6600418)(852.74750337,636.64004168)
\curveto(852.69749856,636.63004183)(852.64749861,636.62504183)(852.59750337,636.62504168)
\curveto(852.5574987,636.62504183)(852.51249875,636.62004184)(852.46250337,636.61004168)
\lineto(852.31250337,636.61004168)
\curveto(852.262499,636.60004186)(852.20249906,636.59504186)(852.13250337,636.59504168)
\curveto(852.07249919,636.59504186)(852.02249924,636.60004186)(851.98250337,636.61004168)
\lineto(851.84750337,636.61004168)
\curveto(851.79749946,636.62004184)(851.75249951,636.62504183)(851.71250337,636.62504168)
\curveto(851.67249959,636.62504183)(851.63249963,636.63004183)(851.59250337,636.64004168)
\curveto(851.54249972,636.65004181)(851.48749977,636.6600418)(851.42750337,636.67004168)
\curveto(851.36749989,636.67004179)(851.31249995,636.67504178)(851.26250337,636.68504168)
\curveto(851.17250009,636.70504175)(851.08250018,636.73004173)(850.99250337,636.76004168)
\curveto(850.90250036,636.78004168)(850.81750044,636.80504165)(850.73750337,636.83504168)
\curveto(850.69750056,636.8550416)(850.6625006,636.86504159)(850.63250337,636.86504168)
\curveto(850.60250066,636.87504158)(850.56750069,636.89004157)(850.52750337,636.91004168)
\curveto(850.37750088,636.98004148)(850.21750104,637.06504139)(850.04750337,637.16504168)
\curveto(849.7575015,637.3550411)(849.50750175,637.58504087)(849.29750337,637.85504168)
\curveto(849.09750216,638.13504032)(848.92750233,638.44504001)(848.78750337,638.78504168)
\curveto(848.73750252,638.89503956)(848.69750256,639.01003945)(848.66750337,639.13004168)
\curveto(848.64750261,639.25003921)(848.61750264,639.37003909)(848.57750337,639.49004168)
\curveto(848.56750269,639.53003893)(848.5625027,639.56503889)(848.56250337,639.59504168)
\curveto(848.5625027,639.62503883)(848.5575027,639.66503879)(848.54750337,639.71504168)
\curveto(848.52750273,639.79503866)(848.51250275,639.88003858)(848.50250337,639.97004168)
\curveto(848.49250277,640.0600384)(848.47750278,640.15003831)(848.45750337,640.24004168)
\lineto(848.45750337,640.45004168)
\curveto(848.44750281,640.49003797)(848.43750282,640.54503791)(848.42750337,640.61504168)
\curveto(848.42750283,640.69503776)(848.43250283,640.7600377)(848.44250337,640.81004168)
\lineto(848.44250337,640.97504168)
\curveto(848.4625028,641.02503743)(848.46750279,641.07503738)(848.45750337,641.12504168)
\curveto(848.4575028,641.18503727)(848.4625028,641.24003722)(848.47250337,641.29004168)
\curveto(848.51250275,641.45003701)(848.54250272,641.61003685)(848.56250337,641.77004168)
\curveto(848.59250267,641.93003653)(848.63750262,642.08003638)(848.69750337,642.22004168)
\curveto(848.74750251,642.33003613)(848.79250247,642.44003602)(848.83250337,642.55004168)
\curveto(848.88250238,642.67003579)(848.93750232,642.78503567)(848.99750337,642.89504168)
\curveto(849.21750204,643.24503521)(849.46750179,643.54503491)(849.74750337,643.79504168)
\curveto(850.02750123,644.0550344)(850.37250089,644.27003419)(850.78250337,644.44004168)
\curveto(850.90250036,644.49003397)(851.02250024,644.52503393)(851.14250337,644.54504168)
\curveto(851.27249999,644.57503388)(851.40749985,644.60503385)(851.54750337,644.63504168)
\curveto(851.59749966,644.64503381)(851.64249962,644.65003381)(851.68250337,644.65004168)
\curveto(851.72249954,644.6600338)(851.76749949,644.66503379)(851.81750337,644.66504168)
\curveto(851.83749942,644.67503378)(851.8624994,644.67503378)(851.89250337,644.66504168)
\curveto(851.92249934,644.6550338)(851.94749931,644.6600338)(851.96750337,644.68004168)
\curveto(852.38749887,644.69003377)(852.75249851,644.64503381)(853.06250337,644.54504168)
\curveto(853.37249789,644.455034)(853.65249761,644.33003413)(853.90250337,644.17004168)
\curveto(853.95249731,644.15003431)(853.99249727,644.12003434)(854.02250337,644.08004168)
\curveto(854.05249721,644.05003441)(854.08749717,644.02503443)(854.12750337,644.00504168)
\curveto(854.20749705,643.94503451)(854.28749697,643.87503458)(854.36750337,643.79504168)
\curveto(854.4574968,643.71503474)(854.53249673,643.63503482)(854.59250337,643.55504168)
\curveto(854.75249651,643.34503511)(854.88749637,643.14503531)(854.99750337,642.95504168)
\curveto(855.06749619,642.84503561)(855.12249614,642.72503573)(855.16250337,642.59504168)
\curveto(855.20249606,642.46503599)(855.24749601,642.33503612)(855.29750337,642.20504168)
\curveto(855.34749591,642.07503638)(855.38249588,641.94003652)(855.40250337,641.80004168)
\curveto(855.43249583,641.6600368)(855.46749579,641.52003694)(855.50750337,641.38004168)
\curveto(855.51749574,641.31003715)(855.52249574,641.24003722)(855.52250337,641.17004168)
\lineto(855.55250337,640.96004168)
\moveto(854.09750337,641.47004168)
\curveto(854.12749713,641.51003695)(854.15249711,641.5600369)(854.17250337,641.62004168)
\curveto(854.19249707,641.69003677)(854.19249707,641.7600367)(854.17250337,641.83004168)
\curveto(854.11249715,642.05003641)(854.02749723,642.2550362)(853.91750337,642.44504168)
\curveto(853.77749748,642.67503578)(853.62249764,642.87003559)(853.45250337,643.03004168)
\curveto(853.28249798,643.19003527)(853.0624982,643.32503513)(852.79250337,643.43504168)
\curveto(852.72249854,643.455035)(852.65249861,643.47003499)(852.58250337,643.48004168)
\curveto(852.51249875,643.50003496)(852.43749882,643.52003494)(852.35750337,643.54004168)
\curveto(852.27749898,643.5600349)(852.19249907,643.57003489)(852.10250337,643.57004168)
\lineto(851.84750337,643.57004168)
\curveto(851.81749944,643.55003491)(851.78249948,643.54003492)(851.74250337,643.54004168)
\curveto(851.70249956,643.55003491)(851.66749959,643.55003491)(851.63750337,643.54004168)
\lineto(851.39750337,643.48004168)
\curveto(851.32749993,643.47003499)(851.2575,643.455035)(851.18750337,643.43504168)
\curveto(850.89750036,643.31503514)(850.6625006,643.16503529)(850.48250337,642.98504168)
\curveto(850.31250095,642.80503565)(850.1575011,642.58003588)(850.01750337,642.31004168)
\curveto(849.98750127,642.2600362)(849.9575013,642.19503626)(849.92750337,642.11504168)
\curveto(849.89750136,642.04503641)(849.87250139,641.96503649)(849.85250337,641.87504168)
\curveto(849.83250143,641.78503667)(849.82750143,641.70003676)(849.83750337,641.62004168)
\curveto(849.84750141,641.54003692)(849.88250138,641.48003698)(849.94250337,641.44004168)
\curveto(850.02250124,641.38003708)(850.1575011,641.35003711)(850.34750337,641.35004168)
\curveto(850.54750071,641.3600371)(850.71750054,641.36503709)(850.85750337,641.36504168)
\lineto(853.13750337,641.36504168)
\curveto(853.28749797,641.36503709)(853.46749779,641.3600371)(853.67750337,641.35004168)
\curveto(853.88749737,641.35003711)(854.02749723,641.39003707)(854.09750337,641.47004168)
}
}
{
\newrgbcolor{curcolor}{0 0 0}
\pscustom[linestyle=none,fillstyle=solid,fillcolor=curcolor]
{
\newpath
\moveto(860.549144,644.66504168)
\curveto(861.17913876,644.68503377)(861.68413826,644.60003386)(862.064144,644.41004168)
\curveto(862.4441375,644.22003424)(862.74913719,643.93503452)(862.979144,643.55504168)
\curveto(863.0391369,643.455035)(863.08413686,643.34503511)(863.114144,643.22504168)
\curveto(863.15413679,643.11503534)(863.18913675,643.00003546)(863.219144,642.88004168)
\curveto(863.26913667,642.69003577)(863.29913664,642.48503597)(863.309144,642.26504168)
\curveto(863.31913662,642.04503641)(863.32413662,641.82003664)(863.324144,641.59004168)
\lineto(863.324144,639.98504168)
\lineto(863.324144,637.64504168)
\curveto(863.32413662,637.47504098)(863.31913662,637.30504115)(863.309144,637.13504168)
\curveto(863.30913663,636.96504149)(863.2441367,636.8550416)(863.114144,636.80504168)
\curveto(863.06413688,636.78504167)(863.00913693,636.77504168)(862.949144,636.77504168)
\curveto(862.89913704,636.76504169)(862.8441371,636.7600417)(862.784144,636.76004168)
\curveto(862.65413729,636.7600417)(862.52913741,636.76504169)(862.409144,636.77504168)
\curveto(862.28913765,636.77504168)(862.20413774,636.81504164)(862.154144,636.89504168)
\curveto(862.10413784,636.96504149)(862.07913786,637.0550414)(862.079144,637.16504168)
\lineto(862.079144,637.49504168)
\lineto(862.079144,638.78504168)
\lineto(862.079144,641.23004168)
\curveto(862.07913786,641.50003696)(862.07413787,641.76503669)(862.064144,642.02504168)
\curveto(862.05413789,642.29503616)(862.00913793,642.52503593)(861.929144,642.71504168)
\curveto(861.84913809,642.91503554)(861.72913821,643.07503538)(861.569144,643.19504168)
\curveto(861.40913853,643.32503513)(861.22413872,643.42503503)(861.014144,643.49504168)
\curveto(860.95413899,643.51503494)(860.88913905,643.52503493)(860.819144,643.52504168)
\curveto(860.75913918,643.53503492)(860.69913924,643.55003491)(860.639144,643.57004168)
\curveto(860.58913935,643.58003488)(860.50913943,643.58003488)(860.399144,643.57004168)
\curveto(860.29913964,643.57003489)(860.22913971,643.56503489)(860.189144,643.55504168)
\curveto(860.14913979,643.53503492)(860.11413983,643.52503493)(860.084144,643.52504168)
\curveto(860.05413989,643.53503492)(860.01913992,643.53503492)(859.979144,643.52504168)
\curveto(859.84914009,643.49503496)(859.72414022,643.460035)(859.604144,643.42004168)
\curveto(859.49414045,643.39003507)(859.38914055,643.34503511)(859.289144,643.28504168)
\curveto(859.24914069,643.26503519)(859.21414073,643.24503521)(859.184144,643.22504168)
\curveto(859.15414079,643.20503525)(859.11914082,643.18503527)(859.079144,643.16504168)
\curveto(858.72914121,642.91503554)(858.47414147,642.54003592)(858.314144,642.04004168)
\curveto(858.28414166,641.9600365)(858.26414168,641.87503658)(858.254144,641.78504168)
\curveto(858.2441417,641.70503675)(858.22914171,641.62503683)(858.209144,641.54504168)
\curveto(858.18914175,641.49503696)(858.18414176,641.44503701)(858.194144,641.39504168)
\curveto(858.20414174,641.3550371)(858.19914174,641.31503714)(858.179144,641.27504168)
\lineto(858.179144,640.96004168)
\curveto(858.16914177,640.93003753)(858.16414178,640.89503756)(858.164144,640.85504168)
\curveto(858.17414177,640.81503764)(858.17914176,640.77003769)(858.179144,640.72004168)
\lineto(858.179144,640.27004168)
\lineto(858.179144,638.83004168)
\lineto(858.179144,637.51004168)
\lineto(858.179144,637.16504168)
\curveto(858.17914176,637.0550414)(858.15414179,636.96504149)(858.104144,636.89504168)
\curveto(858.05414189,636.81504164)(857.96414198,636.77504168)(857.834144,636.77504168)
\curveto(857.71414223,636.76504169)(857.58914235,636.7600417)(857.459144,636.76004168)
\curveto(857.37914256,636.7600417)(857.30414264,636.76504169)(857.234144,636.77504168)
\curveto(857.16414278,636.78504167)(857.10414284,636.81004165)(857.054144,636.85004168)
\curveto(856.97414297,636.90004156)(856.93414301,636.99504146)(856.934144,637.13504168)
\lineto(856.934144,637.54004168)
\lineto(856.934144,639.31004168)
\lineto(856.934144,642.94004168)
\lineto(856.934144,643.85504168)
\lineto(856.934144,644.12504168)
\curveto(856.93414301,644.21503424)(856.95414299,644.28503417)(856.994144,644.33504168)
\curveto(857.02414292,644.39503406)(857.07414287,644.43503402)(857.144144,644.45504168)
\curveto(857.18414276,644.46503399)(857.2391427,644.47503398)(857.309144,644.48504168)
\curveto(857.38914255,644.49503396)(857.46914247,644.50003396)(857.549144,644.50004168)
\curveto(857.62914231,644.50003396)(857.70414224,644.49503396)(857.774144,644.48504168)
\curveto(857.85414209,644.47503398)(857.90914203,644.460034)(857.939144,644.44004168)
\curveto(858.04914189,644.37003409)(858.09914184,644.28003418)(858.089144,644.17004168)
\curveto(858.07914186,644.07003439)(858.09414185,643.9550345)(858.134144,643.82504168)
\curveto(858.15414179,643.76503469)(858.19414175,643.71503474)(858.254144,643.67504168)
\curveto(858.37414157,643.66503479)(858.46914147,643.71003475)(858.539144,643.81004168)
\curveto(858.61914132,643.91003455)(858.69914124,643.99003447)(858.779144,644.05004168)
\curveto(858.91914102,644.15003431)(859.05914088,644.24003422)(859.199144,644.32004168)
\curveto(859.34914059,644.41003405)(859.51914042,644.48503397)(859.709144,644.54504168)
\curveto(859.78914015,644.57503388)(859.87414007,644.59503386)(859.964144,644.60504168)
\curveto(860.06413988,644.61503384)(860.15913978,644.63003383)(860.249144,644.65004168)
\curveto(860.29913964,644.6600338)(860.34913959,644.66503379)(860.399144,644.66504168)
\lineto(860.549144,644.66504168)
}
}
{
\newrgbcolor{curcolor}{0 0 0}
\pscustom[linestyle=none,fillstyle=solid,fillcolor=curcolor]
{
\newpath
\moveto(866.15375337,646.85504168)
\curveto(866.30375136,646.8550316)(866.45375121,646.85003161)(866.60375337,646.84004168)
\curveto(866.75375091,646.84003162)(866.85875081,646.80003166)(866.91875337,646.72004168)
\curveto(866.9687507,646.6600318)(866.99375067,646.57503188)(866.99375337,646.46504168)
\curveto(867.00375066,646.36503209)(867.00875066,646.2600322)(867.00875337,646.15004168)
\lineto(867.00875337,645.28004168)
\curveto(867.00875066,645.20003326)(867.00375066,645.11503334)(866.99375337,645.02504168)
\curveto(866.99375067,644.94503351)(867.00375066,644.87503358)(867.02375337,644.81504168)
\curveto(867.0637506,644.67503378)(867.15375051,644.58503387)(867.29375337,644.54504168)
\curveto(867.34375032,644.53503392)(867.38875028,644.53003393)(867.42875337,644.53004168)
\lineto(867.57875337,644.53004168)
\lineto(867.98375337,644.53004168)
\curveto(868.14374952,644.54003392)(868.25874941,644.53003393)(868.32875337,644.50004168)
\curveto(868.41874925,644.44003402)(868.47874919,644.38003408)(868.50875337,644.32004168)
\curveto(868.52874914,644.28003418)(868.53874913,644.23503422)(868.53875337,644.18504168)
\lineto(868.53875337,644.03504168)
\curveto(868.53874913,643.92503453)(868.53374913,643.82003464)(868.52375337,643.72004168)
\curveto(868.51374915,643.63003483)(868.47874919,643.5600349)(868.41875337,643.51004168)
\curveto(868.35874931,643.460035)(868.27374939,643.43003503)(868.16375337,643.42004168)
\lineto(867.83375337,643.42004168)
\curveto(867.72374994,643.43003503)(867.61375005,643.43503502)(867.50375337,643.43504168)
\curveto(867.39375027,643.43503502)(867.29875037,643.42003504)(867.21875337,643.39004168)
\curveto(867.14875052,643.3600351)(867.09875057,643.31003515)(867.06875337,643.24004168)
\curveto(867.03875063,643.17003529)(867.01875065,643.08503537)(867.00875337,642.98504168)
\curveto(866.99875067,642.89503556)(866.99375067,642.79503566)(866.99375337,642.68504168)
\curveto(867.00375066,642.58503587)(867.00875066,642.48503597)(867.00875337,642.38504168)
\lineto(867.00875337,639.41504168)
\curveto(867.00875066,639.19503926)(867.00375066,638.9600395)(866.99375337,638.71004168)
\curveto(866.99375067,638.47003999)(867.03875063,638.28504017)(867.12875337,638.15504168)
\curveto(867.17875049,638.07504038)(867.24375042,638.02004044)(867.32375337,637.99004168)
\curveto(867.40375026,637.9600405)(867.49875017,637.93504052)(867.60875337,637.91504168)
\curveto(867.63875003,637.90504055)(867.66875,637.90004056)(867.69875337,637.90004168)
\curveto(867.73874993,637.91004055)(867.77374989,637.91004055)(867.80375337,637.90004168)
\lineto(867.99875337,637.90004168)
\curveto(868.09874957,637.90004056)(868.18874948,637.89004057)(868.26875337,637.87004168)
\curveto(868.35874931,637.8600406)(868.42374924,637.82504063)(868.46375337,637.76504168)
\curveto(868.48374918,637.73504072)(868.49874917,637.68004078)(868.50875337,637.60004168)
\curveto(868.52874914,637.53004093)(868.53874913,637.455041)(868.53875337,637.37504168)
\curveto(868.54874912,637.29504116)(868.54874912,637.21504124)(868.53875337,637.13504168)
\curveto(868.52874914,637.06504139)(868.50874916,637.01004145)(868.47875337,636.97004168)
\curveto(868.43874923,636.90004156)(868.3637493,636.85004161)(868.25375337,636.82004168)
\curveto(868.17374949,636.80004166)(868.08374958,636.79004167)(867.98375337,636.79004168)
\curveto(867.88374978,636.80004166)(867.79374987,636.80504165)(867.71375337,636.80504168)
\curveto(867.65375001,636.80504165)(867.59375007,636.80004166)(867.53375337,636.79004168)
\curveto(867.47375019,636.79004167)(867.41875025,636.79504166)(867.36875337,636.80504168)
\lineto(867.18875337,636.80504168)
\curveto(867.13875053,636.81504164)(867.08875058,636.82004164)(867.03875337,636.82004168)
\curveto(866.99875067,636.83004163)(866.95375071,636.83504162)(866.90375337,636.83504168)
\curveto(866.70375096,636.88504157)(866.52875114,636.94004152)(866.37875337,637.00004168)
\curveto(866.23875143,637.0600414)(866.11875155,637.16504129)(866.01875337,637.31504168)
\curveto(865.87875179,637.51504094)(865.79875187,637.76504069)(865.77875337,638.06504168)
\curveto(865.75875191,638.37504008)(865.74875192,638.70503975)(865.74875337,639.05504168)
\lineto(865.74875337,642.98504168)
\curveto(865.71875195,643.11503534)(865.68875198,643.21003525)(865.65875337,643.27004168)
\curveto(865.63875203,643.33003513)(865.5687521,643.38003508)(865.44875337,643.42004168)
\curveto(865.40875226,643.43003503)(865.3687523,643.43003503)(865.32875337,643.42004168)
\curveto(865.28875238,643.41003505)(865.24875242,643.41503504)(865.20875337,643.43504168)
\lineto(864.96875337,643.43504168)
\curveto(864.83875283,643.43503502)(864.72875294,643.44503501)(864.63875337,643.46504168)
\curveto(864.55875311,643.49503496)(864.50375316,643.5550349)(864.47375337,643.64504168)
\curveto(864.45375321,643.68503477)(864.43875323,643.73003473)(864.42875337,643.78004168)
\lineto(864.42875337,643.93004168)
\curveto(864.42875324,644.07003439)(864.43875323,644.18503427)(864.45875337,644.27504168)
\curveto(864.47875319,644.37503408)(864.53875313,644.45003401)(864.63875337,644.50004168)
\curveto(864.74875292,644.54003392)(864.88875278,644.55003391)(865.05875337,644.53004168)
\curveto(865.23875243,644.51003395)(865.38875228,644.52003394)(865.50875337,644.56004168)
\curveto(865.59875207,644.61003385)(865.668752,644.68003378)(865.71875337,644.77004168)
\curveto(865.73875193,644.83003363)(865.74875192,644.90503355)(865.74875337,644.99504168)
\lineto(865.74875337,645.25004168)
\lineto(865.74875337,646.18004168)
\lineto(865.74875337,646.42004168)
\curveto(865.74875192,646.51003195)(865.75875191,646.58503187)(865.77875337,646.64504168)
\curveto(865.81875185,646.72503173)(865.89375177,646.79003167)(866.00375337,646.84004168)
\curveto(866.03375163,646.84003162)(866.05875161,646.84003162)(866.07875337,646.84004168)
\curveto(866.10875156,646.85003161)(866.13375153,646.8550316)(866.15375337,646.85504168)
}
}
{
\newrgbcolor{curcolor}{0 0 0}
\pscustom[linestyle=none,fillstyle=solid,fillcolor=curcolor]
{
\newpath
\moveto(877.05055025,640.99004168)
\curveto(877.07054219,640.93003753)(877.08054218,640.83503762)(877.08055025,640.70504168)
\curveto(877.08054218,640.58503787)(877.07554218,640.50003796)(877.06555025,640.45004168)
\lineto(877.06555025,640.30004168)
\curveto(877.0555422,640.22003824)(877.04554221,640.14503831)(877.03555025,640.07504168)
\curveto(877.03554222,640.01503844)(877.03054223,639.94503851)(877.02055025,639.86504168)
\curveto(877.00054226,639.80503865)(876.98554227,639.74503871)(876.97555025,639.68504168)
\curveto(876.97554228,639.62503883)(876.96554229,639.56503889)(876.94555025,639.50504168)
\curveto(876.90554235,639.37503908)(876.87054239,639.24503921)(876.84055025,639.11504168)
\curveto(876.81054245,638.98503947)(876.77054249,638.86503959)(876.72055025,638.75504168)
\curveto(876.51054275,638.27504018)(876.23054303,637.87004059)(875.88055025,637.54004168)
\curveto(875.53054373,637.22004124)(875.10054416,636.97504148)(874.59055025,636.80504168)
\curveto(874.48054478,636.76504169)(874.3605449,636.73504172)(874.23055025,636.71504168)
\curveto(874.11054515,636.69504176)(873.98554527,636.67504178)(873.85555025,636.65504168)
\curveto(873.79554546,636.64504181)(873.73054553,636.64004182)(873.66055025,636.64004168)
\curveto(873.60054566,636.63004183)(873.54054572,636.62504183)(873.48055025,636.62504168)
\curveto(873.44054582,636.61504184)(873.38054588,636.61004185)(873.30055025,636.61004168)
\curveto(873.23054603,636.61004185)(873.18054608,636.61504184)(873.15055025,636.62504168)
\curveto(873.11054615,636.63504182)(873.07054619,636.64004182)(873.03055025,636.64004168)
\curveto(872.99054627,636.63004183)(872.9555463,636.63004183)(872.92555025,636.64004168)
\lineto(872.83555025,636.64004168)
\lineto(872.47555025,636.68504168)
\curveto(872.33554692,636.72504173)(872.20054706,636.76504169)(872.07055025,636.80504168)
\curveto(871.94054732,636.84504161)(871.81554744,636.89004157)(871.69555025,636.94004168)
\curveto(871.24554801,637.14004132)(870.87554838,637.40004106)(870.58555025,637.72004168)
\curveto(870.29554896,638.04004042)(870.0555492,638.43004003)(869.86555025,638.89004168)
\curveto(869.81554944,638.99003947)(869.77554948,639.09003937)(869.74555025,639.19004168)
\curveto(869.72554953,639.29003917)(869.70554955,639.39503906)(869.68555025,639.50504168)
\curveto(869.66554959,639.54503891)(869.6555496,639.57503888)(869.65555025,639.59504168)
\curveto(869.66554959,639.62503883)(869.66554959,639.6600388)(869.65555025,639.70004168)
\curveto(869.63554962,639.78003868)(869.62054964,639.8600386)(869.61055025,639.94004168)
\curveto(869.61054965,640.03003843)(869.60054966,640.11503834)(869.58055025,640.19504168)
\lineto(869.58055025,640.31504168)
\curveto(869.58054968,640.3550381)(869.57554968,640.40003806)(869.56555025,640.45004168)
\curveto(869.5555497,640.50003796)(869.55054971,640.58503787)(869.55055025,640.70504168)
\curveto(869.55054971,640.83503762)(869.5605497,640.93003753)(869.58055025,640.99004168)
\curveto(869.60054966,641.0600374)(869.60554965,641.13003733)(869.59555025,641.20004168)
\curveto(869.58554967,641.27003719)(869.59054967,641.34003712)(869.61055025,641.41004168)
\curveto(869.62054964,641.460037)(869.62554963,641.50003696)(869.62555025,641.53004168)
\curveto(869.63554962,641.57003689)(869.64554961,641.61503684)(869.65555025,641.66504168)
\curveto(869.68554957,641.78503667)(869.71054955,641.90503655)(869.73055025,642.02504168)
\curveto(869.7605495,642.14503631)(869.80054946,642.2600362)(869.85055025,642.37004168)
\curveto(870.00054926,642.74003572)(870.18054908,643.07003539)(870.39055025,643.36004168)
\curveto(870.61054865,643.6600348)(870.87554838,643.91003455)(871.18555025,644.11004168)
\curveto(871.30554795,644.19003427)(871.43054783,644.2550342)(871.56055025,644.30504168)
\curveto(871.69054757,644.36503409)(871.82554743,644.42503403)(871.96555025,644.48504168)
\curveto(872.08554717,644.53503392)(872.21554704,644.56503389)(872.35555025,644.57504168)
\curveto(872.49554676,644.59503386)(872.63554662,644.62503383)(872.77555025,644.66504168)
\lineto(872.97055025,644.66504168)
\curveto(873.04054622,644.67503378)(873.10554615,644.68503377)(873.16555025,644.69504168)
\curveto(874.0555452,644.70503375)(874.79554446,644.52003394)(875.38555025,644.14004168)
\curveto(875.97554328,643.7600347)(876.40054286,643.26503519)(876.66055025,642.65504168)
\curveto(876.71054255,642.5550359)(876.75054251,642.455036)(876.78055025,642.35504168)
\curveto(876.81054245,642.2550362)(876.84554241,642.15003631)(876.88555025,642.04004168)
\curveto(876.91554234,641.93003653)(876.94054232,641.81003665)(876.96055025,641.68004168)
\curveto(876.98054228,641.5600369)(877.00554225,641.43503702)(877.03555025,641.30504168)
\curveto(877.04554221,641.2550372)(877.04554221,641.20003726)(877.03555025,641.14004168)
\curveto(877.03554222,641.09003737)(877.04054222,641.04003742)(877.05055025,640.99004168)
\moveto(875.71555025,640.13504168)
\curveto(875.73554352,640.20503825)(875.74054352,640.28503817)(875.73055025,640.37504168)
\lineto(875.73055025,640.63004168)
\curveto(875.73054353,641.02003744)(875.69554356,641.35003711)(875.62555025,641.62004168)
\curveto(875.59554366,641.70003676)(875.57054369,641.78003668)(875.55055025,641.86004168)
\curveto(875.53054373,641.94003652)(875.50554375,642.01503644)(875.47555025,642.08504168)
\curveto(875.19554406,642.73503572)(874.75054451,643.18503527)(874.14055025,643.43504168)
\curveto(874.07054519,643.46503499)(873.99554526,643.48503497)(873.91555025,643.49504168)
\lineto(873.67555025,643.55504168)
\curveto(873.59554566,643.57503488)(873.51054575,643.58503487)(873.42055025,643.58504168)
\lineto(873.15055025,643.58504168)
\lineto(872.88055025,643.54004168)
\curveto(872.78054648,643.52003494)(872.68554657,643.49503496)(872.59555025,643.46504168)
\curveto(872.51554674,643.44503501)(872.43554682,643.41503504)(872.35555025,643.37504168)
\curveto(872.28554697,643.3550351)(872.22054704,643.32503513)(872.16055025,643.28504168)
\curveto(872.10054716,643.24503521)(872.04554721,643.20503525)(871.99555025,643.16504168)
\curveto(871.7555475,642.99503546)(871.5605477,642.79003567)(871.41055025,642.55004168)
\curveto(871.260548,642.31003615)(871.13054813,642.03003643)(871.02055025,641.71004168)
\curveto(870.99054827,641.61003685)(870.97054829,641.50503695)(870.96055025,641.39504168)
\curveto(870.95054831,641.29503716)(870.93554832,641.19003727)(870.91555025,641.08004168)
\curveto(870.90554835,641.04003742)(870.90054836,640.97503748)(870.90055025,640.88504168)
\curveto(870.89054837,640.8550376)(870.88554837,640.82003764)(870.88555025,640.78004168)
\curveto(870.89554836,640.74003772)(870.90054836,640.69503776)(870.90055025,640.64504168)
\lineto(870.90055025,640.34504168)
\curveto(870.90054836,640.24503821)(870.91054835,640.1550383)(870.93055025,640.07504168)
\lineto(870.96055025,639.89504168)
\curveto(870.98054828,639.79503866)(870.99554826,639.69503876)(871.00555025,639.59504168)
\curveto(871.02554823,639.50503895)(871.0555482,639.42003904)(871.09555025,639.34004168)
\curveto(871.19554806,639.10003936)(871.31054795,638.87503958)(871.44055025,638.66504168)
\curveto(871.58054768,638.45504)(871.75054751,638.28004018)(871.95055025,638.14004168)
\curveto(872.00054726,638.11004035)(872.04554721,638.08504037)(872.08555025,638.06504168)
\curveto(872.12554713,638.04504041)(872.17054709,638.02004044)(872.22055025,637.99004168)
\curveto(872.30054696,637.94004052)(872.38554687,637.89504056)(872.47555025,637.85504168)
\curveto(872.57554668,637.82504063)(872.68054658,637.79504066)(872.79055025,637.76504168)
\curveto(872.84054642,637.74504071)(872.88554637,637.73504072)(872.92555025,637.73504168)
\curveto(872.97554628,637.74504071)(873.02554623,637.74504071)(873.07555025,637.73504168)
\curveto(873.10554615,637.72504073)(873.16554609,637.71504074)(873.25555025,637.70504168)
\curveto(873.3555459,637.69504076)(873.43054583,637.70004076)(873.48055025,637.72004168)
\curveto(873.52054574,637.73004073)(873.5605457,637.73004073)(873.60055025,637.72004168)
\curveto(873.64054562,637.72004074)(873.68054558,637.73004073)(873.72055025,637.75004168)
\curveto(873.80054546,637.77004069)(873.88054538,637.78504067)(873.96055025,637.79504168)
\curveto(874.04054522,637.81504064)(874.11554514,637.84004062)(874.18555025,637.87004168)
\curveto(874.52554473,638.01004045)(874.80054446,638.20504025)(875.01055025,638.45504168)
\curveto(875.22054404,638.70503975)(875.39554386,639.00003946)(875.53555025,639.34004168)
\curveto(875.58554367,639.460039)(875.61554364,639.58503887)(875.62555025,639.71504168)
\curveto(875.64554361,639.8550386)(875.67554358,639.99503846)(875.71555025,640.13504168)
}
}
{
\newrgbcolor{curcolor}{0 0 0}
\pscustom[linestyle=none,fillstyle=solid,fillcolor=curcolor]
{
\newpath
\moveto(880.9688315,644.69504168)
\curveto(881.68882743,644.70503375)(882.29382683,644.62003384)(882.7838315,644.44004168)
\curveto(883.27382585,644.27003419)(883.65382547,643.96503449)(883.9238315,643.52504168)
\curveto(883.99382513,643.41503504)(884.04882507,643.30003516)(884.0888315,643.18004168)
\curveto(884.12882499,643.07003539)(884.16882495,642.94503551)(884.2088315,642.80504168)
\curveto(884.22882489,642.73503572)(884.23382489,642.6600358)(884.2238315,642.58004168)
\curveto(884.21382491,642.51003595)(884.19882492,642.455036)(884.1788315,642.41504168)
\curveto(884.15882496,642.39503606)(884.13382499,642.37503608)(884.1038315,642.35504168)
\curveto(884.07382505,642.34503611)(884.04882507,642.33003613)(884.0288315,642.31004168)
\curveto(883.97882514,642.29003617)(883.92882519,642.28503617)(883.8788315,642.29504168)
\curveto(883.82882529,642.30503615)(883.77882534,642.30503615)(883.7288315,642.29504168)
\curveto(883.64882547,642.27503618)(883.54382558,642.27003619)(883.4138315,642.28004168)
\curveto(883.28382584,642.30003616)(883.19382593,642.32503613)(883.1438315,642.35504168)
\curveto(883.06382606,642.40503605)(883.00882611,642.47003599)(882.9788315,642.55004168)
\curveto(882.95882616,642.64003582)(882.9238262,642.72503573)(882.8738315,642.80504168)
\curveto(882.78382634,642.96503549)(882.65882646,643.11003535)(882.4988315,643.24004168)
\curveto(882.38882673,643.32003514)(882.26882685,643.38003508)(882.1388315,643.42004168)
\curveto(882.00882711,643.460035)(881.86882725,643.50003496)(881.7188315,643.54004168)
\curveto(881.66882745,643.5600349)(881.6188275,643.56503489)(881.5688315,643.55504168)
\curveto(881.5188276,643.5550349)(881.46882765,643.5600349)(881.4188315,643.57004168)
\curveto(881.35882776,643.59003487)(881.28382784,643.60003486)(881.1938315,643.60004168)
\curveto(881.10382802,643.60003486)(881.02882809,643.59003487)(880.9688315,643.57004168)
\lineto(880.8788315,643.57004168)
\lineto(880.7288315,643.54004168)
\curveto(880.67882844,643.54003492)(880.62882849,643.53503492)(880.5788315,643.52504168)
\curveto(880.3188288,643.46503499)(880.10382902,643.38003508)(879.9338315,643.27004168)
\curveto(879.76382936,643.1600353)(879.64882947,642.97503548)(879.5888315,642.71504168)
\curveto(879.56882955,642.64503581)(879.56382956,642.57503588)(879.5738315,642.50504168)
\curveto(879.59382953,642.43503602)(879.61382951,642.37503608)(879.6338315,642.32504168)
\curveto(879.69382943,642.17503628)(879.76382936,642.06503639)(879.8438315,641.99504168)
\curveto(879.93382919,641.93503652)(880.04382908,641.86503659)(880.1738315,641.78504168)
\curveto(880.33382879,641.68503677)(880.51382861,641.61003685)(880.7138315,641.56004168)
\curveto(880.91382821,641.52003694)(881.11382801,641.47003699)(881.3138315,641.41004168)
\curveto(881.44382768,641.37003709)(881.57382755,641.34003712)(881.7038315,641.32004168)
\curveto(881.83382729,641.30003716)(881.96382716,641.27003719)(882.0938315,641.23004168)
\curveto(882.30382682,641.17003729)(882.50882661,641.11003735)(882.7088315,641.05004168)
\curveto(882.90882621,641.00003746)(883.10882601,640.93503752)(883.3088315,640.85504168)
\lineto(883.4588315,640.79504168)
\curveto(883.50882561,640.77503768)(883.55882556,640.75003771)(883.6088315,640.72004168)
\curveto(883.80882531,640.60003786)(883.98382514,640.46503799)(884.1338315,640.31504168)
\curveto(884.28382484,640.16503829)(884.40882471,639.97503848)(884.5088315,639.74504168)
\curveto(884.52882459,639.67503878)(884.54882457,639.58003888)(884.5688315,639.46004168)
\curveto(884.58882453,639.39003907)(884.59882452,639.31503914)(884.5988315,639.23504168)
\curveto(884.60882451,639.16503929)(884.61382451,639.08503937)(884.6138315,638.99504168)
\lineto(884.6138315,638.84504168)
\curveto(884.59382453,638.77503968)(884.58382454,638.70503975)(884.5838315,638.63504168)
\curveto(884.58382454,638.56503989)(884.57382455,638.49503996)(884.5538315,638.42504168)
\curveto(884.5238246,638.31504014)(884.48882463,638.21004025)(884.4488315,638.11004168)
\curveto(884.40882471,638.01004045)(884.36382476,637.92004054)(884.3138315,637.84004168)
\curveto(884.15382497,637.58004088)(883.94882517,637.37004109)(883.6988315,637.21004168)
\curveto(883.44882567,637.0600414)(883.16882595,636.93004153)(882.8588315,636.82004168)
\curveto(882.76882635,636.79004167)(882.67382645,636.77004169)(882.5738315,636.76004168)
\curveto(882.48382664,636.74004172)(882.39382673,636.71504174)(882.3038315,636.68504168)
\curveto(882.20382692,636.66504179)(882.10382702,636.6550418)(882.0038315,636.65504168)
\curveto(881.90382722,636.6550418)(881.80382732,636.64504181)(881.7038315,636.62504168)
\lineto(881.5538315,636.62504168)
\curveto(881.50382762,636.61504184)(881.43382769,636.61004185)(881.3438315,636.61004168)
\curveto(881.25382787,636.61004185)(881.18382794,636.61504184)(881.1338315,636.62504168)
\lineto(880.9688315,636.62504168)
\curveto(880.90882821,636.64504181)(880.84382828,636.6550418)(880.7738315,636.65504168)
\curveto(880.70382842,636.64504181)(880.64382848,636.65004181)(880.5938315,636.67004168)
\curveto(880.54382858,636.68004178)(880.47882864,636.68504177)(880.3988315,636.68504168)
\lineto(880.1588315,636.74504168)
\curveto(880.08882903,636.7550417)(880.01382911,636.77504168)(879.9338315,636.80504168)
\curveto(879.6238295,636.90504155)(879.35382977,637.03004143)(879.1238315,637.18004168)
\curveto(878.89383023,637.33004113)(878.69383043,637.52504093)(878.5238315,637.76504168)
\curveto(878.43383069,637.89504056)(878.35883076,638.03004043)(878.2988315,638.17004168)
\curveto(878.23883088,638.31004015)(878.18383094,638.46503999)(878.1338315,638.63504168)
\curveto(878.11383101,638.69503976)(878.10383102,638.76503969)(878.1038315,638.84504168)
\curveto(878.11383101,638.93503952)(878.12883099,639.00503945)(878.1488315,639.05504168)
\curveto(878.17883094,639.09503936)(878.22883089,639.13503932)(878.2988315,639.17504168)
\curveto(878.34883077,639.19503926)(878.4188307,639.20503925)(878.5088315,639.20504168)
\curveto(878.59883052,639.21503924)(878.68883043,639.21503924)(878.7788315,639.20504168)
\curveto(878.86883025,639.19503926)(878.95383017,639.18003928)(879.0338315,639.16004168)
\curveto(879.12383,639.15003931)(879.18382994,639.13503932)(879.2138315,639.11504168)
\curveto(879.28382984,639.06503939)(879.32882979,638.99003947)(879.3488315,638.89004168)
\curveto(879.37882974,638.80003966)(879.41382971,638.71503974)(879.4538315,638.63504168)
\curveto(879.55382957,638.41504004)(879.68882943,638.24504021)(879.8588315,638.12504168)
\curveto(879.97882914,638.03504042)(880.11382901,637.96504049)(880.2638315,637.91504168)
\curveto(880.41382871,637.86504059)(880.57382855,637.81504064)(880.7438315,637.76504168)
\lineto(881.0588315,637.72004168)
\lineto(881.1488315,637.72004168)
\curveto(881.2188279,637.70004076)(881.30882781,637.69004077)(881.4188315,637.69004168)
\curveto(881.53882758,637.69004077)(881.63882748,637.70004076)(881.7188315,637.72004168)
\curveto(881.78882733,637.72004074)(881.84382728,637.72504073)(881.8838315,637.73504168)
\curveto(881.94382718,637.74504071)(882.00382712,637.75004071)(882.0638315,637.75004168)
\curveto(882.123827,637.7600407)(882.17882694,637.77004069)(882.2288315,637.78004168)
\curveto(882.5188266,637.8600406)(882.74882637,637.96504049)(882.9188315,638.09504168)
\curveto(883.08882603,638.22504023)(883.20882591,638.44504001)(883.2788315,638.75504168)
\curveto(883.29882582,638.80503965)(883.30382582,638.8600396)(883.2938315,638.92004168)
\curveto(883.28382584,638.98003948)(883.27382585,639.02503943)(883.2638315,639.05504168)
\curveto(883.21382591,639.24503921)(883.14382598,639.38503907)(883.0538315,639.47504168)
\curveto(882.96382616,639.57503888)(882.84882627,639.66503879)(882.7088315,639.74504168)
\curveto(882.6188265,639.80503865)(882.5188266,639.8550386)(882.4088315,639.89504168)
\lineto(882.0788315,640.01504168)
\curveto(882.04882707,640.02503843)(882.0188271,640.03003843)(881.9888315,640.03004168)
\curveto(881.96882715,640.03003843)(881.94382718,640.04003842)(881.9138315,640.06004168)
\curveto(881.57382755,640.17003829)(881.2188279,640.25003821)(880.8488315,640.30004168)
\curveto(880.48882863,640.3600381)(880.14882897,640.455038)(879.8288315,640.58504168)
\curveto(879.72882939,640.62503783)(879.63382949,640.6600378)(879.5438315,640.69004168)
\curveto(879.45382967,640.72003774)(879.36882975,640.7600377)(879.2888315,640.81004168)
\curveto(879.09883002,640.92003754)(878.9238302,641.04503741)(878.7638315,641.18504168)
\curveto(878.60383052,641.32503713)(878.47883064,641.50003696)(878.3888315,641.71004168)
\curveto(878.35883076,641.78003668)(878.33383079,641.85003661)(878.3138315,641.92004168)
\curveto(878.30383082,641.99003647)(878.28883083,642.06503639)(878.2688315,642.14504168)
\curveto(878.23883088,642.26503619)(878.22883089,642.40003606)(878.2388315,642.55004168)
\curveto(878.24883087,642.71003575)(878.26383086,642.84503561)(878.2838315,642.95504168)
\curveto(878.30383082,643.00503545)(878.31383081,643.04503541)(878.3138315,643.07504168)
\curveto(878.3238308,643.11503534)(878.33883078,643.1550353)(878.3588315,643.19504168)
\curveto(878.44883067,643.42503503)(878.56883055,643.62503483)(878.7188315,643.79504168)
\curveto(878.87883024,643.96503449)(879.05883006,644.11503434)(879.2588315,644.24504168)
\curveto(879.40882971,644.33503412)(879.57382955,644.40503405)(879.7538315,644.45504168)
\curveto(879.93382919,644.51503394)(880.123829,644.57003389)(880.3238315,644.62004168)
\curveto(880.39382873,644.63003383)(880.45882866,644.64003382)(880.5188315,644.65004168)
\curveto(880.58882853,644.6600338)(880.66382846,644.67003379)(880.7438315,644.68004168)
\curveto(880.77382835,644.69003377)(880.81382831,644.69003377)(880.8638315,644.68004168)
\curveto(880.91382821,644.67003379)(880.94882817,644.67503378)(880.9688315,644.69504168)
}
}
{
\newrgbcolor{curcolor}{0.7019608 0.7019608 0.7019608}
\pscustom[linestyle=none,fillstyle=solid,fillcolor=curcolor]
{
\newpath
\moveto(812.80437349,647.5000783)
\lineto(827.80437349,647.5000783)
\lineto(827.80437349,632.5000783)
\lineto(812.80437349,632.5000783)
\closepath
}
}
{
\newrgbcolor{curcolor}{0 0 0}
\pscustom[linestyle=none,fillstyle=solid,fillcolor=curcolor]
{
\newpath
\moveto(832.7925815,624.43427508)
\lineto(837.6975815,624.43427508)
\lineto(838.9875815,624.43427508)
\curveto(839.09757362,624.43426438)(839.20757351,624.43426438)(839.3175815,624.43427508)
\curveto(839.42757329,624.44426437)(839.5175732,624.42426439)(839.5875815,624.37427508)
\curveto(839.6175731,624.35426446)(839.64257307,624.32926449)(839.6625815,624.29927508)
\curveto(839.68257303,624.26926455)(839.70257301,624.23926458)(839.7225815,624.20927508)
\curveto(839.74257297,624.13926468)(839.75257296,624.02426479)(839.7525815,623.86427508)
\curveto(839.75257296,623.7142651)(839.74257297,623.59926522)(839.7225815,623.51927508)
\curveto(839.68257303,623.37926544)(839.59757312,623.29926552)(839.4675815,623.27927508)
\curveto(839.33757338,623.26926555)(839.18257353,623.26426555)(839.0025815,623.26427508)
\lineto(837.5025815,623.26427508)
\lineto(834.9825815,623.26427508)
\lineto(834.4125815,623.26427508)
\curveto(834.20257851,623.27426554)(834.04757867,623.24926557)(833.9475815,623.18927508)
\curveto(833.84757887,623.12926569)(833.79257892,623.02426579)(833.7825815,622.87427508)
\lineto(833.7825815,622.40927508)
\lineto(833.7825815,620.87927508)
\curveto(833.78257893,620.76926805)(833.77757894,620.63926818)(833.7675815,620.48927508)
\curveto(833.76757895,620.33926848)(833.77757894,620.2192686)(833.7975815,620.12927508)
\curveto(833.82757889,620.00926881)(833.88757883,619.92926889)(833.9775815,619.88927508)
\curveto(834.0175787,619.86926895)(834.08757863,619.84926897)(834.1875815,619.82927508)
\lineto(834.3375815,619.82927508)
\curveto(834.37757834,619.819269)(834.4175783,619.814269)(834.4575815,619.81427508)
\curveto(834.50757821,619.82426899)(834.55757816,619.82926899)(834.6075815,619.82927508)
\lineto(835.1175815,619.82927508)
\lineto(838.0575815,619.82927508)
\lineto(838.3575815,619.82927508)
\curveto(838.46757425,619.83926898)(838.57757414,619.83926898)(838.6875815,619.82927508)
\curveto(838.80757391,619.82926899)(838.9125738,619.819269)(839.0025815,619.79927508)
\curveto(839.10257361,619.78926903)(839.17757354,619.76926905)(839.2275815,619.73927508)
\curveto(839.25757346,619.7192691)(839.28257343,619.67426914)(839.3025815,619.60427508)
\curveto(839.32257339,619.53426928)(839.33757338,619.45926936)(839.3475815,619.37927508)
\curveto(839.35757336,619.29926952)(839.35757336,619.2142696)(839.3475815,619.12427508)
\curveto(839.34757337,619.04426977)(839.33757338,618.97426984)(839.3175815,618.91427508)
\curveto(839.29757342,618.82426999)(839.25257346,618.75927006)(839.1825815,618.71927508)
\curveto(839.16257355,618.69927012)(839.13257358,618.68427013)(839.0925815,618.67427508)
\curveto(839.06257365,618.67427014)(839.03257368,618.66927015)(839.0025815,618.65927508)
\lineto(838.9125815,618.65927508)
\curveto(838.86257385,618.64927017)(838.8125739,618.64427017)(838.7625815,618.64427508)
\curveto(838.712574,618.65427016)(838.66257405,618.65927016)(838.6125815,618.65927508)
\lineto(838.0575815,618.65927508)
\lineto(834.8925815,618.65927508)
\lineto(834.5325815,618.65927508)
\curveto(834.42257829,618.66927015)(834.3175784,618.66427015)(834.2175815,618.64427508)
\curveto(834.1175786,618.63427018)(834.02757869,618.60927021)(833.9475815,618.56927508)
\curveto(833.87757884,618.52927029)(833.82757889,618.45927036)(833.7975815,618.35927508)
\curveto(833.77757894,618.29927052)(833.76757895,618.22927059)(833.7675815,618.14927508)
\curveto(833.77757894,618.06927075)(833.78257893,617.98927083)(833.7825815,617.90927508)
\lineto(833.7825815,617.06927508)
\lineto(833.7825815,615.64427508)
\curveto(833.78257893,615.50427331)(833.78757893,615.37427344)(833.7975815,615.25427508)
\curveto(833.80757891,615.14427367)(833.84757887,615.06427375)(833.9175815,615.01427508)
\curveto(833.98757873,614.96427385)(834.06757865,614.93427388)(834.1575815,614.92427508)
\lineto(834.4575815,614.92427508)
\lineto(835.4175815,614.92427508)
\lineto(838.1925815,614.92427508)
\lineto(839.0475815,614.92427508)
\lineto(839.2875815,614.92427508)
\curveto(839.36757335,614.93427388)(839.43757328,614.92927389)(839.4975815,614.90927508)
\curveto(839.6175731,614.86927395)(839.69757302,614.814274)(839.7375815,614.74427508)
\curveto(839.75757296,614.7142741)(839.77257294,614.66427415)(839.7825815,614.59427508)
\curveto(839.79257292,614.52427429)(839.79757292,614.44927437)(839.7975815,614.36927508)
\curveto(839.80757291,614.29927452)(839.80757291,614.22427459)(839.7975815,614.14427508)
\curveto(839.78757293,614.07427474)(839.77757294,614.0192748)(839.7675815,613.97927508)
\curveto(839.72757299,613.89927492)(839.68257303,613.84427497)(839.6325815,613.81427508)
\curveto(839.57257314,613.77427504)(839.49257322,613.75427506)(839.3925815,613.75427508)
\lineto(839.1225815,613.75427508)
\lineto(838.0725815,613.75427508)
\lineto(834.0825815,613.75427508)
\lineto(833.0325815,613.75427508)
\curveto(832.89257982,613.75427506)(832.77257994,613.75927506)(832.6725815,613.76927508)
\curveto(832.57258014,613.78927503)(832.49758022,613.83927498)(832.4475815,613.91927508)
\curveto(832.40758031,613.97927484)(832.38758033,614.05427476)(832.3875815,614.14427508)
\lineto(832.3875815,614.42927508)
\lineto(832.3875815,615.47927508)
\lineto(832.3875815,619.49927508)
\lineto(832.3875815,622.85927508)
\lineto(832.3875815,623.78927508)
\lineto(832.3875815,624.05927508)
\curveto(832.38758033,624.14926467)(832.40758031,624.2192646)(832.4475815,624.26927508)
\curveto(832.48758023,624.33926448)(832.56258015,624.38926443)(832.6725815,624.41927508)
\curveto(832.69258002,624.42926439)(832.71258,624.42926439)(832.7325815,624.41927508)
\curveto(832.75257996,624.4192644)(832.77257994,624.42426439)(832.7925815,624.43427508)
}
}
{
\newrgbcolor{curcolor}{0 0 0}
\pscustom[linestyle=none,fillstyle=solid,fillcolor=curcolor]
{
\newpath
\moveto(844.99250337,621.62927508)
\curveto(845.62249814,621.64926717)(846.12749763,621.56426725)(846.50750337,621.37427508)
\curveto(846.88749687,621.18426763)(847.19249657,620.89926792)(847.42250337,620.51927508)
\curveto(847.48249628,620.4192684)(847.52749623,620.30926851)(847.55750337,620.18927508)
\curveto(847.59749616,620.07926874)(847.63249613,619.96426885)(847.66250337,619.84427508)
\curveto(847.71249605,619.65426916)(847.74249602,619.44926937)(847.75250337,619.22927508)
\curveto(847.762496,619.00926981)(847.76749599,618.78427003)(847.76750337,618.55427508)
\lineto(847.76750337,616.94927508)
\lineto(847.76750337,614.60927508)
\curveto(847.76749599,614.43927438)(847.762496,614.26927455)(847.75250337,614.09927508)
\curveto(847.75249601,613.92927489)(847.68749607,613.819275)(847.55750337,613.76927508)
\curveto(847.50749625,613.74927507)(847.45249631,613.73927508)(847.39250337,613.73927508)
\curveto(847.34249642,613.72927509)(847.28749647,613.72427509)(847.22750337,613.72427508)
\curveto(847.09749666,613.72427509)(846.97249679,613.72927509)(846.85250337,613.73927508)
\curveto(846.73249703,613.73927508)(846.64749711,613.77927504)(846.59750337,613.85927508)
\curveto(846.54749721,613.92927489)(846.52249724,614.0192748)(846.52250337,614.12927508)
\lineto(846.52250337,614.45927508)
\lineto(846.52250337,615.74927508)
\lineto(846.52250337,618.19427508)
\curveto(846.52249724,618.46427035)(846.51749724,618.72927009)(846.50750337,618.98927508)
\curveto(846.49749726,619.25926956)(846.45249731,619.48926933)(846.37250337,619.67927508)
\curveto(846.29249747,619.87926894)(846.17249759,620.03926878)(846.01250337,620.15927508)
\curveto(845.85249791,620.28926853)(845.66749809,620.38926843)(845.45750337,620.45927508)
\curveto(845.39749836,620.47926834)(845.33249843,620.48926833)(845.26250337,620.48927508)
\curveto(845.20249856,620.49926832)(845.14249862,620.5142683)(845.08250337,620.53427508)
\curveto(845.03249873,620.54426827)(844.95249881,620.54426827)(844.84250337,620.53427508)
\curveto(844.74249902,620.53426828)(844.67249909,620.52926829)(844.63250337,620.51927508)
\curveto(844.59249917,620.49926832)(844.5574992,620.48926833)(844.52750337,620.48927508)
\curveto(844.49749926,620.49926832)(844.4624993,620.49926832)(844.42250337,620.48927508)
\curveto(844.29249947,620.45926836)(844.16749959,620.42426839)(844.04750337,620.38427508)
\curveto(843.93749982,620.35426846)(843.83249993,620.30926851)(843.73250337,620.24927508)
\curveto(843.69250007,620.22926859)(843.6575001,620.20926861)(843.62750337,620.18927508)
\curveto(843.59750016,620.16926865)(843.5625002,620.14926867)(843.52250337,620.12927508)
\curveto(843.17250059,619.87926894)(842.91750084,619.50426931)(842.75750337,619.00427508)
\curveto(842.72750103,618.92426989)(842.70750105,618.83926998)(842.69750337,618.74927508)
\curveto(842.68750107,618.66927015)(842.67250109,618.58927023)(842.65250337,618.50927508)
\curveto(842.63250113,618.45927036)(842.62750113,618.40927041)(842.63750337,618.35927508)
\curveto(842.64750111,618.3192705)(842.64250112,618.27927054)(842.62250337,618.23927508)
\lineto(842.62250337,617.92427508)
\curveto(842.61250115,617.89427092)(842.60750115,617.85927096)(842.60750337,617.81927508)
\curveto(842.61750114,617.77927104)(842.62250114,617.73427108)(842.62250337,617.68427508)
\lineto(842.62250337,617.23427508)
\lineto(842.62250337,615.79427508)
\lineto(842.62250337,614.47427508)
\lineto(842.62250337,614.12927508)
\curveto(842.62250114,614.0192748)(842.59750116,613.92927489)(842.54750337,613.85927508)
\curveto(842.49750126,613.77927504)(842.40750135,613.73927508)(842.27750337,613.73927508)
\curveto(842.1575016,613.72927509)(842.03250173,613.72427509)(841.90250337,613.72427508)
\curveto(841.82250194,613.72427509)(841.74750201,613.72927509)(841.67750337,613.73927508)
\curveto(841.60750215,613.74927507)(841.54750221,613.77427504)(841.49750337,613.81427508)
\curveto(841.41750234,613.86427495)(841.37750238,613.95927486)(841.37750337,614.09927508)
\lineto(841.37750337,614.50427508)
\lineto(841.37750337,616.27427508)
\lineto(841.37750337,619.90427508)
\lineto(841.37750337,620.81927508)
\lineto(841.37750337,621.08927508)
\curveto(841.37750238,621.17926764)(841.39750236,621.24926757)(841.43750337,621.29927508)
\curveto(841.46750229,621.35926746)(841.51750224,621.39926742)(841.58750337,621.41927508)
\curveto(841.62750213,621.42926739)(841.68250208,621.43926738)(841.75250337,621.44927508)
\curveto(841.83250193,621.45926736)(841.91250185,621.46426735)(841.99250337,621.46427508)
\curveto(842.07250169,621.46426735)(842.14750161,621.45926736)(842.21750337,621.44927508)
\curveto(842.29750146,621.43926738)(842.35250141,621.42426739)(842.38250337,621.40427508)
\curveto(842.49250127,621.33426748)(842.54250122,621.24426757)(842.53250337,621.13427508)
\curveto(842.52250124,621.03426778)(842.53750122,620.9192679)(842.57750337,620.78927508)
\curveto(842.59750116,620.72926809)(842.63750112,620.67926814)(842.69750337,620.63927508)
\curveto(842.81750094,620.62926819)(842.91250085,620.67426814)(842.98250337,620.77427508)
\curveto(843.0625007,620.87426794)(843.14250062,620.95426786)(843.22250337,621.01427508)
\curveto(843.3625004,621.1142677)(843.50250026,621.20426761)(843.64250337,621.28427508)
\curveto(843.79249997,621.37426744)(843.9624998,621.44926737)(844.15250337,621.50927508)
\curveto(844.23249953,621.53926728)(844.31749944,621.55926726)(844.40750337,621.56927508)
\curveto(844.50749925,621.57926724)(844.60249916,621.59426722)(844.69250337,621.61427508)
\curveto(844.74249902,621.62426719)(844.79249897,621.62926719)(844.84250337,621.62927508)
\lineto(844.99250337,621.62927508)
}
}
{
\newrgbcolor{curcolor}{0 0 0}
\pscustom[linestyle=none,fillstyle=solid,fillcolor=curcolor]
{
\newpath
\moveto(850.22211275,624.43427508)
\curveto(850.35211113,624.43426438)(850.487111,624.43426438)(850.62711275,624.43427508)
\curveto(850.77711071,624.43426438)(850.8871106,624.39926442)(850.95711275,624.32927508)
\curveto(851.00711048,624.25926456)(851.03211045,624.16426465)(851.03211275,624.04427508)
\curveto(851.04211044,623.93426488)(851.04711044,623.819265)(851.04711275,623.69927508)
\lineto(851.04711275,622.36427508)
\lineto(851.04711275,616.28927508)
\lineto(851.04711275,614.60927508)
\lineto(851.04711275,614.21927508)
\curveto(851.04711044,614.07927474)(851.02211046,613.96927485)(850.97211275,613.88927508)
\curveto(850.94211054,613.83927498)(850.89711059,613.80927501)(850.83711275,613.79927508)
\curveto(850.7871107,613.78927503)(850.72211076,613.77427504)(850.64211275,613.75427508)
\lineto(850.43211275,613.75427508)
\lineto(850.11711275,613.75427508)
\curveto(850.01711147,613.76427505)(849.94211154,613.79927502)(849.89211275,613.85927508)
\curveto(849.84211164,613.93927488)(849.81211167,614.03927478)(849.80211275,614.15927508)
\lineto(849.80211275,614.53427508)
\lineto(849.80211275,615.91427508)
\lineto(849.80211275,622.15427508)
\lineto(849.80211275,623.62427508)
\curveto(849.80211168,623.73426508)(849.79711169,623.84926497)(849.78711275,623.96927508)
\curveto(849.7871117,624.09926472)(849.81211167,624.19926462)(849.86211275,624.26927508)
\curveto(849.90211158,624.32926449)(849.97711151,624.37926444)(850.08711275,624.41927508)
\curveto(850.10711138,624.42926439)(850.12711136,624.42926439)(850.14711275,624.41927508)
\curveto(850.17711131,624.4192644)(850.20211128,624.42426439)(850.22211275,624.43427508)
}
}
{
\newrgbcolor{curcolor}{0 0 0}
\pscustom[linestyle=none,fillstyle=solid,fillcolor=curcolor]
{
\newpath
\moveto(859.8769565,614.30927508)
\curveto(859.90694867,614.14927467)(859.89194868,614.0142748)(859.8319565,613.90427508)
\curveto(859.7719488,613.80427501)(859.69194888,613.72927509)(859.5919565,613.67927508)
\curveto(859.54194903,613.65927516)(859.48694909,613.64927517)(859.4269565,613.64927508)
\curveto(859.3769492,613.64927517)(859.32194925,613.63927518)(859.2619565,613.61927508)
\curveto(859.04194953,613.56927525)(858.82194975,613.58427523)(858.6019565,613.66427508)
\curveto(858.39195018,613.73427508)(858.24695033,613.82427499)(858.1669565,613.93427508)
\curveto(858.11695046,614.00427481)(858.0719505,614.08427473)(858.0319565,614.17427508)
\curveto(857.99195058,614.27427454)(857.94195063,614.35427446)(857.8819565,614.41427508)
\curveto(857.86195071,614.43427438)(857.83695074,614.45427436)(857.8069565,614.47427508)
\curveto(857.78695079,614.49427432)(857.75695082,614.49927432)(857.7169565,614.48927508)
\curveto(857.60695097,614.45927436)(857.50195107,614.40427441)(857.4019565,614.32427508)
\curveto(857.31195126,614.24427457)(857.22195135,614.17427464)(857.1319565,614.11427508)
\curveto(857.00195157,614.03427478)(856.86195171,613.95927486)(856.7119565,613.88927508)
\curveto(856.56195201,613.82927499)(856.40195217,613.77427504)(856.2319565,613.72427508)
\curveto(856.13195244,613.69427512)(856.02195255,613.67427514)(855.9019565,613.66427508)
\curveto(855.79195278,613.65427516)(855.68195289,613.63927518)(855.5719565,613.61927508)
\curveto(855.52195305,613.60927521)(855.4769531,613.60427521)(855.4369565,613.60427508)
\lineto(855.3319565,613.60427508)
\curveto(855.22195335,613.58427523)(855.11695346,613.58427523)(855.0169565,613.60427508)
\lineto(854.8819565,613.60427508)
\curveto(854.83195374,613.6142752)(854.78195379,613.6192752)(854.7319565,613.61927508)
\curveto(854.68195389,613.6192752)(854.63695394,613.62927519)(854.5969565,613.64927508)
\curveto(854.55695402,613.65927516)(854.52195405,613.66427515)(854.4919565,613.66427508)
\curveto(854.4719541,613.65427516)(854.44695413,613.65427516)(854.4169565,613.66427508)
\lineto(854.1769565,613.72427508)
\curveto(854.09695448,613.73427508)(854.02195455,613.75427506)(853.9519565,613.78427508)
\curveto(853.65195492,613.9142749)(853.40695517,614.05927476)(853.2169565,614.21927508)
\curveto(853.03695554,614.38927443)(852.88695569,614.62427419)(852.7669565,614.92427508)
\curveto(852.6769559,615.14427367)(852.63195594,615.40927341)(852.6319565,615.71927508)
\lineto(852.6319565,616.03427508)
\curveto(852.64195593,616.08427273)(852.64695593,616.13427268)(852.6469565,616.18427508)
\lineto(852.6769565,616.36427508)
\lineto(852.7969565,616.69427508)
\curveto(852.83695574,616.80427201)(852.88695569,616.90427191)(852.9469565,616.99427508)
\curveto(853.12695545,617.28427153)(853.3719552,617.49927132)(853.6819565,617.63927508)
\curveto(853.99195458,617.77927104)(854.33195424,617.90427091)(854.7019565,618.01427508)
\curveto(854.84195373,618.05427076)(854.98695359,618.08427073)(855.1369565,618.10427508)
\curveto(855.28695329,618.12427069)(855.43695314,618.14927067)(855.5869565,618.17927508)
\curveto(855.65695292,618.19927062)(855.72195285,618.20927061)(855.7819565,618.20927508)
\curveto(855.85195272,618.20927061)(855.92695265,618.2192706)(856.0069565,618.23927508)
\curveto(856.0769525,618.25927056)(856.14695243,618.26927055)(856.2169565,618.26927508)
\curveto(856.28695229,618.27927054)(856.36195221,618.29427052)(856.4419565,618.31427508)
\curveto(856.69195188,618.37427044)(856.92695165,618.42427039)(857.1469565,618.46427508)
\curveto(857.36695121,618.5142703)(857.54195103,618.62927019)(857.6719565,618.80927508)
\curveto(857.73195084,618.88926993)(857.78195079,618.98926983)(857.8219565,619.10927508)
\curveto(857.86195071,619.23926958)(857.86195071,619.37926944)(857.8219565,619.52927508)
\curveto(857.76195081,619.76926905)(857.6719509,619.95926886)(857.5519565,620.09927508)
\curveto(857.44195113,620.23926858)(857.28195129,620.34926847)(857.0719565,620.42927508)
\curveto(856.95195162,620.47926834)(856.80695177,620.5142683)(856.6369565,620.53427508)
\curveto(856.4769521,620.55426826)(856.30695227,620.56426825)(856.1269565,620.56427508)
\curveto(855.94695263,620.56426825)(855.7719528,620.55426826)(855.6019565,620.53427508)
\curveto(855.43195314,620.5142683)(855.28695329,620.48426833)(855.1669565,620.44427508)
\curveto(854.99695358,620.38426843)(854.83195374,620.29926852)(854.6719565,620.18927508)
\curveto(854.59195398,620.12926869)(854.51695406,620.04926877)(854.4469565,619.94927508)
\curveto(854.38695419,619.85926896)(854.33195424,619.75926906)(854.2819565,619.64927508)
\curveto(854.25195432,619.56926925)(854.22195435,619.48426933)(854.1919565,619.39427508)
\curveto(854.1719544,619.30426951)(854.12695445,619.23426958)(854.0569565,619.18427508)
\curveto(854.01695456,619.15426966)(853.94695463,619.12926969)(853.8469565,619.10927508)
\curveto(853.75695482,619.09926972)(853.66195491,619.09426972)(853.5619565,619.09427508)
\curveto(853.46195511,619.09426972)(853.36195521,619.09926972)(853.2619565,619.10927508)
\curveto(853.1719554,619.12926969)(853.10695547,619.15426966)(853.0669565,619.18427508)
\curveto(853.02695555,619.2142696)(852.99695558,619.26426955)(852.9769565,619.33427508)
\curveto(852.95695562,619.40426941)(852.95695562,619.47926934)(852.9769565,619.55927508)
\curveto(853.00695557,619.68926913)(853.03695554,619.80926901)(853.0669565,619.91927508)
\curveto(853.10695547,620.03926878)(853.15195542,620.15426866)(853.2019565,620.26427508)
\curveto(853.39195518,620.6142682)(853.63195494,620.88426793)(853.9219565,621.07427508)
\curveto(854.21195436,621.27426754)(854.571954,621.43426738)(855.0019565,621.55427508)
\curveto(855.10195347,621.57426724)(855.20195337,621.58926723)(855.3019565,621.59927508)
\curveto(855.41195316,621.60926721)(855.52195305,621.62426719)(855.6319565,621.64427508)
\curveto(855.6719529,621.65426716)(855.73695284,621.65426716)(855.8269565,621.64427508)
\curveto(855.91695266,621.64426717)(855.9719526,621.65426716)(855.9919565,621.67427508)
\curveto(856.69195188,621.68426713)(857.30195127,621.60426721)(857.8219565,621.43427508)
\curveto(858.34195023,621.26426755)(858.70694987,620.93926788)(858.9169565,620.45927508)
\curveto(859.00694957,620.25926856)(859.05694952,620.02426879)(859.0669565,619.75427508)
\curveto(859.08694949,619.49426932)(859.09694948,619.2192696)(859.0969565,618.92927508)
\lineto(859.0969565,615.61427508)
\curveto(859.09694948,615.47427334)(859.10194947,615.33927348)(859.1119565,615.20927508)
\curveto(859.12194945,615.07927374)(859.15194942,614.97427384)(859.2019565,614.89427508)
\curveto(859.25194932,614.82427399)(859.31694926,614.77427404)(859.3969565,614.74427508)
\curveto(859.48694909,614.70427411)(859.571949,614.67427414)(859.6519565,614.65427508)
\curveto(859.73194884,614.64427417)(859.79194878,614.59927422)(859.8319565,614.51927508)
\curveto(859.85194872,614.48927433)(859.86194871,614.45927436)(859.8619565,614.42927508)
\curveto(859.86194871,614.39927442)(859.86694871,614.35927446)(859.8769565,614.30927508)
\moveto(857.7319565,615.97427508)
\curveto(857.79195078,616.1142727)(857.82195075,616.27427254)(857.8219565,616.45427508)
\curveto(857.83195074,616.64427217)(857.83695074,616.83927198)(857.8369565,617.03927508)
\curveto(857.83695074,617.14927167)(857.83195074,617.24927157)(857.8219565,617.33927508)
\curveto(857.81195076,617.42927139)(857.7719508,617.49927132)(857.7019565,617.54927508)
\curveto(857.6719509,617.56927125)(857.60195097,617.57927124)(857.4919565,617.57927508)
\curveto(857.4719511,617.55927126)(857.43695114,617.54927127)(857.3869565,617.54927508)
\curveto(857.33695124,617.54927127)(857.29195128,617.53927128)(857.2519565,617.51927508)
\curveto(857.1719514,617.49927132)(857.08195149,617.47927134)(856.9819565,617.45927508)
\lineto(856.6819565,617.39927508)
\curveto(856.65195192,617.39927142)(856.61695196,617.39427142)(856.5769565,617.38427508)
\lineto(856.4719565,617.38427508)
\curveto(856.32195225,617.34427147)(856.15695242,617.3192715)(855.9769565,617.30927508)
\curveto(855.80695277,617.30927151)(855.64695293,617.28927153)(855.4969565,617.24927508)
\curveto(855.41695316,617.22927159)(855.34195323,617.20927161)(855.2719565,617.18927508)
\curveto(855.21195336,617.17927164)(855.14195343,617.16427165)(855.0619565,617.14427508)
\curveto(854.90195367,617.09427172)(854.75195382,617.02927179)(854.6119565,616.94927508)
\curveto(854.4719541,616.87927194)(854.35195422,616.78927203)(854.2519565,616.67927508)
\curveto(854.15195442,616.56927225)(854.0769545,616.43427238)(854.0269565,616.27427508)
\curveto(853.9769546,616.12427269)(853.95695462,615.93927288)(853.9669565,615.71927508)
\curveto(853.96695461,615.6192732)(853.98195459,615.52427329)(854.0119565,615.43427508)
\curveto(854.05195452,615.35427346)(854.09695448,615.27927354)(854.1469565,615.20927508)
\curveto(854.22695435,615.09927372)(854.33195424,615.00427381)(854.4619565,614.92427508)
\curveto(854.59195398,614.85427396)(854.73195384,614.79427402)(854.8819565,614.74427508)
\curveto(854.93195364,614.73427408)(854.98195359,614.72927409)(855.0319565,614.72927508)
\curveto(855.08195349,614.72927409)(855.13195344,614.72427409)(855.1819565,614.71427508)
\curveto(855.25195332,614.69427412)(855.33695324,614.67927414)(855.4369565,614.66927508)
\curveto(855.54695303,614.66927415)(855.63695294,614.67927414)(855.7069565,614.69927508)
\curveto(855.76695281,614.7192741)(855.82695275,614.72427409)(855.8869565,614.71427508)
\curveto(855.94695263,614.7142741)(856.00695257,614.72427409)(856.0669565,614.74427508)
\curveto(856.14695243,614.76427405)(856.22195235,614.77927404)(856.2919565,614.78927508)
\curveto(856.3719522,614.79927402)(856.44695213,614.819274)(856.5169565,614.84927508)
\curveto(856.80695177,614.96927385)(857.05195152,615.1142737)(857.2519565,615.28427508)
\curveto(857.46195111,615.45427336)(857.62195095,615.68427313)(857.7319565,615.97427508)
}
}
{
\newrgbcolor{curcolor}{0 0 0}
\pscustom[linestyle=none,fillstyle=solid,fillcolor=curcolor]
{
\newpath
\moveto(864.18359712,621.65927508)
\curveto(864.92359233,621.66926715)(865.53859172,621.55926726)(866.02859712,621.32927508)
\curveto(866.52859073,621.10926771)(866.92359033,620.77426804)(867.21359712,620.32427508)
\curveto(867.34358991,620.12426869)(867.4535898,619.87926894)(867.54359712,619.58927508)
\curveto(867.56358969,619.53926928)(867.57858968,619.47426934)(867.58859712,619.39427508)
\curveto(867.59858966,619.3142695)(867.59358966,619.24426957)(867.57359712,619.18427508)
\curveto(867.54358971,619.13426968)(867.49358976,619.08926973)(867.42359712,619.04927508)
\curveto(867.39358986,619.02926979)(867.36358989,619.0192698)(867.33359712,619.01927508)
\curveto(867.30358995,619.02926979)(867.26858999,619.02926979)(867.22859712,619.01927508)
\curveto(867.18859007,619.00926981)(867.14859011,619.00426981)(867.10859712,619.00427508)
\curveto(867.06859019,619.0142698)(867.02859023,619.0192698)(866.98859712,619.01927508)
\lineto(866.67359712,619.01927508)
\curveto(866.57359068,619.02926979)(866.48859077,619.05926976)(866.41859712,619.10927508)
\curveto(866.33859092,619.16926965)(866.28359097,619.25426956)(866.25359712,619.36427508)
\curveto(866.22359103,619.47426934)(866.18359107,619.56926925)(866.13359712,619.64927508)
\curveto(865.98359127,619.90926891)(865.78859147,620.1142687)(865.54859712,620.26427508)
\curveto(865.46859179,620.3142685)(865.38359187,620.35426846)(865.29359712,620.38427508)
\curveto(865.20359205,620.42426839)(865.10859215,620.45926836)(865.00859712,620.48927508)
\curveto(864.86859239,620.52926829)(864.68359257,620.54926827)(864.45359712,620.54927508)
\curveto(864.22359303,620.55926826)(864.03359322,620.53926828)(863.88359712,620.48927508)
\curveto(863.81359344,620.46926835)(863.74859351,620.45426836)(863.68859712,620.44427508)
\curveto(863.62859363,620.43426838)(863.56359369,620.4192684)(863.49359712,620.39927508)
\curveto(863.23359402,620.28926853)(863.00359425,620.13926868)(862.80359712,619.94927508)
\curveto(862.60359465,619.75926906)(862.44859481,619.53426928)(862.33859712,619.27427508)
\curveto(862.29859496,619.18426963)(862.26359499,619.08926973)(862.23359712,618.98927508)
\curveto(862.20359505,618.89926992)(862.17359508,618.79927002)(862.14359712,618.68927508)
\lineto(862.05359712,618.28427508)
\curveto(862.04359521,618.23427058)(862.03859522,618.17927064)(862.03859712,618.11927508)
\curveto(862.04859521,618.05927076)(862.04359521,618.00427081)(862.02359712,617.95427508)
\lineto(862.02359712,617.83427508)
\curveto(862.01359524,617.79427102)(862.00359525,617.72927109)(861.99359712,617.63927508)
\curveto(861.99359526,617.54927127)(862.00359525,617.48427133)(862.02359712,617.44427508)
\curveto(862.03359522,617.39427142)(862.03359522,617.34427147)(862.02359712,617.29427508)
\curveto(862.01359524,617.24427157)(862.01359524,617.19427162)(862.02359712,617.14427508)
\curveto(862.03359522,617.10427171)(862.03859522,617.03427178)(862.03859712,616.93427508)
\curveto(862.0585952,616.85427196)(862.07359518,616.76927205)(862.08359712,616.67927508)
\curveto(862.10359515,616.58927223)(862.12359513,616.50427231)(862.14359712,616.42427508)
\curveto(862.253595,616.10427271)(862.37859488,615.82427299)(862.51859712,615.58427508)
\curveto(862.66859459,615.35427346)(862.87359438,615.15427366)(863.13359712,614.98427508)
\curveto(863.22359403,614.93427388)(863.31359394,614.88927393)(863.40359712,614.84927508)
\curveto(863.50359375,614.80927401)(863.60859365,614.76927405)(863.71859712,614.72927508)
\curveto(863.76859349,614.7192741)(863.80859345,614.7142741)(863.83859712,614.71427508)
\curveto(863.86859339,614.7142741)(863.90859335,614.70927411)(863.95859712,614.69927508)
\curveto(863.98859327,614.68927413)(864.03859322,614.68427413)(864.10859712,614.68427508)
\lineto(864.27359712,614.68427508)
\curveto(864.27359298,614.67427414)(864.29359296,614.66927415)(864.33359712,614.66927508)
\curveto(864.3535929,614.67927414)(864.37859288,614.67927414)(864.40859712,614.66927508)
\curveto(864.43859282,614.66927415)(864.46859279,614.67427414)(864.49859712,614.68427508)
\curveto(864.56859269,614.70427411)(864.63359262,614.70927411)(864.69359712,614.69927508)
\curveto(864.76359249,614.69927412)(864.83359242,614.70927411)(864.90359712,614.72927508)
\curveto(865.16359209,614.80927401)(865.38859187,614.90927391)(865.57859712,615.02927508)
\curveto(865.76859149,615.15927366)(865.92859133,615.32427349)(866.05859712,615.52427508)
\curveto(866.10859115,615.60427321)(866.1535911,615.68927313)(866.19359712,615.77927508)
\lineto(866.31359712,616.04927508)
\curveto(866.33359092,616.12927269)(866.3535909,616.20427261)(866.37359712,616.27427508)
\curveto(866.40359085,616.35427246)(866.4535908,616.4192724)(866.52359712,616.46927508)
\curveto(866.5535907,616.49927232)(866.61359064,616.5192723)(866.70359712,616.52927508)
\curveto(866.79359046,616.54927227)(866.88859037,616.55927226)(866.98859712,616.55927508)
\curveto(867.09859016,616.56927225)(867.19859006,616.56927225)(867.28859712,616.55927508)
\curveto(867.38858987,616.54927227)(867.4585898,616.52927229)(867.49859712,616.49927508)
\curveto(867.5585897,616.45927236)(867.59358966,616.39927242)(867.60359712,616.31927508)
\curveto(867.62358963,616.23927258)(867.62358963,616.15427266)(867.60359712,616.06427508)
\curveto(867.5535897,615.9142729)(867.50358975,615.76927305)(867.45359712,615.62927508)
\curveto(867.41358984,615.49927332)(867.3585899,615.36927345)(867.28859712,615.23927508)
\curveto(867.13859012,614.93927388)(866.94859031,614.67427414)(866.71859712,614.44427508)
\curveto(866.49859076,614.2142746)(866.22859103,614.02927479)(865.90859712,613.88927508)
\curveto(865.82859143,613.84927497)(865.74359151,613.814275)(865.65359712,613.78427508)
\curveto(865.56359169,613.76427505)(865.46859179,613.73927508)(865.36859712,613.70927508)
\curveto(865.258592,613.66927515)(865.14859211,613.64927517)(865.03859712,613.64927508)
\curveto(864.92859233,613.63927518)(864.81859244,613.62427519)(864.70859712,613.60427508)
\curveto(864.66859259,613.58427523)(864.62859263,613.57927524)(864.58859712,613.58927508)
\curveto(864.54859271,613.59927522)(864.50859275,613.59927522)(864.46859712,613.58927508)
\lineto(864.33359712,613.58927508)
\lineto(864.09359712,613.58927508)
\curveto(864.02359323,613.57927524)(863.9585933,613.58427523)(863.89859712,613.60427508)
\lineto(863.82359712,613.60427508)
\lineto(863.46359712,613.64927508)
\curveto(863.33359392,613.68927513)(863.20859405,613.72427509)(863.08859712,613.75427508)
\curveto(862.96859429,613.78427503)(862.8535944,613.82427499)(862.74359712,613.87427508)
\curveto(862.38359487,614.03427478)(862.08359517,614.22427459)(861.84359712,614.44427508)
\curveto(861.61359564,614.66427415)(861.39859586,614.93427388)(861.19859712,615.25427508)
\curveto(861.14859611,615.33427348)(861.10359615,615.42427339)(861.06359712,615.52427508)
\lineto(860.94359712,615.82427508)
\curveto(860.89359636,615.93427288)(860.8585964,616.04927277)(860.83859712,616.16927508)
\curveto(860.81859644,616.28927253)(860.79359646,616.40927241)(860.76359712,616.52927508)
\curveto(860.7535965,616.56927225)(860.74859651,616.60927221)(860.74859712,616.64927508)
\curveto(860.74859651,616.68927213)(860.74359651,616.72927209)(860.73359712,616.76927508)
\curveto(860.71359654,616.82927199)(860.70359655,616.89427192)(860.70359712,616.96427508)
\curveto(860.71359654,617.03427178)(860.70859655,617.09927172)(860.68859712,617.15927508)
\lineto(860.68859712,617.30927508)
\curveto(860.67859658,617.35927146)(860.67359658,617.42927139)(860.67359712,617.51927508)
\curveto(860.67359658,617.60927121)(860.67859658,617.67927114)(860.68859712,617.72927508)
\curveto(860.69859656,617.77927104)(860.69859656,617.82427099)(860.68859712,617.86427508)
\curveto(860.68859657,617.90427091)(860.69359656,617.94427087)(860.70359712,617.98427508)
\curveto(860.72359653,618.05427076)(860.72859653,618.12427069)(860.71859712,618.19427508)
\curveto(860.71859654,618.26427055)(860.72859653,618.32927049)(860.74859712,618.38927508)
\curveto(860.78859647,618.55927026)(860.82359643,618.72927009)(860.85359712,618.89927508)
\curveto(860.88359637,619.06926975)(860.92859633,619.22926959)(860.98859712,619.37927508)
\curveto(861.19859606,619.89926892)(861.4535958,620.3192685)(861.75359712,620.63927508)
\curveto(862.0535952,620.95926786)(862.46359479,621.22426759)(862.98359712,621.43427508)
\curveto(863.09359416,621.48426733)(863.21359404,621.5192673)(863.34359712,621.53927508)
\curveto(863.47359378,621.55926726)(863.60859365,621.58426723)(863.74859712,621.61427508)
\curveto(863.81859344,621.62426719)(863.88859337,621.62926719)(863.95859712,621.62927508)
\curveto(864.02859323,621.63926718)(864.10359315,621.64926717)(864.18359712,621.65927508)
}
}
{
\newrgbcolor{curcolor}{0 0 0}
\pscustom[linestyle=none,fillstyle=solid,fillcolor=curcolor]
{
\newpath
\moveto(875.85523775,617.92427508)
\curveto(875.87523006,617.82427099)(875.87523006,617.70927111)(875.85523775,617.57927508)
\curveto(875.84523009,617.45927136)(875.81523012,617.37427144)(875.76523775,617.32427508)
\curveto(875.71523022,617.28427153)(875.6402303,617.25427156)(875.54023775,617.23427508)
\curveto(875.45023049,617.22427159)(875.34523059,617.2192716)(875.22523775,617.21927508)
\lineto(874.86523775,617.21927508)
\curveto(874.74523119,617.22927159)(874.6402313,617.23427158)(874.55023775,617.23427508)
\lineto(870.71023775,617.23427508)
\curveto(870.63023531,617.23427158)(870.55023539,617.22927159)(870.47023775,617.21927508)
\curveto(870.39023555,617.2192716)(870.32523561,617.20427161)(870.27523775,617.17427508)
\curveto(870.2352357,617.15427166)(870.19523574,617.1142717)(870.15523775,617.05427508)
\curveto(870.1352358,617.02427179)(870.11523582,616.97927184)(870.09523775,616.91927508)
\curveto(870.07523586,616.86927195)(870.07523586,616.819272)(870.09523775,616.76927508)
\curveto(870.10523583,616.7192721)(870.11023583,616.67427214)(870.11023775,616.63427508)
\curveto(870.11023583,616.59427222)(870.11523582,616.55427226)(870.12523775,616.51427508)
\curveto(870.14523579,616.43427238)(870.16523577,616.34927247)(870.18523775,616.25927508)
\curveto(870.20523573,616.17927264)(870.2352357,616.09927272)(870.27523775,616.01927508)
\curveto(870.50523543,615.47927334)(870.88523505,615.09427372)(871.41523775,614.86427508)
\curveto(871.47523446,614.83427398)(871.5402344,614.80927401)(871.61023775,614.78927508)
\lineto(871.82023775,614.72927508)
\curveto(871.85023409,614.7192741)(871.90023404,614.7142741)(871.97023775,614.71427508)
\curveto(872.11023383,614.67427414)(872.29523364,614.65427416)(872.52523775,614.65427508)
\curveto(872.75523318,614.65427416)(872.940233,614.67427414)(873.08023775,614.71427508)
\curveto(873.22023272,614.75427406)(873.34523259,614.79427402)(873.45523775,614.83427508)
\curveto(873.57523236,614.88427393)(873.68523225,614.94427387)(873.78523775,615.01427508)
\curveto(873.89523204,615.08427373)(873.99023195,615.16427365)(874.07023775,615.25427508)
\curveto(874.15023179,615.35427346)(874.22023172,615.45927336)(874.28023775,615.56927508)
\curveto(874.3402316,615.66927315)(874.39023155,615.77427304)(874.43023775,615.88427508)
\curveto(874.48023146,615.99427282)(874.56023138,616.07427274)(874.67023775,616.12427508)
\curveto(874.71023123,616.14427267)(874.77523116,616.15927266)(874.86523775,616.16927508)
\curveto(874.95523098,616.17927264)(875.04523089,616.17927264)(875.13523775,616.16927508)
\curveto(875.22523071,616.16927265)(875.31023063,616.16427265)(875.39023775,616.15427508)
\curveto(875.47023047,616.14427267)(875.52523041,616.12427269)(875.55523775,616.09427508)
\curveto(875.65523028,616.02427279)(875.68023026,615.90927291)(875.63023775,615.74927508)
\curveto(875.55023039,615.47927334)(875.44523049,615.23927358)(875.31523775,615.02927508)
\curveto(875.11523082,614.70927411)(874.88523105,614.44427437)(874.62523775,614.23427508)
\curveto(874.37523156,614.03427478)(874.05523188,613.86927495)(873.66523775,613.73927508)
\curveto(873.56523237,613.69927512)(873.46523247,613.67427514)(873.36523775,613.66427508)
\curveto(873.26523267,613.64427517)(873.16023278,613.62427519)(873.05023775,613.60427508)
\curveto(873.00023294,613.59427522)(872.95023299,613.58927523)(872.90023775,613.58927508)
\curveto(872.86023308,613.58927523)(872.81523312,613.58427523)(872.76523775,613.57427508)
\lineto(872.61523775,613.57427508)
\curveto(872.56523337,613.56427525)(872.50523343,613.55927526)(872.43523775,613.55927508)
\curveto(872.37523356,613.55927526)(872.32523361,613.56427525)(872.28523775,613.57427508)
\lineto(872.15023775,613.57427508)
\curveto(872.10023384,613.58427523)(872.05523388,613.58927523)(872.01523775,613.58927508)
\curveto(871.97523396,613.58927523)(871.935234,613.59427522)(871.89523775,613.60427508)
\curveto(871.84523409,613.6142752)(871.79023415,613.62427519)(871.73023775,613.63427508)
\curveto(871.67023427,613.63427518)(871.61523432,613.63927518)(871.56523775,613.64927508)
\curveto(871.47523446,613.66927515)(871.38523455,613.69427512)(871.29523775,613.72427508)
\curveto(871.20523473,613.74427507)(871.12023482,613.76927505)(871.04023775,613.79927508)
\curveto(871.00023494,613.819275)(870.96523497,613.82927499)(870.93523775,613.82927508)
\curveto(870.90523503,613.83927498)(870.87023507,613.85427496)(870.83023775,613.87427508)
\curveto(870.68023526,613.94427487)(870.52023542,614.02927479)(870.35023775,614.12927508)
\curveto(870.06023588,614.3192745)(869.81023613,614.54927427)(869.60023775,614.81927508)
\curveto(869.40023654,615.09927372)(869.23023671,615.40927341)(869.09023775,615.74927508)
\curveto(869.0402369,615.85927296)(869.00023694,615.97427284)(868.97023775,616.09427508)
\curveto(868.95023699,616.2142726)(868.92023702,616.33427248)(868.88023775,616.45427508)
\curveto(868.87023707,616.49427232)(868.86523707,616.52927229)(868.86523775,616.55927508)
\curveto(868.86523707,616.58927223)(868.86023708,616.62927219)(868.85023775,616.67927508)
\curveto(868.83023711,616.75927206)(868.81523712,616.84427197)(868.80523775,616.93427508)
\curveto(868.79523714,617.02427179)(868.78023716,617.1142717)(868.76023775,617.20427508)
\lineto(868.76023775,617.41427508)
\curveto(868.75023719,617.45427136)(868.7402372,617.50927131)(868.73023775,617.57927508)
\curveto(868.73023721,617.65927116)(868.7352372,617.72427109)(868.74523775,617.77427508)
\lineto(868.74523775,617.93927508)
\curveto(868.76523717,617.98927083)(868.77023717,618.03927078)(868.76023775,618.08927508)
\curveto(868.76023718,618.14927067)(868.76523717,618.20427061)(868.77523775,618.25427508)
\curveto(868.81523712,618.4142704)(868.84523709,618.57427024)(868.86523775,618.73427508)
\curveto(868.89523704,618.89426992)(868.940237,619.04426977)(869.00023775,619.18427508)
\curveto(869.05023689,619.29426952)(869.09523684,619.40426941)(869.13523775,619.51427508)
\curveto(869.18523675,619.63426918)(869.2402367,619.74926907)(869.30023775,619.85927508)
\curveto(869.52023642,620.20926861)(869.77023617,620.50926831)(870.05023775,620.75927508)
\curveto(870.33023561,621.0192678)(870.67523526,621.23426758)(871.08523775,621.40427508)
\curveto(871.20523473,621.45426736)(871.32523461,621.48926733)(871.44523775,621.50927508)
\curveto(871.57523436,621.53926728)(871.71023423,621.56926725)(871.85023775,621.59927508)
\curveto(871.90023404,621.60926721)(871.94523399,621.6142672)(871.98523775,621.61427508)
\curveto(872.02523391,621.62426719)(872.07023387,621.62926719)(872.12023775,621.62927508)
\curveto(872.1402338,621.63926718)(872.16523377,621.63926718)(872.19523775,621.62927508)
\curveto(872.22523371,621.6192672)(872.25023369,621.62426719)(872.27023775,621.64427508)
\curveto(872.69023325,621.65426716)(873.05523288,621.60926721)(873.36523775,621.50927508)
\curveto(873.67523226,621.4192674)(873.95523198,621.29426752)(874.20523775,621.13427508)
\curveto(874.25523168,621.1142677)(874.29523164,621.08426773)(874.32523775,621.04427508)
\curveto(874.35523158,621.0142678)(874.39023155,620.98926783)(874.43023775,620.96927508)
\curveto(874.51023143,620.90926791)(874.59023135,620.83926798)(874.67023775,620.75927508)
\curveto(874.76023118,620.67926814)(874.8352311,620.59926822)(874.89523775,620.51927508)
\curveto(875.05523088,620.30926851)(875.19023075,620.10926871)(875.30023775,619.91927508)
\curveto(875.37023057,619.80926901)(875.42523051,619.68926913)(875.46523775,619.55927508)
\curveto(875.50523043,619.42926939)(875.55023039,619.29926952)(875.60023775,619.16927508)
\curveto(875.65023029,619.03926978)(875.68523025,618.90426991)(875.70523775,618.76427508)
\curveto(875.7352302,618.62427019)(875.77023017,618.48427033)(875.81023775,618.34427508)
\curveto(875.82023012,618.27427054)(875.82523011,618.20427061)(875.82523775,618.13427508)
\lineto(875.85523775,617.92427508)
\moveto(874.40023775,618.43427508)
\curveto(874.43023151,618.47427034)(874.45523148,618.52427029)(874.47523775,618.58427508)
\curveto(874.49523144,618.65427016)(874.49523144,618.72427009)(874.47523775,618.79427508)
\curveto(874.41523152,619.0142698)(874.33023161,619.2192696)(874.22023775,619.40927508)
\curveto(874.08023186,619.63926918)(873.92523201,619.83426898)(873.75523775,619.99427508)
\curveto(873.58523235,620.15426866)(873.36523257,620.28926853)(873.09523775,620.39927508)
\curveto(873.02523291,620.4192684)(872.95523298,620.43426838)(872.88523775,620.44427508)
\curveto(872.81523312,620.46426835)(872.7402332,620.48426833)(872.66023775,620.50427508)
\curveto(872.58023336,620.52426829)(872.49523344,620.53426828)(872.40523775,620.53427508)
\lineto(872.15023775,620.53427508)
\curveto(872.12023382,620.5142683)(872.08523385,620.50426831)(872.04523775,620.50427508)
\curveto(872.00523393,620.5142683)(871.97023397,620.5142683)(871.94023775,620.50427508)
\lineto(871.70023775,620.44427508)
\curveto(871.63023431,620.43426838)(871.56023438,620.4192684)(871.49023775,620.39927508)
\curveto(871.20023474,620.27926854)(870.96523497,620.12926869)(870.78523775,619.94927508)
\curveto(870.61523532,619.76926905)(870.46023548,619.54426927)(870.32023775,619.27427508)
\curveto(870.29023565,619.22426959)(870.26023568,619.15926966)(870.23023775,619.07927508)
\curveto(870.20023574,619.00926981)(870.17523576,618.92926989)(870.15523775,618.83927508)
\curveto(870.1352358,618.74927007)(870.13023581,618.66427015)(870.14023775,618.58427508)
\curveto(870.15023579,618.50427031)(870.18523575,618.44427037)(870.24523775,618.40427508)
\curveto(870.32523561,618.34427047)(870.46023548,618.3142705)(870.65023775,618.31427508)
\curveto(870.85023509,618.32427049)(871.02023492,618.32927049)(871.16023775,618.32927508)
\lineto(873.44023775,618.32927508)
\curveto(873.59023235,618.32927049)(873.77023217,618.32427049)(873.98023775,618.31427508)
\curveto(874.19023175,618.3142705)(874.33023161,618.35427046)(874.40023775,618.43427508)
}
}
{
\newrgbcolor{curcolor}{0 0 0}
\pscustom[linestyle=none,fillstyle=solid,fillcolor=curcolor]
{
\newpath
\moveto(879.59187837,621.65927508)
\curveto(880.31187431,621.66926715)(880.9168737,621.58426723)(881.40687837,621.40427508)
\curveto(881.89687272,621.23426758)(882.27687234,620.92926789)(882.54687837,620.48927508)
\curveto(882.616872,620.37926844)(882.67187195,620.26426855)(882.71187837,620.14427508)
\curveto(882.75187187,620.03426878)(882.79187183,619.90926891)(882.83187837,619.76927508)
\curveto(882.85187177,619.69926912)(882.85687176,619.62426919)(882.84687837,619.54427508)
\curveto(882.83687178,619.47426934)(882.8218718,619.4192694)(882.80187837,619.37927508)
\curveto(882.78187184,619.35926946)(882.75687186,619.33926948)(882.72687837,619.31927508)
\curveto(882.69687192,619.30926951)(882.67187195,619.29426952)(882.65187837,619.27427508)
\curveto(882.60187202,619.25426956)(882.55187207,619.24926957)(882.50187837,619.25927508)
\curveto(882.45187217,619.26926955)(882.40187222,619.26926955)(882.35187837,619.25927508)
\curveto(882.27187235,619.23926958)(882.16687245,619.23426958)(882.03687837,619.24427508)
\curveto(881.90687271,619.26426955)(881.8168728,619.28926953)(881.76687837,619.31927508)
\curveto(881.68687293,619.36926945)(881.63187299,619.43426938)(881.60187837,619.51427508)
\curveto(881.58187304,619.60426921)(881.54687307,619.68926913)(881.49687837,619.76927508)
\curveto(881.40687321,619.92926889)(881.28187334,620.07426874)(881.12187837,620.20427508)
\curveto(881.01187361,620.28426853)(880.89187373,620.34426847)(880.76187837,620.38427508)
\curveto(880.63187399,620.42426839)(880.49187413,620.46426835)(880.34187837,620.50427508)
\curveto(880.29187433,620.52426829)(880.24187438,620.52926829)(880.19187837,620.51927508)
\curveto(880.14187448,620.5192683)(880.09187453,620.52426829)(880.04187837,620.53427508)
\curveto(879.98187464,620.55426826)(879.90687471,620.56426825)(879.81687837,620.56427508)
\curveto(879.72687489,620.56426825)(879.65187497,620.55426826)(879.59187837,620.53427508)
\lineto(879.50187837,620.53427508)
\lineto(879.35187837,620.50427508)
\curveto(879.30187532,620.50426831)(879.25187537,620.49926832)(879.20187837,620.48927508)
\curveto(878.94187568,620.42926839)(878.72687589,620.34426847)(878.55687837,620.23427508)
\curveto(878.38687623,620.12426869)(878.27187635,619.93926888)(878.21187837,619.67927508)
\curveto(878.19187643,619.60926921)(878.18687643,619.53926928)(878.19687837,619.46927508)
\curveto(878.2168764,619.39926942)(878.23687638,619.33926948)(878.25687837,619.28927508)
\curveto(878.3168763,619.13926968)(878.38687623,619.02926979)(878.46687837,618.95927508)
\curveto(878.55687606,618.89926992)(878.66687595,618.82926999)(878.79687837,618.74927508)
\curveto(878.95687566,618.64927017)(879.13687548,618.57427024)(879.33687837,618.52427508)
\curveto(879.53687508,618.48427033)(879.73687488,618.43427038)(879.93687837,618.37427508)
\curveto(880.06687455,618.33427048)(880.19687442,618.30427051)(880.32687837,618.28427508)
\curveto(880.45687416,618.26427055)(880.58687403,618.23427058)(880.71687837,618.19427508)
\curveto(880.92687369,618.13427068)(881.13187349,618.07427074)(881.33187837,618.01427508)
\curveto(881.53187309,617.96427085)(881.73187289,617.89927092)(881.93187837,617.81927508)
\lineto(882.08187837,617.75927508)
\curveto(882.13187249,617.73927108)(882.18187244,617.7142711)(882.23187837,617.68427508)
\curveto(882.43187219,617.56427125)(882.60687201,617.42927139)(882.75687837,617.27927508)
\curveto(882.90687171,617.12927169)(883.03187159,616.93927188)(883.13187837,616.70927508)
\curveto(883.15187147,616.63927218)(883.17187145,616.54427227)(883.19187837,616.42427508)
\curveto(883.21187141,616.35427246)(883.2218714,616.27927254)(883.22187837,616.19927508)
\curveto(883.23187139,616.12927269)(883.23687138,616.04927277)(883.23687837,615.95927508)
\lineto(883.23687837,615.80927508)
\curveto(883.2168714,615.73927308)(883.20687141,615.66927315)(883.20687837,615.59927508)
\curveto(883.20687141,615.52927329)(883.19687142,615.45927336)(883.17687837,615.38927508)
\curveto(883.14687147,615.27927354)(883.11187151,615.17427364)(883.07187837,615.07427508)
\curveto(883.03187159,614.97427384)(882.98687163,614.88427393)(882.93687837,614.80427508)
\curveto(882.77687184,614.54427427)(882.57187205,614.33427448)(882.32187837,614.17427508)
\curveto(882.07187255,614.02427479)(881.79187283,613.89427492)(881.48187837,613.78427508)
\curveto(881.39187323,613.75427506)(881.29687332,613.73427508)(881.19687837,613.72427508)
\curveto(881.10687351,613.70427511)(881.0168736,613.67927514)(880.92687837,613.64927508)
\curveto(880.82687379,613.62927519)(880.72687389,613.6192752)(880.62687837,613.61927508)
\curveto(880.52687409,613.6192752)(880.42687419,613.60927521)(880.32687837,613.58927508)
\lineto(880.17687837,613.58927508)
\curveto(880.12687449,613.57927524)(880.05687456,613.57427524)(879.96687837,613.57427508)
\curveto(879.87687474,613.57427524)(879.80687481,613.57927524)(879.75687837,613.58927508)
\lineto(879.59187837,613.58927508)
\curveto(879.53187509,613.60927521)(879.46687515,613.6192752)(879.39687837,613.61927508)
\curveto(879.32687529,613.60927521)(879.26687535,613.6142752)(879.21687837,613.63427508)
\curveto(879.16687545,613.64427517)(879.10187552,613.64927517)(879.02187837,613.64927508)
\lineto(878.78187837,613.70927508)
\curveto(878.71187591,613.7192751)(878.63687598,613.73927508)(878.55687837,613.76927508)
\curveto(878.24687637,613.86927495)(877.97687664,613.99427482)(877.74687837,614.14427508)
\curveto(877.5168771,614.29427452)(877.3168773,614.48927433)(877.14687837,614.72927508)
\curveto(877.05687756,614.85927396)(876.98187764,614.99427382)(876.92187837,615.13427508)
\curveto(876.86187776,615.27427354)(876.80687781,615.42927339)(876.75687837,615.59927508)
\curveto(876.73687788,615.65927316)(876.72687789,615.72927309)(876.72687837,615.80927508)
\curveto(876.73687788,615.89927292)(876.75187787,615.96927285)(876.77187837,616.01927508)
\curveto(876.80187782,616.05927276)(876.85187777,616.09927272)(876.92187837,616.13927508)
\curveto(876.97187765,616.15927266)(877.04187758,616.16927265)(877.13187837,616.16927508)
\curveto(877.2218774,616.17927264)(877.31187731,616.17927264)(877.40187837,616.16927508)
\curveto(877.49187713,616.15927266)(877.57687704,616.14427267)(877.65687837,616.12427508)
\curveto(877.74687687,616.1142727)(877.80687681,616.09927272)(877.83687837,616.07927508)
\curveto(877.90687671,616.02927279)(877.95187667,615.95427286)(877.97187837,615.85427508)
\curveto(878.00187662,615.76427305)(878.03687658,615.67927314)(878.07687837,615.59927508)
\curveto(878.17687644,615.37927344)(878.31187631,615.20927361)(878.48187837,615.08927508)
\curveto(878.60187602,614.99927382)(878.73687588,614.92927389)(878.88687837,614.87927508)
\curveto(879.03687558,614.82927399)(879.19687542,614.77927404)(879.36687837,614.72927508)
\lineto(879.68187837,614.68427508)
\lineto(879.77187837,614.68427508)
\curveto(879.84187478,614.66427415)(879.93187469,614.65427416)(880.04187837,614.65427508)
\curveto(880.16187446,614.65427416)(880.26187436,614.66427415)(880.34187837,614.68427508)
\curveto(880.41187421,614.68427413)(880.46687415,614.68927413)(880.50687837,614.69927508)
\curveto(880.56687405,614.70927411)(880.62687399,614.7142741)(880.68687837,614.71427508)
\curveto(880.74687387,614.72427409)(880.80187382,614.73427408)(880.85187837,614.74427508)
\curveto(881.14187348,614.82427399)(881.37187325,614.92927389)(881.54187837,615.05927508)
\curveto(881.71187291,615.18927363)(881.83187279,615.40927341)(881.90187837,615.71927508)
\curveto(881.9218727,615.76927305)(881.92687269,615.82427299)(881.91687837,615.88427508)
\curveto(881.90687271,615.94427287)(881.89687272,615.98927283)(881.88687837,616.01927508)
\curveto(881.83687278,616.20927261)(881.76687285,616.34927247)(881.67687837,616.43927508)
\curveto(881.58687303,616.53927228)(881.47187315,616.62927219)(881.33187837,616.70927508)
\curveto(881.24187338,616.76927205)(881.14187348,616.819272)(881.03187837,616.85927508)
\lineto(880.70187837,616.97927508)
\curveto(880.67187395,616.98927183)(880.64187398,616.99427182)(880.61187837,616.99427508)
\curveto(880.59187403,616.99427182)(880.56687405,617.00427181)(880.53687837,617.02427508)
\curveto(880.19687442,617.13427168)(879.84187478,617.2142716)(879.47187837,617.26427508)
\curveto(879.11187551,617.32427149)(878.77187585,617.4192714)(878.45187837,617.54927508)
\curveto(878.35187627,617.58927123)(878.25687636,617.62427119)(878.16687837,617.65427508)
\curveto(878.07687654,617.68427113)(877.99187663,617.72427109)(877.91187837,617.77427508)
\curveto(877.7218769,617.88427093)(877.54687707,618.00927081)(877.38687837,618.14927508)
\curveto(877.22687739,618.28927053)(877.10187752,618.46427035)(877.01187837,618.67427508)
\curveto(876.98187764,618.74427007)(876.95687766,618.81427)(876.93687837,618.88427508)
\curveto(876.92687769,618.95426986)(876.91187771,619.02926979)(876.89187837,619.10927508)
\curveto(876.86187776,619.22926959)(876.85187777,619.36426945)(876.86187837,619.51427508)
\curveto(876.87187775,619.67426914)(876.88687773,619.80926901)(876.90687837,619.91927508)
\curveto(876.92687769,619.96926885)(876.93687768,620.00926881)(876.93687837,620.03927508)
\curveto(876.94687767,620.07926874)(876.96187766,620.1192687)(876.98187837,620.15927508)
\curveto(877.07187755,620.38926843)(877.19187743,620.58926823)(877.34187837,620.75927508)
\curveto(877.50187712,620.92926789)(877.68187694,621.07926774)(877.88187837,621.20927508)
\curveto(878.03187659,621.29926752)(878.19687642,621.36926745)(878.37687837,621.41927508)
\curveto(878.55687606,621.47926734)(878.74687587,621.53426728)(878.94687837,621.58427508)
\curveto(879.0168756,621.59426722)(879.08187554,621.60426721)(879.14187837,621.61427508)
\curveto(879.21187541,621.62426719)(879.28687533,621.63426718)(879.36687837,621.64427508)
\curveto(879.39687522,621.65426716)(879.43687518,621.65426716)(879.48687837,621.64427508)
\curveto(879.53687508,621.63426718)(879.57187505,621.63926718)(879.59187837,621.65927508)
}
}
{
\newrgbcolor{curcolor}{0.60000002 0.60000002 0.60000002}
\pscustom[linestyle=none,fillstyle=solid,fillcolor=curcolor]
{
\newpath
\moveto(812.80437349,624.4643117)
\lineto(827.80437349,624.4643117)
\lineto(827.80437349,609.4643117)
\lineto(812.80437349,609.4643117)
\closepath
}
}
{
\newrgbcolor{curcolor}{0 0 0}
\pscustom[linestyle=none,fillstyle=solid,fillcolor=curcolor]
{
\newpath
\moveto(832.8075815,601.39856951)
\lineto(837.4425815,601.39856951)
\lineto(838.6575815,601.39856951)
\curveto(838.76757395,601.39855882)(838.87257384,601.39855882)(838.9725815,601.39856951)
\curveto(839.08257363,601.39855882)(839.16757355,601.37855884)(839.2275815,601.33856951)
\curveto(839.30757341,601.28855893)(839.35257336,601.213559)(839.3625815,601.11356951)
\curveto(839.38257333,601.02355919)(839.39257332,600.9135593)(839.3925815,600.78356951)
\lineto(839.3925815,600.63356951)
\curveto(839.40257331,600.59355962)(839.39757332,600.55355966)(839.3775815,600.51356951)
\curveto(839.33757338,600.35355986)(839.24757347,600.26355995)(839.1075815,600.24356951)
\curveto(838.97757374,600.23355998)(838.8125739,600.22855999)(838.6125815,600.22856951)
\lineto(837.0525815,600.22856951)
\lineto(834.8475815,600.22856951)
\lineto(834.3375815,600.22856951)
\curveto(834.15757856,600.23855998)(834.02257869,600.20856001)(833.9325815,600.13856951)
\curveto(833.84257887,600.07856014)(833.79257892,599.97356024)(833.7825815,599.82356951)
\lineto(833.7825815,599.37356951)
\lineto(833.7825815,597.88856951)
\curveto(833.78257893,597.80856241)(833.77757894,597.70856251)(833.7675815,597.58856951)
\curveto(833.76757895,597.46856275)(833.77757894,597.36856285)(833.7975815,597.28856951)
\lineto(833.7975815,597.16856951)
\curveto(833.8175789,597.10856311)(833.83257888,597.04856317)(833.8425815,596.98856951)
\curveto(833.86257885,596.93856328)(833.89757882,596.89856332)(833.9475815,596.86856951)
\curveto(834.03757868,596.80856341)(834.17757854,596.77856344)(834.3675815,596.77856951)
\curveto(834.55757816,596.78856343)(834.72257799,596.79356342)(834.8625815,596.79356951)
\lineto(837.5625815,596.79356951)
\lineto(837.8475815,596.79356951)
\curveto(837.95757476,596.80356341)(838.06257465,596.80356341)(838.1625815,596.79356951)
\curveto(838.27257444,596.79356342)(838.36757435,596.78356343)(838.4475815,596.76356951)
\curveto(838.53757418,596.74356347)(838.59757412,596.70856351)(838.6275815,596.65856951)
\curveto(838.67757404,596.59856362)(838.70257401,596.52356369)(838.7025815,596.43356951)
\lineto(838.7025815,596.13356951)
\lineto(838.7025815,595.96856951)
\curveto(838.70257401,595.9185643)(838.69257402,595.87356434)(838.6725815,595.83356951)
\curveto(838.63257408,595.73356448)(838.57757414,595.67856454)(838.5075815,595.66856951)
\curveto(838.46757425,595.64856457)(838.42757429,595.63856458)(838.3875815,595.63856951)
\curveto(838.35757436,595.63856458)(838.3175744,595.63356458)(838.2675815,595.62356951)
\curveto(838.22757449,595.6135646)(838.18257453,595.60856461)(838.1325815,595.60856951)
\curveto(838.09257462,595.6185646)(838.05257466,595.62356459)(838.0125815,595.62356951)
\lineto(837.4875815,595.62356951)
\lineto(834.9525815,595.62356951)
\lineto(834.3825815,595.62356951)
\curveto(834.17257854,595.63356458)(834.02257869,595.60356461)(833.9325815,595.53356951)
\curveto(833.88257883,595.49356472)(833.84257887,595.42856479)(833.8125815,595.33856951)
\curveto(833.79257892,595.25856496)(833.77757894,595.16356505)(833.7675815,595.05356951)
\lineto(833.7675815,594.70856951)
\curveto(833.77757894,594.59856562)(833.78257893,594.49856572)(833.7825815,594.40856951)
\lineto(833.7825815,591.82856951)
\curveto(833.78257893,591.65856856)(833.78757893,591.47356874)(833.7975815,591.27356951)
\curveto(833.80757891,591.07356914)(833.77257894,590.92356929)(833.6925815,590.82356951)
\curveto(833.66257905,590.78356943)(833.6175791,590.75856946)(833.5575815,590.74856951)
\curveto(833.49757922,590.74856947)(833.43757928,590.73856948)(833.3775815,590.71856951)
\lineto(833.0925815,590.71856951)
\curveto(832.95257976,590.7185695)(832.82257989,590.72356949)(832.7025815,590.73356951)
\curveto(832.58258013,590.74356947)(832.49758022,590.79356942)(832.4475815,590.88356951)
\curveto(832.40758031,590.94356927)(832.38758033,591.02356919)(832.3875815,591.12356951)
\lineto(832.3875815,591.45356951)
\lineto(832.3875815,592.65356951)
\lineto(832.3875815,598.92356951)
\lineto(832.3875815,600.54356951)
\curveto(832.38758033,600.65355956)(832.38258033,600.77355944)(832.3725815,600.90356951)
\curveto(832.37258034,601.04355917)(832.39758032,601.15355906)(832.4475815,601.23356951)
\curveto(832.48758023,601.29355892)(832.56258015,601.34355887)(832.6725815,601.38356951)
\curveto(832.69258002,601.39355882)(832.71258,601.39355882)(832.7325815,601.38356951)
\curveto(832.76257995,601.38355883)(832.78757993,601.38855883)(832.8075815,601.39856951)
}
}
{
\newrgbcolor{curcolor}{0 0 0}
\pscustom[linestyle=none,fillstyle=solid,fillcolor=curcolor]
{
\newpath
\moveto(847.87086275,594.91856951)
\curveto(847.89085469,594.85856536)(847.90085468,594.76356545)(847.90086275,594.63356951)
\curveto(847.90085468,594.5135657)(847.89585468,594.42856579)(847.88586275,594.37856951)
\lineto(847.88586275,594.22856951)
\curveto(847.8758547,594.14856607)(847.86585471,594.07356614)(847.85586275,594.00356951)
\curveto(847.85585472,593.94356627)(847.85085473,593.87356634)(847.84086275,593.79356951)
\curveto(847.82085476,593.73356648)(847.80585477,593.67356654)(847.79586275,593.61356951)
\curveto(847.79585478,593.55356666)(847.78585479,593.49356672)(847.76586275,593.43356951)
\curveto(847.72585485,593.30356691)(847.69085489,593.17356704)(847.66086275,593.04356951)
\curveto(847.63085495,592.9135673)(847.59085499,592.79356742)(847.54086275,592.68356951)
\curveto(847.33085525,592.20356801)(847.05085553,591.79856842)(846.70086275,591.46856951)
\curveto(846.35085623,591.14856907)(845.92085666,590.90356931)(845.41086275,590.73356951)
\curveto(845.30085728,590.69356952)(845.1808574,590.66356955)(845.05086275,590.64356951)
\curveto(844.93085765,590.62356959)(844.80585777,590.60356961)(844.67586275,590.58356951)
\curveto(844.61585796,590.57356964)(844.55085803,590.56856965)(844.48086275,590.56856951)
\curveto(844.42085816,590.55856966)(844.36085822,590.55356966)(844.30086275,590.55356951)
\curveto(844.26085832,590.54356967)(844.20085838,590.53856968)(844.12086275,590.53856951)
\curveto(844.05085853,590.53856968)(844.00085858,590.54356967)(843.97086275,590.55356951)
\curveto(843.93085865,590.56356965)(843.89085869,590.56856965)(843.85086275,590.56856951)
\curveto(843.81085877,590.55856966)(843.7758588,590.55856966)(843.74586275,590.56856951)
\lineto(843.65586275,590.56856951)
\lineto(843.29586275,590.61356951)
\curveto(843.15585942,590.65356956)(843.02085956,590.69356952)(842.89086275,590.73356951)
\curveto(842.76085982,590.77356944)(842.63585994,590.8185694)(842.51586275,590.86856951)
\curveto(842.06586051,591.06856915)(841.69586088,591.32856889)(841.40586275,591.64856951)
\curveto(841.11586146,591.96856825)(840.8758617,592.35856786)(840.68586275,592.81856951)
\curveto(840.63586194,592.9185673)(840.59586198,593.0185672)(840.56586275,593.11856951)
\curveto(840.54586203,593.218567)(840.52586205,593.32356689)(840.50586275,593.43356951)
\curveto(840.48586209,593.47356674)(840.4758621,593.50356671)(840.47586275,593.52356951)
\curveto(840.48586209,593.55356666)(840.48586209,593.58856663)(840.47586275,593.62856951)
\curveto(840.45586212,593.70856651)(840.44086214,593.78856643)(840.43086275,593.86856951)
\curveto(840.43086215,593.95856626)(840.42086216,594.04356617)(840.40086275,594.12356951)
\lineto(840.40086275,594.24356951)
\curveto(840.40086218,594.28356593)(840.39586218,594.32856589)(840.38586275,594.37856951)
\curveto(840.3758622,594.42856579)(840.37086221,594.5135657)(840.37086275,594.63356951)
\curveto(840.37086221,594.76356545)(840.3808622,594.85856536)(840.40086275,594.91856951)
\curveto(840.42086216,594.98856523)(840.42586215,595.05856516)(840.41586275,595.12856951)
\curveto(840.40586217,595.19856502)(840.41086217,595.26856495)(840.43086275,595.33856951)
\curveto(840.44086214,595.38856483)(840.44586213,595.42856479)(840.44586275,595.45856951)
\curveto(840.45586212,595.49856472)(840.46586211,595.54356467)(840.47586275,595.59356951)
\curveto(840.50586207,595.7135645)(840.53086205,595.83356438)(840.55086275,595.95356951)
\curveto(840.580862,596.07356414)(840.62086196,596.18856403)(840.67086275,596.29856951)
\curveto(840.82086176,596.66856355)(841.00086158,596.99856322)(841.21086275,597.28856951)
\curveto(841.43086115,597.58856263)(841.69586088,597.83856238)(842.00586275,598.03856951)
\curveto(842.12586045,598.1185621)(842.25086033,598.18356203)(842.38086275,598.23356951)
\curveto(842.51086007,598.29356192)(842.64585993,598.35356186)(842.78586275,598.41356951)
\curveto(842.90585967,598.46356175)(843.03585954,598.49356172)(843.17586275,598.50356951)
\curveto(843.31585926,598.52356169)(843.45585912,598.55356166)(843.59586275,598.59356951)
\lineto(843.79086275,598.59356951)
\curveto(843.86085872,598.60356161)(843.92585865,598.6135616)(843.98586275,598.62356951)
\curveto(844.8758577,598.63356158)(845.61585696,598.44856177)(846.20586275,598.06856951)
\curveto(846.79585578,597.68856253)(847.22085536,597.19356302)(847.48086275,596.58356951)
\curveto(847.53085505,596.48356373)(847.57085501,596.38356383)(847.60086275,596.28356951)
\curveto(847.63085495,596.18356403)(847.66585491,596.07856414)(847.70586275,595.96856951)
\curveto(847.73585484,595.85856436)(847.76085482,595.73856448)(847.78086275,595.60856951)
\curveto(847.80085478,595.48856473)(847.82585475,595.36356485)(847.85586275,595.23356951)
\curveto(847.86585471,595.18356503)(847.86585471,595.12856509)(847.85586275,595.06856951)
\curveto(847.85585472,595.0185652)(847.86085472,594.96856525)(847.87086275,594.91856951)
\moveto(846.53586275,594.06356951)
\curveto(846.55585602,594.13356608)(846.56085602,594.213566)(846.55086275,594.30356951)
\lineto(846.55086275,594.55856951)
\curveto(846.55085603,594.94856527)(846.51585606,595.27856494)(846.44586275,595.54856951)
\curveto(846.41585616,595.62856459)(846.39085619,595.70856451)(846.37086275,595.78856951)
\curveto(846.35085623,595.86856435)(846.32585625,595.94356427)(846.29586275,596.01356951)
\curveto(846.01585656,596.66356355)(845.57085701,597.1135631)(844.96086275,597.36356951)
\curveto(844.89085769,597.39356282)(844.81585776,597.4135628)(844.73586275,597.42356951)
\lineto(844.49586275,597.48356951)
\curveto(844.41585816,597.50356271)(844.33085825,597.5135627)(844.24086275,597.51356951)
\lineto(843.97086275,597.51356951)
\lineto(843.70086275,597.46856951)
\curveto(843.60085898,597.44856277)(843.50585907,597.42356279)(843.41586275,597.39356951)
\curveto(843.33585924,597.37356284)(843.25585932,597.34356287)(843.17586275,597.30356951)
\curveto(843.10585947,597.28356293)(843.04085954,597.25356296)(842.98086275,597.21356951)
\curveto(842.92085966,597.17356304)(842.86585971,597.13356308)(842.81586275,597.09356951)
\curveto(842.57586,596.92356329)(842.3808602,596.7185635)(842.23086275,596.47856951)
\curveto(842.0808605,596.23856398)(841.95086063,595.95856426)(841.84086275,595.63856951)
\curveto(841.81086077,595.53856468)(841.79086079,595.43356478)(841.78086275,595.32356951)
\curveto(841.77086081,595.22356499)(841.75586082,595.1185651)(841.73586275,595.00856951)
\curveto(841.72586085,594.96856525)(841.72086086,594.90356531)(841.72086275,594.81356951)
\curveto(841.71086087,594.78356543)(841.70586087,594.74856547)(841.70586275,594.70856951)
\curveto(841.71586086,594.66856555)(841.72086086,594.62356559)(841.72086275,594.57356951)
\lineto(841.72086275,594.27356951)
\curveto(841.72086086,594.17356604)(841.73086085,594.08356613)(841.75086275,594.00356951)
\lineto(841.78086275,593.82356951)
\curveto(841.80086078,593.72356649)(841.81586076,593.62356659)(841.82586275,593.52356951)
\curveto(841.84586073,593.43356678)(841.8758607,593.34856687)(841.91586275,593.26856951)
\curveto(842.01586056,593.02856719)(842.13086045,592.80356741)(842.26086275,592.59356951)
\curveto(842.40086018,592.38356783)(842.57086001,592.20856801)(842.77086275,592.06856951)
\curveto(842.82085976,592.03856818)(842.86585971,592.0135682)(842.90586275,591.99356951)
\curveto(842.94585963,591.97356824)(842.99085959,591.94856827)(843.04086275,591.91856951)
\curveto(843.12085946,591.86856835)(843.20585937,591.82356839)(843.29586275,591.78356951)
\curveto(843.39585918,591.75356846)(843.50085908,591.72356849)(843.61086275,591.69356951)
\curveto(843.66085892,591.67356854)(843.70585887,591.66356855)(843.74586275,591.66356951)
\curveto(843.79585878,591.67356854)(843.84585873,591.67356854)(843.89586275,591.66356951)
\curveto(843.92585865,591.65356856)(843.98585859,591.64356857)(844.07586275,591.63356951)
\curveto(844.1758584,591.62356859)(844.25085833,591.62856859)(844.30086275,591.64856951)
\curveto(844.34085824,591.65856856)(844.3808582,591.65856856)(844.42086275,591.64856951)
\curveto(844.46085812,591.64856857)(844.50085808,591.65856856)(844.54086275,591.67856951)
\curveto(844.62085796,591.69856852)(844.70085788,591.7135685)(844.78086275,591.72356951)
\curveto(844.86085772,591.74356847)(844.93585764,591.76856845)(845.00586275,591.79856951)
\curveto(845.34585723,591.93856828)(845.62085696,592.13356808)(845.83086275,592.38356951)
\curveto(846.04085654,592.63356758)(846.21585636,592.92856729)(846.35586275,593.26856951)
\curveto(846.40585617,593.38856683)(846.43585614,593.5135667)(846.44586275,593.64356951)
\curveto(846.46585611,593.78356643)(846.49585608,593.92356629)(846.53586275,594.06356951)
}
}
{
\newrgbcolor{curcolor}{0 0 0}
\pscustom[linestyle=none,fillstyle=solid,fillcolor=curcolor]
{
\newpath
\moveto(850.304144,600.78356951)
\curveto(850.45414199,600.78355943)(850.60414184,600.77855944)(850.754144,600.76856951)
\curveto(850.90414154,600.76855945)(851.00914143,600.72855949)(851.069144,600.64856951)
\curveto(851.11914132,600.58855963)(851.1441413,600.50355971)(851.144144,600.39356951)
\curveto(851.15414129,600.29355992)(851.15914128,600.18856003)(851.159144,600.07856951)
\lineto(851.159144,599.20856951)
\curveto(851.15914128,599.12856109)(851.15414129,599.04356117)(851.144144,598.95356951)
\curveto(851.1441413,598.87356134)(851.15414129,598.80356141)(851.174144,598.74356951)
\curveto(851.21414123,598.60356161)(851.30414114,598.5135617)(851.444144,598.47356951)
\curveto(851.49414095,598.46356175)(851.5391409,598.45856176)(851.579144,598.45856951)
\lineto(851.729144,598.45856951)
\lineto(852.134144,598.45856951)
\curveto(852.29414015,598.46856175)(852.40914003,598.45856176)(852.479144,598.42856951)
\curveto(852.56913987,598.36856185)(852.62913981,598.30856191)(852.659144,598.24856951)
\curveto(852.67913976,598.20856201)(852.68913975,598.16356205)(852.689144,598.11356951)
\lineto(852.689144,597.96356951)
\curveto(852.68913975,597.85356236)(852.68413976,597.74856247)(852.674144,597.64856951)
\curveto(852.66413978,597.55856266)(852.62913981,597.48856273)(852.569144,597.43856951)
\curveto(852.50913993,597.38856283)(852.42414002,597.35856286)(852.314144,597.34856951)
\lineto(851.984144,597.34856951)
\curveto(851.87414057,597.35856286)(851.76414068,597.36356285)(851.654144,597.36356951)
\curveto(851.5441409,597.36356285)(851.44914099,597.34856287)(851.369144,597.31856951)
\curveto(851.29914114,597.28856293)(851.24914119,597.23856298)(851.219144,597.16856951)
\curveto(851.18914125,597.09856312)(851.16914127,597.0135632)(851.159144,596.91356951)
\curveto(851.14914129,596.82356339)(851.1441413,596.72356349)(851.144144,596.61356951)
\curveto(851.15414129,596.5135637)(851.15914128,596.4135638)(851.159144,596.31356951)
\lineto(851.159144,593.34356951)
\curveto(851.15914128,593.12356709)(851.15414129,592.88856733)(851.144144,592.63856951)
\curveto(851.1441413,592.39856782)(851.18914125,592.213568)(851.279144,592.08356951)
\curveto(851.32914111,592.00356821)(851.39414105,591.94856827)(851.474144,591.91856951)
\curveto(851.55414089,591.88856833)(851.64914079,591.86356835)(851.759144,591.84356951)
\curveto(851.78914065,591.83356838)(851.81914062,591.82856839)(851.849144,591.82856951)
\curveto(851.88914055,591.83856838)(851.92414052,591.83856838)(851.954144,591.82856951)
\lineto(852.149144,591.82856951)
\curveto(852.24914019,591.82856839)(852.3391401,591.8185684)(852.419144,591.79856951)
\curveto(852.50913993,591.78856843)(852.57413987,591.75356846)(852.614144,591.69356951)
\curveto(852.63413981,591.66356855)(852.64913979,591.60856861)(852.659144,591.52856951)
\curveto(852.67913976,591.45856876)(852.68913975,591.38356883)(852.689144,591.30356951)
\curveto(852.69913974,591.22356899)(852.69913974,591.14356907)(852.689144,591.06356951)
\curveto(852.67913976,590.99356922)(852.65913978,590.93856928)(852.629144,590.89856951)
\curveto(852.58913985,590.82856939)(852.51413993,590.77856944)(852.404144,590.74856951)
\curveto(852.32414012,590.72856949)(852.23414021,590.7185695)(852.134144,590.71856951)
\curveto(852.03414041,590.72856949)(851.9441405,590.73356948)(851.864144,590.73356951)
\curveto(851.80414064,590.73356948)(851.7441407,590.72856949)(851.684144,590.71856951)
\curveto(851.62414082,590.7185695)(851.56914087,590.72356949)(851.519144,590.73356951)
\lineto(851.339144,590.73356951)
\curveto(851.28914115,590.74356947)(851.2391412,590.74856947)(851.189144,590.74856951)
\curveto(851.14914129,590.75856946)(851.10414134,590.76356945)(851.054144,590.76356951)
\curveto(850.85414159,590.8135694)(850.67914176,590.86856935)(850.529144,590.92856951)
\curveto(850.38914205,590.98856923)(850.26914217,591.09356912)(850.169144,591.24356951)
\curveto(850.02914241,591.44356877)(849.94914249,591.69356852)(849.929144,591.99356951)
\curveto(849.90914253,592.30356791)(849.89914254,592.63356758)(849.899144,592.98356951)
\lineto(849.899144,596.91356951)
\curveto(849.86914257,597.04356317)(849.8391426,597.13856308)(849.809144,597.19856951)
\curveto(849.78914265,597.25856296)(849.71914272,597.30856291)(849.599144,597.34856951)
\curveto(849.55914288,597.35856286)(849.51914292,597.35856286)(849.479144,597.34856951)
\curveto(849.439143,597.33856288)(849.39914304,597.34356287)(849.359144,597.36356951)
\lineto(849.119144,597.36356951)
\curveto(848.98914345,597.36356285)(848.87914356,597.37356284)(848.789144,597.39356951)
\curveto(848.70914373,597.42356279)(848.65414379,597.48356273)(848.624144,597.57356951)
\curveto(848.60414384,597.6135626)(848.58914385,597.65856256)(848.579144,597.70856951)
\lineto(848.579144,597.85856951)
\curveto(848.57914386,597.99856222)(848.58914385,598.1135621)(848.609144,598.20356951)
\curveto(848.62914381,598.30356191)(848.68914375,598.37856184)(848.789144,598.42856951)
\curveto(848.89914354,598.46856175)(849.0391434,598.47856174)(849.209144,598.45856951)
\curveto(849.38914305,598.43856178)(849.5391429,598.44856177)(849.659144,598.48856951)
\curveto(849.74914269,598.53856168)(849.81914262,598.60856161)(849.869144,598.69856951)
\curveto(849.88914255,598.75856146)(849.89914254,598.83356138)(849.899144,598.92356951)
\lineto(849.899144,599.17856951)
\lineto(849.899144,600.10856951)
\lineto(849.899144,600.34856951)
\curveto(849.89914254,600.43855978)(849.90914253,600.5135597)(849.929144,600.57356951)
\curveto(849.96914247,600.65355956)(850.0441424,600.7185595)(850.154144,600.76856951)
\curveto(850.18414226,600.76855945)(850.20914223,600.76855945)(850.229144,600.76856951)
\curveto(850.25914218,600.77855944)(850.28414216,600.78355943)(850.304144,600.78356951)
}
}
{
\newrgbcolor{curcolor}{0 0 0}
\pscustom[linestyle=none,fillstyle=solid,fillcolor=curcolor]
{
\newpath
\moveto(861.20094087,594.91856951)
\curveto(861.22093281,594.85856536)(861.2309328,594.76356545)(861.23094087,594.63356951)
\curveto(861.2309328,594.5135657)(861.22593281,594.42856579)(861.21594087,594.37856951)
\lineto(861.21594087,594.22856951)
\curveto(861.20593283,594.14856607)(861.19593284,594.07356614)(861.18594087,594.00356951)
\curveto(861.18593285,593.94356627)(861.18093285,593.87356634)(861.17094087,593.79356951)
\curveto(861.15093288,593.73356648)(861.1359329,593.67356654)(861.12594087,593.61356951)
\curveto(861.12593291,593.55356666)(861.11593292,593.49356672)(861.09594087,593.43356951)
\curveto(861.05593298,593.30356691)(861.02093301,593.17356704)(860.99094087,593.04356951)
\curveto(860.96093307,592.9135673)(860.92093311,592.79356742)(860.87094087,592.68356951)
\curveto(860.66093337,592.20356801)(860.38093365,591.79856842)(860.03094087,591.46856951)
\curveto(859.68093435,591.14856907)(859.25093478,590.90356931)(858.74094087,590.73356951)
\curveto(858.6309354,590.69356952)(858.51093552,590.66356955)(858.38094087,590.64356951)
\curveto(858.26093577,590.62356959)(858.1359359,590.60356961)(858.00594087,590.58356951)
\curveto(857.94593609,590.57356964)(857.88093615,590.56856965)(857.81094087,590.56856951)
\curveto(857.75093628,590.55856966)(857.69093634,590.55356966)(857.63094087,590.55356951)
\curveto(857.59093644,590.54356967)(857.5309365,590.53856968)(857.45094087,590.53856951)
\curveto(857.38093665,590.53856968)(857.3309367,590.54356967)(857.30094087,590.55356951)
\curveto(857.26093677,590.56356965)(857.22093681,590.56856965)(857.18094087,590.56856951)
\curveto(857.14093689,590.55856966)(857.10593693,590.55856966)(857.07594087,590.56856951)
\lineto(856.98594087,590.56856951)
\lineto(856.62594087,590.61356951)
\curveto(856.48593755,590.65356956)(856.35093768,590.69356952)(856.22094087,590.73356951)
\curveto(856.09093794,590.77356944)(855.96593807,590.8185694)(855.84594087,590.86856951)
\curveto(855.39593864,591.06856915)(855.02593901,591.32856889)(854.73594087,591.64856951)
\curveto(854.44593959,591.96856825)(854.20593983,592.35856786)(854.01594087,592.81856951)
\curveto(853.96594007,592.9185673)(853.92594011,593.0185672)(853.89594087,593.11856951)
\curveto(853.87594016,593.218567)(853.85594018,593.32356689)(853.83594087,593.43356951)
\curveto(853.81594022,593.47356674)(853.80594023,593.50356671)(853.80594087,593.52356951)
\curveto(853.81594022,593.55356666)(853.81594022,593.58856663)(853.80594087,593.62856951)
\curveto(853.78594025,593.70856651)(853.77094026,593.78856643)(853.76094087,593.86856951)
\curveto(853.76094027,593.95856626)(853.75094028,594.04356617)(853.73094087,594.12356951)
\lineto(853.73094087,594.24356951)
\curveto(853.7309403,594.28356593)(853.72594031,594.32856589)(853.71594087,594.37856951)
\curveto(853.70594033,594.42856579)(853.70094033,594.5135657)(853.70094087,594.63356951)
\curveto(853.70094033,594.76356545)(853.71094032,594.85856536)(853.73094087,594.91856951)
\curveto(853.75094028,594.98856523)(853.75594028,595.05856516)(853.74594087,595.12856951)
\curveto(853.7359403,595.19856502)(853.74094029,595.26856495)(853.76094087,595.33856951)
\curveto(853.77094026,595.38856483)(853.77594026,595.42856479)(853.77594087,595.45856951)
\curveto(853.78594025,595.49856472)(853.79594024,595.54356467)(853.80594087,595.59356951)
\curveto(853.8359402,595.7135645)(853.86094017,595.83356438)(853.88094087,595.95356951)
\curveto(853.91094012,596.07356414)(853.95094008,596.18856403)(854.00094087,596.29856951)
\curveto(854.15093988,596.66856355)(854.3309397,596.99856322)(854.54094087,597.28856951)
\curveto(854.76093927,597.58856263)(855.02593901,597.83856238)(855.33594087,598.03856951)
\curveto(855.45593858,598.1185621)(855.58093845,598.18356203)(855.71094087,598.23356951)
\curveto(855.84093819,598.29356192)(855.97593806,598.35356186)(856.11594087,598.41356951)
\curveto(856.2359378,598.46356175)(856.36593767,598.49356172)(856.50594087,598.50356951)
\curveto(856.64593739,598.52356169)(856.78593725,598.55356166)(856.92594087,598.59356951)
\lineto(857.12094087,598.59356951)
\curveto(857.19093684,598.60356161)(857.25593678,598.6135616)(857.31594087,598.62356951)
\curveto(858.20593583,598.63356158)(858.94593509,598.44856177)(859.53594087,598.06856951)
\curveto(860.12593391,597.68856253)(860.55093348,597.19356302)(860.81094087,596.58356951)
\curveto(860.86093317,596.48356373)(860.90093313,596.38356383)(860.93094087,596.28356951)
\curveto(860.96093307,596.18356403)(860.99593304,596.07856414)(861.03594087,595.96856951)
\curveto(861.06593297,595.85856436)(861.09093294,595.73856448)(861.11094087,595.60856951)
\curveto(861.1309329,595.48856473)(861.15593288,595.36356485)(861.18594087,595.23356951)
\curveto(861.19593284,595.18356503)(861.19593284,595.12856509)(861.18594087,595.06856951)
\curveto(861.18593285,595.0185652)(861.19093284,594.96856525)(861.20094087,594.91856951)
\moveto(859.86594087,594.06356951)
\curveto(859.88593415,594.13356608)(859.89093414,594.213566)(859.88094087,594.30356951)
\lineto(859.88094087,594.55856951)
\curveto(859.88093415,594.94856527)(859.84593419,595.27856494)(859.77594087,595.54856951)
\curveto(859.74593429,595.62856459)(859.72093431,595.70856451)(859.70094087,595.78856951)
\curveto(859.68093435,595.86856435)(859.65593438,595.94356427)(859.62594087,596.01356951)
\curveto(859.34593469,596.66356355)(858.90093513,597.1135631)(858.29094087,597.36356951)
\curveto(858.22093581,597.39356282)(858.14593589,597.4135628)(858.06594087,597.42356951)
\lineto(857.82594087,597.48356951)
\curveto(857.74593629,597.50356271)(857.66093637,597.5135627)(857.57094087,597.51356951)
\lineto(857.30094087,597.51356951)
\lineto(857.03094087,597.46856951)
\curveto(856.9309371,597.44856277)(856.8359372,597.42356279)(856.74594087,597.39356951)
\curveto(856.66593737,597.37356284)(856.58593745,597.34356287)(856.50594087,597.30356951)
\curveto(856.4359376,597.28356293)(856.37093766,597.25356296)(856.31094087,597.21356951)
\curveto(856.25093778,597.17356304)(856.19593784,597.13356308)(856.14594087,597.09356951)
\curveto(855.90593813,596.92356329)(855.71093832,596.7185635)(855.56094087,596.47856951)
\curveto(855.41093862,596.23856398)(855.28093875,595.95856426)(855.17094087,595.63856951)
\curveto(855.14093889,595.53856468)(855.12093891,595.43356478)(855.11094087,595.32356951)
\curveto(855.10093893,595.22356499)(855.08593895,595.1185651)(855.06594087,595.00856951)
\curveto(855.05593898,594.96856525)(855.05093898,594.90356531)(855.05094087,594.81356951)
\curveto(855.04093899,594.78356543)(855.035939,594.74856547)(855.03594087,594.70856951)
\curveto(855.04593899,594.66856555)(855.05093898,594.62356559)(855.05094087,594.57356951)
\lineto(855.05094087,594.27356951)
\curveto(855.05093898,594.17356604)(855.06093897,594.08356613)(855.08094087,594.00356951)
\lineto(855.11094087,593.82356951)
\curveto(855.1309389,593.72356649)(855.14593889,593.62356659)(855.15594087,593.52356951)
\curveto(855.17593886,593.43356678)(855.20593883,593.34856687)(855.24594087,593.26856951)
\curveto(855.34593869,593.02856719)(855.46093857,592.80356741)(855.59094087,592.59356951)
\curveto(855.7309383,592.38356783)(855.90093813,592.20856801)(856.10094087,592.06856951)
\curveto(856.15093788,592.03856818)(856.19593784,592.0135682)(856.23594087,591.99356951)
\curveto(856.27593776,591.97356824)(856.32093771,591.94856827)(856.37094087,591.91856951)
\curveto(856.45093758,591.86856835)(856.5359375,591.82356839)(856.62594087,591.78356951)
\curveto(856.72593731,591.75356846)(856.8309372,591.72356849)(856.94094087,591.69356951)
\curveto(856.99093704,591.67356854)(857.035937,591.66356855)(857.07594087,591.66356951)
\curveto(857.12593691,591.67356854)(857.17593686,591.67356854)(857.22594087,591.66356951)
\curveto(857.25593678,591.65356856)(857.31593672,591.64356857)(857.40594087,591.63356951)
\curveto(857.50593653,591.62356859)(857.58093645,591.62856859)(857.63094087,591.64856951)
\curveto(857.67093636,591.65856856)(857.71093632,591.65856856)(857.75094087,591.64856951)
\curveto(857.79093624,591.64856857)(857.8309362,591.65856856)(857.87094087,591.67856951)
\curveto(857.95093608,591.69856852)(858.030936,591.7135685)(858.11094087,591.72356951)
\curveto(858.19093584,591.74356847)(858.26593577,591.76856845)(858.33594087,591.79856951)
\curveto(858.67593536,591.93856828)(858.95093508,592.13356808)(859.16094087,592.38356951)
\curveto(859.37093466,592.63356758)(859.54593449,592.92856729)(859.68594087,593.26856951)
\curveto(859.7359343,593.38856683)(859.76593427,593.5135667)(859.77594087,593.64356951)
\curveto(859.79593424,593.78356643)(859.82593421,593.92356629)(859.86594087,594.06356951)
}
}
{
\newrgbcolor{curcolor}{0 0 0}
\pscustom[linestyle=none,fillstyle=solid,fillcolor=curcolor]
{
\newpath
\moveto(869.30422212,598.33856951)
\curveto(869.37421452,598.28856193)(869.40921449,598.213562)(869.40922212,598.11356951)
\curveto(869.41921448,598.0135622)(869.42421447,597.90856231)(869.42422212,597.79856951)
\lineto(869.42422212,591.52856951)
\lineto(869.42422212,590.92856951)
\curveto(869.40421449,590.87856934)(869.3992145,590.82856939)(869.40922212,590.77856951)
\curveto(869.41921448,590.73856948)(869.41421448,590.69356952)(869.39422212,590.64356951)
\curveto(869.37421452,590.54356967)(869.35921454,590.44356977)(869.34922212,590.34356951)
\curveto(869.34921455,590.23356998)(869.33421456,590.12857009)(869.30422212,590.02856951)
\curveto(869.27421462,589.9185703)(869.24421465,589.8135704)(869.21422212,589.71356951)
\curveto(869.1942147,589.6135706)(869.15921474,589.5135707)(869.10922212,589.41356951)
\curveto(869.00921489,589.15357106)(868.87921502,588.9185713)(868.71922212,588.70856951)
\curveto(868.56921533,588.49857172)(868.38921551,588.32357189)(868.17922212,588.18356951)
\curveto(868.00921589,588.06357215)(867.82921607,587.96857225)(867.63922212,587.89856951)
\curveto(867.44921645,587.8185724)(867.24421665,587.74357247)(867.02422212,587.67356951)
\curveto(866.93421696,587.65357256)(866.84421705,587.64357257)(866.75422212,587.64356951)
\curveto(866.66421723,587.63357258)(866.57421732,587.6185726)(866.48422212,587.59856951)
\lineto(866.39422212,587.59856951)
\curveto(866.37421752,587.58857263)(866.35421754,587.58357263)(866.33422212,587.58356951)
\curveto(866.28421761,587.57357264)(866.23421766,587.57357264)(866.18422212,587.58356951)
\curveto(866.14421775,587.59357262)(866.0992178,587.58857263)(866.04922212,587.56856951)
\curveto(865.97921792,587.54857267)(865.86921803,587.54357267)(865.71922212,587.55356951)
\curveto(865.57921832,587.55357266)(865.47921842,587.56357265)(865.41922212,587.58356951)
\curveto(865.38921851,587.58357263)(865.35921854,587.58857263)(865.32922212,587.59856951)
\lineto(865.26922212,587.59856951)
\curveto(865.17921872,587.6185726)(865.08921881,587.63357258)(864.99922212,587.64356951)
\curveto(864.90921899,587.64357257)(864.82421907,587.65357256)(864.74422212,587.67356951)
\curveto(864.66421923,587.69357252)(864.58421931,587.7185725)(864.50422212,587.74856951)
\curveto(864.42421947,587.76857245)(864.34421955,587.79357242)(864.26422212,587.82356951)
\curveto(863.94421995,587.95357226)(863.67422022,588.09857212)(863.45422212,588.25856951)
\curveto(863.24422065,588.4185718)(863.05422084,588.64357157)(862.88422212,588.93356951)
\curveto(862.86422103,588.95357126)(862.84922105,588.97857124)(862.83922212,589.00856951)
\curveto(862.83922106,589.02857119)(862.82922107,589.05357116)(862.80922212,589.08356951)
\curveto(862.77922112,589.16357105)(862.74422115,589.27857094)(862.70422212,589.42856951)
\curveto(862.67422122,589.56857065)(862.70422119,589.67357054)(862.79422212,589.74356951)
\curveto(862.85422104,589.79357042)(862.93422096,589.8185704)(863.03422212,589.81856951)
\lineto(863.36422212,589.81856951)
\lineto(863.52922212,589.81856951)
\curveto(863.58922031,589.8185704)(863.64422025,589.80857041)(863.69422212,589.78856951)
\curveto(863.78422011,589.75857046)(863.84922005,589.70857051)(863.88922212,589.63856951)
\curveto(863.92921997,589.56857065)(863.97421992,589.49357072)(864.02422212,589.41356951)
\lineto(864.14422212,589.23356951)
\curveto(864.1942197,589.16357105)(864.24421965,589.10857111)(864.29422212,589.06856951)
\curveto(864.54421935,588.87857134)(864.84421905,588.73857148)(865.19422212,588.64856951)
\curveto(865.25421864,588.62857159)(865.31421858,588.6185716)(865.37422212,588.61856951)
\curveto(865.44421845,588.60857161)(865.50921839,588.59357162)(865.56922212,588.57356951)
\lineto(865.65922212,588.57356951)
\curveto(865.72921817,588.55357166)(865.81421808,588.54357167)(865.91422212,588.54356951)
\curveto(866.01421788,588.54357167)(866.10421779,588.55357166)(866.18422212,588.57356951)
\curveto(866.21421768,588.58357163)(866.25421764,588.58857163)(866.30422212,588.58856951)
\curveto(866.40421749,588.60857161)(866.4992174,588.62857159)(866.58922212,588.64856951)
\curveto(866.67921722,588.65857156)(866.76421713,588.68357153)(866.84422212,588.72356951)
\curveto(867.13421676,588.84357137)(867.36921653,589.00857121)(867.54922212,589.21856951)
\curveto(867.73921616,589.4185708)(867.894216,589.66357055)(868.01422212,589.95356951)
\curveto(868.05421584,590.04357017)(868.07921582,590.13857008)(868.08922212,590.23856951)
\curveto(868.10921579,590.33856988)(868.13421576,590.44356977)(868.16422212,590.55356951)
\curveto(868.18421571,590.60356961)(868.1942157,590.65356956)(868.19422212,590.70356951)
\curveto(868.1942157,590.75356946)(868.1992157,590.80356941)(868.20922212,590.85356951)
\curveto(868.21921568,590.88356933)(868.22421567,590.93356928)(868.22422212,591.00356951)
\curveto(868.24421565,591.08356913)(868.24421565,591.16856905)(868.22422212,591.25856951)
\curveto(868.21421568,591.30856891)(868.20921569,591.35356886)(868.20922212,591.39356951)
\curveto(868.21921568,591.43356878)(868.21421568,591.46856875)(868.19422212,591.49856951)
\curveto(868.17421572,591.5185687)(868.15921574,591.52856869)(868.14922212,591.52856951)
\lineto(868.10422212,591.57356951)
\curveto(868.00421589,591.57356864)(867.92921597,591.54356867)(867.87922212,591.48356951)
\curveto(867.83921606,591.43356878)(867.78921611,591.38856883)(867.72922212,591.34856951)
\lineto(867.48922212,591.13856951)
\curveto(867.40921649,591.07856914)(867.31921658,591.02356919)(867.21922212,590.97356951)
\curveto(867.07921682,590.88356933)(866.90421699,590.80856941)(866.69422212,590.74856951)
\curveto(866.48421741,590.69856952)(866.26421763,590.66356955)(866.03422212,590.64356951)
\curveto(865.80421809,590.62356959)(865.57421832,590.62856959)(865.34422212,590.65856951)
\curveto(865.11421878,590.67856954)(864.90421899,590.7185695)(864.71422212,590.77856951)
\curveto(863.77422012,591.08856913)(863.11422078,591.68356853)(862.73422212,592.56356951)
\curveto(862.68422121,592.66356755)(862.64422125,592.75856746)(862.61422212,592.84856951)
\curveto(862.58422131,592.94856727)(862.54922135,593.05356716)(862.50922212,593.16356951)
\curveto(862.48922141,593.213567)(862.47922142,593.25856696)(862.47922212,593.29856951)
\curveto(862.47922142,593.33856688)(862.46922143,593.38356683)(862.44922212,593.43356951)
\curveto(862.42922147,593.50356671)(862.41422148,593.57356664)(862.40422212,593.64356951)
\curveto(862.40422149,593.72356649)(862.3942215,593.79856642)(862.37422212,593.86856951)
\curveto(862.36422153,593.90856631)(862.35922154,593.94356627)(862.35922212,593.97356951)
\curveto(862.36922153,594.0135662)(862.36922153,594.05356616)(862.35922212,594.09356951)
\curveto(862.35922154,594.13356608)(862.35422154,594.17356604)(862.34422212,594.21356951)
\lineto(862.34422212,594.33356951)
\curveto(862.32422157,594.45356576)(862.32422157,594.57856564)(862.34422212,594.70856951)
\curveto(862.35422154,594.76856545)(862.35922154,594.82856539)(862.35922212,594.88856951)
\lineto(862.35922212,595.05356951)
\curveto(862.36922153,595.10356511)(862.37422152,595.14356507)(862.37422212,595.17356951)
\curveto(862.37422152,595.213565)(862.37922152,595.25856496)(862.38922212,595.30856951)
\curveto(862.41922148,595.4185648)(862.43922146,595.52356469)(862.44922212,595.62356951)
\curveto(862.45922144,595.73356448)(862.48422141,595.84356437)(862.52422212,595.95356951)
\curveto(862.56422133,596.07356414)(862.5992213,596.18856403)(862.62922212,596.29856951)
\curveto(862.66922123,596.4185638)(862.71422118,596.53356368)(862.76422212,596.64356951)
\curveto(862.83422106,596.80356341)(862.91422098,596.94856327)(863.00422212,597.07856951)
\curveto(863.0942208,597.218563)(863.18922071,597.35356286)(863.28922212,597.48356951)
\curveto(863.35922054,597.59356262)(863.44922045,597.68356253)(863.55922212,597.75356951)
\lineto(863.61922212,597.81356951)
\lineto(863.67922212,597.87356951)
\lineto(863.82922212,597.99356951)
\lineto(864.00922212,598.11356951)
\curveto(864.13921976,598.19356202)(864.27421962,598.26356195)(864.41422212,598.32356951)
\curveto(864.56421933,598.38356183)(864.72421917,598.43856178)(864.89422212,598.48856951)
\curveto(864.9942189,598.5185617)(865.0942188,598.53856168)(865.19422212,598.54856951)
\curveto(865.30421859,598.55856166)(865.41421848,598.57356164)(865.52422212,598.59356951)
\curveto(865.56421833,598.60356161)(865.61421828,598.60356161)(865.67422212,598.59356951)
\curveto(865.74421815,598.58356163)(865.7942181,598.58856163)(865.82422212,598.60856951)
\curveto(866.14421775,598.6185616)(866.42921747,598.58856163)(866.67922212,598.51856951)
\curveto(866.93921696,598.44856177)(867.16921673,598.34856187)(867.36922212,598.21856951)
\curveto(867.43921646,598.17856204)(867.50421639,598.13356208)(867.56422212,598.08356951)
\lineto(867.74422212,597.93356951)
\curveto(867.7942161,597.89356232)(867.83921606,597.84856237)(867.87922212,597.79856951)
\curveto(867.92921597,597.75856246)(868.00421589,597.73856248)(868.10422212,597.73856951)
\lineto(868.14922212,597.78356951)
\curveto(868.16921573,597.80356241)(868.18921571,597.82856239)(868.20922212,597.85856951)
\curveto(868.23921566,597.93856228)(868.25421564,598.0185622)(868.25422212,598.09856951)
\curveto(868.26421563,598.17856204)(868.2942156,598.24856197)(868.34422212,598.30856951)
\curveto(868.37421552,598.34856187)(868.43421546,598.37856184)(868.52422212,598.39856951)
\curveto(868.61421528,598.42856179)(868.70921519,598.44356177)(868.80922212,598.44356951)
\curveto(868.90921499,598.44356177)(869.00421489,598.43356178)(869.09422212,598.41356951)
\curveto(869.1942147,598.39356182)(869.26421463,598.36856185)(869.30422212,598.33856951)
\moveto(868.17922212,594.55856951)
\curveto(868.18921571,594.59856562)(868.1942157,594.64856557)(868.19422212,594.70856951)
\curveto(868.1942157,594.77856544)(868.18921571,594.83356538)(868.17922212,594.87356951)
\lineto(868.17922212,595.11356951)
\curveto(868.15921574,595.20356501)(868.14421575,595.28856493)(868.13422212,595.36856951)
\curveto(868.12421577,595.45856476)(868.10921579,595.54356467)(868.08922212,595.62356951)
\curveto(868.06921583,595.70356451)(868.04921585,595.77856444)(868.02922212,595.84856951)
\curveto(868.01921588,595.92856429)(867.9992159,596.00356421)(867.96922212,596.07356951)
\curveto(867.85921604,596.35356386)(867.71421618,596.60356361)(867.53422212,596.82356951)
\curveto(867.36421653,597.04356317)(867.14421675,597.20856301)(866.87422212,597.31856951)
\curveto(866.7942171,597.35856286)(866.70921719,597.38856283)(866.61922212,597.40856951)
\curveto(866.52921737,597.43856278)(866.43421746,597.46356275)(866.33422212,597.48356951)
\curveto(866.25421764,597.50356271)(866.16421773,597.50856271)(866.06422212,597.49856951)
\lineto(865.79422212,597.49856951)
\curveto(865.74421815,597.48856273)(865.6942182,597.48356273)(865.64422212,597.48356951)
\curveto(865.60421829,597.48356273)(865.55921834,597.47856274)(865.50922212,597.46856951)
\curveto(865.31921858,597.4185628)(865.15921874,597.36856285)(865.02922212,597.31856951)
\curveto(864.68921921,597.17856304)(864.42421947,596.96856325)(864.23422212,596.68856951)
\curveto(864.04421985,596.40856381)(863.89422,596.08356413)(863.78422212,595.71356951)
\curveto(863.76422013,595.63356458)(863.74922015,595.55356466)(863.73922212,595.47356951)
\curveto(863.73922016,595.40356481)(863.72922017,595.32856489)(863.70922212,595.24856951)
\curveto(863.68922021,595.218565)(863.67922022,595.18356503)(863.67922212,595.14356951)
\curveto(863.68922021,595.10356511)(863.68922021,595.06856515)(863.67922212,595.03856951)
\lineto(863.67922212,594.70856951)
\lineto(863.67922212,594.36356951)
\curveto(863.67922022,594.25356596)(863.68922021,594.14856607)(863.70922212,594.04856951)
\lineto(863.70922212,593.97356951)
\curveto(863.71922018,593.94356627)(863.72422017,593.9185663)(863.72422212,593.89856951)
\curveto(863.74422015,593.80856641)(863.75922014,593.7185665)(863.76922212,593.62856951)
\curveto(863.78922011,593.53856668)(863.81422008,593.45356676)(863.84422212,593.37356951)
\curveto(863.92421997,593.1135671)(864.02421987,592.87356734)(864.14422212,592.65356951)
\curveto(864.26421963,592.43356778)(864.42421947,592.25356796)(864.62422212,592.11356951)
\lineto(864.74422212,592.02356951)
\curveto(864.78421911,592.00356821)(864.82921907,591.98356823)(864.87922212,591.96356951)
\curveto(864.95921894,591.9135683)(865.04421885,591.87356834)(865.13422212,591.84356951)
\curveto(865.22421867,591.8135684)(865.32421857,591.78356843)(865.43422212,591.75356951)
\curveto(865.48421841,591.74356847)(865.52921837,591.73856848)(865.56922212,591.73856951)
\curveto(865.61921828,591.74856847)(865.66921823,591.74356847)(865.71922212,591.72356951)
\curveto(865.74921815,591.7135685)(865.7992181,591.70856851)(865.86922212,591.70856951)
\curveto(865.93921796,591.70856851)(865.98921791,591.7135685)(866.01922212,591.72356951)
\curveto(866.04921785,591.73356848)(866.07921782,591.73356848)(866.10922212,591.72356951)
\curveto(866.14921775,591.72356849)(866.18921771,591.72856849)(866.22922212,591.73856951)
\curveto(866.31921758,591.75856846)(866.40421749,591.77856844)(866.48422212,591.79856951)
\curveto(866.56421733,591.8185684)(866.64421725,591.84356837)(866.72422212,591.87356951)
\curveto(867.06421683,592.02356819)(867.33421656,592.23356798)(867.53422212,592.50356951)
\curveto(867.73421616,592.77356744)(867.894216,593.08856713)(868.01422212,593.44856951)
\curveto(868.04421585,593.53856668)(868.06421583,593.62856659)(868.07422212,593.71856951)
\curveto(868.0942158,593.8185664)(868.11421578,593.9135663)(868.13422212,594.00356951)
\curveto(868.14421575,594.04356617)(868.14921575,594.07856614)(868.14922212,594.10856951)
\curveto(868.14921575,594.14856607)(868.15421574,594.18856603)(868.16422212,594.22856951)
\curveto(868.18421571,594.27856594)(868.18421571,594.32856589)(868.16422212,594.37856951)
\curveto(868.15421574,594.43856578)(868.15921574,594.49856572)(868.17922212,594.55856951)
}
}
{
\newrgbcolor{curcolor}{0 0 0}
\pscustom[linestyle=none,fillstyle=solid,fillcolor=curcolor]
{
\newpath
\moveto(874.94750337,598.62356951)
\curveto(875.17749858,598.62356159)(875.30749845,598.56356165)(875.33750337,598.44356951)
\curveto(875.36749839,598.33356188)(875.38249838,598.16856205)(875.38250337,597.94856951)
\lineto(875.38250337,597.66356951)
\curveto(875.38249838,597.57356264)(875.3574984,597.49856272)(875.30750337,597.43856951)
\curveto(875.24749851,597.35856286)(875.1624986,597.3135629)(875.05250337,597.30356951)
\curveto(874.94249882,597.30356291)(874.83249893,597.28856293)(874.72250337,597.25856951)
\curveto(874.58249918,597.22856299)(874.44749931,597.19856302)(874.31750337,597.16856951)
\curveto(874.19749956,597.13856308)(874.08249968,597.09856312)(873.97250337,597.04856951)
\curveto(873.68250008,596.9185633)(873.44750031,596.73856348)(873.26750337,596.50856951)
\curveto(873.08750067,596.28856393)(872.93250083,596.03356418)(872.80250337,595.74356951)
\curveto(872.762501,595.63356458)(872.73250103,595.5185647)(872.71250337,595.39856951)
\curveto(872.69250107,595.28856493)(872.66750109,595.17356504)(872.63750337,595.05356951)
\curveto(872.62750113,595.00356521)(872.62250114,594.95356526)(872.62250337,594.90356951)
\curveto(872.63250113,594.85356536)(872.63250113,594.80356541)(872.62250337,594.75356951)
\curveto(872.59250117,594.63356558)(872.57750118,594.49356572)(872.57750337,594.33356951)
\curveto(872.58750117,594.18356603)(872.59250117,594.03856618)(872.59250337,593.89856951)
\lineto(872.59250337,592.05356951)
\lineto(872.59250337,591.70856951)
\curveto(872.59250117,591.58856863)(872.58750117,591.47356874)(872.57750337,591.36356951)
\curveto(872.56750119,591.25356896)(872.5625012,591.15856906)(872.56250337,591.07856951)
\curveto(872.57250119,590.99856922)(872.55250121,590.92856929)(872.50250337,590.86856951)
\curveto(872.45250131,590.79856942)(872.37250139,590.75856946)(872.26250337,590.74856951)
\curveto(872.1625016,590.73856948)(872.05250171,590.73356948)(871.93250337,590.73356951)
\lineto(871.66250337,590.73356951)
\curveto(871.61250215,590.75356946)(871.5625022,590.76856945)(871.51250337,590.77856951)
\curveto(871.47250229,590.79856942)(871.44250232,590.82356939)(871.42250337,590.85356951)
\curveto(871.37250239,590.92356929)(871.34250242,591.00856921)(871.33250337,591.10856951)
\lineto(871.33250337,591.43856951)
\lineto(871.33250337,592.59356951)
\lineto(871.33250337,596.74856951)
\lineto(871.33250337,597.78356951)
\lineto(871.33250337,598.08356951)
\curveto(871.34250242,598.18356203)(871.37250239,598.26856195)(871.42250337,598.33856951)
\curveto(871.45250231,598.37856184)(871.50250226,598.40856181)(871.57250337,598.42856951)
\curveto(871.65250211,598.44856177)(871.73750202,598.45856176)(871.82750337,598.45856951)
\curveto(871.91750184,598.46856175)(872.00750175,598.46856175)(872.09750337,598.45856951)
\curveto(872.18750157,598.44856177)(872.2575015,598.43356178)(872.30750337,598.41356951)
\curveto(872.38750137,598.38356183)(872.43750132,598.32356189)(872.45750337,598.23356951)
\curveto(872.48750127,598.15356206)(872.50250126,598.06356215)(872.50250337,597.96356951)
\lineto(872.50250337,597.66356951)
\curveto(872.50250126,597.56356265)(872.52250124,597.47356274)(872.56250337,597.39356951)
\curveto(872.57250119,597.37356284)(872.58250118,597.35856286)(872.59250337,597.34856951)
\lineto(872.63750337,597.30356951)
\curveto(872.74750101,597.30356291)(872.83750092,597.34856287)(872.90750337,597.43856951)
\curveto(872.97750078,597.53856268)(873.03750072,597.6185626)(873.08750337,597.67856951)
\lineto(873.17750337,597.76856951)
\curveto(873.26750049,597.87856234)(873.39250037,597.99356222)(873.55250337,598.11356951)
\curveto(873.71250005,598.23356198)(873.8624999,598.32356189)(874.00250337,598.38356951)
\curveto(874.09249967,598.43356178)(874.18749957,598.46856175)(874.28750337,598.48856951)
\curveto(874.38749937,598.5185617)(874.49249927,598.54856167)(874.60250337,598.57856951)
\curveto(874.6624991,598.58856163)(874.72249904,598.59356162)(874.78250337,598.59356951)
\curveto(874.84249892,598.60356161)(874.89749886,598.6135616)(874.94750337,598.62356951)
}
}
{
\newrgbcolor{curcolor}{0 0 0}
\pscustom[linestyle=none,fillstyle=solid,fillcolor=curcolor]
{
\newpath
\moveto(883.197269,591.27356951)
\curveto(883.22726117,591.1135691)(883.21226118,590.97856924)(883.152269,590.86856951)
\curveto(883.0922613,590.76856945)(883.01226138,590.69356952)(882.912269,590.64356951)
\curveto(882.86226153,590.62356959)(882.80726159,590.6135696)(882.747269,590.61356951)
\curveto(882.6972617,590.6135696)(882.64226175,590.60356961)(882.582269,590.58356951)
\curveto(882.36226203,590.53356968)(882.14226225,590.54856967)(881.922269,590.62856951)
\curveto(881.71226268,590.69856952)(881.56726283,590.78856943)(881.487269,590.89856951)
\curveto(881.43726296,590.96856925)(881.392263,591.04856917)(881.352269,591.13856951)
\curveto(881.31226308,591.23856898)(881.26226313,591.3185689)(881.202269,591.37856951)
\curveto(881.18226321,591.39856882)(881.15726324,591.4185688)(881.127269,591.43856951)
\curveto(881.10726329,591.45856876)(881.07726332,591.46356875)(881.037269,591.45356951)
\curveto(880.92726347,591.42356879)(880.82226357,591.36856885)(880.722269,591.28856951)
\curveto(880.63226376,591.20856901)(880.54226385,591.13856908)(880.452269,591.07856951)
\curveto(880.32226407,590.99856922)(880.18226421,590.92356929)(880.032269,590.85356951)
\curveto(879.88226451,590.79356942)(879.72226467,590.73856948)(879.552269,590.68856951)
\curveto(879.45226494,590.65856956)(879.34226505,590.63856958)(879.222269,590.62856951)
\curveto(879.11226528,590.6185696)(879.00226539,590.60356961)(878.892269,590.58356951)
\curveto(878.84226555,590.57356964)(878.7972656,590.56856965)(878.757269,590.56856951)
\lineto(878.652269,590.56856951)
\curveto(878.54226585,590.54856967)(878.43726596,590.54856967)(878.337269,590.56856951)
\lineto(878.202269,590.56856951)
\curveto(878.15226624,590.57856964)(878.10226629,590.58356963)(878.052269,590.58356951)
\curveto(878.00226639,590.58356963)(877.95726644,590.59356962)(877.917269,590.61356951)
\curveto(877.87726652,590.62356959)(877.84226655,590.62856959)(877.812269,590.62856951)
\curveto(877.7922666,590.6185696)(877.76726663,590.6185696)(877.737269,590.62856951)
\lineto(877.497269,590.68856951)
\curveto(877.41726698,590.69856952)(877.34226705,590.7185695)(877.272269,590.74856951)
\curveto(876.97226742,590.87856934)(876.72726767,591.02356919)(876.537269,591.18356951)
\curveto(876.35726804,591.35356886)(876.20726819,591.58856863)(876.087269,591.88856951)
\curveto(875.9972684,592.10856811)(875.95226844,592.37356784)(875.952269,592.68356951)
\lineto(875.952269,592.99856951)
\curveto(875.96226843,593.04856717)(875.96726843,593.09856712)(875.967269,593.14856951)
\lineto(875.997269,593.32856951)
\lineto(876.117269,593.65856951)
\curveto(876.15726824,593.76856645)(876.20726819,593.86856635)(876.267269,593.95856951)
\curveto(876.44726795,594.24856597)(876.6922677,594.46356575)(877.002269,594.60356951)
\curveto(877.31226708,594.74356547)(877.65226674,594.86856535)(878.022269,594.97856951)
\curveto(878.16226623,595.0185652)(878.30726609,595.04856517)(878.457269,595.06856951)
\curveto(878.60726579,595.08856513)(878.75726564,595.1135651)(878.907269,595.14356951)
\curveto(878.97726542,595.16356505)(879.04226535,595.17356504)(879.102269,595.17356951)
\curveto(879.17226522,595.17356504)(879.24726515,595.18356503)(879.327269,595.20356951)
\curveto(879.397265,595.22356499)(879.46726493,595.23356498)(879.537269,595.23356951)
\curveto(879.60726479,595.24356497)(879.68226471,595.25856496)(879.762269,595.27856951)
\curveto(880.01226438,595.33856488)(880.24726415,595.38856483)(880.467269,595.42856951)
\curveto(880.68726371,595.47856474)(880.86226353,595.59356462)(880.992269,595.77356951)
\curveto(881.05226334,595.85356436)(881.10226329,595.95356426)(881.142269,596.07356951)
\curveto(881.18226321,596.20356401)(881.18226321,596.34356387)(881.142269,596.49356951)
\curveto(881.08226331,596.73356348)(880.9922634,596.92356329)(880.872269,597.06356951)
\curveto(880.76226363,597.20356301)(880.60226379,597.3135629)(880.392269,597.39356951)
\curveto(880.27226412,597.44356277)(880.12726427,597.47856274)(879.957269,597.49856951)
\curveto(879.7972646,597.5185627)(879.62726477,597.52856269)(879.447269,597.52856951)
\curveto(879.26726513,597.52856269)(879.0922653,597.5185627)(878.922269,597.49856951)
\curveto(878.75226564,597.47856274)(878.60726579,597.44856277)(878.487269,597.40856951)
\curveto(878.31726608,597.34856287)(878.15226624,597.26356295)(877.992269,597.15356951)
\curveto(877.91226648,597.09356312)(877.83726656,597.0135632)(877.767269,596.91356951)
\curveto(877.70726669,596.82356339)(877.65226674,596.72356349)(877.602269,596.61356951)
\curveto(877.57226682,596.53356368)(877.54226685,596.44856377)(877.512269,596.35856951)
\curveto(877.4922669,596.26856395)(877.44726695,596.19856402)(877.377269,596.14856951)
\curveto(877.33726706,596.1185641)(877.26726713,596.09356412)(877.167269,596.07356951)
\curveto(877.07726732,596.06356415)(876.98226741,596.05856416)(876.882269,596.05856951)
\curveto(876.78226761,596.05856416)(876.68226771,596.06356415)(876.582269,596.07356951)
\curveto(876.4922679,596.09356412)(876.42726797,596.1185641)(876.387269,596.14856951)
\curveto(876.34726805,596.17856404)(876.31726808,596.22856399)(876.297269,596.29856951)
\curveto(876.27726812,596.36856385)(876.27726812,596.44356377)(876.297269,596.52356951)
\curveto(876.32726807,596.65356356)(876.35726804,596.77356344)(876.387269,596.88356951)
\curveto(876.42726797,597.00356321)(876.47226792,597.1185631)(876.522269,597.22856951)
\curveto(876.71226768,597.57856264)(876.95226744,597.84856237)(877.242269,598.03856951)
\curveto(877.53226686,598.23856198)(877.8922665,598.39856182)(878.322269,598.51856951)
\curveto(878.42226597,598.53856168)(878.52226587,598.55356166)(878.622269,598.56356951)
\curveto(878.73226566,598.57356164)(878.84226555,598.58856163)(878.952269,598.60856951)
\curveto(878.9922654,598.6185616)(879.05726534,598.6185616)(879.147269,598.60856951)
\curveto(879.23726516,598.60856161)(879.2922651,598.6185616)(879.312269,598.63856951)
\curveto(880.01226438,598.64856157)(880.62226377,598.56856165)(881.142269,598.39856951)
\curveto(881.66226273,598.22856199)(882.02726237,597.90356231)(882.237269,597.42356951)
\curveto(882.32726207,597.22356299)(882.37726202,596.98856323)(882.387269,596.71856951)
\curveto(882.40726199,596.45856376)(882.41726198,596.18356403)(882.417269,595.89356951)
\lineto(882.417269,592.57856951)
\curveto(882.41726198,592.43856778)(882.42226197,592.30356791)(882.432269,592.17356951)
\curveto(882.44226195,592.04356817)(882.47226192,591.93856828)(882.522269,591.85856951)
\curveto(882.57226182,591.78856843)(882.63726176,591.73856848)(882.717269,591.70856951)
\curveto(882.80726159,591.66856855)(882.8922615,591.63856858)(882.972269,591.61856951)
\curveto(883.05226134,591.60856861)(883.11226128,591.56356865)(883.152269,591.48356951)
\curveto(883.17226122,591.45356876)(883.18226121,591.42356879)(883.182269,591.39356951)
\curveto(883.18226121,591.36356885)(883.18726121,591.32356889)(883.197269,591.27356951)
\moveto(881.052269,592.93856951)
\curveto(881.11226328,593.07856714)(881.14226325,593.23856698)(881.142269,593.41856951)
\curveto(881.15226324,593.60856661)(881.15726324,593.80356641)(881.157269,594.00356951)
\curveto(881.15726324,594.1135661)(881.15226324,594.213566)(881.142269,594.30356951)
\curveto(881.13226326,594.39356582)(881.0922633,594.46356575)(881.022269,594.51356951)
\curveto(880.9922634,594.53356568)(880.92226347,594.54356567)(880.812269,594.54356951)
\curveto(880.7922636,594.52356569)(880.75726364,594.5135657)(880.707269,594.51356951)
\curveto(880.65726374,594.5135657)(880.61226378,594.50356571)(880.572269,594.48356951)
\curveto(880.4922639,594.46356575)(880.40226399,594.44356577)(880.302269,594.42356951)
\lineto(880.002269,594.36356951)
\curveto(879.97226442,594.36356585)(879.93726446,594.35856586)(879.897269,594.34856951)
\lineto(879.792269,594.34856951)
\curveto(879.64226475,594.30856591)(879.47726492,594.28356593)(879.297269,594.27356951)
\curveto(879.12726527,594.27356594)(878.96726543,594.25356596)(878.817269,594.21356951)
\curveto(878.73726566,594.19356602)(878.66226573,594.17356604)(878.592269,594.15356951)
\curveto(878.53226586,594.14356607)(878.46226593,594.12856609)(878.382269,594.10856951)
\curveto(878.22226617,594.05856616)(878.07226632,593.99356622)(877.932269,593.91356951)
\curveto(877.7922666,593.84356637)(877.67226672,593.75356646)(877.572269,593.64356951)
\curveto(877.47226692,593.53356668)(877.397267,593.39856682)(877.347269,593.23856951)
\curveto(877.2972671,593.08856713)(877.27726712,592.90356731)(877.287269,592.68356951)
\curveto(877.28726711,592.58356763)(877.30226709,592.48856773)(877.332269,592.39856951)
\curveto(877.37226702,592.3185679)(877.41726698,592.24356797)(877.467269,592.17356951)
\curveto(877.54726685,592.06356815)(877.65226674,591.96856825)(877.782269,591.88856951)
\curveto(877.91226648,591.8185684)(878.05226634,591.75856846)(878.202269,591.70856951)
\curveto(878.25226614,591.69856852)(878.30226609,591.69356852)(878.352269,591.69356951)
\curveto(878.40226599,591.69356852)(878.45226594,591.68856853)(878.502269,591.67856951)
\curveto(878.57226582,591.65856856)(878.65726574,591.64356857)(878.757269,591.63356951)
\curveto(878.86726553,591.63356858)(878.95726544,591.64356857)(879.027269,591.66356951)
\curveto(879.08726531,591.68356853)(879.14726525,591.68856853)(879.207269,591.67856951)
\curveto(879.26726513,591.67856854)(879.32726507,591.68856853)(879.387269,591.70856951)
\curveto(879.46726493,591.72856849)(879.54226485,591.74356847)(879.612269,591.75356951)
\curveto(879.6922647,591.76356845)(879.76726463,591.78356843)(879.837269,591.81356951)
\curveto(880.12726427,591.93356828)(880.37226402,592.07856814)(880.572269,592.24856951)
\curveto(880.78226361,592.4185678)(880.94226345,592.64856757)(881.052269,592.93856951)
}
}
{
\newrgbcolor{curcolor}{0 0 0}
\pscustom[linestyle=none,fillstyle=solid,fillcolor=curcolor]
{
\newpath
\moveto(886.81390962,601.51856951)
\curveto(886.99390608,601.52855869)(887.18390589,601.52855869)(887.38390962,601.51856951)
\curveto(887.58390549,601.50855871)(887.72390535,601.44855877)(887.80390962,601.33856951)
\curveto(887.84390523,601.27855894)(887.86890521,601.20355901)(887.87890962,601.11356951)
\curveto(887.88890519,601.03355918)(887.89390518,600.94355927)(887.89390962,600.84356951)
\curveto(887.89390518,600.7135595)(887.86890521,600.60855961)(887.81890962,600.52856951)
\curveto(887.7789053,600.47855974)(887.71890536,600.44355977)(887.63890962,600.42356951)
\curveto(887.56890551,600.4135598)(887.48890559,600.40855981)(887.39890962,600.40856951)
\lineto(887.11390962,600.40856951)
\curveto(887.02390605,600.4185598)(886.94390613,600.4185598)(886.87390962,600.40856951)
\curveto(886.59390648,600.32855989)(886.40890667,600.19856002)(886.31890962,600.01856951)
\curveto(886.23890684,599.84856037)(886.19890688,599.58856063)(886.19890962,599.23856951)
\curveto(886.19890688,599.16856105)(886.19390688,599.09356112)(886.18390962,599.01356951)
\curveto(886.1739069,598.94356127)(886.1789069,598.87856134)(886.19890962,598.81856951)
\curveto(886.22890685,598.66856155)(886.29390678,598.56356165)(886.39390962,598.50356951)
\curveto(886.4739066,598.47356174)(886.5739065,598.45856176)(886.69390962,598.45856951)
\lineto(887.05390962,598.45856951)
\lineto(887.27890962,598.45856951)
\curveto(887.30890577,598.43856178)(887.33890574,598.43356178)(887.36890962,598.44356951)
\curveto(887.39890568,598.45356176)(887.42890565,598.44856177)(887.45890962,598.42856951)
\curveto(887.55890552,598.39856182)(887.62390545,598.33856188)(887.65390962,598.24856951)
\curveto(887.68390539,598.16856205)(887.69890538,598.06356215)(887.69890962,597.93356951)
\curveto(887.68890539,597.89356232)(887.68390539,597.85356236)(887.68390962,597.81356951)
\lineto(887.68390962,597.69356951)
\curveto(887.65390542,597.54356267)(887.58890549,597.44356277)(887.48890962,597.39356951)
\curveto(887.35890572,597.34356287)(887.18890589,597.32856289)(886.97890962,597.34856951)
\curveto(886.7789063,597.37856284)(886.60890647,597.37356284)(886.46890962,597.33356951)
\curveto(886.38890669,597.3135629)(886.32890675,597.27356294)(886.28890962,597.21356951)
\curveto(886.24890683,597.16356305)(886.21890686,597.09356312)(886.19890962,597.00356951)
\curveto(886.1789069,596.93356328)(886.1739069,596.85356336)(886.18390962,596.76356951)
\curveto(886.19390688,596.67356354)(886.19890688,596.58856363)(886.19890962,596.50856951)
\lineto(886.19890962,595.51856951)
\lineto(886.19890962,592.33856951)
\lineto(886.19890962,591.58856951)
\lineto(886.19890962,591.39356951)
\curveto(886.20890687,591.32356889)(886.20390687,591.26356895)(886.18390962,591.21356951)
\lineto(886.18390962,591.09356951)
\lineto(886.15390962,590.97356951)
\curveto(886.14390693,590.93356928)(886.12890695,590.89856932)(886.10890962,590.86856951)
\curveto(886.05890702,590.79856942)(885.98390709,590.75856946)(885.88390962,590.74856951)
\curveto(885.78390729,590.73856948)(885.6739074,590.73356948)(885.55390962,590.73356951)
\lineto(885.26890962,590.73356951)
\curveto(885.21890786,590.75356946)(885.16890791,590.76856945)(885.11890962,590.77856951)
\curveto(885.078908,590.79856942)(885.04390803,590.83356938)(885.01390962,590.88356951)
\curveto(884.99390808,590.9135693)(884.9739081,590.97856924)(884.95390962,591.07856951)
\lineto(884.95390962,591.18356951)
\curveto(884.93390814,591.23356898)(884.92390815,591.28356893)(884.92390962,591.33356951)
\curveto(884.93390814,591.39356882)(884.93890814,591.44856877)(884.93890962,591.49856951)
\lineto(884.93890962,592.09856951)
\lineto(884.93890962,596.19356951)
\lineto(884.93890962,596.53856951)
\curveto(884.94890813,596.65856356)(884.94890813,596.76856345)(884.93890962,596.86856951)
\curveto(884.93890814,596.97856324)(884.91890816,597.07356314)(884.87890962,597.15356951)
\curveto(884.84890823,597.23356298)(884.79390828,597.28856293)(884.71390962,597.31856951)
\curveto(884.65390842,597.34856287)(884.58390849,597.36356285)(884.50390962,597.36356951)
\lineto(884.27890962,597.36356951)
\lineto(884.03890962,597.36356951)
\curveto(883.96890911,597.36356285)(883.90390917,597.37356284)(883.84390962,597.39356951)
\curveto(883.75390932,597.43356278)(883.68890939,597.5185627)(883.64890962,597.64856951)
\curveto(883.63890944,597.69856252)(883.63390944,597.74356247)(883.63390962,597.78356951)
\lineto(883.63390962,597.91856951)
\curveto(883.63390944,598.05856216)(883.64890943,598.16856205)(883.67890962,598.24856951)
\curveto(883.70890937,598.33856188)(883.7739093,598.39856182)(883.87390962,598.42856951)
\curveto(883.94390913,598.45856176)(884.02390905,598.46856175)(884.11390962,598.45856951)
\lineto(884.39890962,598.45856951)
\curveto(884.49890858,598.45856176)(884.58390849,598.46856175)(884.65390962,598.48856951)
\curveto(884.73390834,598.50856171)(884.79890828,598.54856167)(884.84890962,598.60856951)
\curveto(884.91890816,598.68856153)(884.94890813,598.8135614)(884.93890962,598.98356951)
\lineto(884.93890962,599.46356951)
\curveto(884.93890814,599.66356055)(884.94890813,599.84856037)(884.96890962,600.01856951)
\curveto(884.99890808,600.19856002)(885.04390803,600.35855986)(885.10390962,600.49856951)
\curveto(885.21390786,600.73855948)(885.35890772,600.93355928)(885.53890962,601.08356951)
\curveto(885.72890735,601.23355898)(885.95390712,601.34855887)(886.21390962,601.42856951)
\curveto(886.2739068,601.44855877)(886.33390674,601.45855876)(886.39390962,601.45856951)
\curveto(886.46390661,601.46855875)(886.53390654,601.48355873)(886.60390962,601.50356951)
\curveto(886.62390645,601.5135587)(886.65890642,601.5135587)(886.70890962,601.50356951)
\curveto(886.75890632,601.50355871)(886.79390628,601.50855871)(886.81390962,601.51856951)
\moveto(889.07890962,599.94356951)
\curveto(889.14890393,599.89356032)(889.23390384,599.86856035)(889.33390962,599.86856951)
\lineto(889.64890962,599.86856951)
\lineto(889.81390962,599.86856951)
\curveto(889.8739032,599.86856035)(889.92890315,599.87856034)(889.97890962,599.89856951)
\curveto(890.10890297,599.94856027)(890.1739029,600.05356016)(890.17390962,600.21356951)
\curveto(890.18390289,600.37355984)(890.18890289,600.54355967)(890.18890962,600.72356951)
\lineto(890.18890962,600.97856951)
\curveto(890.18890289,601.06855915)(890.1739029,601.14355907)(890.14390962,601.20356951)
\curveto(890.09390298,601.3135589)(889.99390308,601.37355884)(889.84390962,601.38356951)
\curveto(889.69390338,601.39355882)(889.53390354,601.39855882)(889.36390962,601.39856951)
\curveto(889.34390373,601.38855883)(889.31890376,601.38355883)(889.28890962,601.38356951)
\curveto(889.26890381,601.39355882)(889.24890383,601.39355882)(889.22890962,601.38356951)
\curveto(889.10890397,601.34355887)(889.02890405,601.28355893)(888.98890962,601.20356951)
\curveto(888.95890412,601.14355907)(888.94390413,601.06855915)(888.94390962,600.97856951)
\lineto(888.94390962,600.72356951)
\lineto(888.94390962,600.25856951)
\curveto(888.95390412,600.10856011)(888.99890408,600.00356021)(889.07890962,599.94356951)
\moveto(890.18890962,597.78356951)
\lineto(890.18890962,598.06856951)
\curveto(890.18890289,598.16856205)(890.16390291,598.24856197)(890.11390962,598.30856951)
\curveto(890.06390301,598.38856183)(889.96890311,598.42856179)(889.82890962,598.42856951)
\curveto(889.69890338,598.43856178)(889.56890351,598.44356177)(889.43890962,598.44356951)
\curveto(889.41890366,598.43356178)(889.39390368,598.42856179)(889.36390962,598.42856951)
\curveto(889.34390373,598.43856178)(889.32390375,598.44356177)(889.30390962,598.44356951)
\curveto(889.24390383,598.42356179)(889.18890389,598.40856181)(889.13890962,598.39856951)
\curveto(889.08890399,598.38856183)(889.04890403,598.35856186)(889.01890962,598.30856951)
\curveto(888.96890411,598.24856197)(888.94390413,598.16356205)(888.94390962,598.05356951)
\lineto(888.94390962,597.73856951)
\lineto(888.94390962,591.39356951)
\lineto(888.94390962,591.10856951)
\curveto(888.94390413,591.0185692)(888.96390411,590.94356927)(889.00390962,590.88356951)
\curveto(889.05390402,590.80356941)(889.12390395,590.75356946)(889.21390962,590.73356951)
\curveto(889.31390376,590.72356949)(889.42890365,590.7185695)(889.55890962,590.71856951)
\lineto(889.78390962,590.71856951)
\curveto(889.86390321,590.73856948)(889.93390314,590.75356946)(889.99390962,590.76356951)
\curveto(890.05390302,590.78356943)(890.09890298,590.82356939)(890.12890962,590.88356951)
\curveto(890.1789029,590.94356927)(890.19890288,591.02356919)(890.18890962,591.12356951)
\lineto(890.18890962,591.43856951)
\lineto(890.18890962,597.78356951)
}
}
{
\newrgbcolor{curcolor}{0 0 0}
\pscustom[linestyle=none,fillstyle=solid,fillcolor=curcolor]
{
\newpath
\moveto(899.0175815,591.27356951)
\curveto(899.04757367,591.1135691)(899.03257368,590.97856924)(898.9725815,590.86856951)
\curveto(898.9125738,590.76856945)(898.83257388,590.69356952)(898.7325815,590.64356951)
\curveto(898.68257403,590.62356959)(898.62757409,590.6135696)(898.5675815,590.61356951)
\curveto(898.5175742,590.6135696)(898.46257425,590.60356961)(898.4025815,590.58356951)
\curveto(898.18257453,590.53356968)(897.96257475,590.54856967)(897.7425815,590.62856951)
\curveto(897.53257518,590.69856952)(897.38757533,590.78856943)(897.3075815,590.89856951)
\curveto(897.25757546,590.96856925)(897.2125755,591.04856917)(897.1725815,591.13856951)
\curveto(897.13257558,591.23856898)(897.08257563,591.3185689)(897.0225815,591.37856951)
\curveto(897.00257571,591.39856882)(896.97757574,591.4185688)(896.9475815,591.43856951)
\curveto(896.92757579,591.45856876)(896.89757582,591.46356875)(896.8575815,591.45356951)
\curveto(896.74757597,591.42356879)(896.64257607,591.36856885)(896.5425815,591.28856951)
\curveto(896.45257626,591.20856901)(896.36257635,591.13856908)(896.2725815,591.07856951)
\curveto(896.14257657,590.99856922)(896.00257671,590.92356929)(895.8525815,590.85356951)
\curveto(895.70257701,590.79356942)(895.54257717,590.73856948)(895.3725815,590.68856951)
\curveto(895.27257744,590.65856956)(895.16257755,590.63856958)(895.0425815,590.62856951)
\curveto(894.93257778,590.6185696)(894.82257789,590.60356961)(894.7125815,590.58356951)
\curveto(894.66257805,590.57356964)(894.6175781,590.56856965)(894.5775815,590.56856951)
\lineto(894.4725815,590.56856951)
\curveto(894.36257835,590.54856967)(894.25757846,590.54856967)(894.1575815,590.56856951)
\lineto(894.0225815,590.56856951)
\curveto(893.97257874,590.57856964)(893.92257879,590.58356963)(893.8725815,590.58356951)
\curveto(893.82257889,590.58356963)(893.77757894,590.59356962)(893.7375815,590.61356951)
\curveto(893.69757902,590.62356959)(893.66257905,590.62856959)(893.6325815,590.62856951)
\curveto(893.6125791,590.6185696)(893.58757913,590.6185696)(893.5575815,590.62856951)
\lineto(893.3175815,590.68856951)
\curveto(893.23757948,590.69856952)(893.16257955,590.7185695)(893.0925815,590.74856951)
\curveto(892.79257992,590.87856934)(892.54758017,591.02356919)(892.3575815,591.18356951)
\curveto(892.17758054,591.35356886)(892.02758069,591.58856863)(891.9075815,591.88856951)
\curveto(891.8175809,592.10856811)(891.77258094,592.37356784)(891.7725815,592.68356951)
\lineto(891.7725815,592.99856951)
\curveto(891.78258093,593.04856717)(891.78758093,593.09856712)(891.7875815,593.14856951)
\lineto(891.8175815,593.32856951)
\lineto(891.9375815,593.65856951)
\curveto(891.97758074,593.76856645)(892.02758069,593.86856635)(892.0875815,593.95856951)
\curveto(892.26758045,594.24856597)(892.5125802,594.46356575)(892.8225815,594.60356951)
\curveto(893.13257958,594.74356547)(893.47257924,594.86856535)(893.8425815,594.97856951)
\curveto(893.98257873,595.0185652)(894.12757859,595.04856517)(894.2775815,595.06856951)
\curveto(894.42757829,595.08856513)(894.57757814,595.1135651)(894.7275815,595.14356951)
\curveto(894.79757792,595.16356505)(894.86257785,595.17356504)(894.9225815,595.17356951)
\curveto(894.99257772,595.17356504)(895.06757765,595.18356503)(895.1475815,595.20356951)
\curveto(895.2175775,595.22356499)(895.28757743,595.23356498)(895.3575815,595.23356951)
\curveto(895.42757729,595.24356497)(895.50257721,595.25856496)(895.5825815,595.27856951)
\curveto(895.83257688,595.33856488)(896.06757665,595.38856483)(896.2875815,595.42856951)
\curveto(896.50757621,595.47856474)(896.68257603,595.59356462)(896.8125815,595.77356951)
\curveto(896.87257584,595.85356436)(896.92257579,595.95356426)(896.9625815,596.07356951)
\curveto(897.00257571,596.20356401)(897.00257571,596.34356387)(896.9625815,596.49356951)
\curveto(896.90257581,596.73356348)(896.8125759,596.92356329)(896.6925815,597.06356951)
\curveto(896.58257613,597.20356301)(896.42257629,597.3135629)(896.2125815,597.39356951)
\curveto(896.09257662,597.44356277)(895.94757677,597.47856274)(895.7775815,597.49856951)
\curveto(895.6175771,597.5185627)(895.44757727,597.52856269)(895.2675815,597.52856951)
\curveto(895.08757763,597.52856269)(894.9125778,597.5185627)(894.7425815,597.49856951)
\curveto(894.57257814,597.47856274)(894.42757829,597.44856277)(894.3075815,597.40856951)
\curveto(894.13757858,597.34856287)(893.97257874,597.26356295)(893.8125815,597.15356951)
\curveto(893.73257898,597.09356312)(893.65757906,597.0135632)(893.5875815,596.91356951)
\curveto(893.52757919,596.82356339)(893.47257924,596.72356349)(893.4225815,596.61356951)
\curveto(893.39257932,596.53356368)(893.36257935,596.44856377)(893.3325815,596.35856951)
\curveto(893.3125794,596.26856395)(893.26757945,596.19856402)(893.1975815,596.14856951)
\curveto(893.15757956,596.1185641)(893.08757963,596.09356412)(892.9875815,596.07356951)
\curveto(892.89757982,596.06356415)(892.80257991,596.05856416)(892.7025815,596.05856951)
\curveto(892.60258011,596.05856416)(892.50258021,596.06356415)(892.4025815,596.07356951)
\curveto(892.3125804,596.09356412)(892.24758047,596.1185641)(892.2075815,596.14856951)
\curveto(892.16758055,596.17856404)(892.13758058,596.22856399)(892.1175815,596.29856951)
\curveto(892.09758062,596.36856385)(892.09758062,596.44356377)(892.1175815,596.52356951)
\curveto(892.14758057,596.65356356)(892.17758054,596.77356344)(892.2075815,596.88356951)
\curveto(892.24758047,597.00356321)(892.29258042,597.1185631)(892.3425815,597.22856951)
\curveto(892.53258018,597.57856264)(892.77257994,597.84856237)(893.0625815,598.03856951)
\curveto(893.35257936,598.23856198)(893.712579,598.39856182)(894.1425815,598.51856951)
\curveto(894.24257847,598.53856168)(894.34257837,598.55356166)(894.4425815,598.56356951)
\curveto(894.55257816,598.57356164)(894.66257805,598.58856163)(894.7725815,598.60856951)
\curveto(894.8125779,598.6185616)(894.87757784,598.6185616)(894.9675815,598.60856951)
\curveto(895.05757766,598.60856161)(895.1125776,598.6185616)(895.1325815,598.63856951)
\curveto(895.83257688,598.64856157)(896.44257627,598.56856165)(896.9625815,598.39856951)
\curveto(897.48257523,598.22856199)(897.84757487,597.90356231)(898.0575815,597.42356951)
\curveto(898.14757457,597.22356299)(898.19757452,596.98856323)(898.2075815,596.71856951)
\curveto(898.22757449,596.45856376)(898.23757448,596.18356403)(898.2375815,595.89356951)
\lineto(898.2375815,592.57856951)
\curveto(898.23757448,592.43856778)(898.24257447,592.30356791)(898.2525815,592.17356951)
\curveto(898.26257445,592.04356817)(898.29257442,591.93856828)(898.3425815,591.85856951)
\curveto(898.39257432,591.78856843)(898.45757426,591.73856848)(898.5375815,591.70856951)
\curveto(898.62757409,591.66856855)(898.712574,591.63856858)(898.7925815,591.61856951)
\curveto(898.87257384,591.60856861)(898.93257378,591.56356865)(898.9725815,591.48356951)
\curveto(898.99257372,591.45356876)(899.00257371,591.42356879)(899.0025815,591.39356951)
\curveto(899.00257371,591.36356885)(899.00757371,591.32356889)(899.0175815,591.27356951)
\moveto(896.8725815,592.93856951)
\curveto(896.93257578,593.07856714)(896.96257575,593.23856698)(896.9625815,593.41856951)
\curveto(896.97257574,593.60856661)(896.97757574,593.80356641)(896.9775815,594.00356951)
\curveto(896.97757574,594.1135661)(896.97257574,594.213566)(896.9625815,594.30356951)
\curveto(896.95257576,594.39356582)(896.9125758,594.46356575)(896.8425815,594.51356951)
\curveto(896.8125759,594.53356568)(896.74257597,594.54356567)(896.6325815,594.54356951)
\curveto(896.6125761,594.52356569)(896.57757614,594.5135657)(896.5275815,594.51356951)
\curveto(896.47757624,594.5135657)(896.43257628,594.50356571)(896.3925815,594.48356951)
\curveto(896.3125764,594.46356575)(896.22257649,594.44356577)(896.1225815,594.42356951)
\lineto(895.8225815,594.36356951)
\curveto(895.79257692,594.36356585)(895.75757696,594.35856586)(895.7175815,594.34856951)
\lineto(895.6125815,594.34856951)
\curveto(895.46257725,594.30856591)(895.29757742,594.28356593)(895.1175815,594.27356951)
\curveto(894.94757777,594.27356594)(894.78757793,594.25356596)(894.6375815,594.21356951)
\curveto(894.55757816,594.19356602)(894.48257823,594.17356604)(894.4125815,594.15356951)
\curveto(894.35257836,594.14356607)(894.28257843,594.12856609)(894.2025815,594.10856951)
\curveto(894.04257867,594.05856616)(893.89257882,593.99356622)(893.7525815,593.91356951)
\curveto(893.6125791,593.84356637)(893.49257922,593.75356646)(893.3925815,593.64356951)
\curveto(893.29257942,593.53356668)(893.2175795,593.39856682)(893.1675815,593.23856951)
\curveto(893.1175796,593.08856713)(893.09757962,592.90356731)(893.1075815,592.68356951)
\curveto(893.10757961,592.58356763)(893.12257959,592.48856773)(893.1525815,592.39856951)
\curveto(893.19257952,592.3185679)(893.23757948,592.24356797)(893.2875815,592.17356951)
\curveto(893.36757935,592.06356815)(893.47257924,591.96856825)(893.6025815,591.88856951)
\curveto(893.73257898,591.8185684)(893.87257884,591.75856846)(894.0225815,591.70856951)
\curveto(894.07257864,591.69856852)(894.12257859,591.69356852)(894.1725815,591.69356951)
\curveto(894.22257849,591.69356852)(894.27257844,591.68856853)(894.3225815,591.67856951)
\curveto(894.39257832,591.65856856)(894.47757824,591.64356857)(894.5775815,591.63356951)
\curveto(894.68757803,591.63356858)(894.77757794,591.64356857)(894.8475815,591.66356951)
\curveto(894.90757781,591.68356853)(894.96757775,591.68856853)(895.0275815,591.67856951)
\curveto(895.08757763,591.67856854)(895.14757757,591.68856853)(895.2075815,591.70856951)
\curveto(895.28757743,591.72856849)(895.36257735,591.74356847)(895.4325815,591.75356951)
\curveto(895.5125772,591.76356845)(895.58757713,591.78356843)(895.6575815,591.81356951)
\curveto(895.94757677,591.93356828)(896.19257652,592.07856814)(896.3925815,592.24856951)
\curveto(896.60257611,592.4185678)(896.76257595,592.64856757)(896.8725815,592.93856951)
}
}
{
\newrgbcolor{curcolor}{0 0 0}
\pscustom[linestyle=none,fillstyle=solid,fillcolor=curcolor]
{
\newpath
\moveto(902.61922212,598.62356951)
\curveto(903.33921806,598.63356158)(903.94421745,598.54856167)(904.43422212,598.36856951)
\curveto(904.92421647,598.19856202)(905.30421609,597.89356232)(905.57422212,597.45356951)
\curveto(905.64421575,597.34356287)(905.6992157,597.22856299)(905.73922212,597.10856951)
\curveto(905.77921562,596.99856322)(905.81921558,596.87356334)(905.85922212,596.73356951)
\curveto(905.87921552,596.66356355)(905.88421551,596.58856363)(905.87422212,596.50856951)
\curveto(905.86421553,596.43856378)(905.84921555,596.38356383)(905.82922212,596.34356951)
\curveto(905.80921559,596.32356389)(905.78421561,596.30356391)(905.75422212,596.28356951)
\curveto(905.72421567,596.27356394)(905.6992157,596.25856396)(905.67922212,596.23856951)
\curveto(905.62921577,596.218564)(905.57921582,596.213564)(905.52922212,596.22356951)
\curveto(905.47921592,596.23356398)(905.42921597,596.23356398)(905.37922212,596.22356951)
\curveto(905.2992161,596.20356401)(905.1942162,596.19856402)(905.06422212,596.20856951)
\curveto(904.93421646,596.22856399)(904.84421655,596.25356396)(904.79422212,596.28356951)
\curveto(904.71421668,596.33356388)(904.65921674,596.39856382)(904.62922212,596.47856951)
\curveto(904.60921679,596.56856365)(904.57421682,596.65356356)(904.52422212,596.73356951)
\curveto(904.43421696,596.89356332)(904.30921709,597.03856318)(904.14922212,597.16856951)
\curveto(904.03921736,597.24856297)(903.91921748,597.30856291)(903.78922212,597.34856951)
\curveto(903.65921774,597.38856283)(903.51921788,597.42856279)(903.36922212,597.46856951)
\curveto(903.31921808,597.48856273)(903.26921813,597.49356272)(903.21922212,597.48356951)
\curveto(903.16921823,597.48356273)(903.11921828,597.48856273)(903.06922212,597.49856951)
\curveto(903.00921839,597.5185627)(902.93421846,597.52856269)(902.84422212,597.52856951)
\curveto(902.75421864,597.52856269)(902.67921872,597.5185627)(902.61922212,597.49856951)
\lineto(902.52922212,597.49856951)
\lineto(902.37922212,597.46856951)
\curveto(902.32921907,597.46856275)(902.27921912,597.46356275)(902.22922212,597.45356951)
\curveto(901.96921943,597.39356282)(901.75421964,597.30856291)(901.58422212,597.19856951)
\curveto(901.41421998,597.08856313)(901.2992201,596.90356331)(901.23922212,596.64356951)
\curveto(901.21922018,596.57356364)(901.21422018,596.50356371)(901.22422212,596.43356951)
\curveto(901.24422015,596.36356385)(901.26422013,596.30356391)(901.28422212,596.25356951)
\curveto(901.34422005,596.10356411)(901.41421998,595.99356422)(901.49422212,595.92356951)
\curveto(901.58421981,595.86356435)(901.6942197,595.79356442)(901.82422212,595.71356951)
\curveto(901.98421941,595.6135646)(902.16421923,595.53856468)(902.36422212,595.48856951)
\curveto(902.56421883,595.44856477)(902.76421863,595.39856482)(902.96422212,595.33856951)
\curveto(903.0942183,595.29856492)(903.22421817,595.26856495)(903.35422212,595.24856951)
\curveto(903.48421791,595.22856499)(903.61421778,595.19856502)(903.74422212,595.15856951)
\curveto(903.95421744,595.09856512)(904.15921724,595.03856518)(904.35922212,594.97856951)
\curveto(904.55921684,594.92856529)(904.75921664,594.86356535)(904.95922212,594.78356951)
\lineto(905.10922212,594.72356951)
\curveto(905.15921624,594.70356551)(905.20921619,594.67856554)(905.25922212,594.64856951)
\curveto(905.45921594,594.52856569)(905.63421576,594.39356582)(905.78422212,594.24356951)
\curveto(905.93421546,594.09356612)(906.05921534,593.90356631)(906.15922212,593.67356951)
\curveto(906.17921522,593.60356661)(906.1992152,593.50856671)(906.21922212,593.38856951)
\curveto(906.23921516,593.3185669)(906.24921515,593.24356697)(906.24922212,593.16356951)
\curveto(906.25921514,593.09356712)(906.26421513,593.0135672)(906.26422212,592.92356951)
\lineto(906.26422212,592.77356951)
\curveto(906.24421515,592.70356751)(906.23421516,592.63356758)(906.23422212,592.56356951)
\curveto(906.23421516,592.49356772)(906.22421517,592.42356779)(906.20422212,592.35356951)
\curveto(906.17421522,592.24356797)(906.13921526,592.13856808)(906.09922212,592.03856951)
\curveto(906.05921534,591.93856828)(906.01421538,591.84856837)(905.96422212,591.76856951)
\curveto(905.80421559,591.50856871)(905.5992158,591.29856892)(905.34922212,591.13856951)
\curveto(905.0992163,590.98856923)(904.81921658,590.85856936)(904.50922212,590.74856951)
\curveto(904.41921698,590.7185695)(904.32421707,590.69856952)(904.22422212,590.68856951)
\curveto(904.13421726,590.66856955)(904.04421735,590.64356957)(903.95422212,590.61356951)
\curveto(903.85421754,590.59356962)(903.75421764,590.58356963)(903.65422212,590.58356951)
\curveto(903.55421784,590.58356963)(903.45421794,590.57356964)(903.35422212,590.55356951)
\lineto(903.20422212,590.55356951)
\curveto(903.15421824,590.54356967)(903.08421831,590.53856968)(902.99422212,590.53856951)
\curveto(902.90421849,590.53856968)(902.83421856,590.54356967)(902.78422212,590.55356951)
\lineto(902.61922212,590.55356951)
\curveto(902.55921884,590.57356964)(902.4942189,590.58356963)(902.42422212,590.58356951)
\curveto(902.35421904,590.57356964)(902.2942191,590.57856964)(902.24422212,590.59856951)
\curveto(902.1942192,590.60856961)(902.12921927,590.6135696)(902.04922212,590.61356951)
\lineto(901.80922212,590.67356951)
\curveto(901.73921966,590.68356953)(901.66421973,590.70356951)(901.58422212,590.73356951)
\curveto(901.27422012,590.83356938)(901.00422039,590.95856926)(900.77422212,591.10856951)
\curveto(900.54422085,591.25856896)(900.34422105,591.45356876)(900.17422212,591.69356951)
\curveto(900.08422131,591.82356839)(900.00922139,591.95856826)(899.94922212,592.09856951)
\curveto(899.88922151,592.23856798)(899.83422156,592.39356782)(899.78422212,592.56356951)
\curveto(899.76422163,592.62356759)(899.75422164,592.69356752)(899.75422212,592.77356951)
\curveto(899.76422163,592.86356735)(899.77922162,592.93356728)(899.79922212,592.98356951)
\curveto(899.82922157,593.02356719)(899.87922152,593.06356715)(899.94922212,593.10356951)
\curveto(899.9992214,593.12356709)(900.06922133,593.13356708)(900.15922212,593.13356951)
\curveto(900.24922115,593.14356707)(900.33922106,593.14356707)(900.42922212,593.13356951)
\curveto(900.51922088,593.12356709)(900.60422079,593.10856711)(900.68422212,593.08856951)
\curveto(900.77422062,593.07856714)(900.83422056,593.06356715)(900.86422212,593.04356951)
\curveto(900.93422046,592.99356722)(900.97922042,592.9185673)(900.99922212,592.81856951)
\curveto(901.02922037,592.72856749)(901.06422033,592.64356757)(901.10422212,592.56356951)
\curveto(901.20422019,592.34356787)(901.33922006,592.17356804)(901.50922212,592.05356951)
\curveto(901.62921977,591.96356825)(901.76421963,591.89356832)(901.91422212,591.84356951)
\curveto(902.06421933,591.79356842)(902.22421917,591.74356847)(902.39422212,591.69356951)
\lineto(902.70922212,591.64856951)
\lineto(902.79922212,591.64856951)
\curveto(902.86921853,591.62856859)(902.95921844,591.6185686)(903.06922212,591.61856951)
\curveto(903.18921821,591.6185686)(903.28921811,591.62856859)(903.36922212,591.64856951)
\curveto(903.43921796,591.64856857)(903.4942179,591.65356856)(903.53422212,591.66356951)
\curveto(903.5942178,591.67356854)(903.65421774,591.67856854)(903.71422212,591.67856951)
\curveto(903.77421762,591.68856853)(903.82921757,591.69856852)(903.87922212,591.70856951)
\curveto(904.16921723,591.78856843)(904.399217,591.89356832)(904.56922212,592.02356951)
\curveto(904.73921666,592.15356806)(904.85921654,592.37356784)(904.92922212,592.68356951)
\curveto(904.94921645,592.73356748)(904.95421644,592.78856743)(904.94422212,592.84856951)
\curveto(904.93421646,592.90856731)(904.92421647,592.95356726)(904.91422212,592.98356951)
\curveto(904.86421653,593.17356704)(904.7942166,593.3135669)(904.70422212,593.40356951)
\curveto(904.61421678,593.50356671)(904.4992169,593.59356662)(904.35922212,593.67356951)
\curveto(904.26921713,593.73356648)(904.16921723,593.78356643)(904.05922212,593.82356951)
\lineto(903.72922212,593.94356951)
\curveto(903.6992177,593.95356626)(903.66921773,593.95856626)(903.63922212,593.95856951)
\curveto(903.61921778,593.95856626)(903.5942178,593.96856625)(903.56422212,593.98856951)
\curveto(903.22421817,594.09856612)(902.86921853,594.17856604)(902.49922212,594.22856951)
\curveto(902.13921926,594.28856593)(901.7992196,594.38356583)(901.47922212,594.51356951)
\curveto(901.37922002,594.55356566)(901.28422011,594.58856563)(901.19422212,594.61856951)
\curveto(901.10422029,594.64856557)(901.01922038,594.68856553)(900.93922212,594.73856951)
\curveto(900.74922065,594.84856537)(900.57422082,594.97356524)(900.41422212,595.11356951)
\curveto(900.25422114,595.25356496)(900.12922127,595.42856479)(900.03922212,595.63856951)
\curveto(900.00922139,595.70856451)(899.98422141,595.77856444)(899.96422212,595.84856951)
\curveto(899.95422144,595.9185643)(899.93922146,595.99356422)(899.91922212,596.07356951)
\curveto(899.88922151,596.19356402)(899.87922152,596.32856389)(899.88922212,596.47856951)
\curveto(899.8992215,596.63856358)(899.91422148,596.77356344)(899.93422212,596.88356951)
\curveto(899.95422144,596.93356328)(899.96422143,596.97356324)(899.96422212,597.00356951)
\curveto(899.97422142,597.04356317)(899.98922141,597.08356313)(900.00922212,597.12356951)
\curveto(900.0992213,597.35356286)(900.21922118,597.55356266)(900.36922212,597.72356951)
\curveto(900.52922087,597.89356232)(900.70922069,598.04356217)(900.90922212,598.17356951)
\curveto(901.05922034,598.26356195)(901.22422017,598.33356188)(901.40422212,598.38356951)
\curveto(901.58421981,598.44356177)(901.77421962,598.49856172)(901.97422212,598.54856951)
\curveto(902.04421935,598.55856166)(902.10921929,598.56856165)(902.16922212,598.57856951)
\curveto(902.23921916,598.58856163)(902.31421908,598.59856162)(902.39422212,598.60856951)
\curveto(902.42421897,598.6185616)(902.46421893,598.6185616)(902.51422212,598.60856951)
\curveto(902.56421883,598.59856162)(902.5992188,598.60356161)(902.61922212,598.62356951)
}
}
{
\newrgbcolor{curcolor}{0.50196081 0.50196081 0.50196081}
\pscustom[linestyle=none,fillstyle=solid,fillcolor=curcolor]
{
\newpath
\moveto(812.80437349,601.42860613)
\lineto(827.80437349,601.42860613)
\lineto(827.80437349,586.42860613)
\lineto(812.80437349,586.42860613)
\closepath
}
}
{
\newrgbcolor{curcolor}{0 0 0}
\pscustom[linestyle=none,fillstyle=solid,fillcolor=curcolor]
{
\newpath
\moveto(831.7125815,578.52639178)
\curveto(831.87258084,578.5263811)(832.04758067,578.5263811)(832.2375815,578.52639178)
\curveto(832.42758029,578.53638109)(832.57258014,578.51138111)(832.6725815,578.45139178)
\curveto(832.76257995,578.39138123)(832.82257989,578.29638133)(832.8525815,578.16639178)
\curveto(832.89257982,578.03638159)(832.93257978,577.91638171)(832.9725815,577.80639178)
\curveto(833.05257966,577.60638202)(833.12257959,577.40138222)(833.1825815,577.19139178)
\curveto(833.24257947,576.99138263)(833.3125794,576.79138283)(833.3925815,576.59139178)
\curveto(833.4125793,576.54138308)(833.42757929,576.49138313)(833.4375815,576.44139178)
\lineto(833.4675815,576.29139178)
\curveto(833.53757918,576.1213835)(833.59757912,575.94138368)(833.6475815,575.75139178)
\curveto(833.70757901,575.57138405)(833.76757895,575.38638424)(833.8275815,575.19639178)
\curveto(833.96757875,574.78638484)(834.10257861,574.38138524)(834.2325815,573.98139178)
\curveto(834.37257834,573.58138604)(834.5125782,573.17638645)(834.6525815,572.76639178)
\curveto(834.72257799,572.56638706)(834.78257793,572.36138726)(834.8325815,572.15139178)
\curveto(834.89257782,571.95138767)(834.96257775,571.75138787)(835.0425815,571.55139178)
\curveto(835.06257765,571.50138812)(835.07757764,571.44638818)(835.0875815,571.38639178)
\lineto(835.1475815,571.20639178)
\curveto(835.25757746,570.91638871)(835.35757736,570.61638901)(835.4475815,570.30639178)
\curveto(835.48757723,570.20638942)(835.52257719,570.10138952)(835.5525815,569.99139178)
\curveto(835.58257713,569.89138973)(835.62757709,569.80138982)(835.6875815,569.72139178)
\curveto(835.70757701,569.70138992)(835.74257697,569.67138995)(835.7925815,569.63139178)
\curveto(835.9125768,569.64138998)(835.98757673,569.69138993)(836.0175815,569.78139178)
\curveto(836.04757667,569.88138974)(836.08257663,569.97638965)(836.1225815,570.06639178)
\curveto(836.23257648,570.3263893)(836.32257639,570.59138903)(836.3925815,570.86139178)
\curveto(836.46257625,571.13138849)(836.54757617,571.39638823)(836.6475815,571.65639178)
\curveto(836.70757601,571.81638781)(836.75757596,571.97638765)(836.7975815,572.13639178)
\curveto(836.84757587,572.29638733)(836.90257581,572.45638717)(836.9625815,572.61639178)
\curveto(837.0125757,572.73638689)(837.05257566,572.85638677)(837.0825815,572.97639178)
\curveto(837.12257559,573.10638652)(837.16757555,573.23138639)(837.2175815,573.35139178)
\curveto(837.36757535,573.77138585)(837.50757521,574.19638543)(837.6375815,574.62639178)
\curveto(837.76757495,575.05638457)(837.9125748,575.48138414)(838.0725815,575.90139178)
\curveto(838.09257462,575.94138368)(838.10257461,575.97638365)(838.1025815,576.00639178)
\curveto(838.10257461,576.04638358)(838.1125746,576.08638354)(838.1325815,576.12639178)
\curveto(838.19257452,576.27638335)(838.24757447,576.43138319)(838.2975815,576.59139178)
\curveto(838.34757437,576.75138287)(838.39757432,576.90638272)(838.4475815,577.05639178)
\curveto(838.50757421,577.20638242)(838.55757416,577.35638227)(838.5975815,577.50639178)
\curveto(838.64757407,577.66638196)(838.70257401,577.8263818)(838.7625815,577.98639178)
\curveto(838.79257392,578.07638155)(838.82257389,578.16138146)(838.8525815,578.24139178)
\curveto(838.89257382,578.33138129)(838.95257376,578.40138122)(839.0325815,578.45139178)
\curveto(839.09257362,578.50138112)(839.17257354,578.5263811)(839.2725815,578.52639178)
\curveto(839.38257333,578.5263811)(839.49257322,578.5263811)(839.6025815,578.52639178)
\lineto(839.9325815,578.52639178)
\curveto(840.0125727,578.50638112)(840.08257263,578.48638114)(840.1425815,578.46639178)
\curveto(840.20257251,578.45638117)(840.24257247,578.41138121)(840.2625815,578.33139178)
\lineto(840.2625815,578.25639178)
\curveto(840.27257244,578.23638139)(840.27257244,578.21638141)(840.2625815,578.19639178)
\curveto(840.24257247,578.09638153)(840.2125725,577.99638163)(840.1725815,577.89639178)
\curveto(840.14257257,577.80638182)(840.1125726,577.7213819)(840.0825815,577.64139178)
\curveto(840.04257267,577.56138206)(840.00757271,577.47638215)(839.9775815,577.38639178)
\curveto(839.95757276,577.30638232)(839.93257278,577.2263824)(839.9025815,577.14639178)
\curveto(839.84257287,577.00638262)(839.78757293,576.85638277)(839.7375815,576.69639178)
\curveto(839.69757302,576.54638308)(839.64757307,576.40138322)(839.5875815,576.26139178)
\curveto(839.56757315,576.2213834)(839.55757316,576.18638344)(839.5575815,576.15639178)
\curveto(839.55757316,576.1263835)(839.54757317,576.09138353)(839.5275815,576.05139178)
\curveto(839.44757327,575.88138374)(839.37757334,575.70138392)(839.3175815,575.51139178)
\curveto(839.26757345,575.3213843)(839.20257351,575.14138448)(839.1225815,574.97139178)
\curveto(839.10257361,574.93138469)(839.09257362,574.89138473)(839.0925815,574.85139178)
\curveto(839.09257362,574.8213848)(839.08257363,574.79138483)(839.0625815,574.76139178)
\curveto(839.0125737,574.63138499)(838.96257375,574.49638513)(838.9125815,574.35639178)
\curveto(838.87257384,574.2263854)(838.82757389,574.09638553)(838.7775815,573.96639178)
\curveto(838.75757396,573.93638569)(838.74257397,573.90138572)(838.7325815,573.86139178)
\curveto(838.73257398,573.83138579)(838.72257399,573.79638583)(838.7025815,573.75639178)
\curveto(838.55257416,573.37638625)(838.4125743,572.99138663)(838.2825815,572.60139178)
\curveto(838.16257455,572.21138741)(838.02757469,571.8263878)(837.8775815,571.44639178)
\lineto(837.8325815,571.31139178)
\lineto(837.7125815,570.98139178)
\curveto(837.68257503,570.88138874)(837.64757507,570.77638885)(837.6075815,570.66639178)
\curveto(837.55757516,570.54638908)(837.5125752,570.4213892)(837.4725815,570.29139178)
\curveto(837.43257528,570.17138945)(837.38757533,570.05138957)(837.3375815,569.93139178)
\lineto(837.2775815,569.75139178)
\lineto(837.2175815,569.57139178)
\curveto(837.15757556,569.4213902)(837.10257561,569.26639036)(837.0525815,569.10639178)
\curveto(837.00257571,568.94639068)(836.94757577,568.79639083)(836.8875815,568.65639178)
\curveto(836.83757588,568.5263911)(836.78757593,568.38639124)(836.7375815,568.23639178)
\curveto(836.69757602,568.09639153)(836.62757609,567.99139163)(836.5275815,567.92139178)
\curveto(836.48757623,567.90139172)(836.44257627,567.88639174)(836.3925815,567.87639178)
\curveto(836.34257637,567.86639176)(836.28757643,567.85639177)(836.2275815,567.84639178)
\lineto(835.8075815,567.84639178)
\lineto(835.4175815,567.84639178)
\curveto(835.27757744,567.83639179)(835.16757755,567.85639177)(835.0875815,567.90639178)
\curveto(834.99757772,567.95639167)(834.93757778,568.0263916)(834.9075815,568.11639178)
\curveto(834.87757784,568.21639141)(834.83757788,568.3213913)(834.7875815,568.43139178)
\curveto(834.72757799,568.58139104)(834.67257804,568.73639089)(834.6225815,568.89639178)
\curveto(834.57257814,569.06639056)(834.5125782,569.23139039)(834.4425815,569.39139178)
\curveto(834.42257829,569.43139019)(834.40757831,569.47139015)(834.3975815,569.51139178)
\curveto(834.39757832,569.55139007)(834.38757833,569.59139003)(834.3675815,569.63139178)
\curveto(834.28757843,569.83138979)(834.2125785,570.03138959)(834.1425815,570.23139178)
\curveto(834.08257863,570.44138918)(834.0125787,570.64138898)(833.9325815,570.83139178)
\curveto(833.9125788,570.88138874)(833.89757882,570.9263887)(833.8875815,570.96639178)
\curveto(833.88757883,571.00638862)(833.87757884,571.04638858)(833.8575815,571.08639178)
\curveto(833.80757891,571.2263884)(833.75757896,571.36138826)(833.7075815,571.49139178)
\lineto(833.5575815,571.91139178)
\curveto(833.53757918,571.95138767)(833.52257919,571.99138763)(833.5125815,572.03139178)
\curveto(833.5125792,572.07138755)(833.50257921,572.11138751)(833.4825815,572.15139178)
\lineto(833.3325815,572.54139178)
\curveto(833.29257942,572.68138694)(833.24757947,572.8213868)(833.1975815,572.96139178)
\curveto(833.14757957,573.07138655)(833.10757961,573.18138644)(833.0775815,573.29139178)
\curveto(833.04757967,573.41138621)(833.00757971,573.5263861)(832.9575815,573.63639178)
\curveto(832.84757987,573.91638571)(832.74757997,574.20138542)(832.6575815,574.49139178)
\curveto(832.56758015,574.79138483)(832.46258025,575.08138454)(832.3425815,575.36139178)
\curveto(832.30258041,575.45138417)(832.26758045,575.54138408)(832.2375815,575.63139178)
\curveto(832.2175805,575.73138389)(832.19258052,575.8213838)(832.1625815,575.90139178)
\curveto(832.13258058,575.96138366)(832.10758061,576.0213836)(832.0875815,576.08139178)
\curveto(832.07758064,576.15138347)(832.05758066,576.21638341)(832.0275815,576.27639178)
\curveto(831.93758078,576.50638312)(831.85258086,576.74138288)(831.7725815,576.98139178)
\curveto(831.70258101,577.2213824)(831.62258109,577.45638217)(831.5325815,577.68639178)
\curveto(831.5125812,577.75638187)(831.48758123,577.8263818)(831.4575815,577.89639178)
\curveto(831.43758128,577.96638166)(831.4175813,578.04138158)(831.3975815,578.12139178)
\curveto(831.35758136,578.2213814)(831.35258136,578.30638132)(831.3825815,578.37639178)
\curveto(831.4125813,578.44638118)(831.48258123,578.49138113)(831.5925815,578.51139178)
\curveto(831.6125811,578.5213811)(831.63258108,578.5213811)(831.6525815,578.51139178)
\curveto(831.67258104,578.51138111)(831.69258102,578.51638111)(831.7125815,578.52639178)
}
}
{
\newrgbcolor{curcolor}{0 0 0}
\pscustom[linestyle=none,fillstyle=solid,fillcolor=curcolor]
{
\newpath
\moveto(841.58750337,577.08639178)
\curveto(841.50750225,577.14638248)(841.4625023,577.25138237)(841.45250337,577.40139178)
\lineto(841.45250337,577.86639178)
\lineto(841.45250337,578.12139178)
\curveto(841.45250231,578.21138141)(841.46750229,578.28638134)(841.49750337,578.34639178)
\curveto(841.53750222,578.4263812)(841.61750214,578.48638114)(841.73750337,578.52639178)
\curveto(841.757502,578.53638109)(841.77750198,578.53638109)(841.79750337,578.52639178)
\curveto(841.82750193,578.5263811)(841.85250191,578.53138109)(841.87250337,578.54139178)
\curveto(842.04250172,578.54138108)(842.20250156,578.53638109)(842.35250337,578.52639178)
\curveto(842.50250126,578.51638111)(842.60250116,578.45638117)(842.65250337,578.34639178)
\curveto(842.68250108,578.28638134)(842.69750106,578.21138141)(842.69750337,578.12139178)
\lineto(842.69750337,577.86639178)
\curveto(842.69750106,577.68638194)(842.69250107,577.51638211)(842.68250337,577.35639178)
\curveto(842.68250108,577.19638243)(842.61750114,577.09138253)(842.48750337,577.04139178)
\curveto(842.43750132,577.0213826)(842.38250138,577.01138261)(842.32250337,577.01139178)
\lineto(842.15750337,577.01139178)
\lineto(841.84250337,577.01139178)
\curveto(841.74250202,577.01138261)(841.6575021,577.03638259)(841.58750337,577.08639178)
\moveto(842.69750337,568.58139178)
\lineto(842.69750337,568.26639178)
\curveto(842.70750105,568.16639146)(842.68750107,568.08639154)(842.63750337,568.02639178)
\curveto(842.60750115,567.96639166)(842.5625012,567.9263917)(842.50250337,567.90639178)
\curveto(842.44250132,567.89639173)(842.37250139,567.88139174)(842.29250337,567.86139178)
\lineto(842.06750337,567.86139178)
\curveto(841.93750182,567.86139176)(841.82250194,567.86639176)(841.72250337,567.87639178)
\curveto(841.63250213,567.89639173)(841.5625022,567.94639168)(841.51250337,568.02639178)
\curveto(841.47250229,568.08639154)(841.45250231,568.16139146)(841.45250337,568.25139178)
\lineto(841.45250337,568.53639178)
\lineto(841.45250337,574.88139178)
\lineto(841.45250337,575.19639178)
\curveto(841.45250231,575.30638432)(841.47750228,575.39138423)(841.52750337,575.45139178)
\curveto(841.5575022,575.50138412)(841.59750216,575.53138409)(841.64750337,575.54139178)
\curveto(841.69750206,575.55138407)(841.75250201,575.56638406)(841.81250337,575.58639178)
\curveto(841.83250193,575.58638404)(841.85250191,575.58138404)(841.87250337,575.57139178)
\curveto(841.90250186,575.57138405)(841.92750183,575.57638405)(841.94750337,575.58639178)
\curveto(842.07750168,575.58638404)(842.20750155,575.58138404)(842.33750337,575.57139178)
\curveto(842.47750128,575.57138405)(842.57250119,575.53138409)(842.62250337,575.45139178)
\curveto(842.67250109,575.39138423)(842.69750106,575.31138431)(842.69750337,575.21139178)
\lineto(842.69750337,574.92639178)
\lineto(842.69750337,568.58139178)
}
}
{
\newrgbcolor{curcolor}{0 0 0}
\pscustom[linestyle=none,fillstyle=solid,fillcolor=curcolor]
{
\newpath
\moveto(851.60234712,568.67139178)
\lineto(851.60234712,568.28139178)
\curveto(851.60233925,568.16139146)(851.57733927,568.06139156)(851.52734712,567.98139178)
\curveto(851.47733937,567.91139171)(851.39233946,567.87139175)(851.27234712,567.86139178)
\lineto(850.92734712,567.86139178)
\curveto(850.86733998,567.86139176)(850.80734004,567.85639177)(850.74734712,567.84639178)
\curveto(850.69734015,567.84639178)(850.6523402,567.85639177)(850.61234712,567.87639178)
\curveto(850.52234033,567.89639173)(850.46234039,567.93639169)(850.43234712,567.99639178)
\curveto(850.39234046,568.04639158)(850.36734048,568.10639152)(850.35734712,568.17639178)
\curveto(850.35734049,568.24639138)(850.34234051,568.31639131)(850.31234712,568.38639178)
\curveto(850.30234055,568.40639122)(850.28734056,568.4213912)(850.26734712,568.43139178)
\curveto(850.25734059,568.45139117)(850.24234061,568.47139115)(850.22234712,568.49139178)
\curveto(850.12234073,568.50139112)(850.04234081,568.48139114)(849.98234712,568.43139178)
\curveto(849.93234092,568.38139124)(849.87734097,568.33139129)(849.81734712,568.28139178)
\curveto(849.61734123,568.13139149)(849.41734143,568.01639161)(849.21734712,567.93639178)
\curveto(849.03734181,567.85639177)(848.82734202,567.79639183)(848.58734712,567.75639178)
\curveto(848.35734249,567.71639191)(848.11734273,567.69639193)(847.86734712,567.69639178)
\curveto(847.62734322,567.68639194)(847.38734346,567.70139192)(847.14734712,567.74139178)
\curveto(846.90734394,567.77139185)(846.69734415,567.8263918)(846.51734712,567.90639178)
\curveto(845.99734485,568.1263915)(845.57734527,568.4213912)(845.25734712,568.79139178)
\curveto(844.93734591,569.17139045)(844.68734616,569.64138998)(844.50734712,570.20139178)
\curveto(844.46734638,570.29138933)(844.43734641,570.38138924)(844.41734712,570.47139178)
\curveto(844.40734644,570.57138905)(844.38734646,570.67138895)(844.35734712,570.77139178)
\curveto(844.3473465,570.8213888)(844.34234651,570.87138875)(844.34234712,570.92139178)
\curveto(844.34234651,570.97138865)(844.33734651,571.0213886)(844.32734712,571.07139178)
\curveto(844.30734654,571.1213885)(844.29734655,571.17138845)(844.29734712,571.22139178)
\curveto(844.30734654,571.28138834)(844.30734654,571.33638829)(844.29734712,571.38639178)
\lineto(844.29734712,571.53639178)
\curveto(844.27734657,571.58638804)(844.26734658,571.65138797)(844.26734712,571.73139178)
\curveto(844.26734658,571.81138781)(844.27734657,571.87638775)(844.29734712,571.92639178)
\lineto(844.29734712,572.09139178)
\curveto(844.31734653,572.16138746)(844.32234653,572.23138739)(844.31234712,572.30139178)
\curveto(844.31234654,572.38138724)(844.32234653,572.45638717)(844.34234712,572.52639178)
\curveto(844.3523465,572.57638705)(844.35734649,572.621387)(844.35734712,572.66139178)
\curveto(844.35734649,572.70138692)(844.36234649,572.74638688)(844.37234712,572.79639178)
\curveto(844.40234645,572.89638673)(844.42734642,572.99138663)(844.44734712,573.08139178)
\curveto(844.46734638,573.18138644)(844.49234636,573.27638635)(844.52234712,573.36639178)
\curveto(844.6523462,573.74638588)(844.81734603,574.08638554)(845.01734712,574.38639178)
\curveto(845.22734562,574.69638493)(845.47734537,574.95138467)(845.76734712,575.15139178)
\curveto(845.93734491,575.27138435)(846.11234474,575.37138425)(846.29234712,575.45139178)
\curveto(846.48234437,575.53138409)(846.68734416,575.60138402)(846.90734712,575.66139178)
\curveto(846.97734387,575.67138395)(847.04234381,575.68138394)(847.10234712,575.69139178)
\curveto(847.17234368,575.70138392)(847.24234361,575.71638391)(847.31234712,575.73639178)
\lineto(847.46234712,575.73639178)
\curveto(847.54234331,575.75638387)(847.65734319,575.76638386)(847.80734712,575.76639178)
\curveto(847.96734288,575.76638386)(848.08734276,575.75638387)(848.16734712,575.73639178)
\curveto(848.20734264,575.7263839)(848.26234259,575.7213839)(848.33234712,575.72139178)
\curveto(848.44234241,575.69138393)(848.5523423,575.66638396)(848.66234712,575.64639178)
\curveto(848.77234208,575.63638399)(848.87734197,575.60638402)(848.97734712,575.55639178)
\curveto(849.12734172,575.49638413)(849.26734158,575.43138419)(849.39734712,575.36139178)
\curveto(849.53734131,575.29138433)(849.66734118,575.21138441)(849.78734712,575.12139178)
\curveto(849.847341,575.07138455)(849.90734094,575.01638461)(849.96734712,574.95639178)
\curveto(850.03734081,574.90638472)(850.12734072,574.89138473)(850.23734712,574.91139178)
\curveto(850.25734059,574.94138468)(850.27234058,574.96638466)(850.28234712,574.98639178)
\curveto(850.30234055,575.00638462)(850.31734053,575.03638459)(850.32734712,575.07639178)
\curveto(850.35734049,575.16638446)(850.36734048,575.28138434)(850.35734712,575.42139178)
\lineto(850.35734712,575.79639178)
\lineto(850.35734712,577.52139178)
\lineto(850.35734712,577.98639178)
\curveto(850.35734049,578.16638146)(850.38234047,578.29638133)(850.43234712,578.37639178)
\curveto(850.47234038,578.44638118)(850.53234032,578.49138113)(850.61234712,578.51139178)
\curveto(850.63234022,578.51138111)(850.65734019,578.51138111)(850.68734712,578.51139178)
\curveto(850.71734013,578.5213811)(850.74234011,578.5263811)(850.76234712,578.52639178)
\curveto(850.90233995,578.53638109)(851.0473398,578.53638109)(851.19734712,578.52639178)
\curveto(851.35733949,578.5263811)(851.46733938,578.48638114)(851.52734712,578.40639178)
\curveto(851.57733927,578.3263813)(851.60233925,578.2263814)(851.60234712,578.10639178)
\lineto(851.60234712,577.73139178)
\lineto(851.60234712,568.67139178)
\moveto(850.38734712,571.50639178)
\curveto(850.40734044,571.55638807)(850.41734043,571.621388)(850.41734712,571.70139178)
\curveto(850.41734043,571.79138783)(850.40734044,571.86138776)(850.38734712,571.91139178)
\lineto(850.38734712,572.13639178)
\curveto(850.36734048,572.2263874)(850.3523405,572.31638731)(850.34234712,572.40639178)
\curveto(850.33234052,572.50638712)(850.31234054,572.59638703)(850.28234712,572.67639178)
\curveto(850.26234059,572.75638687)(850.24234061,572.83138679)(850.22234712,572.90139178)
\curveto(850.21234064,572.97138665)(850.19234066,573.04138658)(850.16234712,573.11139178)
\curveto(850.04234081,573.41138621)(849.88734096,573.67638595)(849.69734712,573.90639178)
\curveto(849.50734134,574.13638549)(849.26734158,574.31638531)(848.97734712,574.44639178)
\curveto(848.87734197,574.49638513)(848.77234208,574.53138509)(848.66234712,574.55139178)
\curveto(848.56234229,574.58138504)(848.4523424,574.60638502)(848.33234712,574.62639178)
\curveto(848.2523426,574.64638498)(848.16234269,574.65638497)(848.06234712,574.65639178)
\lineto(847.79234712,574.65639178)
\curveto(847.74234311,574.64638498)(847.69734315,574.63638499)(847.65734712,574.62639178)
\lineto(847.52234712,574.62639178)
\curveto(847.44234341,574.60638502)(847.35734349,574.58638504)(847.26734712,574.56639178)
\curveto(847.18734366,574.54638508)(847.10734374,574.5213851)(847.02734712,574.49139178)
\curveto(846.70734414,574.35138527)(846.4473444,574.14638548)(846.24734712,573.87639178)
\curveto(846.05734479,573.61638601)(845.90234495,573.31138631)(845.78234712,572.96139178)
\curveto(845.74234511,572.85138677)(845.71234514,572.73638689)(845.69234712,572.61639178)
\curveto(845.68234517,572.50638712)(845.66734518,572.39638723)(845.64734712,572.28639178)
\curveto(845.6473452,572.24638738)(845.64234521,572.20638742)(845.63234712,572.16639178)
\lineto(845.63234712,572.06139178)
\curveto(845.61234524,572.01138761)(845.60234525,571.95638767)(845.60234712,571.89639178)
\curveto(845.61234524,571.83638779)(845.61734523,571.78138784)(845.61734712,571.73139178)
\lineto(845.61734712,571.40139178)
\curveto(845.61734523,571.30138832)(845.62734522,571.20638842)(845.64734712,571.11639178)
\curveto(845.65734519,571.08638854)(845.66234519,571.03638859)(845.66234712,570.96639178)
\curveto(845.68234517,570.89638873)(845.69734515,570.8263888)(845.70734712,570.75639178)
\lineto(845.76734712,570.54639178)
\curveto(845.87734497,570.19638943)(846.02734482,569.89638973)(846.21734712,569.64639178)
\curveto(846.40734444,569.39639023)(846.6473442,569.19139043)(846.93734712,569.03139178)
\curveto(847.02734382,568.98139064)(847.11734373,568.94139068)(847.20734712,568.91139178)
\curveto(847.29734355,568.88139074)(847.39734345,568.85139077)(847.50734712,568.82139178)
\curveto(847.55734329,568.80139082)(847.60734324,568.79639083)(847.65734712,568.80639178)
\curveto(847.71734313,568.81639081)(847.77234308,568.81139081)(847.82234712,568.79139178)
\curveto(847.86234299,568.78139084)(847.90234295,568.77639085)(847.94234712,568.77639178)
\lineto(848.07734712,568.77639178)
\lineto(848.21234712,568.77639178)
\curveto(848.24234261,568.78639084)(848.29234256,568.79139083)(848.36234712,568.79139178)
\curveto(848.44234241,568.81139081)(848.52234233,568.8263908)(848.60234712,568.83639178)
\curveto(848.68234217,568.85639077)(848.75734209,568.88139074)(848.82734712,568.91139178)
\curveto(849.15734169,569.05139057)(849.42234143,569.2263904)(849.62234712,569.43639178)
\curveto(849.83234102,569.65638997)(850.00734084,569.93138969)(850.14734712,570.26139178)
\curveto(850.19734065,570.37138925)(850.23234062,570.48138914)(850.25234712,570.59139178)
\curveto(850.27234058,570.70138892)(850.29734055,570.81138881)(850.32734712,570.92139178)
\curveto(850.3473405,570.96138866)(850.35734049,570.99638863)(850.35734712,571.02639178)
\curveto(850.35734049,571.06638856)(850.36234049,571.10638852)(850.37234712,571.14639178)
\curveto(850.38234047,571.20638842)(850.38234047,571.26638836)(850.37234712,571.32639178)
\curveto(850.37234048,571.38638824)(850.37734047,571.44638818)(850.38734712,571.50639178)
}
}
{
\newrgbcolor{curcolor}{0 0 0}
\pscustom[linestyle=none,fillstyle=solid,fillcolor=curcolor]
{
\newpath
\moveto(860.29859712,572.03139178)
\curveto(860.31858944,571.93138769)(860.31858944,571.81638781)(860.29859712,571.68639178)
\curveto(860.28858947,571.56638806)(860.2585895,571.48138814)(860.20859712,571.43139178)
\curveto(860.1585896,571.39138823)(860.08358967,571.36138826)(859.98359712,571.34139178)
\curveto(859.89358986,571.33138829)(859.78858997,571.3263883)(859.66859712,571.32639178)
\lineto(859.30859712,571.32639178)
\curveto(859.18859057,571.33638829)(859.08359067,571.34138828)(858.99359712,571.34139178)
\lineto(855.15359712,571.34139178)
\curveto(855.07359468,571.34138828)(854.99359476,571.33638829)(854.91359712,571.32639178)
\curveto(854.83359492,571.3263883)(854.76859499,571.31138831)(854.71859712,571.28139178)
\curveto(854.67859508,571.26138836)(854.63859512,571.2213884)(854.59859712,571.16139178)
\curveto(854.57859518,571.13138849)(854.5585952,571.08638854)(854.53859712,571.02639178)
\curveto(854.51859524,570.97638865)(854.51859524,570.9263887)(854.53859712,570.87639178)
\curveto(854.54859521,570.8263888)(854.5535952,570.78138884)(854.55359712,570.74139178)
\curveto(854.5535952,570.70138892)(854.5585952,570.66138896)(854.56859712,570.62139178)
\curveto(854.58859517,570.54138908)(854.60859515,570.45638917)(854.62859712,570.36639178)
\curveto(854.64859511,570.28638934)(854.67859508,570.20638942)(854.71859712,570.12639178)
\curveto(854.94859481,569.58639004)(855.32859443,569.20139042)(855.85859712,568.97139178)
\curveto(855.91859384,568.94139068)(855.98359377,568.91639071)(856.05359712,568.89639178)
\lineto(856.26359712,568.83639178)
\curveto(856.29359346,568.8263908)(856.34359341,568.8213908)(856.41359712,568.82139178)
\curveto(856.5535932,568.78139084)(856.73859302,568.76139086)(856.96859712,568.76139178)
\curveto(857.19859256,568.76139086)(857.38359237,568.78139084)(857.52359712,568.82139178)
\curveto(857.66359209,568.86139076)(857.78859197,568.90139072)(857.89859712,568.94139178)
\curveto(858.01859174,568.99139063)(858.12859163,569.05139057)(858.22859712,569.12139178)
\curveto(858.33859142,569.19139043)(858.43359132,569.27139035)(858.51359712,569.36139178)
\curveto(858.59359116,569.46139016)(858.66359109,569.56639006)(858.72359712,569.67639178)
\curveto(858.78359097,569.77638985)(858.83359092,569.88138974)(858.87359712,569.99139178)
\curveto(858.92359083,570.10138952)(859.00359075,570.18138944)(859.11359712,570.23139178)
\curveto(859.1535906,570.25138937)(859.21859054,570.26638936)(859.30859712,570.27639178)
\curveto(859.39859036,570.28638934)(859.48859027,570.28638934)(859.57859712,570.27639178)
\curveto(859.66859009,570.27638935)(859.75359,570.27138935)(859.83359712,570.26139178)
\curveto(859.91358984,570.25138937)(859.96858979,570.23138939)(859.99859712,570.20139178)
\curveto(860.09858966,570.13138949)(860.12358963,570.01638961)(860.07359712,569.85639178)
\curveto(859.99358976,569.58639004)(859.88858987,569.34639028)(859.75859712,569.13639178)
\curveto(859.5585902,568.81639081)(859.32859043,568.55139107)(859.06859712,568.34139178)
\curveto(858.81859094,568.14139148)(858.49859126,567.97639165)(858.10859712,567.84639178)
\curveto(858.00859175,567.80639182)(857.90859185,567.78139184)(857.80859712,567.77139178)
\curveto(857.70859205,567.75139187)(857.60359215,567.73139189)(857.49359712,567.71139178)
\curveto(857.44359231,567.70139192)(857.39359236,567.69639193)(857.34359712,567.69639178)
\curveto(857.30359245,567.69639193)(857.2585925,567.69139193)(857.20859712,567.68139178)
\lineto(857.05859712,567.68139178)
\curveto(857.00859275,567.67139195)(856.94859281,567.66639196)(856.87859712,567.66639178)
\curveto(856.81859294,567.66639196)(856.76859299,567.67139195)(856.72859712,567.68139178)
\lineto(856.59359712,567.68139178)
\curveto(856.54359321,567.69139193)(856.49859326,567.69639193)(856.45859712,567.69639178)
\curveto(856.41859334,567.69639193)(856.37859338,567.70139192)(856.33859712,567.71139178)
\curveto(856.28859347,567.7213919)(856.23359352,567.73139189)(856.17359712,567.74139178)
\curveto(856.11359364,567.74139188)(856.0585937,567.74639188)(856.00859712,567.75639178)
\curveto(855.91859384,567.77639185)(855.82859393,567.80139182)(855.73859712,567.83139178)
\curveto(855.64859411,567.85139177)(855.56359419,567.87639175)(855.48359712,567.90639178)
\curveto(855.44359431,567.9263917)(855.40859435,567.93639169)(855.37859712,567.93639178)
\curveto(855.34859441,567.94639168)(855.31359444,567.96139166)(855.27359712,567.98139178)
\curveto(855.12359463,568.05139157)(854.96359479,568.13639149)(854.79359712,568.23639178)
\curveto(854.50359525,568.4263912)(854.2535955,568.65639097)(854.04359712,568.92639178)
\curveto(853.84359591,569.20639042)(853.67359608,569.51639011)(853.53359712,569.85639178)
\curveto(853.48359627,569.96638966)(853.44359631,570.08138954)(853.41359712,570.20139178)
\curveto(853.39359636,570.3213893)(853.36359639,570.44138918)(853.32359712,570.56139178)
\curveto(853.31359644,570.60138902)(853.30859645,570.63638899)(853.30859712,570.66639178)
\curveto(853.30859645,570.69638893)(853.30359645,570.73638889)(853.29359712,570.78639178)
\curveto(853.27359648,570.86638876)(853.2585965,570.95138867)(853.24859712,571.04139178)
\curveto(853.23859652,571.13138849)(853.22359653,571.2213884)(853.20359712,571.31139178)
\lineto(853.20359712,571.52139178)
\curveto(853.19359656,571.56138806)(853.18359657,571.61638801)(853.17359712,571.68639178)
\curveto(853.17359658,571.76638786)(853.17859658,571.83138779)(853.18859712,571.88139178)
\lineto(853.18859712,572.04639178)
\curveto(853.20859655,572.09638753)(853.21359654,572.14638748)(853.20359712,572.19639178)
\curveto(853.20359655,572.25638737)(853.20859655,572.31138731)(853.21859712,572.36139178)
\curveto(853.2585965,572.5213871)(853.28859647,572.68138694)(853.30859712,572.84139178)
\curveto(853.33859642,573.00138662)(853.38359637,573.15138647)(853.44359712,573.29139178)
\curveto(853.49359626,573.40138622)(853.53859622,573.51138611)(853.57859712,573.62139178)
\curveto(853.62859613,573.74138588)(853.68359607,573.85638577)(853.74359712,573.96639178)
\curveto(853.96359579,574.31638531)(854.21359554,574.61638501)(854.49359712,574.86639178)
\curveto(854.77359498,575.1263845)(855.11859464,575.34138428)(855.52859712,575.51139178)
\curveto(855.64859411,575.56138406)(855.76859399,575.59638403)(855.88859712,575.61639178)
\curveto(856.01859374,575.64638398)(856.1535936,575.67638395)(856.29359712,575.70639178)
\curveto(856.34359341,575.71638391)(856.38859337,575.7213839)(856.42859712,575.72139178)
\curveto(856.46859329,575.73138389)(856.51359324,575.73638389)(856.56359712,575.73639178)
\curveto(856.58359317,575.74638388)(856.60859315,575.74638388)(856.63859712,575.73639178)
\curveto(856.66859309,575.7263839)(856.69359306,575.73138389)(856.71359712,575.75139178)
\curveto(857.13359262,575.76138386)(857.49859226,575.71638391)(857.80859712,575.61639178)
\curveto(858.11859164,575.5263841)(858.39859136,575.40138422)(858.64859712,575.24139178)
\curveto(858.69859106,575.2213844)(858.73859102,575.19138443)(858.76859712,575.15139178)
\curveto(858.79859096,575.1213845)(858.83359092,575.09638453)(858.87359712,575.07639178)
\curveto(858.9535908,575.01638461)(859.03359072,574.94638468)(859.11359712,574.86639178)
\curveto(859.20359055,574.78638484)(859.27859048,574.70638492)(859.33859712,574.62639178)
\curveto(859.49859026,574.41638521)(859.63359012,574.21638541)(859.74359712,574.02639178)
\curveto(859.81358994,573.91638571)(859.86858989,573.79638583)(859.90859712,573.66639178)
\curveto(859.94858981,573.53638609)(859.99358976,573.40638622)(860.04359712,573.27639178)
\curveto(860.09358966,573.14638648)(860.12858963,573.01138661)(860.14859712,572.87139178)
\curveto(860.17858958,572.73138689)(860.21358954,572.59138703)(860.25359712,572.45139178)
\curveto(860.26358949,572.38138724)(860.26858949,572.31138731)(860.26859712,572.24139178)
\lineto(860.29859712,572.03139178)
\moveto(858.84359712,572.54139178)
\curveto(858.87359088,572.58138704)(858.89859086,572.63138699)(858.91859712,572.69139178)
\curveto(858.93859082,572.76138686)(858.93859082,572.83138679)(858.91859712,572.90139178)
\curveto(858.8585909,573.1213865)(858.77359098,573.3263863)(858.66359712,573.51639178)
\curveto(858.52359123,573.74638588)(858.36859139,573.94138568)(858.19859712,574.10139178)
\curveto(858.02859173,574.26138536)(857.80859195,574.39638523)(857.53859712,574.50639178)
\curveto(857.46859229,574.5263851)(857.39859236,574.54138508)(857.32859712,574.55139178)
\curveto(857.2585925,574.57138505)(857.18359257,574.59138503)(857.10359712,574.61139178)
\curveto(857.02359273,574.63138499)(856.93859282,574.64138498)(856.84859712,574.64139178)
\lineto(856.59359712,574.64139178)
\curveto(856.56359319,574.621385)(856.52859323,574.61138501)(856.48859712,574.61139178)
\curveto(856.44859331,574.621385)(856.41359334,574.621385)(856.38359712,574.61139178)
\lineto(856.14359712,574.55139178)
\curveto(856.07359368,574.54138508)(856.00359375,574.5263851)(855.93359712,574.50639178)
\curveto(855.64359411,574.38638524)(855.40859435,574.23638539)(855.22859712,574.05639178)
\curveto(855.0585947,573.87638575)(854.90359485,573.65138597)(854.76359712,573.38139178)
\curveto(854.73359502,573.33138629)(854.70359505,573.26638636)(854.67359712,573.18639178)
\curveto(854.64359511,573.11638651)(854.61859514,573.03638659)(854.59859712,572.94639178)
\curveto(854.57859518,572.85638677)(854.57359518,572.77138685)(854.58359712,572.69139178)
\curveto(854.59359516,572.61138701)(854.62859513,572.55138707)(854.68859712,572.51139178)
\curveto(854.76859499,572.45138717)(854.90359485,572.4213872)(855.09359712,572.42139178)
\curveto(855.29359446,572.43138719)(855.46359429,572.43638719)(855.60359712,572.43639178)
\lineto(857.88359712,572.43639178)
\curveto(858.03359172,572.43638719)(858.21359154,572.43138719)(858.42359712,572.42139178)
\curveto(858.63359112,572.4213872)(858.77359098,572.46138716)(858.84359712,572.54139178)
}
}
{
\newrgbcolor{curcolor}{0 0 0}
\pscustom[linestyle=none,fillstyle=solid,fillcolor=curcolor]
{
\newpath
\moveto(868.73023775,572.06139178)
\curveto(868.75022969,572.00138762)(868.76022968,571.90638772)(868.76023775,571.77639178)
\curveto(868.76022968,571.65638797)(868.75522968,571.57138805)(868.74523775,571.52139178)
\lineto(868.74523775,571.37139178)
\curveto(868.7352297,571.29138833)(868.72522971,571.21638841)(868.71523775,571.14639178)
\curveto(868.71522972,571.08638854)(868.71022973,571.01638861)(868.70023775,570.93639178)
\curveto(868.68022976,570.87638875)(868.66522977,570.81638881)(868.65523775,570.75639178)
\curveto(868.65522978,570.69638893)(868.64522979,570.63638899)(868.62523775,570.57639178)
\curveto(868.58522985,570.44638918)(868.55022989,570.31638931)(868.52023775,570.18639178)
\curveto(868.49022995,570.05638957)(868.45022999,569.93638969)(868.40023775,569.82639178)
\curveto(868.19023025,569.34639028)(867.91023053,568.94139068)(867.56023775,568.61139178)
\curveto(867.21023123,568.29139133)(866.78023166,568.04639158)(866.27023775,567.87639178)
\curveto(866.16023228,567.83639179)(866.0402324,567.80639182)(865.91023775,567.78639178)
\curveto(865.79023265,567.76639186)(865.66523277,567.74639188)(865.53523775,567.72639178)
\curveto(865.47523296,567.71639191)(865.41023303,567.71139191)(865.34023775,567.71139178)
\curveto(865.28023316,567.70139192)(865.22023322,567.69639193)(865.16023775,567.69639178)
\curveto(865.12023332,567.68639194)(865.06023338,567.68139194)(864.98023775,567.68139178)
\curveto(864.91023353,567.68139194)(864.86023358,567.68639194)(864.83023775,567.69639178)
\curveto(864.79023365,567.70639192)(864.75023369,567.71139191)(864.71023775,567.71139178)
\curveto(864.67023377,567.70139192)(864.6352338,567.70139192)(864.60523775,567.71139178)
\lineto(864.51523775,567.71139178)
\lineto(864.15523775,567.75639178)
\curveto(864.01523442,567.79639183)(863.88023456,567.83639179)(863.75023775,567.87639178)
\curveto(863.62023482,567.91639171)(863.49523494,567.96139166)(863.37523775,568.01139178)
\curveto(862.92523551,568.21139141)(862.55523588,568.47139115)(862.26523775,568.79139178)
\curveto(861.97523646,569.11139051)(861.7352367,569.50139012)(861.54523775,569.96139178)
\curveto(861.49523694,570.06138956)(861.45523698,570.16138946)(861.42523775,570.26139178)
\curveto(861.40523703,570.36138926)(861.38523705,570.46638916)(861.36523775,570.57639178)
\curveto(861.34523709,570.61638901)(861.3352371,570.64638898)(861.33523775,570.66639178)
\curveto(861.34523709,570.69638893)(861.34523709,570.73138889)(861.33523775,570.77139178)
\curveto(861.31523712,570.85138877)(861.30023714,570.93138869)(861.29023775,571.01139178)
\curveto(861.29023715,571.10138852)(861.28023716,571.18638844)(861.26023775,571.26639178)
\lineto(861.26023775,571.38639178)
\curveto(861.26023718,571.4263882)(861.25523718,571.47138815)(861.24523775,571.52139178)
\curveto(861.2352372,571.57138805)(861.23023721,571.65638797)(861.23023775,571.77639178)
\curveto(861.23023721,571.90638772)(861.2402372,572.00138762)(861.26023775,572.06139178)
\curveto(861.28023716,572.13138749)(861.28523715,572.20138742)(861.27523775,572.27139178)
\curveto(861.26523717,572.34138728)(861.27023717,572.41138721)(861.29023775,572.48139178)
\curveto(861.30023714,572.53138709)(861.30523713,572.57138705)(861.30523775,572.60139178)
\curveto(861.31523712,572.64138698)(861.32523711,572.68638694)(861.33523775,572.73639178)
\curveto(861.36523707,572.85638677)(861.39023705,572.97638665)(861.41023775,573.09639178)
\curveto(861.440237,573.21638641)(861.48023696,573.33138629)(861.53023775,573.44139178)
\curveto(861.68023676,573.81138581)(861.86023658,574.14138548)(862.07023775,574.43139178)
\curveto(862.29023615,574.73138489)(862.55523588,574.98138464)(862.86523775,575.18139178)
\curveto(862.98523545,575.26138436)(863.11023533,575.3263843)(863.24023775,575.37639178)
\curveto(863.37023507,575.43638419)(863.50523493,575.49638413)(863.64523775,575.55639178)
\curveto(863.76523467,575.60638402)(863.89523454,575.63638399)(864.03523775,575.64639178)
\curveto(864.17523426,575.66638396)(864.31523412,575.69638393)(864.45523775,575.73639178)
\lineto(864.65023775,575.73639178)
\curveto(864.72023372,575.74638388)(864.78523365,575.75638387)(864.84523775,575.76639178)
\curveto(865.7352327,575.77638385)(866.47523196,575.59138403)(867.06523775,575.21139178)
\curveto(867.65523078,574.83138479)(868.08023036,574.33638529)(868.34023775,573.72639178)
\curveto(868.39023005,573.626386)(868.43023001,573.5263861)(868.46023775,573.42639178)
\curveto(868.49022995,573.3263863)(868.52522991,573.2213864)(868.56523775,573.11139178)
\curveto(868.59522984,573.00138662)(868.62022982,572.88138674)(868.64023775,572.75139178)
\curveto(868.66022978,572.63138699)(868.68522975,572.50638712)(868.71523775,572.37639178)
\curveto(868.72522971,572.3263873)(868.72522971,572.27138735)(868.71523775,572.21139178)
\curveto(868.71522972,572.16138746)(868.72022972,572.11138751)(868.73023775,572.06139178)
\moveto(867.39523775,571.20639178)
\curveto(867.41523102,571.27638835)(867.42023102,571.35638827)(867.41023775,571.44639178)
\lineto(867.41023775,571.70139178)
\curveto(867.41023103,572.09138753)(867.37523106,572.4213872)(867.30523775,572.69139178)
\curveto(867.27523116,572.77138685)(867.25023119,572.85138677)(867.23023775,572.93139178)
\curveto(867.21023123,573.01138661)(867.18523125,573.08638654)(867.15523775,573.15639178)
\curveto(866.87523156,573.80638582)(866.43023201,574.25638537)(865.82023775,574.50639178)
\curveto(865.75023269,574.53638509)(865.67523276,574.55638507)(865.59523775,574.56639178)
\lineto(865.35523775,574.62639178)
\curveto(865.27523316,574.64638498)(865.19023325,574.65638497)(865.10023775,574.65639178)
\lineto(864.83023775,574.65639178)
\lineto(864.56023775,574.61139178)
\curveto(864.46023398,574.59138503)(864.36523407,574.56638506)(864.27523775,574.53639178)
\curveto(864.19523424,574.51638511)(864.11523432,574.48638514)(864.03523775,574.44639178)
\curveto(863.96523447,574.4263852)(863.90023454,574.39638523)(863.84023775,574.35639178)
\curveto(863.78023466,574.31638531)(863.72523471,574.27638535)(863.67523775,574.23639178)
\curveto(863.435235,574.06638556)(863.2402352,573.86138576)(863.09023775,573.62139178)
\curveto(862.9402355,573.38138624)(862.81023563,573.10138652)(862.70023775,572.78139178)
\curveto(862.67023577,572.68138694)(862.65023579,572.57638705)(862.64023775,572.46639178)
\curveto(862.63023581,572.36638726)(862.61523582,572.26138736)(862.59523775,572.15139178)
\curveto(862.58523585,572.11138751)(862.58023586,572.04638758)(862.58023775,571.95639178)
\curveto(862.57023587,571.9263877)(862.56523587,571.89138773)(862.56523775,571.85139178)
\curveto(862.57523586,571.81138781)(862.58023586,571.76638786)(862.58023775,571.71639178)
\lineto(862.58023775,571.41639178)
\curveto(862.58023586,571.31638831)(862.59023585,571.2263884)(862.61023775,571.14639178)
\lineto(862.64023775,570.96639178)
\curveto(862.66023578,570.86638876)(862.67523576,570.76638886)(862.68523775,570.66639178)
\curveto(862.70523573,570.57638905)(862.7352357,570.49138913)(862.77523775,570.41139178)
\curveto(862.87523556,570.17138945)(862.99023545,569.94638968)(863.12023775,569.73639178)
\curveto(863.26023518,569.5263901)(863.43023501,569.35139027)(863.63023775,569.21139178)
\curveto(863.68023476,569.18139044)(863.72523471,569.15639047)(863.76523775,569.13639178)
\curveto(863.80523463,569.11639051)(863.85023459,569.09139053)(863.90023775,569.06139178)
\curveto(863.98023446,569.01139061)(864.06523437,568.96639066)(864.15523775,568.92639178)
\curveto(864.25523418,568.89639073)(864.36023408,568.86639076)(864.47023775,568.83639178)
\curveto(864.52023392,568.81639081)(864.56523387,568.80639082)(864.60523775,568.80639178)
\curveto(864.65523378,568.81639081)(864.70523373,568.81639081)(864.75523775,568.80639178)
\curveto(864.78523365,568.79639083)(864.84523359,568.78639084)(864.93523775,568.77639178)
\curveto(865.0352334,568.76639086)(865.11023333,568.77139085)(865.16023775,568.79139178)
\curveto(865.20023324,568.80139082)(865.2402332,568.80139082)(865.28023775,568.79139178)
\curveto(865.32023312,568.79139083)(865.36023308,568.80139082)(865.40023775,568.82139178)
\curveto(865.48023296,568.84139078)(865.56023288,568.85639077)(865.64023775,568.86639178)
\curveto(865.72023272,568.88639074)(865.79523264,568.91139071)(865.86523775,568.94139178)
\curveto(866.20523223,569.08139054)(866.48023196,569.27639035)(866.69023775,569.52639178)
\curveto(866.90023154,569.77638985)(867.07523136,570.07138955)(867.21523775,570.41139178)
\curveto(867.26523117,570.53138909)(867.29523114,570.65638897)(867.30523775,570.78639178)
\curveto(867.32523111,570.9263887)(867.35523108,571.06638856)(867.39523775,571.20639178)
}
}
{
\newrgbcolor{curcolor}{0 0 0}
\pscustom[linestyle=none,fillstyle=solid,fillcolor=curcolor]
{
\newpath
\moveto(872.648519,575.76639178)
\curveto(873.36851493,575.77638385)(873.97351433,575.69138393)(874.463519,575.51139178)
\curveto(874.95351335,575.34138428)(875.33351297,575.03638459)(875.603519,574.59639178)
\curveto(875.67351263,574.48638514)(875.72851257,574.37138525)(875.768519,574.25139178)
\curveto(875.80851249,574.14138548)(875.84851245,574.01638561)(875.888519,573.87639178)
\curveto(875.90851239,573.80638582)(875.91351239,573.73138589)(875.903519,573.65139178)
\curveto(875.89351241,573.58138604)(875.87851242,573.5263861)(875.858519,573.48639178)
\curveto(875.83851246,573.46638616)(875.81351249,573.44638618)(875.783519,573.42639178)
\curveto(875.75351255,573.41638621)(875.72851257,573.40138622)(875.708519,573.38139178)
\curveto(875.65851264,573.36138626)(875.60851269,573.35638627)(875.558519,573.36639178)
\curveto(875.50851279,573.37638625)(875.45851284,573.37638625)(875.408519,573.36639178)
\curveto(875.32851297,573.34638628)(875.22351308,573.34138628)(875.093519,573.35139178)
\curveto(874.96351334,573.37138625)(874.87351343,573.39638623)(874.823519,573.42639178)
\curveto(874.74351356,573.47638615)(874.68851361,573.54138608)(874.658519,573.62139178)
\curveto(874.63851366,573.71138591)(874.6035137,573.79638583)(874.553519,573.87639178)
\curveto(874.46351384,574.03638559)(874.33851396,574.18138544)(874.178519,574.31139178)
\curveto(874.06851423,574.39138523)(873.94851435,574.45138517)(873.818519,574.49139178)
\curveto(873.68851461,574.53138509)(873.54851475,574.57138505)(873.398519,574.61139178)
\curveto(873.34851495,574.63138499)(873.298515,574.63638499)(873.248519,574.62639178)
\curveto(873.1985151,574.626385)(873.14851515,574.63138499)(873.098519,574.64139178)
\curveto(873.03851526,574.66138496)(872.96351534,574.67138495)(872.873519,574.67139178)
\curveto(872.78351552,574.67138495)(872.70851559,574.66138496)(872.648519,574.64139178)
\lineto(872.558519,574.64139178)
\lineto(872.408519,574.61139178)
\curveto(872.35851594,574.61138501)(872.30851599,574.60638502)(872.258519,574.59639178)
\curveto(871.9985163,574.53638509)(871.78351652,574.45138517)(871.613519,574.34139178)
\curveto(871.44351686,574.23138539)(871.32851697,574.04638558)(871.268519,573.78639178)
\curveto(871.24851705,573.71638591)(871.24351706,573.64638598)(871.253519,573.57639178)
\curveto(871.27351703,573.50638612)(871.29351701,573.44638618)(871.313519,573.39639178)
\curveto(871.37351693,573.24638638)(871.44351686,573.13638649)(871.523519,573.06639178)
\curveto(871.61351669,573.00638662)(871.72351658,572.93638669)(871.853519,572.85639178)
\curveto(872.01351629,572.75638687)(872.19351611,572.68138694)(872.393519,572.63139178)
\curveto(872.59351571,572.59138703)(872.79351551,572.54138708)(872.993519,572.48139178)
\curveto(873.12351518,572.44138718)(873.25351505,572.41138721)(873.383519,572.39139178)
\curveto(873.51351479,572.37138725)(873.64351466,572.34138728)(873.773519,572.30139178)
\curveto(873.98351432,572.24138738)(874.18851411,572.18138744)(874.388519,572.12139178)
\curveto(874.58851371,572.07138755)(874.78851351,572.00638762)(874.988519,571.92639178)
\lineto(875.138519,571.86639178)
\curveto(875.18851311,571.84638778)(875.23851306,571.8213878)(875.288519,571.79139178)
\curveto(875.48851281,571.67138795)(875.66351264,571.53638809)(875.813519,571.38639178)
\curveto(875.96351234,571.23638839)(876.08851221,571.04638858)(876.188519,570.81639178)
\curveto(876.20851209,570.74638888)(876.22851207,570.65138897)(876.248519,570.53139178)
\curveto(876.26851203,570.46138916)(876.27851202,570.38638924)(876.278519,570.30639178)
\curveto(876.28851201,570.23638939)(876.29351201,570.15638947)(876.293519,570.06639178)
\lineto(876.293519,569.91639178)
\curveto(876.27351203,569.84638978)(876.26351204,569.77638985)(876.263519,569.70639178)
\curveto(876.26351204,569.63638999)(876.25351205,569.56639006)(876.233519,569.49639178)
\curveto(876.2035121,569.38639024)(876.16851213,569.28139034)(876.128519,569.18139178)
\curveto(876.08851221,569.08139054)(876.04351226,568.99139063)(875.993519,568.91139178)
\curveto(875.83351247,568.65139097)(875.62851267,568.44139118)(875.378519,568.28139178)
\curveto(875.12851317,568.13139149)(874.84851345,568.00139162)(874.538519,567.89139178)
\curveto(874.44851385,567.86139176)(874.35351395,567.84139178)(874.253519,567.83139178)
\curveto(874.16351414,567.81139181)(874.07351423,567.78639184)(873.983519,567.75639178)
\curveto(873.88351442,567.73639189)(873.78351452,567.7263919)(873.683519,567.72639178)
\curveto(873.58351472,567.7263919)(873.48351482,567.71639191)(873.383519,567.69639178)
\lineto(873.233519,567.69639178)
\curveto(873.18351512,567.68639194)(873.11351519,567.68139194)(873.023519,567.68139178)
\curveto(872.93351537,567.68139194)(872.86351544,567.68639194)(872.813519,567.69639178)
\lineto(872.648519,567.69639178)
\curveto(872.58851571,567.71639191)(872.52351578,567.7263919)(872.453519,567.72639178)
\curveto(872.38351592,567.71639191)(872.32351598,567.7213919)(872.273519,567.74139178)
\curveto(872.22351608,567.75139187)(872.15851614,567.75639187)(872.078519,567.75639178)
\lineto(871.838519,567.81639178)
\curveto(871.76851653,567.8263918)(871.69351661,567.84639178)(871.613519,567.87639178)
\curveto(871.303517,567.97639165)(871.03351727,568.10139152)(870.803519,568.25139178)
\curveto(870.57351773,568.40139122)(870.37351793,568.59639103)(870.203519,568.83639178)
\curveto(870.11351819,568.96639066)(870.03851826,569.10139052)(869.978519,569.24139178)
\curveto(869.91851838,569.38139024)(869.86351844,569.53639009)(869.813519,569.70639178)
\curveto(869.79351851,569.76638986)(869.78351852,569.83638979)(869.783519,569.91639178)
\curveto(869.79351851,570.00638962)(869.80851849,570.07638955)(869.828519,570.12639178)
\curveto(869.85851844,570.16638946)(869.90851839,570.20638942)(869.978519,570.24639178)
\curveto(870.02851827,570.26638936)(870.0985182,570.27638935)(870.188519,570.27639178)
\curveto(870.27851802,570.28638934)(870.36851793,570.28638934)(870.458519,570.27639178)
\curveto(870.54851775,570.26638936)(870.63351767,570.25138937)(870.713519,570.23139178)
\curveto(870.8035175,570.2213894)(870.86351744,570.20638942)(870.893519,570.18639178)
\curveto(870.96351734,570.13638949)(871.00851729,570.06138956)(871.028519,569.96139178)
\curveto(871.05851724,569.87138975)(871.09351721,569.78638984)(871.133519,569.70639178)
\curveto(871.23351707,569.48639014)(871.36851693,569.31639031)(871.538519,569.19639178)
\curveto(871.65851664,569.10639052)(871.79351651,569.03639059)(871.943519,568.98639178)
\curveto(872.09351621,568.93639069)(872.25351605,568.88639074)(872.423519,568.83639178)
\lineto(872.738519,568.79139178)
\lineto(872.828519,568.79139178)
\curveto(872.8985154,568.77139085)(872.98851531,568.76139086)(873.098519,568.76139178)
\curveto(873.21851508,568.76139086)(873.31851498,568.77139085)(873.398519,568.79139178)
\curveto(873.46851483,568.79139083)(873.52351478,568.79639083)(873.563519,568.80639178)
\curveto(873.62351468,568.81639081)(873.68351462,568.8213908)(873.743519,568.82139178)
\curveto(873.8035145,568.83139079)(873.85851444,568.84139078)(873.908519,568.85139178)
\curveto(874.1985141,568.93139069)(874.42851387,569.03639059)(874.598519,569.16639178)
\curveto(874.76851353,569.29639033)(874.88851341,569.51639011)(874.958519,569.82639178)
\curveto(874.97851332,569.87638975)(874.98351332,569.93138969)(874.973519,569.99139178)
\curveto(874.96351334,570.05138957)(874.95351335,570.09638953)(874.943519,570.12639178)
\curveto(874.89351341,570.31638931)(874.82351348,570.45638917)(874.733519,570.54639178)
\curveto(874.64351366,570.64638898)(874.52851377,570.73638889)(874.388519,570.81639178)
\curveto(874.298514,570.87638875)(874.1985141,570.9263887)(874.088519,570.96639178)
\lineto(873.758519,571.08639178)
\curveto(873.72851457,571.09638853)(873.6985146,571.10138852)(873.668519,571.10139178)
\curveto(873.64851465,571.10138852)(873.62351468,571.11138851)(873.593519,571.13139178)
\curveto(873.25351505,571.24138838)(872.8985154,571.3213883)(872.528519,571.37139178)
\curveto(872.16851613,571.43138819)(871.82851647,571.5263881)(871.508519,571.65639178)
\curveto(871.40851689,571.69638793)(871.31351699,571.73138789)(871.223519,571.76139178)
\curveto(871.13351717,571.79138783)(871.04851725,571.83138779)(870.968519,571.88139178)
\curveto(870.77851752,571.99138763)(870.6035177,572.11638751)(870.443519,572.25639178)
\curveto(870.28351802,572.39638723)(870.15851814,572.57138705)(870.068519,572.78139178)
\curveto(870.03851826,572.85138677)(870.01351829,572.9213867)(869.993519,572.99139178)
\curveto(869.98351832,573.06138656)(869.96851833,573.13638649)(869.948519,573.21639178)
\curveto(869.91851838,573.33638629)(869.90851839,573.47138615)(869.918519,573.62139178)
\curveto(869.92851837,573.78138584)(869.94351836,573.91638571)(869.963519,574.02639178)
\curveto(869.98351832,574.07638555)(869.99351831,574.11638551)(869.993519,574.14639178)
\curveto(870.0035183,574.18638544)(870.01851828,574.2263854)(870.038519,574.26639178)
\curveto(870.12851817,574.49638513)(870.24851805,574.69638493)(870.398519,574.86639178)
\curveto(870.55851774,575.03638459)(870.73851756,575.18638444)(870.938519,575.31639178)
\curveto(871.08851721,575.40638422)(871.25351705,575.47638415)(871.433519,575.52639178)
\curveto(871.61351669,575.58638404)(871.8035165,575.64138398)(872.003519,575.69139178)
\curveto(872.07351623,575.70138392)(872.13851616,575.71138391)(872.198519,575.72139178)
\curveto(872.26851603,575.73138389)(872.34351596,575.74138388)(872.423519,575.75139178)
\curveto(872.45351585,575.76138386)(872.49351581,575.76138386)(872.543519,575.75139178)
\curveto(872.59351571,575.74138388)(872.62851567,575.74638388)(872.648519,575.76639178)
}
}
{
\newrgbcolor{curcolor}{0.40000001 0.40000001 0.40000001}
\pscustom[linestyle=none,fillstyle=solid,fillcolor=curcolor]
{
\newpath
\moveto(812.80437349,578.5714284)
\lineto(827.80437349,578.5714284)
\lineto(827.80437349,563.5714284)
\lineto(812.80437349,563.5714284)
\closepath
}
}
{
\newrgbcolor{curcolor}{0 0 0}
\pscustom[linestyle=none,fillstyle=solid,fillcolor=curcolor]
{
\newpath
\moveto(415.20953951,703.11942034)
\curveto(415.20952902,703.08941467)(415.20952902,703.04941471)(415.20953951,702.99942034)
\curveto(415.21952901,702.94941481)(415.22452901,702.89441487)(415.22453951,702.83442034)
\curveto(415.22452901,702.77441499)(415.21952901,702.71941504)(415.20953951,702.66942034)
\curveto(415.20952902,702.61941514)(415.20952902,702.58441518)(415.20953951,702.56442034)
\curveto(415.20952902,702.49441527)(415.20452903,702.42441534)(415.19453951,702.35442034)
\curveto(415.19452904,702.29441547)(415.19452904,702.23441553)(415.19453951,702.17442034)
\curveto(415.17452906,702.12441564)(415.16452907,702.07441569)(415.16453951,702.02442034)
\curveto(415.17452906,701.97441579)(415.17452906,701.92441584)(415.16453951,701.87442034)
\curveto(415.14452909,701.764416)(415.1295291,701.65441611)(415.11953951,701.54442034)
\curveto(415.10952912,701.43441633)(415.08952914,701.32441644)(415.05953951,701.21442034)
\curveto(415.00952922,701.04441672)(414.96452927,700.87941688)(414.92453951,700.71942034)
\curveto(414.88452935,700.56941719)(414.8345294,700.41941734)(414.77453951,700.26942034)
\curveto(414.60452963,699.84941791)(414.39452984,699.46941829)(414.14453951,699.12942034)
\curveto(413.89453034,698.78941897)(413.59453064,698.49941926)(413.24453951,698.25942034)
\curveto(413.04453119,698.11941964)(412.8345314,697.99941976)(412.61453951,697.89942034)
\curveto(412.40453183,697.79941996)(412.17453206,697.70942005)(411.92453951,697.62942034)
\curveto(411.82453241,697.59942016)(411.71953251,697.57442019)(411.60953951,697.55442034)
\curveto(411.50953272,697.54442022)(411.40453283,697.52442024)(411.29453951,697.49442034)
\curveto(411.24453299,697.48442028)(411.19453304,697.47942028)(411.14453951,697.47942034)
\curveto(411.10453313,697.47942028)(411.05953317,697.47442029)(411.00953951,697.46442034)
\curveto(410.96953326,697.45442031)(410.9295333,697.44942031)(410.88953951,697.44942034)
\curveto(410.84953338,697.4594203)(410.80453343,697.4594203)(410.75453951,697.44942034)
\curveto(410.7345335,697.43942032)(410.70453353,697.43442033)(410.66453951,697.43442034)
\curveto(410.62453361,697.44442032)(410.59453364,697.44442032)(410.57453951,697.43442034)
\curveto(410.49453374,697.41442035)(410.39453384,697.40942035)(410.27453951,697.41942034)
\curveto(410.15453408,697.42942033)(410.04953418,697.43442033)(409.95953951,697.43442034)
\lineto(406.46453951,697.43442034)
\curveto(406.29453794,697.43442033)(406.14953808,697.43942032)(406.02953951,697.44942034)
\curveto(405.91953831,697.46942029)(405.83953839,697.53942022)(405.78953951,697.65942034)
\curveto(405.75953847,697.73942002)(405.74453849,697.8594199)(405.74453951,698.01942034)
\curveto(405.75453848,698.18941957)(405.75953847,698.32941943)(405.75953951,698.43942034)
\lineto(405.75953951,707.24442034)
\curveto(405.75953847,707.3644104)(405.75453848,707.48941027)(405.74453951,707.61942034)
\curveto(405.74453849,707.75941)(405.76953846,707.86940989)(405.81953951,707.94942034)
\curveto(405.85953837,708.00940975)(405.9345383,708.0594097)(406.04453951,708.09942034)
\curveto(406.06453817,708.10940965)(406.08453815,708.10940965)(406.10453951,708.09942034)
\curveto(406.12453811,708.09940966)(406.14453809,708.10440966)(406.16453951,708.11442034)
\lineto(410.19953951,708.11442034)
\curveto(410.25953397,708.11440965)(410.31953391,708.11440965)(410.37953951,708.11442034)
\curveto(410.44953378,708.12440964)(410.50953372,708.12440964)(410.55953951,708.11442034)
\lineto(410.73953951,708.11442034)
\curveto(410.78953344,708.09440967)(410.84453339,708.08440968)(410.90453951,708.08442034)
\curveto(410.96453327,708.09440967)(411.01953321,708.08940967)(411.06953951,708.06942034)
\curveto(411.1295331,708.04940971)(411.18453305,708.03940972)(411.23453951,708.03942034)
\curveto(411.29453294,708.04940971)(411.35453288,708.04440972)(411.41453951,708.02442034)
\curveto(411.55453268,707.99440977)(411.68953254,707.9644098)(411.81953951,707.93442034)
\curveto(411.94953228,707.91440985)(412.07453216,707.87940988)(412.19453951,707.82942034)
\curveto(412.30453193,707.77940998)(412.41453182,707.73441003)(412.52453951,707.69442034)
\curveto(412.6345316,707.65441011)(412.73953149,707.60441016)(412.83953951,707.54442034)
\curveto(413.08953114,707.38441038)(413.31953091,707.22941053)(413.52953951,707.07942034)
\lineto(413.61953951,706.98942034)
\curveto(413.71953051,706.90941085)(413.80953042,706.81941094)(413.88953951,706.71942034)
\lineto(414.02453951,706.59942034)
\curveto(414.07453016,706.51941124)(414.1295301,706.43941132)(414.18953951,706.35942034)
\curveto(414.25952997,706.28941147)(414.31952991,706.21441155)(414.36953951,706.13442034)
\curveto(414.49952973,705.92441184)(414.61452962,705.69941206)(414.71453951,705.45942034)
\curveto(414.81452942,705.22941253)(414.90452933,704.98441278)(414.98453951,704.72442034)
\curveto(415.0345292,704.59441317)(415.06452917,704.4594133)(415.07453951,704.31942034)
\curveto(415.09452914,704.17941358)(415.11952911,704.03941372)(415.14953951,703.89942034)
\curveto(415.14952908,703.84941391)(415.14952908,703.80441396)(415.14953951,703.76442034)
\curveto(415.15952907,703.73441403)(415.16452907,703.69941406)(415.16453951,703.65942034)
\curveto(415.18452905,703.59941416)(415.18952904,703.53441423)(415.17953951,703.46442034)
\curveto(415.17952905,703.39441437)(415.18952904,703.33441443)(415.20953951,703.28442034)
\lineto(415.20953951,703.11942034)
\moveto(412.86953951,702.39942034)
\curveto(412.88953134,702.44941531)(412.89953133,702.52941523)(412.89953951,702.63942034)
\curveto(412.89953133,702.74941501)(412.88953134,702.82941493)(412.86953951,702.87942034)
\lineto(412.86953951,703.16442034)
\curveto(412.84953138,703.25441451)(412.8345314,703.34941441)(412.82453951,703.44942034)
\curveto(412.82453141,703.54941421)(412.81453142,703.63941412)(412.79453951,703.71942034)
\curveto(412.77453146,703.76941399)(412.76453147,703.81441395)(412.76453951,703.85442034)
\curveto(412.77453146,703.90441386)(412.76953146,703.95441381)(412.74953951,704.00442034)
\curveto(412.69953153,704.1644136)(412.64953158,704.31441345)(412.59953951,704.45442034)
\curveto(412.55953167,704.60441316)(412.49953173,704.74441302)(412.41953951,704.87442034)
\curveto(412.26953196,705.11441265)(412.09453214,705.31941244)(411.89453951,705.48942034)
\curveto(411.70453253,705.66941209)(411.46953276,705.81941194)(411.18953951,705.93942034)
\curveto(411.09953313,705.96941179)(411.00953322,705.99441177)(410.91953951,706.01442034)
\curveto(410.8295334,706.04441172)(410.73953349,706.06941169)(410.64953951,706.08942034)
\curveto(410.56953366,706.09941166)(410.49453374,706.10441166)(410.42453951,706.10442034)
\curveto(410.36453387,706.11441165)(410.29453394,706.12941163)(410.21453951,706.14942034)
\curveto(410.17453406,706.1594116)(410.1345341,706.1594116)(410.09453951,706.14942034)
\curveto(410.05453418,706.14941161)(410.01953421,706.15441161)(409.98953951,706.16442034)
\lineto(409.65953951,706.16442034)
\curveto(409.60953462,706.17441159)(409.55453468,706.17441159)(409.49453951,706.16442034)
\lineto(409.31453951,706.16442034)
\lineto(408.63953951,706.16442034)
\curveto(408.61953561,706.14441162)(408.58453565,706.13941162)(408.53453951,706.14942034)
\curveto(408.49453574,706.1594116)(408.45953577,706.1594116)(408.42953951,706.14942034)
\lineto(408.27953951,706.08942034)
\curveto(408.229536,706.07941168)(408.18953604,706.04941171)(408.15953951,705.99942034)
\curveto(408.11953611,705.94941181)(408.09953613,705.87941188)(408.09953951,705.78942034)
\lineto(408.09953951,705.48942034)
\curveto(408.09953613,705.3594124)(408.09453614,705.22441254)(408.08453951,705.08442034)
\lineto(408.08453951,704.66442034)
\lineto(408.08453951,700.47942034)
\curveto(408.08453615,700.41941734)(408.07953615,700.35441741)(408.06953951,700.28442034)
\curveto(408.06953616,700.21441755)(408.07953615,700.15441761)(408.09953951,700.10442034)
\lineto(408.09953951,699.95442034)
\lineto(408.09953951,699.74442034)
\curveto(408.10953612,699.68441808)(408.12453611,699.62941813)(408.14453951,699.57942034)
\curveto(408.20453603,699.4594183)(408.31953591,699.39441837)(408.48953951,699.38442034)
\lineto(409.01453951,699.38442034)
\lineto(410.19953951,699.38442034)
\curveto(410.59953363,699.39441837)(410.93953329,699.45441831)(411.21953951,699.56442034)
\curveto(411.58953264,699.71441805)(411.87953235,699.91441785)(412.08953951,700.16442034)
\curveto(412.30953192,700.41441735)(412.49453174,700.72441704)(412.64453951,701.09442034)
\curveto(412.68453155,701.17441659)(412.71453152,701.2644165)(412.73453951,701.36442034)
\curveto(412.75453148,701.4644163)(412.77953145,701.5644162)(412.80953951,701.66442034)
\lineto(412.80953951,701.78442034)
\curveto(412.8295314,701.85441591)(412.83953139,701.92941583)(412.83953951,702.00942034)
\curveto(412.83953139,702.08941567)(412.84953138,702.16941559)(412.86953951,702.24942034)
\lineto(412.86953951,702.39942034)
}
}
{
\newrgbcolor{curcolor}{0 0 0}
\pscustom[linestyle=none,fillstyle=solid,fillcolor=curcolor]
{
\newpath
\moveto(418.70805513,708.00942034)
\curveto(418.77805218,707.92940983)(418.81305215,707.80940995)(418.81305513,707.64942034)
\lineto(418.81305513,707.18442034)
\lineto(418.81305513,706.77942034)
\curveto(418.81305215,706.63941112)(418.77805218,706.54441122)(418.70805513,706.49442034)
\curveto(418.64805231,706.44441132)(418.56805239,706.41441135)(418.46805513,706.40442034)
\curveto(418.37805258,706.39441137)(418.27805268,706.38941137)(418.16805513,706.38942034)
\lineto(417.32805513,706.38942034)
\curveto(417.21805374,706.38941137)(417.11805384,706.39441137)(417.02805513,706.40442034)
\curveto(416.94805401,706.41441135)(416.87805408,706.44441132)(416.81805513,706.49442034)
\curveto(416.77805418,706.52441124)(416.74805421,706.57941118)(416.72805513,706.65942034)
\curveto(416.71805424,706.74941101)(416.70805425,706.84441092)(416.69805513,706.94442034)
\lineto(416.69805513,707.27442034)
\curveto(416.70805425,707.38441038)(416.71305425,707.47941028)(416.71305513,707.55942034)
\lineto(416.71305513,707.76942034)
\curveto(416.72305424,707.83940992)(416.74305422,707.89940986)(416.77305513,707.94942034)
\curveto(416.79305417,707.98940977)(416.81805414,708.01940974)(416.84805513,708.03942034)
\lineto(416.96805513,708.09942034)
\curveto(416.98805397,708.09940966)(417.01305395,708.09940966)(417.04305513,708.09942034)
\curveto(417.07305389,708.10940965)(417.09805386,708.11440965)(417.11805513,708.11442034)
\lineto(418.21305513,708.11442034)
\curveto(418.31305265,708.11440965)(418.40805255,708.10940965)(418.49805513,708.09942034)
\curveto(418.58805237,708.08940967)(418.6580523,708.0594097)(418.70805513,708.00942034)
\moveto(418.81305513,698.24442034)
\curveto(418.81305215,698.04441972)(418.80805215,697.87441989)(418.79805513,697.73442034)
\curveto(418.78805217,697.59442017)(418.69805226,697.49942026)(418.52805513,697.44942034)
\curveto(418.46805249,697.42942033)(418.40305256,697.41942034)(418.33305513,697.41942034)
\curveto(418.2630527,697.42942033)(418.18805277,697.43442033)(418.10805513,697.43442034)
\lineto(417.26805513,697.43442034)
\curveto(417.17805378,697.43442033)(417.08805387,697.43942032)(416.99805513,697.44942034)
\curveto(416.91805404,697.4594203)(416.8580541,697.48942027)(416.81805513,697.53942034)
\curveto(416.7580542,697.60942015)(416.72305424,697.69442007)(416.71305513,697.79442034)
\lineto(416.71305513,698.13942034)
\lineto(416.71305513,704.46942034)
\lineto(416.71305513,704.76942034)
\curveto(416.71305425,704.86941289)(416.73305423,704.94941281)(416.77305513,705.00942034)
\curveto(416.83305413,705.07941268)(416.91805404,705.12441264)(417.02805513,705.14442034)
\curveto(417.04805391,705.15441261)(417.07305389,705.15441261)(417.10305513,705.14442034)
\curveto(417.14305382,705.14441262)(417.17305379,705.14941261)(417.19305513,705.15942034)
\lineto(417.94305513,705.15942034)
\lineto(418.13805513,705.15942034)
\curveto(418.21805274,705.16941259)(418.28305268,705.16941259)(418.33305513,705.15942034)
\lineto(418.45305513,705.15942034)
\curveto(418.51305245,705.13941262)(418.56805239,705.12441264)(418.61805513,705.11442034)
\curveto(418.66805229,705.10441266)(418.70805225,705.07441269)(418.73805513,705.02442034)
\curveto(418.77805218,704.97441279)(418.79805216,704.90441286)(418.79805513,704.81442034)
\curveto(418.80805215,704.72441304)(418.81305215,704.62941313)(418.81305513,704.52942034)
\lineto(418.81305513,698.24442034)
}
}
{
\newrgbcolor{curcolor}{0 0 0}
\pscustom[linestyle=none,fillstyle=solid,fillcolor=curcolor]
{
\newpath
\moveto(423.44524263,705.36942034)
\curveto(424.19523813,705.38941237)(424.84523748,705.30441246)(425.39524263,705.11442034)
\curveto(425.95523637,704.93441283)(426.38023595,704.61941314)(426.67024263,704.16942034)
\curveto(426.74023559,704.0594137)(426.80023553,703.94441382)(426.85024263,703.82442034)
\curveto(426.91023542,703.71441405)(426.96023537,703.58941417)(427.00024263,703.44942034)
\curveto(427.02023531,703.38941437)(427.0302353,703.32441444)(427.03024263,703.25442034)
\curveto(427.0302353,703.18441458)(427.02023531,703.12441464)(427.00024263,703.07442034)
\curveto(426.96023537,703.01441475)(426.90523542,702.97441479)(426.83524263,702.95442034)
\curveto(426.78523554,702.93441483)(426.7252356,702.92441484)(426.65524263,702.92442034)
\lineto(426.44524263,702.92442034)
\lineto(425.78524263,702.92442034)
\curveto(425.71523661,702.92441484)(425.64523668,702.91941484)(425.57524263,702.90942034)
\curveto(425.50523682,702.90941485)(425.44023689,702.91941484)(425.38024263,702.93942034)
\curveto(425.28023705,702.9594148)(425.20523712,702.99941476)(425.15524263,703.05942034)
\curveto(425.10523722,703.11941464)(425.06023727,703.17941458)(425.02024263,703.23942034)
\lineto(424.90024263,703.44942034)
\curveto(424.87023746,703.52941423)(424.82023751,703.59441417)(424.75024263,703.64442034)
\curveto(424.65023768,703.72441404)(424.55023778,703.78441398)(424.45024263,703.82442034)
\curveto(424.36023797,703.8644139)(424.24523808,703.89941386)(424.10524263,703.92942034)
\curveto(424.03523829,703.94941381)(423.9302384,703.9644138)(423.79024263,703.97442034)
\curveto(423.66023867,703.98441378)(423.56023877,703.97941378)(423.49024263,703.95942034)
\lineto(423.38524263,703.95942034)
\lineto(423.23524263,703.92942034)
\curveto(423.19523913,703.92941383)(423.15023918,703.92441384)(423.10024263,703.91442034)
\curveto(422.9302394,703.8644139)(422.79023954,703.79441397)(422.68024263,703.70442034)
\curveto(422.58023975,703.62441414)(422.51023982,703.49941426)(422.47024263,703.32942034)
\curveto(422.45023988,703.2594145)(422.45023988,703.19441457)(422.47024263,703.13442034)
\curveto(422.49023984,703.07441469)(422.51023982,703.02441474)(422.53024263,702.98442034)
\curveto(422.60023973,702.8644149)(422.68023965,702.76941499)(422.77024263,702.69942034)
\curveto(422.87023946,702.62941513)(422.98523934,702.56941519)(423.11524263,702.51942034)
\curveto(423.30523902,702.43941532)(423.51023882,702.36941539)(423.73024263,702.30942034)
\lineto(424.42024263,702.15942034)
\curveto(424.66023767,702.11941564)(424.89023744,702.06941569)(425.11024263,702.00942034)
\curveto(425.34023699,701.9594158)(425.55523677,701.89441587)(425.75524263,701.81442034)
\curveto(425.84523648,701.77441599)(425.9302364,701.73941602)(426.01024263,701.70942034)
\curveto(426.10023623,701.68941607)(426.18523614,701.65441611)(426.26524263,701.60442034)
\curveto(426.45523587,701.48441628)(426.6252357,701.35441641)(426.77524263,701.21442034)
\curveto(426.93523539,701.07441669)(427.06023527,700.89941686)(427.15024263,700.68942034)
\curveto(427.18023515,700.61941714)(427.20523512,700.54941721)(427.22524263,700.47942034)
\curveto(427.24523508,700.40941735)(427.26523506,700.33441743)(427.28524263,700.25442034)
\curveto(427.29523503,700.19441757)(427.30023503,700.09941766)(427.30024263,699.96942034)
\curveto(427.31023502,699.84941791)(427.31023502,699.75441801)(427.30024263,699.68442034)
\lineto(427.30024263,699.60942034)
\curveto(427.28023505,699.54941821)(427.26523506,699.48941827)(427.25524263,699.42942034)
\curveto(427.25523507,699.37941838)(427.25023508,699.32941843)(427.24024263,699.27942034)
\curveto(427.17023516,698.97941878)(427.06023527,698.71441905)(426.91024263,698.48442034)
\curveto(426.75023558,698.24441952)(426.55523577,698.04941971)(426.32524263,697.89942034)
\curveto(426.09523623,697.74942001)(425.83523649,697.61942014)(425.54524263,697.50942034)
\curveto(425.43523689,697.4594203)(425.31523701,697.42442034)(425.18524263,697.40442034)
\curveto(425.06523726,697.38442038)(424.94523738,697.3594204)(424.82524263,697.32942034)
\curveto(424.73523759,697.30942045)(424.64023769,697.29942046)(424.54024263,697.29942034)
\curveto(424.45023788,697.28942047)(424.36023797,697.27442049)(424.27024263,697.25442034)
\lineto(424.00024263,697.25442034)
\curveto(423.94023839,697.23442053)(423.83523849,697.22442054)(423.68524263,697.22442034)
\curveto(423.54523878,697.22442054)(423.44523888,697.23442053)(423.38524263,697.25442034)
\curveto(423.35523897,697.25442051)(423.32023901,697.2594205)(423.28024263,697.26942034)
\lineto(423.17524263,697.26942034)
\curveto(423.05523927,697.28942047)(422.93523939,697.30442046)(422.81524263,697.31442034)
\curveto(422.69523963,697.32442044)(422.58023975,697.34442042)(422.47024263,697.37442034)
\curveto(422.08024025,697.48442028)(421.73524059,697.60942015)(421.43524263,697.74942034)
\curveto(421.13524119,697.89941986)(420.88024145,698.11941964)(420.67024263,698.40942034)
\curveto(420.5302418,698.59941916)(420.41024192,698.81941894)(420.31024263,699.06942034)
\curveto(420.29024204,699.12941863)(420.27024206,699.20941855)(420.25024263,699.30942034)
\curveto(420.2302421,699.3594184)(420.21524211,699.42941833)(420.20524263,699.51942034)
\curveto(420.19524213,699.60941815)(420.20024213,699.68441808)(420.22024263,699.74442034)
\curveto(420.25024208,699.81441795)(420.30024203,699.8644179)(420.37024263,699.89442034)
\curveto(420.42024191,699.91441785)(420.48024185,699.92441784)(420.55024263,699.92442034)
\lineto(420.77524263,699.92442034)
\lineto(421.48024263,699.92442034)
\lineto(421.72024263,699.92442034)
\curveto(421.80024053,699.92441784)(421.87024046,699.91441785)(421.93024263,699.89442034)
\curveto(422.04024029,699.85441791)(422.11024022,699.78941797)(422.14024263,699.69942034)
\curveto(422.18024015,699.60941815)(422.2252401,699.51441825)(422.27524263,699.41442034)
\curveto(422.29524003,699.3644184)(422.33024,699.29941846)(422.38024263,699.21942034)
\curveto(422.44023989,699.13941862)(422.49023984,699.08941867)(422.53024263,699.06942034)
\curveto(422.65023968,698.96941879)(422.76523956,698.88941887)(422.87524263,698.82942034)
\curveto(422.98523934,698.77941898)(423.1252392,698.72941903)(423.29524263,698.67942034)
\curveto(423.34523898,698.6594191)(423.39523893,698.64941911)(423.44524263,698.64942034)
\curveto(423.49523883,698.6594191)(423.54523878,698.6594191)(423.59524263,698.64942034)
\curveto(423.67523865,698.62941913)(423.76023857,698.61941914)(423.85024263,698.61942034)
\curveto(423.95023838,698.62941913)(424.03523829,698.64441912)(424.10524263,698.66442034)
\curveto(424.15523817,698.67441909)(424.20023813,698.67941908)(424.24024263,698.67942034)
\curveto(424.29023804,698.67941908)(424.34023799,698.68941907)(424.39024263,698.70942034)
\curveto(424.5302378,698.759419)(424.65523767,698.81941894)(424.76524263,698.88942034)
\curveto(424.88523744,698.9594188)(424.98023735,699.04941871)(425.05024263,699.15942034)
\curveto(425.10023723,699.23941852)(425.14023719,699.3644184)(425.17024263,699.53442034)
\curveto(425.19023714,699.60441816)(425.19023714,699.66941809)(425.17024263,699.72942034)
\curveto(425.15023718,699.78941797)(425.1302372,699.83941792)(425.11024263,699.87942034)
\curveto(425.04023729,700.01941774)(424.95023738,700.12441764)(424.84024263,700.19442034)
\curveto(424.74023759,700.2644175)(424.62023771,700.32941743)(424.48024263,700.38942034)
\curveto(424.29023804,700.46941729)(424.09023824,700.53441723)(423.88024263,700.58442034)
\curveto(423.67023866,700.63441713)(423.46023887,700.68941707)(423.25024263,700.74942034)
\curveto(423.17023916,700.76941699)(423.08523924,700.78441698)(422.99524263,700.79442034)
\curveto(422.91523941,700.80441696)(422.83523949,700.81941694)(422.75524263,700.83942034)
\curveto(422.43523989,700.92941683)(422.1302402,701.01441675)(421.84024263,701.09442034)
\curveto(421.55024078,701.18441658)(421.28524104,701.31441645)(421.04524263,701.48442034)
\curveto(420.76524156,701.68441608)(420.56024177,701.95441581)(420.43024263,702.29442034)
\curveto(420.41024192,702.3644154)(420.39024194,702.4594153)(420.37024263,702.57942034)
\curveto(420.35024198,702.64941511)(420.33524199,702.73441503)(420.32524263,702.83442034)
\curveto(420.31524201,702.93441483)(420.32024201,703.02441474)(420.34024263,703.10442034)
\curveto(420.36024197,703.15441461)(420.36524196,703.19441457)(420.35524263,703.22442034)
\curveto(420.34524198,703.2644145)(420.35024198,703.30941445)(420.37024263,703.35942034)
\curveto(420.39024194,703.46941429)(420.41024192,703.56941419)(420.43024263,703.65942034)
\curveto(420.46024187,703.759414)(420.49524183,703.85441391)(420.53524263,703.94442034)
\curveto(420.66524166,704.23441353)(420.84524148,704.46941329)(421.07524263,704.64942034)
\curveto(421.30524102,704.82941293)(421.56524076,704.97441279)(421.85524263,705.08442034)
\curveto(421.96524036,705.13441263)(422.08024025,705.16941259)(422.20024263,705.18942034)
\curveto(422.32024001,705.21941254)(422.44523988,705.24941251)(422.57524263,705.27942034)
\curveto(422.63523969,705.29941246)(422.69523963,705.30941245)(422.75524263,705.30942034)
\lineto(422.93524263,705.33942034)
\curveto(423.01523931,705.34941241)(423.10023923,705.35441241)(423.19024263,705.35442034)
\curveto(423.28023905,705.35441241)(423.36523896,705.3594124)(423.44524263,705.36942034)
}
}
{
\newrgbcolor{curcolor}{0 0 0}
\pscustom[linestyle=none,fillstyle=solid,fillcolor=curcolor]
{
\newpath
\moveto(429.58188326,707.46942034)
\lineto(430.58688326,707.46942034)
\curveto(430.73688027,707.46941029)(430.86688014,707.4594103)(430.97688326,707.43942034)
\curveto(431.09687991,707.42941033)(431.18187983,707.36941039)(431.23188326,707.25942034)
\curveto(431.25187976,707.20941055)(431.26187975,707.14941061)(431.26188326,707.07942034)
\lineto(431.26188326,706.86942034)
\lineto(431.26188326,706.19442034)
\curveto(431.26187975,706.14441162)(431.25687975,706.08441168)(431.24688326,706.01442034)
\curveto(431.24687976,705.95441181)(431.25187976,705.89941186)(431.26188326,705.84942034)
\lineto(431.26188326,705.68442034)
\curveto(431.26187975,705.60441216)(431.26687974,705.52941223)(431.27688326,705.45942034)
\curveto(431.28687972,705.39941236)(431.3118797,705.34441242)(431.35188326,705.29442034)
\curveto(431.42187959,705.20441256)(431.54687946,705.15441261)(431.72688326,705.14442034)
\lineto(432.26688326,705.14442034)
\lineto(432.44688326,705.14442034)
\curveto(432.5068785,705.14441262)(432.56187845,705.13441263)(432.61188326,705.11442034)
\curveto(432.72187829,705.0644127)(432.78187823,704.97441279)(432.79188326,704.84442034)
\curveto(432.8118782,704.71441305)(432.82187819,704.56941319)(432.82188326,704.40942034)
\lineto(432.82188326,704.19942034)
\curveto(432.83187818,704.12941363)(432.82687818,704.06941369)(432.80688326,704.01942034)
\curveto(432.75687825,703.8594139)(432.65187836,703.77441399)(432.49188326,703.76442034)
\curveto(432.33187868,703.75441401)(432.15187886,703.74941401)(431.95188326,703.74942034)
\lineto(431.81688326,703.74942034)
\curveto(431.77687923,703.759414)(431.74187927,703.759414)(431.71188326,703.74942034)
\curveto(431.67187934,703.73941402)(431.63687937,703.73441403)(431.60688326,703.73442034)
\curveto(431.57687943,703.74441402)(431.54687946,703.73941402)(431.51688326,703.71942034)
\curveto(431.43687957,703.69941406)(431.37687963,703.65441411)(431.33688326,703.58442034)
\curveto(431.3068797,703.52441424)(431.28187973,703.44941431)(431.26188326,703.35942034)
\curveto(431.25187976,703.30941445)(431.25187976,703.25441451)(431.26188326,703.19442034)
\curveto(431.27187974,703.13441463)(431.27187974,703.07941468)(431.26188326,703.02942034)
\lineto(431.26188326,702.09942034)
\lineto(431.26188326,700.34442034)
\curveto(431.26187975,700.09441767)(431.26687974,699.87441789)(431.27688326,699.68442034)
\curveto(431.29687971,699.50441826)(431.36187965,699.34441842)(431.47188326,699.20442034)
\curveto(431.52187949,699.14441862)(431.58687942,699.09941866)(431.66688326,699.06942034)
\lineto(431.93688326,699.00942034)
\curveto(431.96687904,698.99941876)(431.99687901,698.99441877)(432.02688326,698.99442034)
\curveto(432.06687894,699.00441876)(432.09687891,699.00441876)(432.11688326,698.99442034)
\lineto(432.28188326,698.99442034)
\curveto(432.39187862,698.99441877)(432.48687852,698.98941877)(432.56688326,698.97942034)
\curveto(432.64687836,698.96941879)(432.7118783,698.92941883)(432.76188326,698.85942034)
\curveto(432.80187821,698.79941896)(432.82187819,698.71941904)(432.82188326,698.61942034)
\lineto(432.82188326,698.33442034)
\curveto(432.82187819,698.12441964)(432.81687819,697.92941983)(432.80688326,697.74942034)
\curveto(432.8068782,697.57942018)(432.72687828,697.4644203)(432.56688326,697.40442034)
\curveto(432.51687849,697.38442038)(432.47187854,697.37942038)(432.43188326,697.38942034)
\curveto(432.39187862,697.38942037)(432.34687866,697.37942038)(432.29688326,697.35942034)
\lineto(432.14688326,697.35942034)
\curveto(432.12687888,697.3594204)(432.09687891,697.3644204)(432.05688326,697.37442034)
\curveto(432.01687899,697.37442039)(431.98187903,697.36942039)(431.95188326,697.35942034)
\curveto(431.90187911,697.34942041)(431.84687916,697.34942041)(431.78688326,697.35942034)
\lineto(431.63688326,697.35942034)
\lineto(431.48688326,697.35942034)
\curveto(431.43687957,697.34942041)(431.39187962,697.34942041)(431.35188326,697.35942034)
\lineto(431.18688326,697.35942034)
\curveto(431.13687987,697.36942039)(431.08187993,697.37442039)(431.02188326,697.37442034)
\curveto(430.96188005,697.37442039)(430.9068801,697.37942038)(430.85688326,697.38942034)
\curveto(430.78688022,697.39942036)(430.72188029,697.40942035)(430.66188326,697.41942034)
\lineto(430.48188326,697.44942034)
\curveto(430.37188064,697.47942028)(430.26688074,697.51442025)(430.16688326,697.55442034)
\curveto(430.06688094,697.59442017)(429.97188104,697.63942012)(429.88188326,697.68942034)
\lineto(429.79188326,697.74942034)
\curveto(429.76188125,697.77941998)(429.72688128,697.80941995)(429.68688326,697.83942034)
\curveto(429.66688134,697.8594199)(429.64188137,697.87941988)(429.61188326,697.89942034)
\lineto(429.53688326,697.97442034)
\curveto(429.39688161,698.1644196)(429.29188172,698.37441939)(429.22188326,698.60442034)
\curveto(429.20188181,698.64441912)(429.19188182,698.67941908)(429.19188326,698.70942034)
\curveto(429.20188181,698.74941901)(429.20188181,698.79441897)(429.19188326,698.84442034)
\curveto(429.18188183,698.8644189)(429.17688183,698.88941887)(429.17688326,698.91942034)
\curveto(429.17688183,698.94941881)(429.17188184,698.97441879)(429.16188326,698.99442034)
\lineto(429.16188326,699.14442034)
\curveto(429.15188186,699.18441858)(429.14688186,699.22941853)(429.14688326,699.27942034)
\curveto(429.15688185,699.32941843)(429.16188185,699.37941838)(429.16188326,699.42942034)
\lineto(429.16188326,699.99942034)
\lineto(429.16188326,702.23442034)
\lineto(429.16188326,703.02942034)
\lineto(429.16188326,703.23942034)
\curveto(429.17188184,703.30941445)(429.16688184,703.37441439)(429.14688326,703.43442034)
\curveto(429.1068819,703.57441419)(429.03688197,703.6644141)(428.93688326,703.70442034)
\curveto(428.82688218,703.75441401)(428.68688232,703.76941399)(428.51688326,703.74942034)
\curveto(428.34688266,703.72941403)(428.20188281,703.74441402)(428.08188326,703.79442034)
\curveto(428.00188301,703.82441394)(427.95188306,703.86941389)(427.93188326,703.92942034)
\curveto(427.9118831,703.98941377)(427.89188312,704.0644137)(427.87188326,704.15442034)
\lineto(427.87188326,704.46942034)
\curveto(427.87188314,704.64941311)(427.88188313,704.79441297)(427.90188326,704.90442034)
\curveto(427.92188309,705.01441275)(428.006883,705.08941267)(428.15688326,705.12942034)
\curveto(428.19688281,705.14941261)(428.23688277,705.15441261)(428.27688326,705.14442034)
\lineto(428.41188326,705.14442034)
\curveto(428.56188245,705.14441262)(428.70188231,705.14941261)(428.83188326,705.15942034)
\curveto(428.96188205,705.17941258)(429.05188196,705.23941252)(429.10188326,705.33942034)
\curveto(429.13188188,705.40941235)(429.14688186,705.48941227)(429.14688326,705.57942034)
\curveto(429.15688185,705.66941209)(429.16188185,705.759412)(429.16188326,705.84942034)
\lineto(429.16188326,706.77942034)
\lineto(429.16188326,707.03442034)
\curveto(429.16188185,707.12441064)(429.17188184,707.19941056)(429.19188326,707.25942034)
\curveto(429.24188177,707.3594104)(429.31688169,707.42441034)(429.41688326,707.45442034)
\curveto(429.43688157,707.4644103)(429.46188155,707.4644103)(429.49188326,707.45442034)
\curveto(429.53188148,707.45441031)(429.56188145,707.4594103)(429.58188326,707.46942034)
}
}
{
\newrgbcolor{curcolor}{0 0 0}
\pscustom[linestyle=none,fillstyle=solid,fillcolor=curcolor]
{
\newpath
\moveto(438.23032076,705.35442034)
\curveto(438.34031544,705.35441241)(438.43531535,705.34441242)(438.51532076,705.32442034)
\curveto(438.60531518,705.30441246)(438.67531511,705.2594125)(438.72532076,705.18942034)
\curveto(438.785315,705.10941265)(438.81531497,704.96941279)(438.81532076,704.76942034)
\lineto(438.81532076,704.25942034)
\lineto(438.81532076,703.88442034)
\curveto(438.82531496,703.74441402)(438.81031497,703.63441413)(438.77032076,703.55442034)
\curveto(438.73031505,703.48441428)(438.67031511,703.43941432)(438.59032076,703.41942034)
\curveto(438.52031526,703.39941436)(438.43531535,703.38941437)(438.33532076,703.38942034)
\curveto(438.24531554,703.38941437)(438.14531564,703.39441437)(438.03532076,703.40442034)
\curveto(437.93531585,703.41441435)(437.84031594,703.40941435)(437.75032076,703.38942034)
\curveto(437.6803161,703.36941439)(437.61031617,703.35441441)(437.54032076,703.34442034)
\curveto(437.47031631,703.34441442)(437.40531638,703.33441443)(437.34532076,703.31442034)
\curveto(437.1853166,703.2644145)(437.02531676,703.18941457)(436.86532076,703.08942034)
\curveto(436.70531708,702.99941476)(436.5803172,702.89441487)(436.49032076,702.77442034)
\curveto(436.44031734,702.69441507)(436.3853174,702.60941515)(436.32532076,702.51942034)
\curveto(436.27531751,702.43941532)(436.22531756,702.35441541)(436.17532076,702.26442034)
\curveto(436.14531764,702.18441558)(436.11531767,702.09941566)(436.08532076,702.00942034)
\lineto(436.02532076,701.76942034)
\curveto(436.00531778,701.69941606)(435.99531779,701.62441614)(435.99532076,701.54442034)
\curveto(435.99531779,701.47441629)(435.9853178,701.40441636)(435.96532076,701.33442034)
\curveto(435.95531783,701.29441647)(435.95031783,701.25441651)(435.95032076,701.21442034)
\curveto(435.96031782,701.18441658)(435.96031782,701.15441661)(435.95032076,701.12442034)
\lineto(435.95032076,700.88442034)
\curveto(435.93031785,700.81441695)(435.92531786,700.73441703)(435.93532076,700.64442034)
\curveto(435.94531784,700.5644172)(435.95031783,700.48441728)(435.95032076,700.40442034)
\lineto(435.95032076,699.44442034)
\lineto(435.95032076,698.16942034)
\curveto(435.95031783,698.03941972)(435.94531784,697.91941984)(435.93532076,697.80942034)
\curveto(435.92531786,697.69942006)(435.89531789,697.60942015)(435.84532076,697.53942034)
\curveto(435.82531796,697.50942025)(435.79031799,697.48442028)(435.74032076,697.46442034)
\curveto(435.70031808,697.45442031)(435.65531813,697.44442032)(435.60532076,697.43442034)
\lineto(435.53032076,697.43442034)
\curveto(435.4803183,697.42442034)(435.42531836,697.41942034)(435.36532076,697.41942034)
\lineto(435.20032076,697.41942034)
\lineto(434.55532076,697.41942034)
\curveto(434.49531929,697.42942033)(434.43031935,697.43442033)(434.36032076,697.43442034)
\lineto(434.16532076,697.43442034)
\curveto(434.11531967,697.45442031)(434.06531972,697.46942029)(434.01532076,697.47942034)
\curveto(433.96531982,697.49942026)(433.93031985,697.53442023)(433.91032076,697.58442034)
\curveto(433.87031991,697.63442013)(433.84531994,697.70442006)(433.83532076,697.79442034)
\lineto(433.83532076,698.09442034)
\lineto(433.83532076,699.11442034)
\lineto(433.83532076,703.34442034)
\lineto(433.83532076,704.45442034)
\lineto(433.83532076,704.73942034)
\curveto(433.83531995,704.83941292)(433.85531993,704.91941284)(433.89532076,704.97942034)
\curveto(433.94531984,705.0594127)(434.02031976,705.10941265)(434.12032076,705.12942034)
\curveto(434.22031956,705.14941261)(434.34031944,705.1594126)(434.48032076,705.15942034)
\lineto(435.24532076,705.15942034)
\curveto(435.36531842,705.1594126)(435.47031831,705.14941261)(435.56032076,705.12942034)
\curveto(435.65031813,705.11941264)(435.72031806,705.07441269)(435.77032076,704.99442034)
\curveto(435.80031798,704.94441282)(435.81531797,704.87441289)(435.81532076,704.78442034)
\lineto(435.84532076,704.51442034)
\curveto(435.85531793,704.43441333)(435.87031791,704.3594134)(435.89032076,704.28942034)
\curveto(435.92031786,704.21941354)(435.97031781,704.18441358)(436.04032076,704.18442034)
\curveto(436.06031772,704.20441356)(436.0803177,704.21441355)(436.10032076,704.21442034)
\curveto(436.12031766,704.21441355)(436.14031764,704.22441354)(436.16032076,704.24442034)
\curveto(436.22031756,704.29441347)(436.27031751,704.34941341)(436.31032076,704.40942034)
\curveto(436.36031742,704.47941328)(436.42031736,704.53941322)(436.49032076,704.58942034)
\curveto(436.53031725,704.61941314)(436.56531722,704.64941311)(436.59532076,704.67942034)
\curveto(436.62531716,704.71941304)(436.66031712,704.75441301)(436.70032076,704.78442034)
\lineto(436.97032076,704.96442034)
\curveto(437.07031671,705.02441274)(437.17031661,705.07941268)(437.27032076,705.12942034)
\curveto(437.37031641,705.16941259)(437.47031631,705.20441256)(437.57032076,705.23442034)
\lineto(437.90032076,705.32442034)
\curveto(437.93031585,705.33441243)(437.9853158,705.33441243)(438.06532076,705.32442034)
\curveto(438.15531563,705.32441244)(438.21031557,705.33441243)(438.23032076,705.35442034)
}
}
{
\newrgbcolor{curcolor}{0 0 0}
\pscustom[linestyle=none,fillstyle=solid,fillcolor=curcolor]
{
\newpath
\moveto(441.73539888,708.00942034)
\curveto(441.80539593,707.92940983)(441.8403959,707.80940995)(441.84039888,707.64942034)
\lineto(441.84039888,707.18442034)
\lineto(441.84039888,706.77942034)
\curveto(441.8403959,706.63941112)(441.80539593,706.54441122)(441.73539888,706.49442034)
\curveto(441.67539606,706.44441132)(441.59539614,706.41441135)(441.49539888,706.40442034)
\curveto(441.40539633,706.39441137)(441.30539643,706.38941137)(441.19539888,706.38942034)
\lineto(440.35539888,706.38942034)
\curveto(440.24539749,706.38941137)(440.14539759,706.39441137)(440.05539888,706.40442034)
\curveto(439.97539776,706.41441135)(439.90539783,706.44441132)(439.84539888,706.49442034)
\curveto(439.80539793,706.52441124)(439.77539796,706.57941118)(439.75539888,706.65942034)
\curveto(439.74539799,706.74941101)(439.735398,706.84441092)(439.72539888,706.94442034)
\lineto(439.72539888,707.27442034)
\curveto(439.735398,707.38441038)(439.740398,707.47941028)(439.74039888,707.55942034)
\lineto(439.74039888,707.76942034)
\curveto(439.75039799,707.83940992)(439.77039797,707.89940986)(439.80039888,707.94942034)
\curveto(439.82039792,707.98940977)(439.84539789,708.01940974)(439.87539888,708.03942034)
\lineto(439.99539888,708.09942034)
\curveto(440.01539772,708.09940966)(440.0403977,708.09940966)(440.07039888,708.09942034)
\curveto(440.10039764,708.10940965)(440.12539761,708.11440965)(440.14539888,708.11442034)
\lineto(441.24039888,708.11442034)
\curveto(441.3403964,708.11440965)(441.4353963,708.10940965)(441.52539888,708.09942034)
\curveto(441.61539612,708.08940967)(441.68539605,708.0594097)(441.73539888,708.00942034)
\moveto(441.84039888,698.24442034)
\curveto(441.8403959,698.04441972)(441.8353959,697.87441989)(441.82539888,697.73442034)
\curveto(441.81539592,697.59442017)(441.72539601,697.49942026)(441.55539888,697.44942034)
\curveto(441.49539624,697.42942033)(441.43039631,697.41942034)(441.36039888,697.41942034)
\curveto(441.29039645,697.42942033)(441.21539652,697.43442033)(441.13539888,697.43442034)
\lineto(440.29539888,697.43442034)
\curveto(440.20539753,697.43442033)(440.11539762,697.43942032)(440.02539888,697.44942034)
\curveto(439.94539779,697.4594203)(439.88539785,697.48942027)(439.84539888,697.53942034)
\curveto(439.78539795,697.60942015)(439.75039799,697.69442007)(439.74039888,697.79442034)
\lineto(439.74039888,698.13942034)
\lineto(439.74039888,704.46942034)
\lineto(439.74039888,704.76942034)
\curveto(439.740398,704.86941289)(439.76039798,704.94941281)(439.80039888,705.00942034)
\curveto(439.86039788,705.07941268)(439.94539779,705.12441264)(440.05539888,705.14442034)
\curveto(440.07539766,705.15441261)(440.10039764,705.15441261)(440.13039888,705.14442034)
\curveto(440.17039757,705.14441262)(440.20039754,705.14941261)(440.22039888,705.15942034)
\lineto(440.97039888,705.15942034)
\lineto(441.16539888,705.15942034)
\curveto(441.24539649,705.16941259)(441.31039643,705.16941259)(441.36039888,705.15942034)
\lineto(441.48039888,705.15942034)
\curveto(441.5403962,705.13941262)(441.59539614,705.12441264)(441.64539888,705.11442034)
\curveto(441.69539604,705.10441266)(441.735396,705.07441269)(441.76539888,705.02442034)
\curveto(441.80539593,704.97441279)(441.82539591,704.90441286)(441.82539888,704.81442034)
\curveto(441.8353959,704.72441304)(441.8403959,704.62941313)(441.84039888,704.52942034)
\lineto(441.84039888,698.24442034)
}
}
{
\newrgbcolor{curcolor}{0 0 0}
\pscustom[linestyle=none,fillstyle=solid,fillcolor=curcolor]
{
\newpath
\moveto(451.30258638,701.67942034)
\curveto(451.32257778,701.61941614)(451.33257777,701.51441625)(451.33258638,701.36442034)
\curveto(451.33257777,701.22441654)(451.32757778,701.12441664)(451.31758638,701.06442034)
\curveto(451.31757779,701.01441675)(451.31257779,700.96941679)(451.30258638,700.92942034)
\lineto(451.30258638,700.80942034)
\curveto(451.28257782,700.72941703)(451.27257783,700.64941711)(451.27258638,700.56942034)
\curveto(451.27257783,700.49941726)(451.26257784,700.42441734)(451.24258638,700.34442034)
\curveto(451.24257786,700.30441746)(451.23257787,700.23441753)(451.21258638,700.13442034)
\curveto(451.18257792,700.01441775)(451.15257795,699.88941787)(451.12258638,699.75942034)
\curveto(451.102578,699.63941812)(451.06757804,699.52441824)(451.01758638,699.41442034)
\curveto(450.83757827,698.9644188)(450.61257849,698.57441919)(450.34258638,698.24442034)
\curveto(450.07257903,697.91441985)(449.71757939,697.65442011)(449.27758638,697.46442034)
\curveto(449.18757992,697.42442034)(449.09258001,697.39442037)(448.99258638,697.37442034)
\curveto(448.9025802,697.34442042)(448.8025803,697.31442045)(448.69258638,697.28442034)
\curveto(448.63258047,697.2644205)(448.56758054,697.25442051)(448.49758638,697.25442034)
\curveto(448.43758067,697.25442051)(448.37758073,697.24942051)(448.31758638,697.23942034)
\lineto(448.18258638,697.23942034)
\curveto(448.12258098,697.21942054)(448.04258106,697.21442055)(447.94258638,697.22442034)
\curveto(447.84258126,697.22442054)(447.76258134,697.23442053)(447.70258638,697.25442034)
\lineto(447.61258638,697.25442034)
\curveto(447.56258154,697.2644205)(447.5075816,697.27442049)(447.44758638,697.28442034)
\curveto(447.38758172,697.28442048)(447.32758178,697.28942047)(447.26758638,697.29942034)
\curveto(447.07758203,697.34942041)(446.9025822,697.39942036)(446.74258638,697.44942034)
\curveto(446.58258252,697.49942026)(446.43258267,697.56942019)(446.29258638,697.65942034)
\lineto(446.11258638,697.77942034)
\curveto(446.06258304,697.81941994)(446.01258309,697.8644199)(445.96258638,697.91442034)
\lineto(445.87258638,697.97442034)
\curveto(445.84258326,697.99441977)(445.81258329,698.00941975)(445.78258638,698.01942034)
\curveto(445.69258341,698.04941971)(445.63758347,698.02941973)(445.61758638,697.95942034)
\curveto(445.56758354,697.88941987)(445.53258357,697.80441996)(445.51258638,697.70442034)
\curveto(445.5025836,697.61442015)(445.46758364,697.54442022)(445.40758638,697.49442034)
\curveto(445.34758376,697.45442031)(445.27758383,697.42942033)(445.19758638,697.41942034)
\lineto(444.92758638,697.41942034)
\lineto(444.20758638,697.41942034)
\lineto(443.98258638,697.41942034)
\curveto(443.91258519,697.40942035)(443.84758526,697.41442035)(443.78758638,697.43442034)
\curveto(443.64758546,697.48442028)(443.56758554,697.57442019)(443.54758638,697.70442034)
\curveto(443.53758557,697.84441992)(443.53258557,697.99941976)(443.53258638,698.16942034)
\lineto(443.53258638,707.31942034)
\lineto(443.53258638,707.66442034)
\curveto(443.53258557,707.78440998)(443.55758555,707.87940988)(443.60758638,707.94942034)
\curveto(443.64758546,708.01940974)(443.71758539,708.0644097)(443.81758638,708.08442034)
\curveto(443.83758527,708.09440967)(443.85758525,708.09440967)(443.87758638,708.08442034)
\curveto(443.9075852,708.08440968)(443.93258517,708.08940967)(443.95258638,708.09942034)
\lineto(444.89758638,708.09942034)
\curveto(445.07758403,708.09940966)(445.23258387,708.08940967)(445.36258638,708.06942034)
\curveto(445.49258361,708.0594097)(445.57758353,707.98440978)(445.61758638,707.84442034)
\curveto(445.64758346,707.74441002)(445.65758345,707.60941015)(445.64758638,707.43942034)
\curveto(445.63758347,707.27941048)(445.63258347,707.13941062)(445.63258638,707.01942034)
\lineto(445.63258638,705.38442034)
\lineto(445.63258638,705.05442034)
\curveto(445.63258347,704.94441282)(445.64258346,704.84941291)(445.66258638,704.76942034)
\curveto(445.67258343,704.71941304)(445.68258342,704.67441309)(445.69258638,704.63442034)
\curveto(445.7025834,704.60441316)(445.72758338,704.58441318)(445.76758638,704.57442034)
\curveto(445.78758332,704.55441321)(445.81258329,704.54441322)(445.84258638,704.54442034)
\curveto(445.88258322,704.54441322)(445.91258319,704.54941321)(445.93258638,704.55942034)
\curveto(446.0025831,704.59941316)(446.06758304,704.63941312)(446.12758638,704.67942034)
\curveto(446.18758292,704.72941303)(446.25258285,704.77941298)(446.32258638,704.82942034)
\curveto(446.45258265,704.91941284)(446.58758252,704.99441277)(446.72758638,705.05442034)
\curveto(446.86758224,705.12441264)(447.02258208,705.18441258)(447.19258638,705.23442034)
\curveto(447.27258183,705.2644125)(447.35258175,705.27941248)(447.43258638,705.27942034)
\curveto(447.51258159,705.28941247)(447.59258151,705.30441246)(447.67258638,705.32442034)
\curveto(447.74258136,705.34441242)(447.81758129,705.35441241)(447.89758638,705.35442034)
\lineto(448.13758638,705.35442034)
\lineto(448.28758638,705.35442034)
\curveto(448.31758079,705.34441242)(448.35258075,705.33941242)(448.39258638,705.33942034)
\curveto(448.43258067,705.34941241)(448.47258063,705.34941241)(448.51258638,705.33942034)
\curveto(448.62258048,705.30941245)(448.72258038,705.28441248)(448.81258638,705.26442034)
\curveto(448.91258019,705.25441251)(449.0075801,705.22941253)(449.09758638,705.18942034)
\curveto(449.55757955,704.99941276)(449.93257917,704.75441301)(450.22258638,704.45442034)
\curveto(450.51257859,704.15441361)(450.75757835,703.77941398)(450.95758638,703.32942034)
\curveto(451.0075781,703.20941455)(451.04757806,703.08441468)(451.07758638,702.95442034)
\curveto(451.11757799,702.82441494)(451.15757795,702.68941507)(451.19758638,702.54942034)
\curveto(451.21757789,702.47941528)(451.22757788,702.40941535)(451.22758638,702.33942034)
\curveto(451.23757787,702.27941548)(451.25257785,702.20941555)(451.27258638,702.12942034)
\curveto(451.29257781,702.07941568)(451.29757781,702.02441574)(451.28758638,701.96442034)
\curveto(451.28757782,701.90441586)(451.29257781,701.84441592)(451.30258638,701.78442034)
\lineto(451.30258638,701.67942034)
\moveto(449.08258638,700.26942034)
\curveto(449.11257999,700.36941739)(449.13757997,700.49441727)(449.15758638,700.64442034)
\curveto(449.18757992,700.79441697)(449.2025799,700.94441682)(449.20258638,701.09442034)
\curveto(449.21257989,701.25441651)(449.21257989,701.40941635)(449.20258638,701.55942034)
\curveto(449.2025799,701.71941604)(449.18757992,701.85441591)(449.15758638,701.96442034)
\curveto(449.12757998,702.0644157)(449.10758,702.1594156)(449.09758638,702.24942034)
\curveto(449.08758002,702.33941542)(449.06258004,702.42441534)(449.02258638,702.50442034)
\curveto(448.88258022,702.85441491)(448.68258042,703.14941461)(448.42258638,703.38942034)
\curveto(448.17258093,703.63941412)(447.8025813,703.764414)(447.31258638,703.76442034)
\curveto(447.27258183,703.764414)(447.23758187,703.759414)(447.20758638,703.74942034)
\lineto(447.10258638,703.74942034)
\curveto(447.03258207,703.72941403)(446.96758214,703.70941405)(446.90758638,703.68942034)
\curveto(446.84758226,703.67941408)(446.78758232,703.6644141)(446.72758638,703.64442034)
\curveto(446.43758267,703.51441425)(446.21758289,703.32941443)(446.06758638,703.08942034)
\curveto(445.91758319,702.8594149)(445.79258331,702.59441517)(445.69258638,702.29442034)
\curveto(445.66258344,702.21441555)(445.64258346,702.12941563)(445.63258638,702.03942034)
\curveto(445.63258347,701.9594158)(445.62258348,701.87941588)(445.60258638,701.79942034)
\curveto(445.59258351,701.76941599)(445.58758352,701.71941604)(445.58758638,701.64942034)
\curveto(445.57758353,701.60941615)(445.57258353,701.56941619)(445.57258638,701.52942034)
\curveto(445.58258352,701.48941627)(445.58258352,701.44941631)(445.57258638,701.40942034)
\curveto(445.55258355,701.32941643)(445.54758356,701.21941654)(445.55758638,701.07942034)
\curveto(445.56758354,700.93941682)(445.58258352,700.83941692)(445.60258638,700.77942034)
\curveto(445.62258348,700.68941707)(445.63258347,700.60441716)(445.63258638,700.52442034)
\curveto(445.64258346,700.44441732)(445.66258344,700.3644174)(445.69258638,700.28442034)
\curveto(445.78258332,700.00441776)(445.88758322,699.759418)(446.00758638,699.54942034)
\curveto(446.13758297,699.34941841)(446.31758279,699.17941858)(446.54758638,699.03942034)
\curveto(446.7075824,698.93941882)(446.87258223,698.86941889)(447.04258638,698.82942034)
\curveto(447.06258204,698.82941893)(447.08258202,698.82441894)(447.10258638,698.81442034)
\lineto(447.19258638,698.81442034)
\curveto(447.22258188,698.80441896)(447.27258183,698.79441897)(447.34258638,698.78442034)
\curveto(447.41258169,698.78441898)(447.47258163,698.78941897)(447.52258638,698.79942034)
\curveto(447.62258148,698.81941894)(447.71258139,698.83441893)(447.79258638,698.84442034)
\curveto(447.88258122,698.8644189)(447.96758114,698.88941887)(448.04758638,698.91942034)
\curveto(448.32758078,699.04941871)(448.54258056,699.22941853)(448.69258638,699.45942034)
\curveto(448.85258025,699.68941807)(448.98258012,699.9594178)(449.08258638,700.26942034)
}
}
{
\newrgbcolor{curcolor}{0 0 0}
\pscustom[linestyle=none,fillstyle=solid,fillcolor=curcolor]
{
\newpath
\moveto(453.09250826,705.14442034)
\lineto(454.21750826,705.14442034)
\curveto(454.32750582,705.14441262)(454.42750572,705.13941262)(454.51750826,705.12942034)
\curveto(454.60750554,705.11941264)(454.67250548,705.08441268)(454.71250826,705.02442034)
\curveto(454.76250539,704.9644128)(454.79250536,704.87941288)(454.80250826,704.76942034)
\curveto(454.81250534,704.66941309)(454.81750533,704.5644132)(454.81750826,704.45442034)
\lineto(454.81750826,703.40442034)
\lineto(454.81750826,701.16942034)
\curveto(454.81750533,700.80941695)(454.83250532,700.46941729)(454.86250826,700.14942034)
\curveto(454.89250526,699.82941793)(454.98250517,699.5644182)(455.13250826,699.35442034)
\curveto(455.27250488,699.14441862)(455.49750465,698.99441877)(455.80750826,698.90442034)
\curveto(455.85750429,698.89441887)(455.89750425,698.88941887)(455.92750826,698.88942034)
\curveto(455.96750418,698.88941887)(456.01250414,698.88441888)(456.06250826,698.87442034)
\curveto(456.11250404,698.8644189)(456.16750398,698.8594189)(456.22750826,698.85942034)
\curveto(456.28750386,698.8594189)(456.33250382,698.8644189)(456.36250826,698.87442034)
\curveto(456.41250374,698.89441887)(456.4525037,698.89941886)(456.48250826,698.88942034)
\curveto(456.52250363,698.87941888)(456.56250359,698.88441888)(456.60250826,698.90442034)
\curveto(456.81250334,698.95441881)(456.97750317,699.01941874)(457.09750826,699.09942034)
\curveto(457.27750287,699.20941855)(457.41750273,699.34941841)(457.51750826,699.51942034)
\curveto(457.62750252,699.69941806)(457.70250245,699.89441787)(457.74250826,700.10442034)
\curveto(457.79250236,700.32441744)(457.82250233,700.5644172)(457.83250826,700.82442034)
\curveto(457.84250231,701.09441667)(457.8475023,701.37441639)(457.84750826,701.66442034)
\lineto(457.84750826,703.47942034)
\lineto(457.84750826,704.45442034)
\lineto(457.84750826,704.72442034)
\curveto(457.8475023,704.82441294)(457.86750228,704.90441286)(457.90750826,704.96442034)
\curveto(457.95750219,705.05441271)(458.03250212,705.10441266)(458.13250826,705.11442034)
\curveto(458.23250192,705.13441263)(458.3525018,705.14441262)(458.49250826,705.14442034)
\lineto(459.28750826,705.14442034)
\lineto(459.57250826,705.14442034)
\curveto(459.66250049,705.14441262)(459.73750041,705.12441264)(459.79750826,705.08442034)
\curveto(459.87750027,705.03441273)(459.92250023,704.9594128)(459.93250826,704.85942034)
\curveto(459.94250021,704.759413)(459.9475002,704.64441312)(459.94750826,704.51442034)
\lineto(459.94750826,703.37442034)
\lineto(459.94750826,699.15942034)
\lineto(459.94750826,698.09442034)
\lineto(459.94750826,697.79442034)
\curveto(459.9475002,697.69442007)(459.92750022,697.61942014)(459.88750826,697.56942034)
\curveto(459.83750031,697.48942027)(459.76250039,697.44442032)(459.66250826,697.43442034)
\curveto(459.56250059,697.42442034)(459.45750069,697.41942034)(459.34750826,697.41942034)
\lineto(458.53750826,697.41942034)
\curveto(458.42750172,697.41942034)(458.32750182,697.42442034)(458.23750826,697.43442034)
\curveto(458.15750199,697.44442032)(458.09250206,697.48442028)(458.04250826,697.55442034)
\curveto(458.02250213,697.58442018)(458.00250215,697.62942013)(457.98250826,697.68942034)
\curveto(457.97250218,697.74942001)(457.95750219,697.80941995)(457.93750826,697.86942034)
\curveto(457.92750222,697.92941983)(457.91250224,697.98441978)(457.89250826,698.03442034)
\curveto(457.87250228,698.08441968)(457.84250231,698.11441965)(457.80250826,698.12442034)
\curveto(457.78250237,698.14441962)(457.75750239,698.14941961)(457.72750826,698.13942034)
\curveto(457.69750245,698.12941963)(457.67250248,698.11941964)(457.65250826,698.10942034)
\curveto(457.58250257,698.06941969)(457.52250263,698.02441974)(457.47250826,697.97442034)
\curveto(457.42250273,697.92441984)(457.36750278,697.87941988)(457.30750826,697.83942034)
\curveto(457.26750288,697.80941995)(457.22750292,697.77441999)(457.18750826,697.73442034)
\curveto(457.15750299,697.70442006)(457.11750303,697.67442009)(457.06750826,697.64442034)
\curveto(456.83750331,697.50442026)(456.56750358,697.39442037)(456.25750826,697.31442034)
\curveto(456.18750396,697.29442047)(456.11750403,697.28442048)(456.04750826,697.28442034)
\curveto(455.97750417,697.27442049)(455.90250425,697.2594205)(455.82250826,697.23942034)
\curveto(455.78250437,697.22942053)(455.73750441,697.22942053)(455.68750826,697.23942034)
\curveto(455.6475045,697.23942052)(455.60750454,697.23442053)(455.56750826,697.22442034)
\curveto(455.53750461,697.21442055)(455.47250468,697.21442055)(455.37250826,697.22442034)
\curveto(455.28250487,697.22442054)(455.22250493,697.22942053)(455.19250826,697.23942034)
\curveto(455.14250501,697.23942052)(455.09250506,697.24442052)(455.04250826,697.25442034)
\lineto(454.89250826,697.25442034)
\curveto(454.77250538,697.28442048)(454.65750549,697.30942045)(454.54750826,697.32942034)
\curveto(454.43750571,697.34942041)(454.32750582,697.37942038)(454.21750826,697.41942034)
\curveto(454.16750598,697.43942032)(454.12250603,697.45442031)(454.08250826,697.46442034)
\curveto(454.0525061,697.48442028)(454.01250614,697.50442026)(453.96250826,697.52442034)
\curveto(453.61250654,697.71442005)(453.33250682,697.97941978)(453.12250826,698.31942034)
\curveto(452.99250716,698.52941923)(452.89750725,698.77941898)(452.83750826,699.06942034)
\curveto(452.77750737,699.36941839)(452.73750741,699.68441808)(452.71750826,700.01442034)
\curveto(452.70750744,700.35441741)(452.70250745,700.69941706)(452.70250826,701.04942034)
\curveto(452.71250744,701.40941635)(452.71750743,701.764416)(452.71750826,702.11442034)
\lineto(452.71750826,704.15442034)
\curveto(452.71750743,704.28441348)(452.71250744,704.43441333)(452.70250826,704.60442034)
\curveto(452.70250745,704.78441298)(452.72750742,704.91441285)(452.77750826,704.99442034)
\curveto(452.80750734,705.04441272)(452.86750728,705.08941267)(452.95750826,705.12942034)
\curveto(453.01750713,705.12941263)(453.06250709,705.13441263)(453.09250826,705.14442034)
}
}
{
\newrgbcolor{curcolor}{0 0 0}
\pscustom[linestyle=none,fillstyle=solid,fillcolor=curcolor]
{
\newpath
\moveto(465.14875826,705.36942034)
\curveto(465.9587531,705.38941237)(466.63375242,705.26941249)(467.17375826,705.00942034)
\curveto(467.72375133,704.74941301)(468.1587509,704.37941338)(468.47875826,703.89942034)
\curveto(468.63875042,703.6594141)(468.7587503,703.38441438)(468.83875826,703.07442034)
\curveto(468.8587502,703.02441474)(468.87375018,702.9594148)(468.88375826,702.87942034)
\curveto(468.90375015,702.79941496)(468.90375015,702.72941503)(468.88375826,702.66942034)
\curveto(468.84375021,702.5594152)(468.77375028,702.49441527)(468.67375826,702.47442034)
\curveto(468.57375048,702.4644153)(468.4537506,702.4594153)(468.31375826,702.45942034)
\lineto(467.53375826,702.45942034)
\lineto(467.24875826,702.45942034)
\curveto(467.1587519,702.4594153)(467.08375197,702.47941528)(467.02375826,702.51942034)
\curveto(466.94375211,702.5594152)(466.88875217,702.61941514)(466.85875826,702.69942034)
\curveto(466.82875223,702.78941497)(466.78875227,702.87941488)(466.73875826,702.96942034)
\curveto(466.67875238,703.07941468)(466.61375244,703.17941458)(466.54375826,703.26942034)
\curveto(466.47375258,703.3594144)(466.39375266,703.43941432)(466.30375826,703.50942034)
\curveto(466.16375289,703.59941416)(466.00875305,703.66941409)(465.83875826,703.71942034)
\curveto(465.77875328,703.73941402)(465.71875334,703.74941401)(465.65875826,703.74942034)
\curveto(465.59875346,703.74941401)(465.54375351,703.759414)(465.49375826,703.77942034)
\lineto(465.34375826,703.77942034)
\curveto(465.14375391,703.77941398)(464.98375407,703.759414)(464.86375826,703.71942034)
\curveto(464.57375448,703.62941413)(464.33875472,703.48941427)(464.15875826,703.29942034)
\curveto(463.97875508,703.11941464)(463.83375522,702.89941486)(463.72375826,702.63942034)
\curveto(463.67375538,702.52941523)(463.63375542,702.40941535)(463.60375826,702.27942034)
\curveto(463.58375547,702.1594156)(463.5587555,702.02941573)(463.52875826,701.88942034)
\curveto(463.51875554,701.84941591)(463.51375554,701.80941595)(463.51375826,701.76942034)
\curveto(463.51375554,701.72941603)(463.50875555,701.68941607)(463.49875826,701.64942034)
\curveto(463.47875558,701.54941621)(463.46875559,701.40941635)(463.46875826,701.22942034)
\curveto(463.47875558,701.04941671)(463.49375556,700.90941685)(463.51375826,700.80942034)
\curveto(463.51375554,700.72941703)(463.51875554,700.67441709)(463.52875826,700.64442034)
\curveto(463.54875551,700.57441719)(463.5587555,700.50441726)(463.55875826,700.43442034)
\curveto(463.56875549,700.3644174)(463.58375547,700.29441747)(463.60375826,700.22442034)
\curveto(463.68375537,699.99441777)(463.77875528,699.78441798)(463.88875826,699.59442034)
\curveto(463.99875506,699.40441836)(464.13875492,699.24441852)(464.30875826,699.11442034)
\curveto(464.34875471,699.08441868)(464.40875465,699.04941871)(464.48875826,699.00942034)
\curveto(464.59875446,698.93941882)(464.70875435,698.89441887)(464.81875826,698.87442034)
\curveto(464.93875412,698.85441891)(465.08375397,698.83441893)(465.25375826,698.81442034)
\lineto(465.34375826,698.81442034)
\curveto(465.38375367,698.81441895)(465.41375364,698.81941894)(465.43375826,698.82942034)
\lineto(465.56875826,698.82942034)
\curveto(465.63875342,698.84941891)(465.70375335,698.8644189)(465.76375826,698.87442034)
\curveto(465.83375322,698.89441887)(465.89875316,698.91441885)(465.95875826,698.93442034)
\curveto(466.2587528,699.0644187)(466.48875257,699.25441851)(466.64875826,699.50442034)
\curveto(466.68875237,699.55441821)(466.72375233,699.60941815)(466.75375826,699.66942034)
\curveto(466.78375227,699.73941802)(466.80875225,699.79941796)(466.82875826,699.84942034)
\curveto(466.86875219,699.9594178)(466.90375215,700.05441771)(466.93375826,700.13442034)
\curveto(466.96375209,700.22441754)(467.03375202,700.29441747)(467.14375826,700.34442034)
\curveto(467.23375182,700.38441738)(467.37875168,700.39941736)(467.57875826,700.38942034)
\lineto(468.07375826,700.38942034)
\lineto(468.28375826,700.38942034)
\curveto(468.36375069,700.39941736)(468.42875063,700.39441737)(468.47875826,700.37442034)
\lineto(468.59875826,700.37442034)
\lineto(468.71875826,700.34442034)
\curveto(468.7587503,700.34441742)(468.78875027,700.33441743)(468.80875826,700.31442034)
\curveto(468.8587502,700.27441749)(468.88875017,700.21441755)(468.89875826,700.13442034)
\curveto(468.91875014,700.0644177)(468.91875014,699.98941777)(468.89875826,699.90942034)
\curveto(468.80875025,699.57941818)(468.69875036,699.28441848)(468.56875826,699.02442034)
\curveto(468.1587509,698.25441951)(467.50375155,697.71942004)(466.60375826,697.41942034)
\curveto(466.50375255,697.38942037)(466.39875266,697.36942039)(466.28875826,697.35942034)
\curveto(466.17875288,697.33942042)(466.06875299,697.31442045)(465.95875826,697.28442034)
\curveto(465.89875316,697.27442049)(465.83875322,697.26942049)(465.77875826,697.26942034)
\curveto(465.71875334,697.26942049)(465.6587534,697.2644205)(465.59875826,697.25442034)
\lineto(465.43375826,697.25442034)
\curveto(465.38375367,697.23442053)(465.30875375,697.22942053)(465.20875826,697.23942034)
\curveto(465.10875395,697.23942052)(465.03375402,697.24442052)(464.98375826,697.25442034)
\curveto(464.90375415,697.27442049)(464.82875423,697.28442048)(464.75875826,697.28442034)
\curveto(464.69875436,697.27442049)(464.63375442,697.27942048)(464.56375826,697.29942034)
\lineto(464.41375826,697.32942034)
\curveto(464.36375469,697.32942043)(464.31375474,697.33442043)(464.26375826,697.34442034)
\curveto(464.1537549,697.37442039)(464.04875501,697.40442036)(463.94875826,697.43442034)
\curveto(463.84875521,697.4644203)(463.7537553,697.49942026)(463.66375826,697.53942034)
\curveto(463.19375586,697.73942002)(462.79875626,697.99441977)(462.47875826,698.30442034)
\curveto(462.1587569,698.62441914)(461.89875716,699.01941874)(461.69875826,699.48942034)
\curveto(461.64875741,699.57941818)(461.60875745,699.67441809)(461.57875826,699.77442034)
\lineto(461.48875826,700.10442034)
\curveto(461.47875758,700.14441762)(461.47375758,700.17941758)(461.47375826,700.20942034)
\curveto(461.47375758,700.24941751)(461.46375759,700.29441747)(461.44375826,700.34442034)
\curveto(461.42375763,700.41441735)(461.41375764,700.48441728)(461.41375826,700.55442034)
\curveto(461.41375764,700.63441713)(461.40375765,700.70941705)(461.38375826,700.77942034)
\lineto(461.38375826,701.03442034)
\curveto(461.36375769,701.08441668)(461.3537577,701.13941662)(461.35375826,701.19942034)
\curveto(461.3537577,701.26941649)(461.36375769,701.32941643)(461.38375826,701.37942034)
\curveto(461.39375766,701.42941633)(461.39375766,701.47441629)(461.38375826,701.51442034)
\curveto(461.37375768,701.55441621)(461.37375768,701.59441617)(461.38375826,701.63442034)
\curveto(461.40375765,701.70441606)(461.40875765,701.76941599)(461.39875826,701.82942034)
\curveto(461.39875766,701.88941587)(461.40875765,701.94941581)(461.42875826,702.00942034)
\curveto(461.47875758,702.18941557)(461.51875754,702.3594154)(461.54875826,702.51942034)
\curveto(461.57875748,702.68941507)(461.62375743,702.85441491)(461.68375826,703.01442034)
\curveto(461.90375715,703.52441424)(462.17875688,703.94941381)(462.50875826,704.28942034)
\curveto(462.84875621,704.62941313)(463.27875578,704.90441286)(463.79875826,705.11442034)
\curveto(463.93875512,705.17441259)(464.08375497,705.21441255)(464.23375826,705.23442034)
\curveto(464.38375467,705.2644125)(464.53875452,705.29941246)(464.69875826,705.33942034)
\curveto(464.77875428,705.34941241)(464.8537542,705.35441241)(464.92375826,705.35442034)
\curveto(464.99375406,705.35441241)(465.06875399,705.3594124)(465.14875826,705.36942034)
}
}
{
\newrgbcolor{curcolor}{0 0 0}
\pscustom[linestyle=none,fillstyle=solid,fillcolor=curcolor]
{
\newpath
\moveto(472.29203951,708.00942034)
\curveto(472.36203656,707.92940983)(472.39703652,707.80940995)(472.39703951,707.64942034)
\lineto(472.39703951,707.18442034)
\lineto(472.39703951,706.77942034)
\curveto(472.39703652,706.63941112)(472.36203656,706.54441122)(472.29203951,706.49442034)
\curveto(472.23203669,706.44441132)(472.15203677,706.41441135)(472.05203951,706.40442034)
\curveto(471.96203696,706.39441137)(471.86203706,706.38941137)(471.75203951,706.38942034)
\lineto(470.91203951,706.38942034)
\curveto(470.80203812,706.38941137)(470.70203822,706.39441137)(470.61203951,706.40442034)
\curveto(470.53203839,706.41441135)(470.46203846,706.44441132)(470.40203951,706.49442034)
\curveto(470.36203856,706.52441124)(470.33203859,706.57941118)(470.31203951,706.65942034)
\curveto(470.30203862,706.74941101)(470.29203863,706.84441092)(470.28203951,706.94442034)
\lineto(470.28203951,707.27442034)
\curveto(470.29203863,707.38441038)(470.29703862,707.47941028)(470.29703951,707.55942034)
\lineto(470.29703951,707.76942034)
\curveto(470.30703861,707.83940992)(470.32703859,707.89940986)(470.35703951,707.94942034)
\curveto(470.37703854,707.98940977)(470.40203852,708.01940974)(470.43203951,708.03942034)
\lineto(470.55203951,708.09942034)
\curveto(470.57203835,708.09940966)(470.59703832,708.09940966)(470.62703951,708.09942034)
\curveto(470.65703826,708.10940965)(470.68203824,708.11440965)(470.70203951,708.11442034)
\lineto(471.79703951,708.11442034)
\curveto(471.89703702,708.11440965)(471.99203693,708.10940965)(472.08203951,708.09942034)
\curveto(472.17203675,708.08940967)(472.24203668,708.0594097)(472.29203951,708.00942034)
\moveto(472.39703951,698.24442034)
\curveto(472.39703652,698.04441972)(472.39203653,697.87441989)(472.38203951,697.73442034)
\curveto(472.37203655,697.59442017)(472.28203664,697.49942026)(472.11203951,697.44942034)
\curveto(472.05203687,697.42942033)(471.98703693,697.41942034)(471.91703951,697.41942034)
\curveto(471.84703707,697.42942033)(471.77203715,697.43442033)(471.69203951,697.43442034)
\lineto(470.85203951,697.43442034)
\curveto(470.76203816,697.43442033)(470.67203825,697.43942032)(470.58203951,697.44942034)
\curveto(470.50203842,697.4594203)(470.44203848,697.48942027)(470.40203951,697.53942034)
\curveto(470.34203858,697.60942015)(470.30703861,697.69442007)(470.29703951,697.79442034)
\lineto(470.29703951,698.13942034)
\lineto(470.29703951,704.46942034)
\lineto(470.29703951,704.76942034)
\curveto(470.29703862,704.86941289)(470.3170386,704.94941281)(470.35703951,705.00942034)
\curveto(470.4170385,705.07941268)(470.50203842,705.12441264)(470.61203951,705.14442034)
\curveto(470.63203829,705.15441261)(470.65703826,705.15441261)(470.68703951,705.14442034)
\curveto(470.72703819,705.14441262)(470.75703816,705.14941261)(470.77703951,705.15942034)
\lineto(471.52703951,705.15942034)
\lineto(471.72203951,705.15942034)
\curveto(471.80203712,705.16941259)(471.86703705,705.16941259)(471.91703951,705.15942034)
\lineto(472.03703951,705.15942034)
\curveto(472.09703682,705.13941262)(472.15203677,705.12441264)(472.20203951,705.11442034)
\curveto(472.25203667,705.10441266)(472.29203663,705.07441269)(472.32203951,705.02442034)
\curveto(472.36203656,704.97441279)(472.38203654,704.90441286)(472.38203951,704.81442034)
\curveto(472.39203653,704.72441304)(472.39703652,704.62941313)(472.39703951,704.52942034)
\lineto(472.39703951,698.24442034)
}
}
{
\newrgbcolor{curcolor}{0 0 0}
\pscustom[linestyle=none,fillstyle=solid,fillcolor=curcolor]
{
\newpath
\moveto(481.82922701,701.60442034)
\curveto(481.80921848,701.65441611)(481.80421848,701.70941605)(481.81422701,701.76942034)
\curveto(481.82421846,701.82941593)(481.81921847,701.88441588)(481.79922701,701.93442034)
\curveto(481.7892185,701.97441579)(481.7842185,702.01441575)(481.78422701,702.05442034)
\curveto(481.7842185,702.09441567)(481.77921851,702.13441563)(481.76922701,702.17442034)
\lineto(481.70922701,702.44442034)
\curveto(481.6892186,702.53441523)(481.66421862,702.61941514)(481.63422701,702.69942034)
\curveto(481.5842187,702.83941492)(481.53921875,702.96941479)(481.49922701,703.08942034)
\curveto(481.45921883,703.21941454)(481.40421888,703.33941442)(481.33422701,703.44942034)
\curveto(481.26421902,703.5594142)(481.19421909,703.6644141)(481.12422701,703.76442034)
\curveto(481.06421922,703.8644139)(480.99421929,703.9644138)(480.91422701,704.06442034)
\curveto(480.83421945,704.17441359)(480.73421955,704.27441349)(480.61422701,704.36442034)
\curveto(480.50421978,704.4644133)(480.39421989,704.55441321)(480.28422701,704.63442034)
\curveto(479.95422033,704.8644129)(479.57422071,705.04441272)(479.14422701,705.17442034)
\curveto(478.72422156,705.30441246)(478.22422206,705.3644124)(477.64422701,705.35442034)
\curveto(477.57422271,705.34441242)(477.50422278,705.33941242)(477.43422701,705.33942034)
\curveto(477.36422292,705.33941242)(477.289223,705.33441243)(477.20922701,705.32442034)
\curveto(477.05922323,705.28441248)(476.91422337,705.25441251)(476.77422701,705.23442034)
\curveto(476.63422365,705.21441255)(476.49922379,705.17941258)(476.36922701,705.12942034)
\curveto(476.25922403,705.07941268)(476.14922414,705.03441273)(476.03922701,704.99442034)
\curveto(475.92922436,704.95441281)(475.82422446,704.90941285)(475.72422701,704.85942034)
\curveto(475.36422492,704.62941313)(475.05922523,704.37441339)(474.80922701,704.09442034)
\curveto(474.55922573,703.82441394)(474.34422594,703.48441428)(474.16422701,703.07442034)
\curveto(474.11422617,702.95441481)(474.07422621,702.82941493)(474.04422701,702.69942034)
\curveto(474.01422627,702.57941518)(473.97922631,702.45441531)(473.93922701,702.32442034)
\curveto(473.91922637,702.27441549)(473.90922638,702.22441554)(473.90922701,702.17442034)
\curveto(473.90922638,702.13441563)(473.90422638,702.08941567)(473.89422701,702.03942034)
\curveto(473.87422641,701.98941577)(473.86422642,701.93441583)(473.86422701,701.87442034)
\curveto(473.87422641,701.82441594)(473.87422641,701.77441599)(473.86422701,701.72442034)
\lineto(473.86422701,701.61942034)
\curveto(473.84422644,701.5594162)(473.82922646,701.47441629)(473.81922701,701.36442034)
\curveto(473.81922647,701.25441651)(473.82922646,701.16941659)(473.84922701,701.10942034)
\lineto(473.84922701,700.97442034)
\curveto(473.84922644,700.93441683)(473.85422643,700.88941687)(473.86422701,700.83942034)
\curveto(473.8842264,700.759417)(473.89422639,700.67441709)(473.89422701,700.58442034)
\curveto(473.89422639,700.50441726)(473.90422638,700.42441734)(473.92422701,700.34442034)
\curveto(473.94422634,700.29441747)(473.95422633,700.24941751)(473.95422701,700.20942034)
\curveto(473.95422633,700.16941759)(473.96422632,700.12441764)(473.98422701,700.07442034)
\curveto(474.01422627,699.9644178)(474.03922625,699.8594179)(474.05922701,699.75942034)
\curveto(474.0892262,699.6594181)(474.12922616,699.5644182)(474.17922701,699.47442034)
\curveto(474.34922594,699.08441868)(474.55922573,698.74941901)(474.80922701,698.46942034)
\curveto(475.05922523,698.18941957)(475.35922493,697.94441982)(475.70922701,697.73442034)
\curveto(475.81922447,697.67442009)(475.92422436,697.62442014)(476.02422701,697.58442034)
\curveto(476.13422415,697.54442022)(476.24922404,697.50442026)(476.36922701,697.46442034)
\curveto(476.45922383,697.42442034)(476.55422373,697.39442037)(476.65422701,697.37442034)
\curveto(476.75422353,697.35442041)(476.85422343,697.32942043)(476.95422701,697.29942034)
\curveto(477.00422328,697.28942047)(477.04422324,697.28442048)(477.07422701,697.28442034)
\curveto(477.11422317,697.28442048)(477.15422313,697.27942048)(477.19422701,697.26942034)
\curveto(477.24422304,697.24942051)(477.29422299,697.24442052)(477.34422701,697.25442034)
\curveto(477.40422288,697.25442051)(477.45922283,697.24942051)(477.50922701,697.23942034)
\lineto(477.65922701,697.23942034)
\curveto(477.71922257,697.21942054)(477.80422248,697.21442055)(477.91422701,697.22442034)
\curveto(478.02422226,697.22442054)(478.10422218,697.22942053)(478.15422701,697.23942034)
\curveto(478.1842221,697.23942052)(478.21422207,697.24442052)(478.24422701,697.25442034)
\lineto(478.34922701,697.25442034)
\curveto(478.39922189,697.2644205)(478.45422183,697.26942049)(478.51422701,697.26942034)
\curveto(478.57422171,697.26942049)(478.62922166,697.27942048)(478.67922701,697.29942034)
\curveto(478.80922148,697.32942043)(478.93422135,697.3594204)(479.05422701,697.38942034)
\curveto(479.1842211,697.40942035)(479.30922098,697.44442032)(479.42922701,697.49442034)
\curveto(479.90922038,697.69442007)(480.31921997,697.94441982)(480.65922701,698.24442034)
\curveto(480.99921929,698.54441922)(481.27421901,698.93441883)(481.48422701,699.41442034)
\curveto(481.53421875,699.51441825)(481.57421871,699.61941814)(481.60422701,699.72942034)
\curveto(481.63421865,699.84941791)(481.66921862,699.9644178)(481.70922701,700.07442034)
\curveto(481.71921857,700.14441762)(481.72921856,700.20941755)(481.73922701,700.26942034)
\curveto(481.74921854,700.32941743)(481.76421852,700.39441737)(481.78422701,700.46442034)
\curveto(481.80421848,700.54441722)(481.80921848,700.62441714)(481.79922701,700.70442034)
\curveto(481.79921849,700.78441698)(481.80921848,700.8644169)(481.82922701,700.94442034)
\lineto(481.82922701,701.09442034)
\curveto(481.84921844,701.15441661)(481.85921843,701.23941652)(481.85922701,701.34942034)
\curveto(481.85921843,701.4594163)(481.84921844,701.54441622)(481.82922701,701.60442034)
\moveto(479.72922701,701.06442034)
\curveto(479.71922057,701.01441675)(479.71422057,700.9644168)(479.71422701,700.91442034)
\lineto(479.71422701,700.77942034)
\curveto(479.70422058,700.73941702)(479.69922059,700.69941706)(479.69922701,700.65942034)
\curveto(479.69922059,700.62941713)(479.69422059,700.59441717)(479.68422701,700.55442034)
\curveto(479.65422063,700.44441732)(479.62922066,700.33941742)(479.60922701,700.23942034)
\curveto(479.5892207,700.13941762)(479.55922073,700.03941772)(479.51922701,699.93942034)
\curveto(479.40922088,699.68941807)(479.27422101,699.47941828)(479.11422701,699.30942034)
\curveto(478.95422133,699.13941862)(478.74422154,699.00441876)(478.48422701,698.90442034)
\curveto(478.41422187,698.87441889)(478.33922195,698.85441891)(478.25922701,698.84442034)
\curveto(478.17922211,698.83441893)(478.09922219,698.81941894)(478.01922701,698.79942034)
\lineto(477.89922701,698.79942034)
\curveto(477.85922243,698.78941897)(477.81422247,698.78441898)(477.76422701,698.78442034)
\lineto(477.64422701,698.81442034)
\curveto(477.60422268,698.82441894)(477.56922272,698.82441894)(477.53922701,698.81442034)
\curveto(477.50922278,698.81441895)(477.47422281,698.81941894)(477.43422701,698.82942034)
\curveto(477.34422294,698.84941891)(477.25422303,698.87441889)(477.16422701,698.90442034)
\curveto(477.0842232,698.93441883)(477.00922328,698.97441879)(476.93922701,699.02442034)
\curveto(476.6892236,699.17441859)(476.50422378,699.33941842)(476.38422701,699.51942034)
\curveto(476.27422401,699.70941805)(476.16922412,699.95441781)(476.06922701,700.25442034)
\curveto(476.04922424,700.33441743)(476.03422425,700.40941735)(476.02422701,700.47942034)
\curveto(476.01422427,700.5594172)(475.99922429,700.63941712)(475.97922701,700.71942034)
\lineto(475.97922701,700.85442034)
\curveto(475.95922433,700.92441684)(475.94422434,701.02941673)(475.93422701,701.16942034)
\curveto(475.93422435,701.30941645)(475.94422434,701.41441635)(475.96422701,701.48442034)
\lineto(475.96422701,701.63442034)
\curveto(475.96422432,701.68441608)(475.96922432,701.73441603)(475.97922701,701.78442034)
\curveto(475.99922429,701.89441587)(476.01422427,702.00441576)(476.02422701,702.11442034)
\curveto(476.04422424,702.22441554)(476.06922422,702.32941543)(476.09922701,702.42942034)
\curveto(476.1892241,702.69941506)(476.30922398,702.93441483)(476.45922701,703.13442034)
\curveto(476.61922367,703.34441442)(476.82422346,703.50441426)(477.07422701,703.61442034)
\curveto(477.12422316,703.64441412)(477.17922311,703.6644141)(477.23922701,703.67442034)
\lineto(477.44922701,703.73442034)
\curveto(477.47922281,703.74441402)(477.51422277,703.74441402)(477.55422701,703.73442034)
\curveto(477.59422269,703.73441403)(477.62922266,703.74441402)(477.65922701,703.76442034)
\lineto(477.92922701,703.76442034)
\curveto(478.01922227,703.77441399)(478.10422218,703.76941399)(478.18422701,703.74942034)
\curveto(478.25422203,703.72941403)(478.31922197,703.70941405)(478.37922701,703.68942034)
\curveto(478.43922185,703.67941408)(478.49922179,703.6644141)(478.55922701,703.64442034)
\curveto(478.80922148,703.53441423)(479.00922128,703.38441438)(479.15922701,703.19442034)
\curveto(479.30922098,703.01441475)(479.43922085,702.79441497)(479.54922701,702.53442034)
\curveto(479.57922071,702.45441531)(479.59922069,702.36941539)(479.60922701,702.27942034)
\lineto(479.66922701,702.03942034)
\curveto(479.67922061,702.01941574)(479.6842206,701.98941577)(479.68422701,701.94942034)
\curveto(479.69422059,701.89941586)(479.69922059,701.84441592)(479.69922701,701.78442034)
\curveto(479.69922059,701.72441604)(479.70922058,701.66941609)(479.72922701,701.61942034)
\lineto(479.72922701,701.49942034)
\curveto(479.73922055,701.44941631)(479.74422054,701.37441639)(479.74422701,701.27442034)
\curveto(479.74422054,701.18441658)(479.73922055,701.11441665)(479.72922701,701.06442034)
\moveto(478.49922701,708.23442034)
\lineto(479.56422701,708.23442034)
\curveto(479.64422064,708.23440953)(479.73922055,708.23440953)(479.84922701,708.23442034)
\curveto(479.95922033,708.23440953)(480.03922025,708.21940954)(480.08922701,708.18942034)
\curveto(480.10922018,708.17940958)(480.11922017,708.1644096)(480.11922701,708.14442034)
\curveto(480.12922016,708.13440963)(480.14422014,708.12440964)(480.16422701,708.11442034)
\curveto(480.17422011,707.99440977)(480.12422016,707.88940987)(480.01422701,707.79942034)
\curveto(479.91422037,707.70941005)(479.82922046,707.62941013)(479.75922701,707.55942034)
\curveto(479.67922061,707.48941027)(479.59922069,707.41441035)(479.51922701,707.33442034)
\curveto(479.44922084,707.2644105)(479.37422091,707.19941056)(479.29422701,707.13942034)
\curveto(479.25422103,707.10941065)(479.21922107,707.07441069)(479.18922701,707.03442034)
\curveto(479.16922112,707.00441076)(479.13922115,706.97941078)(479.09922701,706.95942034)
\curveto(479.07922121,706.92941083)(479.05422123,706.90441086)(479.02422701,706.88442034)
\lineto(478.87422701,706.73442034)
\lineto(478.72422701,706.61442034)
\lineto(478.67922701,706.56942034)
\curveto(478.67922161,706.5594112)(478.66922162,706.54441122)(478.64922701,706.52442034)
\curveto(478.56922172,706.4644113)(478.4892218,706.39941136)(478.40922701,706.32942034)
\curveto(478.33922195,706.2594115)(478.24922204,706.20441156)(478.13922701,706.16442034)
\curveto(478.09922219,706.15441161)(478.05922223,706.14941161)(478.01922701,706.14942034)
\curveto(477.9892223,706.14941161)(477.94922234,706.14441162)(477.89922701,706.13442034)
\curveto(477.86922242,706.12441164)(477.82922246,706.11941164)(477.77922701,706.11942034)
\curveto(477.72922256,706.12941163)(477.6842226,706.13441163)(477.64422701,706.13442034)
\lineto(477.29922701,706.13442034)
\curveto(477.17922311,706.13441163)(477.0892232,706.1594116)(477.02922701,706.20942034)
\curveto(476.96922332,706.24941151)(476.95422333,706.31941144)(476.98422701,706.41942034)
\curveto(477.00422328,706.49941126)(477.03922325,706.56941119)(477.08922701,706.62942034)
\curveto(477.13922315,706.69941106)(477.1842231,706.76941099)(477.22422701,706.83942034)
\curveto(477.32422296,706.97941078)(477.41922287,707.11441065)(477.50922701,707.24442034)
\curveto(477.59922269,707.37441039)(477.6892226,707.50941025)(477.77922701,707.64942034)
\curveto(477.82922246,707.72941003)(477.87922241,707.81440995)(477.92922701,707.90442034)
\curveto(477.9892223,707.99440977)(478.05422223,708.0644097)(478.12422701,708.11442034)
\curveto(478.16422212,708.14440962)(478.23422205,708.17940958)(478.33422701,708.21942034)
\curveto(478.35422193,708.22940953)(478.37922191,708.22940953)(478.40922701,708.21942034)
\curveto(478.44922184,708.21940954)(478.47922181,708.22440954)(478.49922701,708.23442034)
}
}
{
\newrgbcolor{curcolor}{0 0 0}
\pscustom[linestyle=none,fillstyle=solid,fillcolor=curcolor]
{
\newpath
\moveto(487.65414888,705.35442034)
\curveto(488.25414308,705.37441239)(488.75414258,705.28941247)(489.15414888,705.09942034)
\curveto(489.55414178,704.90941285)(489.86914146,704.62941313)(490.09914888,704.25942034)
\curveto(490.16914116,704.14941361)(490.22414111,704.02941373)(490.26414888,703.89942034)
\curveto(490.30414103,703.77941398)(490.34414099,703.65441411)(490.38414888,703.52442034)
\curveto(490.40414093,703.44441432)(490.41414092,703.36941439)(490.41414888,703.29942034)
\curveto(490.42414091,703.22941453)(490.43914089,703.1594146)(490.45914888,703.08942034)
\curveto(490.45914087,703.02941473)(490.46414087,702.98941477)(490.47414888,702.96942034)
\curveto(490.49414084,702.82941493)(490.50414083,702.68441508)(490.50414888,702.53442034)
\lineto(490.50414888,702.09942034)
\lineto(490.50414888,700.76442034)
\lineto(490.50414888,698.33442034)
\curveto(490.50414083,698.14441962)(490.49914083,697.9594198)(490.48914888,697.77942034)
\curveto(490.48914084,697.60942015)(490.41914091,697.49942026)(490.27914888,697.44942034)
\curveto(490.21914111,697.42942033)(490.14914118,697.41942034)(490.06914888,697.41942034)
\lineto(489.82914888,697.41942034)
\lineto(489.01914888,697.41942034)
\curveto(488.89914243,697.41942034)(488.78914254,697.42442034)(488.68914888,697.43442034)
\curveto(488.59914273,697.45442031)(488.5291428,697.49942026)(488.47914888,697.56942034)
\curveto(488.43914289,697.62942013)(488.41414292,697.70442006)(488.40414888,697.79442034)
\lineto(488.40414888,698.10942034)
\lineto(488.40414888,699.15942034)
\lineto(488.40414888,701.39442034)
\curveto(488.40414293,701.764416)(488.38914294,702.10441566)(488.35914888,702.41442034)
\curveto(488.329143,702.73441503)(488.23914309,703.00441476)(488.08914888,703.22442034)
\curveto(487.94914338,703.42441434)(487.74414359,703.5644142)(487.47414888,703.64442034)
\curveto(487.42414391,703.6644141)(487.36914396,703.67441409)(487.30914888,703.67442034)
\curveto(487.25914407,703.67441409)(487.20414413,703.68441408)(487.14414888,703.70442034)
\curveto(487.09414424,703.71441405)(487.0291443,703.71441405)(486.94914888,703.70442034)
\curveto(486.87914445,703.70441406)(486.82414451,703.69941406)(486.78414888,703.68942034)
\curveto(486.74414459,703.67941408)(486.70914462,703.67441409)(486.67914888,703.67442034)
\curveto(486.64914468,703.67441409)(486.61914471,703.66941409)(486.58914888,703.65942034)
\curveto(486.35914497,703.59941416)(486.17414516,703.51941424)(486.03414888,703.41942034)
\curveto(485.71414562,703.18941457)(485.52414581,702.85441491)(485.46414888,702.41442034)
\curveto(485.40414593,701.97441579)(485.37414596,701.47941628)(485.37414888,700.92942034)
\lineto(485.37414888,699.05442034)
\lineto(485.37414888,698.13942034)
\lineto(485.37414888,697.86942034)
\curveto(485.37414596,697.77941998)(485.35914597,697.70442006)(485.32914888,697.64442034)
\curveto(485.27914605,697.53442023)(485.19914613,697.46942029)(485.08914888,697.44942034)
\curveto(484.97914635,697.42942033)(484.84414649,697.41942034)(484.68414888,697.41942034)
\lineto(483.93414888,697.41942034)
\curveto(483.82414751,697.41942034)(483.71414762,697.42442034)(483.60414888,697.43442034)
\curveto(483.49414784,697.44442032)(483.41414792,697.47942028)(483.36414888,697.53942034)
\curveto(483.29414804,697.62942013)(483.25914807,697.75942)(483.25914888,697.92942034)
\curveto(483.26914806,698.09941966)(483.27414806,698.2594195)(483.27414888,698.40942034)
\lineto(483.27414888,700.44942034)
\lineto(483.27414888,703.74942034)
\lineto(483.27414888,704.51442034)
\lineto(483.27414888,704.81442034)
\curveto(483.28414805,704.90441286)(483.31414802,704.97941278)(483.36414888,705.03942034)
\curveto(483.38414795,705.06941269)(483.41414792,705.08941267)(483.45414888,705.09942034)
\curveto(483.50414783,705.11941264)(483.55414778,705.13441263)(483.60414888,705.14442034)
\lineto(483.67914888,705.14442034)
\curveto(483.7291476,705.15441261)(483.77914755,705.1594126)(483.82914888,705.15942034)
\lineto(483.99414888,705.15942034)
\lineto(484.62414888,705.15942034)
\curveto(484.70414663,705.1594126)(484.77914655,705.15441261)(484.84914888,705.14442034)
\curveto(484.9291464,705.14441262)(484.99914633,705.13441263)(485.05914888,705.11442034)
\curveto(485.1291462,705.08441268)(485.17414616,705.03941272)(485.19414888,704.97942034)
\curveto(485.22414611,704.91941284)(485.24914608,704.84941291)(485.26914888,704.76942034)
\curveto(485.27914605,704.72941303)(485.27914605,704.69441307)(485.26914888,704.66442034)
\curveto(485.26914606,704.63441313)(485.27914605,704.60441316)(485.29914888,704.57442034)
\curveto(485.31914601,704.52441324)(485.334146,704.49441327)(485.34414888,704.48442034)
\curveto(485.36414597,704.47441329)(485.38914594,704.4594133)(485.41914888,704.43942034)
\curveto(485.5291458,704.42941333)(485.61914571,704.4644133)(485.68914888,704.54442034)
\curveto(485.75914557,704.63441313)(485.8341455,704.70441306)(485.91414888,704.75442034)
\curveto(486.18414515,704.95441281)(486.48414485,705.11441265)(486.81414888,705.23442034)
\curveto(486.90414443,705.2644125)(486.99414434,705.28441248)(487.08414888,705.29442034)
\curveto(487.18414415,705.30441246)(487.28914404,705.31941244)(487.39914888,705.33942034)
\curveto(487.4291439,705.34941241)(487.47414386,705.34941241)(487.53414888,705.33942034)
\curveto(487.59414374,705.33941242)(487.6341437,705.34441242)(487.65414888,705.35442034)
}
}
{
\newrgbcolor{curcolor}{0 0 0}
\pscustom[linestyle=none,fillstyle=solid,fillcolor=curcolor]
{
}
}
{
\newrgbcolor{curcolor}{0 0 0}
\pscustom[linestyle=none,fillstyle=solid,fillcolor=curcolor]
{
\newpath
\moveto(503.88555513,698.27442034)
\lineto(503.88555513,697.85442034)
\curveto(503.88554676,697.72442004)(503.85554679,697.61942014)(503.79555513,697.53942034)
\curveto(503.7455469,697.48942027)(503.68054697,697.45442031)(503.60055513,697.43442034)
\curveto(503.52054713,697.42442034)(503.43054722,697.41942034)(503.33055513,697.41942034)
\lineto(502.50555513,697.41942034)
\lineto(502.22055513,697.41942034)
\curveto(502.14054851,697.42942033)(502.07554857,697.45442031)(502.02555513,697.49442034)
\curveto(501.95554869,697.54442022)(501.91554873,697.60942015)(501.90555513,697.68942034)
\curveto(501.89554875,697.76941999)(501.87554877,697.84941991)(501.84555513,697.92942034)
\curveto(501.82554882,697.94941981)(501.80554884,697.9644198)(501.78555513,697.97442034)
\curveto(501.77554887,697.99441977)(501.76054889,698.01441975)(501.74055513,698.03442034)
\curveto(501.63054902,698.03441973)(501.5505491,698.00941975)(501.50055513,697.95942034)
\lineto(501.35055513,697.80942034)
\curveto(501.28054937,697.75942)(501.21554943,697.71442005)(501.15555513,697.67442034)
\curveto(501.09554955,697.64442012)(501.03054962,697.60442016)(500.96055513,697.55442034)
\curveto(500.92054973,697.53442023)(500.87554977,697.51442025)(500.82555513,697.49442034)
\curveto(500.78554986,697.47442029)(500.74054991,697.45442031)(500.69055513,697.43442034)
\curveto(500.5505501,697.38442038)(500.40055025,697.33942042)(500.24055513,697.29942034)
\curveto(500.19055046,697.27942048)(500.1455505,697.26942049)(500.10555513,697.26942034)
\curveto(500.06555058,697.26942049)(500.02555062,697.2644205)(499.98555513,697.25442034)
\lineto(499.85055513,697.25442034)
\curveto(499.82055083,697.24442052)(499.78055087,697.23942052)(499.73055513,697.23942034)
\lineto(499.59555513,697.23942034)
\curveto(499.53555111,697.21942054)(499.4455512,697.21442055)(499.32555513,697.22442034)
\curveto(499.20555144,697.22442054)(499.12055153,697.23442053)(499.07055513,697.25442034)
\curveto(499.00055165,697.27442049)(498.93555171,697.28442048)(498.87555513,697.28442034)
\curveto(498.82555182,697.27442049)(498.77055188,697.27942048)(498.71055513,697.29942034)
\lineto(498.35055513,697.41942034)
\curveto(498.24055241,697.44942031)(498.13055252,697.48942027)(498.02055513,697.53942034)
\curveto(497.67055298,697.68942007)(497.35555329,697.91941984)(497.07555513,698.22942034)
\curveto(496.80555384,698.54941921)(496.59055406,698.88441888)(496.43055513,699.23442034)
\curveto(496.38055427,699.34441842)(496.34055431,699.44941831)(496.31055513,699.54942034)
\curveto(496.28055437,699.6594181)(496.2455544,699.76941799)(496.20555513,699.87942034)
\curveto(496.19555445,699.91941784)(496.19055446,699.95441781)(496.19055513,699.98442034)
\curveto(496.19055446,700.02441774)(496.18055447,700.06941769)(496.16055513,700.11942034)
\curveto(496.14055451,700.19941756)(496.12055453,700.28441748)(496.10055513,700.37442034)
\curveto(496.09055456,700.47441729)(496.07555457,700.57441719)(496.05555513,700.67442034)
\curveto(496.0455546,700.70441706)(496.04055461,700.73941702)(496.04055513,700.77942034)
\curveto(496.0505546,700.81941694)(496.0505546,700.85441691)(496.04055513,700.88442034)
\lineto(496.04055513,701.01942034)
\curveto(496.04055461,701.06941669)(496.03555461,701.11941664)(496.02555513,701.16942034)
\curveto(496.01555463,701.21941654)(496.01055464,701.27441649)(496.01055513,701.33442034)
\curveto(496.01055464,701.40441636)(496.01555463,701.4594163)(496.02555513,701.49942034)
\curveto(496.03555461,701.54941621)(496.04055461,701.59441617)(496.04055513,701.63442034)
\lineto(496.04055513,701.78442034)
\curveto(496.0505546,701.83441593)(496.0505546,701.87941588)(496.04055513,701.91942034)
\curveto(496.04055461,701.96941579)(496.0505546,702.01941574)(496.07055513,702.06942034)
\curveto(496.09055456,702.17941558)(496.10555454,702.28441548)(496.11555513,702.38442034)
\curveto(496.13555451,702.48441528)(496.16055449,702.58441518)(496.19055513,702.68442034)
\curveto(496.23055442,702.80441496)(496.26555438,702.91941484)(496.29555513,703.02942034)
\curveto(496.32555432,703.13941462)(496.36555428,703.24941451)(496.41555513,703.35942034)
\curveto(496.55555409,703.6594141)(496.73055392,703.94441382)(496.94055513,704.21442034)
\curveto(496.96055369,704.24441352)(496.98555366,704.26941349)(497.01555513,704.28942034)
\curveto(497.05555359,704.31941344)(497.08555356,704.34941341)(497.10555513,704.37942034)
\curveto(497.1455535,704.42941333)(497.18555346,704.47441329)(497.22555513,704.51442034)
\curveto(497.26555338,704.55441321)(497.31055334,704.59441317)(497.36055513,704.63442034)
\curveto(497.40055325,704.65441311)(497.43555321,704.67941308)(497.46555513,704.70942034)
\curveto(497.49555315,704.74941301)(497.53055312,704.77941298)(497.57055513,704.79942034)
\curveto(497.82055283,704.96941279)(498.11055254,705.10941265)(498.44055513,705.21942034)
\curveto(498.51055214,705.23941252)(498.58055207,705.25441251)(498.65055513,705.26442034)
\curveto(498.73055192,705.27441249)(498.81055184,705.28941247)(498.89055513,705.30942034)
\curveto(498.96055169,705.32941243)(499.0505516,705.33941242)(499.16055513,705.33942034)
\curveto(499.27055138,705.34941241)(499.38055127,705.35441241)(499.49055513,705.35442034)
\curveto(499.60055105,705.35441241)(499.70555094,705.34941241)(499.80555513,705.33942034)
\curveto(499.91555073,705.32941243)(500.00555064,705.31441245)(500.07555513,705.29442034)
\curveto(500.22555042,705.24441252)(500.37055028,705.19941256)(500.51055513,705.15942034)
\curveto(500.65055,705.11941264)(500.78054987,705.0644127)(500.90055513,704.99442034)
\curveto(500.97054968,704.94441282)(501.03554961,704.89441287)(501.09555513,704.84442034)
\curveto(501.15554949,704.80441296)(501.22054943,704.759413)(501.29055513,704.70942034)
\curveto(501.33054932,704.67941308)(501.38554926,704.63941312)(501.45555513,704.58942034)
\curveto(501.53554911,704.53941322)(501.61054904,704.53941322)(501.68055513,704.58942034)
\curveto(501.72054893,704.60941315)(501.74054891,704.64441312)(501.74055513,704.69442034)
\curveto(501.74054891,704.74441302)(501.7505489,704.79441297)(501.77055513,704.84442034)
\lineto(501.77055513,704.99442034)
\curveto(501.78054887,705.02441274)(501.78554886,705.0594127)(501.78555513,705.09942034)
\lineto(501.78555513,705.21942034)
\lineto(501.78555513,707.25942034)
\curveto(501.78554886,707.36941039)(501.78054887,707.48941027)(501.77055513,707.61942034)
\curveto(501.77054888,707.75941)(501.79554885,707.8644099)(501.84555513,707.93442034)
\curveto(501.88554876,708.01440975)(501.96054869,708.0644097)(502.07055513,708.08442034)
\curveto(502.09054856,708.09440967)(502.11054854,708.09440967)(502.13055513,708.08442034)
\curveto(502.1505485,708.08440968)(502.17054848,708.08940967)(502.19055513,708.09942034)
\lineto(503.25555513,708.09942034)
\curveto(503.37554727,708.09940966)(503.48554716,708.09440967)(503.58555513,708.08442034)
\curveto(503.68554696,708.07440969)(503.76054689,708.03440973)(503.81055513,707.96442034)
\curveto(503.86054679,707.88440988)(503.88554676,707.77940998)(503.88555513,707.64942034)
\lineto(503.88555513,707.28942034)
\lineto(503.88555513,698.27442034)
\moveto(501.84555513,701.21442034)
\curveto(501.85554879,701.25441651)(501.85554879,701.29441647)(501.84555513,701.33442034)
\lineto(501.84555513,701.46942034)
\curveto(501.8455488,701.56941619)(501.84054881,701.66941609)(501.83055513,701.76942034)
\curveto(501.82054883,701.86941589)(501.80554884,701.9594158)(501.78555513,702.03942034)
\curveto(501.76554888,702.14941561)(501.7455489,702.24941551)(501.72555513,702.33942034)
\curveto(501.71554893,702.42941533)(501.69054896,702.51441525)(501.65055513,702.59442034)
\curveto(501.51054914,702.95441481)(501.30554934,703.23941452)(501.03555513,703.44942034)
\curveto(500.77554987,703.6594141)(500.39555025,703.764414)(499.89555513,703.76442034)
\curveto(499.83555081,703.764414)(499.75555089,703.75441401)(499.65555513,703.73442034)
\curveto(499.57555107,703.71441405)(499.50055115,703.69441407)(499.43055513,703.67442034)
\curveto(499.37055128,703.6644141)(499.31055134,703.64441412)(499.25055513,703.61442034)
\curveto(498.98055167,703.50441426)(498.77055188,703.33441443)(498.62055513,703.10442034)
\curveto(498.47055218,702.87441489)(498.3505523,702.61441515)(498.26055513,702.32442034)
\curveto(498.23055242,702.22441554)(498.21055244,702.12441564)(498.20055513,702.02442034)
\curveto(498.19055246,701.92441584)(498.17055248,701.81941594)(498.14055513,701.70942034)
\lineto(498.14055513,701.49942034)
\curveto(498.12055253,701.40941635)(498.11555253,701.28441648)(498.12555513,701.12442034)
\curveto(498.13555251,700.97441679)(498.1505525,700.8644169)(498.17055513,700.79442034)
\lineto(498.17055513,700.70442034)
\curveto(498.18055247,700.68441708)(498.18555246,700.6644171)(498.18555513,700.64442034)
\curveto(498.20555244,700.5644172)(498.22055243,700.48941727)(498.23055513,700.41942034)
\curveto(498.2505524,700.34941741)(498.27055238,700.27441749)(498.29055513,700.19442034)
\curveto(498.46055219,699.67441809)(498.7505519,699.28941847)(499.16055513,699.03942034)
\curveto(499.29055136,698.94941881)(499.47055118,698.87941888)(499.70055513,698.82942034)
\curveto(499.74055091,698.81941894)(499.80055085,698.81441895)(499.88055513,698.81442034)
\curveto(499.91055074,698.80441896)(499.95555069,698.79441897)(500.01555513,698.78442034)
\curveto(500.08555056,698.78441898)(500.14055051,698.78941897)(500.18055513,698.79942034)
\curveto(500.26055039,698.81941894)(500.34055031,698.83441893)(500.42055513,698.84442034)
\curveto(500.50055015,698.85441891)(500.58055007,698.87441889)(500.66055513,698.90442034)
\curveto(500.91054974,699.01441875)(501.11054954,699.15441861)(501.26055513,699.32442034)
\curveto(501.41054924,699.49441827)(501.54054911,699.70941805)(501.65055513,699.96942034)
\curveto(501.69054896,700.0594177)(501.72054893,700.14941761)(501.74055513,700.23942034)
\curveto(501.76054889,700.33941742)(501.78054887,700.44441732)(501.80055513,700.55442034)
\curveto(501.81054884,700.60441716)(501.81054884,700.64941711)(501.80055513,700.68942034)
\curveto(501.80054885,700.73941702)(501.81054884,700.78941697)(501.83055513,700.83942034)
\curveto(501.84054881,700.86941689)(501.8455488,700.90441686)(501.84555513,700.94442034)
\lineto(501.84555513,701.07942034)
\lineto(501.84555513,701.21442034)
}
}
{
\newrgbcolor{curcolor}{0 0 0}
\pscustom[linestyle=none,fillstyle=solid,fillcolor=curcolor]
{
\newpath
\moveto(512.83047701,701.36442034)
\curveto(512.85046884,701.28441648)(512.85046884,701.19441657)(512.83047701,701.09442034)
\curveto(512.81046888,700.99441677)(512.77546892,700.92941683)(512.72547701,700.89942034)
\curveto(512.67546902,700.8594169)(512.60046909,700.82941693)(512.50047701,700.80942034)
\curveto(512.41046928,700.79941696)(512.30546939,700.78941697)(512.18547701,700.77942034)
\lineto(511.84047701,700.77942034)
\curveto(511.73046996,700.78941697)(511.63047006,700.79441697)(511.54047701,700.79442034)
\lineto(507.88047701,700.79442034)
\lineto(507.67047701,700.79442034)
\curveto(507.61047408,700.79441697)(507.55547414,700.78441698)(507.50547701,700.76442034)
\curveto(507.42547427,700.72441704)(507.37547432,700.68441708)(507.35547701,700.64442034)
\curveto(507.33547436,700.62441714)(507.31547438,700.58441718)(507.29547701,700.52442034)
\curveto(507.27547442,700.47441729)(507.27047442,700.42441734)(507.28047701,700.37442034)
\curveto(507.30047439,700.31441745)(507.31047438,700.25441751)(507.31047701,700.19442034)
\curveto(507.32047437,700.14441762)(507.33547436,700.08941767)(507.35547701,700.02942034)
\curveto(507.43547426,699.78941797)(507.53047416,699.58941817)(507.64047701,699.42942034)
\curveto(507.76047393,699.27941848)(507.92047377,699.14441862)(508.12047701,699.02442034)
\curveto(508.20047349,698.97441879)(508.28047341,698.93941882)(508.36047701,698.91942034)
\curveto(508.45047324,698.90941885)(508.54047315,698.88941887)(508.63047701,698.85942034)
\curveto(508.71047298,698.83941892)(508.82047287,698.82441894)(508.96047701,698.81442034)
\curveto(509.10047259,698.80441896)(509.22047247,698.80941895)(509.32047701,698.82942034)
\lineto(509.45547701,698.82942034)
\curveto(509.55547214,698.84941891)(509.64547205,698.86941889)(509.72547701,698.88942034)
\curveto(509.81547188,698.91941884)(509.90047179,698.94941881)(509.98047701,698.97942034)
\curveto(510.08047161,699.02941873)(510.1904715,699.09441867)(510.31047701,699.17442034)
\curveto(510.44047125,699.25441851)(510.53547116,699.33441843)(510.59547701,699.41442034)
\curveto(510.64547105,699.48441828)(510.695471,699.54941821)(510.74547701,699.60942034)
\curveto(510.80547089,699.67941808)(510.87547082,699.72941803)(510.95547701,699.75942034)
\curveto(511.05547064,699.80941795)(511.18047051,699.82941793)(511.33047701,699.81942034)
\lineto(511.76547701,699.81942034)
\lineto(511.94547701,699.81942034)
\curveto(512.01546968,699.82941793)(512.07546962,699.82441794)(512.12547701,699.80442034)
\lineto(512.27547701,699.80442034)
\curveto(512.37546932,699.78441798)(512.44546925,699.759418)(512.48547701,699.72942034)
\curveto(512.52546917,699.70941805)(512.54546915,699.6644181)(512.54547701,699.59442034)
\curveto(512.55546914,699.52441824)(512.55046914,699.4644183)(512.53047701,699.41442034)
\curveto(512.48046921,699.27441849)(512.42546927,699.14941861)(512.36547701,699.03942034)
\curveto(512.30546939,698.92941883)(512.23546946,698.81941894)(512.15547701,698.70942034)
\curveto(511.93546976,698.37941938)(511.68547001,698.11441965)(511.40547701,697.91442034)
\curveto(511.12547057,697.71442005)(510.77547092,697.54442022)(510.35547701,697.40442034)
\curveto(510.24547145,697.3644204)(510.13547156,697.33942042)(510.02547701,697.32942034)
\curveto(509.91547178,697.31942044)(509.80047189,697.29942046)(509.68047701,697.26942034)
\curveto(509.64047205,697.2594205)(509.5954721,697.2594205)(509.54547701,697.26942034)
\curveto(509.50547219,697.26942049)(509.46547223,697.2644205)(509.42547701,697.25442034)
\lineto(509.26047701,697.25442034)
\curveto(509.21047248,697.23442053)(509.15047254,697.22942053)(509.08047701,697.23942034)
\curveto(509.02047267,697.23942052)(508.96547273,697.24442052)(508.91547701,697.25442034)
\curveto(508.83547286,697.2644205)(508.76547293,697.2644205)(508.70547701,697.25442034)
\curveto(508.64547305,697.24442052)(508.58047311,697.24942051)(508.51047701,697.26942034)
\curveto(508.46047323,697.28942047)(508.40547329,697.29942046)(508.34547701,697.29942034)
\curveto(508.28547341,697.29942046)(508.23047346,697.30942045)(508.18047701,697.32942034)
\curveto(508.07047362,697.34942041)(507.96047373,697.37442039)(507.85047701,697.40442034)
\curveto(507.74047395,697.42442034)(507.64047405,697.4594203)(507.55047701,697.50942034)
\curveto(507.44047425,697.54942021)(507.33547436,697.58442018)(507.23547701,697.61442034)
\curveto(507.14547455,697.65442011)(507.06047463,697.69942006)(506.98047701,697.74942034)
\curveto(506.66047503,697.94941981)(506.37547532,698.17941958)(506.12547701,698.43942034)
\curveto(505.87547582,698.70941905)(505.67047602,699.01941874)(505.51047701,699.36942034)
\curveto(505.46047623,699.47941828)(505.42047627,699.58941817)(505.39047701,699.69942034)
\curveto(505.36047633,699.81941794)(505.32047637,699.93941782)(505.27047701,700.05942034)
\curveto(505.26047643,700.09941766)(505.25547644,700.13441763)(505.25547701,700.16442034)
\curveto(505.25547644,700.20441756)(505.25047644,700.24441752)(505.24047701,700.28442034)
\curveto(505.20047649,700.40441736)(505.17547652,700.53441723)(505.16547701,700.67442034)
\lineto(505.13547701,701.09442034)
\curveto(505.13547656,701.14441662)(505.13047656,701.19941656)(505.12047701,701.25942034)
\curveto(505.12047657,701.31941644)(505.12547657,701.37441639)(505.13547701,701.42442034)
\lineto(505.13547701,701.60442034)
\lineto(505.18047701,701.96442034)
\curveto(505.22047647,702.13441563)(505.25547644,702.29941546)(505.28547701,702.45942034)
\curveto(505.31547638,702.61941514)(505.36047633,702.76941499)(505.42047701,702.90942034)
\curveto(505.85047584,703.94941381)(506.58047511,704.68441308)(507.61047701,705.11442034)
\curveto(507.75047394,705.17441259)(507.8904738,705.21441255)(508.03047701,705.23442034)
\curveto(508.18047351,705.2644125)(508.33547336,705.29941246)(508.49547701,705.33942034)
\curveto(508.57547312,705.34941241)(508.65047304,705.35441241)(508.72047701,705.35442034)
\curveto(508.7904729,705.35441241)(508.86547283,705.3594124)(508.94547701,705.36942034)
\curveto(509.45547224,705.37941238)(509.8904718,705.31941244)(510.25047701,705.18942034)
\curveto(510.62047107,705.06941269)(510.95047074,704.90941285)(511.24047701,704.70942034)
\curveto(511.33047036,704.64941311)(511.42047027,704.57941318)(511.51047701,704.49942034)
\curveto(511.60047009,704.42941333)(511.68047001,704.35441341)(511.75047701,704.27442034)
\curveto(511.78046991,704.22441354)(511.82046987,704.18441358)(511.87047701,704.15442034)
\curveto(511.95046974,704.04441372)(512.02546967,703.92941383)(512.09547701,703.80942034)
\curveto(512.16546953,703.69941406)(512.24046945,703.58441418)(512.32047701,703.46442034)
\curveto(512.37046932,703.37441439)(512.41046928,703.27941448)(512.44047701,703.17942034)
\curveto(512.48046921,703.08941467)(512.52046917,702.98941477)(512.56047701,702.87942034)
\curveto(512.61046908,702.74941501)(512.65046904,702.61441515)(512.68047701,702.47442034)
\curveto(512.71046898,702.33441543)(512.74546895,702.19441557)(512.78547701,702.05442034)
\curveto(512.80546889,701.97441579)(512.81046888,701.88441588)(512.80047701,701.78442034)
\curveto(512.80046889,701.69441607)(512.81046888,701.60941615)(512.83047701,701.52942034)
\lineto(512.83047701,701.36442034)
\moveto(510.58047701,702.24942034)
\curveto(510.65047104,702.34941541)(510.65547104,702.46941529)(510.59547701,702.60942034)
\curveto(510.54547115,702.759415)(510.50547119,702.86941489)(510.47547701,702.93942034)
\curveto(510.33547136,703.20941455)(510.15047154,703.41441435)(509.92047701,703.55442034)
\curveto(509.690472,703.70441406)(509.37047232,703.78441398)(508.96047701,703.79442034)
\curveto(508.93047276,703.77441399)(508.8954728,703.76941399)(508.85547701,703.77942034)
\curveto(508.81547288,703.78941397)(508.78047291,703.78941397)(508.75047701,703.77942034)
\curveto(508.70047299,703.759414)(508.64547305,703.74441402)(508.58547701,703.73442034)
\curveto(508.52547317,703.73441403)(508.47047322,703.72441404)(508.42047701,703.70442034)
\curveto(507.98047371,703.5644142)(507.65547404,703.28941447)(507.44547701,702.87942034)
\curveto(507.42547427,702.83941492)(507.40047429,702.78441498)(507.37047701,702.71442034)
\curveto(507.35047434,702.65441511)(507.33547436,702.58941517)(507.32547701,702.51942034)
\curveto(507.31547438,702.4594153)(507.31547438,702.39941536)(507.32547701,702.33942034)
\curveto(507.34547435,702.27941548)(507.38047431,702.22941553)(507.43047701,702.18942034)
\curveto(507.51047418,702.13941562)(507.62047407,702.11441565)(507.76047701,702.11442034)
\lineto(508.16547701,702.11442034)
\lineto(509.83047701,702.11442034)
\lineto(510.26547701,702.11442034)
\curveto(510.42547127,702.12441564)(510.53047116,702.16941559)(510.58047701,702.24942034)
}
}
{
\newrgbcolor{curcolor}{0 0 0}
\pscustom[linestyle=none,fillstyle=solid,fillcolor=curcolor]
{
}
}
{
\newrgbcolor{curcolor}{0 0 0}
\pscustom[linestyle=none,fillstyle=solid,fillcolor=curcolor]
{
\newpath
\moveto(518.74891451,708.11442034)
\lineto(519.84391451,708.11442034)
\curveto(519.94391202,708.11440965)(520.03891193,708.10940965)(520.12891451,708.09942034)
\curveto(520.21891175,708.08940967)(520.28891168,708.0594097)(520.33891451,708.00942034)
\curveto(520.39891157,707.93940982)(520.42891154,707.84440992)(520.42891451,707.72442034)
\curveto(520.43891153,707.61441015)(520.44391152,707.49941026)(520.44391451,707.37942034)
\lineto(520.44391451,706.04442034)
\lineto(520.44391451,700.65942034)
\lineto(520.44391451,698.36442034)
\lineto(520.44391451,697.94442034)
\curveto(520.45391151,697.79441997)(520.43391153,697.67942008)(520.38391451,697.59942034)
\curveto(520.33391163,697.51942024)(520.24391172,697.4644203)(520.11391451,697.43442034)
\curveto(520.05391191,697.41442035)(519.98391198,697.40942035)(519.90391451,697.41942034)
\curveto(519.83391213,697.42942033)(519.7639122,697.43442033)(519.69391451,697.43442034)
\lineto(518.97391451,697.43442034)
\curveto(518.8639131,697.43442033)(518.7639132,697.43942032)(518.67391451,697.44942034)
\curveto(518.58391338,697.4594203)(518.50891346,697.48942027)(518.44891451,697.53942034)
\curveto(518.38891358,697.58942017)(518.35391361,697.6644201)(518.34391451,697.76442034)
\lineto(518.34391451,698.09442034)
\lineto(518.34391451,699.42942034)
\lineto(518.34391451,705.05442034)
\lineto(518.34391451,707.09442034)
\curveto(518.34391362,707.22441054)(518.33891363,707.37941038)(518.32891451,707.55942034)
\curveto(518.32891364,707.73941002)(518.35391361,707.86940989)(518.40391451,707.94942034)
\curveto(518.42391354,707.98940977)(518.44891352,708.01940974)(518.47891451,708.03942034)
\lineto(518.59891451,708.09942034)
\curveto(518.61891335,708.09940966)(518.64391332,708.09940966)(518.67391451,708.09942034)
\curveto(518.70391326,708.10940965)(518.72891324,708.11440965)(518.74891451,708.11442034)
}
}
{
\newrgbcolor{curcolor}{0 0 0}
\pscustom[linestyle=none,fillstyle=solid,fillcolor=curcolor]
{
\newpath
\moveto(529.15610201,698.01942034)
\curveto(529.17609416,697.90941985)(529.18609415,697.79941996)(529.18610201,697.68942034)
\curveto(529.19609414,697.57942018)(529.14609419,697.50442026)(529.03610201,697.46442034)
\curveto(528.97609436,697.43442033)(528.90609443,697.41942034)(528.82610201,697.41942034)
\lineto(528.58610201,697.41942034)
\lineto(527.77610201,697.41942034)
\lineto(527.50610201,697.41942034)
\curveto(527.42609591,697.42942033)(527.36109597,697.45442031)(527.31110201,697.49442034)
\curveto(527.24109609,697.53442023)(527.18609615,697.58942017)(527.14610201,697.65942034)
\curveto(527.11609622,697.73942002)(527.07109626,697.80441996)(527.01110201,697.85442034)
\curveto(526.99109634,697.87441989)(526.96609637,697.88941987)(526.93610201,697.89942034)
\curveto(526.90609643,697.91941984)(526.86609647,697.92441984)(526.81610201,697.91442034)
\curveto(526.76609657,697.89441987)(526.71609662,697.86941989)(526.66610201,697.83942034)
\curveto(526.62609671,697.80941995)(526.58109675,697.78441998)(526.53110201,697.76442034)
\curveto(526.48109685,697.72442004)(526.42609691,697.68942007)(526.36610201,697.65942034)
\lineto(526.18610201,697.56942034)
\curveto(526.05609728,697.50942025)(525.92109741,697.4594203)(525.78110201,697.41942034)
\curveto(525.64109769,697.38942037)(525.49609784,697.35442041)(525.34610201,697.31442034)
\curveto(525.27609806,697.29442047)(525.20609813,697.28442048)(525.13610201,697.28442034)
\curveto(525.07609826,697.27442049)(525.01109832,697.2644205)(524.94110201,697.25442034)
\lineto(524.85110201,697.25442034)
\curveto(524.82109851,697.24442052)(524.79109854,697.23942052)(524.76110201,697.23942034)
\lineto(524.59610201,697.23942034)
\curveto(524.49609884,697.21942054)(524.39609894,697.21942054)(524.29610201,697.23942034)
\lineto(524.16110201,697.23942034)
\curveto(524.09109924,697.2594205)(524.02109931,697.26942049)(523.95110201,697.26942034)
\curveto(523.89109944,697.2594205)(523.8310995,697.2644205)(523.77110201,697.28442034)
\curveto(523.67109966,697.30442046)(523.57609976,697.32442044)(523.48610201,697.34442034)
\curveto(523.39609994,697.35442041)(523.31110002,697.37942038)(523.23110201,697.41942034)
\curveto(522.94110039,697.52942023)(522.69110064,697.66942009)(522.48110201,697.83942034)
\curveto(522.28110105,698.01941974)(522.12110121,698.25441951)(522.00110201,698.54442034)
\curveto(521.97110136,698.61441915)(521.94110139,698.68941907)(521.91110201,698.76942034)
\curveto(521.89110144,698.84941891)(521.87110146,698.93441883)(521.85110201,699.02442034)
\curveto(521.8311015,699.07441869)(521.82110151,699.12441864)(521.82110201,699.17442034)
\curveto(521.8311015,699.22441854)(521.8311015,699.27441849)(521.82110201,699.32442034)
\curveto(521.81110152,699.35441841)(521.80110153,699.41441835)(521.79110201,699.50442034)
\curveto(521.79110154,699.60441816)(521.79610154,699.67441809)(521.80610201,699.71442034)
\curveto(521.82610151,699.81441795)(521.8361015,699.89941786)(521.83610201,699.96942034)
\lineto(521.92610201,700.29942034)
\curveto(521.95610138,700.41941734)(521.99610134,700.52441724)(522.04610201,700.61442034)
\curveto(522.21610112,700.90441686)(522.41110092,701.12441664)(522.63110201,701.27442034)
\curveto(522.85110048,701.42441634)(523.1311002,701.55441621)(523.47110201,701.66442034)
\curveto(523.60109973,701.71441605)(523.7360996,701.74941601)(523.87610201,701.76942034)
\curveto(524.01609932,701.78941597)(524.15609918,701.81441595)(524.29610201,701.84442034)
\curveto(524.37609896,701.8644159)(524.46109887,701.87441589)(524.55110201,701.87442034)
\curveto(524.64109869,701.88441588)(524.7310986,701.89941586)(524.82110201,701.91942034)
\curveto(524.89109844,701.93941582)(524.96109837,701.94441582)(525.03110201,701.93442034)
\curveto(525.10109823,701.93441583)(525.17609816,701.94441582)(525.25610201,701.96442034)
\curveto(525.32609801,701.98441578)(525.39609794,701.99441577)(525.46610201,701.99442034)
\curveto(525.5360978,701.99441577)(525.61109772,702.00441576)(525.69110201,702.02442034)
\curveto(525.90109743,702.07441569)(526.09109724,702.11441565)(526.26110201,702.14442034)
\curveto(526.44109689,702.18441558)(526.60109673,702.27441549)(526.74110201,702.41442034)
\curveto(526.8310965,702.50441526)(526.89109644,702.60441516)(526.92110201,702.71442034)
\curveto(526.9310964,702.74441502)(526.9310964,702.76941499)(526.92110201,702.78942034)
\curveto(526.92109641,702.80941495)(526.92609641,702.82941493)(526.93610201,702.84942034)
\curveto(526.94609639,702.86941489)(526.95109638,702.89941486)(526.95110201,702.93942034)
\lineto(526.95110201,703.02942034)
\lineto(526.92110201,703.14942034)
\curveto(526.92109641,703.18941457)(526.91609642,703.22441454)(526.90610201,703.25442034)
\curveto(526.80609653,703.55441421)(526.59609674,703.759414)(526.27610201,703.86942034)
\curveto(526.18609715,703.89941386)(526.07609726,703.91941384)(525.94610201,703.92942034)
\curveto(525.82609751,703.94941381)(525.70109763,703.95441381)(525.57110201,703.94442034)
\curveto(525.44109789,703.94441382)(525.31609802,703.93441383)(525.19610201,703.91442034)
\curveto(525.07609826,703.89441387)(524.97109836,703.86941389)(524.88110201,703.83942034)
\curveto(524.82109851,703.81941394)(524.76109857,703.78941397)(524.70110201,703.74942034)
\curveto(524.65109868,703.71941404)(524.60109873,703.68441408)(524.55110201,703.64442034)
\curveto(524.50109883,703.60441416)(524.44609889,703.54941421)(524.38610201,703.47942034)
\curveto(524.336099,703.40941435)(524.30109903,703.34441442)(524.28110201,703.28442034)
\curveto(524.2310991,703.18441458)(524.18609915,703.08941467)(524.14610201,702.99942034)
\curveto(524.11609922,702.90941485)(524.04609929,702.84941491)(523.93610201,702.81942034)
\curveto(523.85609948,702.79941496)(523.77109956,702.78941497)(523.68110201,702.78942034)
\lineto(523.41110201,702.78942034)
\lineto(522.84110201,702.78942034)
\curveto(522.79110054,702.78941497)(522.74110059,702.78441498)(522.69110201,702.77442034)
\curveto(522.64110069,702.77441499)(522.59610074,702.77941498)(522.55610201,702.78942034)
\lineto(522.42110201,702.78942034)
\curveto(522.40110093,702.79941496)(522.37610096,702.80441496)(522.34610201,702.80442034)
\curveto(522.31610102,702.80441496)(522.29110104,702.81441495)(522.27110201,702.83442034)
\curveto(522.19110114,702.85441491)(522.1361012,702.91941484)(522.10610201,703.02942034)
\curveto(522.09610124,703.07941468)(522.09610124,703.12941463)(522.10610201,703.17942034)
\curveto(522.11610122,703.22941453)(522.12610121,703.27441449)(522.13610201,703.31442034)
\curveto(522.16610117,703.42441434)(522.19610114,703.52441424)(522.22610201,703.61442034)
\curveto(522.26610107,703.71441405)(522.31110102,703.80441396)(522.36110201,703.88442034)
\lineto(522.45110201,704.03442034)
\lineto(522.54110201,704.18442034)
\curveto(522.62110071,704.29441347)(522.72110061,704.39941336)(522.84110201,704.49942034)
\curveto(522.86110047,704.50941325)(522.89110044,704.53441323)(522.93110201,704.57442034)
\curveto(522.98110035,704.61441315)(523.02610031,704.64941311)(523.06610201,704.67942034)
\curveto(523.10610023,704.70941305)(523.15110018,704.73941302)(523.20110201,704.76942034)
\curveto(523.37109996,704.87941288)(523.55109978,704.9644128)(523.74110201,705.02442034)
\curveto(523.9310994,705.09441267)(524.12609921,705.1594126)(524.32610201,705.21942034)
\curveto(524.44609889,705.24941251)(524.57109876,705.26941249)(524.70110201,705.27942034)
\curveto(524.8310985,705.28941247)(524.96109837,705.30941245)(525.09110201,705.33942034)
\curveto(525.1310982,705.34941241)(525.19109814,705.34941241)(525.27110201,705.33942034)
\curveto(525.36109797,705.32941243)(525.41609792,705.33441243)(525.43610201,705.35442034)
\curveto(525.84609749,705.3644124)(526.2360971,705.34941241)(526.60610201,705.30942034)
\curveto(526.98609635,705.26941249)(527.32609601,705.19441257)(527.62610201,705.08442034)
\curveto(527.9360954,704.97441279)(528.20109513,704.82441294)(528.42110201,704.63442034)
\curveto(528.64109469,704.45441331)(528.81109452,704.21941354)(528.93110201,703.92942034)
\curveto(529.00109433,703.759414)(529.04109429,703.5644142)(529.05110201,703.34442034)
\curveto(529.06109427,703.12441464)(529.06609427,702.89941486)(529.06610201,702.66942034)
\lineto(529.06610201,699.32442034)
\lineto(529.06610201,698.73942034)
\curveto(529.06609427,698.54941921)(529.08609425,698.37441939)(529.12610201,698.21442034)
\curveto(529.1360942,698.18441958)(529.14109419,698.14941961)(529.14110201,698.10942034)
\curveto(529.14109419,698.07941968)(529.14609419,698.04941971)(529.15610201,698.01942034)
\moveto(526.95110201,700.32942034)
\curveto(526.96109637,700.37941738)(526.96609637,700.43441733)(526.96610201,700.49442034)
\curveto(526.96609637,700.5644172)(526.96109637,700.62441714)(526.95110201,700.67442034)
\curveto(526.9310964,700.73441703)(526.92109641,700.78941697)(526.92110201,700.83942034)
\curveto(526.92109641,700.88941687)(526.90109643,700.92941683)(526.86110201,700.95942034)
\curveto(526.81109652,700.99941676)(526.7360966,701.01941674)(526.63610201,701.01942034)
\curveto(526.59609674,701.00941675)(526.56109677,700.99941676)(526.53110201,700.98942034)
\curveto(526.50109683,700.98941677)(526.46609687,700.98441678)(526.42610201,700.97442034)
\curveto(526.35609698,700.95441681)(526.28109705,700.93941682)(526.20110201,700.92942034)
\curveto(526.12109721,700.91941684)(526.04109729,700.90441686)(525.96110201,700.88442034)
\curveto(525.9310974,700.87441689)(525.88609745,700.86941689)(525.82610201,700.86942034)
\curveto(525.69609764,700.83941692)(525.56609777,700.81941694)(525.43610201,700.80942034)
\curveto(525.30609803,700.79941696)(525.18109815,700.77441699)(525.06110201,700.73442034)
\curveto(524.98109835,700.71441705)(524.90609843,700.69441707)(524.83610201,700.67442034)
\curveto(524.76609857,700.6644171)(524.69609864,700.64441712)(524.62610201,700.61442034)
\curveto(524.41609892,700.52441724)(524.2360991,700.38941737)(524.08610201,700.20942034)
\curveto(523.94609939,700.02941773)(523.89609944,699.77941798)(523.93610201,699.45942034)
\curveto(523.95609938,699.28941847)(524.01109932,699.14941861)(524.10110201,699.03942034)
\curveto(524.17109916,698.92941883)(524.27609906,698.83941892)(524.41610201,698.76942034)
\curveto(524.55609878,698.70941905)(524.70609863,698.6644191)(524.86610201,698.63442034)
\curveto(525.0360983,698.60441916)(525.21109812,698.59441917)(525.39110201,698.60442034)
\curveto(525.58109775,698.62441914)(525.75609758,698.6594191)(525.91610201,698.70942034)
\curveto(526.17609716,698.78941897)(526.38109695,698.91441885)(526.53110201,699.08442034)
\curveto(526.68109665,699.2644185)(526.79609654,699.48441828)(526.87610201,699.74442034)
\curveto(526.89609644,699.81441795)(526.90609643,699.88441788)(526.90610201,699.95442034)
\curveto(526.91609642,700.03441773)(526.9310964,700.11441765)(526.95110201,700.19442034)
\lineto(526.95110201,700.32942034)
}
}
{
\newrgbcolor{curcolor}{0 0 0}
\pscustom[linestyle=none,fillstyle=solid,fillcolor=curcolor]
{
}
}
{
\newrgbcolor{curcolor}{0 0 0}
\pscustom[linestyle=none,fillstyle=solid,fillcolor=curcolor]
{
\newpath
\moveto(541.92953951,698.01942034)
\curveto(541.94953166,697.90941985)(541.95953165,697.79941996)(541.95953951,697.68942034)
\curveto(541.96953164,697.57942018)(541.91953169,697.50442026)(541.80953951,697.46442034)
\curveto(541.74953186,697.43442033)(541.67953193,697.41942034)(541.59953951,697.41942034)
\lineto(541.35953951,697.41942034)
\lineto(540.54953951,697.41942034)
\lineto(540.27953951,697.41942034)
\curveto(540.19953341,697.42942033)(540.13453347,697.45442031)(540.08453951,697.49442034)
\curveto(540.01453359,697.53442023)(539.95953365,697.58942017)(539.91953951,697.65942034)
\curveto(539.88953372,697.73942002)(539.84453376,697.80441996)(539.78453951,697.85442034)
\curveto(539.76453384,697.87441989)(539.73953387,697.88941987)(539.70953951,697.89942034)
\curveto(539.67953393,697.91941984)(539.63953397,697.92441984)(539.58953951,697.91442034)
\curveto(539.53953407,697.89441987)(539.48953412,697.86941989)(539.43953951,697.83942034)
\curveto(539.39953421,697.80941995)(539.35453425,697.78441998)(539.30453951,697.76442034)
\curveto(539.25453435,697.72442004)(539.19953441,697.68942007)(539.13953951,697.65942034)
\lineto(538.95953951,697.56942034)
\curveto(538.82953478,697.50942025)(538.69453491,697.4594203)(538.55453951,697.41942034)
\curveto(538.41453519,697.38942037)(538.26953534,697.35442041)(538.11953951,697.31442034)
\curveto(538.04953556,697.29442047)(537.97953563,697.28442048)(537.90953951,697.28442034)
\curveto(537.84953576,697.27442049)(537.78453582,697.2644205)(537.71453951,697.25442034)
\lineto(537.62453951,697.25442034)
\curveto(537.59453601,697.24442052)(537.56453604,697.23942052)(537.53453951,697.23942034)
\lineto(537.36953951,697.23942034)
\curveto(537.26953634,697.21942054)(537.16953644,697.21942054)(537.06953951,697.23942034)
\lineto(536.93453951,697.23942034)
\curveto(536.86453674,697.2594205)(536.79453681,697.26942049)(536.72453951,697.26942034)
\curveto(536.66453694,697.2594205)(536.604537,697.2644205)(536.54453951,697.28442034)
\curveto(536.44453716,697.30442046)(536.34953726,697.32442044)(536.25953951,697.34442034)
\curveto(536.16953744,697.35442041)(536.08453752,697.37942038)(536.00453951,697.41942034)
\curveto(535.71453789,697.52942023)(535.46453814,697.66942009)(535.25453951,697.83942034)
\curveto(535.05453855,698.01941974)(534.89453871,698.25441951)(534.77453951,698.54442034)
\curveto(534.74453886,698.61441915)(534.71453889,698.68941907)(534.68453951,698.76942034)
\curveto(534.66453894,698.84941891)(534.64453896,698.93441883)(534.62453951,699.02442034)
\curveto(534.604539,699.07441869)(534.59453901,699.12441864)(534.59453951,699.17442034)
\curveto(534.604539,699.22441854)(534.604539,699.27441849)(534.59453951,699.32442034)
\curveto(534.58453902,699.35441841)(534.57453903,699.41441835)(534.56453951,699.50442034)
\curveto(534.56453904,699.60441816)(534.56953904,699.67441809)(534.57953951,699.71442034)
\curveto(534.59953901,699.81441795)(534.609539,699.89941786)(534.60953951,699.96942034)
\lineto(534.69953951,700.29942034)
\curveto(534.72953888,700.41941734)(534.76953884,700.52441724)(534.81953951,700.61442034)
\curveto(534.98953862,700.90441686)(535.18453842,701.12441664)(535.40453951,701.27442034)
\curveto(535.62453798,701.42441634)(535.9045377,701.55441621)(536.24453951,701.66442034)
\curveto(536.37453723,701.71441605)(536.5095371,701.74941601)(536.64953951,701.76942034)
\curveto(536.78953682,701.78941597)(536.92953668,701.81441595)(537.06953951,701.84442034)
\curveto(537.14953646,701.8644159)(537.23453637,701.87441589)(537.32453951,701.87442034)
\curveto(537.41453619,701.88441588)(537.5045361,701.89941586)(537.59453951,701.91942034)
\curveto(537.66453594,701.93941582)(537.73453587,701.94441582)(537.80453951,701.93442034)
\curveto(537.87453573,701.93441583)(537.94953566,701.94441582)(538.02953951,701.96442034)
\curveto(538.09953551,701.98441578)(538.16953544,701.99441577)(538.23953951,701.99442034)
\curveto(538.3095353,701.99441577)(538.38453522,702.00441576)(538.46453951,702.02442034)
\curveto(538.67453493,702.07441569)(538.86453474,702.11441565)(539.03453951,702.14442034)
\curveto(539.21453439,702.18441558)(539.37453423,702.27441549)(539.51453951,702.41442034)
\curveto(539.604534,702.50441526)(539.66453394,702.60441516)(539.69453951,702.71442034)
\curveto(539.7045339,702.74441502)(539.7045339,702.76941499)(539.69453951,702.78942034)
\curveto(539.69453391,702.80941495)(539.69953391,702.82941493)(539.70953951,702.84942034)
\curveto(539.71953389,702.86941489)(539.72453388,702.89941486)(539.72453951,702.93942034)
\lineto(539.72453951,703.02942034)
\lineto(539.69453951,703.14942034)
\curveto(539.69453391,703.18941457)(539.68953392,703.22441454)(539.67953951,703.25442034)
\curveto(539.57953403,703.55441421)(539.36953424,703.759414)(539.04953951,703.86942034)
\curveto(538.95953465,703.89941386)(538.84953476,703.91941384)(538.71953951,703.92942034)
\curveto(538.59953501,703.94941381)(538.47453513,703.95441381)(538.34453951,703.94442034)
\curveto(538.21453539,703.94441382)(538.08953552,703.93441383)(537.96953951,703.91442034)
\curveto(537.84953576,703.89441387)(537.74453586,703.86941389)(537.65453951,703.83942034)
\curveto(537.59453601,703.81941394)(537.53453607,703.78941397)(537.47453951,703.74942034)
\curveto(537.42453618,703.71941404)(537.37453623,703.68441408)(537.32453951,703.64442034)
\curveto(537.27453633,703.60441416)(537.21953639,703.54941421)(537.15953951,703.47942034)
\curveto(537.1095365,703.40941435)(537.07453653,703.34441442)(537.05453951,703.28442034)
\curveto(537.0045366,703.18441458)(536.95953665,703.08941467)(536.91953951,702.99942034)
\curveto(536.88953672,702.90941485)(536.81953679,702.84941491)(536.70953951,702.81942034)
\curveto(536.62953698,702.79941496)(536.54453706,702.78941497)(536.45453951,702.78942034)
\lineto(536.18453951,702.78942034)
\lineto(535.61453951,702.78942034)
\curveto(535.56453804,702.78941497)(535.51453809,702.78441498)(535.46453951,702.77442034)
\curveto(535.41453819,702.77441499)(535.36953824,702.77941498)(535.32953951,702.78942034)
\lineto(535.19453951,702.78942034)
\curveto(535.17453843,702.79941496)(535.14953846,702.80441496)(535.11953951,702.80442034)
\curveto(535.08953852,702.80441496)(535.06453854,702.81441495)(535.04453951,702.83442034)
\curveto(534.96453864,702.85441491)(534.9095387,702.91941484)(534.87953951,703.02942034)
\curveto(534.86953874,703.07941468)(534.86953874,703.12941463)(534.87953951,703.17942034)
\curveto(534.88953872,703.22941453)(534.89953871,703.27441449)(534.90953951,703.31442034)
\curveto(534.93953867,703.42441434)(534.96953864,703.52441424)(534.99953951,703.61442034)
\curveto(535.03953857,703.71441405)(535.08453852,703.80441396)(535.13453951,703.88442034)
\lineto(535.22453951,704.03442034)
\lineto(535.31453951,704.18442034)
\curveto(535.39453821,704.29441347)(535.49453811,704.39941336)(535.61453951,704.49942034)
\curveto(535.63453797,704.50941325)(535.66453794,704.53441323)(535.70453951,704.57442034)
\curveto(535.75453785,704.61441315)(535.79953781,704.64941311)(535.83953951,704.67942034)
\curveto(535.87953773,704.70941305)(535.92453768,704.73941302)(535.97453951,704.76942034)
\curveto(536.14453746,704.87941288)(536.32453728,704.9644128)(536.51453951,705.02442034)
\curveto(536.7045369,705.09441267)(536.89953671,705.1594126)(537.09953951,705.21942034)
\curveto(537.21953639,705.24941251)(537.34453626,705.26941249)(537.47453951,705.27942034)
\curveto(537.604536,705.28941247)(537.73453587,705.30941245)(537.86453951,705.33942034)
\curveto(537.9045357,705.34941241)(537.96453564,705.34941241)(538.04453951,705.33942034)
\curveto(538.13453547,705.32941243)(538.18953542,705.33441243)(538.20953951,705.35442034)
\curveto(538.61953499,705.3644124)(539.0095346,705.34941241)(539.37953951,705.30942034)
\curveto(539.75953385,705.26941249)(540.09953351,705.19441257)(540.39953951,705.08442034)
\curveto(540.7095329,704.97441279)(540.97453263,704.82441294)(541.19453951,704.63442034)
\curveto(541.41453219,704.45441331)(541.58453202,704.21941354)(541.70453951,703.92942034)
\curveto(541.77453183,703.759414)(541.81453179,703.5644142)(541.82453951,703.34442034)
\curveto(541.83453177,703.12441464)(541.83953177,702.89941486)(541.83953951,702.66942034)
\lineto(541.83953951,699.32442034)
\lineto(541.83953951,698.73942034)
\curveto(541.83953177,698.54941921)(541.85953175,698.37441939)(541.89953951,698.21442034)
\curveto(541.9095317,698.18441958)(541.91453169,698.14941961)(541.91453951,698.10942034)
\curveto(541.91453169,698.07941968)(541.91953169,698.04941971)(541.92953951,698.01942034)
\moveto(539.72453951,700.32942034)
\curveto(539.73453387,700.37941738)(539.73953387,700.43441733)(539.73953951,700.49442034)
\curveto(539.73953387,700.5644172)(539.73453387,700.62441714)(539.72453951,700.67442034)
\curveto(539.7045339,700.73441703)(539.69453391,700.78941697)(539.69453951,700.83942034)
\curveto(539.69453391,700.88941687)(539.67453393,700.92941683)(539.63453951,700.95942034)
\curveto(539.58453402,700.99941676)(539.5095341,701.01941674)(539.40953951,701.01942034)
\curveto(539.36953424,701.00941675)(539.33453427,700.99941676)(539.30453951,700.98942034)
\curveto(539.27453433,700.98941677)(539.23953437,700.98441678)(539.19953951,700.97442034)
\curveto(539.12953448,700.95441681)(539.05453455,700.93941682)(538.97453951,700.92942034)
\curveto(538.89453471,700.91941684)(538.81453479,700.90441686)(538.73453951,700.88442034)
\curveto(538.7045349,700.87441689)(538.65953495,700.86941689)(538.59953951,700.86942034)
\curveto(538.46953514,700.83941692)(538.33953527,700.81941694)(538.20953951,700.80942034)
\curveto(538.07953553,700.79941696)(537.95453565,700.77441699)(537.83453951,700.73442034)
\curveto(537.75453585,700.71441705)(537.67953593,700.69441707)(537.60953951,700.67442034)
\curveto(537.53953607,700.6644171)(537.46953614,700.64441712)(537.39953951,700.61442034)
\curveto(537.18953642,700.52441724)(537.0095366,700.38941737)(536.85953951,700.20942034)
\curveto(536.71953689,700.02941773)(536.66953694,699.77941798)(536.70953951,699.45942034)
\curveto(536.72953688,699.28941847)(536.78453682,699.14941861)(536.87453951,699.03942034)
\curveto(536.94453666,698.92941883)(537.04953656,698.83941892)(537.18953951,698.76942034)
\curveto(537.32953628,698.70941905)(537.47953613,698.6644191)(537.63953951,698.63442034)
\curveto(537.8095358,698.60441916)(537.98453562,698.59441917)(538.16453951,698.60442034)
\curveto(538.35453525,698.62441914)(538.52953508,698.6594191)(538.68953951,698.70942034)
\curveto(538.94953466,698.78941897)(539.15453445,698.91441885)(539.30453951,699.08442034)
\curveto(539.45453415,699.2644185)(539.56953404,699.48441828)(539.64953951,699.74442034)
\curveto(539.66953394,699.81441795)(539.67953393,699.88441788)(539.67953951,699.95442034)
\curveto(539.68953392,700.03441773)(539.7045339,700.11441765)(539.72453951,700.19442034)
\lineto(539.72453951,700.32942034)
}
}
{
\newrgbcolor{curcolor}{0 0 0}
\pscustom[linestyle=none,fillstyle=solid,fillcolor=curcolor]
{
\newpath
\moveto(543.91282076,705.14442034)
\lineto(545.03782076,705.14442034)
\curveto(545.14781832,705.14441262)(545.24781822,705.13941262)(545.33782076,705.12942034)
\curveto(545.42781804,705.11941264)(545.49281798,705.08441268)(545.53282076,705.02442034)
\curveto(545.58281789,704.9644128)(545.61281786,704.87941288)(545.62282076,704.76942034)
\curveto(545.63281784,704.66941309)(545.63781783,704.5644132)(545.63782076,704.45442034)
\lineto(545.63782076,703.40442034)
\lineto(545.63782076,701.16942034)
\curveto(545.63781783,700.80941695)(545.65281782,700.46941729)(545.68282076,700.14942034)
\curveto(545.71281776,699.82941793)(545.80281767,699.5644182)(545.95282076,699.35442034)
\curveto(546.09281738,699.14441862)(546.31781715,698.99441877)(546.62782076,698.90442034)
\curveto(546.67781679,698.89441887)(546.71781675,698.88941887)(546.74782076,698.88942034)
\curveto(546.78781668,698.88941887)(546.83281664,698.88441888)(546.88282076,698.87442034)
\curveto(546.93281654,698.8644189)(546.98781648,698.8594189)(547.04782076,698.85942034)
\curveto(547.10781636,698.8594189)(547.15281632,698.8644189)(547.18282076,698.87442034)
\curveto(547.23281624,698.89441887)(547.2728162,698.89941886)(547.30282076,698.88942034)
\curveto(547.34281613,698.87941888)(547.38281609,698.88441888)(547.42282076,698.90442034)
\curveto(547.63281584,698.95441881)(547.79781567,699.01941874)(547.91782076,699.09942034)
\curveto(548.09781537,699.20941855)(548.23781523,699.34941841)(548.33782076,699.51942034)
\curveto(548.44781502,699.69941806)(548.52281495,699.89441787)(548.56282076,700.10442034)
\curveto(548.61281486,700.32441744)(548.64281483,700.5644172)(548.65282076,700.82442034)
\curveto(548.66281481,701.09441667)(548.6678148,701.37441639)(548.66782076,701.66442034)
\lineto(548.66782076,703.47942034)
\lineto(548.66782076,704.45442034)
\lineto(548.66782076,704.72442034)
\curveto(548.6678148,704.82441294)(548.68781478,704.90441286)(548.72782076,704.96442034)
\curveto(548.77781469,705.05441271)(548.85281462,705.10441266)(548.95282076,705.11442034)
\curveto(549.05281442,705.13441263)(549.1728143,705.14441262)(549.31282076,705.14442034)
\lineto(550.10782076,705.14442034)
\lineto(550.39282076,705.14442034)
\curveto(550.48281299,705.14441262)(550.55781291,705.12441264)(550.61782076,705.08442034)
\curveto(550.69781277,705.03441273)(550.74281273,704.9594128)(550.75282076,704.85942034)
\curveto(550.76281271,704.759413)(550.7678127,704.64441312)(550.76782076,704.51442034)
\lineto(550.76782076,703.37442034)
\lineto(550.76782076,699.15942034)
\lineto(550.76782076,698.09442034)
\lineto(550.76782076,697.79442034)
\curveto(550.7678127,697.69442007)(550.74781272,697.61942014)(550.70782076,697.56942034)
\curveto(550.65781281,697.48942027)(550.58281289,697.44442032)(550.48282076,697.43442034)
\curveto(550.38281309,697.42442034)(550.27781319,697.41942034)(550.16782076,697.41942034)
\lineto(549.35782076,697.41942034)
\curveto(549.24781422,697.41942034)(549.14781432,697.42442034)(549.05782076,697.43442034)
\curveto(548.97781449,697.44442032)(548.91281456,697.48442028)(548.86282076,697.55442034)
\curveto(548.84281463,697.58442018)(548.82281465,697.62942013)(548.80282076,697.68942034)
\curveto(548.79281468,697.74942001)(548.77781469,697.80941995)(548.75782076,697.86942034)
\curveto(548.74781472,697.92941983)(548.73281474,697.98441978)(548.71282076,698.03442034)
\curveto(548.69281478,698.08441968)(548.66281481,698.11441965)(548.62282076,698.12442034)
\curveto(548.60281487,698.14441962)(548.57781489,698.14941961)(548.54782076,698.13942034)
\curveto(548.51781495,698.12941963)(548.49281498,698.11941964)(548.47282076,698.10942034)
\curveto(548.40281507,698.06941969)(548.34281513,698.02441974)(548.29282076,697.97442034)
\curveto(548.24281523,697.92441984)(548.18781528,697.87941988)(548.12782076,697.83942034)
\curveto(548.08781538,697.80941995)(548.04781542,697.77441999)(548.00782076,697.73442034)
\curveto(547.97781549,697.70442006)(547.93781553,697.67442009)(547.88782076,697.64442034)
\curveto(547.65781581,697.50442026)(547.38781608,697.39442037)(547.07782076,697.31442034)
\curveto(547.00781646,697.29442047)(546.93781653,697.28442048)(546.86782076,697.28442034)
\curveto(546.79781667,697.27442049)(546.72281675,697.2594205)(546.64282076,697.23942034)
\curveto(546.60281687,697.22942053)(546.55781691,697.22942053)(546.50782076,697.23942034)
\curveto(546.467817,697.23942052)(546.42781704,697.23442053)(546.38782076,697.22442034)
\curveto(546.35781711,697.21442055)(546.29281718,697.21442055)(546.19282076,697.22442034)
\curveto(546.10281737,697.22442054)(546.04281743,697.22942053)(546.01282076,697.23942034)
\curveto(545.96281751,697.23942052)(545.91281756,697.24442052)(545.86282076,697.25442034)
\lineto(545.71282076,697.25442034)
\curveto(545.59281788,697.28442048)(545.47781799,697.30942045)(545.36782076,697.32942034)
\curveto(545.25781821,697.34942041)(545.14781832,697.37942038)(545.03782076,697.41942034)
\curveto(544.98781848,697.43942032)(544.94281853,697.45442031)(544.90282076,697.46442034)
\curveto(544.8728186,697.48442028)(544.83281864,697.50442026)(544.78282076,697.52442034)
\curveto(544.43281904,697.71442005)(544.15281932,697.97941978)(543.94282076,698.31942034)
\curveto(543.81281966,698.52941923)(543.71781975,698.77941898)(543.65782076,699.06942034)
\curveto(543.59781987,699.36941839)(543.55781991,699.68441808)(543.53782076,700.01442034)
\curveto(543.52781994,700.35441741)(543.52281995,700.69941706)(543.52282076,701.04942034)
\curveto(543.53281994,701.40941635)(543.53781993,701.764416)(543.53782076,702.11442034)
\lineto(543.53782076,704.15442034)
\curveto(543.53781993,704.28441348)(543.53281994,704.43441333)(543.52282076,704.60442034)
\curveto(543.52281995,704.78441298)(543.54781992,704.91441285)(543.59782076,704.99442034)
\curveto(543.62781984,705.04441272)(543.68781978,705.08941267)(543.77782076,705.12942034)
\curveto(543.83781963,705.12941263)(543.88281959,705.13441263)(543.91282076,705.14442034)
}
}
{
\newrgbcolor{curcolor}{0 0 0}
\pscustom[linestyle=none,fillstyle=solid,fillcolor=curcolor]
{
\newpath
\moveto(559.98907076,698.27442034)
\lineto(559.98907076,697.85442034)
\curveto(559.98906239,697.72442004)(559.95906242,697.61942014)(559.89907076,697.53942034)
\curveto(559.84906253,697.48942027)(559.78406259,697.45442031)(559.70407076,697.43442034)
\curveto(559.62406275,697.42442034)(559.53406284,697.41942034)(559.43407076,697.41942034)
\lineto(558.60907076,697.41942034)
\lineto(558.32407076,697.41942034)
\curveto(558.24406413,697.42942033)(558.1790642,697.45442031)(558.12907076,697.49442034)
\curveto(558.05906432,697.54442022)(558.01906436,697.60942015)(558.00907076,697.68942034)
\curveto(557.99906438,697.76941999)(557.9790644,697.84941991)(557.94907076,697.92942034)
\curveto(557.92906445,697.94941981)(557.90906447,697.9644198)(557.88907076,697.97442034)
\curveto(557.8790645,697.99441977)(557.86406451,698.01441975)(557.84407076,698.03442034)
\curveto(557.73406464,698.03441973)(557.65406472,698.00941975)(557.60407076,697.95942034)
\lineto(557.45407076,697.80942034)
\curveto(557.38406499,697.75942)(557.31906506,697.71442005)(557.25907076,697.67442034)
\curveto(557.19906518,697.64442012)(557.13406524,697.60442016)(557.06407076,697.55442034)
\curveto(557.02406535,697.53442023)(556.9790654,697.51442025)(556.92907076,697.49442034)
\curveto(556.88906549,697.47442029)(556.84406553,697.45442031)(556.79407076,697.43442034)
\curveto(556.65406572,697.38442038)(556.50406587,697.33942042)(556.34407076,697.29942034)
\curveto(556.29406608,697.27942048)(556.24906613,697.26942049)(556.20907076,697.26942034)
\curveto(556.16906621,697.26942049)(556.12906625,697.2644205)(556.08907076,697.25442034)
\lineto(555.95407076,697.25442034)
\curveto(555.92406645,697.24442052)(555.88406649,697.23942052)(555.83407076,697.23942034)
\lineto(555.69907076,697.23942034)
\curveto(555.63906674,697.21942054)(555.54906683,697.21442055)(555.42907076,697.22442034)
\curveto(555.30906707,697.22442054)(555.22406715,697.23442053)(555.17407076,697.25442034)
\curveto(555.10406727,697.27442049)(555.03906734,697.28442048)(554.97907076,697.28442034)
\curveto(554.92906745,697.27442049)(554.8740675,697.27942048)(554.81407076,697.29942034)
\lineto(554.45407076,697.41942034)
\curveto(554.34406803,697.44942031)(554.23406814,697.48942027)(554.12407076,697.53942034)
\curveto(553.7740686,697.68942007)(553.45906892,697.91941984)(553.17907076,698.22942034)
\curveto(552.90906947,698.54941921)(552.69406968,698.88441888)(552.53407076,699.23442034)
\curveto(552.48406989,699.34441842)(552.44406993,699.44941831)(552.41407076,699.54942034)
\curveto(552.38406999,699.6594181)(552.34907003,699.76941799)(552.30907076,699.87942034)
\curveto(552.29907008,699.91941784)(552.29407008,699.95441781)(552.29407076,699.98442034)
\curveto(552.29407008,700.02441774)(552.28407009,700.06941769)(552.26407076,700.11942034)
\curveto(552.24407013,700.19941756)(552.22407015,700.28441748)(552.20407076,700.37442034)
\curveto(552.19407018,700.47441729)(552.1790702,700.57441719)(552.15907076,700.67442034)
\curveto(552.14907023,700.70441706)(552.14407023,700.73941702)(552.14407076,700.77942034)
\curveto(552.15407022,700.81941694)(552.15407022,700.85441691)(552.14407076,700.88442034)
\lineto(552.14407076,701.01942034)
\curveto(552.14407023,701.06941669)(552.13907024,701.11941664)(552.12907076,701.16942034)
\curveto(552.11907026,701.21941654)(552.11407026,701.27441649)(552.11407076,701.33442034)
\curveto(552.11407026,701.40441636)(552.11907026,701.4594163)(552.12907076,701.49942034)
\curveto(552.13907024,701.54941621)(552.14407023,701.59441617)(552.14407076,701.63442034)
\lineto(552.14407076,701.78442034)
\curveto(552.15407022,701.83441593)(552.15407022,701.87941588)(552.14407076,701.91942034)
\curveto(552.14407023,701.96941579)(552.15407022,702.01941574)(552.17407076,702.06942034)
\curveto(552.19407018,702.17941558)(552.20907017,702.28441548)(552.21907076,702.38442034)
\curveto(552.23907014,702.48441528)(552.26407011,702.58441518)(552.29407076,702.68442034)
\curveto(552.33407004,702.80441496)(552.36907001,702.91941484)(552.39907076,703.02942034)
\curveto(552.42906995,703.13941462)(552.46906991,703.24941451)(552.51907076,703.35942034)
\curveto(552.65906972,703.6594141)(552.83406954,703.94441382)(553.04407076,704.21442034)
\curveto(553.06406931,704.24441352)(553.08906929,704.26941349)(553.11907076,704.28942034)
\curveto(553.15906922,704.31941344)(553.18906919,704.34941341)(553.20907076,704.37942034)
\curveto(553.24906913,704.42941333)(553.28906909,704.47441329)(553.32907076,704.51442034)
\curveto(553.36906901,704.55441321)(553.41406896,704.59441317)(553.46407076,704.63442034)
\curveto(553.50406887,704.65441311)(553.53906884,704.67941308)(553.56907076,704.70942034)
\curveto(553.59906878,704.74941301)(553.63406874,704.77941298)(553.67407076,704.79942034)
\curveto(553.92406845,704.96941279)(554.21406816,705.10941265)(554.54407076,705.21942034)
\curveto(554.61406776,705.23941252)(554.68406769,705.25441251)(554.75407076,705.26442034)
\curveto(554.83406754,705.27441249)(554.91406746,705.28941247)(554.99407076,705.30942034)
\curveto(555.06406731,705.32941243)(555.15406722,705.33941242)(555.26407076,705.33942034)
\curveto(555.374067,705.34941241)(555.48406689,705.35441241)(555.59407076,705.35442034)
\curveto(555.70406667,705.35441241)(555.80906657,705.34941241)(555.90907076,705.33942034)
\curveto(556.01906636,705.32941243)(556.10906627,705.31441245)(556.17907076,705.29442034)
\curveto(556.32906605,705.24441252)(556.4740659,705.19941256)(556.61407076,705.15942034)
\curveto(556.75406562,705.11941264)(556.88406549,705.0644127)(557.00407076,704.99442034)
\curveto(557.0740653,704.94441282)(557.13906524,704.89441287)(557.19907076,704.84442034)
\curveto(557.25906512,704.80441296)(557.32406505,704.759413)(557.39407076,704.70942034)
\curveto(557.43406494,704.67941308)(557.48906489,704.63941312)(557.55907076,704.58942034)
\curveto(557.63906474,704.53941322)(557.71406466,704.53941322)(557.78407076,704.58942034)
\curveto(557.82406455,704.60941315)(557.84406453,704.64441312)(557.84407076,704.69442034)
\curveto(557.84406453,704.74441302)(557.85406452,704.79441297)(557.87407076,704.84442034)
\lineto(557.87407076,704.99442034)
\curveto(557.88406449,705.02441274)(557.88906449,705.0594127)(557.88907076,705.09942034)
\lineto(557.88907076,705.21942034)
\lineto(557.88907076,707.25942034)
\curveto(557.88906449,707.36941039)(557.88406449,707.48941027)(557.87407076,707.61942034)
\curveto(557.8740645,707.75941)(557.89906448,707.8644099)(557.94907076,707.93442034)
\curveto(557.98906439,708.01440975)(558.06406431,708.0644097)(558.17407076,708.08442034)
\curveto(558.19406418,708.09440967)(558.21406416,708.09440967)(558.23407076,708.08442034)
\curveto(558.25406412,708.08440968)(558.2740641,708.08940967)(558.29407076,708.09942034)
\lineto(559.35907076,708.09942034)
\curveto(559.4790629,708.09940966)(559.58906279,708.09440967)(559.68907076,708.08442034)
\curveto(559.78906259,708.07440969)(559.86406251,708.03440973)(559.91407076,707.96442034)
\curveto(559.96406241,707.88440988)(559.98906239,707.77940998)(559.98907076,707.64942034)
\lineto(559.98907076,707.28942034)
\lineto(559.98907076,698.27442034)
\moveto(557.94907076,701.21442034)
\curveto(557.95906442,701.25441651)(557.95906442,701.29441647)(557.94907076,701.33442034)
\lineto(557.94907076,701.46942034)
\curveto(557.94906443,701.56941619)(557.94406443,701.66941609)(557.93407076,701.76942034)
\curveto(557.92406445,701.86941589)(557.90906447,701.9594158)(557.88907076,702.03942034)
\curveto(557.86906451,702.14941561)(557.84906453,702.24941551)(557.82907076,702.33942034)
\curveto(557.81906456,702.42941533)(557.79406458,702.51441525)(557.75407076,702.59442034)
\curveto(557.61406476,702.95441481)(557.40906497,703.23941452)(557.13907076,703.44942034)
\curveto(556.8790655,703.6594141)(556.49906588,703.764414)(555.99907076,703.76442034)
\curveto(555.93906644,703.764414)(555.85906652,703.75441401)(555.75907076,703.73442034)
\curveto(555.6790667,703.71441405)(555.60406677,703.69441407)(555.53407076,703.67442034)
\curveto(555.4740669,703.6644141)(555.41406696,703.64441412)(555.35407076,703.61442034)
\curveto(555.08406729,703.50441426)(554.8740675,703.33441443)(554.72407076,703.10442034)
\curveto(554.5740678,702.87441489)(554.45406792,702.61441515)(554.36407076,702.32442034)
\curveto(554.33406804,702.22441554)(554.31406806,702.12441564)(554.30407076,702.02442034)
\curveto(554.29406808,701.92441584)(554.2740681,701.81941594)(554.24407076,701.70942034)
\lineto(554.24407076,701.49942034)
\curveto(554.22406815,701.40941635)(554.21906816,701.28441648)(554.22907076,701.12442034)
\curveto(554.23906814,700.97441679)(554.25406812,700.8644169)(554.27407076,700.79442034)
\lineto(554.27407076,700.70442034)
\curveto(554.28406809,700.68441708)(554.28906809,700.6644171)(554.28907076,700.64442034)
\curveto(554.30906807,700.5644172)(554.32406805,700.48941727)(554.33407076,700.41942034)
\curveto(554.35406802,700.34941741)(554.374068,700.27441749)(554.39407076,700.19442034)
\curveto(554.56406781,699.67441809)(554.85406752,699.28941847)(555.26407076,699.03942034)
\curveto(555.39406698,698.94941881)(555.5740668,698.87941888)(555.80407076,698.82942034)
\curveto(555.84406653,698.81941894)(555.90406647,698.81441895)(555.98407076,698.81442034)
\curveto(556.01406636,698.80441896)(556.05906632,698.79441897)(556.11907076,698.78442034)
\curveto(556.18906619,698.78441898)(556.24406613,698.78941897)(556.28407076,698.79942034)
\curveto(556.36406601,698.81941894)(556.44406593,698.83441893)(556.52407076,698.84442034)
\curveto(556.60406577,698.85441891)(556.68406569,698.87441889)(556.76407076,698.90442034)
\curveto(557.01406536,699.01441875)(557.21406516,699.15441861)(557.36407076,699.32442034)
\curveto(557.51406486,699.49441827)(557.64406473,699.70941805)(557.75407076,699.96942034)
\curveto(557.79406458,700.0594177)(557.82406455,700.14941761)(557.84407076,700.23942034)
\curveto(557.86406451,700.33941742)(557.88406449,700.44441732)(557.90407076,700.55442034)
\curveto(557.91406446,700.60441716)(557.91406446,700.64941711)(557.90407076,700.68942034)
\curveto(557.90406447,700.73941702)(557.91406446,700.78941697)(557.93407076,700.83942034)
\curveto(557.94406443,700.86941689)(557.94906443,700.90441686)(557.94907076,700.94442034)
\lineto(557.94907076,701.07942034)
\lineto(557.94907076,701.21442034)
}
}
{
\newrgbcolor{curcolor}{0 0 0}
\pscustom[linestyle=none,fillstyle=solid,fillcolor=curcolor]
{
\newpath
\moveto(563.66899263,708.00942034)
\curveto(563.73898968,707.92940983)(563.77398965,707.80940995)(563.77399263,707.64942034)
\lineto(563.77399263,707.18442034)
\lineto(563.77399263,706.77942034)
\curveto(563.77398965,706.63941112)(563.73898968,706.54441122)(563.66899263,706.49442034)
\curveto(563.60898981,706.44441132)(563.52898989,706.41441135)(563.42899263,706.40442034)
\curveto(563.33899008,706.39441137)(563.23899018,706.38941137)(563.12899263,706.38942034)
\lineto(562.28899263,706.38942034)
\curveto(562.17899124,706.38941137)(562.07899134,706.39441137)(561.98899263,706.40442034)
\curveto(561.90899151,706.41441135)(561.83899158,706.44441132)(561.77899263,706.49442034)
\curveto(561.73899168,706.52441124)(561.70899171,706.57941118)(561.68899263,706.65942034)
\curveto(561.67899174,706.74941101)(561.66899175,706.84441092)(561.65899263,706.94442034)
\lineto(561.65899263,707.27442034)
\curveto(561.66899175,707.38441038)(561.67399175,707.47941028)(561.67399263,707.55942034)
\lineto(561.67399263,707.76942034)
\curveto(561.68399174,707.83940992)(561.70399172,707.89940986)(561.73399263,707.94942034)
\curveto(561.75399167,707.98940977)(561.77899164,708.01940974)(561.80899263,708.03942034)
\lineto(561.92899263,708.09942034)
\curveto(561.94899147,708.09940966)(561.97399145,708.09940966)(562.00399263,708.09942034)
\curveto(562.03399139,708.10940965)(562.05899136,708.11440965)(562.07899263,708.11442034)
\lineto(563.17399263,708.11442034)
\curveto(563.27399015,708.11440965)(563.36899005,708.10940965)(563.45899263,708.09942034)
\curveto(563.54898987,708.08940967)(563.6189898,708.0594097)(563.66899263,708.00942034)
\moveto(563.77399263,698.24442034)
\curveto(563.77398965,698.04441972)(563.76898965,697.87441989)(563.75899263,697.73442034)
\curveto(563.74898967,697.59442017)(563.65898976,697.49942026)(563.48899263,697.44942034)
\curveto(563.42898999,697.42942033)(563.36399006,697.41942034)(563.29399263,697.41942034)
\curveto(563.2239902,697.42942033)(563.14899027,697.43442033)(563.06899263,697.43442034)
\lineto(562.22899263,697.43442034)
\curveto(562.13899128,697.43442033)(562.04899137,697.43942032)(561.95899263,697.44942034)
\curveto(561.87899154,697.4594203)(561.8189916,697.48942027)(561.77899263,697.53942034)
\curveto(561.7189917,697.60942015)(561.68399174,697.69442007)(561.67399263,697.79442034)
\lineto(561.67399263,698.13942034)
\lineto(561.67399263,704.46942034)
\lineto(561.67399263,704.76942034)
\curveto(561.67399175,704.86941289)(561.69399173,704.94941281)(561.73399263,705.00942034)
\curveto(561.79399163,705.07941268)(561.87899154,705.12441264)(561.98899263,705.14442034)
\curveto(562.00899141,705.15441261)(562.03399139,705.15441261)(562.06399263,705.14442034)
\curveto(562.10399132,705.14441262)(562.13399129,705.14941261)(562.15399263,705.15942034)
\lineto(562.90399263,705.15942034)
\lineto(563.09899263,705.15942034)
\curveto(563.17899024,705.16941259)(563.24399018,705.16941259)(563.29399263,705.15942034)
\lineto(563.41399263,705.15942034)
\curveto(563.47398995,705.13941262)(563.52898989,705.12441264)(563.57899263,705.11442034)
\curveto(563.62898979,705.10441266)(563.66898975,705.07441269)(563.69899263,705.02442034)
\curveto(563.73898968,704.97441279)(563.75898966,704.90441286)(563.75899263,704.81442034)
\curveto(563.76898965,704.72441304)(563.77398965,704.62941313)(563.77399263,704.52942034)
\lineto(563.77399263,698.24442034)
}
}
{
\newrgbcolor{curcolor}{0 0 0}
\pscustom[linestyle=none,fillstyle=solid,fillcolor=curcolor]
{
\newpath
\moveto(572.80118013,701.36442034)
\curveto(572.82117197,701.28441648)(572.82117197,701.19441657)(572.80118013,701.09442034)
\curveto(572.78117201,700.99441677)(572.74617204,700.92941683)(572.69618013,700.89942034)
\curveto(572.64617214,700.8594169)(572.57117222,700.82941693)(572.47118013,700.80942034)
\curveto(572.38117241,700.79941696)(572.27617251,700.78941697)(572.15618013,700.77942034)
\lineto(571.81118013,700.77942034)
\curveto(571.70117309,700.78941697)(571.60117319,700.79441697)(571.51118013,700.79442034)
\lineto(567.85118013,700.79442034)
\lineto(567.64118013,700.79442034)
\curveto(567.58117721,700.79441697)(567.52617726,700.78441698)(567.47618013,700.76442034)
\curveto(567.39617739,700.72441704)(567.34617744,700.68441708)(567.32618013,700.64442034)
\curveto(567.30617748,700.62441714)(567.2861775,700.58441718)(567.26618013,700.52442034)
\curveto(567.24617754,700.47441729)(567.24117755,700.42441734)(567.25118013,700.37442034)
\curveto(567.27117752,700.31441745)(567.28117751,700.25441751)(567.28118013,700.19442034)
\curveto(567.2911775,700.14441762)(567.30617748,700.08941767)(567.32618013,700.02942034)
\curveto(567.40617738,699.78941797)(567.50117729,699.58941817)(567.61118013,699.42942034)
\curveto(567.73117706,699.27941848)(567.8911769,699.14441862)(568.09118013,699.02442034)
\curveto(568.17117662,698.97441879)(568.25117654,698.93941882)(568.33118013,698.91942034)
\curveto(568.42117637,698.90941885)(568.51117628,698.88941887)(568.60118013,698.85942034)
\curveto(568.68117611,698.83941892)(568.791176,698.82441894)(568.93118013,698.81442034)
\curveto(569.07117572,698.80441896)(569.1911756,698.80941895)(569.29118013,698.82942034)
\lineto(569.42618013,698.82942034)
\curveto(569.52617526,698.84941891)(569.61617517,698.86941889)(569.69618013,698.88942034)
\curveto(569.786175,698.91941884)(569.87117492,698.94941881)(569.95118013,698.97942034)
\curveto(570.05117474,699.02941873)(570.16117463,699.09441867)(570.28118013,699.17442034)
\curveto(570.41117438,699.25441851)(570.50617428,699.33441843)(570.56618013,699.41442034)
\curveto(570.61617417,699.48441828)(570.66617412,699.54941821)(570.71618013,699.60942034)
\curveto(570.77617401,699.67941808)(570.84617394,699.72941803)(570.92618013,699.75942034)
\curveto(571.02617376,699.80941795)(571.15117364,699.82941793)(571.30118013,699.81942034)
\lineto(571.73618013,699.81942034)
\lineto(571.91618013,699.81942034)
\curveto(571.9861728,699.82941793)(572.04617274,699.82441794)(572.09618013,699.80442034)
\lineto(572.24618013,699.80442034)
\curveto(572.34617244,699.78441798)(572.41617237,699.759418)(572.45618013,699.72942034)
\curveto(572.49617229,699.70941805)(572.51617227,699.6644181)(572.51618013,699.59442034)
\curveto(572.52617226,699.52441824)(572.52117227,699.4644183)(572.50118013,699.41442034)
\curveto(572.45117234,699.27441849)(572.39617239,699.14941861)(572.33618013,699.03942034)
\curveto(572.27617251,698.92941883)(572.20617258,698.81941894)(572.12618013,698.70942034)
\curveto(571.90617288,698.37941938)(571.65617313,698.11441965)(571.37618013,697.91442034)
\curveto(571.09617369,697.71442005)(570.74617404,697.54442022)(570.32618013,697.40442034)
\curveto(570.21617457,697.3644204)(570.10617468,697.33942042)(569.99618013,697.32942034)
\curveto(569.8861749,697.31942044)(569.77117502,697.29942046)(569.65118013,697.26942034)
\curveto(569.61117518,697.2594205)(569.56617522,697.2594205)(569.51618013,697.26942034)
\curveto(569.47617531,697.26942049)(569.43617535,697.2644205)(569.39618013,697.25442034)
\lineto(569.23118013,697.25442034)
\curveto(569.18117561,697.23442053)(569.12117567,697.22942053)(569.05118013,697.23942034)
\curveto(568.9911758,697.23942052)(568.93617585,697.24442052)(568.88618013,697.25442034)
\curveto(568.80617598,697.2644205)(568.73617605,697.2644205)(568.67618013,697.25442034)
\curveto(568.61617617,697.24442052)(568.55117624,697.24942051)(568.48118013,697.26942034)
\curveto(568.43117636,697.28942047)(568.37617641,697.29942046)(568.31618013,697.29942034)
\curveto(568.25617653,697.29942046)(568.20117659,697.30942045)(568.15118013,697.32942034)
\curveto(568.04117675,697.34942041)(567.93117686,697.37442039)(567.82118013,697.40442034)
\curveto(567.71117708,697.42442034)(567.61117718,697.4594203)(567.52118013,697.50942034)
\curveto(567.41117738,697.54942021)(567.30617748,697.58442018)(567.20618013,697.61442034)
\curveto(567.11617767,697.65442011)(567.03117776,697.69942006)(566.95118013,697.74942034)
\curveto(566.63117816,697.94941981)(566.34617844,698.17941958)(566.09618013,698.43942034)
\curveto(565.84617894,698.70941905)(565.64117915,699.01941874)(565.48118013,699.36942034)
\curveto(565.43117936,699.47941828)(565.3911794,699.58941817)(565.36118013,699.69942034)
\curveto(565.33117946,699.81941794)(565.2911795,699.93941782)(565.24118013,700.05942034)
\curveto(565.23117956,700.09941766)(565.22617956,700.13441763)(565.22618013,700.16442034)
\curveto(565.22617956,700.20441756)(565.22117957,700.24441752)(565.21118013,700.28442034)
\curveto(565.17117962,700.40441736)(565.14617964,700.53441723)(565.13618013,700.67442034)
\lineto(565.10618013,701.09442034)
\curveto(565.10617968,701.14441662)(565.10117969,701.19941656)(565.09118013,701.25942034)
\curveto(565.0911797,701.31941644)(565.09617969,701.37441639)(565.10618013,701.42442034)
\lineto(565.10618013,701.60442034)
\lineto(565.15118013,701.96442034)
\curveto(565.1911796,702.13441563)(565.22617956,702.29941546)(565.25618013,702.45942034)
\curveto(565.2861795,702.61941514)(565.33117946,702.76941499)(565.39118013,702.90942034)
\curveto(565.82117897,703.94941381)(566.55117824,704.68441308)(567.58118013,705.11442034)
\curveto(567.72117707,705.17441259)(567.86117693,705.21441255)(568.00118013,705.23442034)
\curveto(568.15117664,705.2644125)(568.30617648,705.29941246)(568.46618013,705.33942034)
\curveto(568.54617624,705.34941241)(568.62117617,705.35441241)(568.69118013,705.35442034)
\curveto(568.76117603,705.35441241)(568.83617595,705.3594124)(568.91618013,705.36942034)
\curveto(569.42617536,705.37941238)(569.86117493,705.31941244)(570.22118013,705.18942034)
\curveto(570.5911742,705.06941269)(570.92117387,704.90941285)(571.21118013,704.70942034)
\curveto(571.30117349,704.64941311)(571.3911734,704.57941318)(571.48118013,704.49942034)
\curveto(571.57117322,704.42941333)(571.65117314,704.35441341)(571.72118013,704.27442034)
\curveto(571.75117304,704.22441354)(571.791173,704.18441358)(571.84118013,704.15442034)
\curveto(571.92117287,704.04441372)(571.99617279,703.92941383)(572.06618013,703.80942034)
\curveto(572.13617265,703.69941406)(572.21117258,703.58441418)(572.29118013,703.46442034)
\curveto(572.34117245,703.37441439)(572.38117241,703.27941448)(572.41118013,703.17942034)
\curveto(572.45117234,703.08941467)(572.4911723,702.98941477)(572.53118013,702.87942034)
\curveto(572.58117221,702.74941501)(572.62117217,702.61441515)(572.65118013,702.47442034)
\curveto(572.68117211,702.33441543)(572.71617207,702.19441557)(572.75618013,702.05442034)
\curveto(572.77617201,701.97441579)(572.78117201,701.88441588)(572.77118013,701.78442034)
\curveto(572.77117202,701.69441607)(572.78117201,701.60941615)(572.80118013,701.52942034)
\lineto(572.80118013,701.36442034)
\moveto(570.55118013,702.24942034)
\curveto(570.62117417,702.34941541)(570.62617416,702.46941529)(570.56618013,702.60942034)
\curveto(570.51617427,702.759415)(570.47617431,702.86941489)(570.44618013,702.93942034)
\curveto(570.30617448,703.20941455)(570.12117467,703.41441435)(569.89118013,703.55442034)
\curveto(569.66117513,703.70441406)(569.34117545,703.78441398)(568.93118013,703.79442034)
\curveto(568.90117589,703.77441399)(568.86617592,703.76941399)(568.82618013,703.77942034)
\curveto(568.786176,703.78941397)(568.75117604,703.78941397)(568.72118013,703.77942034)
\curveto(568.67117612,703.759414)(568.61617617,703.74441402)(568.55618013,703.73442034)
\curveto(568.49617629,703.73441403)(568.44117635,703.72441404)(568.39118013,703.70442034)
\curveto(567.95117684,703.5644142)(567.62617716,703.28941447)(567.41618013,702.87942034)
\curveto(567.39617739,702.83941492)(567.37117742,702.78441498)(567.34118013,702.71442034)
\curveto(567.32117747,702.65441511)(567.30617748,702.58941517)(567.29618013,702.51942034)
\curveto(567.2861775,702.4594153)(567.2861775,702.39941536)(567.29618013,702.33942034)
\curveto(567.31617747,702.27941548)(567.35117744,702.22941553)(567.40118013,702.18942034)
\curveto(567.48117731,702.13941562)(567.5911772,702.11441565)(567.73118013,702.11442034)
\lineto(568.13618013,702.11442034)
\lineto(569.80118013,702.11442034)
\lineto(570.23618013,702.11442034)
\curveto(570.39617439,702.12441564)(570.50117429,702.16941559)(570.55118013,702.24942034)
}
}
{
\newrgbcolor{curcolor}{0 0 0}
\pscustom[linestyle=none,fillstyle=solid,fillcolor=curcolor]
{
\newpath
\moveto(578.47446138,705.35442034)
\curveto(579.07445558,705.37441239)(579.57445508,705.28941247)(579.97446138,705.09942034)
\curveto(580.37445428,704.90941285)(580.68945396,704.62941313)(580.91946138,704.25942034)
\curveto(580.98945366,704.14941361)(581.04445361,704.02941373)(581.08446138,703.89942034)
\curveto(581.12445353,703.77941398)(581.16445349,703.65441411)(581.20446138,703.52442034)
\curveto(581.22445343,703.44441432)(581.23445342,703.36941439)(581.23446138,703.29942034)
\curveto(581.24445341,703.22941453)(581.25945339,703.1594146)(581.27946138,703.08942034)
\curveto(581.27945337,703.02941473)(581.28445337,702.98941477)(581.29446138,702.96942034)
\curveto(581.31445334,702.82941493)(581.32445333,702.68441508)(581.32446138,702.53442034)
\lineto(581.32446138,702.09942034)
\lineto(581.32446138,700.76442034)
\lineto(581.32446138,698.33442034)
\curveto(581.32445333,698.14441962)(581.31945333,697.9594198)(581.30946138,697.77942034)
\curveto(581.30945334,697.60942015)(581.23945341,697.49942026)(581.09946138,697.44942034)
\curveto(581.03945361,697.42942033)(580.96945368,697.41942034)(580.88946138,697.41942034)
\lineto(580.64946138,697.41942034)
\lineto(579.83946138,697.41942034)
\curveto(579.71945493,697.41942034)(579.60945504,697.42442034)(579.50946138,697.43442034)
\curveto(579.41945523,697.45442031)(579.3494553,697.49942026)(579.29946138,697.56942034)
\curveto(579.25945539,697.62942013)(579.23445542,697.70442006)(579.22446138,697.79442034)
\lineto(579.22446138,698.10942034)
\lineto(579.22446138,699.15942034)
\lineto(579.22446138,701.39442034)
\curveto(579.22445543,701.764416)(579.20945544,702.10441566)(579.17946138,702.41442034)
\curveto(579.1494555,702.73441503)(579.05945559,703.00441476)(578.90946138,703.22442034)
\curveto(578.76945588,703.42441434)(578.56445609,703.5644142)(578.29446138,703.64442034)
\curveto(578.24445641,703.6644141)(578.18945646,703.67441409)(578.12946138,703.67442034)
\curveto(578.07945657,703.67441409)(578.02445663,703.68441408)(577.96446138,703.70442034)
\curveto(577.91445674,703.71441405)(577.8494568,703.71441405)(577.76946138,703.70442034)
\curveto(577.69945695,703.70441406)(577.64445701,703.69941406)(577.60446138,703.68942034)
\curveto(577.56445709,703.67941408)(577.52945712,703.67441409)(577.49946138,703.67442034)
\curveto(577.46945718,703.67441409)(577.43945721,703.66941409)(577.40946138,703.65942034)
\curveto(577.17945747,703.59941416)(576.99445766,703.51941424)(576.85446138,703.41942034)
\curveto(576.53445812,703.18941457)(576.34445831,702.85441491)(576.28446138,702.41442034)
\curveto(576.22445843,701.97441579)(576.19445846,701.47941628)(576.19446138,700.92942034)
\lineto(576.19446138,699.05442034)
\lineto(576.19446138,698.13942034)
\lineto(576.19446138,697.86942034)
\curveto(576.19445846,697.77941998)(576.17945847,697.70442006)(576.14946138,697.64442034)
\curveto(576.09945855,697.53442023)(576.01945863,697.46942029)(575.90946138,697.44942034)
\curveto(575.79945885,697.42942033)(575.66445899,697.41942034)(575.50446138,697.41942034)
\lineto(574.75446138,697.41942034)
\curveto(574.64446001,697.41942034)(574.53446012,697.42442034)(574.42446138,697.43442034)
\curveto(574.31446034,697.44442032)(574.23446042,697.47942028)(574.18446138,697.53942034)
\curveto(574.11446054,697.62942013)(574.07946057,697.75942)(574.07946138,697.92942034)
\curveto(574.08946056,698.09941966)(574.09446056,698.2594195)(574.09446138,698.40942034)
\lineto(574.09446138,700.44942034)
\lineto(574.09446138,703.74942034)
\lineto(574.09446138,704.51442034)
\lineto(574.09446138,704.81442034)
\curveto(574.10446055,704.90441286)(574.13446052,704.97941278)(574.18446138,705.03942034)
\curveto(574.20446045,705.06941269)(574.23446042,705.08941267)(574.27446138,705.09942034)
\curveto(574.32446033,705.11941264)(574.37446028,705.13441263)(574.42446138,705.14442034)
\lineto(574.49946138,705.14442034)
\curveto(574.5494601,705.15441261)(574.59946005,705.1594126)(574.64946138,705.15942034)
\lineto(574.81446138,705.15942034)
\lineto(575.44446138,705.15942034)
\curveto(575.52445913,705.1594126)(575.59945905,705.15441261)(575.66946138,705.14442034)
\curveto(575.7494589,705.14441262)(575.81945883,705.13441263)(575.87946138,705.11442034)
\curveto(575.9494587,705.08441268)(575.99445866,705.03941272)(576.01446138,704.97942034)
\curveto(576.04445861,704.91941284)(576.06945858,704.84941291)(576.08946138,704.76942034)
\curveto(576.09945855,704.72941303)(576.09945855,704.69441307)(576.08946138,704.66442034)
\curveto(576.08945856,704.63441313)(576.09945855,704.60441316)(576.11946138,704.57442034)
\curveto(576.13945851,704.52441324)(576.1544585,704.49441327)(576.16446138,704.48442034)
\curveto(576.18445847,704.47441329)(576.20945844,704.4594133)(576.23946138,704.43942034)
\curveto(576.3494583,704.42941333)(576.43945821,704.4644133)(576.50946138,704.54442034)
\curveto(576.57945807,704.63441313)(576.654458,704.70441306)(576.73446138,704.75442034)
\curveto(577.00445765,704.95441281)(577.30445735,705.11441265)(577.63446138,705.23442034)
\curveto(577.72445693,705.2644125)(577.81445684,705.28441248)(577.90446138,705.29442034)
\curveto(578.00445665,705.30441246)(578.10945654,705.31941244)(578.21946138,705.33942034)
\curveto(578.2494564,705.34941241)(578.29445636,705.34941241)(578.35446138,705.33942034)
\curveto(578.41445624,705.33941242)(578.4544562,705.34441242)(578.47446138,705.35442034)
}
}
{
\newrgbcolor{curcolor}{0 0 0}
\pscustom[linestyle=none,fillstyle=solid,fillcolor=curcolor]
{
\newpath
\moveto(586.52571138,705.36942034)
\curveto(587.33570622,705.38941237)(588.01070555,705.26941249)(588.55071138,705.00942034)
\curveto(589.10070446,704.74941301)(589.53570402,704.37941338)(589.85571138,703.89942034)
\curveto(590.01570354,703.6594141)(590.13570342,703.38441438)(590.21571138,703.07442034)
\curveto(590.23570332,703.02441474)(590.25070331,702.9594148)(590.26071138,702.87942034)
\curveto(590.28070328,702.79941496)(590.28070328,702.72941503)(590.26071138,702.66942034)
\curveto(590.22070334,702.5594152)(590.15070341,702.49441527)(590.05071138,702.47442034)
\curveto(589.95070361,702.4644153)(589.83070373,702.4594153)(589.69071138,702.45942034)
\lineto(588.91071138,702.45942034)
\lineto(588.62571138,702.45942034)
\curveto(588.53570502,702.4594153)(588.4607051,702.47941528)(588.40071138,702.51942034)
\curveto(588.32070524,702.5594152)(588.26570529,702.61941514)(588.23571138,702.69942034)
\curveto(588.20570535,702.78941497)(588.16570539,702.87941488)(588.11571138,702.96942034)
\curveto(588.0557055,703.07941468)(587.99070557,703.17941458)(587.92071138,703.26942034)
\curveto(587.85070571,703.3594144)(587.77070579,703.43941432)(587.68071138,703.50942034)
\curveto(587.54070602,703.59941416)(587.38570617,703.66941409)(587.21571138,703.71942034)
\curveto(587.1557064,703.73941402)(587.09570646,703.74941401)(587.03571138,703.74942034)
\curveto(586.97570658,703.74941401)(586.92070664,703.759414)(586.87071138,703.77942034)
\lineto(586.72071138,703.77942034)
\curveto(586.52070704,703.77941398)(586.3607072,703.759414)(586.24071138,703.71942034)
\curveto(585.95070761,703.62941413)(585.71570784,703.48941427)(585.53571138,703.29942034)
\curveto(585.3557082,703.11941464)(585.21070835,702.89941486)(585.10071138,702.63942034)
\curveto(585.05070851,702.52941523)(585.01070855,702.40941535)(584.98071138,702.27942034)
\curveto(584.9607086,702.1594156)(584.93570862,702.02941573)(584.90571138,701.88942034)
\curveto(584.89570866,701.84941591)(584.89070867,701.80941595)(584.89071138,701.76942034)
\curveto(584.89070867,701.72941603)(584.88570867,701.68941607)(584.87571138,701.64942034)
\curveto(584.8557087,701.54941621)(584.84570871,701.40941635)(584.84571138,701.22942034)
\curveto(584.8557087,701.04941671)(584.87070869,700.90941685)(584.89071138,700.80942034)
\curveto(584.89070867,700.72941703)(584.89570866,700.67441709)(584.90571138,700.64442034)
\curveto(584.92570863,700.57441719)(584.93570862,700.50441726)(584.93571138,700.43442034)
\curveto(584.94570861,700.3644174)(584.9607086,700.29441747)(584.98071138,700.22442034)
\curveto(585.0607085,699.99441777)(585.1557084,699.78441798)(585.26571138,699.59442034)
\curveto(585.37570818,699.40441836)(585.51570804,699.24441852)(585.68571138,699.11442034)
\curveto(585.72570783,699.08441868)(585.78570777,699.04941871)(585.86571138,699.00942034)
\curveto(585.97570758,698.93941882)(586.08570747,698.89441887)(586.19571138,698.87442034)
\curveto(586.31570724,698.85441891)(586.4607071,698.83441893)(586.63071138,698.81442034)
\lineto(586.72071138,698.81442034)
\curveto(586.7607068,698.81441895)(586.79070677,698.81941894)(586.81071138,698.82942034)
\lineto(586.94571138,698.82942034)
\curveto(587.01570654,698.84941891)(587.08070648,698.8644189)(587.14071138,698.87442034)
\curveto(587.21070635,698.89441887)(587.27570628,698.91441885)(587.33571138,698.93442034)
\curveto(587.63570592,699.0644187)(587.86570569,699.25441851)(588.02571138,699.50442034)
\curveto(588.06570549,699.55441821)(588.10070546,699.60941815)(588.13071138,699.66942034)
\curveto(588.1607054,699.73941802)(588.18570537,699.79941796)(588.20571138,699.84942034)
\curveto(588.24570531,699.9594178)(588.28070528,700.05441771)(588.31071138,700.13442034)
\curveto(588.34070522,700.22441754)(588.41070515,700.29441747)(588.52071138,700.34442034)
\curveto(588.61070495,700.38441738)(588.7557048,700.39941736)(588.95571138,700.38942034)
\lineto(589.45071138,700.38942034)
\lineto(589.66071138,700.38942034)
\curveto(589.74070382,700.39941736)(589.80570375,700.39441737)(589.85571138,700.37442034)
\lineto(589.97571138,700.37442034)
\lineto(590.09571138,700.34442034)
\curveto(590.13570342,700.34441742)(590.16570339,700.33441743)(590.18571138,700.31442034)
\curveto(590.23570332,700.27441749)(590.26570329,700.21441755)(590.27571138,700.13442034)
\curveto(590.29570326,700.0644177)(590.29570326,699.98941777)(590.27571138,699.90942034)
\curveto(590.18570337,699.57941818)(590.07570348,699.28441848)(589.94571138,699.02442034)
\curveto(589.53570402,698.25441951)(588.88070468,697.71942004)(587.98071138,697.41942034)
\curveto(587.88070568,697.38942037)(587.77570578,697.36942039)(587.66571138,697.35942034)
\curveto(587.555706,697.33942042)(587.44570611,697.31442045)(587.33571138,697.28442034)
\curveto(587.27570628,697.27442049)(587.21570634,697.26942049)(587.15571138,697.26942034)
\curveto(587.09570646,697.26942049)(587.03570652,697.2644205)(586.97571138,697.25442034)
\lineto(586.81071138,697.25442034)
\curveto(586.7607068,697.23442053)(586.68570687,697.22942053)(586.58571138,697.23942034)
\curveto(586.48570707,697.23942052)(586.41070715,697.24442052)(586.36071138,697.25442034)
\curveto(586.28070728,697.27442049)(586.20570735,697.28442048)(586.13571138,697.28442034)
\curveto(586.07570748,697.27442049)(586.01070755,697.27942048)(585.94071138,697.29942034)
\lineto(585.79071138,697.32942034)
\curveto(585.74070782,697.32942043)(585.69070787,697.33442043)(585.64071138,697.34442034)
\curveto(585.53070803,697.37442039)(585.42570813,697.40442036)(585.32571138,697.43442034)
\curveto(585.22570833,697.4644203)(585.13070843,697.49942026)(585.04071138,697.53942034)
\curveto(584.57070899,697.73942002)(584.17570938,697.99441977)(583.85571138,698.30442034)
\curveto(583.53571002,698.62441914)(583.27571028,699.01941874)(583.07571138,699.48942034)
\curveto(583.02571053,699.57941818)(582.98571057,699.67441809)(582.95571138,699.77442034)
\lineto(582.86571138,700.10442034)
\curveto(582.8557107,700.14441762)(582.85071071,700.17941758)(582.85071138,700.20942034)
\curveto(582.85071071,700.24941751)(582.84071072,700.29441747)(582.82071138,700.34442034)
\curveto(582.80071076,700.41441735)(582.79071077,700.48441728)(582.79071138,700.55442034)
\curveto(582.79071077,700.63441713)(582.78071078,700.70941705)(582.76071138,700.77942034)
\lineto(582.76071138,701.03442034)
\curveto(582.74071082,701.08441668)(582.73071083,701.13941662)(582.73071138,701.19942034)
\curveto(582.73071083,701.26941649)(582.74071082,701.32941643)(582.76071138,701.37942034)
\curveto(582.77071079,701.42941633)(582.77071079,701.47441629)(582.76071138,701.51442034)
\curveto(582.75071081,701.55441621)(582.75071081,701.59441617)(582.76071138,701.63442034)
\curveto(582.78071078,701.70441606)(582.78571077,701.76941599)(582.77571138,701.82942034)
\curveto(582.77571078,701.88941587)(582.78571077,701.94941581)(582.80571138,702.00942034)
\curveto(582.8557107,702.18941557)(582.89571066,702.3594154)(582.92571138,702.51942034)
\curveto(582.9557106,702.68941507)(583.00071056,702.85441491)(583.06071138,703.01442034)
\curveto(583.28071028,703.52441424)(583.55571,703.94941381)(583.88571138,704.28942034)
\curveto(584.22570933,704.62941313)(584.6557089,704.90441286)(585.17571138,705.11442034)
\curveto(585.31570824,705.17441259)(585.4607081,705.21441255)(585.61071138,705.23442034)
\curveto(585.7607078,705.2644125)(585.91570764,705.29941246)(586.07571138,705.33942034)
\curveto(586.1557074,705.34941241)(586.23070733,705.35441241)(586.30071138,705.35442034)
\curveto(586.37070719,705.35441241)(586.44570711,705.3594124)(586.52571138,705.36942034)
}
}
{
\newrgbcolor{curcolor}{0 0 0}
\pscustom[linestyle=none,fillstyle=solid,fillcolor=curcolor]
{
\newpath
\moveto(593.66899263,708.00942034)
\curveto(593.73898968,707.92940983)(593.77398965,707.80940995)(593.77399263,707.64942034)
\lineto(593.77399263,707.18442034)
\lineto(593.77399263,706.77942034)
\curveto(593.77398965,706.63941112)(593.73898968,706.54441122)(593.66899263,706.49442034)
\curveto(593.60898981,706.44441132)(593.52898989,706.41441135)(593.42899263,706.40442034)
\curveto(593.33899008,706.39441137)(593.23899018,706.38941137)(593.12899263,706.38942034)
\lineto(592.28899263,706.38942034)
\curveto(592.17899124,706.38941137)(592.07899134,706.39441137)(591.98899263,706.40442034)
\curveto(591.90899151,706.41441135)(591.83899158,706.44441132)(591.77899263,706.49442034)
\curveto(591.73899168,706.52441124)(591.70899171,706.57941118)(591.68899263,706.65942034)
\curveto(591.67899174,706.74941101)(591.66899175,706.84441092)(591.65899263,706.94442034)
\lineto(591.65899263,707.27442034)
\curveto(591.66899175,707.38441038)(591.67399175,707.47941028)(591.67399263,707.55942034)
\lineto(591.67399263,707.76942034)
\curveto(591.68399174,707.83940992)(591.70399172,707.89940986)(591.73399263,707.94942034)
\curveto(591.75399167,707.98940977)(591.77899164,708.01940974)(591.80899263,708.03942034)
\lineto(591.92899263,708.09942034)
\curveto(591.94899147,708.09940966)(591.97399145,708.09940966)(592.00399263,708.09942034)
\curveto(592.03399139,708.10940965)(592.05899136,708.11440965)(592.07899263,708.11442034)
\lineto(593.17399263,708.11442034)
\curveto(593.27399015,708.11440965)(593.36899005,708.10940965)(593.45899263,708.09942034)
\curveto(593.54898987,708.08940967)(593.6189898,708.0594097)(593.66899263,708.00942034)
\moveto(593.77399263,698.24442034)
\curveto(593.77398965,698.04441972)(593.76898965,697.87441989)(593.75899263,697.73442034)
\curveto(593.74898967,697.59442017)(593.65898976,697.49942026)(593.48899263,697.44942034)
\curveto(593.42898999,697.42942033)(593.36399006,697.41942034)(593.29399263,697.41942034)
\curveto(593.2239902,697.42942033)(593.14899027,697.43442033)(593.06899263,697.43442034)
\lineto(592.22899263,697.43442034)
\curveto(592.13899128,697.43442033)(592.04899137,697.43942032)(591.95899263,697.44942034)
\curveto(591.87899154,697.4594203)(591.8189916,697.48942027)(591.77899263,697.53942034)
\curveto(591.7189917,697.60942015)(591.68399174,697.69442007)(591.67399263,697.79442034)
\lineto(591.67399263,698.13942034)
\lineto(591.67399263,704.46942034)
\lineto(591.67399263,704.76942034)
\curveto(591.67399175,704.86941289)(591.69399173,704.94941281)(591.73399263,705.00942034)
\curveto(591.79399163,705.07941268)(591.87899154,705.12441264)(591.98899263,705.14442034)
\curveto(592.00899141,705.15441261)(592.03399139,705.15441261)(592.06399263,705.14442034)
\curveto(592.10399132,705.14441262)(592.13399129,705.14941261)(592.15399263,705.15942034)
\lineto(592.90399263,705.15942034)
\lineto(593.09899263,705.15942034)
\curveto(593.17899024,705.16941259)(593.24399018,705.16941259)(593.29399263,705.15942034)
\lineto(593.41399263,705.15942034)
\curveto(593.47398995,705.13941262)(593.52898989,705.12441264)(593.57899263,705.11442034)
\curveto(593.62898979,705.10441266)(593.66898975,705.07441269)(593.69899263,705.02442034)
\curveto(593.73898968,704.97441279)(593.75898966,704.90441286)(593.75899263,704.81442034)
\curveto(593.76898965,704.72441304)(593.77398965,704.62941313)(593.77399263,704.52942034)
\lineto(593.77399263,698.24442034)
}
}
{
\newrgbcolor{curcolor}{0 0 0}
\pscustom[linestyle=none,fillstyle=solid,fillcolor=curcolor]
{
\newpath
\moveto(602.48618013,698.01942034)
\curveto(602.50617228,697.90941985)(602.51617227,697.79941996)(602.51618013,697.68942034)
\curveto(602.52617226,697.57942018)(602.47617231,697.50442026)(602.36618013,697.46442034)
\curveto(602.30617248,697.43442033)(602.23617255,697.41942034)(602.15618013,697.41942034)
\lineto(601.91618013,697.41942034)
\lineto(601.10618013,697.41942034)
\lineto(600.83618013,697.41942034)
\curveto(600.75617403,697.42942033)(600.6911741,697.45442031)(600.64118013,697.49442034)
\curveto(600.57117422,697.53442023)(600.51617427,697.58942017)(600.47618013,697.65942034)
\curveto(600.44617434,697.73942002)(600.40117439,697.80441996)(600.34118013,697.85442034)
\curveto(600.32117447,697.87441989)(600.29617449,697.88941987)(600.26618013,697.89942034)
\curveto(600.23617455,697.91941984)(600.19617459,697.92441984)(600.14618013,697.91442034)
\curveto(600.09617469,697.89441987)(600.04617474,697.86941989)(599.99618013,697.83942034)
\curveto(599.95617483,697.80941995)(599.91117488,697.78441998)(599.86118013,697.76442034)
\curveto(599.81117498,697.72442004)(599.75617503,697.68942007)(599.69618013,697.65942034)
\lineto(599.51618013,697.56942034)
\curveto(599.3861754,697.50942025)(599.25117554,697.4594203)(599.11118013,697.41942034)
\curveto(598.97117582,697.38942037)(598.82617596,697.35442041)(598.67618013,697.31442034)
\curveto(598.60617618,697.29442047)(598.53617625,697.28442048)(598.46618013,697.28442034)
\curveto(598.40617638,697.27442049)(598.34117645,697.2644205)(598.27118013,697.25442034)
\lineto(598.18118013,697.25442034)
\curveto(598.15117664,697.24442052)(598.12117667,697.23942052)(598.09118013,697.23942034)
\lineto(597.92618013,697.23942034)
\curveto(597.82617696,697.21942054)(597.72617706,697.21942054)(597.62618013,697.23942034)
\lineto(597.49118013,697.23942034)
\curveto(597.42117737,697.2594205)(597.35117744,697.26942049)(597.28118013,697.26942034)
\curveto(597.22117757,697.2594205)(597.16117763,697.2644205)(597.10118013,697.28442034)
\curveto(597.00117779,697.30442046)(596.90617788,697.32442044)(596.81618013,697.34442034)
\curveto(596.72617806,697.35442041)(596.64117815,697.37942038)(596.56118013,697.41942034)
\curveto(596.27117852,697.52942023)(596.02117877,697.66942009)(595.81118013,697.83942034)
\curveto(595.61117918,698.01941974)(595.45117934,698.25441951)(595.33118013,698.54442034)
\curveto(595.30117949,698.61441915)(595.27117952,698.68941907)(595.24118013,698.76942034)
\curveto(595.22117957,698.84941891)(595.20117959,698.93441883)(595.18118013,699.02442034)
\curveto(595.16117963,699.07441869)(595.15117964,699.12441864)(595.15118013,699.17442034)
\curveto(595.16117963,699.22441854)(595.16117963,699.27441849)(595.15118013,699.32442034)
\curveto(595.14117965,699.35441841)(595.13117966,699.41441835)(595.12118013,699.50442034)
\curveto(595.12117967,699.60441816)(595.12617966,699.67441809)(595.13618013,699.71442034)
\curveto(595.15617963,699.81441795)(595.16617962,699.89941786)(595.16618013,699.96942034)
\lineto(595.25618013,700.29942034)
\curveto(595.2861795,700.41941734)(595.32617946,700.52441724)(595.37618013,700.61442034)
\curveto(595.54617924,700.90441686)(595.74117905,701.12441664)(595.96118013,701.27442034)
\curveto(596.18117861,701.42441634)(596.46117833,701.55441621)(596.80118013,701.66442034)
\curveto(596.93117786,701.71441605)(597.06617772,701.74941601)(597.20618013,701.76942034)
\curveto(597.34617744,701.78941597)(597.4861773,701.81441595)(597.62618013,701.84442034)
\curveto(597.70617708,701.8644159)(597.791177,701.87441589)(597.88118013,701.87442034)
\curveto(597.97117682,701.88441588)(598.06117673,701.89941586)(598.15118013,701.91942034)
\curveto(598.22117657,701.93941582)(598.2911765,701.94441582)(598.36118013,701.93442034)
\curveto(598.43117636,701.93441583)(598.50617628,701.94441582)(598.58618013,701.96442034)
\curveto(598.65617613,701.98441578)(598.72617606,701.99441577)(598.79618013,701.99442034)
\curveto(598.86617592,701.99441577)(598.94117585,702.00441576)(599.02118013,702.02442034)
\curveto(599.23117556,702.07441569)(599.42117537,702.11441565)(599.59118013,702.14442034)
\curveto(599.77117502,702.18441558)(599.93117486,702.27441549)(600.07118013,702.41442034)
\curveto(600.16117463,702.50441526)(600.22117457,702.60441516)(600.25118013,702.71442034)
\curveto(600.26117453,702.74441502)(600.26117453,702.76941499)(600.25118013,702.78942034)
\curveto(600.25117454,702.80941495)(600.25617453,702.82941493)(600.26618013,702.84942034)
\curveto(600.27617451,702.86941489)(600.28117451,702.89941486)(600.28118013,702.93942034)
\lineto(600.28118013,703.02942034)
\lineto(600.25118013,703.14942034)
\curveto(600.25117454,703.18941457)(600.24617454,703.22441454)(600.23618013,703.25442034)
\curveto(600.13617465,703.55441421)(599.92617486,703.759414)(599.60618013,703.86942034)
\curveto(599.51617527,703.89941386)(599.40617538,703.91941384)(599.27618013,703.92942034)
\curveto(599.15617563,703.94941381)(599.03117576,703.95441381)(598.90118013,703.94442034)
\curveto(598.77117602,703.94441382)(598.64617614,703.93441383)(598.52618013,703.91442034)
\curveto(598.40617638,703.89441387)(598.30117649,703.86941389)(598.21118013,703.83942034)
\curveto(598.15117664,703.81941394)(598.0911767,703.78941397)(598.03118013,703.74942034)
\curveto(597.98117681,703.71941404)(597.93117686,703.68441408)(597.88118013,703.64442034)
\curveto(597.83117696,703.60441416)(597.77617701,703.54941421)(597.71618013,703.47942034)
\curveto(597.66617712,703.40941435)(597.63117716,703.34441442)(597.61118013,703.28442034)
\curveto(597.56117723,703.18441458)(597.51617727,703.08941467)(597.47618013,702.99942034)
\curveto(597.44617734,702.90941485)(597.37617741,702.84941491)(597.26618013,702.81942034)
\curveto(597.1861776,702.79941496)(597.10117769,702.78941497)(597.01118013,702.78942034)
\lineto(596.74118013,702.78942034)
\lineto(596.17118013,702.78942034)
\curveto(596.12117867,702.78941497)(596.07117872,702.78441498)(596.02118013,702.77442034)
\curveto(595.97117882,702.77441499)(595.92617886,702.77941498)(595.88618013,702.78942034)
\lineto(595.75118013,702.78942034)
\curveto(595.73117906,702.79941496)(595.70617908,702.80441496)(595.67618013,702.80442034)
\curveto(595.64617914,702.80441496)(595.62117917,702.81441495)(595.60118013,702.83442034)
\curveto(595.52117927,702.85441491)(595.46617932,702.91941484)(595.43618013,703.02942034)
\curveto(595.42617936,703.07941468)(595.42617936,703.12941463)(595.43618013,703.17942034)
\curveto(595.44617934,703.22941453)(595.45617933,703.27441449)(595.46618013,703.31442034)
\curveto(595.49617929,703.42441434)(595.52617926,703.52441424)(595.55618013,703.61442034)
\curveto(595.59617919,703.71441405)(595.64117915,703.80441396)(595.69118013,703.88442034)
\lineto(595.78118013,704.03442034)
\lineto(595.87118013,704.18442034)
\curveto(595.95117884,704.29441347)(596.05117874,704.39941336)(596.17118013,704.49942034)
\curveto(596.1911786,704.50941325)(596.22117857,704.53441323)(596.26118013,704.57442034)
\curveto(596.31117848,704.61441315)(596.35617843,704.64941311)(596.39618013,704.67942034)
\curveto(596.43617835,704.70941305)(596.48117831,704.73941302)(596.53118013,704.76942034)
\curveto(596.70117809,704.87941288)(596.88117791,704.9644128)(597.07118013,705.02442034)
\curveto(597.26117753,705.09441267)(597.45617733,705.1594126)(597.65618013,705.21942034)
\curveto(597.77617701,705.24941251)(597.90117689,705.26941249)(598.03118013,705.27942034)
\curveto(598.16117663,705.28941247)(598.2911765,705.30941245)(598.42118013,705.33942034)
\curveto(598.46117633,705.34941241)(598.52117627,705.34941241)(598.60118013,705.33942034)
\curveto(598.6911761,705.32941243)(598.74617604,705.33441243)(598.76618013,705.35442034)
\curveto(599.17617561,705.3644124)(599.56617522,705.34941241)(599.93618013,705.30942034)
\curveto(600.31617447,705.26941249)(600.65617413,705.19441257)(600.95618013,705.08442034)
\curveto(601.26617352,704.97441279)(601.53117326,704.82441294)(601.75118013,704.63442034)
\curveto(601.97117282,704.45441331)(602.14117265,704.21941354)(602.26118013,703.92942034)
\curveto(602.33117246,703.759414)(602.37117242,703.5644142)(602.38118013,703.34442034)
\curveto(602.3911724,703.12441464)(602.39617239,702.89941486)(602.39618013,702.66942034)
\lineto(602.39618013,699.32442034)
\lineto(602.39618013,698.73942034)
\curveto(602.39617239,698.54941921)(602.41617237,698.37441939)(602.45618013,698.21442034)
\curveto(602.46617232,698.18441958)(602.47117232,698.14941961)(602.47118013,698.10942034)
\curveto(602.47117232,698.07941968)(602.47617231,698.04941971)(602.48618013,698.01942034)
\moveto(600.28118013,700.32942034)
\curveto(600.2911745,700.37941738)(600.29617449,700.43441733)(600.29618013,700.49442034)
\curveto(600.29617449,700.5644172)(600.2911745,700.62441714)(600.28118013,700.67442034)
\curveto(600.26117453,700.73441703)(600.25117454,700.78941697)(600.25118013,700.83942034)
\curveto(600.25117454,700.88941687)(600.23117456,700.92941683)(600.19118013,700.95942034)
\curveto(600.14117465,700.99941676)(600.06617472,701.01941674)(599.96618013,701.01942034)
\curveto(599.92617486,701.00941675)(599.8911749,700.99941676)(599.86118013,700.98942034)
\curveto(599.83117496,700.98941677)(599.79617499,700.98441678)(599.75618013,700.97442034)
\curveto(599.6861751,700.95441681)(599.61117518,700.93941682)(599.53118013,700.92942034)
\curveto(599.45117534,700.91941684)(599.37117542,700.90441686)(599.29118013,700.88442034)
\curveto(599.26117553,700.87441689)(599.21617557,700.86941689)(599.15618013,700.86942034)
\curveto(599.02617576,700.83941692)(598.89617589,700.81941694)(598.76618013,700.80942034)
\curveto(598.63617615,700.79941696)(598.51117628,700.77441699)(598.39118013,700.73442034)
\curveto(598.31117648,700.71441705)(598.23617655,700.69441707)(598.16618013,700.67442034)
\curveto(598.09617669,700.6644171)(598.02617676,700.64441712)(597.95618013,700.61442034)
\curveto(597.74617704,700.52441724)(597.56617722,700.38941737)(597.41618013,700.20942034)
\curveto(597.27617751,700.02941773)(597.22617756,699.77941798)(597.26618013,699.45942034)
\curveto(597.2861775,699.28941847)(597.34117745,699.14941861)(597.43118013,699.03942034)
\curveto(597.50117729,698.92941883)(597.60617718,698.83941892)(597.74618013,698.76942034)
\curveto(597.8861769,698.70941905)(598.03617675,698.6644191)(598.19618013,698.63442034)
\curveto(598.36617642,698.60441916)(598.54117625,698.59441917)(598.72118013,698.60442034)
\curveto(598.91117588,698.62441914)(599.0861757,698.6594191)(599.24618013,698.70942034)
\curveto(599.50617528,698.78941897)(599.71117508,698.91441885)(599.86118013,699.08442034)
\curveto(600.01117478,699.2644185)(600.12617466,699.48441828)(600.20618013,699.74442034)
\curveto(600.22617456,699.81441795)(600.23617455,699.88441788)(600.23618013,699.95442034)
\curveto(600.24617454,700.03441773)(600.26117453,700.11441765)(600.28118013,700.19442034)
\lineto(600.28118013,700.32942034)
}
}
{
\newrgbcolor{curcolor}{0 0 0}
\pscustom[linestyle=none,fillstyle=solid,fillcolor=curcolor]
{
\newpath
\moveto(413.39453951,682.62942034)
\curveto(413.40453083,682.56941644)(413.40953082,682.47941653)(413.40953951,682.35942034)
\curveto(413.40953082,682.23941677)(413.39953083,682.15441686)(413.37953951,682.10442034)
\lineto(413.37953951,681.90942034)
\curveto(413.34953088,681.79941721)(413.3295309,681.69441732)(413.31953951,681.59442034)
\curveto(413.31953091,681.49441752)(413.30453093,681.39441762)(413.27453951,681.29442034)
\curveto(413.25453098,681.20441781)(413.234531,681.1094179)(413.21453951,681.00942034)
\curveto(413.19453104,680.91941809)(413.16453107,680.82941818)(413.12453951,680.73942034)
\curveto(413.05453118,680.56941844)(412.98453125,680.4094186)(412.91453951,680.25942034)
\curveto(412.84453139,680.11941889)(412.76453147,679.97941903)(412.67453951,679.83942034)
\curveto(412.61453162,679.74941926)(412.54953168,679.66441935)(412.47953951,679.58442034)
\curveto(412.41953181,679.5144195)(412.34953188,679.43941957)(412.26953951,679.35942034)
\lineto(412.16453951,679.25442034)
\curveto(412.11453212,679.20441981)(412.05953217,679.15941985)(411.99953951,679.11942034)
\lineto(411.84953951,678.99942034)
\curveto(411.76953246,678.93942007)(411.67953255,678.88442013)(411.57953951,678.83442034)
\curveto(411.48953274,678.79442022)(411.39453284,678.74942026)(411.29453951,678.69942034)
\curveto(411.19453304,678.64942036)(411.08953314,678.6144204)(410.97953951,678.59442034)
\curveto(410.87953335,678.57442044)(410.77453346,678.55442046)(410.66453951,678.53442034)
\curveto(410.60453363,678.5144205)(410.53953369,678.50442051)(410.46953951,678.50442034)
\curveto(410.40953382,678.50442051)(410.34453389,678.49442052)(410.27453951,678.47442034)
\lineto(410.13953951,678.47442034)
\curveto(410.05953417,678.45442056)(409.98453425,678.45442056)(409.91453951,678.47442034)
\lineto(409.76453951,678.47442034)
\curveto(409.70453453,678.49442052)(409.63953459,678.50442051)(409.56953951,678.50442034)
\curveto(409.50953472,678.49442052)(409.44953478,678.49942051)(409.38953951,678.51942034)
\curveto(409.229535,678.56942044)(409.07453516,678.6144204)(408.92453951,678.65442034)
\curveto(408.78453545,678.69442032)(408.65453558,678.75442026)(408.53453951,678.83442034)
\curveto(408.46453577,678.87442014)(408.39953583,678.9144201)(408.33953951,678.95442034)
\curveto(408.27953595,679.00442001)(408.21453602,679.05441996)(408.14453951,679.10442034)
\lineto(407.96453951,679.23942034)
\curveto(407.88453635,679.29941971)(407.81453642,679.30441971)(407.75453951,679.25442034)
\curveto(407.70453653,679.22441979)(407.67953655,679.18441983)(407.67953951,679.13442034)
\curveto(407.67953655,679.09441992)(407.66953656,679.04441997)(407.64953951,678.98442034)
\curveto(407.6295366,678.88442013)(407.61953661,678.76942024)(407.61953951,678.63942034)
\curveto(407.6295366,678.5094205)(407.6345366,678.38942062)(407.63453951,678.27942034)
\lineto(407.63453951,676.74942034)
\curveto(407.6345366,676.61942239)(407.6295366,676.49442252)(407.61953951,676.37442034)
\curveto(407.61953661,676.24442277)(407.59453664,676.13942287)(407.54453951,676.05942034)
\curveto(407.51453672,676.01942299)(407.45953677,675.98942302)(407.37953951,675.96942034)
\curveto(407.29953693,675.94942306)(407.20953702,675.93942307)(407.10953951,675.93942034)
\curveto(407.00953722,675.92942308)(406.90953732,675.92942308)(406.80953951,675.93942034)
\lineto(406.55453951,675.93942034)
\lineto(406.14953951,675.93942034)
\lineto(406.04453951,675.93942034)
\curveto(406.00453823,675.93942307)(405.96953826,675.94442307)(405.93953951,675.95442034)
\lineto(405.81953951,675.95442034)
\curveto(405.64953858,676.00442301)(405.55953867,676.10442291)(405.54953951,676.25442034)
\curveto(405.53953869,676.39442262)(405.5345387,676.56442245)(405.53453951,676.76442034)
\lineto(405.53453951,685.56942034)
\curveto(405.5345387,685.67941333)(405.5295387,685.79441322)(405.51953951,685.91442034)
\curveto(405.51953871,686.04441297)(405.54453869,686.14441287)(405.59453951,686.21442034)
\curveto(405.6345386,686.28441273)(405.68953854,686.32941268)(405.75953951,686.34942034)
\curveto(405.80953842,686.36941264)(405.86953836,686.37941263)(405.93953951,686.37942034)
\lineto(406.16453951,686.37942034)
\lineto(406.88453951,686.37942034)
\lineto(407.16953951,686.37942034)
\curveto(407.25953697,686.37941263)(407.3345369,686.35441266)(407.39453951,686.30442034)
\curveto(407.46453677,686.25441276)(407.49953673,686.18941282)(407.49953951,686.10942034)
\curveto(407.50953672,686.03941297)(407.5345367,685.96441305)(407.57453951,685.88442034)
\curveto(407.58453665,685.85441316)(407.59453664,685.82941318)(407.60453951,685.80942034)
\curveto(407.62453661,685.79941321)(407.64453659,685.78441323)(407.66453951,685.76442034)
\curveto(407.77453646,685.75441326)(407.86453637,685.78441323)(407.93453951,685.85442034)
\curveto(408.00453623,685.92441309)(408.07453616,685.98441303)(408.14453951,686.03442034)
\curveto(408.27453596,686.12441289)(408.40953582,686.20441281)(408.54953951,686.27442034)
\curveto(408.68953554,686.35441266)(408.84453539,686.41941259)(409.01453951,686.46942034)
\curveto(409.09453514,686.49941251)(409.17953505,686.51941249)(409.26953951,686.52942034)
\curveto(409.36953486,686.53941247)(409.46453477,686.55441246)(409.55453951,686.57442034)
\curveto(409.59453464,686.58441243)(409.6345346,686.58441243)(409.67453951,686.57442034)
\curveto(409.72453451,686.56441245)(409.76453447,686.56941244)(409.79453951,686.58942034)
\curveto(410.36453387,686.6094124)(410.84453339,686.52941248)(411.23453951,686.34942034)
\curveto(411.6345326,686.17941283)(411.97453226,685.95441306)(412.25453951,685.67442034)
\curveto(412.30453193,685.62441339)(412.34953188,685.57441344)(412.38953951,685.52442034)
\curveto(412.4295318,685.48441353)(412.46953176,685.43941357)(412.50953951,685.38942034)
\curveto(412.57953165,685.29941371)(412.63953159,685.2094138)(412.68953951,685.11942034)
\curveto(412.74953148,685.02941398)(412.80453143,684.93941407)(412.85453951,684.84942034)
\curveto(412.87453136,684.82941418)(412.88453135,684.80441421)(412.88453951,684.77442034)
\curveto(412.89453134,684.74441427)(412.90953132,684.7094143)(412.92953951,684.66942034)
\curveto(412.98953124,684.56941444)(413.04453119,684.44941456)(413.09453951,684.30942034)
\curveto(413.11453112,684.24941476)(413.1345311,684.18441483)(413.15453951,684.11442034)
\curveto(413.17453106,684.05441496)(413.19453104,683.98941502)(413.21453951,683.91942034)
\curveto(413.25453098,683.79941521)(413.27953095,683.67441534)(413.28953951,683.54442034)
\curveto(413.30953092,683.4144156)(413.3345309,683.27941573)(413.36453951,683.13942034)
\lineto(413.36453951,682.97442034)
\lineto(413.39453951,682.79442034)
\lineto(413.39453951,682.62942034)
\moveto(411.27953951,682.28442034)
\curveto(411.28953294,682.33441668)(411.29453294,682.39941661)(411.29453951,682.47942034)
\curveto(411.29453294,682.56941644)(411.28953294,682.63941637)(411.27953951,682.68942034)
\lineto(411.27953951,682.82442034)
\curveto(411.25953297,682.88441613)(411.24953298,682.94941606)(411.24953951,683.01942034)
\curveto(411.24953298,683.08941592)(411.23953299,683.15941585)(411.21953951,683.22942034)
\curveto(411.19953303,683.32941568)(411.17953305,683.42441559)(411.15953951,683.51442034)
\curveto(411.13953309,683.6144154)(411.10953312,683.70441531)(411.06953951,683.78442034)
\curveto(410.94953328,684.10441491)(410.79453344,684.35941465)(410.60453951,684.54942034)
\curveto(410.41453382,684.73941427)(410.14453409,684.87941413)(409.79453951,684.96942034)
\curveto(409.71453452,684.98941402)(409.62453461,684.99941401)(409.52453951,684.99942034)
\lineto(409.25453951,684.99942034)
\curveto(409.21453502,684.98941402)(409.17953505,684.98441403)(409.14953951,684.98442034)
\curveto(409.11953511,684.98441403)(409.08453515,684.97941403)(409.04453951,684.96942034)
\lineto(408.83453951,684.90942034)
\curveto(408.77453546,684.89941411)(408.71453552,684.87941413)(408.65453951,684.84942034)
\curveto(408.39453584,684.73941427)(408.18953604,684.56941444)(408.03953951,684.33942034)
\curveto(407.89953633,684.1094149)(407.78453645,683.85441516)(407.69453951,683.57442034)
\curveto(407.67453656,683.49441552)(407.65953657,683.4094156)(407.64953951,683.31942034)
\curveto(407.63953659,683.23941577)(407.62453661,683.15941585)(407.60453951,683.07942034)
\curveto(407.59453664,683.03941597)(407.58953664,682.97441604)(407.58953951,682.88442034)
\curveto(407.56953666,682.84441617)(407.56453667,682.79441622)(407.57453951,682.73442034)
\curveto(407.58453665,682.68441633)(407.58453665,682.63441638)(407.57453951,682.58442034)
\curveto(407.55453668,682.52441649)(407.55453668,682.46941654)(407.57453951,682.41942034)
\lineto(407.57453951,682.23942034)
\lineto(407.57453951,682.10442034)
\curveto(407.57453666,682.06441695)(407.58453665,682.02441699)(407.60453951,681.98442034)
\curveto(407.60453663,681.9144171)(407.60953662,681.85941715)(407.61953951,681.81942034)
\lineto(407.64953951,681.63942034)
\curveto(407.65953657,681.57941743)(407.67453656,681.51941749)(407.69453951,681.45942034)
\curveto(407.78453645,681.16941784)(407.88953634,680.92941808)(408.00953951,680.73942034)
\curveto(408.13953609,680.55941845)(408.31953591,680.39941861)(408.54953951,680.25942034)
\curveto(408.68953554,680.17941883)(408.85453538,680.1144189)(409.04453951,680.06442034)
\curveto(409.08453515,680.05441896)(409.11953511,680.04941896)(409.14953951,680.04942034)
\curveto(409.17953505,680.05941895)(409.21453502,680.05941895)(409.25453951,680.04942034)
\curveto(409.29453494,680.03941897)(409.35453488,680.02941898)(409.43453951,680.01942034)
\curveto(409.51453472,680.01941899)(409.57953465,680.02441899)(409.62953951,680.03442034)
\curveto(409.70953452,680.05441896)(409.78953444,680.06941894)(409.86953951,680.07942034)
\curveto(409.95953427,680.09941891)(410.04453419,680.12441889)(410.12453951,680.15442034)
\curveto(410.36453387,680.25441876)(410.55953367,680.39441862)(410.70953951,680.57442034)
\curveto(410.85953337,680.75441826)(410.98453325,680.96441805)(411.08453951,681.20442034)
\curveto(411.1345331,681.32441769)(411.16953306,681.44941756)(411.18953951,681.57942034)
\curveto(411.20953302,681.7094173)(411.234533,681.84441717)(411.26453951,681.98442034)
\lineto(411.26453951,682.13442034)
\curveto(411.27453296,682.18441683)(411.27953295,682.23441678)(411.27953951,682.28442034)
}
}
{
\newrgbcolor{curcolor}{0 0 0}
\pscustom[linestyle=none,fillstyle=solid,fillcolor=curcolor]
{
\newpath
\moveto(422.44446138,682.85442034)
\curveto(422.46445281,682.79441622)(422.4744528,682.7094163)(422.47446138,682.59942034)
\curveto(422.4744528,682.48941652)(422.46445281,682.40441661)(422.44446138,682.34442034)
\lineto(422.44446138,682.19442034)
\curveto(422.42445285,682.1144169)(422.41445286,682.03441698)(422.41446138,681.95442034)
\curveto(422.42445285,681.87441714)(422.41945286,681.79441722)(422.39946138,681.71442034)
\curveto(422.3794529,681.64441737)(422.36445291,681.57941743)(422.35446138,681.51942034)
\curveto(422.34445293,681.45941755)(422.33445294,681.39441762)(422.32446138,681.32442034)
\curveto(422.28445299,681.2144178)(422.24945303,681.09941791)(422.21946138,680.97942034)
\curveto(422.18945309,680.86941814)(422.14945313,680.76441825)(422.09946138,680.66442034)
\curveto(421.88945339,680.18441883)(421.61445366,679.79441922)(421.27446138,679.49442034)
\curveto(420.93445434,679.19441982)(420.52445475,678.94442007)(420.04446138,678.74442034)
\curveto(419.92445535,678.69442032)(419.79945548,678.65942035)(419.66946138,678.63942034)
\curveto(419.54945573,678.6094204)(419.42445585,678.57942043)(419.29446138,678.54942034)
\curveto(419.24445603,678.52942048)(419.18945609,678.51942049)(419.12946138,678.51942034)
\curveto(419.06945621,678.51942049)(419.01445626,678.5144205)(418.96446138,678.50442034)
\lineto(418.85946138,678.50442034)
\curveto(418.82945645,678.49442052)(418.79945648,678.48942052)(418.76946138,678.48942034)
\curveto(418.71945656,678.47942053)(418.63945664,678.47442054)(418.52946138,678.47442034)
\curveto(418.41945686,678.46442055)(418.33445694,678.46942054)(418.27446138,678.48942034)
\lineto(418.12446138,678.48942034)
\curveto(418.0744572,678.49942051)(418.01945726,678.50442051)(417.95946138,678.50442034)
\curveto(417.90945737,678.49442052)(417.85945742,678.49942051)(417.80946138,678.51942034)
\curveto(417.76945751,678.52942048)(417.72945755,678.53442048)(417.68946138,678.53442034)
\curveto(417.65945762,678.53442048)(417.61945766,678.53942047)(417.56946138,678.54942034)
\curveto(417.46945781,678.57942043)(417.36945791,678.60442041)(417.26946138,678.62442034)
\curveto(417.16945811,678.64442037)(417.0744582,678.67442034)(416.98446138,678.71442034)
\curveto(416.86445841,678.75442026)(416.74945853,678.79442022)(416.63946138,678.83442034)
\curveto(416.53945874,678.87442014)(416.43445884,678.92442009)(416.32446138,678.98442034)
\curveto(415.9744593,679.19441982)(415.6744596,679.43941957)(415.42446138,679.71942034)
\curveto(415.1744601,679.99941901)(414.96446031,680.33441868)(414.79446138,680.72442034)
\curveto(414.74446053,680.8144182)(414.70446057,680.9094181)(414.67446138,681.00942034)
\curveto(414.65446062,681.1094179)(414.62946065,681.2144178)(414.59946138,681.32442034)
\curveto(414.5794607,681.37441764)(414.56946071,681.41941759)(414.56946138,681.45942034)
\curveto(414.56946071,681.49941751)(414.55946072,681.54441747)(414.53946138,681.59442034)
\curveto(414.51946076,681.67441734)(414.50946077,681.75441726)(414.50946138,681.83442034)
\curveto(414.50946077,681.92441709)(414.49946078,682.009417)(414.47946138,682.08942034)
\curveto(414.46946081,682.13941687)(414.46446081,682.18441683)(414.46446138,682.22442034)
\lineto(414.46446138,682.35942034)
\curveto(414.44446083,682.41941659)(414.43446084,682.50441651)(414.43446138,682.61442034)
\curveto(414.44446083,682.72441629)(414.45946082,682.8094162)(414.47946138,682.86942034)
\lineto(414.47946138,682.97442034)
\curveto(414.48946079,683.02441599)(414.48946079,683.07441594)(414.47946138,683.12442034)
\curveto(414.4794608,683.18441583)(414.48946079,683.23941577)(414.50946138,683.28942034)
\curveto(414.51946076,683.33941567)(414.52446075,683.38441563)(414.52446138,683.42442034)
\curveto(414.52446075,683.47441554)(414.53446074,683.52441549)(414.55446138,683.57442034)
\curveto(414.59446068,683.70441531)(414.62946065,683.82941518)(414.65946138,683.94942034)
\curveto(414.68946059,684.07941493)(414.72946055,684.20441481)(414.77946138,684.32442034)
\curveto(414.95946032,684.73441428)(415.1744601,685.07441394)(415.42446138,685.34442034)
\curveto(415.6744596,685.62441339)(415.9794593,685.87941313)(416.33946138,686.10942034)
\curveto(416.43945884,686.15941285)(416.54445873,686.20441281)(416.65446138,686.24442034)
\curveto(416.76445851,686.28441273)(416.8744584,686.32941268)(416.98446138,686.37942034)
\curveto(417.11445816,686.42941258)(417.24945803,686.46441255)(417.38946138,686.48442034)
\curveto(417.52945775,686.50441251)(417.6744576,686.53441248)(417.82446138,686.57442034)
\curveto(417.90445737,686.58441243)(417.9794573,686.58941242)(418.04946138,686.58942034)
\curveto(418.11945716,686.58941242)(418.18945709,686.59441242)(418.25946138,686.60442034)
\curveto(418.83945644,686.6144124)(419.33945594,686.55441246)(419.75946138,686.42442034)
\curveto(420.18945509,686.29441272)(420.56945471,686.1144129)(420.89946138,685.88442034)
\curveto(421.00945427,685.80441321)(421.11945416,685.7144133)(421.22946138,685.61442034)
\curveto(421.34945393,685.52441349)(421.44945383,685.42441359)(421.52946138,685.31442034)
\curveto(421.60945367,685.2144138)(421.6794536,685.1144139)(421.73946138,685.01442034)
\curveto(421.80945347,684.9144141)(421.8794534,684.8094142)(421.94946138,684.69942034)
\curveto(422.01945326,684.58941442)(422.0744532,684.46941454)(422.11446138,684.33942034)
\curveto(422.15445312,684.21941479)(422.19945308,684.08941492)(422.24946138,683.94942034)
\curveto(422.279453,683.86941514)(422.30445297,683.78441523)(422.32446138,683.69442034)
\lineto(422.38446138,683.42442034)
\curveto(422.39445288,683.38441563)(422.39945288,683.34441567)(422.39946138,683.30442034)
\curveto(422.39945288,683.26441575)(422.40445287,683.22441579)(422.41446138,683.18442034)
\curveto(422.43445284,683.13441588)(422.43945284,683.07941593)(422.42946138,683.01942034)
\curveto(422.41945286,682.95941605)(422.42445285,682.90441611)(422.44446138,682.85442034)
\moveto(420.34446138,682.31442034)
\curveto(420.35445492,682.36441665)(420.35945492,682.43441658)(420.35946138,682.52442034)
\curveto(420.35945492,682.62441639)(420.35445492,682.69941631)(420.34446138,682.74942034)
\lineto(420.34446138,682.86942034)
\curveto(420.32445495,682.91941609)(420.31445496,682.97441604)(420.31446138,683.03442034)
\curveto(420.31445496,683.09441592)(420.30945497,683.14941586)(420.29946138,683.19942034)
\curveto(420.29945498,683.23941577)(420.29445498,683.26941574)(420.28446138,683.28942034)
\lineto(420.22446138,683.52942034)
\curveto(420.21445506,683.61941539)(420.19445508,683.70441531)(420.16446138,683.78442034)
\curveto(420.05445522,684.04441497)(419.92445535,684.26441475)(419.77446138,684.44442034)
\curveto(419.62445565,684.63441438)(419.42445585,684.78441423)(419.17446138,684.89442034)
\curveto(419.11445616,684.9144141)(419.05445622,684.92941408)(418.99446138,684.93942034)
\curveto(418.93445634,684.95941405)(418.86945641,684.97941403)(418.79946138,684.99942034)
\curveto(418.71945656,685.01941399)(418.63445664,685.02441399)(418.54446138,685.01442034)
\lineto(418.27446138,685.01442034)
\curveto(418.24445703,684.99441402)(418.20945707,684.98441403)(418.16946138,684.98442034)
\curveto(418.12945715,684.99441402)(418.09445718,684.99441402)(418.06446138,684.98442034)
\lineto(417.85446138,684.92442034)
\curveto(417.79445748,684.9144141)(417.73945754,684.89441412)(417.68946138,684.86442034)
\curveto(417.43945784,684.75441426)(417.23445804,684.59441442)(417.07446138,684.38442034)
\curveto(416.92445835,684.18441483)(416.80445847,683.94941506)(416.71446138,683.67942034)
\curveto(416.68445859,683.57941543)(416.65945862,683.47441554)(416.63946138,683.36442034)
\curveto(416.62945865,683.25441576)(416.61445866,683.14441587)(416.59446138,683.03442034)
\curveto(416.58445869,682.98441603)(416.5794587,682.93441608)(416.57946138,682.88442034)
\lineto(416.57946138,682.73442034)
\curveto(416.55945872,682.66441635)(416.54945873,682.55941645)(416.54946138,682.41942034)
\curveto(416.55945872,682.27941673)(416.5744587,682.17441684)(416.59446138,682.10442034)
\lineto(416.59446138,681.96942034)
\curveto(416.61445866,681.88941712)(416.62945865,681.8094172)(416.63946138,681.72942034)
\curveto(416.64945863,681.65941735)(416.66445861,681.58441743)(416.68446138,681.50442034)
\curveto(416.78445849,681.20441781)(416.88945839,680.95941805)(416.99946138,680.76942034)
\curveto(417.11945816,680.58941842)(417.30445797,680.42441859)(417.55446138,680.27442034)
\curveto(417.62445765,680.22441879)(417.69945758,680.18441883)(417.77946138,680.15442034)
\curveto(417.86945741,680.12441889)(417.95945732,680.09941891)(418.04946138,680.07942034)
\curveto(418.08945719,680.06941894)(418.12445715,680.06441895)(418.15446138,680.06442034)
\curveto(418.18445709,680.07441894)(418.21945706,680.07441894)(418.25946138,680.06442034)
\lineto(418.37946138,680.03442034)
\curveto(418.42945685,680.03441898)(418.4744568,680.03941897)(418.51446138,680.04942034)
\lineto(418.63446138,680.04942034)
\curveto(418.71445656,680.06941894)(418.79445648,680.08441893)(418.87446138,680.09442034)
\curveto(418.95445632,680.10441891)(419.02945625,680.12441889)(419.09946138,680.15442034)
\curveto(419.35945592,680.25441876)(419.56945571,680.38941862)(419.72946138,680.55942034)
\curveto(419.88945539,680.72941828)(420.02445525,680.93941807)(420.13446138,681.18942034)
\curveto(420.1744551,681.28941772)(420.20445507,681.38941762)(420.22446138,681.48942034)
\curveto(420.24445503,681.58941742)(420.26945501,681.69441732)(420.29946138,681.80442034)
\curveto(420.30945497,681.84441717)(420.31445496,681.87941713)(420.31446138,681.90942034)
\curveto(420.31445496,681.94941706)(420.31945496,681.98941702)(420.32946138,682.02942034)
\lineto(420.32946138,682.16442034)
\curveto(420.32945495,682.2144168)(420.33445494,682.26441675)(420.34446138,682.31442034)
}
}
{
\newrgbcolor{curcolor}{0 0 0}
\pscustom[linestyle=none,fillstyle=solid,fillcolor=curcolor]
{
\newpath
\moveto(428.26938326,686.60442034)
\curveto(428.37937794,686.60441241)(428.47437785,686.59441242)(428.55438326,686.57442034)
\curveto(428.64437768,686.55441246)(428.71437761,686.5094125)(428.76438326,686.43942034)
\curveto(428.8243775,686.35941265)(428.85437747,686.21941279)(428.85438326,686.01942034)
\lineto(428.85438326,685.50942034)
\lineto(428.85438326,685.13442034)
\curveto(428.86437746,684.99441402)(428.84937747,684.88441413)(428.80938326,684.80442034)
\curveto(428.76937755,684.73441428)(428.70937761,684.68941432)(428.62938326,684.66942034)
\curveto(428.55937776,684.64941436)(428.47437785,684.63941437)(428.37438326,684.63942034)
\curveto(428.28437804,684.63941437)(428.18437814,684.64441437)(428.07438326,684.65442034)
\curveto(427.97437835,684.66441435)(427.87937844,684.65941435)(427.78938326,684.63942034)
\curveto(427.7193786,684.61941439)(427.64937867,684.60441441)(427.57938326,684.59442034)
\curveto(427.50937881,684.59441442)(427.44437888,684.58441443)(427.38438326,684.56442034)
\curveto(427.2243791,684.5144145)(427.06437926,684.43941457)(426.90438326,684.33942034)
\curveto(426.74437958,684.24941476)(426.6193797,684.14441487)(426.52938326,684.02442034)
\curveto(426.47937984,683.94441507)(426.4243799,683.85941515)(426.36438326,683.76942034)
\curveto(426.31438001,683.68941532)(426.26438006,683.60441541)(426.21438326,683.51442034)
\curveto(426.18438014,683.43441558)(426.15438017,683.34941566)(426.12438326,683.25942034)
\lineto(426.06438326,683.01942034)
\curveto(426.04438028,682.94941606)(426.03438029,682.87441614)(426.03438326,682.79442034)
\curveto(426.03438029,682.72441629)(426.0243803,682.65441636)(426.00438326,682.58442034)
\curveto(425.99438033,682.54441647)(425.98938033,682.50441651)(425.98938326,682.46442034)
\curveto(425.99938032,682.43441658)(425.99938032,682.40441661)(425.98938326,682.37442034)
\lineto(425.98938326,682.13442034)
\curveto(425.96938035,682.06441695)(425.96438036,681.98441703)(425.97438326,681.89442034)
\curveto(425.98438034,681.8144172)(425.98938033,681.73441728)(425.98938326,681.65442034)
\lineto(425.98938326,680.69442034)
\lineto(425.98938326,679.41942034)
\curveto(425.98938033,679.28941972)(425.98438034,679.16941984)(425.97438326,679.05942034)
\curveto(425.96438036,678.94942006)(425.93438039,678.85942015)(425.88438326,678.78942034)
\curveto(425.86438046,678.75942025)(425.82938049,678.73442028)(425.77938326,678.71442034)
\curveto(425.73938058,678.70442031)(425.69438063,678.69442032)(425.64438326,678.68442034)
\lineto(425.56938326,678.68442034)
\curveto(425.5193808,678.67442034)(425.46438086,678.66942034)(425.40438326,678.66942034)
\lineto(425.23938326,678.66942034)
\lineto(424.59438326,678.66942034)
\curveto(424.53438179,678.67942033)(424.46938185,678.68442033)(424.39938326,678.68442034)
\lineto(424.20438326,678.68442034)
\curveto(424.15438217,678.70442031)(424.10438222,678.71942029)(424.05438326,678.72942034)
\curveto(424.00438232,678.74942026)(423.96938235,678.78442023)(423.94938326,678.83442034)
\curveto(423.90938241,678.88442013)(423.88438244,678.95442006)(423.87438326,679.04442034)
\lineto(423.87438326,679.34442034)
\lineto(423.87438326,680.36442034)
\lineto(423.87438326,684.59442034)
\lineto(423.87438326,685.70442034)
\lineto(423.87438326,685.98942034)
\curveto(423.87438245,686.08941292)(423.89438243,686.16941284)(423.93438326,686.22942034)
\curveto(423.98438234,686.3094127)(424.05938226,686.35941265)(424.15938326,686.37942034)
\curveto(424.25938206,686.39941261)(424.37938194,686.4094126)(424.51938326,686.40942034)
\lineto(425.28438326,686.40942034)
\curveto(425.40438092,686.4094126)(425.50938081,686.39941261)(425.59938326,686.37942034)
\curveto(425.68938063,686.36941264)(425.75938056,686.32441269)(425.80938326,686.24442034)
\curveto(425.83938048,686.19441282)(425.85438047,686.12441289)(425.85438326,686.03442034)
\lineto(425.88438326,685.76442034)
\curveto(425.89438043,685.68441333)(425.90938041,685.6094134)(425.92938326,685.53942034)
\curveto(425.95938036,685.46941354)(426.00938031,685.43441358)(426.07938326,685.43442034)
\curveto(426.09938022,685.45441356)(426.1193802,685.46441355)(426.13938326,685.46442034)
\curveto(426.15938016,685.46441355)(426.17938014,685.47441354)(426.19938326,685.49442034)
\curveto(426.25938006,685.54441347)(426.30938001,685.59941341)(426.34938326,685.65942034)
\curveto(426.39937992,685.72941328)(426.45937986,685.78941322)(426.52938326,685.83942034)
\curveto(426.56937975,685.86941314)(426.60437972,685.89941311)(426.63438326,685.92942034)
\curveto(426.66437966,685.96941304)(426.69937962,686.00441301)(426.73938326,686.03442034)
\lineto(427.00938326,686.21442034)
\curveto(427.10937921,686.27441274)(427.20937911,686.32941268)(427.30938326,686.37942034)
\curveto(427.40937891,686.41941259)(427.50937881,686.45441256)(427.60938326,686.48442034)
\lineto(427.93938326,686.57442034)
\curveto(427.96937835,686.58441243)(428.0243783,686.58441243)(428.10438326,686.57442034)
\curveto(428.19437813,686.57441244)(428.24937807,686.58441243)(428.26938326,686.60442034)
}
}
{
\newrgbcolor{curcolor}{0 0 0}
\pscustom[linestyle=none,fillstyle=solid,fillcolor=curcolor]
{
}
}
{
\newrgbcolor{curcolor}{0 0 0}
\pscustom[linestyle=none,fillstyle=solid,fillcolor=curcolor]
{
\newpath
\moveto(434.88461763,688.71942034)
\lineto(435.88961763,688.71942034)
\curveto(436.03961465,688.71941029)(436.16961452,688.7094103)(436.27961763,688.68942034)
\curveto(436.39961429,688.67941033)(436.4846142,688.61941039)(436.53461763,688.50942034)
\curveto(436.55461413,688.45941055)(436.56461412,688.39941061)(436.56461763,688.32942034)
\lineto(436.56461763,688.11942034)
\lineto(436.56461763,687.44442034)
\curveto(436.56461412,687.39441162)(436.55961413,687.33441168)(436.54961763,687.26442034)
\curveto(436.54961414,687.20441181)(436.55461413,687.14941186)(436.56461763,687.09942034)
\lineto(436.56461763,686.93442034)
\curveto(436.56461412,686.85441216)(436.56961412,686.77941223)(436.57961763,686.70942034)
\curveto(436.5896141,686.64941236)(436.61461407,686.59441242)(436.65461763,686.54442034)
\curveto(436.72461396,686.45441256)(436.84961384,686.40441261)(437.02961763,686.39442034)
\lineto(437.56961763,686.39442034)
\lineto(437.74961763,686.39442034)
\curveto(437.80961288,686.39441262)(437.86461282,686.38441263)(437.91461763,686.36442034)
\curveto(438.02461266,686.3144127)(438.0846126,686.22441279)(438.09461763,686.09442034)
\curveto(438.11461257,685.96441305)(438.12461256,685.81941319)(438.12461763,685.65942034)
\lineto(438.12461763,685.44942034)
\curveto(438.13461255,685.37941363)(438.12961256,685.31941369)(438.10961763,685.26942034)
\curveto(438.05961263,685.1094139)(437.95461273,685.02441399)(437.79461763,685.01442034)
\curveto(437.63461305,685.00441401)(437.45461323,684.99941401)(437.25461763,684.99942034)
\lineto(437.11961763,684.99942034)
\curveto(437.07961361,685.009414)(437.04461364,685.009414)(437.01461763,684.99942034)
\curveto(436.97461371,684.98941402)(436.93961375,684.98441403)(436.90961763,684.98442034)
\curveto(436.87961381,684.99441402)(436.84961384,684.98941402)(436.81961763,684.96942034)
\curveto(436.73961395,684.94941406)(436.67961401,684.90441411)(436.63961763,684.83442034)
\curveto(436.60961408,684.77441424)(436.5846141,684.69941431)(436.56461763,684.60942034)
\curveto(436.55461413,684.55941445)(436.55461413,684.50441451)(436.56461763,684.44442034)
\curveto(436.57461411,684.38441463)(436.57461411,684.32941468)(436.56461763,684.27942034)
\lineto(436.56461763,683.34942034)
\lineto(436.56461763,681.59442034)
\curveto(436.56461412,681.34441767)(436.56961412,681.12441789)(436.57961763,680.93442034)
\curveto(436.59961409,680.75441826)(436.66461402,680.59441842)(436.77461763,680.45442034)
\curveto(436.82461386,680.39441862)(436.8896138,680.34941866)(436.96961763,680.31942034)
\lineto(437.23961763,680.25942034)
\curveto(437.26961342,680.24941876)(437.29961339,680.24441877)(437.32961763,680.24442034)
\curveto(437.36961332,680.25441876)(437.39961329,680.25441876)(437.41961763,680.24442034)
\lineto(437.58461763,680.24442034)
\curveto(437.69461299,680.24441877)(437.7896129,680.23941877)(437.86961763,680.22942034)
\curveto(437.94961274,680.21941879)(438.01461267,680.17941883)(438.06461763,680.10942034)
\curveto(438.10461258,680.04941896)(438.12461256,679.96941904)(438.12461763,679.86942034)
\lineto(438.12461763,679.58442034)
\curveto(438.12461256,679.37441964)(438.11961257,679.17941983)(438.10961763,678.99942034)
\curveto(438.10961258,678.82942018)(438.02961266,678.7144203)(437.86961763,678.65442034)
\curveto(437.81961287,678.63442038)(437.77461291,678.62942038)(437.73461763,678.63942034)
\curveto(437.69461299,678.63942037)(437.64961304,678.62942038)(437.59961763,678.60942034)
\lineto(437.44961763,678.60942034)
\curveto(437.42961326,678.6094204)(437.39961329,678.6144204)(437.35961763,678.62442034)
\curveto(437.31961337,678.62442039)(437.2846134,678.61942039)(437.25461763,678.60942034)
\curveto(437.20461348,678.59942041)(437.14961354,678.59942041)(437.08961763,678.60942034)
\lineto(436.93961763,678.60942034)
\lineto(436.78961763,678.60942034)
\curveto(436.73961395,678.59942041)(436.69461399,678.59942041)(436.65461763,678.60942034)
\lineto(436.48961763,678.60942034)
\curveto(436.43961425,678.61942039)(436.3846143,678.62442039)(436.32461763,678.62442034)
\curveto(436.26461442,678.62442039)(436.20961448,678.62942038)(436.15961763,678.63942034)
\curveto(436.0896146,678.64942036)(436.02461466,678.65942035)(435.96461763,678.66942034)
\lineto(435.78461763,678.69942034)
\curveto(435.67461501,678.72942028)(435.56961512,678.76442025)(435.46961763,678.80442034)
\curveto(435.36961532,678.84442017)(435.27461541,678.88942012)(435.18461763,678.93942034)
\lineto(435.09461763,678.99942034)
\curveto(435.06461562,679.02941998)(435.02961566,679.05941995)(434.98961763,679.08942034)
\curveto(434.96961572,679.1094199)(434.94461574,679.12941988)(434.91461763,679.14942034)
\lineto(434.83961763,679.22442034)
\curveto(434.69961599,679.4144196)(434.59461609,679.62441939)(434.52461763,679.85442034)
\curveto(434.50461618,679.89441912)(434.49461619,679.92941908)(434.49461763,679.95942034)
\curveto(434.50461618,679.99941901)(434.50461618,680.04441897)(434.49461763,680.09442034)
\curveto(434.4846162,680.1144189)(434.47961621,680.13941887)(434.47961763,680.16942034)
\curveto(434.47961621,680.19941881)(434.47461621,680.22441879)(434.46461763,680.24442034)
\lineto(434.46461763,680.39442034)
\curveto(434.45461623,680.43441858)(434.44961624,680.47941853)(434.44961763,680.52942034)
\curveto(434.45961623,680.57941843)(434.46461622,680.62941838)(434.46461763,680.67942034)
\lineto(434.46461763,681.24942034)
\lineto(434.46461763,683.48442034)
\lineto(434.46461763,684.27942034)
\lineto(434.46461763,684.48942034)
\curveto(434.47461621,684.55941445)(434.46961622,684.62441439)(434.44961763,684.68442034)
\curveto(434.40961628,684.82441419)(434.33961635,684.9144141)(434.23961763,684.95442034)
\curveto(434.12961656,685.00441401)(433.9896167,685.01941399)(433.81961763,684.99942034)
\curveto(433.64961704,684.97941403)(433.50461718,684.99441402)(433.38461763,685.04442034)
\curveto(433.30461738,685.07441394)(433.25461743,685.11941389)(433.23461763,685.17942034)
\curveto(433.21461747,685.23941377)(433.19461749,685.3144137)(433.17461763,685.40442034)
\lineto(433.17461763,685.71942034)
\curveto(433.17461751,685.89941311)(433.1846175,686.04441297)(433.20461763,686.15442034)
\curveto(433.22461746,686.26441275)(433.30961738,686.33941267)(433.45961763,686.37942034)
\curveto(433.49961719,686.39941261)(433.53961715,686.40441261)(433.57961763,686.39442034)
\lineto(433.71461763,686.39442034)
\curveto(433.86461682,686.39441262)(434.00461668,686.39941261)(434.13461763,686.40942034)
\curveto(434.26461642,686.42941258)(434.35461633,686.48941252)(434.40461763,686.58942034)
\curveto(434.43461625,686.65941235)(434.44961624,686.73941227)(434.44961763,686.82942034)
\curveto(434.45961623,686.91941209)(434.46461622,687.009412)(434.46461763,687.09942034)
\lineto(434.46461763,688.02942034)
\lineto(434.46461763,688.28442034)
\curveto(434.46461622,688.37441064)(434.47461621,688.44941056)(434.49461763,688.50942034)
\curveto(434.54461614,688.6094104)(434.61961607,688.67441034)(434.71961763,688.70442034)
\curveto(434.73961595,688.7144103)(434.76461592,688.7144103)(434.79461763,688.70442034)
\curveto(434.83461585,688.70441031)(434.86461582,688.7094103)(434.88461763,688.71942034)
}
}
{
\newrgbcolor{curcolor}{0 0 0}
\pscustom[linestyle=none,fillstyle=solid,fillcolor=curcolor]
{
\newpath
\moveto(441.20805513,689.25942034)
\curveto(441.27805218,689.17940983)(441.31305215,689.05940995)(441.31305513,688.89942034)
\lineto(441.31305513,688.43442034)
\lineto(441.31305513,688.02942034)
\curveto(441.31305215,687.88941112)(441.27805218,687.79441122)(441.20805513,687.74442034)
\curveto(441.14805231,687.69441132)(441.06805239,687.66441135)(440.96805513,687.65442034)
\curveto(440.87805258,687.64441137)(440.77805268,687.63941137)(440.66805513,687.63942034)
\lineto(439.82805513,687.63942034)
\curveto(439.71805374,687.63941137)(439.61805384,687.64441137)(439.52805513,687.65442034)
\curveto(439.44805401,687.66441135)(439.37805408,687.69441132)(439.31805513,687.74442034)
\curveto(439.27805418,687.77441124)(439.24805421,687.82941118)(439.22805513,687.90942034)
\curveto(439.21805424,687.99941101)(439.20805425,688.09441092)(439.19805513,688.19442034)
\lineto(439.19805513,688.52442034)
\curveto(439.20805425,688.63441038)(439.21305425,688.72941028)(439.21305513,688.80942034)
\lineto(439.21305513,689.01942034)
\curveto(439.22305424,689.08940992)(439.24305422,689.14940986)(439.27305513,689.19942034)
\curveto(439.29305417,689.23940977)(439.31805414,689.26940974)(439.34805513,689.28942034)
\lineto(439.46805513,689.34942034)
\curveto(439.48805397,689.34940966)(439.51305395,689.34940966)(439.54305513,689.34942034)
\curveto(439.57305389,689.35940965)(439.59805386,689.36440965)(439.61805513,689.36442034)
\lineto(440.71305513,689.36442034)
\curveto(440.81305265,689.36440965)(440.90805255,689.35940965)(440.99805513,689.34942034)
\curveto(441.08805237,689.33940967)(441.1580523,689.3094097)(441.20805513,689.25942034)
\moveto(441.31305513,679.49442034)
\curveto(441.31305215,679.29441972)(441.30805215,679.12441989)(441.29805513,678.98442034)
\curveto(441.28805217,678.84442017)(441.19805226,678.74942026)(441.02805513,678.69942034)
\curveto(440.96805249,678.67942033)(440.90305256,678.66942034)(440.83305513,678.66942034)
\curveto(440.7630527,678.67942033)(440.68805277,678.68442033)(440.60805513,678.68442034)
\lineto(439.76805513,678.68442034)
\curveto(439.67805378,678.68442033)(439.58805387,678.68942032)(439.49805513,678.69942034)
\curveto(439.41805404,678.7094203)(439.3580541,678.73942027)(439.31805513,678.78942034)
\curveto(439.2580542,678.85942015)(439.22305424,678.94442007)(439.21305513,679.04442034)
\lineto(439.21305513,679.38942034)
\lineto(439.21305513,685.71942034)
\lineto(439.21305513,686.01942034)
\curveto(439.21305425,686.11941289)(439.23305423,686.19941281)(439.27305513,686.25942034)
\curveto(439.33305413,686.32941268)(439.41805404,686.37441264)(439.52805513,686.39442034)
\curveto(439.54805391,686.40441261)(439.57305389,686.40441261)(439.60305513,686.39442034)
\curveto(439.64305382,686.39441262)(439.67305379,686.39941261)(439.69305513,686.40942034)
\lineto(440.44305513,686.40942034)
\lineto(440.63805513,686.40942034)
\curveto(440.71805274,686.41941259)(440.78305268,686.41941259)(440.83305513,686.40942034)
\lineto(440.95305513,686.40942034)
\curveto(441.01305245,686.38941262)(441.06805239,686.37441264)(441.11805513,686.36442034)
\curveto(441.16805229,686.35441266)(441.20805225,686.32441269)(441.23805513,686.27442034)
\curveto(441.27805218,686.22441279)(441.29805216,686.15441286)(441.29805513,686.06442034)
\curveto(441.30805215,685.97441304)(441.31305215,685.87941313)(441.31305513,685.77942034)
\lineto(441.31305513,679.49442034)
}
}
{
\newrgbcolor{curcolor}{0 0 0}
\pscustom[linestyle=none,fillstyle=solid,fillcolor=curcolor]
{
\newpath
\moveto(450.86524263,682.62942034)
\curveto(450.87523395,682.56941644)(450.88023395,682.47941653)(450.88024263,682.35942034)
\curveto(450.88023395,682.23941677)(450.87023396,682.15441686)(450.85024263,682.10442034)
\lineto(450.85024263,681.90942034)
\curveto(450.82023401,681.79941721)(450.80023403,681.69441732)(450.79024263,681.59442034)
\curveto(450.79023404,681.49441752)(450.77523405,681.39441762)(450.74524263,681.29442034)
\curveto(450.7252341,681.20441781)(450.70523412,681.1094179)(450.68524263,681.00942034)
\curveto(450.66523416,680.91941809)(450.63523419,680.82941818)(450.59524263,680.73942034)
\curveto(450.5252343,680.56941844)(450.45523437,680.4094186)(450.38524263,680.25942034)
\curveto(450.31523451,680.11941889)(450.23523459,679.97941903)(450.14524263,679.83942034)
\curveto(450.08523474,679.74941926)(450.02023481,679.66441935)(449.95024263,679.58442034)
\curveto(449.89023494,679.5144195)(449.82023501,679.43941957)(449.74024263,679.35942034)
\lineto(449.63524263,679.25442034)
\curveto(449.58523524,679.20441981)(449.5302353,679.15941985)(449.47024263,679.11942034)
\lineto(449.32024263,678.99942034)
\curveto(449.24023559,678.93942007)(449.15023568,678.88442013)(449.05024263,678.83442034)
\curveto(448.96023587,678.79442022)(448.86523596,678.74942026)(448.76524263,678.69942034)
\curveto(448.66523616,678.64942036)(448.56023627,678.6144204)(448.45024263,678.59442034)
\curveto(448.35023648,678.57442044)(448.24523658,678.55442046)(448.13524263,678.53442034)
\curveto(448.07523675,678.5144205)(448.01023682,678.50442051)(447.94024263,678.50442034)
\curveto(447.88023695,678.50442051)(447.81523701,678.49442052)(447.74524263,678.47442034)
\lineto(447.61024263,678.47442034)
\curveto(447.5302373,678.45442056)(447.45523737,678.45442056)(447.38524263,678.47442034)
\lineto(447.23524263,678.47442034)
\curveto(447.17523765,678.49442052)(447.11023772,678.50442051)(447.04024263,678.50442034)
\curveto(446.98023785,678.49442052)(446.92023791,678.49942051)(446.86024263,678.51942034)
\curveto(446.70023813,678.56942044)(446.54523828,678.6144204)(446.39524263,678.65442034)
\curveto(446.25523857,678.69442032)(446.1252387,678.75442026)(446.00524263,678.83442034)
\curveto(445.93523889,678.87442014)(445.87023896,678.9144201)(445.81024263,678.95442034)
\curveto(445.75023908,679.00442001)(445.68523914,679.05441996)(445.61524263,679.10442034)
\lineto(445.43524263,679.23942034)
\curveto(445.35523947,679.29941971)(445.28523954,679.30441971)(445.22524263,679.25442034)
\curveto(445.17523965,679.22441979)(445.15023968,679.18441983)(445.15024263,679.13442034)
\curveto(445.15023968,679.09441992)(445.14023969,679.04441997)(445.12024263,678.98442034)
\curveto(445.10023973,678.88442013)(445.09023974,678.76942024)(445.09024263,678.63942034)
\curveto(445.10023973,678.5094205)(445.10523972,678.38942062)(445.10524263,678.27942034)
\lineto(445.10524263,676.74942034)
\curveto(445.10523972,676.61942239)(445.10023973,676.49442252)(445.09024263,676.37442034)
\curveto(445.09023974,676.24442277)(445.06523976,676.13942287)(445.01524263,676.05942034)
\curveto(444.98523984,676.01942299)(444.9302399,675.98942302)(444.85024263,675.96942034)
\curveto(444.77024006,675.94942306)(444.68024015,675.93942307)(444.58024263,675.93942034)
\curveto(444.48024035,675.92942308)(444.38024045,675.92942308)(444.28024263,675.93942034)
\lineto(444.02524263,675.93942034)
\lineto(443.62024263,675.93942034)
\lineto(443.51524263,675.93942034)
\curveto(443.47524135,675.93942307)(443.44024139,675.94442307)(443.41024263,675.95442034)
\lineto(443.29024263,675.95442034)
\curveto(443.12024171,676.00442301)(443.0302418,676.10442291)(443.02024263,676.25442034)
\curveto(443.01024182,676.39442262)(443.00524182,676.56442245)(443.00524263,676.76442034)
\lineto(443.00524263,685.56942034)
\curveto(443.00524182,685.67941333)(443.00024183,685.79441322)(442.99024263,685.91442034)
\curveto(442.99024184,686.04441297)(443.01524181,686.14441287)(443.06524263,686.21442034)
\curveto(443.10524172,686.28441273)(443.16024167,686.32941268)(443.23024263,686.34942034)
\curveto(443.28024155,686.36941264)(443.34024149,686.37941263)(443.41024263,686.37942034)
\lineto(443.63524263,686.37942034)
\lineto(444.35524263,686.37942034)
\lineto(444.64024263,686.37942034)
\curveto(444.7302401,686.37941263)(444.80524002,686.35441266)(444.86524263,686.30442034)
\curveto(444.93523989,686.25441276)(444.97023986,686.18941282)(444.97024263,686.10942034)
\curveto(444.98023985,686.03941297)(445.00523982,685.96441305)(445.04524263,685.88442034)
\curveto(445.05523977,685.85441316)(445.06523976,685.82941318)(445.07524263,685.80942034)
\curveto(445.09523973,685.79941321)(445.11523971,685.78441323)(445.13524263,685.76442034)
\curveto(445.24523958,685.75441326)(445.33523949,685.78441323)(445.40524263,685.85442034)
\curveto(445.47523935,685.92441309)(445.54523928,685.98441303)(445.61524263,686.03442034)
\curveto(445.74523908,686.12441289)(445.88023895,686.20441281)(446.02024263,686.27442034)
\curveto(446.16023867,686.35441266)(446.31523851,686.41941259)(446.48524263,686.46942034)
\curveto(446.56523826,686.49941251)(446.65023818,686.51941249)(446.74024263,686.52942034)
\curveto(446.84023799,686.53941247)(446.93523789,686.55441246)(447.02524263,686.57442034)
\curveto(447.06523776,686.58441243)(447.10523772,686.58441243)(447.14524263,686.57442034)
\curveto(447.19523763,686.56441245)(447.23523759,686.56941244)(447.26524263,686.58942034)
\curveto(447.83523699,686.6094124)(448.31523651,686.52941248)(448.70524263,686.34942034)
\curveto(449.10523572,686.17941283)(449.44523538,685.95441306)(449.72524263,685.67442034)
\curveto(449.77523505,685.62441339)(449.82023501,685.57441344)(449.86024263,685.52442034)
\curveto(449.90023493,685.48441353)(449.94023489,685.43941357)(449.98024263,685.38942034)
\curveto(450.05023478,685.29941371)(450.11023472,685.2094138)(450.16024263,685.11942034)
\curveto(450.22023461,685.02941398)(450.27523455,684.93941407)(450.32524263,684.84942034)
\curveto(450.34523448,684.82941418)(450.35523447,684.80441421)(450.35524263,684.77442034)
\curveto(450.36523446,684.74441427)(450.38023445,684.7094143)(450.40024263,684.66942034)
\curveto(450.46023437,684.56941444)(450.51523431,684.44941456)(450.56524263,684.30942034)
\curveto(450.58523424,684.24941476)(450.60523422,684.18441483)(450.62524263,684.11442034)
\curveto(450.64523418,684.05441496)(450.66523416,683.98941502)(450.68524263,683.91942034)
\curveto(450.7252341,683.79941521)(450.75023408,683.67441534)(450.76024263,683.54442034)
\curveto(450.78023405,683.4144156)(450.80523402,683.27941573)(450.83524263,683.13942034)
\lineto(450.83524263,682.97442034)
\lineto(450.86524263,682.79442034)
\lineto(450.86524263,682.62942034)
\moveto(448.75024263,682.28442034)
\curveto(448.76023607,682.33441668)(448.76523606,682.39941661)(448.76524263,682.47942034)
\curveto(448.76523606,682.56941644)(448.76023607,682.63941637)(448.75024263,682.68942034)
\lineto(448.75024263,682.82442034)
\curveto(448.7302361,682.88441613)(448.72023611,682.94941606)(448.72024263,683.01942034)
\curveto(448.72023611,683.08941592)(448.71023612,683.15941585)(448.69024263,683.22942034)
\curveto(448.67023616,683.32941568)(448.65023618,683.42441559)(448.63024263,683.51442034)
\curveto(448.61023622,683.6144154)(448.58023625,683.70441531)(448.54024263,683.78442034)
\curveto(448.42023641,684.10441491)(448.26523656,684.35941465)(448.07524263,684.54942034)
\curveto(447.88523694,684.73941427)(447.61523721,684.87941413)(447.26524263,684.96942034)
\curveto(447.18523764,684.98941402)(447.09523773,684.99941401)(446.99524263,684.99942034)
\lineto(446.72524263,684.99942034)
\curveto(446.68523814,684.98941402)(446.65023818,684.98441403)(446.62024263,684.98442034)
\curveto(446.59023824,684.98441403)(446.55523827,684.97941403)(446.51524263,684.96942034)
\lineto(446.30524263,684.90942034)
\curveto(446.24523858,684.89941411)(446.18523864,684.87941413)(446.12524263,684.84942034)
\curveto(445.86523896,684.73941427)(445.66023917,684.56941444)(445.51024263,684.33942034)
\curveto(445.37023946,684.1094149)(445.25523957,683.85441516)(445.16524263,683.57442034)
\curveto(445.14523968,683.49441552)(445.1302397,683.4094156)(445.12024263,683.31942034)
\curveto(445.11023972,683.23941577)(445.09523973,683.15941585)(445.07524263,683.07942034)
\curveto(445.06523976,683.03941597)(445.06023977,682.97441604)(445.06024263,682.88442034)
\curveto(445.04023979,682.84441617)(445.03523979,682.79441622)(445.04524263,682.73442034)
\curveto(445.05523977,682.68441633)(445.05523977,682.63441638)(445.04524263,682.58442034)
\curveto(445.0252398,682.52441649)(445.0252398,682.46941654)(445.04524263,682.41942034)
\lineto(445.04524263,682.23942034)
\lineto(445.04524263,682.10442034)
\curveto(445.04523978,682.06441695)(445.05523977,682.02441699)(445.07524263,681.98442034)
\curveto(445.07523975,681.9144171)(445.08023975,681.85941715)(445.09024263,681.81942034)
\lineto(445.12024263,681.63942034)
\curveto(445.1302397,681.57941743)(445.14523968,681.51941749)(445.16524263,681.45942034)
\curveto(445.25523957,681.16941784)(445.36023947,680.92941808)(445.48024263,680.73942034)
\curveto(445.61023922,680.55941845)(445.79023904,680.39941861)(446.02024263,680.25942034)
\curveto(446.16023867,680.17941883)(446.3252385,680.1144189)(446.51524263,680.06442034)
\curveto(446.55523827,680.05441896)(446.59023824,680.04941896)(446.62024263,680.04942034)
\curveto(446.65023818,680.05941895)(446.68523814,680.05941895)(446.72524263,680.04942034)
\curveto(446.76523806,680.03941897)(446.825238,680.02941898)(446.90524263,680.01942034)
\curveto(446.98523784,680.01941899)(447.05023778,680.02441899)(447.10024263,680.03442034)
\curveto(447.18023765,680.05441896)(447.26023757,680.06941894)(447.34024263,680.07942034)
\curveto(447.4302374,680.09941891)(447.51523731,680.12441889)(447.59524263,680.15442034)
\curveto(447.83523699,680.25441876)(448.0302368,680.39441862)(448.18024263,680.57442034)
\curveto(448.3302365,680.75441826)(448.45523637,680.96441805)(448.55524263,681.20442034)
\curveto(448.60523622,681.32441769)(448.64023619,681.44941756)(448.66024263,681.57942034)
\curveto(448.68023615,681.7094173)(448.70523612,681.84441717)(448.73524263,681.98442034)
\lineto(448.73524263,682.13442034)
\curveto(448.74523608,682.18441683)(448.75023608,682.23441678)(448.75024263,682.28442034)
}
}
{
\newrgbcolor{curcolor}{0 0 0}
\pscustom[linestyle=none,fillstyle=solid,fillcolor=curcolor]
{
\newpath
\moveto(459.91516451,682.85442034)
\curveto(459.93515594,682.79441622)(459.94515593,682.7094163)(459.94516451,682.59942034)
\curveto(459.94515593,682.48941652)(459.93515594,682.40441661)(459.91516451,682.34442034)
\lineto(459.91516451,682.19442034)
\curveto(459.89515598,682.1144169)(459.88515599,682.03441698)(459.88516451,681.95442034)
\curveto(459.89515598,681.87441714)(459.89015598,681.79441722)(459.87016451,681.71442034)
\curveto(459.85015602,681.64441737)(459.83515604,681.57941743)(459.82516451,681.51942034)
\curveto(459.81515606,681.45941755)(459.80515607,681.39441762)(459.79516451,681.32442034)
\curveto(459.75515612,681.2144178)(459.72015615,681.09941791)(459.69016451,680.97942034)
\curveto(459.66015621,680.86941814)(459.62015625,680.76441825)(459.57016451,680.66442034)
\curveto(459.36015651,680.18441883)(459.08515679,679.79441922)(458.74516451,679.49442034)
\curveto(458.40515747,679.19441982)(457.99515788,678.94442007)(457.51516451,678.74442034)
\curveto(457.39515848,678.69442032)(457.2701586,678.65942035)(457.14016451,678.63942034)
\curveto(457.02015885,678.6094204)(456.89515898,678.57942043)(456.76516451,678.54942034)
\curveto(456.71515916,678.52942048)(456.66015921,678.51942049)(456.60016451,678.51942034)
\curveto(456.54015933,678.51942049)(456.48515939,678.5144205)(456.43516451,678.50442034)
\lineto(456.33016451,678.50442034)
\curveto(456.30015957,678.49442052)(456.2701596,678.48942052)(456.24016451,678.48942034)
\curveto(456.19015968,678.47942053)(456.11015976,678.47442054)(456.00016451,678.47442034)
\curveto(455.89015998,678.46442055)(455.80516007,678.46942054)(455.74516451,678.48942034)
\lineto(455.59516451,678.48942034)
\curveto(455.54516033,678.49942051)(455.49016038,678.50442051)(455.43016451,678.50442034)
\curveto(455.38016049,678.49442052)(455.33016054,678.49942051)(455.28016451,678.51942034)
\curveto(455.24016063,678.52942048)(455.20016067,678.53442048)(455.16016451,678.53442034)
\curveto(455.13016074,678.53442048)(455.09016078,678.53942047)(455.04016451,678.54942034)
\curveto(454.94016093,678.57942043)(454.84016103,678.60442041)(454.74016451,678.62442034)
\curveto(454.64016123,678.64442037)(454.54516133,678.67442034)(454.45516451,678.71442034)
\curveto(454.33516154,678.75442026)(454.22016165,678.79442022)(454.11016451,678.83442034)
\curveto(454.01016186,678.87442014)(453.90516197,678.92442009)(453.79516451,678.98442034)
\curveto(453.44516243,679.19441982)(453.14516273,679.43941957)(452.89516451,679.71942034)
\curveto(452.64516323,679.99941901)(452.43516344,680.33441868)(452.26516451,680.72442034)
\curveto(452.21516366,680.8144182)(452.1751637,680.9094181)(452.14516451,681.00942034)
\curveto(452.12516375,681.1094179)(452.10016377,681.2144178)(452.07016451,681.32442034)
\curveto(452.05016382,681.37441764)(452.04016383,681.41941759)(452.04016451,681.45942034)
\curveto(452.04016383,681.49941751)(452.03016384,681.54441747)(452.01016451,681.59442034)
\curveto(451.99016388,681.67441734)(451.98016389,681.75441726)(451.98016451,681.83442034)
\curveto(451.98016389,681.92441709)(451.9701639,682.009417)(451.95016451,682.08942034)
\curveto(451.94016393,682.13941687)(451.93516394,682.18441683)(451.93516451,682.22442034)
\lineto(451.93516451,682.35942034)
\curveto(451.91516396,682.41941659)(451.90516397,682.50441651)(451.90516451,682.61442034)
\curveto(451.91516396,682.72441629)(451.93016394,682.8094162)(451.95016451,682.86942034)
\lineto(451.95016451,682.97442034)
\curveto(451.96016391,683.02441599)(451.96016391,683.07441594)(451.95016451,683.12442034)
\curveto(451.95016392,683.18441583)(451.96016391,683.23941577)(451.98016451,683.28942034)
\curveto(451.99016388,683.33941567)(451.99516388,683.38441563)(451.99516451,683.42442034)
\curveto(451.99516388,683.47441554)(452.00516387,683.52441549)(452.02516451,683.57442034)
\curveto(452.06516381,683.70441531)(452.10016377,683.82941518)(452.13016451,683.94942034)
\curveto(452.16016371,684.07941493)(452.20016367,684.20441481)(452.25016451,684.32442034)
\curveto(452.43016344,684.73441428)(452.64516323,685.07441394)(452.89516451,685.34442034)
\curveto(453.14516273,685.62441339)(453.45016242,685.87941313)(453.81016451,686.10942034)
\curveto(453.91016196,686.15941285)(454.01516186,686.20441281)(454.12516451,686.24442034)
\curveto(454.23516164,686.28441273)(454.34516153,686.32941268)(454.45516451,686.37942034)
\curveto(454.58516129,686.42941258)(454.72016115,686.46441255)(454.86016451,686.48442034)
\curveto(455.00016087,686.50441251)(455.14516073,686.53441248)(455.29516451,686.57442034)
\curveto(455.3751605,686.58441243)(455.45016042,686.58941242)(455.52016451,686.58942034)
\curveto(455.59016028,686.58941242)(455.66016021,686.59441242)(455.73016451,686.60442034)
\curveto(456.31015956,686.6144124)(456.81015906,686.55441246)(457.23016451,686.42442034)
\curveto(457.66015821,686.29441272)(458.04015783,686.1144129)(458.37016451,685.88442034)
\curveto(458.48015739,685.80441321)(458.59015728,685.7144133)(458.70016451,685.61442034)
\curveto(458.82015705,685.52441349)(458.92015695,685.42441359)(459.00016451,685.31442034)
\curveto(459.08015679,685.2144138)(459.15015672,685.1144139)(459.21016451,685.01442034)
\curveto(459.28015659,684.9144141)(459.35015652,684.8094142)(459.42016451,684.69942034)
\curveto(459.49015638,684.58941442)(459.54515633,684.46941454)(459.58516451,684.33942034)
\curveto(459.62515625,684.21941479)(459.6701562,684.08941492)(459.72016451,683.94942034)
\curveto(459.75015612,683.86941514)(459.7751561,683.78441523)(459.79516451,683.69442034)
\lineto(459.85516451,683.42442034)
\curveto(459.86515601,683.38441563)(459.870156,683.34441567)(459.87016451,683.30442034)
\curveto(459.870156,683.26441575)(459.875156,683.22441579)(459.88516451,683.18442034)
\curveto(459.90515597,683.13441588)(459.91015596,683.07941593)(459.90016451,683.01942034)
\curveto(459.89015598,682.95941605)(459.89515598,682.90441611)(459.91516451,682.85442034)
\moveto(457.81516451,682.31442034)
\curveto(457.82515805,682.36441665)(457.83015804,682.43441658)(457.83016451,682.52442034)
\curveto(457.83015804,682.62441639)(457.82515805,682.69941631)(457.81516451,682.74942034)
\lineto(457.81516451,682.86942034)
\curveto(457.79515808,682.91941609)(457.78515809,682.97441604)(457.78516451,683.03442034)
\curveto(457.78515809,683.09441592)(457.78015809,683.14941586)(457.77016451,683.19942034)
\curveto(457.7701581,683.23941577)(457.76515811,683.26941574)(457.75516451,683.28942034)
\lineto(457.69516451,683.52942034)
\curveto(457.68515819,683.61941539)(457.66515821,683.70441531)(457.63516451,683.78442034)
\curveto(457.52515835,684.04441497)(457.39515848,684.26441475)(457.24516451,684.44442034)
\curveto(457.09515878,684.63441438)(456.89515898,684.78441423)(456.64516451,684.89442034)
\curveto(456.58515929,684.9144141)(456.52515935,684.92941408)(456.46516451,684.93942034)
\curveto(456.40515947,684.95941405)(456.34015953,684.97941403)(456.27016451,684.99942034)
\curveto(456.19015968,685.01941399)(456.10515977,685.02441399)(456.01516451,685.01442034)
\lineto(455.74516451,685.01442034)
\curveto(455.71516016,684.99441402)(455.68016019,684.98441403)(455.64016451,684.98442034)
\curveto(455.60016027,684.99441402)(455.56516031,684.99441402)(455.53516451,684.98442034)
\lineto(455.32516451,684.92442034)
\curveto(455.26516061,684.9144141)(455.21016066,684.89441412)(455.16016451,684.86442034)
\curveto(454.91016096,684.75441426)(454.70516117,684.59441442)(454.54516451,684.38442034)
\curveto(454.39516148,684.18441483)(454.2751616,683.94941506)(454.18516451,683.67942034)
\curveto(454.15516172,683.57941543)(454.13016174,683.47441554)(454.11016451,683.36442034)
\curveto(454.10016177,683.25441576)(454.08516179,683.14441587)(454.06516451,683.03442034)
\curveto(454.05516182,682.98441603)(454.05016182,682.93441608)(454.05016451,682.88442034)
\lineto(454.05016451,682.73442034)
\curveto(454.03016184,682.66441635)(454.02016185,682.55941645)(454.02016451,682.41942034)
\curveto(454.03016184,682.27941673)(454.04516183,682.17441684)(454.06516451,682.10442034)
\lineto(454.06516451,681.96942034)
\curveto(454.08516179,681.88941712)(454.10016177,681.8094172)(454.11016451,681.72942034)
\curveto(454.12016175,681.65941735)(454.13516174,681.58441743)(454.15516451,681.50442034)
\curveto(454.25516162,681.20441781)(454.36016151,680.95941805)(454.47016451,680.76942034)
\curveto(454.59016128,680.58941842)(454.7751611,680.42441859)(455.02516451,680.27442034)
\curveto(455.09516078,680.22441879)(455.1701607,680.18441883)(455.25016451,680.15442034)
\curveto(455.34016053,680.12441889)(455.43016044,680.09941891)(455.52016451,680.07942034)
\curveto(455.56016031,680.06941894)(455.59516028,680.06441895)(455.62516451,680.06442034)
\curveto(455.65516022,680.07441894)(455.69016018,680.07441894)(455.73016451,680.06442034)
\lineto(455.85016451,680.03442034)
\curveto(455.90015997,680.03441898)(455.94515993,680.03941897)(455.98516451,680.04942034)
\lineto(456.10516451,680.04942034)
\curveto(456.18515969,680.06941894)(456.26515961,680.08441893)(456.34516451,680.09442034)
\curveto(456.42515945,680.10441891)(456.50015937,680.12441889)(456.57016451,680.15442034)
\curveto(456.83015904,680.25441876)(457.04015883,680.38941862)(457.20016451,680.55942034)
\curveto(457.36015851,680.72941828)(457.49515838,680.93941807)(457.60516451,681.18942034)
\curveto(457.64515823,681.28941772)(457.6751582,681.38941762)(457.69516451,681.48942034)
\curveto(457.71515816,681.58941742)(457.74015813,681.69441732)(457.77016451,681.80442034)
\curveto(457.78015809,681.84441717)(457.78515809,681.87941713)(457.78516451,681.90942034)
\curveto(457.78515809,681.94941706)(457.79015808,681.98941702)(457.80016451,682.02942034)
\lineto(457.80016451,682.16442034)
\curveto(457.80015807,682.2144168)(457.80515807,682.26441675)(457.81516451,682.31442034)
}
}
{
\newrgbcolor{curcolor}{0 0 0}
\pscustom[linestyle=none,fillstyle=solid,fillcolor=curcolor]
{
}
}
{
\newrgbcolor{curcolor}{0 0 0}
\pscustom[linestyle=none,fillstyle=solid,fillcolor=curcolor]
{
\newpath
\moveto(473.06524263,679.52442034)
\lineto(473.06524263,679.10442034)
\curveto(473.06523426,678.97442004)(473.03523429,678.86942014)(472.97524263,678.78942034)
\curveto(472.9252344,678.73942027)(472.86023447,678.70442031)(472.78024263,678.68442034)
\curveto(472.70023463,678.67442034)(472.61023472,678.66942034)(472.51024263,678.66942034)
\lineto(471.68524263,678.66942034)
\lineto(471.40024263,678.66942034)
\curveto(471.32023601,678.67942033)(471.25523607,678.70442031)(471.20524263,678.74442034)
\curveto(471.13523619,678.79442022)(471.09523623,678.85942015)(471.08524263,678.93942034)
\curveto(471.07523625,679.01941999)(471.05523627,679.09941991)(471.02524263,679.17942034)
\curveto(471.00523632,679.19941981)(470.98523634,679.2144198)(470.96524263,679.22442034)
\curveto(470.95523637,679.24441977)(470.94023639,679.26441975)(470.92024263,679.28442034)
\curveto(470.81023652,679.28441973)(470.7302366,679.25941975)(470.68024263,679.20942034)
\lineto(470.53024263,679.05942034)
\curveto(470.46023687,679.00942)(470.39523693,678.96442005)(470.33524263,678.92442034)
\curveto(470.27523705,678.89442012)(470.21023712,678.85442016)(470.14024263,678.80442034)
\curveto(470.10023723,678.78442023)(470.05523727,678.76442025)(470.00524263,678.74442034)
\curveto(469.96523736,678.72442029)(469.92023741,678.70442031)(469.87024263,678.68442034)
\curveto(469.7302376,678.63442038)(469.58023775,678.58942042)(469.42024263,678.54942034)
\curveto(469.37023796,678.52942048)(469.325238,678.51942049)(469.28524263,678.51942034)
\curveto(469.24523808,678.51942049)(469.20523812,678.5144205)(469.16524263,678.50442034)
\lineto(469.03024263,678.50442034)
\curveto(469.00023833,678.49442052)(468.96023837,678.48942052)(468.91024263,678.48942034)
\lineto(468.77524263,678.48942034)
\curveto(468.71523861,678.46942054)(468.6252387,678.46442055)(468.50524263,678.47442034)
\curveto(468.38523894,678.47442054)(468.30023903,678.48442053)(468.25024263,678.50442034)
\curveto(468.18023915,678.52442049)(468.11523921,678.53442048)(468.05524263,678.53442034)
\curveto(468.00523932,678.52442049)(467.95023938,678.52942048)(467.89024263,678.54942034)
\lineto(467.53024263,678.66942034)
\curveto(467.42023991,678.69942031)(467.31024002,678.73942027)(467.20024263,678.78942034)
\curveto(466.85024048,678.93942007)(466.53524079,679.16941984)(466.25524263,679.47942034)
\curveto(465.98524134,679.79941921)(465.77024156,680.13441888)(465.61024263,680.48442034)
\curveto(465.56024177,680.59441842)(465.52024181,680.69941831)(465.49024263,680.79942034)
\curveto(465.46024187,680.9094181)(465.4252419,681.01941799)(465.38524263,681.12942034)
\curveto(465.37524195,681.16941784)(465.37024196,681.20441781)(465.37024263,681.23442034)
\curveto(465.37024196,681.27441774)(465.36024197,681.31941769)(465.34024263,681.36942034)
\curveto(465.32024201,681.44941756)(465.30024203,681.53441748)(465.28024263,681.62442034)
\curveto(465.27024206,681.72441729)(465.25524207,681.82441719)(465.23524263,681.92442034)
\curveto(465.2252421,681.95441706)(465.22024211,681.98941702)(465.22024263,682.02942034)
\curveto(465.2302421,682.06941694)(465.2302421,682.10441691)(465.22024263,682.13442034)
\lineto(465.22024263,682.26942034)
\curveto(465.22024211,682.31941669)(465.21524211,682.36941664)(465.20524263,682.41942034)
\curveto(465.19524213,682.46941654)(465.19024214,682.52441649)(465.19024263,682.58442034)
\curveto(465.19024214,682.65441636)(465.19524213,682.7094163)(465.20524263,682.74942034)
\curveto(465.21524211,682.79941621)(465.22024211,682.84441617)(465.22024263,682.88442034)
\lineto(465.22024263,683.03442034)
\curveto(465.2302421,683.08441593)(465.2302421,683.12941588)(465.22024263,683.16942034)
\curveto(465.22024211,683.21941579)(465.2302421,683.26941574)(465.25024263,683.31942034)
\curveto(465.27024206,683.42941558)(465.28524204,683.53441548)(465.29524263,683.63442034)
\curveto(465.31524201,683.73441528)(465.34024199,683.83441518)(465.37024263,683.93442034)
\curveto(465.41024192,684.05441496)(465.44524188,684.16941484)(465.47524263,684.27942034)
\curveto(465.50524182,684.38941462)(465.54524178,684.49941451)(465.59524263,684.60942034)
\curveto(465.73524159,684.9094141)(465.91024142,685.19441382)(466.12024263,685.46442034)
\curveto(466.14024119,685.49441352)(466.16524116,685.51941349)(466.19524263,685.53942034)
\curveto(466.23524109,685.56941344)(466.26524106,685.59941341)(466.28524263,685.62942034)
\curveto(466.325241,685.67941333)(466.36524096,685.72441329)(466.40524263,685.76442034)
\curveto(466.44524088,685.80441321)(466.49024084,685.84441317)(466.54024263,685.88442034)
\curveto(466.58024075,685.90441311)(466.61524071,685.92941308)(466.64524263,685.95942034)
\curveto(466.67524065,685.99941301)(466.71024062,686.02941298)(466.75024263,686.04942034)
\curveto(467.00024033,686.21941279)(467.29024004,686.35941265)(467.62024263,686.46942034)
\curveto(467.69023964,686.48941252)(467.76023957,686.50441251)(467.83024263,686.51442034)
\curveto(467.91023942,686.52441249)(467.99023934,686.53941247)(468.07024263,686.55942034)
\curveto(468.14023919,686.57941243)(468.2302391,686.58941242)(468.34024263,686.58942034)
\curveto(468.45023888,686.59941241)(468.56023877,686.60441241)(468.67024263,686.60442034)
\curveto(468.78023855,686.60441241)(468.88523844,686.59941241)(468.98524263,686.58942034)
\curveto(469.09523823,686.57941243)(469.18523814,686.56441245)(469.25524263,686.54442034)
\curveto(469.40523792,686.49441252)(469.55023778,686.44941256)(469.69024263,686.40942034)
\curveto(469.8302375,686.36941264)(469.96023737,686.3144127)(470.08024263,686.24442034)
\curveto(470.15023718,686.19441282)(470.21523711,686.14441287)(470.27524263,686.09442034)
\curveto(470.33523699,686.05441296)(470.40023693,686.009413)(470.47024263,685.95942034)
\curveto(470.51023682,685.92941308)(470.56523676,685.88941312)(470.63524263,685.83942034)
\curveto(470.71523661,685.78941322)(470.79023654,685.78941322)(470.86024263,685.83942034)
\curveto(470.90023643,685.85941315)(470.92023641,685.89441312)(470.92024263,685.94442034)
\curveto(470.92023641,685.99441302)(470.9302364,686.04441297)(470.95024263,686.09442034)
\lineto(470.95024263,686.24442034)
\curveto(470.96023637,686.27441274)(470.96523636,686.3094127)(470.96524263,686.34942034)
\lineto(470.96524263,686.46942034)
\lineto(470.96524263,688.50942034)
\curveto(470.96523636,688.61941039)(470.96023637,688.73941027)(470.95024263,688.86942034)
\curveto(470.95023638,689.00941)(470.97523635,689.1144099)(471.02524263,689.18442034)
\curveto(471.06523626,689.26440975)(471.14023619,689.3144097)(471.25024263,689.33442034)
\curveto(471.27023606,689.34440967)(471.29023604,689.34440967)(471.31024263,689.33442034)
\curveto(471.330236,689.33440968)(471.35023598,689.33940967)(471.37024263,689.34942034)
\lineto(472.43524263,689.34942034)
\curveto(472.55523477,689.34940966)(472.66523466,689.34440967)(472.76524263,689.33442034)
\curveto(472.86523446,689.32440969)(472.94023439,689.28440973)(472.99024263,689.21442034)
\curveto(473.04023429,689.13440988)(473.06523426,689.02940998)(473.06524263,688.89942034)
\lineto(473.06524263,688.53942034)
\lineto(473.06524263,679.52442034)
\moveto(471.02524263,682.46442034)
\curveto(471.03523629,682.50441651)(471.03523629,682.54441647)(471.02524263,682.58442034)
\lineto(471.02524263,682.71942034)
\curveto(471.0252363,682.81941619)(471.02023631,682.91941609)(471.01024263,683.01942034)
\curveto(471.00023633,683.11941589)(470.98523634,683.2094158)(470.96524263,683.28942034)
\curveto(470.94523638,683.39941561)(470.9252364,683.49941551)(470.90524263,683.58942034)
\curveto(470.89523643,683.67941533)(470.87023646,683.76441525)(470.83024263,683.84442034)
\curveto(470.69023664,684.20441481)(470.48523684,684.48941452)(470.21524263,684.69942034)
\curveto(469.95523737,684.9094141)(469.57523775,685.014414)(469.07524263,685.01442034)
\curveto(469.01523831,685.014414)(468.93523839,685.00441401)(468.83524263,684.98442034)
\curveto(468.75523857,684.96441405)(468.68023865,684.94441407)(468.61024263,684.92442034)
\curveto(468.55023878,684.9144141)(468.49023884,684.89441412)(468.43024263,684.86442034)
\curveto(468.16023917,684.75441426)(467.95023938,684.58441443)(467.80024263,684.35442034)
\curveto(467.65023968,684.12441489)(467.5302398,683.86441515)(467.44024263,683.57442034)
\curveto(467.41023992,683.47441554)(467.39023994,683.37441564)(467.38024263,683.27442034)
\curveto(467.37023996,683.17441584)(467.35023998,683.06941594)(467.32024263,682.95942034)
\lineto(467.32024263,682.74942034)
\curveto(467.30024003,682.65941635)(467.29524003,682.53441648)(467.30524263,682.37442034)
\curveto(467.31524001,682.22441679)(467.33024,682.1144169)(467.35024263,682.04442034)
\lineto(467.35024263,681.95442034)
\curveto(467.36023997,681.93441708)(467.36523996,681.9144171)(467.36524263,681.89442034)
\curveto(467.38523994,681.8144172)(467.40023993,681.73941727)(467.41024263,681.66942034)
\curveto(467.4302399,681.59941741)(467.45023988,681.52441749)(467.47024263,681.44442034)
\curveto(467.64023969,680.92441809)(467.9302394,680.53941847)(468.34024263,680.28942034)
\curveto(468.47023886,680.19941881)(468.65023868,680.12941888)(468.88024263,680.07942034)
\curveto(468.92023841,680.06941894)(468.98023835,680.06441895)(469.06024263,680.06442034)
\curveto(469.09023824,680.05441896)(469.13523819,680.04441897)(469.19524263,680.03442034)
\curveto(469.26523806,680.03441898)(469.32023801,680.03941897)(469.36024263,680.04942034)
\curveto(469.44023789,680.06941894)(469.52023781,680.08441893)(469.60024263,680.09442034)
\curveto(469.68023765,680.10441891)(469.76023757,680.12441889)(469.84024263,680.15442034)
\curveto(470.09023724,680.26441875)(470.29023704,680.40441861)(470.44024263,680.57442034)
\curveto(470.59023674,680.74441827)(470.72023661,680.95941805)(470.83024263,681.21942034)
\curveto(470.87023646,681.3094177)(470.90023643,681.39941761)(470.92024263,681.48942034)
\curveto(470.94023639,681.58941742)(470.96023637,681.69441732)(470.98024263,681.80442034)
\curveto(470.99023634,681.85441716)(470.99023634,681.89941711)(470.98024263,681.93942034)
\curveto(470.98023635,681.98941702)(470.99023634,682.03941697)(471.01024263,682.08942034)
\curveto(471.02023631,682.11941689)(471.0252363,682.15441686)(471.02524263,682.19442034)
\lineto(471.02524263,682.32942034)
\lineto(471.02524263,682.46442034)
}
}
{
\newrgbcolor{curcolor}{0 0 0}
\pscustom[linestyle=none,fillstyle=solid,fillcolor=curcolor]
{
\newpath
\moveto(482.01016451,682.61442034)
\curveto(482.03015634,682.53441648)(482.03015634,682.44441657)(482.01016451,682.34442034)
\curveto(481.99015638,682.24441677)(481.95515642,682.17941683)(481.90516451,682.14942034)
\curveto(481.85515652,682.1094169)(481.78015659,682.07941693)(481.68016451,682.05942034)
\curveto(481.59015678,682.04941696)(481.48515689,682.03941697)(481.36516451,682.02942034)
\lineto(481.02016451,682.02942034)
\curveto(480.91015746,682.03941697)(480.81015756,682.04441697)(480.72016451,682.04442034)
\lineto(477.06016451,682.04442034)
\lineto(476.85016451,682.04442034)
\curveto(476.79016158,682.04441697)(476.73516164,682.03441698)(476.68516451,682.01442034)
\curveto(476.60516177,681.97441704)(476.55516182,681.93441708)(476.53516451,681.89442034)
\curveto(476.51516186,681.87441714)(476.49516188,681.83441718)(476.47516451,681.77442034)
\curveto(476.45516192,681.72441729)(476.45016192,681.67441734)(476.46016451,681.62442034)
\curveto(476.48016189,681.56441745)(476.49016188,681.50441751)(476.49016451,681.44442034)
\curveto(476.50016187,681.39441762)(476.51516186,681.33941767)(476.53516451,681.27942034)
\curveto(476.61516176,681.03941797)(476.71016166,680.83941817)(476.82016451,680.67942034)
\curveto(476.94016143,680.52941848)(477.10016127,680.39441862)(477.30016451,680.27442034)
\curveto(477.38016099,680.22441879)(477.46016091,680.18941882)(477.54016451,680.16942034)
\curveto(477.63016074,680.15941885)(477.72016065,680.13941887)(477.81016451,680.10942034)
\curveto(477.89016048,680.08941892)(478.00016037,680.07441894)(478.14016451,680.06442034)
\curveto(478.28016009,680.05441896)(478.40015997,680.05941895)(478.50016451,680.07942034)
\lineto(478.63516451,680.07942034)
\curveto(478.73515964,680.09941891)(478.82515955,680.11941889)(478.90516451,680.13942034)
\curveto(478.99515938,680.16941884)(479.08015929,680.19941881)(479.16016451,680.22942034)
\curveto(479.26015911,680.27941873)(479.370159,680.34441867)(479.49016451,680.42442034)
\curveto(479.62015875,680.50441851)(479.71515866,680.58441843)(479.77516451,680.66442034)
\curveto(479.82515855,680.73441828)(479.8751585,680.79941821)(479.92516451,680.85942034)
\curveto(479.98515839,680.92941808)(480.05515832,680.97941803)(480.13516451,681.00942034)
\curveto(480.23515814,681.05941795)(480.36015801,681.07941793)(480.51016451,681.06942034)
\lineto(480.94516451,681.06942034)
\lineto(481.12516451,681.06942034)
\curveto(481.19515718,681.07941793)(481.25515712,681.07441794)(481.30516451,681.05442034)
\lineto(481.45516451,681.05442034)
\curveto(481.55515682,681.03441798)(481.62515675,681.009418)(481.66516451,680.97942034)
\curveto(481.70515667,680.95941805)(481.72515665,680.9144181)(481.72516451,680.84442034)
\curveto(481.73515664,680.77441824)(481.73015664,680.7144183)(481.71016451,680.66442034)
\curveto(481.66015671,680.52441849)(481.60515677,680.39941861)(481.54516451,680.28942034)
\curveto(481.48515689,680.17941883)(481.41515696,680.06941894)(481.33516451,679.95942034)
\curveto(481.11515726,679.62941938)(480.86515751,679.36441965)(480.58516451,679.16442034)
\curveto(480.30515807,678.96442005)(479.95515842,678.79442022)(479.53516451,678.65442034)
\curveto(479.42515895,678.6144204)(479.31515906,678.58942042)(479.20516451,678.57942034)
\curveto(479.09515928,678.56942044)(478.98015939,678.54942046)(478.86016451,678.51942034)
\curveto(478.82015955,678.5094205)(478.7751596,678.5094205)(478.72516451,678.51942034)
\curveto(478.68515969,678.51942049)(478.64515973,678.5144205)(478.60516451,678.50442034)
\lineto(478.44016451,678.50442034)
\curveto(478.39015998,678.48442053)(478.33016004,678.47942053)(478.26016451,678.48942034)
\curveto(478.20016017,678.48942052)(478.14516023,678.49442052)(478.09516451,678.50442034)
\curveto(478.01516036,678.5144205)(477.94516043,678.5144205)(477.88516451,678.50442034)
\curveto(477.82516055,678.49442052)(477.76016061,678.49942051)(477.69016451,678.51942034)
\curveto(477.64016073,678.53942047)(477.58516079,678.54942046)(477.52516451,678.54942034)
\curveto(477.46516091,678.54942046)(477.41016096,678.55942045)(477.36016451,678.57942034)
\curveto(477.25016112,678.59942041)(477.14016123,678.62442039)(477.03016451,678.65442034)
\curveto(476.92016145,678.67442034)(476.82016155,678.7094203)(476.73016451,678.75942034)
\curveto(476.62016175,678.79942021)(476.51516186,678.83442018)(476.41516451,678.86442034)
\curveto(476.32516205,678.90442011)(476.24016213,678.94942006)(476.16016451,678.99942034)
\curveto(475.84016253,679.19941981)(475.55516282,679.42941958)(475.30516451,679.68942034)
\curveto(475.05516332,679.95941905)(474.85016352,680.26941874)(474.69016451,680.61942034)
\curveto(474.64016373,680.72941828)(474.60016377,680.83941817)(474.57016451,680.94942034)
\curveto(474.54016383,681.06941794)(474.50016387,681.18941782)(474.45016451,681.30942034)
\curveto(474.44016393,681.34941766)(474.43516394,681.38441763)(474.43516451,681.41442034)
\curveto(474.43516394,681.45441756)(474.43016394,681.49441752)(474.42016451,681.53442034)
\curveto(474.38016399,681.65441736)(474.35516402,681.78441723)(474.34516451,681.92442034)
\lineto(474.31516451,682.34442034)
\curveto(474.31516406,682.39441662)(474.31016406,682.44941656)(474.30016451,682.50942034)
\curveto(474.30016407,682.56941644)(474.30516407,682.62441639)(474.31516451,682.67442034)
\lineto(474.31516451,682.85442034)
\lineto(474.36016451,683.21442034)
\curveto(474.40016397,683.38441563)(474.43516394,683.54941546)(474.46516451,683.70942034)
\curveto(474.49516388,683.86941514)(474.54016383,684.01941499)(474.60016451,684.15942034)
\curveto(475.03016334,685.19941381)(475.76016261,685.93441308)(476.79016451,686.36442034)
\curveto(476.93016144,686.42441259)(477.0701613,686.46441255)(477.21016451,686.48442034)
\curveto(477.36016101,686.5144125)(477.51516086,686.54941246)(477.67516451,686.58942034)
\curveto(477.75516062,686.59941241)(477.83016054,686.60441241)(477.90016451,686.60442034)
\curveto(477.9701604,686.60441241)(478.04516033,686.6094124)(478.12516451,686.61942034)
\curveto(478.63515974,686.62941238)(479.0701593,686.56941244)(479.43016451,686.43942034)
\curveto(479.80015857,686.31941269)(480.13015824,686.15941285)(480.42016451,685.95942034)
\curveto(480.51015786,685.89941311)(480.60015777,685.82941318)(480.69016451,685.74942034)
\curveto(480.78015759,685.67941333)(480.86015751,685.60441341)(480.93016451,685.52442034)
\curveto(480.96015741,685.47441354)(481.00015737,685.43441358)(481.05016451,685.40442034)
\curveto(481.13015724,685.29441372)(481.20515717,685.17941383)(481.27516451,685.05942034)
\curveto(481.34515703,684.94941406)(481.42015695,684.83441418)(481.50016451,684.71442034)
\curveto(481.55015682,684.62441439)(481.59015678,684.52941448)(481.62016451,684.42942034)
\curveto(481.66015671,684.33941467)(481.70015667,684.23941477)(481.74016451,684.12942034)
\curveto(481.79015658,683.99941501)(481.83015654,683.86441515)(481.86016451,683.72442034)
\curveto(481.89015648,683.58441543)(481.92515645,683.44441557)(481.96516451,683.30442034)
\curveto(481.98515639,683.22441579)(481.99015638,683.13441588)(481.98016451,683.03442034)
\curveto(481.98015639,682.94441607)(481.99015638,682.85941615)(482.01016451,682.77942034)
\lineto(482.01016451,682.61442034)
\moveto(479.76016451,683.49942034)
\curveto(479.83015854,683.59941541)(479.83515854,683.71941529)(479.77516451,683.85942034)
\curveto(479.72515865,684.009415)(479.68515869,684.11941489)(479.65516451,684.18942034)
\curveto(479.51515886,684.45941455)(479.33015904,684.66441435)(479.10016451,684.80442034)
\curveto(478.8701595,684.95441406)(478.55015982,685.03441398)(478.14016451,685.04442034)
\curveto(478.11016026,685.02441399)(478.0751603,685.01941399)(478.03516451,685.02942034)
\curveto(477.99516038,685.03941397)(477.96016041,685.03941397)(477.93016451,685.02942034)
\curveto(477.88016049,685.009414)(477.82516055,684.99441402)(477.76516451,684.98442034)
\curveto(477.70516067,684.98441403)(477.65016072,684.97441404)(477.60016451,684.95442034)
\curveto(477.16016121,684.8144142)(476.83516154,684.53941447)(476.62516451,684.12942034)
\curveto(476.60516177,684.08941492)(476.58016179,684.03441498)(476.55016451,683.96442034)
\curveto(476.53016184,683.90441511)(476.51516186,683.83941517)(476.50516451,683.76942034)
\curveto(476.49516188,683.7094153)(476.49516188,683.64941536)(476.50516451,683.58942034)
\curveto(476.52516185,683.52941548)(476.56016181,683.47941553)(476.61016451,683.43942034)
\curveto(476.69016168,683.38941562)(476.80016157,683.36441565)(476.94016451,683.36442034)
\lineto(477.34516451,683.36442034)
\lineto(479.01016451,683.36442034)
\lineto(479.44516451,683.36442034)
\curveto(479.60515877,683.37441564)(479.71015866,683.41941559)(479.76016451,683.49942034)
}
}
{
\newrgbcolor{curcolor}{0 0 0}
\pscustom[linestyle=none,fillstyle=solid,fillcolor=curcolor]
{
}
}
{
\newrgbcolor{curcolor}{0 0 0}
\pscustom[linestyle=none,fillstyle=solid,fillcolor=curcolor]
{
\newpath
\moveto(491.84360201,686.60442034)
\curveto(491.95359669,686.60441241)(492.0485966,686.59441242)(492.12860201,686.57442034)
\curveto(492.21859643,686.55441246)(492.28859636,686.5094125)(492.33860201,686.43942034)
\curveto(492.39859625,686.35941265)(492.42859622,686.21941279)(492.42860201,686.01942034)
\lineto(492.42860201,685.50942034)
\lineto(492.42860201,685.13442034)
\curveto(492.43859621,684.99441402)(492.42359622,684.88441413)(492.38360201,684.80442034)
\curveto(492.3435963,684.73441428)(492.28359636,684.68941432)(492.20360201,684.66942034)
\curveto(492.13359651,684.64941436)(492.0485966,684.63941437)(491.94860201,684.63942034)
\curveto(491.85859679,684.63941437)(491.75859689,684.64441437)(491.64860201,684.65442034)
\curveto(491.5485971,684.66441435)(491.45359719,684.65941435)(491.36360201,684.63942034)
\curveto(491.29359735,684.61941439)(491.22359742,684.60441441)(491.15360201,684.59442034)
\curveto(491.08359756,684.59441442)(491.01859763,684.58441443)(490.95860201,684.56442034)
\curveto(490.79859785,684.5144145)(490.63859801,684.43941457)(490.47860201,684.33942034)
\curveto(490.31859833,684.24941476)(490.19359845,684.14441487)(490.10360201,684.02442034)
\curveto(490.05359859,683.94441507)(489.99859865,683.85941515)(489.93860201,683.76942034)
\curveto(489.88859876,683.68941532)(489.83859881,683.60441541)(489.78860201,683.51442034)
\curveto(489.75859889,683.43441558)(489.72859892,683.34941566)(489.69860201,683.25942034)
\lineto(489.63860201,683.01942034)
\curveto(489.61859903,682.94941606)(489.60859904,682.87441614)(489.60860201,682.79442034)
\curveto(489.60859904,682.72441629)(489.59859905,682.65441636)(489.57860201,682.58442034)
\curveto(489.56859908,682.54441647)(489.56359908,682.50441651)(489.56360201,682.46442034)
\curveto(489.57359907,682.43441658)(489.57359907,682.40441661)(489.56360201,682.37442034)
\lineto(489.56360201,682.13442034)
\curveto(489.5435991,682.06441695)(489.53859911,681.98441703)(489.54860201,681.89442034)
\curveto(489.55859909,681.8144172)(489.56359908,681.73441728)(489.56360201,681.65442034)
\lineto(489.56360201,680.69442034)
\lineto(489.56360201,679.41942034)
\curveto(489.56359908,679.28941972)(489.55859909,679.16941984)(489.54860201,679.05942034)
\curveto(489.53859911,678.94942006)(489.50859914,678.85942015)(489.45860201,678.78942034)
\curveto(489.43859921,678.75942025)(489.40359924,678.73442028)(489.35360201,678.71442034)
\curveto(489.31359933,678.70442031)(489.26859938,678.69442032)(489.21860201,678.68442034)
\lineto(489.14360201,678.68442034)
\curveto(489.09359955,678.67442034)(489.03859961,678.66942034)(488.97860201,678.66942034)
\lineto(488.81360201,678.66942034)
\lineto(488.16860201,678.66942034)
\curveto(488.10860054,678.67942033)(488.0436006,678.68442033)(487.97360201,678.68442034)
\lineto(487.77860201,678.68442034)
\curveto(487.72860092,678.70442031)(487.67860097,678.71942029)(487.62860201,678.72942034)
\curveto(487.57860107,678.74942026)(487.5436011,678.78442023)(487.52360201,678.83442034)
\curveto(487.48360116,678.88442013)(487.45860119,678.95442006)(487.44860201,679.04442034)
\lineto(487.44860201,679.34442034)
\lineto(487.44860201,680.36442034)
\lineto(487.44860201,684.59442034)
\lineto(487.44860201,685.70442034)
\lineto(487.44860201,685.98942034)
\curveto(487.4486012,686.08941292)(487.46860118,686.16941284)(487.50860201,686.22942034)
\curveto(487.55860109,686.3094127)(487.63360101,686.35941265)(487.73360201,686.37942034)
\curveto(487.83360081,686.39941261)(487.95360069,686.4094126)(488.09360201,686.40942034)
\lineto(488.85860201,686.40942034)
\curveto(488.97859967,686.4094126)(489.08359956,686.39941261)(489.17360201,686.37942034)
\curveto(489.26359938,686.36941264)(489.33359931,686.32441269)(489.38360201,686.24442034)
\curveto(489.41359923,686.19441282)(489.42859922,686.12441289)(489.42860201,686.03442034)
\lineto(489.45860201,685.76442034)
\curveto(489.46859918,685.68441333)(489.48359916,685.6094134)(489.50360201,685.53942034)
\curveto(489.53359911,685.46941354)(489.58359906,685.43441358)(489.65360201,685.43442034)
\curveto(489.67359897,685.45441356)(489.69359895,685.46441355)(489.71360201,685.46442034)
\curveto(489.73359891,685.46441355)(489.75359889,685.47441354)(489.77360201,685.49442034)
\curveto(489.83359881,685.54441347)(489.88359876,685.59941341)(489.92360201,685.65942034)
\curveto(489.97359867,685.72941328)(490.03359861,685.78941322)(490.10360201,685.83942034)
\curveto(490.1435985,685.86941314)(490.17859847,685.89941311)(490.20860201,685.92942034)
\curveto(490.23859841,685.96941304)(490.27359837,686.00441301)(490.31360201,686.03442034)
\lineto(490.58360201,686.21442034)
\curveto(490.68359796,686.27441274)(490.78359786,686.32941268)(490.88360201,686.37942034)
\curveto(490.98359766,686.41941259)(491.08359756,686.45441256)(491.18360201,686.48442034)
\lineto(491.51360201,686.57442034)
\curveto(491.5435971,686.58441243)(491.59859705,686.58441243)(491.67860201,686.57442034)
\curveto(491.76859688,686.57441244)(491.82359682,686.58441243)(491.84360201,686.60442034)
}
}
{
\newrgbcolor{curcolor}{0 0 0}
\pscustom[linestyle=none,fillstyle=solid,fillcolor=curcolor]
{
\newpath
\moveto(500.35000826,682.61442034)
\curveto(500.37000009,682.53441648)(500.37000009,682.44441657)(500.35000826,682.34442034)
\curveto(500.33000013,682.24441677)(500.29500017,682.17941683)(500.24500826,682.14942034)
\curveto(500.19500027,682.1094169)(500.12000034,682.07941693)(500.02000826,682.05942034)
\curveto(499.93000053,682.04941696)(499.82500064,682.03941697)(499.70500826,682.02942034)
\lineto(499.36000826,682.02942034)
\curveto(499.25000121,682.03941697)(499.15000131,682.04441697)(499.06000826,682.04442034)
\lineto(495.40000826,682.04442034)
\lineto(495.19000826,682.04442034)
\curveto(495.13000533,682.04441697)(495.07500539,682.03441698)(495.02500826,682.01442034)
\curveto(494.94500552,681.97441704)(494.89500557,681.93441708)(494.87500826,681.89442034)
\curveto(494.85500561,681.87441714)(494.83500563,681.83441718)(494.81500826,681.77442034)
\curveto(494.79500567,681.72441729)(494.79000567,681.67441734)(494.80000826,681.62442034)
\curveto(494.82000564,681.56441745)(494.83000563,681.50441751)(494.83000826,681.44442034)
\curveto(494.84000562,681.39441762)(494.85500561,681.33941767)(494.87500826,681.27942034)
\curveto(494.95500551,681.03941797)(495.05000541,680.83941817)(495.16000826,680.67942034)
\curveto(495.28000518,680.52941848)(495.44000502,680.39441862)(495.64000826,680.27442034)
\curveto(495.72000474,680.22441879)(495.80000466,680.18941882)(495.88000826,680.16942034)
\curveto(495.97000449,680.15941885)(496.0600044,680.13941887)(496.15000826,680.10942034)
\curveto(496.23000423,680.08941892)(496.34000412,680.07441894)(496.48000826,680.06442034)
\curveto(496.62000384,680.05441896)(496.74000372,680.05941895)(496.84000826,680.07942034)
\lineto(496.97500826,680.07942034)
\curveto(497.07500339,680.09941891)(497.1650033,680.11941889)(497.24500826,680.13942034)
\curveto(497.33500313,680.16941884)(497.42000304,680.19941881)(497.50000826,680.22942034)
\curveto(497.60000286,680.27941873)(497.71000275,680.34441867)(497.83000826,680.42442034)
\curveto(497.9600025,680.50441851)(498.05500241,680.58441843)(498.11500826,680.66442034)
\curveto(498.1650023,680.73441828)(498.21500225,680.79941821)(498.26500826,680.85942034)
\curveto(498.32500214,680.92941808)(498.39500207,680.97941803)(498.47500826,681.00942034)
\curveto(498.57500189,681.05941795)(498.70000176,681.07941793)(498.85000826,681.06942034)
\lineto(499.28500826,681.06942034)
\lineto(499.46500826,681.06942034)
\curveto(499.53500093,681.07941793)(499.59500087,681.07441794)(499.64500826,681.05442034)
\lineto(499.79500826,681.05442034)
\curveto(499.89500057,681.03441798)(499.9650005,681.009418)(500.00500826,680.97942034)
\curveto(500.04500042,680.95941805)(500.0650004,680.9144181)(500.06500826,680.84442034)
\curveto(500.07500039,680.77441824)(500.07000039,680.7144183)(500.05000826,680.66442034)
\curveto(500.00000046,680.52441849)(499.94500052,680.39941861)(499.88500826,680.28942034)
\curveto(499.82500064,680.17941883)(499.75500071,680.06941894)(499.67500826,679.95942034)
\curveto(499.45500101,679.62941938)(499.20500126,679.36441965)(498.92500826,679.16442034)
\curveto(498.64500182,678.96442005)(498.29500217,678.79442022)(497.87500826,678.65442034)
\curveto(497.7650027,678.6144204)(497.65500281,678.58942042)(497.54500826,678.57942034)
\curveto(497.43500303,678.56942044)(497.32000314,678.54942046)(497.20000826,678.51942034)
\curveto(497.1600033,678.5094205)(497.11500335,678.5094205)(497.06500826,678.51942034)
\curveto(497.02500344,678.51942049)(496.98500348,678.5144205)(496.94500826,678.50442034)
\lineto(496.78000826,678.50442034)
\curveto(496.73000373,678.48442053)(496.67000379,678.47942053)(496.60000826,678.48942034)
\curveto(496.54000392,678.48942052)(496.48500398,678.49442052)(496.43500826,678.50442034)
\curveto(496.35500411,678.5144205)(496.28500418,678.5144205)(496.22500826,678.50442034)
\curveto(496.1650043,678.49442052)(496.10000436,678.49942051)(496.03000826,678.51942034)
\curveto(495.98000448,678.53942047)(495.92500454,678.54942046)(495.86500826,678.54942034)
\curveto(495.80500466,678.54942046)(495.75000471,678.55942045)(495.70000826,678.57942034)
\curveto(495.59000487,678.59942041)(495.48000498,678.62442039)(495.37000826,678.65442034)
\curveto(495.2600052,678.67442034)(495.1600053,678.7094203)(495.07000826,678.75942034)
\curveto(494.9600055,678.79942021)(494.85500561,678.83442018)(494.75500826,678.86442034)
\curveto(494.6650058,678.90442011)(494.58000588,678.94942006)(494.50000826,678.99942034)
\curveto(494.18000628,679.19941981)(493.89500657,679.42941958)(493.64500826,679.68942034)
\curveto(493.39500707,679.95941905)(493.19000727,680.26941874)(493.03000826,680.61942034)
\curveto(492.98000748,680.72941828)(492.94000752,680.83941817)(492.91000826,680.94942034)
\curveto(492.88000758,681.06941794)(492.84000762,681.18941782)(492.79000826,681.30942034)
\curveto(492.78000768,681.34941766)(492.77500769,681.38441763)(492.77500826,681.41442034)
\curveto(492.77500769,681.45441756)(492.77000769,681.49441752)(492.76000826,681.53442034)
\curveto(492.72000774,681.65441736)(492.69500777,681.78441723)(492.68500826,681.92442034)
\lineto(492.65500826,682.34442034)
\curveto(492.65500781,682.39441662)(492.65000781,682.44941656)(492.64000826,682.50942034)
\curveto(492.64000782,682.56941644)(492.64500782,682.62441639)(492.65500826,682.67442034)
\lineto(492.65500826,682.85442034)
\lineto(492.70000826,683.21442034)
\curveto(492.74000772,683.38441563)(492.77500769,683.54941546)(492.80500826,683.70942034)
\curveto(492.83500763,683.86941514)(492.88000758,684.01941499)(492.94000826,684.15942034)
\curveto(493.37000709,685.19941381)(494.10000636,685.93441308)(495.13000826,686.36442034)
\curveto(495.27000519,686.42441259)(495.41000505,686.46441255)(495.55000826,686.48442034)
\curveto(495.70000476,686.5144125)(495.85500461,686.54941246)(496.01500826,686.58942034)
\curveto(496.09500437,686.59941241)(496.17000429,686.60441241)(496.24000826,686.60442034)
\curveto(496.31000415,686.60441241)(496.38500408,686.6094124)(496.46500826,686.61942034)
\curveto(496.97500349,686.62941238)(497.41000305,686.56941244)(497.77000826,686.43942034)
\curveto(498.14000232,686.31941269)(498.47000199,686.15941285)(498.76000826,685.95942034)
\curveto(498.85000161,685.89941311)(498.94000152,685.82941318)(499.03000826,685.74942034)
\curveto(499.12000134,685.67941333)(499.20000126,685.60441341)(499.27000826,685.52442034)
\curveto(499.30000116,685.47441354)(499.34000112,685.43441358)(499.39000826,685.40442034)
\curveto(499.47000099,685.29441372)(499.54500092,685.17941383)(499.61500826,685.05942034)
\curveto(499.68500078,684.94941406)(499.7600007,684.83441418)(499.84000826,684.71442034)
\curveto(499.89000057,684.62441439)(499.93000053,684.52941448)(499.96000826,684.42942034)
\curveto(500.00000046,684.33941467)(500.04000042,684.23941477)(500.08000826,684.12942034)
\curveto(500.13000033,683.99941501)(500.17000029,683.86441515)(500.20000826,683.72442034)
\curveto(500.23000023,683.58441543)(500.2650002,683.44441557)(500.30500826,683.30442034)
\curveto(500.32500014,683.22441579)(500.33000013,683.13441588)(500.32000826,683.03442034)
\curveto(500.32000014,682.94441607)(500.33000013,682.85941615)(500.35000826,682.77942034)
\lineto(500.35000826,682.61442034)
\moveto(498.10000826,683.49942034)
\curveto(498.17000229,683.59941541)(498.17500229,683.71941529)(498.11500826,683.85942034)
\curveto(498.0650024,684.009415)(498.02500244,684.11941489)(497.99500826,684.18942034)
\curveto(497.85500261,684.45941455)(497.67000279,684.66441435)(497.44000826,684.80442034)
\curveto(497.21000325,684.95441406)(496.89000357,685.03441398)(496.48000826,685.04442034)
\curveto(496.45000401,685.02441399)(496.41500405,685.01941399)(496.37500826,685.02942034)
\curveto(496.33500413,685.03941397)(496.30000416,685.03941397)(496.27000826,685.02942034)
\curveto(496.22000424,685.009414)(496.1650043,684.99441402)(496.10500826,684.98442034)
\curveto(496.04500442,684.98441403)(495.99000447,684.97441404)(495.94000826,684.95442034)
\curveto(495.50000496,684.8144142)(495.17500529,684.53941447)(494.96500826,684.12942034)
\curveto(494.94500552,684.08941492)(494.92000554,684.03441498)(494.89000826,683.96442034)
\curveto(494.87000559,683.90441511)(494.85500561,683.83941517)(494.84500826,683.76942034)
\curveto(494.83500563,683.7094153)(494.83500563,683.64941536)(494.84500826,683.58942034)
\curveto(494.8650056,683.52941548)(494.90000556,683.47941553)(494.95000826,683.43942034)
\curveto(495.03000543,683.38941562)(495.14000532,683.36441565)(495.28000826,683.36442034)
\lineto(495.68500826,683.36442034)
\lineto(497.35000826,683.36442034)
\lineto(497.78500826,683.36442034)
\curveto(497.94500252,683.37441564)(498.05000241,683.41941559)(498.10000826,683.49942034)
}
}
{
\newrgbcolor{curcolor}{0 0 0}
\pscustom[linestyle=none,fillstyle=solid,fillcolor=curcolor]
{
\newpath
\moveto(505.16828951,686.61942034)
\curveto(505.97828435,686.63941237)(506.65328367,686.51941249)(507.19328951,686.25942034)
\curveto(507.74328258,685.99941301)(508.17828215,685.62941338)(508.49828951,685.14942034)
\curveto(508.65828167,684.9094141)(508.77828155,684.63441438)(508.85828951,684.32442034)
\curveto(508.87828145,684.27441474)(508.89328143,684.2094148)(508.90328951,684.12942034)
\curveto(508.9232814,684.04941496)(508.9232814,683.97941503)(508.90328951,683.91942034)
\curveto(508.86328146,683.8094152)(508.79328153,683.74441527)(508.69328951,683.72442034)
\curveto(508.59328173,683.7144153)(508.47328185,683.7094153)(508.33328951,683.70942034)
\lineto(507.55328951,683.70942034)
\lineto(507.26828951,683.70942034)
\curveto(507.17828315,683.7094153)(507.10328322,683.72941528)(507.04328951,683.76942034)
\curveto(506.96328336,683.8094152)(506.90828342,683.86941514)(506.87828951,683.94942034)
\curveto(506.84828348,684.03941497)(506.80828352,684.12941488)(506.75828951,684.21942034)
\curveto(506.69828363,684.32941468)(506.63328369,684.42941458)(506.56328951,684.51942034)
\curveto(506.49328383,684.6094144)(506.41328391,684.68941432)(506.32328951,684.75942034)
\curveto(506.18328414,684.84941416)(506.0282843,684.91941409)(505.85828951,684.96942034)
\curveto(505.79828453,684.98941402)(505.73828459,684.99941401)(505.67828951,684.99942034)
\curveto(505.61828471,684.99941401)(505.56328476,685.009414)(505.51328951,685.02942034)
\lineto(505.36328951,685.02942034)
\curveto(505.16328516,685.02941398)(505.00328532,685.009414)(504.88328951,684.96942034)
\curveto(504.59328573,684.87941413)(504.35828597,684.73941427)(504.17828951,684.54942034)
\curveto(503.99828633,684.36941464)(503.85328647,684.14941486)(503.74328951,683.88942034)
\curveto(503.69328663,683.77941523)(503.65328667,683.65941535)(503.62328951,683.52942034)
\curveto(503.60328672,683.4094156)(503.57828675,683.27941573)(503.54828951,683.13942034)
\curveto(503.53828679,683.09941591)(503.53328679,683.05941595)(503.53328951,683.01942034)
\curveto(503.53328679,682.97941603)(503.5282868,682.93941607)(503.51828951,682.89942034)
\curveto(503.49828683,682.79941621)(503.48828684,682.65941635)(503.48828951,682.47942034)
\curveto(503.49828683,682.29941671)(503.51328681,682.15941685)(503.53328951,682.05942034)
\curveto(503.53328679,681.97941703)(503.53828679,681.92441709)(503.54828951,681.89442034)
\curveto(503.56828676,681.82441719)(503.57828675,681.75441726)(503.57828951,681.68442034)
\curveto(503.58828674,681.6144174)(503.60328672,681.54441747)(503.62328951,681.47442034)
\curveto(503.70328662,681.24441777)(503.79828653,681.03441798)(503.90828951,680.84442034)
\curveto(504.01828631,680.65441836)(504.15828617,680.49441852)(504.32828951,680.36442034)
\curveto(504.36828596,680.33441868)(504.4282859,680.29941871)(504.50828951,680.25942034)
\curveto(504.61828571,680.18941882)(504.7282856,680.14441887)(504.83828951,680.12442034)
\curveto(504.95828537,680.10441891)(505.10328522,680.08441893)(505.27328951,680.06442034)
\lineto(505.36328951,680.06442034)
\curveto(505.40328492,680.06441895)(505.43328489,680.06941894)(505.45328951,680.07942034)
\lineto(505.58828951,680.07942034)
\curveto(505.65828467,680.09941891)(505.7232846,680.1144189)(505.78328951,680.12442034)
\curveto(505.85328447,680.14441887)(505.91828441,680.16441885)(505.97828951,680.18442034)
\curveto(506.27828405,680.3144187)(506.50828382,680.50441851)(506.66828951,680.75442034)
\curveto(506.70828362,680.80441821)(506.74328358,680.85941815)(506.77328951,680.91942034)
\curveto(506.80328352,680.98941802)(506.8282835,681.04941796)(506.84828951,681.09942034)
\curveto(506.88828344,681.2094178)(506.9232834,681.30441771)(506.95328951,681.38442034)
\curveto(506.98328334,681.47441754)(507.05328327,681.54441747)(507.16328951,681.59442034)
\curveto(507.25328307,681.63441738)(507.39828293,681.64941736)(507.59828951,681.63942034)
\lineto(508.09328951,681.63942034)
\lineto(508.30328951,681.63942034)
\curveto(508.38328194,681.64941736)(508.44828188,681.64441737)(508.49828951,681.62442034)
\lineto(508.61828951,681.62442034)
\lineto(508.73828951,681.59442034)
\curveto(508.77828155,681.59441742)(508.80828152,681.58441743)(508.82828951,681.56442034)
\curveto(508.87828145,681.52441749)(508.90828142,681.46441755)(508.91828951,681.38442034)
\curveto(508.93828139,681.3144177)(508.93828139,681.23941777)(508.91828951,681.15942034)
\curveto(508.8282815,680.82941818)(508.71828161,680.53441848)(508.58828951,680.27442034)
\curveto(508.17828215,679.50441951)(507.5232828,678.96942004)(506.62328951,678.66942034)
\curveto(506.5232838,678.63942037)(506.41828391,678.61942039)(506.30828951,678.60942034)
\curveto(506.19828413,678.58942042)(506.08828424,678.56442045)(505.97828951,678.53442034)
\curveto(505.91828441,678.52442049)(505.85828447,678.51942049)(505.79828951,678.51942034)
\curveto(505.73828459,678.51942049)(505.67828465,678.5144205)(505.61828951,678.50442034)
\lineto(505.45328951,678.50442034)
\curveto(505.40328492,678.48442053)(505.328285,678.47942053)(505.22828951,678.48942034)
\curveto(505.1282852,678.48942052)(505.05328527,678.49442052)(505.00328951,678.50442034)
\curveto(504.9232854,678.52442049)(504.84828548,678.53442048)(504.77828951,678.53442034)
\curveto(504.71828561,678.52442049)(504.65328567,678.52942048)(504.58328951,678.54942034)
\lineto(504.43328951,678.57942034)
\curveto(504.38328594,678.57942043)(504.33328599,678.58442043)(504.28328951,678.59442034)
\curveto(504.17328615,678.62442039)(504.06828626,678.65442036)(503.96828951,678.68442034)
\curveto(503.86828646,678.7144203)(503.77328655,678.74942026)(503.68328951,678.78942034)
\curveto(503.21328711,678.98942002)(502.81828751,679.24441977)(502.49828951,679.55442034)
\curveto(502.17828815,679.87441914)(501.91828841,680.26941874)(501.71828951,680.73942034)
\curveto(501.66828866,680.82941818)(501.6282887,680.92441809)(501.59828951,681.02442034)
\lineto(501.50828951,681.35442034)
\curveto(501.49828883,681.39441762)(501.49328883,681.42941758)(501.49328951,681.45942034)
\curveto(501.49328883,681.49941751)(501.48328884,681.54441747)(501.46328951,681.59442034)
\curveto(501.44328888,681.66441735)(501.43328889,681.73441728)(501.43328951,681.80442034)
\curveto(501.43328889,681.88441713)(501.4232889,681.95941705)(501.40328951,682.02942034)
\lineto(501.40328951,682.28442034)
\curveto(501.38328894,682.33441668)(501.37328895,682.38941662)(501.37328951,682.44942034)
\curveto(501.37328895,682.51941649)(501.38328894,682.57941643)(501.40328951,682.62942034)
\curveto(501.41328891,682.67941633)(501.41328891,682.72441629)(501.40328951,682.76442034)
\curveto(501.39328893,682.80441621)(501.39328893,682.84441617)(501.40328951,682.88442034)
\curveto(501.4232889,682.95441606)(501.4282889,683.01941599)(501.41828951,683.07942034)
\curveto(501.41828891,683.13941587)(501.4282889,683.19941581)(501.44828951,683.25942034)
\curveto(501.49828883,683.43941557)(501.53828879,683.6094154)(501.56828951,683.76942034)
\curveto(501.59828873,683.93941507)(501.64328868,684.10441491)(501.70328951,684.26442034)
\curveto(501.9232884,684.77441424)(502.19828813,685.19941381)(502.52828951,685.53942034)
\curveto(502.86828746,685.87941313)(503.29828703,686.15441286)(503.81828951,686.36442034)
\curveto(503.95828637,686.42441259)(504.10328622,686.46441255)(504.25328951,686.48442034)
\curveto(504.40328592,686.5144125)(504.55828577,686.54941246)(504.71828951,686.58942034)
\curveto(504.79828553,686.59941241)(504.87328545,686.60441241)(504.94328951,686.60442034)
\curveto(505.01328531,686.60441241)(505.08828524,686.6094124)(505.16828951,686.61942034)
}
}
{
\newrgbcolor{curcolor}{0 0 0}
\pscustom[linestyle=none,fillstyle=solid,fillcolor=curcolor]
{
\newpath
\moveto(510.63157076,686.39442034)
\lineto(511.75657076,686.39442034)
\curveto(511.86656832,686.39441262)(511.96656822,686.38941262)(512.05657076,686.37942034)
\curveto(512.14656804,686.36941264)(512.21156798,686.33441268)(512.25157076,686.27442034)
\curveto(512.30156789,686.2144128)(512.33156786,686.12941288)(512.34157076,686.01942034)
\curveto(512.35156784,685.91941309)(512.35656783,685.8144132)(512.35657076,685.70442034)
\lineto(512.35657076,684.65442034)
\lineto(512.35657076,682.41942034)
\curveto(512.35656783,682.05941695)(512.37156782,681.71941729)(512.40157076,681.39942034)
\curveto(512.43156776,681.07941793)(512.52156767,680.8144182)(512.67157076,680.60442034)
\curveto(512.81156738,680.39441862)(513.03656715,680.24441877)(513.34657076,680.15442034)
\curveto(513.39656679,680.14441887)(513.43656675,680.13941887)(513.46657076,680.13942034)
\curveto(513.50656668,680.13941887)(513.55156664,680.13441888)(513.60157076,680.12442034)
\curveto(513.65156654,680.1144189)(513.70656648,680.1094189)(513.76657076,680.10942034)
\curveto(513.82656636,680.1094189)(513.87156632,680.1144189)(513.90157076,680.12442034)
\curveto(513.95156624,680.14441887)(513.9915662,680.14941886)(514.02157076,680.13942034)
\curveto(514.06156613,680.12941888)(514.10156609,680.13441888)(514.14157076,680.15442034)
\curveto(514.35156584,680.20441881)(514.51656567,680.26941874)(514.63657076,680.34942034)
\curveto(514.81656537,680.45941855)(514.95656523,680.59941841)(515.05657076,680.76942034)
\curveto(515.16656502,680.94941806)(515.24156495,681.14441787)(515.28157076,681.35442034)
\curveto(515.33156486,681.57441744)(515.36156483,681.8144172)(515.37157076,682.07442034)
\curveto(515.38156481,682.34441667)(515.3865648,682.62441639)(515.38657076,682.91442034)
\lineto(515.38657076,684.72942034)
\lineto(515.38657076,685.70442034)
\lineto(515.38657076,685.97442034)
\curveto(515.3865648,686.07441294)(515.40656478,686.15441286)(515.44657076,686.21442034)
\curveto(515.49656469,686.30441271)(515.57156462,686.35441266)(515.67157076,686.36442034)
\curveto(515.77156442,686.38441263)(515.8915643,686.39441262)(516.03157076,686.39442034)
\lineto(516.82657076,686.39442034)
\lineto(517.11157076,686.39442034)
\curveto(517.20156299,686.39441262)(517.27656291,686.37441264)(517.33657076,686.33442034)
\curveto(517.41656277,686.28441273)(517.46156273,686.2094128)(517.47157076,686.10942034)
\curveto(517.48156271,686.009413)(517.4865627,685.89441312)(517.48657076,685.76442034)
\lineto(517.48657076,684.62442034)
\lineto(517.48657076,680.40942034)
\lineto(517.48657076,679.34442034)
\lineto(517.48657076,679.04442034)
\curveto(517.4865627,678.94442007)(517.46656272,678.86942014)(517.42657076,678.81942034)
\curveto(517.37656281,678.73942027)(517.30156289,678.69442032)(517.20157076,678.68442034)
\curveto(517.10156309,678.67442034)(516.99656319,678.66942034)(516.88657076,678.66942034)
\lineto(516.07657076,678.66942034)
\curveto(515.96656422,678.66942034)(515.86656432,678.67442034)(515.77657076,678.68442034)
\curveto(515.69656449,678.69442032)(515.63156456,678.73442028)(515.58157076,678.80442034)
\curveto(515.56156463,678.83442018)(515.54156465,678.87942013)(515.52157076,678.93942034)
\curveto(515.51156468,678.99942001)(515.49656469,679.05941995)(515.47657076,679.11942034)
\curveto(515.46656472,679.17941983)(515.45156474,679.23441978)(515.43157076,679.28442034)
\curveto(515.41156478,679.33441968)(515.38156481,679.36441965)(515.34157076,679.37442034)
\curveto(515.32156487,679.39441962)(515.29656489,679.39941961)(515.26657076,679.38942034)
\curveto(515.23656495,679.37941963)(515.21156498,679.36941964)(515.19157076,679.35942034)
\curveto(515.12156507,679.31941969)(515.06156513,679.27441974)(515.01157076,679.22442034)
\curveto(514.96156523,679.17441984)(514.90656528,679.12941988)(514.84657076,679.08942034)
\curveto(514.80656538,679.05941995)(514.76656542,679.02441999)(514.72657076,678.98442034)
\curveto(514.69656549,678.95442006)(514.65656553,678.92442009)(514.60657076,678.89442034)
\curveto(514.37656581,678.75442026)(514.10656608,678.64442037)(513.79657076,678.56442034)
\curveto(513.72656646,678.54442047)(513.65656653,678.53442048)(513.58657076,678.53442034)
\curveto(513.51656667,678.52442049)(513.44156675,678.5094205)(513.36157076,678.48942034)
\curveto(513.32156687,678.47942053)(513.27656691,678.47942053)(513.22657076,678.48942034)
\curveto(513.186567,678.48942052)(513.14656704,678.48442053)(513.10657076,678.47442034)
\curveto(513.07656711,678.46442055)(513.01156718,678.46442055)(512.91157076,678.47442034)
\curveto(512.82156737,678.47442054)(512.76156743,678.47942053)(512.73157076,678.48942034)
\curveto(512.68156751,678.48942052)(512.63156756,678.49442052)(512.58157076,678.50442034)
\lineto(512.43157076,678.50442034)
\curveto(512.31156788,678.53442048)(512.19656799,678.55942045)(512.08657076,678.57942034)
\curveto(511.97656821,678.59942041)(511.86656832,678.62942038)(511.75657076,678.66942034)
\curveto(511.70656848,678.68942032)(511.66156853,678.70442031)(511.62157076,678.71442034)
\curveto(511.5915686,678.73442028)(511.55156864,678.75442026)(511.50157076,678.77442034)
\curveto(511.15156904,678.96442005)(510.87156932,679.22941978)(510.66157076,679.56942034)
\curveto(510.53156966,679.77941923)(510.43656975,680.02941898)(510.37657076,680.31942034)
\curveto(510.31656987,680.61941839)(510.27656991,680.93441808)(510.25657076,681.26442034)
\curveto(510.24656994,681.60441741)(510.24156995,681.94941706)(510.24157076,682.29942034)
\curveto(510.25156994,682.65941635)(510.25656993,683.014416)(510.25657076,683.36442034)
\lineto(510.25657076,685.40442034)
\curveto(510.25656993,685.53441348)(510.25156994,685.68441333)(510.24157076,685.85442034)
\curveto(510.24156995,686.03441298)(510.26656992,686.16441285)(510.31657076,686.24442034)
\curveto(510.34656984,686.29441272)(510.40656978,686.33941267)(510.49657076,686.37942034)
\curveto(510.55656963,686.37941263)(510.60156959,686.38441263)(510.63157076,686.39442034)
}
}
{
\newrgbcolor{curcolor}{0 0 0}
\pscustom[linestyle=none,fillstyle=solid,fillcolor=curcolor]
{
\newpath
\moveto(523.54282076,686.60442034)
\curveto(523.65281544,686.60441241)(523.74781535,686.59441242)(523.82782076,686.57442034)
\curveto(523.91781518,686.55441246)(523.98781511,686.5094125)(524.03782076,686.43942034)
\curveto(524.097815,686.35941265)(524.12781497,686.21941279)(524.12782076,686.01942034)
\lineto(524.12782076,685.50942034)
\lineto(524.12782076,685.13442034)
\curveto(524.13781496,684.99441402)(524.12281497,684.88441413)(524.08282076,684.80442034)
\curveto(524.04281505,684.73441428)(523.98281511,684.68941432)(523.90282076,684.66942034)
\curveto(523.83281526,684.64941436)(523.74781535,684.63941437)(523.64782076,684.63942034)
\curveto(523.55781554,684.63941437)(523.45781564,684.64441437)(523.34782076,684.65442034)
\curveto(523.24781585,684.66441435)(523.15281594,684.65941435)(523.06282076,684.63942034)
\curveto(522.9928161,684.61941439)(522.92281617,684.60441441)(522.85282076,684.59442034)
\curveto(522.78281631,684.59441442)(522.71781638,684.58441443)(522.65782076,684.56442034)
\curveto(522.4978166,684.5144145)(522.33781676,684.43941457)(522.17782076,684.33942034)
\curveto(522.01781708,684.24941476)(521.8928172,684.14441487)(521.80282076,684.02442034)
\curveto(521.75281734,683.94441507)(521.6978174,683.85941515)(521.63782076,683.76942034)
\curveto(521.58781751,683.68941532)(521.53781756,683.60441541)(521.48782076,683.51442034)
\curveto(521.45781764,683.43441558)(521.42781767,683.34941566)(521.39782076,683.25942034)
\lineto(521.33782076,683.01942034)
\curveto(521.31781778,682.94941606)(521.30781779,682.87441614)(521.30782076,682.79442034)
\curveto(521.30781779,682.72441629)(521.2978178,682.65441636)(521.27782076,682.58442034)
\curveto(521.26781783,682.54441647)(521.26281783,682.50441651)(521.26282076,682.46442034)
\curveto(521.27281782,682.43441658)(521.27281782,682.40441661)(521.26282076,682.37442034)
\lineto(521.26282076,682.13442034)
\curveto(521.24281785,682.06441695)(521.23781786,681.98441703)(521.24782076,681.89442034)
\curveto(521.25781784,681.8144172)(521.26281783,681.73441728)(521.26282076,681.65442034)
\lineto(521.26282076,680.69442034)
\lineto(521.26282076,679.41942034)
\curveto(521.26281783,679.28941972)(521.25781784,679.16941984)(521.24782076,679.05942034)
\curveto(521.23781786,678.94942006)(521.20781789,678.85942015)(521.15782076,678.78942034)
\curveto(521.13781796,678.75942025)(521.10281799,678.73442028)(521.05282076,678.71442034)
\curveto(521.01281808,678.70442031)(520.96781813,678.69442032)(520.91782076,678.68442034)
\lineto(520.84282076,678.68442034)
\curveto(520.7928183,678.67442034)(520.73781836,678.66942034)(520.67782076,678.66942034)
\lineto(520.51282076,678.66942034)
\lineto(519.86782076,678.66942034)
\curveto(519.80781929,678.67942033)(519.74281935,678.68442033)(519.67282076,678.68442034)
\lineto(519.47782076,678.68442034)
\curveto(519.42781967,678.70442031)(519.37781972,678.71942029)(519.32782076,678.72942034)
\curveto(519.27781982,678.74942026)(519.24281985,678.78442023)(519.22282076,678.83442034)
\curveto(519.18281991,678.88442013)(519.15781994,678.95442006)(519.14782076,679.04442034)
\lineto(519.14782076,679.34442034)
\lineto(519.14782076,680.36442034)
\lineto(519.14782076,684.59442034)
\lineto(519.14782076,685.70442034)
\lineto(519.14782076,685.98942034)
\curveto(519.14781995,686.08941292)(519.16781993,686.16941284)(519.20782076,686.22942034)
\curveto(519.25781984,686.3094127)(519.33281976,686.35941265)(519.43282076,686.37942034)
\curveto(519.53281956,686.39941261)(519.65281944,686.4094126)(519.79282076,686.40942034)
\lineto(520.55782076,686.40942034)
\curveto(520.67781842,686.4094126)(520.78281831,686.39941261)(520.87282076,686.37942034)
\curveto(520.96281813,686.36941264)(521.03281806,686.32441269)(521.08282076,686.24442034)
\curveto(521.11281798,686.19441282)(521.12781797,686.12441289)(521.12782076,686.03442034)
\lineto(521.15782076,685.76442034)
\curveto(521.16781793,685.68441333)(521.18281791,685.6094134)(521.20282076,685.53942034)
\curveto(521.23281786,685.46941354)(521.28281781,685.43441358)(521.35282076,685.43442034)
\curveto(521.37281772,685.45441356)(521.3928177,685.46441355)(521.41282076,685.46442034)
\curveto(521.43281766,685.46441355)(521.45281764,685.47441354)(521.47282076,685.49442034)
\curveto(521.53281756,685.54441347)(521.58281751,685.59941341)(521.62282076,685.65942034)
\curveto(521.67281742,685.72941328)(521.73281736,685.78941322)(521.80282076,685.83942034)
\curveto(521.84281725,685.86941314)(521.87781722,685.89941311)(521.90782076,685.92942034)
\curveto(521.93781716,685.96941304)(521.97281712,686.00441301)(522.01282076,686.03442034)
\lineto(522.28282076,686.21442034)
\curveto(522.38281671,686.27441274)(522.48281661,686.32941268)(522.58282076,686.37942034)
\curveto(522.68281641,686.41941259)(522.78281631,686.45441256)(522.88282076,686.48442034)
\lineto(523.21282076,686.57442034)
\curveto(523.24281585,686.58441243)(523.2978158,686.58441243)(523.37782076,686.57442034)
\curveto(523.46781563,686.57441244)(523.52281557,686.58441243)(523.54282076,686.60442034)
}
}
{
\newrgbcolor{curcolor}{0 0 0}
\pscustom[linestyle=none,fillstyle=solid,fillcolor=curcolor]
{
\newpath
\moveto(527.91789888,686.61942034)
\curveto(528.66789438,686.63941237)(529.31789373,686.55441246)(529.86789888,686.36442034)
\curveto(530.42789262,686.18441283)(530.8528922,685.86941314)(531.14289888,685.41942034)
\curveto(531.21289184,685.3094137)(531.27289178,685.19441382)(531.32289888,685.07442034)
\curveto(531.38289167,684.96441405)(531.43289162,684.83941417)(531.47289888,684.69942034)
\curveto(531.49289156,684.63941437)(531.50289155,684.57441444)(531.50289888,684.50442034)
\curveto(531.50289155,684.43441458)(531.49289156,684.37441464)(531.47289888,684.32442034)
\curveto(531.43289162,684.26441475)(531.37789167,684.22441479)(531.30789888,684.20442034)
\curveto(531.25789179,684.18441483)(531.19789185,684.17441484)(531.12789888,684.17442034)
\lineto(530.91789888,684.17442034)
\lineto(530.25789888,684.17442034)
\curveto(530.18789286,684.17441484)(530.11789293,684.16941484)(530.04789888,684.15942034)
\curveto(529.97789307,684.15941485)(529.91289314,684.16941484)(529.85289888,684.18942034)
\curveto(529.7528933,684.2094148)(529.67789337,684.24941476)(529.62789888,684.30942034)
\curveto(529.57789347,684.36941464)(529.53289352,684.42941458)(529.49289888,684.48942034)
\lineto(529.37289888,684.69942034)
\curveto(529.34289371,684.77941423)(529.29289376,684.84441417)(529.22289888,684.89442034)
\curveto(529.12289393,684.97441404)(529.02289403,685.03441398)(528.92289888,685.07442034)
\curveto(528.83289422,685.1144139)(528.71789433,685.14941386)(528.57789888,685.17942034)
\curveto(528.50789454,685.19941381)(528.40289465,685.2144138)(528.26289888,685.22442034)
\curveto(528.13289492,685.23441378)(528.03289502,685.22941378)(527.96289888,685.20942034)
\lineto(527.85789888,685.20942034)
\lineto(527.70789888,685.17942034)
\curveto(527.66789538,685.17941383)(527.62289543,685.17441384)(527.57289888,685.16442034)
\curveto(527.40289565,685.1144139)(527.26289579,685.04441397)(527.15289888,684.95442034)
\curveto(527.052896,684.87441414)(526.98289607,684.74941426)(526.94289888,684.57942034)
\curveto(526.92289613,684.5094145)(526.92289613,684.44441457)(526.94289888,684.38442034)
\curveto(526.96289609,684.32441469)(526.98289607,684.27441474)(527.00289888,684.23442034)
\curveto(527.07289598,684.1144149)(527.1528959,684.01941499)(527.24289888,683.94942034)
\curveto(527.34289571,683.87941513)(527.45789559,683.81941519)(527.58789888,683.76942034)
\curveto(527.77789527,683.68941532)(527.98289507,683.61941539)(528.20289888,683.55942034)
\lineto(528.89289888,683.40942034)
\curveto(529.13289392,683.36941564)(529.36289369,683.31941569)(529.58289888,683.25942034)
\curveto(529.81289324,683.2094158)(530.02789302,683.14441587)(530.22789888,683.06442034)
\curveto(530.31789273,683.02441599)(530.40289265,682.98941602)(530.48289888,682.95942034)
\curveto(530.57289248,682.93941607)(530.65789239,682.90441611)(530.73789888,682.85442034)
\curveto(530.92789212,682.73441628)(531.09789195,682.60441641)(531.24789888,682.46442034)
\curveto(531.40789164,682.32441669)(531.53289152,682.14941686)(531.62289888,681.93942034)
\curveto(531.6528914,681.86941714)(531.67789137,681.79941721)(531.69789888,681.72942034)
\curveto(531.71789133,681.65941735)(531.73789131,681.58441743)(531.75789888,681.50442034)
\curveto(531.76789128,681.44441757)(531.77289128,681.34941766)(531.77289888,681.21942034)
\curveto(531.78289127,681.09941791)(531.78289127,681.00441801)(531.77289888,680.93442034)
\lineto(531.77289888,680.85942034)
\curveto(531.7528913,680.79941821)(531.73789131,680.73941827)(531.72789888,680.67942034)
\curveto(531.72789132,680.62941838)(531.72289133,680.57941843)(531.71289888,680.52942034)
\curveto(531.64289141,680.22941878)(531.53289152,679.96441905)(531.38289888,679.73442034)
\curveto(531.22289183,679.49441952)(531.02789202,679.29941971)(530.79789888,679.14942034)
\curveto(530.56789248,678.99942001)(530.30789274,678.86942014)(530.01789888,678.75942034)
\curveto(529.90789314,678.7094203)(529.78789326,678.67442034)(529.65789888,678.65442034)
\curveto(529.53789351,678.63442038)(529.41789363,678.6094204)(529.29789888,678.57942034)
\curveto(529.20789384,678.55942045)(529.11289394,678.54942046)(529.01289888,678.54942034)
\curveto(528.92289413,678.53942047)(528.83289422,678.52442049)(528.74289888,678.50442034)
\lineto(528.47289888,678.50442034)
\curveto(528.41289464,678.48442053)(528.30789474,678.47442054)(528.15789888,678.47442034)
\curveto(528.01789503,678.47442054)(527.91789513,678.48442053)(527.85789888,678.50442034)
\curveto(527.82789522,678.50442051)(527.79289526,678.5094205)(527.75289888,678.51942034)
\lineto(527.64789888,678.51942034)
\curveto(527.52789552,678.53942047)(527.40789564,678.55442046)(527.28789888,678.56442034)
\curveto(527.16789588,678.57442044)(527.052896,678.59442042)(526.94289888,678.62442034)
\curveto(526.5528965,678.73442028)(526.20789684,678.85942015)(525.90789888,678.99942034)
\curveto(525.60789744,679.14941986)(525.3528977,679.36941964)(525.14289888,679.65942034)
\curveto(525.00289805,679.84941916)(524.88289817,680.06941894)(524.78289888,680.31942034)
\curveto(524.76289829,680.37941863)(524.74289831,680.45941855)(524.72289888,680.55942034)
\curveto(524.70289835,680.6094184)(524.68789836,680.67941833)(524.67789888,680.76942034)
\curveto(524.66789838,680.85941815)(524.67289838,680.93441808)(524.69289888,680.99442034)
\curveto(524.72289833,681.06441795)(524.77289828,681.1144179)(524.84289888,681.14442034)
\curveto(524.89289816,681.16441785)(524.9528981,681.17441784)(525.02289888,681.17442034)
\lineto(525.24789888,681.17442034)
\lineto(525.95289888,681.17442034)
\lineto(526.19289888,681.17442034)
\curveto(526.27289678,681.17441784)(526.34289671,681.16441785)(526.40289888,681.14442034)
\curveto(526.51289654,681.10441791)(526.58289647,681.03941797)(526.61289888,680.94942034)
\curveto(526.6528964,680.85941815)(526.69789635,680.76441825)(526.74789888,680.66442034)
\curveto(526.76789628,680.6144184)(526.80289625,680.54941846)(526.85289888,680.46942034)
\curveto(526.91289614,680.38941862)(526.96289609,680.33941867)(527.00289888,680.31942034)
\curveto(527.12289593,680.21941879)(527.23789581,680.13941887)(527.34789888,680.07942034)
\curveto(527.45789559,680.02941898)(527.59789545,679.97941903)(527.76789888,679.92942034)
\curveto(527.81789523,679.9094191)(527.86789518,679.89941911)(527.91789888,679.89942034)
\curveto(527.96789508,679.9094191)(528.01789503,679.9094191)(528.06789888,679.89942034)
\curveto(528.1478949,679.87941913)(528.23289482,679.86941914)(528.32289888,679.86942034)
\curveto(528.42289463,679.87941913)(528.50789454,679.89441912)(528.57789888,679.91442034)
\curveto(528.62789442,679.92441909)(528.67289438,679.92941908)(528.71289888,679.92942034)
\curveto(528.76289429,679.92941908)(528.81289424,679.93941907)(528.86289888,679.95942034)
\curveto(529.00289405,680.009419)(529.12789392,680.06941894)(529.23789888,680.13942034)
\curveto(529.35789369,680.2094188)(529.4528936,680.29941871)(529.52289888,680.40942034)
\curveto(529.57289348,680.48941852)(529.61289344,680.6144184)(529.64289888,680.78442034)
\curveto(529.66289339,680.85441816)(529.66289339,680.91941809)(529.64289888,680.97942034)
\curveto(529.62289343,681.03941797)(529.60289345,681.08941792)(529.58289888,681.12942034)
\curveto(529.51289354,681.26941774)(529.42289363,681.37441764)(529.31289888,681.44442034)
\curveto(529.21289384,681.5144175)(529.09289396,681.57941743)(528.95289888,681.63942034)
\curveto(528.76289429,681.71941729)(528.56289449,681.78441723)(528.35289888,681.83442034)
\curveto(528.14289491,681.88441713)(527.93289512,681.93941707)(527.72289888,681.99942034)
\curveto(527.64289541,682.01941699)(527.55789549,682.03441698)(527.46789888,682.04442034)
\curveto(527.38789566,682.05441696)(527.30789574,682.06941694)(527.22789888,682.08942034)
\curveto(526.90789614,682.17941683)(526.60289645,682.26441675)(526.31289888,682.34442034)
\curveto(526.02289703,682.43441658)(525.75789729,682.56441645)(525.51789888,682.73442034)
\curveto(525.23789781,682.93441608)(525.03289802,683.20441581)(524.90289888,683.54442034)
\curveto(524.88289817,683.6144154)(524.86289819,683.7094153)(524.84289888,683.82942034)
\curveto(524.82289823,683.89941511)(524.80789824,683.98441503)(524.79789888,684.08442034)
\curveto(524.78789826,684.18441483)(524.79289826,684.27441474)(524.81289888,684.35442034)
\curveto(524.83289822,684.40441461)(524.83789821,684.44441457)(524.82789888,684.47442034)
\curveto(524.81789823,684.5144145)(524.82289823,684.55941445)(524.84289888,684.60942034)
\curveto(524.86289819,684.71941429)(524.88289817,684.81941419)(524.90289888,684.90942034)
\curveto(524.93289812,685.009414)(524.96789808,685.10441391)(525.00789888,685.19442034)
\curveto(525.13789791,685.48441353)(525.31789773,685.71941329)(525.54789888,685.89942034)
\curveto(525.77789727,686.07941293)(526.03789701,686.22441279)(526.32789888,686.33442034)
\curveto(526.43789661,686.38441263)(526.5528965,686.41941259)(526.67289888,686.43942034)
\curveto(526.79289626,686.46941254)(526.91789613,686.49941251)(527.04789888,686.52942034)
\curveto(527.10789594,686.54941246)(527.16789588,686.55941245)(527.22789888,686.55942034)
\lineto(527.40789888,686.58942034)
\curveto(527.48789556,686.59941241)(527.57289548,686.60441241)(527.66289888,686.60442034)
\curveto(527.7528953,686.60441241)(527.83789521,686.6094124)(527.91789888,686.61942034)
}
}
{
\newrgbcolor{curcolor}{0 0 0}
\pscustom[linestyle=none,fillstyle=solid,fillcolor=curcolor]
{
\newpath
\moveto(540.77453951,682.85442034)
\curveto(540.79453094,682.79441622)(540.80453093,682.7094163)(540.80453951,682.59942034)
\curveto(540.80453093,682.48941652)(540.79453094,682.40441661)(540.77453951,682.34442034)
\lineto(540.77453951,682.19442034)
\curveto(540.75453098,682.1144169)(540.74453099,682.03441698)(540.74453951,681.95442034)
\curveto(540.75453098,681.87441714)(540.74953098,681.79441722)(540.72953951,681.71442034)
\curveto(540.70953102,681.64441737)(540.69453104,681.57941743)(540.68453951,681.51942034)
\curveto(540.67453106,681.45941755)(540.66453107,681.39441762)(540.65453951,681.32442034)
\curveto(540.61453112,681.2144178)(540.57953115,681.09941791)(540.54953951,680.97942034)
\curveto(540.51953121,680.86941814)(540.47953125,680.76441825)(540.42953951,680.66442034)
\curveto(540.21953151,680.18441883)(539.94453179,679.79441922)(539.60453951,679.49442034)
\curveto(539.26453247,679.19441982)(538.85453288,678.94442007)(538.37453951,678.74442034)
\curveto(538.25453348,678.69442032)(538.1295336,678.65942035)(537.99953951,678.63942034)
\curveto(537.87953385,678.6094204)(537.75453398,678.57942043)(537.62453951,678.54942034)
\curveto(537.57453416,678.52942048)(537.51953421,678.51942049)(537.45953951,678.51942034)
\curveto(537.39953433,678.51942049)(537.34453439,678.5144205)(537.29453951,678.50442034)
\lineto(537.18953951,678.50442034)
\curveto(537.15953457,678.49442052)(537.1295346,678.48942052)(537.09953951,678.48942034)
\curveto(537.04953468,678.47942053)(536.96953476,678.47442054)(536.85953951,678.47442034)
\curveto(536.74953498,678.46442055)(536.66453507,678.46942054)(536.60453951,678.48942034)
\lineto(536.45453951,678.48942034)
\curveto(536.40453533,678.49942051)(536.34953538,678.50442051)(536.28953951,678.50442034)
\curveto(536.23953549,678.49442052)(536.18953554,678.49942051)(536.13953951,678.51942034)
\curveto(536.09953563,678.52942048)(536.05953567,678.53442048)(536.01953951,678.53442034)
\curveto(535.98953574,678.53442048)(535.94953578,678.53942047)(535.89953951,678.54942034)
\curveto(535.79953593,678.57942043)(535.69953603,678.60442041)(535.59953951,678.62442034)
\curveto(535.49953623,678.64442037)(535.40453633,678.67442034)(535.31453951,678.71442034)
\curveto(535.19453654,678.75442026)(535.07953665,678.79442022)(534.96953951,678.83442034)
\curveto(534.86953686,678.87442014)(534.76453697,678.92442009)(534.65453951,678.98442034)
\curveto(534.30453743,679.19441982)(534.00453773,679.43941957)(533.75453951,679.71942034)
\curveto(533.50453823,679.99941901)(533.29453844,680.33441868)(533.12453951,680.72442034)
\curveto(533.07453866,680.8144182)(533.0345387,680.9094181)(533.00453951,681.00942034)
\curveto(532.98453875,681.1094179)(532.95953877,681.2144178)(532.92953951,681.32442034)
\curveto(532.90953882,681.37441764)(532.89953883,681.41941759)(532.89953951,681.45942034)
\curveto(532.89953883,681.49941751)(532.88953884,681.54441747)(532.86953951,681.59442034)
\curveto(532.84953888,681.67441734)(532.83953889,681.75441726)(532.83953951,681.83442034)
\curveto(532.83953889,681.92441709)(532.8295389,682.009417)(532.80953951,682.08942034)
\curveto(532.79953893,682.13941687)(532.79453894,682.18441683)(532.79453951,682.22442034)
\lineto(532.79453951,682.35942034)
\curveto(532.77453896,682.41941659)(532.76453897,682.50441651)(532.76453951,682.61442034)
\curveto(532.77453896,682.72441629)(532.78953894,682.8094162)(532.80953951,682.86942034)
\lineto(532.80953951,682.97442034)
\curveto(532.81953891,683.02441599)(532.81953891,683.07441594)(532.80953951,683.12442034)
\curveto(532.80953892,683.18441583)(532.81953891,683.23941577)(532.83953951,683.28942034)
\curveto(532.84953888,683.33941567)(532.85453888,683.38441563)(532.85453951,683.42442034)
\curveto(532.85453888,683.47441554)(532.86453887,683.52441549)(532.88453951,683.57442034)
\curveto(532.92453881,683.70441531)(532.95953877,683.82941518)(532.98953951,683.94942034)
\curveto(533.01953871,684.07941493)(533.05953867,684.20441481)(533.10953951,684.32442034)
\curveto(533.28953844,684.73441428)(533.50453823,685.07441394)(533.75453951,685.34442034)
\curveto(534.00453773,685.62441339)(534.30953742,685.87941313)(534.66953951,686.10942034)
\curveto(534.76953696,686.15941285)(534.87453686,686.20441281)(534.98453951,686.24442034)
\curveto(535.09453664,686.28441273)(535.20453653,686.32941268)(535.31453951,686.37942034)
\curveto(535.44453629,686.42941258)(535.57953615,686.46441255)(535.71953951,686.48442034)
\curveto(535.85953587,686.50441251)(536.00453573,686.53441248)(536.15453951,686.57442034)
\curveto(536.2345355,686.58441243)(536.30953542,686.58941242)(536.37953951,686.58942034)
\curveto(536.44953528,686.58941242)(536.51953521,686.59441242)(536.58953951,686.60442034)
\curveto(537.16953456,686.6144124)(537.66953406,686.55441246)(538.08953951,686.42442034)
\curveto(538.51953321,686.29441272)(538.89953283,686.1144129)(539.22953951,685.88442034)
\curveto(539.33953239,685.80441321)(539.44953228,685.7144133)(539.55953951,685.61442034)
\curveto(539.67953205,685.52441349)(539.77953195,685.42441359)(539.85953951,685.31442034)
\curveto(539.93953179,685.2144138)(540.00953172,685.1144139)(540.06953951,685.01442034)
\curveto(540.13953159,684.9144141)(540.20953152,684.8094142)(540.27953951,684.69942034)
\curveto(540.34953138,684.58941442)(540.40453133,684.46941454)(540.44453951,684.33942034)
\curveto(540.48453125,684.21941479)(540.5295312,684.08941492)(540.57953951,683.94942034)
\curveto(540.60953112,683.86941514)(540.6345311,683.78441523)(540.65453951,683.69442034)
\lineto(540.71453951,683.42442034)
\curveto(540.72453101,683.38441563)(540.729531,683.34441567)(540.72953951,683.30442034)
\curveto(540.729531,683.26441575)(540.734531,683.22441579)(540.74453951,683.18442034)
\curveto(540.76453097,683.13441588)(540.76953096,683.07941593)(540.75953951,683.01942034)
\curveto(540.74953098,682.95941605)(540.75453098,682.90441611)(540.77453951,682.85442034)
\moveto(538.67453951,682.31442034)
\curveto(538.68453305,682.36441665)(538.68953304,682.43441658)(538.68953951,682.52442034)
\curveto(538.68953304,682.62441639)(538.68453305,682.69941631)(538.67453951,682.74942034)
\lineto(538.67453951,682.86942034)
\curveto(538.65453308,682.91941609)(538.64453309,682.97441604)(538.64453951,683.03442034)
\curveto(538.64453309,683.09441592)(538.63953309,683.14941586)(538.62953951,683.19942034)
\curveto(538.6295331,683.23941577)(538.62453311,683.26941574)(538.61453951,683.28942034)
\lineto(538.55453951,683.52942034)
\curveto(538.54453319,683.61941539)(538.52453321,683.70441531)(538.49453951,683.78442034)
\curveto(538.38453335,684.04441497)(538.25453348,684.26441475)(538.10453951,684.44442034)
\curveto(537.95453378,684.63441438)(537.75453398,684.78441423)(537.50453951,684.89442034)
\curveto(537.44453429,684.9144141)(537.38453435,684.92941408)(537.32453951,684.93942034)
\curveto(537.26453447,684.95941405)(537.19953453,684.97941403)(537.12953951,684.99942034)
\curveto(537.04953468,685.01941399)(536.96453477,685.02441399)(536.87453951,685.01442034)
\lineto(536.60453951,685.01442034)
\curveto(536.57453516,684.99441402)(536.53953519,684.98441403)(536.49953951,684.98442034)
\curveto(536.45953527,684.99441402)(536.42453531,684.99441402)(536.39453951,684.98442034)
\lineto(536.18453951,684.92442034)
\curveto(536.12453561,684.9144141)(536.06953566,684.89441412)(536.01953951,684.86442034)
\curveto(535.76953596,684.75441426)(535.56453617,684.59441442)(535.40453951,684.38442034)
\curveto(535.25453648,684.18441483)(535.1345366,683.94941506)(535.04453951,683.67942034)
\curveto(535.01453672,683.57941543)(534.98953674,683.47441554)(534.96953951,683.36442034)
\curveto(534.95953677,683.25441576)(534.94453679,683.14441587)(534.92453951,683.03442034)
\curveto(534.91453682,682.98441603)(534.90953682,682.93441608)(534.90953951,682.88442034)
\lineto(534.90953951,682.73442034)
\curveto(534.88953684,682.66441635)(534.87953685,682.55941645)(534.87953951,682.41942034)
\curveto(534.88953684,682.27941673)(534.90453683,682.17441684)(534.92453951,682.10442034)
\lineto(534.92453951,681.96942034)
\curveto(534.94453679,681.88941712)(534.95953677,681.8094172)(534.96953951,681.72942034)
\curveto(534.97953675,681.65941735)(534.99453674,681.58441743)(535.01453951,681.50442034)
\curveto(535.11453662,681.20441781)(535.21953651,680.95941805)(535.32953951,680.76942034)
\curveto(535.44953628,680.58941842)(535.6345361,680.42441859)(535.88453951,680.27442034)
\curveto(535.95453578,680.22441879)(536.0295357,680.18441883)(536.10953951,680.15442034)
\curveto(536.19953553,680.12441889)(536.28953544,680.09941891)(536.37953951,680.07942034)
\curveto(536.41953531,680.06941894)(536.45453528,680.06441895)(536.48453951,680.06442034)
\curveto(536.51453522,680.07441894)(536.54953518,680.07441894)(536.58953951,680.06442034)
\lineto(536.70953951,680.03442034)
\curveto(536.75953497,680.03441898)(536.80453493,680.03941897)(536.84453951,680.04942034)
\lineto(536.96453951,680.04942034)
\curveto(537.04453469,680.06941894)(537.12453461,680.08441893)(537.20453951,680.09442034)
\curveto(537.28453445,680.10441891)(537.35953437,680.12441889)(537.42953951,680.15442034)
\curveto(537.68953404,680.25441876)(537.89953383,680.38941862)(538.05953951,680.55942034)
\curveto(538.21953351,680.72941828)(538.35453338,680.93941807)(538.46453951,681.18942034)
\curveto(538.50453323,681.28941772)(538.5345332,681.38941762)(538.55453951,681.48942034)
\curveto(538.57453316,681.58941742)(538.59953313,681.69441732)(538.62953951,681.80442034)
\curveto(538.63953309,681.84441717)(538.64453309,681.87941713)(538.64453951,681.90942034)
\curveto(538.64453309,681.94941706)(538.64953308,681.98941702)(538.65953951,682.02942034)
\lineto(538.65953951,682.16442034)
\curveto(538.65953307,682.2144168)(538.66453307,682.26441675)(538.67453951,682.31442034)
}
}
{
\newrgbcolor{curcolor}{0 0 0}
\pscustom[linestyle=none,fillstyle=solid,fillcolor=curcolor]
{
\newpath
\moveto(28.32111654,343.25226336)
\curveto(28.32110606,343.22225769)(28.32110606,343.18225773)(28.32111654,343.13226336)
\curveto(28.33110605,343.08225783)(28.33610604,343.02725789)(28.33611654,342.96726336)
\curveto(28.33610604,342.90725801)(28.33110605,342.85225806)(28.32111654,342.80226336)
\curveto(28.32110606,342.75225816)(28.32110606,342.7172582)(28.32111654,342.69726336)
\curveto(28.32110606,342.62725829)(28.31610606,342.55725836)(28.30611654,342.48726336)
\curveto(28.30610607,342.42725849)(28.30610607,342.36725855)(28.30611654,342.30726336)
\curveto(28.28610609,342.25725866)(28.2761061,342.20725871)(28.27611654,342.15726336)
\curveto(28.28610609,342.10725881)(28.28610609,342.05725886)(28.27611654,342.00726336)
\curveto(28.25610612,341.89725902)(28.24110614,341.78725913)(28.23111654,341.67726336)
\curveto(28.22110616,341.56725935)(28.20110618,341.45725946)(28.17111654,341.34726336)
\curveto(28.12110626,341.17725974)(28.0761063,341.0122599)(28.03611654,340.85226336)
\curveto(27.99610638,340.70226021)(27.94610643,340.55226036)(27.88611654,340.40226336)
\curveto(27.71610666,339.98226093)(27.50610687,339.60226131)(27.25611654,339.26226336)
\curveto(27.00610737,338.92226199)(26.70610767,338.63226228)(26.35611654,338.39226336)
\curveto(26.15610822,338.25226266)(25.94610843,338.13226278)(25.72611654,338.03226336)
\curveto(25.51610886,337.93226298)(25.28610909,337.84226307)(25.03611654,337.76226336)
\curveto(24.93610944,337.73226318)(24.83110955,337.70726321)(24.72111654,337.68726336)
\curveto(24.62110976,337.67726324)(24.51610986,337.65726326)(24.40611654,337.62726336)
\curveto(24.35611002,337.6172633)(24.30611007,337.6122633)(24.25611654,337.61226336)
\curveto(24.21611016,337.6122633)(24.17111021,337.60726331)(24.12111654,337.59726336)
\curveto(24.0811103,337.58726333)(24.04111034,337.58226333)(24.00111654,337.58226336)
\curveto(23.96111042,337.59226332)(23.91611046,337.59226332)(23.86611654,337.58226336)
\curveto(23.84611053,337.57226334)(23.81611056,337.56726335)(23.77611654,337.56726336)
\curveto(23.73611064,337.57726334)(23.70611067,337.57726334)(23.68611654,337.56726336)
\curveto(23.60611077,337.54726337)(23.50611087,337.54226337)(23.38611654,337.55226336)
\curveto(23.26611111,337.56226335)(23.16111122,337.56726335)(23.07111654,337.56726336)
\lineto(19.57611654,337.56726336)
\curveto(19.40611497,337.56726335)(19.26111512,337.57226334)(19.14111654,337.58226336)
\curveto(19.03111535,337.60226331)(18.95111543,337.67226324)(18.90111654,337.79226336)
\curveto(18.87111551,337.87226304)(18.85611552,337.99226292)(18.85611654,338.15226336)
\curveto(18.86611551,338.32226259)(18.87111551,338.46226245)(18.87111654,338.57226336)
\lineto(18.87111654,347.37726336)
\curveto(18.87111551,347.49725342)(18.86611551,347.62225329)(18.85611654,347.75226336)
\curveto(18.85611552,347.89225302)(18.8811155,348.00225291)(18.93111654,348.08226336)
\curveto(18.97111541,348.14225277)(19.04611533,348.19225272)(19.15611654,348.23226336)
\curveto(19.1761152,348.24225267)(19.19611518,348.24225267)(19.21611654,348.23226336)
\curveto(19.23611514,348.23225268)(19.25611512,348.23725268)(19.27611654,348.24726336)
\lineto(23.31111654,348.24726336)
\curveto(23.37111101,348.24725267)(23.43111095,348.24725267)(23.49111654,348.24726336)
\curveto(23.56111082,348.25725266)(23.62111076,348.25725266)(23.67111654,348.24726336)
\lineto(23.85111654,348.24726336)
\curveto(23.90111048,348.22725269)(23.95611042,348.2172527)(24.01611654,348.21726336)
\curveto(24.0761103,348.22725269)(24.13111025,348.22225269)(24.18111654,348.20226336)
\curveto(24.24111014,348.18225273)(24.29611008,348.17225274)(24.34611654,348.17226336)
\curveto(24.40610997,348.18225273)(24.46610991,348.17725274)(24.52611654,348.15726336)
\curveto(24.66610971,348.12725279)(24.80110958,348.09725282)(24.93111654,348.06726336)
\curveto(25.06110932,348.04725287)(25.18610919,348.0122529)(25.30611654,347.96226336)
\curveto(25.41610896,347.912253)(25.52610885,347.86725305)(25.63611654,347.82726336)
\curveto(25.74610863,347.78725313)(25.85110853,347.73725318)(25.95111654,347.67726336)
\curveto(26.20110818,347.5172534)(26.43110795,347.36225355)(26.64111654,347.21226336)
\lineto(26.73111654,347.12226336)
\curveto(26.83110755,347.04225387)(26.92110746,346.95225396)(27.00111654,346.85226336)
\lineto(27.13611654,346.73226336)
\curveto(27.18610719,346.65225426)(27.24110714,346.57225434)(27.30111654,346.49226336)
\curveto(27.37110701,346.42225449)(27.43110695,346.34725457)(27.48111654,346.26726336)
\curveto(27.61110677,346.05725486)(27.72610665,345.83225508)(27.82611654,345.59226336)
\curveto(27.92610645,345.36225555)(28.01610636,345.1172558)(28.09611654,344.85726336)
\curveto(28.14610623,344.72725619)(28.1761062,344.59225632)(28.18611654,344.45226336)
\curveto(28.20610617,344.3122566)(28.23110615,344.17225674)(28.26111654,344.03226336)
\curveto(28.26110612,343.98225693)(28.26110612,343.93725698)(28.26111654,343.89726336)
\curveto(28.27110611,343.86725705)(28.2761061,343.83225708)(28.27611654,343.79226336)
\curveto(28.29610608,343.73225718)(28.30110608,343.66725725)(28.29111654,343.59726336)
\curveto(28.29110609,343.52725739)(28.30110608,343.46725745)(28.32111654,343.41726336)
\lineto(28.32111654,343.25226336)
\moveto(25.98111654,342.53226336)
\curveto(26.00110838,342.58225833)(26.01110837,342.66225825)(26.01111654,342.77226336)
\curveto(26.01110837,342.88225803)(26.00110838,342.96225795)(25.98111654,343.01226336)
\lineto(25.98111654,343.29726336)
\curveto(25.96110842,343.38725753)(25.94610843,343.48225743)(25.93611654,343.58226336)
\curveto(25.93610844,343.68225723)(25.92610845,343.77225714)(25.90611654,343.85226336)
\curveto(25.88610849,343.90225701)(25.8761085,343.94725697)(25.87611654,343.98726336)
\curveto(25.88610849,344.03725688)(25.8811085,344.08725683)(25.86111654,344.13726336)
\curveto(25.81110857,344.29725662)(25.76110862,344.44725647)(25.71111654,344.58726336)
\curveto(25.67110871,344.73725618)(25.61110877,344.87725604)(25.53111654,345.00726336)
\curveto(25.381109,345.24725567)(25.20610917,345.45225546)(25.00611654,345.62226336)
\curveto(24.81610956,345.80225511)(24.5811098,345.95225496)(24.30111654,346.07226336)
\curveto(24.21111017,346.10225481)(24.12111026,346.12725479)(24.03111654,346.14726336)
\curveto(23.94111044,346.17725474)(23.85111053,346.20225471)(23.76111654,346.22226336)
\curveto(23.6811107,346.23225468)(23.60611077,346.23725468)(23.53611654,346.23726336)
\curveto(23.4761109,346.24725467)(23.40611097,346.26225465)(23.32611654,346.28226336)
\curveto(23.28611109,346.29225462)(23.24611113,346.29225462)(23.20611654,346.28226336)
\curveto(23.16611121,346.28225463)(23.13111125,346.28725463)(23.10111654,346.29726336)
\lineto(22.77111654,346.29726336)
\curveto(22.72111166,346.30725461)(22.66611171,346.30725461)(22.60611654,346.29726336)
\lineto(22.42611654,346.29726336)
\lineto(21.75111654,346.29726336)
\curveto(21.73111265,346.27725464)(21.69611268,346.27225464)(21.64611654,346.28226336)
\curveto(21.60611277,346.29225462)(21.57111281,346.29225462)(21.54111654,346.28226336)
\lineto(21.39111654,346.22226336)
\curveto(21.34111304,346.2122547)(21.30111308,346.18225473)(21.27111654,346.13226336)
\curveto(21.23111315,346.08225483)(21.21111317,346.0122549)(21.21111654,345.92226336)
\lineto(21.21111654,345.62226336)
\curveto(21.21111317,345.49225542)(21.20611317,345.35725556)(21.19611654,345.21726336)
\lineto(21.19611654,344.79726336)
\lineto(21.19611654,340.61226336)
\curveto(21.19611318,340.55226036)(21.19111319,340.48726043)(21.18111654,340.41726336)
\curveto(21.1811132,340.34726057)(21.19111319,340.28726063)(21.21111654,340.23726336)
\lineto(21.21111654,340.08726336)
\lineto(21.21111654,339.87726336)
\curveto(21.22111316,339.8172611)(21.23611314,339.76226115)(21.25611654,339.71226336)
\curveto(21.31611306,339.59226132)(21.43111295,339.52726139)(21.60111654,339.51726336)
\lineto(22.12611654,339.51726336)
\lineto(23.31111654,339.51726336)
\curveto(23.71111067,339.52726139)(24.05111033,339.58726133)(24.33111654,339.69726336)
\curveto(24.70110968,339.84726107)(24.99110939,340.04726087)(25.20111654,340.29726336)
\curveto(25.42110896,340.54726037)(25.60610877,340.85726006)(25.75611654,341.22726336)
\curveto(25.79610858,341.30725961)(25.82610855,341.39725952)(25.84611654,341.49726336)
\curveto(25.86610851,341.59725932)(25.89110849,341.69725922)(25.92111654,341.79726336)
\lineto(25.92111654,341.91726336)
\curveto(25.94110844,341.98725893)(25.95110843,342.06225885)(25.95111654,342.14226336)
\curveto(25.95110843,342.22225869)(25.96110842,342.30225861)(25.98111654,342.38226336)
\lineto(25.98111654,342.53226336)
}
}
{
\newrgbcolor{curcolor}{0 0 0}
\pscustom[linestyle=none,fillstyle=solid,fillcolor=curcolor]
{
\newpath
\moveto(31.81963217,348.14226336)
\curveto(31.88962922,348.06225285)(31.92462918,347.94225297)(31.92463217,347.78226336)
\lineto(31.92463217,347.31726336)
\lineto(31.92463217,346.91226336)
\curveto(31.92462918,346.77225414)(31.88962922,346.67725424)(31.81963217,346.62726336)
\curveto(31.75962935,346.57725434)(31.67962943,346.54725437)(31.57963217,346.53726336)
\curveto(31.48962962,346.52725439)(31.38962972,346.52225439)(31.27963217,346.52226336)
\lineto(30.43963217,346.52226336)
\curveto(30.32963078,346.52225439)(30.22963088,346.52725439)(30.13963217,346.53726336)
\curveto(30.05963105,346.54725437)(29.98963112,346.57725434)(29.92963217,346.62726336)
\curveto(29.88963122,346.65725426)(29.85963125,346.7122542)(29.83963217,346.79226336)
\curveto(29.82963128,346.88225403)(29.81963129,346.97725394)(29.80963217,347.07726336)
\lineto(29.80963217,347.40726336)
\curveto(29.81963129,347.5172534)(29.82463128,347.6122533)(29.82463217,347.69226336)
\lineto(29.82463217,347.90226336)
\curveto(29.83463127,347.97225294)(29.85463125,348.03225288)(29.88463217,348.08226336)
\curveto(29.9046312,348.12225279)(29.92963118,348.15225276)(29.95963217,348.17226336)
\lineto(30.07963217,348.23226336)
\curveto(30.09963101,348.23225268)(30.12463098,348.23225268)(30.15463217,348.23226336)
\curveto(30.18463092,348.24225267)(30.2096309,348.24725267)(30.22963217,348.24726336)
\lineto(31.32463217,348.24726336)
\curveto(31.42462968,348.24725267)(31.51962959,348.24225267)(31.60963217,348.23226336)
\curveto(31.69962941,348.22225269)(31.76962934,348.19225272)(31.81963217,348.14226336)
\moveto(31.92463217,338.37726336)
\curveto(31.92462918,338.17726274)(31.91962919,338.00726291)(31.90963217,337.86726336)
\curveto(31.89962921,337.72726319)(31.8096293,337.63226328)(31.63963217,337.58226336)
\curveto(31.57962953,337.56226335)(31.51462959,337.55226336)(31.44463217,337.55226336)
\curveto(31.37462973,337.56226335)(31.29962981,337.56726335)(31.21963217,337.56726336)
\lineto(30.37963217,337.56726336)
\curveto(30.28963082,337.56726335)(30.19963091,337.57226334)(30.10963217,337.58226336)
\curveto(30.02963108,337.59226332)(29.96963114,337.62226329)(29.92963217,337.67226336)
\curveto(29.86963124,337.74226317)(29.83463127,337.82726309)(29.82463217,337.92726336)
\lineto(29.82463217,338.27226336)
\lineto(29.82463217,344.60226336)
\lineto(29.82463217,344.90226336)
\curveto(29.82463128,345.00225591)(29.84463126,345.08225583)(29.88463217,345.14226336)
\curveto(29.94463116,345.2122557)(30.02963108,345.25725566)(30.13963217,345.27726336)
\curveto(30.15963095,345.28725563)(30.18463092,345.28725563)(30.21463217,345.27726336)
\curveto(30.25463085,345.27725564)(30.28463082,345.28225563)(30.30463217,345.29226336)
\lineto(31.05463217,345.29226336)
\lineto(31.24963217,345.29226336)
\curveto(31.32962978,345.30225561)(31.39462971,345.30225561)(31.44463217,345.29226336)
\lineto(31.56463217,345.29226336)
\curveto(31.62462948,345.27225564)(31.67962943,345.25725566)(31.72963217,345.24726336)
\curveto(31.77962933,345.23725568)(31.81962929,345.20725571)(31.84963217,345.15726336)
\curveto(31.88962922,345.10725581)(31.9096292,345.03725588)(31.90963217,344.94726336)
\curveto(31.91962919,344.85725606)(31.92462918,344.76225615)(31.92463217,344.66226336)
\lineto(31.92463217,338.37726336)
}
}
{
\newrgbcolor{curcolor}{0 0 0}
\pscustom[linestyle=none,fillstyle=solid,fillcolor=curcolor]
{
\newpath
\moveto(36.55681967,345.50226336)
\curveto(37.30681517,345.52225539)(37.95681452,345.43725548)(38.50681967,345.24726336)
\curveto(39.06681341,345.06725585)(39.49181298,344.75225616)(39.78181967,344.30226336)
\curveto(39.85181262,344.19225672)(39.91181256,344.07725684)(39.96181967,343.95726336)
\curveto(40.02181245,343.84725707)(40.0718124,343.72225719)(40.11181967,343.58226336)
\curveto(40.13181234,343.52225739)(40.14181233,343.45725746)(40.14181967,343.38726336)
\curveto(40.14181233,343.3172576)(40.13181234,343.25725766)(40.11181967,343.20726336)
\curveto(40.0718124,343.14725777)(40.01681246,343.10725781)(39.94681967,343.08726336)
\curveto(39.89681258,343.06725785)(39.83681264,343.05725786)(39.76681967,343.05726336)
\lineto(39.55681967,343.05726336)
\lineto(38.89681967,343.05726336)
\curveto(38.82681365,343.05725786)(38.75681372,343.05225786)(38.68681967,343.04226336)
\curveto(38.61681386,343.04225787)(38.55181392,343.05225786)(38.49181967,343.07226336)
\curveto(38.39181408,343.09225782)(38.31681416,343.13225778)(38.26681967,343.19226336)
\curveto(38.21681426,343.25225766)(38.1718143,343.3122576)(38.13181967,343.37226336)
\lineto(38.01181967,343.58226336)
\curveto(37.98181449,343.66225725)(37.93181454,343.72725719)(37.86181967,343.77726336)
\curveto(37.76181471,343.85725706)(37.66181481,343.917257)(37.56181967,343.95726336)
\curveto(37.471815,343.99725692)(37.35681512,344.03225688)(37.21681967,344.06226336)
\curveto(37.14681533,344.08225683)(37.04181543,344.09725682)(36.90181967,344.10726336)
\curveto(36.7718157,344.1172568)(36.6718158,344.1122568)(36.60181967,344.09226336)
\lineto(36.49681967,344.09226336)
\lineto(36.34681967,344.06226336)
\curveto(36.30681617,344.06225685)(36.26181621,344.05725686)(36.21181967,344.04726336)
\curveto(36.04181643,343.99725692)(35.90181657,343.92725699)(35.79181967,343.83726336)
\curveto(35.69181678,343.75725716)(35.62181685,343.63225728)(35.58181967,343.46226336)
\curveto(35.56181691,343.39225752)(35.56181691,343.32725759)(35.58181967,343.26726336)
\curveto(35.60181687,343.20725771)(35.62181685,343.15725776)(35.64181967,343.11726336)
\curveto(35.71181676,342.99725792)(35.79181668,342.90225801)(35.88181967,342.83226336)
\curveto(35.98181649,342.76225815)(36.09681638,342.70225821)(36.22681967,342.65226336)
\curveto(36.41681606,342.57225834)(36.62181585,342.50225841)(36.84181967,342.44226336)
\lineto(37.53181967,342.29226336)
\curveto(37.7718147,342.25225866)(38.00181447,342.20225871)(38.22181967,342.14226336)
\curveto(38.45181402,342.09225882)(38.66681381,342.02725889)(38.86681967,341.94726336)
\curveto(38.95681352,341.90725901)(39.04181343,341.87225904)(39.12181967,341.84226336)
\curveto(39.21181326,341.82225909)(39.29681318,341.78725913)(39.37681967,341.73726336)
\curveto(39.56681291,341.6172593)(39.73681274,341.48725943)(39.88681967,341.34726336)
\curveto(40.04681243,341.20725971)(40.1718123,341.03225988)(40.26181967,340.82226336)
\curveto(40.29181218,340.75226016)(40.31681216,340.68226023)(40.33681967,340.61226336)
\curveto(40.35681212,340.54226037)(40.3768121,340.46726045)(40.39681967,340.38726336)
\curveto(40.40681207,340.32726059)(40.41181206,340.23226068)(40.41181967,340.10226336)
\curveto(40.42181205,339.98226093)(40.42181205,339.88726103)(40.41181967,339.81726336)
\lineto(40.41181967,339.74226336)
\curveto(40.39181208,339.68226123)(40.3768121,339.62226129)(40.36681967,339.56226336)
\curveto(40.36681211,339.5122614)(40.36181211,339.46226145)(40.35181967,339.41226336)
\curveto(40.28181219,339.1122618)(40.1718123,338.84726207)(40.02181967,338.61726336)
\curveto(39.86181261,338.37726254)(39.66681281,338.18226273)(39.43681967,338.03226336)
\curveto(39.20681327,337.88226303)(38.94681353,337.75226316)(38.65681967,337.64226336)
\curveto(38.54681393,337.59226332)(38.42681405,337.55726336)(38.29681967,337.53726336)
\curveto(38.1768143,337.5172634)(38.05681442,337.49226342)(37.93681967,337.46226336)
\curveto(37.84681463,337.44226347)(37.75181472,337.43226348)(37.65181967,337.43226336)
\curveto(37.56181491,337.42226349)(37.471815,337.40726351)(37.38181967,337.38726336)
\lineto(37.11181967,337.38726336)
\curveto(37.05181542,337.36726355)(36.94681553,337.35726356)(36.79681967,337.35726336)
\curveto(36.65681582,337.35726356)(36.55681592,337.36726355)(36.49681967,337.38726336)
\curveto(36.46681601,337.38726353)(36.43181604,337.39226352)(36.39181967,337.40226336)
\lineto(36.28681967,337.40226336)
\curveto(36.16681631,337.42226349)(36.04681643,337.43726348)(35.92681967,337.44726336)
\curveto(35.80681667,337.45726346)(35.69181678,337.47726344)(35.58181967,337.50726336)
\curveto(35.19181728,337.6172633)(34.84681763,337.74226317)(34.54681967,337.88226336)
\curveto(34.24681823,338.03226288)(33.99181848,338.25226266)(33.78181967,338.54226336)
\curveto(33.64181883,338.73226218)(33.52181895,338.95226196)(33.42181967,339.20226336)
\curveto(33.40181907,339.26226165)(33.38181909,339.34226157)(33.36181967,339.44226336)
\curveto(33.34181913,339.49226142)(33.32681915,339.56226135)(33.31681967,339.65226336)
\curveto(33.30681917,339.74226117)(33.31181916,339.8172611)(33.33181967,339.87726336)
\curveto(33.36181911,339.94726097)(33.41181906,339.99726092)(33.48181967,340.02726336)
\curveto(33.53181894,340.04726087)(33.59181888,340.05726086)(33.66181967,340.05726336)
\lineto(33.88681967,340.05726336)
\lineto(34.59181967,340.05726336)
\lineto(34.83181967,340.05726336)
\curveto(34.91181756,340.05726086)(34.98181749,340.04726087)(35.04181967,340.02726336)
\curveto(35.15181732,339.98726093)(35.22181725,339.92226099)(35.25181967,339.83226336)
\curveto(35.29181718,339.74226117)(35.33681714,339.64726127)(35.38681967,339.54726336)
\curveto(35.40681707,339.49726142)(35.44181703,339.43226148)(35.49181967,339.35226336)
\curveto(35.55181692,339.27226164)(35.60181687,339.22226169)(35.64181967,339.20226336)
\curveto(35.76181671,339.10226181)(35.8768166,339.02226189)(35.98681967,338.96226336)
\curveto(36.09681638,338.912262)(36.23681624,338.86226205)(36.40681967,338.81226336)
\curveto(36.45681602,338.79226212)(36.50681597,338.78226213)(36.55681967,338.78226336)
\curveto(36.60681587,338.79226212)(36.65681582,338.79226212)(36.70681967,338.78226336)
\curveto(36.78681569,338.76226215)(36.8718156,338.75226216)(36.96181967,338.75226336)
\curveto(37.06181541,338.76226215)(37.14681533,338.77726214)(37.21681967,338.79726336)
\curveto(37.26681521,338.80726211)(37.31181516,338.8122621)(37.35181967,338.81226336)
\curveto(37.40181507,338.8122621)(37.45181502,338.82226209)(37.50181967,338.84226336)
\curveto(37.64181483,338.89226202)(37.76681471,338.95226196)(37.87681967,339.02226336)
\curveto(37.99681448,339.09226182)(38.09181438,339.18226173)(38.16181967,339.29226336)
\curveto(38.21181426,339.37226154)(38.25181422,339.49726142)(38.28181967,339.66726336)
\curveto(38.30181417,339.73726118)(38.30181417,339.80226111)(38.28181967,339.86226336)
\curveto(38.26181421,339.92226099)(38.24181423,339.97226094)(38.22181967,340.01226336)
\curveto(38.15181432,340.15226076)(38.06181441,340.25726066)(37.95181967,340.32726336)
\curveto(37.85181462,340.39726052)(37.73181474,340.46226045)(37.59181967,340.52226336)
\curveto(37.40181507,340.60226031)(37.20181527,340.66726025)(36.99181967,340.71726336)
\curveto(36.78181569,340.76726015)(36.5718159,340.82226009)(36.36181967,340.88226336)
\curveto(36.28181619,340.90226001)(36.19681628,340.91726)(36.10681967,340.92726336)
\curveto(36.02681645,340.93725998)(35.94681653,340.95225996)(35.86681967,340.97226336)
\curveto(35.54681693,341.06225985)(35.24181723,341.14725977)(34.95181967,341.22726336)
\curveto(34.66181781,341.3172596)(34.39681808,341.44725947)(34.15681967,341.61726336)
\curveto(33.8768186,341.8172591)(33.6718188,342.08725883)(33.54181967,342.42726336)
\curveto(33.52181895,342.49725842)(33.50181897,342.59225832)(33.48181967,342.71226336)
\curveto(33.46181901,342.78225813)(33.44681903,342.86725805)(33.43681967,342.96726336)
\curveto(33.42681905,343.06725785)(33.43181904,343.15725776)(33.45181967,343.23726336)
\curveto(33.471819,343.28725763)(33.476819,343.32725759)(33.46681967,343.35726336)
\curveto(33.45681902,343.39725752)(33.46181901,343.44225747)(33.48181967,343.49226336)
\curveto(33.50181897,343.60225731)(33.52181895,343.70225721)(33.54181967,343.79226336)
\curveto(33.5718189,343.89225702)(33.60681887,343.98725693)(33.64681967,344.07726336)
\curveto(33.7768187,344.36725655)(33.95681852,344.60225631)(34.18681967,344.78226336)
\curveto(34.41681806,344.96225595)(34.6768178,345.10725581)(34.96681967,345.21726336)
\curveto(35.0768174,345.26725565)(35.19181728,345.30225561)(35.31181967,345.32226336)
\curveto(35.43181704,345.35225556)(35.55681692,345.38225553)(35.68681967,345.41226336)
\curveto(35.74681673,345.43225548)(35.80681667,345.44225547)(35.86681967,345.44226336)
\lineto(36.04681967,345.47226336)
\curveto(36.12681635,345.48225543)(36.21181626,345.48725543)(36.30181967,345.48726336)
\curveto(36.39181608,345.48725543)(36.476816,345.49225542)(36.55681967,345.50226336)
}
}
{
\newrgbcolor{curcolor}{0 0 0}
\pscustom[linestyle=none,fillstyle=solid,fillcolor=curcolor]
{
\newpath
\moveto(42.69346029,347.60226336)
\lineto(43.69846029,347.60226336)
\curveto(43.84845731,347.60225331)(43.97845718,347.59225332)(44.08846029,347.57226336)
\curveto(44.20845695,347.56225335)(44.29345686,347.50225341)(44.34346029,347.39226336)
\curveto(44.36345679,347.34225357)(44.37345678,347.28225363)(44.37346029,347.21226336)
\lineto(44.37346029,347.00226336)
\lineto(44.37346029,346.32726336)
\curveto(44.37345678,346.27725464)(44.36845679,346.2172547)(44.35846029,346.14726336)
\curveto(44.3584568,346.08725483)(44.36345679,346.03225488)(44.37346029,345.98226336)
\lineto(44.37346029,345.81726336)
\curveto(44.37345678,345.73725518)(44.37845678,345.66225525)(44.38846029,345.59226336)
\curveto(44.39845676,345.53225538)(44.42345673,345.47725544)(44.46346029,345.42726336)
\curveto(44.53345662,345.33725558)(44.6584565,345.28725563)(44.83846029,345.27726336)
\lineto(45.37846029,345.27726336)
\lineto(45.55846029,345.27726336)
\curveto(45.61845554,345.27725564)(45.67345548,345.26725565)(45.72346029,345.24726336)
\curveto(45.83345532,345.19725572)(45.89345526,345.10725581)(45.90346029,344.97726336)
\curveto(45.92345523,344.84725607)(45.93345522,344.70225621)(45.93346029,344.54226336)
\lineto(45.93346029,344.33226336)
\curveto(45.94345521,344.26225665)(45.93845522,344.20225671)(45.91846029,344.15226336)
\curveto(45.86845529,343.99225692)(45.76345539,343.90725701)(45.60346029,343.89726336)
\curveto(45.44345571,343.88725703)(45.26345589,343.88225703)(45.06346029,343.88226336)
\lineto(44.92846029,343.88226336)
\curveto(44.88845627,343.89225702)(44.8534563,343.89225702)(44.82346029,343.88226336)
\curveto(44.78345637,343.87225704)(44.74845641,343.86725705)(44.71846029,343.86726336)
\curveto(44.68845647,343.87725704)(44.6584565,343.87225704)(44.62846029,343.85226336)
\curveto(44.54845661,343.83225708)(44.48845667,343.78725713)(44.44846029,343.71726336)
\curveto(44.41845674,343.65725726)(44.39345676,343.58225733)(44.37346029,343.49226336)
\curveto(44.36345679,343.44225747)(44.36345679,343.38725753)(44.37346029,343.32726336)
\curveto(44.38345677,343.26725765)(44.38345677,343.2122577)(44.37346029,343.16226336)
\lineto(44.37346029,342.23226336)
\lineto(44.37346029,340.47726336)
\curveto(44.37345678,340.22726069)(44.37845678,340.00726091)(44.38846029,339.81726336)
\curveto(44.40845675,339.63726128)(44.47345668,339.47726144)(44.58346029,339.33726336)
\curveto(44.63345652,339.27726164)(44.69845646,339.23226168)(44.77846029,339.20226336)
\lineto(45.04846029,339.14226336)
\curveto(45.07845608,339.13226178)(45.10845605,339.12726179)(45.13846029,339.12726336)
\curveto(45.17845598,339.13726178)(45.20845595,339.13726178)(45.22846029,339.12726336)
\lineto(45.39346029,339.12726336)
\curveto(45.50345565,339.12726179)(45.59845556,339.12226179)(45.67846029,339.11226336)
\curveto(45.7584554,339.10226181)(45.82345533,339.06226185)(45.87346029,338.99226336)
\curveto(45.91345524,338.93226198)(45.93345522,338.85226206)(45.93346029,338.75226336)
\lineto(45.93346029,338.46726336)
\curveto(45.93345522,338.25726266)(45.92845523,338.06226285)(45.91846029,337.88226336)
\curveto(45.91845524,337.7122632)(45.83845532,337.59726332)(45.67846029,337.53726336)
\curveto(45.62845553,337.5172634)(45.58345557,337.5122634)(45.54346029,337.52226336)
\curveto(45.50345565,337.52226339)(45.4584557,337.5122634)(45.40846029,337.49226336)
\lineto(45.25846029,337.49226336)
\curveto(45.23845592,337.49226342)(45.20845595,337.49726342)(45.16846029,337.50726336)
\curveto(45.12845603,337.50726341)(45.09345606,337.50226341)(45.06346029,337.49226336)
\curveto(45.01345614,337.48226343)(44.9584562,337.48226343)(44.89846029,337.49226336)
\lineto(44.74846029,337.49226336)
\lineto(44.59846029,337.49226336)
\curveto(44.54845661,337.48226343)(44.50345665,337.48226343)(44.46346029,337.49226336)
\lineto(44.29846029,337.49226336)
\curveto(44.24845691,337.50226341)(44.19345696,337.50726341)(44.13346029,337.50726336)
\curveto(44.07345708,337.50726341)(44.01845714,337.5122634)(43.96846029,337.52226336)
\curveto(43.89845726,337.53226338)(43.83345732,337.54226337)(43.77346029,337.55226336)
\lineto(43.59346029,337.58226336)
\curveto(43.48345767,337.6122633)(43.37845778,337.64726327)(43.27846029,337.68726336)
\curveto(43.17845798,337.72726319)(43.08345807,337.77226314)(42.99346029,337.82226336)
\lineto(42.90346029,337.88226336)
\curveto(42.87345828,337.912263)(42.83845832,337.94226297)(42.79846029,337.97226336)
\curveto(42.77845838,337.99226292)(42.7534584,338.0122629)(42.72346029,338.03226336)
\lineto(42.64846029,338.10726336)
\curveto(42.50845865,338.29726262)(42.40345875,338.50726241)(42.33346029,338.73726336)
\curveto(42.31345884,338.77726214)(42.30345885,338.8122621)(42.30346029,338.84226336)
\curveto(42.31345884,338.88226203)(42.31345884,338.92726199)(42.30346029,338.97726336)
\curveto(42.29345886,338.99726192)(42.28845887,339.02226189)(42.28846029,339.05226336)
\curveto(42.28845887,339.08226183)(42.28345887,339.10726181)(42.27346029,339.12726336)
\lineto(42.27346029,339.27726336)
\curveto(42.26345889,339.3172616)(42.2584589,339.36226155)(42.25846029,339.41226336)
\curveto(42.26845889,339.46226145)(42.27345888,339.5122614)(42.27346029,339.56226336)
\lineto(42.27346029,340.13226336)
\lineto(42.27346029,342.36726336)
\lineto(42.27346029,343.16226336)
\lineto(42.27346029,343.37226336)
\curveto(42.28345887,343.44225747)(42.27845888,343.50725741)(42.25846029,343.56726336)
\curveto(42.21845894,343.70725721)(42.14845901,343.79725712)(42.04846029,343.83726336)
\curveto(41.93845922,343.88725703)(41.79845936,343.90225701)(41.62846029,343.88226336)
\curveto(41.4584597,343.86225705)(41.31345984,343.87725704)(41.19346029,343.92726336)
\curveto(41.11346004,343.95725696)(41.06346009,344.00225691)(41.04346029,344.06226336)
\curveto(41.02346013,344.12225679)(41.00346015,344.19725672)(40.98346029,344.28726336)
\lineto(40.98346029,344.60226336)
\curveto(40.98346017,344.78225613)(40.99346016,344.92725599)(41.01346029,345.03726336)
\curveto(41.03346012,345.14725577)(41.11846004,345.22225569)(41.26846029,345.26226336)
\curveto(41.30845985,345.28225563)(41.34845981,345.28725563)(41.38846029,345.27726336)
\lineto(41.52346029,345.27726336)
\curveto(41.67345948,345.27725564)(41.81345934,345.28225563)(41.94346029,345.29226336)
\curveto(42.07345908,345.3122556)(42.16345899,345.37225554)(42.21346029,345.47226336)
\curveto(42.24345891,345.54225537)(42.2584589,345.62225529)(42.25846029,345.71226336)
\curveto(42.26845889,345.80225511)(42.27345888,345.89225502)(42.27346029,345.98226336)
\lineto(42.27346029,346.91226336)
\lineto(42.27346029,347.16726336)
\curveto(42.27345888,347.25725366)(42.28345887,347.33225358)(42.30346029,347.39226336)
\curveto(42.3534588,347.49225342)(42.42845873,347.55725336)(42.52846029,347.58726336)
\curveto(42.54845861,347.59725332)(42.57345858,347.59725332)(42.60346029,347.58726336)
\curveto(42.64345851,347.58725333)(42.67345848,347.59225332)(42.69346029,347.60226336)
}
}
{
\newrgbcolor{curcolor}{0 0 0}
\pscustom[linestyle=none,fillstyle=solid,fillcolor=curcolor]
{
\newpath
\moveto(51.34189779,345.48726336)
\curveto(51.45189248,345.48725543)(51.54689238,345.47725544)(51.62689779,345.45726336)
\curveto(51.71689221,345.43725548)(51.78689214,345.39225552)(51.83689779,345.32226336)
\curveto(51.89689203,345.24225567)(51.926892,345.10225581)(51.92689779,344.90226336)
\lineto(51.92689779,344.39226336)
\lineto(51.92689779,344.01726336)
\curveto(51.93689199,343.87725704)(51.92189201,343.76725715)(51.88189779,343.68726336)
\curveto(51.84189209,343.6172573)(51.78189215,343.57225734)(51.70189779,343.55226336)
\curveto(51.6318923,343.53225738)(51.54689238,343.52225739)(51.44689779,343.52226336)
\curveto(51.35689257,343.52225739)(51.25689267,343.52725739)(51.14689779,343.53726336)
\curveto(51.04689288,343.54725737)(50.95189298,343.54225737)(50.86189779,343.52226336)
\curveto(50.79189314,343.50225741)(50.72189321,343.48725743)(50.65189779,343.47726336)
\curveto(50.58189335,343.47725744)(50.51689341,343.46725745)(50.45689779,343.44726336)
\curveto(50.29689363,343.39725752)(50.13689379,343.32225759)(49.97689779,343.22226336)
\curveto(49.81689411,343.13225778)(49.69189424,343.02725789)(49.60189779,342.90726336)
\curveto(49.55189438,342.82725809)(49.49689443,342.74225817)(49.43689779,342.65226336)
\curveto(49.38689454,342.57225834)(49.33689459,342.48725843)(49.28689779,342.39726336)
\curveto(49.25689467,342.3172586)(49.2268947,342.23225868)(49.19689779,342.14226336)
\lineto(49.13689779,341.90226336)
\curveto(49.11689481,341.83225908)(49.10689482,341.75725916)(49.10689779,341.67726336)
\curveto(49.10689482,341.60725931)(49.09689483,341.53725938)(49.07689779,341.46726336)
\curveto(49.06689486,341.42725949)(49.06189487,341.38725953)(49.06189779,341.34726336)
\curveto(49.07189486,341.3172596)(49.07189486,341.28725963)(49.06189779,341.25726336)
\lineto(49.06189779,341.01726336)
\curveto(49.04189489,340.94725997)(49.03689489,340.86726005)(49.04689779,340.77726336)
\curveto(49.05689487,340.69726022)(49.06189487,340.6172603)(49.06189779,340.53726336)
\lineto(49.06189779,339.57726336)
\lineto(49.06189779,338.30226336)
\curveto(49.06189487,338.17226274)(49.05689487,338.05226286)(49.04689779,337.94226336)
\curveto(49.03689489,337.83226308)(49.00689492,337.74226317)(48.95689779,337.67226336)
\curveto(48.93689499,337.64226327)(48.90189503,337.6172633)(48.85189779,337.59726336)
\curveto(48.81189512,337.58726333)(48.76689516,337.57726334)(48.71689779,337.56726336)
\lineto(48.64189779,337.56726336)
\curveto(48.59189534,337.55726336)(48.53689539,337.55226336)(48.47689779,337.55226336)
\lineto(48.31189779,337.55226336)
\lineto(47.66689779,337.55226336)
\curveto(47.60689632,337.56226335)(47.54189639,337.56726335)(47.47189779,337.56726336)
\lineto(47.27689779,337.56726336)
\curveto(47.2268967,337.58726333)(47.17689675,337.60226331)(47.12689779,337.61226336)
\curveto(47.07689685,337.63226328)(47.04189689,337.66726325)(47.02189779,337.71726336)
\curveto(46.98189695,337.76726315)(46.95689697,337.83726308)(46.94689779,337.92726336)
\lineto(46.94689779,338.22726336)
\lineto(46.94689779,339.24726336)
\lineto(46.94689779,343.47726336)
\lineto(46.94689779,344.58726336)
\lineto(46.94689779,344.87226336)
\curveto(46.94689698,344.97225594)(46.96689696,345.05225586)(47.00689779,345.11226336)
\curveto(47.05689687,345.19225572)(47.1318968,345.24225567)(47.23189779,345.26226336)
\curveto(47.3318966,345.28225563)(47.45189648,345.29225562)(47.59189779,345.29226336)
\lineto(48.35689779,345.29226336)
\curveto(48.47689545,345.29225562)(48.58189535,345.28225563)(48.67189779,345.26226336)
\curveto(48.76189517,345.25225566)(48.8318951,345.20725571)(48.88189779,345.12726336)
\curveto(48.91189502,345.07725584)(48.926895,345.00725591)(48.92689779,344.91726336)
\lineto(48.95689779,344.64726336)
\curveto(48.96689496,344.56725635)(48.98189495,344.49225642)(49.00189779,344.42226336)
\curveto(49.0318949,344.35225656)(49.08189485,344.3172566)(49.15189779,344.31726336)
\curveto(49.17189476,344.33725658)(49.19189474,344.34725657)(49.21189779,344.34726336)
\curveto(49.2318947,344.34725657)(49.25189468,344.35725656)(49.27189779,344.37726336)
\curveto(49.3318946,344.42725649)(49.38189455,344.48225643)(49.42189779,344.54226336)
\curveto(49.47189446,344.6122563)(49.5318944,344.67225624)(49.60189779,344.72226336)
\curveto(49.64189429,344.75225616)(49.67689425,344.78225613)(49.70689779,344.81226336)
\curveto(49.73689419,344.85225606)(49.77189416,344.88725603)(49.81189779,344.91726336)
\lineto(50.08189779,345.09726336)
\curveto(50.18189375,345.15725576)(50.28189365,345.2122557)(50.38189779,345.26226336)
\curveto(50.48189345,345.30225561)(50.58189335,345.33725558)(50.68189779,345.36726336)
\lineto(51.01189779,345.45726336)
\curveto(51.04189289,345.46725545)(51.09689283,345.46725545)(51.17689779,345.45726336)
\curveto(51.26689266,345.45725546)(51.32189261,345.46725545)(51.34189779,345.48726336)
}
}
{
\newrgbcolor{curcolor}{0 0 0}
\pscustom[linestyle=none,fillstyle=solid,fillcolor=curcolor]
{
\newpath
\moveto(54.84697592,348.14226336)
\curveto(54.91697297,348.06225285)(54.95197293,347.94225297)(54.95197592,347.78226336)
\lineto(54.95197592,347.31726336)
\lineto(54.95197592,346.91226336)
\curveto(54.95197293,346.77225414)(54.91697297,346.67725424)(54.84697592,346.62726336)
\curveto(54.7869731,346.57725434)(54.70697318,346.54725437)(54.60697592,346.53726336)
\curveto(54.51697337,346.52725439)(54.41697347,346.52225439)(54.30697592,346.52226336)
\lineto(53.46697592,346.52226336)
\curveto(53.35697453,346.52225439)(53.25697463,346.52725439)(53.16697592,346.53726336)
\curveto(53.0869748,346.54725437)(53.01697487,346.57725434)(52.95697592,346.62726336)
\curveto(52.91697497,346.65725426)(52.886975,346.7122542)(52.86697592,346.79226336)
\curveto(52.85697503,346.88225403)(52.84697504,346.97725394)(52.83697592,347.07726336)
\lineto(52.83697592,347.40726336)
\curveto(52.84697504,347.5172534)(52.85197503,347.6122533)(52.85197592,347.69226336)
\lineto(52.85197592,347.90226336)
\curveto(52.86197502,347.97225294)(52.881975,348.03225288)(52.91197592,348.08226336)
\curveto(52.93197495,348.12225279)(52.95697493,348.15225276)(52.98697592,348.17226336)
\lineto(53.10697592,348.23226336)
\curveto(53.12697476,348.23225268)(53.15197473,348.23225268)(53.18197592,348.23226336)
\curveto(53.21197467,348.24225267)(53.23697465,348.24725267)(53.25697592,348.24726336)
\lineto(54.35197592,348.24726336)
\curveto(54.45197343,348.24725267)(54.54697334,348.24225267)(54.63697592,348.23226336)
\curveto(54.72697316,348.22225269)(54.79697309,348.19225272)(54.84697592,348.14226336)
\moveto(54.95197592,338.37726336)
\curveto(54.95197293,338.17726274)(54.94697294,338.00726291)(54.93697592,337.86726336)
\curveto(54.92697296,337.72726319)(54.83697305,337.63226328)(54.66697592,337.58226336)
\curveto(54.60697328,337.56226335)(54.54197334,337.55226336)(54.47197592,337.55226336)
\curveto(54.40197348,337.56226335)(54.32697356,337.56726335)(54.24697592,337.56726336)
\lineto(53.40697592,337.56726336)
\curveto(53.31697457,337.56726335)(53.22697466,337.57226334)(53.13697592,337.58226336)
\curveto(53.05697483,337.59226332)(52.99697489,337.62226329)(52.95697592,337.67226336)
\curveto(52.89697499,337.74226317)(52.86197502,337.82726309)(52.85197592,337.92726336)
\lineto(52.85197592,338.27226336)
\lineto(52.85197592,344.60226336)
\lineto(52.85197592,344.90226336)
\curveto(52.85197503,345.00225591)(52.87197501,345.08225583)(52.91197592,345.14226336)
\curveto(52.97197491,345.2122557)(53.05697483,345.25725566)(53.16697592,345.27726336)
\curveto(53.1869747,345.28725563)(53.21197467,345.28725563)(53.24197592,345.27726336)
\curveto(53.2819746,345.27725564)(53.31197457,345.28225563)(53.33197592,345.29226336)
\lineto(54.08197592,345.29226336)
\lineto(54.27697592,345.29226336)
\curveto(54.35697353,345.30225561)(54.42197346,345.30225561)(54.47197592,345.29226336)
\lineto(54.59197592,345.29226336)
\curveto(54.65197323,345.27225564)(54.70697318,345.25725566)(54.75697592,345.24726336)
\curveto(54.80697308,345.23725568)(54.84697304,345.20725571)(54.87697592,345.15726336)
\curveto(54.91697297,345.10725581)(54.93697295,345.03725588)(54.93697592,344.94726336)
\curveto(54.94697294,344.85725606)(54.95197293,344.76225615)(54.95197592,344.66226336)
\lineto(54.95197592,338.37726336)
}
}
{
\newrgbcolor{curcolor}{0 0 0}
\pscustom[linestyle=none,fillstyle=solid,fillcolor=curcolor]
{
\newpath
\moveto(64.41416342,341.81226336)
\curveto(64.43415482,341.75225916)(64.44415481,341.64725927)(64.44416342,341.49726336)
\curveto(64.44415481,341.35725956)(64.43915481,341.25725966)(64.42916342,341.19726336)
\curveto(64.42915482,341.14725977)(64.42415483,341.10225981)(64.41416342,341.06226336)
\lineto(64.41416342,340.94226336)
\curveto(64.39415486,340.86226005)(64.38415487,340.78226013)(64.38416342,340.70226336)
\curveto(64.38415487,340.63226028)(64.37415488,340.55726036)(64.35416342,340.47726336)
\curveto(64.3541549,340.43726048)(64.34415491,340.36726055)(64.32416342,340.26726336)
\curveto(64.29415496,340.14726077)(64.26415499,340.02226089)(64.23416342,339.89226336)
\curveto(64.21415504,339.77226114)(64.17915507,339.65726126)(64.12916342,339.54726336)
\curveto(63.9491553,339.09726182)(63.72415553,338.70726221)(63.45416342,338.37726336)
\curveto(63.18415607,338.04726287)(62.82915642,337.78726313)(62.38916342,337.59726336)
\curveto(62.29915695,337.55726336)(62.20415705,337.52726339)(62.10416342,337.50726336)
\curveto(62.01415724,337.47726344)(61.91415734,337.44726347)(61.80416342,337.41726336)
\curveto(61.74415751,337.39726352)(61.67915757,337.38726353)(61.60916342,337.38726336)
\curveto(61.5491577,337.38726353)(61.48915776,337.38226353)(61.42916342,337.37226336)
\lineto(61.29416342,337.37226336)
\curveto(61.23415802,337.35226356)(61.1541581,337.34726357)(61.05416342,337.35726336)
\curveto(60.9541583,337.35726356)(60.87415838,337.36726355)(60.81416342,337.38726336)
\lineto(60.72416342,337.38726336)
\curveto(60.67415858,337.39726352)(60.61915863,337.40726351)(60.55916342,337.41726336)
\curveto(60.49915875,337.4172635)(60.43915881,337.42226349)(60.37916342,337.43226336)
\curveto(60.18915906,337.48226343)(60.01415924,337.53226338)(59.85416342,337.58226336)
\curveto(59.69415956,337.63226328)(59.54415971,337.70226321)(59.40416342,337.79226336)
\lineto(59.22416342,337.91226336)
\curveto(59.17416008,337.95226296)(59.12416013,337.99726292)(59.07416342,338.04726336)
\lineto(58.98416342,338.10726336)
\curveto(58.9541603,338.12726279)(58.92416033,338.14226277)(58.89416342,338.15226336)
\curveto(58.80416045,338.18226273)(58.7491605,338.16226275)(58.72916342,338.09226336)
\curveto(58.67916057,338.02226289)(58.64416061,337.93726298)(58.62416342,337.83726336)
\curveto(58.61416064,337.74726317)(58.57916067,337.67726324)(58.51916342,337.62726336)
\curveto(58.45916079,337.58726333)(58.38916086,337.56226335)(58.30916342,337.55226336)
\lineto(58.03916342,337.55226336)
\lineto(57.31916342,337.55226336)
\lineto(57.09416342,337.55226336)
\curveto(57.02416223,337.54226337)(56.95916229,337.54726337)(56.89916342,337.56726336)
\curveto(56.75916249,337.6172633)(56.67916257,337.70726321)(56.65916342,337.83726336)
\curveto(56.6491626,337.97726294)(56.64416261,338.13226278)(56.64416342,338.30226336)
\lineto(56.64416342,347.45226336)
\lineto(56.64416342,347.79726336)
\curveto(56.64416261,347.917253)(56.66916258,348.0122529)(56.71916342,348.08226336)
\curveto(56.75916249,348.15225276)(56.82916242,348.19725272)(56.92916342,348.21726336)
\curveto(56.9491623,348.22725269)(56.96916228,348.22725269)(56.98916342,348.21726336)
\curveto(57.01916223,348.2172527)(57.04416221,348.22225269)(57.06416342,348.23226336)
\lineto(58.00916342,348.23226336)
\curveto(58.18916106,348.23225268)(58.34416091,348.22225269)(58.47416342,348.20226336)
\curveto(58.60416065,348.19225272)(58.68916056,348.1172528)(58.72916342,347.97726336)
\curveto(58.75916049,347.87725304)(58.76916048,347.74225317)(58.75916342,347.57226336)
\curveto(58.7491605,347.4122535)(58.74416051,347.27225364)(58.74416342,347.15226336)
\lineto(58.74416342,345.51726336)
\lineto(58.74416342,345.18726336)
\curveto(58.74416051,345.07725584)(58.7541605,344.98225593)(58.77416342,344.90226336)
\curveto(58.78416047,344.85225606)(58.79416046,344.80725611)(58.80416342,344.76726336)
\curveto(58.81416044,344.73725618)(58.83916041,344.7172562)(58.87916342,344.70726336)
\curveto(58.89916035,344.68725623)(58.92416033,344.67725624)(58.95416342,344.67726336)
\curveto(58.99416026,344.67725624)(59.02416023,344.68225623)(59.04416342,344.69226336)
\curveto(59.11416014,344.73225618)(59.17916007,344.77225614)(59.23916342,344.81226336)
\curveto(59.29915995,344.86225605)(59.36415989,344.912256)(59.43416342,344.96226336)
\curveto(59.56415969,345.05225586)(59.69915955,345.12725579)(59.83916342,345.18726336)
\curveto(59.97915927,345.25725566)(60.13415912,345.3172556)(60.30416342,345.36726336)
\curveto(60.38415887,345.39725552)(60.46415879,345.4122555)(60.54416342,345.41226336)
\curveto(60.62415863,345.42225549)(60.70415855,345.43725548)(60.78416342,345.45726336)
\curveto(60.8541584,345.47725544)(60.92915832,345.48725543)(61.00916342,345.48726336)
\lineto(61.24916342,345.48726336)
\lineto(61.39916342,345.48726336)
\curveto(61.42915782,345.47725544)(61.46415779,345.47225544)(61.50416342,345.47226336)
\curveto(61.54415771,345.48225543)(61.58415767,345.48225543)(61.62416342,345.47226336)
\curveto(61.73415752,345.44225547)(61.83415742,345.4172555)(61.92416342,345.39726336)
\curveto(62.02415723,345.38725553)(62.11915713,345.36225555)(62.20916342,345.32226336)
\curveto(62.66915658,345.13225578)(63.04415621,344.88725603)(63.33416342,344.58726336)
\curveto(63.62415563,344.28725663)(63.86915538,343.912257)(64.06916342,343.46226336)
\curveto(64.11915513,343.34225757)(64.15915509,343.2172577)(64.18916342,343.08726336)
\curveto(64.22915502,342.95725796)(64.26915498,342.82225809)(64.30916342,342.68226336)
\curveto(64.32915492,342.6122583)(64.33915491,342.54225837)(64.33916342,342.47226336)
\curveto(64.3491549,342.4122585)(64.36415489,342.34225857)(64.38416342,342.26226336)
\curveto(64.40415485,342.2122587)(64.40915484,342.15725876)(64.39916342,342.09726336)
\curveto(64.39915485,342.03725888)(64.40415485,341.97725894)(64.41416342,341.91726336)
\lineto(64.41416342,341.81226336)
\moveto(62.19416342,340.40226336)
\curveto(62.22415703,340.50226041)(62.249157,340.62726029)(62.26916342,340.77726336)
\curveto(62.29915695,340.92725999)(62.31415694,341.07725984)(62.31416342,341.22726336)
\curveto(62.32415693,341.38725953)(62.32415693,341.54225937)(62.31416342,341.69226336)
\curveto(62.31415694,341.85225906)(62.29915695,341.98725893)(62.26916342,342.09726336)
\curveto(62.23915701,342.19725872)(62.21915703,342.29225862)(62.20916342,342.38226336)
\curveto(62.19915705,342.47225844)(62.17415708,342.55725836)(62.13416342,342.63726336)
\curveto(61.99415726,342.98725793)(61.79415746,343.28225763)(61.53416342,343.52226336)
\curveto(61.28415797,343.77225714)(60.91415834,343.89725702)(60.42416342,343.89726336)
\curveto(60.38415887,343.89725702)(60.3491589,343.89225702)(60.31916342,343.88226336)
\lineto(60.21416342,343.88226336)
\curveto(60.14415911,343.86225705)(60.07915917,343.84225707)(60.01916342,343.82226336)
\curveto(59.95915929,343.8122571)(59.89915935,343.79725712)(59.83916342,343.77726336)
\curveto(59.5491597,343.64725727)(59.32915992,343.46225745)(59.17916342,343.22226336)
\curveto(59.02916022,342.99225792)(58.90416035,342.72725819)(58.80416342,342.42726336)
\curveto(58.77416048,342.34725857)(58.7541605,342.26225865)(58.74416342,342.17226336)
\curveto(58.74416051,342.09225882)(58.73416052,342.0122589)(58.71416342,341.93226336)
\curveto(58.70416055,341.90225901)(58.69916055,341.85225906)(58.69916342,341.78226336)
\curveto(58.68916056,341.74225917)(58.68416057,341.70225921)(58.68416342,341.66226336)
\curveto(58.69416056,341.62225929)(58.69416056,341.58225933)(58.68416342,341.54226336)
\curveto(58.66416059,341.46225945)(58.65916059,341.35225956)(58.66916342,341.21226336)
\curveto(58.67916057,341.07225984)(58.69416056,340.97225994)(58.71416342,340.91226336)
\curveto(58.73416052,340.82226009)(58.74416051,340.73726018)(58.74416342,340.65726336)
\curveto(58.7541605,340.57726034)(58.77416048,340.49726042)(58.80416342,340.41726336)
\curveto(58.89416036,340.13726078)(58.99916025,339.89226102)(59.11916342,339.68226336)
\curveto(59.24916,339.48226143)(59.42915982,339.3122616)(59.65916342,339.17226336)
\curveto(59.81915943,339.07226184)(59.98415927,339.00226191)(60.15416342,338.96226336)
\curveto(60.17415908,338.96226195)(60.19415906,338.95726196)(60.21416342,338.94726336)
\lineto(60.30416342,338.94726336)
\curveto(60.33415892,338.93726198)(60.38415887,338.92726199)(60.45416342,338.91726336)
\curveto(60.52415873,338.917262)(60.58415867,338.92226199)(60.63416342,338.93226336)
\curveto(60.73415852,338.95226196)(60.82415843,338.96726195)(60.90416342,338.97726336)
\curveto(60.99415826,338.99726192)(61.07915817,339.02226189)(61.15916342,339.05226336)
\curveto(61.43915781,339.18226173)(61.6541576,339.36226155)(61.80416342,339.59226336)
\curveto(61.96415729,339.82226109)(62.09415716,340.09226082)(62.19416342,340.40226336)
}
}
{
\newrgbcolor{curcolor}{0 0 0}
\pscustom[linestyle=none,fillstyle=solid,fillcolor=curcolor]
{
\newpath
\moveto(66.20408529,345.27726336)
\lineto(67.32908529,345.27726336)
\curveto(67.43908286,345.27725564)(67.53908276,345.27225564)(67.62908529,345.26226336)
\curveto(67.71908258,345.25225566)(67.78408251,345.2172557)(67.82408529,345.15726336)
\curveto(67.87408242,345.09725582)(67.90408239,345.0122559)(67.91408529,344.90226336)
\curveto(67.92408237,344.80225611)(67.92908237,344.69725622)(67.92908529,344.58726336)
\lineto(67.92908529,343.53726336)
\lineto(67.92908529,341.30226336)
\curveto(67.92908237,340.94225997)(67.94408235,340.60226031)(67.97408529,340.28226336)
\curveto(68.00408229,339.96226095)(68.0940822,339.69726122)(68.24408529,339.48726336)
\curveto(68.38408191,339.27726164)(68.60908169,339.12726179)(68.91908529,339.03726336)
\curveto(68.96908133,339.02726189)(69.00908129,339.02226189)(69.03908529,339.02226336)
\curveto(69.07908122,339.02226189)(69.12408117,339.0172619)(69.17408529,339.00726336)
\curveto(69.22408107,338.99726192)(69.27908102,338.99226192)(69.33908529,338.99226336)
\curveto(69.3990809,338.99226192)(69.44408085,338.99726192)(69.47408529,339.00726336)
\curveto(69.52408077,339.02726189)(69.56408073,339.03226188)(69.59408529,339.02226336)
\curveto(69.63408066,339.0122619)(69.67408062,339.0172619)(69.71408529,339.03726336)
\curveto(69.92408037,339.08726183)(70.08908021,339.15226176)(70.20908529,339.23226336)
\curveto(70.38907991,339.34226157)(70.52907977,339.48226143)(70.62908529,339.65226336)
\curveto(70.73907956,339.83226108)(70.81407948,340.02726089)(70.85408529,340.23726336)
\curveto(70.90407939,340.45726046)(70.93407936,340.69726022)(70.94408529,340.95726336)
\curveto(70.95407934,341.22725969)(70.95907934,341.50725941)(70.95908529,341.79726336)
\lineto(70.95908529,343.61226336)
\lineto(70.95908529,344.58726336)
\lineto(70.95908529,344.85726336)
\curveto(70.95907934,344.95725596)(70.97907932,345.03725588)(71.01908529,345.09726336)
\curveto(71.06907923,345.18725573)(71.14407915,345.23725568)(71.24408529,345.24726336)
\curveto(71.34407895,345.26725565)(71.46407883,345.27725564)(71.60408529,345.27726336)
\lineto(72.39908529,345.27726336)
\lineto(72.68408529,345.27726336)
\curveto(72.77407752,345.27725564)(72.84907745,345.25725566)(72.90908529,345.21726336)
\curveto(72.98907731,345.16725575)(73.03407726,345.09225582)(73.04408529,344.99226336)
\curveto(73.05407724,344.89225602)(73.05907724,344.77725614)(73.05908529,344.64726336)
\lineto(73.05908529,343.50726336)
\lineto(73.05908529,339.29226336)
\lineto(73.05908529,338.22726336)
\lineto(73.05908529,337.92726336)
\curveto(73.05907724,337.82726309)(73.03907726,337.75226316)(72.99908529,337.70226336)
\curveto(72.94907735,337.62226329)(72.87407742,337.57726334)(72.77408529,337.56726336)
\curveto(72.67407762,337.55726336)(72.56907773,337.55226336)(72.45908529,337.55226336)
\lineto(71.64908529,337.55226336)
\curveto(71.53907876,337.55226336)(71.43907886,337.55726336)(71.34908529,337.56726336)
\curveto(71.26907903,337.57726334)(71.20407909,337.6172633)(71.15408529,337.68726336)
\curveto(71.13407916,337.7172632)(71.11407918,337.76226315)(71.09408529,337.82226336)
\curveto(71.08407921,337.88226303)(71.06907923,337.94226297)(71.04908529,338.00226336)
\curveto(71.03907926,338.06226285)(71.02407927,338.1172628)(71.00408529,338.16726336)
\curveto(70.98407931,338.2172627)(70.95407934,338.24726267)(70.91408529,338.25726336)
\curveto(70.8940794,338.27726264)(70.86907943,338.28226263)(70.83908529,338.27226336)
\curveto(70.80907949,338.26226265)(70.78407951,338.25226266)(70.76408529,338.24226336)
\curveto(70.6940796,338.20226271)(70.63407966,338.15726276)(70.58408529,338.10726336)
\curveto(70.53407976,338.05726286)(70.47907982,338.0122629)(70.41908529,337.97226336)
\curveto(70.37907992,337.94226297)(70.33907996,337.90726301)(70.29908529,337.86726336)
\curveto(70.26908003,337.83726308)(70.22908007,337.80726311)(70.17908529,337.77726336)
\curveto(69.94908035,337.63726328)(69.67908062,337.52726339)(69.36908529,337.44726336)
\curveto(69.299081,337.42726349)(69.22908107,337.4172635)(69.15908529,337.41726336)
\curveto(69.08908121,337.40726351)(69.01408128,337.39226352)(68.93408529,337.37226336)
\curveto(68.8940814,337.36226355)(68.84908145,337.36226355)(68.79908529,337.37226336)
\curveto(68.75908154,337.37226354)(68.71908158,337.36726355)(68.67908529,337.35726336)
\curveto(68.64908165,337.34726357)(68.58408171,337.34726357)(68.48408529,337.35726336)
\curveto(68.3940819,337.35726356)(68.33408196,337.36226355)(68.30408529,337.37226336)
\curveto(68.25408204,337.37226354)(68.20408209,337.37726354)(68.15408529,337.38726336)
\lineto(68.00408529,337.38726336)
\curveto(67.88408241,337.4172635)(67.76908253,337.44226347)(67.65908529,337.46226336)
\curveto(67.54908275,337.48226343)(67.43908286,337.5122634)(67.32908529,337.55226336)
\curveto(67.27908302,337.57226334)(67.23408306,337.58726333)(67.19408529,337.59726336)
\curveto(67.16408313,337.6172633)(67.12408317,337.63726328)(67.07408529,337.65726336)
\curveto(66.72408357,337.84726307)(66.44408385,338.1122628)(66.23408529,338.45226336)
\curveto(66.10408419,338.66226225)(66.00908429,338.912262)(65.94908529,339.20226336)
\curveto(65.88908441,339.50226141)(65.84908445,339.8172611)(65.82908529,340.14726336)
\curveto(65.81908448,340.48726043)(65.81408448,340.83226008)(65.81408529,341.18226336)
\curveto(65.82408447,341.54225937)(65.82908447,341.89725902)(65.82908529,342.24726336)
\lineto(65.82908529,344.28726336)
\curveto(65.82908447,344.4172565)(65.82408447,344.56725635)(65.81408529,344.73726336)
\curveto(65.81408448,344.917256)(65.83908446,345.04725587)(65.88908529,345.12726336)
\curveto(65.91908438,345.17725574)(65.97908432,345.22225569)(66.06908529,345.26226336)
\curveto(66.12908417,345.26225565)(66.17408412,345.26725565)(66.20408529,345.27726336)
}
}
{
\newrgbcolor{curcolor}{0 0 0}
\pscustom[linestyle=none,fillstyle=solid,fillcolor=curcolor]
{
\newpath
\moveto(78.26033529,345.50226336)
\curveto(79.07033013,345.52225539)(79.74532946,345.40225551)(80.28533529,345.14226336)
\curveto(80.83532837,344.88225603)(81.27032793,344.5122564)(81.59033529,344.03226336)
\curveto(81.75032745,343.79225712)(81.87032733,343.5172574)(81.95033529,343.20726336)
\curveto(81.97032723,343.15725776)(81.98532722,343.09225782)(81.99533529,343.01226336)
\curveto(82.01532719,342.93225798)(82.01532719,342.86225805)(81.99533529,342.80226336)
\curveto(81.95532725,342.69225822)(81.88532732,342.62725829)(81.78533529,342.60726336)
\curveto(81.68532752,342.59725832)(81.56532764,342.59225832)(81.42533529,342.59226336)
\lineto(80.64533529,342.59226336)
\lineto(80.36033529,342.59226336)
\curveto(80.27032893,342.59225832)(80.19532901,342.6122583)(80.13533529,342.65226336)
\curveto(80.05532915,342.69225822)(80.0003292,342.75225816)(79.97033529,342.83226336)
\curveto(79.94032926,342.92225799)(79.9003293,343.0122579)(79.85033529,343.10226336)
\curveto(79.79032941,343.2122577)(79.72532948,343.3122576)(79.65533529,343.40226336)
\curveto(79.58532962,343.49225742)(79.5053297,343.57225734)(79.41533529,343.64226336)
\curveto(79.27532993,343.73225718)(79.12033008,343.80225711)(78.95033529,343.85226336)
\curveto(78.89033031,343.87225704)(78.83033037,343.88225703)(78.77033529,343.88226336)
\curveto(78.71033049,343.88225703)(78.65533055,343.89225702)(78.60533529,343.91226336)
\lineto(78.45533529,343.91226336)
\curveto(78.25533095,343.912257)(78.09533111,343.89225702)(77.97533529,343.85226336)
\curveto(77.68533152,343.76225715)(77.45033175,343.62225729)(77.27033529,343.43226336)
\curveto(77.09033211,343.25225766)(76.94533226,343.03225788)(76.83533529,342.77226336)
\curveto(76.78533242,342.66225825)(76.74533246,342.54225837)(76.71533529,342.41226336)
\curveto(76.69533251,342.29225862)(76.67033253,342.16225875)(76.64033529,342.02226336)
\curveto(76.63033257,341.98225893)(76.62533258,341.94225897)(76.62533529,341.90226336)
\curveto(76.62533258,341.86225905)(76.62033258,341.82225909)(76.61033529,341.78226336)
\curveto(76.59033261,341.68225923)(76.58033262,341.54225937)(76.58033529,341.36226336)
\curveto(76.59033261,341.18225973)(76.6053326,341.04225987)(76.62533529,340.94226336)
\curveto(76.62533258,340.86226005)(76.63033257,340.80726011)(76.64033529,340.77726336)
\curveto(76.66033254,340.70726021)(76.67033253,340.63726028)(76.67033529,340.56726336)
\curveto(76.68033252,340.49726042)(76.69533251,340.42726049)(76.71533529,340.35726336)
\curveto(76.79533241,340.12726079)(76.89033231,339.917261)(77.00033529,339.72726336)
\curveto(77.11033209,339.53726138)(77.25033195,339.37726154)(77.42033529,339.24726336)
\curveto(77.46033174,339.2172617)(77.52033168,339.18226173)(77.60033529,339.14226336)
\curveto(77.71033149,339.07226184)(77.82033138,339.02726189)(77.93033529,339.00726336)
\curveto(78.05033115,338.98726193)(78.19533101,338.96726195)(78.36533529,338.94726336)
\lineto(78.45533529,338.94726336)
\curveto(78.49533071,338.94726197)(78.52533068,338.95226196)(78.54533529,338.96226336)
\lineto(78.68033529,338.96226336)
\curveto(78.75033045,338.98226193)(78.81533039,338.99726192)(78.87533529,339.00726336)
\curveto(78.94533026,339.02726189)(79.01033019,339.04726187)(79.07033529,339.06726336)
\curveto(79.37032983,339.19726172)(79.6003296,339.38726153)(79.76033529,339.63726336)
\curveto(79.8003294,339.68726123)(79.83532937,339.74226117)(79.86533529,339.80226336)
\curveto(79.89532931,339.87226104)(79.92032928,339.93226098)(79.94033529,339.98226336)
\curveto(79.98032922,340.09226082)(80.01532919,340.18726073)(80.04533529,340.26726336)
\curveto(80.07532913,340.35726056)(80.14532906,340.42726049)(80.25533529,340.47726336)
\curveto(80.34532886,340.5172604)(80.49032871,340.53226038)(80.69033529,340.52226336)
\lineto(81.18533529,340.52226336)
\lineto(81.39533529,340.52226336)
\curveto(81.47532773,340.53226038)(81.54032766,340.52726039)(81.59033529,340.50726336)
\lineto(81.71033529,340.50726336)
\lineto(81.83033529,340.47726336)
\curveto(81.87032733,340.47726044)(81.9003273,340.46726045)(81.92033529,340.44726336)
\curveto(81.97032723,340.40726051)(82.0003272,340.34726057)(82.01033529,340.26726336)
\curveto(82.03032717,340.19726072)(82.03032717,340.12226079)(82.01033529,340.04226336)
\curveto(81.92032728,339.7122612)(81.81032739,339.4172615)(81.68033529,339.15726336)
\curveto(81.27032793,338.38726253)(80.61532859,337.85226306)(79.71533529,337.55226336)
\curveto(79.61532959,337.52226339)(79.51032969,337.50226341)(79.40033529,337.49226336)
\curveto(79.29032991,337.47226344)(79.18033002,337.44726347)(79.07033529,337.41726336)
\curveto(79.01033019,337.40726351)(78.95033025,337.40226351)(78.89033529,337.40226336)
\curveto(78.83033037,337.40226351)(78.77033043,337.39726352)(78.71033529,337.38726336)
\lineto(78.54533529,337.38726336)
\curveto(78.49533071,337.36726355)(78.42033078,337.36226355)(78.32033529,337.37226336)
\curveto(78.22033098,337.37226354)(78.14533106,337.37726354)(78.09533529,337.38726336)
\curveto(78.01533119,337.40726351)(77.94033126,337.4172635)(77.87033529,337.41726336)
\curveto(77.81033139,337.40726351)(77.74533146,337.4122635)(77.67533529,337.43226336)
\lineto(77.52533529,337.46226336)
\curveto(77.47533173,337.46226345)(77.42533178,337.46726345)(77.37533529,337.47726336)
\curveto(77.26533194,337.50726341)(77.16033204,337.53726338)(77.06033529,337.56726336)
\curveto(76.96033224,337.59726332)(76.86533234,337.63226328)(76.77533529,337.67226336)
\curveto(76.3053329,337.87226304)(75.91033329,338.12726279)(75.59033529,338.43726336)
\curveto(75.27033393,338.75726216)(75.01033419,339.15226176)(74.81033529,339.62226336)
\curveto(74.76033444,339.7122612)(74.72033448,339.80726111)(74.69033529,339.90726336)
\lineto(74.60033529,340.23726336)
\curveto(74.59033461,340.27726064)(74.58533462,340.3122606)(74.58533529,340.34226336)
\curveto(74.58533462,340.38226053)(74.57533463,340.42726049)(74.55533529,340.47726336)
\curveto(74.53533467,340.54726037)(74.52533468,340.6172603)(74.52533529,340.68726336)
\curveto(74.52533468,340.76726015)(74.51533469,340.84226007)(74.49533529,340.91226336)
\lineto(74.49533529,341.16726336)
\curveto(74.47533473,341.2172597)(74.46533474,341.27225964)(74.46533529,341.33226336)
\curveto(74.46533474,341.40225951)(74.47533473,341.46225945)(74.49533529,341.51226336)
\curveto(74.5053347,341.56225935)(74.5053347,341.60725931)(74.49533529,341.64726336)
\curveto(74.48533472,341.68725923)(74.48533472,341.72725919)(74.49533529,341.76726336)
\curveto(74.51533469,341.83725908)(74.52033468,341.90225901)(74.51033529,341.96226336)
\curveto(74.51033469,342.02225889)(74.52033468,342.08225883)(74.54033529,342.14226336)
\curveto(74.59033461,342.32225859)(74.63033457,342.49225842)(74.66033529,342.65226336)
\curveto(74.69033451,342.82225809)(74.73533447,342.98725793)(74.79533529,343.14726336)
\curveto(75.01533419,343.65725726)(75.29033391,344.08225683)(75.62033529,344.42226336)
\curveto(75.96033324,344.76225615)(76.39033281,345.03725588)(76.91033529,345.24726336)
\curveto(77.05033215,345.30725561)(77.19533201,345.34725557)(77.34533529,345.36726336)
\curveto(77.49533171,345.39725552)(77.65033155,345.43225548)(77.81033529,345.47226336)
\curveto(77.89033131,345.48225543)(77.96533124,345.48725543)(78.03533529,345.48726336)
\curveto(78.1053311,345.48725543)(78.18033102,345.49225542)(78.26033529,345.50226336)
}
}
{
\newrgbcolor{curcolor}{0 0 0}
\pscustom[linestyle=none,fillstyle=solid,fillcolor=curcolor]
{
\newpath
\moveto(85.40361654,348.14226336)
\curveto(85.47361359,348.06225285)(85.50861356,347.94225297)(85.50861654,347.78226336)
\lineto(85.50861654,347.31726336)
\lineto(85.50861654,346.91226336)
\curveto(85.50861356,346.77225414)(85.47361359,346.67725424)(85.40361654,346.62726336)
\curveto(85.34361372,346.57725434)(85.2636138,346.54725437)(85.16361654,346.53726336)
\curveto(85.07361399,346.52725439)(84.97361409,346.52225439)(84.86361654,346.52226336)
\lineto(84.02361654,346.52226336)
\curveto(83.91361515,346.52225439)(83.81361525,346.52725439)(83.72361654,346.53726336)
\curveto(83.64361542,346.54725437)(83.57361549,346.57725434)(83.51361654,346.62726336)
\curveto(83.47361559,346.65725426)(83.44361562,346.7122542)(83.42361654,346.79226336)
\curveto(83.41361565,346.88225403)(83.40361566,346.97725394)(83.39361654,347.07726336)
\lineto(83.39361654,347.40726336)
\curveto(83.40361566,347.5172534)(83.40861566,347.6122533)(83.40861654,347.69226336)
\lineto(83.40861654,347.90226336)
\curveto(83.41861565,347.97225294)(83.43861563,348.03225288)(83.46861654,348.08226336)
\curveto(83.48861558,348.12225279)(83.51361555,348.15225276)(83.54361654,348.17226336)
\lineto(83.66361654,348.23226336)
\curveto(83.68361538,348.23225268)(83.70861536,348.23225268)(83.73861654,348.23226336)
\curveto(83.7686153,348.24225267)(83.79361527,348.24725267)(83.81361654,348.24726336)
\lineto(84.90861654,348.24726336)
\curveto(85.00861406,348.24725267)(85.10361396,348.24225267)(85.19361654,348.23226336)
\curveto(85.28361378,348.22225269)(85.35361371,348.19225272)(85.40361654,348.14226336)
\moveto(85.50861654,338.37726336)
\curveto(85.50861356,338.17726274)(85.50361356,338.00726291)(85.49361654,337.86726336)
\curveto(85.48361358,337.72726319)(85.39361367,337.63226328)(85.22361654,337.58226336)
\curveto(85.1636139,337.56226335)(85.09861397,337.55226336)(85.02861654,337.55226336)
\curveto(84.95861411,337.56226335)(84.88361418,337.56726335)(84.80361654,337.56726336)
\lineto(83.96361654,337.56726336)
\curveto(83.87361519,337.56726335)(83.78361528,337.57226334)(83.69361654,337.58226336)
\curveto(83.61361545,337.59226332)(83.55361551,337.62226329)(83.51361654,337.67226336)
\curveto(83.45361561,337.74226317)(83.41861565,337.82726309)(83.40861654,337.92726336)
\lineto(83.40861654,338.27226336)
\lineto(83.40861654,344.60226336)
\lineto(83.40861654,344.90226336)
\curveto(83.40861566,345.00225591)(83.42861564,345.08225583)(83.46861654,345.14226336)
\curveto(83.52861554,345.2122557)(83.61361545,345.25725566)(83.72361654,345.27726336)
\curveto(83.74361532,345.28725563)(83.7686153,345.28725563)(83.79861654,345.27726336)
\curveto(83.83861523,345.27725564)(83.8686152,345.28225563)(83.88861654,345.29226336)
\lineto(84.63861654,345.29226336)
\lineto(84.83361654,345.29226336)
\curveto(84.91361415,345.30225561)(84.97861409,345.30225561)(85.02861654,345.29226336)
\lineto(85.14861654,345.29226336)
\curveto(85.20861386,345.27225564)(85.2636138,345.25725566)(85.31361654,345.24726336)
\curveto(85.3636137,345.23725568)(85.40361366,345.20725571)(85.43361654,345.15726336)
\curveto(85.47361359,345.10725581)(85.49361357,345.03725588)(85.49361654,344.94726336)
\curveto(85.50361356,344.85725606)(85.50861356,344.76225615)(85.50861654,344.66226336)
\lineto(85.50861654,338.37726336)
}
}
{
\newrgbcolor{curcolor}{0 0 0}
\pscustom[linestyle=none,fillstyle=solid,fillcolor=curcolor]
{
\newpath
\moveto(94.94080404,341.73726336)
\curveto(94.92079551,341.78725913)(94.91579552,341.84225907)(94.92580404,341.90226336)
\curveto(94.9357955,341.96225895)(94.9307955,342.0172589)(94.91080404,342.06726336)
\curveto(94.90079553,342.10725881)(94.89579554,342.14725877)(94.89580404,342.18726336)
\curveto(94.89579554,342.22725869)(94.89079554,342.26725865)(94.88080404,342.30726336)
\lineto(94.82080404,342.57726336)
\curveto(94.80079563,342.66725825)(94.77579566,342.75225816)(94.74580404,342.83226336)
\curveto(94.69579574,342.97225794)(94.65079578,343.10225781)(94.61080404,343.22226336)
\curveto(94.57079586,343.35225756)(94.51579592,343.47225744)(94.44580404,343.58226336)
\curveto(94.37579606,343.69225722)(94.30579613,343.79725712)(94.23580404,343.89726336)
\curveto(94.17579626,343.99725692)(94.10579633,344.09725682)(94.02580404,344.19726336)
\curveto(93.94579649,344.30725661)(93.84579659,344.40725651)(93.72580404,344.49726336)
\curveto(93.61579682,344.59725632)(93.50579693,344.68725623)(93.39580404,344.76726336)
\curveto(93.06579737,344.99725592)(92.68579775,345.17725574)(92.25580404,345.30726336)
\curveto(91.8357986,345.43725548)(91.3357991,345.49725542)(90.75580404,345.48726336)
\curveto(90.68579975,345.47725544)(90.61579982,345.47225544)(90.54580404,345.47226336)
\curveto(90.47579996,345.47225544)(90.40080003,345.46725545)(90.32080404,345.45726336)
\curveto(90.17080026,345.4172555)(90.02580041,345.38725553)(89.88580404,345.36726336)
\curveto(89.74580069,345.34725557)(89.61080082,345.3122556)(89.48080404,345.26226336)
\curveto(89.37080106,345.2122557)(89.26080117,345.16725575)(89.15080404,345.12726336)
\curveto(89.04080139,345.08725583)(88.9358015,345.04225587)(88.83580404,344.99226336)
\curveto(88.47580196,344.76225615)(88.17080226,344.50725641)(87.92080404,344.22726336)
\curveto(87.67080276,343.95725696)(87.45580298,343.6172573)(87.27580404,343.20726336)
\curveto(87.22580321,343.08725783)(87.18580325,342.96225795)(87.15580404,342.83226336)
\curveto(87.12580331,342.7122582)(87.09080334,342.58725833)(87.05080404,342.45726336)
\curveto(87.0308034,342.40725851)(87.02080341,342.35725856)(87.02080404,342.30726336)
\curveto(87.02080341,342.26725865)(87.01580342,342.22225869)(87.00580404,342.17226336)
\curveto(86.98580345,342.12225879)(86.97580346,342.06725885)(86.97580404,342.00726336)
\curveto(86.98580345,341.95725896)(86.98580345,341.90725901)(86.97580404,341.85726336)
\lineto(86.97580404,341.75226336)
\curveto(86.95580348,341.69225922)(86.94080349,341.60725931)(86.93080404,341.49726336)
\curveto(86.9308035,341.38725953)(86.94080349,341.30225961)(86.96080404,341.24226336)
\lineto(86.96080404,341.10726336)
\curveto(86.96080347,341.06725985)(86.96580347,341.02225989)(86.97580404,340.97226336)
\curveto(86.99580344,340.89226002)(87.00580343,340.80726011)(87.00580404,340.71726336)
\curveto(87.00580343,340.63726028)(87.01580342,340.55726036)(87.03580404,340.47726336)
\curveto(87.05580338,340.42726049)(87.06580337,340.38226053)(87.06580404,340.34226336)
\curveto(87.06580337,340.30226061)(87.07580336,340.25726066)(87.09580404,340.20726336)
\curveto(87.12580331,340.09726082)(87.15080328,339.99226092)(87.17080404,339.89226336)
\curveto(87.20080323,339.79226112)(87.24080319,339.69726122)(87.29080404,339.60726336)
\curveto(87.46080297,339.2172617)(87.67080276,338.88226203)(87.92080404,338.60226336)
\curveto(88.17080226,338.32226259)(88.47080196,338.07726284)(88.82080404,337.86726336)
\curveto(88.9308015,337.80726311)(89.0358014,337.75726316)(89.13580404,337.71726336)
\curveto(89.24580119,337.67726324)(89.36080107,337.63726328)(89.48080404,337.59726336)
\curveto(89.57080086,337.55726336)(89.66580077,337.52726339)(89.76580404,337.50726336)
\curveto(89.86580057,337.48726343)(89.96580047,337.46226345)(90.06580404,337.43226336)
\curveto(90.11580032,337.42226349)(90.15580028,337.4172635)(90.18580404,337.41726336)
\curveto(90.22580021,337.4172635)(90.26580017,337.4122635)(90.30580404,337.40226336)
\curveto(90.35580008,337.38226353)(90.40580003,337.37726354)(90.45580404,337.38726336)
\curveto(90.51579992,337.38726353)(90.57079986,337.38226353)(90.62080404,337.37226336)
\lineto(90.77080404,337.37226336)
\curveto(90.8307996,337.35226356)(90.91579952,337.34726357)(91.02580404,337.35726336)
\curveto(91.1357993,337.35726356)(91.21579922,337.36226355)(91.26580404,337.37226336)
\curveto(91.29579914,337.37226354)(91.32579911,337.37726354)(91.35580404,337.38726336)
\lineto(91.46080404,337.38726336)
\curveto(91.51079892,337.39726352)(91.56579887,337.40226351)(91.62580404,337.40226336)
\curveto(91.68579875,337.40226351)(91.74079869,337.4122635)(91.79080404,337.43226336)
\curveto(91.92079851,337.46226345)(92.04579839,337.49226342)(92.16580404,337.52226336)
\curveto(92.29579814,337.54226337)(92.42079801,337.57726334)(92.54080404,337.62726336)
\curveto(93.02079741,337.82726309)(93.430797,338.07726284)(93.77080404,338.37726336)
\curveto(94.11079632,338.67726224)(94.38579605,339.06726185)(94.59580404,339.54726336)
\curveto(94.64579579,339.64726127)(94.68579575,339.75226116)(94.71580404,339.86226336)
\curveto(94.74579569,339.98226093)(94.78079565,340.09726082)(94.82080404,340.20726336)
\curveto(94.8307956,340.27726064)(94.84079559,340.34226057)(94.85080404,340.40226336)
\curveto(94.86079557,340.46226045)(94.87579556,340.52726039)(94.89580404,340.59726336)
\curveto(94.91579552,340.67726024)(94.92079551,340.75726016)(94.91080404,340.83726336)
\curveto(94.91079552,340.91726)(94.92079551,340.99725992)(94.94080404,341.07726336)
\lineto(94.94080404,341.22726336)
\curveto(94.96079547,341.28725963)(94.97079546,341.37225954)(94.97080404,341.48226336)
\curveto(94.97079546,341.59225932)(94.96079547,341.67725924)(94.94080404,341.73726336)
\moveto(92.84080404,341.19726336)
\curveto(92.8307976,341.14725977)(92.82579761,341.09725982)(92.82580404,341.04726336)
\lineto(92.82580404,340.91226336)
\curveto(92.81579762,340.87226004)(92.81079762,340.83226008)(92.81080404,340.79226336)
\curveto(92.81079762,340.76226015)(92.80579763,340.72726019)(92.79580404,340.68726336)
\curveto(92.76579767,340.57726034)(92.74079769,340.47226044)(92.72080404,340.37226336)
\curveto(92.70079773,340.27226064)(92.67079776,340.17226074)(92.63080404,340.07226336)
\curveto(92.52079791,339.82226109)(92.38579805,339.6122613)(92.22580404,339.44226336)
\curveto(92.06579837,339.27226164)(91.85579858,339.13726178)(91.59580404,339.03726336)
\curveto(91.52579891,339.00726191)(91.45079898,338.98726193)(91.37080404,338.97726336)
\curveto(91.29079914,338.96726195)(91.21079922,338.95226196)(91.13080404,338.93226336)
\lineto(91.01080404,338.93226336)
\curveto(90.97079946,338.92226199)(90.92579951,338.917262)(90.87580404,338.91726336)
\lineto(90.75580404,338.94726336)
\curveto(90.71579972,338.95726196)(90.68079975,338.95726196)(90.65080404,338.94726336)
\curveto(90.62079981,338.94726197)(90.58579985,338.95226196)(90.54580404,338.96226336)
\curveto(90.45579998,338.98226193)(90.36580007,339.00726191)(90.27580404,339.03726336)
\curveto(90.19580024,339.06726185)(90.12080031,339.10726181)(90.05080404,339.15726336)
\curveto(89.80080063,339.30726161)(89.61580082,339.47226144)(89.49580404,339.65226336)
\curveto(89.38580105,339.84226107)(89.28080115,340.08726083)(89.18080404,340.38726336)
\curveto(89.16080127,340.46726045)(89.14580129,340.54226037)(89.13580404,340.61226336)
\curveto(89.12580131,340.69226022)(89.11080132,340.77226014)(89.09080404,340.85226336)
\lineto(89.09080404,340.98726336)
\curveto(89.07080136,341.05725986)(89.05580138,341.16225975)(89.04580404,341.30226336)
\curveto(89.04580139,341.44225947)(89.05580138,341.54725937)(89.07580404,341.61726336)
\lineto(89.07580404,341.76726336)
\curveto(89.07580136,341.8172591)(89.08080135,341.86725905)(89.09080404,341.91726336)
\curveto(89.11080132,342.02725889)(89.12580131,342.13725878)(89.13580404,342.24726336)
\curveto(89.15580128,342.35725856)(89.18080125,342.46225845)(89.21080404,342.56226336)
\curveto(89.30080113,342.83225808)(89.42080101,343.06725785)(89.57080404,343.26726336)
\curveto(89.7308007,343.47725744)(89.9358005,343.63725728)(90.18580404,343.74726336)
\curveto(90.2358002,343.77725714)(90.29080014,343.79725712)(90.35080404,343.80726336)
\lineto(90.56080404,343.86726336)
\curveto(90.59079984,343.87725704)(90.62579981,343.87725704)(90.66580404,343.86726336)
\curveto(90.70579973,343.86725705)(90.74079969,343.87725704)(90.77080404,343.89726336)
\lineto(91.04080404,343.89726336)
\curveto(91.1307993,343.90725701)(91.21579922,343.90225701)(91.29580404,343.88226336)
\curveto(91.36579907,343.86225705)(91.430799,343.84225707)(91.49080404,343.82226336)
\curveto(91.55079888,343.8122571)(91.61079882,343.79725712)(91.67080404,343.77726336)
\curveto(91.92079851,343.66725725)(92.12079831,343.5172574)(92.27080404,343.32726336)
\curveto(92.42079801,343.14725777)(92.55079788,342.92725799)(92.66080404,342.66726336)
\curveto(92.69079774,342.58725833)(92.71079772,342.50225841)(92.72080404,342.41226336)
\lineto(92.78080404,342.17226336)
\curveto(92.79079764,342.15225876)(92.79579764,342.12225879)(92.79580404,342.08226336)
\curveto(92.80579763,342.03225888)(92.81079762,341.97725894)(92.81080404,341.91726336)
\curveto(92.81079762,341.85725906)(92.82079761,341.80225911)(92.84080404,341.75226336)
\lineto(92.84080404,341.63226336)
\curveto(92.85079758,341.58225933)(92.85579758,341.50725941)(92.85580404,341.40726336)
\curveto(92.85579758,341.3172596)(92.85079758,341.24725967)(92.84080404,341.19726336)
\moveto(91.61080404,348.36726336)
\lineto(92.67580404,348.36726336)
\curveto(92.75579768,348.36725255)(92.85079758,348.36725255)(92.96080404,348.36726336)
\curveto(93.07079736,348.36725255)(93.15079728,348.35225256)(93.20080404,348.32226336)
\curveto(93.22079721,348.3122526)(93.2307972,348.29725262)(93.23080404,348.27726336)
\curveto(93.24079719,348.26725265)(93.25579718,348.25725266)(93.27580404,348.24726336)
\curveto(93.28579715,348.12725279)(93.2357972,348.02225289)(93.12580404,347.93226336)
\curveto(93.02579741,347.84225307)(92.94079749,347.76225315)(92.87080404,347.69226336)
\curveto(92.79079764,347.62225329)(92.71079772,347.54725337)(92.63080404,347.46726336)
\curveto(92.56079787,347.39725352)(92.48579795,347.33225358)(92.40580404,347.27226336)
\curveto(92.36579807,347.24225367)(92.3307981,347.20725371)(92.30080404,347.16726336)
\curveto(92.28079815,347.13725378)(92.25079818,347.1122538)(92.21080404,347.09226336)
\curveto(92.19079824,347.06225385)(92.16579827,347.03725388)(92.13580404,347.01726336)
\lineto(91.98580404,346.86726336)
\lineto(91.83580404,346.74726336)
\lineto(91.79080404,346.70226336)
\curveto(91.79079864,346.69225422)(91.78079865,346.67725424)(91.76080404,346.65726336)
\curveto(91.68079875,346.59725432)(91.60079883,346.53225438)(91.52080404,346.46226336)
\curveto(91.45079898,346.39225452)(91.36079907,346.33725458)(91.25080404,346.29726336)
\curveto(91.21079922,346.28725463)(91.17079926,346.28225463)(91.13080404,346.28226336)
\curveto(91.10079933,346.28225463)(91.06079937,346.27725464)(91.01080404,346.26726336)
\curveto(90.98079945,346.25725466)(90.94079949,346.25225466)(90.89080404,346.25226336)
\curveto(90.84079959,346.26225465)(90.79579964,346.26725465)(90.75580404,346.26726336)
\lineto(90.41080404,346.26726336)
\curveto(90.29080014,346.26725465)(90.20080023,346.29225462)(90.14080404,346.34226336)
\curveto(90.08080035,346.38225453)(90.06580037,346.45225446)(90.09580404,346.55226336)
\curveto(90.11580032,346.63225428)(90.15080028,346.70225421)(90.20080404,346.76226336)
\curveto(90.25080018,346.83225408)(90.29580014,346.90225401)(90.33580404,346.97226336)
\curveto(90.4358,347.1122538)(90.5307999,347.24725367)(90.62080404,347.37726336)
\curveto(90.71079972,347.50725341)(90.80079963,347.64225327)(90.89080404,347.78226336)
\curveto(90.94079949,347.86225305)(90.99079944,347.94725297)(91.04080404,348.03726336)
\curveto(91.10079933,348.12725279)(91.16579927,348.19725272)(91.23580404,348.24726336)
\curveto(91.27579916,348.27725264)(91.34579909,348.3122526)(91.44580404,348.35226336)
\curveto(91.46579897,348.36225255)(91.49079894,348.36225255)(91.52080404,348.35226336)
\curveto(91.56079887,348.35225256)(91.59079884,348.35725256)(91.61080404,348.36726336)
}
}
{
\newrgbcolor{curcolor}{0 0 0}
\pscustom[linestyle=none,fillstyle=solid,fillcolor=curcolor]
{
\newpath
\moveto(100.76572592,345.48726336)
\curveto(101.36572011,345.50725541)(101.86571961,345.42225549)(102.26572592,345.23226336)
\curveto(102.66571881,345.04225587)(102.9807185,344.76225615)(103.21072592,344.39226336)
\curveto(103.2807182,344.28225663)(103.33571814,344.16225675)(103.37572592,344.03226336)
\curveto(103.41571806,343.912257)(103.45571802,343.78725713)(103.49572592,343.65726336)
\curveto(103.51571796,343.57725734)(103.52571795,343.50225741)(103.52572592,343.43226336)
\curveto(103.53571794,343.36225755)(103.55071793,343.29225762)(103.57072592,343.22226336)
\curveto(103.57071791,343.16225775)(103.5757179,343.12225779)(103.58572592,343.10226336)
\curveto(103.60571787,342.96225795)(103.61571786,342.8172581)(103.61572592,342.66726336)
\lineto(103.61572592,342.23226336)
\lineto(103.61572592,340.89726336)
\lineto(103.61572592,338.46726336)
\curveto(103.61571786,338.27726264)(103.61071787,338.09226282)(103.60072592,337.91226336)
\curveto(103.60071788,337.74226317)(103.53071795,337.63226328)(103.39072592,337.58226336)
\curveto(103.33071815,337.56226335)(103.26071822,337.55226336)(103.18072592,337.55226336)
\lineto(102.94072592,337.55226336)
\lineto(102.13072592,337.55226336)
\curveto(102.01071947,337.55226336)(101.90071958,337.55726336)(101.80072592,337.56726336)
\curveto(101.71071977,337.58726333)(101.64071984,337.63226328)(101.59072592,337.70226336)
\curveto(101.55071993,337.76226315)(101.52571995,337.83726308)(101.51572592,337.92726336)
\lineto(101.51572592,338.24226336)
\lineto(101.51572592,339.29226336)
\lineto(101.51572592,341.52726336)
\curveto(101.51571996,341.89725902)(101.50071998,342.23725868)(101.47072592,342.54726336)
\curveto(101.44072004,342.86725805)(101.35072013,343.13725778)(101.20072592,343.35726336)
\curveto(101.06072042,343.55725736)(100.85572062,343.69725722)(100.58572592,343.77726336)
\curveto(100.53572094,343.79725712)(100.480721,343.80725711)(100.42072592,343.80726336)
\curveto(100.37072111,343.80725711)(100.31572116,343.8172571)(100.25572592,343.83726336)
\curveto(100.20572127,343.84725707)(100.14072134,343.84725707)(100.06072592,343.83726336)
\curveto(99.99072149,343.83725708)(99.93572154,343.83225708)(99.89572592,343.82226336)
\curveto(99.85572162,343.8122571)(99.82072166,343.80725711)(99.79072592,343.80726336)
\curveto(99.76072172,343.80725711)(99.73072175,343.80225711)(99.70072592,343.79226336)
\curveto(99.47072201,343.73225718)(99.28572219,343.65225726)(99.14572592,343.55226336)
\curveto(98.82572265,343.32225759)(98.63572284,342.98725793)(98.57572592,342.54726336)
\curveto(98.51572296,342.10725881)(98.48572299,341.6122593)(98.48572592,341.06226336)
\lineto(98.48572592,339.18726336)
\lineto(98.48572592,338.27226336)
\lineto(98.48572592,338.00226336)
\curveto(98.48572299,337.912263)(98.47072301,337.83726308)(98.44072592,337.77726336)
\curveto(98.39072309,337.66726325)(98.31072317,337.60226331)(98.20072592,337.58226336)
\curveto(98.09072339,337.56226335)(97.95572352,337.55226336)(97.79572592,337.55226336)
\lineto(97.04572592,337.55226336)
\curveto(96.93572454,337.55226336)(96.82572465,337.55726336)(96.71572592,337.56726336)
\curveto(96.60572487,337.57726334)(96.52572495,337.6122633)(96.47572592,337.67226336)
\curveto(96.40572507,337.76226315)(96.37072511,337.89226302)(96.37072592,338.06226336)
\curveto(96.3807251,338.23226268)(96.38572509,338.39226252)(96.38572592,338.54226336)
\lineto(96.38572592,340.58226336)
\lineto(96.38572592,343.88226336)
\lineto(96.38572592,344.64726336)
\lineto(96.38572592,344.94726336)
\curveto(96.39572508,345.03725588)(96.42572505,345.1122558)(96.47572592,345.17226336)
\curveto(96.49572498,345.20225571)(96.52572495,345.22225569)(96.56572592,345.23226336)
\curveto(96.61572486,345.25225566)(96.66572481,345.26725565)(96.71572592,345.27726336)
\lineto(96.79072592,345.27726336)
\curveto(96.84072464,345.28725563)(96.89072459,345.29225562)(96.94072592,345.29226336)
\lineto(97.10572592,345.29226336)
\lineto(97.73572592,345.29226336)
\curveto(97.81572366,345.29225562)(97.89072359,345.28725563)(97.96072592,345.27726336)
\curveto(98.04072344,345.27725564)(98.11072337,345.26725565)(98.17072592,345.24726336)
\curveto(98.24072324,345.2172557)(98.28572319,345.17225574)(98.30572592,345.11226336)
\curveto(98.33572314,345.05225586)(98.36072312,344.98225593)(98.38072592,344.90226336)
\curveto(98.39072309,344.86225605)(98.39072309,344.82725609)(98.38072592,344.79726336)
\curveto(98.3807231,344.76725615)(98.39072309,344.73725618)(98.41072592,344.70726336)
\curveto(98.43072305,344.65725626)(98.44572303,344.62725629)(98.45572592,344.61726336)
\curveto(98.475723,344.60725631)(98.50072298,344.59225632)(98.53072592,344.57226336)
\curveto(98.64072284,344.56225635)(98.73072275,344.59725632)(98.80072592,344.67726336)
\curveto(98.87072261,344.76725615)(98.94572253,344.83725608)(99.02572592,344.88726336)
\curveto(99.29572218,345.08725583)(99.59572188,345.24725567)(99.92572592,345.36726336)
\curveto(100.01572146,345.39725552)(100.10572137,345.4172555)(100.19572592,345.42726336)
\curveto(100.29572118,345.43725548)(100.40072108,345.45225546)(100.51072592,345.47226336)
\curveto(100.54072094,345.48225543)(100.58572089,345.48225543)(100.64572592,345.47226336)
\curveto(100.70572077,345.47225544)(100.74572073,345.47725544)(100.76572592,345.48726336)
}
}
{
\newrgbcolor{curcolor}{0 0 0}
\pscustom[linestyle=none,fillstyle=solid,fillcolor=curcolor]
{
}
}
{
\newrgbcolor{curcolor}{0 0 0}
\pscustom[linestyle=none,fillstyle=solid,fillcolor=curcolor]
{
\newpath
\moveto(116.99713217,338.40726336)
\lineto(116.99713217,337.98726336)
\curveto(116.9971238,337.85726306)(116.96712383,337.75226316)(116.90713217,337.67226336)
\curveto(116.85712394,337.62226329)(116.792124,337.58726333)(116.71213217,337.56726336)
\curveto(116.63212416,337.55726336)(116.54212425,337.55226336)(116.44213217,337.55226336)
\lineto(115.61713217,337.55226336)
\lineto(115.33213217,337.55226336)
\curveto(115.25212554,337.56226335)(115.18712561,337.58726333)(115.13713217,337.62726336)
\curveto(115.06712573,337.67726324)(115.02712577,337.74226317)(115.01713217,337.82226336)
\curveto(115.00712579,337.90226301)(114.98712581,337.98226293)(114.95713217,338.06226336)
\curveto(114.93712586,338.08226283)(114.91712588,338.09726282)(114.89713217,338.10726336)
\curveto(114.88712591,338.12726279)(114.87212592,338.14726277)(114.85213217,338.16726336)
\curveto(114.74212605,338.16726275)(114.66212613,338.14226277)(114.61213217,338.09226336)
\lineto(114.46213217,337.94226336)
\curveto(114.3921264,337.89226302)(114.32712647,337.84726307)(114.26713217,337.80726336)
\curveto(114.20712659,337.77726314)(114.14212665,337.73726318)(114.07213217,337.68726336)
\curveto(114.03212676,337.66726325)(113.98712681,337.64726327)(113.93713217,337.62726336)
\curveto(113.8971269,337.60726331)(113.85212694,337.58726333)(113.80213217,337.56726336)
\curveto(113.66212713,337.5172634)(113.51212728,337.47226344)(113.35213217,337.43226336)
\curveto(113.30212749,337.4122635)(113.25712754,337.40226351)(113.21713217,337.40226336)
\curveto(113.17712762,337.40226351)(113.13712766,337.39726352)(113.09713217,337.38726336)
\lineto(112.96213217,337.38726336)
\curveto(112.93212786,337.37726354)(112.8921279,337.37226354)(112.84213217,337.37226336)
\lineto(112.70713217,337.37226336)
\curveto(112.64712815,337.35226356)(112.55712824,337.34726357)(112.43713217,337.35726336)
\curveto(112.31712848,337.35726356)(112.23212856,337.36726355)(112.18213217,337.38726336)
\curveto(112.11212868,337.40726351)(112.04712875,337.4172635)(111.98713217,337.41726336)
\curveto(111.93712886,337.40726351)(111.88212891,337.4122635)(111.82213217,337.43226336)
\lineto(111.46213217,337.55226336)
\curveto(111.35212944,337.58226333)(111.24212955,337.62226329)(111.13213217,337.67226336)
\curveto(110.78213001,337.82226309)(110.46713033,338.05226286)(110.18713217,338.36226336)
\curveto(109.91713088,338.68226223)(109.70213109,339.0172619)(109.54213217,339.36726336)
\curveto(109.4921313,339.47726144)(109.45213134,339.58226133)(109.42213217,339.68226336)
\curveto(109.3921314,339.79226112)(109.35713144,339.90226101)(109.31713217,340.01226336)
\curveto(109.30713149,340.05226086)(109.30213149,340.08726083)(109.30213217,340.11726336)
\curveto(109.30213149,340.15726076)(109.2921315,340.20226071)(109.27213217,340.25226336)
\curveto(109.25213154,340.33226058)(109.23213156,340.4172605)(109.21213217,340.50726336)
\curveto(109.20213159,340.60726031)(109.18713161,340.70726021)(109.16713217,340.80726336)
\curveto(109.15713164,340.83726008)(109.15213164,340.87226004)(109.15213217,340.91226336)
\curveto(109.16213163,340.95225996)(109.16213163,340.98725993)(109.15213217,341.01726336)
\lineto(109.15213217,341.15226336)
\curveto(109.15213164,341.20225971)(109.14713165,341.25225966)(109.13713217,341.30226336)
\curveto(109.12713167,341.35225956)(109.12213167,341.40725951)(109.12213217,341.46726336)
\curveto(109.12213167,341.53725938)(109.12713167,341.59225932)(109.13713217,341.63226336)
\curveto(109.14713165,341.68225923)(109.15213164,341.72725919)(109.15213217,341.76726336)
\lineto(109.15213217,341.91726336)
\curveto(109.16213163,341.96725895)(109.16213163,342.0122589)(109.15213217,342.05226336)
\curveto(109.15213164,342.10225881)(109.16213163,342.15225876)(109.18213217,342.20226336)
\curveto(109.20213159,342.3122586)(109.21713158,342.4172585)(109.22713217,342.51726336)
\curveto(109.24713155,342.6172583)(109.27213152,342.7172582)(109.30213217,342.81726336)
\curveto(109.34213145,342.93725798)(109.37713142,343.05225786)(109.40713217,343.16226336)
\curveto(109.43713136,343.27225764)(109.47713132,343.38225753)(109.52713217,343.49226336)
\curveto(109.66713113,343.79225712)(109.84213095,344.07725684)(110.05213217,344.34726336)
\curveto(110.07213072,344.37725654)(110.0971307,344.40225651)(110.12713217,344.42226336)
\curveto(110.16713063,344.45225646)(110.1971306,344.48225643)(110.21713217,344.51226336)
\curveto(110.25713054,344.56225635)(110.2971305,344.60725631)(110.33713217,344.64726336)
\curveto(110.37713042,344.68725623)(110.42213037,344.72725619)(110.47213217,344.76726336)
\curveto(110.51213028,344.78725613)(110.54713025,344.8122561)(110.57713217,344.84226336)
\curveto(110.60713019,344.88225603)(110.64213015,344.912256)(110.68213217,344.93226336)
\curveto(110.93212986,345.10225581)(111.22212957,345.24225567)(111.55213217,345.35226336)
\curveto(111.62212917,345.37225554)(111.6921291,345.38725553)(111.76213217,345.39726336)
\curveto(111.84212895,345.40725551)(111.92212887,345.42225549)(112.00213217,345.44226336)
\curveto(112.07212872,345.46225545)(112.16212863,345.47225544)(112.27213217,345.47226336)
\curveto(112.38212841,345.48225543)(112.4921283,345.48725543)(112.60213217,345.48726336)
\curveto(112.71212808,345.48725543)(112.81712798,345.48225543)(112.91713217,345.47226336)
\curveto(113.02712777,345.46225545)(113.11712768,345.44725547)(113.18713217,345.42726336)
\curveto(113.33712746,345.37725554)(113.48212731,345.33225558)(113.62213217,345.29226336)
\curveto(113.76212703,345.25225566)(113.8921269,345.19725572)(114.01213217,345.12726336)
\curveto(114.08212671,345.07725584)(114.14712665,345.02725589)(114.20713217,344.97726336)
\curveto(114.26712653,344.93725598)(114.33212646,344.89225602)(114.40213217,344.84226336)
\curveto(114.44212635,344.8122561)(114.4971263,344.77225614)(114.56713217,344.72226336)
\curveto(114.64712615,344.67225624)(114.72212607,344.67225624)(114.79213217,344.72226336)
\curveto(114.83212596,344.74225617)(114.85212594,344.77725614)(114.85213217,344.82726336)
\curveto(114.85212594,344.87725604)(114.86212593,344.92725599)(114.88213217,344.97726336)
\lineto(114.88213217,345.12726336)
\curveto(114.8921259,345.15725576)(114.8971259,345.19225572)(114.89713217,345.23226336)
\lineto(114.89713217,345.35226336)
\lineto(114.89713217,347.39226336)
\curveto(114.8971259,347.50225341)(114.8921259,347.62225329)(114.88213217,347.75226336)
\curveto(114.88212591,347.89225302)(114.90712589,347.99725292)(114.95713217,348.06726336)
\curveto(114.9971258,348.14725277)(115.07212572,348.19725272)(115.18213217,348.21726336)
\curveto(115.20212559,348.22725269)(115.22212557,348.22725269)(115.24213217,348.21726336)
\curveto(115.26212553,348.2172527)(115.28212551,348.22225269)(115.30213217,348.23226336)
\lineto(116.36713217,348.23226336)
\curveto(116.48712431,348.23225268)(116.5971242,348.22725269)(116.69713217,348.21726336)
\curveto(116.797124,348.20725271)(116.87212392,348.16725275)(116.92213217,348.09726336)
\curveto(116.97212382,348.0172529)(116.9971238,347.912253)(116.99713217,347.78226336)
\lineto(116.99713217,347.42226336)
\lineto(116.99713217,338.40726336)
\moveto(114.95713217,341.34726336)
\curveto(114.96712583,341.38725953)(114.96712583,341.42725949)(114.95713217,341.46726336)
\lineto(114.95713217,341.60226336)
\curveto(114.95712584,341.70225921)(114.95212584,341.80225911)(114.94213217,341.90226336)
\curveto(114.93212586,342.00225891)(114.91712588,342.09225882)(114.89713217,342.17226336)
\curveto(114.87712592,342.28225863)(114.85712594,342.38225853)(114.83713217,342.47226336)
\curveto(114.82712597,342.56225835)(114.80212599,342.64725827)(114.76213217,342.72726336)
\curveto(114.62212617,343.08725783)(114.41712638,343.37225754)(114.14713217,343.58226336)
\curveto(113.88712691,343.79225712)(113.50712729,343.89725702)(113.00713217,343.89726336)
\curveto(112.94712785,343.89725702)(112.86712793,343.88725703)(112.76713217,343.86726336)
\curveto(112.68712811,343.84725707)(112.61212818,343.82725709)(112.54213217,343.80726336)
\curveto(112.48212831,343.79725712)(112.42212837,343.77725714)(112.36213217,343.74726336)
\curveto(112.0921287,343.63725728)(111.88212891,343.46725745)(111.73213217,343.23726336)
\curveto(111.58212921,343.00725791)(111.46212933,342.74725817)(111.37213217,342.45726336)
\curveto(111.34212945,342.35725856)(111.32212947,342.25725866)(111.31213217,342.15726336)
\curveto(111.30212949,342.05725886)(111.28212951,341.95225896)(111.25213217,341.84226336)
\lineto(111.25213217,341.63226336)
\curveto(111.23212956,341.54225937)(111.22712957,341.4172595)(111.23713217,341.25726336)
\curveto(111.24712955,341.10725981)(111.26212953,340.99725992)(111.28213217,340.92726336)
\lineto(111.28213217,340.83726336)
\curveto(111.2921295,340.8172601)(111.2971295,340.79726012)(111.29713217,340.77726336)
\curveto(111.31712948,340.69726022)(111.33212946,340.62226029)(111.34213217,340.55226336)
\curveto(111.36212943,340.48226043)(111.38212941,340.40726051)(111.40213217,340.32726336)
\curveto(111.57212922,339.80726111)(111.86212893,339.42226149)(112.27213217,339.17226336)
\curveto(112.40212839,339.08226183)(112.58212821,339.0122619)(112.81213217,338.96226336)
\curveto(112.85212794,338.95226196)(112.91212788,338.94726197)(112.99213217,338.94726336)
\curveto(113.02212777,338.93726198)(113.06712773,338.92726199)(113.12713217,338.91726336)
\curveto(113.1971276,338.917262)(113.25212754,338.92226199)(113.29213217,338.93226336)
\curveto(113.37212742,338.95226196)(113.45212734,338.96726195)(113.53213217,338.97726336)
\curveto(113.61212718,338.98726193)(113.6921271,339.00726191)(113.77213217,339.03726336)
\curveto(114.02212677,339.14726177)(114.22212657,339.28726163)(114.37213217,339.45726336)
\curveto(114.52212627,339.62726129)(114.65212614,339.84226107)(114.76213217,340.10226336)
\curveto(114.80212599,340.19226072)(114.83212596,340.28226063)(114.85213217,340.37226336)
\curveto(114.87212592,340.47226044)(114.8921259,340.57726034)(114.91213217,340.68726336)
\curveto(114.92212587,340.73726018)(114.92212587,340.78226013)(114.91213217,340.82226336)
\curveto(114.91212588,340.87226004)(114.92212587,340.92225999)(114.94213217,340.97226336)
\curveto(114.95212584,341.00225991)(114.95712584,341.03725988)(114.95713217,341.07726336)
\lineto(114.95713217,341.21226336)
\lineto(114.95713217,341.34726336)
}
}
{
\newrgbcolor{curcolor}{0 0 0}
\pscustom[linestyle=none,fillstyle=solid,fillcolor=curcolor]
{
\newpath
\moveto(125.94205404,341.49726336)
\curveto(125.96204588,341.4172595)(125.96204588,341.32725959)(125.94205404,341.22726336)
\curveto(125.92204592,341.12725979)(125.88704595,341.06225985)(125.83705404,341.03226336)
\curveto(125.78704605,340.99225992)(125.71204613,340.96225995)(125.61205404,340.94226336)
\curveto(125.52204632,340.93225998)(125.41704642,340.92225999)(125.29705404,340.91226336)
\lineto(124.95205404,340.91226336)
\curveto(124.842047,340.92225999)(124.7420471,340.92725999)(124.65205404,340.92726336)
\lineto(120.99205404,340.92726336)
\lineto(120.78205404,340.92726336)
\curveto(120.72205112,340.92725999)(120.66705117,340.91726)(120.61705404,340.89726336)
\curveto(120.5370513,340.85726006)(120.48705135,340.8172601)(120.46705404,340.77726336)
\curveto(120.44705139,340.75726016)(120.42705141,340.7172602)(120.40705404,340.65726336)
\curveto(120.38705145,340.60726031)(120.38205146,340.55726036)(120.39205404,340.50726336)
\curveto(120.41205143,340.44726047)(120.42205142,340.38726053)(120.42205404,340.32726336)
\curveto(120.43205141,340.27726064)(120.44705139,340.22226069)(120.46705404,340.16226336)
\curveto(120.54705129,339.92226099)(120.6420512,339.72226119)(120.75205404,339.56226336)
\curveto(120.87205097,339.4122615)(121.03205081,339.27726164)(121.23205404,339.15726336)
\curveto(121.31205053,339.10726181)(121.39205045,339.07226184)(121.47205404,339.05226336)
\curveto(121.56205028,339.04226187)(121.65205019,339.02226189)(121.74205404,338.99226336)
\curveto(121.82205002,338.97226194)(121.93204991,338.95726196)(122.07205404,338.94726336)
\curveto(122.21204963,338.93726198)(122.33204951,338.94226197)(122.43205404,338.96226336)
\lineto(122.56705404,338.96226336)
\curveto(122.66704917,338.98226193)(122.75704908,339.00226191)(122.83705404,339.02226336)
\curveto(122.92704891,339.05226186)(123.01204883,339.08226183)(123.09205404,339.11226336)
\curveto(123.19204865,339.16226175)(123.30204854,339.22726169)(123.42205404,339.30726336)
\curveto(123.55204829,339.38726153)(123.64704819,339.46726145)(123.70705404,339.54726336)
\curveto(123.75704808,339.6172613)(123.80704803,339.68226123)(123.85705404,339.74226336)
\curveto(123.91704792,339.8122611)(123.98704785,339.86226105)(124.06705404,339.89226336)
\curveto(124.16704767,339.94226097)(124.29204755,339.96226095)(124.44205404,339.95226336)
\lineto(124.87705404,339.95226336)
\lineto(125.05705404,339.95226336)
\curveto(125.12704671,339.96226095)(125.18704665,339.95726096)(125.23705404,339.93726336)
\lineto(125.38705404,339.93726336)
\curveto(125.48704635,339.917261)(125.55704628,339.89226102)(125.59705404,339.86226336)
\curveto(125.6370462,339.84226107)(125.65704618,339.79726112)(125.65705404,339.72726336)
\curveto(125.66704617,339.65726126)(125.66204618,339.59726132)(125.64205404,339.54726336)
\curveto(125.59204625,339.40726151)(125.5370463,339.28226163)(125.47705404,339.17226336)
\curveto(125.41704642,339.06226185)(125.34704649,338.95226196)(125.26705404,338.84226336)
\curveto(125.04704679,338.5122624)(124.79704704,338.24726267)(124.51705404,338.04726336)
\curveto(124.2370476,337.84726307)(123.88704795,337.67726324)(123.46705404,337.53726336)
\curveto(123.35704848,337.49726342)(123.24704859,337.47226344)(123.13705404,337.46226336)
\curveto(123.02704881,337.45226346)(122.91204893,337.43226348)(122.79205404,337.40226336)
\curveto(122.75204909,337.39226352)(122.70704913,337.39226352)(122.65705404,337.40226336)
\curveto(122.61704922,337.40226351)(122.57704926,337.39726352)(122.53705404,337.38726336)
\lineto(122.37205404,337.38726336)
\curveto(122.32204952,337.36726355)(122.26204958,337.36226355)(122.19205404,337.37226336)
\curveto(122.13204971,337.37226354)(122.07704976,337.37726354)(122.02705404,337.38726336)
\curveto(121.94704989,337.39726352)(121.87704996,337.39726352)(121.81705404,337.38726336)
\curveto(121.75705008,337.37726354)(121.69205015,337.38226353)(121.62205404,337.40226336)
\curveto(121.57205027,337.42226349)(121.51705032,337.43226348)(121.45705404,337.43226336)
\curveto(121.39705044,337.43226348)(121.3420505,337.44226347)(121.29205404,337.46226336)
\curveto(121.18205066,337.48226343)(121.07205077,337.50726341)(120.96205404,337.53726336)
\curveto(120.85205099,337.55726336)(120.75205109,337.59226332)(120.66205404,337.64226336)
\curveto(120.55205129,337.68226323)(120.44705139,337.7172632)(120.34705404,337.74726336)
\curveto(120.25705158,337.78726313)(120.17205167,337.83226308)(120.09205404,337.88226336)
\curveto(119.77205207,338.08226283)(119.48705235,338.3122626)(119.23705404,338.57226336)
\curveto(118.98705285,338.84226207)(118.78205306,339.15226176)(118.62205404,339.50226336)
\curveto(118.57205327,339.6122613)(118.53205331,339.72226119)(118.50205404,339.83226336)
\curveto(118.47205337,339.95226096)(118.43205341,340.07226084)(118.38205404,340.19226336)
\curveto(118.37205347,340.23226068)(118.36705347,340.26726065)(118.36705404,340.29726336)
\curveto(118.36705347,340.33726058)(118.36205348,340.37726054)(118.35205404,340.41726336)
\curveto(118.31205353,340.53726038)(118.28705355,340.66726025)(118.27705404,340.80726336)
\lineto(118.24705404,341.22726336)
\curveto(118.24705359,341.27725964)(118.2420536,341.33225958)(118.23205404,341.39226336)
\curveto(118.23205361,341.45225946)(118.2370536,341.50725941)(118.24705404,341.55726336)
\lineto(118.24705404,341.73726336)
\lineto(118.29205404,342.09726336)
\curveto(118.33205351,342.26725865)(118.36705347,342.43225848)(118.39705404,342.59226336)
\curveto(118.42705341,342.75225816)(118.47205337,342.90225801)(118.53205404,343.04226336)
\curveto(118.96205288,344.08225683)(119.69205215,344.8172561)(120.72205404,345.24726336)
\curveto(120.86205098,345.30725561)(121.00205084,345.34725557)(121.14205404,345.36726336)
\curveto(121.29205055,345.39725552)(121.44705039,345.43225548)(121.60705404,345.47226336)
\curveto(121.68705015,345.48225543)(121.76205008,345.48725543)(121.83205404,345.48726336)
\curveto(121.90204994,345.48725543)(121.97704986,345.49225542)(122.05705404,345.50226336)
\curveto(122.56704927,345.5122554)(123.00204884,345.45225546)(123.36205404,345.32226336)
\curveto(123.73204811,345.20225571)(124.06204778,345.04225587)(124.35205404,344.84226336)
\curveto(124.4420474,344.78225613)(124.53204731,344.7122562)(124.62205404,344.63226336)
\curveto(124.71204713,344.56225635)(124.79204705,344.48725643)(124.86205404,344.40726336)
\curveto(124.89204695,344.35725656)(124.93204691,344.3172566)(124.98205404,344.28726336)
\curveto(125.06204678,344.17725674)(125.1370467,344.06225685)(125.20705404,343.94226336)
\curveto(125.27704656,343.83225708)(125.35204649,343.7172572)(125.43205404,343.59726336)
\curveto(125.48204636,343.50725741)(125.52204632,343.4122575)(125.55205404,343.31226336)
\curveto(125.59204625,343.22225769)(125.63204621,343.12225779)(125.67205404,343.01226336)
\curveto(125.72204612,342.88225803)(125.76204608,342.74725817)(125.79205404,342.60726336)
\curveto(125.82204602,342.46725845)(125.85704598,342.32725859)(125.89705404,342.18726336)
\curveto(125.91704592,342.10725881)(125.92204592,342.0172589)(125.91205404,341.91726336)
\curveto(125.91204593,341.82725909)(125.92204592,341.74225917)(125.94205404,341.66226336)
\lineto(125.94205404,341.49726336)
\moveto(123.69205404,342.38226336)
\curveto(123.76204808,342.48225843)(123.76704807,342.60225831)(123.70705404,342.74226336)
\curveto(123.65704818,342.89225802)(123.61704822,343.00225791)(123.58705404,343.07226336)
\curveto(123.44704839,343.34225757)(123.26204858,343.54725737)(123.03205404,343.68726336)
\curveto(122.80204904,343.83725708)(122.48204936,343.917257)(122.07205404,343.92726336)
\curveto(122.0420498,343.90725701)(122.00704983,343.90225701)(121.96705404,343.91226336)
\curveto(121.92704991,343.92225699)(121.89204995,343.92225699)(121.86205404,343.91226336)
\curveto(121.81205003,343.89225702)(121.75705008,343.87725704)(121.69705404,343.86726336)
\curveto(121.6370502,343.86725705)(121.58205026,343.85725706)(121.53205404,343.83726336)
\curveto(121.09205075,343.69725722)(120.76705107,343.42225749)(120.55705404,343.01226336)
\curveto(120.5370513,342.97225794)(120.51205133,342.917258)(120.48205404,342.84726336)
\curveto(120.46205138,342.78725813)(120.44705139,342.72225819)(120.43705404,342.65226336)
\curveto(120.42705141,342.59225832)(120.42705141,342.53225838)(120.43705404,342.47226336)
\curveto(120.45705138,342.4122585)(120.49205135,342.36225855)(120.54205404,342.32226336)
\curveto(120.62205122,342.27225864)(120.73205111,342.24725867)(120.87205404,342.24726336)
\lineto(121.27705404,342.24726336)
\lineto(122.94205404,342.24726336)
\lineto(123.37705404,342.24726336)
\curveto(123.5370483,342.25725866)(123.6420482,342.30225861)(123.69205404,342.38226336)
}
}
{
\newrgbcolor{curcolor}{0 0 0}
\pscustom[linestyle=none,fillstyle=solid,fillcolor=curcolor]
{
}
}
{
\newrgbcolor{curcolor}{0 0 0}
\pscustom[linestyle=none,fillstyle=solid,fillcolor=curcolor]
{
\newpath
\moveto(131.86049154,348.24726336)
\lineto(132.95549154,348.24726336)
\curveto(133.05548906,348.24725267)(133.15048896,348.24225267)(133.24049154,348.23226336)
\curveto(133.33048878,348.22225269)(133.40048871,348.19225272)(133.45049154,348.14226336)
\curveto(133.5104886,348.07225284)(133.54048857,347.97725294)(133.54049154,347.85726336)
\curveto(133.55048856,347.74725317)(133.55548856,347.63225328)(133.55549154,347.51226336)
\lineto(133.55549154,346.17726336)
\lineto(133.55549154,340.79226336)
\lineto(133.55549154,338.49726336)
\lineto(133.55549154,338.07726336)
\curveto(133.56548855,337.92726299)(133.54548857,337.8122631)(133.49549154,337.73226336)
\curveto(133.44548867,337.65226326)(133.35548876,337.59726332)(133.22549154,337.56726336)
\curveto(133.16548895,337.54726337)(133.09548902,337.54226337)(133.01549154,337.55226336)
\curveto(132.94548917,337.56226335)(132.87548924,337.56726335)(132.80549154,337.56726336)
\lineto(132.08549154,337.56726336)
\curveto(131.97549014,337.56726335)(131.87549024,337.57226334)(131.78549154,337.58226336)
\curveto(131.69549042,337.59226332)(131.62049049,337.62226329)(131.56049154,337.67226336)
\curveto(131.50049061,337.72226319)(131.46549065,337.79726312)(131.45549154,337.89726336)
\lineto(131.45549154,338.22726336)
\lineto(131.45549154,339.56226336)
\lineto(131.45549154,345.18726336)
\lineto(131.45549154,347.22726336)
\curveto(131.45549066,347.35725356)(131.45049066,347.5122534)(131.44049154,347.69226336)
\curveto(131.44049067,347.87225304)(131.46549065,348.00225291)(131.51549154,348.08226336)
\curveto(131.53549058,348.12225279)(131.56049055,348.15225276)(131.59049154,348.17226336)
\lineto(131.71049154,348.23226336)
\curveto(131.73049038,348.23225268)(131.75549036,348.23225268)(131.78549154,348.23226336)
\curveto(131.8154903,348.24225267)(131.84049027,348.24725267)(131.86049154,348.24726336)
}
}
{
\newrgbcolor{curcolor}{0 0 0}
\pscustom[linestyle=none,fillstyle=solid,fillcolor=curcolor]
{
\newpath
\moveto(142.98767904,341.73726336)
\curveto(143.00767047,341.67725924)(143.01767046,341.59225932)(143.01767904,341.48226336)
\curveto(143.01767046,341.37225954)(143.00767047,341.28725963)(142.98767904,341.22726336)
\lineto(142.98767904,341.07726336)
\curveto(142.96767051,340.99725992)(142.95767052,340.91726)(142.95767904,340.83726336)
\curveto(142.96767051,340.75726016)(142.96267052,340.67726024)(142.94267904,340.59726336)
\curveto(142.92267056,340.52726039)(142.90767057,340.46226045)(142.89767904,340.40226336)
\curveto(142.88767059,340.34226057)(142.8776706,340.27726064)(142.86767904,340.20726336)
\curveto(142.82767065,340.09726082)(142.79267069,339.98226093)(142.76267904,339.86226336)
\curveto(142.73267075,339.75226116)(142.69267079,339.64726127)(142.64267904,339.54726336)
\curveto(142.43267105,339.06726185)(142.15767132,338.67726224)(141.81767904,338.37726336)
\curveto(141.477672,338.07726284)(141.06767241,337.82726309)(140.58767904,337.62726336)
\curveto(140.46767301,337.57726334)(140.34267314,337.54226337)(140.21267904,337.52226336)
\curveto(140.09267339,337.49226342)(139.96767351,337.46226345)(139.83767904,337.43226336)
\curveto(139.78767369,337.4122635)(139.73267375,337.40226351)(139.67267904,337.40226336)
\curveto(139.61267387,337.40226351)(139.55767392,337.39726352)(139.50767904,337.38726336)
\lineto(139.40267904,337.38726336)
\curveto(139.37267411,337.37726354)(139.34267414,337.37226354)(139.31267904,337.37226336)
\curveto(139.26267422,337.36226355)(139.1826743,337.35726356)(139.07267904,337.35726336)
\curveto(138.96267452,337.34726357)(138.8776746,337.35226356)(138.81767904,337.37226336)
\lineto(138.66767904,337.37226336)
\curveto(138.61767486,337.38226353)(138.56267492,337.38726353)(138.50267904,337.38726336)
\curveto(138.45267503,337.37726354)(138.40267508,337.38226353)(138.35267904,337.40226336)
\curveto(138.31267517,337.4122635)(138.27267521,337.4172635)(138.23267904,337.41726336)
\curveto(138.20267528,337.4172635)(138.16267532,337.42226349)(138.11267904,337.43226336)
\curveto(138.01267547,337.46226345)(137.91267557,337.48726343)(137.81267904,337.50726336)
\curveto(137.71267577,337.52726339)(137.61767586,337.55726336)(137.52767904,337.59726336)
\curveto(137.40767607,337.63726328)(137.29267619,337.67726324)(137.18267904,337.71726336)
\curveto(137.0826764,337.75726316)(136.9776765,337.80726311)(136.86767904,337.86726336)
\curveto(136.51767696,338.07726284)(136.21767726,338.32226259)(135.96767904,338.60226336)
\curveto(135.71767776,338.88226203)(135.50767797,339.2172617)(135.33767904,339.60726336)
\curveto(135.28767819,339.69726122)(135.24767823,339.79226112)(135.21767904,339.89226336)
\curveto(135.19767828,339.99226092)(135.17267831,340.09726082)(135.14267904,340.20726336)
\curveto(135.12267836,340.25726066)(135.11267837,340.30226061)(135.11267904,340.34226336)
\curveto(135.11267837,340.38226053)(135.10267838,340.42726049)(135.08267904,340.47726336)
\curveto(135.06267842,340.55726036)(135.05267843,340.63726028)(135.05267904,340.71726336)
\curveto(135.05267843,340.80726011)(135.04267844,340.89226002)(135.02267904,340.97226336)
\curveto(135.01267847,341.02225989)(135.00767847,341.06725985)(135.00767904,341.10726336)
\lineto(135.00767904,341.24226336)
\curveto(134.98767849,341.30225961)(134.9776785,341.38725953)(134.97767904,341.49726336)
\curveto(134.98767849,341.60725931)(135.00267848,341.69225922)(135.02267904,341.75226336)
\lineto(135.02267904,341.85726336)
\curveto(135.03267845,341.90725901)(135.03267845,341.95725896)(135.02267904,342.00726336)
\curveto(135.02267846,342.06725885)(135.03267845,342.12225879)(135.05267904,342.17226336)
\curveto(135.06267842,342.22225869)(135.06767841,342.26725865)(135.06767904,342.30726336)
\curveto(135.06767841,342.35725856)(135.0776784,342.40725851)(135.09767904,342.45726336)
\curveto(135.13767834,342.58725833)(135.17267831,342.7122582)(135.20267904,342.83226336)
\curveto(135.23267825,342.96225795)(135.27267821,343.08725783)(135.32267904,343.20726336)
\curveto(135.50267798,343.6172573)(135.71767776,343.95725696)(135.96767904,344.22726336)
\curveto(136.21767726,344.50725641)(136.52267696,344.76225615)(136.88267904,344.99226336)
\curveto(136.9826765,345.04225587)(137.08767639,345.08725583)(137.19767904,345.12726336)
\curveto(137.30767617,345.16725575)(137.41767606,345.2122557)(137.52767904,345.26226336)
\curveto(137.65767582,345.3122556)(137.79267569,345.34725557)(137.93267904,345.36726336)
\curveto(138.07267541,345.38725553)(138.21767526,345.4172555)(138.36767904,345.45726336)
\curveto(138.44767503,345.46725545)(138.52267496,345.47225544)(138.59267904,345.47226336)
\curveto(138.66267482,345.47225544)(138.73267475,345.47725544)(138.80267904,345.48726336)
\curveto(139.3826741,345.49725542)(139.8826736,345.43725548)(140.30267904,345.30726336)
\curveto(140.73267275,345.17725574)(141.11267237,344.99725592)(141.44267904,344.76726336)
\curveto(141.55267193,344.68725623)(141.66267182,344.59725632)(141.77267904,344.49726336)
\curveto(141.89267159,344.40725651)(141.99267149,344.30725661)(142.07267904,344.19726336)
\curveto(142.15267133,344.09725682)(142.22267126,343.99725692)(142.28267904,343.89726336)
\curveto(142.35267113,343.79725712)(142.42267106,343.69225722)(142.49267904,343.58226336)
\curveto(142.56267092,343.47225744)(142.61767086,343.35225756)(142.65767904,343.22226336)
\curveto(142.69767078,343.10225781)(142.74267074,342.97225794)(142.79267904,342.83226336)
\curveto(142.82267066,342.75225816)(142.84767063,342.66725825)(142.86767904,342.57726336)
\lineto(142.92767904,342.30726336)
\curveto(142.93767054,342.26725865)(142.94267054,342.22725869)(142.94267904,342.18726336)
\curveto(142.94267054,342.14725877)(142.94767053,342.10725881)(142.95767904,342.06726336)
\curveto(142.9776705,342.0172589)(142.9826705,341.96225895)(142.97267904,341.90226336)
\curveto(142.96267052,341.84225907)(142.96767051,341.78725913)(142.98767904,341.73726336)
\moveto(140.88767904,341.19726336)
\curveto(140.89767258,341.24725967)(140.90267258,341.3172596)(140.90267904,341.40726336)
\curveto(140.90267258,341.50725941)(140.89767258,341.58225933)(140.88767904,341.63226336)
\lineto(140.88767904,341.75226336)
\curveto(140.86767261,341.80225911)(140.85767262,341.85725906)(140.85767904,341.91726336)
\curveto(140.85767262,341.97725894)(140.85267263,342.03225888)(140.84267904,342.08226336)
\curveto(140.84267264,342.12225879)(140.83767264,342.15225876)(140.82767904,342.17226336)
\lineto(140.76767904,342.41226336)
\curveto(140.75767272,342.50225841)(140.73767274,342.58725833)(140.70767904,342.66726336)
\curveto(140.59767288,342.92725799)(140.46767301,343.14725777)(140.31767904,343.32726336)
\curveto(140.16767331,343.5172574)(139.96767351,343.66725725)(139.71767904,343.77726336)
\curveto(139.65767382,343.79725712)(139.59767388,343.8122571)(139.53767904,343.82226336)
\curveto(139.477674,343.84225707)(139.41267407,343.86225705)(139.34267904,343.88226336)
\curveto(139.26267422,343.90225701)(139.1776743,343.90725701)(139.08767904,343.89726336)
\lineto(138.81767904,343.89726336)
\curveto(138.78767469,343.87725704)(138.75267473,343.86725705)(138.71267904,343.86726336)
\curveto(138.67267481,343.87725704)(138.63767484,343.87725704)(138.60767904,343.86726336)
\lineto(138.39767904,343.80726336)
\curveto(138.33767514,343.79725712)(138.2826752,343.77725714)(138.23267904,343.74726336)
\curveto(137.9826755,343.63725728)(137.7776757,343.47725744)(137.61767904,343.26726336)
\curveto(137.46767601,343.06725785)(137.34767613,342.83225808)(137.25767904,342.56226336)
\curveto(137.22767625,342.46225845)(137.20267628,342.35725856)(137.18267904,342.24726336)
\curveto(137.17267631,342.13725878)(137.15767632,342.02725889)(137.13767904,341.91726336)
\curveto(137.12767635,341.86725905)(137.12267636,341.8172591)(137.12267904,341.76726336)
\lineto(137.12267904,341.61726336)
\curveto(137.10267638,341.54725937)(137.09267639,341.44225947)(137.09267904,341.30226336)
\curveto(137.10267638,341.16225975)(137.11767636,341.05725986)(137.13767904,340.98726336)
\lineto(137.13767904,340.85226336)
\curveto(137.15767632,340.77226014)(137.17267631,340.69226022)(137.18267904,340.61226336)
\curveto(137.19267629,340.54226037)(137.20767627,340.46726045)(137.22767904,340.38726336)
\curveto(137.32767615,340.08726083)(137.43267605,339.84226107)(137.54267904,339.65226336)
\curveto(137.66267582,339.47226144)(137.84767563,339.30726161)(138.09767904,339.15726336)
\curveto(138.16767531,339.10726181)(138.24267524,339.06726185)(138.32267904,339.03726336)
\curveto(138.41267507,339.00726191)(138.50267498,338.98226193)(138.59267904,338.96226336)
\curveto(138.63267485,338.95226196)(138.66767481,338.94726197)(138.69767904,338.94726336)
\curveto(138.72767475,338.95726196)(138.76267472,338.95726196)(138.80267904,338.94726336)
\lineto(138.92267904,338.91726336)
\curveto(138.97267451,338.917262)(139.01767446,338.92226199)(139.05767904,338.93226336)
\lineto(139.17767904,338.93226336)
\curveto(139.25767422,338.95226196)(139.33767414,338.96726195)(139.41767904,338.97726336)
\curveto(139.49767398,338.98726193)(139.57267391,339.00726191)(139.64267904,339.03726336)
\curveto(139.90267358,339.13726178)(140.11267337,339.27226164)(140.27267904,339.44226336)
\curveto(140.43267305,339.6122613)(140.56767291,339.82226109)(140.67767904,340.07226336)
\curveto(140.71767276,340.17226074)(140.74767273,340.27226064)(140.76767904,340.37226336)
\curveto(140.78767269,340.47226044)(140.81267267,340.57726034)(140.84267904,340.68726336)
\curveto(140.85267263,340.72726019)(140.85767262,340.76226015)(140.85767904,340.79226336)
\curveto(140.85767262,340.83226008)(140.86267262,340.87226004)(140.87267904,340.91226336)
\lineto(140.87267904,341.04726336)
\curveto(140.87267261,341.09725982)(140.8776726,341.14725977)(140.88767904,341.19726336)
}
}
{
\newrgbcolor{curcolor}{0 0 0}
\pscustom[linestyle=none,fillstyle=solid,fillcolor=curcolor]
{
\newpath
\moveto(147.35760092,345.50226336)
\curveto(148.10759642,345.52225539)(148.75759577,345.43725548)(149.30760092,345.24726336)
\curveto(149.86759466,345.06725585)(150.29259423,344.75225616)(150.58260092,344.30226336)
\curveto(150.65259387,344.19225672)(150.71259381,344.07725684)(150.76260092,343.95726336)
\curveto(150.8225937,343.84725707)(150.87259365,343.72225719)(150.91260092,343.58226336)
\curveto(150.93259359,343.52225739)(150.94259358,343.45725746)(150.94260092,343.38726336)
\curveto(150.94259358,343.3172576)(150.93259359,343.25725766)(150.91260092,343.20726336)
\curveto(150.87259365,343.14725777)(150.81759371,343.10725781)(150.74760092,343.08726336)
\curveto(150.69759383,343.06725785)(150.63759389,343.05725786)(150.56760092,343.05726336)
\lineto(150.35760092,343.05726336)
\lineto(149.69760092,343.05726336)
\curveto(149.6275949,343.05725786)(149.55759497,343.05225786)(149.48760092,343.04226336)
\curveto(149.41759511,343.04225787)(149.35259517,343.05225786)(149.29260092,343.07226336)
\curveto(149.19259533,343.09225782)(149.11759541,343.13225778)(149.06760092,343.19226336)
\curveto(149.01759551,343.25225766)(148.97259555,343.3122576)(148.93260092,343.37226336)
\lineto(148.81260092,343.58226336)
\curveto(148.78259574,343.66225725)(148.73259579,343.72725719)(148.66260092,343.77726336)
\curveto(148.56259596,343.85725706)(148.46259606,343.917257)(148.36260092,343.95726336)
\curveto(148.27259625,343.99725692)(148.15759637,344.03225688)(148.01760092,344.06226336)
\curveto(147.94759658,344.08225683)(147.84259668,344.09725682)(147.70260092,344.10726336)
\curveto(147.57259695,344.1172568)(147.47259705,344.1122568)(147.40260092,344.09226336)
\lineto(147.29760092,344.09226336)
\lineto(147.14760092,344.06226336)
\curveto(147.10759742,344.06225685)(147.06259746,344.05725686)(147.01260092,344.04726336)
\curveto(146.84259768,343.99725692)(146.70259782,343.92725699)(146.59260092,343.83726336)
\curveto(146.49259803,343.75725716)(146.4225981,343.63225728)(146.38260092,343.46226336)
\curveto(146.36259816,343.39225752)(146.36259816,343.32725759)(146.38260092,343.26726336)
\curveto(146.40259812,343.20725771)(146.4225981,343.15725776)(146.44260092,343.11726336)
\curveto(146.51259801,342.99725792)(146.59259793,342.90225801)(146.68260092,342.83226336)
\curveto(146.78259774,342.76225815)(146.89759763,342.70225821)(147.02760092,342.65226336)
\curveto(147.21759731,342.57225834)(147.4225971,342.50225841)(147.64260092,342.44226336)
\lineto(148.33260092,342.29226336)
\curveto(148.57259595,342.25225866)(148.80259572,342.20225871)(149.02260092,342.14226336)
\curveto(149.25259527,342.09225882)(149.46759506,342.02725889)(149.66760092,341.94726336)
\curveto(149.75759477,341.90725901)(149.84259468,341.87225904)(149.92260092,341.84226336)
\curveto(150.01259451,341.82225909)(150.09759443,341.78725913)(150.17760092,341.73726336)
\curveto(150.36759416,341.6172593)(150.53759399,341.48725943)(150.68760092,341.34726336)
\curveto(150.84759368,341.20725971)(150.97259355,341.03225988)(151.06260092,340.82226336)
\curveto(151.09259343,340.75226016)(151.11759341,340.68226023)(151.13760092,340.61226336)
\curveto(151.15759337,340.54226037)(151.17759335,340.46726045)(151.19760092,340.38726336)
\curveto(151.20759332,340.32726059)(151.21259331,340.23226068)(151.21260092,340.10226336)
\curveto(151.2225933,339.98226093)(151.2225933,339.88726103)(151.21260092,339.81726336)
\lineto(151.21260092,339.74226336)
\curveto(151.19259333,339.68226123)(151.17759335,339.62226129)(151.16760092,339.56226336)
\curveto(151.16759336,339.5122614)(151.16259336,339.46226145)(151.15260092,339.41226336)
\curveto(151.08259344,339.1122618)(150.97259355,338.84726207)(150.82260092,338.61726336)
\curveto(150.66259386,338.37726254)(150.46759406,338.18226273)(150.23760092,338.03226336)
\curveto(150.00759452,337.88226303)(149.74759478,337.75226316)(149.45760092,337.64226336)
\curveto(149.34759518,337.59226332)(149.2275953,337.55726336)(149.09760092,337.53726336)
\curveto(148.97759555,337.5172634)(148.85759567,337.49226342)(148.73760092,337.46226336)
\curveto(148.64759588,337.44226347)(148.55259597,337.43226348)(148.45260092,337.43226336)
\curveto(148.36259616,337.42226349)(148.27259625,337.40726351)(148.18260092,337.38726336)
\lineto(147.91260092,337.38726336)
\curveto(147.85259667,337.36726355)(147.74759678,337.35726356)(147.59760092,337.35726336)
\curveto(147.45759707,337.35726356)(147.35759717,337.36726355)(147.29760092,337.38726336)
\curveto(147.26759726,337.38726353)(147.23259729,337.39226352)(147.19260092,337.40226336)
\lineto(147.08760092,337.40226336)
\curveto(146.96759756,337.42226349)(146.84759768,337.43726348)(146.72760092,337.44726336)
\curveto(146.60759792,337.45726346)(146.49259803,337.47726344)(146.38260092,337.50726336)
\curveto(145.99259853,337.6172633)(145.64759888,337.74226317)(145.34760092,337.88226336)
\curveto(145.04759948,338.03226288)(144.79259973,338.25226266)(144.58260092,338.54226336)
\curveto(144.44260008,338.73226218)(144.3226002,338.95226196)(144.22260092,339.20226336)
\curveto(144.20260032,339.26226165)(144.18260034,339.34226157)(144.16260092,339.44226336)
\curveto(144.14260038,339.49226142)(144.1276004,339.56226135)(144.11760092,339.65226336)
\curveto(144.10760042,339.74226117)(144.11260041,339.8172611)(144.13260092,339.87726336)
\curveto(144.16260036,339.94726097)(144.21260031,339.99726092)(144.28260092,340.02726336)
\curveto(144.33260019,340.04726087)(144.39260013,340.05726086)(144.46260092,340.05726336)
\lineto(144.68760092,340.05726336)
\lineto(145.39260092,340.05726336)
\lineto(145.63260092,340.05726336)
\curveto(145.71259881,340.05726086)(145.78259874,340.04726087)(145.84260092,340.02726336)
\curveto(145.95259857,339.98726093)(146.0225985,339.92226099)(146.05260092,339.83226336)
\curveto(146.09259843,339.74226117)(146.13759839,339.64726127)(146.18760092,339.54726336)
\curveto(146.20759832,339.49726142)(146.24259828,339.43226148)(146.29260092,339.35226336)
\curveto(146.35259817,339.27226164)(146.40259812,339.22226169)(146.44260092,339.20226336)
\curveto(146.56259796,339.10226181)(146.67759785,339.02226189)(146.78760092,338.96226336)
\curveto(146.89759763,338.912262)(147.03759749,338.86226205)(147.20760092,338.81226336)
\curveto(147.25759727,338.79226212)(147.30759722,338.78226213)(147.35760092,338.78226336)
\curveto(147.40759712,338.79226212)(147.45759707,338.79226212)(147.50760092,338.78226336)
\curveto(147.58759694,338.76226215)(147.67259685,338.75226216)(147.76260092,338.75226336)
\curveto(147.86259666,338.76226215)(147.94759658,338.77726214)(148.01760092,338.79726336)
\curveto(148.06759646,338.80726211)(148.11259641,338.8122621)(148.15260092,338.81226336)
\curveto(148.20259632,338.8122621)(148.25259627,338.82226209)(148.30260092,338.84226336)
\curveto(148.44259608,338.89226202)(148.56759596,338.95226196)(148.67760092,339.02226336)
\curveto(148.79759573,339.09226182)(148.89259563,339.18226173)(148.96260092,339.29226336)
\curveto(149.01259551,339.37226154)(149.05259547,339.49726142)(149.08260092,339.66726336)
\curveto(149.10259542,339.73726118)(149.10259542,339.80226111)(149.08260092,339.86226336)
\curveto(149.06259546,339.92226099)(149.04259548,339.97226094)(149.02260092,340.01226336)
\curveto(148.95259557,340.15226076)(148.86259566,340.25726066)(148.75260092,340.32726336)
\curveto(148.65259587,340.39726052)(148.53259599,340.46226045)(148.39260092,340.52226336)
\curveto(148.20259632,340.60226031)(148.00259652,340.66726025)(147.79260092,340.71726336)
\curveto(147.58259694,340.76726015)(147.37259715,340.82226009)(147.16260092,340.88226336)
\curveto(147.08259744,340.90226001)(146.99759753,340.91726)(146.90760092,340.92726336)
\curveto(146.8275977,340.93725998)(146.74759778,340.95225996)(146.66760092,340.97226336)
\curveto(146.34759818,341.06225985)(146.04259848,341.14725977)(145.75260092,341.22726336)
\curveto(145.46259906,341.3172596)(145.19759933,341.44725947)(144.95760092,341.61726336)
\curveto(144.67759985,341.8172591)(144.47260005,342.08725883)(144.34260092,342.42726336)
\curveto(144.3226002,342.49725842)(144.30260022,342.59225832)(144.28260092,342.71226336)
\curveto(144.26260026,342.78225813)(144.24760028,342.86725805)(144.23760092,342.96726336)
\curveto(144.2276003,343.06725785)(144.23260029,343.15725776)(144.25260092,343.23726336)
\curveto(144.27260025,343.28725763)(144.27760025,343.32725759)(144.26760092,343.35726336)
\curveto(144.25760027,343.39725752)(144.26260026,343.44225747)(144.28260092,343.49226336)
\curveto(144.30260022,343.60225731)(144.3226002,343.70225721)(144.34260092,343.79226336)
\curveto(144.37260015,343.89225702)(144.40760012,343.98725693)(144.44760092,344.07726336)
\curveto(144.57759995,344.36725655)(144.75759977,344.60225631)(144.98760092,344.78226336)
\curveto(145.21759931,344.96225595)(145.47759905,345.10725581)(145.76760092,345.21726336)
\curveto(145.87759865,345.26725565)(145.99259853,345.30225561)(146.11260092,345.32226336)
\curveto(146.23259829,345.35225556)(146.35759817,345.38225553)(146.48760092,345.41226336)
\curveto(146.54759798,345.43225548)(146.60759792,345.44225547)(146.66760092,345.44226336)
\lineto(146.84760092,345.47226336)
\curveto(146.9275976,345.48225543)(147.01259751,345.48725543)(147.10260092,345.48726336)
\curveto(147.19259733,345.48725543)(147.27759725,345.49225542)(147.35760092,345.50226336)
}
}
{
\newrgbcolor{curcolor}{0 0 0}
\pscustom[linestyle=none,fillstyle=solid,fillcolor=curcolor]
{
}
}
{
\newrgbcolor{curcolor}{0 0 0}
\pscustom[linestyle=none,fillstyle=solid,fillcolor=curcolor]
{
\newpath
\moveto(160.17439779,345.50226336)
\curveto(160.98439263,345.52225539)(161.65939196,345.40225551)(162.19939779,345.14226336)
\curveto(162.74939087,344.88225603)(163.18439043,344.5122564)(163.50439779,344.03226336)
\curveto(163.66438995,343.79225712)(163.78438983,343.5172574)(163.86439779,343.20726336)
\curveto(163.88438973,343.15725776)(163.89938972,343.09225782)(163.90939779,343.01226336)
\curveto(163.92938969,342.93225798)(163.92938969,342.86225805)(163.90939779,342.80226336)
\curveto(163.86938975,342.69225822)(163.79938982,342.62725829)(163.69939779,342.60726336)
\curveto(163.59939002,342.59725832)(163.47939014,342.59225832)(163.33939779,342.59226336)
\lineto(162.55939779,342.59226336)
\lineto(162.27439779,342.59226336)
\curveto(162.18439143,342.59225832)(162.10939151,342.6122583)(162.04939779,342.65226336)
\curveto(161.96939165,342.69225822)(161.9143917,342.75225816)(161.88439779,342.83226336)
\curveto(161.85439176,342.92225799)(161.8143918,343.0122579)(161.76439779,343.10226336)
\curveto(161.70439191,343.2122577)(161.63939198,343.3122576)(161.56939779,343.40226336)
\curveto(161.49939212,343.49225742)(161.4193922,343.57225734)(161.32939779,343.64226336)
\curveto(161.18939243,343.73225718)(161.03439258,343.80225711)(160.86439779,343.85226336)
\curveto(160.80439281,343.87225704)(160.74439287,343.88225703)(160.68439779,343.88226336)
\curveto(160.62439299,343.88225703)(160.56939305,343.89225702)(160.51939779,343.91226336)
\lineto(160.36939779,343.91226336)
\curveto(160.16939345,343.912257)(160.00939361,343.89225702)(159.88939779,343.85226336)
\curveto(159.59939402,343.76225715)(159.36439425,343.62225729)(159.18439779,343.43226336)
\curveto(159.00439461,343.25225766)(158.85939476,343.03225788)(158.74939779,342.77226336)
\curveto(158.69939492,342.66225825)(158.65939496,342.54225837)(158.62939779,342.41226336)
\curveto(158.60939501,342.29225862)(158.58439503,342.16225875)(158.55439779,342.02226336)
\curveto(158.54439507,341.98225893)(158.53939508,341.94225897)(158.53939779,341.90226336)
\curveto(158.53939508,341.86225905)(158.53439508,341.82225909)(158.52439779,341.78226336)
\curveto(158.50439511,341.68225923)(158.49439512,341.54225937)(158.49439779,341.36226336)
\curveto(158.50439511,341.18225973)(158.5193951,341.04225987)(158.53939779,340.94226336)
\curveto(158.53939508,340.86226005)(158.54439507,340.80726011)(158.55439779,340.77726336)
\curveto(158.57439504,340.70726021)(158.58439503,340.63726028)(158.58439779,340.56726336)
\curveto(158.59439502,340.49726042)(158.60939501,340.42726049)(158.62939779,340.35726336)
\curveto(158.70939491,340.12726079)(158.80439481,339.917261)(158.91439779,339.72726336)
\curveto(159.02439459,339.53726138)(159.16439445,339.37726154)(159.33439779,339.24726336)
\curveto(159.37439424,339.2172617)(159.43439418,339.18226173)(159.51439779,339.14226336)
\curveto(159.62439399,339.07226184)(159.73439388,339.02726189)(159.84439779,339.00726336)
\curveto(159.96439365,338.98726193)(160.10939351,338.96726195)(160.27939779,338.94726336)
\lineto(160.36939779,338.94726336)
\curveto(160.40939321,338.94726197)(160.43939318,338.95226196)(160.45939779,338.96226336)
\lineto(160.59439779,338.96226336)
\curveto(160.66439295,338.98226193)(160.72939289,338.99726192)(160.78939779,339.00726336)
\curveto(160.85939276,339.02726189)(160.92439269,339.04726187)(160.98439779,339.06726336)
\curveto(161.28439233,339.19726172)(161.5143921,339.38726153)(161.67439779,339.63726336)
\curveto(161.7143919,339.68726123)(161.74939187,339.74226117)(161.77939779,339.80226336)
\curveto(161.80939181,339.87226104)(161.83439178,339.93226098)(161.85439779,339.98226336)
\curveto(161.89439172,340.09226082)(161.92939169,340.18726073)(161.95939779,340.26726336)
\curveto(161.98939163,340.35726056)(162.05939156,340.42726049)(162.16939779,340.47726336)
\curveto(162.25939136,340.5172604)(162.40439121,340.53226038)(162.60439779,340.52226336)
\lineto(163.09939779,340.52226336)
\lineto(163.30939779,340.52226336)
\curveto(163.38939023,340.53226038)(163.45439016,340.52726039)(163.50439779,340.50726336)
\lineto(163.62439779,340.50726336)
\lineto(163.74439779,340.47726336)
\curveto(163.78438983,340.47726044)(163.8143898,340.46726045)(163.83439779,340.44726336)
\curveto(163.88438973,340.40726051)(163.9143897,340.34726057)(163.92439779,340.26726336)
\curveto(163.94438967,340.19726072)(163.94438967,340.12226079)(163.92439779,340.04226336)
\curveto(163.83438978,339.7122612)(163.72438989,339.4172615)(163.59439779,339.15726336)
\curveto(163.18439043,338.38726253)(162.52939109,337.85226306)(161.62939779,337.55226336)
\curveto(161.52939209,337.52226339)(161.42439219,337.50226341)(161.31439779,337.49226336)
\curveto(161.20439241,337.47226344)(161.09439252,337.44726347)(160.98439779,337.41726336)
\curveto(160.92439269,337.40726351)(160.86439275,337.40226351)(160.80439779,337.40226336)
\curveto(160.74439287,337.40226351)(160.68439293,337.39726352)(160.62439779,337.38726336)
\lineto(160.45939779,337.38726336)
\curveto(160.40939321,337.36726355)(160.33439328,337.36226355)(160.23439779,337.37226336)
\curveto(160.13439348,337.37226354)(160.05939356,337.37726354)(160.00939779,337.38726336)
\curveto(159.92939369,337.40726351)(159.85439376,337.4172635)(159.78439779,337.41726336)
\curveto(159.72439389,337.40726351)(159.65939396,337.4122635)(159.58939779,337.43226336)
\lineto(159.43939779,337.46226336)
\curveto(159.38939423,337.46226345)(159.33939428,337.46726345)(159.28939779,337.47726336)
\curveto(159.17939444,337.50726341)(159.07439454,337.53726338)(158.97439779,337.56726336)
\curveto(158.87439474,337.59726332)(158.77939484,337.63226328)(158.68939779,337.67226336)
\curveto(158.2193954,337.87226304)(157.82439579,338.12726279)(157.50439779,338.43726336)
\curveto(157.18439643,338.75726216)(156.92439669,339.15226176)(156.72439779,339.62226336)
\curveto(156.67439694,339.7122612)(156.63439698,339.80726111)(156.60439779,339.90726336)
\lineto(156.51439779,340.23726336)
\curveto(156.50439711,340.27726064)(156.49939712,340.3122606)(156.49939779,340.34226336)
\curveto(156.49939712,340.38226053)(156.48939713,340.42726049)(156.46939779,340.47726336)
\curveto(156.44939717,340.54726037)(156.43939718,340.6172603)(156.43939779,340.68726336)
\curveto(156.43939718,340.76726015)(156.42939719,340.84226007)(156.40939779,340.91226336)
\lineto(156.40939779,341.16726336)
\curveto(156.38939723,341.2172597)(156.37939724,341.27225964)(156.37939779,341.33226336)
\curveto(156.37939724,341.40225951)(156.38939723,341.46225945)(156.40939779,341.51226336)
\curveto(156.4193972,341.56225935)(156.4193972,341.60725931)(156.40939779,341.64726336)
\curveto(156.39939722,341.68725923)(156.39939722,341.72725919)(156.40939779,341.76726336)
\curveto(156.42939719,341.83725908)(156.43439718,341.90225901)(156.42439779,341.96226336)
\curveto(156.42439719,342.02225889)(156.43439718,342.08225883)(156.45439779,342.14226336)
\curveto(156.50439711,342.32225859)(156.54439707,342.49225842)(156.57439779,342.65226336)
\curveto(156.60439701,342.82225809)(156.64939697,342.98725793)(156.70939779,343.14726336)
\curveto(156.92939669,343.65725726)(157.20439641,344.08225683)(157.53439779,344.42226336)
\curveto(157.87439574,344.76225615)(158.30439531,345.03725588)(158.82439779,345.24726336)
\curveto(158.96439465,345.30725561)(159.10939451,345.34725557)(159.25939779,345.36726336)
\curveto(159.40939421,345.39725552)(159.56439405,345.43225548)(159.72439779,345.47226336)
\curveto(159.80439381,345.48225543)(159.87939374,345.48725543)(159.94939779,345.48726336)
\curveto(160.0193936,345.48725543)(160.09439352,345.49225542)(160.17439779,345.50226336)
}
}
{
\newrgbcolor{curcolor}{0 0 0}
\pscustom[linestyle=none,fillstyle=solid,fillcolor=curcolor]
{
\newpath
\moveto(172.98767904,341.73726336)
\curveto(173.00767047,341.67725924)(173.01767046,341.59225932)(173.01767904,341.48226336)
\curveto(173.01767046,341.37225954)(173.00767047,341.28725963)(172.98767904,341.22726336)
\lineto(172.98767904,341.07726336)
\curveto(172.96767051,340.99725992)(172.95767052,340.91726)(172.95767904,340.83726336)
\curveto(172.96767051,340.75726016)(172.96267052,340.67726024)(172.94267904,340.59726336)
\curveto(172.92267056,340.52726039)(172.90767057,340.46226045)(172.89767904,340.40226336)
\curveto(172.88767059,340.34226057)(172.8776706,340.27726064)(172.86767904,340.20726336)
\curveto(172.82767065,340.09726082)(172.79267069,339.98226093)(172.76267904,339.86226336)
\curveto(172.73267075,339.75226116)(172.69267079,339.64726127)(172.64267904,339.54726336)
\curveto(172.43267105,339.06726185)(172.15767132,338.67726224)(171.81767904,338.37726336)
\curveto(171.477672,338.07726284)(171.06767241,337.82726309)(170.58767904,337.62726336)
\curveto(170.46767301,337.57726334)(170.34267314,337.54226337)(170.21267904,337.52226336)
\curveto(170.09267339,337.49226342)(169.96767351,337.46226345)(169.83767904,337.43226336)
\curveto(169.78767369,337.4122635)(169.73267375,337.40226351)(169.67267904,337.40226336)
\curveto(169.61267387,337.40226351)(169.55767392,337.39726352)(169.50767904,337.38726336)
\lineto(169.40267904,337.38726336)
\curveto(169.37267411,337.37726354)(169.34267414,337.37226354)(169.31267904,337.37226336)
\curveto(169.26267422,337.36226355)(169.1826743,337.35726356)(169.07267904,337.35726336)
\curveto(168.96267452,337.34726357)(168.8776746,337.35226356)(168.81767904,337.37226336)
\lineto(168.66767904,337.37226336)
\curveto(168.61767486,337.38226353)(168.56267492,337.38726353)(168.50267904,337.38726336)
\curveto(168.45267503,337.37726354)(168.40267508,337.38226353)(168.35267904,337.40226336)
\curveto(168.31267517,337.4122635)(168.27267521,337.4172635)(168.23267904,337.41726336)
\curveto(168.20267528,337.4172635)(168.16267532,337.42226349)(168.11267904,337.43226336)
\curveto(168.01267547,337.46226345)(167.91267557,337.48726343)(167.81267904,337.50726336)
\curveto(167.71267577,337.52726339)(167.61767586,337.55726336)(167.52767904,337.59726336)
\curveto(167.40767607,337.63726328)(167.29267619,337.67726324)(167.18267904,337.71726336)
\curveto(167.0826764,337.75726316)(166.9776765,337.80726311)(166.86767904,337.86726336)
\curveto(166.51767696,338.07726284)(166.21767726,338.32226259)(165.96767904,338.60226336)
\curveto(165.71767776,338.88226203)(165.50767797,339.2172617)(165.33767904,339.60726336)
\curveto(165.28767819,339.69726122)(165.24767823,339.79226112)(165.21767904,339.89226336)
\curveto(165.19767828,339.99226092)(165.17267831,340.09726082)(165.14267904,340.20726336)
\curveto(165.12267836,340.25726066)(165.11267837,340.30226061)(165.11267904,340.34226336)
\curveto(165.11267837,340.38226053)(165.10267838,340.42726049)(165.08267904,340.47726336)
\curveto(165.06267842,340.55726036)(165.05267843,340.63726028)(165.05267904,340.71726336)
\curveto(165.05267843,340.80726011)(165.04267844,340.89226002)(165.02267904,340.97226336)
\curveto(165.01267847,341.02225989)(165.00767847,341.06725985)(165.00767904,341.10726336)
\lineto(165.00767904,341.24226336)
\curveto(164.98767849,341.30225961)(164.9776785,341.38725953)(164.97767904,341.49726336)
\curveto(164.98767849,341.60725931)(165.00267848,341.69225922)(165.02267904,341.75226336)
\lineto(165.02267904,341.85726336)
\curveto(165.03267845,341.90725901)(165.03267845,341.95725896)(165.02267904,342.00726336)
\curveto(165.02267846,342.06725885)(165.03267845,342.12225879)(165.05267904,342.17226336)
\curveto(165.06267842,342.22225869)(165.06767841,342.26725865)(165.06767904,342.30726336)
\curveto(165.06767841,342.35725856)(165.0776784,342.40725851)(165.09767904,342.45726336)
\curveto(165.13767834,342.58725833)(165.17267831,342.7122582)(165.20267904,342.83226336)
\curveto(165.23267825,342.96225795)(165.27267821,343.08725783)(165.32267904,343.20726336)
\curveto(165.50267798,343.6172573)(165.71767776,343.95725696)(165.96767904,344.22726336)
\curveto(166.21767726,344.50725641)(166.52267696,344.76225615)(166.88267904,344.99226336)
\curveto(166.9826765,345.04225587)(167.08767639,345.08725583)(167.19767904,345.12726336)
\curveto(167.30767617,345.16725575)(167.41767606,345.2122557)(167.52767904,345.26226336)
\curveto(167.65767582,345.3122556)(167.79267569,345.34725557)(167.93267904,345.36726336)
\curveto(168.07267541,345.38725553)(168.21767526,345.4172555)(168.36767904,345.45726336)
\curveto(168.44767503,345.46725545)(168.52267496,345.47225544)(168.59267904,345.47226336)
\curveto(168.66267482,345.47225544)(168.73267475,345.47725544)(168.80267904,345.48726336)
\curveto(169.3826741,345.49725542)(169.8826736,345.43725548)(170.30267904,345.30726336)
\curveto(170.73267275,345.17725574)(171.11267237,344.99725592)(171.44267904,344.76726336)
\curveto(171.55267193,344.68725623)(171.66267182,344.59725632)(171.77267904,344.49726336)
\curveto(171.89267159,344.40725651)(171.99267149,344.30725661)(172.07267904,344.19726336)
\curveto(172.15267133,344.09725682)(172.22267126,343.99725692)(172.28267904,343.89726336)
\curveto(172.35267113,343.79725712)(172.42267106,343.69225722)(172.49267904,343.58226336)
\curveto(172.56267092,343.47225744)(172.61767086,343.35225756)(172.65767904,343.22226336)
\curveto(172.69767078,343.10225781)(172.74267074,342.97225794)(172.79267904,342.83226336)
\curveto(172.82267066,342.75225816)(172.84767063,342.66725825)(172.86767904,342.57726336)
\lineto(172.92767904,342.30726336)
\curveto(172.93767054,342.26725865)(172.94267054,342.22725869)(172.94267904,342.18726336)
\curveto(172.94267054,342.14725877)(172.94767053,342.10725881)(172.95767904,342.06726336)
\curveto(172.9776705,342.0172589)(172.9826705,341.96225895)(172.97267904,341.90226336)
\curveto(172.96267052,341.84225907)(172.96767051,341.78725913)(172.98767904,341.73726336)
\moveto(170.88767904,341.19726336)
\curveto(170.89767258,341.24725967)(170.90267258,341.3172596)(170.90267904,341.40726336)
\curveto(170.90267258,341.50725941)(170.89767258,341.58225933)(170.88767904,341.63226336)
\lineto(170.88767904,341.75226336)
\curveto(170.86767261,341.80225911)(170.85767262,341.85725906)(170.85767904,341.91726336)
\curveto(170.85767262,341.97725894)(170.85267263,342.03225888)(170.84267904,342.08226336)
\curveto(170.84267264,342.12225879)(170.83767264,342.15225876)(170.82767904,342.17226336)
\lineto(170.76767904,342.41226336)
\curveto(170.75767272,342.50225841)(170.73767274,342.58725833)(170.70767904,342.66726336)
\curveto(170.59767288,342.92725799)(170.46767301,343.14725777)(170.31767904,343.32726336)
\curveto(170.16767331,343.5172574)(169.96767351,343.66725725)(169.71767904,343.77726336)
\curveto(169.65767382,343.79725712)(169.59767388,343.8122571)(169.53767904,343.82226336)
\curveto(169.477674,343.84225707)(169.41267407,343.86225705)(169.34267904,343.88226336)
\curveto(169.26267422,343.90225701)(169.1776743,343.90725701)(169.08767904,343.89726336)
\lineto(168.81767904,343.89726336)
\curveto(168.78767469,343.87725704)(168.75267473,343.86725705)(168.71267904,343.86726336)
\curveto(168.67267481,343.87725704)(168.63767484,343.87725704)(168.60767904,343.86726336)
\lineto(168.39767904,343.80726336)
\curveto(168.33767514,343.79725712)(168.2826752,343.77725714)(168.23267904,343.74726336)
\curveto(167.9826755,343.63725728)(167.7776757,343.47725744)(167.61767904,343.26726336)
\curveto(167.46767601,343.06725785)(167.34767613,342.83225808)(167.25767904,342.56226336)
\curveto(167.22767625,342.46225845)(167.20267628,342.35725856)(167.18267904,342.24726336)
\curveto(167.17267631,342.13725878)(167.15767632,342.02725889)(167.13767904,341.91726336)
\curveto(167.12767635,341.86725905)(167.12267636,341.8172591)(167.12267904,341.76726336)
\lineto(167.12267904,341.61726336)
\curveto(167.10267638,341.54725937)(167.09267639,341.44225947)(167.09267904,341.30226336)
\curveto(167.10267638,341.16225975)(167.11767636,341.05725986)(167.13767904,340.98726336)
\lineto(167.13767904,340.85226336)
\curveto(167.15767632,340.77226014)(167.17267631,340.69226022)(167.18267904,340.61226336)
\curveto(167.19267629,340.54226037)(167.20767627,340.46726045)(167.22767904,340.38726336)
\curveto(167.32767615,340.08726083)(167.43267605,339.84226107)(167.54267904,339.65226336)
\curveto(167.66267582,339.47226144)(167.84767563,339.30726161)(168.09767904,339.15726336)
\curveto(168.16767531,339.10726181)(168.24267524,339.06726185)(168.32267904,339.03726336)
\curveto(168.41267507,339.00726191)(168.50267498,338.98226193)(168.59267904,338.96226336)
\curveto(168.63267485,338.95226196)(168.66767481,338.94726197)(168.69767904,338.94726336)
\curveto(168.72767475,338.95726196)(168.76267472,338.95726196)(168.80267904,338.94726336)
\lineto(168.92267904,338.91726336)
\curveto(168.97267451,338.917262)(169.01767446,338.92226199)(169.05767904,338.93226336)
\lineto(169.17767904,338.93226336)
\curveto(169.25767422,338.95226196)(169.33767414,338.96726195)(169.41767904,338.97726336)
\curveto(169.49767398,338.98726193)(169.57267391,339.00726191)(169.64267904,339.03726336)
\curveto(169.90267358,339.13726178)(170.11267337,339.27226164)(170.27267904,339.44226336)
\curveto(170.43267305,339.6122613)(170.56767291,339.82226109)(170.67767904,340.07226336)
\curveto(170.71767276,340.17226074)(170.74767273,340.27226064)(170.76767904,340.37226336)
\curveto(170.78767269,340.47226044)(170.81267267,340.57726034)(170.84267904,340.68726336)
\curveto(170.85267263,340.72726019)(170.85767262,340.76226015)(170.85767904,340.79226336)
\curveto(170.85767262,340.83226008)(170.86267262,340.87226004)(170.87267904,340.91226336)
\lineto(170.87267904,341.04726336)
\curveto(170.87267261,341.09725982)(170.8776726,341.14725977)(170.88767904,341.19726336)
}
}
{
\newrgbcolor{curcolor}{0 0 0}
\pscustom[linestyle=none,fillstyle=solid,fillcolor=curcolor]
{
\newpath
\moveto(178.84260092,345.48726336)
\curveto(179.21259531,345.49725542)(179.53759499,345.45725546)(179.81760092,345.36726336)
\curveto(180.09759443,345.27725564)(180.34259418,345.15225576)(180.55260092,344.99226336)
\curveto(180.63259389,344.93225598)(180.70259382,344.86225605)(180.76260092,344.78226336)
\curveto(180.83259369,344.70225621)(180.90759362,344.62225629)(180.98760092,344.54226336)
\curveto(181.00759352,344.52225639)(181.03759349,344.49225642)(181.07760092,344.45226336)
\curveto(181.1275934,344.42225649)(181.17759335,344.4172565)(181.22760092,344.43726336)
\curveto(181.33759319,344.46725645)(181.44259308,344.53725638)(181.54260092,344.64726336)
\curveto(181.64259288,344.76725615)(181.73759279,344.85725606)(181.82760092,344.91726336)
\curveto(181.96759256,345.02725589)(182.11759241,345.1172558)(182.27760092,345.18726336)
\curveto(182.43759209,345.26725565)(182.61759191,345.34225557)(182.81760092,345.41226336)
\curveto(182.89759163,345.43225548)(182.99259153,345.44725547)(183.10260092,345.45726336)
\curveto(183.2225913,345.47725544)(183.34259118,345.48725543)(183.46260092,345.48726336)
\curveto(183.59259093,345.49725542)(183.71259081,345.49725542)(183.82260092,345.48726336)
\curveto(183.94259058,345.47725544)(184.04759048,345.46225545)(184.13760092,345.44226336)
\curveto(184.18759034,345.43225548)(184.23259029,345.42725549)(184.27260092,345.42726336)
\curveto(184.31259021,345.42725549)(184.35759017,345.4172555)(184.40760092,345.39726336)
\curveto(184.54758998,345.35725556)(184.68258984,345.3172556)(184.81260092,345.27726336)
\curveto(184.94258958,345.23725568)(185.06258946,345.18225573)(185.17260092,345.11226336)
\curveto(185.59258893,344.85225606)(185.90758862,344.47225644)(186.11760092,343.97226336)
\curveto(186.15758837,343.88225703)(186.18758834,343.78725713)(186.20760092,343.68726336)
\curveto(186.2275883,343.59725732)(186.24758828,343.50725741)(186.26760092,343.41726336)
\curveto(186.27758825,343.34725757)(186.28258824,343.28225763)(186.28260092,343.22226336)
\curveto(186.29258823,343.16225775)(186.30258822,343.10225781)(186.31260092,343.04226336)
\lineto(186.31260092,342.89226336)
\curveto(186.3225882,342.83225808)(186.3225882,342.76225815)(186.31260092,342.68226336)
\curveto(186.31258821,342.60225831)(186.31258821,342.52725839)(186.31260092,342.45726336)
\lineto(186.31260092,341.58726336)
\lineto(186.31260092,338.66226336)
\curveto(186.31258821,338.58226233)(186.31258821,338.48726243)(186.31260092,338.37726336)
\curveto(186.3225882,338.27726264)(186.3225882,338.17726274)(186.31260092,338.07726336)
\curveto(186.31258821,337.98726293)(186.30258822,337.89726302)(186.28260092,337.80726336)
\curveto(186.26258826,337.72726319)(186.23258829,337.67226324)(186.19260092,337.64226336)
\curveto(186.13258839,337.59226332)(186.05258847,337.56226335)(185.95260092,337.55226336)
\lineto(185.65260092,337.55226336)
\lineto(184.85760092,337.55226336)
\curveto(184.71758981,337.55226336)(184.59258993,337.56226335)(184.48260092,337.58226336)
\curveto(184.37259015,337.60226331)(184.29759023,337.65726326)(184.25760092,337.74726336)
\curveto(184.2275903,337.8172631)(184.21259031,337.89226302)(184.21260092,337.97226336)
\curveto(184.21259031,338.06226285)(184.21259031,338.14726277)(184.21260092,338.22726336)
\lineto(184.21260092,339.06726336)
\lineto(184.21260092,341.09226336)
\lineto(184.21260092,341.72226336)
\curveto(184.21259031,341.77225914)(184.21259031,341.82725909)(184.21260092,341.88726336)
\curveto(184.2225903,341.94725897)(184.21759031,342.00225891)(184.19760092,342.05226336)
\lineto(184.19760092,342.17226336)
\curveto(184.19759033,342.23225868)(184.19759033,342.29225862)(184.19760092,342.35226336)
\curveto(184.19759033,342.4122585)(184.19259033,342.47225844)(184.18260092,342.53226336)
\curveto(184.17259035,342.57225834)(184.16759036,342.6122583)(184.16760092,342.65226336)
\curveto(184.16759036,342.70225821)(184.16259036,342.74725817)(184.15260092,342.78726336)
\curveto(184.11259041,342.93725798)(184.06759046,343.06725785)(184.01760092,343.17726336)
\curveto(183.97759055,343.29725762)(183.91259061,343.40225751)(183.82260092,343.49226336)
\curveto(183.68259084,343.63225728)(183.51259101,343.73225718)(183.31260092,343.79226336)
\curveto(183.27259125,343.80225711)(183.23759129,343.80225711)(183.20760092,343.79226336)
\curveto(183.17759135,343.79225712)(183.14259138,343.80225711)(183.10260092,343.82226336)
\curveto(183.06259146,343.83225708)(183.01259151,343.83725708)(182.95260092,343.83726336)
\curveto(182.90259162,343.84725707)(182.85259167,343.84725707)(182.80260092,343.83726336)
\curveto(182.74259178,343.8172571)(182.68259184,343.80725711)(182.62260092,343.80726336)
\curveto(182.56259196,343.80725711)(182.50259202,343.79725712)(182.44260092,343.77726336)
\curveto(182.15259237,343.67725724)(181.94259258,343.52725739)(181.81260092,343.32726336)
\curveto(181.64259288,343.09725782)(181.53759299,342.80725811)(181.49760092,342.45726336)
\curveto(181.46759306,342.1172588)(181.45259307,341.74225917)(181.45260092,341.33226336)
\lineto(181.45260092,339.35226336)
\lineto(181.45260092,338.24226336)
\lineto(181.45260092,337.94226336)
\curveto(181.45259307,337.84226307)(181.4275931,337.76226315)(181.37760092,337.70226336)
\curveto(181.3275932,337.63226328)(181.25259327,337.58726333)(181.15260092,337.56726336)
\curveto(181.06259346,337.55726336)(180.95759357,337.55226336)(180.83760092,337.55226336)
\lineto(180.02760092,337.55226336)
\lineto(179.75760092,337.55226336)
\curveto(179.67759485,337.56226335)(179.60759492,337.57726334)(179.54760092,337.59726336)
\curveto(179.44759508,337.64726327)(179.38759514,337.72726319)(179.36760092,337.83726336)
\curveto(179.35759517,337.94726297)(179.35259517,338.07226284)(179.35260092,338.21226336)
\lineto(179.35260092,339.48726336)
\lineto(179.35260092,341.84226336)
\curveto(179.35259517,342.13225878)(179.34259518,342.40725851)(179.32260092,342.66726336)
\curveto(179.30259522,342.92725799)(179.23759529,343.14225777)(179.12760092,343.31226336)
\curveto(179.04759548,343.45225746)(178.94259558,343.55725736)(178.81260092,343.62726336)
\curveto(178.69259583,343.69725722)(178.54259598,343.75725716)(178.36260092,343.80726336)
\curveto(178.3225962,343.8172571)(178.28259624,343.8172571)(178.24260092,343.80726336)
\curveto(178.20259632,343.80725711)(178.15759637,343.8122571)(178.10760092,343.82226336)
\curveto(177.99759653,343.84225707)(177.89259663,343.83225708)(177.79260092,343.79226336)
\curveto(177.77259675,343.79225712)(177.75259677,343.78725713)(177.73260092,343.77726336)
\lineto(177.67260092,343.77726336)
\curveto(177.51259701,343.72725719)(177.35759717,343.64225727)(177.20760092,343.52226336)
\curveto(177.04759748,343.40225751)(176.9225976,343.26225765)(176.83260092,343.10226336)
\curveto(176.75259777,342.95225796)(176.69259783,342.77725814)(176.65260092,342.57726336)
\curveto(176.6225979,342.38725853)(176.60259792,342.17725874)(176.59260092,341.94726336)
\lineto(176.59260092,341.19726336)
\lineto(176.59260092,339.17226336)
\lineto(176.59260092,338.25726336)
\lineto(176.59260092,337.98726336)
\curveto(176.59259793,337.89726302)(176.57759795,337.8172631)(176.54760092,337.74726336)
\curveto(176.50759802,337.65726326)(176.43259809,337.60226331)(176.32260092,337.58226336)
\curveto(176.21259831,337.56226335)(176.08759844,337.55226336)(175.94760092,337.55226336)
\lineto(175.16760092,337.55226336)
\lineto(174.86760092,337.55226336)
\curveto(174.77759975,337.56226335)(174.70259982,337.58726333)(174.64260092,337.62726336)
\curveto(174.55259997,337.67726324)(174.50260002,337.76726315)(174.49260092,337.89726336)
\lineto(174.49260092,338.33226336)
\lineto(174.49260092,340.08726336)
\lineto(174.49260092,343.74726336)
\lineto(174.49260092,344.64726336)
\lineto(174.49260092,344.93226336)
\curveto(174.50260002,345.02225589)(174.5276,345.09725582)(174.56760092,345.15726336)
\curveto(174.61759991,345.2172557)(174.69759983,345.25725566)(174.80760092,345.27726336)
\lineto(174.89760092,345.27726336)
\curveto(174.94759958,345.28725563)(174.99759953,345.29225562)(175.04760092,345.29226336)
\lineto(175.21260092,345.29226336)
\lineto(175.82760092,345.29226336)
\curveto(175.90759862,345.29225562)(175.98259854,345.28725563)(176.05260092,345.27726336)
\curveto(176.13259839,345.27725564)(176.20259832,345.26725565)(176.26260092,345.24726336)
\curveto(176.34259818,345.2172557)(176.39259813,345.16725575)(176.41260092,345.09726336)
\curveto(176.44259808,345.02725589)(176.46759806,344.94725597)(176.48760092,344.85726336)
\curveto(176.49759803,344.82725609)(176.49759803,344.79725612)(176.48760092,344.76726336)
\curveto(176.48759804,344.74725617)(176.49759803,344.72725619)(176.51760092,344.70726336)
\curveto(176.527598,344.67725624)(176.53759799,344.65225626)(176.54760092,344.63226336)
\curveto(176.56759796,344.62225629)(176.58759794,344.60725631)(176.60760092,344.58726336)
\curveto(176.7275978,344.57725634)(176.8275977,344.6122563)(176.90760092,344.69226336)
\curveto(176.98759754,344.78225613)(177.06259746,344.85225606)(177.13260092,344.90226336)
\curveto(177.27259725,345.00225591)(177.41259711,345.09225582)(177.55260092,345.17226336)
\curveto(177.70259682,345.25225566)(177.86259666,345.3172556)(178.03260092,345.36726336)
\curveto(178.1225964,345.39725552)(178.21259631,345.4172555)(178.30260092,345.42726336)
\curveto(178.39259613,345.43725548)(178.48759604,345.45225546)(178.58760092,345.47226336)
\curveto(178.61759591,345.48225543)(178.66259586,345.48225543)(178.72260092,345.47226336)
\curveto(178.78259574,345.47225544)(178.8225957,345.47725544)(178.84260092,345.48726336)
}
}
{
\newrgbcolor{curcolor}{0 0 0}
\pscustom[linestyle=none,fillstyle=solid,fillcolor=curcolor]
{
\newpath
\moveto(195.34635092,341.49726336)
\curveto(195.36634275,341.4172595)(195.36634275,341.32725959)(195.34635092,341.22726336)
\curveto(195.32634279,341.12725979)(195.29134283,341.06225985)(195.24135092,341.03226336)
\curveto(195.19134293,340.99225992)(195.116343,340.96225995)(195.01635092,340.94226336)
\curveto(194.92634319,340.93225998)(194.8213433,340.92225999)(194.70135092,340.91226336)
\lineto(194.35635092,340.91226336)
\curveto(194.24634387,340.92225999)(194.14634397,340.92725999)(194.05635092,340.92726336)
\lineto(190.39635092,340.92726336)
\lineto(190.18635092,340.92726336)
\curveto(190.12634799,340.92725999)(190.07134805,340.91726)(190.02135092,340.89726336)
\curveto(189.94134818,340.85726006)(189.89134823,340.8172601)(189.87135092,340.77726336)
\curveto(189.85134827,340.75726016)(189.83134829,340.7172602)(189.81135092,340.65726336)
\curveto(189.79134833,340.60726031)(189.78634833,340.55726036)(189.79635092,340.50726336)
\curveto(189.8163483,340.44726047)(189.82634829,340.38726053)(189.82635092,340.32726336)
\curveto(189.83634828,340.27726064)(189.85134827,340.22226069)(189.87135092,340.16226336)
\curveto(189.95134817,339.92226099)(190.04634807,339.72226119)(190.15635092,339.56226336)
\curveto(190.27634784,339.4122615)(190.43634768,339.27726164)(190.63635092,339.15726336)
\curveto(190.7163474,339.10726181)(190.79634732,339.07226184)(190.87635092,339.05226336)
\curveto(190.96634715,339.04226187)(191.05634706,339.02226189)(191.14635092,338.99226336)
\curveto(191.22634689,338.97226194)(191.33634678,338.95726196)(191.47635092,338.94726336)
\curveto(191.6163465,338.93726198)(191.73634638,338.94226197)(191.83635092,338.96226336)
\lineto(191.97135092,338.96226336)
\curveto(192.07134605,338.98226193)(192.16134596,339.00226191)(192.24135092,339.02226336)
\curveto(192.33134579,339.05226186)(192.4163457,339.08226183)(192.49635092,339.11226336)
\curveto(192.59634552,339.16226175)(192.70634541,339.22726169)(192.82635092,339.30726336)
\curveto(192.95634516,339.38726153)(193.05134507,339.46726145)(193.11135092,339.54726336)
\curveto(193.16134496,339.6172613)(193.21134491,339.68226123)(193.26135092,339.74226336)
\curveto(193.3213448,339.8122611)(193.39134473,339.86226105)(193.47135092,339.89226336)
\curveto(193.57134455,339.94226097)(193.69634442,339.96226095)(193.84635092,339.95226336)
\lineto(194.28135092,339.95226336)
\lineto(194.46135092,339.95226336)
\curveto(194.53134359,339.96226095)(194.59134353,339.95726096)(194.64135092,339.93726336)
\lineto(194.79135092,339.93726336)
\curveto(194.89134323,339.917261)(194.96134316,339.89226102)(195.00135092,339.86226336)
\curveto(195.04134308,339.84226107)(195.06134306,339.79726112)(195.06135092,339.72726336)
\curveto(195.07134305,339.65726126)(195.06634305,339.59726132)(195.04635092,339.54726336)
\curveto(194.99634312,339.40726151)(194.94134318,339.28226163)(194.88135092,339.17226336)
\curveto(194.8213433,339.06226185)(194.75134337,338.95226196)(194.67135092,338.84226336)
\curveto(194.45134367,338.5122624)(194.20134392,338.24726267)(193.92135092,338.04726336)
\curveto(193.64134448,337.84726307)(193.29134483,337.67726324)(192.87135092,337.53726336)
\curveto(192.76134536,337.49726342)(192.65134547,337.47226344)(192.54135092,337.46226336)
\curveto(192.43134569,337.45226346)(192.3163458,337.43226348)(192.19635092,337.40226336)
\curveto(192.15634596,337.39226352)(192.11134601,337.39226352)(192.06135092,337.40226336)
\curveto(192.0213461,337.40226351)(191.98134614,337.39726352)(191.94135092,337.38726336)
\lineto(191.77635092,337.38726336)
\curveto(191.72634639,337.36726355)(191.66634645,337.36226355)(191.59635092,337.37226336)
\curveto(191.53634658,337.37226354)(191.48134664,337.37726354)(191.43135092,337.38726336)
\curveto(191.35134677,337.39726352)(191.28134684,337.39726352)(191.22135092,337.38726336)
\curveto(191.16134696,337.37726354)(191.09634702,337.38226353)(191.02635092,337.40226336)
\curveto(190.97634714,337.42226349)(190.9213472,337.43226348)(190.86135092,337.43226336)
\curveto(190.80134732,337.43226348)(190.74634737,337.44226347)(190.69635092,337.46226336)
\curveto(190.58634753,337.48226343)(190.47634764,337.50726341)(190.36635092,337.53726336)
\curveto(190.25634786,337.55726336)(190.15634796,337.59226332)(190.06635092,337.64226336)
\curveto(189.95634816,337.68226323)(189.85134827,337.7172632)(189.75135092,337.74726336)
\curveto(189.66134846,337.78726313)(189.57634854,337.83226308)(189.49635092,337.88226336)
\curveto(189.17634894,338.08226283)(188.89134923,338.3122626)(188.64135092,338.57226336)
\curveto(188.39134973,338.84226207)(188.18634993,339.15226176)(188.02635092,339.50226336)
\curveto(187.97635014,339.6122613)(187.93635018,339.72226119)(187.90635092,339.83226336)
\curveto(187.87635024,339.95226096)(187.83635028,340.07226084)(187.78635092,340.19226336)
\curveto(187.77635034,340.23226068)(187.77135035,340.26726065)(187.77135092,340.29726336)
\curveto(187.77135035,340.33726058)(187.76635035,340.37726054)(187.75635092,340.41726336)
\curveto(187.7163504,340.53726038)(187.69135043,340.66726025)(187.68135092,340.80726336)
\lineto(187.65135092,341.22726336)
\curveto(187.65135047,341.27725964)(187.64635047,341.33225958)(187.63635092,341.39226336)
\curveto(187.63635048,341.45225946)(187.64135048,341.50725941)(187.65135092,341.55726336)
\lineto(187.65135092,341.73726336)
\lineto(187.69635092,342.09726336)
\curveto(187.73635038,342.26725865)(187.77135035,342.43225848)(187.80135092,342.59226336)
\curveto(187.83135029,342.75225816)(187.87635024,342.90225801)(187.93635092,343.04226336)
\curveto(188.36634975,344.08225683)(189.09634902,344.8172561)(190.12635092,345.24726336)
\curveto(190.26634785,345.30725561)(190.40634771,345.34725557)(190.54635092,345.36726336)
\curveto(190.69634742,345.39725552)(190.85134727,345.43225548)(191.01135092,345.47226336)
\curveto(191.09134703,345.48225543)(191.16634695,345.48725543)(191.23635092,345.48726336)
\curveto(191.30634681,345.48725543)(191.38134674,345.49225542)(191.46135092,345.50226336)
\curveto(191.97134615,345.5122554)(192.40634571,345.45225546)(192.76635092,345.32226336)
\curveto(193.13634498,345.20225571)(193.46634465,345.04225587)(193.75635092,344.84226336)
\curveto(193.84634427,344.78225613)(193.93634418,344.7122562)(194.02635092,344.63226336)
\curveto(194.116344,344.56225635)(194.19634392,344.48725643)(194.26635092,344.40726336)
\curveto(194.29634382,344.35725656)(194.33634378,344.3172566)(194.38635092,344.28726336)
\curveto(194.46634365,344.17725674)(194.54134358,344.06225685)(194.61135092,343.94226336)
\curveto(194.68134344,343.83225708)(194.75634336,343.7172572)(194.83635092,343.59726336)
\curveto(194.88634323,343.50725741)(194.92634319,343.4122575)(194.95635092,343.31226336)
\curveto(194.99634312,343.22225769)(195.03634308,343.12225779)(195.07635092,343.01226336)
\curveto(195.12634299,342.88225803)(195.16634295,342.74725817)(195.19635092,342.60726336)
\curveto(195.22634289,342.46725845)(195.26134286,342.32725859)(195.30135092,342.18726336)
\curveto(195.3213428,342.10725881)(195.32634279,342.0172589)(195.31635092,341.91726336)
\curveto(195.3163428,341.82725909)(195.32634279,341.74225917)(195.34635092,341.66226336)
\lineto(195.34635092,341.49726336)
\moveto(193.09635092,342.38226336)
\curveto(193.16634495,342.48225843)(193.17134495,342.60225831)(193.11135092,342.74226336)
\curveto(193.06134506,342.89225802)(193.0213451,343.00225791)(192.99135092,343.07226336)
\curveto(192.85134527,343.34225757)(192.66634545,343.54725737)(192.43635092,343.68726336)
\curveto(192.20634591,343.83725708)(191.88634623,343.917257)(191.47635092,343.92726336)
\curveto(191.44634667,343.90725701)(191.41134671,343.90225701)(191.37135092,343.91226336)
\curveto(191.33134679,343.92225699)(191.29634682,343.92225699)(191.26635092,343.91226336)
\curveto(191.2163469,343.89225702)(191.16134696,343.87725704)(191.10135092,343.86726336)
\curveto(191.04134708,343.86725705)(190.98634713,343.85725706)(190.93635092,343.83726336)
\curveto(190.49634762,343.69725722)(190.17134795,343.42225749)(189.96135092,343.01226336)
\curveto(189.94134818,342.97225794)(189.9163482,342.917258)(189.88635092,342.84726336)
\curveto(189.86634825,342.78725813)(189.85134827,342.72225819)(189.84135092,342.65226336)
\curveto(189.83134829,342.59225832)(189.83134829,342.53225838)(189.84135092,342.47226336)
\curveto(189.86134826,342.4122585)(189.89634822,342.36225855)(189.94635092,342.32226336)
\curveto(190.02634809,342.27225864)(190.13634798,342.24725867)(190.27635092,342.24726336)
\lineto(190.68135092,342.24726336)
\lineto(192.34635092,342.24726336)
\lineto(192.78135092,342.24726336)
\curveto(192.94134518,342.25725866)(193.04634507,342.30225861)(193.09635092,342.38226336)
}
}
{
\newrgbcolor{curcolor}{0 0 0}
\pscustom[linestyle=none,fillstyle=solid,fillcolor=curcolor]
{
\newpath
\moveto(201.01963217,345.48726336)
\curveto(201.61962636,345.50725541)(202.11962586,345.42225549)(202.51963217,345.23226336)
\curveto(202.91962506,345.04225587)(203.23462475,344.76225615)(203.46463217,344.39226336)
\curveto(203.53462445,344.28225663)(203.58962439,344.16225675)(203.62963217,344.03226336)
\curveto(203.66962431,343.912257)(203.70962427,343.78725713)(203.74963217,343.65726336)
\curveto(203.76962421,343.57725734)(203.7796242,343.50225741)(203.77963217,343.43226336)
\curveto(203.78962419,343.36225755)(203.80462418,343.29225762)(203.82463217,343.22226336)
\curveto(203.82462416,343.16225775)(203.82962415,343.12225779)(203.83963217,343.10226336)
\curveto(203.85962412,342.96225795)(203.86962411,342.8172581)(203.86963217,342.66726336)
\lineto(203.86963217,342.23226336)
\lineto(203.86963217,340.89726336)
\lineto(203.86963217,338.46726336)
\curveto(203.86962411,338.27726264)(203.86462412,338.09226282)(203.85463217,337.91226336)
\curveto(203.85462413,337.74226317)(203.7846242,337.63226328)(203.64463217,337.58226336)
\curveto(203.5846244,337.56226335)(203.51462447,337.55226336)(203.43463217,337.55226336)
\lineto(203.19463217,337.55226336)
\lineto(202.38463217,337.55226336)
\curveto(202.26462572,337.55226336)(202.15462583,337.55726336)(202.05463217,337.56726336)
\curveto(201.96462602,337.58726333)(201.89462609,337.63226328)(201.84463217,337.70226336)
\curveto(201.80462618,337.76226315)(201.7796262,337.83726308)(201.76963217,337.92726336)
\lineto(201.76963217,338.24226336)
\lineto(201.76963217,339.29226336)
\lineto(201.76963217,341.52726336)
\curveto(201.76962621,341.89725902)(201.75462623,342.23725868)(201.72463217,342.54726336)
\curveto(201.69462629,342.86725805)(201.60462638,343.13725778)(201.45463217,343.35726336)
\curveto(201.31462667,343.55725736)(201.10962687,343.69725722)(200.83963217,343.77726336)
\curveto(200.78962719,343.79725712)(200.73462725,343.80725711)(200.67463217,343.80726336)
\curveto(200.62462736,343.80725711)(200.56962741,343.8172571)(200.50963217,343.83726336)
\curveto(200.45962752,343.84725707)(200.39462759,343.84725707)(200.31463217,343.83726336)
\curveto(200.24462774,343.83725708)(200.18962779,343.83225708)(200.14963217,343.82226336)
\curveto(200.10962787,343.8122571)(200.07462791,343.80725711)(200.04463217,343.80726336)
\curveto(200.01462797,343.80725711)(199.984628,343.80225711)(199.95463217,343.79226336)
\curveto(199.72462826,343.73225718)(199.53962844,343.65225726)(199.39963217,343.55226336)
\curveto(199.0796289,343.32225759)(198.88962909,342.98725793)(198.82963217,342.54726336)
\curveto(198.76962921,342.10725881)(198.73962924,341.6122593)(198.73963217,341.06226336)
\lineto(198.73963217,339.18726336)
\lineto(198.73963217,338.27226336)
\lineto(198.73963217,338.00226336)
\curveto(198.73962924,337.912263)(198.72462926,337.83726308)(198.69463217,337.77726336)
\curveto(198.64462934,337.66726325)(198.56462942,337.60226331)(198.45463217,337.58226336)
\curveto(198.34462964,337.56226335)(198.20962977,337.55226336)(198.04963217,337.55226336)
\lineto(197.29963217,337.55226336)
\curveto(197.18963079,337.55226336)(197.0796309,337.55726336)(196.96963217,337.56726336)
\curveto(196.85963112,337.57726334)(196.7796312,337.6122633)(196.72963217,337.67226336)
\curveto(196.65963132,337.76226315)(196.62463136,337.89226302)(196.62463217,338.06226336)
\curveto(196.63463135,338.23226268)(196.63963134,338.39226252)(196.63963217,338.54226336)
\lineto(196.63963217,340.58226336)
\lineto(196.63963217,343.88226336)
\lineto(196.63963217,344.64726336)
\lineto(196.63963217,344.94726336)
\curveto(196.64963133,345.03725588)(196.6796313,345.1122558)(196.72963217,345.17226336)
\curveto(196.74963123,345.20225571)(196.7796312,345.22225569)(196.81963217,345.23226336)
\curveto(196.86963111,345.25225566)(196.91963106,345.26725565)(196.96963217,345.27726336)
\lineto(197.04463217,345.27726336)
\curveto(197.09463089,345.28725563)(197.14463084,345.29225562)(197.19463217,345.29226336)
\lineto(197.35963217,345.29226336)
\lineto(197.98963217,345.29226336)
\curveto(198.06962991,345.29225562)(198.14462984,345.28725563)(198.21463217,345.27726336)
\curveto(198.29462969,345.27725564)(198.36462962,345.26725565)(198.42463217,345.24726336)
\curveto(198.49462949,345.2172557)(198.53962944,345.17225574)(198.55963217,345.11226336)
\curveto(198.58962939,345.05225586)(198.61462937,344.98225593)(198.63463217,344.90226336)
\curveto(198.64462934,344.86225605)(198.64462934,344.82725609)(198.63463217,344.79726336)
\curveto(198.63462935,344.76725615)(198.64462934,344.73725618)(198.66463217,344.70726336)
\curveto(198.6846293,344.65725626)(198.69962928,344.62725629)(198.70963217,344.61726336)
\curveto(198.72962925,344.60725631)(198.75462923,344.59225632)(198.78463217,344.57226336)
\curveto(198.89462909,344.56225635)(198.984629,344.59725632)(199.05463217,344.67726336)
\curveto(199.12462886,344.76725615)(199.19962878,344.83725608)(199.27963217,344.88726336)
\curveto(199.54962843,345.08725583)(199.84962813,345.24725567)(200.17963217,345.36726336)
\curveto(200.26962771,345.39725552)(200.35962762,345.4172555)(200.44963217,345.42726336)
\curveto(200.54962743,345.43725548)(200.65462733,345.45225546)(200.76463217,345.47226336)
\curveto(200.79462719,345.48225543)(200.83962714,345.48225543)(200.89963217,345.47226336)
\curveto(200.95962702,345.47225544)(200.99962698,345.47725544)(201.01963217,345.48726336)
}
}
{
\newrgbcolor{curcolor}{0 0 0}
\pscustom[linestyle=none,fillstyle=solid,fillcolor=curcolor]
{
\newpath
\moveto(206.55088217,347.60226336)
\lineto(207.55588217,347.60226336)
\curveto(207.70587918,347.60225331)(207.83587905,347.59225332)(207.94588217,347.57226336)
\curveto(208.06587882,347.56225335)(208.15087874,347.50225341)(208.20088217,347.39226336)
\curveto(208.22087867,347.34225357)(208.23087866,347.28225363)(208.23088217,347.21226336)
\lineto(208.23088217,347.00226336)
\lineto(208.23088217,346.32726336)
\curveto(208.23087866,346.27725464)(208.22587866,346.2172547)(208.21588217,346.14726336)
\curveto(208.21587867,346.08725483)(208.22087867,346.03225488)(208.23088217,345.98226336)
\lineto(208.23088217,345.81726336)
\curveto(208.23087866,345.73725518)(208.23587865,345.66225525)(208.24588217,345.59226336)
\curveto(208.25587863,345.53225538)(208.28087861,345.47725544)(208.32088217,345.42726336)
\curveto(208.3908785,345.33725558)(208.51587837,345.28725563)(208.69588217,345.27726336)
\lineto(209.23588217,345.27726336)
\lineto(209.41588217,345.27726336)
\curveto(209.47587741,345.27725564)(209.53087736,345.26725565)(209.58088217,345.24726336)
\curveto(209.6908772,345.19725572)(209.75087714,345.10725581)(209.76088217,344.97726336)
\curveto(209.78087711,344.84725607)(209.7908771,344.70225621)(209.79088217,344.54226336)
\lineto(209.79088217,344.33226336)
\curveto(209.80087709,344.26225665)(209.79587709,344.20225671)(209.77588217,344.15226336)
\curveto(209.72587716,343.99225692)(209.62087727,343.90725701)(209.46088217,343.89726336)
\curveto(209.30087759,343.88725703)(209.12087777,343.88225703)(208.92088217,343.88226336)
\lineto(208.78588217,343.88226336)
\curveto(208.74587814,343.89225702)(208.71087818,343.89225702)(208.68088217,343.88226336)
\curveto(208.64087825,343.87225704)(208.60587828,343.86725705)(208.57588217,343.86726336)
\curveto(208.54587834,343.87725704)(208.51587837,343.87225704)(208.48588217,343.85226336)
\curveto(208.40587848,343.83225708)(208.34587854,343.78725713)(208.30588217,343.71726336)
\curveto(208.27587861,343.65725726)(208.25087864,343.58225733)(208.23088217,343.49226336)
\curveto(208.22087867,343.44225747)(208.22087867,343.38725753)(208.23088217,343.32726336)
\curveto(208.24087865,343.26725765)(208.24087865,343.2122577)(208.23088217,343.16226336)
\lineto(208.23088217,342.23226336)
\lineto(208.23088217,340.47726336)
\curveto(208.23087866,340.22726069)(208.23587865,340.00726091)(208.24588217,339.81726336)
\curveto(208.26587862,339.63726128)(208.33087856,339.47726144)(208.44088217,339.33726336)
\curveto(208.4908784,339.27726164)(208.55587833,339.23226168)(208.63588217,339.20226336)
\lineto(208.90588217,339.14226336)
\curveto(208.93587795,339.13226178)(208.96587792,339.12726179)(208.99588217,339.12726336)
\curveto(209.03587785,339.13726178)(209.06587782,339.13726178)(209.08588217,339.12726336)
\lineto(209.25088217,339.12726336)
\curveto(209.36087753,339.12726179)(209.45587743,339.12226179)(209.53588217,339.11226336)
\curveto(209.61587727,339.10226181)(209.68087721,339.06226185)(209.73088217,338.99226336)
\curveto(209.77087712,338.93226198)(209.7908771,338.85226206)(209.79088217,338.75226336)
\lineto(209.79088217,338.46726336)
\curveto(209.7908771,338.25726266)(209.7858771,338.06226285)(209.77588217,337.88226336)
\curveto(209.77587711,337.7122632)(209.69587719,337.59726332)(209.53588217,337.53726336)
\curveto(209.4858774,337.5172634)(209.44087745,337.5122634)(209.40088217,337.52226336)
\curveto(209.36087753,337.52226339)(209.31587757,337.5122634)(209.26588217,337.49226336)
\lineto(209.11588217,337.49226336)
\curveto(209.09587779,337.49226342)(209.06587782,337.49726342)(209.02588217,337.50726336)
\curveto(208.9858779,337.50726341)(208.95087794,337.50226341)(208.92088217,337.49226336)
\curveto(208.87087802,337.48226343)(208.81587807,337.48226343)(208.75588217,337.49226336)
\lineto(208.60588217,337.49226336)
\lineto(208.45588217,337.49226336)
\curveto(208.40587848,337.48226343)(208.36087853,337.48226343)(208.32088217,337.49226336)
\lineto(208.15588217,337.49226336)
\curveto(208.10587878,337.50226341)(208.05087884,337.50726341)(207.99088217,337.50726336)
\curveto(207.93087896,337.50726341)(207.87587901,337.5122634)(207.82588217,337.52226336)
\curveto(207.75587913,337.53226338)(207.6908792,337.54226337)(207.63088217,337.55226336)
\lineto(207.45088217,337.58226336)
\curveto(207.34087955,337.6122633)(207.23587965,337.64726327)(207.13588217,337.68726336)
\curveto(207.03587985,337.72726319)(206.94087995,337.77226314)(206.85088217,337.82226336)
\lineto(206.76088217,337.88226336)
\curveto(206.73088016,337.912263)(206.69588019,337.94226297)(206.65588217,337.97226336)
\curveto(206.63588025,337.99226292)(206.61088028,338.0122629)(206.58088217,338.03226336)
\lineto(206.50588217,338.10726336)
\curveto(206.36588052,338.29726262)(206.26088063,338.50726241)(206.19088217,338.73726336)
\curveto(206.17088072,338.77726214)(206.16088073,338.8122621)(206.16088217,338.84226336)
\curveto(206.17088072,338.88226203)(206.17088072,338.92726199)(206.16088217,338.97726336)
\curveto(206.15088074,338.99726192)(206.14588074,339.02226189)(206.14588217,339.05226336)
\curveto(206.14588074,339.08226183)(206.14088075,339.10726181)(206.13088217,339.12726336)
\lineto(206.13088217,339.27726336)
\curveto(206.12088077,339.3172616)(206.11588077,339.36226155)(206.11588217,339.41226336)
\curveto(206.12588076,339.46226145)(206.13088076,339.5122614)(206.13088217,339.56226336)
\lineto(206.13088217,340.13226336)
\lineto(206.13088217,342.36726336)
\lineto(206.13088217,343.16226336)
\lineto(206.13088217,343.37226336)
\curveto(206.14088075,343.44225747)(206.13588075,343.50725741)(206.11588217,343.56726336)
\curveto(206.07588081,343.70725721)(206.00588088,343.79725712)(205.90588217,343.83726336)
\curveto(205.79588109,343.88725703)(205.65588123,343.90225701)(205.48588217,343.88226336)
\curveto(205.31588157,343.86225705)(205.17088172,343.87725704)(205.05088217,343.92726336)
\curveto(204.97088192,343.95725696)(204.92088197,344.00225691)(204.90088217,344.06226336)
\curveto(204.88088201,344.12225679)(204.86088203,344.19725672)(204.84088217,344.28726336)
\lineto(204.84088217,344.60226336)
\curveto(204.84088205,344.78225613)(204.85088204,344.92725599)(204.87088217,345.03726336)
\curveto(204.890882,345.14725577)(204.97588191,345.22225569)(205.12588217,345.26226336)
\curveto(205.16588172,345.28225563)(205.20588168,345.28725563)(205.24588217,345.27726336)
\lineto(205.38088217,345.27726336)
\curveto(205.53088136,345.27725564)(205.67088122,345.28225563)(205.80088217,345.29226336)
\curveto(205.93088096,345.3122556)(206.02088087,345.37225554)(206.07088217,345.47226336)
\curveto(206.10088079,345.54225537)(206.11588077,345.62225529)(206.11588217,345.71226336)
\curveto(206.12588076,345.80225511)(206.13088076,345.89225502)(206.13088217,345.98226336)
\lineto(206.13088217,346.91226336)
\lineto(206.13088217,347.16726336)
\curveto(206.13088076,347.25725366)(206.14088075,347.33225358)(206.16088217,347.39226336)
\curveto(206.21088068,347.49225342)(206.2858806,347.55725336)(206.38588217,347.58726336)
\curveto(206.40588048,347.59725332)(206.43088046,347.59725332)(206.46088217,347.58726336)
\curveto(206.50088039,347.58725333)(206.53088036,347.59225332)(206.55088217,347.60226336)
}
}
{
\newrgbcolor{curcolor}{0 0 0}
\pscustom[linestyle=none,fillstyle=solid,fillcolor=curcolor]
{
\newpath
\moveto(217.82431967,338.15226336)
\curveto(217.84431182,338.04226287)(217.85431181,337.93226298)(217.85431967,337.82226336)
\curveto(217.8643118,337.7122632)(217.81431185,337.63726328)(217.70431967,337.59726336)
\curveto(217.64431202,337.56726335)(217.57431209,337.55226336)(217.49431967,337.55226336)
\lineto(217.25431967,337.55226336)
\lineto(216.44431967,337.55226336)
\lineto(216.17431967,337.55226336)
\curveto(216.09431357,337.56226335)(216.02931363,337.58726333)(215.97931967,337.62726336)
\curveto(215.90931375,337.66726325)(215.85431381,337.72226319)(215.81431967,337.79226336)
\curveto(215.78431388,337.87226304)(215.73931392,337.93726298)(215.67931967,337.98726336)
\curveto(215.659314,338.00726291)(215.63431403,338.02226289)(215.60431967,338.03226336)
\curveto(215.57431409,338.05226286)(215.53431413,338.05726286)(215.48431967,338.04726336)
\curveto(215.43431423,338.02726289)(215.38431428,338.00226291)(215.33431967,337.97226336)
\curveto(215.29431437,337.94226297)(215.24931441,337.917263)(215.19931967,337.89726336)
\curveto(215.14931451,337.85726306)(215.09431457,337.82226309)(215.03431967,337.79226336)
\lineto(214.85431967,337.70226336)
\curveto(214.72431494,337.64226327)(214.58931507,337.59226332)(214.44931967,337.55226336)
\curveto(214.30931535,337.52226339)(214.1643155,337.48726343)(214.01431967,337.44726336)
\curveto(213.94431572,337.42726349)(213.87431579,337.4172635)(213.80431967,337.41726336)
\curveto(213.74431592,337.40726351)(213.67931598,337.39726352)(213.60931967,337.38726336)
\lineto(213.51931967,337.38726336)
\curveto(213.48931617,337.37726354)(213.4593162,337.37226354)(213.42931967,337.37226336)
\lineto(213.26431967,337.37226336)
\curveto(213.1643165,337.35226356)(213.0643166,337.35226356)(212.96431967,337.37226336)
\lineto(212.82931967,337.37226336)
\curveto(212.7593169,337.39226352)(212.68931697,337.40226351)(212.61931967,337.40226336)
\curveto(212.5593171,337.39226352)(212.49931716,337.39726352)(212.43931967,337.41726336)
\curveto(212.33931732,337.43726348)(212.24431742,337.45726346)(212.15431967,337.47726336)
\curveto(212.0643176,337.48726343)(211.97931768,337.5122634)(211.89931967,337.55226336)
\curveto(211.60931805,337.66226325)(211.3593183,337.80226311)(211.14931967,337.97226336)
\curveto(210.94931871,338.15226276)(210.78931887,338.38726253)(210.66931967,338.67726336)
\curveto(210.63931902,338.74726217)(210.60931905,338.82226209)(210.57931967,338.90226336)
\curveto(210.5593191,338.98226193)(210.53931912,339.06726185)(210.51931967,339.15726336)
\curveto(210.49931916,339.20726171)(210.48931917,339.25726166)(210.48931967,339.30726336)
\curveto(210.49931916,339.35726156)(210.49931916,339.40726151)(210.48931967,339.45726336)
\curveto(210.47931918,339.48726143)(210.46931919,339.54726137)(210.45931967,339.63726336)
\curveto(210.4593192,339.73726118)(210.4643192,339.80726111)(210.47431967,339.84726336)
\curveto(210.49431917,339.94726097)(210.50431916,340.03226088)(210.50431967,340.10226336)
\lineto(210.59431967,340.43226336)
\curveto(210.62431904,340.55226036)(210.664319,340.65726026)(210.71431967,340.74726336)
\curveto(210.88431878,341.03725988)(211.07931858,341.25725966)(211.29931967,341.40726336)
\curveto(211.51931814,341.55725936)(211.79931786,341.68725923)(212.13931967,341.79726336)
\curveto(212.26931739,341.84725907)(212.40431726,341.88225903)(212.54431967,341.90226336)
\curveto(212.68431698,341.92225899)(212.82431684,341.94725897)(212.96431967,341.97726336)
\curveto(213.04431662,341.99725892)(213.12931653,342.00725891)(213.21931967,342.00726336)
\curveto(213.30931635,342.0172589)(213.39931626,342.03225888)(213.48931967,342.05226336)
\curveto(213.5593161,342.07225884)(213.62931603,342.07725884)(213.69931967,342.06726336)
\curveto(213.76931589,342.06725885)(213.84431582,342.07725884)(213.92431967,342.09726336)
\curveto(213.99431567,342.1172588)(214.0643156,342.12725879)(214.13431967,342.12726336)
\curveto(214.20431546,342.12725879)(214.27931538,342.13725878)(214.35931967,342.15726336)
\curveto(214.56931509,342.20725871)(214.7593149,342.24725867)(214.92931967,342.27726336)
\curveto(215.10931455,342.3172586)(215.26931439,342.40725851)(215.40931967,342.54726336)
\curveto(215.49931416,342.63725828)(215.5593141,342.73725818)(215.58931967,342.84726336)
\curveto(215.59931406,342.87725804)(215.59931406,342.90225801)(215.58931967,342.92226336)
\curveto(215.58931407,342.94225797)(215.59431407,342.96225795)(215.60431967,342.98226336)
\curveto(215.61431405,343.00225791)(215.61931404,343.03225788)(215.61931967,343.07226336)
\lineto(215.61931967,343.16226336)
\lineto(215.58931967,343.28226336)
\curveto(215.58931407,343.32225759)(215.58431408,343.35725756)(215.57431967,343.38726336)
\curveto(215.47431419,343.68725723)(215.2643144,343.89225702)(214.94431967,344.00226336)
\curveto(214.85431481,344.03225688)(214.74431492,344.05225686)(214.61431967,344.06226336)
\curveto(214.49431517,344.08225683)(214.36931529,344.08725683)(214.23931967,344.07726336)
\curveto(214.10931555,344.07725684)(213.98431568,344.06725685)(213.86431967,344.04726336)
\curveto(213.74431592,344.02725689)(213.63931602,344.00225691)(213.54931967,343.97226336)
\curveto(213.48931617,343.95225696)(213.42931623,343.92225699)(213.36931967,343.88226336)
\curveto(213.31931634,343.85225706)(213.26931639,343.8172571)(213.21931967,343.77726336)
\curveto(213.16931649,343.73725718)(213.11431655,343.68225723)(213.05431967,343.61226336)
\curveto(213.00431666,343.54225737)(212.96931669,343.47725744)(212.94931967,343.41726336)
\curveto(212.89931676,343.3172576)(212.85431681,343.22225769)(212.81431967,343.13226336)
\curveto(212.78431688,343.04225787)(212.71431695,342.98225793)(212.60431967,342.95226336)
\curveto(212.52431714,342.93225798)(212.43931722,342.92225799)(212.34931967,342.92226336)
\lineto(212.07931967,342.92226336)
\lineto(211.50931967,342.92226336)
\curveto(211.4593182,342.92225799)(211.40931825,342.917258)(211.35931967,342.90726336)
\curveto(211.30931835,342.90725801)(211.2643184,342.912258)(211.22431967,342.92226336)
\lineto(211.08931967,342.92226336)
\curveto(211.06931859,342.93225798)(211.04431862,342.93725798)(211.01431967,342.93726336)
\curveto(210.98431868,342.93725798)(210.9593187,342.94725797)(210.93931967,342.96726336)
\curveto(210.8593188,342.98725793)(210.80431886,343.05225786)(210.77431967,343.16226336)
\curveto(210.7643189,343.2122577)(210.7643189,343.26225765)(210.77431967,343.31226336)
\curveto(210.78431888,343.36225755)(210.79431887,343.40725751)(210.80431967,343.44726336)
\curveto(210.83431883,343.55725736)(210.8643188,343.65725726)(210.89431967,343.74726336)
\curveto(210.93431873,343.84725707)(210.97931868,343.93725698)(211.02931967,344.01726336)
\lineto(211.11931967,344.16726336)
\lineto(211.20931967,344.31726336)
\curveto(211.28931837,344.42725649)(211.38931827,344.53225638)(211.50931967,344.63226336)
\curveto(211.52931813,344.64225627)(211.5593181,344.66725625)(211.59931967,344.70726336)
\curveto(211.64931801,344.74725617)(211.69431797,344.78225613)(211.73431967,344.81226336)
\curveto(211.77431789,344.84225607)(211.81931784,344.87225604)(211.86931967,344.90226336)
\curveto(212.03931762,345.0122559)(212.21931744,345.09725582)(212.40931967,345.15726336)
\curveto(212.59931706,345.22725569)(212.79431687,345.29225562)(212.99431967,345.35226336)
\curveto(213.11431655,345.38225553)(213.23931642,345.40225551)(213.36931967,345.41226336)
\curveto(213.49931616,345.42225549)(213.62931603,345.44225547)(213.75931967,345.47226336)
\curveto(213.79931586,345.48225543)(213.8593158,345.48225543)(213.93931967,345.47226336)
\curveto(214.02931563,345.46225545)(214.08431558,345.46725545)(214.10431967,345.48726336)
\curveto(214.51431515,345.49725542)(214.90431476,345.48225543)(215.27431967,345.44226336)
\curveto(215.65431401,345.40225551)(215.99431367,345.32725559)(216.29431967,345.21726336)
\curveto(216.60431306,345.10725581)(216.86931279,344.95725596)(217.08931967,344.76726336)
\curveto(217.30931235,344.58725633)(217.47931218,344.35225656)(217.59931967,344.06226336)
\curveto(217.66931199,343.89225702)(217.70931195,343.69725722)(217.71931967,343.47726336)
\curveto(217.72931193,343.25725766)(217.73431193,343.03225788)(217.73431967,342.80226336)
\lineto(217.73431967,339.45726336)
\lineto(217.73431967,338.87226336)
\curveto(217.73431193,338.68226223)(217.75431191,338.50726241)(217.79431967,338.34726336)
\curveto(217.80431186,338.3172626)(217.80931185,338.28226263)(217.80931967,338.24226336)
\curveto(217.80931185,338.2122627)(217.81431185,338.18226273)(217.82431967,338.15226336)
\moveto(215.61931967,340.46226336)
\curveto(215.62931403,340.5122604)(215.63431403,340.56726035)(215.63431967,340.62726336)
\curveto(215.63431403,340.69726022)(215.62931403,340.75726016)(215.61931967,340.80726336)
\curveto(215.59931406,340.86726005)(215.58931407,340.92225999)(215.58931967,340.97226336)
\curveto(215.58931407,341.02225989)(215.56931409,341.06225985)(215.52931967,341.09226336)
\curveto(215.47931418,341.13225978)(215.40431426,341.15225976)(215.30431967,341.15226336)
\curveto(215.2643144,341.14225977)(215.22931443,341.13225978)(215.19931967,341.12226336)
\curveto(215.16931449,341.12225979)(215.13431453,341.1172598)(215.09431967,341.10726336)
\curveto(215.02431464,341.08725983)(214.94931471,341.07225984)(214.86931967,341.06226336)
\curveto(214.78931487,341.05225986)(214.70931495,341.03725988)(214.62931967,341.01726336)
\curveto(214.59931506,341.00725991)(214.55431511,341.00225991)(214.49431967,341.00226336)
\curveto(214.3643153,340.97225994)(214.23431543,340.95225996)(214.10431967,340.94226336)
\curveto(213.97431569,340.93225998)(213.84931581,340.90726001)(213.72931967,340.86726336)
\curveto(213.64931601,340.84726007)(213.57431609,340.82726009)(213.50431967,340.80726336)
\curveto(213.43431623,340.79726012)(213.3643163,340.77726014)(213.29431967,340.74726336)
\curveto(213.08431658,340.65726026)(212.90431676,340.52226039)(212.75431967,340.34226336)
\curveto(212.61431705,340.16226075)(212.5643171,339.912261)(212.60431967,339.59226336)
\curveto(212.62431704,339.42226149)(212.67931698,339.28226163)(212.76931967,339.17226336)
\curveto(212.83931682,339.06226185)(212.94431672,338.97226194)(213.08431967,338.90226336)
\curveto(213.22431644,338.84226207)(213.37431629,338.79726212)(213.53431967,338.76726336)
\curveto(213.70431596,338.73726218)(213.87931578,338.72726219)(214.05931967,338.73726336)
\curveto(214.24931541,338.75726216)(214.42431524,338.79226212)(214.58431967,338.84226336)
\curveto(214.84431482,338.92226199)(215.04931461,339.04726187)(215.19931967,339.21726336)
\curveto(215.34931431,339.39726152)(215.4643142,339.6172613)(215.54431967,339.87726336)
\curveto(215.5643141,339.94726097)(215.57431409,340.0172609)(215.57431967,340.08726336)
\curveto(215.58431408,340.16726075)(215.59931406,340.24726067)(215.61931967,340.32726336)
\lineto(215.61931967,340.46226336)
}
}
{
\newrgbcolor{curcolor}{0 0 0}
\pscustom[linestyle=none,fillstyle=solid,fillcolor=curcolor]
{
\newpath
\moveto(223.81260092,345.48726336)
\curveto(223.9225956,345.48725543)(224.01759551,345.47725544)(224.09760092,345.45726336)
\curveto(224.18759534,345.43725548)(224.25759527,345.39225552)(224.30760092,345.32226336)
\curveto(224.36759516,345.24225567)(224.39759513,345.10225581)(224.39760092,344.90226336)
\lineto(224.39760092,344.39226336)
\lineto(224.39760092,344.01726336)
\curveto(224.40759512,343.87725704)(224.39259513,343.76725715)(224.35260092,343.68726336)
\curveto(224.31259521,343.6172573)(224.25259527,343.57225734)(224.17260092,343.55226336)
\curveto(224.10259542,343.53225738)(224.01759551,343.52225739)(223.91760092,343.52226336)
\curveto(223.8275957,343.52225739)(223.7275958,343.52725739)(223.61760092,343.53726336)
\curveto(223.51759601,343.54725737)(223.4225961,343.54225737)(223.33260092,343.52226336)
\curveto(223.26259626,343.50225741)(223.19259633,343.48725743)(223.12260092,343.47726336)
\curveto(223.05259647,343.47725744)(222.98759654,343.46725745)(222.92760092,343.44726336)
\curveto(222.76759676,343.39725752)(222.60759692,343.32225759)(222.44760092,343.22226336)
\curveto(222.28759724,343.13225778)(222.16259736,343.02725789)(222.07260092,342.90726336)
\curveto(222.0225975,342.82725809)(221.96759756,342.74225817)(221.90760092,342.65226336)
\curveto(221.85759767,342.57225834)(221.80759772,342.48725843)(221.75760092,342.39726336)
\curveto(221.7275978,342.3172586)(221.69759783,342.23225868)(221.66760092,342.14226336)
\lineto(221.60760092,341.90226336)
\curveto(221.58759794,341.83225908)(221.57759795,341.75725916)(221.57760092,341.67726336)
\curveto(221.57759795,341.60725931)(221.56759796,341.53725938)(221.54760092,341.46726336)
\curveto(221.53759799,341.42725949)(221.53259799,341.38725953)(221.53260092,341.34726336)
\curveto(221.54259798,341.3172596)(221.54259798,341.28725963)(221.53260092,341.25726336)
\lineto(221.53260092,341.01726336)
\curveto(221.51259801,340.94725997)(221.50759802,340.86726005)(221.51760092,340.77726336)
\curveto(221.527598,340.69726022)(221.53259799,340.6172603)(221.53260092,340.53726336)
\lineto(221.53260092,339.57726336)
\lineto(221.53260092,338.30226336)
\curveto(221.53259799,338.17226274)(221.527598,338.05226286)(221.51760092,337.94226336)
\curveto(221.50759802,337.83226308)(221.47759805,337.74226317)(221.42760092,337.67226336)
\curveto(221.40759812,337.64226327)(221.37259815,337.6172633)(221.32260092,337.59726336)
\curveto(221.28259824,337.58726333)(221.23759829,337.57726334)(221.18760092,337.56726336)
\lineto(221.11260092,337.56726336)
\curveto(221.06259846,337.55726336)(221.00759852,337.55226336)(220.94760092,337.55226336)
\lineto(220.78260092,337.55226336)
\lineto(220.13760092,337.55226336)
\curveto(220.07759945,337.56226335)(220.01259951,337.56726335)(219.94260092,337.56726336)
\lineto(219.74760092,337.56726336)
\curveto(219.69759983,337.58726333)(219.64759988,337.60226331)(219.59760092,337.61226336)
\curveto(219.54759998,337.63226328)(219.51260001,337.66726325)(219.49260092,337.71726336)
\curveto(219.45260007,337.76726315)(219.4276001,337.83726308)(219.41760092,337.92726336)
\lineto(219.41760092,338.22726336)
\lineto(219.41760092,339.24726336)
\lineto(219.41760092,343.47726336)
\lineto(219.41760092,344.58726336)
\lineto(219.41760092,344.87226336)
\curveto(219.41760011,344.97225594)(219.43760009,345.05225586)(219.47760092,345.11226336)
\curveto(219.5276,345.19225572)(219.60259992,345.24225567)(219.70260092,345.26226336)
\curveto(219.80259972,345.28225563)(219.9225996,345.29225562)(220.06260092,345.29226336)
\lineto(220.82760092,345.29226336)
\curveto(220.94759858,345.29225562)(221.05259847,345.28225563)(221.14260092,345.26226336)
\curveto(221.23259829,345.25225566)(221.30259822,345.20725571)(221.35260092,345.12726336)
\curveto(221.38259814,345.07725584)(221.39759813,345.00725591)(221.39760092,344.91726336)
\lineto(221.42760092,344.64726336)
\curveto(221.43759809,344.56725635)(221.45259807,344.49225642)(221.47260092,344.42226336)
\curveto(221.50259802,344.35225656)(221.55259797,344.3172566)(221.62260092,344.31726336)
\curveto(221.64259788,344.33725658)(221.66259786,344.34725657)(221.68260092,344.34726336)
\curveto(221.70259782,344.34725657)(221.7225978,344.35725656)(221.74260092,344.37726336)
\curveto(221.80259772,344.42725649)(221.85259767,344.48225643)(221.89260092,344.54226336)
\curveto(221.94259758,344.6122563)(222.00259752,344.67225624)(222.07260092,344.72226336)
\curveto(222.11259741,344.75225616)(222.14759738,344.78225613)(222.17760092,344.81226336)
\curveto(222.20759732,344.85225606)(222.24259728,344.88725603)(222.28260092,344.91726336)
\lineto(222.55260092,345.09726336)
\curveto(222.65259687,345.15725576)(222.75259677,345.2122557)(222.85260092,345.26226336)
\curveto(222.95259657,345.30225561)(223.05259647,345.33725558)(223.15260092,345.36726336)
\lineto(223.48260092,345.45726336)
\curveto(223.51259601,345.46725545)(223.56759596,345.46725545)(223.64760092,345.45726336)
\curveto(223.73759579,345.45725546)(223.79259573,345.46725545)(223.81260092,345.48726336)
}
}
{
\newrgbcolor{curcolor}{0 0 0}
\pscustom[linestyle=none,fillstyle=solid,fillcolor=curcolor]
{
\newpath
\moveto(227.31767904,348.14226336)
\curveto(227.38767609,348.06225285)(227.42267606,347.94225297)(227.42267904,347.78226336)
\lineto(227.42267904,347.31726336)
\lineto(227.42267904,346.91226336)
\curveto(227.42267606,346.77225414)(227.38767609,346.67725424)(227.31767904,346.62726336)
\curveto(227.25767622,346.57725434)(227.1776763,346.54725437)(227.07767904,346.53726336)
\curveto(226.98767649,346.52725439)(226.88767659,346.52225439)(226.77767904,346.52226336)
\lineto(225.93767904,346.52226336)
\curveto(225.82767765,346.52225439)(225.72767775,346.52725439)(225.63767904,346.53726336)
\curveto(225.55767792,346.54725437)(225.48767799,346.57725434)(225.42767904,346.62726336)
\curveto(225.38767809,346.65725426)(225.35767812,346.7122542)(225.33767904,346.79226336)
\curveto(225.32767815,346.88225403)(225.31767816,346.97725394)(225.30767904,347.07726336)
\lineto(225.30767904,347.40726336)
\curveto(225.31767816,347.5172534)(225.32267816,347.6122533)(225.32267904,347.69226336)
\lineto(225.32267904,347.90226336)
\curveto(225.33267815,347.97225294)(225.35267813,348.03225288)(225.38267904,348.08226336)
\curveto(225.40267808,348.12225279)(225.42767805,348.15225276)(225.45767904,348.17226336)
\lineto(225.57767904,348.23226336)
\curveto(225.59767788,348.23225268)(225.62267786,348.23225268)(225.65267904,348.23226336)
\curveto(225.6826778,348.24225267)(225.70767777,348.24725267)(225.72767904,348.24726336)
\lineto(226.82267904,348.24726336)
\curveto(226.92267656,348.24725267)(227.01767646,348.24225267)(227.10767904,348.23226336)
\curveto(227.19767628,348.22225269)(227.26767621,348.19225272)(227.31767904,348.14226336)
\moveto(227.42267904,338.37726336)
\curveto(227.42267606,338.17726274)(227.41767606,338.00726291)(227.40767904,337.86726336)
\curveto(227.39767608,337.72726319)(227.30767617,337.63226328)(227.13767904,337.58226336)
\curveto(227.0776764,337.56226335)(227.01267647,337.55226336)(226.94267904,337.55226336)
\curveto(226.87267661,337.56226335)(226.79767668,337.56726335)(226.71767904,337.56726336)
\lineto(225.87767904,337.56726336)
\curveto(225.78767769,337.56726335)(225.69767778,337.57226334)(225.60767904,337.58226336)
\curveto(225.52767795,337.59226332)(225.46767801,337.62226329)(225.42767904,337.67226336)
\curveto(225.36767811,337.74226317)(225.33267815,337.82726309)(225.32267904,337.92726336)
\lineto(225.32267904,338.27226336)
\lineto(225.32267904,344.60226336)
\lineto(225.32267904,344.90226336)
\curveto(225.32267816,345.00225591)(225.34267814,345.08225583)(225.38267904,345.14226336)
\curveto(225.44267804,345.2122557)(225.52767795,345.25725566)(225.63767904,345.27726336)
\curveto(225.65767782,345.28725563)(225.6826778,345.28725563)(225.71267904,345.27726336)
\curveto(225.75267773,345.27725564)(225.7826777,345.28225563)(225.80267904,345.29226336)
\lineto(226.55267904,345.29226336)
\lineto(226.74767904,345.29226336)
\curveto(226.82767665,345.30225561)(226.89267659,345.30225561)(226.94267904,345.29226336)
\lineto(227.06267904,345.29226336)
\curveto(227.12267636,345.27225564)(227.1776763,345.25725566)(227.22767904,345.24726336)
\curveto(227.2776762,345.23725568)(227.31767616,345.20725571)(227.34767904,345.15726336)
\curveto(227.38767609,345.10725581)(227.40767607,345.03725588)(227.40767904,344.94726336)
\curveto(227.41767606,344.85725606)(227.42267606,344.76225615)(227.42267904,344.66226336)
\lineto(227.42267904,338.37726336)
}
}
{
\newrgbcolor{curcolor}{0 0 0}
\pscustom[linestyle=none,fillstyle=solid,fillcolor=curcolor]
{
\newpath
\moveto(236.85486654,341.73726336)
\curveto(236.87485797,341.67725924)(236.88485796,341.59225932)(236.88486654,341.48226336)
\curveto(236.88485796,341.37225954)(236.87485797,341.28725963)(236.85486654,341.22726336)
\lineto(236.85486654,341.07726336)
\curveto(236.83485801,340.99725992)(236.82485802,340.91726)(236.82486654,340.83726336)
\curveto(236.83485801,340.75726016)(236.82985802,340.67726024)(236.80986654,340.59726336)
\curveto(236.78985806,340.52726039)(236.77485807,340.46226045)(236.76486654,340.40226336)
\curveto(236.75485809,340.34226057)(236.7448581,340.27726064)(236.73486654,340.20726336)
\curveto(236.69485815,340.09726082)(236.65985819,339.98226093)(236.62986654,339.86226336)
\curveto(236.59985825,339.75226116)(236.55985829,339.64726127)(236.50986654,339.54726336)
\curveto(236.29985855,339.06726185)(236.02485882,338.67726224)(235.68486654,338.37726336)
\curveto(235.3448595,338.07726284)(234.93485991,337.82726309)(234.45486654,337.62726336)
\curveto(234.33486051,337.57726334)(234.20986064,337.54226337)(234.07986654,337.52226336)
\curveto(233.95986089,337.49226342)(233.83486101,337.46226345)(233.70486654,337.43226336)
\curveto(233.65486119,337.4122635)(233.59986125,337.40226351)(233.53986654,337.40226336)
\curveto(233.47986137,337.40226351)(233.42486142,337.39726352)(233.37486654,337.38726336)
\lineto(233.26986654,337.38726336)
\curveto(233.23986161,337.37726354)(233.20986164,337.37226354)(233.17986654,337.37226336)
\curveto(233.12986172,337.36226355)(233.0498618,337.35726356)(232.93986654,337.35726336)
\curveto(232.82986202,337.34726357)(232.7448621,337.35226356)(232.68486654,337.37226336)
\lineto(232.53486654,337.37226336)
\curveto(232.48486236,337.38226353)(232.42986242,337.38726353)(232.36986654,337.38726336)
\curveto(232.31986253,337.37726354)(232.26986258,337.38226353)(232.21986654,337.40226336)
\curveto(232.17986267,337.4122635)(232.13986271,337.4172635)(232.09986654,337.41726336)
\curveto(232.06986278,337.4172635)(232.02986282,337.42226349)(231.97986654,337.43226336)
\curveto(231.87986297,337.46226345)(231.77986307,337.48726343)(231.67986654,337.50726336)
\curveto(231.57986327,337.52726339)(231.48486336,337.55726336)(231.39486654,337.59726336)
\curveto(231.27486357,337.63726328)(231.15986369,337.67726324)(231.04986654,337.71726336)
\curveto(230.9498639,337.75726316)(230.844864,337.80726311)(230.73486654,337.86726336)
\curveto(230.38486446,338.07726284)(230.08486476,338.32226259)(229.83486654,338.60226336)
\curveto(229.58486526,338.88226203)(229.37486547,339.2172617)(229.20486654,339.60726336)
\curveto(229.15486569,339.69726122)(229.11486573,339.79226112)(229.08486654,339.89226336)
\curveto(229.06486578,339.99226092)(229.03986581,340.09726082)(229.00986654,340.20726336)
\curveto(228.98986586,340.25726066)(228.97986587,340.30226061)(228.97986654,340.34226336)
\curveto(228.97986587,340.38226053)(228.96986588,340.42726049)(228.94986654,340.47726336)
\curveto(228.92986592,340.55726036)(228.91986593,340.63726028)(228.91986654,340.71726336)
\curveto(228.91986593,340.80726011)(228.90986594,340.89226002)(228.88986654,340.97226336)
\curveto(228.87986597,341.02225989)(228.87486597,341.06725985)(228.87486654,341.10726336)
\lineto(228.87486654,341.24226336)
\curveto(228.85486599,341.30225961)(228.844866,341.38725953)(228.84486654,341.49726336)
\curveto(228.85486599,341.60725931)(228.86986598,341.69225922)(228.88986654,341.75226336)
\lineto(228.88986654,341.85726336)
\curveto(228.89986595,341.90725901)(228.89986595,341.95725896)(228.88986654,342.00726336)
\curveto(228.88986596,342.06725885)(228.89986595,342.12225879)(228.91986654,342.17226336)
\curveto(228.92986592,342.22225869)(228.93486591,342.26725865)(228.93486654,342.30726336)
\curveto(228.93486591,342.35725856)(228.9448659,342.40725851)(228.96486654,342.45726336)
\curveto(229.00486584,342.58725833)(229.03986581,342.7122582)(229.06986654,342.83226336)
\curveto(229.09986575,342.96225795)(229.13986571,343.08725783)(229.18986654,343.20726336)
\curveto(229.36986548,343.6172573)(229.58486526,343.95725696)(229.83486654,344.22726336)
\curveto(230.08486476,344.50725641)(230.38986446,344.76225615)(230.74986654,344.99226336)
\curveto(230.849864,345.04225587)(230.95486389,345.08725583)(231.06486654,345.12726336)
\curveto(231.17486367,345.16725575)(231.28486356,345.2122557)(231.39486654,345.26226336)
\curveto(231.52486332,345.3122556)(231.65986319,345.34725557)(231.79986654,345.36726336)
\curveto(231.93986291,345.38725553)(232.08486276,345.4172555)(232.23486654,345.45726336)
\curveto(232.31486253,345.46725545)(232.38986246,345.47225544)(232.45986654,345.47226336)
\curveto(232.52986232,345.47225544)(232.59986225,345.47725544)(232.66986654,345.48726336)
\curveto(233.2498616,345.49725542)(233.7498611,345.43725548)(234.16986654,345.30726336)
\curveto(234.59986025,345.17725574)(234.97985987,344.99725592)(235.30986654,344.76726336)
\curveto(235.41985943,344.68725623)(235.52985932,344.59725632)(235.63986654,344.49726336)
\curveto(235.75985909,344.40725651)(235.85985899,344.30725661)(235.93986654,344.19726336)
\curveto(236.01985883,344.09725682)(236.08985876,343.99725692)(236.14986654,343.89726336)
\curveto(236.21985863,343.79725712)(236.28985856,343.69225722)(236.35986654,343.58226336)
\curveto(236.42985842,343.47225744)(236.48485836,343.35225756)(236.52486654,343.22226336)
\curveto(236.56485828,343.10225781)(236.60985824,342.97225794)(236.65986654,342.83226336)
\curveto(236.68985816,342.75225816)(236.71485813,342.66725825)(236.73486654,342.57726336)
\lineto(236.79486654,342.30726336)
\curveto(236.80485804,342.26725865)(236.80985804,342.22725869)(236.80986654,342.18726336)
\curveto(236.80985804,342.14725877)(236.81485803,342.10725881)(236.82486654,342.06726336)
\curveto(236.844858,342.0172589)(236.849858,341.96225895)(236.83986654,341.90226336)
\curveto(236.82985802,341.84225907)(236.83485801,341.78725913)(236.85486654,341.73726336)
\moveto(234.75486654,341.19726336)
\curveto(234.76486008,341.24725967)(234.76986008,341.3172596)(234.76986654,341.40726336)
\curveto(234.76986008,341.50725941)(234.76486008,341.58225933)(234.75486654,341.63226336)
\lineto(234.75486654,341.75226336)
\curveto(234.73486011,341.80225911)(234.72486012,341.85725906)(234.72486654,341.91726336)
\curveto(234.72486012,341.97725894)(234.71986013,342.03225888)(234.70986654,342.08226336)
\curveto(234.70986014,342.12225879)(234.70486014,342.15225876)(234.69486654,342.17226336)
\lineto(234.63486654,342.41226336)
\curveto(234.62486022,342.50225841)(234.60486024,342.58725833)(234.57486654,342.66726336)
\curveto(234.46486038,342.92725799)(234.33486051,343.14725777)(234.18486654,343.32726336)
\curveto(234.03486081,343.5172574)(233.83486101,343.66725725)(233.58486654,343.77726336)
\curveto(233.52486132,343.79725712)(233.46486138,343.8122571)(233.40486654,343.82226336)
\curveto(233.3448615,343.84225707)(233.27986157,343.86225705)(233.20986654,343.88226336)
\curveto(233.12986172,343.90225701)(233.0448618,343.90725701)(232.95486654,343.89726336)
\lineto(232.68486654,343.89726336)
\curveto(232.65486219,343.87725704)(232.61986223,343.86725705)(232.57986654,343.86726336)
\curveto(232.53986231,343.87725704)(232.50486234,343.87725704)(232.47486654,343.86726336)
\lineto(232.26486654,343.80726336)
\curveto(232.20486264,343.79725712)(232.1498627,343.77725714)(232.09986654,343.74726336)
\curveto(231.849863,343.63725728)(231.6448632,343.47725744)(231.48486654,343.26726336)
\curveto(231.33486351,343.06725785)(231.21486363,342.83225808)(231.12486654,342.56226336)
\curveto(231.09486375,342.46225845)(231.06986378,342.35725856)(231.04986654,342.24726336)
\curveto(231.03986381,342.13725878)(231.02486382,342.02725889)(231.00486654,341.91726336)
\curveto(230.99486385,341.86725905)(230.98986386,341.8172591)(230.98986654,341.76726336)
\lineto(230.98986654,341.61726336)
\curveto(230.96986388,341.54725937)(230.95986389,341.44225947)(230.95986654,341.30226336)
\curveto(230.96986388,341.16225975)(230.98486386,341.05725986)(231.00486654,340.98726336)
\lineto(231.00486654,340.85226336)
\curveto(231.02486382,340.77226014)(231.03986381,340.69226022)(231.04986654,340.61226336)
\curveto(231.05986379,340.54226037)(231.07486377,340.46726045)(231.09486654,340.38726336)
\curveto(231.19486365,340.08726083)(231.29986355,339.84226107)(231.40986654,339.65226336)
\curveto(231.52986332,339.47226144)(231.71486313,339.30726161)(231.96486654,339.15726336)
\curveto(232.03486281,339.10726181)(232.10986274,339.06726185)(232.18986654,339.03726336)
\curveto(232.27986257,339.00726191)(232.36986248,338.98226193)(232.45986654,338.96226336)
\curveto(232.49986235,338.95226196)(232.53486231,338.94726197)(232.56486654,338.94726336)
\curveto(232.59486225,338.95726196)(232.62986222,338.95726196)(232.66986654,338.94726336)
\lineto(232.78986654,338.91726336)
\curveto(232.83986201,338.917262)(232.88486196,338.92226199)(232.92486654,338.93226336)
\lineto(233.04486654,338.93226336)
\curveto(233.12486172,338.95226196)(233.20486164,338.96726195)(233.28486654,338.97726336)
\curveto(233.36486148,338.98726193)(233.43986141,339.00726191)(233.50986654,339.03726336)
\curveto(233.76986108,339.13726178)(233.97986087,339.27226164)(234.13986654,339.44226336)
\curveto(234.29986055,339.6122613)(234.43486041,339.82226109)(234.54486654,340.07226336)
\curveto(234.58486026,340.17226074)(234.61486023,340.27226064)(234.63486654,340.37226336)
\curveto(234.65486019,340.47226044)(234.67986017,340.57726034)(234.70986654,340.68726336)
\curveto(234.71986013,340.72726019)(234.72486012,340.76226015)(234.72486654,340.79226336)
\curveto(234.72486012,340.83226008)(234.72986012,340.87226004)(234.73986654,340.91226336)
\lineto(234.73986654,341.04726336)
\curveto(234.73986011,341.09725982)(234.7448601,341.14725977)(234.75486654,341.19726336)
}
}
{
\newrgbcolor{curcolor}{0 0 0}
\pscustom[linestyle=none,fillstyle=solid,fillcolor=curcolor]
{
\newpath
\moveto(241.22478842,345.50226336)
\curveto(241.97478392,345.52225539)(242.62478327,345.43725548)(243.17478842,345.24726336)
\curveto(243.73478216,345.06725585)(244.15978173,344.75225616)(244.44978842,344.30226336)
\curveto(244.51978137,344.19225672)(244.57978131,344.07725684)(244.62978842,343.95726336)
\curveto(244.6897812,343.84725707)(244.73978115,343.72225719)(244.77978842,343.58226336)
\curveto(244.79978109,343.52225739)(244.80978108,343.45725746)(244.80978842,343.38726336)
\curveto(244.80978108,343.3172576)(244.79978109,343.25725766)(244.77978842,343.20726336)
\curveto(244.73978115,343.14725777)(244.68478121,343.10725781)(244.61478842,343.08726336)
\curveto(244.56478133,343.06725785)(244.50478139,343.05725786)(244.43478842,343.05726336)
\lineto(244.22478842,343.05726336)
\lineto(243.56478842,343.05726336)
\curveto(243.4947824,343.05725786)(243.42478247,343.05225786)(243.35478842,343.04226336)
\curveto(243.28478261,343.04225787)(243.21978267,343.05225786)(243.15978842,343.07226336)
\curveto(243.05978283,343.09225782)(242.98478291,343.13225778)(242.93478842,343.19226336)
\curveto(242.88478301,343.25225766)(242.83978305,343.3122576)(242.79978842,343.37226336)
\lineto(242.67978842,343.58226336)
\curveto(242.64978324,343.66225725)(242.59978329,343.72725719)(242.52978842,343.77726336)
\curveto(242.42978346,343.85725706)(242.32978356,343.917257)(242.22978842,343.95726336)
\curveto(242.13978375,343.99725692)(242.02478387,344.03225688)(241.88478842,344.06226336)
\curveto(241.81478408,344.08225683)(241.70978418,344.09725682)(241.56978842,344.10726336)
\curveto(241.43978445,344.1172568)(241.33978455,344.1122568)(241.26978842,344.09226336)
\lineto(241.16478842,344.09226336)
\lineto(241.01478842,344.06226336)
\curveto(240.97478492,344.06225685)(240.92978496,344.05725686)(240.87978842,344.04726336)
\curveto(240.70978518,343.99725692)(240.56978532,343.92725699)(240.45978842,343.83726336)
\curveto(240.35978553,343.75725716)(240.2897856,343.63225728)(240.24978842,343.46226336)
\curveto(240.22978566,343.39225752)(240.22978566,343.32725759)(240.24978842,343.26726336)
\curveto(240.26978562,343.20725771)(240.2897856,343.15725776)(240.30978842,343.11726336)
\curveto(240.37978551,342.99725792)(240.45978543,342.90225801)(240.54978842,342.83226336)
\curveto(240.64978524,342.76225815)(240.76478513,342.70225821)(240.89478842,342.65226336)
\curveto(241.08478481,342.57225834)(241.2897846,342.50225841)(241.50978842,342.44226336)
\lineto(242.19978842,342.29226336)
\curveto(242.43978345,342.25225866)(242.66978322,342.20225871)(242.88978842,342.14226336)
\curveto(243.11978277,342.09225882)(243.33478256,342.02725889)(243.53478842,341.94726336)
\curveto(243.62478227,341.90725901)(243.70978218,341.87225904)(243.78978842,341.84226336)
\curveto(243.87978201,341.82225909)(243.96478193,341.78725913)(244.04478842,341.73726336)
\curveto(244.23478166,341.6172593)(244.40478149,341.48725943)(244.55478842,341.34726336)
\curveto(244.71478118,341.20725971)(244.83978105,341.03225988)(244.92978842,340.82226336)
\curveto(244.95978093,340.75226016)(244.98478091,340.68226023)(245.00478842,340.61226336)
\curveto(245.02478087,340.54226037)(245.04478085,340.46726045)(245.06478842,340.38726336)
\curveto(245.07478082,340.32726059)(245.07978081,340.23226068)(245.07978842,340.10226336)
\curveto(245.0897808,339.98226093)(245.0897808,339.88726103)(245.07978842,339.81726336)
\lineto(245.07978842,339.74226336)
\curveto(245.05978083,339.68226123)(245.04478085,339.62226129)(245.03478842,339.56226336)
\curveto(245.03478086,339.5122614)(245.02978086,339.46226145)(245.01978842,339.41226336)
\curveto(244.94978094,339.1122618)(244.83978105,338.84726207)(244.68978842,338.61726336)
\curveto(244.52978136,338.37726254)(244.33478156,338.18226273)(244.10478842,338.03226336)
\curveto(243.87478202,337.88226303)(243.61478228,337.75226316)(243.32478842,337.64226336)
\curveto(243.21478268,337.59226332)(243.0947828,337.55726336)(242.96478842,337.53726336)
\curveto(242.84478305,337.5172634)(242.72478317,337.49226342)(242.60478842,337.46226336)
\curveto(242.51478338,337.44226347)(242.41978347,337.43226348)(242.31978842,337.43226336)
\curveto(242.22978366,337.42226349)(242.13978375,337.40726351)(242.04978842,337.38726336)
\lineto(241.77978842,337.38726336)
\curveto(241.71978417,337.36726355)(241.61478428,337.35726356)(241.46478842,337.35726336)
\curveto(241.32478457,337.35726356)(241.22478467,337.36726355)(241.16478842,337.38726336)
\curveto(241.13478476,337.38726353)(241.09978479,337.39226352)(241.05978842,337.40226336)
\lineto(240.95478842,337.40226336)
\curveto(240.83478506,337.42226349)(240.71478518,337.43726348)(240.59478842,337.44726336)
\curveto(240.47478542,337.45726346)(240.35978553,337.47726344)(240.24978842,337.50726336)
\curveto(239.85978603,337.6172633)(239.51478638,337.74226317)(239.21478842,337.88226336)
\curveto(238.91478698,338.03226288)(238.65978723,338.25226266)(238.44978842,338.54226336)
\curveto(238.30978758,338.73226218)(238.1897877,338.95226196)(238.08978842,339.20226336)
\curveto(238.06978782,339.26226165)(238.04978784,339.34226157)(238.02978842,339.44226336)
\curveto(238.00978788,339.49226142)(237.9947879,339.56226135)(237.98478842,339.65226336)
\curveto(237.97478792,339.74226117)(237.97978791,339.8172611)(237.99978842,339.87726336)
\curveto(238.02978786,339.94726097)(238.07978781,339.99726092)(238.14978842,340.02726336)
\curveto(238.19978769,340.04726087)(238.25978763,340.05726086)(238.32978842,340.05726336)
\lineto(238.55478842,340.05726336)
\lineto(239.25978842,340.05726336)
\lineto(239.49978842,340.05726336)
\curveto(239.57978631,340.05726086)(239.64978624,340.04726087)(239.70978842,340.02726336)
\curveto(239.81978607,339.98726093)(239.889786,339.92226099)(239.91978842,339.83226336)
\curveto(239.95978593,339.74226117)(240.00478589,339.64726127)(240.05478842,339.54726336)
\curveto(240.07478582,339.49726142)(240.10978578,339.43226148)(240.15978842,339.35226336)
\curveto(240.21978567,339.27226164)(240.26978562,339.22226169)(240.30978842,339.20226336)
\curveto(240.42978546,339.10226181)(240.54478535,339.02226189)(240.65478842,338.96226336)
\curveto(240.76478513,338.912262)(240.90478499,338.86226205)(241.07478842,338.81226336)
\curveto(241.12478477,338.79226212)(241.17478472,338.78226213)(241.22478842,338.78226336)
\curveto(241.27478462,338.79226212)(241.32478457,338.79226212)(241.37478842,338.78226336)
\curveto(241.45478444,338.76226215)(241.53978435,338.75226216)(241.62978842,338.75226336)
\curveto(241.72978416,338.76226215)(241.81478408,338.77726214)(241.88478842,338.79726336)
\curveto(241.93478396,338.80726211)(241.97978391,338.8122621)(242.01978842,338.81226336)
\curveto(242.06978382,338.8122621)(242.11978377,338.82226209)(242.16978842,338.84226336)
\curveto(242.30978358,338.89226202)(242.43478346,338.95226196)(242.54478842,339.02226336)
\curveto(242.66478323,339.09226182)(242.75978313,339.18226173)(242.82978842,339.29226336)
\curveto(242.87978301,339.37226154)(242.91978297,339.49726142)(242.94978842,339.66726336)
\curveto(242.96978292,339.73726118)(242.96978292,339.80226111)(242.94978842,339.86226336)
\curveto(242.92978296,339.92226099)(242.90978298,339.97226094)(242.88978842,340.01226336)
\curveto(242.81978307,340.15226076)(242.72978316,340.25726066)(242.61978842,340.32726336)
\curveto(242.51978337,340.39726052)(242.39978349,340.46226045)(242.25978842,340.52226336)
\curveto(242.06978382,340.60226031)(241.86978402,340.66726025)(241.65978842,340.71726336)
\curveto(241.44978444,340.76726015)(241.23978465,340.82226009)(241.02978842,340.88226336)
\curveto(240.94978494,340.90226001)(240.86478503,340.91726)(240.77478842,340.92726336)
\curveto(240.6947852,340.93725998)(240.61478528,340.95225996)(240.53478842,340.97226336)
\curveto(240.21478568,341.06225985)(239.90978598,341.14725977)(239.61978842,341.22726336)
\curveto(239.32978656,341.3172596)(239.06478683,341.44725947)(238.82478842,341.61726336)
\curveto(238.54478735,341.8172591)(238.33978755,342.08725883)(238.20978842,342.42726336)
\curveto(238.1897877,342.49725842)(238.16978772,342.59225832)(238.14978842,342.71226336)
\curveto(238.12978776,342.78225813)(238.11478778,342.86725805)(238.10478842,342.96726336)
\curveto(238.0947878,343.06725785)(238.09978779,343.15725776)(238.11978842,343.23726336)
\curveto(238.13978775,343.28725763)(238.14478775,343.32725759)(238.13478842,343.35726336)
\curveto(238.12478777,343.39725752)(238.12978776,343.44225747)(238.14978842,343.49226336)
\curveto(238.16978772,343.60225731)(238.1897877,343.70225721)(238.20978842,343.79226336)
\curveto(238.23978765,343.89225702)(238.27478762,343.98725693)(238.31478842,344.07726336)
\curveto(238.44478745,344.36725655)(238.62478727,344.60225631)(238.85478842,344.78226336)
\curveto(239.08478681,344.96225595)(239.34478655,345.10725581)(239.63478842,345.21726336)
\curveto(239.74478615,345.26725565)(239.85978603,345.30225561)(239.97978842,345.32226336)
\curveto(240.09978579,345.35225556)(240.22478567,345.38225553)(240.35478842,345.41226336)
\curveto(240.41478548,345.43225548)(240.47478542,345.44225547)(240.53478842,345.44226336)
\lineto(240.71478842,345.47226336)
\curveto(240.7947851,345.48225543)(240.87978501,345.48725543)(240.96978842,345.48726336)
\curveto(241.05978483,345.48725543)(241.14478475,345.49225542)(241.22478842,345.50226336)
}
}
{
\newrgbcolor{curcolor}{0 0 0}
\pscustom[linestyle=none,fillstyle=solid,fillcolor=curcolor]
{
\newpath
\moveto(26.50611654,322.76226336)
\curveto(26.51610786,322.70225946)(26.52110786,322.61225955)(26.52111654,322.49226336)
\curveto(26.52110786,322.37225979)(26.51110787,322.28725988)(26.49111654,322.23726336)
\lineto(26.49111654,322.04226336)
\curveto(26.46110792,321.93226023)(26.44110794,321.82726034)(26.43111654,321.72726336)
\curveto(26.43110795,321.62726054)(26.41610796,321.52726064)(26.38611654,321.42726336)
\curveto(26.36610801,321.33726083)(26.34610803,321.24226092)(26.32611654,321.14226336)
\curveto(26.30610807,321.05226111)(26.2761081,320.9622612)(26.23611654,320.87226336)
\curveto(26.16610821,320.70226146)(26.09610828,320.54226162)(26.02611654,320.39226336)
\curveto(25.95610842,320.25226191)(25.8761085,320.11226205)(25.78611654,319.97226336)
\curveto(25.72610865,319.88226228)(25.66110872,319.79726237)(25.59111654,319.71726336)
\curveto(25.53110885,319.64726252)(25.46110892,319.57226259)(25.38111654,319.49226336)
\lineto(25.27611654,319.38726336)
\curveto(25.22610915,319.33726283)(25.17110921,319.29226287)(25.11111654,319.25226336)
\lineto(24.96111654,319.13226336)
\curveto(24.8811095,319.07226309)(24.79110959,319.01726315)(24.69111654,318.96726336)
\curveto(24.60110978,318.92726324)(24.50610987,318.88226328)(24.40611654,318.83226336)
\curveto(24.30611007,318.78226338)(24.20111018,318.74726342)(24.09111654,318.72726336)
\curveto(23.99111039,318.70726346)(23.88611049,318.68726348)(23.77611654,318.66726336)
\curveto(23.71611066,318.64726352)(23.65111073,318.63726353)(23.58111654,318.63726336)
\curveto(23.52111086,318.63726353)(23.45611092,318.62726354)(23.38611654,318.60726336)
\lineto(23.25111654,318.60726336)
\curveto(23.17111121,318.58726358)(23.09611128,318.58726358)(23.02611654,318.60726336)
\lineto(22.87611654,318.60726336)
\curveto(22.81611156,318.62726354)(22.75111163,318.63726353)(22.68111654,318.63726336)
\curveto(22.62111176,318.62726354)(22.56111182,318.63226353)(22.50111654,318.65226336)
\curveto(22.34111204,318.70226346)(22.18611219,318.74726342)(22.03611654,318.78726336)
\curveto(21.89611248,318.82726334)(21.76611261,318.88726328)(21.64611654,318.96726336)
\curveto(21.5761128,319.00726316)(21.51111287,319.04726312)(21.45111654,319.08726336)
\curveto(21.39111299,319.13726303)(21.32611305,319.18726298)(21.25611654,319.23726336)
\lineto(21.07611654,319.37226336)
\curveto(20.99611338,319.43226273)(20.92611345,319.43726273)(20.86611654,319.38726336)
\curveto(20.81611356,319.35726281)(20.79111359,319.31726285)(20.79111654,319.26726336)
\curveto(20.79111359,319.22726294)(20.7811136,319.17726299)(20.76111654,319.11726336)
\curveto(20.74111364,319.01726315)(20.73111365,318.90226326)(20.73111654,318.77226336)
\curveto(20.74111364,318.64226352)(20.74611363,318.52226364)(20.74611654,318.41226336)
\lineto(20.74611654,316.88226336)
\curveto(20.74611363,316.75226541)(20.74111364,316.62726554)(20.73111654,316.50726336)
\curveto(20.73111365,316.37726579)(20.70611367,316.27226589)(20.65611654,316.19226336)
\curveto(20.62611375,316.15226601)(20.57111381,316.12226604)(20.49111654,316.10226336)
\curveto(20.41111397,316.08226608)(20.32111406,316.07226609)(20.22111654,316.07226336)
\curveto(20.12111426,316.0622661)(20.02111436,316.0622661)(19.92111654,316.07226336)
\lineto(19.66611654,316.07226336)
\lineto(19.26111654,316.07226336)
\lineto(19.15611654,316.07226336)
\curveto(19.11611526,316.07226609)(19.0811153,316.07726609)(19.05111654,316.08726336)
\lineto(18.93111654,316.08726336)
\curveto(18.76111562,316.13726603)(18.67111571,316.23726593)(18.66111654,316.38726336)
\curveto(18.65111573,316.52726564)(18.64611573,316.69726547)(18.64611654,316.89726336)
\lineto(18.64611654,325.70226336)
\curveto(18.64611573,325.81225635)(18.64111574,325.92725624)(18.63111654,326.04726336)
\curveto(18.63111575,326.17725599)(18.65611572,326.27725589)(18.70611654,326.34726336)
\curveto(18.74611563,326.41725575)(18.80111558,326.4622557)(18.87111654,326.48226336)
\curveto(18.92111546,326.50225566)(18.9811154,326.51225565)(19.05111654,326.51226336)
\lineto(19.27611654,326.51226336)
\lineto(19.99611654,326.51226336)
\lineto(20.28111654,326.51226336)
\curveto(20.37111401,326.51225565)(20.44611393,326.48725568)(20.50611654,326.43726336)
\curveto(20.5761138,326.38725578)(20.61111377,326.32225584)(20.61111654,326.24226336)
\curveto(20.62111376,326.17225599)(20.64611373,326.09725607)(20.68611654,326.01726336)
\curveto(20.69611368,325.98725618)(20.70611367,325.9622562)(20.71611654,325.94226336)
\curveto(20.73611364,325.93225623)(20.75611362,325.91725625)(20.77611654,325.89726336)
\curveto(20.88611349,325.88725628)(20.9761134,325.91725625)(21.04611654,325.98726336)
\curveto(21.11611326,326.05725611)(21.18611319,326.11725605)(21.25611654,326.16726336)
\curveto(21.38611299,326.25725591)(21.52111286,326.33725583)(21.66111654,326.40726336)
\curveto(21.80111258,326.48725568)(21.95611242,326.55225561)(22.12611654,326.60226336)
\curveto(22.20611217,326.63225553)(22.29111209,326.65225551)(22.38111654,326.66226336)
\curveto(22.4811119,326.67225549)(22.5761118,326.68725548)(22.66611654,326.70726336)
\curveto(22.70611167,326.71725545)(22.74611163,326.71725545)(22.78611654,326.70726336)
\curveto(22.83611154,326.69725547)(22.8761115,326.70225546)(22.90611654,326.72226336)
\curveto(23.4761109,326.74225542)(23.95611042,326.6622555)(24.34611654,326.48226336)
\curveto(24.74610963,326.31225585)(25.08610929,326.08725608)(25.36611654,325.80726336)
\curveto(25.41610896,325.75725641)(25.46110892,325.70725646)(25.50111654,325.65726336)
\curveto(25.54110884,325.61725655)(25.5811088,325.57225659)(25.62111654,325.52226336)
\curveto(25.69110869,325.43225673)(25.75110863,325.34225682)(25.80111654,325.25226336)
\curveto(25.86110852,325.162257)(25.91610846,325.07225709)(25.96611654,324.98226336)
\curveto(25.98610839,324.9622572)(25.99610838,324.93725723)(25.99611654,324.90726336)
\curveto(26.00610837,324.87725729)(26.02110836,324.84225732)(26.04111654,324.80226336)
\curveto(26.10110828,324.70225746)(26.15610822,324.58225758)(26.20611654,324.44226336)
\curveto(26.22610815,324.38225778)(26.24610813,324.31725785)(26.26611654,324.24726336)
\curveto(26.28610809,324.18725798)(26.30610807,324.12225804)(26.32611654,324.05226336)
\curveto(26.36610801,323.93225823)(26.39110799,323.80725836)(26.40111654,323.67726336)
\curveto(26.42110796,323.54725862)(26.44610793,323.41225875)(26.47611654,323.27226336)
\lineto(26.47611654,323.10726336)
\lineto(26.50611654,322.92726336)
\lineto(26.50611654,322.76226336)
\moveto(24.39111654,322.41726336)
\curveto(24.40110998,322.4672597)(24.40610997,322.53225963)(24.40611654,322.61226336)
\curveto(24.40610997,322.70225946)(24.40110998,322.77225939)(24.39111654,322.82226336)
\lineto(24.39111654,322.95726336)
\curveto(24.37111001,323.01725915)(24.36111002,323.08225908)(24.36111654,323.15226336)
\curveto(24.36111002,323.22225894)(24.35111003,323.29225887)(24.33111654,323.36226336)
\curveto(24.31111007,323.4622587)(24.29111009,323.55725861)(24.27111654,323.64726336)
\curveto(24.25111013,323.74725842)(24.22111016,323.83725833)(24.18111654,323.91726336)
\curveto(24.06111032,324.23725793)(23.90611047,324.49225767)(23.71611654,324.68226336)
\curveto(23.52611085,324.87225729)(23.25611112,325.01225715)(22.90611654,325.10226336)
\curveto(22.82611155,325.12225704)(22.73611164,325.13225703)(22.63611654,325.13226336)
\lineto(22.36611654,325.13226336)
\curveto(22.32611205,325.12225704)(22.29111209,325.11725705)(22.26111654,325.11726336)
\curveto(22.23111215,325.11725705)(22.19611218,325.11225705)(22.15611654,325.10226336)
\lineto(21.94611654,325.04226336)
\curveto(21.88611249,325.03225713)(21.82611255,325.01225715)(21.76611654,324.98226336)
\curveto(21.50611287,324.87225729)(21.30111308,324.70225746)(21.15111654,324.47226336)
\curveto(21.01111337,324.24225792)(20.89611348,323.98725818)(20.80611654,323.70726336)
\curveto(20.78611359,323.62725854)(20.77111361,323.54225862)(20.76111654,323.45226336)
\curveto(20.75111363,323.37225879)(20.73611364,323.29225887)(20.71611654,323.21226336)
\curveto(20.70611367,323.17225899)(20.70111368,323.10725906)(20.70111654,323.01726336)
\curveto(20.6811137,322.97725919)(20.6761137,322.92725924)(20.68611654,322.86726336)
\curveto(20.69611368,322.81725935)(20.69611368,322.7672594)(20.68611654,322.71726336)
\curveto(20.66611371,322.65725951)(20.66611371,322.60225956)(20.68611654,322.55226336)
\lineto(20.68611654,322.37226336)
\lineto(20.68611654,322.23726336)
\curveto(20.68611369,322.19725997)(20.69611368,322.15726001)(20.71611654,322.11726336)
\curveto(20.71611366,322.04726012)(20.72111366,321.99226017)(20.73111654,321.95226336)
\lineto(20.76111654,321.77226336)
\curveto(20.77111361,321.71226045)(20.78611359,321.65226051)(20.80611654,321.59226336)
\curveto(20.89611348,321.30226086)(21.00111338,321.0622611)(21.12111654,320.87226336)
\curveto(21.25111313,320.69226147)(21.43111295,320.53226163)(21.66111654,320.39226336)
\curveto(21.80111258,320.31226185)(21.96611241,320.24726192)(22.15611654,320.19726336)
\curveto(22.19611218,320.18726198)(22.23111215,320.18226198)(22.26111654,320.18226336)
\curveto(22.29111209,320.19226197)(22.32611205,320.19226197)(22.36611654,320.18226336)
\curveto(22.40611197,320.17226199)(22.46611191,320.162262)(22.54611654,320.15226336)
\curveto(22.62611175,320.15226201)(22.69111169,320.15726201)(22.74111654,320.16726336)
\curveto(22.82111156,320.18726198)(22.90111148,320.20226196)(22.98111654,320.21226336)
\curveto(23.07111131,320.23226193)(23.15611122,320.25726191)(23.23611654,320.28726336)
\curveto(23.4761109,320.38726178)(23.67111071,320.52726164)(23.82111654,320.70726336)
\curveto(23.97111041,320.88726128)(24.09611028,321.09726107)(24.19611654,321.33726336)
\curveto(24.24611013,321.45726071)(24.2811101,321.58226058)(24.30111654,321.71226336)
\curveto(24.32111006,321.84226032)(24.34611003,321.97726019)(24.37611654,322.11726336)
\lineto(24.37611654,322.26726336)
\curveto(24.38610999,322.31725985)(24.39110999,322.3672598)(24.39111654,322.41726336)
}
}
{
\newrgbcolor{curcolor}{0 0 0}
\pscustom[linestyle=none,fillstyle=solid,fillcolor=curcolor]
{
\newpath
\moveto(35.55603842,322.98726336)
\curveto(35.57602985,322.92725924)(35.58602984,322.84225932)(35.58603842,322.73226336)
\curveto(35.58602984,322.62225954)(35.57602985,322.53725963)(35.55603842,322.47726336)
\lineto(35.55603842,322.32726336)
\curveto(35.53602989,322.24725992)(35.5260299,322.16726)(35.52603842,322.08726336)
\curveto(35.53602989,322.00726016)(35.53102989,321.92726024)(35.51103842,321.84726336)
\curveto(35.49102993,321.77726039)(35.47602995,321.71226045)(35.46603842,321.65226336)
\curveto(35.45602997,321.59226057)(35.44602998,321.52726064)(35.43603842,321.45726336)
\curveto(35.39603003,321.34726082)(35.36103006,321.23226093)(35.33103842,321.11226336)
\curveto(35.30103012,321.00226116)(35.26103016,320.89726127)(35.21103842,320.79726336)
\curveto(35.00103042,320.31726185)(34.7260307,319.92726224)(34.38603842,319.62726336)
\curveto(34.04603138,319.32726284)(33.63603179,319.07726309)(33.15603842,318.87726336)
\curveto(33.03603239,318.82726334)(32.91103251,318.79226337)(32.78103842,318.77226336)
\curveto(32.66103276,318.74226342)(32.53603289,318.71226345)(32.40603842,318.68226336)
\curveto(32.35603307,318.6622635)(32.30103312,318.65226351)(32.24103842,318.65226336)
\curveto(32.18103324,318.65226351)(32.1260333,318.64726352)(32.07603842,318.63726336)
\lineto(31.97103842,318.63726336)
\curveto(31.94103348,318.62726354)(31.91103351,318.62226354)(31.88103842,318.62226336)
\curveto(31.83103359,318.61226355)(31.75103367,318.60726356)(31.64103842,318.60726336)
\curveto(31.53103389,318.59726357)(31.44603398,318.60226356)(31.38603842,318.62226336)
\lineto(31.23603842,318.62226336)
\curveto(31.18603424,318.63226353)(31.13103429,318.63726353)(31.07103842,318.63726336)
\curveto(31.0210344,318.62726354)(30.97103445,318.63226353)(30.92103842,318.65226336)
\curveto(30.88103454,318.6622635)(30.84103458,318.6672635)(30.80103842,318.66726336)
\curveto(30.77103465,318.6672635)(30.73103469,318.67226349)(30.68103842,318.68226336)
\curveto(30.58103484,318.71226345)(30.48103494,318.73726343)(30.38103842,318.75726336)
\curveto(30.28103514,318.77726339)(30.18603524,318.80726336)(30.09603842,318.84726336)
\curveto(29.97603545,318.88726328)(29.86103556,318.92726324)(29.75103842,318.96726336)
\curveto(29.65103577,319.00726316)(29.54603588,319.05726311)(29.43603842,319.11726336)
\curveto(29.08603634,319.32726284)(28.78603664,319.57226259)(28.53603842,319.85226336)
\curveto(28.28603714,320.13226203)(28.07603735,320.4672617)(27.90603842,320.85726336)
\curveto(27.85603757,320.94726122)(27.81603761,321.04226112)(27.78603842,321.14226336)
\curveto(27.76603766,321.24226092)(27.74103768,321.34726082)(27.71103842,321.45726336)
\curveto(27.69103773,321.50726066)(27.68103774,321.55226061)(27.68103842,321.59226336)
\curveto(27.68103774,321.63226053)(27.67103775,321.67726049)(27.65103842,321.72726336)
\curveto(27.63103779,321.80726036)(27.6210378,321.88726028)(27.62103842,321.96726336)
\curveto(27.6210378,322.05726011)(27.61103781,322.14226002)(27.59103842,322.22226336)
\curveto(27.58103784,322.27225989)(27.57603785,322.31725985)(27.57603842,322.35726336)
\lineto(27.57603842,322.49226336)
\curveto(27.55603787,322.55225961)(27.54603788,322.63725953)(27.54603842,322.74726336)
\curveto(27.55603787,322.85725931)(27.57103785,322.94225922)(27.59103842,323.00226336)
\lineto(27.59103842,323.10726336)
\curveto(27.60103782,323.15725901)(27.60103782,323.20725896)(27.59103842,323.25726336)
\curveto(27.59103783,323.31725885)(27.60103782,323.37225879)(27.62103842,323.42226336)
\curveto(27.63103779,323.47225869)(27.63603779,323.51725865)(27.63603842,323.55726336)
\curveto(27.63603779,323.60725856)(27.64603778,323.65725851)(27.66603842,323.70726336)
\curveto(27.70603772,323.83725833)(27.74103768,323.9622582)(27.77103842,324.08226336)
\curveto(27.80103762,324.21225795)(27.84103758,324.33725783)(27.89103842,324.45726336)
\curveto(28.07103735,324.8672573)(28.28603714,325.20725696)(28.53603842,325.47726336)
\curveto(28.78603664,325.75725641)(29.09103633,326.01225615)(29.45103842,326.24226336)
\curveto(29.55103587,326.29225587)(29.65603577,326.33725583)(29.76603842,326.37726336)
\curveto(29.87603555,326.41725575)(29.98603544,326.4622557)(30.09603842,326.51226336)
\curveto(30.2260352,326.5622556)(30.36103506,326.59725557)(30.50103842,326.61726336)
\curveto(30.64103478,326.63725553)(30.78603464,326.6672555)(30.93603842,326.70726336)
\curveto(31.01603441,326.71725545)(31.09103433,326.72225544)(31.16103842,326.72226336)
\curveto(31.23103419,326.72225544)(31.30103412,326.72725544)(31.37103842,326.73726336)
\curveto(31.95103347,326.74725542)(32.45103297,326.68725548)(32.87103842,326.55726336)
\curveto(33.30103212,326.42725574)(33.68103174,326.24725592)(34.01103842,326.01726336)
\curveto(34.1210313,325.93725623)(34.23103119,325.84725632)(34.34103842,325.74726336)
\curveto(34.46103096,325.65725651)(34.56103086,325.55725661)(34.64103842,325.44726336)
\curveto(34.7210307,325.34725682)(34.79103063,325.24725692)(34.85103842,325.14726336)
\curveto(34.9210305,325.04725712)(34.99103043,324.94225722)(35.06103842,324.83226336)
\curveto(35.13103029,324.72225744)(35.18603024,324.60225756)(35.22603842,324.47226336)
\curveto(35.26603016,324.35225781)(35.31103011,324.22225794)(35.36103842,324.08226336)
\curveto(35.39103003,324.00225816)(35.41603001,323.91725825)(35.43603842,323.82726336)
\lineto(35.49603842,323.55726336)
\curveto(35.50602992,323.51725865)(35.51102991,323.47725869)(35.51103842,323.43726336)
\curveto(35.51102991,323.39725877)(35.51602991,323.35725881)(35.52603842,323.31726336)
\curveto(35.54602988,323.2672589)(35.55102987,323.21225895)(35.54103842,323.15226336)
\curveto(35.53102989,323.09225907)(35.53602989,323.03725913)(35.55603842,322.98726336)
\moveto(33.45603842,322.44726336)
\curveto(33.46603196,322.49725967)(33.47103195,322.5672596)(33.47103842,322.65726336)
\curveto(33.47103195,322.75725941)(33.46603196,322.83225933)(33.45603842,322.88226336)
\lineto(33.45603842,323.00226336)
\curveto(33.43603199,323.05225911)(33.426032,323.10725906)(33.42603842,323.16726336)
\curveto(33.426032,323.22725894)(33.421032,323.28225888)(33.41103842,323.33226336)
\curveto(33.41103201,323.37225879)(33.40603202,323.40225876)(33.39603842,323.42226336)
\lineto(33.33603842,323.66226336)
\curveto(33.3260321,323.75225841)(33.30603212,323.83725833)(33.27603842,323.91726336)
\curveto(33.16603226,324.17725799)(33.03603239,324.39725777)(32.88603842,324.57726336)
\curveto(32.73603269,324.7672574)(32.53603289,324.91725725)(32.28603842,325.02726336)
\curveto(32.2260332,325.04725712)(32.16603326,325.0622571)(32.10603842,325.07226336)
\curveto(32.04603338,325.09225707)(31.98103344,325.11225705)(31.91103842,325.13226336)
\curveto(31.83103359,325.15225701)(31.74603368,325.15725701)(31.65603842,325.14726336)
\lineto(31.38603842,325.14726336)
\curveto(31.35603407,325.12725704)(31.3210341,325.11725705)(31.28103842,325.11726336)
\curveto(31.24103418,325.12725704)(31.20603422,325.12725704)(31.17603842,325.11726336)
\lineto(30.96603842,325.05726336)
\curveto(30.90603452,325.04725712)(30.85103457,325.02725714)(30.80103842,324.99726336)
\curveto(30.55103487,324.88725728)(30.34603508,324.72725744)(30.18603842,324.51726336)
\curveto(30.03603539,324.31725785)(29.91603551,324.08225808)(29.82603842,323.81226336)
\curveto(29.79603563,323.71225845)(29.77103565,323.60725856)(29.75103842,323.49726336)
\curveto(29.74103568,323.38725878)(29.7260357,323.27725889)(29.70603842,323.16726336)
\curveto(29.69603573,323.11725905)(29.69103573,323.0672591)(29.69103842,323.01726336)
\lineto(29.69103842,322.86726336)
\curveto(29.67103575,322.79725937)(29.66103576,322.69225947)(29.66103842,322.55226336)
\curveto(29.67103575,322.41225975)(29.68603574,322.30725986)(29.70603842,322.23726336)
\lineto(29.70603842,322.10226336)
\curveto(29.7260357,322.02226014)(29.74103568,321.94226022)(29.75103842,321.86226336)
\curveto(29.76103566,321.79226037)(29.77603565,321.71726045)(29.79603842,321.63726336)
\curveto(29.89603553,321.33726083)(30.00103542,321.09226107)(30.11103842,320.90226336)
\curveto(30.23103519,320.72226144)(30.41603501,320.55726161)(30.66603842,320.40726336)
\curveto(30.73603469,320.35726181)(30.81103461,320.31726185)(30.89103842,320.28726336)
\curveto(30.98103444,320.25726191)(31.07103435,320.23226193)(31.16103842,320.21226336)
\curveto(31.20103422,320.20226196)(31.23603419,320.19726197)(31.26603842,320.19726336)
\curveto(31.29603413,320.20726196)(31.33103409,320.20726196)(31.37103842,320.19726336)
\lineto(31.49103842,320.16726336)
\curveto(31.54103388,320.167262)(31.58603384,320.17226199)(31.62603842,320.18226336)
\lineto(31.74603842,320.18226336)
\curveto(31.8260336,320.20226196)(31.90603352,320.21726195)(31.98603842,320.22726336)
\curveto(32.06603336,320.23726193)(32.14103328,320.25726191)(32.21103842,320.28726336)
\curveto(32.47103295,320.38726178)(32.68103274,320.52226164)(32.84103842,320.69226336)
\curveto(33.00103242,320.8622613)(33.13603229,321.07226109)(33.24603842,321.32226336)
\curveto(33.28603214,321.42226074)(33.31603211,321.52226064)(33.33603842,321.62226336)
\curveto(33.35603207,321.72226044)(33.38103204,321.82726034)(33.41103842,321.93726336)
\curveto(33.421032,321.97726019)(33.426032,322.01226015)(33.42603842,322.04226336)
\curveto(33.426032,322.08226008)(33.43103199,322.12226004)(33.44103842,322.16226336)
\lineto(33.44103842,322.29726336)
\curveto(33.44103198,322.34725982)(33.44603198,322.39725977)(33.45603842,322.44726336)
}
}
{
\newrgbcolor{curcolor}{0 0 0}
\pscustom[linestyle=none,fillstyle=solid,fillcolor=curcolor]
{
\newpath
\moveto(41.38096029,326.73726336)
\curveto(41.49095498,326.73725543)(41.58595488,326.72725544)(41.66596029,326.70726336)
\curveto(41.75595471,326.68725548)(41.82595464,326.64225552)(41.87596029,326.57226336)
\curveto(41.93595453,326.49225567)(41.9659545,326.35225581)(41.96596029,326.15226336)
\lineto(41.96596029,325.64226336)
\lineto(41.96596029,325.26726336)
\curveto(41.97595449,325.12725704)(41.96095451,325.01725715)(41.92096029,324.93726336)
\curveto(41.88095459,324.8672573)(41.82095465,324.82225734)(41.74096029,324.80226336)
\curveto(41.6709548,324.78225738)(41.58595488,324.77225739)(41.48596029,324.77226336)
\curveto(41.39595507,324.77225739)(41.29595517,324.77725739)(41.18596029,324.78726336)
\curveto(41.08595538,324.79725737)(40.99095548,324.79225737)(40.90096029,324.77226336)
\curveto(40.83095564,324.75225741)(40.76095571,324.73725743)(40.69096029,324.72726336)
\curveto(40.62095585,324.72725744)(40.55595591,324.71725745)(40.49596029,324.69726336)
\curveto(40.33595613,324.64725752)(40.17595629,324.57225759)(40.01596029,324.47226336)
\curveto(39.85595661,324.38225778)(39.73095674,324.27725789)(39.64096029,324.15726336)
\curveto(39.59095688,324.07725809)(39.53595693,323.99225817)(39.47596029,323.90226336)
\curveto(39.42595704,323.82225834)(39.37595709,323.73725843)(39.32596029,323.64726336)
\curveto(39.29595717,323.5672586)(39.2659572,323.48225868)(39.23596029,323.39226336)
\lineto(39.17596029,323.15226336)
\curveto(39.15595731,323.08225908)(39.14595732,323.00725916)(39.14596029,322.92726336)
\curveto(39.14595732,322.85725931)(39.13595733,322.78725938)(39.11596029,322.71726336)
\curveto(39.10595736,322.67725949)(39.10095737,322.63725953)(39.10096029,322.59726336)
\curveto(39.11095736,322.5672596)(39.11095736,322.53725963)(39.10096029,322.50726336)
\lineto(39.10096029,322.26726336)
\curveto(39.08095739,322.19725997)(39.07595739,322.11726005)(39.08596029,322.02726336)
\curveto(39.09595737,321.94726022)(39.10095737,321.8672603)(39.10096029,321.78726336)
\lineto(39.10096029,320.82726336)
\lineto(39.10096029,319.55226336)
\curveto(39.10095737,319.42226274)(39.09595737,319.30226286)(39.08596029,319.19226336)
\curveto(39.07595739,319.08226308)(39.04595742,318.99226317)(38.99596029,318.92226336)
\curveto(38.97595749,318.89226327)(38.94095753,318.8672633)(38.89096029,318.84726336)
\curveto(38.85095762,318.83726333)(38.80595766,318.82726334)(38.75596029,318.81726336)
\lineto(38.68096029,318.81726336)
\curveto(38.63095784,318.80726336)(38.57595789,318.80226336)(38.51596029,318.80226336)
\lineto(38.35096029,318.80226336)
\lineto(37.70596029,318.80226336)
\curveto(37.64595882,318.81226335)(37.58095889,318.81726335)(37.51096029,318.81726336)
\lineto(37.31596029,318.81726336)
\curveto(37.2659592,318.83726333)(37.21595925,318.85226331)(37.16596029,318.86226336)
\curveto(37.11595935,318.88226328)(37.08095939,318.91726325)(37.06096029,318.96726336)
\curveto(37.02095945,319.01726315)(36.99595947,319.08726308)(36.98596029,319.17726336)
\lineto(36.98596029,319.47726336)
\lineto(36.98596029,320.49726336)
\lineto(36.98596029,324.72726336)
\lineto(36.98596029,325.83726336)
\lineto(36.98596029,326.12226336)
\curveto(36.98595948,326.22225594)(37.00595946,326.30225586)(37.04596029,326.36226336)
\curveto(37.09595937,326.44225572)(37.1709593,326.49225567)(37.27096029,326.51226336)
\curveto(37.3709591,326.53225563)(37.49095898,326.54225562)(37.63096029,326.54226336)
\lineto(38.39596029,326.54226336)
\curveto(38.51595795,326.54225562)(38.62095785,326.53225563)(38.71096029,326.51226336)
\curveto(38.80095767,326.50225566)(38.8709576,326.45725571)(38.92096029,326.37726336)
\curveto(38.95095752,326.32725584)(38.9659575,326.25725591)(38.96596029,326.16726336)
\lineto(38.99596029,325.89726336)
\curveto(39.00595746,325.81725635)(39.02095745,325.74225642)(39.04096029,325.67226336)
\curveto(39.0709574,325.60225656)(39.12095735,325.5672566)(39.19096029,325.56726336)
\curveto(39.21095726,325.58725658)(39.23095724,325.59725657)(39.25096029,325.59726336)
\curveto(39.2709572,325.59725657)(39.29095718,325.60725656)(39.31096029,325.62726336)
\curveto(39.3709571,325.67725649)(39.42095705,325.73225643)(39.46096029,325.79226336)
\curveto(39.51095696,325.8622563)(39.5709569,325.92225624)(39.64096029,325.97226336)
\curveto(39.68095679,326.00225616)(39.71595675,326.03225613)(39.74596029,326.06226336)
\curveto(39.77595669,326.10225606)(39.81095666,326.13725603)(39.85096029,326.16726336)
\lineto(40.12096029,326.34726336)
\curveto(40.22095625,326.40725576)(40.32095615,326.4622557)(40.42096029,326.51226336)
\curveto(40.52095595,326.55225561)(40.62095585,326.58725558)(40.72096029,326.61726336)
\lineto(41.05096029,326.70726336)
\curveto(41.08095539,326.71725545)(41.13595533,326.71725545)(41.21596029,326.70726336)
\curveto(41.30595516,326.70725546)(41.36095511,326.71725545)(41.38096029,326.73726336)
}
}
{
\newrgbcolor{curcolor}{0 0 0}
\pscustom[linestyle=none,fillstyle=solid,fillcolor=curcolor]
{
}
}
{
\newrgbcolor{curcolor}{0 0 0}
\pscustom[linestyle=none,fillstyle=solid,fillcolor=curcolor]
{
\newpath
\moveto(47.99619467,328.85226336)
\lineto(49.00119467,328.85226336)
\curveto(49.15119168,328.85225331)(49.28119155,328.84225332)(49.39119467,328.82226336)
\curveto(49.51119132,328.81225335)(49.59619124,328.75225341)(49.64619467,328.64226336)
\curveto(49.66619117,328.59225357)(49.67619116,328.53225363)(49.67619467,328.46226336)
\lineto(49.67619467,328.25226336)
\lineto(49.67619467,327.57726336)
\curveto(49.67619116,327.52725464)(49.67119116,327.4672547)(49.66119467,327.39726336)
\curveto(49.66119117,327.33725483)(49.66619117,327.28225488)(49.67619467,327.23226336)
\lineto(49.67619467,327.06726336)
\curveto(49.67619116,326.98725518)(49.68119115,326.91225525)(49.69119467,326.84226336)
\curveto(49.70119113,326.78225538)(49.72619111,326.72725544)(49.76619467,326.67726336)
\curveto(49.836191,326.58725558)(49.96119087,326.53725563)(50.14119467,326.52726336)
\lineto(50.68119467,326.52726336)
\lineto(50.86119467,326.52726336)
\curveto(50.92118991,326.52725564)(50.97618986,326.51725565)(51.02619467,326.49726336)
\curveto(51.1361897,326.44725572)(51.19618964,326.35725581)(51.20619467,326.22726336)
\curveto(51.22618961,326.09725607)(51.2361896,325.95225621)(51.23619467,325.79226336)
\lineto(51.23619467,325.58226336)
\curveto(51.24618959,325.51225665)(51.24118959,325.45225671)(51.22119467,325.40226336)
\curveto(51.17118966,325.24225692)(51.06618977,325.15725701)(50.90619467,325.14726336)
\curveto(50.74619009,325.13725703)(50.56619027,325.13225703)(50.36619467,325.13226336)
\lineto(50.23119467,325.13226336)
\curveto(50.19119064,325.14225702)(50.15619068,325.14225702)(50.12619467,325.13226336)
\curveto(50.08619075,325.12225704)(50.05119078,325.11725705)(50.02119467,325.11726336)
\curveto(49.99119084,325.12725704)(49.96119087,325.12225704)(49.93119467,325.10226336)
\curveto(49.85119098,325.08225708)(49.79119104,325.03725713)(49.75119467,324.96726336)
\curveto(49.72119111,324.90725726)(49.69619114,324.83225733)(49.67619467,324.74226336)
\curveto(49.66619117,324.69225747)(49.66619117,324.63725753)(49.67619467,324.57726336)
\curveto(49.68619115,324.51725765)(49.68619115,324.4622577)(49.67619467,324.41226336)
\lineto(49.67619467,323.48226336)
\lineto(49.67619467,321.72726336)
\curveto(49.67619116,321.47726069)(49.68119115,321.25726091)(49.69119467,321.06726336)
\curveto(49.71119112,320.88726128)(49.77619106,320.72726144)(49.88619467,320.58726336)
\curveto(49.9361909,320.52726164)(50.00119083,320.48226168)(50.08119467,320.45226336)
\lineto(50.35119467,320.39226336)
\curveto(50.38119045,320.38226178)(50.41119042,320.37726179)(50.44119467,320.37726336)
\curveto(50.48119035,320.38726178)(50.51119032,320.38726178)(50.53119467,320.37726336)
\lineto(50.69619467,320.37726336)
\curveto(50.80619003,320.37726179)(50.90118993,320.37226179)(50.98119467,320.36226336)
\curveto(51.06118977,320.35226181)(51.12618971,320.31226185)(51.17619467,320.24226336)
\curveto(51.21618962,320.18226198)(51.2361896,320.10226206)(51.23619467,320.00226336)
\lineto(51.23619467,319.71726336)
\curveto(51.2361896,319.50726266)(51.2311896,319.31226285)(51.22119467,319.13226336)
\curveto(51.22118961,318.9622632)(51.14118969,318.84726332)(50.98119467,318.78726336)
\curveto(50.9311899,318.7672634)(50.88618995,318.7622634)(50.84619467,318.77226336)
\curveto(50.80619003,318.77226339)(50.76119007,318.7622634)(50.71119467,318.74226336)
\lineto(50.56119467,318.74226336)
\curveto(50.54119029,318.74226342)(50.51119032,318.74726342)(50.47119467,318.75726336)
\curveto(50.4311904,318.75726341)(50.39619044,318.75226341)(50.36619467,318.74226336)
\curveto(50.31619052,318.73226343)(50.26119057,318.73226343)(50.20119467,318.74226336)
\lineto(50.05119467,318.74226336)
\lineto(49.90119467,318.74226336)
\curveto(49.85119098,318.73226343)(49.80619103,318.73226343)(49.76619467,318.74226336)
\lineto(49.60119467,318.74226336)
\curveto(49.55119128,318.75226341)(49.49619134,318.75726341)(49.43619467,318.75726336)
\curveto(49.37619146,318.75726341)(49.32119151,318.7622634)(49.27119467,318.77226336)
\curveto(49.20119163,318.78226338)(49.1361917,318.79226337)(49.07619467,318.80226336)
\lineto(48.89619467,318.83226336)
\curveto(48.78619205,318.8622633)(48.68119215,318.89726327)(48.58119467,318.93726336)
\curveto(48.48119235,318.97726319)(48.38619245,319.02226314)(48.29619467,319.07226336)
\lineto(48.20619467,319.13226336)
\curveto(48.17619266,319.162263)(48.14119269,319.19226297)(48.10119467,319.22226336)
\curveto(48.08119275,319.24226292)(48.05619278,319.2622629)(48.02619467,319.28226336)
\lineto(47.95119467,319.35726336)
\curveto(47.81119302,319.54726262)(47.70619313,319.75726241)(47.63619467,319.98726336)
\curveto(47.61619322,320.02726214)(47.60619323,320.0622621)(47.60619467,320.09226336)
\curveto(47.61619322,320.13226203)(47.61619322,320.17726199)(47.60619467,320.22726336)
\curveto(47.59619324,320.24726192)(47.59119324,320.27226189)(47.59119467,320.30226336)
\curveto(47.59119324,320.33226183)(47.58619325,320.35726181)(47.57619467,320.37726336)
\lineto(47.57619467,320.52726336)
\curveto(47.56619327,320.5672616)(47.56119327,320.61226155)(47.56119467,320.66226336)
\curveto(47.57119326,320.71226145)(47.57619326,320.7622614)(47.57619467,320.81226336)
\lineto(47.57619467,321.38226336)
\lineto(47.57619467,323.61726336)
\lineto(47.57619467,324.41226336)
\lineto(47.57619467,324.62226336)
\curveto(47.58619325,324.69225747)(47.58119325,324.75725741)(47.56119467,324.81726336)
\curveto(47.52119331,324.95725721)(47.45119338,325.04725712)(47.35119467,325.08726336)
\curveto(47.24119359,325.13725703)(47.10119373,325.15225701)(46.93119467,325.13226336)
\curveto(46.76119407,325.11225705)(46.61619422,325.12725704)(46.49619467,325.17726336)
\curveto(46.41619442,325.20725696)(46.36619447,325.25225691)(46.34619467,325.31226336)
\curveto(46.32619451,325.37225679)(46.30619453,325.44725672)(46.28619467,325.53726336)
\lineto(46.28619467,325.85226336)
\curveto(46.28619455,326.03225613)(46.29619454,326.17725599)(46.31619467,326.28726336)
\curveto(46.3361945,326.39725577)(46.42119441,326.47225569)(46.57119467,326.51226336)
\curveto(46.61119422,326.53225563)(46.65119418,326.53725563)(46.69119467,326.52726336)
\lineto(46.82619467,326.52726336)
\curveto(46.97619386,326.52725564)(47.11619372,326.53225563)(47.24619467,326.54226336)
\curveto(47.37619346,326.5622556)(47.46619337,326.62225554)(47.51619467,326.72226336)
\curveto(47.54619329,326.79225537)(47.56119327,326.87225529)(47.56119467,326.96226336)
\curveto(47.57119326,327.05225511)(47.57619326,327.14225502)(47.57619467,327.23226336)
\lineto(47.57619467,328.16226336)
\lineto(47.57619467,328.41726336)
\curveto(47.57619326,328.50725366)(47.58619325,328.58225358)(47.60619467,328.64226336)
\curveto(47.65619318,328.74225342)(47.7311931,328.80725336)(47.83119467,328.83726336)
\curveto(47.85119298,328.84725332)(47.87619296,328.84725332)(47.90619467,328.83726336)
\curveto(47.94619289,328.83725333)(47.97619286,328.84225332)(47.99619467,328.85226336)
}
}
{
\newrgbcolor{curcolor}{0 0 0}
\pscustom[linestyle=none,fillstyle=solid,fillcolor=curcolor]
{
\newpath
\moveto(54.31963217,329.39226336)
\curveto(54.38962922,329.31225285)(54.42462918,329.19225297)(54.42463217,329.03226336)
\lineto(54.42463217,328.56726336)
\lineto(54.42463217,328.16226336)
\curveto(54.42462918,328.02225414)(54.38962922,327.92725424)(54.31963217,327.87726336)
\curveto(54.25962935,327.82725434)(54.17962943,327.79725437)(54.07963217,327.78726336)
\curveto(53.98962962,327.77725439)(53.88962972,327.77225439)(53.77963217,327.77226336)
\lineto(52.93963217,327.77226336)
\curveto(52.82963078,327.77225439)(52.72963088,327.77725439)(52.63963217,327.78726336)
\curveto(52.55963105,327.79725437)(52.48963112,327.82725434)(52.42963217,327.87726336)
\curveto(52.38963122,327.90725426)(52.35963125,327.9622542)(52.33963217,328.04226336)
\curveto(52.32963128,328.13225403)(52.31963129,328.22725394)(52.30963217,328.32726336)
\lineto(52.30963217,328.65726336)
\curveto(52.31963129,328.7672534)(52.32463128,328.8622533)(52.32463217,328.94226336)
\lineto(52.32463217,329.15226336)
\curveto(52.33463127,329.22225294)(52.35463125,329.28225288)(52.38463217,329.33226336)
\curveto(52.4046312,329.37225279)(52.42963118,329.40225276)(52.45963217,329.42226336)
\lineto(52.57963217,329.48226336)
\curveto(52.59963101,329.48225268)(52.62463098,329.48225268)(52.65463217,329.48226336)
\curveto(52.68463092,329.49225267)(52.7096309,329.49725267)(52.72963217,329.49726336)
\lineto(53.82463217,329.49726336)
\curveto(53.92462968,329.49725267)(54.01962959,329.49225267)(54.10963217,329.48226336)
\curveto(54.19962941,329.47225269)(54.26962934,329.44225272)(54.31963217,329.39226336)
\moveto(54.42463217,319.62726336)
\curveto(54.42462918,319.42726274)(54.41962919,319.25726291)(54.40963217,319.11726336)
\curveto(54.39962921,318.97726319)(54.3096293,318.88226328)(54.13963217,318.83226336)
\curveto(54.07962953,318.81226335)(54.01462959,318.80226336)(53.94463217,318.80226336)
\curveto(53.87462973,318.81226335)(53.79962981,318.81726335)(53.71963217,318.81726336)
\lineto(52.87963217,318.81726336)
\curveto(52.78963082,318.81726335)(52.69963091,318.82226334)(52.60963217,318.83226336)
\curveto(52.52963108,318.84226332)(52.46963114,318.87226329)(52.42963217,318.92226336)
\curveto(52.36963124,318.99226317)(52.33463127,319.07726309)(52.32463217,319.17726336)
\lineto(52.32463217,319.52226336)
\lineto(52.32463217,325.85226336)
\lineto(52.32463217,326.15226336)
\curveto(52.32463128,326.25225591)(52.34463126,326.33225583)(52.38463217,326.39226336)
\curveto(52.44463116,326.4622557)(52.52963108,326.50725566)(52.63963217,326.52726336)
\curveto(52.65963095,326.53725563)(52.68463092,326.53725563)(52.71463217,326.52726336)
\curveto(52.75463085,326.52725564)(52.78463082,326.53225563)(52.80463217,326.54226336)
\lineto(53.55463217,326.54226336)
\lineto(53.74963217,326.54226336)
\curveto(53.82962978,326.55225561)(53.89462971,326.55225561)(53.94463217,326.54226336)
\lineto(54.06463217,326.54226336)
\curveto(54.12462948,326.52225564)(54.17962943,326.50725566)(54.22963217,326.49726336)
\curveto(54.27962933,326.48725568)(54.31962929,326.45725571)(54.34963217,326.40726336)
\curveto(54.38962922,326.35725581)(54.4096292,326.28725588)(54.40963217,326.19726336)
\curveto(54.41962919,326.10725606)(54.42462918,326.01225615)(54.42463217,325.91226336)
\lineto(54.42463217,319.62726336)
}
}
{
\newrgbcolor{curcolor}{0 0 0}
\pscustom[linestyle=none,fillstyle=solid,fillcolor=curcolor]
{
\newpath
\moveto(63.97681967,322.76226336)
\curveto(63.98681099,322.70225946)(63.99181098,322.61225955)(63.99181967,322.49226336)
\curveto(63.99181098,322.37225979)(63.98181099,322.28725988)(63.96181967,322.23726336)
\lineto(63.96181967,322.04226336)
\curveto(63.93181104,321.93226023)(63.91181106,321.82726034)(63.90181967,321.72726336)
\curveto(63.90181107,321.62726054)(63.88681109,321.52726064)(63.85681967,321.42726336)
\curveto(63.83681114,321.33726083)(63.81681116,321.24226092)(63.79681967,321.14226336)
\curveto(63.7768112,321.05226111)(63.74681123,320.9622612)(63.70681967,320.87226336)
\curveto(63.63681134,320.70226146)(63.56681141,320.54226162)(63.49681967,320.39226336)
\curveto(63.42681155,320.25226191)(63.34681163,320.11226205)(63.25681967,319.97226336)
\curveto(63.19681178,319.88226228)(63.13181184,319.79726237)(63.06181967,319.71726336)
\curveto(63.00181197,319.64726252)(62.93181204,319.57226259)(62.85181967,319.49226336)
\lineto(62.74681967,319.38726336)
\curveto(62.69681228,319.33726283)(62.64181233,319.29226287)(62.58181967,319.25226336)
\lineto(62.43181967,319.13226336)
\curveto(62.35181262,319.07226309)(62.26181271,319.01726315)(62.16181967,318.96726336)
\curveto(62.0718129,318.92726324)(61.976813,318.88226328)(61.87681967,318.83226336)
\curveto(61.7768132,318.78226338)(61.6718133,318.74726342)(61.56181967,318.72726336)
\curveto(61.46181351,318.70726346)(61.35681362,318.68726348)(61.24681967,318.66726336)
\curveto(61.18681379,318.64726352)(61.12181385,318.63726353)(61.05181967,318.63726336)
\curveto(60.99181398,318.63726353)(60.92681405,318.62726354)(60.85681967,318.60726336)
\lineto(60.72181967,318.60726336)
\curveto(60.64181433,318.58726358)(60.56681441,318.58726358)(60.49681967,318.60726336)
\lineto(60.34681967,318.60726336)
\curveto(60.28681469,318.62726354)(60.22181475,318.63726353)(60.15181967,318.63726336)
\curveto(60.09181488,318.62726354)(60.03181494,318.63226353)(59.97181967,318.65226336)
\curveto(59.81181516,318.70226346)(59.65681532,318.74726342)(59.50681967,318.78726336)
\curveto(59.36681561,318.82726334)(59.23681574,318.88726328)(59.11681967,318.96726336)
\curveto(59.04681593,319.00726316)(58.98181599,319.04726312)(58.92181967,319.08726336)
\curveto(58.86181611,319.13726303)(58.79681618,319.18726298)(58.72681967,319.23726336)
\lineto(58.54681967,319.37226336)
\curveto(58.46681651,319.43226273)(58.39681658,319.43726273)(58.33681967,319.38726336)
\curveto(58.28681669,319.35726281)(58.26181671,319.31726285)(58.26181967,319.26726336)
\curveto(58.26181671,319.22726294)(58.25181672,319.17726299)(58.23181967,319.11726336)
\curveto(58.21181676,319.01726315)(58.20181677,318.90226326)(58.20181967,318.77226336)
\curveto(58.21181676,318.64226352)(58.21681676,318.52226364)(58.21681967,318.41226336)
\lineto(58.21681967,316.88226336)
\curveto(58.21681676,316.75226541)(58.21181676,316.62726554)(58.20181967,316.50726336)
\curveto(58.20181677,316.37726579)(58.1768168,316.27226589)(58.12681967,316.19226336)
\curveto(58.09681688,316.15226601)(58.04181693,316.12226604)(57.96181967,316.10226336)
\curveto(57.88181709,316.08226608)(57.79181718,316.07226609)(57.69181967,316.07226336)
\curveto(57.59181738,316.0622661)(57.49181748,316.0622661)(57.39181967,316.07226336)
\lineto(57.13681967,316.07226336)
\lineto(56.73181967,316.07226336)
\lineto(56.62681967,316.07226336)
\curveto(56.58681839,316.07226609)(56.55181842,316.07726609)(56.52181967,316.08726336)
\lineto(56.40181967,316.08726336)
\curveto(56.23181874,316.13726603)(56.14181883,316.23726593)(56.13181967,316.38726336)
\curveto(56.12181885,316.52726564)(56.11681886,316.69726547)(56.11681967,316.89726336)
\lineto(56.11681967,325.70226336)
\curveto(56.11681886,325.81225635)(56.11181886,325.92725624)(56.10181967,326.04726336)
\curveto(56.10181887,326.17725599)(56.12681885,326.27725589)(56.17681967,326.34726336)
\curveto(56.21681876,326.41725575)(56.2718187,326.4622557)(56.34181967,326.48226336)
\curveto(56.39181858,326.50225566)(56.45181852,326.51225565)(56.52181967,326.51226336)
\lineto(56.74681967,326.51226336)
\lineto(57.46681967,326.51226336)
\lineto(57.75181967,326.51226336)
\curveto(57.84181713,326.51225565)(57.91681706,326.48725568)(57.97681967,326.43726336)
\curveto(58.04681693,326.38725578)(58.08181689,326.32225584)(58.08181967,326.24226336)
\curveto(58.09181688,326.17225599)(58.11681686,326.09725607)(58.15681967,326.01726336)
\curveto(58.16681681,325.98725618)(58.1768168,325.9622562)(58.18681967,325.94226336)
\curveto(58.20681677,325.93225623)(58.22681675,325.91725625)(58.24681967,325.89726336)
\curveto(58.35681662,325.88725628)(58.44681653,325.91725625)(58.51681967,325.98726336)
\curveto(58.58681639,326.05725611)(58.65681632,326.11725605)(58.72681967,326.16726336)
\curveto(58.85681612,326.25725591)(58.99181598,326.33725583)(59.13181967,326.40726336)
\curveto(59.2718157,326.48725568)(59.42681555,326.55225561)(59.59681967,326.60226336)
\curveto(59.6768153,326.63225553)(59.76181521,326.65225551)(59.85181967,326.66226336)
\curveto(59.95181502,326.67225549)(60.04681493,326.68725548)(60.13681967,326.70726336)
\curveto(60.1768148,326.71725545)(60.21681476,326.71725545)(60.25681967,326.70726336)
\curveto(60.30681467,326.69725547)(60.34681463,326.70225546)(60.37681967,326.72226336)
\curveto(60.94681403,326.74225542)(61.42681355,326.6622555)(61.81681967,326.48226336)
\curveto(62.21681276,326.31225585)(62.55681242,326.08725608)(62.83681967,325.80726336)
\curveto(62.88681209,325.75725641)(62.93181204,325.70725646)(62.97181967,325.65726336)
\curveto(63.01181196,325.61725655)(63.05181192,325.57225659)(63.09181967,325.52226336)
\curveto(63.16181181,325.43225673)(63.22181175,325.34225682)(63.27181967,325.25226336)
\curveto(63.33181164,325.162257)(63.38681159,325.07225709)(63.43681967,324.98226336)
\curveto(63.45681152,324.9622572)(63.46681151,324.93725723)(63.46681967,324.90726336)
\curveto(63.4768115,324.87725729)(63.49181148,324.84225732)(63.51181967,324.80226336)
\curveto(63.5718114,324.70225746)(63.62681135,324.58225758)(63.67681967,324.44226336)
\curveto(63.69681128,324.38225778)(63.71681126,324.31725785)(63.73681967,324.24726336)
\curveto(63.75681122,324.18725798)(63.7768112,324.12225804)(63.79681967,324.05226336)
\curveto(63.83681114,323.93225823)(63.86181111,323.80725836)(63.87181967,323.67726336)
\curveto(63.89181108,323.54725862)(63.91681106,323.41225875)(63.94681967,323.27226336)
\lineto(63.94681967,323.10726336)
\lineto(63.97681967,322.92726336)
\lineto(63.97681967,322.76226336)
\moveto(61.86181967,322.41726336)
\curveto(61.8718131,322.4672597)(61.8768131,322.53225963)(61.87681967,322.61226336)
\curveto(61.8768131,322.70225946)(61.8718131,322.77225939)(61.86181967,322.82226336)
\lineto(61.86181967,322.95726336)
\curveto(61.84181313,323.01725915)(61.83181314,323.08225908)(61.83181967,323.15226336)
\curveto(61.83181314,323.22225894)(61.82181315,323.29225887)(61.80181967,323.36226336)
\curveto(61.78181319,323.4622587)(61.76181321,323.55725861)(61.74181967,323.64726336)
\curveto(61.72181325,323.74725842)(61.69181328,323.83725833)(61.65181967,323.91726336)
\curveto(61.53181344,324.23725793)(61.3768136,324.49225767)(61.18681967,324.68226336)
\curveto(60.99681398,324.87225729)(60.72681425,325.01225715)(60.37681967,325.10226336)
\curveto(60.29681468,325.12225704)(60.20681477,325.13225703)(60.10681967,325.13226336)
\lineto(59.83681967,325.13226336)
\curveto(59.79681518,325.12225704)(59.76181521,325.11725705)(59.73181967,325.11726336)
\curveto(59.70181527,325.11725705)(59.66681531,325.11225705)(59.62681967,325.10226336)
\lineto(59.41681967,325.04226336)
\curveto(59.35681562,325.03225713)(59.29681568,325.01225715)(59.23681967,324.98226336)
\curveto(58.976816,324.87225729)(58.7718162,324.70225746)(58.62181967,324.47226336)
\curveto(58.48181649,324.24225792)(58.36681661,323.98725818)(58.27681967,323.70726336)
\curveto(58.25681672,323.62725854)(58.24181673,323.54225862)(58.23181967,323.45226336)
\curveto(58.22181675,323.37225879)(58.20681677,323.29225887)(58.18681967,323.21226336)
\curveto(58.1768168,323.17225899)(58.1718168,323.10725906)(58.17181967,323.01726336)
\curveto(58.15181682,322.97725919)(58.14681683,322.92725924)(58.15681967,322.86726336)
\curveto(58.16681681,322.81725935)(58.16681681,322.7672594)(58.15681967,322.71726336)
\curveto(58.13681684,322.65725951)(58.13681684,322.60225956)(58.15681967,322.55226336)
\lineto(58.15681967,322.37226336)
\lineto(58.15681967,322.23726336)
\curveto(58.15681682,322.19725997)(58.16681681,322.15726001)(58.18681967,322.11726336)
\curveto(58.18681679,322.04726012)(58.19181678,321.99226017)(58.20181967,321.95226336)
\lineto(58.23181967,321.77226336)
\curveto(58.24181673,321.71226045)(58.25681672,321.65226051)(58.27681967,321.59226336)
\curveto(58.36681661,321.30226086)(58.4718165,321.0622611)(58.59181967,320.87226336)
\curveto(58.72181625,320.69226147)(58.90181607,320.53226163)(59.13181967,320.39226336)
\curveto(59.2718157,320.31226185)(59.43681554,320.24726192)(59.62681967,320.19726336)
\curveto(59.66681531,320.18726198)(59.70181527,320.18226198)(59.73181967,320.18226336)
\curveto(59.76181521,320.19226197)(59.79681518,320.19226197)(59.83681967,320.18226336)
\curveto(59.8768151,320.17226199)(59.93681504,320.162262)(60.01681967,320.15226336)
\curveto(60.09681488,320.15226201)(60.16181481,320.15726201)(60.21181967,320.16726336)
\curveto(60.29181468,320.18726198)(60.3718146,320.20226196)(60.45181967,320.21226336)
\curveto(60.54181443,320.23226193)(60.62681435,320.25726191)(60.70681967,320.28726336)
\curveto(60.94681403,320.38726178)(61.14181383,320.52726164)(61.29181967,320.70726336)
\curveto(61.44181353,320.88726128)(61.56681341,321.09726107)(61.66681967,321.33726336)
\curveto(61.71681326,321.45726071)(61.75181322,321.58226058)(61.77181967,321.71226336)
\curveto(61.79181318,321.84226032)(61.81681316,321.97726019)(61.84681967,322.11726336)
\lineto(61.84681967,322.26726336)
\curveto(61.85681312,322.31725985)(61.86181311,322.3672598)(61.86181967,322.41726336)
}
}
{
\newrgbcolor{curcolor}{0 0 0}
\pscustom[linestyle=none,fillstyle=solid,fillcolor=curcolor]
{
\newpath
\moveto(73.02674154,322.98726336)
\curveto(73.04673297,322.92725924)(73.05673296,322.84225932)(73.05674154,322.73226336)
\curveto(73.05673296,322.62225954)(73.04673297,322.53725963)(73.02674154,322.47726336)
\lineto(73.02674154,322.32726336)
\curveto(73.00673301,322.24725992)(72.99673302,322.16726)(72.99674154,322.08726336)
\curveto(73.00673301,322.00726016)(73.00173302,321.92726024)(72.98174154,321.84726336)
\curveto(72.96173306,321.77726039)(72.94673307,321.71226045)(72.93674154,321.65226336)
\curveto(72.92673309,321.59226057)(72.9167331,321.52726064)(72.90674154,321.45726336)
\curveto(72.86673315,321.34726082)(72.83173319,321.23226093)(72.80174154,321.11226336)
\curveto(72.77173325,321.00226116)(72.73173329,320.89726127)(72.68174154,320.79726336)
\curveto(72.47173355,320.31726185)(72.19673382,319.92726224)(71.85674154,319.62726336)
\curveto(71.5167345,319.32726284)(71.10673491,319.07726309)(70.62674154,318.87726336)
\curveto(70.50673551,318.82726334)(70.38173564,318.79226337)(70.25174154,318.77226336)
\curveto(70.13173589,318.74226342)(70.00673601,318.71226345)(69.87674154,318.68226336)
\curveto(69.82673619,318.6622635)(69.77173625,318.65226351)(69.71174154,318.65226336)
\curveto(69.65173637,318.65226351)(69.59673642,318.64726352)(69.54674154,318.63726336)
\lineto(69.44174154,318.63726336)
\curveto(69.41173661,318.62726354)(69.38173664,318.62226354)(69.35174154,318.62226336)
\curveto(69.30173672,318.61226355)(69.2217368,318.60726356)(69.11174154,318.60726336)
\curveto(69.00173702,318.59726357)(68.9167371,318.60226356)(68.85674154,318.62226336)
\lineto(68.70674154,318.62226336)
\curveto(68.65673736,318.63226353)(68.60173742,318.63726353)(68.54174154,318.63726336)
\curveto(68.49173753,318.62726354)(68.44173758,318.63226353)(68.39174154,318.65226336)
\curveto(68.35173767,318.6622635)(68.31173771,318.6672635)(68.27174154,318.66726336)
\curveto(68.24173778,318.6672635)(68.20173782,318.67226349)(68.15174154,318.68226336)
\curveto(68.05173797,318.71226345)(67.95173807,318.73726343)(67.85174154,318.75726336)
\curveto(67.75173827,318.77726339)(67.65673836,318.80726336)(67.56674154,318.84726336)
\curveto(67.44673857,318.88726328)(67.33173869,318.92726324)(67.22174154,318.96726336)
\curveto(67.1217389,319.00726316)(67.016739,319.05726311)(66.90674154,319.11726336)
\curveto(66.55673946,319.32726284)(66.25673976,319.57226259)(66.00674154,319.85226336)
\curveto(65.75674026,320.13226203)(65.54674047,320.4672617)(65.37674154,320.85726336)
\curveto(65.32674069,320.94726122)(65.28674073,321.04226112)(65.25674154,321.14226336)
\curveto(65.23674078,321.24226092)(65.21174081,321.34726082)(65.18174154,321.45726336)
\curveto(65.16174086,321.50726066)(65.15174087,321.55226061)(65.15174154,321.59226336)
\curveto(65.15174087,321.63226053)(65.14174088,321.67726049)(65.12174154,321.72726336)
\curveto(65.10174092,321.80726036)(65.09174093,321.88726028)(65.09174154,321.96726336)
\curveto(65.09174093,322.05726011)(65.08174094,322.14226002)(65.06174154,322.22226336)
\curveto(65.05174097,322.27225989)(65.04674097,322.31725985)(65.04674154,322.35726336)
\lineto(65.04674154,322.49226336)
\curveto(65.02674099,322.55225961)(65.016741,322.63725953)(65.01674154,322.74726336)
\curveto(65.02674099,322.85725931)(65.04174098,322.94225922)(65.06174154,323.00226336)
\lineto(65.06174154,323.10726336)
\curveto(65.07174095,323.15725901)(65.07174095,323.20725896)(65.06174154,323.25726336)
\curveto(65.06174096,323.31725885)(65.07174095,323.37225879)(65.09174154,323.42226336)
\curveto(65.10174092,323.47225869)(65.10674091,323.51725865)(65.10674154,323.55726336)
\curveto(65.10674091,323.60725856)(65.1167409,323.65725851)(65.13674154,323.70726336)
\curveto(65.17674084,323.83725833)(65.21174081,323.9622582)(65.24174154,324.08226336)
\curveto(65.27174075,324.21225795)(65.31174071,324.33725783)(65.36174154,324.45726336)
\curveto(65.54174048,324.8672573)(65.75674026,325.20725696)(66.00674154,325.47726336)
\curveto(66.25673976,325.75725641)(66.56173946,326.01225615)(66.92174154,326.24226336)
\curveto(67.021739,326.29225587)(67.12673889,326.33725583)(67.23674154,326.37726336)
\curveto(67.34673867,326.41725575)(67.45673856,326.4622557)(67.56674154,326.51226336)
\curveto(67.69673832,326.5622556)(67.83173819,326.59725557)(67.97174154,326.61726336)
\curveto(68.11173791,326.63725553)(68.25673776,326.6672555)(68.40674154,326.70726336)
\curveto(68.48673753,326.71725545)(68.56173746,326.72225544)(68.63174154,326.72226336)
\curveto(68.70173732,326.72225544)(68.77173725,326.72725544)(68.84174154,326.73726336)
\curveto(69.4217366,326.74725542)(69.9217361,326.68725548)(70.34174154,326.55726336)
\curveto(70.77173525,326.42725574)(71.15173487,326.24725592)(71.48174154,326.01726336)
\curveto(71.59173443,325.93725623)(71.70173432,325.84725632)(71.81174154,325.74726336)
\curveto(71.93173409,325.65725651)(72.03173399,325.55725661)(72.11174154,325.44726336)
\curveto(72.19173383,325.34725682)(72.26173376,325.24725692)(72.32174154,325.14726336)
\curveto(72.39173363,325.04725712)(72.46173356,324.94225722)(72.53174154,324.83226336)
\curveto(72.60173342,324.72225744)(72.65673336,324.60225756)(72.69674154,324.47226336)
\curveto(72.73673328,324.35225781)(72.78173324,324.22225794)(72.83174154,324.08226336)
\curveto(72.86173316,324.00225816)(72.88673313,323.91725825)(72.90674154,323.82726336)
\lineto(72.96674154,323.55726336)
\curveto(72.97673304,323.51725865)(72.98173304,323.47725869)(72.98174154,323.43726336)
\curveto(72.98173304,323.39725877)(72.98673303,323.35725881)(72.99674154,323.31726336)
\curveto(73.016733,323.2672589)(73.021733,323.21225895)(73.01174154,323.15226336)
\curveto(73.00173302,323.09225907)(73.00673301,323.03725913)(73.02674154,322.98726336)
\moveto(70.92674154,322.44726336)
\curveto(70.93673508,322.49725967)(70.94173508,322.5672596)(70.94174154,322.65726336)
\curveto(70.94173508,322.75725941)(70.93673508,322.83225933)(70.92674154,322.88226336)
\lineto(70.92674154,323.00226336)
\curveto(70.90673511,323.05225911)(70.89673512,323.10725906)(70.89674154,323.16726336)
\curveto(70.89673512,323.22725894)(70.89173513,323.28225888)(70.88174154,323.33226336)
\curveto(70.88173514,323.37225879)(70.87673514,323.40225876)(70.86674154,323.42226336)
\lineto(70.80674154,323.66226336)
\curveto(70.79673522,323.75225841)(70.77673524,323.83725833)(70.74674154,323.91726336)
\curveto(70.63673538,324.17725799)(70.50673551,324.39725777)(70.35674154,324.57726336)
\curveto(70.20673581,324.7672574)(70.00673601,324.91725725)(69.75674154,325.02726336)
\curveto(69.69673632,325.04725712)(69.63673638,325.0622571)(69.57674154,325.07226336)
\curveto(69.5167365,325.09225707)(69.45173657,325.11225705)(69.38174154,325.13226336)
\curveto(69.30173672,325.15225701)(69.2167368,325.15725701)(69.12674154,325.14726336)
\lineto(68.85674154,325.14726336)
\curveto(68.82673719,325.12725704)(68.79173723,325.11725705)(68.75174154,325.11726336)
\curveto(68.71173731,325.12725704)(68.67673734,325.12725704)(68.64674154,325.11726336)
\lineto(68.43674154,325.05726336)
\curveto(68.37673764,325.04725712)(68.3217377,325.02725714)(68.27174154,324.99726336)
\curveto(68.021738,324.88725728)(67.8167382,324.72725744)(67.65674154,324.51726336)
\curveto(67.50673851,324.31725785)(67.38673863,324.08225808)(67.29674154,323.81226336)
\curveto(67.26673875,323.71225845)(67.24173878,323.60725856)(67.22174154,323.49726336)
\curveto(67.21173881,323.38725878)(67.19673882,323.27725889)(67.17674154,323.16726336)
\curveto(67.16673885,323.11725905)(67.16173886,323.0672591)(67.16174154,323.01726336)
\lineto(67.16174154,322.86726336)
\curveto(67.14173888,322.79725937)(67.13173889,322.69225947)(67.13174154,322.55226336)
\curveto(67.14173888,322.41225975)(67.15673886,322.30725986)(67.17674154,322.23726336)
\lineto(67.17674154,322.10226336)
\curveto(67.19673882,322.02226014)(67.21173881,321.94226022)(67.22174154,321.86226336)
\curveto(67.23173879,321.79226037)(67.24673877,321.71726045)(67.26674154,321.63726336)
\curveto(67.36673865,321.33726083)(67.47173855,321.09226107)(67.58174154,320.90226336)
\curveto(67.70173832,320.72226144)(67.88673813,320.55726161)(68.13674154,320.40726336)
\curveto(68.20673781,320.35726181)(68.28173774,320.31726185)(68.36174154,320.28726336)
\curveto(68.45173757,320.25726191)(68.54173748,320.23226193)(68.63174154,320.21226336)
\curveto(68.67173735,320.20226196)(68.70673731,320.19726197)(68.73674154,320.19726336)
\curveto(68.76673725,320.20726196)(68.80173722,320.20726196)(68.84174154,320.19726336)
\lineto(68.96174154,320.16726336)
\curveto(69.01173701,320.167262)(69.05673696,320.17226199)(69.09674154,320.18226336)
\lineto(69.21674154,320.18226336)
\curveto(69.29673672,320.20226196)(69.37673664,320.21726195)(69.45674154,320.22726336)
\curveto(69.53673648,320.23726193)(69.61173641,320.25726191)(69.68174154,320.28726336)
\curveto(69.94173608,320.38726178)(70.15173587,320.52226164)(70.31174154,320.69226336)
\curveto(70.47173555,320.8622613)(70.60673541,321.07226109)(70.71674154,321.32226336)
\curveto(70.75673526,321.42226074)(70.78673523,321.52226064)(70.80674154,321.62226336)
\curveto(70.82673519,321.72226044)(70.85173517,321.82726034)(70.88174154,321.93726336)
\curveto(70.89173513,321.97726019)(70.89673512,322.01226015)(70.89674154,322.04226336)
\curveto(70.89673512,322.08226008)(70.90173512,322.12226004)(70.91174154,322.16226336)
\lineto(70.91174154,322.29726336)
\curveto(70.91173511,322.34725982)(70.9167351,322.39725977)(70.92674154,322.44726336)
}
}
{
\newrgbcolor{curcolor}{0 0 0}
\pscustom[linestyle=none,fillstyle=solid,fillcolor=curcolor]
{
}
}
{
\newrgbcolor{curcolor}{0 0 0}
\pscustom[linestyle=none,fillstyle=solid,fillcolor=curcolor]
{
\newpath
\moveto(86.17681967,319.65726336)
\lineto(86.17681967,319.23726336)
\curveto(86.1768113,319.10726306)(86.14681133,319.00226316)(86.08681967,318.92226336)
\curveto(86.03681144,318.87226329)(85.9718115,318.83726333)(85.89181967,318.81726336)
\curveto(85.81181166,318.80726336)(85.72181175,318.80226336)(85.62181967,318.80226336)
\lineto(84.79681967,318.80226336)
\lineto(84.51181967,318.80226336)
\curveto(84.43181304,318.81226335)(84.36681311,318.83726333)(84.31681967,318.87726336)
\curveto(84.24681323,318.92726324)(84.20681327,318.99226317)(84.19681967,319.07226336)
\curveto(84.18681329,319.15226301)(84.16681331,319.23226293)(84.13681967,319.31226336)
\curveto(84.11681336,319.33226283)(84.09681338,319.34726282)(84.07681967,319.35726336)
\curveto(84.06681341,319.37726279)(84.05181342,319.39726277)(84.03181967,319.41726336)
\curveto(83.92181355,319.41726275)(83.84181363,319.39226277)(83.79181967,319.34226336)
\lineto(83.64181967,319.19226336)
\curveto(83.5718139,319.14226302)(83.50681397,319.09726307)(83.44681967,319.05726336)
\curveto(83.38681409,319.02726314)(83.32181415,318.98726318)(83.25181967,318.93726336)
\curveto(83.21181426,318.91726325)(83.16681431,318.89726327)(83.11681967,318.87726336)
\curveto(83.0768144,318.85726331)(83.03181444,318.83726333)(82.98181967,318.81726336)
\curveto(82.84181463,318.7672634)(82.69181478,318.72226344)(82.53181967,318.68226336)
\curveto(82.48181499,318.6622635)(82.43681504,318.65226351)(82.39681967,318.65226336)
\curveto(82.35681512,318.65226351)(82.31681516,318.64726352)(82.27681967,318.63726336)
\lineto(82.14181967,318.63726336)
\curveto(82.11181536,318.62726354)(82.0718154,318.62226354)(82.02181967,318.62226336)
\lineto(81.88681967,318.62226336)
\curveto(81.82681565,318.60226356)(81.73681574,318.59726357)(81.61681967,318.60726336)
\curveto(81.49681598,318.60726356)(81.41181606,318.61726355)(81.36181967,318.63726336)
\curveto(81.29181618,318.65726351)(81.22681625,318.6672635)(81.16681967,318.66726336)
\curveto(81.11681636,318.65726351)(81.06181641,318.6622635)(81.00181967,318.68226336)
\lineto(80.64181967,318.80226336)
\curveto(80.53181694,318.83226333)(80.42181705,318.87226329)(80.31181967,318.92226336)
\curveto(79.96181751,319.07226309)(79.64681783,319.30226286)(79.36681967,319.61226336)
\curveto(79.09681838,319.93226223)(78.88181859,320.2672619)(78.72181967,320.61726336)
\curveto(78.6718188,320.72726144)(78.63181884,320.83226133)(78.60181967,320.93226336)
\curveto(78.5718189,321.04226112)(78.53681894,321.15226101)(78.49681967,321.26226336)
\curveto(78.48681899,321.30226086)(78.48181899,321.33726083)(78.48181967,321.36726336)
\curveto(78.48181899,321.40726076)(78.471819,321.45226071)(78.45181967,321.50226336)
\curveto(78.43181904,321.58226058)(78.41181906,321.6672605)(78.39181967,321.75726336)
\curveto(78.38181909,321.85726031)(78.36681911,321.95726021)(78.34681967,322.05726336)
\curveto(78.33681914,322.08726008)(78.33181914,322.12226004)(78.33181967,322.16226336)
\curveto(78.34181913,322.20225996)(78.34181913,322.23725993)(78.33181967,322.26726336)
\lineto(78.33181967,322.40226336)
\curveto(78.33181914,322.45225971)(78.32681915,322.50225966)(78.31681967,322.55226336)
\curveto(78.30681917,322.60225956)(78.30181917,322.65725951)(78.30181967,322.71726336)
\curveto(78.30181917,322.78725938)(78.30681917,322.84225932)(78.31681967,322.88226336)
\curveto(78.32681915,322.93225923)(78.33181914,322.97725919)(78.33181967,323.01726336)
\lineto(78.33181967,323.16726336)
\curveto(78.34181913,323.21725895)(78.34181913,323.2622589)(78.33181967,323.30226336)
\curveto(78.33181914,323.35225881)(78.34181913,323.40225876)(78.36181967,323.45226336)
\curveto(78.38181909,323.5622586)(78.39681908,323.6672585)(78.40681967,323.76726336)
\curveto(78.42681905,323.8672583)(78.45181902,323.9672582)(78.48181967,324.06726336)
\curveto(78.52181895,324.18725798)(78.55681892,324.30225786)(78.58681967,324.41226336)
\curveto(78.61681886,324.52225764)(78.65681882,324.63225753)(78.70681967,324.74226336)
\curveto(78.84681863,325.04225712)(79.02181845,325.32725684)(79.23181967,325.59726336)
\curveto(79.25181822,325.62725654)(79.2768182,325.65225651)(79.30681967,325.67226336)
\curveto(79.34681813,325.70225646)(79.3768181,325.73225643)(79.39681967,325.76226336)
\curveto(79.43681804,325.81225635)(79.476818,325.85725631)(79.51681967,325.89726336)
\curveto(79.55681792,325.93725623)(79.60181787,325.97725619)(79.65181967,326.01726336)
\curveto(79.69181778,326.03725613)(79.72681775,326.0622561)(79.75681967,326.09226336)
\curveto(79.78681769,326.13225603)(79.82181765,326.162256)(79.86181967,326.18226336)
\curveto(80.11181736,326.35225581)(80.40181707,326.49225567)(80.73181967,326.60226336)
\curveto(80.80181667,326.62225554)(80.8718166,326.63725553)(80.94181967,326.64726336)
\curveto(81.02181645,326.65725551)(81.10181637,326.67225549)(81.18181967,326.69226336)
\curveto(81.25181622,326.71225545)(81.34181613,326.72225544)(81.45181967,326.72226336)
\curveto(81.56181591,326.73225543)(81.6718158,326.73725543)(81.78181967,326.73726336)
\curveto(81.89181558,326.73725543)(81.99681548,326.73225543)(82.09681967,326.72226336)
\curveto(82.20681527,326.71225545)(82.29681518,326.69725547)(82.36681967,326.67726336)
\curveto(82.51681496,326.62725554)(82.66181481,326.58225558)(82.80181967,326.54226336)
\curveto(82.94181453,326.50225566)(83.0718144,326.44725572)(83.19181967,326.37726336)
\curveto(83.26181421,326.32725584)(83.32681415,326.27725589)(83.38681967,326.22726336)
\curveto(83.44681403,326.18725598)(83.51181396,326.14225602)(83.58181967,326.09226336)
\curveto(83.62181385,326.0622561)(83.6768138,326.02225614)(83.74681967,325.97226336)
\curveto(83.82681365,325.92225624)(83.90181357,325.92225624)(83.97181967,325.97226336)
\curveto(84.01181346,325.99225617)(84.03181344,326.02725614)(84.03181967,326.07726336)
\curveto(84.03181344,326.12725604)(84.04181343,326.17725599)(84.06181967,326.22726336)
\lineto(84.06181967,326.37726336)
\curveto(84.0718134,326.40725576)(84.0768134,326.44225572)(84.07681967,326.48226336)
\lineto(84.07681967,326.60226336)
\lineto(84.07681967,328.64226336)
\curveto(84.0768134,328.75225341)(84.0718134,328.87225329)(84.06181967,329.00226336)
\curveto(84.06181341,329.14225302)(84.08681339,329.24725292)(84.13681967,329.31726336)
\curveto(84.1768133,329.39725277)(84.25181322,329.44725272)(84.36181967,329.46726336)
\curveto(84.38181309,329.47725269)(84.40181307,329.47725269)(84.42181967,329.46726336)
\curveto(84.44181303,329.4672527)(84.46181301,329.47225269)(84.48181967,329.48226336)
\lineto(85.54681967,329.48226336)
\curveto(85.66681181,329.48225268)(85.7768117,329.47725269)(85.87681967,329.46726336)
\curveto(85.9768115,329.45725271)(86.05181142,329.41725275)(86.10181967,329.34726336)
\curveto(86.15181132,329.2672529)(86.1768113,329.162253)(86.17681967,329.03226336)
\lineto(86.17681967,328.67226336)
\lineto(86.17681967,319.65726336)
\moveto(84.13681967,322.59726336)
\curveto(84.14681333,322.63725953)(84.14681333,322.67725949)(84.13681967,322.71726336)
\lineto(84.13681967,322.85226336)
\curveto(84.13681334,322.95225921)(84.13181334,323.05225911)(84.12181967,323.15226336)
\curveto(84.11181336,323.25225891)(84.09681338,323.34225882)(84.07681967,323.42226336)
\curveto(84.05681342,323.53225863)(84.03681344,323.63225853)(84.01681967,323.72226336)
\curveto(84.00681347,323.81225835)(83.98181349,323.89725827)(83.94181967,323.97726336)
\curveto(83.80181367,324.33725783)(83.59681388,324.62225754)(83.32681967,324.83226336)
\curveto(83.06681441,325.04225712)(82.68681479,325.14725702)(82.18681967,325.14726336)
\curveto(82.12681535,325.14725702)(82.04681543,325.13725703)(81.94681967,325.11726336)
\curveto(81.86681561,325.09725707)(81.79181568,325.07725709)(81.72181967,325.05726336)
\curveto(81.66181581,325.04725712)(81.60181587,325.02725714)(81.54181967,324.99726336)
\curveto(81.2718162,324.88725728)(81.06181641,324.71725745)(80.91181967,324.48726336)
\curveto(80.76181671,324.25725791)(80.64181683,323.99725817)(80.55181967,323.70726336)
\curveto(80.52181695,323.60725856)(80.50181697,323.50725866)(80.49181967,323.40726336)
\curveto(80.48181699,323.30725886)(80.46181701,323.20225896)(80.43181967,323.09226336)
\lineto(80.43181967,322.88226336)
\curveto(80.41181706,322.79225937)(80.40681707,322.6672595)(80.41681967,322.50726336)
\curveto(80.42681705,322.35725981)(80.44181703,322.24725992)(80.46181967,322.17726336)
\lineto(80.46181967,322.08726336)
\curveto(80.471817,322.0672601)(80.476817,322.04726012)(80.47681967,322.02726336)
\curveto(80.49681698,321.94726022)(80.51181696,321.87226029)(80.52181967,321.80226336)
\curveto(80.54181693,321.73226043)(80.56181691,321.65726051)(80.58181967,321.57726336)
\curveto(80.75181672,321.05726111)(81.04181643,320.67226149)(81.45181967,320.42226336)
\curveto(81.58181589,320.33226183)(81.76181571,320.2622619)(81.99181967,320.21226336)
\curveto(82.03181544,320.20226196)(82.09181538,320.19726197)(82.17181967,320.19726336)
\curveto(82.20181527,320.18726198)(82.24681523,320.17726199)(82.30681967,320.16726336)
\curveto(82.3768151,320.167262)(82.43181504,320.17226199)(82.47181967,320.18226336)
\curveto(82.55181492,320.20226196)(82.63181484,320.21726195)(82.71181967,320.22726336)
\curveto(82.79181468,320.23726193)(82.8718146,320.25726191)(82.95181967,320.28726336)
\curveto(83.20181427,320.39726177)(83.40181407,320.53726163)(83.55181967,320.70726336)
\curveto(83.70181377,320.87726129)(83.83181364,321.09226107)(83.94181967,321.35226336)
\curveto(83.98181349,321.44226072)(84.01181346,321.53226063)(84.03181967,321.62226336)
\curveto(84.05181342,321.72226044)(84.0718134,321.82726034)(84.09181967,321.93726336)
\curveto(84.10181337,321.98726018)(84.10181337,322.03226013)(84.09181967,322.07226336)
\curveto(84.09181338,322.12226004)(84.10181337,322.17225999)(84.12181967,322.22226336)
\curveto(84.13181334,322.25225991)(84.13681334,322.28725988)(84.13681967,322.32726336)
\lineto(84.13681967,322.46226336)
\lineto(84.13681967,322.59726336)
}
}
{
\newrgbcolor{curcolor}{0 0 0}
\pscustom[linestyle=none,fillstyle=solid,fillcolor=curcolor]
{
\newpath
\moveto(95.12174154,322.74726336)
\curveto(95.14173338,322.6672595)(95.14173338,322.57725959)(95.12174154,322.47726336)
\curveto(95.10173342,322.37725979)(95.06673345,322.31225985)(95.01674154,322.28226336)
\curveto(94.96673355,322.24225992)(94.89173363,322.21225995)(94.79174154,322.19226336)
\curveto(94.70173382,322.18225998)(94.59673392,322.17225999)(94.47674154,322.16226336)
\lineto(94.13174154,322.16226336)
\curveto(94.0217345,322.17225999)(93.9217346,322.17725999)(93.83174154,322.17726336)
\lineto(90.17174154,322.17726336)
\lineto(89.96174154,322.17726336)
\curveto(89.90173862,322.17725999)(89.84673867,322.16726)(89.79674154,322.14726336)
\curveto(89.7167388,322.10726006)(89.66673885,322.0672601)(89.64674154,322.02726336)
\curveto(89.62673889,322.00726016)(89.60673891,321.9672602)(89.58674154,321.90726336)
\curveto(89.56673895,321.85726031)(89.56173896,321.80726036)(89.57174154,321.75726336)
\curveto(89.59173893,321.69726047)(89.60173892,321.63726053)(89.60174154,321.57726336)
\curveto(89.61173891,321.52726064)(89.62673889,321.47226069)(89.64674154,321.41226336)
\curveto(89.72673879,321.17226099)(89.8217387,320.97226119)(89.93174154,320.81226336)
\curveto(90.05173847,320.6622615)(90.21173831,320.52726164)(90.41174154,320.40726336)
\curveto(90.49173803,320.35726181)(90.57173795,320.32226184)(90.65174154,320.30226336)
\curveto(90.74173778,320.29226187)(90.83173769,320.27226189)(90.92174154,320.24226336)
\curveto(91.00173752,320.22226194)(91.11173741,320.20726196)(91.25174154,320.19726336)
\curveto(91.39173713,320.18726198)(91.51173701,320.19226197)(91.61174154,320.21226336)
\lineto(91.74674154,320.21226336)
\curveto(91.84673667,320.23226193)(91.93673658,320.25226191)(92.01674154,320.27226336)
\curveto(92.10673641,320.30226186)(92.19173633,320.33226183)(92.27174154,320.36226336)
\curveto(92.37173615,320.41226175)(92.48173604,320.47726169)(92.60174154,320.55726336)
\curveto(92.73173579,320.63726153)(92.82673569,320.71726145)(92.88674154,320.79726336)
\curveto(92.93673558,320.8672613)(92.98673553,320.93226123)(93.03674154,320.99226336)
\curveto(93.09673542,321.0622611)(93.16673535,321.11226105)(93.24674154,321.14226336)
\curveto(93.34673517,321.19226097)(93.47173505,321.21226095)(93.62174154,321.20226336)
\lineto(94.05674154,321.20226336)
\lineto(94.23674154,321.20226336)
\curveto(94.30673421,321.21226095)(94.36673415,321.20726096)(94.41674154,321.18726336)
\lineto(94.56674154,321.18726336)
\curveto(94.66673385,321.167261)(94.73673378,321.14226102)(94.77674154,321.11226336)
\curveto(94.8167337,321.09226107)(94.83673368,321.04726112)(94.83674154,320.97726336)
\curveto(94.84673367,320.90726126)(94.84173368,320.84726132)(94.82174154,320.79726336)
\curveto(94.77173375,320.65726151)(94.7167338,320.53226163)(94.65674154,320.42226336)
\curveto(94.59673392,320.31226185)(94.52673399,320.20226196)(94.44674154,320.09226336)
\curveto(94.22673429,319.7622624)(93.97673454,319.49726267)(93.69674154,319.29726336)
\curveto(93.4167351,319.09726307)(93.06673545,318.92726324)(92.64674154,318.78726336)
\curveto(92.53673598,318.74726342)(92.42673609,318.72226344)(92.31674154,318.71226336)
\curveto(92.20673631,318.70226346)(92.09173643,318.68226348)(91.97174154,318.65226336)
\curveto(91.93173659,318.64226352)(91.88673663,318.64226352)(91.83674154,318.65226336)
\curveto(91.79673672,318.65226351)(91.75673676,318.64726352)(91.71674154,318.63726336)
\lineto(91.55174154,318.63726336)
\curveto(91.50173702,318.61726355)(91.44173708,318.61226355)(91.37174154,318.62226336)
\curveto(91.31173721,318.62226354)(91.25673726,318.62726354)(91.20674154,318.63726336)
\curveto(91.12673739,318.64726352)(91.05673746,318.64726352)(90.99674154,318.63726336)
\curveto(90.93673758,318.62726354)(90.87173765,318.63226353)(90.80174154,318.65226336)
\curveto(90.75173777,318.67226349)(90.69673782,318.68226348)(90.63674154,318.68226336)
\curveto(90.57673794,318.68226348)(90.521738,318.69226347)(90.47174154,318.71226336)
\curveto(90.36173816,318.73226343)(90.25173827,318.75726341)(90.14174154,318.78726336)
\curveto(90.03173849,318.80726336)(89.93173859,318.84226332)(89.84174154,318.89226336)
\curveto(89.73173879,318.93226323)(89.62673889,318.9672632)(89.52674154,318.99726336)
\curveto(89.43673908,319.03726313)(89.35173917,319.08226308)(89.27174154,319.13226336)
\curveto(88.95173957,319.33226283)(88.66673985,319.5622626)(88.41674154,319.82226336)
\curveto(88.16674035,320.09226207)(87.96174056,320.40226176)(87.80174154,320.75226336)
\curveto(87.75174077,320.8622613)(87.71174081,320.97226119)(87.68174154,321.08226336)
\curveto(87.65174087,321.20226096)(87.61174091,321.32226084)(87.56174154,321.44226336)
\curveto(87.55174097,321.48226068)(87.54674097,321.51726065)(87.54674154,321.54726336)
\curveto(87.54674097,321.58726058)(87.54174098,321.62726054)(87.53174154,321.66726336)
\curveto(87.49174103,321.78726038)(87.46674105,321.91726025)(87.45674154,322.05726336)
\lineto(87.42674154,322.47726336)
\curveto(87.42674109,322.52725964)(87.4217411,322.58225958)(87.41174154,322.64226336)
\curveto(87.41174111,322.70225946)(87.4167411,322.75725941)(87.42674154,322.80726336)
\lineto(87.42674154,322.98726336)
\lineto(87.47174154,323.34726336)
\curveto(87.51174101,323.51725865)(87.54674097,323.68225848)(87.57674154,323.84226336)
\curveto(87.60674091,324.00225816)(87.65174087,324.15225801)(87.71174154,324.29226336)
\curveto(88.14174038,325.33225683)(88.87173965,326.0672561)(89.90174154,326.49726336)
\curveto(90.04173848,326.55725561)(90.18173834,326.59725557)(90.32174154,326.61726336)
\curveto(90.47173805,326.64725552)(90.62673789,326.68225548)(90.78674154,326.72226336)
\curveto(90.86673765,326.73225543)(90.94173758,326.73725543)(91.01174154,326.73726336)
\curveto(91.08173744,326.73725543)(91.15673736,326.74225542)(91.23674154,326.75226336)
\curveto(91.74673677,326.7622554)(92.18173634,326.70225546)(92.54174154,326.57226336)
\curveto(92.91173561,326.45225571)(93.24173528,326.29225587)(93.53174154,326.09226336)
\curveto(93.6217349,326.03225613)(93.71173481,325.9622562)(93.80174154,325.88226336)
\curveto(93.89173463,325.81225635)(93.97173455,325.73725643)(94.04174154,325.65726336)
\curveto(94.07173445,325.60725656)(94.11173441,325.5672566)(94.16174154,325.53726336)
\curveto(94.24173428,325.42725674)(94.3167342,325.31225685)(94.38674154,325.19226336)
\curveto(94.45673406,325.08225708)(94.53173399,324.9672572)(94.61174154,324.84726336)
\curveto(94.66173386,324.75725741)(94.70173382,324.6622575)(94.73174154,324.56226336)
\curveto(94.77173375,324.47225769)(94.81173371,324.37225779)(94.85174154,324.26226336)
\curveto(94.90173362,324.13225803)(94.94173358,323.99725817)(94.97174154,323.85726336)
\curveto(95.00173352,323.71725845)(95.03673348,323.57725859)(95.07674154,323.43726336)
\curveto(95.09673342,323.35725881)(95.10173342,323.2672589)(95.09174154,323.16726336)
\curveto(95.09173343,323.07725909)(95.10173342,322.99225917)(95.12174154,322.91226336)
\lineto(95.12174154,322.74726336)
\moveto(92.87174154,323.63226336)
\curveto(92.94173558,323.73225843)(92.94673557,323.85225831)(92.88674154,323.99226336)
\curveto(92.83673568,324.14225802)(92.79673572,324.25225791)(92.76674154,324.32226336)
\curveto(92.62673589,324.59225757)(92.44173608,324.79725737)(92.21174154,324.93726336)
\curveto(91.98173654,325.08725708)(91.66173686,325.167257)(91.25174154,325.17726336)
\curveto(91.2217373,325.15725701)(91.18673733,325.15225701)(91.14674154,325.16226336)
\curveto(91.10673741,325.17225699)(91.07173745,325.17225699)(91.04174154,325.16226336)
\curveto(90.99173753,325.14225702)(90.93673758,325.12725704)(90.87674154,325.11726336)
\curveto(90.8167377,325.11725705)(90.76173776,325.10725706)(90.71174154,325.08726336)
\curveto(90.27173825,324.94725722)(89.94673857,324.67225749)(89.73674154,324.26226336)
\curveto(89.7167388,324.22225794)(89.69173883,324.167258)(89.66174154,324.09726336)
\curveto(89.64173888,324.03725813)(89.62673889,323.97225819)(89.61674154,323.90226336)
\curveto(89.60673891,323.84225832)(89.60673891,323.78225838)(89.61674154,323.72226336)
\curveto(89.63673888,323.6622585)(89.67173885,323.61225855)(89.72174154,323.57226336)
\curveto(89.80173872,323.52225864)(89.91173861,323.49725867)(90.05174154,323.49726336)
\lineto(90.45674154,323.49726336)
\lineto(92.12174154,323.49726336)
\lineto(92.55674154,323.49726336)
\curveto(92.7167358,323.50725866)(92.8217357,323.55225861)(92.87174154,323.63226336)
}
}
{
\newrgbcolor{curcolor}{0 0 0}
\pscustom[linestyle=none,fillstyle=solid,fillcolor=curcolor]
{
}
}
{
\newrgbcolor{curcolor}{0 0 0}
\pscustom[linestyle=none,fillstyle=solid,fillcolor=curcolor]
{
\newpath
\moveto(104.95517904,326.73726336)
\curveto(105.06517373,326.73725543)(105.16017363,326.72725544)(105.24017904,326.70726336)
\curveto(105.33017346,326.68725548)(105.40017339,326.64225552)(105.45017904,326.57226336)
\curveto(105.51017328,326.49225567)(105.54017325,326.35225581)(105.54017904,326.15226336)
\lineto(105.54017904,325.64226336)
\lineto(105.54017904,325.26726336)
\curveto(105.55017324,325.12725704)(105.53517326,325.01725715)(105.49517904,324.93726336)
\curveto(105.45517334,324.8672573)(105.3951734,324.82225734)(105.31517904,324.80226336)
\curveto(105.24517355,324.78225738)(105.16017363,324.77225739)(105.06017904,324.77226336)
\curveto(104.97017382,324.77225739)(104.87017392,324.77725739)(104.76017904,324.78726336)
\curveto(104.66017413,324.79725737)(104.56517423,324.79225737)(104.47517904,324.77226336)
\curveto(104.40517439,324.75225741)(104.33517446,324.73725743)(104.26517904,324.72726336)
\curveto(104.1951746,324.72725744)(104.13017466,324.71725745)(104.07017904,324.69726336)
\curveto(103.91017488,324.64725752)(103.75017504,324.57225759)(103.59017904,324.47226336)
\curveto(103.43017536,324.38225778)(103.30517549,324.27725789)(103.21517904,324.15726336)
\curveto(103.16517563,324.07725809)(103.11017568,323.99225817)(103.05017904,323.90226336)
\curveto(103.00017579,323.82225834)(102.95017584,323.73725843)(102.90017904,323.64726336)
\curveto(102.87017592,323.5672586)(102.84017595,323.48225868)(102.81017904,323.39226336)
\lineto(102.75017904,323.15226336)
\curveto(102.73017606,323.08225908)(102.72017607,323.00725916)(102.72017904,322.92726336)
\curveto(102.72017607,322.85725931)(102.71017608,322.78725938)(102.69017904,322.71726336)
\curveto(102.68017611,322.67725949)(102.67517612,322.63725953)(102.67517904,322.59726336)
\curveto(102.68517611,322.5672596)(102.68517611,322.53725963)(102.67517904,322.50726336)
\lineto(102.67517904,322.26726336)
\curveto(102.65517614,322.19725997)(102.65017614,322.11726005)(102.66017904,322.02726336)
\curveto(102.67017612,321.94726022)(102.67517612,321.8672603)(102.67517904,321.78726336)
\lineto(102.67517904,320.82726336)
\lineto(102.67517904,319.55226336)
\curveto(102.67517612,319.42226274)(102.67017612,319.30226286)(102.66017904,319.19226336)
\curveto(102.65017614,319.08226308)(102.62017617,318.99226317)(102.57017904,318.92226336)
\curveto(102.55017624,318.89226327)(102.51517628,318.8672633)(102.46517904,318.84726336)
\curveto(102.42517637,318.83726333)(102.38017641,318.82726334)(102.33017904,318.81726336)
\lineto(102.25517904,318.81726336)
\curveto(102.20517659,318.80726336)(102.15017664,318.80226336)(102.09017904,318.80226336)
\lineto(101.92517904,318.80226336)
\lineto(101.28017904,318.80226336)
\curveto(101.22017757,318.81226335)(101.15517764,318.81726335)(101.08517904,318.81726336)
\lineto(100.89017904,318.81726336)
\curveto(100.84017795,318.83726333)(100.790178,318.85226331)(100.74017904,318.86226336)
\curveto(100.6901781,318.88226328)(100.65517814,318.91726325)(100.63517904,318.96726336)
\curveto(100.5951782,319.01726315)(100.57017822,319.08726308)(100.56017904,319.17726336)
\lineto(100.56017904,319.47726336)
\lineto(100.56017904,320.49726336)
\lineto(100.56017904,324.72726336)
\lineto(100.56017904,325.83726336)
\lineto(100.56017904,326.12226336)
\curveto(100.56017823,326.22225594)(100.58017821,326.30225586)(100.62017904,326.36226336)
\curveto(100.67017812,326.44225572)(100.74517805,326.49225567)(100.84517904,326.51226336)
\curveto(100.94517785,326.53225563)(101.06517773,326.54225562)(101.20517904,326.54226336)
\lineto(101.97017904,326.54226336)
\curveto(102.0901767,326.54225562)(102.1951766,326.53225563)(102.28517904,326.51226336)
\curveto(102.37517642,326.50225566)(102.44517635,326.45725571)(102.49517904,326.37726336)
\curveto(102.52517627,326.32725584)(102.54017625,326.25725591)(102.54017904,326.16726336)
\lineto(102.57017904,325.89726336)
\curveto(102.58017621,325.81725635)(102.5951762,325.74225642)(102.61517904,325.67226336)
\curveto(102.64517615,325.60225656)(102.6951761,325.5672566)(102.76517904,325.56726336)
\curveto(102.78517601,325.58725658)(102.80517599,325.59725657)(102.82517904,325.59726336)
\curveto(102.84517595,325.59725657)(102.86517593,325.60725656)(102.88517904,325.62726336)
\curveto(102.94517585,325.67725649)(102.9951758,325.73225643)(103.03517904,325.79226336)
\curveto(103.08517571,325.8622563)(103.14517565,325.92225624)(103.21517904,325.97226336)
\curveto(103.25517554,326.00225616)(103.2901755,326.03225613)(103.32017904,326.06226336)
\curveto(103.35017544,326.10225606)(103.38517541,326.13725603)(103.42517904,326.16726336)
\lineto(103.69517904,326.34726336)
\curveto(103.795175,326.40725576)(103.8951749,326.4622557)(103.99517904,326.51226336)
\curveto(104.0951747,326.55225561)(104.1951746,326.58725558)(104.29517904,326.61726336)
\lineto(104.62517904,326.70726336)
\curveto(104.65517414,326.71725545)(104.71017408,326.71725545)(104.79017904,326.70726336)
\curveto(104.88017391,326.70725546)(104.93517386,326.71725545)(104.95517904,326.73726336)
}
}
{
\newrgbcolor{curcolor}{0 0 0}
\pscustom[linestyle=none,fillstyle=solid,fillcolor=curcolor]
{
\newpath
\moveto(113.46158529,322.74726336)
\curveto(113.48157713,322.6672595)(113.48157713,322.57725959)(113.46158529,322.47726336)
\curveto(113.44157717,322.37725979)(113.4065772,322.31225985)(113.35658529,322.28226336)
\curveto(113.3065773,322.24225992)(113.23157738,322.21225995)(113.13158529,322.19226336)
\curveto(113.04157757,322.18225998)(112.93657767,322.17225999)(112.81658529,322.16226336)
\lineto(112.47158529,322.16226336)
\curveto(112.36157825,322.17225999)(112.26157835,322.17725999)(112.17158529,322.17726336)
\lineto(108.51158529,322.17726336)
\lineto(108.30158529,322.17726336)
\curveto(108.24158237,322.17725999)(108.18658242,322.16726)(108.13658529,322.14726336)
\curveto(108.05658255,322.10726006)(108.0065826,322.0672601)(107.98658529,322.02726336)
\curveto(107.96658264,322.00726016)(107.94658266,321.9672602)(107.92658529,321.90726336)
\curveto(107.9065827,321.85726031)(107.90158271,321.80726036)(107.91158529,321.75726336)
\curveto(107.93158268,321.69726047)(107.94158267,321.63726053)(107.94158529,321.57726336)
\curveto(107.95158266,321.52726064)(107.96658264,321.47226069)(107.98658529,321.41226336)
\curveto(108.06658254,321.17226099)(108.16158245,320.97226119)(108.27158529,320.81226336)
\curveto(108.39158222,320.6622615)(108.55158206,320.52726164)(108.75158529,320.40726336)
\curveto(108.83158178,320.35726181)(108.9115817,320.32226184)(108.99158529,320.30226336)
\curveto(109.08158153,320.29226187)(109.17158144,320.27226189)(109.26158529,320.24226336)
\curveto(109.34158127,320.22226194)(109.45158116,320.20726196)(109.59158529,320.19726336)
\curveto(109.73158088,320.18726198)(109.85158076,320.19226197)(109.95158529,320.21226336)
\lineto(110.08658529,320.21226336)
\curveto(110.18658042,320.23226193)(110.27658033,320.25226191)(110.35658529,320.27226336)
\curveto(110.44658016,320.30226186)(110.53158008,320.33226183)(110.61158529,320.36226336)
\curveto(110.7115799,320.41226175)(110.82157979,320.47726169)(110.94158529,320.55726336)
\curveto(111.07157954,320.63726153)(111.16657944,320.71726145)(111.22658529,320.79726336)
\curveto(111.27657933,320.8672613)(111.32657928,320.93226123)(111.37658529,320.99226336)
\curveto(111.43657917,321.0622611)(111.5065791,321.11226105)(111.58658529,321.14226336)
\curveto(111.68657892,321.19226097)(111.8115788,321.21226095)(111.96158529,321.20226336)
\lineto(112.39658529,321.20226336)
\lineto(112.57658529,321.20226336)
\curveto(112.64657796,321.21226095)(112.7065779,321.20726096)(112.75658529,321.18726336)
\lineto(112.90658529,321.18726336)
\curveto(113.0065776,321.167261)(113.07657753,321.14226102)(113.11658529,321.11226336)
\curveto(113.15657745,321.09226107)(113.17657743,321.04726112)(113.17658529,320.97726336)
\curveto(113.18657742,320.90726126)(113.18157743,320.84726132)(113.16158529,320.79726336)
\curveto(113.1115775,320.65726151)(113.05657755,320.53226163)(112.99658529,320.42226336)
\curveto(112.93657767,320.31226185)(112.86657774,320.20226196)(112.78658529,320.09226336)
\curveto(112.56657804,319.7622624)(112.31657829,319.49726267)(112.03658529,319.29726336)
\curveto(111.75657885,319.09726307)(111.4065792,318.92726324)(110.98658529,318.78726336)
\curveto(110.87657973,318.74726342)(110.76657984,318.72226344)(110.65658529,318.71226336)
\curveto(110.54658006,318.70226346)(110.43158018,318.68226348)(110.31158529,318.65226336)
\curveto(110.27158034,318.64226352)(110.22658038,318.64226352)(110.17658529,318.65226336)
\curveto(110.13658047,318.65226351)(110.09658051,318.64726352)(110.05658529,318.63726336)
\lineto(109.89158529,318.63726336)
\curveto(109.84158077,318.61726355)(109.78158083,318.61226355)(109.71158529,318.62226336)
\curveto(109.65158096,318.62226354)(109.59658101,318.62726354)(109.54658529,318.63726336)
\curveto(109.46658114,318.64726352)(109.39658121,318.64726352)(109.33658529,318.63726336)
\curveto(109.27658133,318.62726354)(109.2115814,318.63226353)(109.14158529,318.65226336)
\curveto(109.09158152,318.67226349)(109.03658157,318.68226348)(108.97658529,318.68226336)
\curveto(108.91658169,318.68226348)(108.86158175,318.69226347)(108.81158529,318.71226336)
\curveto(108.70158191,318.73226343)(108.59158202,318.75726341)(108.48158529,318.78726336)
\curveto(108.37158224,318.80726336)(108.27158234,318.84226332)(108.18158529,318.89226336)
\curveto(108.07158254,318.93226323)(107.96658264,318.9672632)(107.86658529,318.99726336)
\curveto(107.77658283,319.03726313)(107.69158292,319.08226308)(107.61158529,319.13226336)
\curveto(107.29158332,319.33226283)(107.0065836,319.5622626)(106.75658529,319.82226336)
\curveto(106.5065841,320.09226207)(106.30158431,320.40226176)(106.14158529,320.75226336)
\curveto(106.09158452,320.8622613)(106.05158456,320.97226119)(106.02158529,321.08226336)
\curveto(105.99158462,321.20226096)(105.95158466,321.32226084)(105.90158529,321.44226336)
\curveto(105.89158472,321.48226068)(105.88658472,321.51726065)(105.88658529,321.54726336)
\curveto(105.88658472,321.58726058)(105.88158473,321.62726054)(105.87158529,321.66726336)
\curveto(105.83158478,321.78726038)(105.8065848,321.91726025)(105.79658529,322.05726336)
\lineto(105.76658529,322.47726336)
\curveto(105.76658484,322.52725964)(105.76158485,322.58225958)(105.75158529,322.64226336)
\curveto(105.75158486,322.70225946)(105.75658485,322.75725941)(105.76658529,322.80726336)
\lineto(105.76658529,322.98726336)
\lineto(105.81158529,323.34726336)
\curveto(105.85158476,323.51725865)(105.88658472,323.68225848)(105.91658529,323.84226336)
\curveto(105.94658466,324.00225816)(105.99158462,324.15225801)(106.05158529,324.29226336)
\curveto(106.48158413,325.33225683)(107.2115834,326.0672561)(108.24158529,326.49726336)
\curveto(108.38158223,326.55725561)(108.52158209,326.59725557)(108.66158529,326.61726336)
\curveto(108.8115818,326.64725552)(108.96658164,326.68225548)(109.12658529,326.72226336)
\curveto(109.2065814,326.73225543)(109.28158133,326.73725543)(109.35158529,326.73726336)
\curveto(109.42158119,326.73725543)(109.49658111,326.74225542)(109.57658529,326.75226336)
\curveto(110.08658052,326.7622554)(110.52158009,326.70225546)(110.88158529,326.57226336)
\curveto(111.25157936,326.45225571)(111.58157903,326.29225587)(111.87158529,326.09226336)
\curveto(111.96157865,326.03225613)(112.05157856,325.9622562)(112.14158529,325.88226336)
\curveto(112.23157838,325.81225635)(112.3115783,325.73725643)(112.38158529,325.65726336)
\curveto(112.4115782,325.60725656)(112.45157816,325.5672566)(112.50158529,325.53726336)
\curveto(112.58157803,325.42725674)(112.65657795,325.31225685)(112.72658529,325.19226336)
\curveto(112.79657781,325.08225708)(112.87157774,324.9672572)(112.95158529,324.84726336)
\curveto(113.00157761,324.75725741)(113.04157757,324.6622575)(113.07158529,324.56226336)
\curveto(113.1115775,324.47225769)(113.15157746,324.37225779)(113.19158529,324.26226336)
\curveto(113.24157737,324.13225803)(113.28157733,323.99725817)(113.31158529,323.85726336)
\curveto(113.34157727,323.71725845)(113.37657723,323.57725859)(113.41658529,323.43726336)
\curveto(113.43657717,323.35725881)(113.44157717,323.2672589)(113.43158529,323.16726336)
\curveto(113.43157718,323.07725909)(113.44157717,322.99225917)(113.46158529,322.91226336)
\lineto(113.46158529,322.74726336)
\moveto(111.21158529,323.63226336)
\curveto(111.28157933,323.73225843)(111.28657932,323.85225831)(111.22658529,323.99226336)
\curveto(111.17657943,324.14225802)(111.13657947,324.25225791)(111.10658529,324.32226336)
\curveto(110.96657964,324.59225757)(110.78157983,324.79725737)(110.55158529,324.93726336)
\curveto(110.32158029,325.08725708)(110.00158061,325.167257)(109.59158529,325.17726336)
\curveto(109.56158105,325.15725701)(109.52658108,325.15225701)(109.48658529,325.16226336)
\curveto(109.44658116,325.17225699)(109.4115812,325.17225699)(109.38158529,325.16226336)
\curveto(109.33158128,325.14225702)(109.27658133,325.12725704)(109.21658529,325.11726336)
\curveto(109.15658145,325.11725705)(109.10158151,325.10725706)(109.05158529,325.08726336)
\curveto(108.611582,324.94725722)(108.28658232,324.67225749)(108.07658529,324.26226336)
\curveto(108.05658255,324.22225794)(108.03158258,324.167258)(108.00158529,324.09726336)
\curveto(107.98158263,324.03725813)(107.96658264,323.97225819)(107.95658529,323.90226336)
\curveto(107.94658266,323.84225832)(107.94658266,323.78225838)(107.95658529,323.72226336)
\curveto(107.97658263,323.6622585)(108.0115826,323.61225855)(108.06158529,323.57226336)
\curveto(108.14158247,323.52225864)(108.25158236,323.49725867)(108.39158529,323.49726336)
\lineto(108.79658529,323.49726336)
\lineto(110.46158529,323.49726336)
\lineto(110.89658529,323.49726336)
\curveto(111.05657955,323.50725866)(111.16157945,323.55225861)(111.21158529,323.63226336)
}
}
{
\newrgbcolor{curcolor}{0 0 0}
\pscustom[linestyle=none,fillstyle=solid,fillcolor=curcolor]
{
\newpath
\moveto(118.27986654,326.75226336)
\curveto(119.08986138,326.77225539)(119.76486071,326.65225551)(120.30486654,326.39226336)
\curveto(120.85485962,326.13225603)(121.28985918,325.7622564)(121.60986654,325.28226336)
\curveto(121.7698587,325.04225712)(121.88985858,324.7672574)(121.96986654,324.45726336)
\curveto(121.98985848,324.40725776)(122.00485847,324.34225782)(122.01486654,324.26226336)
\curveto(122.03485844,324.18225798)(122.03485844,324.11225805)(122.01486654,324.05226336)
\curveto(121.9748585,323.94225822)(121.90485857,323.87725829)(121.80486654,323.85726336)
\curveto(121.70485877,323.84725832)(121.58485889,323.84225832)(121.44486654,323.84226336)
\lineto(120.66486654,323.84226336)
\lineto(120.37986654,323.84226336)
\curveto(120.28986018,323.84225832)(120.21486026,323.8622583)(120.15486654,323.90226336)
\curveto(120.0748604,323.94225822)(120.01986045,324.00225816)(119.98986654,324.08226336)
\curveto(119.95986051,324.17225799)(119.91986055,324.2622579)(119.86986654,324.35226336)
\curveto(119.80986066,324.4622577)(119.74486073,324.5622576)(119.67486654,324.65226336)
\curveto(119.60486087,324.74225742)(119.52486095,324.82225734)(119.43486654,324.89226336)
\curveto(119.29486118,324.98225718)(119.13986133,325.05225711)(118.96986654,325.10226336)
\curveto(118.90986156,325.12225704)(118.84986162,325.13225703)(118.78986654,325.13226336)
\curveto(118.72986174,325.13225703)(118.6748618,325.14225702)(118.62486654,325.16226336)
\lineto(118.47486654,325.16226336)
\curveto(118.2748622,325.162257)(118.11486236,325.14225702)(117.99486654,325.10226336)
\curveto(117.70486277,325.01225715)(117.469863,324.87225729)(117.28986654,324.68226336)
\curveto(117.10986336,324.50225766)(116.96486351,324.28225788)(116.85486654,324.02226336)
\curveto(116.80486367,323.91225825)(116.76486371,323.79225837)(116.73486654,323.66226336)
\curveto(116.71486376,323.54225862)(116.68986378,323.41225875)(116.65986654,323.27226336)
\curveto(116.64986382,323.23225893)(116.64486383,323.19225897)(116.64486654,323.15226336)
\curveto(116.64486383,323.11225905)(116.63986383,323.07225909)(116.62986654,323.03226336)
\curveto(116.60986386,322.93225923)(116.59986387,322.79225937)(116.59986654,322.61226336)
\curveto(116.60986386,322.43225973)(116.62486385,322.29225987)(116.64486654,322.19226336)
\curveto(116.64486383,322.11226005)(116.64986382,322.05726011)(116.65986654,322.02726336)
\curveto(116.67986379,321.95726021)(116.68986378,321.88726028)(116.68986654,321.81726336)
\curveto(116.69986377,321.74726042)(116.71486376,321.67726049)(116.73486654,321.60726336)
\curveto(116.81486366,321.37726079)(116.90986356,321.167261)(117.01986654,320.97726336)
\curveto(117.12986334,320.78726138)(117.2698632,320.62726154)(117.43986654,320.49726336)
\curveto(117.47986299,320.4672617)(117.53986293,320.43226173)(117.61986654,320.39226336)
\curveto(117.72986274,320.32226184)(117.83986263,320.27726189)(117.94986654,320.25726336)
\curveto(118.0698624,320.23726193)(118.21486226,320.21726195)(118.38486654,320.19726336)
\lineto(118.47486654,320.19726336)
\curveto(118.51486196,320.19726197)(118.54486193,320.20226196)(118.56486654,320.21226336)
\lineto(118.69986654,320.21226336)
\curveto(118.7698617,320.23226193)(118.83486164,320.24726192)(118.89486654,320.25726336)
\curveto(118.96486151,320.27726189)(119.02986144,320.29726187)(119.08986654,320.31726336)
\curveto(119.38986108,320.44726172)(119.61986085,320.63726153)(119.77986654,320.88726336)
\curveto(119.81986065,320.93726123)(119.85486062,320.99226117)(119.88486654,321.05226336)
\curveto(119.91486056,321.12226104)(119.93986053,321.18226098)(119.95986654,321.23226336)
\curveto(119.99986047,321.34226082)(120.03486044,321.43726073)(120.06486654,321.51726336)
\curveto(120.09486038,321.60726056)(120.16486031,321.67726049)(120.27486654,321.72726336)
\curveto(120.36486011,321.7672604)(120.50985996,321.78226038)(120.70986654,321.77226336)
\lineto(121.20486654,321.77226336)
\lineto(121.41486654,321.77226336)
\curveto(121.49485898,321.78226038)(121.55985891,321.77726039)(121.60986654,321.75726336)
\lineto(121.72986654,321.75726336)
\lineto(121.84986654,321.72726336)
\curveto(121.88985858,321.72726044)(121.91985855,321.71726045)(121.93986654,321.69726336)
\curveto(121.98985848,321.65726051)(122.01985845,321.59726057)(122.02986654,321.51726336)
\curveto(122.04985842,321.44726072)(122.04985842,321.37226079)(122.02986654,321.29226336)
\curveto(121.93985853,320.9622612)(121.82985864,320.6672615)(121.69986654,320.40726336)
\curveto(121.28985918,319.63726253)(120.63485984,319.10226306)(119.73486654,318.80226336)
\curveto(119.63486084,318.77226339)(119.52986094,318.75226341)(119.41986654,318.74226336)
\curveto(119.30986116,318.72226344)(119.19986127,318.69726347)(119.08986654,318.66726336)
\curveto(119.02986144,318.65726351)(118.9698615,318.65226351)(118.90986654,318.65226336)
\curveto(118.84986162,318.65226351)(118.78986168,318.64726352)(118.72986654,318.63726336)
\lineto(118.56486654,318.63726336)
\curveto(118.51486196,318.61726355)(118.43986203,318.61226355)(118.33986654,318.62226336)
\curveto(118.23986223,318.62226354)(118.16486231,318.62726354)(118.11486654,318.63726336)
\curveto(118.03486244,318.65726351)(117.95986251,318.6672635)(117.88986654,318.66726336)
\curveto(117.82986264,318.65726351)(117.76486271,318.6622635)(117.69486654,318.68226336)
\lineto(117.54486654,318.71226336)
\curveto(117.49486298,318.71226345)(117.44486303,318.71726345)(117.39486654,318.72726336)
\curveto(117.28486319,318.75726341)(117.17986329,318.78726338)(117.07986654,318.81726336)
\curveto(116.97986349,318.84726332)(116.88486359,318.88226328)(116.79486654,318.92226336)
\curveto(116.32486415,319.12226304)(115.92986454,319.37726279)(115.60986654,319.68726336)
\curveto(115.28986518,320.00726216)(115.02986544,320.40226176)(114.82986654,320.87226336)
\curveto(114.77986569,320.9622612)(114.73986573,321.05726111)(114.70986654,321.15726336)
\lineto(114.61986654,321.48726336)
\curveto(114.60986586,321.52726064)(114.60486587,321.5622606)(114.60486654,321.59226336)
\curveto(114.60486587,321.63226053)(114.59486588,321.67726049)(114.57486654,321.72726336)
\curveto(114.55486592,321.79726037)(114.54486593,321.8672603)(114.54486654,321.93726336)
\curveto(114.54486593,322.01726015)(114.53486594,322.09226007)(114.51486654,322.16226336)
\lineto(114.51486654,322.41726336)
\curveto(114.49486598,322.4672597)(114.48486599,322.52225964)(114.48486654,322.58226336)
\curveto(114.48486599,322.65225951)(114.49486598,322.71225945)(114.51486654,322.76226336)
\curveto(114.52486595,322.81225935)(114.52486595,322.85725931)(114.51486654,322.89726336)
\curveto(114.50486597,322.93725923)(114.50486597,322.97725919)(114.51486654,323.01726336)
\curveto(114.53486594,323.08725908)(114.53986593,323.15225901)(114.52986654,323.21226336)
\curveto(114.52986594,323.27225889)(114.53986593,323.33225883)(114.55986654,323.39226336)
\curveto(114.60986586,323.57225859)(114.64986582,323.74225842)(114.67986654,323.90226336)
\curveto(114.70986576,324.07225809)(114.75486572,324.23725793)(114.81486654,324.39726336)
\curveto(115.03486544,324.90725726)(115.30986516,325.33225683)(115.63986654,325.67226336)
\curveto(115.97986449,326.01225615)(116.40986406,326.28725588)(116.92986654,326.49726336)
\curveto(117.0698634,326.55725561)(117.21486326,326.59725557)(117.36486654,326.61726336)
\curveto(117.51486296,326.64725552)(117.6698628,326.68225548)(117.82986654,326.72226336)
\curveto(117.90986256,326.73225543)(117.98486249,326.73725543)(118.05486654,326.73726336)
\curveto(118.12486235,326.73725543)(118.19986227,326.74225542)(118.27986654,326.75226336)
}
}
{
\newrgbcolor{curcolor}{0 0 0}
\pscustom[linestyle=none,fillstyle=solid,fillcolor=curcolor]
{
\newpath
\moveto(123.74314779,326.52726336)
\lineto(124.86814779,326.52726336)
\curveto(124.97814536,326.52725564)(125.07814526,326.52225564)(125.16814779,326.51226336)
\curveto(125.25814508,326.50225566)(125.32314501,326.4672557)(125.36314779,326.40726336)
\curveto(125.41314492,326.34725582)(125.44314489,326.2622559)(125.45314779,326.15226336)
\curveto(125.46314487,326.05225611)(125.46814487,325.94725622)(125.46814779,325.83726336)
\lineto(125.46814779,324.78726336)
\lineto(125.46814779,322.55226336)
\curveto(125.46814487,322.19225997)(125.48314485,321.85226031)(125.51314779,321.53226336)
\curveto(125.54314479,321.21226095)(125.6331447,320.94726122)(125.78314779,320.73726336)
\curveto(125.92314441,320.52726164)(126.14814419,320.37726179)(126.45814779,320.28726336)
\curveto(126.50814383,320.27726189)(126.54814379,320.27226189)(126.57814779,320.27226336)
\curveto(126.61814372,320.27226189)(126.66314367,320.2672619)(126.71314779,320.25726336)
\curveto(126.76314357,320.24726192)(126.81814352,320.24226192)(126.87814779,320.24226336)
\curveto(126.9381434,320.24226192)(126.98314335,320.24726192)(127.01314779,320.25726336)
\curveto(127.06314327,320.27726189)(127.10314323,320.28226188)(127.13314779,320.27226336)
\curveto(127.17314316,320.2622619)(127.21314312,320.2672619)(127.25314779,320.28726336)
\curveto(127.46314287,320.33726183)(127.62814271,320.40226176)(127.74814779,320.48226336)
\curveto(127.92814241,320.59226157)(128.06814227,320.73226143)(128.16814779,320.90226336)
\curveto(128.27814206,321.08226108)(128.35314198,321.27726089)(128.39314779,321.48726336)
\curveto(128.44314189,321.70726046)(128.47314186,321.94726022)(128.48314779,322.20726336)
\curveto(128.49314184,322.47725969)(128.49814184,322.75725941)(128.49814779,323.04726336)
\lineto(128.49814779,324.86226336)
\lineto(128.49814779,325.83726336)
\lineto(128.49814779,326.10726336)
\curveto(128.49814184,326.20725596)(128.51814182,326.28725588)(128.55814779,326.34726336)
\curveto(128.60814173,326.43725573)(128.68314165,326.48725568)(128.78314779,326.49726336)
\curveto(128.88314145,326.51725565)(129.00314133,326.52725564)(129.14314779,326.52726336)
\lineto(129.93814779,326.52726336)
\lineto(130.22314779,326.52726336)
\curveto(130.31314002,326.52725564)(130.38813995,326.50725566)(130.44814779,326.46726336)
\curveto(130.52813981,326.41725575)(130.57313976,326.34225582)(130.58314779,326.24226336)
\curveto(130.59313974,326.14225602)(130.59813974,326.02725614)(130.59814779,325.89726336)
\lineto(130.59814779,324.75726336)
\lineto(130.59814779,320.54226336)
\lineto(130.59814779,319.47726336)
\lineto(130.59814779,319.17726336)
\curveto(130.59813974,319.07726309)(130.57813976,319.00226316)(130.53814779,318.95226336)
\curveto(130.48813985,318.87226329)(130.41313992,318.82726334)(130.31314779,318.81726336)
\curveto(130.21314012,318.80726336)(130.10814023,318.80226336)(129.99814779,318.80226336)
\lineto(129.18814779,318.80226336)
\curveto(129.07814126,318.80226336)(128.97814136,318.80726336)(128.88814779,318.81726336)
\curveto(128.80814153,318.82726334)(128.74314159,318.8672633)(128.69314779,318.93726336)
\curveto(128.67314166,318.9672632)(128.65314168,319.01226315)(128.63314779,319.07226336)
\curveto(128.62314171,319.13226303)(128.60814173,319.19226297)(128.58814779,319.25226336)
\curveto(128.57814176,319.31226285)(128.56314177,319.3672628)(128.54314779,319.41726336)
\curveto(128.52314181,319.4672627)(128.49314184,319.49726267)(128.45314779,319.50726336)
\curveto(128.4331419,319.52726264)(128.40814193,319.53226263)(128.37814779,319.52226336)
\curveto(128.34814199,319.51226265)(128.32314201,319.50226266)(128.30314779,319.49226336)
\curveto(128.2331421,319.45226271)(128.17314216,319.40726276)(128.12314779,319.35726336)
\curveto(128.07314226,319.30726286)(128.01814232,319.2622629)(127.95814779,319.22226336)
\curveto(127.91814242,319.19226297)(127.87814246,319.15726301)(127.83814779,319.11726336)
\curveto(127.80814253,319.08726308)(127.76814257,319.05726311)(127.71814779,319.02726336)
\curveto(127.48814285,318.88726328)(127.21814312,318.77726339)(126.90814779,318.69726336)
\curveto(126.8381435,318.67726349)(126.76814357,318.6672635)(126.69814779,318.66726336)
\curveto(126.62814371,318.65726351)(126.55314378,318.64226352)(126.47314779,318.62226336)
\curveto(126.4331439,318.61226355)(126.38814395,318.61226355)(126.33814779,318.62226336)
\curveto(126.29814404,318.62226354)(126.25814408,318.61726355)(126.21814779,318.60726336)
\curveto(126.18814415,318.59726357)(126.12314421,318.59726357)(126.02314779,318.60726336)
\curveto(125.9331444,318.60726356)(125.87314446,318.61226355)(125.84314779,318.62226336)
\curveto(125.79314454,318.62226354)(125.74314459,318.62726354)(125.69314779,318.63726336)
\lineto(125.54314779,318.63726336)
\curveto(125.42314491,318.6672635)(125.30814503,318.69226347)(125.19814779,318.71226336)
\curveto(125.08814525,318.73226343)(124.97814536,318.7622634)(124.86814779,318.80226336)
\curveto(124.81814552,318.82226334)(124.77314556,318.83726333)(124.73314779,318.84726336)
\curveto(124.70314563,318.8672633)(124.66314567,318.88726328)(124.61314779,318.90726336)
\curveto(124.26314607,319.09726307)(123.98314635,319.3622628)(123.77314779,319.70226336)
\curveto(123.64314669,319.91226225)(123.54814679,320.162262)(123.48814779,320.45226336)
\curveto(123.42814691,320.75226141)(123.38814695,321.0672611)(123.36814779,321.39726336)
\curveto(123.35814698,321.73726043)(123.35314698,322.08226008)(123.35314779,322.43226336)
\curveto(123.36314697,322.79225937)(123.36814697,323.14725902)(123.36814779,323.49726336)
\lineto(123.36814779,325.53726336)
\curveto(123.36814697,325.6672565)(123.36314697,325.81725635)(123.35314779,325.98726336)
\curveto(123.35314698,326.167256)(123.37814696,326.29725587)(123.42814779,326.37726336)
\curveto(123.45814688,326.42725574)(123.51814682,326.47225569)(123.60814779,326.51226336)
\curveto(123.66814667,326.51225565)(123.71314662,326.51725565)(123.74314779,326.52726336)
}
}
{
\newrgbcolor{curcolor}{0 0 0}
\pscustom[linestyle=none,fillstyle=solid,fillcolor=curcolor]
{
\newpath
\moveto(136.65439779,326.73726336)
\curveto(136.76439248,326.73725543)(136.85939238,326.72725544)(136.93939779,326.70726336)
\curveto(137.02939221,326.68725548)(137.09939214,326.64225552)(137.14939779,326.57226336)
\curveto(137.20939203,326.49225567)(137.239392,326.35225581)(137.23939779,326.15226336)
\lineto(137.23939779,325.64226336)
\lineto(137.23939779,325.26726336)
\curveto(137.24939199,325.12725704)(137.23439201,325.01725715)(137.19439779,324.93726336)
\curveto(137.15439209,324.8672573)(137.09439215,324.82225734)(137.01439779,324.80226336)
\curveto(136.9443923,324.78225738)(136.85939238,324.77225739)(136.75939779,324.77226336)
\curveto(136.66939257,324.77225739)(136.56939267,324.77725739)(136.45939779,324.78726336)
\curveto(136.35939288,324.79725737)(136.26439298,324.79225737)(136.17439779,324.77226336)
\curveto(136.10439314,324.75225741)(136.03439321,324.73725743)(135.96439779,324.72726336)
\curveto(135.89439335,324.72725744)(135.82939341,324.71725745)(135.76939779,324.69726336)
\curveto(135.60939363,324.64725752)(135.44939379,324.57225759)(135.28939779,324.47226336)
\curveto(135.12939411,324.38225778)(135.00439424,324.27725789)(134.91439779,324.15726336)
\curveto(134.86439438,324.07725809)(134.80939443,323.99225817)(134.74939779,323.90226336)
\curveto(134.69939454,323.82225834)(134.64939459,323.73725843)(134.59939779,323.64726336)
\curveto(134.56939467,323.5672586)(134.5393947,323.48225868)(134.50939779,323.39226336)
\lineto(134.44939779,323.15226336)
\curveto(134.42939481,323.08225908)(134.41939482,323.00725916)(134.41939779,322.92726336)
\curveto(134.41939482,322.85725931)(134.40939483,322.78725938)(134.38939779,322.71726336)
\curveto(134.37939486,322.67725949)(134.37439487,322.63725953)(134.37439779,322.59726336)
\curveto(134.38439486,322.5672596)(134.38439486,322.53725963)(134.37439779,322.50726336)
\lineto(134.37439779,322.26726336)
\curveto(134.35439489,322.19725997)(134.34939489,322.11726005)(134.35939779,322.02726336)
\curveto(134.36939487,321.94726022)(134.37439487,321.8672603)(134.37439779,321.78726336)
\lineto(134.37439779,320.82726336)
\lineto(134.37439779,319.55226336)
\curveto(134.37439487,319.42226274)(134.36939487,319.30226286)(134.35939779,319.19226336)
\curveto(134.34939489,319.08226308)(134.31939492,318.99226317)(134.26939779,318.92226336)
\curveto(134.24939499,318.89226327)(134.21439503,318.8672633)(134.16439779,318.84726336)
\curveto(134.12439512,318.83726333)(134.07939516,318.82726334)(134.02939779,318.81726336)
\lineto(133.95439779,318.81726336)
\curveto(133.90439534,318.80726336)(133.84939539,318.80226336)(133.78939779,318.80226336)
\lineto(133.62439779,318.80226336)
\lineto(132.97939779,318.80226336)
\curveto(132.91939632,318.81226335)(132.85439639,318.81726335)(132.78439779,318.81726336)
\lineto(132.58939779,318.81726336)
\curveto(132.5393967,318.83726333)(132.48939675,318.85226331)(132.43939779,318.86226336)
\curveto(132.38939685,318.88226328)(132.35439689,318.91726325)(132.33439779,318.96726336)
\curveto(132.29439695,319.01726315)(132.26939697,319.08726308)(132.25939779,319.17726336)
\lineto(132.25939779,319.47726336)
\lineto(132.25939779,320.49726336)
\lineto(132.25939779,324.72726336)
\lineto(132.25939779,325.83726336)
\lineto(132.25939779,326.12226336)
\curveto(132.25939698,326.22225594)(132.27939696,326.30225586)(132.31939779,326.36226336)
\curveto(132.36939687,326.44225572)(132.4443968,326.49225567)(132.54439779,326.51226336)
\curveto(132.6443966,326.53225563)(132.76439648,326.54225562)(132.90439779,326.54226336)
\lineto(133.66939779,326.54226336)
\curveto(133.78939545,326.54225562)(133.89439535,326.53225563)(133.98439779,326.51226336)
\curveto(134.07439517,326.50225566)(134.1443951,326.45725571)(134.19439779,326.37726336)
\curveto(134.22439502,326.32725584)(134.239395,326.25725591)(134.23939779,326.16726336)
\lineto(134.26939779,325.89726336)
\curveto(134.27939496,325.81725635)(134.29439495,325.74225642)(134.31439779,325.67226336)
\curveto(134.3443949,325.60225656)(134.39439485,325.5672566)(134.46439779,325.56726336)
\curveto(134.48439476,325.58725658)(134.50439474,325.59725657)(134.52439779,325.59726336)
\curveto(134.5443947,325.59725657)(134.56439468,325.60725656)(134.58439779,325.62726336)
\curveto(134.6443946,325.67725649)(134.69439455,325.73225643)(134.73439779,325.79226336)
\curveto(134.78439446,325.8622563)(134.8443944,325.92225624)(134.91439779,325.97226336)
\curveto(134.95439429,326.00225616)(134.98939425,326.03225613)(135.01939779,326.06226336)
\curveto(135.04939419,326.10225606)(135.08439416,326.13725603)(135.12439779,326.16726336)
\lineto(135.39439779,326.34726336)
\curveto(135.49439375,326.40725576)(135.59439365,326.4622557)(135.69439779,326.51226336)
\curveto(135.79439345,326.55225561)(135.89439335,326.58725558)(135.99439779,326.61726336)
\lineto(136.32439779,326.70726336)
\curveto(136.35439289,326.71725545)(136.40939283,326.71725545)(136.48939779,326.70726336)
\curveto(136.57939266,326.70725546)(136.63439261,326.71725545)(136.65439779,326.73726336)
}
}
{
\newrgbcolor{curcolor}{0 0 0}
\pscustom[linestyle=none,fillstyle=solid,fillcolor=curcolor]
{
\newpath
\moveto(141.02947592,326.75226336)
\curveto(141.77947142,326.77225539)(142.42947077,326.68725548)(142.97947592,326.49726336)
\curveto(143.53946966,326.31725585)(143.96446923,326.00225616)(144.25447592,325.55226336)
\curveto(144.32446887,325.44225672)(144.38446881,325.32725684)(144.43447592,325.20726336)
\curveto(144.4944687,325.09725707)(144.54446865,324.97225719)(144.58447592,324.83226336)
\curveto(144.60446859,324.77225739)(144.61446858,324.70725746)(144.61447592,324.63726336)
\curveto(144.61446858,324.5672576)(144.60446859,324.50725766)(144.58447592,324.45726336)
\curveto(144.54446865,324.39725777)(144.48946871,324.35725781)(144.41947592,324.33726336)
\curveto(144.36946883,324.31725785)(144.30946889,324.30725786)(144.23947592,324.30726336)
\lineto(144.02947592,324.30726336)
\lineto(143.36947592,324.30726336)
\curveto(143.2994699,324.30725786)(143.22946997,324.30225786)(143.15947592,324.29226336)
\curveto(143.08947011,324.29225787)(143.02447017,324.30225786)(142.96447592,324.32226336)
\curveto(142.86447033,324.34225782)(142.78947041,324.38225778)(142.73947592,324.44226336)
\curveto(142.68947051,324.50225766)(142.64447055,324.5622576)(142.60447592,324.62226336)
\lineto(142.48447592,324.83226336)
\curveto(142.45447074,324.91225725)(142.40447079,324.97725719)(142.33447592,325.02726336)
\curveto(142.23447096,325.10725706)(142.13447106,325.167257)(142.03447592,325.20726336)
\curveto(141.94447125,325.24725692)(141.82947137,325.28225688)(141.68947592,325.31226336)
\curveto(141.61947158,325.33225683)(141.51447168,325.34725682)(141.37447592,325.35726336)
\curveto(141.24447195,325.3672568)(141.14447205,325.3622568)(141.07447592,325.34226336)
\lineto(140.96947592,325.34226336)
\lineto(140.81947592,325.31226336)
\curveto(140.77947242,325.31225685)(140.73447246,325.30725686)(140.68447592,325.29726336)
\curveto(140.51447268,325.24725692)(140.37447282,325.17725699)(140.26447592,325.08726336)
\curveto(140.16447303,325.00725716)(140.0944731,324.88225728)(140.05447592,324.71226336)
\curveto(140.03447316,324.64225752)(140.03447316,324.57725759)(140.05447592,324.51726336)
\curveto(140.07447312,324.45725771)(140.0944731,324.40725776)(140.11447592,324.36726336)
\curveto(140.18447301,324.24725792)(140.26447293,324.15225801)(140.35447592,324.08226336)
\curveto(140.45447274,324.01225815)(140.56947263,323.95225821)(140.69947592,323.90226336)
\curveto(140.88947231,323.82225834)(141.0944721,323.75225841)(141.31447592,323.69226336)
\lineto(142.00447592,323.54226336)
\curveto(142.24447095,323.50225866)(142.47447072,323.45225871)(142.69447592,323.39226336)
\curveto(142.92447027,323.34225882)(143.13947006,323.27725889)(143.33947592,323.19726336)
\curveto(143.42946977,323.15725901)(143.51446968,323.12225904)(143.59447592,323.09226336)
\curveto(143.68446951,323.07225909)(143.76946943,323.03725913)(143.84947592,322.98726336)
\curveto(144.03946916,322.8672593)(144.20946899,322.73725943)(144.35947592,322.59726336)
\curveto(144.51946868,322.45725971)(144.64446855,322.28225988)(144.73447592,322.07226336)
\curveto(144.76446843,322.00226016)(144.78946841,321.93226023)(144.80947592,321.86226336)
\curveto(144.82946837,321.79226037)(144.84946835,321.71726045)(144.86947592,321.63726336)
\curveto(144.87946832,321.57726059)(144.88446831,321.48226068)(144.88447592,321.35226336)
\curveto(144.8944683,321.23226093)(144.8944683,321.13726103)(144.88447592,321.06726336)
\lineto(144.88447592,320.99226336)
\curveto(144.86446833,320.93226123)(144.84946835,320.87226129)(144.83947592,320.81226336)
\curveto(144.83946836,320.7622614)(144.83446836,320.71226145)(144.82447592,320.66226336)
\curveto(144.75446844,320.3622618)(144.64446855,320.09726207)(144.49447592,319.86726336)
\curveto(144.33446886,319.62726254)(144.13946906,319.43226273)(143.90947592,319.28226336)
\curveto(143.67946952,319.13226303)(143.41946978,319.00226316)(143.12947592,318.89226336)
\curveto(143.01947018,318.84226332)(142.8994703,318.80726336)(142.76947592,318.78726336)
\curveto(142.64947055,318.7672634)(142.52947067,318.74226342)(142.40947592,318.71226336)
\curveto(142.31947088,318.69226347)(142.22447097,318.68226348)(142.12447592,318.68226336)
\curveto(142.03447116,318.67226349)(141.94447125,318.65726351)(141.85447592,318.63726336)
\lineto(141.58447592,318.63726336)
\curveto(141.52447167,318.61726355)(141.41947178,318.60726356)(141.26947592,318.60726336)
\curveto(141.12947207,318.60726356)(141.02947217,318.61726355)(140.96947592,318.63726336)
\curveto(140.93947226,318.63726353)(140.90447229,318.64226352)(140.86447592,318.65226336)
\lineto(140.75947592,318.65226336)
\curveto(140.63947256,318.67226349)(140.51947268,318.68726348)(140.39947592,318.69726336)
\curveto(140.27947292,318.70726346)(140.16447303,318.72726344)(140.05447592,318.75726336)
\curveto(139.66447353,318.8672633)(139.31947388,318.99226317)(139.01947592,319.13226336)
\curveto(138.71947448,319.28226288)(138.46447473,319.50226266)(138.25447592,319.79226336)
\curveto(138.11447508,319.98226218)(137.9944752,320.20226196)(137.89447592,320.45226336)
\curveto(137.87447532,320.51226165)(137.85447534,320.59226157)(137.83447592,320.69226336)
\curveto(137.81447538,320.74226142)(137.7994754,320.81226135)(137.78947592,320.90226336)
\curveto(137.77947542,320.99226117)(137.78447541,321.0672611)(137.80447592,321.12726336)
\curveto(137.83447536,321.19726097)(137.88447531,321.24726092)(137.95447592,321.27726336)
\curveto(138.00447519,321.29726087)(138.06447513,321.30726086)(138.13447592,321.30726336)
\lineto(138.35947592,321.30726336)
\lineto(139.06447592,321.30726336)
\lineto(139.30447592,321.30726336)
\curveto(139.38447381,321.30726086)(139.45447374,321.29726087)(139.51447592,321.27726336)
\curveto(139.62447357,321.23726093)(139.6944735,321.17226099)(139.72447592,321.08226336)
\curveto(139.76447343,320.99226117)(139.80947339,320.89726127)(139.85947592,320.79726336)
\curveto(139.87947332,320.74726142)(139.91447328,320.68226148)(139.96447592,320.60226336)
\curveto(140.02447317,320.52226164)(140.07447312,320.47226169)(140.11447592,320.45226336)
\curveto(140.23447296,320.35226181)(140.34947285,320.27226189)(140.45947592,320.21226336)
\curveto(140.56947263,320.162262)(140.70947249,320.11226205)(140.87947592,320.06226336)
\curveto(140.92947227,320.04226212)(140.97947222,320.03226213)(141.02947592,320.03226336)
\curveto(141.07947212,320.04226212)(141.12947207,320.04226212)(141.17947592,320.03226336)
\curveto(141.25947194,320.01226215)(141.34447185,320.00226216)(141.43447592,320.00226336)
\curveto(141.53447166,320.01226215)(141.61947158,320.02726214)(141.68947592,320.04726336)
\curveto(141.73947146,320.05726211)(141.78447141,320.0622621)(141.82447592,320.06226336)
\curveto(141.87447132,320.0622621)(141.92447127,320.07226209)(141.97447592,320.09226336)
\curveto(142.11447108,320.14226202)(142.23947096,320.20226196)(142.34947592,320.27226336)
\curveto(142.46947073,320.34226182)(142.56447063,320.43226173)(142.63447592,320.54226336)
\curveto(142.68447051,320.62226154)(142.72447047,320.74726142)(142.75447592,320.91726336)
\curveto(142.77447042,320.98726118)(142.77447042,321.05226111)(142.75447592,321.11226336)
\curveto(142.73447046,321.17226099)(142.71447048,321.22226094)(142.69447592,321.26226336)
\curveto(142.62447057,321.40226076)(142.53447066,321.50726066)(142.42447592,321.57726336)
\curveto(142.32447087,321.64726052)(142.20447099,321.71226045)(142.06447592,321.77226336)
\curveto(141.87447132,321.85226031)(141.67447152,321.91726025)(141.46447592,321.96726336)
\curveto(141.25447194,322.01726015)(141.04447215,322.07226009)(140.83447592,322.13226336)
\curveto(140.75447244,322.15226001)(140.66947253,322.16726)(140.57947592,322.17726336)
\curveto(140.4994727,322.18725998)(140.41947278,322.20225996)(140.33947592,322.22226336)
\curveto(140.01947318,322.31225985)(139.71447348,322.39725977)(139.42447592,322.47726336)
\curveto(139.13447406,322.5672596)(138.86947433,322.69725947)(138.62947592,322.86726336)
\curveto(138.34947485,323.0672591)(138.14447505,323.33725883)(138.01447592,323.67726336)
\curveto(137.9944752,323.74725842)(137.97447522,323.84225832)(137.95447592,323.96226336)
\curveto(137.93447526,324.03225813)(137.91947528,324.11725805)(137.90947592,324.21726336)
\curveto(137.8994753,324.31725785)(137.90447529,324.40725776)(137.92447592,324.48726336)
\curveto(137.94447525,324.53725763)(137.94947525,324.57725759)(137.93947592,324.60726336)
\curveto(137.92947527,324.64725752)(137.93447526,324.69225747)(137.95447592,324.74226336)
\curveto(137.97447522,324.85225731)(137.9944752,324.95225721)(138.01447592,325.04226336)
\curveto(138.04447515,325.14225702)(138.07947512,325.23725693)(138.11947592,325.32726336)
\curveto(138.24947495,325.61725655)(138.42947477,325.85225631)(138.65947592,326.03226336)
\curveto(138.88947431,326.21225595)(139.14947405,326.35725581)(139.43947592,326.46726336)
\curveto(139.54947365,326.51725565)(139.66447353,326.55225561)(139.78447592,326.57226336)
\curveto(139.90447329,326.60225556)(140.02947317,326.63225553)(140.15947592,326.66226336)
\curveto(140.21947298,326.68225548)(140.27947292,326.69225547)(140.33947592,326.69226336)
\lineto(140.51947592,326.72226336)
\curveto(140.5994726,326.73225543)(140.68447251,326.73725543)(140.77447592,326.73726336)
\curveto(140.86447233,326.73725543)(140.94947225,326.74225542)(141.02947592,326.75226336)
}
}
{
\newrgbcolor{curcolor}{0 0 0}
\pscustom[linestyle=none,fillstyle=solid,fillcolor=curcolor]
{
\newpath
\moveto(153.88611654,322.98726336)
\curveto(153.90610797,322.92725924)(153.91610796,322.84225932)(153.91611654,322.73226336)
\curveto(153.91610796,322.62225954)(153.90610797,322.53725963)(153.88611654,322.47726336)
\lineto(153.88611654,322.32726336)
\curveto(153.86610801,322.24725992)(153.85610802,322.16726)(153.85611654,322.08726336)
\curveto(153.86610801,322.00726016)(153.86110802,321.92726024)(153.84111654,321.84726336)
\curveto(153.82110806,321.77726039)(153.80610807,321.71226045)(153.79611654,321.65226336)
\curveto(153.78610809,321.59226057)(153.7761081,321.52726064)(153.76611654,321.45726336)
\curveto(153.72610815,321.34726082)(153.69110819,321.23226093)(153.66111654,321.11226336)
\curveto(153.63110825,321.00226116)(153.59110829,320.89726127)(153.54111654,320.79726336)
\curveto(153.33110855,320.31726185)(153.05610882,319.92726224)(152.71611654,319.62726336)
\curveto(152.3761095,319.32726284)(151.96610991,319.07726309)(151.48611654,318.87726336)
\curveto(151.36611051,318.82726334)(151.24111064,318.79226337)(151.11111654,318.77226336)
\curveto(150.99111089,318.74226342)(150.86611101,318.71226345)(150.73611654,318.68226336)
\curveto(150.68611119,318.6622635)(150.63111125,318.65226351)(150.57111654,318.65226336)
\curveto(150.51111137,318.65226351)(150.45611142,318.64726352)(150.40611654,318.63726336)
\lineto(150.30111654,318.63726336)
\curveto(150.27111161,318.62726354)(150.24111164,318.62226354)(150.21111654,318.62226336)
\curveto(150.16111172,318.61226355)(150.0811118,318.60726356)(149.97111654,318.60726336)
\curveto(149.86111202,318.59726357)(149.7761121,318.60226356)(149.71611654,318.62226336)
\lineto(149.56611654,318.62226336)
\curveto(149.51611236,318.63226353)(149.46111242,318.63726353)(149.40111654,318.63726336)
\curveto(149.35111253,318.62726354)(149.30111258,318.63226353)(149.25111654,318.65226336)
\curveto(149.21111267,318.6622635)(149.17111271,318.6672635)(149.13111654,318.66726336)
\curveto(149.10111278,318.6672635)(149.06111282,318.67226349)(149.01111654,318.68226336)
\curveto(148.91111297,318.71226345)(148.81111307,318.73726343)(148.71111654,318.75726336)
\curveto(148.61111327,318.77726339)(148.51611336,318.80726336)(148.42611654,318.84726336)
\curveto(148.30611357,318.88726328)(148.19111369,318.92726324)(148.08111654,318.96726336)
\curveto(147.9811139,319.00726316)(147.876114,319.05726311)(147.76611654,319.11726336)
\curveto(147.41611446,319.32726284)(147.11611476,319.57226259)(146.86611654,319.85226336)
\curveto(146.61611526,320.13226203)(146.40611547,320.4672617)(146.23611654,320.85726336)
\curveto(146.18611569,320.94726122)(146.14611573,321.04226112)(146.11611654,321.14226336)
\curveto(146.09611578,321.24226092)(146.07111581,321.34726082)(146.04111654,321.45726336)
\curveto(146.02111586,321.50726066)(146.01111587,321.55226061)(146.01111654,321.59226336)
\curveto(146.01111587,321.63226053)(146.00111588,321.67726049)(145.98111654,321.72726336)
\curveto(145.96111592,321.80726036)(145.95111593,321.88726028)(145.95111654,321.96726336)
\curveto(145.95111593,322.05726011)(145.94111594,322.14226002)(145.92111654,322.22226336)
\curveto(145.91111597,322.27225989)(145.90611597,322.31725985)(145.90611654,322.35726336)
\lineto(145.90611654,322.49226336)
\curveto(145.88611599,322.55225961)(145.876116,322.63725953)(145.87611654,322.74726336)
\curveto(145.88611599,322.85725931)(145.90111598,322.94225922)(145.92111654,323.00226336)
\lineto(145.92111654,323.10726336)
\curveto(145.93111595,323.15725901)(145.93111595,323.20725896)(145.92111654,323.25726336)
\curveto(145.92111596,323.31725885)(145.93111595,323.37225879)(145.95111654,323.42226336)
\curveto(145.96111592,323.47225869)(145.96611591,323.51725865)(145.96611654,323.55726336)
\curveto(145.96611591,323.60725856)(145.9761159,323.65725851)(145.99611654,323.70726336)
\curveto(146.03611584,323.83725833)(146.07111581,323.9622582)(146.10111654,324.08226336)
\curveto(146.13111575,324.21225795)(146.17111571,324.33725783)(146.22111654,324.45726336)
\curveto(146.40111548,324.8672573)(146.61611526,325.20725696)(146.86611654,325.47726336)
\curveto(147.11611476,325.75725641)(147.42111446,326.01225615)(147.78111654,326.24226336)
\curveto(147.881114,326.29225587)(147.98611389,326.33725583)(148.09611654,326.37726336)
\curveto(148.20611367,326.41725575)(148.31611356,326.4622557)(148.42611654,326.51226336)
\curveto(148.55611332,326.5622556)(148.69111319,326.59725557)(148.83111654,326.61726336)
\curveto(148.97111291,326.63725553)(149.11611276,326.6672555)(149.26611654,326.70726336)
\curveto(149.34611253,326.71725545)(149.42111246,326.72225544)(149.49111654,326.72226336)
\curveto(149.56111232,326.72225544)(149.63111225,326.72725544)(149.70111654,326.73726336)
\curveto(150.2811116,326.74725542)(150.7811111,326.68725548)(151.20111654,326.55726336)
\curveto(151.63111025,326.42725574)(152.01110987,326.24725592)(152.34111654,326.01726336)
\curveto(152.45110943,325.93725623)(152.56110932,325.84725632)(152.67111654,325.74726336)
\curveto(152.79110909,325.65725651)(152.89110899,325.55725661)(152.97111654,325.44726336)
\curveto(153.05110883,325.34725682)(153.12110876,325.24725692)(153.18111654,325.14726336)
\curveto(153.25110863,325.04725712)(153.32110856,324.94225722)(153.39111654,324.83226336)
\curveto(153.46110842,324.72225744)(153.51610836,324.60225756)(153.55611654,324.47226336)
\curveto(153.59610828,324.35225781)(153.64110824,324.22225794)(153.69111654,324.08226336)
\curveto(153.72110816,324.00225816)(153.74610813,323.91725825)(153.76611654,323.82726336)
\lineto(153.82611654,323.55726336)
\curveto(153.83610804,323.51725865)(153.84110804,323.47725869)(153.84111654,323.43726336)
\curveto(153.84110804,323.39725877)(153.84610803,323.35725881)(153.85611654,323.31726336)
\curveto(153.876108,323.2672589)(153.881108,323.21225895)(153.87111654,323.15226336)
\curveto(153.86110802,323.09225907)(153.86610801,323.03725913)(153.88611654,322.98726336)
\moveto(151.78611654,322.44726336)
\curveto(151.79611008,322.49725967)(151.80111008,322.5672596)(151.80111654,322.65726336)
\curveto(151.80111008,322.75725941)(151.79611008,322.83225933)(151.78611654,322.88226336)
\lineto(151.78611654,323.00226336)
\curveto(151.76611011,323.05225911)(151.75611012,323.10725906)(151.75611654,323.16726336)
\curveto(151.75611012,323.22725894)(151.75111013,323.28225888)(151.74111654,323.33226336)
\curveto(151.74111014,323.37225879)(151.73611014,323.40225876)(151.72611654,323.42226336)
\lineto(151.66611654,323.66226336)
\curveto(151.65611022,323.75225841)(151.63611024,323.83725833)(151.60611654,323.91726336)
\curveto(151.49611038,324.17725799)(151.36611051,324.39725777)(151.21611654,324.57726336)
\curveto(151.06611081,324.7672574)(150.86611101,324.91725725)(150.61611654,325.02726336)
\curveto(150.55611132,325.04725712)(150.49611138,325.0622571)(150.43611654,325.07226336)
\curveto(150.3761115,325.09225707)(150.31111157,325.11225705)(150.24111654,325.13226336)
\curveto(150.16111172,325.15225701)(150.0761118,325.15725701)(149.98611654,325.14726336)
\lineto(149.71611654,325.14726336)
\curveto(149.68611219,325.12725704)(149.65111223,325.11725705)(149.61111654,325.11726336)
\curveto(149.57111231,325.12725704)(149.53611234,325.12725704)(149.50611654,325.11726336)
\lineto(149.29611654,325.05726336)
\curveto(149.23611264,325.04725712)(149.1811127,325.02725714)(149.13111654,324.99726336)
\curveto(148.881113,324.88725728)(148.6761132,324.72725744)(148.51611654,324.51726336)
\curveto(148.36611351,324.31725785)(148.24611363,324.08225808)(148.15611654,323.81226336)
\curveto(148.12611375,323.71225845)(148.10111378,323.60725856)(148.08111654,323.49726336)
\curveto(148.07111381,323.38725878)(148.05611382,323.27725889)(148.03611654,323.16726336)
\curveto(148.02611385,323.11725905)(148.02111386,323.0672591)(148.02111654,323.01726336)
\lineto(148.02111654,322.86726336)
\curveto(148.00111388,322.79725937)(147.99111389,322.69225947)(147.99111654,322.55226336)
\curveto(148.00111388,322.41225975)(148.01611386,322.30725986)(148.03611654,322.23726336)
\lineto(148.03611654,322.10226336)
\curveto(148.05611382,322.02226014)(148.07111381,321.94226022)(148.08111654,321.86226336)
\curveto(148.09111379,321.79226037)(148.10611377,321.71726045)(148.12611654,321.63726336)
\curveto(148.22611365,321.33726083)(148.33111355,321.09226107)(148.44111654,320.90226336)
\curveto(148.56111332,320.72226144)(148.74611313,320.55726161)(148.99611654,320.40726336)
\curveto(149.06611281,320.35726181)(149.14111274,320.31726185)(149.22111654,320.28726336)
\curveto(149.31111257,320.25726191)(149.40111248,320.23226193)(149.49111654,320.21226336)
\curveto(149.53111235,320.20226196)(149.56611231,320.19726197)(149.59611654,320.19726336)
\curveto(149.62611225,320.20726196)(149.66111222,320.20726196)(149.70111654,320.19726336)
\lineto(149.82111654,320.16726336)
\curveto(149.87111201,320.167262)(149.91611196,320.17226199)(149.95611654,320.18226336)
\lineto(150.07611654,320.18226336)
\curveto(150.15611172,320.20226196)(150.23611164,320.21726195)(150.31611654,320.22726336)
\curveto(150.39611148,320.23726193)(150.47111141,320.25726191)(150.54111654,320.28726336)
\curveto(150.80111108,320.38726178)(151.01111087,320.52226164)(151.17111654,320.69226336)
\curveto(151.33111055,320.8622613)(151.46611041,321.07226109)(151.57611654,321.32226336)
\curveto(151.61611026,321.42226074)(151.64611023,321.52226064)(151.66611654,321.62226336)
\curveto(151.68611019,321.72226044)(151.71111017,321.82726034)(151.74111654,321.93726336)
\curveto(151.75111013,321.97726019)(151.75611012,322.01226015)(151.75611654,322.04226336)
\curveto(151.75611012,322.08226008)(151.76111012,322.12226004)(151.77111654,322.16226336)
\lineto(151.77111654,322.29726336)
\curveto(151.77111011,322.34725982)(151.7761101,322.39725977)(151.78611654,322.44726336)
}
}
{
\newrgbcolor{curcolor}{0 0 0}
\pscustom[linestyle=none,fillstyle=solid,fillcolor=curcolor]
{
\newpath
\moveto(415.20953951,343.25226336)
\curveto(415.20952902,343.22225769)(415.20952902,343.18225773)(415.20953951,343.13226336)
\curveto(415.21952901,343.08225783)(415.22452901,343.02725789)(415.22453951,342.96726336)
\curveto(415.22452901,342.90725801)(415.21952901,342.85225806)(415.20953951,342.80226336)
\curveto(415.20952902,342.75225816)(415.20952902,342.7172582)(415.20953951,342.69726336)
\curveto(415.20952902,342.62725829)(415.20452903,342.55725836)(415.19453951,342.48726336)
\curveto(415.19452904,342.42725849)(415.19452904,342.36725855)(415.19453951,342.30726336)
\curveto(415.17452906,342.25725866)(415.16452907,342.20725871)(415.16453951,342.15726336)
\curveto(415.17452906,342.10725881)(415.17452906,342.05725886)(415.16453951,342.00726336)
\curveto(415.14452909,341.89725902)(415.1295291,341.78725913)(415.11953951,341.67726336)
\curveto(415.10952912,341.56725935)(415.08952914,341.45725946)(415.05953951,341.34726336)
\curveto(415.00952922,341.17725974)(414.96452927,341.0122599)(414.92453951,340.85226336)
\curveto(414.88452935,340.70226021)(414.8345294,340.55226036)(414.77453951,340.40226336)
\curveto(414.60452963,339.98226093)(414.39452984,339.60226131)(414.14453951,339.26226336)
\curveto(413.89453034,338.92226199)(413.59453064,338.63226228)(413.24453951,338.39226336)
\curveto(413.04453119,338.25226266)(412.8345314,338.13226278)(412.61453951,338.03226336)
\curveto(412.40453183,337.93226298)(412.17453206,337.84226307)(411.92453951,337.76226336)
\curveto(411.82453241,337.73226318)(411.71953251,337.70726321)(411.60953951,337.68726336)
\curveto(411.50953272,337.67726324)(411.40453283,337.65726326)(411.29453951,337.62726336)
\curveto(411.24453299,337.6172633)(411.19453304,337.6122633)(411.14453951,337.61226336)
\curveto(411.10453313,337.6122633)(411.05953317,337.60726331)(411.00953951,337.59726336)
\curveto(410.96953326,337.58726333)(410.9295333,337.58226333)(410.88953951,337.58226336)
\curveto(410.84953338,337.59226332)(410.80453343,337.59226332)(410.75453951,337.58226336)
\curveto(410.7345335,337.57226334)(410.70453353,337.56726335)(410.66453951,337.56726336)
\curveto(410.62453361,337.57726334)(410.59453364,337.57726334)(410.57453951,337.56726336)
\curveto(410.49453374,337.54726337)(410.39453384,337.54226337)(410.27453951,337.55226336)
\curveto(410.15453408,337.56226335)(410.04953418,337.56726335)(409.95953951,337.56726336)
\lineto(406.46453951,337.56726336)
\curveto(406.29453794,337.56726335)(406.14953808,337.57226334)(406.02953951,337.58226336)
\curveto(405.91953831,337.60226331)(405.83953839,337.67226324)(405.78953951,337.79226336)
\curveto(405.75953847,337.87226304)(405.74453849,337.99226292)(405.74453951,338.15226336)
\curveto(405.75453848,338.32226259)(405.75953847,338.46226245)(405.75953951,338.57226336)
\lineto(405.75953951,347.37726336)
\curveto(405.75953847,347.49725342)(405.75453848,347.62225329)(405.74453951,347.75226336)
\curveto(405.74453849,347.89225302)(405.76953846,348.00225291)(405.81953951,348.08226336)
\curveto(405.85953837,348.14225277)(405.9345383,348.19225272)(406.04453951,348.23226336)
\curveto(406.06453817,348.24225267)(406.08453815,348.24225267)(406.10453951,348.23226336)
\curveto(406.12453811,348.23225268)(406.14453809,348.23725268)(406.16453951,348.24726336)
\lineto(410.19953951,348.24726336)
\curveto(410.25953397,348.24725267)(410.31953391,348.24725267)(410.37953951,348.24726336)
\curveto(410.44953378,348.25725266)(410.50953372,348.25725266)(410.55953951,348.24726336)
\lineto(410.73953951,348.24726336)
\curveto(410.78953344,348.22725269)(410.84453339,348.2172527)(410.90453951,348.21726336)
\curveto(410.96453327,348.22725269)(411.01953321,348.22225269)(411.06953951,348.20226336)
\curveto(411.1295331,348.18225273)(411.18453305,348.17225274)(411.23453951,348.17226336)
\curveto(411.29453294,348.18225273)(411.35453288,348.17725274)(411.41453951,348.15726336)
\curveto(411.55453268,348.12725279)(411.68953254,348.09725282)(411.81953951,348.06726336)
\curveto(411.94953228,348.04725287)(412.07453216,348.0122529)(412.19453951,347.96226336)
\curveto(412.30453193,347.912253)(412.41453182,347.86725305)(412.52453951,347.82726336)
\curveto(412.6345316,347.78725313)(412.73953149,347.73725318)(412.83953951,347.67726336)
\curveto(413.08953114,347.5172534)(413.31953091,347.36225355)(413.52953951,347.21226336)
\lineto(413.61953951,347.12226336)
\curveto(413.71953051,347.04225387)(413.80953042,346.95225396)(413.88953951,346.85226336)
\lineto(414.02453951,346.73226336)
\curveto(414.07453016,346.65225426)(414.1295301,346.57225434)(414.18953951,346.49226336)
\curveto(414.25952997,346.42225449)(414.31952991,346.34725457)(414.36953951,346.26726336)
\curveto(414.49952973,346.05725486)(414.61452962,345.83225508)(414.71453951,345.59226336)
\curveto(414.81452942,345.36225555)(414.90452933,345.1172558)(414.98453951,344.85726336)
\curveto(415.0345292,344.72725619)(415.06452917,344.59225632)(415.07453951,344.45226336)
\curveto(415.09452914,344.3122566)(415.11952911,344.17225674)(415.14953951,344.03226336)
\curveto(415.14952908,343.98225693)(415.14952908,343.93725698)(415.14953951,343.89726336)
\curveto(415.15952907,343.86725705)(415.16452907,343.83225708)(415.16453951,343.79226336)
\curveto(415.18452905,343.73225718)(415.18952904,343.66725725)(415.17953951,343.59726336)
\curveto(415.17952905,343.52725739)(415.18952904,343.46725745)(415.20953951,343.41726336)
\lineto(415.20953951,343.25226336)
\moveto(412.86953951,342.53226336)
\curveto(412.88953134,342.58225833)(412.89953133,342.66225825)(412.89953951,342.77226336)
\curveto(412.89953133,342.88225803)(412.88953134,342.96225795)(412.86953951,343.01226336)
\lineto(412.86953951,343.29726336)
\curveto(412.84953138,343.38725753)(412.8345314,343.48225743)(412.82453951,343.58226336)
\curveto(412.82453141,343.68225723)(412.81453142,343.77225714)(412.79453951,343.85226336)
\curveto(412.77453146,343.90225701)(412.76453147,343.94725697)(412.76453951,343.98726336)
\curveto(412.77453146,344.03725688)(412.76953146,344.08725683)(412.74953951,344.13726336)
\curveto(412.69953153,344.29725662)(412.64953158,344.44725647)(412.59953951,344.58726336)
\curveto(412.55953167,344.73725618)(412.49953173,344.87725604)(412.41953951,345.00726336)
\curveto(412.26953196,345.24725567)(412.09453214,345.45225546)(411.89453951,345.62226336)
\curveto(411.70453253,345.80225511)(411.46953276,345.95225496)(411.18953951,346.07226336)
\curveto(411.09953313,346.10225481)(411.00953322,346.12725479)(410.91953951,346.14726336)
\curveto(410.8295334,346.17725474)(410.73953349,346.20225471)(410.64953951,346.22226336)
\curveto(410.56953366,346.23225468)(410.49453374,346.23725468)(410.42453951,346.23726336)
\curveto(410.36453387,346.24725467)(410.29453394,346.26225465)(410.21453951,346.28226336)
\curveto(410.17453406,346.29225462)(410.1345341,346.29225462)(410.09453951,346.28226336)
\curveto(410.05453418,346.28225463)(410.01953421,346.28725463)(409.98953951,346.29726336)
\lineto(409.65953951,346.29726336)
\curveto(409.60953462,346.30725461)(409.55453468,346.30725461)(409.49453951,346.29726336)
\lineto(409.31453951,346.29726336)
\lineto(408.63953951,346.29726336)
\curveto(408.61953561,346.27725464)(408.58453565,346.27225464)(408.53453951,346.28226336)
\curveto(408.49453574,346.29225462)(408.45953577,346.29225462)(408.42953951,346.28226336)
\lineto(408.27953951,346.22226336)
\curveto(408.229536,346.2122547)(408.18953604,346.18225473)(408.15953951,346.13226336)
\curveto(408.11953611,346.08225483)(408.09953613,346.0122549)(408.09953951,345.92226336)
\lineto(408.09953951,345.62226336)
\curveto(408.09953613,345.49225542)(408.09453614,345.35725556)(408.08453951,345.21726336)
\lineto(408.08453951,344.79726336)
\lineto(408.08453951,340.61226336)
\curveto(408.08453615,340.55226036)(408.07953615,340.48726043)(408.06953951,340.41726336)
\curveto(408.06953616,340.34726057)(408.07953615,340.28726063)(408.09953951,340.23726336)
\lineto(408.09953951,340.08726336)
\lineto(408.09953951,339.87726336)
\curveto(408.10953612,339.8172611)(408.12453611,339.76226115)(408.14453951,339.71226336)
\curveto(408.20453603,339.59226132)(408.31953591,339.52726139)(408.48953951,339.51726336)
\lineto(409.01453951,339.51726336)
\lineto(410.19953951,339.51726336)
\curveto(410.59953363,339.52726139)(410.93953329,339.58726133)(411.21953951,339.69726336)
\curveto(411.58953264,339.84726107)(411.87953235,340.04726087)(412.08953951,340.29726336)
\curveto(412.30953192,340.54726037)(412.49453174,340.85726006)(412.64453951,341.22726336)
\curveto(412.68453155,341.30725961)(412.71453152,341.39725952)(412.73453951,341.49726336)
\curveto(412.75453148,341.59725932)(412.77953145,341.69725922)(412.80953951,341.79726336)
\lineto(412.80953951,341.91726336)
\curveto(412.8295314,341.98725893)(412.83953139,342.06225885)(412.83953951,342.14226336)
\curveto(412.83953139,342.22225869)(412.84953138,342.30225861)(412.86953951,342.38226336)
\lineto(412.86953951,342.53226336)
}
}
{
\newrgbcolor{curcolor}{0 0 0}
\pscustom[linestyle=none,fillstyle=solid,fillcolor=curcolor]
{
\newpath
\moveto(418.70805513,348.14226336)
\curveto(418.77805218,348.06225285)(418.81305215,347.94225297)(418.81305513,347.78226336)
\lineto(418.81305513,347.31726336)
\lineto(418.81305513,346.91226336)
\curveto(418.81305215,346.77225414)(418.77805218,346.67725424)(418.70805513,346.62726336)
\curveto(418.64805231,346.57725434)(418.56805239,346.54725437)(418.46805513,346.53726336)
\curveto(418.37805258,346.52725439)(418.27805268,346.52225439)(418.16805513,346.52226336)
\lineto(417.32805513,346.52226336)
\curveto(417.21805374,346.52225439)(417.11805384,346.52725439)(417.02805513,346.53726336)
\curveto(416.94805401,346.54725437)(416.87805408,346.57725434)(416.81805513,346.62726336)
\curveto(416.77805418,346.65725426)(416.74805421,346.7122542)(416.72805513,346.79226336)
\curveto(416.71805424,346.88225403)(416.70805425,346.97725394)(416.69805513,347.07726336)
\lineto(416.69805513,347.40726336)
\curveto(416.70805425,347.5172534)(416.71305425,347.6122533)(416.71305513,347.69226336)
\lineto(416.71305513,347.90226336)
\curveto(416.72305424,347.97225294)(416.74305422,348.03225288)(416.77305513,348.08226336)
\curveto(416.79305417,348.12225279)(416.81805414,348.15225276)(416.84805513,348.17226336)
\lineto(416.96805513,348.23226336)
\curveto(416.98805397,348.23225268)(417.01305395,348.23225268)(417.04305513,348.23226336)
\curveto(417.07305389,348.24225267)(417.09805386,348.24725267)(417.11805513,348.24726336)
\lineto(418.21305513,348.24726336)
\curveto(418.31305265,348.24725267)(418.40805255,348.24225267)(418.49805513,348.23226336)
\curveto(418.58805237,348.22225269)(418.6580523,348.19225272)(418.70805513,348.14226336)
\moveto(418.81305513,338.37726336)
\curveto(418.81305215,338.17726274)(418.80805215,338.00726291)(418.79805513,337.86726336)
\curveto(418.78805217,337.72726319)(418.69805226,337.63226328)(418.52805513,337.58226336)
\curveto(418.46805249,337.56226335)(418.40305256,337.55226336)(418.33305513,337.55226336)
\curveto(418.2630527,337.56226335)(418.18805277,337.56726335)(418.10805513,337.56726336)
\lineto(417.26805513,337.56726336)
\curveto(417.17805378,337.56726335)(417.08805387,337.57226334)(416.99805513,337.58226336)
\curveto(416.91805404,337.59226332)(416.8580541,337.62226329)(416.81805513,337.67226336)
\curveto(416.7580542,337.74226317)(416.72305424,337.82726309)(416.71305513,337.92726336)
\lineto(416.71305513,338.27226336)
\lineto(416.71305513,344.60226336)
\lineto(416.71305513,344.90226336)
\curveto(416.71305425,345.00225591)(416.73305423,345.08225583)(416.77305513,345.14226336)
\curveto(416.83305413,345.2122557)(416.91805404,345.25725566)(417.02805513,345.27726336)
\curveto(417.04805391,345.28725563)(417.07305389,345.28725563)(417.10305513,345.27726336)
\curveto(417.14305382,345.27725564)(417.17305379,345.28225563)(417.19305513,345.29226336)
\lineto(417.94305513,345.29226336)
\lineto(418.13805513,345.29226336)
\curveto(418.21805274,345.30225561)(418.28305268,345.30225561)(418.33305513,345.29226336)
\lineto(418.45305513,345.29226336)
\curveto(418.51305245,345.27225564)(418.56805239,345.25725566)(418.61805513,345.24726336)
\curveto(418.66805229,345.23725568)(418.70805225,345.20725571)(418.73805513,345.15726336)
\curveto(418.77805218,345.10725581)(418.79805216,345.03725588)(418.79805513,344.94726336)
\curveto(418.80805215,344.85725606)(418.81305215,344.76225615)(418.81305513,344.66226336)
\lineto(418.81305513,338.37726336)
}
}
{
\newrgbcolor{curcolor}{0 0 0}
\pscustom[linestyle=none,fillstyle=solid,fillcolor=curcolor]
{
\newpath
\moveto(423.44524263,345.50226336)
\curveto(424.19523813,345.52225539)(424.84523748,345.43725548)(425.39524263,345.24726336)
\curveto(425.95523637,345.06725585)(426.38023595,344.75225616)(426.67024263,344.30226336)
\curveto(426.74023559,344.19225672)(426.80023553,344.07725684)(426.85024263,343.95726336)
\curveto(426.91023542,343.84725707)(426.96023537,343.72225719)(427.00024263,343.58226336)
\curveto(427.02023531,343.52225739)(427.0302353,343.45725746)(427.03024263,343.38726336)
\curveto(427.0302353,343.3172576)(427.02023531,343.25725766)(427.00024263,343.20726336)
\curveto(426.96023537,343.14725777)(426.90523542,343.10725781)(426.83524263,343.08726336)
\curveto(426.78523554,343.06725785)(426.7252356,343.05725786)(426.65524263,343.05726336)
\lineto(426.44524263,343.05726336)
\lineto(425.78524263,343.05726336)
\curveto(425.71523661,343.05725786)(425.64523668,343.05225786)(425.57524263,343.04226336)
\curveto(425.50523682,343.04225787)(425.44023689,343.05225786)(425.38024263,343.07226336)
\curveto(425.28023705,343.09225782)(425.20523712,343.13225778)(425.15524263,343.19226336)
\curveto(425.10523722,343.25225766)(425.06023727,343.3122576)(425.02024263,343.37226336)
\lineto(424.90024263,343.58226336)
\curveto(424.87023746,343.66225725)(424.82023751,343.72725719)(424.75024263,343.77726336)
\curveto(424.65023768,343.85725706)(424.55023778,343.917257)(424.45024263,343.95726336)
\curveto(424.36023797,343.99725692)(424.24523808,344.03225688)(424.10524263,344.06226336)
\curveto(424.03523829,344.08225683)(423.9302384,344.09725682)(423.79024263,344.10726336)
\curveto(423.66023867,344.1172568)(423.56023877,344.1122568)(423.49024263,344.09226336)
\lineto(423.38524263,344.09226336)
\lineto(423.23524263,344.06226336)
\curveto(423.19523913,344.06225685)(423.15023918,344.05725686)(423.10024263,344.04726336)
\curveto(422.9302394,343.99725692)(422.79023954,343.92725699)(422.68024263,343.83726336)
\curveto(422.58023975,343.75725716)(422.51023982,343.63225728)(422.47024263,343.46226336)
\curveto(422.45023988,343.39225752)(422.45023988,343.32725759)(422.47024263,343.26726336)
\curveto(422.49023984,343.20725771)(422.51023982,343.15725776)(422.53024263,343.11726336)
\curveto(422.60023973,342.99725792)(422.68023965,342.90225801)(422.77024263,342.83226336)
\curveto(422.87023946,342.76225815)(422.98523934,342.70225821)(423.11524263,342.65226336)
\curveto(423.30523902,342.57225834)(423.51023882,342.50225841)(423.73024263,342.44226336)
\lineto(424.42024263,342.29226336)
\curveto(424.66023767,342.25225866)(424.89023744,342.20225871)(425.11024263,342.14226336)
\curveto(425.34023699,342.09225882)(425.55523677,342.02725889)(425.75524263,341.94726336)
\curveto(425.84523648,341.90725901)(425.9302364,341.87225904)(426.01024263,341.84226336)
\curveto(426.10023623,341.82225909)(426.18523614,341.78725913)(426.26524263,341.73726336)
\curveto(426.45523587,341.6172593)(426.6252357,341.48725943)(426.77524263,341.34726336)
\curveto(426.93523539,341.20725971)(427.06023527,341.03225988)(427.15024263,340.82226336)
\curveto(427.18023515,340.75226016)(427.20523512,340.68226023)(427.22524263,340.61226336)
\curveto(427.24523508,340.54226037)(427.26523506,340.46726045)(427.28524263,340.38726336)
\curveto(427.29523503,340.32726059)(427.30023503,340.23226068)(427.30024263,340.10226336)
\curveto(427.31023502,339.98226093)(427.31023502,339.88726103)(427.30024263,339.81726336)
\lineto(427.30024263,339.74226336)
\curveto(427.28023505,339.68226123)(427.26523506,339.62226129)(427.25524263,339.56226336)
\curveto(427.25523507,339.5122614)(427.25023508,339.46226145)(427.24024263,339.41226336)
\curveto(427.17023516,339.1122618)(427.06023527,338.84726207)(426.91024263,338.61726336)
\curveto(426.75023558,338.37726254)(426.55523577,338.18226273)(426.32524263,338.03226336)
\curveto(426.09523623,337.88226303)(425.83523649,337.75226316)(425.54524263,337.64226336)
\curveto(425.43523689,337.59226332)(425.31523701,337.55726336)(425.18524263,337.53726336)
\curveto(425.06523726,337.5172634)(424.94523738,337.49226342)(424.82524263,337.46226336)
\curveto(424.73523759,337.44226347)(424.64023769,337.43226348)(424.54024263,337.43226336)
\curveto(424.45023788,337.42226349)(424.36023797,337.40726351)(424.27024263,337.38726336)
\lineto(424.00024263,337.38726336)
\curveto(423.94023839,337.36726355)(423.83523849,337.35726356)(423.68524263,337.35726336)
\curveto(423.54523878,337.35726356)(423.44523888,337.36726355)(423.38524263,337.38726336)
\curveto(423.35523897,337.38726353)(423.32023901,337.39226352)(423.28024263,337.40226336)
\lineto(423.17524263,337.40226336)
\curveto(423.05523927,337.42226349)(422.93523939,337.43726348)(422.81524263,337.44726336)
\curveto(422.69523963,337.45726346)(422.58023975,337.47726344)(422.47024263,337.50726336)
\curveto(422.08024025,337.6172633)(421.73524059,337.74226317)(421.43524263,337.88226336)
\curveto(421.13524119,338.03226288)(420.88024145,338.25226266)(420.67024263,338.54226336)
\curveto(420.5302418,338.73226218)(420.41024192,338.95226196)(420.31024263,339.20226336)
\curveto(420.29024204,339.26226165)(420.27024206,339.34226157)(420.25024263,339.44226336)
\curveto(420.2302421,339.49226142)(420.21524211,339.56226135)(420.20524263,339.65226336)
\curveto(420.19524213,339.74226117)(420.20024213,339.8172611)(420.22024263,339.87726336)
\curveto(420.25024208,339.94726097)(420.30024203,339.99726092)(420.37024263,340.02726336)
\curveto(420.42024191,340.04726087)(420.48024185,340.05726086)(420.55024263,340.05726336)
\lineto(420.77524263,340.05726336)
\lineto(421.48024263,340.05726336)
\lineto(421.72024263,340.05726336)
\curveto(421.80024053,340.05726086)(421.87024046,340.04726087)(421.93024263,340.02726336)
\curveto(422.04024029,339.98726093)(422.11024022,339.92226099)(422.14024263,339.83226336)
\curveto(422.18024015,339.74226117)(422.2252401,339.64726127)(422.27524263,339.54726336)
\curveto(422.29524003,339.49726142)(422.33024,339.43226148)(422.38024263,339.35226336)
\curveto(422.44023989,339.27226164)(422.49023984,339.22226169)(422.53024263,339.20226336)
\curveto(422.65023968,339.10226181)(422.76523956,339.02226189)(422.87524263,338.96226336)
\curveto(422.98523934,338.912262)(423.1252392,338.86226205)(423.29524263,338.81226336)
\curveto(423.34523898,338.79226212)(423.39523893,338.78226213)(423.44524263,338.78226336)
\curveto(423.49523883,338.79226212)(423.54523878,338.79226212)(423.59524263,338.78226336)
\curveto(423.67523865,338.76226215)(423.76023857,338.75226216)(423.85024263,338.75226336)
\curveto(423.95023838,338.76226215)(424.03523829,338.77726214)(424.10524263,338.79726336)
\curveto(424.15523817,338.80726211)(424.20023813,338.8122621)(424.24024263,338.81226336)
\curveto(424.29023804,338.8122621)(424.34023799,338.82226209)(424.39024263,338.84226336)
\curveto(424.5302378,338.89226202)(424.65523767,338.95226196)(424.76524263,339.02226336)
\curveto(424.88523744,339.09226182)(424.98023735,339.18226173)(425.05024263,339.29226336)
\curveto(425.10023723,339.37226154)(425.14023719,339.49726142)(425.17024263,339.66726336)
\curveto(425.19023714,339.73726118)(425.19023714,339.80226111)(425.17024263,339.86226336)
\curveto(425.15023718,339.92226099)(425.1302372,339.97226094)(425.11024263,340.01226336)
\curveto(425.04023729,340.15226076)(424.95023738,340.25726066)(424.84024263,340.32726336)
\curveto(424.74023759,340.39726052)(424.62023771,340.46226045)(424.48024263,340.52226336)
\curveto(424.29023804,340.60226031)(424.09023824,340.66726025)(423.88024263,340.71726336)
\curveto(423.67023866,340.76726015)(423.46023887,340.82226009)(423.25024263,340.88226336)
\curveto(423.17023916,340.90226001)(423.08523924,340.91726)(422.99524263,340.92726336)
\curveto(422.91523941,340.93725998)(422.83523949,340.95225996)(422.75524263,340.97226336)
\curveto(422.43523989,341.06225985)(422.1302402,341.14725977)(421.84024263,341.22726336)
\curveto(421.55024078,341.3172596)(421.28524104,341.44725947)(421.04524263,341.61726336)
\curveto(420.76524156,341.8172591)(420.56024177,342.08725883)(420.43024263,342.42726336)
\curveto(420.41024192,342.49725842)(420.39024194,342.59225832)(420.37024263,342.71226336)
\curveto(420.35024198,342.78225813)(420.33524199,342.86725805)(420.32524263,342.96726336)
\curveto(420.31524201,343.06725785)(420.32024201,343.15725776)(420.34024263,343.23726336)
\curveto(420.36024197,343.28725763)(420.36524196,343.32725759)(420.35524263,343.35726336)
\curveto(420.34524198,343.39725752)(420.35024198,343.44225747)(420.37024263,343.49226336)
\curveto(420.39024194,343.60225731)(420.41024192,343.70225721)(420.43024263,343.79226336)
\curveto(420.46024187,343.89225702)(420.49524183,343.98725693)(420.53524263,344.07726336)
\curveto(420.66524166,344.36725655)(420.84524148,344.60225631)(421.07524263,344.78226336)
\curveto(421.30524102,344.96225595)(421.56524076,345.10725581)(421.85524263,345.21726336)
\curveto(421.96524036,345.26725565)(422.08024025,345.30225561)(422.20024263,345.32226336)
\curveto(422.32024001,345.35225556)(422.44523988,345.38225553)(422.57524263,345.41226336)
\curveto(422.63523969,345.43225548)(422.69523963,345.44225547)(422.75524263,345.44226336)
\lineto(422.93524263,345.47226336)
\curveto(423.01523931,345.48225543)(423.10023923,345.48725543)(423.19024263,345.48726336)
\curveto(423.28023905,345.48725543)(423.36523896,345.49225542)(423.44524263,345.50226336)
}
}
{
\newrgbcolor{curcolor}{0 0 0}
\pscustom[linestyle=none,fillstyle=solid,fillcolor=curcolor]
{
\newpath
\moveto(429.58188326,347.60226336)
\lineto(430.58688326,347.60226336)
\curveto(430.73688027,347.60225331)(430.86688014,347.59225332)(430.97688326,347.57226336)
\curveto(431.09687991,347.56225335)(431.18187983,347.50225341)(431.23188326,347.39226336)
\curveto(431.25187976,347.34225357)(431.26187975,347.28225363)(431.26188326,347.21226336)
\lineto(431.26188326,347.00226336)
\lineto(431.26188326,346.32726336)
\curveto(431.26187975,346.27725464)(431.25687975,346.2172547)(431.24688326,346.14726336)
\curveto(431.24687976,346.08725483)(431.25187976,346.03225488)(431.26188326,345.98226336)
\lineto(431.26188326,345.81726336)
\curveto(431.26187975,345.73725518)(431.26687974,345.66225525)(431.27688326,345.59226336)
\curveto(431.28687972,345.53225538)(431.3118797,345.47725544)(431.35188326,345.42726336)
\curveto(431.42187959,345.33725558)(431.54687946,345.28725563)(431.72688326,345.27726336)
\lineto(432.26688326,345.27726336)
\lineto(432.44688326,345.27726336)
\curveto(432.5068785,345.27725564)(432.56187845,345.26725565)(432.61188326,345.24726336)
\curveto(432.72187829,345.19725572)(432.78187823,345.10725581)(432.79188326,344.97726336)
\curveto(432.8118782,344.84725607)(432.82187819,344.70225621)(432.82188326,344.54226336)
\lineto(432.82188326,344.33226336)
\curveto(432.83187818,344.26225665)(432.82687818,344.20225671)(432.80688326,344.15226336)
\curveto(432.75687825,343.99225692)(432.65187836,343.90725701)(432.49188326,343.89726336)
\curveto(432.33187868,343.88725703)(432.15187886,343.88225703)(431.95188326,343.88226336)
\lineto(431.81688326,343.88226336)
\curveto(431.77687923,343.89225702)(431.74187927,343.89225702)(431.71188326,343.88226336)
\curveto(431.67187934,343.87225704)(431.63687937,343.86725705)(431.60688326,343.86726336)
\curveto(431.57687943,343.87725704)(431.54687946,343.87225704)(431.51688326,343.85226336)
\curveto(431.43687957,343.83225708)(431.37687963,343.78725713)(431.33688326,343.71726336)
\curveto(431.3068797,343.65725726)(431.28187973,343.58225733)(431.26188326,343.49226336)
\curveto(431.25187976,343.44225747)(431.25187976,343.38725753)(431.26188326,343.32726336)
\curveto(431.27187974,343.26725765)(431.27187974,343.2122577)(431.26188326,343.16226336)
\lineto(431.26188326,342.23226336)
\lineto(431.26188326,340.47726336)
\curveto(431.26187975,340.22726069)(431.26687974,340.00726091)(431.27688326,339.81726336)
\curveto(431.29687971,339.63726128)(431.36187965,339.47726144)(431.47188326,339.33726336)
\curveto(431.52187949,339.27726164)(431.58687942,339.23226168)(431.66688326,339.20226336)
\lineto(431.93688326,339.14226336)
\curveto(431.96687904,339.13226178)(431.99687901,339.12726179)(432.02688326,339.12726336)
\curveto(432.06687894,339.13726178)(432.09687891,339.13726178)(432.11688326,339.12726336)
\lineto(432.28188326,339.12726336)
\curveto(432.39187862,339.12726179)(432.48687852,339.12226179)(432.56688326,339.11226336)
\curveto(432.64687836,339.10226181)(432.7118783,339.06226185)(432.76188326,338.99226336)
\curveto(432.80187821,338.93226198)(432.82187819,338.85226206)(432.82188326,338.75226336)
\lineto(432.82188326,338.46726336)
\curveto(432.82187819,338.25726266)(432.81687819,338.06226285)(432.80688326,337.88226336)
\curveto(432.8068782,337.7122632)(432.72687828,337.59726332)(432.56688326,337.53726336)
\curveto(432.51687849,337.5172634)(432.47187854,337.5122634)(432.43188326,337.52226336)
\curveto(432.39187862,337.52226339)(432.34687866,337.5122634)(432.29688326,337.49226336)
\lineto(432.14688326,337.49226336)
\curveto(432.12687888,337.49226342)(432.09687891,337.49726342)(432.05688326,337.50726336)
\curveto(432.01687899,337.50726341)(431.98187903,337.50226341)(431.95188326,337.49226336)
\curveto(431.90187911,337.48226343)(431.84687916,337.48226343)(431.78688326,337.49226336)
\lineto(431.63688326,337.49226336)
\lineto(431.48688326,337.49226336)
\curveto(431.43687957,337.48226343)(431.39187962,337.48226343)(431.35188326,337.49226336)
\lineto(431.18688326,337.49226336)
\curveto(431.13687987,337.50226341)(431.08187993,337.50726341)(431.02188326,337.50726336)
\curveto(430.96188005,337.50726341)(430.9068801,337.5122634)(430.85688326,337.52226336)
\curveto(430.78688022,337.53226338)(430.72188029,337.54226337)(430.66188326,337.55226336)
\lineto(430.48188326,337.58226336)
\curveto(430.37188064,337.6122633)(430.26688074,337.64726327)(430.16688326,337.68726336)
\curveto(430.06688094,337.72726319)(429.97188104,337.77226314)(429.88188326,337.82226336)
\lineto(429.79188326,337.88226336)
\curveto(429.76188125,337.912263)(429.72688128,337.94226297)(429.68688326,337.97226336)
\curveto(429.66688134,337.99226292)(429.64188137,338.0122629)(429.61188326,338.03226336)
\lineto(429.53688326,338.10726336)
\curveto(429.39688161,338.29726262)(429.29188172,338.50726241)(429.22188326,338.73726336)
\curveto(429.20188181,338.77726214)(429.19188182,338.8122621)(429.19188326,338.84226336)
\curveto(429.20188181,338.88226203)(429.20188181,338.92726199)(429.19188326,338.97726336)
\curveto(429.18188183,338.99726192)(429.17688183,339.02226189)(429.17688326,339.05226336)
\curveto(429.17688183,339.08226183)(429.17188184,339.10726181)(429.16188326,339.12726336)
\lineto(429.16188326,339.27726336)
\curveto(429.15188186,339.3172616)(429.14688186,339.36226155)(429.14688326,339.41226336)
\curveto(429.15688185,339.46226145)(429.16188185,339.5122614)(429.16188326,339.56226336)
\lineto(429.16188326,340.13226336)
\lineto(429.16188326,342.36726336)
\lineto(429.16188326,343.16226336)
\lineto(429.16188326,343.37226336)
\curveto(429.17188184,343.44225747)(429.16688184,343.50725741)(429.14688326,343.56726336)
\curveto(429.1068819,343.70725721)(429.03688197,343.79725712)(428.93688326,343.83726336)
\curveto(428.82688218,343.88725703)(428.68688232,343.90225701)(428.51688326,343.88226336)
\curveto(428.34688266,343.86225705)(428.20188281,343.87725704)(428.08188326,343.92726336)
\curveto(428.00188301,343.95725696)(427.95188306,344.00225691)(427.93188326,344.06226336)
\curveto(427.9118831,344.12225679)(427.89188312,344.19725672)(427.87188326,344.28726336)
\lineto(427.87188326,344.60226336)
\curveto(427.87188314,344.78225613)(427.88188313,344.92725599)(427.90188326,345.03726336)
\curveto(427.92188309,345.14725577)(428.006883,345.22225569)(428.15688326,345.26226336)
\curveto(428.19688281,345.28225563)(428.23688277,345.28725563)(428.27688326,345.27726336)
\lineto(428.41188326,345.27726336)
\curveto(428.56188245,345.27725564)(428.70188231,345.28225563)(428.83188326,345.29226336)
\curveto(428.96188205,345.3122556)(429.05188196,345.37225554)(429.10188326,345.47226336)
\curveto(429.13188188,345.54225537)(429.14688186,345.62225529)(429.14688326,345.71226336)
\curveto(429.15688185,345.80225511)(429.16188185,345.89225502)(429.16188326,345.98226336)
\lineto(429.16188326,346.91226336)
\lineto(429.16188326,347.16726336)
\curveto(429.16188185,347.25725366)(429.17188184,347.33225358)(429.19188326,347.39226336)
\curveto(429.24188177,347.49225342)(429.31688169,347.55725336)(429.41688326,347.58726336)
\curveto(429.43688157,347.59725332)(429.46188155,347.59725332)(429.49188326,347.58726336)
\curveto(429.53188148,347.58725333)(429.56188145,347.59225332)(429.58188326,347.60226336)
}
}
{
\newrgbcolor{curcolor}{0 0 0}
\pscustom[linestyle=none,fillstyle=solid,fillcolor=curcolor]
{
\newpath
\moveto(438.23032076,345.48726336)
\curveto(438.34031544,345.48725543)(438.43531535,345.47725544)(438.51532076,345.45726336)
\curveto(438.60531518,345.43725548)(438.67531511,345.39225552)(438.72532076,345.32226336)
\curveto(438.785315,345.24225567)(438.81531497,345.10225581)(438.81532076,344.90226336)
\lineto(438.81532076,344.39226336)
\lineto(438.81532076,344.01726336)
\curveto(438.82531496,343.87725704)(438.81031497,343.76725715)(438.77032076,343.68726336)
\curveto(438.73031505,343.6172573)(438.67031511,343.57225734)(438.59032076,343.55226336)
\curveto(438.52031526,343.53225738)(438.43531535,343.52225739)(438.33532076,343.52226336)
\curveto(438.24531554,343.52225739)(438.14531564,343.52725739)(438.03532076,343.53726336)
\curveto(437.93531585,343.54725737)(437.84031594,343.54225737)(437.75032076,343.52226336)
\curveto(437.6803161,343.50225741)(437.61031617,343.48725743)(437.54032076,343.47726336)
\curveto(437.47031631,343.47725744)(437.40531638,343.46725745)(437.34532076,343.44726336)
\curveto(437.1853166,343.39725752)(437.02531676,343.32225759)(436.86532076,343.22226336)
\curveto(436.70531708,343.13225778)(436.5803172,343.02725789)(436.49032076,342.90726336)
\curveto(436.44031734,342.82725809)(436.3853174,342.74225817)(436.32532076,342.65226336)
\curveto(436.27531751,342.57225834)(436.22531756,342.48725843)(436.17532076,342.39726336)
\curveto(436.14531764,342.3172586)(436.11531767,342.23225868)(436.08532076,342.14226336)
\lineto(436.02532076,341.90226336)
\curveto(436.00531778,341.83225908)(435.99531779,341.75725916)(435.99532076,341.67726336)
\curveto(435.99531779,341.60725931)(435.9853178,341.53725938)(435.96532076,341.46726336)
\curveto(435.95531783,341.42725949)(435.95031783,341.38725953)(435.95032076,341.34726336)
\curveto(435.96031782,341.3172596)(435.96031782,341.28725963)(435.95032076,341.25726336)
\lineto(435.95032076,341.01726336)
\curveto(435.93031785,340.94725997)(435.92531786,340.86726005)(435.93532076,340.77726336)
\curveto(435.94531784,340.69726022)(435.95031783,340.6172603)(435.95032076,340.53726336)
\lineto(435.95032076,339.57726336)
\lineto(435.95032076,338.30226336)
\curveto(435.95031783,338.17226274)(435.94531784,338.05226286)(435.93532076,337.94226336)
\curveto(435.92531786,337.83226308)(435.89531789,337.74226317)(435.84532076,337.67226336)
\curveto(435.82531796,337.64226327)(435.79031799,337.6172633)(435.74032076,337.59726336)
\curveto(435.70031808,337.58726333)(435.65531813,337.57726334)(435.60532076,337.56726336)
\lineto(435.53032076,337.56726336)
\curveto(435.4803183,337.55726336)(435.42531836,337.55226336)(435.36532076,337.55226336)
\lineto(435.20032076,337.55226336)
\lineto(434.55532076,337.55226336)
\curveto(434.49531929,337.56226335)(434.43031935,337.56726335)(434.36032076,337.56726336)
\lineto(434.16532076,337.56726336)
\curveto(434.11531967,337.58726333)(434.06531972,337.60226331)(434.01532076,337.61226336)
\curveto(433.96531982,337.63226328)(433.93031985,337.66726325)(433.91032076,337.71726336)
\curveto(433.87031991,337.76726315)(433.84531994,337.83726308)(433.83532076,337.92726336)
\lineto(433.83532076,338.22726336)
\lineto(433.83532076,339.24726336)
\lineto(433.83532076,343.47726336)
\lineto(433.83532076,344.58726336)
\lineto(433.83532076,344.87226336)
\curveto(433.83531995,344.97225594)(433.85531993,345.05225586)(433.89532076,345.11226336)
\curveto(433.94531984,345.19225572)(434.02031976,345.24225567)(434.12032076,345.26226336)
\curveto(434.22031956,345.28225563)(434.34031944,345.29225562)(434.48032076,345.29226336)
\lineto(435.24532076,345.29226336)
\curveto(435.36531842,345.29225562)(435.47031831,345.28225563)(435.56032076,345.26226336)
\curveto(435.65031813,345.25225566)(435.72031806,345.20725571)(435.77032076,345.12726336)
\curveto(435.80031798,345.07725584)(435.81531797,345.00725591)(435.81532076,344.91726336)
\lineto(435.84532076,344.64726336)
\curveto(435.85531793,344.56725635)(435.87031791,344.49225642)(435.89032076,344.42226336)
\curveto(435.92031786,344.35225656)(435.97031781,344.3172566)(436.04032076,344.31726336)
\curveto(436.06031772,344.33725658)(436.0803177,344.34725657)(436.10032076,344.34726336)
\curveto(436.12031766,344.34725657)(436.14031764,344.35725656)(436.16032076,344.37726336)
\curveto(436.22031756,344.42725649)(436.27031751,344.48225643)(436.31032076,344.54226336)
\curveto(436.36031742,344.6122563)(436.42031736,344.67225624)(436.49032076,344.72226336)
\curveto(436.53031725,344.75225616)(436.56531722,344.78225613)(436.59532076,344.81226336)
\curveto(436.62531716,344.85225606)(436.66031712,344.88725603)(436.70032076,344.91726336)
\lineto(436.97032076,345.09726336)
\curveto(437.07031671,345.15725576)(437.17031661,345.2122557)(437.27032076,345.26226336)
\curveto(437.37031641,345.30225561)(437.47031631,345.33725558)(437.57032076,345.36726336)
\lineto(437.90032076,345.45726336)
\curveto(437.93031585,345.46725545)(437.9853158,345.46725545)(438.06532076,345.45726336)
\curveto(438.15531563,345.45725546)(438.21031557,345.46725545)(438.23032076,345.48726336)
}
}
{
\newrgbcolor{curcolor}{0 0 0}
\pscustom[linestyle=none,fillstyle=solid,fillcolor=curcolor]
{
\newpath
\moveto(441.73539888,348.14226336)
\curveto(441.80539593,348.06225285)(441.8403959,347.94225297)(441.84039888,347.78226336)
\lineto(441.84039888,347.31726336)
\lineto(441.84039888,346.91226336)
\curveto(441.8403959,346.77225414)(441.80539593,346.67725424)(441.73539888,346.62726336)
\curveto(441.67539606,346.57725434)(441.59539614,346.54725437)(441.49539888,346.53726336)
\curveto(441.40539633,346.52725439)(441.30539643,346.52225439)(441.19539888,346.52226336)
\lineto(440.35539888,346.52226336)
\curveto(440.24539749,346.52225439)(440.14539759,346.52725439)(440.05539888,346.53726336)
\curveto(439.97539776,346.54725437)(439.90539783,346.57725434)(439.84539888,346.62726336)
\curveto(439.80539793,346.65725426)(439.77539796,346.7122542)(439.75539888,346.79226336)
\curveto(439.74539799,346.88225403)(439.735398,346.97725394)(439.72539888,347.07726336)
\lineto(439.72539888,347.40726336)
\curveto(439.735398,347.5172534)(439.740398,347.6122533)(439.74039888,347.69226336)
\lineto(439.74039888,347.90226336)
\curveto(439.75039799,347.97225294)(439.77039797,348.03225288)(439.80039888,348.08226336)
\curveto(439.82039792,348.12225279)(439.84539789,348.15225276)(439.87539888,348.17226336)
\lineto(439.99539888,348.23226336)
\curveto(440.01539772,348.23225268)(440.0403977,348.23225268)(440.07039888,348.23226336)
\curveto(440.10039764,348.24225267)(440.12539761,348.24725267)(440.14539888,348.24726336)
\lineto(441.24039888,348.24726336)
\curveto(441.3403964,348.24725267)(441.4353963,348.24225267)(441.52539888,348.23226336)
\curveto(441.61539612,348.22225269)(441.68539605,348.19225272)(441.73539888,348.14226336)
\moveto(441.84039888,338.37726336)
\curveto(441.8403959,338.17726274)(441.8353959,338.00726291)(441.82539888,337.86726336)
\curveto(441.81539592,337.72726319)(441.72539601,337.63226328)(441.55539888,337.58226336)
\curveto(441.49539624,337.56226335)(441.43039631,337.55226336)(441.36039888,337.55226336)
\curveto(441.29039645,337.56226335)(441.21539652,337.56726335)(441.13539888,337.56726336)
\lineto(440.29539888,337.56726336)
\curveto(440.20539753,337.56726335)(440.11539762,337.57226334)(440.02539888,337.58226336)
\curveto(439.94539779,337.59226332)(439.88539785,337.62226329)(439.84539888,337.67226336)
\curveto(439.78539795,337.74226317)(439.75039799,337.82726309)(439.74039888,337.92726336)
\lineto(439.74039888,338.27226336)
\lineto(439.74039888,344.60226336)
\lineto(439.74039888,344.90226336)
\curveto(439.740398,345.00225591)(439.76039798,345.08225583)(439.80039888,345.14226336)
\curveto(439.86039788,345.2122557)(439.94539779,345.25725566)(440.05539888,345.27726336)
\curveto(440.07539766,345.28725563)(440.10039764,345.28725563)(440.13039888,345.27726336)
\curveto(440.17039757,345.27725564)(440.20039754,345.28225563)(440.22039888,345.29226336)
\lineto(440.97039888,345.29226336)
\lineto(441.16539888,345.29226336)
\curveto(441.24539649,345.30225561)(441.31039643,345.30225561)(441.36039888,345.29226336)
\lineto(441.48039888,345.29226336)
\curveto(441.5403962,345.27225564)(441.59539614,345.25725566)(441.64539888,345.24726336)
\curveto(441.69539604,345.23725568)(441.735396,345.20725571)(441.76539888,345.15726336)
\curveto(441.80539593,345.10725581)(441.82539591,345.03725588)(441.82539888,344.94726336)
\curveto(441.8353959,344.85725606)(441.8403959,344.76225615)(441.84039888,344.66226336)
\lineto(441.84039888,338.37726336)
}
}
{
\newrgbcolor{curcolor}{0 0 0}
\pscustom[linestyle=none,fillstyle=solid,fillcolor=curcolor]
{
\newpath
\moveto(451.30258638,341.81226336)
\curveto(451.32257778,341.75225916)(451.33257777,341.64725927)(451.33258638,341.49726336)
\curveto(451.33257777,341.35725956)(451.32757778,341.25725966)(451.31758638,341.19726336)
\curveto(451.31757779,341.14725977)(451.31257779,341.10225981)(451.30258638,341.06226336)
\lineto(451.30258638,340.94226336)
\curveto(451.28257782,340.86226005)(451.27257783,340.78226013)(451.27258638,340.70226336)
\curveto(451.27257783,340.63226028)(451.26257784,340.55726036)(451.24258638,340.47726336)
\curveto(451.24257786,340.43726048)(451.23257787,340.36726055)(451.21258638,340.26726336)
\curveto(451.18257792,340.14726077)(451.15257795,340.02226089)(451.12258638,339.89226336)
\curveto(451.102578,339.77226114)(451.06757804,339.65726126)(451.01758638,339.54726336)
\curveto(450.83757827,339.09726182)(450.61257849,338.70726221)(450.34258638,338.37726336)
\curveto(450.07257903,338.04726287)(449.71757939,337.78726313)(449.27758638,337.59726336)
\curveto(449.18757992,337.55726336)(449.09258001,337.52726339)(448.99258638,337.50726336)
\curveto(448.9025802,337.47726344)(448.8025803,337.44726347)(448.69258638,337.41726336)
\curveto(448.63258047,337.39726352)(448.56758054,337.38726353)(448.49758638,337.38726336)
\curveto(448.43758067,337.38726353)(448.37758073,337.38226353)(448.31758638,337.37226336)
\lineto(448.18258638,337.37226336)
\curveto(448.12258098,337.35226356)(448.04258106,337.34726357)(447.94258638,337.35726336)
\curveto(447.84258126,337.35726356)(447.76258134,337.36726355)(447.70258638,337.38726336)
\lineto(447.61258638,337.38726336)
\curveto(447.56258154,337.39726352)(447.5075816,337.40726351)(447.44758638,337.41726336)
\curveto(447.38758172,337.4172635)(447.32758178,337.42226349)(447.26758638,337.43226336)
\curveto(447.07758203,337.48226343)(446.9025822,337.53226338)(446.74258638,337.58226336)
\curveto(446.58258252,337.63226328)(446.43258267,337.70226321)(446.29258638,337.79226336)
\lineto(446.11258638,337.91226336)
\curveto(446.06258304,337.95226296)(446.01258309,337.99726292)(445.96258638,338.04726336)
\lineto(445.87258638,338.10726336)
\curveto(445.84258326,338.12726279)(445.81258329,338.14226277)(445.78258638,338.15226336)
\curveto(445.69258341,338.18226273)(445.63758347,338.16226275)(445.61758638,338.09226336)
\curveto(445.56758354,338.02226289)(445.53258357,337.93726298)(445.51258638,337.83726336)
\curveto(445.5025836,337.74726317)(445.46758364,337.67726324)(445.40758638,337.62726336)
\curveto(445.34758376,337.58726333)(445.27758383,337.56226335)(445.19758638,337.55226336)
\lineto(444.92758638,337.55226336)
\lineto(444.20758638,337.55226336)
\lineto(443.98258638,337.55226336)
\curveto(443.91258519,337.54226337)(443.84758526,337.54726337)(443.78758638,337.56726336)
\curveto(443.64758546,337.6172633)(443.56758554,337.70726321)(443.54758638,337.83726336)
\curveto(443.53758557,337.97726294)(443.53258557,338.13226278)(443.53258638,338.30226336)
\lineto(443.53258638,347.45226336)
\lineto(443.53258638,347.79726336)
\curveto(443.53258557,347.917253)(443.55758555,348.0122529)(443.60758638,348.08226336)
\curveto(443.64758546,348.15225276)(443.71758539,348.19725272)(443.81758638,348.21726336)
\curveto(443.83758527,348.22725269)(443.85758525,348.22725269)(443.87758638,348.21726336)
\curveto(443.9075852,348.2172527)(443.93258517,348.22225269)(443.95258638,348.23226336)
\lineto(444.89758638,348.23226336)
\curveto(445.07758403,348.23225268)(445.23258387,348.22225269)(445.36258638,348.20226336)
\curveto(445.49258361,348.19225272)(445.57758353,348.1172528)(445.61758638,347.97726336)
\curveto(445.64758346,347.87725304)(445.65758345,347.74225317)(445.64758638,347.57226336)
\curveto(445.63758347,347.4122535)(445.63258347,347.27225364)(445.63258638,347.15226336)
\lineto(445.63258638,345.51726336)
\lineto(445.63258638,345.18726336)
\curveto(445.63258347,345.07725584)(445.64258346,344.98225593)(445.66258638,344.90226336)
\curveto(445.67258343,344.85225606)(445.68258342,344.80725611)(445.69258638,344.76726336)
\curveto(445.7025834,344.73725618)(445.72758338,344.7172562)(445.76758638,344.70726336)
\curveto(445.78758332,344.68725623)(445.81258329,344.67725624)(445.84258638,344.67726336)
\curveto(445.88258322,344.67725624)(445.91258319,344.68225623)(445.93258638,344.69226336)
\curveto(446.0025831,344.73225618)(446.06758304,344.77225614)(446.12758638,344.81226336)
\curveto(446.18758292,344.86225605)(446.25258285,344.912256)(446.32258638,344.96226336)
\curveto(446.45258265,345.05225586)(446.58758252,345.12725579)(446.72758638,345.18726336)
\curveto(446.86758224,345.25725566)(447.02258208,345.3172556)(447.19258638,345.36726336)
\curveto(447.27258183,345.39725552)(447.35258175,345.4122555)(447.43258638,345.41226336)
\curveto(447.51258159,345.42225549)(447.59258151,345.43725548)(447.67258638,345.45726336)
\curveto(447.74258136,345.47725544)(447.81758129,345.48725543)(447.89758638,345.48726336)
\lineto(448.13758638,345.48726336)
\lineto(448.28758638,345.48726336)
\curveto(448.31758079,345.47725544)(448.35258075,345.47225544)(448.39258638,345.47226336)
\curveto(448.43258067,345.48225543)(448.47258063,345.48225543)(448.51258638,345.47226336)
\curveto(448.62258048,345.44225547)(448.72258038,345.4172555)(448.81258638,345.39726336)
\curveto(448.91258019,345.38725553)(449.0075801,345.36225555)(449.09758638,345.32226336)
\curveto(449.55757955,345.13225578)(449.93257917,344.88725603)(450.22258638,344.58726336)
\curveto(450.51257859,344.28725663)(450.75757835,343.912257)(450.95758638,343.46226336)
\curveto(451.0075781,343.34225757)(451.04757806,343.2172577)(451.07758638,343.08726336)
\curveto(451.11757799,342.95725796)(451.15757795,342.82225809)(451.19758638,342.68226336)
\curveto(451.21757789,342.6122583)(451.22757788,342.54225837)(451.22758638,342.47226336)
\curveto(451.23757787,342.4122585)(451.25257785,342.34225857)(451.27258638,342.26226336)
\curveto(451.29257781,342.2122587)(451.29757781,342.15725876)(451.28758638,342.09726336)
\curveto(451.28757782,342.03725888)(451.29257781,341.97725894)(451.30258638,341.91726336)
\lineto(451.30258638,341.81226336)
\moveto(449.08258638,340.40226336)
\curveto(449.11257999,340.50226041)(449.13757997,340.62726029)(449.15758638,340.77726336)
\curveto(449.18757992,340.92725999)(449.2025799,341.07725984)(449.20258638,341.22726336)
\curveto(449.21257989,341.38725953)(449.21257989,341.54225937)(449.20258638,341.69226336)
\curveto(449.2025799,341.85225906)(449.18757992,341.98725893)(449.15758638,342.09726336)
\curveto(449.12757998,342.19725872)(449.10758,342.29225862)(449.09758638,342.38226336)
\curveto(449.08758002,342.47225844)(449.06258004,342.55725836)(449.02258638,342.63726336)
\curveto(448.88258022,342.98725793)(448.68258042,343.28225763)(448.42258638,343.52226336)
\curveto(448.17258093,343.77225714)(447.8025813,343.89725702)(447.31258638,343.89726336)
\curveto(447.27258183,343.89725702)(447.23758187,343.89225702)(447.20758638,343.88226336)
\lineto(447.10258638,343.88226336)
\curveto(447.03258207,343.86225705)(446.96758214,343.84225707)(446.90758638,343.82226336)
\curveto(446.84758226,343.8122571)(446.78758232,343.79725712)(446.72758638,343.77726336)
\curveto(446.43758267,343.64725727)(446.21758289,343.46225745)(446.06758638,343.22226336)
\curveto(445.91758319,342.99225792)(445.79258331,342.72725819)(445.69258638,342.42726336)
\curveto(445.66258344,342.34725857)(445.64258346,342.26225865)(445.63258638,342.17226336)
\curveto(445.63258347,342.09225882)(445.62258348,342.0122589)(445.60258638,341.93226336)
\curveto(445.59258351,341.90225901)(445.58758352,341.85225906)(445.58758638,341.78226336)
\curveto(445.57758353,341.74225917)(445.57258353,341.70225921)(445.57258638,341.66226336)
\curveto(445.58258352,341.62225929)(445.58258352,341.58225933)(445.57258638,341.54226336)
\curveto(445.55258355,341.46225945)(445.54758356,341.35225956)(445.55758638,341.21226336)
\curveto(445.56758354,341.07225984)(445.58258352,340.97225994)(445.60258638,340.91226336)
\curveto(445.62258348,340.82226009)(445.63258347,340.73726018)(445.63258638,340.65726336)
\curveto(445.64258346,340.57726034)(445.66258344,340.49726042)(445.69258638,340.41726336)
\curveto(445.78258332,340.13726078)(445.88758322,339.89226102)(446.00758638,339.68226336)
\curveto(446.13758297,339.48226143)(446.31758279,339.3122616)(446.54758638,339.17226336)
\curveto(446.7075824,339.07226184)(446.87258223,339.00226191)(447.04258638,338.96226336)
\curveto(447.06258204,338.96226195)(447.08258202,338.95726196)(447.10258638,338.94726336)
\lineto(447.19258638,338.94726336)
\curveto(447.22258188,338.93726198)(447.27258183,338.92726199)(447.34258638,338.91726336)
\curveto(447.41258169,338.917262)(447.47258163,338.92226199)(447.52258638,338.93226336)
\curveto(447.62258148,338.95226196)(447.71258139,338.96726195)(447.79258638,338.97726336)
\curveto(447.88258122,338.99726192)(447.96758114,339.02226189)(448.04758638,339.05226336)
\curveto(448.32758078,339.18226173)(448.54258056,339.36226155)(448.69258638,339.59226336)
\curveto(448.85258025,339.82226109)(448.98258012,340.09226082)(449.08258638,340.40226336)
}
}
{
\newrgbcolor{curcolor}{0 0 0}
\pscustom[linestyle=none,fillstyle=solid,fillcolor=curcolor]
{
\newpath
\moveto(453.09250826,345.27726336)
\lineto(454.21750826,345.27726336)
\curveto(454.32750582,345.27725564)(454.42750572,345.27225564)(454.51750826,345.26226336)
\curveto(454.60750554,345.25225566)(454.67250548,345.2172557)(454.71250826,345.15726336)
\curveto(454.76250539,345.09725582)(454.79250536,345.0122559)(454.80250826,344.90226336)
\curveto(454.81250534,344.80225611)(454.81750533,344.69725622)(454.81750826,344.58726336)
\lineto(454.81750826,343.53726336)
\lineto(454.81750826,341.30226336)
\curveto(454.81750533,340.94225997)(454.83250532,340.60226031)(454.86250826,340.28226336)
\curveto(454.89250526,339.96226095)(454.98250517,339.69726122)(455.13250826,339.48726336)
\curveto(455.27250488,339.27726164)(455.49750465,339.12726179)(455.80750826,339.03726336)
\curveto(455.85750429,339.02726189)(455.89750425,339.02226189)(455.92750826,339.02226336)
\curveto(455.96750418,339.02226189)(456.01250414,339.0172619)(456.06250826,339.00726336)
\curveto(456.11250404,338.99726192)(456.16750398,338.99226192)(456.22750826,338.99226336)
\curveto(456.28750386,338.99226192)(456.33250382,338.99726192)(456.36250826,339.00726336)
\curveto(456.41250374,339.02726189)(456.4525037,339.03226188)(456.48250826,339.02226336)
\curveto(456.52250363,339.0122619)(456.56250359,339.0172619)(456.60250826,339.03726336)
\curveto(456.81250334,339.08726183)(456.97750317,339.15226176)(457.09750826,339.23226336)
\curveto(457.27750287,339.34226157)(457.41750273,339.48226143)(457.51750826,339.65226336)
\curveto(457.62750252,339.83226108)(457.70250245,340.02726089)(457.74250826,340.23726336)
\curveto(457.79250236,340.45726046)(457.82250233,340.69726022)(457.83250826,340.95726336)
\curveto(457.84250231,341.22725969)(457.8475023,341.50725941)(457.84750826,341.79726336)
\lineto(457.84750826,343.61226336)
\lineto(457.84750826,344.58726336)
\lineto(457.84750826,344.85726336)
\curveto(457.8475023,344.95725596)(457.86750228,345.03725588)(457.90750826,345.09726336)
\curveto(457.95750219,345.18725573)(458.03250212,345.23725568)(458.13250826,345.24726336)
\curveto(458.23250192,345.26725565)(458.3525018,345.27725564)(458.49250826,345.27726336)
\lineto(459.28750826,345.27726336)
\lineto(459.57250826,345.27726336)
\curveto(459.66250049,345.27725564)(459.73750041,345.25725566)(459.79750826,345.21726336)
\curveto(459.87750027,345.16725575)(459.92250023,345.09225582)(459.93250826,344.99226336)
\curveto(459.94250021,344.89225602)(459.9475002,344.77725614)(459.94750826,344.64726336)
\lineto(459.94750826,343.50726336)
\lineto(459.94750826,339.29226336)
\lineto(459.94750826,338.22726336)
\lineto(459.94750826,337.92726336)
\curveto(459.9475002,337.82726309)(459.92750022,337.75226316)(459.88750826,337.70226336)
\curveto(459.83750031,337.62226329)(459.76250039,337.57726334)(459.66250826,337.56726336)
\curveto(459.56250059,337.55726336)(459.45750069,337.55226336)(459.34750826,337.55226336)
\lineto(458.53750826,337.55226336)
\curveto(458.42750172,337.55226336)(458.32750182,337.55726336)(458.23750826,337.56726336)
\curveto(458.15750199,337.57726334)(458.09250206,337.6172633)(458.04250826,337.68726336)
\curveto(458.02250213,337.7172632)(458.00250215,337.76226315)(457.98250826,337.82226336)
\curveto(457.97250218,337.88226303)(457.95750219,337.94226297)(457.93750826,338.00226336)
\curveto(457.92750222,338.06226285)(457.91250224,338.1172628)(457.89250826,338.16726336)
\curveto(457.87250228,338.2172627)(457.84250231,338.24726267)(457.80250826,338.25726336)
\curveto(457.78250237,338.27726264)(457.75750239,338.28226263)(457.72750826,338.27226336)
\curveto(457.69750245,338.26226265)(457.67250248,338.25226266)(457.65250826,338.24226336)
\curveto(457.58250257,338.20226271)(457.52250263,338.15726276)(457.47250826,338.10726336)
\curveto(457.42250273,338.05726286)(457.36750278,338.0122629)(457.30750826,337.97226336)
\curveto(457.26750288,337.94226297)(457.22750292,337.90726301)(457.18750826,337.86726336)
\curveto(457.15750299,337.83726308)(457.11750303,337.80726311)(457.06750826,337.77726336)
\curveto(456.83750331,337.63726328)(456.56750358,337.52726339)(456.25750826,337.44726336)
\curveto(456.18750396,337.42726349)(456.11750403,337.4172635)(456.04750826,337.41726336)
\curveto(455.97750417,337.40726351)(455.90250425,337.39226352)(455.82250826,337.37226336)
\curveto(455.78250437,337.36226355)(455.73750441,337.36226355)(455.68750826,337.37226336)
\curveto(455.6475045,337.37226354)(455.60750454,337.36726355)(455.56750826,337.35726336)
\curveto(455.53750461,337.34726357)(455.47250468,337.34726357)(455.37250826,337.35726336)
\curveto(455.28250487,337.35726356)(455.22250493,337.36226355)(455.19250826,337.37226336)
\curveto(455.14250501,337.37226354)(455.09250506,337.37726354)(455.04250826,337.38726336)
\lineto(454.89250826,337.38726336)
\curveto(454.77250538,337.4172635)(454.65750549,337.44226347)(454.54750826,337.46226336)
\curveto(454.43750571,337.48226343)(454.32750582,337.5122634)(454.21750826,337.55226336)
\curveto(454.16750598,337.57226334)(454.12250603,337.58726333)(454.08250826,337.59726336)
\curveto(454.0525061,337.6172633)(454.01250614,337.63726328)(453.96250826,337.65726336)
\curveto(453.61250654,337.84726307)(453.33250682,338.1122628)(453.12250826,338.45226336)
\curveto(452.99250716,338.66226225)(452.89750725,338.912262)(452.83750826,339.20226336)
\curveto(452.77750737,339.50226141)(452.73750741,339.8172611)(452.71750826,340.14726336)
\curveto(452.70750744,340.48726043)(452.70250745,340.83226008)(452.70250826,341.18226336)
\curveto(452.71250744,341.54225937)(452.71750743,341.89725902)(452.71750826,342.24726336)
\lineto(452.71750826,344.28726336)
\curveto(452.71750743,344.4172565)(452.71250744,344.56725635)(452.70250826,344.73726336)
\curveto(452.70250745,344.917256)(452.72750742,345.04725587)(452.77750826,345.12726336)
\curveto(452.80750734,345.17725574)(452.86750728,345.22225569)(452.95750826,345.26226336)
\curveto(453.01750713,345.26225565)(453.06250709,345.26725565)(453.09250826,345.27726336)
}
}
{
\newrgbcolor{curcolor}{0 0 0}
\pscustom[linestyle=none,fillstyle=solid,fillcolor=curcolor]
{
\newpath
\moveto(465.14875826,345.50226336)
\curveto(465.9587531,345.52225539)(466.63375242,345.40225551)(467.17375826,345.14226336)
\curveto(467.72375133,344.88225603)(468.1587509,344.5122564)(468.47875826,344.03226336)
\curveto(468.63875042,343.79225712)(468.7587503,343.5172574)(468.83875826,343.20726336)
\curveto(468.8587502,343.15725776)(468.87375018,343.09225782)(468.88375826,343.01226336)
\curveto(468.90375015,342.93225798)(468.90375015,342.86225805)(468.88375826,342.80226336)
\curveto(468.84375021,342.69225822)(468.77375028,342.62725829)(468.67375826,342.60726336)
\curveto(468.57375048,342.59725832)(468.4537506,342.59225832)(468.31375826,342.59226336)
\lineto(467.53375826,342.59226336)
\lineto(467.24875826,342.59226336)
\curveto(467.1587519,342.59225832)(467.08375197,342.6122583)(467.02375826,342.65226336)
\curveto(466.94375211,342.69225822)(466.88875217,342.75225816)(466.85875826,342.83226336)
\curveto(466.82875223,342.92225799)(466.78875227,343.0122579)(466.73875826,343.10226336)
\curveto(466.67875238,343.2122577)(466.61375244,343.3122576)(466.54375826,343.40226336)
\curveto(466.47375258,343.49225742)(466.39375266,343.57225734)(466.30375826,343.64226336)
\curveto(466.16375289,343.73225718)(466.00875305,343.80225711)(465.83875826,343.85226336)
\curveto(465.77875328,343.87225704)(465.71875334,343.88225703)(465.65875826,343.88226336)
\curveto(465.59875346,343.88225703)(465.54375351,343.89225702)(465.49375826,343.91226336)
\lineto(465.34375826,343.91226336)
\curveto(465.14375391,343.912257)(464.98375407,343.89225702)(464.86375826,343.85226336)
\curveto(464.57375448,343.76225715)(464.33875472,343.62225729)(464.15875826,343.43226336)
\curveto(463.97875508,343.25225766)(463.83375522,343.03225788)(463.72375826,342.77226336)
\curveto(463.67375538,342.66225825)(463.63375542,342.54225837)(463.60375826,342.41226336)
\curveto(463.58375547,342.29225862)(463.5587555,342.16225875)(463.52875826,342.02226336)
\curveto(463.51875554,341.98225893)(463.51375554,341.94225897)(463.51375826,341.90226336)
\curveto(463.51375554,341.86225905)(463.50875555,341.82225909)(463.49875826,341.78226336)
\curveto(463.47875558,341.68225923)(463.46875559,341.54225937)(463.46875826,341.36226336)
\curveto(463.47875558,341.18225973)(463.49375556,341.04225987)(463.51375826,340.94226336)
\curveto(463.51375554,340.86226005)(463.51875554,340.80726011)(463.52875826,340.77726336)
\curveto(463.54875551,340.70726021)(463.5587555,340.63726028)(463.55875826,340.56726336)
\curveto(463.56875549,340.49726042)(463.58375547,340.42726049)(463.60375826,340.35726336)
\curveto(463.68375537,340.12726079)(463.77875528,339.917261)(463.88875826,339.72726336)
\curveto(463.99875506,339.53726138)(464.13875492,339.37726154)(464.30875826,339.24726336)
\curveto(464.34875471,339.2172617)(464.40875465,339.18226173)(464.48875826,339.14226336)
\curveto(464.59875446,339.07226184)(464.70875435,339.02726189)(464.81875826,339.00726336)
\curveto(464.93875412,338.98726193)(465.08375397,338.96726195)(465.25375826,338.94726336)
\lineto(465.34375826,338.94726336)
\curveto(465.38375367,338.94726197)(465.41375364,338.95226196)(465.43375826,338.96226336)
\lineto(465.56875826,338.96226336)
\curveto(465.63875342,338.98226193)(465.70375335,338.99726192)(465.76375826,339.00726336)
\curveto(465.83375322,339.02726189)(465.89875316,339.04726187)(465.95875826,339.06726336)
\curveto(466.2587528,339.19726172)(466.48875257,339.38726153)(466.64875826,339.63726336)
\curveto(466.68875237,339.68726123)(466.72375233,339.74226117)(466.75375826,339.80226336)
\curveto(466.78375227,339.87226104)(466.80875225,339.93226098)(466.82875826,339.98226336)
\curveto(466.86875219,340.09226082)(466.90375215,340.18726073)(466.93375826,340.26726336)
\curveto(466.96375209,340.35726056)(467.03375202,340.42726049)(467.14375826,340.47726336)
\curveto(467.23375182,340.5172604)(467.37875168,340.53226038)(467.57875826,340.52226336)
\lineto(468.07375826,340.52226336)
\lineto(468.28375826,340.52226336)
\curveto(468.36375069,340.53226038)(468.42875063,340.52726039)(468.47875826,340.50726336)
\lineto(468.59875826,340.50726336)
\lineto(468.71875826,340.47726336)
\curveto(468.7587503,340.47726044)(468.78875027,340.46726045)(468.80875826,340.44726336)
\curveto(468.8587502,340.40726051)(468.88875017,340.34726057)(468.89875826,340.26726336)
\curveto(468.91875014,340.19726072)(468.91875014,340.12226079)(468.89875826,340.04226336)
\curveto(468.80875025,339.7122612)(468.69875036,339.4172615)(468.56875826,339.15726336)
\curveto(468.1587509,338.38726253)(467.50375155,337.85226306)(466.60375826,337.55226336)
\curveto(466.50375255,337.52226339)(466.39875266,337.50226341)(466.28875826,337.49226336)
\curveto(466.17875288,337.47226344)(466.06875299,337.44726347)(465.95875826,337.41726336)
\curveto(465.89875316,337.40726351)(465.83875322,337.40226351)(465.77875826,337.40226336)
\curveto(465.71875334,337.40226351)(465.6587534,337.39726352)(465.59875826,337.38726336)
\lineto(465.43375826,337.38726336)
\curveto(465.38375367,337.36726355)(465.30875375,337.36226355)(465.20875826,337.37226336)
\curveto(465.10875395,337.37226354)(465.03375402,337.37726354)(464.98375826,337.38726336)
\curveto(464.90375415,337.40726351)(464.82875423,337.4172635)(464.75875826,337.41726336)
\curveto(464.69875436,337.40726351)(464.63375442,337.4122635)(464.56375826,337.43226336)
\lineto(464.41375826,337.46226336)
\curveto(464.36375469,337.46226345)(464.31375474,337.46726345)(464.26375826,337.47726336)
\curveto(464.1537549,337.50726341)(464.04875501,337.53726338)(463.94875826,337.56726336)
\curveto(463.84875521,337.59726332)(463.7537553,337.63226328)(463.66375826,337.67226336)
\curveto(463.19375586,337.87226304)(462.79875626,338.12726279)(462.47875826,338.43726336)
\curveto(462.1587569,338.75726216)(461.89875716,339.15226176)(461.69875826,339.62226336)
\curveto(461.64875741,339.7122612)(461.60875745,339.80726111)(461.57875826,339.90726336)
\lineto(461.48875826,340.23726336)
\curveto(461.47875758,340.27726064)(461.47375758,340.3122606)(461.47375826,340.34226336)
\curveto(461.47375758,340.38226053)(461.46375759,340.42726049)(461.44375826,340.47726336)
\curveto(461.42375763,340.54726037)(461.41375764,340.6172603)(461.41375826,340.68726336)
\curveto(461.41375764,340.76726015)(461.40375765,340.84226007)(461.38375826,340.91226336)
\lineto(461.38375826,341.16726336)
\curveto(461.36375769,341.2172597)(461.3537577,341.27225964)(461.35375826,341.33226336)
\curveto(461.3537577,341.40225951)(461.36375769,341.46225945)(461.38375826,341.51226336)
\curveto(461.39375766,341.56225935)(461.39375766,341.60725931)(461.38375826,341.64726336)
\curveto(461.37375768,341.68725923)(461.37375768,341.72725919)(461.38375826,341.76726336)
\curveto(461.40375765,341.83725908)(461.40875765,341.90225901)(461.39875826,341.96226336)
\curveto(461.39875766,342.02225889)(461.40875765,342.08225883)(461.42875826,342.14226336)
\curveto(461.47875758,342.32225859)(461.51875754,342.49225842)(461.54875826,342.65226336)
\curveto(461.57875748,342.82225809)(461.62375743,342.98725793)(461.68375826,343.14726336)
\curveto(461.90375715,343.65725726)(462.17875688,344.08225683)(462.50875826,344.42226336)
\curveto(462.84875621,344.76225615)(463.27875578,345.03725588)(463.79875826,345.24726336)
\curveto(463.93875512,345.30725561)(464.08375497,345.34725557)(464.23375826,345.36726336)
\curveto(464.38375467,345.39725552)(464.53875452,345.43225548)(464.69875826,345.47226336)
\curveto(464.77875428,345.48225543)(464.8537542,345.48725543)(464.92375826,345.48726336)
\curveto(464.99375406,345.48725543)(465.06875399,345.49225542)(465.14875826,345.50226336)
}
}
{
\newrgbcolor{curcolor}{0 0 0}
\pscustom[linestyle=none,fillstyle=solid,fillcolor=curcolor]
{
\newpath
\moveto(472.29203951,348.14226336)
\curveto(472.36203656,348.06225285)(472.39703652,347.94225297)(472.39703951,347.78226336)
\lineto(472.39703951,347.31726336)
\lineto(472.39703951,346.91226336)
\curveto(472.39703652,346.77225414)(472.36203656,346.67725424)(472.29203951,346.62726336)
\curveto(472.23203669,346.57725434)(472.15203677,346.54725437)(472.05203951,346.53726336)
\curveto(471.96203696,346.52725439)(471.86203706,346.52225439)(471.75203951,346.52226336)
\lineto(470.91203951,346.52226336)
\curveto(470.80203812,346.52225439)(470.70203822,346.52725439)(470.61203951,346.53726336)
\curveto(470.53203839,346.54725437)(470.46203846,346.57725434)(470.40203951,346.62726336)
\curveto(470.36203856,346.65725426)(470.33203859,346.7122542)(470.31203951,346.79226336)
\curveto(470.30203862,346.88225403)(470.29203863,346.97725394)(470.28203951,347.07726336)
\lineto(470.28203951,347.40726336)
\curveto(470.29203863,347.5172534)(470.29703862,347.6122533)(470.29703951,347.69226336)
\lineto(470.29703951,347.90226336)
\curveto(470.30703861,347.97225294)(470.32703859,348.03225288)(470.35703951,348.08226336)
\curveto(470.37703854,348.12225279)(470.40203852,348.15225276)(470.43203951,348.17226336)
\lineto(470.55203951,348.23226336)
\curveto(470.57203835,348.23225268)(470.59703832,348.23225268)(470.62703951,348.23226336)
\curveto(470.65703826,348.24225267)(470.68203824,348.24725267)(470.70203951,348.24726336)
\lineto(471.79703951,348.24726336)
\curveto(471.89703702,348.24725267)(471.99203693,348.24225267)(472.08203951,348.23226336)
\curveto(472.17203675,348.22225269)(472.24203668,348.19225272)(472.29203951,348.14226336)
\moveto(472.39703951,338.37726336)
\curveto(472.39703652,338.17726274)(472.39203653,338.00726291)(472.38203951,337.86726336)
\curveto(472.37203655,337.72726319)(472.28203664,337.63226328)(472.11203951,337.58226336)
\curveto(472.05203687,337.56226335)(471.98703693,337.55226336)(471.91703951,337.55226336)
\curveto(471.84703707,337.56226335)(471.77203715,337.56726335)(471.69203951,337.56726336)
\lineto(470.85203951,337.56726336)
\curveto(470.76203816,337.56726335)(470.67203825,337.57226334)(470.58203951,337.58226336)
\curveto(470.50203842,337.59226332)(470.44203848,337.62226329)(470.40203951,337.67226336)
\curveto(470.34203858,337.74226317)(470.30703861,337.82726309)(470.29703951,337.92726336)
\lineto(470.29703951,338.27226336)
\lineto(470.29703951,344.60226336)
\lineto(470.29703951,344.90226336)
\curveto(470.29703862,345.00225591)(470.3170386,345.08225583)(470.35703951,345.14226336)
\curveto(470.4170385,345.2122557)(470.50203842,345.25725566)(470.61203951,345.27726336)
\curveto(470.63203829,345.28725563)(470.65703826,345.28725563)(470.68703951,345.27726336)
\curveto(470.72703819,345.27725564)(470.75703816,345.28225563)(470.77703951,345.29226336)
\lineto(471.52703951,345.29226336)
\lineto(471.72203951,345.29226336)
\curveto(471.80203712,345.30225561)(471.86703705,345.30225561)(471.91703951,345.29226336)
\lineto(472.03703951,345.29226336)
\curveto(472.09703682,345.27225564)(472.15203677,345.25725566)(472.20203951,345.24726336)
\curveto(472.25203667,345.23725568)(472.29203663,345.20725571)(472.32203951,345.15726336)
\curveto(472.36203656,345.10725581)(472.38203654,345.03725588)(472.38203951,344.94726336)
\curveto(472.39203653,344.85725606)(472.39703652,344.76225615)(472.39703951,344.66226336)
\lineto(472.39703951,338.37726336)
}
}
{
\newrgbcolor{curcolor}{0 0 0}
\pscustom[linestyle=none,fillstyle=solid,fillcolor=curcolor]
{
\newpath
\moveto(481.82922701,341.73726336)
\curveto(481.80921848,341.78725913)(481.80421848,341.84225907)(481.81422701,341.90226336)
\curveto(481.82421846,341.96225895)(481.81921847,342.0172589)(481.79922701,342.06726336)
\curveto(481.7892185,342.10725881)(481.7842185,342.14725877)(481.78422701,342.18726336)
\curveto(481.7842185,342.22725869)(481.77921851,342.26725865)(481.76922701,342.30726336)
\lineto(481.70922701,342.57726336)
\curveto(481.6892186,342.66725825)(481.66421862,342.75225816)(481.63422701,342.83226336)
\curveto(481.5842187,342.97225794)(481.53921875,343.10225781)(481.49922701,343.22226336)
\curveto(481.45921883,343.35225756)(481.40421888,343.47225744)(481.33422701,343.58226336)
\curveto(481.26421902,343.69225722)(481.19421909,343.79725712)(481.12422701,343.89726336)
\curveto(481.06421922,343.99725692)(480.99421929,344.09725682)(480.91422701,344.19726336)
\curveto(480.83421945,344.30725661)(480.73421955,344.40725651)(480.61422701,344.49726336)
\curveto(480.50421978,344.59725632)(480.39421989,344.68725623)(480.28422701,344.76726336)
\curveto(479.95422033,344.99725592)(479.57422071,345.17725574)(479.14422701,345.30726336)
\curveto(478.72422156,345.43725548)(478.22422206,345.49725542)(477.64422701,345.48726336)
\curveto(477.57422271,345.47725544)(477.50422278,345.47225544)(477.43422701,345.47226336)
\curveto(477.36422292,345.47225544)(477.289223,345.46725545)(477.20922701,345.45726336)
\curveto(477.05922323,345.4172555)(476.91422337,345.38725553)(476.77422701,345.36726336)
\curveto(476.63422365,345.34725557)(476.49922379,345.3122556)(476.36922701,345.26226336)
\curveto(476.25922403,345.2122557)(476.14922414,345.16725575)(476.03922701,345.12726336)
\curveto(475.92922436,345.08725583)(475.82422446,345.04225587)(475.72422701,344.99226336)
\curveto(475.36422492,344.76225615)(475.05922523,344.50725641)(474.80922701,344.22726336)
\curveto(474.55922573,343.95725696)(474.34422594,343.6172573)(474.16422701,343.20726336)
\curveto(474.11422617,343.08725783)(474.07422621,342.96225795)(474.04422701,342.83226336)
\curveto(474.01422627,342.7122582)(473.97922631,342.58725833)(473.93922701,342.45726336)
\curveto(473.91922637,342.40725851)(473.90922638,342.35725856)(473.90922701,342.30726336)
\curveto(473.90922638,342.26725865)(473.90422638,342.22225869)(473.89422701,342.17226336)
\curveto(473.87422641,342.12225879)(473.86422642,342.06725885)(473.86422701,342.00726336)
\curveto(473.87422641,341.95725896)(473.87422641,341.90725901)(473.86422701,341.85726336)
\lineto(473.86422701,341.75226336)
\curveto(473.84422644,341.69225922)(473.82922646,341.60725931)(473.81922701,341.49726336)
\curveto(473.81922647,341.38725953)(473.82922646,341.30225961)(473.84922701,341.24226336)
\lineto(473.84922701,341.10726336)
\curveto(473.84922644,341.06725985)(473.85422643,341.02225989)(473.86422701,340.97226336)
\curveto(473.8842264,340.89226002)(473.89422639,340.80726011)(473.89422701,340.71726336)
\curveto(473.89422639,340.63726028)(473.90422638,340.55726036)(473.92422701,340.47726336)
\curveto(473.94422634,340.42726049)(473.95422633,340.38226053)(473.95422701,340.34226336)
\curveto(473.95422633,340.30226061)(473.96422632,340.25726066)(473.98422701,340.20726336)
\curveto(474.01422627,340.09726082)(474.03922625,339.99226092)(474.05922701,339.89226336)
\curveto(474.0892262,339.79226112)(474.12922616,339.69726122)(474.17922701,339.60726336)
\curveto(474.34922594,339.2172617)(474.55922573,338.88226203)(474.80922701,338.60226336)
\curveto(475.05922523,338.32226259)(475.35922493,338.07726284)(475.70922701,337.86726336)
\curveto(475.81922447,337.80726311)(475.92422436,337.75726316)(476.02422701,337.71726336)
\curveto(476.13422415,337.67726324)(476.24922404,337.63726328)(476.36922701,337.59726336)
\curveto(476.45922383,337.55726336)(476.55422373,337.52726339)(476.65422701,337.50726336)
\curveto(476.75422353,337.48726343)(476.85422343,337.46226345)(476.95422701,337.43226336)
\curveto(477.00422328,337.42226349)(477.04422324,337.4172635)(477.07422701,337.41726336)
\curveto(477.11422317,337.4172635)(477.15422313,337.4122635)(477.19422701,337.40226336)
\curveto(477.24422304,337.38226353)(477.29422299,337.37726354)(477.34422701,337.38726336)
\curveto(477.40422288,337.38726353)(477.45922283,337.38226353)(477.50922701,337.37226336)
\lineto(477.65922701,337.37226336)
\curveto(477.71922257,337.35226356)(477.80422248,337.34726357)(477.91422701,337.35726336)
\curveto(478.02422226,337.35726356)(478.10422218,337.36226355)(478.15422701,337.37226336)
\curveto(478.1842221,337.37226354)(478.21422207,337.37726354)(478.24422701,337.38726336)
\lineto(478.34922701,337.38726336)
\curveto(478.39922189,337.39726352)(478.45422183,337.40226351)(478.51422701,337.40226336)
\curveto(478.57422171,337.40226351)(478.62922166,337.4122635)(478.67922701,337.43226336)
\curveto(478.80922148,337.46226345)(478.93422135,337.49226342)(479.05422701,337.52226336)
\curveto(479.1842211,337.54226337)(479.30922098,337.57726334)(479.42922701,337.62726336)
\curveto(479.90922038,337.82726309)(480.31921997,338.07726284)(480.65922701,338.37726336)
\curveto(480.99921929,338.67726224)(481.27421901,339.06726185)(481.48422701,339.54726336)
\curveto(481.53421875,339.64726127)(481.57421871,339.75226116)(481.60422701,339.86226336)
\curveto(481.63421865,339.98226093)(481.66921862,340.09726082)(481.70922701,340.20726336)
\curveto(481.71921857,340.27726064)(481.72921856,340.34226057)(481.73922701,340.40226336)
\curveto(481.74921854,340.46226045)(481.76421852,340.52726039)(481.78422701,340.59726336)
\curveto(481.80421848,340.67726024)(481.80921848,340.75726016)(481.79922701,340.83726336)
\curveto(481.79921849,340.91726)(481.80921848,340.99725992)(481.82922701,341.07726336)
\lineto(481.82922701,341.22726336)
\curveto(481.84921844,341.28725963)(481.85921843,341.37225954)(481.85922701,341.48226336)
\curveto(481.85921843,341.59225932)(481.84921844,341.67725924)(481.82922701,341.73726336)
\moveto(479.72922701,341.19726336)
\curveto(479.71922057,341.14725977)(479.71422057,341.09725982)(479.71422701,341.04726336)
\lineto(479.71422701,340.91226336)
\curveto(479.70422058,340.87226004)(479.69922059,340.83226008)(479.69922701,340.79226336)
\curveto(479.69922059,340.76226015)(479.69422059,340.72726019)(479.68422701,340.68726336)
\curveto(479.65422063,340.57726034)(479.62922066,340.47226044)(479.60922701,340.37226336)
\curveto(479.5892207,340.27226064)(479.55922073,340.17226074)(479.51922701,340.07226336)
\curveto(479.40922088,339.82226109)(479.27422101,339.6122613)(479.11422701,339.44226336)
\curveto(478.95422133,339.27226164)(478.74422154,339.13726178)(478.48422701,339.03726336)
\curveto(478.41422187,339.00726191)(478.33922195,338.98726193)(478.25922701,338.97726336)
\curveto(478.17922211,338.96726195)(478.09922219,338.95226196)(478.01922701,338.93226336)
\lineto(477.89922701,338.93226336)
\curveto(477.85922243,338.92226199)(477.81422247,338.917262)(477.76422701,338.91726336)
\lineto(477.64422701,338.94726336)
\curveto(477.60422268,338.95726196)(477.56922272,338.95726196)(477.53922701,338.94726336)
\curveto(477.50922278,338.94726197)(477.47422281,338.95226196)(477.43422701,338.96226336)
\curveto(477.34422294,338.98226193)(477.25422303,339.00726191)(477.16422701,339.03726336)
\curveto(477.0842232,339.06726185)(477.00922328,339.10726181)(476.93922701,339.15726336)
\curveto(476.6892236,339.30726161)(476.50422378,339.47226144)(476.38422701,339.65226336)
\curveto(476.27422401,339.84226107)(476.16922412,340.08726083)(476.06922701,340.38726336)
\curveto(476.04922424,340.46726045)(476.03422425,340.54226037)(476.02422701,340.61226336)
\curveto(476.01422427,340.69226022)(475.99922429,340.77226014)(475.97922701,340.85226336)
\lineto(475.97922701,340.98726336)
\curveto(475.95922433,341.05725986)(475.94422434,341.16225975)(475.93422701,341.30226336)
\curveto(475.93422435,341.44225947)(475.94422434,341.54725937)(475.96422701,341.61726336)
\lineto(475.96422701,341.76726336)
\curveto(475.96422432,341.8172591)(475.96922432,341.86725905)(475.97922701,341.91726336)
\curveto(475.99922429,342.02725889)(476.01422427,342.13725878)(476.02422701,342.24726336)
\curveto(476.04422424,342.35725856)(476.06922422,342.46225845)(476.09922701,342.56226336)
\curveto(476.1892241,342.83225808)(476.30922398,343.06725785)(476.45922701,343.26726336)
\curveto(476.61922367,343.47725744)(476.82422346,343.63725728)(477.07422701,343.74726336)
\curveto(477.12422316,343.77725714)(477.17922311,343.79725712)(477.23922701,343.80726336)
\lineto(477.44922701,343.86726336)
\curveto(477.47922281,343.87725704)(477.51422277,343.87725704)(477.55422701,343.86726336)
\curveto(477.59422269,343.86725705)(477.62922266,343.87725704)(477.65922701,343.89726336)
\lineto(477.92922701,343.89726336)
\curveto(478.01922227,343.90725701)(478.10422218,343.90225701)(478.18422701,343.88226336)
\curveto(478.25422203,343.86225705)(478.31922197,343.84225707)(478.37922701,343.82226336)
\curveto(478.43922185,343.8122571)(478.49922179,343.79725712)(478.55922701,343.77726336)
\curveto(478.80922148,343.66725725)(479.00922128,343.5172574)(479.15922701,343.32726336)
\curveto(479.30922098,343.14725777)(479.43922085,342.92725799)(479.54922701,342.66726336)
\curveto(479.57922071,342.58725833)(479.59922069,342.50225841)(479.60922701,342.41226336)
\lineto(479.66922701,342.17226336)
\curveto(479.67922061,342.15225876)(479.6842206,342.12225879)(479.68422701,342.08226336)
\curveto(479.69422059,342.03225888)(479.69922059,341.97725894)(479.69922701,341.91726336)
\curveto(479.69922059,341.85725906)(479.70922058,341.80225911)(479.72922701,341.75226336)
\lineto(479.72922701,341.63226336)
\curveto(479.73922055,341.58225933)(479.74422054,341.50725941)(479.74422701,341.40726336)
\curveto(479.74422054,341.3172596)(479.73922055,341.24725967)(479.72922701,341.19726336)
\moveto(478.49922701,348.36726336)
\lineto(479.56422701,348.36726336)
\curveto(479.64422064,348.36725255)(479.73922055,348.36725255)(479.84922701,348.36726336)
\curveto(479.95922033,348.36725255)(480.03922025,348.35225256)(480.08922701,348.32226336)
\curveto(480.10922018,348.3122526)(480.11922017,348.29725262)(480.11922701,348.27726336)
\curveto(480.12922016,348.26725265)(480.14422014,348.25725266)(480.16422701,348.24726336)
\curveto(480.17422011,348.12725279)(480.12422016,348.02225289)(480.01422701,347.93226336)
\curveto(479.91422037,347.84225307)(479.82922046,347.76225315)(479.75922701,347.69226336)
\curveto(479.67922061,347.62225329)(479.59922069,347.54725337)(479.51922701,347.46726336)
\curveto(479.44922084,347.39725352)(479.37422091,347.33225358)(479.29422701,347.27226336)
\curveto(479.25422103,347.24225367)(479.21922107,347.20725371)(479.18922701,347.16726336)
\curveto(479.16922112,347.13725378)(479.13922115,347.1122538)(479.09922701,347.09226336)
\curveto(479.07922121,347.06225385)(479.05422123,347.03725388)(479.02422701,347.01726336)
\lineto(478.87422701,346.86726336)
\lineto(478.72422701,346.74726336)
\lineto(478.67922701,346.70226336)
\curveto(478.67922161,346.69225422)(478.66922162,346.67725424)(478.64922701,346.65726336)
\curveto(478.56922172,346.59725432)(478.4892218,346.53225438)(478.40922701,346.46226336)
\curveto(478.33922195,346.39225452)(478.24922204,346.33725458)(478.13922701,346.29726336)
\curveto(478.09922219,346.28725463)(478.05922223,346.28225463)(478.01922701,346.28226336)
\curveto(477.9892223,346.28225463)(477.94922234,346.27725464)(477.89922701,346.26726336)
\curveto(477.86922242,346.25725466)(477.82922246,346.25225466)(477.77922701,346.25226336)
\curveto(477.72922256,346.26225465)(477.6842226,346.26725465)(477.64422701,346.26726336)
\lineto(477.29922701,346.26726336)
\curveto(477.17922311,346.26725465)(477.0892232,346.29225462)(477.02922701,346.34226336)
\curveto(476.96922332,346.38225453)(476.95422333,346.45225446)(476.98422701,346.55226336)
\curveto(477.00422328,346.63225428)(477.03922325,346.70225421)(477.08922701,346.76226336)
\curveto(477.13922315,346.83225408)(477.1842231,346.90225401)(477.22422701,346.97226336)
\curveto(477.32422296,347.1122538)(477.41922287,347.24725367)(477.50922701,347.37726336)
\curveto(477.59922269,347.50725341)(477.6892226,347.64225327)(477.77922701,347.78226336)
\curveto(477.82922246,347.86225305)(477.87922241,347.94725297)(477.92922701,348.03726336)
\curveto(477.9892223,348.12725279)(478.05422223,348.19725272)(478.12422701,348.24726336)
\curveto(478.16422212,348.27725264)(478.23422205,348.3122526)(478.33422701,348.35226336)
\curveto(478.35422193,348.36225255)(478.37922191,348.36225255)(478.40922701,348.35226336)
\curveto(478.44922184,348.35225256)(478.47922181,348.35725256)(478.49922701,348.36726336)
}
}
{
\newrgbcolor{curcolor}{0 0 0}
\pscustom[linestyle=none,fillstyle=solid,fillcolor=curcolor]
{
\newpath
\moveto(487.65414888,345.48726336)
\curveto(488.25414308,345.50725541)(488.75414258,345.42225549)(489.15414888,345.23226336)
\curveto(489.55414178,345.04225587)(489.86914146,344.76225615)(490.09914888,344.39226336)
\curveto(490.16914116,344.28225663)(490.22414111,344.16225675)(490.26414888,344.03226336)
\curveto(490.30414103,343.912257)(490.34414099,343.78725713)(490.38414888,343.65726336)
\curveto(490.40414093,343.57725734)(490.41414092,343.50225741)(490.41414888,343.43226336)
\curveto(490.42414091,343.36225755)(490.43914089,343.29225762)(490.45914888,343.22226336)
\curveto(490.45914087,343.16225775)(490.46414087,343.12225779)(490.47414888,343.10226336)
\curveto(490.49414084,342.96225795)(490.50414083,342.8172581)(490.50414888,342.66726336)
\lineto(490.50414888,342.23226336)
\lineto(490.50414888,340.89726336)
\lineto(490.50414888,338.46726336)
\curveto(490.50414083,338.27726264)(490.49914083,338.09226282)(490.48914888,337.91226336)
\curveto(490.48914084,337.74226317)(490.41914091,337.63226328)(490.27914888,337.58226336)
\curveto(490.21914111,337.56226335)(490.14914118,337.55226336)(490.06914888,337.55226336)
\lineto(489.82914888,337.55226336)
\lineto(489.01914888,337.55226336)
\curveto(488.89914243,337.55226336)(488.78914254,337.55726336)(488.68914888,337.56726336)
\curveto(488.59914273,337.58726333)(488.5291428,337.63226328)(488.47914888,337.70226336)
\curveto(488.43914289,337.76226315)(488.41414292,337.83726308)(488.40414888,337.92726336)
\lineto(488.40414888,338.24226336)
\lineto(488.40414888,339.29226336)
\lineto(488.40414888,341.52726336)
\curveto(488.40414293,341.89725902)(488.38914294,342.23725868)(488.35914888,342.54726336)
\curveto(488.329143,342.86725805)(488.23914309,343.13725778)(488.08914888,343.35726336)
\curveto(487.94914338,343.55725736)(487.74414359,343.69725722)(487.47414888,343.77726336)
\curveto(487.42414391,343.79725712)(487.36914396,343.80725711)(487.30914888,343.80726336)
\curveto(487.25914407,343.80725711)(487.20414413,343.8172571)(487.14414888,343.83726336)
\curveto(487.09414424,343.84725707)(487.0291443,343.84725707)(486.94914888,343.83726336)
\curveto(486.87914445,343.83725708)(486.82414451,343.83225708)(486.78414888,343.82226336)
\curveto(486.74414459,343.8122571)(486.70914462,343.80725711)(486.67914888,343.80726336)
\curveto(486.64914468,343.80725711)(486.61914471,343.80225711)(486.58914888,343.79226336)
\curveto(486.35914497,343.73225718)(486.17414516,343.65225726)(486.03414888,343.55226336)
\curveto(485.71414562,343.32225759)(485.52414581,342.98725793)(485.46414888,342.54726336)
\curveto(485.40414593,342.10725881)(485.37414596,341.6122593)(485.37414888,341.06226336)
\lineto(485.37414888,339.18726336)
\lineto(485.37414888,338.27226336)
\lineto(485.37414888,338.00226336)
\curveto(485.37414596,337.912263)(485.35914597,337.83726308)(485.32914888,337.77726336)
\curveto(485.27914605,337.66726325)(485.19914613,337.60226331)(485.08914888,337.58226336)
\curveto(484.97914635,337.56226335)(484.84414649,337.55226336)(484.68414888,337.55226336)
\lineto(483.93414888,337.55226336)
\curveto(483.82414751,337.55226336)(483.71414762,337.55726336)(483.60414888,337.56726336)
\curveto(483.49414784,337.57726334)(483.41414792,337.6122633)(483.36414888,337.67226336)
\curveto(483.29414804,337.76226315)(483.25914807,337.89226302)(483.25914888,338.06226336)
\curveto(483.26914806,338.23226268)(483.27414806,338.39226252)(483.27414888,338.54226336)
\lineto(483.27414888,340.58226336)
\lineto(483.27414888,343.88226336)
\lineto(483.27414888,344.64726336)
\lineto(483.27414888,344.94726336)
\curveto(483.28414805,345.03725588)(483.31414802,345.1122558)(483.36414888,345.17226336)
\curveto(483.38414795,345.20225571)(483.41414792,345.22225569)(483.45414888,345.23226336)
\curveto(483.50414783,345.25225566)(483.55414778,345.26725565)(483.60414888,345.27726336)
\lineto(483.67914888,345.27726336)
\curveto(483.7291476,345.28725563)(483.77914755,345.29225562)(483.82914888,345.29226336)
\lineto(483.99414888,345.29226336)
\lineto(484.62414888,345.29226336)
\curveto(484.70414663,345.29225562)(484.77914655,345.28725563)(484.84914888,345.27726336)
\curveto(484.9291464,345.27725564)(484.99914633,345.26725565)(485.05914888,345.24726336)
\curveto(485.1291462,345.2172557)(485.17414616,345.17225574)(485.19414888,345.11226336)
\curveto(485.22414611,345.05225586)(485.24914608,344.98225593)(485.26914888,344.90226336)
\curveto(485.27914605,344.86225605)(485.27914605,344.82725609)(485.26914888,344.79726336)
\curveto(485.26914606,344.76725615)(485.27914605,344.73725618)(485.29914888,344.70726336)
\curveto(485.31914601,344.65725626)(485.334146,344.62725629)(485.34414888,344.61726336)
\curveto(485.36414597,344.60725631)(485.38914594,344.59225632)(485.41914888,344.57226336)
\curveto(485.5291458,344.56225635)(485.61914571,344.59725632)(485.68914888,344.67726336)
\curveto(485.75914557,344.76725615)(485.8341455,344.83725608)(485.91414888,344.88726336)
\curveto(486.18414515,345.08725583)(486.48414485,345.24725567)(486.81414888,345.36726336)
\curveto(486.90414443,345.39725552)(486.99414434,345.4172555)(487.08414888,345.42726336)
\curveto(487.18414415,345.43725548)(487.28914404,345.45225546)(487.39914888,345.47226336)
\curveto(487.4291439,345.48225543)(487.47414386,345.48225543)(487.53414888,345.47226336)
\curveto(487.59414374,345.47225544)(487.6341437,345.47725544)(487.65414888,345.48726336)
}
}
{
\newrgbcolor{curcolor}{0 0 0}
\pscustom[linestyle=none,fillstyle=solid,fillcolor=curcolor]
{
}
}
{
\newrgbcolor{curcolor}{0 0 0}
\pscustom[linestyle=none,fillstyle=solid,fillcolor=curcolor]
{
\newpath
\moveto(503.88555513,338.40726336)
\lineto(503.88555513,337.98726336)
\curveto(503.88554676,337.85726306)(503.85554679,337.75226316)(503.79555513,337.67226336)
\curveto(503.7455469,337.62226329)(503.68054697,337.58726333)(503.60055513,337.56726336)
\curveto(503.52054713,337.55726336)(503.43054722,337.55226336)(503.33055513,337.55226336)
\lineto(502.50555513,337.55226336)
\lineto(502.22055513,337.55226336)
\curveto(502.14054851,337.56226335)(502.07554857,337.58726333)(502.02555513,337.62726336)
\curveto(501.95554869,337.67726324)(501.91554873,337.74226317)(501.90555513,337.82226336)
\curveto(501.89554875,337.90226301)(501.87554877,337.98226293)(501.84555513,338.06226336)
\curveto(501.82554882,338.08226283)(501.80554884,338.09726282)(501.78555513,338.10726336)
\curveto(501.77554887,338.12726279)(501.76054889,338.14726277)(501.74055513,338.16726336)
\curveto(501.63054902,338.16726275)(501.5505491,338.14226277)(501.50055513,338.09226336)
\lineto(501.35055513,337.94226336)
\curveto(501.28054937,337.89226302)(501.21554943,337.84726307)(501.15555513,337.80726336)
\curveto(501.09554955,337.77726314)(501.03054962,337.73726318)(500.96055513,337.68726336)
\curveto(500.92054973,337.66726325)(500.87554977,337.64726327)(500.82555513,337.62726336)
\curveto(500.78554986,337.60726331)(500.74054991,337.58726333)(500.69055513,337.56726336)
\curveto(500.5505501,337.5172634)(500.40055025,337.47226344)(500.24055513,337.43226336)
\curveto(500.19055046,337.4122635)(500.1455505,337.40226351)(500.10555513,337.40226336)
\curveto(500.06555058,337.40226351)(500.02555062,337.39726352)(499.98555513,337.38726336)
\lineto(499.85055513,337.38726336)
\curveto(499.82055083,337.37726354)(499.78055087,337.37226354)(499.73055513,337.37226336)
\lineto(499.59555513,337.37226336)
\curveto(499.53555111,337.35226356)(499.4455512,337.34726357)(499.32555513,337.35726336)
\curveto(499.20555144,337.35726356)(499.12055153,337.36726355)(499.07055513,337.38726336)
\curveto(499.00055165,337.40726351)(498.93555171,337.4172635)(498.87555513,337.41726336)
\curveto(498.82555182,337.40726351)(498.77055188,337.4122635)(498.71055513,337.43226336)
\lineto(498.35055513,337.55226336)
\curveto(498.24055241,337.58226333)(498.13055252,337.62226329)(498.02055513,337.67226336)
\curveto(497.67055298,337.82226309)(497.35555329,338.05226286)(497.07555513,338.36226336)
\curveto(496.80555384,338.68226223)(496.59055406,339.0172619)(496.43055513,339.36726336)
\curveto(496.38055427,339.47726144)(496.34055431,339.58226133)(496.31055513,339.68226336)
\curveto(496.28055437,339.79226112)(496.2455544,339.90226101)(496.20555513,340.01226336)
\curveto(496.19555445,340.05226086)(496.19055446,340.08726083)(496.19055513,340.11726336)
\curveto(496.19055446,340.15726076)(496.18055447,340.20226071)(496.16055513,340.25226336)
\curveto(496.14055451,340.33226058)(496.12055453,340.4172605)(496.10055513,340.50726336)
\curveto(496.09055456,340.60726031)(496.07555457,340.70726021)(496.05555513,340.80726336)
\curveto(496.0455546,340.83726008)(496.04055461,340.87226004)(496.04055513,340.91226336)
\curveto(496.0505546,340.95225996)(496.0505546,340.98725993)(496.04055513,341.01726336)
\lineto(496.04055513,341.15226336)
\curveto(496.04055461,341.20225971)(496.03555461,341.25225966)(496.02555513,341.30226336)
\curveto(496.01555463,341.35225956)(496.01055464,341.40725951)(496.01055513,341.46726336)
\curveto(496.01055464,341.53725938)(496.01555463,341.59225932)(496.02555513,341.63226336)
\curveto(496.03555461,341.68225923)(496.04055461,341.72725919)(496.04055513,341.76726336)
\lineto(496.04055513,341.91726336)
\curveto(496.0505546,341.96725895)(496.0505546,342.0122589)(496.04055513,342.05226336)
\curveto(496.04055461,342.10225881)(496.0505546,342.15225876)(496.07055513,342.20226336)
\curveto(496.09055456,342.3122586)(496.10555454,342.4172585)(496.11555513,342.51726336)
\curveto(496.13555451,342.6172583)(496.16055449,342.7172582)(496.19055513,342.81726336)
\curveto(496.23055442,342.93725798)(496.26555438,343.05225786)(496.29555513,343.16226336)
\curveto(496.32555432,343.27225764)(496.36555428,343.38225753)(496.41555513,343.49226336)
\curveto(496.55555409,343.79225712)(496.73055392,344.07725684)(496.94055513,344.34726336)
\curveto(496.96055369,344.37725654)(496.98555366,344.40225651)(497.01555513,344.42226336)
\curveto(497.05555359,344.45225646)(497.08555356,344.48225643)(497.10555513,344.51226336)
\curveto(497.1455535,344.56225635)(497.18555346,344.60725631)(497.22555513,344.64726336)
\curveto(497.26555338,344.68725623)(497.31055334,344.72725619)(497.36055513,344.76726336)
\curveto(497.40055325,344.78725613)(497.43555321,344.8122561)(497.46555513,344.84226336)
\curveto(497.49555315,344.88225603)(497.53055312,344.912256)(497.57055513,344.93226336)
\curveto(497.82055283,345.10225581)(498.11055254,345.24225567)(498.44055513,345.35226336)
\curveto(498.51055214,345.37225554)(498.58055207,345.38725553)(498.65055513,345.39726336)
\curveto(498.73055192,345.40725551)(498.81055184,345.42225549)(498.89055513,345.44226336)
\curveto(498.96055169,345.46225545)(499.0505516,345.47225544)(499.16055513,345.47226336)
\curveto(499.27055138,345.48225543)(499.38055127,345.48725543)(499.49055513,345.48726336)
\curveto(499.60055105,345.48725543)(499.70555094,345.48225543)(499.80555513,345.47226336)
\curveto(499.91555073,345.46225545)(500.00555064,345.44725547)(500.07555513,345.42726336)
\curveto(500.22555042,345.37725554)(500.37055028,345.33225558)(500.51055513,345.29226336)
\curveto(500.65055,345.25225566)(500.78054987,345.19725572)(500.90055513,345.12726336)
\curveto(500.97054968,345.07725584)(501.03554961,345.02725589)(501.09555513,344.97726336)
\curveto(501.15554949,344.93725598)(501.22054943,344.89225602)(501.29055513,344.84226336)
\curveto(501.33054932,344.8122561)(501.38554926,344.77225614)(501.45555513,344.72226336)
\curveto(501.53554911,344.67225624)(501.61054904,344.67225624)(501.68055513,344.72226336)
\curveto(501.72054893,344.74225617)(501.74054891,344.77725614)(501.74055513,344.82726336)
\curveto(501.74054891,344.87725604)(501.7505489,344.92725599)(501.77055513,344.97726336)
\lineto(501.77055513,345.12726336)
\curveto(501.78054887,345.15725576)(501.78554886,345.19225572)(501.78555513,345.23226336)
\lineto(501.78555513,345.35226336)
\lineto(501.78555513,347.39226336)
\curveto(501.78554886,347.50225341)(501.78054887,347.62225329)(501.77055513,347.75226336)
\curveto(501.77054888,347.89225302)(501.79554885,347.99725292)(501.84555513,348.06726336)
\curveto(501.88554876,348.14725277)(501.96054869,348.19725272)(502.07055513,348.21726336)
\curveto(502.09054856,348.22725269)(502.11054854,348.22725269)(502.13055513,348.21726336)
\curveto(502.1505485,348.2172527)(502.17054848,348.22225269)(502.19055513,348.23226336)
\lineto(503.25555513,348.23226336)
\curveto(503.37554727,348.23225268)(503.48554716,348.22725269)(503.58555513,348.21726336)
\curveto(503.68554696,348.20725271)(503.76054689,348.16725275)(503.81055513,348.09726336)
\curveto(503.86054679,348.0172529)(503.88554676,347.912253)(503.88555513,347.78226336)
\lineto(503.88555513,347.42226336)
\lineto(503.88555513,338.40726336)
\moveto(501.84555513,341.34726336)
\curveto(501.85554879,341.38725953)(501.85554879,341.42725949)(501.84555513,341.46726336)
\lineto(501.84555513,341.60226336)
\curveto(501.8455488,341.70225921)(501.84054881,341.80225911)(501.83055513,341.90226336)
\curveto(501.82054883,342.00225891)(501.80554884,342.09225882)(501.78555513,342.17226336)
\curveto(501.76554888,342.28225863)(501.7455489,342.38225853)(501.72555513,342.47226336)
\curveto(501.71554893,342.56225835)(501.69054896,342.64725827)(501.65055513,342.72726336)
\curveto(501.51054914,343.08725783)(501.30554934,343.37225754)(501.03555513,343.58226336)
\curveto(500.77554987,343.79225712)(500.39555025,343.89725702)(499.89555513,343.89726336)
\curveto(499.83555081,343.89725702)(499.75555089,343.88725703)(499.65555513,343.86726336)
\curveto(499.57555107,343.84725707)(499.50055115,343.82725709)(499.43055513,343.80726336)
\curveto(499.37055128,343.79725712)(499.31055134,343.77725714)(499.25055513,343.74726336)
\curveto(498.98055167,343.63725728)(498.77055188,343.46725745)(498.62055513,343.23726336)
\curveto(498.47055218,343.00725791)(498.3505523,342.74725817)(498.26055513,342.45726336)
\curveto(498.23055242,342.35725856)(498.21055244,342.25725866)(498.20055513,342.15726336)
\curveto(498.19055246,342.05725886)(498.17055248,341.95225896)(498.14055513,341.84226336)
\lineto(498.14055513,341.63226336)
\curveto(498.12055253,341.54225937)(498.11555253,341.4172595)(498.12555513,341.25726336)
\curveto(498.13555251,341.10725981)(498.1505525,340.99725992)(498.17055513,340.92726336)
\lineto(498.17055513,340.83726336)
\curveto(498.18055247,340.8172601)(498.18555246,340.79726012)(498.18555513,340.77726336)
\curveto(498.20555244,340.69726022)(498.22055243,340.62226029)(498.23055513,340.55226336)
\curveto(498.2505524,340.48226043)(498.27055238,340.40726051)(498.29055513,340.32726336)
\curveto(498.46055219,339.80726111)(498.7505519,339.42226149)(499.16055513,339.17226336)
\curveto(499.29055136,339.08226183)(499.47055118,339.0122619)(499.70055513,338.96226336)
\curveto(499.74055091,338.95226196)(499.80055085,338.94726197)(499.88055513,338.94726336)
\curveto(499.91055074,338.93726198)(499.95555069,338.92726199)(500.01555513,338.91726336)
\curveto(500.08555056,338.917262)(500.14055051,338.92226199)(500.18055513,338.93226336)
\curveto(500.26055039,338.95226196)(500.34055031,338.96726195)(500.42055513,338.97726336)
\curveto(500.50055015,338.98726193)(500.58055007,339.00726191)(500.66055513,339.03726336)
\curveto(500.91054974,339.14726177)(501.11054954,339.28726163)(501.26055513,339.45726336)
\curveto(501.41054924,339.62726129)(501.54054911,339.84226107)(501.65055513,340.10226336)
\curveto(501.69054896,340.19226072)(501.72054893,340.28226063)(501.74055513,340.37226336)
\curveto(501.76054889,340.47226044)(501.78054887,340.57726034)(501.80055513,340.68726336)
\curveto(501.81054884,340.73726018)(501.81054884,340.78226013)(501.80055513,340.82226336)
\curveto(501.80054885,340.87226004)(501.81054884,340.92225999)(501.83055513,340.97226336)
\curveto(501.84054881,341.00225991)(501.8455488,341.03725988)(501.84555513,341.07726336)
\lineto(501.84555513,341.21226336)
\lineto(501.84555513,341.34726336)
}
}
{
\newrgbcolor{curcolor}{0 0 0}
\pscustom[linestyle=none,fillstyle=solid,fillcolor=curcolor]
{
\newpath
\moveto(512.83047701,341.49726336)
\curveto(512.85046884,341.4172595)(512.85046884,341.32725959)(512.83047701,341.22726336)
\curveto(512.81046888,341.12725979)(512.77546892,341.06225985)(512.72547701,341.03226336)
\curveto(512.67546902,340.99225992)(512.60046909,340.96225995)(512.50047701,340.94226336)
\curveto(512.41046928,340.93225998)(512.30546939,340.92225999)(512.18547701,340.91226336)
\lineto(511.84047701,340.91226336)
\curveto(511.73046996,340.92225999)(511.63047006,340.92725999)(511.54047701,340.92726336)
\lineto(507.88047701,340.92726336)
\lineto(507.67047701,340.92726336)
\curveto(507.61047408,340.92725999)(507.55547414,340.91726)(507.50547701,340.89726336)
\curveto(507.42547427,340.85726006)(507.37547432,340.8172601)(507.35547701,340.77726336)
\curveto(507.33547436,340.75726016)(507.31547438,340.7172602)(507.29547701,340.65726336)
\curveto(507.27547442,340.60726031)(507.27047442,340.55726036)(507.28047701,340.50726336)
\curveto(507.30047439,340.44726047)(507.31047438,340.38726053)(507.31047701,340.32726336)
\curveto(507.32047437,340.27726064)(507.33547436,340.22226069)(507.35547701,340.16226336)
\curveto(507.43547426,339.92226099)(507.53047416,339.72226119)(507.64047701,339.56226336)
\curveto(507.76047393,339.4122615)(507.92047377,339.27726164)(508.12047701,339.15726336)
\curveto(508.20047349,339.10726181)(508.28047341,339.07226184)(508.36047701,339.05226336)
\curveto(508.45047324,339.04226187)(508.54047315,339.02226189)(508.63047701,338.99226336)
\curveto(508.71047298,338.97226194)(508.82047287,338.95726196)(508.96047701,338.94726336)
\curveto(509.10047259,338.93726198)(509.22047247,338.94226197)(509.32047701,338.96226336)
\lineto(509.45547701,338.96226336)
\curveto(509.55547214,338.98226193)(509.64547205,339.00226191)(509.72547701,339.02226336)
\curveto(509.81547188,339.05226186)(509.90047179,339.08226183)(509.98047701,339.11226336)
\curveto(510.08047161,339.16226175)(510.1904715,339.22726169)(510.31047701,339.30726336)
\curveto(510.44047125,339.38726153)(510.53547116,339.46726145)(510.59547701,339.54726336)
\curveto(510.64547105,339.6172613)(510.695471,339.68226123)(510.74547701,339.74226336)
\curveto(510.80547089,339.8122611)(510.87547082,339.86226105)(510.95547701,339.89226336)
\curveto(511.05547064,339.94226097)(511.18047051,339.96226095)(511.33047701,339.95226336)
\lineto(511.76547701,339.95226336)
\lineto(511.94547701,339.95226336)
\curveto(512.01546968,339.96226095)(512.07546962,339.95726096)(512.12547701,339.93726336)
\lineto(512.27547701,339.93726336)
\curveto(512.37546932,339.917261)(512.44546925,339.89226102)(512.48547701,339.86226336)
\curveto(512.52546917,339.84226107)(512.54546915,339.79726112)(512.54547701,339.72726336)
\curveto(512.55546914,339.65726126)(512.55046914,339.59726132)(512.53047701,339.54726336)
\curveto(512.48046921,339.40726151)(512.42546927,339.28226163)(512.36547701,339.17226336)
\curveto(512.30546939,339.06226185)(512.23546946,338.95226196)(512.15547701,338.84226336)
\curveto(511.93546976,338.5122624)(511.68547001,338.24726267)(511.40547701,338.04726336)
\curveto(511.12547057,337.84726307)(510.77547092,337.67726324)(510.35547701,337.53726336)
\curveto(510.24547145,337.49726342)(510.13547156,337.47226344)(510.02547701,337.46226336)
\curveto(509.91547178,337.45226346)(509.80047189,337.43226348)(509.68047701,337.40226336)
\curveto(509.64047205,337.39226352)(509.5954721,337.39226352)(509.54547701,337.40226336)
\curveto(509.50547219,337.40226351)(509.46547223,337.39726352)(509.42547701,337.38726336)
\lineto(509.26047701,337.38726336)
\curveto(509.21047248,337.36726355)(509.15047254,337.36226355)(509.08047701,337.37226336)
\curveto(509.02047267,337.37226354)(508.96547273,337.37726354)(508.91547701,337.38726336)
\curveto(508.83547286,337.39726352)(508.76547293,337.39726352)(508.70547701,337.38726336)
\curveto(508.64547305,337.37726354)(508.58047311,337.38226353)(508.51047701,337.40226336)
\curveto(508.46047323,337.42226349)(508.40547329,337.43226348)(508.34547701,337.43226336)
\curveto(508.28547341,337.43226348)(508.23047346,337.44226347)(508.18047701,337.46226336)
\curveto(508.07047362,337.48226343)(507.96047373,337.50726341)(507.85047701,337.53726336)
\curveto(507.74047395,337.55726336)(507.64047405,337.59226332)(507.55047701,337.64226336)
\curveto(507.44047425,337.68226323)(507.33547436,337.7172632)(507.23547701,337.74726336)
\curveto(507.14547455,337.78726313)(507.06047463,337.83226308)(506.98047701,337.88226336)
\curveto(506.66047503,338.08226283)(506.37547532,338.3122626)(506.12547701,338.57226336)
\curveto(505.87547582,338.84226207)(505.67047602,339.15226176)(505.51047701,339.50226336)
\curveto(505.46047623,339.6122613)(505.42047627,339.72226119)(505.39047701,339.83226336)
\curveto(505.36047633,339.95226096)(505.32047637,340.07226084)(505.27047701,340.19226336)
\curveto(505.26047643,340.23226068)(505.25547644,340.26726065)(505.25547701,340.29726336)
\curveto(505.25547644,340.33726058)(505.25047644,340.37726054)(505.24047701,340.41726336)
\curveto(505.20047649,340.53726038)(505.17547652,340.66726025)(505.16547701,340.80726336)
\lineto(505.13547701,341.22726336)
\curveto(505.13547656,341.27725964)(505.13047656,341.33225958)(505.12047701,341.39226336)
\curveto(505.12047657,341.45225946)(505.12547657,341.50725941)(505.13547701,341.55726336)
\lineto(505.13547701,341.73726336)
\lineto(505.18047701,342.09726336)
\curveto(505.22047647,342.26725865)(505.25547644,342.43225848)(505.28547701,342.59226336)
\curveto(505.31547638,342.75225816)(505.36047633,342.90225801)(505.42047701,343.04226336)
\curveto(505.85047584,344.08225683)(506.58047511,344.8172561)(507.61047701,345.24726336)
\curveto(507.75047394,345.30725561)(507.8904738,345.34725557)(508.03047701,345.36726336)
\curveto(508.18047351,345.39725552)(508.33547336,345.43225548)(508.49547701,345.47226336)
\curveto(508.57547312,345.48225543)(508.65047304,345.48725543)(508.72047701,345.48726336)
\curveto(508.7904729,345.48725543)(508.86547283,345.49225542)(508.94547701,345.50226336)
\curveto(509.45547224,345.5122554)(509.8904718,345.45225546)(510.25047701,345.32226336)
\curveto(510.62047107,345.20225571)(510.95047074,345.04225587)(511.24047701,344.84226336)
\curveto(511.33047036,344.78225613)(511.42047027,344.7122562)(511.51047701,344.63226336)
\curveto(511.60047009,344.56225635)(511.68047001,344.48725643)(511.75047701,344.40726336)
\curveto(511.78046991,344.35725656)(511.82046987,344.3172566)(511.87047701,344.28726336)
\curveto(511.95046974,344.17725674)(512.02546967,344.06225685)(512.09547701,343.94226336)
\curveto(512.16546953,343.83225708)(512.24046945,343.7172572)(512.32047701,343.59726336)
\curveto(512.37046932,343.50725741)(512.41046928,343.4122575)(512.44047701,343.31226336)
\curveto(512.48046921,343.22225769)(512.52046917,343.12225779)(512.56047701,343.01226336)
\curveto(512.61046908,342.88225803)(512.65046904,342.74725817)(512.68047701,342.60726336)
\curveto(512.71046898,342.46725845)(512.74546895,342.32725859)(512.78547701,342.18726336)
\curveto(512.80546889,342.10725881)(512.81046888,342.0172589)(512.80047701,341.91726336)
\curveto(512.80046889,341.82725909)(512.81046888,341.74225917)(512.83047701,341.66226336)
\lineto(512.83047701,341.49726336)
\moveto(510.58047701,342.38226336)
\curveto(510.65047104,342.48225843)(510.65547104,342.60225831)(510.59547701,342.74226336)
\curveto(510.54547115,342.89225802)(510.50547119,343.00225791)(510.47547701,343.07226336)
\curveto(510.33547136,343.34225757)(510.15047154,343.54725737)(509.92047701,343.68726336)
\curveto(509.690472,343.83725708)(509.37047232,343.917257)(508.96047701,343.92726336)
\curveto(508.93047276,343.90725701)(508.8954728,343.90225701)(508.85547701,343.91226336)
\curveto(508.81547288,343.92225699)(508.78047291,343.92225699)(508.75047701,343.91226336)
\curveto(508.70047299,343.89225702)(508.64547305,343.87725704)(508.58547701,343.86726336)
\curveto(508.52547317,343.86725705)(508.47047322,343.85725706)(508.42047701,343.83726336)
\curveto(507.98047371,343.69725722)(507.65547404,343.42225749)(507.44547701,343.01226336)
\curveto(507.42547427,342.97225794)(507.40047429,342.917258)(507.37047701,342.84726336)
\curveto(507.35047434,342.78725813)(507.33547436,342.72225819)(507.32547701,342.65226336)
\curveto(507.31547438,342.59225832)(507.31547438,342.53225838)(507.32547701,342.47226336)
\curveto(507.34547435,342.4122585)(507.38047431,342.36225855)(507.43047701,342.32226336)
\curveto(507.51047418,342.27225864)(507.62047407,342.24725867)(507.76047701,342.24726336)
\lineto(508.16547701,342.24726336)
\lineto(509.83047701,342.24726336)
\lineto(510.26547701,342.24726336)
\curveto(510.42547127,342.25725866)(510.53047116,342.30225861)(510.58047701,342.38226336)
}
}
{
\newrgbcolor{curcolor}{0 0 0}
\pscustom[linestyle=none,fillstyle=solid,fillcolor=curcolor]
{
}
}
{
\newrgbcolor{curcolor}{0 0 0}
\pscustom[linestyle=none,fillstyle=solid,fillcolor=curcolor]
{
\newpath
\moveto(518.74891451,348.24726336)
\lineto(519.84391451,348.24726336)
\curveto(519.94391202,348.24725267)(520.03891193,348.24225267)(520.12891451,348.23226336)
\curveto(520.21891175,348.22225269)(520.28891168,348.19225272)(520.33891451,348.14226336)
\curveto(520.39891157,348.07225284)(520.42891154,347.97725294)(520.42891451,347.85726336)
\curveto(520.43891153,347.74725317)(520.44391152,347.63225328)(520.44391451,347.51226336)
\lineto(520.44391451,346.17726336)
\lineto(520.44391451,340.79226336)
\lineto(520.44391451,338.49726336)
\lineto(520.44391451,338.07726336)
\curveto(520.45391151,337.92726299)(520.43391153,337.8122631)(520.38391451,337.73226336)
\curveto(520.33391163,337.65226326)(520.24391172,337.59726332)(520.11391451,337.56726336)
\curveto(520.05391191,337.54726337)(519.98391198,337.54226337)(519.90391451,337.55226336)
\curveto(519.83391213,337.56226335)(519.7639122,337.56726335)(519.69391451,337.56726336)
\lineto(518.97391451,337.56726336)
\curveto(518.8639131,337.56726335)(518.7639132,337.57226334)(518.67391451,337.58226336)
\curveto(518.58391338,337.59226332)(518.50891346,337.62226329)(518.44891451,337.67226336)
\curveto(518.38891358,337.72226319)(518.35391361,337.79726312)(518.34391451,337.89726336)
\lineto(518.34391451,338.22726336)
\lineto(518.34391451,339.56226336)
\lineto(518.34391451,345.18726336)
\lineto(518.34391451,347.22726336)
\curveto(518.34391362,347.35725356)(518.33891363,347.5122534)(518.32891451,347.69226336)
\curveto(518.32891364,347.87225304)(518.35391361,348.00225291)(518.40391451,348.08226336)
\curveto(518.42391354,348.12225279)(518.44891352,348.15225276)(518.47891451,348.17226336)
\lineto(518.59891451,348.23226336)
\curveto(518.61891335,348.23225268)(518.64391332,348.23225268)(518.67391451,348.23226336)
\curveto(518.70391326,348.24225267)(518.72891324,348.24725267)(518.74891451,348.24726336)
}
}
{
\newrgbcolor{curcolor}{0 0 0}
\pscustom[linestyle=none,fillstyle=solid,fillcolor=curcolor]
{
\newpath
\moveto(529.87610201,341.73726336)
\curveto(529.89609344,341.67725924)(529.90609343,341.59225932)(529.90610201,341.48226336)
\curveto(529.90609343,341.37225954)(529.89609344,341.28725963)(529.87610201,341.22726336)
\lineto(529.87610201,341.07726336)
\curveto(529.85609348,340.99725992)(529.84609349,340.91726)(529.84610201,340.83726336)
\curveto(529.85609348,340.75726016)(529.85109348,340.67726024)(529.83110201,340.59726336)
\curveto(529.81109352,340.52726039)(529.79609354,340.46226045)(529.78610201,340.40226336)
\curveto(529.77609356,340.34226057)(529.76609357,340.27726064)(529.75610201,340.20726336)
\curveto(529.71609362,340.09726082)(529.68109365,339.98226093)(529.65110201,339.86226336)
\curveto(529.62109371,339.75226116)(529.58109375,339.64726127)(529.53110201,339.54726336)
\curveto(529.32109401,339.06726185)(529.04609429,338.67726224)(528.70610201,338.37726336)
\curveto(528.36609497,338.07726284)(527.95609538,337.82726309)(527.47610201,337.62726336)
\curveto(527.35609598,337.57726334)(527.2310961,337.54226337)(527.10110201,337.52226336)
\curveto(526.98109635,337.49226342)(526.85609648,337.46226345)(526.72610201,337.43226336)
\curveto(526.67609666,337.4122635)(526.62109671,337.40226351)(526.56110201,337.40226336)
\curveto(526.50109683,337.40226351)(526.44609689,337.39726352)(526.39610201,337.38726336)
\lineto(526.29110201,337.38726336)
\curveto(526.26109707,337.37726354)(526.2310971,337.37226354)(526.20110201,337.37226336)
\curveto(526.15109718,337.36226355)(526.07109726,337.35726356)(525.96110201,337.35726336)
\curveto(525.85109748,337.34726357)(525.76609757,337.35226356)(525.70610201,337.37226336)
\lineto(525.55610201,337.37226336)
\curveto(525.50609783,337.38226353)(525.45109788,337.38726353)(525.39110201,337.38726336)
\curveto(525.34109799,337.37726354)(525.29109804,337.38226353)(525.24110201,337.40226336)
\curveto(525.20109813,337.4122635)(525.16109817,337.4172635)(525.12110201,337.41726336)
\curveto(525.09109824,337.4172635)(525.05109828,337.42226349)(525.00110201,337.43226336)
\curveto(524.90109843,337.46226345)(524.80109853,337.48726343)(524.70110201,337.50726336)
\curveto(524.60109873,337.52726339)(524.50609883,337.55726336)(524.41610201,337.59726336)
\curveto(524.29609904,337.63726328)(524.18109915,337.67726324)(524.07110201,337.71726336)
\curveto(523.97109936,337.75726316)(523.86609947,337.80726311)(523.75610201,337.86726336)
\curveto(523.40609993,338.07726284)(523.10610023,338.32226259)(522.85610201,338.60226336)
\curveto(522.60610073,338.88226203)(522.39610094,339.2172617)(522.22610201,339.60726336)
\curveto(522.17610116,339.69726122)(522.1361012,339.79226112)(522.10610201,339.89226336)
\curveto(522.08610125,339.99226092)(522.06110127,340.09726082)(522.03110201,340.20726336)
\curveto(522.01110132,340.25726066)(522.00110133,340.30226061)(522.00110201,340.34226336)
\curveto(522.00110133,340.38226053)(521.99110134,340.42726049)(521.97110201,340.47726336)
\curveto(521.95110138,340.55726036)(521.94110139,340.63726028)(521.94110201,340.71726336)
\curveto(521.94110139,340.80726011)(521.9311014,340.89226002)(521.91110201,340.97226336)
\curveto(521.90110143,341.02225989)(521.89610144,341.06725985)(521.89610201,341.10726336)
\lineto(521.89610201,341.24226336)
\curveto(521.87610146,341.30225961)(521.86610147,341.38725953)(521.86610201,341.49726336)
\curveto(521.87610146,341.60725931)(521.89110144,341.69225922)(521.91110201,341.75226336)
\lineto(521.91110201,341.85726336)
\curveto(521.92110141,341.90725901)(521.92110141,341.95725896)(521.91110201,342.00726336)
\curveto(521.91110142,342.06725885)(521.92110141,342.12225879)(521.94110201,342.17226336)
\curveto(521.95110138,342.22225869)(521.95610138,342.26725865)(521.95610201,342.30726336)
\curveto(521.95610138,342.35725856)(521.96610137,342.40725851)(521.98610201,342.45726336)
\curveto(522.02610131,342.58725833)(522.06110127,342.7122582)(522.09110201,342.83226336)
\curveto(522.12110121,342.96225795)(522.16110117,343.08725783)(522.21110201,343.20726336)
\curveto(522.39110094,343.6172573)(522.60610073,343.95725696)(522.85610201,344.22726336)
\curveto(523.10610023,344.50725641)(523.41109992,344.76225615)(523.77110201,344.99226336)
\curveto(523.87109946,345.04225587)(523.97609936,345.08725583)(524.08610201,345.12726336)
\curveto(524.19609914,345.16725575)(524.30609903,345.2122557)(524.41610201,345.26226336)
\curveto(524.54609879,345.3122556)(524.68109865,345.34725557)(524.82110201,345.36726336)
\curveto(524.96109837,345.38725553)(525.10609823,345.4172555)(525.25610201,345.45726336)
\curveto(525.336098,345.46725545)(525.41109792,345.47225544)(525.48110201,345.47226336)
\curveto(525.55109778,345.47225544)(525.62109771,345.47725544)(525.69110201,345.48726336)
\curveto(526.27109706,345.49725542)(526.77109656,345.43725548)(527.19110201,345.30726336)
\curveto(527.62109571,345.17725574)(528.00109533,344.99725592)(528.33110201,344.76726336)
\curveto(528.44109489,344.68725623)(528.55109478,344.59725632)(528.66110201,344.49726336)
\curveto(528.78109455,344.40725651)(528.88109445,344.30725661)(528.96110201,344.19726336)
\curveto(529.04109429,344.09725682)(529.11109422,343.99725692)(529.17110201,343.89726336)
\curveto(529.24109409,343.79725712)(529.31109402,343.69225722)(529.38110201,343.58226336)
\curveto(529.45109388,343.47225744)(529.50609383,343.35225756)(529.54610201,343.22226336)
\curveto(529.58609375,343.10225781)(529.6310937,342.97225794)(529.68110201,342.83226336)
\curveto(529.71109362,342.75225816)(529.7360936,342.66725825)(529.75610201,342.57726336)
\lineto(529.81610201,342.30726336)
\curveto(529.82609351,342.26725865)(529.8310935,342.22725869)(529.83110201,342.18726336)
\curveto(529.8310935,342.14725877)(529.8360935,342.10725881)(529.84610201,342.06726336)
\curveto(529.86609347,342.0172589)(529.87109346,341.96225895)(529.86110201,341.90226336)
\curveto(529.85109348,341.84225907)(529.85609348,341.78725913)(529.87610201,341.73726336)
\moveto(527.77610201,341.19726336)
\curveto(527.78609555,341.24725967)(527.79109554,341.3172596)(527.79110201,341.40726336)
\curveto(527.79109554,341.50725941)(527.78609555,341.58225933)(527.77610201,341.63226336)
\lineto(527.77610201,341.75226336)
\curveto(527.75609558,341.80225911)(527.74609559,341.85725906)(527.74610201,341.91726336)
\curveto(527.74609559,341.97725894)(527.74109559,342.03225888)(527.73110201,342.08226336)
\curveto(527.7310956,342.12225879)(527.72609561,342.15225876)(527.71610201,342.17226336)
\lineto(527.65610201,342.41226336)
\curveto(527.64609569,342.50225841)(527.62609571,342.58725833)(527.59610201,342.66726336)
\curveto(527.48609585,342.92725799)(527.35609598,343.14725777)(527.20610201,343.32726336)
\curveto(527.05609628,343.5172574)(526.85609648,343.66725725)(526.60610201,343.77726336)
\curveto(526.54609679,343.79725712)(526.48609685,343.8122571)(526.42610201,343.82226336)
\curveto(526.36609697,343.84225707)(526.30109703,343.86225705)(526.23110201,343.88226336)
\curveto(526.15109718,343.90225701)(526.06609727,343.90725701)(525.97610201,343.89726336)
\lineto(525.70610201,343.89726336)
\curveto(525.67609766,343.87725704)(525.64109769,343.86725705)(525.60110201,343.86726336)
\curveto(525.56109777,343.87725704)(525.52609781,343.87725704)(525.49610201,343.86726336)
\lineto(525.28610201,343.80726336)
\curveto(525.22609811,343.79725712)(525.17109816,343.77725714)(525.12110201,343.74726336)
\curveto(524.87109846,343.63725728)(524.66609867,343.47725744)(524.50610201,343.26726336)
\curveto(524.35609898,343.06725785)(524.2360991,342.83225808)(524.14610201,342.56226336)
\curveto(524.11609922,342.46225845)(524.09109924,342.35725856)(524.07110201,342.24726336)
\curveto(524.06109927,342.13725878)(524.04609929,342.02725889)(524.02610201,341.91726336)
\curveto(524.01609932,341.86725905)(524.01109932,341.8172591)(524.01110201,341.76726336)
\lineto(524.01110201,341.61726336)
\curveto(523.99109934,341.54725937)(523.98109935,341.44225947)(523.98110201,341.30226336)
\curveto(523.99109934,341.16225975)(524.00609933,341.05725986)(524.02610201,340.98726336)
\lineto(524.02610201,340.85226336)
\curveto(524.04609929,340.77226014)(524.06109927,340.69226022)(524.07110201,340.61226336)
\curveto(524.08109925,340.54226037)(524.09609924,340.46726045)(524.11610201,340.38726336)
\curveto(524.21609912,340.08726083)(524.32109901,339.84226107)(524.43110201,339.65226336)
\curveto(524.55109878,339.47226144)(524.7360986,339.30726161)(524.98610201,339.15726336)
\curveto(525.05609828,339.10726181)(525.1310982,339.06726185)(525.21110201,339.03726336)
\curveto(525.30109803,339.00726191)(525.39109794,338.98226193)(525.48110201,338.96226336)
\curveto(525.52109781,338.95226196)(525.55609778,338.94726197)(525.58610201,338.94726336)
\curveto(525.61609772,338.95726196)(525.65109768,338.95726196)(525.69110201,338.94726336)
\lineto(525.81110201,338.91726336)
\curveto(525.86109747,338.917262)(525.90609743,338.92226199)(525.94610201,338.93226336)
\lineto(526.06610201,338.93226336)
\curveto(526.14609719,338.95226196)(526.22609711,338.96726195)(526.30610201,338.97726336)
\curveto(526.38609695,338.98726193)(526.46109687,339.00726191)(526.53110201,339.03726336)
\curveto(526.79109654,339.13726178)(527.00109633,339.27226164)(527.16110201,339.44226336)
\curveto(527.32109601,339.6122613)(527.45609588,339.82226109)(527.56610201,340.07226336)
\curveto(527.60609573,340.17226074)(527.6360957,340.27226064)(527.65610201,340.37226336)
\curveto(527.67609566,340.47226044)(527.70109563,340.57726034)(527.73110201,340.68726336)
\curveto(527.74109559,340.72726019)(527.74609559,340.76226015)(527.74610201,340.79226336)
\curveto(527.74609559,340.83226008)(527.75109558,340.87226004)(527.76110201,340.91226336)
\lineto(527.76110201,341.04726336)
\curveto(527.76109557,341.09725982)(527.76609557,341.14725977)(527.77610201,341.19726336)
}
}
{
\newrgbcolor{curcolor}{0 0 0}
\pscustom[linestyle=none,fillstyle=solid,fillcolor=curcolor]
{
\newpath
\moveto(534.24602388,345.50226336)
\curveto(534.99601938,345.52225539)(535.64601873,345.43725548)(536.19602388,345.24726336)
\curveto(536.75601762,345.06725585)(537.1810172,344.75225616)(537.47102388,344.30226336)
\curveto(537.54101684,344.19225672)(537.60101678,344.07725684)(537.65102388,343.95726336)
\curveto(537.71101667,343.84725707)(537.76101662,343.72225719)(537.80102388,343.58226336)
\curveto(537.82101656,343.52225739)(537.83101655,343.45725746)(537.83102388,343.38726336)
\curveto(537.83101655,343.3172576)(537.82101656,343.25725766)(537.80102388,343.20726336)
\curveto(537.76101662,343.14725777)(537.70601667,343.10725781)(537.63602388,343.08726336)
\curveto(537.58601679,343.06725785)(537.52601685,343.05725786)(537.45602388,343.05726336)
\lineto(537.24602388,343.05726336)
\lineto(536.58602388,343.05726336)
\curveto(536.51601786,343.05725786)(536.44601793,343.05225786)(536.37602388,343.04226336)
\curveto(536.30601807,343.04225787)(536.24101814,343.05225786)(536.18102388,343.07226336)
\curveto(536.0810183,343.09225782)(536.00601837,343.13225778)(535.95602388,343.19226336)
\curveto(535.90601847,343.25225766)(535.86101852,343.3122576)(535.82102388,343.37226336)
\lineto(535.70102388,343.58226336)
\curveto(535.67101871,343.66225725)(535.62101876,343.72725719)(535.55102388,343.77726336)
\curveto(535.45101893,343.85725706)(535.35101903,343.917257)(535.25102388,343.95726336)
\curveto(535.16101922,343.99725692)(535.04601933,344.03225688)(534.90602388,344.06226336)
\curveto(534.83601954,344.08225683)(534.73101965,344.09725682)(534.59102388,344.10726336)
\curveto(534.46101992,344.1172568)(534.36102002,344.1122568)(534.29102388,344.09226336)
\lineto(534.18602388,344.09226336)
\lineto(534.03602388,344.06226336)
\curveto(533.99602038,344.06225685)(533.95102043,344.05725686)(533.90102388,344.04726336)
\curveto(533.73102065,343.99725692)(533.59102079,343.92725699)(533.48102388,343.83726336)
\curveto(533.381021,343.75725716)(533.31102107,343.63225728)(533.27102388,343.46226336)
\curveto(533.25102113,343.39225752)(533.25102113,343.32725759)(533.27102388,343.26726336)
\curveto(533.29102109,343.20725771)(533.31102107,343.15725776)(533.33102388,343.11726336)
\curveto(533.40102098,342.99725792)(533.4810209,342.90225801)(533.57102388,342.83226336)
\curveto(533.67102071,342.76225815)(533.78602059,342.70225821)(533.91602388,342.65226336)
\curveto(534.10602027,342.57225834)(534.31102007,342.50225841)(534.53102388,342.44226336)
\lineto(535.22102388,342.29226336)
\curveto(535.46101892,342.25225866)(535.69101869,342.20225871)(535.91102388,342.14226336)
\curveto(536.14101824,342.09225882)(536.35601802,342.02725889)(536.55602388,341.94726336)
\curveto(536.64601773,341.90725901)(536.73101765,341.87225904)(536.81102388,341.84226336)
\curveto(536.90101748,341.82225909)(536.98601739,341.78725913)(537.06602388,341.73726336)
\curveto(537.25601712,341.6172593)(537.42601695,341.48725943)(537.57602388,341.34726336)
\curveto(537.73601664,341.20725971)(537.86101652,341.03225988)(537.95102388,340.82226336)
\curveto(537.9810164,340.75226016)(538.00601637,340.68226023)(538.02602388,340.61226336)
\curveto(538.04601633,340.54226037)(538.06601631,340.46726045)(538.08602388,340.38726336)
\curveto(538.09601628,340.32726059)(538.10101628,340.23226068)(538.10102388,340.10226336)
\curveto(538.11101627,339.98226093)(538.11101627,339.88726103)(538.10102388,339.81726336)
\lineto(538.10102388,339.74226336)
\curveto(538.0810163,339.68226123)(538.06601631,339.62226129)(538.05602388,339.56226336)
\curveto(538.05601632,339.5122614)(538.05101633,339.46226145)(538.04102388,339.41226336)
\curveto(537.97101641,339.1122618)(537.86101652,338.84726207)(537.71102388,338.61726336)
\curveto(537.55101683,338.37726254)(537.35601702,338.18226273)(537.12602388,338.03226336)
\curveto(536.89601748,337.88226303)(536.63601774,337.75226316)(536.34602388,337.64226336)
\curveto(536.23601814,337.59226332)(536.11601826,337.55726336)(535.98602388,337.53726336)
\curveto(535.86601851,337.5172634)(535.74601863,337.49226342)(535.62602388,337.46226336)
\curveto(535.53601884,337.44226347)(535.44101894,337.43226348)(535.34102388,337.43226336)
\curveto(535.25101913,337.42226349)(535.16101922,337.40726351)(535.07102388,337.38726336)
\lineto(534.80102388,337.38726336)
\curveto(534.74101964,337.36726355)(534.63601974,337.35726356)(534.48602388,337.35726336)
\curveto(534.34602003,337.35726356)(534.24602013,337.36726355)(534.18602388,337.38726336)
\curveto(534.15602022,337.38726353)(534.12102026,337.39226352)(534.08102388,337.40226336)
\lineto(533.97602388,337.40226336)
\curveto(533.85602052,337.42226349)(533.73602064,337.43726348)(533.61602388,337.44726336)
\curveto(533.49602088,337.45726346)(533.381021,337.47726344)(533.27102388,337.50726336)
\curveto(532.8810215,337.6172633)(532.53602184,337.74226317)(532.23602388,337.88226336)
\curveto(531.93602244,338.03226288)(531.6810227,338.25226266)(531.47102388,338.54226336)
\curveto(531.33102305,338.73226218)(531.21102317,338.95226196)(531.11102388,339.20226336)
\curveto(531.09102329,339.26226165)(531.07102331,339.34226157)(531.05102388,339.44226336)
\curveto(531.03102335,339.49226142)(531.01602336,339.56226135)(531.00602388,339.65226336)
\curveto(530.99602338,339.74226117)(531.00102338,339.8172611)(531.02102388,339.87726336)
\curveto(531.05102333,339.94726097)(531.10102328,339.99726092)(531.17102388,340.02726336)
\curveto(531.22102316,340.04726087)(531.2810231,340.05726086)(531.35102388,340.05726336)
\lineto(531.57602388,340.05726336)
\lineto(532.28102388,340.05726336)
\lineto(532.52102388,340.05726336)
\curveto(532.60102178,340.05726086)(532.67102171,340.04726087)(532.73102388,340.02726336)
\curveto(532.84102154,339.98726093)(532.91102147,339.92226099)(532.94102388,339.83226336)
\curveto(532.9810214,339.74226117)(533.02602135,339.64726127)(533.07602388,339.54726336)
\curveto(533.09602128,339.49726142)(533.13102125,339.43226148)(533.18102388,339.35226336)
\curveto(533.24102114,339.27226164)(533.29102109,339.22226169)(533.33102388,339.20226336)
\curveto(533.45102093,339.10226181)(533.56602081,339.02226189)(533.67602388,338.96226336)
\curveto(533.78602059,338.912262)(533.92602045,338.86226205)(534.09602388,338.81226336)
\curveto(534.14602023,338.79226212)(534.19602018,338.78226213)(534.24602388,338.78226336)
\curveto(534.29602008,338.79226212)(534.34602003,338.79226212)(534.39602388,338.78226336)
\curveto(534.4760199,338.76226215)(534.56101982,338.75226216)(534.65102388,338.75226336)
\curveto(534.75101963,338.76226215)(534.83601954,338.77726214)(534.90602388,338.79726336)
\curveto(534.95601942,338.80726211)(535.00101938,338.8122621)(535.04102388,338.81226336)
\curveto(535.09101929,338.8122621)(535.14101924,338.82226209)(535.19102388,338.84226336)
\curveto(535.33101905,338.89226202)(535.45601892,338.95226196)(535.56602388,339.02226336)
\curveto(535.68601869,339.09226182)(535.7810186,339.18226173)(535.85102388,339.29226336)
\curveto(535.90101848,339.37226154)(535.94101844,339.49726142)(535.97102388,339.66726336)
\curveto(535.99101839,339.73726118)(535.99101839,339.80226111)(535.97102388,339.86226336)
\curveto(535.95101843,339.92226099)(535.93101845,339.97226094)(535.91102388,340.01226336)
\curveto(535.84101854,340.15226076)(535.75101863,340.25726066)(535.64102388,340.32726336)
\curveto(535.54101884,340.39726052)(535.42101896,340.46226045)(535.28102388,340.52226336)
\curveto(535.09101929,340.60226031)(534.89101949,340.66726025)(534.68102388,340.71726336)
\curveto(534.47101991,340.76726015)(534.26102012,340.82226009)(534.05102388,340.88226336)
\curveto(533.97102041,340.90226001)(533.88602049,340.91726)(533.79602388,340.92726336)
\curveto(533.71602066,340.93725998)(533.63602074,340.95225996)(533.55602388,340.97226336)
\curveto(533.23602114,341.06225985)(532.93102145,341.14725977)(532.64102388,341.22726336)
\curveto(532.35102203,341.3172596)(532.08602229,341.44725947)(531.84602388,341.61726336)
\curveto(531.56602281,341.8172591)(531.36102302,342.08725883)(531.23102388,342.42726336)
\curveto(531.21102317,342.49725842)(531.19102319,342.59225832)(531.17102388,342.71226336)
\curveto(531.15102323,342.78225813)(531.13602324,342.86725805)(531.12602388,342.96726336)
\curveto(531.11602326,343.06725785)(531.12102326,343.15725776)(531.14102388,343.23726336)
\curveto(531.16102322,343.28725763)(531.16602321,343.32725759)(531.15602388,343.35726336)
\curveto(531.14602323,343.39725752)(531.15102323,343.44225747)(531.17102388,343.49226336)
\curveto(531.19102319,343.60225731)(531.21102317,343.70225721)(531.23102388,343.79226336)
\curveto(531.26102312,343.89225702)(531.29602308,343.98725693)(531.33602388,344.07726336)
\curveto(531.46602291,344.36725655)(531.64602273,344.60225631)(531.87602388,344.78226336)
\curveto(532.10602227,344.96225595)(532.36602201,345.10725581)(532.65602388,345.21726336)
\curveto(532.76602161,345.26725565)(532.8810215,345.30225561)(533.00102388,345.32226336)
\curveto(533.12102126,345.35225556)(533.24602113,345.38225553)(533.37602388,345.41226336)
\curveto(533.43602094,345.43225548)(533.49602088,345.44225547)(533.55602388,345.44226336)
\lineto(533.73602388,345.47226336)
\curveto(533.81602056,345.48225543)(533.90102048,345.48725543)(533.99102388,345.48726336)
\curveto(534.0810203,345.48725543)(534.16602021,345.49225542)(534.24602388,345.50226336)
}
}
{
\newrgbcolor{curcolor}{0 0 0}
\pscustom[linestyle=none,fillstyle=solid,fillcolor=curcolor]
{
}
}
{
\newrgbcolor{curcolor}{0 0 0}
\pscustom[linestyle=none,fillstyle=solid,fillcolor=curcolor]
{
\newpath
\moveto(547.06282076,345.50226336)
\curveto(547.8728156,345.52225539)(548.54781492,345.40225551)(549.08782076,345.14226336)
\curveto(549.63781383,344.88225603)(550.0728134,344.5122564)(550.39282076,344.03226336)
\curveto(550.55281292,343.79225712)(550.6728128,343.5172574)(550.75282076,343.20726336)
\curveto(550.7728127,343.15725776)(550.78781268,343.09225782)(550.79782076,343.01226336)
\curveto(550.81781265,342.93225798)(550.81781265,342.86225805)(550.79782076,342.80226336)
\curveto(550.75781271,342.69225822)(550.68781278,342.62725829)(550.58782076,342.60726336)
\curveto(550.48781298,342.59725832)(550.3678131,342.59225832)(550.22782076,342.59226336)
\lineto(549.44782076,342.59226336)
\lineto(549.16282076,342.59226336)
\curveto(549.0728144,342.59225832)(548.99781447,342.6122583)(548.93782076,342.65226336)
\curveto(548.85781461,342.69225822)(548.80281467,342.75225816)(548.77282076,342.83226336)
\curveto(548.74281473,342.92225799)(548.70281477,343.0122579)(548.65282076,343.10226336)
\curveto(548.59281488,343.2122577)(548.52781494,343.3122576)(548.45782076,343.40226336)
\curveto(548.38781508,343.49225742)(548.30781516,343.57225734)(548.21782076,343.64226336)
\curveto(548.07781539,343.73225718)(547.92281555,343.80225711)(547.75282076,343.85226336)
\curveto(547.69281578,343.87225704)(547.63281584,343.88225703)(547.57282076,343.88226336)
\curveto(547.51281596,343.88225703)(547.45781601,343.89225702)(547.40782076,343.91226336)
\lineto(547.25782076,343.91226336)
\curveto(547.05781641,343.912257)(546.89781657,343.89225702)(546.77782076,343.85226336)
\curveto(546.48781698,343.76225715)(546.25281722,343.62225729)(546.07282076,343.43226336)
\curveto(545.89281758,343.25225766)(545.74781772,343.03225788)(545.63782076,342.77226336)
\curveto(545.58781788,342.66225825)(545.54781792,342.54225837)(545.51782076,342.41226336)
\curveto(545.49781797,342.29225862)(545.472818,342.16225875)(545.44282076,342.02226336)
\curveto(545.43281804,341.98225893)(545.42781804,341.94225897)(545.42782076,341.90226336)
\curveto(545.42781804,341.86225905)(545.42281805,341.82225909)(545.41282076,341.78226336)
\curveto(545.39281808,341.68225923)(545.38281809,341.54225937)(545.38282076,341.36226336)
\curveto(545.39281808,341.18225973)(545.40781806,341.04225987)(545.42782076,340.94226336)
\curveto(545.42781804,340.86226005)(545.43281804,340.80726011)(545.44282076,340.77726336)
\curveto(545.46281801,340.70726021)(545.472818,340.63726028)(545.47282076,340.56726336)
\curveto(545.48281799,340.49726042)(545.49781797,340.42726049)(545.51782076,340.35726336)
\curveto(545.59781787,340.12726079)(545.69281778,339.917261)(545.80282076,339.72726336)
\curveto(545.91281756,339.53726138)(546.05281742,339.37726154)(546.22282076,339.24726336)
\curveto(546.26281721,339.2172617)(546.32281715,339.18226173)(546.40282076,339.14226336)
\curveto(546.51281696,339.07226184)(546.62281685,339.02726189)(546.73282076,339.00726336)
\curveto(546.85281662,338.98726193)(546.99781647,338.96726195)(547.16782076,338.94726336)
\lineto(547.25782076,338.94726336)
\curveto(547.29781617,338.94726197)(547.32781614,338.95226196)(547.34782076,338.96226336)
\lineto(547.48282076,338.96226336)
\curveto(547.55281592,338.98226193)(547.61781585,338.99726192)(547.67782076,339.00726336)
\curveto(547.74781572,339.02726189)(547.81281566,339.04726187)(547.87282076,339.06726336)
\curveto(548.1728153,339.19726172)(548.40281507,339.38726153)(548.56282076,339.63726336)
\curveto(548.60281487,339.68726123)(548.63781483,339.74226117)(548.66782076,339.80226336)
\curveto(548.69781477,339.87226104)(548.72281475,339.93226098)(548.74282076,339.98226336)
\curveto(548.78281469,340.09226082)(548.81781465,340.18726073)(548.84782076,340.26726336)
\curveto(548.87781459,340.35726056)(548.94781452,340.42726049)(549.05782076,340.47726336)
\curveto(549.14781432,340.5172604)(549.29281418,340.53226038)(549.49282076,340.52226336)
\lineto(549.98782076,340.52226336)
\lineto(550.19782076,340.52226336)
\curveto(550.27781319,340.53226038)(550.34281313,340.52726039)(550.39282076,340.50726336)
\lineto(550.51282076,340.50726336)
\lineto(550.63282076,340.47726336)
\curveto(550.6728128,340.47726044)(550.70281277,340.46726045)(550.72282076,340.44726336)
\curveto(550.7728127,340.40726051)(550.80281267,340.34726057)(550.81282076,340.26726336)
\curveto(550.83281264,340.19726072)(550.83281264,340.12226079)(550.81282076,340.04226336)
\curveto(550.72281275,339.7122612)(550.61281286,339.4172615)(550.48282076,339.15726336)
\curveto(550.0728134,338.38726253)(549.41781405,337.85226306)(548.51782076,337.55226336)
\curveto(548.41781505,337.52226339)(548.31281516,337.50226341)(548.20282076,337.49226336)
\curveto(548.09281538,337.47226344)(547.98281549,337.44726347)(547.87282076,337.41726336)
\curveto(547.81281566,337.40726351)(547.75281572,337.40226351)(547.69282076,337.40226336)
\curveto(547.63281584,337.40226351)(547.5728159,337.39726352)(547.51282076,337.38726336)
\lineto(547.34782076,337.38726336)
\curveto(547.29781617,337.36726355)(547.22281625,337.36226355)(547.12282076,337.37226336)
\curveto(547.02281645,337.37226354)(546.94781652,337.37726354)(546.89782076,337.38726336)
\curveto(546.81781665,337.40726351)(546.74281673,337.4172635)(546.67282076,337.41726336)
\curveto(546.61281686,337.40726351)(546.54781692,337.4122635)(546.47782076,337.43226336)
\lineto(546.32782076,337.46226336)
\curveto(546.27781719,337.46226345)(546.22781724,337.46726345)(546.17782076,337.47726336)
\curveto(546.0678174,337.50726341)(545.96281751,337.53726338)(545.86282076,337.56726336)
\curveto(545.76281771,337.59726332)(545.6678178,337.63226328)(545.57782076,337.67226336)
\curveto(545.10781836,337.87226304)(544.71281876,338.12726279)(544.39282076,338.43726336)
\curveto(544.0728194,338.75726216)(543.81281966,339.15226176)(543.61282076,339.62226336)
\curveto(543.56281991,339.7122612)(543.52281995,339.80726111)(543.49282076,339.90726336)
\lineto(543.40282076,340.23726336)
\curveto(543.39282008,340.27726064)(543.38782008,340.3122606)(543.38782076,340.34226336)
\curveto(543.38782008,340.38226053)(543.37782009,340.42726049)(543.35782076,340.47726336)
\curveto(543.33782013,340.54726037)(543.32782014,340.6172603)(543.32782076,340.68726336)
\curveto(543.32782014,340.76726015)(543.31782015,340.84226007)(543.29782076,340.91226336)
\lineto(543.29782076,341.16726336)
\curveto(543.27782019,341.2172597)(543.2678202,341.27225964)(543.26782076,341.33226336)
\curveto(543.2678202,341.40225951)(543.27782019,341.46225945)(543.29782076,341.51226336)
\curveto(543.30782016,341.56225935)(543.30782016,341.60725931)(543.29782076,341.64726336)
\curveto(543.28782018,341.68725923)(543.28782018,341.72725919)(543.29782076,341.76726336)
\curveto(543.31782015,341.83725908)(543.32282015,341.90225901)(543.31282076,341.96226336)
\curveto(543.31282016,342.02225889)(543.32282015,342.08225883)(543.34282076,342.14226336)
\curveto(543.39282008,342.32225859)(543.43282004,342.49225842)(543.46282076,342.65226336)
\curveto(543.49281998,342.82225809)(543.53781993,342.98725793)(543.59782076,343.14726336)
\curveto(543.81781965,343.65725726)(544.09281938,344.08225683)(544.42282076,344.42226336)
\curveto(544.76281871,344.76225615)(545.19281828,345.03725588)(545.71282076,345.24726336)
\curveto(545.85281762,345.30725561)(545.99781747,345.34725557)(546.14782076,345.36726336)
\curveto(546.29781717,345.39725552)(546.45281702,345.43225548)(546.61282076,345.47226336)
\curveto(546.69281678,345.48225543)(546.7678167,345.48725543)(546.83782076,345.48726336)
\curveto(546.90781656,345.48725543)(546.98281649,345.49225542)(547.06282076,345.50226336)
}
}
{
\newrgbcolor{curcolor}{0 0 0}
\pscustom[linestyle=none,fillstyle=solid,fillcolor=curcolor]
{
\newpath
\moveto(559.15610201,338.15226336)
\curveto(559.17609416,338.04226287)(559.18609415,337.93226298)(559.18610201,337.82226336)
\curveto(559.19609414,337.7122632)(559.14609419,337.63726328)(559.03610201,337.59726336)
\curveto(558.97609436,337.56726335)(558.90609443,337.55226336)(558.82610201,337.55226336)
\lineto(558.58610201,337.55226336)
\lineto(557.77610201,337.55226336)
\lineto(557.50610201,337.55226336)
\curveto(557.42609591,337.56226335)(557.36109597,337.58726333)(557.31110201,337.62726336)
\curveto(557.24109609,337.66726325)(557.18609615,337.72226319)(557.14610201,337.79226336)
\curveto(557.11609622,337.87226304)(557.07109626,337.93726298)(557.01110201,337.98726336)
\curveto(556.99109634,338.00726291)(556.96609637,338.02226289)(556.93610201,338.03226336)
\curveto(556.90609643,338.05226286)(556.86609647,338.05726286)(556.81610201,338.04726336)
\curveto(556.76609657,338.02726289)(556.71609662,338.00226291)(556.66610201,337.97226336)
\curveto(556.62609671,337.94226297)(556.58109675,337.917263)(556.53110201,337.89726336)
\curveto(556.48109685,337.85726306)(556.42609691,337.82226309)(556.36610201,337.79226336)
\lineto(556.18610201,337.70226336)
\curveto(556.05609728,337.64226327)(555.92109741,337.59226332)(555.78110201,337.55226336)
\curveto(555.64109769,337.52226339)(555.49609784,337.48726343)(555.34610201,337.44726336)
\curveto(555.27609806,337.42726349)(555.20609813,337.4172635)(555.13610201,337.41726336)
\curveto(555.07609826,337.40726351)(555.01109832,337.39726352)(554.94110201,337.38726336)
\lineto(554.85110201,337.38726336)
\curveto(554.82109851,337.37726354)(554.79109854,337.37226354)(554.76110201,337.37226336)
\lineto(554.59610201,337.37226336)
\curveto(554.49609884,337.35226356)(554.39609894,337.35226356)(554.29610201,337.37226336)
\lineto(554.16110201,337.37226336)
\curveto(554.09109924,337.39226352)(554.02109931,337.40226351)(553.95110201,337.40226336)
\curveto(553.89109944,337.39226352)(553.8310995,337.39726352)(553.77110201,337.41726336)
\curveto(553.67109966,337.43726348)(553.57609976,337.45726346)(553.48610201,337.47726336)
\curveto(553.39609994,337.48726343)(553.31110002,337.5122634)(553.23110201,337.55226336)
\curveto(552.94110039,337.66226325)(552.69110064,337.80226311)(552.48110201,337.97226336)
\curveto(552.28110105,338.15226276)(552.12110121,338.38726253)(552.00110201,338.67726336)
\curveto(551.97110136,338.74726217)(551.94110139,338.82226209)(551.91110201,338.90226336)
\curveto(551.89110144,338.98226193)(551.87110146,339.06726185)(551.85110201,339.15726336)
\curveto(551.8311015,339.20726171)(551.82110151,339.25726166)(551.82110201,339.30726336)
\curveto(551.8311015,339.35726156)(551.8311015,339.40726151)(551.82110201,339.45726336)
\curveto(551.81110152,339.48726143)(551.80110153,339.54726137)(551.79110201,339.63726336)
\curveto(551.79110154,339.73726118)(551.79610154,339.80726111)(551.80610201,339.84726336)
\curveto(551.82610151,339.94726097)(551.8361015,340.03226088)(551.83610201,340.10226336)
\lineto(551.92610201,340.43226336)
\curveto(551.95610138,340.55226036)(551.99610134,340.65726026)(552.04610201,340.74726336)
\curveto(552.21610112,341.03725988)(552.41110092,341.25725966)(552.63110201,341.40726336)
\curveto(552.85110048,341.55725936)(553.1311002,341.68725923)(553.47110201,341.79726336)
\curveto(553.60109973,341.84725907)(553.7360996,341.88225903)(553.87610201,341.90226336)
\curveto(554.01609932,341.92225899)(554.15609918,341.94725897)(554.29610201,341.97726336)
\curveto(554.37609896,341.99725892)(554.46109887,342.00725891)(554.55110201,342.00726336)
\curveto(554.64109869,342.0172589)(554.7310986,342.03225888)(554.82110201,342.05226336)
\curveto(554.89109844,342.07225884)(554.96109837,342.07725884)(555.03110201,342.06726336)
\curveto(555.10109823,342.06725885)(555.17609816,342.07725884)(555.25610201,342.09726336)
\curveto(555.32609801,342.1172588)(555.39609794,342.12725879)(555.46610201,342.12726336)
\curveto(555.5360978,342.12725879)(555.61109772,342.13725878)(555.69110201,342.15726336)
\curveto(555.90109743,342.20725871)(556.09109724,342.24725867)(556.26110201,342.27726336)
\curveto(556.44109689,342.3172586)(556.60109673,342.40725851)(556.74110201,342.54726336)
\curveto(556.8310965,342.63725828)(556.89109644,342.73725818)(556.92110201,342.84726336)
\curveto(556.9310964,342.87725804)(556.9310964,342.90225801)(556.92110201,342.92226336)
\curveto(556.92109641,342.94225797)(556.92609641,342.96225795)(556.93610201,342.98226336)
\curveto(556.94609639,343.00225791)(556.95109638,343.03225788)(556.95110201,343.07226336)
\lineto(556.95110201,343.16226336)
\lineto(556.92110201,343.28226336)
\curveto(556.92109641,343.32225759)(556.91609642,343.35725756)(556.90610201,343.38726336)
\curveto(556.80609653,343.68725723)(556.59609674,343.89225702)(556.27610201,344.00226336)
\curveto(556.18609715,344.03225688)(556.07609726,344.05225686)(555.94610201,344.06226336)
\curveto(555.82609751,344.08225683)(555.70109763,344.08725683)(555.57110201,344.07726336)
\curveto(555.44109789,344.07725684)(555.31609802,344.06725685)(555.19610201,344.04726336)
\curveto(555.07609826,344.02725689)(554.97109836,344.00225691)(554.88110201,343.97226336)
\curveto(554.82109851,343.95225696)(554.76109857,343.92225699)(554.70110201,343.88226336)
\curveto(554.65109868,343.85225706)(554.60109873,343.8172571)(554.55110201,343.77726336)
\curveto(554.50109883,343.73725718)(554.44609889,343.68225723)(554.38610201,343.61226336)
\curveto(554.336099,343.54225737)(554.30109903,343.47725744)(554.28110201,343.41726336)
\curveto(554.2310991,343.3172576)(554.18609915,343.22225769)(554.14610201,343.13226336)
\curveto(554.11609922,343.04225787)(554.04609929,342.98225793)(553.93610201,342.95226336)
\curveto(553.85609948,342.93225798)(553.77109956,342.92225799)(553.68110201,342.92226336)
\lineto(553.41110201,342.92226336)
\lineto(552.84110201,342.92226336)
\curveto(552.79110054,342.92225799)(552.74110059,342.917258)(552.69110201,342.90726336)
\curveto(552.64110069,342.90725801)(552.59610074,342.912258)(552.55610201,342.92226336)
\lineto(552.42110201,342.92226336)
\curveto(552.40110093,342.93225798)(552.37610096,342.93725798)(552.34610201,342.93726336)
\curveto(552.31610102,342.93725798)(552.29110104,342.94725797)(552.27110201,342.96726336)
\curveto(552.19110114,342.98725793)(552.1361012,343.05225786)(552.10610201,343.16226336)
\curveto(552.09610124,343.2122577)(552.09610124,343.26225765)(552.10610201,343.31226336)
\curveto(552.11610122,343.36225755)(552.12610121,343.40725751)(552.13610201,343.44726336)
\curveto(552.16610117,343.55725736)(552.19610114,343.65725726)(552.22610201,343.74726336)
\curveto(552.26610107,343.84725707)(552.31110102,343.93725698)(552.36110201,344.01726336)
\lineto(552.45110201,344.16726336)
\lineto(552.54110201,344.31726336)
\curveto(552.62110071,344.42725649)(552.72110061,344.53225638)(552.84110201,344.63226336)
\curveto(552.86110047,344.64225627)(552.89110044,344.66725625)(552.93110201,344.70726336)
\curveto(552.98110035,344.74725617)(553.02610031,344.78225613)(553.06610201,344.81226336)
\curveto(553.10610023,344.84225607)(553.15110018,344.87225604)(553.20110201,344.90226336)
\curveto(553.37109996,345.0122559)(553.55109978,345.09725582)(553.74110201,345.15726336)
\curveto(553.9310994,345.22725569)(554.12609921,345.29225562)(554.32610201,345.35226336)
\curveto(554.44609889,345.38225553)(554.57109876,345.40225551)(554.70110201,345.41226336)
\curveto(554.8310985,345.42225549)(554.96109837,345.44225547)(555.09110201,345.47226336)
\curveto(555.1310982,345.48225543)(555.19109814,345.48225543)(555.27110201,345.47226336)
\curveto(555.36109797,345.46225545)(555.41609792,345.46725545)(555.43610201,345.48726336)
\curveto(555.84609749,345.49725542)(556.2360971,345.48225543)(556.60610201,345.44226336)
\curveto(556.98609635,345.40225551)(557.32609601,345.32725559)(557.62610201,345.21726336)
\curveto(557.9360954,345.10725581)(558.20109513,344.95725596)(558.42110201,344.76726336)
\curveto(558.64109469,344.58725633)(558.81109452,344.35225656)(558.93110201,344.06226336)
\curveto(559.00109433,343.89225702)(559.04109429,343.69725722)(559.05110201,343.47726336)
\curveto(559.06109427,343.25725766)(559.06609427,343.03225788)(559.06610201,342.80226336)
\lineto(559.06610201,339.45726336)
\lineto(559.06610201,338.87226336)
\curveto(559.06609427,338.68226223)(559.08609425,338.50726241)(559.12610201,338.34726336)
\curveto(559.1360942,338.3172626)(559.14109419,338.28226263)(559.14110201,338.24226336)
\curveto(559.14109419,338.2122627)(559.14609419,338.18226273)(559.15610201,338.15226336)
\moveto(556.95110201,340.46226336)
\curveto(556.96109637,340.5122604)(556.96609637,340.56726035)(556.96610201,340.62726336)
\curveto(556.96609637,340.69726022)(556.96109637,340.75726016)(556.95110201,340.80726336)
\curveto(556.9310964,340.86726005)(556.92109641,340.92225999)(556.92110201,340.97226336)
\curveto(556.92109641,341.02225989)(556.90109643,341.06225985)(556.86110201,341.09226336)
\curveto(556.81109652,341.13225978)(556.7360966,341.15225976)(556.63610201,341.15226336)
\curveto(556.59609674,341.14225977)(556.56109677,341.13225978)(556.53110201,341.12226336)
\curveto(556.50109683,341.12225979)(556.46609687,341.1172598)(556.42610201,341.10726336)
\curveto(556.35609698,341.08725983)(556.28109705,341.07225984)(556.20110201,341.06226336)
\curveto(556.12109721,341.05225986)(556.04109729,341.03725988)(555.96110201,341.01726336)
\curveto(555.9310974,341.00725991)(555.88609745,341.00225991)(555.82610201,341.00226336)
\curveto(555.69609764,340.97225994)(555.56609777,340.95225996)(555.43610201,340.94226336)
\curveto(555.30609803,340.93225998)(555.18109815,340.90726001)(555.06110201,340.86726336)
\curveto(554.98109835,340.84726007)(554.90609843,340.82726009)(554.83610201,340.80726336)
\curveto(554.76609857,340.79726012)(554.69609864,340.77726014)(554.62610201,340.74726336)
\curveto(554.41609892,340.65726026)(554.2360991,340.52226039)(554.08610201,340.34226336)
\curveto(553.94609939,340.16226075)(553.89609944,339.912261)(553.93610201,339.59226336)
\curveto(553.95609938,339.42226149)(554.01109932,339.28226163)(554.10110201,339.17226336)
\curveto(554.17109916,339.06226185)(554.27609906,338.97226194)(554.41610201,338.90226336)
\curveto(554.55609878,338.84226207)(554.70609863,338.79726212)(554.86610201,338.76726336)
\curveto(555.0360983,338.73726218)(555.21109812,338.72726219)(555.39110201,338.73726336)
\curveto(555.58109775,338.75726216)(555.75609758,338.79226212)(555.91610201,338.84226336)
\curveto(556.17609716,338.92226199)(556.38109695,339.04726187)(556.53110201,339.21726336)
\curveto(556.68109665,339.39726152)(556.79609654,339.6172613)(556.87610201,339.87726336)
\curveto(556.89609644,339.94726097)(556.90609643,340.0172609)(556.90610201,340.08726336)
\curveto(556.91609642,340.16726075)(556.9310964,340.24726067)(556.95110201,340.32726336)
\lineto(556.95110201,340.46226336)
}
}
{
\newrgbcolor{curcolor}{0 0 0}
\pscustom[linestyle=none,fillstyle=solid,fillcolor=curcolor]
{
\newpath
\moveto(561.22938326,348.24726336)
\lineto(562.32438326,348.24726336)
\curveto(562.42438077,348.24725267)(562.51938068,348.24225267)(562.60938326,348.23226336)
\curveto(562.6993805,348.22225269)(562.76938043,348.19225272)(562.81938326,348.14226336)
\curveto(562.87938032,348.07225284)(562.90938029,347.97725294)(562.90938326,347.85726336)
\curveto(562.91938028,347.74725317)(562.92438027,347.63225328)(562.92438326,347.51226336)
\lineto(562.92438326,346.17726336)
\lineto(562.92438326,340.79226336)
\lineto(562.92438326,338.49726336)
\lineto(562.92438326,338.07726336)
\curveto(562.93438026,337.92726299)(562.91438028,337.8122631)(562.86438326,337.73226336)
\curveto(562.81438038,337.65226326)(562.72438047,337.59726332)(562.59438326,337.56726336)
\curveto(562.53438066,337.54726337)(562.46438073,337.54226337)(562.38438326,337.55226336)
\curveto(562.31438088,337.56226335)(562.24438095,337.56726335)(562.17438326,337.56726336)
\lineto(561.45438326,337.56726336)
\curveto(561.34438185,337.56726335)(561.24438195,337.57226334)(561.15438326,337.58226336)
\curveto(561.06438213,337.59226332)(560.98938221,337.62226329)(560.92938326,337.67226336)
\curveto(560.86938233,337.72226319)(560.83438236,337.79726312)(560.82438326,337.89726336)
\lineto(560.82438326,338.22726336)
\lineto(560.82438326,339.56226336)
\lineto(560.82438326,345.18726336)
\lineto(560.82438326,347.22726336)
\curveto(560.82438237,347.35725356)(560.81938238,347.5122534)(560.80938326,347.69226336)
\curveto(560.80938239,347.87225304)(560.83438236,348.00225291)(560.88438326,348.08226336)
\curveto(560.90438229,348.12225279)(560.92938227,348.15225276)(560.95938326,348.17226336)
\lineto(561.07938326,348.23226336)
\curveto(561.0993821,348.23225268)(561.12438207,348.23225268)(561.15438326,348.23226336)
\curveto(561.18438201,348.24225267)(561.20938199,348.24725267)(561.22938326,348.24726336)
}
}
{
\newrgbcolor{curcolor}{0 0 0}
\pscustom[linestyle=none,fillstyle=solid,fillcolor=curcolor]
{
\newpath
\moveto(566.68657076,348.14226336)
\curveto(566.75656781,348.06225285)(566.79156777,347.94225297)(566.79157076,347.78226336)
\lineto(566.79157076,347.31726336)
\lineto(566.79157076,346.91226336)
\curveto(566.79156777,346.77225414)(566.75656781,346.67725424)(566.68657076,346.62726336)
\curveto(566.62656794,346.57725434)(566.54656802,346.54725437)(566.44657076,346.53726336)
\curveto(566.35656821,346.52725439)(566.25656831,346.52225439)(566.14657076,346.52226336)
\lineto(565.30657076,346.52226336)
\curveto(565.19656937,346.52225439)(565.09656947,346.52725439)(565.00657076,346.53726336)
\curveto(564.92656964,346.54725437)(564.85656971,346.57725434)(564.79657076,346.62726336)
\curveto(564.75656981,346.65725426)(564.72656984,346.7122542)(564.70657076,346.79226336)
\curveto(564.69656987,346.88225403)(564.68656988,346.97725394)(564.67657076,347.07726336)
\lineto(564.67657076,347.40726336)
\curveto(564.68656988,347.5172534)(564.69156987,347.6122533)(564.69157076,347.69226336)
\lineto(564.69157076,347.90226336)
\curveto(564.70156986,347.97225294)(564.72156984,348.03225288)(564.75157076,348.08226336)
\curveto(564.77156979,348.12225279)(564.79656977,348.15225276)(564.82657076,348.17226336)
\lineto(564.94657076,348.23226336)
\curveto(564.9665696,348.23225268)(564.99156957,348.23225268)(565.02157076,348.23226336)
\curveto(565.05156951,348.24225267)(565.07656949,348.24725267)(565.09657076,348.24726336)
\lineto(566.19157076,348.24726336)
\curveto(566.29156827,348.24725267)(566.38656818,348.24225267)(566.47657076,348.23226336)
\curveto(566.566568,348.22225269)(566.63656793,348.19225272)(566.68657076,348.14226336)
\moveto(566.79157076,338.37726336)
\curveto(566.79156777,338.17726274)(566.78656778,338.00726291)(566.77657076,337.86726336)
\curveto(566.7665678,337.72726319)(566.67656789,337.63226328)(566.50657076,337.58226336)
\curveto(566.44656812,337.56226335)(566.38156818,337.55226336)(566.31157076,337.55226336)
\curveto(566.24156832,337.56226335)(566.1665684,337.56726335)(566.08657076,337.56726336)
\lineto(565.24657076,337.56726336)
\curveto(565.15656941,337.56726335)(565.0665695,337.57226334)(564.97657076,337.58226336)
\curveto(564.89656967,337.59226332)(564.83656973,337.62226329)(564.79657076,337.67226336)
\curveto(564.73656983,337.74226317)(564.70156986,337.82726309)(564.69157076,337.92726336)
\lineto(564.69157076,338.27226336)
\lineto(564.69157076,344.60226336)
\lineto(564.69157076,344.90226336)
\curveto(564.69156987,345.00225591)(564.71156985,345.08225583)(564.75157076,345.14226336)
\curveto(564.81156975,345.2122557)(564.89656967,345.25725566)(565.00657076,345.27726336)
\curveto(565.02656954,345.28725563)(565.05156951,345.28725563)(565.08157076,345.27726336)
\curveto(565.12156944,345.27725564)(565.15156941,345.28225563)(565.17157076,345.29226336)
\lineto(565.92157076,345.29226336)
\lineto(566.11657076,345.29226336)
\curveto(566.19656837,345.30225561)(566.2615683,345.30225561)(566.31157076,345.29226336)
\lineto(566.43157076,345.29226336)
\curveto(566.49156807,345.27225564)(566.54656802,345.25725566)(566.59657076,345.24726336)
\curveto(566.64656792,345.23725568)(566.68656788,345.20725571)(566.71657076,345.15726336)
\curveto(566.75656781,345.10725581)(566.77656779,345.03725588)(566.77657076,344.94726336)
\curveto(566.78656778,344.85725606)(566.79156777,344.76225615)(566.79157076,344.66226336)
\lineto(566.79157076,338.37726336)
}
}
{
\newrgbcolor{curcolor}{0 0 0}
\pscustom[linestyle=none,fillstyle=solid,fillcolor=curcolor]
{
\newpath
\moveto(571.42375826,348.24726336)
\curveto(571.51375442,348.24725267)(571.61375432,348.24725267)(571.72375826,348.24726336)
\curveto(571.84375409,348.24725267)(571.95875397,348.24225267)(572.06875826,348.23226336)
\curveto(572.18875374,348.22225269)(572.29375364,348.20225271)(572.38375826,348.17226336)
\curveto(572.47375346,348.15225276)(572.5337534,348.1172528)(572.56375826,348.06726336)
\curveto(572.62375331,347.98725293)(572.65375328,347.87225304)(572.65375826,347.72226336)
\lineto(572.65375826,347.31726336)
\curveto(572.65375328,347.2172537)(572.64875328,347.1172538)(572.63875826,347.01726336)
\curveto(572.63875329,346.917254)(572.61875331,346.84225407)(572.57875826,346.79226336)
\curveto(572.53875339,346.73225418)(572.48875344,346.69225422)(572.42875826,346.67226336)
\curveto(572.36875356,346.66225425)(572.29875363,346.65725426)(572.21875826,346.65726336)
\lineto(571.99375826,346.65726336)
\curveto(571.92375401,346.66725425)(571.85375408,346.66725425)(571.78375826,346.65726336)
\curveto(571.60375433,346.6172543)(571.46375447,346.56725435)(571.36375826,346.50726336)
\curveto(571.26375467,346.45725446)(571.18375475,346.34725457)(571.12375826,346.17726336)
\curveto(571.10375483,346.14725477)(571.09375484,346.1172548)(571.09375826,346.08726336)
\curveto(571.10375483,346.06725485)(571.10375483,346.04225487)(571.09375826,346.01226336)
\curveto(571.08375485,345.97225494)(571.07375486,345.912255)(571.06375826,345.83226336)
\curveto(571.05375488,345.75225516)(571.05375488,345.68725523)(571.06375826,345.63726336)
\curveto(571.08375485,345.56725535)(571.10875482,345.50725541)(571.13875826,345.45726336)
\curveto(571.16875476,345.40725551)(571.21375472,345.36725555)(571.27375826,345.33726336)
\curveto(571.37375456,345.28725563)(571.49375444,345.27225564)(571.63375826,345.29226336)
\curveto(571.77375416,345.3122556)(571.90375403,345.3122556)(572.02375826,345.29226336)
\curveto(572.07375386,345.28225563)(572.11375382,345.27725564)(572.14375826,345.27726336)
\curveto(572.18375375,345.28725563)(572.22375371,345.28725563)(572.26375826,345.27726336)
\curveto(572.35375358,345.23725568)(572.41875351,345.19225572)(572.45875826,345.14226336)
\curveto(572.47875345,345.1122558)(572.49375344,345.06225585)(572.50375826,344.99226336)
\curveto(572.51375342,344.93225598)(572.52375341,344.86225605)(572.53375826,344.78226336)
\curveto(572.54375339,344.7122562)(572.54375339,344.63725628)(572.53375826,344.55726336)
\curveto(572.5337534,344.48725643)(572.5287534,344.43225648)(572.51875826,344.39226336)
\curveto(572.50875342,344.35225656)(572.50875342,344.3122566)(572.51875826,344.27226336)
\curveto(572.5287534,344.24225667)(572.52375341,344.20725671)(572.50375826,344.16726336)
\curveto(572.48375345,344.04725687)(572.42375351,343.97225694)(572.32375826,343.94226336)
\curveto(572.24375369,343.90225701)(572.14875378,343.88225703)(572.03875826,343.88226336)
\curveto(571.928754,343.89225702)(571.81875411,343.89725702)(571.70875826,343.89726336)
\lineto(571.60375826,343.89726336)
\curveto(571.56375437,343.89725702)(571.5287544,343.89225702)(571.49875826,343.88226336)
\lineto(571.37875826,343.88226336)
\curveto(571.20875472,343.84225707)(571.10375483,343.73225718)(571.06375826,343.55226336)
\curveto(571.04375489,343.49225742)(571.03875489,343.42225749)(571.04875826,343.34226336)
\curveto(571.05875487,343.26225765)(571.06375487,343.18225773)(571.06375826,343.10226336)
\lineto(571.06375826,342.18726336)
\lineto(571.06375826,339.26226336)
\lineto(571.06375826,338.55726336)
\lineto(571.06375826,338.36226336)
\curveto(571.07375486,338.30226261)(571.06875486,338.24726267)(571.04875826,338.19726336)
\lineto(571.04875826,338.03226336)
\curveto(571.04875488,337.87226304)(571.02375491,337.75726316)(570.97375826,337.68726336)
\curveto(570.95375498,337.65726326)(570.91875501,337.63226328)(570.86875826,337.61226336)
\curveto(570.81875511,337.60226331)(570.76875516,337.58726333)(570.71875826,337.56726336)
\lineto(570.64375826,337.56726336)
\curveto(570.59375534,337.55726336)(570.53875539,337.55226336)(570.47875826,337.55226336)
\curveto(570.41875551,337.56226335)(570.36375557,337.56726335)(570.31375826,337.56726336)
\lineto(569.65375826,337.56726336)
\curveto(569.58375635,337.56726335)(569.50875642,337.56226335)(569.42875826,337.55226336)
\curveto(569.35875657,337.55226336)(569.29875663,337.56226335)(569.24875826,337.58226336)
\curveto(569.1287568,337.6122633)(569.04875688,337.66226325)(569.00875826,337.73226336)
\curveto(568.97875695,337.78226313)(568.95875697,337.84726307)(568.94875826,337.92726336)
\lineto(568.94875826,338.16726336)
\lineto(568.94875826,338.94726336)
\lineto(568.94875826,343.14726336)
\curveto(568.94875698,343.3172576)(568.93875699,343.46225745)(568.91875826,343.58226336)
\curveto(568.89875703,343.7122572)(568.8287571,343.80225711)(568.70875826,343.85226336)
\curveto(568.59875733,343.90225701)(568.46375747,343.912257)(568.30375826,343.88226336)
\curveto(568.14375779,343.86225705)(568.00875792,343.87725704)(567.89875826,343.92726336)
\curveto(567.78875814,343.97725694)(567.71875821,344.06225685)(567.68875826,344.18226336)
\curveto(567.66875826,344.23225668)(567.66375827,344.29225662)(567.67375826,344.36226336)
\lineto(567.67375826,344.57226336)
\curveto(567.67375826,344.75225616)(567.68375825,344.90225601)(567.70375826,345.02226336)
\curveto(567.72375821,345.14225577)(567.80875812,345.22725569)(567.95875826,345.27726336)
\curveto(568.03875789,345.29725562)(568.12375781,345.30725561)(568.21375826,345.30726336)
\lineto(568.46875826,345.30726336)
\curveto(568.55875737,345.30725561)(568.63875729,345.3122556)(568.70875826,345.32226336)
\curveto(568.77875715,345.34225557)(568.8337571,345.38225553)(568.87375826,345.44226336)
\curveto(568.94375699,345.54225537)(568.96875696,345.66725525)(568.94875826,345.81726336)
\curveto(568.93875699,345.97725494)(568.94875698,346.12725479)(568.97875826,346.26726336)
\curveto(568.98875694,346.30725461)(568.99375694,346.34725457)(568.99375826,346.38726336)
\curveto(569.00375693,346.42725449)(569.01375692,346.47225444)(569.02375826,346.52226336)
\curveto(569.06375687,346.66225425)(569.10375683,346.78725413)(569.14375826,346.89726336)
\curveto(569.18375675,347.0172539)(569.23875669,347.12725379)(569.30875826,347.22726336)
\curveto(569.44875648,347.46725345)(569.6337563,347.65725326)(569.86375826,347.79726336)
\curveto(570.09375584,347.94725297)(570.35375558,348.06225285)(570.64375826,348.14226336)
\curveto(570.72375521,348.17225274)(570.80875512,348.18725273)(570.89875826,348.18726336)
\curveto(570.98875494,348.19725272)(571.07875485,348.2122527)(571.16875826,348.23226336)
\curveto(571.19875473,348.24225267)(571.24375469,348.24225267)(571.30375826,348.23226336)
\curveto(571.36375457,348.22225269)(571.40375453,348.22725269)(571.42375826,348.24726336)
\moveto(575.84875826,348.14226336)
\curveto(575.79875013,348.19225272)(575.7287502,348.22225269)(575.63875826,348.23226336)
\curveto(575.54875038,348.24225267)(575.45375048,348.24725267)(575.35375826,348.24726336)
\lineto(574.25875826,348.24726336)
\curveto(574.23875169,348.24725267)(574.21375172,348.24225267)(574.18375826,348.23226336)
\curveto(574.15375178,348.23225268)(574.1287518,348.23225268)(574.10875826,348.23226336)
\lineto(573.98875826,348.17226336)
\curveto(573.95875197,348.15225276)(573.933752,348.12225279)(573.91375826,348.08226336)
\curveto(573.88375205,348.03225288)(573.86375207,347.97225294)(573.85375826,347.90226336)
\lineto(573.85375826,347.69226336)
\curveto(573.85375208,347.6122533)(573.84875208,347.5172534)(573.83875826,347.40726336)
\lineto(573.83875826,347.07726336)
\curveto(573.84875208,346.97725394)(573.85875207,346.88225403)(573.86875826,346.79226336)
\curveto(573.88875204,346.7122542)(573.91875201,346.65725426)(573.95875826,346.62726336)
\curveto(574.01875191,346.57725434)(574.08875184,346.54725437)(574.16875826,346.53726336)
\curveto(574.25875167,346.52725439)(574.35875157,346.52225439)(574.46875826,346.52226336)
\lineto(575.30875826,346.52226336)
\curveto(575.41875051,346.52225439)(575.51875041,346.52725439)(575.60875826,346.53726336)
\curveto(575.70875022,346.54725437)(575.78875014,346.57725434)(575.84875826,346.62726336)
\curveto(575.91875001,346.67725424)(575.95374998,346.77225414)(575.95375826,346.91226336)
\lineto(575.95375826,347.31726336)
\lineto(575.95375826,347.78226336)
\curveto(575.95374998,347.94225297)(575.91875001,348.06225285)(575.84875826,348.14226336)
\moveto(575.95375826,344.66226336)
\curveto(575.95374998,344.76225615)(575.94874998,344.85725606)(575.93875826,344.94726336)
\curveto(575.93874999,345.03725588)(575.91875001,345.10725581)(575.87875826,345.15726336)
\curveto(575.84875008,345.20725571)(575.80875012,345.23725568)(575.75875826,345.24726336)
\curveto(575.70875022,345.25725566)(575.65375028,345.27225564)(575.59375826,345.29226336)
\lineto(575.47375826,345.29226336)
\curveto(575.42375051,345.30225561)(575.35875057,345.30225561)(575.27875826,345.29226336)
\lineto(575.08375826,345.29226336)
\lineto(574.33375826,345.29226336)
\curveto(574.31375162,345.28225563)(574.28375165,345.27725564)(574.24375826,345.27726336)
\curveto(574.21375172,345.28725563)(574.18875174,345.28725563)(574.16875826,345.27726336)
\curveto(574.05875187,345.25725566)(573.97375196,345.2122557)(573.91375826,345.14226336)
\curveto(573.87375206,345.08225583)(573.85375208,345.00225591)(573.85375826,344.90226336)
\lineto(573.85375826,344.60226336)
\lineto(573.85375826,338.27226336)
\lineto(573.85375826,337.92726336)
\curveto(573.86375207,337.82726309)(573.89875203,337.74226317)(573.95875826,337.67226336)
\curveto(573.99875193,337.62226329)(574.05875187,337.59226332)(574.13875826,337.58226336)
\curveto(574.2287517,337.57226334)(574.31875161,337.56726335)(574.40875826,337.56726336)
\lineto(575.24875826,337.56726336)
\curveto(575.3287506,337.56726335)(575.40375053,337.56226335)(575.47375826,337.55226336)
\curveto(575.54375039,337.55226336)(575.60875032,337.56226335)(575.66875826,337.58226336)
\curveto(575.83875009,337.63226328)(575.92875,337.72726319)(575.93875826,337.86726336)
\curveto(575.94874998,338.00726291)(575.95374998,338.17726274)(575.95375826,338.37726336)
\lineto(575.95375826,344.66226336)
}
}
{
\newrgbcolor{curcolor}{0 0 0}
\pscustom[linestyle=none,fillstyle=solid,fillcolor=curcolor]
{
\newpath
\moveto(581.19368013,345.50226336)
\curveto(582.00367497,345.52225539)(582.6786743,345.40225551)(583.21868013,345.14226336)
\curveto(583.76867321,344.88225603)(584.20367277,344.5122564)(584.52368013,344.03226336)
\curveto(584.68367229,343.79225712)(584.80367217,343.5172574)(584.88368013,343.20726336)
\curveto(584.90367207,343.15725776)(584.91867206,343.09225782)(584.92868013,343.01226336)
\curveto(584.94867203,342.93225798)(584.94867203,342.86225805)(584.92868013,342.80226336)
\curveto(584.88867209,342.69225822)(584.81867216,342.62725829)(584.71868013,342.60726336)
\curveto(584.61867236,342.59725832)(584.49867248,342.59225832)(584.35868013,342.59226336)
\lineto(583.57868013,342.59226336)
\lineto(583.29368013,342.59226336)
\curveto(583.20367377,342.59225832)(583.12867385,342.6122583)(583.06868013,342.65226336)
\curveto(582.98867399,342.69225822)(582.93367404,342.75225816)(582.90368013,342.83226336)
\curveto(582.8736741,342.92225799)(582.83367414,343.0122579)(582.78368013,343.10226336)
\curveto(582.72367425,343.2122577)(582.65867432,343.3122576)(582.58868013,343.40226336)
\curveto(582.51867446,343.49225742)(582.43867454,343.57225734)(582.34868013,343.64226336)
\curveto(582.20867477,343.73225718)(582.05367492,343.80225711)(581.88368013,343.85226336)
\curveto(581.82367515,343.87225704)(581.76367521,343.88225703)(581.70368013,343.88226336)
\curveto(581.64367533,343.88225703)(581.58867539,343.89225702)(581.53868013,343.91226336)
\lineto(581.38868013,343.91226336)
\curveto(581.18867579,343.912257)(581.02867595,343.89225702)(580.90868013,343.85226336)
\curveto(580.61867636,343.76225715)(580.38367659,343.62225729)(580.20368013,343.43226336)
\curveto(580.02367695,343.25225766)(579.8786771,343.03225788)(579.76868013,342.77226336)
\curveto(579.71867726,342.66225825)(579.6786773,342.54225837)(579.64868013,342.41226336)
\curveto(579.62867735,342.29225862)(579.60367737,342.16225875)(579.57368013,342.02226336)
\curveto(579.56367741,341.98225893)(579.55867742,341.94225897)(579.55868013,341.90226336)
\curveto(579.55867742,341.86225905)(579.55367742,341.82225909)(579.54368013,341.78226336)
\curveto(579.52367745,341.68225923)(579.51367746,341.54225937)(579.51368013,341.36226336)
\curveto(579.52367745,341.18225973)(579.53867744,341.04225987)(579.55868013,340.94226336)
\curveto(579.55867742,340.86226005)(579.56367741,340.80726011)(579.57368013,340.77726336)
\curveto(579.59367738,340.70726021)(579.60367737,340.63726028)(579.60368013,340.56726336)
\curveto(579.61367736,340.49726042)(579.62867735,340.42726049)(579.64868013,340.35726336)
\curveto(579.72867725,340.12726079)(579.82367715,339.917261)(579.93368013,339.72726336)
\curveto(580.04367693,339.53726138)(580.18367679,339.37726154)(580.35368013,339.24726336)
\curveto(580.39367658,339.2172617)(580.45367652,339.18226173)(580.53368013,339.14226336)
\curveto(580.64367633,339.07226184)(580.75367622,339.02726189)(580.86368013,339.00726336)
\curveto(580.98367599,338.98726193)(581.12867585,338.96726195)(581.29868013,338.94726336)
\lineto(581.38868013,338.94726336)
\curveto(581.42867555,338.94726197)(581.45867552,338.95226196)(581.47868013,338.96226336)
\lineto(581.61368013,338.96226336)
\curveto(581.68367529,338.98226193)(581.74867523,338.99726192)(581.80868013,339.00726336)
\curveto(581.8786751,339.02726189)(581.94367503,339.04726187)(582.00368013,339.06726336)
\curveto(582.30367467,339.19726172)(582.53367444,339.38726153)(582.69368013,339.63726336)
\curveto(582.73367424,339.68726123)(582.76867421,339.74226117)(582.79868013,339.80226336)
\curveto(582.82867415,339.87226104)(582.85367412,339.93226098)(582.87368013,339.98226336)
\curveto(582.91367406,340.09226082)(582.94867403,340.18726073)(582.97868013,340.26726336)
\curveto(583.00867397,340.35726056)(583.0786739,340.42726049)(583.18868013,340.47726336)
\curveto(583.2786737,340.5172604)(583.42367355,340.53226038)(583.62368013,340.52226336)
\lineto(584.11868013,340.52226336)
\lineto(584.32868013,340.52226336)
\curveto(584.40867257,340.53226038)(584.4736725,340.52726039)(584.52368013,340.50726336)
\lineto(584.64368013,340.50726336)
\lineto(584.76368013,340.47726336)
\curveto(584.80367217,340.47726044)(584.83367214,340.46726045)(584.85368013,340.44726336)
\curveto(584.90367207,340.40726051)(584.93367204,340.34726057)(584.94368013,340.26726336)
\curveto(584.96367201,340.19726072)(584.96367201,340.12226079)(584.94368013,340.04226336)
\curveto(584.85367212,339.7122612)(584.74367223,339.4172615)(584.61368013,339.15726336)
\curveto(584.20367277,338.38726253)(583.54867343,337.85226306)(582.64868013,337.55226336)
\curveto(582.54867443,337.52226339)(582.44367453,337.50226341)(582.33368013,337.49226336)
\curveto(582.22367475,337.47226344)(582.11367486,337.44726347)(582.00368013,337.41726336)
\curveto(581.94367503,337.40726351)(581.88367509,337.40226351)(581.82368013,337.40226336)
\curveto(581.76367521,337.40226351)(581.70367527,337.39726352)(581.64368013,337.38726336)
\lineto(581.47868013,337.38726336)
\curveto(581.42867555,337.36726355)(581.35367562,337.36226355)(581.25368013,337.37226336)
\curveto(581.15367582,337.37226354)(581.0786759,337.37726354)(581.02868013,337.38726336)
\curveto(580.94867603,337.40726351)(580.8736761,337.4172635)(580.80368013,337.41726336)
\curveto(580.74367623,337.40726351)(580.6786763,337.4122635)(580.60868013,337.43226336)
\lineto(580.45868013,337.46226336)
\curveto(580.40867657,337.46226345)(580.35867662,337.46726345)(580.30868013,337.47726336)
\curveto(580.19867678,337.50726341)(580.09367688,337.53726338)(579.99368013,337.56726336)
\curveto(579.89367708,337.59726332)(579.79867718,337.63226328)(579.70868013,337.67226336)
\curveto(579.23867774,337.87226304)(578.84367813,338.12726279)(578.52368013,338.43726336)
\curveto(578.20367877,338.75726216)(577.94367903,339.15226176)(577.74368013,339.62226336)
\curveto(577.69367928,339.7122612)(577.65367932,339.80726111)(577.62368013,339.90726336)
\lineto(577.53368013,340.23726336)
\curveto(577.52367945,340.27726064)(577.51867946,340.3122606)(577.51868013,340.34226336)
\curveto(577.51867946,340.38226053)(577.50867947,340.42726049)(577.48868013,340.47726336)
\curveto(577.46867951,340.54726037)(577.45867952,340.6172603)(577.45868013,340.68726336)
\curveto(577.45867952,340.76726015)(577.44867953,340.84226007)(577.42868013,340.91226336)
\lineto(577.42868013,341.16726336)
\curveto(577.40867957,341.2172597)(577.39867958,341.27225964)(577.39868013,341.33226336)
\curveto(577.39867958,341.40225951)(577.40867957,341.46225945)(577.42868013,341.51226336)
\curveto(577.43867954,341.56225935)(577.43867954,341.60725931)(577.42868013,341.64726336)
\curveto(577.41867956,341.68725923)(577.41867956,341.72725919)(577.42868013,341.76726336)
\curveto(577.44867953,341.83725908)(577.45367952,341.90225901)(577.44368013,341.96226336)
\curveto(577.44367953,342.02225889)(577.45367952,342.08225883)(577.47368013,342.14226336)
\curveto(577.52367945,342.32225859)(577.56367941,342.49225842)(577.59368013,342.65226336)
\curveto(577.62367935,342.82225809)(577.66867931,342.98725793)(577.72868013,343.14726336)
\curveto(577.94867903,343.65725726)(578.22367875,344.08225683)(578.55368013,344.42226336)
\curveto(578.89367808,344.76225615)(579.32367765,345.03725588)(579.84368013,345.24726336)
\curveto(579.98367699,345.30725561)(580.12867685,345.34725557)(580.27868013,345.36726336)
\curveto(580.42867655,345.39725552)(580.58367639,345.43225548)(580.74368013,345.47226336)
\curveto(580.82367615,345.48225543)(580.89867608,345.48725543)(580.96868013,345.48726336)
\curveto(581.03867594,345.48725543)(581.11367586,345.49225542)(581.19368013,345.50226336)
}
}
{
\newrgbcolor{curcolor}{0 0 0}
\pscustom[linestyle=none,fillstyle=solid,fillcolor=curcolor]
{
\newpath
\moveto(593.28696138,338.15226336)
\curveto(593.30695353,338.04226287)(593.31695352,337.93226298)(593.31696138,337.82226336)
\curveto(593.32695351,337.7122632)(593.27695356,337.63726328)(593.16696138,337.59726336)
\curveto(593.10695373,337.56726335)(593.0369538,337.55226336)(592.95696138,337.55226336)
\lineto(592.71696138,337.55226336)
\lineto(591.90696138,337.55226336)
\lineto(591.63696138,337.55226336)
\curveto(591.55695528,337.56226335)(591.49195535,337.58726333)(591.44196138,337.62726336)
\curveto(591.37195547,337.66726325)(591.31695552,337.72226319)(591.27696138,337.79226336)
\curveto(591.24695559,337.87226304)(591.20195564,337.93726298)(591.14196138,337.98726336)
\curveto(591.12195572,338.00726291)(591.09695574,338.02226289)(591.06696138,338.03226336)
\curveto(591.0369558,338.05226286)(590.99695584,338.05726286)(590.94696138,338.04726336)
\curveto(590.89695594,338.02726289)(590.84695599,338.00226291)(590.79696138,337.97226336)
\curveto(590.75695608,337.94226297)(590.71195613,337.917263)(590.66196138,337.89726336)
\curveto(590.61195623,337.85726306)(590.55695628,337.82226309)(590.49696138,337.79226336)
\lineto(590.31696138,337.70226336)
\curveto(590.18695665,337.64226327)(590.05195679,337.59226332)(589.91196138,337.55226336)
\curveto(589.77195707,337.52226339)(589.62695721,337.48726343)(589.47696138,337.44726336)
\curveto(589.40695743,337.42726349)(589.3369575,337.4172635)(589.26696138,337.41726336)
\curveto(589.20695763,337.40726351)(589.1419577,337.39726352)(589.07196138,337.38726336)
\lineto(588.98196138,337.38726336)
\curveto(588.95195789,337.37726354)(588.92195792,337.37226354)(588.89196138,337.37226336)
\lineto(588.72696138,337.37226336)
\curveto(588.62695821,337.35226356)(588.52695831,337.35226356)(588.42696138,337.37226336)
\lineto(588.29196138,337.37226336)
\curveto(588.22195862,337.39226352)(588.15195869,337.40226351)(588.08196138,337.40226336)
\curveto(588.02195882,337.39226352)(587.96195888,337.39726352)(587.90196138,337.41726336)
\curveto(587.80195904,337.43726348)(587.70695913,337.45726346)(587.61696138,337.47726336)
\curveto(587.52695931,337.48726343)(587.4419594,337.5122634)(587.36196138,337.55226336)
\curveto(587.07195977,337.66226325)(586.82196002,337.80226311)(586.61196138,337.97226336)
\curveto(586.41196043,338.15226276)(586.25196059,338.38726253)(586.13196138,338.67726336)
\curveto(586.10196074,338.74726217)(586.07196077,338.82226209)(586.04196138,338.90226336)
\curveto(586.02196082,338.98226193)(586.00196084,339.06726185)(585.98196138,339.15726336)
\curveto(585.96196088,339.20726171)(585.95196089,339.25726166)(585.95196138,339.30726336)
\curveto(585.96196088,339.35726156)(585.96196088,339.40726151)(585.95196138,339.45726336)
\curveto(585.9419609,339.48726143)(585.93196091,339.54726137)(585.92196138,339.63726336)
\curveto(585.92196092,339.73726118)(585.92696091,339.80726111)(585.93696138,339.84726336)
\curveto(585.95696088,339.94726097)(585.96696087,340.03226088)(585.96696138,340.10226336)
\lineto(586.05696138,340.43226336)
\curveto(586.08696075,340.55226036)(586.12696071,340.65726026)(586.17696138,340.74726336)
\curveto(586.34696049,341.03725988)(586.5419603,341.25725966)(586.76196138,341.40726336)
\curveto(586.98195986,341.55725936)(587.26195958,341.68725923)(587.60196138,341.79726336)
\curveto(587.73195911,341.84725907)(587.86695897,341.88225903)(588.00696138,341.90226336)
\curveto(588.14695869,341.92225899)(588.28695855,341.94725897)(588.42696138,341.97726336)
\curveto(588.50695833,341.99725892)(588.59195825,342.00725891)(588.68196138,342.00726336)
\curveto(588.77195807,342.0172589)(588.86195798,342.03225888)(588.95196138,342.05226336)
\curveto(589.02195782,342.07225884)(589.09195775,342.07725884)(589.16196138,342.06726336)
\curveto(589.23195761,342.06725885)(589.30695753,342.07725884)(589.38696138,342.09726336)
\curveto(589.45695738,342.1172588)(589.52695731,342.12725879)(589.59696138,342.12726336)
\curveto(589.66695717,342.12725879)(589.7419571,342.13725878)(589.82196138,342.15726336)
\curveto(590.03195681,342.20725871)(590.22195662,342.24725867)(590.39196138,342.27726336)
\curveto(590.57195627,342.3172586)(590.73195611,342.40725851)(590.87196138,342.54726336)
\curveto(590.96195588,342.63725828)(591.02195582,342.73725818)(591.05196138,342.84726336)
\curveto(591.06195578,342.87725804)(591.06195578,342.90225801)(591.05196138,342.92226336)
\curveto(591.05195579,342.94225797)(591.05695578,342.96225795)(591.06696138,342.98226336)
\curveto(591.07695576,343.00225791)(591.08195576,343.03225788)(591.08196138,343.07226336)
\lineto(591.08196138,343.16226336)
\lineto(591.05196138,343.28226336)
\curveto(591.05195579,343.32225759)(591.04695579,343.35725756)(591.03696138,343.38726336)
\curveto(590.9369559,343.68725723)(590.72695611,343.89225702)(590.40696138,344.00226336)
\curveto(590.31695652,344.03225688)(590.20695663,344.05225686)(590.07696138,344.06226336)
\curveto(589.95695688,344.08225683)(589.83195701,344.08725683)(589.70196138,344.07726336)
\curveto(589.57195727,344.07725684)(589.44695739,344.06725685)(589.32696138,344.04726336)
\curveto(589.20695763,344.02725689)(589.10195774,344.00225691)(589.01196138,343.97226336)
\curveto(588.95195789,343.95225696)(588.89195795,343.92225699)(588.83196138,343.88226336)
\curveto(588.78195806,343.85225706)(588.73195811,343.8172571)(588.68196138,343.77726336)
\curveto(588.63195821,343.73725718)(588.57695826,343.68225723)(588.51696138,343.61226336)
\curveto(588.46695837,343.54225737)(588.43195841,343.47725744)(588.41196138,343.41726336)
\curveto(588.36195848,343.3172576)(588.31695852,343.22225769)(588.27696138,343.13226336)
\curveto(588.24695859,343.04225787)(588.17695866,342.98225793)(588.06696138,342.95226336)
\curveto(587.98695885,342.93225798)(587.90195894,342.92225799)(587.81196138,342.92226336)
\lineto(587.54196138,342.92226336)
\lineto(586.97196138,342.92226336)
\curveto(586.92195992,342.92225799)(586.87195997,342.917258)(586.82196138,342.90726336)
\curveto(586.77196007,342.90725801)(586.72696011,342.912258)(586.68696138,342.92226336)
\lineto(586.55196138,342.92226336)
\curveto(586.53196031,342.93225798)(586.50696033,342.93725798)(586.47696138,342.93726336)
\curveto(586.44696039,342.93725798)(586.42196042,342.94725797)(586.40196138,342.96726336)
\curveto(586.32196052,342.98725793)(586.26696057,343.05225786)(586.23696138,343.16226336)
\curveto(586.22696061,343.2122577)(586.22696061,343.26225765)(586.23696138,343.31226336)
\curveto(586.24696059,343.36225755)(586.25696058,343.40725751)(586.26696138,343.44726336)
\curveto(586.29696054,343.55725736)(586.32696051,343.65725726)(586.35696138,343.74726336)
\curveto(586.39696044,343.84725707)(586.4419604,343.93725698)(586.49196138,344.01726336)
\lineto(586.58196138,344.16726336)
\lineto(586.67196138,344.31726336)
\curveto(586.75196009,344.42725649)(586.85195999,344.53225638)(586.97196138,344.63226336)
\curveto(586.99195985,344.64225627)(587.02195982,344.66725625)(587.06196138,344.70726336)
\curveto(587.11195973,344.74725617)(587.15695968,344.78225613)(587.19696138,344.81226336)
\curveto(587.2369596,344.84225607)(587.28195956,344.87225604)(587.33196138,344.90226336)
\curveto(587.50195934,345.0122559)(587.68195916,345.09725582)(587.87196138,345.15726336)
\curveto(588.06195878,345.22725569)(588.25695858,345.29225562)(588.45696138,345.35226336)
\curveto(588.57695826,345.38225553)(588.70195814,345.40225551)(588.83196138,345.41226336)
\curveto(588.96195788,345.42225549)(589.09195775,345.44225547)(589.22196138,345.47226336)
\curveto(589.26195758,345.48225543)(589.32195752,345.48225543)(589.40196138,345.47226336)
\curveto(589.49195735,345.46225545)(589.54695729,345.46725545)(589.56696138,345.48726336)
\curveto(589.97695686,345.49725542)(590.36695647,345.48225543)(590.73696138,345.44226336)
\curveto(591.11695572,345.40225551)(591.45695538,345.32725559)(591.75696138,345.21726336)
\curveto(592.06695477,345.10725581)(592.33195451,344.95725596)(592.55196138,344.76726336)
\curveto(592.77195407,344.58725633)(592.9419539,344.35225656)(593.06196138,344.06226336)
\curveto(593.13195371,343.89225702)(593.17195367,343.69725722)(593.18196138,343.47726336)
\curveto(593.19195365,343.25725766)(593.19695364,343.03225788)(593.19696138,342.80226336)
\lineto(593.19696138,339.45726336)
\lineto(593.19696138,338.87226336)
\curveto(593.19695364,338.68226223)(593.21695362,338.50726241)(593.25696138,338.34726336)
\curveto(593.26695357,338.3172626)(593.27195357,338.28226263)(593.27196138,338.24226336)
\curveto(593.27195357,338.2122627)(593.27695356,338.18226273)(593.28696138,338.15226336)
\moveto(591.08196138,340.46226336)
\curveto(591.09195575,340.5122604)(591.09695574,340.56726035)(591.09696138,340.62726336)
\curveto(591.09695574,340.69726022)(591.09195575,340.75726016)(591.08196138,340.80726336)
\curveto(591.06195578,340.86726005)(591.05195579,340.92225999)(591.05196138,340.97226336)
\curveto(591.05195579,341.02225989)(591.03195581,341.06225985)(590.99196138,341.09226336)
\curveto(590.9419559,341.13225978)(590.86695597,341.15225976)(590.76696138,341.15226336)
\curveto(590.72695611,341.14225977)(590.69195615,341.13225978)(590.66196138,341.12226336)
\curveto(590.63195621,341.12225979)(590.59695624,341.1172598)(590.55696138,341.10726336)
\curveto(590.48695635,341.08725983)(590.41195643,341.07225984)(590.33196138,341.06226336)
\curveto(590.25195659,341.05225986)(590.17195667,341.03725988)(590.09196138,341.01726336)
\curveto(590.06195678,341.00725991)(590.01695682,341.00225991)(589.95696138,341.00226336)
\curveto(589.82695701,340.97225994)(589.69695714,340.95225996)(589.56696138,340.94226336)
\curveto(589.4369574,340.93225998)(589.31195753,340.90726001)(589.19196138,340.86726336)
\curveto(589.11195773,340.84726007)(589.0369578,340.82726009)(588.96696138,340.80726336)
\curveto(588.89695794,340.79726012)(588.82695801,340.77726014)(588.75696138,340.74726336)
\curveto(588.54695829,340.65726026)(588.36695847,340.52226039)(588.21696138,340.34226336)
\curveto(588.07695876,340.16226075)(588.02695881,339.912261)(588.06696138,339.59226336)
\curveto(588.08695875,339.42226149)(588.1419587,339.28226163)(588.23196138,339.17226336)
\curveto(588.30195854,339.06226185)(588.40695843,338.97226194)(588.54696138,338.90226336)
\curveto(588.68695815,338.84226207)(588.836958,338.79726212)(588.99696138,338.76726336)
\curveto(589.16695767,338.73726218)(589.3419575,338.72726219)(589.52196138,338.73726336)
\curveto(589.71195713,338.75726216)(589.88695695,338.79226212)(590.04696138,338.84226336)
\curveto(590.30695653,338.92226199)(590.51195633,339.04726187)(590.66196138,339.21726336)
\curveto(590.81195603,339.39726152)(590.92695591,339.6172613)(591.00696138,339.87726336)
\curveto(591.02695581,339.94726097)(591.0369558,340.0172609)(591.03696138,340.08726336)
\curveto(591.04695579,340.16726075)(591.06195578,340.24726067)(591.08196138,340.32726336)
\lineto(591.08196138,340.46226336)
}
}
{
\newrgbcolor{curcolor}{0 0 0}
\pscustom[linestyle=none,fillstyle=solid,fillcolor=curcolor]
{
\newpath
\moveto(602.44024263,338.40726336)
\lineto(602.44024263,337.98726336)
\curveto(602.44023426,337.85726306)(602.41023429,337.75226316)(602.35024263,337.67226336)
\curveto(602.3002344,337.62226329)(602.23523447,337.58726333)(602.15524263,337.56726336)
\curveto(602.07523463,337.55726336)(601.98523472,337.55226336)(601.88524263,337.55226336)
\lineto(601.06024263,337.55226336)
\lineto(600.77524263,337.55226336)
\curveto(600.69523601,337.56226335)(600.63023607,337.58726333)(600.58024263,337.62726336)
\curveto(600.51023619,337.67726324)(600.47023623,337.74226317)(600.46024263,337.82226336)
\curveto(600.45023625,337.90226301)(600.43023627,337.98226293)(600.40024263,338.06226336)
\curveto(600.38023632,338.08226283)(600.36023634,338.09726282)(600.34024263,338.10726336)
\curveto(600.33023637,338.12726279)(600.31523639,338.14726277)(600.29524263,338.16726336)
\curveto(600.18523652,338.16726275)(600.1052366,338.14226277)(600.05524263,338.09226336)
\lineto(599.90524263,337.94226336)
\curveto(599.83523687,337.89226302)(599.77023693,337.84726307)(599.71024263,337.80726336)
\curveto(599.65023705,337.77726314)(599.58523712,337.73726318)(599.51524263,337.68726336)
\curveto(599.47523723,337.66726325)(599.43023727,337.64726327)(599.38024263,337.62726336)
\curveto(599.34023736,337.60726331)(599.29523741,337.58726333)(599.24524263,337.56726336)
\curveto(599.1052376,337.5172634)(598.95523775,337.47226344)(598.79524263,337.43226336)
\curveto(598.74523796,337.4122635)(598.700238,337.40226351)(598.66024263,337.40226336)
\curveto(598.62023808,337.40226351)(598.58023812,337.39726352)(598.54024263,337.38726336)
\lineto(598.40524263,337.38726336)
\curveto(598.37523833,337.37726354)(598.33523837,337.37226354)(598.28524263,337.37226336)
\lineto(598.15024263,337.37226336)
\curveto(598.09023861,337.35226356)(598.0002387,337.34726357)(597.88024263,337.35726336)
\curveto(597.76023894,337.35726356)(597.67523903,337.36726355)(597.62524263,337.38726336)
\curveto(597.55523915,337.40726351)(597.49023921,337.4172635)(597.43024263,337.41726336)
\curveto(597.38023932,337.40726351)(597.32523938,337.4122635)(597.26524263,337.43226336)
\lineto(596.90524263,337.55226336)
\curveto(596.79523991,337.58226333)(596.68524002,337.62226329)(596.57524263,337.67226336)
\curveto(596.22524048,337.82226309)(595.91024079,338.05226286)(595.63024263,338.36226336)
\curveto(595.36024134,338.68226223)(595.14524156,339.0172619)(594.98524263,339.36726336)
\curveto(594.93524177,339.47726144)(594.89524181,339.58226133)(594.86524263,339.68226336)
\curveto(594.83524187,339.79226112)(594.8002419,339.90226101)(594.76024263,340.01226336)
\curveto(594.75024195,340.05226086)(594.74524196,340.08726083)(594.74524263,340.11726336)
\curveto(594.74524196,340.15726076)(594.73524197,340.20226071)(594.71524263,340.25226336)
\curveto(594.69524201,340.33226058)(594.67524203,340.4172605)(594.65524263,340.50726336)
\curveto(594.64524206,340.60726031)(594.63024207,340.70726021)(594.61024263,340.80726336)
\curveto(594.6002421,340.83726008)(594.59524211,340.87226004)(594.59524263,340.91226336)
\curveto(594.6052421,340.95225996)(594.6052421,340.98725993)(594.59524263,341.01726336)
\lineto(594.59524263,341.15226336)
\curveto(594.59524211,341.20225971)(594.59024211,341.25225966)(594.58024263,341.30226336)
\curveto(594.57024213,341.35225956)(594.56524214,341.40725951)(594.56524263,341.46726336)
\curveto(594.56524214,341.53725938)(594.57024213,341.59225932)(594.58024263,341.63226336)
\curveto(594.59024211,341.68225923)(594.59524211,341.72725919)(594.59524263,341.76726336)
\lineto(594.59524263,341.91726336)
\curveto(594.6052421,341.96725895)(594.6052421,342.0122589)(594.59524263,342.05226336)
\curveto(594.59524211,342.10225881)(594.6052421,342.15225876)(594.62524263,342.20226336)
\curveto(594.64524206,342.3122586)(594.66024204,342.4172585)(594.67024263,342.51726336)
\curveto(594.69024201,342.6172583)(594.71524199,342.7172582)(594.74524263,342.81726336)
\curveto(594.78524192,342.93725798)(594.82024188,343.05225786)(594.85024263,343.16226336)
\curveto(594.88024182,343.27225764)(594.92024178,343.38225753)(594.97024263,343.49226336)
\curveto(595.11024159,343.79225712)(595.28524142,344.07725684)(595.49524263,344.34726336)
\curveto(595.51524119,344.37725654)(595.54024116,344.40225651)(595.57024263,344.42226336)
\curveto(595.61024109,344.45225646)(595.64024106,344.48225643)(595.66024263,344.51226336)
\curveto(595.700241,344.56225635)(595.74024096,344.60725631)(595.78024263,344.64726336)
\curveto(595.82024088,344.68725623)(595.86524084,344.72725619)(595.91524263,344.76726336)
\curveto(595.95524075,344.78725613)(595.99024071,344.8122561)(596.02024263,344.84226336)
\curveto(596.05024065,344.88225603)(596.08524062,344.912256)(596.12524263,344.93226336)
\curveto(596.37524033,345.10225581)(596.66524004,345.24225567)(596.99524263,345.35226336)
\curveto(597.06523964,345.37225554)(597.13523957,345.38725553)(597.20524263,345.39726336)
\curveto(597.28523942,345.40725551)(597.36523934,345.42225549)(597.44524263,345.44226336)
\curveto(597.51523919,345.46225545)(597.6052391,345.47225544)(597.71524263,345.47226336)
\curveto(597.82523888,345.48225543)(597.93523877,345.48725543)(598.04524263,345.48726336)
\curveto(598.15523855,345.48725543)(598.26023844,345.48225543)(598.36024263,345.47226336)
\curveto(598.47023823,345.46225545)(598.56023814,345.44725547)(598.63024263,345.42726336)
\curveto(598.78023792,345.37725554)(598.92523778,345.33225558)(599.06524263,345.29226336)
\curveto(599.2052375,345.25225566)(599.33523737,345.19725572)(599.45524263,345.12726336)
\curveto(599.52523718,345.07725584)(599.59023711,345.02725589)(599.65024263,344.97726336)
\curveto(599.71023699,344.93725598)(599.77523693,344.89225602)(599.84524263,344.84226336)
\curveto(599.88523682,344.8122561)(599.94023676,344.77225614)(600.01024263,344.72226336)
\curveto(600.09023661,344.67225624)(600.16523654,344.67225624)(600.23524263,344.72226336)
\curveto(600.27523643,344.74225617)(600.29523641,344.77725614)(600.29524263,344.82726336)
\curveto(600.29523641,344.87725604)(600.3052364,344.92725599)(600.32524263,344.97726336)
\lineto(600.32524263,345.12726336)
\curveto(600.33523637,345.15725576)(600.34023636,345.19225572)(600.34024263,345.23226336)
\lineto(600.34024263,345.35226336)
\lineto(600.34024263,347.39226336)
\curveto(600.34023636,347.50225341)(600.33523637,347.62225329)(600.32524263,347.75226336)
\curveto(600.32523638,347.89225302)(600.35023635,347.99725292)(600.40024263,348.06726336)
\curveto(600.44023626,348.14725277)(600.51523619,348.19725272)(600.62524263,348.21726336)
\curveto(600.64523606,348.22725269)(600.66523604,348.22725269)(600.68524263,348.21726336)
\curveto(600.705236,348.2172527)(600.72523598,348.22225269)(600.74524263,348.23226336)
\lineto(601.81024263,348.23226336)
\curveto(601.93023477,348.23225268)(602.04023466,348.22725269)(602.14024263,348.21726336)
\curveto(602.24023446,348.20725271)(602.31523439,348.16725275)(602.36524263,348.09726336)
\curveto(602.41523429,348.0172529)(602.44023426,347.912253)(602.44024263,347.78226336)
\lineto(602.44024263,347.42226336)
\lineto(602.44024263,338.40726336)
\moveto(600.40024263,341.34726336)
\curveto(600.41023629,341.38725953)(600.41023629,341.42725949)(600.40024263,341.46726336)
\lineto(600.40024263,341.60226336)
\curveto(600.4002363,341.70225921)(600.39523631,341.80225911)(600.38524263,341.90226336)
\curveto(600.37523633,342.00225891)(600.36023634,342.09225882)(600.34024263,342.17226336)
\curveto(600.32023638,342.28225863)(600.3002364,342.38225853)(600.28024263,342.47226336)
\curveto(600.27023643,342.56225835)(600.24523646,342.64725827)(600.20524263,342.72726336)
\curveto(600.06523664,343.08725783)(599.86023684,343.37225754)(599.59024263,343.58226336)
\curveto(599.33023737,343.79225712)(598.95023775,343.89725702)(598.45024263,343.89726336)
\curveto(598.39023831,343.89725702)(598.31023839,343.88725703)(598.21024263,343.86726336)
\curveto(598.13023857,343.84725707)(598.05523865,343.82725709)(597.98524263,343.80726336)
\curveto(597.92523878,343.79725712)(597.86523884,343.77725714)(597.80524263,343.74726336)
\curveto(597.53523917,343.63725728)(597.32523938,343.46725745)(597.17524263,343.23726336)
\curveto(597.02523968,343.00725791)(596.9052398,342.74725817)(596.81524263,342.45726336)
\curveto(596.78523992,342.35725856)(596.76523994,342.25725866)(596.75524263,342.15726336)
\curveto(596.74523996,342.05725886)(596.72523998,341.95225896)(596.69524263,341.84226336)
\lineto(596.69524263,341.63226336)
\curveto(596.67524003,341.54225937)(596.67024003,341.4172595)(596.68024263,341.25726336)
\curveto(596.69024001,341.10725981)(596.70524,340.99725992)(596.72524263,340.92726336)
\lineto(596.72524263,340.83726336)
\curveto(596.73523997,340.8172601)(596.74023996,340.79726012)(596.74024263,340.77726336)
\curveto(596.76023994,340.69726022)(596.77523993,340.62226029)(596.78524263,340.55226336)
\curveto(596.8052399,340.48226043)(596.82523988,340.40726051)(596.84524263,340.32726336)
\curveto(597.01523969,339.80726111)(597.3052394,339.42226149)(597.71524263,339.17226336)
\curveto(597.84523886,339.08226183)(598.02523868,339.0122619)(598.25524263,338.96226336)
\curveto(598.29523841,338.95226196)(598.35523835,338.94726197)(598.43524263,338.94726336)
\curveto(598.46523824,338.93726198)(598.51023819,338.92726199)(598.57024263,338.91726336)
\curveto(598.64023806,338.917262)(598.69523801,338.92226199)(598.73524263,338.93226336)
\curveto(598.81523789,338.95226196)(598.89523781,338.96726195)(598.97524263,338.97726336)
\curveto(599.05523765,338.98726193)(599.13523757,339.00726191)(599.21524263,339.03726336)
\curveto(599.46523724,339.14726177)(599.66523704,339.28726163)(599.81524263,339.45726336)
\curveto(599.96523674,339.62726129)(600.09523661,339.84226107)(600.20524263,340.10226336)
\curveto(600.24523646,340.19226072)(600.27523643,340.28226063)(600.29524263,340.37226336)
\curveto(600.31523639,340.47226044)(600.33523637,340.57726034)(600.35524263,340.68726336)
\curveto(600.36523634,340.73726018)(600.36523634,340.78226013)(600.35524263,340.82226336)
\curveto(600.35523635,340.87226004)(600.36523634,340.92225999)(600.38524263,340.97226336)
\curveto(600.39523631,341.00225991)(600.4002363,341.03725988)(600.40024263,341.07726336)
\lineto(600.40024263,341.21226336)
\lineto(600.40024263,341.34726336)
}
}
{
\newrgbcolor{curcolor}{0 0 0}
\pscustom[linestyle=none,fillstyle=solid,fillcolor=curcolor]
{
\newpath
\moveto(611.79016451,341.73726336)
\curveto(611.81015594,341.67725924)(611.82015593,341.59225932)(611.82016451,341.48226336)
\curveto(611.82015593,341.37225954)(611.81015594,341.28725963)(611.79016451,341.22726336)
\lineto(611.79016451,341.07726336)
\curveto(611.77015598,340.99725992)(611.76015599,340.91726)(611.76016451,340.83726336)
\curveto(611.77015598,340.75726016)(611.76515598,340.67726024)(611.74516451,340.59726336)
\curveto(611.72515602,340.52726039)(611.71015604,340.46226045)(611.70016451,340.40226336)
\curveto(611.69015606,340.34226057)(611.68015607,340.27726064)(611.67016451,340.20726336)
\curveto(611.63015612,340.09726082)(611.59515615,339.98226093)(611.56516451,339.86226336)
\curveto(611.53515621,339.75226116)(611.49515625,339.64726127)(611.44516451,339.54726336)
\curveto(611.23515651,339.06726185)(610.96015679,338.67726224)(610.62016451,338.37726336)
\curveto(610.28015747,338.07726284)(609.87015788,337.82726309)(609.39016451,337.62726336)
\curveto(609.27015848,337.57726334)(609.1451586,337.54226337)(609.01516451,337.52226336)
\curveto(608.89515885,337.49226342)(608.77015898,337.46226345)(608.64016451,337.43226336)
\curveto(608.59015916,337.4122635)(608.53515921,337.40226351)(608.47516451,337.40226336)
\curveto(608.41515933,337.40226351)(608.36015939,337.39726352)(608.31016451,337.38726336)
\lineto(608.20516451,337.38726336)
\curveto(608.17515957,337.37726354)(608.1451596,337.37226354)(608.11516451,337.37226336)
\curveto(608.06515968,337.36226355)(607.98515976,337.35726356)(607.87516451,337.35726336)
\curveto(607.76515998,337.34726357)(607.68016007,337.35226356)(607.62016451,337.37226336)
\lineto(607.47016451,337.37226336)
\curveto(607.42016033,337.38226353)(607.36516038,337.38726353)(607.30516451,337.38726336)
\curveto(607.25516049,337.37726354)(607.20516054,337.38226353)(607.15516451,337.40226336)
\curveto(607.11516063,337.4122635)(607.07516067,337.4172635)(607.03516451,337.41726336)
\curveto(607.00516074,337.4172635)(606.96516078,337.42226349)(606.91516451,337.43226336)
\curveto(606.81516093,337.46226345)(606.71516103,337.48726343)(606.61516451,337.50726336)
\curveto(606.51516123,337.52726339)(606.42016133,337.55726336)(606.33016451,337.59726336)
\curveto(606.21016154,337.63726328)(606.09516165,337.67726324)(605.98516451,337.71726336)
\curveto(605.88516186,337.75726316)(605.78016197,337.80726311)(605.67016451,337.86726336)
\curveto(605.32016243,338.07726284)(605.02016273,338.32226259)(604.77016451,338.60226336)
\curveto(604.52016323,338.88226203)(604.31016344,339.2172617)(604.14016451,339.60726336)
\curveto(604.09016366,339.69726122)(604.0501637,339.79226112)(604.02016451,339.89226336)
\curveto(604.00016375,339.99226092)(603.97516377,340.09726082)(603.94516451,340.20726336)
\curveto(603.92516382,340.25726066)(603.91516383,340.30226061)(603.91516451,340.34226336)
\curveto(603.91516383,340.38226053)(603.90516384,340.42726049)(603.88516451,340.47726336)
\curveto(603.86516388,340.55726036)(603.85516389,340.63726028)(603.85516451,340.71726336)
\curveto(603.85516389,340.80726011)(603.8451639,340.89226002)(603.82516451,340.97226336)
\curveto(603.81516393,341.02225989)(603.81016394,341.06725985)(603.81016451,341.10726336)
\lineto(603.81016451,341.24226336)
\curveto(603.79016396,341.30225961)(603.78016397,341.38725953)(603.78016451,341.49726336)
\curveto(603.79016396,341.60725931)(603.80516394,341.69225922)(603.82516451,341.75226336)
\lineto(603.82516451,341.85726336)
\curveto(603.83516391,341.90725901)(603.83516391,341.95725896)(603.82516451,342.00726336)
\curveto(603.82516392,342.06725885)(603.83516391,342.12225879)(603.85516451,342.17226336)
\curveto(603.86516388,342.22225869)(603.87016388,342.26725865)(603.87016451,342.30726336)
\curveto(603.87016388,342.35725856)(603.88016387,342.40725851)(603.90016451,342.45726336)
\curveto(603.94016381,342.58725833)(603.97516377,342.7122582)(604.00516451,342.83226336)
\curveto(604.03516371,342.96225795)(604.07516367,343.08725783)(604.12516451,343.20726336)
\curveto(604.30516344,343.6172573)(604.52016323,343.95725696)(604.77016451,344.22726336)
\curveto(605.02016273,344.50725641)(605.32516242,344.76225615)(605.68516451,344.99226336)
\curveto(605.78516196,345.04225587)(605.89016186,345.08725583)(606.00016451,345.12726336)
\curveto(606.11016164,345.16725575)(606.22016153,345.2122557)(606.33016451,345.26226336)
\curveto(606.46016129,345.3122556)(606.59516115,345.34725557)(606.73516451,345.36726336)
\curveto(606.87516087,345.38725553)(607.02016073,345.4172555)(607.17016451,345.45726336)
\curveto(607.2501605,345.46725545)(607.32516042,345.47225544)(607.39516451,345.47226336)
\curveto(607.46516028,345.47225544)(607.53516021,345.47725544)(607.60516451,345.48726336)
\curveto(608.18515956,345.49725542)(608.68515906,345.43725548)(609.10516451,345.30726336)
\curveto(609.53515821,345.17725574)(609.91515783,344.99725592)(610.24516451,344.76726336)
\curveto(610.35515739,344.68725623)(610.46515728,344.59725632)(610.57516451,344.49726336)
\curveto(610.69515705,344.40725651)(610.79515695,344.30725661)(610.87516451,344.19726336)
\curveto(610.95515679,344.09725682)(611.02515672,343.99725692)(611.08516451,343.89726336)
\curveto(611.15515659,343.79725712)(611.22515652,343.69225722)(611.29516451,343.58226336)
\curveto(611.36515638,343.47225744)(611.42015633,343.35225756)(611.46016451,343.22226336)
\curveto(611.50015625,343.10225781)(611.5451562,342.97225794)(611.59516451,342.83226336)
\curveto(611.62515612,342.75225816)(611.6501561,342.66725825)(611.67016451,342.57726336)
\lineto(611.73016451,342.30726336)
\curveto(611.74015601,342.26725865)(611.745156,342.22725869)(611.74516451,342.18726336)
\curveto(611.745156,342.14725877)(611.750156,342.10725881)(611.76016451,342.06726336)
\curveto(611.78015597,342.0172589)(611.78515596,341.96225895)(611.77516451,341.90226336)
\curveto(611.76515598,341.84225907)(611.77015598,341.78725913)(611.79016451,341.73726336)
\moveto(609.69016451,341.19726336)
\curveto(609.70015805,341.24725967)(609.70515804,341.3172596)(609.70516451,341.40726336)
\curveto(609.70515804,341.50725941)(609.70015805,341.58225933)(609.69016451,341.63226336)
\lineto(609.69016451,341.75226336)
\curveto(609.67015808,341.80225911)(609.66015809,341.85725906)(609.66016451,341.91726336)
\curveto(609.66015809,341.97725894)(609.65515809,342.03225888)(609.64516451,342.08226336)
\curveto(609.6451581,342.12225879)(609.64015811,342.15225876)(609.63016451,342.17226336)
\lineto(609.57016451,342.41226336)
\curveto(609.56015819,342.50225841)(609.54015821,342.58725833)(609.51016451,342.66726336)
\curveto(609.40015835,342.92725799)(609.27015848,343.14725777)(609.12016451,343.32726336)
\curveto(608.97015878,343.5172574)(608.77015898,343.66725725)(608.52016451,343.77726336)
\curveto(608.46015929,343.79725712)(608.40015935,343.8122571)(608.34016451,343.82226336)
\curveto(608.28015947,343.84225707)(608.21515953,343.86225705)(608.14516451,343.88226336)
\curveto(608.06515968,343.90225701)(607.98015977,343.90725701)(607.89016451,343.89726336)
\lineto(607.62016451,343.89726336)
\curveto(607.59016016,343.87725704)(607.55516019,343.86725705)(607.51516451,343.86726336)
\curveto(607.47516027,343.87725704)(607.44016031,343.87725704)(607.41016451,343.86726336)
\lineto(607.20016451,343.80726336)
\curveto(607.14016061,343.79725712)(607.08516066,343.77725714)(607.03516451,343.74726336)
\curveto(606.78516096,343.63725728)(606.58016117,343.47725744)(606.42016451,343.26726336)
\curveto(606.27016148,343.06725785)(606.1501616,342.83225808)(606.06016451,342.56226336)
\curveto(606.03016172,342.46225845)(606.00516174,342.35725856)(605.98516451,342.24726336)
\curveto(605.97516177,342.13725878)(605.96016179,342.02725889)(605.94016451,341.91726336)
\curveto(605.93016182,341.86725905)(605.92516182,341.8172591)(605.92516451,341.76726336)
\lineto(605.92516451,341.61726336)
\curveto(605.90516184,341.54725937)(605.89516185,341.44225947)(605.89516451,341.30226336)
\curveto(605.90516184,341.16225975)(605.92016183,341.05725986)(605.94016451,340.98726336)
\lineto(605.94016451,340.85226336)
\curveto(605.96016179,340.77226014)(605.97516177,340.69226022)(605.98516451,340.61226336)
\curveto(605.99516175,340.54226037)(606.01016174,340.46726045)(606.03016451,340.38726336)
\curveto(606.13016162,340.08726083)(606.23516151,339.84226107)(606.34516451,339.65226336)
\curveto(606.46516128,339.47226144)(606.6501611,339.30726161)(606.90016451,339.15726336)
\curveto(606.97016078,339.10726181)(607.0451607,339.06726185)(607.12516451,339.03726336)
\curveto(607.21516053,339.00726191)(607.30516044,338.98226193)(607.39516451,338.96226336)
\curveto(607.43516031,338.95226196)(607.47016028,338.94726197)(607.50016451,338.94726336)
\curveto(607.53016022,338.95726196)(607.56516018,338.95726196)(607.60516451,338.94726336)
\lineto(607.72516451,338.91726336)
\curveto(607.77515997,338.917262)(607.82015993,338.92226199)(607.86016451,338.93226336)
\lineto(607.98016451,338.93226336)
\curveto(608.06015969,338.95226196)(608.14015961,338.96726195)(608.22016451,338.97726336)
\curveto(608.30015945,338.98726193)(608.37515937,339.00726191)(608.44516451,339.03726336)
\curveto(608.70515904,339.13726178)(608.91515883,339.27226164)(609.07516451,339.44226336)
\curveto(609.23515851,339.6122613)(609.37015838,339.82226109)(609.48016451,340.07226336)
\curveto(609.52015823,340.17226074)(609.5501582,340.27226064)(609.57016451,340.37226336)
\curveto(609.59015816,340.47226044)(609.61515813,340.57726034)(609.64516451,340.68726336)
\curveto(609.65515809,340.72726019)(609.66015809,340.76226015)(609.66016451,340.79226336)
\curveto(609.66015809,340.83226008)(609.66515808,340.87226004)(609.67516451,340.91226336)
\lineto(609.67516451,341.04726336)
\curveto(609.67515807,341.09725982)(609.68015807,341.14725977)(609.69016451,341.19726336)
}
}
{
\newrgbcolor{curcolor}{0 0 0}
\pscustom[linestyle=none,fillstyle=solid,fillcolor=curcolor]
{
\newpath
\moveto(617.61508638,345.48726336)
\curveto(617.72508107,345.48725543)(617.82008097,345.47725544)(617.90008638,345.45726336)
\curveto(617.9900808,345.43725548)(618.06008073,345.39225552)(618.11008638,345.32226336)
\curveto(618.17008062,345.24225567)(618.20008059,345.10225581)(618.20008638,344.90226336)
\lineto(618.20008638,344.39226336)
\lineto(618.20008638,344.01726336)
\curveto(618.21008058,343.87725704)(618.1950806,343.76725715)(618.15508638,343.68726336)
\curveto(618.11508068,343.6172573)(618.05508074,343.57225734)(617.97508638,343.55226336)
\curveto(617.90508089,343.53225738)(617.82008097,343.52225739)(617.72008638,343.52226336)
\curveto(617.63008116,343.52225739)(617.53008126,343.52725739)(617.42008638,343.53726336)
\curveto(617.32008147,343.54725737)(617.22508157,343.54225737)(617.13508638,343.52226336)
\curveto(617.06508173,343.50225741)(616.9950818,343.48725743)(616.92508638,343.47726336)
\curveto(616.85508194,343.47725744)(616.790082,343.46725745)(616.73008638,343.44726336)
\curveto(616.57008222,343.39725752)(616.41008238,343.32225759)(616.25008638,343.22226336)
\curveto(616.0900827,343.13225778)(615.96508283,343.02725789)(615.87508638,342.90726336)
\curveto(615.82508297,342.82725809)(615.77008302,342.74225817)(615.71008638,342.65226336)
\curveto(615.66008313,342.57225834)(615.61008318,342.48725843)(615.56008638,342.39726336)
\curveto(615.53008326,342.3172586)(615.50008329,342.23225868)(615.47008638,342.14226336)
\lineto(615.41008638,341.90226336)
\curveto(615.3900834,341.83225908)(615.38008341,341.75725916)(615.38008638,341.67726336)
\curveto(615.38008341,341.60725931)(615.37008342,341.53725938)(615.35008638,341.46726336)
\curveto(615.34008345,341.42725949)(615.33508346,341.38725953)(615.33508638,341.34726336)
\curveto(615.34508345,341.3172596)(615.34508345,341.28725963)(615.33508638,341.25726336)
\lineto(615.33508638,341.01726336)
\curveto(615.31508348,340.94725997)(615.31008348,340.86726005)(615.32008638,340.77726336)
\curveto(615.33008346,340.69726022)(615.33508346,340.6172603)(615.33508638,340.53726336)
\lineto(615.33508638,339.57726336)
\lineto(615.33508638,338.30226336)
\curveto(615.33508346,338.17226274)(615.33008346,338.05226286)(615.32008638,337.94226336)
\curveto(615.31008348,337.83226308)(615.28008351,337.74226317)(615.23008638,337.67226336)
\curveto(615.21008358,337.64226327)(615.17508362,337.6172633)(615.12508638,337.59726336)
\curveto(615.08508371,337.58726333)(615.04008375,337.57726334)(614.99008638,337.56726336)
\lineto(614.91508638,337.56726336)
\curveto(614.86508393,337.55726336)(614.81008398,337.55226336)(614.75008638,337.55226336)
\lineto(614.58508638,337.55226336)
\lineto(613.94008638,337.55226336)
\curveto(613.88008491,337.56226335)(613.81508498,337.56726335)(613.74508638,337.56726336)
\lineto(613.55008638,337.56726336)
\curveto(613.50008529,337.58726333)(613.45008534,337.60226331)(613.40008638,337.61226336)
\curveto(613.35008544,337.63226328)(613.31508548,337.66726325)(613.29508638,337.71726336)
\curveto(613.25508554,337.76726315)(613.23008556,337.83726308)(613.22008638,337.92726336)
\lineto(613.22008638,338.22726336)
\lineto(613.22008638,339.24726336)
\lineto(613.22008638,343.47726336)
\lineto(613.22008638,344.58726336)
\lineto(613.22008638,344.87226336)
\curveto(613.22008557,344.97225594)(613.24008555,345.05225586)(613.28008638,345.11226336)
\curveto(613.33008546,345.19225572)(613.40508539,345.24225567)(613.50508638,345.26226336)
\curveto(613.60508519,345.28225563)(613.72508507,345.29225562)(613.86508638,345.29226336)
\lineto(614.63008638,345.29226336)
\curveto(614.75008404,345.29225562)(614.85508394,345.28225563)(614.94508638,345.26226336)
\curveto(615.03508376,345.25225566)(615.10508369,345.20725571)(615.15508638,345.12726336)
\curveto(615.18508361,345.07725584)(615.20008359,345.00725591)(615.20008638,344.91726336)
\lineto(615.23008638,344.64726336)
\curveto(615.24008355,344.56725635)(615.25508354,344.49225642)(615.27508638,344.42226336)
\curveto(615.30508349,344.35225656)(615.35508344,344.3172566)(615.42508638,344.31726336)
\curveto(615.44508335,344.33725658)(615.46508333,344.34725657)(615.48508638,344.34726336)
\curveto(615.50508329,344.34725657)(615.52508327,344.35725656)(615.54508638,344.37726336)
\curveto(615.60508319,344.42725649)(615.65508314,344.48225643)(615.69508638,344.54226336)
\curveto(615.74508305,344.6122563)(615.80508299,344.67225624)(615.87508638,344.72226336)
\curveto(615.91508288,344.75225616)(615.95008284,344.78225613)(615.98008638,344.81226336)
\curveto(616.01008278,344.85225606)(616.04508275,344.88725603)(616.08508638,344.91726336)
\lineto(616.35508638,345.09726336)
\curveto(616.45508234,345.15725576)(616.55508224,345.2122557)(616.65508638,345.26226336)
\curveto(616.75508204,345.30225561)(616.85508194,345.33725558)(616.95508638,345.36726336)
\lineto(617.28508638,345.45726336)
\curveto(617.31508148,345.46725545)(617.37008142,345.46725545)(617.45008638,345.45726336)
\curveto(617.54008125,345.45725546)(617.5950812,345.46725545)(617.61508638,345.48726336)
}
}
{
\newrgbcolor{curcolor}{0 0 0}
\pscustom[linestyle=none,fillstyle=solid,fillcolor=curcolor]
{
\newpath
\moveto(626.12149263,341.49726336)
\curveto(626.14148447,341.4172595)(626.14148447,341.32725959)(626.12149263,341.22726336)
\curveto(626.10148451,341.12725979)(626.06648454,341.06225985)(626.01649263,341.03226336)
\curveto(625.96648464,340.99225992)(625.89148472,340.96225995)(625.79149263,340.94226336)
\curveto(625.70148491,340.93225998)(625.59648501,340.92225999)(625.47649263,340.91226336)
\lineto(625.13149263,340.91226336)
\curveto(625.02148559,340.92225999)(624.92148569,340.92725999)(624.83149263,340.92726336)
\lineto(621.17149263,340.92726336)
\lineto(620.96149263,340.92726336)
\curveto(620.90148971,340.92725999)(620.84648976,340.91726)(620.79649263,340.89726336)
\curveto(620.71648989,340.85726006)(620.66648994,340.8172601)(620.64649263,340.77726336)
\curveto(620.62648998,340.75726016)(620.60649,340.7172602)(620.58649263,340.65726336)
\curveto(620.56649004,340.60726031)(620.56149005,340.55726036)(620.57149263,340.50726336)
\curveto(620.59149002,340.44726047)(620.60149001,340.38726053)(620.60149263,340.32726336)
\curveto(620.61149,340.27726064)(620.62648998,340.22226069)(620.64649263,340.16226336)
\curveto(620.72648988,339.92226099)(620.82148979,339.72226119)(620.93149263,339.56226336)
\curveto(621.05148956,339.4122615)(621.2114894,339.27726164)(621.41149263,339.15726336)
\curveto(621.49148912,339.10726181)(621.57148904,339.07226184)(621.65149263,339.05226336)
\curveto(621.74148887,339.04226187)(621.83148878,339.02226189)(621.92149263,338.99226336)
\curveto(622.00148861,338.97226194)(622.1114885,338.95726196)(622.25149263,338.94726336)
\curveto(622.39148822,338.93726198)(622.5114881,338.94226197)(622.61149263,338.96226336)
\lineto(622.74649263,338.96226336)
\curveto(622.84648776,338.98226193)(622.93648767,339.00226191)(623.01649263,339.02226336)
\curveto(623.1064875,339.05226186)(623.19148742,339.08226183)(623.27149263,339.11226336)
\curveto(623.37148724,339.16226175)(623.48148713,339.22726169)(623.60149263,339.30726336)
\curveto(623.73148688,339.38726153)(623.82648678,339.46726145)(623.88649263,339.54726336)
\curveto(623.93648667,339.6172613)(623.98648662,339.68226123)(624.03649263,339.74226336)
\curveto(624.09648651,339.8122611)(624.16648644,339.86226105)(624.24649263,339.89226336)
\curveto(624.34648626,339.94226097)(624.47148614,339.96226095)(624.62149263,339.95226336)
\lineto(625.05649263,339.95226336)
\lineto(625.23649263,339.95226336)
\curveto(625.3064853,339.96226095)(625.36648524,339.95726096)(625.41649263,339.93726336)
\lineto(625.56649263,339.93726336)
\curveto(625.66648494,339.917261)(625.73648487,339.89226102)(625.77649263,339.86226336)
\curveto(625.81648479,339.84226107)(625.83648477,339.79726112)(625.83649263,339.72726336)
\curveto(625.84648476,339.65726126)(625.84148477,339.59726132)(625.82149263,339.54726336)
\curveto(625.77148484,339.40726151)(625.71648489,339.28226163)(625.65649263,339.17226336)
\curveto(625.59648501,339.06226185)(625.52648508,338.95226196)(625.44649263,338.84226336)
\curveto(625.22648538,338.5122624)(624.97648563,338.24726267)(624.69649263,338.04726336)
\curveto(624.41648619,337.84726307)(624.06648654,337.67726324)(623.64649263,337.53726336)
\curveto(623.53648707,337.49726342)(623.42648718,337.47226344)(623.31649263,337.46226336)
\curveto(623.2064874,337.45226346)(623.09148752,337.43226348)(622.97149263,337.40226336)
\curveto(622.93148768,337.39226352)(622.88648772,337.39226352)(622.83649263,337.40226336)
\curveto(622.79648781,337.40226351)(622.75648785,337.39726352)(622.71649263,337.38726336)
\lineto(622.55149263,337.38726336)
\curveto(622.50148811,337.36726355)(622.44148817,337.36226355)(622.37149263,337.37226336)
\curveto(622.3114883,337.37226354)(622.25648835,337.37726354)(622.20649263,337.38726336)
\curveto(622.12648848,337.39726352)(622.05648855,337.39726352)(621.99649263,337.38726336)
\curveto(621.93648867,337.37726354)(621.87148874,337.38226353)(621.80149263,337.40226336)
\curveto(621.75148886,337.42226349)(621.69648891,337.43226348)(621.63649263,337.43226336)
\curveto(621.57648903,337.43226348)(621.52148909,337.44226347)(621.47149263,337.46226336)
\curveto(621.36148925,337.48226343)(621.25148936,337.50726341)(621.14149263,337.53726336)
\curveto(621.03148958,337.55726336)(620.93148968,337.59226332)(620.84149263,337.64226336)
\curveto(620.73148988,337.68226323)(620.62648998,337.7172632)(620.52649263,337.74726336)
\curveto(620.43649017,337.78726313)(620.35149026,337.83226308)(620.27149263,337.88226336)
\curveto(619.95149066,338.08226283)(619.66649094,338.3122626)(619.41649263,338.57226336)
\curveto(619.16649144,338.84226207)(618.96149165,339.15226176)(618.80149263,339.50226336)
\curveto(618.75149186,339.6122613)(618.7114919,339.72226119)(618.68149263,339.83226336)
\curveto(618.65149196,339.95226096)(618.611492,340.07226084)(618.56149263,340.19226336)
\curveto(618.55149206,340.23226068)(618.54649206,340.26726065)(618.54649263,340.29726336)
\curveto(618.54649206,340.33726058)(618.54149207,340.37726054)(618.53149263,340.41726336)
\curveto(618.49149212,340.53726038)(618.46649214,340.66726025)(618.45649263,340.80726336)
\lineto(618.42649263,341.22726336)
\curveto(618.42649218,341.27725964)(618.42149219,341.33225958)(618.41149263,341.39226336)
\curveto(618.4114922,341.45225946)(618.41649219,341.50725941)(618.42649263,341.55726336)
\lineto(618.42649263,341.73726336)
\lineto(618.47149263,342.09726336)
\curveto(618.5114921,342.26725865)(618.54649206,342.43225848)(618.57649263,342.59226336)
\curveto(618.606492,342.75225816)(618.65149196,342.90225801)(618.71149263,343.04226336)
\curveto(619.14149147,344.08225683)(619.87149074,344.8172561)(620.90149263,345.24726336)
\curveto(621.04148957,345.30725561)(621.18148943,345.34725557)(621.32149263,345.36726336)
\curveto(621.47148914,345.39725552)(621.62648898,345.43225548)(621.78649263,345.47226336)
\curveto(621.86648874,345.48225543)(621.94148867,345.48725543)(622.01149263,345.48726336)
\curveto(622.08148853,345.48725543)(622.15648845,345.49225542)(622.23649263,345.50226336)
\curveto(622.74648786,345.5122554)(623.18148743,345.45225546)(623.54149263,345.32226336)
\curveto(623.9114867,345.20225571)(624.24148637,345.04225587)(624.53149263,344.84226336)
\curveto(624.62148599,344.78225613)(624.7114859,344.7122562)(624.80149263,344.63226336)
\curveto(624.89148572,344.56225635)(624.97148564,344.48725643)(625.04149263,344.40726336)
\curveto(625.07148554,344.35725656)(625.1114855,344.3172566)(625.16149263,344.28726336)
\curveto(625.24148537,344.17725674)(625.31648529,344.06225685)(625.38649263,343.94226336)
\curveto(625.45648515,343.83225708)(625.53148508,343.7172572)(625.61149263,343.59726336)
\curveto(625.66148495,343.50725741)(625.70148491,343.4122575)(625.73149263,343.31226336)
\curveto(625.77148484,343.22225769)(625.8114848,343.12225779)(625.85149263,343.01226336)
\curveto(625.90148471,342.88225803)(625.94148467,342.74725817)(625.97149263,342.60726336)
\curveto(626.00148461,342.46725845)(626.03648457,342.32725859)(626.07649263,342.18726336)
\curveto(626.09648451,342.10725881)(626.10148451,342.0172589)(626.09149263,341.91726336)
\curveto(626.09148452,341.82725909)(626.10148451,341.74225917)(626.12149263,341.66226336)
\lineto(626.12149263,341.49726336)
\moveto(623.87149263,342.38226336)
\curveto(623.94148667,342.48225843)(623.94648666,342.60225831)(623.88649263,342.74226336)
\curveto(623.83648677,342.89225802)(623.79648681,343.00225791)(623.76649263,343.07226336)
\curveto(623.62648698,343.34225757)(623.44148717,343.54725737)(623.21149263,343.68726336)
\curveto(622.98148763,343.83725708)(622.66148795,343.917257)(622.25149263,343.92726336)
\curveto(622.22148839,343.90725701)(622.18648842,343.90225701)(622.14649263,343.91226336)
\curveto(622.1064885,343.92225699)(622.07148854,343.92225699)(622.04149263,343.91226336)
\curveto(621.99148862,343.89225702)(621.93648867,343.87725704)(621.87649263,343.86726336)
\curveto(621.81648879,343.86725705)(621.76148885,343.85725706)(621.71149263,343.83726336)
\curveto(621.27148934,343.69725722)(620.94648966,343.42225749)(620.73649263,343.01226336)
\curveto(620.71648989,342.97225794)(620.69148992,342.917258)(620.66149263,342.84726336)
\curveto(620.64148997,342.78725813)(620.62648998,342.72225819)(620.61649263,342.65226336)
\curveto(620.60649,342.59225832)(620.60649,342.53225838)(620.61649263,342.47226336)
\curveto(620.63648997,342.4122585)(620.67148994,342.36225855)(620.72149263,342.32226336)
\curveto(620.80148981,342.27225864)(620.9114897,342.24725867)(621.05149263,342.24726336)
\lineto(621.45649263,342.24726336)
\lineto(623.12149263,342.24726336)
\lineto(623.55649263,342.24726336)
\curveto(623.71648689,342.25725866)(623.82148679,342.30225861)(623.87149263,342.38226336)
}
}
{
\newrgbcolor{curcolor}{0 0 0}
\pscustom[linestyle=none,fillstyle=solid,fillcolor=curcolor]
{
\newpath
\moveto(630.33977388,345.50226336)
\curveto(631.08976938,345.52225539)(631.73976873,345.43725548)(632.28977388,345.24726336)
\curveto(632.84976762,345.06725585)(633.2747672,344.75225616)(633.56477388,344.30226336)
\curveto(633.63476684,344.19225672)(633.69476678,344.07725684)(633.74477388,343.95726336)
\curveto(633.80476667,343.84725707)(633.85476662,343.72225719)(633.89477388,343.58226336)
\curveto(633.91476656,343.52225739)(633.92476655,343.45725746)(633.92477388,343.38726336)
\curveto(633.92476655,343.3172576)(633.91476656,343.25725766)(633.89477388,343.20726336)
\curveto(633.85476662,343.14725777)(633.79976667,343.10725781)(633.72977388,343.08726336)
\curveto(633.67976679,343.06725785)(633.61976685,343.05725786)(633.54977388,343.05726336)
\lineto(633.33977388,343.05726336)
\lineto(632.67977388,343.05726336)
\curveto(632.60976786,343.05725786)(632.53976793,343.05225786)(632.46977388,343.04226336)
\curveto(632.39976807,343.04225787)(632.33476814,343.05225786)(632.27477388,343.07226336)
\curveto(632.1747683,343.09225782)(632.09976837,343.13225778)(632.04977388,343.19226336)
\curveto(631.99976847,343.25225766)(631.95476852,343.3122576)(631.91477388,343.37226336)
\lineto(631.79477388,343.58226336)
\curveto(631.76476871,343.66225725)(631.71476876,343.72725719)(631.64477388,343.77726336)
\curveto(631.54476893,343.85725706)(631.44476903,343.917257)(631.34477388,343.95726336)
\curveto(631.25476922,343.99725692)(631.13976933,344.03225688)(630.99977388,344.06226336)
\curveto(630.92976954,344.08225683)(630.82476965,344.09725682)(630.68477388,344.10726336)
\curveto(630.55476992,344.1172568)(630.45477002,344.1122568)(630.38477388,344.09226336)
\lineto(630.27977388,344.09226336)
\lineto(630.12977388,344.06226336)
\curveto(630.08977038,344.06225685)(630.04477043,344.05725686)(629.99477388,344.04726336)
\curveto(629.82477065,343.99725692)(629.68477079,343.92725699)(629.57477388,343.83726336)
\curveto(629.474771,343.75725716)(629.40477107,343.63225728)(629.36477388,343.46226336)
\curveto(629.34477113,343.39225752)(629.34477113,343.32725759)(629.36477388,343.26726336)
\curveto(629.38477109,343.20725771)(629.40477107,343.15725776)(629.42477388,343.11726336)
\curveto(629.49477098,342.99725792)(629.5747709,342.90225801)(629.66477388,342.83226336)
\curveto(629.76477071,342.76225815)(629.87977059,342.70225821)(630.00977388,342.65226336)
\curveto(630.19977027,342.57225834)(630.40477007,342.50225841)(630.62477388,342.44226336)
\lineto(631.31477388,342.29226336)
\curveto(631.55476892,342.25225866)(631.78476869,342.20225871)(632.00477388,342.14226336)
\curveto(632.23476824,342.09225882)(632.44976802,342.02725889)(632.64977388,341.94726336)
\curveto(632.73976773,341.90725901)(632.82476765,341.87225904)(632.90477388,341.84226336)
\curveto(632.99476748,341.82225909)(633.07976739,341.78725913)(633.15977388,341.73726336)
\curveto(633.34976712,341.6172593)(633.51976695,341.48725943)(633.66977388,341.34726336)
\curveto(633.82976664,341.20725971)(633.95476652,341.03225988)(634.04477388,340.82226336)
\curveto(634.0747664,340.75226016)(634.09976637,340.68226023)(634.11977388,340.61226336)
\curveto(634.13976633,340.54226037)(634.15976631,340.46726045)(634.17977388,340.38726336)
\curveto(634.18976628,340.32726059)(634.19476628,340.23226068)(634.19477388,340.10226336)
\curveto(634.20476627,339.98226093)(634.20476627,339.88726103)(634.19477388,339.81726336)
\lineto(634.19477388,339.74226336)
\curveto(634.1747663,339.68226123)(634.15976631,339.62226129)(634.14977388,339.56226336)
\curveto(634.14976632,339.5122614)(634.14476633,339.46226145)(634.13477388,339.41226336)
\curveto(634.06476641,339.1122618)(633.95476652,338.84726207)(633.80477388,338.61726336)
\curveto(633.64476683,338.37726254)(633.44976702,338.18226273)(633.21977388,338.03226336)
\curveto(632.98976748,337.88226303)(632.72976774,337.75226316)(632.43977388,337.64226336)
\curveto(632.32976814,337.59226332)(632.20976826,337.55726336)(632.07977388,337.53726336)
\curveto(631.95976851,337.5172634)(631.83976863,337.49226342)(631.71977388,337.46226336)
\curveto(631.62976884,337.44226347)(631.53476894,337.43226348)(631.43477388,337.43226336)
\curveto(631.34476913,337.42226349)(631.25476922,337.40726351)(631.16477388,337.38726336)
\lineto(630.89477388,337.38726336)
\curveto(630.83476964,337.36726355)(630.72976974,337.35726356)(630.57977388,337.35726336)
\curveto(630.43977003,337.35726356)(630.33977013,337.36726355)(630.27977388,337.38726336)
\curveto(630.24977022,337.38726353)(630.21477026,337.39226352)(630.17477388,337.40226336)
\lineto(630.06977388,337.40226336)
\curveto(629.94977052,337.42226349)(629.82977064,337.43726348)(629.70977388,337.44726336)
\curveto(629.58977088,337.45726346)(629.474771,337.47726344)(629.36477388,337.50726336)
\curveto(628.9747715,337.6172633)(628.62977184,337.74226317)(628.32977388,337.88226336)
\curveto(628.02977244,338.03226288)(627.7747727,338.25226266)(627.56477388,338.54226336)
\curveto(627.42477305,338.73226218)(627.30477317,338.95226196)(627.20477388,339.20226336)
\curveto(627.18477329,339.26226165)(627.16477331,339.34226157)(627.14477388,339.44226336)
\curveto(627.12477335,339.49226142)(627.10977336,339.56226135)(627.09977388,339.65226336)
\curveto(627.08977338,339.74226117)(627.09477338,339.8172611)(627.11477388,339.87726336)
\curveto(627.14477333,339.94726097)(627.19477328,339.99726092)(627.26477388,340.02726336)
\curveto(627.31477316,340.04726087)(627.3747731,340.05726086)(627.44477388,340.05726336)
\lineto(627.66977388,340.05726336)
\lineto(628.37477388,340.05726336)
\lineto(628.61477388,340.05726336)
\curveto(628.69477178,340.05726086)(628.76477171,340.04726087)(628.82477388,340.02726336)
\curveto(628.93477154,339.98726093)(629.00477147,339.92226099)(629.03477388,339.83226336)
\curveto(629.0747714,339.74226117)(629.11977135,339.64726127)(629.16977388,339.54726336)
\curveto(629.18977128,339.49726142)(629.22477125,339.43226148)(629.27477388,339.35226336)
\curveto(629.33477114,339.27226164)(629.38477109,339.22226169)(629.42477388,339.20226336)
\curveto(629.54477093,339.10226181)(629.65977081,339.02226189)(629.76977388,338.96226336)
\curveto(629.87977059,338.912262)(630.01977045,338.86226205)(630.18977388,338.81226336)
\curveto(630.23977023,338.79226212)(630.28977018,338.78226213)(630.33977388,338.78226336)
\curveto(630.38977008,338.79226212)(630.43977003,338.79226212)(630.48977388,338.78226336)
\curveto(630.5697699,338.76226215)(630.65476982,338.75226216)(630.74477388,338.75226336)
\curveto(630.84476963,338.76226215)(630.92976954,338.77726214)(630.99977388,338.79726336)
\curveto(631.04976942,338.80726211)(631.09476938,338.8122621)(631.13477388,338.81226336)
\curveto(631.18476929,338.8122621)(631.23476924,338.82226209)(631.28477388,338.84226336)
\curveto(631.42476905,338.89226202)(631.54976892,338.95226196)(631.65977388,339.02226336)
\curveto(631.77976869,339.09226182)(631.8747686,339.18226173)(631.94477388,339.29226336)
\curveto(631.99476848,339.37226154)(632.03476844,339.49726142)(632.06477388,339.66726336)
\curveto(632.08476839,339.73726118)(632.08476839,339.80226111)(632.06477388,339.86226336)
\curveto(632.04476843,339.92226099)(632.02476845,339.97226094)(632.00477388,340.01226336)
\curveto(631.93476854,340.15226076)(631.84476863,340.25726066)(631.73477388,340.32726336)
\curveto(631.63476884,340.39726052)(631.51476896,340.46226045)(631.37477388,340.52226336)
\curveto(631.18476929,340.60226031)(630.98476949,340.66726025)(630.77477388,340.71726336)
\curveto(630.56476991,340.76726015)(630.35477012,340.82226009)(630.14477388,340.88226336)
\curveto(630.06477041,340.90226001)(629.97977049,340.91726)(629.88977388,340.92726336)
\curveto(629.80977066,340.93725998)(629.72977074,340.95225996)(629.64977388,340.97226336)
\curveto(629.32977114,341.06225985)(629.02477145,341.14725977)(628.73477388,341.22726336)
\curveto(628.44477203,341.3172596)(628.17977229,341.44725947)(627.93977388,341.61726336)
\curveto(627.65977281,341.8172591)(627.45477302,342.08725883)(627.32477388,342.42726336)
\curveto(627.30477317,342.49725842)(627.28477319,342.59225832)(627.26477388,342.71226336)
\curveto(627.24477323,342.78225813)(627.22977324,342.86725805)(627.21977388,342.96726336)
\curveto(627.20977326,343.06725785)(627.21477326,343.15725776)(627.23477388,343.23726336)
\curveto(627.25477322,343.28725763)(627.25977321,343.32725759)(627.24977388,343.35726336)
\curveto(627.23977323,343.39725752)(627.24477323,343.44225747)(627.26477388,343.49226336)
\curveto(627.28477319,343.60225731)(627.30477317,343.70225721)(627.32477388,343.79226336)
\curveto(627.35477312,343.89225702)(627.38977308,343.98725693)(627.42977388,344.07726336)
\curveto(627.55977291,344.36725655)(627.73977273,344.60225631)(627.96977388,344.78226336)
\curveto(628.19977227,344.96225595)(628.45977201,345.10725581)(628.74977388,345.21726336)
\curveto(628.85977161,345.26725565)(628.9747715,345.30225561)(629.09477388,345.32226336)
\curveto(629.21477126,345.35225556)(629.33977113,345.38225553)(629.46977388,345.41226336)
\curveto(629.52977094,345.43225548)(629.58977088,345.44225547)(629.64977388,345.44226336)
\lineto(629.82977388,345.47226336)
\curveto(629.90977056,345.48225543)(629.99477048,345.48725543)(630.08477388,345.48726336)
\curveto(630.1747703,345.48725543)(630.25977021,345.49225542)(630.33977388,345.50226336)
}
}
{
\newrgbcolor{curcolor}{0 0 0}
\pscustom[linestyle=none,fillstyle=solid,fillcolor=curcolor]
{
\newpath
\moveto(413.39453951,322.76226336)
\curveto(413.40453083,322.70225946)(413.40953082,322.61225955)(413.40953951,322.49226336)
\curveto(413.40953082,322.37225979)(413.39953083,322.28725988)(413.37953951,322.23726336)
\lineto(413.37953951,322.04226336)
\curveto(413.34953088,321.93226023)(413.3295309,321.82726034)(413.31953951,321.72726336)
\curveto(413.31953091,321.62726054)(413.30453093,321.52726064)(413.27453951,321.42726336)
\curveto(413.25453098,321.33726083)(413.234531,321.24226092)(413.21453951,321.14226336)
\curveto(413.19453104,321.05226111)(413.16453107,320.9622612)(413.12453951,320.87226336)
\curveto(413.05453118,320.70226146)(412.98453125,320.54226162)(412.91453951,320.39226336)
\curveto(412.84453139,320.25226191)(412.76453147,320.11226205)(412.67453951,319.97226336)
\curveto(412.61453162,319.88226228)(412.54953168,319.79726237)(412.47953951,319.71726336)
\curveto(412.41953181,319.64726252)(412.34953188,319.57226259)(412.26953951,319.49226336)
\lineto(412.16453951,319.38726336)
\curveto(412.11453212,319.33726283)(412.05953217,319.29226287)(411.99953951,319.25226336)
\lineto(411.84953951,319.13226336)
\curveto(411.76953246,319.07226309)(411.67953255,319.01726315)(411.57953951,318.96726336)
\curveto(411.48953274,318.92726324)(411.39453284,318.88226328)(411.29453951,318.83226336)
\curveto(411.19453304,318.78226338)(411.08953314,318.74726342)(410.97953951,318.72726336)
\curveto(410.87953335,318.70726346)(410.77453346,318.68726348)(410.66453951,318.66726336)
\curveto(410.60453363,318.64726352)(410.53953369,318.63726353)(410.46953951,318.63726336)
\curveto(410.40953382,318.63726353)(410.34453389,318.62726354)(410.27453951,318.60726336)
\lineto(410.13953951,318.60726336)
\curveto(410.05953417,318.58726358)(409.98453425,318.58726358)(409.91453951,318.60726336)
\lineto(409.76453951,318.60726336)
\curveto(409.70453453,318.62726354)(409.63953459,318.63726353)(409.56953951,318.63726336)
\curveto(409.50953472,318.62726354)(409.44953478,318.63226353)(409.38953951,318.65226336)
\curveto(409.229535,318.70226346)(409.07453516,318.74726342)(408.92453951,318.78726336)
\curveto(408.78453545,318.82726334)(408.65453558,318.88726328)(408.53453951,318.96726336)
\curveto(408.46453577,319.00726316)(408.39953583,319.04726312)(408.33953951,319.08726336)
\curveto(408.27953595,319.13726303)(408.21453602,319.18726298)(408.14453951,319.23726336)
\lineto(407.96453951,319.37226336)
\curveto(407.88453635,319.43226273)(407.81453642,319.43726273)(407.75453951,319.38726336)
\curveto(407.70453653,319.35726281)(407.67953655,319.31726285)(407.67953951,319.26726336)
\curveto(407.67953655,319.22726294)(407.66953656,319.17726299)(407.64953951,319.11726336)
\curveto(407.6295366,319.01726315)(407.61953661,318.90226326)(407.61953951,318.77226336)
\curveto(407.6295366,318.64226352)(407.6345366,318.52226364)(407.63453951,318.41226336)
\lineto(407.63453951,316.88226336)
\curveto(407.6345366,316.75226541)(407.6295366,316.62726554)(407.61953951,316.50726336)
\curveto(407.61953661,316.37726579)(407.59453664,316.27226589)(407.54453951,316.19226336)
\curveto(407.51453672,316.15226601)(407.45953677,316.12226604)(407.37953951,316.10226336)
\curveto(407.29953693,316.08226608)(407.20953702,316.07226609)(407.10953951,316.07226336)
\curveto(407.00953722,316.0622661)(406.90953732,316.0622661)(406.80953951,316.07226336)
\lineto(406.55453951,316.07226336)
\lineto(406.14953951,316.07226336)
\lineto(406.04453951,316.07226336)
\curveto(406.00453823,316.07226609)(405.96953826,316.07726609)(405.93953951,316.08726336)
\lineto(405.81953951,316.08726336)
\curveto(405.64953858,316.13726603)(405.55953867,316.23726593)(405.54953951,316.38726336)
\curveto(405.53953869,316.52726564)(405.5345387,316.69726547)(405.53453951,316.89726336)
\lineto(405.53453951,325.70226336)
\curveto(405.5345387,325.81225635)(405.5295387,325.92725624)(405.51953951,326.04726336)
\curveto(405.51953871,326.17725599)(405.54453869,326.27725589)(405.59453951,326.34726336)
\curveto(405.6345386,326.41725575)(405.68953854,326.4622557)(405.75953951,326.48226336)
\curveto(405.80953842,326.50225566)(405.86953836,326.51225565)(405.93953951,326.51226336)
\lineto(406.16453951,326.51226336)
\lineto(406.88453951,326.51226336)
\lineto(407.16953951,326.51226336)
\curveto(407.25953697,326.51225565)(407.3345369,326.48725568)(407.39453951,326.43726336)
\curveto(407.46453677,326.38725578)(407.49953673,326.32225584)(407.49953951,326.24226336)
\curveto(407.50953672,326.17225599)(407.5345367,326.09725607)(407.57453951,326.01726336)
\curveto(407.58453665,325.98725618)(407.59453664,325.9622562)(407.60453951,325.94226336)
\curveto(407.62453661,325.93225623)(407.64453659,325.91725625)(407.66453951,325.89726336)
\curveto(407.77453646,325.88725628)(407.86453637,325.91725625)(407.93453951,325.98726336)
\curveto(408.00453623,326.05725611)(408.07453616,326.11725605)(408.14453951,326.16726336)
\curveto(408.27453596,326.25725591)(408.40953582,326.33725583)(408.54953951,326.40726336)
\curveto(408.68953554,326.48725568)(408.84453539,326.55225561)(409.01453951,326.60226336)
\curveto(409.09453514,326.63225553)(409.17953505,326.65225551)(409.26953951,326.66226336)
\curveto(409.36953486,326.67225549)(409.46453477,326.68725548)(409.55453951,326.70726336)
\curveto(409.59453464,326.71725545)(409.6345346,326.71725545)(409.67453951,326.70726336)
\curveto(409.72453451,326.69725547)(409.76453447,326.70225546)(409.79453951,326.72226336)
\curveto(410.36453387,326.74225542)(410.84453339,326.6622555)(411.23453951,326.48226336)
\curveto(411.6345326,326.31225585)(411.97453226,326.08725608)(412.25453951,325.80726336)
\curveto(412.30453193,325.75725641)(412.34953188,325.70725646)(412.38953951,325.65726336)
\curveto(412.4295318,325.61725655)(412.46953176,325.57225659)(412.50953951,325.52226336)
\curveto(412.57953165,325.43225673)(412.63953159,325.34225682)(412.68953951,325.25226336)
\curveto(412.74953148,325.162257)(412.80453143,325.07225709)(412.85453951,324.98226336)
\curveto(412.87453136,324.9622572)(412.88453135,324.93725723)(412.88453951,324.90726336)
\curveto(412.89453134,324.87725729)(412.90953132,324.84225732)(412.92953951,324.80226336)
\curveto(412.98953124,324.70225746)(413.04453119,324.58225758)(413.09453951,324.44226336)
\curveto(413.11453112,324.38225778)(413.1345311,324.31725785)(413.15453951,324.24726336)
\curveto(413.17453106,324.18725798)(413.19453104,324.12225804)(413.21453951,324.05226336)
\curveto(413.25453098,323.93225823)(413.27953095,323.80725836)(413.28953951,323.67726336)
\curveto(413.30953092,323.54725862)(413.3345309,323.41225875)(413.36453951,323.27226336)
\lineto(413.36453951,323.10726336)
\lineto(413.39453951,322.92726336)
\lineto(413.39453951,322.76226336)
\moveto(411.27953951,322.41726336)
\curveto(411.28953294,322.4672597)(411.29453294,322.53225963)(411.29453951,322.61226336)
\curveto(411.29453294,322.70225946)(411.28953294,322.77225939)(411.27953951,322.82226336)
\lineto(411.27953951,322.95726336)
\curveto(411.25953297,323.01725915)(411.24953298,323.08225908)(411.24953951,323.15226336)
\curveto(411.24953298,323.22225894)(411.23953299,323.29225887)(411.21953951,323.36226336)
\curveto(411.19953303,323.4622587)(411.17953305,323.55725861)(411.15953951,323.64726336)
\curveto(411.13953309,323.74725842)(411.10953312,323.83725833)(411.06953951,323.91726336)
\curveto(410.94953328,324.23725793)(410.79453344,324.49225767)(410.60453951,324.68226336)
\curveto(410.41453382,324.87225729)(410.14453409,325.01225715)(409.79453951,325.10226336)
\curveto(409.71453452,325.12225704)(409.62453461,325.13225703)(409.52453951,325.13226336)
\lineto(409.25453951,325.13226336)
\curveto(409.21453502,325.12225704)(409.17953505,325.11725705)(409.14953951,325.11726336)
\curveto(409.11953511,325.11725705)(409.08453515,325.11225705)(409.04453951,325.10226336)
\lineto(408.83453951,325.04226336)
\curveto(408.77453546,325.03225713)(408.71453552,325.01225715)(408.65453951,324.98226336)
\curveto(408.39453584,324.87225729)(408.18953604,324.70225746)(408.03953951,324.47226336)
\curveto(407.89953633,324.24225792)(407.78453645,323.98725818)(407.69453951,323.70726336)
\curveto(407.67453656,323.62725854)(407.65953657,323.54225862)(407.64953951,323.45226336)
\curveto(407.63953659,323.37225879)(407.62453661,323.29225887)(407.60453951,323.21226336)
\curveto(407.59453664,323.17225899)(407.58953664,323.10725906)(407.58953951,323.01726336)
\curveto(407.56953666,322.97725919)(407.56453667,322.92725924)(407.57453951,322.86726336)
\curveto(407.58453665,322.81725935)(407.58453665,322.7672594)(407.57453951,322.71726336)
\curveto(407.55453668,322.65725951)(407.55453668,322.60225956)(407.57453951,322.55226336)
\lineto(407.57453951,322.37226336)
\lineto(407.57453951,322.23726336)
\curveto(407.57453666,322.19725997)(407.58453665,322.15726001)(407.60453951,322.11726336)
\curveto(407.60453663,322.04726012)(407.60953662,321.99226017)(407.61953951,321.95226336)
\lineto(407.64953951,321.77226336)
\curveto(407.65953657,321.71226045)(407.67453656,321.65226051)(407.69453951,321.59226336)
\curveto(407.78453645,321.30226086)(407.88953634,321.0622611)(408.00953951,320.87226336)
\curveto(408.13953609,320.69226147)(408.31953591,320.53226163)(408.54953951,320.39226336)
\curveto(408.68953554,320.31226185)(408.85453538,320.24726192)(409.04453951,320.19726336)
\curveto(409.08453515,320.18726198)(409.11953511,320.18226198)(409.14953951,320.18226336)
\curveto(409.17953505,320.19226197)(409.21453502,320.19226197)(409.25453951,320.18226336)
\curveto(409.29453494,320.17226199)(409.35453488,320.162262)(409.43453951,320.15226336)
\curveto(409.51453472,320.15226201)(409.57953465,320.15726201)(409.62953951,320.16726336)
\curveto(409.70953452,320.18726198)(409.78953444,320.20226196)(409.86953951,320.21226336)
\curveto(409.95953427,320.23226193)(410.04453419,320.25726191)(410.12453951,320.28726336)
\curveto(410.36453387,320.38726178)(410.55953367,320.52726164)(410.70953951,320.70726336)
\curveto(410.85953337,320.88726128)(410.98453325,321.09726107)(411.08453951,321.33726336)
\curveto(411.1345331,321.45726071)(411.16953306,321.58226058)(411.18953951,321.71226336)
\curveto(411.20953302,321.84226032)(411.234533,321.97726019)(411.26453951,322.11726336)
\lineto(411.26453951,322.26726336)
\curveto(411.27453296,322.31725985)(411.27953295,322.3672598)(411.27953951,322.41726336)
}
}
{
\newrgbcolor{curcolor}{0 0 0}
\pscustom[linestyle=none,fillstyle=solid,fillcolor=curcolor]
{
\newpath
\moveto(422.44446138,322.98726336)
\curveto(422.46445281,322.92725924)(422.4744528,322.84225932)(422.47446138,322.73226336)
\curveto(422.4744528,322.62225954)(422.46445281,322.53725963)(422.44446138,322.47726336)
\lineto(422.44446138,322.32726336)
\curveto(422.42445285,322.24725992)(422.41445286,322.16726)(422.41446138,322.08726336)
\curveto(422.42445285,322.00726016)(422.41945286,321.92726024)(422.39946138,321.84726336)
\curveto(422.3794529,321.77726039)(422.36445291,321.71226045)(422.35446138,321.65226336)
\curveto(422.34445293,321.59226057)(422.33445294,321.52726064)(422.32446138,321.45726336)
\curveto(422.28445299,321.34726082)(422.24945303,321.23226093)(422.21946138,321.11226336)
\curveto(422.18945309,321.00226116)(422.14945313,320.89726127)(422.09946138,320.79726336)
\curveto(421.88945339,320.31726185)(421.61445366,319.92726224)(421.27446138,319.62726336)
\curveto(420.93445434,319.32726284)(420.52445475,319.07726309)(420.04446138,318.87726336)
\curveto(419.92445535,318.82726334)(419.79945548,318.79226337)(419.66946138,318.77226336)
\curveto(419.54945573,318.74226342)(419.42445585,318.71226345)(419.29446138,318.68226336)
\curveto(419.24445603,318.6622635)(419.18945609,318.65226351)(419.12946138,318.65226336)
\curveto(419.06945621,318.65226351)(419.01445626,318.64726352)(418.96446138,318.63726336)
\lineto(418.85946138,318.63726336)
\curveto(418.82945645,318.62726354)(418.79945648,318.62226354)(418.76946138,318.62226336)
\curveto(418.71945656,318.61226355)(418.63945664,318.60726356)(418.52946138,318.60726336)
\curveto(418.41945686,318.59726357)(418.33445694,318.60226356)(418.27446138,318.62226336)
\lineto(418.12446138,318.62226336)
\curveto(418.0744572,318.63226353)(418.01945726,318.63726353)(417.95946138,318.63726336)
\curveto(417.90945737,318.62726354)(417.85945742,318.63226353)(417.80946138,318.65226336)
\curveto(417.76945751,318.6622635)(417.72945755,318.6672635)(417.68946138,318.66726336)
\curveto(417.65945762,318.6672635)(417.61945766,318.67226349)(417.56946138,318.68226336)
\curveto(417.46945781,318.71226345)(417.36945791,318.73726343)(417.26946138,318.75726336)
\curveto(417.16945811,318.77726339)(417.0744582,318.80726336)(416.98446138,318.84726336)
\curveto(416.86445841,318.88726328)(416.74945853,318.92726324)(416.63946138,318.96726336)
\curveto(416.53945874,319.00726316)(416.43445884,319.05726311)(416.32446138,319.11726336)
\curveto(415.9744593,319.32726284)(415.6744596,319.57226259)(415.42446138,319.85226336)
\curveto(415.1744601,320.13226203)(414.96446031,320.4672617)(414.79446138,320.85726336)
\curveto(414.74446053,320.94726122)(414.70446057,321.04226112)(414.67446138,321.14226336)
\curveto(414.65446062,321.24226092)(414.62946065,321.34726082)(414.59946138,321.45726336)
\curveto(414.5794607,321.50726066)(414.56946071,321.55226061)(414.56946138,321.59226336)
\curveto(414.56946071,321.63226053)(414.55946072,321.67726049)(414.53946138,321.72726336)
\curveto(414.51946076,321.80726036)(414.50946077,321.88726028)(414.50946138,321.96726336)
\curveto(414.50946077,322.05726011)(414.49946078,322.14226002)(414.47946138,322.22226336)
\curveto(414.46946081,322.27225989)(414.46446081,322.31725985)(414.46446138,322.35726336)
\lineto(414.46446138,322.49226336)
\curveto(414.44446083,322.55225961)(414.43446084,322.63725953)(414.43446138,322.74726336)
\curveto(414.44446083,322.85725931)(414.45946082,322.94225922)(414.47946138,323.00226336)
\lineto(414.47946138,323.10726336)
\curveto(414.48946079,323.15725901)(414.48946079,323.20725896)(414.47946138,323.25726336)
\curveto(414.4794608,323.31725885)(414.48946079,323.37225879)(414.50946138,323.42226336)
\curveto(414.51946076,323.47225869)(414.52446075,323.51725865)(414.52446138,323.55726336)
\curveto(414.52446075,323.60725856)(414.53446074,323.65725851)(414.55446138,323.70726336)
\curveto(414.59446068,323.83725833)(414.62946065,323.9622582)(414.65946138,324.08226336)
\curveto(414.68946059,324.21225795)(414.72946055,324.33725783)(414.77946138,324.45726336)
\curveto(414.95946032,324.8672573)(415.1744601,325.20725696)(415.42446138,325.47726336)
\curveto(415.6744596,325.75725641)(415.9794593,326.01225615)(416.33946138,326.24226336)
\curveto(416.43945884,326.29225587)(416.54445873,326.33725583)(416.65446138,326.37726336)
\curveto(416.76445851,326.41725575)(416.8744584,326.4622557)(416.98446138,326.51226336)
\curveto(417.11445816,326.5622556)(417.24945803,326.59725557)(417.38946138,326.61726336)
\curveto(417.52945775,326.63725553)(417.6744576,326.6672555)(417.82446138,326.70726336)
\curveto(417.90445737,326.71725545)(417.9794573,326.72225544)(418.04946138,326.72226336)
\curveto(418.11945716,326.72225544)(418.18945709,326.72725544)(418.25946138,326.73726336)
\curveto(418.83945644,326.74725542)(419.33945594,326.68725548)(419.75946138,326.55726336)
\curveto(420.18945509,326.42725574)(420.56945471,326.24725592)(420.89946138,326.01726336)
\curveto(421.00945427,325.93725623)(421.11945416,325.84725632)(421.22946138,325.74726336)
\curveto(421.34945393,325.65725651)(421.44945383,325.55725661)(421.52946138,325.44726336)
\curveto(421.60945367,325.34725682)(421.6794536,325.24725692)(421.73946138,325.14726336)
\curveto(421.80945347,325.04725712)(421.8794534,324.94225722)(421.94946138,324.83226336)
\curveto(422.01945326,324.72225744)(422.0744532,324.60225756)(422.11446138,324.47226336)
\curveto(422.15445312,324.35225781)(422.19945308,324.22225794)(422.24946138,324.08226336)
\curveto(422.279453,324.00225816)(422.30445297,323.91725825)(422.32446138,323.82726336)
\lineto(422.38446138,323.55726336)
\curveto(422.39445288,323.51725865)(422.39945288,323.47725869)(422.39946138,323.43726336)
\curveto(422.39945288,323.39725877)(422.40445287,323.35725881)(422.41446138,323.31726336)
\curveto(422.43445284,323.2672589)(422.43945284,323.21225895)(422.42946138,323.15226336)
\curveto(422.41945286,323.09225907)(422.42445285,323.03725913)(422.44446138,322.98726336)
\moveto(420.34446138,322.44726336)
\curveto(420.35445492,322.49725967)(420.35945492,322.5672596)(420.35946138,322.65726336)
\curveto(420.35945492,322.75725941)(420.35445492,322.83225933)(420.34446138,322.88226336)
\lineto(420.34446138,323.00226336)
\curveto(420.32445495,323.05225911)(420.31445496,323.10725906)(420.31446138,323.16726336)
\curveto(420.31445496,323.22725894)(420.30945497,323.28225888)(420.29946138,323.33226336)
\curveto(420.29945498,323.37225879)(420.29445498,323.40225876)(420.28446138,323.42226336)
\lineto(420.22446138,323.66226336)
\curveto(420.21445506,323.75225841)(420.19445508,323.83725833)(420.16446138,323.91726336)
\curveto(420.05445522,324.17725799)(419.92445535,324.39725777)(419.77446138,324.57726336)
\curveto(419.62445565,324.7672574)(419.42445585,324.91725725)(419.17446138,325.02726336)
\curveto(419.11445616,325.04725712)(419.05445622,325.0622571)(418.99446138,325.07226336)
\curveto(418.93445634,325.09225707)(418.86945641,325.11225705)(418.79946138,325.13226336)
\curveto(418.71945656,325.15225701)(418.63445664,325.15725701)(418.54446138,325.14726336)
\lineto(418.27446138,325.14726336)
\curveto(418.24445703,325.12725704)(418.20945707,325.11725705)(418.16946138,325.11726336)
\curveto(418.12945715,325.12725704)(418.09445718,325.12725704)(418.06446138,325.11726336)
\lineto(417.85446138,325.05726336)
\curveto(417.79445748,325.04725712)(417.73945754,325.02725714)(417.68946138,324.99726336)
\curveto(417.43945784,324.88725728)(417.23445804,324.72725744)(417.07446138,324.51726336)
\curveto(416.92445835,324.31725785)(416.80445847,324.08225808)(416.71446138,323.81226336)
\curveto(416.68445859,323.71225845)(416.65945862,323.60725856)(416.63946138,323.49726336)
\curveto(416.62945865,323.38725878)(416.61445866,323.27725889)(416.59446138,323.16726336)
\curveto(416.58445869,323.11725905)(416.5794587,323.0672591)(416.57946138,323.01726336)
\lineto(416.57946138,322.86726336)
\curveto(416.55945872,322.79725937)(416.54945873,322.69225947)(416.54946138,322.55226336)
\curveto(416.55945872,322.41225975)(416.5744587,322.30725986)(416.59446138,322.23726336)
\lineto(416.59446138,322.10226336)
\curveto(416.61445866,322.02226014)(416.62945865,321.94226022)(416.63946138,321.86226336)
\curveto(416.64945863,321.79226037)(416.66445861,321.71726045)(416.68446138,321.63726336)
\curveto(416.78445849,321.33726083)(416.88945839,321.09226107)(416.99946138,320.90226336)
\curveto(417.11945816,320.72226144)(417.30445797,320.55726161)(417.55446138,320.40726336)
\curveto(417.62445765,320.35726181)(417.69945758,320.31726185)(417.77946138,320.28726336)
\curveto(417.86945741,320.25726191)(417.95945732,320.23226193)(418.04946138,320.21226336)
\curveto(418.08945719,320.20226196)(418.12445715,320.19726197)(418.15446138,320.19726336)
\curveto(418.18445709,320.20726196)(418.21945706,320.20726196)(418.25946138,320.19726336)
\lineto(418.37946138,320.16726336)
\curveto(418.42945685,320.167262)(418.4744568,320.17226199)(418.51446138,320.18226336)
\lineto(418.63446138,320.18226336)
\curveto(418.71445656,320.20226196)(418.79445648,320.21726195)(418.87446138,320.22726336)
\curveto(418.95445632,320.23726193)(419.02945625,320.25726191)(419.09946138,320.28726336)
\curveto(419.35945592,320.38726178)(419.56945571,320.52226164)(419.72946138,320.69226336)
\curveto(419.88945539,320.8622613)(420.02445525,321.07226109)(420.13446138,321.32226336)
\curveto(420.1744551,321.42226074)(420.20445507,321.52226064)(420.22446138,321.62226336)
\curveto(420.24445503,321.72226044)(420.26945501,321.82726034)(420.29946138,321.93726336)
\curveto(420.30945497,321.97726019)(420.31445496,322.01226015)(420.31446138,322.04226336)
\curveto(420.31445496,322.08226008)(420.31945496,322.12226004)(420.32946138,322.16226336)
\lineto(420.32946138,322.29726336)
\curveto(420.32945495,322.34725982)(420.33445494,322.39725977)(420.34446138,322.44726336)
}
}
{
\newrgbcolor{curcolor}{0 0 0}
\pscustom[linestyle=none,fillstyle=solid,fillcolor=curcolor]
{
\newpath
\moveto(428.26938326,326.73726336)
\curveto(428.37937794,326.73725543)(428.47437785,326.72725544)(428.55438326,326.70726336)
\curveto(428.64437768,326.68725548)(428.71437761,326.64225552)(428.76438326,326.57226336)
\curveto(428.8243775,326.49225567)(428.85437747,326.35225581)(428.85438326,326.15226336)
\lineto(428.85438326,325.64226336)
\lineto(428.85438326,325.26726336)
\curveto(428.86437746,325.12725704)(428.84937747,325.01725715)(428.80938326,324.93726336)
\curveto(428.76937755,324.8672573)(428.70937761,324.82225734)(428.62938326,324.80226336)
\curveto(428.55937776,324.78225738)(428.47437785,324.77225739)(428.37438326,324.77226336)
\curveto(428.28437804,324.77225739)(428.18437814,324.77725739)(428.07438326,324.78726336)
\curveto(427.97437835,324.79725737)(427.87937844,324.79225737)(427.78938326,324.77226336)
\curveto(427.7193786,324.75225741)(427.64937867,324.73725743)(427.57938326,324.72726336)
\curveto(427.50937881,324.72725744)(427.44437888,324.71725745)(427.38438326,324.69726336)
\curveto(427.2243791,324.64725752)(427.06437926,324.57225759)(426.90438326,324.47226336)
\curveto(426.74437958,324.38225778)(426.6193797,324.27725789)(426.52938326,324.15726336)
\curveto(426.47937984,324.07725809)(426.4243799,323.99225817)(426.36438326,323.90226336)
\curveto(426.31438001,323.82225834)(426.26438006,323.73725843)(426.21438326,323.64726336)
\curveto(426.18438014,323.5672586)(426.15438017,323.48225868)(426.12438326,323.39226336)
\lineto(426.06438326,323.15226336)
\curveto(426.04438028,323.08225908)(426.03438029,323.00725916)(426.03438326,322.92726336)
\curveto(426.03438029,322.85725931)(426.0243803,322.78725938)(426.00438326,322.71726336)
\curveto(425.99438033,322.67725949)(425.98938033,322.63725953)(425.98938326,322.59726336)
\curveto(425.99938032,322.5672596)(425.99938032,322.53725963)(425.98938326,322.50726336)
\lineto(425.98938326,322.26726336)
\curveto(425.96938035,322.19725997)(425.96438036,322.11726005)(425.97438326,322.02726336)
\curveto(425.98438034,321.94726022)(425.98938033,321.8672603)(425.98938326,321.78726336)
\lineto(425.98938326,320.82726336)
\lineto(425.98938326,319.55226336)
\curveto(425.98938033,319.42226274)(425.98438034,319.30226286)(425.97438326,319.19226336)
\curveto(425.96438036,319.08226308)(425.93438039,318.99226317)(425.88438326,318.92226336)
\curveto(425.86438046,318.89226327)(425.82938049,318.8672633)(425.77938326,318.84726336)
\curveto(425.73938058,318.83726333)(425.69438063,318.82726334)(425.64438326,318.81726336)
\lineto(425.56938326,318.81726336)
\curveto(425.5193808,318.80726336)(425.46438086,318.80226336)(425.40438326,318.80226336)
\lineto(425.23938326,318.80226336)
\lineto(424.59438326,318.80226336)
\curveto(424.53438179,318.81226335)(424.46938185,318.81726335)(424.39938326,318.81726336)
\lineto(424.20438326,318.81726336)
\curveto(424.15438217,318.83726333)(424.10438222,318.85226331)(424.05438326,318.86226336)
\curveto(424.00438232,318.88226328)(423.96938235,318.91726325)(423.94938326,318.96726336)
\curveto(423.90938241,319.01726315)(423.88438244,319.08726308)(423.87438326,319.17726336)
\lineto(423.87438326,319.47726336)
\lineto(423.87438326,320.49726336)
\lineto(423.87438326,324.72726336)
\lineto(423.87438326,325.83726336)
\lineto(423.87438326,326.12226336)
\curveto(423.87438245,326.22225594)(423.89438243,326.30225586)(423.93438326,326.36226336)
\curveto(423.98438234,326.44225572)(424.05938226,326.49225567)(424.15938326,326.51226336)
\curveto(424.25938206,326.53225563)(424.37938194,326.54225562)(424.51938326,326.54226336)
\lineto(425.28438326,326.54226336)
\curveto(425.40438092,326.54225562)(425.50938081,326.53225563)(425.59938326,326.51226336)
\curveto(425.68938063,326.50225566)(425.75938056,326.45725571)(425.80938326,326.37726336)
\curveto(425.83938048,326.32725584)(425.85438047,326.25725591)(425.85438326,326.16726336)
\lineto(425.88438326,325.89726336)
\curveto(425.89438043,325.81725635)(425.90938041,325.74225642)(425.92938326,325.67226336)
\curveto(425.95938036,325.60225656)(426.00938031,325.5672566)(426.07938326,325.56726336)
\curveto(426.09938022,325.58725658)(426.1193802,325.59725657)(426.13938326,325.59726336)
\curveto(426.15938016,325.59725657)(426.17938014,325.60725656)(426.19938326,325.62726336)
\curveto(426.25938006,325.67725649)(426.30938001,325.73225643)(426.34938326,325.79226336)
\curveto(426.39937992,325.8622563)(426.45937986,325.92225624)(426.52938326,325.97226336)
\curveto(426.56937975,326.00225616)(426.60437972,326.03225613)(426.63438326,326.06226336)
\curveto(426.66437966,326.10225606)(426.69937962,326.13725603)(426.73938326,326.16726336)
\lineto(427.00938326,326.34726336)
\curveto(427.10937921,326.40725576)(427.20937911,326.4622557)(427.30938326,326.51226336)
\curveto(427.40937891,326.55225561)(427.50937881,326.58725558)(427.60938326,326.61726336)
\lineto(427.93938326,326.70726336)
\curveto(427.96937835,326.71725545)(428.0243783,326.71725545)(428.10438326,326.70726336)
\curveto(428.19437813,326.70725546)(428.24937807,326.71725545)(428.26938326,326.73726336)
}
}
{
\newrgbcolor{curcolor}{0 0 0}
\pscustom[linestyle=none,fillstyle=solid,fillcolor=curcolor]
{
}
}
{
\newrgbcolor{curcolor}{0 0 0}
\pscustom[linestyle=none,fillstyle=solid,fillcolor=curcolor]
{
\newpath
\moveto(434.88461763,328.85226336)
\lineto(435.88961763,328.85226336)
\curveto(436.03961465,328.85225331)(436.16961452,328.84225332)(436.27961763,328.82226336)
\curveto(436.39961429,328.81225335)(436.4846142,328.75225341)(436.53461763,328.64226336)
\curveto(436.55461413,328.59225357)(436.56461412,328.53225363)(436.56461763,328.46226336)
\lineto(436.56461763,328.25226336)
\lineto(436.56461763,327.57726336)
\curveto(436.56461412,327.52725464)(436.55961413,327.4672547)(436.54961763,327.39726336)
\curveto(436.54961414,327.33725483)(436.55461413,327.28225488)(436.56461763,327.23226336)
\lineto(436.56461763,327.06726336)
\curveto(436.56461412,326.98725518)(436.56961412,326.91225525)(436.57961763,326.84226336)
\curveto(436.5896141,326.78225538)(436.61461407,326.72725544)(436.65461763,326.67726336)
\curveto(436.72461396,326.58725558)(436.84961384,326.53725563)(437.02961763,326.52726336)
\lineto(437.56961763,326.52726336)
\lineto(437.74961763,326.52726336)
\curveto(437.80961288,326.52725564)(437.86461282,326.51725565)(437.91461763,326.49726336)
\curveto(438.02461266,326.44725572)(438.0846126,326.35725581)(438.09461763,326.22726336)
\curveto(438.11461257,326.09725607)(438.12461256,325.95225621)(438.12461763,325.79226336)
\lineto(438.12461763,325.58226336)
\curveto(438.13461255,325.51225665)(438.12961256,325.45225671)(438.10961763,325.40226336)
\curveto(438.05961263,325.24225692)(437.95461273,325.15725701)(437.79461763,325.14726336)
\curveto(437.63461305,325.13725703)(437.45461323,325.13225703)(437.25461763,325.13226336)
\lineto(437.11961763,325.13226336)
\curveto(437.07961361,325.14225702)(437.04461364,325.14225702)(437.01461763,325.13226336)
\curveto(436.97461371,325.12225704)(436.93961375,325.11725705)(436.90961763,325.11726336)
\curveto(436.87961381,325.12725704)(436.84961384,325.12225704)(436.81961763,325.10226336)
\curveto(436.73961395,325.08225708)(436.67961401,325.03725713)(436.63961763,324.96726336)
\curveto(436.60961408,324.90725726)(436.5846141,324.83225733)(436.56461763,324.74226336)
\curveto(436.55461413,324.69225747)(436.55461413,324.63725753)(436.56461763,324.57726336)
\curveto(436.57461411,324.51725765)(436.57461411,324.4622577)(436.56461763,324.41226336)
\lineto(436.56461763,323.48226336)
\lineto(436.56461763,321.72726336)
\curveto(436.56461412,321.47726069)(436.56961412,321.25726091)(436.57961763,321.06726336)
\curveto(436.59961409,320.88726128)(436.66461402,320.72726144)(436.77461763,320.58726336)
\curveto(436.82461386,320.52726164)(436.8896138,320.48226168)(436.96961763,320.45226336)
\lineto(437.23961763,320.39226336)
\curveto(437.26961342,320.38226178)(437.29961339,320.37726179)(437.32961763,320.37726336)
\curveto(437.36961332,320.38726178)(437.39961329,320.38726178)(437.41961763,320.37726336)
\lineto(437.58461763,320.37726336)
\curveto(437.69461299,320.37726179)(437.7896129,320.37226179)(437.86961763,320.36226336)
\curveto(437.94961274,320.35226181)(438.01461267,320.31226185)(438.06461763,320.24226336)
\curveto(438.10461258,320.18226198)(438.12461256,320.10226206)(438.12461763,320.00226336)
\lineto(438.12461763,319.71726336)
\curveto(438.12461256,319.50726266)(438.11961257,319.31226285)(438.10961763,319.13226336)
\curveto(438.10961258,318.9622632)(438.02961266,318.84726332)(437.86961763,318.78726336)
\curveto(437.81961287,318.7672634)(437.77461291,318.7622634)(437.73461763,318.77226336)
\curveto(437.69461299,318.77226339)(437.64961304,318.7622634)(437.59961763,318.74226336)
\lineto(437.44961763,318.74226336)
\curveto(437.42961326,318.74226342)(437.39961329,318.74726342)(437.35961763,318.75726336)
\curveto(437.31961337,318.75726341)(437.2846134,318.75226341)(437.25461763,318.74226336)
\curveto(437.20461348,318.73226343)(437.14961354,318.73226343)(437.08961763,318.74226336)
\lineto(436.93961763,318.74226336)
\lineto(436.78961763,318.74226336)
\curveto(436.73961395,318.73226343)(436.69461399,318.73226343)(436.65461763,318.74226336)
\lineto(436.48961763,318.74226336)
\curveto(436.43961425,318.75226341)(436.3846143,318.75726341)(436.32461763,318.75726336)
\curveto(436.26461442,318.75726341)(436.20961448,318.7622634)(436.15961763,318.77226336)
\curveto(436.0896146,318.78226338)(436.02461466,318.79226337)(435.96461763,318.80226336)
\lineto(435.78461763,318.83226336)
\curveto(435.67461501,318.8622633)(435.56961512,318.89726327)(435.46961763,318.93726336)
\curveto(435.36961532,318.97726319)(435.27461541,319.02226314)(435.18461763,319.07226336)
\lineto(435.09461763,319.13226336)
\curveto(435.06461562,319.162263)(435.02961566,319.19226297)(434.98961763,319.22226336)
\curveto(434.96961572,319.24226292)(434.94461574,319.2622629)(434.91461763,319.28226336)
\lineto(434.83961763,319.35726336)
\curveto(434.69961599,319.54726262)(434.59461609,319.75726241)(434.52461763,319.98726336)
\curveto(434.50461618,320.02726214)(434.49461619,320.0622621)(434.49461763,320.09226336)
\curveto(434.50461618,320.13226203)(434.50461618,320.17726199)(434.49461763,320.22726336)
\curveto(434.4846162,320.24726192)(434.47961621,320.27226189)(434.47961763,320.30226336)
\curveto(434.47961621,320.33226183)(434.47461621,320.35726181)(434.46461763,320.37726336)
\lineto(434.46461763,320.52726336)
\curveto(434.45461623,320.5672616)(434.44961624,320.61226155)(434.44961763,320.66226336)
\curveto(434.45961623,320.71226145)(434.46461622,320.7622614)(434.46461763,320.81226336)
\lineto(434.46461763,321.38226336)
\lineto(434.46461763,323.61726336)
\lineto(434.46461763,324.41226336)
\lineto(434.46461763,324.62226336)
\curveto(434.47461621,324.69225747)(434.46961622,324.75725741)(434.44961763,324.81726336)
\curveto(434.40961628,324.95725721)(434.33961635,325.04725712)(434.23961763,325.08726336)
\curveto(434.12961656,325.13725703)(433.9896167,325.15225701)(433.81961763,325.13226336)
\curveto(433.64961704,325.11225705)(433.50461718,325.12725704)(433.38461763,325.17726336)
\curveto(433.30461738,325.20725696)(433.25461743,325.25225691)(433.23461763,325.31226336)
\curveto(433.21461747,325.37225679)(433.19461749,325.44725672)(433.17461763,325.53726336)
\lineto(433.17461763,325.85226336)
\curveto(433.17461751,326.03225613)(433.1846175,326.17725599)(433.20461763,326.28726336)
\curveto(433.22461746,326.39725577)(433.30961738,326.47225569)(433.45961763,326.51226336)
\curveto(433.49961719,326.53225563)(433.53961715,326.53725563)(433.57961763,326.52726336)
\lineto(433.71461763,326.52726336)
\curveto(433.86461682,326.52725564)(434.00461668,326.53225563)(434.13461763,326.54226336)
\curveto(434.26461642,326.5622556)(434.35461633,326.62225554)(434.40461763,326.72226336)
\curveto(434.43461625,326.79225537)(434.44961624,326.87225529)(434.44961763,326.96226336)
\curveto(434.45961623,327.05225511)(434.46461622,327.14225502)(434.46461763,327.23226336)
\lineto(434.46461763,328.16226336)
\lineto(434.46461763,328.41726336)
\curveto(434.46461622,328.50725366)(434.47461621,328.58225358)(434.49461763,328.64226336)
\curveto(434.54461614,328.74225342)(434.61961607,328.80725336)(434.71961763,328.83726336)
\curveto(434.73961595,328.84725332)(434.76461592,328.84725332)(434.79461763,328.83726336)
\curveto(434.83461585,328.83725333)(434.86461582,328.84225332)(434.88461763,328.85226336)
}
}
{
\newrgbcolor{curcolor}{0 0 0}
\pscustom[linestyle=none,fillstyle=solid,fillcolor=curcolor]
{
\newpath
\moveto(441.20805513,329.39226336)
\curveto(441.27805218,329.31225285)(441.31305215,329.19225297)(441.31305513,329.03226336)
\lineto(441.31305513,328.56726336)
\lineto(441.31305513,328.16226336)
\curveto(441.31305215,328.02225414)(441.27805218,327.92725424)(441.20805513,327.87726336)
\curveto(441.14805231,327.82725434)(441.06805239,327.79725437)(440.96805513,327.78726336)
\curveto(440.87805258,327.77725439)(440.77805268,327.77225439)(440.66805513,327.77226336)
\lineto(439.82805513,327.77226336)
\curveto(439.71805374,327.77225439)(439.61805384,327.77725439)(439.52805513,327.78726336)
\curveto(439.44805401,327.79725437)(439.37805408,327.82725434)(439.31805513,327.87726336)
\curveto(439.27805418,327.90725426)(439.24805421,327.9622542)(439.22805513,328.04226336)
\curveto(439.21805424,328.13225403)(439.20805425,328.22725394)(439.19805513,328.32726336)
\lineto(439.19805513,328.65726336)
\curveto(439.20805425,328.7672534)(439.21305425,328.8622533)(439.21305513,328.94226336)
\lineto(439.21305513,329.15226336)
\curveto(439.22305424,329.22225294)(439.24305422,329.28225288)(439.27305513,329.33226336)
\curveto(439.29305417,329.37225279)(439.31805414,329.40225276)(439.34805513,329.42226336)
\lineto(439.46805513,329.48226336)
\curveto(439.48805397,329.48225268)(439.51305395,329.48225268)(439.54305513,329.48226336)
\curveto(439.57305389,329.49225267)(439.59805386,329.49725267)(439.61805513,329.49726336)
\lineto(440.71305513,329.49726336)
\curveto(440.81305265,329.49725267)(440.90805255,329.49225267)(440.99805513,329.48226336)
\curveto(441.08805237,329.47225269)(441.1580523,329.44225272)(441.20805513,329.39226336)
\moveto(441.31305513,319.62726336)
\curveto(441.31305215,319.42726274)(441.30805215,319.25726291)(441.29805513,319.11726336)
\curveto(441.28805217,318.97726319)(441.19805226,318.88226328)(441.02805513,318.83226336)
\curveto(440.96805249,318.81226335)(440.90305256,318.80226336)(440.83305513,318.80226336)
\curveto(440.7630527,318.81226335)(440.68805277,318.81726335)(440.60805513,318.81726336)
\lineto(439.76805513,318.81726336)
\curveto(439.67805378,318.81726335)(439.58805387,318.82226334)(439.49805513,318.83226336)
\curveto(439.41805404,318.84226332)(439.3580541,318.87226329)(439.31805513,318.92226336)
\curveto(439.2580542,318.99226317)(439.22305424,319.07726309)(439.21305513,319.17726336)
\lineto(439.21305513,319.52226336)
\lineto(439.21305513,325.85226336)
\lineto(439.21305513,326.15226336)
\curveto(439.21305425,326.25225591)(439.23305423,326.33225583)(439.27305513,326.39226336)
\curveto(439.33305413,326.4622557)(439.41805404,326.50725566)(439.52805513,326.52726336)
\curveto(439.54805391,326.53725563)(439.57305389,326.53725563)(439.60305513,326.52726336)
\curveto(439.64305382,326.52725564)(439.67305379,326.53225563)(439.69305513,326.54226336)
\lineto(440.44305513,326.54226336)
\lineto(440.63805513,326.54226336)
\curveto(440.71805274,326.55225561)(440.78305268,326.55225561)(440.83305513,326.54226336)
\lineto(440.95305513,326.54226336)
\curveto(441.01305245,326.52225564)(441.06805239,326.50725566)(441.11805513,326.49726336)
\curveto(441.16805229,326.48725568)(441.20805225,326.45725571)(441.23805513,326.40726336)
\curveto(441.27805218,326.35725581)(441.29805216,326.28725588)(441.29805513,326.19726336)
\curveto(441.30805215,326.10725606)(441.31305215,326.01225615)(441.31305513,325.91226336)
\lineto(441.31305513,319.62726336)
}
}
{
\newrgbcolor{curcolor}{0 0 0}
\pscustom[linestyle=none,fillstyle=solid,fillcolor=curcolor]
{
\newpath
\moveto(450.86524263,322.76226336)
\curveto(450.87523395,322.70225946)(450.88023395,322.61225955)(450.88024263,322.49226336)
\curveto(450.88023395,322.37225979)(450.87023396,322.28725988)(450.85024263,322.23726336)
\lineto(450.85024263,322.04226336)
\curveto(450.82023401,321.93226023)(450.80023403,321.82726034)(450.79024263,321.72726336)
\curveto(450.79023404,321.62726054)(450.77523405,321.52726064)(450.74524263,321.42726336)
\curveto(450.7252341,321.33726083)(450.70523412,321.24226092)(450.68524263,321.14226336)
\curveto(450.66523416,321.05226111)(450.63523419,320.9622612)(450.59524263,320.87226336)
\curveto(450.5252343,320.70226146)(450.45523437,320.54226162)(450.38524263,320.39226336)
\curveto(450.31523451,320.25226191)(450.23523459,320.11226205)(450.14524263,319.97226336)
\curveto(450.08523474,319.88226228)(450.02023481,319.79726237)(449.95024263,319.71726336)
\curveto(449.89023494,319.64726252)(449.82023501,319.57226259)(449.74024263,319.49226336)
\lineto(449.63524263,319.38726336)
\curveto(449.58523524,319.33726283)(449.5302353,319.29226287)(449.47024263,319.25226336)
\lineto(449.32024263,319.13226336)
\curveto(449.24023559,319.07226309)(449.15023568,319.01726315)(449.05024263,318.96726336)
\curveto(448.96023587,318.92726324)(448.86523596,318.88226328)(448.76524263,318.83226336)
\curveto(448.66523616,318.78226338)(448.56023627,318.74726342)(448.45024263,318.72726336)
\curveto(448.35023648,318.70726346)(448.24523658,318.68726348)(448.13524263,318.66726336)
\curveto(448.07523675,318.64726352)(448.01023682,318.63726353)(447.94024263,318.63726336)
\curveto(447.88023695,318.63726353)(447.81523701,318.62726354)(447.74524263,318.60726336)
\lineto(447.61024263,318.60726336)
\curveto(447.5302373,318.58726358)(447.45523737,318.58726358)(447.38524263,318.60726336)
\lineto(447.23524263,318.60726336)
\curveto(447.17523765,318.62726354)(447.11023772,318.63726353)(447.04024263,318.63726336)
\curveto(446.98023785,318.62726354)(446.92023791,318.63226353)(446.86024263,318.65226336)
\curveto(446.70023813,318.70226346)(446.54523828,318.74726342)(446.39524263,318.78726336)
\curveto(446.25523857,318.82726334)(446.1252387,318.88726328)(446.00524263,318.96726336)
\curveto(445.93523889,319.00726316)(445.87023896,319.04726312)(445.81024263,319.08726336)
\curveto(445.75023908,319.13726303)(445.68523914,319.18726298)(445.61524263,319.23726336)
\lineto(445.43524263,319.37226336)
\curveto(445.35523947,319.43226273)(445.28523954,319.43726273)(445.22524263,319.38726336)
\curveto(445.17523965,319.35726281)(445.15023968,319.31726285)(445.15024263,319.26726336)
\curveto(445.15023968,319.22726294)(445.14023969,319.17726299)(445.12024263,319.11726336)
\curveto(445.10023973,319.01726315)(445.09023974,318.90226326)(445.09024263,318.77226336)
\curveto(445.10023973,318.64226352)(445.10523972,318.52226364)(445.10524263,318.41226336)
\lineto(445.10524263,316.88226336)
\curveto(445.10523972,316.75226541)(445.10023973,316.62726554)(445.09024263,316.50726336)
\curveto(445.09023974,316.37726579)(445.06523976,316.27226589)(445.01524263,316.19226336)
\curveto(444.98523984,316.15226601)(444.9302399,316.12226604)(444.85024263,316.10226336)
\curveto(444.77024006,316.08226608)(444.68024015,316.07226609)(444.58024263,316.07226336)
\curveto(444.48024035,316.0622661)(444.38024045,316.0622661)(444.28024263,316.07226336)
\lineto(444.02524263,316.07226336)
\lineto(443.62024263,316.07226336)
\lineto(443.51524263,316.07226336)
\curveto(443.47524135,316.07226609)(443.44024139,316.07726609)(443.41024263,316.08726336)
\lineto(443.29024263,316.08726336)
\curveto(443.12024171,316.13726603)(443.0302418,316.23726593)(443.02024263,316.38726336)
\curveto(443.01024182,316.52726564)(443.00524182,316.69726547)(443.00524263,316.89726336)
\lineto(443.00524263,325.70226336)
\curveto(443.00524182,325.81225635)(443.00024183,325.92725624)(442.99024263,326.04726336)
\curveto(442.99024184,326.17725599)(443.01524181,326.27725589)(443.06524263,326.34726336)
\curveto(443.10524172,326.41725575)(443.16024167,326.4622557)(443.23024263,326.48226336)
\curveto(443.28024155,326.50225566)(443.34024149,326.51225565)(443.41024263,326.51226336)
\lineto(443.63524263,326.51226336)
\lineto(444.35524263,326.51226336)
\lineto(444.64024263,326.51226336)
\curveto(444.7302401,326.51225565)(444.80524002,326.48725568)(444.86524263,326.43726336)
\curveto(444.93523989,326.38725578)(444.97023986,326.32225584)(444.97024263,326.24226336)
\curveto(444.98023985,326.17225599)(445.00523982,326.09725607)(445.04524263,326.01726336)
\curveto(445.05523977,325.98725618)(445.06523976,325.9622562)(445.07524263,325.94226336)
\curveto(445.09523973,325.93225623)(445.11523971,325.91725625)(445.13524263,325.89726336)
\curveto(445.24523958,325.88725628)(445.33523949,325.91725625)(445.40524263,325.98726336)
\curveto(445.47523935,326.05725611)(445.54523928,326.11725605)(445.61524263,326.16726336)
\curveto(445.74523908,326.25725591)(445.88023895,326.33725583)(446.02024263,326.40726336)
\curveto(446.16023867,326.48725568)(446.31523851,326.55225561)(446.48524263,326.60226336)
\curveto(446.56523826,326.63225553)(446.65023818,326.65225551)(446.74024263,326.66226336)
\curveto(446.84023799,326.67225549)(446.93523789,326.68725548)(447.02524263,326.70726336)
\curveto(447.06523776,326.71725545)(447.10523772,326.71725545)(447.14524263,326.70726336)
\curveto(447.19523763,326.69725547)(447.23523759,326.70225546)(447.26524263,326.72226336)
\curveto(447.83523699,326.74225542)(448.31523651,326.6622555)(448.70524263,326.48226336)
\curveto(449.10523572,326.31225585)(449.44523538,326.08725608)(449.72524263,325.80726336)
\curveto(449.77523505,325.75725641)(449.82023501,325.70725646)(449.86024263,325.65726336)
\curveto(449.90023493,325.61725655)(449.94023489,325.57225659)(449.98024263,325.52226336)
\curveto(450.05023478,325.43225673)(450.11023472,325.34225682)(450.16024263,325.25226336)
\curveto(450.22023461,325.162257)(450.27523455,325.07225709)(450.32524263,324.98226336)
\curveto(450.34523448,324.9622572)(450.35523447,324.93725723)(450.35524263,324.90726336)
\curveto(450.36523446,324.87725729)(450.38023445,324.84225732)(450.40024263,324.80226336)
\curveto(450.46023437,324.70225746)(450.51523431,324.58225758)(450.56524263,324.44226336)
\curveto(450.58523424,324.38225778)(450.60523422,324.31725785)(450.62524263,324.24726336)
\curveto(450.64523418,324.18725798)(450.66523416,324.12225804)(450.68524263,324.05226336)
\curveto(450.7252341,323.93225823)(450.75023408,323.80725836)(450.76024263,323.67726336)
\curveto(450.78023405,323.54725862)(450.80523402,323.41225875)(450.83524263,323.27226336)
\lineto(450.83524263,323.10726336)
\lineto(450.86524263,322.92726336)
\lineto(450.86524263,322.76226336)
\moveto(448.75024263,322.41726336)
\curveto(448.76023607,322.4672597)(448.76523606,322.53225963)(448.76524263,322.61226336)
\curveto(448.76523606,322.70225946)(448.76023607,322.77225939)(448.75024263,322.82226336)
\lineto(448.75024263,322.95726336)
\curveto(448.7302361,323.01725915)(448.72023611,323.08225908)(448.72024263,323.15226336)
\curveto(448.72023611,323.22225894)(448.71023612,323.29225887)(448.69024263,323.36226336)
\curveto(448.67023616,323.4622587)(448.65023618,323.55725861)(448.63024263,323.64726336)
\curveto(448.61023622,323.74725842)(448.58023625,323.83725833)(448.54024263,323.91726336)
\curveto(448.42023641,324.23725793)(448.26523656,324.49225767)(448.07524263,324.68226336)
\curveto(447.88523694,324.87225729)(447.61523721,325.01225715)(447.26524263,325.10226336)
\curveto(447.18523764,325.12225704)(447.09523773,325.13225703)(446.99524263,325.13226336)
\lineto(446.72524263,325.13226336)
\curveto(446.68523814,325.12225704)(446.65023818,325.11725705)(446.62024263,325.11726336)
\curveto(446.59023824,325.11725705)(446.55523827,325.11225705)(446.51524263,325.10226336)
\lineto(446.30524263,325.04226336)
\curveto(446.24523858,325.03225713)(446.18523864,325.01225715)(446.12524263,324.98226336)
\curveto(445.86523896,324.87225729)(445.66023917,324.70225746)(445.51024263,324.47226336)
\curveto(445.37023946,324.24225792)(445.25523957,323.98725818)(445.16524263,323.70726336)
\curveto(445.14523968,323.62725854)(445.1302397,323.54225862)(445.12024263,323.45226336)
\curveto(445.11023972,323.37225879)(445.09523973,323.29225887)(445.07524263,323.21226336)
\curveto(445.06523976,323.17225899)(445.06023977,323.10725906)(445.06024263,323.01726336)
\curveto(445.04023979,322.97725919)(445.03523979,322.92725924)(445.04524263,322.86726336)
\curveto(445.05523977,322.81725935)(445.05523977,322.7672594)(445.04524263,322.71726336)
\curveto(445.0252398,322.65725951)(445.0252398,322.60225956)(445.04524263,322.55226336)
\lineto(445.04524263,322.37226336)
\lineto(445.04524263,322.23726336)
\curveto(445.04523978,322.19725997)(445.05523977,322.15726001)(445.07524263,322.11726336)
\curveto(445.07523975,322.04726012)(445.08023975,321.99226017)(445.09024263,321.95226336)
\lineto(445.12024263,321.77226336)
\curveto(445.1302397,321.71226045)(445.14523968,321.65226051)(445.16524263,321.59226336)
\curveto(445.25523957,321.30226086)(445.36023947,321.0622611)(445.48024263,320.87226336)
\curveto(445.61023922,320.69226147)(445.79023904,320.53226163)(446.02024263,320.39226336)
\curveto(446.16023867,320.31226185)(446.3252385,320.24726192)(446.51524263,320.19726336)
\curveto(446.55523827,320.18726198)(446.59023824,320.18226198)(446.62024263,320.18226336)
\curveto(446.65023818,320.19226197)(446.68523814,320.19226197)(446.72524263,320.18226336)
\curveto(446.76523806,320.17226199)(446.825238,320.162262)(446.90524263,320.15226336)
\curveto(446.98523784,320.15226201)(447.05023778,320.15726201)(447.10024263,320.16726336)
\curveto(447.18023765,320.18726198)(447.26023757,320.20226196)(447.34024263,320.21226336)
\curveto(447.4302374,320.23226193)(447.51523731,320.25726191)(447.59524263,320.28726336)
\curveto(447.83523699,320.38726178)(448.0302368,320.52726164)(448.18024263,320.70726336)
\curveto(448.3302365,320.88726128)(448.45523637,321.09726107)(448.55524263,321.33726336)
\curveto(448.60523622,321.45726071)(448.64023619,321.58226058)(448.66024263,321.71226336)
\curveto(448.68023615,321.84226032)(448.70523612,321.97726019)(448.73524263,322.11726336)
\lineto(448.73524263,322.26726336)
\curveto(448.74523608,322.31725985)(448.75023608,322.3672598)(448.75024263,322.41726336)
}
}
{
\newrgbcolor{curcolor}{0 0 0}
\pscustom[linestyle=none,fillstyle=solid,fillcolor=curcolor]
{
\newpath
\moveto(459.91516451,322.98726336)
\curveto(459.93515594,322.92725924)(459.94515593,322.84225932)(459.94516451,322.73226336)
\curveto(459.94515593,322.62225954)(459.93515594,322.53725963)(459.91516451,322.47726336)
\lineto(459.91516451,322.32726336)
\curveto(459.89515598,322.24725992)(459.88515599,322.16726)(459.88516451,322.08726336)
\curveto(459.89515598,322.00726016)(459.89015598,321.92726024)(459.87016451,321.84726336)
\curveto(459.85015602,321.77726039)(459.83515604,321.71226045)(459.82516451,321.65226336)
\curveto(459.81515606,321.59226057)(459.80515607,321.52726064)(459.79516451,321.45726336)
\curveto(459.75515612,321.34726082)(459.72015615,321.23226093)(459.69016451,321.11226336)
\curveto(459.66015621,321.00226116)(459.62015625,320.89726127)(459.57016451,320.79726336)
\curveto(459.36015651,320.31726185)(459.08515679,319.92726224)(458.74516451,319.62726336)
\curveto(458.40515747,319.32726284)(457.99515788,319.07726309)(457.51516451,318.87726336)
\curveto(457.39515848,318.82726334)(457.2701586,318.79226337)(457.14016451,318.77226336)
\curveto(457.02015885,318.74226342)(456.89515898,318.71226345)(456.76516451,318.68226336)
\curveto(456.71515916,318.6622635)(456.66015921,318.65226351)(456.60016451,318.65226336)
\curveto(456.54015933,318.65226351)(456.48515939,318.64726352)(456.43516451,318.63726336)
\lineto(456.33016451,318.63726336)
\curveto(456.30015957,318.62726354)(456.2701596,318.62226354)(456.24016451,318.62226336)
\curveto(456.19015968,318.61226355)(456.11015976,318.60726356)(456.00016451,318.60726336)
\curveto(455.89015998,318.59726357)(455.80516007,318.60226356)(455.74516451,318.62226336)
\lineto(455.59516451,318.62226336)
\curveto(455.54516033,318.63226353)(455.49016038,318.63726353)(455.43016451,318.63726336)
\curveto(455.38016049,318.62726354)(455.33016054,318.63226353)(455.28016451,318.65226336)
\curveto(455.24016063,318.6622635)(455.20016067,318.6672635)(455.16016451,318.66726336)
\curveto(455.13016074,318.6672635)(455.09016078,318.67226349)(455.04016451,318.68226336)
\curveto(454.94016093,318.71226345)(454.84016103,318.73726343)(454.74016451,318.75726336)
\curveto(454.64016123,318.77726339)(454.54516133,318.80726336)(454.45516451,318.84726336)
\curveto(454.33516154,318.88726328)(454.22016165,318.92726324)(454.11016451,318.96726336)
\curveto(454.01016186,319.00726316)(453.90516197,319.05726311)(453.79516451,319.11726336)
\curveto(453.44516243,319.32726284)(453.14516273,319.57226259)(452.89516451,319.85226336)
\curveto(452.64516323,320.13226203)(452.43516344,320.4672617)(452.26516451,320.85726336)
\curveto(452.21516366,320.94726122)(452.1751637,321.04226112)(452.14516451,321.14226336)
\curveto(452.12516375,321.24226092)(452.10016377,321.34726082)(452.07016451,321.45726336)
\curveto(452.05016382,321.50726066)(452.04016383,321.55226061)(452.04016451,321.59226336)
\curveto(452.04016383,321.63226053)(452.03016384,321.67726049)(452.01016451,321.72726336)
\curveto(451.99016388,321.80726036)(451.98016389,321.88726028)(451.98016451,321.96726336)
\curveto(451.98016389,322.05726011)(451.9701639,322.14226002)(451.95016451,322.22226336)
\curveto(451.94016393,322.27225989)(451.93516394,322.31725985)(451.93516451,322.35726336)
\lineto(451.93516451,322.49226336)
\curveto(451.91516396,322.55225961)(451.90516397,322.63725953)(451.90516451,322.74726336)
\curveto(451.91516396,322.85725931)(451.93016394,322.94225922)(451.95016451,323.00226336)
\lineto(451.95016451,323.10726336)
\curveto(451.96016391,323.15725901)(451.96016391,323.20725896)(451.95016451,323.25726336)
\curveto(451.95016392,323.31725885)(451.96016391,323.37225879)(451.98016451,323.42226336)
\curveto(451.99016388,323.47225869)(451.99516388,323.51725865)(451.99516451,323.55726336)
\curveto(451.99516388,323.60725856)(452.00516387,323.65725851)(452.02516451,323.70726336)
\curveto(452.06516381,323.83725833)(452.10016377,323.9622582)(452.13016451,324.08226336)
\curveto(452.16016371,324.21225795)(452.20016367,324.33725783)(452.25016451,324.45726336)
\curveto(452.43016344,324.8672573)(452.64516323,325.20725696)(452.89516451,325.47726336)
\curveto(453.14516273,325.75725641)(453.45016242,326.01225615)(453.81016451,326.24226336)
\curveto(453.91016196,326.29225587)(454.01516186,326.33725583)(454.12516451,326.37726336)
\curveto(454.23516164,326.41725575)(454.34516153,326.4622557)(454.45516451,326.51226336)
\curveto(454.58516129,326.5622556)(454.72016115,326.59725557)(454.86016451,326.61726336)
\curveto(455.00016087,326.63725553)(455.14516073,326.6672555)(455.29516451,326.70726336)
\curveto(455.3751605,326.71725545)(455.45016042,326.72225544)(455.52016451,326.72226336)
\curveto(455.59016028,326.72225544)(455.66016021,326.72725544)(455.73016451,326.73726336)
\curveto(456.31015956,326.74725542)(456.81015906,326.68725548)(457.23016451,326.55726336)
\curveto(457.66015821,326.42725574)(458.04015783,326.24725592)(458.37016451,326.01726336)
\curveto(458.48015739,325.93725623)(458.59015728,325.84725632)(458.70016451,325.74726336)
\curveto(458.82015705,325.65725651)(458.92015695,325.55725661)(459.00016451,325.44726336)
\curveto(459.08015679,325.34725682)(459.15015672,325.24725692)(459.21016451,325.14726336)
\curveto(459.28015659,325.04725712)(459.35015652,324.94225722)(459.42016451,324.83226336)
\curveto(459.49015638,324.72225744)(459.54515633,324.60225756)(459.58516451,324.47226336)
\curveto(459.62515625,324.35225781)(459.6701562,324.22225794)(459.72016451,324.08226336)
\curveto(459.75015612,324.00225816)(459.7751561,323.91725825)(459.79516451,323.82726336)
\lineto(459.85516451,323.55726336)
\curveto(459.86515601,323.51725865)(459.870156,323.47725869)(459.87016451,323.43726336)
\curveto(459.870156,323.39725877)(459.875156,323.35725881)(459.88516451,323.31726336)
\curveto(459.90515597,323.2672589)(459.91015596,323.21225895)(459.90016451,323.15226336)
\curveto(459.89015598,323.09225907)(459.89515598,323.03725913)(459.91516451,322.98726336)
\moveto(457.81516451,322.44726336)
\curveto(457.82515805,322.49725967)(457.83015804,322.5672596)(457.83016451,322.65726336)
\curveto(457.83015804,322.75725941)(457.82515805,322.83225933)(457.81516451,322.88226336)
\lineto(457.81516451,323.00226336)
\curveto(457.79515808,323.05225911)(457.78515809,323.10725906)(457.78516451,323.16726336)
\curveto(457.78515809,323.22725894)(457.78015809,323.28225888)(457.77016451,323.33226336)
\curveto(457.7701581,323.37225879)(457.76515811,323.40225876)(457.75516451,323.42226336)
\lineto(457.69516451,323.66226336)
\curveto(457.68515819,323.75225841)(457.66515821,323.83725833)(457.63516451,323.91726336)
\curveto(457.52515835,324.17725799)(457.39515848,324.39725777)(457.24516451,324.57726336)
\curveto(457.09515878,324.7672574)(456.89515898,324.91725725)(456.64516451,325.02726336)
\curveto(456.58515929,325.04725712)(456.52515935,325.0622571)(456.46516451,325.07226336)
\curveto(456.40515947,325.09225707)(456.34015953,325.11225705)(456.27016451,325.13226336)
\curveto(456.19015968,325.15225701)(456.10515977,325.15725701)(456.01516451,325.14726336)
\lineto(455.74516451,325.14726336)
\curveto(455.71516016,325.12725704)(455.68016019,325.11725705)(455.64016451,325.11726336)
\curveto(455.60016027,325.12725704)(455.56516031,325.12725704)(455.53516451,325.11726336)
\lineto(455.32516451,325.05726336)
\curveto(455.26516061,325.04725712)(455.21016066,325.02725714)(455.16016451,324.99726336)
\curveto(454.91016096,324.88725728)(454.70516117,324.72725744)(454.54516451,324.51726336)
\curveto(454.39516148,324.31725785)(454.2751616,324.08225808)(454.18516451,323.81226336)
\curveto(454.15516172,323.71225845)(454.13016174,323.60725856)(454.11016451,323.49726336)
\curveto(454.10016177,323.38725878)(454.08516179,323.27725889)(454.06516451,323.16726336)
\curveto(454.05516182,323.11725905)(454.05016182,323.0672591)(454.05016451,323.01726336)
\lineto(454.05016451,322.86726336)
\curveto(454.03016184,322.79725937)(454.02016185,322.69225947)(454.02016451,322.55226336)
\curveto(454.03016184,322.41225975)(454.04516183,322.30725986)(454.06516451,322.23726336)
\lineto(454.06516451,322.10226336)
\curveto(454.08516179,322.02226014)(454.10016177,321.94226022)(454.11016451,321.86226336)
\curveto(454.12016175,321.79226037)(454.13516174,321.71726045)(454.15516451,321.63726336)
\curveto(454.25516162,321.33726083)(454.36016151,321.09226107)(454.47016451,320.90226336)
\curveto(454.59016128,320.72226144)(454.7751611,320.55726161)(455.02516451,320.40726336)
\curveto(455.09516078,320.35726181)(455.1701607,320.31726185)(455.25016451,320.28726336)
\curveto(455.34016053,320.25726191)(455.43016044,320.23226193)(455.52016451,320.21226336)
\curveto(455.56016031,320.20226196)(455.59516028,320.19726197)(455.62516451,320.19726336)
\curveto(455.65516022,320.20726196)(455.69016018,320.20726196)(455.73016451,320.19726336)
\lineto(455.85016451,320.16726336)
\curveto(455.90015997,320.167262)(455.94515993,320.17226199)(455.98516451,320.18226336)
\lineto(456.10516451,320.18226336)
\curveto(456.18515969,320.20226196)(456.26515961,320.21726195)(456.34516451,320.22726336)
\curveto(456.42515945,320.23726193)(456.50015937,320.25726191)(456.57016451,320.28726336)
\curveto(456.83015904,320.38726178)(457.04015883,320.52226164)(457.20016451,320.69226336)
\curveto(457.36015851,320.8622613)(457.49515838,321.07226109)(457.60516451,321.32226336)
\curveto(457.64515823,321.42226074)(457.6751582,321.52226064)(457.69516451,321.62226336)
\curveto(457.71515816,321.72226044)(457.74015813,321.82726034)(457.77016451,321.93726336)
\curveto(457.78015809,321.97726019)(457.78515809,322.01226015)(457.78516451,322.04226336)
\curveto(457.78515809,322.08226008)(457.79015808,322.12226004)(457.80016451,322.16226336)
\lineto(457.80016451,322.29726336)
\curveto(457.80015807,322.34725982)(457.80515807,322.39725977)(457.81516451,322.44726336)
}
}
{
\newrgbcolor{curcolor}{0 0 0}
\pscustom[linestyle=none,fillstyle=solid,fillcolor=curcolor]
{
}
}
{
\newrgbcolor{curcolor}{0 0 0}
\pscustom[linestyle=none,fillstyle=solid,fillcolor=curcolor]
{
\newpath
\moveto(473.06524263,319.65726336)
\lineto(473.06524263,319.23726336)
\curveto(473.06523426,319.10726306)(473.03523429,319.00226316)(472.97524263,318.92226336)
\curveto(472.9252344,318.87226329)(472.86023447,318.83726333)(472.78024263,318.81726336)
\curveto(472.70023463,318.80726336)(472.61023472,318.80226336)(472.51024263,318.80226336)
\lineto(471.68524263,318.80226336)
\lineto(471.40024263,318.80226336)
\curveto(471.32023601,318.81226335)(471.25523607,318.83726333)(471.20524263,318.87726336)
\curveto(471.13523619,318.92726324)(471.09523623,318.99226317)(471.08524263,319.07226336)
\curveto(471.07523625,319.15226301)(471.05523627,319.23226293)(471.02524263,319.31226336)
\curveto(471.00523632,319.33226283)(470.98523634,319.34726282)(470.96524263,319.35726336)
\curveto(470.95523637,319.37726279)(470.94023639,319.39726277)(470.92024263,319.41726336)
\curveto(470.81023652,319.41726275)(470.7302366,319.39226277)(470.68024263,319.34226336)
\lineto(470.53024263,319.19226336)
\curveto(470.46023687,319.14226302)(470.39523693,319.09726307)(470.33524263,319.05726336)
\curveto(470.27523705,319.02726314)(470.21023712,318.98726318)(470.14024263,318.93726336)
\curveto(470.10023723,318.91726325)(470.05523727,318.89726327)(470.00524263,318.87726336)
\curveto(469.96523736,318.85726331)(469.92023741,318.83726333)(469.87024263,318.81726336)
\curveto(469.7302376,318.7672634)(469.58023775,318.72226344)(469.42024263,318.68226336)
\curveto(469.37023796,318.6622635)(469.325238,318.65226351)(469.28524263,318.65226336)
\curveto(469.24523808,318.65226351)(469.20523812,318.64726352)(469.16524263,318.63726336)
\lineto(469.03024263,318.63726336)
\curveto(469.00023833,318.62726354)(468.96023837,318.62226354)(468.91024263,318.62226336)
\lineto(468.77524263,318.62226336)
\curveto(468.71523861,318.60226356)(468.6252387,318.59726357)(468.50524263,318.60726336)
\curveto(468.38523894,318.60726356)(468.30023903,318.61726355)(468.25024263,318.63726336)
\curveto(468.18023915,318.65726351)(468.11523921,318.6672635)(468.05524263,318.66726336)
\curveto(468.00523932,318.65726351)(467.95023938,318.6622635)(467.89024263,318.68226336)
\lineto(467.53024263,318.80226336)
\curveto(467.42023991,318.83226333)(467.31024002,318.87226329)(467.20024263,318.92226336)
\curveto(466.85024048,319.07226309)(466.53524079,319.30226286)(466.25524263,319.61226336)
\curveto(465.98524134,319.93226223)(465.77024156,320.2672619)(465.61024263,320.61726336)
\curveto(465.56024177,320.72726144)(465.52024181,320.83226133)(465.49024263,320.93226336)
\curveto(465.46024187,321.04226112)(465.4252419,321.15226101)(465.38524263,321.26226336)
\curveto(465.37524195,321.30226086)(465.37024196,321.33726083)(465.37024263,321.36726336)
\curveto(465.37024196,321.40726076)(465.36024197,321.45226071)(465.34024263,321.50226336)
\curveto(465.32024201,321.58226058)(465.30024203,321.6672605)(465.28024263,321.75726336)
\curveto(465.27024206,321.85726031)(465.25524207,321.95726021)(465.23524263,322.05726336)
\curveto(465.2252421,322.08726008)(465.22024211,322.12226004)(465.22024263,322.16226336)
\curveto(465.2302421,322.20225996)(465.2302421,322.23725993)(465.22024263,322.26726336)
\lineto(465.22024263,322.40226336)
\curveto(465.22024211,322.45225971)(465.21524211,322.50225966)(465.20524263,322.55226336)
\curveto(465.19524213,322.60225956)(465.19024214,322.65725951)(465.19024263,322.71726336)
\curveto(465.19024214,322.78725938)(465.19524213,322.84225932)(465.20524263,322.88226336)
\curveto(465.21524211,322.93225923)(465.22024211,322.97725919)(465.22024263,323.01726336)
\lineto(465.22024263,323.16726336)
\curveto(465.2302421,323.21725895)(465.2302421,323.2622589)(465.22024263,323.30226336)
\curveto(465.22024211,323.35225881)(465.2302421,323.40225876)(465.25024263,323.45226336)
\curveto(465.27024206,323.5622586)(465.28524204,323.6672585)(465.29524263,323.76726336)
\curveto(465.31524201,323.8672583)(465.34024199,323.9672582)(465.37024263,324.06726336)
\curveto(465.41024192,324.18725798)(465.44524188,324.30225786)(465.47524263,324.41226336)
\curveto(465.50524182,324.52225764)(465.54524178,324.63225753)(465.59524263,324.74226336)
\curveto(465.73524159,325.04225712)(465.91024142,325.32725684)(466.12024263,325.59726336)
\curveto(466.14024119,325.62725654)(466.16524116,325.65225651)(466.19524263,325.67226336)
\curveto(466.23524109,325.70225646)(466.26524106,325.73225643)(466.28524263,325.76226336)
\curveto(466.325241,325.81225635)(466.36524096,325.85725631)(466.40524263,325.89726336)
\curveto(466.44524088,325.93725623)(466.49024084,325.97725619)(466.54024263,326.01726336)
\curveto(466.58024075,326.03725613)(466.61524071,326.0622561)(466.64524263,326.09226336)
\curveto(466.67524065,326.13225603)(466.71024062,326.162256)(466.75024263,326.18226336)
\curveto(467.00024033,326.35225581)(467.29024004,326.49225567)(467.62024263,326.60226336)
\curveto(467.69023964,326.62225554)(467.76023957,326.63725553)(467.83024263,326.64726336)
\curveto(467.91023942,326.65725551)(467.99023934,326.67225549)(468.07024263,326.69226336)
\curveto(468.14023919,326.71225545)(468.2302391,326.72225544)(468.34024263,326.72226336)
\curveto(468.45023888,326.73225543)(468.56023877,326.73725543)(468.67024263,326.73726336)
\curveto(468.78023855,326.73725543)(468.88523844,326.73225543)(468.98524263,326.72226336)
\curveto(469.09523823,326.71225545)(469.18523814,326.69725547)(469.25524263,326.67726336)
\curveto(469.40523792,326.62725554)(469.55023778,326.58225558)(469.69024263,326.54226336)
\curveto(469.8302375,326.50225566)(469.96023737,326.44725572)(470.08024263,326.37726336)
\curveto(470.15023718,326.32725584)(470.21523711,326.27725589)(470.27524263,326.22726336)
\curveto(470.33523699,326.18725598)(470.40023693,326.14225602)(470.47024263,326.09226336)
\curveto(470.51023682,326.0622561)(470.56523676,326.02225614)(470.63524263,325.97226336)
\curveto(470.71523661,325.92225624)(470.79023654,325.92225624)(470.86024263,325.97226336)
\curveto(470.90023643,325.99225617)(470.92023641,326.02725614)(470.92024263,326.07726336)
\curveto(470.92023641,326.12725604)(470.9302364,326.17725599)(470.95024263,326.22726336)
\lineto(470.95024263,326.37726336)
\curveto(470.96023637,326.40725576)(470.96523636,326.44225572)(470.96524263,326.48226336)
\lineto(470.96524263,326.60226336)
\lineto(470.96524263,328.64226336)
\curveto(470.96523636,328.75225341)(470.96023637,328.87225329)(470.95024263,329.00226336)
\curveto(470.95023638,329.14225302)(470.97523635,329.24725292)(471.02524263,329.31726336)
\curveto(471.06523626,329.39725277)(471.14023619,329.44725272)(471.25024263,329.46726336)
\curveto(471.27023606,329.47725269)(471.29023604,329.47725269)(471.31024263,329.46726336)
\curveto(471.330236,329.4672527)(471.35023598,329.47225269)(471.37024263,329.48226336)
\lineto(472.43524263,329.48226336)
\curveto(472.55523477,329.48225268)(472.66523466,329.47725269)(472.76524263,329.46726336)
\curveto(472.86523446,329.45725271)(472.94023439,329.41725275)(472.99024263,329.34726336)
\curveto(473.04023429,329.2672529)(473.06523426,329.162253)(473.06524263,329.03226336)
\lineto(473.06524263,328.67226336)
\lineto(473.06524263,319.65726336)
\moveto(471.02524263,322.59726336)
\curveto(471.03523629,322.63725953)(471.03523629,322.67725949)(471.02524263,322.71726336)
\lineto(471.02524263,322.85226336)
\curveto(471.0252363,322.95225921)(471.02023631,323.05225911)(471.01024263,323.15226336)
\curveto(471.00023633,323.25225891)(470.98523634,323.34225882)(470.96524263,323.42226336)
\curveto(470.94523638,323.53225863)(470.9252364,323.63225853)(470.90524263,323.72226336)
\curveto(470.89523643,323.81225835)(470.87023646,323.89725827)(470.83024263,323.97726336)
\curveto(470.69023664,324.33725783)(470.48523684,324.62225754)(470.21524263,324.83226336)
\curveto(469.95523737,325.04225712)(469.57523775,325.14725702)(469.07524263,325.14726336)
\curveto(469.01523831,325.14725702)(468.93523839,325.13725703)(468.83524263,325.11726336)
\curveto(468.75523857,325.09725707)(468.68023865,325.07725709)(468.61024263,325.05726336)
\curveto(468.55023878,325.04725712)(468.49023884,325.02725714)(468.43024263,324.99726336)
\curveto(468.16023917,324.88725728)(467.95023938,324.71725745)(467.80024263,324.48726336)
\curveto(467.65023968,324.25725791)(467.5302398,323.99725817)(467.44024263,323.70726336)
\curveto(467.41023992,323.60725856)(467.39023994,323.50725866)(467.38024263,323.40726336)
\curveto(467.37023996,323.30725886)(467.35023998,323.20225896)(467.32024263,323.09226336)
\lineto(467.32024263,322.88226336)
\curveto(467.30024003,322.79225937)(467.29524003,322.6672595)(467.30524263,322.50726336)
\curveto(467.31524001,322.35725981)(467.33024,322.24725992)(467.35024263,322.17726336)
\lineto(467.35024263,322.08726336)
\curveto(467.36023997,322.0672601)(467.36523996,322.04726012)(467.36524263,322.02726336)
\curveto(467.38523994,321.94726022)(467.40023993,321.87226029)(467.41024263,321.80226336)
\curveto(467.4302399,321.73226043)(467.45023988,321.65726051)(467.47024263,321.57726336)
\curveto(467.64023969,321.05726111)(467.9302394,320.67226149)(468.34024263,320.42226336)
\curveto(468.47023886,320.33226183)(468.65023868,320.2622619)(468.88024263,320.21226336)
\curveto(468.92023841,320.20226196)(468.98023835,320.19726197)(469.06024263,320.19726336)
\curveto(469.09023824,320.18726198)(469.13523819,320.17726199)(469.19524263,320.16726336)
\curveto(469.26523806,320.167262)(469.32023801,320.17226199)(469.36024263,320.18226336)
\curveto(469.44023789,320.20226196)(469.52023781,320.21726195)(469.60024263,320.22726336)
\curveto(469.68023765,320.23726193)(469.76023757,320.25726191)(469.84024263,320.28726336)
\curveto(470.09023724,320.39726177)(470.29023704,320.53726163)(470.44024263,320.70726336)
\curveto(470.59023674,320.87726129)(470.72023661,321.09226107)(470.83024263,321.35226336)
\curveto(470.87023646,321.44226072)(470.90023643,321.53226063)(470.92024263,321.62226336)
\curveto(470.94023639,321.72226044)(470.96023637,321.82726034)(470.98024263,321.93726336)
\curveto(470.99023634,321.98726018)(470.99023634,322.03226013)(470.98024263,322.07226336)
\curveto(470.98023635,322.12226004)(470.99023634,322.17225999)(471.01024263,322.22226336)
\curveto(471.02023631,322.25225991)(471.0252363,322.28725988)(471.02524263,322.32726336)
\lineto(471.02524263,322.46226336)
\lineto(471.02524263,322.59726336)
}
}
{
\newrgbcolor{curcolor}{0 0 0}
\pscustom[linestyle=none,fillstyle=solid,fillcolor=curcolor]
{
\newpath
\moveto(482.01016451,322.74726336)
\curveto(482.03015634,322.6672595)(482.03015634,322.57725959)(482.01016451,322.47726336)
\curveto(481.99015638,322.37725979)(481.95515642,322.31225985)(481.90516451,322.28226336)
\curveto(481.85515652,322.24225992)(481.78015659,322.21225995)(481.68016451,322.19226336)
\curveto(481.59015678,322.18225998)(481.48515689,322.17225999)(481.36516451,322.16226336)
\lineto(481.02016451,322.16226336)
\curveto(480.91015746,322.17225999)(480.81015756,322.17725999)(480.72016451,322.17726336)
\lineto(477.06016451,322.17726336)
\lineto(476.85016451,322.17726336)
\curveto(476.79016158,322.17725999)(476.73516164,322.16726)(476.68516451,322.14726336)
\curveto(476.60516177,322.10726006)(476.55516182,322.0672601)(476.53516451,322.02726336)
\curveto(476.51516186,322.00726016)(476.49516188,321.9672602)(476.47516451,321.90726336)
\curveto(476.45516192,321.85726031)(476.45016192,321.80726036)(476.46016451,321.75726336)
\curveto(476.48016189,321.69726047)(476.49016188,321.63726053)(476.49016451,321.57726336)
\curveto(476.50016187,321.52726064)(476.51516186,321.47226069)(476.53516451,321.41226336)
\curveto(476.61516176,321.17226099)(476.71016166,320.97226119)(476.82016451,320.81226336)
\curveto(476.94016143,320.6622615)(477.10016127,320.52726164)(477.30016451,320.40726336)
\curveto(477.38016099,320.35726181)(477.46016091,320.32226184)(477.54016451,320.30226336)
\curveto(477.63016074,320.29226187)(477.72016065,320.27226189)(477.81016451,320.24226336)
\curveto(477.89016048,320.22226194)(478.00016037,320.20726196)(478.14016451,320.19726336)
\curveto(478.28016009,320.18726198)(478.40015997,320.19226197)(478.50016451,320.21226336)
\lineto(478.63516451,320.21226336)
\curveto(478.73515964,320.23226193)(478.82515955,320.25226191)(478.90516451,320.27226336)
\curveto(478.99515938,320.30226186)(479.08015929,320.33226183)(479.16016451,320.36226336)
\curveto(479.26015911,320.41226175)(479.370159,320.47726169)(479.49016451,320.55726336)
\curveto(479.62015875,320.63726153)(479.71515866,320.71726145)(479.77516451,320.79726336)
\curveto(479.82515855,320.8672613)(479.8751585,320.93226123)(479.92516451,320.99226336)
\curveto(479.98515839,321.0622611)(480.05515832,321.11226105)(480.13516451,321.14226336)
\curveto(480.23515814,321.19226097)(480.36015801,321.21226095)(480.51016451,321.20226336)
\lineto(480.94516451,321.20226336)
\lineto(481.12516451,321.20226336)
\curveto(481.19515718,321.21226095)(481.25515712,321.20726096)(481.30516451,321.18726336)
\lineto(481.45516451,321.18726336)
\curveto(481.55515682,321.167261)(481.62515675,321.14226102)(481.66516451,321.11226336)
\curveto(481.70515667,321.09226107)(481.72515665,321.04726112)(481.72516451,320.97726336)
\curveto(481.73515664,320.90726126)(481.73015664,320.84726132)(481.71016451,320.79726336)
\curveto(481.66015671,320.65726151)(481.60515677,320.53226163)(481.54516451,320.42226336)
\curveto(481.48515689,320.31226185)(481.41515696,320.20226196)(481.33516451,320.09226336)
\curveto(481.11515726,319.7622624)(480.86515751,319.49726267)(480.58516451,319.29726336)
\curveto(480.30515807,319.09726307)(479.95515842,318.92726324)(479.53516451,318.78726336)
\curveto(479.42515895,318.74726342)(479.31515906,318.72226344)(479.20516451,318.71226336)
\curveto(479.09515928,318.70226346)(478.98015939,318.68226348)(478.86016451,318.65226336)
\curveto(478.82015955,318.64226352)(478.7751596,318.64226352)(478.72516451,318.65226336)
\curveto(478.68515969,318.65226351)(478.64515973,318.64726352)(478.60516451,318.63726336)
\lineto(478.44016451,318.63726336)
\curveto(478.39015998,318.61726355)(478.33016004,318.61226355)(478.26016451,318.62226336)
\curveto(478.20016017,318.62226354)(478.14516023,318.62726354)(478.09516451,318.63726336)
\curveto(478.01516036,318.64726352)(477.94516043,318.64726352)(477.88516451,318.63726336)
\curveto(477.82516055,318.62726354)(477.76016061,318.63226353)(477.69016451,318.65226336)
\curveto(477.64016073,318.67226349)(477.58516079,318.68226348)(477.52516451,318.68226336)
\curveto(477.46516091,318.68226348)(477.41016096,318.69226347)(477.36016451,318.71226336)
\curveto(477.25016112,318.73226343)(477.14016123,318.75726341)(477.03016451,318.78726336)
\curveto(476.92016145,318.80726336)(476.82016155,318.84226332)(476.73016451,318.89226336)
\curveto(476.62016175,318.93226323)(476.51516186,318.9672632)(476.41516451,318.99726336)
\curveto(476.32516205,319.03726313)(476.24016213,319.08226308)(476.16016451,319.13226336)
\curveto(475.84016253,319.33226283)(475.55516282,319.5622626)(475.30516451,319.82226336)
\curveto(475.05516332,320.09226207)(474.85016352,320.40226176)(474.69016451,320.75226336)
\curveto(474.64016373,320.8622613)(474.60016377,320.97226119)(474.57016451,321.08226336)
\curveto(474.54016383,321.20226096)(474.50016387,321.32226084)(474.45016451,321.44226336)
\curveto(474.44016393,321.48226068)(474.43516394,321.51726065)(474.43516451,321.54726336)
\curveto(474.43516394,321.58726058)(474.43016394,321.62726054)(474.42016451,321.66726336)
\curveto(474.38016399,321.78726038)(474.35516402,321.91726025)(474.34516451,322.05726336)
\lineto(474.31516451,322.47726336)
\curveto(474.31516406,322.52725964)(474.31016406,322.58225958)(474.30016451,322.64226336)
\curveto(474.30016407,322.70225946)(474.30516407,322.75725941)(474.31516451,322.80726336)
\lineto(474.31516451,322.98726336)
\lineto(474.36016451,323.34726336)
\curveto(474.40016397,323.51725865)(474.43516394,323.68225848)(474.46516451,323.84226336)
\curveto(474.49516388,324.00225816)(474.54016383,324.15225801)(474.60016451,324.29226336)
\curveto(475.03016334,325.33225683)(475.76016261,326.0672561)(476.79016451,326.49726336)
\curveto(476.93016144,326.55725561)(477.0701613,326.59725557)(477.21016451,326.61726336)
\curveto(477.36016101,326.64725552)(477.51516086,326.68225548)(477.67516451,326.72226336)
\curveto(477.75516062,326.73225543)(477.83016054,326.73725543)(477.90016451,326.73726336)
\curveto(477.9701604,326.73725543)(478.04516033,326.74225542)(478.12516451,326.75226336)
\curveto(478.63515974,326.7622554)(479.0701593,326.70225546)(479.43016451,326.57226336)
\curveto(479.80015857,326.45225571)(480.13015824,326.29225587)(480.42016451,326.09226336)
\curveto(480.51015786,326.03225613)(480.60015777,325.9622562)(480.69016451,325.88226336)
\curveto(480.78015759,325.81225635)(480.86015751,325.73725643)(480.93016451,325.65726336)
\curveto(480.96015741,325.60725656)(481.00015737,325.5672566)(481.05016451,325.53726336)
\curveto(481.13015724,325.42725674)(481.20515717,325.31225685)(481.27516451,325.19226336)
\curveto(481.34515703,325.08225708)(481.42015695,324.9672572)(481.50016451,324.84726336)
\curveto(481.55015682,324.75725741)(481.59015678,324.6622575)(481.62016451,324.56226336)
\curveto(481.66015671,324.47225769)(481.70015667,324.37225779)(481.74016451,324.26226336)
\curveto(481.79015658,324.13225803)(481.83015654,323.99725817)(481.86016451,323.85726336)
\curveto(481.89015648,323.71725845)(481.92515645,323.57725859)(481.96516451,323.43726336)
\curveto(481.98515639,323.35725881)(481.99015638,323.2672589)(481.98016451,323.16726336)
\curveto(481.98015639,323.07725909)(481.99015638,322.99225917)(482.01016451,322.91226336)
\lineto(482.01016451,322.74726336)
\moveto(479.76016451,323.63226336)
\curveto(479.83015854,323.73225843)(479.83515854,323.85225831)(479.77516451,323.99226336)
\curveto(479.72515865,324.14225802)(479.68515869,324.25225791)(479.65516451,324.32226336)
\curveto(479.51515886,324.59225757)(479.33015904,324.79725737)(479.10016451,324.93726336)
\curveto(478.8701595,325.08725708)(478.55015982,325.167257)(478.14016451,325.17726336)
\curveto(478.11016026,325.15725701)(478.0751603,325.15225701)(478.03516451,325.16226336)
\curveto(477.99516038,325.17225699)(477.96016041,325.17225699)(477.93016451,325.16226336)
\curveto(477.88016049,325.14225702)(477.82516055,325.12725704)(477.76516451,325.11726336)
\curveto(477.70516067,325.11725705)(477.65016072,325.10725706)(477.60016451,325.08726336)
\curveto(477.16016121,324.94725722)(476.83516154,324.67225749)(476.62516451,324.26226336)
\curveto(476.60516177,324.22225794)(476.58016179,324.167258)(476.55016451,324.09726336)
\curveto(476.53016184,324.03725813)(476.51516186,323.97225819)(476.50516451,323.90226336)
\curveto(476.49516188,323.84225832)(476.49516188,323.78225838)(476.50516451,323.72226336)
\curveto(476.52516185,323.6622585)(476.56016181,323.61225855)(476.61016451,323.57226336)
\curveto(476.69016168,323.52225864)(476.80016157,323.49725867)(476.94016451,323.49726336)
\lineto(477.34516451,323.49726336)
\lineto(479.01016451,323.49726336)
\lineto(479.44516451,323.49726336)
\curveto(479.60515877,323.50725866)(479.71015866,323.55225861)(479.76016451,323.63226336)
}
}
{
\newrgbcolor{curcolor}{0 0 0}
\pscustom[linestyle=none,fillstyle=solid,fillcolor=curcolor]
{
}
}
{
\newrgbcolor{curcolor}{0 0 0}
\pscustom[linestyle=none,fillstyle=solid,fillcolor=curcolor]
{
\newpath
\moveto(491.84360201,326.73726336)
\curveto(491.95359669,326.73725543)(492.0485966,326.72725544)(492.12860201,326.70726336)
\curveto(492.21859643,326.68725548)(492.28859636,326.64225552)(492.33860201,326.57226336)
\curveto(492.39859625,326.49225567)(492.42859622,326.35225581)(492.42860201,326.15226336)
\lineto(492.42860201,325.64226336)
\lineto(492.42860201,325.26726336)
\curveto(492.43859621,325.12725704)(492.42359622,325.01725715)(492.38360201,324.93726336)
\curveto(492.3435963,324.8672573)(492.28359636,324.82225734)(492.20360201,324.80226336)
\curveto(492.13359651,324.78225738)(492.0485966,324.77225739)(491.94860201,324.77226336)
\curveto(491.85859679,324.77225739)(491.75859689,324.77725739)(491.64860201,324.78726336)
\curveto(491.5485971,324.79725737)(491.45359719,324.79225737)(491.36360201,324.77226336)
\curveto(491.29359735,324.75225741)(491.22359742,324.73725743)(491.15360201,324.72726336)
\curveto(491.08359756,324.72725744)(491.01859763,324.71725745)(490.95860201,324.69726336)
\curveto(490.79859785,324.64725752)(490.63859801,324.57225759)(490.47860201,324.47226336)
\curveto(490.31859833,324.38225778)(490.19359845,324.27725789)(490.10360201,324.15726336)
\curveto(490.05359859,324.07725809)(489.99859865,323.99225817)(489.93860201,323.90226336)
\curveto(489.88859876,323.82225834)(489.83859881,323.73725843)(489.78860201,323.64726336)
\curveto(489.75859889,323.5672586)(489.72859892,323.48225868)(489.69860201,323.39226336)
\lineto(489.63860201,323.15226336)
\curveto(489.61859903,323.08225908)(489.60859904,323.00725916)(489.60860201,322.92726336)
\curveto(489.60859904,322.85725931)(489.59859905,322.78725938)(489.57860201,322.71726336)
\curveto(489.56859908,322.67725949)(489.56359908,322.63725953)(489.56360201,322.59726336)
\curveto(489.57359907,322.5672596)(489.57359907,322.53725963)(489.56360201,322.50726336)
\lineto(489.56360201,322.26726336)
\curveto(489.5435991,322.19725997)(489.53859911,322.11726005)(489.54860201,322.02726336)
\curveto(489.55859909,321.94726022)(489.56359908,321.8672603)(489.56360201,321.78726336)
\lineto(489.56360201,320.82726336)
\lineto(489.56360201,319.55226336)
\curveto(489.56359908,319.42226274)(489.55859909,319.30226286)(489.54860201,319.19226336)
\curveto(489.53859911,319.08226308)(489.50859914,318.99226317)(489.45860201,318.92226336)
\curveto(489.43859921,318.89226327)(489.40359924,318.8672633)(489.35360201,318.84726336)
\curveto(489.31359933,318.83726333)(489.26859938,318.82726334)(489.21860201,318.81726336)
\lineto(489.14360201,318.81726336)
\curveto(489.09359955,318.80726336)(489.03859961,318.80226336)(488.97860201,318.80226336)
\lineto(488.81360201,318.80226336)
\lineto(488.16860201,318.80226336)
\curveto(488.10860054,318.81226335)(488.0436006,318.81726335)(487.97360201,318.81726336)
\lineto(487.77860201,318.81726336)
\curveto(487.72860092,318.83726333)(487.67860097,318.85226331)(487.62860201,318.86226336)
\curveto(487.57860107,318.88226328)(487.5436011,318.91726325)(487.52360201,318.96726336)
\curveto(487.48360116,319.01726315)(487.45860119,319.08726308)(487.44860201,319.17726336)
\lineto(487.44860201,319.47726336)
\lineto(487.44860201,320.49726336)
\lineto(487.44860201,324.72726336)
\lineto(487.44860201,325.83726336)
\lineto(487.44860201,326.12226336)
\curveto(487.4486012,326.22225594)(487.46860118,326.30225586)(487.50860201,326.36226336)
\curveto(487.55860109,326.44225572)(487.63360101,326.49225567)(487.73360201,326.51226336)
\curveto(487.83360081,326.53225563)(487.95360069,326.54225562)(488.09360201,326.54226336)
\lineto(488.85860201,326.54226336)
\curveto(488.97859967,326.54225562)(489.08359956,326.53225563)(489.17360201,326.51226336)
\curveto(489.26359938,326.50225566)(489.33359931,326.45725571)(489.38360201,326.37726336)
\curveto(489.41359923,326.32725584)(489.42859922,326.25725591)(489.42860201,326.16726336)
\lineto(489.45860201,325.89726336)
\curveto(489.46859918,325.81725635)(489.48359916,325.74225642)(489.50360201,325.67226336)
\curveto(489.53359911,325.60225656)(489.58359906,325.5672566)(489.65360201,325.56726336)
\curveto(489.67359897,325.58725658)(489.69359895,325.59725657)(489.71360201,325.59726336)
\curveto(489.73359891,325.59725657)(489.75359889,325.60725656)(489.77360201,325.62726336)
\curveto(489.83359881,325.67725649)(489.88359876,325.73225643)(489.92360201,325.79226336)
\curveto(489.97359867,325.8622563)(490.03359861,325.92225624)(490.10360201,325.97226336)
\curveto(490.1435985,326.00225616)(490.17859847,326.03225613)(490.20860201,326.06226336)
\curveto(490.23859841,326.10225606)(490.27359837,326.13725603)(490.31360201,326.16726336)
\lineto(490.58360201,326.34726336)
\curveto(490.68359796,326.40725576)(490.78359786,326.4622557)(490.88360201,326.51226336)
\curveto(490.98359766,326.55225561)(491.08359756,326.58725558)(491.18360201,326.61726336)
\lineto(491.51360201,326.70726336)
\curveto(491.5435971,326.71725545)(491.59859705,326.71725545)(491.67860201,326.70726336)
\curveto(491.76859688,326.70725546)(491.82359682,326.71725545)(491.84360201,326.73726336)
}
}
{
\newrgbcolor{curcolor}{0 0 0}
\pscustom[linestyle=none,fillstyle=solid,fillcolor=curcolor]
{
\newpath
\moveto(500.35000826,322.74726336)
\curveto(500.37000009,322.6672595)(500.37000009,322.57725959)(500.35000826,322.47726336)
\curveto(500.33000013,322.37725979)(500.29500017,322.31225985)(500.24500826,322.28226336)
\curveto(500.19500027,322.24225992)(500.12000034,322.21225995)(500.02000826,322.19226336)
\curveto(499.93000053,322.18225998)(499.82500064,322.17225999)(499.70500826,322.16226336)
\lineto(499.36000826,322.16226336)
\curveto(499.25000121,322.17225999)(499.15000131,322.17725999)(499.06000826,322.17726336)
\lineto(495.40000826,322.17726336)
\lineto(495.19000826,322.17726336)
\curveto(495.13000533,322.17725999)(495.07500539,322.16726)(495.02500826,322.14726336)
\curveto(494.94500552,322.10726006)(494.89500557,322.0672601)(494.87500826,322.02726336)
\curveto(494.85500561,322.00726016)(494.83500563,321.9672602)(494.81500826,321.90726336)
\curveto(494.79500567,321.85726031)(494.79000567,321.80726036)(494.80000826,321.75726336)
\curveto(494.82000564,321.69726047)(494.83000563,321.63726053)(494.83000826,321.57726336)
\curveto(494.84000562,321.52726064)(494.85500561,321.47226069)(494.87500826,321.41226336)
\curveto(494.95500551,321.17226099)(495.05000541,320.97226119)(495.16000826,320.81226336)
\curveto(495.28000518,320.6622615)(495.44000502,320.52726164)(495.64000826,320.40726336)
\curveto(495.72000474,320.35726181)(495.80000466,320.32226184)(495.88000826,320.30226336)
\curveto(495.97000449,320.29226187)(496.0600044,320.27226189)(496.15000826,320.24226336)
\curveto(496.23000423,320.22226194)(496.34000412,320.20726196)(496.48000826,320.19726336)
\curveto(496.62000384,320.18726198)(496.74000372,320.19226197)(496.84000826,320.21226336)
\lineto(496.97500826,320.21226336)
\curveto(497.07500339,320.23226193)(497.1650033,320.25226191)(497.24500826,320.27226336)
\curveto(497.33500313,320.30226186)(497.42000304,320.33226183)(497.50000826,320.36226336)
\curveto(497.60000286,320.41226175)(497.71000275,320.47726169)(497.83000826,320.55726336)
\curveto(497.9600025,320.63726153)(498.05500241,320.71726145)(498.11500826,320.79726336)
\curveto(498.1650023,320.8672613)(498.21500225,320.93226123)(498.26500826,320.99226336)
\curveto(498.32500214,321.0622611)(498.39500207,321.11226105)(498.47500826,321.14226336)
\curveto(498.57500189,321.19226097)(498.70000176,321.21226095)(498.85000826,321.20226336)
\lineto(499.28500826,321.20226336)
\lineto(499.46500826,321.20226336)
\curveto(499.53500093,321.21226095)(499.59500087,321.20726096)(499.64500826,321.18726336)
\lineto(499.79500826,321.18726336)
\curveto(499.89500057,321.167261)(499.9650005,321.14226102)(500.00500826,321.11226336)
\curveto(500.04500042,321.09226107)(500.0650004,321.04726112)(500.06500826,320.97726336)
\curveto(500.07500039,320.90726126)(500.07000039,320.84726132)(500.05000826,320.79726336)
\curveto(500.00000046,320.65726151)(499.94500052,320.53226163)(499.88500826,320.42226336)
\curveto(499.82500064,320.31226185)(499.75500071,320.20226196)(499.67500826,320.09226336)
\curveto(499.45500101,319.7622624)(499.20500126,319.49726267)(498.92500826,319.29726336)
\curveto(498.64500182,319.09726307)(498.29500217,318.92726324)(497.87500826,318.78726336)
\curveto(497.7650027,318.74726342)(497.65500281,318.72226344)(497.54500826,318.71226336)
\curveto(497.43500303,318.70226346)(497.32000314,318.68226348)(497.20000826,318.65226336)
\curveto(497.1600033,318.64226352)(497.11500335,318.64226352)(497.06500826,318.65226336)
\curveto(497.02500344,318.65226351)(496.98500348,318.64726352)(496.94500826,318.63726336)
\lineto(496.78000826,318.63726336)
\curveto(496.73000373,318.61726355)(496.67000379,318.61226355)(496.60000826,318.62226336)
\curveto(496.54000392,318.62226354)(496.48500398,318.62726354)(496.43500826,318.63726336)
\curveto(496.35500411,318.64726352)(496.28500418,318.64726352)(496.22500826,318.63726336)
\curveto(496.1650043,318.62726354)(496.10000436,318.63226353)(496.03000826,318.65226336)
\curveto(495.98000448,318.67226349)(495.92500454,318.68226348)(495.86500826,318.68226336)
\curveto(495.80500466,318.68226348)(495.75000471,318.69226347)(495.70000826,318.71226336)
\curveto(495.59000487,318.73226343)(495.48000498,318.75726341)(495.37000826,318.78726336)
\curveto(495.2600052,318.80726336)(495.1600053,318.84226332)(495.07000826,318.89226336)
\curveto(494.9600055,318.93226323)(494.85500561,318.9672632)(494.75500826,318.99726336)
\curveto(494.6650058,319.03726313)(494.58000588,319.08226308)(494.50000826,319.13226336)
\curveto(494.18000628,319.33226283)(493.89500657,319.5622626)(493.64500826,319.82226336)
\curveto(493.39500707,320.09226207)(493.19000727,320.40226176)(493.03000826,320.75226336)
\curveto(492.98000748,320.8622613)(492.94000752,320.97226119)(492.91000826,321.08226336)
\curveto(492.88000758,321.20226096)(492.84000762,321.32226084)(492.79000826,321.44226336)
\curveto(492.78000768,321.48226068)(492.77500769,321.51726065)(492.77500826,321.54726336)
\curveto(492.77500769,321.58726058)(492.77000769,321.62726054)(492.76000826,321.66726336)
\curveto(492.72000774,321.78726038)(492.69500777,321.91726025)(492.68500826,322.05726336)
\lineto(492.65500826,322.47726336)
\curveto(492.65500781,322.52725964)(492.65000781,322.58225958)(492.64000826,322.64226336)
\curveto(492.64000782,322.70225946)(492.64500782,322.75725941)(492.65500826,322.80726336)
\lineto(492.65500826,322.98726336)
\lineto(492.70000826,323.34726336)
\curveto(492.74000772,323.51725865)(492.77500769,323.68225848)(492.80500826,323.84226336)
\curveto(492.83500763,324.00225816)(492.88000758,324.15225801)(492.94000826,324.29226336)
\curveto(493.37000709,325.33225683)(494.10000636,326.0672561)(495.13000826,326.49726336)
\curveto(495.27000519,326.55725561)(495.41000505,326.59725557)(495.55000826,326.61726336)
\curveto(495.70000476,326.64725552)(495.85500461,326.68225548)(496.01500826,326.72226336)
\curveto(496.09500437,326.73225543)(496.17000429,326.73725543)(496.24000826,326.73726336)
\curveto(496.31000415,326.73725543)(496.38500408,326.74225542)(496.46500826,326.75226336)
\curveto(496.97500349,326.7622554)(497.41000305,326.70225546)(497.77000826,326.57226336)
\curveto(498.14000232,326.45225571)(498.47000199,326.29225587)(498.76000826,326.09226336)
\curveto(498.85000161,326.03225613)(498.94000152,325.9622562)(499.03000826,325.88226336)
\curveto(499.12000134,325.81225635)(499.20000126,325.73725643)(499.27000826,325.65726336)
\curveto(499.30000116,325.60725656)(499.34000112,325.5672566)(499.39000826,325.53726336)
\curveto(499.47000099,325.42725674)(499.54500092,325.31225685)(499.61500826,325.19226336)
\curveto(499.68500078,325.08225708)(499.7600007,324.9672572)(499.84000826,324.84726336)
\curveto(499.89000057,324.75725741)(499.93000053,324.6622575)(499.96000826,324.56226336)
\curveto(500.00000046,324.47225769)(500.04000042,324.37225779)(500.08000826,324.26226336)
\curveto(500.13000033,324.13225803)(500.17000029,323.99725817)(500.20000826,323.85726336)
\curveto(500.23000023,323.71725845)(500.2650002,323.57725859)(500.30500826,323.43726336)
\curveto(500.32500014,323.35725881)(500.33000013,323.2672589)(500.32000826,323.16726336)
\curveto(500.32000014,323.07725909)(500.33000013,322.99225917)(500.35000826,322.91226336)
\lineto(500.35000826,322.74726336)
\moveto(498.10000826,323.63226336)
\curveto(498.17000229,323.73225843)(498.17500229,323.85225831)(498.11500826,323.99226336)
\curveto(498.0650024,324.14225802)(498.02500244,324.25225791)(497.99500826,324.32226336)
\curveto(497.85500261,324.59225757)(497.67000279,324.79725737)(497.44000826,324.93726336)
\curveto(497.21000325,325.08725708)(496.89000357,325.167257)(496.48000826,325.17726336)
\curveto(496.45000401,325.15725701)(496.41500405,325.15225701)(496.37500826,325.16226336)
\curveto(496.33500413,325.17225699)(496.30000416,325.17225699)(496.27000826,325.16226336)
\curveto(496.22000424,325.14225702)(496.1650043,325.12725704)(496.10500826,325.11726336)
\curveto(496.04500442,325.11725705)(495.99000447,325.10725706)(495.94000826,325.08726336)
\curveto(495.50000496,324.94725722)(495.17500529,324.67225749)(494.96500826,324.26226336)
\curveto(494.94500552,324.22225794)(494.92000554,324.167258)(494.89000826,324.09726336)
\curveto(494.87000559,324.03725813)(494.85500561,323.97225819)(494.84500826,323.90226336)
\curveto(494.83500563,323.84225832)(494.83500563,323.78225838)(494.84500826,323.72226336)
\curveto(494.8650056,323.6622585)(494.90000556,323.61225855)(494.95000826,323.57226336)
\curveto(495.03000543,323.52225864)(495.14000532,323.49725867)(495.28000826,323.49726336)
\lineto(495.68500826,323.49726336)
\lineto(497.35000826,323.49726336)
\lineto(497.78500826,323.49726336)
\curveto(497.94500252,323.50725866)(498.05000241,323.55225861)(498.10000826,323.63226336)
}
}
{
\newrgbcolor{curcolor}{0 0 0}
\pscustom[linestyle=none,fillstyle=solid,fillcolor=curcolor]
{
\newpath
\moveto(505.16828951,326.75226336)
\curveto(505.97828435,326.77225539)(506.65328367,326.65225551)(507.19328951,326.39226336)
\curveto(507.74328258,326.13225603)(508.17828215,325.7622564)(508.49828951,325.28226336)
\curveto(508.65828167,325.04225712)(508.77828155,324.7672574)(508.85828951,324.45726336)
\curveto(508.87828145,324.40725776)(508.89328143,324.34225782)(508.90328951,324.26226336)
\curveto(508.9232814,324.18225798)(508.9232814,324.11225805)(508.90328951,324.05226336)
\curveto(508.86328146,323.94225822)(508.79328153,323.87725829)(508.69328951,323.85726336)
\curveto(508.59328173,323.84725832)(508.47328185,323.84225832)(508.33328951,323.84226336)
\lineto(507.55328951,323.84226336)
\lineto(507.26828951,323.84226336)
\curveto(507.17828315,323.84225832)(507.10328322,323.8622583)(507.04328951,323.90226336)
\curveto(506.96328336,323.94225822)(506.90828342,324.00225816)(506.87828951,324.08226336)
\curveto(506.84828348,324.17225799)(506.80828352,324.2622579)(506.75828951,324.35226336)
\curveto(506.69828363,324.4622577)(506.63328369,324.5622576)(506.56328951,324.65226336)
\curveto(506.49328383,324.74225742)(506.41328391,324.82225734)(506.32328951,324.89226336)
\curveto(506.18328414,324.98225718)(506.0282843,325.05225711)(505.85828951,325.10226336)
\curveto(505.79828453,325.12225704)(505.73828459,325.13225703)(505.67828951,325.13226336)
\curveto(505.61828471,325.13225703)(505.56328476,325.14225702)(505.51328951,325.16226336)
\lineto(505.36328951,325.16226336)
\curveto(505.16328516,325.162257)(505.00328532,325.14225702)(504.88328951,325.10226336)
\curveto(504.59328573,325.01225715)(504.35828597,324.87225729)(504.17828951,324.68226336)
\curveto(503.99828633,324.50225766)(503.85328647,324.28225788)(503.74328951,324.02226336)
\curveto(503.69328663,323.91225825)(503.65328667,323.79225837)(503.62328951,323.66226336)
\curveto(503.60328672,323.54225862)(503.57828675,323.41225875)(503.54828951,323.27226336)
\curveto(503.53828679,323.23225893)(503.53328679,323.19225897)(503.53328951,323.15226336)
\curveto(503.53328679,323.11225905)(503.5282868,323.07225909)(503.51828951,323.03226336)
\curveto(503.49828683,322.93225923)(503.48828684,322.79225937)(503.48828951,322.61226336)
\curveto(503.49828683,322.43225973)(503.51328681,322.29225987)(503.53328951,322.19226336)
\curveto(503.53328679,322.11226005)(503.53828679,322.05726011)(503.54828951,322.02726336)
\curveto(503.56828676,321.95726021)(503.57828675,321.88726028)(503.57828951,321.81726336)
\curveto(503.58828674,321.74726042)(503.60328672,321.67726049)(503.62328951,321.60726336)
\curveto(503.70328662,321.37726079)(503.79828653,321.167261)(503.90828951,320.97726336)
\curveto(504.01828631,320.78726138)(504.15828617,320.62726154)(504.32828951,320.49726336)
\curveto(504.36828596,320.4672617)(504.4282859,320.43226173)(504.50828951,320.39226336)
\curveto(504.61828571,320.32226184)(504.7282856,320.27726189)(504.83828951,320.25726336)
\curveto(504.95828537,320.23726193)(505.10328522,320.21726195)(505.27328951,320.19726336)
\lineto(505.36328951,320.19726336)
\curveto(505.40328492,320.19726197)(505.43328489,320.20226196)(505.45328951,320.21226336)
\lineto(505.58828951,320.21226336)
\curveto(505.65828467,320.23226193)(505.7232846,320.24726192)(505.78328951,320.25726336)
\curveto(505.85328447,320.27726189)(505.91828441,320.29726187)(505.97828951,320.31726336)
\curveto(506.27828405,320.44726172)(506.50828382,320.63726153)(506.66828951,320.88726336)
\curveto(506.70828362,320.93726123)(506.74328358,320.99226117)(506.77328951,321.05226336)
\curveto(506.80328352,321.12226104)(506.8282835,321.18226098)(506.84828951,321.23226336)
\curveto(506.88828344,321.34226082)(506.9232834,321.43726073)(506.95328951,321.51726336)
\curveto(506.98328334,321.60726056)(507.05328327,321.67726049)(507.16328951,321.72726336)
\curveto(507.25328307,321.7672604)(507.39828293,321.78226038)(507.59828951,321.77226336)
\lineto(508.09328951,321.77226336)
\lineto(508.30328951,321.77226336)
\curveto(508.38328194,321.78226038)(508.44828188,321.77726039)(508.49828951,321.75726336)
\lineto(508.61828951,321.75726336)
\lineto(508.73828951,321.72726336)
\curveto(508.77828155,321.72726044)(508.80828152,321.71726045)(508.82828951,321.69726336)
\curveto(508.87828145,321.65726051)(508.90828142,321.59726057)(508.91828951,321.51726336)
\curveto(508.93828139,321.44726072)(508.93828139,321.37226079)(508.91828951,321.29226336)
\curveto(508.8282815,320.9622612)(508.71828161,320.6672615)(508.58828951,320.40726336)
\curveto(508.17828215,319.63726253)(507.5232828,319.10226306)(506.62328951,318.80226336)
\curveto(506.5232838,318.77226339)(506.41828391,318.75226341)(506.30828951,318.74226336)
\curveto(506.19828413,318.72226344)(506.08828424,318.69726347)(505.97828951,318.66726336)
\curveto(505.91828441,318.65726351)(505.85828447,318.65226351)(505.79828951,318.65226336)
\curveto(505.73828459,318.65226351)(505.67828465,318.64726352)(505.61828951,318.63726336)
\lineto(505.45328951,318.63726336)
\curveto(505.40328492,318.61726355)(505.328285,318.61226355)(505.22828951,318.62226336)
\curveto(505.1282852,318.62226354)(505.05328527,318.62726354)(505.00328951,318.63726336)
\curveto(504.9232854,318.65726351)(504.84828548,318.6672635)(504.77828951,318.66726336)
\curveto(504.71828561,318.65726351)(504.65328567,318.6622635)(504.58328951,318.68226336)
\lineto(504.43328951,318.71226336)
\curveto(504.38328594,318.71226345)(504.33328599,318.71726345)(504.28328951,318.72726336)
\curveto(504.17328615,318.75726341)(504.06828626,318.78726338)(503.96828951,318.81726336)
\curveto(503.86828646,318.84726332)(503.77328655,318.88226328)(503.68328951,318.92226336)
\curveto(503.21328711,319.12226304)(502.81828751,319.37726279)(502.49828951,319.68726336)
\curveto(502.17828815,320.00726216)(501.91828841,320.40226176)(501.71828951,320.87226336)
\curveto(501.66828866,320.9622612)(501.6282887,321.05726111)(501.59828951,321.15726336)
\lineto(501.50828951,321.48726336)
\curveto(501.49828883,321.52726064)(501.49328883,321.5622606)(501.49328951,321.59226336)
\curveto(501.49328883,321.63226053)(501.48328884,321.67726049)(501.46328951,321.72726336)
\curveto(501.44328888,321.79726037)(501.43328889,321.8672603)(501.43328951,321.93726336)
\curveto(501.43328889,322.01726015)(501.4232889,322.09226007)(501.40328951,322.16226336)
\lineto(501.40328951,322.41726336)
\curveto(501.38328894,322.4672597)(501.37328895,322.52225964)(501.37328951,322.58226336)
\curveto(501.37328895,322.65225951)(501.38328894,322.71225945)(501.40328951,322.76226336)
\curveto(501.41328891,322.81225935)(501.41328891,322.85725931)(501.40328951,322.89726336)
\curveto(501.39328893,322.93725923)(501.39328893,322.97725919)(501.40328951,323.01726336)
\curveto(501.4232889,323.08725908)(501.4282889,323.15225901)(501.41828951,323.21226336)
\curveto(501.41828891,323.27225889)(501.4282889,323.33225883)(501.44828951,323.39226336)
\curveto(501.49828883,323.57225859)(501.53828879,323.74225842)(501.56828951,323.90226336)
\curveto(501.59828873,324.07225809)(501.64328868,324.23725793)(501.70328951,324.39726336)
\curveto(501.9232884,324.90725726)(502.19828813,325.33225683)(502.52828951,325.67226336)
\curveto(502.86828746,326.01225615)(503.29828703,326.28725588)(503.81828951,326.49726336)
\curveto(503.95828637,326.55725561)(504.10328622,326.59725557)(504.25328951,326.61726336)
\curveto(504.40328592,326.64725552)(504.55828577,326.68225548)(504.71828951,326.72226336)
\curveto(504.79828553,326.73225543)(504.87328545,326.73725543)(504.94328951,326.73726336)
\curveto(505.01328531,326.73725543)(505.08828524,326.74225542)(505.16828951,326.75226336)
}
}
{
\newrgbcolor{curcolor}{0 0 0}
\pscustom[linestyle=none,fillstyle=solid,fillcolor=curcolor]
{
\newpath
\moveto(510.63157076,326.52726336)
\lineto(511.75657076,326.52726336)
\curveto(511.86656832,326.52725564)(511.96656822,326.52225564)(512.05657076,326.51226336)
\curveto(512.14656804,326.50225566)(512.21156798,326.4672557)(512.25157076,326.40726336)
\curveto(512.30156789,326.34725582)(512.33156786,326.2622559)(512.34157076,326.15226336)
\curveto(512.35156784,326.05225611)(512.35656783,325.94725622)(512.35657076,325.83726336)
\lineto(512.35657076,324.78726336)
\lineto(512.35657076,322.55226336)
\curveto(512.35656783,322.19225997)(512.37156782,321.85226031)(512.40157076,321.53226336)
\curveto(512.43156776,321.21226095)(512.52156767,320.94726122)(512.67157076,320.73726336)
\curveto(512.81156738,320.52726164)(513.03656715,320.37726179)(513.34657076,320.28726336)
\curveto(513.39656679,320.27726189)(513.43656675,320.27226189)(513.46657076,320.27226336)
\curveto(513.50656668,320.27226189)(513.55156664,320.2672619)(513.60157076,320.25726336)
\curveto(513.65156654,320.24726192)(513.70656648,320.24226192)(513.76657076,320.24226336)
\curveto(513.82656636,320.24226192)(513.87156632,320.24726192)(513.90157076,320.25726336)
\curveto(513.95156624,320.27726189)(513.9915662,320.28226188)(514.02157076,320.27226336)
\curveto(514.06156613,320.2622619)(514.10156609,320.2672619)(514.14157076,320.28726336)
\curveto(514.35156584,320.33726183)(514.51656567,320.40226176)(514.63657076,320.48226336)
\curveto(514.81656537,320.59226157)(514.95656523,320.73226143)(515.05657076,320.90226336)
\curveto(515.16656502,321.08226108)(515.24156495,321.27726089)(515.28157076,321.48726336)
\curveto(515.33156486,321.70726046)(515.36156483,321.94726022)(515.37157076,322.20726336)
\curveto(515.38156481,322.47725969)(515.3865648,322.75725941)(515.38657076,323.04726336)
\lineto(515.38657076,324.86226336)
\lineto(515.38657076,325.83726336)
\lineto(515.38657076,326.10726336)
\curveto(515.3865648,326.20725596)(515.40656478,326.28725588)(515.44657076,326.34726336)
\curveto(515.49656469,326.43725573)(515.57156462,326.48725568)(515.67157076,326.49726336)
\curveto(515.77156442,326.51725565)(515.8915643,326.52725564)(516.03157076,326.52726336)
\lineto(516.82657076,326.52726336)
\lineto(517.11157076,326.52726336)
\curveto(517.20156299,326.52725564)(517.27656291,326.50725566)(517.33657076,326.46726336)
\curveto(517.41656277,326.41725575)(517.46156273,326.34225582)(517.47157076,326.24226336)
\curveto(517.48156271,326.14225602)(517.4865627,326.02725614)(517.48657076,325.89726336)
\lineto(517.48657076,324.75726336)
\lineto(517.48657076,320.54226336)
\lineto(517.48657076,319.47726336)
\lineto(517.48657076,319.17726336)
\curveto(517.4865627,319.07726309)(517.46656272,319.00226316)(517.42657076,318.95226336)
\curveto(517.37656281,318.87226329)(517.30156289,318.82726334)(517.20157076,318.81726336)
\curveto(517.10156309,318.80726336)(516.99656319,318.80226336)(516.88657076,318.80226336)
\lineto(516.07657076,318.80226336)
\curveto(515.96656422,318.80226336)(515.86656432,318.80726336)(515.77657076,318.81726336)
\curveto(515.69656449,318.82726334)(515.63156456,318.8672633)(515.58157076,318.93726336)
\curveto(515.56156463,318.9672632)(515.54156465,319.01226315)(515.52157076,319.07226336)
\curveto(515.51156468,319.13226303)(515.49656469,319.19226297)(515.47657076,319.25226336)
\curveto(515.46656472,319.31226285)(515.45156474,319.3672628)(515.43157076,319.41726336)
\curveto(515.41156478,319.4672627)(515.38156481,319.49726267)(515.34157076,319.50726336)
\curveto(515.32156487,319.52726264)(515.29656489,319.53226263)(515.26657076,319.52226336)
\curveto(515.23656495,319.51226265)(515.21156498,319.50226266)(515.19157076,319.49226336)
\curveto(515.12156507,319.45226271)(515.06156513,319.40726276)(515.01157076,319.35726336)
\curveto(514.96156523,319.30726286)(514.90656528,319.2622629)(514.84657076,319.22226336)
\curveto(514.80656538,319.19226297)(514.76656542,319.15726301)(514.72657076,319.11726336)
\curveto(514.69656549,319.08726308)(514.65656553,319.05726311)(514.60657076,319.02726336)
\curveto(514.37656581,318.88726328)(514.10656608,318.77726339)(513.79657076,318.69726336)
\curveto(513.72656646,318.67726349)(513.65656653,318.6672635)(513.58657076,318.66726336)
\curveto(513.51656667,318.65726351)(513.44156675,318.64226352)(513.36157076,318.62226336)
\curveto(513.32156687,318.61226355)(513.27656691,318.61226355)(513.22657076,318.62226336)
\curveto(513.186567,318.62226354)(513.14656704,318.61726355)(513.10657076,318.60726336)
\curveto(513.07656711,318.59726357)(513.01156718,318.59726357)(512.91157076,318.60726336)
\curveto(512.82156737,318.60726356)(512.76156743,318.61226355)(512.73157076,318.62226336)
\curveto(512.68156751,318.62226354)(512.63156756,318.62726354)(512.58157076,318.63726336)
\lineto(512.43157076,318.63726336)
\curveto(512.31156788,318.6672635)(512.19656799,318.69226347)(512.08657076,318.71226336)
\curveto(511.97656821,318.73226343)(511.86656832,318.7622634)(511.75657076,318.80226336)
\curveto(511.70656848,318.82226334)(511.66156853,318.83726333)(511.62157076,318.84726336)
\curveto(511.5915686,318.8672633)(511.55156864,318.88726328)(511.50157076,318.90726336)
\curveto(511.15156904,319.09726307)(510.87156932,319.3622628)(510.66157076,319.70226336)
\curveto(510.53156966,319.91226225)(510.43656975,320.162262)(510.37657076,320.45226336)
\curveto(510.31656987,320.75226141)(510.27656991,321.0672611)(510.25657076,321.39726336)
\curveto(510.24656994,321.73726043)(510.24156995,322.08226008)(510.24157076,322.43226336)
\curveto(510.25156994,322.79225937)(510.25656993,323.14725902)(510.25657076,323.49726336)
\lineto(510.25657076,325.53726336)
\curveto(510.25656993,325.6672565)(510.25156994,325.81725635)(510.24157076,325.98726336)
\curveto(510.24156995,326.167256)(510.26656992,326.29725587)(510.31657076,326.37726336)
\curveto(510.34656984,326.42725574)(510.40656978,326.47225569)(510.49657076,326.51226336)
\curveto(510.55656963,326.51225565)(510.60156959,326.51725565)(510.63157076,326.52726336)
}
}
{
\newrgbcolor{curcolor}{0 0 0}
\pscustom[linestyle=none,fillstyle=solid,fillcolor=curcolor]
{
\newpath
\moveto(523.54282076,326.73726336)
\curveto(523.65281544,326.73725543)(523.74781535,326.72725544)(523.82782076,326.70726336)
\curveto(523.91781518,326.68725548)(523.98781511,326.64225552)(524.03782076,326.57226336)
\curveto(524.097815,326.49225567)(524.12781497,326.35225581)(524.12782076,326.15226336)
\lineto(524.12782076,325.64226336)
\lineto(524.12782076,325.26726336)
\curveto(524.13781496,325.12725704)(524.12281497,325.01725715)(524.08282076,324.93726336)
\curveto(524.04281505,324.8672573)(523.98281511,324.82225734)(523.90282076,324.80226336)
\curveto(523.83281526,324.78225738)(523.74781535,324.77225739)(523.64782076,324.77226336)
\curveto(523.55781554,324.77225739)(523.45781564,324.77725739)(523.34782076,324.78726336)
\curveto(523.24781585,324.79725737)(523.15281594,324.79225737)(523.06282076,324.77226336)
\curveto(522.9928161,324.75225741)(522.92281617,324.73725743)(522.85282076,324.72726336)
\curveto(522.78281631,324.72725744)(522.71781638,324.71725745)(522.65782076,324.69726336)
\curveto(522.4978166,324.64725752)(522.33781676,324.57225759)(522.17782076,324.47226336)
\curveto(522.01781708,324.38225778)(521.8928172,324.27725789)(521.80282076,324.15726336)
\curveto(521.75281734,324.07725809)(521.6978174,323.99225817)(521.63782076,323.90226336)
\curveto(521.58781751,323.82225834)(521.53781756,323.73725843)(521.48782076,323.64726336)
\curveto(521.45781764,323.5672586)(521.42781767,323.48225868)(521.39782076,323.39226336)
\lineto(521.33782076,323.15226336)
\curveto(521.31781778,323.08225908)(521.30781779,323.00725916)(521.30782076,322.92726336)
\curveto(521.30781779,322.85725931)(521.2978178,322.78725938)(521.27782076,322.71726336)
\curveto(521.26781783,322.67725949)(521.26281783,322.63725953)(521.26282076,322.59726336)
\curveto(521.27281782,322.5672596)(521.27281782,322.53725963)(521.26282076,322.50726336)
\lineto(521.26282076,322.26726336)
\curveto(521.24281785,322.19725997)(521.23781786,322.11726005)(521.24782076,322.02726336)
\curveto(521.25781784,321.94726022)(521.26281783,321.8672603)(521.26282076,321.78726336)
\lineto(521.26282076,320.82726336)
\lineto(521.26282076,319.55226336)
\curveto(521.26281783,319.42226274)(521.25781784,319.30226286)(521.24782076,319.19226336)
\curveto(521.23781786,319.08226308)(521.20781789,318.99226317)(521.15782076,318.92226336)
\curveto(521.13781796,318.89226327)(521.10281799,318.8672633)(521.05282076,318.84726336)
\curveto(521.01281808,318.83726333)(520.96781813,318.82726334)(520.91782076,318.81726336)
\lineto(520.84282076,318.81726336)
\curveto(520.7928183,318.80726336)(520.73781836,318.80226336)(520.67782076,318.80226336)
\lineto(520.51282076,318.80226336)
\lineto(519.86782076,318.80226336)
\curveto(519.80781929,318.81226335)(519.74281935,318.81726335)(519.67282076,318.81726336)
\lineto(519.47782076,318.81726336)
\curveto(519.42781967,318.83726333)(519.37781972,318.85226331)(519.32782076,318.86226336)
\curveto(519.27781982,318.88226328)(519.24281985,318.91726325)(519.22282076,318.96726336)
\curveto(519.18281991,319.01726315)(519.15781994,319.08726308)(519.14782076,319.17726336)
\lineto(519.14782076,319.47726336)
\lineto(519.14782076,320.49726336)
\lineto(519.14782076,324.72726336)
\lineto(519.14782076,325.83726336)
\lineto(519.14782076,326.12226336)
\curveto(519.14781995,326.22225594)(519.16781993,326.30225586)(519.20782076,326.36226336)
\curveto(519.25781984,326.44225572)(519.33281976,326.49225567)(519.43282076,326.51226336)
\curveto(519.53281956,326.53225563)(519.65281944,326.54225562)(519.79282076,326.54226336)
\lineto(520.55782076,326.54226336)
\curveto(520.67781842,326.54225562)(520.78281831,326.53225563)(520.87282076,326.51226336)
\curveto(520.96281813,326.50225566)(521.03281806,326.45725571)(521.08282076,326.37726336)
\curveto(521.11281798,326.32725584)(521.12781797,326.25725591)(521.12782076,326.16726336)
\lineto(521.15782076,325.89726336)
\curveto(521.16781793,325.81725635)(521.18281791,325.74225642)(521.20282076,325.67226336)
\curveto(521.23281786,325.60225656)(521.28281781,325.5672566)(521.35282076,325.56726336)
\curveto(521.37281772,325.58725658)(521.3928177,325.59725657)(521.41282076,325.59726336)
\curveto(521.43281766,325.59725657)(521.45281764,325.60725656)(521.47282076,325.62726336)
\curveto(521.53281756,325.67725649)(521.58281751,325.73225643)(521.62282076,325.79226336)
\curveto(521.67281742,325.8622563)(521.73281736,325.92225624)(521.80282076,325.97226336)
\curveto(521.84281725,326.00225616)(521.87781722,326.03225613)(521.90782076,326.06226336)
\curveto(521.93781716,326.10225606)(521.97281712,326.13725603)(522.01282076,326.16726336)
\lineto(522.28282076,326.34726336)
\curveto(522.38281671,326.40725576)(522.48281661,326.4622557)(522.58282076,326.51226336)
\curveto(522.68281641,326.55225561)(522.78281631,326.58725558)(522.88282076,326.61726336)
\lineto(523.21282076,326.70726336)
\curveto(523.24281585,326.71725545)(523.2978158,326.71725545)(523.37782076,326.70726336)
\curveto(523.46781563,326.70725546)(523.52281557,326.71725545)(523.54282076,326.73726336)
}
}
{
\newrgbcolor{curcolor}{0 0 0}
\pscustom[linestyle=none,fillstyle=solid,fillcolor=curcolor]
{
\newpath
\moveto(527.91789888,326.75226336)
\curveto(528.66789438,326.77225539)(529.31789373,326.68725548)(529.86789888,326.49726336)
\curveto(530.42789262,326.31725585)(530.8528922,326.00225616)(531.14289888,325.55226336)
\curveto(531.21289184,325.44225672)(531.27289178,325.32725684)(531.32289888,325.20726336)
\curveto(531.38289167,325.09725707)(531.43289162,324.97225719)(531.47289888,324.83226336)
\curveto(531.49289156,324.77225739)(531.50289155,324.70725746)(531.50289888,324.63726336)
\curveto(531.50289155,324.5672576)(531.49289156,324.50725766)(531.47289888,324.45726336)
\curveto(531.43289162,324.39725777)(531.37789167,324.35725781)(531.30789888,324.33726336)
\curveto(531.25789179,324.31725785)(531.19789185,324.30725786)(531.12789888,324.30726336)
\lineto(530.91789888,324.30726336)
\lineto(530.25789888,324.30726336)
\curveto(530.18789286,324.30725786)(530.11789293,324.30225786)(530.04789888,324.29226336)
\curveto(529.97789307,324.29225787)(529.91289314,324.30225786)(529.85289888,324.32226336)
\curveto(529.7528933,324.34225782)(529.67789337,324.38225778)(529.62789888,324.44226336)
\curveto(529.57789347,324.50225766)(529.53289352,324.5622576)(529.49289888,324.62226336)
\lineto(529.37289888,324.83226336)
\curveto(529.34289371,324.91225725)(529.29289376,324.97725719)(529.22289888,325.02726336)
\curveto(529.12289393,325.10725706)(529.02289403,325.167257)(528.92289888,325.20726336)
\curveto(528.83289422,325.24725692)(528.71789433,325.28225688)(528.57789888,325.31226336)
\curveto(528.50789454,325.33225683)(528.40289465,325.34725682)(528.26289888,325.35726336)
\curveto(528.13289492,325.3672568)(528.03289502,325.3622568)(527.96289888,325.34226336)
\lineto(527.85789888,325.34226336)
\lineto(527.70789888,325.31226336)
\curveto(527.66789538,325.31225685)(527.62289543,325.30725686)(527.57289888,325.29726336)
\curveto(527.40289565,325.24725692)(527.26289579,325.17725699)(527.15289888,325.08726336)
\curveto(527.052896,325.00725716)(526.98289607,324.88225728)(526.94289888,324.71226336)
\curveto(526.92289613,324.64225752)(526.92289613,324.57725759)(526.94289888,324.51726336)
\curveto(526.96289609,324.45725771)(526.98289607,324.40725776)(527.00289888,324.36726336)
\curveto(527.07289598,324.24725792)(527.1528959,324.15225801)(527.24289888,324.08226336)
\curveto(527.34289571,324.01225815)(527.45789559,323.95225821)(527.58789888,323.90226336)
\curveto(527.77789527,323.82225834)(527.98289507,323.75225841)(528.20289888,323.69226336)
\lineto(528.89289888,323.54226336)
\curveto(529.13289392,323.50225866)(529.36289369,323.45225871)(529.58289888,323.39226336)
\curveto(529.81289324,323.34225882)(530.02789302,323.27725889)(530.22789888,323.19726336)
\curveto(530.31789273,323.15725901)(530.40289265,323.12225904)(530.48289888,323.09226336)
\curveto(530.57289248,323.07225909)(530.65789239,323.03725913)(530.73789888,322.98726336)
\curveto(530.92789212,322.8672593)(531.09789195,322.73725943)(531.24789888,322.59726336)
\curveto(531.40789164,322.45725971)(531.53289152,322.28225988)(531.62289888,322.07226336)
\curveto(531.6528914,322.00226016)(531.67789137,321.93226023)(531.69789888,321.86226336)
\curveto(531.71789133,321.79226037)(531.73789131,321.71726045)(531.75789888,321.63726336)
\curveto(531.76789128,321.57726059)(531.77289128,321.48226068)(531.77289888,321.35226336)
\curveto(531.78289127,321.23226093)(531.78289127,321.13726103)(531.77289888,321.06726336)
\lineto(531.77289888,320.99226336)
\curveto(531.7528913,320.93226123)(531.73789131,320.87226129)(531.72789888,320.81226336)
\curveto(531.72789132,320.7622614)(531.72289133,320.71226145)(531.71289888,320.66226336)
\curveto(531.64289141,320.3622618)(531.53289152,320.09726207)(531.38289888,319.86726336)
\curveto(531.22289183,319.62726254)(531.02789202,319.43226273)(530.79789888,319.28226336)
\curveto(530.56789248,319.13226303)(530.30789274,319.00226316)(530.01789888,318.89226336)
\curveto(529.90789314,318.84226332)(529.78789326,318.80726336)(529.65789888,318.78726336)
\curveto(529.53789351,318.7672634)(529.41789363,318.74226342)(529.29789888,318.71226336)
\curveto(529.20789384,318.69226347)(529.11289394,318.68226348)(529.01289888,318.68226336)
\curveto(528.92289413,318.67226349)(528.83289422,318.65726351)(528.74289888,318.63726336)
\lineto(528.47289888,318.63726336)
\curveto(528.41289464,318.61726355)(528.30789474,318.60726356)(528.15789888,318.60726336)
\curveto(528.01789503,318.60726356)(527.91789513,318.61726355)(527.85789888,318.63726336)
\curveto(527.82789522,318.63726353)(527.79289526,318.64226352)(527.75289888,318.65226336)
\lineto(527.64789888,318.65226336)
\curveto(527.52789552,318.67226349)(527.40789564,318.68726348)(527.28789888,318.69726336)
\curveto(527.16789588,318.70726346)(527.052896,318.72726344)(526.94289888,318.75726336)
\curveto(526.5528965,318.8672633)(526.20789684,318.99226317)(525.90789888,319.13226336)
\curveto(525.60789744,319.28226288)(525.3528977,319.50226266)(525.14289888,319.79226336)
\curveto(525.00289805,319.98226218)(524.88289817,320.20226196)(524.78289888,320.45226336)
\curveto(524.76289829,320.51226165)(524.74289831,320.59226157)(524.72289888,320.69226336)
\curveto(524.70289835,320.74226142)(524.68789836,320.81226135)(524.67789888,320.90226336)
\curveto(524.66789838,320.99226117)(524.67289838,321.0672611)(524.69289888,321.12726336)
\curveto(524.72289833,321.19726097)(524.77289828,321.24726092)(524.84289888,321.27726336)
\curveto(524.89289816,321.29726087)(524.9528981,321.30726086)(525.02289888,321.30726336)
\lineto(525.24789888,321.30726336)
\lineto(525.95289888,321.30726336)
\lineto(526.19289888,321.30726336)
\curveto(526.27289678,321.30726086)(526.34289671,321.29726087)(526.40289888,321.27726336)
\curveto(526.51289654,321.23726093)(526.58289647,321.17226099)(526.61289888,321.08226336)
\curveto(526.6528964,320.99226117)(526.69789635,320.89726127)(526.74789888,320.79726336)
\curveto(526.76789628,320.74726142)(526.80289625,320.68226148)(526.85289888,320.60226336)
\curveto(526.91289614,320.52226164)(526.96289609,320.47226169)(527.00289888,320.45226336)
\curveto(527.12289593,320.35226181)(527.23789581,320.27226189)(527.34789888,320.21226336)
\curveto(527.45789559,320.162262)(527.59789545,320.11226205)(527.76789888,320.06226336)
\curveto(527.81789523,320.04226212)(527.86789518,320.03226213)(527.91789888,320.03226336)
\curveto(527.96789508,320.04226212)(528.01789503,320.04226212)(528.06789888,320.03226336)
\curveto(528.1478949,320.01226215)(528.23289482,320.00226216)(528.32289888,320.00226336)
\curveto(528.42289463,320.01226215)(528.50789454,320.02726214)(528.57789888,320.04726336)
\curveto(528.62789442,320.05726211)(528.67289438,320.0622621)(528.71289888,320.06226336)
\curveto(528.76289429,320.0622621)(528.81289424,320.07226209)(528.86289888,320.09226336)
\curveto(529.00289405,320.14226202)(529.12789392,320.20226196)(529.23789888,320.27226336)
\curveto(529.35789369,320.34226182)(529.4528936,320.43226173)(529.52289888,320.54226336)
\curveto(529.57289348,320.62226154)(529.61289344,320.74726142)(529.64289888,320.91726336)
\curveto(529.66289339,320.98726118)(529.66289339,321.05226111)(529.64289888,321.11226336)
\curveto(529.62289343,321.17226099)(529.60289345,321.22226094)(529.58289888,321.26226336)
\curveto(529.51289354,321.40226076)(529.42289363,321.50726066)(529.31289888,321.57726336)
\curveto(529.21289384,321.64726052)(529.09289396,321.71226045)(528.95289888,321.77226336)
\curveto(528.76289429,321.85226031)(528.56289449,321.91726025)(528.35289888,321.96726336)
\curveto(528.14289491,322.01726015)(527.93289512,322.07226009)(527.72289888,322.13226336)
\curveto(527.64289541,322.15226001)(527.55789549,322.16726)(527.46789888,322.17726336)
\curveto(527.38789566,322.18725998)(527.30789574,322.20225996)(527.22789888,322.22226336)
\curveto(526.90789614,322.31225985)(526.60289645,322.39725977)(526.31289888,322.47726336)
\curveto(526.02289703,322.5672596)(525.75789729,322.69725947)(525.51789888,322.86726336)
\curveto(525.23789781,323.0672591)(525.03289802,323.33725883)(524.90289888,323.67726336)
\curveto(524.88289817,323.74725842)(524.86289819,323.84225832)(524.84289888,323.96226336)
\curveto(524.82289823,324.03225813)(524.80789824,324.11725805)(524.79789888,324.21726336)
\curveto(524.78789826,324.31725785)(524.79289826,324.40725776)(524.81289888,324.48726336)
\curveto(524.83289822,324.53725763)(524.83789821,324.57725759)(524.82789888,324.60726336)
\curveto(524.81789823,324.64725752)(524.82289823,324.69225747)(524.84289888,324.74226336)
\curveto(524.86289819,324.85225731)(524.88289817,324.95225721)(524.90289888,325.04226336)
\curveto(524.93289812,325.14225702)(524.96789808,325.23725693)(525.00789888,325.32726336)
\curveto(525.13789791,325.61725655)(525.31789773,325.85225631)(525.54789888,326.03226336)
\curveto(525.77789727,326.21225595)(526.03789701,326.35725581)(526.32789888,326.46726336)
\curveto(526.43789661,326.51725565)(526.5528965,326.55225561)(526.67289888,326.57226336)
\curveto(526.79289626,326.60225556)(526.91789613,326.63225553)(527.04789888,326.66226336)
\curveto(527.10789594,326.68225548)(527.16789588,326.69225547)(527.22789888,326.69226336)
\lineto(527.40789888,326.72226336)
\curveto(527.48789556,326.73225543)(527.57289548,326.73725543)(527.66289888,326.73726336)
\curveto(527.7528953,326.73725543)(527.83789521,326.74225542)(527.91789888,326.75226336)
}
}
{
\newrgbcolor{curcolor}{0 0 0}
\pscustom[linestyle=none,fillstyle=solid,fillcolor=curcolor]
{
\newpath
\moveto(540.77453951,322.98726336)
\curveto(540.79453094,322.92725924)(540.80453093,322.84225932)(540.80453951,322.73226336)
\curveto(540.80453093,322.62225954)(540.79453094,322.53725963)(540.77453951,322.47726336)
\lineto(540.77453951,322.32726336)
\curveto(540.75453098,322.24725992)(540.74453099,322.16726)(540.74453951,322.08726336)
\curveto(540.75453098,322.00726016)(540.74953098,321.92726024)(540.72953951,321.84726336)
\curveto(540.70953102,321.77726039)(540.69453104,321.71226045)(540.68453951,321.65226336)
\curveto(540.67453106,321.59226057)(540.66453107,321.52726064)(540.65453951,321.45726336)
\curveto(540.61453112,321.34726082)(540.57953115,321.23226093)(540.54953951,321.11226336)
\curveto(540.51953121,321.00226116)(540.47953125,320.89726127)(540.42953951,320.79726336)
\curveto(540.21953151,320.31726185)(539.94453179,319.92726224)(539.60453951,319.62726336)
\curveto(539.26453247,319.32726284)(538.85453288,319.07726309)(538.37453951,318.87726336)
\curveto(538.25453348,318.82726334)(538.1295336,318.79226337)(537.99953951,318.77226336)
\curveto(537.87953385,318.74226342)(537.75453398,318.71226345)(537.62453951,318.68226336)
\curveto(537.57453416,318.6622635)(537.51953421,318.65226351)(537.45953951,318.65226336)
\curveto(537.39953433,318.65226351)(537.34453439,318.64726352)(537.29453951,318.63726336)
\lineto(537.18953951,318.63726336)
\curveto(537.15953457,318.62726354)(537.1295346,318.62226354)(537.09953951,318.62226336)
\curveto(537.04953468,318.61226355)(536.96953476,318.60726356)(536.85953951,318.60726336)
\curveto(536.74953498,318.59726357)(536.66453507,318.60226356)(536.60453951,318.62226336)
\lineto(536.45453951,318.62226336)
\curveto(536.40453533,318.63226353)(536.34953538,318.63726353)(536.28953951,318.63726336)
\curveto(536.23953549,318.62726354)(536.18953554,318.63226353)(536.13953951,318.65226336)
\curveto(536.09953563,318.6622635)(536.05953567,318.6672635)(536.01953951,318.66726336)
\curveto(535.98953574,318.6672635)(535.94953578,318.67226349)(535.89953951,318.68226336)
\curveto(535.79953593,318.71226345)(535.69953603,318.73726343)(535.59953951,318.75726336)
\curveto(535.49953623,318.77726339)(535.40453633,318.80726336)(535.31453951,318.84726336)
\curveto(535.19453654,318.88726328)(535.07953665,318.92726324)(534.96953951,318.96726336)
\curveto(534.86953686,319.00726316)(534.76453697,319.05726311)(534.65453951,319.11726336)
\curveto(534.30453743,319.32726284)(534.00453773,319.57226259)(533.75453951,319.85226336)
\curveto(533.50453823,320.13226203)(533.29453844,320.4672617)(533.12453951,320.85726336)
\curveto(533.07453866,320.94726122)(533.0345387,321.04226112)(533.00453951,321.14226336)
\curveto(532.98453875,321.24226092)(532.95953877,321.34726082)(532.92953951,321.45726336)
\curveto(532.90953882,321.50726066)(532.89953883,321.55226061)(532.89953951,321.59226336)
\curveto(532.89953883,321.63226053)(532.88953884,321.67726049)(532.86953951,321.72726336)
\curveto(532.84953888,321.80726036)(532.83953889,321.88726028)(532.83953951,321.96726336)
\curveto(532.83953889,322.05726011)(532.8295389,322.14226002)(532.80953951,322.22226336)
\curveto(532.79953893,322.27225989)(532.79453894,322.31725985)(532.79453951,322.35726336)
\lineto(532.79453951,322.49226336)
\curveto(532.77453896,322.55225961)(532.76453897,322.63725953)(532.76453951,322.74726336)
\curveto(532.77453896,322.85725931)(532.78953894,322.94225922)(532.80953951,323.00226336)
\lineto(532.80953951,323.10726336)
\curveto(532.81953891,323.15725901)(532.81953891,323.20725896)(532.80953951,323.25726336)
\curveto(532.80953892,323.31725885)(532.81953891,323.37225879)(532.83953951,323.42226336)
\curveto(532.84953888,323.47225869)(532.85453888,323.51725865)(532.85453951,323.55726336)
\curveto(532.85453888,323.60725856)(532.86453887,323.65725851)(532.88453951,323.70726336)
\curveto(532.92453881,323.83725833)(532.95953877,323.9622582)(532.98953951,324.08226336)
\curveto(533.01953871,324.21225795)(533.05953867,324.33725783)(533.10953951,324.45726336)
\curveto(533.28953844,324.8672573)(533.50453823,325.20725696)(533.75453951,325.47726336)
\curveto(534.00453773,325.75725641)(534.30953742,326.01225615)(534.66953951,326.24226336)
\curveto(534.76953696,326.29225587)(534.87453686,326.33725583)(534.98453951,326.37726336)
\curveto(535.09453664,326.41725575)(535.20453653,326.4622557)(535.31453951,326.51226336)
\curveto(535.44453629,326.5622556)(535.57953615,326.59725557)(535.71953951,326.61726336)
\curveto(535.85953587,326.63725553)(536.00453573,326.6672555)(536.15453951,326.70726336)
\curveto(536.2345355,326.71725545)(536.30953542,326.72225544)(536.37953951,326.72226336)
\curveto(536.44953528,326.72225544)(536.51953521,326.72725544)(536.58953951,326.73726336)
\curveto(537.16953456,326.74725542)(537.66953406,326.68725548)(538.08953951,326.55726336)
\curveto(538.51953321,326.42725574)(538.89953283,326.24725592)(539.22953951,326.01726336)
\curveto(539.33953239,325.93725623)(539.44953228,325.84725632)(539.55953951,325.74726336)
\curveto(539.67953205,325.65725651)(539.77953195,325.55725661)(539.85953951,325.44726336)
\curveto(539.93953179,325.34725682)(540.00953172,325.24725692)(540.06953951,325.14726336)
\curveto(540.13953159,325.04725712)(540.20953152,324.94225722)(540.27953951,324.83226336)
\curveto(540.34953138,324.72225744)(540.40453133,324.60225756)(540.44453951,324.47226336)
\curveto(540.48453125,324.35225781)(540.5295312,324.22225794)(540.57953951,324.08226336)
\curveto(540.60953112,324.00225816)(540.6345311,323.91725825)(540.65453951,323.82726336)
\lineto(540.71453951,323.55726336)
\curveto(540.72453101,323.51725865)(540.729531,323.47725869)(540.72953951,323.43726336)
\curveto(540.729531,323.39725877)(540.734531,323.35725881)(540.74453951,323.31726336)
\curveto(540.76453097,323.2672589)(540.76953096,323.21225895)(540.75953951,323.15226336)
\curveto(540.74953098,323.09225907)(540.75453098,323.03725913)(540.77453951,322.98726336)
\moveto(538.67453951,322.44726336)
\curveto(538.68453305,322.49725967)(538.68953304,322.5672596)(538.68953951,322.65726336)
\curveto(538.68953304,322.75725941)(538.68453305,322.83225933)(538.67453951,322.88226336)
\lineto(538.67453951,323.00226336)
\curveto(538.65453308,323.05225911)(538.64453309,323.10725906)(538.64453951,323.16726336)
\curveto(538.64453309,323.22725894)(538.63953309,323.28225888)(538.62953951,323.33226336)
\curveto(538.6295331,323.37225879)(538.62453311,323.40225876)(538.61453951,323.42226336)
\lineto(538.55453951,323.66226336)
\curveto(538.54453319,323.75225841)(538.52453321,323.83725833)(538.49453951,323.91726336)
\curveto(538.38453335,324.17725799)(538.25453348,324.39725777)(538.10453951,324.57726336)
\curveto(537.95453378,324.7672574)(537.75453398,324.91725725)(537.50453951,325.02726336)
\curveto(537.44453429,325.04725712)(537.38453435,325.0622571)(537.32453951,325.07226336)
\curveto(537.26453447,325.09225707)(537.19953453,325.11225705)(537.12953951,325.13226336)
\curveto(537.04953468,325.15225701)(536.96453477,325.15725701)(536.87453951,325.14726336)
\lineto(536.60453951,325.14726336)
\curveto(536.57453516,325.12725704)(536.53953519,325.11725705)(536.49953951,325.11726336)
\curveto(536.45953527,325.12725704)(536.42453531,325.12725704)(536.39453951,325.11726336)
\lineto(536.18453951,325.05726336)
\curveto(536.12453561,325.04725712)(536.06953566,325.02725714)(536.01953951,324.99726336)
\curveto(535.76953596,324.88725728)(535.56453617,324.72725744)(535.40453951,324.51726336)
\curveto(535.25453648,324.31725785)(535.1345366,324.08225808)(535.04453951,323.81226336)
\curveto(535.01453672,323.71225845)(534.98953674,323.60725856)(534.96953951,323.49726336)
\curveto(534.95953677,323.38725878)(534.94453679,323.27725889)(534.92453951,323.16726336)
\curveto(534.91453682,323.11725905)(534.90953682,323.0672591)(534.90953951,323.01726336)
\lineto(534.90953951,322.86726336)
\curveto(534.88953684,322.79725937)(534.87953685,322.69225947)(534.87953951,322.55226336)
\curveto(534.88953684,322.41225975)(534.90453683,322.30725986)(534.92453951,322.23726336)
\lineto(534.92453951,322.10226336)
\curveto(534.94453679,322.02226014)(534.95953677,321.94226022)(534.96953951,321.86226336)
\curveto(534.97953675,321.79226037)(534.99453674,321.71726045)(535.01453951,321.63726336)
\curveto(535.11453662,321.33726083)(535.21953651,321.09226107)(535.32953951,320.90226336)
\curveto(535.44953628,320.72226144)(535.6345361,320.55726161)(535.88453951,320.40726336)
\curveto(535.95453578,320.35726181)(536.0295357,320.31726185)(536.10953951,320.28726336)
\curveto(536.19953553,320.25726191)(536.28953544,320.23226193)(536.37953951,320.21226336)
\curveto(536.41953531,320.20226196)(536.45453528,320.19726197)(536.48453951,320.19726336)
\curveto(536.51453522,320.20726196)(536.54953518,320.20726196)(536.58953951,320.19726336)
\lineto(536.70953951,320.16726336)
\curveto(536.75953497,320.167262)(536.80453493,320.17226199)(536.84453951,320.18226336)
\lineto(536.96453951,320.18226336)
\curveto(537.04453469,320.20226196)(537.12453461,320.21726195)(537.20453951,320.22726336)
\curveto(537.28453445,320.23726193)(537.35953437,320.25726191)(537.42953951,320.28726336)
\curveto(537.68953404,320.38726178)(537.89953383,320.52226164)(538.05953951,320.69226336)
\curveto(538.21953351,320.8622613)(538.35453338,321.07226109)(538.46453951,321.32226336)
\curveto(538.50453323,321.42226074)(538.5345332,321.52226064)(538.55453951,321.62226336)
\curveto(538.57453316,321.72226044)(538.59953313,321.82726034)(538.62953951,321.93726336)
\curveto(538.63953309,321.97726019)(538.64453309,322.01226015)(538.64453951,322.04226336)
\curveto(538.64453309,322.08226008)(538.64953308,322.12226004)(538.65953951,322.16226336)
\lineto(538.65953951,322.29726336)
\curveto(538.65953307,322.34725982)(538.66453307,322.39725977)(538.67453951,322.44726336)
}
}
{
\newrgbcolor{curcolor}{0.80000001 0.80000001 0.80000001}
\pscustom[linestyle=none,fillstyle=solid,fillcolor=curcolor]
{
\newpath
\moveto(118.65951904,414.52157101)
\curveto(73.49932006,455.49594901)(58.33004643,520.09999194)(80.53437162,576.891815)
\lineto(217.42099381,523.37222524)
\closepath
}
}
{
\newrgbcolor{curcolor}{0.90196079 0.90196079 0.90196079}
\pscustom[linestyle=none,fillstyle=solid,fillcolor=curcolor]
{
\newpath
\moveto(217.42099669,670.34941335)
\curveto(298.59425626,670.34941175)(364.39818351,604.54548192)(364.39818192,523.37222235)
\curveto(364.39818033,442.19896279)(298.59425049,376.39503553)(217.42099093,376.39503713)
\curveto(180.77407546,376.39503784)(145.4490839,390.08556265)(118.37343246,414.7818278)
\lineto(217.42099381,523.37222524)
\closepath
}
}
{
\newrgbcolor{curcolor}{0.7019608 0.7019608 0.7019608}
\pscustom[linestyle=none,fillstyle=solid,fillcolor=curcolor]
{
\newpath
\moveto(80.23284328,576.11409834)
\curveto(86.82093395,593.25053892)(96.56128519,609.00098904)(108.94981728,622.55033616)
\lineto(217.42099381,523.37222524)
\closepath
}
}
{
\newrgbcolor{curcolor}{0.60000002 0.60000002 0.60000002}
\pscustom[linestyle=none,fillstyle=solid,fillcolor=curcolor]
{
\newpath
\moveto(108.68177944,622.25638516)
\curveto(117.79795225,632.28109964)(128.25658038,640.99705557)(139.76050587,648.15660902)
\lineto(217.42099381,523.37222524)
\closepath
}
}
{
\newrgbcolor{curcolor}{0.50196081 0.50196081 0.50196081}
\pscustom[linestyle=none,fillstyle=solid,fillcolor=curcolor]
{
\newpath
\moveto(139.70655636,648.12301696)
\curveto(159.14991326,660.23540137)(181.15530701,667.63534516)(203.96663592,669.73230886)
\lineto(217.42099381,523.37222524)
\closepath
}
}
{
\newrgbcolor{curcolor}{0.40000001 0.40000001 0.40000001}
\pscustom[linestyle=none,fillstyle=solid,fillcolor=curcolor]
{
\newpath
\moveto(203.94254523,669.73009229)
\curveto(208.42351273,670.14275538)(212.92106775,670.34941343)(217.42099669,670.34941335)
\lineto(217.42099381,523.37222524)
\closepath
}
}
{
\newrgbcolor{curcolor}{0.80000001 0.80000001 0.80000001}
\pscustom[linestyle=none,fillstyle=solid,fillcolor=curcolor]
{
\newpath
\moveto(704.72646361,635.350286)
\curveto(712.53948867,628.56461731)(719.61402493,620.97278015)(725.83236598,612.70108777)
\lineto(608.35003181,524.38237524)
\closepath
}
}
{
\newrgbcolor{curcolor}{0.90196079 0.90196079 0.90196079}
\pscustom[linestyle=none,fillstyle=solid,fillcolor=curcolor]
{
\newpath
\moveto(608.35003469,671.35956335)
\curveto(643.8877702,671.35956265)(678.22199624,658.48357977)(704.99622946,635.11541595)
\lineto(608.35003181,524.38237524)
\closepath
}
}
{
\newrgbcolor{curcolor}{0.7019608 0.7019608 0.7019608}
\pscustom[linestyle=none,fillstyle=solid,fillcolor=curcolor]
{
\newpath
\moveto(725.83125315,612.70256804)
\curveto(774.60914874,547.81948111)(761.55331155,455.67904949)(696.67022462,406.9011539)
\curveto(669.07295791,386.15404309)(635.05267376,375.78454728)(600.57491475,377.61098368)
\lineto(608.35003181,524.38237524)
\closepath
}
}
{
\newrgbcolor{curcolor}{0.60000002 0.60000002 0.60000002}
\pscustom[linestyle=none,fillstyle=solid,fillcolor=curcolor]
{
\newpath
\moveto(600.67182853,377.60588182)
\curveto(594.40598237,377.9336621)(588.16755517,378.66233636)(581.99483875,379.78743195)
\lineto(608.35003181,524.38237524)
\closepath
}
}
{
\newrgbcolor{curcolor}{0.50196081 0.50196081 0.50196081}
\pscustom[linestyle=none,fillstyle=solid,fillcolor=curcolor]
{
\newpath
\moveto(582.09966254,379.76836506)
\curveto(502.23155011,394.26604369)(449.23834301,470.76463207)(463.73602164,550.6327445)
\curveto(475.99085466,618.14495849)(533.3278146,668.19489458)(601.8766819,671.21694053)
\lineto(608.35003181,524.38237524)
\closepath
}
}
{
\newrgbcolor{curcolor}{0.40000001 0.40000001 0.40000001}
\pscustom[linestyle=none,fillstyle=solid,fillcolor=curcolor]
{
\newpath
\moveto(601.74948156,671.21127757)
\curveto(603.94827809,671.31012233)(606.14901755,671.35956339)(608.35003469,671.35956335)
\lineto(608.35003181,524.38237524)
\closepath
}
}
{
\newrgbcolor{curcolor}{0.80000001 0.80000001 0.80000001}
\pscustom[linestyle=none,fillstyle=solid,fillcolor=curcolor]
{
\newpath
\moveto(563.57744453,23.76609671)
\curveto(486.26209806,48.49331388)(443.63099612,131.21515604)(468.35821329,208.53050252)
\curveto(476.07417851,232.65624705)(489.89413762,254.3784572)(508.4776025,271.59015387)
\lineto(608.35003181,163.75791524)
\closepath
}
}
{
\newrgbcolor{curcolor}{0.90196079 0.90196079 0.90196079}
\pscustom[linestyle=none,fillstyle=solid,fillcolor=curcolor]
{
\newpath
\moveto(608.35003469,310.73510335)
\curveto(689.52329426,310.73510175)(755.32722151,244.93117192)(755.32721992,163.75791235)
\curveto(755.32721833,82.58465279)(689.52328849,16.78072553)(608.35002893,16.78072713)
\curveto(593.14703499,16.78072742)(578.03607959,19.1394892)(563.55633528,23.77284969)
\lineto(608.35003181,163.75791524)
\closepath
}
}
{
\newrgbcolor{curcolor}{0.50196081 0.50196081 0.50196081}
\pscustom[linestyle=none,fillstyle=solid,fillcolor=curcolor]
{
\newpath
\moveto(508.18492676,271.3183427)
\curveto(535.38765694,296.65074783)(571.17854994,310.73510408)(608.35003469,310.73510335)
\lineto(608.35003181,163.75791524)
\closepath
}
}
{
\newrgbcolor{curcolor}{0.80000001 0.80000001 0.80000001}
\pscustom[linestyle=none,fillstyle=solid,fillcolor=curcolor]
{
\newpath
\moveto(287.43904147,292.98549614)
\curveto(358.80946367,254.31559952)(385.31847734,165.11029576)(346.64858072,93.73987357)
\curveto(312.26175498,30.27442668)(236.87849306,1.17250225)(168.76582939,25.06772735)
\lineto(217.42099981,163.75791524)
\closepath
}
}
{
\newrgbcolor{curcolor}{0.90196079 0.90196079 0.90196079}
\pscustom[linestyle=none,fillstyle=solid,fillcolor=curcolor]
{
\newpath
\moveto(217.42100269,310.73510335)
\curveto(242.05075825,310.73510286)(266.28560427,304.54558911)(287.89902103,292.73520916)
\lineto(217.42099981,163.75791524)
\closepath
}
}
{
\newrgbcolor{curcolor}{0.7019608 0.7019608 0.7019608}
\pscustom[linestyle=none,fillstyle=solid,fillcolor=curcolor]
{
\newpath
\moveto(168.96141599,24.99926664)
\curveto(159.99590911,28.13034882)(151.35496264,32.12297859)(143.15975622,36.92116376)
\lineto(217.42099981,163.75791524)
\closepath
}
}
{
\newrgbcolor{curcolor}{0.60000002 0.60000002 0.60000002}
\pscustom[linestyle=none,fillstyle=solid,fillcolor=curcolor]
{
\newpath
\moveto(143.31857592,36.82831034)
\curveto(73.21729083,77.75394898)(49.56575628,167.75905404)(90.49139492,237.86033913)
\curveto(110.59831307,272.30136066)(143.83768808,297.09306507)(182.58182663,306.54631055)
\lineto(217.42099981,163.75791524)
\closepath
}
}
{
\newrgbcolor{curcolor}{0.50196081 0.50196081 0.50196081}
\pscustom[linestyle=none,fillstyle=solid,fillcolor=curcolor]
{
\newpath
\moveto(182.53069634,306.53382549)
\curveto(193.95091974,309.32459782)(205.66473148,310.73510358)(217.42100269,310.73510335)
\lineto(217.42099981,163.75791524)
\closepath
}
}
{
\newrgbcolor{curcolor}{0 0 0}
\pscustom[linestyle=none,fillstyle=solid,fillcolor=curcolor]
{
\newpath
\moveto(100.48703462,590.54351702)
\lineto(104.08703462,590.54351702)
\lineto(104.73203462,590.54351702)
\curveto(104.81202809,590.5435066)(104.88702802,590.5385066)(104.95703462,590.52851702)
\curveto(105.02702788,590.52850661)(105.08702782,590.51850662)(105.13703462,590.49851702)
\curveto(105.2070277,590.46850667)(105.26202764,590.40850673)(105.30203462,590.31851702)
\curveto(105.32202758,590.28850685)(105.33202757,590.24850689)(105.33203462,590.19851702)
\lineto(105.33203462,590.06351702)
\curveto(105.34202756,589.95350719)(105.33702757,589.84850729)(105.31703462,589.74851702)
\curveto(105.3070276,589.64850749)(105.27202763,589.57850756)(105.21203462,589.53851702)
\curveto(105.12202778,589.46850767)(104.98702792,589.43350771)(104.80703462,589.43351702)
\curveto(104.62702828,589.4435077)(104.46202844,589.44850769)(104.31203462,589.44851702)
\lineto(102.31703462,589.44851702)
\lineto(101.82203462,589.44851702)
\lineto(101.68703462,589.44851702)
\curveto(101.64703126,589.44850769)(101.6070313,589.4435077)(101.56703462,589.43351702)
\lineto(101.35703462,589.43351702)
\curveto(101.24703166,589.40350774)(101.16703174,589.36350778)(101.11703462,589.31351702)
\curveto(101.06703184,589.27350787)(101.03203187,589.21850792)(101.01203462,589.14851702)
\curveto(100.99203191,589.08850805)(100.97703193,589.01850812)(100.96703462,588.93851702)
\curveto(100.95703195,588.85850828)(100.93703197,588.76850837)(100.90703462,588.66851702)
\curveto(100.85703205,588.46850867)(100.81703209,588.26350888)(100.78703462,588.05351702)
\curveto(100.75703215,587.8435093)(100.71703219,587.6385095)(100.66703462,587.43851702)
\curveto(100.64703226,587.36850977)(100.63703227,587.29850984)(100.63703462,587.22851702)
\curveto(100.63703227,587.16850997)(100.62703228,587.10351004)(100.60703462,587.03351702)
\curveto(100.59703231,587.00351014)(100.58703232,586.96351018)(100.57703462,586.91351702)
\curveto(100.57703233,586.87351027)(100.58203232,586.83351031)(100.59203462,586.79351702)
\curveto(100.61203229,586.7435104)(100.63703227,586.69851044)(100.66703462,586.65851702)
\curveto(100.7070322,586.62851051)(100.76703214,586.62351052)(100.84703462,586.64351702)
\curveto(100.907032,586.66351048)(100.96703194,586.68851045)(101.02703462,586.71851702)
\curveto(101.08703182,586.75851038)(101.14703176,586.79351035)(101.20703462,586.82351702)
\curveto(101.26703164,586.8435103)(101.31703159,586.85851028)(101.35703462,586.86851702)
\curveto(101.54703136,586.94851019)(101.75203115,587.00351014)(101.97203462,587.03351702)
\curveto(102.2020307,587.06351008)(102.43203047,587.07351007)(102.66203462,587.06351702)
\curveto(102.90203,587.06351008)(103.13202977,587.0385101)(103.35203462,586.98851702)
\curveto(103.57202933,586.94851019)(103.77202913,586.88851025)(103.95203462,586.80851702)
\curveto(104.0020289,586.78851035)(104.04702886,586.76851037)(104.08703462,586.74851702)
\curveto(104.13702877,586.72851041)(104.18702872,586.70351044)(104.23703462,586.67351702)
\curveto(104.58702832,586.46351068)(104.86702804,586.23351091)(105.07703462,585.98351702)
\curveto(105.29702761,585.73351141)(105.49202741,585.40851173)(105.66203462,585.00851702)
\curveto(105.71202719,584.89851224)(105.74702716,584.78851235)(105.76703462,584.67851702)
\curveto(105.78702712,584.56851257)(105.81202709,584.45351269)(105.84203462,584.33351702)
\curveto(105.85202705,584.30351284)(105.85702705,584.25851288)(105.85703462,584.19851702)
\curveto(105.87702703,584.138513)(105.88702702,584.06851307)(105.88703462,583.98851702)
\curveto(105.88702702,583.91851322)(105.89702701,583.85351329)(105.91703462,583.79351702)
\lineto(105.91703462,583.62851702)
\curveto(105.92702698,583.57851356)(105.93202697,583.50851363)(105.93203462,583.41851702)
\curveto(105.93202697,583.32851381)(105.92202698,583.25851388)(105.90203462,583.20851702)
\curveto(105.88202702,583.14851399)(105.87702703,583.08851405)(105.88703462,583.02851702)
\curveto(105.89702701,582.97851416)(105.89202701,582.92851421)(105.87203462,582.87851702)
\curveto(105.83202707,582.71851442)(105.79702711,582.56851457)(105.76703462,582.42851702)
\curveto(105.73702717,582.28851485)(105.69202721,582.15351499)(105.63203462,582.02351702)
\curveto(105.47202743,581.65351549)(105.25202765,581.31851582)(104.97203462,581.01851702)
\curveto(104.69202821,580.71851642)(104.37202853,580.48851665)(104.01203462,580.32851702)
\curveto(103.84202906,580.24851689)(103.64202926,580.17351697)(103.41203462,580.10351702)
\curveto(103.3020296,580.06351708)(103.18702972,580.0385171)(103.06703462,580.02851702)
\curveto(102.94702996,580.01851712)(102.82703008,579.99851714)(102.70703462,579.96851702)
\curveto(102.65703025,579.94851719)(102.6020303,579.94851719)(102.54203462,579.96851702)
\curveto(102.48203042,579.97851716)(102.42203048,579.97351717)(102.36203462,579.95351702)
\curveto(102.26203064,579.93351721)(102.16203074,579.93351721)(102.06203462,579.95351702)
\lineto(101.92703462,579.95351702)
\curveto(101.87703103,579.97351717)(101.81703109,579.98351716)(101.74703462,579.98351702)
\curveto(101.68703122,579.97351717)(101.63203127,579.97851716)(101.58203462,579.99851702)
\curveto(101.54203136,580.00851713)(101.5070314,580.01351713)(101.47703462,580.01351702)
\curveto(101.44703146,580.01351713)(101.41203149,580.01851712)(101.37203462,580.02851702)
\lineto(101.10203462,580.08851702)
\curveto(101.01203189,580.10851703)(100.92703198,580.138517)(100.84703462,580.17851702)
\curveto(100.5070324,580.31851682)(100.21703269,580.47351667)(99.97703462,580.64351702)
\curveto(99.73703317,580.82351632)(99.51703339,581.05351609)(99.31703462,581.33351702)
\curveto(99.16703374,581.56351558)(99.05203385,581.80351534)(98.97203462,582.05351702)
\curveto(98.95203395,582.10351504)(98.94203396,582.14851499)(98.94203462,582.18851702)
\curveto(98.94203396,582.2385149)(98.93203397,582.28851485)(98.91203462,582.33851702)
\curveto(98.89203401,582.39851474)(98.87703403,582.47851466)(98.86703462,582.57851702)
\curveto(98.86703404,582.67851446)(98.88703402,582.75351439)(98.92703462,582.80351702)
\curveto(98.97703393,582.88351426)(99.05703385,582.92851421)(99.16703462,582.93851702)
\curveto(99.27703363,582.94851419)(99.39203351,582.95351419)(99.51203462,582.95351702)
\lineto(99.67703462,582.95351702)
\curveto(99.73703317,582.95351419)(99.79203311,582.9435142)(99.84203462,582.92351702)
\curveto(99.93203297,582.90351424)(100.0020329,582.86351428)(100.05203462,582.80351702)
\curveto(100.12203278,582.71351443)(100.16703274,582.60351454)(100.18703462,582.47351702)
\curveto(100.21703269,582.35351479)(100.26203264,582.24851489)(100.32203462,582.15851702)
\curveto(100.51203239,581.81851532)(100.77203213,581.54851559)(101.10203462,581.34851702)
\curveto(101.2020317,581.28851585)(101.3070316,581.2385159)(101.41703462,581.19851702)
\curveto(101.53703137,581.16851597)(101.65703125,581.13351601)(101.77703462,581.09351702)
\curveto(101.94703096,581.0435161)(102.15203075,581.02351612)(102.39203462,581.03351702)
\curveto(102.64203026,581.05351609)(102.84203006,581.08851605)(102.99203462,581.13851702)
\curveto(103.36202954,581.25851588)(103.65202925,581.41851572)(103.86203462,581.61851702)
\curveto(104.08202882,581.82851531)(104.26202864,582.10851503)(104.40203462,582.45851702)
\curveto(104.45202845,582.55851458)(104.48202842,582.66351448)(104.49203462,582.77351702)
\curveto(104.51202839,582.88351426)(104.53702837,582.99851414)(104.56703462,583.11851702)
\lineto(104.56703462,583.22351702)
\curveto(104.57702833,583.26351388)(104.58202832,583.30351384)(104.58203462,583.34351702)
\curveto(104.59202831,583.37351377)(104.59202831,583.40851373)(104.58203462,583.44851702)
\lineto(104.58203462,583.56851702)
\curveto(104.58202832,583.82851331)(104.55202835,584.07351307)(104.49203462,584.30351702)
\curveto(104.38202852,584.65351249)(104.22702868,584.94851219)(104.02703462,585.18851702)
\curveto(103.82702908,585.4385117)(103.56702934,585.63351151)(103.24703462,585.77351702)
\lineto(103.06703462,585.83351702)
\curveto(103.01702989,585.85351129)(102.95702995,585.87351127)(102.88703462,585.89351702)
\curveto(102.83703007,585.91351123)(102.77703013,585.92351122)(102.70703462,585.92351702)
\curveto(102.64703026,585.93351121)(102.58203032,585.94851119)(102.51203462,585.96851702)
\lineto(102.36203462,585.96851702)
\curveto(102.32203058,585.98851115)(102.26703064,585.99851114)(102.19703462,585.99851702)
\curveto(102.13703077,585.99851114)(102.08203082,585.98851115)(102.03203462,585.96851702)
\lineto(101.92703462,585.96851702)
\curveto(101.89703101,585.96851117)(101.86203104,585.96351118)(101.82203462,585.95351702)
\lineto(101.58203462,585.89351702)
\curveto(101.5020314,585.88351126)(101.42203148,585.86351128)(101.34203462,585.83351702)
\curveto(101.1020318,585.73351141)(100.87203203,585.59851154)(100.65203462,585.42851702)
\curveto(100.56203234,585.35851178)(100.47703243,585.28351186)(100.39703462,585.20351702)
\curveto(100.31703259,585.13351201)(100.21703269,585.07851206)(100.09703462,585.03851702)
\curveto(100.0070329,585.00851213)(99.86703304,584.99851214)(99.67703462,585.00851702)
\curveto(99.49703341,585.01851212)(99.37703353,585.0435121)(99.31703462,585.08351702)
\curveto(99.26703364,585.12351202)(99.22703368,585.18351196)(99.19703462,585.26351702)
\curveto(99.17703373,585.3435118)(99.17703373,585.42851171)(99.19703462,585.51851702)
\curveto(99.22703368,585.6385115)(99.24703366,585.75851138)(99.25703462,585.87851702)
\curveto(99.27703363,586.00851113)(99.3020336,586.13351101)(99.33203462,586.25351702)
\curveto(99.35203355,586.29351085)(99.35703355,586.32851081)(99.34703462,586.35851702)
\curveto(99.34703356,586.39851074)(99.35703355,586.4435107)(99.37703462,586.49351702)
\curveto(99.39703351,586.58351056)(99.41203349,586.67351047)(99.42203462,586.76351702)
\curveto(99.43203347,586.86351028)(99.45203345,586.95851018)(99.48203462,587.04851702)
\curveto(99.49203341,587.10851003)(99.49703341,587.16850997)(99.49703462,587.22851702)
\curveto(99.5070334,587.28850985)(99.52203338,587.34850979)(99.54203462,587.40851702)
\curveto(99.59203331,587.60850953)(99.62703328,587.81350933)(99.64703462,588.02351702)
\curveto(99.67703323,588.2435089)(99.71703319,588.45350869)(99.76703462,588.65351702)
\curveto(99.79703311,588.75350839)(99.81703309,588.85350829)(99.82703462,588.95351702)
\curveto(99.83703307,589.05350809)(99.85203305,589.15350799)(99.87203462,589.25351702)
\curveto(99.88203302,589.28350786)(99.88703302,589.32350782)(99.88703462,589.37351702)
\curveto(99.91703299,589.48350766)(99.93703297,589.58850755)(99.94703462,589.68851702)
\curveto(99.96703294,589.79850734)(99.99203291,589.90850723)(100.02203462,590.01851702)
\curveto(100.04203286,590.09850704)(100.05703285,590.16850697)(100.06703462,590.22851702)
\curveto(100.07703283,590.29850684)(100.1020328,590.35850678)(100.14203462,590.40851702)
\curveto(100.16203274,590.4385067)(100.19203271,590.45850668)(100.23203462,590.46851702)
\curveto(100.27203263,590.48850665)(100.31703259,590.50850663)(100.36703462,590.52851702)
\curveto(100.42703248,590.52850661)(100.46703244,590.53350661)(100.48703462,590.54351702)
}
}
{
\newrgbcolor{curcolor}{0 0 0}
\pscustom[linestyle=none,fillstyle=solid,fillcolor=curcolor]
{
\newpath
\moveto(108.281644,581.76851702)
\lineto(108.581644,581.76851702)
\curveto(108.69164194,581.77851536)(108.79664183,581.77851536)(108.896644,581.76851702)
\curveto(109.00664162,581.76851537)(109.10664152,581.75851538)(109.196644,581.73851702)
\curveto(109.28664134,581.72851541)(109.35664127,581.70351544)(109.406644,581.66351702)
\curveto(109.4266412,581.6435155)(109.44164119,581.61351553)(109.451644,581.57351702)
\curveto(109.47164116,581.53351561)(109.49164114,581.48851565)(109.511644,581.43851702)
\lineto(109.511644,581.36351702)
\curveto(109.52164111,581.31351583)(109.52164111,581.25851588)(109.511644,581.19851702)
\lineto(109.511644,581.04851702)
\lineto(109.511644,580.56851702)
\curveto(109.51164112,580.39851674)(109.47164116,580.27851686)(109.391644,580.20851702)
\curveto(109.32164131,580.15851698)(109.2316414,580.13351701)(109.121644,580.13351702)
\lineto(108.791644,580.13351702)
\lineto(108.341644,580.13351702)
\curveto(108.19164244,580.13351701)(108.07664255,580.16351698)(107.996644,580.22351702)
\curveto(107.95664267,580.25351689)(107.9266427,580.30351684)(107.906644,580.37351702)
\curveto(107.88664274,580.45351669)(107.87164276,580.5385166)(107.861644,580.62851702)
\lineto(107.861644,580.91351702)
\curveto(107.87164276,581.01351613)(107.87664275,581.09851604)(107.876644,581.16851702)
\lineto(107.876644,581.36351702)
\curveto(107.87664275,581.42351572)(107.88664274,581.47851566)(107.906644,581.52851702)
\curveto(107.94664268,581.6385155)(108.01664261,581.70851543)(108.116644,581.73851702)
\curveto(108.14664248,581.7385154)(108.20164243,581.74851539)(108.281644,581.76851702)
}
}
{
\newrgbcolor{curcolor}{0 0 0}
\pscustom[linestyle=none,fillstyle=solid,fillcolor=curcolor]
{
\newpath
\moveto(118.42680025,585.72851702)
\curveto(118.42679261,585.64851149)(118.43179261,585.56851157)(118.44180025,585.48851702)
\curveto(118.45179259,585.40851173)(118.44679259,585.33351181)(118.42680025,585.26351702)
\curveto(118.40679263,585.22351192)(118.40179264,585.17851196)(118.41180025,585.12851702)
\curveto(118.42179262,585.08851205)(118.42179262,585.04851209)(118.41180025,585.00851702)
\lineto(118.41180025,584.85851702)
\curveto(118.40179264,584.76851237)(118.39679264,584.67851246)(118.39680025,584.58851702)
\curveto(118.39679264,584.50851263)(118.39179265,584.42851271)(118.38180025,584.34851702)
\lineto(118.35180025,584.10851702)
\curveto(118.3417927,584.0385131)(118.33179271,583.96351318)(118.32180025,583.88351702)
\curveto(118.31179273,583.8435133)(118.30679273,583.80351334)(118.30680025,583.76351702)
\curveto(118.30679273,583.72351342)(118.30179274,583.67851346)(118.29180025,583.62851702)
\curveto(118.25179279,583.48851365)(118.22179282,583.34851379)(118.20180025,583.20851702)
\curveto(118.19179285,583.06851407)(118.16179288,582.93351421)(118.11180025,582.80351702)
\curveto(118.06179298,582.63351451)(118.00679303,582.46851467)(117.94680025,582.30851702)
\curveto(117.89679314,582.14851499)(117.8367932,581.99351515)(117.76680025,581.84351702)
\curveto(117.74679329,581.78351536)(117.71679332,581.72351542)(117.67680025,581.66351702)
\lineto(117.58680025,581.51351702)
\curveto(117.38679365,581.19351595)(117.17179387,580.92851621)(116.94180025,580.71851702)
\curveto(116.71179433,580.50851663)(116.41679462,580.32851681)(116.05680025,580.17851702)
\curveto(115.9367951,580.12851701)(115.80679523,580.09351705)(115.66680025,580.07351702)
\curveto(115.5367955,580.05351709)(115.40179564,580.02851711)(115.26180025,579.99851702)
\curveto(115.20179584,579.98851715)(115.1417959,579.98351716)(115.08180025,579.98351702)
\curveto(115.02179602,579.98351716)(114.95679608,579.97851716)(114.88680025,579.96851702)
\curveto(114.85679618,579.95851718)(114.80679623,579.95851718)(114.73680025,579.96851702)
\lineto(114.58680025,579.96851702)
\lineto(114.43680025,579.96851702)
\curveto(114.35679668,579.98851715)(114.27179677,580.00351714)(114.18180025,580.01351702)
\curveto(114.10179694,580.01351713)(114.02679701,580.02351712)(113.95680025,580.04351702)
\curveto(113.91679712,580.05351709)(113.88179716,580.05851708)(113.85180025,580.05851702)
\curveto(113.83179721,580.04851709)(113.80679723,580.05351709)(113.77680025,580.07351702)
\lineto(113.50680025,580.13351702)
\curveto(113.41679762,580.16351698)(113.33179771,580.19351695)(113.25180025,580.22351702)
\curveto(112.67179837,580.46351668)(112.2367988,580.83351631)(111.94680025,581.33351702)
\curveto(111.86679917,581.46351568)(111.80179924,581.59851554)(111.75180025,581.73851702)
\curveto(111.71179933,581.87851526)(111.66679937,582.02851511)(111.61680025,582.18851702)
\curveto(111.59679944,582.26851487)(111.59179945,582.34851479)(111.60180025,582.42851702)
\curveto(111.62179942,582.50851463)(111.65679938,582.56351458)(111.70680025,582.59351702)
\curveto(111.7367993,582.61351453)(111.79179925,582.62851451)(111.87180025,582.63851702)
\curveto(111.95179909,582.65851448)(112.036799,582.66851447)(112.12680025,582.66851702)
\curveto(112.21679882,582.67851446)(112.30179874,582.67851446)(112.38180025,582.66851702)
\curveto(112.47179857,582.65851448)(112.5417985,582.64851449)(112.59180025,582.63851702)
\curveto(112.61179843,582.62851451)(112.6367984,582.61351453)(112.66680025,582.59351702)
\curveto(112.70679833,582.57351457)(112.7367983,582.55351459)(112.75680025,582.53351702)
\curveto(112.81679822,582.45351469)(112.86179818,582.35851478)(112.89180025,582.24851702)
\curveto(112.93179811,582.138515)(112.97679806,582.0385151)(113.02680025,581.94851702)
\curveto(113.27679776,581.55851558)(113.64679739,581.28851585)(114.13680025,581.13851702)
\curveto(114.20679683,581.11851602)(114.27679676,581.10351604)(114.34680025,581.09351702)
\curveto(114.42679661,581.09351605)(114.50679653,581.08351606)(114.58680025,581.06351702)
\curveto(114.62679641,581.05351609)(114.68179636,581.04851609)(114.75180025,581.04851702)
\curveto(114.83179621,581.04851609)(114.88679615,581.05351609)(114.91680025,581.06351702)
\curveto(114.94679609,581.07351607)(114.97679606,581.07851606)(115.00680025,581.07851702)
\lineto(115.11180025,581.07851702)
\curveto(115.19179585,581.09851604)(115.26679577,581.11851602)(115.33680025,581.13851702)
\curveto(115.41679562,581.15851598)(115.49179555,581.18351596)(115.56180025,581.21351702)
\curveto(115.91179513,581.36351578)(116.18179486,581.57851556)(116.37180025,581.85851702)
\curveto(116.56179448,582.138515)(116.71679432,582.46351468)(116.83680025,582.83351702)
\curveto(116.86679417,582.91351423)(116.88679415,582.98851415)(116.89680025,583.05851702)
\curveto(116.91679412,583.12851401)(116.9367941,583.20351394)(116.95680025,583.28351702)
\curveto(116.97679406,583.37351377)(116.99179405,583.46851367)(117.00180025,583.56851702)
\curveto(117.02179402,583.67851346)(117.041794,583.78351336)(117.06180025,583.88351702)
\curveto(117.07179397,583.93351321)(117.07679396,583.98351316)(117.07680025,584.03351702)
\curveto(117.08679395,584.09351305)(117.09179395,584.14851299)(117.09180025,584.19851702)
\curveto(117.11179393,584.25851288)(117.12179392,584.33351281)(117.12180025,584.42351702)
\curveto(117.12179392,584.52351262)(117.11179393,584.60351254)(117.09180025,584.66351702)
\curveto(117.06179398,584.75351239)(117.01179403,584.79351235)(116.94180025,584.78351702)
\curveto(116.88179416,584.77351237)(116.82679421,584.7435124)(116.77680025,584.69351702)
\curveto(116.69679434,584.6435125)(116.62679441,584.58351256)(116.56680025,584.51351702)
\curveto(116.51679452,584.4435127)(116.45179459,584.38351276)(116.37180025,584.33351702)
\curveto(116.21179483,584.22351292)(116.04679499,584.12351302)(115.87680025,584.03351702)
\curveto(115.70679533,583.95351319)(115.51179553,583.88351326)(115.29180025,583.82351702)
\curveto(115.19179585,583.79351335)(115.09179595,583.77851336)(114.99180025,583.77851702)
\curveto(114.90179614,583.77851336)(114.80179624,583.76851337)(114.69180025,583.74851702)
\lineto(114.54180025,583.74851702)
\curveto(114.49179655,583.76851337)(114.4417966,583.77351337)(114.39180025,583.76351702)
\curveto(114.35179669,583.75351339)(114.31179673,583.75351339)(114.27180025,583.76351702)
\curveto(114.2417968,583.77351337)(114.19679684,583.77851336)(114.13680025,583.77851702)
\curveto(114.07679696,583.78851335)(114.01179703,583.79851334)(113.94180025,583.80851702)
\lineto(113.76180025,583.83851702)
\curveto(113.31179773,583.95851318)(112.93179811,584.12351302)(112.62180025,584.33351702)
\curveto(112.35179869,584.52351262)(112.12179892,584.75351239)(111.93180025,585.02351702)
\curveto(111.75179929,585.30351184)(111.60679943,585.61851152)(111.49680025,585.96851702)
\lineto(111.43680025,586.17851702)
\curveto(111.42679961,586.25851088)(111.41179963,586.3385108)(111.39180025,586.41851702)
\curveto(111.38179966,586.44851069)(111.37679966,586.47851066)(111.37680025,586.50851702)
\curveto(111.37679966,586.5385106)(111.37179967,586.56851057)(111.36180025,586.59851702)
\curveto(111.35179969,586.65851048)(111.34679969,586.71851042)(111.34680025,586.77851702)
\curveto(111.34679969,586.84851029)(111.3367997,586.90851023)(111.31680025,586.95851702)
\lineto(111.31680025,587.13851702)
\curveto(111.30679973,587.18850995)(111.30179974,587.25850988)(111.30180025,587.34851702)
\curveto(111.30179974,587.4385097)(111.31179973,587.50850963)(111.33180025,587.55851702)
\lineto(111.33180025,587.72351702)
\curveto(111.35179969,587.80350934)(111.36179968,587.87850926)(111.36180025,587.94851702)
\curveto(111.37179967,588.01850912)(111.38679965,588.08850905)(111.40680025,588.15851702)
\curveto(111.46679957,588.35850878)(111.52679951,588.54850859)(111.58680025,588.72851702)
\curveto(111.65679938,588.90850823)(111.74679929,589.07850806)(111.85680025,589.23851702)
\curveto(111.89679914,589.30850783)(111.9367991,589.37350777)(111.97680025,589.43351702)
\lineto(112.12680025,589.61351702)
\curveto(112.14679889,589.62350752)(112.16679887,589.6385075)(112.18680025,589.65851702)
\curveto(112.27679876,589.78850735)(112.38679865,589.89850724)(112.51680025,589.98851702)
\curveto(112.77679826,590.18850695)(113.041798,590.3435068)(113.31180025,590.45351702)
\curveto(113.39179765,590.49350665)(113.47179757,590.52350662)(113.55180025,590.54351702)
\curveto(113.6417974,590.57350657)(113.73179731,590.59850654)(113.82180025,590.61851702)
\curveto(113.92179712,590.64850649)(114.02179702,590.66850647)(114.12180025,590.67851702)
\curveto(114.22179682,590.68850645)(114.32679671,590.70350644)(114.43680025,590.72351702)
\curveto(114.46679657,590.73350641)(114.50679653,590.73350641)(114.55680025,590.72351702)
\curveto(114.61679642,590.71350643)(114.65679638,590.71850642)(114.67680025,590.73851702)
\curveto(115.39679564,590.75850638)(115.99679504,590.6435065)(116.47680025,590.39351702)
\curveto(116.95679408,590.143507)(117.33179371,589.80350734)(117.60180025,589.37351702)
\curveto(117.69179335,589.23350791)(117.77179327,589.08850805)(117.84180025,588.93851702)
\curveto(117.91179313,588.78850835)(117.98179306,588.62850851)(118.05180025,588.45851702)
\curveto(118.10179294,588.31850882)(118.1417929,588.16850897)(118.17180025,588.00851702)
\curveto(118.20179284,587.84850929)(118.2367928,587.68850945)(118.27680025,587.52851702)
\curveto(118.29679274,587.47850966)(118.30679273,587.42350972)(118.30680025,587.36351702)
\curveto(118.30679273,587.31350983)(118.31179273,587.26350988)(118.32180025,587.21351702)
\curveto(118.3417927,587.15350999)(118.35179269,587.08851005)(118.35180025,587.01851702)
\curveto(118.35179269,586.95851018)(118.36179268,586.90351024)(118.38180025,586.85351702)
\lineto(118.38180025,586.68851702)
\curveto(118.40179264,586.6385105)(118.40679263,586.58851055)(118.39680025,586.53851702)
\curveto(118.38679265,586.48851065)(118.39179265,586.4385107)(118.41180025,586.38851702)
\curveto(118.41179263,586.36851077)(118.40679263,586.3435108)(118.39680025,586.31351702)
\curveto(118.39679264,586.28351086)(118.40179264,586.25851088)(118.41180025,586.23851702)
\curveto(118.42179262,586.20851093)(118.42179262,586.17351097)(118.41180025,586.13351702)
\curveto(118.41179263,586.09351105)(118.41679262,586.05351109)(118.42680025,586.01351702)
\curveto(118.4367926,585.97351117)(118.4367926,585.92851121)(118.42680025,585.87851702)
\lineto(118.42680025,585.72851702)
\moveto(116.92680025,587.03351702)
\curveto(116.9367941,587.08351006)(116.9417941,587.14351)(116.94180025,587.21351702)
\curveto(116.9417941,587.28350986)(116.9367941,587.3435098)(116.92680025,587.39351702)
\curveto(116.91679412,587.4435097)(116.91179413,587.51850962)(116.91180025,587.61851702)
\curveto(116.89179415,587.69850944)(116.87179417,587.77350937)(116.85180025,587.84351702)
\curveto(116.8417942,587.91350923)(116.82679421,587.98350916)(116.80680025,588.05351702)
\curveto(116.66679437,588.48350866)(116.47179457,588.81850832)(116.22180025,589.05851702)
\curveto(115.98179506,589.29850784)(115.6367954,589.47850766)(115.18680025,589.59851702)
\curveto(115.09679594,589.61850752)(114.99679604,589.62850751)(114.88680025,589.62851702)
\lineto(114.55680025,589.62851702)
\curveto(114.5367965,589.60850753)(114.50179654,589.59850754)(114.45180025,589.59851702)
\curveto(114.40179664,589.60850753)(114.35679668,589.60850753)(114.31680025,589.59851702)
\curveto(114.2367968,589.57850756)(114.16179688,589.55850758)(114.09180025,589.53851702)
\lineto(113.88180025,589.47851702)
\curveto(113.59179745,589.34850779)(113.36179768,589.16850797)(113.19180025,588.93851702)
\curveto(113.02179802,588.71850842)(112.88679815,588.45850868)(112.78680025,588.15851702)
\curveto(112.75679828,588.06850907)(112.73179831,587.97350917)(112.71180025,587.87351702)
\curveto(112.70179834,587.78350936)(112.68679835,587.68850945)(112.66680025,587.58851702)
\lineto(112.66680025,587.45351702)
\curveto(112.6367984,587.3435098)(112.62679841,587.20350994)(112.63680025,587.03351702)
\curveto(112.65679838,586.87351027)(112.67679836,586.7435104)(112.69680025,586.64351702)
\curveto(112.71679832,586.58351056)(112.73179831,586.52351062)(112.74180025,586.46351702)
\curveto(112.75179829,586.41351073)(112.76679827,586.36351078)(112.78680025,586.31351702)
\curveto(112.86679817,586.11351103)(112.96179808,585.92351122)(113.07180025,585.74351702)
\curveto(113.19179785,585.56351158)(113.33179771,585.41851172)(113.49180025,585.30851702)
\curveto(113.5417975,585.25851188)(113.59679744,585.21851192)(113.65680025,585.18851702)
\curveto(113.71679732,585.15851198)(113.77679726,585.12351202)(113.83680025,585.08351702)
\curveto(113.98679705,585.00351214)(114.17179687,584.9385122)(114.39180025,584.88851702)
\curveto(114.4417966,584.86851227)(114.48179656,584.86351228)(114.51180025,584.87351702)
\curveto(114.55179649,584.88351226)(114.59679644,584.87851226)(114.64680025,584.85851702)
\curveto(114.68679635,584.84851229)(114.7417963,584.8435123)(114.81180025,584.84351702)
\curveto(114.88179616,584.8435123)(114.9417961,584.84851229)(114.99180025,584.85851702)
\curveto(115.09179595,584.87851226)(115.18679585,584.89351225)(115.27680025,584.90351702)
\curveto(115.36679567,584.92351222)(115.45679558,584.95351219)(115.54680025,584.99351702)
\curveto(116.08679495,585.21351193)(116.48179456,585.60851153)(116.73180025,586.17851702)
\curveto(116.78179426,586.27851086)(116.81679422,586.37851076)(116.83680025,586.47851702)
\curveto(116.85679418,586.58851055)(116.88179416,586.69851044)(116.91180025,586.80851702)
\curveto(116.91179413,586.90851023)(116.91679412,586.98351016)(116.92680025,587.03351702)
}
}
{
\newrgbcolor{curcolor}{0 0 0}
\pscustom[linestyle=none,fillstyle=solid,fillcolor=curcolor]
{
\newpath
\moveto(129.64140962,588.65351702)
\curveto(129.44139932,588.36350878)(129.23139953,588.07850906)(129.01140962,587.79851702)
\curveto(128.80139996,587.51850962)(128.59640017,587.23350991)(128.39640962,586.94351702)
\curveto(127.79640097,586.09351105)(127.19140157,585.25351189)(126.58140962,584.42351702)
\curveto(125.97140279,583.60351354)(125.3664034,582.76851437)(124.76640962,581.91851702)
\lineto(124.25640962,581.19851702)
\lineto(123.74640962,580.50851702)
\curveto(123.6664051,580.39851674)(123.58640518,580.28351686)(123.50640962,580.16351702)
\curveto(123.42640534,580.0435171)(123.33140543,579.94851719)(123.22140962,579.87851702)
\curveto(123.18140558,579.85851728)(123.11640565,579.8435173)(123.02640962,579.83351702)
\curveto(122.94640582,579.81351733)(122.85640591,579.80351734)(122.75640962,579.80351702)
\curveto(122.65640611,579.80351734)(122.5614062,579.80851733)(122.47140962,579.81851702)
\curveto(122.39140637,579.82851731)(122.33140643,579.84851729)(122.29140962,579.87851702)
\curveto(122.2614065,579.89851724)(122.23640653,579.93351721)(122.21640962,579.98351702)
\curveto(122.20640656,580.02351712)(122.21140655,580.06851707)(122.23140962,580.11851702)
\curveto(122.27140649,580.19851694)(122.31640645,580.27351687)(122.36640962,580.34351702)
\curveto(122.42640634,580.42351672)(122.48140628,580.50351664)(122.53140962,580.58351702)
\curveto(122.77140599,580.92351622)(123.01640575,581.25851588)(123.26640962,581.58851702)
\curveto(123.51640525,581.91851522)(123.75640501,582.25351489)(123.98640962,582.59351702)
\curveto(124.14640462,582.81351433)(124.30640446,583.02851411)(124.46640962,583.23851702)
\curveto(124.62640414,583.44851369)(124.78640398,583.66351348)(124.94640962,583.88351702)
\curveto(125.30640346,584.40351274)(125.67140309,584.91351223)(126.04140962,585.41351702)
\curveto(126.41140235,585.91351123)(126.78140198,586.42351072)(127.15140962,586.94351702)
\curveto(127.29140147,587.14351)(127.43140133,587.3385098)(127.57140962,587.52851702)
\curveto(127.72140104,587.71850942)(127.8664009,587.91350923)(128.00640962,588.11351702)
\curveto(128.21640055,588.41350873)(128.43140033,588.71350843)(128.65140962,589.01351702)
\lineto(129.31140962,589.91351702)
\lineto(129.49140962,590.18351702)
\lineto(129.70140962,590.45351702)
\lineto(129.82140962,590.63351702)
\curveto(129.87139889,590.69350645)(129.92139884,590.74850639)(129.97140962,590.79851702)
\curveto(130.04139872,590.84850629)(130.11639865,590.88350626)(130.19640962,590.90351702)
\curveto(130.21639855,590.91350623)(130.24139852,590.91350623)(130.27140962,590.90351702)
\curveto(130.31139845,590.90350624)(130.34139842,590.91350623)(130.36140962,590.93351702)
\curveto(130.48139828,590.93350621)(130.61639815,590.92850621)(130.76640962,590.91851702)
\curveto(130.91639785,590.91850622)(131.00639776,590.87350627)(131.03640962,590.78351702)
\curveto(131.05639771,590.75350639)(131.0613977,590.71850642)(131.05140962,590.67851702)
\curveto(131.04139772,590.6385065)(131.02639774,590.60850653)(131.00640962,590.58851702)
\curveto(130.9663978,590.50850663)(130.92639784,590.4385067)(130.88640962,590.37851702)
\curveto(130.84639792,590.31850682)(130.80139796,590.25850688)(130.75140962,590.19851702)
\lineto(130.18140962,589.41851702)
\curveto(130.00139876,589.16850797)(129.82139894,588.91350823)(129.64140962,588.65351702)
\moveto(122.78640962,584.75351702)
\curveto(122.73640603,584.77351237)(122.68640608,584.77851236)(122.63640962,584.76851702)
\curveto(122.58640618,584.75851238)(122.53640623,584.76351238)(122.48640962,584.78351702)
\curveto(122.37640639,584.80351234)(122.27140649,584.82351232)(122.17140962,584.84351702)
\curveto(122.08140668,584.87351227)(121.98640678,584.91351223)(121.88640962,584.96351702)
\curveto(121.55640721,585.10351204)(121.30140746,585.29851184)(121.12140962,585.54851702)
\curveto(120.94140782,585.80851133)(120.79640797,586.11851102)(120.68640962,586.47851702)
\curveto(120.65640811,586.55851058)(120.63640813,586.6385105)(120.62640962,586.71851702)
\curveto(120.61640815,586.80851033)(120.60140816,586.89351025)(120.58140962,586.97351702)
\curveto(120.57140819,587.02351012)(120.5664082,587.08851005)(120.56640962,587.16851702)
\curveto(120.55640821,587.19850994)(120.55140821,587.22850991)(120.55140962,587.25851702)
\curveto(120.55140821,587.29850984)(120.54640822,587.33350981)(120.53640962,587.36351702)
\lineto(120.53640962,587.51351702)
\curveto(120.52640824,587.56350958)(120.52140824,587.62350952)(120.52140962,587.69351702)
\curveto(120.52140824,587.77350937)(120.52640824,587.8385093)(120.53640962,587.88851702)
\lineto(120.53640962,588.05351702)
\curveto(120.55640821,588.10350904)(120.5614082,588.14850899)(120.55140962,588.18851702)
\curveto(120.55140821,588.2385089)(120.55640821,588.28350886)(120.56640962,588.32351702)
\curveto(120.57640819,588.36350878)(120.58140818,588.39850874)(120.58140962,588.42851702)
\curveto(120.58140818,588.46850867)(120.58640818,588.50850863)(120.59640962,588.54851702)
\curveto(120.62640814,588.65850848)(120.64640812,588.76850837)(120.65640962,588.87851702)
\curveto(120.67640809,588.99850814)(120.71140805,589.11350803)(120.76140962,589.22351702)
\curveto(120.90140786,589.56350758)(121.0614077,589.8385073)(121.24140962,590.04851702)
\curveto(121.43140733,590.26850687)(121.70140706,590.44850669)(122.05140962,590.58851702)
\curveto(122.13140663,590.61850652)(122.21640655,590.6385065)(122.30640962,590.64851702)
\curveto(122.39640637,590.66850647)(122.49140627,590.68850645)(122.59140962,590.70851702)
\curveto(122.62140614,590.71850642)(122.67640609,590.71850642)(122.75640962,590.70851702)
\curveto(122.83640593,590.70850643)(122.88640588,590.71850642)(122.90640962,590.73851702)
\curveto(123.4664053,590.74850639)(123.91640485,590.6385065)(124.25640962,590.40851702)
\curveto(124.60640416,590.17850696)(124.8664039,589.87350727)(125.03640962,589.49351702)
\curveto(125.07640369,589.40350774)(125.11140365,589.30850783)(125.14140962,589.20851702)
\curveto(125.17140359,589.10850803)(125.19640357,589.00850813)(125.21640962,588.90851702)
\curveto(125.23640353,588.87850826)(125.24140352,588.84850829)(125.23140962,588.81851702)
\curveto(125.23140353,588.78850835)(125.23640353,588.75850838)(125.24640962,588.72851702)
\curveto(125.27640349,588.61850852)(125.29640347,588.49350865)(125.30640962,588.35351702)
\curveto(125.31640345,588.22350892)(125.32640344,588.08850905)(125.33640962,587.94851702)
\lineto(125.33640962,587.78351702)
\curveto(125.34640342,587.72350942)(125.34640342,587.66850947)(125.33640962,587.61851702)
\curveto(125.32640344,587.56850957)(125.32140344,587.51850962)(125.32140962,587.46851702)
\lineto(125.32140962,587.33351702)
\curveto(125.31140345,587.29350985)(125.30640346,587.25350989)(125.30640962,587.21351702)
\curveto(125.31640345,587.17350997)(125.31140345,587.12851001)(125.29140962,587.07851702)
\curveto(125.27140349,586.96851017)(125.25140351,586.86351028)(125.23140962,586.76351702)
\curveto(125.22140354,586.66351048)(125.20140356,586.56351058)(125.17140962,586.46351702)
\curveto(125.04140372,586.10351104)(124.87640389,585.78851135)(124.67640962,585.51851702)
\curveto(124.47640429,585.24851189)(124.20140456,585.0435121)(123.85140962,584.90351702)
\curveto(123.77140499,584.87351227)(123.68640508,584.84851229)(123.59640962,584.82851702)
\lineto(123.32640962,584.76851702)
\curveto(123.27640549,584.75851238)(123.23140553,584.75351239)(123.19140962,584.75351702)
\curveto(123.15140561,584.76351238)(123.11140565,584.76351238)(123.07140962,584.75351702)
\curveto(122.97140579,584.73351241)(122.87640589,584.73351241)(122.78640962,584.75351702)
\moveto(121.94640962,586.14851702)
\curveto(121.98640678,586.07851106)(122.02640674,586.01351113)(122.06640962,585.95351702)
\curveto(122.10640666,585.90351124)(122.15640661,585.85351129)(122.21640962,585.80351702)
\lineto(122.36640962,585.68351702)
\curveto(122.42640634,585.65351149)(122.49140627,585.62851151)(122.56140962,585.60851702)
\curveto(122.60140616,585.58851155)(122.63640613,585.57851156)(122.66640962,585.57851702)
\curveto(122.70640606,585.58851155)(122.74640602,585.58351156)(122.78640962,585.56351702)
\curveto(122.81640595,585.56351158)(122.85640591,585.55851158)(122.90640962,585.54851702)
\curveto(122.95640581,585.54851159)(122.99640577,585.55351159)(123.02640962,585.56351702)
\lineto(123.25140962,585.60851702)
\curveto(123.50140526,585.68851145)(123.68640508,585.81351133)(123.80640962,585.98351702)
\curveto(123.88640488,586.08351106)(123.95640481,586.21351093)(124.01640962,586.37351702)
\curveto(124.09640467,586.55351059)(124.15640461,586.77851036)(124.19640962,587.04851702)
\curveto(124.23640453,587.32850981)(124.25140451,587.60850953)(124.24140962,587.88851702)
\curveto(124.23140453,588.17850896)(124.20140456,588.45350869)(124.15140962,588.71351702)
\curveto(124.10140466,588.97350817)(124.02640474,589.18350796)(123.92640962,589.34351702)
\curveto(123.80640496,589.5435076)(123.65640511,589.69350745)(123.47640962,589.79351702)
\curveto(123.39640537,589.8435073)(123.30640546,589.87350727)(123.20640962,589.88351702)
\curveto(123.10640566,589.90350724)(123.00140576,589.91350723)(122.89140962,589.91351702)
\curveto(122.87140589,589.90350724)(122.84640592,589.89850724)(122.81640962,589.89851702)
\curveto(122.79640597,589.90850723)(122.77640599,589.90850723)(122.75640962,589.89851702)
\curveto(122.70640606,589.88850725)(122.6614061,589.87850726)(122.62140962,589.86851702)
\curveto(122.58140618,589.86850727)(122.54140622,589.85850728)(122.50140962,589.83851702)
\curveto(122.32140644,589.75850738)(122.17140659,589.6385075)(122.05140962,589.47851702)
\curveto(121.94140682,589.31850782)(121.85140691,589.138508)(121.78140962,588.93851702)
\curveto(121.72140704,588.74850839)(121.67640709,588.52350862)(121.64640962,588.26351702)
\curveto(121.62640714,588.00350914)(121.62140714,587.7385094)(121.63140962,587.46851702)
\curveto(121.64140712,587.20850993)(121.67140709,586.95851018)(121.72140962,586.71851702)
\curveto(121.78140698,586.48851065)(121.85640691,586.29851084)(121.94640962,586.14851702)
\moveto(132.74640962,583.16351702)
\curveto(132.75639601,583.11351403)(132.761396,583.02351412)(132.76140962,582.89351702)
\curveto(132.761396,582.76351438)(132.75139601,582.67351447)(132.73140962,582.62351702)
\curveto(132.71139605,582.57351457)(132.70639606,582.51851462)(132.71640962,582.45851702)
\curveto(132.72639604,582.40851473)(132.72639604,582.35851478)(132.71640962,582.30851702)
\curveto(132.67639609,582.16851497)(132.64639612,582.03351511)(132.62640962,581.90351702)
\curveto(132.61639615,581.77351537)(132.58639618,581.65351549)(132.53640962,581.54351702)
\curveto(132.39639637,581.19351595)(132.23139653,580.89851624)(132.04140962,580.65851702)
\curveto(131.85139691,580.42851671)(131.58139718,580.2435169)(131.23140962,580.10351702)
\curveto(131.15139761,580.07351707)(131.0663977,580.05351709)(130.97640962,580.04351702)
\curveto(130.88639788,580.02351712)(130.80139796,580.00351714)(130.72140962,579.98351702)
\curveto(130.67139809,579.97351717)(130.62139814,579.96851717)(130.57140962,579.96851702)
\curveto(130.52139824,579.96851717)(130.47139829,579.96351718)(130.42140962,579.95351702)
\curveto(130.39139837,579.9435172)(130.34139842,579.9435172)(130.27140962,579.95351702)
\curveto(130.20139856,579.95351719)(130.15139861,579.95851718)(130.12140962,579.96851702)
\curveto(130.0613987,579.98851715)(130.00139876,579.99851714)(129.94140962,579.99851702)
\curveto(129.89139887,579.98851715)(129.84139892,579.99351715)(129.79140962,580.01351702)
\curveto(129.70139906,580.03351711)(129.61139915,580.05851708)(129.52140962,580.08851702)
\curveto(129.44139932,580.10851703)(129.3613994,580.138517)(129.28140962,580.17851702)
\curveto(128.9613998,580.31851682)(128.71140005,580.51351663)(128.53140962,580.76351702)
\curveto(128.35140041,581.02351612)(128.20140056,581.32851581)(128.08140962,581.67851702)
\curveto(128.0614007,581.75851538)(128.04640072,581.8435153)(128.03640962,581.93351702)
\curveto(128.02640074,582.02351512)(128.01140075,582.10851503)(127.99140962,582.18851702)
\curveto(127.98140078,582.21851492)(127.97640079,582.24851489)(127.97640962,582.27851702)
\lineto(127.97640962,582.38351702)
\curveto(127.95640081,582.46351468)(127.94640082,582.5435146)(127.94640962,582.62351702)
\lineto(127.94640962,582.75851702)
\curveto(127.92640084,582.85851428)(127.92640084,582.95851418)(127.94640962,583.05851702)
\lineto(127.94640962,583.23851702)
\curveto(127.95640081,583.28851385)(127.9614008,583.33351381)(127.96140962,583.37351702)
\curveto(127.9614008,583.42351372)(127.9664008,583.46851367)(127.97640962,583.50851702)
\curveto(127.98640078,583.54851359)(127.99140077,583.58351356)(127.99140962,583.61351702)
\curveto(127.99140077,583.65351349)(127.99640077,583.69351345)(128.00640962,583.73351702)
\lineto(128.06640962,584.06351702)
\curveto(128.08640068,584.18351296)(128.11640065,584.29351285)(128.15640962,584.39351702)
\curveto(128.29640047,584.72351242)(128.45640031,584.99851214)(128.63640962,585.21851702)
\curveto(128.82639994,585.44851169)(129.08639968,585.63351151)(129.41640962,585.77351702)
\curveto(129.49639927,585.81351133)(129.58139918,585.8385113)(129.67140962,585.84851702)
\lineto(129.97140962,585.90851702)
\lineto(130.10640962,585.90851702)
\curveto(130.15639861,585.91851122)(130.20639856,585.92351122)(130.25640962,585.92351702)
\curveto(130.82639794,585.9435112)(131.28639748,585.8385113)(131.63640962,585.60851702)
\curveto(131.99639677,585.38851175)(132.2613965,585.08851205)(132.43140962,584.70851702)
\curveto(132.48139628,584.60851253)(132.52139624,584.50851263)(132.55140962,584.40851702)
\curveto(132.58139618,584.30851283)(132.61139615,584.20351294)(132.64140962,584.09351702)
\curveto(132.65139611,584.05351309)(132.65639611,584.01851312)(132.65640962,583.98851702)
\curveto(132.65639611,583.96851317)(132.6613961,583.9385132)(132.67140962,583.89851702)
\curveto(132.69139607,583.82851331)(132.70139606,583.75351339)(132.70140962,583.67351702)
\curveto(132.70139606,583.59351355)(132.71139605,583.51351363)(132.73140962,583.43351702)
\curveto(132.73139603,583.38351376)(132.73139603,583.3385138)(132.73140962,583.29851702)
\curveto(132.73139603,583.25851388)(132.73639603,583.21351393)(132.74640962,583.16351702)
\moveto(131.63640962,582.72851702)
\curveto(131.64639712,582.77851436)(131.65139711,582.85351429)(131.65140962,582.95351702)
\curveto(131.6613971,583.05351409)(131.65639711,583.12851401)(131.63640962,583.17851702)
\curveto(131.61639715,583.2385139)(131.61139715,583.29351385)(131.62140962,583.34351702)
\curveto(131.64139712,583.40351374)(131.64139712,583.46351368)(131.62140962,583.52351702)
\curveto(131.61139715,583.55351359)(131.60639716,583.58851355)(131.60640962,583.62851702)
\curveto(131.60639716,583.66851347)(131.60139716,583.70851343)(131.59140962,583.74851702)
\curveto(131.57139719,583.82851331)(131.55139721,583.90351324)(131.53140962,583.97351702)
\curveto(131.52139724,584.05351309)(131.50639726,584.13351301)(131.48640962,584.21351702)
\curveto(131.45639731,584.27351287)(131.43139733,584.33351281)(131.41140962,584.39351702)
\curveto(131.39139737,584.45351269)(131.3613974,584.51351263)(131.32140962,584.57351702)
\curveto(131.22139754,584.7435124)(131.09139767,584.87851226)(130.93140962,584.97851702)
\curveto(130.85139791,585.02851211)(130.75639801,585.06351208)(130.64640962,585.08351702)
\curveto(130.53639823,585.10351204)(130.41139835,585.11351203)(130.27140962,585.11351702)
\curveto(130.25139851,585.10351204)(130.22639854,585.09851204)(130.19640962,585.09851702)
\curveto(130.1663986,585.10851203)(130.13639863,585.10851203)(130.10640962,585.09851702)
\lineto(129.95640962,585.03851702)
\curveto(129.90639886,585.02851211)(129.8613989,585.01351213)(129.82140962,584.99351702)
\curveto(129.63139913,584.88351226)(129.48639928,584.7385124)(129.38640962,584.55851702)
\curveto(129.29639947,584.37851276)(129.21639955,584.17351297)(129.14640962,583.94351702)
\curveto(129.10639966,583.81351333)(129.08639968,583.67851346)(129.08640962,583.53851702)
\curveto(129.08639968,583.40851373)(129.07639969,583.26351388)(129.05640962,583.10351702)
\curveto(129.04639972,583.05351409)(129.03639973,582.99351415)(129.02640962,582.92351702)
\curveto(129.02639974,582.85351429)(129.03639973,582.79351435)(129.05640962,582.74351702)
\lineto(129.05640962,582.57851702)
\lineto(129.05640962,582.39851702)
\curveto(129.0663997,582.34851479)(129.07639969,582.29351485)(129.08640962,582.23351702)
\curveto(129.09639967,582.18351496)(129.10139966,582.12851501)(129.10140962,582.06851702)
\curveto(129.11139965,582.00851513)(129.12639964,581.95351519)(129.14640962,581.90351702)
\curveto(129.19639957,581.71351543)(129.25639951,581.5385156)(129.32640962,581.37851702)
\curveto(129.39639937,581.21851592)(129.50139926,581.08851605)(129.64140962,580.98851702)
\curveto(129.77139899,580.88851625)(129.91139885,580.81851632)(130.06140962,580.77851702)
\curveto(130.09139867,580.76851637)(130.11639865,580.76351638)(130.13640962,580.76351702)
\curveto(130.1663986,580.77351637)(130.19639857,580.77351637)(130.22640962,580.76351702)
\curveto(130.24639852,580.76351638)(130.27639849,580.75851638)(130.31640962,580.74851702)
\curveto(130.35639841,580.74851639)(130.39139837,580.75351639)(130.42140962,580.76351702)
\curveto(130.4613983,580.77351637)(130.50139826,580.77851636)(130.54140962,580.77851702)
\curveto(130.58139818,580.77851636)(130.62139814,580.78851635)(130.66140962,580.80851702)
\curveto(130.90139786,580.88851625)(131.09639767,581.02351612)(131.24640962,581.21351702)
\curveto(131.3663974,581.39351575)(131.45639731,581.59851554)(131.51640962,581.82851702)
\curveto(131.53639723,581.89851524)(131.55139721,581.96851517)(131.56140962,582.03851702)
\curveto(131.57139719,582.11851502)(131.58639718,582.19851494)(131.60640962,582.27851702)
\curveto(131.60639716,582.3385148)(131.61139715,582.38351476)(131.62140962,582.41351702)
\curveto(131.62139714,582.43351471)(131.62139714,582.45851468)(131.62140962,582.48851702)
\curveto(131.62139714,582.52851461)(131.62639714,582.55851458)(131.63640962,582.57851702)
\lineto(131.63640962,582.72851702)
}
}
{
\newrgbcolor{curcolor}{0 0 0}
\pscustom[linestyle=none,fillstyle=solid,fillcolor=curcolor]
{
\newpath
\moveto(92.93551027,503.86543963)
\curveto(93.03550542,503.86542901)(93.13050532,503.85542902)(93.22051027,503.83543963)
\curveto(93.31050514,503.82542905)(93.37550508,503.79542908)(93.41551027,503.74543963)
\curveto(93.47550498,503.66542921)(93.50550495,503.56042932)(93.50551027,503.43043963)
\lineto(93.50551027,503.04043963)
\lineto(93.50551027,501.54043963)
\lineto(93.50551027,495.15043963)
\lineto(93.50551027,493.98043963)
\lineto(93.50551027,493.66543963)
\curveto(93.51550494,493.56543931)(93.50050495,493.48543939)(93.46051027,493.42543963)
\curveto(93.41050504,493.34543953)(93.33550512,493.29543958)(93.23551027,493.27543963)
\curveto(93.14550531,493.26543961)(93.03550542,493.26043962)(92.90551027,493.26043963)
\lineto(92.68051027,493.26043963)
\curveto(92.60050585,493.2804396)(92.53050592,493.29543958)(92.47051027,493.30543963)
\curveto(92.41050604,493.32543955)(92.36050609,493.36543951)(92.32051027,493.42543963)
\curveto(92.28050617,493.48543939)(92.26050619,493.56043932)(92.26051027,493.65043963)
\lineto(92.26051027,493.95043963)
\lineto(92.26051027,495.04543963)
\lineto(92.26051027,500.38543963)
\curveto(92.24050621,500.4754324)(92.22550623,500.55043233)(92.21551027,500.61043963)
\curveto(92.21550624,500.6804322)(92.18550627,500.74043214)(92.12551027,500.79043963)
\curveto(92.0555064,500.84043204)(91.96550649,500.86543201)(91.85551027,500.86543963)
\curveto(91.7555067,500.875432)(91.64550681,500.880432)(91.52551027,500.88043963)
\lineto(90.38551027,500.88043963)
\lineto(89.89051027,500.88043963)
\curveto(89.73050872,500.89043199)(89.62050883,500.95043193)(89.56051027,501.06043963)
\curveto(89.54050891,501.09043179)(89.53050892,501.12043176)(89.53051027,501.15043963)
\curveto(89.53050892,501.19043169)(89.52550893,501.23543164)(89.51551027,501.28543963)
\curveto(89.49550896,501.40543147)(89.50050895,501.51543136)(89.53051027,501.61543963)
\curveto(89.57050888,501.71543116)(89.62550883,501.78543109)(89.69551027,501.82543963)
\curveto(89.77550868,501.875431)(89.89550856,501.90043098)(90.05551027,501.90043963)
\curveto(90.21550824,501.90043098)(90.3505081,501.91543096)(90.46051027,501.94543963)
\curveto(90.51050794,501.95543092)(90.56550789,501.96043092)(90.62551027,501.96043963)
\curveto(90.68550777,501.97043091)(90.74550771,501.98543089)(90.80551027,502.00543963)
\curveto(90.9555075,502.05543082)(91.10050735,502.10543077)(91.24051027,502.15543963)
\curveto(91.38050707,502.21543066)(91.51550694,502.28543059)(91.64551027,502.36543963)
\curveto(91.78550667,502.45543042)(91.90550655,502.56043032)(92.00551027,502.68043963)
\curveto(92.10550635,502.80043008)(92.20050625,502.93042995)(92.29051027,503.07043963)
\curveto(92.3505061,503.17042971)(92.39550606,503.2804296)(92.42551027,503.40043963)
\curveto(92.46550599,503.52042936)(92.51550594,503.62542925)(92.57551027,503.71543963)
\curveto(92.62550583,503.7754291)(92.69550576,503.81542906)(92.78551027,503.83543963)
\curveto(92.80550565,503.84542903)(92.83050562,503.85042903)(92.86051027,503.85043963)
\curveto(92.89050556,503.85042903)(92.91550554,503.85542902)(92.93551027,503.86543963)
}
}
{
\newrgbcolor{curcolor}{0 0 0}
\pscustom[linestyle=none,fillstyle=solid,fillcolor=curcolor]
{
\newpath
\moveto(104.16511965,498.85543963)
\curveto(104.16511201,498.7754341)(104.17011201,498.69543418)(104.18011965,498.61543963)
\curveto(104.19011199,498.53543434)(104.18511199,498.46043442)(104.16511965,498.39043963)
\curveto(104.14511203,498.35043453)(104.14011204,498.30543457)(104.15011965,498.25543963)
\curveto(104.16011202,498.21543466)(104.16011202,498.1754347)(104.15011965,498.13543963)
\lineto(104.15011965,497.98543963)
\curveto(104.14011204,497.89543498)(104.13511204,497.80543507)(104.13511965,497.71543963)
\curveto(104.13511204,497.63543524)(104.13011205,497.55543532)(104.12011965,497.47543963)
\lineto(104.09011965,497.23543963)
\curveto(104.0801121,497.16543571)(104.07011211,497.09043579)(104.06011965,497.01043963)
\curveto(104.05011213,496.97043591)(104.04511213,496.93043595)(104.04511965,496.89043963)
\curveto(104.04511213,496.85043603)(104.04011214,496.80543607)(104.03011965,496.75543963)
\curveto(103.99011219,496.61543626)(103.96011222,496.4754364)(103.94011965,496.33543963)
\curveto(103.93011225,496.19543668)(103.90011228,496.06043682)(103.85011965,495.93043963)
\curveto(103.80011238,495.76043712)(103.74511243,495.59543728)(103.68511965,495.43543963)
\curveto(103.63511254,495.2754376)(103.5751126,495.12043776)(103.50511965,494.97043963)
\curveto(103.48511269,494.91043797)(103.45511272,494.85043803)(103.41511965,494.79043963)
\lineto(103.32511965,494.64043963)
\curveto(103.12511305,494.32043856)(102.91011327,494.05543882)(102.68011965,493.84543963)
\curveto(102.45011373,493.63543924)(102.15511402,493.45543942)(101.79511965,493.30543963)
\curveto(101.6751145,493.25543962)(101.54511463,493.22043966)(101.40511965,493.20043963)
\curveto(101.2751149,493.1804397)(101.14011504,493.15543972)(101.00011965,493.12543963)
\curveto(100.94011524,493.11543976)(100.8801153,493.11043977)(100.82011965,493.11043963)
\curveto(100.76011542,493.11043977)(100.69511548,493.10543977)(100.62511965,493.09543963)
\curveto(100.59511558,493.08543979)(100.54511563,493.08543979)(100.47511965,493.09543963)
\lineto(100.32511965,493.09543963)
\lineto(100.17511965,493.09543963)
\curveto(100.09511608,493.11543976)(100.01011617,493.13043975)(99.92011965,493.14043963)
\curveto(99.84011634,493.14043974)(99.76511641,493.15043973)(99.69511965,493.17043963)
\curveto(99.65511652,493.1804397)(99.62011656,493.18543969)(99.59011965,493.18543963)
\curveto(99.57011661,493.1754397)(99.54511663,493.1804397)(99.51511965,493.20043963)
\lineto(99.24511965,493.26043963)
\curveto(99.15511702,493.29043959)(99.07011711,493.32043956)(98.99011965,493.35043963)
\curveto(98.41011777,493.59043929)(97.9751182,493.96043892)(97.68511965,494.46043963)
\curveto(97.60511857,494.59043829)(97.54011864,494.72543815)(97.49011965,494.86543963)
\curveto(97.45011873,495.00543787)(97.40511877,495.15543772)(97.35511965,495.31543963)
\curveto(97.33511884,495.39543748)(97.33011885,495.4754374)(97.34011965,495.55543963)
\curveto(97.36011882,495.63543724)(97.39511878,495.69043719)(97.44511965,495.72043963)
\curveto(97.4751187,495.74043714)(97.53011865,495.75543712)(97.61011965,495.76543963)
\curveto(97.69011849,495.78543709)(97.7751184,495.79543708)(97.86511965,495.79543963)
\curveto(97.95511822,495.80543707)(98.04011814,495.80543707)(98.12011965,495.79543963)
\curveto(98.21011797,495.78543709)(98.2801179,495.7754371)(98.33011965,495.76543963)
\curveto(98.35011783,495.75543712)(98.3751178,495.74043714)(98.40511965,495.72043963)
\curveto(98.44511773,495.70043718)(98.4751177,495.6804372)(98.49511965,495.66043963)
\curveto(98.55511762,495.5804373)(98.60011758,495.48543739)(98.63011965,495.37543963)
\curveto(98.67011751,495.26543761)(98.71511746,495.16543771)(98.76511965,495.07543963)
\curveto(99.01511716,494.68543819)(99.38511679,494.41543846)(99.87511965,494.26543963)
\curveto(99.94511623,494.24543863)(100.01511616,494.23043865)(100.08511965,494.22043963)
\curveto(100.16511601,494.22043866)(100.24511593,494.21043867)(100.32511965,494.19043963)
\curveto(100.36511581,494.1804387)(100.42011576,494.1754387)(100.49011965,494.17543963)
\curveto(100.57011561,494.1754387)(100.62511555,494.1804387)(100.65511965,494.19043963)
\curveto(100.68511549,494.20043868)(100.71511546,494.20543867)(100.74511965,494.20543963)
\lineto(100.85011965,494.20543963)
\curveto(100.93011525,494.22543865)(101.00511517,494.24543863)(101.07511965,494.26543963)
\curveto(101.15511502,494.28543859)(101.23011495,494.31043857)(101.30011965,494.34043963)
\curveto(101.65011453,494.49043839)(101.92011426,494.70543817)(102.11011965,494.98543963)
\curveto(102.30011388,495.26543761)(102.45511372,495.59043729)(102.57511965,495.96043963)
\curveto(102.60511357,496.04043684)(102.62511355,496.11543676)(102.63511965,496.18543963)
\curveto(102.65511352,496.25543662)(102.6751135,496.33043655)(102.69511965,496.41043963)
\curveto(102.71511346,496.50043638)(102.73011345,496.59543628)(102.74011965,496.69543963)
\curveto(102.76011342,496.80543607)(102.7801134,496.91043597)(102.80011965,497.01043963)
\curveto(102.81011337,497.06043582)(102.81511336,497.11043577)(102.81511965,497.16043963)
\curveto(102.82511335,497.22043566)(102.83011335,497.2754356)(102.83011965,497.32543963)
\curveto(102.85011333,497.38543549)(102.86011332,497.46043542)(102.86011965,497.55043963)
\curveto(102.86011332,497.65043523)(102.85011333,497.73043515)(102.83011965,497.79043963)
\curveto(102.80011338,497.880435)(102.75011343,497.92043496)(102.68011965,497.91043963)
\curveto(102.62011356,497.90043498)(102.56511361,497.87043501)(102.51511965,497.82043963)
\curveto(102.43511374,497.77043511)(102.36511381,497.71043517)(102.30511965,497.64043963)
\curveto(102.25511392,497.57043531)(102.19011399,497.51043537)(102.11011965,497.46043963)
\curveto(101.95011423,497.35043553)(101.78511439,497.25043563)(101.61511965,497.16043963)
\curveto(101.44511473,497.0804358)(101.25011493,497.01043587)(101.03011965,496.95043963)
\curveto(100.93011525,496.92043596)(100.83011535,496.90543597)(100.73011965,496.90543963)
\curveto(100.64011554,496.90543597)(100.54011564,496.89543598)(100.43011965,496.87543963)
\lineto(100.28011965,496.87543963)
\curveto(100.23011595,496.89543598)(100.180116,496.90043598)(100.13011965,496.89043963)
\curveto(100.09011609,496.880436)(100.05011613,496.880436)(100.01011965,496.89043963)
\curveto(99.9801162,496.90043598)(99.93511624,496.90543597)(99.87511965,496.90543963)
\curveto(99.81511636,496.91543596)(99.75011643,496.92543595)(99.68011965,496.93543963)
\lineto(99.50011965,496.96543963)
\curveto(99.05011713,497.08543579)(98.67011751,497.25043563)(98.36011965,497.46043963)
\curveto(98.09011809,497.65043523)(97.86011832,497.880435)(97.67011965,498.15043963)
\curveto(97.49011869,498.43043445)(97.34511883,498.74543413)(97.23511965,499.09543963)
\lineto(97.17511965,499.30543963)
\curveto(97.16511901,499.38543349)(97.15011903,499.46543341)(97.13011965,499.54543963)
\curveto(97.12011906,499.5754333)(97.11511906,499.60543327)(97.11511965,499.63543963)
\curveto(97.11511906,499.66543321)(97.11011907,499.69543318)(97.10011965,499.72543963)
\curveto(97.09011909,499.78543309)(97.08511909,499.84543303)(97.08511965,499.90543963)
\curveto(97.08511909,499.9754329)(97.0751191,500.03543284)(97.05511965,500.08543963)
\lineto(97.05511965,500.26543963)
\curveto(97.04511913,500.31543256)(97.04011914,500.38543249)(97.04011965,500.47543963)
\curveto(97.04011914,500.56543231)(97.05011913,500.63543224)(97.07011965,500.68543963)
\lineto(97.07011965,500.85043963)
\curveto(97.09011909,500.93043195)(97.10011908,501.00543187)(97.10011965,501.07543963)
\curveto(97.11011907,501.14543173)(97.12511905,501.21543166)(97.14511965,501.28543963)
\curveto(97.20511897,501.48543139)(97.26511891,501.6754312)(97.32511965,501.85543963)
\curveto(97.39511878,502.03543084)(97.48511869,502.20543067)(97.59511965,502.36543963)
\curveto(97.63511854,502.43543044)(97.6751185,502.50043038)(97.71511965,502.56043963)
\lineto(97.86511965,502.74043963)
\curveto(97.88511829,502.75043013)(97.90511827,502.76543011)(97.92511965,502.78543963)
\curveto(98.01511816,502.91542996)(98.12511805,503.02542985)(98.25511965,503.11543963)
\curveto(98.51511766,503.31542956)(98.7801174,503.47042941)(99.05011965,503.58043963)
\curveto(99.13011705,503.62042926)(99.21011697,503.65042923)(99.29011965,503.67043963)
\curveto(99.3801168,503.70042918)(99.47011671,503.72542915)(99.56011965,503.74543963)
\curveto(99.66011652,503.7754291)(99.76011642,503.79542908)(99.86011965,503.80543963)
\curveto(99.96011622,503.81542906)(100.06511611,503.83042905)(100.17511965,503.85043963)
\curveto(100.20511597,503.86042902)(100.24511593,503.86042902)(100.29511965,503.85043963)
\curveto(100.35511582,503.84042904)(100.39511578,503.84542903)(100.41511965,503.86543963)
\curveto(101.13511504,503.88542899)(101.73511444,503.77042911)(102.21511965,503.52043963)
\curveto(102.69511348,503.27042961)(103.07011311,502.93042995)(103.34011965,502.50043963)
\curveto(103.43011275,502.36043052)(103.51011267,502.21543066)(103.58011965,502.06543963)
\curveto(103.65011253,501.91543096)(103.72011246,501.75543112)(103.79011965,501.58543963)
\curveto(103.84011234,501.44543143)(103.8801123,501.29543158)(103.91011965,501.13543963)
\curveto(103.94011224,500.9754319)(103.9751122,500.81543206)(104.01511965,500.65543963)
\curveto(104.03511214,500.60543227)(104.04511213,500.55043233)(104.04511965,500.49043963)
\curveto(104.04511213,500.44043244)(104.05011213,500.39043249)(104.06011965,500.34043963)
\curveto(104.0801121,500.2804326)(104.09011209,500.21543266)(104.09011965,500.14543963)
\curveto(104.09011209,500.08543279)(104.10011208,500.03043285)(104.12011965,499.98043963)
\lineto(104.12011965,499.81543963)
\curveto(104.14011204,499.76543311)(104.14511203,499.71543316)(104.13511965,499.66543963)
\curveto(104.12511205,499.61543326)(104.13011205,499.56543331)(104.15011965,499.51543963)
\curveto(104.15011203,499.49543338)(104.14511203,499.47043341)(104.13511965,499.44043963)
\curveto(104.13511204,499.41043347)(104.14011204,499.38543349)(104.15011965,499.36543963)
\curveto(104.16011202,499.33543354)(104.16011202,499.30043358)(104.15011965,499.26043963)
\curveto(104.15011203,499.22043366)(104.15511202,499.1804337)(104.16511965,499.14043963)
\curveto(104.175112,499.10043378)(104.175112,499.05543382)(104.16511965,499.00543963)
\lineto(104.16511965,498.85543963)
\moveto(102.66511965,500.16043963)
\curveto(102.6751135,500.21043267)(102.6801135,500.27043261)(102.68011965,500.34043963)
\curveto(102.6801135,500.41043247)(102.6751135,500.47043241)(102.66511965,500.52043963)
\curveto(102.65511352,500.57043231)(102.65011353,500.64543223)(102.65011965,500.74543963)
\curveto(102.63011355,500.82543205)(102.61011357,500.90043198)(102.59011965,500.97043963)
\curveto(102.5801136,501.04043184)(102.56511361,501.11043177)(102.54511965,501.18043963)
\curveto(102.40511377,501.61043127)(102.21011397,501.94543093)(101.96011965,502.18543963)
\curveto(101.72011446,502.42543045)(101.3751148,502.60543027)(100.92511965,502.72543963)
\curveto(100.83511534,502.74543013)(100.73511544,502.75543012)(100.62511965,502.75543963)
\lineto(100.29511965,502.75543963)
\curveto(100.2751159,502.73543014)(100.24011594,502.72543015)(100.19011965,502.72543963)
\curveto(100.14011604,502.73543014)(100.09511608,502.73543014)(100.05511965,502.72543963)
\curveto(99.9751162,502.70543017)(99.90011628,502.68543019)(99.83011965,502.66543963)
\lineto(99.62011965,502.60543963)
\curveto(99.33011685,502.4754304)(99.10011708,502.29543058)(98.93011965,502.06543963)
\curveto(98.76011742,501.84543103)(98.62511755,501.58543129)(98.52511965,501.28543963)
\curveto(98.49511768,501.19543168)(98.47011771,501.10043178)(98.45011965,501.00043963)
\curveto(98.44011774,500.91043197)(98.42511775,500.81543206)(98.40511965,500.71543963)
\lineto(98.40511965,500.58043963)
\curveto(98.3751178,500.47043241)(98.36511781,500.33043255)(98.37511965,500.16043963)
\curveto(98.39511778,500.00043288)(98.41511776,499.87043301)(98.43511965,499.77043963)
\curveto(98.45511772,499.71043317)(98.47011771,499.65043323)(98.48011965,499.59043963)
\curveto(98.49011769,499.54043334)(98.50511767,499.49043339)(98.52511965,499.44043963)
\curveto(98.60511757,499.24043364)(98.70011748,499.05043383)(98.81011965,498.87043963)
\curveto(98.93011725,498.69043419)(99.07011711,498.54543433)(99.23011965,498.43543963)
\curveto(99.2801169,498.38543449)(99.33511684,498.34543453)(99.39511965,498.31543963)
\curveto(99.45511672,498.28543459)(99.51511666,498.25043463)(99.57511965,498.21043963)
\curveto(99.72511645,498.13043475)(99.91011627,498.06543481)(100.13011965,498.01543963)
\curveto(100.180116,497.99543488)(100.22011596,497.99043489)(100.25011965,498.00043963)
\curveto(100.29011589,498.01043487)(100.33511584,498.00543487)(100.38511965,497.98543963)
\curveto(100.42511575,497.9754349)(100.4801157,497.97043491)(100.55011965,497.97043963)
\curveto(100.62011556,497.97043491)(100.6801155,497.9754349)(100.73011965,497.98543963)
\curveto(100.83011535,498.00543487)(100.92511525,498.02043486)(101.01511965,498.03043963)
\curveto(101.10511507,498.05043483)(101.19511498,498.0804348)(101.28511965,498.12043963)
\curveto(101.82511435,498.34043454)(102.22011396,498.73543414)(102.47011965,499.30543963)
\curveto(102.52011366,499.40543347)(102.55511362,499.50543337)(102.57511965,499.60543963)
\curveto(102.59511358,499.71543316)(102.62011356,499.82543305)(102.65011965,499.93543963)
\curveto(102.65011353,500.03543284)(102.65511352,500.11043277)(102.66511965,500.16043963)
}
}
{
\newrgbcolor{curcolor}{0 0 0}
\pscustom[linestyle=none,fillstyle=solid,fillcolor=curcolor]
{
\newpath
\moveto(106.52972902,494.89543963)
\lineto(106.82972902,494.89543963)
\curveto(106.93972696,494.90543797)(107.04472686,494.90543797)(107.14472902,494.89543963)
\curveto(107.25472665,494.89543798)(107.35472655,494.88543799)(107.44472902,494.86543963)
\curveto(107.53472637,494.85543802)(107.6047263,494.83043805)(107.65472902,494.79043963)
\curveto(107.67472623,494.77043811)(107.68972621,494.74043814)(107.69972902,494.70043963)
\curveto(107.71972618,494.66043822)(107.73972616,494.61543826)(107.75972902,494.56543963)
\lineto(107.75972902,494.49043963)
\curveto(107.76972613,494.44043844)(107.76972613,494.38543849)(107.75972902,494.32543963)
\lineto(107.75972902,494.17543963)
\lineto(107.75972902,493.69543963)
\curveto(107.75972614,493.52543935)(107.71972618,493.40543947)(107.63972902,493.33543963)
\curveto(107.56972633,493.28543959)(107.47972642,493.26043962)(107.36972902,493.26043963)
\lineto(107.03972902,493.26043963)
\lineto(106.58972902,493.26043963)
\curveto(106.43972746,493.26043962)(106.32472758,493.29043959)(106.24472902,493.35043963)
\curveto(106.2047277,493.3804395)(106.17472773,493.43043945)(106.15472902,493.50043963)
\curveto(106.13472777,493.5804393)(106.11972778,493.66543921)(106.10972902,493.75543963)
\lineto(106.10972902,494.04043963)
\curveto(106.11972778,494.14043874)(106.12472778,494.22543865)(106.12472902,494.29543963)
\lineto(106.12472902,494.49043963)
\curveto(106.12472778,494.55043833)(106.13472777,494.60543827)(106.15472902,494.65543963)
\curveto(106.19472771,494.76543811)(106.26472764,494.83543804)(106.36472902,494.86543963)
\curveto(106.39472751,494.86543801)(106.44972745,494.875438)(106.52972902,494.89543963)
}
}
{
\newrgbcolor{curcolor}{0 0 0}
\pscustom[linestyle=none,fillstyle=solid,fillcolor=curcolor]
{
\newpath
\moveto(113.79488527,503.86543963)
\curveto(113.89488042,503.86542901)(113.98988032,503.85542902)(114.07988527,503.83543963)
\curveto(114.16988014,503.82542905)(114.23488008,503.79542908)(114.27488527,503.74543963)
\curveto(114.33487998,503.66542921)(114.36487995,503.56042932)(114.36488527,503.43043963)
\lineto(114.36488527,503.04043963)
\lineto(114.36488527,501.54043963)
\lineto(114.36488527,495.15043963)
\lineto(114.36488527,493.98043963)
\lineto(114.36488527,493.66543963)
\curveto(114.37487994,493.56543931)(114.35987995,493.48543939)(114.31988527,493.42543963)
\curveto(114.26988004,493.34543953)(114.19488012,493.29543958)(114.09488527,493.27543963)
\curveto(114.00488031,493.26543961)(113.89488042,493.26043962)(113.76488527,493.26043963)
\lineto(113.53988527,493.26043963)
\curveto(113.45988085,493.2804396)(113.38988092,493.29543958)(113.32988527,493.30543963)
\curveto(113.26988104,493.32543955)(113.21988109,493.36543951)(113.17988527,493.42543963)
\curveto(113.13988117,493.48543939)(113.11988119,493.56043932)(113.11988527,493.65043963)
\lineto(113.11988527,493.95043963)
\lineto(113.11988527,495.04543963)
\lineto(113.11988527,500.38543963)
\curveto(113.09988121,500.4754324)(113.08488123,500.55043233)(113.07488527,500.61043963)
\curveto(113.07488124,500.6804322)(113.04488127,500.74043214)(112.98488527,500.79043963)
\curveto(112.9148814,500.84043204)(112.82488149,500.86543201)(112.71488527,500.86543963)
\curveto(112.6148817,500.875432)(112.50488181,500.880432)(112.38488527,500.88043963)
\lineto(111.24488527,500.88043963)
\lineto(110.74988527,500.88043963)
\curveto(110.58988372,500.89043199)(110.47988383,500.95043193)(110.41988527,501.06043963)
\curveto(110.39988391,501.09043179)(110.38988392,501.12043176)(110.38988527,501.15043963)
\curveto(110.38988392,501.19043169)(110.38488393,501.23543164)(110.37488527,501.28543963)
\curveto(110.35488396,501.40543147)(110.35988395,501.51543136)(110.38988527,501.61543963)
\curveto(110.42988388,501.71543116)(110.48488383,501.78543109)(110.55488527,501.82543963)
\curveto(110.63488368,501.875431)(110.75488356,501.90043098)(110.91488527,501.90043963)
\curveto(111.07488324,501.90043098)(111.2098831,501.91543096)(111.31988527,501.94543963)
\curveto(111.36988294,501.95543092)(111.42488289,501.96043092)(111.48488527,501.96043963)
\curveto(111.54488277,501.97043091)(111.60488271,501.98543089)(111.66488527,502.00543963)
\curveto(111.8148825,502.05543082)(111.95988235,502.10543077)(112.09988527,502.15543963)
\curveto(112.23988207,502.21543066)(112.37488194,502.28543059)(112.50488527,502.36543963)
\curveto(112.64488167,502.45543042)(112.76488155,502.56043032)(112.86488527,502.68043963)
\curveto(112.96488135,502.80043008)(113.05988125,502.93042995)(113.14988527,503.07043963)
\curveto(113.2098811,503.17042971)(113.25488106,503.2804296)(113.28488527,503.40043963)
\curveto(113.32488099,503.52042936)(113.37488094,503.62542925)(113.43488527,503.71543963)
\curveto(113.48488083,503.7754291)(113.55488076,503.81542906)(113.64488527,503.83543963)
\curveto(113.66488065,503.84542903)(113.68988062,503.85042903)(113.71988527,503.85043963)
\curveto(113.74988056,503.85042903)(113.77488054,503.85542902)(113.79488527,503.86543963)
}
}
{
\newrgbcolor{curcolor}{0 0 0}
\pscustom[linestyle=none,fillstyle=solid,fillcolor=curcolor]
{
\newpath
\moveto(127.88949465,501.78043963)
\curveto(127.68948435,501.49043139)(127.47948456,501.20543167)(127.25949465,500.92543963)
\curveto(127.04948499,500.64543223)(126.84448519,500.36043252)(126.64449465,500.07043963)
\curveto(126.04448599,499.22043366)(125.4394866,498.3804345)(124.82949465,497.55043963)
\curveto(124.21948782,496.73043615)(123.61448842,495.89543698)(123.01449465,495.04543963)
\lineto(122.50449465,494.32543963)
\lineto(121.99449465,493.63543963)
\curveto(121.91449012,493.52543935)(121.8344902,493.41043947)(121.75449465,493.29043963)
\curveto(121.67449036,493.17043971)(121.57949046,493.0754398)(121.46949465,493.00543963)
\curveto(121.42949061,492.98543989)(121.36449067,492.97043991)(121.27449465,492.96043963)
\curveto(121.19449084,492.94043994)(121.10449093,492.93043995)(121.00449465,492.93043963)
\curveto(120.90449113,492.93043995)(120.80949123,492.93543994)(120.71949465,492.94543963)
\curveto(120.6394914,492.95543992)(120.57949146,492.9754399)(120.53949465,493.00543963)
\curveto(120.50949153,493.02543985)(120.48449155,493.06043982)(120.46449465,493.11043963)
\curveto(120.45449158,493.15043973)(120.45949158,493.19543968)(120.47949465,493.24543963)
\curveto(120.51949152,493.32543955)(120.56449147,493.40043948)(120.61449465,493.47043963)
\curveto(120.67449136,493.55043933)(120.72949131,493.63043925)(120.77949465,493.71043963)
\curveto(121.01949102,494.05043883)(121.26449077,494.38543849)(121.51449465,494.71543963)
\curveto(121.76449027,495.04543783)(122.00449003,495.3804375)(122.23449465,495.72043963)
\curveto(122.39448964,495.94043694)(122.55448948,496.15543672)(122.71449465,496.36543963)
\curveto(122.87448916,496.5754363)(123.034489,496.79043609)(123.19449465,497.01043963)
\curveto(123.55448848,497.53043535)(123.91948812,498.04043484)(124.28949465,498.54043963)
\curveto(124.65948738,499.04043384)(125.02948701,499.55043333)(125.39949465,500.07043963)
\curveto(125.5394865,500.27043261)(125.67948636,500.46543241)(125.81949465,500.65543963)
\curveto(125.96948607,500.84543203)(126.11448592,501.04043184)(126.25449465,501.24043963)
\curveto(126.46448557,501.54043134)(126.67948536,501.84043104)(126.89949465,502.14043963)
\lineto(127.55949465,503.04043963)
\lineto(127.73949465,503.31043963)
\lineto(127.94949465,503.58043963)
\lineto(128.06949465,503.76043963)
\curveto(128.11948392,503.82042906)(128.16948387,503.875429)(128.21949465,503.92543963)
\curveto(128.28948375,503.9754289)(128.36448367,504.01042887)(128.44449465,504.03043963)
\curveto(128.46448357,504.04042884)(128.48948355,504.04042884)(128.51949465,504.03043963)
\curveto(128.55948348,504.03042885)(128.58948345,504.04042884)(128.60949465,504.06043963)
\curveto(128.72948331,504.06042882)(128.86448317,504.05542882)(129.01449465,504.04543963)
\curveto(129.16448287,504.04542883)(129.25448278,504.00042888)(129.28449465,503.91043963)
\curveto(129.30448273,503.880429)(129.30948273,503.84542903)(129.29949465,503.80543963)
\curveto(129.28948275,503.76542911)(129.27448276,503.73542914)(129.25449465,503.71543963)
\curveto(129.21448282,503.63542924)(129.17448286,503.56542931)(129.13449465,503.50543963)
\curveto(129.09448294,503.44542943)(129.04948299,503.38542949)(128.99949465,503.32543963)
\lineto(128.42949465,502.54543963)
\curveto(128.24948379,502.29543058)(128.06948397,502.04043084)(127.88949465,501.78043963)
\moveto(121.03449465,497.88043963)
\curveto(120.98449105,497.90043498)(120.9344911,497.90543497)(120.88449465,497.89543963)
\curveto(120.8344912,497.88543499)(120.78449125,497.89043499)(120.73449465,497.91043963)
\curveto(120.62449141,497.93043495)(120.51949152,497.95043493)(120.41949465,497.97043963)
\curveto(120.32949171,498.00043488)(120.2344918,498.04043484)(120.13449465,498.09043963)
\curveto(119.80449223,498.23043465)(119.54949249,498.42543445)(119.36949465,498.67543963)
\curveto(119.18949285,498.93543394)(119.04449299,499.24543363)(118.93449465,499.60543963)
\curveto(118.90449313,499.68543319)(118.88449315,499.76543311)(118.87449465,499.84543963)
\curveto(118.86449317,499.93543294)(118.84949319,500.02043286)(118.82949465,500.10043963)
\curveto(118.81949322,500.15043273)(118.81449322,500.21543266)(118.81449465,500.29543963)
\curveto(118.80449323,500.32543255)(118.79949324,500.35543252)(118.79949465,500.38543963)
\curveto(118.79949324,500.42543245)(118.79449324,500.46043242)(118.78449465,500.49043963)
\lineto(118.78449465,500.64043963)
\curveto(118.77449326,500.69043219)(118.76949327,500.75043213)(118.76949465,500.82043963)
\curveto(118.76949327,500.90043198)(118.77449326,500.96543191)(118.78449465,501.01543963)
\lineto(118.78449465,501.18043963)
\curveto(118.80449323,501.23043165)(118.80949323,501.2754316)(118.79949465,501.31543963)
\curveto(118.79949324,501.36543151)(118.80449323,501.41043147)(118.81449465,501.45043963)
\curveto(118.82449321,501.49043139)(118.82949321,501.52543135)(118.82949465,501.55543963)
\curveto(118.82949321,501.59543128)(118.8344932,501.63543124)(118.84449465,501.67543963)
\curveto(118.87449316,501.78543109)(118.89449314,501.89543098)(118.90449465,502.00543963)
\curveto(118.92449311,502.12543075)(118.95949308,502.24043064)(119.00949465,502.35043963)
\curveto(119.14949289,502.69043019)(119.30949273,502.96542991)(119.48949465,503.17543963)
\curveto(119.67949236,503.39542948)(119.94949209,503.5754293)(120.29949465,503.71543963)
\curveto(120.37949166,503.74542913)(120.46449157,503.76542911)(120.55449465,503.77543963)
\curveto(120.64449139,503.79542908)(120.7394913,503.81542906)(120.83949465,503.83543963)
\curveto(120.86949117,503.84542903)(120.92449111,503.84542903)(121.00449465,503.83543963)
\curveto(121.08449095,503.83542904)(121.1344909,503.84542903)(121.15449465,503.86543963)
\curveto(121.71449032,503.875429)(122.16448987,503.76542911)(122.50449465,503.53543963)
\curveto(122.85448918,503.30542957)(123.11448892,503.00042988)(123.28449465,502.62043963)
\curveto(123.32448871,502.53043035)(123.35948868,502.43543044)(123.38949465,502.33543963)
\curveto(123.41948862,502.23543064)(123.44448859,502.13543074)(123.46449465,502.03543963)
\curveto(123.48448855,502.00543087)(123.48948855,501.9754309)(123.47949465,501.94543963)
\curveto(123.47948856,501.91543096)(123.48448855,501.88543099)(123.49449465,501.85543963)
\curveto(123.52448851,501.74543113)(123.54448849,501.62043126)(123.55449465,501.48043963)
\curveto(123.56448847,501.35043153)(123.57448846,501.21543166)(123.58449465,501.07543963)
\lineto(123.58449465,500.91043963)
\curveto(123.59448844,500.85043203)(123.59448844,500.79543208)(123.58449465,500.74543963)
\curveto(123.57448846,500.69543218)(123.56948847,500.64543223)(123.56949465,500.59543963)
\lineto(123.56949465,500.46043963)
\curveto(123.55948848,500.42043246)(123.55448848,500.3804325)(123.55449465,500.34043963)
\curveto(123.56448847,500.30043258)(123.55948848,500.25543262)(123.53949465,500.20543963)
\curveto(123.51948852,500.09543278)(123.49948854,499.99043289)(123.47949465,499.89043963)
\curveto(123.46948857,499.79043309)(123.44948859,499.69043319)(123.41949465,499.59043963)
\curveto(123.28948875,499.23043365)(123.12448891,498.91543396)(122.92449465,498.64543963)
\curveto(122.72448931,498.3754345)(122.44948959,498.17043471)(122.09949465,498.03043963)
\curveto(122.01949002,498.00043488)(121.9344901,497.9754349)(121.84449465,497.95543963)
\lineto(121.57449465,497.89543963)
\curveto(121.52449051,497.88543499)(121.47949056,497.880435)(121.43949465,497.88043963)
\curveto(121.39949064,497.89043499)(121.35949068,497.89043499)(121.31949465,497.88043963)
\curveto(121.21949082,497.86043502)(121.12449091,497.86043502)(121.03449465,497.88043963)
\moveto(120.19449465,499.27543963)
\curveto(120.2344918,499.20543367)(120.27449176,499.14043374)(120.31449465,499.08043963)
\curveto(120.35449168,499.03043385)(120.40449163,498.9804339)(120.46449465,498.93043963)
\lineto(120.61449465,498.81043963)
\curveto(120.67449136,498.7804341)(120.7394913,498.75543412)(120.80949465,498.73543963)
\curveto(120.84949119,498.71543416)(120.88449115,498.70543417)(120.91449465,498.70543963)
\curveto(120.95449108,498.71543416)(120.99449104,498.71043417)(121.03449465,498.69043963)
\curveto(121.06449097,498.69043419)(121.10449093,498.68543419)(121.15449465,498.67543963)
\curveto(121.20449083,498.6754342)(121.24449079,498.6804342)(121.27449465,498.69043963)
\lineto(121.49949465,498.73543963)
\curveto(121.74949029,498.81543406)(121.9344901,498.94043394)(122.05449465,499.11043963)
\curveto(122.1344899,499.21043367)(122.20448983,499.34043354)(122.26449465,499.50043963)
\curveto(122.34448969,499.6804332)(122.40448963,499.90543297)(122.44449465,500.17543963)
\curveto(122.48448955,500.45543242)(122.49948954,500.73543214)(122.48949465,501.01543963)
\curveto(122.47948956,501.30543157)(122.44948959,501.5804313)(122.39949465,501.84043963)
\curveto(122.34948969,502.10043078)(122.27448976,502.31043057)(122.17449465,502.47043963)
\curveto(122.05448998,502.67043021)(121.90449013,502.82043006)(121.72449465,502.92043963)
\curveto(121.64449039,502.97042991)(121.55449048,503.00042988)(121.45449465,503.01043963)
\curveto(121.35449068,503.03042985)(121.24949079,503.04042984)(121.13949465,503.04043963)
\curveto(121.11949092,503.03042985)(121.09449094,503.02542985)(121.06449465,503.02543963)
\curveto(121.04449099,503.03542984)(121.02449101,503.03542984)(121.00449465,503.02543963)
\curveto(120.95449108,503.01542986)(120.90949113,503.00542987)(120.86949465,502.99543963)
\curveto(120.82949121,502.99542988)(120.78949125,502.98542989)(120.74949465,502.96543963)
\curveto(120.56949147,502.88542999)(120.41949162,502.76543011)(120.29949465,502.60543963)
\curveto(120.18949185,502.44543043)(120.09949194,502.26543061)(120.02949465,502.06543963)
\curveto(119.96949207,501.875431)(119.92449211,501.65043123)(119.89449465,501.39043963)
\curveto(119.87449216,501.13043175)(119.86949217,500.86543201)(119.87949465,500.59543963)
\curveto(119.88949215,500.33543254)(119.91949212,500.08543279)(119.96949465,499.84543963)
\curveto(120.02949201,499.61543326)(120.10449193,499.42543345)(120.19449465,499.27543963)
\moveto(130.99449465,496.29043963)
\curveto(131.00448103,496.24043664)(131.00948103,496.15043673)(131.00949465,496.02043963)
\curveto(131.00948103,495.89043699)(130.99948104,495.80043708)(130.97949465,495.75043963)
\curveto(130.95948108,495.70043718)(130.95448108,495.64543723)(130.96449465,495.58543963)
\curveto(130.97448106,495.53543734)(130.97448106,495.48543739)(130.96449465,495.43543963)
\curveto(130.92448111,495.29543758)(130.89448114,495.16043772)(130.87449465,495.03043963)
\curveto(130.86448117,494.90043798)(130.8344812,494.7804381)(130.78449465,494.67043963)
\curveto(130.64448139,494.32043856)(130.47948156,494.02543885)(130.28949465,493.78543963)
\curveto(130.09948194,493.55543932)(129.82948221,493.37043951)(129.47949465,493.23043963)
\curveto(129.39948264,493.20043968)(129.31448272,493.1804397)(129.22449465,493.17043963)
\curveto(129.1344829,493.15043973)(129.04948299,493.13043975)(128.96949465,493.11043963)
\curveto(128.91948312,493.10043978)(128.86948317,493.09543978)(128.81949465,493.09543963)
\curveto(128.76948327,493.09543978)(128.71948332,493.09043979)(128.66949465,493.08043963)
\curveto(128.6394834,493.07043981)(128.58948345,493.07043981)(128.51949465,493.08043963)
\curveto(128.44948359,493.0804398)(128.39948364,493.08543979)(128.36949465,493.09543963)
\curveto(128.30948373,493.11543976)(128.24948379,493.12543975)(128.18949465,493.12543963)
\curveto(128.1394839,493.11543976)(128.08948395,493.12043976)(128.03949465,493.14043963)
\curveto(127.94948409,493.16043972)(127.85948418,493.18543969)(127.76949465,493.21543963)
\curveto(127.68948435,493.23543964)(127.60948443,493.26543961)(127.52949465,493.30543963)
\curveto(127.20948483,493.44543943)(126.95948508,493.64043924)(126.77949465,493.89043963)
\curveto(126.59948544,494.15043873)(126.44948559,494.45543842)(126.32949465,494.80543963)
\curveto(126.30948573,494.88543799)(126.29448574,494.97043791)(126.28449465,495.06043963)
\curveto(126.27448576,495.15043773)(126.25948578,495.23543764)(126.23949465,495.31543963)
\curveto(126.22948581,495.34543753)(126.22448581,495.3754375)(126.22449465,495.40543963)
\lineto(126.22449465,495.51043963)
\curveto(126.20448583,495.59043729)(126.19448584,495.67043721)(126.19449465,495.75043963)
\lineto(126.19449465,495.88543963)
\curveto(126.17448586,495.98543689)(126.17448586,496.08543679)(126.19449465,496.18543963)
\lineto(126.19449465,496.36543963)
\curveto(126.20448583,496.41543646)(126.20948583,496.46043642)(126.20949465,496.50043963)
\curveto(126.20948583,496.55043633)(126.21448582,496.59543628)(126.22449465,496.63543963)
\curveto(126.2344858,496.6754362)(126.2394858,496.71043617)(126.23949465,496.74043963)
\curveto(126.2394858,496.7804361)(126.24448579,496.82043606)(126.25449465,496.86043963)
\lineto(126.31449465,497.19043963)
\curveto(126.3344857,497.31043557)(126.36448567,497.42043546)(126.40449465,497.52043963)
\curveto(126.54448549,497.85043503)(126.70448533,498.12543475)(126.88449465,498.34543963)
\curveto(127.07448496,498.5754343)(127.3344847,498.76043412)(127.66449465,498.90043963)
\curveto(127.74448429,498.94043394)(127.82948421,498.96543391)(127.91949465,498.97543963)
\lineto(128.21949465,499.03543963)
\lineto(128.35449465,499.03543963)
\curveto(128.40448363,499.04543383)(128.45448358,499.05043383)(128.50449465,499.05043963)
\curveto(129.07448296,499.07043381)(129.5344825,498.96543391)(129.88449465,498.73543963)
\curveto(130.24448179,498.51543436)(130.50948153,498.21543466)(130.67949465,497.83543963)
\curveto(130.72948131,497.73543514)(130.76948127,497.63543524)(130.79949465,497.53543963)
\curveto(130.82948121,497.43543544)(130.85948118,497.33043555)(130.88949465,497.22043963)
\curveto(130.89948114,497.1804357)(130.90448113,497.14543573)(130.90449465,497.11543963)
\curveto(130.90448113,497.09543578)(130.90948113,497.06543581)(130.91949465,497.02543963)
\curveto(130.9394811,496.95543592)(130.94948109,496.880436)(130.94949465,496.80043963)
\curveto(130.94948109,496.72043616)(130.95948108,496.64043624)(130.97949465,496.56043963)
\curveto(130.97948106,496.51043637)(130.97948106,496.46543641)(130.97949465,496.42543963)
\curveto(130.97948106,496.38543649)(130.98448105,496.34043654)(130.99449465,496.29043963)
\moveto(129.88449465,495.85543963)
\curveto(129.89448214,495.90543697)(129.89948214,495.9804369)(129.89949465,496.08043963)
\curveto(129.90948213,496.1804367)(129.90448213,496.25543662)(129.88449465,496.30543963)
\curveto(129.86448217,496.36543651)(129.85948218,496.42043646)(129.86949465,496.47043963)
\curveto(129.88948215,496.53043635)(129.88948215,496.59043629)(129.86949465,496.65043963)
\curveto(129.85948218,496.6804362)(129.85448218,496.71543616)(129.85449465,496.75543963)
\curveto(129.85448218,496.79543608)(129.84948219,496.83543604)(129.83949465,496.87543963)
\curveto(129.81948222,496.95543592)(129.79948224,497.03043585)(129.77949465,497.10043963)
\curveto(129.76948227,497.1804357)(129.75448228,497.26043562)(129.73449465,497.34043963)
\curveto(129.70448233,497.40043548)(129.67948236,497.46043542)(129.65949465,497.52043963)
\curveto(129.6394824,497.5804353)(129.60948243,497.64043524)(129.56949465,497.70043963)
\curveto(129.46948257,497.87043501)(129.3394827,498.00543487)(129.17949465,498.10543963)
\curveto(129.09948294,498.15543472)(129.00448303,498.19043469)(128.89449465,498.21043963)
\curveto(128.78448325,498.23043465)(128.65948338,498.24043464)(128.51949465,498.24043963)
\curveto(128.49948354,498.23043465)(128.47448356,498.22543465)(128.44449465,498.22543963)
\curveto(128.41448362,498.23543464)(128.38448365,498.23543464)(128.35449465,498.22543963)
\lineto(128.20449465,498.16543963)
\curveto(128.15448388,498.15543472)(128.10948393,498.14043474)(128.06949465,498.12043963)
\curveto(127.87948416,498.01043487)(127.7344843,497.86543501)(127.63449465,497.68543963)
\curveto(127.54448449,497.50543537)(127.46448457,497.30043558)(127.39449465,497.07043963)
\curveto(127.35448468,496.94043594)(127.3344847,496.80543607)(127.33449465,496.66543963)
\curveto(127.3344847,496.53543634)(127.32448471,496.39043649)(127.30449465,496.23043963)
\curveto(127.29448474,496.1804367)(127.28448475,496.12043676)(127.27449465,496.05043963)
\curveto(127.27448476,495.9804369)(127.28448475,495.92043696)(127.30449465,495.87043963)
\lineto(127.30449465,495.70543963)
\lineto(127.30449465,495.52543963)
\curveto(127.31448472,495.4754374)(127.32448471,495.42043746)(127.33449465,495.36043963)
\curveto(127.34448469,495.31043757)(127.34948469,495.25543762)(127.34949465,495.19543963)
\curveto(127.35948468,495.13543774)(127.37448466,495.0804378)(127.39449465,495.03043963)
\curveto(127.44448459,494.84043804)(127.50448453,494.66543821)(127.57449465,494.50543963)
\curveto(127.64448439,494.34543853)(127.74948429,494.21543866)(127.88949465,494.11543963)
\curveto(128.01948402,494.01543886)(128.15948388,493.94543893)(128.30949465,493.90543963)
\curveto(128.3394837,493.89543898)(128.36448367,493.89043899)(128.38449465,493.89043963)
\curveto(128.41448362,493.90043898)(128.44448359,493.90043898)(128.47449465,493.89043963)
\curveto(128.49448354,493.89043899)(128.52448351,493.88543899)(128.56449465,493.87543963)
\curveto(128.60448343,493.875439)(128.6394834,493.880439)(128.66949465,493.89043963)
\curveto(128.70948333,493.90043898)(128.74948329,493.90543897)(128.78949465,493.90543963)
\curveto(128.82948321,493.90543897)(128.86948317,493.91543896)(128.90949465,493.93543963)
\curveto(129.14948289,494.01543886)(129.34448269,494.15043873)(129.49449465,494.34043963)
\curveto(129.61448242,494.52043836)(129.70448233,494.72543815)(129.76449465,494.95543963)
\curveto(129.78448225,495.02543785)(129.79948224,495.09543778)(129.80949465,495.16543963)
\curveto(129.81948222,495.24543763)(129.8344822,495.32543755)(129.85449465,495.40543963)
\curveto(129.85448218,495.46543741)(129.85948218,495.51043737)(129.86949465,495.54043963)
\curveto(129.86948217,495.56043732)(129.86948217,495.58543729)(129.86949465,495.61543963)
\curveto(129.86948217,495.65543722)(129.87448216,495.68543719)(129.88449465,495.70543963)
\lineto(129.88449465,495.85543963)
}
}
{
\newrgbcolor{curcolor}{0 0 0}
\pscustom[linestyle=none,fillstyle=solid,fillcolor=curcolor]
{
\newpath
\moveto(309.01768404,481.41796404)
\curveto(309.04767631,481.29795983)(309.07267629,481.15795997)(309.09268404,480.99796404)
\curveto(309.11267625,480.83796029)(309.12267624,480.67296045)(309.12268404,480.50296404)
\curveto(309.12267624,480.33296079)(309.11267625,480.16796096)(309.09268404,480.00796404)
\curveto(309.07267629,479.84796128)(309.04767631,479.70796142)(309.01768404,479.58796404)
\curveto(308.97767638,479.44796168)(308.94267642,479.3229618)(308.91268404,479.21296404)
\curveto(308.88267648,479.10296202)(308.84267652,478.99296213)(308.79268404,478.88296404)
\curveto(308.52267684,478.24296288)(308.10767725,477.75796337)(307.54768404,477.42796404)
\curveto(307.46767789,477.36796376)(307.38267798,477.31796381)(307.29268404,477.27796404)
\curveto(307.20267816,477.24796388)(307.10267826,477.21296391)(306.99268404,477.17296404)
\curveto(306.88267848,477.122964)(306.7626786,477.08796404)(306.63268404,477.06796404)
\curveto(306.51267885,477.03796409)(306.38267898,477.00796412)(306.24268404,476.97796404)
\curveto(306.18267918,476.95796417)(306.12267924,476.95296417)(306.06268404,476.96296404)
\curveto(306.01267935,476.97296415)(305.95267941,476.96796416)(305.88268404,476.94796404)
\curveto(305.8626795,476.93796419)(305.83767952,476.93796419)(305.80768404,476.94796404)
\curveto(305.77767958,476.94796418)(305.75267961,476.94296418)(305.73268404,476.93296404)
\lineto(305.58268404,476.93296404)
\curveto(305.51267985,476.9229642)(305.4626799,476.9229642)(305.43268404,476.93296404)
\curveto(305.39267997,476.94296418)(305.34768001,476.94796418)(305.29768404,476.94796404)
\curveto(305.2576801,476.93796419)(305.21768014,476.93796419)(305.17768404,476.94796404)
\curveto(305.08768027,476.96796416)(304.99768036,476.98296414)(304.90768404,476.99296404)
\curveto(304.81768054,476.99296413)(304.72768063,477.00296412)(304.63768404,477.02296404)
\curveto(304.54768081,477.05296407)(304.4576809,477.07796405)(304.36768404,477.09796404)
\curveto(304.27768108,477.11796401)(304.19268117,477.14796398)(304.11268404,477.18796404)
\curveto(303.87268149,477.29796383)(303.64768171,477.4279637)(303.43768404,477.57796404)
\curveto(303.22768213,477.73796339)(303.04768231,477.91796321)(302.89768404,478.11796404)
\curveto(302.77768258,478.28796284)(302.67268269,478.46296266)(302.58268404,478.64296404)
\curveto(302.49268287,478.8229623)(302.40268296,479.01296211)(302.31268404,479.21296404)
\curveto(302.27268309,479.31296181)(302.23768312,479.41296171)(302.20768404,479.51296404)
\curveto(302.18768317,479.6229615)(302.1626832,479.73296139)(302.13268404,479.84296404)
\curveto(302.09268327,479.98296114)(302.06768329,480.122961)(302.05768404,480.26296404)
\curveto(302.04768331,480.40296072)(302.02768333,480.54296058)(301.99768404,480.68296404)
\curveto(301.98768337,480.79296033)(301.97768338,480.89296023)(301.96768404,480.98296404)
\curveto(301.96768339,481.08296004)(301.9576834,481.18295994)(301.93768404,481.28296404)
\lineto(301.93768404,481.37296404)
\curveto(301.94768341,481.40295972)(301.94768341,481.4279597)(301.93768404,481.44796404)
\lineto(301.93768404,481.65796404)
\curveto(301.91768344,481.71795941)(301.90768345,481.78295934)(301.90768404,481.85296404)
\curveto(301.91768344,481.93295919)(301.92268344,482.00795912)(301.92268404,482.07796404)
\lineto(301.92268404,482.22796404)
\curveto(301.92268344,482.27795885)(301.92768343,482.3279588)(301.93768404,482.37796404)
\lineto(301.93768404,482.75296404)
\curveto(301.94768341,482.78295834)(301.94768341,482.81795831)(301.93768404,482.85796404)
\curveto(301.93768342,482.89795823)(301.94268342,482.93795819)(301.95268404,482.97796404)
\curveto(301.97268339,483.08795804)(301.98768337,483.19795793)(301.99768404,483.30796404)
\curveto(302.00768335,483.4279577)(302.01768334,483.54295758)(302.02768404,483.65296404)
\curveto(302.06768329,483.80295732)(302.09268327,483.94795718)(302.10268404,484.08796404)
\curveto(302.12268324,484.23795689)(302.15268321,484.38295674)(302.19268404,484.52296404)
\curveto(302.28268308,484.8229563)(302.37768298,485.10795602)(302.47768404,485.37796404)
\curveto(302.57768278,485.64795548)(302.70268266,485.89795523)(302.85268404,486.12796404)
\curveto(303.05268231,486.44795468)(303.29768206,486.7279544)(303.58768404,486.96796404)
\curveto(303.87768148,487.20795392)(304.21768114,487.39295373)(304.60768404,487.52296404)
\curveto(304.71768064,487.56295356)(304.82768053,487.58795354)(304.93768404,487.59796404)
\curveto(305.0576803,487.61795351)(305.17768018,487.64295348)(305.29768404,487.67296404)
\curveto(305.36767999,487.68295344)(305.43267993,487.68795344)(305.49268404,487.68796404)
\curveto(305.55267981,487.68795344)(305.61767974,487.69295343)(305.68768404,487.70296404)
\curveto(306.38767897,487.7229534)(306.9626784,487.60795352)(307.41268404,487.35796404)
\curveto(307.8626775,487.10795402)(308.20767715,486.75795437)(308.44768404,486.30796404)
\curveto(308.5576768,486.07795505)(308.6576767,485.80295532)(308.74768404,485.48296404)
\curveto(308.76767659,485.41295571)(308.76767659,485.33795579)(308.74768404,485.25796404)
\curveto(308.73767662,485.18795594)(308.71267665,485.13795599)(308.67268404,485.10796404)
\curveto(308.64267672,485.07795605)(308.58267678,485.05295607)(308.49268404,485.03296404)
\curveto(308.40267696,485.0229561)(308.30267706,485.01295611)(308.19268404,485.00296404)
\curveto(308.09267727,485.00295612)(307.99267737,485.00795612)(307.89268404,485.01796404)
\curveto(307.80267756,485.0279561)(307.73767762,485.04795608)(307.69768404,485.07796404)
\curveto(307.58767777,485.14795598)(307.50767785,485.25795587)(307.45768404,485.40796404)
\curveto(307.41767794,485.55795557)(307.362678,485.68795544)(307.29268404,485.79796404)
\curveto(307.10267826,486.10795502)(306.82267854,486.33795479)(306.45268404,486.48796404)
\curveto(306.38267898,486.51795461)(306.30767905,486.53795459)(306.22768404,486.54796404)
\curveto(306.1576792,486.55795457)(306.08267928,486.57295455)(306.00268404,486.59296404)
\curveto(305.95267941,486.60295452)(305.88267948,486.60795452)(305.79268404,486.60796404)
\curveto(305.71267965,486.60795452)(305.64767971,486.60295452)(305.59768404,486.59296404)
\curveto(305.5576798,486.57295455)(305.52267984,486.56795456)(305.49268404,486.57796404)
\curveto(305.4626799,486.58795454)(305.42767993,486.58795454)(305.38768404,486.57796404)
\lineto(305.14768404,486.51796404)
\curveto(305.07768028,486.49795463)(305.00768035,486.47295465)(304.93768404,486.44296404)
\curveto(304.5576808,486.28295484)(304.26768109,486.07295505)(304.06768404,485.81296404)
\curveto(303.87768148,485.55295557)(303.70268166,485.23795589)(303.54268404,484.86796404)
\curveto(303.51268185,484.78795634)(303.48768187,484.70795642)(303.46768404,484.62796404)
\curveto(303.4576819,484.54795658)(303.43768192,484.46795666)(303.40768404,484.38796404)
\curveto(303.37768198,484.27795685)(303.35268201,484.16295696)(303.33268404,484.04296404)
\curveto(303.32268204,483.9229572)(303.30268206,483.80295732)(303.27268404,483.68296404)
\curveto(303.25268211,483.63295749)(303.24268212,483.58295754)(303.24268404,483.53296404)
\curveto(303.25268211,483.48295764)(303.24768211,483.43295769)(303.22768404,483.38296404)
\curveto(303.21768214,483.3229578)(303.21768214,483.24295788)(303.22768404,483.14296404)
\curveto(303.23768212,483.05295807)(303.25268211,482.99795813)(303.27268404,482.97796404)
\curveto(303.29268207,482.93795819)(303.32268204,482.91795821)(303.36268404,482.91796404)
\curveto(303.41268195,482.91795821)(303.4576819,482.9279582)(303.49768404,482.94796404)
\curveto(303.56768179,482.98795814)(303.62768173,483.03295809)(303.67768404,483.08296404)
\curveto(303.72768163,483.13295799)(303.78768157,483.18295794)(303.85768404,483.23296404)
\lineto(303.91768404,483.29296404)
\curveto(303.94768141,483.3229578)(303.97768138,483.34795778)(304.00768404,483.36796404)
\curveto(304.23768112,483.5279576)(304.51268085,483.66295746)(304.83268404,483.77296404)
\curveto(304.90268046,483.79295733)(304.97268039,483.80795732)(305.04268404,483.81796404)
\curveto(305.11268025,483.8279573)(305.18768017,483.84295728)(305.26768404,483.86296404)
\curveto(305.30768005,483.86295726)(305.34268002,483.86795726)(305.37268404,483.87796404)
\curveto(305.40267996,483.88795724)(305.43767992,483.88795724)(305.47768404,483.87796404)
\curveto(305.52767983,483.87795725)(305.56767979,483.88795724)(305.59768404,483.90796404)
\lineto(305.76268404,483.90796404)
\lineto(305.85268404,483.90796404)
\curveto(305.90267946,483.91795721)(305.94267942,483.91795721)(305.97268404,483.90796404)
\curveto(306.02267934,483.89795723)(306.07267929,483.89295723)(306.12268404,483.89296404)
\curveto(306.18267918,483.90295722)(306.23767912,483.90295722)(306.28768404,483.89296404)
\curveto(306.39767896,483.86295726)(306.50267886,483.84295728)(306.60268404,483.83296404)
\curveto(306.71267865,483.8229573)(306.81767854,483.79795733)(306.91768404,483.75796404)
\curveto(307.33767802,483.61795751)(307.68267768,483.43295769)(307.95268404,483.20296404)
\curveto(308.22267714,482.98295814)(308.4626769,482.69795843)(308.67268404,482.34796404)
\curveto(308.75267661,482.20795892)(308.81767654,482.05795907)(308.86768404,481.89796404)
\curveto(308.91767644,481.74795938)(308.96767639,481.58795954)(309.01768404,481.41796404)
\moveto(307.77268404,480.11296404)
\curveto(307.78267758,480.16296096)(307.78767757,480.20796092)(307.78768404,480.24796404)
\lineto(307.78768404,480.39796404)
\curveto(307.78767757,480.70796042)(307.74767761,480.99296013)(307.66768404,481.25296404)
\curveto(307.64767771,481.31295981)(307.62767773,481.36795976)(307.60768404,481.41796404)
\curveto(307.59767776,481.47795965)(307.58267778,481.53295959)(307.56268404,481.58296404)
\curveto(307.34267802,482.07295905)(306.99767836,482.4229587)(306.52768404,482.63296404)
\curveto(306.44767891,482.66295846)(306.36767899,482.68795844)(306.28768404,482.70796404)
\lineto(306.04768404,482.76796404)
\curveto(305.96767939,482.78795834)(305.87767948,482.79795833)(305.77768404,482.79796404)
\lineto(305.46268404,482.79796404)
\curveto(305.44267992,482.77795835)(305.40267996,482.76795836)(305.34268404,482.76796404)
\curveto(305.29268007,482.77795835)(305.24768011,482.77795835)(305.20768404,482.76796404)
\lineto(304.96768404,482.70796404)
\curveto(304.89768046,482.69795843)(304.82768053,482.67795845)(304.75768404,482.64796404)
\curveto(304.1576812,482.38795874)(303.75268161,481.9229592)(303.54268404,481.25296404)
\curveto(303.51268185,481.17295995)(303.49268187,481.09296003)(303.48268404,481.01296404)
\curveto(303.47268189,480.93296019)(303.4576819,480.84796028)(303.43768404,480.75796404)
\lineto(303.43768404,480.60796404)
\curveto(303.42768193,480.56796056)(303.42268194,480.49796063)(303.42268404,480.39796404)
\curveto(303.42268194,480.16796096)(303.44268192,479.97296115)(303.48268404,479.81296404)
\curveto(303.50268186,479.74296138)(303.51768184,479.67796145)(303.52768404,479.61796404)
\curveto(303.53768182,479.55796157)(303.5576818,479.49296163)(303.58768404,479.42296404)
\curveto(303.69768166,479.14296198)(303.84268152,478.89796223)(304.02268404,478.68796404)
\curveto(304.20268116,478.48796264)(304.43768092,478.3279628)(304.72768404,478.20796404)
\lineto(304.96768404,478.11796404)
\lineto(305.20768404,478.05796404)
\curveto(305.2576801,478.03796309)(305.29768006,478.03296309)(305.32768404,478.04296404)
\curveto(305.36767999,478.05296307)(305.41267995,478.04796308)(305.46268404,478.02796404)
\curveto(305.49267987,478.01796311)(305.54767981,478.01296311)(305.62768404,478.01296404)
\curveto(305.70767965,478.01296311)(305.76767959,478.01796311)(305.80768404,478.02796404)
\curveto(305.91767944,478.04796308)(306.02267934,478.06296306)(306.12268404,478.07296404)
\curveto(306.22267914,478.08296304)(306.31767904,478.11296301)(306.40768404,478.16296404)
\curveto(306.93767842,478.36296276)(307.32767803,478.73796239)(307.57768404,479.28796404)
\curveto(307.61767774,479.38796174)(307.64767771,479.49296163)(307.66768404,479.60296404)
\lineto(307.75768404,479.93296404)
\curveto(307.7576776,480.01296111)(307.7626776,480.07296105)(307.77268404,480.11296404)
}
}
{
\newrgbcolor{curcolor}{0 0 0}
\pscustom[linestyle=none,fillstyle=solid,fillcolor=curcolor]
{
\newpath
\moveto(314.42729341,487.70296404)
\curveto(314.52728856,487.70295342)(314.62228846,487.69295343)(314.71229341,487.67296404)
\curveto(314.80228828,487.66295346)(314.86728822,487.63295349)(314.90729341,487.58296404)
\curveto(314.96728812,487.50295362)(314.99728809,487.39795373)(314.99729341,487.26796404)
\lineto(314.99729341,486.87796404)
\lineto(314.99729341,485.37796404)
\lineto(314.99729341,478.98796404)
\lineto(314.99729341,477.81796404)
\lineto(314.99729341,477.50296404)
\curveto(315.00728808,477.40296372)(314.99228809,477.3229638)(314.95229341,477.26296404)
\curveto(314.90228818,477.18296394)(314.82728826,477.13296399)(314.72729341,477.11296404)
\curveto(314.63728845,477.10296402)(314.52728856,477.09796403)(314.39729341,477.09796404)
\lineto(314.17229341,477.09796404)
\curveto(314.09228899,477.11796401)(314.02228906,477.13296399)(313.96229341,477.14296404)
\curveto(313.90228918,477.16296396)(313.85228923,477.20296392)(313.81229341,477.26296404)
\curveto(313.77228931,477.3229638)(313.75228933,477.39796373)(313.75229341,477.48796404)
\lineto(313.75229341,477.78796404)
\lineto(313.75229341,478.88296404)
\lineto(313.75229341,484.22296404)
\curveto(313.73228935,484.31295681)(313.71728937,484.38795674)(313.70729341,484.44796404)
\curveto(313.70728938,484.51795661)(313.67728941,484.57795655)(313.61729341,484.62796404)
\curveto(313.54728954,484.67795645)(313.45728963,484.70295642)(313.34729341,484.70296404)
\curveto(313.24728984,484.71295641)(313.13728995,484.71795641)(313.01729341,484.71796404)
\lineto(311.87729341,484.71796404)
\lineto(311.38229341,484.71796404)
\curveto(311.22229186,484.7279564)(311.11229197,484.78795634)(311.05229341,484.89796404)
\curveto(311.03229205,484.9279562)(311.02229206,484.95795617)(311.02229341,484.98796404)
\curveto(311.02229206,485.0279561)(311.01729207,485.07295605)(311.00729341,485.12296404)
\curveto(310.9872921,485.24295588)(310.99229209,485.35295577)(311.02229341,485.45296404)
\curveto(311.06229202,485.55295557)(311.11729197,485.6229555)(311.18729341,485.66296404)
\curveto(311.26729182,485.71295541)(311.3872917,485.73795539)(311.54729341,485.73796404)
\curveto(311.70729138,485.73795539)(311.84229124,485.75295537)(311.95229341,485.78296404)
\curveto(312.00229108,485.79295533)(312.05729103,485.79795533)(312.11729341,485.79796404)
\curveto(312.17729091,485.80795532)(312.23729085,485.8229553)(312.29729341,485.84296404)
\curveto(312.44729064,485.89295523)(312.59229049,485.94295518)(312.73229341,485.99296404)
\curveto(312.87229021,486.05295507)(313.00729008,486.122955)(313.13729341,486.20296404)
\curveto(313.27728981,486.29295483)(313.39728969,486.39795473)(313.49729341,486.51796404)
\curveto(313.59728949,486.63795449)(313.69228939,486.76795436)(313.78229341,486.90796404)
\curveto(313.84228924,487.00795412)(313.8872892,487.11795401)(313.91729341,487.23796404)
\curveto(313.95728913,487.35795377)(314.00728908,487.46295366)(314.06729341,487.55296404)
\curveto(314.11728897,487.61295351)(314.1872889,487.65295347)(314.27729341,487.67296404)
\curveto(314.29728879,487.68295344)(314.32228876,487.68795344)(314.35229341,487.68796404)
\curveto(314.3822887,487.68795344)(314.40728868,487.69295343)(314.42729341,487.70296404)
}
}
{
\newrgbcolor{curcolor}{0 0 0}
\pscustom[linestyle=none,fillstyle=solid,fillcolor=curcolor]
{
\newpath
\moveto(319.67190279,478.73296404)
\lineto(319.97190279,478.73296404)
\curveto(320.08190073,478.74296238)(320.18690062,478.74296238)(320.28690279,478.73296404)
\curveto(320.39690041,478.73296239)(320.49690031,478.7229624)(320.58690279,478.70296404)
\curveto(320.67690013,478.69296243)(320.74690006,478.66796246)(320.79690279,478.62796404)
\curveto(320.81689999,478.60796252)(320.83189998,478.57796255)(320.84190279,478.53796404)
\curveto(320.86189995,478.49796263)(320.88189993,478.45296267)(320.90190279,478.40296404)
\lineto(320.90190279,478.32796404)
\curveto(320.9118999,478.27796285)(320.9118999,478.2229629)(320.90190279,478.16296404)
\lineto(320.90190279,478.01296404)
\lineto(320.90190279,477.53296404)
\curveto(320.90189991,477.36296376)(320.86189995,477.24296388)(320.78190279,477.17296404)
\curveto(320.7119001,477.122964)(320.62190019,477.09796403)(320.51190279,477.09796404)
\lineto(320.18190279,477.09796404)
\lineto(319.73190279,477.09796404)
\curveto(319.58190123,477.09796403)(319.46690134,477.127964)(319.38690279,477.18796404)
\curveto(319.34690146,477.21796391)(319.31690149,477.26796386)(319.29690279,477.33796404)
\curveto(319.27690153,477.41796371)(319.26190155,477.50296362)(319.25190279,477.59296404)
\lineto(319.25190279,477.87796404)
\curveto(319.26190155,477.97796315)(319.26690154,478.06296306)(319.26690279,478.13296404)
\lineto(319.26690279,478.32796404)
\curveto(319.26690154,478.38796274)(319.27690153,478.44296268)(319.29690279,478.49296404)
\curveto(319.33690147,478.60296252)(319.4069014,478.67296245)(319.50690279,478.70296404)
\curveto(319.53690127,478.70296242)(319.59190122,478.71296241)(319.67190279,478.73296404)
}
}
{
\newrgbcolor{curcolor}{0 0 0}
\pscustom[linestyle=none,fillstyle=solid,fillcolor=curcolor]
{
\newpath
\moveto(329.89205904,480.17296404)
\curveto(329.90205132,480.13296099)(329.90205132,480.08296104)(329.89205904,480.02296404)
\curveto(329.89205133,479.96296116)(329.88705133,479.91296121)(329.87705904,479.87296404)
\curveto(329.87705134,479.83296129)(329.87205135,479.79296133)(329.86205904,479.75296404)
\lineto(329.86205904,479.64796404)
\curveto(329.84205138,479.56796156)(329.82705139,479.48796164)(329.81705904,479.40796404)
\curveto(329.80705141,479.3279618)(329.78705143,479.25296187)(329.75705904,479.18296404)
\curveto(329.73705148,479.10296202)(329.7170515,479.0279621)(329.69705904,478.95796404)
\curveto(329.67705154,478.88796224)(329.64705157,478.81296231)(329.60705904,478.73296404)
\curveto(329.42705179,478.31296281)(329.17205205,477.97296315)(328.84205904,477.71296404)
\curveto(328.51205271,477.45296367)(328.1220531,477.24796388)(327.67205904,477.09796404)
\curveto(327.55205367,477.05796407)(327.42705379,477.03296409)(327.29705904,477.02296404)
\curveto(327.17705404,477.00296412)(327.05205417,476.97796415)(326.92205904,476.94796404)
\curveto(326.86205436,476.93796419)(326.79705442,476.93296419)(326.72705904,476.93296404)
\curveto(326.66705455,476.93296419)(326.60205462,476.9279642)(326.53205904,476.91796404)
\lineto(326.41205904,476.91796404)
\lineto(326.21705904,476.91796404)
\curveto(326.15705506,476.90796422)(326.10205512,476.91296421)(326.05205904,476.93296404)
\curveto(325.98205524,476.95296417)(325.9170553,476.95796417)(325.85705904,476.94796404)
\curveto(325.79705542,476.93796419)(325.73705548,476.94296418)(325.67705904,476.96296404)
\curveto(325.62705559,476.97296415)(325.58205564,476.97796415)(325.54205904,476.97796404)
\curveto(325.50205572,476.97796415)(325.45705576,476.98796414)(325.40705904,477.00796404)
\curveto(325.32705589,477.0279641)(325.25205597,477.04796408)(325.18205904,477.06796404)
\curveto(325.11205611,477.07796405)(325.04205618,477.09296403)(324.97205904,477.11296404)
\curveto(324.49205673,477.28296384)(324.09205713,477.49296363)(323.77205904,477.74296404)
\curveto(323.46205776,478.00296312)(323.21205801,478.35796277)(323.02205904,478.80796404)
\curveto(322.99205823,478.86796226)(322.96705825,478.9279622)(322.94705904,478.98796404)
\curveto(322.93705828,479.05796207)(322.9220583,479.13296199)(322.90205904,479.21296404)
\curveto(322.88205834,479.27296185)(322.86705835,479.33796179)(322.85705904,479.40796404)
\curveto(322.84705837,479.47796165)(322.83205839,479.54796158)(322.81205904,479.61796404)
\curveto(322.80205842,479.66796146)(322.79705842,479.70796142)(322.79705904,479.73796404)
\lineto(322.79705904,479.85796404)
\curveto(322.78705843,479.89796123)(322.77705844,479.94796118)(322.76705904,480.00796404)
\curveto(322.76705845,480.06796106)(322.77205845,480.11796101)(322.78205904,480.15796404)
\lineto(322.78205904,480.29296404)
\curveto(322.79205843,480.34296078)(322.79705842,480.39296073)(322.79705904,480.44296404)
\curveto(322.8170584,480.54296058)(322.83205839,480.63796049)(322.84205904,480.72796404)
\curveto(322.85205837,480.8279603)(322.87205835,480.9229602)(322.90205904,481.01296404)
\curveto(322.95205827,481.16295996)(323.00705821,481.30295982)(323.06705904,481.43296404)
\curveto(323.12705809,481.56295956)(323.19705802,481.68295944)(323.27705904,481.79296404)
\curveto(323.30705791,481.84295928)(323.33705788,481.88295924)(323.36705904,481.91296404)
\curveto(323.40705781,481.94295918)(323.44205778,481.97795915)(323.47205904,482.01796404)
\curveto(323.53205769,482.09795903)(323.60205762,482.16795896)(323.68205904,482.22796404)
\curveto(323.74205748,482.27795885)(323.80205742,482.3229588)(323.86205904,482.36296404)
\lineto(324.07205904,482.51296404)
\curveto(324.1220571,482.55295857)(324.17205705,482.58795854)(324.22205904,482.61796404)
\curveto(324.27205695,482.65795847)(324.30705691,482.71295841)(324.32705904,482.78296404)
\curveto(324.32705689,482.81295831)(324.3170569,482.83795829)(324.29705904,482.85796404)
\curveto(324.28705693,482.88795824)(324.27705694,482.91295821)(324.26705904,482.93296404)
\curveto(324.22705699,482.98295814)(324.17705704,483.0279581)(324.11705904,483.06796404)
\curveto(324.06705715,483.11795801)(324.0170572,483.16295796)(323.96705904,483.20296404)
\curveto(323.92705729,483.23295789)(323.87705734,483.28795784)(323.81705904,483.36796404)
\curveto(323.79705742,483.39795773)(323.76705745,483.4229577)(323.72705904,483.44296404)
\curveto(323.69705752,483.47295765)(323.67205755,483.50795762)(323.65205904,483.54796404)
\curveto(323.48205774,483.75795737)(323.35205787,484.00295712)(323.26205904,484.28296404)
\curveto(323.24205798,484.36295676)(323.22705799,484.44295668)(323.21705904,484.52296404)
\curveto(323.20705801,484.60295652)(323.19205803,484.68295644)(323.17205904,484.76296404)
\curveto(323.15205807,484.81295631)(323.14205808,484.87795625)(323.14205904,484.95796404)
\curveto(323.14205808,485.04795608)(323.15205807,485.11795601)(323.17205904,485.16796404)
\curveto(323.17205805,485.26795586)(323.17705804,485.33795579)(323.18705904,485.37796404)
\curveto(323.20705801,485.45795567)(323.222058,485.5279556)(323.23205904,485.58796404)
\curveto(323.24205798,485.65795547)(323.25705796,485.7279554)(323.27705904,485.79796404)
\curveto(323.42705779,486.2279549)(323.64205758,486.57295455)(323.92205904,486.83296404)
\curveto(324.21205701,487.09295403)(324.56205666,487.30795382)(324.97205904,487.47796404)
\curveto(325.08205614,487.5279536)(325.19705602,487.55795357)(325.31705904,487.56796404)
\curveto(325.44705577,487.58795354)(325.57705564,487.61795351)(325.70705904,487.65796404)
\curveto(325.78705543,487.65795347)(325.85705536,487.65795347)(325.91705904,487.65796404)
\curveto(325.98705523,487.66795346)(326.06205516,487.67795345)(326.14205904,487.68796404)
\curveto(326.93205429,487.70795342)(327.58705363,487.57795355)(328.10705904,487.29796404)
\curveto(328.63705258,487.01795411)(329.0170522,486.60795452)(329.24705904,486.06796404)
\curveto(329.35705186,485.83795529)(329.42705179,485.55295557)(329.45705904,485.21296404)
\curveto(329.49705172,484.88295624)(329.46705175,484.57795655)(329.36705904,484.29796404)
\curveto(329.32705189,484.16795696)(329.27705194,484.04795708)(329.21705904,483.93796404)
\curveto(329.16705205,483.8279573)(329.10705211,483.7229574)(329.03705904,483.62296404)
\curveto(329.0170522,483.58295754)(328.98705223,483.54795758)(328.94705904,483.51796404)
\lineto(328.85705904,483.42796404)
\curveto(328.80705241,483.33795779)(328.74705247,483.27295785)(328.67705904,483.23296404)
\curveto(328.62705259,483.18295794)(328.57205265,483.13295799)(328.51205904,483.08296404)
\curveto(328.46205276,483.04295808)(328.4170528,482.99795813)(328.37705904,482.94796404)
\curveto(328.35705286,482.9279582)(328.33705288,482.90295822)(328.31705904,482.87296404)
\curveto(328.30705291,482.85295827)(328.30705291,482.8279583)(328.31705904,482.79796404)
\curveto(328.32705289,482.74795838)(328.35705286,482.69795843)(328.40705904,482.64796404)
\curveto(328.45705276,482.60795852)(328.51205271,482.56795856)(328.57205904,482.52796404)
\lineto(328.75205904,482.40796404)
\curveto(328.81205241,482.37795875)(328.86205236,482.34795878)(328.90205904,482.31796404)
\curveto(329.23205199,482.07795905)(329.48205174,481.76795936)(329.65205904,481.38796404)
\curveto(329.69205153,481.30795982)(329.7220515,481.2229599)(329.74205904,481.13296404)
\curveto(329.77205145,481.04296008)(329.79705142,480.95296017)(329.81705904,480.86296404)
\curveto(329.82705139,480.81296031)(329.83705138,480.75796037)(329.84705904,480.69796404)
\lineto(329.87705904,480.54796404)
\curveto(329.88705133,480.48796064)(329.88705133,480.4229607)(329.87705904,480.35296404)
\curveto(329.86705135,480.29296083)(329.87205135,480.23296089)(329.89205904,480.17296404)
\moveto(324.50705904,485.21296404)
\curveto(324.47705674,485.10295602)(324.47205675,484.96295616)(324.49205904,484.79296404)
\curveto(324.51205671,484.63295649)(324.53705668,484.50795662)(324.56705904,484.41796404)
\curveto(324.67705654,484.09795703)(324.82705639,483.85295727)(325.01705904,483.68296404)
\curveto(325.20705601,483.5229576)(325.47205575,483.39295773)(325.81205904,483.29296404)
\curveto(325.94205528,483.26295786)(326.10705511,483.23795789)(326.30705904,483.21796404)
\curveto(326.50705471,483.20795792)(326.67705454,483.2229579)(326.81705904,483.26296404)
\curveto(327.10705411,483.34295778)(327.34705387,483.45295767)(327.53705904,483.59296404)
\curveto(327.73705348,483.74295738)(327.89205333,483.94295718)(328.00205904,484.19296404)
\curveto(328.0220532,484.24295688)(328.03205319,484.28795684)(328.03205904,484.32796404)
\curveto(328.04205318,484.36795676)(328.05705316,484.41295671)(328.07705904,484.46296404)
\curveto(328.10705311,484.57295655)(328.12705309,484.71295641)(328.13705904,484.88296404)
\curveto(328.14705307,485.05295607)(328.13705308,485.19795593)(328.10705904,485.31796404)
\curveto(328.08705313,485.40795572)(328.06205316,485.49295563)(328.03205904,485.57296404)
\curveto(328.01205321,485.65295547)(327.97705324,485.73295539)(327.92705904,485.81296404)
\curveto(327.75705346,486.08295504)(327.53205369,486.27795485)(327.25205904,486.39796404)
\curveto(326.98205424,486.51795461)(326.6220546,486.57795455)(326.17205904,486.57796404)
\curveto(326.15205507,486.55795457)(326.1220551,486.55295457)(326.08205904,486.56296404)
\curveto(326.04205518,486.57295455)(326.00705521,486.57295455)(325.97705904,486.56296404)
\curveto(325.92705529,486.54295458)(325.87205535,486.5279546)(325.81205904,486.51796404)
\curveto(325.76205546,486.51795461)(325.71205551,486.50795462)(325.66205904,486.48796404)
\curveto(325.4220558,486.39795473)(325.21205601,486.28295484)(325.03205904,486.14296404)
\curveto(324.85205637,486.01295511)(324.71205651,485.83295529)(324.61205904,485.60296404)
\curveto(324.59205663,485.54295558)(324.57205665,485.47795565)(324.55205904,485.40796404)
\curveto(324.54205668,485.34795578)(324.52705669,485.28295584)(324.50705904,485.21296404)
\moveto(328.52705904,479.67796404)
\curveto(328.57705264,479.86796126)(328.58205264,480.07296105)(328.54205904,480.29296404)
\curveto(328.51205271,480.51296061)(328.46705275,480.69296043)(328.40705904,480.83296404)
\curveto(328.23705298,481.20295992)(327.97705324,481.50795962)(327.62705904,481.74796404)
\curveto(327.28705393,481.98795914)(326.85205437,482.10795902)(326.32205904,482.10796404)
\curveto(326.29205493,482.08795904)(326.25205497,482.08295904)(326.20205904,482.09296404)
\curveto(326.15205507,482.11295901)(326.11205511,482.11795901)(326.08205904,482.10796404)
\lineto(325.81205904,482.04796404)
\curveto(325.73205549,482.03795909)(325.65205557,482.0229591)(325.57205904,482.00296404)
\curveto(325.27205595,481.89295923)(325.00705621,481.74795938)(324.77705904,481.56796404)
\curveto(324.55705666,481.38795974)(324.38705683,481.15795997)(324.26705904,480.87796404)
\curveto(324.23705698,480.79796033)(324.21205701,480.71796041)(324.19205904,480.63796404)
\curveto(324.17205705,480.55796057)(324.15205707,480.47296065)(324.13205904,480.38296404)
\curveto(324.10205712,480.26296086)(324.09205713,480.11296101)(324.10205904,479.93296404)
\curveto(324.1220571,479.75296137)(324.14705707,479.61296151)(324.17705904,479.51296404)
\curveto(324.19705702,479.46296166)(324.20705701,479.41796171)(324.20705904,479.37796404)
\curveto(324.217057,479.34796178)(324.23205699,479.30796182)(324.25205904,479.25796404)
\curveto(324.35205687,479.03796209)(324.48205674,478.83796229)(324.64205904,478.65796404)
\curveto(324.81205641,478.47796265)(325.00705621,478.34296278)(325.22705904,478.25296404)
\curveto(325.29705592,478.21296291)(325.39205583,478.17796295)(325.51205904,478.14796404)
\curveto(325.73205549,478.05796307)(325.98705523,478.01296311)(326.27705904,478.01296404)
\lineto(326.56205904,478.01296404)
\curveto(326.66205456,478.03296309)(326.75705446,478.04796308)(326.84705904,478.05796404)
\curveto(326.93705428,478.06796306)(327.02705419,478.08796304)(327.11705904,478.11796404)
\curveto(327.37705384,478.19796293)(327.6170536,478.3279628)(327.83705904,478.50796404)
\curveto(328.06705315,478.69796243)(328.23705298,478.91296221)(328.34705904,479.15296404)
\curveto(328.38705283,479.23296189)(328.4170528,479.31296181)(328.43705904,479.39296404)
\curveto(328.46705275,479.48296164)(328.49705272,479.57796155)(328.52705904,479.67796404)
}
}
{
\newrgbcolor{curcolor}{0 0 0}
\pscustom[linestyle=none,fillstyle=solid,fillcolor=curcolor]
{
\newpath
\moveto(341.03166841,485.61796404)
\curveto(340.83165811,485.3279558)(340.62165832,485.04295608)(340.40166841,484.76296404)
\curveto(340.19165875,484.48295664)(339.98665896,484.19795693)(339.78666841,483.90796404)
\curveto(339.18665976,483.05795807)(338.58166036,482.21795891)(337.97166841,481.38796404)
\curveto(337.36166158,480.56796056)(336.75666219,479.73296139)(336.15666841,478.88296404)
\lineto(335.64666841,478.16296404)
\lineto(335.13666841,477.47296404)
\curveto(335.05666389,477.36296376)(334.97666397,477.24796388)(334.89666841,477.12796404)
\curveto(334.81666413,477.00796412)(334.72166422,476.91296421)(334.61166841,476.84296404)
\curveto(334.57166437,476.8229643)(334.50666444,476.80796432)(334.41666841,476.79796404)
\curveto(334.33666461,476.77796435)(334.2466647,476.76796436)(334.14666841,476.76796404)
\curveto(334.0466649,476.76796436)(333.95166499,476.77296435)(333.86166841,476.78296404)
\curveto(333.78166516,476.79296433)(333.72166522,476.81296431)(333.68166841,476.84296404)
\curveto(333.65166529,476.86296426)(333.62666532,476.89796423)(333.60666841,476.94796404)
\curveto(333.59666535,476.98796414)(333.60166534,477.03296409)(333.62166841,477.08296404)
\curveto(333.66166528,477.16296396)(333.70666524,477.23796389)(333.75666841,477.30796404)
\curveto(333.81666513,477.38796374)(333.87166507,477.46796366)(333.92166841,477.54796404)
\curveto(334.16166478,477.88796324)(334.40666454,478.2229629)(334.65666841,478.55296404)
\curveto(334.90666404,478.88296224)(335.1466638,479.21796191)(335.37666841,479.55796404)
\curveto(335.53666341,479.77796135)(335.69666325,479.99296113)(335.85666841,480.20296404)
\curveto(336.01666293,480.41296071)(336.17666277,480.6279605)(336.33666841,480.84796404)
\curveto(336.69666225,481.36795976)(337.06166188,481.87795925)(337.43166841,482.37796404)
\curveto(337.80166114,482.87795825)(338.17166077,483.38795774)(338.54166841,483.90796404)
\curveto(338.68166026,484.10795702)(338.82166012,484.30295682)(338.96166841,484.49296404)
\curveto(339.11165983,484.68295644)(339.25665969,484.87795625)(339.39666841,485.07796404)
\curveto(339.60665934,485.37795575)(339.82165912,485.67795545)(340.04166841,485.97796404)
\lineto(340.70166841,486.87796404)
\lineto(340.88166841,487.14796404)
\lineto(341.09166841,487.41796404)
\lineto(341.21166841,487.59796404)
\curveto(341.26165768,487.65795347)(341.31165763,487.71295341)(341.36166841,487.76296404)
\curveto(341.43165751,487.81295331)(341.50665744,487.84795328)(341.58666841,487.86796404)
\curveto(341.60665734,487.87795325)(341.63165731,487.87795325)(341.66166841,487.86796404)
\curveto(341.70165724,487.86795326)(341.73165721,487.87795325)(341.75166841,487.89796404)
\curveto(341.87165707,487.89795323)(342.00665694,487.89295323)(342.15666841,487.88296404)
\curveto(342.30665664,487.88295324)(342.39665655,487.83795329)(342.42666841,487.74796404)
\curveto(342.4466565,487.71795341)(342.45165649,487.68295344)(342.44166841,487.64296404)
\curveto(342.43165651,487.60295352)(342.41665653,487.57295355)(342.39666841,487.55296404)
\curveto(342.35665659,487.47295365)(342.31665663,487.40295372)(342.27666841,487.34296404)
\curveto(342.23665671,487.28295384)(342.19165675,487.2229539)(342.14166841,487.16296404)
\lineto(341.57166841,486.38296404)
\curveto(341.39165755,486.13295499)(341.21165773,485.87795525)(341.03166841,485.61796404)
\moveto(334.17666841,481.71796404)
\curveto(334.12666482,481.73795939)(334.07666487,481.74295938)(334.02666841,481.73296404)
\curveto(333.97666497,481.7229594)(333.92666502,481.7279594)(333.87666841,481.74796404)
\curveto(333.76666518,481.76795936)(333.66166528,481.78795934)(333.56166841,481.80796404)
\curveto(333.47166547,481.83795929)(333.37666557,481.87795925)(333.27666841,481.92796404)
\curveto(332.946666,482.06795906)(332.69166625,482.26295886)(332.51166841,482.51296404)
\curveto(332.33166661,482.77295835)(332.18666676,483.08295804)(332.07666841,483.44296404)
\curveto(332.0466669,483.5229576)(332.02666692,483.60295752)(332.01666841,483.68296404)
\curveto(332.00666694,483.77295735)(331.99166695,483.85795727)(331.97166841,483.93796404)
\curveto(331.96166698,483.98795714)(331.95666699,484.05295707)(331.95666841,484.13296404)
\curveto(331.946667,484.16295696)(331.941667,484.19295693)(331.94166841,484.22296404)
\curveto(331.941667,484.26295686)(331.93666701,484.29795683)(331.92666841,484.32796404)
\lineto(331.92666841,484.47796404)
\curveto(331.91666703,484.5279566)(331.91166703,484.58795654)(331.91166841,484.65796404)
\curveto(331.91166703,484.73795639)(331.91666703,484.80295632)(331.92666841,484.85296404)
\lineto(331.92666841,485.01796404)
\curveto(331.946667,485.06795606)(331.95166699,485.11295601)(331.94166841,485.15296404)
\curveto(331.941667,485.20295592)(331.946667,485.24795588)(331.95666841,485.28796404)
\curveto(331.96666698,485.3279558)(331.97166697,485.36295576)(331.97166841,485.39296404)
\curveto(331.97166697,485.43295569)(331.97666697,485.47295565)(331.98666841,485.51296404)
\curveto(332.01666693,485.6229555)(332.03666691,485.73295539)(332.04666841,485.84296404)
\curveto(332.06666688,485.96295516)(332.10166684,486.07795505)(332.15166841,486.18796404)
\curveto(332.29166665,486.5279546)(332.45166649,486.80295432)(332.63166841,487.01296404)
\curveto(332.82166612,487.23295389)(333.09166585,487.41295371)(333.44166841,487.55296404)
\curveto(333.52166542,487.58295354)(333.60666534,487.60295352)(333.69666841,487.61296404)
\curveto(333.78666516,487.63295349)(333.88166506,487.65295347)(333.98166841,487.67296404)
\curveto(334.01166493,487.68295344)(334.06666488,487.68295344)(334.14666841,487.67296404)
\curveto(334.22666472,487.67295345)(334.27666467,487.68295344)(334.29666841,487.70296404)
\curveto(334.85666409,487.71295341)(335.30666364,487.60295352)(335.64666841,487.37296404)
\curveto(335.99666295,487.14295398)(336.25666269,486.83795429)(336.42666841,486.45796404)
\curveto(336.46666248,486.36795476)(336.50166244,486.27295485)(336.53166841,486.17296404)
\curveto(336.56166238,486.07295505)(336.58666236,485.97295515)(336.60666841,485.87296404)
\curveto(336.62666232,485.84295528)(336.63166231,485.81295531)(336.62166841,485.78296404)
\curveto(336.62166232,485.75295537)(336.62666232,485.7229554)(336.63666841,485.69296404)
\curveto(336.66666228,485.58295554)(336.68666226,485.45795567)(336.69666841,485.31796404)
\curveto(336.70666224,485.18795594)(336.71666223,485.05295607)(336.72666841,484.91296404)
\lineto(336.72666841,484.74796404)
\curveto(336.73666221,484.68795644)(336.73666221,484.63295649)(336.72666841,484.58296404)
\curveto(336.71666223,484.53295659)(336.71166223,484.48295664)(336.71166841,484.43296404)
\lineto(336.71166841,484.29796404)
\curveto(336.70166224,484.25795687)(336.69666225,484.21795691)(336.69666841,484.17796404)
\curveto(336.70666224,484.13795699)(336.70166224,484.09295703)(336.68166841,484.04296404)
\curveto(336.66166228,483.93295719)(336.6416623,483.8279573)(336.62166841,483.72796404)
\curveto(336.61166233,483.6279575)(336.59166235,483.5279576)(336.56166841,483.42796404)
\curveto(336.43166251,483.06795806)(336.26666268,482.75295837)(336.06666841,482.48296404)
\curveto(335.86666308,482.21295891)(335.59166335,482.00795912)(335.24166841,481.86796404)
\curveto(335.16166378,481.83795929)(335.07666387,481.81295931)(334.98666841,481.79296404)
\lineto(334.71666841,481.73296404)
\curveto(334.66666428,481.7229594)(334.62166432,481.71795941)(334.58166841,481.71796404)
\curveto(334.5416644,481.7279594)(334.50166444,481.7279594)(334.46166841,481.71796404)
\curveto(334.36166458,481.69795943)(334.26666468,481.69795943)(334.17666841,481.71796404)
\moveto(333.33666841,483.11296404)
\curveto(333.37666557,483.04295808)(333.41666553,482.97795815)(333.45666841,482.91796404)
\curveto(333.49666545,482.86795826)(333.5466654,482.81795831)(333.60666841,482.76796404)
\lineto(333.75666841,482.64796404)
\curveto(333.81666513,482.61795851)(333.88166506,482.59295853)(333.95166841,482.57296404)
\curveto(333.99166495,482.55295857)(334.02666492,482.54295858)(334.05666841,482.54296404)
\curveto(334.09666485,482.55295857)(334.13666481,482.54795858)(334.17666841,482.52796404)
\curveto(334.20666474,482.5279586)(334.2466647,482.5229586)(334.29666841,482.51296404)
\curveto(334.3466646,482.51295861)(334.38666456,482.51795861)(334.41666841,482.52796404)
\lineto(334.64166841,482.57296404)
\curveto(334.89166405,482.65295847)(335.07666387,482.77795835)(335.19666841,482.94796404)
\curveto(335.27666367,483.04795808)(335.3466636,483.17795795)(335.40666841,483.33796404)
\curveto(335.48666346,483.51795761)(335.5466634,483.74295738)(335.58666841,484.01296404)
\curveto(335.62666332,484.29295683)(335.6416633,484.57295655)(335.63166841,484.85296404)
\curveto(335.62166332,485.14295598)(335.59166335,485.41795571)(335.54166841,485.67796404)
\curveto(335.49166345,485.93795519)(335.41666353,486.14795498)(335.31666841,486.30796404)
\curveto(335.19666375,486.50795462)(335.0466639,486.65795447)(334.86666841,486.75796404)
\curveto(334.78666416,486.80795432)(334.69666425,486.83795429)(334.59666841,486.84796404)
\curveto(334.49666445,486.86795426)(334.39166455,486.87795425)(334.28166841,486.87796404)
\curveto(334.26166468,486.86795426)(334.23666471,486.86295426)(334.20666841,486.86296404)
\curveto(334.18666476,486.87295425)(334.16666478,486.87295425)(334.14666841,486.86296404)
\curveto(334.09666485,486.85295427)(334.05166489,486.84295428)(334.01166841,486.83296404)
\curveto(333.97166497,486.83295429)(333.93166501,486.8229543)(333.89166841,486.80296404)
\curveto(333.71166523,486.7229544)(333.56166538,486.60295452)(333.44166841,486.44296404)
\curveto(333.33166561,486.28295484)(333.2416657,486.10295502)(333.17166841,485.90296404)
\curveto(333.11166583,485.71295541)(333.06666588,485.48795564)(333.03666841,485.22796404)
\curveto(333.01666593,484.96795616)(333.01166593,484.70295642)(333.02166841,484.43296404)
\curveto(333.03166591,484.17295695)(333.06166588,483.9229572)(333.11166841,483.68296404)
\curveto(333.17166577,483.45295767)(333.2466657,483.26295786)(333.33666841,483.11296404)
\moveto(344.13666841,480.12796404)
\curveto(344.1466548,480.07796105)(344.15165479,479.98796114)(344.15166841,479.85796404)
\curveto(344.15165479,479.7279614)(344.1416548,479.63796149)(344.12166841,479.58796404)
\curveto(344.10165484,479.53796159)(344.09665485,479.48296164)(344.10666841,479.42296404)
\curveto(344.11665483,479.37296175)(344.11665483,479.3229618)(344.10666841,479.27296404)
\curveto(344.06665488,479.13296199)(344.03665491,478.99796213)(344.01666841,478.86796404)
\curveto(344.00665494,478.73796239)(343.97665497,478.61796251)(343.92666841,478.50796404)
\curveto(343.78665516,478.15796297)(343.62165532,477.86296326)(343.43166841,477.62296404)
\curveto(343.2416557,477.39296373)(342.97165597,477.20796392)(342.62166841,477.06796404)
\curveto(342.5416564,477.03796409)(342.45665649,477.01796411)(342.36666841,477.00796404)
\curveto(342.27665667,476.98796414)(342.19165675,476.96796416)(342.11166841,476.94796404)
\curveto(342.06165688,476.93796419)(342.01165693,476.93296419)(341.96166841,476.93296404)
\curveto(341.91165703,476.93296419)(341.86165708,476.9279642)(341.81166841,476.91796404)
\curveto(341.78165716,476.90796422)(341.73165721,476.90796422)(341.66166841,476.91796404)
\curveto(341.59165735,476.91796421)(341.5416574,476.9229642)(341.51166841,476.93296404)
\curveto(341.45165749,476.95296417)(341.39165755,476.96296416)(341.33166841,476.96296404)
\curveto(341.28165766,476.95296417)(341.23165771,476.95796417)(341.18166841,476.97796404)
\curveto(341.09165785,476.99796413)(341.00165794,477.0229641)(340.91166841,477.05296404)
\curveto(340.83165811,477.07296405)(340.75165819,477.10296402)(340.67166841,477.14296404)
\curveto(340.35165859,477.28296384)(340.10165884,477.47796365)(339.92166841,477.72796404)
\curveto(339.7416592,477.98796314)(339.59165935,478.29296283)(339.47166841,478.64296404)
\curveto(339.45165949,478.7229624)(339.43665951,478.80796232)(339.42666841,478.89796404)
\curveto(339.41665953,478.98796214)(339.40165954,479.07296205)(339.38166841,479.15296404)
\curveto(339.37165957,479.18296194)(339.36665958,479.21296191)(339.36666841,479.24296404)
\lineto(339.36666841,479.34796404)
\curveto(339.3466596,479.4279617)(339.33665961,479.50796162)(339.33666841,479.58796404)
\lineto(339.33666841,479.72296404)
\curveto(339.31665963,479.8229613)(339.31665963,479.9229612)(339.33666841,480.02296404)
\lineto(339.33666841,480.20296404)
\curveto(339.3466596,480.25296087)(339.35165959,480.29796083)(339.35166841,480.33796404)
\curveto(339.35165959,480.38796074)(339.35665959,480.43296069)(339.36666841,480.47296404)
\curveto(339.37665957,480.51296061)(339.38165956,480.54796058)(339.38166841,480.57796404)
\curveto(339.38165956,480.61796051)(339.38665956,480.65796047)(339.39666841,480.69796404)
\lineto(339.45666841,481.02796404)
\curveto(339.47665947,481.14795998)(339.50665944,481.25795987)(339.54666841,481.35796404)
\curveto(339.68665926,481.68795944)(339.8466591,481.96295916)(340.02666841,482.18296404)
\curveto(340.21665873,482.41295871)(340.47665847,482.59795853)(340.80666841,482.73796404)
\curveto(340.88665806,482.77795835)(340.97165797,482.80295832)(341.06166841,482.81296404)
\lineto(341.36166841,482.87296404)
\lineto(341.49666841,482.87296404)
\curveto(341.5466574,482.88295824)(341.59665735,482.88795824)(341.64666841,482.88796404)
\curveto(342.21665673,482.90795822)(342.67665627,482.80295832)(343.02666841,482.57296404)
\curveto(343.38665556,482.35295877)(343.65165529,482.05295907)(343.82166841,481.67296404)
\curveto(343.87165507,481.57295955)(343.91165503,481.47295965)(343.94166841,481.37296404)
\curveto(343.97165497,481.27295985)(344.00165494,481.16795996)(344.03166841,481.05796404)
\curveto(344.0416549,481.01796011)(344.0466549,480.98296014)(344.04666841,480.95296404)
\curveto(344.0466549,480.93296019)(344.05165489,480.90296022)(344.06166841,480.86296404)
\curveto(344.08165486,480.79296033)(344.09165485,480.71796041)(344.09166841,480.63796404)
\curveto(344.09165485,480.55796057)(344.10165484,480.47796065)(344.12166841,480.39796404)
\curveto(344.12165482,480.34796078)(344.12165482,480.30296082)(344.12166841,480.26296404)
\curveto(344.12165482,480.2229609)(344.12665482,480.17796095)(344.13666841,480.12796404)
\moveto(343.02666841,479.69296404)
\curveto(343.03665591,479.74296138)(343.0416559,479.81796131)(343.04166841,479.91796404)
\curveto(343.05165589,480.01796111)(343.0466559,480.09296103)(343.02666841,480.14296404)
\curveto(343.00665594,480.20296092)(343.00165594,480.25796087)(343.01166841,480.30796404)
\curveto(343.03165591,480.36796076)(343.03165591,480.4279607)(343.01166841,480.48796404)
\curveto(343.00165594,480.51796061)(342.99665595,480.55296057)(342.99666841,480.59296404)
\curveto(342.99665595,480.63296049)(342.99165595,480.67296045)(342.98166841,480.71296404)
\curveto(342.96165598,480.79296033)(342.941656,480.86796026)(342.92166841,480.93796404)
\curveto(342.91165603,481.01796011)(342.89665605,481.09796003)(342.87666841,481.17796404)
\curveto(342.8466561,481.23795989)(342.82165612,481.29795983)(342.80166841,481.35796404)
\curveto(342.78165616,481.41795971)(342.75165619,481.47795965)(342.71166841,481.53796404)
\curveto(342.61165633,481.70795942)(342.48165646,481.84295928)(342.32166841,481.94296404)
\curveto(342.2416567,481.99295913)(342.1466568,482.0279591)(342.03666841,482.04796404)
\curveto(341.92665702,482.06795906)(341.80165714,482.07795905)(341.66166841,482.07796404)
\curveto(341.6416573,482.06795906)(341.61665733,482.06295906)(341.58666841,482.06296404)
\curveto(341.55665739,482.07295905)(341.52665742,482.07295905)(341.49666841,482.06296404)
\lineto(341.34666841,482.00296404)
\curveto(341.29665765,481.99295913)(341.25165769,481.97795915)(341.21166841,481.95796404)
\curveto(341.02165792,481.84795928)(340.87665807,481.70295942)(340.77666841,481.52296404)
\curveto(340.68665826,481.34295978)(340.60665834,481.13795999)(340.53666841,480.90796404)
\curveto(340.49665845,480.77796035)(340.47665847,480.64296048)(340.47666841,480.50296404)
\curveto(340.47665847,480.37296075)(340.46665848,480.2279609)(340.44666841,480.06796404)
\curveto(340.43665851,480.01796111)(340.42665852,479.95796117)(340.41666841,479.88796404)
\curveto(340.41665853,479.81796131)(340.42665852,479.75796137)(340.44666841,479.70796404)
\lineto(340.44666841,479.54296404)
\lineto(340.44666841,479.36296404)
\curveto(340.45665849,479.31296181)(340.46665848,479.25796187)(340.47666841,479.19796404)
\curveto(340.48665846,479.14796198)(340.49165845,479.09296203)(340.49166841,479.03296404)
\curveto(340.50165844,478.97296215)(340.51665843,478.91796221)(340.53666841,478.86796404)
\curveto(340.58665836,478.67796245)(340.6466583,478.50296262)(340.71666841,478.34296404)
\curveto(340.78665816,478.18296294)(340.89165805,478.05296307)(341.03166841,477.95296404)
\curveto(341.16165778,477.85296327)(341.30165764,477.78296334)(341.45166841,477.74296404)
\curveto(341.48165746,477.73296339)(341.50665744,477.7279634)(341.52666841,477.72796404)
\curveto(341.55665739,477.73796339)(341.58665736,477.73796339)(341.61666841,477.72796404)
\curveto(341.63665731,477.7279634)(341.66665728,477.7229634)(341.70666841,477.71296404)
\curveto(341.7466572,477.71296341)(341.78165716,477.71796341)(341.81166841,477.72796404)
\curveto(341.85165709,477.73796339)(341.89165705,477.74296338)(341.93166841,477.74296404)
\curveto(341.97165697,477.74296338)(342.01165693,477.75296337)(342.05166841,477.77296404)
\curveto(342.29165665,477.85296327)(342.48665646,477.98796314)(342.63666841,478.17796404)
\curveto(342.75665619,478.35796277)(342.8466561,478.56296256)(342.90666841,478.79296404)
\curveto(342.92665602,478.86296226)(342.941656,478.93296219)(342.95166841,479.00296404)
\curveto(342.96165598,479.08296204)(342.97665597,479.16296196)(342.99666841,479.24296404)
\curveto(342.99665595,479.30296182)(343.00165594,479.34796178)(343.01166841,479.37796404)
\curveto(343.01165593,479.39796173)(343.01165593,479.4229617)(343.01166841,479.45296404)
\curveto(343.01165593,479.49296163)(343.01665593,479.5229616)(343.02666841,479.54296404)
\lineto(343.02666841,479.69296404)
}
}
{
\newrgbcolor{curcolor}{0 0 0}
\pscustom[linestyle=none,fillstyle=solid,fillcolor=curcolor]
{
\newpath
\moveto(632.35698091,642.25630877)
\curveto(632.45697606,642.25629815)(632.55197596,642.24629816)(632.64198091,642.22630877)
\curveto(632.73197578,642.21629819)(632.79697572,642.18629822)(632.83698091,642.13630877)
\curveto(632.89697562,642.05629835)(632.92697559,641.95129846)(632.92698091,641.82130877)
\lineto(632.92698091,641.43130877)
\lineto(632.92698091,639.93130877)
\lineto(632.92698091,633.54130877)
\lineto(632.92698091,632.37130877)
\lineto(632.92698091,632.05630877)
\curveto(632.93697558,631.95630845)(632.92197559,631.87630853)(632.88198091,631.81630877)
\curveto(632.83197568,631.73630867)(632.75697576,631.68630872)(632.65698091,631.66630877)
\curveto(632.56697595,631.65630875)(632.45697606,631.65130876)(632.32698091,631.65130877)
\lineto(632.10198091,631.65130877)
\curveto(632.02197649,631.67130874)(631.95197656,631.68630872)(631.89198091,631.69630877)
\curveto(631.83197668,631.71630869)(631.78197673,631.75630865)(631.74198091,631.81630877)
\curveto(631.70197681,631.87630853)(631.68197683,631.95130846)(631.68198091,632.04130877)
\lineto(631.68198091,632.34130877)
\lineto(631.68198091,633.43630877)
\lineto(631.68198091,638.77630877)
\curveto(631.66197685,638.86630154)(631.64697687,638.94130147)(631.63698091,639.00130877)
\curveto(631.63697688,639.07130134)(631.60697691,639.13130128)(631.54698091,639.18130877)
\curveto(631.47697704,639.23130118)(631.38697713,639.25630115)(631.27698091,639.25630877)
\curveto(631.17697734,639.26630114)(631.06697745,639.27130114)(630.94698091,639.27130877)
\lineto(629.80698091,639.27130877)
\lineto(629.31198091,639.27130877)
\curveto(629.15197936,639.28130113)(629.04197947,639.34130107)(628.98198091,639.45130877)
\curveto(628.96197955,639.48130093)(628.95197956,639.5113009)(628.95198091,639.54130877)
\curveto(628.95197956,639.58130083)(628.94697957,639.62630078)(628.93698091,639.67630877)
\curveto(628.9169796,639.79630061)(628.92197959,639.9063005)(628.95198091,640.00630877)
\curveto(628.99197952,640.1063003)(629.04697947,640.17630023)(629.11698091,640.21630877)
\curveto(629.19697932,640.26630014)(629.3169792,640.29130012)(629.47698091,640.29130877)
\curveto(629.63697888,640.29130012)(629.77197874,640.3063001)(629.88198091,640.33630877)
\curveto(629.93197858,640.34630006)(629.98697853,640.35130006)(630.04698091,640.35130877)
\curveto(630.10697841,640.36130005)(630.16697835,640.37630003)(630.22698091,640.39630877)
\curveto(630.37697814,640.44629996)(630.52197799,640.49629991)(630.66198091,640.54630877)
\curveto(630.80197771,640.6062998)(630.93697758,640.67629973)(631.06698091,640.75630877)
\curveto(631.20697731,640.84629956)(631.32697719,640.95129946)(631.42698091,641.07130877)
\curveto(631.52697699,641.19129922)(631.62197689,641.32129909)(631.71198091,641.46130877)
\curveto(631.77197674,641.56129885)(631.8169767,641.67129874)(631.84698091,641.79130877)
\curveto(631.88697663,641.9112985)(631.93697658,642.01629839)(631.99698091,642.10630877)
\curveto(632.04697647,642.16629824)(632.1169764,642.2062982)(632.20698091,642.22630877)
\curveto(632.22697629,642.23629817)(632.25197626,642.24129817)(632.28198091,642.24130877)
\curveto(632.3119762,642.24129817)(632.33697618,642.24629816)(632.35698091,642.25630877)
}
}
{
\newrgbcolor{curcolor}{0 0 0}
\pscustom[linestyle=none,fillstyle=solid,fillcolor=curcolor]
{
\newpath
\moveto(640.70659029,642.25630877)
\curveto(640.80658543,642.25629815)(640.90158534,642.24629816)(640.99159029,642.22630877)
\curveto(641.08158516,642.21629819)(641.14658509,642.18629822)(641.18659029,642.13630877)
\curveto(641.24658499,642.05629835)(641.27658496,641.95129846)(641.27659029,641.82130877)
\lineto(641.27659029,641.43130877)
\lineto(641.27659029,639.93130877)
\lineto(641.27659029,633.54130877)
\lineto(641.27659029,632.37130877)
\lineto(641.27659029,632.05630877)
\curveto(641.28658495,631.95630845)(641.27158497,631.87630853)(641.23159029,631.81630877)
\curveto(641.18158506,631.73630867)(641.10658513,631.68630872)(641.00659029,631.66630877)
\curveto(640.91658532,631.65630875)(640.80658543,631.65130876)(640.67659029,631.65130877)
\lineto(640.45159029,631.65130877)
\curveto(640.37158587,631.67130874)(640.30158594,631.68630872)(640.24159029,631.69630877)
\curveto(640.18158606,631.71630869)(640.13158611,631.75630865)(640.09159029,631.81630877)
\curveto(640.05158619,631.87630853)(640.03158621,631.95130846)(640.03159029,632.04130877)
\lineto(640.03159029,632.34130877)
\lineto(640.03159029,633.43630877)
\lineto(640.03159029,638.77630877)
\curveto(640.01158623,638.86630154)(639.99658624,638.94130147)(639.98659029,639.00130877)
\curveto(639.98658625,639.07130134)(639.95658628,639.13130128)(639.89659029,639.18130877)
\curveto(639.82658641,639.23130118)(639.7365865,639.25630115)(639.62659029,639.25630877)
\curveto(639.52658671,639.26630114)(639.41658682,639.27130114)(639.29659029,639.27130877)
\lineto(638.15659029,639.27130877)
\lineto(637.66159029,639.27130877)
\curveto(637.50158874,639.28130113)(637.39158885,639.34130107)(637.33159029,639.45130877)
\curveto(637.31158893,639.48130093)(637.30158894,639.5113009)(637.30159029,639.54130877)
\curveto(637.30158894,639.58130083)(637.29658894,639.62630078)(637.28659029,639.67630877)
\curveto(637.26658897,639.79630061)(637.27158897,639.9063005)(637.30159029,640.00630877)
\curveto(637.3415889,640.1063003)(637.39658884,640.17630023)(637.46659029,640.21630877)
\curveto(637.54658869,640.26630014)(637.66658857,640.29130012)(637.82659029,640.29130877)
\curveto(637.98658825,640.29130012)(638.12158812,640.3063001)(638.23159029,640.33630877)
\curveto(638.28158796,640.34630006)(638.3365879,640.35130006)(638.39659029,640.35130877)
\curveto(638.45658778,640.36130005)(638.51658772,640.37630003)(638.57659029,640.39630877)
\curveto(638.72658751,640.44629996)(638.87158737,640.49629991)(639.01159029,640.54630877)
\curveto(639.15158709,640.6062998)(639.28658695,640.67629973)(639.41659029,640.75630877)
\curveto(639.55658668,640.84629956)(639.67658656,640.95129946)(639.77659029,641.07130877)
\curveto(639.87658636,641.19129922)(639.97158627,641.32129909)(640.06159029,641.46130877)
\curveto(640.12158612,641.56129885)(640.16658607,641.67129874)(640.19659029,641.79130877)
\curveto(640.236586,641.9112985)(640.28658595,642.01629839)(640.34659029,642.10630877)
\curveto(640.39658584,642.16629824)(640.46658577,642.2062982)(640.55659029,642.22630877)
\curveto(640.57658566,642.23629817)(640.60158564,642.24129817)(640.63159029,642.24130877)
\curveto(640.66158558,642.24129817)(640.68658555,642.24629816)(640.70659029,642.25630877)
}
}
{
\newrgbcolor{curcolor}{0 0 0}
\pscustom[linestyle=none,fillstyle=solid,fillcolor=curcolor]
{
\newpath
\moveto(645.95119966,633.28630877)
\lineto(646.25119966,633.28630877)
\curveto(646.3611976,633.29630711)(646.4661975,633.29630711)(646.56619966,633.28630877)
\curveto(646.67619729,633.28630712)(646.77619719,633.27630713)(646.86619966,633.25630877)
\curveto(646.95619701,633.24630716)(647.02619694,633.22130719)(647.07619966,633.18130877)
\curveto(647.09619687,633.16130725)(647.11119685,633.13130728)(647.12119966,633.09130877)
\curveto(647.14119682,633.05130736)(647.1611968,633.0063074)(647.18119966,632.95630877)
\lineto(647.18119966,632.88130877)
\curveto(647.19119677,632.83130758)(647.19119677,632.77630763)(647.18119966,632.71630877)
\lineto(647.18119966,632.56630877)
\lineto(647.18119966,632.08630877)
\curveto(647.18119678,631.91630849)(647.14119682,631.79630861)(647.06119966,631.72630877)
\curveto(646.99119697,631.67630873)(646.90119706,631.65130876)(646.79119966,631.65130877)
\lineto(646.46119966,631.65130877)
\lineto(646.01119966,631.65130877)
\curveto(645.8611981,631.65130876)(645.74619822,631.68130873)(645.66619966,631.74130877)
\curveto(645.62619834,631.77130864)(645.59619837,631.82130859)(645.57619966,631.89130877)
\curveto(645.55619841,631.97130844)(645.54119842,632.05630835)(645.53119966,632.14630877)
\lineto(645.53119966,632.43130877)
\curveto(645.54119842,632.53130788)(645.54619842,632.61630779)(645.54619966,632.68630877)
\lineto(645.54619966,632.88130877)
\curveto(645.54619842,632.94130747)(645.55619841,632.99630741)(645.57619966,633.04630877)
\curveto(645.61619835,633.15630725)(645.68619828,633.22630718)(645.78619966,633.25630877)
\curveto(645.81619815,633.25630715)(645.87119809,633.26630714)(645.95119966,633.28630877)
}
}
{
\newrgbcolor{curcolor}{0 0 0}
\pscustom[linestyle=none,fillstyle=solid,fillcolor=curcolor]
{
\newpath
\moveto(656.03635591,635.14630877)
\curveto(656.10634827,635.09630531)(656.14634823,635.02630538)(656.15635591,634.93630877)
\curveto(656.1763482,634.84630556)(656.18634819,634.74130567)(656.18635591,634.62130877)
\curveto(656.18634819,634.57130584)(656.18134819,634.52130589)(656.17135591,634.47130877)
\curveto(656.1713482,634.42130599)(656.16134821,634.37630603)(656.14135591,634.33630877)
\curveto(656.11134826,634.24630616)(656.05134832,634.18630622)(655.96135591,634.15630877)
\curveto(655.88134849,634.13630627)(655.78634859,634.12630628)(655.67635591,634.12630877)
\lineto(655.36135591,634.12630877)
\curveto(655.25134912,634.13630627)(655.14634923,634.12630628)(655.04635591,634.09630877)
\curveto(654.90634947,634.06630634)(654.81634956,633.98630642)(654.77635591,633.85630877)
\curveto(654.75634962,633.78630662)(654.74634963,633.70130671)(654.74635591,633.60130877)
\lineto(654.74635591,633.33130877)
\lineto(654.74635591,632.38630877)
\lineto(654.74635591,632.05630877)
\curveto(654.74634963,631.94630846)(654.72634965,631.86130855)(654.68635591,631.80130877)
\curveto(654.64634973,631.74130867)(654.59634978,631.70130871)(654.53635591,631.68130877)
\curveto(654.48634989,631.67130874)(654.42134995,631.65630875)(654.34135591,631.63630877)
\lineto(654.14635591,631.63630877)
\curveto(654.02635035,631.63630877)(653.92135045,631.64130877)(653.83135591,631.65130877)
\curveto(653.74135063,631.67130874)(653.6713507,631.72130869)(653.62135591,631.80130877)
\curveto(653.59135078,631.85130856)(653.5763508,631.92130849)(653.57635591,632.01130877)
\lineto(653.57635591,632.31130877)
\lineto(653.57635591,633.34630877)
\curveto(653.5763508,633.5063069)(653.56635081,633.65130676)(653.54635591,633.78130877)
\curveto(653.53635084,633.92130649)(653.48135089,634.01630639)(653.38135591,634.06630877)
\curveto(653.33135104,634.08630632)(653.26135111,634.10130631)(653.17135591,634.11130877)
\curveto(653.09135128,634.12130629)(653.00135137,634.12630628)(652.90135591,634.12630877)
\lineto(652.61635591,634.12630877)
\lineto(652.37635591,634.12630877)
\lineto(650.11135591,634.12630877)
\curveto(650.02135435,634.12630628)(649.91635446,634.12130629)(649.79635591,634.11130877)
\lineto(649.46635591,634.11130877)
\curveto(649.35635502,634.1113063)(649.25635512,634.12130629)(649.16635591,634.14130877)
\curveto(649.0763553,634.16130625)(649.01635536,634.19630621)(648.98635591,634.24630877)
\curveto(648.93635544,634.31630609)(648.91135546,634.411306)(648.91135591,634.53130877)
\lineto(648.91135591,634.87630877)
\lineto(648.91135591,635.14630877)
\curveto(648.95135542,635.31630509)(649.00635537,635.45630495)(649.07635591,635.56630877)
\curveto(649.14635523,635.67630473)(649.22635515,635.79130462)(649.31635591,635.91130877)
\lineto(649.67635591,636.45130877)
\curveto(650.11635426,637.08130333)(650.55135382,637.70130271)(650.98135591,638.31130877)
\lineto(652.30135591,640.17130877)
\curveto(652.46135191,640.40130001)(652.61635176,640.62129979)(652.76635591,640.83130877)
\curveto(652.91635146,641.05129936)(653.0713513,641.27629913)(653.23135591,641.50630877)
\curveto(653.28135109,641.57629883)(653.33135104,641.64129877)(653.38135591,641.70130877)
\curveto(653.43135094,641.77129864)(653.48135089,641.84629856)(653.53135591,641.92630877)
\lineto(653.59135591,642.01630877)
\curveto(653.62135075,642.05629835)(653.65135072,642.08629832)(653.68135591,642.10630877)
\curveto(653.72135065,642.13629827)(653.76135061,642.15629825)(653.80135591,642.16630877)
\curveto(653.84135053,642.18629822)(653.88635049,642.2062982)(653.93635591,642.22630877)
\curveto(653.95635042,642.22629818)(653.9763504,642.22129819)(653.99635591,642.21130877)
\curveto(654.02635035,642.2112982)(654.05135032,642.22129819)(654.07135591,642.24130877)
\curveto(654.20135017,642.24129817)(654.32135005,642.23629817)(654.43135591,642.22630877)
\curveto(654.54134983,642.21629819)(654.62134975,642.17129824)(654.67135591,642.09130877)
\curveto(654.71134966,642.04129837)(654.73134964,641.97129844)(654.73135591,641.88130877)
\curveto(654.74134963,641.79129862)(654.74634963,641.69629871)(654.74635591,641.59630877)
\lineto(654.74635591,636.13630877)
\curveto(654.74634963,636.06630434)(654.74134963,635.99130442)(654.73135591,635.91130877)
\curveto(654.73134964,635.84130457)(654.73634964,635.77130464)(654.74635591,635.70130877)
\lineto(654.74635591,635.59630877)
\curveto(654.76634961,635.54630486)(654.78134959,635.49130492)(654.79135591,635.43130877)
\curveto(654.80134957,635.38130503)(654.82634955,635.34130507)(654.86635591,635.31130877)
\curveto(654.93634944,635.26130515)(655.02134935,635.23130518)(655.12135591,635.22130877)
\lineto(655.45135591,635.22130877)
\curveto(655.56134881,635.22130519)(655.66634871,635.21630519)(655.76635591,635.20630877)
\curveto(655.8763485,635.2063052)(655.96634841,635.18630522)(656.03635591,635.14630877)
\moveto(653.47135591,635.34130877)
\curveto(653.55135082,635.45130496)(653.58635079,635.62130479)(653.57635591,635.85130877)
\lineto(653.57635591,636.46630877)
\lineto(653.57635591,638.94130877)
\lineto(653.57635591,639.25630877)
\curveto(653.58635079,639.37630103)(653.58135079,639.47630093)(653.56135591,639.55630877)
\lineto(653.56135591,639.70630877)
\curveto(653.56135081,639.79630061)(653.54635083,639.88130053)(653.51635591,639.96130877)
\curveto(653.50635087,639.98130043)(653.49635088,639.99130042)(653.48635591,639.99130877)
\lineto(653.44135591,640.03630877)
\curveto(653.42135095,640.04630036)(653.39135098,640.05130036)(653.35135591,640.05130877)
\curveto(653.33135104,640.03130038)(653.31135106,640.01630039)(653.29135591,640.00630877)
\curveto(653.28135109,640.0063004)(653.26635111,640.00130041)(653.24635591,639.99130877)
\curveto(653.18635119,639.94130047)(653.12635125,639.87130054)(653.06635591,639.78130877)
\curveto(653.00635137,639.69130072)(652.95135142,639.6113008)(652.90135591,639.54130877)
\curveto(652.80135157,639.40130101)(652.70635167,639.25630115)(652.61635591,639.10630877)
\curveto(652.52635185,638.96630144)(652.43135194,638.82630158)(652.33135591,638.68630877)
\lineto(651.79135591,637.90630877)
\curveto(651.62135275,637.64630276)(651.44635293,637.38630302)(651.26635591,637.12630877)
\curveto(651.18635319,637.01630339)(651.11135326,636.9113035)(651.04135591,636.81130877)
\lineto(650.83135591,636.51130877)
\curveto(650.78135359,636.43130398)(650.73135364,636.35630405)(650.68135591,636.28630877)
\curveto(650.64135373,636.21630419)(650.59635378,636.14130427)(650.54635591,636.06130877)
\curveto(650.49635388,636.00130441)(650.44635393,635.93630447)(650.39635591,635.86630877)
\curveto(650.35635402,635.8063046)(650.31635406,635.73630467)(650.27635591,635.65630877)
\curveto(650.23635414,635.59630481)(650.21135416,635.52630488)(650.20135591,635.44630877)
\curveto(650.19135418,635.37630503)(650.22635415,635.32130509)(650.30635591,635.28130877)
\curveto(650.376354,635.23130518)(650.48635389,635.2063052)(650.63635591,635.20630877)
\curveto(650.79635358,635.21630519)(650.93135344,635.22130519)(651.04135591,635.22130877)
\lineto(652.72135591,635.22130877)
\lineto(653.15635591,635.22130877)
\curveto(653.30635107,635.22130519)(653.41135096,635.26130515)(653.47135591,635.34130877)
}
}
{
\newrgbcolor{curcolor}{0 0 0}
\pscustom[linestyle=none,fillstyle=solid,fillcolor=curcolor]
{
\newpath
\moveto(667.31096529,640.17130877)
\curveto(667.11095499,639.88130053)(666.9009552,639.59630081)(666.68096529,639.31630877)
\curveto(666.47095563,639.03630137)(666.26595583,638.75130166)(666.06596529,638.46130877)
\curveto(665.46595663,637.6113028)(664.86095724,636.77130364)(664.25096529,635.94130877)
\curveto(663.64095846,635.12130529)(663.03595906,634.28630612)(662.43596529,633.43630877)
\lineto(661.92596529,632.71630877)
\lineto(661.41596529,632.02630877)
\curveto(661.33596076,631.91630849)(661.25596084,631.80130861)(661.17596529,631.68130877)
\curveto(661.095961,631.56130885)(661.0009611,631.46630894)(660.89096529,631.39630877)
\curveto(660.85096125,631.37630903)(660.78596131,631.36130905)(660.69596529,631.35130877)
\curveto(660.61596148,631.33130908)(660.52596157,631.32130909)(660.42596529,631.32130877)
\curveto(660.32596177,631.32130909)(660.23096187,631.32630908)(660.14096529,631.33630877)
\curveto(660.06096204,631.34630906)(660.0009621,631.36630904)(659.96096529,631.39630877)
\curveto(659.93096217,631.41630899)(659.90596219,631.45130896)(659.88596529,631.50130877)
\curveto(659.87596222,631.54130887)(659.88096222,631.58630882)(659.90096529,631.63630877)
\curveto(659.94096216,631.71630869)(659.98596211,631.79130862)(660.03596529,631.86130877)
\curveto(660.095962,631.94130847)(660.15096195,632.02130839)(660.20096529,632.10130877)
\curveto(660.44096166,632.44130797)(660.68596141,632.77630763)(660.93596529,633.10630877)
\curveto(661.18596091,633.43630697)(661.42596067,633.77130664)(661.65596529,634.11130877)
\curveto(661.81596028,634.33130608)(661.97596012,634.54630586)(662.13596529,634.75630877)
\curveto(662.2959598,634.96630544)(662.45595964,635.18130523)(662.61596529,635.40130877)
\curveto(662.97595912,635.92130449)(663.34095876,636.43130398)(663.71096529,636.93130877)
\curveto(664.08095802,637.43130298)(664.45095765,637.94130247)(664.82096529,638.46130877)
\curveto(664.96095714,638.66130175)(665.100957,638.85630155)(665.24096529,639.04630877)
\curveto(665.39095671,639.23630117)(665.53595656,639.43130098)(665.67596529,639.63130877)
\curveto(665.88595621,639.93130048)(666.100956,640.23130018)(666.32096529,640.53130877)
\lineto(666.98096529,641.43130877)
\lineto(667.16096529,641.70130877)
\lineto(667.37096529,641.97130877)
\lineto(667.49096529,642.15130877)
\curveto(667.54095456,642.2112982)(667.59095451,642.26629814)(667.64096529,642.31630877)
\curveto(667.71095439,642.36629804)(667.78595431,642.40129801)(667.86596529,642.42130877)
\curveto(667.88595421,642.43129798)(667.91095419,642.43129798)(667.94096529,642.42130877)
\curveto(667.98095412,642.42129799)(668.01095409,642.43129798)(668.03096529,642.45130877)
\curveto(668.15095395,642.45129796)(668.28595381,642.44629796)(668.43596529,642.43630877)
\curveto(668.58595351,642.43629797)(668.67595342,642.39129802)(668.70596529,642.30130877)
\curveto(668.72595337,642.27129814)(668.73095337,642.23629817)(668.72096529,642.19630877)
\curveto(668.71095339,642.15629825)(668.6959534,642.12629828)(668.67596529,642.10630877)
\curveto(668.63595346,642.02629838)(668.5959535,641.95629845)(668.55596529,641.89630877)
\curveto(668.51595358,641.83629857)(668.47095363,641.77629863)(668.42096529,641.71630877)
\lineto(667.85096529,640.93630877)
\curveto(667.67095443,640.68629972)(667.49095461,640.43129998)(667.31096529,640.17130877)
\moveto(660.45596529,636.27130877)
\curveto(660.40596169,636.29130412)(660.35596174,636.29630411)(660.30596529,636.28630877)
\curveto(660.25596184,636.27630413)(660.20596189,636.28130413)(660.15596529,636.30130877)
\curveto(660.04596205,636.32130409)(659.94096216,636.34130407)(659.84096529,636.36130877)
\curveto(659.75096235,636.39130402)(659.65596244,636.43130398)(659.55596529,636.48130877)
\curveto(659.22596287,636.62130379)(658.97096313,636.81630359)(658.79096529,637.06630877)
\curveto(658.61096349,637.32630308)(658.46596363,637.63630277)(658.35596529,637.99630877)
\curveto(658.32596377,638.07630233)(658.30596379,638.15630225)(658.29596529,638.23630877)
\curveto(658.28596381,638.32630208)(658.27096383,638.411302)(658.25096529,638.49130877)
\curveto(658.24096386,638.54130187)(658.23596386,638.6063018)(658.23596529,638.68630877)
\curveto(658.22596387,638.71630169)(658.22096388,638.74630166)(658.22096529,638.77630877)
\curveto(658.22096388,638.81630159)(658.21596388,638.85130156)(658.20596529,638.88130877)
\lineto(658.20596529,639.03130877)
\curveto(658.1959639,639.08130133)(658.19096391,639.14130127)(658.19096529,639.21130877)
\curveto(658.19096391,639.29130112)(658.1959639,639.35630105)(658.20596529,639.40630877)
\lineto(658.20596529,639.57130877)
\curveto(658.22596387,639.62130079)(658.23096387,639.66630074)(658.22096529,639.70630877)
\curveto(658.22096388,639.75630065)(658.22596387,639.80130061)(658.23596529,639.84130877)
\curveto(658.24596385,639.88130053)(658.25096385,639.91630049)(658.25096529,639.94630877)
\curveto(658.25096385,639.98630042)(658.25596384,640.02630038)(658.26596529,640.06630877)
\curveto(658.2959638,640.17630023)(658.31596378,640.28630012)(658.32596529,640.39630877)
\curveto(658.34596375,640.51629989)(658.38096372,640.63129978)(658.43096529,640.74130877)
\curveto(658.57096353,641.08129933)(658.73096337,641.35629905)(658.91096529,641.56630877)
\curveto(659.100963,641.78629862)(659.37096273,641.96629844)(659.72096529,642.10630877)
\curveto(659.8009623,642.13629827)(659.88596221,642.15629825)(659.97596529,642.16630877)
\curveto(660.06596203,642.18629822)(660.16096194,642.2062982)(660.26096529,642.22630877)
\curveto(660.29096181,642.23629817)(660.34596175,642.23629817)(660.42596529,642.22630877)
\curveto(660.50596159,642.22629818)(660.55596154,642.23629817)(660.57596529,642.25630877)
\curveto(661.13596096,642.26629814)(661.58596051,642.15629825)(661.92596529,641.92630877)
\curveto(662.27595982,641.69629871)(662.53595956,641.39129902)(662.70596529,641.01130877)
\curveto(662.74595935,640.92129949)(662.78095932,640.82629958)(662.81096529,640.72630877)
\curveto(662.84095926,640.62629978)(662.86595923,640.52629988)(662.88596529,640.42630877)
\curveto(662.90595919,640.39630001)(662.91095919,640.36630004)(662.90096529,640.33630877)
\curveto(662.9009592,640.3063001)(662.90595919,640.27630013)(662.91596529,640.24630877)
\curveto(662.94595915,640.13630027)(662.96595913,640.0113004)(662.97596529,639.87130877)
\curveto(662.98595911,639.74130067)(662.9959591,639.6063008)(663.00596529,639.46630877)
\lineto(663.00596529,639.30130877)
\curveto(663.01595908,639.24130117)(663.01595908,639.18630122)(663.00596529,639.13630877)
\curveto(662.9959591,639.08630132)(662.99095911,639.03630137)(662.99096529,638.98630877)
\lineto(662.99096529,638.85130877)
\curveto(662.98095912,638.8113016)(662.97595912,638.77130164)(662.97596529,638.73130877)
\curveto(662.98595911,638.69130172)(662.98095912,638.64630176)(662.96096529,638.59630877)
\curveto(662.94095916,638.48630192)(662.92095918,638.38130203)(662.90096529,638.28130877)
\curveto(662.89095921,638.18130223)(662.87095923,638.08130233)(662.84096529,637.98130877)
\curveto(662.71095939,637.62130279)(662.54595955,637.3063031)(662.34596529,637.03630877)
\curveto(662.14595995,636.76630364)(661.87096023,636.56130385)(661.52096529,636.42130877)
\curveto(661.44096066,636.39130402)(661.35596074,636.36630404)(661.26596529,636.34630877)
\lineto(660.99596529,636.28630877)
\curveto(660.94596115,636.27630413)(660.9009612,636.27130414)(660.86096529,636.27130877)
\curveto(660.82096128,636.28130413)(660.78096132,636.28130413)(660.74096529,636.27130877)
\curveto(660.64096146,636.25130416)(660.54596155,636.25130416)(660.45596529,636.27130877)
\moveto(659.61596529,637.66630877)
\curveto(659.65596244,637.59630281)(659.6959624,637.53130288)(659.73596529,637.47130877)
\curveto(659.77596232,637.42130299)(659.82596227,637.37130304)(659.88596529,637.32130877)
\lineto(660.03596529,637.20130877)
\curveto(660.095962,637.17130324)(660.16096194,637.14630326)(660.23096529,637.12630877)
\curveto(660.27096183,637.1063033)(660.30596179,637.09630331)(660.33596529,637.09630877)
\curveto(660.37596172,637.1063033)(660.41596168,637.10130331)(660.45596529,637.08130877)
\curveto(660.48596161,637.08130333)(660.52596157,637.07630333)(660.57596529,637.06630877)
\curveto(660.62596147,637.06630334)(660.66596143,637.07130334)(660.69596529,637.08130877)
\lineto(660.92096529,637.12630877)
\curveto(661.17096093,637.2063032)(661.35596074,637.33130308)(661.47596529,637.50130877)
\curveto(661.55596054,637.60130281)(661.62596047,637.73130268)(661.68596529,637.89130877)
\curveto(661.76596033,638.07130234)(661.82596027,638.29630211)(661.86596529,638.56630877)
\curveto(661.90596019,638.84630156)(661.92096018,639.12630128)(661.91096529,639.40630877)
\curveto(661.9009602,639.69630071)(661.87096023,639.97130044)(661.82096529,640.23130877)
\curveto(661.77096033,640.49129992)(661.6959604,640.70129971)(661.59596529,640.86130877)
\curveto(661.47596062,641.06129935)(661.32596077,641.2112992)(661.14596529,641.31130877)
\curveto(661.06596103,641.36129905)(660.97596112,641.39129902)(660.87596529,641.40130877)
\curveto(660.77596132,641.42129899)(660.67096143,641.43129898)(660.56096529,641.43130877)
\curveto(660.54096156,641.42129899)(660.51596158,641.41629899)(660.48596529,641.41630877)
\curveto(660.46596163,641.42629898)(660.44596165,641.42629898)(660.42596529,641.41630877)
\curveto(660.37596172,641.406299)(660.33096177,641.39629901)(660.29096529,641.38630877)
\curveto(660.25096185,641.38629902)(660.21096189,641.37629903)(660.17096529,641.35630877)
\curveto(659.99096211,641.27629913)(659.84096226,641.15629925)(659.72096529,640.99630877)
\curveto(659.61096249,640.83629957)(659.52096258,640.65629975)(659.45096529,640.45630877)
\curveto(659.39096271,640.26630014)(659.34596275,640.04130037)(659.31596529,639.78130877)
\curveto(659.2959628,639.52130089)(659.29096281,639.25630115)(659.30096529,638.98630877)
\curveto(659.31096279,638.72630168)(659.34096276,638.47630193)(659.39096529,638.23630877)
\curveto(659.45096265,638.0063024)(659.52596257,637.81630259)(659.61596529,637.66630877)
\moveto(670.41596529,634.68130877)
\curveto(670.42595167,634.63130578)(670.43095167,634.54130587)(670.43096529,634.41130877)
\curveto(670.43095167,634.28130613)(670.42095168,634.19130622)(670.40096529,634.14130877)
\curveto(670.38095172,634.09130632)(670.37595172,634.03630637)(670.38596529,633.97630877)
\curveto(670.3959517,633.92630648)(670.3959517,633.87630653)(670.38596529,633.82630877)
\curveto(670.34595175,633.68630672)(670.31595178,633.55130686)(670.29596529,633.42130877)
\curveto(670.28595181,633.29130712)(670.25595184,633.17130724)(670.20596529,633.06130877)
\curveto(670.06595203,632.7113077)(669.9009522,632.41630799)(669.71096529,632.17630877)
\curveto(669.52095258,631.94630846)(669.25095285,631.76130865)(668.90096529,631.62130877)
\curveto(668.82095328,631.59130882)(668.73595336,631.57130884)(668.64596529,631.56130877)
\curveto(668.55595354,631.54130887)(668.47095363,631.52130889)(668.39096529,631.50130877)
\curveto(668.34095376,631.49130892)(668.29095381,631.48630892)(668.24096529,631.48630877)
\curveto(668.19095391,631.48630892)(668.14095396,631.48130893)(668.09096529,631.47130877)
\curveto(668.06095404,631.46130895)(668.01095409,631.46130895)(667.94096529,631.47130877)
\curveto(667.87095423,631.47130894)(667.82095428,631.47630893)(667.79096529,631.48630877)
\curveto(667.73095437,631.5063089)(667.67095443,631.51630889)(667.61096529,631.51630877)
\curveto(667.56095454,631.5063089)(667.51095459,631.5113089)(667.46096529,631.53130877)
\curveto(667.37095473,631.55130886)(667.28095482,631.57630883)(667.19096529,631.60630877)
\curveto(667.11095499,631.62630878)(667.03095507,631.65630875)(666.95096529,631.69630877)
\curveto(666.63095547,631.83630857)(666.38095572,632.03130838)(666.20096529,632.28130877)
\curveto(666.02095608,632.54130787)(665.87095623,632.84630756)(665.75096529,633.19630877)
\curveto(665.73095637,633.27630713)(665.71595638,633.36130705)(665.70596529,633.45130877)
\curveto(665.6959564,633.54130687)(665.68095642,633.62630678)(665.66096529,633.70630877)
\curveto(665.65095645,633.73630667)(665.64595645,633.76630664)(665.64596529,633.79630877)
\lineto(665.64596529,633.90130877)
\curveto(665.62595647,633.98130643)(665.61595648,634.06130635)(665.61596529,634.14130877)
\lineto(665.61596529,634.27630877)
\curveto(665.5959565,634.37630603)(665.5959565,634.47630593)(665.61596529,634.57630877)
\lineto(665.61596529,634.75630877)
\curveto(665.62595647,634.8063056)(665.63095647,634.85130556)(665.63096529,634.89130877)
\curveto(665.63095647,634.94130547)(665.63595646,634.98630542)(665.64596529,635.02630877)
\curveto(665.65595644,635.06630534)(665.66095644,635.10130531)(665.66096529,635.13130877)
\curveto(665.66095644,635.17130524)(665.66595643,635.2113052)(665.67596529,635.25130877)
\lineto(665.73596529,635.58130877)
\curveto(665.75595634,635.70130471)(665.78595631,635.8113046)(665.82596529,635.91130877)
\curveto(665.96595613,636.24130417)(666.12595597,636.51630389)(666.30596529,636.73630877)
\curveto(666.4959556,636.96630344)(666.75595534,637.15130326)(667.08596529,637.29130877)
\curveto(667.16595493,637.33130308)(667.25095485,637.35630305)(667.34096529,637.36630877)
\lineto(667.64096529,637.42630877)
\lineto(667.77596529,637.42630877)
\curveto(667.82595427,637.43630297)(667.87595422,637.44130297)(667.92596529,637.44130877)
\curveto(668.4959536,637.46130295)(668.95595314,637.35630305)(669.30596529,637.12630877)
\curveto(669.66595243,636.9063035)(669.93095217,636.6063038)(670.10096529,636.22630877)
\curveto(670.15095195,636.12630428)(670.19095191,636.02630438)(670.22096529,635.92630877)
\curveto(670.25095185,635.82630458)(670.28095182,635.72130469)(670.31096529,635.61130877)
\curveto(670.32095178,635.57130484)(670.32595177,635.53630487)(670.32596529,635.50630877)
\curveto(670.32595177,635.48630492)(670.33095177,635.45630495)(670.34096529,635.41630877)
\curveto(670.36095174,635.34630506)(670.37095173,635.27130514)(670.37096529,635.19130877)
\curveto(670.37095173,635.1113053)(670.38095172,635.03130538)(670.40096529,634.95130877)
\curveto(670.4009517,634.90130551)(670.4009517,634.85630555)(670.40096529,634.81630877)
\curveto(670.4009517,634.77630563)(670.40595169,634.73130568)(670.41596529,634.68130877)
\moveto(669.30596529,634.24630877)
\curveto(669.31595278,634.29630611)(669.32095278,634.37130604)(669.32096529,634.47130877)
\curveto(669.33095277,634.57130584)(669.32595277,634.64630576)(669.30596529,634.69630877)
\curveto(669.28595281,634.75630565)(669.28095282,634.8113056)(669.29096529,634.86130877)
\curveto(669.31095279,634.92130549)(669.31095279,634.98130543)(669.29096529,635.04130877)
\curveto(669.28095282,635.07130534)(669.27595282,635.1063053)(669.27596529,635.14630877)
\curveto(669.27595282,635.18630522)(669.27095283,635.22630518)(669.26096529,635.26630877)
\curveto(669.24095286,635.34630506)(669.22095288,635.42130499)(669.20096529,635.49130877)
\curveto(669.19095291,635.57130484)(669.17595292,635.65130476)(669.15596529,635.73130877)
\curveto(669.12595297,635.79130462)(669.100953,635.85130456)(669.08096529,635.91130877)
\curveto(669.06095304,635.97130444)(669.03095307,636.03130438)(668.99096529,636.09130877)
\curveto(668.89095321,636.26130415)(668.76095334,636.39630401)(668.60096529,636.49630877)
\curveto(668.52095358,636.54630386)(668.42595367,636.58130383)(668.31596529,636.60130877)
\curveto(668.20595389,636.62130379)(668.08095402,636.63130378)(667.94096529,636.63130877)
\curveto(667.92095418,636.62130379)(667.8959542,636.61630379)(667.86596529,636.61630877)
\curveto(667.83595426,636.62630378)(667.80595429,636.62630378)(667.77596529,636.61630877)
\lineto(667.62596529,636.55630877)
\curveto(667.57595452,636.54630386)(667.53095457,636.53130388)(667.49096529,636.51130877)
\curveto(667.3009548,636.40130401)(667.15595494,636.25630415)(667.05596529,636.07630877)
\curveto(666.96595513,635.89630451)(666.88595521,635.69130472)(666.81596529,635.46130877)
\curveto(666.77595532,635.33130508)(666.75595534,635.19630521)(666.75596529,635.05630877)
\curveto(666.75595534,634.92630548)(666.74595535,634.78130563)(666.72596529,634.62130877)
\curveto(666.71595538,634.57130584)(666.70595539,634.5113059)(666.69596529,634.44130877)
\curveto(666.6959554,634.37130604)(666.70595539,634.3113061)(666.72596529,634.26130877)
\lineto(666.72596529,634.09630877)
\lineto(666.72596529,633.91630877)
\curveto(666.73595536,633.86630654)(666.74595535,633.8113066)(666.75596529,633.75130877)
\curveto(666.76595533,633.70130671)(666.77095533,633.64630676)(666.77096529,633.58630877)
\curveto(666.78095532,633.52630688)(666.7959553,633.47130694)(666.81596529,633.42130877)
\curveto(666.86595523,633.23130718)(666.92595517,633.05630735)(666.99596529,632.89630877)
\curveto(667.06595503,632.73630767)(667.17095493,632.6063078)(667.31096529,632.50630877)
\curveto(667.44095466,632.406308)(667.58095452,632.33630807)(667.73096529,632.29630877)
\curveto(667.76095434,632.28630812)(667.78595431,632.28130813)(667.80596529,632.28130877)
\curveto(667.83595426,632.29130812)(667.86595423,632.29130812)(667.89596529,632.28130877)
\curveto(667.91595418,632.28130813)(667.94595415,632.27630813)(667.98596529,632.26630877)
\curveto(668.02595407,632.26630814)(668.06095404,632.27130814)(668.09096529,632.28130877)
\curveto(668.13095397,632.29130812)(668.17095393,632.29630811)(668.21096529,632.29630877)
\curveto(668.25095385,632.29630811)(668.29095381,632.3063081)(668.33096529,632.32630877)
\curveto(668.57095353,632.406308)(668.76595333,632.54130787)(668.91596529,632.73130877)
\curveto(669.03595306,632.9113075)(669.12595297,633.11630729)(669.18596529,633.34630877)
\curveto(669.20595289,633.41630699)(669.22095288,633.48630692)(669.23096529,633.55630877)
\curveto(669.24095286,633.63630677)(669.25595284,633.71630669)(669.27596529,633.79630877)
\curveto(669.27595282,633.85630655)(669.28095282,633.90130651)(669.29096529,633.93130877)
\curveto(669.29095281,633.95130646)(669.29095281,633.97630643)(669.29096529,634.00630877)
\curveto(669.29095281,634.04630636)(669.2959528,634.07630633)(669.30596529,634.09630877)
\lineto(669.30596529,634.24630877)
}
}
{
\newrgbcolor{curcolor}{0 0 0}
\pscustom[linestyle=none,fillstyle=solid,fillcolor=curcolor]
{
\newpath
\moveto(701.1124814,475.58113787)
\curveto(702.74247596,475.61112722)(703.79247491,475.05612778)(704.2624814,473.91613787)
\curveto(704.36247434,473.68612915)(704.42747428,473.39612944)(704.4574814,473.04613787)
\curveto(704.49747421,472.70613013)(704.47247423,472.39613044)(704.3824814,472.11613787)
\curveto(704.29247441,471.85613098)(704.17247453,471.6311312)(704.0224814,471.44113787)
\curveto(704.0024747,471.40113143)(703.97747473,471.36613147)(703.9474814,471.33613787)
\curveto(703.91747479,471.31613152)(703.89247481,471.29113154)(703.8724814,471.26113787)
\lineto(703.7824814,471.14113787)
\curveto(703.75247495,471.11113172)(703.71747499,471.08613175)(703.6774814,471.06613787)
\curveto(703.62747508,471.01613182)(703.57247513,470.97113186)(703.5124814,470.93113787)
\curveto(703.46247524,470.89113194)(703.41747529,470.84113199)(703.3774814,470.78113787)
\curveto(703.33747537,470.74113209)(703.32247538,470.69113214)(703.3324814,470.63113787)
\curveto(703.34247536,470.58113225)(703.37247533,470.5361323)(703.4224814,470.49613787)
\curveto(703.47247523,470.45613238)(703.52747518,470.41613242)(703.5874814,470.37613787)
\curveto(703.65747505,470.34613249)(703.72247498,470.31613252)(703.7824814,470.28613787)
\curveto(703.84247486,470.25613258)(703.89247481,470.22613261)(703.9324814,470.19613787)
\curveto(704.25247445,469.97613286)(704.5074742,469.66613317)(704.6974814,469.26613787)
\curveto(704.73747397,469.17613366)(704.76747394,469.08113375)(704.7874814,468.98113787)
\curveto(704.81747389,468.89113394)(704.84247386,468.80113403)(704.8624814,468.71113787)
\curveto(704.87247383,468.66113417)(704.87747383,468.61113422)(704.8774814,468.56113787)
\curveto(704.88747382,468.52113431)(704.89747381,468.47613436)(704.9074814,468.42613787)
\curveto(704.91747379,468.37613446)(704.91747379,468.32613451)(704.9074814,468.27613787)
\curveto(704.89747381,468.22613461)(704.9024738,468.17613466)(704.9224814,468.12613787)
\curveto(704.93247377,468.07613476)(704.93747377,468.01613482)(704.9374814,467.94613787)
\curveto(704.93747377,467.87613496)(704.92747378,467.81613502)(704.9074814,467.76613787)
\lineto(704.9074814,467.54113787)
\lineto(704.8474814,467.30113787)
\curveto(704.83747387,467.2311356)(704.82247388,467.16113567)(704.8024814,467.09113787)
\curveto(704.77247393,467.00113583)(704.74247396,466.91613592)(704.7124814,466.83613787)
\curveto(704.69247401,466.75613608)(704.66247404,466.67613616)(704.6224814,466.59613787)
\curveto(704.6024741,466.5361363)(704.57247413,466.47613636)(704.5324814,466.41613787)
\curveto(704.5024742,466.36613647)(704.46747424,466.31613652)(704.4274814,466.26613787)
\curveto(704.22747448,465.95613688)(703.97747473,465.69613714)(703.6774814,465.48613787)
\curveto(703.37747533,465.28613755)(703.03247567,465.12113771)(702.6424814,464.99113787)
\curveto(702.52247618,464.95113788)(702.39247631,464.92613791)(702.2524814,464.91613787)
\curveto(702.12247658,464.89613794)(701.98747672,464.87113796)(701.8474814,464.84113787)
\curveto(701.77747693,464.831138)(701.707477,464.82613801)(701.6374814,464.82613787)
\curveto(701.57747713,464.82613801)(701.51247719,464.82113801)(701.4424814,464.81113787)
\curveto(701.4024773,464.80113803)(701.34247736,464.79613804)(701.2624814,464.79613787)
\curveto(701.19247751,464.79613804)(701.14247756,464.80113803)(701.1124814,464.81113787)
\curveto(701.06247764,464.82113801)(701.01747769,464.82613801)(700.9774814,464.82613787)
\lineto(700.8574814,464.82613787)
\curveto(700.75747795,464.84613799)(700.65747805,464.86113797)(700.5574814,464.87113787)
\curveto(700.45747825,464.88113795)(700.36247834,464.89613794)(700.2724814,464.91613787)
\curveto(700.16247854,464.94613789)(700.05247865,464.97113786)(699.9424814,464.99113787)
\curveto(699.84247886,465.02113781)(699.73747897,465.06113777)(699.6274814,465.11113787)
\curveto(699.25747945,465.27113756)(698.94247976,465.47113736)(698.6824814,465.71113787)
\curveto(698.42248028,465.96113687)(698.21248049,466.27113656)(698.0524814,466.64113787)
\curveto(698.01248069,466.7311361)(697.97748073,466.82613601)(697.9474814,466.92613787)
\curveto(697.91748079,467.02613581)(697.88748082,467.1311357)(697.8574814,467.24113787)
\curveto(697.83748087,467.29113554)(697.82748088,467.34113549)(697.8274814,467.39113787)
\curveto(697.82748088,467.45113538)(697.81748089,467.51113532)(697.7974814,467.57113787)
\curveto(697.77748093,467.6311352)(697.76748094,467.71113512)(697.7674814,467.81113787)
\curveto(697.76748094,467.91113492)(697.78248092,467.98613485)(697.8124814,468.03613787)
\curveto(697.82248088,468.06613477)(697.83748087,468.09113474)(697.8574814,468.11113787)
\lineto(697.9174814,468.17113787)
\curveto(697.95748075,468.19113464)(698.01748069,468.20613463)(698.0974814,468.21613787)
\curveto(698.18748052,468.22613461)(698.27748043,468.2311346)(698.3674814,468.23113787)
\curveto(698.45748025,468.2311346)(698.54248016,468.22613461)(698.6224814,468.21613787)
\curveto(698.71247999,468.20613463)(698.77747993,468.19613464)(698.8174814,468.18613787)
\curveto(698.83747987,468.16613467)(698.85747985,468.15113468)(698.8774814,468.14113787)
\curveto(698.89747981,468.14113469)(698.91747979,468.1311347)(698.9374814,468.11113787)
\curveto(699.0074797,468.02113481)(699.04747966,467.90613493)(699.0574814,467.76613787)
\curveto(699.07747963,467.62613521)(699.1074796,467.50113533)(699.1474814,467.39113787)
\lineto(699.2974814,467.03113787)
\curveto(699.34747936,466.92113591)(699.41247929,466.81613602)(699.4924814,466.71613787)
\curveto(699.51247919,466.68613615)(699.53247917,466.66113617)(699.5524814,466.64113787)
\curveto(699.58247912,466.62113621)(699.6074791,466.59613624)(699.6274814,466.56613787)
\curveto(699.66747904,466.50613633)(699.702479,466.46113637)(699.7324814,466.43113787)
\curveto(699.77247893,466.40113643)(699.8074789,466.37113646)(699.8374814,466.34113787)
\curveto(699.87747883,466.31113652)(699.92247878,466.28113655)(699.9724814,466.25113787)
\curveto(700.06247864,466.19113664)(700.15747855,466.14113669)(700.2574814,466.10113787)
\lineto(700.5874814,465.98113787)
\curveto(700.73747797,465.9311369)(700.93747777,465.90113693)(701.1874814,465.89113787)
\curveto(701.43747727,465.88113695)(701.64747706,465.90113693)(701.8174814,465.95113787)
\curveto(701.89747681,465.97113686)(701.96747674,465.98613685)(702.0274814,465.99613787)
\lineto(702.2374814,466.05613787)
\curveto(702.51747619,466.17613666)(702.75747595,466.32613651)(702.9574814,466.50613787)
\curveto(703.16747554,466.68613615)(703.33247537,466.91613592)(703.4524814,467.19613787)
\curveto(703.48247522,467.26613557)(703.5024752,467.3361355)(703.5124814,467.40613787)
\lineto(703.5724814,467.64613787)
\curveto(703.61247509,467.78613505)(703.62247508,467.94613489)(703.6024814,468.12613787)
\curveto(703.58247512,468.31613452)(703.55247515,468.46613437)(703.5124814,468.57613787)
\curveto(703.38247532,468.95613388)(703.19747551,469.24613359)(702.9574814,469.44613787)
\curveto(702.72747598,469.64613319)(702.41747629,469.80613303)(702.0274814,469.92613787)
\curveto(701.91747679,469.95613288)(701.79747691,469.97613286)(701.6674814,469.98613787)
\curveto(701.54747716,469.99613284)(701.42247728,470.00113283)(701.2924814,470.00113787)
\curveto(701.13247757,470.00113283)(700.99247771,470.00613283)(700.8724814,470.01613787)
\curveto(700.75247795,470.02613281)(700.66747804,470.08613275)(700.6174814,470.19613787)
\curveto(700.59747811,470.22613261)(700.58747812,470.26113257)(700.5874814,470.30113787)
\lineto(700.5874814,470.43613787)
\curveto(700.57747813,470.5361323)(700.57747813,470.6311322)(700.5874814,470.72113787)
\curveto(700.6074781,470.81113202)(700.64747806,470.87613196)(700.7074814,470.91613787)
\curveto(700.74747796,470.94613189)(700.78747792,470.96613187)(700.8274814,470.97613787)
\curveto(700.87747783,470.98613185)(700.93247777,470.99613184)(700.9924814,471.00613787)
\curveto(701.01247769,471.01613182)(701.03747767,471.01613182)(701.0674814,471.00613787)
\curveto(701.09747761,471.00613183)(701.12247758,471.01113182)(701.1424814,471.02113787)
\lineto(701.2774814,471.02113787)
\curveto(701.38747732,471.04113179)(701.48747722,471.05113178)(701.5774814,471.05113787)
\curveto(701.67747703,471.06113177)(701.77247693,471.08113175)(701.8624814,471.11113787)
\curveto(702.18247652,471.22113161)(702.43747627,471.36613147)(702.6274814,471.54613787)
\curveto(702.81747589,471.72613111)(702.96747574,471.97613086)(703.0774814,472.29613787)
\curveto(703.1074756,472.39613044)(703.12747558,472.52113031)(703.1374814,472.67113787)
\curveto(703.15747555,472.83113)(703.15247555,472.97612986)(703.1224814,473.10613787)
\curveto(703.1024756,473.17612966)(703.08247562,473.24112959)(703.0624814,473.30113787)
\curveto(703.05247565,473.37112946)(703.03247567,473.4361294)(703.0024814,473.49613787)
\curveto(702.9024758,473.7361291)(702.75747595,473.92612891)(702.5674814,474.06613787)
\curveto(702.37747633,474.20612863)(702.15247655,474.31612852)(701.8924814,474.39613787)
\curveto(701.83247687,474.41612842)(701.77247693,474.42612841)(701.7124814,474.42613787)
\curveto(701.65247705,474.42612841)(701.58747712,474.4361284)(701.5174814,474.45613787)
\curveto(701.43747727,474.47612836)(701.34247736,474.48612835)(701.2324814,474.48613787)
\curveto(701.12247758,474.48612835)(701.02747768,474.47612836)(700.9474814,474.45613787)
\curveto(700.89747781,474.4361284)(700.84747786,474.42612841)(700.7974814,474.42613787)
\curveto(700.75747795,474.42612841)(700.71247799,474.41612842)(700.6624814,474.39613787)
\curveto(700.48247822,474.34612849)(700.31247839,474.27112856)(700.1524814,474.17113787)
\curveto(700.0024787,474.08112875)(699.87247883,473.96612887)(699.7624814,473.82613787)
\curveto(699.67247903,473.70612913)(699.59247911,473.57612926)(699.5224814,473.43613787)
\curveto(699.45247925,473.29612954)(699.38747932,473.14112969)(699.3274814,472.97113787)
\curveto(699.29747941,472.86112997)(699.27747943,472.74113009)(699.2674814,472.61113787)
\curveto(699.25747945,472.49113034)(699.22247948,472.39113044)(699.1624814,472.31113787)
\curveto(699.14247956,472.27113056)(699.08247962,472.2311306)(698.9824814,472.19113787)
\curveto(698.94247976,472.18113065)(698.88247982,472.17113066)(698.8024814,472.16113787)
\lineto(698.5474814,472.16113787)
\curveto(698.45748025,472.17113066)(698.37248033,472.18113065)(698.2924814,472.19113787)
\curveto(698.22248048,472.20113063)(698.17248053,472.21613062)(698.1424814,472.23613787)
\curveto(698.1024806,472.26613057)(698.06748064,472.32113051)(698.0374814,472.40113787)
\curveto(698.0074807,472.48113035)(698.0024807,472.56613027)(698.0224814,472.65613787)
\curveto(698.03248067,472.70613013)(698.03748067,472.75613008)(698.0374814,472.80613787)
\lineto(698.0674814,472.98613787)
\curveto(698.09748061,473.08612975)(698.12248058,473.18612965)(698.1424814,473.28613787)
\curveto(698.17248053,473.38612945)(698.2074805,473.47612936)(698.2474814,473.55613787)
\curveto(698.29748041,473.66612917)(698.34248036,473.77112906)(698.3824814,473.87113787)
\curveto(698.42248028,473.98112885)(698.47248023,474.08612875)(698.5324814,474.18613787)
\curveto(698.86247984,474.72612811)(699.33247937,475.12112771)(699.9424814,475.37113787)
\curveto(700.06247864,475.42112741)(700.18747852,475.45612738)(700.3174814,475.47613787)
\curveto(700.45747825,475.49612734)(700.59747811,475.52112731)(700.7374814,475.55113787)
\curveto(700.79747791,475.56112727)(700.85747785,475.56612727)(700.9174814,475.56613787)
\curveto(700.98747772,475.56612727)(701.05247765,475.57112726)(701.1124814,475.58113787)
}
}
{
\newrgbcolor{curcolor}{0 0 0}
\pscustom[linestyle=none,fillstyle=solid,fillcolor=curcolor]
{
\newpath
\moveto(713.34709078,469.29613787)
\curveto(713.37708305,469.17613366)(713.40208303,469.0361338)(713.42209078,468.87613787)
\curveto(713.44208299,468.71613412)(713.45208298,468.55113428)(713.45209078,468.38113787)
\curveto(713.45208298,468.21113462)(713.44208299,468.04613479)(713.42209078,467.88613787)
\curveto(713.40208303,467.72613511)(713.37708305,467.58613525)(713.34709078,467.46613787)
\curveto(713.30708312,467.32613551)(713.27208316,467.20113563)(713.24209078,467.09113787)
\curveto(713.21208322,466.98113585)(713.17208326,466.87113596)(713.12209078,466.76113787)
\curveto(712.85208358,466.12113671)(712.43708399,465.6361372)(711.87709078,465.30613787)
\curveto(711.79708463,465.24613759)(711.71208472,465.19613764)(711.62209078,465.15613787)
\curveto(711.5320849,465.12613771)(711.432085,465.09113774)(711.32209078,465.05113787)
\curveto(711.21208522,465.00113783)(711.09208534,464.96613787)(710.96209078,464.94613787)
\curveto(710.84208559,464.91613792)(710.71208572,464.88613795)(710.57209078,464.85613787)
\curveto(710.51208592,464.836138)(710.45208598,464.831138)(710.39209078,464.84113787)
\curveto(710.34208609,464.85113798)(710.28208615,464.84613799)(710.21209078,464.82613787)
\curveto(710.19208624,464.81613802)(710.16708626,464.81613802)(710.13709078,464.82613787)
\curveto(710.10708632,464.82613801)(710.08208635,464.82113801)(710.06209078,464.81113787)
\lineto(709.91209078,464.81113787)
\curveto(709.84208659,464.80113803)(709.79208664,464.80113803)(709.76209078,464.81113787)
\curveto(709.72208671,464.82113801)(709.67708675,464.82613801)(709.62709078,464.82613787)
\curveto(709.58708684,464.81613802)(709.54708688,464.81613802)(709.50709078,464.82613787)
\curveto(709.41708701,464.84613799)(709.3270871,464.86113797)(709.23709078,464.87113787)
\curveto(709.14708728,464.87113796)(709.05708737,464.88113795)(708.96709078,464.90113787)
\curveto(708.87708755,464.9311379)(708.78708764,464.95613788)(708.69709078,464.97613787)
\curveto(708.60708782,464.99613784)(708.52208791,465.02613781)(708.44209078,465.06613787)
\curveto(708.20208823,465.17613766)(707.97708845,465.30613753)(707.76709078,465.45613787)
\curveto(707.55708887,465.61613722)(707.37708905,465.79613704)(707.22709078,465.99613787)
\curveto(707.10708932,466.16613667)(707.00208943,466.34113649)(706.91209078,466.52113787)
\curveto(706.82208961,466.70113613)(706.7320897,466.89113594)(706.64209078,467.09113787)
\curveto(706.60208983,467.19113564)(706.56708986,467.29113554)(706.53709078,467.39113787)
\curveto(706.51708991,467.50113533)(706.49208994,467.61113522)(706.46209078,467.72113787)
\curveto(706.42209001,467.86113497)(706.39709003,468.00113483)(706.38709078,468.14113787)
\curveto(706.37709005,468.28113455)(706.35709007,468.42113441)(706.32709078,468.56113787)
\curveto(706.31709011,468.67113416)(706.30709012,468.77113406)(706.29709078,468.86113787)
\curveto(706.29709013,468.96113387)(706.28709014,469.06113377)(706.26709078,469.16113787)
\lineto(706.26709078,469.25113787)
\curveto(706.27709015,469.28113355)(706.27709015,469.30613353)(706.26709078,469.32613787)
\lineto(706.26709078,469.53613787)
\curveto(706.24709018,469.59613324)(706.23709019,469.66113317)(706.23709078,469.73113787)
\curveto(706.24709018,469.81113302)(706.25209018,469.88613295)(706.25209078,469.95613787)
\lineto(706.25209078,470.10613787)
\curveto(706.25209018,470.15613268)(706.25709017,470.20613263)(706.26709078,470.25613787)
\lineto(706.26709078,470.63113787)
\curveto(706.27709015,470.66113217)(706.27709015,470.69613214)(706.26709078,470.73613787)
\curveto(706.26709016,470.77613206)(706.27209016,470.81613202)(706.28209078,470.85613787)
\curveto(706.30209013,470.96613187)(706.31709011,471.07613176)(706.32709078,471.18613787)
\curveto(706.33709009,471.30613153)(706.34709008,471.42113141)(706.35709078,471.53113787)
\curveto(706.39709003,471.68113115)(706.42209001,471.82613101)(706.43209078,471.96613787)
\curveto(706.45208998,472.11613072)(706.48208995,472.26113057)(706.52209078,472.40113787)
\curveto(706.61208982,472.70113013)(706.70708972,472.98612985)(706.80709078,473.25613787)
\curveto(706.90708952,473.52612931)(707.0320894,473.77612906)(707.18209078,474.00613787)
\curveto(707.38208905,474.32612851)(707.6270888,474.60612823)(707.91709078,474.84613787)
\curveto(708.20708822,475.08612775)(708.54708788,475.27112756)(708.93709078,475.40113787)
\curveto(709.04708738,475.44112739)(709.15708727,475.46612737)(709.26709078,475.47613787)
\curveto(709.38708704,475.49612734)(709.50708692,475.52112731)(709.62709078,475.55113787)
\curveto(709.69708673,475.56112727)(709.76208667,475.56612727)(709.82209078,475.56613787)
\curveto(709.88208655,475.56612727)(709.94708648,475.57112726)(710.01709078,475.58113787)
\curveto(710.71708571,475.60112723)(711.29208514,475.48612735)(711.74209078,475.23613787)
\curveto(712.19208424,474.98612785)(712.53708389,474.6361282)(712.77709078,474.18613787)
\curveto(712.88708354,473.95612888)(712.98708344,473.68112915)(713.07709078,473.36113787)
\curveto(713.09708333,473.29112954)(713.09708333,473.21612962)(713.07709078,473.13613787)
\curveto(713.06708336,473.06612977)(713.04208339,473.01612982)(713.00209078,472.98613787)
\curveto(712.97208346,472.95612988)(712.91208352,472.9311299)(712.82209078,472.91113787)
\curveto(712.7320837,472.90112993)(712.6320838,472.89112994)(712.52209078,472.88113787)
\curveto(712.42208401,472.88112995)(712.32208411,472.88612995)(712.22209078,472.89613787)
\curveto(712.1320843,472.90612993)(712.06708436,472.92612991)(712.02709078,472.95613787)
\curveto(711.91708451,473.02612981)(711.83708459,473.1361297)(711.78709078,473.28613787)
\curveto(711.74708468,473.4361294)(711.69208474,473.56612927)(711.62209078,473.67613787)
\curveto(711.432085,473.98612885)(711.15208528,474.21612862)(710.78209078,474.36613787)
\curveto(710.71208572,474.39612844)(710.63708579,474.41612842)(710.55709078,474.42613787)
\curveto(710.48708594,474.4361284)(710.41208602,474.45112838)(710.33209078,474.47113787)
\curveto(710.28208615,474.48112835)(710.21208622,474.48612835)(710.12209078,474.48613787)
\curveto(710.04208639,474.48612835)(709.97708645,474.48112835)(709.92709078,474.47113787)
\curveto(709.88708654,474.45112838)(709.85208658,474.44612839)(709.82209078,474.45613787)
\curveto(709.79208664,474.46612837)(709.75708667,474.46612837)(709.71709078,474.45613787)
\lineto(709.47709078,474.39613787)
\curveto(709.40708702,474.37612846)(709.33708709,474.35112848)(709.26709078,474.32113787)
\curveto(708.88708754,474.16112867)(708.59708783,473.95112888)(708.39709078,473.69113787)
\curveto(708.20708822,473.4311294)(708.0320884,473.11612972)(707.87209078,472.74613787)
\curveto(707.84208859,472.66613017)(707.81708861,472.58613025)(707.79709078,472.50613787)
\curveto(707.78708864,472.42613041)(707.76708866,472.34613049)(707.73709078,472.26613787)
\curveto(707.70708872,472.15613068)(707.68208875,472.04113079)(707.66209078,471.92113787)
\curveto(707.65208878,471.80113103)(707.6320888,471.68113115)(707.60209078,471.56113787)
\curveto(707.58208885,471.51113132)(707.57208886,471.46113137)(707.57209078,471.41113787)
\curveto(707.58208885,471.36113147)(707.57708885,471.31113152)(707.55709078,471.26113787)
\curveto(707.54708888,471.20113163)(707.54708888,471.12113171)(707.55709078,471.02113787)
\curveto(707.56708886,470.9311319)(707.58208885,470.87613196)(707.60209078,470.85613787)
\curveto(707.62208881,470.81613202)(707.65208878,470.79613204)(707.69209078,470.79613787)
\curveto(707.74208869,470.79613204)(707.78708864,470.80613203)(707.82709078,470.82613787)
\curveto(707.89708853,470.86613197)(707.95708847,470.91113192)(708.00709078,470.96113787)
\curveto(708.05708837,471.01113182)(708.11708831,471.06113177)(708.18709078,471.11113787)
\lineto(708.24709078,471.17113787)
\curveto(708.27708815,471.20113163)(708.30708812,471.22613161)(708.33709078,471.24613787)
\curveto(708.56708786,471.40613143)(708.84208759,471.54113129)(709.16209078,471.65113787)
\curveto(709.2320872,471.67113116)(709.30208713,471.68613115)(709.37209078,471.69613787)
\curveto(709.44208699,471.70613113)(709.51708691,471.72113111)(709.59709078,471.74113787)
\curveto(709.63708679,471.74113109)(709.67208676,471.74613109)(709.70209078,471.75613787)
\curveto(709.7320867,471.76613107)(709.76708666,471.76613107)(709.80709078,471.75613787)
\curveto(709.85708657,471.75613108)(709.89708653,471.76613107)(709.92709078,471.78613787)
\lineto(710.09209078,471.78613787)
\lineto(710.18209078,471.78613787)
\curveto(710.2320862,471.79613104)(710.27208616,471.79613104)(710.30209078,471.78613787)
\curveto(710.35208608,471.77613106)(710.40208603,471.77113106)(710.45209078,471.77113787)
\curveto(710.51208592,471.78113105)(710.56708586,471.78113105)(710.61709078,471.77113787)
\curveto(710.7270857,471.74113109)(710.8320856,471.72113111)(710.93209078,471.71113787)
\curveto(711.04208539,471.70113113)(711.14708528,471.67613116)(711.24709078,471.63613787)
\curveto(711.66708476,471.49613134)(712.01208442,471.31113152)(712.28209078,471.08113787)
\curveto(712.55208388,470.86113197)(712.79208364,470.57613226)(713.00209078,470.22613787)
\curveto(713.08208335,470.08613275)(713.14708328,469.9361329)(713.19709078,469.77613787)
\curveto(713.24708318,469.62613321)(713.29708313,469.46613337)(713.34709078,469.29613787)
\moveto(712.10209078,467.99113787)
\curveto(712.11208432,468.04113479)(712.11708431,468.08613475)(712.11709078,468.12613787)
\lineto(712.11709078,468.27613787)
\curveto(712.11708431,468.58613425)(712.07708435,468.87113396)(711.99709078,469.13113787)
\curveto(711.97708445,469.19113364)(711.95708447,469.24613359)(711.93709078,469.29613787)
\curveto(711.9270845,469.35613348)(711.91208452,469.41113342)(711.89209078,469.46113787)
\curveto(711.67208476,469.95113288)(711.3270851,470.30113253)(710.85709078,470.51113787)
\curveto(710.77708565,470.54113229)(710.69708573,470.56613227)(710.61709078,470.58613787)
\lineto(710.37709078,470.64613787)
\curveto(710.29708613,470.66613217)(710.20708622,470.67613216)(710.10709078,470.67613787)
\lineto(709.79209078,470.67613787)
\curveto(709.77208666,470.65613218)(709.7320867,470.64613219)(709.67209078,470.64613787)
\curveto(709.62208681,470.65613218)(709.57708685,470.65613218)(709.53709078,470.64613787)
\lineto(709.29709078,470.58613787)
\curveto(709.2270872,470.57613226)(709.15708727,470.55613228)(709.08709078,470.52613787)
\curveto(708.48708794,470.26613257)(708.08208835,469.80113303)(707.87209078,469.13113787)
\curveto(707.84208859,469.05113378)(707.82208861,468.97113386)(707.81209078,468.89113787)
\curveto(707.80208863,468.81113402)(707.78708864,468.72613411)(707.76709078,468.63613787)
\lineto(707.76709078,468.48613787)
\curveto(707.75708867,468.44613439)(707.75208868,468.37613446)(707.75209078,468.27613787)
\curveto(707.75208868,468.04613479)(707.77208866,467.85113498)(707.81209078,467.69113787)
\curveto(707.8320886,467.62113521)(707.84708858,467.55613528)(707.85709078,467.49613787)
\curveto(707.86708856,467.4361354)(707.88708854,467.37113546)(707.91709078,467.30113787)
\curveto(708.0270884,467.02113581)(708.17208826,466.77613606)(708.35209078,466.56613787)
\curveto(708.5320879,466.36613647)(708.76708766,466.20613663)(709.05709078,466.08613787)
\lineto(709.29709078,465.99613787)
\lineto(709.53709078,465.93613787)
\curveto(709.58708684,465.91613692)(709.6270868,465.91113692)(709.65709078,465.92113787)
\curveto(709.69708673,465.9311369)(709.74208669,465.92613691)(709.79209078,465.90613787)
\curveto(709.82208661,465.89613694)(709.87708655,465.89113694)(709.95709078,465.89113787)
\curveto(710.03708639,465.89113694)(710.09708633,465.89613694)(710.13709078,465.90613787)
\curveto(710.24708618,465.92613691)(710.35208608,465.94113689)(710.45209078,465.95113787)
\curveto(710.55208588,465.96113687)(710.64708578,465.99113684)(710.73709078,466.04113787)
\curveto(711.26708516,466.24113659)(711.65708477,466.61613622)(711.90709078,467.16613787)
\curveto(711.94708448,467.26613557)(711.97708445,467.37113546)(711.99709078,467.48113787)
\lineto(712.08709078,467.81113787)
\curveto(712.08708434,467.89113494)(712.09208434,467.95113488)(712.10209078,467.99113787)
}
}
{
\newrgbcolor{curcolor}{0 0 0}
\pscustom[linestyle=none,fillstyle=solid,fillcolor=curcolor]
{
\newpath
\moveto(724.50170015,473.49613787)
\curveto(724.30168985,473.20612963)(724.09169006,472.92112991)(723.87170015,472.64113787)
\curveto(723.66169049,472.36113047)(723.4566907,472.07613076)(723.25670015,471.78613787)
\curveto(722.6566915,470.9361319)(722.0516921,470.09613274)(721.44170015,469.26613787)
\curveto(720.83169332,468.44613439)(720.22669393,467.61113522)(719.62670015,466.76113787)
\lineto(719.11670015,466.04113787)
\lineto(718.60670015,465.35113787)
\curveto(718.52669563,465.24113759)(718.44669571,465.12613771)(718.36670015,465.00613787)
\curveto(718.28669587,464.88613795)(718.19169596,464.79113804)(718.08170015,464.72113787)
\curveto(718.04169611,464.70113813)(717.97669618,464.68613815)(717.88670015,464.67613787)
\curveto(717.80669635,464.65613818)(717.71669644,464.64613819)(717.61670015,464.64613787)
\curveto(717.51669664,464.64613819)(717.42169673,464.65113818)(717.33170015,464.66113787)
\curveto(717.2516969,464.67113816)(717.19169696,464.69113814)(717.15170015,464.72113787)
\curveto(717.12169703,464.74113809)(717.09669706,464.77613806)(717.07670015,464.82613787)
\curveto(717.06669709,464.86613797)(717.07169708,464.91113792)(717.09170015,464.96113787)
\curveto(717.13169702,465.04113779)(717.17669698,465.11613772)(717.22670015,465.18613787)
\curveto(717.28669687,465.26613757)(717.34169681,465.34613749)(717.39170015,465.42613787)
\curveto(717.63169652,465.76613707)(717.87669628,466.10113673)(718.12670015,466.43113787)
\curveto(718.37669578,466.76113607)(718.61669554,467.09613574)(718.84670015,467.43613787)
\curveto(719.00669515,467.65613518)(719.16669499,467.87113496)(719.32670015,468.08113787)
\curveto(719.48669467,468.29113454)(719.64669451,468.50613433)(719.80670015,468.72613787)
\curveto(720.16669399,469.24613359)(720.53169362,469.75613308)(720.90170015,470.25613787)
\curveto(721.27169288,470.75613208)(721.64169251,471.26613157)(722.01170015,471.78613787)
\curveto(722.151692,471.98613085)(722.29169186,472.18113065)(722.43170015,472.37113787)
\curveto(722.58169157,472.56113027)(722.72669143,472.75613008)(722.86670015,472.95613787)
\curveto(723.07669108,473.25612958)(723.29169086,473.55612928)(723.51170015,473.85613787)
\lineto(724.17170015,474.75613787)
\lineto(724.35170015,475.02613787)
\lineto(724.56170015,475.29613787)
\lineto(724.68170015,475.47613787)
\curveto(724.73168942,475.5361273)(724.78168937,475.59112724)(724.83170015,475.64113787)
\curveto(724.90168925,475.69112714)(724.97668918,475.72612711)(725.05670015,475.74613787)
\curveto(725.07668908,475.75612708)(725.10168905,475.75612708)(725.13170015,475.74613787)
\curveto(725.17168898,475.74612709)(725.20168895,475.75612708)(725.22170015,475.77613787)
\curveto(725.34168881,475.77612706)(725.47668868,475.77112706)(725.62670015,475.76113787)
\curveto(725.77668838,475.76112707)(725.86668829,475.71612712)(725.89670015,475.62613787)
\curveto(725.91668824,475.59612724)(725.92168823,475.56112727)(725.91170015,475.52113787)
\curveto(725.90168825,475.48112735)(725.88668827,475.45112738)(725.86670015,475.43113787)
\curveto(725.82668833,475.35112748)(725.78668837,475.28112755)(725.74670015,475.22113787)
\curveto(725.70668845,475.16112767)(725.66168849,475.10112773)(725.61170015,475.04113787)
\lineto(725.04170015,474.26113787)
\curveto(724.86168929,474.01112882)(724.68168947,473.75612908)(724.50170015,473.49613787)
\moveto(717.64670015,469.59613787)
\curveto(717.59669656,469.61613322)(717.54669661,469.62113321)(717.49670015,469.61113787)
\curveto(717.44669671,469.60113323)(717.39669676,469.60613323)(717.34670015,469.62613787)
\curveto(717.23669692,469.64613319)(717.13169702,469.66613317)(717.03170015,469.68613787)
\curveto(716.94169721,469.71613312)(716.84669731,469.75613308)(716.74670015,469.80613787)
\curveto(716.41669774,469.94613289)(716.16169799,470.14113269)(715.98170015,470.39113787)
\curveto(715.80169835,470.65113218)(715.6566985,470.96113187)(715.54670015,471.32113787)
\curveto(715.51669864,471.40113143)(715.49669866,471.48113135)(715.48670015,471.56113787)
\curveto(715.47669868,471.65113118)(715.46169869,471.7361311)(715.44170015,471.81613787)
\curveto(715.43169872,471.86613097)(715.42669873,471.9311309)(715.42670015,472.01113787)
\curveto(715.41669874,472.04113079)(715.41169874,472.07113076)(715.41170015,472.10113787)
\curveto(715.41169874,472.14113069)(715.40669875,472.17613066)(715.39670015,472.20613787)
\lineto(715.39670015,472.35613787)
\curveto(715.38669877,472.40613043)(715.38169877,472.46613037)(715.38170015,472.53613787)
\curveto(715.38169877,472.61613022)(715.38669877,472.68113015)(715.39670015,472.73113787)
\lineto(715.39670015,472.89613787)
\curveto(715.41669874,472.94612989)(715.42169873,472.99112984)(715.41170015,473.03113787)
\curveto(715.41169874,473.08112975)(715.41669874,473.12612971)(715.42670015,473.16613787)
\curveto(715.43669872,473.20612963)(715.44169871,473.24112959)(715.44170015,473.27113787)
\curveto(715.44169871,473.31112952)(715.44669871,473.35112948)(715.45670015,473.39113787)
\curveto(715.48669867,473.50112933)(715.50669865,473.61112922)(715.51670015,473.72113787)
\curveto(715.53669862,473.84112899)(715.57169858,473.95612888)(715.62170015,474.06613787)
\curveto(715.76169839,474.40612843)(715.92169823,474.68112815)(716.10170015,474.89113787)
\curveto(716.29169786,475.11112772)(716.56169759,475.29112754)(716.91170015,475.43113787)
\curveto(716.99169716,475.46112737)(717.07669708,475.48112735)(717.16670015,475.49113787)
\curveto(717.2566969,475.51112732)(717.3516968,475.5311273)(717.45170015,475.55113787)
\curveto(717.48169667,475.56112727)(717.53669662,475.56112727)(717.61670015,475.55113787)
\curveto(717.69669646,475.55112728)(717.74669641,475.56112727)(717.76670015,475.58113787)
\curveto(718.32669583,475.59112724)(718.77669538,475.48112735)(719.11670015,475.25113787)
\curveto(719.46669469,475.02112781)(719.72669443,474.71612812)(719.89670015,474.33613787)
\curveto(719.93669422,474.24612859)(719.97169418,474.15112868)(720.00170015,474.05113787)
\curveto(720.03169412,473.95112888)(720.0566941,473.85112898)(720.07670015,473.75113787)
\curveto(720.09669406,473.72112911)(720.10169405,473.69112914)(720.09170015,473.66113787)
\curveto(720.09169406,473.6311292)(720.09669406,473.60112923)(720.10670015,473.57113787)
\curveto(720.13669402,473.46112937)(720.156694,473.3361295)(720.16670015,473.19613787)
\curveto(720.17669398,473.06612977)(720.18669397,472.9311299)(720.19670015,472.79113787)
\lineto(720.19670015,472.62613787)
\curveto(720.20669395,472.56613027)(720.20669395,472.51113032)(720.19670015,472.46113787)
\curveto(720.18669397,472.41113042)(720.18169397,472.36113047)(720.18170015,472.31113787)
\lineto(720.18170015,472.17613787)
\curveto(720.17169398,472.1361307)(720.16669399,472.09613074)(720.16670015,472.05613787)
\curveto(720.17669398,472.01613082)(720.17169398,471.97113086)(720.15170015,471.92113787)
\curveto(720.13169402,471.81113102)(720.11169404,471.70613113)(720.09170015,471.60613787)
\curveto(720.08169407,471.50613133)(720.06169409,471.40613143)(720.03170015,471.30613787)
\curveto(719.90169425,470.94613189)(719.73669442,470.6311322)(719.53670015,470.36113787)
\curveto(719.33669482,470.09113274)(719.06169509,469.88613295)(718.71170015,469.74613787)
\curveto(718.63169552,469.71613312)(718.54669561,469.69113314)(718.45670015,469.67113787)
\lineto(718.18670015,469.61113787)
\curveto(718.13669602,469.60113323)(718.09169606,469.59613324)(718.05170015,469.59613787)
\curveto(718.01169614,469.60613323)(717.97169618,469.60613323)(717.93170015,469.59613787)
\curveto(717.83169632,469.57613326)(717.73669642,469.57613326)(717.64670015,469.59613787)
\moveto(716.80670015,470.99113787)
\curveto(716.84669731,470.92113191)(716.88669727,470.85613198)(716.92670015,470.79613787)
\curveto(716.96669719,470.74613209)(717.01669714,470.69613214)(717.07670015,470.64613787)
\lineto(717.22670015,470.52613787)
\curveto(717.28669687,470.49613234)(717.3516968,470.47113236)(717.42170015,470.45113787)
\curveto(717.46169669,470.4311324)(717.49669666,470.42113241)(717.52670015,470.42113787)
\curveto(717.56669659,470.4311324)(717.60669655,470.42613241)(717.64670015,470.40613787)
\curveto(717.67669648,470.40613243)(717.71669644,470.40113243)(717.76670015,470.39113787)
\curveto(717.81669634,470.39113244)(717.8566963,470.39613244)(717.88670015,470.40613787)
\lineto(718.11170015,470.45113787)
\curveto(718.36169579,470.5311323)(718.54669561,470.65613218)(718.66670015,470.82613787)
\curveto(718.74669541,470.92613191)(718.81669534,471.05613178)(718.87670015,471.21613787)
\curveto(718.9566952,471.39613144)(719.01669514,471.62113121)(719.05670015,471.89113787)
\curveto(719.09669506,472.17113066)(719.11169504,472.45113038)(719.10170015,472.73113787)
\curveto(719.09169506,473.02112981)(719.06169509,473.29612954)(719.01170015,473.55613787)
\curveto(718.96169519,473.81612902)(718.88669527,474.02612881)(718.78670015,474.18613787)
\curveto(718.66669549,474.38612845)(718.51669564,474.5361283)(718.33670015,474.63613787)
\curveto(718.2566959,474.68612815)(718.16669599,474.71612812)(718.06670015,474.72613787)
\curveto(717.96669619,474.74612809)(717.86169629,474.75612808)(717.75170015,474.75613787)
\curveto(717.73169642,474.74612809)(717.70669645,474.74112809)(717.67670015,474.74113787)
\curveto(717.6566965,474.75112808)(717.63669652,474.75112808)(717.61670015,474.74113787)
\curveto(717.56669659,474.7311281)(717.52169663,474.72112811)(717.48170015,474.71113787)
\curveto(717.44169671,474.71112812)(717.40169675,474.70112813)(717.36170015,474.68113787)
\curveto(717.18169697,474.60112823)(717.03169712,474.48112835)(716.91170015,474.32113787)
\curveto(716.80169735,474.16112867)(716.71169744,473.98112885)(716.64170015,473.78113787)
\curveto(716.58169757,473.59112924)(716.53669762,473.36612947)(716.50670015,473.10613787)
\curveto(716.48669767,472.84612999)(716.48169767,472.58113025)(716.49170015,472.31113787)
\curveto(716.50169765,472.05113078)(716.53169762,471.80113103)(716.58170015,471.56113787)
\curveto(716.64169751,471.3311315)(716.71669744,471.14113169)(716.80670015,470.99113787)
\moveto(727.60670015,468.00613787)
\curveto(727.61668654,467.95613488)(727.62168653,467.86613497)(727.62170015,467.73613787)
\curveto(727.62168653,467.60613523)(727.61168654,467.51613532)(727.59170015,467.46613787)
\curveto(727.57168658,467.41613542)(727.56668659,467.36113547)(727.57670015,467.30113787)
\curveto(727.58668657,467.25113558)(727.58668657,467.20113563)(727.57670015,467.15113787)
\curveto(727.53668662,467.01113582)(727.50668665,466.87613596)(727.48670015,466.74613787)
\curveto(727.47668668,466.61613622)(727.44668671,466.49613634)(727.39670015,466.38613787)
\curveto(727.2566869,466.0361368)(727.09168706,465.74113709)(726.90170015,465.50113787)
\curveto(726.71168744,465.27113756)(726.44168771,465.08613775)(726.09170015,464.94613787)
\curveto(726.01168814,464.91613792)(725.92668823,464.89613794)(725.83670015,464.88613787)
\curveto(725.74668841,464.86613797)(725.66168849,464.84613799)(725.58170015,464.82613787)
\curveto(725.53168862,464.81613802)(725.48168867,464.81113802)(725.43170015,464.81113787)
\curveto(725.38168877,464.81113802)(725.33168882,464.80613803)(725.28170015,464.79613787)
\curveto(725.2516889,464.78613805)(725.20168895,464.78613805)(725.13170015,464.79613787)
\curveto(725.06168909,464.79613804)(725.01168914,464.80113803)(724.98170015,464.81113787)
\curveto(724.92168923,464.831138)(724.86168929,464.84113799)(724.80170015,464.84113787)
\curveto(724.7516894,464.831138)(724.70168945,464.836138)(724.65170015,464.85613787)
\curveto(724.56168959,464.87613796)(724.47168968,464.90113793)(724.38170015,464.93113787)
\curveto(724.30168985,464.95113788)(724.22168993,464.98113785)(724.14170015,465.02113787)
\curveto(723.82169033,465.16113767)(723.57169058,465.35613748)(723.39170015,465.60613787)
\curveto(723.21169094,465.86613697)(723.06169109,466.17113666)(722.94170015,466.52113787)
\curveto(722.92169123,466.60113623)(722.90669125,466.68613615)(722.89670015,466.77613787)
\curveto(722.88669127,466.86613597)(722.87169128,466.95113588)(722.85170015,467.03113787)
\curveto(722.84169131,467.06113577)(722.83669132,467.09113574)(722.83670015,467.12113787)
\lineto(722.83670015,467.22613787)
\curveto(722.81669134,467.30613553)(722.80669135,467.38613545)(722.80670015,467.46613787)
\lineto(722.80670015,467.60113787)
\curveto(722.78669137,467.70113513)(722.78669137,467.80113503)(722.80670015,467.90113787)
\lineto(722.80670015,468.08113787)
\curveto(722.81669134,468.1311347)(722.82169133,468.17613466)(722.82170015,468.21613787)
\curveto(722.82169133,468.26613457)(722.82669133,468.31113452)(722.83670015,468.35113787)
\curveto(722.84669131,468.39113444)(722.8516913,468.42613441)(722.85170015,468.45613787)
\curveto(722.8516913,468.49613434)(722.8566913,468.5361343)(722.86670015,468.57613787)
\lineto(722.92670015,468.90613787)
\curveto(722.94669121,469.02613381)(722.97669118,469.1361337)(723.01670015,469.23613787)
\curveto(723.156691,469.56613327)(723.31669084,469.84113299)(723.49670015,470.06113787)
\curveto(723.68669047,470.29113254)(723.94669021,470.47613236)(724.27670015,470.61613787)
\curveto(724.3566898,470.65613218)(724.44168971,470.68113215)(724.53170015,470.69113787)
\lineto(724.83170015,470.75113787)
\lineto(724.96670015,470.75113787)
\curveto(725.01668914,470.76113207)(725.06668909,470.76613207)(725.11670015,470.76613787)
\curveto(725.68668847,470.78613205)(726.14668801,470.68113215)(726.49670015,470.45113787)
\curveto(726.8566873,470.2311326)(727.12168703,469.9311329)(727.29170015,469.55113787)
\curveto(727.34168681,469.45113338)(727.38168677,469.35113348)(727.41170015,469.25113787)
\curveto(727.44168671,469.15113368)(727.47168668,469.04613379)(727.50170015,468.93613787)
\curveto(727.51168664,468.89613394)(727.51668664,468.86113397)(727.51670015,468.83113787)
\curveto(727.51668664,468.81113402)(727.52168663,468.78113405)(727.53170015,468.74113787)
\curveto(727.5516866,468.67113416)(727.56168659,468.59613424)(727.56170015,468.51613787)
\curveto(727.56168659,468.4361344)(727.57168658,468.35613448)(727.59170015,468.27613787)
\curveto(727.59168656,468.22613461)(727.59168656,468.18113465)(727.59170015,468.14113787)
\curveto(727.59168656,468.10113473)(727.59668656,468.05613478)(727.60670015,468.00613787)
\moveto(726.49670015,467.57113787)
\curveto(726.50668765,467.62113521)(726.51168764,467.69613514)(726.51170015,467.79613787)
\curveto(726.52168763,467.89613494)(726.51668764,467.97113486)(726.49670015,468.02113787)
\curveto(726.47668768,468.08113475)(726.47168768,468.1361347)(726.48170015,468.18613787)
\curveto(726.50168765,468.24613459)(726.50168765,468.30613453)(726.48170015,468.36613787)
\curveto(726.47168768,468.39613444)(726.46668769,468.4311344)(726.46670015,468.47113787)
\curveto(726.46668769,468.51113432)(726.46168769,468.55113428)(726.45170015,468.59113787)
\curveto(726.43168772,468.67113416)(726.41168774,468.74613409)(726.39170015,468.81613787)
\curveto(726.38168777,468.89613394)(726.36668779,468.97613386)(726.34670015,469.05613787)
\curveto(726.31668784,469.11613372)(726.29168786,469.17613366)(726.27170015,469.23613787)
\curveto(726.2516879,469.29613354)(726.22168793,469.35613348)(726.18170015,469.41613787)
\curveto(726.08168807,469.58613325)(725.9516882,469.72113311)(725.79170015,469.82113787)
\curveto(725.71168844,469.87113296)(725.61668854,469.90613293)(725.50670015,469.92613787)
\curveto(725.39668876,469.94613289)(725.27168888,469.95613288)(725.13170015,469.95613787)
\curveto(725.11168904,469.94613289)(725.08668907,469.94113289)(725.05670015,469.94113787)
\curveto(725.02668913,469.95113288)(724.99668916,469.95113288)(724.96670015,469.94113787)
\lineto(724.81670015,469.88113787)
\curveto(724.76668939,469.87113296)(724.72168943,469.85613298)(724.68170015,469.83613787)
\curveto(724.49168966,469.72613311)(724.34668981,469.58113325)(724.24670015,469.40113787)
\curveto(724.15669,469.22113361)(724.07669008,469.01613382)(724.00670015,468.78613787)
\curveto(723.96669019,468.65613418)(723.94669021,468.52113431)(723.94670015,468.38113787)
\curveto(723.94669021,468.25113458)(723.93669022,468.10613473)(723.91670015,467.94613787)
\curveto(723.90669025,467.89613494)(723.89669026,467.836135)(723.88670015,467.76613787)
\curveto(723.88669027,467.69613514)(723.89669026,467.6361352)(723.91670015,467.58613787)
\lineto(723.91670015,467.42113787)
\lineto(723.91670015,467.24113787)
\curveto(723.92669023,467.19113564)(723.93669022,467.1361357)(723.94670015,467.07613787)
\curveto(723.9566902,467.02613581)(723.96169019,466.97113586)(723.96170015,466.91113787)
\curveto(723.97169018,466.85113598)(723.98669017,466.79613604)(724.00670015,466.74613787)
\curveto(724.0566901,466.55613628)(724.11669004,466.38113645)(724.18670015,466.22113787)
\curveto(724.2566899,466.06113677)(724.36168979,465.9311369)(724.50170015,465.83113787)
\curveto(724.63168952,465.7311371)(724.77168938,465.66113717)(724.92170015,465.62113787)
\curveto(724.9516892,465.61113722)(724.97668918,465.60613723)(724.99670015,465.60613787)
\curveto(725.02668913,465.61613722)(725.0566891,465.61613722)(725.08670015,465.60613787)
\curveto(725.10668905,465.60613723)(725.13668902,465.60113723)(725.17670015,465.59113787)
\curveto(725.21668894,465.59113724)(725.2516889,465.59613724)(725.28170015,465.60613787)
\curveto(725.32168883,465.61613722)(725.36168879,465.62113721)(725.40170015,465.62113787)
\curveto(725.44168871,465.62113721)(725.48168867,465.6311372)(725.52170015,465.65113787)
\curveto(725.76168839,465.7311371)(725.9566882,465.86613697)(726.10670015,466.05613787)
\curveto(726.22668793,466.2361366)(726.31668784,466.44113639)(726.37670015,466.67113787)
\curveto(726.39668776,466.74113609)(726.41168774,466.81113602)(726.42170015,466.88113787)
\curveto(726.43168772,466.96113587)(726.44668771,467.04113579)(726.46670015,467.12113787)
\curveto(726.46668769,467.18113565)(726.47168768,467.22613561)(726.48170015,467.25613787)
\curveto(726.48168767,467.27613556)(726.48168767,467.30113553)(726.48170015,467.33113787)
\curveto(726.48168767,467.37113546)(726.48668767,467.40113543)(726.49670015,467.42113787)
\lineto(726.49670015,467.57113787)
}
}
{
\newrgbcolor{curcolor}{0 0 0}
\pscustom[linestyle=none,fillstyle=solid,fillcolor=curcolor]
{
\newpath
\moveto(479.61347383,529.08032977)
\curveto(479.68346619,529.03032631)(479.72346615,528.96032638)(479.73347383,528.87032977)
\curveto(479.75346612,528.78032656)(479.76346611,528.67532666)(479.76347383,528.55532977)
\curveto(479.76346611,528.50532683)(479.75846611,528.45532688)(479.74847383,528.40532977)
\curveto(479.74846612,528.35532698)(479.73846613,528.31032703)(479.71847383,528.27032977)
\curveto(479.68846618,528.18032716)(479.62846624,528.12032722)(479.53847383,528.09032977)
\curveto(479.45846641,528.07032727)(479.36346651,528.06032728)(479.25347383,528.06032977)
\lineto(478.93847383,528.06032977)
\curveto(478.82846704,528.07032727)(478.72346715,528.06032728)(478.62347383,528.03032977)
\curveto(478.48346739,528.00032734)(478.39346748,527.92032742)(478.35347383,527.79032977)
\curveto(478.33346754,527.72032762)(478.32346755,527.6353277)(478.32347383,527.53532977)
\lineto(478.32347383,527.26532977)
\lineto(478.32347383,526.32032977)
\lineto(478.32347383,525.99032977)
\curveto(478.32346755,525.88032946)(478.30346757,525.79532954)(478.26347383,525.73532977)
\curveto(478.22346765,525.67532966)(478.1734677,525.6353297)(478.11347383,525.61532977)
\curveto(478.06346781,525.60532973)(477.99846787,525.59032975)(477.91847383,525.57032977)
\lineto(477.72347383,525.57032977)
\curveto(477.60346827,525.57032977)(477.49846837,525.57532976)(477.40847383,525.58532977)
\curveto(477.31846855,525.60532973)(477.24846862,525.65532968)(477.19847383,525.73532977)
\curveto(477.1684687,525.78532955)(477.15346872,525.85532948)(477.15347383,525.94532977)
\lineto(477.15347383,526.24532977)
\lineto(477.15347383,527.28032977)
\curveto(477.15346872,527.4403279)(477.14346873,527.58532775)(477.12347383,527.71532977)
\curveto(477.11346876,527.85532748)(477.05846881,527.95032739)(476.95847383,528.00032977)
\curveto(476.90846896,528.02032732)(476.83846903,528.0353273)(476.74847383,528.04532977)
\curveto(476.6684692,528.05532728)(476.57846929,528.06032728)(476.47847383,528.06032977)
\lineto(476.19347383,528.06032977)
\lineto(475.95347383,528.06032977)
\lineto(473.68847383,528.06032977)
\curveto(473.59847227,528.06032728)(473.49347238,528.05532728)(473.37347383,528.04532977)
\lineto(473.04347383,528.04532977)
\curveto(472.93347294,528.04532729)(472.83347304,528.05532728)(472.74347383,528.07532977)
\curveto(472.65347322,528.09532724)(472.59347328,528.13032721)(472.56347383,528.18032977)
\curveto(472.51347336,528.25032709)(472.48847338,528.34532699)(472.48847383,528.46532977)
\lineto(472.48847383,528.81032977)
\lineto(472.48847383,529.08032977)
\curveto(472.52847334,529.25032609)(472.58347329,529.39032595)(472.65347383,529.50032977)
\curveto(472.72347315,529.61032573)(472.80347307,529.72532561)(472.89347383,529.84532977)
\lineto(473.25347383,530.38532977)
\curveto(473.69347218,531.01532432)(474.12847174,531.6353237)(474.55847383,532.24532977)
\lineto(475.87847383,534.10532977)
\curveto(476.03846983,534.335321)(476.19346968,534.55532078)(476.34347383,534.76532977)
\curveto(476.49346938,534.98532035)(476.64846922,535.21032013)(476.80847383,535.44032977)
\curveto(476.85846901,535.51031983)(476.90846896,535.57531976)(476.95847383,535.63532977)
\curveto(477.00846886,535.70531963)(477.05846881,535.78031956)(477.10847383,535.86032977)
\lineto(477.16847383,535.95032977)
\curveto(477.19846867,535.99031935)(477.22846864,536.02031932)(477.25847383,536.04032977)
\curveto(477.29846857,536.07031927)(477.33846853,536.09031925)(477.37847383,536.10032977)
\curveto(477.41846845,536.12031922)(477.46346841,536.1403192)(477.51347383,536.16032977)
\curveto(477.53346834,536.16031918)(477.55346832,536.15531918)(477.57347383,536.14532977)
\curveto(477.60346827,536.14531919)(477.62846824,536.15531918)(477.64847383,536.17532977)
\curveto(477.77846809,536.17531916)(477.89846797,536.17031917)(478.00847383,536.16032977)
\curveto(478.11846775,536.15031919)(478.19846767,536.10531923)(478.24847383,536.02532977)
\curveto(478.28846758,535.97531936)(478.30846756,535.90531943)(478.30847383,535.81532977)
\curveto(478.31846755,535.72531961)(478.32346755,535.63031971)(478.32347383,535.53032977)
\lineto(478.32347383,530.07032977)
\curveto(478.32346755,530.00032534)(478.31846755,529.92532541)(478.30847383,529.84532977)
\curveto(478.30846756,529.77532556)(478.31346756,529.70532563)(478.32347383,529.63532977)
\lineto(478.32347383,529.53032977)
\curveto(478.34346753,529.48032586)(478.35846751,529.42532591)(478.36847383,529.36532977)
\curveto(478.37846749,529.31532602)(478.40346747,529.27532606)(478.44347383,529.24532977)
\curveto(478.51346736,529.19532614)(478.59846727,529.16532617)(478.69847383,529.15532977)
\lineto(479.02847383,529.15532977)
\curveto(479.13846673,529.15532618)(479.24346663,529.15032619)(479.34347383,529.14032977)
\curveto(479.45346642,529.1403262)(479.54346633,529.12032622)(479.61347383,529.08032977)
\moveto(477.04847383,529.27532977)
\curveto(477.12846874,529.38532595)(477.16346871,529.55532578)(477.15347383,529.78532977)
\lineto(477.15347383,530.40032977)
\lineto(477.15347383,532.87532977)
\lineto(477.15347383,533.19032977)
\curveto(477.16346871,533.31032203)(477.15846871,533.41032193)(477.13847383,533.49032977)
\lineto(477.13847383,533.64032977)
\curveto(477.13846873,533.73032161)(477.12346875,533.81532152)(477.09347383,533.89532977)
\curveto(477.08346879,533.91532142)(477.0734688,533.92532141)(477.06347383,533.92532977)
\lineto(477.01847383,533.97032977)
\curveto(476.99846887,533.98032136)(476.9684689,533.98532135)(476.92847383,533.98532977)
\curveto(476.90846896,533.96532137)(476.88846898,533.95032139)(476.86847383,533.94032977)
\curveto(476.85846901,533.9403214)(476.84346903,533.9353214)(476.82347383,533.92532977)
\curveto(476.76346911,533.87532146)(476.70346917,533.80532153)(476.64347383,533.71532977)
\curveto(476.58346929,533.62532171)(476.52846934,533.54532179)(476.47847383,533.47532977)
\curveto(476.37846949,533.335322)(476.28346959,533.19032215)(476.19347383,533.04032977)
\curveto(476.10346977,532.90032244)(476.00846986,532.76032258)(475.90847383,532.62032977)
\lineto(475.36847383,531.84032977)
\curveto(475.19847067,531.58032376)(475.02347085,531.32032402)(474.84347383,531.06032977)
\curveto(474.76347111,530.95032439)(474.68847118,530.84532449)(474.61847383,530.74532977)
\lineto(474.40847383,530.44532977)
\curveto(474.35847151,530.36532497)(474.30847156,530.29032505)(474.25847383,530.22032977)
\curveto(474.21847165,530.15032519)(474.1734717,530.07532526)(474.12347383,529.99532977)
\curveto(474.0734718,529.9353254)(474.02347185,529.87032547)(473.97347383,529.80032977)
\curveto(473.93347194,529.7403256)(473.89347198,529.67032567)(473.85347383,529.59032977)
\curveto(473.81347206,529.53032581)(473.78847208,529.46032588)(473.77847383,529.38032977)
\curveto(473.7684721,529.31032603)(473.80347207,529.25532608)(473.88347383,529.21532977)
\curveto(473.95347192,529.16532617)(474.06347181,529.1403262)(474.21347383,529.14032977)
\curveto(474.3734715,529.15032619)(474.50847136,529.15532618)(474.61847383,529.15532977)
\lineto(476.29847383,529.15532977)
\lineto(476.73347383,529.15532977)
\curveto(476.88346899,529.15532618)(476.98846888,529.19532614)(477.04847383,529.27532977)
}
}
{
\newrgbcolor{curcolor}{0 0 0}
\pscustom[linestyle=none,fillstyle=solid,fillcolor=curcolor]
{
\newpath
\moveto(488.08308321,529.90532977)
\curveto(488.11307548,529.78532555)(488.13807546,529.64532569)(488.15808321,529.48532977)
\curveto(488.17807542,529.32532601)(488.18807541,529.16032618)(488.18808321,528.99032977)
\curveto(488.18807541,528.82032652)(488.17807542,528.65532668)(488.15808321,528.49532977)
\curveto(488.13807546,528.335327)(488.11307548,528.19532714)(488.08308321,528.07532977)
\curveto(488.04307555,527.9353274)(488.00807559,527.81032753)(487.97808321,527.70032977)
\curveto(487.94807565,527.59032775)(487.90807569,527.48032786)(487.85808321,527.37032977)
\curveto(487.58807601,526.73032861)(487.17307642,526.24532909)(486.61308321,525.91532977)
\curveto(486.53307706,525.85532948)(486.44807715,525.80532953)(486.35808321,525.76532977)
\curveto(486.26807733,525.7353296)(486.16807743,525.70032964)(486.05808321,525.66032977)
\curveto(485.94807765,525.61032973)(485.82807777,525.57532976)(485.69808321,525.55532977)
\curveto(485.57807802,525.52532981)(485.44807815,525.49532984)(485.30808321,525.46532977)
\curveto(485.24807835,525.44532989)(485.18807841,525.4403299)(485.12808321,525.45032977)
\curveto(485.07807852,525.46032988)(485.01807858,525.45532988)(484.94808321,525.43532977)
\curveto(484.92807867,525.42532991)(484.90307869,525.42532991)(484.87308321,525.43532977)
\curveto(484.84307875,525.4353299)(484.81807878,525.43032991)(484.79808321,525.42032977)
\lineto(484.64808321,525.42032977)
\curveto(484.57807902,525.41032993)(484.52807907,525.41032993)(484.49808321,525.42032977)
\curveto(484.45807914,525.43032991)(484.41307918,525.4353299)(484.36308321,525.43532977)
\curveto(484.32307927,525.42532991)(484.28307931,525.42532991)(484.24308321,525.43532977)
\curveto(484.15307944,525.45532988)(484.06307953,525.47032987)(483.97308321,525.48032977)
\curveto(483.88307971,525.48032986)(483.7930798,525.49032985)(483.70308321,525.51032977)
\curveto(483.61307998,525.5403298)(483.52308007,525.56532977)(483.43308321,525.58532977)
\curveto(483.34308025,525.60532973)(483.25808034,525.6353297)(483.17808321,525.67532977)
\curveto(482.93808066,525.78532955)(482.71308088,525.91532942)(482.50308321,526.06532977)
\curveto(482.2930813,526.22532911)(482.11308148,526.40532893)(481.96308321,526.60532977)
\curveto(481.84308175,526.77532856)(481.73808186,526.95032839)(481.64808321,527.13032977)
\curveto(481.55808204,527.31032803)(481.46808213,527.50032784)(481.37808321,527.70032977)
\curveto(481.33808226,527.80032754)(481.30308229,527.90032744)(481.27308321,528.00032977)
\curveto(481.25308234,528.11032723)(481.22808237,528.22032712)(481.19808321,528.33032977)
\curveto(481.15808244,528.47032687)(481.13308246,528.61032673)(481.12308321,528.75032977)
\curveto(481.11308248,528.89032645)(481.0930825,529.03032631)(481.06308321,529.17032977)
\curveto(481.05308254,529.28032606)(481.04308255,529.38032596)(481.03308321,529.47032977)
\curveto(481.03308256,529.57032577)(481.02308257,529.67032567)(481.00308321,529.77032977)
\lineto(481.00308321,529.86032977)
\curveto(481.01308258,529.89032545)(481.01308258,529.91532542)(481.00308321,529.93532977)
\lineto(481.00308321,530.14532977)
\curveto(480.98308261,530.20532513)(480.97308262,530.27032507)(480.97308321,530.34032977)
\curveto(480.98308261,530.42032492)(480.98808261,530.49532484)(480.98808321,530.56532977)
\lineto(480.98808321,530.71532977)
\curveto(480.98808261,530.76532457)(480.9930826,530.81532452)(481.00308321,530.86532977)
\lineto(481.00308321,531.24032977)
\curveto(481.01308258,531.27032407)(481.01308258,531.30532403)(481.00308321,531.34532977)
\curveto(481.00308259,531.38532395)(481.00808259,531.42532391)(481.01808321,531.46532977)
\curveto(481.03808256,531.57532376)(481.05308254,531.68532365)(481.06308321,531.79532977)
\curveto(481.07308252,531.91532342)(481.08308251,532.03032331)(481.09308321,532.14032977)
\curveto(481.13308246,532.29032305)(481.15808244,532.4353229)(481.16808321,532.57532977)
\curveto(481.18808241,532.72532261)(481.21808238,532.87032247)(481.25808321,533.01032977)
\curveto(481.34808225,533.31032203)(481.44308215,533.59532174)(481.54308321,533.86532977)
\curveto(481.64308195,534.1353212)(481.76808183,534.38532095)(481.91808321,534.61532977)
\curveto(482.11808148,534.9353204)(482.36308123,535.21532012)(482.65308321,535.45532977)
\curveto(482.94308065,535.69531964)(483.28308031,535.88031946)(483.67308321,536.01032977)
\curveto(483.78307981,536.05031929)(483.8930797,536.07531926)(484.00308321,536.08532977)
\curveto(484.12307947,536.10531923)(484.24307935,536.13031921)(484.36308321,536.16032977)
\curveto(484.43307916,536.17031917)(484.4980791,536.17531916)(484.55808321,536.17532977)
\curveto(484.61807898,536.17531916)(484.68307891,536.18031916)(484.75308321,536.19032977)
\curveto(485.45307814,536.21031913)(486.02807757,536.09531924)(486.47808321,535.84532977)
\curveto(486.92807667,535.59531974)(487.27307632,535.24532009)(487.51308321,534.79532977)
\curveto(487.62307597,534.56532077)(487.72307587,534.29032105)(487.81308321,533.97032977)
\curveto(487.83307576,533.90032144)(487.83307576,533.82532151)(487.81308321,533.74532977)
\curveto(487.80307579,533.67532166)(487.77807582,533.62532171)(487.73808321,533.59532977)
\curveto(487.70807589,533.56532177)(487.64807595,533.5403218)(487.55808321,533.52032977)
\curveto(487.46807613,533.51032183)(487.36807623,533.50032184)(487.25808321,533.49032977)
\curveto(487.15807644,533.49032185)(487.05807654,533.49532184)(486.95808321,533.50532977)
\curveto(486.86807673,533.51532182)(486.80307679,533.5353218)(486.76308321,533.56532977)
\curveto(486.65307694,533.6353217)(486.57307702,533.74532159)(486.52308321,533.89532977)
\curveto(486.48307711,534.04532129)(486.42807717,534.17532116)(486.35808321,534.28532977)
\curveto(486.16807743,534.59532074)(485.88807771,534.82532051)(485.51808321,534.97532977)
\curveto(485.44807815,535.00532033)(485.37307822,535.02532031)(485.29308321,535.03532977)
\curveto(485.22307837,535.04532029)(485.14807845,535.06032028)(485.06808321,535.08032977)
\curveto(485.01807858,535.09032025)(484.94807865,535.09532024)(484.85808321,535.09532977)
\curveto(484.77807882,535.09532024)(484.71307888,535.09032025)(484.66308321,535.08032977)
\curveto(484.62307897,535.06032028)(484.58807901,535.05532028)(484.55808321,535.06532977)
\curveto(484.52807907,535.07532026)(484.4930791,535.07532026)(484.45308321,535.06532977)
\lineto(484.21308321,535.00532977)
\curveto(484.14307945,534.98532035)(484.07307952,534.96032038)(484.00308321,534.93032977)
\curveto(483.62307997,534.77032057)(483.33308026,534.56032078)(483.13308321,534.30032977)
\curveto(482.94308065,534.0403213)(482.76808083,533.72532161)(482.60808321,533.35532977)
\curveto(482.57808102,533.27532206)(482.55308104,533.19532214)(482.53308321,533.11532977)
\curveto(482.52308107,533.0353223)(482.50308109,532.95532238)(482.47308321,532.87532977)
\curveto(482.44308115,532.76532257)(482.41808118,532.65032269)(482.39808321,532.53032977)
\curveto(482.38808121,532.41032293)(482.36808123,532.29032305)(482.33808321,532.17032977)
\curveto(482.31808128,532.12032322)(482.30808129,532.07032327)(482.30808321,532.02032977)
\curveto(482.31808128,531.97032337)(482.31308128,531.92032342)(482.29308321,531.87032977)
\curveto(482.28308131,531.81032353)(482.28308131,531.73032361)(482.29308321,531.63032977)
\curveto(482.30308129,531.5403238)(482.31808128,531.48532385)(482.33808321,531.46532977)
\curveto(482.35808124,531.42532391)(482.38808121,531.40532393)(482.42808321,531.40532977)
\curveto(482.47808112,531.40532393)(482.52308107,531.41532392)(482.56308321,531.43532977)
\curveto(482.63308096,531.47532386)(482.6930809,531.52032382)(482.74308321,531.57032977)
\curveto(482.7930808,531.62032372)(482.85308074,531.67032367)(482.92308321,531.72032977)
\lineto(482.98308321,531.78032977)
\curveto(483.01308058,531.81032353)(483.04308055,531.8353235)(483.07308321,531.85532977)
\curveto(483.30308029,532.01532332)(483.57808002,532.15032319)(483.89808321,532.26032977)
\curveto(483.96807963,532.28032306)(484.03807956,532.29532304)(484.10808321,532.30532977)
\curveto(484.17807942,532.31532302)(484.25307934,532.33032301)(484.33308321,532.35032977)
\curveto(484.37307922,532.35032299)(484.40807919,532.35532298)(484.43808321,532.36532977)
\curveto(484.46807913,532.37532296)(484.50307909,532.37532296)(484.54308321,532.36532977)
\curveto(484.593079,532.36532297)(484.63307896,532.37532296)(484.66308321,532.39532977)
\lineto(484.82808321,532.39532977)
\lineto(484.91808321,532.39532977)
\curveto(484.96807863,532.40532293)(485.00807859,532.40532293)(485.03808321,532.39532977)
\curveto(485.08807851,532.38532295)(485.13807846,532.38032296)(485.18808321,532.38032977)
\curveto(485.24807835,532.39032295)(485.30307829,532.39032295)(485.35308321,532.38032977)
\curveto(485.46307813,532.35032299)(485.56807803,532.33032301)(485.66808321,532.32032977)
\curveto(485.77807782,532.31032303)(485.88307771,532.28532305)(485.98308321,532.24532977)
\curveto(486.40307719,532.10532323)(486.74807685,531.92032342)(487.01808321,531.69032977)
\curveto(487.28807631,531.47032387)(487.52807607,531.18532415)(487.73808321,530.83532977)
\curveto(487.81807578,530.69532464)(487.88307571,530.54532479)(487.93308321,530.38532977)
\curveto(487.98307561,530.2353251)(488.03307556,530.07532526)(488.08308321,529.90532977)
\moveto(486.83808321,528.60032977)
\curveto(486.84807675,528.65032669)(486.85307674,528.69532664)(486.85308321,528.73532977)
\lineto(486.85308321,528.88532977)
\curveto(486.85307674,529.19532614)(486.81307678,529.48032586)(486.73308321,529.74032977)
\curveto(486.71307688,529.80032554)(486.6930769,529.85532548)(486.67308321,529.90532977)
\curveto(486.66307693,529.96532537)(486.64807695,530.02032532)(486.62808321,530.07032977)
\curveto(486.40807719,530.56032478)(486.06307753,530.91032443)(485.59308321,531.12032977)
\curveto(485.51307808,531.15032419)(485.43307816,531.17532416)(485.35308321,531.19532977)
\lineto(485.11308321,531.25532977)
\curveto(485.03307856,531.27532406)(484.94307865,531.28532405)(484.84308321,531.28532977)
\lineto(484.52808321,531.28532977)
\curveto(484.50807909,531.26532407)(484.46807913,531.25532408)(484.40808321,531.25532977)
\curveto(484.35807924,531.26532407)(484.31307928,531.26532407)(484.27308321,531.25532977)
\lineto(484.03308321,531.19532977)
\curveto(483.96307963,531.18532415)(483.8930797,531.16532417)(483.82308321,531.13532977)
\curveto(483.22308037,530.87532446)(482.81808078,530.41032493)(482.60808321,529.74032977)
\curveto(482.57808102,529.66032568)(482.55808104,529.58032576)(482.54808321,529.50032977)
\curveto(482.53808106,529.42032592)(482.52308107,529.335326)(482.50308321,529.24532977)
\lineto(482.50308321,529.09532977)
\curveto(482.4930811,529.05532628)(482.48808111,528.98532635)(482.48808321,528.88532977)
\curveto(482.48808111,528.65532668)(482.50808109,528.46032688)(482.54808321,528.30032977)
\curveto(482.56808103,528.23032711)(482.58308101,528.16532717)(482.59308321,528.10532977)
\curveto(482.60308099,528.04532729)(482.62308097,527.98032736)(482.65308321,527.91032977)
\curveto(482.76308083,527.63032771)(482.90808069,527.38532795)(483.08808321,527.17532977)
\curveto(483.26808033,526.97532836)(483.50308009,526.81532852)(483.79308321,526.69532977)
\lineto(484.03308321,526.60532977)
\lineto(484.27308321,526.54532977)
\curveto(484.32307927,526.52532881)(484.36307923,526.52032882)(484.39308321,526.53032977)
\curveto(484.43307916,526.5403288)(484.47807912,526.5353288)(484.52808321,526.51532977)
\curveto(484.55807904,526.50532883)(484.61307898,526.50032884)(484.69308321,526.50032977)
\curveto(484.77307882,526.50032884)(484.83307876,526.50532883)(484.87308321,526.51532977)
\curveto(484.98307861,526.5353288)(485.08807851,526.55032879)(485.18808321,526.56032977)
\curveto(485.28807831,526.57032877)(485.38307821,526.60032874)(485.47308321,526.65032977)
\curveto(486.00307759,526.85032849)(486.3930772,527.22532811)(486.64308321,527.77532977)
\curveto(486.68307691,527.87532746)(486.71307688,527.98032736)(486.73308321,528.09032977)
\lineto(486.82308321,528.42032977)
\curveto(486.82307677,528.50032684)(486.82807677,528.56032678)(486.83808321,528.60032977)
}
}
{
\newrgbcolor{curcolor}{0 0 0}
\pscustom[linestyle=none,fillstyle=solid,fillcolor=curcolor]
{
\newpath
\moveto(490.38769258,527.22032977)
\lineto(490.68769258,527.22032977)
\curveto(490.79769052,527.23032811)(490.90269042,527.23032811)(491.00269258,527.22032977)
\curveto(491.11269021,527.22032812)(491.21269011,527.21032813)(491.30269258,527.19032977)
\curveto(491.39268993,527.18032816)(491.46268986,527.15532818)(491.51269258,527.11532977)
\curveto(491.53268979,527.09532824)(491.54768977,527.06532827)(491.55769258,527.02532977)
\curveto(491.57768974,526.98532835)(491.59768972,526.9403284)(491.61769258,526.89032977)
\lineto(491.61769258,526.81532977)
\curveto(491.62768969,526.76532857)(491.62768969,526.71032863)(491.61769258,526.65032977)
\lineto(491.61769258,526.50032977)
\lineto(491.61769258,526.02032977)
\curveto(491.6176897,525.85032949)(491.57768974,525.73032961)(491.49769258,525.66032977)
\curveto(491.42768989,525.61032973)(491.33768998,525.58532975)(491.22769258,525.58532977)
\lineto(490.89769258,525.58532977)
\lineto(490.44769258,525.58532977)
\curveto(490.29769102,525.58532975)(490.18269114,525.61532972)(490.10269258,525.67532977)
\curveto(490.06269126,525.70532963)(490.03269129,525.75532958)(490.01269258,525.82532977)
\curveto(489.99269133,525.90532943)(489.97769134,525.99032935)(489.96769258,526.08032977)
\lineto(489.96769258,526.36532977)
\curveto(489.97769134,526.46532887)(489.98269134,526.55032879)(489.98269258,526.62032977)
\lineto(489.98269258,526.81532977)
\curveto(489.98269134,526.87532846)(489.99269133,526.93032841)(490.01269258,526.98032977)
\curveto(490.05269127,527.09032825)(490.1226912,527.16032818)(490.22269258,527.19032977)
\curveto(490.25269107,527.19032815)(490.30769101,527.20032814)(490.38769258,527.22032977)
}
}
{
\newrgbcolor{curcolor}{0 0 0}
\pscustom[linestyle=none,fillstyle=solid,fillcolor=curcolor]
{
\newpath
\moveto(494.05284883,535.99532977)
\lineto(498.85284883,535.99532977)
\lineto(499.85784883,535.99532977)
\curveto(499.99784173,535.99531934)(500.11784161,535.98531935)(500.21784883,535.96532977)
\curveto(500.3278414,535.95531938)(500.40784132,535.91031943)(500.45784883,535.83032977)
\curveto(500.47784125,535.79031955)(500.48784124,535.7403196)(500.48784883,535.68032977)
\curveto(500.49784123,535.62031972)(500.50284123,535.55531978)(500.50284883,535.48532977)
\lineto(500.50284883,535.21532977)
\curveto(500.50284123,535.12532021)(500.49284124,535.04532029)(500.47284883,534.97532977)
\curveto(500.4328413,534.89532044)(500.38784134,534.82532051)(500.33784883,534.76532977)
\lineto(500.18784883,534.58532977)
\curveto(500.15784157,534.5353208)(500.12284161,534.49532084)(500.08284883,534.46532977)
\curveto(500.04284169,534.4353209)(500.00284173,534.39532094)(499.96284883,534.34532977)
\curveto(499.88284185,534.2353211)(499.79784193,534.12532121)(499.70784883,534.01532977)
\curveto(499.61784211,533.91532142)(499.5328422,533.81032153)(499.45284883,533.70032977)
\curveto(499.31284242,533.50032184)(499.17284256,533.29032205)(499.03284883,533.07032977)
\curveto(498.89284284,532.86032248)(498.75284298,532.64532269)(498.61284883,532.42532977)
\curveto(498.56284317,532.335323)(498.51284322,532.2403231)(498.46284883,532.14032977)
\curveto(498.41284332,532.0403233)(498.35784337,531.94532339)(498.29784883,531.85532977)
\curveto(498.27784345,531.8353235)(498.26784346,531.81032353)(498.26784883,531.78032977)
\curveto(498.26784346,531.75032359)(498.25784347,531.72532361)(498.23784883,531.70532977)
\curveto(498.16784356,531.60532373)(498.10284363,531.49032385)(498.04284883,531.36032977)
\curveto(497.98284375,531.2403241)(497.9278438,531.12532421)(497.87784883,531.01532977)
\curveto(497.77784395,530.78532455)(497.68284405,530.55032479)(497.59284883,530.31032977)
\curveto(497.50284423,530.07032527)(497.40284433,529.83032551)(497.29284883,529.59032977)
\curveto(497.27284446,529.5403258)(497.25784447,529.49532584)(497.24784883,529.45532977)
\curveto(497.24784448,529.41532592)(497.23784449,529.37032597)(497.21784883,529.32032977)
\curveto(497.16784456,529.20032614)(497.12284461,529.07532626)(497.08284883,528.94532977)
\curveto(497.05284468,528.82532651)(497.01784471,528.70532663)(496.97784883,528.58532977)
\curveto(496.89784483,528.35532698)(496.8328449,528.11532722)(496.78284883,527.86532977)
\curveto(496.74284499,527.62532771)(496.69284504,527.38532795)(496.63284883,527.14532977)
\curveto(496.59284514,526.99532834)(496.56784516,526.84532849)(496.55784883,526.69532977)
\curveto(496.54784518,526.54532879)(496.5278452,526.39532894)(496.49784883,526.24532977)
\curveto(496.48784524,526.20532913)(496.48284525,526.14532919)(496.48284883,526.06532977)
\curveto(496.45284528,525.94532939)(496.42284531,525.84532949)(496.39284883,525.76532977)
\curveto(496.36284537,525.68532965)(496.29284544,525.63032971)(496.18284883,525.60032977)
\curveto(496.1328456,525.58032976)(496.07784565,525.57032977)(496.01784883,525.57032977)
\lineto(495.82284883,525.57032977)
\curveto(495.68284605,525.57032977)(495.54284619,525.57532976)(495.40284883,525.58532977)
\curveto(495.27284646,525.59532974)(495.17784655,525.6403297)(495.11784883,525.72032977)
\curveto(495.07784665,525.78032956)(495.05784667,525.86532947)(495.05784883,525.97532977)
\curveto(495.06784666,526.08532925)(495.08284665,526.18032916)(495.10284883,526.26032977)
\lineto(495.10284883,526.33532977)
\curveto(495.11284662,526.36532897)(495.11784661,526.39532894)(495.11784883,526.42532977)
\curveto(495.13784659,526.50532883)(495.14784658,526.58032876)(495.14784883,526.65032977)
\curveto(495.14784658,526.72032862)(495.15784657,526.79032855)(495.17784883,526.86032977)
\curveto(495.2278465,527.05032829)(495.26784646,527.2353281)(495.29784883,527.41532977)
\curveto(495.3278464,527.60532773)(495.36784636,527.78532755)(495.41784883,527.95532977)
\curveto(495.43784629,528.00532733)(495.44784628,528.04532729)(495.44784883,528.07532977)
\curveto(495.44784628,528.10532723)(495.45284628,528.1403272)(495.46284883,528.18032977)
\curveto(495.56284617,528.48032686)(495.65284608,528.77532656)(495.73284883,529.06532977)
\curveto(495.82284591,529.35532598)(495.9278458,529.6353257)(496.04784883,529.90532977)
\curveto(496.30784542,530.48532485)(496.57784515,531.0353243)(496.85784883,531.55532977)
\curveto(497.13784459,532.08532325)(497.44784428,532.59032275)(497.78784883,533.07032977)
\curveto(497.9278438,533.27032207)(498.07784365,533.46032188)(498.23784883,533.64032977)
\curveto(498.39784333,533.83032151)(498.54784318,534.02032132)(498.68784883,534.21032977)
\curveto(498.727843,534.26032108)(498.76284297,534.30532103)(498.79284883,534.34532977)
\curveto(498.8328429,534.39532094)(498.86784286,534.44532089)(498.89784883,534.49532977)
\curveto(498.90784282,534.51532082)(498.91784281,534.5403208)(498.92784883,534.57032977)
\curveto(498.94784278,534.60032074)(498.94784278,534.63032071)(498.92784883,534.66032977)
\curveto(498.90784282,534.72032062)(498.87284286,534.75532058)(498.82284883,534.76532977)
\curveto(498.77284296,534.78532055)(498.72284301,534.80532053)(498.67284883,534.82532977)
\lineto(498.56784883,534.82532977)
\curveto(498.5278432,534.8353205)(498.47784325,534.8353205)(498.41784883,534.82532977)
\lineto(498.26784883,534.82532977)
\lineto(497.66784883,534.82532977)
\lineto(495.02784883,534.82532977)
\lineto(494.29284883,534.82532977)
\lineto(494.05284883,534.82532977)
\curveto(493.98284775,534.8353205)(493.92284781,534.85032049)(493.87284883,534.87032977)
\curveto(493.78284795,534.91032043)(493.72284801,534.97032037)(493.69284883,535.05032977)
\curveto(493.64284809,535.15032019)(493.6278481,535.29532004)(493.64784883,535.48532977)
\curveto(493.66784806,535.68531965)(493.70284803,535.82031952)(493.75284883,535.89032977)
\curveto(493.77284796,535.91031943)(493.79784793,535.92531941)(493.82784883,535.93532977)
\lineto(493.94784883,535.99532977)
\curveto(493.96784776,535.99531934)(493.98284775,535.99031935)(493.99284883,535.98032977)
\curveto(494.01284772,535.98031936)(494.0328477,535.98531935)(494.05284883,535.99532977)
}
}
{
\newrgbcolor{curcolor}{0 0 0}
\pscustom[linestyle=none,fillstyle=solid,fillcolor=curcolor]
{
\newpath
\moveto(511.74745821,534.10532977)
\curveto(511.54744791,533.81532152)(511.33744812,533.53032181)(511.11745821,533.25032977)
\curveto(510.90744855,532.97032237)(510.70244875,532.68532265)(510.50245821,532.39532977)
\curveto(509.90244955,531.54532379)(509.29745016,530.70532463)(508.68745821,529.87532977)
\curveto(508.07745138,529.05532628)(507.47245198,528.22032712)(506.87245821,527.37032977)
\lineto(506.36245821,526.65032977)
\lineto(505.85245821,525.96032977)
\curveto(505.77245368,525.85032949)(505.69245376,525.7353296)(505.61245821,525.61532977)
\curveto(505.53245392,525.49532984)(505.43745402,525.40032994)(505.32745821,525.33032977)
\curveto(505.28745417,525.31033003)(505.22245423,525.29533004)(505.13245821,525.28532977)
\curveto(505.0524544,525.26533007)(504.96245449,525.25533008)(504.86245821,525.25532977)
\curveto(504.76245469,525.25533008)(504.66745479,525.26033008)(504.57745821,525.27032977)
\curveto(504.49745496,525.28033006)(504.43745502,525.30033004)(504.39745821,525.33032977)
\curveto(504.36745509,525.35032999)(504.34245511,525.38532995)(504.32245821,525.43532977)
\curveto(504.31245514,525.47532986)(504.31745514,525.52032982)(504.33745821,525.57032977)
\curveto(504.37745508,525.65032969)(504.42245503,525.72532961)(504.47245821,525.79532977)
\curveto(504.53245492,525.87532946)(504.58745487,525.95532938)(504.63745821,526.03532977)
\curveto(504.87745458,526.37532896)(505.12245433,526.71032863)(505.37245821,527.04032977)
\curveto(505.62245383,527.37032797)(505.86245359,527.70532763)(506.09245821,528.04532977)
\curveto(506.2524532,528.26532707)(506.41245304,528.48032686)(506.57245821,528.69032977)
\curveto(506.73245272,528.90032644)(506.89245256,529.11532622)(507.05245821,529.33532977)
\curveto(507.41245204,529.85532548)(507.77745168,530.36532497)(508.14745821,530.86532977)
\curveto(508.51745094,531.36532397)(508.88745057,531.87532346)(509.25745821,532.39532977)
\curveto(509.39745006,532.59532274)(509.53744992,532.79032255)(509.67745821,532.98032977)
\curveto(509.82744963,533.17032217)(509.97244948,533.36532197)(510.11245821,533.56532977)
\curveto(510.32244913,533.86532147)(510.53744892,534.16532117)(510.75745821,534.46532977)
\lineto(511.41745821,535.36532977)
\lineto(511.59745821,535.63532977)
\lineto(511.80745821,535.90532977)
\lineto(511.92745821,536.08532977)
\curveto(511.97744748,536.14531919)(512.02744743,536.20031914)(512.07745821,536.25032977)
\curveto(512.14744731,536.30031904)(512.22244723,536.335319)(512.30245821,536.35532977)
\curveto(512.32244713,536.36531897)(512.34744711,536.36531897)(512.37745821,536.35532977)
\curveto(512.41744704,536.35531898)(512.44744701,536.36531897)(512.46745821,536.38532977)
\curveto(512.58744687,536.38531895)(512.72244673,536.38031896)(512.87245821,536.37032977)
\curveto(513.02244643,536.37031897)(513.11244634,536.32531901)(513.14245821,536.23532977)
\curveto(513.16244629,536.20531913)(513.16744629,536.17031917)(513.15745821,536.13032977)
\curveto(513.14744631,536.09031925)(513.13244632,536.06031928)(513.11245821,536.04032977)
\curveto(513.07244638,535.96031938)(513.03244642,535.89031945)(512.99245821,535.83032977)
\curveto(512.9524465,535.77031957)(512.90744655,535.71031963)(512.85745821,535.65032977)
\lineto(512.28745821,534.87032977)
\curveto(512.10744735,534.62032072)(511.92744753,534.36532097)(511.74745821,534.10532977)
\moveto(504.89245821,530.20532977)
\curveto(504.84245461,530.22532511)(504.79245466,530.23032511)(504.74245821,530.22032977)
\curveto(504.69245476,530.21032513)(504.64245481,530.21532512)(504.59245821,530.23532977)
\curveto(504.48245497,530.25532508)(504.37745508,530.27532506)(504.27745821,530.29532977)
\curveto(504.18745527,530.32532501)(504.09245536,530.36532497)(503.99245821,530.41532977)
\curveto(503.66245579,530.55532478)(503.40745605,530.75032459)(503.22745821,531.00032977)
\curveto(503.04745641,531.26032408)(502.90245655,531.57032377)(502.79245821,531.93032977)
\curveto(502.76245669,532.01032333)(502.74245671,532.09032325)(502.73245821,532.17032977)
\curveto(502.72245673,532.26032308)(502.70745675,532.34532299)(502.68745821,532.42532977)
\curveto(502.67745678,532.47532286)(502.67245678,532.5403228)(502.67245821,532.62032977)
\curveto(502.66245679,532.65032269)(502.6574568,532.68032266)(502.65745821,532.71032977)
\curveto(502.6574568,532.75032259)(502.6524568,532.78532255)(502.64245821,532.81532977)
\lineto(502.64245821,532.96532977)
\curveto(502.63245682,533.01532232)(502.62745683,533.07532226)(502.62745821,533.14532977)
\curveto(502.62745683,533.22532211)(502.63245682,533.29032205)(502.64245821,533.34032977)
\lineto(502.64245821,533.50532977)
\curveto(502.66245679,533.55532178)(502.66745679,533.60032174)(502.65745821,533.64032977)
\curveto(502.6574568,533.69032165)(502.66245679,533.7353216)(502.67245821,533.77532977)
\curveto(502.68245677,533.81532152)(502.68745677,533.85032149)(502.68745821,533.88032977)
\curveto(502.68745677,533.92032142)(502.69245676,533.96032138)(502.70245821,534.00032977)
\curveto(502.73245672,534.11032123)(502.7524567,534.22032112)(502.76245821,534.33032977)
\curveto(502.78245667,534.45032089)(502.81745664,534.56532077)(502.86745821,534.67532977)
\curveto(503.00745645,535.01532032)(503.16745629,535.29032005)(503.34745821,535.50032977)
\curveto(503.53745592,535.72031962)(503.80745565,535.90031944)(504.15745821,536.04032977)
\curveto(504.23745522,536.07031927)(504.32245513,536.09031925)(504.41245821,536.10032977)
\curveto(504.50245495,536.12031922)(504.59745486,536.1403192)(504.69745821,536.16032977)
\curveto(504.72745473,536.17031917)(504.78245467,536.17031917)(504.86245821,536.16032977)
\curveto(504.94245451,536.16031918)(504.99245446,536.17031917)(505.01245821,536.19032977)
\curveto(505.57245388,536.20031914)(506.02245343,536.09031925)(506.36245821,535.86032977)
\curveto(506.71245274,535.63031971)(506.97245248,535.32532001)(507.14245821,534.94532977)
\curveto(507.18245227,534.85532048)(507.21745224,534.76032058)(507.24745821,534.66032977)
\curveto(507.27745218,534.56032078)(507.30245215,534.46032088)(507.32245821,534.36032977)
\curveto(507.34245211,534.33032101)(507.34745211,534.30032104)(507.33745821,534.27032977)
\curveto(507.33745212,534.2403211)(507.34245211,534.21032113)(507.35245821,534.18032977)
\curveto(507.38245207,534.07032127)(507.40245205,533.94532139)(507.41245821,533.80532977)
\curveto(507.42245203,533.67532166)(507.43245202,533.5403218)(507.44245821,533.40032977)
\lineto(507.44245821,533.23532977)
\curveto(507.452452,533.17532216)(507.452452,533.12032222)(507.44245821,533.07032977)
\curveto(507.43245202,533.02032232)(507.42745203,532.97032237)(507.42745821,532.92032977)
\lineto(507.42745821,532.78532977)
\curveto(507.41745204,532.74532259)(507.41245204,532.70532263)(507.41245821,532.66532977)
\curveto(507.42245203,532.62532271)(507.41745204,532.58032276)(507.39745821,532.53032977)
\curveto(507.37745208,532.42032292)(507.3574521,532.31532302)(507.33745821,532.21532977)
\curveto(507.32745213,532.11532322)(507.30745215,532.01532332)(507.27745821,531.91532977)
\curveto(507.14745231,531.55532378)(506.98245247,531.2403241)(506.78245821,530.97032977)
\curveto(506.58245287,530.70032464)(506.30745315,530.49532484)(505.95745821,530.35532977)
\curveto(505.87745358,530.32532501)(505.79245366,530.30032504)(505.70245821,530.28032977)
\lineto(505.43245821,530.22032977)
\curveto(505.38245407,530.21032513)(505.33745412,530.20532513)(505.29745821,530.20532977)
\curveto(505.2574542,530.21532512)(505.21745424,530.21532512)(505.17745821,530.20532977)
\curveto(505.07745438,530.18532515)(504.98245447,530.18532515)(504.89245821,530.20532977)
\moveto(504.05245821,531.60032977)
\curveto(504.09245536,531.53032381)(504.13245532,531.46532387)(504.17245821,531.40532977)
\curveto(504.21245524,531.35532398)(504.26245519,531.30532403)(504.32245821,531.25532977)
\lineto(504.47245821,531.13532977)
\curveto(504.53245492,531.10532423)(504.59745486,531.08032426)(504.66745821,531.06032977)
\curveto(504.70745475,531.0403243)(504.74245471,531.03032431)(504.77245821,531.03032977)
\curveto(504.81245464,531.0403243)(504.8524546,531.0353243)(504.89245821,531.01532977)
\curveto(504.92245453,531.01532432)(504.96245449,531.01032433)(505.01245821,531.00032977)
\curveto(505.06245439,531.00032434)(505.10245435,531.00532433)(505.13245821,531.01532977)
\lineto(505.35745821,531.06032977)
\curveto(505.60745385,531.1403242)(505.79245366,531.26532407)(505.91245821,531.43532977)
\curveto(505.99245346,531.5353238)(506.06245339,531.66532367)(506.12245821,531.82532977)
\curveto(506.20245325,532.00532333)(506.26245319,532.23032311)(506.30245821,532.50032977)
\curveto(506.34245311,532.78032256)(506.3574531,533.06032228)(506.34745821,533.34032977)
\curveto(506.33745312,533.63032171)(506.30745315,533.90532143)(506.25745821,534.16532977)
\curveto(506.20745325,534.42532091)(506.13245332,534.6353207)(506.03245821,534.79532977)
\curveto(505.91245354,534.99532034)(505.76245369,535.14532019)(505.58245821,535.24532977)
\curveto(505.50245395,535.29532004)(505.41245404,535.32532001)(505.31245821,535.33532977)
\curveto(505.21245424,535.35531998)(505.10745435,535.36531997)(504.99745821,535.36532977)
\curveto(504.97745448,535.35531998)(504.9524545,535.35031999)(504.92245821,535.35032977)
\curveto(504.90245455,535.36031998)(504.88245457,535.36031998)(504.86245821,535.35032977)
\curveto(504.81245464,535.34032)(504.76745469,535.33032001)(504.72745821,535.32032977)
\curveto(504.68745477,535.32032002)(504.64745481,535.31032003)(504.60745821,535.29032977)
\curveto(504.42745503,535.21032013)(504.27745518,535.09032025)(504.15745821,534.93032977)
\curveto(504.04745541,534.77032057)(503.9574555,534.59032075)(503.88745821,534.39032977)
\curveto(503.82745563,534.20032114)(503.78245567,533.97532136)(503.75245821,533.71532977)
\curveto(503.73245572,533.45532188)(503.72745573,533.19032215)(503.73745821,532.92032977)
\curveto(503.74745571,532.66032268)(503.77745568,532.41032293)(503.82745821,532.17032977)
\curveto(503.88745557,531.9403234)(503.96245549,531.75032359)(504.05245821,531.60032977)
\moveto(514.85245821,528.61532977)
\curveto(514.86244459,528.56532677)(514.86744459,528.47532686)(514.86745821,528.34532977)
\curveto(514.86744459,528.21532712)(514.8574446,528.12532721)(514.83745821,528.07532977)
\curveto(514.81744464,528.02532731)(514.81244464,527.97032737)(514.82245821,527.91032977)
\curveto(514.83244462,527.86032748)(514.83244462,527.81032753)(514.82245821,527.76032977)
\curveto(514.78244467,527.62032772)(514.7524447,527.48532785)(514.73245821,527.35532977)
\curveto(514.72244473,527.22532811)(514.69244476,527.10532823)(514.64245821,526.99532977)
\curveto(514.50244495,526.64532869)(514.33744512,526.35032899)(514.14745821,526.11032977)
\curveto(513.9574455,525.88032946)(513.68744577,525.69532964)(513.33745821,525.55532977)
\curveto(513.2574462,525.52532981)(513.17244628,525.50532983)(513.08245821,525.49532977)
\curveto(512.99244646,525.47532986)(512.90744655,525.45532988)(512.82745821,525.43532977)
\curveto(512.77744668,525.42532991)(512.72744673,525.42032992)(512.67745821,525.42032977)
\curveto(512.62744683,525.42032992)(512.57744688,525.41532992)(512.52745821,525.40532977)
\curveto(512.49744696,525.39532994)(512.44744701,525.39532994)(512.37745821,525.40532977)
\curveto(512.30744715,525.40532993)(512.2574472,525.41032993)(512.22745821,525.42032977)
\curveto(512.16744729,525.4403299)(512.10744735,525.45032989)(512.04745821,525.45032977)
\curveto(511.99744746,525.4403299)(511.94744751,525.44532989)(511.89745821,525.46532977)
\curveto(511.80744765,525.48532985)(511.71744774,525.51032983)(511.62745821,525.54032977)
\curveto(511.54744791,525.56032978)(511.46744799,525.59032975)(511.38745821,525.63032977)
\curveto(511.06744839,525.77032957)(510.81744864,525.96532937)(510.63745821,526.21532977)
\curveto(510.457449,526.47532886)(510.30744915,526.78032856)(510.18745821,527.13032977)
\curveto(510.16744929,527.21032813)(510.1524493,527.29532804)(510.14245821,527.38532977)
\curveto(510.13244932,527.47532786)(510.11744934,527.56032778)(510.09745821,527.64032977)
\curveto(510.08744937,527.67032767)(510.08244937,527.70032764)(510.08245821,527.73032977)
\lineto(510.08245821,527.83532977)
\curveto(510.06244939,527.91532742)(510.0524494,527.99532734)(510.05245821,528.07532977)
\lineto(510.05245821,528.21032977)
\curveto(510.03244942,528.31032703)(510.03244942,528.41032693)(510.05245821,528.51032977)
\lineto(510.05245821,528.69032977)
\curveto(510.06244939,528.7403266)(510.06744939,528.78532655)(510.06745821,528.82532977)
\curveto(510.06744939,528.87532646)(510.07244938,528.92032642)(510.08245821,528.96032977)
\curveto(510.09244936,529.00032634)(510.09744936,529.0353263)(510.09745821,529.06532977)
\curveto(510.09744936,529.10532623)(510.10244935,529.14532619)(510.11245821,529.18532977)
\lineto(510.17245821,529.51532977)
\curveto(510.19244926,529.6353257)(510.22244923,529.74532559)(510.26245821,529.84532977)
\curveto(510.40244905,530.17532516)(510.56244889,530.45032489)(510.74245821,530.67032977)
\curveto(510.93244852,530.90032444)(511.19244826,531.08532425)(511.52245821,531.22532977)
\curveto(511.60244785,531.26532407)(511.68744777,531.29032405)(511.77745821,531.30032977)
\lineto(512.07745821,531.36032977)
\lineto(512.21245821,531.36032977)
\curveto(512.26244719,531.37032397)(512.31244714,531.37532396)(512.36245821,531.37532977)
\curveto(512.93244652,531.39532394)(513.39244606,531.29032405)(513.74245821,531.06032977)
\curveto(514.10244535,530.8403245)(514.36744509,530.5403248)(514.53745821,530.16032977)
\curveto(514.58744487,530.06032528)(514.62744483,529.96032538)(514.65745821,529.86032977)
\curveto(514.68744477,529.76032558)(514.71744474,529.65532568)(514.74745821,529.54532977)
\curveto(514.7574447,529.50532583)(514.76244469,529.47032587)(514.76245821,529.44032977)
\curveto(514.76244469,529.42032592)(514.76744469,529.39032595)(514.77745821,529.35032977)
\curveto(514.79744466,529.28032606)(514.80744465,529.20532613)(514.80745821,529.12532977)
\curveto(514.80744465,529.04532629)(514.81744464,528.96532637)(514.83745821,528.88532977)
\curveto(514.83744462,528.8353265)(514.83744462,528.79032655)(514.83745821,528.75032977)
\curveto(514.83744462,528.71032663)(514.84244461,528.66532667)(514.85245821,528.61532977)
\moveto(513.74245821,528.18032977)
\curveto(513.7524457,528.23032711)(513.7574457,528.30532703)(513.75745821,528.40532977)
\curveto(513.76744569,528.50532683)(513.76244569,528.58032676)(513.74245821,528.63032977)
\curveto(513.72244573,528.69032665)(513.71744574,528.74532659)(513.72745821,528.79532977)
\curveto(513.74744571,528.85532648)(513.74744571,528.91532642)(513.72745821,528.97532977)
\curveto(513.71744574,529.00532633)(513.71244574,529.0403263)(513.71245821,529.08032977)
\curveto(513.71244574,529.12032622)(513.70744575,529.16032618)(513.69745821,529.20032977)
\curveto(513.67744578,529.28032606)(513.6574458,529.35532598)(513.63745821,529.42532977)
\curveto(513.62744583,529.50532583)(513.61244584,529.58532575)(513.59245821,529.66532977)
\curveto(513.56244589,529.72532561)(513.53744592,529.78532555)(513.51745821,529.84532977)
\curveto(513.49744596,529.90532543)(513.46744599,529.96532537)(513.42745821,530.02532977)
\curveto(513.32744613,530.19532514)(513.19744626,530.33032501)(513.03745821,530.43032977)
\curveto(512.9574465,530.48032486)(512.86244659,530.51532482)(512.75245821,530.53532977)
\curveto(512.64244681,530.55532478)(512.51744694,530.56532477)(512.37745821,530.56532977)
\curveto(512.3574471,530.55532478)(512.33244712,530.55032479)(512.30245821,530.55032977)
\curveto(512.27244718,530.56032478)(512.24244721,530.56032478)(512.21245821,530.55032977)
\lineto(512.06245821,530.49032977)
\curveto(512.01244744,530.48032486)(511.96744749,530.46532487)(511.92745821,530.44532977)
\curveto(511.73744772,530.335325)(511.59244786,530.19032515)(511.49245821,530.01032977)
\curveto(511.40244805,529.83032551)(511.32244813,529.62532571)(511.25245821,529.39532977)
\curveto(511.21244824,529.26532607)(511.19244826,529.13032621)(511.19245821,528.99032977)
\curveto(511.19244826,528.86032648)(511.18244827,528.71532662)(511.16245821,528.55532977)
\curveto(511.1524483,528.50532683)(511.14244831,528.44532689)(511.13245821,528.37532977)
\curveto(511.13244832,528.30532703)(511.14244831,528.24532709)(511.16245821,528.19532977)
\lineto(511.16245821,528.03032977)
\lineto(511.16245821,527.85032977)
\curveto(511.17244828,527.80032754)(511.18244827,527.74532759)(511.19245821,527.68532977)
\curveto(511.20244825,527.6353277)(511.20744825,527.58032776)(511.20745821,527.52032977)
\curveto(511.21744824,527.46032788)(511.23244822,527.40532793)(511.25245821,527.35532977)
\curveto(511.30244815,527.16532817)(511.36244809,526.99032835)(511.43245821,526.83032977)
\curveto(511.50244795,526.67032867)(511.60744785,526.5403288)(511.74745821,526.44032977)
\curveto(511.87744758,526.340329)(512.01744744,526.27032907)(512.16745821,526.23032977)
\curveto(512.19744726,526.22032912)(512.22244723,526.21532912)(512.24245821,526.21532977)
\curveto(512.27244718,526.22532911)(512.30244715,526.22532911)(512.33245821,526.21532977)
\curveto(512.3524471,526.21532912)(512.38244707,526.21032913)(512.42245821,526.20032977)
\curveto(512.46244699,526.20032914)(512.49744696,526.20532913)(512.52745821,526.21532977)
\curveto(512.56744689,526.22532911)(512.60744685,526.23032911)(512.64745821,526.23032977)
\curveto(512.68744677,526.23032911)(512.72744673,526.2403291)(512.76745821,526.26032977)
\curveto(513.00744645,526.340329)(513.20244625,526.47532886)(513.35245821,526.66532977)
\curveto(513.47244598,526.84532849)(513.56244589,527.05032829)(513.62245821,527.28032977)
\curveto(513.64244581,527.35032799)(513.6574458,527.42032792)(513.66745821,527.49032977)
\curveto(513.67744578,527.57032777)(513.69244576,527.65032769)(513.71245821,527.73032977)
\curveto(513.71244574,527.79032755)(513.71744574,527.8353275)(513.72745821,527.86532977)
\curveto(513.72744573,527.88532745)(513.72744573,527.91032743)(513.72745821,527.94032977)
\curveto(513.72744573,527.98032736)(513.73244572,528.01032733)(513.74245821,528.03032977)
\lineto(513.74245821,528.18032977)
}
}
{
\newrgbcolor{curcolor}{0 0 0}
\pscustom[linestyle=none,fillstyle=solid,fillcolor=curcolor]
{
\newpath
\moveto(548.51432955,281.63185809)
\curveto(548.61432469,281.63184747)(548.7093246,281.62184748)(548.79932955,281.60185809)
\curveto(548.88932442,281.59184751)(548.95432435,281.56184754)(548.99432955,281.51185809)
\curveto(549.05432425,281.43184767)(549.08432422,281.32684777)(549.08432955,281.19685809)
\lineto(549.08432955,280.80685809)
\lineto(549.08432955,279.30685809)
\lineto(549.08432955,272.91685809)
\lineto(549.08432955,271.74685809)
\lineto(549.08432955,271.43185809)
\curveto(549.09432421,271.33185777)(549.07932423,271.25185785)(549.03932955,271.19185809)
\curveto(548.98932432,271.11185799)(548.91432439,271.06185804)(548.81432955,271.04185809)
\curveto(548.72432458,271.03185807)(548.61432469,271.02685807)(548.48432955,271.02685809)
\lineto(548.25932955,271.02685809)
\curveto(548.17932513,271.04685805)(548.1093252,271.06185804)(548.04932955,271.07185809)
\curveto(547.98932532,271.09185801)(547.93932537,271.13185797)(547.89932955,271.19185809)
\curveto(547.85932545,271.25185785)(547.83932547,271.32685777)(547.83932955,271.41685809)
\lineto(547.83932955,271.71685809)
\lineto(547.83932955,272.81185809)
\lineto(547.83932955,278.15185809)
\curveto(547.81932549,278.24185086)(547.8043255,278.31685078)(547.79432955,278.37685809)
\curveto(547.79432551,278.44685065)(547.76432554,278.50685059)(547.70432955,278.55685809)
\curveto(547.63432567,278.60685049)(547.54432576,278.63185047)(547.43432955,278.63185809)
\curveto(547.33432597,278.64185046)(547.22432608,278.64685045)(547.10432955,278.64685809)
\lineto(545.96432955,278.64685809)
\lineto(545.46932955,278.64685809)
\curveto(545.309328,278.65685044)(545.19932811,278.71685038)(545.13932955,278.82685809)
\curveto(545.11932819,278.85685024)(545.1093282,278.88685021)(545.10932955,278.91685809)
\curveto(545.1093282,278.95685014)(545.1043282,279.0018501)(545.09432955,279.05185809)
\curveto(545.07432823,279.17184993)(545.07932823,279.28184982)(545.10932955,279.38185809)
\curveto(545.14932816,279.48184962)(545.2043281,279.55184955)(545.27432955,279.59185809)
\curveto(545.35432795,279.64184946)(545.47432783,279.66684943)(545.63432955,279.66685809)
\curveto(545.79432751,279.66684943)(545.92932738,279.68184942)(546.03932955,279.71185809)
\curveto(546.08932722,279.72184938)(546.14432716,279.72684937)(546.20432955,279.72685809)
\curveto(546.26432704,279.73684936)(546.32432698,279.75184935)(546.38432955,279.77185809)
\curveto(546.53432677,279.82184928)(546.67932663,279.87184923)(546.81932955,279.92185809)
\curveto(546.95932635,279.98184912)(547.09432621,280.05184905)(547.22432955,280.13185809)
\curveto(547.36432594,280.22184888)(547.48432582,280.32684877)(547.58432955,280.44685809)
\curveto(547.68432562,280.56684853)(547.77932553,280.6968484)(547.86932955,280.83685809)
\curveto(547.92932538,280.93684816)(547.97432533,281.04684805)(548.00432955,281.16685809)
\curveto(548.04432526,281.28684781)(548.09432521,281.39184771)(548.15432955,281.48185809)
\curveto(548.2043251,281.54184756)(548.27432503,281.58184752)(548.36432955,281.60185809)
\curveto(548.38432492,281.61184749)(548.4093249,281.61684748)(548.43932955,281.61685809)
\curveto(548.46932484,281.61684748)(548.49432481,281.62184748)(548.51432955,281.63185809)
}
}
{
\newrgbcolor{curcolor}{0 0 0}
\pscustom[linestyle=none,fillstyle=solid,fillcolor=curcolor]
{
\newpath
\moveto(556.86393892,281.63185809)
\curveto(556.96393407,281.63184747)(557.05893397,281.62184748)(557.14893892,281.60185809)
\curveto(557.23893379,281.59184751)(557.30393373,281.56184754)(557.34393892,281.51185809)
\curveto(557.40393363,281.43184767)(557.4339336,281.32684777)(557.43393892,281.19685809)
\lineto(557.43393892,280.80685809)
\lineto(557.43393892,279.30685809)
\lineto(557.43393892,272.91685809)
\lineto(557.43393892,271.74685809)
\lineto(557.43393892,271.43185809)
\curveto(557.44393359,271.33185777)(557.4289336,271.25185785)(557.38893892,271.19185809)
\curveto(557.33893369,271.11185799)(557.26393377,271.06185804)(557.16393892,271.04185809)
\curveto(557.07393396,271.03185807)(556.96393407,271.02685807)(556.83393892,271.02685809)
\lineto(556.60893892,271.02685809)
\curveto(556.5289345,271.04685805)(556.45893457,271.06185804)(556.39893892,271.07185809)
\curveto(556.33893469,271.09185801)(556.28893474,271.13185797)(556.24893892,271.19185809)
\curveto(556.20893482,271.25185785)(556.18893484,271.32685777)(556.18893892,271.41685809)
\lineto(556.18893892,271.71685809)
\lineto(556.18893892,272.81185809)
\lineto(556.18893892,278.15185809)
\curveto(556.16893486,278.24185086)(556.15393488,278.31685078)(556.14393892,278.37685809)
\curveto(556.14393489,278.44685065)(556.11393492,278.50685059)(556.05393892,278.55685809)
\curveto(555.98393505,278.60685049)(555.89393514,278.63185047)(555.78393892,278.63185809)
\curveto(555.68393535,278.64185046)(555.57393546,278.64685045)(555.45393892,278.64685809)
\lineto(554.31393892,278.64685809)
\lineto(553.81893892,278.64685809)
\curveto(553.65893737,278.65685044)(553.54893748,278.71685038)(553.48893892,278.82685809)
\curveto(553.46893756,278.85685024)(553.45893757,278.88685021)(553.45893892,278.91685809)
\curveto(553.45893757,278.95685014)(553.45393758,279.0018501)(553.44393892,279.05185809)
\curveto(553.42393761,279.17184993)(553.4289376,279.28184982)(553.45893892,279.38185809)
\curveto(553.49893753,279.48184962)(553.55393748,279.55184955)(553.62393892,279.59185809)
\curveto(553.70393733,279.64184946)(553.82393721,279.66684943)(553.98393892,279.66685809)
\curveto(554.14393689,279.66684943)(554.27893675,279.68184942)(554.38893892,279.71185809)
\curveto(554.43893659,279.72184938)(554.49393654,279.72684937)(554.55393892,279.72685809)
\curveto(554.61393642,279.73684936)(554.67393636,279.75184935)(554.73393892,279.77185809)
\curveto(554.88393615,279.82184928)(555.028936,279.87184923)(555.16893892,279.92185809)
\curveto(555.30893572,279.98184912)(555.44393559,280.05184905)(555.57393892,280.13185809)
\curveto(555.71393532,280.22184888)(555.8339352,280.32684877)(555.93393892,280.44685809)
\curveto(556.033935,280.56684853)(556.1289349,280.6968484)(556.21893892,280.83685809)
\curveto(556.27893475,280.93684816)(556.32393471,281.04684805)(556.35393892,281.16685809)
\curveto(556.39393464,281.28684781)(556.44393459,281.39184771)(556.50393892,281.48185809)
\curveto(556.55393448,281.54184756)(556.62393441,281.58184752)(556.71393892,281.60185809)
\curveto(556.7339343,281.61184749)(556.75893427,281.61684748)(556.78893892,281.61685809)
\curveto(556.81893421,281.61684748)(556.84393419,281.62184748)(556.86393892,281.63185809)
}
}
{
\newrgbcolor{curcolor}{0 0 0}
\pscustom[linestyle=none,fillstyle=solid,fillcolor=curcolor]
{
\newpath
\moveto(562.1085483,272.66185809)
\lineto(562.4085483,272.66185809)
\curveto(562.51854624,272.67185643)(562.62354613,272.67185643)(562.7235483,272.66185809)
\curveto(562.83354592,272.66185644)(562.93354582,272.65185645)(563.0235483,272.63185809)
\curveto(563.11354564,272.62185648)(563.18354557,272.5968565)(563.2335483,272.55685809)
\curveto(563.2535455,272.53685656)(563.26854549,272.50685659)(563.2785483,272.46685809)
\curveto(563.29854546,272.42685667)(563.31854544,272.38185672)(563.3385483,272.33185809)
\lineto(563.3385483,272.25685809)
\curveto(563.34854541,272.20685689)(563.34854541,272.15185695)(563.3385483,272.09185809)
\lineto(563.3385483,271.94185809)
\lineto(563.3385483,271.46185809)
\curveto(563.33854542,271.29185781)(563.29854546,271.17185793)(563.2185483,271.10185809)
\curveto(563.14854561,271.05185805)(563.0585457,271.02685807)(562.9485483,271.02685809)
\lineto(562.6185483,271.02685809)
\lineto(562.1685483,271.02685809)
\curveto(562.01854674,271.02685807)(561.90354685,271.05685804)(561.8235483,271.11685809)
\curveto(561.78354697,271.14685795)(561.753547,271.1968579)(561.7335483,271.26685809)
\curveto(561.71354704,271.34685775)(561.69854706,271.43185767)(561.6885483,271.52185809)
\lineto(561.6885483,271.80685809)
\curveto(561.69854706,271.90685719)(561.70354705,271.99185711)(561.7035483,272.06185809)
\lineto(561.7035483,272.25685809)
\curveto(561.70354705,272.31685678)(561.71354704,272.37185673)(561.7335483,272.42185809)
\curveto(561.77354698,272.53185657)(561.84354691,272.6018565)(561.9435483,272.63185809)
\curveto(561.97354678,272.63185647)(562.02854673,272.64185646)(562.1085483,272.66185809)
}
}
{
\newrgbcolor{curcolor}{0 0 0}
\pscustom[linestyle=none,fillstyle=solid,fillcolor=curcolor]
{
\newpath
\moveto(572.25370455,276.62185809)
\curveto(572.25369691,276.54185256)(572.25869691,276.46185264)(572.26870455,276.38185809)
\curveto(572.27869689,276.3018528)(572.27369689,276.22685287)(572.25370455,276.15685809)
\curveto(572.23369693,276.11685298)(572.22869694,276.07185303)(572.23870455,276.02185809)
\curveto(572.24869692,275.98185312)(572.24869692,275.94185316)(572.23870455,275.90185809)
\lineto(572.23870455,275.75185809)
\curveto(572.22869694,275.66185344)(572.22369694,275.57185353)(572.22370455,275.48185809)
\curveto(572.22369694,275.4018537)(572.21869695,275.32185378)(572.20870455,275.24185809)
\lineto(572.17870455,275.00185809)
\curveto(572.168697,274.93185417)(572.15869701,274.85685424)(572.14870455,274.77685809)
\curveto(572.13869703,274.73685436)(572.13369703,274.6968544)(572.13370455,274.65685809)
\curveto(572.13369703,274.61685448)(572.12869704,274.57185453)(572.11870455,274.52185809)
\curveto(572.07869709,274.38185472)(572.04869712,274.24185486)(572.02870455,274.10185809)
\curveto(572.01869715,273.96185514)(571.98869718,273.82685527)(571.93870455,273.69685809)
\curveto(571.88869728,273.52685557)(571.83369733,273.36185574)(571.77370455,273.20185809)
\curveto(571.72369744,273.04185606)(571.6636975,272.88685621)(571.59370455,272.73685809)
\curveto(571.57369759,272.67685642)(571.54369762,272.61685648)(571.50370455,272.55685809)
\lineto(571.41370455,272.40685809)
\curveto(571.21369795,272.08685701)(570.99869817,271.82185728)(570.76870455,271.61185809)
\curveto(570.53869863,271.4018577)(570.24369892,271.22185788)(569.88370455,271.07185809)
\curveto(569.7636994,271.02185808)(569.63369953,270.98685811)(569.49370455,270.96685809)
\curveto(569.3636998,270.94685815)(569.22869994,270.92185818)(569.08870455,270.89185809)
\curveto(569.02870014,270.88185822)(568.9687002,270.87685822)(568.90870455,270.87685809)
\curveto(568.84870032,270.87685822)(568.78370038,270.87185823)(568.71370455,270.86185809)
\curveto(568.68370048,270.85185825)(568.63370053,270.85185825)(568.56370455,270.86185809)
\lineto(568.41370455,270.86185809)
\lineto(568.26370455,270.86185809)
\curveto(568.18370098,270.88185822)(568.09870107,270.8968582)(568.00870455,270.90685809)
\curveto(567.92870124,270.90685819)(567.85370131,270.91685818)(567.78370455,270.93685809)
\curveto(567.74370142,270.94685815)(567.70870146,270.95185815)(567.67870455,270.95185809)
\curveto(567.65870151,270.94185816)(567.63370153,270.94685815)(567.60370455,270.96685809)
\lineto(567.33370455,271.02685809)
\curveto(567.24370192,271.05685804)(567.15870201,271.08685801)(567.07870455,271.11685809)
\curveto(566.49870267,271.35685774)(566.0637031,271.72685737)(565.77370455,272.22685809)
\curveto(565.69370347,272.35685674)(565.62870354,272.49185661)(565.57870455,272.63185809)
\curveto(565.53870363,272.77185633)(565.49370367,272.92185618)(565.44370455,273.08185809)
\curveto(565.42370374,273.16185594)(565.41870375,273.24185586)(565.42870455,273.32185809)
\curveto(565.44870372,273.4018557)(565.48370368,273.45685564)(565.53370455,273.48685809)
\curveto(565.5637036,273.50685559)(565.61870355,273.52185558)(565.69870455,273.53185809)
\curveto(565.77870339,273.55185555)(565.8637033,273.56185554)(565.95370455,273.56185809)
\curveto(566.04370312,273.57185553)(566.12870304,273.57185553)(566.20870455,273.56185809)
\curveto(566.29870287,273.55185555)(566.3687028,273.54185556)(566.41870455,273.53185809)
\curveto(566.43870273,273.52185558)(566.4637027,273.50685559)(566.49370455,273.48685809)
\curveto(566.53370263,273.46685563)(566.5637026,273.44685565)(566.58370455,273.42685809)
\curveto(566.64370252,273.34685575)(566.68870248,273.25185585)(566.71870455,273.14185809)
\curveto(566.75870241,273.03185607)(566.80370236,272.93185617)(566.85370455,272.84185809)
\curveto(567.10370206,272.45185665)(567.47370169,272.18185692)(567.96370455,272.03185809)
\curveto(568.03370113,272.01185709)(568.10370106,271.9968571)(568.17370455,271.98685809)
\curveto(568.25370091,271.98685711)(568.33370083,271.97685712)(568.41370455,271.95685809)
\curveto(568.45370071,271.94685715)(568.50870066,271.94185716)(568.57870455,271.94185809)
\curveto(568.65870051,271.94185716)(568.71370045,271.94685715)(568.74370455,271.95685809)
\curveto(568.77370039,271.96685713)(568.80370036,271.97185713)(568.83370455,271.97185809)
\lineto(568.93870455,271.97185809)
\curveto(569.01870015,271.99185711)(569.09370007,272.01185709)(569.16370455,272.03185809)
\curveto(569.24369992,272.05185705)(569.31869985,272.07685702)(569.38870455,272.10685809)
\curveto(569.73869943,272.25685684)(570.00869916,272.47185663)(570.19870455,272.75185809)
\curveto(570.38869878,273.03185607)(570.54369862,273.35685574)(570.66370455,273.72685809)
\curveto(570.69369847,273.80685529)(570.71369845,273.88185522)(570.72370455,273.95185809)
\curveto(570.74369842,274.02185508)(570.7636984,274.096855)(570.78370455,274.17685809)
\curveto(570.80369836,274.26685483)(570.81869835,274.36185474)(570.82870455,274.46185809)
\curveto(570.84869832,274.57185453)(570.8686983,274.67685442)(570.88870455,274.77685809)
\curveto(570.89869827,274.82685427)(570.90369826,274.87685422)(570.90370455,274.92685809)
\curveto(570.91369825,274.98685411)(570.91869825,275.04185406)(570.91870455,275.09185809)
\curveto(570.93869823,275.15185395)(570.94869822,275.22685387)(570.94870455,275.31685809)
\curveto(570.94869822,275.41685368)(570.93869823,275.4968536)(570.91870455,275.55685809)
\curveto(570.88869828,275.64685345)(570.83869833,275.68685341)(570.76870455,275.67685809)
\curveto(570.70869846,275.66685343)(570.65369851,275.63685346)(570.60370455,275.58685809)
\curveto(570.52369864,275.53685356)(570.45369871,275.47685362)(570.39370455,275.40685809)
\curveto(570.34369882,275.33685376)(570.27869889,275.27685382)(570.19870455,275.22685809)
\curveto(570.03869913,275.11685398)(569.87369929,275.01685408)(569.70370455,274.92685809)
\curveto(569.53369963,274.84685425)(569.33869983,274.77685432)(569.11870455,274.71685809)
\curveto(569.01870015,274.68685441)(568.91870025,274.67185443)(568.81870455,274.67185809)
\curveto(568.72870044,274.67185443)(568.62870054,274.66185444)(568.51870455,274.64185809)
\lineto(568.36870455,274.64185809)
\curveto(568.31870085,274.66185444)(568.2687009,274.66685443)(568.21870455,274.65685809)
\curveto(568.17870099,274.64685445)(568.13870103,274.64685445)(568.09870455,274.65685809)
\curveto(568.0687011,274.66685443)(568.02370114,274.67185443)(567.96370455,274.67185809)
\curveto(567.90370126,274.68185442)(567.83870133,274.69185441)(567.76870455,274.70185809)
\lineto(567.58870455,274.73185809)
\curveto(567.13870203,274.85185425)(566.75870241,275.01685408)(566.44870455,275.22685809)
\curveto(566.17870299,275.41685368)(565.94870322,275.64685345)(565.75870455,275.91685809)
\curveto(565.57870359,276.1968529)(565.43370373,276.51185259)(565.32370455,276.86185809)
\lineto(565.26370455,277.07185809)
\curveto(565.25370391,277.15185195)(565.23870393,277.23185187)(565.21870455,277.31185809)
\curveto(565.20870396,277.34185176)(565.20370396,277.37185173)(565.20370455,277.40185809)
\curveto(565.20370396,277.43185167)(565.19870397,277.46185164)(565.18870455,277.49185809)
\curveto(565.17870399,277.55185155)(565.17370399,277.61185149)(565.17370455,277.67185809)
\curveto(565.17370399,277.74185136)(565.163704,277.8018513)(565.14370455,277.85185809)
\lineto(565.14370455,278.03185809)
\curveto(565.13370403,278.08185102)(565.12870404,278.15185095)(565.12870455,278.24185809)
\curveto(565.12870404,278.33185077)(565.13870403,278.4018507)(565.15870455,278.45185809)
\lineto(565.15870455,278.61685809)
\curveto(565.17870399,278.6968504)(565.18870398,278.77185033)(565.18870455,278.84185809)
\curveto(565.19870397,278.91185019)(565.21370395,278.98185012)(565.23370455,279.05185809)
\curveto(565.29370387,279.25184985)(565.35370381,279.44184966)(565.41370455,279.62185809)
\curveto(565.48370368,279.8018493)(565.57370359,279.97184913)(565.68370455,280.13185809)
\curveto(565.72370344,280.2018489)(565.7637034,280.26684883)(565.80370455,280.32685809)
\lineto(565.95370455,280.50685809)
\curveto(565.97370319,280.51684858)(565.99370317,280.53184857)(566.01370455,280.55185809)
\curveto(566.10370306,280.68184842)(566.21370295,280.79184831)(566.34370455,280.88185809)
\curveto(566.60370256,281.08184802)(566.8687023,281.23684786)(567.13870455,281.34685809)
\curveto(567.21870195,281.38684771)(567.29870187,281.41684768)(567.37870455,281.43685809)
\curveto(567.4687017,281.46684763)(567.55870161,281.49184761)(567.64870455,281.51185809)
\curveto(567.74870142,281.54184756)(567.84870132,281.56184754)(567.94870455,281.57185809)
\curveto(568.04870112,281.58184752)(568.15370101,281.5968475)(568.26370455,281.61685809)
\curveto(568.29370087,281.62684747)(568.33370083,281.62684747)(568.38370455,281.61685809)
\curveto(568.44370072,281.60684749)(568.48370068,281.61184749)(568.50370455,281.63185809)
\curveto(569.22369994,281.65184745)(569.82369934,281.53684756)(570.30370455,281.28685809)
\curveto(570.78369838,281.03684806)(571.15869801,280.6968484)(571.42870455,280.26685809)
\curveto(571.51869765,280.12684897)(571.59869757,279.98184912)(571.66870455,279.83185809)
\curveto(571.73869743,279.68184942)(571.80869736,279.52184958)(571.87870455,279.35185809)
\curveto(571.92869724,279.21184989)(571.9686972,279.06185004)(571.99870455,278.90185809)
\curveto(572.02869714,278.74185036)(572.0636971,278.58185052)(572.10370455,278.42185809)
\curveto(572.12369704,278.37185073)(572.13369703,278.31685078)(572.13370455,278.25685809)
\curveto(572.13369703,278.20685089)(572.13869703,278.15685094)(572.14870455,278.10685809)
\curveto(572.168697,278.04685105)(572.17869699,277.98185112)(572.17870455,277.91185809)
\curveto(572.17869699,277.85185125)(572.18869698,277.7968513)(572.20870455,277.74685809)
\lineto(572.20870455,277.58185809)
\curveto(572.22869694,277.53185157)(572.23369693,277.48185162)(572.22370455,277.43185809)
\curveto(572.21369695,277.38185172)(572.21869695,277.33185177)(572.23870455,277.28185809)
\curveto(572.23869693,277.26185184)(572.23369693,277.23685186)(572.22370455,277.20685809)
\curveto(572.22369694,277.17685192)(572.22869694,277.15185195)(572.23870455,277.13185809)
\curveto(572.24869692,277.101852)(572.24869692,277.06685203)(572.23870455,277.02685809)
\curveto(572.23869693,276.98685211)(572.24369692,276.94685215)(572.25370455,276.90685809)
\curveto(572.2636969,276.86685223)(572.2636969,276.82185228)(572.25370455,276.77185809)
\lineto(572.25370455,276.62185809)
\moveto(570.75370455,277.92685809)
\curveto(570.7636984,277.97685112)(570.7686984,278.03685106)(570.76870455,278.10685809)
\curveto(570.7686984,278.17685092)(570.7636984,278.23685086)(570.75370455,278.28685809)
\curveto(570.74369842,278.33685076)(570.73869843,278.41185069)(570.73870455,278.51185809)
\curveto(570.71869845,278.59185051)(570.69869847,278.66685043)(570.67870455,278.73685809)
\curveto(570.6686985,278.80685029)(570.65369851,278.87685022)(570.63370455,278.94685809)
\curveto(570.49369867,279.37684972)(570.29869887,279.71184939)(570.04870455,279.95185809)
\curveto(569.80869936,280.19184891)(569.4636997,280.37184873)(569.01370455,280.49185809)
\curveto(568.92370024,280.51184859)(568.82370034,280.52184858)(568.71370455,280.52185809)
\lineto(568.38370455,280.52185809)
\curveto(568.3637008,280.5018486)(568.32870084,280.49184861)(568.27870455,280.49185809)
\curveto(568.22870094,280.5018486)(568.18370098,280.5018486)(568.14370455,280.49185809)
\curveto(568.0637011,280.47184863)(567.98870118,280.45184865)(567.91870455,280.43185809)
\lineto(567.70870455,280.37185809)
\curveto(567.41870175,280.24184886)(567.18870198,280.06184904)(567.01870455,279.83185809)
\curveto(566.84870232,279.61184949)(566.71370245,279.35184975)(566.61370455,279.05185809)
\curveto(566.58370258,278.96185014)(566.55870261,278.86685023)(566.53870455,278.76685809)
\curveto(566.52870264,278.67685042)(566.51370265,278.58185052)(566.49370455,278.48185809)
\lineto(566.49370455,278.34685809)
\curveto(566.4637027,278.23685086)(566.45370271,278.096851)(566.46370455,277.92685809)
\curveto(566.48370268,277.76685133)(566.50370266,277.63685146)(566.52370455,277.53685809)
\curveto(566.54370262,277.47685162)(566.55870261,277.41685168)(566.56870455,277.35685809)
\curveto(566.57870259,277.30685179)(566.59370257,277.25685184)(566.61370455,277.20685809)
\curveto(566.69370247,277.00685209)(566.78870238,276.81685228)(566.89870455,276.63685809)
\curveto(567.01870215,276.45685264)(567.15870201,276.31185279)(567.31870455,276.20185809)
\curveto(567.3687018,276.15185295)(567.42370174,276.11185299)(567.48370455,276.08185809)
\curveto(567.54370162,276.05185305)(567.60370156,276.01685308)(567.66370455,275.97685809)
\curveto(567.81370135,275.8968532)(567.99870117,275.83185327)(568.21870455,275.78185809)
\curveto(568.2687009,275.76185334)(568.30870086,275.75685334)(568.33870455,275.76685809)
\curveto(568.37870079,275.77685332)(568.42370074,275.77185333)(568.47370455,275.75185809)
\curveto(568.51370065,275.74185336)(568.5687006,275.73685336)(568.63870455,275.73685809)
\curveto(568.70870046,275.73685336)(568.7687004,275.74185336)(568.81870455,275.75185809)
\curveto(568.91870025,275.77185333)(569.01370015,275.78685331)(569.10370455,275.79685809)
\curveto(569.19369997,275.81685328)(569.28369988,275.84685325)(569.37370455,275.88685809)
\curveto(569.91369925,276.10685299)(570.30869886,276.5018526)(570.55870455,277.07185809)
\curveto(570.60869856,277.17185193)(570.64369852,277.27185183)(570.66370455,277.37185809)
\curveto(570.68369848,277.48185162)(570.70869846,277.59185151)(570.73870455,277.70185809)
\curveto(570.73869843,277.8018513)(570.74369842,277.87685122)(570.75370455,277.92685809)
}
}
{
\newrgbcolor{curcolor}{0 0 0}
\pscustom[linestyle=none,fillstyle=solid,fillcolor=curcolor]
{
\newpath
\moveto(583.46831392,279.54685809)
\curveto(583.26830362,279.25684984)(583.05830383,278.97185013)(582.83831392,278.69185809)
\curveto(582.62830426,278.41185069)(582.42330447,278.12685097)(582.22331392,277.83685809)
\curveto(581.62330527,276.98685211)(581.01830587,276.14685295)(580.40831392,275.31685809)
\curveto(579.79830709,274.4968546)(579.1933077,273.66185544)(578.59331392,272.81185809)
\lineto(578.08331392,272.09185809)
\lineto(577.57331392,271.40185809)
\curveto(577.4933094,271.29185781)(577.41330948,271.17685792)(577.33331392,271.05685809)
\curveto(577.25330964,270.93685816)(577.15830973,270.84185826)(577.04831392,270.77185809)
\curveto(577.00830988,270.75185835)(576.94330995,270.73685836)(576.85331392,270.72685809)
\curveto(576.77331012,270.70685839)(576.68331021,270.6968584)(576.58331392,270.69685809)
\curveto(576.48331041,270.6968584)(576.3883105,270.7018584)(576.29831392,270.71185809)
\curveto(576.21831067,270.72185838)(576.15831073,270.74185836)(576.11831392,270.77185809)
\curveto(576.0883108,270.79185831)(576.06331083,270.82685827)(576.04331392,270.87685809)
\curveto(576.03331086,270.91685818)(576.03831085,270.96185814)(576.05831392,271.01185809)
\curveto(576.09831079,271.09185801)(576.14331075,271.16685793)(576.19331392,271.23685809)
\curveto(576.25331064,271.31685778)(576.30831058,271.3968577)(576.35831392,271.47685809)
\curveto(576.59831029,271.81685728)(576.84331005,272.15185695)(577.09331392,272.48185809)
\curveto(577.34330955,272.81185629)(577.58330931,273.14685595)(577.81331392,273.48685809)
\curveto(577.97330892,273.70685539)(578.13330876,273.92185518)(578.29331392,274.13185809)
\curveto(578.45330844,274.34185476)(578.61330828,274.55685454)(578.77331392,274.77685809)
\curveto(579.13330776,275.2968538)(579.49830739,275.80685329)(579.86831392,276.30685809)
\curveto(580.23830665,276.80685229)(580.60830628,277.31685178)(580.97831392,277.83685809)
\curveto(581.11830577,278.03685106)(581.25830563,278.23185087)(581.39831392,278.42185809)
\curveto(581.54830534,278.61185049)(581.6933052,278.80685029)(581.83331392,279.00685809)
\curveto(582.04330485,279.30684979)(582.25830463,279.60684949)(582.47831392,279.90685809)
\lineto(583.13831392,280.80685809)
\lineto(583.31831392,281.07685809)
\lineto(583.52831392,281.34685809)
\lineto(583.64831392,281.52685809)
\curveto(583.69830319,281.58684751)(583.74830314,281.64184746)(583.79831392,281.69185809)
\curveto(583.86830302,281.74184736)(583.94330295,281.77684732)(584.02331392,281.79685809)
\curveto(584.04330285,281.80684729)(584.06830282,281.80684729)(584.09831392,281.79685809)
\curveto(584.13830275,281.7968473)(584.16830272,281.80684729)(584.18831392,281.82685809)
\curveto(584.30830258,281.82684727)(584.44330245,281.82184728)(584.59331392,281.81185809)
\curveto(584.74330215,281.81184729)(584.83330206,281.76684733)(584.86331392,281.67685809)
\curveto(584.88330201,281.64684745)(584.888302,281.61184749)(584.87831392,281.57185809)
\curveto(584.86830202,281.53184757)(584.85330204,281.5018476)(584.83331392,281.48185809)
\curveto(584.7933021,281.4018477)(584.75330214,281.33184777)(584.71331392,281.27185809)
\curveto(584.67330222,281.21184789)(584.62830226,281.15184795)(584.57831392,281.09185809)
\lineto(584.00831392,280.31185809)
\curveto(583.82830306,280.06184904)(583.64830324,279.80684929)(583.46831392,279.54685809)
\moveto(576.61331392,275.64685809)
\curveto(576.56331033,275.66685343)(576.51331038,275.67185343)(576.46331392,275.66185809)
\curveto(576.41331048,275.65185345)(576.36331053,275.65685344)(576.31331392,275.67685809)
\curveto(576.20331069,275.6968534)(576.09831079,275.71685338)(575.99831392,275.73685809)
\curveto(575.90831098,275.76685333)(575.81331108,275.80685329)(575.71331392,275.85685809)
\curveto(575.38331151,275.9968531)(575.12831176,276.19185291)(574.94831392,276.44185809)
\curveto(574.76831212,276.7018524)(574.62331227,277.01185209)(574.51331392,277.37185809)
\curveto(574.48331241,277.45185165)(574.46331243,277.53185157)(574.45331392,277.61185809)
\curveto(574.44331245,277.7018514)(574.42831246,277.78685131)(574.40831392,277.86685809)
\curveto(574.39831249,277.91685118)(574.3933125,277.98185112)(574.39331392,278.06185809)
\curveto(574.38331251,278.09185101)(574.37831251,278.12185098)(574.37831392,278.15185809)
\curveto(574.37831251,278.19185091)(574.37331252,278.22685087)(574.36331392,278.25685809)
\lineto(574.36331392,278.40685809)
\curveto(574.35331254,278.45685064)(574.34831254,278.51685058)(574.34831392,278.58685809)
\curveto(574.34831254,278.66685043)(574.35331254,278.73185037)(574.36331392,278.78185809)
\lineto(574.36331392,278.94685809)
\curveto(574.38331251,278.9968501)(574.3883125,279.04185006)(574.37831392,279.08185809)
\curveto(574.37831251,279.13184997)(574.38331251,279.17684992)(574.39331392,279.21685809)
\curveto(574.40331249,279.25684984)(574.40831248,279.29184981)(574.40831392,279.32185809)
\curveto(574.40831248,279.36184974)(574.41331248,279.4018497)(574.42331392,279.44185809)
\curveto(574.45331244,279.55184955)(574.47331242,279.66184944)(574.48331392,279.77185809)
\curveto(574.50331239,279.89184921)(574.53831235,280.00684909)(574.58831392,280.11685809)
\curveto(574.72831216,280.45684864)(574.888312,280.73184837)(575.06831392,280.94185809)
\curveto(575.25831163,281.16184794)(575.52831136,281.34184776)(575.87831392,281.48185809)
\curveto(575.95831093,281.51184759)(576.04331085,281.53184757)(576.13331392,281.54185809)
\curveto(576.22331067,281.56184754)(576.31831057,281.58184752)(576.41831392,281.60185809)
\curveto(576.44831044,281.61184749)(576.50331039,281.61184749)(576.58331392,281.60185809)
\curveto(576.66331023,281.6018475)(576.71331018,281.61184749)(576.73331392,281.63185809)
\curveto(577.2933096,281.64184746)(577.74330915,281.53184757)(578.08331392,281.30185809)
\curveto(578.43330846,281.07184803)(578.6933082,280.76684833)(578.86331392,280.38685809)
\curveto(578.90330799,280.2968488)(578.93830795,280.2018489)(578.96831392,280.10185809)
\curveto(578.99830789,280.0018491)(579.02330787,279.9018492)(579.04331392,279.80185809)
\curveto(579.06330783,279.77184933)(579.06830782,279.74184936)(579.05831392,279.71185809)
\curveto(579.05830783,279.68184942)(579.06330783,279.65184945)(579.07331392,279.62185809)
\curveto(579.10330779,279.51184959)(579.12330777,279.38684971)(579.13331392,279.24685809)
\curveto(579.14330775,279.11684998)(579.15330774,278.98185012)(579.16331392,278.84185809)
\lineto(579.16331392,278.67685809)
\curveto(579.17330772,278.61685048)(579.17330772,278.56185054)(579.16331392,278.51185809)
\curveto(579.15330774,278.46185064)(579.14830774,278.41185069)(579.14831392,278.36185809)
\lineto(579.14831392,278.22685809)
\curveto(579.13830775,278.18685091)(579.13330776,278.14685095)(579.13331392,278.10685809)
\curveto(579.14330775,278.06685103)(579.13830775,278.02185108)(579.11831392,277.97185809)
\curveto(579.09830779,277.86185124)(579.07830781,277.75685134)(579.05831392,277.65685809)
\curveto(579.04830784,277.55685154)(579.02830786,277.45685164)(578.99831392,277.35685809)
\curveto(578.86830802,276.9968521)(578.70330819,276.68185242)(578.50331392,276.41185809)
\curveto(578.30330859,276.14185296)(578.02830886,275.93685316)(577.67831392,275.79685809)
\curveto(577.59830929,275.76685333)(577.51330938,275.74185336)(577.42331392,275.72185809)
\lineto(577.15331392,275.66185809)
\curveto(577.10330979,275.65185345)(577.05830983,275.64685345)(577.01831392,275.64685809)
\curveto(576.97830991,275.65685344)(576.93830995,275.65685344)(576.89831392,275.64685809)
\curveto(576.79831009,275.62685347)(576.70331019,275.62685347)(576.61331392,275.64685809)
\moveto(575.77331392,277.04185809)
\curveto(575.81331108,276.97185213)(575.85331104,276.90685219)(575.89331392,276.84685809)
\curveto(575.93331096,276.7968523)(575.98331091,276.74685235)(576.04331392,276.69685809)
\lineto(576.19331392,276.57685809)
\curveto(576.25331064,276.54685255)(576.31831057,276.52185258)(576.38831392,276.50185809)
\curveto(576.42831046,276.48185262)(576.46331043,276.47185263)(576.49331392,276.47185809)
\curveto(576.53331036,276.48185262)(576.57331032,276.47685262)(576.61331392,276.45685809)
\curveto(576.64331025,276.45685264)(576.68331021,276.45185265)(576.73331392,276.44185809)
\curveto(576.78331011,276.44185266)(576.82331007,276.44685265)(576.85331392,276.45685809)
\lineto(577.07831392,276.50185809)
\curveto(577.32830956,276.58185252)(577.51330938,276.70685239)(577.63331392,276.87685809)
\curveto(577.71330918,276.97685212)(577.78330911,277.10685199)(577.84331392,277.26685809)
\curveto(577.92330897,277.44685165)(577.98330891,277.67185143)(578.02331392,277.94185809)
\curveto(578.06330883,278.22185088)(578.07830881,278.5018506)(578.06831392,278.78185809)
\curveto(578.05830883,279.07185003)(578.02830886,279.34684975)(577.97831392,279.60685809)
\curveto(577.92830896,279.86684923)(577.85330904,280.07684902)(577.75331392,280.23685809)
\curveto(577.63330926,280.43684866)(577.48330941,280.58684851)(577.30331392,280.68685809)
\curveto(577.22330967,280.73684836)(577.13330976,280.76684833)(577.03331392,280.77685809)
\curveto(576.93330996,280.7968483)(576.82831006,280.80684829)(576.71831392,280.80685809)
\curveto(576.69831019,280.7968483)(576.67331022,280.79184831)(576.64331392,280.79185809)
\curveto(576.62331027,280.8018483)(576.60331029,280.8018483)(576.58331392,280.79185809)
\curveto(576.53331036,280.78184832)(576.4883104,280.77184833)(576.44831392,280.76185809)
\curveto(576.40831048,280.76184834)(576.36831052,280.75184835)(576.32831392,280.73185809)
\curveto(576.14831074,280.65184845)(575.99831089,280.53184857)(575.87831392,280.37185809)
\curveto(575.76831112,280.21184889)(575.67831121,280.03184907)(575.60831392,279.83185809)
\curveto(575.54831134,279.64184946)(575.50331139,279.41684968)(575.47331392,279.15685809)
\curveto(575.45331144,278.8968502)(575.44831144,278.63185047)(575.45831392,278.36185809)
\curveto(575.46831142,278.101851)(575.49831139,277.85185125)(575.54831392,277.61185809)
\curveto(575.60831128,277.38185172)(575.68331121,277.19185191)(575.77331392,277.04185809)
\moveto(586.57331392,274.05685809)
\curveto(586.58330031,274.00685509)(586.5883003,273.91685518)(586.58831392,273.78685809)
\curveto(586.5883003,273.65685544)(586.57830031,273.56685553)(586.55831392,273.51685809)
\curveto(586.53830035,273.46685563)(586.53330036,273.41185569)(586.54331392,273.35185809)
\curveto(586.55330034,273.3018558)(586.55330034,273.25185585)(586.54331392,273.20185809)
\curveto(586.50330039,273.06185604)(586.47330042,272.92685617)(586.45331392,272.79685809)
\curveto(586.44330045,272.66685643)(586.41330048,272.54685655)(586.36331392,272.43685809)
\curveto(586.22330067,272.08685701)(586.05830083,271.79185731)(585.86831392,271.55185809)
\curveto(585.67830121,271.32185778)(585.40830148,271.13685796)(585.05831392,270.99685809)
\curveto(584.97830191,270.96685813)(584.893302,270.94685815)(584.80331392,270.93685809)
\curveto(584.71330218,270.91685818)(584.62830226,270.8968582)(584.54831392,270.87685809)
\curveto(584.49830239,270.86685823)(584.44830244,270.86185824)(584.39831392,270.86185809)
\curveto(584.34830254,270.86185824)(584.29830259,270.85685824)(584.24831392,270.84685809)
\curveto(584.21830267,270.83685826)(584.16830272,270.83685826)(584.09831392,270.84685809)
\curveto(584.02830286,270.84685825)(583.97830291,270.85185825)(583.94831392,270.86185809)
\curveto(583.888303,270.88185822)(583.82830306,270.89185821)(583.76831392,270.89185809)
\curveto(583.71830317,270.88185822)(583.66830322,270.88685821)(583.61831392,270.90685809)
\curveto(583.52830336,270.92685817)(583.43830345,270.95185815)(583.34831392,270.98185809)
\curveto(583.26830362,271.0018581)(583.1883037,271.03185807)(583.10831392,271.07185809)
\curveto(582.7883041,271.21185789)(582.53830435,271.40685769)(582.35831392,271.65685809)
\curveto(582.17830471,271.91685718)(582.02830486,272.22185688)(581.90831392,272.57185809)
\curveto(581.888305,272.65185645)(581.87330502,272.73685636)(581.86331392,272.82685809)
\curveto(581.85330504,272.91685618)(581.83830505,273.0018561)(581.81831392,273.08185809)
\curveto(581.80830508,273.11185599)(581.80330509,273.14185596)(581.80331392,273.17185809)
\lineto(581.80331392,273.27685809)
\curveto(581.78330511,273.35685574)(581.77330512,273.43685566)(581.77331392,273.51685809)
\lineto(581.77331392,273.65185809)
\curveto(581.75330514,273.75185535)(581.75330514,273.85185525)(581.77331392,273.95185809)
\lineto(581.77331392,274.13185809)
\curveto(581.78330511,274.18185492)(581.7883051,274.22685487)(581.78831392,274.26685809)
\curveto(581.7883051,274.31685478)(581.7933051,274.36185474)(581.80331392,274.40185809)
\curveto(581.81330508,274.44185466)(581.81830507,274.47685462)(581.81831392,274.50685809)
\curveto(581.81830507,274.54685455)(581.82330507,274.58685451)(581.83331392,274.62685809)
\lineto(581.89331392,274.95685809)
\curveto(581.91330498,275.07685402)(581.94330495,275.18685391)(581.98331392,275.28685809)
\curveto(582.12330477,275.61685348)(582.28330461,275.89185321)(582.46331392,276.11185809)
\curveto(582.65330424,276.34185276)(582.91330398,276.52685257)(583.24331392,276.66685809)
\curveto(583.32330357,276.70685239)(583.40830348,276.73185237)(583.49831392,276.74185809)
\lineto(583.79831392,276.80185809)
\lineto(583.93331392,276.80185809)
\curveto(583.98330291,276.81185229)(584.03330286,276.81685228)(584.08331392,276.81685809)
\curveto(584.65330224,276.83685226)(585.11330178,276.73185237)(585.46331392,276.50185809)
\curveto(585.82330107,276.28185282)(586.0883008,275.98185312)(586.25831392,275.60185809)
\curveto(586.30830058,275.5018536)(586.34830054,275.4018537)(586.37831392,275.30185809)
\curveto(586.40830048,275.2018539)(586.43830045,275.096854)(586.46831392,274.98685809)
\curveto(586.47830041,274.94685415)(586.48330041,274.91185419)(586.48331392,274.88185809)
\curveto(586.48330041,274.86185424)(586.4883004,274.83185427)(586.49831392,274.79185809)
\curveto(586.51830037,274.72185438)(586.52830036,274.64685445)(586.52831392,274.56685809)
\curveto(586.52830036,274.48685461)(586.53830035,274.40685469)(586.55831392,274.32685809)
\curveto(586.55830033,274.27685482)(586.55830033,274.23185487)(586.55831392,274.19185809)
\curveto(586.55830033,274.15185495)(586.56330033,274.10685499)(586.57331392,274.05685809)
\moveto(585.46331392,273.62185809)
\curveto(585.47330142,273.67185543)(585.47830141,273.74685535)(585.47831392,273.84685809)
\curveto(585.4883014,273.94685515)(585.48330141,274.02185508)(585.46331392,274.07185809)
\curveto(585.44330145,274.13185497)(585.43830145,274.18685491)(585.44831392,274.23685809)
\curveto(585.46830142,274.2968548)(585.46830142,274.35685474)(585.44831392,274.41685809)
\curveto(585.43830145,274.44685465)(585.43330146,274.48185462)(585.43331392,274.52185809)
\curveto(585.43330146,274.56185454)(585.42830146,274.6018545)(585.41831392,274.64185809)
\curveto(585.39830149,274.72185438)(585.37830151,274.7968543)(585.35831392,274.86685809)
\curveto(585.34830154,274.94685415)(585.33330156,275.02685407)(585.31331392,275.10685809)
\curveto(585.28330161,275.16685393)(585.25830163,275.22685387)(585.23831392,275.28685809)
\curveto(585.21830167,275.34685375)(585.1883017,275.40685369)(585.14831392,275.46685809)
\curveto(585.04830184,275.63685346)(584.91830197,275.77185333)(584.75831392,275.87185809)
\curveto(584.67830221,275.92185318)(584.58330231,275.95685314)(584.47331392,275.97685809)
\curveto(584.36330253,275.9968531)(584.23830265,276.00685309)(584.09831392,276.00685809)
\curveto(584.07830281,275.9968531)(584.05330284,275.99185311)(584.02331392,275.99185809)
\curveto(583.9933029,276.0018531)(583.96330293,276.0018531)(583.93331392,275.99185809)
\lineto(583.78331392,275.93185809)
\curveto(583.73330316,275.92185318)(583.6883032,275.90685319)(583.64831392,275.88685809)
\curveto(583.45830343,275.77685332)(583.31330358,275.63185347)(583.21331392,275.45185809)
\curveto(583.12330377,275.27185383)(583.04330385,275.06685403)(582.97331392,274.83685809)
\curveto(582.93330396,274.70685439)(582.91330398,274.57185453)(582.91331392,274.43185809)
\curveto(582.91330398,274.3018548)(582.90330399,274.15685494)(582.88331392,273.99685809)
\curveto(582.87330402,273.94685515)(582.86330403,273.88685521)(582.85331392,273.81685809)
\curveto(582.85330404,273.74685535)(582.86330403,273.68685541)(582.88331392,273.63685809)
\lineto(582.88331392,273.47185809)
\lineto(582.88331392,273.29185809)
\curveto(582.893304,273.24185586)(582.90330399,273.18685591)(582.91331392,273.12685809)
\curveto(582.92330397,273.07685602)(582.92830396,273.02185608)(582.92831392,272.96185809)
\curveto(582.93830395,272.9018562)(582.95330394,272.84685625)(582.97331392,272.79685809)
\curveto(583.02330387,272.60685649)(583.08330381,272.43185667)(583.15331392,272.27185809)
\curveto(583.22330367,272.11185699)(583.32830356,271.98185712)(583.46831392,271.88185809)
\curveto(583.59830329,271.78185732)(583.73830315,271.71185739)(583.88831392,271.67185809)
\curveto(583.91830297,271.66185744)(583.94330295,271.65685744)(583.96331392,271.65685809)
\curveto(583.9933029,271.66685743)(584.02330287,271.66685743)(584.05331392,271.65685809)
\curveto(584.07330282,271.65685744)(584.10330279,271.65185745)(584.14331392,271.64185809)
\curveto(584.18330271,271.64185746)(584.21830267,271.64685745)(584.24831392,271.65685809)
\curveto(584.2883026,271.66685743)(584.32830256,271.67185743)(584.36831392,271.67185809)
\curveto(584.40830248,271.67185743)(584.44830244,271.68185742)(584.48831392,271.70185809)
\curveto(584.72830216,271.78185732)(584.92330197,271.91685718)(585.07331392,272.10685809)
\curveto(585.1933017,272.28685681)(585.28330161,272.49185661)(585.34331392,272.72185809)
\curveto(585.36330153,272.79185631)(585.37830151,272.86185624)(585.38831392,272.93185809)
\curveto(585.39830149,273.01185609)(585.41330148,273.09185601)(585.43331392,273.17185809)
\curveto(585.43330146,273.23185587)(585.43830145,273.27685582)(585.44831392,273.30685809)
\curveto(585.44830144,273.32685577)(585.44830144,273.35185575)(585.44831392,273.38185809)
\curveto(585.44830144,273.42185568)(585.45330144,273.45185565)(585.46331392,273.47185809)
\lineto(585.46331392,273.62185809)
}
}
{
\newrgbcolor{curcolor}{0 0 0}
\pscustom[linestyle=none,fillstyle=solid,fillcolor=curcolor]
{
\newpath
\moveto(478.87893038,144.25112078)
\curveto(480.50892494,144.28111013)(481.55892389,143.72611069)(482.02893038,142.58612078)
\curveto(482.12892332,142.35611206)(482.19392325,142.06611235)(482.22393038,141.71612078)
\curveto(482.26392318,141.37611304)(482.23892321,141.06611335)(482.14893038,140.78612078)
\curveto(482.05892339,140.52611389)(481.93892351,140.30111411)(481.78893038,140.11112078)
\curveto(481.76892368,140.07111434)(481.7439237,140.03611438)(481.71393038,140.00612078)
\curveto(481.68392376,139.98611443)(481.65892379,139.96111445)(481.63893038,139.93112078)
\lineto(481.54893038,139.81112078)
\curveto(481.51892393,139.78111463)(481.48392396,139.75611466)(481.44393038,139.73612078)
\curveto(481.39392405,139.68611473)(481.33892411,139.64111477)(481.27893038,139.60112078)
\curveto(481.22892422,139.56111485)(481.18392426,139.5111149)(481.14393038,139.45112078)
\curveto(481.10392434,139.411115)(481.08892436,139.36111505)(481.09893038,139.30112078)
\curveto(481.10892434,139.25111516)(481.13892431,139.20611521)(481.18893038,139.16612078)
\curveto(481.23892421,139.12611529)(481.29392415,139.08611533)(481.35393038,139.04612078)
\curveto(481.42392402,139.0161154)(481.48892396,138.98611543)(481.54893038,138.95612078)
\curveto(481.60892384,138.92611549)(481.65892379,138.89611552)(481.69893038,138.86612078)
\curveto(482.01892343,138.64611577)(482.27392317,138.33611608)(482.46393038,137.93612078)
\curveto(482.50392294,137.84611657)(482.53392291,137.75111666)(482.55393038,137.65112078)
\curveto(482.58392286,137.56111685)(482.60892284,137.47111694)(482.62893038,137.38112078)
\curveto(482.63892281,137.33111708)(482.6439228,137.28111713)(482.64393038,137.23112078)
\curveto(482.65392279,137.19111722)(482.66392278,137.14611727)(482.67393038,137.09612078)
\curveto(482.68392276,137.04611737)(482.68392276,136.99611742)(482.67393038,136.94612078)
\curveto(482.66392278,136.89611752)(482.66892278,136.84611757)(482.68893038,136.79612078)
\curveto(482.69892275,136.74611767)(482.70392274,136.68611773)(482.70393038,136.61612078)
\curveto(482.70392274,136.54611787)(482.69392275,136.48611793)(482.67393038,136.43612078)
\lineto(482.67393038,136.21112078)
\lineto(482.61393038,135.97112078)
\curveto(482.60392284,135.90111851)(482.58892286,135.83111858)(482.56893038,135.76112078)
\curveto(482.53892291,135.67111874)(482.50892294,135.58611883)(482.47893038,135.50612078)
\curveto(482.45892299,135.42611899)(482.42892302,135.34611907)(482.38893038,135.26612078)
\curveto(482.36892308,135.20611921)(482.33892311,135.14611927)(482.29893038,135.08612078)
\curveto(482.26892318,135.03611938)(482.23392321,134.98611943)(482.19393038,134.93612078)
\curveto(481.99392345,134.62611979)(481.7439237,134.36612005)(481.44393038,134.15612078)
\curveto(481.1439243,133.95612046)(480.79892465,133.79112062)(480.40893038,133.66112078)
\curveto(480.28892516,133.62112079)(480.15892529,133.59612082)(480.01893038,133.58612078)
\curveto(479.88892556,133.56612085)(479.75392569,133.54112087)(479.61393038,133.51112078)
\curveto(479.5439259,133.50112091)(479.47392597,133.49612092)(479.40393038,133.49612078)
\curveto(479.3439261,133.49612092)(479.27892617,133.49112092)(479.20893038,133.48112078)
\curveto(479.16892628,133.47112094)(479.10892634,133.46612095)(479.02893038,133.46612078)
\curveto(478.95892649,133.46612095)(478.90892654,133.47112094)(478.87893038,133.48112078)
\curveto(478.82892662,133.49112092)(478.78392666,133.49612092)(478.74393038,133.49612078)
\lineto(478.62393038,133.49612078)
\curveto(478.52392692,133.5161209)(478.42392702,133.53112088)(478.32393038,133.54112078)
\curveto(478.22392722,133.55112086)(478.12892732,133.56612085)(478.03893038,133.58612078)
\curveto(477.92892752,133.6161208)(477.81892763,133.64112077)(477.70893038,133.66112078)
\curveto(477.60892784,133.69112072)(477.50392794,133.73112068)(477.39393038,133.78112078)
\curveto(477.02392842,133.94112047)(476.70892874,134.14112027)(476.44893038,134.38112078)
\curveto(476.18892926,134.63111978)(475.97892947,134.94111947)(475.81893038,135.31112078)
\curveto(475.77892967,135.40111901)(475.7439297,135.49611892)(475.71393038,135.59612078)
\curveto(475.68392976,135.69611872)(475.65392979,135.80111861)(475.62393038,135.91112078)
\curveto(475.60392984,135.96111845)(475.59392985,136.0111184)(475.59393038,136.06112078)
\curveto(475.59392985,136.12111829)(475.58392986,136.18111823)(475.56393038,136.24112078)
\curveto(475.5439299,136.30111811)(475.53392991,136.38111803)(475.53393038,136.48112078)
\curveto(475.53392991,136.58111783)(475.5489299,136.65611776)(475.57893038,136.70612078)
\curveto(475.58892986,136.73611768)(475.60392984,136.76111765)(475.62393038,136.78112078)
\lineto(475.68393038,136.84112078)
\curveto(475.72392972,136.86111755)(475.78392966,136.87611754)(475.86393038,136.88612078)
\curveto(475.95392949,136.89611752)(476.0439294,136.90111751)(476.13393038,136.90112078)
\curveto(476.22392922,136.90111751)(476.30892914,136.89611752)(476.38893038,136.88612078)
\curveto(476.47892897,136.87611754)(476.5439289,136.86611755)(476.58393038,136.85612078)
\curveto(476.60392884,136.83611758)(476.62392882,136.82111759)(476.64393038,136.81112078)
\curveto(476.66392878,136.8111176)(476.68392876,136.80111761)(476.70393038,136.78112078)
\curveto(476.77392867,136.69111772)(476.81392863,136.57611784)(476.82393038,136.43612078)
\curveto(476.8439286,136.29611812)(476.87392857,136.17111824)(476.91393038,136.06112078)
\lineto(477.06393038,135.70112078)
\curveto(477.11392833,135.59111882)(477.17892827,135.48611893)(477.25893038,135.38612078)
\curveto(477.27892817,135.35611906)(477.29892815,135.33111908)(477.31893038,135.31112078)
\curveto(477.3489281,135.29111912)(477.37392807,135.26611915)(477.39393038,135.23612078)
\curveto(477.43392801,135.17611924)(477.46892798,135.13111928)(477.49893038,135.10112078)
\curveto(477.53892791,135.07111934)(477.57392787,135.04111937)(477.60393038,135.01112078)
\curveto(477.6439278,134.98111943)(477.68892776,134.95111946)(477.73893038,134.92112078)
\curveto(477.82892762,134.86111955)(477.92392752,134.8111196)(478.02393038,134.77112078)
\lineto(478.35393038,134.65112078)
\curveto(478.50392694,134.60111981)(478.70392674,134.57111984)(478.95393038,134.56112078)
\curveto(479.20392624,134.55111986)(479.41392603,134.57111984)(479.58393038,134.62112078)
\curveto(479.66392578,134.64111977)(479.73392571,134.65611976)(479.79393038,134.66612078)
\lineto(480.00393038,134.72612078)
\curveto(480.28392516,134.84611957)(480.52392492,134.99611942)(480.72393038,135.17612078)
\curveto(480.93392451,135.35611906)(481.09892435,135.58611883)(481.21893038,135.86612078)
\curveto(481.2489242,135.93611848)(481.26892418,136.00611841)(481.27893038,136.07612078)
\lineto(481.33893038,136.31612078)
\curveto(481.37892407,136.45611796)(481.38892406,136.6161178)(481.36893038,136.79612078)
\curveto(481.3489241,136.98611743)(481.31892413,137.13611728)(481.27893038,137.24612078)
\curveto(481.1489243,137.62611679)(480.96392448,137.9161165)(480.72393038,138.11612078)
\curveto(480.49392495,138.3161161)(480.18392526,138.47611594)(479.79393038,138.59612078)
\curveto(479.68392576,138.62611579)(479.56392588,138.64611577)(479.43393038,138.65612078)
\curveto(479.31392613,138.66611575)(479.18892626,138.67111574)(479.05893038,138.67112078)
\curveto(478.89892655,138.67111574)(478.75892669,138.67611574)(478.63893038,138.68612078)
\curveto(478.51892693,138.69611572)(478.43392701,138.75611566)(478.38393038,138.86612078)
\curveto(478.36392708,138.89611552)(478.35392709,138.93111548)(478.35393038,138.97112078)
\lineto(478.35393038,139.10612078)
\curveto(478.3439271,139.20611521)(478.3439271,139.30111511)(478.35393038,139.39112078)
\curveto(478.37392707,139.48111493)(478.41392703,139.54611487)(478.47393038,139.58612078)
\curveto(478.51392693,139.6161148)(478.55392689,139.63611478)(478.59393038,139.64612078)
\curveto(478.6439268,139.65611476)(478.69892675,139.66611475)(478.75893038,139.67612078)
\curveto(478.77892667,139.68611473)(478.80392664,139.68611473)(478.83393038,139.67612078)
\curveto(478.86392658,139.67611474)(478.88892656,139.68111473)(478.90893038,139.69112078)
\lineto(479.04393038,139.69112078)
\curveto(479.15392629,139.7111147)(479.25392619,139.72111469)(479.34393038,139.72112078)
\curveto(479.443926,139.73111468)(479.53892591,139.75111466)(479.62893038,139.78112078)
\curveto(479.9489255,139.89111452)(480.20392524,140.03611438)(480.39393038,140.21612078)
\curveto(480.58392486,140.39611402)(480.73392471,140.64611377)(480.84393038,140.96612078)
\curveto(480.87392457,141.06611335)(480.89392455,141.19111322)(480.90393038,141.34112078)
\curveto(480.92392452,141.50111291)(480.91892453,141.64611277)(480.88893038,141.77612078)
\curveto(480.86892458,141.84611257)(480.8489246,141.9111125)(480.82893038,141.97112078)
\curveto(480.81892463,142.04111237)(480.79892465,142.10611231)(480.76893038,142.16612078)
\curveto(480.66892478,142.40611201)(480.52392492,142.59611182)(480.33393038,142.73612078)
\curveto(480.1439253,142.87611154)(479.91892553,142.98611143)(479.65893038,143.06612078)
\curveto(479.59892585,143.08611133)(479.53892591,143.09611132)(479.47893038,143.09612078)
\curveto(479.41892603,143.09611132)(479.35392609,143.10611131)(479.28393038,143.12612078)
\curveto(479.20392624,143.14611127)(479.10892634,143.15611126)(478.99893038,143.15612078)
\curveto(478.88892656,143.15611126)(478.79392665,143.14611127)(478.71393038,143.12612078)
\curveto(478.66392678,143.10611131)(478.61392683,143.09611132)(478.56393038,143.09612078)
\curveto(478.52392692,143.09611132)(478.47892697,143.08611133)(478.42893038,143.06612078)
\curveto(478.2489272,143.0161114)(478.07892737,142.94111147)(477.91893038,142.84112078)
\curveto(477.76892768,142.75111166)(477.63892781,142.63611178)(477.52893038,142.49612078)
\curveto(477.43892801,142.37611204)(477.35892809,142.24611217)(477.28893038,142.10612078)
\curveto(477.21892823,141.96611245)(477.15392829,141.8111126)(477.09393038,141.64112078)
\curveto(477.06392838,141.53111288)(477.0439284,141.411113)(477.03393038,141.28112078)
\curveto(477.02392842,141.16111325)(476.98892846,141.06111335)(476.92893038,140.98112078)
\curveto(476.90892854,140.94111347)(476.8489286,140.90111351)(476.74893038,140.86112078)
\curveto(476.70892874,140.85111356)(476.6489288,140.84111357)(476.56893038,140.83112078)
\lineto(476.31393038,140.83112078)
\curveto(476.22392922,140.84111357)(476.13892931,140.85111356)(476.05893038,140.86112078)
\curveto(475.98892946,140.87111354)(475.93892951,140.88611353)(475.90893038,140.90612078)
\curveto(475.86892958,140.93611348)(475.83392961,140.99111342)(475.80393038,141.07112078)
\curveto(475.77392967,141.15111326)(475.76892968,141.23611318)(475.78893038,141.32612078)
\curveto(475.79892965,141.37611304)(475.80392964,141.42611299)(475.80393038,141.47612078)
\lineto(475.83393038,141.65612078)
\curveto(475.86392958,141.75611266)(475.88892956,141.85611256)(475.90893038,141.95612078)
\curveto(475.93892951,142.05611236)(475.97392947,142.14611227)(476.01393038,142.22612078)
\curveto(476.06392938,142.33611208)(476.10892934,142.44111197)(476.14893038,142.54112078)
\curveto(476.18892926,142.65111176)(476.23892921,142.75611166)(476.29893038,142.85612078)
\curveto(476.62892882,143.39611102)(477.09892835,143.79111062)(477.70893038,144.04112078)
\curveto(477.82892762,144.09111032)(477.95392749,144.12611029)(478.08393038,144.14612078)
\curveto(478.22392722,144.16611025)(478.36392708,144.19111022)(478.50393038,144.22112078)
\curveto(478.56392688,144.23111018)(478.62392682,144.23611018)(478.68393038,144.23612078)
\curveto(478.75392669,144.23611018)(478.81892663,144.24111017)(478.87893038,144.25112078)
}
}
{
\newrgbcolor{curcolor}{0 0 0}
\pscustom[linestyle=none,fillstyle=solid,fillcolor=curcolor]
{
\newpath
\moveto(487.22853975,144.25112078)
\curveto(488.85853431,144.28111013)(489.90853326,143.72611069)(490.37853975,142.58612078)
\curveto(490.47853269,142.35611206)(490.54353263,142.06611235)(490.57353975,141.71612078)
\curveto(490.61353256,141.37611304)(490.58853258,141.06611335)(490.49853975,140.78612078)
\curveto(490.40853276,140.52611389)(490.28853288,140.30111411)(490.13853975,140.11112078)
\curveto(490.11853305,140.07111434)(490.09353308,140.03611438)(490.06353975,140.00612078)
\curveto(490.03353314,139.98611443)(490.00853316,139.96111445)(489.98853975,139.93112078)
\lineto(489.89853975,139.81112078)
\curveto(489.8685333,139.78111463)(489.83353334,139.75611466)(489.79353975,139.73612078)
\curveto(489.74353343,139.68611473)(489.68853348,139.64111477)(489.62853975,139.60112078)
\curveto(489.57853359,139.56111485)(489.53353364,139.5111149)(489.49353975,139.45112078)
\curveto(489.45353372,139.411115)(489.43853373,139.36111505)(489.44853975,139.30112078)
\curveto(489.45853371,139.25111516)(489.48853368,139.20611521)(489.53853975,139.16612078)
\curveto(489.58853358,139.12611529)(489.64353353,139.08611533)(489.70353975,139.04612078)
\curveto(489.7735334,139.0161154)(489.83853333,138.98611543)(489.89853975,138.95612078)
\curveto(489.95853321,138.92611549)(490.00853316,138.89611552)(490.04853975,138.86612078)
\curveto(490.3685328,138.64611577)(490.62353255,138.33611608)(490.81353975,137.93612078)
\curveto(490.85353232,137.84611657)(490.88353229,137.75111666)(490.90353975,137.65112078)
\curveto(490.93353224,137.56111685)(490.95853221,137.47111694)(490.97853975,137.38112078)
\curveto(490.98853218,137.33111708)(490.99353218,137.28111713)(490.99353975,137.23112078)
\curveto(491.00353217,137.19111722)(491.01353216,137.14611727)(491.02353975,137.09612078)
\curveto(491.03353214,137.04611737)(491.03353214,136.99611742)(491.02353975,136.94612078)
\curveto(491.01353216,136.89611752)(491.01853215,136.84611757)(491.03853975,136.79612078)
\curveto(491.04853212,136.74611767)(491.05353212,136.68611773)(491.05353975,136.61612078)
\curveto(491.05353212,136.54611787)(491.04353213,136.48611793)(491.02353975,136.43612078)
\lineto(491.02353975,136.21112078)
\lineto(490.96353975,135.97112078)
\curveto(490.95353222,135.90111851)(490.93853223,135.83111858)(490.91853975,135.76112078)
\curveto(490.88853228,135.67111874)(490.85853231,135.58611883)(490.82853975,135.50612078)
\curveto(490.80853236,135.42611899)(490.77853239,135.34611907)(490.73853975,135.26612078)
\curveto(490.71853245,135.20611921)(490.68853248,135.14611927)(490.64853975,135.08612078)
\curveto(490.61853255,135.03611938)(490.58353259,134.98611943)(490.54353975,134.93612078)
\curveto(490.34353283,134.62611979)(490.09353308,134.36612005)(489.79353975,134.15612078)
\curveto(489.49353368,133.95612046)(489.14853402,133.79112062)(488.75853975,133.66112078)
\curveto(488.63853453,133.62112079)(488.50853466,133.59612082)(488.36853975,133.58612078)
\curveto(488.23853493,133.56612085)(488.10353507,133.54112087)(487.96353975,133.51112078)
\curveto(487.89353528,133.50112091)(487.82353535,133.49612092)(487.75353975,133.49612078)
\curveto(487.69353548,133.49612092)(487.62853554,133.49112092)(487.55853975,133.48112078)
\curveto(487.51853565,133.47112094)(487.45853571,133.46612095)(487.37853975,133.46612078)
\curveto(487.30853586,133.46612095)(487.25853591,133.47112094)(487.22853975,133.48112078)
\curveto(487.17853599,133.49112092)(487.13353604,133.49612092)(487.09353975,133.49612078)
\lineto(486.97353975,133.49612078)
\curveto(486.8735363,133.5161209)(486.7735364,133.53112088)(486.67353975,133.54112078)
\curveto(486.5735366,133.55112086)(486.47853669,133.56612085)(486.38853975,133.58612078)
\curveto(486.27853689,133.6161208)(486.168537,133.64112077)(486.05853975,133.66112078)
\curveto(485.95853721,133.69112072)(485.85353732,133.73112068)(485.74353975,133.78112078)
\curveto(485.3735378,133.94112047)(485.05853811,134.14112027)(484.79853975,134.38112078)
\curveto(484.53853863,134.63111978)(484.32853884,134.94111947)(484.16853975,135.31112078)
\curveto(484.12853904,135.40111901)(484.09353908,135.49611892)(484.06353975,135.59612078)
\curveto(484.03353914,135.69611872)(484.00353917,135.80111861)(483.97353975,135.91112078)
\curveto(483.95353922,135.96111845)(483.94353923,136.0111184)(483.94353975,136.06112078)
\curveto(483.94353923,136.12111829)(483.93353924,136.18111823)(483.91353975,136.24112078)
\curveto(483.89353928,136.30111811)(483.88353929,136.38111803)(483.88353975,136.48112078)
\curveto(483.88353929,136.58111783)(483.89853927,136.65611776)(483.92853975,136.70612078)
\curveto(483.93853923,136.73611768)(483.95353922,136.76111765)(483.97353975,136.78112078)
\lineto(484.03353975,136.84112078)
\curveto(484.0735391,136.86111755)(484.13353904,136.87611754)(484.21353975,136.88612078)
\curveto(484.30353887,136.89611752)(484.39353878,136.90111751)(484.48353975,136.90112078)
\curveto(484.5735386,136.90111751)(484.65853851,136.89611752)(484.73853975,136.88612078)
\curveto(484.82853834,136.87611754)(484.89353828,136.86611755)(484.93353975,136.85612078)
\curveto(484.95353822,136.83611758)(484.9735382,136.82111759)(484.99353975,136.81112078)
\curveto(485.01353816,136.8111176)(485.03353814,136.80111761)(485.05353975,136.78112078)
\curveto(485.12353805,136.69111772)(485.16353801,136.57611784)(485.17353975,136.43612078)
\curveto(485.19353798,136.29611812)(485.22353795,136.17111824)(485.26353975,136.06112078)
\lineto(485.41353975,135.70112078)
\curveto(485.46353771,135.59111882)(485.52853764,135.48611893)(485.60853975,135.38612078)
\curveto(485.62853754,135.35611906)(485.64853752,135.33111908)(485.66853975,135.31112078)
\curveto(485.69853747,135.29111912)(485.72353745,135.26611915)(485.74353975,135.23612078)
\curveto(485.78353739,135.17611924)(485.81853735,135.13111928)(485.84853975,135.10112078)
\curveto(485.88853728,135.07111934)(485.92353725,135.04111937)(485.95353975,135.01112078)
\curveto(485.99353718,134.98111943)(486.03853713,134.95111946)(486.08853975,134.92112078)
\curveto(486.17853699,134.86111955)(486.2735369,134.8111196)(486.37353975,134.77112078)
\lineto(486.70353975,134.65112078)
\curveto(486.85353632,134.60111981)(487.05353612,134.57111984)(487.30353975,134.56112078)
\curveto(487.55353562,134.55111986)(487.76353541,134.57111984)(487.93353975,134.62112078)
\curveto(488.01353516,134.64111977)(488.08353509,134.65611976)(488.14353975,134.66612078)
\lineto(488.35353975,134.72612078)
\curveto(488.63353454,134.84611957)(488.8735343,134.99611942)(489.07353975,135.17612078)
\curveto(489.28353389,135.35611906)(489.44853372,135.58611883)(489.56853975,135.86612078)
\curveto(489.59853357,135.93611848)(489.61853355,136.00611841)(489.62853975,136.07612078)
\lineto(489.68853975,136.31612078)
\curveto(489.72853344,136.45611796)(489.73853343,136.6161178)(489.71853975,136.79612078)
\curveto(489.69853347,136.98611743)(489.6685335,137.13611728)(489.62853975,137.24612078)
\curveto(489.49853367,137.62611679)(489.31353386,137.9161165)(489.07353975,138.11612078)
\curveto(488.84353433,138.3161161)(488.53353464,138.47611594)(488.14353975,138.59612078)
\curveto(488.03353514,138.62611579)(487.91353526,138.64611577)(487.78353975,138.65612078)
\curveto(487.66353551,138.66611575)(487.53853563,138.67111574)(487.40853975,138.67112078)
\curveto(487.24853592,138.67111574)(487.10853606,138.67611574)(486.98853975,138.68612078)
\curveto(486.8685363,138.69611572)(486.78353639,138.75611566)(486.73353975,138.86612078)
\curveto(486.71353646,138.89611552)(486.70353647,138.93111548)(486.70353975,138.97112078)
\lineto(486.70353975,139.10612078)
\curveto(486.69353648,139.20611521)(486.69353648,139.30111511)(486.70353975,139.39112078)
\curveto(486.72353645,139.48111493)(486.76353641,139.54611487)(486.82353975,139.58612078)
\curveto(486.86353631,139.6161148)(486.90353627,139.63611478)(486.94353975,139.64612078)
\curveto(486.99353618,139.65611476)(487.04853612,139.66611475)(487.10853975,139.67612078)
\curveto(487.12853604,139.68611473)(487.15353602,139.68611473)(487.18353975,139.67612078)
\curveto(487.21353596,139.67611474)(487.23853593,139.68111473)(487.25853975,139.69112078)
\lineto(487.39353975,139.69112078)
\curveto(487.50353567,139.7111147)(487.60353557,139.72111469)(487.69353975,139.72112078)
\curveto(487.79353538,139.73111468)(487.88853528,139.75111466)(487.97853975,139.78112078)
\curveto(488.29853487,139.89111452)(488.55353462,140.03611438)(488.74353975,140.21612078)
\curveto(488.93353424,140.39611402)(489.08353409,140.64611377)(489.19353975,140.96612078)
\curveto(489.22353395,141.06611335)(489.24353393,141.19111322)(489.25353975,141.34112078)
\curveto(489.2735339,141.50111291)(489.2685339,141.64611277)(489.23853975,141.77612078)
\curveto(489.21853395,141.84611257)(489.19853397,141.9111125)(489.17853975,141.97112078)
\curveto(489.168534,142.04111237)(489.14853402,142.10611231)(489.11853975,142.16612078)
\curveto(489.01853415,142.40611201)(488.8735343,142.59611182)(488.68353975,142.73612078)
\curveto(488.49353468,142.87611154)(488.2685349,142.98611143)(488.00853975,143.06612078)
\curveto(487.94853522,143.08611133)(487.88853528,143.09611132)(487.82853975,143.09612078)
\curveto(487.7685354,143.09611132)(487.70353547,143.10611131)(487.63353975,143.12612078)
\curveto(487.55353562,143.14611127)(487.45853571,143.15611126)(487.34853975,143.15612078)
\curveto(487.23853593,143.15611126)(487.14353603,143.14611127)(487.06353975,143.12612078)
\curveto(487.01353616,143.10611131)(486.96353621,143.09611132)(486.91353975,143.09612078)
\curveto(486.8735363,143.09611132)(486.82853634,143.08611133)(486.77853975,143.06612078)
\curveto(486.59853657,143.0161114)(486.42853674,142.94111147)(486.26853975,142.84112078)
\curveto(486.11853705,142.75111166)(485.98853718,142.63611178)(485.87853975,142.49612078)
\curveto(485.78853738,142.37611204)(485.70853746,142.24611217)(485.63853975,142.10612078)
\curveto(485.5685376,141.96611245)(485.50353767,141.8111126)(485.44353975,141.64112078)
\curveto(485.41353776,141.53111288)(485.39353778,141.411113)(485.38353975,141.28112078)
\curveto(485.3735378,141.16111325)(485.33853783,141.06111335)(485.27853975,140.98112078)
\curveto(485.25853791,140.94111347)(485.19853797,140.90111351)(485.09853975,140.86112078)
\curveto(485.05853811,140.85111356)(484.99853817,140.84111357)(484.91853975,140.83112078)
\lineto(484.66353975,140.83112078)
\curveto(484.5735386,140.84111357)(484.48853868,140.85111356)(484.40853975,140.86112078)
\curveto(484.33853883,140.87111354)(484.28853888,140.88611353)(484.25853975,140.90612078)
\curveto(484.21853895,140.93611348)(484.18353899,140.99111342)(484.15353975,141.07112078)
\curveto(484.12353905,141.15111326)(484.11853905,141.23611318)(484.13853975,141.32612078)
\curveto(484.14853902,141.37611304)(484.15353902,141.42611299)(484.15353975,141.47612078)
\lineto(484.18353975,141.65612078)
\curveto(484.21353896,141.75611266)(484.23853893,141.85611256)(484.25853975,141.95612078)
\curveto(484.28853888,142.05611236)(484.32353885,142.14611227)(484.36353975,142.22612078)
\curveto(484.41353876,142.33611208)(484.45853871,142.44111197)(484.49853975,142.54112078)
\curveto(484.53853863,142.65111176)(484.58853858,142.75611166)(484.64853975,142.85612078)
\curveto(484.97853819,143.39611102)(485.44853772,143.79111062)(486.05853975,144.04112078)
\curveto(486.17853699,144.09111032)(486.30353687,144.12611029)(486.43353975,144.14612078)
\curveto(486.5735366,144.16611025)(486.71353646,144.19111022)(486.85353975,144.22112078)
\curveto(486.91353626,144.23111018)(486.9735362,144.23611018)(487.03353975,144.23612078)
\curveto(487.10353607,144.23611018)(487.168536,144.24111017)(487.22853975,144.25112078)
}
}
{
\newrgbcolor{curcolor}{0 0 0}
\pscustom[linestyle=none,fillstyle=solid,fillcolor=curcolor]
{
\newpath
\moveto(493.41814913,135.28112078)
\lineto(493.71814913,135.28112078)
\curveto(493.82814707,135.29111912)(493.93314696,135.29111912)(494.03314913,135.28112078)
\curveto(494.14314675,135.28111913)(494.24314665,135.27111914)(494.33314913,135.25112078)
\curveto(494.42314647,135.24111917)(494.4931464,135.2161192)(494.54314913,135.17612078)
\curveto(494.56314633,135.15611926)(494.57814632,135.12611929)(494.58814913,135.08612078)
\curveto(494.60814629,135.04611937)(494.62814627,135.00111941)(494.64814913,134.95112078)
\lineto(494.64814913,134.87612078)
\curveto(494.65814624,134.82611959)(494.65814624,134.77111964)(494.64814913,134.71112078)
\lineto(494.64814913,134.56112078)
\lineto(494.64814913,134.08112078)
\curveto(494.64814625,133.9111205)(494.60814629,133.79112062)(494.52814913,133.72112078)
\curveto(494.45814644,133.67112074)(494.36814653,133.64612077)(494.25814913,133.64612078)
\lineto(493.92814913,133.64612078)
\lineto(493.47814913,133.64612078)
\curveto(493.32814757,133.64612077)(493.21314768,133.67612074)(493.13314913,133.73612078)
\curveto(493.0931478,133.76612065)(493.06314783,133.8161206)(493.04314913,133.88612078)
\curveto(493.02314787,133.96612045)(493.00814789,134.05112036)(492.99814913,134.14112078)
\lineto(492.99814913,134.42612078)
\curveto(493.00814789,134.52611989)(493.01314788,134.6111198)(493.01314913,134.68112078)
\lineto(493.01314913,134.87612078)
\curveto(493.01314788,134.93611948)(493.02314787,134.99111942)(493.04314913,135.04112078)
\curveto(493.08314781,135.15111926)(493.15314774,135.22111919)(493.25314913,135.25112078)
\curveto(493.28314761,135.25111916)(493.33814756,135.26111915)(493.41814913,135.28112078)
}
}
{
\newrgbcolor{curcolor}{0 0 0}
\pscustom[linestyle=none,fillstyle=solid,fillcolor=curcolor]
{
\newpath
\moveto(499.73830538,144.25112078)
\curveto(501.36829994,144.28111013)(502.41829889,143.72611069)(502.88830538,142.58612078)
\curveto(502.98829832,142.35611206)(503.05329825,142.06611235)(503.08330538,141.71612078)
\curveto(503.12329818,141.37611304)(503.09829821,141.06611335)(503.00830538,140.78612078)
\curveto(502.91829839,140.52611389)(502.79829851,140.30111411)(502.64830538,140.11112078)
\curveto(502.62829868,140.07111434)(502.6032987,140.03611438)(502.57330538,140.00612078)
\curveto(502.54329876,139.98611443)(502.51829879,139.96111445)(502.49830538,139.93112078)
\lineto(502.40830538,139.81112078)
\curveto(502.37829893,139.78111463)(502.34329896,139.75611466)(502.30330538,139.73612078)
\curveto(502.25329905,139.68611473)(502.19829911,139.64111477)(502.13830538,139.60112078)
\curveto(502.08829922,139.56111485)(502.04329926,139.5111149)(502.00330538,139.45112078)
\curveto(501.96329934,139.411115)(501.94829936,139.36111505)(501.95830538,139.30112078)
\curveto(501.96829934,139.25111516)(501.99829931,139.20611521)(502.04830538,139.16612078)
\curveto(502.09829921,139.12611529)(502.15329915,139.08611533)(502.21330538,139.04612078)
\curveto(502.28329902,139.0161154)(502.34829896,138.98611543)(502.40830538,138.95612078)
\curveto(502.46829884,138.92611549)(502.51829879,138.89611552)(502.55830538,138.86612078)
\curveto(502.87829843,138.64611577)(503.13329817,138.33611608)(503.32330538,137.93612078)
\curveto(503.36329794,137.84611657)(503.39329791,137.75111666)(503.41330538,137.65112078)
\curveto(503.44329786,137.56111685)(503.46829784,137.47111694)(503.48830538,137.38112078)
\curveto(503.49829781,137.33111708)(503.5032978,137.28111713)(503.50330538,137.23112078)
\curveto(503.51329779,137.19111722)(503.52329778,137.14611727)(503.53330538,137.09612078)
\curveto(503.54329776,137.04611737)(503.54329776,136.99611742)(503.53330538,136.94612078)
\curveto(503.52329778,136.89611752)(503.52829778,136.84611757)(503.54830538,136.79612078)
\curveto(503.55829775,136.74611767)(503.56329774,136.68611773)(503.56330538,136.61612078)
\curveto(503.56329774,136.54611787)(503.55329775,136.48611793)(503.53330538,136.43612078)
\lineto(503.53330538,136.21112078)
\lineto(503.47330538,135.97112078)
\curveto(503.46329784,135.90111851)(503.44829786,135.83111858)(503.42830538,135.76112078)
\curveto(503.39829791,135.67111874)(503.36829794,135.58611883)(503.33830538,135.50612078)
\curveto(503.31829799,135.42611899)(503.28829802,135.34611907)(503.24830538,135.26612078)
\curveto(503.22829808,135.20611921)(503.19829811,135.14611927)(503.15830538,135.08612078)
\curveto(503.12829818,135.03611938)(503.09329821,134.98611943)(503.05330538,134.93612078)
\curveto(502.85329845,134.62611979)(502.6032987,134.36612005)(502.30330538,134.15612078)
\curveto(502.0032993,133.95612046)(501.65829965,133.79112062)(501.26830538,133.66112078)
\curveto(501.14830016,133.62112079)(501.01830029,133.59612082)(500.87830538,133.58612078)
\curveto(500.74830056,133.56612085)(500.61330069,133.54112087)(500.47330538,133.51112078)
\curveto(500.4033009,133.50112091)(500.33330097,133.49612092)(500.26330538,133.49612078)
\curveto(500.2033011,133.49612092)(500.13830117,133.49112092)(500.06830538,133.48112078)
\curveto(500.02830128,133.47112094)(499.96830134,133.46612095)(499.88830538,133.46612078)
\curveto(499.81830149,133.46612095)(499.76830154,133.47112094)(499.73830538,133.48112078)
\curveto(499.68830162,133.49112092)(499.64330166,133.49612092)(499.60330538,133.49612078)
\lineto(499.48330538,133.49612078)
\curveto(499.38330192,133.5161209)(499.28330202,133.53112088)(499.18330538,133.54112078)
\curveto(499.08330222,133.55112086)(498.98830232,133.56612085)(498.89830538,133.58612078)
\curveto(498.78830252,133.6161208)(498.67830263,133.64112077)(498.56830538,133.66112078)
\curveto(498.46830284,133.69112072)(498.36330294,133.73112068)(498.25330538,133.78112078)
\curveto(497.88330342,133.94112047)(497.56830374,134.14112027)(497.30830538,134.38112078)
\curveto(497.04830426,134.63111978)(496.83830447,134.94111947)(496.67830538,135.31112078)
\curveto(496.63830467,135.40111901)(496.6033047,135.49611892)(496.57330538,135.59612078)
\curveto(496.54330476,135.69611872)(496.51330479,135.80111861)(496.48330538,135.91112078)
\curveto(496.46330484,135.96111845)(496.45330485,136.0111184)(496.45330538,136.06112078)
\curveto(496.45330485,136.12111829)(496.44330486,136.18111823)(496.42330538,136.24112078)
\curveto(496.4033049,136.30111811)(496.39330491,136.38111803)(496.39330538,136.48112078)
\curveto(496.39330491,136.58111783)(496.4083049,136.65611776)(496.43830538,136.70612078)
\curveto(496.44830486,136.73611768)(496.46330484,136.76111765)(496.48330538,136.78112078)
\lineto(496.54330538,136.84112078)
\curveto(496.58330472,136.86111755)(496.64330466,136.87611754)(496.72330538,136.88612078)
\curveto(496.81330449,136.89611752)(496.9033044,136.90111751)(496.99330538,136.90112078)
\curveto(497.08330422,136.90111751)(497.16830414,136.89611752)(497.24830538,136.88612078)
\curveto(497.33830397,136.87611754)(497.4033039,136.86611755)(497.44330538,136.85612078)
\curveto(497.46330384,136.83611758)(497.48330382,136.82111759)(497.50330538,136.81112078)
\curveto(497.52330378,136.8111176)(497.54330376,136.80111761)(497.56330538,136.78112078)
\curveto(497.63330367,136.69111772)(497.67330363,136.57611784)(497.68330538,136.43612078)
\curveto(497.7033036,136.29611812)(497.73330357,136.17111824)(497.77330538,136.06112078)
\lineto(497.92330538,135.70112078)
\curveto(497.97330333,135.59111882)(498.03830327,135.48611893)(498.11830538,135.38612078)
\curveto(498.13830317,135.35611906)(498.15830315,135.33111908)(498.17830538,135.31112078)
\curveto(498.2083031,135.29111912)(498.23330307,135.26611915)(498.25330538,135.23612078)
\curveto(498.29330301,135.17611924)(498.32830298,135.13111928)(498.35830538,135.10112078)
\curveto(498.39830291,135.07111934)(498.43330287,135.04111937)(498.46330538,135.01112078)
\curveto(498.5033028,134.98111943)(498.54830276,134.95111946)(498.59830538,134.92112078)
\curveto(498.68830262,134.86111955)(498.78330252,134.8111196)(498.88330538,134.77112078)
\lineto(499.21330538,134.65112078)
\curveto(499.36330194,134.60111981)(499.56330174,134.57111984)(499.81330538,134.56112078)
\curveto(500.06330124,134.55111986)(500.27330103,134.57111984)(500.44330538,134.62112078)
\curveto(500.52330078,134.64111977)(500.59330071,134.65611976)(500.65330538,134.66612078)
\lineto(500.86330538,134.72612078)
\curveto(501.14330016,134.84611957)(501.38329992,134.99611942)(501.58330538,135.17612078)
\curveto(501.79329951,135.35611906)(501.95829935,135.58611883)(502.07830538,135.86612078)
\curveto(502.1082992,135.93611848)(502.12829918,136.00611841)(502.13830538,136.07612078)
\lineto(502.19830538,136.31612078)
\curveto(502.23829907,136.45611796)(502.24829906,136.6161178)(502.22830538,136.79612078)
\curveto(502.2082991,136.98611743)(502.17829913,137.13611728)(502.13830538,137.24612078)
\curveto(502.0082993,137.62611679)(501.82329948,137.9161165)(501.58330538,138.11612078)
\curveto(501.35329995,138.3161161)(501.04330026,138.47611594)(500.65330538,138.59612078)
\curveto(500.54330076,138.62611579)(500.42330088,138.64611577)(500.29330538,138.65612078)
\curveto(500.17330113,138.66611575)(500.04830126,138.67111574)(499.91830538,138.67112078)
\curveto(499.75830155,138.67111574)(499.61830169,138.67611574)(499.49830538,138.68612078)
\curveto(499.37830193,138.69611572)(499.29330201,138.75611566)(499.24330538,138.86612078)
\curveto(499.22330208,138.89611552)(499.21330209,138.93111548)(499.21330538,138.97112078)
\lineto(499.21330538,139.10612078)
\curveto(499.2033021,139.20611521)(499.2033021,139.30111511)(499.21330538,139.39112078)
\curveto(499.23330207,139.48111493)(499.27330203,139.54611487)(499.33330538,139.58612078)
\curveto(499.37330193,139.6161148)(499.41330189,139.63611478)(499.45330538,139.64612078)
\curveto(499.5033018,139.65611476)(499.55830175,139.66611475)(499.61830538,139.67612078)
\curveto(499.63830167,139.68611473)(499.66330164,139.68611473)(499.69330538,139.67612078)
\curveto(499.72330158,139.67611474)(499.74830156,139.68111473)(499.76830538,139.69112078)
\lineto(499.90330538,139.69112078)
\curveto(500.01330129,139.7111147)(500.11330119,139.72111469)(500.20330538,139.72112078)
\curveto(500.303301,139.73111468)(500.39830091,139.75111466)(500.48830538,139.78112078)
\curveto(500.8083005,139.89111452)(501.06330024,140.03611438)(501.25330538,140.21612078)
\curveto(501.44329986,140.39611402)(501.59329971,140.64611377)(501.70330538,140.96612078)
\curveto(501.73329957,141.06611335)(501.75329955,141.19111322)(501.76330538,141.34112078)
\curveto(501.78329952,141.50111291)(501.77829953,141.64611277)(501.74830538,141.77612078)
\curveto(501.72829958,141.84611257)(501.7082996,141.9111125)(501.68830538,141.97112078)
\curveto(501.67829963,142.04111237)(501.65829965,142.10611231)(501.62830538,142.16612078)
\curveto(501.52829978,142.40611201)(501.38329992,142.59611182)(501.19330538,142.73612078)
\curveto(501.0033003,142.87611154)(500.77830053,142.98611143)(500.51830538,143.06612078)
\curveto(500.45830085,143.08611133)(500.39830091,143.09611132)(500.33830538,143.09612078)
\curveto(500.27830103,143.09611132)(500.21330109,143.10611131)(500.14330538,143.12612078)
\curveto(500.06330124,143.14611127)(499.96830134,143.15611126)(499.85830538,143.15612078)
\curveto(499.74830156,143.15611126)(499.65330165,143.14611127)(499.57330538,143.12612078)
\curveto(499.52330178,143.10611131)(499.47330183,143.09611132)(499.42330538,143.09612078)
\curveto(499.38330192,143.09611132)(499.33830197,143.08611133)(499.28830538,143.06612078)
\curveto(499.1083022,143.0161114)(498.93830237,142.94111147)(498.77830538,142.84112078)
\curveto(498.62830268,142.75111166)(498.49830281,142.63611178)(498.38830538,142.49612078)
\curveto(498.29830301,142.37611204)(498.21830309,142.24611217)(498.14830538,142.10612078)
\curveto(498.07830323,141.96611245)(498.01330329,141.8111126)(497.95330538,141.64112078)
\curveto(497.92330338,141.53111288)(497.9033034,141.411113)(497.89330538,141.28112078)
\curveto(497.88330342,141.16111325)(497.84830346,141.06111335)(497.78830538,140.98112078)
\curveto(497.76830354,140.94111347)(497.7083036,140.90111351)(497.60830538,140.86112078)
\curveto(497.56830374,140.85111356)(497.5083038,140.84111357)(497.42830538,140.83112078)
\lineto(497.17330538,140.83112078)
\curveto(497.08330422,140.84111357)(496.99830431,140.85111356)(496.91830538,140.86112078)
\curveto(496.84830446,140.87111354)(496.79830451,140.88611353)(496.76830538,140.90612078)
\curveto(496.72830458,140.93611348)(496.69330461,140.99111342)(496.66330538,141.07112078)
\curveto(496.63330467,141.15111326)(496.62830468,141.23611318)(496.64830538,141.32612078)
\curveto(496.65830465,141.37611304)(496.66330464,141.42611299)(496.66330538,141.47612078)
\lineto(496.69330538,141.65612078)
\curveto(496.72330458,141.75611266)(496.74830456,141.85611256)(496.76830538,141.95612078)
\curveto(496.79830451,142.05611236)(496.83330447,142.14611227)(496.87330538,142.22612078)
\curveto(496.92330438,142.33611208)(496.96830434,142.44111197)(497.00830538,142.54112078)
\curveto(497.04830426,142.65111176)(497.09830421,142.75611166)(497.15830538,142.85612078)
\curveto(497.48830382,143.39611102)(497.95830335,143.79111062)(498.56830538,144.04112078)
\curveto(498.68830262,144.09111032)(498.81330249,144.12611029)(498.94330538,144.14612078)
\curveto(499.08330222,144.16611025)(499.22330208,144.19111022)(499.36330538,144.22112078)
\curveto(499.42330188,144.23111018)(499.48330182,144.23611018)(499.54330538,144.23612078)
\curveto(499.61330169,144.23611018)(499.67830163,144.24111017)(499.73830538,144.25112078)
}
}
{
\newrgbcolor{curcolor}{0 0 0}
\pscustom[linestyle=none,fillstyle=solid,fillcolor=curcolor]
{
\newpath
\moveto(514.77791475,142.16612078)
\curveto(514.57790445,141.87611254)(514.36790466,141.59111282)(514.14791475,141.31112078)
\curveto(513.93790509,141.03111338)(513.7329053,140.74611367)(513.53291475,140.45612078)
\curveto(512.9329061,139.60611481)(512.3279067,138.76611565)(511.71791475,137.93612078)
\curveto(511.10790792,137.1161173)(510.50290853,136.28111813)(509.90291475,135.43112078)
\lineto(509.39291475,134.71112078)
\lineto(508.88291475,134.02112078)
\curveto(508.80291023,133.9111205)(508.72291031,133.79612062)(508.64291475,133.67612078)
\curveto(508.56291047,133.55612086)(508.46791056,133.46112095)(508.35791475,133.39112078)
\curveto(508.31791071,133.37112104)(508.25291078,133.35612106)(508.16291475,133.34612078)
\curveto(508.08291095,133.32612109)(507.99291104,133.3161211)(507.89291475,133.31612078)
\curveto(507.79291124,133.3161211)(507.69791133,133.32112109)(507.60791475,133.33112078)
\curveto(507.5279115,133.34112107)(507.46791156,133.36112105)(507.42791475,133.39112078)
\curveto(507.39791163,133.411121)(507.37291166,133.44612097)(507.35291475,133.49612078)
\curveto(507.34291169,133.53612088)(507.34791168,133.58112083)(507.36791475,133.63112078)
\curveto(507.40791162,133.7111207)(507.45291158,133.78612063)(507.50291475,133.85612078)
\curveto(507.56291147,133.93612048)(507.61791141,134.0161204)(507.66791475,134.09612078)
\curveto(507.90791112,134.43611998)(508.15291088,134.77111964)(508.40291475,135.10112078)
\curveto(508.65291038,135.43111898)(508.89291014,135.76611865)(509.12291475,136.10612078)
\curveto(509.28290975,136.32611809)(509.44290959,136.54111787)(509.60291475,136.75112078)
\curveto(509.76290927,136.96111745)(509.92290911,137.17611724)(510.08291475,137.39612078)
\curveto(510.44290859,137.9161165)(510.80790822,138.42611599)(511.17791475,138.92612078)
\curveto(511.54790748,139.42611499)(511.91790711,139.93611448)(512.28791475,140.45612078)
\curveto(512.4279066,140.65611376)(512.56790646,140.85111356)(512.70791475,141.04112078)
\curveto(512.85790617,141.23111318)(513.00290603,141.42611299)(513.14291475,141.62612078)
\curveto(513.35290568,141.92611249)(513.56790546,142.22611219)(513.78791475,142.52612078)
\lineto(514.44791475,143.42612078)
\lineto(514.62791475,143.69612078)
\lineto(514.83791475,143.96612078)
\lineto(514.95791475,144.14612078)
\curveto(515.00790402,144.20611021)(515.05790397,144.26111015)(515.10791475,144.31112078)
\curveto(515.17790385,144.36111005)(515.25290378,144.39611002)(515.33291475,144.41612078)
\curveto(515.35290368,144.42610999)(515.37790365,144.42610999)(515.40791475,144.41612078)
\curveto(515.44790358,144.41611)(515.47790355,144.42610999)(515.49791475,144.44612078)
\curveto(515.61790341,144.44610997)(515.75290328,144.44110997)(515.90291475,144.43112078)
\curveto(516.05290298,144.43110998)(516.14290289,144.38611003)(516.17291475,144.29612078)
\curveto(516.19290284,144.26611015)(516.19790283,144.23111018)(516.18791475,144.19112078)
\curveto(516.17790285,144.15111026)(516.16290287,144.12111029)(516.14291475,144.10112078)
\curveto(516.10290293,144.02111039)(516.06290297,143.95111046)(516.02291475,143.89112078)
\curveto(515.98290305,143.83111058)(515.93790309,143.77111064)(515.88791475,143.71112078)
\lineto(515.31791475,142.93112078)
\curveto(515.13790389,142.68111173)(514.95790407,142.42611199)(514.77791475,142.16612078)
\moveto(507.92291475,138.26612078)
\curveto(507.87291116,138.28611613)(507.82291121,138.29111612)(507.77291475,138.28112078)
\curveto(507.72291131,138.27111614)(507.67291136,138.27611614)(507.62291475,138.29612078)
\curveto(507.51291152,138.3161161)(507.40791162,138.33611608)(507.30791475,138.35612078)
\curveto(507.21791181,138.38611603)(507.12291191,138.42611599)(507.02291475,138.47612078)
\curveto(506.69291234,138.6161158)(506.43791259,138.8111156)(506.25791475,139.06112078)
\curveto(506.07791295,139.32111509)(505.9329131,139.63111478)(505.82291475,139.99112078)
\curveto(505.79291324,140.07111434)(505.77291326,140.15111426)(505.76291475,140.23112078)
\curveto(505.75291328,140.32111409)(505.73791329,140.40611401)(505.71791475,140.48612078)
\curveto(505.70791332,140.53611388)(505.70291333,140.60111381)(505.70291475,140.68112078)
\curveto(505.69291334,140.7111137)(505.68791334,140.74111367)(505.68791475,140.77112078)
\curveto(505.68791334,140.8111136)(505.68291335,140.84611357)(505.67291475,140.87612078)
\lineto(505.67291475,141.02612078)
\curveto(505.66291337,141.07611334)(505.65791337,141.13611328)(505.65791475,141.20612078)
\curveto(505.65791337,141.28611313)(505.66291337,141.35111306)(505.67291475,141.40112078)
\lineto(505.67291475,141.56612078)
\curveto(505.69291334,141.6161128)(505.69791333,141.66111275)(505.68791475,141.70112078)
\curveto(505.68791334,141.75111266)(505.69291334,141.79611262)(505.70291475,141.83612078)
\curveto(505.71291332,141.87611254)(505.71791331,141.9111125)(505.71791475,141.94112078)
\curveto(505.71791331,141.98111243)(505.72291331,142.02111239)(505.73291475,142.06112078)
\curveto(505.76291327,142.17111224)(505.78291325,142.28111213)(505.79291475,142.39112078)
\curveto(505.81291322,142.5111119)(505.84791318,142.62611179)(505.89791475,142.73612078)
\curveto(506.03791299,143.07611134)(506.19791283,143.35111106)(506.37791475,143.56112078)
\curveto(506.56791246,143.78111063)(506.83791219,143.96111045)(507.18791475,144.10112078)
\curveto(507.26791176,144.13111028)(507.35291168,144.15111026)(507.44291475,144.16112078)
\curveto(507.5329115,144.18111023)(507.6279114,144.20111021)(507.72791475,144.22112078)
\curveto(507.75791127,144.23111018)(507.81291122,144.23111018)(507.89291475,144.22112078)
\curveto(507.97291106,144.22111019)(508.02291101,144.23111018)(508.04291475,144.25112078)
\curveto(508.60291043,144.26111015)(509.05290998,144.15111026)(509.39291475,143.92112078)
\curveto(509.74290929,143.69111072)(510.00290903,143.38611103)(510.17291475,143.00612078)
\curveto(510.21290882,142.9161115)(510.24790878,142.82111159)(510.27791475,142.72112078)
\curveto(510.30790872,142.62111179)(510.3329087,142.52111189)(510.35291475,142.42112078)
\curveto(510.37290866,142.39111202)(510.37790865,142.36111205)(510.36791475,142.33112078)
\curveto(510.36790866,142.30111211)(510.37290866,142.27111214)(510.38291475,142.24112078)
\curveto(510.41290862,142.13111228)(510.4329086,142.00611241)(510.44291475,141.86612078)
\curveto(510.45290858,141.73611268)(510.46290857,141.60111281)(510.47291475,141.46112078)
\lineto(510.47291475,141.29612078)
\curveto(510.48290855,141.23611318)(510.48290855,141.18111323)(510.47291475,141.13112078)
\curveto(510.46290857,141.08111333)(510.45790857,141.03111338)(510.45791475,140.98112078)
\lineto(510.45791475,140.84612078)
\curveto(510.44790858,140.80611361)(510.44290859,140.76611365)(510.44291475,140.72612078)
\curveto(510.45290858,140.68611373)(510.44790858,140.64111377)(510.42791475,140.59112078)
\curveto(510.40790862,140.48111393)(510.38790864,140.37611404)(510.36791475,140.27612078)
\curveto(510.35790867,140.17611424)(510.33790869,140.07611434)(510.30791475,139.97612078)
\curveto(510.17790885,139.6161148)(510.01290902,139.30111511)(509.81291475,139.03112078)
\curveto(509.61290942,138.76111565)(509.33790969,138.55611586)(508.98791475,138.41612078)
\curveto(508.90791012,138.38611603)(508.82291021,138.36111605)(508.73291475,138.34112078)
\lineto(508.46291475,138.28112078)
\curveto(508.41291062,138.27111614)(508.36791066,138.26611615)(508.32791475,138.26612078)
\curveto(508.28791074,138.27611614)(508.24791078,138.27611614)(508.20791475,138.26612078)
\curveto(508.10791092,138.24611617)(508.01291102,138.24611617)(507.92291475,138.26612078)
\moveto(507.08291475,139.66112078)
\curveto(507.12291191,139.59111482)(507.16291187,139.52611489)(507.20291475,139.46612078)
\curveto(507.24291179,139.416115)(507.29291174,139.36611505)(507.35291475,139.31612078)
\lineto(507.50291475,139.19612078)
\curveto(507.56291147,139.16611525)(507.6279114,139.14111527)(507.69791475,139.12112078)
\curveto(507.73791129,139.10111531)(507.77291126,139.09111532)(507.80291475,139.09112078)
\curveto(507.84291119,139.10111531)(507.88291115,139.09611532)(507.92291475,139.07612078)
\curveto(507.95291108,139.07611534)(507.99291104,139.07111534)(508.04291475,139.06112078)
\curveto(508.09291094,139.06111535)(508.1329109,139.06611535)(508.16291475,139.07612078)
\lineto(508.38791475,139.12112078)
\curveto(508.63791039,139.20111521)(508.82291021,139.32611509)(508.94291475,139.49612078)
\curveto(509.02291001,139.59611482)(509.09290994,139.72611469)(509.15291475,139.88612078)
\curveto(509.2329098,140.06611435)(509.29290974,140.29111412)(509.33291475,140.56112078)
\curveto(509.37290966,140.84111357)(509.38790964,141.12111329)(509.37791475,141.40112078)
\curveto(509.36790966,141.69111272)(509.33790969,141.96611245)(509.28791475,142.22612078)
\curveto(509.23790979,142.48611193)(509.16290987,142.69611172)(509.06291475,142.85612078)
\curveto(508.94291009,143.05611136)(508.79291024,143.20611121)(508.61291475,143.30612078)
\curveto(508.5329105,143.35611106)(508.44291059,143.38611103)(508.34291475,143.39612078)
\curveto(508.24291079,143.416111)(508.13791089,143.42611099)(508.02791475,143.42612078)
\curveto(508.00791102,143.416111)(507.98291105,143.411111)(507.95291475,143.41112078)
\curveto(507.9329111,143.42111099)(507.91291112,143.42111099)(507.89291475,143.41112078)
\curveto(507.84291119,143.40111101)(507.79791123,143.39111102)(507.75791475,143.38112078)
\curveto(507.71791131,143.38111103)(507.67791135,143.37111104)(507.63791475,143.35112078)
\curveto(507.45791157,143.27111114)(507.30791172,143.15111126)(507.18791475,142.99112078)
\curveto(507.07791195,142.83111158)(506.98791204,142.65111176)(506.91791475,142.45112078)
\curveto(506.85791217,142.26111215)(506.81291222,142.03611238)(506.78291475,141.77612078)
\curveto(506.76291227,141.5161129)(506.75791227,141.25111316)(506.76791475,140.98112078)
\curveto(506.77791225,140.72111369)(506.80791222,140.47111394)(506.85791475,140.23112078)
\curveto(506.91791211,140.00111441)(506.99291204,139.8111146)(507.08291475,139.66112078)
\moveto(517.88291475,136.67612078)
\curveto(517.89290114,136.62611779)(517.89790113,136.53611788)(517.89791475,136.40612078)
\curveto(517.89790113,136.27611814)(517.88790114,136.18611823)(517.86791475,136.13612078)
\curveto(517.84790118,136.08611833)(517.84290119,136.03111838)(517.85291475,135.97112078)
\curveto(517.86290117,135.92111849)(517.86290117,135.87111854)(517.85291475,135.82112078)
\curveto(517.81290122,135.68111873)(517.78290125,135.54611887)(517.76291475,135.41612078)
\curveto(517.75290128,135.28611913)(517.72290131,135.16611925)(517.67291475,135.05612078)
\curveto(517.5329015,134.70611971)(517.36790166,134.41112)(517.17791475,134.17112078)
\curveto(516.98790204,133.94112047)(516.71790231,133.75612066)(516.36791475,133.61612078)
\curveto(516.28790274,133.58612083)(516.20290283,133.56612085)(516.11291475,133.55612078)
\curveto(516.02290301,133.53612088)(515.93790309,133.5161209)(515.85791475,133.49612078)
\curveto(515.80790322,133.48612093)(515.75790327,133.48112093)(515.70791475,133.48112078)
\curveto(515.65790337,133.48112093)(515.60790342,133.47612094)(515.55791475,133.46612078)
\curveto(515.5279035,133.45612096)(515.47790355,133.45612096)(515.40791475,133.46612078)
\curveto(515.33790369,133.46612095)(515.28790374,133.47112094)(515.25791475,133.48112078)
\curveto(515.19790383,133.50112091)(515.13790389,133.5111209)(515.07791475,133.51112078)
\curveto(515.027904,133.50112091)(514.97790405,133.50612091)(514.92791475,133.52612078)
\curveto(514.83790419,133.54612087)(514.74790428,133.57112084)(514.65791475,133.60112078)
\curveto(514.57790445,133.62112079)(514.49790453,133.65112076)(514.41791475,133.69112078)
\curveto(514.09790493,133.83112058)(513.84790518,134.02612039)(513.66791475,134.27612078)
\curveto(513.48790554,134.53611988)(513.33790569,134.84111957)(513.21791475,135.19112078)
\curveto(513.19790583,135.27111914)(513.18290585,135.35611906)(513.17291475,135.44612078)
\curveto(513.16290587,135.53611888)(513.14790588,135.62111879)(513.12791475,135.70112078)
\curveto(513.11790591,135.73111868)(513.11290592,135.76111865)(513.11291475,135.79112078)
\lineto(513.11291475,135.89612078)
\curveto(513.09290594,135.97611844)(513.08290595,136.05611836)(513.08291475,136.13612078)
\lineto(513.08291475,136.27112078)
\curveto(513.06290597,136.37111804)(513.06290597,136.47111794)(513.08291475,136.57112078)
\lineto(513.08291475,136.75112078)
\curveto(513.09290594,136.80111761)(513.09790593,136.84611757)(513.09791475,136.88612078)
\curveto(513.09790593,136.93611748)(513.10290593,136.98111743)(513.11291475,137.02112078)
\curveto(513.12290591,137.06111735)(513.1279059,137.09611732)(513.12791475,137.12612078)
\curveto(513.1279059,137.16611725)(513.1329059,137.20611721)(513.14291475,137.24612078)
\lineto(513.20291475,137.57612078)
\curveto(513.22290581,137.69611672)(513.25290578,137.80611661)(513.29291475,137.90612078)
\curveto(513.4329056,138.23611618)(513.59290544,138.5111159)(513.77291475,138.73112078)
\curveto(513.96290507,138.96111545)(514.22290481,139.14611527)(514.55291475,139.28612078)
\curveto(514.6329044,139.32611509)(514.71790431,139.35111506)(514.80791475,139.36112078)
\lineto(515.10791475,139.42112078)
\lineto(515.24291475,139.42112078)
\curveto(515.29290374,139.43111498)(515.34290369,139.43611498)(515.39291475,139.43612078)
\curveto(515.96290307,139.45611496)(516.42290261,139.35111506)(516.77291475,139.12112078)
\curveto(517.1329019,138.90111551)(517.39790163,138.60111581)(517.56791475,138.22112078)
\curveto(517.61790141,138.12111629)(517.65790137,138.02111639)(517.68791475,137.92112078)
\curveto(517.71790131,137.82111659)(517.74790128,137.7161167)(517.77791475,137.60612078)
\curveto(517.78790124,137.56611685)(517.79290124,137.53111688)(517.79291475,137.50112078)
\curveto(517.79290124,137.48111693)(517.79790123,137.45111696)(517.80791475,137.41112078)
\curveto(517.8279012,137.34111707)(517.83790119,137.26611715)(517.83791475,137.18612078)
\curveto(517.83790119,137.10611731)(517.84790118,137.02611739)(517.86791475,136.94612078)
\curveto(517.86790116,136.89611752)(517.86790116,136.85111756)(517.86791475,136.81112078)
\curveto(517.86790116,136.77111764)(517.87290116,136.72611769)(517.88291475,136.67612078)
\moveto(516.77291475,136.24112078)
\curveto(516.78290225,136.29111812)(516.78790224,136.36611805)(516.78791475,136.46612078)
\curveto(516.79790223,136.56611785)(516.79290224,136.64111777)(516.77291475,136.69112078)
\curveto(516.75290228,136.75111766)(516.74790228,136.80611761)(516.75791475,136.85612078)
\curveto(516.77790225,136.9161175)(516.77790225,136.97611744)(516.75791475,137.03612078)
\curveto(516.74790228,137.06611735)(516.74290229,137.10111731)(516.74291475,137.14112078)
\curveto(516.74290229,137.18111723)(516.73790229,137.22111719)(516.72791475,137.26112078)
\curveto(516.70790232,137.34111707)(516.68790234,137.416117)(516.66791475,137.48612078)
\curveto(516.65790237,137.56611685)(516.64290239,137.64611677)(516.62291475,137.72612078)
\curveto(516.59290244,137.78611663)(516.56790246,137.84611657)(516.54791475,137.90612078)
\curveto(516.5279025,137.96611645)(516.49790253,138.02611639)(516.45791475,138.08612078)
\curveto(516.35790267,138.25611616)(516.2279028,138.39111602)(516.06791475,138.49112078)
\curveto(515.98790304,138.54111587)(515.89290314,138.57611584)(515.78291475,138.59612078)
\curveto(515.67290336,138.6161158)(515.54790348,138.62611579)(515.40791475,138.62612078)
\curveto(515.38790364,138.6161158)(515.36290367,138.6111158)(515.33291475,138.61112078)
\curveto(515.30290373,138.62111579)(515.27290376,138.62111579)(515.24291475,138.61112078)
\lineto(515.09291475,138.55112078)
\curveto(515.04290399,138.54111587)(514.99790403,138.52611589)(514.95791475,138.50612078)
\curveto(514.76790426,138.39611602)(514.62290441,138.25111616)(514.52291475,138.07112078)
\curveto(514.4329046,137.89111652)(514.35290468,137.68611673)(514.28291475,137.45612078)
\curveto(514.24290479,137.32611709)(514.22290481,137.19111722)(514.22291475,137.05112078)
\curveto(514.22290481,136.92111749)(514.21290482,136.77611764)(514.19291475,136.61612078)
\curveto(514.18290485,136.56611785)(514.17290486,136.50611791)(514.16291475,136.43612078)
\curveto(514.16290487,136.36611805)(514.17290486,136.30611811)(514.19291475,136.25612078)
\lineto(514.19291475,136.09112078)
\lineto(514.19291475,135.91112078)
\curveto(514.20290483,135.86111855)(514.21290482,135.80611861)(514.22291475,135.74612078)
\curveto(514.2329048,135.69611872)(514.23790479,135.64111877)(514.23791475,135.58112078)
\curveto(514.24790478,135.52111889)(514.26290477,135.46611895)(514.28291475,135.41612078)
\curveto(514.3329047,135.22611919)(514.39290464,135.05111936)(514.46291475,134.89112078)
\curveto(514.5329045,134.73111968)(514.63790439,134.60111981)(514.77791475,134.50112078)
\curveto(514.90790412,134.40112001)(515.04790398,134.33112008)(515.19791475,134.29112078)
\curveto(515.2279038,134.28112013)(515.25290378,134.27612014)(515.27291475,134.27612078)
\curveto(515.30290373,134.28612013)(515.3329037,134.28612013)(515.36291475,134.27612078)
\curveto(515.38290365,134.27612014)(515.41290362,134.27112014)(515.45291475,134.26112078)
\curveto(515.49290354,134.26112015)(515.5279035,134.26612015)(515.55791475,134.27612078)
\curveto(515.59790343,134.28612013)(515.63790339,134.29112012)(515.67791475,134.29112078)
\curveto(515.71790331,134.29112012)(515.75790327,134.30112011)(515.79791475,134.32112078)
\curveto(516.03790299,134.40112001)(516.2329028,134.53611988)(516.38291475,134.72612078)
\curveto(516.50290253,134.90611951)(516.59290244,135.1111193)(516.65291475,135.34112078)
\curveto(516.67290236,135.411119)(516.68790234,135.48111893)(516.69791475,135.55112078)
\curveto(516.70790232,135.63111878)(516.72290231,135.7111187)(516.74291475,135.79112078)
\curveto(516.74290229,135.85111856)(516.74790228,135.89611852)(516.75791475,135.92612078)
\curveto(516.75790227,135.94611847)(516.75790227,135.97111844)(516.75791475,136.00112078)
\curveto(516.75790227,136.04111837)(516.76290227,136.07111834)(516.77291475,136.09112078)
\lineto(516.77291475,136.24112078)
}
}
{
\newrgbcolor{curcolor}{0 0 0}
\pscustom[linestyle=none,fillstyle=solid,fillcolor=curcolor]
{
\newpath
\moveto(698.49734956,154.15762225)
\lineto(702.09734956,154.15762225)
\lineto(702.74234956,154.15762225)
\curveto(702.82234303,154.15761182)(702.89734296,154.15261183)(702.96734956,154.14262225)
\curveto(703.03734282,154.14261184)(703.09734276,154.13261185)(703.14734956,154.11262225)
\curveto(703.21734264,154.0826119)(703.27234258,154.02261196)(703.31234956,153.93262225)
\curveto(703.33234252,153.90261208)(703.34234251,153.86261212)(703.34234956,153.81262225)
\lineto(703.34234956,153.67762225)
\curveto(703.3523425,153.56761241)(703.34734251,153.46261252)(703.32734956,153.36262225)
\curveto(703.31734254,153.26261272)(703.28234257,153.19261279)(703.22234956,153.15262225)
\curveto(703.13234272,153.0826129)(702.99734286,153.04761293)(702.81734956,153.04762225)
\curveto(702.63734322,153.05761292)(702.47234338,153.06261292)(702.32234956,153.06262225)
\lineto(700.32734956,153.06262225)
\lineto(699.83234956,153.06262225)
\lineto(699.69734956,153.06262225)
\curveto(699.6573462,153.06261292)(699.61734624,153.05761292)(699.57734956,153.04762225)
\lineto(699.36734956,153.04762225)
\curveto(699.2573466,153.01761296)(699.17734668,152.977613)(699.12734956,152.92762225)
\curveto(699.07734678,152.88761309)(699.04234681,152.83261315)(699.02234956,152.76262225)
\curveto(699.00234685,152.70261328)(698.98734687,152.63261335)(698.97734956,152.55262225)
\curveto(698.96734689,152.47261351)(698.94734691,152.3826136)(698.91734956,152.28262225)
\curveto(698.86734699,152.0826139)(698.82734703,151.8776141)(698.79734956,151.66762225)
\curveto(698.76734709,151.45761452)(698.72734713,151.25261473)(698.67734956,151.05262225)
\curveto(698.6573472,150.982615)(698.64734721,150.91261507)(698.64734956,150.84262225)
\curveto(698.64734721,150.7826152)(698.63734722,150.71761526)(698.61734956,150.64762225)
\curveto(698.60734725,150.61761536)(698.59734726,150.5776154)(698.58734956,150.52762225)
\curveto(698.58734727,150.48761549)(698.59234726,150.44761553)(698.60234956,150.40762225)
\curveto(698.62234723,150.35761562)(698.64734721,150.31261567)(698.67734956,150.27262225)
\curveto(698.71734714,150.24261574)(698.77734708,150.23761574)(698.85734956,150.25762225)
\curveto(698.91734694,150.2776157)(698.97734688,150.30261568)(699.03734956,150.33262225)
\curveto(699.09734676,150.37261561)(699.1573467,150.40761557)(699.21734956,150.43762225)
\curveto(699.27734658,150.45761552)(699.32734653,150.47261551)(699.36734956,150.48262225)
\curveto(699.5573463,150.56261542)(699.76234609,150.61761536)(699.98234956,150.64762225)
\curveto(700.21234564,150.6776153)(700.44234541,150.68761529)(700.67234956,150.67762225)
\curveto(700.91234494,150.6776153)(701.14234471,150.65261533)(701.36234956,150.60262225)
\curveto(701.58234427,150.56261542)(701.78234407,150.50261548)(701.96234956,150.42262225)
\curveto(702.01234384,150.40261558)(702.0573438,150.3826156)(702.09734956,150.36262225)
\curveto(702.14734371,150.34261564)(702.19734366,150.31761566)(702.24734956,150.28762225)
\curveto(702.59734326,150.0776159)(702.87734298,149.84761613)(703.08734956,149.59762225)
\curveto(703.30734255,149.34761663)(703.50234235,149.02261696)(703.67234956,148.62262225)
\curveto(703.72234213,148.51261747)(703.7573421,148.40261758)(703.77734956,148.29262225)
\curveto(703.79734206,148.1826178)(703.82234203,148.06761791)(703.85234956,147.94762225)
\curveto(703.86234199,147.91761806)(703.86734199,147.87261811)(703.86734956,147.81262225)
\curveto(703.88734197,147.75261823)(703.89734196,147.6826183)(703.89734956,147.60262225)
\curveto(703.89734196,147.53261845)(703.90734195,147.46761851)(703.92734956,147.40762225)
\lineto(703.92734956,147.24262225)
\curveto(703.93734192,147.19261879)(703.94234191,147.12261886)(703.94234956,147.03262225)
\curveto(703.94234191,146.94261904)(703.93234192,146.87261911)(703.91234956,146.82262225)
\curveto(703.89234196,146.76261922)(703.88734197,146.70261928)(703.89734956,146.64262225)
\curveto(703.90734195,146.59261939)(703.90234195,146.54261944)(703.88234956,146.49262225)
\curveto(703.84234201,146.33261965)(703.80734205,146.1826198)(703.77734956,146.04262225)
\curveto(703.74734211,145.90262008)(703.70234215,145.76762021)(703.64234956,145.63762225)
\curveto(703.48234237,145.26762071)(703.26234259,144.93262105)(702.98234956,144.63262225)
\curveto(702.70234315,144.33262165)(702.38234347,144.10262188)(702.02234956,143.94262225)
\curveto(701.852344,143.86262212)(701.6523442,143.78762219)(701.42234956,143.71762225)
\curveto(701.31234454,143.6776223)(701.19734466,143.65262233)(701.07734956,143.64262225)
\curveto(700.9573449,143.63262235)(700.83734502,143.61262237)(700.71734956,143.58262225)
\curveto(700.66734519,143.56262242)(700.61234524,143.56262242)(700.55234956,143.58262225)
\curveto(700.49234536,143.59262239)(700.43234542,143.58762239)(700.37234956,143.56762225)
\curveto(700.27234558,143.54762243)(700.17234568,143.54762243)(700.07234956,143.56762225)
\lineto(699.93734956,143.56762225)
\curveto(699.88734597,143.58762239)(699.82734603,143.59762238)(699.75734956,143.59762225)
\curveto(699.69734616,143.58762239)(699.64234621,143.59262239)(699.59234956,143.61262225)
\curveto(699.5523463,143.62262236)(699.51734634,143.62762235)(699.48734956,143.62762225)
\curveto(699.4573464,143.62762235)(699.42234643,143.63262235)(699.38234956,143.64262225)
\lineto(699.11234956,143.70262225)
\curveto(699.02234683,143.72262226)(698.93734692,143.75262223)(698.85734956,143.79262225)
\curveto(698.51734734,143.93262205)(698.22734763,144.08762189)(697.98734956,144.25762225)
\curveto(697.74734811,144.43762154)(697.52734833,144.66762131)(697.32734956,144.94762225)
\curveto(697.17734868,145.1776208)(697.06234879,145.41762056)(696.98234956,145.66762225)
\curveto(696.96234889,145.71762026)(696.9523489,145.76262022)(696.95234956,145.80262225)
\curveto(696.9523489,145.85262013)(696.94234891,145.90262008)(696.92234956,145.95262225)
\curveto(696.90234895,146.01261997)(696.88734897,146.09261989)(696.87734956,146.19262225)
\curveto(696.87734898,146.29261969)(696.89734896,146.36761961)(696.93734956,146.41762225)
\curveto(696.98734887,146.49761948)(697.06734879,146.54261944)(697.17734956,146.55262225)
\curveto(697.28734857,146.56261942)(697.40234845,146.56761941)(697.52234956,146.56762225)
\lineto(697.68734956,146.56762225)
\curveto(697.74734811,146.56761941)(697.80234805,146.55761942)(697.85234956,146.53762225)
\curveto(697.94234791,146.51761946)(698.01234784,146.4776195)(698.06234956,146.41762225)
\curveto(698.13234772,146.32761965)(698.17734768,146.21761976)(698.19734956,146.08762225)
\curveto(698.22734763,145.96762001)(698.27234758,145.86262012)(698.33234956,145.77262225)
\curveto(698.52234733,145.43262055)(698.78234707,145.16262082)(699.11234956,144.96262225)
\curveto(699.21234664,144.90262108)(699.31734654,144.85262113)(699.42734956,144.81262225)
\curveto(699.54734631,144.7826212)(699.66734619,144.74762123)(699.78734956,144.70762225)
\curveto(699.9573459,144.65762132)(700.16234569,144.63762134)(700.40234956,144.64762225)
\curveto(700.6523452,144.66762131)(700.852345,144.70262128)(701.00234956,144.75262225)
\curveto(701.37234448,144.87262111)(701.66234419,145.03262095)(701.87234956,145.23262225)
\curveto(702.09234376,145.44262054)(702.27234358,145.72262026)(702.41234956,146.07262225)
\curveto(702.46234339,146.17261981)(702.49234336,146.2776197)(702.50234956,146.38762225)
\curveto(702.52234333,146.49761948)(702.54734331,146.61261937)(702.57734956,146.73262225)
\lineto(702.57734956,146.83762225)
\curveto(702.58734327,146.8776191)(702.59234326,146.91761906)(702.59234956,146.95762225)
\curveto(702.60234325,146.98761899)(702.60234325,147.02261896)(702.59234956,147.06262225)
\lineto(702.59234956,147.18262225)
\curveto(702.59234326,147.44261854)(702.56234329,147.68761829)(702.50234956,147.91762225)
\curveto(702.39234346,148.26761771)(702.23734362,148.56261742)(702.03734956,148.80262225)
\curveto(701.83734402,149.05261693)(701.57734428,149.24761673)(701.25734956,149.38762225)
\lineto(701.07734956,149.44762225)
\curveto(701.02734483,149.46761651)(700.96734489,149.48761649)(700.89734956,149.50762225)
\curveto(700.84734501,149.52761645)(700.78734507,149.53761644)(700.71734956,149.53762225)
\curveto(700.6573452,149.54761643)(700.59234526,149.56261642)(700.52234956,149.58262225)
\lineto(700.37234956,149.58262225)
\curveto(700.33234552,149.60261638)(700.27734558,149.61261637)(700.20734956,149.61262225)
\curveto(700.14734571,149.61261637)(700.09234576,149.60261638)(700.04234956,149.58262225)
\lineto(699.93734956,149.58262225)
\curveto(699.90734595,149.5826164)(699.87234598,149.5776164)(699.83234956,149.56762225)
\lineto(699.59234956,149.50762225)
\curveto(699.51234634,149.49761648)(699.43234642,149.4776165)(699.35234956,149.44762225)
\curveto(699.11234674,149.34761663)(698.88234697,149.21261677)(698.66234956,149.04262225)
\curveto(698.57234728,148.97261701)(698.48734737,148.89761708)(698.40734956,148.81762225)
\curveto(698.32734753,148.74761723)(698.22734763,148.69261729)(698.10734956,148.65262225)
\curveto(698.01734784,148.62261736)(697.87734798,148.61261737)(697.68734956,148.62262225)
\curveto(697.50734835,148.63261735)(697.38734847,148.65761732)(697.32734956,148.69762225)
\curveto(697.27734858,148.73761724)(697.23734862,148.79761718)(697.20734956,148.87762225)
\curveto(697.18734867,148.95761702)(697.18734867,149.04261694)(697.20734956,149.13262225)
\curveto(697.23734862,149.25261673)(697.2573486,149.37261661)(697.26734956,149.49262225)
\curveto(697.28734857,149.62261636)(697.31234854,149.74761623)(697.34234956,149.86762225)
\curveto(697.36234849,149.90761607)(697.36734849,149.94261604)(697.35734956,149.97262225)
\curveto(697.3573485,150.01261597)(697.36734849,150.05761592)(697.38734956,150.10762225)
\curveto(697.40734845,150.19761578)(697.42234843,150.28761569)(697.43234956,150.37762225)
\curveto(697.44234841,150.4776155)(697.46234839,150.57261541)(697.49234956,150.66262225)
\curveto(697.50234835,150.72261526)(697.50734835,150.7826152)(697.50734956,150.84262225)
\curveto(697.51734834,150.90261508)(697.53234832,150.96261502)(697.55234956,151.02262225)
\curveto(697.60234825,151.22261476)(697.63734822,151.42761455)(697.65734956,151.63762225)
\curveto(697.68734817,151.85761412)(697.72734813,152.06761391)(697.77734956,152.26762225)
\curveto(697.80734805,152.36761361)(697.82734803,152.46761351)(697.83734956,152.56762225)
\curveto(697.84734801,152.66761331)(697.86234799,152.76761321)(697.88234956,152.86762225)
\curveto(697.89234796,152.89761308)(697.89734796,152.93761304)(697.89734956,152.98762225)
\curveto(697.92734793,153.09761288)(697.94734791,153.20261278)(697.95734956,153.30262225)
\curveto(697.97734788,153.41261257)(698.00234785,153.52261246)(698.03234956,153.63262225)
\curveto(698.0523478,153.71261227)(698.06734779,153.7826122)(698.07734956,153.84262225)
\curveto(698.08734777,153.91261207)(698.11234774,153.97261201)(698.15234956,154.02262225)
\curveto(698.17234768,154.05261193)(698.20234765,154.07261191)(698.24234956,154.08262225)
\curveto(698.28234757,154.10261188)(698.32734753,154.12261186)(698.37734956,154.14262225)
\curveto(698.43734742,154.14261184)(698.47734738,154.14761183)(698.49734956,154.15762225)
}
}
{
\newrgbcolor{curcolor}{0 0 0}
\pscustom[linestyle=none,fillstyle=solid,fillcolor=curcolor]
{
\newpath
\moveto(712.21695894,147.24262225)
\curveto(712.28695129,147.19261879)(712.32695125,147.12261886)(712.33695894,147.03262225)
\curveto(712.35695122,146.94261904)(712.36695121,146.83761914)(712.36695894,146.71762225)
\curveto(712.36695121,146.66761931)(712.36195122,146.61761936)(712.35195894,146.56762225)
\curveto(712.35195123,146.51761946)(712.34195124,146.47261951)(712.32195894,146.43262225)
\curveto(712.29195129,146.34261964)(712.23195135,146.2826197)(712.14195894,146.25262225)
\curveto(712.06195152,146.23261975)(711.96695161,146.22261976)(711.85695894,146.22262225)
\lineto(711.54195894,146.22262225)
\curveto(711.43195215,146.23261975)(711.32695225,146.22261976)(711.22695894,146.19262225)
\curveto(711.08695249,146.16261982)(710.99695258,146.0826199)(710.95695894,145.95262225)
\curveto(710.93695264,145.8826201)(710.92695265,145.79762018)(710.92695894,145.69762225)
\lineto(710.92695894,145.42762225)
\lineto(710.92695894,144.48262225)
\lineto(710.92695894,144.15262225)
\curveto(710.92695265,144.04262194)(710.90695267,143.95762202)(710.86695894,143.89762225)
\curveto(710.82695275,143.83762214)(710.7769528,143.79762218)(710.71695894,143.77762225)
\curveto(710.66695291,143.76762221)(710.60195298,143.75262223)(710.52195894,143.73262225)
\lineto(710.32695894,143.73262225)
\curveto(710.20695337,143.73262225)(710.10195348,143.73762224)(710.01195894,143.74762225)
\curveto(709.92195366,143.76762221)(709.85195373,143.81762216)(709.80195894,143.89762225)
\curveto(709.77195381,143.94762203)(709.75695382,144.01762196)(709.75695894,144.10762225)
\lineto(709.75695894,144.40762225)
\lineto(709.75695894,145.44262225)
\curveto(709.75695382,145.60262038)(709.74695383,145.74762023)(709.72695894,145.87762225)
\curveto(709.71695386,146.01761996)(709.66195392,146.11261987)(709.56195894,146.16262225)
\curveto(709.51195407,146.1826198)(709.44195414,146.19761978)(709.35195894,146.20762225)
\curveto(709.27195431,146.21761976)(709.1819544,146.22261976)(709.08195894,146.22262225)
\lineto(708.79695894,146.22262225)
\lineto(708.55695894,146.22262225)
\lineto(706.29195894,146.22262225)
\curveto(706.20195738,146.22261976)(706.09695748,146.21761976)(705.97695894,146.20762225)
\lineto(705.64695894,146.20762225)
\curveto(705.53695804,146.20761977)(705.43695814,146.21761976)(705.34695894,146.23762225)
\curveto(705.25695832,146.25761972)(705.19695838,146.29261969)(705.16695894,146.34262225)
\curveto(705.11695846,146.41261957)(705.09195849,146.50761947)(705.09195894,146.62762225)
\lineto(705.09195894,146.97262225)
\lineto(705.09195894,147.24262225)
\curveto(705.13195845,147.41261857)(705.18695839,147.55261843)(705.25695894,147.66262225)
\curveto(705.32695825,147.77261821)(705.40695817,147.88761809)(705.49695894,148.00762225)
\lineto(705.85695894,148.54762225)
\curveto(706.29695728,149.1776168)(706.73195685,149.79761618)(707.16195894,150.40762225)
\lineto(708.48195894,152.26762225)
\curveto(708.64195494,152.49761348)(708.79695478,152.71761326)(708.94695894,152.92762225)
\curveto(709.09695448,153.14761283)(709.25195433,153.37261261)(709.41195894,153.60262225)
\curveto(709.46195412,153.67261231)(709.51195407,153.73761224)(709.56195894,153.79762225)
\curveto(709.61195397,153.86761211)(709.66195392,153.94261204)(709.71195894,154.02262225)
\lineto(709.77195894,154.11262225)
\curveto(709.80195378,154.15261183)(709.83195375,154.1826118)(709.86195894,154.20262225)
\curveto(709.90195368,154.23261175)(709.94195364,154.25261173)(709.98195894,154.26262225)
\curveto(710.02195356,154.2826117)(710.06695351,154.30261168)(710.11695894,154.32262225)
\curveto(710.13695344,154.32261166)(710.15695342,154.31761166)(710.17695894,154.30762225)
\curveto(710.20695337,154.30761167)(710.23195335,154.31761166)(710.25195894,154.33762225)
\curveto(710.3819532,154.33761164)(710.50195308,154.33261165)(710.61195894,154.32262225)
\curveto(710.72195286,154.31261167)(710.80195278,154.26761171)(710.85195894,154.18762225)
\curveto(710.89195269,154.13761184)(710.91195267,154.06761191)(710.91195894,153.97762225)
\curveto(710.92195266,153.88761209)(710.92695265,153.79261219)(710.92695894,153.69262225)
\lineto(710.92695894,148.23262225)
\curveto(710.92695265,148.16261782)(710.92195266,148.08761789)(710.91195894,148.00762225)
\curveto(710.91195267,147.93761804)(710.91695266,147.86761811)(710.92695894,147.79762225)
\lineto(710.92695894,147.69262225)
\curveto(710.94695263,147.64261834)(710.96195262,147.58761839)(710.97195894,147.52762225)
\curveto(710.9819526,147.4776185)(711.00695257,147.43761854)(711.04695894,147.40762225)
\curveto(711.11695246,147.35761862)(711.20195238,147.32761865)(711.30195894,147.31762225)
\lineto(711.63195894,147.31762225)
\curveto(711.74195184,147.31761866)(711.84695173,147.31261867)(711.94695894,147.30262225)
\curveto(712.05695152,147.30261868)(712.14695143,147.2826187)(712.21695894,147.24262225)
\moveto(709.65195894,147.43762225)
\curveto(709.73195385,147.54761843)(709.76695381,147.71761826)(709.75695894,147.94762225)
\lineto(709.75695894,148.56262225)
\lineto(709.75695894,151.03762225)
\lineto(709.75695894,151.35262225)
\curveto(709.76695381,151.47261451)(709.76195382,151.57261441)(709.74195894,151.65262225)
\lineto(709.74195894,151.80262225)
\curveto(709.74195384,151.89261409)(709.72695385,151.977614)(709.69695894,152.05762225)
\curveto(709.68695389,152.0776139)(709.6769539,152.08761389)(709.66695894,152.08762225)
\lineto(709.62195894,152.13262225)
\curveto(709.60195398,152.14261384)(709.57195401,152.14761383)(709.53195894,152.14762225)
\curveto(709.51195407,152.12761385)(709.49195409,152.11261387)(709.47195894,152.10262225)
\curveto(709.46195412,152.10261388)(709.44695413,152.09761388)(709.42695894,152.08762225)
\curveto(709.36695421,152.03761394)(709.30695427,151.96761401)(709.24695894,151.87762225)
\curveto(709.18695439,151.78761419)(709.13195445,151.70761427)(709.08195894,151.63762225)
\curveto(708.9819546,151.49761448)(708.88695469,151.35261463)(708.79695894,151.20262225)
\curveto(708.70695487,151.06261492)(708.61195497,150.92261506)(708.51195894,150.78262225)
\lineto(707.97195894,150.00262225)
\curveto(707.80195578,149.74261624)(707.62695595,149.4826165)(707.44695894,149.22262225)
\curveto(707.36695621,149.11261687)(707.29195629,149.00761697)(707.22195894,148.90762225)
\lineto(707.01195894,148.60762225)
\curveto(706.96195662,148.52761745)(706.91195667,148.45261753)(706.86195894,148.38262225)
\curveto(706.82195676,148.31261767)(706.7769568,148.23761774)(706.72695894,148.15762225)
\curveto(706.6769569,148.09761788)(706.62695695,148.03261795)(706.57695894,147.96262225)
\curveto(706.53695704,147.90261808)(706.49695708,147.83261815)(706.45695894,147.75262225)
\curveto(706.41695716,147.69261829)(706.39195719,147.62261836)(706.38195894,147.54262225)
\curveto(706.37195721,147.47261851)(706.40695717,147.41761856)(706.48695894,147.37762225)
\curveto(706.55695702,147.32761865)(706.66695691,147.30261868)(706.81695894,147.30262225)
\curveto(706.9769566,147.31261867)(707.11195647,147.31761866)(707.22195894,147.31762225)
\lineto(708.90195894,147.31762225)
\lineto(709.33695894,147.31762225)
\curveto(709.48695409,147.31761866)(709.59195399,147.35761862)(709.65195894,147.43762225)
}
}
{
\newrgbcolor{curcolor}{0 0 0}
\pscustom[linestyle=none,fillstyle=solid,fillcolor=curcolor]
{
\newpath
\moveto(714.64156831,145.38262225)
\lineto(714.94156831,145.38262225)
\curveto(715.05156625,145.39262059)(715.15656615,145.39262059)(715.25656831,145.38262225)
\curveto(715.36656594,145.3826206)(715.46656584,145.37262061)(715.55656831,145.35262225)
\curveto(715.64656566,145.34262064)(715.71656559,145.31762066)(715.76656831,145.27762225)
\curveto(715.78656552,145.25762072)(715.8015655,145.22762075)(715.81156831,145.18762225)
\curveto(715.83156547,145.14762083)(715.85156545,145.10262088)(715.87156831,145.05262225)
\lineto(715.87156831,144.97762225)
\curveto(715.88156542,144.92762105)(715.88156542,144.87262111)(715.87156831,144.81262225)
\lineto(715.87156831,144.66262225)
\lineto(715.87156831,144.18262225)
\curveto(715.87156543,144.01262197)(715.83156547,143.89262209)(715.75156831,143.82262225)
\curveto(715.68156562,143.77262221)(715.59156571,143.74762223)(715.48156831,143.74762225)
\lineto(715.15156831,143.74762225)
\lineto(714.70156831,143.74762225)
\curveto(714.55156675,143.74762223)(714.43656687,143.7776222)(714.35656831,143.83762225)
\curveto(714.31656699,143.86762211)(714.28656702,143.91762206)(714.26656831,143.98762225)
\curveto(714.24656706,144.06762191)(714.23156707,144.15262183)(714.22156831,144.24262225)
\lineto(714.22156831,144.52762225)
\curveto(714.23156707,144.62762135)(714.23656707,144.71262127)(714.23656831,144.78262225)
\lineto(714.23656831,144.97762225)
\curveto(714.23656707,145.03762094)(714.24656706,145.09262089)(714.26656831,145.14262225)
\curveto(714.306567,145.25262073)(714.37656693,145.32262066)(714.47656831,145.35262225)
\curveto(714.5065668,145.35262063)(714.56156674,145.36262062)(714.64156831,145.38262225)
}
}
{
\newrgbcolor{curcolor}{0 0 0}
\pscustom[linestyle=none,fillstyle=solid,fillcolor=curcolor]
{
\newpath
\moveto(724.86172456,146.82262225)
\curveto(724.87171684,146.7826192)(724.87171684,146.73261925)(724.86172456,146.67262225)
\curveto(724.86171685,146.61261937)(724.85671686,146.56261942)(724.84672456,146.52262225)
\curveto(724.84671687,146.4826195)(724.84171687,146.44261954)(724.83172456,146.40262225)
\lineto(724.83172456,146.29762225)
\curveto(724.8117169,146.21761976)(724.79671692,146.13761984)(724.78672456,146.05762225)
\curveto(724.77671694,145.97762)(724.75671696,145.90262008)(724.72672456,145.83262225)
\curveto(724.70671701,145.75262023)(724.68671703,145.6776203)(724.66672456,145.60762225)
\curveto(724.64671707,145.53762044)(724.6167171,145.46262052)(724.57672456,145.38262225)
\curveto(724.39671732,144.96262102)(724.14171757,144.62262136)(723.81172456,144.36262225)
\curveto(723.48171823,144.10262188)(723.09171862,143.89762208)(722.64172456,143.74762225)
\curveto(722.52171919,143.70762227)(722.39671932,143.6826223)(722.26672456,143.67262225)
\curveto(722.14671957,143.65262233)(722.02171969,143.62762235)(721.89172456,143.59762225)
\curveto(721.83171988,143.58762239)(721.76671995,143.5826224)(721.69672456,143.58262225)
\curveto(721.63672008,143.5826224)(721.57172014,143.5776224)(721.50172456,143.56762225)
\lineto(721.38172456,143.56762225)
\lineto(721.18672456,143.56762225)
\curveto(721.12672059,143.55762242)(721.07172064,143.56262242)(721.02172456,143.58262225)
\curveto(720.95172076,143.60262238)(720.88672083,143.60762237)(720.82672456,143.59762225)
\curveto(720.76672095,143.58762239)(720.70672101,143.59262239)(720.64672456,143.61262225)
\curveto(720.59672112,143.62262236)(720.55172116,143.62762235)(720.51172456,143.62762225)
\curveto(720.47172124,143.62762235)(720.42672129,143.63762234)(720.37672456,143.65762225)
\curveto(720.29672142,143.6776223)(720.22172149,143.69762228)(720.15172456,143.71762225)
\curveto(720.08172163,143.72762225)(720.0117217,143.74262224)(719.94172456,143.76262225)
\curveto(719.46172225,143.93262205)(719.06172265,144.14262184)(718.74172456,144.39262225)
\curveto(718.43172328,144.65262133)(718.18172353,145.00762097)(717.99172456,145.45762225)
\curveto(717.96172375,145.51762046)(717.93672378,145.5776204)(717.91672456,145.63762225)
\curveto(717.90672381,145.70762027)(717.89172382,145.7826202)(717.87172456,145.86262225)
\curveto(717.85172386,145.92262006)(717.83672388,145.98761999)(717.82672456,146.05762225)
\curveto(717.8167239,146.12761985)(717.80172391,146.19761978)(717.78172456,146.26762225)
\curveto(717.77172394,146.31761966)(717.76672395,146.35761962)(717.76672456,146.38762225)
\lineto(717.76672456,146.50762225)
\curveto(717.75672396,146.54761943)(717.74672397,146.59761938)(717.73672456,146.65762225)
\curveto(717.73672398,146.71761926)(717.74172397,146.76761921)(717.75172456,146.80762225)
\lineto(717.75172456,146.94262225)
\curveto(717.76172395,146.99261899)(717.76672395,147.04261894)(717.76672456,147.09262225)
\curveto(717.78672393,147.19261879)(717.80172391,147.28761869)(717.81172456,147.37762225)
\curveto(717.82172389,147.4776185)(717.84172387,147.57261841)(717.87172456,147.66262225)
\curveto(717.92172379,147.81261817)(717.97672374,147.95261803)(718.03672456,148.08262225)
\curveto(718.09672362,148.21261777)(718.16672355,148.33261765)(718.24672456,148.44262225)
\curveto(718.27672344,148.49261749)(718.30672341,148.53261745)(718.33672456,148.56262225)
\curveto(718.37672334,148.59261739)(718.4117233,148.62761735)(718.44172456,148.66762225)
\curveto(718.50172321,148.74761723)(718.57172314,148.81761716)(718.65172456,148.87762225)
\curveto(718.711723,148.92761705)(718.77172294,148.97261701)(718.83172456,149.01262225)
\lineto(719.04172456,149.16262225)
\curveto(719.09172262,149.20261678)(719.14172257,149.23761674)(719.19172456,149.26762225)
\curveto(719.24172247,149.30761667)(719.27672244,149.36261662)(719.29672456,149.43262225)
\curveto(719.29672242,149.46261652)(719.28672243,149.48761649)(719.26672456,149.50762225)
\curveto(719.25672246,149.53761644)(719.24672247,149.56261642)(719.23672456,149.58262225)
\curveto(719.19672252,149.63261635)(719.14672257,149.6776163)(719.08672456,149.71762225)
\curveto(719.03672268,149.76761621)(718.98672273,149.81261617)(718.93672456,149.85262225)
\curveto(718.89672282,149.8826161)(718.84672287,149.93761604)(718.78672456,150.01762225)
\curveto(718.76672295,150.04761593)(718.73672298,150.07261591)(718.69672456,150.09262225)
\curveto(718.66672305,150.12261586)(718.64172307,150.15761582)(718.62172456,150.19762225)
\curveto(718.45172326,150.40761557)(718.32172339,150.65261533)(718.23172456,150.93262225)
\curveto(718.2117235,151.01261497)(718.19672352,151.09261489)(718.18672456,151.17262225)
\curveto(718.17672354,151.25261473)(718.16172355,151.33261465)(718.14172456,151.41262225)
\curveto(718.12172359,151.46261452)(718.1117236,151.52761445)(718.11172456,151.60762225)
\curveto(718.1117236,151.69761428)(718.12172359,151.76761421)(718.14172456,151.81762225)
\curveto(718.14172357,151.91761406)(718.14672357,151.98761399)(718.15672456,152.02762225)
\curveto(718.17672354,152.10761387)(718.19172352,152.1776138)(718.20172456,152.23762225)
\curveto(718.2117235,152.30761367)(718.22672349,152.3776136)(718.24672456,152.44762225)
\curveto(718.39672332,152.8776131)(718.6117231,153.22261276)(718.89172456,153.48262225)
\curveto(719.18172253,153.74261224)(719.53172218,153.95761202)(719.94172456,154.12762225)
\curveto(720.05172166,154.1776118)(720.16672155,154.20761177)(720.28672456,154.21762225)
\curveto(720.4167213,154.23761174)(720.54672117,154.26761171)(720.67672456,154.30762225)
\curveto(720.75672096,154.30761167)(720.82672089,154.30761167)(720.88672456,154.30762225)
\curveto(720.95672076,154.31761166)(721.03172068,154.32761165)(721.11172456,154.33762225)
\curveto(721.90171981,154.35761162)(722.55671916,154.22761175)(723.07672456,153.94762225)
\curveto(723.60671811,153.66761231)(723.98671773,153.25761272)(724.21672456,152.71762225)
\curveto(724.32671739,152.48761349)(724.39671732,152.20261378)(724.42672456,151.86262225)
\curveto(724.46671725,151.53261445)(724.43671728,151.22761475)(724.33672456,150.94762225)
\curveto(724.29671742,150.81761516)(724.24671747,150.69761528)(724.18672456,150.58762225)
\curveto(724.13671758,150.4776155)(724.07671764,150.37261561)(724.00672456,150.27262225)
\curveto(723.98671773,150.23261575)(723.95671776,150.19761578)(723.91672456,150.16762225)
\lineto(723.82672456,150.07762225)
\curveto(723.77671794,149.98761599)(723.716718,149.92261606)(723.64672456,149.88262225)
\curveto(723.59671812,149.83261615)(723.54171817,149.7826162)(723.48172456,149.73262225)
\curveto(723.43171828,149.69261629)(723.38671833,149.64761633)(723.34672456,149.59762225)
\curveto(723.32671839,149.5776164)(723.30671841,149.55261643)(723.28672456,149.52262225)
\curveto(723.27671844,149.50261648)(723.27671844,149.4776165)(723.28672456,149.44762225)
\curveto(723.29671842,149.39761658)(723.32671839,149.34761663)(723.37672456,149.29762225)
\curveto(723.42671829,149.25761672)(723.48171823,149.21761676)(723.54172456,149.17762225)
\lineto(723.72172456,149.05762225)
\curveto(723.78171793,149.02761695)(723.83171788,148.99761698)(723.87172456,148.96762225)
\curveto(724.20171751,148.72761725)(724.45171726,148.41761756)(724.62172456,148.03762225)
\curveto(724.66171705,147.95761802)(724.69171702,147.87261811)(724.71172456,147.78262225)
\curveto(724.74171697,147.69261829)(724.76671695,147.60261838)(724.78672456,147.51262225)
\curveto(724.79671692,147.46261852)(724.80671691,147.40761857)(724.81672456,147.34762225)
\lineto(724.84672456,147.19762225)
\curveto(724.85671686,147.13761884)(724.85671686,147.07261891)(724.84672456,147.00262225)
\curveto(724.83671688,146.94261904)(724.84171687,146.8826191)(724.86172456,146.82262225)
\moveto(719.47672456,151.86262225)
\curveto(719.44672227,151.75261423)(719.44172227,151.61261437)(719.46172456,151.44262225)
\curveto(719.48172223,151.2826147)(719.50672221,151.15761482)(719.53672456,151.06762225)
\curveto(719.64672207,150.74761523)(719.79672192,150.50261548)(719.98672456,150.33262225)
\curveto(720.17672154,150.17261581)(720.44172127,150.04261594)(720.78172456,149.94262225)
\curveto(720.9117208,149.91261607)(721.07672064,149.88761609)(721.27672456,149.86762225)
\curveto(721.47672024,149.85761612)(721.64672007,149.87261611)(721.78672456,149.91262225)
\curveto(722.07671964,149.99261599)(722.3167194,150.10261588)(722.50672456,150.24262225)
\curveto(722.70671901,150.39261559)(722.86171885,150.59261539)(722.97172456,150.84262225)
\curveto(722.99171872,150.89261509)(723.00171871,150.93761504)(723.00172456,150.97762225)
\curveto(723.0117187,151.01761496)(723.02671869,151.06261492)(723.04672456,151.11262225)
\curveto(723.07671864,151.22261476)(723.09671862,151.36261462)(723.10672456,151.53262225)
\curveto(723.1167186,151.70261428)(723.10671861,151.84761413)(723.07672456,151.96762225)
\curveto(723.05671866,152.05761392)(723.03171868,152.14261384)(723.00172456,152.22262225)
\curveto(722.98171873,152.30261368)(722.94671877,152.3826136)(722.89672456,152.46262225)
\curveto(722.72671899,152.73261325)(722.50171921,152.92761305)(722.22172456,153.04762225)
\curveto(721.95171976,153.16761281)(721.59172012,153.22761275)(721.14172456,153.22762225)
\curveto(721.12172059,153.20761277)(721.09172062,153.20261278)(721.05172456,153.21262225)
\curveto(721.0117207,153.22261276)(720.97672074,153.22261276)(720.94672456,153.21262225)
\curveto(720.89672082,153.19261279)(720.84172087,153.1776128)(720.78172456,153.16762225)
\curveto(720.73172098,153.16761281)(720.68172103,153.15761282)(720.63172456,153.13762225)
\curveto(720.39172132,153.04761293)(720.18172153,152.93261305)(720.00172456,152.79262225)
\curveto(719.82172189,152.66261332)(719.68172203,152.4826135)(719.58172456,152.25262225)
\curveto(719.56172215,152.19261379)(719.54172217,152.12761385)(719.52172456,152.05762225)
\curveto(719.5117222,151.99761398)(719.49672222,151.93261405)(719.47672456,151.86262225)
\moveto(723.49672456,146.32762225)
\curveto(723.54671817,146.51761946)(723.55171816,146.72261926)(723.51172456,146.94262225)
\curveto(723.48171823,147.16261882)(723.43671828,147.34261864)(723.37672456,147.48262225)
\curveto(723.20671851,147.85261813)(722.94671877,148.15761782)(722.59672456,148.39762225)
\curveto(722.25671946,148.63761734)(721.82171989,148.75761722)(721.29172456,148.75762225)
\curveto(721.26172045,148.73761724)(721.22172049,148.73261725)(721.17172456,148.74262225)
\curveto(721.12172059,148.76261722)(721.08172063,148.76761721)(721.05172456,148.75762225)
\lineto(720.78172456,148.69762225)
\curveto(720.70172101,148.68761729)(720.62172109,148.67261731)(720.54172456,148.65262225)
\curveto(720.24172147,148.54261744)(719.97672174,148.39761758)(719.74672456,148.21762225)
\curveto(719.52672219,148.03761794)(719.35672236,147.80761817)(719.23672456,147.52762225)
\curveto(719.20672251,147.44761853)(719.18172253,147.36761861)(719.16172456,147.28762225)
\curveto(719.14172257,147.20761877)(719.12172259,147.12261886)(719.10172456,147.03262225)
\curveto(719.07172264,146.91261907)(719.06172265,146.76261922)(719.07172456,146.58262225)
\curveto(719.09172262,146.40261958)(719.1167226,146.26261972)(719.14672456,146.16262225)
\curveto(719.16672255,146.11261987)(719.17672254,146.06761991)(719.17672456,146.02762225)
\curveto(719.18672253,145.99761998)(719.20172251,145.95762002)(719.22172456,145.90762225)
\curveto(719.32172239,145.68762029)(719.45172226,145.48762049)(719.61172456,145.30762225)
\curveto(719.78172193,145.12762085)(719.97672174,144.99262099)(720.19672456,144.90262225)
\curveto(720.26672145,144.86262112)(720.36172135,144.82762115)(720.48172456,144.79762225)
\curveto(720.70172101,144.70762127)(720.95672076,144.66262132)(721.24672456,144.66262225)
\lineto(721.53172456,144.66262225)
\curveto(721.63172008,144.6826213)(721.72671999,144.69762128)(721.81672456,144.70762225)
\curveto(721.90671981,144.71762126)(721.99671972,144.73762124)(722.08672456,144.76762225)
\curveto(722.34671937,144.84762113)(722.58671913,144.977621)(722.80672456,145.15762225)
\curveto(723.03671868,145.34762063)(723.20671851,145.56262042)(723.31672456,145.80262225)
\curveto(723.35671836,145.8826201)(723.38671833,145.96262002)(723.40672456,146.04262225)
\curveto(723.43671828,146.13261985)(723.46671825,146.22761975)(723.49672456,146.32762225)
}
}
{
\newrgbcolor{curcolor}{0 0 0}
\pscustom[linestyle=none,fillstyle=solid,fillcolor=curcolor]
{
\newpath
\moveto(736.00133394,152.26762225)
\curveto(735.80132364,151.977614)(735.59132385,151.69261429)(735.37133394,151.41262225)
\curveto(735.16132428,151.13261485)(734.95632448,150.84761513)(734.75633394,150.55762225)
\curveto(734.15632528,149.70761627)(733.55132589,148.86761711)(732.94133394,148.03762225)
\curveto(732.33132711,147.21761876)(731.72632771,146.3826196)(731.12633394,145.53262225)
\lineto(730.61633394,144.81262225)
\lineto(730.10633394,144.12262225)
\curveto(730.02632941,144.01262197)(729.94632949,143.89762208)(729.86633394,143.77762225)
\curveto(729.78632965,143.65762232)(729.69132975,143.56262242)(729.58133394,143.49262225)
\curveto(729.5413299,143.47262251)(729.47632996,143.45762252)(729.38633394,143.44762225)
\curveto(729.30633013,143.42762255)(729.21633022,143.41762256)(729.11633394,143.41762225)
\curveto(729.01633042,143.41762256)(728.92133052,143.42262256)(728.83133394,143.43262225)
\curveto(728.75133069,143.44262254)(728.69133075,143.46262252)(728.65133394,143.49262225)
\curveto(728.62133082,143.51262247)(728.59633084,143.54762243)(728.57633394,143.59762225)
\curveto(728.56633087,143.63762234)(728.57133087,143.6826223)(728.59133394,143.73262225)
\curveto(728.63133081,143.81262217)(728.67633076,143.88762209)(728.72633394,143.95762225)
\curveto(728.78633065,144.03762194)(728.8413306,144.11762186)(728.89133394,144.19762225)
\curveto(729.13133031,144.53762144)(729.37633006,144.87262111)(729.62633394,145.20262225)
\curveto(729.87632956,145.53262045)(730.11632932,145.86762011)(730.34633394,146.20762225)
\curveto(730.50632893,146.42761955)(730.66632877,146.64261934)(730.82633394,146.85262225)
\curveto(730.98632845,147.06261892)(731.14632829,147.2776187)(731.30633394,147.49762225)
\curveto(731.66632777,148.01761796)(732.03132741,148.52761745)(732.40133394,149.02762225)
\curveto(732.77132667,149.52761645)(733.1413263,150.03761594)(733.51133394,150.55762225)
\curveto(733.65132579,150.75761522)(733.79132565,150.95261503)(733.93133394,151.14262225)
\curveto(734.08132536,151.33261465)(734.22632521,151.52761445)(734.36633394,151.72762225)
\curveto(734.57632486,152.02761395)(734.79132465,152.32761365)(735.01133394,152.62762225)
\lineto(735.67133394,153.52762225)
\lineto(735.85133394,153.79762225)
\lineto(736.06133394,154.06762225)
\lineto(736.18133394,154.24762225)
\curveto(736.23132321,154.30761167)(736.28132316,154.36261162)(736.33133394,154.41262225)
\curveto(736.40132304,154.46261152)(736.47632296,154.49761148)(736.55633394,154.51762225)
\curveto(736.57632286,154.52761145)(736.60132284,154.52761145)(736.63133394,154.51762225)
\curveto(736.67132277,154.51761146)(736.70132274,154.52761145)(736.72133394,154.54762225)
\curveto(736.8413226,154.54761143)(736.97632246,154.54261144)(737.12633394,154.53262225)
\curveto(737.27632216,154.53261145)(737.36632207,154.48761149)(737.39633394,154.39762225)
\curveto(737.41632202,154.36761161)(737.42132202,154.33261165)(737.41133394,154.29262225)
\curveto(737.40132204,154.25261173)(737.38632205,154.22261176)(737.36633394,154.20262225)
\curveto(737.32632211,154.12261186)(737.28632215,154.05261193)(737.24633394,153.99262225)
\curveto(737.20632223,153.93261205)(737.16132228,153.87261211)(737.11133394,153.81262225)
\lineto(736.54133394,153.03262225)
\curveto(736.36132308,152.7826132)(736.18132326,152.52761345)(736.00133394,152.26762225)
\moveto(729.14633394,148.36762225)
\curveto(729.09633034,148.38761759)(729.04633039,148.39261759)(728.99633394,148.38262225)
\curveto(728.94633049,148.37261761)(728.89633054,148.3776176)(728.84633394,148.39762225)
\curveto(728.7363307,148.41761756)(728.63133081,148.43761754)(728.53133394,148.45762225)
\curveto(728.441331,148.48761749)(728.34633109,148.52761745)(728.24633394,148.57762225)
\curveto(727.91633152,148.71761726)(727.66133178,148.91261707)(727.48133394,149.16262225)
\curveto(727.30133214,149.42261656)(727.15633228,149.73261625)(727.04633394,150.09262225)
\curveto(727.01633242,150.17261581)(726.99633244,150.25261573)(726.98633394,150.33262225)
\curveto(726.97633246,150.42261556)(726.96133248,150.50761547)(726.94133394,150.58762225)
\curveto(726.93133251,150.63761534)(726.92633251,150.70261528)(726.92633394,150.78262225)
\curveto(726.91633252,150.81261517)(726.91133253,150.84261514)(726.91133394,150.87262225)
\curveto(726.91133253,150.91261507)(726.90633253,150.94761503)(726.89633394,150.97762225)
\lineto(726.89633394,151.12762225)
\curveto(726.88633255,151.1776148)(726.88133256,151.23761474)(726.88133394,151.30762225)
\curveto(726.88133256,151.38761459)(726.88633255,151.45261453)(726.89633394,151.50262225)
\lineto(726.89633394,151.66762225)
\curveto(726.91633252,151.71761426)(726.92133252,151.76261422)(726.91133394,151.80262225)
\curveto(726.91133253,151.85261413)(726.91633252,151.89761408)(726.92633394,151.93762225)
\curveto(726.9363325,151.977614)(726.9413325,152.01261397)(726.94133394,152.04262225)
\curveto(726.9413325,152.0826139)(726.94633249,152.12261386)(726.95633394,152.16262225)
\curveto(726.98633245,152.27261371)(727.00633243,152.3826136)(727.01633394,152.49262225)
\curveto(727.0363324,152.61261337)(727.07133237,152.72761325)(727.12133394,152.83762225)
\curveto(727.26133218,153.1776128)(727.42133202,153.45261253)(727.60133394,153.66262225)
\curveto(727.79133165,153.8826121)(728.06133138,154.06261192)(728.41133394,154.20262225)
\curveto(728.49133095,154.23261175)(728.57633086,154.25261173)(728.66633394,154.26262225)
\curveto(728.75633068,154.2826117)(728.85133059,154.30261168)(728.95133394,154.32262225)
\curveto(728.98133046,154.33261165)(729.0363304,154.33261165)(729.11633394,154.32262225)
\curveto(729.19633024,154.32261166)(729.24633019,154.33261165)(729.26633394,154.35262225)
\curveto(729.82632961,154.36261162)(730.27632916,154.25261173)(730.61633394,154.02262225)
\curveto(730.96632847,153.79261219)(731.22632821,153.48761249)(731.39633394,153.10762225)
\curveto(731.436328,153.01761296)(731.47132797,152.92261306)(731.50133394,152.82262225)
\curveto(731.53132791,152.72261326)(731.55632788,152.62261336)(731.57633394,152.52262225)
\curveto(731.59632784,152.49261349)(731.60132784,152.46261352)(731.59133394,152.43262225)
\curveto(731.59132785,152.40261358)(731.59632784,152.37261361)(731.60633394,152.34262225)
\curveto(731.6363278,152.23261375)(731.65632778,152.10761387)(731.66633394,151.96762225)
\curveto(731.67632776,151.83761414)(731.68632775,151.70261428)(731.69633394,151.56262225)
\lineto(731.69633394,151.39762225)
\curveto(731.70632773,151.33761464)(731.70632773,151.2826147)(731.69633394,151.23262225)
\curveto(731.68632775,151.1826148)(731.68132776,151.13261485)(731.68133394,151.08262225)
\lineto(731.68133394,150.94762225)
\curveto(731.67132777,150.90761507)(731.66632777,150.86761511)(731.66633394,150.82762225)
\curveto(731.67632776,150.78761519)(731.67132777,150.74261524)(731.65133394,150.69262225)
\curveto(731.63132781,150.5826154)(731.61132783,150.4776155)(731.59133394,150.37762225)
\curveto(731.58132786,150.2776157)(731.56132788,150.1776158)(731.53133394,150.07762225)
\curveto(731.40132804,149.71761626)(731.2363282,149.40261658)(731.03633394,149.13262225)
\curveto(730.8363286,148.86261712)(730.56132888,148.65761732)(730.21133394,148.51762225)
\curveto(730.13132931,148.48761749)(730.04632939,148.46261752)(729.95633394,148.44262225)
\lineto(729.68633394,148.38262225)
\curveto(729.6363298,148.37261761)(729.59132985,148.36761761)(729.55133394,148.36762225)
\curveto(729.51132993,148.3776176)(729.47132997,148.3776176)(729.43133394,148.36762225)
\curveto(729.33133011,148.34761763)(729.2363302,148.34761763)(729.14633394,148.36762225)
\moveto(728.30633394,149.76262225)
\curveto(728.34633109,149.69261629)(728.38633105,149.62761635)(728.42633394,149.56762225)
\curveto(728.46633097,149.51761646)(728.51633092,149.46761651)(728.57633394,149.41762225)
\lineto(728.72633394,149.29762225)
\curveto(728.78633065,149.26761671)(728.85133059,149.24261674)(728.92133394,149.22262225)
\curveto(728.96133048,149.20261678)(728.99633044,149.19261679)(729.02633394,149.19262225)
\curveto(729.06633037,149.20261678)(729.10633033,149.19761678)(729.14633394,149.17762225)
\curveto(729.17633026,149.1776168)(729.21633022,149.17261681)(729.26633394,149.16262225)
\curveto(729.31633012,149.16261682)(729.35633008,149.16761681)(729.38633394,149.17762225)
\lineto(729.61133394,149.22262225)
\curveto(729.86132958,149.30261668)(730.04632939,149.42761655)(730.16633394,149.59762225)
\curveto(730.24632919,149.69761628)(730.31632912,149.82761615)(730.37633394,149.98762225)
\curveto(730.45632898,150.16761581)(730.51632892,150.39261559)(730.55633394,150.66262225)
\curveto(730.59632884,150.94261504)(730.61132883,151.22261476)(730.60133394,151.50262225)
\curveto(730.59132885,151.79261419)(730.56132888,152.06761391)(730.51133394,152.32762225)
\curveto(730.46132898,152.58761339)(730.38632905,152.79761318)(730.28633394,152.95762225)
\curveto(730.16632927,153.15761282)(730.01632942,153.30761267)(729.83633394,153.40762225)
\curveto(729.75632968,153.45761252)(729.66632977,153.48761249)(729.56633394,153.49762225)
\curveto(729.46632997,153.51761246)(729.36133008,153.52761245)(729.25133394,153.52762225)
\curveto(729.23133021,153.51761246)(729.20633023,153.51261247)(729.17633394,153.51262225)
\curveto(729.15633028,153.52261246)(729.1363303,153.52261246)(729.11633394,153.51262225)
\curveto(729.06633037,153.50261248)(729.02133042,153.49261249)(728.98133394,153.48262225)
\curveto(728.9413305,153.4826125)(728.90133054,153.47261251)(728.86133394,153.45262225)
\curveto(728.68133076,153.37261261)(728.53133091,153.25261273)(728.41133394,153.09262225)
\curveto(728.30133114,152.93261305)(728.21133123,152.75261323)(728.14133394,152.55262225)
\curveto(728.08133136,152.36261362)(728.0363314,152.13761384)(728.00633394,151.87762225)
\curveto(727.98633145,151.61761436)(727.98133146,151.35261463)(727.99133394,151.08262225)
\curveto(728.00133144,150.82261516)(728.03133141,150.57261541)(728.08133394,150.33262225)
\curveto(728.1413313,150.10261588)(728.21633122,149.91261607)(728.30633394,149.76262225)
\moveto(739.10633394,146.77762225)
\curveto(739.11632032,146.72761925)(739.12132032,146.63761934)(739.12133394,146.50762225)
\curveto(739.12132032,146.3776196)(739.11132033,146.28761969)(739.09133394,146.23762225)
\curveto(739.07132037,146.18761979)(739.06632037,146.13261985)(739.07633394,146.07262225)
\curveto(739.08632035,146.02261996)(739.08632035,145.97262001)(739.07633394,145.92262225)
\curveto(739.0363204,145.7826202)(739.00632043,145.64762033)(738.98633394,145.51762225)
\curveto(738.97632046,145.38762059)(738.94632049,145.26762071)(738.89633394,145.15762225)
\curveto(738.75632068,144.80762117)(738.59132085,144.51262147)(738.40133394,144.27262225)
\curveto(738.21132123,144.04262194)(737.9413215,143.85762212)(737.59133394,143.71762225)
\curveto(737.51132193,143.68762229)(737.42632201,143.66762231)(737.33633394,143.65762225)
\curveto(737.24632219,143.63762234)(737.16132228,143.61762236)(737.08133394,143.59762225)
\curveto(737.03132241,143.58762239)(736.98132246,143.5826224)(736.93133394,143.58262225)
\curveto(736.88132256,143.5826224)(736.83132261,143.5776224)(736.78133394,143.56762225)
\curveto(736.75132269,143.55762242)(736.70132274,143.55762242)(736.63133394,143.56762225)
\curveto(736.56132288,143.56762241)(736.51132293,143.57262241)(736.48133394,143.58262225)
\curveto(736.42132302,143.60262238)(736.36132308,143.61262237)(736.30133394,143.61262225)
\curveto(736.25132319,143.60262238)(736.20132324,143.60762237)(736.15133394,143.62762225)
\curveto(736.06132338,143.64762233)(735.97132347,143.67262231)(735.88133394,143.70262225)
\curveto(735.80132364,143.72262226)(735.72132372,143.75262223)(735.64133394,143.79262225)
\curveto(735.32132412,143.93262205)(735.07132437,144.12762185)(734.89133394,144.37762225)
\curveto(734.71132473,144.63762134)(734.56132488,144.94262104)(734.44133394,145.29262225)
\curveto(734.42132502,145.37262061)(734.40632503,145.45762052)(734.39633394,145.54762225)
\curveto(734.38632505,145.63762034)(734.37132507,145.72262026)(734.35133394,145.80262225)
\curveto(734.3413251,145.83262015)(734.3363251,145.86262012)(734.33633394,145.89262225)
\lineto(734.33633394,145.99762225)
\curveto(734.31632512,146.0776199)(734.30632513,146.15761982)(734.30633394,146.23762225)
\lineto(734.30633394,146.37262225)
\curveto(734.28632515,146.47261951)(734.28632515,146.57261941)(734.30633394,146.67262225)
\lineto(734.30633394,146.85262225)
\curveto(734.31632512,146.90261908)(734.32132512,146.94761903)(734.32133394,146.98762225)
\curveto(734.32132512,147.03761894)(734.32632511,147.0826189)(734.33633394,147.12262225)
\curveto(734.34632509,147.16261882)(734.35132509,147.19761878)(734.35133394,147.22762225)
\curveto(734.35132509,147.26761871)(734.35632508,147.30761867)(734.36633394,147.34762225)
\lineto(734.42633394,147.67762225)
\curveto(734.44632499,147.79761818)(734.47632496,147.90761807)(734.51633394,148.00762225)
\curveto(734.65632478,148.33761764)(734.81632462,148.61261737)(734.99633394,148.83262225)
\curveto(735.18632425,149.06261692)(735.44632399,149.24761673)(735.77633394,149.38762225)
\curveto(735.85632358,149.42761655)(735.9413235,149.45261653)(736.03133394,149.46262225)
\lineto(736.33133394,149.52262225)
\lineto(736.46633394,149.52262225)
\curveto(736.51632292,149.53261645)(736.56632287,149.53761644)(736.61633394,149.53762225)
\curveto(737.18632225,149.55761642)(737.64632179,149.45261653)(737.99633394,149.22262225)
\curveto(738.35632108,149.00261698)(738.62132082,148.70261728)(738.79133394,148.32262225)
\curveto(738.8413206,148.22261776)(738.88132056,148.12261786)(738.91133394,148.02262225)
\curveto(738.9413205,147.92261806)(738.97132047,147.81761816)(739.00133394,147.70762225)
\curveto(739.01132043,147.66761831)(739.01632042,147.63261835)(739.01633394,147.60262225)
\curveto(739.01632042,147.5826184)(739.02132042,147.55261843)(739.03133394,147.51262225)
\curveto(739.05132039,147.44261854)(739.06132038,147.36761861)(739.06133394,147.28762225)
\curveto(739.06132038,147.20761877)(739.07132037,147.12761885)(739.09133394,147.04762225)
\curveto(739.09132035,146.99761898)(739.09132035,146.95261903)(739.09133394,146.91262225)
\curveto(739.09132035,146.87261911)(739.09632034,146.82761915)(739.10633394,146.77762225)
\moveto(737.99633394,146.34262225)
\curveto(738.00632143,146.39261959)(738.01132143,146.46761951)(738.01133394,146.56762225)
\curveto(738.02132142,146.66761931)(738.01632142,146.74261924)(737.99633394,146.79262225)
\curveto(737.97632146,146.85261913)(737.97132147,146.90761907)(737.98133394,146.95762225)
\curveto(738.00132144,147.01761896)(738.00132144,147.0776189)(737.98133394,147.13762225)
\curveto(737.97132147,147.16761881)(737.96632147,147.20261878)(737.96633394,147.24262225)
\curveto(737.96632147,147.2826187)(737.96132148,147.32261866)(737.95133394,147.36262225)
\curveto(737.93132151,147.44261854)(737.91132153,147.51761846)(737.89133394,147.58762225)
\curveto(737.88132156,147.66761831)(737.86632157,147.74761823)(737.84633394,147.82762225)
\curveto(737.81632162,147.88761809)(737.79132165,147.94761803)(737.77133394,148.00762225)
\curveto(737.75132169,148.06761791)(737.72132172,148.12761785)(737.68133394,148.18762225)
\curveto(737.58132186,148.35761762)(737.45132199,148.49261749)(737.29133394,148.59262225)
\curveto(737.21132223,148.64261734)(737.11632232,148.6776173)(737.00633394,148.69762225)
\curveto(736.89632254,148.71761726)(736.77132267,148.72761725)(736.63133394,148.72762225)
\curveto(736.61132283,148.71761726)(736.58632285,148.71261727)(736.55633394,148.71262225)
\curveto(736.52632291,148.72261726)(736.49632294,148.72261726)(736.46633394,148.71262225)
\lineto(736.31633394,148.65262225)
\curveto(736.26632317,148.64261734)(736.22132322,148.62761735)(736.18133394,148.60762225)
\curveto(735.99132345,148.49761748)(735.84632359,148.35261763)(735.74633394,148.17262225)
\curveto(735.65632378,147.99261799)(735.57632386,147.78761819)(735.50633394,147.55762225)
\curveto(735.46632397,147.42761855)(735.44632399,147.29261869)(735.44633394,147.15262225)
\curveto(735.44632399,147.02261896)(735.436324,146.8776191)(735.41633394,146.71762225)
\curveto(735.40632403,146.66761931)(735.39632404,146.60761937)(735.38633394,146.53762225)
\curveto(735.38632405,146.46761951)(735.39632404,146.40761957)(735.41633394,146.35762225)
\lineto(735.41633394,146.19262225)
\lineto(735.41633394,146.01262225)
\curveto(735.42632401,145.96262002)(735.436324,145.90762007)(735.44633394,145.84762225)
\curveto(735.45632398,145.79762018)(735.46132398,145.74262024)(735.46133394,145.68262225)
\curveto(735.47132397,145.62262036)(735.48632395,145.56762041)(735.50633394,145.51762225)
\curveto(735.55632388,145.32762065)(735.61632382,145.15262083)(735.68633394,144.99262225)
\curveto(735.75632368,144.83262115)(735.86132358,144.70262128)(736.00133394,144.60262225)
\curveto(736.13132331,144.50262148)(736.27132317,144.43262155)(736.42133394,144.39262225)
\curveto(736.45132299,144.3826216)(736.47632296,144.3776216)(736.49633394,144.37762225)
\curveto(736.52632291,144.38762159)(736.55632288,144.38762159)(736.58633394,144.37762225)
\curveto(736.60632283,144.3776216)(736.6363228,144.37262161)(736.67633394,144.36262225)
\curveto(736.71632272,144.36262162)(736.75132269,144.36762161)(736.78133394,144.37762225)
\curveto(736.82132262,144.38762159)(736.86132258,144.39262159)(736.90133394,144.39262225)
\curveto(736.9413225,144.39262159)(736.98132246,144.40262158)(737.02133394,144.42262225)
\curveto(737.26132218,144.50262148)(737.45632198,144.63762134)(737.60633394,144.82762225)
\curveto(737.72632171,145.00762097)(737.81632162,145.21262077)(737.87633394,145.44262225)
\curveto(737.89632154,145.51262047)(737.91132153,145.5826204)(737.92133394,145.65262225)
\curveto(737.93132151,145.73262025)(737.94632149,145.81262017)(737.96633394,145.89262225)
\curveto(737.96632147,145.95262003)(737.97132147,145.99761998)(737.98133394,146.02762225)
\curveto(737.98132146,146.04761993)(737.98132146,146.07261991)(737.98133394,146.10262225)
\curveto(737.98132146,146.14261984)(737.98632145,146.17261981)(737.99633394,146.19262225)
\lineto(737.99633394,146.34262225)
}
}
{
\newrgbcolor{curcolor}{0 0 0}
\pscustom[linestyle=none,fillstyle=solid,fillcolor=curcolor]
{
\newpath
\moveto(244.38293428,284.20342059)
\curveto(244.39292656,284.16341754)(244.39292656,284.11341759)(244.38293428,284.05342059)
\curveto(244.38292657,283.99341771)(244.37792658,283.94341776)(244.36793428,283.90342059)
\curveto(244.36792659,283.86341784)(244.36292659,283.82341788)(244.35293428,283.78342059)
\lineto(244.35293428,283.67842059)
\curveto(244.33292662,283.5984181)(244.31792664,283.51841818)(244.30793428,283.43842059)
\curveto(244.29792666,283.35841834)(244.27792668,283.28341842)(244.24793428,283.21342059)
\curveto(244.22792673,283.13341857)(244.20792675,283.05841864)(244.18793428,282.98842059)
\curveto(244.16792679,282.91841878)(244.13792682,282.84341886)(244.09793428,282.76342059)
\curveto(243.91792704,282.34341936)(243.66292729,282.0034197)(243.33293428,281.74342059)
\curveto(243.00292795,281.48342022)(242.61292834,281.27842042)(242.16293428,281.12842059)
\curveto(242.04292891,281.08842061)(241.91792904,281.06342064)(241.78793428,281.05342059)
\curveto(241.66792929,281.03342067)(241.54292941,281.00842069)(241.41293428,280.97842059)
\curveto(241.3529296,280.96842073)(241.28792967,280.96342074)(241.21793428,280.96342059)
\curveto(241.1579298,280.96342074)(241.09292986,280.95842074)(241.02293428,280.94842059)
\lineto(240.90293428,280.94842059)
\lineto(240.70793428,280.94842059)
\curveto(240.64793031,280.93842076)(240.59293036,280.94342076)(240.54293428,280.96342059)
\curveto(240.47293048,280.98342072)(240.40793055,280.98842071)(240.34793428,280.97842059)
\curveto(240.28793067,280.96842073)(240.22793073,280.97342073)(240.16793428,280.99342059)
\curveto(240.11793084,281.0034207)(240.07293088,281.00842069)(240.03293428,281.00842059)
\curveto(239.99293096,281.00842069)(239.94793101,281.01842068)(239.89793428,281.03842059)
\curveto(239.81793114,281.05842064)(239.74293121,281.07842062)(239.67293428,281.09842059)
\curveto(239.60293135,281.10842059)(239.53293142,281.12342058)(239.46293428,281.14342059)
\curveto(238.98293197,281.31342039)(238.58293237,281.52342018)(238.26293428,281.77342059)
\curveto(237.952933,282.03341967)(237.70293325,282.38841931)(237.51293428,282.83842059)
\curveto(237.48293347,282.8984188)(237.4579335,282.95841874)(237.43793428,283.01842059)
\curveto(237.42793353,283.08841861)(237.41293354,283.16341854)(237.39293428,283.24342059)
\curveto(237.37293358,283.3034184)(237.3579336,283.36841833)(237.34793428,283.43842059)
\curveto(237.33793362,283.50841819)(237.32293363,283.57841812)(237.30293428,283.64842059)
\curveto(237.29293366,283.698418)(237.28793367,283.73841796)(237.28793428,283.76842059)
\lineto(237.28793428,283.88842059)
\curveto(237.27793368,283.92841777)(237.26793369,283.97841772)(237.25793428,284.03842059)
\curveto(237.2579337,284.0984176)(237.26293369,284.14841755)(237.27293428,284.18842059)
\lineto(237.27293428,284.32342059)
\curveto(237.28293367,284.37341733)(237.28793367,284.42341728)(237.28793428,284.47342059)
\curveto(237.30793365,284.57341713)(237.32293363,284.66841703)(237.33293428,284.75842059)
\curveto(237.34293361,284.85841684)(237.36293359,284.95341675)(237.39293428,285.04342059)
\curveto(237.44293351,285.19341651)(237.49793346,285.33341637)(237.55793428,285.46342059)
\curveto(237.61793334,285.59341611)(237.68793327,285.71341599)(237.76793428,285.82342059)
\curveto(237.79793316,285.87341583)(237.82793313,285.91341579)(237.85793428,285.94342059)
\curveto(237.89793306,285.97341573)(237.93293302,286.00841569)(237.96293428,286.04842059)
\curveto(238.02293293,286.12841557)(238.09293286,286.1984155)(238.17293428,286.25842059)
\curveto(238.23293272,286.30841539)(238.29293266,286.35341535)(238.35293428,286.39342059)
\lineto(238.56293428,286.54342059)
\curveto(238.61293234,286.58341512)(238.66293229,286.61841508)(238.71293428,286.64842059)
\curveto(238.76293219,286.68841501)(238.79793216,286.74341496)(238.81793428,286.81342059)
\curveto(238.81793214,286.84341486)(238.80793215,286.86841483)(238.78793428,286.88842059)
\curveto(238.77793218,286.91841478)(238.76793219,286.94341476)(238.75793428,286.96342059)
\curveto(238.71793224,287.01341469)(238.66793229,287.05841464)(238.60793428,287.09842059)
\curveto(238.5579324,287.14841455)(238.50793245,287.19341451)(238.45793428,287.23342059)
\curveto(238.41793254,287.26341444)(238.36793259,287.31841438)(238.30793428,287.39842059)
\curveto(238.28793267,287.42841427)(238.2579327,287.45341425)(238.21793428,287.47342059)
\curveto(238.18793277,287.5034142)(238.16293279,287.53841416)(238.14293428,287.57842059)
\curveto(237.97293298,287.78841391)(237.84293311,288.03341367)(237.75293428,288.31342059)
\curveto(237.73293322,288.39341331)(237.71793324,288.47341323)(237.70793428,288.55342059)
\curveto(237.69793326,288.63341307)(237.68293327,288.71341299)(237.66293428,288.79342059)
\curveto(237.64293331,288.84341286)(237.63293332,288.90841279)(237.63293428,288.98842059)
\curveto(237.63293332,289.07841262)(237.64293331,289.14841255)(237.66293428,289.19842059)
\curveto(237.66293329,289.2984124)(237.66793329,289.36841233)(237.67793428,289.40842059)
\curveto(237.69793326,289.48841221)(237.71293324,289.55841214)(237.72293428,289.61842059)
\curveto(237.73293322,289.68841201)(237.74793321,289.75841194)(237.76793428,289.82842059)
\curveto(237.91793304,290.25841144)(238.13293282,290.6034111)(238.41293428,290.86342059)
\curveto(238.70293225,291.12341058)(239.0529319,291.33841036)(239.46293428,291.50842059)
\curveto(239.57293138,291.55841014)(239.68793127,291.58841011)(239.80793428,291.59842059)
\curveto(239.93793102,291.61841008)(240.06793089,291.64841005)(240.19793428,291.68842059)
\curveto(240.27793068,291.68841001)(240.34793061,291.68841001)(240.40793428,291.68842059)
\curveto(240.47793048,291.69841)(240.5529304,291.70840999)(240.63293428,291.71842059)
\curveto(241.42292953,291.73840996)(242.07792888,291.60841009)(242.59793428,291.32842059)
\curveto(243.12792783,291.04841065)(243.50792745,290.63841106)(243.73793428,290.09842059)
\curveto(243.84792711,289.86841183)(243.91792704,289.58341212)(243.94793428,289.24342059)
\curveto(243.98792697,288.91341279)(243.957927,288.60841309)(243.85793428,288.32842059)
\curveto(243.81792714,288.1984135)(243.76792719,288.07841362)(243.70793428,287.96842059)
\curveto(243.6579273,287.85841384)(243.59792736,287.75341395)(243.52793428,287.65342059)
\curveto(243.50792745,287.61341409)(243.47792748,287.57841412)(243.43793428,287.54842059)
\lineto(243.34793428,287.45842059)
\curveto(243.29792766,287.36841433)(243.23792772,287.3034144)(243.16793428,287.26342059)
\curveto(243.11792784,287.21341449)(243.06292789,287.16341454)(243.00293428,287.11342059)
\curveto(242.952928,287.07341463)(242.90792805,287.02841467)(242.86793428,286.97842059)
\curveto(242.84792811,286.95841474)(242.82792813,286.93341477)(242.80793428,286.90342059)
\curveto(242.79792816,286.88341482)(242.79792816,286.85841484)(242.80793428,286.82842059)
\curveto(242.81792814,286.77841492)(242.84792811,286.72841497)(242.89793428,286.67842059)
\curveto(242.94792801,286.63841506)(243.00292795,286.5984151)(243.06293428,286.55842059)
\lineto(243.24293428,286.43842059)
\curveto(243.30292765,286.40841529)(243.3529276,286.37841532)(243.39293428,286.34842059)
\curveto(243.72292723,286.10841559)(243.97292698,285.7984159)(244.14293428,285.41842059)
\curveto(244.18292677,285.33841636)(244.21292674,285.25341645)(244.23293428,285.16342059)
\curveto(244.26292669,285.07341663)(244.28792667,284.98341672)(244.30793428,284.89342059)
\curveto(244.31792664,284.84341686)(244.32792663,284.78841691)(244.33793428,284.72842059)
\lineto(244.36793428,284.57842059)
\curveto(244.37792658,284.51841718)(244.37792658,284.45341725)(244.36793428,284.38342059)
\curveto(244.3579266,284.32341738)(244.36292659,284.26341744)(244.38293428,284.20342059)
\moveto(238.99793428,289.24342059)
\curveto(238.96793199,289.13341257)(238.96293199,288.99341271)(238.98293428,288.82342059)
\curveto(239.00293195,288.66341304)(239.02793193,288.53841316)(239.05793428,288.44842059)
\curveto(239.16793179,288.12841357)(239.31793164,287.88341382)(239.50793428,287.71342059)
\curveto(239.69793126,287.55341415)(239.96293099,287.42341428)(240.30293428,287.32342059)
\curveto(240.43293052,287.29341441)(240.59793036,287.26841443)(240.79793428,287.24842059)
\curveto(240.99792996,287.23841446)(241.16792979,287.25341445)(241.30793428,287.29342059)
\curveto(241.59792936,287.37341433)(241.83792912,287.48341422)(242.02793428,287.62342059)
\curveto(242.22792873,287.77341393)(242.38292857,287.97341373)(242.49293428,288.22342059)
\curveto(242.51292844,288.27341343)(242.52292843,288.31841338)(242.52293428,288.35842059)
\curveto(242.53292842,288.3984133)(242.54792841,288.44341326)(242.56793428,288.49342059)
\curveto(242.59792836,288.6034131)(242.61792834,288.74341296)(242.62793428,288.91342059)
\curveto(242.63792832,289.08341262)(242.62792833,289.22841247)(242.59793428,289.34842059)
\curveto(242.57792838,289.43841226)(242.5529284,289.52341218)(242.52293428,289.60342059)
\curveto(242.50292845,289.68341202)(242.46792849,289.76341194)(242.41793428,289.84342059)
\curveto(242.24792871,290.11341159)(242.02292893,290.30841139)(241.74293428,290.42842059)
\curveto(241.47292948,290.54841115)(241.11292984,290.60841109)(240.66293428,290.60842059)
\curveto(240.64293031,290.58841111)(240.61293034,290.58341112)(240.57293428,290.59342059)
\curveto(240.53293042,290.6034111)(240.49793046,290.6034111)(240.46793428,290.59342059)
\curveto(240.41793054,290.57341113)(240.36293059,290.55841114)(240.30293428,290.54842059)
\curveto(240.2529307,290.54841115)(240.20293075,290.53841116)(240.15293428,290.51842059)
\curveto(239.91293104,290.42841127)(239.70293125,290.31341139)(239.52293428,290.17342059)
\curveto(239.34293161,290.04341166)(239.20293175,289.86341184)(239.10293428,289.63342059)
\curveto(239.08293187,289.57341213)(239.06293189,289.50841219)(239.04293428,289.43842059)
\curveto(239.03293192,289.37841232)(239.01793194,289.31341239)(238.99793428,289.24342059)
\moveto(243.01793428,283.70842059)
\curveto(243.06792789,283.8984178)(243.07292788,284.1034176)(243.03293428,284.32342059)
\curveto(243.00292795,284.54341716)(242.957928,284.72341698)(242.89793428,284.86342059)
\curveto(242.72792823,285.23341647)(242.46792849,285.53841616)(242.11793428,285.77842059)
\curveto(241.77792918,286.01841568)(241.34292961,286.13841556)(240.81293428,286.13842059)
\curveto(240.78293017,286.11841558)(240.74293021,286.11341559)(240.69293428,286.12342059)
\curveto(240.64293031,286.14341556)(240.60293035,286.14841555)(240.57293428,286.13842059)
\lineto(240.30293428,286.07842059)
\curveto(240.22293073,286.06841563)(240.14293081,286.05341565)(240.06293428,286.03342059)
\curveto(239.76293119,285.92341578)(239.49793146,285.77841592)(239.26793428,285.59842059)
\curveto(239.04793191,285.41841628)(238.87793208,285.18841651)(238.75793428,284.90842059)
\curveto(238.72793223,284.82841687)(238.70293225,284.74841695)(238.68293428,284.66842059)
\curveto(238.66293229,284.58841711)(238.64293231,284.5034172)(238.62293428,284.41342059)
\curveto(238.59293236,284.29341741)(238.58293237,284.14341756)(238.59293428,283.96342059)
\curveto(238.61293234,283.78341792)(238.63793232,283.64341806)(238.66793428,283.54342059)
\curveto(238.68793227,283.49341821)(238.69793226,283.44841825)(238.69793428,283.40842059)
\curveto(238.70793225,283.37841832)(238.72293223,283.33841836)(238.74293428,283.28842059)
\curveto(238.84293211,283.06841863)(238.97293198,282.86841883)(239.13293428,282.68842059)
\curveto(239.30293165,282.50841919)(239.49793146,282.37341933)(239.71793428,282.28342059)
\curveto(239.78793117,282.24341946)(239.88293107,282.20841949)(240.00293428,282.17842059)
\curveto(240.22293073,282.08841961)(240.47793048,282.04341966)(240.76793428,282.04342059)
\lineto(241.05293428,282.04342059)
\curveto(241.1529298,282.06341964)(241.24792971,282.07841962)(241.33793428,282.08842059)
\curveto(241.42792953,282.0984196)(241.51792944,282.11841958)(241.60793428,282.14842059)
\curveto(241.86792909,282.22841947)(242.10792885,282.35841934)(242.32793428,282.53842059)
\curveto(242.5579284,282.72841897)(242.72792823,282.94341876)(242.83793428,283.18342059)
\curveto(242.87792808,283.26341844)(242.90792805,283.34341836)(242.92793428,283.42342059)
\curveto(242.957928,283.51341819)(242.98792797,283.60841809)(243.01793428,283.70842059)
}
}
{
\newrgbcolor{curcolor}{0 0 0}
\pscustom[linestyle=none,fillstyle=solid,fillcolor=curcolor]
{
\newpath
\moveto(255.52254366,289.64842059)
\curveto(255.32253336,289.35841234)(255.11253357,289.07341263)(254.89254366,288.79342059)
\curveto(254.682534,288.51341319)(254.4775342,288.22841347)(254.27754366,287.93842059)
\curveto(253.677535,287.08841461)(253.07253561,286.24841545)(252.46254366,285.41842059)
\curveto(251.85253683,284.5984171)(251.24753743,283.76341794)(250.64754366,282.91342059)
\lineto(250.13754366,282.19342059)
\lineto(249.62754366,281.50342059)
\curveto(249.54753913,281.39342031)(249.46753921,281.27842042)(249.38754366,281.15842059)
\curveto(249.30753937,281.03842066)(249.21253947,280.94342076)(249.10254366,280.87342059)
\curveto(249.06253962,280.85342085)(248.99753968,280.83842086)(248.90754366,280.82842059)
\curveto(248.82753985,280.80842089)(248.73753994,280.7984209)(248.63754366,280.79842059)
\curveto(248.53754014,280.7984209)(248.44254024,280.8034209)(248.35254366,280.81342059)
\curveto(248.27254041,280.82342088)(248.21254047,280.84342086)(248.17254366,280.87342059)
\curveto(248.14254054,280.89342081)(248.11754056,280.92842077)(248.09754366,280.97842059)
\curveto(248.08754059,281.01842068)(248.09254059,281.06342064)(248.11254366,281.11342059)
\curveto(248.15254053,281.19342051)(248.19754048,281.26842043)(248.24754366,281.33842059)
\curveto(248.30754037,281.41842028)(248.36254032,281.4984202)(248.41254366,281.57842059)
\curveto(248.65254003,281.91841978)(248.89753978,282.25341945)(249.14754366,282.58342059)
\curveto(249.39753928,282.91341879)(249.63753904,283.24841845)(249.86754366,283.58842059)
\curveto(250.02753865,283.80841789)(250.18753849,284.02341768)(250.34754366,284.23342059)
\curveto(250.50753817,284.44341726)(250.66753801,284.65841704)(250.82754366,284.87842059)
\curveto(251.18753749,285.3984163)(251.55253713,285.90841579)(251.92254366,286.40842059)
\curveto(252.29253639,286.90841479)(252.66253602,287.41841428)(253.03254366,287.93842059)
\curveto(253.17253551,288.13841356)(253.31253537,288.33341337)(253.45254366,288.52342059)
\curveto(253.60253508,288.71341299)(253.74753493,288.90841279)(253.88754366,289.10842059)
\curveto(254.09753458,289.40841229)(254.31253437,289.70841199)(254.53254366,290.00842059)
\lineto(255.19254366,290.90842059)
\lineto(255.37254366,291.17842059)
\lineto(255.58254366,291.44842059)
\lineto(255.70254366,291.62842059)
\curveto(255.75253293,291.68841001)(255.80253288,291.74340996)(255.85254366,291.79342059)
\curveto(255.92253276,291.84340986)(255.99753268,291.87840982)(256.07754366,291.89842059)
\curveto(256.09753258,291.90840979)(256.12253256,291.90840979)(256.15254366,291.89842059)
\curveto(256.19253249,291.8984098)(256.22253246,291.90840979)(256.24254366,291.92842059)
\curveto(256.36253232,291.92840977)(256.49753218,291.92340978)(256.64754366,291.91342059)
\curveto(256.79753188,291.91340979)(256.88753179,291.86840983)(256.91754366,291.77842059)
\curveto(256.93753174,291.74840995)(256.94253174,291.71340999)(256.93254366,291.67342059)
\curveto(256.92253176,291.63341007)(256.90753177,291.6034101)(256.88754366,291.58342059)
\curveto(256.84753183,291.5034102)(256.80753187,291.43341027)(256.76754366,291.37342059)
\curveto(256.72753195,291.31341039)(256.682532,291.25341045)(256.63254366,291.19342059)
\lineto(256.06254366,290.41342059)
\curveto(255.8825328,290.16341154)(255.70253298,289.90841179)(255.52254366,289.64842059)
\moveto(248.66754366,285.74842059)
\curveto(248.61754006,285.76841593)(248.56754011,285.77341593)(248.51754366,285.76342059)
\curveto(248.46754021,285.75341595)(248.41754026,285.75841594)(248.36754366,285.77842059)
\curveto(248.25754042,285.7984159)(248.15254053,285.81841588)(248.05254366,285.83842059)
\curveto(247.96254072,285.86841583)(247.86754081,285.90841579)(247.76754366,285.95842059)
\curveto(247.43754124,286.0984156)(247.1825415,286.29341541)(247.00254366,286.54342059)
\curveto(246.82254186,286.8034149)(246.677542,287.11341459)(246.56754366,287.47342059)
\curveto(246.53754214,287.55341415)(246.51754216,287.63341407)(246.50754366,287.71342059)
\curveto(246.49754218,287.8034139)(246.4825422,287.88841381)(246.46254366,287.96842059)
\curveto(246.45254223,288.01841368)(246.44754223,288.08341362)(246.44754366,288.16342059)
\curveto(246.43754224,288.19341351)(246.43254225,288.22341348)(246.43254366,288.25342059)
\curveto(246.43254225,288.29341341)(246.42754225,288.32841337)(246.41754366,288.35842059)
\lineto(246.41754366,288.50842059)
\curveto(246.40754227,288.55841314)(246.40254228,288.61841308)(246.40254366,288.68842059)
\curveto(246.40254228,288.76841293)(246.40754227,288.83341287)(246.41754366,288.88342059)
\lineto(246.41754366,289.04842059)
\curveto(246.43754224,289.0984126)(246.44254224,289.14341256)(246.43254366,289.18342059)
\curveto(246.43254225,289.23341247)(246.43754224,289.27841242)(246.44754366,289.31842059)
\curveto(246.45754222,289.35841234)(246.46254222,289.39341231)(246.46254366,289.42342059)
\curveto(246.46254222,289.46341224)(246.46754221,289.5034122)(246.47754366,289.54342059)
\curveto(246.50754217,289.65341205)(246.52754215,289.76341194)(246.53754366,289.87342059)
\curveto(246.55754212,289.99341171)(246.59254209,290.10841159)(246.64254366,290.21842059)
\curveto(246.7825419,290.55841114)(246.94254174,290.83341087)(247.12254366,291.04342059)
\curveto(247.31254137,291.26341044)(247.5825411,291.44341026)(247.93254366,291.58342059)
\curveto(248.01254067,291.61341009)(248.09754058,291.63341007)(248.18754366,291.64342059)
\curveto(248.2775404,291.66341004)(248.37254031,291.68341002)(248.47254366,291.70342059)
\curveto(248.50254018,291.71340999)(248.55754012,291.71340999)(248.63754366,291.70342059)
\curveto(248.71753996,291.70341)(248.76753991,291.71340999)(248.78754366,291.73342059)
\curveto(249.34753933,291.74340996)(249.79753888,291.63341007)(250.13754366,291.40342059)
\curveto(250.48753819,291.17341053)(250.74753793,290.86841083)(250.91754366,290.48842059)
\curveto(250.95753772,290.3984113)(250.99253769,290.3034114)(251.02254366,290.20342059)
\curveto(251.05253763,290.1034116)(251.0775376,290.0034117)(251.09754366,289.90342059)
\curveto(251.11753756,289.87341183)(251.12253756,289.84341186)(251.11254366,289.81342059)
\curveto(251.11253757,289.78341192)(251.11753756,289.75341195)(251.12754366,289.72342059)
\curveto(251.15753752,289.61341209)(251.1775375,289.48841221)(251.18754366,289.34842059)
\curveto(251.19753748,289.21841248)(251.20753747,289.08341262)(251.21754366,288.94342059)
\lineto(251.21754366,288.77842059)
\curveto(251.22753745,288.71841298)(251.22753745,288.66341304)(251.21754366,288.61342059)
\curveto(251.20753747,288.56341314)(251.20253748,288.51341319)(251.20254366,288.46342059)
\lineto(251.20254366,288.32842059)
\curveto(251.19253749,288.28841341)(251.18753749,288.24841345)(251.18754366,288.20842059)
\curveto(251.19753748,288.16841353)(251.19253749,288.12341358)(251.17254366,288.07342059)
\curveto(251.15253753,287.96341374)(251.13253755,287.85841384)(251.11254366,287.75842059)
\curveto(251.10253758,287.65841404)(251.0825376,287.55841414)(251.05254366,287.45842059)
\curveto(250.92253776,287.0984146)(250.75753792,286.78341492)(250.55754366,286.51342059)
\curveto(250.35753832,286.24341546)(250.0825386,286.03841566)(249.73254366,285.89842059)
\curveto(249.65253903,285.86841583)(249.56753911,285.84341586)(249.47754366,285.82342059)
\lineto(249.20754366,285.76342059)
\curveto(249.15753952,285.75341595)(249.11253957,285.74841595)(249.07254366,285.74842059)
\curveto(249.03253965,285.75841594)(248.99253969,285.75841594)(248.95254366,285.74842059)
\curveto(248.85253983,285.72841597)(248.75753992,285.72841597)(248.66754366,285.74842059)
\moveto(247.82754366,287.14342059)
\curveto(247.86754081,287.07341463)(247.90754077,287.00841469)(247.94754366,286.94842059)
\curveto(247.98754069,286.8984148)(248.03754064,286.84841485)(248.09754366,286.79842059)
\lineto(248.24754366,286.67842059)
\curveto(248.30754037,286.64841505)(248.37254031,286.62341508)(248.44254366,286.60342059)
\curveto(248.4825402,286.58341512)(248.51754016,286.57341513)(248.54754366,286.57342059)
\curveto(248.58754009,286.58341512)(248.62754005,286.57841512)(248.66754366,286.55842059)
\curveto(248.69753998,286.55841514)(248.73753994,286.55341515)(248.78754366,286.54342059)
\curveto(248.83753984,286.54341516)(248.8775398,286.54841515)(248.90754366,286.55842059)
\lineto(249.13254366,286.60342059)
\curveto(249.3825393,286.68341502)(249.56753911,286.80841489)(249.68754366,286.97842059)
\curveto(249.76753891,287.07841462)(249.83753884,287.20841449)(249.89754366,287.36842059)
\curveto(249.9775387,287.54841415)(250.03753864,287.77341393)(250.07754366,288.04342059)
\curveto(250.11753856,288.32341338)(250.13253855,288.6034131)(250.12254366,288.88342059)
\curveto(250.11253857,289.17341253)(250.0825386,289.44841225)(250.03254366,289.70842059)
\curveto(249.9825387,289.96841173)(249.90753877,290.17841152)(249.80754366,290.33842059)
\curveto(249.68753899,290.53841116)(249.53753914,290.68841101)(249.35754366,290.78842059)
\curveto(249.2775394,290.83841086)(249.18753949,290.86841083)(249.08754366,290.87842059)
\curveto(248.98753969,290.8984108)(248.8825398,290.90841079)(248.77254366,290.90842059)
\curveto(248.75253993,290.8984108)(248.72753995,290.89341081)(248.69754366,290.89342059)
\curveto(248.67754,290.9034108)(248.65754002,290.9034108)(248.63754366,290.89342059)
\curveto(248.58754009,290.88341082)(248.54254014,290.87341083)(248.50254366,290.86342059)
\curveto(248.46254022,290.86341084)(248.42254026,290.85341085)(248.38254366,290.83342059)
\curveto(248.20254048,290.75341095)(248.05254063,290.63341107)(247.93254366,290.47342059)
\curveto(247.82254086,290.31341139)(247.73254095,290.13341157)(247.66254366,289.93342059)
\curveto(247.60254108,289.74341196)(247.55754112,289.51841218)(247.52754366,289.25842059)
\curveto(247.50754117,288.9984127)(247.50254118,288.73341297)(247.51254366,288.46342059)
\curveto(247.52254116,288.2034135)(247.55254113,287.95341375)(247.60254366,287.71342059)
\curveto(247.66254102,287.48341422)(247.73754094,287.29341441)(247.82754366,287.14342059)
\moveto(258.62754366,284.15842059)
\curveto(258.63753004,284.10841759)(258.64253004,284.01841768)(258.64254366,283.88842059)
\curveto(258.64253004,283.75841794)(258.63253005,283.66841803)(258.61254366,283.61842059)
\curveto(258.59253009,283.56841813)(258.58753009,283.51341819)(258.59754366,283.45342059)
\curveto(258.60753007,283.4034183)(258.60753007,283.35341835)(258.59754366,283.30342059)
\curveto(258.55753012,283.16341854)(258.52753015,283.02841867)(258.50754366,282.89842059)
\curveto(258.49753018,282.76841893)(258.46753021,282.64841905)(258.41754366,282.53842059)
\curveto(258.2775304,282.18841951)(258.11253057,281.89341981)(257.92254366,281.65342059)
\curveto(257.73253095,281.42342028)(257.46253122,281.23842046)(257.11254366,281.09842059)
\curveto(257.03253165,281.06842063)(256.94753173,281.04842065)(256.85754366,281.03842059)
\curveto(256.76753191,281.01842068)(256.682532,280.9984207)(256.60254366,280.97842059)
\curveto(256.55253213,280.96842073)(256.50253218,280.96342074)(256.45254366,280.96342059)
\curveto(256.40253228,280.96342074)(256.35253233,280.95842074)(256.30254366,280.94842059)
\curveto(256.27253241,280.93842076)(256.22253246,280.93842076)(256.15254366,280.94842059)
\curveto(256.0825326,280.94842075)(256.03253265,280.95342075)(256.00254366,280.96342059)
\curveto(255.94253274,280.98342072)(255.8825328,280.99342071)(255.82254366,280.99342059)
\curveto(255.77253291,280.98342072)(255.72253296,280.98842071)(255.67254366,281.00842059)
\curveto(255.5825331,281.02842067)(255.49253319,281.05342065)(255.40254366,281.08342059)
\curveto(255.32253336,281.1034206)(255.24253344,281.13342057)(255.16254366,281.17342059)
\curveto(254.84253384,281.31342039)(254.59253409,281.50842019)(254.41254366,281.75842059)
\curveto(254.23253445,282.01841968)(254.0825346,282.32341938)(253.96254366,282.67342059)
\curveto(253.94253474,282.75341895)(253.92753475,282.83841886)(253.91754366,282.92842059)
\curveto(253.90753477,283.01841868)(253.89253479,283.1034186)(253.87254366,283.18342059)
\curveto(253.86253482,283.21341849)(253.85753482,283.24341846)(253.85754366,283.27342059)
\lineto(253.85754366,283.37842059)
\curveto(253.83753484,283.45841824)(253.82753485,283.53841816)(253.82754366,283.61842059)
\lineto(253.82754366,283.75342059)
\curveto(253.80753487,283.85341785)(253.80753487,283.95341775)(253.82754366,284.05342059)
\lineto(253.82754366,284.23342059)
\curveto(253.83753484,284.28341742)(253.84253484,284.32841737)(253.84254366,284.36842059)
\curveto(253.84253484,284.41841728)(253.84753483,284.46341724)(253.85754366,284.50342059)
\curveto(253.86753481,284.54341716)(253.87253481,284.57841712)(253.87254366,284.60842059)
\curveto(253.87253481,284.64841705)(253.8775348,284.68841701)(253.88754366,284.72842059)
\lineto(253.94754366,285.05842059)
\curveto(253.96753471,285.17841652)(253.99753468,285.28841641)(254.03754366,285.38842059)
\curveto(254.1775345,285.71841598)(254.33753434,285.99341571)(254.51754366,286.21342059)
\curveto(254.70753397,286.44341526)(254.96753371,286.62841507)(255.29754366,286.76842059)
\curveto(255.3775333,286.80841489)(255.46253322,286.83341487)(255.55254366,286.84342059)
\lineto(255.85254366,286.90342059)
\lineto(255.98754366,286.90342059)
\curveto(256.03753264,286.91341479)(256.08753259,286.91841478)(256.13754366,286.91842059)
\curveto(256.70753197,286.93841476)(257.16753151,286.83341487)(257.51754366,286.60342059)
\curveto(257.8775308,286.38341532)(258.14253054,286.08341562)(258.31254366,285.70342059)
\curveto(258.36253032,285.6034161)(258.40253028,285.5034162)(258.43254366,285.40342059)
\curveto(258.46253022,285.3034164)(258.49253019,285.1984165)(258.52254366,285.08842059)
\curveto(258.53253015,285.04841665)(258.53753014,285.01341669)(258.53754366,284.98342059)
\curveto(258.53753014,284.96341674)(258.54253014,284.93341677)(258.55254366,284.89342059)
\curveto(258.57253011,284.82341688)(258.5825301,284.74841695)(258.58254366,284.66842059)
\curveto(258.5825301,284.58841711)(258.59253009,284.50841719)(258.61254366,284.42842059)
\curveto(258.61253007,284.37841732)(258.61253007,284.33341737)(258.61254366,284.29342059)
\curveto(258.61253007,284.25341745)(258.61753006,284.20841749)(258.62754366,284.15842059)
\moveto(257.51754366,283.72342059)
\curveto(257.52753115,283.77341793)(257.53253115,283.84841785)(257.53254366,283.94842059)
\curveto(257.54253114,284.04841765)(257.53753114,284.12341758)(257.51754366,284.17342059)
\curveto(257.49753118,284.23341747)(257.49253119,284.28841741)(257.50254366,284.33842059)
\curveto(257.52253116,284.3984173)(257.52253116,284.45841724)(257.50254366,284.51842059)
\curveto(257.49253119,284.54841715)(257.48753119,284.58341712)(257.48754366,284.62342059)
\curveto(257.48753119,284.66341704)(257.4825312,284.703417)(257.47254366,284.74342059)
\curveto(257.45253123,284.82341688)(257.43253125,284.8984168)(257.41254366,284.96842059)
\curveto(257.40253128,285.04841665)(257.38753129,285.12841657)(257.36754366,285.20842059)
\curveto(257.33753134,285.26841643)(257.31253137,285.32841637)(257.29254366,285.38842059)
\curveto(257.27253141,285.44841625)(257.24253144,285.50841619)(257.20254366,285.56842059)
\curveto(257.10253158,285.73841596)(256.97253171,285.87341583)(256.81254366,285.97342059)
\curveto(256.73253195,286.02341568)(256.63753204,286.05841564)(256.52754366,286.07842059)
\curveto(256.41753226,286.0984156)(256.29253239,286.10841559)(256.15254366,286.10842059)
\curveto(256.13253255,286.0984156)(256.10753257,286.09341561)(256.07754366,286.09342059)
\curveto(256.04753263,286.1034156)(256.01753266,286.1034156)(255.98754366,286.09342059)
\lineto(255.83754366,286.03342059)
\curveto(255.78753289,286.02341568)(255.74253294,286.00841569)(255.70254366,285.98842059)
\curveto(255.51253317,285.87841582)(255.36753331,285.73341597)(255.26754366,285.55342059)
\curveto(255.1775335,285.37341633)(255.09753358,285.16841653)(255.02754366,284.93842059)
\curveto(254.98753369,284.80841689)(254.96753371,284.67341703)(254.96754366,284.53342059)
\curveto(254.96753371,284.4034173)(254.95753372,284.25841744)(254.93754366,284.09842059)
\curveto(254.92753375,284.04841765)(254.91753376,283.98841771)(254.90754366,283.91842059)
\curveto(254.90753377,283.84841785)(254.91753376,283.78841791)(254.93754366,283.73842059)
\lineto(254.93754366,283.57342059)
\lineto(254.93754366,283.39342059)
\curveto(254.94753373,283.34341836)(254.95753372,283.28841841)(254.96754366,283.22842059)
\curveto(254.9775337,283.17841852)(254.9825337,283.12341858)(254.98254366,283.06342059)
\curveto(254.99253369,283.0034187)(255.00753367,282.94841875)(255.02754366,282.89842059)
\curveto(255.0775336,282.70841899)(255.13753354,282.53341917)(255.20754366,282.37342059)
\curveto(255.2775334,282.21341949)(255.3825333,282.08341962)(255.52254366,281.98342059)
\curveto(255.65253303,281.88341982)(255.79253289,281.81341989)(255.94254366,281.77342059)
\curveto(255.97253271,281.76341994)(255.99753268,281.75841994)(256.01754366,281.75842059)
\curveto(256.04753263,281.76841993)(256.0775326,281.76841993)(256.10754366,281.75842059)
\curveto(256.12753255,281.75841994)(256.15753252,281.75341995)(256.19754366,281.74342059)
\curveto(256.23753244,281.74341996)(256.27253241,281.74841995)(256.30254366,281.75842059)
\curveto(256.34253234,281.76841993)(256.3825323,281.77341993)(256.42254366,281.77342059)
\curveto(256.46253222,281.77341993)(256.50253218,281.78341992)(256.54254366,281.80342059)
\curveto(256.7825319,281.88341982)(256.9775317,282.01841968)(257.12754366,282.20842059)
\curveto(257.24753143,282.38841931)(257.33753134,282.59341911)(257.39754366,282.82342059)
\curveto(257.41753126,282.89341881)(257.43253125,282.96341874)(257.44254366,283.03342059)
\curveto(257.45253123,283.11341859)(257.46753121,283.19341851)(257.48754366,283.27342059)
\curveto(257.48753119,283.33341837)(257.49253119,283.37841832)(257.50254366,283.40842059)
\curveto(257.50253118,283.42841827)(257.50253118,283.45341825)(257.50254366,283.48342059)
\curveto(257.50253118,283.52341818)(257.50753117,283.55341815)(257.51754366,283.57342059)
\lineto(257.51754366,283.72342059)
}
}
{
\newrgbcolor{curcolor}{0 0 0}
\pscustom[linestyle=none,fillstyle=solid,fillcolor=curcolor]
{
\newpath
\moveto(306.87738985,120.9786452)
\curveto(306.9473822,120.92864174)(306.98738216,120.85864181)(306.99738985,120.7686452)
\curveto(307.01738213,120.67864199)(307.02738212,120.57364209)(307.02738985,120.4536452)
\curveto(307.02738212,120.40364226)(307.02238213,120.35364231)(307.01238985,120.3036452)
\curveto(307.01238214,120.25364241)(307.00238215,120.20864246)(306.98238985,120.1686452)
\curveto(306.9523822,120.07864259)(306.89238226,120.01864265)(306.80238985,119.9886452)
\curveto(306.72238243,119.9686427)(306.62738252,119.95864271)(306.51738985,119.9586452)
\lineto(306.20238985,119.9586452)
\curveto(306.09238306,119.9686427)(305.98738316,119.95864271)(305.88738985,119.9286452)
\curveto(305.7473834,119.89864277)(305.65738349,119.81864285)(305.61738985,119.6886452)
\curveto(305.59738355,119.61864305)(305.58738356,119.53364313)(305.58738985,119.4336452)
\lineto(305.58738985,119.1636452)
\lineto(305.58738985,118.2186452)
\lineto(305.58738985,117.8886452)
\curveto(305.58738356,117.77864489)(305.56738358,117.69364497)(305.52738985,117.6336452)
\curveto(305.48738366,117.57364509)(305.43738371,117.53364513)(305.37738985,117.5136452)
\curveto(305.32738382,117.50364516)(305.26238389,117.48864518)(305.18238985,117.4686452)
\lineto(304.98738985,117.4686452)
\curveto(304.86738428,117.4686452)(304.76238439,117.47364519)(304.67238985,117.4836452)
\curveto(304.58238457,117.50364516)(304.51238464,117.55364511)(304.46238985,117.6336452)
\curveto(304.43238472,117.68364498)(304.41738473,117.75364491)(304.41738985,117.8436452)
\lineto(304.41738985,118.1436452)
\lineto(304.41738985,119.1786452)
\curveto(304.41738473,119.33864333)(304.40738474,119.48364318)(304.38738985,119.6136452)
\curveto(304.37738477,119.75364291)(304.32238483,119.84864282)(304.22238985,119.8986452)
\curveto(304.17238498,119.91864275)(304.10238505,119.93364273)(304.01238985,119.9436452)
\curveto(303.93238522,119.95364271)(303.84238531,119.95864271)(303.74238985,119.9586452)
\lineto(303.45738985,119.9586452)
\lineto(303.21738985,119.9586452)
\lineto(300.95238985,119.9586452)
\curveto(300.86238829,119.95864271)(300.75738839,119.95364271)(300.63738985,119.9436452)
\lineto(300.30738985,119.9436452)
\curveto(300.19738895,119.94364272)(300.09738905,119.95364271)(300.00738985,119.9736452)
\curveto(299.91738923,119.99364267)(299.85738929,120.02864264)(299.82738985,120.0786452)
\curveto(299.77738937,120.14864252)(299.7523894,120.24364242)(299.75238985,120.3636452)
\lineto(299.75238985,120.7086452)
\lineto(299.75238985,120.9786452)
\curveto(299.79238936,121.14864152)(299.8473893,121.28864138)(299.91738985,121.3986452)
\curveto(299.98738916,121.50864116)(300.06738908,121.62364104)(300.15738985,121.7436452)
\lineto(300.51738985,122.2836452)
\curveto(300.95738819,122.91363975)(301.39238776,123.53363913)(301.82238985,124.1436452)
\lineto(303.14238985,126.0036452)
\curveto(303.30238585,126.23363643)(303.45738569,126.45363621)(303.60738985,126.6636452)
\curveto(303.75738539,126.88363578)(303.91238524,127.10863556)(304.07238985,127.3386452)
\curveto(304.12238503,127.40863526)(304.17238498,127.47363519)(304.22238985,127.5336452)
\curveto(304.27238488,127.60363506)(304.32238483,127.67863499)(304.37238985,127.7586452)
\lineto(304.43238985,127.8486452)
\curveto(304.46238469,127.88863478)(304.49238466,127.91863475)(304.52238985,127.9386452)
\curveto(304.56238459,127.9686347)(304.60238455,127.98863468)(304.64238985,127.9986452)
\curveto(304.68238447,128.01863465)(304.72738442,128.03863463)(304.77738985,128.0586452)
\curveto(304.79738435,128.05863461)(304.81738433,128.05363461)(304.83738985,128.0436452)
\curveto(304.86738428,128.04363462)(304.89238426,128.05363461)(304.91238985,128.0736452)
\curveto(305.04238411,128.07363459)(305.16238399,128.0686346)(305.27238985,128.0586452)
\curveto(305.38238377,128.04863462)(305.46238369,128.00363466)(305.51238985,127.9236452)
\curveto(305.5523836,127.87363479)(305.57238358,127.80363486)(305.57238985,127.7136452)
\curveto(305.58238357,127.62363504)(305.58738356,127.52863514)(305.58738985,127.4286452)
\lineto(305.58738985,121.9686452)
\curveto(305.58738356,121.89864077)(305.58238357,121.82364084)(305.57238985,121.7436452)
\curveto(305.57238358,121.67364099)(305.57738357,121.60364106)(305.58738985,121.5336452)
\lineto(305.58738985,121.4286452)
\curveto(305.60738354,121.37864129)(305.62238353,121.32364134)(305.63238985,121.2636452)
\curveto(305.64238351,121.21364145)(305.66738348,121.17364149)(305.70738985,121.1436452)
\curveto(305.77738337,121.09364157)(305.86238329,121.0636416)(305.96238985,121.0536452)
\lineto(306.29238985,121.0536452)
\curveto(306.40238275,121.05364161)(306.50738264,121.04864162)(306.60738985,121.0386452)
\curveto(306.71738243,121.03864163)(306.80738234,121.01864165)(306.87738985,120.9786452)
\moveto(304.31238985,121.1736452)
\curveto(304.39238476,121.28364138)(304.42738472,121.45364121)(304.41738985,121.6836452)
\lineto(304.41738985,122.2986452)
\lineto(304.41738985,124.7736452)
\lineto(304.41738985,125.0886452)
\curveto(304.42738472,125.20863746)(304.42238473,125.30863736)(304.40238985,125.3886452)
\lineto(304.40238985,125.5386452)
\curveto(304.40238475,125.62863704)(304.38738476,125.71363695)(304.35738985,125.7936452)
\curveto(304.3473848,125.81363685)(304.33738481,125.82363684)(304.32738985,125.8236452)
\lineto(304.28238985,125.8686452)
\curveto(304.26238489,125.87863679)(304.23238492,125.88363678)(304.19238985,125.8836452)
\curveto(304.17238498,125.8636368)(304.152385,125.84863682)(304.13238985,125.8386452)
\curveto(304.12238503,125.83863683)(304.10738504,125.83363683)(304.08738985,125.8236452)
\curveto(304.02738512,125.77363689)(303.96738518,125.70363696)(303.90738985,125.6136452)
\curveto(303.8473853,125.52363714)(303.79238536,125.44363722)(303.74238985,125.3736452)
\curveto(303.64238551,125.23363743)(303.5473856,125.08863758)(303.45738985,124.9386452)
\curveto(303.36738578,124.79863787)(303.27238588,124.65863801)(303.17238985,124.5186452)
\lineto(302.63238985,123.7386452)
\curveto(302.46238669,123.47863919)(302.28738686,123.21863945)(302.10738985,122.9586452)
\curveto(302.02738712,122.84863982)(301.9523872,122.74363992)(301.88238985,122.6436452)
\lineto(301.67238985,122.3436452)
\curveto(301.62238753,122.2636404)(301.57238758,122.18864048)(301.52238985,122.1186452)
\curveto(301.48238767,122.04864062)(301.43738771,121.97364069)(301.38738985,121.8936452)
\curveto(301.33738781,121.83364083)(301.28738786,121.7686409)(301.23738985,121.6986452)
\curveto(301.19738795,121.63864103)(301.15738799,121.5686411)(301.11738985,121.4886452)
\curveto(301.07738807,121.42864124)(301.0523881,121.35864131)(301.04238985,121.2786452)
\curveto(301.03238812,121.20864146)(301.06738808,121.15364151)(301.14738985,121.1136452)
\curveto(301.21738793,121.0636416)(301.32738782,121.03864163)(301.47738985,121.0386452)
\curveto(301.63738751,121.04864162)(301.77238738,121.05364161)(301.88238985,121.0536452)
\lineto(303.56238985,121.0536452)
\lineto(303.99738985,121.0536452)
\curveto(304.147385,121.05364161)(304.2523849,121.09364157)(304.31238985,121.1736452)
}
}
{
\newrgbcolor{curcolor}{0 0 0}
\pscustom[linestyle=none,fillstyle=solid,fillcolor=curcolor]
{
\newpath
\moveto(308.80699922,127.8936452)
\lineto(313.60699922,127.8936452)
\lineto(314.61199922,127.8936452)
\curveto(314.75199212,127.89363477)(314.871992,127.88363478)(314.97199922,127.8636452)
\curveto(315.08199179,127.85363481)(315.16199171,127.80863486)(315.21199922,127.7286452)
\curveto(315.23199164,127.68863498)(315.24199163,127.63863503)(315.24199922,127.5786452)
\curveto(315.25199162,127.51863515)(315.25699162,127.45363521)(315.25699922,127.3836452)
\lineto(315.25699922,127.1136452)
\curveto(315.25699162,127.02363564)(315.24699163,126.94363572)(315.22699922,126.8736452)
\curveto(315.18699169,126.79363587)(315.14199173,126.72363594)(315.09199922,126.6636452)
\lineto(314.94199922,126.4836452)
\curveto(314.91199196,126.43363623)(314.876992,126.39363627)(314.83699922,126.3636452)
\curveto(314.79699208,126.33363633)(314.75699212,126.29363637)(314.71699922,126.2436452)
\curveto(314.63699224,126.13363653)(314.55199232,126.02363664)(314.46199922,125.9136452)
\curveto(314.3719925,125.81363685)(314.28699259,125.70863696)(314.20699922,125.5986452)
\curveto(314.06699281,125.39863727)(313.92699295,125.18863748)(313.78699922,124.9686452)
\curveto(313.64699323,124.75863791)(313.50699337,124.54363812)(313.36699922,124.3236452)
\curveto(313.31699356,124.23363843)(313.26699361,124.13863853)(313.21699922,124.0386452)
\curveto(313.16699371,123.93863873)(313.11199376,123.84363882)(313.05199922,123.7536452)
\curveto(313.03199384,123.73363893)(313.02199385,123.70863896)(313.02199922,123.6786452)
\curveto(313.02199385,123.64863902)(313.01199386,123.62363904)(312.99199922,123.6036452)
\curveto(312.92199395,123.50363916)(312.85699402,123.38863928)(312.79699922,123.2586452)
\curveto(312.73699414,123.13863953)(312.68199419,123.02363964)(312.63199922,122.9136452)
\curveto(312.53199434,122.68363998)(312.43699444,122.44864022)(312.34699922,122.2086452)
\curveto(312.25699462,121.9686407)(312.15699472,121.72864094)(312.04699922,121.4886452)
\curveto(312.02699485,121.43864123)(312.01199486,121.39364127)(312.00199922,121.3536452)
\curveto(312.00199487,121.31364135)(311.99199488,121.2686414)(311.97199922,121.2186452)
\curveto(311.92199495,121.09864157)(311.876995,120.97364169)(311.83699922,120.8436452)
\curveto(311.80699507,120.72364194)(311.7719951,120.60364206)(311.73199922,120.4836452)
\curveto(311.65199522,120.25364241)(311.58699529,120.01364265)(311.53699922,119.7636452)
\curveto(311.49699538,119.52364314)(311.44699543,119.28364338)(311.38699922,119.0436452)
\curveto(311.34699553,118.89364377)(311.32199555,118.74364392)(311.31199922,118.5936452)
\curveto(311.30199557,118.44364422)(311.28199559,118.29364437)(311.25199922,118.1436452)
\curveto(311.24199563,118.10364456)(311.23699564,118.04364462)(311.23699922,117.9636452)
\curveto(311.20699567,117.84364482)(311.1769957,117.74364492)(311.14699922,117.6636452)
\curveto(311.11699576,117.58364508)(311.04699583,117.52864514)(310.93699922,117.4986452)
\curveto(310.88699599,117.47864519)(310.83199604,117.4686452)(310.77199922,117.4686452)
\lineto(310.57699922,117.4686452)
\curveto(310.43699644,117.4686452)(310.29699658,117.47364519)(310.15699922,117.4836452)
\curveto(310.02699685,117.49364517)(309.93199694,117.53864513)(309.87199922,117.6186452)
\curveto(309.83199704,117.67864499)(309.81199706,117.7636449)(309.81199922,117.8736452)
\curveto(309.82199705,117.98364468)(309.83699704,118.07864459)(309.85699922,118.1586452)
\lineto(309.85699922,118.2336452)
\curveto(309.86699701,118.2636444)(309.871997,118.29364437)(309.87199922,118.3236452)
\curveto(309.89199698,118.40364426)(309.90199697,118.47864419)(309.90199922,118.5486452)
\curveto(309.90199697,118.61864405)(309.91199696,118.68864398)(309.93199922,118.7586452)
\curveto(309.98199689,118.94864372)(310.02199685,119.13364353)(310.05199922,119.3136452)
\curveto(310.08199679,119.50364316)(310.12199675,119.68364298)(310.17199922,119.8536452)
\curveto(310.19199668,119.90364276)(310.20199667,119.94364272)(310.20199922,119.9736452)
\curveto(310.20199667,120.00364266)(310.20699667,120.03864263)(310.21699922,120.0786452)
\curveto(310.31699656,120.37864229)(310.40699647,120.67364199)(310.48699922,120.9636452)
\curveto(310.5769963,121.25364141)(310.68199619,121.53364113)(310.80199922,121.8036452)
\curveto(311.06199581,122.38364028)(311.33199554,122.93363973)(311.61199922,123.4536452)
\curveto(311.89199498,123.98363868)(312.20199467,124.48863818)(312.54199922,124.9686452)
\curveto(312.68199419,125.1686375)(312.83199404,125.35863731)(312.99199922,125.5386452)
\curveto(313.15199372,125.72863694)(313.30199357,125.91863675)(313.44199922,126.1086452)
\curveto(313.48199339,126.15863651)(313.51699336,126.20363646)(313.54699922,126.2436452)
\curveto(313.58699329,126.29363637)(313.62199325,126.34363632)(313.65199922,126.3936452)
\curveto(313.66199321,126.41363625)(313.6719932,126.43863623)(313.68199922,126.4686452)
\curveto(313.70199317,126.49863617)(313.70199317,126.52863614)(313.68199922,126.5586452)
\curveto(313.66199321,126.61863605)(313.62699325,126.65363601)(313.57699922,126.6636452)
\curveto(313.52699335,126.68363598)(313.4769934,126.70363596)(313.42699922,126.7236452)
\lineto(313.32199922,126.7236452)
\curveto(313.28199359,126.73363593)(313.23199364,126.73363593)(313.17199922,126.7236452)
\lineto(313.02199922,126.7236452)
\lineto(312.42199922,126.7236452)
\lineto(309.78199922,126.7236452)
\lineto(309.04699922,126.7236452)
\lineto(308.80699922,126.7236452)
\curveto(308.73699814,126.73363593)(308.6769982,126.74863592)(308.62699922,126.7686452)
\curveto(308.53699834,126.80863586)(308.4769984,126.8686358)(308.44699922,126.9486452)
\curveto(308.39699848,127.04863562)(308.38199849,127.19363547)(308.40199922,127.3836452)
\curveto(308.42199845,127.58363508)(308.45699842,127.71863495)(308.50699922,127.7886452)
\curveto(308.52699835,127.80863486)(308.55199832,127.82363484)(308.58199922,127.8336452)
\lineto(308.70199922,127.8936452)
\curveto(308.72199815,127.89363477)(308.73699814,127.88863478)(308.74699922,127.8786452)
\curveto(308.76699811,127.87863479)(308.78699809,127.88363478)(308.80699922,127.8936452)
}
}
{
\newrgbcolor{curcolor}{0 0 0}
\pscustom[linestyle=none,fillstyle=solid,fillcolor=curcolor]
{
\newpath
\moveto(317.6516086,119.1186452)
\lineto(317.9516086,119.1186452)
\curveto(318.06160654,119.12864354)(318.16660643,119.12864354)(318.2666086,119.1186452)
\curveto(318.37660622,119.11864355)(318.47660612,119.10864356)(318.5666086,119.0886452)
\curveto(318.65660594,119.07864359)(318.72660587,119.05364361)(318.7766086,119.0136452)
\curveto(318.7966058,118.99364367)(318.81160579,118.9636437)(318.8216086,118.9236452)
\curveto(318.84160576,118.88364378)(318.86160574,118.83864383)(318.8816086,118.7886452)
\lineto(318.8816086,118.7136452)
\curveto(318.89160571,118.663644)(318.89160571,118.60864406)(318.8816086,118.5486452)
\lineto(318.8816086,118.3986452)
\lineto(318.8816086,117.9186452)
\curveto(318.88160572,117.74864492)(318.84160576,117.62864504)(318.7616086,117.5586452)
\curveto(318.69160591,117.50864516)(318.601606,117.48364518)(318.4916086,117.4836452)
\lineto(318.1616086,117.4836452)
\lineto(317.7116086,117.4836452)
\curveto(317.56160704,117.48364518)(317.44660715,117.51364515)(317.3666086,117.5736452)
\curveto(317.32660727,117.60364506)(317.2966073,117.65364501)(317.2766086,117.7236452)
\curveto(317.25660734,117.80364486)(317.24160736,117.88864478)(317.2316086,117.9786452)
\lineto(317.2316086,118.2636452)
\curveto(317.24160736,118.3636443)(317.24660735,118.44864422)(317.2466086,118.5186452)
\lineto(317.2466086,118.7136452)
\curveto(317.24660735,118.77364389)(317.25660734,118.82864384)(317.2766086,118.8786452)
\curveto(317.31660728,118.98864368)(317.38660721,119.05864361)(317.4866086,119.0886452)
\curveto(317.51660708,119.08864358)(317.57160703,119.09864357)(317.6516086,119.1186452)
}
}
{
\newrgbcolor{curcolor}{0 0 0}
\pscustom[linestyle=none,fillstyle=solid,fillcolor=curcolor]
{
\newpath
\moveto(323.97176485,128.0886452)
\curveto(325.60175941,128.11863455)(326.65175836,127.5636351)(327.12176485,126.4236452)
\curveto(327.22175779,126.19363647)(327.28675772,125.90363676)(327.31676485,125.5536452)
\curveto(327.35675765,125.21363745)(327.33175768,124.90363776)(327.24176485,124.6236452)
\curveto(327.15175786,124.3636383)(327.03175798,124.13863853)(326.88176485,123.9486452)
\curveto(326.86175815,123.90863876)(326.83675817,123.87363879)(326.80676485,123.8436452)
\curveto(326.77675823,123.82363884)(326.75175826,123.79863887)(326.73176485,123.7686452)
\lineto(326.64176485,123.6486452)
\curveto(326.6117584,123.61863905)(326.57675843,123.59363907)(326.53676485,123.5736452)
\curveto(326.48675852,123.52363914)(326.43175858,123.47863919)(326.37176485,123.4386452)
\curveto(326.32175869,123.39863927)(326.27675873,123.34863932)(326.23676485,123.2886452)
\curveto(326.19675881,123.24863942)(326.18175883,123.19863947)(326.19176485,123.1386452)
\curveto(326.20175881,123.08863958)(326.23175878,123.04363962)(326.28176485,123.0036452)
\curveto(326.33175868,122.9636397)(326.38675862,122.92363974)(326.44676485,122.8836452)
\curveto(326.51675849,122.85363981)(326.58175843,122.82363984)(326.64176485,122.7936452)
\curveto(326.70175831,122.7636399)(326.75175826,122.73363993)(326.79176485,122.7036452)
\curveto(327.1117579,122.48364018)(327.36675764,122.17364049)(327.55676485,121.7736452)
\curveto(327.59675741,121.68364098)(327.62675738,121.58864108)(327.64676485,121.4886452)
\curveto(327.67675733,121.39864127)(327.70175731,121.30864136)(327.72176485,121.2186452)
\curveto(327.73175728,121.1686415)(327.73675727,121.11864155)(327.73676485,121.0686452)
\curveto(327.74675726,121.02864164)(327.75675725,120.98364168)(327.76676485,120.9336452)
\curveto(327.77675723,120.88364178)(327.77675723,120.83364183)(327.76676485,120.7836452)
\curveto(327.75675725,120.73364193)(327.76175725,120.68364198)(327.78176485,120.6336452)
\curveto(327.79175722,120.58364208)(327.79675721,120.52364214)(327.79676485,120.4536452)
\curveto(327.79675721,120.38364228)(327.78675722,120.32364234)(327.76676485,120.2736452)
\lineto(327.76676485,120.0486452)
\lineto(327.70676485,119.8086452)
\curveto(327.69675731,119.73864293)(327.68175733,119.668643)(327.66176485,119.5986452)
\curveto(327.63175738,119.50864316)(327.60175741,119.42364324)(327.57176485,119.3436452)
\curveto(327.55175746,119.2636434)(327.52175749,119.18364348)(327.48176485,119.1036452)
\curveto(327.46175755,119.04364362)(327.43175758,118.98364368)(327.39176485,118.9236452)
\curveto(327.36175765,118.87364379)(327.32675768,118.82364384)(327.28676485,118.7736452)
\curveto(327.08675792,118.4636442)(326.83675817,118.20364446)(326.53676485,117.9936452)
\curveto(326.23675877,117.79364487)(325.89175912,117.62864504)(325.50176485,117.4986452)
\curveto(325.38175963,117.45864521)(325.25175976,117.43364523)(325.11176485,117.4236452)
\curveto(324.98176003,117.40364526)(324.84676016,117.37864529)(324.70676485,117.3486452)
\curveto(324.63676037,117.33864533)(324.56676044,117.33364533)(324.49676485,117.3336452)
\curveto(324.43676057,117.33364533)(324.37176064,117.32864534)(324.30176485,117.3186452)
\curveto(324.26176075,117.30864536)(324.20176081,117.30364536)(324.12176485,117.3036452)
\curveto(324.05176096,117.30364536)(324.00176101,117.30864536)(323.97176485,117.3186452)
\curveto(323.92176109,117.32864534)(323.87676113,117.33364533)(323.83676485,117.3336452)
\lineto(323.71676485,117.3336452)
\curveto(323.61676139,117.35364531)(323.51676149,117.3686453)(323.41676485,117.3786452)
\curveto(323.31676169,117.38864528)(323.22176179,117.40364526)(323.13176485,117.4236452)
\curveto(323.02176199,117.45364521)(322.9117621,117.47864519)(322.80176485,117.4986452)
\curveto(322.70176231,117.52864514)(322.59676241,117.5686451)(322.48676485,117.6186452)
\curveto(322.11676289,117.77864489)(321.80176321,117.97864469)(321.54176485,118.2186452)
\curveto(321.28176373,118.4686442)(321.07176394,118.77864389)(320.91176485,119.1486452)
\curveto(320.87176414,119.23864343)(320.83676417,119.33364333)(320.80676485,119.4336452)
\curveto(320.77676423,119.53364313)(320.74676426,119.63864303)(320.71676485,119.7486452)
\curveto(320.69676431,119.79864287)(320.68676432,119.84864282)(320.68676485,119.8986452)
\curveto(320.68676432,119.95864271)(320.67676433,120.01864265)(320.65676485,120.0786452)
\curveto(320.63676437,120.13864253)(320.62676438,120.21864245)(320.62676485,120.3186452)
\curveto(320.62676438,120.41864225)(320.64176437,120.49364217)(320.67176485,120.5436452)
\curveto(320.68176433,120.57364209)(320.69676431,120.59864207)(320.71676485,120.6186452)
\lineto(320.77676485,120.6786452)
\curveto(320.81676419,120.69864197)(320.87676413,120.71364195)(320.95676485,120.7236452)
\curveto(321.04676396,120.73364193)(321.13676387,120.73864193)(321.22676485,120.7386452)
\curveto(321.31676369,120.73864193)(321.40176361,120.73364193)(321.48176485,120.7236452)
\curveto(321.57176344,120.71364195)(321.63676337,120.70364196)(321.67676485,120.6936452)
\curveto(321.69676331,120.67364199)(321.71676329,120.65864201)(321.73676485,120.6486452)
\curveto(321.75676325,120.64864202)(321.77676323,120.63864203)(321.79676485,120.6186452)
\curveto(321.86676314,120.52864214)(321.9067631,120.41364225)(321.91676485,120.2736452)
\curveto(321.93676307,120.13364253)(321.96676304,120.00864266)(322.00676485,119.8986452)
\lineto(322.15676485,119.5386452)
\curveto(322.2067628,119.42864324)(322.27176274,119.32364334)(322.35176485,119.2236452)
\curveto(322.37176264,119.19364347)(322.39176262,119.1686435)(322.41176485,119.1486452)
\curveto(322.44176257,119.12864354)(322.46676254,119.10364356)(322.48676485,119.0736452)
\curveto(322.52676248,119.01364365)(322.56176245,118.9686437)(322.59176485,118.9386452)
\curveto(322.63176238,118.90864376)(322.66676234,118.87864379)(322.69676485,118.8486452)
\curveto(322.73676227,118.81864385)(322.78176223,118.78864388)(322.83176485,118.7586452)
\curveto(322.92176209,118.69864397)(323.01676199,118.64864402)(323.11676485,118.6086452)
\lineto(323.44676485,118.4886452)
\curveto(323.59676141,118.43864423)(323.79676121,118.40864426)(324.04676485,118.3986452)
\curveto(324.29676071,118.38864428)(324.5067605,118.40864426)(324.67676485,118.4586452)
\curveto(324.75676025,118.47864419)(324.82676018,118.49364417)(324.88676485,118.5036452)
\lineto(325.09676485,118.5636452)
\curveto(325.37675963,118.68364398)(325.61675939,118.83364383)(325.81676485,119.0136452)
\curveto(326.02675898,119.19364347)(326.19175882,119.42364324)(326.31176485,119.7036452)
\curveto(326.34175867,119.77364289)(326.36175865,119.84364282)(326.37176485,119.9136452)
\lineto(326.43176485,120.1536452)
\curveto(326.47175854,120.29364237)(326.48175853,120.45364221)(326.46176485,120.6336452)
\curveto(326.44175857,120.82364184)(326.4117586,120.97364169)(326.37176485,121.0836452)
\curveto(326.24175877,121.4636412)(326.05675895,121.75364091)(325.81676485,121.9536452)
\curveto(325.58675942,122.15364051)(325.27675973,122.31364035)(324.88676485,122.4336452)
\curveto(324.77676023,122.4636402)(324.65676035,122.48364018)(324.52676485,122.4936452)
\curveto(324.4067606,122.50364016)(324.28176073,122.50864016)(324.15176485,122.5086452)
\curveto(323.99176102,122.50864016)(323.85176116,122.51364015)(323.73176485,122.5236452)
\curveto(323.6117614,122.53364013)(323.52676148,122.59364007)(323.47676485,122.7036452)
\curveto(323.45676155,122.73363993)(323.44676156,122.7686399)(323.44676485,122.8086452)
\lineto(323.44676485,122.9436452)
\curveto(323.43676157,123.04363962)(323.43676157,123.13863953)(323.44676485,123.2286452)
\curveto(323.46676154,123.31863935)(323.5067615,123.38363928)(323.56676485,123.4236452)
\curveto(323.6067614,123.45363921)(323.64676136,123.47363919)(323.68676485,123.4836452)
\curveto(323.73676127,123.49363917)(323.79176122,123.50363916)(323.85176485,123.5136452)
\curveto(323.87176114,123.52363914)(323.89676111,123.52363914)(323.92676485,123.5136452)
\curveto(323.95676105,123.51363915)(323.98176103,123.51863915)(324.00176485,123.5286452)
\lineto(324.13676485,123.5286452)
\curveto(324.24676076,123.54863912)(324.34676066,123.55863911)(324.43676485,123.5586452)
\curveto(324.53676047,123.5686391)(324.63176038,123.58863908)(324.72176485,123.6186452)
\curveto(325.04175997,123.72863894)(325.29675971,123.87363879)(325.48676485,124.0536452)
\curveto(325.67675933,124.23363843)(325.82675918,124.48363818)(325.93676485,124.8036452)
\curveto(325.96675904,124.90363776)(325.98675902,125.02863764)(325.99676485,125.1786452)
\curveto(326.01675899,125.33863733)(326.011759,125.48363718)(325.98176485,125.6136452)
\curveto(325.96175905,125.68363698)(325.94175907,125.74863692)(325.92176485,125.8086452)
\curveto(325.9117591,125.87863679)(325.89175912,125.94363672)(325.86176485,126.0036452)
\curveto(325.76175925,126.24363642)(325.61675939,126.43363623)(325.42676485,126.5736452)
\curveto(325.23675977,126.71363595)(325.01176,126.82363584)(324.75176485,126.9036452)
\curveto(324.69176032,126.92363574)(324.63176038,126.93363573)(324.57176485,126.9336452)
\curveto(324.5117605,126.93363573)(324.44676056,126.94363572)(324.37676485,126.9636452)
\curveto(324.29676071,126.98363568)(324.20176081,126.99363567)(324.09176485,126.9936452)
\curveto(323.98176103,126.99363567)(323.88676112,126.98363568)(323.80676485,126.9636452)
\curveto(323.75676125,126.94363572)(323.7067613,126.93363573)(323.65676485,126.9336452)
\curveto(323.61676139,126.93363573)(323.57176144,126.92363574)(323.52176485,126.9036452)
\curveto(323.34176167,126.85363581)(323.17176184,126.77863589)(323.01176485,126.6786452)
\curveto(322.86176215,126.58863608)(322.73176228,126.47363619)(322.62176485,126.3336452)
\curveto(322.53176248,126.21363645)(322.45176256,126.08363658)(322.38176485,125.9436452)
\curveto(322.3117627,125.80363686)(322.24676276,125.64863702)(322.18676485,125.4786452)
\curveto(322.15676285,125.3686373)(322.13676287,125.24863742)(322.12676485,125.1186452)
\curveto(322.11676289,124.99863767)(322.08176293,124.89863777)(322.02176485,124.8186452)
\curveto(322.00176301,124.77863789)(321.94176307,124.73863793)(321.84176485,124.6986452)
\curveto(321.80176321,124.68863798)(321.74176327,124.67863799)(321.66176485,124.6686452)
\lineto(321.40676485,124.6686452)
\curveto(321.31676369,124.67863799)(321.23176378,124.68863798)(321.15176485,124.6986452)
\curveto(321.08176393,124.70863796)(321.03176398,124.72363794)(321.00176485,124.7436452)
\curveto(320.96176405,124.77363789)(320.92676408,124.82863784)(320.89676485,124.9086452)
\curveto(320.86676414,124.98863768)(320.86176415,125.07363759)(320.88176485,125.1636452)
\curveto(320.89176412,125.21363745)(320.89676411,125.2636374)(320.89676485,125.3136452)
\lineto(320.92676485,125.4936452)
\curveto(320.95676405,125.59363707)(320.98176403,125.69363697)(321.00176485,125.7936452)
\curveto(321.03176398,125.89363677)(321.06676394,125.98363668)(321.10676485,126.0636452)
\curveto(321.15676385,126.17363649)(321.20176381,126.27863639)(321.24176485,126.3786452)
\curveto(321.28176373,126.48863618)(321.33176368,126.59363607)(321.39176485,126.6936452)
\curveto(321.72176329,127.23363543)(322.19176282,127.62863504)(322.80176485,127.8786452)
\curveto(322.92176209,127.92863474)(323.04676196,127.9636347)(323.17676485,127.9836452)
\curveto(323.31676169,128.00363466)(323.45676155,128.02863464)(323.59676485,128.0586452)
\curveto(323.65676135,128.0686346)(323.71676129,128.07363459)(323.77676485,128.0736452)
\curveto(323.84676116,128.07363459)(323.9117611,128.07863459)(323.97176485,128.0886452)
}
}
{
\newrgbcolor{curcolor}{0 0 0}
\pscustom[linestyle=none,fillstyle=solid,fillcolor=curcolor]
{
\newpath
\moveto(339.01137422,126.0036452)
\curveto(338.81136392,125.71363695)(338.60136413,125.42863724)(338.38137422,125.1486452)
\curveto(338.17136456,124.8686378)(337.96636477,124.58363808)(337.76637422,124.2936452)
\curveto(337.16636557,123.44363922)(336.56136617,122.60364006)(335.95137422,121.7736452)
\curveto(335.34136739,120.95364171)(334.736368,120.11864255)(334.13637422,119.2686452)
\lineto(333.62637422,118.5486452)
\lineto(333.11637422,117.8586452)
\curveto(333.0363697,117.74864492)(332.95636978,117.63364503)(332.87637422,117.5136452)
\curveto(332.79636994,117.39364527)(332.70137003,117.29864537)(332.59137422,117.2286452)
\curveto(332.55137018,117.20864546)(332.48637025,117.19364547)(332.39637422,117.1836452)
\curveto(332.31637042,117.1636455)(332.22637051,117.15364551)(332.12637422,117.1536452)
\curveto(332.02637071,117.15364551)(331.9313708,117.15864551)(331.84137422,117.1686452)
\curveto(331.76137097,117.17864549)(331.70137103,117.19864547)(331.66137422,117.2286452)
\curveto(331.6313711,117.24864542)(331.60637113,117.28364538)(331.58637422,117.3336452)
\curveto(331.57637116,117.37364529)(331.58137115,117.41864525)(331.60137422,117.4686452)
\curveto(331.64137109,117.54864512)(331.68637105,117.62364504)(331.73637422,117.6936452)
\curveto(331.79637094,117.77364489)(331.85137088,117.85364481)(331.90137422,117.9336452)
\curveto(332.14137059,118.27364439)(332.38637035,118.60864406)(332.63637422,118.9386452)
\curveto(332.88636985,119.2686434)(333.12636961,119.60364306)(333.35637422,119.9436452)
\curveto(333.51636922,120.1636425)(333.67636906,120.37864229)(333.83637422,120.5886452)
\curveto(333.99636874,120.79864187)(334.15636858,121.01364165)(334.31637422,121.2336452)
\curveto(334.67636806,121.75364091)(335.04136769,122.2636404)(335.41137422,122.7636452)
\curveto(335.78136695,123.2636394)(336.15136658,123.77363889)(336.52137422,124.2936452)
\curveto(336.66136607,124.49363817)(336.80136593,124.68863798)(336.94137422,124.8786452)
\curveto(337.09136564,125.0686376)(337.2363655,125.2636374)(337.37637422,125.4636452)
\curveto(337.58636515,125.7636369)(337.80136493,126.0636366)(338.02137422,126.3636452)
\lineto(338.68137422,127.2636452)
\lineto(338.86137422,127.5336452)
\lineto(339.07137422,127.8036452)
\lineto(339.19137422,127.9836452)
\curveto(339.24136349,128.04363462)(339.29136344,128.09863457)(339.34137422,128.1486452)
\curveto(339.41136332,128.19863447)(339.48636325,128.23363443)(339.56637422,128.2536452)
\curveto(339.58636315,128.2636344)(339.61136312,128.2636344)(339.64137422,128.2536452)
\curveto(339.68136305,128.25363441)(339.71136302,128.2636344)(339.73137422,128.2836452)
\curveto(339.85136288,128.28363438)(339.98636275,128.27863439)(340.13637422,128.2686452)
\curveto(340.28636245,128.2686344)(340.37636236,128.22363444)(340.40637422,128.1336452)
\curveto(340.42636231,128.10363456)(340.4313623,128.0686346)(340.42137422,128.0286452)
\curveto(340.41136232,127.98863468)(340.39636234,127.95863471)(340.37637422,127.9386452)
\curveto(340.3363624,127.85863481)(340.29636244,127.78863488)(340.25637422,127.7286452)
\curveto(340.21636252,127.668635)(340.17136256,127.60863506)(340.12137422,127.5486452)
\lineto(339.55137422,126.7686452)
\curveto(339.37136336,126.51863615)(339.19136354,126.2636364)(339.01137422,126.0036452)
\moveto(332.15637422,122.1036452)
\curveto(332.10637063,122.12364054)(332.05637068,122.12864054)(332.00637422,122.1186452)
\curveto(331.95637078,122.10864056)(331.90637083,122.11364055)(331.85637422,122.1336452)
\curveto(331.74637099,122.15364051)(331.64137109,122.17364049)(331.54137422,122.1936452)
\curveto(331.45137128,122.22364044)(331.35637138,122.2636404)(331.25637422,122.3136452)
\curveto(330.92637181,122.45364021)(330.67137206,122.64864002)(330.49137422,122.8986452)
\curveto(330.31137242,123.15863951)(330.16637257,123.4686392)(330.05637422,123.8286452)
\curveto(330.02637271,123.90863876)(330.00637273,123.98863868)(329.99637422,124.0686452)
\curveto(329.98637275,124.15863851)(329.97137276,124.24363842)(329.95137422,124.3236452)
\curveto(329.94137279,124.37363829)(329.9363728,124.43863823)(329.93637422,124.5186452)
\curveto(329.92637281,124.54863812)(329.92137281,124.57863809)(329.92137422,124.6086452)
\curveto(329.92137281,124.64863802)(329.91637282,124.68363798)(329.90637422,124.7136452)
\lineto(329.90637422,124.8636452)
\curveto(329.89637284,124.91363775)(329.89137284,124.97363769)(329.89137422,125.0436452)
\curveto(329.89137284,125.12363754)(329.89637284,125.18863748)(329.90637422,125.2386452)
\lineto(329.90637422,125.4036452)
\curveto(329.92637281,125.45363721)(329.9313728,125.49863717)(329.92137422,125.5386452)
\curveto(329.92137281,125.58863708)(329.92637281,125.63363703)(329.93637422,125.6736452)
\curveto(329.94637279,125.71363695)(329.95137278,125.74863692)(329.95137422,125.7786452)
\curveto(329.95137278,125.81863685)(329.95637278,125.85863681)(329.96637422,125.8986452)
\curveto(329.99637274,126.00863666)(330.01637272,126.11863655)(330.02637422,126.2286452)
\curveto(330.04637269,126.34863632)(330.08137265,126.4636362)(330.13137422,126.5736452)
\curveto(330.27137246,126.91363575)(330.4313723,127.18863548)(330.61137422,127.3986452)
\curveto(330.80137193,127.61863505)(331.07137166,127.79863487)(331.42137422,127.9386452)
\curveto(331.50137123,127.9686347)(331.58637115,127.98863468)(331.67637422,127.9986452)
\curveto(331.76637097,128.01863465)(331.86137087,128.03863463)(331.96137422,128.0586452)
\curveto(331.99137074,128.0686346)(332.04637069,128.0686346)(332.12637422,128.0586452)
\curveto(332.20637053,128.05863461)(332.25637048,128.0686346)(332.27637422,128.0886452)
\curveto(332.8363699,128.09863457)(333.28636945,127.98863468)(333.62637422,127.7586452)
\curveto(333.97636876,127.52863514)(334.2363685,127.22363544)(334.40637422,126.8436452)
\curveto(334.44636829,126.75363591)(334.48136825,126.65863601)(334.51137422,126.5586452)
\curveto(334.54136819,126.45863621)(334.56636817,126.35863631)(334.58637422,126.2586452)
\curveto(334.60636813,126.22863644)(334.61136812,126.19863647)(334.60137422,126.1686452)
\curveto(334.60136813,126.13863653)(334.60636813,126.10863656)(334.61637422,126.0786452)
\curveto(334.64636809,125.9686367)(334.66636807,125.84363682)(334.67637422,125.7036452)
\curveto(334.68636805,125.57363709)(334.69636804,125.43863723)(334.70637422,125.2986452)
\lineto(334.70637422,125.1336452)
\curveto(334.71636802,125.07363759)(334.71636802,125.01863765)(334.70637422,124.9686452)
\curveto(334.69636804,124.91863775)(334.69136804,124.8686378)(334.69137422,124.8186452)
\lineto(334.69137422,124.6836452)
\curveto(334.68136805,124.64363802)(334.67636806,124.60363806)(334.67637422,124.5636452)
\curveto(334.68636805,124.52363814)(334.68136805,124.47863819)(334.66137422,124.4286452)
\curveto(334.64136809,124.31863835)(334.62136811,124.21363845)(334.60137422,124.1136452)
\curveto(334.59136814,124.01363865)(334.57136816,123.91363875)(334.54137422,123.8136452)
\curveto(334.41136832,123.45363921)(334.24636849,123.13863953)(334.04637422,122.8686452)
\curveto(333.84636889,122.59864007)(333.57136916,122.39364027)(333.22137422,122.2536452)
\curveto(333.14136959,122.22364044)(333.05636968,122.19864047)(332.96637422,122.1786452)
\lineto(332.69637422,122.1186452)
\curveto(332.64637009,122.10864056)(332.60137013,122.10364056)(332.56137422,122.1036452)
\curveto(332.52137021,122.11364055)(332.48137025,122.11364055)(332.44137422,122.1036452)
\curveto(332.34137039,122.08364058)(332.24637049,122.08364058)(332.15637422,122.1036452)
\moveto(331.31637422,123.4986452)
\curveto(331.35637138,123.42863924)(331.39637134,123.3636393)(331.43637422,123.3036452)
\curveto(331.47637126,123.25363941)(331.52637121,123.20363946)(331.58637422,123.1536452)
\lineto(331.73637422,123.0336452)
\curveto(331.79637094,123.00363966)(331.86137087,122.97863969)(331.93137422,122.9586452)
\curveto(331.97137076,122.93863973)(332.00637073,122.92863974)(332.03637422,122.9286452)
\curveto(332.07637066,122.93863973)(332.11637062,122.93363973)(332.15637422,122.9136452)
\curveto(332.18637055,122.91363975)(332.22637051,122.90863976)(332.27637422,122.8986452)
\curveto(332.32637041,122.89863977)(332.36637037,122.90363976)(332.39637422,122.9136452)
\lineto(332.62137422,122.9586452)
\curveto(332.87136986,123.03863963)(333.05636968,123.1636395)(333.17637422,123.3336452)
\curveto(333.25636948,123.43363923)(333.32636941,123.5636391)(333.38637422,123.7236452)
\curveto(333.46636927,123.90363876)(333.52636921,124.12863854)(333.56637422,124.3986452)
\curveto(333.60636913,124.67863799)(333.62136911,124.95863771)(333.61137422,125.2386452)
\curveto(333.60136913,125.52863714)(333.57136916,125.80363686)(333.52137422,126.0636452)
\curveto(333.47136926,126.32363634)(333.39636934,126.53363613)(333.29637422,126.6936452)
\curveto(333.17636956,126.89363577)(333.02636971,127.04363562)(332.84637422,127.1436452)
\curveto(332.76636997,127.19363547)(332.67637006,127.22363544)(332.57637422,127.2336452)
\curveto(332.47637026,127.25363541)(332.37137036,127.2636354)(332.26137422,127.2636452)
\curveto(332.24137049,127.25363541)(332.21637052,127.24863542)(332.18637422,127.2486452)
\curveto(332.16637057,127.25863541)(332.14637059,127.25863541)(332.12637422,127.2486452)
\curveto(332.07637066,127.23863543)(332.0313707,127.22863544)(331.99137422,127.2186452)
\curveto(331.95137078,127.21863545)(331.91137082,127.20863546)(331.87137422,127.1886452)
\curveto(331.69137104,127.10863556)(331.54137119,126.98863568)(331.42137422,126.8286452)
\curveto(331.31137142,126.668636)(331.22137151,126.48863618)(331.15137422,126.2886452)
\curveto(331.09137164,126.09863657)(331.04637169,125.87363679)(331.01637422,125.6136452)
\curveto(330.99637174,125.35363731)(330.99137174,125.08863758)(331.00137422,124.8186452)
\curveto(331.01137172,124.55863811)(331.04137169,124.30863836)(331.09137422,124.0686452)
\curveto(331.15137158,123.83863883)(331.22637151,123.64863902)(331.31637422,123.4986452)
\moveto(342.11637422,120.5136452)
\curveto(342.12636061,120.4636422)(342.1313606,120.37364229)(342.13137422,120.2436452)
\curveto(342.1313606,120.11364255)(342.12136061,120.02364264)(342.10137422,119.9736452)
\curveto(342.08136065,119.92364274)(342.07636066,119.8686428)(342.08637422,119.8086452)
\curveto(342.09636064,119.75864291)(342.09636064,119.70864296)(342.08637422,119.6586452)
\curveto(342.04636069,119.51864315)(342.01636072,119.38364328)(341.99637422,119.2536452)
\curveto(341.98636075,119.12364354)(341.95636078,119.00364366)(341.90637422,118.8936452)
\curveto(341.76636097,118.54364412)(341.60136113,118.24864442)(341.41137422,118.0086452)
\curveto(341.22136151,117.77864489)(340.95136178,117.59364507)(340.60137422,117.4536452)
\curveto(340.52136221,117.42364524)(340.4363623,117.40364526)(340.34637422,117.3936452)
\curveto(340.25636248,117.37364529)(340.17136256,117.35364531)(340.09137422,117.3336452)
\curveto(340.04136269,117.32364534)(339.99136274,117.31864535)(339.94137422,117.3186452)
\curveto(339.89136284,117.31864535)(339.84136289,117.31364535)(339.79137422,117.3036452)
\curveto(339.76136297,117.29364537)(339.71136302,117.29364537)(339.64137422,117.3036452)
\curveto(339.57136316,117.30364536)(339.52136321,117.30864536)(339.49137422,117.3186452)
\curveto(339.4313633,117.33864533)(339.37136336,117.34864532)(339.31137422,117.3486452)
\curveto(339.26136347,117.33864533)(339.21136352,117.34364532)(339.16137422,117.3636452)
\curveto(339.07136366,117.38364528)(338.98136375,117.40864526)(338.89137422,117.4386452)
\curveto(338.81136392,117.45864521)(338.731364,117.48864518)(338.65137422,117.5286452)
\curveto(338.3313644,117.668645)(338.08136465,117.8636448)(337.90137422,118.1136452)
\curveto(337.72136501,118.37364429)(337.57136516,118.67864399)(337.45137422,119.0286452)
\curveto(337.4313653,119.10864356)(337.41636532,119.19364347)(337.40637422,119.2836452)
\curveto(337.39636534,119.37364329)(337.38136535,119.45864321)(337.36137422,119.5386452)
\curveto(337.35136538,119.5686431)(337.34636539,119.59864307)(337.34637422,119.6286452)
\lineto(337.34637422,119.7336452)
\curveto(337.32636541,119.81364285)(337.31636542,119.89364277)(337.31637422,119.9736452)
\lineto(337.31637422,120.1086452)
\curveto(337.29636544,120.20864246)(337.29636544,120.30864236)(337.31637422,120.4086452)
\lineto(337.31637422,120.5886452)
\curveto(337.32636541,120.63864203)(337.3313654,120.68364198)(337.33137422,120.7236452)
\curveto(337.3313654,120.77364189)(337.3363654,120.81864185)(337.34637422,120.8586452)
\curveto(337.35636538,120.89864177)(337.36136537,120.93364173)(337.36137422,120.9636452)
\curveto(337.36136537,121.00364166)(337.36636537,121.04364162)(337.37637422,121.0836452)
\lineto(337.43637422,121.4136452)
\curveto(337.45636528,121.53364113)(337.48636525,121.64364102)(337.52637422,121.7436452)
\curveto(337.66636507,122.07364059)(337.82636491,122.34864032)(338.00637422,122.5686452)
\curveto(338.19636454,122.79863987)(338.45636428,122.98363968)(338.78637422,123.1236452)
\curveto(338.86636387,123.1636395)(338.95136378,123.18863948)(339.04137422,123.1986452)
\lineto(339.34137422,123.2586452)
\lineto(339.47637422,123.2586452)
\curveto(339.52636321,123.2686394)(339.57636316,123.27363939)(339.62637422,123.2736452)
\curveto(340.19636254,123.29363937)(340.65636208,123.18863948)(341.00637422,122.9586452)
\curveto(341.36636137,122.73863993)(341.6313611,122.43864023)(341.80137422,122.0586452)
\curveto(341.85136088,121.95864071)(341.89136084,121.85864081)(341.92137422,121.7586452)
\curveto(341.95136078,121.65864101)(341.98136075,121.55364111)(342.01137422,121.4436452)
\curveto(342.02136071,121.40364126)(342.02636071,121.3686413)(342.02637422,121.3386452)
\curveto(342.02636071,121.31864135)(342.0313607,121.28864138)(342.04137422,121.2486452)
\curveto(342.06136067,121.17864149)(342.07136066,121.10364156)(342.07137422,121.0236452)
\curveto(342.07136066,120.94364172)(342.08136065,120.8636418)(342.10137422,120.7836452)
\curveto(342.10136063,120.73364193)(342.10136063,120.68864198)(342.10137422,120.6486452)
\curveto(342.10136063,120.60864206)(342.10636063,120.5636421)(342.11637422,120.5136452)
\moveto(341.00637422,120.0786452)
\curveto(341.01636172,120.12864254)(341.02136171,120.20364246)(341.02137422,120.3036452)
\curveto(341.0313617,120.40364226)(341.02636171,120.47864219)(341.00637422,120.5286452)
\curveto(340.98636175,120.58864208)(340.98136175,120.64364202)(340.99137422,120.6936452)
\curveto(341.01136172,120.75364191)(341.01136172,120.81364185)(340.99137422,120.8736452)
\curveto(340.98136175,120.90364176)(340.97636176,120.93864173)(340.97637422,120.9786452)
\curveto(340.97636176,121.01864165)(340.97136176,121.05864161)(340.96137422,121.0986452)
\curveto(340.94136179,121.17864149)(340.92136181,121.25364141)(340.90137422,121.3236452)
\curveto(340.89136184,121.40364126)(340.87636186,121.48364118)(340.85637422,121.5636452)
\curveto(340.82636191,121.62364104)(340.80136193,121.68364098)(340.78137422,121.7436452)
\curveto(340.76136197,121.80364086)(340.731362,121.8636408)(340.69137422,121.9236452)
\curveto(340.59136214,122.09364057)(340.46136227,122.22864044)(340.30137422,122.3286452)
\curveto(340.22136251,122.37864029)(340.12636261,122.41364025)(340.01637422,122.4336452)
\curveto(339.90636283,122.45364021)(339.78136295,122.4636402)(339.64137422,122.4636452)
\curveto(339.62136311,122.45364021)(339.59636314,122.44864022)(339.56637422,122.4486452)
\curveto(339.5363632,122.45864021)(339.50636323,122.45864021)(339.47637422,122.4486452)
\lineto(339.32637422,122.3886452)
\curveto(339.27636346,122.37864029)(339.2313635,122.3636403)(339.19137422,122.3436452)
\curveto(339.00136373,122.23364043)(338.85636388,122.08864058)(338.75637422,121.9086452)
\curveto(338.66636407,121.72864094)(338.58636415,121.52364114)(338.51637422,121.2936452)
\curveto(338.47636426,121.1636415)(338.45636428,121.02864164)(338.45637422,120.8886452)
\curveto(338.45636428,120.75864191)(338.44636429,120.61364205)(338.42637422,120.4536452)
\curveto(338.41636432,120.40364226)(338.40636433,120.34364232)(338.39637422,120.2736452)
\curveto(338.39636434,120.20364246)(338.40636433,120.14364252)(338.42637422,120.0936452)
\lineto(338.42637422,119.9286452)
\lineto(338.42637422,119.7486452)
\curveto(338.4363643,119.69864297)(338.44636429,119.64364302)(338.45637422,119.5836452)
\curveto(338.46636427,119.53364313)(338.47136426,119.47864319)(338.47137422,119.4186452)
\curveto(338.48136425,119.35864331)(338.49636424,119.30364336)(338.51637422,119.2536452)
\curveto(338.56636417,119.0636436)(338.62636411,118.88864378)(338.69637422,118.7286452)
\curveto(338.76636397,118.5686441)(338.87136386,118.43864423)(339.01137422,118.3386452)
\curveto(339.14136359,118.23864443)(339.28136345,118.1686445)(339.43137422,118.1286452)
\curveto(339.46136327,118.11864455)(339.48636325,118.11364455)(339.50637422,118.1136452)
\curveto(339.5363632,118.12364454)(339.56636317,118.12364454)(339.59637422,118.1136452)
\curveto(339.61636312,118.11364455)(339.64636309,118.10864456)(339.68637422,118.0986452)
\curveto(339.72636301,118.09864457)(339.76136297,118.10364456)(339.79137422,118.1136452)
\curveto(339.8313629,118.12364454)(339.87136286,118.12864454)(339.91137422,118.1286452)
\curveto(339.95136278,118.12864454)(339.99136274,118.13864453)(340.03137422,118.1586452)
\curveto(340.27136246,118.23864443)(340.46636227,118.37364429)(340.61637422,118.5636452)
\curveto(340.736362,118.74364392)(340.82636191,118.94864372)(340.88637422,119.1786452)
\curveto(340.90636183,119.24864342)(340.92136181,119.31864335)(340.93137422,119.3886452)
\curveto(340.94136179,119.4686432)(340.95636178,119.54864312)(340.97637422,119.6286452)
\curveto(340.97636176,119.68864298)(340.98136175,119.73364293)(340.99137422,119.7636452)
\curveto(340.99136174,119.78364288)(340.99136174,119.80864286)(340.99137422,119.8386452)
\curveto(340.99136174,119.87864279)(340.99636174,119.90864276)(341.00637422,119.9286452)
\lineto(341.00637422,120.0786452)
}
}
{
\newrgbcolor{curcolor}{0 0 0}
\pscustom[linestyle=none,fillstyle=solid,fillcolor=curcolor]
{
\newpath
\moveto(84.91944246,187.68764422)
\curveto(86.54943702,187.71763357)(87.59943597,187.16263413)(88.06944246,186.02264422)
\curveto(88.1694354,185.7926355)(88.23443534,185.50263579)(88.26444246,185.15264422)
\curveto(88.30443527,184.81263648)(88.27943529,184.50263679)(88.18944246,184.22264422)
\curveto(88.09943547,183.96263733)(87.97943559,183.73763755)(87.82944246,183.54764422)
\curveto(87.80943576,183.50763778)(87.78443579,183.47263782)(87.75444246,183.44264422)
\curveto(87.72443585,183.42263787)(87.69943587,183.39763789)(87.67944246,183.36764422)
\lineto(87.58944246,183.24764422)
\curveto(87.55943601,183.21763807)(87.52443605,183.1926381)(87.48444246,183.17264422)
\curveto(87.43443614,183.12263817)(87.37943619,183.07763821)(87.31944246,183.03764422)
\curveto(87.2694363,182.99763829)(87.22443635,182.94763834)(87.18444246,182.88764422)
\curveto(87.14443643,182.84763844)(87.12943644,182.79763849)(87.13944246,182.73764422)
\curveto(87.14943642,182.6876386)(87.17943639,182.64263865)(87.22944246,182.60264422)
\curveto(87.27943629,182.56263873)(87.33443624,182.52263877)(87.39444246,182.48264422)
\curveto(87.46443611,182.45263884)(87.52943604,182.42263887)(87.58944246,182.39264422)
\curveto(87.64943592,182.36263893)(87.69943587,182.33263896)(87.73944246,182.30264422)
\curveto(88.05943551,182.08263921)(88.31443526,181.77263952)(88.50444246,181.37264422)
\curveto(88.54443503,181.28264001)(88.574435,181.1876401)(88.59444246,181.08764422)
\curveto(88.62443495,180.99764029)(88.64943492,180.90764038)(88.66944246,180.81764422)
\curveto(88.67943489,180.76764052)(88.68443489,180.71764057)(88.68444246,180.66764422)
\curveto(88.69443488,180.62764066)(88.70443487,180.58264071)(88.71444246,180.53264422)
\curveto(88.72443485,180.48264081)(88.72443485,180.43264086)(88.71444246,180.38264422)
\curveto(88.70443487,180.33264096)(88.70943486,180.28264101)(88.72944246,180.23264422)
\curveto(88.73943483,180.18264111)(88.74443483,180.12264117)(88.74444246,180.05264422)
\curveto(88.74443483,179.98264131)(88.73443484,179.92264137)(88.71444246,179.87264422)
\lineto(88.71444246,179.64764422)
\lineto(88.65444246,179.40764422)
\curveto(88.64443493,179.33764195)(88.62943494,179.26764202)(88.60944246,179.19764422)
\curveto(88.57943499,179.10764218)(88.54943502,179.02264227)(88.51944246,178.94264422)
\curveto(88.49943507,178.86264243)(88.4694351,178.78264251)(88.42944246,178.70264422)
\curveto(88.40943516,178.64264265)(88.37943519,178.58264271)(88.33944246,178.52264422)
\curveto(88.30943526,178.47264282)(88.2744353,178.42264287)(88.23444246,178.37264422)
\curveto(88.03443554,178.06264323)(87.78443579,177.80264349)(87.48444246,177.59264422)
\curveto(87.18443639,177.3926439)(86.83943673,177.22764406)(86.44944246,177.09764422)
\curveto(86.32943724,177.05764423)(86.19943737,177.03264426)(86.05944246,177.02264422)
\curveto(85.92943764,177.00264429)(85.79443778,176.97764431)(85.65444246,176.94764422)
\curveto(85.58443799,176.93764435)(85.51443806,176.93264436)(85.44444246,176.93264422)
\curveto(85.38443819,176.93264436)(85.31943825,176.92764436)(85.24944246,176.91764422)
\curveto(85.20943836,176.90764438)(85.14943842,176.90264439)(85.06944246,176.90264422)
\curveto(84.99943857,176.90264439)(84.94943862,176.90764438)(84.91944246,176.91764422)
\curveto(84.8694387,176.92764436)(84.82443875,176.93264436)(84.78444246,176.93264422)
\lineto(84.66444246,176.93264422)
\curveto(84.56443901,176.95264434)(84.46443911,176.96764432)(84.36444246,176.97764422)
\curveto(84.26443931,176.9876443)(84.1694394,177.00264429)(84.07944246,177.02264422)
\curveto(83.9694396,177.05264424)(83.85943971,177.07764421)(83.74944246,177.09764422)
\curveto(83.64943992,177.12764416)(83.54444003,177.16764412)(83.43444246,177.21764422)
\curveto(83.06444051,177.37764391)(82.74944082,177.57764371)(82.48944246,177.81764422)
\curveto(82.22944134,178.06764322)(82.01944155,178.37764291)(81.85944246,178.74764422)
\curveto(81.81944175,178.83764245)(81.78444179,178.93264236)(81.75444246,179.03264422)
\curveto(81.72444185,179.13264216)(81.69444188,179.23764205)(81.66444246,179.34764422)
\curveto(81.64444193,179.39764189)(81.63444194,179.44764184)(81.63444246,179.49764422)
\curveto(81.63444194,179.55764173)(81.62444195,179.61764167)(81.60444246,179.67764422)
\curveto(81.58444199,179.73764155)(81.574442,179.81764147)(81.57444246,179.91764422)
\curveto(81.574442,180.01764127)(81.58944198,180.0926412)(81.61944246,180.14264422)
\curveto(81.62944194,180.17264112)(81.64444193,180.19764109)(81.66444246,180.21764422)
\lineto(81.72444246,180.27764422)
\curveto(81.76444181,180.29764099)(81.82444175,180.31264098)(81.90444246,180.32264422)
\curveto(81.99444158,180.33264096)(82.08444149,180.33764095)(82.17444246,180.33764422)
\curveto(82.26444131,180.33764095)(82.34944122,180.33264096)(82.42944246,180.32264422)
\curveto(82.51944105,180.31264098)(82.58444099,180.30264099)(82.62444246,180.29264422)
\curveto(82.64444093,180.27264102)(82.66444091,180.25764103)(82.68444246,180.24764422)
\curveto(82.70444087,180.24764104)(82.72444085,180.23764105)(82.74444246,180.21764422)
\curveto(82.81444076,180.12764116)(82.85444072,180.01264128)(82.86444246,179.87264422)
\curveto(82.88444069,179.73264156)(82.91444066,179.60764168)(82.95444246,179.49764422)
\lineto(83.10444246,179.13764422)
\curveto(83.15444042,179.02764226)(83.21944035,178.92264237)(83.29944246,178.82264422)
\curveto(83.31944025,178.7926425)(83.33944023,178.76764252)(83.35944246,178.74764422)
\curveto(83.38944018,178.72764256)(83.41444016,178.70264259)(83.43444246,178.67264422)
\curveto(83.4744401,178.61264268)(83.50944006,178.56764272)(83.53944246,178.53764422)
\curveto(83.57943999,178.50764278)(83.61443996,178.47764281)(83.64444246,178.44764422)
\curveto(83.68443989,178.41764287)(83.72943984,178.3876429)(83.77944246,178.35764422)
\curveto(83.8694397,178.29764299)(83.96443961,178.24764304)(84.06444246,178.20764422)
\lineto(84.39444246,178.08764422)
\curveto(84.54443903,178.03764325)(84.74443883,178.00764328)(84.99444246,177.99764422)
\curveto(85.24443833,177.9876433)(85.45443812,178.00764328)(85.62444246,178.05764422)
\curveto(85.70443787,178.07764321)(85.7744378,178.0926432)(85.83444246,178.10264422)
\lineto(86.04444246,178.16264422)
\curveto(86.32443725,178.28264301)(86.56443701,178.43264286)(86.76444246,178.61264422)
\curveto(86.9744366,178.7926425)(87.13943643,179.02264227)(87.25944246,179.30264422)
\curveto(87.28943628,179.37264192)(87.30943626,179.44264185)(87.31944246,179.51264422)
\lineto(87.37944246,179.75264422)
\curveto(87.41943615,179.8926414)(87.42943614,180.05264124)(87.40944246,180.23264422)
\curveto(87.38943618,180.42264087)(87.35943621,180.57264072)(87.31944246,180.68264422)
\curveto(87.18943638,181.06264023)(87.00443657,181.35263994)(86.76444246,181.55264422)
\curveto(86.53443704,181.75263954)(86.22443735,181.91263938)(85.83444246,182.03264422)
\curveto(85.72443785,182.06263923)(85.60443797,182.08263921)(85.47444246,182.09264422)
\curveto(85.35443822,182.10263919)(85.22943834,182.10763918)(85.09944246,182.10764422)
\curveto(84.93943863,182.10763918)(84.79943877,182.11263918)(84.67944246,182.12264422)
\curveto(84.55943901,182.13263916)(84.4744391,182.1926391)(84.42444246,182.30264422)
\curveto(84.40443917,182.33263896)(84.39443918,182.36763892)(84.39444246,182.40764422)
\lineto(84.39444246,182.54264422)
\curveto(84.38443919,182.64263865)(84.38443919,182.73763855)(84.39444246,182.82764422)
\curveto(84.41443916,182.91763837)(84.45443912,182.98263831)(84.51444246,183.02264422)
\curveto(84.55443902,183.05263824)(84.59443898,183.07263822)(84.63444246,183.08264422)
\curveto(84.68443889,183.0926382)(84.73943883,183.10263819)(84.79944246,183.11264422)
\curveto(84.81943875,183.12263817)(84.84443873,183.12263817)(84.87444246,183.11264422)
\curveto(84.90443867,183.11263818)(84.92943864,183.11763817)(84.94944246,183.12764422)
\lineto(85.08444246,183.12764422)
\curveto(85.19443838,183.14763814)(85.29443828,183.15763813)(85.38444246,183.15764422)
\curveto(85.48443809,183.16763812)(85.57943799,183.1876381)(85.66944246,183.21764422)
\curveto(85.98943758,183.32763796)(86.24443733,183.47263782)(86.43444246,183.65264422)
\curveto(86.62443695,183.83263746)(86.7744368,184.08263721)(86.88444246,184.40264422)
\curveto(86.91443666,184.50263679)(86.93443664,184.62763666)(86.94444246,184.77764422)
\curveto(86.96443661,184.93763635)(86.95943661,185.08263621)(86.92944246,185.21264422)
\curveto(86.90943666,185.28263601)(86.88943668,185.34763594)(86.86944246,185.40764422)
\curveto(86.85943671,185.47763581)(86.83943673,185.54263575)(86.80944246,185.60264422)
\curveto(86.70943686,185.84263545)(86.56443701,186.03263526)(86.37444246,186.17264422)
\curveto(86.18443739,186.31263498)(85.95943761,186.42263487)(85.69944246,186.50264422)
\curveto(85.63943793,186.52263477)(85.57943799,186.53263476)(85.51944246,186.53264422)
\curveto(85.45943811,186.53263476)(85.39443818,186.54263475)(85.32444246,186.56264422)
\curveto(85.24443833,186.58263471)(85.14943842,186.5926347)(85.03944246,186.59264422)
\curveto(84.92943864,186.5926347)(84.83443874,186.58263471)(84.75444246,186.56264422)
\curveto(84.70443887,186.54263475)(84.65443892,186.53263476)(84.60444246,186.53264422)
\curveto(84.56443901,186.53263476)(84.51943905,186.52263477)(84.46944246,186.50264422)
\curveto(84.28943928,186.45263484)(84.11943945,186.37763491)(83.95944246,186.27764422)
\curveto(83.80943976,186.1876351)(83.67943989,186.07263522)(83.56944246,185.93264422)
\curveto(83.47944009,185.81263548)(83.39944017,185.68263561)(83.32944246,185.54264422)
\curveto(83.25944031,185.40263589)(83.19444038,185.24763604)(83.13444246,185.07764422)
\curveto(83.10444047,184.96763632)(83.08444049,184.84763644)(83.07444246,184.71764422)
\curveto(83.06444051,184.59763669)(83.02944054,184.49763679)(82.96944246,184.41764422)
\curveto(82.94944062,184.37763691)(82.88944068,184.33763695)(82.78944246,184.29764422)
\curveto(82.74944082,184.287637)(82.68944088,184.27763701)(82.60944246,184.26764422)
\lineto(82.35444246,184.26764422)
\curveto(82.26444131,184.27763701)(82.17944139,184.287637)(82.09944246,184.29764422)
\curveto(82.02944154,184.30763698)(81.97944159,184.32263697)(81.94944246,184.34264422)
\curveto(81.90944166,184.37263692)(81.8744417,184.42763686)(81.84444246,184.50764422)
\curveto(81.81444176,184.5876367)(81.80944176,184.67263662)(81.82944246,184.76264422)
\curveto(81.83944173,184.81263648)(81.84444173,184.86263643)(81.84444246,184.91264422)
\lineto(81.87444246,185.09264422)
\curveto(81.90444167,185.1926361)(81.92944164,185.292636)(81.94944246,185.39264422)
\curveto(81.97944159,185.4926358)(82.01444156,185.58263571)(82.05444246,185.66264422)
\curveto(82.10444147,185.77263552)(82.14944142,185.87763541)(82.18944246,185.97764422)
\curveto(82.22944134,186.0876352)(82.27944129,186.1926351)(82.33944246,186.29264422)
\curveto(82.6694409,186.83263446)(83.13944043,187.22763406)(83.74944246,187.47764422)
\curveto(83.8694397,187.52763376)(83.99443958,187.56263373)(84.12444246,187.58264422)
\curveto(84.26443931,187.60263369)(84.40443917,187.62763366)(84.54444246,187.65764422)
\curveto(84.60443897,187.66763362)(84.66443891,187.67263362)(84.72444246,187.67264422)
\curveto(84.79443878,187.67263362)(84.85943871,187.67763361)(84.91944246,187.68764422)
}
}
{
\newrgbcolor{curcolor}{0 0 0}
\pscustom[linestyle=none,fillstyle=solid,fillcolor=curcolor]
{
\newpath
\moveto(90.61405184,187.49264422)
\lineto(95.41405184,187.49264422)
\lineto(96.41905184,187.49264422)
\curveto(96.55904474,187.4926338)(96.67904462,187.48263381)(96.77905184,187.46264422)
\curveto(96.88904441,187.45263384)(96.96904433,187.40763388)(97.01905184,187.32764422)
\curveto(97.03904426,187.287634)(97.04904425,187.23763405)(97.04905184,187.17764422)
\curveto(97.05904424,187.11763417)(97.06404423,187.05263424)(97.06405184,186.98264422)
\lineto(97.06405184,186.71264422)
\curveto(97.06404423,186.62263467)(97.05404424,186.54263475)(97.03405184,186.47264422)
\curveto(96.9940443,186.3926349)(96.94904435,186.32263497)(96.89905184,186.26264422)
\lineto(96.74905184,186.08264422)
\curveto(96.71904458,186.03263526)(96.68404461,185.9926353)(96.64405184,185.96264422)
\curveto(96.60404469,185.93263536)(96.56404473,185.8926354)(96.52405184,185.84264422)
\curveto(96.44404485,185.73263556)(96.35904494,185.62263567)(96.26905184,185.51264422)
\curveto(96.17904512,185.41263588)(96.0940452,185.30763598)(96.01405184,185.19764422)
\curveto(95.87404542,184.99763629)(95.73404556,184.7876365)(95.59405184,184.56764422)
\curveto(95.45404584,184.35763693)(95.31404598,184.14263715)(95.17405184,183.92264422)
\curveto(95.12404617,183.83263746)(95.07404622,183.73763755)(95.02405184,183.63764422)
\curveto(94.97404632,183.53763775)(94.91904638,183.44263785)(94.85905184,183.35264422)
\curveto(94.83904646,183.33263796)(94.82904647,183.30763798)(94.82905184,183.27764422)
\curveto(94.82904647,183.24763804)(94.81904648,183.22263807)(94.79905184,183.20264422)
\curveto(94.72904657,183.10263819)(94.66404663,182.9876383)(94.60405184,182.85764422)
\curveto(94.54404675,182.73763855)(94.48904681,182.62263867)(94.43905184,182.51264422)
\curveto(94.33904696,182.28263901)(94.24404705,182.04763924)(94.15405184,181.80764422)
\curveto(94.06404723,181.56763972)(93.96404733,181.32763996)(93.85405184,181.08764422)
\curveto(93.83404746,181.03764025)(93.81904748,180.9926403)(93.80905184,180.95264422)
\curveto(93.80904749,180.91264038)(93.7990475,180.86764042)(93.77905184,180.81764422)
\curveto(93.72904757,180.69764059)(93.68404761,180.57264072)(93.64405184,180.44264422)
\curveto(93.61404768,180.32264097)(93.57904772,180.20264109)(93.53905184,180.08264422)
\curveto(93.45904784,179.85264144)(93.3940479,179.61264168)(93.34405184,179.36264422)
\curveto(93.30404799,179.12264217)(93.25404804,178.88264241)(93.19405184,178.64264422)
\curveto(93.15404814,178.4926428)(93.12904817,178.34264295)(93.11905184,178.19264422)
\curveto(93.10904819,178.04264325)(93.08904821,177.8926434)(93.05905184,177.74264422)
\curveto(93.04904825,177.70264359)(93.04404825,177.64264365)(93.04405184,177.56264422)
\curveto(93.01404828,177.44264385)(92.98404831,177.34264395)(92.95405184,177.26264422)
\curveto(92.92404837,177.18264411)(92.85404844,177.12764416)(92.74405184,177.09764422)
\curveto(92.6940486,177.07764421)(92.63904866,177.06764422)(92.57905184,177.06764422)
\lineto(92.38405184,177.06764422)
\curveto(92.24404905,177.06764422)(92.10404919,177.07264422)(91.96405184,177.08264422)
\curveto(91.83404946,177.0926442)(91.73904956,177.13764415)(91.67905184,177.21764422)
\curveto(91.63904966,177.27764401)(91.61904968,177.36264393)(91.61905184,177.47264422)
\curveto(91.62904967,177.58264371)(91.64404965,177.67764361)(91.66405184,177.75764422)
\lineto(91.66405184,177.83264422)
\curveto(91.67404962,177.86264343)(91.67904962,177.8926434)(91.67905184,177.92264422)
\curveto(91.6990496,178.00264329)(91.70904959,178.07764321)(91.70905184,178.14764422)
\curveto(91.70904959,178.21764307)(91.71904958,178.287643)(91.73905184,178.35764422)
\curveto(91.78904951,178.54764274)(91.82904947,178.73264256)(91.85905184,178.91264422)
\curveto(91.88904941,179.10264219)(91.92904937,179.28264201)(91.97905184,179.45264422)
\curveto(91.9990493,179.50264179)(92.00904929,179.54264175)(92.00905184,179.57264422)
\curveto(92.00904929,179.60264169)(92.01404928,179.63764165)(92.02405184,179.67764422)
\curveto(92.12404917,179.97764131)(92.21404908,180.27264102)(92.29405184,180.56264422)
\curveto(92.38404891,180.85264044)(92.48904881,181.13264016)(92.60905184,181.40264422)
\curveto(92.86904843,181.98263931)(93.13904816,182.53263876)(93.41905184,183.05264422)
\curveto(93.6990476,183.58263771)(94.00904729,184.0876372)(94.34905184,184.56764422)
\curveto(94.48904681,184.76763652)(94.63904666,184.95763633)(94.79905184,185.13764422)
\curveto(94.95904634,185.32763596)(95.10904619,185.51763577)(95.24905184,185.70764422)
\curveto(95.28904601,185.75763553)(95.32404597,185.80263549)(95.35405184,185.84264422)
\curveto(95.3940459,185.8926354)(95.42904587,185.94263535)(95.45905184,185.99264422)
\curveto(95.46904583,186.01263528)(95.47904582,186.03763525)(95.48905184,186.06764422)
\curveto(95.50904579,186.09763519)(95.50904579,186.12763516)(95.48905184,186.15764422)
\curveto(95.46904583,186.21763507)(95.43404586,186.25263504)(95.38405184,186.26264422)
\curveto(95.33404596,186.28263501)(95.28404601,186.30263499)(95.23405184,186.32264422)
\lineto(95.12905184,186.32264422)
\curveto(95.08904621,186.33263496)(95.03904626,186.33263496)(94.97905184,186.32264422)
\lineto(94.82905184,186.32264422)
\lineto(94.22905184,186.32264422)
\lineto(91.58905184,186.32264422)
\lineto(90.85405184,186.32264422)
\lineto(90.61405184,186.32264422)
\curveto(90.54405075,186.33263496)(90.48405081,186.34763494)(90.43405184,186.36764422)
\curveto(90.34405095,186.40763488)(90.28405101,186.46763482)(90.25405184,186.54764422)
\curveto(90.20405109,186.64763464)(90.18905111,186.7926345)(90.20905184,186.98264422)
\curveto(90.22905107,187.18263411)(90.26405103,187.31763397)(90.31405184,187.38764422)
\curveto(90.33405096,187.40763388)(90.35905094,187.42263387)(90.38905184,187.43264422)
\lineto(90.50905184,187.49264422)
\curveto(90.52905077,187.4926338)(90.54405075,187.4876338)(90.55405184,187.47764422)
\curveto(90.57405072,187.47763381)(90.5940507,187.48263381)(90.61405184,187.49264422)
}
}
{
\newrgbcolor{curcolor}{0 0 0}
\pscustom[linestyle=none,fillstyle=solid,fillcolor=curcolor]
{
\newpath
\moveto(99.45866121,178.71764422)
\lineto(99.75866121,178.71764422)
\curveto(99.86865915,178.72764256)(99.97365905,178.72764256)(100.07366121,178.71764422)
\curveto(100.18365884,178.71764257)(100.28365874,178.70764258)(100.37366121,178.68764422)
\curveto(100.46365856,178.67764261)(100.53365849,178.65264264)(100.58366121,178.61264422)
\curveto(100.60365842,178.5926427)(100.6186584,178.56264273)(100.62866121,178.52264422)
\curveto(100.64865837,178.48264281)(100.66865835,178.43764285)(100.68866121,178.38764422)
\lineto(100.68866121,178.31264422)
\curveto(100.69865832,178.26264303)(100.69865832,178.20764308)(100.68866121,178.14764422)
\lineto(100.68866121,177.99764422)
\lineto(100.68866121,177.51764422)
\curveto(100.68865833,177.34764394)(100.64865837,177.22764406)(100.56866121,177.15764422)
\curveto(100.49865852,177.10764418)(100.40865861,177.08264421)(100.29866121,177.08264422)
\lineto(99.96866121,177.08264422)
\lineto(99.51866121,177.08264422)
\curveto(99.36865965,177.08264421)(99.25365977,177.11264418)(99.17366121,177.17264422)
\curveto(99.13365989,177.20264409)(99.10365992,177.25264404)(99.08366121,177.32264422)
\curveto(99.06365996,177.40264389)(99.04865997,177.4876438)(99.03866121,177.57764422)
\lineto(99.03866121,177.86264422)
\curveto(99.04865997,177.96264333)(99.05365997,178.04764324)(99.05366121,178.11764422)
\lineto(99.05366121,178.31264422)
\curveto(99.05365997,178.37264292)(99.06365996,178.42764286)(99.08366121,178.47764422)
\curveto(99.1236599,178.5876427)(99.19365983,178.65764263)(99.29366121,178.68764422)
\curveto(99.3236597,178.6876426)(99.37865964,178.69764259)(99.45866121,178.71764422)
}
}
{
\newrgbcolor{curcolor}{0 0 0}
\pscustom[linestyle=none,fillstyle=solid,fillcolor=curcolor]
{
\newpath
\moveto(109.60381746,182.67764422)
\curveto(109.60380983,182.59763869)(109.60880982,182.51763877)(109.61881746,182.43764422)
\curveto(109.6288098,182.35763893)(109.62380981,182.28263901)(109.60381746,182.21264422)
\curveto(109.58380985,182.17263912)(109.57880985,182.12763916)(109.58881746,182.07764422)
\curveto(109.59880983,182.03763925)(109.59880983,181.99763929)(109.58881746,181.95764422)
\lineto(109.58881746,181.80764422)
\curveto(109.57880985,181.71763957)(109.57380986,181.62763966)(109.57381746,181.53764422)
\curveto(109.57380986,181.45763983)(109.56880986,181.37763991)(109.55881746,181.29764422)
\lineto(109.52881746,181.05764422)
\curveto(109.51880991,180.9876403)(109.50880992,180.91264038)(109.49881746,180.83264422)
\curveto(109.48880994,180.7926405)(109.48380995,180.75264054)(109.48381746,180.71264422)
\curveto(109.48380995,180.67264062)(109.47880995,180.62764066)(109.46881746,180.57764422)
\curveto(109.42881,180.43764085)(109.39881003,180.29764099)(109.37881746,180.15764422)
\curveto(109.36881006,180.01764127)(109.33881009,179.88264141)(109.28881746,179.75264422)
\curveto(109.23881019,179.58264171)(109.18381025,179.41764187)(109.12381746,179.25764422)
\curveto(109.07381036,179.09764219)(109.01381042,178.94264235)(108.94381746,178.79264422)
\curveto(108.92381051,178.73264256)(108.89381054,178.67264262)(108.85381746,178.61264422)
\lineto(108.76381746,178.46264422)
\curveto(108.56381087,178.14264315)(108.34881108,177.87764341)(108.11881746,177.66764422)
\curveto(107.88881154,177.45764383)(107.59381184,177.27764401)(107.23381746,177.12764422)
\curveto(107.11381232,177.07764421)(106.98381245,177.04264425)(106.84381746,177.02264422)
\curveto(106.71381272,177.00264429)(106.57881285,176.97764431)(106.43881746,176.94764422)
\curveto(106.37881305,176.93764435)(106.31881311,176.93264436)(106.25881746,176.93264422)
\curveto(106.19881323,176.93264436)(106.1338133,176.92764436)(106.06381746,176.91764422)
\curveto(106.0338134,176.90764438)(105.98381345,176.90764438)(105.91381746,176.91764422)
\lineto(105.76381746,176.91764422)
\lineto(105.61381746,176.91764422)
\curveto(105.5338139,176.93764435)(105.44881398,176.95264434)(105.35881746,176.96264422)
\curveto(105.27881415,176.96264433)(105.20381423,176.97264432)(105.13381746,176.99264422)
\curveto(105.09381434,177.00264429)(105.05881437,177.00764428)(105.02881746,177.00764422)
\curveto(105.00881442,176.99764429)(104.98381445,177.00264429)(104.95381746,177.02264422)
\lineto(104.68381746,177.08264422)
\curveto(104.59381484,177.11264418)(104.50881492,177.14264415)(104.42881746,177.17264422)
\curveto(103.84881558,177.41264388)(103.41381602,177.78264351)(103.12381746,178.28264422)
\curveto(103.04381639,178.41264288)(102.97881645,178.54764274)(102.92881746,178.68764422)
\curveto(102.88881654,178.82764246)(102.84381659,178.97764231)(102.79381746,179.13764422)
\curveto(102.77381666,179.21764207)(102.76881666,179.29764199)(102.77881746,179.37764422)
\curveto(102.79881663,179.45764183)(102.8338166,179.51264178)(102.88381746,179.54264422)
\curveto(102.91381652,179.56264173)(102.96881646,179.57764171)(103.04881746,179.58764422)
\curveto(103.1288163,179.60764168)(103.21381622,179.61764167)(103.30381746,179.61764422)
\curveto(103.39381604,179.62764166)(103.47881595,179.62764166)(103.55881746,179.61764422)
\curveto(103.64881578,179.60764168)(103.71881571,179.59764169)(103.76881746,179.58764422)
\curveto(103.78881564,179.57764171)(103.81381562,179.56264173)(103.84381746,179.54264422)
\curveto(103.88381555,179.52264177)(103.91381552,179.50264179)(103.93381746,179.48264422)
\curveto(103.99381544,179.40264189)(104.03881539,179.30764198)(104.06881746,179.19764422)
\curveto(104.10881532,179.0876422)(104.15381528,178.9876423)(104.20381746,178.89764422)
\curveto(104.45381498,178.50764278)(104.82381461,178.23764305)(105.31381746,178.08764422)
\curveto(105.38381405,178.06764322)(105.45381398,178.05264324)(105.52381746,178.04264422)
\curveto(105.60381383,178.04264325)(105.68381375,178.03264326)(105.76381746,178.01264422)
\curveto(105.80381363,178.00264329)(105.85881357,177.99764329)(105.92881746,177.99764422)
\curveto(106.00881342,177.99764329)(106.06381337,178.00264329)(106.09381746,178.01264422)
\curveto(106.12381331,178.02264327)(106.15381328,178.02764326)(106.18381746,178.02764422)
\lineto(106.28881746,178.02764422)
\curveto(106.36881306,178.04764324)(106.44381299,178.06764322)(106.51381746,178.08764422)
\curveto(106.59381284,178.10764318)(106.66881276,178.13264316)(106.73881746,178.16264422)
\curveto(107.08881234,178.31264298)(107.35881207,178.52764276)(107.54881746,178.80764422)
\curveto(107.73881169,179.0876422)(107.89381154,179.41264188)(108.01381746,179.78264422)
\curveto(108.04381139,179.86264143)(108.06381137,179.93764135)(108.07381746,180.00764422)
\curveto(108.09381134,180.07764121)(108.11381132,180.15264114)(108.13381746,180.23264422)
\curveto(108.15381128,180.32264097)(108.16881126,180.41764087)(108.17881746,180.51764422)
\curveto(108.19881123,180.62764066)(108.21881121,180.73264056)(108.23881746,180.83264422)
\curveto(108.24881118,180.88264041)(108.25381118,180.93264036)(108.25381746,180.98264422)
\curveto(108.26381117,181.04264025)(108.26881116,181.09764019)(108.26881746,181.14764422)
\curveto(108.28881114,181.20764008)(108.29881113,181.28264001)(108.29881746,181.37264422)
\curveto(108.29881113,181.47263982)(108.28881114,181.55263974)(108.26881746,181.61264422)
\curveto(108.23881119,181.70263959)(108.18881124,181.74263955)(108.11881746,181.73264422)
\curveto(108.05881137,181.72263957)(108.00381143,181.6926396)(107.95381746,181.64264422)
\curveto(107.87381156,181.5926397)(107.80381163,181.53263976)(107.74381746,181.46264422)
\curveto(107.69381174,181.3926399)(107.6288118,181.33263996)(107.54881746,181.28264422)
\curveto(107.38881204,181.17264012)(107.22381221,181.07264022)(107.05381746,180.98264422)
\curveto(106.88381255,180.90264039)(106.68881274,180.83264046)(106.46881746,180.77264422)
\curveto(106.36881306,180.74264055)(106.26881316,180.72764056)(106.16881746,180.72764422)
\curveto(106.07881335,180.72764056)(105.97881345,180.71764057)(105.86881746,180.69764422)
\lineto(105.71881746,180.69764422)
\curveto(105.66881376,180.71764057)(105.61881381,180.72264057)(105.56881746,180.71264422)
\curveto(105.5288139,180.70264059)(105.48881394,180.70264059)(105.44881746,180.71264422)
\curveto(105.41881401,180.72264057)(105.37381406,180.72764056)(105.31381746,180.72764422)
\curveto(105.25381418,180.73764055)(105.18881424,180.74764054)(105.11881746,180.75764422)
\lineto(104.93881746,180.78764422)
\curveto(104.48881494,180.90764038)(104.10881532,181.07264022)(103.79881746,181.28264422)
\curveto(103.5288159,181.47263982)(103.29881613,181.70263959)(103.10881746,181.97264422)
\curveto(102.9288165,182.25263904)(102.78381665,182.56763872)(102.67381746,182.91764422)
\lineto(102.61381746,183.12764422)
\curveto(102.60381683,183.20763808)(102.58881684,183.287638)(102.56881746,183.36764422)
\curveto(102.55881687,183.39763789)(102.55381688,183.42763786)(102.55381746,183.45764422)
\curveto(102.55381688,183.4876378)(102.54881688,183.51763777)(102.53881746,183.54764422)
\curveto(102.5288169,183.60763768)(102.52381691,183.66763762)(102.52381746,183.72764422)
\curveto(102.52381691,183.79763749)(102.51381692,183.85763743)(102.49381746,183.90764422)
\lineto(102.49381746,184.08764422)
\curveto(102.48381695,184.13763715)(102.47881695,184.20763708)(102.47881746,184.29764422)
\curveto(102.47881695,184.3876369)(102.48881694,184.45763683)(102.50881746,184.50764422)
\lineto(102.50881746,184.67264422)
\curveto(102.5288169,184.75263654)(102.53881689,184.82763646)(102.53881746,184.89764422)
\curveto(102.54881688,184.96763632)(102.56381687,185.03763625)(102.58381746,185.10764422)
\curveto(102.64381679,185.30763598)(102.70381673,185.49763579)(102.76381746,185.67764422)
\curveto(102.8338166,185.85763543)(102.92381651,186.02763526)(103.03381746,186.18764422)
\curveto(103.07381636,186.25763503)(103.11381632,186.32263497)(103.15381746,186.38264422)
\lineto(103.30381746,186.56264422)
\curveto(103.32381611,186.57263472)(103.34381609,186.5876347)(103.36381746,186.60764422)
\curveto(103.45381598,186.73763455)(103.56381587,186.84763444)(103.69381746,186.93764422)
\curveto(103.95381548,187.13763415)(104.21881521,187.292634)(104.48881746,187.40264422)
\curveto(104.56881486,187.44263385)(104.64881478,187.47263382)(104.72881746,187.49264422)
\curveto(104.81881461,187.52263377)(104.90881452,187.54763374)(104.99881746,187.56764422)
\curveto(105.09881433,187.59763369)(105.19881423,187.61763367)(105.29881746,187.62764422)
\curveto(105.39881403,187.63763365)(105.50381393,187.65263364)(105.61381746,187.67264422)
\curveto(105.64381379,187.68263361)(105.68381375,187.68263361)(105.73381746,187.67264422)
\curveto(105.79381364,187.66263363)(105.8338136,187.66763362)(105.85381746,187.68764422)
\curveto(106.57381286,187.70763358)(107.17381226,187.5926337)(107.65381746,187.34264422)
\curveto(108.1338113,187.0926342)(108.50881092,186.75263454)(108.77881746,186.32264422)
\curveto(108.86881056,186.18263511)(108.94881048,186.03763525)(109.01881746,185.88764422)
\curveto(109.08881034,185.73763555)(109.15881027,185.57763571)(109.22881746,185.40764422)
\curveto(109.27881015,185.26763602)(109.31881011,185.11763617)(109.34881746,184.95764422)
\curveto(109.37881005,184.79763649)(109.41381002,184.63763665)(109.45381746,184.47764422)
\curveto(109.47380996,184.42763686)(109.48380995,184.37263692)(109.48381746,184.31264422)
\curveto(109.48380995,184.26263703)(109.48880994,184.21263708)(109.49881746,184.16264422)
\curveto(109.51880991,184.10263719)(109.5288099,184.03763725)(109.52881746,183.96764422)
\curveto(109.5288099,183.90763738)(109.53880989,183.85263744)(109.55881746,183.80264422)
\lineto(109.55881746,183.63764422)
\curveto(109.57880985,183.5876377)(109.58380985,183.53763775)(109.57381746,183.48764422)
\curveto(109.56380987,183.43763785)(109.56880986,183.3876379)(109.58881746,183.33764422)
\curveto(109.58880984,183.31763797)(109.58380985,183.292638)(109.57381746,183.26264422)
\curveto(109.57380986,183.23263806)(109.57880985,183.20763808)(109.58881746,183.18764422)
\curveto(109.59880983,183.15763813)(109.59880983,183.12263817)(109.58881746,183.08264422)
\curveto(109.58880984,183.04263825)(109.59380984,183.00263829)(109.60381746,182.96264422)
\curveto(109.61380982,182.92263837)(109.61380982,182.87763841)(109.60381746,182.82764422)
\lineto(109.60381746,182.67764422)
\moveto(108.10381746,183.98264422)
\curveto(108.11381132,184.03263726)(108.11881131,184.0926372)(108.11881746,184.16264422)
\curveto(108.11881131,184.23263706)(108.11381132,184.292637)(108.10381746,184.34264422)
\curveto(108.09381134,184.3926369)(108.08881134,184.46763682)(108.08881746,184.56764422)
\curveto(108.06881136,184.64763664)(108.04881138,184.72263657)(108.02881746,184.79264422)
\curveto(108.01881141,184.86263643)(108.00381143,184.93263636)(107.98381746,185.00264422)
\curveto(107.84381159,185.43263586)(107.64881178,185.76763552)(107.39881746,186.00764422)
\curveto(107.15881227,186.24763504)(106.81381262,186.42763486)(106.36381746,186.54764422)
\curveto(106.27381316,186.56763472)(106.17381326,186.57763471)(106.06381746,186.57764422)
\lineto(105.73381746,186.57764422)
\curveto(105.71381372,186.55763473)(105.67881375,186.54763474)(105.62881746,186.54764422)
\curveto(105.57881385,186.55763473)(105.5338139,186.55763473)(105.49381746,186.54764422)
\curveto(105.41381402,186.52763476)(105.33881409,186.50763478)(105.26881746,186.48764422)
\lineto(105.05881746,186.42764422)
\curveto(104.76881466,186.29763499)(104.53881489,186.11763517)(104.36881746,185.88764422)
\curveto(104.19881523,185.66763562)(104.06381537,185.40763588)(103.96381746,185.10764422)
\curveto(103.9338155,185.01763627)(103.90881552,184.92263637)(103.88881746,184.82264422)
\curveto(103.87881555,184.73263656)(103.86381557,184.63763665)(103.84381746,184.53764422)
\lineto(103.84381746,184.40264422)
\curveto(103.81381562,184.292637)(103.80381563,184.15263714)(103.81381746,183.98264422)
\curveto(103.8338156,183.82263747)(103.85381558,183.6926376)(103.87381746,183.59264422)
\curveto(103.89381554,183.53263776)(103.90881552,183.47263782)(103.91881746,183.41264422)
\curveto(103.9288155,183.36263793)(103.94381549,183.31263798)(103.96381746,183.26264422)
\curveto(104.04381539,183.06263823)(104.13881529,182.87263842)(104.24881746,182.69264422)
\curveto(104.36881506,182.51263878)(104.50881492,182.36763892)(104.66881746,182.25764422)
\curveto(104.71881471,182.20763908)(104.77381466,182.16763912)(104.83381746,182.13764422)
\curveto(104.89381454,182.10763918)(104.95381448,182.07263922)(105.01381746,182.03264422)
\curveto(105.16381427,181.95263934)(105.34881408,181.8876394)(105.56881746,181.83764422)
\curveto(105.61881381,181.81763947)(105.65881377,181.81263948)(105.68881746,181.82264422)
\curveto(105.7288137,181.83263946)(105.77381366,181.82763946)(105.82381746,181.80764422)
\curveto(105.86381357,181.79763949)(105.91881351,181.7926395)(105.98881746,181.79264422)
\curveto(106.05881337,181.7926395)(106.11881331,181.79763949)(106.16881746,181.80764422)
\curveto(106.26881316,181.82763946)(106.36381307,181.84263945)(106.45381746,181.85264422)
\curveto(106.54381289,181.87263942)(106.6338128,181.90263939)(106.72381746,181.94264422)
\curveto(107.26381217,182.16263913)(107.65881177,182.55763873)(107.90881746,183.12764422)
\curveto(107.95881147,183.22763806)(107.99381144,183.32763796)(108.01381746,183.42764422)
\curveto(108.0338114,183.53763775)(108.05881137,183.64763764)(108.08881746,183.75764422)
\curveto(108.08881134,183.85763743)(108.09381134,183.93263736)(108.10381746,183.98264422)
}
}
{
\newrgbcolor{curcolor}{0 0 0}
\pscustom[linestyle=none,fillstyle=solid,fillcolor=curcolor]
{
\newpath
\moveto(120.81842684,185.60264422)
\curveto(120.61841654,185.31263598)(120.40841675,185.02763626)(120.18842684,184.74764422)
\curveto(119.97841718,184.46763682)(119.77341738,184.18263711)(119.57342684,183.89264422)
\curveto(118.97341818,183.04263825)(118.36841879,182.20263909)(117.75842684,181.37264422)
\curveto(117.14842001,180.55264074)(116.54342061,179.71764157)(115.94342684,178.86764422)
\lineto(115.43342684,178.14764422)
\lineto(114.92342684,177.45764422)
\curveto(114.84342231,177.34764394)(114.76342239,177.23264406)(114.68342684,177.11264422)
\curveto(114.60342255,176.9926443)(114.50842265,176.89764439)(114.39842684,176.82764422)
\curveto(114.3584228,176.80764448)(114.29342286,176.7926445)(114.20342684,176.78264422)
\curveto(114.12342303,176.76264453)(114.03342312,176.75264454)(113.93342684,176.75264422)
\curveto(113.83342332,176.75264454)(113.73842342,176.75764453)(113.64842684,176.76764422)
\curveto(113.56842359,176.77764451)(113.50842365,176.79764449)(113.46842684,176.82764422)
\curveto(113.43842372,176.84764444)(113.41342374,176.88264441)(113.39342684,176.93264422)
\curveto(113.38342377,176.97264432)(113.38842377,177.01764427)(113.40842684,177.06764422)
\curveto(113.44842371,177.14764414)(113.49342366,177.22264407)(113.54342684,177.29264422)
\curveto(113.60342355,177.37264392)(113.6584235,177.45264384)(113.70842684,177.53264422)
\curveto(113.94842321,177.87264342)(114.19342296,178.20764308)(114.44342684,178.53764422)
\curveto(114.69342246,178.86764242)(114.93342222,179.20264209)(115.16342684,179.54264422)
\curveto(115.32342183,179.76264153)(115.48342167,179.97764131)(115.64342684,180.18764422)
\curveto(115.80342135,180.39764089)(115.96342119,180.61264068)(116.12342684,180.83264422)
\curveto(116.48342067,181.35263994)(116.84842031,181.86263943)(117.21842684,182.36264422)
\curveto(117.58841957,182.86263843)(117.9584192,183.37263792)(118.32842684,183.89264422)
\curveto(118.46841869,184.0926372)(118.60841855,184.287637)(118.74842684,184.47764422)
\curveto(118.89841826,184.66763662)(119.04341811,184.86263643)(119.18342684,185.06264422)
\curveto(119.39341776,185.36263593)(119.60841755,185.66263563)(119.82842684,185.96264422)
\lineto(120.48842684,186.86264422)
\lineto(120.66842684,187.13264422)
\lineto(120.87842684,187.40264422)
\lineto(120.99842684,187.58264422)
\curveto(121.04841611,187.64263365)(121.09841606,187.69763359)(121.14842684,187.74764422)
\curveto(121.21841594,187.79763349)(121.29341586,187.83263346)(121.37342684,187.85264422)
\curveto(121.39341576,187.86263343)(121.41841574,187.86263343)(121.44842684,187.85264422)
\curveto(121.48841567,187.85263344)(121.51841564,187.86263343)(121.53842684,187.88264422)
\curveto(121.6584155,187.88263341)(121.79341536,187.87763341)(121.94342684,187.86764422)
\curveto(122.09341506,187.86763342)(122.18341497,187.82263347)(122.21342684,187.73264422)
\curveto(122.23341492,187.70263359)(122.23841492,187.66763362)(122.22842684,187.62764422)
\curveto(122.21841494,187.5876337)(122.20341495,187.55763373)(122.18342684,187.53764422)
\curveto(122.14341501,187.45763383)(122.10341505,187.3876339)(122.06342684,187.32764422)
\curveto(122.02341513,187.26763402)(121.97841518,187.20763408)(121.92842684,187.14764422)
\lineto(121.35842684,186.36764422)
\curveto(121.17841598,186.11763517)(120.99841616,185.86263543)(120.81842684,185.60264422)
\moveto(113.96342684,181.70264422)
\curveto(113.91342324,181.72263957)(113.86342329,181.72763956)(113.81342684,181.71764422)
\curveto(113.76342339,181.70763958)(113.71342344,181.71263958)(113.66342684,181.73264422)
\curveto(113.5534236,181.75263954)(113.44842371,181.77263952)(113.34842684,181.79264422)
\curveto(113.2584239,181.82263947)(113.16342399,181.86263943)(113.06342684,181.91264422)
\curveto(112.73342442,182.05263924)(112.47842468,182.24763904)(112.29842684,182.49764422)
\curveto(112.11842504,182.75763853)(111.97342518,183.06763822)(111.86342684,183.42764422)
\curveto(111.83342532,183.50763778)(111.81342534,183.5876377)(111.80342684,183.66764422)
\curveto(111.79342536,183.75763753)(111.77842538,183.84263745)(111.75842684,183.92264422)
\curveto(111.74842541,183.97263732)(111.74342541,184.03763725)(111.74342684,184.11764422)
\curveto(111.73342542,184.14763714)(111.72842543,184.17763711)(111.72842684,184.20764422)
\curveto(111.72842543,184.24763704)(111.72342543,184.28263701)(111.71342684,184.31264422)
\lineto(111.71342684,184.46264422)
\curveto(111.70342545,184.51263678)(111.69842546,184.57263672)(111.69842684,184.64264422)
\curveto(111.69842546,184.72263657)(111.70342545,184.7876365)(111.71342684,184.83764422)
\lineto(111.71342684,185.00264422)
\curveto(111.73342542,185.05263624)(111.73842542,185.09763619)(111.72842684,185.13764422)
\curveto(111.72842543,185.1876361)(111.73342542,185.23263606)(111.74342684,185.27264422)
\curveto(111.7534254,185.31263598)(111.7584254,185.34763594)(111.75842684,185.37764422)
\curveto(111.7584254,185.41763587)(111.76342539,185.45763583)(111.77342684,185.49764422)
\curveto(111.80342535,185.60763568)(111.82342533,185.71763557)(111.83342684,185.82764422)
\curveto(111.8534253,185.94763534)(111.88842527,186.06263523)(111.93842684,186.17264422)
\curveto(112.07842508,186.51263478)(112.23842492,186.7876345)(112.41842684,186.99764422)
\curveto(112.60842455,187.21763407)(112.87842428,187.39763389)(113.22842684,187.53764422)
\curveto(113.30842385,187.56763372)(113.39342376,187.5876337)(113.48342684,187.59764422)
\curveto(113.57342358,187.61763367)(113.66842349,187.63763365)(113.76842684,187.65764422)
\curveto(113.79842336,187.66763362)(113.8534233,187.66763362)(113.93342684,187.65764422)
\curveto(114.01342314,187.65763363)(114.06342309,187.66763362)(114.08342684,187.68764422)
\curveto(114.64342251,187.69763359)(115.09342206,187.5876337)(115.43342684,187.35764422)
\curveto(115.78342137,187.12763416)(116.04342111,186.82263447)(116.21342684,186.44264422)
\curveto(116.2534209,186.35263494)(116.28842087,186.25763503)(116.31842684,186.15764422)
\curveto(116.34842081,186.05763523)(116.37342078,185.95763533)(116.39342684,185.85764422)
\curveto(116.41342074,185.82763546)(116.41842074,185.79763549)(116.40842684,185.76764422)
\curveto(116.40842075,185.73763555)(116.41342074,185.70763558)(116.42342684,185.67764422)
\curveto(116.4534207,185.56763572)(116.47342068,185.44263585)(116.48342684,185.30264422)
\curveto(116.49342066,185.17263612)(116.50342065,185.03763625)(116.51342684,184.89764422)
\lineto(116.51342684,184.73264422)
\curveto(116.52342063,184.67263662)(116.52342063,184.61763667)(116.51342684,184.56764422)
\curveto(116.50342065,184.51763677)(116.49842066,184.46763682)(116.49842684,184.41764422)
\lineto(116.49842684,184.28264422)
\curveto(116.48842067,184.24263705)(116.48342067,184.20263709)(116.48342684,184.16264422)
\curveto(116.49342066,184.12263717)(116.48842067,184.07763721)(116.46842684,184.02764422)
\curveto(116.44842071,183.91763737)(116.42842073,183.81263748)(116.40842684,183.71264422)
\curveto(116.39842076,183.61263768)(116.37842078,183.51263778)(116.34842684,183.41264422)
\curveto(116.21842094,183.05263824)(116.0534211,182.73763855)(115.85342684,182.46764422)
\curveto(115.6534215,182.19763909)(115.37842178,181.9926393)(115.02842684,181.85264422)
\curveto(114.94842221,181.82263947)(114.86342229,181.79763949)(114.77342684,181.77764422)
\lineto(114.50342684,181.71764422)
\curveto(114.4534227,181.70763958)(114.40842275,181.70263959)(114.36842684,181.70264422)
\curveto(114.32842283,181.71263958)(114.28842287,181.71263958)(114.24842684,181.70264422)
\curveto(114.14842301,181.68263961)(114.0534231,181.68263961)(113.96342684,181.70264422)
\moveto(113.12342684,183.09764422)
\curveto(113.16342399,183.02763826)(113.20342395,182.96263833)(113.24342684,182.90264422)
\curveto(113.28342387,182.85263844)(113.33342382,182.80263849)(113.39342684,182.75264422)
\lineto(113.54342684,182.63264422)
\curveto(113.60342355,182.60263869)(113.66842349,182.57763871)(113.73842684,182.55764422)
\curveto(113.77842338,182.53763875)(113.81342334,182.52763876)(113.84342684,182.52764422)
\curveto(113.88342327,182.53763875)(113.92342323,182.53263876)(113.96342684,182.51264422)
\curveto(113.99342316,182.51263878)(114.03342312,182.50763878)(114.08342684,182.49764422)
\curveto(114.13342302,182.49763879)(114.17342298,182.50263879)(114.20342684,182.51264422)
\lineto(114.42842684,182.55764422)
\curveto(114.67842248,182.63763865)(114.86342229,182.76263853)(114.98342684,182.93264422)
\curveto(115.06342209,183.03263826)(115.13342202,183.16263813)(115.19342684,183.32264422)
\curveto(115.27342188,183.50263779)(115.33342182,183.72763756)(115.37342684,183.99764422)
\curveto(115.41342174,184.27763701)(115.42842173,184.55763673)(115.41842684,184.83764422)
\curveto(115.40842175,185.12763616)(115.37842178,185.40263589)(115.32842684,185.66264422)
\curveto(115.27842188,185.92263537)(115.20342195,186.13263516)(115.10342684,186.29264422)
\curveto(114.98342217,186.4926348)(114.83342232,186.64263465)(114.65342684,186.74264422)
\curveto(114.57342258,186.7926345)(114.48342267,186.82263447)(114.38342684,186.83264422)
\curveto(114.28342287,186.85263444)(114.17842298,186.86263443)(114.06842684,186.86264422)
\curveto(114.04842311,186.85263444)(114.02342313,186.84763444)(113.99342684,186.84764422)
\curveto(113.97342318,186.85763443)(113.9534232,186.85763443)(113.93342684,186.84764422)
\curveto(113.88342327,186.83763445)(113.83842332,186.82763446)(113.79842684,186.81764422)
\curveto(113.7584234,186.81763447)(113.71842344,186.80763448)(113.67842684,186.78764422)
\curveto(113.49842366,186.70763458)(113.34842381,186.5876347)(113.22842684,186.42764422)
\curveto(113.11842404,186.26763502)(113.02842413,186.0876352)(112.95842684,185.88764422)
\curveto(112.89842426,185.69763559)(112.8534243,185.47263582)(112.82342684,185.21264422)
\curveto(112.80342435,184.95263634)(112.79842436,184.6876366)(112.80842684,184.41764422)
\curveto(112.81842434,184.15763713)(112.84842431,183.90763738)(112.89842684,183.66764422)
\curveto(112.9584242,183.43763785)(113.03342412,183.24763804)(113.12342684,183.09764422)
\moveto(123.92342684,180.11264422)
\curveto(123.93341322,180.06264123)(123.93841322,179.97264132)(123.93842684,179.84264422)
\curveto(123.93841322,179.71264158)(123.92841323,179.62264167)(123.90842684,179.57264422)
\curveto(123.88841327,179.52264177)(123.88341327,179.46764182)(123.89342684,179.40764422)
\curveto(123.90341325,179.35764193)(123.90341325,179.30764198)(123.89342684,179.25764422)
\curveto(123.8534133,179.11764217)(123.82341333,178.98264231)(123.80342684,178.85264422)
\curveto(123.79341336,178.72264257)(123.76341339,178.60264269)(123.71342684,178.49264422)
\curveto(123.57341358,178.14264315)(123.40841375,177.84764344)(123.21842684,177.60764422)
\curveto(123.02841413,177.37764391)(122.7584144,177.1926441)(122.40842684,177.05264422)
\curveto(122.32841483,177.02264427)(122.24341491,177.00264429)(122.15342684,176.99264422)
\curveto(122.06341509,176.97264432)(121.97841518,176.95264434)(121.89842684,176.93264422)
\curveto(121.84841531,176.92264437)(121.79841536,176.91764437)(121.74842684,176.91764422)
\curveto(121.69841546,176.91764437)(121.64841551,176.91264438)(121.59842684,176.90264422)
\curveto(121.56841559,176.8926444)(121.51841564,176.8926444)(121.44842684,176.90264422)
\curveto(121.37841578,176.90264439)(121.32841583,176.90764438)(121.29842684,176.91764422)
\curveto(121.23841592,176.93764435)(121.17841598,176.94764434)(121.11842684,176.94764422)
\curveto(121.06841609,176.93764435)(121.01841614,176.94264435)(120.96842684,176.96264422)
\curveto(120.87841628,176.98264431)(120.78841637,177.00764428)(120.69842684,177.03764422)
\curveto(120.61841654,177.05764423)(120.53841662,177.0876442)(120.45842684,177.12764422)
\curveto(120.13841702,177.26764402)(119.88841727,177.46264383)(119.70842684,177.71264422)
\curveto(119.52841763,177.97264332)(119.37841778,178.27764301)(119.25842684,178.62764422)
\curveto(119.23841792,178.70764258)(119.22341793,178.7926425)(119.21342684,178.88264422)
\curveto(119.20341795,178.97264232)(119.18841797,179.05764223)(119.16842684,179.13764422)
\curveto(119.158418,179.16764212)(119.153418,179.19764209)(119.15342684,179.22764422)
\lineto(119.15342684,179.33264422)
\curveto(119.13341802,179.41264188)(119.12341803,179.4926418)(119.12342684,179.57264422)
\lineto(119.12342684,179.70764422)
\curveto(119.10341805,179.80764148)(119.10341805,179.90764138)(119.12342684,180.00764422)
\lineto(119.12342684,180.18764422)
\curveto(119.13341802,180.23764105)(119.13841802,180.28264101)(119.13842684,180.32264422)
\curveto(119.13841802,180.37264092)(119.14341801,180.41764087)(119.15342684,180.45764422)
\curveto(119.16341799,180.49764079)(119.16841799,180.53264076)(119.16842684,180.56264422)
\curveto(119.16841799,180.60264069)(119.17341798,180.64264065)(119.18342684,180.68264422)
\lineto(119.24342684,181.01264422)
\curveto(119.26341789,181.13264016)(119.29341786,181.24264005)(119.33342684,181.34264422)
\curveto(119.47341768,181.67263962)(119.63341752,181.94763934)(119.81342684,182.16764422)
\curveto(120.00341715,182.39763889)(120.26341689,182.58263871)(120.59342684,182.72264422)
\curveto(120.67341648,182.76263853)(120.7584164,182.7876385)(120.84842684,182.79764422)
\lineto(121.14842684,182.85764422)
\lineto(121.28342684,182.85764422)
\curveto(121.33341582,182.86763842)(121.38341577,182.87263842)(121.43342684,182.87264422)
\curveto(122.00341515,182.8926384)(122.46341469,182.7876385)(122.81342684,182.55764422)
\curveto(123.17341398,182.33763895)(123.43841372,182.03763925)(123.60842684,181.65764422)
\curveto(123.6584135,181.55763973)(123.69841346,181.45763983)(123.72842684,181.35764422)
\curveto(123.7584134,181.25764003)(123.78841337,181.15264014)(123.81842684,181.04264422)
\curveto(123.82841333,181.00264029)(123.83341332,180.96764032)(123.83342684,180.93764422)
\curveto(123.83341332,180.91764037)(123.83841332,180.8876404)(123.84842684,180.84764422)
\curveto(123.86841329,180.77764051)(123.87841328,180.70264059)(123.87842684,180.62264422)
\curveto(123.87841328,180.54264075)(123.88841327,180.46264083)(123.90842684,180.38264422)
\curveto(123.90841325,180.33264096)(123.90841325,180.287641)(123.90842684,180.24764422)
\curveto(123.90841325,180.20764108)(123.91341324,180.16264113)(123.92342684,180.11264422)
\moveto(122.81342684,179.67764422)
\curveto(122.82341433,179.72764156)(122.82841433,179.80264149)(122.82842684,179.90264422)
\curveto(122.83841432,180.00264129)(122.83341432,180.07764121)(122.81342684,180.12764422)
\curveto(122.79341436,180.1876411)(122.78841437,180.24264105)(122.79842684,180.29264422)
\curveto(122.81841434,180.35264094)(122.81841434,180.41264088)(122.79842684,180.47264422)
\curveto(122.78841437,180.50264079)(122.78341437,180.53764075)(122.78342684,180.57764422)
\curveto(122.78341437,180.61764067)(122.77841438,180.65764063)(122.76842684,180.69764422)
\curveto(122.74841441,180.77764051)(122.72841443,180.85264044)(122.70842684,180.92264422)
\curveto(122.69841446,181.00264029)(122.68341447,181.08264021)(122.66342684,181.16264422)
\curveto(122.63341452,181.22264007)(122.60841455,181.28264001)(122.58842684,181.34264422)
\curveto(122.56841459,181.40263989)(122.53841462,181.46263983)(122.49842684,181.52264422)
\curveto(122.39841476,181.6926396)(122.26841489,181.82763946)(122.10842684,181.92764422)
\curveto(122.02841513,181.97763931)(121.93341522,182.01263928)(121.82342684,182.03264422)
\curveto(121.71341544,182.05263924)(121.58841557,182.06263923)(121.44842684,182.06264422)
\curveto(121.42841573,182.05263924)(121.40341575,182.04763924)(121.37342684,182.04764422)
\curveto(121.34341581,182.05763923)(121.31341584,182.05763923)(121.28342684,182.04764422)
\lineto(121.13342684,181.98764422)
\curveto(121.08341607,181.97763931)(121.03841612,181.96263933)(120.99842684,181.94264422)
\curveto(120.80841635,181.83263946)(120.66341649,181.6876396)(120.56342684,181.50764422)
\curveto(120.47341668,181.32763996)(120.39341676,181.12264017)(120.32342684,180.89264422)
\curveto(120.28341687,180.76264053)(120.26341689,180.62764066)(120.26342684,180.48764422)
\curveto(120.26341689,180.35764093)(120.2534169,180.21264108)(120.23342684,180.05264422)
\curveto(120.22341693,180.00264129)(120.21341694,179.94264135)(120.20342684,179.87264422)
\curveto(120.20341695,179.80264149)(120.21341694,179.74264155)(120.23342684,179.69264422)
\lineto(120.23342684,179.52764422)
\lineto(120.23342684,179.34764422)
\curveto(120.24341691,179.29764199)(120.2534169,179.24264205)(120.26342684,179.18264422)
\curveto(120.27341688,179.13264216)(120.27841688,179.07764221)(120.27842684,179.01764422)
\curveto(120.28841687,178.95764233)(120.30341685,178.90264239)(120.32342684,178.85264422)
\curveto(120.37341678,178.66264263)(120.43341672,178.4876428)(120.50342684,178.32764422)
\curveto(120.57341658,178.16764312)(120.67841648,178.03764325)(120.81842684,177.93764422)
\curveto(120.94841621,177.83764345)(121.08841607,177.76764352)(121.23842684,177.72764422)
\curveto(121.26841589,177.71764357)(121.29341586,177.71264358)(121.31342684,177.71264422)
\curveto(121.34341581,177.72264357)(121.37341578,177.72264357)(121.40342684,177.71264422)
\curveto(121.42341573,177.71264358)(121.4534157,177.70764358)(121.49342684,177.69764422)
\curveto(121.53341562,177.69764359)(121.56841559,177.70264359)(121.59842684,177.71264422)
\curveto(121.63841552,177.72264357)(121.67841548,177.72764356)(121.71842684,177.72764422)
\curveto(121.7584154,177.72764356)(121.79841536,177.73764355)(121.83842684,177.75764422)
\curveto(122.07841508,177.83764345)(122.27341488,177.97264332)(122.42342684,178.16264422)
\curveto(122.54341461,178.34264295)(122.63341452,178.54764274)(122.69342684,178.77764422)
\curveto(122.71341444,178.84764244)(122.72841443,178.91764237)(122.73842684,178.98764422)
\curveto(122.74841441,179.06764222)(122.76341439,179.14764214)(122.78342684,179.22764422)
\curveto(122.78341437,179.287642)(122.78841437,179.33264196)(122.79842684,179.36264422)
\curveto(122.79841436,179.38264191)(122.79841436,179.40764188)(122.79842684,179.43764422)
\curveto(122.79841436,179.47764181)(122.80341435,179.50764178)(122.81342684,179.52764422)
\lineto(122.81342684,179.67764422)
}
}
{
\newrgbcolor{curcolor}{0 0 0}
\pscustom[linestyle=none,fillstyle=solid,fillcolor=curcolor]
{
\newpath
\moveto(164.08679964,644.08163348)
\lineto(168.88679964,644.08163348)
\lineto(169.89179964,644.08163348)
\curveto(170.03179254,644.08162305)(170.15179242,644.07162306)(170.25179964,644.05163348)
\curveto(170.36179221,644.04162309)(170.44179213,643.99662314)(170.49179964,643.91663348)
\curveto(170.51179206,643.87662326)(170.52179205,643.82662331)(170.52179964,643.76663348)
\curveto(170.53179204,643.70662343)(170.53679203,643.64162349)(170.53679964,643.57163348)
\lineto(170.53679964,643.30163348)
\curveto(170.53679203,643.21162392)(170.52679204,643.131624)(170.50679964,643.06163348)
\curveto(170.4667921,642.98162415)(170.42179215,642.91162422)(170.37179964,642.85163348)
\lineto(170.22179964,642.67163348)
\curveto(170.19179238,642.62162451)(170.15679241,642.58162455)(170.11679964,642.55163348)
\curveto(170.07679249,642.52162461)(170.03679253,642.48162465)(169.99679964,642.43163348)
\curveto(169.91679265,642.32162481)(169.83179274,642.21162492)(169.74179964,642.10163348)
\curveto(169.65179292,642.00162513)(169.566793,641.89662524)(169.48679964,641.78663348)
\curveto(169.34679322,641.58662555)(169.20679336,641.37662576)(169.06679964,641.15663348)
\curveto(168.92679364,640.94662619)(168.78679378,640.7316264)(168.64679964,640.51163348)
\curveto(168.59679397,640.42162671)(168.54679402,640.32662681)(168.49679964,640.22663348)
\curveto(168.44679412,640.12662701)(168.39179418,640.0316271)(168.33179964,639.94163348)
\curveto(168.31179426,639.92162721)(168.30179427,639.89662724)(168.30179964,639.86663348)
\curveto(168.30179427,639.8366273)(168.29179428,639.81162732)(168.27179964,639.79163348)
\curveto(168.20179437,639.69162744)(168.13679443,639.57662756)(168.07679964,639.44663348)
\curveto(168.01679455,639.32662781)(167.96179461,639.21162792)(167.91179964,639.10163348)
\curveto(167.81179476,638.87162826)(167.71679485,638.6366285)(167.62679964,638.39663348)
\curveto(167.53679503,638.15662898)(167.43679513,637.91662922)(167.32679964,637.67663348)
\curveto(167.30679526,637.62662951)(167.29179528,637.58162955)(167.28179964,637.54163348)
\curveto(167.28179529,637.50162963)(167.2717953,637.45662968)(167.25179964,637.40663348)
\curveto(167.20179537,637.28662985)(167.15679541,637.16162997)(167.11679964,637.03163348)
\curveto(167.08679548,636.91163022)(167.05179552,636.79163034)(167.01179964,636.67163348)
\curveto(166.93179564,636.44163069)(166.8667957,636.20163093)(166.81679964,635.95163348)
\curveto(166.77679579,635.71163142)(166.72679584,635.47163166)(166.66679964,635.23163348)
\curveto(166.62679594,635.08163205)(166.60179597,634.9316322)(166.59179964,634.78163348)
\curveto(166.58179599,634.6316325)(166.56179601,634.48163265)(166.53179964,634.33163348)
\curveto(166.52179605,634.29163284)(166.51679605,634.2316329)(166.51679964,634.15163348)
\curveto(166.48679608,634.0316331)(166.45679611,633.9316332)(166.42679964,633.85163348)
\curveto(166.39679617,633.77163336)(166.32679624,633.71663342)(166.21679964,633.68663348)
\curveto(166.1667964,633.66663347)(166.11179646,633.65663348)(166.05179964,633.65663348)
\lineto(165.85679964,633.65663348)
\curveto(165.71679685,633.65663348)(165.57679699,633.66163347)(165.43679964,633.67163348)
\curveto(165.30679726,633.68163345)(165.21179736,633.72663341)(165.15179964,633.80663348)
\curveto(165.11179746,633.86663327)(165.09179748,633.95163318)(165.09179964,634.06163348)
\curveto(165.10179747,634.17163296)(165.11679745,634.26663287)(165.13679964,634.34663348)
\lineto(165.13679964,634.42163348)
\curveto(165.14679742,634.45163268)(165.15179742,634.48163265)(165.15179964,634.51163348)
\curveto(165.1717974,634.59163254)(165.18179739,634.66663247)(165.18179964,634.73663348)
\curveto(165.18179739,634.80663233)(165.19179738,634.87663226)(165.21179964,634.94663348)
\curveto(165.26179731,635.136632)(165.30179727,635.32163181)(165.33179964,635.50163348)
\curveto(165.36179721,635.69163144)(165.40179717,635.87163126)(165.45179964,636.04163348)
\curveto(165.4717971,636.09163104)(165.48179709,636.131631)(165.48179964,636.16163348)
\curveto(165.48179709,636.19163094)(165.48679708,636.22663091)(165.49679964,636.26663348)
\curveto(165.59679697,636.56663057)(165.68679688,636.86163027)(165.76679964,637.15163348)
\curveto(165.85679671,637.44162969)(165.96179661,637.72162941)(166.08179964,637.99163348)
\curveto(166.34179623,638.57162856)(166.61179596,639.12162801)(166.89179964,639.64163348)
\curveto(167.1717954,640.17162696)(167.48179509,640.67662646)(167.82179964,641.15663348)
\curveto(167.96179461,641.35662578)(168.11179446,641.54662559)(168.27179964,641.72663348)
\curveto(168.43179414,641.91662522)(168.58179399,642.10662503)(168.72179964,642.29663348)
\curveto(168.76179381,642.34662479)(168.79679377,642.39162474)(168.82679964,642.43163348)
\curveto(168.8667937,642.48162465)(168.90179367,642.5316246)(168.93179964,642.58163348)
\curveto(168.94179363,642.60162453)(168.95179362,642.62662451)(168.96179964,642.65663348)
\curveto(168.98179359,642.68662445)(168.98179359,642.71662442)(168.96179964,642.74663348)
\curveto(168.94179363,642.80662433)(168.90679366,642.84162429)(168.85679964,642.85163348)
\curveto(168.80679376,642.87162426)(168.75679381,642.89162424)(168.70679964,642.91163348)
\lineto(168.60179964,642.91163348)
\curveto(168.56179401,642.92162421)(168.51179406,642.92162421)(168.45179964,642.91163348)
\lineto(168.30179964,642.91163348)
\lineto(167.70179964,642.91163348)
\lineto(165.06179964,642.91163348)
\lineto(164.32679964,642.91163348)
\lineto(164.08679964,642.91163348)
\curveto(164.01679855,642.92162421)(163.95679861,642.9366242)(163.90679964,642.95663348)
\curveto(163.81679875,642.99662414)(163.75679881,643.05662408)(163.72679964,643.13663348)
\curveto(163.67679889,643.2366239)(163.66179891,643.38162375)(163.68179964,643.57163348)
\curveto(163.70179887,643.77162336)(163.73679883,643.90662323)(163.78679964,643.97663348)
\curveto(163.80679876,643.99662314)(163.83179874,644.01162312)(163.86179964,644.02163348)
\lineto(163.98179964,644.08163348)
\curveto(164.00179857,644.08162305)(164.01679855,644.07662306)(164.02679964,644.06663348)
\curveto(164.04679852,644.06662307)(164.0667985,644.07162306)(164.08679964,644.08163348)
}
}
{
\newrgbcolor{curcolor}{0 0 0}
\pscustom[linestyle=none,fillstyle=solid,fillcolor=curcolor]
{
\newpath
\moveto(172.93140901,635.30663348)
\lineto(173.23140901,635.30663348)
\curveto(173.34140695,635.31663182)(173.44640685,635.31663182)(173.54640901,635.30663348)
\curveto(173.65640664,635.30663183)(173.75640654,635.29663184)(173.84640901,635.27663348)
\curveto(173.93640636,635.26663187)(174.00640629,635.24163189)(174.05640901,635.20163348)
\curveto(174.07640622,635.18163195)(174.0914062,635.15163198)(174.10140901,635.11163348)
\curveto(174.12140617,635.07163206)(174.14140615,635.02663211)(174.16140901,634.97663348)
\lineto(174.16140901,634.90163348)
\curveto(174.17140612,634.85163228)(174.17140612,634.79663234)(174.16140901,634.73663348)
\lineto(174.16140901,634.58663348)
\lineto(174.16140901,634.10663348)
\curveto(174.16140613,633.9366332)(174.12140617,633.81663332)(174.04140901,633.74663348)
\curveto(173.97140632,633.69663344)(173.88140641,633.67163346)(173.77140901,633.67163348)
\lineto(173.44140901,633.67163348)
\lineto(172.99140901,633.67163348)
\curveto(172.84140745,633.67163346)(172.72640757,633.70163343)(172.64640901,633.76163348)
\curveto(172.60640769,633.79163334)(172.57640772,633.84163329)(172.55640901,633.91163348)
\curveto(172.53640776,633.99163314)(172.52140777,634.07663306)(172.51140901,634.16663348)
\lineto(172.51140901,634.45163348)
\curveto(172.52140777,634.55163258)(172.52640777,634.6366325)(172.52640901,634.70663348)
\lineto(172.52640901,634.90163348)
\curveto(172.52640777,634.96163217)(172.53640776,635.01663212)(172.55640901,635.06663348)
\curveto(172.5964077,635.17663196)(172.66640763,635.24663189)(172.76640901,635.27663348)
\curveto(172.7964075,635.27663186)(172.85140744,635.28663185)(172.93140901,635.30663348)
}
}
{
\newrgbcolor{curcolor}{0 0 0}
\pscustom[linestyle=none,fillstyle=solid,fillcolor=curcolor]
{
\newpath
\moveto(183.01656526,637.16663348)
\curveto(183.08655762,637.11663002)(183.12655758,637.04663009)(183.13656526,636.95663348)
\curveto(183.15655755,636.86663027)(183.16655754,636.76163037)(183.16656526,636.64163348)
\curveto(183.16655754,636.59163054)(183.16155754,636.54163059)(183.15156526,636.49163348)
\curveto(183.15155755,636.44163069)(183.14155756,636.39663074)(183.12156526,636.35663348)
\curveto(183.09155761,636.26663087)(183.03155767,636.20663093)(182.94156526,636.17663348)
\curveto(182.86155784,636.15663098)(182.76655794,636.14663099)(182.65656526,636.14663348)
\lineto(182.34156526,636.14663348)
\curveto(182.23155847,636.15663098)(182.12655858,636.14663099)(182.02656526,636.11663348)
\curveto(181.88655882,636.08663105)(181.79655891,636.00663113)(181.75656526,635.87663348)
\curveto(181.73655897,635.80663133)(181.72655898,635.72163141)(181.72656526,635.62163348)
\lineto(181.72656526,635.35163348)
\lineto(181.72656526,634.40663348)
\lineto(181.72656526,634.07663348)
\curveto(181.72655898,633.96663317)(181.706559,633.88163325)(181.66656526,633.82163348)
\curveto(181.62655908,633.76163337)(181.57655913,633.72163341)(181.51656526,633.70163348)
\curveto(181.46655924,633.69163344)(181.4015593,633.67663346)(181.32156526,633.65663348)
\lineto(181.12656526,633.65663348)
\curveto(181.0065597,633.65663348)(180.9015598,633.66163347)(180.81156526,633.67163348)
\curveto(180.72155998,633.69163344)(180.65156005,633.74163339)(180.60156526,633.82163348)
\curveto(180.57156013,633.87163326)(180.55656015,633.94163319)(180.55656526,634.03163348)
\lineto(180.55656526,634.33163348)
\lineto(180.55656526,635.36663348)
\curveto(180.55656015,635.52663161)(180.54656016,635.67163146)(180.52656526,635.80163348)
\curveto(180.51656019,635.94163119)(180.46156024,636.0366311)(180.36156526,636.08663348)
\curveto(180.31156039,636.10663103)(180.24156046,636.12163101)(180.15156526,636.13163348)
\curveto(180.07156063,636.14163099)(179.98156072,636.14663099)(179.88156526,636.14663348)
\lineto(179.59656526,636.14663348)
\lineto(179.35656526,636.14663348)
\lineto(177.09156526,636.14663348)
\curveto(177.0015637,636.14663099)(176.89656381,636.14163099)(176.77656526,636.13163348)
\lineto(176.44656526,636.13163348)
\curveto(176.33656437,636.131631)(176.23656447,636.14163099)(176.14656526,636.16163348)
\curveto(176.05656465,636.18163095)(175.99656471,636.21663092)(175.96656526,636.26663348)
\curveto(175.91656479,636.3366308)(175.89156481,636.4316307)(175.89156526,636.55163348)
\lineto(175.89156526,636.89663348)
\lineto(175.89156526,637.16663348)
\curveto(175.93156477,637.3366298)(175.98656472,637.47662966)(176.05656526,637.58663348)
\curveto(176.12656458,637.69662944)(176.2065645,637.81162932)(176.29656526,637.93163348)
\lineto(176.65656526,638.47163348)
\curveto(177.09656361,639.10162803)(177.53156317,639.72162741)(177.96156526,640.33163348)
\lineto(179.28156526,642.19163348)
\curveto(179.44156126,642.42162471)(179.59656111,642.64162449)(179.74656526,642.85163348)
\curveto(179.89656081,643.07162406)(180.05156065,643.29662384)(180.21156526,643.52663348)
\curveto(180.26156044,643.59662354)(180.31156039,643.66162347)(180.36156526,643.72163348)
\curveto(180.41156029,643.79162334)(180.46156024,643.86662327)(180.51156526,643.94663348)
\lineto(180.57156526,644.03663348)
\curveto(180.6015601,644.07662306)(180.63156007,644.10662303)(180.66156526,644.12663348)
\curveto(180.70156,644.15662298)(180.74155996,644.17662296)(180.78156526,644.18663348)
\curveto(180.82155988,644.20662293)(180.86655984,644.22662291)(180.91656526,644.24663348)
\curveto(180.93655977,644.24662289)(180.95655975,644.24162289)(180.97656526,644.23163348)
\curveto(181.0065597,644.2316229)(181.03155967,644.24162289)(181.05156526,644.26163348)
\curveto(181.18155952,644.26162287)(181.3015594,644.25662288)(181.41156526,644.24663348)
\curveto(181.52155918,644.2366229)(181.6015591,644.19162294)(181.65156526,644.11163348)
\curveto(181.69155901,644.06162307)(181.71155899,643.99162314)(181.71156526,643.90163348)
\curveto(181.72155898,643.81162332)(181.72655898,643.71662342)(181.72656526,643.61663348)
\lineto(181.72656526,638.15663348)
\curveto(181.72655898,638.08662905)(181.72155898,638.01162912)(181.71156526,637.93163348)
\curveto(181.71155899,637.86162927)(181.71655899,637.79162934)(181.72656526,637.72163348)
\lineto(181.72656526,637.61663348)
\curveto(181.74655896,637.56662957)(181.76155894,637.51162962)(181.77156526,637.45163348)
\curveto(181.78155892,637.40162973)(181.8065589,637.36162977)(181.84656526,637.33163348)
\curveto(181.91655879,637.28162985)(182.0015587,637.25162988)(182.10156526,637.24163348)
\lineto(182.43156526,637.24163348)
\curveto(182.54155816,637.24162989)(182.64655806,637.2366299)(182.74656526,637.22663348)
\curveto(182.85655785,637.22662991)(182.94655776,637.20662993)(183.01656526,637.16663348)
\moveto(180.45156526,637.36163348)
\curveto(180.53156017,637.47162966)(180.56656014,637.64162949)(180.55656526,637.87163348)
\lineto(180.55656526,638.48663348)
\lineto(180.55656526,640.96163348)
\lineto(180.55656526,641.27663348)
\curveto(180.56656014,641.39662574)(180.56156014,641.49662564)(180.54156526,641.57663348)
\lineto(180.54156526,641.72663348)
\curveto(180.54156016,641.81662532)(180.52656018,641.90162523)(180.49656526,641.98163348)
\curveto(180.48656022,642.00162513)(180.47656023,642.01162512)(180.46656526,642.01163348)
\lineto(180.42156526,642.05663348)
\curveto(180.4015603,642.06662507)(180.37156033,642.07162506)(180.33156526,642.07163348)
\curveto(180.31156039,642.05162508)(180.29156041,642.0366251)(180.27156526,642.02663348)
\curveto(180.26156044,642.02662511)(180.24656046,642.02162511)(180.22656526,642.01163348)
\curveto(180.16656054,641.96162517)(180.1065606,641.89162524)(180.04656526,641.80163348)
\curveto(179.98656072,641.71162542)(179.93156077,641.6316255)(179.88156526,641.56163348)
\curveto(179.78156092,641.42162571)(179.68656102,641.27662586)(179.59656526,641.12663348)
\curveto(179.5065612,640.98662615)(179.41156129,640.84662629)(179.31156526,640.70663348)
\lineto(178.77156526,639.92663348)
\curveto(178.6015621,639.66662747)(178.42656228,639.40662773)(178.24656526,639.14663348)
\curveto(178.16656254,639.0366281)(178.09156261,638.9316282)(178.02156526,638.83163348)
\lineto(177.81156526,638.53163348)
\curveto(177.76156294,638.45162868)(177.71156299,638.37662876)(177.66156526,638.30663348)
\curveto(177.62156308,638.2366289)(177.57656313,638.16162897)(177.52656526,638.08163348)
\curveto(177.47656323,638.02162911)(177.42656328,637.95662918)(177.37656526,637.88663348)
\curveto(177.33656337,637.82662931)(177.29656341,637.75662938)(177.25656526,637.67663348)
\curveto(177.21656349,637.61662952)(177.19156351,637.54662959)(177.18156526,637.46663348)
\curveto(177.17156353,637.39662974)(177.2065635,637.34162979)(177.28656526,637.30163348)
\curveto(177.35656335,637.25162988)(177.46656324,637.22662991)(177.61656526,637.22663348)
\curveto(177.77656293,637.2366299)(177.91156279,637.24162989)(178.02156526,637.24163348)
\lineto(179.70156526,637.24163348)
\lineto(180.13656526,637.24163348)
\curveto(180.28656042,637.24162989)(180.39156031,637.28162985)(180.45156526,637.36163348)
}
}
{
\newrgbcolor{curcolor}{0 0 0}
\pscustom[linestyle=none,fillstyle=solid,fillcolor=curcolor]
{
\newpath
\moveto(194.29117464,642.19163348)
\curveto(194.09116434,641.90162523)(193.88116455,641.61662552)(193.66117464,641.33663348)
\curveto(193.45116498,641.05662608)(193.24616518,640.77162636)(193.04617464,640.48163348)
\curveto(192.44616598,639.6316275)(191.84116659,638.79162834)(191.23117464,637.96163348)
\curveto(190.62116781,637.14162999)(190.01616841,636.30663083)(189.41617464,635.45663348)
\lineto(188.90617464,634.73663348)
\lineto(188.39617464,634.04663348)
\curveto(188.31617011,633.9366332)(188.23617019,633.82163331)(188.15617464,633.70163348)
\curveto(188.07617035,633.58163355)(187.98117045,633.48663365)(187.87117464,633.41663348)
\curveto(187.8311706,633.39663374)(187.76617066,633.38163375)(187.67617464,633.37163348)
\curveto(187.59617083,633.35163378)(187.50617092,633.34163379)(187.40617464,633.34163348)
\curveto(187.30617112,633.34163379)(187.21117122,633.34663379)(187.12117464,633.35663348)
\curveto(187.04117139,633.36663377)(186.98117145,633.38663375)(186.94117464,633.41663348)
\curveto(186.91117152,633.4366337)(186.88617154,633.47163366)(186.86617464,633.52163348)
\curveto(186.85617157,633.56163357)(186.86117157,633.60663353)(186.88117464,633.65663348)
\curveto(186.92117151,633.7366334)(186.96617146,633.81163332)(187.01617464,633.88163348)
\curveto(187.07617135,633.96163317)(187.1311713,634.04163309)(187.18117464,634.12163348)
\curveto(187.42117101,634.46163267)(187.66617076,634.79663234)(187.91617464,635.12663348)
\curveto(188.16617026,635.45663168)(188.40617002,635.79163134)(188.63617464,636.13163348)
\curveto(188.79616963,636.35163078)(188.95616947,636.56663057)(189.11617464,636.77663348)
\curveto(189.27616915,636.98663015)(189.43616899,637.20162993)(189.59617464,637.42163348)
\curveto(189.95616847,637.94162919)(190.32116811,638.45162868)(190.69117464,638.95163348)
\curveto(191.06116737,639.45162768)(191.431167,639.96162717)(191.80117464,640.48163348)
\curveto(191.94116649,640.68162645)(192.08116635,640.87662626)(192.22117464,641.06663348)
\curveto(192.37116606,641.25662588)(192.51616591,641.45162568)(192.65617464,641.65163348)
\curveto(192.86616556,641.95162518)(193.08116535,642.25162488)(193.30117464,642.55163348)
\lineto(193.96117464,643.45163348)
\lineto(194.14117464,643.72163348)
\lineto(194.35117464,643.99163348)
\lineto(194.47117464,644.17163348)
\curveto(194.52116391,644.2316229)(194.57116386,644.28662285)(194.62117464,644.33663348)
\curveto(194.69116374,644.38662275)(194.76616366,644.42162271)(194.84617464,644.44163348)
\curveto(194.86616356,644.45162268)(194.89116354,644.45162268)(194.92117464,644.44163348)
\curveto(194.96116347,644.44162269)(194.99116344,644.45162268)(195.01117464,644.47163348)
\curveto(195.1311633,644.47162266)(195.26616316,644.46662267)(195.41617464,644.45663348)
\curveto(195.56616286,644.45662268)(195.65616277,644.41162272)(195.68617464,644.32163348)
\curveto(195.70616272,644.29162284)(195.71116272,644.25662288)(195.70117464,644.21663348)
\curveto(195.69116274,644.17662296)(195.67616275,644.14662299)(195.65617464,644.12663348)
\curveto(195.61616281,644.04662309)(195.57616285,643.97662316)(195.53617464,643.91663348)
\curveto(195.49616293,643.85662328)(195.45116298,643.79662334)(195.40117464,643.73663348)
\lineto(194.83117464,642.95663348)
\curveto(194.65116378,642.70662443)(194.47116396,642.45162468)(194.29117464,642.19163348)
\moveto(187.43617464,638.29163348)
\curveto(187.38617104,638.31162882)(187.33617109,638.31662882)(187.28617464,638.30663348)
\curveto(187.23617119,638.29662884)(187.18617124,638.30162883)(187.13617464,638.32163348)
\curveto(187.0261714,638.34162879)(186.92117151,638.36162877)(186.82117464,638.38163348)
\curveto(186.7311717,638.41162872)(186.63617179,638.45162868)(186.53617464,638.50163348)
\curveto(186.20617222,638.64162849)(185.95117248,638.8366283)(185.77117464,639.08663348)
\curveto(185.59117284,639.34662779)(185.44617298,639.65662748)(185.33617464,640.01663348)
\curveto(185.30617312,640.09662704)(185.28617314,640.17662696)(185.27617464,640.25663348)
\curveto(185.26617316,640.34662679)(185.25117318,640.4316267)(185.23117464,640.51163348)
\curveto(185.22117321,640.56162657)(185.21617321,640.62662651)(185.21617464,640.70663348)
\curveto(185.20617322,640.7366264)(185.20117323,640.76662637)(185.20117464,640.79663348)
\curveto(185.20117323,640.8366263)(185.19617323,640.87162626)(185.18617464,640.90163348)
\lineto(185.18617464,641.05163348)
\curveto(185.17617325,641.10162603)(185.17117326,641.16162597)(185.17117464,641.23163348)
\curveto(185.17117326,641.31162582)(185.17617325,641.37662576)(185.18617464,641.42663348)
\lineto(185.18617464,641.59163348)
\curveto(185.20617322,641.64162549)(185.21117322,641.68662545)(185.20117464,641.72663348)
\curveto(185.20117323,641.77662536)(185.20617322,641.82162531)(185.21617464,641.86163348)
\curveto(185.2261732,641.90162523)(185.2311732,641.9366252)(185.23117464,641.96663348)
\curveto(185.2311732,642.00662513)(185.23617319,642.04662509)(185.24617464,642.08663348)
\curveto(185.27617315,642.19662494)(185.29617313,642.30662483)(185.30617464,642.41663348)
\curveto(185.3261731,642.5366246)(185.36117307,642.65162448)(185.41117464,642.76163348)
\curveto(185.55117288,643.10162403)(185.71117272,643.37662376)(185.89117464,643.58663348)
\curveto(186.08117235,643.80662333)(186.35117208,643.98662315)(186.70117464,644.12663348)
\curveto(186.78117165,644.15662298)(186.86617156,644.17662296)(186.95617464,644.18663348)
\curveto(187.04617138,644.20662293)(187.14117129,644.22662291)(187.24117464,644.24663348)
\curveto(187.27117116,644.25662288)(187.3261711,644.25662288)(187.40617464,644.24663348)
\curveto(187.48617094,644.24662289)(187.53617089,644.25662288)(187.55617464,644.27663348)
\curveto(188.11617031,644.28662285)(188.56616986,644.17662296)(188.90617464,643.94663348)
\curveto(189.25616917,643.71662342)(189.51616891,643.41162372)(189.68617464,643.03163348)
\curveto(189.7261687,642.94162419)(189.76116867,642.84662429)(189.79117464,642.74663348)
\curveto(189.82116861,642.64662449)(189.84616858,642.54662459)(189.86617464,642.44663348)
\curveto(189.88616854,642.41662472)(189.89116854,642.38662475)(189.88117464,642.35663348)
\curveto(189.88116855,642.32662481)(189.88616854,642.29662484)(189.89617464,642.26663348)
\curveto(189.9261685,642.15662498)(189.94616848,642.0316251)(189.95617464,641.89163348)
\curveto(189.96616846,641.76162537)(189.97616845,641.62662551)(189.98617464,641.48663348)
\lineto(189.98617464,641.32163348)
\curveto(189.99616843,641.26162587)(189.99616843,641.20662593)(189.98617464,641.15663348)
\curveto(189.97616845,641.10662603)(189.97116846,641.05662608)(189.97117464,641.00663348)
\lineto(189.97117464,640.87163348)
\curveto(189.96116847,640.8316263)(189.95616847,640.79162634)(189.95617464,640.75163348)
\curveto(189.96616846,640.71162642)(189.96116847,640.66662647)(189.94117464,640.61663348)
\curveto(189.92116851,640.50662663)(189.90116853,640.40162673)(189.88117464,640.30163348)
\curveto(189.87116856,640.20162693)(189.85116858,640.10162703)(189.82117464,640.00163348)
\curveto(189.69116874,639.64162749)(189.5261689,639.32662781)(189.32617464,639.05663348)
\curveto(189.1261693,638.78662835)(188.85116958,638.58162855)(188.50117464,638.44163348)
\curveto(188.42117001,638.41162872)(188.33617009,638.38662875)(188.24617464,638.36663348)
\lineto(187.97617464,638.30663348)
\curveto(187.9261705,638.29662884)(187.88117055,638.29162884)(187.84117464,638.29163348)
\curveto(187.80117063,638.30162883)(187.76117067,638.30162883)(187.72117464,638.29163348)
\curveto(187.62117081,638.27162886)(187.5261709,638.27162886)(187.43617464,638.29163348)
\moveto(186.59617464,639.68663348)
\curveto(186.63617179,639.61662752)(186.67617175,639.55162758)(186.71617464,639.49163348)
\curveto(186.75617167,639.44162769)(186.80617162,639.39162774)(186.86617464,639.34163348)
\lineto(187.01617464,639.22163348)
\curveto(187.07617135,639.19162794)(187.14117129,639.16662797)(187.21117464,639.14663348)
\curveto(187.25117118,639.12662801)(187.28617114,639.11662802)(187.31617464,639.11663348)
\curveto(187.35617107,639.12662801)(187.39617103,639.12162801)(187.43617464,639.10163348)
\curveto(187.46617096,639.10162803)(187.50617092,639.09662804)(187.55617464,639.08663348)
\curveto(187.60617082,639.08662805)(187.64617078,639.09162804)(187.67617464,639.10163348)
\lineto(187.90117464,639.14663348)
\curveto(188.15117028,639.22662791)(188.33617009,639.35162778)(188.45617464,639.52163348)
\curveto(188.53616989,639.62162751)(188.60616982,639.75162738)(188.66617464,639.91163348)
\curveto(188.74616968,640.09162704)(188.80616962,640.31662682)(188.84617464,640.58663348)
\curveto(188.88616954,640.86662627)(188.90116953,641.14662599)(188.89117464,641.42663348)
\curveto(188.88116955,641.71662542)(188.85116958,641.99162514)(188.80117464,642.25163348)
\curveto(188.75116968,642.51162462)(188.67616975,642.72162441)(188.57617464,642.88163348)
\curveto(188.45616997,643.08162405)(188.30617012,643.2316239)(188.12617464,643.33163348)
\curveto(188.04617038,643.38162375)(187.95617047,643.41162372)(187.85617464,643.42163348)
\curveto(187.75617067,643.44162369)(187.65117078,643.45162368)(187.54117464,643.45163348)
\curveto(187.52117091,643.44162369)(187.49617093,643.4366237)(187.46617464,643.43663348)
\curveto(187.44617098,643.44662369)(187.426171,643.44662369)(187.40617464,643.43663348)
\curveto(187.35617107,643.42662371)(187.31117112,643.41662372)(187.27117464,643.40663348)
\curveto(187.2311712,643.40662373)(187.19117124,643.39662374)(187.15117464,643.37663348)
\curveto(186.97117146,643.29662384)(186.82117161,643.17662396)(186.70117464,643.01663348)
\curveto(186.59117184,642.85662428)(186.50117193,642.67662446)(186.43117464,642.47663348)
\curveto(186.37117206,642.28662485)(186.3261721,642.06162507)(186.29617464,641.80163348)
\curveto(186.27617215,641.54162559)(186.27117216,641.27662586)(186.28117464,641.00663348)
\curveto(186.29117214,640.74662639)(186.32117211,640.49662664)(186.37117464,640.25663348)
\curveto(186.431172,640.02662711)(186.50617192,639.8366273)(186.59617464,639.68663348)
\moveto(197.39617464,636.70163348)
\curveto(197.40616102,636.65163048)(197.41116102,636.56163057)(197.41117464,636.43163348)
\curveto(197.41116102,636.30163083)(197.40116103,636.21163092)(197.38117464,636.16163348)
\curveto(197.36116107,636.11163102)(197.35616107,636.05663108)(197.36617464,635.99663348)
\curveto(197.37616105,635.94663119)(197.37616105,635.89663124)(197.36617464,635.84663348)
\curveto(197.3261611,635.70663143)(197.29616113,635.57163156)(197.27617464,635.44163348)
\curveto(197.26616116,635.31163182)(197.23616119,635.19163194)(197.18617464,635.08163348)
\curveto(197.04616138,634.7316324)(196.88116155,634.4366327)(196.69117464,634.19663348)
\curveto(196.50116193,633.96663317)(196.2311622,633.78163335)(195.88117464,633.64163348)
\curveto(195.80116263,633.61163352)(195.71616271,633.59163354)(195.62617464,633.58163348)
\curveto(195.53616289,633.56163357)(195.45116298,633.54163359)(195.37117464,633.52163348)
\curveto(195.32116311,633.51163362)(195.27116316,633.50663363)(195.22117464,633.50663348)
\curveto(195.17116326,633.50663363)(195.12116331,633.50163363)(195.07117464,633.49163348)
\curveto(195.04116339,633.48163365)(194.99116344,633.48163365)(194.92117464,633.49163348)
\curveto(194.85116358,633.49163364)(194.80116363,633.49663364)(194.77117464,633.50663348)
\curveto(194.71116372,633.52663361)(194.65116378,633.5366336)(194.59117464,633.53663348)
\curveto(194.54116389,633.52663361)(194.49116394,633.5316336)(194.44117464,633.55163348)
\curveto(194.35116408,633.57163356)(194.26116417,633.59663354)(194.17117464,633.62663348)
\curveto(194.09116434,633.64663349)(194.01116442,633.67663346)(193.93117464,633.71663348)
\curveto(193.61116482,633.85663328)(193.36116507,634.05163308)(193.18117464,634.30163348)
\curveto(193.00116543,634.56163257)(192.85116558,634.86663227)(192.73117464,635.21663348)
\curveto(192.71116572,635.29663184)(192.69616573,635.38163175)(192.68617464,635.47163348)
\curveto(192.67616575,635.56163157)(192.66116577,635.64663149)(192.64117464,635.72663348)
\curveto(192.6311658,635.75663138)(192.6261658,635.78663135)(192.62617464,635.81663348)
\lineto(192.62617464,635.92163348)
\curveto(192.60616582,636.00163113)(192.59616583,636.08163105)(192.59617464,636.16163348)
\lineto(192.59617464,636.29663348)
\curveto(192.57616585,636.39663074)(192.57616585,636.49663064)(192.59617464,636.59663348)
\lineto(192.59617464,636.77663348)
\curveto(192.60616582,636.82663031)(192.61116582,636.87163026)(192.61117464,636.91163348)
\curveto(192.61116582,636.96163017)(192.61616581,637.00663013)(192.62617464,637.04663348)
\curveto(192.63616579,637.08663005)(192.64116579,637.12163001)(192.64117464,637.15163348)
\curveto(192.64116579,637.19162994)(192.64616578,637.2316299)(192.65617464,637.27163348)
\lineto(192.71617464,637.60163348)
\curveto(192.73616569,637.72162941)(192.76616566,637.8316293)(192.80617464,637.93163348)
\curveto(192.94616548,638.26162887)(193.10616532,638.5366286)(193.28617464,638.75663348)
\curveto(193.47616495,638.98662815)(193.73616469,639.17162796)(194.06617464,639.31163348)
\curveto(194.14616428,639.35162778)(194.2311642,639.37662776)(194.32117464,639.38663348)
\lineto(194.62117464,639.44663348)
\lineto(194.75617464,639.44663348)
\curveto(194.80616362,639.45662768)(194.85616357,639.46162767)(194.90617464,639.46163348)
\curveto(195.47616295,639.48162765)(195.93616249,639.37662776)(196.28617464,639.14663348)
\curveto(196.64616178,638.92662821)(196.91116152,638.62662851)(197.08117464,638.24663348)
\curveto(197.1311613,638.14662899)(197.17116126,638.04662909)(197.20117464,637.94663348)
\curveto(197.2311612,637.84662929)(197.26116117,637.74162939)(197.29117464,637.63163348)
\curveto(197.30116113,637.59162954)(197.30616112,637.55662958)(197.30617464,637.52663348)
\curveto(197.30616112,637.50662963)(197.31116112,637.47662966)(197.32117464,637.43663348)
\curveto(197.34116109,637.36662977)(197.35116108,637.29162984)(197.35117464,637.21163348)
\curveto(197.35116108,637.13163)(197.36116107,637.05163008)(197.38117464,636.97163348)
\curveto(197.38116105,636.92163021)(197.38116105,636.87663026)(197.38117464,636.83663348)
\curveto(197.38116105,636.79663034)(197.38616104,636.75163038)(197.39617464,636.70163348)
\moveto(196.28617464,636.26663348)
\curveto(196.29616213,636.31663082)(196.30116213,636.39163074)(196.30117464,636.49163348)
\curveto(196.31116212,636.59163054)(196.30616212,636.66663047)(196.28617464,636.71663348)
\curveto(196.26616216,636.77663036)(196.26116217,636.8316303)(196.27117464,636.88163348)
\curveto(196.29116214,636.94163019)(196.29116214,637.00163013)(196.27117464,637.06163348)
\curveto(196.26116217,637.09163004)(196.25616217,637.12663001)(196.25617464,637.16663348)
\curveto(196.25616217,637.20662993)(196.25116218,637.24662989)(196.24117464,637.28663348)
\curveto(196.22116221,637.36662977)(196.20116223,637.44162969)(196.18117464,637.51163348)
\curveto(196.17116226,637.59162954)(196.15616227,637.67162946)(196.13617464,637.75163348)
\curveto(196.10616232,637.81162932)(196.08116235,637.87162926)(196.06117464,637.93163348)
\curveto(196.04116239,637.99162914)(196.01116242,638.05162908)(195.97117464,638.11163348)
\curveto(195.87116256,638.28162885)(195.74116269,638.41662872)(195.58117464,638.51663348)
\curveto(195.50116293,638.56662857)(195.40616302,638.60162853)(195.29617464,638.62163348)
\curveto(195.18616324,638.64162849)(195.06116337,638.65162848)(194.92117464,638.65163348)
\curveto(194.90116353,638.64162849)(194.87616355,638.6366285)(194.84617464,638.63663348)
\curveto(194.81616361,638.64662849)(194.78616364,638.64662849)(194.75617464,638.63663348)
\lineto(194.60617464,638.57663348)
\curveto(194.55616387,638.56662857)(194.51116392,638.55162858)(194.47117464,638.53163348)
\curveto(194.28116415,638.42162871)(194.13616429,638.27662886)(194.03617464,638.09663348)
\curveto(193.94616448,637.91662922)(193.86616456,637.71162942)(193.79617464,637.48163348)
\curveto(193.75616467,637.35162978)(193.73616469,637.21662992)(193.73617464,637.07663348)
\curveto(193.73616469,636.94663019)(193.7261647,636.80163033)(193.70617464,636.64163348)
\curveto(193.69616473,636.59163054)(193.68616474,636.5316306)(193.67617464,636.46163348)
\curveto(193.67616475,636.39163074)(193.68616474,636.3316308)(193.70617464,636.28163348)
\lineto(193.70617464,636.11663348)
\lineto(193.70617464,635.93663348)
\curveto(193.71616471,635.88663125)(193.7261647,635.8316313)(193.73617464,635.77163348)
\curveto(193.74616468,635.72163141)(193.75116468,635.66663147)(193.75117464,635.60663348)
\curveto(193.76116467,635.54663159)(193.77616465,635.49163164)(193.79617464,635.44163348)
\curveto(193.84616458,635.25163188)(193.90616452,635.07663206)(193.97617464,634.91663348)
\curveto(194.04616438,634.75663238)(194.15116428,634.62663251)(194.29117464,634.52663348)
\curveto(194.42116401,634.42663271)(194.56116387,634.35663278)(194.71117464,634.31663348)
\curveto(194.74116369,634.30663283)(194.76616366,634.30163283)(194.78617464,634.30163348)
\curveto(194.81616361,634.31163282)(194.84616358,634.31163282)(194.87617464,634.30163348)
\curveto(194.89616353,634.30163283)(194.9261635,634.29663284)(194.96617464,634.28663348)
\curveto(195.00616342,634.28663285)(195.04116339,634.29163284)(195.07117464,634.30163348)
\curveto(195.11116332,634.31163282)(195.15116328,634.31663282)(195.19117464,634.31663348)
\curveto(195.2311632,634.31663282)(195.27116316,634.32663281)(195.31117464,634.34663348)
\curveto(195.55116288,634.42663271)(195.74616268,634.56163257)(195.89617464,634.75163348)
\curveto(196.01616241,634.9316322)(196.10616232,635.136632)(196.16617464,635.36663348)
\curveto(196.18616224,635.4366317)(196.20116223,635.50663163)(196.21117464,635.57663348)
\curveto(196.22116221,635.65663148)(196.23616219,635.7366314)(196.25617464,635.81663348)
\curveto(196.25616217,635.87663126)(196.26116217,635.92163121)(196.27117464,635.95163348)
\curveto(196.27116216,635.97163116)(196.27116216,635.99663114)(196.27117464,636.02663348)
\curveto(196.27116216,636.06663107)(196.27616215,636.09663104)(196.28617464,636.11663348)
\lineto(196.28617464,636.26663348)
}
}
\end{pspicture}

%\caption{Porcentajes de los recursos según su tipo}
%\label{recursos_pie_1}
%\end{figure}

\subsection{Linea de tiempo}
Finalizados los elementos propios de la herramienta, se observan ahora las
lineas de tiempo, donde se presentan los tiempos en los que estos elementos han
sido creados.
En la figura \ref{tiempos_area_1} puede apreciarse en la linea de creación de
los usuarios, los registros automáticos de los estudiantes, de parte del
docente de su materia, siendo la creación de usuarios la linea predominante en
esta gráfica.

%\begin{figure}
%\centering
%%LaTeX with PSTricks extensions
%%Creator: inkscape 0.48.5
%%Please note this file requires PSTricks extensions
\psset{xunit=.5pt,yunit=.5pt,runit=.5pt}
\begin{pspicture}(1052.35998535,480)
{
\newrgbcolor{curcolor}{0 0 0}
\pscustom[linestyle=none,fillstyle=solid,fillcolor=curcolor]
{
\newpath
\moveto(28.10452492,447.20227663)
\curveto(28.13451719,447.08227242)(28.15951717,446.94227256)(28.17952492,446.78227663)
\curveto(28.19951713,446.62227288)(28.20951712,446.45727304)(28.20952492,446.28727663)
\curveto(28.20951712,446.11727338)(28.19951713,445.95227355)(28.17952492,445.79227663)
\curveto(28.15951717,445.63227387)(28.13451719,445.49227401)(28.10452492,445.37227663)
\curveto(28.06451726,445.23227427)(28.0295173,445.10727439)(27.99952492,444.99727663)
\curveto(27.96951736,444.88727461)(27.9295174,444.77727472)(27.87952492,444.66727663)
\curveto(27.60951772,444.02727547)(27.19451813,443.54227596)(26.63452492,443.21227663)
\curveto(26.55451877,443.15227635)(26.46951886,443.1022764)(26.37952492,443.06227663)
\curveto(26.28951904,443.03227647)(26.18951914,442.9972765)(26.07952492,442.95727663)
\curveto(25.96951936,442.90727659)(25.84951948,442.87227663)(25.71952492,442.85227663)
\curveto(25.59951973,442.82227668)(25.46951986,442.79227671)(25.32952492,442.76227663)
\curveto(25.26952006,442.74227676)(25.20952012,442.73727676)(25.14952492,442.74727663)
\curveto(25.09952023,442.75727674)(25.03952029,442.75227675)(24.96952492,442.73227663)
\curveto(24.94952038,442.72227678)(24.9245204,442.72227678)(24.89452492,442.73227663)
\curveto(24.86452046,442.73227677)(24.83952049,442.72727677)(24.81952492,442.71727663)
\lineto(24.66952492,442.71727663)
\curveto(24.59952073,442.70727679)(24.54952078,442.70727679)(24.51952492,442.71727663)
\curveto(24.47952085,442.72727677)(24.43452089,442.73227677)(24.38452492,442.73227663)
\curveto(24.34452098,442.72227678)(24.30452102,442.72227678)(24.26452492,442.73227663)
\curveto(24.17452115,442.75227675)(24.08452124,442.76727673)(23.99452492,442.77727663)
\curveto(23.90452142,442.77727672)(23.81452151,442.78727671)(23.72452492,442.80727663)
\curveto(23.63452169,442.83727666)(23.54452178,442.86227664)(23.45452492,442.88227663)
\curveto(23.36452196,442.9022766)(23.27952205,442.93227657)(23.19952492,442.97227663)
\curveto(22.95952237,443.08227642)(22.73452259,443.21227629)(22.52452492,443.36227663)
\curveto(22.31452301,443.52227598)(22.13452319,443.7022758)(21.98452492,443.90227663)
\curveto(21.86452346,444.07227543)(21.75952357,444.24727525)(21.66952492,444.42727663)
\curveto(21.57952375,444.60727489)(21.48952384,444.7972747)(21.39952492,444.99727663)
\curveto(21.35952397,445.0972744)(21.324524,445.1972743)(21.29452492,445.29727663)
\curveto(21.27452405,445.40727409)(21.24952408,445.51727398)(21.21952492,445.62727663)
\curveto(21.17952415,445.76727373)(21.15452417,445.90727359)(21.14452492,446.04727663)
\curveto(21.13452419,446.18727331)(21.11452421,446.32727317)(21.08452492,446.46727663)
\curveto(21.07452425,446.57727292)(21.06452426,446.67727282)(21.05452492,446.76727663)
\curveto(21.05452427,446.86727263)(21.04452428,446.96727253)(21.02452492,447.06727663)
\lineto(21.02452492,447.15727663)
\curveto(21.03452429,447.18727231)(21.03452429,447.21227229)(21.02452492,447.23227663)
\lineto(21.02452492,447.44227663)
\curveto(21.00452432,447.502272)(20.99452433,447.56727193)(20.99452492,447.63727663)
\curveto(21.00452432,447.71727178)(21.00952432,447.79227171)(21.00952492,447.86227663)
\lineto(21.00952492,448.01227663)
\curveto(21.00952432,448.06227144)(21.01452431,448.11227139)(21.02452492,448.16227663)
\lineto(21.02452492,448.53727663)
\curveto(21.03452429,448.56727093)(21.03452429,448.6022709)(21.02452492,448.64227663)
\curveto(21.0245243,448.68227082)(21.0295243,448.72227078)(21.03952492,448.76227663)
\curveto(21.05952427,448.87227063)(21.07452425,448.98227052)(21.08452492,449.09227663)
\curveto(21.09452423,449.21227029)(21.10452422,449.32727017)(21.11452492,449.43727663)
\curveto(21.15452417,449.58726991)(21.17952415,449.73226977)(21.18952492,449.87227663)
\curveto(21.20952412,450.02226948)(21.23952409,450.16726933)(21.27952492,450.30727663)
\curveto(21.36952396,450.60726889)(21.46452386,450.89226861)(21.56452492,451.16227663)
\curveto(21.66452366,451.43226807)(21.78952354,451.68226782)(21.93952492,451.91227663)
\curveto(22.13952319,452.23226727)(22.38452294,452.51226699)(22.67452492,452.75227663)
\curveto(22.96452236,452.99226651)(23.30452202,453.17726632)(23.69452492,453.30727663)
\curveto(23.80452152,453.34726615)(23.91452141,453.37226613)(24.02452492,453.38227663)
\curveto(24.14452118,453.4022661)(24.26452106,453.42726607)(24.38452492,453.45727663)
\curveto(24.45452087,453.46726603)(24.51952081,453.47226603)(24.57952492,453.47227663)
\curveto(24.63952069,453.47226603)(24.70452062,453.47726602)(24.77452492,453.48727663)
\curveto(25.47451985,453.50726599)(26.04951928,453.39226611)(26.49952492,453.14227663)
\curveto(26.94951838,452.89226661)(27.29451803,452.54226696)(27.53452492,452.09227663)
\curveto(27.64451768,451.86226764)(27.74451758,451.58726791)(27.83452492,451.26727663)
\curveto(27.85451747,451.1972683)(27.85451747,451.12226838)(27.83452492,451.04227663)
\curveto(27.8245175,450.97226853)(27.79951753,450.92226858)(27.75952492,450.89227663)
\curveto(27.7295176,450.86226864)(27.66951766,450.83726866)(27.57952492,450.81727663)
\curveto(27.48951784,450.80726869)(27.38951794,450.7972687)(27.27952492,450.78727663)
\curveto(27.17951815,450.78726871)(27.07951825,450.79226871)(26.97952492,450.80227663)
\curveto(26.88951844,450.81226869)(26.8245185,450.83226867)(26.78452492,450.86227663)
\curveto(26.67451865,450.93226857)(26.59451873,451.04226846)(26.54452492,451.19227663)
\curveto(26.50451882,451.34226816)(26.44951888,451.47226803)(26.37952492,451.58227663)
\curveto(26.18951914,451.89226761)(25.90951942,452.12226738)(25.53952492,452.27227663)
\curveto(25.46951986,452.3022672)(25.39451993,452.32226718)(25.31452492,452.33227663)
\curveto(25.24452008,452.34226716)(25.16952016,452.35726714)(25.08952492,452.37727663)
\curveto(25.03952029,452.38726711)(24.96952036,452.39226711)(24.87952492,452.39227663)
\curveto(24.79952053,452.39226711)(24.73452059,452.38726711)(24.68452492,452.37727663)
\curveto(24.64452068,452.35726714)(24.60952072,452.35226715)(24.57952492,452.36227663)
\curveto(24.54952078,452.37226713)(24.51452081,452.37226713)(24.47452492,452.36227663)
\lineto(24.23452492,452.30227663)
\curveto(24.16452116,452.28226722)(24.09452123,452.25726724)(24.02452492,452.22727663)
\curveto(23.64452168,452.06726743)(23.35452197,451.85726764)(23.15452492,451.59727663)
\curveto(22.96452236,451.33726816)(22.78952254,451.02226848)(22.62952492,450.65227663)
\curveto(22.59952273,450.57226893)(22.57452275,450.49226901)(22.55452492,450.41227663)
\curveto(22.54452278,450.33226917)(22.5245228,450.25226925)(22.49452492,450.17227663)
\curveto(22.46452286,450.06226944)(22.43952289,449.94726955)(22.41952492,449.82727663)
\curveto(22.40952292,449.70726979)(22.38952294,449.58726991)(22.35952492,449.46727663)
\curveto(22.33952299,449.41727008)(22.329523,449.36727013)(22.32952492,449.31727663)
\curveto(22.33952299,449.26727023)(22.33452299,449.21727028)(22.31452492,449.16727663)
\curveto(22.30452302,449.10727039)(22.30452302,449.02727047)(22.31452492,448.92727663)
\curveto(22.324523,448.83727066)(22.33952299,448.78227072)(22.35952492,448.76227663)
\curveto(22.37952295,448.72227078)(22.40952292,448.7022708)(22.44952492,448.70227663)
\curveto(22.49952283,448.7022708)(22.54452278,448.71227079)(22.58452492,448.73227663)
\curveto(22.65452267,448.77227073)(22.71452261,448.81727068)(22.76452492,448.86727663)
\curveto(22.81452251,448.91727058)(22.87452245,448.96727053)(22.94452492,449.01727663)
\lineto(23.00452492,449.07727663)
\curveto(23.03452229,449.10727039)(23.06452226,449.13227037)(23.09452492,449.15227663)
\curveto(23.324522,449.31227019)(23.59952173,449.44727005)(23.91952492,449.55727663)
\curveto(23.98952134,449.57726992)(24.05952127,449.59226991)(24.12952492,449.60227663)
\curveto(24.19952113,449.61226989)(24.27452105,449.62726987)(24.35452492,449.64727663)
\curveto(24.39452093,449.64726985)(24.4295209,449.65226985)(24.45952492,449.66227663)
\curveto(24.48952084,449.67226983)(24.5245208,449.67226983)(24.56452492,449.66227663)
\curveto(24.61452071,449.66226984)(24.65452067,449.67226983)(24.68452492,449.69227663)
\lineto(24.84952492,449.69227663)
\lineto(24.93952492,449.69227663)
\curveto(24.98952034,449.7022698)(25.0295203,449.7022698)(25.05952492,449.69227663)
\curveto(25.10952022,449.68226982)(25.15952017,449.67726982)(25.20952492,449.67727663)
\curveto(25.26952006,449.68726981)(25.32452,449.68726981)(25.37452492,449.67727663)
\curveto(25.48451984,449.64726985)(25.58951974,449.62726987)(25.68952492,449.61727663)
\curveto(25.79951953,449.60726989)(25.90451942,449.58226992)(26.00452492,449.54227663)
\curveto(26.4245189,449.4022701)(26.76951856,449.21727028)(27.03952492,448.98727663)
\curveto(27.30951802,448.76727073)(27.54951778,448.48227102)(27.75952492,448.13227663)
\curveto(27.83951749,447.99227151)(27.90451742,447.84227166)(27.95452492,447.68227663)
\curveto(28.00451732,447.53227197)(28.05451727,447.37227213)(28.10452492,447.20227663)
\moveto(26.85952492,445.89727663)
\curveto(26.86951846,445.94727355)(26.87451845,445.99227351)(26.87452492,446.03227663)
\lineto(26.87452492,446.18227663)
\curveto(26.87451845,446.49227301)(26.83451849,446.77727272)(26.75452492,447.03727663)
\curveto(26.73451859,447.0972724)(26.71451861,447.15227235)(26.69452492,447.20227663)
\curveto(26.68451864,447.26227224)(26.66951866,447.31727218)(26.64952492,447.36727663)
\curveto(26.4295189,447.85727164)(26.08451924,448.20727129)(25.61452492,448.41727663)
\curveto(25.53451979,448.44727105)(25.45451987,448.47227103)(25.37452492,448.49227663)
\lineto(25.13452492,448.55227663)
\curveto(25.05452027,448.57227093)(24.96452036,448.58227092)(24.86452492,448.58227663)
\lineto(24.54952492,448.58227663)
\curveto(24.5295208,448.56227094)(24.48952084,448.55227095)(24.42952492,448.55227663)
\curveto(24.37952095,448.56227094)(24.33452099,448.56227094)(24.29452492,448.55227663)
\lineto(24.05452492,448.49227663)
\curveto(23.98452134,448.48227102)(23.91452141,448.46227104)(23.84452492,448.43227663)
\curveto(23.24452208,448.17227133)(22.83952249,447.70727179)(22.62952492,447.03727663)
\curveto(22.59952273,446.95727254)(22.57952275,446.87727262)(22.56952492,446.79727663)
\curveto(22.55952277,446.71727278)(22.54452278,446.63227287)(22.52452492,446.54227663)
\lineto(22.52452492,446.39227663)
\curveto(22.51452281,446.35227315)(22.50952282,446.28227322)(22.50952492,446.18227663)
\curveto(22.50952282,445.95227355)(22.5295228,445.75727374)(22.56952492,445.59727663)
\curveto(22.58952274,445.52727397)(22.60452272,445.46227404)(22.61452492,445.40227663)
\curveto(22.6245227,445.34227416)(22.64452268,445.27727422)(22.67452492,445.20727663)
\curveto(22.78452254,444.92727457)(22.9295224,444.68227482)(23.10952492,444.47227663)
\curveto(23.28952204,444.27227523)(23.5245218,444.11227539)(23.81452492,443.99227663)
\lineto(24.05452492,443.90227663)
\lineto(24.29452492,443.84227663)
\curveto(24.34452098,443.82227568)(24.38452094,443.81727568)(24.41452492,443.82727663)
\curveto(24.45452087,443.83727566)(24.49952083,443.83227567)(24.54952492,443.81227663)
\curveto(24.57952075,443.8022757)(24.63452069,443.7972757)(24.71452492,443.79727663)
\curveto(24.79452053,443.7972757)(24.85452047,443.8022757)(24.89452492,443.81227663)
\curveto(25.00452032,443.83227567)(25.10952022,443.84727565)(25.20952492,443.85727663)
\curveto(25.30952002,443.86727563)(25.40451992,443.8972756)(25.49452492,443.94727663)
\curveto(26.0245193,444.14727535)(26.41451891,444.52227498)(26.66452492,445.07227663)
\curveto(26.70451862,445.17227433)(26.73451859,445.27727422)(26.75452492,445.38727663)
\lineto(26.84452492,445.71727663)
\curveto(26.84451848,445.7972737)(26.84951848,445.85727364)(26.85952492,445.89727663)
}
}
{
\newrgbcolor{curcolor}{0 0 0}
\pscustom[linestyle=none,fillstyle=solid,fillcolor=curcolor]
{
\newpath
\moveto(36.45413429,447.96727663)
\lineto(36.45413429,447.71227663)
\curveto(36.46412659,447.63227187)(36.45912659,447.55727194)(36.43913429,447.48727663)
\lineto(36.43913429,447.24727663)
\lineto(36.43913429,447.08227663)
\curveto(36.41912663,446.98227252)(36.40912664,446.87727262)(36.40913429,446.76727663)
\curveto(36.40912664,446.66727283)(36.39912665,446.56727293)(36.37913429,446.46727663)
\lineto(36.37913429,446.31727663)
\curveto(36.3491267,446.17727332)(36.32912672,446.03727346)(36.31913429,445.89727663)
\curveto(36.30912674,445.76727373)(36.28412677,445.63727386)(36.24413429,445.50727663)
\curveto(36.22412683,445.42727407)(36.20412685,445.34227416)(36.18413429,445.25227663)
\lineto(36.12413429,445.01227663)
\lineto(36.00413429,444.71227663)
\curveto(35.97412708,444.62227488)(35.93912711,444.53227497)(35.89913429,444.44227663)
\curveto(35.79912725,444.22227528)(35.66412739,444.00727549)(35.49413429,443.79727663)
\curveto(35.33412772,443.58727591)(35.15912789,443.41727608)(34.96913429,443.28727663)
\curveto(34.91912813,443.24727625)(34.85912819,443.20727629)(34.78913429,443.16727663)
\curveto(34.72912832,443.13727636)(34.66912838,443.1022764)(34.60913429,443.06227663)
\curveto(34.52912852,443.01227649)(34.43412862,442.97227653)(34.32413429,442.94227663)
\curveto(34.21412884,442.91227659)(34.10912894,442.88227662)(34.00913429,442.85227663)
\curveto(33.89912915,442.81227669)(33.78912926,442.78727671)(33.67913429,442.77727663)
\curveto(33.56912948,442.76727673)(33.4541296,442.75227675)(33.33413429,442.73227663)
\curveto(33.29412976,442.72227678)(33.2491298,442.72227678)(33.19913429,442.73227663)
\curveto(33.15912989,442.73227677)(33.11912993,442.72727677)(33.07913429,442.71727663)
\curveto(33.03913001,442.70727679)(32.98413007,442.7022768)(32.91413429,442.70227663)
\curveto(32.84413021,442.7022768)(32.79413026,442.70727679)(32.76413429,442.71727663)
\curveto(32.71413034,442.73727676)(32.66913038,442.74227676)(32.62913429,442.73227663)
\curveto(32.58913046,442.72227678)(32.5541305,442.72227678)(32.52413429,442.73227663)
\lineto(32.43413429,442.73227663)
\curveto(32.37413068,442.75227675)(32.30913074,442.76727673)(32.23913429,442.77727663)
\curveto(32.17913087,442.77727672)(32.11413094,442.78227672)(32.04413429,442.79227663)
\curveto(31.87413118,442.84227666)(31.71413134,442.89227661)(31.56413429,442.94227663)
\curveto(31.41413164,442.99227651)(31.26913178,443.05727644)(31.12913429,443.13727663)
\curveto(31.07913197,443.17727632)(31.02413203,443.20727629)(30.96413429,443.22727663)
\curveto(30.91413214,443.25727624)(30.86413219,443.29227621)(30.81413429,443.33227663)
\curveto(30.57413248,443.51227599)(30.37413268,443.73227577)(30.21413429,443.99227663)
\curveto(30.054133,444.25227525)(29.91413314,444.53727496)(29.79413429,444.84727663)
\curveto(29.73413332,444.98727451)(29.68913336,445.12727437)(29.65913429,445.26727663)
\curveto(29.62913342,445.41727408)(29.59413346,445.57227393)(29.55413429,445.73227663)
\curveto(29.53413352,445.84227366)(29.51913353,445.95227355)(29.50913429,446.06227663)
\curveto(29.49913355,446.17227333)(29.48413357,446.28227322)(29.46413429,446.39227663)
\curveto(29.4541336,446.43227307)(29.4491336,446.47227303)(29.44913429,446.51227663)
\curveto(29.45913359,446.55227295)(29.45913359,446.59227291)(29.44913429,446.63227663)
\curveto(29.43913361,446.68227282)(29.43413362,446.73227277)(29.43413429,446.78227663)
\lineto(29.43413429,446.94727663)
\curveto(29.41413364,446.9972725)(29.40913364,447.04727245)(29.41913429,447.09727663)
\curveto(29.42913362,447.15727234)(29.42913362,447.21227229)(29.41913429,447.26227663)
\curveto(29.40913364,447.3022722)(29.40913364,447.34727215)(29.41913429,447.39727663)
\curveto(29.42913362,447.44727205)(29.42413363,447.497272)(29.40413429,447.54727663)
\curveto(29.38413367,447.61727188)(29.37913367,447.69227181)(29.38913429,447.77227663)
\curveto(29.39913365,447.86227164)(29.40413365,447.94727155)(29.40413429,448.02727663)
\curveto(29.40413365,448.11727138)(29.39913365,448.21727128)(29.38913429,448.32727663)
\curveto(29.37913367,448.44727105)(29.38413367,448.54727095)(29.40413429,448.62727663)
\lineto(29.40413429,448.91227663)
\lineto(29.44913429,449.54227663)
\curveto(29.45913359,449.64226986)(29.46913358,449.73726976)(29.47913429,449.82727663)
\lineto(29.50913429,450.12727663)
\curveto(29.52913352,450.17726932)(29.53413352,450.22726927)(29.52413429,450.27727663)
\curveto(29.52413353,450.33726916)(29.53413352,450.39226911)(29.55413429,450.44227663)
\curveto(29.60413345,450.61226889)(29.64413341,450.77726872)(29.67413429,450.93727663)
\curveto(29.70413335,451.10726839)(29.7541333,451.26726823)(29.82413429,451.41727663)
\curveto(30.01413304,451.87726762)(30.23413282,452.25226725)(30.48413429,452.54227663)
\curveto(30.74413231,452.83226667)(31.10413195,453.07726642)(31.56413429,453.27727663)
\curveto(31.69413136,453.32726617)(31.82413123,453.36226614)(31.95413429,453.38227663)
\curveto(32.09413096,453.4022661)(32.23413082,453.42726607)(32.37413429,453.45727663)
\curveto(32.44413061,453.46726603)(32.50913054,453.47226603)(32.56913429,453.47227663)
\curveto(32.62913042,453.47226603)(32.69413036,453.47726602)(32.76413429,453.48727663)
\curveto(33.59412946,453.50726599)(34.26412879,453.35726614)(34.77413429,453.03727663)
\curveto(35.28412777,452.72726677)(35.66412739,452.28726721)(35.91413429,451.71727663)
\curveto(35.96412709,451.5972679)(36.00912704,451.47226803)(36.04913429,451.34227663)
\curveto(36.08912696,451.21226829)(36.13412692,451.07726842)(36.18413429,450.93727663)
\curveto(36.20412685,450.85726864)(36.21912683,450.77226873)(36.22913429,450.68227663)
\lineto(36.28913429,450.44227663)
\curveto(36.31912673,450.33226917)(36.33412672,450.22226928)(36.33413429,450.11227663)
\curveto(36.34412671,450.0022695)(36.35912669,449.89226961)(36.37913429,449.78227663)
\curveto(36.39912665,449.73226977)(36.40412665,449.68726981)(36.39413429,449.64727663)
\curveto(36.39412666,449.60726989)(36.39912665,449.56726993)(36.40913429,449.52727663)
\curveto(36.41912663,449.47727002)(36.41912663,449.42227008)(36.40913429,449.36227663)
\curveto(36.40912664,449.31227019)(36.41412664,449.26227024)(36.42413429,449.21227663)
\lineto(36.42413429,449.07727663)
\curveto(36.44412661,449.01727048)(36.44412661,448.94727055)(36.42413429,448.86727663)
\curveto(36.41412664,448.7972707)(36.41912663,448.73227077)(36.43913429,448.67227663)
\curveto(36.4491266,448.64227086)(36.4541266,448.6022709)(36.45413429,448.55227663)
\lineto(36.45413429,448.43227663)
\lineto(36.45413429,447.96727663)
\moveto(34.90913429,445.64227663)
\curveto(35.00912804,445.96227354)(35.06912798,446.32727317)(35.08913429,446.73727663)
\curveto(35.10912794,447.14727235)(35.11912793,447.55727194)(35.11913429,447.96727663)
\curveto(35.11912793,448.3972711)(35.10912794,448.81727068)(35.08913429,449.22727663)
\curveto(35.06912798,449.63726986)(35.02412803,450.02226948)(34.95413429,450.38227663)
\curveto(34.88412817,450.74226876)(34.77412828,451.06226844)(34.62413429,451.34227663)
\curveto(34.48412857,451.63226787)(34.28912876,451.86726763)(34.03913429,452.04727663)
\curveto(33.87912917,452.15726734)(33.69912935,452.23726726)(33.49913429,452.28727663)
\curveto(33.29912975,452.34726715)(33.05413,452.37726712)(32.76413429,452.37727663)
\curveto(32.74413031,452.35726714)(32.70913034,452.34726715)(32.65913429,452.34727663)
\curveto(32.60913044,452.35726714)(32.56913048,452.35726714)(32.53913429,452.34727663)
\curveto(32.45913059,452.32726717)(32.38413067,452.30726719)(32.31413429,452.28727663)
\curveto(32.2541308,452.27726722)(32.18913086,452.25726724)(32.11913429,452.22727663)
\curveto(31.8491312,452.10726739)(31.62913142,451.93726756)(31.45913429,451.71727663)
\curveto(31.29913175,451.50726799)(31.16413189,451.26226824)(31.05413429,450.98227663)
\curveto(31.00413205,450.87226863)(30.96413209,450.75226875)(30.93413429,450.62227663)
\curveto(30.91413214,450.502269)(30.88913216,450.37726912)(30.85913429,450.24727663)
\curveto(30.83913221,450.1972693)(30.82913222,450.14226936)(30.82913429,450.08227663)
\curveto(30.82913222,450.03226947)(30.82413223,449.98226952)(30.81413429,449.93227663)
\curveto(30.80413225,449.84226966)(30.79413226,449.74726975)(30.78413429,449.64727663)
\curveto(30.77413228,449.55726994)(30.76413229,449.46227004)(30.75413429,449.36227663)
\curveto(30.7541323,449.28227022)(30.7491323,449.1972703)(30.73913429,449.10727663)
\lineto(30.73913429,448.86727663)
\lineto(30.73913429,448.68727663)
\curveto(30.72913232,448.65727084)(30.72413233,448.62227088)(30.72413429,448.58227663)
\lineto(30.72413429,448.44727663)
\lineto(30.72413429,447.99727663)
\curveto(30.72413233,447.91727158)(30.71913233,447.83227167)(30.70913429,447.74227663)
\curveto(30.70913234,447.66227184)(30.71913233,447.58727191)(30.73913429,447.51727663)
\lineto(30.73913429,447.24727663)
\curveto(30.73913231,447.22727227)(30.73413232,447.1972723)(30.72413429,447.15727663)
\curveto(30.72413233,447.12727237)(30.72913232,447.1022724)(30.73913429,447.08227663)
\curveto(30.7491323,446.98227252)(30.7541323,446.88227262)(30.75413429,446.78227663)
\curveto(30.76413229,446.69227281)(30.77413228,446.59227291)(30.78413429,446.48227663)
\curveto(30.81413224,446.36227314)(30.82913222,446.23727326)(30.82913429,446.10727663)
\curveto(30.83913221,445.98727351)(30.86413219,445.87227363)(30.90413429,445.76227663)
\curveto(30.98413207,445.46227404)(31.06913198,445.1972743)(31.15913429,444.96727663)
\curveto(31.25913179,444.73727476)(31.40413165,444.52227498)(31.59413429,444.32227663)
\curveto(31.80413125,444.12227538)(32.06913098,443.97227553)(32.38913429,443.87227663)
\curveto(32.42913062,443.85227565)(32.46413059,443.84227566)(32.49413429,443.84227663)
\curveto(32.53413052,443.85227565)(32.57913047,443.84727565)(32.62913429,443.82727663)
\curveto(32.66913038,443.81727568)(32.73913031,443.80727569)(32.83913429,443.79727663)
\curveto(32.9491301,443.78727571)(33.03413002,443.79227571)(33.09413429,443.81227663)
\curveto(33.16412989,443.83227567)(33.23412982,443.84227566)(33.30413429,443.84227663)
\curveto(33.37412968,443.85227565)(33.43912961,443.86727563)(33.49913429,443.88727663)
\curveto(33.69912935,443.94727555)(33.87912917,444.03227547)(34.03913429,444.14227663)
\curveto(34.06912898,444.16227534)(34.09412896,444.18227532)(34.11413429,444.20227663)
\lineto(34.17413429,444.26227663)
\curveto(34.21412884,444.28227522)(34.26412879,444.32227518)(34.32413429,444.38227663)
\curveto(34.42412863,444.52227498)(34.50912854,444.65227485)(34.57913429,444.77227663)
\curveto(34.6491284,444.89227461)(34.71912833,445.03727446)(34.78913429,445.20727663)
\curveto(34.81912823,445.27727422)(34.83912821,445.34727415)(34.84913429,445.41727663)
\curveto(34.86912818,445.48727401)(34.88912816,445.56227394)(34.90913429,445.64227663)
}
}
{
\newrgbcolor{curcolor}{0 0 0}
\pscustom[linestyle=none,fillstyle=solid,fillcolor=curcolor]
{
\newpath
\moveto(44.80374367,447.96727663)
\lineto(44.80374367,447.71227663)
\curveto(44.81373596,447.63227187)(44.80873597,447.55727194)(44.78874367,447.48727663)
\lineto(44.78874367,447.24727663)
\lineto(44.78874367,447.08227663)
\curveto(44.76873601,446.98227252)(44.75873602,446.87727262)(44.75874367,446.76727663)
\curveto(44.75873602,446.66727283)(44.74873603,446.56727293)(44.72874367,446.46727663)
\lineto(44.72874367,446.31727663)
\curveto(44.69873608,446.17727332)(44.6787361,446.03727346)(44.66874367,445.89727663)
\curveto(44.65873612,445.76727373)(44.63373614,445.63727386)(44.59374367,445.50727663)
\curveto(44.5737362,445.42727407)(44.55373622,445.34227416)(44.53374367,445.25227663)
\lineto(44.47374367,445.01227663)
\lineto(44.35374367,444.71227663)
\curveto(44.32373645,444.62227488)(44.28873649,444.53227497)(44.24874367,444.44227663)
\curveto(44.14873663,444.22227528)(44.01373676,444.00727549)(43.84374367,443.79727663)
\curveto(43.68373709,443.58727591)(43.50873727,443.41727608)(43.31874367,443.28727663)
\curveto(43.26873751,443.24727625)(43.20873757,443.20727629)(43.13874367,443.16727663)
\curveto(43.0787377,443.13727636)(43.01873776,443.1022764)(42.95874367,443.06227663)
\curveto(42.8787379,443.01227649)(42.78373799,442.97227653)(42.67374367,442.94227663)
\curveto(42.56373821,442.91227659)(42.45873832,442.88227662)(42.35874367,442.85227663)
\curveto(42.24873853,442.81227669)(42.13873864,442.78727671)(42.02874367,442.77727663)
\curveto(41.91873886,442.76727673)(41.80373897,442.75227675)(41.68374367,442.73227663)
\curveto(41.64373913,442.72227678)(41.59873918,442.72227678)(41.54874367,442.73227663)
\curveto(41.50873927,442.73227677)(41.46873931,442.72727677)(41.42874367,442.71727663)
\curveto(41.38873939,442.70727679)(41.33373944,442.7022768)(41.26374367,442.70227663)
\curveto(41.19373958,442.7022768)(41.14373963,442.70727679)(41.11374367,442.71727663)
\curveto(41.06373971,442.73727676)(41.01873976,442.74227676)(40.97874367,442.73227663)
\curveto(40.93873984,442.72227678)(40.90373987,442.72227678)(40.87374367,442.73227663)
\lineto(40.78374367,442.73227663)
\curveto(40.72374005,442.75227675)(40.65874012,442.76727673)(40.58874367,442.77727663)
\curveto(40.52874025,442.77727672)(40.46374031,442.78227672)(40.39374367,442.79227663)
\curveto(40.22374055,442.84227666)(40.06374071,442.89227661)(39.91374367,442.94227663)
\curveto(39.76374101,442.99227651)(39.61874116,443.05727644)(39.47874367,443.13727663)
\curveto(39.42874135,443.17727632)(39.3737414,443.20727629)(39.31374367,443.22727663)
\curveto(39.26374151,443.25727624)(39.21374156,443.29227621)(39.16374367,443.33227663)
\curveto(38.92374185,443.51227599)(38.72374205,443.73227577)(38.56374367,443.99227663)
\curveto(38.40374237,444.25227525)(38.26374251,444.53727496)(38.14374367,444.84727663)
\curveto(38.08374269,444.98727451)(38.03874274,445.12727437)(38.00874367,445.26727663)
\curveto(37.9787428,445.41727408)(37.94374283,445.57227393)(37.90374367,445.73227663)
\curveto(37.88374289,445.84227366)(37.86874291,445.95227355)(37.85874367,446.06227663)
\curveto(37.84874293,446.17227333)(37.83374294,446.28227322)(37.81374367,446.39227663)
\curveto(37.80374297,446.43227307)(37.79874298,446.47227303)(37.79874367,446.51227663)
\curveto(37.80874297,446.55227295)(37.80874297,446.59227291)(37.79874367,446.63227663)
\curveto(37.78874299,446.68227282)(37.78374299,446.73227277)(37.78374367,446.78227663)
\lineto(37.78374367,446.94727663)
\curveto(37.76374301,446.9972725)(37.75874302,447.04727245)(37.76874367,447.09727663)
\curveto(37.778743,447.15727234)(37.778743,447.21227229)(37.76874367,447.26227663)
\curveto(37.75874302,447.3022722)(37.75874302,447.34727215)(37.76874367,447.39727663)
\curveto(37.778743,447.44727205)(37.773743,447.497272)(37.75374367,447.54727663)
\curveto(37.73374304,447.61727188)(37.72874305,447.69227181)(37.73874367,447.77227663)
\curveto(37.74874303,447.86227164)(37.75374302,447.94727155)(37.75374367,448.02727663)
\curveto(37.75374302,448.11727138)(37.74874303,448.21727128)(37.73874367,448.32727663)
\curveto(37.72874305,448.44727105)(37.73374304,448.54727095)(37.75374367,448.62727663)
\lineto(37.75374367,448.91227663)
\lineto(37.79874367,449.54227663)
\curveto(37.80874297,449.64226986)(37.81874296,449.73726976)(37.82874367,449.82727663)
\lineto(37.85874367,450.12727663)
\curveto(37.8787429,450.17726932)(37.88374289,450.22726927)(37.87374367,450.27727663)
\curveto(37.8737429,450.33726916)(37.88374289,450.39226911)(37.90374367,450.44227663)
\curveto(37.95374282,450.61226889)(37.99374278,450.77726872)(38.02374367,450.93727663)
\curveto(38.05374272,451.10726839)(38.10374267,451.26726823)(38.17374367,451.41727663)
\curveto(38.36374241,451.87726762)(38.58374219,452.25226725)(38.83374367,452.54227663)
\curveto(39.09374168,452.83226667)(39.45374132,453.07726642)(39.91374367,453.27727663)
\curveto(40.04374073,453.32726617)(40.1737406,453.36226614)(40.30374367,453.38227663)
\curveto(40.44374033,453.4022661)(40.58374019,453.42726607)(40.72374367,453.45727663)
\curveto(40.79373998,453.46726603)(40.85873992,453.47226603)(40.91874367,453.47227663)
\curveto(40.9787398,453.47226603)(41.04373973,453.47726602)(41.11374367,453.48727663)
\curveto(41.94373883,453.50726599)(42.61373816,453.35726614)(43.12374367,453.03727663)
\curveto(43.63373714,452.72726677)(44.01373676,452.28726721)(44.26374367,451.71727663)
\curveto(44.31373646,451.5972679)(44.35873642,451.47226803)(44.39874367,451.34227663)
\curveto(44.43873634,451.21226829)(44.48373629,451.07726842)(44.53374367,450.93727663)
\curveto(44.55373622,450.85726864)(44.56873621,450.77226873)(44.57874367,450.68227663)
\lineto(44.63874367,450.44227663)
\curveto(44.66873611,450.33226917)(44.68373609,450.22226928)(44.68374367,450.11227663)
\curveto(44.69373608,450.0022695)(44.70873607,449.89226961)(44.72874367,449.78227663)
\curveto(44.74873603,449.73226977)(44.75373602,449.68726981)(44.74374367,449.64727663)
\curveto(44.74373603,449.60726989)(44.74873603,449.56726993)(44.75874367,449.52727663)
\curveto(44.76873601,449.47727002)(44.76873601,449.42227008)(44.75874367,449.36227663)
\curveto(44.75873602,449.31227019)(44.76373601,449.26227024)(44.77374367,449.21227663)
\lineto(44.77374367,449.07727663)
\curveto(44.79373598,449.01727048)(44.79373598,448.94727055)(44.77374367,448.86727663)
\curveto(44.76373601,448.7972707)(44.76873601,448.73227077)(44.78874367,448.67227663)
\curveto(44.79873598,448.64227086)(44.80373597,448.6022709)(44.80374367,448.55227663)
\lineto(44.80374367,448.43227663)
\lineto(44.80374367,447.96727663)
\moveto(43.25874367,445.64227663)
\curveto(43.35873742,445.96227354)(43.41873736,446.32727317)(43.43874367,446.73727663)
\curveto(43.45873732,447.14727235)(43.46873731,447.55727194)(43.46874367,447.96727663)
\curveto(43.46873731,448.3972711)(43.45873732,448.81727068)(43.43874367,449.22727663)
\curveto(43.41873736,449.63726986)(43.3737374,450.02226948)(43.30374367,450.38227663)
\curveto(43.23373754,450.74226876)(43.12373765,451.06226844)(42.97374367,451.34227663)
\curveto(42.83373794,451.63226787)(42.63873814,451.86726763)(42.38874367,452.04727663)
\curveto(42.22873855,452.15726734)(42.04873873,452.23726726)(41.84874367,452.28727663)
\curveto(41.64873913,452.34726715)(41.40373937,452.37726712)(41.11374367,452.37727663)
\curveto(41.09373968,452.35726714)(41.05873972,452.34726715)(41.00874367,452.34727663)
\curveto(40.95873982,452.35726714)(40.91873986,452.35726714)(40.88874367,452.34727663)
\curveto(40.80873997,452.32726717)(40.73374004,452.30726719)(40.66374367,452.28727663)
\curveto(40.60374017,452.27726722)(40.53874024,452.25726724)(40.46874367,452.22727663)
\curveto(40.19874058,452.10726739)(39.9787408,451.93726756)(39.80874367,451.71727663)
\curveto(39.64874113,451.50726799)(39.51374126,451.26226824)(39.40374367,450.98227663)
\curveto(39.35374142,450.87226863)(39.31374146,450.75226875)(39.28374367,450.62227663)
\curveto(39.26374151,450.502269)(39.23874154,450.37726912)(39.20874367,450.24727663)
\curveto(39.18874159,450.1972693)(39.1787416,450.14226936)(39.17874367,450.08227663)
\curveto(39.1787416,450.03226947)(39.1737416,449.98226952)(39.16374367,449.93227663)
\curveto(39.15374162,449.84226966)(39.14374163,449.74726975)(39.13374367,449.64727663)
\curveto(39.12374165,449.55726994)(39.11374166,449.46227004)(39.10374367,449.36227663)
\curveto(39.10374167,449.28227022)(39.09874168,449.1972703)(39.08874367,449.10727663)
\lineto(39.08874367,448.86727663)
\lineto(39.08874367,448.68727663)
\curveto(39.0787417,448.65727084)(39.0737417,448.62227088)(39.07374367,448.58227663)
\lineto(39.07374367,448.44727663)
\lineto(39.07374367,447.99727663)
\curveto(39.0737417,447.91727158)(39.06874171,447.83227167)(39.05874367,447.74227663)
\curveto(39.05874172,447.66227184)(39.06874171,447.58727191)(39.08874367,447.51727663)
\lineto(39.08874367,447.24727663)
\curveto(39.08874169,447.22727227)(39.08374169,447.1972723)(39.07374367,447.15727663)
\curveto(39.0737417,447.12727237)(39.0787417,447.1022724)(39.08874367,447.08227663)
\curveto(39.09874168,446.98227252)(39.10374167,446.88227262)(39.10374367,446.78227663)
\curveto(39.11374166,446.69227281)(39.12374165,446.59227291)(39.13374367,446.48227663)
\curveto(39.16374161,446.36227314)(39.1787416,446.23727326)(39.17874367,446.10727663)
\curveto(39.18874159,445.98727351)(39.21374156,445.87227363)(39.25374367,445.76227663)
\curveto(39.33374144,445.46227404)(39.41874136,445.1972743)(39.50874367,444.96727663)
\curveto(39.60874117,444.73727476)(39.75374102,444.52227498)(39.94374367,444.32227663)
\curveto(40.15374062,444.12227538)(40.41874036,443.97227553)(40.73874367,443.87227663)
\curveto(40.77874,443.85227565)(40.81373996,443.84227566)(40.84374367,443.84227663)
\curveto(40.88373989,443.85227565)(40.92873985,443.84727565)(40.97874367,443.82727663)
\curveto(41.01873976,443.81727568)(41.08873969,443.80727569)(41.18874367,443.79727663)
\curveto(41.29873948,443.78727571)(41.38373939,443.79227571)(41.44374367,443.81227663)
\curveto(41.51373926,443.83227567)(41.58373919,443.84227566)(41.65374367,443.84227663)
\curveto(41.72373905,443.85227565)(41.78873899,443.86727563)(41.84874367,443.88727663)
\curveto(42.04873873,443.94727555)(42.22873855,444.03227547)(42.38874367,444.14227663)
\curveto(42.41873836,444.16227534)(42.44373833,444.18227532)(42.46374367,444.20227663)
\lineto(42.52374367,444.26227663)
\curveto(42.56373821,444.28227522)(42.61373816,444.32227518)(42.67374367,444.38227663)
\curveto(42.773738,444.52227498)(42.85873792,444.65227485)(42.92874367,444.77227663)
\curveto(42.99873778,444.89227461)(43.06873771,445.03727446)(43.13874367,445.20727663)
\curveto(43.16873761,445.27727422)(43.18873759,445.34727415)(43.19874367,445.41727663)
\curveto(43.21873756,445.48727401)(43.23873754,445.56227394)(43.25874367,445.64227663)
}
}
{
\newrgbcolor{curcolor}{0 0 0}
\pscustom[linestyle=none,fillstyle=solid,fillcolor=curcolor]
{
\newpath
\moveto(27.98452492,366.21516726)
\curveto(28.05451727,366.1651638)(28.09451723,366.09516387)(28.10452492,366.00516726)
\curveto(28.1245172,365.91516405)(28.13451719,365.81016415)(28.13452492,365.69016726)
\curveto(28.13451719,365.64016432)(28.1295172,365.59016437)(28.11952492,365.54016726)
\curveto(28.11951721,365.49016447)(28.10951722,365.44516452)(28.08952492,365.40516726)
\curveto(28.05951727,365.31516465)(27.99951733,365.25516471)(27.90952492,365.22516726)
\curveto(27.8295175,365.20516476)(27.73451759,365.19516477)(27.62452492,365.19516726)
\lineto(27.30952492,365.19516726)
\curveto(27.19951813,365.20516476)(27.09451823,365.19516477)(26.99452492,365.16516726)
\curveto(26.85451847,365.13516483)(26.76451856,365.05516491)(26.72452492,364.92516726)
\curveto(26.70451862,364.85516511)(26.69451863,364.77016519)(26.69452492,364.67016726)
\lineto(26.69452492,364.40016726)
\lineto(26.69452492,363.45516726)
\lineto(26.69452492,363.12516726)
\curveto(26.69451863,363.01516695)(26.67451865,362.93016703)(26.63452492,362.87016726)
\curveto(26.59451873,362.81016715)(26.54451878,362.77016719)(26.48452492,362.75016726)
\curveto(26.43451889,362.74016722)(26.36951896,362.72516724)(26.28952492,362.70516726)
\lineto(26.09452492,362.70516726)
\curveto(25.97451935,362.70516726)(25.86951946,362.71016725)(25.77952492,362.72016726)
\curveto(25.68951964,362.74016722)(25.61951971,362.79016717)(25.56952492,362.87016726)
\curveto(25.53951979,362.92016704)(25.5245198,362.99016697)(25.52452492,363.08016726)
\lineto(25.52452492,363.38016726)
\lineto(25.52452492,364.41516726)
\curveto(25.5245198,364.57516539)(25.51451981,364.72016524)(25.49452492,364.85016726)
\curveto(25.48451984,364.99016497)(25.4295199,365.08516488)(25.32952492,365.13516726)
\curveto(25.27952005,365.15516481)(25.20952012,365.17016479)(25.11952492,365.18016726)
\curveto(25.03952029,365.19016477)(24.94952038,365.19516477)(24.84952492,365.19516726)
\lineto(24.56452492,365.19516726)
\lineto(24.32452492,365.19516726)
\lineto(22.05952492,365.19516726)
\curveto(21.96952336,365.19516477)(21.86452346,365.19016477)(21.74452492,365.18016726)
\lineto(21.41452492,365.18016726)
\curveto(21.30452402,365.18016478)(21.20452412,365.19016477)(21.11452492,365.21016726)
\curveto(21.0245243,365.23016473)(20.96452436,365.2651647)(20.93452492,365.31516726)
\curveto(20.88452444,365.38516458)(20.85952447,365.48016448)(20.85952492,365.60016726)
\lineto(20.85952492,365.94516726)
\lineto(20.85952492,366.21516726)
\curveto(20.89952443,366.38516358)(20.95452437,366.52516344)(21.02452492,366.63516726)
\curveto(21.09452423,366.74516322)(21.17452415,366.8601631)(21.26452492,366.98016726)
\lineto(21.62452492,367.52016726)
\curveto(22.06452326,368.15016181)(22.49952283,368.77016119)(22.92952492,369.38016726)
\lineto(24.24952492,371.24016726)
\curveto(24.40952092,371.47015849)(24.56452076,371.69015827)(24.71452492,371.90016726)
\curveto(24.86452046,372.12015784)(25.01952031,372.34515762)(25.17952492,372.57516726)
\curveto(25.2295201,372.64515732)(25.27952005,372.71015725)(25.32952492,372.77016726)
\curveto(25.37951995,372.84015712)(25.4295199,372.91515705)(25.47952492,372.99516726)
\lineto(25.53952492,373.08516726)
\curveto(25.56951976,373.12515684)(25.59951973,373.15515681)(25.62952492,373.17516726)
\curveto(25.66951966,373.20515676)(25.70951962,373.22515674)(25.74952492,373.23516726)
\curveto(25.78951954,373.25515671)(25.83451949,373.27515669)(25.88452492,373.29516726)
\curveto(25.90451942,373.29515667)(25.9245194,373.29015667)(25.94452492,373.28016726)
\curveto(25.97451935,373.28015668)(25.99951933,373.29015667)(26.01952492,373.31016726)
\curveto(26.14951918,373.31015665)(26.26951906,373.30515666)(26.37952492,373.29516726)
\curveto(26.48951884,373.28515668)(26.56951876,373.24015672)(26.61952492,373.16016726)
\curveto(26.65951867,373.11015685)(26.67951865,373.04015692)(26.67952492,372.95016726)
\curveto(26.68951864,372.8601571)(26.69451863,372.7651572)(26.69452492,372.66516726)
\lineto(26.69452492,367.20516726)
\curveto(26.69451863,367.13516283)(26.68951864,367.0601629)(26.67952492,366.98016726)
\curveto(26.67951865,366.91016305)(26.68451864,366.84016312)(26.69452492,366.77016726)
\lineto(26.69452492,366.66516726)
\curveto(26.71451861,366.61516335)(26.7295186,366.5601634)(26.73952492,366.50016726)
\curveto(26.74951858,366.45016351)(26.77451855,366.41016355)(26.81452492,366.38016726)
\curveto(26.88451844,366.33016363)(26.96951836,366.30016366)(27.06952492,366.29016726)
\lineto(27.39952492,366.29016726)
\curveto(27.50951782,366.29016367)(27.61451771,366.28516368)(27.71452492,366.27516726)
\curveto(27.8245175,366.27516369)(27.91451741,366.25516371)(27.98452492,366.21516726)
\moveto(25.41952492,366.41016726)
\curveto(25.49951983,366.52016344)(25.53451979,366.69016327)(25.52452492,366.92016726)
\lineto(25.52452492,367.53516726)
\lineto(25.52452492,370.01016726)
\lineto(25.52452492,370.32516726)
\curveto(25.53451979,370.44515952)(25.5295198,370.54515942)(25.50952492,370.62516726)
\lineto(25.50952492,370.77516726)
\curveto(25.50951982,370.8651591)(25.49451983,370.95015901)(25.46452492,371.03016726)
\curveto(25.45451987,371.05015891)(25.44451988,371.0601589)(25.43452492,371.06016726)
\lineto(25.38952492,371.10516726)
\curveto(25.36951996,371.11515885)(25.33951999,371.12015884)(25.29952492,371.12016726)
\curveto(25.27952005,371.10015886)(25.25952007,371.08515888)(25.23952492,371.07516726)
\curveto(25.2295201,371.07515889)(25.21452011,371.07015889)(25.19452492,371.06016726)
\curveto(25.13452019,371.01015895)(25.07452025,370.94015902)(25.01452492,370.85016726)
\curveto(24.95452037,370.7601592)(24.89952043,370.68015928)(24.84952492,370.61016726)
\curveto(24.74952058,370.47015949)(24.65452067,370.32515964)(24.56452492,370.17516726)
\curveto(24.47452085,370.03515993)(24.37952095,369.89516007)(24.27952492,369.75516726)
\lineto(23.73952492,368.97516726)
\curveto(23.56952176,368.71516125)(23.39452193,368.45516151)(23.21452492,368.19516726)
\curveto(23.13452219,368.08516188)(23.05952227,367.98016198)(22.98952492,367.88016726)
\lineto(22.77952492,367.58016726)
\curveto(22.7295226,367.50016246)(22.67952265,367.42516254)(22.62952492,367.35516726)
\curveto(22.58952274,367.28516268)(22.54452278,367.21016275)(22.49452492,367.13016726)
\curveto(22.44452288,367.07016289)(22.39452293,367.00516296)(22.34452492,366.93516726)
\curveto(22.30452302,366.87516309)(22.26452306,366.80516316)(22.22452492,366.72516726)
\curveto(22.18452314,366.6651633)(22.15952317,366.59516337)(22.14952492,366.51516726)
\curveto(22.13952319,366.44516352)(22.17452315,366.39016357)(22.25452492,366.35016726)
\curveto(22.324523,366.30016366)(22.43452289,366.27516369)(22.58452492,366.27516726)
\curveto(22.74452258,366.28516368)(22.87952245,366.29016367)(22.98952492,366.29016726)
\lineto(24.66952492,366.29016726)
\lineto(25.10452492,366.29016726)
\curveto(25.25452007,366.29016367)(25.35951997,366.33016363)(25.41952492,366.41016726)
}
}
{
\newrgbcolor{curcolor}{0 0 0}
\pscustom[linestyle=none,fillstyle=solid,fillcolor=curcolor]
{
\newpath
\moveto(30.96413429,373.13016726)
\lineto(34.56413429,373.13016726)
\lineto(35.20913429,373.13016726)
\curveto(35.28912776,373.13015683)(35.36412769,373.12515684)(35.43413429,373.11516726)
\curveto(35.50412755,373.11515685)(35.56412749,373.10515686)(35.61413429,373.08516726)
\curveto(35.68412737,373.05515691)(35.73912731,372.99515697)(35.77913429,372.90516726)
\curveto(35.79912725,372.87515709)(35.80912724,372.83515713)(35.80913429,372.78516726)
\lineto(35.80913429,372.65016726)
\curveto(35.81912723,372.54015742)(35.81412724,372.43515753)(35.79413429,372.33516726)
\curveto(35.78412727,372.23515773)(35.7491273,372.1651578)(35.68913429,372.12516726)
\curveto(35.59912745,372.05515791)(35.46412759,372.02015794)(35.28413429,372.02016726)
\curveto(35.10412795,372.03015793)(34.93912811,372.03515793)(34.78913429,372.03516726)
\lineto(32.79413429,372.03516726)
\lineto(32.29913429,372.03516726)
\lineto(32.16413429,372.03516726)
\curveto(32.12413093,372.03515793)(32.08413097,372.03015793)(32.04413429,372.02016726)
\lineto(31.83413429,372.02016726)
\curveto(31.72413133,371.99015797)(31.64413141,371.95015801)(31.59413429,371.90016726)
\curveto(31.54413151,371.8601581)(31.50913154,371.80515816)(31.48913429,371.73516726)
\curveto(31.46913158,371.67515829)(31.4541316,371.60515836)(31.44413429,371.52516726)
\curveto(31.43413162,371.44515852)(31.41413164,371.35515861)(31.38413429,371.25516726)
\curveto(31.33413172,371.05515891)(31.29413176,370.85015911)(31.26413429,370.64016726)
\curveto(31.23413182,370.43015953)(31.19413186,370.22515974)(31.14413429,370.02516726)
\curveto(31.12413193,369.95516001)(31.11413194,369.88516008)(31.11413429,369.81516726)
\curveto(31.11413194,369.75516021)(31.10413195,369.69016027)(31.08413429,369.62016726)
\curveto(31.07413198,369.59016037)(31.06413199,369.55016041)(31.05413429,369.50016726)
\curveto(31.054132,369.4601605)(31.05913199,369.42016054)(31.06913429,369.38016726)
\curveto(31.08913196,369.33016063)(31.11413194,369.28516068)(31.14413429,369.24516726)
\curveto(31.18413187,369.21516075)(31.24413181,369.21016075)(31.32413429,369.23016726)
\curveto(31.38413167,369.25016071)(31.44413161,369.27516069)(31.50413429,369.30516726)
\curveto(31.56413149,369.34516062)(31.62413143,369.38016058)(31.68413429,369.41016726)
\curveto(31.74413131,369.43016053)(31.79413126,369.44516052)(31.83413429,369.45516726)
\curveto(32.02413103,369.53516043)(32.22913082,369.59016037)(32.44913429,369.62016726)
\curveto(32.67913037,369.65016031)(32.90913014,369.6601603)(33.13913429,369.65016726)
\curveto(33.37912967,369.65016031)(33.60912944,369.62516034)(33.82913429,369.57516726)
\curveto(34.049129,369.53516043)(34.2491288,369.47516049)(34.42913429,369.39516726)
\curveto(34.47912857,369.37516059)(34.52412853,369.35516061)(34.56413429,369.33516726)
\curveto(34.61412844,369.31516065)(34.66412839,369.29016067)(34.71413429,369.26016726)
\curveto(35.06412799,369.05016091)(35.34412771,368.82016114)(35.55413429,368.57016726)
\curveto(35.77412728,368.32016164)(35.96912708,367.99516197)(36.13913429,367.59516726)
\curveto(36.18912686,367.48516248)(36.22412683,367.37516259)(36.24413429,367.26516726)
\curveto(36.26412679,367.15516281)(36.28912676,367.04016292)(36.31913429,366.92016726)
\curveto(36.32912672,366.89016307)(36.33412672,366.84516312)(36.33413429,366.78516726)
\curveto(36.3541267,366.72516324)(36.36412669,366.65516331)(36.36413429,366.57516726)
\curveto(36.36412669,366.50516346)(36.37412668,366.44016352)(36.39413429,366.38016726)
\lineto(36.39413429,366.21516726)
\curveto(36.40412665,366.1651638)(36.40912664,366.09516387)(36.40913429,366.00516726)
\curveto(36.40912664,365.91516405)(36.39912665,365.84516412)(36.37913429,365.79516726)
\curveto(36.35912669,365.73516423)(36.3541267,365.67516429)(36.36413429,365.61516726)
\curveto(36.37412668,365.5651644)(36.36912668,365.51516445)(36.34913429,365.46516726)
\curveto(36.30912674,365.30516466)(36.27412678,365.15516481)(36.24413429,365.01516726)
\curveto(36.21412684,364.87516509)(36.16912688,364.74016522)(36.10913429,364.61016726)
\curveto(35.9491271,364.24016572)(35.72912732,363.90516606)(35.44913429,363.60516726)
\curveto(35.16912788,363.30516666)(34.8491282,363.07516689)(34.48913429,362.91516726)
\curveto(34.31912873,362.83516713)(34.11912893,362.7601672)(33.88913429,362.69016726)
\curveto(33.77912927,362.65016731)(33.66412939,362.62516734)(33.54413429,362.61516726)
\curveto(33.42412963,362.60516736)(33.30412975,362.58516738)(33.18413429,362.55516726)
\curveto(33.13412992,362.53516743)(33.07912997,362.53516743)(33.01913429,362.55516726)
\curveto(32.95913009,362.5651674)(32.89913015,362.5601674)(32.83913429,362.54016726)
\curveto(32.73913031,362.52016744)(32.63913041,362.52016744)(32.53913429,362.54016726)
\lineto(32.40413429,362.54016726)
\curveto(32.3541307,362.5601674)(32.29413076,362.57016739)(32.22413429,362.57016726)
\curveto(32.16413089,362.5601674)(32.10913094,362.5651674)(32.05913429,362.58516726)
\curveto(32.01913103,362.59516737)(31.98413107,362.60016736)(31.95413429,362.60016726)
\curveto(31.92413113,362.60016736)(31.88913116,362.60516736)(31.84913429,362.61516726)
\lineto(31.57913429,362.67516726)
\curveto(31.48913156,362.69516727)(31.40413165,362.72516724)(31.32413429,362.76516726)
\curveto(30.98413207,362.90516706)(30.69413236,363.0601669)(30.45413429,363.23016726)
\curveto(30.21413284,363.41016655)(29.99413306,363.64016632)(29.79413429,363.92016726)
\curveto(29.64413341,364.15016581)(29.52913352,364.39016557)(29.44913429,364.64016726)
\curveto(29.42913362,364.69016527)(29.41913363,364.73516523)(29.41913429,364.77516726)
\curveto(29.41913363,364.82516514)(29.40913364,364.87516509)(29.38913429,364.92516726)
\curveto(29.36913368,364.98516498)(29.3541337,365.0651649)(29.34413429,365.16516726)
\curveto(29.34413371,365.2651647)(29.36413369,365.34016462)(29.40413429,365.39016726)
\curveto(29.4541336,365.47016449)(29.53413352,365.51516445)(29.64413429,365.52516726)
\curveto(29.7541333,365.53516443)(29.86913318,365.54016442)(29.98913429,365.54016726)
\lineto(30.15413429,365.54016726)
\curveto(30.21413284,365.54016442)(30.26913278,365.53016443)(30.31913429,365.51016726)
\curveto(30.40913264,365.49016447)(30.47913257,365.45016451)(30.52913429,365.39016726)
\curveto(30.59913245,365.30016466)(30.64413241,365.19016477)(30.66413429,365.06016726)
\curveto(30.69413236,364.94016502)(30.73913231,364.83516513)(30.79913429,364.74516726)
\curveto(30.98913206,364.40516556)(31.2491318,364.13516583)(31.57913429,363.93516726)
\curveto(31.67913137,363.87516609)(31.78413127,363.82516614)(31.89413429,363.78516726)
\curveto(32.01413104,363.75516621)(32.13413092,363.72016624)(32.25413429,363.68016726)
\curveto(32.42413063,363.63016633)(32.62913042,363.61016635)(32.86913429,363.62016726)
\curveto(33.11912993,363.64016632)(33.31912973,363.67516629)(33.46913429,363.72516726)
\curveto(33.83912921,363.84516612)(34.12912892,364.00516596)(34.33913429,364.20516726)
\curveto(34.55912849,364.41516555)(34.73912831,364.69516527)(34.87913429,365.04516726)
\curveto(34.92912812,365.14516482)(34.95912809,365.25016471)(34.96913429,365.36016726)
\curveto(34.98912806,365.47016449)(35.01412804,365.58516438)(35.04413429,365.70516726)
\lineto(35.04413429,365.81016726)
\curveto(35.054128,365.85016411)(35.05912799,365.89016407)(35.05913429,365.93016726)
\curveto(35.06912798,365.960164)(35.06912798,365.99516397)(35.05913429,366.03516726)
\lineto(35.05913429,366.15516726)
\curveto(35.05912799,366.41516355)(35.02912802,366.6601633)(34.96913429,366.89016726)
\curveto(34.85912819,367.24016272)(34.70412835,367.53516243)(34.50413429,367.77516726)
\curveto(34.30412875,368.02516194)(34.04412901,368.22016174)(33.72413429,368.36016726)
\lineto(33.54413429,368.42016726)
\curveto(33.49412956,368.44016152)(33.43412962,368.4601615)(33.36413429,368.48016726)
\curveto(33.31412974,368.50016146)(33.2541298,368.51016145)(33.18413429,368.51016726)
\curveto(33.12412993,368.52016144)(33.05912999,368.53516143)(32.98913429,368.55516726)
\lineto(32.83913429,368.55516726)
\curveto(32.79913025,368.57516139)(32.74413031,368.58516138)(32.67413429,368.58516726)
\curveto(32.61413044,368.58516138)(32.55913049,368.57516139)(32.50913429,368.55516726)
\lineto(32.40413429,368.55516726)
\curveto(32.37413068,368.55516141)(32.33913071,368.55016141)(32.29913429,368.54016726)
\lineto(32.05913429,368.48016726)
\curveto(31.97913107,368.47016149)(31.89913115,368.45016151)(31.81913429,368.42016726)
\curveto(31.57913147,368.32016164)(31.3491317,368.18516178)(31.12913429,368.01516726)
\curveto(31.03913201,367.94516202)(30.9541321,367.87016209)(30.87413429,367.79016726)
\curveto(30.79413226,367.72016224)(30.69413236,367.6651623)(30.57413429,367.62516726)
\curveto(30.48413257,367.59516237)(30.34413271,367.58516238)(30.15413429,367.59516726)
\curveto(29.97413308,367.60516236)(29.8541332,367.63016233)(29.79413429,367.67016726)
\curveto(29.74413331,367.71016225)(29.70413335,367.77016219)(29.67413429,367.85016726)
\curveto(29.6541334,367.93016203)(29.6541334,368.01516195)(29.67413429,368.10516726)
\curveto(29.70413335,368.22516174)(29.72413333,368.34516162)(29.73413429,368.46516726)
\curveto(29.7541333,368.59516137)(29.77913327,368.72016124)(29.80913429,368.84016726)
\curveto(29.82913322,368.88016108)(29.83413322,368.91516105)(29.82413429,368.94516726)
\curveto(29.82413323,368.98516098)(29.83413322,369.03016093)(29.85413429,369.08016726)
\curveto(29.87413318,369.17016079)(29.88913316,369.2601607)(29.89913429,369.35016726)
\curveto(29.90913314,369.45016051)(29.92913312,369.54516042)(29.95913429,369.63516726)
\curveto(29.96913308,369.69516027)(29.97413308,369.75516021)(29.97413429,369.81516726)
\curveto(29.98413307,369.87516009)(29.99913305,369.93516003)(30.01913429,369.99516726)
\curveto(30.06913298,370.19515977)(30.10413295,370.40015956)(30.12413429,370.61016726)
\curveto(30.1541329,370.83015913)(30.19413286,371.04015892)(30.24413429,371.24016726)
\curveto(30.27413278,371.34015862)(30.29413276,371.44015852)(30.30413429,371.54016726)
\curveto(30.31413274,371.64015832)(30.32913272,371.74015822)(30.34913429,371.84016726)
\curveto(30.35913269,371.87015809)(30.36413269,371.91015805)(30.36413429,371.96016726)
\curveto(30.39413266,372.07015789)(30.41413264,372.17515779)(30.42413429,372.27516726)
\curveto(30.44413261,372.38515758)(30.46913258,372.49515747)(30.49913429,372.60516726)
\curveto(30.51913253,372.68515728)(30.53413252,372.75515721)(30.54413429,372.81516726)
\curveto(30.5541325,372.88515708)(30.57913247,372.94515702)(30.61913429,372.99516726)
\curveto(30.63913241,373.02515694)(30.66913238,373.04515692)(30.70913429,373.05516726)
\curveto(30.7491323,373.07515689)(30.79413226,373.09515687)(30.84413429,373.11516726)
\curveto(30.90413215,373.11515685)(30.94413211,373.12015684)(30.96413429,373.13016726)
}
}
{
\newrgbcolor{curcolor}{0 0 0}
\pscustom[linestyle=none,fillstyle=solid,fillcolor=curcolor]
{
\newpath
\moveto(44.80374367,367.80516726)
\lineto(44.80374367,367.55016726)
\curveto(44.81373596,367.47016249)(44.80873597,367.39516257)(44.78874367,367.32516726)
\lineto(44.78874367,367.08516726)
\lineto(44.78874367,366.92016726)
\curveto(44.76873601,366.82016314)(44.75873602,366.71516325)(44.75874367,366.60516726)
\curveto(44.75873602,366.50516346)(44.74873603,366.40516356)(44.72874367,366.30516726)
\lineto(44.72874367,366.15516726)
\curveto(44.69873608,366.01516395)(44.6787361,365.87516409)(44.66874367,365.73516726)
\curveto(44.65873612,365.60516436)(44.63373614,365.47516449)(44.59374367,365.34516726)
\curveto(44.5737362,365.2651647)(44.55373622,365.18016478)(44.53374367,365.09016726)
\lineto(44.47374367,364.85016726)
\lineto(44.35374367,364.55016726)
\curveto(44.32373645,364.4601655)(44.28873649,364.37016559)(44.24874367,364.28016726)
\curveto(44.14873663,364.0601659)(44.01373676,363.84516612)(43.84374367,363.63516726)
\curveto(43.68373709,363.42516654)(43.50873727,363.25516671)(43.31874367,363.12516726)
\curveto(43.26873751,363.08516688)(43.20873757,363.04516692)(43.13874367,363.00516726)
\curveto(43.0787377,362.97516699)(43.01873776,362.94016702)(42.95874367,362.90016726)
\curveto(42.8787379,362.85016711)(42.78373799,362.81016715)(42.67374367,362.78016726)
\curveto(42.56373821,362.75016721)(42.45873832,362.72016724)(42.35874367,362.69016726)
\curveto(42.24873853,362.65016731)(42.13873864,362.62516734)(42.02874367,362.61516726)
\curveto(41.91873886,362.60516736)(41.80373897,362.59016737)(41.68374367,362.57016726)
\curveto(41.64373913,362.5601674)(41.59873918,362.5601674)(41.54874367,362.57016726)
\curveto(41.50873927,362.57016739)(41.46873931,362.5651674)(41.42874367,362.55516726)
\curveto(41.38873939,362.54516742)(41.33373944,362.54016742)(41.26374367,362.54016726)
\curveto(41.19373958,362.54016742)(41.14373963,362.54516742)(41.11374367,362.55516726)
\curveto(41.06373971,362.57516739)(41.01873976,362.58016738)(40.97874367,362.57016726)
\curveto(40.93873984,362.5601674)(40.90373987,362.5601674)(40.87374367,362.57016726)
\lineto(40.78374367,362.57016726)
\curveto(40.72374005,362.59016737)(40.65874012,362.60516736)(40.58874367,362.61516726)
\curveto(40.52874025,362.61516735)(40.46374031,362.62016734)(40.39374367,362.63016726)
\curveto(40.22374055,362.68016728)(40.06374071,362.73016723)(39.91374367,362.78016726)
\curveto(39.76374101,362.83016713)(39.61874116,362.89516707)(39.47874367,362.97516726)
\curveto(39.42874135,363.01516695)(39.3737414,363.04516692)(39.31374367,363.06516726)
\curveto(39.26374151,363.09516687)(39.21374156,363.13016683)(39.16374367,363.17016726)
\curveto(38.92374185,363.35016661)(38.72374205,363.57016639)(38.56374367,363.83016726)
\curveto(38.40374237,364.09016587)(38.26374251,364.37516559)(38.14374367,364.68516726)
\curveto(38.08374269,364.82516514)(38.03874274,364.965165)(38.00874367,365.10516726)
\curveto(37.9787428,365.25516471)(37.94374283,365.41016455)(37.90374367,365.57016726)
\curveto(37.88374289,365.68016428)(37.86874291,365.79016417)(37.85874367,365.90016726)
\curveto(37.84874293,366.01016395)(37.83374294,366.12016384)(37.81374367,366.23016726)
\curveto(37.80374297,366.27016369)(37.79874298,366.31016365)(37.79874367,366.35016726)
\curveto(37.80874297,366.39016357)(37.80874297,366.43016353)(37.79874367,366.47016726)
\curveto(37.78874299,366.52016344)(37.78374299,366.57016339)(37.78374367,366.62016726)
\lineto(37.78374367,366.78516726)
\curveto(37.76374301,366.83516313)(37.75874302,366.88516308)(37.76874367,366.93516726)
\curveto(37.778743,366.99516297)(37.778743,367.05016291)(37.76874367,367.10016726)
\curveto(37.75874302,367.14016282)(37.75874302,367.18516278)(37.76874367,367.23516726)
\curveto(37.778743,367.28516268)(37.773743,367.33516263)(37.75374367,367.38516726)
\curveto(37.73374304,367.45516251)(37.72874305,367.53016243)(37.73874367,367.61016726)
\curveto(37.74874303,367.70016226)(37.75374302,367.78516218)(37.75374367,367.86516726)
\curveto(37.75374302,367.95516201)(37.74874303,368.05516191)(37.73874367,368.16516726)
\curveto(37.72874305,368.28516168)(37.73374304,368.38516158)(37.75374367,368.46516726)
\lineto(37.75374367,368.75016726)
\lineto(37.79874367,369.38016726)
\curveto(37.80874297,369.48016048)(37.81874296,369.57516039)(37.82874367,369.66516726)
\lineto(37.85874367,369.96516726)
\curveto(37.8787429,370.01515995)(37.88374289,370.0651599)(37.87374367,370.11516726)
\curveto(37.8737429,370.17515979)(37.88374289,370.23015973)(37.90374367,370.28016726)
\curveto(37.95374282,370.45015951)(37.99374278,370.61515935)(38.02374367,370.77516726)
\curveto(38.05374272,370.94515902)(38.10374267,371.10515886)(38.17374367,371.25516726)
\curveto(38.36374241,371.71515825)(38.58374219,372.09015787)(38.83374367,372.38016726)
\curveto(39.09374168,372.67015729)(39.45374132,372.91515705)(39.91374367,373.11516726)
\curveto(40.04374073,373.1651568)(40.1737406,373.20015676)(40.30374367,373.22016726)
\curveto(40.44374033,373.24015672)(40.58374019,373.2651567)(40.72374367,373.29516726)
\curveto(40.79373998,373.30515666)(40.85873992,373.31015665)(40.91874367,373.31016726)
\curveto(40.9787398,373.31015665)(41.04373973,373.31515665)(41.11374367,373.32516726)
\curveto(41.94373883,373.34515662)(42.61373816,373.19515677)(43.12374367,372.87516726)
\curveto(43.63373714,372.5651574)(44.01373676,372.12515784)(44.26374367,371.55516726)
\curveto(44.31373646,371.43515853)(44.35873642,371.31015865)(44.39874367,371.18016726)
\curveto(44.43873634,371.05015891)(44.48373629,370.91515905)(44.53374367,370.77516726)
\curveto(44.55373622,370.69515927)(44.56873621,370.61015935)(44.57874367,370.52016726)
\lineto(44.63874367,370.28016726)
\curveto(44.66873611,370.17015979)(44.68373609,370.0601599)(44.68374367,369.95016726)
\curveto(44.69373608,369.84016012)(44.70873607,369.73016023)(44.72874367,369.62016726)
\curveto(44.74873603,369.57016039)(44.75373602,369.52516044)(44.74374367,369.48516726)
\curveto(44.74373603,369.44516052)(44.74873603,369.40516056)(44.75874367,369.36516726)
\curveto(44.76873601,369.31516065)(44.76873601,369.2601607)(44.75874367,369.20016726)
\curveto(44.75873602,369.15016081)(44.76373601,369.10016086)(44.77374367,369.05016726)
\lineto(44.77374367,368.91516726)
\curveto(44.79373598,368.85516111)(44.79373598,368.78516118)(44.77374367,368.70516726)
\curveto(44.76373601,368.63516133)(44.76873601,368.57016139)(44.78874367,368.51016726)
\curveto(44.79873598,368.48016148)(44.80373597,368.44016152)(44.80374367,368.39016726)
\lineto(44.80374367,368.27016726)
\lineto(44.80374367,367.80516726)
\moveto(43.25874367,365.48016726)
\curveto(43.35873742,365.80016416)(43.41873736,366.1651638)(43.43874367,366.57516726)
\curveto(43.45873732,366.98516298)(43.46873731,367.39516257)(43.46874367,367.80516726)
\curveto(43.46873731,368.23516173)(43.45873732,368.65516131)(43.43874367,369.06516726)
\curveto(43.41873736,369.47516049)(43.3737374,369.8601601)(43.30374367,370.22016726)
\curveto(43.23373754,370.58015938)(43.12373765,370.90015906)(42.97374367,371.18016726)
\curveto(42.83373794,371.47015849)(42.63873814,371.70515826)(42.38874367,371.88516726)
\curveto(42.22873855,371.99515797)(42.04873873,372.07515789)(41.84874367,372.12516726)
\curveto(41.64873913,372.18515778)(41.40373937,372.21515775)(41.11374367,372.21516726)
\curveto(41.09373968,372.19515777)(41.05873972,372.18515778)(41.00874367,372.18516726)
\curveto(40.95873982,372.19515777)(40.91873986,372.19515777)(40.88874367,372.18516726)
\curveto(40.80873997,372.1651578)(40.73374004,372.14515782)(40.66374367,372.12516726)
\curveto(40.60374017,372.11515785)(40.53874024,372.09515787)(40.46874367,372.06516726)
\curveto(40.19874058,371.94515802)(39.9787408,371.77515819)(39.80874367,371.55516726)
\curveto(39.64874113,371.34515862)(39.51374126,371.10015886)(39.40374367,370.82016726)
\curveto(39.35374142,370.71015925)(39.31374146,370.59015937)(39.28374367,370.46016726)
\curveto(39.26374151,370.34015962)(39.23874154,370.21515975)(39.20874367,370.08516726)
\curveto(39.18874159,370.03515993)(39.1787416,369.98015998)(39.17874367,369.92016726)
\curveto(39.1787416,369.87016009)(39.1737416,369.82016014)(39.16374367,369.77016726)
\curveto(39.15374162,369.68016028)(39.14374163,369.58516038)(39.13374367,369.48516726)
\curveto(39.12374165,369.39516057)(39.11374166,369.30016066)(39.10374367,369.20016726)
\curveto(39.10374167,369.12016084)(39.09874168,369.03516093)(39.08874367,368.94516726)
\lineto(39.08874367,368.70516726)
\lineto(39.08874367,368.52516726)
\curveto(39.0787417,368.49516147)(39.0737417,368.4601615)(39.07374367,368.42016726)
\lineto(39.07374367,368.28516726)
\lineto(39.07374367,367.83516726)
\curveto(39.0737417,367.75516221)(39.06874171,367.67016229)(39.05874367,367.58016726)
\curveto(39.05874172,367.50016246)(39.06874171,367.42516254)(39.08874367,367.35516726)
\lineto(39.08874367,367.08516726)
\curveto(39.08874169,367.0651629)(39.08374169,367.03516293)(39.07374367,366.99516726)
\curveto(39.0737417,366.965163)(39.0787417,366.94016302)(39.08874367,366.92016726)
\curveto(39.09874168,366.82016314)(39.10374167,366.72016324)(39.10374367,366.62016726)
\curveto(39.11374166,366.53016343)(39.12374165,366.43016353)(39.13374367,366.32016726)
\curveto(39.16374161,366.20016376)(39.1787416,366.07516389)(39.17874367,365.94516726)
\curveto(39.18874159,365.82516414)(39.21374156,365.71016425)(39.25374367,365.60016726)
\curveto(39.33374144,365.30016466)(39.41874136,365.03516493)(39.50874367,364.80516726)
\curveto(39.60874117,364.57516539)(39.75374102,364.3601656)(39.94374367,364.16016726)
\curveto(40.15374062,363.960166)(40.41874036,363.81016615)(40.73874367,363.71016726)
\curveto(40.77874,363.69016627)(40.81373996,363.68016628)(40.84374367,363.68016726)
\curveto(40.88373989,363.69016627)(40.92873985,363.68516628)(40.97874367,363.66516726)
\curveto(41.01873976,363.65516631)(41.08873969,363.64516632)(41.18874367,363.63516726)
\curveto(41.29873948,363.62516634)(41.38373939,363.63016633)(41.44374367,363.65016726)
\curveto(41.51373926,363.67016629)(41.58373919,363.68016628)(41.65374367,363.68016726)
\curveto(41.72373905,363.69016627)(41.78873899,363.70516626)(41.84874367,363.72516726)
\curveto(42.04873873,363.78516618)(42.22873855,363.87016609)(42.38874367,363.98016726)
\curveto(42.41873836,364.00016596)(42.44373833,364.02016594)(42.46374367,364.04016726)
\lineto(42.52374367,364.10016726)
\curveto(42.56373821,364.12016584)(42.61373816,364.1601658)(42.67374367,364.22016726)
\curveto(42.773738,364.3601656)(42.85873792,364.49016547)(42.92874367,364.61016726)
\curveto(42.99873778,364.73016523)(43.06873771,364.87516509)(43.13874367,365.04516726)
\curveto(43.16873761,365.11516485)(43.18873759,365.18516478)(43.19874367,365.25516726)
\curveto(43.21873756,365.32516464)(43.23873754,365.40016456)(43.25874367,365.48016726)
}
}
{
\newrgbcolor{curcolor}{0 0 0}
\pscustom[linestyle=none,fillstyle=solid,fillcolor=curcolor]
{
\newpath
\moveto(24.21952492,293.11807497)
\curveto(25.84951948,293.14806432)(26.89951843,292.59306488)(27.36952492,291.45307497)
\curveto(27.46951786,291.22306625)(27.53451779,290.93306654)(27.56452492,290.58307497)
\curveto(27.60451772,290.24306723)(27.57951775,289.93306754)(27.48952492,289.65307497)
\curveto(27.39951793,289.39306808)(27.27951805,289.1680683)(27.12952492,288.97807497)
\curveto(27.10951822,288.93806853)(27.08451824,288.90306857)(27.05452492,288.87307497)
\curveto(27.0245183,288.85306862)(26.99951833,288.82806864)(26.97952492,288.79807497)
\lineto(26.88952492,288.67807497)
\curveto(26.85951847,288.64806882)(26.8245185,288.62306885)(26.78452492,288.60307497)
\curveto(26.73451859,288.55306892)(26.67951865,288.50806896)(26.61952492,288.46807497)
\curveto(26.56951876,288.42806904)(26.5245188,288.37806909)(26.48452492,288.31807497)
\curveto(26.44451888,288.27806919)(26.4295189,288.22806924)(26.43952492,288.16807497)
\curveto(26.44951888,288.11806935)(26.47951885,288.0730694)(26.52952492,288.03307497)
\curveto(26.57951875,287.99306948)(26.63451869,287.95306952)(26.69452492,287.91307497)
\curveto(26.76451856,287.88306959)(26.8295185,287.85306962)(26.88952492,287.82307497)
\curveto(26.94951838,287.79306968)(26.99951833,287.76306971)(27.03952492,287.73307497)
\curveto(27.35951797,287.51306996)(27.61451771,287.20307027)(27.80452492,286.80307497)
\curveto(27.84451748,286.71307076)(27.87451745,286.61807085)(27.89452492,286.51807497)
\curveto(27.9245174,286.42807104)(27.94951738,286.33807113)(27.96952492,286.24807497)
\curveto(27.97951735,286.19807127)(27.98451734,286.14807132)(27.98452492,286.09807497)
\curveto(27.99451733,286.05807141)(28.00451732,286.01307146)(28.01452492,285.96307497)
\curveto(28.0245173,285.91307156)(28.0245173,285.86307161)(28.01452492,285.81307497)
\curveto(28.00451732,285.76307171)(28.00951732,285.71307176)(28.02952492,285.66307497)
\curveto(28.03951729,285.61307186)(28.04451728,285.55307192)(28.04452492,285.48307497)
\curveto(28.04451728,285.41307206)(28.03451729,285.35307212)(28.01452492,285.30307497)
\lineto(28.01452492,285.07807497)
\lineto(27.95452492,284.83807497)
\curveto(27.94451738,284.7680727)(27.9295174,284.69807277)(27.90952492,284.62807497)
\curveto(27.87951745,284.53807293)(27.84951748,284.45307302)(27.81952492,284.37307497)
\curveto(27.79951753,284.29307318)(27.76951756,284.21307326)(27.72952492,284.13307497)
\curveto(27.70951762,284.0730734)(27.67951765,284.01307346)(27.63952492,283.95307497)
\curveto(27.60951772,283.90307357)(27.57451775,283.85307362)(27.53452492,283.80307497)
\curveto(27.33451799,283.49307398)(27.08451824,283.23307424)(26.78452492,283.02307497)
\curveto(26.48451884,282.82307465)(26.13951919,282.65807481)(25.74952492,282.52807497)
\curveto(25.6295197,282.48807498)(25.49951983,282.46307501)(25.35952492,282.45307497)
\curveto(25.2295201,282.43307504)(25.09452023,282.40807506)(24.95452492,282.37807497)
\curveto(24.88452044,282.3680751)(24.81452051,282.36307511)(24.74452492,282.36307497)
\curveto(24.68452064,282.36307511)(24.61952071,282.35807511)(24.54952492,282.34807497)
\curveto(24.50952082,282.33807513)(24.44952088,282.33307514)(24.36952492,282.33307497)
\curveto(24.29952103,282.33307514)(24.24952108,282.33807513)(24.21952492,282.34807497)
\curveto(24.16952116,282.35807511)(24.1245212,282.36307511)(24.08452492,282.36307497)
\lineto(23.96452492,282.36307497)
\curveto(23.86452146,282.38307509)(23.76452156,282.39807507)(23.66452492,282.40807497)
\curveto(23.56452176,282.41807505)(23.46952186,282.43307504)(23.37952492,282.45307497)
\curveto(23.26952206,282.48307499)(23.15952217,282.50807496)(23.04952492,282.52807497)
\curveto(22.94952238,282.55807491)(22.84452248,282.59807487)(22.73452492,282.64807497)
\curveto(22.36452296,282.80807466)(22.04952328,283.00807446)(21.78952492,283.24807497)
\curveto(21.5295238,283.49807397)(21.31952401,283.80807366)(21.15952492,284.17807497)
\curveto(21.11952421,284.2680732)(21.08452424,284.36307311)(21.05452492,284.46307497)
\curveto(21.0245243,284.56307291)(20.99452433,284.6680728)(20.96452492,284.77807497)
\curveto(20.94452438,284.82807264)(20.93452439,284.87807259)(20.93452492,284.92807497)
\curveto(20.93452439,284.98807248)(20.9245244,285.04807242)(20.90452492,285.10807497)
\curveto(20.88452444,285.1680723)(20.87452445,285.24807222)(20.87452492,285.34807497)
\curveto(20.87452445,285.44807202)(20.88952444,285.52307195)(20.91952492,285.57307497)
\curveto(20.9295244,285.60307187)(20.94452438,285.62807184)(20.96452492,285.64807497)
\lineto(21.02452492,285.70807497)
\curveto(21.06452426,285.72807174)(21.1245242,285.74307173)(21.20452492,285.75307497)
\curveto(21.29452403,285.76307171)(21.38452394,285.7680717)(21.47452492,285.76807497)
\curveto(21.56452376,285.7680717)(21.64952368,285.76307171)(21.72952492,285.75307497)
\curveto(21.81952351,285.74307173)(21.88452344,285.73307174)(21.92452492,285.72307497)
\curveto(21.94452338,285.70307177)(21.96452336,285.68807178)(21.98452492,285.67807497)
\curveto(22.00452332,285.67807179)(22.0245233,285.6680718)(22.04452492,285.64807497)
\curveto(22.11452321,285.55807191)(22.15452317,285.44307203)(22.16452492,285.30307497)
\curveto(22.18452314,285.16307231)(22.21452311,285.03807243)(22.25452492,284.92807497)
\lineto(22.40452492,284.56807497)
\curveto(22.45452287,284.45807301)(22.51952281,284.35307312)(22.59952492,284.25307497)
\curveto(22.61952271,284.22307325)(22.63952269,284.19807327)(22.65952492,284.17807497)
\curveto(22.68952264,284.15807331)(22.71452261,284.13307334)(22.73452492,284.10307497)
\curveto(22.77452255,284.04307343)(22.80952252,283.99807347)(22.83952492,283.96807497)
\curveto(22.87952245,283.93807353)(22.91452241,283.90807356)(22.94452492,283.87807497)
\curveto(22.98452234,283.84807362)(23.0295223,283.81807365)(23.07952492,283.78807497)
\curveto(23.16952216,283.72807374)(23.26452206,283.67807379)(23.36452492,283.63807497)
\lineto(23.69452492,283.51807497)
\curveto(23.84452148,283.468074)(24.04452128,283.43807403)(24.29452492,283.42807497)
\curveto(24.54452078,283.41807405)(24.75452057,283.43807403)(24.92452492,283.48807497)
\curveto(25.00452032,283.50807396)(25.07452025,283.52307395)(25.13452492,283.53307497)
\lineto(25.34452492,283.59307497)
\curveto(25.6245197,283.71307376)(25.86451946,283.86307361)(26.06452492,284.04307497)
\curveto(26.27451905,284.22307325)(26.43951889,284.45307302)(26.55952492,284.73307497)
\curveto(26.58951874,284.80307267)(26.60951872,284.8730726)(26.61952492,284.94307497)
\lineto(26.67952492,285.18307497)
\curveto(26.71951861,285.32307215)(26.7295186,285.48307199)(26.70952492,285.66307497)
\curveto(26.68951864,285.85307162)(26.65951867,286.00307147)(26.61952492,286.11307497)
\curveto(26.48951884,286.49307098)(26.30451902,286.78307069)(26.06452492,286.98307497)
\curveto(25.83451949,287.18307029)(25.5245198,287.34307013)(25.13452492,287.46307497)
\curveto(25.0245203,287.49306998)(24.90452042,287.51306996)(24.77452492,287.52307497)
\curveto(24.65452067,287.53306994)(24.5295208,287.53806993)(24.39952492,287.53807497)
\curveto(24.23952109,287.53806993)(24.09952123,287.54306993)(23.97952492,287.55307497)
\curveto(23.85952147,287.56306991)(23.77452155,287.62306985)(23.72452492,287.73307497)
\curveto(23.70452162,287.76306971)(23.69452163,287.79806967)(23.69452492,287.83807497)
\lineto(23.69452492,287.97307497)
\curveto(23.68452164,288.0730694)(23.68452164,288.1680693)(23.69452492,288.25807497)
\curveto(23.71452161,288.34806912)(23.75452157,288.41306906)(23.81452492,288.45307497)
\curveto(23.85452147,288.48306899)(23.89452143,288.50306897)(23.93452492,288.51307497)
\curveto(23.98452134,288.52306895)(24.03952129,288.53306894)(24.09952492,288.54307497)
\curveto(24.11952121,288.55306892)(24.14452118,288.55306892)(24.17452492,288.54307497)
\curveto(24.20452112,288.54306893)(24.2295211,288.54806892)(24.24952492,288.55807497)
\lineto(24.38452492,288.55807497)
\curveto(24.49452083,288.57806889)(24.59452073,288.58806888)(24.68452492,288.58807497)
\curveto(24.78452054,288.59806887)(24.87952045,288.61806885)(24.96952492,288.64807497)
\curveto(25.28952004,288.75806871)(25.54451978,288.90306857)(25.73452492,289.08307497)
\curveto(25.9245194,289.26306821)(26.07451925,289.51306796)(26.18452492,289.83307497)
\curveto(26.21451911,289.93306754)(26.23451909,290.05806741)(26.24452492,290.20807497)
\curveto(26.26451906,290.3680671)(26.25951907,290.51306696)(26.22952492,290.64307497)
\curveto(26.20951912,290.71306676)(26.18951914,290.77806669)(26.16952492,290.83807497)
\curveto(26.15951917,290.90806656)(26.13951919,290.9730665)(26.10952492,291.03307497)
\curveto(26.00951932,291.2730662)(25.86451946,291.46306601)(25.67452492,291.60307497)
\curveto(25.48451984,291.74306573)(25.25952007,291.85306562)(24.99952492,291.93307497)
\curveto(24.93952039,291.95306552)(24.87952045,291.96306551)(24.81952492,291.96307497)
\curveto(24.75952057,291.96306551)(24.69452063,291.9730655)(24.62452492,291.99307497)
\curveto(24.54452078,292.01306546)(24.44952088,292.02306545)(24.33952492,292.02307497)
\curveto(24.2295211,292.02306545)(24.13452119,292.01306546)(24.05452492,291.99307497)
\curveto(24.00452132,291.9730655)(23.95452137,291.96306551)(23.90452492,291.96307497)
\curveto(23.86452146,291.96306551)(23.81952151,291.95306552)(23.76952492,291.93307497)
\curveto(23.58952174,291.88306559)(23.41952191,291.80806566)(23.25952492,291.70807497)
\curveto(23.10952222,291.61806585)(22.97952235,291.50306597)(22.86952492,291.36307497)
\curveto(22.77952255,291.24306623)(22.69952263,291.11306636)(22.62952492,290.97307497)
\curveto(22.55952277,290.83306664)(22.49452283,290.67806679)(22.43452492,290.50807497)
\curveto(22.40452292,290.39806707)(22.38452294,290.27806719)(22.37452492,290.14807497)
\curveto(22.36452296,290.02806744)(22.329523,289.92806754)(22.26952492,289.84807497)
\curveto(22.24952308,289.80806766)(22.18952314,289.7680677)(22.08952492,289.72807497)
\curveto(22.04952328,289.71806775)(21.98952334,289.70806776)(21.90952492,289.69807497)
\lineto(21.65452492,289.69807497)
\curveto(21.56452376,289.70806776)(21.47952385,289.71806775)(21.39952492,289.72807497)
\curveto(21.329524,289.73806773)(21.27952405,289.75306772)(21.24952492,289.77307497)
\curveto(21.20952412,289.80306767)(21.17452415,289.85806761)(21.14452492,289.93807497)
\curveto(21.11452421,290.01806745)(21.10952422,290.10306737)(21.12952492,290.19307497)
\curveto(21.13952419,290.24306723)(21.14452418,290.29306718)(21.14452492,290.34307497)
\lineto(21.17452492,290.52307497)
\curveto(21.20452412,290.62306685)(21.2295241,290.72306675)(21.24952492,290.82307497)
\curveto(21.27952405,290.92306655)(21.31452401,291.01306646)(21.35452492,291.09307497)
\curveto(21.40452392,291.20306627)(21.44952388,291.30806616)(21.48952492,291.40807497)
\curveto(21.5295238,291.51806595)(21.57952375,291.62306585)(21.63952492,291.72307497)
\curveto(21.96952336,292.26306521)(22.43952289,292.65806481)(23.04952492,292.90807497)
\curveto(23.16952216,292.95806451)(23.29452203,292.99306448)(23.42452492,293.01307497)
\curveto(23.56452176,293.03306444)(23.70452162,293.05806441)(23.84452492,293.08807497)
\curveto(23.90452142,293.09806437)(23.96452136,293.10306437)(24.02452492,293.10307497)
\curveto(24.09452123,293.10306437)(24.15952117,293.10806436)(24.21952492,293.11807497)
}
}
{
\newrgbcolor{curcolor}{0 0 0}
\pscustom[linestyle=none,fillstyle=solid,fillcolor=curcolor]
{
\newpath
\moveto(36.45413429,287.59807497)
\lineto(36.45413429,287.34307497)
\curveto(36.46412659,287.26307021)(36.45912659,287.18807028)(36.43913429,287.11807497)
\lineto(36.43913429,286.87807497)
\lineto(36.43913429,286.71307497)
\curveto(36.41912663,286.61307086)(36.40912664,286.50807096)(36.40913429,286.39807497)
\curveto(36.40912664,286.29807117)(36.39912665,286.19807127)(36.37913429,286.09807497)
\lineto(36.37913429,285.94807497)
\curveto(36.3491267,285.80807166)(36.32912672,285.6680718)(36.31913429,285.52807497)
\curveto(36.30912674,285.39807207)(36.28412677,285.2680722)(36.24413429,285.13807497)
\curveto(36.22412683,285.05807241)(36.20412685,284.9730725)(36.18413429,284.88307497)
\lineto(36.12413429,284.64307497)
\lineto(36.00413429,284.34307497)
\curveto(35.97412708,284.25307322)(35.93912711,284.16307331)(35.89913429,284.07307497)
\curveto(35.79912725,283.85307362)(35.66412739,283.63807383)(35.49413429,283.42807497)
\curveto(35.33412772,283.21807425)(35.15912789,283.04807442)(34.96913429,282.91807497)
\curveto(34.91912813,282.87807459)(34.85912819,282.83807463)(34.78913429,282.79807497)
\curveto(34.72912832,282.7680747)(34.66912838,282.73307474)(34.60913429,282.69307497)
\curveto(34.52912852,282.64307483)(34.43412862,282.60307487)(34.32413429,282.57307497)
\curveto(34.21412884,282.54307493)(34.10912894,282.51307496)(34.00913429,282.48307497)
\curveto(33.89912915,282.44307503)(33.78912926,282.41807505)(33.67913429,282.40807497)
\curveto(33.56912948,282.39807507)(33.4541296,282.38307509)(33.33413429,282.36307497)
\curveto(33.29412976,282.35307512)(33.2491298,282.35307512)(33.19913429,282.36307497)
\curveto(33.15912989,282.36307511)(33.11912993,282.35807511)(33.07913429,282.34807497)
\curveto(33.03913001,282.33807513)(32.98413007,282.33307514)(32.91413429,282.33307497)
\curveto(32.84413021,282.33307514)(32.79413026,282.33807513)(32.76413429,282.34807497)
\curveto(32.71413034,282.3680751)(32.66913038,282.3730751)(32.62913429,282.36307497)
\curveto(32.58913046,282.35307512)(32.5541305,282.35307512)(32.52413429,282.36307497)
\lineto(32.43413429,282.36307497)
\curveto(32.37413068,282.38307509)(32.30913074,282.39807507)(32.23913429,282.40807497)
\curveto(32.17913087,282.40807506)(32.11413094,282.41307506)(32.04413429,282.42307497)
\curveto(31.87413118,282.473075)(31.71413134,282.52307495)(31.56413429,282.57307497)
\curveto(31.41413164,282.62307485)(31.26913178,282.68807478)(31.12913429,282.76807497)
\curveto(31.07913197,282.80807466)(31.02413203,282.83807463)(30.96413429,282.85807497)
\curveto(30.91413214,282.88807458)(30.86413219,282.92307455)(30.81413429,282.96307497)
\curveto(30.57413248,283.14307433)(30.37413268,283.36307411)(30.21413429,283.62307497)
\curveto(30.054133,283.88307359)(29.91413314,284.1680733)(29.79413429,284.47807497)
\curveto(29.73413332,284.61807285)(29.68913336,284.75807271)(29.65913429,284.89807497)
\curveto(29.62913342,285.04807242)(29.59413346,285.20307227)(29.55413429,285.36307497)
\curveto(29.53413352,285.473072)(29.51913353,285.58307189)(29.50913429,285.69307497)
\curveto(29.49913355,285.80307167)(29.48413357,285.91307156)(29.46413429,286.02307497)
\curveto(29.4541336,286.06307141)(29.4491336,286.10307137)(29.44913429,286.14307497)
\curveto(29.45913359,286.18307129)(29.45913359,286.22307125)(29.44913429,286.26307497)
\curveto(29.43913361,286.31307116)(29.43413362,286.36307111)(29.43413429,286.41307497)
\lineto(29.43413429,286.57807497)
\curveto(29.41413364,286.62807084)(29.40913364,286.67807079)(29.41913429,286.72807497)
\curveto(29.42913362,286.78807068)(29.42913362,286.84307063)(29.41913429,286.89307497)
\curveto(29.40913364,286.93307054)(29.40913364,286.97807049)(29.41913429,287.02807497)
\curveto(29.42913362,287.07807039)(29.42413363,287.12807034)(29.40413429,287.17807497)
\curveto(29.38413367,287.24807022)(29.37913367,287.32307015)(29.38913429,287.40307497)
\curveto(29.39913365,287.49306998)(29.40413365,287.57806989)(29.40413429,287.65807497)
\curveto(29.40413365,287.74806972)(29.39913365,287.84806962)(29.38913429,287.95807497)
\curveto(29.37913367,288.07806939)(29.38413367,288.17806929)(29.40413429,288.25807497)
\lineto(29.40413429,288.54307497)
\lineto(29.44913429,289.17307497)
\curveto(29.45913359,289.2730682)(29.46913358,289.3680681)(29.47913429,289.45807497)
\lineto(29.50913429,289.75807497)
\curveto(29.52913352,289.80806766)(29.53413352,289.85806761)(29.52413429,289.90807497)
\curveto(29.52413353,289.9680675)(29.53413352,290.02306745)(29.55413429,290.07307497)
\curveto(29.60413345,290.24306723)(29.64413341,290.40806706)(29.67413429,290.56807497)
\curveto(29.70413335,290.73806673)(29.7541333,290.89806657)(29.82413429,291.04807497)
\curveto(30.01413304,291.50806596)(30.23413282,291.88306559)(30.48413429,292.17307497)
\curveto(30.74413231,292.46306501)(31.10413195,292.70806476)(31.56413429,292.90807497)
\curveto(31.69413136,292.95806451)(31.82413123,292.99306448)(31.95413429,293.01307497)
\curveto(32.09413096,293.03306444)(32.23413082,293.05806441)(32.37413429,293.08807497)
\curveto(32.44413061,293.09806437)(32.50913054,293.10306437)(32.56913429,293.10307497)
\curveto(32.62913042,293.10306437)(32.69413036,293.10806436)(32.76413429,293.11807497)
\curveto(33.59412946,293.13806433)(34.26412879,292.98806448)(34.77413429,292.66807497)
\curveto(35.28412777,292.35806511)(35.66412739,291.91806555)(35.91413429,291.34807497)
\curveto(35.96412709,291.22806624)(36.00912704,291.10306637)(36.04913429,290.97307497)
\curveto(36.08912696,290.84306663)(36.13412692,290.70806676)(36.18413429,290.56807497)
\curveto(36.20412685,290.48806698)(36.21912683,290.40306707)(36.22913429,290.31307497)
\lineto(36.28913429,290.07307497)
\curveto(36.31912673,289.96306751)(36.33412672,289.85306762)(36.33413429,289.74307497)
\curveto(36.34412671,289.63306784)(36.35912669,289.52306795)(36.37913429,289.41307497)
\curveto(36.39912665,289.36306811)(36.40412665,289.31806815)(36.39413429,289.27807497)
\curveto(36.39412666,289.23806823)(36.39912665,289.19806827)(36.40913429,289.15807497)
\curveto(36.41912663,289.10806836)(36.41912663,289.05306842)(36.40913429,288.99307497)
\curveto(36.40912664,288.94306853)(36.41412664,288.89306858)(36.42413429,288.84307497)
\lineto(36.42413429,288.70807497)
\curveto(36.44412661,288.64806882)(36.44412661,288.57806889)(36.42413429,288.49807497)
\curveto(36.41412664,288.42806904)(36.41912663,288.36306911)(36.43913429,288.30307497)
\curveto(36.4491266,288.2730692)(36.4541266,288.23306924)(36.45413429,288.18307497)
\lineto(36.45413429,288.06307497)
\lineto(36.45413429,287.59807497)
\moveto(34.90913429,285.27307497)
\curveto(35.00912804,285.59307188)(35.06912798,285.95807151)(35.08913429,286.36807497)
\curveto(35.10912794,286.77807069)(35.11912793,287.18807028)(35.11913429,287.59807497)
\curveto(35.11912793,288.02806944)(35.10912794,288.44806902)(35.08913429,288.85807497)
\curveto(35.06912798,289.2680682)(35.02412803,289.65306782)(34.95413429,290.01307497)
\curveto(34.88412817,290.3730671)(34.77412828,290.69306678)(34.62413429,290.97307497)
\curveto(34.48412857,291.26306621)(34.28912876,291.49806597)(34.03913429,291.67807497)
\curveto(33.87912917,291.78806568)(33.69912935,291.8680656)(33.49913429,291.91807497)
\curveto(33.29912975,291.97806549)(33.05413,292.00806546)(32.76413429,292.00807497)
\curveto(32.74413031,291.98806548)(32.70913034,291.97806549)(32.65913429,291.97807497)
\curveto(32.60913044,291.98806548)(32.56913048,291.98806548)(32.53913429,291.97807497)
\curveto(32.45913059,291.95806551)(32.38413067,291.93806553)(32.31413429,291.91807497)
\curveto(32.2541308,291.90806556)(32.18913086,291.88806558)(32.11913429,291.85807497)
\curveto(31.8491312,291.73806573)(31.62913142,291.5680659)(31.45913429,291.34807497)
\curveto(31.29913175,291.13806633)(31.16413189,290.89306658)(31.05413429,290.61307497)
\curveto(31.00413205,290.50306697)(30.96413209,290.38306709)(30.93413429,290.25307497)
\curveto(30.91413214,290.13306734)(30.88913216,290.00806746)(30.85913429,289.87807497)
\curveto(30.83913221,289.82806764)(30.82913222,289.7730677)(30.82913429,289.71307497)
\curveto(30.82913222,289.66306781)(30.82413223,289.61306786)(30.81413429,289.56307497)
\curveto(30.80413225,289.473068)(30.79413226,289.37806809)(30.78413429,289.27807497)
\curveto(30.77413228,289.18806828)(30.76413229,289.09306838)(30.75413429,288.99307497)
\curveto(30.7541323,288.91306856)(30.7491323,288.82806864)(30.73913429,288.73807497)
\lineto(30.73913429,288.49807497)
\lineto(30.73913429,288.31807497)
\curveto(30.72913232,288.28806918)(30.72413233,288.25306922)(30.72413429,288.21307497)
\lineto(30.72413429,288.07807497)
\lineto(30.72413429,287.62807497)
\curveto(30.72413233,287.54806992)(30.71913233,287.46307001)(30.70913429,287.37307497)
\curveto(30.70913234,287.29307018)(30.71913233,287.21807025)(30.73913429,287.14807497)
\lineto(30.73913429,286.87807497)
\curveto(30.73913231,286.85807061)(30.73413232,286.82807064)(30.72413429,286.78807497)
\curveto(30.72413233,286.75807071)(30.72913232,286.73307074)(30.73913429,286.71307497)
\curveto(30.7491323,286.61307086)(30.7541323,286.51307096)(30.75413429,286.41307497)
\curveto(30.76413229,286.32307115)(30.77413228,286.22307125)(30.78413429,286.11307497)
\curveto(30.81413224,285.99307148)(30.82913222,285.8680716)(30.82913429,285.73807497)
\curveto(30.83913221,285.61807185)(30.86413219,285.50307197)(30.90413429,285.39307497)
\curveto(30.98413207,285.09307238)(31.06913198,284.82807264)(31.15913429,284.59807497)
\curveto(31.25913179,284.3680731)(31.40413165,284.15307332)(31.59413429,283.95307497)
\curveto(31.80413125,283.75307372)(32.06913098,283.60307387)(32.38913429,283.50307497)
\curveto(32.42913062,283.48307399)(32.46413059,283.473074)(32.49413429,283.47307497)
\curveto(32.53413052,283.48307399)(32.57913047,283.47807399)(32.62913429,283.45807497)
\curveto(32.66913038,283.44807402)(32.73913031,283.43807403)(32.83913429,283.42807497)
\curveto(32.9491301,283.41807405)(33.03413002,283.42307405)(33.09413429,283.44307497)
\curveto(33.16412989,283.46307401)(33.23412982,283.473074)(33.30413429,283.47307497)
\curveto(33.37412968,283.48307399)(33.43912961,283.49807397)(33.49913429,283.51807497)
\curveto(33.69912935,283.57807389)(33.87912917,283.66307381)(34.03913429,283.77307497)
\curveto(34.06912898,283.79307368)(34.09412896,283.81307366)(34.11413429,283.83307497)
\lineto(34.17413429,283.89307497)
\curveto(34.21412884,283.91307356)(34.26412879,283.95307352)(34.32413429,284.01307497)
\curveto(34.42412863,284.15307332)(34.50912854,284.28307319)(34.57913429,284.40307497)
\curveto(34.6491284,284.52307295)(34.71912833,284.6680728)(34.78913429,284.83807497)
\curveto(34.81912823,284.90807256)(34.83912821,284.97807249)(34.84913429,285.04807497)
\curveto(34.86912818,285.11807235)(34.88912816,285.19307228)(34.90913429,285.27307497)
}
}
{
\newrgbcolor{curcolor}{0 0 0}
\pscustom[linestyle=none,fillstyle=solid,fillcolor=curcolor]
{
\newpath
\moveto(44.80374367,287.59807497)
\lineto(44.80374367,287.34307497)
\curveto(44.81373596,287.26307021)(44.80873597,287.18807028)(44.78874367,287.11807497)
\lineto(44.78874367,286.87807497)
\lineto(44.78874367,286.71307497)
\curveto(44.76873601,286.61307086)(44.75873602,286.50807096)(44.75874367,286.39807497)
\curveto(44.75873602,286.29807117)(44.74873603,286.19807127)(44.72874367,286.09807497)
\lineto(44.72874367,285.94807497)
\curveto(44.69873608,285.80807166)(44.6787361,285.6680718)(44.66874367,285.52807497)
\curveto(44.65873612,285.39807207)(44.63373614,285.2680722)(44.59374367,285.13807497)
\curveto(44.5737362,285.05807241)(44.55373622,284.9730725)(44.53374367,284.88307497)
\lineto(44.47374367,284.64307497)
\lineto(44.35374367,284.34307497)
\curveto(44.32373645,284.25307322)(44.28873649,284.16307331)(44.24874367,284.07307497)
\curveto(44.14873663,283.85307362)(44.01373676,283.63807383)(43.84374367,283.42807497)
\curveto(43.68373709,283.21807425)(43.50873727,283.04807442)(43.31874367,282.91807497)
\curveto(43.26873751,282.87807459)(43.20873757,282.83807463)(43.13874367,282.79807497)
\curveto(43.0787377,282.7680747)(43.01873776,282.73307474)(42.95874367,282.69307497)
\curveto(42.8787379,282.64307483)(42.78373799,282.60307487)(42.67374367,282.57307497)
\curveto(42.56373821,282.54307493)(42.45873832,282.51307496)(42.35874367,282.48307497)
\curveto(42.24873853,282.44307503)(42.13873864,282.41807505)(42.02874367,282.40807497)
\curveto(41.91873886,282.39807507)(41.80373897,282.38307509)(41.68374367,282.36307497)
\curveto(41.64373913,282.35307512)(41.59873918,282.35307512)(41.54874367,282.36307497)
\curveto(41.50873927,282.36307511)(41.46873931,282.35807511)(41.42874367,282.34807497)
\curveto(41.38873939,282.33807513)(41.33373944,282.33307514)(41.26374367,282.33307497)
\curveto(41.19373958,282.33307514)(41.14373963,282.33807513)(41.11374367,282.34807497)
\curveto(41.06373971,282.3680751)(41.01873976,282.3730751)(40.97874367,282.36307497)
\curveto(40.93873984,282.35307512)(40.90373987,282.35307512)(40.87374367,282.36307497)
\lineto(40.78374367,282.36307497)
\curveto(40.72374005,282.38307509)(40.65874012,282.39807507)(40.58874367,282.40807497)
\curveto(40.52874025,282.40807506)(40.46374031,282.41307506)(40.39374367,282.42307497)
\curveto(40.22374055,282.473075)(40.06374071,282.52307495)(39.91374367,282.57307497)
\curveto(39.76374101,282.62307485)(39.61874116,282.68807478)(39.47874367,282.76807497)
\curveto(39.42874135,282.80807466)(39.3737414,282.83807463)(39.31374367,282.85807497)
\curveto(39.26374151,282.88807458)(39.21374156,282.92307455)(39.16374367,282.96307497)
\curveto(38.92374185,283.14307433)(38.72374205,283.36307411)(38.56374367,283.62307497)
\curveto(38.40374237,283.88307359)(38.26374251,284.1680733)(38.14374367,284.47807497)
\curveto(38.08374269,284.61807285)(38.03874274,284.75807271)(38.00874367,284.89807497)
\curveto(37.9787428,285.04807242)(37.94374283,285.20307227)(37.90374367,285.36307497)
\curveto(37.88374289,285.473072)(37.86874291,285.58307189)(37.85874367,285.69307497)
\curveto(37.84874293,285.80307167)(37.83374294,285.91307156)(37.81374367,286.02307497)
\curveto(37.80374297,286.06307141)(37.79874298,286.10307137)(37.79874367,286.14307497)
\curveto(37.80874297,286.18307129)(37.80874297,286.22307125)(37.79874367,286.26307497)
\curveto(37.78874299,286.31307116)(37.78374299,286.36307111)(37.78374367,286.41307497)
\lineto(37.78374367,286.57807497)
\curveto(37.76374301,286.62807084)(37.75874302,286.67807079)(37.76874367,286.72807497)
\curveto(37.778743,286.78807068)(37.778743,286.84307063)(37.76874367,286.89307497)
\curveto(37.75874302,286.93307054)(37.75874302,286.97807049)(37.76874367,287.02807497)
\curveto(37.778743,287.07807039)(37.773743,287.12807034)(37.75374367,287.17807497)
\curveto(37.73374304,287.24807022)(37.72874305,287.32307015)(37.73874367,287.40307497)
\curveto(37.74874303,287.49306998)(37.75374302,287.57806989)(37.75374367,287.65807497)
\curveto(37.75374302,287.74806972)(37.74874303,287.84806962)(37.73874367,287.95807497)
\curveto(37.72874305,288.07806939)(37.73374304,288.17806929)(37.75374367,288.25807497)
\lineto(37.75374367,288.54307497)
\lineto(37.79874367,289.17307497)
\curveto(37.80874297,289.2730682)(37.81874296,289.3680681)(37.82874367,289.45807497)
\lineto(37.85874367,289.75807497)
\curveto(37.8787429,289.80806766)(37.88374289,289.85806761)(37.87374367,289.90807497)
\curveto(37.8737429,289.9680675)(37.88374289,290.02306745)(37.90374367,290.07307497)
\curveto(37.95374282,290.24306723)(37.99374278,290.40806706)(38.02374367,290.56807497)
\curveto(38.05374272,290.73806673)(38.10374267,290.89806657)(38.17374367,291.04807497)
\curveto(38.36374241,291.50806596)(38.58374219,291.88306559)(38.83374367,292.17307497)
\curveto(39.09374168,292.46306501)(39.45374132,292.70806476)(39.91374367,292.90807497)
\curveto(40.04374073,292.95806451)(40.1737406,292.99306448)(40.30374367,293.01307497)
\curveto(40.44374033,293.03306444)(40.58374019,293.05806441)(40.72374367,293.08807497)
\curveto(40.79373998,293.09806437)(40.85873992,293.10306437)(40.91874367,293.10307497)
\curveto(40.9787398,293.10306437)(41.04373973,293.10806436)(41.11374367,293.11807497)
\curveto(41.94373883,293.13806433)(42.61373816,292.98806448)(43.12374367,292.66807497)
\curveto(43.63373714,292.35806511)(44.01373676,291.91806555)(44.26374367,291.34807497)
\curveto(44.31373646,291.22806624)(44.35873642,291.10306637)(44.39874367,290.97307497)
\curveto(44.43873634,290.84306663)(44.48373629,290.70806676)(44.53374367,290.56807497)
\curveto(44.55373622,290.48806698)(44.56873621,290.40306707)(44.57874367,290.31307497)
\lineto(44.63874367,290.07307497)
\curveto(44.66873611,289.96306751)(44.68373609,289.85306762)(44.68374367,289.74307497)
\curveto(44.69373608,289.63306784)(44.70873607,289.52306795)(44.72874367,289.41307497)
\curveto(44.74873603,289.36306811)(44.75373602,289.31806815)(44.74374367,289.27807497)
\curveto(44.74373603,289.23806823)(44.74873603,289.19806827)(44.75874367,289.15807497)
\curveto(44.76873601,289.10806836)(44.76873601,289.05306842)(44.75874367,288.99307497)
\curveto(44.75873602,288.94306853)(44.76373601,288.89306858)(44.77374367,288.84307497)
\lineto(44.77374367,288.70807497)
\curveto(44.79373598,288.64806882)(44.79373598,288.57806889)(44.77374367,288.49807497)
\curveto(44.76373601,288.42806904)(44.76873601,288.36306911)(44.78874367,288.30307497)
\curveto(44.79873598,288.2730692)(44.80373597,288.23306924)(44.80374367,288.18307497)
\lineto(44.80374367,288.06307497)
\lineto(44.80374367,287.59807497)
\moveto(43.25874367,285.27307497)
\curveto(43.35873742,285.59307188)(43.41873736,285.95807151)(43.43874367,286.36807497)
\curveto(43.45873732,286.77807069)(43.46873731,287.18807028)(43.46874367,287.59807497)
\curveto(43.46873731,288.02806944)(43.45873732,288.44806902)(43.43874367,288.85807497)
\curveto(43.41873736,289.2680682)(43.3737374,289.65306782)(43.30374367,290.01307497)
\curveto(43.23373754,290.3730671)(43.12373765,290.69306678)(42.97374367,290.97307497)
\curveto(42.83373794,291.26306621)(42.63873814,291.49806597)(42.38874367,291.67807497)
\curveto(42.22873855,291.78806568)(42.04873873,291.8680656)(41.84874367,291.91807497)
\curveto(41.64873913,291.97806549)(41.40373937,292.00806546)(41.11374367,292.00807497)
\curveto(41.09373968,291.98806548)(41.05873972,291.97806549)(41.00874367,291.97807497)
\curveto(40.95873982,291.98806548)(40.91873986,291.98806548)(40.88874367,291.97807497)
\curveto(40.80873997,291.95806551)(40.73374004,291.93806553)(40.66374367,291.91807497)
\curveto(40.60374017,291.90806556)(40.53874024,291.88806558)(40.46874367,291.85807497)
\curveto(40.19874058,291.73806573)(39.9787408,291.5680659)(39.80874367,291.34807497)
\curveto(39.64874113,291.13806633)(39.51374126,290.89306658)(39.40374367,290.61307497)
\curveto(39.35374142,290.50306697)(39.31374146,290.38306709)(39.28374367,290.25307497)
\curveto(39.26374151,290.13306734)(39.23874154,290.00806746)(39.20874367,289.87807497)
\curveto(39.18874159,289.82806764)(39.1787416,289.7730677)(39.17874367,289.71307497)
\curveto(39.1787416,289.66306781)(39.1737416,289.61306786)(39.16374367,289.56307497)
\curveto(39.15374162,289.473068)(39.14374163,289.37806809)(39.13374367,289.27807497)
\curveto(39.12374165,289.18806828)(39.11374166,289.09306838)(39.10374367,288.99307497)
\curveto(39.10374167,288.91306856)(39.09874168,288.82806864)(39.08874367,288.73807497)
\lineto(39.08874367,288.49807497)
\lineto(39.08874367,288.31807497)
\curveto(39.0787417,288.28806918)(39.0737417,288.25306922)(39.07374367,288.21307497)
\lineto(39.07374367,288.07807497)
\lineto(39.07374367,287.62807497)
\curveto(39.0737417,287.54806992)(39.06874171,287.46307001)(39.05874367,287.37307497)
\curveto(39.05874172,287.29307018)(39.06874171,287.21807025)(39.08874367,287.14807497)
\lineto(39.08874367,286.87807497)
\curveto(39.08874169,286.85807061)(39.08374169,286.82807064)(39.07374367,286.78807497)
\curveto(39.0737417,286.75807071)(39.0787417,286.73307074)(39.08874367,286.71307497)
\curveto(39.09874168,286.61307086)(39.10374167,286.51307096)(39.10374367,286.41307497)
\curveto(39.11374166,286.32307115)(39.12374165,286.22307125)(39.13374367,286.11307497)
\curveto(39.16374161,285.99307148)(39.1787416,285.8680716)(39.17874367,285.73807497)
\curveto(39.18874159,285.61807185)(39.21374156,285.50307197)(39.25374367,285.39307497)
\curveto(39.33374144,285.09307238)(39.41874136,284.82807264)(39.50874367,284.59807497)
\curveto(39.60874117,284.3680731)(39.75374102,284.15307332)(39.94374367,283.95307497)
\curveto(40.15374062,283.75307372)(40.41874036,283.60307387)(40.73874367,283.50307497)
\curveto(40.77874,283.48307399)(40.81373996,283.473074)(40.84374367,283.47307497)
\curveto(40.88373989,283.48307399)(40.92873985,283.47807399)(40.97874367,283.45807497)
\curveto(41.01873976,283.44807402)(41.08873969,283.43807403)(41.18874367,283.42807497)
\curveto(41.29873948,283.41807405)(41.38373939,283.42307405)(41.44374367,283.44307497)
\curveto(41.51373926,283.46307401)(41.58373919,283.473074)(41.65374367,283.47307497)
\curveto(41.72373905,283.48307399)(41.78873899,283.49807397)(41.84874367,283.51807497)
\curveto(42.04873873,283.57807389)(42.22873855,283.66307381)(42.38874367,283.77307497)
\curveto(42.41873836,283.79307368)(42.44373833,283.81307366)(42.46374367,283.83307497)
\lineto(42.52374367,283.89307497)
\curveto(42.56373821,283.91307356)(42.61373816,283.95307352)(42.67374367,284.01307497)
\curveto(42.773738,284.15307332)(42.85873792,284.28307319)(42.92874367,284.40307497)
\curveto(42.99873778,284.52307295)(43.06873771,284.6680728)(43.13874367,284.83807497)
\curveto(43.16873761,284.90807256)(43.18873759,284.97807249)(43.19874367,285.04807497)
\curveto(43.21873756,285.11807235)(43.23873754,285.19307228)(43.25874367,285.27307497)
}
}
{
\newrgbcolor{curcolor}{0 0 0}
\pscustom[linestyle=none,fillstyle=solid,fillcolor=curcolor]
{
\newpath
\moveto(25.16452492,212.95593508)
\curveto(25.26452006,212.95592446)(25.35951997,212.94592447)(25.44952492,212.92593508)
\curveto(25.53951979,212.9159245)(25.60451972,212.88592453)(25.64452492,212.83593508)
\curveto(25.70451962,212.75592466)(25.73451959,212.65092476)(25.73452492,212.52093508)
\lineto(25.73452492,212.13093508)
\lineto(25.73452492,210.63093508)
\lineto(25.73452492,204.24093508)
\lineto(25.73452492,203.07093508)
\lineto(25.73452492,202.75593508)
\curveto(25.74451958,202.65593476)(25.7295196,202.57593484)(25.68952492,202.51593508)
\curveto(25.63951969,202.43593498)(25.56451976,202.38593503)(25.46452492,202.36593508)
\curveto(25.37451995,202.35593506)(25.26452006,202.35093506)(25.13452492,202.35093508)
\lineto(24.90952492,202.35093508)
\curveto(24.8295205,202.37093504)(24.75952057,202.38593503)(24.69952492,202.39593508)
\curveto(24.63952069,202.415935)(24.58952074,202.45593496)(24.54952492,202.51593508)
\curveto(24.50952082,202.57593484)(24.48952084,202.65093476)(24.48952492,202.74093508)
\lineto(24.48952492,203.04093508)
\lineto(24.48952492,204.13593508)
\lineto(24.48952492,209.47593508)
\curveto(24.46952086,209.56592785)(24.45452087,209.64092777)(24.44452492,209.70093508)
\curveto(24.44452088,209.77092764)(24.41452091,209.83092758)(24.35452492,209.88093508)
\curveto(24.28452104,209.93092748)(24.19452113,209.95592746)(24.08452492,209.95593508)
\curveto(23.98452134,209.96592745)(23.87452145,209.97092744)(23.75452492,209.97093508)
\lineto(22.61452492,209.97093508)
\lineto(22.11952492,209.97093508)
\curveto(21.95952337,209.98092743)(21.84952348,210.04092737)(21.78952492,210.15093508)
\curveto(21.76952356,210.18092723)(21.75952357,210.2109272)(21.75952492,210.24093508)
\curveto(21.75952357,210.28092713)(21.75452357,210.32592709)(21.74452492,210.37593508)
\curveto(21.7245236,210.49592692)(21.7295236,210.60592681)(21.75952492,210.70593508)
\curveto(21.79952353,210.80592661)(21.85452347,210.87592654)(21.92452492,210.91593508)
\curveto(22.00452332,210.96592645)(22.1245232,210.99092642)(22.28452492,210.99093508)
\curveto(22.44452288,210.99092642)(22.57952275,211.00592641)(22.68952492,211.03593508)
\curveto(22.73952259,211.04592637)(22.79452253,211.05092636)(22.85452492,211.05093508)
\curveto(22.91452241,211.06092635)(22.97452235,211.07592634)(23.03452492,211.09593508)
\curveto(23.18452214,211.14592627)(23.329522,211.19592622)(23.46952492,211.24593508)
\curveto(23.60952172,211.30592611)(23.74452158,211.37592604)(23.87452492,211.45593508)
\curveto(24.01452131,211.54592587)(24.13452119,211.65092576)(24.23452492,211.77093508)
\curveto(24.33452099,211.89092552)(24.4295209,212.02092539)(24.51952492,212.16093508)
\curveto(24.57952075,212.26092515)(24.6245207,212.37092504)(24.65452492,212.49093508)
\curveto(24.69452063,212.6109248)(24.74452058,212.7159247)(24.80452492,212.80593508)
\curveto(24.85452047,212.86592455)(24.9245204,212.90592451)(25.01452492,212.92593508)
\curveto(25.03452029,212.93592448)(25.05952027,212.94092447)(25.08952492,212.94093508)
\curveto(25.11952021,212.94092447)(25.14452018,212.94592447)(25.16452492,212.95593508)
}
}
{
\newrgbcolor{curcolor}{0 0 0}
\pscustom[linestyle=none,fillstyle=solid,fillcolor=curcolor]
{
\newpath
\moveto(30.96413429,212.76093508)
\lineto(34.56413429,212.76093508)
\lineto(35.20913429,212.76093508)
\curveto(35.28912776,212.76092465)(35.36412769,212.75592466)(35.43413429,212.74593508)
\curveto(35.50412755,212.74592467)(35.56412749,212.73592468)(35.61413429,212.71593508)
\curveto(35.68412737,212.68592473)(35.73912731,212.62592479)(35.77913429,212.53593508)
\curveto(35.79912725,212.50592491)(35.80912724,212.46592495)(35.80913429,212.41593508)
\lineto(35.80913429,212.28093508)
\curveto(35.81912723,212.17092524)(35.81412724,212.06592535)(35.79413429,211.96593508)
\curveto(35.78412727,211.86592555)(35.7491273,211.79592562)(35.68913429,211.75593508)
\curveto(35.59912745,211.68592573)(35.46412759,211.65092576)(35.28413429,211.65093508)
\curveto(35.10412795,211.66092575)(34.93912811,211.66592575)(34.78913429,211.66593508)
\lineto(32.79413429,211.66593508)
\lineto(32.29913429,211.66593508)
\lineto(32.16413429,211.66593508)
\curveto(32.12413093,211.66592575)(32.08413097,211.66092575)(32.04413429,211.65093508)
\lineto(31.83413429,211.65093508)
\curveto(31.72413133,211.62092579)(31.64413141,211.58092583)(31.59413429,211.53093508)
\curveto(31.54413151,211.49092592)(31.50913154,211.43592598)(31.48913429,211.36593508)
\curveto(31.46913158,211.30592611)(31.4541316,211.23592618)(31.44413429,211.15593508)
\curveto(31.43413162,211.07592634)(31.41413164,210.98592643)(31.38413429,210.88593508)
\curveto(31.33413172,210.68592673)(31.29413176,210.48092693)(31.26413429,210.27093508)
\curveto(31.23413182,210.06092735)(31.19413186,209.85592756)(31.14413429,209.65593508)
\curveto(31.12413193,209.58592783)(31.11413194,209.5159279)(31.11413429,209.44593508)
\curveto(31.11413194,209.38592803)(31.10413195,209.32092809)(31.08413429,209.25093508)
\curveto(31.07413198,209.22092819)(31.06413199,209.18092823)(31.05413429,209.13093508)
\curveto(31.054132,209.09092832)(31.05913199,209.05092836)(31.06913429,209.01093508)
\curveto(31.08913196,208.96092845)(31.11413194,208.9159285)(31.14413429,208.87593508)
\curveto(31.18413187,208.84592857)(31.24413181,208.84092857)(31.32413429,208.86093508)
\curveto(31.38413167,208.88092853)(31.44413161,208.90592851)(31.50413429,208.93593508)
\curveto(31.56413149,208.97592844)(31.62413143,209.0109284)(31.68413429,209.04093508)
\curveto(31.74413131,209.06092835)(31.79413126,209.07592834)(31.83413429,209.08593508)
\curveto(32.02413103,209.16592825)(32.22913082,209.22092819)(32.44913429,209.25093508)
\curveto(32.67913037,209.28092813)(32.90913014,209.29092812)(33.13913429,209.28093508)
\curveto(33.37912967,209.28092813)(33.60912944,209.25592816)(33.82913429,209.20593508)
\curveto(34.049129,209.16592825)(34.2491288,209.10592831)(34.42913429,209.02593508)
\curveto(34.47912857,209.00592841)(34.52412853,208.98592843)(34.56413429,208.96593508)
\curveto(34.61412844,208.94592847)(34.66412839,208.92092849)(34.71413429,208.89093508)
\curveto(35.06412799,208.68092873)(35.34412771,208.45092896)(35.55413429,208.20093508)
\curveto(35.77412728,207.95092946)(35.96912708,207.62592979)(36.13913429,207.22593508)
\curveto(36.18912686,207.1159303)(36.22412683,207.00593041)(36.24413429,206.89593508)
\curveto(36.26412679,206.78593063)(36.28912676,206.67093074)(36.31913429,206.55093508)
\curveto(36.32912672,206.52093089)(36.33412672,206.47593094)(36.33413429,206.41593508)
\curveto(36.3541267,206.35593106)(36.36412669,206.28593113)(36.36413429,206.20593508)
\curveto(36.36412669,206.13593128)(36.37412668,206.07093134)(36.39413429,206.01093508)
\lineto(36.39413429,205.84593508)
\curveto(36.40412665,205.79593162)(36.40912664,205.72593169)(36.40913429,205.63593508)
\curveto(36.40912664,205.54593187)(36.39912665,205.47593194)(36.37913429,205.42593508)
\curveto(36.35912669,205.36593205)(36.3541267,205.30593211)(36.36413429,205.24593508)
\curveto(36.37412668,205.19593222)(36.36912668,205.14593227)(36.34913429,205.09593508)
\curveto(36.30912674,204.93593248)(36.27412678,204.78593263)(36.24413429,204.64593508)
\curveto(36.21412684,204.50593291)(36.16912688,204.37093304)(36.10913429,204.24093508)
\curveto(35.9491271,203.87093354)(35.72912732,203.53593388)(35.44913429,203.23593508)
\curveto(35.16912788,202.93593448)(34.8491282,202.70593471)(34.48913429,202.54593508)
\curveto(34.31912873,202.46593495)(34.11912893,202.39093502)(33.88913429,202.32093508)
\curveto(33.77912927,202.28093513)(33.66412939,202.25593516)(33.54413429,202.24593508)
\curveto(33.42412963,202.23593518)(33.30412975,202.2159352)(33.18413429,202.18593508)
\curveto(33.13412992,202.16593525)(33.07912997,202.16593525)(33.01913429,202.18593508)
\curveto(32.95913009,202.19593522)(32.89913015,202.19093522)(32.83913429,202.17093508)
\curveto(32.73913031,202.15093526)(32.63913041,202.15093526)(32.53913429,202.17093508)
\lineto(32.40413429,202.17093508)
\curveto(32.3541307,202.19093522)(32.29413076,202.20093521)(32.22413429,202.20093508)
\curveto(32.16413089,202.19093522)(32.10913094,202.19593522)(32.05913429,202.21593508)
\curveto(32.01913103,202.22593519)(31.98413107,202.23093518)(31.95413429,202.23093508)
\curveto(31.92413113,202.23093518)(31.88913116,202.23593518)(31.84913429,202.24593508)
\lineto(31.57913429,202.30593508)
\curveto(31.48913156,202.32593509)(31.40413165,202.35593506)(31.32413429,202.39593508)
\curveto(30.98413207,202.53593488)(30.69413236,202.69093472)(30.45413429,202.86093508)
\curveto(30.21413284,203.04093437)(29.99413306,203.27093414)(29.79413429,203.55093508)
\curveto(29.64413341,203.78093363)(29.52913352,204.02093339)(29.44913429,204.27093508)
\curveto(29.42913362,204.32093309)(29.41913363,204.36593305)(29.41913429,204.40593508)
\curveto(29.41913363,204.45593296)(29.40913364,204.50593291)(29.38913429,204.55593508)
\curveto(29.36913368,204.6159328)(29.3541337,204.69593272)(29.34413429,204.79593508)
\curveto(29.34413371,204.89593252)(29.36413369,204.97093244)(29.40413429,205.02093508)
\curveto(29.4541336,205.10093231)(29.53413352,205.14593227)(29.64413429,205.15593508)
\curveto(29.7541333,205.16593225)(29.86913318,205.17093224)(29.98913429,205.17093508)
\lineto(30.15413429,205.17093508)
\curveto(30.21413284,205.17093224)(30.26913278,205.16093225)(30.31913429,205.14093508)
\curveto(30.40913264,205.12093229)(30.47913257,205.08093233)(30.52913429,205.02093508)
\curveto(30.59913245,204.93093248)(30.64413241,204.82093259)(30.66413429,204.69093508)
\curveto(30.69413236,204.57093284)(30.73913231,204.46593295)(30.79913429,204.37593508)
\curveto(30.98913206,204.03593338)(31.2491318,203.76593365)(31.57913429,203.56593508)
\curveto(31.67913137,203.50593391)(31.78413127,203.45593396)(31.89413429,203.41593508)
\curveto(32.01413104,203.38593403)(32.13413092,203.35093406)(32.25413429,203.31093508)
\curveto(32.42413063,203.26093415)(32.62913042,203.24093417)(32.86913429,203.25093508)
\curveto(33.11912993,203.27093414)(33.31912973,203.30593411)(33.46913429,203.35593508)
\curveto(33.83912921,203.47593394)(34.12912892,203.63593378)(34.33913429,203.83593508)
\curveto(34.55912849,204.04593337)(34.73912831,204.32593309)(34.87913429,204.67593508)
\curveto(34.92912812,204.77593264)(34.95912809,204.88093253)(34.96913429,204.99093508)
\curveto(34.98912806,205.10093231)(35.01412804,205.2159322)(35.04413429,205.33593508)
\lineto(35.04413429,205.44093508)
\curveto(35.054128,205.48093193)(35.05912799,205.52093189)(35.05913429,205.56093508)
\curveto(35.06912798,205.59093182)(35.06912798,205.62593179)(35.05913429,205.66593508)
\lineto(35.05913429,205.78593508)
\curveto(35.05912799,206.04593137)(35.02912802,206.29093112)(34.96913429,206.52093508)
\curveto(34.85912819,206.87093054)(34.70412835,207.16593025)(34.50413429,207.40593508)
\curveto(34.30412875,207.65592976)(34.04412901,207.85092956)(33.72413429,207.99093508)
\lineto(33.54413429,208.05093508)
\curveto(33.49412956,208.07092934)(33.43412962,208.09092932)(33.36413429,208.11093508)
\curveto(33.31412974,208.13092928)(33.2541298,208.14092927)(33.18413429,208.14093508)
\curveto(33.12412993,208.15092926)(33.05912999,208.16592925)(32.98913429,208.18593508)
\lineto(32.83913429,208.18593508)
\curveto(32.79913025,208.20592921)(32.74413031,208.2159292)(32.67413429,208.21593508)
\curveto(32.61413044,208.2159292)(32.55913049,208.20592921)(32.50913429,208.18593508)
\lineto(32.40413429,208.18593508)
\curveto(32.37413068,208.18592923)(32.33913071,208.18092923)(32.29913429,208.17093508)
\lineto(32.05913429,208.11093508)
\curveto(31.97913107,208.10092931)(31.89913115,208.08092933)(31.81913429,208.05093508)
\curveto(31.57913147,207.95092946)(31.3491317,207.8159296)(31.12913429,207.64593508)
\curveto(31.03913201,207.57592984)(30.9541321,207.50092991)(30.87413429,207.42093508)
\curveto(30.79413226,207.35093006)(30.69413236,207.29593012)(30.57413429,207.25593508)
\curveto(30.48413257,207.22593019)(30.34413271,207.2159302)(30.15413429,207.22593508)
\curveto(29.97413308,207.23593018)(29.8541332,207.26093015)(29.79413429,207.30093508)
\curveto(29.74413331,207.34093007)(29.70413335,207.40093001)(29.67413429,207.48093508)
\curveto(29.6541334,207.56092985)(29.6541334,207.64592977)(29.67413429,207.73593508)
\curveto(29.70413335,207.85592956)(29.72413333,207.97592944)(29.73413429,208.09593508)
\curveto(29.7541333,208.22592919)(29.77913327,208.35092906)(29.80913429,208.47093508)
\curveto(29.82913322,208.5109289)(29.83413322,208.54592887)(29.82413429,208.57593508)
\curveto(29.82413323,208.6159288)(29.83413322,208.66092875)(29.85413429,208.71093508)
\curveto(29.87413318,208.80092861)(29.88913316,208.89092852)(29.89913429,208.98093508)
\curveto(29.90913314,209.08092833)(29.92913312,209.17592824)(29.95913429,209.26593508)
\curveto(29.96913308,209.32592809)(29.97413308,209.38592803)(29.97413429,209.44593508)
\curveto(29.98413307,209.50592791)(29.99913305,209.56592785)(30.01913429,209.62593508)
\curveto(30.06913298,209.82592759)(30.10413295,210.03092738)(30.12413429,210.24093508)
\curveto(30.1541329,210.46092695)(30.19413286,210.67092674)(30.24413429,210.87093508)
\curveto(30.27413278,210.97092644)(30.29413276,211.07092634)(30.30413429,211.17093508)
\curveto(30.31413274,211.27092614)(30.32913272,211.37092604)(30.34913429,211.47093508)
\curveto(30.35913269,211.50092591)(30.36413269,211.54092587)(30.36413429,211.59093508)
\curveto(30.39413266,211.70092571)(30.41413264,211.80592561)(30.42413429,211.90593508)
\curveto(30.44413261,212.0159254)(30.46913258,212.12592529)(30.49913429,212.23593508)
\curveto(30.51913253,212.3159251)(30.53413252,212.38592503)(30.54413429,212.44593508)
\curveto(30.5541325,212.5159249)(30.57913247,212.57592484)(30.61913429,212.62593508)
\curveto(30.63913241,212.65592476)(30.66913238,212.67592474)(30.70913429,212.68593508)
\curveto(30.7491323,212.70592471)(30.79413226,212.72592469)(30.84413429,212.74593508)
\curveto(30.90413215,212.74592467)(30.94413211,212.75092466)(30.96413429,212.76093508)
}
}
{
\newrgbcolor{curcolor}{0 0 0}
\pscustom[linestyle=none,fillstyle=solid,fillcolor=curcolor]
{
\newpath
\moveto(44.80374367,207.43593508)
\lineto(44.80374367,207.18093508)
\curveto(44.81373596,207.10093031)(44.80873597,207.02593039)(44.78874367,206.95593508)
\lineto(44.78874367,206.71593508)
\lineto(44.78874367,206.55093508)
\curveto(44.76873601,206.45093096)(44.75873602,206.34593107)(44.75874367,206.23593508)
\curveto(44.75873602,206.13593128)(44.74873603,206.03593138)(44.72874367,205.93593508)
\lineto(44.72874367,205.78593508)
\curveto(44.69873608,205.64593177)(44.6787361,205.50593191)(44.66874367,205.36593508)
\curveto(44.65873612,205.23593218)(44.63373614,205.10593231)(44.59374367,204.97593508)
\curveto(44.5737362,204.89593252)(44.55373622,204.8109326)(44.53374367,204.72093508)
\lineto(44.47374367,204.48093508)
\lineto(44.35374367,204.18093508)
\curveto(44.32373645,204.09093332)(44.28873649,204.00093341)(44.24874367,203.91093508)
\curveto(44.14873663,203.69093372)(44.01373676,203.47593394)(43.84374367,203.26593508)
\curveto(43.68373709,203.05593436)(43.50873727,202.88593453)(43.31874367,202.75593508)
\curveto(43.26873751,202.7159347)(43.20873757,202.67593474)(43.13874367,202.63593508)
\curveto(43.0787377,202.60593481)(43.01873776,202.57093484)(42.95874367,202.53093508)
\curveto(42.8787379,202.48093493)(42.78373799,202.44093497)(42.67374367,202.41093508)
\curveto(42.56373821,202.38093503)(42.45873832,202.35093506)(42.35874367,202.32093508)
\curveto(42.24873853,202.28093513)(42.13873864,202.25593516)(42.02874367,202.24593508)
\curveto(41.91873886,202.23593518)(41.80373897,202.22093519)(41.68374367,202.20093508)
\curveto(41.64373913,202.19093522)(41.59873918,202.19093522)(41.54874367,202.20093508)
\curveto(41.50873927,202.20093521)(41.46873931,202.19593522)(41.42874367,202.18593508)
\curveto(41.38873939,202.17593524)(41.33373944,202.17093524)(41.26374367,202.17093508)
\curveto(41.19373958,202.17093524)(41.14373963,202.17593524)(41.11374367,202.18593508)
\curveto(41.06373971,202.20593521)(41.01873976,202.2109352)(40.97874367,202.20093508)
\curveto(40.93873984,202.19093522)(40.90373987,202.19093522)(40.87374367,202.20093508)
\lineto(40.78374367,202.20093508)
\curveto(40.72374005,202.22093519)(40.65874012,202.23593518)(40.58874367,202.24593508)
\curveto(40.52874025,202.24593517)(40.46374031,202.25093516)(40.39374367,202.26093508)
\curveto(40.22374055,202.3109351)(40.06374071,202.36093505)(39.91374367,202.41093508)
\curveto(39.76374101,202.46093495)(39.61874116,202.52593489)(39.47874367,202.60593508)
\curveto(39.42874135,202.64593477)(39.3737414,202.67593474)(39.31374367,202.69593508)
\curveto(39.26374151,202.72593469)(39.21374156,202.76093465)(39.16374367,202.80093508)
\curveto(38.92374185,202.98093443)(38.72374205,203.20093421)(38.56374367,203.46093508)
\curveto(38.40374237,203.72093369)(38.26374251,204.00593341)(38.14374367,204.31593508)
\curveto(38.08374269,204.45593296)(38.03874274,204.59593282)(38.00874367,204.73593508)
\curveto(37.9787428,204.88593253)(37.94374283,205.04093237)(37.90374367,205.20093508)
\curveto(37.88374289,205.3109321)(37.86874291,205.42093199)(37.85874367,205.53093508)
\curveto(37.84874293,205.64093177)(37.83374294,205.75093166)(37.81374367,205.86093508)
\curveto(37.80374297,205.90093151)(37.79874298,205.94093147)(37.79874367,205.98093508)
\curveto(37.80874297,206.02093139)(37.80874297,206.06093135)(37.79874367,206.10093508)
\curveto(37.78874299,206.15093126)(37.78374299,206.20093121)(37.78374367,206.25093508)
\lineto(37.78374367,206.41593508)
\curveto(37.76374301,206.46593095)(37.75874302,206.5159309)(37.76874367,206.56593508)
\curveto(37.778743,206.62593079)(37.778743,206.68093073)(37.76874367,206.73093508)
\curveto(37.75874302,206.77093064)(37.75874302,206.8159306)(37.76874367,206.86593508)
\curveto(37.778743,206.9159305)(37.773743,206.96593045)(37.75374367,207.01593508)
\curveto(37.73374304,207.08593033)(37.72874305,207.16093025)(37.73874367,207.24093508)
\curveto(37.74874303,207.33093008)(37.75374302,207.41593)(37.75374367,207.49593508)
\curveto(37.75374302,207.58592983)(37.74874303,207.68592973)(37.73874367,207.79593508)
\curveto(37.72874305,207.9159295)(37.73374304,208.0159294)(37.75374367,208.09593508)
\lineto(37.75374367,208.38093508)
\lineto(37.79874367,209.01093508)
\curveto(37.80874297,209.1109283)(37.81874296,209.20592821)(37.82874367,209.29593508)
\lineto(37.85874367,209.59593508)
\curveto(37.8787429,209.64592777)(37.88374289,209.69592772)(37.87374367,209.74593508)
\curveto(37.8737429,209.80592761)(37.88374289,209.86092755)(37.90374367,209.91093508)
\curveto(37.95374282,210.08092733)(37.99374278,210.24592717)(38.02374367,210.40593508)
\curveto(38.05374272,210.57592684)(38.10374267,210.73592668)(38.17374367,210.88593508)
\curveto(38.36374241,211.34592607)(38.58374219,211.72092569)(38.83374367,212.01093508)
\curveto(39.09374168,212.30092511)(39.45374132,212.54592487)(39.91374367,212.74593508)
\curveto(40.04374073,212.79592462)(40.1737406,212.83092458)(40.30374367,212.85093508)
\curveto(40.44374033,212.87092454)(40.58374019,212.89592452)(40.72374367,212.92593508)
\curveto(40.79373998,212.93592448)(40.85873992,212.94092447)(40.91874367,212.94093508)
\curveto(40.9787398,212.94092447)(41.04373973,212.94592447)(41.11374367,212.95593508)
\curveto(41.94373883,212.97592444)(42.61373816,212.82592459)(43.12374367,212.50593508)
\curveto(43.63373714,212.19592522)(44.01373676,211.75592566)(44.26374367,211.18593508)
\curveto(44.31373646,211.06592635)(44.35873642,210.94092647)(44.39874367,210.81093508)
\curveto(44.43873634,210.68092673)(44.48373629,210.54592687)(44.53374367,210.40593508)
\curveto(44.55373622,210.32592709)(44.56873621,210.24092717)(44.57874367,210.15093508)
\lineto(44.63874367,209.91093508)
\curveto(44.66873611,209.80092761)(44.68373609,209.69092772)(44.68374367,209.58093508)
\curveto(44.69373608,209.47092794)(44.70873607,209.36092805)(44.72874367,209.25093508)
\curveto(44.74873603,209.20092821)(44.75373602,209.15592826)(44.74374367,209.11593508)
\curveto(44.74373603,209.07592834)(44.74873603,209.03592838)(44.75874367,208.99593508)
\curveto(44.76873601,208.94592847)(44.76873601,208.89092852)(44.75874367,208.83093508)
\curveto(44.75873602,208.78092863)(44.76373601,208.73092868)(44.77374367,208.68093508)
\lineto(44.77374367,208.54593508)
\curveto(44.79373598,208.48592893)(44.79373598,208.415929)(44.77374367,208.33593508)
\curveto(44.76373601,208.26592915)(44.76873601,208.20092921)(44.78874367,208.14093508)
\curveto(44.79873598,208.1109293)(44.80373597,208.07092934)(44.80374367,208.02093508)
\lineto(44.80374367,207.90093508)
\lineto(44.80374367,207.43593508)
\moveto(43.25874367,205.11093508)
\curveto(43.35873742,205.43093198)(43.41873736,205.79593162)(43.43874367,206.20593508)
\curveto(43.45873732,206.6159308)(43.46873731,207.02593039)(43.46874367,207.43593508)
\curveto(43.46873731,207.86592955)(43.45873732,208.28592913)(43.43874367,208.69593508)
\curveto(43.41873736,209.10592831)(43.3737374,209.49092792)(43.30374367,209.85093508)
\curveto(43.23373754,210.2109272)(43.12373765,210.53092688)(42.97374367,210.81093508)
\curveto(42.83373794,211.10092631)(42.63873814,211.33592608)(42.38874367,211.51593508)
\curveto(42.22873855,211.62592579)(42.04873873,211.70592571)(41.84874367,211.75593508)
\curveto(41.64873913,211.8159256)(41.40373937,211.84592557)(41.11374367,211.84593508)
\curveto(41.09373968,211.82592559)(41.05873972,211.8159256)(41.00874367,211.81593508)
\curveto(40.95873982,211.82592559)(40.91873986,211.82592559)(40.88874367,211.81593508)
\curveto(40.80873997,211.79592562)(40.73374004,211.77592564)(40.66374367,211.75593508)
\curveto(40.60374017,211.74592567)(40.53874024,211.72592569)(40.46874367,211.69593508)
\curveto(40.19874058,211.57592584)(39.9787408,211.40592601)(39.80874367,211.18593508)
\curveto(39.64874113,210.97592644)(39.51374126,210.73092668)(39.40374367,210.45093508)
\curveto(39.35374142,210.34092707)(39.31374146,210.22092719)(39.28374367,210.09093508)
\curveto(39.26374151,209.97092744)(39.23874154,209.84592757)(39.20874367,209.71593508)
\curveto(39.18874159,209.66592775)(39.1787416,209.6109278)(39.17874367,209.55093508)
\curveto(39.1787416,209.50092791)(39.1737416,209.45092796)(39.16374367,209.40093508)
\curveto(39.15374162,209.3109281)(39.14374163,209.2159282)(39.13374367,209.11593508)
\curveto(39.12374165,209.02592839)(39.11374166,208.93092848)(39.10374367,208.83093508)
\curveto(39.10374167,208.75092866)(39.09874168,208.66592875)(39.08874367,208.57593508)
\lineto(39.08874367,208.33593508)
\lineto(39.08874367,208.15593508)
\curveto(39.0787417,208.12592929)(39.0737417,208.09092932)(39.07374367,208.05093508)
\lineto(39.07374367,207.91593508)
\lineto(39.07374367,207.46593508)
\curveto(39.0737417,207.38593003)(39.06874171,207.30093011)(39.05874367,207.21093508)
\curveto(39.05874172,207.13093028)(39.06874171,207.05593036)(39.08874367,206.98593508)
\lineto(39.08874367,206.71593508)
\curveto(39.08874169,206.69593072)(39.08374169,206.66593075)(39.07374367,206.62593508)
\curveto(39.0737417,206.59593082)(39.0787417,206.57093084)(39.08874367,206.55093508)
\curveto(39.09874168,206.45093096)(39.10374167,206.35093106)(39.10374367,206.25093508)
\curveto(39.11374166,206.16093125)(39.12374165,206.06093135)(39.13374367,205.95093508)
\curveto(39.16374161,205.83093158)(39.1787416,205.70593171)(39.17874367,205.57593508)
\curveto(39.18874159,205.45593196)(39.21374156,205.34093207)(39.25374367,205.23093508)
\curveto(39.33374144,204.93093248)(39.41874136,204.66593275)(39.50874367,204.43593508)
\curveto(39.60874117,204.20593321)(39.75374102,203.99093342)(39.94374367,203.79093508)
\curveto(40.15374062,203.59093382)(40.41874036,203.44093397)(40.73874367,203.34093508)
\curveto(40.77874,203.32093409)(40.81373996,203.3109341)(40.84374367,203.31093508)
\curveto(40.88373989,203.32093409)(40.92873985,203.3159341)(40.97874367,203.29593508)
\curveto(41.01873976,203.28593413)(41.08873969,203.27593414)(41.18874367,203.26593508)
\curveto(41.29873948,203.25593416)(41.38373939,203.26093415)(41.44374367,203.28093508)
\curveto(41.51373926,203.30093411)(41.58373919,203.3109341)(41.65374367,203.31093508)
\curveto(41.72373905,203.32093409)(41.78873899,203.33593408)(41.84874367,203.35593508)
\curveto(42.04873873,203.415934)(42.22873855,203.50093391)(42.38874367,203.61093508)
\curveto(42.41873836,203.63093378)(42.44373833,203.65093376)(42.46374367,203.67093508)
\lineto(42.52374367,203.73093508)
\curveto(42.56373821,203.75093366)(42.61373816,203.79093362)(42.67374367,203.85093508)
\curveto(42.773738,203.99093342)(42.85873792,204.12093329)(42.92874367,204.24093508)
\curveto(42.99873778,204.36093305)(43.06873771,204.50593291)(43.13874367,204.67593508)
\curveto(43.16873761,204.74593267)(43.18873759,204.8159326)(43.19874367,204.88593508)
\curveto(43.21873756,204.95593246)(43.23873754,205.03093238)(43.25874367,205.11093508)
}
}
{
\newrgbcolor{curcolor}{0 0 0}
\pscustom[linestyle=none,fillstyle=solid,fillcolor=curcolor]
{
\newpath
\moveto(44.80374176,127.24385744)
\lineto(44.80374176,126.98885744)
\curveto(44.81373406,126.90885268)(44.80873406,126.83385275)(44.78874176,126.76385744)
\lineto(44.78874176,126.52385744)
\lineto(44.78874176,126.35885744)
\curveto(44.7687341,126.25885333)(44.75873411,126.15385343)(44.75874176,126.04385744)
\curveto(44.75873411,125.94385364)(44.74873412,125.84385374)(44.72874176,125.74385744)
\lineto(44.72874176,125.59385744)
\curveto(44.69873417,125.45385413)(44.67873419,125.31385427)(44.66874176,125.17385744)
\curveto(44.65873421,125.04385454)(44.63373424,124.91385467)(44.59374176,124.78385744)
\curveto(44.5737343,124.70385488)(44.55373432,124.61885497)(44.53374176,124.52885744)
\lineto(44.47374176,124.28885744)
\lineto(44.35374176,123.98885744)
\curveto(44.32373455,123.89885569)(44.28873458,123.80885578)(44.24874176,123.71885744)
\curveto(44.14873472,123.49885609)(44.01373486,123.2838563)(43.84374176,123.07385744)
\curveto(43.68373519,122.86385672)(43.50873536,122.69385689)(43.31874176,122.56385744)
\curveto(43.2687356,122.52385706)(43.20873566,122.4838571)(43.13874176,122.44385744)
\curveto(43.07873579,122.41385717)(43.01873585,122.37885721)(42.95874176,122.33885744)
\curveto(42.87873599,122.2888573)(42.78373609,122.24885734)(42.67374176,122.21885744)
\curveto(42.56373631,122.1888574)(42.45873641,122.15885743)(42.35874176,122.12885744)
\curveto(42.24873662,122.0888575)(42.13873673,122.06385752)(42.02874176,122.05385744)
\curveto(41.91873695,122.04385754)(41.80373707,122.02885756)(41.68374176,122.00885744)
\curveto(41.64373723,121.99885759)(41.59873727,121.99885759)(41.54874176,122.00885744)
\curveto(41.50873736,122.00885758)(41.4687374,122.00385758)(41.42874176,121.99385744)
\curveto(41.38873748,121.9838576)(41.33373754,121.97885761)(41.26374176,121.97885744)
\curveto(41.19373768,121.97885761)(41.14373773,121.9838576)(41.11374176,121.99385744)
\curveto(41.06373781,122.01385757)(41.01873785,122.01885757)(40.97874176,122.00885744)
\curveto(40.93873793,121.99885759)(40.90373797,121.99885759)(40.87374176,122.00885744)
\lineto(40.78374176,122.00885744)
\curveto(40.72373815,122.02885756)(40.65873821,122.04385754)(40.58874176,122.05385744)
\curveto(40.52873834,122.05385753)(40.46373841,122.05885753)(40.39374176,122.06885744)
\curveto(40.22373865,122.11885747)(40.06373881,122.16885742)(39.91374176,122.21885744)
\curveto(39.76373911,122.26885732)(39.61873925,122.33385725)(39.47874176,122.41385744)
\curveto(39.42873944,122.45385713)(39.3737395,122.4838571)(39.31374176,122.50385744)
\curveto(39.26373961,122.53385705)(39.21373966,122.56885702)(39.16374176,122.60885744)
\curveto(38.92373995,122.7888568)(38.72374015,123.00885658)(38.56374176,123.26885744)
\curveto(38.40374047,123.52885606)(38.26374061,123.81385577)(38.14374176,124.12385744)
\curveto(38.08374079,124.26385532)(38.03874083,124.40385518)(38.00874176,124.54385744)
\curveto(37.97874089,124.69385489)(37.94374093,124.84885474)(37.90374176,125.00885744)
\curveto(37.88374099,125.11885447)(37.868741,125.22885436)(37.85874176,125.33885744)
\curveto(37.84874102,125.44885414)(37.83374104,125.55885403)(37.81374176,125.66885744)
\curveto(37.80374107,125.70885388)(37.79874107,125.74885384)(37.79874176,125.78885744)
\curveto(37.80874106,125.82885376)(37.80874106,125.86885372)(37.79874176,125.90885744)
\curveto(37.78874108,125.95885363)(37.78374109,126.00885358)(37.78374176,126.05885744)
\lineto(37.78374176,126.22385744)
\curveto(37.76374111,126.27385331)(37.75874111,126.32385326)(37.76874176,126.37385744)
\curveto(37.77874109,126.43385315)(37.77874109,126.4888531)(37.76874176,126.53885744)
\curveto(37.75874111,126.57885301)(37.75874111,126.62385296)(37.76874176,126.67385744)
\curveto(37.77874109,126.72385286)(37.7737411,126.77385281)(37.75374176,126.82385744)
\curveto(37.73374114,126.89385269)(37.72874114,126.96885262)(37.73874176,127.04885744)
\curveto(37.74874112,127.13885245)(37.75374112,127.22385236)(37.75374176,127.30385744)
\curveto(37.75374112,127.39385219)(37.74874112,127.49385209)(37.73874176,127.60385744)
\curveto(37.72874114,127.72385186)(37.73374114,127.82385176)(37.75374176,127.90385744)
\lineto(37.75374176,128.18885744)
\lineto(37.79874176,128.81885744)
\curveto(37.80874106,128.91885067)(37.81874105,129.01385057)(37.82874176,129.10385744)
\lineto(37.85874176,129.40385744)
\curveto(37.87874099,129.45385013)(37.88374099,129.50385008)(37.87374176,129.55385744)
\curveto(37.873741,129.61384997)(37.88374099,129.66884992)(37.90374176,129.71885744)
\curveto(37.95374092,129.8888497)(37.99374088,130.05384953)(38.02374176,130.21385744)
\curveto(38.05374082,130.3838492)(38.10374077,130.54384904)(38.17374176,130.69385744)
\curveto(38.36374051,131.15384843)(38.58374029,131.52884806)(38.83374176,131.81885744)
\curveto(39.09373978,132.10884748)(39.45373942,132.35384723)(39.91374176,132.55385744)
\curveto(40.04373883,132.60384698)(40.1737387,132.63884695)(40.30374176,132.65885744)
\curveto(40.44373843,132.67884691)(40.58373829,132.70384688)(40.72374176,132.73385744)
\curveto(40.79373808,132.74384684)(40.85873801,132.74884684)(40.91874176,132.74885744)
\curveto(40.97873789,132.74884684)(41.04373783,132.75384683)(41.11374176,132.76385744)
\curveto(41.94373693,132.7838468)(42.61373626,132.63384695)(43.12374176,132.31385744)
\curveto(43.63373524,132.00384758)(44.01373486,131.56384802)(44.26374176,130.99385744)
\curveto(44.31373456,130.87384871)(44.35873451,130.74884884)(44.39874176,130.61885744)
\curveto(44.43873443,130.4888491)(44.48373439,130.35384923)(44.53374176,130.21385744)
\curveto(44.55373432,130.13384945)(44.5687343,130.04884954)(44.57874176,129.95885744)
\lineto(44.63874176,129.71885744)
\curveto(44.6687342,129.60884998)(44.68373419,129.49885009)(44.68374176,129.38885744)
\curveto(44.69373418,129.27885031)(44.70873416,129.16885042)(44.72874176,129.05885744)
\curveto(44.74873412,129.00885058)(44.75373412,128.96385062)(44.74374176,128.92385744)
\curveto(44.74373413,128.8838507)(44.74873412,128.84385074)(44.75874176,128.80385744)
\curveto(44.7687341,128.75385083)(44.7687341,128.69885089)(44.75874176,128.63885744)
\curveto(44.75873411,128.588851)(44.76373411,128.53885105)(44.77374176,128.48885744)
\lineto(44.77374176,128.35385744)
\curveto(44.79373408,128.29385129)(44.79373408,128.22385136)(44.77374176,128.14385744)
\curveto(44.76373411,128.07385151)(44.7687341,128.00885158)(44.78874176,127.94885744)
\curveto(44.79873407,127.91885167)(44.80373407,127.87885171)(44.80374176,127.82885744)
\lineto(44.80374176,127.70885744)
\lineto(44.80374176,127.24385744)
\moveto(43.25874176,124.91885744)
\curveto(43.35873551,125.23885435)(43.41873545,125.60385398)(43.43874176,126.01385744)
\curveto(43.45873541,126.42385316)(43.4687354,126.83385275)(43.46874176,127.24385744)
\curveto(43.4687354,127.67385191)(43.45873541,128.09385149)(43.43874176,128.50385744)
\curveto(43.41873545,128.91385067)(43.3737355,129.29885029)(43.30374176,129.65885744)
\curveto(43.23373564,130.01884957)(43.12373575,130.33884925)(42.97374176,130.61885744)
\curveto(42.83373604,130.90884868)(42.63873623,131.14384844)(42.38874176,131.32385744)
\curveto(42.22873664,131.43384815)(42.04873682,131.51384807)(41.84874176,131.56385744)
\curveto(41.64873722,131.62384796)(41.40373747,131.65384793)(41.11374176,131.65385744)
\curveto(41.09373778,131.63384795)(41.05873781,131.62384796)(41.00874176,131.62385744)
\curveto(40.95873791,131.63384795)(40.91873795,131.63384795)(40.88874176,131.62385744)
\curveto(40.80873806,131.60384798)(40.73373814,131.583848)(40.66374176,131.56385744)
\curveto(40.60373827,131.55384803)(40.53873833,131.53384805)(40.46874176,131.50385744)
\curveto(40.19873867,131.3838482)(39.97873889,131.21384837)(39.80874176,130.99385744)
\curveto(39.64873922,130.7838488)(39.51373936,130.53884905)(39.40374176,130.25885744)
\curveto(39.35373952,130.14884944)(39.31373956,130.02884956)(39.28374176,129.89885744)
\curveto(39.26373961,129.77884981)(39.23873963,129.65384993)(39.20874176,129.52385744)
\curveto(39.18873968,129.47385011)(39.17873969,129.41885017)(39.17874176,129.35885744)
\curveto(39.17873969,129.30885028)(39.1737397,129.25885033)(39.16374176,129.20885744)
\curveto(39.15373972,129.11885047)(39.14373973,129.02385056)(39.13374176,128.92385744)
\curveto(39.12373975,128.83385075)(39.11373976,128.73885085)(39.10374176,128.63885744)
\curveto(39.10373977,128.55885103)(39.09873977,128.47385111)(39.08874176,128.38385744)
\lineto(39.08874176,128.14385744)
\lineto(39.08874176,127.96385744)
\curveto(39.07873979,127.93385165)(39.0737398,127.89885169)(39.07374176,127.85885744)
\lineto(39.07374176,127.72385744)
\lineto(39.07374176,127.27385744)
\curveto(39.0737398,127.19385239)(39.0687398,127.10885248)(39.05874176,127.01885744)
\curveto(39.05873981,126.93885265)(39.0687398,126.86385272)(39.08874176,126.79385744)
\lineto(39.08874176,126.52385744)
\curveto(39.08873978,126.50385308)(39.08373979,126.47385311)(39.07374176,126.43385744)
\curveto(39.0737398,126.40385318)(39.07873979,126.37885321)(39.08874176,126.35885744)
\curveto(39.09873977,126.25885333)(39.10373977,126.15885343)(39.10374176,126.05885744)
\curveto(39.11373976,125.96885362)(39.12373975,125.86885372)(39.13374176,125.75885744)
\curveto(39.16373971,125.63885395)(39.17873969,125.51385407)(39.17874176,125.38385744)
\curveto(39.18873968,125.26385432)(39.21373966,125.14885444)(39.25374176,125.03885744)
\curveto(39.33373954,124.73885485)(39.41873945,124.47385511)(39.50874176,124.24385744)
\curveto(39.60873926,124.01385557)(39.75373912,123.79885579)(39.94374176,123.59885744)
\curveto(40.15373872,123.39885619)(40.41873845,123.24885634)(40.73874176,123.14885744)
\curveto(40.77873809,123.12885646)(40.81373806,123.11885647)(40.84374176,123.11885744)
\curveto(40.88373799,123.12885646)(40.92873794,123.12385646)(40.97874176,123.10385744)
\curveto(41.01873785,123.09385649)(41.08873778,123.0838565)(41.18874176,123.07385744)
\curveto(41.29873757,123.06385652)(41.38373749,123.06885652)(41.44374176,123.08885744)
\curveto(41.51373736,123.10885648)(41.58373729,123.11885647)(41.65374176,123.11885744)
\curveto(41.72373715,123.12885646)(41.78873708,123.14385644)(41.84874176,123.16385744)
\curveto(42.04873682,123.22385636)(42.22873664,123.30885628)(42.38874176,123.41885744)
\curveto(42.41873645,123.43885615)(42.44373643,123.45885613)(42.46374176,123.47885744)
\lineto(42.52374176,123.53885744)
\curveto(42.56373631,123.55885603)(42.61373626,123.59885599)(42.67374176,123.65885744)
\curveto(42.7737361,123.79885579)(42.85873601,123.92885566)(42.92874176,124.04885744)
\curveto(42.99873587,124.16885542)(43.0687358,124.31385527)(43.13874176,124.48385744)
\curveto(43.1687357,124.55385503)(43.18873568,124.62385496)(43.19874176,124.69385744)
\curveto(43.21873565,124.76385482)(43.23873563,124.83885475)(43.25874176,124.91885744)
}
}
{
\newrgbcolor{curcolor}{0.90196079 0.90196079 0.90196079}
\pscustom[linestyle=none,fillstyle=solid,fillcolor=curcolor]
{
\newpath
\moveto(67.142857,127.49991)
\lineto(67.142857,131.95108)
\lineto(115.78874,131.95108)
\lineto(115.78874,132.99098)
\lineto(160.27393,132.98698)
\lineto(160.27393,134.03081)
\lineto(188.28124,134.03081)
\lineto(189.30952,134.99676)
\lineto(274.28568,134.95536)
\lineto(279.35265,178.92857)
\lineto(296.27229,178.92857)
\lineto(302.23212,258.03571)
\lineto(368.32586,259.93304)
\lineto(374.26337,308.99553)
\lineto(562.32627,311.04911)
\lineto(568.25573,355.9375)
\lineto(607.3214,355.9375)
\lineto(613.41263,361.51786)
\lineto(740.10365,361.51786)
\lineto(745.7589,377.96875)
\lineto(757.23211,377.96875)
\lineto(762.27676,416.96428)
\lineto(1024.2857,417.98434)
\lineto(1024.2857,127.5)
\closepath
}
}
{
\newrgbcolor{curcolor}{0 0 0}
\pscustom[linewidth=1,linecolor=curcolor]
{
\newpath
\moveto(67.142857,287.5)
\lineto(1024.2857,287.5)
}
}
{
\newrgbcolor{curcolor}{0 0 0}
\pscustom[linewidth=1,linecolor=curcolor]
{
\newpath
\moveto(67.142857,207.54464)
\lineto(1024.2857,207.54464)
}
}
{
\newrgbcolor{curcolor}{0 0 0}
\pscustom[linewidth=1,linecolor=curcolor]
{
\newpath
\moveto(67.142857,127.5)
\lineto(1024.2857,127.5)
}
}
{
\newrgbcolor{curcolor}{0 0 0}
\pscustom[linewidth=1,linecolor=curcolor]
{
\newpath
\moveto(67.142857,367.5)
\lineto(1024.2857,367.5)
}
}
{
\newrgbcolor{curcolor}{1 1 1}
\pscustom[linestyle=none,fillstyle=solid,fillcolor=curcolor]
{
\newpath
\moveto(107.83379364,420.7285132)
\lineto(221.22342682,420.7285132)
\lineto(221.22342682,314.40995546)
\lineto(107.83379364,314.40995546)
\closepath
}
}
{
\newrgbcolor{curcolor}{0 0 0}
\pscustom[linewidth=1,linecolor=curcolor]
{
\newpath
\moveto(107.83379364,420.7285132)
\lineto(221.22342682,420.7285132)
\lineto(221.22342682,314.40995546)
\lineto(107.83379364,314.40995546)
\closepath
}
}
{
\newrgbcolor{curcolor}{0 0 0}
\pscustom[linewidth=1,linecolor=curcolor]
{
\newpath
\moveto(67.142857,447.45536)
\lineto(1024.2857,447.45536)
}
}
{
\newrgbcolor{curcolor}{0 0 0}
\pscustom[linestyle=none,fillstyle=solid,fillcolor=curcolor]
{
\newpath
\moveto(140.78876512,407.57102054)
\curveto(140.92876345,407.57100984)(141.09376328,407.57100984)(141.28376512,407.57102054)
\curveto(141.48376289,407.58100983)(141.61376276,407.54600987)(141.67376512,407.46602054)
\curveto(141.75376262,407.37601004)(141.78876259,407.24101017)(141.77876512,407.06102054)
\lineto(141.77876512,406.55102054)
\lineto(141.77876512,404.30102054)
\lineto(141.77876512,401.51102054)
\curveto(141.7787626,401.16101625)(141.78376259,400.8160166)(141.79376512,400.47602054)
\curveto(141.80376257,400.14601727)(141.85376252,399.85101756)(141.94376512,399.59102054)
\curveto(142.0737623,399.20101821)(142.25876212,398.88101853)(142.49876512,398.63102054)
\curveto(142.73876164,398.38101903)(143.04376133,398.17601924)(143.41376512,398.01602054)
\curveto(143.52376085,397.96601945)(143.63876074,397.93101948)(143.75876512,397.91102054)
\curveto(143.8787605,397.90101951)(143.99876038,397.88101953)(144.11876512,397.85102054)
\lineto(144.29876512,397.82102054)
\curveto(144.36876001,397.82101959)(144.43375994,397.8160196)(144.49376512,397.80602054)
\lineto(144.64376512,397.80602054)
\curveto(144.70375967,397.80601961)(144.76375961,397.80101961)(144.82376512,397.79102054)
\curveto(144.89375948,397.79101962)(144.95375942,397.80101961)(145.00376512,397.82102054)
\curveto(145.05375932,397.83101958)(145.10375927,397.83101958)(145.15376512,397.82102054)
\curveto(145.21375916,397.82101959)(145.26875911,397.82601959)(145.31876512,397.83602054)
\curveto(145.43875894,397.86601955)(145.55375882,397.89101952)(145.66376512,397.91102054)
\curveto(145.7737586,397.93101948)(145.88375849,397.96601945)(145.99376512,398.01602054)
\curveto(146.373758,398.16601925)(146.66875771,398.36101905)(146.87876512,398.60102054)
\curveto(147.09875728,398.84101857)(147.2737571,399.16101825)(147.40376512,399.56102054)
\curveto(147.48375689,399.8110176)(147.52875685,400.09101732)(147.53876512,400.40102054)
\curveto(147.55875682,400.72101669)(147.56875681,401.04601637)(147.56876512,401.37602054)
\lineto(147.56876512,403.83602054)
\lineto(147.56876512,406.43102054)
\curveto(147.56875681,406.60101081)(147.56375681,406.79101062)(147.55376512,407.00102054)
\curveto(147.55375682,407.22101019)(147.59375678,407.37601004)(147.67376512,407.46602054)
\curveto(147.72375665,407.53600988)(147.81375656,407.57100984)(147.94376512,407.57102054)
\curveto(148.08375629,407.57100984)(148.21875616,407.57100984)(148.34876512,407.57102054)
\curveto(148.39875598,407.57100984)(148.44375593,407.56600985)(148.48376512,407.55602054)
\curveto(148.53375584,407.55600986)(148.58375579,407.55600986)(148.63376512,407.55602054)
\curveto(148.7737556,407.52600989)(148.86375551,407.47600994)(148.90376512,407.40602054)
\curveto(148.95375542,407.32601009)(148.9737554,407.2110102)(148.96376512,407.06102054)
\lineto(148.96376512,406.65602054)
\lineto(148.96376512,404.58602054)
\lineto(148.96376512,401.46602054)
\curveto(148.96375541,401.09601632)(148.95875542,400.73101668)(148.94876512,400.37102054)
\curveto(148.93875544,400.0110174)(148.89375548,399.68601773)(148.81376512,399.39602054)
\curveto(148.6737557,398.90601851)(148.48875589,398.48601893)(148.25876512,398.13602054)
\curveto(148.03875634,397.79601962)(147.73875664,397.50601991)(147.35876512,397.26602054)
\curveto(147.20875717,397.17602024)(147.04875733,397.09602032)(146.87876512,397.02602054)
\curveto(146.70875767,396.95602046)(146.52875785,396.89102052)(146.33876512,396.83102054)
\curveto(146.24875813,396.80102061)(146.15875822,396.78102063)(146.06876512,396.77102054)
\curveto(145.9787584,396.75102066)(145.88375849,396.73102068)(145.78376512,396.71102054)
\curveto(145.68375869,396.69102072)(145.5787588,396.68102073)(145.46876512,396.68102054)
\curveto(145.36875901,396.68102073)(145.26875911,396.67102074)(145.16876512,396.65102054)
\lineto(144.98876512,396.65102054)
\curveto(144.93875944,396.64102077)(144.85875952,396.63602078)(144.74876512,396.63602054)
\curveto(144.63875974,396.63602078)(144.55875982,396.64102077)(144.50876512,396.65102054)
\lineto(144.32876512,396.65102054)
\curveto(144.25876012,396.67102074)(144.18376019,396.68102073)(144.10376512,396.68102054)
\curveto(144.03376034,396.67102074)(143.96376041,396.67602074)(143.89376512,396.69602054)
\lineto(143.77376512,396.69602054)
\curveto(143.69376068,396.7160207)(143.61376076,396.73102068)(143.53376512,396.74102054)
\curveto(143.46376091,396.75102066)(143.38876099,396.76602065)(143.30876512,396.78602054)
\curveto(143.13876124,396.83602058)(142.96876141,396.88102053)(142.79876512,396.92102054)
\curveto(142.63876174,396.97102044)(142.48376189,397.03102038)(142.33376512,397.10102054)
\curveto(142.26376211,397.13102028)(142.19376218,397.16602025)(142.12376512,397.20602054)
\curveto(142.06376231,397.24602017)(141.99876238,397.29102012)(141.92876512,397.34102054)
\curveto(141.62876275,397.53101988)(141.378763,397.75601966)(141.17876512,398.01602054)
\curveto(140.9787634,398.27601914)(140.80876357,398.58101883)(140.66876512,398.93102054)
\curveto(140.60876377,399.08101833)(140.56376381,399.23601818)(140.53376512,399.39602054)
\curveto(140.50376387,399.55601786)(140.46876391,399.72101769)(140.42876512,399.89102054)
\curveto(140.41876396,399.97101744)(140.40876397,400.04601737)(140.39876512,400.11602054)
\curveto(140.39876398,400.19601722)(140.39376398,400.27601714)(140.38376512,400.35602054)
\lineto(140.38376512,400.50602054)
\curveto(140.36376401,400.58601683)(140.35876402,400.66601675)(140.36876512,400.74602054)
\curveto(140.378764,400.83601658)(140.38376399,400.92101649)(140.38376512,401.00102054)
\lineto(140.38376512,401.97602054)
\lineto(140.38376512,405.89102054)
\lineto(140.38376512,406.91102054)
\lineto(140.38376512,407.18102054)
\curveto(140.38376399,407.27101014)(140.40376397,407.34601007)(140.44376512,407.40602054)
\curveto(140.48376389,407.47600994)(140.55876382,407.52600989)(140.66876512,407.55602054)
\curveto(140.72876365,407.55600986)(140.76876361,407.56100985)(140.78876512,407.57102054)
}
}
{
\newrgbcolor{curcolor}{0 0 0}
\pscustom[linestyle=none,fillstyle=solid,fillcolor=curcolor]
{
\newpath
\moveto(153.44360887,404.79602054)
\curveto(154.1636048,404.80601261)(154.7686042,404.72101269)(155.25860887,404.54102054)
\curveto(155.74860322,404.37101304)(156.12860284,404.06601335)(156.39860887,403.62602054)
\curveto(156.4686025,403.5160139)(156.52360244,403.40101401)(156.56360887,403.28102054)
\curveto(156.60360236,403.17101424)(156.64360232,403.04601437)(156.68360887,402.90602054)
\curveto(156.70360226,402.83601458)(156.70860226,402.76101465)(156.69860887,402.68102054)
\curveto(156.68860228,402.6110148)(156.67360229,402.55601486)(156.65360887,402.51602054)
\curveto(156.63360233,402.49601492)(156.60860236,402.47601494)(156.57860887,402.45602054)
\curveto(156.54860242,402.44601497)(156.52360244,402.43101498)(156.50360887,402.41102054)
\curveto(156.45360251,402.39101502)(156.40360256,402.38601503)(156.35360887,402.39602054)
\curveto(156.30360266,402.40601501)(156.25360271,402.40601501)(156.20360887,402.39602054)
\curveto(156.12360284,402.37601504)(156.01860295,402.37101504)(155.88860887,402.38102054)
\curveto(155.75860321,402.40101501)(155.6686033,402.42601499)(155.61860887,402.45602054)
\curveto(155.53860343,402.50601491)(155.48360348,402.57101484)(155.45360887,402.65102054)
\curveto(155.43360353,402.74101467)(155.39860357,402.82601459)(155.34860887,402.90602054)
\curveto(155.25860371,403.06601435)(155.13360383,403.2110142)(154.97360887,403.34102054)
\curveto(154.8636041,403.42101399)(154.74360422,403.48101393)(154.61360887,403.52102054)
\curveto(154.48360448,403.56101385)(154.34360462,403.60101381)(154.19360887,403.64102054)
\curveto(154.14360482,403.66101375)(154.09360487,403.66601375)(154.04360887,403.65602054)
\curveto(153.99360497,403.65601376)(153.94360502,403.66101375)(153.89360887,403.67102054)
\curveto(153.83360513,403.69101372)(153.75860521,403.70101371)(153.66860887,403.70102054)
\curveto(153.57860539,403.70101371)(153.50360546,403.69101372)(153.44360887,403.67102054)
\lineto(153.35360887,403.67102054)
\lineto(153.20360887,403.64102054)
\curveto(153.15360581,403.64101377)(153.10360586,403.63601378)(153.05360887,403.62602054)
\curveto(152.79360617,403.56601385)(152.57860639,403.48101393)(152.40860887,403.37102054)
\curveto(152.23860673,403.26101415)(152.12360684,403.07601434)(152.06360887,402.81602054)
\curveto(152.04360692,402.74601467)(152.03860693,402.67601474)(152.04860887,402.60602054)
\curveto(152.0686069,402.53601488)(152.08860688,402.47601494)(152.10860887,402.42602054)
\curveto(152.1686068,402.27601514)(152.23860673,402.16601525)(152.31860887,402.09602054)
\curveto(152.40860656,402.03601538)(152.51860645,401.96601545)(152.64860887,401.88602054)
\curveto(152.80860616,401.78601563)(152.98860598,401.7110157)(153.18860887,401.66102054)
\curveto(153.38860558,401.62101579)(153.58860538,401.57101584)(153.78860887,401.51102054)
\curveto(153.91860505,401.47101594)(154.04860492,401.44101597)(154.17860887,401.42102054)
\curveto(154.30860466,401.40101601)(154.43860453,401.37101604)(154.56860887,401.33102054)
\curveto(154.77860419,401.27101614)(154.98360398,401.2110162)(155.18360887,401.15102054)
\curveto(155.38360358,401.10101631)(155.58360338,401.03601638)(155.78360887,400.95602054)
\lineto(155.93360887,400.89602054)
\curveto(155.98360298,400.87601654)(156.03360293,400.85101656)(156.08360887,400.82102054)
\curveto(156.28360268,400.70101671)(156.45860251,400.56601685)(156.60860887,400.41602054)
\curveto(156.75860221,400.26601715)(156.88360208,400.07601734)(156.98360887,399.84602054)
\curveto(157.00360196,399.77601764)(157.02360194,399.68101773)(157.04360887,399.56102054)
\curveto(157.0636019,399.49101792)(157.07360189,399.416018)(157.07360887,399.33602054)
\curveto(157.08360188,399.26601815)(157.08860188,399.18601823)(157.08860887,399.09602054)
\lineto(157.08860887,398.94602054)
\curveto(157.0686019,398.87601854)(157.05860191,398.80601861)(157.05860887,398.73602054)
\curveto(157.05860191,398.66601875)(157.04860192,398.59601882)(157.02860887,398.52602054)
\curveto(156.99860197,398.416019)(156.963602,398.3110191)(156.92360887,398.21102054)
\curveto(156.88360208,398.1110193)(156.83860213,398.02101939)(156.78860887,397.94102054)
\curveto(156.62860234,397.68101973)(156.42360254,397.47101994)(156.17360887,397.31102054)
\curveto(155.92360304,397.16102025)(155.64360332,397.03102038)(155.33360887,396.92102054)
\curveto(155.24360372,396.89102052)(155.14860382,396.87102054)(155.04860887,396.86102054)
\curveto(154.95860401,396.84102057)(154.8686041,396.8160206)(154.77860887,396.78602054)
\curveto(154.67860429,396.76602065)(154.57860439,396.75602066)(154.47860887,396.75602054)
\curveto(154.37860459,396.75602066)(154.27860469,396.74602067)(154.17860887,396.72602054)
\lineto(154.02860887,396.72602054)
\curveto(153.97860499,396.7160207)(153.90860506,396.7110207)(153.81860887,396.71102054)
\curveto(153.72860524,396.7110207)(153.65860531,396.7160207)(153.60860887,396.72602054)
\lineto(153.44360887,396.72602054)
\curveto(153.38360558,396.74602067)(153.31860565,396.75602066)(153.24860887,396.75602054)
\curveto(153.17860579,396.74602067)(153.11860585,396.75102066)(153.06860887,396.77102054)
\curveto(153.01860595,396.78102063)(152.95360601,396.78602063)(152.87360887,396.78602054)
\lineto(152.63360887,396.84602054)
\curveto(152.5636064,396.85602056)(152.48860648,396.87602054)(152.40860887,396.90602054)
\curveto(152.09860687,397.00602041)(151.82860714,397.13102028)(151.59860887,397.28102054)
\curveto(151.3686076,397.43101998)(151.1686078,397.62601979)(150.99860887,397.86602054)
\curveto(150.90860806,397.99601942)(150.83360813,398.13101928)(150.77360887,398.27102054)
\curveto(150.71360825,398.411019)(150.65860831,398.56601885)(150.60860887,398.73602054)
\curveto(150.58860838,398.79601862)(150.57860839,398.86601855)(150.57860887,398.94602054)
\curveto(150.58860838,399.03601838)(150.60360836,399.10601831)(150.62360887,399.15602054)
\curveto(150.65360831,399.19601822)(150.70360826,399.23601818)(150.77360887,399.27602054)
\curveto(150.82360814,399.29601812)(150.89360807,399.30601811)(150.98360887,399.30602054)
\curveto(151.07360789,399.3160181)(151.1636078,399.3160181)(151.25360887,399.30602054)
\curveto(151.34360762,399.29601812)(151.42860754,399.28101813)(151.50860887,399.26102054)
\curveto(151.59860737,399.25101816)(151.65860731,399.23601818)(151.68860887,399.21602054)
\curveto(151.75860721,399.16601825)(151.80360716,399.09101832)(151.82360887,398.99102054)
\curveto(151.85360711,398.90101851)(151.88860708,398.8160186)(151.92860887,398.73602054)
\curveto(152.02860694,398.5160189)(152.1636068,398.34601907)(152.33360887,398.22602054)
\curveto(152.45360651,398.13601928)(152.58860638,398.06601935)(152.73860887,398.01602054)
\curveto(152.88860608,397.96601945)(153.04860592,397.9160195)(153.21860887,397.86602054)
\lineto(153.53360887,397.82102054)
\lineto(153.62360887,397.82102054)
\curveto(153.69360527,397.80101961)(153.78360518,397.79101962)(153.89360887,397.79102054)
\curveto(154.01360495,397.79101962)(154.11360485,397.80101961)(154.19360887,397.82102054)
\curveto(154.2636047,397.82101959)(154.31860465,397.82601959)(154.35860887,397.83602054)
\curveto(154.41860455,397.84601957)(154.47860449,397.85101956)(154.53860887,397.85102054)
\curveto(154.59860437,397.86101955)(154.65360431,397.87101954)(154.70360887,397.88102054)
\curveto(154.99360397,397.96101945)(155.22360374,398.06601935)(155.39360887,398.19602054)
\curveto(155.5636034,398.32601909)(155.68360328,398.54601887)(155.75360887,398.85602054)
\curveto(155.77360319,398.90601851)(155.77860319,398.96101845)(155.76860887,399.02102054)
\curveto(155.75860321,399.08101833)(155.74860322,399.12601829)(155.73860887,399.15602054)
\curveto(155.68860328,399.34601807)(155.61860335,399.48601793)(155.52860887,399.57602054)
\curveto(155.43860353,399.67601774)(155.32360364,399.76601765)(155.18360887,399.84602054)
\curveto(155.09360387,399.90601751)(154.99360397,399.95601746)(154.88360887,399.99602054)
\lineto(154.55360887,400.11602054)
\curveto(154.52360444,400.12601729)(154.49360447,400.13101728)(154.46360887,400.13102054)
\curveto(154.44360452,400.13101728)(154.41860455,400.14101727)(154.38860887,400.16102054)
\curveto(154.04860492,400.27101714)(153.69360527,400.35101706)(153.32360887,400.40102054)
\curveto(152.963606,400.46101695)(152.62360634,400.55601686)(152.30360887,400.68602054)
\curveto(152.20360676,400.72601669)(152.10860686,400.76101665)(152.01860887,400.79102054)
\curveto(151.92860704,400.82101659)(151.84360712,400.86101655)(151.76360887,400.91102054)
\curveto(151.57360739,401.02101639)(151.39860757,401.14601627)(151.23860887,401.28602054)
\curveto(151.07860789,401.42601599)(150.95360801,401.60101581)(150.86360887,401.81102054)
\curveto(150.83360813,401.88101553)(150.80860816,401.95101546)(150.78860887,402.02102054)
\curveto(150.77860819,402.09101532)(150.7636082,402.16601525)(150.74360887,402.24602054)
\curveto(150.71360825,402.36601505)(150.70360826,402.50101491)(150.71360887,402.65102054)
\curveto(150.72360824,402.8110146)(150.73860823,402.94601447)(150.75860887,403.05602054)
\curveto(150.77860819,403.10601431)(150.78860818,403.14601427)(150.78860887,403.17602054)
\curveto(150.79860817,403.2160142)(150.81360815,403.25601416)(150.83360887,403.29602054)
\curveto(150.92360804,403.52601389)(151.04360792,403.72601369)(151.19360887,403.89602054)
\curveto(151.35360761,404.06601335)(151.53360743,404.2160132)(151.73360887,404.34602054)
\curveto(151.88360708,404.43601298)(152.04860692,404.50601291)(152.22860887,404.55602054)
\curveto(152.40860656,404.6160128)(152.59860637,404.67101274)(152.79860887,404.72102054)
\curveto(152.8686061,404.73101268)(152.93360603,404.74101267)(152.99360887,404.75102054)
\curveto(153.0636059,404.76101265)(153.13860583,404.77101264)(153.21860887,404.78102054)
\curveto(153.24860572,404.79101262)(153.28860568,404.79101262)(153.33860887,404.78102054)
\curveto(153.38860558,404.77101264)(153.42360554,404.77601264)(153.44360887,404.79602054)
}
}
{
\newrgbcolor{curcolor}{0 0 0}
\pscustom[linestyle=none,fillstyle=solid,fillcolor=curcolor]
{
\newpath
\moveto(158.97860887,404.61602054)
\lineto(159.41360887,404.61602054)
\curveto(159.5636069,404.6160128)(159.6686068,404.57601284)(159.72860887,404.49602054)
\curveto(159.77860669,404.416013)(159.80360666,404.3160131)(159.80360887,404.19602054)
\curveto(159.81360665,404.07601334)(159.81860665,403.95601346)(159.81860887,403.83602054)
\lineto(159.81860887,402.41102054)
\lineto(159.81860887,400.14602054)
\lineto(159.81860887,399.45602054)
\curveto(159.81860665,399.22601819)(159.84360662,399.02601839)(159.89360887,398.85602054)
\curveto(160.05360641,398.40601901)(160.35360611,398.09101932)(160.79360887,397.91102054)
\curveto(161.01360545,397.82101959)(161.27860519,397.78601963)(161.58860887,397.80602054)
\curveto(161.89860457,397.83601958)(162.14860432,397.89101952)(162.33860887,397.97102054)
\curveto(162.6686038,398.1110193)(162.92860354,398.28601913)(163.11860887,398.49602054)
\curveto(163.31860315,398.7160187)(163.47360299,399.00101841)(163.58360887,399.35102054)
\curveto(163.61360285,399.43101798)(163.63360283,399.5110179)(163.64360887,399.59102054)
\curveto(163.65360281,399.67101774)(163.6686028,399.75601766)(163.68860887,399.84602054)
\curveto(163.69860277,399.89601752)(163.69860277,399.94101747)(163.68860887,399.98102054)
\curveto(163.68860278,400.02101739)(163.69860277,400.06601735)(163.71860887,400.11602054)
\lineto(163.71860887,400.43102054)
\curveto(163.73860273,400.5110169)(163.74360272,400.60101681)(163.73360887,400.70102054)
\curveto(163.72360274,400.8110166)(163.71860275,400.9110165)(163.71860887,401.00102054)
\lineto(163.71860887,402.17102054)
\lineto(163.71860887,403.76102054)
\curveto(163.71860275,403.88101353)(163.71360275,404.00601341)(163.70360887,404.13602054)
\curveto(163.70360276,404.27601314)(163.72860274,404.38601303)(163.77860887,404.46602054)
\curveto(163.81860265,404.5160129)(163.8636026,404.54601287)(163.91360887,404.55602054)
\curveto(163.97360249,404.57601284)(164.04360242,404.59601282)(164.12360887,404.61602054)
\lineto(164.34860887,404.61602054)
\curveto(164.468602,404.6160128)(164.57360189,404.6110128)(164.66360887,404.60102054)
\curveto(164.7636017,404.59101282)(164.83860163,404.54601287)(164.88860887,404.46602054)
\curveto(164.93860153,404.416013)(164.9636015,404.34101307)(164.96360887,404.24102054)
\lineto(164.96360887,403.95602054)
\lineto(164.96360887,402.93602054)
\lineto(164.96360887,398.90102054)
\lineto(164.96360887,397.55102054)
\curveto(164.9636015,397.43101998)(164.95860151,397.3160201)(164.94860887,397.20602054)
\curveto(164.94860152,397.10602031)(164.91360155,397.03102038)(164.84360887,396.98102054)
\curveto(164.80360166,396.95102046)(164.74360172,396.92602049)(164.66360887,396.90602054)
\curveto(164.58360188,396.89602052)(164.49360197,396.88602053)(164.39360887,396.87602054)
\curveto(164.30360216,396.87602054)(164.21360225,396.88102053)(164.12360887,396.89102054)
\curveto(164.04360242,396.90102051)(163.98360248,396.92102049)(163.94360887,396.95102054)
\curveto(163.89360257,396.99102042)(163.84860262,397.05602036)(163.80860887,397.14602054)
\curveto(163.79860267,397.18602023)(163.78860268,397.24102017)(163.77860887,397.31102054)
\curveto(163.77860269,397.38102003)(163.77360269,397.44601997)(163.76360887,397.50602054)
\curveto(163.75360271,397.57601984)(163.73360273,397.63101978)(163.70360887,397.67102054)
\curveto(163.67360279,397.7110197)(163.62860284,397.72601969)(163.56860887,397.71602054)
\curveto(163.48860298,397.69601972)(163.40860306,397.63601978)(163.32860887,397.53602054)
\curveto(163.24860322,397.44601997)(163.17360329,397.37602004)(163.10360887,397.32602054)
\curveto(162.88360358,397.16602025)(162.63360383,397.02602039)(162.35360887,396.90602054)
\curveto(162.24360422,396.85602056)(162.12860434,396.82602059)(162.00860887,396.81602054)
\curveto(161.89860457,396.79602062)(161.78360468,396.77102064)(161.66360887,396.74102054)
\curveto(161.61360485,396.73102068)(161.55860491,396.73102068)(161.49860887,396.74102054)
\curveto(161.44860502,396.75102066)(161.39860507,396.74602067)(161.34860887,396.72602054)
\curveto(161.24860522,396.70602071)(161.15860531,396.70602071)(161.07860887,396.72602054)
\lineto(160.92860887,396.72602054)
\curveto(160.87860559,396.74602067)(160.81860565,396.75602066)(160.74860887,396.75602054)
\curveto(160.68860578,396.75602066)(160.63360583,396.76102065)(160.58360887,396.77102054)
\curveto(160.54360592,396.79102062)(160.50360596,396.80102061)(160.46360887,396.80102054)
\curveto(160.43360603,396.79102062)(160.39360607,396.79602062)(160.34360887,396.81602054)
\lineto(160.10360887,396.87602054)
\curveto(160.03360643,396.89602052)(159.95860651,396.92602049)(159.87860887,396.96602054)
\curveto(159.61860685,397.07602034)(159.39860707,397.22102019)(159.21860887,397.40102054)
\curveto(159.04860742,397.59101982)(158.90860756,397.8160196)(158.79860887,398.07602054)
\curveto(158.75860771,398.16601925)(158.72860774,398.25601916)(158.70860887,398.34602054)
\lineto(158.64860887,398.64602054)
\curveto(158.62860784,398.70601871)(158.61860785,398.76101865)(158.61860887,398.81102054)
\curveto(158.62860784,398.87101854)(158.62360784,398.93601848)(158.60360887,399.00602054)
\curveto(158.59360787,399.02601839)(158.58860788,399.05101836)(158.58860887,399.08102054)
\curveto(158.58860788,399.12101829)(158.58360788,399.15601826)(158.57360887,399.18602054)
\lineto(158.57360887,399.33602054)
\curveto(158.5636079,399.37601804)(158.55860791,399.42101799)(158.55860887,399.47102054)
\curveto(158.5686079,399.53101788)(158.57360789,399.58601783)(158.57360887,399.63602054)
\lineto(158.57360887,400.23602054)
\lineto(158.57360887,402.99602054)
\lineto(158.57360887,403.95602054)
\lineto(158.57360887,404.22602054)
\curveto(158.57360789,404.3160131)(158.59360787,404.39101302)(158.63360887,404.45102054)
\curveto(158.67360779,404.52101289)(158.74860772,404.57101284)(158.85860887,404.60102054)
\curveto(158.87860759,404.6110128)(158.89860757,404.6110128)(158.91860887,404.60102054)
\curveto(158.93860753,404.60101281)(158.95860751,404.60601281)(158.97860887,404.61602054)
}
}
{
\newrgbcolor{curcolor}{0 0 0}
\pscustom[linestyle=none,fillstyle=solid,fillcolor=curcolor]
{
\newpath
\moveto(173.74821824,397.44602054)
\curveto(173.77821041,397.28602013)(173.76321043,397.15102026)(173.70321824,397.04102054)
\curveto(173.64321055,396.94102047)(173.56321063,396.86602055)(173.46321824,396.81602054)
\curveto(173.41321078,396.79602062)(173.35821083,396.78602063)(173.29821824,396.78602054)
\curveto(173.24821094,396.78602063)(173.193211,396.77602064)(173.13321824,396.75602054)
\curveto(172.91321128,396.70602071)(172.6932115,396.72102069)(172.47321824,396.80102054)
\curveto(172.26321193,396.87102054)(172.11821207,396.96102045)(172.03821824,397.07102054)
\curveto(171.9882122,397.14102027)(171.94321225,397.22102019)(171.90321824,397.31102054)
\curveto(171.86321233,397.41102)(171.81321238,397.49101992)(171.75321824,397.55102054)
\curveto(171.73321246,397.57101984)(171.70821248,397.59101982)(171.67821824,397.61102054)
\curveto(171.65821253,397.63101978)(171.62821256,397.63601978)(171.58821824,397.62602054)
\curveto(171.47821271,397.59601982)(171.37321282,397.54101987)(171.27321824,397.46102054)
\curveto(171.18321301,397.38102003)(171.0932131,397.3110201)(171.00321824,397.25102054)
\curveto(170.87321332,397.17102024)(170.73321346,397.09602032)(170.58321824,397.02602054)
\curveto(170.43321376,396.96602045)(170.27321392,396.9110205)(170.10321824,396.86102054)
\curveto(170.00321419,396.83102058)(169.8932143,396.8110206)(169.77321824,396.80102054)
\curveto(169.66321453,396.79102062)(169.55321464,396.77602064)(169.44321824,396.75602054)
\curveto(169.3932148,396.74602067)(169.34821484,396.74102067)(169.30821824,396.74102054)
\lineto(169.20321824,396.74102054)
\curveto(169.0932151,396.72102069)(168.9882152,396.72102069)(168.88821824,396.74102054)
\lineto(168.75321824,396.74102054)
\curveto(168.70321549,396.75102066)(168.65321554,396.75602066)(168.60321824,396.75602054)
\curveto(168.55321564,396.75602066)(168.50821568,396.76602065)(168.46821824,396.78602054)
\curveto(168.42821576,396.79602062)(168.3932158,396.80102061)(168.36321824,396.80102054)
\curveto(168.34321585,396.79102062)(168.31821587,396.79102062)(168.28821824,396.80102054)
\lineto(168.04821824,396.86102054)
\curveto(167.96821622,396.87102054)(167.8932163,396.89102052)(167.82321824,396.92102054)
\curveto(167.52321667,397.05102036)(167.27821691,397.19602022)(167.08821824,397.35602054)
\curveto(166.90821728,397.52601989)(166.75821743,397.76101965)(166.63821824,398.06102054)
\curveto(166.54821764,398.28101913)(166.50321769,398.54601887)(166.50321824,398.85602054)
\lineto(166.50321824,399.17102054)
\curveto(166.51321768,399.22101819)(166.51821767,399.27101814)(166.51821824,399.32102054)
\lineto(166.54821824,399.50102054)
\lineto(166.66821824,399.83102054)
\curveto(166.70821748,399.94101747)(166.75821743,400.04101737)(166.81821824,400.13102054)
\curveto(166.99821719,400.42101699)(167.24321695,400.63601678)(167.55321824,400.77602054)
\curveto(167.86321633,400.9160165)(168.20321599,401.04101637)(168.57321824,401.15102054)
\curveto(168.71321548,401.19101622)(168.85821533,401.22101619)(169.00821824,401.24102054)
\curveto(169.15821503,401.26101615)(169.30821488,401.28601613)(169.45821824,401.31602054)
\curveto(169.52821466,401.33601608)(169.5932146,401.34601607)(169.65321824,401.34602054)
\curveto(169.72321447,401.34601607)(169.79821439,401.35601606)(169.87821824,401.37602054)
\curveto(169.94821424,401.39601602)(170.01821417,401.40601601)(170.08821824,401.40602054)
\curveto(170.15821403,401.416016)(170.23321396,401.43101598)(170.31321824,401.45102054)
\curveto(170.56321363,401.5110159)(170.79821339,401.56101585)(171.01821824,401.60102054)
\curveto(171.23821295,401.65101576)(171.41321278,401.76601565)(171.54321824,401.94602054)
\curveto(171.60321259,402.02601539)(171.65321254,402.12601529)(171.69321824,402.24602054)
\curveto(171.73321246,402.37601504)(171.73321246,402.5160149)(171.69321824,402.66602054)
\curveto(171.63321256,402.90601451)(171.54321265,403.09601432)(171.42321824,403.23602054)
\curveto(171.31321288,403.37601404)(171.15321304,403.48601393)(170.94321824,403.56602054)
\curveto(170.82321337,403.6160138)(170.67821351,403.65101376)(170.50821824,403.67102054)
\curveto(170.34821384,403.69101372)(170.17821401,403.70101371)(169.99821824,403.70102054)
\curveto(169.81821437,403.70101371)(169.64321455,403.69101372)(169.47321824,403.67102054)
\curveto(169.30321489,403.65101376)(169.15821503,403.62101379)(169.03821824,403.58102054)
\curveto(168.86821532,403.52101389)(168.70321549,403.43601398)(168.54321824,403.32602054)
\curveto(168.46321573,403.26601415)(168.3882158,403.18601423)(168.31821824,403.08602054)
\curveto(168.25821593,402.99601442)(168.20321599,402.89601452)(168.15321824,402.78602054)
\curveto(168.12321607,402.70601471)(168.0932161,402.62101479)(168.06321824,402.53102054)
\curveto(168.04321615,402.44101497)(167.99821619,402.37101504)(167.92821824,402.32102054)
\curveto(167.8882163,402.29101512)(167.81821637,402.26601515)(167.71821824,402.24602054)
\curveto(167.62821656,402.23601518)(167.53321666,402.23101518)(167.43321824,402.23102054)
\curveto(167.33321686,402.23101518)(167.23321696,402.23601518)(167.13321824,402.24602054)
\curveto(167.04321715,402.26601515)(166.97821721,402.29101512)(166.93821824,402.32102054)
\curveto(166.89821729,402.35101506)(166.86821732,402.40101501)(166.84821824,402.47102054)
\curveto(166.82821736,402.54101487)(166.82821736,402.6160148)(166.84821824,402.69602054)
\curveto(166.87821731,402.82601459)(166.90821728,402.94601447)(166.93821824,403.05602054)
\curveto(166.97821721,403.17601424)(167.02321717,403.29101412)(167.07321824,403.40102054)
\curveto(167.26321693,403.75101366)(167.50321669,404.02101339)(167.79321824,404.21102054)
\curveto(168.08321611,404.411013)(168.44321575,404.57101284)(168.87321824,404.69102054)
\curveto(168.97321522,404.7110127)(169.07321512,404.72601269)(169.17321824,404.73602054)
\curveto(169.28321491,404.74601267)(169.3932148,404.76101265)(169.50321824,404.78102054)
\curveto(169.54321465,404.79101262)(169.60821458,404.79101262)(169.69821824,404.78102054)
\curveto(169.7882144,404.78101263)(169.84321435,404.79101262)(169.86321824,404.81102054)
\curveto(170.56321363,404.82101259)(171.17321302,404.74101267)(171.69321824,404.57102054)
\curveto(172.21321198,404.40101301)(172.57821161,404.07601334)(172.78821824,403.59602054)
\curveto(172.87821131,403.39601402)(172.92821126,403.16101425)(172.93821824,402.89102054)
\curveto(172.95821123,402.63101478)(172.96821122,402.35601506)(172.96821824,402.06602054)
\lineto(172.96821824,398.75102054)
\curveto(172.96821122,398.6110188)(172.97321122,398.47601894)(172.98321824,398.34602054)
\curveto(172.9932112,398.2160192)(173.02321117,398.1110193)(173.07321824,398.03102054)
\curveto(173.12321107,397.96101945)(173.188211,397.9110195)(173.26821824,397.88102054)
\curveto(173.35821083,397.84101957)(173.44321075,397.8110196)(173.52321824,397.79102054)
\curveto(173.60321059,397.78101963)(173.66321053,397.73601968)(173.70321824,397.65602054)
\curveto(173.72321047,397.62601979)(173.73321046,397.59601982)(173.73321824,397.56602054)
\curveto(173.73321046,397.53601988)(173.73821045,397.49601992)(173.74821824,397.44602054)
\moveto(171.60321824,399.11102054)
\curveto(171.66321253,399.25101816)(171.6932125,399.411018)(171.69321824,399.59102054)
\curveto(171.70321249,399.78101763)(171.70821248,399.97601744)(171.70821824,400.17602054)
\curveto(171.70821248,400.28601713)(171.70321249,400.38601703)(171.69321824,400.47602054)
\curveto(171.68321251,400.56601685)(171.64321255,400.63601678)(171.57321824,400.68602054)
\curveto(171.54321265,400.70601671)(171.47321272,400.7160167)(171.36321824,400.71602054)
\curveto(171.34321285,400.69601672)(171.30821288,400.68601673)(171.25821824,400.68602054)
\curveto(171.20821298,400.68601673)(171.16321303,400.67601674)(171.12321824,400.65602054)
\curveto(171.04321315,400.63601678)(170.95321324,400.6160168)(170.85321824,400.59602054)
\lineto(170.55321824,400.53602054)
\curveto(170.52321367,400.53601688)(170.4882137,400.53101688)(170.44821824,400.52102054)
\lineto(170.34321824,400.52102054)
\curveto(170.193214,400.48101693)(170.02821416,400.45601696)(169.84821824,400.44602054)
\curveto(169.67821451,400.44601697)(169.51821467,400.42601699)(169.36821824,400.38602054)
\curveto(169.2882149,400.36601705)(169.21321498,400.34601707)(169.14321824,400.32602054)
\curveto(169.08321511,400.3160171)(169.01321518,400.30101711)(168.93321824,400.28102054)
\curveto(168.77321542,400.23101718)(168.62321557,400.16601725)(168.48321824,400.08602054)
\curveto(168.34321585,400.0160174)(168.22321597,399.92601749)(168.12321824,399.81602054)
\curveto(168.02321617,399.70601771)(167.94821624,399.57101784)(167.89821824,399.41102054)
\curveto(167.84821634,399.26101815)(167.82821636,399.07601834)(167.83821824,398.85602054)
\curveto(167.83821635,398.75601866)(167.85321634,398.66101875)(167.88321824,398.57102054)
\curveto(167.92321627,398.49101892)(167.96821622,398.416019)(168.01821824,398.34602054)
\curveto(168.09821609,398.23601918)(168.20321599,398.14101927)(168.33321824,398.06102054)
\curveto(168.46321573,397.99101942)(168.60321559,397.93101948)(168.75321824,397.88102054)
\curveto(168.80321539,397.87101954)(168.85321534,397.86601955)(168.90321824,397.86602054)
\curveto(168.95321524,397.86601955)(169.00321519,397.86101955)(169.05321824,397.85102054)
\curveto(169.12321507,397.83101958)(169.20821498,397.8160196)(169.30821824,397.80602054)
\curveto(169.41821477,397.80601961)(169.50821468,397.8160196)(169.57821824,397.83602054)
\curveto(169.63821455,397.85601956)(169.69821449,397.86101955)(169.75821824,397.85102054)
\curveto(169.81821437,397.85101956)(169.87821431,397.86101955)(169.93821824,397.88102054)
\curveto(170.01821417,397.90101951)(170.0932141,397.9160195)(170.16321824,397.92602054)
\curveto(170.24321395,397.93601948)(170.31821387,397.95601946)(170.38821824,397.98602054)
\curveto(170.67821351,398.10601931)(170.92321327,398.25101916)(171.12321824,398.42102054)
\curveto(171.33321286,398.59101882)(171.4932127,398.82101859)(171.60321824,399.11102054)
}
}
{
\newrgbcolor{curcolor}{0 0 0}
\pscustom[linestyle=none,fillstyle=solid,fillcolor=curcolor]
{
\newpath
\moveto(178.56485887,404.79602054)
\curveto(178.79485408,404.79601262)(178.92485395,404.73601268)(178.95485887,404.61602054)
\curveto(178.98485389,404.50601291)(178.99985387,404.34101307)(178.99985887,404.12102054)
\lineto(178.99985887,403.83602054)
\curveto(178.99985387,403.74601367)(178.9748539,403.67101374)(178.92485887,403.61102054)
\curveto(178.86485401,403.53101388)(178.77985409,403.48601393)(178.66985887,403.47602054)
\curveto(178.55985431,403.47601394)(178.44985442,403.46101395)(178.33985887,403.43102054)
\curveto(178.19985467,403.40101401)(178.06485481,403.37101404)(177.93485887,403.34102054)
\curveto(177.81485506,403.3110141)(177.69985517,403.27101414)(177.58985887,403.22102054)
\curveto(177.29985557,403.09101432)(177.06485581,402.9110145)(176.88485887,402.68102054)
\curveto(176.70485617,402.46101495)(176.54985632,402.20601521)(176.41985887,401.91602054)
\curveto(176.37985649,401.80601561)(176.34985652,401.69101572)(176.32985887,401.57102054)
\curveto(176.30985656,401.46101595)(176.28485659,401.34601607)(176.25485887,401.22602054)
\curveto(176.24485663,401.17601624)(176.23985663,401.12601629)(176.23985887,401.07602054)
\curveto(176.24985662,401.02601639)(176.24985662,400.97601644)(176.23985887,400.92602054)
\curveto(176.20985666,400.80601661)(176.19485668,400.66601675)(176.19485887,400.50602054)
\curveto(176.20485667,400.35601706)(176.20985666,400.2110172)(176.20985887,400.07102054)
\lineto(176.20985887,398.22602054)
\lineto(176.20985887,397.88102054)
\curveto(176.20985666,397.76101965)(176.20485667,397.64601977)(176.19485887,397.53602054)
\curveto(176.18485669,397.42601999)(176.17985669,397.33102008)(176.17985887,397.25102054)
\curveto(176.18985668,397.17102024)(176.1698567,397.10102031)(176.11985887,397.04102054)
\curveto(176.0698568,396.97102044)(175.98985688,396.93102048)(175.87985887,396.92102054)
\curveto(175.77985709,396.9110205)(175.6698572,396.90602051)(175.54985887,396.90602054)
\lineto(175.27985887,396.90602054)
\curveto(175.22985764,396.92602049)(175.17985769,396.94102047)(175.12985887,396.95102054)
\curveto(175.08985778,396.97102044)(175.05985781,396.99602042)(175.03985887,397.02602054)
\curveto(174.98985788,397.09602032)(174.95985791,397.18102023)(174.94985887,397.28102054)
\lineto(174.94985887,397.61102054)
\lineto(174.94985887,398.76602054)
\lineto(174.94985887,402.92102054)
\lineto(174.94985887,403.95602054)
\lineto(174.94985887,404.25602054)
\curveto(174.95985791,404.35601306)(174.98985788,404.44101297)(175.03985887,404.51102054)
\curveto(175.0698578,404.55101286)(175.11985775,404.58101283)(175.18985887,404.60102054)
\curveto(175.2698576,404.62101279)(175.35485752,404.63101278)(175.44485887,404.63102054)
\curveto(175.53485734,404.64101277)(175.62485725,404.64101277)(175.71485887,404.63102054)
\curveto(175.80485707,404.62101279)(175.874857,404.60601281)(175.92485887,404.58602054)
\curveto(176.00485687,404.55601286)(176.05485682,404.49601292)(176.07485887,404.40602054)
\curveto(176.10485677,404.32601309)(176.11985675,404.23601318)(176.11985887,404.13602054)
\lineto(176.11985887,403.83602054)
\curveto(176.11985675,403.73601368)(176.13985673,403.64601377)(176.17985887,403.56602054)
\curveto(176.18985668,403.54601387)(176.19985667,403.53101388)(176.20985887,403.52102054)
\lineto(176.25485887,403.47602054)
\curveto(176.36485651,403.47601394)(176.45485642,403.52101389)(176.52485887,403.61102054)
\curveto(176.59485628,403.7110137)(176.65485622,403.79101362)(176.70485887,403.85102054)
\lineto(176.79485887,403.94102054)
\curveto(176.88485599,404.05101336)(177.00985586,404.16601325)(177.16985887,404.28602054)
\curveto(177.32985554,404.40601301)(177.47985539,404.49601292)(177.61985887,404.55602054)
\curveto(177.70985516,404.60601281)(177.80485507,404.64101277)(177.90485887,404.66102054)
\curveto(178.00485487,404.69101272)(178.10985476,404.72101269)(178.21985887,404.75102054)
\curveto(178.27985459,404.76101265)(178.33985453,404.76601265)(178.39985887,404.76602054)
\curveto(178.45985441,404.77601264)(178.51485436,404.78601263)(178.56485887,404.79602054)
}
}
{
\newrgbcolor{curcolor}{0 0 0}
\pscustom[linestyle=none,fillstyle=solid,fillcolor=curcolor]
{
\newpath
\moveto(180.21462449,406.11602054)
\curveto(180.13462337,406.17601124)(180.08962342,406.28101113)(180.07962449,406.43102054)
\lineto(180.07962449,406.89602054)
\lineto(180.07962449,407.15102054)
\curveto(180.07962343,407.24101017)(180.09462341,407.3160101)(180.12462449,407.37602054)
\curveto(180.16462334,407.45600996)(180.24462326,407.5160099)(180.36462449,407.55602054)
\curveto(180.38462312,407.56600985)(180.4046231,407.56600985)(180.42462449,407.55602054)
\curveto(180.45462305,407.55600986)(180.47962303,407.56100985)(180.49962449,407.57102054)
\curveto(180.66962284,407.57100984)(180.82962268,407.56600985)(180.97962449,407.55602054)
\curveto(181.12962238,407.54600987)(181.22962228,407.48600993)(181.27962449,407.37602054)
\curveto(181.3096222,407.3160101)(181.32462218,407.24101017)(181.32462449,407.15102054)
\lineto(181.32462449,406.89602054)
\curveto(181.32462218,406.7160107)(181.31962219,406.54601087)(181.30962449,406.38602054)
\curveto(181.3096222,406.22601119)(181.24462226,406.12101129)(181.11462449,406.07102054)
\curveto(181.06462244,406.05101136)(181.0096225,406.04101137)(180.94962449,406.04102054)
\lineto(180.78462449,406.04102054)
\lineto(180.46962449,406.04102054)
\curveto(180.36962314,406.04101137)(180.28462322,406.06601135)(180.21462449,406.11602054)
\moveto(181.32462449,397.61102054)
\lineto(181.32462449,397.29602054)
\curveto(181.33462217,397.19602022)(181.31462219,397.1160203)(181.26462449,397.05602054)
\curveto(181.23462227,396.99602042)(181.18962232,396.95602046)(181.12962449,396.93602054)
\curveto(181.06962244,396.92602049)(180.99962251,396.9110205)(180.91962449,396.89102054)
\lineto(180.69462449,396.89102054)
\curveto(180.56462294,396.89102052)(180.44962306,396.89602052)(180.34962449,396.90602054)
\curveto(180.25962325,396.92602049)(180.18962332,396.97602044)(180.13962449,397.05602054)
\curveto(180.09962341,397.1160203)(180.07962343,397.19102022)(180.07962449,397.28102054)
\lineto(180.07962449,397.56602054)
\lineto(180.07962449,403.91102054)
\lineto(180.07962449,404.22602054)
\curveto(180.07962343,404.33601308)(180.1046234,404.42101299)(180.15462449,404.48102054)
\curveto(180.18462332,404.53101288)(180.22462328,404.56101285)(180.27462449,404.57102054)
\curveto(180.32462318,404.58101283)(180.37962313,404.59601282)(180.43962449,404.61602054)
\curveto(180.45962305,404.6160128)(180.47962303,404.6110128)(180.49962449,404.60102054)
\curveto(180.52962298,404.60101281)(180.55462295,404.60601281)(180.57462449,404.61602054)
\curveto(180.7046228,404.6160128)(180.83462267,404.6110128)(180.96462449,404.60102054)
\curveto(181.1046224,404.60101281)(181.19962231,404.56101285)(181.24962449,404.48102054)
\curveto(181.29962221,404.42101299)(181.32462218,404.34101307)(181.32462449,404.24102054)
\lineto(181.32462449,403.95602054)
\lineto(181.32462449,397.61102054)
}
}
{
\newrgbcolor{curcolor}{0 0 0}
\pscustom[linestyle=none,fillstyle=solid,fillcolor=curcolor]
{
\newpath
\moveto(190.39446824,401.09102054)
\curveto(190.41446018,401.03101638)(190.42446017,400.93601648)(190.42446824,400.80602054)
\curveto(190.42446017,400.68601673)(190.41946018,400.60101681)(190.40946824,400.55102054)
\lineto(190.40946824,400.40102054)
\curveto(190.3994602,400.32101709)(190.38946021,400.24601717)(190.37946824,400.17602054)
\curveto(190.37946022,400.1160173)(190.37446022,400.04601737)(190.36446824,399.96602054)
\curveto(190.34446025,399.90601751)(190.32946027,399.84601757)(190.31946824,399.78602054)
\curveto(190.31946028,399.72601769)(190.30946029,399.66601775)(190.28946824,399.60602054)
\curveto(190.24946035,399.47601794)(190.21446038,399.34601807)(190.18446824,399.21602054)
\curveto(190.15446044,399.08601833)(190.11446048,398.96601845)(190.06446824,398.85602054)
\curveto(189.85446074,398.37601904)(189.57446102,397.97101944)(189.22446824,397.64102054)
\curveto(188.87446172,397.32102009)(188.44446215,397.07602034)(187.93446824,396.90602054)
\curveto(187.82446277,396.86602055)(187.70446289,396.83602058)(187.57446824,396.81602054)
\curveto(187.45446314,396.79602062)(187.32946327,396.77602064)(187.19946824,396.75602054)
\curveto(187.13946346,396.74602067)(187.07446352,396.74102067)(187.00446824,396.74102054)
\curveto(186.94446365,396.73102068)(186.88446371,396.72602069)(186.82446824,396.72602054)
\curveto(186.78446381,396.7160207)(186.72446387,396.7110207)(186.64446824,396.71102054)
\curveto(186.57446402,396.7110207)(186.52446407,396.7160207)(186.49446824,396.72602054)
\curveto(186.45446414,396.73602068)(186.41446418,396.74102067)(186.37446824,396.74102054)
\curveto(186.33446426,396.73102068)(186.2994643,396.73102068)(186.26946824,396.74102054)
\lineto(186.17946824,396.74102054)
\lineto(185.81946824,396.78602054)
\curveto(185.67946492,396.82602059)(185.54446505,396.86602055)(185.41446824,396.90602054)
\curveto(185.28446531,396.94602047)(185.15946544,396.99102042)(185.03946824,397.04102054)
\curveto(184.58946601,397.24102017)(184.21946638,397.50101991)(183.92946824,397.82102054)
\curveto(183.63946696,398.14101927)(183.3994672,398.53101888)(183.20946824,398.99102054)
\curveto(183.15946744,399.09101832)(183.11946748,399.19101822)(183.08946824,399.29102054)
\curveto(183.06946753,399.39101802)(183.04946755,399.49601792)(183.02946824,399.60602054)
\curveto(183.00946759,399.64601777)(182.9994676,399.67601774)(182.99946824,399.69602054)
\curveto(183.00946759,399.72601769)(183.00946759,399.76101765)(182.99946824,399.80102054)
\curveto(182.97946762,399.88101753)(182.96446763,399.96101745)(182.95446824,400.04102054)
\curveto(182.95446764,400.13101728)(182.94446765,400.2160172)(182.92446824,400.29602054)
\lineto(182.92446824,400.41602054)
\curveto(182.92446767,400.45601696)(182.91946768,400.50101691)(182.90946824,400.55102054)
\curveto(182.8994677,400.60101681)(182.8944677,400.68601673)(182.89446824,400.80602054)
\curveto(182.8944677,400.93601648)(182.90446769,401.03101638)(182.92446824,401.09102054)
\curveto(182.94446765,401.16101625)(182.94946765,401.23101618)(182.93946824,401.30102054)
\curveto(182.92946767,401.37101604)(182.93446766,401.44101597)(182.95446824,401.51102054)
\curveto(182.96446763,401.56101585)(182.96946763,401.60101581)(182.96946824,401.63102054)
\curveto(182.97946762,401.67101574)(182.98946761,401.7160157)(182.99946824,401.76602054)
\curveto(183.02946757,401.88601553)(183.05446754,402.00601541)(183.07446824,402.12602054)
\curveto(183.10446749,402.24601517)(183.14446745,402.36101505)(183.19446824,402.47102054)
\curveto(183.34446725,402.84101457)(183.52446707,403.17101424)(183.73446824,403.46102054)
\curveto(183.95446664,403.76101365)(184.21946638,404.0110134)(184.52946824,404.21102054)
\curveto(184.64946595,404.29101312)(184.77446582,404.35601306)(184.90446824,404.40602054)
\curveto(185.03446556,404.46601295)(185.16946543,404.52601289)(185.30946824,404.58602054)
\curveto(185.42946517,404.63601278)(185.55946504,404.66601275)(185.69946824,404.67602054)
\curveto(185.83946476,404.69601272)(185.97946462,404.72601269)(186.11946824,404.76602054)
\lineto(186.31446824,404.76602054)
\curveto(186.38446421,404.77601264)(186.44946415,404.78601263)(186.50946824,404.79602054)
\curveto(187.3994632,404.80601261)(188.13946246,404.62101279)(188.72946824,404.24102054)
\curveto(189.31946128,403.86101355)(189.74446085,403.36601405)(190.00446824,402.75602054)
\curveto(190.05446054,402.65601476)(190.0944605,402.55601486)(190.12446824,402.45602054)
\curveto(190.15446044,402.35601506)(190.18946041,402.25101516)(190.22946824,402.14102054)
\curveto(190.25946034,402.03101538)(190.28446031,401.9110155)(190.30446824,401.78102054)
\curveto(190.32446027,401.66101575)(190.34946025,401.53601588)(190.37946824,401.40602054)
\curveto(190.38946021,401.35601606)(190.38946021,401.30101611)(190.37946824,401.24102054)
\curveto(190.37946022,401.19101622)(190.38446021,401.14101627)(190.39446824,401.09102054)
\moveto(189.05946824,400.23602054)
\curveto(189.07946152,400.30601711)(189.08446151,400.38601703)(189.07446824,400.47602054)
\lineto(189.07446824,400.73102054)
\curveto(189.07446152,401.12101629)(189.03946156,401.45101596)(188.96946824,401.72102054)
\curveto(188.93946166,401.80101561)(188.91446168,401.88101553)(188.89446824,401.96102054)
\curveto(188.87446172,402.04101537)(188.84946175,402.1160153)(188.81946824,402.18602054)
\curveto(188.53946206,402.83601458)(188.0944625,403.28601413)(187.48446824,403.53602054)
\curveto(187.41446318,403.56601385)(187.33946326,403.58601383)(187.25946824,403.59602054)
\lineto(187.01946824,403.65602054)
\curveto(186.93946366,403.67601374)(186.85446374,403.68601373)(186.76446824,403.68602054)
\lineto(186.49446824,403.68602054)
\lineto(186.22446824,403.64102054)
\curveto(186.12446447,403.62101379)(186.02946457,403.59601382)(185.93946824,403.56602054)
\curveto(185.85946474,403.54601387)(185.77946482,403.5160139)(185.69946824,403.47602054)
\curveto(185.62946497,403.45601396)(185.56446503,403.42601399)(185.50446824,403.38602054)
\curveto(185.44446515,403.34601407)(185.38946521,403.30601411)(185.33946824,403.26602054)
\curveto(185.0994655,403.09601432)(184.90446569,402.89101452)(184.75446824,402.65102054)
\curveto(184.60446599,402.411015)(184.47446612,402.13101528)(184.36446824,401.81102054)
\curveto(184.33446626,401.7110157)(184.31446628,401.60601581)(184.30446824,401.49602054)
\curveto(184.2944663,401.39601602)(184.27946632,401.29101612)(184.25946824,401.18102054)
\curveto(184.24946635,401.14101627)(184.24446635,401.07601634)(184.24446824,400.98602054)
\curveto(184.23446636,400.95601646)(184.22946637,400.92101649)(184.22946824,400.88102054)
\curveto(184.23946636,400.84101657)(184.24446635,400.79601662)(184.24446824,400.74602054)
\lineto(184.24446824,400.44602054)
\curveto(184.24446635,400.34601707)(184.25446634,400.25601716)(184.27446824,400.17602054)
\lineto(184.30446824,399.99602054)
\curveto(184.32446627,399.89601752)(184.33946626,399.79601762)(184.34946824,399.69602054)
\curveto(184.36946623,399.60601781)(184.3994662,399.52101789)(184.43946824,399.44102054)
\curveto(184.53946606,399.20101821)(184.65446594,398.97601844)(184.78446824,398.76602054)
\curveto(184.92446567,398.55601886)(185.0944655,398.38101903)(185.29446824,398.24102054)
\curveto(185.34446525,398.2110192)(185.38946521,398.18601923)(185.42946824,398.16602054)
\curveto(185.46946513,398.14601927)(185.51446508,398.12101929)(185.56446824,398.09102054)
\curveto(185.64446495,398.04101937)(185.72946487,397.99601942)(185.81946824,397.95602054)
\curveto(185.91946468,397.92601949)(186.02446457,397.89601952)(186.13446824,397.86602054)
\curveto(186.18446441,397.84601957)(186.22946437,397.83601958)(186.26946824,397.83602054)
\curveto(186.31946428,397.84601957)(186.36946423,397.84601957)(186.41946824,397.83602054)
\curveto(186.44946415,397.82601959)(186.50946409,397.8160196)(186.59946824,397.80602054)
\curveto(186.6994639,397.79601962)(186.77446382,397.80101961)(186.82446824,397.82102054)
\curveto(186.86446373,397.83101958)(186.90446369,397.83101958)(186.94446824,397.82102054)
\curveto(186.98446361,397.82101959)(187.02446357,397.83101958)(187.06446824,397.85102054)
\curveto(187.14446345,397.87101954)(187.22446337,397.88601953)(187.30446824,397.89602054)
\curveto(187.38446321,397.9160195)(187.45946314,397.94101947)(187.52946824,397.97102054)
\curveto(187.86946273,398.1110193)(188.14446245,398.30601911)(188.35446824,398.55602054)
\curveto(188.56446203,398.80601861)(188.73946186,399.10101831)(188.87946824,399.44102054)
\curveto(188.92946167,399.56101785)(188.95946164,399.68601773)(188.96946824,399.81602054)
\curveto(188.98946161,399.95601746)(189.01946158,400.09601732)(189.05946824,400.23602054)
}
}
{
\newrgbcolor{curcolor}{0 0 0}
\pscustom[linestyle=none,fillstyle=solid,fillcolor=curcolor]
{
\newpath
\moveto(194.31274949,404.79602054)
\curveto(195.03274543,404.80601261)(195.63774482,404.72101269)(196.12774949,404.54102054)
\curveto(196.61774384,404.37101304)(196.99774346,404.06601335)(197.26774949,403.62602054)
\curveto(197.33774312,403.5160139)(197.39274307,403.40101401)(197.43274949,403.28102054)
\curveto(197.47274299,403.17101424)(197.51274295,403.04601437)(197.55274949,402.90602054)
\curveto(197.57274289,402.83601458)(197.57774288,402.76101465)(197.56774949,402.68102054)
\curveto(197.5577429,402.6110148)(197.54274292,402.55601486)(197.52274949,402.51602054)
\curveto(197.50274296,402.49601492)(197.47774298,402.47601494)(197.44774949,402.45602054)
\curveto(197.41774304,402.44601497)(197.39274307,402.43101498)(197.37274949,402.41102054)
\curveto(197.32274314,402.39101502)(197.27274319,402.38601503)(197.22274949,402.39602054)
\curveto(197.17274329,402.40601501)(197.12274334,402.40601501)(197.07274949,402.39602054)
\curveto(196.99274347,402.37601504)(196.88774357,402.37101504)(196.75774949,402.38102054)
\curveto(196.62774383,402.40101501)(196.53774392,402.42601499)(196.48774949,402.45602054)
\curveto(196.40774405,402.50601491)(196.35274411,402.57101484)(196.32274949,402.65102054)
\curveto(196.30274416,402.74101467)(196.26774419,402.82601459)(196.21774949,402.90602054)
\curveto(196.12774433,403.06601435)(196.00274446,403.2110142)(195.84274949,403.34102054)
\curveto(195.73274473,403.42101399)(195.61274485,403.48101393)(195.48274949,403.52102054)
\curveto(195.35274511,403.56101385)(195.21274525,403.60101381)(195.06274949,403.64102054)
\curveto(195.01274545,403.66101375)(194.9627455,403.66601375)(194.91274949,403.65602054)
\curveto(194.8627456,403.65601376)(194.81274565,403.66101375)(194.76274949,403.67102054)
\curveto(194.70274576,403.69101372)(194.62774583,403.70101371)(194.53774949,403.70102054)
\curveto(194.44774601,403.70101371)(194.37274609,403.69101372)(194.31274949,403.67102054)
\lineto(194.22274949,403.67102054)
\lineto(194.07274949,403.64102054)
\curveto(194.02274644,403.64101377)(193.97274649,403.63601378)(193.92274949,403.62602054)
\curveto(193.6627468,403.56601385)(193.44774701,403.48101393)(193.27774949,403.37102054)
\curveto(193.10774735,403.26101415)(192.99274747,403.07601434)(192.93274949,402.81602054)
\curveto(192.91274755,402.74601467)(192.90774755,402.67601474)(192.91774949,402.60602054)
\curveto(192.93774752,402.53601488)(192.9577475,402.47601494)(192.97774949,402.42602054)
\curveto(193.03774742,402.27601514)(193.10774735,402.16601525)(193.18774949,402.09602054)
\curveto(193.27774718,402.03601538)(193.38774707,401.96601545)(193.51774949,401.88602054)
\curveto(193.67774678,401.78601563)(193.8577466,401.7110157)(194.05774949,401.66102054)
\curveto(194.2577462,401.62101579)(194.457746,401.57101584)(194.65774949,401.51102054)
\curveto(194.78774567,401.47101594)(194.91774554,401.44101597)(195.04774949,401.42102054)
\curveto(195.17774528,401.40101601)(195.30774515,401.37101604)(195.43774949,401.33102054)
\curveto(195.64774481,401.27101614)(195.85274461,401.2110162)(196.05274949,401.15102054)
\curveto(196.25274421,401.10101631)(196.45274401,401.03601638)(196.65274949,400.95602054)
\lineto(196.80274949,400.89602054)
\curveto(196.85274361,400.87601654)(196.90274356,400.85101656)(196.95274949,400.82102054)
\curveto(197.15274331,400.70101671)(197.32774313,400.56601685)(197.47774949,400.41602054)
\curveto(197.62774283,400.26601715)(197.75274271,400.07601734)(197.85274949,399.84602054)
\curveto(197.87274259,399.77601764)(197.89274257,399.68101773)(197.91274949,399.56102054)
\curveto(197.93274253,399.49101792)(197.94274252,399.416018)(197.94274949,399.33602054)
\curveto(197.95274251,399.26601815)(197.9577425,399.18601823)(197.95774949,399.09602054)
\lineto(197.95774949,398.94602054)
\curveto(197.93774252,398.87601854)(197.92774253,398.80601861)(197.92774949,398.73602054)
\curveto(197.92774253,398.66601875)(197.91774254,398.59601882)(197.89774949,398.52602054)
\curveto(197.86774259,398.416019)(197.83274263,398.3110191)(197.79274949,398.21102054)
\curveto(197.75274271,398.1110193)(197.70774275,398.02101939)(197.65774949,397.94102054)
\curveto(197.49774296,397.68101973)(197.29274317,397.47101994)(197.04274949,397.31102054)
\curveto(196.79274367,397.16102025)(196.51274395,397.03102038)(196.20274949,396.92102054)
\curveto(196.11274435,396.89102052)(196.01774444,396.87102054)(195.91774949,396.86102054)
\curveto(195.82774463,396.84102057)(195.73774472,396.8160206)(195.64774949,396.78602054)
\curveto(195.54774491,396.76602065)(195.44774501,396.75602066)(195.34774949,396.75602054)
\curveto(195.24774521,396.75602066)(195.14774531,396.74602067)(195.04774949,396.72602054)
\lineto(194.89774949,396.72602054)
\curveto(194.84774561,396.7160207)(194.77774568,396.7110207)(194.68774949,396.71102054)
\curveto(194.59774586,396.7110207)(194.52774593,396.7160207)(194.47774949,396.72602054)
\lineto(194.31274949,396.72602054)
\curveto(194.25274621,396.74602067)(194.18774627,396.75602066)(194.11774949,396.75602054)
\curveto(194.04774641,396.74602067)(193.98774647,396.75102066)(193.93774949,396.77102054)
\curveto(193.88774657,396.78102063)(193.82274664,396.78602063)(193.74274949,396.78602054)
\lineto(193.50274949,396.84602054)
\curveto(193.43274703,396.85602056)(193.3577471,396.87602054)(193.27774949,396.90602054)
\curveto(192.96774749,397.00602041)(192.69774776,397.13102028)(192.46774949,397.28102054)
\curveto(192.23774822,397.43101998)(192.03774842,397.62601979)(191.86774949,397.86602054)
\curveto(191.77774868,397.99601942)(191.70274876,398.13101928)(191.64274949,398.27102054)
\curveto(191.58274888,398.411019)(191.52774893,398.56601885)(191.47774949,398.73602054)
\curveto(191.457749,398.79601862)(191.44774901,398.86601855)(191.44774949,398.94602054)
\curveto(191.457749,399.03601838)(191.47274899,399.10601831)(191.49274949,399.15602054)
\curveto(191.52274894,399.19601822)(191.57274889,399.23601818)(191.64274949,399.27602054)
\curveto(191.69274877,399.29601812)(191.7627487,399.30601811)(191.85274949,399.30602054)
\curveto(191.94274852,399.3160181)(192.03274843,399.3160181)(192.12274949,399.30602054)
\curveto(192.21274825,399.29601812)(192.29774816,399.28101813)(192.37774949,399.26102054)
\curveto(192.46774799,399.25101816)(192.52774793,399.23601818)(192.55774949,399.21602054)
\curveto(192.62774783,399.16601825)(192.67274779,399.09101832)(192.69274949,398.99102054)
\curveto(192.72274774,398.90101851)(192.7577477,398.8160186)(192.79774949,398.73602054)
\curveto(192.89774756,398.5160189)(193.03274743,398.34601907)(193.20274949,398.22602054)
\curveto(193.32274714,398.13601928)(193.457747,398.06601935)(193.60774949,398.01602054)
\curveto(193.7577467,397.96601945)(193.91774654,397.9160195)(194.08774949,397.86602054)
\lineto(194.40274949,397.82102054)
\lineto(194.49274949,397.82102054)
\curveto(194.5627459,397.80101961)(194.65274581,397.79101962)(194.76274949,397.79102054)
\curveto(194.88274558,397.79101962)(194.98274548,397.80101961)(195.06274949,397.82102054)
\curveto(195.13274533,397.82101959)(195.18774527,397.82601959)(195.22774949,397.83602054)
\curveto(195.28774517,397.84601957)(195.34774511,397.85101956)(195.40774949,397.85102054)
\curveto(195.46774499,397.86101955)(195.52274494,397.87101954)(195.57274949,397.88102054)
\curveto(195.8627446,397.96101945)(196.09274437,398.06601935)(196.26274949,398.19602054)
\curveto(196.43274403,398.32601909)(196.55274391,398.54601887)(196.62274949,398.85602054)
\curveto(196.64274382,398.90601851)(196.64774381,398.96101845)(196.63774949,399.02102054)
\curveto(196.62774383,399.08101833)(196.61774384,399.12601829)(196.60774949,399.15602054)
\curveto(196.5577439,399.34601807)(196.48774397,399.48601793)(196.39774949,399.57602054)
\curveto(196.30774415,399.67601774)(196.19274427,399.76601765)(196.05274949,399.84602054)
\curveto(195.9627445,399.90601751)(195.8627446,399.95601746)(195.75274949,399.99602054)
\lineto(195.42274949,400.11602054)
\curveto(195.39274507,400.12601729)(195.3627451,400.13101728)(195.33274949,400.13102054)
\curveto(195.31274515,400.13101728)(195.28774517,400.14101727)(195.25774949,400.16102054)
\curveto(194.91774554,400.27101714)(194.5627459,400.35101706)(194.19274949,400.40102054)
\curveto(193.83274663,400.46101695)(193.49274697,400.55601686)(193.17274949,400.68602054)
\curveto(193.07274739,400.72601669)(192.97774748,400.76101665)(192.88774949,400.79102054)
\curveto(192.79774766,400.82101659)(192.71274775,400.86101655)(192.63274949,400.91102054)
\curveto(192.44274802,401.02101639)(192.26774819,401.14601627)(192.10774949,401.28602054)
\curveto(191.94774851,401.42601599)(191.82274864,401.60101581)(191.73274949,401.81102054)
\curveto(191.70274876,401.88101553)(191.67774878,401.95101546)(191.65774949,402.02102054)
\curveto(191.64774881,402.09101532)(191.63274883,402.16601525)(191.61274949,402.24602054)
\curveto(191.58274888,402.36601505)(191.57274889,402.50101491)(191.58274949,402.65102054)
\curveto(191.59274887,402.8110146)(191.60774885,402.94601447)(191.62774949,403.05602054)
\curveto(191.64774881,403.10601431)(191.6577488,403.14601427)(191.65774949,403.17602054)
\curveto(191.66774879,403.2160142)(191.68274878,403.25601416)(191.70274949,403.29602054)
\curveto(191.79274867,403.52601389)(191.91274855,403.72601369)(192.06274949,403.89602054)
\curveto(192.22274824,404.06601335)(192.40274806,404.2160132)(192.60274949,404.34602054)
\curveto(192.75274771,404.43601298)(192.91774754,404.50601291)(193.09774949,404.55602054)
\curveto(193.27774718,404.6160128)(193.46774699,404.67101274)(193.66774949,404.72102054)
\curveto(193.73774672,404.73101268)(193.80274666,404.74101267)(193.86274949,404.75102054)
\curveto(193.93274653,404.76101265)(194.00774645,404.77101264)(194.08774949,404.78102054)
\curveto(194.11774634,404.79101262)(194.1577463,404.79101262)(194.20774949,404.78102054)
\curveto(194.2577462,404.77101264)(194.29274617,404.77601264)(194.31274949,404.79602054)
}
}
{
\newrgbcolor{curcolor}{0.90196079 0.90196079 0.90196079}
\pscustom[linestyle=none,fillstyle=solid,fillcolor=curcolor]
{
\newpath
\moveto(120.84556474,410.10108768)
\lineto(135.84556474,410.10108768)
\lineto(135.84556474,395.10108768)
\lineto(120.84556474,395.10108768)
\closepath
}
}
{
\newrgbcolor{curcolor}{0 0 0}
\pscustom[linestyle=none,fillstyle=solid,fillcolor=curcolor]
{
\newpath
\moveto(144.77876512,384.77531497)
\curveto(145.75875862,384.79530402)(146.5787578,384.63530418)(147.23876512,384.29531497)
\curveto(147.90875647,383.96530485)(148.42875595,383.50530531)(148.79876512,382.91531497)
\curveto(148.89875548,382.75530606)(148.9787554,382.60030621)(149.03876512,382.45031497)
\curveto(149.10875527,382.3103065)(149.1737552,382.14030667)(149.23376512,381.94031497)
\curveto(149.25375512,381.89030692)(149.2737551,381.82030699)(149.29376512,381.73031497)
\curveto(149.31375506,381.65030716)(149.30875507,381.57530724)(149.27876512,381.50531497)
\curveto(149.25875512,381.44530737)(149.21875516,381.40530741)(149.15876512,381.38531497)
\curveto(149.10875527,381.37530744)(149.05375532,381.36030745)(148.99376512,381.34031497)
\lineto(148.84376512,381.34031497)
\curveto(148.81375556,381.33030748)(148.7737556,381.32530749)(148.72376512,381.32531497)
\lineto(148.60376512,381.32531497)
\curveto(148.46375591,381.32530749)(148.33375604,381.33030748)(148.21376512,381.34031497)
\curveto(148.10375627,381.36030745)(148.02375635,381.4103074)(147.97376512,381.49031497)
\curveto(147.90375647,381.59030722)(147.84875653,381.70530711)(147.80876512,381.83531497)
\curveto(147.76875661,381.96530685)(147.71375666,382.08530673)(147.64376512,382.19531497)
\curveto(147.51375686,382.4153064)(147.36375701,382.60530621)(147.19376512,382.76531497)
\curveto(147.03375734,382.92530589)(146.84375753,383.07530574)(146.62376512,383.21531497)
\curveto(146.50375787,383.29530552)(146.36875801,383.35530546)(146.21876512,383.39531497)
\curveto(146.0787583,383.43530538)(145.93375844,383.47530534)(145.78376512,383.51531497)
\curveto(145.6737587,383.54530527)(145.54875883,383.56530525)(145.40876512,383.57531497)
\curveto(145.26875911,383.59530522)(145.11875926,383.60530521)(144.95876512,383.60531497)
\curveto(144.80875957,383.60530521)(144.65875972,383.59530522)(144.50876512,383.57531497)
\curveto(144.36876001,383.56530525)(144.24876013,383.54530527)(144.14876512,383.51531497)
\curveto(144.04876033,383.49530532)(143.95376042,383.47530534)(143.86376512,383.45531497)
\curveto(143.7737606,383.43530538)(143.68376069,383.40530541)(143.59376512,383.36531497)
\curveto(142.75376162,383.0153058)(142.14876223,382.4153064)(141.77876512,381.56531497)
\curveto(141.70876267,381.42530739)(141.64876273,381.27530754)(141.59876512,381.11531497)
\curveto(141.55876282,380.96530785)(141.51376286,380.810308)(141.46376512,380.65031497)
\curveto(141.44376293,380.59030822)(141.43376294,380.52530829)(141.43376512,380.45531497)
\curveto(141.43376294,380.39530842)(141.42376295,380.33530848)(141.40376512,380.27531497)
\curveto(141.39376298,380.23530858)(141.38876299,380.20030861)(141.38876512,380.17031497)
\curveto(141.38876299,380.14030867)(141.38376299,380.10530871)(141.37376512,380.06531497)
\curveto(141.35376302,379.95530886)(141.33876304,379.84030897)(141.32876512,379.72031497)
\lineto(141.32876512,379.37531497)
\curveto(141.32876305,379.30530951)(141.32376305,379.23030958)(141.31376512,379.15031497)
\curveto(141.31376306,379.08030973)(141.31876306,379.0153098)(141.32876512,378.95531497)
\lineto(141.32876512,378.80531497)
\curveto(141.34876303,378.73531008)(141.35376302,378.66531015)(141.34376512,378.59531497)
\curveto(141.34376303,378.52531029)(141.35376302,378.45531036)(141.37376512,378.38531497)
\curveto(141.39376298,378.32531049)(141.39876298,378.26531055)(141.38876512,378.20531497)
\curveto(141.38876299,378.14531067)(141.39876298,378.09031072)(141.41876512,378.04031497)
\curveto(141.44876293,377.9103109)(141.4737629,377.78031103)(141.49376512,377.65031497)
\curveto(141.52376285,377.53031128)(141.55876282,377.4103114)(141.59876512,377.29031497)
\curveto(141.76876261,376.79031202)(141.98876239,376.36031245)(142.25876512,376.00031497)
\curveto(142.52876185,375.65031316)(142.88376149,375.36031345)(143.32376512,375.13031497)
\curveto(143.46376091,375.06031375)(143.60376077,375.00531381)(143.74376512,374.96531497)
\curveto(143.89376048,374.92531389)(144.05376032,374.88031393)(144.22376512,374.83031497)
\curveto(144.29376008,374.810314)(144.35876002,374.80031401)(144.41876512,374.80031497)
\curveto(144.4787599,374.810314)(144.54875983,374.80531401)(144.62876512,374.78531497)
\curveto(144.6787597,374.77531404)(144.76875961,374.76531405)(144.89876512,374.75531497)
\curveto(145.02875935,374.75531406)(145.12375925,374.76531405)(145.18376512,374.78531497)
\lineto(145.28876512,374.78531497)
\curveto(145.32875905,374.79531402)(145.36875901,374.79531402)(145.40876512,374.78531497)
\curveto(145.44875893,374.78531403)(145.48875889,374.79531402)(145.52876512,374.81531497)
\curveto(145.62875875,374.83531398)(145.72375865,374.85031396)(145.81376512,374.86031497)
\curveto(145.91375846,374.88031393)(146.00875837,374.9103139)(146.09876512,374.95031497)
\curveto(146.8787575,375.27031354)(147.42875695,375.79531302)(147.74876512,376.52531497)
\curveto(147.82875655,376.70531211)(147.90375647,376.92031189)(147.97376512,377.17031497)
\curveto(147.99375638,377.26031155)(148.00875637,377.35031146)(148.01876512,377.44031497)
\curveto(148.03875634,377.54031127)(148.0737563,377.63031118)(148.12376512,377.71031497)
\curveto(148.1737562,377.79031102)(148.25375612,377.83531098)(148.36376512,377.84531497)
\curveto(148.4737559,377.85531096)(148.59375578,377.86031095)(148.72376512,377.86031497)
\lineto(148.87376512,377.86031497)
\curveto(148.92375545,377.86031095)(148.96875541,377.85531096)(149.00876512,377.84531497)
\lineto(149.11376512,377.84531497)
\lineto(149.20376512,377.81531497)
\curveto(149.24375513,377.815311)(149.2737551,377.80531101)(149.29376512,377.78531497)
\curveto(149.36375501,377.74531107)(149.40375497,377.67031114)(149.41376512,377.56031497)
\curveto(149.42375495,377.46031135)(149.41375496,377.36031145)(149.38376512,377.26031497)
\curveto(149.32375505,377.03031178)(149.26875511,376.810312)(149.21876512,376.60031497)
\curveto(149.16875521,376.39031242)(149.09375528,376.19031262)(148.99376512,376.00031497)
\curveto(148.91375546,375.87031294)(148.83875554,375.74531307)(148.76876512,375.62531497)
\curveto(148.70875567,375.50531331)(148.63875574,375.38531343)(148.55876512,375.26531497)
\curveto(148.378756,375.00531381)(148.15375622,374.76531405)(147.88376512,374.54531497)
\curveto(147.62375675,374.33531448)(147.33875704,374.16031465)(147.02876512,374.02031497)
\curveto(146.91875746,373.97031484)(146.80875757,373.93031488)(146.69876512,373.90031497)
\curveto(146.59875778,373.87031494)(146.49375788,373.83531498)(146.38376512,373.79531497)
\curveto(146.2737581,373.75531506)(146.15875822,373.73031508)(146.03876512,373.72031497)
\curveto(145.92875845,373.70031511)(145.81375856,373.68031513)(145.69376512,373.66031497)
\curveto(145.64375873,373.64031517)(145.59875878,373.63531518)(145.55876512,373.64531497)
\curveto(145.51875886,373.64531517)(145.4787589,373.64031517)(145.43876512,373.63031497)
\curveto(145.378759,373.62031519)(145.31875906,373.6153152)(145.25876512,373.61531497)
\curveto(145.19875918,373.6153152)(145.13375924,373.6103152)(145.06376512,373.60031497)
\curveto(145.03375934,373.59031522)(144.96375941,373.59031522)(144.85376512,373.60031497)
\curveto(144.75375962,373.60031521)(144.68875969,373.60531521)(144.65876512,373.61531497)
\curveto(144.60875977,373.62531519)(144.55875982,373.63031518)(144.50876512,373.63031497)
\curveto(144.46875991,373.62031519)(144.42375995,373.62031519)(144.37376512,373.63031497)
\lineto(144.22376512,373.63031497)
\curveto(144.14376023,373.65031516)(144.06876031,373.66531515)(143.99876512,373.67531497)
\curveto(143.92876045,373.67531514)(143.85376052,373.68531513)(143.77376512,373.70531497)
\lineto(143.50376512,373.76531497)
\curveto(143.41376096,373.77531504)(143.32876105,373.79531502)(143.24876512,373.82531497)
\curveto(143.03876134,373.88531493)(142.84876153,373.96031485)(142.67876512,374.05031497)
\curveto(142.04876233,374.32031449)(141.53876284,374.70531411)(141.14876512,375.20531497)
\curveto(140.75876362,375.70531311)(140.44876393,376.29531252)(140.21876512,376.97531497)
\curveto(140.1787642,377.09531172)(140.14376423,377.22031159)(140.11376512,377.35031497)
\curveto(140.09376428,377.48031133)(140.06876431,377.6153112)(140.03876512,377.75531497)
\curveto(140.01876436,377.80531101)(140.00876437,377.85031096)(140.00876512,377.89031497)
\curveto(140.01876436,377.93031088)(140.01876436,377.97531084)(140.00876512,378.02531497)
\curveto(139.98876439,378.1153107)(139.9737644,378.2103106)(139.96376512,378.31031497)
\curveto(139.96376441,378.4103104)(139.95376442,378.50531031)(139.93376512,378.59531497)
\lineto(139.93376512,378.88031497)
\curveto(139.91376446,378.93030988)(139.90376447,379.0153098)(139.90376512,379.13531497)
\curveto(139.90376447,379.25530956)(139.91376446,379.34030947)(139.93376512,379.39031497)
\curveto(139.94376443,379.42030939)(139.94376443,379.45030936)(139.93376512,379.48031497)
\curveto(139.92376445,379.52030929)(139.92376445,379.55030926)(139.93376512,379.57031497)
\lineto(139.93376512,379.70531497)
\curveto(139.94376443,379.78530903)(139.94876443,379.86530895)(139.94876512,379.94531497)
\curveto(139.95876442,380.03530878)(139.9737644,380.12030869)(139.99376512,380.20031497)
\curveto(140.01376436,380.26030855)(140.02376435,380.32030849)(140.02376512,380.38031497)
\curveto(140.02376435,380.45030836)(140.03376434,380.52030829)(140.05376512,380.59031497)
\curveto(140.10376427,380.76030805)(140.14376423,380.92530789)(140.17376512,381.08531497)
\curveto(140.20376417,381.24530757)(140.24876413,381.39530742)(140.30876512,381.53531497)
\lineto(140.45876512,381.92531497)
\curveto(140.51876386,382.06530675)(140.58376379,382.19030662)(140.65376512,382.30031497)
\curveto(140.80376357,382.56030625)(140.95376342,382.79530602)(141.10376512,383.00531497)
\curveto(141.13376324,383.05530576)(141.16876321,383.09530572)(141.20876512,383.12531497)
\curveto(141.25876312,383.16530565)(141.29876308,383.2103056)(141.32876512,383.26031497)
\curveto(141.38876299,383.34030547)(141.44876293,383.4103054)(141.50876512,383.47031497)
\lineto(141.71876512,383.65031497)
\curveto(141.7787626,383.70030511)(141.83376254,383.74530507)(141.88376512,383.78531497)
\curveto(141.94376243,383.83530498)(142.00876237,383.88530493)(142.07876512,383.93531497)
\curveto(142.22876215,384.04530477)(142.38376199,384.14030467)(142.54376512,384.22031497)
\curveto(142.71376166,384.30030451)(142.88876149,384.38030443)(143.06876512,384.46031497)
\curveto(143.1787612,384.5103043)(143.29376108,384.54530427)(143.41376512,384.56531497)
\curveto(143.54376083,384.59530422)(143.66876071,384.63030418)(143.78876512,384.67031497)
\curveto(143.85876052,384.68030413)(143.92376045,384.69030412)(143.98376512,384.70031497)
\lineto(144.16376512,384.73031497)
\curveto(144.24376013,384.74030407)(144.31876006,384.74530407)(144.38876512,384.74531497)
\curveto(144.46875991,384.75530406)(144.54875983,384.76530405)(144.62876512,384.77531497)
\curveto(144.64875973,384.78530403)(144.6737597,384.78530403)(144.70376512,384.77531497)
\curveto(144.73375964,384.76530405)(144.75875962,384.76530405)(144.77876512,384.77531497)
}
}
{
\newrgbcolor{curcolor}{0 0 0}
\pscustom[linestyle=none,fillstyle=solid,fillcolor=curcolor]
{
\newpath
\moveto(158.13860887,378.05531497)
\curveto(158.15860081,377.99531082)(158.1686008,377.90031091)(158.16860887,377.77031497)
\curveto(158.1686008,377.65031116)(158.1636008,377.56531125)(158.15360887,377.51531497)
\lineto(158.15360887,377.36531497)
\curveto(158.14360082,377.28531153)(158.13360083,377.2103116)(158.12360887,377.14031497)
\curveto(158.12360084,377.08031173)(158.11860085,377.0103118)(158.10860887,376.93031497)
\curveto(158.08860088,376.87031194)(158.07360089,376.810312)(158.06360887,376.75031497)
\curveto(158.0636009,376.69031212)(158.05360091,376.63031218)(158.03360887,376.57031497)
\curveto(157.99360097,376.44031237)(157.95860101,376.3103125)(157.92860887,376.18031497)
\curveto(157.89860107,376.05031276)(157.85860111,375.93031288)(157.80860887,375.82031497)
\curveto(157.59860137,375.34031347)(157.31860165,374.93531388)(156.96860887,374.60531497)
\curveto(156.61860235,374.28531453)(156.18860278,374.04031477)(155.67860887,373.87031497)
\curveto(155.5686034,373.83031498)(155.44860352,373.80031501)(155.31860887,373.78031497)
\curveto(155.19860377,373.76031505)(155.07360389,373.74031507)(154.94360887,373.72031497)
\curveto(154.88360408,373.7103151)(154.81860415,373.70531511)(154.74860887,373.70531497)
\curveto(154.68860428,373.69531512)(154.62860434,373.69031512)(154.56860887,373.69031497)
\curveto(154.52860444,373.68031513)(154.4686045,373.67531514)(154.38860887,373.67531497)
\curveto(154.31860465,373.67531514)(154.2686047,373.68031513)(154.23860887,373.69031497)
\curveto(154.19860477,373.70031511)(154.15860481,373.70531511)(154.11860887,373.70531497)
\curveto(154.07860489,373.69531512)(154.04360492,373.69531512)(154.01360887,373.70531497)
\lineto(153.92360887,373.70531497)
\lineto(153.56360887,373.75031497)
\curveto(153.42360554,373.79031502)(153.28860568,373.83031498)(153.15860887,373.87031497)
\curveto(153.02860594,373.9103149)(152.90360606,373.95531486)(152.78360887,374.00531497)
\curveto(152.33360663,374.20531461)(151.963607,374.46531435)(151.67360887,374.78531497)
\curveto(151.38360758,375.10531371)(151.14360782,375.49531332)(150.95360887,375.95531497)
\curveto(150.90360806,376.05531276)(150.8636081,376.15531266)(150.83360887,376.25531497)
\curveto(150.81360815,376.35531246)(150.79360817,376.46031235)(150.77360887,376.57031497)
\curveto(150.75360821,376.6103122)(150.74360822,376.64031217)(150.74360887,376.66031497)
\curveto(150.75360821,376.69031212)(150.75360821,376.72531209)(150.74360887,376.76531497)
\curveto(150.72360824,376.84531197)(150.70860826,376.92531189)(150.69860887,377.00531497)
\curveto(150.69860827,377.09531172)(150.68860828,377.18031163)(150.66860887,377.26031497)
\lineto(150.66860887,377.38031497)
\curveto(150.6686083,377.42031139)(150.6636083,377.46531135)(150.65360887,377.51531497)
\curveto(150.64360832,377.56531125)(150.63860833,377.65031116)(150.63860887,377.77031497)
\curveto(150.63860833,377.90031091)(150.64860832,377.99531082)(150.66860887,378.05531497)
\curveto(150.68860828,378.12531069)(150.69360827,378.19531062)(150.68360887,378.26531497)
\curveto(150.67360829,378.33531048)(150.67860829,378.40531041)(150.69860887,378.47531497)
\curveto(150.70860826,378.52531029)(150.71360825,378.56531025)(150.71360887,378.59531497)
\curveto(150.72360824,378.63531018)(150.73360823,378.68031013)(150.74360887,378.73031497)
\curveto(150.77360819,378.85030996)(150.79860817,378.97030984)(150.81860887,379.09031497)
\curveto(150.84860812,379.2103096)(150.88860808,379.32530949)(150.93860887,379.43531497)
\curveto(151.08860788,379.80530901)(151.2686077,380.13530868)(151.47860887,380.42531497)
\curveto(151.69860727,380.72530809)(151.963607,380.97530784)(152.27360887,381.17531497)
\curveto(152.39360657,381.25530756)(152.51860645,381.32030749)(152.64860887,381.37031497)
\curveto(152.77860619,381.43030738)(152.91360605,381.49030732)(153.05360887,381.55031497)
\curveto(153.17360579,381.60030721)(153.30360566,381.63030718)(153.44360887,381.64031497)
\curveto(153.58360538,381.66030715)(153.72360524,381.69030712)(153.86360887,381.73031497)
\lineto(154.05860887,381.73031497)
\curveto(154.12860484,381.74030707)(154.19360477,381.75030706)(154.25360887,381.76031497)
\curveto(155.14360382,381.77030704)(155.88360308,381.58530723)(156.47360887,381.20531497)
\curveto(157.0636019,380.82530799)(157.48860148,380.33030848)(157.74860887,379.72031497)
\curveto(157.79860117,379.62030919)(157.83860113,379.52030929)(157.86860887,379.42031497)
\curveto(157.89860107,379.32030949)(157.93360103,379.2153096)(157.97360887,379.10531497)
\curveto(158.00360096,378.99530982)(158.02860094,378.87530994)(158.04860887,378.74531497)
\curveto(158.0686009,378.62531019)(158.09360087,378.50031031)(158.12360887,378.37031497)
\curveto(158.13360083,378.32031049)(158.13360083,378.26531055)(158.12360887,378.20531497)
\curveto(158.12360084,378.15531066)(158.12860084,378.10531071)(158.13860887,378.05531497)
\moveto(156.80360887,377.20031497)
\curveto(156.82360214,377.27031154)(156.82860214,377.35031146)(156.81860887,377.44031497)
\lineto(156.81860887,377.69531497)
\curveto(156.81860215,378.08531073)(156.78360218,378.4153104)(156.71360887,378.68531497)
\curveto(156.68360228,378.76531005)(156.65860231,378.84530997)(156.63860887,378.92531497)
\curveto(156.61860235,379.00530981)(156.59360237,379.08030973)(156.56360887,379.15031497)
\curveto(156.28360268,379.80030901)(155.83860313,380.25030856)(155.22860887,380.50031497)
\curveto(155.15860381,380.53030828)(155.08360388,380.55030826)(155.00360887,380.56031497)
\lineto(154.76360887,380.62031497)
\curveto(154.68360428,380.64030817)(154.59860437,380.65030816)(154.50860887,380.65031497)
\lineto(154.23860887,380.65031497)
\lineto(153.96860887,380.60531497)
\curveto(153.8686051,380.58530823)(153.77360519,380.56030825)(153.68360887,380.53031497)
\curveto(153.60360536,380.5103083)(153.52360544,380.48030833)(153.44360887,380.44031497)
\curveto(153.37360559,380.42030839)(153.30860566,380.39030842)(153.24860887,380.35031497)
\curveto(153.18860578,380.3103085)(153.13360583,380.27030854)(153.08360887,380.23031497)
\curveto(152.84360612,380.06030875)(152.64860632,379.85530896)(152.49860887,379.61531497)
\curveto(152.34860662,379.37530944)(152.21860675,379.09530972)(152.10860887,378.77531497)
\curveto(152.07860689,378.67531014)(152.05860691,378.57031024)(152.04860887,378.46031497)
\curveto(152.03860693,378.36031045)(152.02360694,378.25531056)(152.00360887,378.14531497)
\curveto(151.99360697,378.10531071)(151.98860698,378.04031077)(151.98860887,377.95031497)
\curveto(151.97860699,377.92031089)(151.97360699,377.88531093)(151.97360887,377.84531497)
\curveto(151.98360698,377.80531101)(151.98860698,377.76031105)(151.98860887,377.71031497)
\lineto(151.98860887,377.41031497)
\curveto(151.98860698,377.3103115)(151.99860697,377.22031159)(152.01860887,377.14031497)
\lineto(152.04860887,376.96031497)
\curveto(152.0686069,376.86031195)(152.08360688,376.76031205)(152.09360887,376.66031497)
\curveto(152.11360685,376.57031224)(152.14360682,376.48531233)(152.18360887,376.40531497)
\curveto(152.28360668,376.16531265)(152.39860657,375.94031287)(152.52860887,375.73031497)
\curveto(152.6686063,375.52031329)(152.83860613,375.34531347)(153.03860887,375.20531497)
\curveto(153.08860588,375.17531364)(153.13360583,375.15031366)(153.17360887,375.13031497)
\curveto(153.21360575,375.1103137)(153.25860571,375.08531373)(153.30860887,375.05531497)
\curveto(153.38860558,375.00531381)(153.47360549,374.96031385)(153.56360887,374.92031497)
\curveto(153.6636053,374.89031392)(153.7686052,374.86031395)(153.87860887,374.83031497)
\curveto(153.92860504,374.810314)(153.97360499,374.80031401)(154.01360887,374.80031497)
\curveto(154.0636049,374.810314)(154.11360485,374.810314)(154.16360887,374.80031497)
\curveto(154.19360477,374.79031402)(154.25360471,374.78031403)(154.34360887,374.77031497)
\curveto(154.44360452,374.76031405)(154.51860445,374.76531405)(154.56860887,374.78531497)
\curveto(154.60860436,374.79531402)(154.64860432,374.79531402)(154.68860887,374.78531497)
\curveto(154.72860424,374.78531403)(154.7686042,374.79531402)(154.80860887,374.81531497)
\curveto(154.88860408,374.83531398)(154.968604,374.85031396)(155.04860887,374.86031497)
\curveto(155.12860384,374.88031393)(155.20360376,374.90531391)(155.27360887,374.93531497)
\curveto(155.61360335,375.07531374)(155.88860308,375.27031354)(156.09860887,375.52031497)
\curveto(156.30860266,375.77031304)(156.48360248,376.06531275)(156.62360887,376.40531497)
\curveto(156.67360229,376.52531229)(156.70360226,376.65031216)(156.71360887,376.78031497)
\curveto(156.73360223,376.92031189)(156.7636022,377.06031175)(156.80360887,377.20031497)
}
}
{
\newrgbcolor{curcolor}{0 0 0}
\pscustom[linestyle=none,fillstyle=solid,fillcolor=curcolor]
{
\newpath
\moveto(163.31689012,381.73031497)
\curveto(163.94688488,381.75030706)(164.45188438,381.66530715)(164.83189012,381.47531497)
\curveto(165.21188362,381.28530753)(165.51688331,381.00030781)(165.74689012,380.62031497)
\curveto(165.80688302,380.52030829)(165.85188298,380.4103084)(165.88189012,380.29031497)
\curveto(165.92188291,380.18030863)(165.95688287,380.06530875)(165.98689012,379.94531497)
\curveto(166.03688279,379.75530906)(166.06688276,379.55030926)(166.07689012,379.33031497)
\curveto(166.08688274,379.1103097)(166.09188274,378.88530993)(166.09189012,378.65531497)
\lineto(166.09189012,377.05031497)
\lineto(166.09189012,374.71031497)
\curveto(166.09188274,374.54031427)(166.08688274,374.37031444)(166.07689012,374.20031497)
\curveto(166.07688275,374.03031478)(166.01188282,373.92031489)(165.88189012,373.87031497)
\curveto(165.831883,373.85031496)(165.77688305,373.84031497)(165.71689012,373.84031497)
\curveto(165.66688316,373.83031498)(165.61188322,373.82531499)(165.55189012,373.82531497)
\curveto(165.42188341,373.82531499)(165.29688353,373.83031498)(165.17689012,373.84031497)
\curveto(165.05688377,373.84031497)(164.97188386,373.88031493)(164.92189012,373.96031497)
\curveto(164.87188396,374.03031478)(164.84688398,374.12031469)(164.84689012,374.23031497)
\lineto(164.84689012,374.56031497)
\lineto(164.84689012,375.85031497)
\lineto(164.84689012,378.29531497)
\curveto(164.84688398,378.56531025)(164.84188399,378.83030998)(164.83189012,379.09031497)
\curveto(164.82188401,379.36030945)(164.77688405,379.59030922)(164.69689012,379.78031497)
\curveto(164.61688421,379.98030883)(164.49688433,380.14030867)(164.33689012,380.26031497)
\curveto(164.17688465,380.39030842)(163.99188484,380.49030832)(163.78189012,380.56031497)
\curveto(163.72188511,380.58030823)(163.65688517,380.59030822)(163.58689012,380.59031497)
\curveto(163.5268853,380.60030821)(163.46688536,380.6153082)(163.40689012,380.63531497)
\curveto(163.35688547,380.64530817)(163.27688555,380.64530817)(163.16689012,380.63531497)
\curveto(163.06688576,380.63530818)(162.99688583,380.63030818)(162.95689012,380.62031497)
\curveto(162.91688591,380.60030821)(162.88188595,380.59030822)(162.85189012,380.59031497)
\curveto(162.82188601,380.60030821)(162.78688604,380.60030821)(162.74689012,380.59031497)
\curveto(162.61688621,380.56030825)(162.49188634,380.52530829)(162.37189012,380.48531497)
\curveto(162.26188657,380.45530836)(162.15688667,380.4103084)(162.05689012,380.35031497)
\curveto(162.01688681,380.33030848)(161.98188685,380.3103085)(161.95189012,380.29031497)
\curveto(161.92188691,380.27030854)(161.88688694,380.25030856)(161.84689012,380.23031497)
\curveto(161.49688733,379.98030883)(161.24188759,379.60530921)(161.08189012,379.10531497)
\curveto(161.05188778,379.02530979)(161.0318878,378.94030987)(161.02189012,378.85031497)
\curveto(161.01188782,378.77031004)(160.99688783,378.69031012)(160.97689012,378.61031497)
\curveto(160.95688787,378.56031025)(160.95188788,378.5103103)(160.96189012,378.46031497)
\curveto(160.97188786,378.42031039)(160.96688786,378.38031043)(160.94689012,378.34031497)
\lineto(160.94689012,378.02531497)
\curveto(160.93688789,377.99531082)(160.9318879,377.96031085)(160.93189012,377.92031497)
\curveto(160.94188789,377.88031093)(160.94688788,377.83531098)(160.94689012,377.78531497)
\lineto(160.94689012,377.33531497)
\lineto(160.94689012,375.89531497)
\lineto(160.94689012,374.57531497)
\lineto(160.94689012,374.23031497)
\curveto(160.94688788,374.12031469)(160.92188791,374.03031478)(160.87189012,373.96031497)
\curveto(160.82188801,373.88031493)(160.7318881,373.84031497)(160.60189012,373.84031497)
\curveto(160.48188835,373.83031498)(160.35688847,373.82531499)(160.22689012,373.82531497)
\curveto(160.14688868,373.82531499)(160.07188876,373.83031498)(160.00189012,373.84031497)
\curveto(159.9318889,373.85031496)(159.87188896,373.87531494)(159.82189012,373.91531497)
\curveto(159.74188909,373.96531485)(159.70188913,374.06031475)(159.70189012,374.20031497)
\lineto(159.70189012,374.60531497)
\lineto(159.70189012,376.37531497)
\lineto(159.70189012,380.00531497)
\lineto(159.70189012,380.92031497)
\lineto(159.70189012,381.19031497)
\curveto(159.70188913,381.28030753)(159.72188911,381.35030746)(159.76189012,381.40031497)
\curveto(159.79188904,381.46030735)(159.84188899,381.50030731)(159.91189012,381.52031497)
\curveto(159.95188888,381.53030728)(160.00688882,381.54030727)(160.07689012,381.55031497)
\curveto(160.15688867,381.56030725)(160.23688859,381.56530725)(160.31689012,381.56531497)
\curveto(160.39688843,381.56530725)(160.47188836,381.56030725)(160.54189012,381.55031497)
\curveto(160.62188821,381.54030727)(160.67688815,381.52530729)(160.70689012,381.50531497)
\curveto(160.81688801,381.43530738)(160.86688796,381.34530747)(160.85689012,381.23531497)
\curveto(160.84688798,381.13530768)(160.86188797,381.02030779)(160.90189012,380.89031497)
\curveto(160.92188791,380.83030798)(160.96188787,380.78030803)(161.02189012,380.74031497)
\curveto(161.14188769,380.73030808)(161.23688759,380.77530804)(161.30689012,380.87531497)
\curveto(161.38688744,380.97530784)(161.46688736,381.05530776)(161.54689012,381.11531497)
\curveto(161.68688714,381.2153076)(161.826887,381.30530751)(161.96689012,381.38531497)
\curveto(162.11688671,381.47530734)(162.28688654,381.55030726)(162.47689012,381.61031497)
\curveto(162.55688627,381.64030717)(162.64188619,381.66030715)(162.73189012,381.67031497)
\curveto(162.831886,381.68030713)(162.9268859,381.69530712)(163.01689012,381.71531497)
\curveto(163.06688576,381.72530709)(163.11688571,381.73030708)(163.16689012,381.73031497)
\lineto(163.31689012,381.73031497)
}
}
{
\newrgbcolor{curcolor}{0 0 0}
\pscustom[linestyle=none,fillstyle=solid,fillcolor=curcolor]
{
\newpath
\moveto(168.92149949,383.92031497)
\curveto(169.07149748,383.92030489)(169.22149733,383.9153049)(169.37149949,383.90531497)
\curveto(169.52149703,383.90530491)(169.62649693,383.86530495)(169.68649949,383.78531497)
\curveto(169.73649682,383.72530509)(169.76149679,383.64030517)(169.76149949,383.53031497)
\curveto(169.77149678,383.43030538)(169.77649678,383.32530549)(169.77649949,383.21531497)
\lineto(169.77649949,382.34531497)
\curveto(169.77649678,382.26530655)(169.77149678,382.18030663)(169.76149949,382.09031497)
\curveto(169.76149679,382.0103068)(169.77149678,381.94030687)(169.79149949,381.88031497)
\curveto(169.83149672,381.74030707)(169.92149663,381.65030716)(170.06149949,381.61031497)
\curveto(170.11149644,381.60030721)(170.1564964,381.59530722)(170.19649949,381.59531497)
\lineto(170.34649949,381.59531497)
\lineto(170.75149949,381.59531497)
\curveto(170.91149564,381.60530721)(171.02649553,381.59530722)(171.09649949,381.56531497)
\curveto(171.18649537,381.50530731)(171.24649531,381.44530737)(171.27649949,381.38531497)
\curveto(171.29649526,381.34530747)(171.30649525,381.30030751)(171.30649949,381.25031497)
\lineto(171.30649949,381.10031497)
\curveto(171.30649525,380.99030782)(171.30149525,380.88530793)(171.29149949,380.78531497)
\curveto(171.28149527,380.69530812)(171.24649531,380.62530819)(171.18649949,380.57531497)
\curveto(171.12649543,380.52530829)(171.04149551,380.49530832)(170.93149949,380.48531497)
\lineto(170.60149949,380.48531497)
\curveto(170.49149606,380.49530832)(170.38149617,380.50030831)(170.27149949,380.50031497)
\curveto(170.16149639,380.50030831)(170.06649649,380.48530833)(169.98649949,380.45531497)
\curveto(169.91649664,380.42530839)(169.86649669,380.37530844)(169.83649949,380.30531497)
\curveto(169.80649675,380.23530858)(169.78649677,380.15030866)(169.77649949,380.05031497)
\curveto(169.76649679,379.96030885)(169.76149679,379.86030895)(169.76149949,379.75031497)
\curveto(169.77149678,379.65030916)(169.77649678,379.55030926)(169.77649949,379.45031497)
\lineto(169.77649949,376.48031497)
\curveto(169.77649678,376.26031255)(169.77149678,376.02531279)(169.76149949,375.77531497)
\curveto(169.76149679,375.53531328)(169.80649675,375.35031346)(169.89649949,375.22031497)
\curveto(169.94649661,375.14031367)(170.01149654,375.08531373)(170.09149949,375.05531497)
\curveto(170.17149638,375.02531379)(170.26649629,375.00031381)(170.37649949,374.98031497)
\curveto(170.40649615,374.97031384)(170.43649612,374.96531385)(170.46649949,374.96531497)
\curveto(170.50649605,374.97531384)(170.54149601,374.97531384)(170.57149949,374.96531497)
\lineto(170.76649949,374.96531497)
\curveto(170.86649569,374.96531385)(170.9564956,374.95531386)(171.03649949,374.93531497)
\curveto(171.12649543,374.92531389)(171.19149536,374.89031392)(171.23149949,374.83031497)
\curveto(171.2514953,374.80031401)(171.26649529,374.74531407)(171.27649949,374.66531497)
\curveto(171.29649526,374.59531422)(171.30649525,374.52031429)(171.30649949,374.44031497)
\curveto(171.31649524,374.36031445)(171.31649524,374.28031453)(171.30649949,374.20031497)
\curveto(171.29649526,374.13031468)(171.27649528,374.07531474)(171.24649949,374.03531497)
\curveto(171.20649535,373.96531485)(171.13149542,373.9153149)(171.02149949,373.88531497)
\curveto(170.94149561,373.86531495)(170.8514957,373.85531496)(170.75149949,373.85531497)
\curveto(170.6514959,373.86531495)(170.56149599,373.87031494)(170.48149949,373.87031497)
\curveto(170.42149613,373.87031494)(170.36149619,373.86531495)(170.30149949,373.85531497)
\curveto(170.24149631,373.85531496)(170.18649637,373.86031495)(170.13649949,373.87031497)
\lineto(169.95649949,373.87031497)
\curveto(169.90649665,373.88031493)(169.8564967,373.88531493)(169.80649949,373.88531497)
\curveto(169.76649679,373.89531492)(169.72149683,373.90031491)(169.67149949,373.90031497)
\curveto(169.47149708,373.95031486)(169.29649726,374.00531481)(169.14649949,374.06531497)
\curveto(169.00649755,374.12531469)(168.88649767,374.23031458)(168.78649949,374.38031497)
\curveto(168.64649791,374.58031423)(168.56649799,374.83031398)(168.54649949,375.13031497)
\curveto(168.52649803,375.44031337)(168.51649804,375.77031304)(168.51649949,376.12031497)
\lineto(168.51649949,380.05031497)
\curveto(168.48649807,380.18030863)(168.4564981,380.27530854)(168.42649949,380.33531497)
\curveto(168.40649815,380.39530842)(168.33649822,380.44530837)(168.21649949,380.48531497)
\curveto(168.17649838,380.49530832)(168.13649842,380.49530832)(168.09649949,380.48531497)
\curveto(168.0564985,380.47530834)(168.01649854,380.48030833)(167.97649949,380.50031497)
\lineto(167.73649949,380.50031497)
\curveto(167.60649895,380.50030831)(167.49649906,380.5103083)(167.40649949,380.53031497)
\curveto(167.32649923,380.56030825)(167.27149928,380.62030819)(167.24149949,380.71031497)
\curveto(167.22149933,380.75030806)(167.20649935,380.79530802)(167.19649949,380.84531497)
\lineto(167.19649949,380.99531497)
\curveto(167.19649936,381.13530768)(167.20649935,381.25030756)(167.22649949,381.34031497)
\curveto(167.24649931,381.44030737)(167.30649925,381.5153073)(167.40649949,381.56531497)
\curveto(167.51649904,381.60530721)(167.6564989,381.6153072)(167.82649949,381.59531497)
\curveto(168.00649855,381.57530724)(168.1564984,381.58530723)(168.27649949,381.62531497)
\curveto(168.36649819,381.67530714)(168.43649812,381.74530707)(168.48649949,381.83531497)
\curveto(168.50649805,381.89530692)(168.51649804,381.97030684)(168.51649949,382.06031497)
\lineto(168.51649949,382.31531497)
\lineto(168.51649949,383.24531497)
\lineto(168.51649949,383.48531497)
\curveto(168.51649804,383.57530524)(168.52649803,383.65030516)(168.54649949,383.71031497)
\curveto(168.58649797,383.79030502)(168.66149789,383.85530496)(168.77149949,383.90531497)
\curveto(168.80149775,383.90530491)(168.82649773,383.90530491)(168.84649949,383.90531497)
\curveto(168.87649768,383.9153049)(168.90149765,383.92030489)(168.92149949,383.92031497)
}
}
{
\newrgbcolor{curcolor}{0 0 0}
\pscustom[linestyle=none,fillstyle=solid,fillcolor=curcolor]
{
\newpath
\moveto(179.57829637,374.41031497)
\curveto(179.60828854,374.25031456)(179.59328855,374.1153147)(179.53329637,374.00531497)
\curveto(179.47328867,373.90531491)(179.39328875,373.83031498)(179.29329637,373.78031497)
\curveto(179.2432889,373.76031505)(179.18828896,373.75031506)(179.12829637,373.75031497)
\curveto(179.07828907,373.75031506)(179.02328912,373.74031507)(178.96329637,373.72031497)
\curveto(178.7432894,373.67031514)(178.52328962,373.68531513)(178.30329637,373.76531497)
\curveto(178.09329005,373.83531498)(177.9482902,373.92531489)(177.86829637,374.03531497)
\curveto(177.81829033,374.10531471)(177.77329037,374.18531463)(177.73329637,374.27531497)
\curveto(177.69329045,374.37531444)(177.6432905,374.45531436)(177.58329637,374.51531497)
\curveto(177.56329058,374.53531428)(177.53829061,374.55531426)(177.50829637,374.57531497)
\curveto(177.48829066,374.59531422)(177.45829069,374.60031421)(177.41829637,374.59031497)
\curveto(177.30829084,374.56031425)(177.20329094,374.50531431)(177.10329637,374.42531497)
\curveto(177.01329113,374.34531447)(176.92329122,374.27531454)(176.83329637,374.21531497)
\curveto(176.70329144,374.13531468)(176.56329158,374.06031475)(176.41329637,373.99031497)
\curveto(176.26329188,373.93031488)(176.10329204,373.87531494)(175.93329637,373.82531497)
\curveto(175.83329231,373.79531502)(175.72329242,373.77531504)(175.60329637,373.76531497)
\curveto(175.49329265,373.75531506)(175.38329276,373.74031507)(175.27329637,373.72031497)
\curveto(175.22329292,373.7103151)(175.17829297,373.70531511)(175.13829637,373.70531497)
\lineto(175.03329637,373.70531497)
\curveto(174.92329322,373.68531513)(174.81829333,373.68531513)(174.71829637,373.70531497)
\lineto(174.58329637,373.70531497)
\curveto(174.53329361,373.7153151)(174.48329366,373.72031509)(174.43329637,373.72031497)
\curveto(174.38329376,373.72031509)(174.33829381,373.73031508)(174.29829637,373.75031497)
\curveto(174.25829389,373.76031505)(174.22329392,373.76531505)(174.19329637,373.76531497)
\curveto(174.17329397,373.75531506)(174.148294,373.75531506)(174.11829637,373.76531497)
\lineto(173.87829637,373.82531497)
\curveto(173.79829435,373.83531498)(173.72329442,373.85531496)(173.65329637,373.88531497)
\curveto(173.35329479,374.0153148)(173.10829504,374.16031465)(172.91829637,374.32031497)
\curveto(172.73829541,374.49031432)(172.58829556,374.72531409)(172.46829637,375.02531497)
\curveto(172.37829577,375.24531357)(172.33329581,375.5103133)(172.33329637,375.82031497)
\lineto(172.33329637,376.13531497)
\curveto(172.3432958,376.18531263)(172.3482958,376.23531258)(172.34829637,376.28531497)
\lineto(172.37829637,376.46531497)
\lineto(172.49829637,376.79531497)
\curveto(172.53829561,376.90531191)(172.58829556,377.00531181)(172.64829637,377.09531497)
\curveto(172.82829532,377.38531143)(173.07329507,377.60031121)(173.38329637,377.74031497)
\curveto(173.69329445,377.88031093)(174.03329411,378.00531081)(174.40329637,378.11531497)
\curveto(174.5432936,378.15531066)(174.68829346,378.18531063)(174.83829637,378.20531497)
\curveto(174.98829316,378.22531059)(175.13829301,378.25031056)(175.28829637,378.28031497)
\curveto(175.35829279,378.30031051)(175.42329272,378.3103105)(175.48329637,378.31031497)
\curveto(175.55329259,378.3103105)(175.62829252,378.32031049)(175.70829637,378.34031497)
\curveto(175.77829237,378.36031045)(175.8482923,378.37031044)(175.91829637,378.37031497)
\curveto(175.98829216,378.38031043)(176.06329208,378.39531042)(176.14329637,378.41531497)
\curveto(176.39329175,378.47531034)(176.62829152,378.52531029)(176.84829637,378.56531497)
\curveto(177.06829108,378.6153102)(177.2432909,378.73031008)(177.37329637,378.91031497)
\curveto(177.43329071,378.99030982)(177.48329066,379.09030972)(177.52329637,379.21031497)
\curveto(177.56329058,379.34030947)(177.56329058,379.48030933)(177.52329637,379.63031497)
\curveto(177.46329068,379.87030894)(177.37329077,380.06030875)(177.25329637,380.20031497)
\curveto(177.143291,380.34030847)(176.98329116,380.45030836)(176.77329637,380.53031497)
\curveto(176.65329149,380.58030823)(176.50829164,380.6153082)(176.33829637,380.63531497)
\curveto(176.17829197,380.65530816)(176.00829214,380.66530815)(175.82829637,380.66531497)
\curveto(175.6482925,380.66530815)(175.47329267,380.65530816)(175.30329637,380.63531497)
\curveto(175.13329301,380.6153082)(174.98829316,380.58530823)(174.86829637,380.54531497)
\curveto(174.69829345,380.48530833)(174.53329361,380.40030841)(174.37329637,380.29031497)
\curveto(174.29329385,380.23030858)(174.21829393,380.15030866)(174.14829637,380.05031497)
\curveto(174.08829406,379.96030885)(174.03329411,379.86030895)(173.98329637,379.75031497)
\curveto(173.95329419,379.67030914)(173.92329422,379.58530923)(173.89329637,379.49531497)
\curveto(173.87329427,379.40530941)(173.82829432,379.33530948)(173.75829637,379.28531497)
\curveto(173.71829443,379.25530956)(173.6482945,379.23030958)(173.54829637,379.21031497)
\curveto(173.45829469,379.20030961)(173.36329478,379.19530962)(173.26329637,379.19531497)
\curveto(173.16329498,379.19530962)(173.06329508,379.20030961)(172.96329637,379.21031497)
\curveto(172.87329527,379.23030958)(172.80829534,379.25530956)(172.76829637,379.28531497)
\curveto(172.72829542,379.3153095)(172.69829545,379.36530945)(172.67829637,379.43531497)
\curveto(172.65829549,379.50530931)(172.65829549,379.58030923)(172.67829637,379.66031497)
\curveto(172.70829544,379.79030902)(172.73829541,379.9103089)(172.76829637,380.02031497)
\curveto(172.80829534,380.14030867)(172.85329529,380.25530856)(172.90329637,380.36531497)
\curveto(173.09329505,380.7153081)(173.33329481,380.98530783)(173.62329637,381.17531497)
\curveto(173.91329423,381.37530744)(174.27329387,381.53530728)(174.70329637,381.65531497)
\curveto(174.80329334,381.67530714)(174.90329324,381.69030712)(175.00329637,381.70031497)
\curveto(175.11329303,381.7103071)(175.22329292,381.72530709)(175.33329637,381.74531497)
\curveto(175.37329277,381.75530706)(175.43829271,381.75530706)(175.52829637,381.74531497)
\curveto(175.61829253,381.74530707)(175.67329247,381.75530706)(175.69329637,381.77531497)
\curveto(176.39329175,381.78530703)(177.00329114,381.70530711)(177.52329637,381.53531497)
\curveto(178.0432901,381.36530745)(178.40828974,381.04030777)(178.61829637,380.56031497)
\curveto(178.70828944,380.36030845)(178.75828939,380.12530869)(178.76829637,379.85531497)
\curveto(178.78828936,379.59530922)(178.79828935,379.32030949)(178.79829637,379.03031497)
\lineto(178.79829637,375.71531497)
\curveto(178.79828935,375.57531324)(178.80328934,375.44031337)(178.81329637,375.31031497)
\curveto(178.82328932,375.18031363)(178.85328929,375.07531374)(178.90329637,374.99531497)
\curveto(178.95328919,374.92531389)(179.01828913,374.87531394)(179.09829637,374.84531497)
\curveto(179.18828896,374.80531401)(179.27328887,374.77531404)(179.35329637,374.75531497)
\curveto(179.43328871,374.74531407)(179.49328865,374.70031411)(179.53329637,374.62031497)
\curveto(179.55328859,374.59031422)(179.56328858,374.56031425)(179.56329637,374.53031497)
\curveto(179.56328858,374.50031431)(179.56828858,374.46031435)(179.57829637,374.41031497)
\moveto(177.43329637,376.07531497)
\curveto(177.49329065,376.2153126)(177.52329062,376.37531244)(177.52329637,376.55531497)
\curveto(177.53329061,376.74531207)(177.53829061,376.94031187)(177.53829637,377.14031497)
\curveto(177.53829061,377.25031156)(177.53329061,377.35031146)(177.52329637,377.44031497)
\curveto(177.51329063,377.53031128)(177.47329067,377.60031121)(177.40329637,377.65031497)
\curveto(177.37329077,377.67031114)(177.30329084,377.68031113)(177.19329637,377.68031497)
\curveto(177.17329097,377.66031115)(177.13829101,377.65031116)(177.08829637,377.65031497)
\curveto(177.03829111,377.65031116)(176.99329115,377.64031117)(176.95329637,377.62031497)
\curveto(176.87329127,377.60031121)(176.78329136,377.58031123)(176.68329637,377.56031497)
\lineto(176.38329637,377.50031497)
\curveto(176.35329179,377.50031131)(176.31829183,377.49531132)(176.27829637,377.48531497)
\lineto(176.17329637,377.48531497)
\curveto(176.02329212,377.44531137)(175.85829229,377.42031139)(175.67829637,377.41031497)
\curveto(175.50829264,377.4103114)(175.3482928,377.39031142)(175.19829637,377.35031497)
\curveto(175.11829303,377.33031148)(175.0432931,377.3103115)(174.97329637,377.29031497)
\curveto(174.91329323,377.28031153)(174.8432933,377.26531155)(174.76329637,377.24531497)
\curveto(174.60329354,377.19531162)(174.45329369,377.13031168)(174.31329637,377.05031497)
\curveto(174.17329397,376.98031183)(174.05329409,376.89031192)(173.95329637,376.78031497)
\curveto(173.85329429,376.67031214)(173.77829437,376.53531228)(173.72829637,376.37531497)
\curveto(173.67829447,376.22531259)(173.65829449,376.04031277)(173.66829637,375.82031497)
\curveto(173.66829448,375.72031309)(173.68329446,375.62531319)(173.71329637,375.53531497)
\curveto(173.75329439,375.45531336)(173.79829435,375.38031343)(173.84829637,375.31031497)
\curveto(173.92829422,375.20031361)(174.03329411,375.10531371)(174.16329637,375.02531497)
\curveto(174.29329385,374.95531386)(174.43329371,374.89531392)(174.58329637,374.84531497)
\curveto(174.63329351,374.83531398)(174.68329346,374.83031398)(174.73329637,374.83031497)
\curveto(174.78329336,374.83031398)(174.83329331,374.82531399)(174.88329637,374.81531497)
\curveto(174.95329319,374.79531402)(175.03829311,374.78031403)(175.13829637,374.77031497)
\curveto(175.2482929,374.77031404)(175.33829281,374.78031403)(175.40829637,374.80031497)
\curveto(175.46829268,374.82031399)(175.52829262,374.82531399)(175.58829637,374.81531497)
\curveto(175.6482925,374.815314)(175.70829244,374.82531399)(175.76829637,374.84531497)
\curveto(175.8482923,374.86531395)(175.92329222,374.88031393)(175.99329637,374.89031497)
\curveto(176.07329207,374.90031391)(176.148292,374.92031389)(176.21829637,374.95031497)
\curveto(176.50829164,375.07031374)(176.75329139,375.2153136)(176.95329637,375.38531497)
\curveto(177.16329098,375.55531326)(177.32329082,375.78531303)(177.43329637,376.07531497)
}
}
{
\newrgbcolor{curcolor}{0 0 0}
\pscustom[linestyle=none,fillstyle=solid,fillcolor=curcolor]
{
\newpath
\moveto(183.88493699,381.76031497)
\curveto(184.6249322,381.77030704)(185.23993159,381.66030715)(185.72993699,381.43031497)
\curveto(186.2299306,381.2103076)(186.6249302,380.87530794)(186.91493699,380.42531497)
\curveto(187.04492978,380.22530859)(187.15492967,379.98030883)(187.24493699,379.69031497)
\curveto(187.26492956,379.64030917)(187.27992955,379.57530924)(187.28993699,379.49531497)
\curveto(187.29992953,379.4153094)(187.29492953,379.34530947)(187.27493699,379.28531497)
\curveto(187.24492958,379.23530958)(187.19492963,379.19030962)(187.12493699,379.15031497)
\curveto(187.09492973,379.13030968)(187.06492976,379.12030969)(187.03493699,379.12031497)
\curveto(187.00492982,379.13030968)(186.96992986,379.13030968)(186.92993699,379.12031497)
\curveto(186.88992994,379.1103097)(186.84992998,379.10530971)(186.80993699,379.10531497)
\curveto(186.76993006,379.1153097)(186.7299301,379.12030969)(186.68993699,379.12031497)
\lineto(186.37493699,379.12031497)
\curveto(186.27493055,379.13030968)(186.18993064,379.16030965)(186.11993699,379.21031497)
\curveto(186.03993079,379.27030954)(185.98493084,379.35530946)(185.95493699,379.46531497)
\curveto(185.9249309,379.57530924)(185.88493094,379.67030914)(185.83493699,379.75031497)
\curveto(185.68493114,380.0103088)(185.48993134,380.2153086)(185.24993699,380.36531497)
\curveto(185.16993166,380.4153084)(185.08493174,380.45530836)(184.99493699,380.48531497)
\curveto(184.90493192,380.52530829)(184.80993202,380.56030825)(184.70993699,380.59031497)
\curveto(184.56993226,380.63030818)(184.38493244,380.65030816)(184.15493699,380.65031497)
\curveto(183.9249329,380.66030815)(183.73493309,380.64030817)(183.58493699,380.59031497)
\curveto(183.51493331,380.57030824)(183.44993338,380.55530826)(183.38993699,380.54531497)
\curveto(183.3299335,380.53530828)(183.26493356,380.52030829)(183.19493699,380.50031497)
\curveto(182.93493389,380.39030842)(182.70493412,380.24030857)(182.50493699,380.05031497)
\curveto(182.30493452,379.86030895)(182.14993468,379.63530918)(182.03993699,379.37531497)
\curveto(181.99993483,379.28530953)(181.96493486,379.19030962)(181.93493699,379.09031497)
\curveto(181.90493492,379.00030981)(181.87493495,378.90030991)(181.84493699,378.79031497)
\lineto(181.75493699,378.38531497)
\curveto(181.74493508,378.33531048)(181.73993509,378.28031053)(181.73993699,378.22031497)
\curveto(181.74993508,378.16031065)(181.74493508,378.10531071)(181.72493699,378.05531497)
\lineto(181.72493699,377.93531497)
\curveto(181.71493511,377.89531092)(181.70493512,377.83031098)(181.69493699,377.74031497)
\curveto(181.69493513,377.65031116)(181.70493512,377.58531123)(181.72493699,377.54531497)
\curveto(181.73493509,377.49531132)(181.73493509,377.44531137)(181.72493699,377.39531497)
\curveto(181.71493511,377.34531147)(181.71493511,377.29531152)(181.72493699,377.24531497)
\curveto(181.73493509,377.20531161)(181.73993509,377.13531168)(181.73993699,377.03531497)
\curveto(181.75993507,376.95531186)(181.77493505,376.87031194)(181.78493699,376.78031497)
\curveto(181.80493502,376.69031212)(181.824935,376.60531221)(181.84493699,376.52531497)
\curveto(181.95493487,376.20531261)(182.07993475,375.92531289)(182.21993699,375.68531497)
\curveto(182.36993446,375.45531336)(182.57493425,375.25531356)(182.83493699,375.08531497)
\curveto(182.9249339,375.03531378)(183.01493381,374.99031382)(183.10493699,374.95031497)
\curveto(183.20493362,374.9103139)(183.30993352,374.87031394)(183.41993699,374.83031497)
\curveto(183.46993336,374.82031399)(183.50993332,374.815314)(183.53993699,374.81531497)
\curveto(183.56993326,374.815314)(183.60993322,374.810314)(183.65993699,374.80031497)
\curveto(183.68993314,374.79031402)(183.73993309,374.78531403)(183.80993699,374.78531497)
\lineto(183.97493699,374.78531497)
\curveto(183.97493285,374.77531404)(183.99493283,374.77031404)(184.03493699,374.77031497)
\curveto(184.05493277,374.78031403)(184.07993275,374.78031403)(184.10993699,374.77031497)
\curveto(184.13993269,374.77031404)(184.16993266,374.77531404)(184.19993699,374.78531497)
\curveto(184.26993256,374.80531401)(184.33493249,374.810314)(184.39493699,374.80031497)
\curveto(184.46493236,374.80031401)(184.53493229,374.810314)(184.60493699,374.83031497)
\curveto(184.86493196,374.9103139)(185.08993174,375.0103138)(185.27993699,375.13031497)
\curveto(185.46993136,375.26031355)(185.6299312,375.42531339)(185.75993699,375.62531497)
\curveto(185.80993102,375.70531311)(185.85493097,375.79031302)(185.89493699,375.88031497)
\lineto(186.01493699,376.15031497)
\curveto(186.03493079,376.23031258)(186.05493077,376.30531251)(186.07493699,376.37531497)
\curveto(186.10493072,376.45531236)(186.15493067,376.52031229)(186.22493699,376.57031497)
\curveto(186.25493057,376.60031221)(186.31493051,376.62031219)(186.40493699,376.63031497)
\curveto(186.49493033,376.65031216)(186.58993024,376.66031215)(186.68993699,376.66031497)
\curveto(186.79993003,376.67031214)(186.89992993,376.67031214)(186.98993699,376.66031497)
\curveto(187.08992974,376.65031216)(187.15992967,376.63031218)(187.19993699,376.60031497)
\curveto(187.25992957,376.56031225)(187.29492953,376.50031231)(187.30493699,376.42031497)
\curveto(187.3249295,376.34031247)(187.3249295,376.25531256)(187.30493699,376.16531497)
\curveto(187.25492957,376.0153128)(187.20492962,375.87031294)(187.15493699,375.73031497)
\curveto(187.11492971,375.60031321)(187.05992977,375.47031334)(186.98993699,375.34031497)
\curveto(186.83992999,375.04031377)(186.64993018,374.77531404)(186.41993699,374.54531497)
\curveto(186.19993063,374.3153145)(185.9299309,374.13031468)(185.60993699,373.99031497)
\curveto(185.5299313,373.95031486)(185.44493138,373.9153149)(185.35493699,373.88531497)
\curveto(185.26493156,373.86531495)(185.16993166,373.84031497)(185.06993699,373.81031497)
\curveto(184.95993187,373.77031504)(184.84993198,373.75031506)(184.73993699,373.75031497)
\curveto(184.6299322,373.74031507)(184.51993231,373.72531509)(184.40993699,373.70531497)
\curveto(184.36993246,373.68531513)(184.3299325,373.68031513)(184.28993699,373.69031497)
\curveto(184.24993258,373.70031511)(184.20993262,373.70031511)(184.16993699,373.69031497)
\lineto(184.03493699,373.69031497)
\lineto(183.79493699,373.69031497)
\curveto(183.7249331,373.68031513)(183.65993317,373.68531513)(183.59993699,373.70531497)
\lineto(183.52493699,373.70531497)
\lineto(183.16493699,373.75031497)
\curveto(183.03493379,373.79031502)(182.90993392,373.82531499)(182.78993699,373.85531497)
\curveto(182.66993416,373.88531493)(182.55493427,373.92531489)(182.44493699,373.97531497)
\curveto(182.08493474,374.13531468)(181.78493504,374.32531449)(181.54493699,374.54531497)
\curveto(181.31493551,374.76531405)(181.09993573,375.03531378)(180.89993699,375.35531497)
\curveto(180.84993598,375.43531338)(180.80493602,375.52531329)(180.76493699,375.62531497)
\lineto(180.64493699,375.92531497)
\curveto(180.59493623,376.03531278)(180.55993627,376.15031266)(180.53993699,376.27031497)
\curveto(180.51993631,376.39031242)(180.49493633,376.5103123)(180.46493699,376.63031497)
\curveto(180.45493637,376.67031214)(180.44993638,376.7103121)(180.44993699,376.75031497)
\curveto(180.44993638,376.79031202)(180.44493638,376.83031198)(180.43493699,376.87031497)
\curveto(180.41493641,376.93031188)(180.40493642,376.99531182)(180.40493699,377.06531497)
\curveto(180.41493641,377.13531168)(180.40993642,377.20031161)(180.38993699,377.26031497)
\lineto(180.38993699,377.41031497)
\curveto(180.37993645,377.46031135)(180.37493645,377.53031128)(180.37493699,377.62031497)
\curveto(180.37493645,377.7103111)(180.37993645,377.78031103)(180.38993699,377.83031497)
\curveto(180.39993643,377.88031093)(180.39993643,377.92531089)(180.38993699,377.96531497)
\curveto(180.38993644,378.00531081)(180.39493643,378.04531077)(180.40493699,378.08531497)
\curveto(180.4249364,378.15531066)(180.4299364,378.22531059)(180.41993699,378.29531497)
\curveto(180.41993641,378.36531045)(180.4299364,378.43031038)(180.44993699,378.49031497)
\curveto(180.48993634,378.66031015)(180.5249363,378.83030998)(180.55493699,379.00031497)
\curveto(180.58493624,379.17030964)(180.6299362,379.33030948)(180.68993699,379.48031497)
\curveto(180.89993593,380.00030881)(181.15493567,380.42030839)(181.45493699,380.74031497)
\curveto(181.75493507,381.06030775)(182.16493466,381.32530749)(182.68493699,381.53531497)
\curveto(182.79493403,381.58530723)(182.91493391,381.62030719)(183.04493699,381.64031497)
\curveto(183.17493365,381.66030715)(183.30993352,381.68530713)(183.44993699,381.71531497)
\curveto(183.51993331,381.72530709)(183.58993324,381.73030708)(183.65993699,381.73031497)
\curveto(183.7299331,381.74030707)(183.80493302,381.75030706)(183.88493699,381.76031497)
}
}
{
\newrgbcolor{curcolor}{0 0 0}
\pscustom[linestyle=none,fillstyle=solid,fillcolor=curcolor]
{
\newpath
\moveto(189.75157762,383.92031497)
\curveto(189.90157561,383.92030489)(190.05157546,383.9153049)(190.20157762,383.90531497)
\curveto(190.35157516,383.90530491)(190.45657505,383.86530495)(190.51657762,383.78531497)
\curveto(190.56657494,383.72530509)(190.59157492,383.64030517)(190.59157762,383.53031497)
\curveto(190.60157491,383.43030538)(190.6065749,383.32530549)(190.60657762,383.21531497)
\lineto(190.60657762,382.34531497)
\curveto(190.6065749,382.26530655)(190.60157491,382.18030663)(190.59157762,382.09031497)
\curveto(190.59157492,382.0103068)(190.60157491,381.94030687)(190.62157762,381.88031497)
\curveto(190.66157485,381.74030707)(190.75157476,381.65030716)(190.89157762,381.61031497)
\curveto(190.94157457,381.60030721)(190.98657452,381.59530722)(191.02657762,381.59531497)
\lineto(191.17657762,381.59531497)
\lineto(191.58157762,381.59531497)
\curveto(191.74157377,381.60530721)(191.85657365,381.59530722)(191.92657762,381.56531497)
\curveto(192.01657349,381.50530731)(192.07657343,381.44530737)(192.10657762,381.38531497)
\curveto(192.12657338,381.34530747)(192.13657337,381.30030751)(192.13657762,381.25031497)
\lineto(192.13657762,381.10031497)
\curveto(192.13657337,380.99030782)(192.13157338,380.88530793)(192.12157762,380.78531497)
\curveto(192.1115734,380.69530812)(192.07657343,380.62530819)(192.01657762,380.57531497)
\curveto(191.95657355,380.52530829)(191.87157364,380.49530832)(191.76157762,380.48531497)
\lineto(191.43157762,380.48531497)
\curveto(191.32157419,380.49530832)(191.2115743,380.50030831)(191.10157762,380.50031497)
\curveto(190.99157452,380.50030831)(190.89657461,380.48530833)(190.81657762,380.45531497)
\curveto(190.74657476,380.42530839)(190.69657481,380.37530844)(190.66657762,380.30531497)
\curveto(190.63657487,380.23530858)(190.61657489,380.15030866)(190.60657762,380.05031497)
\curveto(190.59657491,379.96030885)(190.59157492,379.86030895)(190.59157762,379.75031497)
\curveto(190.60157491,379.65030916)(190.6065749,379.55030926)(190.60657762,379.45031497)
\lineto(190.60657762,376.48031497)
\curveto(190.6065749,376.26031255)(190.60157491,376.02531279)(190.59157762,375.77531497)
\curveto(190.59157492,375.53531328)(190.63657487,375.35031346)(190.72657762,375.22031497)
\curveto(190.77657473,375.14031367)(190.84157467,375.08531373)(190.92157762,375.05531497)
\curveto(191.00157451,375.02531379)(191.09657441,375.00031381)(191.20657762,374.98031497)
\curveto(191.23657427,374.97031384)(191.26657424,374.96531385)(191.29657762,374.96531497)
\curveto(191.33657417,374.97531384)(191.37157414,374.97531384)(191.40157762,374.96531497)
\lineto(191.59657762,374.96531497)
\curveto(191.69657381,374.96531385)(191.78657372,374.95531386)(191.86657762,374.93531497)
\curveto(191.95657355,374.92531389)(192.02157349,374.89031392)(192.06157762,374.83031497)
\curveto(192.08157343,374.80031401)(192.09657341,374.74531407)(192.10657762,374.66531497)
\curveto(192.12657338,374.59531422)(192.13657337,374.52031429)(192.13657762,374.44031497)
\curveto(192.14657336,374.36031445)(192.14657336,374.28031453)(192.13657762,374.20031497)
\curveto(192.12657338,374.13031468)(192.1065734,374.07531474)(192.07657762,374.03531497)
\curveto(192.03657347,373.96531485)(191.96157355,373.9153149)(191.85157762,373.88531497)
\curveto(191.77157374,373.86531495)(191.68157383,373.85531496)(191.58157762,373.85531497)
\curveto(191.48157403,373.86531495)(191.39157412,373.87031494)(191.31157762,373.87031497)
\curveto(191.25157426,373.87031494)(191.19157432,373.86531495)(191.13157762,373.85531497)
\curveto(191.07157444,373.85531496)(191.01657449,373.86031495)(190.96657762,373.87031497)
\lineto(190.78657762,373.87031497)
\curveto(190.73657477,373.88031493)(190.68657482,373.88531493)(190.63657762,373.88531497)
\curveto(190.59657491,373.89531492)(190.55157496,373.90031491)(190.50157762,373.90031497)
\curveto(190.30157521,373.95031486)(190.12657538,374.00531481)(189.97657762,374.06531497)
\curveto(189.83657567,374.12531469)(189.71657579,374.23031458)(189.61657762,374.38031497)
\curveto(189.47657603,374.58031423)(189.39657611,374.83031398)(189.37657762,375.13031497)
\curveto(189.35657615,375.44031337)(189.34657616,375.77031304)(189.34657762,376.12031497)
\lineto(189.34657762,380.05031497)
\curveto(189.31657619,380.18030863)(189.28657622,380.27530854)(189.25657762,380.33531497)
\curveto(189.23657627,380.39530842)(189.16657634,380.44530837)(189.04657762,380.48531497)
\curveto(189.0065765,380.49530832)(188.96657654,380.49530832)(188.92657762,380.48531497)
\curveto(188.88657662,380.47530834)(188.84657666,380.48030833)(188.80657762,380.50031497)
\lineto(188.56657762,380.50031497)
\curveto(188.43657707,380.50030831)(188.32657718,380.5103083)(188.23657762,380.53031497)
\curveto(188.15657735,380.56030825)(188.10157741,380.62030819)(188.07157762,380.71031497)
\curveto(188.05157746,380.75030806)(188.03657747,380.79530802)(188.02657762,380.84531497)
\lineto(188.02657762,380.99531497)
\curveto(188.02657748,381.13530768)(188.03657747,381.25030756)(188.05657762,381.34031497)
\curveto(188.07657743,381.44030737)(188.13657737,381.5153073)(188.23657762,381.56531497)
\curveto(188.34657716,381.60530721)(188.48657702,381.6153072)(188.65657762,381.59531497)
\curveto(188.83657667,381.57530724)(188.98657652,381.58530723)(189.10657762,381.62531497)
\curveto(189.19657631,381.67530714)(189.26657624,381.74530707)(189.31657762,381.83531497)
\curveto(189.33657617,381.89530692)(189.34657616,381.97030684)(189.34657762,382.06031497)
\lineto(189.34657762,382.31531497)
\lineto(189.34657762,383.24531497)
\lineto(189.34657762,383.48531497)
\curveto(189.34657616,383.57530524)(189.35657615,383.65030516)(189.37657762,383.71031497)
\curveto(189.41657609,383.79030502)(189.49157602,383.85530496)(189.60157762,383.90531497)
\curveto(189.63157588,383.90530491)(189.65657585,383.90530491)(189.67657762,383.90531497)
\curveto(189.7065758,383.9153049)(189.73157578,383.92030489)(189.75157762,383.92031497)
}
}
{
\newrgbcolor{curcolor}{0 0 0}
\pscustom[linestyle=none,fillstyle=solid,fillcolor=curcolor]
{
\newpath
\moveto(200.64837449,378.05531497)
\curveto(200.66836643,377.99531082)(200.67836642,377.90031091)(200.67837449,377.77031497)
\curveto(200.67836642,377.65031116)(200.67336643,377.56531125)(200.66337449,377.51531497)
\lineto(200.66337449,377.36531497)
\curveto(200.65336645,377.28531153)(200.64336646,377.2103116)(200.63337449,377.14031497)
\curveto(200.63336647,377.08031173)(200.62836647,377.0103118)(200.61837449,376.93031497)
\curveto(200.5983665,376.87031194)(200.58336652,376.810312)(200.57337449,376.75031497)
\curveto(200.57336653,376.69031212)(200.56336654,376.63031218)(200.54337449,376.57031497)
\curveto(200.5033666,376.44031237)(200.46836663,376.3103125)(200.43837449,376.18031497)
\curveto(200.40836669,376.05031276)(200.36836673,375.93031288)(200.31837449,375.82031497)
\curveto(200.10836699,375.34031347)(199.82836727,374.93531388)(199.47837449,374.60531497)
\curveto(199.12836797,374.28531453)(198.6983684,374.04031477)(198.18837449,373.87031497)
\curveto(198.07836902,373.83031498)(197.95836914,373.80031501)(197.82837449,373.78031497)
\curveto(197.70836939,373.76031505)(197.58336952,373.74031507)(197.45337449,373.72031497)
\curveto(197.39336971,373.7103151)(197.32836977,373.70531511)(197.25837449,373.70531497)
\curveto(197.1983699,373.69531512)(197.13836996,373.69031512)(197.07837449,373.69031497)
\curveto(197.03837006,373.68031513)(196.97837012,373.67531514)(196.89837449,373.67531497)
\curveto(196.82837027,373.67531514)(196.77837032,373.68031513)(196.74837449,373.69031497)
\curveto(196.70837039,373.70031511)(196.66837043,373.70531511)(196.62837449,373.70531497)
\curveto(196.58837051,373.69531512)(196.55337055,373.69531512)(196.52337449,373.70531497)
\lineto(196.43337449,373.70531497)
\lineto(196.07337449,373.75031497)
\curveto(195.93337117,373.79031502)(195.7983713,373.83031498)(195.66837449,373.87031497)
\curveto(195.53837156,373.9103149)(195.41337169,373.95531486)(195.29337449,374.00531497)
\curveto(194.84337226,374.20531461)(194.47337263,374.46531435)(194.18337449,374.78531497)
\curveto(193.89337321,375.10531371)(193.65337345,375.49531332)(193.46337449,375.95531497)
\curveto(193.41337369,376.05531276)(193.37337373,376.15531266)(193.34337449,376.25531497)
\curveto(193.32337378,376.35531246)(193.3033738,376.46031235)(193.28337449,376.57031497)
\curveto(193.26337384,376.6103122)(193.25337385,376.64031217)(193.25337449,376.66031497)
\curveto(193.26337384,376.69031212)(193.26337384,376.72531209)(193.25337449,376.76531497)
\curveto(193.23337387,376.84531197)(193.21837388,376.92531189)(193.20837449,377.00531497)
\curveto(193.20837389,377.09531172)(193.1983739,377.18031163)(193.17837449,377.26031497)
\lineto(193.17837449,377.38031497)
\curveto(193.17837392,377.42031139)(193.17337393,377.46531135)(193.16337449,377.51531497)
\curveto(193.15337395,377.56531125)(193.14837395,377.65031116)(193.14837449,377.77031497)
\curveto(193.14837395,377.90031091)(193.15837394,377.99531082)(193.17837449,378.05531497)
\curveto(193.1983739,378.12531069)(193.2033739,378.19531062)(193.19337449,378.26531497)
\curveto(193.18337392,378.33531048)(193.18837391,378.40531041)(193.20837449,378.47531497)
\curveto(193.21837388,378.52531029)(193.22337388,378.56531025)(193.22337449,378.59531497)
\curveto(193.23337387,378.63531018)(193.24337386,378.68031013)(193.25337449,378.73031497)
\curveto(193.28337382,378.85030996)(193.30837379,378.97030984)(193.32837449,379.09031497)
\curveto(193.35837374,379.2103096)(193.3983737,379.32530949)(193.44837449,379.43531497)
\curveto(193.5983735,379.80530901)(193.77837332,380.13530868)(193.98837449,380.42531497)
\curveto(194.20837289,380.72530809)(194.47337263,380.97530784)(194.78337449,381.17531497)
\curveto(194.9033722,381.25530756)(195.02837207,381.32030749)(195.15837449,381.37031497)
\curveto(195.28837181,381.43030738)(195.42337168,381.49030732)(195.56337449,381.55031497)
\curveto(195.68337142,381.60030721)(195.81337129,381.63030718)(195.95337449,381.64031497)
\curveto(196.09337101,381.66030715)(196.23337087,381.69030712)(196.37337449,381.73031497)
\lineto(196.56837449,381.73031497)
\curveto(196.63837046,381.74030707)(196.7033704,381.75030706)(196.76337449,381.76031497)
\curveto(197.65336945,381.77030704)(198.39336871,381.58530723)(198.98337449,381.20531497)
\curveto(199.57336753,380.82530799)(199.9983671,380.33030848)(200.25837449,379.72031497)
\curveto(200.30836679,379.62030919)(200.34836675,379.52030929)(200.37837449,379.42031497)
\curveto(200.40836669,379.32030949)(200.44336666,379.2153096)(200.48337449,379.10531497)
\curveto(200.51336659,378.99530982)(200.53836656,378.87530994)(200.55837449,378.74531497)
\curveto(200.57836652,378.62531019)(200.6033665,378.50031031)(200.63337449,378.37031497)
\curveto(200.64336646,378.32031049)(200.64336646,378.26531055)(200.63337449,378.20531497)
\curveto(200.63336647,378.15531066)(200.63836646,378.10531071)(200.64837449,378.05531497)
\moveto(199.31337449,377.20031497)
\curveto(199.33336777,377.27031154)(199.33836776,377.35031146)(199.32837449,377.44031497)
\lineto(199.32837449,377.69531497)
\curveto(199.32836777,378.08531073)(199.29336781,378.4153104)(199.22337449,378.68531497)
\curveto(199.19336791,378.76531005)(199.16836793,378.84530997)(199.14837449,378.92531497)
\curveto(199.12836797,379.00530981)(199.103368,379.08030973)(199.07337449,379.15031497)
\curveto(198.79336831,379.80030901)(198.34836875,380.25030856)(197.73837449,380.50031497)
\curveto(197.66836943,380.53030828)(197.59336951,380.55030826)(197.51337449,380.56031497)
\lineto(197.27337449,380.62031497)
\curveto(197.19336991,380.64030817)(197.10836999,380.65030816)(197.01837449,380.65031497)
\lineto(196.74837449,380.65031497)
\lineto(196.47837449,380.60531497)
\curveto(196.37837072,380.58530823)(196.28337082,380.56030825)(196.19337449,380.53031497)
\curveto(196.11337099,380.5103083)(196.03337107,380.48030833)(195.95337449,380.44031497)
\curveto(195.88337122,380.42030839)(195.81837128,380.39030842)(195.75837449,380.35031497)
\curveto(195.6983714,380.3103085)(195.64337146,380.27030854)(195.59337449,380.23031497)
\curveto(195.35337175,380.06030875)(195.15837194,379.85530896)(195.00837449,379.61531497)
\curveto(194.85837224,379.37530944)(194.72837237,379.09530972)(194.61837449,378.77531497)
\curveto(194.58837251,378.67531014)(194.56837253,378.57031024)(194.55837449,378.46031497)
\curveto(194.54837255,378.36031045)(194.53337257,378.25531056)(194.51337449,378.14531497)
\curveto(194.5033726,378.10531071)(194.4983726,378.04031077)(194.49837449,377.95031497)
\curveto(194.48837261,377.92031089)(194.48337262,377.88531093)(194.48337449,377.84531497)
\curveto(194.49337261,377.80531101)(194.4983726,377.76031105)(194.49837449,377.71031497)
\lineto(194.49837449,377.41031497)
\curveto(194.4983726,377.3103115)(194.50837259,377.22031159)(194.52837449,377.14031497)
\lineto(194.55837449,376.96031497)
\curveto(194.57837252,376.86031195)(194.59337251,376.76031205)(194.60337449,376.66031497)
\curveto(194.62337248,376.57031224)(194.65337245,376.48531233)(194.69337449,376.40531497)
\curveto(194.79337231,376.16531265)(194.90837219,375.94031287)(195.03837449,375.73031497)
\curveto(195.17837192,375.52031329)(195.34837175,375.34531347)(195.54837449,375.20531497)
\curveto(195.5983715,375.17531364)(195.64337146,375.15031366)(195.68337449,375.13031497)
\curveto(195.72337138,375.1103137)(195.76837133,375.08531373)(195.81837449,375.05531497)
\curveto(195.8983712,375.00531381)(195.98337112,374.96031385)(196.07337449,374.92031497)
\curveto(196.17337093,374.89031392)(196.27837082,374.86031395)(196.38837449,374.83031497)
\curveto(196.43837066,374.810314)(196.48337062,374.80031401)(196.52337449,374.80031497)
\curveto(196.57337053,374.810314)(196.62337048,374.810314)(196.67337449,374.80031497)
\curveto(196.7033704,374.79031402)(196.76337034,374.78031403)(196.85337449,374.77031497)
\curveto(196.95337015,374.76031405)(197.02837007,374.76531405)(197.07837449,374.78531497)
\curveto(197.11836998,374.79531402)(197.15836994,374.79531402)(197.19837449,374.78531497)
\curveto(197.23836986,374.78531403)(197.27836982,374.79531402)(197.31837449,374.81531497)
\curveto(197.3983697,374.83531398)(197.47836962,374.85031396)(197.55837449,374.86031497)
\curveto(197.63836946,374.88031393)(197.71336939,374.90531391)(197.78337449,374.93531497)
\curveto(198.12336898,375.07531374)(198.3983687,375.27031354)(198.60837449,375.52031497)
\curveto(198.81836828,375.77031304)(198.99336811,376.06531275)(199.13337449,376.40531497)
\curveto(199.18336792,376.52531229)(199.21336789,376.65031216)(199.22337449,376.78031497)
\curveto(199.24336786,376.92031189)(199.27336783,377.06031175)(199.31337449,377.20031497)
}
}
{
\newrgbcolor{curcolor}{0 0 0}
\pscustom[linestyle=none,fillstyle=solid,fillcolor=curcolor]
{
\newpath
\moveto(204.56665574,381.76031497)
\curveto(205.28665168,381.77030704)(205.89165107,381.68530713)(206.38165574,381.50531497)
\curveto(206.87165009,381.33530748)(207.25164971,381.03030778)(207.52165574,380.59031497)
\curveto(207.59164937,380.48030833)(207.64664932,380.36530845)(207.68665574,380.24531497)
\curveto(207.72664924,380.13530868)(207.7666492,380.0103088)(207.80665574,379.87031497)
\curveto(207.82664914,379.80030901)(207.83164913,379.72530909)(207.82165574,379.64531497)
\curveto(207.81164915,379.57530924)(207.79664917,379.52030929)(207.77665574,379.48031497)
\curveto(207.75664921,379.46030935)(207.73164923,379.44030937)(207.70165574,379.42031497)
\curveto(207.67164929,379.4103094)(207.64664932,379.39530942)(207.62665574,379.37531497)
\curveto(207.57664939,379.35530946)(207.52664944,379.35030946)(207.47665574,379.36031497)
\curveto(207.42664954,379.37030944)(207.37664959,379.37030944)(207.32665574,379.36031497)
\curveto(207.24664972,379.34030947)(207.14164982,379.33530948)(207.01165574,379.34531497)
\curveto(206.88165008,379.36530945)(206.79165017,379.39030942)(206.74165574,379.42031497)
\curveto(206.6616503,379.47030934)(206.60665036,379.53530928)(206.57665574,379.61531497)
\curveto(206.55665041,379.70530911)(206.52165044,379.79030902)(206.47165574,379.87031497)
\curveto(206.38165058,380.03030878)(206.25665071,380.17530864)(206.09665574,380.30531497)
\curveto(205.98665098,380.38530843)(205.8666511,380.44530837)(205.73665574,380.48531497)
\curveto(205.60665136,380.52530829)(205.4666515,380.56530825)(205.31665574,380.60531497)
\curveto(205.2666517,380.62530819)(205.21665175,380.63030818)(205.16665574,380.62031497)
\curveto(205.11665185,380.62030819)(205.0666519,380.62530819)(205.01665574,380.63531497)
\curveto(204.95665201,380.65530816)(204.88165208,380.66530815)(204.79165574,380.66531497)
\curveto(204.70165226,380.66530815)(204.62665234,380.65530816)(204.56665574,380.63531497)
\lineto(204.47665574,380.63531497)
\lineto(204.32665574,380.60531497)
\curveto(204.27665269,380.60530821)(204.22665274,380.60030821)(204.17665574,380.59031497)
\curveto(203.91665305,380.53030828)(203.70165326,380.44530837)(203.53165574,380.33531497)
\curveto(203.3616536,380.22530859)(203.24665372,380.04030877)(203.18665574,379.78031497)
\curveto(203.1666538,379.7103091)(203.1616538,379.64030917)(203.17165574,379.57031497)
\curveto(203.19165377,379.50030931)(203.21165375,379.44030937)(203.23165574,379.39031497)
\curveto(203.29165367,379.24030957)(203.3616536,379.13030968)(203.44165574,379.06031497)
\curveto(203.53165343,379.00030981)(203.64165332,378.93030988)(203.77165574,378.85031497)
\curveto(203.93165303,378.75031006)(204.11165285,378.67531014)(204.31165574,378.62531497)
\curveto(204.51165245,378.58531023)(204.71165225,378.53531028)(204.91165574,378.47531497)
\curveto(205.04165192,378.43531038)(205.17165179,378.40531041)(205.30165574,378.38531497)
\curveto(205.43165153,378.36531045)(205.5616514,378.33531048)(205.69165574,378.29531497)
\curveto(205.90165106,378.23531058)(206.10665086,378.17531064)(206.30665574,378.11531497)
\curveto(206.50665046,378.06531075)(206.70665026,378.00031081)(206.90665574,377.92031497)
\lineto(207.05665574,377.86031497)
\curveto(207.10664986,377.84031097)(207.15664981,377.815311)(207.20665574,377.78531497)
\curveto(207.40664956,377.66531115)(207.58164938,377.53031128)(207.73165574,377.38031497)
\curveto(207.88164908,377.23031158)(208.00664896,377.04031177)(208.10665574,376.81031497)
\curveto(208.12664884,376.74031207)(208.14664882,376.64531217)(208.16665574,376.52531497)
\curveto(208.18664878,376.45531236)(208.19664877,376.38031243)(208.19665574,376.30031497)
\curveto(208.20664876,376.23031258)(208.21164875,376.15031266)(208.21165574,376.06031497)
\lineto(208.21165574,375.91031497)
\curveto(208.19164877,375.84031297)(208.18164878,375.77031304)(208.18165574,375.70031497)
\curveto(208.18164878,375.63031318)(208.17164879,375.56031325)(208.15165574,375.49031497)
\curveto(208.12164884,375.38031343)(208.08664888,375.27531354)(208.04665574,375.17531497)
\curveto(208.00664896,375.07531374)(207.961649,374.98531383)(207.91165574,374.90531497)
\curveto(207.75164921,374.64531417)(207.54664942,374.43531438)(207.29665574,374.27531497)
\curveto(207.04664992,374.12531469)(206.7666502,373.99531482)(206.45665574,373.88531497)
\curveto(206.3666506,373.85531496)(206.27165069,373.83531498)(206.17165574,373.82531497)
\curveto(206.08165088,373.80531501)(205.99165097,373.78031503)(205.90165574,373.75031497)
\curveto(205.80165116,373.73031508)(205.70165126,373.72031509)(205.60165574,373.72031497)
\curveto(205.50165146,373.72031509)(205.40165156,373.7103151)(205.30165574,373.69031497)
\lineto(205.15165574,373.69031497)
\curveto(205.10165186,373.68031513)(205.03165193,373.67531514)(204.94165574,373.67531497)
\curveto(204.85165211,373.67531514)(204.78165218,373.68031513)(204.73165574,373.69031497)
\lineto(204.56665574,373.69031497)
\curveto(204.50665246,373.7103151)(204.44165252,373.72031509)(204.37165574,373.72031497)
\curveto(204.30165266,373.7103151)(204.24165272,373.7153151)(204.19165574,373.73531497)
\curveto(204.14165282,373.74531507)(204.07665289,373.75031506)(203.99665574,373.75031497)
\lineto(203.75665574,373.81031497)
\curveto(203.68665328,373.82031499)(203.61165335,373.84031497)(203.53165574,373.87031497)
\curveto(203.22165374,373.97031484)(202.95165401,374.09531472)(202.72165574,374.24531497)
\curveto(202.49165447,374.39531442)(202.29165467,374.59031422)(202.12165574,374.83031497)
\curveto(202.03165493,374.96031385)(201.95665501,375.09531372)(201.89665574,375.23531497)
\curveto(201.83665513,375.37531344)(201.78165518,375.53031328)(201.73165574,375.70031497)
\curveto(201.71165525,375.76031305)(201.70165526,375.83031298)(201.70165574,375.91031497)
\curveto(201.71165525,376.00031281)(201.72665524,376.07031274)(201.74665574,376.12031497)
\curveto(201.77665519,376.16031265)(201.82665514,376.20031261)(201.89665574,376.24031497)
\curveto(201.94665502,376.26031255)(202.01665495,376.27031254)(202.10665574,376.27031497)
\curveto(202.19665477,376.28031253)(202.28665468,376.28031253)(202.37665574,376.27031497)
\curveto(202.4666545,376.26031255)(202.55165441,376.24531257)(202.63165574,376.22531497)
\curveto(202.72165424,376.2153126)(202.78165418,376.20031261)(202.81165574,376.18031497)
\curveto(202.88165408,376.13031268)(202.92665404,376.05531276)(202.94665574,375.95531497)
\curveto(202.97665399,375.86531295)(203.01165395,375.78031303)(203.05165574,375.70031497)
\curveto(203.15165381,375.48031333)(203.28665368,375.3103135)(203.45665574,375.19031497)
\curveto(203.57665339,375.10031371)(203.71165325,375.03031378)(203.86165574,374.98031497)
\curveto(204.01165295,374.93031388)(204.17165279,374.88031393)(204.34165574,374.83031497)
\lineto(204.65665574,374.78531497)
\lineto(204.74665574,374.78531497)
\curveto(204.81665215,374.76531405)(204.90665206,374.75531406)(205.01665574,374.75531497)
\curveto(205.13665183,374.75531406)(205.23665173,374.76531405)(205.31665574,374.78531497)
\curveto(205.38665158,374.78531403)(205.44165152,374.79031402)(205.48165574,374.80031497)
\curveto(205.54165142,374.810314)(205.60165136,374.815314)(205.66165574,374.81531497)
\curveto(205.72165124,374.82531399)(205.77665119,374.83531398)(205.82665574,374.84531497)
\curveto(206.11665085,374.92531389)(206.34665062,375.03031378)(206.51665574,375.16031497)
\curveto(206.68665028,375.29031352)(206.80665016,375.5103133)(206.87665574,375.82031497)
\curveto(206.89665007,375.87031294)(206.90165006,375.92531289)(206.89165574,375.98531497)
\curveto(206.88165008,376.04531277)(206.87165009,376.09031272)(206.86165574,376.12031497)
\curveto(206.81165015,376.3103125)(206.74165022,376.45031236)(206.65165574,376.54031497)
\curveto(206.5616504,376.64031217)(206.44665052,376.73031208)(206.30665574,376.81031497)
\curveto(206.21665075,376.87031194)(206.11665085,376.92031189)(206.00665574,376.96031497)
\lineto(205.67665574,377.08031497)
\curveto(205.64665132,377.09031172)(205.61665135,377.09531172)(205.58665574,377.09531497)
\curveto(205.5666514,377.09531172)(205.54165142,377.10531171)(205.51165574,377.12531497)
\curveto(205.17165179,377.23531158)(204.81665215,377.3153115)(204.44665574,377.36531497)
\curveto(204.08665288,377.42531139)(203.74665322,377.52031129)(203.42665574,377.65031497)
\curveto(203.32665364,377.69031112)(203.23165373,377.72531109)(203.14165574,377.75531497)
\curveto(203.05165391,377.78531103)(202.966654,377.82531099)(202.88665574,377.87531497)
\curveto(202.69665427,377.98531083)(202.52165444,378.1103107)(202.36165574,378.25031497)
\curveto(202.20165476,378.39031042)(202.07665489,378.56531025)(201.98665574,378.77531497)
\curveto(201.95665501,378.84530997)(201.93165503,378.9153099)(201.91165574,378.98531497)
\curveto(201.90165506,379.05530976)(201.88665508,379.13030968)(201.86665574,379.21031497)
\curveto(201.83665513,379.33030948)(201.82665514,379.46530935)(201.83665574,379.61531497)
\curveto(201.84665512,379.77530904)(201.8616551,379.9103089)(201.88165574,380.02031497)
\curveto(201.90165506,380.07030874)(201.91165505,380.1103087)(201.91165574,380.14031497)
\curveto(201.92165504,380.18030863)(201.93665503,380.22030859)(201.95665574,380.26031497)
\curveto(202.04665492,380.49030832)(202.1666548,380.69030812)(202.31665574,380.86031497)
\curveto(202.47665449,381.03030778)(202.65665431,381.18030763)(202.85665574,381.31031497)
\curveto(203.00665396,381.40030741)(203.17165379,381.47030734)(203.35165574,381.52031497)
\curveto(203.53165343,381.58030723)(203.72165324,381.63530718)(203.92165574,381.68531497)
\curveto(203.99165297,381.69530712)(204.05665291,381.70530711)(204.11665574,381.71531497)
\curveto(204.18665278,381.72530709)(204.2616527,381.73530708)(204.34165574,381.74531497)
\curveto(204.37165259,381.75530706)(204.41165255,381.75530706)(204.46165574,381.74531497)
\curveto(204.51165245,381.73530708)(204.54665242,381.74030707)(204.56665574,381.76031497)
}
}
{
\newrgbcolor{curcolor}{0.80000001 0.80000001 0.80000001}
\pscustom[linestyle=none,fillstyle=solid,fillcolor=curcolor]
{
\newpath
\moveto(120.84556474,387.06538211)
\lineto(135.84556474,387.06538211)
\lineto(135.84556474,372.06538211)
\lineto(120.84556474,372.06538211)
\closepath
}
}
{
\newrgbcolor{curcolor}{0 0 0}
\pscustom[linestyle=none,fillstyle=solid,fillcolor=curcolor]
{
\newpath
\moveto(148.82876512,351.37313724)
\curveto(148.8787555,351.24313698)(148.85875552,351.14313708)(148.76876512,351.07313724)
\curveto(148.71875566,351.04313718)(148.65375572,351.0231372)(148.57376512,351.01313724)
\lineto(148.34876512,351.01313724)
\lineto(147.86876512,351.01313724)
\curveto(147.70875667,351.01313721)(147.58375679,351.04813717)(147.49376512,351.11813724)
\curveto(147.41375696,351.16813705)(147.35875702,351.24313698)(147.32876512,351.34313724)
\lineto(147.26876512,351.67313724)
\curveto(147.25875712,351.71313651)(147.25375712,351.74813647)(147.25376512,351.77813724)
\lineto(147.25376512,351.88313724)
\curveto(147.23375714,351.93313629)(147.22875715,351.97813624)(147.23876512,352.01813724)
\curveto(147.24875713,352.05813616)(147.24875713,352.09813612)(147.23876512,352.13813724)
\curveto(147.22875715,352.19813602)(147.22375715,352.25813596)(147.22376512,352.31813724)
\lineto(147.22376512,352.49813724)
\lineto(147.17876512,353.17313724)
\curveto(147.15875722,353.24313498)(147.14875723,353.31313491)(147.14876512,353.38313724)
\curveto(147.14875723,353.45313477)(147.13875724,353.52813469)(147.11876512,353.60813724)
\curveto(147.06875731,353.78813443)(147.02875735,353.96813425)(146.99876512,354.14813724)
\curveto(146.9787574,354.32813389)(146.93375744,354.49813372)(146.86376512,354.65813724)
\curveto(146.6737577,355.07813314)(146.35875802,355.35813286)(145.91876512,355.49813724)
\curveto(145.78875859,355.54813267)(145.64375873,355.57313265)(145.48376512,355.57313724)
\curveto(145.33375904,355.58313264)(145.1737592,355.58813263)(145.00376512,355.58813724)
\lineto(142.24376512,355.58813724)
\curveto(142.1737622,355.56813265)(142.10876227,355.54813267)(142.04876512,355.52813724)
\curveto(141.99876238,355.5181327)(141.95376242,355.48813273)(141.91376512,355.43813724)
\curveto(141.84376253,355.33813288)(141.80876257,355.17313305)(141.80876512,354.94313724)
\curveto(141.81876256,354.7231335)(141.82376255,354.52813369)(141.82376512,354.35813724)
\lineto(141.82376512,352.18313724)
\curveto(141.82376255,352.04313618)(141.82876255,351.86813635)(141.83876512,351.65813724)
\curveto(141.84876253,351.45813676)(141.82876255,351.30813691)(141.77876512,351.20813724)
\curveto(141.75876262,351.13813708)(141.71876266,351.09313713)(141.65876512,351.07313724)
\curveto(141.61876276,351.05313717)(141.5787628,351.04313718)(141.53876512,351.04313724)
\curveto(141.50876287,351.04313718)(141.46876291,351.03313719)(141.41876512,351.01313724)
\curveto(141.378763,351.00313722)(141.33376304,350.99813722)(141.28376512,350.99813724)
\curveto(141.23376314,351.00813721)(141.18376319,351.01313721)(141.13376512,351.01313724)
\lineto(140.80376512,351.01313724)
\curveto(140.70376367,351.0231372)(140.61876376,351.05313717)(140.54876512,351.10313724)
\curveto(140.46876391,351.15313707)(140.42876395,351.24313698)(140.42876512,351.37313724)
\lineto(140.42876512,351.77813724)
\lineto(140.42876512,360.89813724)
\curveto(140.42876395,361.00812721)(140.42376395,361.1231271)(140.41376512,361.24313724)
\curveto(140.41376396,361.36312686)(140.43876394,361.45812676)(140.48876512,361.52813724)
\curveto(140.52876385,361.58812663)(140.60376377,361.63812658)(140.71376512,361.67813724)
\curveto(140.73376364,361.68812653)(140.75376362,361.68812653)(140.77376512,361.67813724)
\curveto(140.79376358,361.67812654)(140.81376356,361.68312654)(140.83376512,361.69313724)
\lineto(145.18376512,361.69313724)
\curveto(145.25375912,361.69312653)(145.32875905,361.69312653)(145.40876512,361.69313724)
\curveto(145.48875889,361.70312652)(145.55875882,361.70312652)(145.61876512,361.69313724)
\lineto(145.78376512,361.69313724)
\curveto(145.84375853,361.68312654)(145.90375847,361.67312655)(145.96376512,361.66313724)
\curveto(146.02375835,361.66312656)(146.08875829,361.65812656)(146.15876512,361.64813724)
\curveto(146.23875814,361.62812659)(146.31875806,361.61312661)(146.39876512,361.60313724)
\curveto(146.48875789,361.59312663)(146.5737578,361.57812664)(146.65376512,361.55813724)
\curveto(146.84375753,361.49812672)(147.01875736,361.43312679)(147.17876512,361.36313724)
\curveto(147.33875704,361.29312693)(147.48875689,361.20812701)(147.62876512,361.10813724)
\curveto(147.8787565,360.93812728)(148.0787563,360.72812749)(148.22876512,360.47813724)
\curveto(148.38875599,360.23812798)(148.51875586,359.95312827)(148.61876512,359.62313724)
\curveto(148.63875574,359.54312868)(148.64875573,359.45812876)(148.64876512,359.36813724)
\curveto(148.65875572,359.28812893)(148.6737557,359.20812901)(148.69376512,359.12813724)
\lineto(148.69376512,358.97813724)
\curveto(148.70375567,358.92812929)(148.70375567,358.86812935)(148.69376512,358.79813724)
\curveto(148.69375568,358.73812948)(148.68875569,358.68312954)(148.67876512,358.63313724)
\lineto(148.67876512,358.46813724)
\curveto(148.65875572,358.38812983)(148.64375573,358.31312991)(148.63376512,358.24313724)
\curveto(148.63375574,358.17313005)(148.62375575,358.10313012)(148.60376512,358.03313724)
\curveto(148.55375582,357.88313034)(148.50375587,357.73813048)(148.45376512,357.59813724)
\curveto(148.41375596,357.46813075)(148.35375602,357.34313088)(148.27376512,357.22313724)
\curveto(148.24375613,357.17313105)(148.20875617,357.12813109)(148.16876512,357.08813724)
\curveto(148.13875624,357.04813117)(148.10875627,357.00313122)(148.07876512,356.95313724)
\lineto(148.04876512,356.92313724)
\curveto(148.03875634,356.9231313)(148.02875635,356.9181313)(148.01876512,356.90813724)
\lineto(147.94376512,356.83313724)
\curveto(147.92375645,356.80313142)(147.90375647,356.77813144)(147.88376512,356.75813724)
\curveto(147.80375657,356.69813152)(147.72875665,356.63813158)(147.65876512,356.57813724)
\curveto(147.58875679,356.52813169)(147.51375686,356.47813174)(147.43376512,356.42813724)
\curveto(147.38375699,356.39813182)(147.33875704,356.36313186)(147.29876512,356.32313724)
\curveto(147.25875712,356.29313193)(147.23375714,356.24813197)(147.22376512,356.18813724)
\curveto(147.21375716,356.12813209)(147.23375714,356.07813214)(147.28376512,356.03813724)
\curveto(147.34375703,355.99813222)(147.39375698,355.96813225)(147.43376512,355.94813724)
\curveto(147.54375683,355.87813234)(147.64375673,355.80313242)(147.73376512,355.72313724)
\curveto(147.83375654,355.64313258)(147.91875646,355.54813267)(147.98876512,355.43813724)
\curveto(148.09875628,355.29813292)(148.1787562,355.13813308)(148.22876512,354.95813724)
\curveto(148.2787561,354.78813343)(148.32875605,354.60313362)(148.37876512,354.40313724)
\lineto(148.40876512,354.16313724)
\curveto(148.41875596,354.09313413)(148.42875595,354.0181342)(148.43876512,353.93813724)
\curveto(148.45875592,353.86813435)(148.46375591,353.79813442)(148.45376512,353.72813724)
\curveto(148.44375593,353.65813456)(148.44875593,353.58813463)(148.46876512,353.51813724)
\lineto(148.46876512,353.38313724)
\curveto(148.48875589,353.31313491)(148.49375588,353.23813498)(148.48376512,353.15813724)
\curveto(148.4737559,353.07813514)(148.4787559,352.99813522)(148.49876512,352.91813724)
\curveto(148.50875587,352.87813534)(148.50875587,352.83813538)(148.49876512,352.79813724)
\curveto(148.49875588,352.76813545)(148.50375587,352.72813549)(148.51376512,352.67813724)
\curveto(148.53375584,352.57813564)(148.54875583,352.47313575)(148.55876512,352.36313724)
\curveto(148.56875581,352.26313596)(148.58875579,352.16813605)(148.61876512,352.07813724)
\curveto(148.63875574,352.0181362)(148.64875573,351.95813626)(148.64876512,351.89813724)
\curveto(148.65875572,351.84813637)(148.6737557,351.79313643)(148.69376512,351.73313724)
\lineto(148.82876512,351.37313724)
\moveto(147.01376512,357.59813724)
\curveto(147.08375729,357.70813051)(147.13375724,357.8231304)(147.16376512,357.94313724)
\curveto(147.20375717,358.06313016)(147.23875714,358.19313003)(147.26876512,358.33313724)
\lineto(147.26876512,358.46813724)
\curveto(147.29875708,358.60812961)(147.30375707,358.75812946)(147.28376512,358.91813724)
\curveto(147.26375711,359.08812913)(147.23375714,359.22812899)(147.19376512,359.33813724)
\curveto(147.03375734,359.83812838)(146.71875766,360.18312804)(146.24876512,360.37313724)
\curveto(146.04875833,360.45312777)(145.81375856,360.49812772)(145.54376512,360.50813724)
\curveto(145.28375909,360.5181277)(145.01375936,360.5231277)(144.73376512,360.52313724)
\lineto(142.25876512,360.52313724)
\curveto(142.23876214,360.51312771)(142.21376216,360.50812771)(142.18376512,360.50813724)
\curveto(142.16376221,360.50812771)(142.13876224,360.50312772)(142.10876512,360.49313724)
\curveto(141.98876239,360.46312776)(141.90876247,360.39812782)(141.86876512,360.29813724)
\curveto(141.82876255,360.20812801)(141.80876257,360.08312814)(141.80876512,359.92313724)
\curveto(141.81876256,359.76312846)(141.82376255,359.6181286)(141.82376512,359.48813724)
\lineto(141.82376512,357.76313724)
\curveto(141.82376255,357.61313061)(141.81876256,357.45313077)(141.80876512,357.28313724)
\curveto(141.80876257,357.1231311)(141.84376253,356.99813122)(141.91376512,356.90813724)
\curveto(141.96376241,356.83813138)(142.03876234,356.79313143)(142.13876512,356.77313724)
\curveto(142.23876214,356.76313146)(142.34876203,356.75813146)(142.46876512,356.75813724)
\lineto(143.39876512,356.75813724)
\curveto(143.78876059,356.75813146)(144.16876021,356.75313147)(144.53876512,356.74313724)
\curveto(144.90875947,356.74313148)(145.24875913,356.76313146)(145.55876512,356.80313724)
\curveto(145.8787585,356.85313137)(146.16375821,356.93813128)(146.41376512,357.05813724)
\curveto(146.66375771,357.17813104)(146.86375751,357.35813086)(147.01376512,357.59813724)
}
}
{
\newrgbcolor{curcolor}{0 0 0}
\pscustom[linestyle=none,fillstyle=solid,fillcolor=curcolor]
{
\newpath
\moveto(157.20696824,355.16813724)
\curveto(157.22696056,355.06813315)(157.22696056,354.95313327)(157.20696824,354.82313724)
\curveto(157.19696059,354.70313352)(157.16696062,354.6181336)(157.11696824,354.56813724)
\curveto(157.06696072,354.52813369)(156.99196079,354.49813372)(156.89196824,354.47813724)
\curveto(156.80196098,354.46813375)(156.69696109,354.46313376)(156.57696824,354.46313724)
\lineto(156.21696824,354.46313724)
\curveto(156.09696169,354.47313375)(155.99196179,354.47813374)(155.90196824,354.47813724)
\lineto(152.06196824,354.47813724)
\curveto(151.9819658,354.47813374)(151.90196588,354.47313375)(151.82196824,354.46313724)
\curveto(151.74196604,354.46313376)(151.67696611,354.44813377)(151.62696824,354.41813724)
\curveto(151.5869662,354.39813382)(151.54696624,354.35813386)(151.50696824,354.29813724)
\curveto(151.4869663,354.26813395)(151.46696632,354.223134)(151.44696824,354.16313724)
\curveto(151.42696636,354.11313411)(151.42696636,354.06313416)(151.44696824,354.01313724)
\curveto(151.45696633,353.96313426)(151.46196632,353.9181343)(151.46196824,353.87813724)
\curveto(151.46196632,353.83813438)(151.46696632,353.79813442)(151.47696824,353.75813724)
\curveto(151.49696629,353.67813454)(151.51696627,353.59313463)(151.53696824,353.50313724)
\curveto(151.55696623,353.4231348)(151.5869662,353.34313488)(151.62696824,353.26313724)
\curveto(151.85696593,352.7231355)(152.23696555,352.33813588)(152.76696824,352.10813724)
\curveto(152.82696496,352.07813614)(152.89196489,352.05313617)(152.96196824,352.03313724)
\lineto(153.17196824,351.97313724)
\curveto(153.20196458,351.96313626)(153.25196453,351.95813626)(153.32196824,351.95813724)
\curveto(153.46196432,351.9181363)(153.64696414,351.89813632)(153.87696824,351.89813724)
\curveto(154.10696368,351.89813632)(154.29196349,351.9181363)(154.43196824,351.95813724)
\curveto(154.57196321,351.99813622)(154.69696309,352.03813618)(154.80696824,352.07813724)
\curveto(154.92696286,352.12813609)(155.03696275,352.18813603)(155.13696824,352.25813724)
\curveto(155.24696254,352.32813589)(155.34196244,352.40813581)(155.42196824,352.49813724)
\curveto(155.50196228,352.59813562)(155.57196221,352.70313552)(155.63196824,352.81313724)
\curveto(155.69196209,352.91313531)(155.74196204,353.0181352)(155.78196824,353.12813724)
\curveto(155.83196195,353.23813498)(155.91196187,353.3181349)(156.02196824,353.36813724)
\curveto(156.06196172,353.38813483)(156.12696166,353.40313482)(156.21696824,353.41313724)
\curveto(156.30696148,353.4231348)(156.39696139,353.4231348)(156.48696824,353.41313724)
\curveto(156.57696121,353.41313481)(156.66196112,353.40813481)(156.74196824,353.39813724)
\curveto(156.82196096,353.38813483)(156.87696091,353.36813485)(156.90696824,353.33813724)
\curveto(157.00696078,353.26813495)(157.03196075,353.15313507)(156.98196824,352.99313724)
\curveto(156.90196088,352.7231355)(156.79696099,352.48313574)(156.66696824,352.27313724)
\curveto(156.46696132,351.95313627)(156.23696155,351.68813653)(155.97696824,351.47813724)
\curveto(155.72696206,351.27813694)(155.40696238,351.11313711)(155.01696824,350.98313724)
\curveto(154.91696287,350.94313728)(154.81696297,350.9181373)(154.71696824,350.90813724)
\curveto(154.61696317,350.88813733)(154.51196327,350.86813735)(154.40196824,350.84813724)
\curveto(154.35196343,350.83813738)(154.30196348,350.83313739)(154.25196824,350.83313724)
\curveto(154.21196357,350.83313739)(154.16696362,350.82813739)(154.11696824,350.81813724)
\lineto(153.96696824,350.81813724)
\curveto(153.91696387,350.80813741)(153.85696393,350.80313742)(153.78696824,350.80313724)
\curveto(153.72696406,350.80313742)(153.67696411,350.80813741)(153.63696824,350.81813724)
\lineto(153.50196824,350.81813724)
\curveto(153.45196433,350.82813739)(153.40696438,350.83313739)(153.36696824,350.83313724)
\curveto(153.32696446,350.83313739)(153.2869645,350.83813738)(153.24696824,350.84813724)
\curveto(153.19696459,350.85813736)(153.14196464,350.86813735)(153.08196824,350.87813724)
\curveto(153.02196476,350.87813734)(152.96696482,350.88313734)(152.91696824,350.89313724)
\curveto(152.82696496,350.91313731)(152.73696505,350.93813728)(152.64696824,350.96813724)
\curveto(152.55696523,350.98813723)(152.47196531,351.01313721)(152.39196824,351.04313724)
\curveto(152.35196543,351.06313716)(152.31696547,351.07313715)(152.28696824,351.07313724)
\curveto(152.25696553,351.08313714)(152.22196556,351.09813712)(152.18196824,351.11813724)
\curveto(152.03196575,351.18813703)(151.87196591,351.27313695)(151.70196824,351.37313724)
\curveto(151.41196637,351.56313666)(151.16196662,351.79313643)(150.95196824,352.06313724)
\curveto(150.75196703,352.34313588)(150.5819672,352.65313557)(150.44196824,352.99313724)
\curveto(150.39196739,353.10313512)(150.35196743,353.218135)(150.32196824,353.33813724)
\curveto(150.30196748,353.45813476)(150.27196751,353.57813464)(150.23196824,353.69813724)
\curveto(150.22196756,353.73813448)(150.21696757,353.77313445)(150.21696824,353.80313724)
\curveto(150.21696757,353.83313439)(150.21196757,353.87313435)(150.20196824,353.92313724)
\curveto(150.1819676,354.00313422)(150.16696762,354.08813413)(150.15696824,354.17813724)
\curveto(150.14696764,354.26813395)(150.13196765,354.35813386)(150.11196824,354.44813724)
\lineto(150.11196824,354.65813724)
\curveto(150.10196768,354.69813352)(150.09196769,354.75313347)(150.08196824,354.82313724)
\curveto(150.0819677,354.90313332)(150.0869677,354.96813325)(150.09696824,355.01813724)
\lineto(150.09696824,355.18313724)
\curveto(150.11696767,355.23313299)(150.12196766,355.28313294)(150.11196824,355.33313724)
\curveto(150.11196767,355.39313283)(150.11696767,355.44813277)(150.12696824,355.49813724)
\curveto(150.16696762,355.65813256)(150.19696759,355.8181324)(150.21696824,355.97813724)
\curveto(150.24696754,356.13813208)(150.29196749,356.28813193)(150.35196824,356.42813724)
\curveto(150.40196738,356.53813168)(150.44696734,356.64813157)(150.48696824,356.75813724)
\curveto(150.53696725,356.87813134)(150.59196719,356.99313123)(150.65196824,357.10313724)
\curveto(150.87196691,357.45313077)(151.12196666,357.75313047)(151.40196824,358.00313724)
\curveto(151.6819661,358.26312996)(152.02696576,358.47812974)(152.43696824,358.64813724)
\curveto(152.55696523,358.69812952)(152.67696511,358.73312949)(152.79696824,358.75313724)
\curveto(152.92696486,358.78312944)(153.06196472,358.81312941)(153.20196824,358.84313724)
\curveto(153.25196453,358.85312937)(153.29696449,358.85812936)(153.33696824,358.85813724)
\curveto(153.37696441,358.86812935)(153.42196436,358.87312935)(153.47196824,358.87313724)
\curveto(153.49196429,358.88312934)(153.51696427,358.88312934)(153.54696824,358.87313724)
\curveto(153.57696421,358.86312936)(153.60196418,358.86812935)(153.62196824,358.88813724)
\curveto(154.04196374,358.89812932)(154.40696338,358.85312937)(154.71696824,358.75313724)
\curveto(155.02696276,358.66312956)(155.30696248,358.53812968)(155.55696824,358.37813724)
\curveto(155.60696218,358.35812986)(155.64696214,358.32812989)(155.67696824,358.28813724)
\curveto(155.70696208,358.25812996)(155.74196204,358.23312999)(155.78196824,358.21313724)
\curveto(155.86196192,358.15313007)(155.94196184,358.08313014)(156.02196824,358.00313724)
\curveto(156.11196167,357.9231303)(156.1869616,357.84313038)(156.24696824,357.76313724)
\curveto(156.40696138,357.55313067)(156.54196124,357.35313087)(156.65196824,357.16313724)
\curveto(156.72196106,357.05313117)(156.77696101,356.93313129)(156.81696824,356.80313724)
\curveto(156.85696093,356.67313155)(156.90196088,356.54313168)(156.95196824,356.41313724)
\curveto(157.00196078,356.28313194)(157.03696075,356.14813207)(157.05696824,356.00813724)
\curveto(157.0869607,355.86813235)(157.12196066,355.72813249)(157.16196824,355.58813724)
\curveto(157.17196061,355.5181327)(157.17696061,355.44813277)(157.17696824,355.37813724)
\lineto(157.20696824,355.16813724)
\moveto(155.75196824,355.67813724)
\curveto(155.781962,355.7181325)(155.80696198,355.76813245)(155.82696824,355.82813724)
\curveto(155.84696194,355.89813232)(155.84696194,355.96813225)(155.82696824,356.03813724)
\curveto(155.76696202,356.25813196)(155.6819621,356.46313176)(155.57196824,356.65313724)
\curveto(155.43196235,356.88313134)(155.27696251,357.07813114)(155.10696824,357.23813724)
\curveto(154.93696285,357.39813082)(154.71696307,357.53313069)(154.44696824,357.64313724)
\curveto(154.37696341,357.66313056)(154.30696348,357.67813054)(154.23696824,357.68813724)
\curveto(154.16696362,357.70813051)(154.09196369,357.72813049)(154.01196824,357.74813724)
\curveto(153.93196385,357.76813045)(153.84696394,357.77813044)(153.75696824,357.77813724)
\lineto(153.50196824,357.77813724)
\curveto(153.47196431,357.75813046)(153.43696435,357.74813047)(153.39696824,357.74813724)
\curveto(153.35696443,357.75813046)(153.32196446,357.75813046)(153.29196824,357.74813724)
\lineto(153.05196824,357.68813724)
\curveto(152.9819648,357.67813054)(152.91196487,357.66313056)(152.84196824,357.64313724)
\curveto(152.55196523,357.5231307)(152.31696547,357.37313085)(152.13696824,357.19313724)
\curveto(151.96696582,357.01313121)(151.81196597,356.78813143)(151.67196824,356.51813724)
\curveto(151.64196614,356.46813175)(151.61196617,356.40313182)(151.58196824,356.32313724)
\curveto(151.55196623,356.25313197)(151.52696626,356.17313205)(151.50696824,356.08313724)
\curveto(151.4869663,355.99313223)(151.4819663,355.90813231)(151.49196824,355.82813724)
\curveto(151.50196628,355.74813247)(151.53696625,355.68813253)(151.59696824,355.64813724)
\curveto(151.67696611,355.58813263)(151.81196597,355.55813266)(152.00196824,355.55813724)
\curveto(152.20196558,355.56813265)(152.37196541,355.57313265)(152.51196824,355.57313724)
\lineto(154.79196824,355.57313724)
\curveto(154.94196284,355.57313265)(155.12196266,355.56813265)(155.33196824,355.55813724)
\curveto(155.54196224,355.55813266)(155.6819621,355.59813262)(155.75196824,355.67813724)
}
}
{
\newrgbcolor{curcolor}{0 0 0}
\pscustom[linestyle=none,fillstyle=solid,fillcolor=curcolor]
{
\newpath
\moveto(161.64860887,358.90313724)
\curveto(162.38860408,358.91312931)(163.00360346,358.80312942)(163.49360887,358.57313724)
\curveto(163.99360247,358.35312987)(164.38860208,358.0181302)(164.67860887,357.56813724)
\curveto(164.80860166,357.36813085)(164.91860155,357.1231311)(165.00860887,356.83313724)
\curveto(165.02860144,356.78313144)(165.04360142,356.7181315)(165.05360887,356.63813724)
\curveto(165.0636014,356.55813166)(165.05860141,356.48813173)(165.03860887,356.42813724)
\curveto(165.00860146,356.37813184)(164.95860151,356.33313189)(164.88860887,356.29313724)
\curveto(164.85860161,356.27313195)(164.82860164,356.26313196)(164.79860887,356.26313724)
\curveto(164.7686017,356.27313195)(164.73360173,356.27313195)(164.69360887,356.26313724)
\curveto(164.65360181,356.25313197)(164.61360185,356.24813197)(164.57360887,356.24813724)
\curveto(164.53360193,356.25813196)(164.49360197,356.26313196)(164.45360887,356.26313724)
\lineto(164.13860887,356.26313724)
\curveto(164.03860243,356.27313195)(163.95360251,356.30313192)(163.88360887,356.35313724)
\curveto(163.80360266,356.41313181)(163.74860272,356.49813172)(163.71860887,356.60813724)
\curveto(163.68860278,356.7181315)(163.64860282,356.81313141)(163.59860887,356.89313724)
\curveto(163.44860302,357.15313107)(163.25360321,357.35813086)(163.01360887,357.50813724)
\curveto(162.93360353,357.55813066)(162.84860362,357.59813062)(162.75860887,357.62813724)
\curveto(162.6686038,357.66813055)(162.57360389,357.70313052)(162.47360887,357.73313724)
\curveto(162.33360413,357.77313045)(162.14860432,357.79313043)(161.91860887,357.79313724)
\curveto(161.68860478,357.80313042)(161.49860497,357.78313044)(161.34860887,357.73313724)
\curveto(161.27860519,357.71313051)(161.21360525,357.69813052)(161.15360887,357.68813724)
\curveto(161.09360537,357.67813054)(161.02860544,357.66313056)(160.95860887,357.64313724)
\curveto(160.69860577,357.53313069)(160.468606,357.38313084)(160.26860887,357.19313724)
\curveto(160.0686064,357.00313122)(159.91360655,356.77813144)(159.80360887,356.51813724)
\curveto(159.7636067,356.42813179)(159.72860674,356.33313189)(159.69860887,356.23313724)
\curveto(159.6686068,356.14313208)(159.63860683,356.04313218)(159.60860887,355.93313724)
\lineto(159.51860887,355.52813724)
\curveto(159.50860696,355.47813274)(159.50360696,355.4231328)(159.50360887,355.36313724)
\curveto(159.51360695,355.30313292)(159.50860696,355.24813297)(159.48860887,355.19813724)
\lineto(159.48860887,355.07813724)
\curveto(159.47860699,355.03813318)(159.468607,354.97313325)(159.45860887,354.88313724)
\curveto(159.45860701,354.79313343)(159.468607,354.72813349)(159.48860887,354.68813724)
\curveto(159.49860697,354.63813358)(159.49860697,354.58813363)(159.48860887,354.53813724)
\curveto(159.47860699,354.48813373)(159.47860699,354.43813378)(159.48860887,354.38813724)
\curveto(159.49860697,354.34813387)(159.50360696,354.27813394)(159.50360887,354.17813724)
\curveto(159.52360694,354.09813412)(159.53860693,354.01313421)(159.54860887,353.92313724)
\curveto(159.5686069,353.83313439)(159.58860688,353.74813447)(159.60860887,353.66813724)
\curveto(159.71860675,353.34813487)(159.84360662,353.06813515)(159.98360887,352.82813724)
\curveto(160.13360633,352.59813562)(160.33860613,352.39813582)(160.59860887,352.22813724)
\curveto(160.68860578,352.17813604)(160.77860569,352.13313609)(160.86860887,352.09313724)
\curveto(160.9686055,352.05313617)(161.07360539,352.01313621)(161.18360887,351.97313724)
\curveto(161.23360523,351.96313626)(161.27360519,351.95813626)(161.30360887,351.95813724)
\curveto(161.33360513,351.95813626)(161.37360509,351.95313627)(161.42360887,351.94313724)
\curveto(161.45360501,351.93313629)(161.50360496,351.92813629)(161.57360887,351.92813724)
\lineto(161.73860887,351.92813724)
\curveto(161.73860473,351.9181363)(161.75860471,351.91313631)(161.79860887,351.91313724)
\curveto(161.81860465,351.9231363)(161.84360462,351.9231363)(161.87360887,351.91313724)
\curveto(161.90360456,351.91313631)(161.93360453,351.9181363)(161.96360887,351.92813724)
\curveto(162.03360443,351.94813627)(162.09860437,351.95313627)(162.15860887,351.94313724)
\curveto(162.22860424,351.94313628)(162.29860417,351.95313627)(162.36860887,351.97313724)
\curveto(162.62860384,352.05313617)(162.85360361,352.15313607)(163.04360887,352.27313724)
\curveto(163.23360323,352.40313582)(163.39360307,352.56813565)(163.52360887,352.76813724)
\curveto(163.57360289,352.84813537)(163.61860285,352.93313529)(163.65860887,353.02313724)
\lineto(163.77860887,353.29313724)
\curveto(163.79860267,353.37313485)(163.81860265,353.44813477)(163.83860887,353.51813724)
\curveto(163.8686026,353.59813462)(163.91860255,353.66313456)(163.98860887,353.71313724)
\curveto(164.01860245,353.74313448)(164.07860239,353.76313446)(164.16860887,353.77313724)
\curveto(164.25860221,353.79313443)(164.35360211,353.80313442)(164.45360887,353.80313724)
\curveto(164.5636019,353.81313441)(164.6636018,353.81313441)(164.75360887,353.80313724)
\curveto(164.85360161,353.79313443)(164.92360154,353.77313445)(164.96360887,353.74313724)
\curveto(165.02360144,353.70313452)(165.05860141,353.64313458)(165.06860887,353.56313724)
\curveto(165.08860138,353.48313474)(165.08860138,353.39813482)(165.06860887,353.30813724)
\curveto(165.01860145,353.15813506)(164.9686015,353.01313521)(164.91860887,352.87313724)
\curveto(164.87860159,352.74313548)(164.82360164,352.61313561)(164.75360887,352.48313724)
\curveto(164.60360186,352.18313604)(164.41360205,351.9181363)(164.18360887,351.68813724)
\curveto(163.9636025,351.45813676)(163.69360277,351.27313695)(163.37360887,351.13313724)
\curveto(163.29360317,351.09313713)(163.20860326,351.05813716)(163.11860887,351.02813724)
\curveto(163.02860344,351.00813721)(162.93360353,350.98313724)(162.83360887,350.95313724)
\curveto(162.72360374,350.91313731)(162.61360385,350.89313733)(162.50360887,350.89313724)
\curveto(162.39360407,350.88313734)(162.28360418,350.86813735)(162.17360887,350.84813724)
\curveto(162.13360433,350.82813739)(162.09360437,350.8231374)(162.05360887,350.83313724)
\curveto(162.01360445,350.84313738)(161.97360449,350.84313738)(161.93360887,350.83313724)
\lineto(161.79860887,350.83313724)
\lineto(161.55860887,350.83313724)
\curveto(161.48860498,350.8231374)(161.42360504,350.82813739)(161.36360887,350.84813724)
\lineto(161.28860887,350.84813724)
\lineto(160.92860887,350.89313724)
\curveto(160.79860567,350.93313729)(160.67360579,350.96813725)(160.55360887,350.99813724)
\curveto(160.43360603,351.02813719)(160.31860615,351.06813715)(160.20860887,351.11813724)
\curveto(159.84860662,351.27813694)(159.54860692,351.46813675)(159.30860887,351.68813724)
\curveto(159.07860739,351.90813631)(158.8636076,352.17813604)(158.66360887,352.49813724)
\curveto(158.61360785,352.57813564)(158.5686079,352.66813555)(158.52860887,352.76813724)
\lineto(158.40860887,353.06813724)
\curveto(158.35860811,353.17813504)(158.32360814,353.29313493)(158.30360887,353.41313724)
\curveto(158.28360818,353.53313469)(158.25860821,353.65313457)(158.22860887,353.77313724)
\curveto(158.21860825,353.81313441)(158.21360825,353.85313437)(158.21360887,353.89313724)
\curveto(158.21360825,353.93313429)(158.20860826,353.97313425)(158.19860887,354.01313724)
\curveto(158.17860829,354.07313415)(158.1686083,354.13813408)(158.16860887,354.20813724)
\curveto(158.17860829,354.27813394)(158.17360829,354.34313388)(158.15360887,354.40313724)
\lineto(158.15360887,354.55313724)
\curveto(158.14360832,354.60313362)(158.13860833,354.67313355)(158.13860887,354.76313724)
\curveto(158.13860833,354.85313337)(158.14360832,354.9231333)(158.15360887,354.97313724)
\curveto(158.1636083,355.0231332)(158.1636083,355.06813315)(158.15360887,355.10813724)
\curveto(158.15360831,355.14813307)(158.15860831,355.18813303)(158.16860887,355.22813724)
\curveto(158.18860828,355.29813292)(158.19360827,355.36813285)(158.18360887,355.43813724)
\curveto(158.18360828,355.50813271)(158.19360827,355.57313265)(158.21360887,355.63313724)
\curveto(158.25360821,355.80313242)(158.28860818,355.97313225)(158.31860887,356.14313724)
\curveto(158.34860812,356.31313191)(158.39360807,356.47313175)(158.45360887,356.62313724)
\curveto(158.6636078,357.14313108)(158.91860755,357.56313066)(159.21860887,357.88313724)
\curveto(159.51860695,358.20313002)(159.92860654,358.46812975)(160.44860887,358.67813724)
\curveto(160.55860591,358.72812949)(160.67860579,358.76312946)(160.80860887,358.78313724)
\curveto(160.93860553,358.80312942)(161.07360539,358.82812939)(161.21360887,358.85813724)
\curveto(161.28360518,358.86812935)(161.35360511,358.87312935)(161.42360887,358.87313724)
\curveto(161.49360497,358.88312934)(161.5686049,358.89312933)(161.64860887,358.90313724)
}
}
{
\newrgbcolor{curcolor}{0 0 0}
\pscustom[linestyle=none,fillstyle=solid,fillcolor=curcolor]
{
\newpath
\moveto(167.03524949,358.72313724)
\lineto(167.47024949,358.72313724)
\curveto(167.62024753,358.7231295)(167.72524742,358.68312954)(167.78524949,358.60313724)
\curveto(167.83524731,358.5231297)(167.86024729,358.4231298)(167.86024949,358.30313724)
\curveto(167.87024728,358.18313004)(167.87524727,358.06313016)(167.87524949,357.94313724)
\lineto(167.87524949,356.51813724)
\lineto(167.87524949,354.25313724)
\lineto(167.87524949,353.56313724)
\curveto(167.87524727,353.33313489)(167.90024725,353.13313509)(167.95024949,352.96313724)
\curveto(168.11024704,352.51313571)(168.41024674,352.19813602)(168.85024949,352.01813724)
\curveto(169.07024608,351.92813629)(169.33524581,351.89313633)(169.64524949,351.91313724)
\curveto(169.95524519,351.94313628)(170.20524494,351.99813622)(170.39524949,352.07813724)
\curveto(170.72524442,352.218136)(170.98524416,352.39313583)(171.17524949,352.60313724)
\curveto(171.37524377,352.8231354)(171.53024362,353.10813511)(171.64024949,353.45813724)
\curveto(171.67024348,353.53813468)(171.69024346,353.6181346)(171.70024949,353.69813724)
\curveto(171.71024344,353.77813444)(171.72524342,353.86313436)(171.74524949,353.95313724)
\curveto(171.75524339,354.00313422)(171.75524339,354.04813417)(171.74524949,354.08813724)
\curveto(171.7452434,354.12813409)(171.75524339,354.17313405)(171.77524949,354.22313724)
\lineto(171.77524949,354.53813724)
\curveto(171.79524335,354.6181336)(171.80024335,354.70813351)(171.79024949,354.80813724)
\curveto(171.78024337,354.9181333)(171.77524337,355.0181332)(171.77524949,355.10813724)
\lineto(171.77524949,356.27813724)
\lineto(171.77524949,357.86813724)
\curveto(171.77524337,357.98813023)(171.77024338,358.11313011)(171.76024949,358.24313724)
\curveto(171.76024339,358.38312984)(171.78524336,358.49312973)(171.83524949,358.57313724)
\curveto(171.87524327,358.6231296)(171.92024323,358.65312957)(171.97024949,358.66313724)
\curveto(172.03024312,358.68312954)(172.10024305,358.70312952)(172.18024949,358.72313724)
\lineto(172.40524949,358.72313724)
\curveto(172.52524262,358.7231295)(172.63024252,358.7181295)(172.72024949,358.70813724)
\curveto(172.82024233,358.69812952)(172.89524225,358.65312957)(172.94524949,358.57313724)
\curveto(172.99524215,358.5231297)(173.02024213,358.44812977)(173.02024949,358.34813724)
\lineto(173.02024949,358.06313724)
\lineto(173.02024949,357.04313724)
\lineto(173.02024949,353.00813724)
\lineto(173.02024949,351.65813724)
\curveto(173.02024213,351.53813668)(173.01524213,351.4231368)(173.00524949,351.31313724)
\curveto(173.00524214,351.21313701)(172.97024218,351.13813708)(172.90024949,351.08813724)
\curveto(172.86024229,351.05813716)(172.80024235,351.03313719)(172.72024949,351.01313724)
\curveto(172.64024251,351.00313722)(172.5502426,350.99313723)(172.45024949,350.98313724)
\curveto(172.36024279,350.98313724)(172.27024288,350.98813723)(172.18024949,350.99813724)
\curveto(172.10024305,351.00813721)(172.04024311,351.02813719)(172.00024949,351.05813724)
\curveto(171.9502432,351.09813712)(171.90524324,351.16313706)(171.86524949,351.25313724)
\curveto(171.85524329,351.29313693)(171.8452433,351.34813687)(171.83524949,351.41813724)
\curveto(171.83524331,351.48813673)(171.83024332,351.55313667)(171.82024949,351.61313724)
\curveto(171.81024334,351.68313654)(171.79024336,351.73813648)(171.76024949,351.77813724)
\curveto(171.73024342,351.8181364)(171.68524346,351.83313639)(171.62524949,351.82313724)
\curveto(171.5452436,351.80313642)(171.46524368,351.74313648)(171.38524949,351.64313724)
\curveto(171.30524384,351.55313667)(171.23024392,351.48313674)(171.16024949,351.43313724)
\curveto(170.94024421,351.27313695)(170.69024446,351.13313709)(170.41024949,351.01313724)
\curveto(170.30024485,350.96313726)(170.18524496,350.93313729)(170.06524949,350.92313724)
\curveto(169.95524519,350.90313732)(169.84024531,350.87813734)(169.72024949,350.84813724)
\curveto(169.67024548,350.83813738)(169.61524553,350.83813738)(169.55524949,350.84813724)
\curveto(169.50524564,350.85813736)(169.45524569,350.85313737)(169.40524949,350.83313724)
\curveto(169.30524584,350.81313741)(169.21524593,350.81313741)(169.13524949,350.83313724)
\lineto(168.98524949,350.83313724)
\curveto(168.93524621,350.85313737)(168.87524627,350.86313736)(168.80524949,350.86313724)
\curveto(168.7452464,350.86313736)(168.69024646,350.86813735)(168.64024949,350.87813724)
\curveto(168.60024655,350.89813732)(168.56024659,350.90813731)(168.52024949,350.90813724)
\curveto(168.49024666,350.89813732)(168.4502467,350.90313732)(168.40024949,350.92313724)
\lineto(168.16024949,350.98313724)
\curveto(168.09024706,351.00313722)(168.01524713,351.03313719)(167.93524949,351.07313724)
\curveto(167.67524747,351.18313704)(167.45524769,351.32813689)(167.27524949,351.50813724)
\curveto(167.10524804,351.69813652)(166.96524818,351.9231363)(166.85524949,352.18313724)
\curveto(166.81524833,352.27313595)(166.78524836,352.36313586)(166.76524949,352.45313724)
\lineto(166.70524949,352.75313724)
\curveto(166.68524846,352.81313541)(166.67524847,352.86813535)(166.67524949,352.91813724)
\curveto(166.68524846,352.97813524)(166.68024847,353.04313518)(166.66024949,353.11313724)
\curveto(166.6502485,353.13313509)(166.6452485,353.15813506)(166.64524949,353.18813724)
\curveto(166.6452485,353.22813499)(166.64024851,353.26313496)(166.63024949,353.29313724)
\lineto(166.63024949,353.44313724)
\curveto(166.62024853,353.48313474)(166.61524853,353.52813469)(166.61524949,353.57813724)
\curveto(166.62524852,353.63813458)(166.63024852,353.69313453)(166.63024949,353.74313724)
\lineto(166.63024949,354.34313724)
\lineto(166.63024949,357.10313724)
\lineto(166.63024949,358.06313724)
\lineto(166.63024949,358.33313724)
\curveto(166.63024852,358.4231298)(166.6502485,358.49812972)(166.69024949,358.55813724)
\curveto(166.73024842,358.62812959)(166.80524834,358.67812954)(166.91524949,358.70813724)
\curveto(166.93524821,358.7181295)(166.95524819,358.7181295)(166.97524949,358.70813724)
\curveto(166.99524815,358.70812951)(167.01524813,358.71312951)(167.03524949,358.72313724)
}
}
{
\newrgbcolor{curcolor}{0 0 0}
\pscustom[linestyle=none,fillstyle=solid,fillcolor=curcolor]
{
\newpath
\moveto(178.56485887,358.90313724)
\curveto(178.79485408,358.90312932)(178.92485395,358.84312938)(178.95485887,358.72313724)
\curveto(178.98485389,358.61312961)(178.99985387,358.44812977)(178.99985887,358.22813724)
\lineto(178.99985887,357.94313724)
\curveto(178.99985387,357.85313037)(178.9748539,357.77813044)(178.92485887,357.71813724)
\curveto(178.86485401,357.63813058)(178.77985409,357.59313063)(178.66985887,357.58313724)
\curveto(178.55985431,357.58313064)(178.44985442,357.56813065)(178.33985887,357.53813724)
\curveto(178.19985467,357.50813071)(178.06485481,357.47813074)(177.93485887,357.44813724)
\curveto(177.81485506,357.4181308)(177.69985517,357.37813084)(177.58985887,357.32813724)
\curveto(177.29985557,357.19813102)(177.06485581,357.0181312)(176.88485887,356.78813724)
\curveto(176.70485617,356.56813165)(176.54985632,356.31313191)(176.41985887,356.02313724)
\curveto(176.37985649,355.91313231)(176.34985652,355.79813242)(176.32985887,355.67813724)
\curveto(176.30985656,355.56813265)(176.28485659,355.45313277)(176.25485887,355.33313724)
\curveto(176.24485663,355.28313294)(176.23985663,355.23313299)(176.23985887,355.18313724)
\curveto(176.24985662,355.13313309)(176.24985662,355.08313314)(176.23985887,355.03313724)
\curveto(176.20985666,354.91313331)(176.19485668,354.77313345)(176.19485887,354.61313724)
\curveto(176.20485667,354.46313376)(176.20985666,354.3181339)(176.20985887,354.17813724)
\lineto(176.20985887,352.33313724)
\lineto(176.20985887,351.98813724)
\curveto(176.20985666,351.86813635)(176.20485667,351.75313647)(176.19485887,351.64313724)
\curveto(176.18485669,351.53313669)(176.17985669,351.43813678)(176.17985887,351.35813724)
\curveto(176.18985668,351.27813694)(176.1698567,351.20813701)(176.11985887,351.14813724)
\curveto(176.0698568,351.07813714)(175.98985688,351.03813718)(175.87985887,351.02813724)
\curveto(175.77985709,351.0181372)(175.6698572,351.01313721)(175.54985887,351.01313724)
\lineto(175.27985887,351.01313724)
\curveto(175.22985764,351.03313719)(175.17985769,351.04813717)(175.12985887,351.05813724)
\curveto(175.08985778,351.07813714)(175.05985781,351.10313712)(175.03985887,351.13313724)
\curveto(174.98985788,351.20313702)(174.95985791,351.28813693)(174.94985887,351.38813724)
\lineto(174.94985887,351.71813724)
\lineto(174.94985887,352.87313724)
\lineto(174.94985887,357.02813724)
\lineto(174.94985887,358.06313724)
\lineto(174.94985887,358.36313724)
\curveto(174.95985791,358.46312976)(174.98985788,358.54812967)(175.03985887,358.61813724)
\curveto(175.0698578,358.65812956)(175.11985775,358.68812953)(175.18985887,358.70813724)
\curveto(175.2698576,358.72812949)(175.35485752,358.73812948)(175.44485887,358.73813724)
\curveto(175.53485734,358.74812947)(175.62485725,358.74812947)(175.71485887,358.73813724)
\curveto(175.80485707,358.72812949)(175.874857,358.71312951)(175.92485887,358.69313724)
\curveto(176.00485687,358.66312956)(176.05485682,358.60312962)(176.07485887,358.51313724)
\curveto(176.10485677,358.43312979)(176.11985675,358.34312988)(176.11985887,358.24313724)
\lineto(176.11985887,357.94313724)
\curveto(176.11985675,357.84313038)(176.13985673,357.75313047)(176.17985887,357.67313724)
\curveto(176.18985668,357.65313057)(176.19985667,357.63813058)(176.20985887,357.62813724)
\lineto(176.25485887,357.58313724)
\curveto(176.36485651,357.58313064)(176.45485642,357.62813059)(176.52485887,357.71813724)
\curveto(176.59485628,357.8181304)(176.65485622,357.89813032)(176.70485887,357.95813724)
\lineto(176.79485887,358.04813724)
\curveto(176.88485599,358.15813006)(177.00985586,358.27312995)(177.16985887,358.39313724)
\curveto(177.32985554,358.51312971)(177.47985539,358.60312962)(177.61985887,358.66313724)
\curveto(177.70985516,358.71312951)(177.80485507,358.74812947)(177.90485887,358.76813724)
\curveto(178.00485487,358.79812942)(178.10985476,358.82812939)(178.21985887,358.85813724)
\curveto(178.27985459,358.86812935)(178.33985453,358.87312935)(178.39985887,358.87313724)
\curveto(178.45985441,358.88312934)(178.51485436,358.89312933)(178.56485887,358.90313724)
}
}
{
\newrgbcolor{curcolor}{0 0 0}
\pscustom[linestyle=none,fillstyle=solid,fillcolor=curcolor]
{
\newpath
\moveto(182.35962449,358.90313724)
\curveto(183.07962043,358.91312931)(183.68461982,358.82812939)(184.17462449,358.64813724)
\curveto(184.66461884,358.47812974)(185.04461846,358.17313005)(185.31462449,357.73313724)
\curveto(185.38461812,357.6231306)(185.43961807,357.50813071)(185.47962449,357.38813724)
\curveto(185.51961799,357.27813094)(185.55961795,357.15313107)(185.59962449,357.01313724)
\curveto(185.61961789,356.94313128)(185.62461788,356.86813135)(185.61462449,356.78813724)
\curveto(185.6046179,356.7181315)(185.58961792,356.66313156)(185.56962449,356.62313724)
\curveto(185.54961796,356.60313162)(185.52461798,356.58313164)(185.49462449,356.56313724)
\curveto(185.46461804,356.55313167)(185.43961807,356.53813168)(185.41962449,356.51813724)
\curveto(185.36961814,356.49813172)(185.31961819,356.49313173)(185.26962449,356.50313724)
\curveto(185.21961829,356.51313171)(185.16961834,356.51313171)(185.11962449,356.50313724)
\curveto(185.03961847,356.48313174)(184.93461857,356.47813174)(184.80462449,356.48813724)
\curveto(184.67461883,356.50813171)(184.58461892,356.53313169)(184.53462449,356.56313724)
\curveto(184.45461905,356.61313161)(184.39961911,356.67813154)(184.36962449,356.75813724)
\curveto(184.34961916,356.84813137)(184.31461919,356.93313129)(184.26462449,357.01313724)
\curveto(184.17461933,357.17313105)(184.04961946,357.3181309)(183.88962449,357.44813724)
\curveto(183.77961973,357.52813069)(183.65961985,357.58813063)(183.52962449,357.62813724)
\curveto(183.39962011,357.66813055)(183.25962025,357.70813051)(183.10962449,357.74813724)
\curveto(183.05962045,357.76813045)(183.0096205,357.77313045)(182.95962449,357.76313724)
\curveto(182.9096206,357.76313046)(182.85962065,357.76813045)(182.80962449,357.77813724)
\curveto(182.74962076,357.79813042)(182.67462083,357.80813041)(182.58462449,357.80813724)
\curveto(182.49462101,357.80813041)(182.41962109,357.79813042)(182.35962449,357.77813724)
\lineto(182.26962449,357.77813724)
\lineto(182.11962449,357.74813724)
\curveto(182.06962144,357.74813047)(182.01962149,357.74313048)(181.96962449,357.73313724)
\curveto(181.7096218,357.67313055)(181.49462201,357.58813063)(181.32462449,357.47813724)
\curveto(181.15462235,357.36813085)(181.03962247,357.18313104)(180.97962449,356.92313724)
\curveto(180.95962255,356.85313137)(180.95462255,356.78313144)(180.96462449,356.71313724)
\curveto(180.98462252,356.64313158)(181.0046225,356.58313164)(181.02462449,356.53313724)
\curveto(181.08462242,356.38313184)(181.15462235,356.27313195)(181.23462449,356.20313724)
\curveto(181.32462218,356.14313208)(181.43462207,356.07313215)(181.56462449,355.99313724)
\curveto(181.72462178,355.89313233)(181.9046216,355.8181324)(182.10462449,355.76813724)
\curveto(182.3046212,355.72813249)(182.504621,355.67813254)(182.70462449,355.61813724)
\curveto(182.83462067,355.57813264)(182.96462054,355.54813267)(183.09462449,355.52813724)
\curveto(183.22462028,355.50813271)(183.35462015,355.47813274)(183.48462449,355.43813724)
\curveto(183.69461981,355.37813284)(183.89961961,355.3181329)(184.09962449,355.25813724)
\curveto(184.29961921,355.20813301)(184.49961901,355.14313308)(184.69962449,355.06313724)
\lineto(184.84962449,355.00313724)
\curveto(184.89961861,354.98313324)(184.94961856,354.95813326)(184.99962449,354.92813724)
\curveto(185.19961831,354.80813341)(185.37461813,354.67313355)(185.52462449,354.52313724)
\curveto(185.67461783,354.37313385)(185.79961771,354.18313404)(185.89962449,353.95313724)
\curveto(185.91961759,353.88313434)(185.93961757,353.78813443)(185.95962449,353.66813724)
\curveto(185.97961753,353.59813462)(185.98961752,353.5231347)(185.98962449,353.44313724)
\curveto(185.99961751,353.37313485)(186.0046175,353.29313493)(186.00462449,353.20313724)
\lineto(186.00462449,353.05313724)
\curveto(185.98461752,352.98313524)(185.97461753,352.91313531)(185.97462449,352.84313724)
\curveto(185.97461753,352.77313545)(185.96461754,352.70313552)(185.94462449,352.63313724)
\curveto(185.91461759,352.5231357)(185.87961763,352.4181358)(185.83962449,352.31813724)
\curveto(185.79961771,352.218136)(185.75461775,352.12813609)(185.70462449,352.04813724)
\curveto(185.54461796,351.78813643)(185.33961817,351.57813664)(185.08962449,351.41813724)
\curveto(184.83961867,351.26813695)(184.55961895,351.13813708)(184.24962449,351.02813724)
\curveto(184.15961935,350.99813722)(184.06461944,350.97813724)(183.96462449,350.96813724)
\curveto(183.87461963,350.94813727)(183.78461972,350.9231373)(183.69462449,350.89313724)
\curveto(183.59461991,350.87313735)(183.49462001,350.86313736)(183.39462449,350.86313724)
\curveto(183.29462021,350.86313736)(183.19462031,350.85313737)(183.09462449,350.83313724)
\lineto(182.94462449,350.83313724)
\curveto(182.89462061,350.8231374)(182.82462068,350.8181374)(182.73462449,350.81813724)
\curveto(182.64462086,350.8181374)(182.57462093,350.8231374)(182.52462449,350.83313724)
\lineto(182.35962449,350.83313724)
\curveto(182.29962121,350.85313737)(182.23462127,350.86313736)(182.16462449,350.86313724)
\curveto(182.09462141,350.85313737)(182.03462147,350.85813736)(181.98462449,350.87813724)
\curveto(181.93462157,350.88813733)(181.86962164,350.89313733)(181.78962449,350.89313724)
\lineto(181.54962449,350.95313724)
\curveto(181.47962203,350.96313726)(181.4046221,350.98313724)(181.32462449,351.01313724)
\curveto(181.01462249,351.11313711)(180.74462276,351.23813698)(180.51462449,351.38813724)
\curveto(180.28462322,351.53813668)(180.08462342,351.73313649)(179.91462449,351.97313724)
\curveto(179.82462368,352.10313612)(179.74962376,352.23813598)(179.68962449,352.37813724)
\curveto(179.62962388,352.5181357)(179.57462393,352.67313555)(179.52462449,352.84313724)
\curveto(179.504624,352.90313532)(179.49462401,352.97313525)(179.49462449,353.05313724)
\curveto(179.504624,353.14313508)(179.51962399,353.21313501)(179.53962449,353.26313724)
\curveto(179.56962394,353.30313492)(179.61962389,353.34313488)(179.68962449,353.38313724)
\curveto(179.73962377,353.40313482)(179.8096237,353.41313481)(179.89962449,353.41313724)
\curveto(179.98962352,353.4231348)(180.07962343,353.4231348)(180.16962449,353.41313724)
\curveto(180.25962325,353.40313482)(180.34462316,353.38813483)(180.42462449,353.36813724)
\curveto(180.51462299,353.35813486)(180.57462293,353.34313488)(180.60462449,353.32313724)
\curveto(180.67462283,353.27313495)(180.71962279,353.19813502)(180.73962449,353.09813724)
\curveto(180.76962274,353.00813521)(180.8046227,352.9231353)(180.84462449,352.84313724)
\curveto(180.94462256,352.6231356)(181.07962243,352.45313577)(181.24962449,352.33313724)
\curveto(181.36962214,352.24313598)(181.504622,352.17313605)(181.65462449,352.12313724)
\curveto(181.8046217,352.07313615)(181.96462154,352.0231362)(182.13462449,351.97313724)
\lineto(182.44962449,351.92813724)
\lineto(182.53962449,351.92813724)
\curveto(182.6096209,351.90813631)(182.69962081,351.89813632)(182.80962449,351.89813724)
\curveto(182.92962058,351.89813632)(183.02962048,351.90813631)(183.10962449,351.92813724)
\curveto(183.17962033,351.92813629)(183.23462027,351.93313629)(183.27462449,351.94313724)
\curveto(183.33462017,351.95313627)(183.39462011,351.95813626)(183.45462449,351.95813724)
\curveto(183.51461999,351.96813625)(183.56961994,351.97813624)(183.61962449,351.98813724)
\curveto(183.9096196,352.06813615)(184.13961937,352.17313605)(184.30962449,352.30313724)
\curveto(184.47961903,352.43313579)(184.59961891,352.65313557)(184.66962449,352.96313724)
\curveto(184.68961882,353.01313521)(184.69461881,353.06813515)(184.68462449,353.12813724)
\curveto(184.67461883,353.18813503)(184.66461884,353.23313499)(184.65462449,353.26313724)
\curveto(184.6046189,353.45313477)(184.53461897,353.59313463)(184.44462449,353.68313724)
\curveto(184.35461915,353.78313444)(184.23961927,353.87313435)(184.09962449,353.95313724)
\curveto(184.0096195,354.01313421)(183.9096196,354.06313416)(183.79962449,354.10313724)
\lineto(183.46962449,354.22313724)
\curveto(183.43962007,354.23313399)(183.4096201,354.23813398)(183.37962449,354.23813724)
\curveto(183.35962015,354.23813398)(183.33462017,354.24813397)(183.30462449,354.26813724)
\curveto(182.96462054,354.37813384)(182.6096209,354.45813376)(182.23962449,354.50813724)
\curveto(181.87962163,354.56813365)(181.53962197,354.66313356)(181.21962449,354.79313724)
\curveto(181.11962239,354.83313339)(181.02462248,354.86813335)(180.93462449,354.89813724)
\curveto(180.84462266,354.92813329)(180.75962275,354.96813325)(180.67962449,355.01813724)
\curveto(180.48962302,355.12813309)(180.31462319,355.25313297)(180.15462449,355.39313724)
\curveto(179.99462351,355.53313269)(179.86962364,355.70813251)(179.77962449,355.91813724)
\curveto(179.74962376,355.98813223)(179.72462378,356.05813216)(179.70462449,356.12813724)
\curveto(179.69462381,356.19813202)(179.67962383,356.27313195)(179.65962449,356.35313724)
\curveto(179.62962388,356.47313175)(179.61962389,356.60813161)(179.62962449,356.75813724)
\curveto(179.63962387,356.9181313)(179.65462385,357.05313117)(179.67462449,357.16313724)
\curveto(179.69462381,357.21313101)(179.7046238,357.25313097)(179.70462449,357.28313724)
\curveto(179.71462379,357.3231309)(179.72962378,357.36313086)(179.74962449,357.40313724)
\curveto(179.83962367,357.63313059)(179.95962355,357.83313039)(180.10962449,358.00313724)
\curveto(180.26962324,358.17313005)(180.44962306,358.3231299)(180.64962449,358.45313724)
\curveto(180.79962271,358.54312968)(180.96462254,358.61312961)(181.14462449,358.66313724)
\curveto(181.32462218,358.7231295)(181.51462199,358.77812944)(181.71462449,358.82813724)
\curveto(181.78462172,358.83812938)(181.84962166,358.84812937)(181.90962449,358.85813724)
\curveto(181.97962153,358.86812935)(182.05462145,358.87812934)(182.13462449,358.88813724)
\curveto(182.16462134,358.89812932)(182.2046213,358.89812932)(182.25462449,358.88813724)
\curveto(182.3046212,358.87812934)(182.33962117,358.88312934)(182.35962449,358.90313724)
}
}
{
\newrgbcolor{curcolor}{0 0 0}
\pscustom[linestyle=none,fillstyle=solid,fillcolor=curcolor]
{
\newpath
\moveto(194.55462449,355.19813724)
\curveto(194.57461643,355.13813308)(194.58461642,355.04313318)(194.58462449,354.91313724)
\curveto(194.58461642,354.79313343)(194.57961643,354.70813351)(194.56962449,354.65813724)
\lineto(194.56962449,354.50813724)
\curveto(194.55961645,354.42813379)(194.54961646,354.35313387)(194.53962449,354.28313724)
\curveto(194.53961647,354.223134)(194.53461647,354.15313407)(194.52462449,354.07313724)
\curveto(194.5046165,354.01313421)(194.48961652,353.95313427)(194.47962449,353.89313724)
\curveto(194.47961653,353.83313439)(194.46961654,353.77313445)(194.44962449,353.71313724)
\curveto(194.4096166,353.58313464)(194.37461663,353.45313477)(194.34462449,353.32313724)
\curveto(194.31461669,353.19313503)(194.27461673,353.07313515)(194.22462449,352.96313724)
\curveto(194.01461699,352.48313574)(193.73461727,352.07813614)(193.38462449,351.74813724)
\curveto(193.03461797,351.42813679)(192.6046184,351.18313704)(192.09462449,351.01313724)
\curveto(191.98461902,350.97313725)(191.86461914,350.94313728)(191.73462449,350.92313724)
\curveto(191.61461939,350.90313732)(191.48961952,350.88313734)(191.35962449,350.86313724)
\curveto(191.29961971,350.85313737)(191.23461977,350.84813737)(191.16462449,350.84813724)
\curveto(191.1046199,350.83813738)(191.04461996,350.83313739)(190.98462449,350.83313724)
\curveto(190.94462006,350.8231374)(190.88462012,350.8181374)(190.80462449,350.81813724)
\curveto(190.73462027,350.8181374)(190.68462032,350.8231374)(190.65462449,350.83313724)
\curveto(190.61462039,350.84313738)(190.57462043,350.84813737)(190.53462449,350.84813724)
\curveto(190.49462051,350.83813738)(190.45962055,350.83813738)(190.42962449,350.84813724)
\lineto(190.33962449,350.84813724)
\lineto(189.97962449,350.89313724)
\curveto(189.83962117,350.93313729)(189.7046213,350.97313725)(189.57462449,351.01313724)
\curveto(189.44462156,351.05313717)(189.31962169,351.09813712)(189.19962449,351.14813724)
\curveto(188.74962226,351.34813687)(188.37962263,351.60813661)(188.08962449,351.92813724)
\curveto(187.79962321,352.24813597)(187.55962345,352.63813558)(187.36962449,353.09813724)
\curveto(187.31962369,353.19813502)(187.27962373,353.29813492)(187.24962449,353.39813724)
\curveto(187.22962378,353.49813472)(187.2096238,353.60313462)(187.18962449,353.71313724)
\curveto(187.16962384,353.75313447)(187.15962385,353.78313444)(187.15962449,353.80313724)
\curveto(187.16962384,353.83313439)(187.16962384,353.86813435)(187.15962449,353.90813724)
\curveto(187.13962387,353.98813423)(187.12462388,354.06813415)(187.11462449,354.14813724)
\curveto(187.11462389,354.23813398)(187.1046239,354.3231339)(187.08462449,354.40313724)
\lineto(187.08462449,354.52313724)
\curveto(187.08462392,354.56313366)(187.07962393,354.60813361)(187.06962449,354.65813724)
\curveto(187.05962395,354.70813351)(187.05462395,354.79313343)(187.05462449,354.91313724)
\curveto(187.05462395,355.04313318)(187.06462394,355.13813308)(187.08462449,355.19813724)
\curveto(187.1046239,355.26813295)(187.1096239,355.33813288)(187.09962449,355.40813724)
\curveto(187.08962392,355.47813274)(187.09462391,355.54813267)(187.11462449,355.61813724)
\curveto(187.12462388,355.66813255)(187.12962388,355.70813251)(187.12962449,355.73813724)
\curveto(187.13962387,355.77813244)(187.14962386,355.8231324)(187.15962449,355.87313724)
\curveto(187.18962382,355.99313223)(187.21462379,356.11313211)(187.23462449,356.23313724)
\curveto(187.26462374,356.35313187)(187.3046237,356.46813175)(187.35462449,356.57813724)
\curveto(187.5046235,356.94813127)(187.68462332,357.27813094)(187.89462449,357.56813724)
\curveto(188.11462289,357.86813035)(188.37962263,358.1181301)(188.68962449,358.31813724)
\curveto(188.8096222,358.39812982)(188.93462207,358.46312976)(189.06462449,358.51313724)
\curveto(189.19462181,358.57312965)(189.32962168,358.63312959)(189.46962449,358.69313724)
\curveto(189.58962142,358.74312948)(189.71962129,358.77312945)(189.85962449,358.78313724)
\curveto(189.99962101,358.80312942)(190.13962087,358.83312939)(190.27962449,358.87313724)
\lineto(190.47462449,358.87313724)
\curveto(190.54462046,358.88312934)(190.6096204,358.89312933)(190.66962449,358.90313724)
\curveto(191.55961945,358.91312931)(192.29961871,358.72812949)(192.88962449,358.34813724)
\curveto(193.47961753,357.96813025)(193.9046171,357.47313075)(194.16462449,356.86313724)
\curveto(194.21461679,356.76313146)(194.25461675,356.66313156)(194.28462449,356.56313724)
\curveto(194.31461669,356.46313176)(194.34961666,356.35813186)(194.38962449,356.24813724)
\curveto(194.41961659,356.13813208)(194.44461656,356.0181322)(194.46462449,355.88813724)
\curveto(194.48461652,355.76813245)(194.5096165,355.64313258)(194.53962449,355.51313724)
\curveto(194.54961646,355.46313276)(194.54961646,355.40813281)(194.53962449,355.34813724)
\curveto(194.53961647,355.29813292)(194.54461646,355.24813297)(194.55462449,355.19813724)
\moveto(193.21962449,354.34313724)
\curveto(193.23961777,354.41313381)(193.24461776,354.49313373)(193.23462449,354.58313724)
\lineto(193.23462449,354.83813724)
\curveto(193.23461777,355.22813299)(193.19961781,355.55813266)(193.12962449,355.82813724)
\curveto(193.09961791,355.90813231)(193.07461793,355.98813223)(193.05462449,356.06813724)
\curveto(193.03461797,356.14813207)(193.009618,356.223132)(192.97962449,356.29313724)
\curveto(192.69961831,356.94313128)(192.25461875,357.39313083)(191.64462449,357.64313724)
\curveto(191.57461943,357.67313055)(191.49961951,357.69313053)(191.41962449,357.70313724)
\lineto(191.17962449,357.76313724)
\curveto(191.09961991,357.78313044)(191.01461999,357.79313043)(190.92462449,357.79313724)
\lineto(190.65462449,357.79313724)
\lineto(190.38462449,357.74813724)
\curveto(190.28462072,357.72813049)(190.18962082,357.70313052)(190.09962449,357.67313724)
\curveto(190.01962099,357.65313057)(189.93962107,357.6231306)(189.85962449,357.58313724)
\curveto(189.78962122,357.56313066)(189.72462128,357.53313069)(189.66462449,357.49313724)
\curveto(189.6046214,357.45313077)(189.54962146,357.41313081)(189.49962449,357.37313724)
\curveto(189.25962175,357.20313102)(189.06462194,356.99813122)(188.91462449,356.75813724)
\curveto(188.76462224,356.5181317)(188.63462237,356.23813198)(188.52462449,355.91813724)
\curveto(188.49462251,355.8181324)(188.47462253,355.71313251)(188.46462449,355.60313724)
\curveto(188.45462255,355.50313272)(188.43962257,355.39813282)(188.41962449,355.28813724)
\curveto(188.4096226,355.24813297)(188.4046226,355.18313304)(188.40462449,355.09313724)
\curveto(188.39462261,355.06313316)(188.38962262,355.02813319)(188.38962449,354.98813724)
\curveto(188.39962261,354.94813327)(188.4046226,354.90313332)(188.40462449,354.85313724)
\lineto(188.40462449,354.55313724)
\curveto(188.4046226,354.45313377)(188.41462259,354.36313386)(188.43462449,354.28313724)
\lineto(188.46462449,354.10313724)
\curveto(188.48462252,354.00313422)(188.49962251,353.90313432)(188.50962449,353.80313724)
\curveto(188.52962248,353.71313451)(188.55962245,353.62813459)(188.59962449,353.54813724)
\curveto(188.69962231,353.30813491)(188.81462219,353.08313514)(188.94462449,352.87313724)
\curveto(189.08462192,352.66313556)(189.25462175,352.48813573)(189.45462449,352.34813724)
\curveto(189.5046215,352.3181359)(189.54962146,352.29313593)(189.58962449,352.27313724)
\curveto(189.62962138,352.25313597)(189.67462133,352.22813599)(189.72462449,352.19813724)
\curveto(189.8046212,352.14813607)(189.88962112,352.10313612)(189.97962449,352.06313724)
\curveto(190.07962093,352.03313619)(190.18462082,352.00313622)(190.29462449,351.97313724)
\curveto(190.34462066,351.95313627)(190.38962062,351.94313628)(190.42962449,351.94313724)
\curveto(190.47962053,351.95313627)(190.52962048,351.95313627)(190.57962449,351.94313724)
\curveto(190.6096204,351.93313629)(190.66962034,351.9231363)(190.75962449,351.91313724)
\curveto(190.85962015,351.90313632)(190.93462007,351.90813631)(190.98462449,351.92813724)
\curveto(191.02461998,351.93813628)(191.06461994,351.93813628)(191.10462449,351.92813724)
\curveto(191.14461986,351.92813629)(191.18461982,351.93813628)(191.22462449,351.95813724)
\curveto(191.3046197,351.97813624)(191.38461962,351.99313623)(191.46462449,352.00313724)
\curveto(191.54461946,352.0231362)(191.61961939,352.04813617)(191.68962449,352.07813724)
\curveto(192.02961898,352.218136)(192.3046187,352.41313581)(192.51462449,352.66313724)
\curveto(192.72461828,352.91313531)(192.89961811,353.20813501)(193.03962449,353.54813724)
\curveto(193.08961792,353.66813455)(193.11961789,353.79313443)(193.12962449,353.92313724)
\curveto(193.14961786,354.06313416)(193.17961783,354.20313402)(193.21962449,354.34313724)
}
}
{
\newrgbcolor{curcolor}{0 0 0}
\pscustom[linestyle=none,fillstyle=solid,fillcolor=curcolor]
{
\newpath
\moveto(198.47290574,358.90313724)
\curveto(199.19290168,358.91312931)(199.79790107,358.82812939)(200.28790574,358.64813724)
\curveto(200.77790009,358.47812974)(201.15789971,358.17313005)(201.42790574,357.73313724)
\curveto(201.49789937,357.6231306)(201.55289932,357.50813071)(201.59290574,357.38813724)
\curveto(201.63289924,357.27813094)(201.6728992,357.15313107)(201.71290574,357.01313724)
\curveto(201.73289914,356.94313128)(201.73789913,356.86813135)(201.72790574,356.78813724)
\curveto(201.71789915,356.7181315)(201.70289917,356.66313156)(201.68290574,356.62313724)
\curveto(201.66289921,356.60313162)(201.63789923,356.58313164)(201.60790574,356.56313724)
\curveto(201.57789929,356.55313167)(201.55289932,356.53813168)(201.53290574,356.51813724)
\curveto(201.48289939,356.49813172)(201.43289944,356.49313173)(201.38290574,356.50313724)
\curveto(201.33289954,356.51313171)(201.28289959,356.51313171)(201.23290574,356.50313724)
\curveto(201.15289972,356.48313174)(201.04789982,356.47813174)(200.91790574,356.48813724)
\curveto(200.78790008,356.50813171)(200.69790017,356.53313169)(200.64790574,356.56313724)
\curveto(200.5679003,356.61313161)(200.51290036,356.67813154)(200.48290574,356.75813724)
\curveto(200.46290041,356.84813137)(200.42790044,356.93313129)(200.37790574,357.01313724)
\curveto(200.28790058,357.17313105)(200.16290071,357.3181309)(200.00290574,357.44813724)
\curveto(199.89290098,357.52813069)(199.7729011,357.58813063)(199.64290574,357.62813724)
\curveto(199.51290136,357.66813055)(199.3729015,357.70813051)(199.22290574,357.74813724)
\curveto(199.1729017,357.76813045)(199.12290175,357.77313045)(199.07290574,357.76313724)
\curveto(199.02290185,357.76313046)(198.9729019,357.76813045)(198.92290574,357.77813724)
\curveto(198.86290201,357.79813042)(198.78790208,357.80813041)(198.69790574,357.80813724)
\curveto(198.60790226,357.80813041)(198.53290234,357.79813042)(198.47290574,357.77813724)
\lineto(198.38290574,357.77813724)
\lineto(198.23290574,357.74813724)
\curveto(198.18290269,357.74813047)(198.13290274,357.74313048)(198.08290574,357.73313724)
\curveto(197.82290305,357.67313055)(197.60790326,357.58813063)(197.43790574,357.47813724)
\curveto(197.2679036,357.36813085)(197.15290372,357.18313104)(197.09290574,356.92313724)
\curveto(197.0729038,356.85313137)(197.0679038,356.78313144)(197.07790574,356.71313724)
\curveto(197.09790377,356.64313158)(197.11790375,356.58313164)(197.13790574,356.53313724)
\curveto(197.19790367,356.38313184)(197.2679036,356.27313195)(197.34790574,356.20313724)
\curveto(197.43790343,356.14313208)(197.54790332,356.07313215)(197.67790574,355.99313724)
\curveto(197.83790303,355.89313233)(198.01790285,355.8181324)(198.21790574,355.76813724)
\curveto(198.41790245,355.72813249)(198.61790225,355.67813254)(198.81790574,355.61813724)
\curveto(198.94790192,355.57813264)(199.07790179,355.54813267)(199.20790574,355.52813724)
\curveto(199.33790153,355.50813271)(199.4679014,355.47813274)(199.59790574,355.43813724)
\curveto(199.80790106,355.37813284)(200.01290086,355.3181329)(200.21290574,355.25813724)
\curveto(200.41290046,355.20813301)(200.61290026,355.14313308)(200.81290574,355.06313724)
\lineto(200.96290574,355.00313724)
\curveto(201.01289986,354.98313324)(201.06289981,354.95813326)(201.11290574,354.92813724)
\curveto(201.31289956,354.80813341)(201.48789938,354.67313355)(201.63790574,354.52313724)
\curveto(201.78789908,354.37313385)(201.91289896,354.18313404)(202.01290574,353.95313724)
\curveto(202.03289884,353.88313434)(202.05289882,353.78813443)(202.07290574,353.66813724)
\curveto(202.09289878,353.59813462)(202.10289877,353.5231347)(202.10290574,353.44313724)
\curveto(202.11289876,353.37313485)(202.11789875,353.29313493)(202.11790574,353.20313724)
\lineto(202.11790574,353.05313724)
\curveto(202.09789877,352.98313524)(202.08789878,352.91313531)(202.08790574,352.84313724)
\curveto(202.08789878,352.77313545)(202.07789879,352.70313552)(202.05790574,352.63313724)
\curveto(202.02789884,352.5231357)(201.99289888,352.4181358)(201.95290574,352.31813724)
\curveto(201.91289896,352.218136)(201.867899,352.12813609)(201.81790574,352.04813724)
\curveto(201.65789921,351.78813643)(201.45289942,351.57813664)(201.20290574,351.41813724)
\curveto(200.95289992,351.26813695)(200.6729002,351.13813708)(200.36290574,351.02813724)
\curveto(200.2729006,350.99813722)(200.17790069,350.97813724)(200.07790574,350.96813724)
\curveto(199.98790088,350.94813727)(199.89790097,350.9231373)(199.80790574,350.89313724)
\curveto(199.70790116,350.87313735)(199.60790126,350.86313736)(199.50790574,350.86313724)
\curveto(199.40790146,350.86313736)(199.30790156,350.85313737)(199.20790574,350.83313724)
\lineto(199.05790574,350.83313724)
\curveto(199.00790186,350.8231374)(198.93790193,350.8181374)(198.84790574,350.81813724)
\curveto(198.75790211,350.8181374)(198.68790218,350.8231374)(198.63790574,350.83313724)
\lineto(198.47290574,350.83313724)
\curveto(198.41290246,350.85313737)(198.34790252,350.86313736)(198.27790574,350.86313724)
\curveto(198.20790266,350.85313737)(198.14790272,350.85813736)(198.09790574,350.87813724)
\curveto(198.04790282,350.88813733)(197.98290289,350.89313733)(197.90290574,350.89313724)
\lineto(197.66290574,350.95313724)
\curveto(197.59290328,350.96313726)(197.51790335,350.98313724)(197.43790574,351.01313724)
\curveto(197.12790374,351.11313711)(196.85790401,351.23813698)(196.62790574,351.38813724)
\curveto(196.39790447,351.53813668)(196.19790467,351.73313649)(196.02790574,351.97313724)
\curveto(195.93790493,352.10313612)(195.86290501,352.23813598)(195.80290574,352.37813724)
\curveto(195.74290513,352.5181357)(195.68790518,352.67313555)(195.63790574,352.84313724)
\curveto(195.61790525,352.90313532)(195.60790526,352.97313525)(195.60790574,353.05313724)
\curveto(195.61790525,353.14313508)(195.63290524,353.21313501)(195.65290574,353.26313724)
\curveto(195.68290519,353.30313492)(195.73290514,353.34313488)(195.80290574,353.38313724)
\curveto(195.85290502,353.40313482)(195.92290495,353.41313481)(196.01290574,353.41313724)
\curveto(196.10290477,353.4231348)(196.19290468,353.4231348)(196.28290574,353.41313724)
\curveto(196.3729045,353.40313482)(196.45790441,353.38813483)(196.53790574,353.36813724)
\curveto(196.62790424,353.35813486)(196.68790418,353.34313488)(196.71790574,353.32313724)
\curveto(196.78790408,353.27313495)(196.83290404,353.19813502)(196.85290574,353.09813724)
\curveto(196.88290399,353.00813521)(196.91790395,352.9231353)(196.95790574,352.84313724)
\curveto(197.05790381,352.6231356)(197.19290368,352.45313577)(197.36290574,352.33313724)
\curveto(197.48290339,352.24313598)(197.61790325,352.17313605)(197.76790574,352.12313724)
\curveto(197.91790295,352.07313615)(198.07790279,352.0231362)(198.24790574,351.97313724)
\lineto(198.56290574,351.92813724)
\lineto(198.65290574,351.92813724)
\curveto(198.72290215,351.90813631)(198.81290206,351.89813632)(198.92290574,351.89813724)
\curveto(199.04290183,351.89813632)(199.14290173,351.90813631)(199.22290574,351.92813724)
\curveto(199.29290158,351.92813629)(199.34790152,351.93313629)(199.38790574,351.94313724)
\curveto(199.44790142,351.95313627)(199.50790136,351.95813626)(199.56790574,351.95813724)
\curveto(199.62790124,351.96813625)(199.68290119,351.97813624)(199.73290574,351.98813724)
\curveto(200.02290085,352.06813615)(200.25290062,352.17313605)(200.42290574,352.30313724)
\curveto(200.59290028,352.43313579)(200.71290016,352.65313557)(200.78290574,352.96313724)
\curveto(200.80290007,353.01313521)(200.80790006,353.06813515)(200.79790574,353.12813724)
\curveto(200.78790008,353.18813503)(200.77790009,353.23313499)(200.76790574,353.26313724)
\curveto(200.71790015,353.45313477)(200.64790022,353.59313463)(200.55790574,353.68313724)
\curveto(200.4679004,353.78313444)(200.35290052,353.87313435)(200.21290574,353.95313724)
\curveto(200.12290075,354.01313421)(200.02290085,354.06313416)(199.91290574,354.10313724)
\lineto(199.58290574,354.22313724)
\curveto(199.55290132,354.23313399)(199.52290135,354.23813398)(199.49290574,354.23813724)
\curveto(199.4729014,354.23813398)(199.44790142,354.24813397)(199.41790574,354.26813724)
\curveto(199.07790179,354.37813384)(198.72290215,354.45813376)(198.35290574,354.50813724)
\curveto(197.99290288,354.56813365)(197.65290322,354.66313356)(197.33290574,354.79313724)
\curveto(197.23290364,354.83313339)(197.13790373,354.86813335)(197.04790574,354.89813724)
\curveto(196.95790391,354.92813329)(196.872904,354.96813325)(196.79290574,355.01813724)
\curveto(196.60290427,355.12813309)(196.42790444,355.25313297)(196.26790574,355.39313724)
\curveto(196.10790476,355.53313269)(195.98290489,355.70813251)(195.89290574,355.91813724)
\curveto(195.86290501,355.98813223)(195.83790503,356.05813216)(195.81790574,356.12813724)
\curveto(195.80790506,356.19813202)(195.79290508,356.27313195)(195.77290574,356.35313724)
\curveto(195.74290513,356.47313175)(195.73290514,356.60813161)(195.74290574,356.75813724)
\curveto(195.75290512,356.9181313)(195.7679051,357.05313117)(195.78790574,357.16313724)
\curveto(195.80790506,357.21313101)(195.81790505,357.25313097)(195.81790574,357.28313724)
\curveto(195.82790504,357.3231309)(195.84290503,357.36313086)(195.86290574,357.40313724)
\curveto(195.95290492,357.63313059)(196.0729048,357.83313039)(196.22290574,358.00313724)
\curveto(196.38290449,358.17313005)(196.56290431,358.3231299)(196.76290574,358.45313724)
\curveto(196.91290396,358.54312968)(197.07790379,358.61312961)(197.25790574,358.66313724)
\curveto(197.43790343,358.7231295)(197.62790324,358.77812944)(197.82790574,358.82813724)
\curveto(197.89790297,358.83812938)(197.96290291,358.84812937)(198.02290574,358.85813724)
\curveto(198.09290278,358.86812935)(198.1679027,358.87812934)(198.24790574,358.88813724)
\curveto(198.27790259,358.89812932)(198.31790255,358.89812932)(198.36790574,358.88813724)
\curveto(198.41790245,358.87812934)(198.45290242,358.88312934)(198.47290574,358.90313724)
}
}
{
\newrgbcolor{curcolor}{0.7019608 0.7019608 0.7019608}
\pscustom[linestyle=none,fillstyle=solid,fillcolor=curcolor]
{
\newpath
\moveto(120.84556474,364.20820438)
\lineto(135.84556474,364.20820438)
\lineto(135.84556474,349.20820438)
\lineto(120.84556474,349.20820438)
\closepath
}
}
{
\newrgbcolor{curcolor}{0 0 0}
\pscustom[linestyle=none,fillstyle=solid,fillcolor=curcolor]
{
\newpath
\moveto(140.83376512,338.64237064)
\lineto(145.73876512,338.64237064)
\lineto(147.02876512,338.64237064)
\curveto(147.13875724,338.64235994)(147.24875713,338.64235994)(147.35876512,338.64237064)
\curveto(147.46875691,338.65235993)(147.55875682,338.63235995)(147.62876512,338.58237064)
\curveto(147.65875672,338.56236002)(147.68375669,338.53736005)(147.70376512,338.50737064)
\curveto(147.72375665,338.47736011)(147.74375663,338.44736014)(147.76376512,338.41737064)
\curveto(147.78375659,338.34736024)(147.79375658,338.23236035)(147.79376512,338.07237064)
\curveto(147.79375658,337.92236066)(147.78375659,337.80736078)(147.76376512,337.72737064)
\curveto(147.72375665,337.587361)(147.63875674,337.50736108)(147.50876512,337.48737064)
\curveto(147.378757,337.47736111)(147.22375715,337.47236111)(147.04376512,337.47237064)
\lineto(145.54376512,337.47237064)
\lineto(143.02376512,337.47237064)
\lineto(142.45376512,337.47237064)
\curveto(142.24376213,337.4823611)(142.08876229,337.45736113)(141.98876512,337.39737064)
\curveto(141.88876249,337.33736125)(141.83376254,337.23236135)(141.82376512,337.08237064)
\lineto(141.82376512,336.61737064)
\lineto(141.82376512,335.08737064)
\curveto(141.82376255,334.97736361)(141.81876256,334.84736374)(141.80876512,334.69737064)
\curveto(141.80876257,334.54736404)(141.81876256,334.42736416)(141.83876512,334.33737064)
\curveto(141.86876251,334.21736437)(141.92876245,334.13736445)(142.01876512,334.09737064)
\curveto(142.05876232,334.07736451)(142.12876225,334.05736453)(142.22876512,334.03737064)
\lineto(142.37876512,334.03737064)
\curveto(142.41876196,334.02736456)(142.45876192,334.02236456)(142.49876512,334.02237064)
\curveto(142.54876183,334.03236455)(142.59876178,334.03736455)(142.64876512,334.03737064)
\lineto(143.15876512,334.03737064)
\lineto(146.09876512,334.03737064)
\lineto(146.39876512,334.03737064)
\curveto(146.50875787,334.04736454)(146.61875776,334.04736454)(146.72876512,334.03737064)
\curveto(146.84875753,334.03736455)(146.95375742,334.02736456)(147.04376512,334.00737064)
\curveto(147.14375723,333.99736459)(147.21875716,333.97736461)(147.26876512,333.94737064)
\curveto(147.29875708,333.92736466)(147.32375705,333.8823647)(147.34376512,333.81237064)
\curveto(147.36375701,333.74236484)(147.378757,333.66736492)(147.38876512,333.58737064)
\curveto(147.39875698,333.50736508)(147.39875698,333.42236516)(147.38876512,333.33237064)
\curveto(147.38875699,333.25236533)(147.378757,333.1823654)(147.35876512,333.12237064)
\curveto(147.33875704,333.03236555)(147.29375708,332.96736562)(147.22376512,332.92737064)
\curveto(147.20375717,332.90736568)(147.1737572,332.89236569)(147.13376512,332.88237064)
\curveto(147.10375727,332.8823657)(147.0737573,332.87736571)(147.04376512,332.86737064)
\lineto(146.95376512,332.86737064)
\curveto(146.90375747,332.85736573)(146.85375752,332.85236573)(146.80376512,332.85237064)
\curveto(146.75375762,332.86236572)(146.70375767,332.86736572)(146.65376512,332.86737064)
\lineto(146.09876512,332.86737064)
\lineto(142.93376512,332.86737064)
\lineto(142.57376512,332.86737064)
\curveto(142.46376191,332.87736571)(142.35876202,332.87236571)(142.25876512,332.85237064)
\curveto(142.15876222,332.84236574)(142.06876231,332.81736577)(141.98876512,332.77737064)
\curveto(141.91876246,332.73736585)(141.86876251,332.66736592)(141.83876512,332.56737064)
\curveto(141.81876256,332.50736608)(141.80876257,332.43736615)(141.80876512,332.35737064)
\curveto(141.81876256,332.27736631)(141.82376255,332.19736639)(141.82376512,332.11737064)
\lineto(141.82376512,331.27737064)
\lineto(141.82376512,329.85237064)
\curveto(141.82376255,329.71236887)(141.82876255,329.582369)(141.83876512,329.46237064)
\curveto(141.84876253,329.35236923)(141.88876249,329.27236931)(141.95876512,329.22237064)
\curveto(142.02876235,329.17236941)(142.10876227,329.14236944)(142.19876512,329.13237064)
\lineto(142.49876512,329.13237064)
\lineto(143.45876512,329.13237064)
\lineto(146.23376512,329.13237064)
\lineto(147.08876512,329.13237064)
\lineto(147.32876512,329.13237064)
\curveto(147.40875697,329.14236944)(147.4787569,329.13736945)(147.53876512,329.11737064)
\curveto(147.65875672,329.07736951)(147.73875664,329.02236956)(147.77876512,328.95237064)
\curveto(147.79875658,328.92236966)(147.81375656,328.87236971)(147.82376512,328.80237064)
\curveto(147.83375654,328.73236985)(147.83875654,328.65736993)(147.83876512,328.57737064)
\curveto(147.84875653,328.50737008)(147.84875653,328.43237015)(147.83876512,328.35237064)
\curveto(147.82875655,328.2823703)(147.81875656,328.22737036)(147.80876512,328.18737064)
\curveto(147.76875661,328.10737048)(147.72375665,328.05237053)(147.67376512,328.02237064)
\curveto(147.61375676,327.9823706)(147.53375684,327.96237062)(147.43376512,327.96237064)
\lineto(147.16376512,327.96237064)
\lineto(146.11376512,327.96237064)
\lineto(142.12376512,327.96237064)
\lineto(141.07376512,327.96237064)
\curveto(140.93376344,327.96237062)(140.81376356,327.96737062)(140.71376512,327.97737064)
\curveto(140.61376376,327.99737059)(140.53876384,328.04737054)(140.48876512,328.12737064)
\curveto(140.44876393,328.1873704)(140.42876395,328.26237032)(140.42876512,328.35237064)
\lineto(140.42876512,328.63737064)
\lineto(140.42876512,329.68737064)
\lineto(140.42876512,333.70737064)
\lineto(140.42876512,337.06737064)
\lineto(140.42876512,337.99737064)
\lineto(140.42876512,338.26737064)
\curveto(140.42876395,338.35736023)(140.44876393,338.42736016)(140.48876512,338.47737064)
\curveto(140.52876385,338.54736004)(140.60376377,338.59735999)(140.71376512,338.62737064)
\curveto(140.73376364,338.63735995)(140.75376362,338.63735995)(140.77376512,338.62737064)
\curveto(140.79376358,338.62735996)(140.81376356,338.63235995)(140.83376512,338.64237064)
}
}
{
\newrgbcolor{curcolor}{0 0 0}
\pscustom[linestyle=none,fillstyle=solid,fillcolor=curcolor]
{
\newpath
\moveto(151.77368699,335.86737064)
\curveto(152.49368293,335.87736271)(153.09868232,335.79236279)(153.58868699,335.61237064)
\curveto(154.07868134,335.44236314)(154.45868096,335.13736345)(154.72868699,334.69737064)
\curveto(154.79868062,334.587364)(154.85368057,334.47236411)(154.89368699,334.35237064)
\curveto(154.93368049,334.24236434)(154.97368045,334.11736447)(155.01368699,333.97737064)
\curveto(155.03368039,333.90736468)(155.03868038,333.83236475)(155.02868699,333.75237064)
\curveto(155.0186804,333.6823649)(155.00368042,333.62736496)(154.98368699,333.58737064)
\curveto(154.96368046,333.56736502)(154.93868048,333.54736504)(154.90868699,333.52737064)
\curveto(154.87868054,333.51736507)(154.85368057,333.50236508)(154.83368699,333.48237064)
\curveto(154.78368064,333.46236512)(154.73368069,333.45736513)(154.68368699,333.46737064)
\curveto(154.63368079,333.47736511)(154.58368084,333.47736511)(154.53368699,333.46737064)
\curveto(154.45368097,333.44736514)(154.34868107,333.44236514)(154.21868699,333.45237064)
\curveto(154.08868133,333.47236511)(153.99868142,333.49736509)(153.94868699,333.52737064)
\curveto(153.86868155,333.57736501)(153.81368161,333.64236494)(153.78368699,333.72237064)
\curveto(153.76368166,333.81236477)(153.72868169,333.89736469)(153.67868699,333.97737064)
\curveto(153.58868183,334.13736445)(153.46368196,334.2823643)(153.30368699,334.41237064)
\curveto(153.19368223,334.49236409)(153.07368235,334.55236403)(152.94368699,334.59237064)
\curveto(152.81368261,334.63236395)(152.67368275,334.67236391)(152.52368699,334.71237064)
\curveto(152.47368295,334.73236385)(152.423683,334.73736385)(152.37368699,334.72737064)
\curveto(152.3236831,334.72736386)(152.27368315,334.73236385)(152.22368699,334.74237064)
\curveto(152.16368326,334.76236382)(152.08868333,334.77236381)(151.99868699,334.77237064)
\curveto(151.90868351,334.77236381)(151.83368359,334.76236382)(151.77368699,334.74237064)
\lineto(151.68368699,334.74237064)
\lineto(151.53368699,334.71237064)
\curveto(151.48368394,334.71236387)(151.43368399,334.70736388)(151.38368699,334.69737064)
\curveto(151.1236843,334.63736395)(150.90868451,334.55236403)(150.73868699,334.44237064)
\curveto(150.56868485,334.33236425)(150.45368497,334.14736444)(150.39368699,333.88737064)
\curveto(150.37368505,333.81736477)(150.36868505,333.74736484)(150.37868699,333.67737064)
\curveto(150.39868502,333.60736498)(150.418685,333.54736504)(150.43868699,333.49737064)
\curveto(150.49868492,333.34736524)(150.56868485,333.23736535)(150.64868699,333.16737064)
\curveto(150.73868468,333.10736548)(150.84868457,333.03736555)(150.97868699,332.95737064)
\curveto(151.13868428,332.85736573)(151.3186841,332.7823658)(151.51868699,332.73237064)
\curveto(151.7186837,332.69236589)(151.9186835,332.64236594)(152.11868699,332.58237064)
\curveto(152.24868317,332.54236604)(152.37868304,332.51236607)(152.50868699,332.49237064)
\curveto(152.63868278,332.47236611)(152.76868265,332.44236614)(152.89868699,332.40237064)
\curveto(153.10868231,332.34236624)(153.31368211,332.2823663)(153.51368699,332.22237064)
\curveto(153.71368171,332.17236641)(153.91368151,332.10736648)(154.11368699,332.02737064)
\lineto(154.26368699,331.96737064)
\curveto(154.31368111,331.94736664)(154.36368106,331.92236666)(154.41368699,331.89237064)
\curveto(154.61368081,331.77236681)(154.78868063,331.63736695)(154.93868699,331.48737064)
\curveto(155.08868033,331.33736725)(155.21368021,331.14736744)(155.31368699,330.91737064)
\curveto(155.33368009,330.84736774)(155.35368007,330.75236783)(155.37368699,330.63237064)
\curveto(155.39368003,330.56236802)(155.40368002,330.4873681)(155.40368699,330.40737064)
\curveto(155.41368001,330.33736825)(155.41868,330.25736833)(155.41868699,330.16737064)
\lineto(155.41868699,330.01737064)
\curveto(155.39868002,329.94736864)(155.38868003,329.87736871)(155.38868699,329.80737064)
\curveto(155.38868003,329.73736885)(155.37868004,329.66736892)(155.35868699,329.59737064)
\curveto(155.32868009,329.4873691)(155.29368013,329.3823692)(155.25368699,329.28237064)
\curveto(155.21368021,329.1823694)(155.16868025,329.09236949)(155.11868699,329.01237064)
\curveto(154.95868046,328.75236983)(154.75368067,328.54237004)(154.50368699,328.38237064)
\curveto(154.25368117,328.23237035)(153.97368145,328.10237048)(153.66368699,327.99237064)
\curveto(153.57368185,327.96237062)(153.47868194,327.94237064)(153.37868699,327.93237064)
\curveto(153.28868213,327.91237067)(153.19868222,327.8873707)(153.10868699,327.85737064)
\curveto(153.00868241,327.83737075)(152.90868251,327.82737076)(152.80868699,327.82737064)
\curveto(152.70868271,327.82737076)(152.60868281,327.81737077)(152.50868699,327.79737064)
\lineto(152.35868699,327.79737064)
\curveto(152.30868311,327.7873708)(152.23868318,327.7823708)(152.14868699,327.78237064)
\curveto(152.05868336,327.7823708)(151.98868343,327.7873708)(151.93868699,327.79737064)
\lineto(151.77368699,327.79737064)
\curveto(151.71368371,327.81737077)(151.64868377,327.82737076)(151.57868699,327.82737064)
\curveto(151.50868391,327.81737077)(151.44868397,327.82237076)(151.39868699,327.84237064)
\curveto(151.34868407,327.85237073)(151.28368414,327.85737073)(151.20368699,327.85737064)
\lineto(150.96368699,327.91737064)
\curveto(150.89368453,327.92737066)(150.8186846,327.94737064)(150.73868699,327.97737064)
\curveto(150.42868499,328.07737051)(150.15868526,328.20237038)(149.92868699,328.35237064)
\curveto(149.69868572,328.50237008)(149.49868592,328.69736989)(149.32868699,328.93737064)
\curveto(149.23868618,329.06736952)(149.16368626,329.20236938)(149.10368699,329.34237064)
\curveto(149.04368638,329.4823691)(148.98868643,329.63736895)(148.93868699,329.80737064)
\curveto(148.9186865,329.86736872)(148.90868651,329.93736865)(148.90868699,330.01737064)
\curveto(148.9186865,330.10736848)(148.93368649,330.17736841)(148.95368699,330.22737064)
\curveto(148.98368644,330.26736832)(149.03368639,330.30736828)(149.10368699,330.34737064)
\curveto(149.15368627,330.36736822)(149.2236862,330.37736821)(149.31368699,330.37737064)
\curveto(149.40368602,330.3873682)(149.49368593,330.3873682)(149.58368699,330.37737064)
\curveto(149.67368575,330.36736822)(149.75868566,330.35236823)(149.83868699,330.33237064)
\curveto(149.92868549,330.32236826)(149.98868543,330.30736828)(150.01868699,330.28737064)
\curveto(150.08868533,330.23736835)(150.13368529,330.16236842)(150.15368699,330.06237064)
\curveto(150.18368524,329.97236861)(150.2186852,329.8873687)(150.25868699,329.80737064)
\curveto(150.35868506,329.587369)(150.49368493,329.41736917)(150.66368699,329.29737064)
\curveto(150.78368464,329.20736938)(150.9186845,329.13736945)(151.06868699,329.08737064)
\curveto(151.2186842,329.03736955)(151.37868404,328.9873696)(151.54868699,328.93737064)
\lineto(151.86368699,328.89237064)
\lineto(151.95368699,328.89237064)
\curveto(152.0236834,328.87236971)(152.11368331,328.86236972)(152.22368699,328.86237064)
\curveto(152.34368308,328.86236972)(152.44368298,328.87236971)(152.52368699,328.89237064)
\curveto(152.59368283,328.89236969)(152.64868277,328.89736969)(152.68868699,328.90737064)
\curveto(152.74868267,328.91736967)(152.80868261,328.92236966)(152.86868699,328.92237064)
\curveto(152.92868249,328.93236965)(152.98368244,328.94236964)(153.03368699,328.95237064)
\curveto(153.3236821,329.03236955)(153.55368187,329.13736945)(153.72368699,329.26737064)
\curveto(153.89368153,329.39736919)(154.01368141,329.61736897)(154.08368699,329.92737064)
\curveto(154.10368132,329.97736861)(154.10868131,330.03236855)(154.09868699,330.09237064)
\curveto(154.08868133,330.15236843)(154.07868134,330.19736839)(154.06868699,330.22737064)
\curveto(154.0186814,330.41736817)(153.94868147,330.55736803)(153.85868699,330.64737064)
\curveto(153.76868165,330.74736784)(153.65368177,330.83736775)(153.51368699,330.91737064)
\curveto(153.423682,330.97736761)(153.3236821,331.02736756)(153.21368699,331.06737064)
\lineto(152.88368699,331.18737064)
\curveto(152.85368257,331.19736739)(152.8236826,331.20236738)(152.79368699,331.20237064)
\curveto(152.77368265,331.20236738)(152.74868267,331.21236737)(152.71868699,331.23237064)
\curveto(152.37868304,331.34236724)(152.0236834,331.42236716)(151.65368699,331.47237064)
\curveto(151.29368413,331.53236705)(150.95368447,331.62736696)(150.63368699,331.75737064)
\curveto(150.53368489,331.79736679)(150.43868498,331.83236675)(150.34868699,331.86237064)
\curveto(150.25868516,331.89236669)(150.17368525,331.93236665)(150.09368699,331.98237064)
\curveto(149.90368552,332.09236649)(149.72868569,332.21736637)(149.56868699,332.35737064)
\curveto(149.40868601,332.49736609)(149.28368614,332.67236591)(149.19368699,332.88237064)
\curveto(149.16368626,332.95236563)(149.13868628,333.02236556)(149.11868699,333.09237064)
\curveto(149.10868631,333.16236542)(149.09368633,333.23736535)(149.07368699,333.31737064)
\curveto(149.04368638,333.43736515)(149.03368639,333.57236501)(149.04368699,333.72237064)
\curveto(149.05368637,333.8823647)(149.06868635,334.01736457)(149.08868699,334.12737064)
\curveto(149.10868631,334.17736441)(149.1186863,334.21736437)(149.11868699,334.24737064)
\curveto(149.12868629,334.2873643)(149.14368628,334.32736426)(149.16368699,334.36737064)
\curveto(149.25368617,334.59736399)(149.37368605,334.79736379)(149.52368699,334.96737064)
\curveto(149.68368574,335.13736345)(149.86368556,335.2873633)(150.06368699,335.41737064)
\curveto(150.21368521,335.50736308)(150.37868504,335.57736301)(150.55868699,335.62737064)
\curveto(150.73868468,335.6873629)(150.92868449,335.74236284)(151.12868699,335.79237064)
\curveto(151.19868422,335.80236278)(151.26368416,335.81236277)(151.32368699,335.82237064)
\curveto(151.39368403,335.83236275)(151.46868395,335.84236274)(151.54868699,335.85237064)
\curveto(151.57868384,335.86236272)(151.6186838,335.86236272)(151.66868699,335.85237064)
\curveto(151.7186837,335.84236274)(151.75368367,335.84736274)(151.77368699,335.86737064)
}
}
{
\newrgbcolor{curcolor}{0 0 0}
\pscustom[linestyle=none,fillstyle=solid,fillcolor=curcolor]
{
\newpath
\moveto(164.26868699,332.02737064)
\curveto(164.27867864,331.97736661)(164.28367864,331.91236667)(164.28368699,331.83237064)
\curveto(164.28367864,331.75236683)(164.27867864,331.6873669)(164.26868699,331.63737064)
\curveto(164.24867867,331.587367)(164.24367868,331.53736705)(164.25368699,331.48737064)
\curveto(164.26367866,331.44736714)(164.26367866,331.40736718)(164.25368699,331.36737064)
\curveto(164.25367867,331.29736729)(164.24867867,331.24236734)(164.23868699,331.20237064)
\curveto(164.2186787,331.11236747)(164.20367872,331.02236756)(164.19368699,330.93237064)
\curveto(164.19367873,330.84236774)(164.18367874,330.75236783)(164.16368699,330.66237064)
\lineto(164.10368699,330.42237064)
\curveto(164.08367884,330.35236823)(164.05867886,330.27736831)(164.02868699,330.19737064)
\curveto(163.90867901,329.82736876)(163.74367918,329.49236909)(163.53368699,329.19237064)
\curveto(163.47367945,329.10236948)(163.40867951,329.01236957)(163.33868699,328.92237064)
\curveto(163.26867965,328.84236974)(163.19367973,328.76736982)(163.11368699,328.69737064)
\lineto(163.03868699,328.62237064)
\curveto(162.96867995,328.57237001)(162.90368002,328.52237006)(162.84368699,328.47237064)
\curveto(162.78368014,328.42237016)(162.71368021,328.37237021)(162.63368699,328.32237064)
\curveto(162.5236804,328.24237034)(162.39868052,328.17237041)(162.25868699,328.11237064)
\curveto(162.12868079,328.06237052)(161.99368093,328.01237057)(161.85368699,327.96237064)
\curveto(161.77368115,327.94237064)(161.69368123,327.92737066)(161.61368699,327.91737064)
\curveto(161.54368138,327.90737068)(161.46868145,327.89237069)(161.38868699,327.87237064)
\lineto(161.32868699,327.87237064)
\curveto(161.3186816,327.86237072)(161.30368162,327.85737073)(161.28368699,327.85737064)
\curveto(161.19368173,327.83737075)(161.05868186,327.82737076)(160.87868699,327.82737064)
\curveto(160.70868221,327.81737077)(160.57368235,327.82237076)(160.47368699,327.84237064)
\lineto(160.39868699,327.84237064)
\curveto(160.32868259,327.85237073)(160.26368266,327.86237072)(160.20368699,327.87237064)
\curveto(160.14368278,327.87237071)(160.08368284,327.8823707)(160.02368699,327.90237064)
\curveto(159.85368307,327.95237063)(159.69368323,327.99737059)(159.54368699,328.03737064)
\curveto(159.39368353,328.07737051)(159.25368367,328.13737045)(159.12368699,328.21737064)
\curveto(158.96368396,328.30737028)(158.8236841,328.40237018)(158.70368699,328.50237064)
\curveto(158.66368426,328.53237005)(158.60368432,328.57237001)(158.52368699,328.62237064)
\curveto(158.44368448,328.6823699)(158.36868455,328.6873699)(158.29868699,328.63737064)
\curveto(158.25868466,328.60736998)(158.23868468,328.56737002)(158.23868699,328.51737064)
\curveto(158.23868468,328.46737012)(158.22868469,328.41237017)(158.20868699,328.35237064)
\curveto(158.19868472,328.32237026)(158.19868472,328.2873703)(158.20868699,328.24737064)
\curveto(158.2186847,328.21737037)(158.2186847,328.1823704)(158.20868699,328.14237064)
\curveto(158.18868473,328.0823705)(158.17868474,328.01737057)(158.17868699,327.94737064)
\curveto(158.18868473,327.86737072)(158.19368473,327.79737079)(158.19368699,327.73737064)
\lineto(158.19368699,325.93737064)
\lineto(158.19368699,325.50237064)
\curveto(158.19368473,325.35237323)(158.16368476,325.23737335)(158.10368699,325.15737064)
\curveto(158.05368487,325.0873735)(157.97368495,325.05237353)(157.86368699,325.05237064)
\curveto(157.75368517,325.04237354)(157.64368528,325.03737355)(157.53368699,325.03737064)
\lineto(157.29368699,325.03737064)
\curveto(157.2236857,325.05737353)(157.16368576,325.07737351)(157.11368699,325.09737064)
\curveto(157.07368585,325.11737347)(157.03868588,325.15237343)(157.00868699,325.20237064)
\curveto(156.95868596,325.27237331)(156.93368599,325.3823732)(156.93368699,325.53237064)
\curveto(156.94368598,325.6823729)(156.94868597,325.81237277)(156.94868699,325.92237064)
\lineto(156.94868699,334.92237064)
\lineto(156.94868699,335.28237064)
\curveto(156.95868596,335.41236317)(156.98868593,335.51736307)(157.03868699,335.59737064)
\curveto(157.06868585,335.63736295)(157.13368579,335.66736292)(157.23368699,335.68737064)
\curveto(157.34368558,335.71736287)(157.45868546,335.72736286)(157.57868699,335.71737064)
\curveto(157.69868522,335.71736287)(157.80868511,335.70236288)(157.90868699,335.67237064)
\curveto(158.0186849,335.65236293)(158.08868483,335.62236296)(158.11868699,335.58237064)
\curveto(158.15868476,335.53236305)(158.17868474,335.47236311)(158.17868699,335.40237064)
\curveto(158.18868473,335.33236325)(158.20868471,335.26236332)(158.23868699,335.19237064)
\curveto(158.25868466,335.16236342)(158.27368465,335.13736345)(158.28368699,335.11737064)
\curveto(158.30368462,335.10736348)(158.3236846,335.09236349)(158.34368699,335.07237064)
\curveto(158.45368447,335.06236352)(158.54368438,335.09736349)(158.61368699,335.17737064)
\curveto(158.69368423,335.25736333)(158.76868415,335.32236326)(158.83868699,335.37237064)
\curveto(159.09868382,335.55236303)(159.40868351,335.69236289)(159.76868699,335.79237064)
\curveto(159.85868306,335.81236277)(159.94868297,335.82736276)(160.03868699,335.83737064)
\curveto(160.13868278,335.84736274)(160.23868268,335.86236272)(160.33868699,335.88237064)
\curveto(160.37868254,335.89236269)(160.42868249,335.89236269)(160.48868699,335.88237064)
\curveto(160.54868237,335.87236271)(160.58868233,335.87736271)(160.60868699,335.89737064)
\curveto(161.03868188,335.90736268)(161.4186815,335.86236272)(161.74868699,335.76237064)
\curveto(162.07868084,335.67236291)(162.37368055,335.54236304)(162.63368699,335.37237064)
\lineto(162.78368699,335.25237064)
\curveto(162.83368009,335.22236336)(162.88368004,335.1873634)(162.93368699,335.14737064)
\curveto(162.95367997,335.12736346)(162.96867995,335.10736348)(162.97868699,335.08737064)
\curveto(162.99867992,335.07736351)(163.0186799,335.06236352)(163.03868699,335.04237064)
\curveto(163.08867983,334.99236359)(163.14367978,334.93736365)(163.20368699,334.87737064)
\curveto(163.26367966,334.81736377)(163.3186796,334.75736383)(163.36868699,334.69737064)
\curveto(163.48867943,334.52736406)(163.61367931,334.34236424)(163.74368699,334.14237064)
\curveto(163.8236791,334.01236457)(163.88867903,333.86736472)(163.93868699,333.70737064)
\curveto(163.99867892,333.54736504)(164.05367887,333.3873652)(164.10368699,333.22737064)
\curveto(164.1236788,333.14736544)(164.13867878,333.06236552)(164.14868699,332.97237064)
\curveto(164.16867875,332.8823657)(164.18867873,332.79736579)(164.20868699,332.71737064)
\lineto(164.20868699,332.59737064)
\curveto(164.2186787,332.56736602)(164.2236787,332.53736605)(164.22368699,332.50737064)
\curveto(164.24367868,332.45736613)(164.24867867,332.40236618)(164.23868699,332.34237064)
\curveto(164.23867868,332.2823663)(164.24867867,332.22736636)(164.26868699,332.17737064)
\lineto(164.26868699,332.02737064)
\moveto(162.93368699,331.62237064)
\curveto(162.95367997,331.67236691)(162.95867996,331.73236685)(162.94868699,331.80237064)
\curveto(162.93867998,331.8823667)(162.93367999,331.95236663)(162.93368699,332.01237064)
\curveto(162.93367999,332.1823664)(162.92368,332.34236624)(162.90368699,332.49237064)
\curveto(162.89368003,332.64236594)(162.86368006,332.7873658)(162.81368699,332.92737064)
\lineto(162.75368699,333.10737064)
\curveto(162.74368018,333.17736541)(162.7236802,333.24236534)(162.69368699,333.30237064)
\curveto(162.58368034,333.57236501)(162.40868051,333.83236475)(162.16868699,334.08237064)
\curveto(161.93868098,334.33236425)(161.7186812,334.50236408)(161.50868699,334.59237064)
\curveto(161.42868149,334.63236395)(161.34368158,334.66236392)(161.25368699,334.68237064)
\curveto(161.17368175,334.70236388)(161.08868183,334.72736386)(160.99868699,334.75737064)
\curveto(160.90868201,334.77736381)(160.80368212,334.7873638)(160.68368699,334.78737064)
\lineto(160.35368699,334.78737064)
\curveto(160.33368259,334.76736382)(160.29368263,334.75736383)(160.23368699,334.75737064)
\curveto(160.18368274,334.76736382)(160.13868278,334.76736382)(160.09868699,334.75737064)
\lineto(159.82868699,334.69737064)
\curveto(159.74868317,334.67736391)(159.66868325,334.64736394)(159.58868699,334.60737064)
\curveto(159.26868365,334.46736412)(159.00368392,334.26236432)(158.79368699,333.99237064)
\curveto(158.59368433,333.73236485)(158.43868448,333.42736516)(158.32868699,333.07737064)
\curveto(158.28868463,332.96736562)(158.25868466,332.85736573)(158.23868699,332.74737064)
\curveto(158.22868469,332.63736595)(158.21368471,332.52736606)(158.19368699,332.41737064)
\curveto(158.18368474,332.37736621)(158.17868474,332.33736625)(158.17868699,332.29737064)
\curveto(158.17868474,332.26736632)(158.17368475,332.23236635)(158.16368699,332.19237064)
\lineto(158.16368699,332.07237064)
\curveto(158.15368477,332.02236656)(158.14868477,331.94736664)(158.14868699,331.84737064)
\curveto(158.14868477,331.75736683)(158.15368477,331.6873669)(158.16368699,331.63737064)
\lineto(158.16368699,331.51737064)
\curveto(158.17368475,331.47736711)(158.17868474,331.43736715)(158.17868699,331.39737064)
\curveto(158.17868474,331.35736723)(158.18368474,331.32236726)(158.19368699,331.29237064)
\curveto(158.20368472,331.26236732)(158.20868471,331.23236735)(158.20868699,331.20237064)
\curveto(158.20868471,331.17236741)(158.21368471,331.13736745)(158.22368699,331.09737064)
\curveto(158.24368468,331.01736757)(158.25868466,330.93736765)(158.26868699,330.85737064)
\lineto(158.32868699,330.61737064)
\curveto(158.43868448,330.27736831)(158.58868433,329.97736861)(158.77868699,329.71737064)
\curveto(158.97868394,329.46736912)(159.23868368,329.27236931)(159.55868699,329.13237064)
\curveto(159.74868317,329.05236953)(159.94368298,328.99236959)(160.14368699,328.95237064)
\curveto(160.18368274,328.93236965)(160.2236827,328.92236966)(160.26368699,328.92237064)
\curveto(160.30368262,328.93236965)(160.34368258,328.93236965)(160.38368699,328.92237064)
\lineto(160.50368699,328.92237064)
\curveto(160.57368235,328.90236968)(160.64368228,328.90236968)(160.71368699,328.92237064)
\lineto(160.83368699,328.92237064)
\curveto(160.94368198,328.94236964)(161.04868187,328.95736963)(161.14868699,328.96737064)
\curveto(161.24868167,328.97736961)(161.34868157,329.00236958)(161.44868699,329.04237064)
\curveto(161.75868116,329.17236941)(162.00868091,329.34236924)(162.19868699,329.55237064)
\curveto(162.39868052,329.77236881)(162.56368036,330.03736855)(162.69368699,330.34737064)
\curveto(162.74368018,330.4873681)(162.77868014,330.62736796)(162.79868699,330.76737064)
\curveto(162.82868009,330.91736767)(162.86368006,331.07236751)(162.90368699,331.23237064)
\curveto(162.91368001,331.2823673)(162.91868,331.32736726)(162.91868699,331.36737064)
\curveto(162.91868,331.40736718)(162.92368,331.45236713)(162.93368699,331.50237064)
\lineto(162.93368699,331.62237064)
}
}
{
\newrgbcolor{curcolor}{0 0 0}
\pscustom[linestyle=none,fillstyle=solid,fillcolor=curcolor]
{
\newpath
\moveto(172.63493699,328.51737064)
\curveto(172.66492916,328.35737023)(172.64992918,328.22237036)(172.58993699,328.11237064)
\curveto(172.5299293,328.01237057)(172.44992938,327.93737065)(172.34993699,327.88737064)
\curveto(172.29992953,327.86737072)(172.24492958,327.85737073)(172.18493699,327.85737064)
\curveto(172.13492969,327.85737073)(172.07992975,327.84737074)(172.01993699,327.82737064)
\curveto(171.79993003,327.77737081)(171.57993025,327.79237079)(171.35993699,327.87237064)
\curveto(171.14993068,327.94237064)(171.00493082,328.03237055)(170.92493699,328.14237064)
\curveto(170.87493095,328.21237037)(170.829931,328.29237029)(170.78993699,328.38237064)
\curveto(170.74993108,328.4823701)(170.69993113,328.56237002)(170.63993699,328.62237064)
\curveto(170.61993121,328.64236994)(170.59493123,328.66236992)(170.56493699,328.68237064)
\curveto(170.54493128,328.70236988)(170.51493131,328.70736988)(170.47493699,328.69737064)
\curveto(170.36493146,328.66736992)(170.25993157,328.61236997)(170.15993699,328.53237064)
\curveto(170.06993176,328.45237013)(169.97993185,328.3823702)(169.88993699,328.32237064)
\curveto(169.75993207,328.24237034)(169.61993221,328.16737042)(169.46993699,328.09737064)
\curveto(169.31993251,328.03737055)(169.15993267,327.9823706)(168.98993699,327.93237064)
\curveto(168.88993294,327.90237068)(168.77993305,327.8823707)(168.65993699,327.87237064)
\curveto(168.54993328,327.86237072)(168.43993339,327.84737074)(168.32993699,327.82737064)
\curveto(168.27993355,327.81737077)(168.23493359,327.81237077)(168.19493699,327.81237064)
\lineto(168.08993699,327.81237064)
\curveto(167.97993385,327.79237079)(167.87493395,327.79237079)(167.77493699,327.81237064)
\lineto(167.63993699,327.81237064)
\curveto(167.58993424,327.82237076)(167.53993429,327.82737076)(167.48993699,327.82737064)
\curveto(167.43993439,327.82737076)(167.39493443,327.83737075)(167.35493699,327.85737064)
\curveto(167.31493451,327.86737072)(167.27993455,327.87237071)(167.24993699,327.87237064)
\curveto(167.2299346,327.86237072)(167.20493462,327.86237072)(167.17493699,327.87237064)
\lineto(166.93493699,327.93237064)
\curveto(166.85493497,327.94237064)(166.77993505,327.96237062)(166.70993699,327.99237064)
\curveto(166.40993542,328.12237046)(166.16493566,328.26737032)(165.97493699,328.42737064)
\curveto(165.79493603,328.59736999)(165.64493618,328.83236975)(165.52493699,329.13237064)
\curveto(165.43493639,329.35236923)(165.38993644,329.61736897)(165.38993699,329.92737064)
\lineto(165.38993699,330.24237064)
\curveto(165.39993643,330.29236829)(165.40493642,330.34236824)(165.40493699,330.39237064)
\lineto(165.43493699,330.57237064)
\lineto(165.55493699,330.90237064)
\curveto(165.59493623,331.01236757)(165.64493618,331.11236747)(165.70493699,331.20237064)
\curveto(165.88493594,331.49236709)(166.1299357,331.70736688)(166.43993699,331.84737064)
\curveto(166.74993508,331.9873666)(167.08993474,332.11236647)(167.45993699,332.22237064)
\curveto(167.59993423,332.26236632)(167.74493408,332.29236629)(167.89493699,332.31237064)
\curveto(168.04493378,332.33236625)(168.19493363,332.35736623)(168.34493699,332.38737064)
\curveto(168.41493341,332.40736618)(168.47993335,332.41736617)(168.53993699,332.41737064)
\curveto(168.60993322,332.41736617)(168.68493314,332.42736616)(168.76493699,332.44737064)
\curveto(168.83493299,332.46736612)(168.90493292,332.47736611)(168.97493699,332.47737064)
\curveto(169.04493278,332.4873661)(169.11993271,332.50236608)(169.19993699,332.52237064)
\curveto(169.44993238,332.582366)(169.68493214,332.63236595)(169.90493699,332.67237064)
\curveto(170.1249317,332.72236586)(170.29993153,332.83736575)(170.42993699,333.01737064)
\curveto(170.48993134,333.09736549)(170.53993129,333.19736539)(170.57993699,333.31737064)
\curveto(170.61993121,333.44736514)(170.61993121,333.587365)(170.57993699,333.73737064)
\curveto(170.51993131,333.97736461)(170.4299314,334.16736442)(170.30993699,334.30737064)
\curveto(170.19993163,334.44736414)(170.03993179,334.55736403)(169.82993699,334.63737064)
\curveto(169.70993212,334.6873639)(169.56493226,334.72236386)(169.39493699,334.74237064)
\curveto(169.23493259,334.76236382)(169.06493276,334.77236381)(168.88493699,334.77237064)
\curveto(168.70493312,334.77236381)(168.5299333,334.76236382)(168.35993699,334.74237064)
\curveto(168.18993364,334.72236386)(168.04493378,334.69236389)(167.92493699,334.65237064)
\curveto(167.75493407,334.59236399)(167.58993424,334.50736408)(167.42993699,334.39737064)
\curveto(167.34993448,334.33736425)(167.27493455,334.25736433)(167.20493699,334.15737064)
\curveto(167.14493468,334.06736452)(167.08993474,333.96736462)(167.03993699,333.85737064)
\curveto(167.00993482,333.77736481)(166.97993485,333.69236489)(166.94993699,333.60237064)
\curveto(166.9299349,333.51236507)(166.88493494,333.44236514)(166.81493699,333.39237064)
\curveto(166.77493505,333.36236522)(166.70493512,333.33736525)(166.60493699,333.31737064)
\curveto(166.51493531,333.30736528)(166.41993541,333.30236528)(166.31993699,333.30237064)
\curveto(166.21993561,333.30236528)(166.11993571,333.30736528)(166.01993699,333.31737064)
\curveto(165.9299359,333.33736525)(165.86493596,333.36236522)(165.82493699,333.39237064)
\curveto(165.78493604,333.42236516)(165.75493607,333.47236511)(165.73493699,333.54237064)
\curveto(165.71493611,333.61236497)(165.71493611,333.6873649)(165.73493699,333.76737064)
\curveto(165.76493606,333.89736469)(165.79493603,334.01736457)(165.82493699,334.12737064)
\curveto(165.86493596,334.24736434)(165.90993592,334.36236422)(165.95993699,334.47237064)
\curveto(166.14993568,334.82236376)(166.38993544,335.09236349)(166.67993699,335.28237064)
\curveto(166.96993486,335.4823631)(167.3299345,335.64236294)(167.75993699,335.76237064)
\curveto(167.85993397,335.7823628)(167.95993387,335.79736279)(168.05993699,335.80737064)
\curveto(168.16993366,335.81736277)(168.27993355,335.83236275)(168.38993699,335.85237064)
\curveto(168.4299334,335.86236272)(168.49493333,335.86236272)(168.58493699,335.85237064)
\curveto(168.67493315,335.85236273)(168.7299331,335.86236272)(168.74993699,335.88237064)
\curveto(169.44993238,335.89236269)(170.05993177,335.81236277)(170.57993699,335.64237064)
\curveto(171.09993073,335.47236311)(171.46493036,335.14736344)(171.67493699,334.66737064)
\curveto(171.76493006,334.46736412)(171.81493001,334.23236435)(171.82493699,333.96237064)
\curveto(171.84492998,333.70236488)(171.85492997,333.42736516)(171.85493699,333.13737064)
\lineto(171.85493699,329.82237064)
\curveto(171.85492997,329.6823689)(171.85992997,329.54736904)(171.86993699,329.41737064)
\curveto(171.87992995,329.2873693)(171.90992992,329.1823694)(171.95993699,329.10237064)
\curveto(172.00992982,329.03236955)(172.07492975,328.9823696)(172.15493699,328.95237064)
\curveto(172.24492958,328.91236967)(172.3299295,328.8823697)(172.40993699,328.86237064)
\curveto(172.48992934,328.85236973)(172.54992928,328.80736978)(172.58993699,328.72737064)
\curveto(172.60992922,328.69736989)(172.61992921,328.66736992)(172.61993699,328.63737064)
\curveto(172.61992921,328.60736998)(172.6249292,328.56737002)(172.63493699,328.51737064)
\moveto(170.48993699,330.18237064)
\curveto(170.54993128,330.32236826)(170.57993125,330.4823681)(170.57993699,330.66237064)
\curveto(170.58993124,330.85236773)(170.59493123,331.04736754)(170.59493699,331.24737064)
\curveto(170.59493123,331.35736723)(170.58993124,331.45736713)(170.57993699,331.54737064)
\curveto(170.56993126,331.63736695)(170.5299313,331.70736688)(170.45993699,331.75737064)
\curveto(170.4299314,331.77736681)(170.35993147,331.7873668)(170.24993699,331.78737064)
\curveto(170.2299316,331.76736682)(170.19493163,331.75736683)(170.14493699,331.75737064)
\curveto(170.09493173,331.75736683)(170.04993178,331.74736684)(170.00993699,331.72737064)
\curveto(169.9299319,331.70736688)(169.83993199,331.6873669)(169.73993699,331.66737064)
\lineto(169.43993699,331.60737064)
\curveto(169.40993242,331.60736698)(169.37493245,331.60236698)(169.33493699,331.59237064)
\lineto(169.22993699,331.59237064)
\curveto(169.07993275,331.55236703)(168.91493291,331.52736706)(168.73493699,331.51737064)
\curveto(168.56493326,331.51736707)(168.40493342,331.49736709)(168.25493699,331.45737064)
\curveto(168.17493365,331.43736715)(168.09993373,331.41736717)(168.02993699,331.39737064)
\curveto(167.96993386,331.3873672)(167.89993393,331.37236721)(167.81993699,331.35237064)
\curveto(167.65993417,331.30236728)(167.50993432,331.23736735)(167.36993699,331.15737064)
\curveto(167.2299346,331.0873675)(167.10993472,330.99736759)(167.00993699,330.88737064)
\curveto(166.90993492,330.77736781)(166.83493499,330.64236794)(166.78493699,330.48237064)
\curveto(166.73493509,330.33236825)(166.71493511,330.14736844)(166.72493699,329.92737064)
\curveto(166.7249351,329.82736876)(166.73993509,329.73236885)(166.76993699,329.64237064)
\curveto(166.80993502,329.56236902)(166.85493497,329.4873691)(166.90493699,329.41737064)
\curveto(166.98493484,329.30736928)(167.08993474,329.21236937)(167.21993699,329.13237064)
\curveto(167.34993448,329.06236952)(167.48993434,329.00236958)(167.63993699,328.95237064)
\curveto(167.68993414,328.94236964)(167.73993409,328.93736965)(167.78993699,328.93737064)
\curveto(167.83993399,328.93736965)(167.88993394,328.93236965)(167.93993699,328.92237064)
\curveto(168.00993382,328.90236968)(168.09493373,328.8873697)(168.19493699,328.87737064)
\curveto(168.30493352,328.87736971)(168.39493343,328.8873697)(168.46493699,328.90737064)
\curveto(168.5249333,328.92736966)(168.58493324,328.93236965)(168.64493699,328.92237064)
\curveto(168.70493312,328.92236966)(168.76493306,328.93236965)(168.82493699,328.95237064)
\curveto(168.90493292,328.97236961)(168.97993285,328.9873696)(169.04993699,328.99737064)
\curveto(169.1299327,329.00736958)(169.20493262,329.02736956)(169.27493699,329.05737064)
\curveto(169.56493226,329.17736941)(169.80993202,329.32236926)(170.00993699,329.49237064)
\curveto(170.21993161,329.66236892)(170.37993145,329.89236869)(170.48993699,330.18237064)
}
}
{
\newrgbcolor{curcolor}{0 0 0}
\pscustom[linestyle=none,fillstyle=solid,fillcolor=curcolor]
{
\newpath
\moveto(176.94157762,335.86737064)
\curveto(177.68157283,335.87736271)(178.29657221,335.76736282)(178.78657762,335.53737064)
\curveto(179.28657122,335.31736327)(179.68157083,334.9823636)(179.97157762,334.53237064)
\curveto(180.10157041,334.33236425)(180.2115703,334.0873645)(180.30157762,333.79737064)
\curveto(180.32157019,333.74736484)(180.33657017,333.6823649)(180.34657762,333.60237064)
\curveto(180.35657015,333.52236506)(180.35157016,333.45236513)(180.33157762,333.39237064)
\curveto(180.30157021,333.34236524)(180.25157026,333.29736529)(180.18157762,333.25737064)
\curveto(180.15157036,333.23736535)(180.12157039,333.22736536)(180.09157762,333.22737064)
\curveto(180.06157045,333.23736535)(180.02657048,333.23736535)(179.98657762,333.22737064)
\curveto(179.94657056,333.21736537)(179.9065706,333.21236537)(179.86657762,333.21237064)
\curveto(179.82657068,333.22236536)(179.78657072,333.22736536)(179.74657762,333.22737064)
\lineto(179.43157762,333.22737064)
\curveto(179.33157118,333.23736535)(179.24657126,333.26736532)(179.17657762,333.31737064)
\curveto(179.09657141,333.37736521)(179.04157147,333.46236512)(179.01157762,333.57237064)
\curveto(178.98157153,333.6823649)(178.94157157,333.77736481)(178.89157762,333.85737064)
\curveto(178.74157177,334.11736447)(178.54657196,334.32236426)(178.30657762,334.47237064)
\curveto(178.22657228,334.52236406)(178.14157237,334.56236402)(178.05157762,334.59237064)
\curveto(177.96157255,334.63236395)(177.86657264,334.66736392)(177.76657762,334.69737064)
\curveto(177.62657288,334.73736385)(177.44157307,334.75736383)(177.21157762,334.75737064)
\curveto(176.98157353,334.76736382)(176.79157372,334.74736384)(176.64157762,334.69737064)
\curveto(176.57157394,334.67736391)(176.506574,334.66236392)(176.44657762,334.65237064)
\curveto(176.38657412,334.64236394)(176.32157419,334.62736396)(176.25157762,334.60737064)
\curveto(175.99157452,334.49736409)(175.76157475,334.34736424)(175.56157762,334.15737064)
\curveto(175.36157515,333.96736462)(175.2065753,333.74236484)(175.09657762,333.48237064)
\curveto(175.05657545,333.39236519)(175.02157549,333.29736529)(174.99157762,333.19737064)
\curveto(174.96157555,333.10736548)(174.93157558,333.00736558)(174.90157762,332.89737064)
\lineto(174.81157762,332.49237064)
\curveto(174.80157571,332.44236614)(174.79657571,332.3873662)(174.79657762,332.32737064)
\curveto(174.8065757,332.26736632)(174.80157571,332.21236637)(174.78157762,332.16237064)
\lineto(174.78157762,332.04237064)
\curveto(174.77157574,332.00236658)(174.76157575,331.93736665)(174.75157762,331.84737064)
\curveto(174.75157576,331.75736683)(174.76157575,331.69236689)(174.78157762,331.65237064)
\curveto(174.79157572,331.60236698)(174.79157572,331.55236703)(174.78157762,331.50237064)
\curveto(174.77157574,331.45236713)(174.77157574,331.40236718)(174.78157762,331.35237064)
\curveto(174.79157572,331.31236727)(174.79657571,331.24236734)(174.79657762,331.14237064)
\curveto(174.81657569,331.06236752)(174.83157568,330.97736761)(174.84157762,330.88737064)
\curveto(174.86157565,330.79736779)(174.88157563,330.71236787)(174.90157762,330.63237064)
\curveto(175.0115755,330.31236827)(175.13657537,330.03236855)(175.27657762,329.79237064)
\curveto(175.42657508,329.56236902)(175.63157488,329.36236922)(175.89157762,329.19237064)
\curveto(175.98157453,329.14236944)(176.07157444,329.09736949)(176.16157762,329.05737064)
\curveto(176.26157425,329.01736957)(176.36657414,328.97736961)(176.47657762,328.93737064)
\curveto(176.52657398,328.92736966)(176.56657394,328.92236966)(176.59657762,328.92237064)
\curveto(176.62657388,328.92236966)(176.66657384,328.91736967)(176.71657762,328.90737064)
\curveto(176.74657376,328.89736969)(176.79657371,328.89236969)(176.86657762,328.89237064)
\lineto(177.03157762,328.89237064)
\curveto(177.03157348,328.8823697)(177.05157346,328.87736971)(177.09157762,328.87737064)
\curveto(177.1115734,328.8873697)(177.13657337,328.8873697)(177.16657762,328.87737064)
\curveto(177.19657331,328.87736971)(177.22657328,328.8823697)(177.25657762,328.89237064)
\curveto(177.32657318,328.91236967)(177.39157312,328.91736967)(177.45157762,328.90737064)
\curveto(177.52157299,328.90736968)(177.59157292,328.91736967)(177.66157762,328.93737064)
\curveto(177.92157259,329.01736957)(178.14657236,329.11736947)(178.33657762,329.23737064)
\curveto(178.52657198,329.36736922)(178.68657182,329.53236905)(178.81657762,329.73237064)
\curveto(178.86657164,329.81236877)(178.9115716,329.89736869)(178.95157762,329.98737064)
\lineto(179.07157762,330.25737064)
\curveto(179.09157142,330.33736825)(179.1115714,330.41236817)(179.13157762,330.48237064)
\curveto(179.16157135,330.56236802)(179.2115713,330.62736796)(179.28157762,330.67737064)
\curveto(179.3115712,330.70736788)(179.37157114,330.72736786)(179.46157762,330.73737064)
\curveto(179.55157096,330.75736783)(179.64657086,330.76736782)(179.74657762,330.76737064)
\curveto(179.85657065,330.77736781)(179.95657055,330.77736781)(180.04657762,330.76737064)
\curveto(180.14657036,330.75736783)(180.21657029,330.73736785)(180.25657762,330.70737064)
\curveto(180.31657019,330.66736792)(180.35157016,330.60736798)(180.36157762,330.52737064)
\curveto(180.38157013,330.44736814)(180.38157013,330.36236822)(180.36157762,330.27237064)
\curveto(180.3115702,330.12236846)(180.26157025,329.97736861)(180.21157762,329.83737064)
\curveto(180.17157034,329.70736888)(180.11657039,329.57736901)(180.04657762,329.44737064)
\curveto(179.89657061,329.14736944)(179.7065708,328.8823697)(179.47657762,328.65237064)
\curveto(179.25657125,328.42237016)(178.98657152,328.23737035)(178.66657762,328.09737064)
\curveto(178.58657192,328.05737053)(178.50157201,328.02237056)(178.41157762,327.99237064)
\curveto(178.32157219,327.97237061)(178.22657228,327.94737064)(178.12657762,327.91737064)
\curveto(178.01657249,327.87737071)(177.9065726,327.85737073)(177.79657762,327.85737064)
\curveto(177.68657282,327.84737074)(177.57657293,327.83237075)(177.46657762,327.81237064)
\curveto(177.42657308,327.79237079)(177.38657312,327.7873708)(177.34657762,327.79737064)
\curveto(177.3065732,327.80737078)(177.26657324,327.80737078)(177.22657762,327.79737064)
\lineto(177.09157762,327.79737064)
\lineto(176.85157762,327.79737064)
\curveto(176.78157373,327.7873708)(176.71657379,327.79237079)(176.65657762,327.81237064)
\lineto(176.58157762,327.81237064)
\lineto(176.22157762,327.85737064)
\curveto(176.09157442,327.89737069)(175.96657454,327.93237065)(175.84657762,327.96237064)
\curveto(175.72657478,327.99237059)(175.6115749,328.03237055)(175.50157762,328.08237064)
\curveto(175.14157537,328.24237034)(174.84157567,328.43237015)(174.60157762,328.65237064)
\curveto(174.37157614,328.87236971)(174.15657635,329.14236944)(173.95657762,329.46237064)
\curveto(173.9065766,329.54236904)(173.86157665,329.63236895)(173.82157762,329.73237064)
\lineto(173.70157762,330.03237064)
\curveto(173.65157686,330.14236844)(173.61657689,330.25736833)(173.59657762,330.37737064)
\curveto(173.57657693,330.49736809)(173.55157696,330.61736797)(173.52157762,330.73737064)
\curveto(173.511577,330.77736781)(173.506577,330.81736777)(173.50657762,330.85737064)
\curveto(173.506577,330.89736769)(173.50157701,330.93736765)(173.49157762,330.97737064)
\curveto(173.47157704,331.03736755)(173.46157705,331.10236748)(173.46157762,331.17237064)
\curveto(173.47157704,331.24236734)(173.46657704,331.30736728)(173.44657762,331.36737064)
\lineto(173.44657762,331.51737064)
\curveto(173.43657707,331.56736702)(173.43157708,331.63736695)(173.43157762,331.72737064)
\curveto(173.43157708,331.81736677)(173.43657707,331.8873667)(173.44657762,331.93737064)
\curveto(173.45657705,331.9873666)(173.45657705,332.03236655)(173.44657762,332.07237064)
\curveto(173.44657706,332.11236647)(173.45157706,332.15236643)(173.46157762,332.19237064)
\curveto(173.48157703,332.26236632)(173.48657702,332.33236625)(173.47657762,332.40237064)
\curveto(173.47657703,332.47236611)(173.48657702,332.53736605)(173.50657762,332.59737064)
\curveto(173.54657696,332.76736582)(173.58157693,332.93736565)(173.61157762,333.10737064)
\curveto(173.64157687,333.27736531)(173.68657682,333.43736515)(173.74657762,333.58737064)
\curveto(173.95657655,334.10736448)(174.2115763,334.52736406)(174.51157762,334.84737064)
\curveto(174.8115757,335.16736342)(175.22157529,335.43236315)(175.74157762,335.64237064)
\curveto(175.85157466,335.69236289)(175.97157454,335.72736286)(176.10157762,335.74737064)
\curveto(176.23157428,335.76736282)(176.36657414,335.79236279)(176.50657762,335.82237064)
\curveto(176.57657393,335.83236275)(176.64657386,335.83736275)(176.71657762,335.83737064)
\curveto(176.78657372,335.84736274)(176.86157365,335.85736273)(176.94157762,335.86737064)
}
}
{
\newrgbcolor{curcolor}{0 0 0}
\pscustom[linestyle=none,fillstyle=solid,fillcolor=curcolor]
{
\newpath
\moveto(182.14821824,337.18737064)
\curveto(182.06821712,337.24736134)(182.02321717,337.35236123)(182.01321824,337.50237064)
\lineto(182.01321824,337.96737064)
\lineto(182.01321824,338.22237064)
\curveto(182.01321718,338.31236027)(182.02821716,338.3873602)(182.05821824,338.44737064)
\curveto(182.09821709,338.52736006)(182.17821701,338.58736)(182.29821824,338.62737064)
\curveto(182.31821687,338.63735995)(182.33821685,338.63735995)(182.35821824,338.62737064)
\curveto(182.3882168,338.62735996)(182.41321678,338.63235995)(182.43321824,338.64237064)
\curveto(182.60321659,338.64235994)(182.76321643,338.63735995)(182.91321824,338.62737064)
\curveto(183.06321613,338.61735997)(183.16321603,338.55736003)(183.21321824,338.44737064)
\curveto(183.24321595,338.3873602)(183.25821593,338.31236027)(183.25821824,338.22237064)
\lineto(183.25821824,337.96737064)
\curveto(183.25821593,337.7873608)(183.25321594,337.61736097)(183.24321824,337.45737064)
\curveto(183.24321595,337.29736129)(183.17821601,337.19236139)(183.04821824,337.14237064)
\curveto(182.99821619,337.12236146)(182.94321625,337.11236147)(182.88321824,337.11237064)
\lineto(182.71821824,337.11237064)
\lineto(182.40321824,337.11237064)
\curveto(182.30321689,337.11236147)(182.21821697,337.13736145)(182.14821824,337.18737064)
\moveto(183.25821824,328.68237064)
\lineto(183.25821824,328.36737064)
\curveto(183.26821592,328.26737032)(183.24821594,328.1873704)(183.19821824,328.12737064)
\curveto(183.16821602,328.06737052)(183.12321607,328.02737056)(183.06321824,328.00737064)
\curveto(183.00321619,327.99737059)(182.93321626,327.9823706)(182.85321824,327.96237064)
\lineto(182.62821824,327.96237064)
\curveto(182.49821669,327.96237062)(182.38321681,327.96737062)(182.28321824,327.97737064)
\curveto(182.193217,327.99737059)(182.12321707,328.04737054)(182.07321824,328.12737064)
\curveto(182.03321716,328.1873704)(182.01321718,328.26237032)(182.01321824,328.35237064)
\lineto(182.01321824,328.63737064)
\lineto(182.01321824,334.98237064)
\lineto(182.01321824,335.29737064)
\curveto(182.01321718,335.40736318)(182.03821715,335.49236309)(182.08821824,335.55237064)
\curveto(182.11821707,335.60236298)(182.15821703,335.63236295)(182.20821824,335.64237064)
\curveto(182.25821693,335.65236293)(182.31321688,335.66736292)(182.37321824,335.68737064)
\curveto(182.3932168,335.6873629)(182.41321678,335.6823629)(182.43321824,335.67237064)
\curveto(182.46321673,335.67236291)(182.4882167,335.67736291)(182.50821824,335.68737064)
\curveto(182.63821655,335.6873629)(182.76821642,335.6823629)(182.89821824,335.67237064)
\curveto(183.03821615,335.67236291)(183.13321606,335.63236295)(183.18321824,335.55237064)
\curveto(183.23321596,335.49236309)(183.25821593,335.41236317)(183.25821824,335.31237064)
\lineto(183.25821824,335.02737064)
\lineto(183.25821824,328.68237064)
}
}
{
\newrgbcolor{curcolor}{0 0 0}
\pscustom[linestyle=none,fillstyle=solid,fillcolor=curcolor]
{
\newpath
\moveto(192.32806199,332.16237064)
\curveto(192.34805393,332.10236648)(192.35805392,332.00736658)(192.35806199,331.87737064)
\curveto(192.35805392,331.75736683)(192.35305393,331.67236691)(192.34306199,331.62237064)
\lineto(192.34306199,331.47237064)
\curveto(192.33305395,331.39236719)(192.32305396,331.31736727)(192.31306199,331.24737064)
\curveto(192.31305397,331.1873674)(192.30805397,331.11736747)(192.29806199,331.03737064)
\curveto(192.278054,330.97736761)(192.26305402,330.91736767)(192.25306199,330.85737064)
\curveto(192.25305403,330.79736779)(192.24305404,330.73736785)(192.22306199,330.67737064)
\curveto(192.1830541,330.54736804)(192.14805413,330.41736817)(192.11806199,330.28737064)
\curveto(192.08805419,330.15736843)(192.04805423,330.03736855)(191.99806199,329.92737064)
\curveto(191.78805449,329.44736914)(191.50805477,329.04236954)(191.15806199,328.71237064)
\curveto(190.80805547,328.39237019)(190.3780559,328.14737044)(189.86806199,327.97737064)
\curveto(189.75805652,327.93737065)(189.63805664,327.90737068)(189.50806199,327.88737064)
\curveto(189.38805689,327.86737072)(189.26305702,327.84737074)(189.13306199,327.82737064)
\curveto(189.07305721,327.81737077)(189.00805727,327.81237077)(188.93806199,327.81237064)
\curveto(188.8780574,327.80237078)(188.81805746,327.79737079)(188.75806199,327.79737064)
\curveto(188.71805756,327.7873708)(188.65805762,327.7823708)(188.57806199,327.78237064)
\curveto(188.50805777,327.7823708)(188.45805782,327.7873708)(188.42806199,327.79737064)
\curveto(188.38805789,327.80737078)(188.34805793,327.81237077)(188.30806199,327.81237064)
\curveto(188.26805801,327.80237078)(188.23305805,327.80237078)(188.20306199,327.81237064)
\lineto(188.11306199,327.81237064)
\lineto(187.75306199,327.85737064)
\curveto(187.61305867,327.89737069)(187.4780588,327.93737065)(187.34806199,327.97737064)
\curveto(187.21805906,328.01737057)(187.09305919,328.06237052)(186.97306199,328.11237064)
\curveto(186.52305976,328.31237027)(186.15306013,328.57237001)(185.86306199,328.89237064)
\curveto(185.57306071,329.21236937)(185.33306095,329.60236898)(185.14306199,330.06237064)
\curveto(185.09306119,330.16236842)(185.05306123,330.26236832)(185.02306199,330.36237064)
\curveto(185.00306128,330.46236812)(184.9830613,330.56736802)(184.96306199,330.67737064)
\curveto(184.94306134,330.71736787)(184.93306135,330.74736784)(184.93306199,330.76737064)
\curveto(184.94306134,330.79736779)(184.94306134,330.83236775)(184.93306199,330.87237064)
\curveto(184.91306137,330.95236763)(184.89806138,331.03236755)(184.88806199,331.11237064)
\curveto(184.88806139,331.20236738)(184.8780614,331.2873673)(184.85806199,331.36737064)
\lineto(184.85806199,331.48737064)
\curveto(184.85806142,331.52736706)(184.85306143,331.57236701)(184.84306199,331.62237064)
\curveto(184.83306145,331.67236691)(184.82806145,331.75736683)(184.82806199,331.87737064)
\curveto(184.82806145,332.00736658)(184.83806144,332.10236648)(184.85806199,332.16237064)
\curveto(184.8780614,332.23236635)(184.8830614,332.30236628)(184.87306199,332.37237064)
\curveto(184.86306142,332.44236614)(184.86806141,332.51236607)(184.88806199,332.58237064)
\curveto(184.89806138,332.63236595)(184.90306138,332.67236591)(184.90306199,332.70237064)
\curveto(184.91306137,332.74236584)(184.92306136,332.7873658)(184.93306199,332.83737064)
\curveto(184.96306132,332.95736563)(184.98806129,333.07736551)(185.00806199,333.19737064)
\curveto(185.03806124,333.31736527)(185.0780612,333.43236515)(185.12806199,333.54237064)
\curveto(185.278061,333.91236467)(185.45806082,334.24236434)(185.66806199,334.53237064)
\curveto(185.88806039,334.83236375)(186.15306013,335.0823635)(186.46306199,335.28237064)
\curveto(186.5830597,335.36236322)(186.70805957,335.42736316)(186.83806199,335.47737064)
\curveto(186.96805931,335.53736305)(187.10305918,335.59736299)(187.24306199,335.65737064)
\curveto(187.36305892,335.70736288)(187.49305879,335.73736285)(187.63306199,335.74737064)
\curveto(187.77305851,335.76736282)(187.91305837,335.79736279)(188.05306199,335.83737064)
\lineto(188.24806199,335.83737064)
\curveto(188.31805796,335.84736274)(188.3830579,335.85736273)(188.44306199,335.86737064)
\curveto(189.33305695,335.87736271)(190.07305621,335.69236289)(190.66306199,335.31237064)
\curveto(191.25305503,334.93236365)(191.6780546,334.43736415)(191.93806199,333.82737064)
\curveto(191.98805429,333.72736486)(192.02805425,333.62736496)(192.05806199,333.52737064)
\curveto(192.08805419,333.42736516)(192.12305416,333.32236526)(192.16306199,333.21237064)
\curveto(192.19305409,333.10236548)(192.21805406,332.9823656)(192.23806199,332.85237064)
\curveto(192.25805402,332.73236585)(192.283054,332.60736598)(192.31306199,332.47737064)
\curveto(192.32305396,332.42736616)(192.32305396,332.37236621)(192.31306199,332.31237064)
\curveto(192.31305397,332.26236632)(192.31805396,332.21236637)(192.32806199,332.16237064)
\moveto(190.99306199,331.30737064)
\curveto(191.01305527,331.37736721)(191.01805526,331.45736713)(191.00806199,331.54737064)
\lineto(191.00806199,331.80237064)
\curveto(191.00805527,332.19236639)(190.97305531,332.52236606)(190.90306199,332.79237064)
\curveto(190.87305541,332.87236571)(190.84805543,332.95236563)(190.82806199,333.03237064)
\curveto(190.80805547,333.11236547)(190.7830555,333.1873654)(190.75306199,333.25737064)
\curveto(190.47305581,333.90736468)(190.02805625,334.35736423)(189.41806199,334.60737064)
\curveto(189.34805693,334.63736395)(189.27305701,334.65736393)(189.19306199,334.66737064)
\lineto(188.95306199,334.72737064)
\curveto(188.87305741,334.74736384)(188.78805749,334.75736383)(188.69806199,334.75737064)
\lineto(188.42806199,334.75737064)
\lineto(188.15806199,334.71237064)
\curveto(188.05805822,334.69236389)(187.96305832,334.66736392)(187.87306199,334.63737064)
\curveto(187.79305849,334.61736397)(187.71305857,334.587364)(187.63306199,334.54737064)
\curveto(187.56305872,334.52736406)(187.49805878,334.49736409)(187.43806199,334.45737064)
\curveto(187.3780589,334.41736417)(187.32305896,334.37736421)(187.27306199,334.33737064)
\curveto(187.03305925,334.16736442)(186.83805944,333.96236462)(186.68806199,333.72237064)
\curveto(186.53805974,333.4823651)(186.40805987,333.20236538)(186.29806199,332.88237064)
\curveto(186.26806001,332.7823658)(186.24806003,332.67736591)(186.23806199,332.56737064)
\curveto(186.22806005,332.46736612)(186.21306007,332.36236622)(186.19306199,332.25237064)
\curveto(186.1830601,332.21236637)(186.1780601,332.14736644)(186.17806199,332.05737064)
\curveto(186.16806011,332.02736656)(186.16306012,331.99236659)(186.16306199,331.95237064)
\curveto(186.17306011,331.91236667)(186.1780601,331.86736672)(186.17806199,331.81737064)
\lineto(186.17806199,331.51737064)
\curveto(186.1780601,331.41736717)(186.18806009,331.32736726)(186.20806199,331.24737064)
\lineto(186.23806199,331.06737064)
\curveto(186.25806002,330.96736762)(186.27306001,330.86736772)(186.28306199,330.76737064)
\curveto(186.30305998,330.67736791)(186.33305995,330.59236799)(186.37306199,330.51237064)
\curveto(186.47305981,330.27236831)(186.58805969,330.04736854)(186.71806199,329.83737064)
\curveto(186.85805942,329.62736896)(187.02805925,329.45236913)(187.22806199,329.31237064)
\curveto(187.278059,329.2823693)(187.32305896,329.25736933)(187.36306199,329.23737064)
\curveto(187.40305888,329.21736937)(187.44805883,329.19236939)(187.49806199,329.16237064)
\curveto(187.5780587,329.11236947)(187.66305862,329.06736952)(187.75306199,329.02737064)
\curveto(187.85305843,328.99736959)(187.95805832,328.96736962)(188.06806199,328.93737064)
\curveto(188.11805816,328.91736967)(188.16305812,328.90736968)(188.20306199,328.90737064)
\curveto(188.25305803,328.91736967)(188.30305798,328.91736967)(188.35306199,328.90737064)
\curveto(188.3830579,328.89736969)(188.44305784,328.8873697)(188.53306199,328.87737064)
\curveto(188.63305765,328.86736972)(188.70805757,328.87236971)(188.75806199,328.89237064)
\curveto(188.79805748,328.90236968)(188.83805744,328.90236968)(188.87806199,328.89237064)
\curveto(188.91805736,328.89236969)(188.95805732,328.90236968)(188.99806199,328.92237064)
\curveto(189.0780572,328.94236964)(189.15805712,328.95736963)(189.23806199,328.96737064)
\curveto(189.31805696,328.9873696)(189.39305689,329.01236957)(189.46306199,329.04237064)
\curveto(189.80305648,329.1823694)(190.0780562,329.37736921)(190.28806199,329.62737064)
\curveto(190.49805578,329.87736871)(190.67305561,330.17236841)(190.81306199,330.51237064)
\curveto(190.86305542,330.63236795)(190.89305539,330.75736783)(190.90306199,330.88737064)
\curveto(190.92305536,331.02736756)(190.95305533,331.16736742)(190.99306199,331.30737064)
}
}
{
\newrgbcolor{curcolor}{0 0 0}
\pscustom[linestyle=none,fillstyle=solid,fillcolor=curcolor]
{
\newpath
\moveto(196.24634324,335.86737064)
\curveto(196.96633918,335.87736271)(197.57133857,335.79236279)(198.06134324,335.61237064)
\curveto(198.55133759,335.44236314)(198.93133721,335.13736345)(199.20134324,334.69737064)
\curveto(199.27133687,334.587364)(199.32633682,334.47236411)(199.36634324,334.35237064)
\curveto(199.40633674,334.24236434)(199.4463367,334.11736447)(199.48634324,333.97737064)
\curveto(199.50633664,333.90736468)(199.51133663,333.83236475)(199.50134324,333.75237064)
\curveto(199.49133665,333.6823649)(199.47633667,333.62736496)(199.45634324,333.58737064)
\curveto(199.43633671,333.56736502)(199.41133673,333.54736504)(199.38134324,333.52737064)
\curveto(199.35133679,333.51736507)(199.32633682,333.50236508)(199.30634324,333.48237064)
\curveto(199.25633689,333.46236512)(199.20633694,333.45736513)(199.15634324,333.46737064)
\curveto(199.10633704,333.47736511)(199.05633709,333.47736511)(199.00634324,333.46737064)
\curveto(198.92633722,333.44736514)(198.82133732,333.44236514)(198.69134324,333.45237064)
\curveto(198.56133758,333.47236511)(198.47133767,333.49736509)(198.42134324,333.52737064)
\curveto(198.3413378,333.57736501)(198.28633786,333.64236494)(198.25634324,333.72237064)
\curveto(198.23633791,333.81236477)(198.20133794,333.89736469)(198.15134324,333.97737064)
\curveto(198.06133808,334.13736445)(197.93633821,334.2823643)(197.77634324,334.41237064)
\curveto(197.66633848,334.49236409)(197.5463386,334.55236403)(197.41634324,334.59237064)
\curveto(197.28633886,334.63236395)(197.146339,334.67236391)(196.99634324,334.71237064)
\curveto(196.9463392,334.73236385)(196.89633925,334.73736385)(196.84634324,334.72737064)
\curveto(196.79633935,334.72736386)(196.7463394,334.73236385)(196.69634324,334.74237064)
\curveto(196.63633951,334.76236382)(196.56133958,334.77236381)(196.47134324,334.77237064)
\curveto(196.38133976,334.77236381)(196.30633984,334.76236382)(196.24634324,334.74237064)
\lineto(196.15634324,334.74237064)
\lineto(196.00634324,334.71237064)
\curveto(195.95634019,334.71236387)(195.90634024,334.70736388)(195.85634324,334.69737064)
\curveto(195.59634055,334.63736395)(195.38134076,334.55236403)(195.21134324,334.44237064)
\curveto(195.0413411,334.33236425)(194.92634122,334.14736444)(194.86634324,333.88737064)
\curveto(194.8463413,333.81736477)(194.8413413,333.74736484)(194.85134324,333.67737064)
\curveto(194.87134127,333.60736498)(194.89134125,333.54736504)(194.91134324,333.49737064)
\curveto(194.97134117,333.34736524)(195.0413411,333.23736535)(195.12134324,333.16737064)
\curveto(195.21134093,333.10736548)(195.32134082,333.03736555)(195.45134324,332.95737064)
\curveto(195.61134053,332.85736573)(195.79134035,332.7823658)(195.99134324,332.73237064)
\curveto(196.19133995,332.69236589)(196.39133975,332.64236594)(196.59134324,332.58237064)
\curveto(196.72133942,332.54236604)(196.85133929,332.51236607)(196.98134324,332.49237064)
\curveto(197.11133903,332.47236611)(197.2413389,332.44236614)(197.37134324,332.40237064)
\curveto(197.58133856,332.34236624)(197.78633836,332.2823663)(197.98634324,332.22237064)
\curveto(198.18633796,332.17236641)(198.38633776,332.10736648)(198.58634324,332.02737064)
\lineto(198.73634324,331.96737064)
\curveto(198.78633736,331.94736664)(198.83633731,331.92236666)(198.88634324,331.89237064)
\curveto(199.08633706,331.77236681)(199.26133688,331.63736695)(199.41134324,331.48737064)
\curveto(199.56133658,331.33736725)(199.68633646,331.14736744)(199.78634324,330.91737064)
\curveto(199.80633634,330.84736774)(199.82633632,330.75236783)(199.84634324,330.63237064)
\curveto(199.86633628,330.56236802)(199.87633627,330.4873681)(199.87634324,330.40737064)
\curveto(199.88633626,330.33736825)(199.89133625,330.25736833)(199.89134324,330.16737064)
\lineto(199.89134324,330.01737064)
\curveto(199.87133627,329.94736864)(199.86133628,329.87736871)(199.86134324,329.80737064)
\curveto(199.86133628,329.73736885)(199.85133629,329.66736892)(199.83134324,329.59737064)
\curveto(199.80133634,329.4873691)(199.76633638,329.3823692)(199.72634324,329.28237064)
\curveto(199.68633646,329.1823694)(199.6413365,329.09236949)(199.59134324,329.01237064)
\curveto(199.43133671,328.75236983)(199.22633692,328.54237004)(198.97634324,328.38237064)
\curveto(198.72633742,328.23237035)(198.4463377,328.10237048)(198.13634324,327.99237064)
\curveto(198.0463381,327.96237062)(197.95133819,327.94237064)(197.85134324,327.93237064)
\curveto(197.76133838,327.91237067)(197.67133847,327.8873707)(197.58134324,327.85737064)
\curveto(197.48133866,327.83737075)(197.38133876,327.82737076)(197.28134324,327.82737064)
\curveto(197.18133896,327.82737076)(197.08133906,327.81737077)(196.98134324,327.79737064)
\lineto(196.83134324,327.79737064)
\curveto(196.78133936,327.7873708)(196.71133943,327.7823708)(196.62134324,327.78237064)
\curveto(196.53133961,327.7823708)(196.46133968,327.7873708)(196.41134324,327.79737064)
\lineto(196.24634324,327.79737064)
\curveto(196.18633996,327.81737077)(196.12134002,327.82737076)(196.05134324,327.82737064)
\curveto(195.98134016,327.81737077)(195.92134022,327.82237076)(195.87134324,327.84237064)
\curveto(195.82134032,327.85237073)(195.75634039,327.85737073)(195.67634324,327.85737064)
\lineto(195.43634324,327.91737064)
\curveto(195.36634078,327.92737066)(195.29134085,327.94737064)(195.21134324,327.97737064)
\curveto(194.90134124,328.07737051)(194.63134151,328.20237038)(194.40134324,328.35237064)
\curveto(194.17134197,328.50237008)(193.97134217,328.69736989)(193.80134324,328.93737064)
\curveto(193.71134243,329.06736952)(193.63634251,329.20236938)(193.57634324,329.34237064)
\curveto(193.51634263,329.4823691)(193.46134268,329.63736895)(193.41134324,329.80737064)
\curveto(193.39134275,329.86736872)(193.38134276,329.93736865)(193.38134324,330.01737064)
\curveto(193.39134275,330.10736848)(193.40634274,330.17736841)(193.42634324,330.22737064)
\curveto(193.45634269,330.26736832)(193.50634264,330.30736828)(193.57634324,330.34737064)
\curveto(193.62634252,330.36736822)(193.69634245,330.37736821)(193.78634324,330.37737064)
\curveto(193.87634227,330.3873682)(193.96634218,330.3873682)(194.05634324,330.37737064)
\curveto(194.146342,330.36736822)(194.23134191,330.35236823)(194.31134324,330.33237064)
\curveto(194.40134174,330.32236826)(194.46134168,330.30736828)(194.49134324,330.28737064)
\curveto(194.56134158,330.23736835)(194.60634154,330.16236842)(194.62634324,330.06237064)
\curveto(194.65634149,329.97236861)(194.69134145,329.8873687)(194.73134324,329.80737064)
\curveto(194.83134131,329.587369)(194.96634118,329.41736917)(195.13634324,329.29737064)
\curveto(195.25634089,329.20736938)(195.39134075,329.13736945)(195.54134324,329.08737064)
\curveto(195.69134045,329.03736955)(195.85134029,328.9873696)(196.02134324,328.93737064)
\lineto(196.33634324,328.89237064)
\lineto(196.42634324,328.89237064)
\curveto(196.49633965,328.87236971)(196.58633956,328.86236972)(196.69634324,328.86237064)
\curveto(196.81633933,328.86236972)(196.91633923,328.87236971)(196.99634324,328.89237064)
\curveto(197.06633908,328.89236969)(197.12133902,328.89736969)(197.16134324,328.90737064)
\curveto(197.22133892,328.91736967)(197.28133886,328.92236966)(197.34134324,328.92237064)
\curveto(197.40133874,328.93236965)(197.45633869,328.94236964)(197.50634324,328.95237064)
\curveto(197.79633835,329.03236955)(198.02633812,329.13736945)(198.19634324,329.26737064)
\curveto(198.36633778,329.39736919)(198.48633766,329.61736897)(198.55634324,329.92737064)
\curveto(198.57633757,329.97736861)(198.58133756,330.03236855)(198.57134324,330.09237064)
\curveto(198.56133758,330.15236843)(198.55133759,330.19736839)(198.54134324,330.22737064)
\curveto(198.49133765,330.41736817)(198.42133772,330.55736803)(198.33134324,330.64737064)
\curveto(198.2413379,330.74736784)(198.12633802,330.83736775)(197.98634324,330.91737064)
\curveto(197.89633825,330.97736761)(197.79633835,331.02736756)(197.68634324,331.06737064)
\lineto(197.35634324,331.18737064)
\curveto(197.32633882,331.19736739)(197.29633885,331.20236738)(197.26634324,331.20237064)
\curveto(197.2463389,331.20236738)(197.22133892,331.21236737)(197.19134324,331.23237064)
\curveto(196.85133929,331.34236724)(196.49633965,331.42236716)(196.12634324,331.47237064)
\curveto(195.76634038,331.53236705)(195.42634072,331.62736696)(195.10634324,331.75737064)
\curveto(195.00634114,331.79736679)(194.91134123,331.83236675)(194.82134324,331.86237064)
\curveto(194.73134141,331.89236669)(194.6463415,331.93236665)(194.56634324,331.98237064)
\curveto(194.37634177,332.09236649)(194.20134194,332.21736637)(194.04134324,332.35737064)
\curveto(193.88134226,332.49736609)(193.75634239,332.67236591)(193.66634324,332.88237064)
\curveto(193.63634251,332.95236563)(193.61134253,333.02236556)(193.59134324,333.09237064)
\curveto(193.58134256,333.16236542)(193.56634258,333.23736535)(193.54634324,333.31737064)
\curveto(193.51634263,333.43736515)(193.50634264,333.57236501)(193.51634324,333.72237064)
\curveto(193.52634262,333.8823647)(193.5413426,334.01736457)(193.56134324,334.12737064)
\curveto(193.58134256,334.17736441)(193.59134255,334.21736437)(193.59134324,334.24737064)
\curveto(193.60134254,334.2873643)(193.61634253,334.32736426)(193.63634324,334.36737064)
\curveto(193.72634242,334.59736399)(193.8463423,334.79736379)(193.99634324,334.96737064)
\curveto(194.15634199,335.13736345)(194.33634181,335.2873633)(194.53634324,335.41737064)
\curveto(194.68634146,335.50736308)(194.85134129,335.57736301)(195.03134324,335.62737064)
\curveto(195.21134093,335.6873629)(195.40134074,335.74236284)(195.60134324,335.79237064)
\curveto(195.67134047,335.80236278)(195.73634041,335.81236277)(195.79634324,335.82237064)
\curveto(195.86634028,335.83236275)(195.9413402,335.84236274)(196.02134324,335.85237064)
\curveto(196.05134009,335.86236272)(196.09134005,335.86236272)(196.14134324,335.85237064)
\curveto(196.19133995,335.84236274)(196.22633992,335.84736274)(196.24634324,335.86737064)
}
}
{
\newrgbcolor{curcolor}{0.60000002 0.60000002 0.60000002}
\pscustom[linestyle=none,fillstyle=solid,fillcolor=curcolor]
{
\newpath
\moveto(120.84556474,341.17243777)
\lineto(135.84556474,341.17243777)
\lineto(135.84556474,326.17243777)
\lineto(120.84556474,326.17243777)
\closepath
}
}
{
\newrgbcolor{curcolor}{0.80000001 0.80000001 0.80000001}
\pscustom[linestyle=none,fillstyle=solid,fillcolor=curcolor]
{
\newpath
\moveto(1024.2857,127.5)
\lineto(1024.2857,163.97321)
\lineto(763.61219,163.97321)
\lineto(757.0462,161.81381)
\lineto(734.3809,161.81381)
\lineto(729.51954,158.90962)
\lineto(690.37613,159.03589)
\lineto(686.08298,156.63678)
\lineto(621.30695,156.88932)
\lineto(543.03572,154.82143)
\lineto(531.78572,153.21428)
\lineto(468.21429,153.39286)
\lineto(461.25,149.28571)
\lineto(410,149.82143)
\lineto(350,149.10714)
\lineto(337.85714,146.96428)
\lineto(315.53571,147.14286)
\lineto(307.32143,143.92857)
\lineto(302.32143,143.57143)
\lineto(297.32143,140.53571)
\lineto(261.78571,140.71428)
\lineto(250.17857,140.35714)
\lineto(245.71429,137.67857)
\lineto(235.17857,137.85714)
\lineto(227.85714,133.39286)
\lineto(195.89286,133.03571)
\lineto(189.28571,131.25)
\lineto(161.81381,131.38296)
\lineto(159.28843,130.05714)
\lineto(118.06158,130.43595)
\lineto(114.19643,127.5)
\lineto(1024.2857,127.5)
\closepath
}
}
{
\newrgbcolor{curcolor}{0.7019608 0.7019608 0.7019608}
\pscustom[linestyle=none,fillstyle=solid,fillcolor=curcolor]
{
\newpath
\moveto(115.17857,127.5)
\lineto(1024.2857,127.5)
\lineto(1024.2857,160)
\lineto(882.21041,160.046)
\lineto(882.17881,158.97271)
\lineto(692.3017,159.00431)
\lineto(685.23063,153.92198)
\lineto(547.2501,153.98508)
\lineto(543.28121,152.92406)
\lineto(494.21871,153.01336)
\lineto(486.2946,149.95533)
\lineto(458.32585,149.99993)
\lineto(453.23657,148.97314)
\lineto(435.33478,149.01774)
\lineto(431.20532,147.96863)
\lineto(420.29014,147.96863)
\lineto(418.25889,146.98649)
\lineto(354.26335,147.03109)
\lineto(347.2321,144.99984)
\lineto(317.2321,144.99984)
\lineto(307.20978,139.99984)
\lineto(298.28121,139.99984)
\lineto(296.31693,138.95073)
\lineto(261.2946,138.97303)
\lineto(258.21425,137.99089)
\lineto(251.27228,137.94629)
\lineto(247.25443,136.94182)
\lineto(236.27228,136.94182)
\lineto(229.24145,133.15059)
\lineto(195.14881,133.15059)
\lineto(190.47685,130.56207)
\lineto(119.91896,130.56207)
\closepath
}
}
{
\newrgbcolor{curcolor}{0.60000002 0.60000002 0.60000002}
\pscustom[linestyle=none,fillstyle=solid,fillcolor=curcolor]
{
\newpath
\moveto(1024.2857,127.5)
\lineto(1024.2857,135.1079)
\lineto(760.26606,135.48671)
\lineto(526.54201,133.0876)
\lineto(234.10285,133.59267)
\lineto(229.6875,130.80357)
\lineto(195.98214,130.9375)
\lineto(191.20536,128.75)
\lineto(116.74107,129.01786)
\lineto(114.82143,127.5)
\lineto(1024.2857,127.5)
\closepath
}
}
{
\newrgbcolor{curcolor}{0 0 0}
\pscustom[linestyle=none,fillstyle=solid,fillcolor=curcolor]
{
\newpath
\moveto(263.84668701,31.67142873)
\lineto(263.84668701,32.58642873)
\curveto(263.84669771,32.68642608)(263.84669771,32.78142599)(263.84668701,32.87142873)
\curveto(263.84669771,32.96142581)(263.86669769,33.03642573)(263.90668701,33.09642873)
\curveto(263.96669759,33.18642558)(264.04669751,33.24642552)(264.14668701,33.27642873)
\curveto(264.24669731,33.31642545)(264.3516972,33.36142541)(264.46168701,33.41142873)
\curveto(264.6516969,33.49142528)(264.84169671,33.56142521)(265.03168701,33.62142873)
\curveto(265.22169633,33.69142508)(265.41169614,33.766425)(265.60168701,33.84642873)
\curveto(265.78169577,33.91642485)(265.96669559,33.98142479)(266.15668701,34.04142873)
\curveto(266.33669522,34.10142467)(266.51669504,34.1714246)(266.69668701,34.25142873)
\curveto(266.83669472,34.31142446)(266.98169457,34.3664244)(267.13168701,34.41642873)
\curveto(267.28169427,34.4664243)(267.42669413,34.52142425)(267.56668701,34.58142873)
\curveto(268.01669354,34.76142401)(268.47169308,34.93142384)(268.93168701,35.09142873)
\curveto(269.38169217,35.25142352)(269.83169172,35.42142335)(270.28168701,35.60142873)
\curveto(270.33169122,35.62142315)(270.38169117,35.63642313)(270.43168701,35.64642873)
\lineto(270.58168701,35.70642873)
\curveto(270.80169075,35.79642297)(271.02669053,35.88142289)(271.25668701,35.96142873)
\curveto(271.47669008,36.04142273)(271.69668986,36.12642264)(271.91668701,36.21642873)
\curveto(272.00668955,36.25642251)(272.11668944,36.29642247)(272.24668701,36.33642873)
\curveto(272.36668919,36.37642239)(272.43668912,36.44142233)(272.45668701,36.53142873)
\curveto(272.46668909,36.5714222)(272.46668909,36.60142217)(272.45668701,36.62142873)
\lineto(272.39668701,36.68142873)
\curveto(272.34668921,36.73142204)(272.29168926,36.766422)(272.23168701,36.78642873)
\curveto(272.17168938,36.81642195)(272.10668945,36.84642192)(272.03668701,36.87642873)
\lineto(271.40668701,37.11642873)
\curveto(271.18669037,37.19642157)(270.97169058,37.27642149)(270.76168701,37.35642873)
\lineto(270.61168701,37.41642873)
\lineto(270.43168701,37.47642873)
\curveto(270.24169131,37.55642121)(270.0516915,37.62642114)(269.86168701,37.68642873)
\curveto(269.66169189,37.75642101)(269.46169209,37.83142094)(269.26168701,37.91142873)
\curveto(268.68169287,38.15142062)(268.09669346,38.3714204)(267.50668701,38.57142873)
\curveto(266.91669464,38.78141999)(266.33169522,39.00641976)(265.75168701,39.24642873)
\curveto(265.551696,39.32641944)(265.34669621,39.40141937)(265.13668701,39.47142873)
\curveto(264.92669663,39.55141922)(264.72169683,39.63141914)(264.52168701,39.71142873)
\curveto(264.44169711,39.75141902)(264.34169721,39.78641898)(264.22168701,39.81642873)
\curveto(264.10169745,39.85641891)(264.01669754,39.91141886)(263.96668701,39.98142873)
\curveto(263.92669763,40.04141873)(263.89669766,40.11641865)(263.87668701,40.20642873)
\curveto(263.8566977,40.30641846)(263.84669771,40.41641835)(263.84668701,40.53642873)
\curveto(263.83669772,40.65641811)(263.83669772,40.77641799)(263.84668701,40.89642873)
\curveto(263.84669771,41.01641775)(263.84669771,41.12641764)(263.84668701,41.22642873)
\curveto(263.84669771,41.31641745)(263.84669771,41.40641736)(263.84668701,41.49642873)
\curveto(263.84669771,41.59641717)(263.86669769,41.6714171)(263.90668701,41.72142873)
\curveto(263.9566976,41.81141696)(264.04669751,41.86141691)(264.17668701,41.87142873)
\curveto(264.30669725,41.88141689)(264.44669711,41.88641688)(264.59668701,41.88642873)
\lineto(266.24668701,41.88642873)
\lineto(272.51668701,41.88642873)
\lineto(273.77668701,41.88642873)
\curveto(273.88668767,41.88641688)(273.99668756,41.88641688)(274.10668701,41.88642873)
\curveto(274.21668734,41.89641687)(274.30168725,41.87641689)(274.36168701,41.82642873)
\curveto(274.42168713,41.79641697)(274.46168709,41.75141702)(274.48168701,41.69142873)
\curveto(274.49168706,41.63141714)(274.50668705,41.56141721)(274.52668701,41.48142873)
\lineto(274.52668701,41.24142873)
\lineto(274.52668701,40.88142873)
\curveto(274.51668704,40.771418)(274.47168708,40.69141808)(274.39168701,40.64142873)
\curveto(274.36168719,40.62141815)(274.33168722,40.60641816)(274.30168701,40.59642873)
\curveto(274.26168729,40.59641817)(274.21668734,40.58641818)(274.16668701,40.56642873)
\lineto(274.00168701,40.56642873)
\curveto(273.94168761,40.55641821)(273.87168768,40.55141822)(273.79168701,40.55142873)
\curveto(273.71168784,40.56141821)(273.63668792,40.5664182)(273.56668701,40.56642873)
\lineto(272.72668701,40.56642873)
\lineto(268.30168701,40.56642873)
\curveto(268.0516935,40.5664182)(267.80169375,40.5664182)(267.55168701,40.56642873)
\curveto(267.29169426,40.5664182)(267.04169451,40.56141821)(266.80168701,40.55142873)
\curveto(266.70169485,40.55141822)(266.59169496,40.54641822)(266.47168701,40.53642873)
\curveto(266.3516952,40.52641824)(266.29169526,40.4714183)(266.29168701,40.37142873)
\lineto(266.30668701,40.37142873)
\curveto(266.32669523,40.30141847)(266.39169516,40.24141853)(266.50168701,40.19142873)
\curveto(266.61169494,40.15141862)(266.70669485,40.11641865)(266.78668701,40.08642873)
\curveto(266.9566946,40.01641875)(267.13169442,39.95141882)(267.31168701,39.89142873)
\curveto(267.48169407,39.83141894)(267.6516939,39.76141901)(267.82168701,39.68142873)
\curveto(267.87169368,39.66141911)(267.91669364,39.64641912)(267.95668701,39.63642873)
\curveto(267.99669356,39.62641914)(268.04169351,39.61141916)(268.09168701,39.59142873)
\curveto(268.27169328,39.51141926)(268.4566931,39.44141933)(268.64668701,39.38142873)
\curveto(268.82669273,39.33141944)(269.00669255,39.2664195)(269.18668701,39.18642873)
\curveto(269.33669222,39.11641965)(269.49169206,39.05641971)(269.65168701,39.00642873)
\curveto(269.80169175,38.95641981)(269.9516916,38.90141987)(270.10168701,38.84142873)
\curveto(270.57169098,38.64142013)(271.04669051,38.46142031)(271.52668701,38.30142873)
\curveto(271.99668956,38.14142063)(272.46168909,37.9664208)(272.92168701,37.77642873)
\curveto(273.10168845,37.69642107)(273.28168827,37.62642114)(273.46168701,37.56642873)
\curveto(273.64168791,37.50642126)(273.82168773,37.44142133)(274.00168701,37.37142873)
\curveto(274.11168744,37.32142145)(274.21668734,37.2714215)(274.31668701,37.22142873)
\curveto(274.40668715,37.18142159)(274.47168708,37.09642167)(274.51168701,36.96642873)
\curveto(274.52168703,36.94642182)(274.52668703,36.92142185)(274.52668701,36.89142873)
\curveto(274.51668704,36.8714219)(274.51668704,36.84642192)(274.52668701,36.81642873)
\curveto(274.53668702,36.78642198)(274.54168701,36.75142202)(274.54168701,36.71142873)
\curveto(274.53168702,36.6714221)(274.52668703,36.63142214)(274.52668701,36.59142873)
\lineto(274.52668701,36.29142873)
\curveto(274.52668703,36.19142258)(274.50168705,36.11142266)(274.45168701,36.05142873)
\curveto(274.40168715,35.9714228)(274.33168722,35.91142286)(274.24168701,35.87142873)
\curveto(274.14168741,35.84142293)(274.04168751,35.80142297)(273.94168701,35.75142873)
\curveto(273.74168781,35.6714231)(273.53668802,35.59142318)(273.32668701,35.51142873)
\curveto(273.10668845,35.44142333)(272.89668866,35.3664234)(272.69668701,35.28642873)
\curveto(272.51668904,35.20642356)(272.33668922,35.13642363)(272.15668701,35.07642873)
\curveto(271.96668959,35.02642374)(271.78168977,34.96142381)(271.60168701,34.88142873)
\curveto(271.04169051,34.65142412)(270.47669108,34.43642433)(269.90668701,34.23642873)
\curveto(269.33669222,34.03642473)(268.77169278,33.82142495)(268.21168701,33.59142873)
\lineto(267.58168701,33.35142873)
\curveto(267.36169419,33.28142549)(267.1516944,33.20642556)(266.95168701,33.12642873)
\curveto(266.84169471,33.07642569)(266.73669482,33.03142574)(266.63668701,32.99142873)
\curveto(266.52669503,32.96142581)(266.43169512,32.91142586)(266.35168701,32.84142873)
\curveto(266.33169522,32.83142594)(266.32169523,32.82142595)(266.32168701,32.81142873)
\lineto(266.29168701,32.78142873)
\lineto(266.29168701,32.70642873)
\lineto(266.32168701,32.67642873)
\curveto(266.32169523,32.6664261)(266.32669523,32.65642611)(266.33668701,32.64642873)
\curveto(266.38669517,32.62642614)(266.44169511,32.61642615)(266.50168701,32.61642873)
\curveto(266.56169499,32.61642615)(266.62169493,32.60642616)(266.68168701,32.58642873)
\lineto(266.84668701,32.58642873)
\curveto(266.90669465,32.5664262)(266.97169458,32.56142621)(267.04168701,32.57142873)
\curveto(267.11169444,32.58142619)(267.18169437,32.58642618)(267.25168701,32.58642873)
\lineto(268.06168701,32.58642873)
\lineto(272.62168701,32.58642873)
\lineto(273.80668701,32.58642873)
\curveto(273.91668764,32.58642618)(274.02668753,32.58142619)(274.13668701,32.57142873)
\curveto(274.24668731,32.5714262)(274.33168722,32.54642622)(274.39168701,32.49642873)
\curveto(274.47168708,32.44642632)(274.51668704,32.35642641)(274.52668701,32.22642873)
\lineto(274.52668701,31.83642873)
\lineto(274.52668701,31.64142873)
\curveto(274.52668703,31.59142718)(274.51668704,31.54142723)(274.49668701,31.49142873)
\curveto(274.4566871,31.36142741)(274.37168718,31.28642748)(274.24168701,31.26642873)
\curveto(274.11168744,31.25642751)(273.96168759,31.25142752)(273.79168701,31.25142873)
\lineto(272.05168701,31.25142873)
\lineto(266.05168701,31.25142873)
\lineto(264.64168701,31.25142873)
\curveto(264.53169702,31.25142752)(264.41669714,31.24642752)(264.29668701,31.23642873)
\curveto(264.17669738,31.23642753)(264.08169747,31.26142751)(264.01168701,31.31142873)
\curveto(263.9516976,31.35142742)(263.90169765,31.42642734)(263.86168701,31.53642873)
\curveto(263.8516977,31.55642721)(263.8516977,31.57642719)(263.86168701,31.59642873)
\curveto(263.86169769,31.62642714)(263.8566977,31.65142712)(263.84668701,31.67142873)
}
}
{
\newrgbcolor{curcolor}{0 0 0}
\pscustom[linestyle=none,fillstyle=solid,fillcolor=curcolor]
{
\newpath
\moveto(273.97168701,50.87353811)
\curveto(274.13168742,50.90353028)(274.26668729,50.88853029)(274.37668701,50.82853811)
\curveto(274.47668708,50.76853041)(274.551687,50.68853049)(274.60168701,50.58853811)
\curveto(274.62168693,50.53853064)(274.63168692,50.4835307)(274.63168701,50.42353811)
\curveto(274.63168692,50.37353081)(274.64168691,50.31853086)(274.66168701,50.25853811)
\curveto(274.71168684,50.03853114)(274.69668686,49.81853136)(274.61668701,49.59853811)
\curveto(274.54668701,49.38853179)(274.4566871,49.24353194)(274.34668701,49.16353811)
\curveto(274.27668728,49.11353207)(274.19668736,49.06853211)(274.10668701,49.02853811)
\curveto(274.00668755,48.98853219)(273.92668763,48.93853224)(273.86668701,48.87853811)
\curveto(273.84668771,48.85853232)(273.82668773,48.83353235)(273.80668701,48.80353811)
\curveto(273.78668777,48.7835324)(273.78168777,48.75353243)(273.79168701,48.71353811)
\curveto(273.82168773,48.60353258)(273.87668768,48.49853268)(273.95668701,48.39853811)
\curveto(274.03668752,48.30853287)(274.10668745,48.21853296)(274.16668701,48.12853811)
\curveto(274.24668731,47.99853318)(274.32168723,47.85853332)(274.39168701,47.70853811)
\curveto(274.4516871,47.55853362)(274.50668705,47.39853378)(274.55668701,47.22853811)
\curveto(274.58668697,47.12853405)(274.60668695,47.01853416)(274.61668701,46.89853811)
\curveto(274.62668693,46.78853439)(274.64168691,46.6785345)(274.66168701,46.56853811)
\curveto(274.67168688,46.51853466)(274.67668688,46.47353471)(274.67668701,46.43353811)
\lineto(274.67668701,46.32853811)
\curveto(274.69668686,46.21853496)(274.69668686,46.11353507)(274.67668701,46.01353811)
\lineto(274.67668701,45.87853811)
\curveto(274.66668689,45.82853535)(274.66168689,45.7785354)(274.66168701,45.72853811)
\curveto(274.66168689,45.6785355)(274.6516869,45.63353555)(274.63168701,45.59353811)
\curveto(274.62168693,45.55353563)(274.61668694,45.51853566)(274.61668701,45.48853811)
\curveto(274.62668693,45.46853571)(274.62668693,45.44353574)(274.61668701,45.41353811)
\lineto(274.55668701,45.17353811)
\curveto(274.54668701,45.09353609)(274.52668703,45.01853616)(274.49668701,44.94853811)
\curveto(274.36668719,44.64853653)(274.22168733,44.40353678)(274.06168701,44.21353811)
\curveto(273.89168766,44.03353715)(273.6566879,43.8835373)(273.35668701,43.76353811)
\curveto(273.13668842,43.67353751)(272.87168868,43.62853755)(272.56168701,43.62853811)
\lineto(272.24668701,43.62853811)
\curveto(272.19668936,43.63853754)(272.14668941,43.64353754)(272.09668701,43.64353811)
\lineto(271.91668701,43.67353811)
\lineto(271.58668701,43.79353811)
\curveto(271.47669008,43.83353735)(271.37669018,43.8835373)(271.28668701,43.94353811)
\curveto(270.99669056,44.12353706)(270.78169077,44.36853681)(270.64168701,44.67853811)
\curveto(270.50169105,44.98853619)(270.37669118,45.32853585)(270.26668701,45.69853811)
\curveto(270.22669133,45.83853534)(270.19669136,45.9835352)(270.17668701,46.13353811)
\curveto(270.1566914,46.2835349)(270.13169142,46.43353475)(270.10168701,46.58353811)
\curveto(270.08169147,46.65353453)(270.07169148,46.71853446)(270.07168701,46.77853811)
\curveto(270.07169148,46.84853433)(270.06169149,46.92353426)(270.04168701,47.00353811)
\curveto(270.02169153,47.07353411)(270.01169154,47.14353404)(270.01168701,47.21353811)
\curveto(270.00169155,47.2835339)(269.98669157,47.35853382)(269.96668701,47.43853811)
\curveto(269.90669165,47.68853349)(269.8566917,47.92353326)(269.81668701,48.14353811)
\curveto(269.76669179,48.36353282)(269.6516919,48.53853264)(269.47168701,48.66853811)
\curveto(269.39169216,48.72853245)(269.29169226,48.7785324)(269.17168701,48.81853811)
\curveto(269.04169251,48.85853232)(268.90169265,48.85853232)(268.75168701,48.81853811)
\curveto(268.51169304,48.75853242)(268.32169323,48.66853251)(268.18168701,48.54853811)
\curveto(268.04169351,48.43853274)(267.93169362,48.2785329)(267.85168701,48.06853811)
\curveto(267.80169375,47.94853323)(267.76669379,47.80353338)(267.74668701,47.63353811)
\curveto(267.72669383,47.47353371)(267.71669384,47.30353388)(267.71668701,47.12353811)
\curveto(267.71669384,46.94353424)(267.72669383,46.76853441)(267.74668701,46.59853811)
\curveto(267.76669379,46.42853475)(267.79669376,46.2835349)(267.83668701,46.16353811)
\curveto(267.89669366,45.99353519)(267.98169357,45.82853535)(268.09168701,45.66853811)
\curveto(268.1516934,45.58853559)(268.23169332,45.51353567)(268.33168701,45.44353811)
\curveto(268.42169313,45.3835358)(268.52169303,45.32853585)(268.63168701,45.27853811)
\curveto(268.71169284,45.24853593)(268.79669276,45.21853596)(268.88668701,45.18853811)
\curveto(268.97669258,45.16853601)(269.04669251,45.12353606)(269.09668701,45.05353811)
\curveto(269.12669243,45.01353617)(269.1516924,44.94353624)(269.17168701,44.84353811)
\curveto(269.18169237,44.75353643)(269.18669237,44.65853652)(269.18668701,44.55853811)
\curveto(269.18669237,44.45853672)(269.18169237,44.35853682)(269.17168701,44.25853811)
\curveto(269.1516924,44.16853701)(269.12669243,44.10353708)(269.09668701,44.06353811)
\curveto(269.06669249,44.02353716)(269.01669254,43.99353719)(268.94668701,43.97353811)
\curveto(268.87669268,43.95353723)(268.80169275,43.95353723)(268.72168701,43.97353811)
\curveto(268.59169296,44.00353718)(268.47169308,44.03353715)(268.36168701,44.06353811)
\curveto(268.24169331,44.10353708)(268.12669343,44.14853703)(268.01668701,44.19853811)
\curveto(267.66669389,44.38853679)(267.39669416,44.62853655)(267.20668701,44.91853811)
\curveto(267.00669455,45.20853597)(266.84669471,45.56853561)(266.72668701,45.99853811)
\curveto(266.70669485,46.09853508)(266.69169486,46.19853498)(266.68168701,46.29853811)
\curveto(266.67169488,46.40853477)(266.6566949,46.51853466)(266.63668701,46.62853811)
\curveto(266.62669493,46.66853451)(266.62669493,46.73353445)(266.63668701,46.82353811)
\curveto(266.63669492,46.91353427)(266.62669493,46.96853421)(266.60668701,46.98853811)
\curveto(266.59669496,47.68853349)(266.67669488,48.29853288)(266.84668701,48.81853811)
\curveto(267.01669454,49.33853184)(267.34169421,49.70353148)(267.82168701,49.91353811)
\curveto(268.02169353,50.00353118)(268.2566933,50.05353113)(268.52668701,50.06353811)
\curveto(268.78669277,50.0835311)(269.06169249,50.09353109)(269.35168701,50.09353811)
\lineto(272.66668701,50.09353811)
\curveto(272.80668875,50.09353109)(272.94168861,50.09853108)(273.07168701,50.10853811)
\curveto(273.20168835,50.11853106)(273.30668825,50.14853103)(273.38668701,50.19853811)
\curveto(273.4566881,50.24853093)(273.50668805,50.31353087)(273.53668701,50.39353811)
\curveto(273.57668798,50.4835307)(273.60668795,50.56853061)(273.62668701,50.64853811)
\curveto(273.63668792,50.72853045)(273.68168787,50.78853039)(273.76168701,50.82853811)
\curveto(273.79168776,50.84853033)(273.82168773,50.85853032)(273.85168701,50.85853811)
\curveto(273.88168767,50.85853032)(273.92168763,50.86353032)(273.97168701,50.87353811)
\moveto(272.30668701,48.72853811)
\curveto(272.16668939,48.78853239)(272.00668955,48.81853236)(271.82668701,48.81853811)
\curveto(271.63668992,48.82853235)(271.44169011,48.83353235)(271.24168701,48.83353811)
\curveto(271.13169042,48.83353235)(271.03169052,48.82853235)(270.94168701,48.81853811)
\curveto(270.8516907,48.80853237)(270.78169077,48.76853241)(270.73168701,48.69853811)
\curveto(270.71169084,48.66853251)(270.70169085,48.59853258)(270.70168701,48.48853811)
\curveto(270.72169083,48.46853271)(270.73169082,48.43353275)(270.73168701,48.38353811)
\curveto(270.73169082,48.33353285)(270.74169081,48.28853289)(270.76168701,48.24853811)
\curveto(270.78169077,48.16853301)(270.80169075,48.0785331)(270.82168701,47.97853811)
\lineto(270.88168701,47.67853811)
\curveto(270.88169067,47.64853353)(270.88669067,47.61353357)(270.89668701,47.57353811)
\lineto(270.89668701,47.46853811)
\curveto(270.93669062,47.31853386)(270.96169059,47.15353403)(270.97168701,46.97353811)
\curveto(270.97169058,46.80353438)(270.99169056,46.64353454)(271.03168701,46.49353811)
\curveto(271.0516905,46.41353477)(271.07169048,46.33853484)(271.09168701,46.26853811)
\curveto(271.10169045,46.20853497)(271.11669044,46.13853504)(271.13668701,46.05853811)
\curveto(271.18669037,45.89853528)(271.2516903,45.74853543)(271.33168701,45.60853811)
\curveto(271.40169015,45.46853571)(271.49169006,45.34853583)(271.60168701,45.24853811)
\curveto(271.71168984,45.14853603)(271.84668971,45.07353611)(272.00668701,45.02353811)
\curveto(272.1566894,44.97353621)(272.34168921,44.95353623)(272.56168701,44.96353811)
\curveto(272.66168889,44.96353622)(272.7566888,44.9785362)(272.84668701,45.00853811)
\curveto(272.92668863,45.04853613)(273.00168855,45.09353609)(273.07168701,45.14353811)
\curveto(273.18168837,45.22353596)(273.27668828,45.32853585)(273.35668701,45.45853811)
\curveto(273.42668813,45.58853559)(273.48668807,45.72853545)(273.53668701,45.87853811)
\curveto(273.54668801,45.92853525)(273.551688,45.9785352)(273.55168701,46.02853811)
\curveto(273.551688,46.0785351)(273.556688,46.12853505)(273.56668701,46.17853811)
\curveto(273.58668797,46.24853493)(273.60168795,46.33353485)(273.61168701,46.43353811)
\curveto(273.61168794,46.54353464)(273.60168795,46.63353455)(273.58168701,46.70353811)
\curveto(273.56168799,46.76353442)(273.556688,46.82353436)(273.56668701,46.88353811)
\curveto(273.56668799,46.94353424)(273.556688,47.00353418)(273.53668701,47.06353811)
\curveto(273.51668804,47.14353404)(273.50168805,47.21853396)(273.49168701,47.28853811)
\curveto(273.48168807,47.36853381)(273.46168809,47.44353374)(273.43168701,47.51353811)
\curveto(273.31168824,47.80353338)(273.16668839,48.04853313)(272.99668701,48.24853811)
\curveto(272.82668873,48.45853272)(272.59668896,48.61853256)(272.30668701,48.72853811)
}
}
{
\newrgbcolor{curcolor}{0 0 0}
\pscustom[linestyle=none,fillstyle=solid,fillcolor=curcolor]
{
\newpath
\moveto(266.62168701,55.69017873)
\curveto(266.62169493,55.92017394)(266.68169487,56.05017381)(266.80168701,56.08017873)
\curveto(266.91169464,56.11017375)(267.07669448,56.12517374)(267.29668701,56.12517873)
\lineto(267.58168701,56.12517873)
\curveto(267.67169388,56.12517374)(267.74669381,56.10017376)(267.80668701,56.05017873)
\curveto(267.88669367,55.99017387)(267.93169362,55.90517396)(267.94168701,55.79517873)
\curveto(267.94169361,55.68517418)(267.9566936,55.57517429)(267.98668701,55.46517873)
\curveto(268.01669354,55.32517454)(268.04669351,55.19017467)(268.07668701,55.06017873)
\curveto(268.10669345,54.94017492)(268.14669341,54.82517504)(268.19668701,54.71517873)
\curveto(268.32669323,54.42517544)(268.50669305,54.19017567)(268.73668701,54.01017873)
\curveto(268.9566926,53.83017603)(269.21169234,53.67517619)(269.50168701,53.54517873)
\curveto(269.61169194,53.50517636)(269.72669183,53.47517639)(269.84668701,53.45517873)
\curveto(269.9566916,53.43517643)(270.07169148,53.41017645)(270.19168701,53.38017873)
\curveto(270.24169131,53.37017649)(270.29169126,53.3651765)(270.34168701,53.36517873)
\curveto(270.39169116,53.37517649)(270.44169111,53.37517649)(270.49168701,53.36517873)
\curveto(270.61169094,53.33517653)(270.7516908,53.32017654)(270.91168701,53.32017873)
\curveto(271.06169049,53.33017653)(271.20669035,53.33517653)(271.34668701,53.33517873)
\lineto(273.19168701,53.33517873)
\lineto(273.53668701,53.33517873)
\curveto(273.6566879,53.33517653)(273.77168778,53.33017653)(273.88168701,53.32017873)
\curveto(273.99168756,53.31017655)(274.08668747,53.30517656)(274.16668701,53.30517873)
\curveto(274.24668731,53.31517655)(274.31668724,53.29517657)(274.37668701,53.24517873)
\curveto(274.44668711,53.19517667)(274.48668707,53.11517675)(274.49668701,53.00517873)
\curveto(274.50668705,52.90517696)(274.51168704,52.79517707)(274.51168701,52.67517873)
\lineto(274.51168701,52.40517873)
\curveto(274.49168706,52.35517751)(274.47668708,52.30517756)(274.46668701,52.25517873)
\curveto(274.44668711,52.21517765)(274.42168713,52.18517768)(274.39168701,52.16517873)
\curveto(274.32168723,52.11517775)(274.23668732,52.08517778)(274.13668701,52.07517873)
\lineto(273.80668701,52.07517873)
\lineto(272.65168701,52.07517873)
\lineto(268.49668701,52.07517873)
\lineto(267.46168701,52.07517873)
\lineto(267.16168701,52.07517873)
\curveto(267.06169449,52.08517778)(266.97669458,52.11517775)(266.90668701,52.16517873)
\curveto(266.86669469,52.19517767)(266.83669472,52.24517762)(266.81668701,52.31517873)
\curveto(266.79669476,52.39517747)(266.78669477,52.48017738)(266.78668701,52.57017873)
\curveto(266.77669478,52.6601772)(266.77669478,52.75017711)(266.78668701,52.84017873)
\curveto(266.79669476,52.93017693)(266.81169474,53.00017686)(266.83168701,53.05017873)
\curveto(266.86169469,53.13017673)(266.92169463,53.18017668)(267.01168701,53.20017873)
\curveto(267.09169446,53.23017663)(267.18169437,53.24517662)(267.28168701,53.24517873)
\lineto(267.58168701,53.24517873)
\curveto(267.68169387,53.24517662)(267.77169378,53.2651766)(267.85168701,53.30517873)
\curveto(267.87169368,53.31517655)(267.88669367,53.32517654)(267.89668701,53.33517873)
\lineto(267.94168701,53.38017873)
\curveto(267.94169361,53.49017637)(267.89669366,53.58017628)(267.80668701,53.65017873)
\curveto(267.70669385,53.72017614)(267.62669393,53.78017608)(267.56668701,53.83017873)
\lineto(267.47668701,53.92017873)
\curveto(267.36669419,54.01017585)(267.2516943,54.13517573)(267.13168701,54.29517873)
\curveto(267.01169454,54.45517541)(266.92169463,54.60517526)(266.86168701,54.74517873)
\curveto(266.81169474,54.83517503)(266.77669478,54.93017493)(266.75668701,55.03017873)
\curveto(266.72669483,55.13017473)(266.69669486,55.23517463)(266.66668701,55.34517873)
\curveto(266.6566949,55.40517446)(266.6516949,55.4651744)(266.65168701,55.52517873)
\curveto(266.64169491,55.58517428)(266.63169492,55.64017422)(266.62168701,55.69017873)
}
}
{
\newrgbcolor{curcolor}{0 0 0}
\pscustom[linestyle=none,fillstyle=solid,fillcolor=curcolor]
{
}
}
{
\newrgbcolor{curcolor}{0 0 0}
\pscustom[linestyle=none,fillstyle=solid,fillcolor=curcolor]
{
\newpath
\moveto(263.92168701,64.24510061)
\curveto(263.91169764,64.93509597)(264.03169752,65.53509537)(264.28168701,66.04510061)
\curveto(264.53169702,66.56509434)(264.86669669,66.96009395)(265.28668701,67.23010061)
\curveto(265.36669619,67.28009363)(265.4566961,67.32509358)(265.55668701,67.36510061)
\curveto(265.64669591,67.4050935)(265.74169581,67.45009346)(265.84168701,67.50010061)
\curveto(265.94169561,67.54009337)(266.04169551,67.57009334)(266.14168701,67.59010061)
\curveto(266.24169531,67.6100933)(266.34669521,67.63009328)(266.45668701,67.65010061)
\curveto(266.50669505,67.67009324)(266.551695,67.67509323)(266.59168701,67.66510061)
\curveto(266.63169492,67.65509325)(266.67669488,67.66009325)(266.72668701,67.68010061)
\curveto(266.77669478,67.69009322)(266.86169469,67.69509321)(266.98168701,67.69510061)
\curveto(267.09169446,67.69509321)(267.17669438,67.69009322)(267.23668701,67.68010061)
\curveto(267.29669426,67.66009325)(267.3566942,67.65009326)(267.41668701,67.65010061)
\curveto(267.47669408,67.66009325)(267.53669402,67.65509325)(267.59668701,67.63510061)
\curveto(267.73669382,67.59509331)(267.87169368,67.56009335)(268.00168701,67.53010061)
\curveto(268.13169342,67.50009341)(268.2566933,67.46009345)(268.37668701,67.41010061)
\curveto(268.51669304,67.35009356)(268.64169291,67.28009363)(268.75168701,67.20010061)
\curveto(268.86169269,67.13009378)(268.97169258,67.05509385)(269.08168701,66.97510061)
\lineto(269.14168701,66.91510061)
\curveto(269.16169239,66.905094)(269.18169237,66.89009402)(269.20168701,66.87010061)
\curveto(269.36169219,66.75009416)(269.50669205,66.61509429)(269.63668701,66.46510061)
\curveto(269.76669179,66.31509459)(269.89169166,66.15509475)(270.01168701,65.98510061)
\curveto(270.23169132,65.67509523)(270.43669112,65.38009553)(270.62668701,65.10010061)
\curveto(270.76669079,64.87009604)(270.90169065,64.64009627)(271.03168701,64.41010061)
\curveto(271.16169039,64.19009672)(271.29669026,63.97009694)(271.43668701,63.75010061)
\curveto(271.60668995,63.50009741)(271.78668977,63.26009765)(271.97668701,63.03010061)
\curveto(272.16668939,62.8100981)(272.39168916,62.62009829)(272.65168701,62.46010061)
\curveto(272.71168884,62.42009849)(272.77168878,62.38509852)(272.83168701,62.35510061)
\curveto(272.88168867,62.32509858)(272.94668861,62.29509861)(273.02668701,62.26510061)
\curveto(273.09668846,62.24509866)(273.1566884,62.24009867)(273.20668701,62.25010061)
\curveto(273.27668828,62.27009864)(273.33168822,62.3050986)(273.37168701,62.35510061)
\curveto(273.40168815,62.4050985)(273.42168813,62.46509844)(273.43168701,62.53510061)
\lineto(273.43168701,62.77510061)
\lineto(273.43168701,63.52510061)
\lineto(273.43168701,66.33010061)
\lineto(273.43168701,66.99010061)
\curveto(273.43168812,67.08009383)(273.43668812,67.16509374)(273.44668701,67.24510061)
\curveto(273.44668811,67.32509358)(273.46668809,67.39009352)(273.50668701,67.44010061)
\curveto(273.54668801,67.49009342)(273.62168793,67.53009338)(273.73168701,67.56010061)
\curveto(273.83168772,67.60009331)(273.93168762,67.6100933)(274.03168701,67.59010061)
\lineto(274.16668701,67.59010061)
\curveto(274.23668732,67.57009334)(274.29668726,67.55009336)(274.34668701,67.53010061)
\curveto(274.39668716,67.5100934)(274.43668712,67.47509343)(274.46668701,67.42510061)
\curveto(274.50668705,67.37509353)(274.52668703,67.3050936)(274.52668701,67.21510061)
\lineto(274.52668701,66.94510061)
\lineto(274.52668701,66.04510061)
\lineto(274.52668701,62.53510061)
\lineto(274.52668701,61.47010061)
\curveto(274.52668703,61.39009952)(274.53168702,61.30009961)(274.54168701,61.20010061)
\curveto(274.54168701,61.10009981)(274.53168702,61.01509989)(274.51168701,60.94510061)
\curveto(274.44168711,60.73510017)(274.26168729,60.67010024)(273.97168701,60.75010061)
\curveto(273.93168762,60.76010015)(273.89668766,60.76010015)(273.86668701,60.75010061)
\curveto(273.82668773,60.75010016)(273.78168777,60.76010015)(273.73168701,60.78010061)
\curveto(273.6516879,60.80010011)(273.56668799,60.82010009)(273.47668701,60.84010061)
\curveto(273.38668817,60.86010005)(273.30168825,60.88510002)(273.22168701,60.91510061)
\curveto(272.73168882,61.07509983)(272.31668924,61.27509963)(271.97668701,61.51510061)
\curveto(271.72668983,61.69509921)(271.50169005,61.90009901)(271.30168701,62.13010061)
\curveto(271.09169046,62.36009855)(270.89669066,62.60009831)(270.71668701,62.85010061)
\curveto(270.53669102,63.1100978)(270.36669119,63.37509753)(270.20668701,63.64510061)
\curveto(270.03669152,63.92509698)(269.86169169,64.19509671)(269.68168701,64.45510061)
\curveto(269.60169195,64.56509634)(269.52669203,64.67009624)(269.45668701,64.77010061)
\curveto(269.38669217,64.88009603)(269.31169224,64.99009592)(269.23168701,65.10010061)
\curveto(269.20169235,65.14009577)(269.17169238,65.17509573)(269.14168701,65.20510061)
\curveto(269.10169245,65.24509566)(269.07169248,65.28509562)(269.05168701,65.32510061)
\curveto(268.94169261,65.46509544)(268.81669274,65.59009532)(268.67668701,65.70010061)
\curveto(268.64669291,65.72009519)(268.62169293,65.74509516)(268.60168701,65.77510061)
\curveto(268.57169298,65.8050951)(268.54169301,65.83009508)(268.51168701,65.85010061)
\curveto(268.41169314,65.93009498)(268.31169324,65.99509491)(268.21168701,66.04510061)
\curveto(268.11169344,66.1050948)(268.00169355,66.16009475)(267.88168701,66.21010061)
\curveto(267.81169374,66.24009467)(267.73669382,66.26009465)(267.65668701,66.27010061)
\lineto(267.41668701,66.33010061)
\lineto(267.32668701,66.33010061)
\curveto(267.29669426,66.34009457)(267.26669429,66.34509456)(267.23668701,66.34510061)
\curveto(267.16669439,66.36509454)(267.07169448,66.37009454)(266.95168701,66.36010061)
\curveto(266.82169473,66.36009455)(266.72169483,66.35009456)(266.65168701,66.33010061)
\curveto(266.57169498,66.3100946)(266.49669506,66.29009462)(266.42668701,66.27010061)
\curveto(266.34669521,66.26009465)(266.26669529,66.24009467)(266.18668701,66.21010061)
\curveto(265.94669561,66.10009481)(265.74669581,65.95009496)(265.58668701,65.76010061)
\curveto(265.41669614,65.58009533)(265.27669628,65.36009555)(265.16668701,65.10010061)
\curveto(265.14669641,65.03009588)(265.13169642,64.96009595)(265.12168701,64.89010061)
\curveto(265.10169645,64.82009609)(265.08169647,64.74509616)(265.06168701,64.66510061)
\curveto(265.04169651,64.58509632)(265.03169652,64.47509643)(265.03168701,64.33510061)
\curveto(265.03169652,64.2050967)(265.04169651,64.10009681)(265.06168701,64.02010061)
\curveto(265.07169648,63.96009695)(265.07669648,63.905097)(265.07668701,63.85510061)
\curveto(265.07669648,63.8050971)(265.08669647,63.75509715)(265.10668701,63.70510061)
\curveto(265.14669641,63.6050973)(265.18669637,63.5100974)(265.22668701,63.42010061)
\curveto(265.26669629,63.34009757)(265.31169624,63.26009765)(265.36168701,63.18010061)
\curveto(265.38169617,63.15009776)(265.40669615,63.12009779)(265.43668701,63.09010061)
\curveto(265.46669609,63.07009784)(265.49169606,63.04509786)(265.51168701,63.01510061)
\lineto(265.58668701,62.94010061)
\curveto(265.60669595,62.910098)(265.62669593,62.88509802)(265.64668701,62.86510061)
\lineto(265.85668701,62.71510061)
\curveto(265.91669564,62.67509823)(265.98169557,62.63009828)(266.05168701,62.58010061)
\curveto(266.14169541,62.52009839)(266.24669531,62.47009844)(266.36668701,62.43010061)
\curveto(266.47669508,62.40009851)(266.58669497,62.36509854)(266.69668701,62.32510061)
\curveto(266.80669475,62.28509862)(266.9516946,62.26009865)(267.13168701,62.25010061)
\curveto(267.30169425,62.24009867)(267.42669413,62.2100987)(267.50668701,62.16010061)
\curveto(267.58669397,62.1100988)(267.63169392,62.03509887)(267.64168701,61.93510061)
\curveto(267.6516939,61.83509907)(267.6566939,61.72509918)(267.65668701,61.60510061)
\curveto(267.6566939,61.56509934)(267.66169389,61.52509938)(267.67168701,61.48510061)
\curveto(267.67169388,61.44509946)(267.66669389,61.4100995)(267.65668701,61.38010061)
\curveto(267.63669392,61.33009958)(267.62669393,61.28009963)(267.62668701,61.23010061)
\curveto(267.62669393,61.19009972)(267.61669394,61.15009976)(267.59668701,61.11010061)
\curveto(267.53669402,61.02009989)(267.40169415,60.97509993)(267.19168701,60.97510061)
\lineto(267.07168701,60.97510061)
\curveto(267.01169454,60.98509992)(266.9516946,60.99009992)(266.89168701,60.99010061)
\curveto(266.82169473,61.00009991)(266.7566948,61.0100999)(266.69668701,61.02010061)
\curveto(266.58669497,61.04009987)(266.48669507,61.06009985)(266.39668701,61.08010061)
\curveto(266.29669526,61.10009981)(266.20169535,61.13009978)(266.11168701,61.17010061)
\curveto(266.04169551,61.19009972)(265.98169557,61.2100997)(265.93168701,61.23010061)
\lineto(265.75168701,61.29010061)
\curveto(265.49169606,61.4100995)(265.24669631,61.56509934)(265.01668701,61.75510061)
\curveto(264.78669677,61.95509895)(264.60169695,62.17009874)(264.46168701,62.40010061)
\curveto(264.38169717,62.5100984)(264.31669724,62.62509828)(264.26668701,62.74510061)
\lineto(264.11668701,63.13510061)
\curveto(264.06669749,63.24509766)(264.03669752,63.36009755)(264.02668701,63.48010061)
\curveto(264.00669755,63.60009731)(263.98169757,63.72509718)(263.95168701,63.85510061)
\curveto(263.9516976,63.92509698)(263.9516976,63.99009692)(263.95168701,64.05010061)
\curveto(263.94169761,64.1100968)(263.93169762,64.17509673)(263.92168701,64.24510061)
}
}
{
\newrgbcolor{curcolor}{0 0 0}
\pscustom[linestyle=none,fillstyle=solid,fillcolor=curcolor]
{
\newpath
\moveto(263.92168701,72.59470998)
\curveto(263.91169764,73.28470535)(264.03169752,73.88470475)(264.28168701,74.39470998)
\curveto(264.53169702,74.91470372)(264.86669669,75.30970332)(265.28668701,75.57970998)
\curveto(265.36669619,75.629703)(265.4566961,75.67470296)(265.55668701,75.71470998)
\curveto(265.64669591,75.75470288)(265.74169581,75.79970283)(265.84168701,75.84970998)
\curveto(265.94169561,75.88970274)(266.04169551,75.91970271)(266.14168701,75.93970998)
\curveto(266.24169531,75.95970267)(266.34669521,75.97970265)(266.45668701,75.99970998)
\curveto(266.50669505,76.01970261)(266.551695,76.02470261)(266.59168701,76.01470998)
\curveto(266.63169492,76.00470263)(266.67669488,76.00970262)(266.72668701,76.02970998)
\curveto(266.77669478,76.03970259)(266.86169469,76.04470259)(266.98168701,76.04470998)
\curveto(267.09169446,76.04470259)(267.17669438,76.03970259)(267.23668701,76.02970998)
\curveto(267.29669426,76.00970262)(267.3566942,75.99970263)(267.41668701,75.99970998)
\curveto(267.47669408,76.00970262)(267.53669402,76.00470263)(267.59668701,75.98470998)
\curveto(267.73669382,75.94470269)(267.87169368,75.90970272)(268.00168701,75.87970998)
\curveto(268.13169342,75.84970278)(268.2566933,75.80970282)(268.37668701,75.75970998)
\curveto(268.51669304,75.69970293)(268.64169291,75.629703)(268.75168701,75.54970998)
\curveto(268.86169269,75.47970315)(268.97169258,75.40470323)(269.08168701,75.32470998)
\lineto(269.14168701,75.26470998)
\curveto(269.16169239,75.25470338)(269.18169237,75.23970339)(269.20168701,75.21970998)
\curveto(269.36169219,75.09970353)(269.50669205,74.96470367)(269.63668701,74.81470998)
\curveto(269.76669179,74.66470397)(269.89169166,74.50470413)(270.01168701,74.33470998)
\curveto(270.23169132,74.02470461)(270.43669112,73.7297049)(270.62668701,73.44970998)
\curveto(270.76669079,73.21970541)(270.90169065,72.98970564)(271.03168701,72.75970998)
\curveto(271.16169039,72.53970609)(271.29669026,72.31970631)(271.43668701,72.09970998)
\curveto(271.60668995,71.84970678)(271.78668977,71.60970702)(271.97668701,71.37970998)
\curveto(272.16668939,71.15970747)(272.39168916,70.96970766)(272.65168701,70.80970998)
\curveto(272.71168884,70.76970786)(272.77168878,70.7347079)(272.83168701,70.70470998)
\curveto(272.88168867,70.67470796)(272.94668861,70.64470799)(273.02668701,70.61470998)
\curveto(273.09668846,70.59470804)(273.1566884,70.58970804)(273.20668701,70.59970998)
\curveto(273.27668828,70.61970801)(273.33168822,70.65470798)(273.37168701,70.70470998)
\curveto(273.40168815,70.75470788)(273.42168813,70.81470782)(273.43168701,70.88470998)
\lineto(273.43168701,71.12470998)
\lineto(273.43168701,71.87470998)
\lineto(273.43168701,74.67970998)
\lineto(273.43168701,75.33970998)
\curveto(273.43168812,75.4297032)(273.43668812,75.51470312)(273.44668701,75.59470998)
\curveto(273.44668811,75.67470296)(273.46668809,75.73970289)(273.50668701,75.78970998)
\curveto(273.54668801,75.83970279)(273.62168793,75.87970275)(273.73168701,75.90970998)
\curveto(273.83168772,75.94970268)(273.93168762,75.95970267)(274.03168701,75.93970998)
\lineto(274.16668701,75.93970998)
\curveto(274.23668732,75.91970271)(274.29668726,75.89970273)(274.34668701,75.87970998)
\curveto(274.39668716,75.85970277)(274.43668712,75.82470281)(274.46668701,75.77470998)
\curveto(274.50668705,75.72470291)(274.52668703,75.65470298)(274.52668701,75.56470998)
\lineto(274.52668701,75.29470998)
\lineto(274.52668701,74.39470998)
\lineto(274.52668701,70.88470998)
\lineto(274.52668701,69.81970998)
\curveto(274.52668703,69.73970889)(274.53168702,69.64970898)(274.54168701,69.54970998)
\curveto(274.54168701,69.44970918)(274.53168702,69.36470927)(274.51168701,69.29470998)
\curveto(274.44168711,69.08470955)(274.26168729,69.01970961)(273.97168701,69.09970998)
\curveto(273.93168762,69.10970952)(273.89668766,69.10970952)(273.86668701,69.09970998)
\curveto(273.82668773,69.09970953)(273.78168777,69.10970952)(273.73168701,69.12970998)
\curveto(273.6516879,69.14970948)(273.56668799,69.16970946)(273.47668701,69.18970998)
\curveto(273.38668817,69.20970942)(273.30168825,69.2347094)(273.22168701,69.26470998)
\curveto(272.73168882,69.42470921)(272.31668924,69.62470901)(271.97668701,69.86470998)
\curveto(271.72668983,70.04470859)(271.50169005,70.24970838)(271.30168701,70.47970998)
\curveto(271.09169046,70.70970792)(270.89669066,70.94970768)(270.71668701,71.19970998)
\curveto(270.53669102,71.45970717)(270.36669119,71.72470691)(270.20668701,71.99470998)
\curveto(270.03669152,72.27470636)(269.86169169,72.54470609)(269.68168701,72.80470998)
\curveto(269.60169195,72.91470572)(269.52669203,73.01970561)(269.45668701,73.11970998)
\curveto(269.38669217,73.2297054)(269.31169224,73.33970529)(269.23168701,73.44970998)
\curveto(269.20169235,73.48970514)(269.17169238,73.52470511)(269.14168701,73.55470998)
\curveto(269.10169245,73.59470504)(269.07169248,73.634705)(269.05168701,73.67470998)
\curveto(268.94169261,73.81470482)(268.81669274,73.93970469)(268.67668701,74.04970998)
\curveto(268.64669291,74.06970456)(268.62169293,74.09470454)(268.60168701,74.12470998)
\curveto(268.57169298,74.15470448)(268.54169301,74.17970445)(268.51168701,74.19970998)
\curveto(268.41169314,74.27970435)(268.31169324,74.34470429)(268.21168701,74.39470998)
\curveto(268.11169344,74.45470418)(268.00169355,74.50970412)(267.88168701,74.55970998)
\curveto(267.81169374,74.58970404)(267.73669382,74.60970402)(267.65668701,74.61970998)
\lineto(267.41668701,74.67970998)
\lineto(267.32668701,74.67970998)
\curveto(267.29669426,74.68970394)(267.26669429,74.69470394)(267.23668701,74.69470998)
\curveto(267.16669439,74.71470392)(267.07169448,74.71970391)(266.95168701,74.70970998)
\curveto(266.82169473,74.70970392)(266.72169483,74.69970393)(266.65168701,74.67970998)
\curveto(266.57169498,74.65970397)(266.49669506,74.63970399)(266.42668701,74.61970998)
\curveto(266.34669521,74.60970402)(266.26669529,74.58970404)(266.18668701,74.55970998)
\curveto(265.94669561,74.44970418)(265.74669581,74.29970433)(265.58668701,74.10970998)
\curveto(265.41669614,73.9297047)(265.27669628,73.70970492)(265.16668701,73.44970998)
\curveto(265.14669641,73.37970525)(265.13169642,73.30970532)(265.12168701,73.23970998)
\curveto(265.10169645,73.16970546)(265.08169647,73.09470554)(265.06168701,73.01470998)
\curveto(265.04169651,72.9347057)(265.03169652,72.82470581)(265.03168701,72.68470998)
\curveto(265.03169652,72.55470608)(265.04169651,72.44970618)(265.06168701,72.36970998)
\curveto(265.07169648,72.30970632)(265.07669648,72.25470638)(265.07668701,72.20470998)
\curveto(265.07669648,72.15470648)(265.08669647,72.10470653)(265.10668701,72.05470998)
\curveto(265.14669641,71.95470668)(265.18669637,71.85970677)(265.22668701,71.76970998)
\curveto(265.26669629,71.68970694)(265.31169624,71.60970702)(265.36168701,71.52970998)
\curveto(265.38169617,71.49970713)(265.40669615,71.46970716)(265.43668701,71.43970998)
\curveto(265.46669609,71.41970721)(265.49169606,71.39470724)(265.51168701,71.36470998)
\lineto(265.58668701,71.28970998)
\curveto(265.60669595,71.25970737)(265.62669593,71.2347074)(265.64668701,71.21470998)
\lineto(265.85668701,71.06470998)
\curveto(265.91669564,71.02470761)(265.98169557,70.97970765)(266.05168701,70.92970998)
\curveto(266.14169541,70.86970776)(266.24669531,70.81970781)(266.36668701,70.77970998)
\curveto(266.47669508,70.74970788)(266.58669497,70.71470792)(266.69668701,70.67470998)
\curveto(266.80669475,70.634708)(266.9516946,70.60970802)(267.13168701,70.59970998)
\curveto(267.30169425,70.58970804)(267.42669413,70.55970807)(267.50668701,70.50970998)
\curveto(267.58669397,70.45970817)(267.63169392,70.38470825)(267.64168701,70.28470998)
\curveto(267.6516939,70.18470845)(267.6566939,70.07470856)(267.65668701,69.95470998)
\curveto(267.6566939,69.91470872)(267.66169389,69.87470876)(267.67168701,69.83470998)
\curveto(267.67169388,69.79470884)(267.66669389,69.75970887)(267.65668701,69.72970998)
\curveto(267.63669392,69.67970895)(267.62669393,69.629709)(267.62668701,69.57970998)
\curveto(267.62669393,69.53970909)(267.61669394,69.49970913)(267.59668701,69.45970998)
\curveto(267.53669402,69.36970926)(267.40169415,69.32470931)(267.19168701,69.32470998)
\lineto(267.07168701,69.32470998)
\curveto(267.01169454,69.3347093)(266.9516946,69.33970929)(266.89168701,69.33970998)
\curveto(266.82169473,69.34970928)(266.7566948,69.35970927)(266.69668701,69.36970998)
\curveto(266.58669497,69.38970924)(266.48669507,69.40970922)(266.39668701,69.42970998)
\curveto(266.29669526,69.44970918)(266.20169535,69.47970915)(266.11168701,69.51970998)
\curveto(266.04169551,69.53970909)(265.98169557,69.55970907)(265.93168701,69.57970998)
\lineto(265.75168701,69.63970998)
\curveto(265.49169606,69.75970887)(265.24669631,69.91470872)(265.01668701,70.10470998)
\curveto(264.78669677,70.30470833)(264.60169695,70.51970811)(264.46168701,70.74970998)
\curveto(264.38169717,70.85970777)(264.31669724,70.97470766)(264.26668701,71.09470998)
\lineto(264.11668701,71.48470998)
\curveto(264.06669749,71.59470704)(264.03669752,71.70970692)(264.02668701,71.82970998)
\curveto(264.00669755,71.94970668)(263.98169757,72.07470656)(263.95168701,72.20470998)
\curveto(263.9516976,72.27470636)(263.9516976,72.33970629)(263.95168701,72.39970998)
\curveto(263.94169761,72.45970617)(263.93169762,72.52470611)(263.92168701,72.59470998)
}
}
{
\newrgbcolor{curcolor}{0 0 0}
\pscustom[linestyle=none,fillstyle=solid,fillcolor=curcolor]
{
\newpath
\moveto(272.89168701,78.63431936)
\lineto(272.89168701,79.26431936)
\lineto(272.89168701,79.45931936)
\curveto(272.89168866,79.52931683)(272.90168865,79.58931677)(272.92168701,79.63931936)
\curveto(272.96168859,79.70931665)(273.00168855,79.7593166)(273.04168701,79.78931936)
\curveto(273.09168846,79.82931653)(273.1566884,79.84931651)(273.23668701,79.84931936)
\curveto(273.31668824,79.8593165)(273.40168815,79.86431649)(273.49168701,79.86431936)
\lineto(274.21168701,79.86431936)
\curveto(274.69168686,79.86431649)(275.10168645,79.80431655)(275.44168701,79.68431936)
\curveto(275.78168577,79.56431679)(276.0566855,79.36931699)(276.26668701,79.09931936)
\curveto(276.31668524,79.02931733)(276.36168519,78.9593174)(276.40168701,78.88931936)
\curveto(276.4516851,78.82931753)(276.49668506,78.7543176)(276.53668701,78.66431936)
\curveto(276.54668501,78.64431771)(276.556685,78.61431774)(276.56668701,78.57431936)
\curveto(276.58668497,78.53431782)(276.59168496,78.48931787)(276.58168701,78.43931936)
\curveto(276.551685,78.34931801)(276.47668508,78.29431806)(276.35668701,78.27431936)
\curveto(276.24668531,78.2543181)(276.1516854,78.26931809)(276.07168701,78.31931936)
\curveto(276.00168555,78.34931801)(275.93668562,78.39431796)(275.87668701,78.45431936)
\curveto(275.82668573,78.52431783)(275.77668578,78.58931777)(275.72668701,78.64931936)
\curveto(275.67668588,78.71931764)(275.60168595,78.77931758)(275.50168701,78.82931936)
\curveto(275.41168614,78.88931747)(275.32168623,78.93931742)(275.23168701,78.97931936)
\curveto(275.20168635,78.99931736)(275.14168641,79.02431733)(275.05168701,79.05431936)
\curveto(274.97168658,79.08431727)(274.90168665,79.08931727)(274.84168701,79.06931936)
\curveto(274.70168685,79.03931732)(274.61168694,78.97931738)(274.57168701,78.88931936)
\curveto(274.54168701,78.80931755)(274.52668703,78.71931764)(274.52668701,78.61931936)
\curveto(274.52668703,78.51931784)(274.50168705,78.43431792)(274.45168701,78.36431936)
\curveto(274.38168717,78.27431808)(274.24168731,78.22931813)(274.03168701,78.22931936)
\lineto(273.47668701,78.22931936)
\lineto(273.25168701,78.22931936)
\curveto(273.17168838,78.23931812)(273.10668845,78.2593181)(273.05668701,78.28931936)
\curveto(272.97668858,78.34931801)(272.93168862,78.41931794)(272.92168701,78.49931936)
\curveto(272.91168864,78.51931784)(272.90668865,78.53931782)(272.90668701,78.55931936)
\curveto(272.90668865,78.58931777)(272.90168865,78.61431774)(272.89168701,78.63431936)
}
}
{
\newrgbcolor{curcolor}{0 0 0}
\pscustom[linestyle=none,fillstyle=solid,fillcolor=curcolor]
{
}
}
{
\newrgbcolor{curcolor}{0 0 0}
\pscustom[linestyle=none,fillstyle=solid,fillcolor=curcolor]
{
\newpath
\moveto(263.92168701,89.26463186)
\curveto(263.91169764,89.95462722)(264.03169752,90.55462662)(264.28168701,91.06463186)
\curveto(264.53169702,91.58462559)(264.86669669,91.9796252)(265.28668701,92.24963186)
\curveto(265.36669619,92.29962488)(265.4566961,92.34462483)(265.55668701,92.38463186)
\curveto(265.64669591,92.42462475)(265.74169581,92.46962471)(265.84168701,92.51963186)
\curveto(265.94169561,92.55962462)(266.04169551,92.58962459)(266.14168701,92.60963186)
\curveto(266.24169531,92.62962455)(266.34669521,92.64962453)(266.45668701,92.66963186)
\curveto(266.50669505,92.68962449)(266.551695,92.69462448)(266.59168701,92.68463186)
\curveto(266.63169492,92.6746245)(266.67669488,92.6796245)(266.72668701,92.69963186)
\curveto(266.77669478,92.70962447)(266.86169469,92.71462446)(266.98168701,92.71463186)
\curveto(267.09169446,92.71462446)(267.17669438,92.70962447)(267.23668701,92.69963186)
\curveto(267.29669426,92.6796245)(267.3566942,92.66962451)(267.41668701,92.66963186)
\curveto(267.47669408,92.6796245)(267.53669402,92.6746245)(267.59668701,92.65463186)
\curveto(267.73669382,92.61462456)(267.87169368,92.5796246)(268.00168701,92.54963186)
\curveto(268.13169342,92.51962466)(268.2566933,92.4796247)(268.37668701,92.42963186)
\curveto(268.51669304,92.36962481)(268.64169291,92.29962488)(268.75168701,92.21963186)
\curveto(268.86169269,92.14962503)(268.97169258,92.0746251)(269.08168701,91.99463186)
\lineto(269.14168701,91.93463186)
\curveto(269.16169239,91.92462525)(269.18169237,91.90962527)(269.20168701,91.88963186)
\curveto(269.36169219,91.76962541)(269.50669205,91.63462554)(269.63668701,91.48463186)
\curveto(269.76669179,91.33462584)(269.89169166,91.174626)(270.01168701,91.00463186)
\curveto(270.23169132,90.69462648)(270.43669112,90.39962678)(270.62668701,90.11963186)
\curveto(270.76669079,89.88962729)(270.90169065,89.65962752)(271.03168701,89.42963186)
\curveto(271.16169039,89.20962797)(271.29669026,88.98962819)(271.43668701,88.76963186)
\curveto(271.60668995,88.51962866)(271.78668977,88.2796289)(271.97668701,88.04963186)
\curveto(272.16668939,87.82962935)(272.39168916,87.63962954)(272.65168701,87.47963186)
\curveto(272.71168884,87.43962974)(272.77168878,87.40462977)(272.83168701,87.37463186)
\curveto(272.88168867,87.34462983)(272.94668861,87.31462986)(273.02668701,87.28463186)
\curveto(273.09668846,87.26462991)(273.1566884,87.25962992)(273.20668701,87.26963186)
\curveto(273.27668828,87.28962989)(273.33168822,87.32462985)(273.37168701,87.37463186)
\curveto(273.40168815,87.42462975)(273.42168813,87.48462969)(273.43168701,87.55463186)
\lineto(273.43168701,87.79463186)
\lineto(273.43168701,88.54463186)
\lineto(273.43168701,91.34963186)
\lineto(273.43168701,92.00963186)
\curveto(273.43168812,92.09962508)(273.43668812,92.18462499)(273.44668701,92.26463186)
\curveto(273.44668811,92.34462483)(273.46668809,92.40962477)(273.50668701,92.45963186)
\curveto(273.54668801,92.50962467)(273.62168793,92.54962463)(273.73168701,92.57963186)
\curveto(273.83168772,92.61962456)(273.93168762,92.62962455)(274.03168701,92.60963186)
\lineto(274.16668701,92.60963186)
\curveto(274.23668732,92.58962459)(274.29668726,92.56962461)(274.34668701,92.54963186)
\curveto(274.39668716,92.52962465)(274.43668712,92.49462468)(274.46668701,92.44463186)
\curveto(274.50668705,92.39462478)(274.52668703,92.32462485)(274.52668701,92.23463186)
\lineto(274.52668701,91.96463186)
\lineto(274.52668701,91.06463186)
\lineto(274.52668701,87.55463186)
\lineto(274.52668701,86.48963186)
\curveto(274.52668703,86.40963077)(274.53168702,86.31963086)(274.54168701,86.21963186)
\curveto(274.54168701,86.11963106)(274.53168702,86.03463114)(274.51168701,85.96463186)
\curveto(274.44168711,85.75463142)(274.26168729,85.68963149)(273.97168701,85.76963186)
\curveto(273.93168762,85.7796314)(273.89668766,85.7796314)(273.86668701,85.76963186)
\curveto(273.82668773,85.76963141)(273.78168777,85.7796314)(273.73168701,85.79963186)
\curveto(273.6516879,85.81963136)(273.56668799,85.83963134)(273.47668701,85.85963186)
\curveto(273.38668817,85.8796313)(273.30168825,85.90463127)(273.22168701,85.93463186)
\curveto(272.73168882,86.09463108)(272.31668924,86.29463088)(271.97668701,86.53463186)
\curveto(271.72668983,86.71463046)(271.50169005,86.91963026)(271.30168701,87.14963186)
\curveto(271.09169046,87.3796298)(270.89669066,87.61962956)(270.71668701,87.86963186)
\curveto(270.53669102,88.12962905)(270.36669119,88.39462878)(270.20668701,88.66463186)
\curveto(270.03669152,88.94462823)(269.86169169,89.21462796)(269.68168701,89.47463186)
\curveto(269.60169195,89.58462759)(269.52669203,89.68962749)(269.45668701,89.78963186)
\curveto(269.38669217,89.89962728)(269.31169224,90.00962717)(269.23168701,90.11963186)
\curveto(269.20169235,90.15962702)(269.17169238,90.19462698)(269.14168701,90.22463186)
\curveto(269.10169245,90.26462691)(269.07169248,90.30462687)(269.05168701,90.34463186)
\curveto(268.94169261,90.48462669)(268.81669274,90.60962657)(268.67668701,90.71963186)
\curveto(268.64669291,90.73962644)(268.62169293,90.76462641)(268.60168701,90.79463186)
\curveto(268.57169298,90.82462635)(268.54169301,90.84962633)(268.51168701,90.86963186)
\curveto(268.41169314,90.94962623)(268.31169324,91.01462616)(268.21168701,91.06463186)
\curveto(268.11169344,91.12462605)(268.00169355,91.179626)(267.88168701,91.22963186)
\curveto(267.81169374,91.25962592)(267.73669382,91.2796259)(267.65668701,91.28963186)
\lineto(267.41668701,91.34963186)
\lineto(267.32668701,91.34963186)
\curveto(267.29669426,91.35962582)(267.26669429,91.36462581)(267.23668701,91.36463186)
\curveto(267.16669439,91.38462579)(267.07169448,91.38962579)(266.95168701,91.37963186)
\curveto(266.82169473,91.3796258)(266.72169483,91.36962581)(266.65168701,91.34963186)
\curveto(266.57169498,91.32962585)(266.49669506,91.30962587)(266.42668701,91.28963186)
\curveto(266.34669521,91.2796259)(266.26669529,91.25962592)(266.18668701,91.22963186)
\curveto(265.94669561,91.11962606)(265.74669581,90.96962621)(265.58668701,90.77963186)
\curveto(265.41669614,90.59962658)(265.27669628,90.3796268)(265.16668701,90.11963186)
\curveto(265.14669641,90.04962713)(265.13169642,89.9796272)(265.12168701,89.90963186)
\curveto(265.10169645,89.83962734)(265.08169647,89.76462741)(265.06168701,89.68463186)
\curveto(265.04169651,89.60462757)(265.03169652,89.49462768)(265.03168701,89.35463186)
\curveto(265.03169652,89.22462795)(265.04169651,89.11962806)(265.06168701,89.03963186)
\curveto(265.07169648,88.9796282)(265.07669648,88.92462825)(265.07668701,88.87463186)
\curveto(265.07669648,88.82462835)(265.08669647,88.7746284)(265.10668701,88.72463186)
\curveto(265.14669641,88.62462855)(265.18669637,88.52962865)(265.22668701,88.43963186)
\curveto(265.26669629,88.35962882)(265.31169624,88.2796289)(265.36168701,88.19963186)
\curveto(265.38169617,88.16962901)(265.40669615,88.13962904)(265.43668701,88.10963186)
\curveto(265.46669609,88.08962909)(265.49169606,88.06462911)(265.51168701,88.03463186)
\lineto(265.58668701,87.95963186)
\curveto(265.60669595,87.92962925)(265.62669593,87.90462927)(265.64668701,87.88463186)
\lineto(265.85668701,87.73463186)
\curveto(265.91669564,87.69462948)(265.98169557,87.64962953)(266.05168701,87.59963186)
\curveto(266.14169541,87.53962964)(266.24669531,87.48962969)(266.36668701,87.44963186)
\curveto(266.47669508,87.41962976)(266.58669497,87.38462979)(266.69668701,87.34463186)
\curveto(266.80669475,87.30462987)(266.9516946,87.2796299)(267.13168701,87.26963186)
\curveto(267.30169425,87.25962992)(267.42669413,87.22962995)(267.50668701,87.17963186)
\curveto(267.58669397,87.12963005)(267.63169392,87.05463012)(267.64168701,86.95463186)
\curveto(267.6516939,86.85463032)(267.6566939,86.74463043)(267.65668701,86.62463186)
\curveto(267.6566939,86.58463059)(267.66169389,86.54463063)(267.67168701,86.50463186)
\curveto(267.67169388,86.46463071)(267.66669389,86.42963075)(267.65668701,86.39963186)
\curveto(267.63669392,86.34963083)(267.62669393,86.29963088)(267.62668701,86.24963186)
\curveto(267.62669393,86.20963097)(267.61669394,86.16963101)(267.59668701,86.12963186)
\curveto(267.53669402,86.03963114)(267.40169415,85.99463118)(267.19168701,85.99463186)
\lineto(267.07168701,85.99463186)
\curveto(267.01169454,86.00463117)(266.9516946,86.00963117)(266.89168701,86.00963186)
\curveto(266.82169473,86.01963116)(266.7566948,86.02963115)(266.69668701,86.03963186)
\curveto(266.58669497,86.05963112)(266.48669507,86.0796311)(266.39668701,86.09963186)
\curveto(266.29669526,86.11963106)(266.20169535,86.14963103)(266.11168701,86.18963186)
\curveto(266.04169551,86.20963097)(265.98169557,86.22963095)(265.93168701,86.24963186)
\lineto(265.75168701,86.30963186)
\curveto(265.49169606,86.42963075)(265.24669631,86.58463059)(265.01668701,86.77463186)
\curveto(264.78669677,86.9746302)(264.60169695,87.18962999)(264.46168701,87.41963186)
\curveto(264.38169717,87.52962965)(264.31669724,87.64462953)(264.26668701,87.76463186)
\lineto(264.11668701,88.15463186)
\curveto(264.06669749,88.26462891)(264.03669752,88.3796288)(264.02668701,88.49963186)
\curveto(264.00669755,88.61962856)(263.98169757,88.74462843)(263.95168701,88.87463186)
\curveto(263.9516976,88.94462823)(263.9516976,89.00962817)(263.95168701,89.06963186)
\curveto(263.94169761,89.12962805)(263.93169762,89.19462798)(263.92168701,89.26463186)
}
}
{
\newrgbcolor{curcolor}{0 0 0}
\pscustom[linestyle=none,fillstyle=solid,fillcolor=curcolor]
{
\newpath
\moveto(269.44168701,101.36424123)
\lineto(269.69668701,101.36424123)
\curveto(269.77669178,101.37423353)(269.8516917,101.36923353)(269.92168701,101.34924123)
\lineto(270.16168701,101.34924123)
\lineto(270.32668701,101.34924123)
\curveto(270.42669113,101.32923357)(270.53169102,101.31923358)(270.64168701,101.31924123)
\curveto(270.74169081,101.31923358)(270.84169071,101.30923359)(270.94168701,101.28924123)
\lineto(271.09168701,101.28924123)
\curveto(271.23169032,101.25923364)(271.37169018,101.23923366)(271.51168701,101.22924123)
\curveto(271.64168991,101.21923368)(271.77168978,101.19423371)(271.90168701,101.15424123)
\curveto(271.98168957,101.13423377)(272.06668949,101.11423379)(272.15668701,101.09424123)
\lineto(272.39668701,101.03424123)
\lineto(272.69668701,100.91424123)
\curveto(272.78668877,100.88423402)(272.87668868,100.84923405)(272.96668701,100.80924123)
\curveto(273.18668837,100.70923419)(273.40168815,100.57423433)(273.61168701,100.40424123)
\curveto(273.82168773,100.24423466)(273.99168756,100.06923483)(274.12168701,99.87924123)
\curveto(274.16168739,99.82923507)(274.20168735,99.76923513)(274.24168701,99.69924123)
\curveto(274.27168728,99.63923526)(274.30668725,99.57923532)(274.34668701,99.51924123)
\curveto(274.39668716,99.43923546)(274.43668712,99.34423556)(274.46668701,99.23424123)
\curveto(274.49668706,99.12423578)(274.52668703,99.01923588)(274.55668701,98.91924123)
\curveto(274.59668696,98.80923609)(274.62168693,98.6992362)(274.63168701,98.58924123)
\curveto(274.64168691,98.47923642)(274.6566869,98.36423654)(274.67668701,98.24424123)
\curveto(274.68668687,98.2042367)(274.68668687,98.15923674)(274.67668701,98.10924123)
\curveto(274.67668688,98.06923683)(274.68168687,98.02923687)(274.69168701,97.98924123)
\curveto(274.70168685,97.94923695)(274.70668685,97.89423701)(274.70668701,97.82424123)
\curveto(274.70668685,97.75423715)(274.70168685,97.7042372)(274.69168701,97.67424123)
\curveto(274.67168688,97.62423728)(274.66668689,97.57923732)(274.67668701,97.53924123)
\curveto(274.68668687,97.4992374)(274.68668687,97.46423744)(274.67668701,97.43424123)
\lineto(274.67668701,97.34424123)
\curveto(274.6566869,97.28423762)(274.64168691,97.21923768)(274.63168701,97.14924123)
\curveto(274.63168692,97.08923781)(274.62668693,97.02423788)(274.61668701,96.95424123)
\curveto(274.56668699,96.78423812)(274.51668704,96.62423828)(274.46668701,96.47424123)
\curveto(274.41668714,96.32423858)(274.3516872,96.17923872)(274.27168701,96.03924123)
\curveto(274.23168732,95.98923891)(274.20168735,95.93423897)(274.18168701,95.87424123)
\curveto(274.1516874,95.82423908)(274.11668744,95.77423913)(274.07668701,95.72424123)
\curveto(273.89668766,95.48423942)(273.67668788,95.28423962)(273.41668701,95.12424123)
\curveto(273.1566884,94.96423994)(272.87168868,94.82424008)(272.56168701,94.70424123)
\curveto(272.42168913,94.64424026)(272.28168927,94.5992403)(272.14168701,94.56924123)
\curveto(271.99168956,94.53924036)(271.83668972,94.5042404)(271.67668701,94.46424123)
\curveto(271.56668999,94.44424046)(271.4566901,94.42924047)(271.34668701,94.41924123)
\curveto(271.23669032,94.40924049)(271.12669043,94.39424051)(271.01668701,94.37424123)
\curveto(270.97669058,94.36424054)(270.93669062,94.35924054)(270.89668701,94.35924123)
\curveto(270.8566907,94.36924053)(270.81669074,94.36924053)(270.77668701,94.35924123)
\curveto(270.72669083,94.34924055)(270.67669088,94.34424056)(270.62668701,94.34424123)
\lineto(270.46168701,94.34424123)
\curveto(270.41169114,94.32424058)(270.36169119,94.31924058)(270.31168701,94.32924123)
\curveto(270.2516913,94.33924056)(270.19669136,94.33924056)(270.14668701,94.32924123)
\curveto(270.10669145,94.31924058)(270.06169149,94.31924058)(270.01168701,94.32924123)
\curveto(269.96169159,94.33924056)(269.91169164,94.33424057)(269.86168701,94.31424123)
\curveto(269.79169176,94.29424061)(269.71669184,94.28924061)(269.63668701,94.29924123)
\curveto(269.54669201,94.30924059)(269.46169209,94.31424059)(269.38168701,94.31424123)
\curveto(269.29169226,94.31424059)(269.19169236,94.30924059)(269.08168701,94.29924123)
\curveto(268.96169259,94.28924061)(268.86169269,94.29424061)(268.78168701,94.31424123)
\lineto(268.49668701,94.31424123)
\lineto(267.86668701,94.35924123)
\curveto(267.76669379,94.36924053)(267.67169388,94.37924052)(267.58168701,94.38924123)
\lineto(267.28168701,94.41924123)
\curveto(267.23169432,94.43924046)(267.18169437,94.44424046)(267.13168701,94.43424123)
\curveto(267.07169448,94.43424047)(267.01669454,94.44424046)(266.96668701,94.46424123)
\curveto(266.79669476,94.51424039)(266.63169492,94.55424035)(266.47168701,94.58424123)
\curveto(266.30169525,94.61424029)(266.14169541,94.66424024)(265.99168701,94.73424123)
\curveto(265.53169602,94.92423998)(265.1566964,95.14423976)(264.86668701,95.39424123)
\curveto(264.57669698,95.65423925)(264.33169722,96.01423889)(264.13168701,96.47424123)
\curveto(264.08169747,96.6042383)(264.04669751,96.73423817)(264.02668701,96.86424123)
\curveto(264.00669755,97.0042379)(263.98169757,97.14423776)(263.95168701,97.28424123)
\curveto(263.94169761,97.35423755)(263.93669762,97.41923748)(263.93668701,97.47924123)
\curveto(263.93669762,97.53923736)(263.93169762,97.6042373)(263.92168701,97.67424123)
\curveto(263.90169765,98.5042364)(264.0516975,99.17423573)(264.37168701,99.68424123)
\curveto(264.68169687,100.19423471)(265.12169643,100.57423433)(265.69168701,100.82424123)
\curveto(265.81169574,100.87423403)(265.93669562,100.91923398)(266.06668701,100.95924123)
\curveto(266.19669536,100.9992339)(266.33169522,101.04423386)(266.47168701,101.09424123)
\curveto(266.551695,101.11423379)(266.63669492,101.12923377)(266.72668701,101.13924123)
\lineto(266.96668701,101.19924123)
\curveto(267.07669448,101.22923367)(267.18669437,101.24423366)(267.29668701,101.24424123)
\curveto(267.40669415,101.25423365)(267.51669404,101.26923363)(267.62668701,101.28924123)
\curveto(267.67669388,101.30923359)(267.72169383,101.31423359)(267.76168701,101.30424123)
\curveto(267.80169375,101.3042336)(267.84169371,101.30923359)(267.88168701,101.31924123)
\curveto(267.93169362,101.32923357)(267.98669357,101.32923357)(268.04668701,101.31924123)
\curveto(268.09669346,101.31923358)(268.14669341,101.32423358)(268.19668701,101.33424123)
\lineto(268.33168701,101.33424123)
\curveto(268.39169316,101.35423355)(268.46169309,101.35423355)(268.54168701,101.33424123)
\curveto(268.61169294,101.32423358)(268.67669288,101.32923357)(268.73668701,101.34924123)
\curveto(268.76669279,101.35923354)(268.80669275,101.36423354)(268.85668701,101.36424123)
\lineto(268.97668701,101.36424123)
\lineto(269.44168701,101.36424123)
\moveto(271.76668701,99.81924123)
\curveto(271.44669011,99.91923498)(271.08169047,99.97923492)(270.67168701,99.99924123)
\curveto(270.26169129,100.01923488)(269.8516917,100.02923487)(269.44168701,100.02924123)
\curveto(269.01169254,100.02923487)(268.59169296,100.01923488)(268.18168701,99.99924123)
\curveto(267.77169378,99.97923492)(267.38669417,99.93423497)(267.02668701,99.86424123)
\curveto(266.66669489,99.79423511)(266.34669521,99.68423522)(266.06668701,99.53424123)
\curveto(265.77669578,99.39423551)(265.54169601,99.1992357)(265.36168701,98.94924123)
\curveto(265.2516963,98.78923611)(265.17169638,98.60923629)(265.12168701,98.40924123)
\curveto(265.06169649,98.20923669)(265.03169652,97.96423694)(265.03168701,97.67424123)
\curveto(265.0516965,97.65423725)(265.06169649,97.61923728)(265.06168701,97.56924123)
\curveto(265.0516965,97.51923738)(265.0516965,97.47923742)(265.06168701,97.44924123)
\curveto(265.08169647,97.36923753)(265.10169645,97.29423761)(265.12168701,97.22424123)
\curveto(265.13169642,97.16423774)(265.1516964,97.0992378)(265.18168701,97.02924123)
\curveto(265.30169625,96.75923814)(265.47169608,96.53923836)(265.69168701,96.36924123)
\curveto(265.90169565,96.20923869)(266.14669541,96.07423883)(266.42668701,95.96424123)
\curveto(266.53669502,95.91423899)(266.6566949,95.87423903)(266.78668701,95.84424123)
\curveto(266.90669465,95.82423908)(267.03169452,95.7992391)(267.16168701,95.76924123)
\curveto(267.21169434,95.74923915)(267.26669429,95.73923916)(267.32668701,95.73924123)
\curveto(267.37669418,95.73923916)(267.42669413,95.73423917)(267.47668701,95.72424123)
\curveto(267.56669399,95.71423919)(267.66169389,95.7042392)(267.76168701,95.69424123)
\curveto(267.8516937,95.68423922)(267.94669361,95.67423923)(268.04668701,95.66424123)
\curveto(268.12669343,95.66423924)(268.21169334,95.65923924)(268.30168701,95.64924123)
\lineto(268.54168701,95.64924123)
\lineto(268.72168701,95.64924123)
\curveto(268.7516928,95.63923926)(268.78669277,95.63423927)(268.82668701,95.63424123)
\lineto(268.96168701,95.63424123)
\lineto(269.41168701,95.63424123)
\curveto(269.49169206,95.63423927)(269.57669198,95.62923927)(269.66668701,95.61924123)
\curveto(269.74669181,95.61923928)(269.82169173,95.62923927)(269.89168701,95.64924123)
\lineto(270.16168701,95.64924123)
\curveto(270.18169137,95.64923925)(270.21169134,95.64423926)(270.25168701,95.63424123)
\curveto(270.28169127,95.63423927)(270.30669125,95.63923926)(270.32668701,95.64924123)
\curveto(270.42669113,95.65923924)(270.52669103,95.66423924)(270.62668701,95.66424123)
\curveto(270.71669084,95.67423923)(270.81669074,95.68423922)(270.92668701,95.69424123)
\curveto(271.04669051,95.72423918)(271.17169038,95.73923916)(271.30168701,95.73924123)
\curveto(271.42169013,95.74923915)(271.53669002,95.77423913)(271.64668701,95.81424123)
\curveto(271.94668961,95.89423901)(272.21168934,95.97923892)(272.44168701,96.06924123)
\curveto(272.67168888,96.16923873)(272.88668867,96.31423859)(273.08668701,96.50424123)
\curveto(273.28668827,96.71423819)(273.43668812,96.97923792)(273.53668701,97.29924123)
\curveto(273.556688,97.33923756)(273.56668799,97.37423753)(273.56668701,97.40424123)
\curveto(273.556688,97.44423746)(273.56168799,97.48923741)(273.58168701,97.53924123)
\curveto(273.59168796,97.57923732)(273.60168795,97.64923725)(273.61168701,97.74924123)
\curveto(273.62168793,97.85923704)(273.61668794,97.94423696)(273.59668701,98.00424123)
\curveto(273.57668798,98.07423683)(273.56668799,98.14423676)(273.56668701,98.21424123)
\curveto(273.556688,98.28423662)(273.54168801,98.34923655)(273.52168701,98.40924123)
\curveto(273.46168809,98.60923629)(273.37668818,98.78923611)(273.26668701,98.94924123)
\curveto(273.24668831,98.97923592)(273.22668833,99.0042359)(273.20668701,99.02424123)
\lineto(273.14668701,99.08424123)
\curveto(273.12668843,99.12423578)(273.08668847,99.17423573)(273.02668701,99.23424123)
\curveto(272.88668867,99.33423557)(272.7566888,99.41923548)(272.63668701,99.48924123)
\curveto(272.51668904,99.55923534)(272.37168918,99.62923527)(272.20168701,99.69924123)
\curveto(272.13168942,99.72923517)(272.06168949,99.74923515)(271.99168701,99.75924123)
\curveto(271.92168963,99.77923512)(271.84668971,99.7992351)(271.76668701,99.81924123)
}
}
{
\newrgbcolor{curcolor}{0 0 0}
\pscustom[linestyle=none,fillstyle=solid,fillcolor=curcolor]
{
\newpath
\moveto(263.92168701,106.77385061)
\curveto(263.92169763,106.87384575)(263.93169762,106.96884566)(263.95168701,107.05885061)
\curveto(263.96169759,107.14884548)(263.99169756,107.21384541)(264.04168701,107.25385061)
\curveto(264.12169743,107.31384531)(264.22669733,107.34384528)(264.35668701,107.34385061)
\lineto(264.74668701,107.34385061)
\lineto(266.24668701,107.34385061)
\lineto(272.63668701,107.34385061)
\lineto(273.80668701,107.34385061)
\lineto(274.12168701,107.34385061)
\curveto(274.22168733,107.35384527)(274.30168725,107.33884529)(274.36168701,107.29885061)
\curveto(274.44168711,107.24884538)(274.49168706,107.17384545)(274.51168701,107.07385061)
\curveto(274.52168703,106.98384564)(274.52668703,106.87384575)(274.52668701,106.74385061)
\lineto(274.52668701,106.51885061)
\curveto(274.50668705,106.43884619)(274.49168706,106.36884626)(274.48168701,106.30885061)
\curveto(274.46168709,106.24884638)(274.42168713,106.19884643)(274.36168701,106.15885061)
\curveto(274.30168725,106.11884651)(274.22668733,106.09884653)(274.13668701,106.09885061)
\lineto(273.83668701,106.09885061)
\lineto(272.74168701,106.09885061)
\lineto(267.40168701,106.09885061)
\curveto(267.31169424,106.07884655)(267.23669432,106.06384656)(267.17668701,106.05385061)
\curveto(267.10669445,106.05384657)(267.04669451,106.0238466)(266.99668701,105.96385061)
\curveto(266.94669461,105.89384673)(266.92169463,105.80384682)(266.92168701,105.69385061)
\curveto(266.91169464,105.59384703)(266.90669465,105.48384714)(266.90668701,105.36385061)
\lineto(266.90668701,104.22385061)
\lineto(266.90668701,103.72885061)
\curveto(266.89669466,103.56884906)(266.83669472,103.45884917)(266.72668701,103.39885061)
\curveto(266.69669486,103.37884925)(266.66669489,103.36884926)(266.63668701,103.36885061)
\curveto(266.59669496,103.36884926)(266.551695,103.36384926)(266.50168701,103.35385061)
\curveto(266.38169517,103.33384929)(266.27169528,103.33884929)(266.17168701,103.36885061)
\curveto(266.07169548,103.40884922)(266.00169555,103.46384916)(265.96168701,103.53385061)
\curveto(265.91169564,103.61384901)(265.88669567,103.73384889)(265.88668701,103.89385061)
\curveto(265.88669567,104.05384857)(265.87169568,104.18884844)(265.84168701,104.29885061)
\curveto(265.83169572,104.34884828)(265.82669573,104.40384822)(265.82668701,104.46385061)
\curveto(265.81669574,104.5238481)(265.80169575,104.58384804)(265.78168701,104.64385061)
\curveto(265.73169582,104.79384783)(265.68169587,104.93884769)(265.63168701,105.07885061)
\curveto(265.57169598,105.21884741)(265.50169605,105.35384727)(265.42168701,105.48385061)
\curveto(265.33169622,105.623847)(265.22669633,105.74384688)(265.10668701,105.84385061)
\curveto(264.98669657,105.94384668)(264.8566967,106.03884659)(264.71668701,106.12885061)
\curveto(264.61669694,106.18884644)(264.50669705,106.23384639)(264.38668701,106.26385061)
\curveto(264.26669729,106.30384632)(264.16169739,106.35384627)(264.07168701,106.41385061)
\curveto(264.01169754,106.46384616)(263.97169758,106.53384609)(263.95168701,106.62385061)
\curveto(263.94169761,106.64384598)(263.93669762,106.66884596)(263.93668701,106.69885061)
\curveto(263.93669762,106.7288459)(263.93169762,106.75384587)(263.92168701,106.77385061)
}
}
{
\newrgbcolor{curcolor}{0 0 0}
\pscustom[linestyle=none,fillstyle=solid,fillcolor=curcolor]
{
\newpath
\moveto(263.92168701,115.12345998)
\curveto(263.92169763,115.22345513)(263.93169762,115.31845503)(263.95168701,115.40845998)
\curveto(263.96169759,115.49845485)(263.99169756,115.56345479)(264.04168701,115.60345998)
\curveto(264.12169743,115.66345469)(264.22669733,115.69345466)(264.35668701,115.69345998)
\lineto(264.74668701,115.69345998)
\lineto(266.24668701,115.69345998)
\lineto(272.63668701,115.69345998)
\lineto(273.80668701,115.69345998)
\lineto(274.12168701,115.69345998)
\curveto(274.22168733,115.70345465)(274.30168725,115.68845466)(274.36168701,115.64845998)
\curveto(274.44168711,115.59845475)(274.49168706,115.52345483)(274.51168701,115.42345998)
\curveto(274.52168703,115.33345502)(274.52668703,115.22345513)(274.52668701,115.09345998)
\lineto(274.52668701,114.86845998)
\curveto(274.50668705,114.78845556)(274.49168706,114.71845563)(274.48168701,114.65845998)
\curveto(274.46168709,114.59845575)(274.42168713,114.5484558)(274.36168701,114.50845998)
\curveto(274.30168725,114.46845588)(274.22668733,114.4484559)(274.13668701,114.44845998)
\lineto(273.83668701,114.44845998)
\lineto(272.74168701,114.44845998)
\lineto(267.40168701,114.44845998)
\curveto(267.31169424,114.42845592)(267.23669432,114.41345594)(267.17668701,114.40345998)
\curveto(267.10669445,114.40345595)(267.04669451,114.37345598)(266.99668701,114.31345998)
\curveto(266.94669461,114.24345611)(266.92169463,114.1534562)(266.92168701,114.04345998)
\curveto(266.91169464,113.94345641)(266.90669465,113.83345652)(266.90668701,113.71345998)
\lineto(266.90668701,112.57345998)
\lineto(266.90668701,112.07845998)
\curveto(266.89669466,111.91845843)(266.83669472,111.80845854)(266.72668701,111.74845998)
\curveto(266.69669486,111.72845862)(266.66669489,111.71845863)(266.63668701,111.71845998)
\curveto(266.59669496,111.71845863)(266.551695,111.71345864)(266.50168701,111.70345998)
\curveto(266.38169517,111.68345867)(266.27169528,111.68845866)(266.17168701,111.71845998)
\curveto(266.07169548,111.75845859)(266.00169555,111.81345854)(265.96168701,111.88345998)
\curveto(265.91169564,111.96345839)(265.88669567,112.08345827)(265.88668701,112.24345998)
\curveto(265.88669567,112.40345795)(265.87169568,112.53845781)(265.84168701,112.64845998)
\curveto(265.83169572,112.69845765)(265.82669573,112.7534576)(265.82668701,112.81345998)
\curveto(265.81669574,112.87345748)(265.80169575,112.93345742)(265.78168701,112.99345998)
\curveto(265.73169582,113.14345721)(265.68169587,113.28845706)(265.63168701,113.42845998)
\curveto(265.57169598,113.56845678)(265.50169605,113.70345665)(265.42168701,113.83345998)
\curveto(265.33169622,113.97345638)(265.22669633,114.09345626)(265.10668701,114.19345998)
\curveto(264.98669657,114.29345606)(264.8566967,114.38845596)(264.71668701,114.47845998)
\curveto(264.61669694,114.53845581)(264.50669705,114.58345577)(264.38668701,114.61345998)
\curveto(264.26669729,114.6534557)(264.16169739,114.70345565)(264.07168701,114.76345998)
\curveto(264.01169754,114.81345554)(263.97169758,114.88345547)(263.95168701,114.97345998)
\curveto(263.94169761,114.99345536)(263.93669762,115.01845533)(263.93668701,115.04845998)
\curveto(263.93669762,115.07845527)(263.93169762,115.10345525)(263.92168701,115.12345998)
}
}
{
\newrgbcolor{curcolor}{0 0 0}
\pscustom[linestyle=none,fillstyle=solid,fillcolor=curcolor]
{
\newpath
\moveto(284.75803345,31.67142873)
\lineto(284.75803345,32.58642873)
\curveto(284.75804414,32.68642608)(284.75804414,32.78142599)(284.75803345,32.87142873)
\curveto(284.75804414,32.96142581)(284.77804412,33.03642573)(284.81803345,33.09642873)
\curveto(284.87804402,33.18642558)(284.95804394,33.24642552)(285.05803345,33.27642873)
\curveto(285.15804374,33.31642545)(285.26304364,33.36142541)(285.37303345,33.41142873)
\curveto(285.56304334,33.49142528)(285.75304315,33.56142521)(285.94303345,33.62142873)
\curveto(286.13304277,33.69142508)(286.32304258,33.766425)(286.51303345,33.84642873)
\curveto(286.69304221,33.91642485)(286.87804202,33.98142479)(287.06803345,34.04142873)
\curveto(287.24804165,34.10142467)(287.42804147,34.1714246)(287.60803345,34.25142873)
\curveto(287.74804115,34.31142446)(287.89304101,34.3664244)(288.04303345,34.41642873)
\curveto(288.19304071,34.4664243)(288.33804056,34.52142425)(288.47803345,34.58142873)
\curveto(288.92803997,34.76142401)(289.38303952,34.93142384)(289.84303345,35.09142873)
\curveto(290.29303861,35.25142352)(290.74303816,35.42142335)(291.19303345,35.60142873)
\curveto(291.24303766,35.62142315)(291.29303761,35.63642313)(291.34303345,35.64642873)
\lineto(291.49303345,35.70642873)
\curveto(291.71303719,35.79642297)(291.93803696,35.88142289)(292.16803345,35.96142873)
\curveto(292.38803651,36.04142273)(292.60803629,36.12642264)(292.82803345,36.21642873)
\curveto(292.91803598,36.25642251)(293.02803587,36.29642247)(293.15803345,36.33642873)
\curveto(293.27803562,36.37642239)(293.34803555,36.44142233)(293.36803345,36.53142873)
\curveto(293.37803552,36.5714222)(293.37803552,36.60142217)(293.36803345,36.62142873)
\lineto(293.30803345,36.68142873)
\curveto(293.25803564,36.73142204)(293.2030357,36.766422)(293.14303345,36.78642873)
\curveto(293.08303582,36.81642195)(293.01803588,36.84642192)(292.94803345,36.87642873)
\lineto(292.31803345,37.11642873)
\curveto(292.0980368,37.19642157)(291.88303702,37.27642149)(291.67303345,37.35642873)
\lineto(291.52303345,37.41642873)
\lineto(291.34303345,37.47642873)
\curveto(291.15303775,37.55642121)(290.96303794,37.62642114)(290.77303345,37.68642873)
\curveto(290.57303833,37.75642101)(290.37303853,37.83142094)(290.17303345,37.91142873)
\curveto(289.59303931,38.15142062)(289.00803989,38.3714204)(288.41803345,38.57142873)
\curveto(287.82804107,38.78141999)(287.24304166,39.00641976)(286.66303345,39.24642873)
\curveto(286.46304244,39.32641944)(286.25804264,39.40141937)(286.04803345,39.47142873)
\curveto(285.83804306,39.55141922)(285.63304327,39.63141914)(285.43303345,39.71142873)
\curveto(285.35304355,39.75141902)(285.25304365,39.78641898)(285.13303345,39.81642873)
\curveto(285.01304389,39.85641891)(284.92804397,39.91141886)(284.87803345,39.98142873)
\curveto(284.83804406,40.04141873)(284.80804409,40.11641865)(284.78803345,40.20642873)
\curveto(284.76804413,40.30641846)(284.75804414,40.41641835)(284.75803345,40.53642873)
\curveto(284.74804415,40.65641811)(284.74804415,40.77641799)(284.75803345,40.89642873)
\curveto(284.75804414,41.01641775)(284.75804414,41.12641764)(284.75803345,41.22642873)
\curveto(284.75804414,41.31641745)(284.75804414,41.40641736)(284.75803345,41.49642873)
\curveto(284.75804414,41.59641717)(284.77804412,41.6714171)(284.81803345,41.72142873)
\curveto(284.86804403,41.81141696)(284.95804394,41.86141691)(285.08803345,41.87142873)
\curveto(285.21804368,41.88141689)(285.35804354,41.88641688)(285.50803345,41.88642873)
\lineto(287.15803345,41.88642873)
\lineto(293.42803345,41.88642873)
\lineto(294.68803345,41.88642873)
\curveto(294.7980341,41.88641688)(294.90803399,41.88641688)(295.01803345,41.88642873)
\curveto(295.12803377,41.89641687)(295.21303369,41.87641689)(295.27303345,41.82642873)
\curveto(295.33303357,41.79641697)(295.37303353,41.75141702)(295.39303345,41.69142873)
\curveto(295.4030335,41.63141714)(295.41803348,41.56141721)(295.43803345,41.48142873)
\lineto(295.43803345,41.24142873)
\lineto(295.43803345,40.88142873)
\curveto(295.42803347,40.771418)(295.38303352,40.69141808)(295.30303345,40.64142873)
\curveto(295.27303363,40.62141815)(295.24303366,40.60641816)(295.21303345,40.59642873)
\curveto(295.17303373,40.59641817)(295.12803377,40.58641818)(295.07803345,40.56642873)
\lineto(294.91303345,40.56642873)
\curveto(294.85303405,40.55641821)(294.78303412,40.55141822)(294.70303345,40.55142873)
\curveto(294.62303428,40.56141821)(294.54803435,40.5664182)(294.47803345,40.56642873)
\lineto(293.63803345,40.56642873)
\lineto(289.21303345,40.56642873)
\curveto(288.96303994,40.5664182)(288.71304019,40.5664182)(288.46303345,40.56642873)
\curveto(288.2030407,40.5664182)(287.95304095,40.56141821)(287.71303345,40.55142873)
\curveto(287.61304129,40.55141822)(287.5030414,40.54641822)(287.38303345,40.53642873)
\curveto(287.26304164,40.52641824)(287.2030417,40.4714183)(287.20303345,40.37142873)
\lineto(287.21803345,40.37142873)
\curveto(287.23804166,40.30141847)(287.3030416,40.24141853)(287.41303345,40.19142873)
\curveto(287.52304138,40.15141862)(287.61804128,40.11641865)(287.69803345,40.08642873)
\curveto(287.86804103,40.01641875)(288.04304086,39.95141882)(288.22303345,39.89142873)
\curveto(288.39304051,39.83141894)(288.56304034,39.76141901)(288.73303345,39.68142873)
\curveto(288.78304012,39.66141911)(288.82804007,39.64641912)(288.86803345,39.63642873)
\curveto(288.90803999,39.62641914)(288.95303995,39.61141916)(289.00303345,39.59142873)
\curveto(289.18303972,39.51141926)(289.36803953,39.44141933)(289.55803345,39.38142873)
\curveto(289.73803916,39.33141944)(289.91803898,39.2664195)(290.09803345,39.18642873)
\curveto(290.24803865,39.11641965)(290.4030385,39.05641971)(290.56303345,39.00642873)
\curveto(290.71303819,38.95641981)(290.86303804,38.90141987)(291.01303345,38.84142873)
\curveto(291.48303742,38.64142013)(291.95803694,38.46142031)(292.43803345,38.30142873)
\curveto(292.90803599,38.14142063)(293.37303553,37.9664208)(293.83303345,37.77642873)
\curveto(294.01303489,37.69642107)(294.19303471,37.62642114)(294.37303345,37.56642873)
\curveto(294.55303435,37.50642126)(294.73303417,37.44142133)(294.91303345,37.37142873)
\curveto(295.02303388,37.32142145)(295.12803377,37.2714215)(295.22803345,37.22142873)
\curveto(295.31803358,37.18142159)(295.38303352,37.09642167)(295.42303345,36.96642873)
\curveto(295.43303347,36.94642182)(295.43803346,36.92142185)(295.43803345,36.89142873)
\curveto(295.42803347,36.8714219)(295.42803347,36.84642192)(295.43803345,36.81642873)
\curveto(295.44803345,36.78642198)(295.45303345,36.75142202)(295.45303345,36.71142873)
\curveto(295.44303346,36.6714221)(295.43803346,36.63142214)(295.43803345,36.59142873)
\lineto(295.43803345,36.29142873)
\curveto(295.43803346,36.19142258)(295.41303349,36.11142266)(295.36303345,36.05142873)
\curveto(295.31303359,35.9714228)(295.24303366,35.91142286)(295.15303345,35.87142873)
\curveto(295.05303385,35.84142293)(294.95303395,35.80142297)(294.85303345,35.75142873)
\curveto(294.65303425,35.6714231)(294.44803445,35.59142318)(294.23803345,35.51142873)
\curveto(294.01803488,35.44142333)(293.80803509,35.3664234)(293.60803345,35.28642873)
\curveto(293.42803547,35.20642356)(293.24803565,35.13642363)(293.06803345,35.07642873)
\curveto(292.87803602,35.02642374)(292.69303621,34.96142381)(292.51303345,34.88142873)
\curveto(291.95303695,34.65142412)(291.38803751,34.43642433)(290.81803345,34.23642873)
\curveto(290.24803865,34.03642473)(289.68303922,33.82142495)(289.12303345,33.59142873)
\lineto(288.49303345,33.35142873)
\curveto(288.27304063,33.28142549)(288.06304084,33.20642556)(287.86303345,33.12642873)
\curveto(287.75304115,33.07642569)(287.64804125,33.03142574)(287.54803345,32.99142873)
\curveto(287.43804146,32.96142581)(287.34304156,32.91142586)(287.26303345,32.84142873)
\curveto(287.24304166,32.83142594)(287.23304167,32.82142595)(287.23303345,32.81142873)
\lineto(287.20303345,32.78142873)
\lineto(287.20303345,32.70642873)
\lineto(287.23303345,32.67642873)
\curveto(287.23304167,32.6664261)(287.23804166,32.65642611)(287.24803345,32.64642873)
\curveto(287.2980416,32.62642614)(287.35304155,32.61642615)(287.41303345,32.61642873)
\curveto(287.47304143,32.61642615)(287.53304137,32.60642616)(287.59303345,32.58642873)
\lineto(287.75803345,32.58642873)
\curveto(287.81804108,32.5664262)(287.88304102,32.56142621)(287.95303345,32.57142873)
\curveto(288.02304088,32.58142619)(288.09304081,32.58642618)(288.16303345,32.58642873)
\lineto(288.97303345,32.58642873)
\lineto(293.53303345,32.58642873)
\lineto(294.71803345,32.58642873)
\curveto(294.82803407,32.58642618)(294.93803396,32.58142619)(295.04803345,32.57142873)
\curveto(295.15803374,32.5714262)(295.24303366,32.54642622)(295.30303345,32.49642873)
\curveto(295.38303352,32.44642632)(295.42803347,32.35642641)(295.43803345,32.22642873)
\lineto(295.43803345,31.83642873)
\lineto(295.43803345,31.64142873)
\curveto(295.43803346,31.59142718)(295.42803347,31.54142723)(295.40803345,31.49142873)
\curveto(295.36803353,31.36142741)(295.28303362,31.28642748)(295.15303345,31.26642873)
\curveto(295.02303388,31.25642751)(294.87303403,31.25142752)(294.70303345,31.25142873)
\lineto(292.96303345,31.25142873)
\lineto(286.96303345,31.25142873)
\lineto(285.55303345,31.25142873)
\curveto(285.44304346,31.25142752)(285.32804357,31.24642752)(285.20803345,31.23642873)
\curveto(285.08804381,31.23642753)(284.99304391,31.26142751)(284.92303345,31.31142873)
\curveto(284.86304404,31.35142742)(284.81304409,31.42642734)(284.77303345,31.53642873)
\curveto(284.76304414,31.55642721)(284.76304414,31.57642719)(284.77303345,31.59642873)
\curveto(284.77304413,31.62642714)(284.76804413,31.65142712)(284.75803345,31.67142873)
}
}
{
\newrgbcolor{curcolor}{0 0 0}
\pscustom[linestyle=none,fillstyle=solid,fillcolor=curcolor]
{
\newpath
\moveto(294.88303345,50.87353811)
\curveto(295.04303386,50.90353028)(295.17803372,50.88853029)(295.28803345,50.82853811)
\curveto(295.38803351,50.76853041)(295.46303344,50.68853049)(295.51303345,50.58853811)
\curveto(295.53303337,50.53853064)(295.54303336,50.4835307)(295.54303345,50.42353811)
\curveto(295.54303336,50.37353081)(295.55303335,50.31853086)(295.57303345,50.25853811)
\curveto(295.62303328,50.03853114)(295.60803329,49.81853136)(295.52803345,49.59853811)
\curveto(295.45803344,49.38853179)(295.36803353,49.24353194)(295.25803345,49.16353811)
\curveto(295.18803371,49.11353207)(295.10803379,49.06853211)(295.01803345,49.02853811)
\curveto(294.91803398,48.98853219)(294.83803406,48.93853224)(294.77803345,48.87853811)
\curveto(294.75803414,48.85853232)(294.73803416,48.83353235)(294.71803345,48.80353811)
\curveto(294.6980342,48.7835324)(294.69303421,48.75353243)(294.70303345,48.71353811)
\curveto(294.73303417,48.60353258)(294.78803411,48.49853268)(294.86803345,48.39853811)
\curveto(294.94803395,48.30853287)(295.01803388,48.21853296)(295.07803345,48.12853811)
\curveto(295.15803374,47.99853318)(295.23303367,47.85853332)(295.30303345,47.70853811)
\curveto(295.36303354,47.55853362)(295.41803348,47.39853378)(295.46803345,47.22853811)
\curveto(295.4980334,47.12853405)(295.51803338,47.01853416)(295.52803345,46.89853811)
\curveto(295.53803336,46.78853439)(295.55303335,46.6785345)(295.57303345,46.56853811)
\curveto(295.58303332,46.51853466)(295.58803331,46.47353471)(295.58803345,46.43353811)
\lineto(295.58803345,46.32853811)
\curveto(295.60803329,46.21853496)(295.60803329,46.11353507)(295.58803345,46.01353811)
\lineto(295.58803345,45.87853811)
\curveto(295.57803332,45.82853535)(295.57303333,45.7785354)(295.57303345,45.72853811)
\curveto(295.57303333,45.6785355)(295.56303334,45.63353555)(295.54303345,45.59353811)
\curveto(295.53303337,45.55353563)(295.52803337,45.51853566)(295.52803345,45.48853811)
\curveto(295.53803336,45.46853571)(295.53803336,45.44353574)(295.52803345,45.41353811)
\lineto(295.46803345,45.17353811)
\curveto(295.45803344,45.09353609)(295.43803346,45.01853616)(295.40803345,44.94853811)
\curveto(295.27803362,44.64853653)(295.13303377,44.40353678)(294.97303345,44.21353811)
\curveto(294.8030341,44.03353715)(294.56803433,43.8835373)(294.26803345,43.76353811)
\curveto(294.04803485,43.67353751)(293.78303512,43.62853755)(293.47303345,43.62853811)
\lineto(293.15803345,43.62853811)
\curveto(293.10803579,43.63853754)(293.05803584,43.64353754)(293.00803345,43.64353811)
\lineto(292.82803345,43.67353811)
\lineto(292.49803345,43.79353811)
\curveto(292.38803651,43.83353735)(292.28803661,43.8835373)(292.19803345,43.94353811)
\curveto(291.90803699,44.12353706)(291.69303721,44.36853681)(291.55303345,44.67853811)
\curveto(291.41303749,44.98853619)(291.28803761,45.32853585)(291.17803345,45.69853811)
\curveto(291.13803776,45.83853534)(291.10803779,45.9835352)(291.08803345,46.13353811)
\curveto(291.06803783,46.2835349)(291.04303786,46.43353475)(291.01303345,46.58353811)
\curveto(290.99303791,46.65353453)(290.98303792,46.71853446)(290.98303345,46.77853811)
\curveto(290.98303792,46.84853433)(290.97303793,46.92353426)(290.95303345,47.00353811)
\curveto(290.93303797,47.07353411)(290.92303798,47.14353404)(290.92303345,47.21353811)
\curveto(290.91303799,47.2835339)(290.898038,47.35853382)(290.87803345,47.43853811)
\curveto(290.81803808,47.68853349)(290.76803813,47.92353326)(290.72803345,48.14353811)
\curveto(290.67803822,48.36353282)(290.56303834,48.53853264)(290.38303345,48.66853811)
\curveto(290.3030386,48.72853245)(290.2030387,48.7785324)(290.08303345,48.81853811)
\curveto(289.95303895,48.85853232)(289.81303909,48.85853232)(289.66303345,48.81853811)
\curveto(289.42303948,48.75853242)(289.23303967,48.66853251)(289.09303345,48.54853811)
\curveto(288.95303995,48.43853274)(288.84304006,48.2785329)(288.76303345,48.06853811)
\curveto(288.71304019,47.94853323)(288.67804022,47.80353338)(288.65803345,47.63353811)
\curveto(288.63804026,47.47353371)(288.62804027,47.30353388)(288.62803345,47.12353811)
\curveto(288.62804027,46.94353424)(288.63804026,46.76853441)(288.65803345,46.59853811)
\curveto(288.67804022,46.42853475)(288.70804019,46.2835349)(288.74803345,46.16353811)
\curveto(288.80804009,45.99353519)(288.89304001,45.82853535)(289.00303345,45.66853811)
\curveto(289.06303984,45.58853559)(289.14303976,45.51353567)(289.24303345,45.44353811)
\curveto(289.33303957,45.3835358)(289.43303947,45.32853585)(289.54303345,45.27853811)
\curveto(289.62303928,45.24853593)(289.70803919,45.21853596)(289.79803345,45.18853811)
\curveto(289.88803901,45.16853601)(289.95803894,45.12353606)(290.00803345,45.05353811)
\curveto(290.03803886,45.01353617)(290.06303884,44.94353624)(290.08303345,44.84353811)
\curveto(290.09303881,44.75353643)(290.0980388,44.65853652)(290.09803345,44.55853811)
\curveto(290.0980388,44.45853672)(290.09303881,44.35853682)(290.08303345,44.25853811)
\curveto(290.06303884,44.16853701)(290.03803886,44.10353708)(290.00803345,44.06353811)
\curveto(289.97803892,44.02353716)(289.92803897,43.99353719)(289.85803345,43.97353811)
\curveto(289.78803911,43.95353723)(289.71303919,43.95353723)(289.63303345,43.97353811)
\curveto(289.5030394,44.00353718)(289.38303952,44.03353715)(289.27303345,44.06353811)
\curveto(289.15303975,44.10353708)(289.03803986,44.14853703)(288.92803345,44.19853811)
\curveto(288.57804032,44.38853679)(288.30804059,44.62853655)(288.11803345,44.91853811)
\curveto(287.91804098,45.20853597)(287.75804114,45.56853561)(287.63803345,45.99853811)
\curveto(287.61804128,46.09853508)(287.6030413,46.19853498)(287.59303345,46.29853811)
\curveto(287.58304132,46.40853477)(287.56804133,46.51853466)(287.54803345,46.62853811)
\curveto(287.53804136,46.66853451)(287.53804136,46.73353445)(287.54803345,46.82353811)
\curveto(287.54804135,46.91353427)(287.53804136,46.96853421)(287.51803345,46.98853811)
\curveto(287.50804139,47.68853349)(287.58804131,48.29853288)(287.75803345,48.81853811)
\curveto(287.92804097,49.33853184)(288.25304065,49.70353148)(288.73303345,49.91353811)
\curveto(288.93303997,50.00353118)(289.16803973,50.05353113)(289.43803345,50.06353811)
\curveto(289.6980392,50.0835311)(289.97303893,50.09353109)(290.26303345,50.09353811)
\lineto(293.57803345,50.09353811)
\curveto(293.71803518,50.09353109)(293.85303505,50.09853108)(293.98303345,50.10853811)
\curveto(294.11303479,50.11853106)(294.21803468,50.14853103)(294.29803345,50.19853811)
\curveto(294.36803453,50.24853093)(294.41803448,50.31353087)(294.44803345,50.39353811)
\curveto(294.48803441,50.4835307)(294.51803438,50.56853061)(294.53803345,50.64853811)
\curveto(294.54803435,50.72853045)(294.59303431,50.78853039)(294.67303345,50.82853811)
\curveto(294.7030342,50.84853033)(294.73303417,50.85853032)(294.76303345,50.85853811)
\curveto(294.79303411,50.85853032)(294.83303407,50.86353032)(294.88303345,50.87353811)
\moveto(293.21803345,48.72853811)
\curveto(293.07803582,48.78853239)(292.91803598,48.81853236)(292.73803345,48.81853811)
\curveto(292.54803635,48.82853235)(292.35303655,48.83353235)(292.15303345,48.83353811)
\curveto(292.04303686,48.83353235)(291.94303696,48.82853235)(291.85303345,48.81853811)
\curveto(291.76303714,48.80853237)(291.69303721,48.76853241)(291.64303345,48.69853811)
\curveto(291.62303728,48.66853251)(291.61303729,48.59853258)(291.61303345,48.48853811)
\curveto(291.63303727,48.46853271)(291.64303726,48.43353275)(291.64303345,48.38353811)
\curveto(291.64303726,48.33353285)(291.65303725,48.28853289)(291.67303345,48.24853811)
\curveto(291.69303721,48.16853301)(291.71303719,48.0785331)(291.73303345,47.97853811)
\lineto(291.79303345,47.67853811)
\curveto(291.79303711,47.64853353)(291.7980371,47.61353357)(291.80803345,47.57353811)
\lineto(291.80803345,47.46853811)
\curveto(291.84803705,47.31853386)(291.87303703,47.15353403)(291.88303345,46.97353811)
\curveto(291.88303702,46.80353438)(291.903037,46.64353454)(291.94303345,46.49353811)
\curveto(291.96303694,46.41353477)(291.98303692,46.33853484)(292.00303345,46.26853811)
\curveto(292.01303689,46.20853497)(292.02803687,46.13853504)(292.04803345,46.05853811)
\curveto(292.0980368,45.89853528)(292.16303674,45.74853543)(292.24303345,45.60853811)
\curveto(292.31303659,45.46853571)(292.4030365,45.34853583)(292.51303345,45.24853811)
\curveto(292.62303628,45.14853603)(292.75803614,45.07353611)(292.91803345,45.02353811)
\curveto(293.06803583,44.97353621)(293.25303565,44.95353623)(293.47303345,44.96353811)
\curveto(293.57303533,44.96353622)(293.66803523,44.9785362)(293.75803345,45.00853811)
\curveto(293.83803506,45.04853613)(293.91303499,45.09353609)(293.98303345,45.14353811)
\curveto(294.09303481,45.22353596)(294.18803471,45.32853585)(294.26803345,45.45853811)
\curveto(294.33803456,45.58853559)(294.3980345,45.72853545)(294.44803345,45.87853811)
\curveto(294.45803444,45.92853525)(294.46303444,45.9785352)(294.46303345,46.02853811)
\curveto(294.46303444,46.0785351)(294.46803443,46.12853505)(294.47803345,46.17853811)
\curveto(294.4980344,46.24853493)(294.51303439,46.33353485)(294.52303345,46.43353811)
\curveto(294.52303438,46.54353464)(294.51303439,46.63353455)(294.49303345,46.70353811)
\curveto(294.47303443,46.76353442)(294.46803443,46.82353436)(294.47803345,46.88353811)
\curveto(294.47803442,46.94353424)(294.46803443,47.00353418)(294.44803345,47.06353811)
\curveto(294.42803447,47.14353404)(294.41303449,47.21853396)(294.40303345,47.28853811)
\curveto(294.39303451,47.36853381)(294.37303453,47.44353374)(294.34303345,47.51353811)
\curveto(294.22303468,47.80353338)(294.07803482,48.04853313)(293.90803345,48.24853811)
\curveto(293.73803516,48.45853272)(293.50803539,48.61853256)(293.21803345,48.72853811)
}
}
{
\newrgbcolor{curcolor}{0 0 0}
\pscustom[linestyle=none,fillstyle=solid,fillcolor=curcolor]
{
\newpath
\moveto(287.53303345,55.69017873)
\curveto(287.53304137,55.92017394)(287.59304131,56.05017381)(287.71303345,56.08017873)
\curveto(287.82304108,56.11017375)(287.98804091,56.12517374)(288.20803345,56.12517873)
\lineto(288.49303345,56.12517873)
\curveto(288.58304032,56.12517374)(288.65804024,56.10017376)(288.71803345,56.05017873)
\curveto(288.7980401,55.99017387)(288.84304006,55.90517396)(288.85303345,55.79517873)
\curveto(288.85304005,55.68517418)(288.86804003,55.57517429)(288.89803345,55.46517873)
\curveto(288.92803997,55.32517454)(288.95803994,55.19017467)(288.98803345,55.06017873)
\curveto(289.01803988,54.94017492)(289.05803984,54.82517504)(289.10803345,54.71517873)
\curveto(289.23803966,54.42517544)(289.41803948,54.19017567)(289.64803345,54.01017873)
\curveto(289.86803903,53.83017603)(290.12303878,53.67517619)(290.41303345,53.54517873)
\curveto(290.52303838,53.50517636)(290.63803826,53.47517639)(290.75803345,53.45517873)
\curveto(290.86803803,53.43517643)(290.98303792,53.41017645)(291.10303345,53.38017873)
\curveto(291.15303775,53.37017649)(291.2030377,53.3651765)(291.25303345,53.36517873)
\curveto(291.3030376,53.37517649)(291.35303755,53.37517649)(291.40303345,53.36517873)
\curveto(291.52303738,53.33517653)(291.66303724,53.32017654)(291.82303345,53.32017873)
\curveto(291.97303693,53.33017653)(292.11803678,53.33517653)(292.25803345,53.33517873)
\lineto(294.10303345,53.33517873)
\lineto(294.44803345,53.33517873)
\curveto(294.56803433,53.33517653)(294.68303422,53.33017653)(294.79303345,53.32017873)
\curveto(294.903034,53.31017655)(294.9980339,53.30517656)(295.07803345,53.30517873)
\curveto(295.15803374,53.31517655)(295.22803367,53.29517657)(295.28803345,53.24517873)
\curveto(295.35803354,53.19517667)(295.3980335,53.11517675)(295.40803345,53.00517873)
\curveto(295.41803348,52.90517696)(295.42303348,52.79517707)(295.42303345,52.67517873)
\lineto(295.42303345,52.40517873)
\curveto(295.4030335,52.35517751)(295.38803351,52.30517756)(295.37803345,52.25517873)
\curveto(295.35803354,52.21517765)(295.33303357,52.18517768)(295.30303345,52.16517873)
\curveto(295.23303367,52.11517775)(295.14803375,52.08517778)(295.04803345,52.07517873)
\lineto(294.71803345,52.07517873)
\lineto(293.56303345,52.07517873)
\lineto(289.40803345,52.07517873)
\lineto(288.37303345,52.07517873)
\lineto(288.07303345,52.07517873)
\curveto(287.97304093,52.08517778)(287.88804101,52.11517775)(287.81803345,52.16517873)
\curveto(287.77804112,52.19517767)(287.74804115,52.24517762)(287.72803345,52.31517873)
\curveto(287.70804119,52.39517747)(287.6980412,52.48017738)(287.69803345,52.57017873)
\curveto(287.68804121,52.6601772)(287.68804121,52.75017711)(287.69803345,52.84017873)
\curveto(287.70804119,52.93017693)(287.72304118,53.00017686)(287.74303345,53.05017873)
\curveto(287.77304113,53.13017673)(287.83304107,53.18017668)(287.92303345,53.20017873)
\curveto(288.0030409,53.23017663)(288.09304081,53.24517662)(288.19303345,53.24517873)
\lineto(288.49303345,53.24517873)
\curveto(288.59304031,53.24517662)(288.68304022,53.2651766)(288.76303345,53.30517873)
\curveto(288.78304012,53.31517655)(288.7980401,53.32517654)(288.80803345,53.33517873)
\lineto(288.85303345,53.38017873)
\curveto(288.85304005,53.49017637)(288.80804009,53.58017628)(288.71803345,53.65017873)
\curveto(288.61804028,53.72017614)(288.53804036,53.78017608)(288.47803345,53.83017873)
\lineto(288.38803345,53.92017873)
\curveto(288.27804062,54.01017585)(288.16304074,54.13517573)(288.04303345,54.29517873)
\curveto(287.92304098,54.45517541)(287.83304107,54.60517526)(287.77303345,54.74517873)
\curveto(287.72304118,54.83517503)(287.68804121,54.93017493)(287.66803345,55.03017873)
\curveto(287.63804126,55.13017473)(287.60804129,55.23517463)(287.57803345,55.34517873)
\curveto(287.56804133,55.40517446)(287.56304134,55.4651744)(287.56303345,55.52517873)
\curveto(287.55304135,55.58517428)(287.54304136,55.64017422)(287.53303345,55.69017873)
}
}
{
\newrgbcolor{curcolor}{0 0 0}
\pscustom[linestyle=none,fillstyle=solid,fillcolor=curcolor]
{
}
}
{
\newrgbcolor{curcolor}{0 0 0}
\pscustom[linestyle=none,fillstyle=solid,fillcolor=curcolor]
{
\newpath
\moveto(284.83303345,64.24510061)
\curveto(284.82304408,64.93509597)(284.94304396,65.53509537)(285.19303345,66.04510061)
\curveto(285.44304346,66.56509434)(285.77804312,66.96009395)(286.19803345,67.23010061)
\curveto(286.27804262,67.28009363)(286.36804253,67.32509358)(286.46803345,67.36510061)
\curveto(286.55804234,67.4050935)(286.65304225,67.45009346)(286.75303345,67.50010061)
\curveto(286.85304205,67.54009337)(286.95304195,67.57009334)(287.05303345,67.59010061)
\curveto(287.15304175,67.6100933)(287.25804164,67.63009328)(287.36803345,67.65010061)
\curveto(287.41804148,67.67009324)(287.46304144,67.67509323)(287.50303345,67.66510061)
\curveto(287.54304136,67.65509325)(287.58804131,67.66009325)(287.63803345,67.68010061)
\curveto(287.68804121,67.69009322)(287.77304113,67.69509321)(287.89303345,67.69510061)
\curveto(288.0030409,67.69509321)(288.08804081,67.69009322)(288.14803345,67.68010061)
\curveto(288.20804069,67.66009325)(288.26804063,67.65009326)(288.32803345,67.65010061)
\curveto(288.38804051,67.66009325)(288.44804045,67.65509325)(288.50803345,67.63510061)
\curveto(288.64804025,67.59509331)(288.78304012,67.56009335)(288.91303345,67.53010061)
\curveto(289.04303986,67.50009341)(289.16803973,67.46009345)(289.28803345,67.41010061)
\curveto(289.42803947,67.35009356)(289.55303935,67.28009363)(289.66303345,67.20010061)
\curveto(289.77303913,67.13009378)(289.88303902,67.05509385)(289.99303345,66.97510061)
\lineto(290.05303345,66.91510061)
\curveto(290.07303883,66.905094)(290.09303881,66.89009402)(290.11303345,66.87010061)
\curveto(290.27303863,66.75009416)(290.41803848,66.61509429)(290.54803345,66.46510061)
\curveto(290.67803822,66.31509459)(290.8030381,66.15509475)(290.92303345,65.98510061)
\curveto(291.14303776,65.67509523)(291.34803755,65.38009553)(291.53803345,65.10010061)
\curveto(291.67803722,64.87009604)(291.81303709,64.64009627)(291.94303345,64.41010061)
\curveto(292.07303683,64.19009672)(292.20803669,63.97009694)(292.34803345,63.75010061)
\curveto(292.51803638,63.50009741)(292.6980362,63.26009765)(292.88803345,63.03010061)
\curveto(293.07803582,62.8100981)(293.3030356,62.62009829)(293.56303345,62.46010061)
\curveto(293.62303528,62.42009849)(293.68303522,62.38509852)(293.74303345,62.35510061)
\curveto(293.79303511,62.32509858)(293.85803504,62.29509861)(293.93803345,62.26510061)
\curveto(294.00803489,62.24509866)(294.06803483,62.24009867)(294.11803345,62.25010061)
\curveto(294.18803471,62.27009864)(294.24303466,62.3050986)(294.28303345,62.35510061)
\curveto(294.31303459,62.4050985)(294.33303457,62.46509844)(294.34303345,62.53510061)
\lineto(294.34303345,62.77510061)
\lineto(294.34303345,63.52510061)
\lineto(294.34303345,66.33010061)
\lineto(294.34303345,66.99010061)
\curveto(294.34303456,67.08009383)(294.34803455,67.16509374)(294.35803345,67.24510061)
\curveto(294.35803454,67.32509358)(294.37803452,67.39009352)(294.41803345,67.44010061)
\curveto(294.45803444,67.49009342)(294.53303437,67.53009338)(294.64303345,67.56010061)
\curveto(294.74303416,67.60009331)(294.84303406,67.6100933)(294.94303345,67.59010061)
\lineto(295.07803345,67.59010061)
\curveto(295.14803375,67.57009334)(295.20803369,67.55009336)(295.25803345,67.53010061)
\curveto(295.30803359,67.5100934)(295.34803355,67.47509343)(295.37803345,67.42510061)
\curveto(295.41803348,67.37509353)(295.43803346,67.3050936)(295.43803345,67.21510061)
\lineto(295.43803345,66.94510061)
\lineto(295.43803345,66.04510061)
\lineto(295.43803345,62.53510061)
\lineto(295.43803345,61.47010061)
\curveto(295.43803346,61.39009952)(295.44303346,61.30009961)(295.45303345,61.20010061)
\curveto(295.45303345,61.10009981)(295.44303346,61.01509989)(295.42303345,60.94510061)
\curveto(295.35303355,60.73510017)(295.17303373,60.67010024)(294.88303345,60.75010061)
\curveto(294.84303406,60.76010015)(294.80803409,60.76010015)(294.77803345,60.75010061)
\curveto(294.73803416,60.75010016)(294.69303421,60.76010015)(294.64303345,60.78010061)
\curveto(294.56303434,60.80010011)(294.47803442,60.82010009)(294.38803345,60.84010061)
\curveto(294.2980346,60.86010005)(294.21303469,60.88510002)(294.13303345,60.91510061)
\curveto(293.64303526,61.07509983)(293.22803567,61.27509963)(292.88803345,61.51510061)
\curveto(292.63803626,61.69509921)(292.41303649,61.90009901)(292.21303345,62.13010061)
\curveto(292.0030369,62.36009855)(291.80803709,62.60009831)(291.62803345,62.85010061)
\curveto(291.44803745,63.1100978)(291.27803762,63.37509753)(291.11803345,63.64510061)
\curveto(290.94803795,63.92509698)(290.77303813,64.19509671)(290.59303345,64.45510061)
\curveto(290.51303839,64.56509634)(290.43803846,64.67009624)(290.36803345,64.77010061)
\curveto(290.2980386,64.88009603)(290.22303868,64.99009592)(290.14303345,65.10010061)
\curveto(290.11303879,65.14009577)(290.08303882,65.17509573)(290.05303345,65.20510061)
\curveto(290.01303889,65.24509566)(289.98303892,65.28509562)(289.96303345,65.32510061)
\curveto(289.85303905,65.46509544)(289.72803917,65.59009532)(289.58803345,65.70010061)
\curveto(289.55803934,65.72009519)(289.53303937,65.74509516)(289.51303345,65.77510061)
\curveto(289.48303942,65.8050951)(289.45303945,65.83009508)(289.42303345,65.85010061)
\curveto(289.32303958,65.93009498)(289.22303968,65.99509491)(289.12303345,66.04510061)
\curveto(289.02303988,66.1050948)(288.91303999,66.16009475)(288.79303345,66.21010061)
\curveto(288.72304018,66.24009467)(288.64804025,66.26009465)(288.56803345,66.27010061)
\lineto(288.32803345,66.33010061)
\lineto(288.23803345,66.33010061)
\curveto(288.20804069,66.34009457)(288.17804072,66.34509456)(288.14803345,66.34510061)
\curveto(288.07804082,66.36509454)(287.98304092,66.37009454)(287.86303345,66.36010061)
\curveto(287.73304117,66.36009455)(287.63304127,66.35009456)(287.56303345,66.33010061)
\curveto(287.48304142,66.3100946)(287.40804149,66.29009462)(287.33803345,66.27010061)
\curveto(287.25804164,66.26009465)(287.17804172,66.24009467)(287.09803345,66.21010061)
\curveto(286.85804204,66.10009481)(286.65804224,65.95009496)(286.49803345,65.76010061)
\curveto(286.32804257,65.58009533)(286.18804271,65.36009555)(286.07803345,65.10010061)
\curveto(286.05804284,65.03009588)(286.04304286,64.96009595)(286.03303345,64.89010061)
\curveto(286.01304289,64.82009609)(285.99304291,64.74509616)(285.97303345,64.66510061)
\curveto(285.95304295,64.58509632)(285.94304296,64.47509643)(285.94303345,64.33510061)
\curveto(285.94304296,64.2050967)(285.95304295,64.10009681)(285.97303345,64.02010061)
\curveto(285.98304292,63.96009695)(285.98804291,63.905097)(285.98803345,63.85510061)
\curveto(285.98804291,63.8050971)(285.9980429,63.75509715)(286.01803345,63.70510061)
\curveto(286.05804284,63.6050973)(286.0980428,63.5100974)(286.13803345,63.42010061)
\curveto(286.17804272,63.34009757)(286.22304268,63.26009765)(286.27303345,63.18010061)
\curveto(286.29304261,63.15009776)(286.31804258,63.12009779)(286.34803345,63.09010061)
\curveto(286.37804252,63.07009784)(286.4030425,63.04509786)(286.42303345,63.01510061)
\lineto(286.49803345,62.94010061)
\curveto(286.51804238,62.910098)(286.53804236,62.88509802)(286.55803345,62.86510061)
\lineto(286.76803345,62.71510061)
\curveto(286.82804207,62.67509823)(286.89304201,62.63009828)(286.96303345,62.58010061)
\curveto(287.05304185,62.52009839)(287.15804174,62.47009844)(287.27803345,62.43010061)
\curveto(287.38804151,62.40009851)(287.4980414,62.36509854)(287.60803345,62.32510061)
\curveto(287.71804118,62.28509862)(287.86304104,62.26009865)(288.04303345,62.25010061)
\curveto(288.21304069,62.24009867)(288.33804056,62.2100987)(288.41803345,62.16010061)
\curveto(288.4980404,62.1100988)(288.54304036,62.03509887)(288.55303345,61.93510061)
\curveto(288.56304034,61.83509907)(288.56804033,61.72509918)(288.56803345,61.60510061)
\curveto(288.56804033,61.56509934)(288.57304033,61.52509938)(288.58303345,61.48510061)
\curveto(288.58304032,61.44509946)(288.57804032,61.4100995)(288.56803345,61.38010061)
\curveto(288.54804035,61.33009958)(288.53804036,61.28009963)(288.53803345,61.23010061)
\curveto(288.53804036,61.19009972)(288.52804037,61.15009976)(288.50803345,61.11010061)
\curveto(288.44804045,61.02009989)(288.31304059,60.97509993)(288.10303345,60.97510061)
\lineto(287.98303345,60.97510061)
\curveto(287.92304098,60.98509992)(287.86304104,60.99009992)(287.80303345,60.99010061)
\curveto(287.73304117,61.00009991)(287.66804123,61.0100999)(287.60803345,61.02010061)
\curveto(287.4980414,61.04009987)(287.3980415,61.06009985)(287.30803345,61.08010061)
\curveto(287.20804169,61.10009981)(287.11304179,61.13009978)(287.02303345,61.17010061)
\curveto(286.95304195,61.19009972)(286.89304201,61.2100997)(286.84303345,61.23010061)
\lineto(286.66303345,61.29010061)
\curveto(286.4030425,61.4100995)(286.15804274,61.56509934)(285.92803345,61.75510061)
\curveto(285.6980432,61.95509895)(285.51304339,62.17009874)(285.37303345,62.40010061)
\curveto(285.29304361,62.5100984)(285.22804367,62.62509828)(285.17803345,62.74510061)
\lineto(285.02803345,63.13510061)
\curveto(284.97804392,63.24509766)(284.94804395,63.36009755)(284.93803345,63.48010061)
\curveto(284.91804398,63.60009731)(284.89304401,63.72509718)(284.86303345,63.85510061)
\curveto(284.86304404,63.92509698)(284.86304404,63.99009692)(284.86303345,64.05010061)
\curveto(284.85304405,64.1100968)(284.84304406,64.17509673)(284.83303345,64.24510061)
}
}
{
\newrgbcolor{curcolor}{0 0 0}
\pscustom[linestyle=none,fillstyle=solid,fillcolor=curcolor]
{
\newpath
\moveto(291.11803345,76.34470998)
\curveto(291.23803766,76.37470226)(291.37803752,76.39970223)(291.53803345,76.41970998)
\curveto(291.6980372,76.43970219)(291.86303704,76.44970218)(292.03303345,76.44970998)
\curveto(292.2030367,76.44970218)(292.36803653,76.43970219)(292.52803345,76.41970998)
\curveto(292.68803621,76.39970223)(292.82803607,76.37470226)(292.94803345,76.34470998)
\curveto(293.08803581,76.30470233)(293.21303569,76.26970236)(293.32303345,76.23970998)
\curveto(293.43303547,76.20970242)(293.54303536,76.16970246)(293.65303345,76.11970998)
\curveto(294.29303461,75.84970278)(294.77803412,75.4347032)(295.10803345,74.87470998)
\curveto(295.16803373,74.79470384)(295.21803368,74.70970392)(295.25803345,74.61970998)
\curveto(295.28803361,74.5297041)(295.32303358,74.4297042)(295.36303345,74.31970998)
\curveto(295.41303349,74.20970442)(295.44803345,74.08970454)(295.46803345,73.95970998)
\curveto(295.4980334,73.83970479)(295.52803337,73.70970492)(295.55803345,73.56970998)
\curveto(295.57803332,73.50970512)(295.58303332,73.44970518)(295.57303345,73.38970998)
\curveto(295.56303334,73.33970529)(295.56803333,73.27970535)(295.58803345,73.20970998)
\curveto(295.5980333,73.18970544)(295.5980333,73.16470547)(295.58803345,73.13470998)
\curveto(295.58803331,73.10470553)(295.59303331,73.07970555)(295.60303345,73.05970998)
\lineto(295.60303345,72.90970998)
\curveto(295.61303329,72.83970579)(295.61303329,72.78970584)(295.60303345,72.75970998)
\curveto(295.59303331,72.71970591)(295.58803331,72.67470596)(295.58803345,72.62470998)
\curveto(295.5980333,72.58470605)(295.5980333,72.54470609)(295.58803345,72.50470998)
\curveto(295.56803333,72.41470622)(295.55303335,72.32470631)(295.54303345,72.23470998)
\curveto(295.54303336,72.14470649)(295.53303337,72.05470658)(295.51303345,71.96470998)
\curveto(295.48303342,71.87470676)(295.45803344,71.78470685)(295.43803345,71.69470998)
\curveto(295.41803348,71.60470703)(295.38803351,71.51970711)(295.34803345,71.43970998)
\curveto(295.23803366,71.19970743)(295.10803379,70.97470766)(294.95803345,70.76470998)
\curveto(294.7980341,70.55470808)(294.61803428,70.37470826)(294.41803345,70.22470998)
\curveto(294.24803465,70.10470853)(294.07303483,69.99970863)(293.89303345,69.90970998)
\curveto(293.71303519,69.81970881)(293.52303538,69.7297089)(293.32303345,69.63970998)
\curveto(293.22303568,69.59970903)(293.12303578,69.56470907)(293.02303345,69.53470998)
\curveto(292.91303599,69.51470912)(292.8030361,69.48970914)(292.69303345,69.45970998)
\curveto(292.55303635,69.41970921)(292.41303649,69.39470924)(292.27303345,69.38470998)
\curveto(292.13303677,69.37470926)(291.99303691,69.35470928)(291.85303345,69.32470998)
\curveto(291.74303716,69.31470932)(291.64303726,69.30470933)(291.55303345,69.29470998)
\curveto(291.45303745,69.29470934)(291.35303755,69.28470935)(291.25303345,69.26470998)
\lineto(291.16303345,69.26470998)
\curveto(291.13303777,69.27470936)(291.10803779,69.27470936)(291.08803345,69.26470998)
\lineto(290.87803345,69.26470998)
\curveto(290.81803808,69.24470939)(290.75303815,69.2347094)(290.68303345,69.23470998)
\curveto(290.6030383,69.24470939)(290.52803837,69.24970938)(290.45803345,69.24970998)
\lineto(290.30803345,69.24970998)
\curveto(290.25803864,69.24970938)(290.20803869,69.25470938)(290.15803345,69.26470998)
\lineto(289.78303345,69.26470998)
\curveto(289.75303915,69.27470936)(289.71803918,69.27470936)(289.67803345,69.26470998)
\curveto(289.63803926,69.26470937)(289.5980393,69.26970936)(289.55803345,69.27970998)
\curveto(289.44803945,69.29970933)(289.33803956,69.31470932)(289.22803345,69.32470998)
\curveto(289.10803979,69.3347093)(288.99303991,69.34470929)(288.88303345,69.35470998)
\curveto(288.73304017,69.39470924)(288.58804031,69.41970921)(288.44803345,69.42970998)
\curveto(288.2980406,69.44970918)(288.15304075,69.47970915)(288.01303345,69.51970998)
\curveto(287.71304119,69.60970902)(287.42804147,69.70470893)(287.15803345,69.80470998)
\curveto(286.88804201,69.90470873)(286.63804226,70.0297086)(286.40803345,70.17970998)
\curveto(286.08804281,70.37970825)(285.80804309,70.62470801)(285.56803345,70.91470998)
\curveto(285.32804357,71.20470743)(285.14304376,71.54470709)(285.01303345,71.93470998)
\curveto(284.97304393,72.04470659)(284.94804395,72.15470648)(284.93803345,72.26470998)
\curveto(284.91804398,72.38470625)(284.89304401,72.50470613)(284.86303345,72.62470998)
\curveto(284.85304405,72.69470594)(284.84804405,72.75970587)(284.84803345,72.81970998)
\curveto(284.84804405,72.87970575)(284.84304406,72.94470569)(284.83303345,73.01470998)
\curveto(284.81304409,73.71470492)(284.92804397,74.28970434)(285.17803345,74.73970998)
\curveto(285.42804347,75.18970344)(285.77804312,75.5347031)(286.22803345,75.77470998)
\curveto(286.45804244,75.88470275)(286.73304217,75.98470265)(287.05303345,76.07470998)
\curveto(287.12304178,76.09470254)(287.1980417,76.09470254)(287.27803345,76.07470998)
\curveto(287.34804155,76.06470257)(287.3980415,76.03970259)(287.42803345,75.99970998)
\curveto(287.45804144,75.96970266)(287.48304142,75.90970272)(287.50303345,75.81970998)
\curveto(287.51304139,75.7297029)(287.52304138,75.629703)(287.53303345,75.51970998)
\curveto(287.53304137,75.41970321)(287.52804137,75.31970331)(287.51803345,75.21970998)
\curveto(287.50804139,75.1297035)(287.48804141,75.06470357)(287.45803345,75.02470998)
\curveto(287.38804151,74.91470372)(287.27804162,74.8347038)(287.12803345,74.78470998)
\curveto(286.97804192,74.74470389)(286.84804205,74.68970394)(286.73803345,74.61970998)
\curveto(286.42804247,74.4297042)(286.1980427,74.14970448)(286.04803345,73.77970998)
\curveto(286.01804288,73.70970492)(285.9980429,73.634705)(285.98803345,73.55470998)
\curveto(285.97804292,73.48470515)(285.96304294,73.40970522)(285.94303345,73.32970998)
\curveto(285.93304297,73.27970535)(285.92804297,73.20970542)(285.92803345,73.11970998)
\curveto(285.92804297,73.03970559)(285.93304297,72.97470566)(285.94303345,72.92470998)
\curveto(285.96304294,72.88470575)(285.96804293,72.84970578)(285.95803345,72.81970998)
\curveto(285.94804295,72.78970584)(285.94804295,72.75470588)(285.95803345,72.71470998)
\lineto(286.01803345,72.47470998)
\curveto(286.03804286,72.40470623)(286.06304284,72.3347063)(286.09303345,72.26470998)
\curveto(286.25304265,71.88470675)(286.46304244,71.59470704)(286.72303345,71.39470998)
\curveto(286.98304192,71.20470743)(287.2980416,71.0297076)(287.66803345,70.86970998)
\curveto(287.74804115,70.83970779)(287.82804107,70.81470782)(287.90803345,70.79470998)
\curveto(287.98804091,70.78470785)(288.06804083,70.76470787)(288.14803345,70.73470998)
\curveto(288.25804064,70.70470793)(288.37304053,70.67970795)(288.49303345,70.65970998)
\curveto(288.61304029,70.64970798)(288.73304017,70.629708)(288.85303345,70.59970998)
\curveto(288.90304,70.57970805)(288.95303995,70.56970806)(289.00303345,70.56970998)
\curveto(289.05303985,70.57970805)(289.1030398,70.57470806)(289.15303345,70.55470998)
\curveto(289.21303969,70.54470809)(289.29303961,70.54470809)(289.39303345,70.55470998)
\curveto(289.48303942,70.56470807)(289.53803936,70.57970805)(289.55803345,70.59970998)
\curveto(289.5980393,70.61970801)(289.61803928,70.64970798)(289.61803345,70.68970998)
\curveto(289.61803928,70.73970789)(289.60803929,70.78470785)(289.58803345,70.82470998)
\curveto(289.54803935,70.89470774)(289.5030394,70.95470768)(289.45303345,71.00470998)
\curveto(289.4030395,71.05470758)(289.35303955,71.11470752)(289.30303345,71.18470998)
\lineto(289.24303345,71.24470998)
\curveto(289.21303969,71.27470736)(289.18803971,71.30470733)(289.16803345,71.33470998)
\curveto(289.00803989,71.56470707)(288.87304003,71.83970679)(288.76303345,72.15970998)
\curveto(288.74304016,72.2297064)(288.72804017,72.29970633)(288.71803345,72.36970998)
\curveto(288.70804019,72.43970619)(288.69304021,72.51470612)(288.67303345,72.59470998)
\curveto(288.67304023,72.634706)(288.66804023,72.66970596)(288.65803345,72.69970998)
\curveto(288.64804025,72.7297059)(288.64804025,72.76470587)(288.65803345,72.80470998)
\curveto(288.65804024,72.85470578)(288.64804025,72.89470574)(288.62803345,72.92470998)
\lineto(288.62803345,73.08970998)
\lineto(288.62803345,73.17970998)
\curveto(288.61804028,73.2297054)(288.61804028,73.26970536)(288.62803345,73.29970998)
\curveto(288.63804026,73.34970528)(288.64304026,73.39970523)(288.64303345,73.44970998)
\curveto(288.63304027,73.50970512)(288.63304027,73.56470507)(288.64303345,73.61470998)
\curveto(288.67304023,73.72470491)(288.69304021,73.8297048)(288.70303345,73.92970998)
\curveto(288.71304019,74.03970459)(288.73804016,74.14470449)(288.77803345,74.24470998)
\curveto(288.91803998,74.66470397)(289.1030398,75.00970362)(289.33303345,75.27970998)
\curveto(289.55303935,75.54970308)(289.83803906,75.78970284)(290.18803345,75.99970998)
\curveto(290.32803857,76.07970255)(290.47803842,76.14470249)(290.63803345,76.19470998)
\curveto(290.78803811,76.24470239)(290.94803795,76.29470234)(291.11803345,76.34470998)
\moveto(292.42303345,75.09970998)
\curveto(292.37303653,75.10970352)(292.32803657,75.11470352)(292.28803345,75.11470998)
\lineto(292.13803345,75.11470998)
\curveto(291.82803707,75.11470352)(291.54303736,75.07470356)(291.28303345,74.99470998)
\curveto(291.22303768,74.97470366)(291.16803773,74.95470368)(291.11803345,74.93470998)
\curveto(291.05803784,74.92470371)(291.0030379,74.90970372)(290.95303345,74.88970998)
\curveto(290.46303844,74.66970396)(290.11303879,74.32470431)(289.90303345,73.85470998)
\curveto(289.87303903,73.77470486)(289.84803905,73.69470494)(289.82803345,73.61470998)
\lineto(289.76803345,73.37470998)
\curveto(289.74803915,73.29470534)(289.73803916,73.20470543)(289.73803345,73.10470998)
\lineto(289.73803345,72.78970998)
\curveto(289.75803914,72.76970586)(289.76803913,72.7297059)(289.76803345,72.66970998)
\curveto(289.75803914,72.61970601)(289.75803914,72.57470606)(289.76803345,72.53470998)
\lineto(289.82803345,72.29470998)
\curveto(289.83803906,72.22470641)(289.85803904,72.15470648)(289.88803345,72.08470998)
\curveto(290.14803875,71.48470715)(290.61303829,71.07970755)(291.28303345,70.86970998)
\curveto(291.36303754,70.83970779)(291.44303746,70.81970781)(291.52303345,70.80970998)
\curveto(291.6030373,70.79970783)(291.68803721,70.78470785)(291.77803345,70.76470998)
\lineto(291.92803345,70.76470998)
\curveto(291.96803693,70.75470788)(292.03803686,70.74970788)(292.13803345,70.74970998)
\curveto(292.36803653,70.74970788)(292.56303634,70.76970786)(292.72303345,70.80970998)
\curveto(292.79303611,70.8297078)(292.85803604,70.84470779)(292.91803345,70.85470998)
\curveto(292.97803592,70.86470777)(293.04303586,70.88470775)(293.11303345,70.91470998)
\curveto(293.39303551,71.02470761)(293.63803526,71.16970746)(293.84803345,71.34970998)
\curveto(294.04803485,71.5297071)(294.20803469,71.76470687)(294.32803345,72.05470998)
\lineto(294.41803345,72.29470998)
\lineto(294.47803345,72.53470998)
\curveto(294.4980344,72.58470605)(294.5030344,72.62470601)(294.49303345,72.65470998)
\curveto(294.48303442,72.69470594)(294.48803441,72.73970589)(294.50803345,72.78970998)
\curveto(294.51803438,72.81970581)(294.52303438,72.87470576)(294.52303345,72.95470998)
\curveto(294.52303438,73.0347056)(294.51803438,73.09470554)(294.50803345,73.13470998)
\curveto(294.48803441,73.24470539)(294.47303443,73.34970528)(294.46303345,73.44970998)
\curveto(294.45303445,73.54970508)(294.42303448,73.64470499)(294.37303345,73.73470998)
\curveto(294.17303473,74.26470437)(293.7980351,74.65470398)(293.24803345,74.90470998)
\curveto(293.14803575,74.94470369)(293.04303586,74.97470366)(292.93303345,74.99470998)
\lineto(292.60303345,75.08470998)
\curveto(292.52303638,75.08470355)(292.46303644,75.08970354)(292.42303345,75.09970998)
}
}
{
\newrgbcolor{curcolor}{0 0 0}
\pscustom[linestyle=none,fillstyle=solid,fillcolor=curcolor]
{
\newpath
\moveto(293.80303345,78.63431936)
\lineto(293.80303345,79.26431936)
\lineto(293.80303345,79.45931936)
\curveto(293.8030351,79.52931683)(293.81303509,79.58931677)(293.83303345,79.63931936)
\curveto(293.87303503,79.70931665)(293.91303499,79.7593166)(293.95303345,79.78931936)
\curveto(294.0030349,79.82931653)(294.06803483,79.84931651)(294.14803345,79.84931936)
\curveto(294.22803467,79.8593165)(294.31303459,79.86431649)(294.40303345,79.86431936)
\lineto(295.12303345,79.86431936)
\curveto(295.6030333,79.86431649)(296.01303289,79.80431655)(296.35303345,79.68431936)
\curveto(296.69303221,79.56431679)(296.96803193,79.36931699)(297.17803345,79.09931936)
\curveto(297.22803167,79.02931733)(297.27303163,78.9593174)(297.31303345,78.88931936)
\curveto(297.36303154,78.82931753)(297.40803149,78.7543176)(297.44803345,78.66431936)
\curveto(297.45803144,78.64431771)(297.46803143,78.61431774)(297.47803345,78.57431936)
\curveto(297.4980314,78.53431782)(297.5030314,78.48931787)(297.49303345,78.43931936)
\curveto(297.46303144,78.34931801)(297.38803151,78.29431806)(297.26803345,78.27431936)
\curveto(297.15803174,78.2543181)(297.06303184,78.26931809)(296.98303345,78.31931936)
\curveto(296.91303199,78.34931801)(296.84803205,78.39431796)(296.78803345,78.45431936)
\curveto(296.73803216,78.52431783)(296.68803221,78.58931777)(296.63803345,78.64931936)
\curveto(296.58803231,78.71931764)(296.51303239,78.77931758)(296.41303345,78.82931936)
\curveto(296.32303258,78.88931747)(296.23303267,78.93931742)(296.14303345,78.97931936)
\curveto(296.11303279,78.99931736)(296.05303285,79.02431733)(295.96303345,79.05431936)
\curveto(295.88303302,79.08431727)(295.81303309,79.08931727)(295.75303345,79.06931936)
\curveto(295.61303329,79.03931732)(295.52303338,78.97931738)(295.48303345,78.88931936)
\curveto(295.45303345,78.80931755)(295.43803346,78.71931764)(295.43803345,78.61931936)
\curveto(295.43803346,78.51931784)(295.41303349,78.43431792)(295.36303345,78.36431936)
\curveto(295.29303361,78.27431808)(295.15303375,78.22931813)(294.94303345,78.22931936)
\lineto(294.38803345,78.22931936)
\lineto(294.16303345,78.22931936)
\curveto(294.08303482,78.23931812)(294.01803488,78.2593181)(293.96803345,78.28931936)
\curveto(293.88803501,78.34931801)(293.84303506,78.41931794)(293.83303345,78.49931936)
\curveto(293.82303508,78.51931784)(293.81803508,78.53931782)(293.81803345,78.55931936)
\curveto(293.81803508,78.58931777)(293.81303509,78.61431774)(293.80303345,78.63431936)
}
}
{
\newrgbcolor{curcolor}{0 0 0}
\pscustom[linestyle=none,fillstyle=solid,fillcolor=curcolor]
{
}
}
{
\newrgbcolor{curcolor}{0 0 0}
\pscustom[linestyle=none,fillstyle=solid,fillcolor=curcolor]
{
\newpath
\moveto(284.83303345,89.26463186)
\curveto(284.82304408,89.95462722)(284.94304396,90.55462662)(285.19303345,91.06463186)
\curveto(285.44304346,91.58462559)(285.77804312,91.9796252)(286.19803345,92.24963186)
\curveto(286.27804262,92.29962488)(286.36804253,92.34462483)(286.46803345,92.38463186)
\curveto(286.55804234,92.42462475)(286.65304225,92.46962471)(286.75303345,92.51963186)
\curveto(286.85304205,92.55962462)(286.95304195,92.58962459)(287.05303345,92.60963186)
\curveto(287.15304175,92.62962455)(287.25804164,92.64962453)(287.36803345,92.66963186)
\curveto(287.41804148,92.68962449)(287.46304144,92.69462448)(287.50303345,92.68463186)
\curveto(287.54304136,92.6746245)(287.58804131,92.6796245)(287.63803345,92.69963186)
\curveto(287.68804121,92.70962447)(287.77304113,92.71462446)(287.89303345,92.71463186)
\curveto(288.0030409,92.71462446)(288.08804081,92.70962447)(288.14803345,92.69963186)
\curveto(288.20804069,92.6796245)(288.26804063,92.66962451)(288.32803345,92.66963186)
\curveto(288.38804051,92.6796245)(288.44804045,92.6746245)(288.50803345,92.65463186)
\curveto(288.64804025,92.61462456)(288.78304012,92.5796246)(288.91303345,92.54963186)
\curveto(289.04303986,92.51962466)(289.16803973,92.4796247)(289.28803345,92.42963186)
\curveto(289.42803947,92.36962481)(289.55303935,92.29962488)(289.66303345,92.21963186)
\curveto(289.77303913,92.14962503)(289.88303902,92.0746251)(289.99303345,91.99463186)
\lineto(290.05303345,91.93463186)
\curveto(290.07303883,91.92462525)(290.09303881,91.90962527)(290.11303345,91.88963186)
\curveto(290.27303863,91.76962541)(290.41803848,91.63462554)(290.54803345,91.48463186)
\curveto(290.67803822,91.33462584)(290.8030381,91.174626)(290.92303345,91.00463186)
\curveto(291.14303776,90.69462648)(291.34803755,90.39962678)(291.53803345,90.11963186)
\curveto(291.67803722,89.88962729)(291.81303709,89.65962752)(291.94303345,89.42963186)
\curveto(292.07303683,89.20962797)(292.20803669,88.98962819)(292.34803345,88.76963186)
\curveto(292.51803638,88.51962866)(292.6980362,88.2796289)(292.88803345,88.04963186)
\curveto(293.07803582,87.82962935)(293.3030356,87.63962954)(293.56303345,87.47963186)
\curveto(293.62303528,87.43962974)(293.68303522,87.40462977)(293.74303345,87.37463186)
\curveto(293.79303511,87.34462983)(293.85803504,87.31462986)(293.93803345,87.28463186)
\curveto(294.00803489,87.26462991)(294.06803483,87.25962992)(294.11803345,87.26963186)
\curveto(294.18803471,87.28962989)(294.24303466,87.32462985)(294.28303345,87.37463186)
\curveto(294.31303459,87.42462975)(294.33303457,87.48462969)(294.34303345,87.55463186)
\lineto(294.34303345,87.79463186)
\lineto(294.34303345,88.54463186)
\lineto(294.34303345,91.34963186)
\lineto(294.34303345,92.00963186)
\curveto(294.34303456,92.09962508)(294.34803455,92.18462499)(294.35803345,92.26463186)
\curveto(294.35803454,92.34462483)(294.37803452,92.40962477)(294.41803345,92.45963186)
\curveto(294.45803444,92.50962467)(294.53303437,92.54962463)(294.64303345,92.57963186)
\curveto(294.74303416,92.61962456)(294.84303406,92.62962455)(294.94303345,92.60963186)
\lineto(295.07803345,92.60963186)
\curveto(295.14803375,92.58962459)(295.20803369,92.56962461)(295.25803345,92.54963186)
\curveto(295.30803359,92.52962465)(295.34803355,92.49462468)(295.37803345,92.44463186)
\curveto(295.41803348,92.39462478)(295.43803346,92.32462485)(295.43803345,92.23463186)
\lineto(295.43803345,91.96463186)
\lineto(295.43803345,91.06463186)
\lineto(295.43803345,87.55463186)
\lineto(295.43803345,86.48963186)
\curveto(295.43803346,86.40963077)(295.44303346,86.31963086)(295.45303345,86.21963186)
\curveto(295.45303345,86.11963106)(295.44303346,86.03463114)(295.42303345,85.96463186)
\curveto(295.35303355,85.75463142)(295.17303373,85.68963149)(294.88303345,85.76963186)
\curveto(294.84303406,85.7796314)(294.80803409,85.7796314)(294.77803345,85.76963186)
\curveto(294.73803416,85.76963141)(294.69303421,85.7796314)(294.64303345,85.79963186)
\curveto(294.56303434,85.81963136)(294.47803442,85.83963134)(294.38803345,85.85963186)
\curveto(294.2980346,85.8796313)(294.21303469,85.90463127)(294.13303345,85.93463186)
\curveto(293.64303526,86.09463108)(293.22803567,86.29463088)(292.88803345,86.53463186)
\curveto(292.63803626,86.71463046)(292.41303649,86.91963026)(292.21303345,87.14963186)
\curveto(292.0030369,87.3796298)(291.80803709,87.61962956)(291.62803345,87.86963186)
\curveto(291.44803745,88.12962905)(291.27803762,88.39462878)(291.11803345,88.66463186)
\curveto(290.94803795,88.94462823)(290.77303813,89.21462796)(290.59303345,89.47463186)
\curveto(290.51303839,89.58462759)(290.43803846,89.68962749)(290.36803345,89.78963186)
\curveto(290.2980386,89.89962728)(290.22303868,90.00962717)(290.14303345,90.11963186)
\curveto(290.11303879,90.15962702)(290.08303882,90.19462698)(290.05303345,90.22463186)
\curveto(290.01303889,90.26462691)(289.98303892,90.30462687)(289.96303345,90.34463186)
\curveto(289.85303905,90.48462669)(289.72803917,90.60962657)(289.58803345,90.71963186)
\curveto(289.55803934,90.73962644)(289.53303937,90.76462641)(289.51303345,90.79463186)
\curveto(289.48303942,90.82462635)(289.45303945,90.84962633)(289.42303345,90.86963186)
\curveto(289.32303958,90.94962623)(289.22303968,91.01462616)(289.12303345,91.06463186)
\curveto(289.02303988,91.12462605)(288.91303999,91.179626)(288.79303345,91.22963186)
\curveto(288.72304018,91.25962592)(288.64804025,91.2796259)(288.56803345,91.28963186)
\lineto(288.32803345,91.34963186)
\lineto(288.23803345,91.34963186)
\curveto(288.20804069,91.35962582)(288.17804072,91.36462581)(288.14803345,91.36463186)
\curveto(288.07804082,91.38462579)(287.98304092,91.38962579)(287.86303345,91.37963186)
\curveto(287.73304117,91.3796258)(287.63304127,91.36962581)(287.56303345,91.34963186)
\curveto(287.48304142,91.32962585)(287.40804149,91.30962587)(287.33803345,91.28963186)
\curveto(287.25804164,91.2796259)(287.17804172,91.25962592)(287.09803345,91.22963186)
\curveto(286.85804204,91.11962606)(286.65804224,90.96962621)(286.49803345,90.77963186)
\curveto(286.32804257,90.59962658)(286.18804271,90.3796268)(286.07803345,90.11963186)
\curveto(286.05804284,90.04962713)(286.04304286,89.9796272)(286.03303345,89.90963186)
\curveto(286.01304289,89.83962734)(285.99304291,89.76462741)(285.97303345,89.68463186)
\curveto(285.95304295,89.60462757)(285.94304296,89.49462768)(285.94303345,89.35463186)
\curveto(285.94304296,89.22462795)(285.95304295,89.11962806)(285.97303345,89.03963186)
\curveto(285.98304292,88.9796282)(285.98804291,88.92462825)(285.98803345,88.87463186)
\curveto(285.98804291,88.82462835)(285.9980429,88.7746284)(286.01803345,88.72463186)
\curveto(286.05804284,88.62462855)(286.0980428,88.52962865)(286.13803345,88.43963186)
\curveto(286.17804272,88.35962882)(286.22304268,88.2796289)(286.27303345,88.19963186)
\curveto(286.29304261,88.16962901)(286.31804258,88.13962904)(286.34803345,88.10963186)
\curveto(286.37804252,88.08962909)(286.4030425,88.06462911)(286.42303345,88.03463186)
\lineto(286.49803345,87.95963186)
\curveto(286.51804238,87.92962925)(286.53804236,87.90462927)(286.55803345,87.88463186)
\lineto(286.76803345,87.73463186)
\curveto(286.82804207,87.69462948)(286.89304201,87.64962953)(286.96303345,87.59963186)
\curveto(287.05304185,87.53962964)(287.15804174,87.48962969)(287.27803345,87.44963186)
\curveto(287.38804151,87.41962976)(287.4980414,87.38462979)(287.60803345,87.34463186)
\curveto(287.71804118,87.30462987)(287.86304104,87.2796299)(288.04303345,87.26963186)
\curveto(288.21304069,87.25962992)(288.33804056,87.22962995)(288.41803345,87.17963186)
\curveto(288.4980404,87.12963005)(288.54304036,87.05463012)(288.55303345,86.95463186)
\curveto(288.56304034,86.85463032)(288.56804033,86.74463043)(288.56803345,86.62463186)
\curveto(288.56804033,86.58463059)(288.57304033,86.54463063)(288.58303345,86.50463186)
\curveto(288.58304032,86.46463071)(288.57804032,86.42963075)(288.56803345,86.39963186)
\curveto(288.54804035,86.34963083)(288.53804036,86.29963088)(288.53803345,86.24963186)
\curveto(288.53804036,86.20963097)(288.52804037,86.16963101)(288.50803345,86.12963186)
\curveto(288.44804045,86.03963114)(288.31304059,85.99463118)(288.10303345,85.99463186)
\lineto(287.98303345,85.99463186)
\curveto(287.92304098,86.00463117)(287.86304104,86.00963117)(287.80303345,86.00963186)
\curveto(287.73304117,86.01963116)(287.66804123,86.02963115)(287.60803345,86.03963186)
\curveto(287.4980414,86.05963112)(287.3980415,86.0796311)(287.30803345,86.09963186)
\curveto(287.20804169,86.11963106)(287.11304179,86.14963103)(287.02303345,86.18963186)
\curveto(286.95304195,86.20963097)(286.89304201,86.22963095)(286.84303345,86.24963186)
\lineto(286.66303345,86.30963186)
\curveto(286.4030425,86.42963075)(286.15804274,86.58463059)(285.92803345,86.77463186)
\curveto(285.6980432,86.9746302)(285.51304339,87.18962999)(285.37303345,87.41963186)
\curveto(285.29304361,87.52962965)(285.22804367,87.64462953)(285.17803345,87.76463186)
\lineto(285.02803345,88.15463186)
\curveto(284.97804392,88.26462891)(284.94804395,88.3796288)(284.93803345,88.49963186)
\curveto(284.91804398,88.61962856)(284.89304401,88.74462843)(284.86303345,88.87463186)
\curveto(284.86304404,88.94462823)(284.86304404,89.00962817)(284.86303345,89.06963186)
\curveto(284.85304405,89.12962805)(284.84304406,89.19462798)(284.83303345,89.26463186)
}
}
{
\newrgbcolor{curcolor}{0 0 0}
\pscustom[linestyle=none,fillstyle=solid,fillcolor=curcolor]
{
\newpath
\moveto(290.35303345,101.36424123)
\lineto(290.60803345,101.36424123)
\curveto(290.68803821,101.37423353)(290.76303814,101.36923353)(290.83303345,101.34924123)
\lineto(291.07303345,101.34924123)
\lineto(291.23803345,101.34924123)
\curveto(291.33803756,101.32923357)(291.44303746,101.31923358)(291.55303345,101.31924123)
\curveto(291.65303725,101.31923358)(291.75303715,101.30923359)(291.85303345,101.28924123)
\lineto(292.00303345,101.28924123)
\curveto(292.14303676,101.25923364)(292.28303662,101.23923366)(292.42303345,101.22924123)
\curveto(292.55303635,101.21923368)(292.68303622,101.19423371)(292.81303345,101.15424123)
\curveto(292.89303601,101.13423377)(292.97803592,101.11423379)(293.06803345,101.09424123)
\lineto(293.30803345,101.03424123)
\lineto(293.60803345,100.91424123)
\curveto(293.6980352,100.88423402)(293.78803511,100.84923405)(293.87803345,100.80924123)
\curveto(294.0980348,100.70923419)(294.31303459,100.57423433)(294.52303345,100.40424123)
\curveto(294.73303417,100.24423466)(294.903034,100.06923483)(295.03303345,99.87924123)
\curveto(295.07303383,99.82923507)(295.11303379,99.76923513)(295.15303345,99.69924123)
\curveto(295.18303372,99.63923526)(295.21803368,99.57923532)(295.25803345,99.51924123)
\curveto(295.30803359,99.43923546)(295.34803355,99.34423556)(295.37803345,99.23424123)
\curveto(295.40803349,99.12423578)(295.43803346,99.01923588)(295.46803345,98.91924123)
\curveto(295.50803339,98.80923609)(295.53303337,98.6992362)(295.54303345,98.58924123)
\curveto(295.55303335,98.47923642)(295.56803333,98.36423654)(295.58803345,98.24424123)
\curveto(295.5980333,98.2042367)(295.5980333,98.15923674)(295.58803345,98.10924123)
\curveto(295.58803331,98.06923683)(295.59303331,98.02923687)(295.60303345,97.98924123)
\curveto(295.61303329,97.94923695)(295.61803328,97.89423701)(295.61803345,97.82424123)
\curveto(295.61803328,97.75423715)(295.61303329,97.7042372)(295.60303345,97.67424123)
\curveto(295.58303332,97.62423728)(295.57803332,97.57923732)(295.58803345,97.53924123)
\curveto(295.5980333,97.4992374)(295.5980333,97.46423744)(295.58803345,97.43424123)
\lineto(295.58803345,97.34424123)
\curveto(295.56803333,97.28423762)(295.55303335,97.21923768)(295.54303345,97.14924123)
\curveto(295.54303336,97.08923781)(295.53803336,97.02423788)(295.52803345,96.95424123)
\curveto(295.47803342,96.78423812)(295.42803347,96.62423828)(295.37803345,96.47424123)
\curveto(295.32803357,96.32423858)(295.26303364,96.17923872)(295.18303345,96.03924123)
\curveto(295.14303376,95.98923891)(295.11303379,95.93423897)(295.09303345,95.87424123)
\curveto(295.06303384,95.82423908)(295.02803387,95.77423913)(294.98803345,95.72424123)
\curveto(294.80803409,95.48423942)(294.58803431,95.28423962)(294.32803345,95.12424123)
\curveto(294.06803483,94.96423994)(293.78303512,94.82424008)(293.47303345,94.70424123)
\curveto(293.33303557,94.64424026)(293.19303571,94.5992403)(293.05303345,94.56924123)
\curveto(292.903036,94.53924036)(292.74803615,94.5042404)(292.58803345,94.46424123)
\curveto(292.47803642,94.44424046)(292.36803653,94.42924047)(292.25803345,94.41924123)
\curveto(292.14803675,94.40924049)(292.03803686,94.39424051)(291.92803345,94.37424123)
\curveto(291.88803701,94.36424054)(291.84803705,94.35924054)(291.80803345,94.35924123)
\curveto(291.76803713,94.36924053)(291.72803717,94.36924053)(291.68803345,94.35924123)
\curveto(291.63803726,94.34924055)(291.58803731,94.34424056)(291.53803345,94.34424123)
\lineto(291.37303345,94.34424123)
\curveto(291.32303758,94.32424058)(291.27303763,94.31924058)(291.22303345,94.32924123)
\curveto(291.16303774,94.33924056)(291.10803779,94.33924056)(291.05803345,94.32924123)
\curveto(291.01803788,94.31924058)(290.97303793,94.31924058)(290.92303345,94.32924123)
\curveto(290.87303803,94.33924056)(290.82303808,94.33424057)(290.77303345,94.31424123)
\curveto(290.7030382,94.29424061)(290.62803827,94.28924061)(290.54803345,94.29924123)
\curveto(290.45803844,94.30924059)(290.37303853,94.31424059)(290.29303345,94.31424123)
\curveto(290.2030387,94.31424059)(290.1030388,94.30924059)(289.99303345,94.29924123)
\curveto(289.87303903,94.28924061)(289.77303913,94.29424061)(289.69303345,94.31424123)
\lineto(289.40803345,94.31424123)
\lineto(288.77803345,94.35924123)
\curveto(288.67804022,94.36924053)(288.58304032,94.37924052)(288.49303345,94.38924123)
\lineto(288.19303345,94.41924123)
\curveto(288.14304076,94.43924046)(288.09304081,94.44424046)(288.04303345,94.43424123)
\curveto(287.98304092,94.43424047)(287.92804097,94.44424046)(287.87803345,94.46424123)
\curveto(287.70804119,94.51424039)(287.54304136,94.55424035)(287.38303345,94.58424123)
\curveto(287.21304169,94.61424029)(287.05304185,94.66424024)(286.90303345,94.73424123)
\curveto(286.44304246,94.92423998)(286.06804283,95.14423976)(285.77803345,95.39424123)
\curveto(285.48804341,95.65423925)(285.24304366,96.01423889)(285.04303345,96.47424123)
\curveto(284.99304391,96.6042383)(284.95804394,96.73423817)(284.93803345,96.86424123)
\curveto(284.91804398,97.0042379)(284.89304401,97.14423776)(284.86303345,97.28424123)
\curveto(284.85304405,97.35423755)(284.84804405,97.41923748)(284.84803345,97.47924123)
\curveto(284.84804405,97.53923736)(284.84304406,97.6042373)(284.83303345,97.67424123)
\curveto(284.81304409,98.5042364)(284.96304394,99.17423573)(285.28303345,99.68424123)
\curveto(285.59304331,100.19423471)(286.03304287,100.57423433)(286.60303345,100.82424123)
\curveto(286.72304218,100.87423403)(286.84804205,100.91923398)(286.97803345,100.95924123)
\curveto(287.10804179,100.9992339)(287.24304166,101.04423386)(287.38303345,101.09424123)
\curveto(287.46304144,101.11423379)(287.54804135,101.12923377)(287.63803345,101.13924123)
\lineto(287.87803345,101.19924123)
\curveto(287.98804091,101.22923367)(288.0980408,101.24423366)(288.20803345,101.24424123)
\curveto(288.31804058,101.25423365)(288.42804047,101.26923363)(288.53803345,101.28924123)
\curveto(288.58804031,101.30923359)(288.63304027,101.31423359)(288.67303345,101.30424123)
\curveto(288.71304019,101.3042336)(288.75304015,101.30923359)(288.79303345,101.31924123)
\curveto(288.84304006,101.32923357)(288.89804,101.32923357)(288.95803345,101.31924123)
\curveto(289.00803989,101.31923358)(289.05803984,101.32423358)(289.10803345,101.33424123)
\lineto(289.24303345,101.33424123)
\curveto(289.3030396,101.35423355)(289.37303953,101.35423355)(289.45303345,101.33424123)
\curveto(289.52303938,101.32423358)(289.58803931,101.32923357)(289.64803345,101.34924123)
\curveto(289.67803922,101.35923354)(289.71803918,101.36423354)(289.76803345,101.36424123)
\lineto(289.88803345,101.36424123)
\lineto(290.35303345,101.36424123)
\moveto(292.67803345,99.81924123)
\curveto(292.35803654,99.91923498)(291.99303691,99.97923492)(291.58303345,99.99924123)
\curveto(291.17303773,100.01923488)(290.76303814,100.02923487)(290.35303345,100.02924123)
\curveto(289.92303898,100.02923487)(289.5030394,100.01923488)(289.09303345,99.99924123)
\curveto(288.68304022,99.97923492)(288.2980406,99.93423497)(287.93803345,99.86424123)
\curveto(287.57804132,99.79423511)(287.25804164,99.68423522)(286.97803345,99.53424123)
\curveto(286.68804221,99.39423551)(286.45304245,99.1992357)(286.27303345,98.94924123)
\curveto(286.16304274,98.78923611)(286.08304282,98.60923629)(286.03303345,98.40924123)
\curveto(285.97304293,98.20923669)(285.94304296,97.96423694)(285.94303345,97.67424123)
\curveto(285.96304294,97.65423725)(285.97304293,97.61923728)(285.97303345,97.56924123)
\curveto(285.96304294,97.51923738)(285.96304294,97.47923742)(285.97303345,97.44924123)
\curveto(285.99304291,97.36923753)(286.01304289,97.29423761)(286.03303345,97.22424123)
\curveto(286.04304286,97.16423774)(286.06304284,97.0992378)(286.09303345,97.02924123)
\curveto(286.21304269,96.75923814)(286.38304252,96.53923836)(286.60303345,96.36924123)
\curveto(286.81304209,96.20923869)(287.05804184,96.07423883)(287.33803345,95.96424123)
\curveto(287.44804145,95.91423899)(287.56804133,95.87423903)(287.69803345,95.84424123)
\curveto(287.81804108,95.82423908)(287.94304096,95.7992391)(288.07303345,95.76924123)
\curveto(288.12304078,95.74923915)(288.17804072,95.73923916)(288.23803345,95.73924123)
\curveto(288.28804061,95.73923916)(288.33804056,95.73423917)(288.38803345,95.72424123)
\curveto(288.47804042,95.71423919)(288.57304033,95.7042392)(288.67303345,95.69424123)
\curveto(288.76304014,95.68423922)(288.85804004,95.67423923)(288.95803345,95.66424123)
\curveto(289.03803986,95.66423924)(289.12303978,95.65923924)(289.21303345,95.64924123)
\lineto(289.45303345,95.64924123)
\lineto(289.63303345,95.64924123)
\curveto(289.66303924,95.63923926)(289.6980392,95.63423927)(289.73803345,95.63424123)
\lineto(289.87303345,95.63424123)
\lineto(290.32303345,95.63424123)
\curveto(290.4030385,95.63423927)(290.48803841,95.62923927)(290.57803345,95.61924123)
\curveto(290.65803824,95.61923928)(290.73303817,95.62923927)(290.80303345,95.64924123)
\lineto(291.07303345,95.64924123)
\curveto(291.09303781,95.64923925)(291.12303778,95.64423926)(291.16303345,95.63424123)
\curveto(291.19303771,95.63423927)(291.21803768,95.63923926)(291.23803345,95.64924123)
\curveto(291.33803756,95.65923924)(291.43803746,95.66423924)(291.53803345,95.66424123)
\curveto(291.62803727,95.67423923)(291.72803717,95.68423922)(291.83803345,95.69424123)
\curveto(291.95803694,95.72423918)(292.08303682,95.73923916)(292.21303345,95.73924123)
\curveto(292.33303657,95.74923915)(292.44803645,95.77423913)(292.55803345,95.81424123)
\curveto(292.85803604,95.89423901)(293.12303578,95.97923892)(293.35303345,96.06924123)
\curveto(293.58303532,96.16923873)(293.7980351,96.31423859)(293.99803345,96.50424123)
\curveto(294.1980347,96.71423819)(294.34803455,96.97923792)(294.44803345,97.29924123)
\curveto(294.46803443,97.33923756)(294.47803442,97.37423753)(294.47803345,97.40424123)
\curveto(294.46803443,97.44423746)(294.47303443,97.48923741)(294.49303345,97.53924123)
\curveto(294.5030344,97.57923732)(294.51303439,97.64923725)(294.52303345,97.74924123)
\curveto(294.53303437,97.85923704)(294.52803437,97.94423696)(294.50803345,98.00424123)
\curveto(294.48803441,98.07423683)(294.47803442,98.14423676)(294.47803345,98.21424123)
\curveto(294.46803443,98.28423662)(294.45303445,98.34923655)(294.43303345,98.40924123)
\curveto(294.37303453,98.60923629)(294.28803461,98.78923611)(294.17803345,98.94924123)
\curveto(294.15803474,98.97923592)(294.13803476,99.0042359)(294.11803345,99.02424123)
\lineto(294.05803345,99.08424123)
\curveto(294.03803486,99.12423578)(293.9980349,99.17423573)(293.93803345,99.23424123)
\curveto(293.7980351,99.33423557)(293.66803523,99.41923548)(293.54803345,99.48924123)
\curveto(293.42803547,99.55923534)(293.28303562,99.62923527)(293.11303345,99.69924123)
\curveto(293.04303586,99.72923517)(292.97303593,99.74923515)(292.90303345,99.75924123)
\curveto(292.83303607,99.77923512)(292.75803614,99.7992351)(292.67803345,99.81924123)
}
}
{
\newrgbcolor{curcolor}{0 0 0}
\pscustom[linestyle=none,fillstyle=solid,fillcolor=curcolor]
{
\newpath
\moveto(284.83303345,106.77385061)
\curveto(284.83304407,106.87384575)(284.84304406,106.96884566)(284.86303345,107.05885061)
\curveto(284.87304403,107.14884548)(284.903044,107.21384541)(284.95303345,107.25385061)
\curveto(285.03304387,107.31384531)(285.13804376,107.34384528)(285.26803345,107.34385061)
\lineto(285.65803345,107.34385061)
\lineto(287.15803345,107.34385061)
\lineto(293.54803345,107.34385061)
\lineto(294.71803345,107.34385061)
\lineto(295.03303345,107.34385061)
\curveto(295.13303377,107.35384527)(295.21303369,107.33884529)(295.27303345,107.29885061)
\curveto(295.35303355,107.24884538)(295.4030335,107.17384545)(295.42303345,107.07385061)
\curveto(295.43303347,106.98384564)(295.43803346,106.87384575)(295.43803345,106.74385061)
\lineto(295.43803345,106.51885061)
\curveto(295.41803348,106.43884619)(295.4030335,106.36884626)(295.39303345,106.30885061)
\curveto(295.37303353,106.24884638)(295.33303357,106.19884643)(295.27303345,106.15885061)
\curveto(295.21303369,106.11884651)(295.13803376,106.09884653)(295.04803345,106.09885061)
\lineto(294.74803345,106.09885061)
\lineto(293.65303345,106.09885061)
\lineto(288.31303345,106.09885061)
\curveto(288.22304068,106.07884655)(288.14804075,106.06384656)(288.08803345,106.05385061)
\curveto(288.01804088,106.05384657)(287.95804094,106.0238466)(287.90803345,105.96385061)
\curveto(287.85804104,105.89384673)(287.83304107,105.80384682)(287.83303345,105.69385061)
\curveto(287.82304108,105.59384703)(287.81804108,105.48384714)(287.81803345,105.36385061)
\lineto(287.81803345,104.22385061)
\lineto(287.81803345,103.72885061)
\curveto(287.80804109,103.56884906)(287.74804115,103.45884917)(287.63803345,103.39885061)
\curveto(287.60804129,103.37884925)(287.57804132,103.36884926)(287.54803345,103.36885061)
\curveto(287.50804139,103.36884926)(287.46304144,103.36384926)(287.41303345,103.35385061)
\curveto(287.29304161,103.33384929)(287.18304172,103.33884929)(287.08303345,103.36885061)
\curveto(286.98304192,103.40884922)(286.91304199,103.46384916)(286.87303345,103.53385061)
\curveto(286.82304208,103.61384901)(286.7980421,103.73384889)(286.79803345,103.89385061)
\curveto(286.7980421,104.05384857)(286.78304212,104.18884844)(286.75303345,104.29885061)
\curveto(286.74304216,104.34884828)(286.73804216,104.40384822)(286.73803345,104.46385061)
\curveto(286.72804217,104.5238481)(286.71304219,104.58384804)(286.69303345,104.64385061)
\curveto(286.64304226,104.79384783)(286.59304231,104.93884769)(286.54303345,105.07885061)
\curveto(286.48304242,105.21884741)(286.41304249,105.35384727)(286.33303345,105.48385061)
\curveto(286.24304266,105.623847)(286.13804276,105.74384688)(286.01803345,105.84385061)
\curveto(285.898043,105.94384668)(285.76804313,106.03884659)(285.62803345,106.12885061)
\curveto(285.52804337,106.18884644)(285.41804348,106.23384639)(285.29803345,106.26385061)
\curveto(285.17804372,106.30384632)(285.07304383,106.35384627)(284.98303345,106.41385061)
\curveto(284.92304398,106.46384616)(284.88304402,106.53384609)(284.86303345,106.62385061)
\curveto(284.85304405,106.64384598)(284.84804405,106.66884596)(284.84803345,106.69885061)
\curveto(284.84804405,106.7288459)(284.84304406,106.75384587)(284.83303345,106.77385061)
}
}
{
\newrgbcolor{curcolor}{0 0 0}
\pscustom[linestyle=none,fillstyle=solid,fillcolor=curcolor]
{
\newpath
\moveto(284.83303345,115.12345998)
\curveto(284.83304407,115.22345513)(284.84304406,115.31845503)(284.86303345,115.40845998)
\curveto(284.87304403,115.49845485)(284.903044,115.56345479)(284.95303345,115.60345998)
\curveto(285.03304387,115.66345469)(285.13804376,115.69345466)(285.26803345,115.69345998)
\lineto(285.65803345,115.69345998)
\lineto(287.15803345,115.69345998)
\lineto(293.54803345,115.69345998)
\lineto(294.71803345,115.69345998)
\lineto(295.03303345,115.69345998)
\curveto(295.13303377,115.70345465)(295.21303369,115.68845466)(295.27303345,115.64845998)
\curveto(295.35303355,115.59845475)(295.4030335,115.52345483)(295.42303345,115.42345998)
\curveto(295.43303347,115.33345502)(295.43803346,115.22345513)(295.43803345,115.09345998)
\lineto(295.43803345,114.86845998)
\curveto(295.41803348,114.78845556)(295.4030335,114.71845563)(295.39303345,114.65845998)
\curveto(295.37303353,114.59845575)(295.33303357,114.5484558)(295.27303345,114.50845998)
\curveto(295.21303369,114.46845588)(295.13803376,114.4484559)(295.04803345,114.44845998)
\lineto(294.74803345,114.44845998)
\lineto(293.65303345,114.44845998)
\lineto(288.31303345,114.44845998)
\curveto(288.22304068,114.42845592)(288.14804075,114.41345594)(288.08803345,114.40345998)
\curveto(288.01804088,114.40345595)(287.95804094,114.37345598)(287.90803345,114.31345998)
\curveto(287.85804104,114.24345611)(287.83304107,114.1534562)(287.83303345,114.04345998)
\curveto(287.82304108,113.94345641)(287.81804108,113.83345652)(287.81803345,113.71345998)
\lineto(287.81803345,112.57345998)
\lineto(287.81803345,112.07845998)
\curveto(287.80804109,111.91845843)(287.74804115,111.80845854)(287.63803345,111.74845998)
\curveto(287.60804129,111.72845862)(287.57804132,111.71845863)(287.54803345,111.71845998)
\curveto(287.50804139,111.71845863)(287.46304144,111.71345864)(287.41303345,111.70345998)
\curveto(287.29304161,111.68345867)(287.18304172,111.68845866)(287.08303345,111.71845998)
\curveto(286.98304192,111.75845859)(286.91304199,111.81345854)(286.87303345,111.88345998)
\curveto(286.82304208,111.96345839)(286.7980421,112.08345827)(286.79803345,112.24345998)
\curveto(286.7980421,112.40345795)(286.78304212,112.53845781)(286.75303345,112.64845998)
\curveto(286.74304216,112.69845765)(286.73804216,112.7534576)(286.73803345,112.81345998)
\curveto(286.72804217,112.87345748)(286.71304219,112.93345742)(286.69303345,112.99345998)
\curveto(286.64304226,113.14345721)(286.59304231,113.28845706)(286.54303345,113.42845998)
\curveto(286.48304242,113.56845678)(286.41304249,113.70345665)(286.33303345,113.83345998)
\curveto(286.24304266,113.97345638)(286.13804276,114.09345626)(286.01803345,114.19345998)
\curveto(285.898043,114.29345606)(285.76804313,114.38845596)(285.62803345,114.47845998)
\curveto(285.52804337,114.53845581)(285.41804348,114.58345577)(285.29803345,114.61345998)
\curveto(285.17804372,114.6534557)(285.07304383,114.70345565)(284.98303345,114.76345998)
\curveto(284.92304398,114.81345554)(284.88304402,114.88345547)(284.86303345,114.97345998)
\curveto(284.85304405,114.99345536)(284.84804405,115.01845533)(284.84803345,115.04845998)
\curveto(284.84804405,115.07845527)(284.84304406,115.10345525)(284.83303345,115.12345998)
}
}
{
\newrgbcolor{curcolor}{0 0 0}
\pscustom[linestyle=none,fillstyle=solid,fillcolor=curcolor]
{
\newpath
\moveto(305.66937988,31.67142873)
\lineto(305.66937988,32.58642873)
\curveto(305.66939058,32.68642608)(305.66939058,32.78142599)(305.66937988,32.87142873)
\curveto(305.66939058,32.96142581)(305.68939056,33.03642573)(305.72937988,33.09642873)
\curveto(305.78939046,33.18642558)(305.86939038,33.24642552)(305.96937988,33.27642873)
\curveto(306.06939018,33.31642545)(306.17439007,33.36142541)(306.28437988,33.41142873)
\curveto(306.47438977,33.49142528)(306.66438958,33.56142521)(306.85437988,33.62142873)
\curveto(307.0443892,33.69142508)(307.23438901,33.766425)(307.42437988,33.84642873)
\curveto(307.60438864,33.91642485)(307.78938846,33.98142479)(307.97937988,34.04142873)
\curveto(308.15938809,34.10142467)(308.33938791,34.1714246)(308.51937988,34.25142873)
\curveto(308.65938759,34.31142446)(308.80438744,34.3664244)(308.95437988,34.41642873)
\curveto(309.10438714,34.4664243)(309.249387,34.52142425)(309.38937988,34.58142873)
\curveto(309.83938641,34.76142401)(310.29438595,34.93142384)(310.75437988,35.09142873)
\curveto(311.20438504,35.25142352)(311.65438459,35.42142335)(312.10437988,35.60142873)
\curveto(312.15438409,35.62142315)(312.20438404,35.63642313)(312.25437988,35.64642873)
\lineto(312.40437988,35.70642873)
\curveto(312.62438362,35.79642297)(312.8493834,35.88142289)(313.07937988,35.96142873)
\curveto(313.29938295,36.04142273)(313.51938273,36.12642264)(313.73937988,36.21642873)
\curveto(313.82938242,36.25642251)(313.93938231,36.29642247)(314.06937988,36.33642873)
\curveto(314.18938206,36.37642239)(314.25938199,36.44142233)(314.27937988,36.53142873)
\curveto(314.28938196,36.5714222)(314.28938196,36.60142217)(314.27937988,36.62142873)
\lineto(314.21937988,36.68142873)
\curveto(314.16938208,36.73142204)(314.11438213,36.766422)(314.05437988,36.78642873)
\curveto(313.99438225,36.81642195)(313.92938232,36.84642192)(313.85937988,36.87642873)
\lineto(313.22937988,37.11642873)
\curveto(313.00938324,37.19642157)(312.79438345,37.27642149)(312.58437988,37.35642873)
\lineto(312.43437988,37.41642873)
\lineto(312.25437988,37.47642873)
\curveto(312.06438418,37.55642121)(311.87438437,37.62642114)(311.68437988,37.68642873)
\curveto(311.48438476,37.75642101)(311.28438496,37.83142094)(311.08437988,37.91142873)
\curveto(310.50438574,38.15142062)(309.91938633,38.3714204)(309.32937988,38.57142873)
\curveto(308.73938751,38.78141999)(308.15438809,39.00641976)(307.57437988,39.24642873)
\curveto(307.37438887,39.32641944)(307.16938908,39.40141937)(306.95937988,39.47142873)
\curveto(306.7493895,39.55141922)(306.5443897,39.63141914)(306.34437988,39.71142873)
\curveto(306.26438998,39.75141902)(306.16439008,39.78641898)(306.04437988,39.81642873)
\curveto(305.92439032,39.85641891)(305.83939041,39.91141886)(305.78937988,39.98142873)
\curveto(305.7493905,40.04141873)(305.71939053,40.11641865)(305.69937988,40.20642873)
\curveto(305.67939057,40.30641846)(305.66939058,40.41641835)(305.66937988,40.53642873)
\curveto(305.65939059,40.65641811)(305.65939059,40.77641799)(305.66937988,40.89642873)
\curveto(305.66939058,41.01641775)(305.66939058,41.12641764)(305.66937988,41.22642873)
\curveto(305.66939058,41.31641745)(305.66939058,41.40641736)(305.66937988,41.49642873)
\curveto(305.66939058,41.59641717)(305.68939056,41.6714171)(305.72937988,41.72142873)
\curveto(305.77939047,41.81141696)(305.86939038,41.86141691)(305.99937988,41.87142873)
\curveto(306.12939012,41.88141689)(306.26938998,41.88641688)(306.41937988,41.88642873)
\lineto(308.06937988,41.88642873)
\lineto(314.33937988,41.88642873)
\lineto(315.59937988,41.88642873)
\curveto(315.70938054,41.88641688)(315.81938043,41.88641688)(315.92937988,41.88642873)
\curveto(316.03938021,41.89641687)(316.12438012,41.87641689)(316.18437988,41.82642873)
\curveto(316.24438,41.79641697)(316.28437996,41.75141702)(316.30437988,41.69142873)
\curveto(316.31437993,41.63141714)(316.32937992,41.56141721)(316.34937988,41.48142873)
\lineto(316.34937988,41.24142873)
\lineto(316.34937988,40.88142873)
\curveto(316.33937991,40.771418)(316.29437995,40.69141808)(316.21437988,40.64142873)
\curveto(316.18438006,40.62141815)(316.15438009,40.60641816)(316.12437988,40.59642873)
\curveto(316.08438016,40.59641817)(316.03938021,40.58641818)(315.98937988,40.56642873)
\lineto(315.82437988,40.56642873)
\curveto(315.76438048,40.55641821)(315.69438055,40.55141822)(315.61437988,40.55142873)
\curveto(315.53438071,40.56141821)(315.45938079,40.5664182)(315.38937988,40.56642873)
\lineto(314.54937988,40.56642873)
\lineto(310.12437988,40.56642873)
\curveto(309.87438637,40.5664182)(309.62438662,40.5664182)(309.37437988,40.56642873)
\curveto(309.11438713,40.5664182)(308.86438738,40.56141821)(308.62437988,40.55142873)
\curveto(308.52438772,40.55141822)(308.41438783,40.54641822)(308.29437988,40.53642873)
\curveto(308.17438807,40.52641824)(308.11438813,40.4714183)(308.11437988,40.37142873)
\lineto(308.12937988,40.37142873)
\curveto(308.1493881,40.30141847)(308.21438803,40.24141853)(308.32437988,40.19142873)
\curveto(308.43438781,40.15141862)(308.52938772,40.11641865)(308.60937988,40.08642873)
\curveto(308.77938747,40.01641875)(308.95438729,39.95141882)(309.13437988,39.89142873)
\curveto(309.30438694,39.83141894)(309.47438677,39.76141901)(309.64437988,39.68142873)
\curveto(309.69438655,39.66141911)(309.73938651,39.64641912)(309.77937988,39.63642873)
\curveto(309.81938643,39.62641914)(309.86438638,39.61141916)(309.91437988,39.59142873)
\curveto(310.09438615,39.51141926)(310.27938597,39.44141933)(310.46937988,39.38142873)
\curveto(310.6493856,39.33141944)(310.82938542,39.2664195)(311.00937988,39.18642873)
\curveto(311.15938509,39.11641965)(311.31438493,39.05641971)(311.47437988,39.00642873)
\curveto(311.62438462,38.95641981)(311.77438447,38.90141987)(311.92437988,38.84142873)
\curveto(312.39438385,38.64142013)(312.86938338,38.46142031)(313.34937988,38.30142873)
\curveto(313.81938243,38.14142063)(314.28438196,37.9664208)(314.74437988,37.77642873)
\curveto(314.92438132,37.69642107)(315.10438114,37.62642114)(315.28437988,37.56642873)
\curveto(315.46438078,37.50642126)(315.6443806,37.44142133)(315.82437988,37.37142873)
\curveto(315.93438031,37.32142145)(316.03938021,37.2714215)(316.13937988,37.22142873)
\curveto(316.22938002,37.18142159)(316.29437995,37.09642167)(316.33437988,36.96642873)
\curveto(316.3443799,36.94642182)(316.3493799,36.92142185)(316.34937988,36.89142873)
\curveto(316.33937991,36.8714219)(316.33937991,36.84642192)(316.34937988,36.81642873)
\curveto(316.35937989,36.78642198)(316.36437988,36.75142202)(316.36437988,36.71142873)
\curveto(316.35437989,36.6714221)(316.3493799,36.63142214)(316.34937988,36.59142873)
\lineto(316.34937988,36.29142873)
\curveto(316.3493799,36.19142258)(316.32437992,36.11142266)(316.27437988,36.05142873)
\curveto(316.22438002,35.9714228)(316.15438009,35.91142286)(316.06437988,35.87142873)
\curveto(315.96438028,35.84142293)(315.86438038,35.80142297)(315.76437988,35.75142873)
\curveto(315.56438068,35.6714231)(315.35938089,35.59142318)(315.14937988,35.51142873)
\curveto(314.92938132,35.44142333)(314.71938153,35.3664234)(314.51937988,35.28642873)
\curveto(314.33938191,35.20642356)(314.15938209,35.13642363)(313.97937988,35.07642873)
\curveto(313.78938246,35.02642374)(313.60438264,34.96142381)(313.42437988,34.88142873)
\curveto(312.86438338,34.65142412)(312.29938395,34.43642433)(311.72937988,34.23642873)
\curveto(311.15938509,34.03642473)(310.59438565,33.82142495)(310.03437988,33.59142873)
\lineto(309.40437988,33.35142873)
\curveto(309.18438706,33.28142549)(308.97438727,33.20642556)(308.77437988,33.12642873)
\curveto(308.66438758,33.07642569)(308.55938769,33.03142574)(308.45937988,32.99142873)
\curveto(308.3493879,32.96142581)(308.25438799,32.91142586)(308.17437988,32.84142873)
\curveto(308.15438809,32.83142594)(308.1443881,32.82142595)(308.14437988,32.81142873)
\lineto(308.11437988,32.78142873)
\lineto(308.11437988,32.70642873)
\lineto(308.14437988,32.67642873)
\curveto(308.1443881,32.6664261)(308.1493881,32.65642611)(308.15937988,32.64642873)
\curveto(308.20938804,32.62642614)(308.26438798,32.61642615)(308.32437988,32.61642873)
\curveto(308.38438786,32.61642615)(308.4443878,32.60642616)(308.50437988,32.58642873)
\lineto(308.66937988,32.58642873)
\curveto(308.72938752,32.5664262)(308.79438745,32.56142621)(308.86437988,32.57142873)
\curveto(308.93438731,32.58142619)(309.00438724,32.58642618)(309.07437988,32.58642873)
\lineto(309.88437988,32.58642873)
\lineto(314.44437988,32.58642873)
\lineto(315.62937988,32.58642873)
\curveto(315.73938051,32.58642618)(315.8493804,32.58142619)(315.95937988,32.57142873)
\curveto(316.06938018,32.5714262)(316.15438009,32.54642622)(316.21437988,32.49642873)
\curveto(316.29437995,32.44642632)(316.33937991,32.35642641)(316.34937988,32.22642873)
\lineto(316.34937988,31.83642873)
\lineto(316.34937988,31.64142873)
\curveto(316.3493799,31.59142718)(316.33937991,31.54142723)(316.31937988,31.49142873)
\curveto(316.27937997,31.36142741)(316.19438005,31.28642748)(316.06437988,31.26642873)
\curveto(315.93438031,31.25642751)(315.78438046,31.25142752)(315.61437988,31.25142873)
\lineto(313.87437988,31.25142873)
\lineto(307.87437988,31.25142873)
\lineto(306.46437988,31.25142873)
\curveto(306.35438989,31.25142752)(306.23939001,31.24642752)(306.11937988,31.23642873)
\curveto(305.99939025,31.23642753)(305.90439034,31.26142751)(305.83437988,31.31142873)
\curveto(305.77439047,31.35142742)(305.72439052,31.42642734)(305.68437988,31.53642873)
\curveto(305.67439057,31.55642721)(305.67439057,31.57642719)(305.68437988,31.59642873)
\curveto(305.68439056,31.62642714)(305.67939057,31.65142712)(305.66937988,31.67142873)
}
}
{
\newrgbcolor{curcolor}{0 0 0}
\pscustom[linestyle=none,fillstyle=solid,fillcolor=curcolor]
{
\newpath
\moveto(315.79437988,50.87353811)
\curveto(315.95438029,50.90353028)(316.08938016,50.88853029)(316.19937988,50.82853811)
\curveto(316.29937995,50.76853041)(316.37437987,50.68853049)(316.42437988,50.58853811)
\curveto(316.4443798,50.53853064)(316.45437979,50.4835307)(316.45437988,50.42353811)
\curveto(316.45437979,50.37353081)(316.46437978,50.31853086)(316.48437988,50.25853811)
\curveto(316.53437971,50.03853114)(316.51937973,49.81853136)(316.43937988,49.59853811)
\curveto(316.36937988,49.38853179)(316.27937997,49.24353194)(316.16937988,49.16353811)
\curveto(316.09938015,49.11353207)(316.01938023,49.06853211)(315.92937988,49.02853811)
\curveto(315.82938042,48.98853219)(315.7493805,48.93853224)(315.68937988,48.87853811)
\curveto(315.66938058,48.85853232)(315.6493806,48.83353235)(315.62937988,48.80353811)
\curveto(315.60938064,48.7835324)(315.60438064,48.75353243)(315.61437988,48.71353811)
\curveto(315.6443806,48.60353258)(315.69938055,48.49853268)(315.77937988,48.39853811)
\curveto(315.85938039,48.30853287)(315.92938032,48.21853296)(315.98937988,48.12853811)
\curveto(316.06938018,47.99853318)(316.1443801,47.85853332)(316.21437988,47.70853811)
\curveto(316.27437997,47.55853362)(316.32937992,47.39853378)(316.37937988,47.22853811)
\curveto(316.40937984,47.12853405)(316.42937982,47.01853416)(316.43937988,46.89853811)
\curveto(316.4493798,46.78853439)(316.46437978,46.6785345)(316.48437988,46.56853811)
\curveto(316.49437975,46.51853466)(316.49937975,46.47353471)(316.49937988,46.43353811)
\lineto(316.49937988,46.32853811)
\curveto(316.51937973,46.21853496)(316.51937973,46.11353507)(316.49937988,46.01353811)
\lineto(316.49937988,45.87853811)
\curveto(316.48937976,45.82853535)(316.48437976,45.7785354)(316.48437988,45.72853811)
\curveto(316.48437976,45.6785355)(316.47437977,45.63353555)(316.45437988,45.59353811)
\curveto(316.4443798,45.55353563)(316.43937981,45.51853566)(316.43937988,45.48853811)
\curveto(316.4493798,45.46853571)(316.4493798,45.44353574)(316.43937988,45.41353811)
\lineto(316.37937988,45.17353811)
\curveto(316.36937988,45.09353609)(316.3493799,45.01853616)(316.31937988,44.94853811)
\curveto(316.18938006,44.64853653)(316.0443802,44.40353678)(315.88437988,44.21353811)
\curveto(315.71438053,44.03353715)(315.47938077,43.8835373)(315.17937988,43.76353811)
\curveto(314.95938129,43.67353751)(314.69438155,43.62853755)(314.38437988,43.62853811)
\lineto(314.06937988,43.62853811)
\curveto(314.01938223,43.63853754)(313.96938228,43.64353754)(313.91937988,43.64353811)
\lineto(313.73937988,43.67353811)
\lineto(313.40937988,43.79353811)
\curveto(313.29938295,43.83353735)(313.19938305,43.8835373)(313.10937988,43.94353811)
\curveto(312.81938343,44.12353706)(312.60438364,44.36853681)(312.46437988,44.67853811)
\curveto(312.32438392,44.98853619)(312.19938405,45.32853585)(312.08937988,45.69853811)
\curveto(312.0493842,45.83853534)(312.01938423,45.9835352)(311.99937988,46.13353811)
\curveto(311.97938427,46.2835349)(311.95438429,46.43353475)(311.92437988,46.58353811)
\curveto(311.90438434,46.65353453)(311.89438435,46.71853446)(311.89437988,46.77853811)
\curveto(311.89438435,46.84853433)(311.88438436,46.92353426)(311.86437988,47.00353811)
\curveto(311.8443844,47.07353411)(311.83438441,47.14353404)(311.83437988,47.21353811)
\curveto(311.82438442,47.2835339)(311.80938444,47.35853382)(311.78937988,47.43853811)
\curveto(311.72938452,47.68853349)(311.67938457,47.92353326)(311.63937988,48.14353811)
\curveto(311.58938466,48.36353282)(311.47438477,48.53853264)(311.29437988,48.66853811)
\curveto(311.21438503,48.72853245)(311.11438513,48.7785324)(310.99437988,48.81853811)
\curveto(310.86438538,48.85853232)(310.72438552,48.85853232)(310.57437988,48.81853811)
\curveto(310.33438591,48.75853242)(310.1443861,48.66853251)(310.00437988,48.54853811)
\curveto(309.86438638,48.43853274)(309.75438649,48.2785329)(309.67437988,48.06853811)
\curveto(309.62438662,47.94853323)(309.58938666,47.80353338)(309.56937988,47.63353811)
\curveto(309.5493867,47.47353371)(309.53938671,47.30353388)(309.53937988,47.12353811)
\curveto(309.53938671,46.94353424)(309.5493867,46.76853441)(309.56937988,46.59853811)
\curveto(309.58938666,46.42853475)(309.61938663,46.2835349)(309.65937988,46.16353811)
\curveto(309.71938653,45.99353519)(309.80438644,45.82853535)(309.91437988,45.66853811)
\curveto(309.97438627,45.58853559)(310.05438619,45.51353567)(310.15437988,45.44353811)
\curveto(310.244386,45.3835358)(310.3443859,45.32853585)(310.45437988,45.27853811)
\curveto(310.53438571,45.24853593)(310.61938563,45.21853596)(310.70937988,45.18853811)
\curveto(310.79938545,45.16853601)(310.86938538,45.12353606)(310.91937988,45.05353811)
\curveto(310.9493853,45.01353617)(310.97438527,44.94353624)(310.99437988,44.84353811)
\curveto(311.00438524,44.75353643)(311.00938524,44.65853652)(311.00937988,44.55853811)
\curveto(311.00938524,44.45853672)(311.00438524,44.35853682)(310.99437988,44.25853811)
\curveto(310.97438527,44.16853701)(310.9493853,44.10353708)(310.91937988,44.06353811)
\curveto(310.88938536,44.02353716)(310.83938541,43.99353719)(310.76937988,43.97353811)
\curveto(310.69938555,43.95353723)(310.62438562,43.95353723)(310.54437988,43.97353811)
\curveto(310.41438583,44.00353718)(310.29438595,44.03353715)(310.18437988,44.06353811)
\curveto(310.06438618,44.10353708)(309.9493863,44.14853703)(309.83937988,44.19853811)
\curveto(309.48938676,44.38853679)(309.21938703,44.62853655)(309.02937988,44.91853811)
\curveto(308.82938742,45.20853597)(308.66938758,45.56853561)(308.54937988,45.99853811)
\curveto(308.52938772,46.09853508)(308.51438773,46.19853498)(308.50437988,46.29853811)
\curveto(308.49438775,46.40853477)(308.47938777,46.51853466)(308.45937988,46.62853811)
\curveto(308.4493878,46.66853451)(308.4493878,46.73353445)(308.45937988,46.82353811)
\curveto(308.45938779,46.91353427)(308.4493878,46.96853421)(308.42937988,46.98853811)
\curveto(308.41938783,47.68853349)(308.49938775,48.29853288)(308.66937988,48.81853811)
\curveto(308.83938741,49.33853184)(309.16438708,49.70353148)(309.64437988,49.91353811)
\curveto(309.8443864,50.00353118)(310.07938617,50.05353113)(310.34937988,50.06353811)
\curveto(310.60938564,50.0835311)(310.88438536,50.09353109)(311.17437988,50.09353811)
\lineto(314.48937988,50.09353811)
\curveto(314.62938162,50.09353109)(314.76438148,50.09853108)(314.89437988,50.10853811)
\curveto(315.02438122,50.11853106)(315.12938112,50.14853103)(315.20937988,50.19853811)
\curveto(315.27938097,50.24853093)(315.32938092,50.31353087)(315.35937988,50.39353811)
\curveto(315.39938085,50.4835307)(315.42938082,50.56853061)(315.44937988,50.64853811)
\curveto(315.45938079,50.72853045)(315.50438074,50.78853039)(315.58437988,50.82853811)
\curveto(315.61438063,50.84853033)(315.6443806,50.85853032)(315.67437988,50.85853811)
\curveto(315.70438054,50.85853032)(315.7443805,50.86353032)(315.79437988,50.87353811)
\moveto(314.12937988,48.72853811)
\curveto(313.98938226,48.78853239)(313.82938242,48.81853236)(313.64937988,48.81853811)
\curveto(313.45938279,48.82853235)(313.26438298,48.83353235)(313.06437988,48.83353811)
\curveto(312.95438329,48.83353235)(312.85438339,48.82853235)(312.76437988,48.81853811)
\curveto(312.67438357,48.80853237)(312.60438364,48.76853241)(312.55437988,48.69853811)
\curveto(312.53438371,48.66853251)(312.52438372,48.59853258)(312.52437988,48.48853811)
\curveto(312.5443837,48.46853271)(312.55438369,48.43353275)(312.55437988,48.38353811)
\curveto(312.55438369,48.33353285)(312.56438368,48.28853289)(312.58437988,48.24853811)
\curveto(312.60438364,48.16853301)(312.62438362,48.0785331)(312.64437988,47.97853811)
\lineto(312.70437988,47.67853811)
\curveto(312.70438354,47.64853353)(312.70938354,47.61353357)(312.71937988,47.57353811)
\lineto(312.71937988,47.46853811)
\curveto(312.75938349,47.31853386)(312.78438346,47.15353403)(312.79437988,46.97353811)
\curveto(312.79438345,46.80353438)(312.81438343,46.64353454)(312.85437988,46.49353811)
\curveto(312.87438337,46.41353477)(312.89438335,46.33853484)(312.91437988,46.26853811)
\curveto(312.92438332,46.20853497)(312.93938331,46.13853504)(312.95937988,46.05853811)
\curveto(313.00938324,45.89853528)(313.07438317,45.74853543)(313.15437988,45.60853811)
\curveto(313.22438302,45.46853571)(313.31438293,45.34853583)(313.42437988,45.24853811)
\curveto(313.53438271,45.14853603)(313.66938258,45.07353611)(313.82937988,45.02353811)
\curveto(313.97938227,44.97353621)(314.16438208,44.95353623)(314.38437988,44.96353811)
\curveto(314.48438176,44.96353622)(314.57938167,44.9785362)(314.66937988,45.00853811)
\curveto(314.7493815,45.04853613)(314.82438142,45.09353609)(314.89437988,45.14353811)
\curveto(315.00438124,45.22353596)(315.09938115,45.32853585)(315.17937988,45.45853811)
\curveto(315.249381,45.58853559)(315.30938094,45.72853545)(315.35937988,45.87853811)
\curveto(315.36938088,45.92853525)(315.37438087,45.9785352)(315.37437988,46.02853811)
\curveto(315.37438087,46.0785351)(315.37938087,46.12853505)(315.38937988,46.17853811)
\curveto(315.40938084,46.24853493)(315.42438082,46.33353485)(315.43437988,46.43353811)
\curveto(315.43438081,46.54353464)(315.42438082,46.63353455)(315.40437988,46.70353811)
\curveto(315.38438086,46.76353442)(315.37938087,46.82353436)(315.38937988,46.88353811)
\curveto(315.38938086,46.94353424)(315.37938087,47.00353418)(315.35937988,47.06353811)
\curveto(315.33938091,47.14353404)(315.32438092,47.21853396)(315.31437988,47.28853811)
\curveto(315.30438094,47.36853381)(315.28438096,47.44353374)(315.25437988,47.51353811)
\curveto(315.13438111,47.80353338)(314.98938126,48.04853313)(314.81937988,48.24853811)
\curveto(314.6493816,48.45853272)(314.41938183,48.61853256)(314.12937988,48.72853811)
}
}
{
\newrgbcolor{curcolor}{0 0 0}
\pscustom[linestyle=none,fillstyle=solid,fillcolor=curcolor]
{
\newpath
\moveto(308.44437988,55.69017873)
\curveto(308.4443878,55.92017394)(308.50438774,56.05017381)(308.62437988,56.08017873)
\curveto(308.73438751,56.11017375)(308.89938735,56.12517374)(309.11937988,56.12517873)
\lineto(309.40437988,56.12517873)
\curveto(309.49438675,56.12517374)(309.56938668,56.10017376)(309.62937988,56.05017873)
\curveto(309.70938654,55.99017387)(309.75438649,55.90517396)(309.76437988,55.79517873)
\curveto(309.76438648,55.68517418)(309.77938647,55.57517429)(309.80937988,55.46517873)
\curveto(309.83938641,55.32517454)(309.86938638,55.19017467)(309.89937988,55.06017873)
\curveto(309.92938632,54.94017492)(309.96938628,54.82517504)(310.01937988,54.71517873)
\curveto(310.1493861,54.42517544)(310.32938592,54.19017567)(310.55937988,54.01017873)
\curveto(310.77938547,53.83017603)(311.03438521,53.67517619)(311.32437988,53.54517873)
\curveto(311.43438481,53.50517636)(311.5493847,53.47517639)(311.66937988,53.45517873)
\curveto(311.77938447,53.43517643)(311.89438435,53.41017645)(312.01437988,53.38017873)
\curveto(312.06438418,53.37017649)(312.11438413,53.3651765)(312.16437988,53.36517873)
\curveto(312.21438403,53.37517649)(312.26438398,53.37517649)(312.31437988,53.36517873)
\curveto(312.43438381,53.33517653)(312.57438367,53.32017654)(312.73437988,53.32017873)
\curveto(312.88438336,53.33017653)(313.02938322,53.33517653)(313.16937988,53.33517873)
\lineto(315.01437988,53.33517873)
\lineto(315.35937988,53.33517873)
\curveto(315.47938077,53.33517653)(315.59438065,53.33017653)(315.70437988,53.32017873)
\curveto(315.81438043,53.31017655)(315.90938034,53.30517656)(315.98937988,53.30517873)
\curveto(316.06938018,53.31517655)(316.13938011,53.29517657)(316.19937988,53.24517873)
\curveto(316.26937998,53.19517667)(316.30937994,53.11517675)(316.31937988,53.00517873)
\curveto(316.32937992,52.90517696)(316.33437991,52.79517707)(316.33437988,52.67517873)
\lineto(316.33437988,52.40517873)
\curveto(316.31437993,52.35517751)(316.29937995,52.30517756)(316.28937988,52.25517873)
\curveto(316.26937998,52.21517765)(316.24438,52.18517768)(316.21437988,52.16517873)
\curveto(316.1443801,52.11517775)(316.05938019,52.08517778)(315.95937988,52.07517873)
\lineto(315.62937988,52.07517873)
\lineto(314.47437988,52.07517873)
\lineto(310.31937988,52.07517873)
\lineto(309.28437988,52.07517873)
\lineto(308.98437988,52.07517873)
\curveto(308.88438736,52.08517778)(308.79938745,52.11517775)(308.72937988,52.16517873)
\curveto(308.68938756,52.19517767)(308.65938759,52.24517762)(308.63937988,52.31517873)
\curveto(308.61938763,52.39517747)(308.60938764,52.48017738)(308.60937988,52.57017873)
\curveto(308.59938765,52.6601772)(308.59938765,52.75017711)(308.60937988,52.84017873)
\curveto(308.61938763,52.93017693)(308.63438761,53.00017686)(308.65437988,53.05017873)
\curveto(308.68438756,53.13017673)(308.7443875,53.18017668)(308.83437988,53.20017873)
\curveto(308.91438733,53.23017663)(309.00438724,53.24517662)(309.10437988,53.24517873)
\lineto(309.40437988,53.24517873)
\curveto(309.50438674,53.24517662)(309.59438665,53.2651766)(309.67437988,53.30517873)
\curveto(309.69438655,53.31517655)(309.70938654,53.32517654)(309.71937988,53.33517873)
\lineto(309.76437988,53.38017873)
\curveto(309.76438648,53.49017637)(309.71938653,53.58017628)(309.62937988,53.65017873)
\curveto(309.52938672,53.72017614)(309.4493868,53.78017608)(309.38937988,53.83017873)
\lineto(309.29937988,53.92017873)
\curveto(309.18938706,54.01017585)(309.07438717,54.13517573)(308.95437988,54.29517873)
\curveto(308.83438741,54.45517541)(308.7443875,54.60517526)(308.68437988,54.74517873)
\curveto(308.63438761,54.83517503)(308.59938765,54.93017493)(308.57937988,55.03017873)
\curveto(308.5493877,55.13017473)(308.51938773,55.23517463)(308.48937988,55.34517873)
\curveto(308.47938777,55.40517446)(308.47438777,55.4651744)(308.47437988,55.52517873)
\curveto(308.46438778,55.58517428)(308.45438779,55.64017422)(308.44437988,55.69017873)
}
}
{
\newrgbcolor{curcolor}{0 0 0}
\pscustom[linestyle=none,fillstyle=solid,fillcolor=curcolor]
{
}
}
{
\newrgbcolor{curcolor}{0 0 0}
\pscustom[linestyle=none,fillstyle=solid,fillcolor=curcolor]
{
\newpath
\moveto(305.74437988,64.11010061)
\curveto(305.71439053,65.74009517)(306.26938998,66.79009412)(307.40937988,67.26010061)
\curveto(307.63938861,67.36009355)(307.92938832,67.42509348)(308.27937988,67.45510061)
\curveto(308.61938763,67.49509341)(308.92938732,67.47009344)(309.20937988,67.38010061)
\curveto(309.46938678,67.29009362)(309.69438655,67.17009374)(309.88437988,67.02010061)
\curveto(309.92438632,67.00009391)(309.95938629,66.97509393)(309.98937988,66.94510061)
\curveto(310.00938624,66.91509399)(310.03438621,66.89009402)(310.06437988,66.87010061)
\lineto(310.18437988,66.78010061)
\curveto(310.21438603,66.75009416)(310.23938601,66.71509419)(310.25937988,66.67510061)
\curveto(310.30938594,66.62509428)(310.35438589,66.57009434)(310.39437988,66.51010061)
\curveto(310.43438581,66.46009445)(310.48438576,66.41509449)(310.54437988,66.37510061)
\curveto(310.58438566,66.33509457)(310.63438561,66.32009459)(310.69437988,66.33010061)
\curveto(310.7443855,66.34009457)(310.78938546,66.37009454)(310.82937988,66.42010061)
\curveto(310.86938538,66.47009444)(310.90938534,66.52509438)(310.94937988,66.58510061)
\curveto(310.97938527,66.65509425)(311.00938524,66.72009419)(311.03937988,66.78010061)
\curveto(311.06938518,66.84009407)(311.09938515,66.89009402)(311.12937988,66.93010061)
\curveto(311.3493849,67.25009366)(311.65938459,67.5050934)(312.05937988,67.69510061)
\curveto(312.1493841,67.73509317)(312.244384,67.76509314)(312.34437988,67.78510061)
\curveto(312.43438381,67.81509309)(312.52438372,67.84009307)(312.61437988,67.86010061)
\curveto(312.66438358,67.87009304)(312.71438353,67.87509303)(312.76437988,67.87510061)
\curveto(312.80438344,67.88509302)(312.8493834,67.89509301)(312.89937988,67.90510061)
\curveto(312.9493833,67.91509299)(312.99938325,67.91509299)(313.04937988,67.90510061)
\curveto(313.09938315,67.89509301)(313.1493831,67.90009301)(313.19937988,67.92010061)
\curveto(313.249383,67.93009298)(313.30938294,67.93509297)(313.37937988,67.93510061)
\curveto(313.4493828,67.93509297)(313.50938274,67.92509298)(313.55937988,67.90510061)
\lineto(313.78437988,67.90510061)
\lineto(314.02437988,67.84510061)
\curveto(314.09438215,67.83509307)(314.16438208,67.82009309)(314.23437988,67.80010061)
\curveto(314.32438192,67.77009314)(314.40938184,67.74009317)(314.48937988,67.71010061)
\curveto(314.56938168,67.69009322)(314.6493816,67.66009325)(314.72937988,67.62010061)
\curveto(314.78938146,67.60009331)(314.8493814,67.57009334)(314.90937988,67.53010061)
\curveto(314.95938129,67.50009341)(315.00938124,67.46509344)(315.05937988,67.42510061)
\curveto(315.36938088,67.22509368)(315.62938062,66.97509393)(315.83937988,66.67510061)
\curveto(316.03938021,66.37509453)(316.20438004,66.03009488)(316.33437988,65.64010061)
\curveto(316.37437987,65.52009539)(316.39937985,65.39009552)(316.40937988,65.25010061)
\curveto(316.42937982,65.12009579)(316.45437979,64.98509592)(316.48437988,64.84510061)
\curveto(316.49437975,64.77509613)(316.49937975,64.7050962)(316.49937988,64.63510061)
\curveto(316.49937975,64.57509633)(316.50437974,64.5100964)(316.51437988,64.44010061)
\curveto(316.52437972,64.40009651)(316.52937972,64.34009657)(316.52937988,64.26010061)
\curveto(316.52937972,64.19009672)(316.52437972,64.14009677)(316.51437988,64.11010061)
\curveto(316.50437974,64.06009685)(316.49937975,64.01509689)(316.49937988,63.97510061)
\lineto(316.49937988,63.85510061)
\curveto(316.47937977,63.75509715)(316.46437978,63.65509725)(316.45437988,63.55510061)
\curveto(316.4443798,63.45509745)(316.42937982,63.36009755)(316.40937988,63.27010061)
\curveto(316.37937987,63.16009775)(316.35437989,63.05009786)(316.33437988,62.94010061)
\curveto(316.30437994,62.84009807)(316.26437998,62.73509817)(316.21437988,62.62510061)
\curveto(316.05438019,62.25509865)(315.85438039,61.94009897)(315.61437988,61.68010061)
\curveto(315.36438088,61.42009949)(315.05438119,61.2100997)(314.68437988,61.05010061)
\curveto(314.59438165,61.0100999)(314.49938175,60.97509993)(314.39937988,60.94510061)
\curveto(314.29938195,60.91509999)(314.19438205,60.88510002)(314.08437988,60.85510061)
\curveto(314.03438221,60.83510007)(313.98438226,60.82510008)(313.93437988,60.82510061)
\curveto(313.87438237,60.82510008)(313.81438243,60.81510009)(313.75437988,60.79510061)
\curveto(313.69438255,60.77510013)(313.61438263,60.76510014)(313.51437988,60.76510061)
\curveto(313.41438283,60.76510014)(313.33938291,60.78010013)(313.28937988,60.81010061)
\curveto(313.25938299,60.82010009)(313.23438301,60.83510007)(313.21437988,60.85510061)
\lineto(313.15437988,60.91510061)
\curveto(313.13438311,60.95509995)(313.11938313,61.01509989)(313.10937988,61.09510061)
\curveto(313.09938315,61.18509972)(313.09438315,61.27509963)(313.09437988,61.36510061)
\curveto(313.09438315,61.45509945)(313.09938315,61.54009937)(313.10937988,61.62010061)
\curveto(313.11938313,61.7100992)(313.12938312,61.77509913)(313.13937988,61.81510061)
\curveto(313.15938309,61.83509907)(313.17438307,61.85509905)(313.18437988,61.87510061)
\curveto(313.18438306,61.89509901)(313.19438305,61.91509899)(313.21437988,61.93510061)
\curveto(313.30438294,62.0050989)(313.41938283,62.04509886)(313.55937988,62.05510061)
\curveto(313.69938255,62.07509883)(313.82438242,62.1050988)(313.93437988,62.14510061)
\lineto(314.29437988,62.29510061)
\curveto(314.40438184,62.34509856)(314.50938174,62.4100985)(314.60937988,62.49010061)
\curveto(314.63938161,62.5100984)(314.66438158,62.53009838)(314.68437988,62.55010061)
\curveto(314.70438154,62.58009833)(314.72938152,62.6050983)(314.75937988,62.62510061)
\curveto(314.81938143,62.66509824)(314.86438138,62.70009821)(314.89437988,62.73010061)
\curveto(314.92438132,62.77009814)(314.95438129,62.8050981)(314.98437988,62.83510061)
\curveto(315.01438123,62.87509803)(315.0443812,62.92009799)(315.07437988,62.97010061)
\curveto(315.13438111,63.06009785)(315.18438106,63.15509775)(315.22437988,63.25510061)
\lineto(315.34437988,63.58510061)
\curveto(315.39438085,63.73509717)(315.42438082,63.93509697)(315.43437988,64.18510061)
\curveto(315.4443808,64.43509647)(315.42438082,64.64509626)(315.37437988,64.81510061)
\curveto(315.35438089,64.89509601)(315.33938091,64.96509594)(315.32937988,65.02510061)
\lineto(315.26937988,65.23510061)
\curveto(315.1493811,65.51509539)(314.99938125,65.75509515)(314.81937988,65.95510061)
\curveto(314.63938161,66.16509474)(314.40938184,66.33009458)(314.12937988,66.45010061)
\curveto(314.05938219,66.48009443)(313.98938226,66.50009441)(313.91937988,66.51010061)
\lineto(313.67937988,66.57010061)
\curveto(313.53938271,66.6100943)(313.37938287,66.62009429)(313.19937988,66.60010061)
\curveto(313.00938324,66.58009433)(312.85938339,66.55009436)(312.74937988,66.51010061)
\curveto(312.36938388,66.38009453)(312.07938417,66.19509471)(311.87937988,65.95510061)
\curveto(311.67938457,65.72509518)(311.51938473,65.41509549)(311.39937988,65.02510061)
\curveto(311.36938488,64.91509599)(311.3493849,64.79509611)(311.33937988,64.66510061)
\curveto(311.32938492,64.54509636)(311.32438492,64.42009649)(311.32437988,64.29010061)
\curveto(311.32438492,64.13009678)(311.31938493,63.99009692)(311.30937988,63.87010061)
\curveto(311.29938495,63.75009716)(311.23938501,63.66509724)(311.12937988,63.61510061)
\curveto(311.09938515,63.59509731)(311.06438518,63.58509732)(311.02437988,63.58510061)
\lineto(310.88937988,63.58510061)
\curveto(310.78938546,63.57509733)(310.69438555,63.57509733)(310.60437988,63.58510061)
\curveto(310.51438573,63.6050973)(310.4493858,63.64509726)(310.40937988,63.70510061)
\curveto(310.37938587,63.74509716)(310.35938589,63.78509712)(310.34937988,63.82510061)
\curveto(310.33938591,63.87509703)(310.32938592,63.93009698)(310.31937988,63.99010061)
\curveto(310.30938594,64.0100969)(310.30938594,64.03509687)(310.31937988,64.06510061)
\curveto(310.31938593,64.09509681)(310.31438593,64.12009679)(310.30437988,64.14010061)
\lineto(310.30437988,64.27510061)
\curveto(310.28438596,64.38509652)(310.27438597,64.48509642)(310.27437988,64.57510061)
\curveto(310.26438598,64.67509623)(310.244386,64.77009614)(310.21437988,64.86010061)
\curveto(310.10438614,65.18009573)(309.95938629,65.43509547)(309.77937988,65.62510061)
\curveto(309.59938665,65.81509509)(309.3493869,65.96509494)(309.02937988,66.07510061)
\curveto(308.92938732,66.1050948)(308.80438744,66.12509478)(308.65437988,66.13510061)
\curveto(308.49438775,66.15509475)(308.3493879,66.15009476)(308.21937988,66.12010061)
\curveto(308.1493881,66.10009481)(308.08438816,66.08009483)(308.02437988,66.06010061)
\curveto(307.95438829,66.05009486)(307.88938836,66.03009488)(307.82937988,66.00010061)
\curveto(307.58938866,65.90009501)(307.39938885,65.75509515)(307.25937988,65.56510061)
\curveto(307.11938913,65.37509553)(307.00938924,65.15009576)(306.92937988,64.89010061)
\curveto(306.90938934,64.83009608)(306.89938935,64.77009614)(306.89937988,64.71010061)
\curveto(306.89938935,64.65009626)(306.88938936,64.58509632)(306.86937988,64.51510061)
\curveto(306.8493894,64.43509647)(306.83938941,64.34009657)(306.83937988,64.23010061)
\curveto(306.83938941,64.12009679)(306.8493894,64.02509688)(306.86937988,63.94510061)
\curveto(306.88938936,63.89509701)(306.89938935,63.84509706)(306.89937988,63.79510061)
\curveto(306.89938935,63.75509715)(306.90938934,63.7100972)(306.92937988,63.66010061)
\curveto(306.97938927,63.48009743)(307.05438919,63.3100976)(307.15437988,63.15010061)
\curveto(307.244389,63.00009791)(307.35938889,62.87009804)(307.49937988,62.76010061)
\curveto(307.61938863,62.67009824)(307.7493885,62.59009832)(307.88937988,62.52010061)
\curveto(308.02938822,62.45009846)(308.18438806,62.38509852)(308.35437988,62.32510061)
\curveto(308.46438778,62.29509861)(308.58438766,62.27509863)(308.71437988,62.26510061)
\curveto(308.83438741,62.25509865)(308.93438731,62.22009869)(309.01437988,62.16010061)
\curveto(309.05438719,62.14009877)(309.09438715,62.08009883)(309.13437988,61.98010061)
\curveto(309.1443871,61.94009897)(309.15438709,61.88009903)(309.16437988,61.80010061)
\lineto(309.16437988,61.54510061)
\curveto(309.15438709,61.45509945)(309.1443871,61.37009954)(309.13437988,61.29010061)
\curveto(309.12438712,61.22009969)(309.10938714,61.17009974)(309.08937988,61.14010061)
\curveto(309.05938719,61.10009981)(309.00438724,61.06509984)(308.92437988,61.03510061)
\curveto(308.8443874,61.0050999)(308.75938749,61.00009991)(308.66937988,61.02010061)
\curveto(308.61938763,61.03009988)(308.56938768,61.03509987)(308.51937988,61.03510061)
\lineto(308.33937988,61.06510061)
\curveto(308.23938801,61.09509981)(308.13938811,61.12009979)(308.03937988,61.14010061)
\curveto(307.93938831,61.17009974)(307.8493884,61.2050997)(307.76937988,61.24510061)
\curveto(307.65938859,61.29509961)(307.55438869,61.34009957)(307.45437988,61.38010061)
\curveto(307.3443889,61.42009949)(307.23938901,61.47009944)(307.13937988,61.53010061)
\curveto(306.59938965,61.86009905)(306.20439004,62.33009858)(305.95437988,62.94010061)
\curveto(305.90439034,63.06009785)(305.86939038,63.18509772)(305.84937988,63.31510061)
\curveto(305.82939042,63.45509745)(305.80439044,63.59509731)(305.77437988,63.73510061)
\curveto(305.76439048,63.79509711)(305.75939049,63.85509705)(305.75937988,63.91510061)
\curveto(305.75939049,63.98509692)(305.75439049,64.05009686)(305.74437988,64.11010061)
}
}
{
\newrgbcolor{curcolor}{0 0 0}
\pscustom[linestyle=none,fillstyle=solid,fillcolor=curcolor]
{
\newpath
\moveto(311.26437988,76.34470998)
\lineto(311.51937988,76.34470998)
\curveto(311.59938465,76.35470228)(311.67438457,76.34970228)(311.74437988,76.32970998)
\lineto(311.98437988,76.32970998)
\lineto(312.14937988,76.32970998)
\curveto(312.249384,76.30970232)(312.35438389,76.29970233)(312.46437988,76.29970998)
\curveto(312.56438368,76.29970233)(312.66438358,76.28970234)(312.76437988,76.26970998)
\lineto(312.91437988,76.26970998)
\curveto(313.05438319,76.23970239)(313.19438305,76.21970241)(313.33437988,76.20970998)
\curveto(313.46438278,76.19970243)(313.59438265,76.17470246)(313.72437988,76.13470998)
\curveto(313.80438244,76.11470252)(313.88938236,76.09470254)(313.97937988,76.07470998)
\lineto(314.21937988,76.01470998)
\lineto(314.51937988,75.89470998)
\curveto(314.60938164,75.86470277)(314.69938155,75.8297028)(314.78937988,75.78970998)
\curveto(315.00938124,75.68970294)(315.22438102,75.55470308)(315.43437988,75.38470998)
\curveto(315.6443806,75.22470341)(315.81438043,75.04970358)(315.94437988,74.85970998)
\curveto(315.98438026,74.80970382)(316.02438022,74.74970388)(316.06437988,74.67970998)
\curveto(316.09438015,74.61970401)(316.12938012,74.55970407)(316.16937988,74.49970998)
\curveto(316.21938003,74.41970421)(316.25937999,74.32470431)(316.28937988,74.21470998)
\curveto(316.31937993,74.10470453)(316.3493799,73.99970463)(316.37937988,73.89970998)
\curveto(316.41937983,73.78970484)(316.4443798,73.67970495)(316.45437988,73.56970998)
\curveto(316.46437978,73.45970517)(316.47937977,73.34470529)(316.49937988,73.22470998)
\curveto(316.50937974,73.18470545)(316.50937974,73.13970549)(316.49937988,73.08970998)
\curveto(316.49937975,73.04970558)(316.50437974,73.00970562)(316.51437988,72.96970998)
\curveto(316.52437972,72.9297057)(316.52937972,72.87470576)(316.52937988,72.80470998)
\curveto(316.52937972,72.7347059)(316.52437972,72.68470595)(316.51437988,72.65470998)
\curveto(316.49437975,72.60470603)(316.48937976,72.55970607)(316.49937988,72.51970998)
\curveto(316.50937974,72.47970615)(316.50937974,72.44470619)(316.49937988,72.41470998)
\lineto(316.49937988,72.32470998)
\curveto(316.47937977,72.26470637)(316.46437978,72.19970643)(316.45437988,72.12970998)
\curveto(316.45437979,72.06970656)(316.4493798,72.00470663)(316.43937988,71.93470998)
\curveto(316.38937986,71.76470687)(316.33937991,71.60470703)(316.28937988,71.45470998)
\curveto(316.23938001,71.30470733)(316.17438007,71.15970747)(316.09437988,71.01970998)
\curveto(316.05438019,70.96970766)(316.02438022,70.91470772)(316.00437988,70.85470998)
\curveto(315.97438027,70.80470783)(315.93938031,70.75470788)(315.89937988,70.70470998)
\curveto(315.71938053,70.46470817)(315.49938075,70.26470837)(315.23937988,70.10470998)
\curveto(314.97938127,69.94470869)(314.69438155,69.80470883)(314.38437988,69.68470998)
\curveto(314.244382,69.62470901)(314.10438214,69.57970905)(313.96437988,69.54970998)
\curveto(313.81438243,69.51970911)(313.65938259,69.48470915)(313.49937988,69.44470998)
\curveto(313.38938286,69.42470921)(313.27938297,69.40970922)(313.16937988,69.39970998)
\curveto(313.05938319,69.38970924)(312.9493833,69.37470926)(312.83937988,69.35470998)
\curveto(312.79938345,69.34470929)(312.75938349,69.33970929)(312.71937988,69.33970998)
\curveto(312.67938357,69.34970928)(312.63938361,69.34970928)(312.59937988,69.33970998)
\curveto(312.5493837,69.3297093)(312.49938375,69.32470931)(312.44937988,69.32470998)
\lineto(312.28437988,69.32470998)
\curveto(312.23438401,69.30470933)(312.18438406,69.29970933)(312.13437988,69.30970998)
\curveto(312.07438417,69.31970931)(312.01938423,69.31970931)(311.96937988,69.30970998)
\curveto(311.92938432,69.29970933)(311.88438436,69.29970933)(311.83437988,69.30970998)
\curveto(311.78438446,69.31970931)(311.73438451,69.31470932)(311.68437988,69.29470998)
\curveto(311.61438463,69.27470936)(311.53938471,69.26970936)(311.45937988,69.27970998)
\curveto(311.36938488,69.28970934)(311.28438496,69.29470934)(311.20437988,69.29470998)
\curveto(311.11438513,69.29470934)(311.01438523,69.28970934)(310.90437988,69.27970998)
\curveto(310.78438546,69.26970936)(310.68438556,69.27470936)(310.60437988,69.29470998)
\lineto(310.31937988,69.29470998)
\lineto(309.68937988,69.33970998)
\curveto(309.58938666,69.34970928)(309.49438675,69.35970927)(309.40437988,69.36970998)
\lineto(309.10437988,69.39970998)
\curveto(309.05438719,69.41970921)(309.00438724,69.42470921)(308.95437988,69.41470998)
\curveto(308.89438735,69.41470922)(308.83938741,69.42470921)(308.78937988,69.44470998)
\curveto(308.61938763,69.49470914)(308.45438779,69.5347091)(308.29437988,69.56470998)
\curveto(308.12438812,69.59470904)(307.96438828,69.64470899)(307.81437988,69.71470998)
\curveto(307.35438889,69.90470873)(306.97938927,70.12470851)(306.68937988,70.37470998)
\curveto(306.39938985,70.634708)(306.15439009,70.99470764)(305.95437988,71.45470998)
\curveto(305.90439034,71.58470705)(305.86939038,71.71470692)(305.84937988,71.84470998)
\curveto(305.82939042,71.98470665)(305.80439044,72.12470651)(305.77437988,72.26470998)
\curveto(305.76439048,72.3347063)(305.75939049,72.39970623)(305.75937988,72.45970998)
\curveto(305.75939049,72.51970611)(305.75439049,72.58470605)(305.74437988,72.65470998)
\curveto(305.72439052,73.48470515)(305.87439037,74.15470448)(306.19437988,74.66470998)
\curveto(306.50438974,75.17470346)(306.9443893,75.55470308)(307.51437988,75.80470998)
\curveto(307.63438861,75.85470278)(307.75938849,75.89970273)(307.88937988,75.93970998)
\curveto(308.01938823,75.97970265)(308.15438809,76.02470261)(308.29437988,76.07470998)
\curveto(308.37438787,76.09470254)(308.45938779,76.10970252)(308.54937988,76.11970998)
\lineto(308.78937988,76.17970998)
\curveto(308.89938735,76.20970242)(309.00938724,76.22470241)(309.11937988,76.22470998)
\curveto(309.22938702,76.2347024)(309.33938691,76.24970238)(309.44937988,76.26970998)
\curveto(309.49938675,76.28970234)(309.5443867,76.29470234)(309.58437988,76.28470998)
\curveto(309.62438662,76.28470235)(309.66438658,76.28970234)(309.70437988,76.29970998)
\curveto(309.75438649,76.30970232)(309.80938644,76.30970232)(309.86937988,76.29970998)
\curveto(309.91938633,76.29970233)(309.96938628,76.30470233)(310.01937988,76.31470998)
\lineto(310.15437988,76.31470998)
\curveto(310.21438603,76.3347023)(310.28438596,76.3347023)(310.36437988,76.31470998)
\curveto(310.43438581,76.30470233)(310.49938575,76.30970232)(310.55937988,76.32970998)
\curveto(310.58938566,76.33970229)(310.62938562,76.34470229)(310.67937988,76.34470998)
\lineto(310.79937988,76.34470998)
\lineto(311.26437988,76.34470998)
\moveto(313.58937988,74.79970998)
\curveto(313.26938298,74.89970373)(312.90438334,74.95970367)(312.49437988,74.97970998)
\curveto(312.08438416,74.99970363)(311.67438457,75.00970362)(311.26437988,75.00970998)
\curveto(310.83438541,75.00970362)(310.41438583,74.99970363)(310.00437988,74.97970998)
\curveto(309.59438665,74.95970367)(309.20938704,74.91470372)(308.84937988,74.84470998)
\curveto(308.48938776,74.77470386)(308.16938808,74.66470397)(307.88937988,74.51470998)
\curveto(307.59938865,74.37470426)(307.36438888,74.17970445)(307.18437988,73.92970998)
\curveto(307.07438917,73.76970486)(306.99438925,73.58970504)(306.94437988,73.38970998)
\curveto(306.88438936,73.18970544)(306.85438939,72.94470569)(306.85437988,72.65470998)
\curveto(306.87438937,72.634706)(306.88438936,72.59970603)(306.88437988,72.54970998)
\curveto(306.87438937,72.49970613)(306.87438937,72.45970617)(306.88437988,72.42970998)
\curveto(306.90438934,72.34970628)(306.92438932,72.27470636)(306.94437988,72.20470998)
\curveto(306.95438929,72.14470649)(306.97438927,72.07970655)(307.00437988,72.00970998)
\curveto(307.12438912,71.73970689)(307.29438895,71.51970711)(307.51437988,71.34970998)
\curveto(307.72438852,71.18970744)(307.96938828,71.05470758)(308.24937988,70.94470998)
\curveto(308.35938789,70.89470774)(308.47938777,70.85470778)(308.60937988,70.82470998)
\curveto(308.72938752,70.80470783)(308.85438739,70.77970785)(308.98437988,70.74970998)
\curveto(309.03438721,70.7297079)(309.08938716,70.71970791)(309.14937988,70.71970998)
\curveto(309.19938705,70.71970791)(309.249387,70.71470792)(309.29937988,70.70470998)
\curveto(309.38938686,70.69470794)(309.48438676,70.68470795)(309.58437988,70.67470998)
\curveto(309.67438657,70.66470797)(309.76938648,70.65470798)(309.86937988,70.64470998)
\curveto(309.9493863,70.64470799)(310.03438621,70.63970799)(310.12437988,70.62970998)
\lineto(310.36437988,70.62970998)
\lineto(310.54437988,70.62970998)
\curveto(310.57438567,70.61970801)(310.60938564,70.61470802)(310.64937988,70.61470998)
\lineto(310.78437988,70.61470998)
\lineto(311.23437988,70.61470998)
\curveto(311.31438493,70.61470802)(311.39938485,70.60970802)(311.48937988,70.59970998)
\curveto(311.56938468,70.59970803)(311.6443846,70.60970802)(311.71437988,70.62970998)
\lineto(311.98437988,70.62970998)
\curveto(312.00438424,70.629708)(312.03438421,70.62470801)(312.07437988,70.61470998)
\curveto(312.10438414,70.61470802)(312.12938412,70.61970801)(312.14937988,70.62970998)
\curveto(312.249384,70.63970799)(312.3493839,70.64470799)(312.44937988,70.64470998)
\curveto(312.53938371,70.65470798)(312.63938361,70.66470797)(312.74937988,70.67470998)
\curveto(312.86938338,70.70470793)(312.99438325,70.71970791)(313.12437988,70.71970998)
\curveto(313.244383,70.7297079)(313.35938289,70.75470788)(313.46937988,70.79470998)
\curveto(313.76938248,70.87470776)(314.03438221,70.95970767)(314.26437988,71.04970998)
\curveto(314.49438175,71.14970748)(314.70938154,71.29470734)(314.90937988,71.48470998)
\curveto(315.10938114,71.69470694)(315.25938099,71.95970667)(315.35937988,72.27970998)
\curveto(315.37938087,72.31970631)(315.38938086,72.35470628)(315.38937988,72.38470998)
\curveto(315.37938087,72.42470621)(315.38438086,72.46970616)(315.40437988,72.51970998)
\curveto(315.41438083,72.55970607)(315.42438082,72.629706)(315.43437988,72.72970998)
\curveto(315.4443808,72.83970579)(315.43938081,72.92470571)(315.41937988,72.98470998)
\curveto(315.39938085,73.05470558)(315.38938086,73.12470551)(315.38937988,73.19470998)
\curveto(315.37938087,73.26470537)(315.36438088,73.3297053)(315.34437988,73.38970998)
\curveto(315.28438096,73.58970504)(315.19938105,73.76970486)(315.08937988,73.92970998)
\curveto(315.06938118,73.95970467)(315.0493812,73.98470465)(315.02937988,74.00470998)
\lineto(314.96937988,74.06470998)
\curveto(314.9493813,74.10470453)(314.90938134,74.15470448)(314.84937988,74.21470998)
\curveto(314.70938154,74.31470432)(314.57938167,74.39970423)(314.45937988,74.46970998)
\curveto(314.33938191,74.53970409)(314.19438205,74.60970402)(314.02437988,74.67970998)
\curveto(313.95438229,74.70970392)(313.88438236,74.7297039)(313.81437988,74.73970998)
\curveto(313.7443825,74.75970387)(313.66938258,74.77970385)(313.58937988,74.79970998)
}
}
{
\newrgbcolor{curcolor}{0 0 0}
\pscustom[linestyle=none,fillstyle=solid,fillcolor=curcolor]
{
\newpath
\moveto(314.71437988,78.63431936)
\lineto(314.71437988,79.26431936)
\lineto(314.71437988,79.45931936)
\curveto(314.71438153,79.52931683)(314.72438152,79.58931677)(314.74437988,79.63931936)
\curveto(314.78438146,79.70931665)(314.82438142,79.7593166)(314.86437988,79.78931936)
\curveto(314.91438133,79.82931653)(314.97938127,79.84931651)(315.05937988,79.84931936)
\curveto(315.13938111,79.8593165)(315.22438102,79.86431649)(315.31437988,79.86431936)
\lineto(316.03437988,79.86431936)
\curveto(316.51437973,79.86431649)(316.92437932,79.80431655)(317.26437988,79.68431936)
\curveto(317.60437864,79.56431679)(317.87937837,79.36931699)(318.08937988,79.09931936)
\curveto(318.13937811,79.02931733)(318.18437806,78.9593174)(318.22437988,78.88931936)
\curveto(318.27437797,78.82931753)(318.31937793,78.7543176)(318.35937988,78.66431936)
\curveto(318.36937788,78.64431771)(318.37937787,78.61431774)(318.38937988,78.57431936)
\curveto(318.40937784,78.53431782)(318.41437783,78.48931787)(318.40437988,78.43931936)
\curveto(318.37437787,78.34931801)(318.29937795,78.29431806)(318.17937988,78.27431936)
\curveto(318.06937818,78.2543181)(317.97437827,78.26931809)(317.89437988,78.31931936)
\curveto(317.82437842,78.34931801)(317.75937849,78.39431796)(317.69937988,78.45431936)
\curveto(317.6493786,78.52431783)(317.59937865,78.58931777)(317.54937988,78.64931936)
\curveto(317.49937875,78.71931764)(317.42437882,78.77931758)(317.32437988,78.82931936)
\curveto(317.23437901,78.88931747)(317.1443791,78.93931742)(317.05437988,78.97931936)
\curveto(317.02437922,78.99931736)(316.96437928,79.02431733)(316.87437988,79.05431936)
\curveto(316.79437945,79.08431727)(316.72437952,79.08931727)(316.66437988,79.06931936)
\curveto(316.52437972,79.03931732)(316.43437981,78.97931738)(316.39437988,78.88931936)
\curveto(316.36437988,78.80931755)(316.3493799,78.71931764)(316.34937988,78.61931936)
\curveto(316.3493799,78.51931784)(316.32437992,78.43431792)(316.27437988,78.36431936)
\curveto(316.20438004,78.27431808)(316.06438018,78.22931813)(315.85437988,78.22931936)
\lineto(315.29937988,78.22931936)
\lineto(315.07437988,78.22931936)
\curveto(314.99438125,78.23931812)(314.92938132,78.2593181)(314.87937988,78.28931936)
\curveto(314.79938145,78.34931801)(314.75438149,78.41931794)(314.74437988,78.49931936)
\curveto(314.73438151,78.51931784)(314.72938152,78.53931782)(314.72937988,78.55931936)
\curveto(314.72938152,78.58931777)(314.72438152,78.61431774)(314.71437988,78.63431936)
}
}
{
\newrgbcolor{curcolor}{0 0 0}
\pscustom[linestyle=none,fillstyle=solid,fillcolor=curcolor]
{
}
}
{
\newrgbcolor{curcolor}{0 0 0}
\pscustom[linestyle=none,fillstyle=solid,fillcolor=curcolor]
{
\newpath
\moveto(305.74437988,89.26463186)
\curveto(305.73439051,89.95462722)(305.85439039,90.55462662)(306.10437988,91.06463186)
\curveto(306.35438989,91.58462559)(306.68938956,91.9796252)(307.10937988,92.24963186)
\curveto(307.18938906,92.29962488)(307.27938897,92.34462483)(307.37937988,92.38463186)
\curveto(307.46938878,92.42462475)(307.56438868,92.46962471)(307.66437988,92.51963186)
\curveto(307.76438848,92.55962462)(307.86438838,92.58962459)(307.96437988,92.60963186)
\curveto(308.06438818,92.62962455)(308.16938808,92.64962453)(308.27937988,92.66963186)
\curveto(308.32938792,92.68962449)(308.37438787,92.69462448)(308.41437988,92.68463186)
\curveto(308.45438779,92.6746245)(308.49938775,92.6796245)(308.54937988,92.69963186)
\curveto(308.59938765,92.70962447)(308.68438756,92.71462446)(308.80437988,92.71463186)
\curveto(308.91438733,92.71462446)(308.99938725,92.70962447)(309.05937988,92.69963186)
\curveto(309.11938713,92.6796245)(309.17938707,92.66962451)(309.23937988,92.66963186)
\curveto(309.29938695,92.6796245)(309.35938689,92.6746245)(309.41937988,92.65463186)
\curveto(309.55938669,92.61462456)(309.69438655,92.5796246)(309.82437988,92.54963186)
\curveto(309.95438629,92.51962466)(310.07938617,92.4796247)(310.19937988,92.42963186)
\curveto(310.33938591,92.36962481)(310.46438578,92.29962488)(310.57437988,92.21963186)
\curveto(310.68438556,92.14962503)(310.79438545,92.0746251)(310.90437988,91.99463186)
\lineto(310.96437988,91.93463186)
\curveto(310.98438526,91.92462525)(311.00438524,91.90962527)(311.02437988,91.88963186)
\curveto(311.18438506,91.76962541)(311.32938492,91.63462554)(311.45937988,91.48463186)
\curveto(311.58938466,91.33462584)(311.71438453,91.174626)(311.83437988,91.00463186)
\curveto(312.05438419,90.69462648)(312.25938399,90.39962678)(312.44937988,90.11963186)
\curveto(312.58938366,89.88962729)(312.72438352,89.65962752)(312.85437988,89.42963186)
\curveto(312.98438326,89.20962797)(313.11938313,88.98962819)(313.25937988,88.76963186)
\curveto(313.42938282,88.51962866)(313.60938264,88.2796289)(313.79937988,88.04963186)
\curveto(313.98938226,87.82962935)(314.21438203,87.63962954)(314.47437988,87.47963186)
\curveto(314.53438171,87.43962974)(314.59438165,87.40462977)(314.65437988,87.37463186)
\curveto(314.70438154,87.34462983)(314.76938148,87.31462986)(314.84937988,87.28463186)
\curveto(314.91938133,87.26462991)(314.97938127,87.25962992)(315.02937988,87.26963186)
\curveto(315.09938115,87.28962989)(315.15438109,87.32462985)(315.19437988,87.37463186)
\curveto(315.22438102,87.42462975)(315.244381,87.48462969)(315.25437988,87.55463186)
\lineto(315.25437988,87.79463186)
\lineto(315.25437988,88.54463186)
\lineto(315.25437988,91.34963186)
\lineto(315.25437988,92.00963186)
\curveto(315.25438099,92.09962508)(315.25938099,92.18462499)(315.26937988,92.26463186)
\curveto(315.26938098,92.34462483)(315.28938096,92.40962477)(315.32937988,92.45963186)
\curveto(315.36938088,92.50962467)(315.4443808,92.54962463)(315.55437988,92.57963186)
\curveto(315.65438059,92.61962456)(315.75438049,92.62962455)(315.85437988,92.60963186)
\lineto(315.98937988,92.60963186)
\curveto(316.05938019,92.58962459)(316.11938013,92.56962461)(316.16937988,92.54963186)
\curveto(316.21938003,92.52962465)(316.25937999,92.49462468)(316.28937988,92.44463186)
\curveto(316.32937992,92.39462478)(316.3493799,92.32462485)(316.34937988,92.23463186)
\lineto(316.34937988,91.96463186)
\lineto(316.34937988,91.06463186)
\lineto(316.34937988,87.55463186)
\lineto(316.34937988,86.48963186)
\curveto(316.3493799,86.40963077)(316.35437989,86.31963086)(316.36437988,86.21963186)
\curveto(316.36437988,86.11963106)(316.35437989,86.03463114)(316.33437988,85.96463186)
\curveto(316.26437998,85.75463142)(316.08438016,85.68963149)(315.79437988,85.76963186)
\curveto(315.75438049,85.7796314)(315.71938053,85.7796314)(315.68937988,85.76963186)
\curveto(315.6493806,85.76963141)(315.60438064,85.7796314)(315.55437988,85.79963186)
\curveto(315.47438077,85.81963136)(315.38938086,85.83963134)(315.29937988,85.85963186)
\curveto(315.20938104,85.8796313)(315.12438112,85.90463127)(315.04437988,85.93463186)
\curveto(314.55438169,86.09463108)(314.13938211,86.29463088)(313.79937988,86.53463186)
\curveto(313.5493827,86.71463046)(313.32438292,86.91963026)(313.12437988,87.14963186)
\curveto(312.91438333,87.3796298)(312.71938353,87.61962956)(312.53937988,87.86963186)
\curveto(312.35938389,88.12962905)(312.18938406,88.39462878)(312.02937988,88.66463186)
\curveto(311.85938439,88.94462823)(311.68438456,89.21462796)(311.50437988,89.47463186)
\curveto(311.42438482,89.58462759)(311.3493849,89.68962749)(311.27937988,89.78963186)
\curveto(311.20938504,89.89962728)(311.13438511,90.00962717)(311.05437988,90.11963186)
\curveto(311.02438522,90.15962702)(310.99438525,90.19462698)(310.96437988,90.22463186)
\curveto(310.92438532,90.26462691)(310.89438535,90.30462687)(310.87437988,90.34463186)
\curveto(310.76438548,90.48462669)(310.63938561,90.60962657)(310.49937988,90.71963186)
\curveto(310.46938578,90.73962644)(310.4443858,90.76462641)(310.42437988,90.79463186)
\curveto(310.39438585,90.82462635)(310.36438588,90.84962633)(310.33437988,90.86963186)
\curveto(310.23438601,90.94962623)(310.13438611,91.01462616)(310.03437988,91.06463186)
\curveto(309.93438631,91.12462605)(309.82438642,91.179626)(309.70437988,91.22963186)
\curveto(309.63438661,91.25962592)(309.55938669,91.2796259)(309.47937988,91.28963186)
\lineto(309.23937988,91.34963186)
\lineto(309.14937988,91.34963186)
\curveto(309.11938713,91.35962582)(309.08938716,91.36462581)(309.05937988,91.36463186)
\curveto(308.98938726,91.38462579)(308.89438735,91.38962579)(308.77437988,91.37963186)
\curveto(308.6443876,91.3796258)(308.5443877,91.36962581)(308.47437988,91.34963186)
\curveto(308.39438785,91.32962585)(308.31938793,91.30962587)(308.24937988,91.28963186)
\curveto(308.16938808,91.2796259)(308.08938816,91.25962592)(308.00937988,91.22963186)
\curveto(307.76938848,91.11962606)(307.56938868,90.96962621)(307.40937988,90.77963186)
\curveto(307.23938901,90.59962658)(307.09938915,90.3796268)(306.98937988,90.11963186)
\curveto(306.96938928,90.04962713)(306.95438929,89.9796272)(306.94437988,89.90963186)
\curveto(306.92438932,89.83962734)(306.90438934,89.76462741)(306.88437988,89.68463186)
\curveto(306.86438938,89.60462757)(306.85438939,89.49462768)(306.85437988,89.35463186)
\curveto(306.85438939,89.22462795)(306.86438938,89.11962806)(306.88437988,89.03963186)
\curveto(306.89438935,88.9796282)(306.89938935,88.92462825)(306.89937988,88.87463186)
\curveto(306.89938935,88.82462835)(306.90938934,88.7746284)(306.92937988,88.72463186)
\curveto(306.96938928,88.62462855)(307.00938924,88.52962865)(307.04937988,88.43963186)
\curveto(307.08938916,88.35962882)(307.13438911,88.2796289)(307.18437988,88.19963186)
\curveto(307.20438904,88.16962901)(307.22938902,88.13962904)(307.25937988,88.10963186)
\curveto(307.28938896,88.08962909)(307.31438893,88.06462911)(307.33437988,88.03463186)
\lineto(307.40937988,87.95963186)
\curveto(307.42938882,87.92962925)(307.4493888,87.90462927)(307.46937988,87.88463186)
\lineto(307.67937988,87.73463186)
\curveto(307.73938851,87.69462948)(307.80438844,87.64962953)(307.87437988,87.59963186)
\curveto(307.96438828,87.53962964)(308.06938818,87.48962969)(308.18937988,87.44963186)
\curveto(308.29938795,87.41962976)(308.40938784,87.38462979)(308.51937988,87.34463186)
\curveto(308.62938762,87.30462987)(308.77438747,87.2796299)(308.95437988,87.26963186)
\curveto(309.12438712,87.25962992)(309.249387,87.22962995)(309.32937988,87.17963186)
\curveto(309.40938684,87.12963005)(309.45438679,87.05463012)(309.46437988,86.95463186)
\curveto(309.47438677,86.85463032)(309.47938677,86.74463043)(309.47937988,86.62463186)
\curveto(309.47938677,86.58463059)(309.48438676,86.54463063)(309.49437988,86.50463186)
\curveto(309.49438675,86.46463071)(309.48938676,86.42963075)(309.47937988,86.39963186)
\curveto(309.45938679,86.34963083)(309.4493868,86.29963088)(309.44937988,86.24963186)
\curveto(309.4493868,86.20963097)(309.43938681,86.16963101)(309.41937988,86.12963186)
\curveto(309.35938689,86.03963114)(309.22438702,85.99463118)(309.01437988,85.99463186)
\lineto(308.89437988,85.99463186)
\curveto(308.83438741,86.00463117)(308.77438747,86.00963117)(308.71437988,86.00963186)
\curveto(308.6443876,86.01963116)(308.57938767,86.02963115)(308.51937988,86.03963186)
\curveto(308.40938784,86.05963112)(308.30938794,86.0796311)(308.21937988,86.09963186)
\curveto(308.11938813,86.11963106)(308.02438822,86.14963103)(307.93437988,86.18963186)
\curveto(307.86438838,86.20963097)(307.80438844,86.22963095)(307.75437988,86.24963186)
\lineto(307.57437988,86.30963186)
\curveto(307.31438893,86.42963075)(307.06938918,86.58463059)(306.83937988,86.77463186)
\curveto(306.60938964,86.9746302)(306.42438982,87.18962999)(306.28437988,87.41963186)
\curveto(306.20439004,87.52962965)(306.13939011,87.64462953)(306.08937988,87.76463186)
\lineto(305.93937988,88.15463186)
\curveto(305.88939036,88.26462891)(305.85939039,88.3796288)(305.84937988,88.49963186)
\curveto(305.82939042,88.61962856)(305.80439044,88.74462843)(305.77437988,88.87463186)
\curveto(305.77439047,88.94462823)(305.77439047,89.00962817)(305.77437988,89.06963186)
\curveto(305.76439048,89.12962805)(305.75439049,89.19462798)(305.74437988,89.26463186)
}
}
{
\newrgbcolor{curcolor}{0 0 0}
\pscustom[linestyle=none,fillstyle=solid,fillcolor=curcolor]
{
\newpath
\moveto(311.26437988,101.36424123)
\lineto(311.51937988,101.36424123)
\curveto(311.59938465,101.37423353)(311.67438457,101.36923353)(311.74437988,101.34924123)
\lineto(311.98437988,101.34924123)
\lineto(312.14937988,101.34924123)
\curveto(312.249384,101.32923357)(312.35438389,101.31923358)(312.46437988,101.31924123)
\curveto(312.56438368,101.31923358)(312.66438358,101.30923359)(312.76437988,101.28924123)
\lineto(312.91437988,101.28924123)
\curveto(313.05438319,101.25923364)(313.19438305,101.23923366)(313.33437988,101.22924123)
\curveto(313.46438278,101.21923368)(313.59438265,101.19423371)(313.72437988,101.15424123)
\curveto(313.80438244,101.13423377)(313.88938236,101.11423379)(313.97937988,101.09424123)
\lineto(314.21937988,101.03424123)
\lineto(314.51937988,100.91424123)
\curveto(314.60938164,100.88423402)(314.69938155,100.84923405)(314.78937988,100.80924123)
\curveto(315.00938124,100.70923419)(315.22438102,100.57423433)(315.43437988,100.40424123)
\curveto(315.6443806,100.24423466)(315.81438043,100.06923483)(315.94437988,99.87924123)
\curveto(315.98438026,99.82923507)(316.02438022,99.76923513)(316.06437988,99.69924123)
\curveto(316.09438015,99.63923526)(316.12938012,99.57923532)(316.16937988,99.51924123)
\curveto(316.21938003,99.43923546)(316.25937999,99.34423556)(316.28937988,99.23424123)
\curveto(316.31937993,99.12423578)(316.3493799,99.01923588)(316.37937988,98.91924123)
\curveto(316.41937983,98.80923609)(316.4443798,98.6992362)(316.45437988,98.58924123)
\curveto(316.46437978,98.47923642)(316.47937977,98.36423654)(316.49937988,98.24424123)
\curveto(316.50937974,98.2042367)(316.50937974,98.15923674)(316.49937988,98.10924123)
\curveto(316.49937975,98.06923683)(316.50437974,98.02923687)(316.51437988,97.98924123)
\curveto(316.52437972,97.94923695)(316.52937972,97.89423701)(316.52937988,97.82424123)
\curveto(316.52937972,97.75423715)(316.52437972,97.7042372)(316.51437988,97.67424123)
\curveto(316.49437975,97.62423728)(316.48937976,97.57923732)(316.49937988,97.53924123)
\curveto(316.50937974,97.4992374)(316.50937974,97.46423744)(316.49937988,97.43424123)
\lineto(316.49937988,97.34424123)
\curveto(316.47937977,97.28423762)(316.46437978,97.21923768)(316.45437988,97.14924123)
\curveto(316.45437979,97.08923781)(316.4493798,97.02423788)(316.43937988,96.95424123)
\curveto(316.38937986,96.78423812)(316.33937991,96.62423828)(316.28937988,96.47424123)
\curveto(316.23938001,96.32423858)(316.17438007,96.17923872)(316.09437988,96.03924123)
\curveto(316.05438019,95.98923891)(316.02438022,95.93423897)(316.00437988,95.87424123)
\curveto(315.97438027,95.82423908)(315.93938031,95.77423913)(315.89937988,95.72424123)
\curveto(315.71938053,95.48423942)(315.49938075,95.28423962)(315.23937988,95.12424123)
\curveto(314.97938127,94.96423994)(314.69438155,94.82424008)(314.38437988,94.70424123)
\curveto(314.244382,94.64424026)(314.10438214,94.5992403)(313.96437988,94.56924123)
\curveto(313.81438243,94.53924036)(313.65938259,94.5042404)(313.49937988,94.46424123)
\curveto(313.38938286,94.44424046)(313.27938297,94.42924047)(313.16937988,94.41924123)
\curveto(313.05938319,94.40924049)(312.9493833,94.39424051)(312.83937988,94.37424123)
\curveto(312.79938345,94.36424054)(312.75938349,94.35924054)(312.71937988,94.35924123)
\curveto(312.67938357,94.36924053)(312.63938361,94.36924053)(312.59937988,94.35924123)
\curveto(312.5493837,94.34924055)(312.49938375,94.34424056)(312.44937988,94.34424123)
\lineto(312.28437988,94.34424123)
\curveto(312.23438401,94.32424058)(312.18438406,94.31924058)(312.13437988,94.32924123)
\curveto(312.07438417,94.33924056)(312.01938423,94.33924056)(311.96937988,94.32924123)
\curveto(311.92938432,94.31924058)(311.88438436,94.31924058)(311.83437988,94.32924123)
\curveto(311.78438446,94.33924056)(311.73438451,94.33424057)(311.68437988,94.31424123)
\curveto(311.61438463,94.29424061)(311.53938471,94.28924061)(311.45937988,94.29924123)
\curveto(311.36938488,94.30924059)(311.28438496,94.31424059)(311.20437988,94.31424123)
\curveto(311.11438513,94.31424059)(311.01438523,94.30924059)(310.90437988,94.29924123)
\curveto(310.78438546,94.28924061)(310.68438556,94.29424061)(310.60437988,94.31424123)
\lineto(310.31937988,94.31424123)
\lineto(309.68937988,94.35924123)
\curveto(309.58938666,94.36924053)(309.49438675,94.37924052)(309.40437988,94.38924123)
\lineto(309.10437988,94.41924123)
\curveto(309.05438719,94.43924046)(309.00438724,94.44424046)(308.95437988,94.43424123)
\curveto(308.89438735,94.43424047)(308.83938741,94.44424046)(308.78937988,94.46424123)
\curveto(308.61938763,94.51424039)(308.45438779,94.55424035)(308.29437988,94.58424123)
\curveto(308.12438812,94.61424029)(307.96438828,94.66424024)(307.81437988,94.73424123)
\curveto(307.35438889,94.92423998)(306.97938927,95.14423976)(306.68937988,95.39424123)
\curveto(306.39938985,95.65423925)(306.15439009,96.01423889)(305.95437988,96.47424123)
\curveto(305.90439034,96.6042383)(305.86939038,96.73423817)(305.84937988,96.86424123)
\curveto(305.82939042,97.0042379)(305.80439044,97.14423776)(305.77437988,97.28424123)
\curveto(305.76439048,97.35423755)(305.75939049,97.41923748)(305.75937988,97.47924123)
\curveto(305.75939049,97.53923736)(305.75439049,97.6042373)(305.74437988,97.67424123)
\curveto(305.72439052,98.5042364)(305.87439037,99.17423573)(306.19437988,99.68424123)
\curveto(306.50438974,100.19423471)(306.9443893,100.57423433)(307.51437988,100.82424123)
\curveto(307.63438861,100.87423403)(307.75938849,100.91923398)(307.88937988,100.95924123)
\curveto(308.01938823,100.9992339)(308.15438809,101.04423386)(308.29437988,101.09424123)
\curveto(308.37438787,101.11423379)(308.45938779,101.12923377)(308.54937988,101.13924123)
\lineto(308.78937988,101.19924123)
\curveto(308.89938735,101.22923367)(309.00938724,101.24423366)(309.11937988,101.24424123)
\curveto(309.22938702,101.25423365)(309.33938691,101.26923363)(309.44937988,101.28924123)
\curveto(309.49938675,101.30923359)(309.5443867,101.31423359)(309.58437988,101.30424123)
\curveto(309.62438662,101.3042336)(309.66438658,101.30923359)(309.70437988,101.31924123)
\curveto(309.75438649,101.32923357)(309.80938644,101.32923357)(309.86937988,101.31924123)
\curveto(309.91938633,101.31923358)(309.96938628,101.32423358)(310.01937988,101.33424123)
\lineto(310.15437988,101.33424123)
\curveto(310.21438603,101.35423355)(310.28438596,101.35423355)(310.36437988,101.33424123)
\curveto(310.43438581,101.32423358)(310.49938575,101.32923357)(310.55937988,101.34924123)
\curveto(310.58938566,101.35923354)(310.62938562,101.36423354)(310.67937988,101.36424123)
\lineto(310.79937988,101.36424123)
\lineto(311.26437988,101.36424123)
\moveto(313.58937988,99.81924123)
\curveto(313.26938298,99.91923498)(312.90438334,99.97923492)(312.49437988,99.99924123)
\curveto(312.08438416,100.01923488)(311.67438457,100.02923487)(311.26437988,100.02924123)
\curveto(310.83438541,100.02923487)(310.41438583,100.01923488)(310.00437988,99.99924123)
\curveto(309.59438665,99.97923492)(309.20938704,99.93423497)(308.84937988,99.86424123)
\curveto(308.48938776,99.79423511)(308.16938808,99.68423522)(307.88937988,99.53424123)
\curveto(307.59938865,99.39423551)(307.36438888,99.1992357)(307.18437988,98.94924123)
\curveto(307.07438917,98.78923611)(306.99438925,98.60923629)(306.94437988,98.40924123)
\curveto(306.88438936,98.20923669)(306.85438939,97.96423694)(306.85437988,97.67424123)
\curveto(306.87438937,97.65423725)(306.88438936,97.61923728)(306.88437988,97.56924123)
\curveto(306.87438937,97.51923738)(306.87438937,97.47923742)(306.88437988,97.44924123)
\curveto(306.90438934,97.36923753)(306.92438932,97.29423761)(306.94437988,97.22424123)
\curveto(306.95438929,97.16423774)(306.97438927,97.0992378)(307.00437988,97.02924123)
\curveto(307.12438912,96.75923814)(307.29438895,96.53923836)(307.51437988,96.36924123)
\curveto(307.72438852,96.20923869)(307.96938828,96.07423883)(308.24937988,95.96424123)
\curveto(308.35938789,95.91423899)(308.47938777,95.87423903)(308.60937988,95.84424123)
\curveto(308.72938752,95.82423908)(308.85438739,95.7992391)(308.98437988,95.76924123)
\curveto(309.03438721,95.74923915)(309.08938716,95.73923916)(309.14937988,95.73924123)
\curveto(309.19938705,95.73923916)(309.249387,95.73423917)(309.29937988,95.72424123)
\curveto(309.38938686,95.71423919)(309.48438676,95.7042392)(309.58437988,95.69424123)
\curveto(309.67438657,95.68423922)(309.76938648,95.67423923)(309.86937988,95.66424123)
\curveto(309.9493863,95.66423924)(310.03438621,95.65923924)(310.12437988,95.64924123)
\lineto(310.36437988,95.64924123)
\lineto(310.54437988,95.64924123)
\curveto(310.57438567,95.63923926)(310.60938564,95.63423927)(310.64937988,95.63424123)
\lineto(310.78437988,95.63424123)
\lineto(311.23437988,95.63424123)
\curveto(311.31438493,95.63423927)(311.39938485,95.62923927)(311.48937988,95.61924123)
\curveto(311.56938468,95.61923928)(311.6443846,95.62923927)(311.71437988,95.64924123)
\lineto(311.98437988,95.64924123)
\curveto(312.00438424,95.64923925)(312.03438421,95.64423926)(312.07437988,95.63424123)
\curveto(312.10438414,95.63423927)(312.12938412,95.63923926)(312.14937988,95.64924123)
\curveto(312.249384,95.65923924)(312.3493839,95.66423924)(312.44937988,95.66424123)
\curveto(312.53938371,95.67423923)(312.63938361,95.68423922)(312.74937988,95.69424123)
\curveto(312.86938338,95.72423918)(312.99438325,95.73923916)(313.12437988,95.73924123)
\curveto(313.244383,95.74923915)(313.35938289,95.77423913)(313.46937988,95.81424123)
\curveto(313.76938248,95.89423901)(314.03438221,95.97923892)(314.26437988,96.06924123)
\curveto(314.49438175,96.16923873)(314.70938154,96.31423859)(314.90937988,96.50424123)
\curveto(315.10938114,96.71423819)(315.25938099,96.97923792)(315.35937988,97.29924123)
\curveto(315.37938087,97.33923756)(315.38938086,97.37423753)(315.38937988,97.40424123)
\curveto(315.37938087,97.44423746)(315.38438086,97.48923741)(315.40437988,97.53924123)
\curveto(315.41438083,97.57923732)(315.42438082,97.64923725)(315.43437988,97.74924123)
\curveto(315.4443808,97.85923704)(315.43938081,97.94423696)(315.41937988,98.00424123)
\curveto(315.39938085,98.07423683)(315.38938086,98.14423676)(315.38937988,98.21424123)
\curveto(315.37938087,98.28423662)(315.36438088,98.34923655)(315.34437988,98.40924123)
\curveto(315.28438096,98.60923629)(315.19938105,98.78923611)(315.08937988,98.94924123)
\curveto(315.06938118,98.97923592)(315.0493812,99.0042359)(315.02937988,99.02424123)
\lineto(314.96937988,99.08424123)
\curveto(314.9493813,99.12423578)(314.90938134,99.17423573)(314.84937988,99.23424123)
\curveto(314.70938154,99.33423557)(314.57938167,99.41923548)(314.45937988,99.48924123)
\curveto(314.33938191,99.55923534)(314.19438205,99.62923527)(314.02437988,99.69924123)
\curveto(313.95438229,99.72923517)(313.88438236,99.74923515)(313.81437988,99.75924123)
\curveto(313.7443825,99.77923512)(313.66938258,99.7992351)(313.58937988,99.81924123)
}
}
{
\newrgbcolor{curcolor}{0 0 0}
\pscustom[linestyle=none,fillstyle=solid,fillcolor=curcolor]
{
\newpath
\moveto(305.74437988,106.77385061)
\curveto(305.7443905,106.87384575)(305.75439049,106.96884566)(305.77437988,107.05885061)
\curveto(305.78439046,107.14884548)(305.81439043,107.21384541)(305.86437988,107.25385061)
\curveto(305.9443903,107.31384531)(306.0493902,107.34384528)(306.17937988,107.34385061)
\lineto(306.56937988,107.34385061)
\lineto(308.06937988,107.34385061)
\lineto(314.45937988,107.34385061)
\lineto(315.62937988,107.34385061)
\lineto(315.94437988,107.34385061)
\curveto(316.0443802,107.35384527)(316.12438012,107.33884529)(316.18437988,107.29885061)
\curveto(316.26437998,107.24884538)(316.31437993,107.17384545)(316.33437988,107.07385061)
\curveto(316.3443799,106.98384564)(316.3493799,106.87384575)(316.34937988,106.74385061)
\lineto(316.34937988,106.51885061)
\curveto(316.32937992,106.43884619)(316.31437993,106.36884626)(316.30437988,106.30885061)
\curveto(316.28437996,106.24884638)(316.24438,106.19884643)(316.18437988,106.15885061)
\curveto(316.12438012,106.11884651)(316.0493802,106.09884653)(315.95937988,106.09885061)
\lineto(315.65937988,106.09885061)
\lineto(314.56437988,106.09885061)
\lineto(309.22437988,106.09885061)
\curveto(309.13438711,106.07884655)(309.05938719,106.06384656)(308.99937988,106.05385061)
\curveto(308.92938732,106.05384657)(308.86938738,106.0238466)(308.81937988,105.96385061)
\curveto(308.76938748,105.89384673)(308.7443875,105.80384682)(308.74437988,105.69385061)
\curveto(308.73438751,105.59384703)(308.72938752,105.48384714)(308.72937988,105.36385061)
\lineto(308.72937988,104.22385061)
\lineto(308.72937988,103.72885061)
\curveto(308.71938753,103.56884906)(308.65938759,103.45884917)(308.54937988,103.39885061)
\curveto(308.51938773,103.37884925)(308.48938776,103.36884926)(308.45937988,103.36885061)
\curveto(308.41938783,103.36884926)(308.37438787,103.36384926)(308.32437988,103.35385061)
\curveto(308.20438804,103.33384929)(308.09438815,103.33884929)(307.99437988,103.36885061)
\curveto(307.89438835,103.40884922)(307.82438842,103.46384916)(307.78437988,103.53385061)
\curveto(307.73438851,103.61384901)(307.70938854,103.73384889)(307.70937988,103.89385061)
\curveto(307.70938854,104.05384857)(307.69438855,104.18884844)(307.66437988,104.29885061)
\curveto(307.65438859,104.34884828)(307.6493886,104.40384822)(307.64937988,104.46385061)
\curveto(307.63938861,104.5238481)(307.62438862,104.58384804)(307.60437988,104.64385061)
\curveto(307.55438869,104.79384783)(307.50438874,104.93884769)(307.45437988,105.07885061)
\curveto(307.39438885,105.21884741)(307.32438892,105.35384727)(307.24437988,105.48385061)
\curveto(307.15438909,105.623847)(307.0493892,105.74384688)(306.92937988,105.84385061)
\curveto(306.80938944,105.94384668)(306.67938957,106.03884659)(306.53937988,106.12885061)
\curveto(306.43938981,106.18884644)(306.32938992,106.23384639)(306.20937988,106.26385061)
\curveto(306.08939016,106.30384632)(305.98439026,106.35384627)(305.89437988,106.41385061)
\curveto(305.83439041,106.46384616)(305.79439045,106.53384609)(305.77437988,106.62385061)
\curveto(305.76439048,106.64384598)(305.75939049,106.66884596)(305.75937988,106.69885061)
\curveto(305.75939049,106.7288459)(305.75439049,106.75384587)(305.74437988,106.77385061)
}
}
{
\newrgbcolor{curcolor}{0 0 0}
\pscustom[linestyle=none,fillstyle=solid,fillcolor=curcolor]
{
\newpath
\moveto(305.74437988,115.12345998)
\curveto(305.7443905,115.22345513)(305.75439049,115.31845503)(305.77437988,115.40845998)
\curveto(305.78439046,115.49845485)(305.81439043,115.56345479)(305.86437988,115.60345998)
\curveto(305.9443903,115.66345469)(306.0493902,115.69345466)(306.17937988,115.69345998)
\lineto(306.56937988,115.69345998)
\lineto(308.06937988,115.69345998)
\lineto(314.45937988,115.69345998)
\lineto(315.62937988,115.69345998)
\lineto(315.94437988,115.69345998)
\curveto(316.0443802,115.70345465)(316.12438012,115.68845466)(316.18437988,115.64845998)
\curveto(316.26437998,115.59845475)(316.31437993,115.52345483)(316.33437988,115.42345998)
\curveto(316.3443799,115.33345502)(316.3493799,115.22345513)(316.34937988,115.09345998)
\lineto(316.34937988,114.86845998)
\curveto(316.32937992,114.78845556)(316.31437993,114.71845563)(316.30437988,114.65845998)
\curveto(316.28437996,114.59845575)(316.24438,114.5484558)(316.18437988,114.50845998)
\curveto(316.12438012,114.46845588)(316.0493802,114.4484559)(315.95937988,114.44845998)
\lineto(315.65937988,114.44845998)
\lineto(314.56437988,114.44845998)
\lineto(309.22437988,114.44845998)
\curveto(309.13438711,114.42845592)(309.05938719,114.41345594)(308.99937988,114.40345998)
\curveto(308.92938732,114.40345595)(308.86938738,114.37345598)(308.81937988,114.31345998)
\curveto(308.76938748,114.24345611)(308.7443875,114.1534562)(308.74437988,114.04345998)
\curveto(308.73438751,113.94345641)(308.72938752,113.83345652)(308.72937988,113.71345998)
\lineto(308.72937988,112.57345998)
\lineto(308.72937988,112.07845998)
\curveto(308.71938753,111.91845843)(308.65938759,111.80845854)(308.54937988,111.74845998)
\curveto(308.51938773,111.72845862)(308.48938776,111.71845863)(308.45937988,111.71845998)
\curveto(308.41938783,111.71845863)(308.37438787,111.71345864)(308.32437988,111.70345998)
\curveto(308.20438804,111.68345867)(308.09438815,111.68845866)(307.99437988,111.71845998)
\curveto(307.89438835,111.75845859)(307.82438842,111.81345854)(307.78437988,111.88345998)
\curveto(307.73438851,111.96345839)(307.70938854,112.08345827)(307.70937988,112.24345998)
\curveto(307.70938854,112.40345795)(307.69438855,112.53845781)(307.66437988,112.64845998)
\curveto(307.65438859,112.69845765)(307.6493886,112.7534576)(307.64937988,112.81345998)
\curveto(307.63938861,112.87345748)(307.62438862,112.93345742)(307.60437988,112.99345998)
\curveto(307.55438869,113.14345721)(307.50438874,113.28845706)(307.45437988,113.42845998)
\curveto(307.39438885,113.56845678)(307.32438892,113.70345665)(307.24437988,113.83345998)
\curveto(307.15438909,113.97345638)(307.0493892,114.09345626)(306.92937988,114.19345998)
\curveto(306.80938944,114.29345606)(306.67938957,114.38845596)(306.53937988,114.47845998)
\curveto(306.43938981,114.53845581)(306.32938992,114.58345577)(306.20937988,114.61345998)
\curveto(306.08939016,114.6534557)(305.98439026,114.70345565)(305.89437988,114.76345998)
\curveto(305.83439041,114.81345554)(305.79439045,114.88345547)(305.77437988,114.97345998)
\curveto(305.76439048,114.99345536)(305.75939049,115.01845533)(305.75937988,115.04845998)
\curveto(305.75939049,115.07845527)(305.75439049,115.10345525)(305.74437988,115.12345998)
}
}
{
\newrgbcolor{curcolor}{0 0 0}
\pscustom[linestyle=none,fillstyle=solid,fillcolor=curcolor]
{
\newpath
\moveto(336.48072632,42.02236623)
\curveto(336.53072706,42.04235669)(336.590727,42.06735666)(336.66072632,42.09736623)
\curveto(336.73072686,42.1273566)(336.80572679,42.14735658)(336.88572632,42.15736623)
\curveto(336.95572664,42.17735655)(337.02572657,42.17735655)(337.09572632,42.15736623)
\curveto(337.15572644,42.14735658)(337.20072639,42.10735662)(337.23072632,42.03736623)
\curveto(337.25072634,41.98735674)(337.26072633,41.9273568)(337.26072632,41.85736623)
\lineto(337.26072632,41.64736623)
\lineto(337.26072632,41.19736623)
\curveto(337.26072633,41.04735768)(337.23572636,40.9273578)(337.18572632,40.83736623)
\curveto(337.12572647,40.73735799)(337.02072657,40.66235807)(336.87072632,40.61236623)
\curveto(336.72072687,40.57235816)(336.58572701,40.5273582)(336.46572632,40.47736623)
\curveto(336.20572739,40.36735836)(335.93572766,40.26735846)(335.65572632,40.17736623)
\curveto(335.37572822,40.08735864)(335.10072849,39.98735874)(334.83072632,39.87736623)
\curveto(334.74072885,39.84735888)(334.65572894,39.81735891)(334.57572632,39.78736623)
\curveto(334.4957291,39.76735896)(334.42072917,39.73735899)(334.35072632,39.69736623)
\curveto(334.28072931,39.66735906)(334.22072937,39.62235911)(334.17072632,39.56236623)
\curveto(334.12072947,39.50235923)(334.08072951,39.42235931)(334.05072632,39.32236623)
\curveto(334.03072956,39.27235946)(334.02572957,39.21235952)(334.03572632,39.14236623)
\lineto(334.03572632,38.94736623)
\lineto(334.03572632,36.11236623)
\lineto(334.03572632,35.81236623)
\curveto(334.02572957,35.70236303)(334.02572957,35.59736313)(334.03572632,35.49736623)
\curveto(334.04572955,35.39736333)(334.06072953,35.30236343)(334.08072632,35.21236623)
\curveto(334.10072949,35.1323636)(334.14072945,35.07236366)(334.20072632,35.03236623)
\curveto(334.30072929,34.95236378)(334.41572918,34.89236384)(334.54572632,34.85236623)
\curveto(334.66572893,34.82236391)(334.7907288,34.78236395)(334.92072632,34.73236623)
\curveto(335.15072844,34.6323641)(335.3907282,34.53736419)(335.64072632,34.44736623)
\curveto(335.8907277,34.36736436)(336.13072746,34.27736445)(336.36072632,34.17736623)
\curveto(336.42072717,34.15736457)(336.4907271,34.1323646)(336.57072632,34.10236623)
\curveto(336.64072695,34.08236465)(336.71572688,34.05736467)(336.79572632,34.02736623)
\curveto(336.87572672,33.99736473)(336.95072664,33.96236477)(337.02072632,33.92236623)
\curveto(337.08072651,33.89236484)(337.12572647,33.85736487)(337.15572632,33.81736623)
\curveto(337.21572638,33.73736499)(337.25072634,33.6273651)(337.26072632,33.48736623)
\lineto(337.26072632,33.06736623)
\lineto(337.26072632,32.82736623)
\curveto(337.25072634,32.75736597)(337.22572637,32.69736603)(337.18572632,32.64736623)
\curveto(337.15572644,32.59736613)(337.11072648,32.56736616)(337.05072632,32.55736623)
\curveto(336.9907266,32.55736617)(336.93072666,32.56236617)(336.87072632,32.57236623)
\curveto(336.80072679,32.59236614)(336.73572686,32.61236612)(336.67572632,32.63236623)
\curveto(336.60572699,32.66236607)(336.55572704,32.68736604)(336.52572632,32.70736623)
\curveto(336.20572739,32.84736588)(335.8907277,32.97236576)(335.58072632,33.08236623)
\curveto(335.26072833,33.19236554)(334.94072865,33.31236542)(334.62072632,33.44236623)
\curveto(334.40072919,33.5323652)(334.18572941,33.61736511)(333.97572632,33.69736623)
\curveto(333.75572984,33.77736495)(333.53573006,33.86236487)(333.31572632,33.95236623)
\curveto(332.595731,34.25236448)(331.87073172,34.53736419)(331.14072632,34.80736623)
\curveto(330.40073319,35.07736365)(329.66573393,35.36236337)(328.93572632,35.66236623)
\curveto(328.67573492,35.77236296)(328.41073518,35.87236286)(328.14072632,35.96236623)
\curveto(327.87073572,36.06236267)(327.60573599,36.16736256)(327.34572632,36.27736623)
\curveto(327.23573636,36.3273624)(327.11573648,36.37236236)(326.98572632,36.41236623)
\curveto(326.84573675,36.46236227)(326.74573685,36.5323622)(326.68572632,36.62236623)
\curveto(326.64573695,36.66236207)(326.61573698,36.727362)(326.59572632,36.81736623)
\curveto(326.58573701,36.83736189)(326.58573701,36.85736187)(326.59572632,36.87736623)
\curveto(326.595737,36.90736182)(326.590737,36.9323618)(326.58072632,36.95236623)
\curveto(326.58073701,37.1323616)(326.58073701,37.34236139)(326.58072632,37.58236623)
\curveto(326.57073702,37.82236091)(326.60573699,37.99736073)(326.68572632,38.10736623)
\curveto(326.74573685,38.18736054)(326.84573675,38.24736048)(326.98572632,38.28736623)
\curveto(327.11573648,38.33736039)(327.23573636,38.38736034)(327.34572632,38.43736623)
\curveto(327.57573602,38.53736019)(327.80573579,38.6273601)(328.03572632,38.70736623)
\curveto(328.26573533,38.78735994)(328.4957351,38.87735985)(328.72572632,38.97736623)
\curveto(328.92573467,39.05735967)(329.13073446,39.1323596)(329.34072632,39.20236623)
\curveto(329.55073404,39.28235945)(329.75573384,39.36735936)(329.95572632,39.45736623)
\curveto(330.68573291,39.75735897)(331.42573217,40.04235869)(332.17572632,40.31236623)
\curveto(332.91573068,40.59235814)(333.65072994,40.88735784)(334.38072632,41.19736623)
\curveto(334.47072912,41.23735749)(334.55572904,41.26735746)(334.63572632,41.28736623)
\curveto(334.71572888,41.31735741)(334.80072879,41.34735738)(334.89072632,41.37736623)
\curveto(335.15072844,41.48735724)(335.41572818,41.59235714)(335.68572632,41.69236623)
\curveto(335.95572764,41.80235693)(336.22072737,41.91235682)(336.48072632,42.02236623)
\moveto(332.83572632,38.81236623)
\curveto(332.80573079,38.90235983)(332.75573084,38.95735977)(332.68572632,38.97736623)
\curveto(332.61573098,39.00735972)(332.54073105,39.01235972)(332.46072632,38.99236623)
\curveto(332.37073122,38.98235975)(332.28573131,38.95735977)(332.20572632,38.91736623)
\curveto(332.11573148,38.88735984)(332.04073155,38.85735987)(331.98072632,38.82736623)
\curveto(331.94073165,38.80735992)(331.90573169,38.79735993)(331.87572632,38.79736623)
\curveto(331.84573175,38.79735993)(331.81073178,38.78735994)(331.77072632,38.76736623)
\lineto(331.53072632,38.67736623)
\curveto(331.44073215,38.65736007)(331.35073224,38.6273601)(331.26072632,38.58736623)
\curveto(330.90073269,38.43736029)(330.53573306,38.30236043)(330.16572632,38.18236623)
\curveto(329.78573381,38.07236066)(329.41573418,37.94236079)(329.05572632,37.79236623)
\curveto(328.94573465,37.74236099)(328.83573476,37.69736103)(328.72572632,37.65736623)
\curveto(328.61573498,37.6273611)(328.51073508,37.58736114)(328.41072632,37.53736623)
\curveto(328.36073523,37.51736121)(328.31573528,37.49236124)(328.27572632,37.46236623)
\curveto(328.22573537,37.44236129)(328.20073539,37.39236134)(328.20072632,37.31236623)
\curveto(328.22073537,37.29236144)(328.23573536,37.27236146)(328.24572632,37.25236623)
\curveto(328.25573534,37.2323615)(328.27073532,37.21236152)(328.29072632,37.19236623)
\curveto(328.34073525,37.15236158)(328.3957352,37.12236161)(328.45572632,37.10236623)
\curveto(328.50573509,37.08236165)(328.56073503,37.06236167)(328.62072632,37.04236623)
\curveto(328.73073486,36.99236174)(328.84073475,36.95236178)(328.95072632,36.92236623)
\curveto(329.06073453,36.89236184)(329.17073442,36.85236188)(329.28072632,36.80236623)
\curveto(329.67073392,36.6323621)(330.06573353,36.48236225)(330.46572632,36.35236623)
\curveto(330.86573273,36.2323625)(331.25573234,36.09236264)(331.63572632,35.93236623)
\lineto(331.78572632,35.87236623)
\curveto(331.83573176,35.86236287)(331.88573171,35.84736288)(331.93572632,35.82736623)
\lineto(332.17572632,35.73736623)
\curveto(332.25573134,35.70736302)(332.33573126,35.68236305)(332.41572632,35.66236623)
\curveto(332.46573113,35.64236309)(332.52073107,35.6323631)(332.58072632,35.63236623)
\curveto(332.64073095,35.64236309)(332.6907309,35.65736307)(332.73072632,35.67736623)
\curveto(332.81073078,35.727363)(332.85573074,35.8323629)(332.86572632,35.99236623)
\lineto(332.86572632,36.44236623)
\lineto(332.86572632,38.04736623)
\curveto(332.86573073,38.15736057)(332.87073072,38.29236044)(332.88072632,38.45236623)
\curveto(332.88073071,38.61236012)(332.86573073,38.73236)(332.83572632,38.81236623)
}
}
{
\newrgbcolor{curcolor}{0 0 0}
\pscustom[linestyle=none,fillstyle=solid,fillcolor=curcolor]
{
\newpath
\moveto(333.22572632,50.56392873)
\curveto(333.27573032,50.57392038)(333.34573025,50.57892038)(333.43572632,50.57892873)
\curveto(333.51573008,50.57892038)(333.58073001,50.57392038)(333.63072632,50.56392873)
\curveto(333.67072992,50.56392039)(333.71072988,50.5589204)(333.75072632,50.54892873)
\lineto(333.87072632,50.54892873)
\curveto(333.95072964,50.52892043)(334.03072956,50.51892044)(334.11072632,50.51892873)
\curveto(334.1907294,50.51892044)(334.27072932,50.50892045)(334.35072632,50.48892873)
\curveto(334.3907292,50.47892048)(334.43072916,50.47392048)(334.47072632,50.47392873)
\curveto(334.50072909,50.47392048)(334.53572906,50.46892049)(334.57572632,50.45892873)
\curveto(334.68572891,50.42892053)(334.7907288,50.39892056)(334.89072632,50.36892873)
\curveto(334.9907286,50.34892061)(335.0907285,50.31892064)(335.19072632,50.27892873)
\curveto(335.54072805,50.13892082)(335.85572774,49.96892099)(336.13572632,49.76892873)
\curveto(336.41572718,49.56892139)(336.65572694,49.31892164)(336.85572632,49.01892873)
\curveto(336.95572664,48.86892209)(337.04072655,48.72392223)(337.11072632,48.58392873)
\curveto(337.16072643,48.47392248)(337.20072639,48.36392259)(337.23072632,48.25392873)
\curveto(337.26072633,48.1539228)(337.2907263,48.04892291)(337.32072632,47.93892873)
\curveto(337.34072625,47.86892309)(337.35072624,47.80392315)(337.35072632,47.74392873)
\curveto(337.36072623,47.68392327)(337.37572622,47.62392333)(337.39572632,47.56392873)
\lineto(337.39572632,47.41392873)
\curveto(337.41572618,47.36392359)(337.42572617,47.28892367)(337.42572632,47.18892873)
\curveto(337.43572616,47.08892387)(337.43072616,47.00892395)(337.41072632,46.94892873)
\lineto(337.41072632,46.79892873)
\curveto(337.40072619,46.7589242)(337.3957262,46.71392424)(337.39572632,46.66392873)
\curveto(337.3957262,46.62392433)(337.3907262,46.57892438)(337.38072632,46.52892873)
\curveto(337.34072625,46.37892458)(337.30572629,46.22892473)(337.27572632,46.07892873)
\curveto(337.24572635,45.93892502)(337.20072639,45.79892516)(337.14072632,45.65892873)
\curveto(337.06072653,45.4589255)(336.96072663,45.27892568)(336.84072632,45.11892873)
\lineto(336.69072632,44.93892873)
\curveto(336.63072696,44.87892608)(336.590727,44.80892615)(336.57072632,44.72892873)
\curveto(336.56072703,44.66892629)(336.57572702,44.61892634)(336.61572632,44.57892873)
\curveto(336.64572695,44.54892641)(336.6907269,44.52392643)(336.75072632,44.50392873)
\curveto(336.81072678,44.49392646)(336.87572672,44.48392647)(336.94572632,44.47392873)
\curveto(337.00572659,44.47392648)(337.05072654,44.46392649)(337.08072632,44.44392873)
\curveto(337.13072646,44.40392655)(337.17572642,44.3589266)(337.21572632,44.30892873)
\curveto(337.23572636,44.2589267)(337.25072634,44.18892677)(337.26072632,44.09892873)
\lineto(337.26072632,43.82892873)
\curveto(337.26072633,43.73892722)(337.25572634,43.6539273)(337.24572632,43.57392873)
\curveto(337.22572637,43.49392746)(337.20572639,43.43392752)(337.18572632,43.39392873)
\curveto(337.16572643,43.37392758)(337.14072645,43.3539276)(337.11072632,43.33392873)
\lineto(337.02072632,43.27392873)
\curveto(336.94072665,43.24392771)(336.82072677,43.22892773)(336.66072632,43.22892873)
\curveto(336.50072709,43.23892772)(336.36572723,43.24392771)(336.25572632,43.24392873)
\lineto(327.45072632,43.24392873)
\curveto(327.33073626,43.24392771)(327.20573639,43.23892772)(327.07572632,43.22892873)
\curveto(326.93573666,43.22892773)(326.82573677,43.2539277)(326.74572632,43.30392873)
\curveto(326.68573691,43.34392761)(326.63573696,43.40892755)(326.59572632,43.49892873)
\curveto(326.595737,43.51892744)(326.595737,43.54392741)(326.59572632,43.57392873)
\curveto(326.58573701,43.60392735)(326.58073701,43.62892733)(326.58072632,43.64892873)
\curveto(326.57073702,43.78892717)(326.57073702,43.93392702)(326.58072632,44.08392873)
\curveto(326.58073701,44.24392671)(326.62073697,44.3539266)(326.70072632,44.41392873)
\curveto(326.78073681,44.46392649)(326.8957367,44.48892647)(327.04572632,44.48892873)
\lineto(327.45072632,44.48892873)
\lineto(329.20572632,44.48892873)
\lineto(329.46072632,44.48892873)
\lineto(329.74572632,44.48892873)
\curveto(329.83573376,44.49892646)(329.92073367,44.50892645)(330.00072632,44.51892873)
\curveto(330.07073352,44.53892642)(330.12073347,44.56892639)(330.15072632,44.60892873)
\curveto(330.18073341,44.64892631)(330.18573341,44.69392626)(330.16572632,44.74392873)
\curveto(330.14573345,44.79392616)(330.12573347,44.83392612)(330.10572632,44.86392873)
\curveto(330.06573353,44.91392604)(330.02573357,44.958926)(329.98572632,44.99892873)
\lineto(329.86572632,45.14892873)
\curveto(329.81573378,45.21892574)(329.77073382,45.28892567)(329.73072632,45.35892873)
\lineto(329.61072632,45.59892873)
\curveto(329.52073407,45.77892518)(329.45573414,45.99392496)(329.41572632,46.24392873)
\curveto(329.37573422,46.49392446)(329.35573424,46.74892421)(329.35572632,47.00892873)
\curveto(329.35573424,47.26892369)(329.38073421,47.52392343)(329.43072632,47.77392873)
\curveto(329.47073412,48.02392293)(329.53073406,48.24392271)(329.61072632,48.43392873)
\curveto(329.78073381,48.83392212)(330.01573358,49.17892178)(330.31572632,49.46892873)
\curveto(330.61573298,49.7589212)(330.96573263,49.98892097)(331.36572632,50.15892873)
\curveto(331.47573212,50.20892075)(331.58573201,50.24892071)(331.69572632,50.27892873)
\curveto(331.7957318,50.31892064)(331.90073169,50.3589206)(332.01072632,50.39892873)
\curveto(332.12073147,50.42892053)(332.23573136,50.44892051)(332.35572632,50.45892873)
\lineto(332.68572632,50.51892873)
\curveto(332.71573088,50.52892043)(332.75073084,50.53392042)(332.79072632,50.53392873)
\curveto(332.82073077,50.53392042)(332.85073074,50.53892042)(332.88072632,50.54892873)
\curveto(332.94073065,50.56892039)(333.00073059,50.56892039)(333.06072632,50.54892873)
\curveto(333.11073048,50.53892042)(333.16573043,50.54392041)(333.22572632,50.56392873)
\moveto(333.61572632,49.22892873)
\curveto(333.56573003,49.24892171)(333.50573009,49.2539217)(333.43572632,49.24392873)
\curveto(333.36573023,49.23392172)(333.30073029,49.22892173)(333.24072632,49.22892873)
\curveto(333.07073052,49.22892173)(332.91073068,49.21892174)(332.76072632,49.19892873)
\curveto(332.61073098,49.18892177)(332.47573112,49.1589218)(332.35572632,49.10892873)
\curveto(332.25573134,49.07892188)(332.16573143,49.0539219)(332.08572632,49.03392873)
\curveto(332.00573159,49.01392194)(331.92573167,48.98392197)(331.84572632,48.94392873)
\curveto(331.595732,48.83392212)(331.36573223,48.68392227)(331.15572632,48.49392873)
\curveto(330.93573266,48.30392265)(330.77073282,48.08392287)(330.66072632,47.83392873)
\curveto(330.63073296,47.7539232)(330.60573299,47.67392328)(330.58572632,47.59392873)
\curveto(330.55573304,47.52392343)(330.53073306,47.44892351)(330.51072632,47.36892873)
\curveto(330.48073311,47.2589237)(330.46573313,47.14892381)(330.46572632,47.03892873)
\curveto(330.45573314,46.92892403)(330.45073314,46.80892415)(330.45072632,46.67892873)
\curveto(330.46073313,46.62892433)(330.47073312,46.58392437)(330.48072632,46.54392873)
\lineto(330.48072632,46.40892873)
\lineto(330.54072632,46.13892873)
\curveto(330.56073303,46.0589249)(330.590733,45.97892498)(330.63072632,45.89892873)
\curveto(330.77073282,45.5589254)(330.98073261,45.28892567)(331.26072632,45.08892873)
\curveto(331.53073206,44.88892607)(331.85073174,44.72892623)(332.22072632,44.60892873)
\curveto(332.33073126,44.56892639)(332.44073115,44.54392641)(332.55072632,44.53392873)
\curveto(332.66073093,44.52392643)(332.77573082,44.50392645)(332.89572632,44.47392873)
\curveto(332.94573065,44.46392649)(332.9907306,44.46392649)(333.03072632,44.47392873)
\curveto(333.07073052,44.48392647)(333.11573048,44.47892648)(333.16572632,44.45892873)
\curveto(333.21573038,44.44892651)(333.2907303,44.44392651)(333.39072632,44.44392873)
\curveto(333.48073011,44.44392651)(333.55073004,44.44892651)(333.60072632,44.45892873)
\lineto(333.72072632,44.45892873)
\curveto(333.76072983,44.46892649)(333.80072979,44.47392648)(333.84072632,44.47392873)
\curveto(333.88072971,44.47392648)(333.91572968,44.47892648)(333.94572632,44.48892873)
\curveto(333.97572962,44.49892646)(334.01072958,44.50392645)(334.05072632,44.50392873)
\curveto(334.08072951,44.50392645)(334.11072948,44.50892645)(334.14072632,44.51892873)
\curveto(334.22072937,44.53892642)(334.30072929,44.5539264)(334.38072632,44.56392873)
\lineto(334.62072632,44.62392873)
\curveto(334.96072863,44.73392622)(335.25072834,44.88392607)(335.49072632,45.07392873)
\curveto(335.73072786,45.27392568)(335.93072766,45.51892544)(336.09072632,45.80892873)
\curveto(336.14072745,45.89892506)(336.18072741,45.99392496)(336.21072632,46.09392873)
\curveto(336.23072736,46.19392476)(336.25572734,46.29892466)(336.28572632,46.40892873)
\curveto(336.30572729,46.4589245)(336.31572728,46.50392445)(336.31572632,46.54392873)
\curveto(336.30572729,46.59392436)(336.30572729,46.64392431)(336.31572632,46.69392873)
\curveto(336.32572727,46.73392422)(336.33072726,46.77892418)(336.33072632,46.82892873)
\lineto(336.33072632,46.96392873)
\lineto(336.33072632,47.09892873)
\curveto(336.32072727,47.13892382)(336.31572728,47.17392378)(336.31572632,47.20392873)
\curveto(336.31572728,47.23392372)(336.31072728,47.26892369)(336.30072632,47.30892873)
\curveto(336.28072731,47.38892357)(336.26572733,47.46392349)(336.25572632,47.53392873)
\curveto(336.23572736,47.60392335)(336.21072738,47.67892328)(336.18072632,47.75892873)
\curveto(336.05072754,48.06892289)(335.88072771,48.31892264)(335.67072632,48.50892873)
\curveto(335.45072814,48.69892226)(335.18572841,48.8589221)(334.87572632,48.98892873)
\curveto(334.73572886,49.03892192)(334.595729,49.07392188)(334.45572632,49.09392873)
\curveto(334.30572929,49.12392183)(334.15572944,49.1589218)(334.00572632,49.19892873)
\curveto(333.95572964,49.21892174)(333.91072968,49.22392173)(333.87072632,49.21392873)
\curveto(333.82072977,49.21392174)(333.77072982,49.21892174)(333.72072632,49.22892873)
\lineto(333.61572632,49.22892873)
}
}
{
\newrgbcolor{curcolor}{0 0 0}
\pscustom[linestyle=none,fillstyle=solid,fillcolor=curcolor]
{
\newpath
\moveto(329.35572632,55.69017873)
\curveto(329.35573424,55.92017394)(329.41573418,56.05017381)(329.53572632,56.08017873)
\curveto(329.64573395,56.11017375)(329.81073378,56.12517374)(330.03072632,56.12517873)
\lineto(330.31572632,56.12517873)
\curveto(330.40573319,56.12517374)(330.48073311,56.10017376)(330.54072632,56.05017873)
\curveto(330.62073297,55.99017387)(330.66573293,55.90517396)(330.67572632,55.79517873)
\curveto(330.67573292,55.68517418)(330.6907329,55.57517429)(330.72072632,55.46517873)
\curveto(330.75073284,55.32517454)(330.78073281,55.19017467)(330.81072632,55.06017873)
\curveto(330.84073275,54.94017492)(330.88073271,54.82517504)(330.93072632,54.71517873)
\curveto(331.06073253,54.42517544)(331.24073235,54.19017567)(331.47072632,54.01017873)
\curveto(331.6907319,53.83017603)(331.94573165,53.67517619)(332.23572632,53.54517873)
\curveto(332.34573125,53.50517636)(332.46073113,53.47517639)(332.58072632,53.45517873)
\curveto(332.6907309,53.43517643)(332.80573079,53.41017645)(332.92572632,53.38017873)
\curveto(332.97573062,53.37017649)(333.02573057,53.3651765)(333.07572632,53.36517873)
\curveto(333.12573047,53.37517649)(333.17573042,53.37517649)(333.22572632,53.36517873)
\curveto(333.34573025,53.33517653)(333.48573011,53.32017654)(333.64572632,53.32017873)
\curveto(333.7957298,53.33017653)(333.94072965,53.33517653)(334.08072632,53.33517873)
\lineto(335.92572632,53.33517873)
\lineto(336.27072632,53.33517873)
\curveto(336.3907272,53.33517653)(336.50572709,53.33017653)(336.61572632,53.32017873)
\curveto(336.72572687,53.31017655)(336.82072677,53.30517656)(336.90072632,53.30517873)
\curveto(336.98072661,53.31517655)(337.05072654,53.29517657)(337.11072632,53.24517873)
\curveto(337.18072641,53.19517667)(337.22072637,53.11517675)(337.23072632,53.00517873)
\curveto(337.24072635,52.90517696)(337.24572635,52.79517707)(337.24572632,52.67517873)
\lineto(337.24572632,52.40517873)
\curveto(337.22572637,52.35517751)(337.21072638,52.30517756)(337.20072632,52.25517873)
\curveto(337.18072641,52.21517765)(337.15572644,52.18517768)(337.12572632,52.16517873)
\curveto(337.05572654,52.11517775)(336.97072662,52.08517778)(336.87072632,52.07517873)
\lineto(336.54072632,52.07517873)
\lineto(335.38572632,52.07517873)
\lineto(331.23072632,52.07517873)
\lineto(330.19572632,52.07517873)
\lineto(329.89572632,52.07517873)
\curveto(329.7957338,52.08517778)(329.71073388,52.11517775)(329.64072632,52.16517873)
\curveto(329.60073399,52.19517767)(329.57073402,52.24517762)(329.55072632,52.31517873)
\curveto(329.53073406,52.39517747)(329.52073407,52.48017738)(329.52072632,52.57017873)
\curveto(329.51073408,52.6601772)(329.51073408,52.75017711)(329.52072632,52.84017873)
\curveto(329.53073406,52.93017693)(329.54573405,53.00017686)(329.56572632,53.05017873)
\curveto(329.595734,53.13017673)(329.65573394,53.18017668)(329.74572632,53.20017873)
\curveto(329.82573377,53.23017663)(329.91573368,53.24517662)(330.01572632,53.24517873)
\lineto(330.31572632,53.24517873)
\curveto(330.41573318,53.24517662)(330.50573309,53.2651766)(330.58572632,53.30517873)
\curveto(330.60573299,53.31517655)(330.62073297,53.32517654)(330.63072632,53.33517873)
\lineto(330.67572632,53.38017873)
\curveto(330.67573292,53.49017637)(330.63073296,53.58017628)(330.54072632,53.65017873)
\curveto(330.44073315,53.72017614)(330.36073323,53.78017608)(330.30072632,53.83017873)
\lineto(330.21072632,53.92017873)
\curveto(330.10073349,54.01017585)(329.98573361,54.13517573)(329.86572632,54.29517873)
\curveto(329.74573385,54.45517541)(329.65573394,54.60517526)(329.59572632,54.74517873)
\curveto(329.54573405,54.83517503)(329.51073408,54.93017493)(329.49072632,55.03017873)
\curveto(329.46073413,55.13017473)(329.43073416,55.23517463)(329.40072632,55.34517873)
\curveto(329.3907342,55.40517446)(329.38573421,55.4651744)(329.38572632,55.52517873)
\curveto(329.37573422,55.58517428)(329.36573423,55.64017422)(329.35572632,55.69017873)
}
}
{
\newrgbcolor{curcolor}{0 0 0}
\pscustom[linestyle=none,fillstyle=solid,fillcolor=curcolor]
{
}
}
{
\newrgbcolor{curcolor}{0 0 0}
\pscustom[linestyle=none,fillstyle=solid,fillcolor=curcolor]
{
\newpath
\moveto(332.17572632,67.99510061)
\lineto(332.43072632,67.99510061)
\curveto(332.51073108,68.0050929)(332.58573101,68.00009291)(332.65572632,67.98010061)
\lineto(332.89572632,67.98010061)
\lineto(333.06072632,67.98010061)
\curveto(333.16073043,67.96009295)(333.26573033,67.95009296)(333.37572632,67.95010061)
\curveto(333.47573012,67.95009296)(333.57573002,67.94009297)(333.67572632,67.92010061)
\lineto(333.82572632,67.92010061)
\curveto(333.96572963,67.89009302)(334.10572949,67.87009304)(334.24572632,67.86010061)
\curveto(334.37572922,67.85009306)(334.50572909,67.82509308)(334.63572632,67.78510061)
\curveto(334.71572888,67.76509314)(334.80072879,67.74509316)(334.89072632,67.72510061)
\lineto(335.13072632,67.66510061)
\lineto(335.43072632,67.54510061)
\curveto(335.52072807,67.51509339)(335.61072798,67.48009343)(335.70072632,67.44010061)
\curveto(335.92072767,67.34009357)(336.13572746,67.2050937)(336.34572632,67.03510061)
\curveto(336.55572704,66.87509403)(336.72572687,66.70009421)(336.85572632,66.51010061)
\curveto(336.8957267,66.46009445)(336.93572666,66.40009451)(336.97572632,66.33010061)
\curveto(337.00572659,66.27009464)(337.04072655,66.2100947)(337.08072632,66.15010061)
\curveto(337.13072646,66.07009484)(337.17072642,65.97509493)(337.20072632,65.86510061)
\curveto(337.23072636,65.75509515)(337.26072633,65.65009526)(337.29072632,65.55010061)
\curveto(337.33072626,65.44009547)(337.35572624,65.33009558)(337.36572632,65.22010061)
\curveto(337.37572622,65.1100958)(337.3907262,64.99509591)(337.41072632,64.87510061)
\curveto(337.42072617,64.83509607)(337.42072617,64.79009612)(337.41072632,64.74010061)
\curveto(337.41072618,64.70009621)(337.41572618,64.66009625)(337.42572632,64.62010061)
\curveto(337.43572616,64.58009633)(337.44072615,64.52509638)(337.44072632,64.45510061)
\curveto(337.44072615,64.38509652)(337.43572616,64.33509657)(337.42572632,64.30510061)
\curveto(337.40572619,64.25509665)(337.40072619,64.2100967)(337.41072632,64.17010061)
\curveto(337.42072617,64.13009678)(337.42072617,64.09509681)(337.41072632,64.06510061)
\lineto(337.41072632,63.97510061)
\curveto(337.3907262,63.91509699)(337.37572622,63.85009706)(337.36572632,63.78010061)
\curveto(337.36572623,63.72009719)(337.36072623,63.65509725)(337.35072632,63.58510061)
\curveto(337.30072629,63.41509749)(337.25072634,63.25509765)(337.20072632,63.10510061)
\curveto(337.15072644,62.95509795)(337.08572651,62.8100981)(337.00572632,62.67010061)
\curveto(336.96572663,62.62009829)(336.93572666,62.56509834)(336.91572632,62.50510061)
\curveto(336.88572671,62.45509845)(336.85072674,62.4050985)(336.81072632,62.35510061)
\curveto(336.63072696,62.11509879)(336.41072718,61.91509899)(336.15072632,61.75510061)
\curveto(335.8907277,61.59509931)(335.60572799,61.45509945)(335.29572632,61.33510061)
\curveto(335.15572844,61.27509963)(335.01572858,61.23009968)(334.87572632,61.20010061)
\curveto(334.72572887,61.17009974)(334.57072902,61.13509977)(334.41072632,61.09510061)
\curveto(334.30072929,61.07509983)(334.1907294,61.06009985)(334.08072632,61.05010061)
\curveto(333.97072962,61.04009987)(333.86072973,61.02509988)(333.75072632,61.00510061)
\curveto(333.71072988,60.99509991)(333.67072992,60.99009992)(333.63072632,60.99010061)
\curveto(333.59073,61.00009991)(333.55073004,61.00009991)(333.51072632,60.99010061)
\curveto(333.46073013,60.98009993)(333.41073018,60.97509993)(333.36072632,60.97510061)
\lineto(333.19572632,60.97510061)
\curveto(333.14573045,60.95509995)(333.0957305,60.95009996)(333.04572632,60.96010061)
\curveto(332.98573061,60.97009994)(332.93073066,60.97009994)(332.88072632,60.96010061)
\curveto(332.84073075,60.95009996)(332.7957308,60.95009996)(332.74572632,60.96010061)
\curveto(332.6957309,60.97009994)(332.64573095,60.96509994)(332.59572632,60.94510061)
\curveto(332.52573107,60.92509998)(332.45073114,60.92009999)(332.37072632,60.93010061)
\curveto(332.28073131,60.94009997)(332.1957314,60.94509996)(332.11572632,60.94510061)
\curveto(332.02573157,60.94509996)(331.92573167,60.94009997)(331.81572632,60.93010061)
\curveto(331.6957319,60.92009999)(331.595732,60.92509998)(331.51572632,60.94510061)
\lineto(331.23072632,60.94510061)
\lineto(330.60072632,60.99010061)
\curveto(330.50073309,61.00009991)(330.40573319,61.0100999)(330.31572632,61.02010061)
\lineto(330.01572632,61.05010061)
\curveto(329.96573363,61.07009984)(329.91573368,61.07509983)(329.86572632,61.06510061)
\curveto(329.80573379,61.06509984)(329.75073384,61.07509983)(329.70072632,61.09510061)
\curveto(329.53073406,61.14509976)(329.36573423,61.18509972)(329.20572632,61.21510061)
\curveto(329.03573456,61.24509966)(328.87573472,61.29509961)(328.72572632,61.36510061)
\curveto(328.26573533,61.55509935)(327.8907357,61.77509913)(327.60072632,62.02510061)
\curveto(327.31073628,62.28509862)(327.06573653,62.64509826)(326.86572632,63.10510061)
\curveto(326.81573678,63.23509767)(326.78073681,63.36509754)(326.76072632,63.49510061)
\curveto(326.74073685,63.63509727)(326.71573688,63.77509713)(326.68572632,63.91510061)
\curveto(326.67573692,63.98509692)(326.67073692,64.05009686)(326.67072632,64.11010061)
\curveto(326.67073692,64.17009674)(326.66573693,64.23509667)(326.65572632,64.30510061)
\curveto(326.63573696,65.13509577)(326.78573681,65.8050951)(327.10572632,66.31510061)
\curveto(327.41573618,66.82509408)(327.85573574,67.2050937)(328.42572632,67.45510061)
\curveto(328.54573505,67.5050934)(328.67073492,67.55009336)(328.80072632,67.59010061)
\curveto(328.93073466,67.63009328)(329.06573453,67.67509323)(329.20572632,67.72510061)
\curveto(329.28573431,67.74509316)(329.37073422,67.76009315)(329.46072632,67.77010061)
\lineto(329.70072632,67.83010061)
\curveto(329.81073378,67.86009305)(329.92073367,67.87509303)(330.03072632,67.87510061)
\curveto(330.14073345,67.88509302)(330.25073334,67.90009301)(330.36072632,67.92010061)
\curveto(330.41073318,67.94009297)(330.45573314,67.94509296)(330.49572632,67.93510061)
\curveto(330.53573306,67.93509297)(330.57573302,67.94009297)(330.61572632,67.95010061)
\curveto(330.66573293,67.96009295)(330.72073287,67.96009295)(330.78072632,67.95010061)
\curveto(330.83073276,67.95009296)(330.88073271,67.95509295)(330.93072632,67.96510061)
\lineto(331.06572632,67.96510061)
\curveto(331.12573247,67.98509292)(331.1957324,67.98509292)(331.27572632,67.96510061)
\curveto(331.34573225,67.95509295)(331.41073218,67.96009295)(331.47072632,67.98010061)
\curveto(331.50073209,67.99009292)(331.54073205,67.99509291)(331.59072632,67.99510061)
\lineto(331.71072632,67.99510061)
\lineto(332.17572632,67.99510061)
\moveto(334.50072632,66.45010061)
\curveto(334.18072941,66.55009436)(333.81572978,66.6100943)(333.40572632,66.63010061)
\curveto(332.9957306,66.65009426)(332.58573101,66.66009425)(332.17572632,66.66010061)
\curveto(331.74573185,66.66009425)(331.32573227,66.65009426)(330.91572632,66.63010061)
\curveto(330.50573309,66.6100943)(330.12073347,66.56509434)(329.76072632,66.49510061)
\curveto(329.40073419,66.42509448)(329.08073451,66.31509459)(328.80072632,66.16510061)
\curveto(328.51073508,66.02509488)(328.27573532,65.83009508)(328.09572632,65.58010061)
\curveto(327.98573561,65.42009549)(327.90573569,65.24009567)(327.85572632,65.04010061)
\curveto(327.7957358,64.84009607)(327.76573583,64.59509631)(327.76572632,64.30510061)
\curveto(327.78573581,64.28509662)(327.7957358,64.25009666)(327.79572632,64.20010061)
\curveto(327.78573581,64.15009676)(327.78573581,64.1100968)(327.79572632,64.08010061)
\curveto(327.81573578,64.00009691)(327.83573576,63.92509698)(327.85572632,63.85510061)
\curveto(327.86573573,63.79509711)(327.88573571,63.73009718)(327.91572632,63.66010061)
\curveto(328.03573556,63.39009752)(328.20573539,63.17009774)(328.42572632,63.00010061)
\curveto(328.63573496,62.84009807)(328.88073471,62.7050982)(329.16072632,62.59510061)
\curveto(329.27073432,62.54509836)(329.3907342,62.5050984)(329.52072632,62.47510061)
\curveto(329.64073395,62.45509845)(329.76573383,62.43009848)(329.89572632,62.40010061)
\curveto(329.94573365,62.38009853)(330.00073359,62.37009854)(330.06072632,62.37010061)
\curveto(330.11073348,62.37009854)(330.16073343,62.36509854)(330.21072632,62.35510061)
\curveto(330.30073329,62.34509856)(330.3957332,62.33509857)(330.49572632,62.32510061)
\curveto(330.58573301,62.31509859)(330.68073291,62.3050986)(330.78072632,62.29510061)
\curveto(330.86073273,62.29509861)(330.94573265,62.29009862)(331.03572632,62.28010061)
\lineto(331.27572632,62.28010061)
\lineto(331.45572632,62.28010061)
\curveto(331.48573211,62.27009864)(331.52073207,62.26509864)(331.56072632,62.26510061)
\lineto(331.69572632,62.26510061)
\lineto(332.14572632,62.26510061)
\curveto(332.22573137,62.26509864)(332.31073128,62.26009865)(332.40072632,62.25010061)
\curveto(332.48073111,62.25009866)(332.55573104,62.26009865)(332.62572632,62.28010061)
\lineto(332.89572632,62.28010061)
\curveto(332.91573068,62.28009863)(332.94573065,62.27509863)(332.98572632,62.26510061)
\curveto(333.01573058,62.26509864)(333.04073055,62.27009864)(333.06072632,62.28010061)
\curveto(333.16073043,62.29009862)(333.26073033,62.29509861)(333.36072632,62.29510061)
\curveto(333.45073014,62.3050986)(333.55073004,62.31509859)(333.66072632,62.32510061)
\curveto(333.78072981,62.35509855)(333.90572969,62.37009854)(334.03572632,62.37010061)
\curveto(334.15572944,62.38009853)(334.27072932,62.4050985)(334.38072632,62.44510061)
\curveto(334.68072891,62.52509838)(334.94572865,62.6100983)(335.17572632,62.70010061)
\curveto(335.40572819,62.80009811)(335.62072797,62.94509796)(335.82072632,63.13510061)
\curveto(336.02072757,63.34509756)(336.17072742,63.6100973)(336.27072632,63.93010061)
\curveto(336.2907273,63.97009694)(336.30072729,64.0050969)(336.30072632,64.03510061)
\curveto(336.2907273,64.07509683)(336.2957273,64.12009679)(336.31572632,64.17010061)
\curveto(336.32572727,64.2100967)(336.33572726,64.28009663)(336.34572632,64.38010061)
\curveto(336.35572724,64.49009642)(336.35072724,64.57509633)(336.33072632,64.63510061)
\curveto(336.31072728,64.7050962)(336.30072729,64.77509613)(336.30072632,64.84510061)
\curveto(336.2907273,64.91509599)(336.27572732,64.98009593)(336.25572632,65.04010061)
\curveto(336.1957274,65.24009567)(336.11072748,65.42009549)(336.00072632,65.58010061)
\curveto(335.98072761,65.6100953)(335.96072763,65.63509527)(335.94072632,65.65510061)
\lineto(335.88072632,65.71510061)
\curveto(335.86072773,65.75509515)(335.82072777,65.8050951)(335.76072632,65.86510061)
\curveto(335.62072797,65.96509494)(335.4907281,66.05009486)(335.37072632,66.12010061)
\curveto(335.25072834,66.19009472)(335.10572849,66.26009465)(334.93572632,66.33010061)
\curveto(334.86572873,66.36009455)(334.7957288,66.38009453)(334.72572632,66.39010061)
\curveto(334.65572894,66.4100945)(334.58072901,66.43009448)(334.50072632,66.45010061)
}
}
{
\newrgbcolor{curcolor}{0 0 0}
\pscustom[linestyle=none,fillstyle=solid,fillcolor=curcolor]
{
\newpath
\moveto(326.65572632,72.45970998)
\curveto(326.62573697,74.08970454)(327.18073641,75.13970349)(328.32072632,75.60970998)
\curveto(328.55073504,75.70970292)(328.84073475,75.77470286)(329.19072632,75.80470998)
\curveto(329.53073406,75.84470279)(329.84073375,75.81970281)(330.12072632,75.72970998)
\curveto(330.38073321,75.63970299)(330.60573299,75.51970311)(330.79572632,75.36970998)
\curveto(330.83573276,75.34970328)(330.87073272,75.32470331)(330.90072632,75.29470998)
\curveto(330.92073267,75.26470337)(330.94573265,75.23970339)(330.97572632,75.21970998)
\lineto(331.09572632,75.12970998)
\curveto(331.12573247,75.09970353)(331.15073244,75.06470357)(331.17072632,75.02470998)
\curveto(331.22073237,74.97470366)(331.26573233,74.91970371)(331.30572632,74.85970998)
\curveto(331.34573225,74.80970382)(331.3957322,74.76470387)(331.45572632,74.72470998)
\curveto(331.4957321,74.68470395)(331.54573205,74.66970396)(331.60572632,74.67970998)
\curveto(331.65573194,74.68970394)(331.70073189,74.71970391)(331.74072632,74.76970998)
\curveto(331.78073181,74.81970381)(331.82073177,74.87470376)(331.86072632,74.93470998)
\curveto(331.8907317,75.00470363)(331.92073167,75.06970356)(331.95072632,75.12970998)
\curveto(331.98073161,75.18970344)(332.01073158,75.23970339)(332.04072632,75.27970998)
\curveto(332.26073133,75.59970303)(332.57073102,75.85470278)(332.97072632,76.04470998)
\curveto(333.06073053,76.08470255)(333.15573044,76.11470252)(333.25572632,76.13470998)
\curveto(333.34573025,76.16470247)(333.43573016,76.18970244)(333.52572632,76.20970998)
\curveto(333.57573002,76.21970241)(333.62572997,76.22470241)(333.67572632,76.22470998)
\curveto(333.71572988,76.2347024)(333.76072983,76.24470239)(333.81072632,76.25470998)
\curveto(333.86072973,76.26470237)(333.91072968,76.26470237)(333.96072632,76.25470998)
\curveto(334.01072958,76.24470239)(334.06072953,76.24970238)(334.11072632,76.26970998)
\curveto(334.16072943,76.27970235)(334.22072937,76.28470235)(334.29072632,76.28470998)
\curveto(334.36072923,76.28470235)(334.42072917,76.27470236)(334.47072632,76.25470998)
\lineto(334.69572632,76.25470998)
\lineto(334.93572632,76.19470998)
\curveto(335.00572859,76.18470245)(335.07572852,76.16970246)(335.14572632,76.14970998)
\curveto(335.23572836,76.11970251)(335.32072827,76.08970254)(335.40072632,76.05970998)
\curveto(335.48072811,76.03970259)(335.56072803,76.00970262)(335.64072632,75.96970998)
\curveto(335.70072789,75.94970268)(335.76072783,75.91970271)(335.82072632,75.87970998)
\curveto(335.87072772,75.84970278)(335.92072767,75.81470282)(335.97072632,75.77470998)
\curveto(336.28072731,75.57470306)(336.54072705,75.32470331)(336.75072632,75.02470998)
\curveto(336.95072664,74.72470391)(337.11572648,74.37970425)(337.24572632,73.98970998)
\curveto(337.28572631,73.86970476)(337.31072628,73.73970489)(337.32072632,73.59970998)
\curveto(337.34072625,73.46970516)(337.36572623,73.3347053)(337.39572632,73.19470998)
\curveto(337.40572619,73.12470551)(337.41072618,73.05470558)(337.41072632,72.98470998)
\curveto(337.41072618,72.92470571)(337.41572618,72.85970577)(337.42572632,72.78970998)
\curveto(337.43572616,72.74970588)(337.44072615,72.68970594)(337.44072632,72.60970998)
\curveto(337.44072615,72.53970609)(337.43572616,72.48970614)(337.42572632,72.45970998)
\curveto(337.41572618,72.40970622)(337.41072618,72.36470627)(337.41072632,72.32470998)
\lineto(337.41072632,72.20470998)
\curveto(337.3907262,72.10470653)(337.37572622,72.00470663)(337.36572632,71.90470998)
\curveto(337.35572624,71.80470683)(337.34072625,71.70970692)(337.32072632,71.61970998)
\curveto(337.2907263,71.50970712)(337.26572633,71.39970723)(337.24572632,71.28970998)
\curveto(337.21572638,71.18970744)(337.17572642,71.08470755)(337.12572632,70.97470998)
\curveto(336.96572663,70.60470803)(336.76572683,70.28970834)(336.52572632,70.02970998)
\curveto(336.27572732,69.76970886)(335.96572763,69.55970907)(335.59572632,69.39970998)
\curveto(335.50572809,69.35970927)(335.41072818,69.32470931)(335.31072632,69.29470998)
\curveto(335.21072838,69.26470937)(335.10572849,69.2347094)(334.99572632,69.20470998)
\curveto(334.94572865,69.18470945)(334.8957287,69.17470946)(334.84572632,69.17470998)
\curveto(334.78572881,69.17470946)(334.72572887,69.16470947)(334.66572632,69.14470998)
\curveto(334.60572899,69.12470951)(334.52572907,69.11470952)(334.42572632,69.11470998)
\curveto(334.32572927,69.11470952)(334.25072934,69.1297095)(334.20072632,69.15970998)
\curveto(334.17072942,69.16970946)(334.14572945,69.18470945)(334.12572632,69.20470998)
\lineto(334.06572632,69.26470998)
\curveto(334.04572955,69.30470933)(334.03072956,69.36470927)(334.02072632,69.44470998)
\curveto(334.01072958,69.5347091)(334.00572959,69.62470901)(334.00572632,69.71470998)
\curveto(334.00572959,69.80470883)(334.01072958,69.88970874)(334.02072632,69.96970998)
\curveto(334.03072956,70.05970857)(334.04072955,70.12470851)(334.05072632,70.16470998)
\curveto(334.07072952,70.18470845)(334.08572951,70.20470843)(334.09572632,70.22470998)
\curveto(334.0957295,70.24470839)(334.10572949,70.26470837)(334.12572632,70.28470998)
\curveto(334.21572938,70.35470828)(334.33072926,70.39470824)(334.47072632,70.40470998)
\curveto(334.61072898,70.42470821)(334.73572886,70.45470818)(334.84572632,70.49470998)
\lineto(335.20572632,70.64470998)
\curveto(335.31572828,70.69470794)(335.42072817,70.75970787)(335.52072632,70.83970998)
\curveto(335.55072804,70.85970777)(335.57572802,70.87970775)(335.59572632,70.89970998)
\curveto(335.61572798,70.9297077)(335.64072795,70.95470768)(335.67072632,70.97470998)
\curveto(335.73072786,71.01470762)(335.77572782,71.04970758)(335.80572632,71.07970998)
\curveto(335.83572776,71.11970751)(335.86572773,71.15470748)(335.89572632,71.18470998)
\curveto(335.92572767,71.22470741)(335.95572764,71.26970736)(335.98572632,71.31970998)
\curveto(336.04572755,71.40970722)(336.0957275,71.50470713)(336.13572632,71.60470998)
\lineto(336.25572632,71.93470998)
\curveto(336.30572729,72.08470655)(336.33572726,72.28470635)(336.34572632,72.53470998)
\curveto(336.35572724,72.78470585)(336.33572726,72.99470564)(336.28572632,73.16470998)
\curveto(336.26572733,73.24470539)(336.25072734,73.31470532)(336.24072632,73.37470998)
\lineto(336.18072632,73.58470998)
\curveto(336.06072753,73.86470477)(335.91072768,74.10470453)(335.73072632,74.30470998)
\curveto(335.55072804,74.51470412)(335.32072827,74.67970395)(335.04072632,74.79970998)
\curveto(334.97072862,74.8297038)(334.90072869,74.84970378)(334.83072632,74.85970998)
\lineto(334.59072632,74.91970998)
\curveto(334.45072914,74.95970367)(334.2907293,74.96970366)(334.11072632,74.94970998)
\curveto(333.92072967,74.9297037)(333.77072982,74.89970373)(333.66072632,74.85970998)
\curveto(333.28073031,74.7297039)(332.9907306,74.54470409)(332.79072632,74.30470998)
\curveto(332.590731,74.07470456)(332.43073116,73.76470487)(332.31072632,73.37470998)
\curveto(332.28073131,73.26470537)(332.26073133,73.14470549)(332.25072632,73.01470998)
\curveto(332.24073135,72.89470574)(332.23573136,72.76970586)(332.23572632,72.63970998)
\curveto(332.23573136,72.47970615)(332.23073136,72.33970629)(332.22072632,72.21970998)
\curveto(332.21073138,72.09970653)(332.15073144,72.01470662)(332.04072632,71.96470998)
\curveto(332.01073158,71.94470669)(331.97573162,71.9347067)(331.93572632,71.93470998)
\lineto(331.80072632,71.93470998)
\curveto(331.70073189,71.92470671)(331.60573199,71.92470671)(331.51572632,71.93470998)
\curveto(331.42573217,71.95470668)(331.36073223,71.99470664)(331.32072632,72.05470998)
\curveto(331.2907323,72.09470654)(331.27073232,72.1347065)(331.26072632,72.17470998)
\curveto(331.25073234,72.22470641)(331.24073235,72.27970635)(331.23072632,72.33970998)
\curveto(331.22073237,72.35970627)(331.22073237,72.38470625)(331.23072632,72.41470998)
\curveto(331.23073236,72.44470619)(331.22573237,72.46970616)(331.21572632,72.48970998)
\lineto(331.21572632,72.62470998)
\curveto(331.1957324,72.7347059)(331.18573241,72.8347058)(331.18572632,72.92470998)
\curveto(331.17573242,73.02470561)(331.15573244,73.11970551)(331.12572632,73.20970998)
\curveto(331.01573258,73.5297051)(330.87073272,73.78470485)(330.69072632,73.97470998)
\curveto(330.51073308,74.16470447)(330.26073333,74.31470432)(329.94072632,74.42470998)
\curveto(329.84073375,74.45470418)(329.71573388,74.47470416)(329.56572632,74.48470998)
\curveto(329.40573419,74.50470413)(329.26073433,74.49970413)(329.13072632,74.46970998)
\curveto(329.06073453,74.44970418)(328.9957346,74.4297042)(328.93572632,74.40970998)
\curveto(328.86573473,74.39970423)(328.80073479,74.37970425)(328.74072632,74.34970998)
\curveto(328.50073509,74.24970438)(328.31073528,74.10470453)(328.17072632,73.91470998)
\curveto(328.03073556,73.72470491)(327.92073567,73.49970513)(327.84072632,73.23970998)
\curveto(327.82073577,73.17970545)(327.81073578,73.11970551)(327.81072632,73.05970998)
\curveto(327.81073578,72.99970563)(327.80073579,72.9347057)(327.78072632,72.86470998)
\curveto(327.76073583,72.78470585)(327.75073584,72.68970594)(327.75072632,72.57970998)
\curveto(327.75073584,72.46970616)(327.76073583,72.37470626)(327.78072632,72.29470998)
\curveto(327.80073579,72.24470639)(327.81073578,72.19470644)(327.81072632,72.14470998)
\curveto(327.81073578,72.10470653)(327.82073577,72.05970657)(327.84072632,72.00970998)
\curveto(327.8907357,71.8297068)(327.96573563,71.65970697)(328.06572632,71.49970998)
\curveto(328.15573544,71.34970728)(328.27073532,71.21970741)(328.41072632,71.10970998)
\curveto(328.53073506,71.01970761)(328.66073493,70.93970769)(328.80072632,70.86970998)
\curveto(328.94073465,70.79970783)(329.0957345,70.7347079)(329.26572632,70.67470998)
\curveto(329.37573422,70.64470799)(329.4957341,70.62470801)(329.62572632,70.61470998)
\curveto(329.74573385,70.60470803)(329.84573375,70.56970806)(329.92572632,70.50970998)
\curveto(329.96573363,70.48970814)(330.00573359,70.4297082)(330.04572632,70.32970998)
\curveto(330.05573354,70.28970834)(330.06573353,70.2297084)(330.07572632,70.14970998)
\lineto(330.07572632,69.89470998)
\curveto(330.06573353,69.80470883)(330.05573354,69.71970891)(330.04572632,69.63970998)
\curveto(330.03573356,69.56970906)(330.02073357,69.51970911)(330.00072632,69.48970998)
\curveto(329.97073362,69.44970918)(329.91573368,69.41470922)(329.83572632,69.38470998)
\curveto(329.75573384,69.35470928)(329.67073392,69.34970928)(329.58072632,69.36970998)
\curveto(329.53073406,69.37970925)(329.48073411,69.38470925)(329.43072632,69.38470998)
\lineto(329.25072632,69.41470998)
\curveto(329.15073444,69.44470919)(329.05073454,69.46970916)(328.95072632,69.48970998)
\curveto(328.85073474,69.51970911)(328.76073483,69.55470908)(328.68072632,69.59470998)
\curveto(328.57073502,69.64470899)(328.46573513,69.68970894)(328.36572632,69.72970998)
\curveto(328.25573534,69.76970886)(328.15073544,69.81970881)(328.05072632,69.87970998)
\curveto(327.51073608,70.20970842)(327.11573648,70.67970795)(326.86572632,71.28970998)
\curveto(326.81573678,71.40970722)(326.78073681,71.5347071)(326.76072632,71.66470998)
\curveto(326.74073685,71.80470683)(326.71573688,71.94470669)(326.68572632,72.08470998)
\curveto(326.67573692,72.14470649)(326.67073692,72.20470643)(326.67072632,72.26470998)
\curveto(326.67073692,72.3347063)(326.66573693,72.39970623)(326.65572632,72.45970998)
}
}
{
\newrgbcolor{curcolor}{0 0 0}
\pscustom[linestyle=none,fillstyle=solid,fillcolor=curcolor]
{
\newpath
\moveto(335.62572632,78.63431936)
\lineto(335.62572632,79.26431936)
\lineto(335.62572632,79.45931936)
\curveto(335.62572797,79.52931683)(335.63572796,79.58931677)(335.65572632,79.63931936)
\curveto(335.6957279,79.70931665)(335.73572786,79.7593166)(335.77572632,79.78931936)
\curveto(335.82572777,79.82931653)(335.8907277,79.84931651)(335.97072632,79.84931936)
\curveto(336.05072754,79.8593165)(336.13572746,79.86431649)(336.22572632,79.86431936)
\lineto(336.94572632,79.86431936)
\curveto(337.42572617,79.86431649)(337.83572576,79.80431655)(338.17572632,79.68431936)
\curveto(338.51572508,79.56431679)(338.7907248,79.36931699)(339.00072632,79.09931936)
\curveto(339.05072454,79.02931733)(339.0957245,78.9593174)(339.13572632,78.88931936)
\curveto(339.18572441,78.82931753)(339.23072436,78.7543176)(339.27072632,78.66431936)
\curveto(339.28072431,78.64431771)(339.2907243,78.61431774)(339.30072632,78.57431936)
\curveto(339.32072427,78.53431782)(339.32572427,78.48931787)(339.31572632,78.43931936)
\curveto(339.28572431,78.34931801)(339.21072438,78.29431806)(339.09072632,78.27431936)
\curveto(338.98072461,78.2543181)(338.88572471,78.26931809)(338.80572632,78.31931936)
\curveto(338.73572486,78.34931801)(338.67072492,78.39431796)(338.61072632,78.45431936)
\curveto(338.56072503,78.52431783)(338.51072508,78.58931777)(338.46072632,78.64931936)
\curveto(338.41072518,78.71931764)(338.33572526,78.77931758)(338.23572632,78.82931936)
\curveto(338.14572545,78.88931747)(338.05572554,78.93931742)(337.96572632,78.97931936)
\curveto(337.93572566,78.99931736)(337.87572572,79.02431733)(337.78572632,79.05431936)
\curveto(337.70572589,79.08431727)(337.63572596,79.08931727)(337.57572632,79.06931936)
\curveto(337.43572616,79.03931732)(337.34572625,78.97931738)(337.30572632,78.88931936)
\curveto(337.27572632,78.80931755)(337.26072633,78.71931764)(337.26072632,78.61931936)
\curveto(337.26072633,78.51931784)(337.23572636,78.43431792)(337.18572632,78.36431936)
\curveto(337.11572648,78.27431808)(336.97572662,78.22931813)(336.76572632,78.22931936)
\lineto(336.21072632,78.22931936)
\lineto(335.98572632,78.22931936)
\curveto(335.90572769,78.23931812)(335.84072775,78.2593181)(335.79072632,78.28931936)
\curveto(335.71072788,78.34931801)(335.66572793,78.41931794)(335.65572632,78.49931936)
\curveto(335.64572795,78.51931784)(335.64072795,78.53931782)(335.64072632,78.55931936)
\curveto(335.64072795,78.58931777)(335.63572796,78.61431774)(335.62572632,78.63431936)
}
}
{
\newrgbcolor{curcolor}{0 0 0}
\pscustom[linestyle=none,fillstyle=solid,fillcolor=curcolor]
{
}
}
{
\newrgbcolor{curcolor}{0 0 0}
\pscustom[linestyle=none,fillstyle=solid,fillcolor=curcolor]
{
\newpath
\moveto(326.65572632,89.26463186)
\curveto(326.64573695,89.95462722)(326.76573683,90.55462662)(327.01572632,91.06463186)
\curveto(327.26573633,91.58462559)(327.60073599,91.9796252)(328.02072632,92.24963186)
\curveto(328.10073549,92.29962488)(328.1907354,92.34462483)(328.29072632,92.38463186)
\curveto(328.38073521,92.42462475)(328.47573512,92.46962471)(328.57572632,92.51963186)
\curveto(328.67573492,92.55962462)(328.77573482,92.58962459)(328.87572632,92.60963186)
\curveto(328.97573462,92.62962455)(329.08073451,92.64962453)(329.19072632,92.66963186)
\curveto(329.24073435,92.68962449)(329.28573431,92.69462448)(329.32572632,92.68463186)
\curveto(329.36573423,92.6746245)(329.41073418,92.6796245)(329.46072632,92.69963186)
\curveto(329.51073408,92.70962447)(329.595734,92.71462446)(329.71572632,92.71463186)
\curveto(329.82573377,92.71462446)(329.91073368,92.70962447)(329.97072632,92.69963186)
\curveto(330.03073356,92.6796245)(330.0907335,92.66962451)(330.15072632,92.66963186)
\curveto(330.21073338,92.6796245)(330.27073332,92.6746245)(330.33072632,92.65463186)
\curveto(330.47073312,92.61462456)(330.60573299,92.5796246)(330.73572632,92.54963186)
\curveto(330.86573273,92.51962466)(330.9907326,92.4796247)(331.11072632,92.42963186)
\curveto(331.25073234,92.36962481)(331.37573222,92.29962488)(331.48572632,92.21963186)
\curveto(331.595732,92.14962503)(331.70573189,92.0746251)(331.81572632,91.99463186)
\lineto(331.87572632,91.93463186)
\curveto(331.8957317,91.92462525)(331.91573168,91.90962527)(331.93572632,91.88963186)
\curveto(332.0957315,91.76962541)(332.24073135,91.63462554)(332.37072632,91.48463186)
\curveto(332.50073109,91.33462584)(332.62573097,91.174626)(332.74572632,91.00463186)
\curveto(332.96573063,90.69462648)(333.17073042,90.39962678)(333.36072632,90.11963186)
\curveto(333.50073009,89.88962729)(333.63572996,89.65962752)(333.76572632,89.42963186)
\curveto(333.8957297,89.20962797)(334.03072956,88.98962819)(334.17072632,88.76963186)
\curveto(334.34072925,88.51962866)(334.52072907,88.2796289)(334.71072632,88.04963186)
\curveto(334.90072869,87.82962935)(335.12572847,87.63962954)(335.38572632,87.47963186)
\curveto(335.44572815,87.43962974)(335.50572809,87.40462977)(335.56572632,87.37463186)
\curveto(335.61572798,87.34462983)(335.68072791,87.31462986)(335.76072632,87.28463186)
\curveto(335.83072776,87.26462991)(335.8907277,87.25962992)(335.94072632,87.26963186)
\curveto(336.01072758,87.28962989)(336.06572753,87.32462985)(336.10572632,87.37463186)
\curveto(336.13572746,87.42462975)(336.15572744,87.48462969)(336.16572632,87.55463186)
\lineto(336.16572632,87.79463186)
\lineto(336.16572632,88.54463186)
\lineto(336.16572632,91.34963186)
\lineto(336.16572632,92.00963186)
\curveto(336.16572743,92.09962508)(336.17072742,92.18462499)(336.18072632,92.26463186)
\curveto(336.18072741,92.34462483)(336.20072739,92.40962477)(336.24072632,92.45963186)
\curveto(336.28072731,92.50962467)(336.35572724,92.54962463)(336.46572632,92.57963186)
\curveto(336.56572703,92.61962456)(336.66572693,92.62962455)(336.76572632,92.60963186)
\lineto(336.90072632,92.60963186)
\curveto(336.97072662,92.58962459)(337.03072656,92.56962461)(337.08072632,92.54963186)
\curveto(337.13072646,92.52962465)(337.17072642,92.49462468)(337.20072632,92.44463186)
\curveto(337.24072635,92.39462478)(337.26072633,92.32462485)(337.26072632,92.23463186)
\lineto(337.26072632,91.96463186)
\lineto(337.26072632,91.06463186)
\lineto(337.26072632,87.55463186)
\lineto(337.26072632,86.48963186)
\curveto(337.26072633,86.40963077)(337.26572633,86.31963086)(337.27572632,86.21963186)
\curveto(337.27572632,86.11963106)(337.26572633,86.03463114)(337.24572632,85.96463186)
\curveto(337.17572642,85.75463142)(336.9957266,85.68963149)(336.70572632,85.76963186)
\curveto(336.66572693,85.7796314)(336.63072696,85.7796314)(336.60072632,85.76963186)
\curveto(336.56072703,85.76963141)(336.51572708,85.7796314)(336.46572632,85.79963186)
\curveto(336.38572721,85.81963136)(336.30072729,85.83963134)(336.21072632,85.85963186)
\curveto(336.12072747,85.8796313)(336.03572756,85.90463127)(335.95572632,85.93463186)
\curveto(335.46572813,86.09463108)(335.05072854,86.29463088)(334.71072632,86.53463186)
\curveto(334.46072913,86.71463046)(334.23572936,86.91963026)(334.03572632,87.14963186)
\curveto(333.82572977,87.3796298)(333.63072996,87.61962956)(333.45072632,87.86963186)
\curveto(333.27073032,88.12962905)(333.10073049,88.39462878)(332.94072632,88.66463186)
\curveto(332.77073082,88.94462823)(332.595731,89.21462796)(332.41572632,89.47463186)
\curveto(332.33573126,89.58462759)(332.26073133,89.68962749)(332.19072632,89.78963186)
\curveto(332.12073147,89.89962728)(332.04573155,90.00962717)(331.96572632,90.11963186)
\curveto(331.93573166,90.15962702)(331.90573169,90.19462698)(331.87572632,90.22463186)
\curveto(331.83573176,90.26462691)(331.80573179,90.30462687)(331.78572632,90.34463186)
\curveto(331.67573192,90.48462669)(331.55073204,90.60962657)(331.41072632,90.71963186)
\curveto(331.38073221,90.73962644)(331.35573224,90.76462641)(331.33572632,90.79463186)
\curveto(331.30573229,90.82462635)(331.27573232,90.84962633)(331.24572632,90.86963186)
\curveto(331.14573245,90.94962623)(331.04573255,91.01462616)(330.94572632,91.06463186)
\curveto(330.84573275,91.12462605)(330.73573286,91.179626)(330.61572632,91.22963186)
\curveto(330.54573305,91.25962592)(330.47073312,91.2796259)(330.39072632,91.28963186)
\lineto(330.15072632,91.34963186)
\lineto(330.06072632,91.34963186)
\curveto(330.03073356,91.35962582)(330.00073359,91.36462581)(329.97072632,91.36463186)
\curveto(329.90073369,91.38462579)(329.80573379,91.38962579)(329.68572632,91.37963186)
\curveto(329.55573404,91.3796258)(329.45573414,91.36962581)(329.38572632,91.34963186)
\curveto(329.30573429,91.32962585)(329.23073436,91.30962587)(329.16072632,91.28963186)
\curveto(329.08073451,91.2796259)(329.00073459,91.25962592)(328.92072632,91.22963186)
\curveto(328.68073491,91.11962606)(328.48073511,90.96962621)(328.32072632,90.77963186)
\curveto(328.15073544,90.59962658)(328.01073558,90.3796268)(327.90072632,90.11963186)
\curveto(327.88073571,90.04962713)(327.86573573,89.9796272)(327.85572632,89.90963186)
\curveto(327.83573576,89.83962734)(327.81573578,89.76462741)(327.79572632,89.68463186)
\curveto(327.77573582,89.60462757)(327.76573583,89.49462768)(327.76572632,89.35463186)
\curveto(327.76573583,89.22462795)(327.77573582,89.11962806)(327.79572632,89.03963186)
\curveto(327.80573579,88.9796282)(327.81073578,88.92462825)(327.81072632,88.87463186)
\curveto(327.81073578,88.82462835)(327.82073577,88.7746284)(327.84072632,88.72463186)
\curveto(327.88073571,88.62462855)(327.92073567,88.52962865)(327.96072632,88.43963186)
\curveto(328.00073559,88.35962882)(328.04573555,88.2796289)(328.09572632,88.19963186)
\curveto(328.11573548,88.16962901)(328.14073545,88.13962904)(328.17072632,88.10963186)
\curveto(328.20073539,88.08962909)(328.22573537,88.06462911)(328.24572632,88.03463186)
\lineto(328.32072632,87.95963186)
\curveto(328.34073525,87.92962925)(328.36073523,87.90462927)(328.38072632,87.88463186)
\lineto(328.59072632,87.73463186)
\curveto(328.65073494,87.69462948)(328.71573488,87.64962953)(328.78572632,87.59963186)
\curveto(328.87573472,87.53962964)(328.98073461,87.48962969)(329.10072632,87.44963186)
\curveto(329.21073438,87.41962976)(329.32073427,87.38462979)(329.43072632,87.34463186)
\curveto(329.54073405,87.30462987)(329.68573391,87.2796299)(329.86572632,87.26963186)
\curveto(330.03573356,87.25962992)(330.16073343,87.22962995)(330.24072632,87.17963186)
\curveto(330.32073327,87.12963005)(330.36573323,87.05463012)(330.37572632,86.95463186)
\curveto(330.38573321,86.85463032)(330.3907332,86.74463043)(330.39072632,86.62463186)
\curveto(330.3907332,86.58463059)(330.3957332,86.54463063)(330.40572632,86.50463186)
\curveto(330.40573319,86.46463071)(330.40073319,86.42963075)(330.39072632,86.39963186)
\curveto(330.37073322,86.34963083)(330.36073323,86.29963088)(330.36072632,86.24963186)
\curveto(330.36073323,86.20963097)(330.35073324,86.16963101)(330.33072632,86.12963186)
\curveto(330.27073332,86.03963114)(330.13573346,85.99463118)(329.92572632,85.99463186)
\lineto(329.80572632,85.99463186)
\curveto(329.74573385,86.00463117)(329.68573391,86.00963117)(329.62572632,86.00963186)
\curveto(329.55573404,86.01963116)(329.4907341,86.02963115)(329.43072632,86.03963186)
\curveto(329.32073427,86.05963112)(329.22073437,86.0796311)(329.13072632,86.09963186)
\curveto(329.03073456,86.11963106)(328.93573466,86.14963103)(328.84572632,86.18963186)
\curveto(328.77573482,86.20963097)(328.71573488,86.22963095)(328.66572632,86.24963186)
\lineto(328.48572632,86.30963186)
\curveto(328.22573537,86.42963075)(327.98073561,86.58463059)(327.75072632,86.77463186)
\curveto(327.52073607,86.9746302)(327.33573626,87.18962999)(327.19572632,87.41963186)
\curveto(327.11573648,87.52962965)(327.05073654,87.64462953)(327.00072632,87.76463186)
\lineto(326.85072632,88.15463186)
\curveto(326.80073679,88.26462891)(326.77073682,88.3796288)(326.76072632,88.49963186)
\curveto(326.74073685,88.61962856)(326.71573688,88.74462843)(326.68572632,88.87463186)
\curveto(326.68573691,88.94462823)(326.68573691,89.00962817)(326.68572632,89.06963186)
\curveto(326.67573692,89.12962805)(326.66573693,89.19462798)(326.65572632,89.26463186)
}
}
{
\newrgbcolor{curcolor}{0 0 0}
\pscustom[linestyle=none,fillstyle=solid,fillcolor=curcolor]
{
\newpath
\moveto(332.17572632,101.36424123)
\lineto(332.43072632,101.36424123)
\curveto(332.51073108,101.37423353)(332.58573101,101.36923353)(332.65572632,101.34924123)
\lineto(332.89572632,101.34924123)
\lineto(333.06072632,101.34924123)
\curveto(333.16073043,101.32923357)(333.26573033,101.31923358)(333.37572632,101.31924123)
\curveto(333.47573012,101.31923358)(333.57573002,101.30923359)(333.67572632,101.28924123)
\lineto(333.82572632,101.28924123)
\curveto(333.96572963,101.25923364)(334.10572949,101.23923366)(334.24572632,101.22924123)
\curveto(334.37572922,101.21923368)(334.50572909,101.19423371)(334.63572632,101.15424123)
\curveto(334.71572888,101.13423377)(334.80072879,101.11423379)(334.89072632,101.09424123)
\lineto(335.13072632,101.03424123)
\lineto(335.43072632,100.91424123)
\curveto(335.52072807,100.88423402)(335.61072798,100.84923405)(335.70072632,100.80924123)
\curveto(335.92072767,100.70923419)(336.13572746,100.57423433)(336.34572632,100.40424123)
\curveto(336.55572704,100.24423466)(336.72572687,100.06923483)(336.85572632,99.87924123)
\curveto(336.8957267,99.82923507)(336.93572666,99.76923513)(336.97572632,99.69924123)
\curveto(337.00572659,99.63923526)(337.04072655,99.57923532)(337.08072632,99.51924123)
\curveto(337.13072646,99.43923546)(337.17072642,99.34423556)(337.20072632,99.23424123)
\curveto(337.23072636,99.12423578)(337.26072633,99.01923588)(337.29072632,98.91924123)
\curveto(337.33072626,98.80923609)(337.35572624,98.6992362)(337.36572632,98.58924123)
\curveto(337.37572622,98.47923642)(337.3907262,98.36423654)(337.41072632,98.24424123)
\curveto(337.42072617,98.2042367)(337.42072617,98.15923674)(337.41072632,98.10924123)
\curveto(337.41072618,98.06923683)(337.41572618,98.02923687)(337.42572632,97.98924123)
\curveto(337.43572616,97.94923695)(337.44072615,97.89423701)(337.44072632,97.82424123)
\curveto(337.44072615,97.75423715)(337.43572616,97.7042372)(337.42572632,97.67424123)
\curveto(337.40572619,97.62423728)(337.40072619,97.57923732)(337.41072632,97.53924123)
\curveto(337.42072617,97.4992374)(337.42072617,97.46423744)(337.41072632,97.43424123)
\lineto(337.41072632,97.34424123)
\curveto(337.3907262,97.28423762)(337.37572622,97.21923768)(337.36572632,97.14924123)
\curveto(337.36572623,97.08923781)(337.36072623,97.02423788)(337.35072632,96.95424123)
\curveto(337.30072629,96.78423812)(337.25072634,96.62423828)(337.20072632,96.47424123)
\curveto(337.15072644,96.32423858)(337.08572651,96.17923872)(337.00572632,96.03924123)
\curveto(336.96572663,95.98923891)(336.93572666,95.93423897)(336.91572632,95.87424123)
\curveto(336.88572671,95.82423908)(336.85072674,95.77423913)(336.81072632,95.72424123)
\curveto(336.63072696,95.48423942)(336.41072718,95.28423962)(336.15072632,95.12424123)
\curveto(335.8907277,94.96423994)(335.60572799,94.82424008)(335.29572632,94.70424123)
\curveto(335.15572844,94.64424026)(335.01572858,94.5992403)(334.87572632,94.56924123)
\curveto(334.72572887,94.53924036)(334.57072902,94.5042404)(334.41072632,94.46424123)
\curveto(334.30072929,94.44424046)(334.1907294,94.42924047)(334.08072632,94.41924123)
\curveto(333.97072962,94.40924049)(333.86072973,94.39424051)(333.75072632,94.37424123)
\curveto(333.71072988,94.36424054)(333.67072992,94.35924054)(333.63072632,94.35924123)
\curveto(333.59073,94.36924053)(333.55073004,94.36924053)(333.51072632,94.35924123)
\curveto(333.46073013,94.34924055)(333.41073018,94.34424056)(333.36072632,94.34424123)
\lineto(333.19572632,94.34424123)
\curveto(333.14573045,94.32424058)(333.0957305,94.31924058)(333.04572632,94.32924123)
\curveto(332.98573061,94.33924056)(332.93073066,94.33924056)(332.88072632,94.32924123)
\curveto(332.84073075,94.31924058)(332.7957308,94.31924058)(332.74572632,94.32924123)
\curveto(332.6957309,94.33924056)(332.64573095,94.33424057)(332.59572632,94.31424123)
\curveto(332.52573107,94.29424061)(332.45073114,94.28924061)(332.37072632,94.29924123)
\curveto(332.28073131,94.30924059)(332.1957314,94.31424059)(332.11572632,94.31424123)
\curveto(332.02573157,94.31424059)(331.92573167,94.30924059)(331.81572632,94.29924123)
\curveto(331.6957319,94.28924061)(331.595732,94.29424061)(331.51572632,94.31424123)
\lineto(331.23072632,94.31424123)
\lineto(330.60072632,94.35924123)
\curveto(330.50073309,94.36924053)(330.40573319,94.37924052)(330.31572632,94.38924123)
\lineto(330.01572632,94.41924123)
\curveto(329.96573363,94.43924046)(329.91573368,94.44424046)(329.86572632,94.43424123)
\curveto(329.80573379,94.43424047)(329.75073384,94.44424046)(329.70072632,94.46424123)
\curveto(329.53073406,94.51424039)(329.36573423,94.55424035)(329.20572632,94.58424123)
\curveto(329.03573456,94.61424029)(328.87573472,94.66424024)(328.72572632,94.73424123)
\curveto(328.26573533,94.92423998)(327.8907357,95.14423976)(327.60072632,95.39424123)
\curveto(327.31073628,95.65423925)(327.06573653,96.01423889)(326.86572632,96.47424123)
\curveto(326.81573678,96.6042383)(326.78073681,96.73423817)(326.76072632,96.86424123)
\curveto(326.74073685,97.0042379)(326.71573688,97.14423776)(326.68572632,97.28424123)
\curveto(326.67573692,97.35423755)(326.67073692,97.41923748)(326.67072632,97.47924123)
\curveto(326.67073692,97.53923736)(326.66573693,97.6042373)(326.65572632,97.67424123)
\curveto(326.63573696,98.5042364)(326.78573681,99.17423573)(327.10572632,99.68424123)
\curveto(327.41573618,100.19423471)(327.85573574,100.57423433)(328.42572632,100.82424123)
\curveto(328.54573505,100.87423403)(328.67073492,100.91923398)(328.80072632,100.95924123)
\curveto(328.93073466,100.9992339)(329.06573453,101.04423386)(329.20572632,101.09424123)
\curveto(329.28573431,101.11423379)(329.37073422,101.12923377)(329.46072632,101.13924123)
\lineto(329.70072632,101.19924123)
\curveto(329.81073378,101.22923367)(329.92073367,101.24423366)(330.03072632,101.24424123)
\curveto(330.14073345,101.25423365)(330.25073334,101.26923363)(330.36072632,101.28924123)
\curveto(330.41073318,101.30923359)(330.45573314,101.31423359)(330.49572632,101.30424123)
\curveto(330.53573306,101.3042336)(330.57573302,101.30923359)(330.61572632,101.31924123)
\curveto(330.66573293,101.32923357)(330.72073287,101.32923357)(330.78072632,101.31924123)
\curveto(330.83073276,101.31923358)(330.88073271,101.32423358)(330.93072632,101.33424123)
\lineto(331.06572632,101.33424123)
\curveto(331.12573247,101.35423355)(331.1957324,101.35423355)(331.27572632,101.33424123)
\curveto(331.34573225,101.32423358)(331.41073218,101.32923357)(331.47072632,101.34924123)
\curveto(331.50073209,101.35923354)(331.54073205,101.36423354)(331.59072632,101.36424123)
\lineto(331.71072632,101.36424123)
\lineto(332.17572632,101.36424123)
\moveto(334.50072632,99.81924123)
\curveto(334.18072941,99.91923498)(333.81572978,99.97923492)(333.40572632,99.99924123)
\curveto(332.9957306,100.01923488)(332.58573101,100.02923487)(332.17572632,100.02924123)
\curveto(331.74573185,100.02923487)(331.32573227,100.01923488)(330.91572632,99.99924123)
\curveto(330.50573309,99.97923492)(330.12073347,99.93423497)(329.76072632,99.86424123)
\curveto(329.40073419,99.79423511)(329.08073451,99.68423522)(328.80072632,99.53424123)
\curveto(328.51073508,99.39423551)(328.27573532,99.1992357)(328.09572632,98.94924123)
\curveto(327.98573561,98.78923611)(327.90573569,98.60923629)(327.85572632,98.40924123)
\curveto(327.7957358,98.20923669)(327.76573583,97.96423694)(327.76572632,97.67424123)
\curveto(327.78573581,97.65423725)(327.7957358,97.61923728)(327.79572632,97.56924123)
\curveto(327.78573581,97.51923738)(327.78573581,97.47923742)(327.79572632,97.44924123)
\curveto(327.81573578,97.36923753)(327.83573576,97.29423761)(327.85572632,97.22424123)
\curveto(327.86573573,97.16423774)(327.88573571,97.0992378)(327.91572632,97.02924123)
\curveto(328.03573556,96.75923814)(328.20573539,96.53923836)(328.42572632,96.36924123)
\curveto(328.63573496,96.20923869)(328.88073471,96.07423883)(329.16072632,95.96424123)
\curveto(329.27073432,95.91423899)(329.3907342,95.87423903)(329.52072632,95.84424123)
\curveto(329.64073395,95.82423908)(329.76573383,95.7992391)(329.89572632,95.76924123)
\curveto(329.94573365,95.74923915)(330.00073359,95.73923916)(330.06072632,95.73924123)
\curveto(330.11073348,95.73923916)(330.16073343,95.73423917)(330.21072632,95.72424123)
\curveto(330.30073329,95.71423919)(330.3957332,95.7042392)(330.49572632,95.69424123)
\curveto(330.58573301,95.68423922)(330.68073291,95.67423923)(330.78072632,95.66424123)
\curveto(330.86073273,95.66423924)(330.94573265,95.65923924)(331.03572632,95.64924123)
\lineto(331.27572632,95.64924123)
\lineto(331.45572632,95.64924123)
\curveto(331.48573211,95.63923926)(331.52073207,95.63423927)(331.56072632,95.63424123)
\lineto(331.69572632,95.63424123)
\lineto(332.14572632,95.63424123)
\curveto(332.22573137,95.63423927)(332.31073128,95.62923927)(332.40072632,95.61924123)
\curveto(332.48073111,95.61923928)(332.55573104,95.62923927)(332.62572632,95.64924123)
\lineto(332.89572632,95.64924123)
\curveto(332.91573068,95.64923925)(332.94573065,95.64423926)(332.98572632,95.63424123)
\curveto(333.01573058,95.63423927)(333.04073055,95.63923926)(333.06072632,95.64924123)
\curveto(333.16073043,95.65923924)(333.26073033,95.66423924)(333.36072632,95.66424123)
\curveto(333.45073014,95.67423923)(333.55073004,95.68423922)(333.66072632,95.69424123)
\curveto(333.78072981,95.72423918)(333.90572969,95.73923916)(334.03572632,95.73924123)
\curveto(334.15572944,95.74923915)(334.27072932,95.77423913)(334.38072632,95.81424123)
\curveto(334.68072891,95.89423901)(334.94572865,95.97923892)(335.17572632,96.06924123)
\curveto(335.40572819,96.16923873)(335.62072797,96.31423859)(335.82072632,96.50424123)
\curveto(336.02072757,96.71423819)(336.17072742,96.97923792)(336.27072632,97.29924123)
\curveto(336.2907273,97.33923756)(336.30072729,97.37423753)(336.30072632,97.40424123)
\curveto(336.2907273,97.44423746)(336.2957273,97.48923741)(336.31572632,97.53924123)
\curveto(336.32572727,97.57923732)(336.33572726,97.64923725)(336.34572632,97.74924123)
\curveto(336.35572724,97.85923704)(336.35072724,97.94423696)(336.33072632,98.00424123)
\curveto(336.31072728,98.07423683)(336.30072729,98.14423676)(336.30072632,98.21424123)
\curveto(336.2907273,98.28423662)(336.27572732,98.34923655)(336.25572632,98.40924123)
\curveto(336.1957274,98.60923629)(336.11072748,98.78923611)(336.00072632,98.94924123)
\curveto(335.98072761,98.97923592)(335.96072763,99.0042359)(335.94072632,99.02424123)
\lineto(335.88072632,99.08424123)
\curveto(335.86072773,99.12423578)(335.82072777,99.17423573)(335.76072632,99.23424123)
\curveto(335.62072797,99.33423557)(335.4907281,99.41923548)(335.37072632,99.48924123)
\curveto(335.25072834,99.55923534)(335.10572849,99.62923527)(334.93572632,99.69924123)
\curveto(334.86572873,99.72923517)(334.7957288,99.74923515)(334.72572632,99.75924123)
\curveto(334.65572894,99.77923512)(334.58072901,99.7992351)(334.50072632,99.81924123)
}
}
{
\newrgbcolor{curcolor}{0 0 0}
\pscustom[linestyle=none,fillstyle=solid,fillcolor=curcolor]
{
\newpath
\moveto(326.65572632,106.77385061)
\curveto(326.65573694,106.87384575)(326.66573693,106.96884566)(326.68572632,107.05885061)
\curveto(326.6957369,107.14884548)(326.72573687,107.21384541)(326.77572632,107.25385061)
\curveto(326.85573674,107.31384531)(326.96073663,107.34384528)(327.09072632,107.34385061)
\lineto(327.48072632,107.34385061)
\lineto(328.98072632,107.34385061)
\lineto(335.37072632,107.34385061)
\lineto(336.54072632,107.34385061)
\lineto(336.85572632,107.34385061)
\curveto(336.95572664,107.35384527)(337.03572656,107.33884529)(337.09572632,107.29885061)
\curveto(337.17572642,107.24884538)(337.22572637,107.17384545)(337.24572632,107.07385061)
\curveto(337.25572634,106.98384564)(337.26072633,106.87384575)(337.26072632,106.74385061)
\lineto(337.26072632,106.51885061)
\curveto(337.24072635,106.43884619)(337.22572637,106.36884626)(337.21572632,106.30885061)
\curveto(337.1957264,106.24884638)(337.15572644,106.19884643)(337.09572632,106.15885061)
\curveto(337.03572656,106.11884651)(336.96072663,106.09884653)(336.87072632,106.09885061)
\lineto(336.57072632,106.09885061)
\lineto(335.47572632,106.09885061)
\lineto(330.13572632,106.09885061)
\curveto(330.04573355,106.07884655)(329.97073362,106.06384656)(329.91072632,106.05385061)
\curveto(329.84073375,106.05384657)(329.78073381,106.0238466)(329.73072632,105.96385061)
\curveto(329.68073391,105.89384673)(329.65573394,105.80384682)(329.65572632,105.69385061)
\curveto(329.64573395,105.59384703)(329.64073395,105.48384714)(329.64072632,105.36385061)
\lineto(329.64072632,104.22385061)
\lineto(329.64072632,103.72885061)
\curveto(329.63073396,103.56884906)(329.57073402,103.45884917)(329.46072632,103.39885061)
\curveto(329.43073416,103.37884925)(329.40073419,103.36884926)(329.37072632,103.36885061)
\curveto(329.33073426,103.36884926)(329.28573431,103.36384926)(329.23572632,103.35385061)
\curveto(329.11573448,103.33384929)(329.00573459,103.33884929)(328.90572632,103.36885061)
\curveto(328.80573479,103.40884922)(328.73573486,103.46384916)(328.69572632,103.53385061)
\curveto(328.64573495,103.61384901)(328.62073497,103.73384889)(328.62072632,103.89385061)
\curveto(328.62073497,104.05384857)(328.60573499,104.18884844)(328.57572632,104.29885061)
\curveto(328.56573503,104.34884828)(328.56073503,104.40384822)(328.56072632,104.46385061)
\curveto(328.55073504,104.5238481)(328.53573506,104.58384804)(328.51572632,104.64385061)
\curveto(328.46573513,104.79384783)(328.41573518,104.93884769)(328.36572632,105.07885061)
\curveto(328.30573529,105.21884741)(328.23573536,105.35384727)(328.15572632,105.48385061)
\curveto(328.06573553,105.623847)(327.96073563,105.74384688)(327.84072632,105.84385061)
\curveto(327.72073587,105.94384668)(327.590736,106.03884659)(327.45072632,106.12885061)
\curveto(327.35073624,106.18884644)(327.24073635,106.23384639)(327.12072632,106.26385061)
\curveto(327.00073659,106.30384632)(326.8957367,106.35384627)(326.80572632,106.41385061)
\curveto(326.74573685,106.46384616)(326.70573689,106.53384609)(326.68572632,106.62385061)
\curveto(326.67573692,106.64384598)(326.67073692,106.66884596)(326.67072632,106.69885061)
\curveto(326.67073692,106.7288459)(326.66573693,106.75384587)(326.65572632,106.77385061)
}
}
{
\newrgbcolor{curcolor}{0 0 0}
\pscustom[linestyle=none,fillstyle=solid,fillcolor=curcolor]
{
\newpath
\moveto(326.65572632,115.12345998)
\curveto(326.65573694,115.22345513)(326.66573693,115.31845503)(326.68572632,115.40845998)
\curveto(326.6957369,115.49845485)(326.72573687,115.56345479)(326.77572632,115.60345998)
\curveto(326.85573674,115.66345469)(326.96073663,115.69345466)(327.09072632,115.69345998)
\lineto(327.48072632,115.69345998)
\lineto(328.98072632,115.69345998)
\lineto(335.37072632,115.69345998)
\lineto(336.54072632,115.69345998)
\lineto(336.85572632,115.69345998)
\curveto(336.95572664,115.70345465)(337.03572656,115.68845466)(337.09572632,115.64845998)
\curveto(337.17572642,115.59845475)(337.22572637,115.52345483)(337.24572632,115.42345998)
\curveto(337.25572634,115.33345502)(337.26072633,115.22345513)(337.26072632,115.09345998)
\lineto(337.26072632,114.86845998)
\curveto(337.24072635,114.78845556)(337.22572637,114.71845563)(337.21572632,114.65845998)
\curveto(337.1957264,114.59845575)(337.15572644,114.5484558)(337.09572632,114.50845998)
\curveto(337.03572656,114.46845588)(336.96072663,114.4484559)(336.87072632,114.44845998)
\lineto(336.57072632,114.44845998)
\lineto(335.47572632,114.44845998)
\lineto(330.13572632,114.44845998)
\curveto(330.04573355,114.42845592)(329.97073362,114.41345594)(329.91072632,114.40345998)
\curveto(329.84073375,114.40345595)(329.78073381,114.37345598)(329.73072632,114.31345998)
\curveto(329.68073391,114.24345611)(329.65573394,114.1534562)(329.65572632,114.04345998)
\curveto(329.64573395,113.94345641)(329.64073395,113.83345652)(329.64072632,113.71345998)
\lineto(329.64072632,112.57345998)
\lineto(329.64072632,112.07845998)
\curveto(329.63073396,111.91845843)(329.57073402,111.80845854)(329.46072632,111.74845998)
\curveto(329.43073416,111.72845862)(329.40073419,111.71845863)(329.37072632,111.71845998)
\curveto(329.33073426,111.71845863)(329.28573431,111.71345864)(329.23572632,111.70345998)
\curveto(329.11573448,111.68345867)(329.00573459,111.68845866)(328.90572632,111.71845998)
\curveto(328.80573479,111.75845859)(328.73573486,111.81345854)(328.69572632,111.88345998)
\curveto(328.64573495,111.96345839)(328.62073497,112.08345827)(328.62072632,112.24345998)
\curveto(328.62073497,112.40345795)(328.60573499,112.53845781)(328.57572632,112.64845998)
\curveto(328.56573503,112.69845765)(328.56073503,112.7534576)(328.56072632,112.81345998)
\curveto(328.55073504,112.87345748)(328.53573506,112.93345742)(328.51572632,112.99345998)
\curveto(328.46573513,113.14345721)(328.41573518,113.28845706)(328.36572632,113.42845998)
\curveto(328.30573529,113.56845678)(328.23573536,113.70345665)(328.15572632,113.83345998)
\curveto(328.06573553,113.97345638)(327.96073563,114.09345626)(327.84072632,114.19345998)
\curveto(327.72073587,114.29345606)(327.590736,114.38845596)(327.45072632,114.47845998)
\curveto(327.35073624,114.53845581)(327.24073635,114.58345577)(327.12072632,114.61345998)
\curveto(327.00073659,114.6534557)(326.8957367,114.70345565)(326.80572632,114.76345998)
\curveto(326.74573685,114.81345554)(326.70573689,114.88345547)(326.68572632,114.97345998)
\curveto(326.67573692,114.99345536)(326.67073692,115.01845533)(326.67072632,115.04845998)
\curveto(326.67073692,115.07845527)(326.66573693,115.10345525)(326.65572632,115.12345998)
}
}
{
\newrgbcolor{curcolor}{0 0 0}
\pscustom[linestyle=none,fillstyle=solid,fillcolor=curcolor]
{
\newpath
\moveto(357.39207275,42.02236623)
\curveto(357.4420735,42.04235669)(357.50207344,42.06735666)(357.57207275,42.09736623)
\curveto(357.6420733,42.1273566)(357.71707322,42.14735658)(357.79707275,42.15736623)
\curveto(357.86707307,42.17735655)(357.937073,42.17735655)(358.00707275,42.15736623)
\curveto(358.06707287,42.14735658)(358.11207283,42.10735662)(358.14207275,42.03736623)
\curveto(358.16207278,41.98735674)(358.17207277,41.9273568)(358.17207275,41.85736623)
\lineto(358.17207275,41.64736623)
\lineto(358.17207275,41.19736623)
\curveto(358.17207277,41.04735768)(358.14707279,40.9273578)(358.09707275,40.83736623)
\curveto(358.0370729,40.73735799)(357.93207301,40.66235807)(357.78207275,40.61236623)
\curveto(357.63207331,40.57235816)(357.49707344,40.5273582)(357.37707275,40.47736623)
\curveto(357.11707382,40.36735836)(356.84707409,40.26735846)(356.56707275,40.17736623)
\curveto(356.28707465,40.08735864)(356.01207493,39.98735874)(355.74207275,39.87736623)
\curveto(355.65207529,39.84735888)(355.56707537,39.81735891)(355.48707275,39.78736623)
\curveto(355.40707553,39.76735896)(355.33207561,39.73735899)(355.26207275,39.69736623)
\curveto(355.19207575,39.66735906)(355.13207581,39.62235911)(355.08207275,39.56236623)
\curveto(355.03207591,39.50235923)(354.99207595,39.42235931)(354.96207275,39.32236623)
\curveto(354.942076,39.27235946)(354.937076,39.21235952)(354.94707275,39.14236623)
\lineto(354.94707275,38.94736623)
\lineto(354.94707275,36.11236623)
\lineto(354.94707275,35.81236623)
\curveto(354.937076,35.70236303)(354.937076,35.59736313)(354.94707275,35.49736623)
\curveto(354.95707598,35.39736333)(354.97207597,35.30236343)(354.99207275,35.21236623)
\curveto(355.01207593,35.1323636)(355.05207589,35.07236366)(355.11207275,35.03236623)
\curveto(355.21207573,34.95236378)(355.32707561,34.89236384)(355.45707275,34.85236623)
\curveto(355.57707536,34.82236391)(355.70207524,34.78236395)(355.83207275,34.73236623)
\curveto(356.06207488,34.6323641)(356.30207464,34.53736419)(356.55207275,34.44736623)
\curveto(356.80207414,34.36736436)(357.0420739,34.27736445)(357.27207275,34.17736623)
\curveto(357.33207361,34.15736457)(357.40207354,34.1323646)(357.48207275,34.10236623)
\curveto(357.55207339,34.08236465)(357.62707331,34.05736467)(357.70707275,34.02736623)
\curveto(357.78707315,33.99736473)(357.86207308,33.96236477)(357.93207275,33.92236623)
\curveto(357.99207295,33.89236484)(358.0370729,33.85736487)(358.06707275,33.81736623)
\curveto(358.12707281,33.73736499)(358.16207278,33.6273651)(358.17207275,33.48736623)
\lineto(358.17207275,33.06736623)
\lineto(358.17207275,32.82736623)
\curveto(358.16207278,32.75736597)(358.1370728,32.69736603)(358.09707275,32.64736623)
\curveto(358.06707287,32.59736613)(358.02207292,32.56736616)(357.96207275,32.55736623)
\curveto(357.90207304,32.55736617)(357.8420731,32.56236617)(357.78207275,32.57236623)
\curveto(357.71207323,32.59236614)(357.64707329,32.61236612)(357.58707275,32.63236623)
\curveto(357.51707342,32.66236607)(357.46707347,32.68736604)(357.43707275,32.70736623)
\curveto(357.11707382,32.84736588)(356.80207414,32.97236576)(356.49207275,33.08236623)
\curveto(356.17207477,33.19236554)(355.85207509,33.31236542)(355.53207275,33.44236623)
\curveto(355.31207563,33.5323652)(355.09707584,33.61736511)(354.88707275,33.69736623)
\curveto(354.66707627,33.77736495)(354.44707649,33.86236487)(354.22707275,33.95236623)
\curveto(353.50707743,34.25236448)(352.78207816,34.53736419)(352.05207275,34.80736623)
\curveto(351.31207963,35.07736365)(350.57708036,35.36236337)(349.84707275,35.66236623)
\curveto(349.58708135,35.77236296)(349.32208162,35.87236286)(349.05207275,35.96236623)
\curveto(348.78208216,36.06236267)(348.51708242,36.16736256)(348.25707275,36.27736623)
\curveto(348.14708279,36.3273624)(348.02708291,36.37236236)(347.89707275,36.41236623)
\curveto(347.75708318,36.46236227)(347.65708328,36.5323622)(347.59707275,36.62236623)
\curveto(347.55708338,36.66236207)(347.52708341,36.727362)(347.50707275,36.81736623)
\curveto(347.49708344,36.83736189)(347.49708344,36.85736187)(347.50707275,36.87736623)
\curveto(347.50708343,36.90736182)(347.50208344,36.9323618)(347.49207275,36.95236623)
\curveto(347.49208345,37.1323616)(347.49208345,37.34236139)(347.49207275,37.58236623)
\curveto(347.48208346,37.82236091)(347.51708342,37.99736073)(347.59707275,38.10736623)
\curveto(347.65708328,38.18736054)(347.75708318,38.24736048)(347.89707275,38.28736623)
\curveto(348.02708291,38.33736039)(348.14708279,38.38736034)(348.25707275,38.43736623)
\curveto(348.48708245,38.53736019)(348.71708222,38.6273601)(348.94707275,38.70736623)
\curveto(349.17708176,38.78735994)(349.40708153,38.87735985)(349.63707275,38.97736623)
\curveto(349.8370811,39.05735967)(350.0420809,39.1323596)(350.25207275,39.20236623)
\curveto(350.46208048,39.28235945)(350.66708027,39.36735936)(350.86707275,39.45736623)
\curveto(351.59707934,39.75735897)(352.3370786,40.04235869)(353.08707275,40.31236623)
\curveto(353.82707711,40.59235814)(354.56207638,40.88735784)(355.29207275,41.19736623)
\curveto(355.38207556,41.23735749)(355.46707547,41.26735746)(355.54707275,41.28736623)
\curveto(355.62707531,41.31735741)(355.71207523,41.34735738)(355.80207275,41.37736623)
\curveto(356.06207488,41.48735724)(356.32707461,41.59235714)(356.59707275,41.69236623)
\curveto(356.86707407,41.80235693)(357.13207381,41.91235682)(357.39207275,42.02236623)
\moveto(353.74707275,38.81236623)
\curveto(353.71707722,38.90235983)(353.66707727,38.95735977)(353.59707275,38.97736623)
\curveto(353.52707741,39.00735972)(353.45207749,39.01235972)(353.37207275,38.99236623)
\curveto(353.28207766,38.98235975)(353.19707774,38.95735977)(353.11707275,38.91736623)
\curveto(353.02707791,38.88735984)(352.95207799,38.85735987)(352.89207275,38.82736623)
\curveto(352.85207809,38.80735992)(352.81707812,38.79735993)(352.78707275,38.79736623)
\curveto(352.75707818,38.79735993)(352.72207822,38.78735994)(352.68207275,38.76736623)
\lineto(352.44207275,38.67736623)
\curveto(352.35207859,38.65736007)(352.26207868,38.6273601)(352.17207275,38.58736623)
\curveto(351.81207913,38.43736029)(351.44707949,38.30236043)(351.07707275,38.18236623)
\curveto(350.69708024,38.07236066)(350.32708061,37.94236079)(349.96707275,37.79236623)
\curveto(349.85708108,37.74236099)(349.74708119,37.69736103)(349.63707275,37.65736623)
\curveto(349.52708141,37.6273611)(349.42208152,37.58736114)(349.32207275,37.53736623)
\curveto(349.27208167,37.51736121)(349.22708171,37.49236124)(349.18707275,37.46236623)
\curveto(349.1370818,37.44236129)(349.11208183,37.39236134)(349.11207275,37.31236623)
\curveto(349.13208181,37.29236144)(349.14708179,37.27236146)(349.15707275,37.25236623)
\curveto(349.16708177,37.2323615)(349.18208176,37.21236152)(349.20207275,37.19236623)
\curveto(349.25208169,37.15236158)(349.30708163,37.12236161)(349.36707275,37.10236623)
\curveto(349.41708152,37.08236165)(349.47208147,37.06236167)(349.53207275,37.04236623)
\curveto(349.6420813,36.99236174)(349.75208119,36.95236178)(349.86207275,36.92236623)
\curveto(349.97208097,36.89236184)(350.08208086,36.85236188)(350.19207275,36.80236623)
\curveto(350.58208036,36.6323621)(350.97707996,36.48236225)(351.37707275,36.35236623)
\curveto(351.77707916,36.2323625)(352.16707877,36.09236264)(352.54707275,35.93236623)
\lineto(352.69707275,35.87236623)
\curveto(352.74707819,35.86236287)(352.79707814,35.84736288)(352.84707275,35.82736623)
\lineto(353.08707275,35.73736623)
\curveto(353.16707777,35.70736302)(353.24707769,35.68236305)(353.32707275,35.66236623)
\curveto(353.37707756,35.64236309)(353.43207751,35.6323631)(353.49207275,35.63236623)
\curveto(353.55207739,35.64236309)(353.60207734,35.65736307)(353.64207275,35.67736623)
\curveto(353.72207722,35.727363)(353.76707717,35.8323629)(353.77707275,35.99236623)
\lineto(353.77707275,36.44236623)
\lineto(353.77707275,38.04736623)
\curveto(353.77707716,38.15736057)(353.78207716,38.29236044)(353.79207275,38.45236623)
\curveto(353.79207715,38.61236012)(353.77707716,38.73236)(353.74707275,38.81236623)
}
}
{
\newrgbcolor{curcolor}{0 0 0}
\pscustom[linestyle=none,fillstyle=solid,fillcolor=curcolor]
{
\newpath
\moveto(354.13707275,50.56392873)
\curveto(354.18707675,50.57392038)(354.25707668,50.57892038)(354.34707275,50.57892873)
\curveto(354.42707651,50.57892038)(354.49207645,50.57392038)(354.54207275,50.56392873)
\curveto(354.58207636,50.56392039)(354.62207632,50.5589204)(354.66207275,50.54892873)
\lineto(354.78207275,50.54892873)
\curveto(354.86207608,50.52892043)(354.942076,50.51892044)(355.02207275,50.51892873)
\curveto(355.10207584,50.51892044)(355.18207576,50.50892045)(355.26207275,50.48892873)
\curveto(355.30207564,50.47892048)(355.3420756,50.47392048)(355.38207275,50.47392873)
\curveto(355.41207553,50.47392048)(355.44707549,50.46892049)(355.48707275,50.45892873)
\curveto(355.59707534,50.42892053)(355.70207524,50.39892056)(355.80207275,50.36892873)
\curveto(355.90207504,50.34892061)(356.00207494,50.31892064)(356.10207275,50.27892873)
\curveto(356.45207449,50.13892082)(356.76707417,49.96892099)(357.04707275,49.76892873)
\curveto(357.32707361,49.56892139)(357.56707337,49.31892164)(357.76707275,49.01892873)
\curveto(357.86707307,48.86892209)(357.95207299,48.72392223)(358.02207275,48.58392873)
\curveto(358.07207287,48.47392248)(358.11207283,48.36392259)(358.14207275,48.25392873)
\curveto(358.17207277,48.1539228)(358.20207274,48.04892291)(358.23207275,47.93892873)
\curveto(358.25207269,47.86892309)(358.26207268,47.80392315)(358.26207275,47.74392873)
\curveto(358.27207267,47.68392327)(358.28707265,47.62392333)(358.30707275,47.56392873)
\lineto(358.30707275,47.41392873)
\curveto(358.32707261,47.36392359)(358.3370726,47.28892367)(358.33707275,47.18892873)
\curveto(358.34707259,47.08892387)(358.3420726,47.00892395)(358.32207275,46.94892873)
\lineto(358.32207275,46.79892873)
\curveto(358.31207263,46.7589242)(358.30707263,46.71392424)(358.30707275,46.66392873)
\curveto(358.30707263,46.62392433)(358.30207264,46.57892438)(358.29207275,46.52892873)
\curveto(358.25207269,46.37892458)(358.21707272,46.22892473)(358.18707275,46.07892873)
\curveto(358.15707278,45.93892502)(358.11207283,45.79892516)(358.05207275,45.65892873)
\curveto(357.97207297,45.4589255)(357.87207307,45.27892568)(357.75207275,45.11892873)
\lineto(357.60207275,44.93892873)
\curveto(357.5420734,44.87892608)(357.50207344,44.80892615)(357.48207275,44.72892873)
\curveto(357.47207347,44.66892629)(357.48707345,44.61892634)(357.52707275,44.57892873)
\curveto(357.55707338,44.54892641)(357.60207334,44.52392643)(357.66207275,44.50392873)
\curveto(357.72207322,44.49392646)(357.78707315,44.48392647)(357.85707275,44.47392873)
\curveto(357.91707302,44.47392648)(357.96207298,44.46392649)(357.99207275,44.44392873)
\curveto(358.0420729,44.40392655)(358.08707285,44.3589266)(358.12707275,44.30892873)
\curveto(358.14707279,44.2589267)(358.16207278,44.18892677)(358.17207275,44.09892873)
\lineto(358.17207275,43.82892873)
\curveto(358.17207277,43.73892722)(358.16707277,43.6539273)(358.15707275,43.57392873)
\curveto(358.1370728,43.49392746)(358.11707282,43.43392752)(358.09707275,43.39392873)
\curveto(358.07707286,43.37392758)(358.05207289,43.3539276)(358.02207275,43.33392873)
\lineto(357.93207275,43.27392873)
\curveto(357.85207309,43.24392771)(357.73207321,43.22892773)(357.57207275,43.22892873)
\curveto(357.41207353,43.23892772)(357.27707366,43.24392771)(357.16707275,43.24392873)
\lineto(348.36207275,43.24392873)
\curveto(348.2420827,43.24392771)(348.11708282,43.23892772)(347.98707275,43.22892873)
\curveto(347.84708309,43.22892773)(347.7370832,43.2539277)(347.65707275,43.30392873)
\curveto(347.59708334,43.34392761)(347.54708339,43.40892755)(347.50707275,43.49892873)
\curveto(347.50708343,43.51892744)(347.50708343,43.54392741)(347.50707275,43.57392873)
\curveto(347.49708344,43.60392735)(347.49208345,43.62892733)(347.49207275,43.64892873)
\curveto(347.48208346,43.78892717)(347.48208346,43.93392702)(347.49207275,44.08392873)
\curveto(347.49208345,44.24392671)(347.53208341,44.3539266)(347.61207275,44.41392873)
\curveto(347.69208325,44.46392649)(347.80708313,44.48892647)(347.95707275,44.48892873)
\lineto(348.36207275,44.48892873)
\lineto(350.11707275,44.48892873)
\lineto(350.37207275,44.48892873)
\lineto(350.65707275,44.48892873)
\curveto(350.74708019,44.49892646)(350.83208011,44.50892645)(350.91207275,44.51892873)
\curveto(350.98207996,44.53892642)(351.03207991,44.56892639)(351.06207275,44.60892873)
\curveto(351.09207985,44.64892631)(351.09707984,44.69392626)(351.07707275,44.74392873)
\curveto(351.05707988,44.79392616)(351.0370799,44.83392612)(351.01707275,44.86392873)
\curveto(350.97707996,44.91392604)(350.93708,44.958926)(350.89707275,44.99892873)
\lineto(350.77707275,45.14892873)
\curveto(350.72708021,45.21892574)(350.68208026,45.28892567)(350.64207275,45.35892873)
\lineto(350.52207275,45.59892873)
\curveto(350.43208051,45.77892518)(350.36708057,45.99392496)(350.32707275,46.24392873)
\curveto(350.28708065,46.49392446)(350.26708067,46.74892421)(350.26707275,47.00892873)
\curveto(350.26708067,47.26892369)(350.29208065,47.52392343)(350.34207275,47.77392873)
\curveto(350.38208056,48.02392293)(350.4420805,48.24392271)(350.52207275,48.43392873)
\curveto(350.69208025,48.83392212)(350.92708001,49.17892178)(351.22707275,49.46892873)
\curveto(351.52707941,49.7589212)(351.87707906,49.98892097)(352.27707275,50.15892873)
\curveto(352.38707855,50.20892075)(352.49707844,50.24892071)(352.60707275,50.27892873)
\curveto(352.70707823,50.31892064)(352.81207813,50.3589206)(352.92207275,50.39892873)
\curveto(353.03207791,50.42892053)(353.14707779,50.44892051)(353.26707275,50.45892873)
\lineto(353.59707275,50.51892873)
\curveto(353.62707731,50.52892043)(353.66207728,50.53392042)(353.70207275,50.53392873)
\curveto(353.73207721,50.53392042)(353.76207718,50.53892042)(353.79207275,50.54892873)
\curveto(353.85207709,50.56892039)(353.91207703,50.56892039)(353.97207275,50.54892873)
\curveto(354.02207692,50.53892042)(354.07707686,50.54392041)(354.13707275,50.56392873)
\moveto(354.52707275,49.22892873)
\curveto(354.47707646,49.24892171)(354.41707652,49.2539217)(354.34707275,49.24392873)
\curveto(354.27707666,49.23392172)(354.21207673,49.22892173)(354.15207275,49.22892873)
\curveto(353.98207696,49.22892173)(353.82207712,49.21892174)(353.67207275,49.19892873)
\curveto(353.52207742,49.18892177)(353.38707755,49.1589218)(353.26707275,49.10892873)
\curveto(353.16707777,49.07892188)(353.07707786,49.0539219)(352.99707275,49.03392873)
\curveto(352.91707802,49.01392194)(352.8370781,48.98392197)(352.75707275,48.94392873)
\curveto(352.50707843,48.83392212)(352.27707866,48.68392227)(352.06707275,48.49392873)
\curveto(351.84707909,48.30392265)(351.68207926,48.08392287)(351.57207275,47.83392873)
\curveto(351.5420794,47.7539232)(351.51707942,47.67392328)(351.49707275,47.59392873)
\curveto(351.46707947,47.52392343)(351.4420795,47.44892351)(351.42207275,47.36892873)
\curveto(351.39207955,47.2589237)(351.37707956,47.14892381)(351.37707275,47.03892873)
\curveto(351.36707957,46.92892403)(351.36207958,46.80892415)(351.36207275,46.67892873)
\curveto(351.37207957,46.62892433)(351.38207956,46.58392437)(351.39207275,46.54392873)
\lineto(351.39207275,46.40892873)
\lineto(351.45207275,46.13892873)
\curveto(351.47207947,46.0589249)(351.50207944,45.97892498)(351.54207275,45.89892873)
\curveto(351.68207926,45.5589254)(351.89207905,45.28892567)(352.17207275,45.08892873)
\curveto(352.4420785,44.88892607)(352.76207818,44.72892623)(353.13207275,44.60892873)
\curveto(353.2420777,44.56892639)(353.35207759,44.54392641)(353.46207275,44.53392873)
\curveto(353.57207737,44.52392643)(353.68707725,44.50392645)(353.80707275,44.47392873)
\curveto(353.85707708,44.46392649)(353.90207704,44.46392649)(353.94207275,44.47392873)
\curveto(353.98207696,44.48392647)(354.02707691,44.47892648)(354.07707275,44.45892873)
\curveto(354.12707681,44.44892651)(354.20207674,44.44392651)(354.30207275,44.44392873)
\curveto(354.39207655,44.44392651)(354.46207648,44.44892651)(354.51207275,44.45892873)
\lineto(354.63207275,44.45892873)
\curveto(354.67207627,44.46892649)(354.71207623,44.47392648)(354.75207275,44.47392873)
\curveto(354.79207615,44.47392648)(354.82707611,44.47892648)(354.85707275,44.48892873)
\curveto(354.88707605,44.49892646)(354.92207602,44.50392645)(354.96207275,44.50392873)
\curveto(354.99207595,44.50392645)(355.02207592,44.50892645)(355.05207275,44.51892873)
\curveto(355.13207581,44.53892642)(355.21207573,44.5539264)(355.29207275,44.56392873)
\lineto(355.53207275,44.62392873)
\curveto(355.87207507,44.73392622)(356.16207478,44.88392607)(356.40207275,45.07392873)
\curveto(356.6420743,45.27392568)(356.8420741,45.51892544)(357.00207275,45.80892873)
\curveto(357.05207389,45.89892506)(357.09207385,45.99392496)(357.12207275,46.09392873)
\curveto(357.1420738,46.19392476)(357.16707377,46.29892466)(357.19707275,46.40892873)
\curveto(357.21707372,46.4589245)(357.22707371,46.50392445)(357.22707275,46.54392873)
\curveto(357.21707372,46.59392436)(357.21707372,46.64392431)(357.22707275,46.69392873)
\curveto(357.2370737,46.73392422)(357.2420737,46.77892418)(357.24207275,46.82892873)
\lineto(357.24207275,46.96392873)
\lineto(357.24207275,47.09892873)
\curveto(357.23207371,47.13892382)(357.22707371,47.17392378)(357.22707275,47.20392873)
\curveto(357.22707371,47.23392372)(357.22207372,47.26892369)(357.21207275,47.30892873)
\curveto(357.19207375,47.38892357)(357.17707376,47.46392349)(357.16707275,47.53392873)
\curveto(357.14707379,47.60392335)(357.12207382,47.67892328)(357.09207275,47.75892873)
\curveto(356.96207398,48.06892289)(356.79207415,48.31892264)(356.58207275,48.50892873)
\curveto(356.36207458,48.69892226)(356.09707484,48.8589221)(355.78707275,48.98892873)
\curveto(355.64707529,49.03892192)(355.50707543,49.07392188)(355.36707275,49.09392873)
\curveto(355.21707572,49.12392183)(355.06707587,49.1589218)(354.91707275,49.19892873)
\curveto(354.86707607,49.21892174)(354.82207612,49.22392173)(354.78207275,49.21392873)
\curveto(354.73207621,49.21392174)(354.68207626,49.21892174)(354.63207275,49.22892873)
\lineto(354.52707275,49.22892873)
}
}
{
\newrgbcolor{curcolor}{0 0 0}
\pscustom[linestyle=none,fillstyle=solid,fillcolor=curcolor]
{
\newpath
\moveto(350.26707275,55.69017873)
\curveto(350.26708067,55.92017394)(350.32708061,56.05017381)(350.44707275,56.08017873)
\curveto(350.55708038,56.11017375)(350.72208022,56.12517374)(350.94207275,56.12517873)
\lineto(351.22707275,56.12517873)
\curveto(351.31707962,56.12517374)(351.39207955,56.10017376)(351.45207275,56.05017873)
\curveto(351.53207941,55.99017387)(351.57707936,55.90517396)(351.58707275,55.79517873)
\curveto(351.58707935,55.68517418)(351.60207934,55.57517429)(351.63207275,55.46517873)
\curveto(351.66207928,55.32517454)(351.69207925,55.19017467)(351.72207275,55.06017873)
\curveto(351.75207919,54.94017492)(351.79207915,54.82517504)(351.84207275,54.71517873)
\curveto(351.97207897,54.42517544)(352.15207879,54.19017567)(352.38207275,54.01017873)
\curveto(352.60207834,53.83017603)(352.85707808,53.67517619)(353.14707275,53.54517873)
\curveto(353.25707768,53.50517636)(353.37207757,53.47517639)(353.49207275,53.45517873)
\curveto(353.60207734,53.43517643)(353.71707722,53.41017645)(353.83707275,53.38017873)
\curveto(353.88707705,53.37017649)(353.937077,53.3651765)(353.98707275,53.36517873)
\curveto(354.0370769,53.37517649)(354.08707685,53.37517649)(354.13707275,53.36517873)
\curveto(354.25707668,53.33517653)(354.39707654,53.32017654)(354.55707275,53.32017873)
\curveto(354.70707623,53.33017653)(354.85207609,53.33517653)(354.99207275,53.33517873)
\lineto(356.83707275,53.33517873)
\lineto(357.18207275,53.33517873)
\curveto(357.30207364,53.33517653)(357.41707352,53.33017653)(357.52707275,53.32017873)
\curveto(357.6370733,53.31017655)(357.73207321,53.30517656)(357.81207275,53.30517873)
\curveto(357.89207305,53.31517655)(357.96207298,53.29517657)(358.02207275,53.24517873)
\curveto(358.09207285,53.19517667)(358.13207281,53.11517675)(358.14207275,53.00517873)
\curveto(358.15207279,52.90517696)(358.15707278,52.79517707)(358.15707275,52.67517873)
\lineto(358.15707275,52.40517873)
\curveto(358.1370728,52.35517751)(358.12207282,52.30517756)(358.11207275,52.25517873)
\curveto(358.09207285,52.21517765)(358.06707287,52.18517768)(358.03707275,52.16517873)
\curveto(357.96707297,52.11517775)(357.88207306,52.08517778)(357.78207275,52.07517873)
\lineto(357.45207275,52.07517873)
\lineto(356.29707275,52.07517873)
\lineto(352.14207275,52.07517873)
\lineto(351.10707275,52.07517873)
\lineto(350.80707275,52.07517873)
\curveto(350.70708023,52.08517778)(350.62208032,52.11517775)(350.55207275,52.16517873)
\curveto(350.51208043,52.19517767)(350.48208046,52.24517762)(350.46207275,52.31517873)
\curveto(350.4420805,52.39517747)(350.43208051,52.48017738)(350.43207275,52.57017873)
\curveto(350.42208052,52.6601772)(350.42208052,52.75017711)(350.43207275,52.84017873)
\curveto(350.4420805,52.93017693)(350.45708048,53.00017686)(350.47707275,53.05017873)
\curveto(350.50708043,53.13017673)(350.56708037,53.18017668)(350.65707275,53.20017873)
\curveto(350.7370802,53.23017663)(350.82708011,53.24517662)(350.92707275,53.24517873)
\lineto(351.22707275,53.24517873)
\curveto(351.32707961,53.24517662)(351.41707952,53.2651766)(351.49707275,53.30517873)
\curveto(351.51707942,53.31517655)(351.53207941,53.32517654)(351.54207275,53.33517873)
\lineto(351.58707275,53.38017873)
\curveto(351.58707935,53.49017637)(351.5420794,53.58017628)(351.45207275,53.65017873)
\curveto(351.35207959,53.72017614)(351.27207967,53.78017608)(351.21207275,53.83017873)
\lineto(351.12207275,53.92017873)
\curveto(351.01207993,54.01017585)(350.89708004,54.13517573)(350.77707275,54.29517873)
\curveto(350.65708028,54.45517541)(350.56708037,54.60517526)(350.50707275,54.74517873)
\curveto(350.45708048,54.83517503)(350.42208052,54.93017493)(350.40207275,55.03017873)
\curveto(350.37208057,55.13017473)(350.3420806,55.23517463)(350.31207275,55.34517873)
\curveto(350.30208064,55.40517446)(350.29708064,55.4651744)(350.29707275,55.52517873)
\curveto(350.28708065,55.58517428)(350.27708066,55.64017422)(350.26707275,55.69017873)
}
}
{
\newrgbcolor{curcolor}{0 0 0}
\pscustom[linestyle=none,fillstyle=solid,fillcolor=curcolor]
{
}
}
{
\newrgbcolor{curcolor}{0 0 0}
\pscustom[linestyle=none,fillstyle=solid,fillcolor=curcolor]
{
\newpath
\moveto(353.08707275,67.99510061)
\lineto(353.34207275,67.99510061)
\curveto(353.42207752,68.0050929)(353.49707744,68.00009291)(353.56707275,67.98010061)
\lineto(353.80707275,67.98010061)
\lineto(353.97207275,67.98010061)
\curveto(354.07207687,67.96009295)(354.17707676,67.95009296)(354.28707275,67.95010061)
\curveto(354.38707655,67.95009296)(354.48707645,67.94009297)(354.58707275,67.92010061)
\lineto(354.73707275,67.92010061)
\curveto(354.87707606,67.89009302)(355.01707592,67.87009304)(355.15707275,67.86010061)
\curveto(355.28707565,67.85009306)(355.41707552,67.82509308)(355.54707275,67.78510061)
\curveto(355.62707531,67.76509314)(355.71207523,67.74509316)(355.80207275,67.72510061)
\lineto(356.04207275,67.66510061)
\lineto(356.34207275,67.54510061)
\curveto(356.43207451,67.51509339)(356.52207442,67.48009343)(356.61207275,67.44010061)
\curveto(356.83207411,67.34009357)(357.04707389,67.2050937)(357.25707275,67.03510061)
\curveto(357.46707347,66.87509403)(357.6370733,66.70009421)(357.76707275,66.51010061)
\curveto(357.80707313,66.46009445)(357.84707309,66.40009451)(357.88707275,66.33010061)
\curveto(357.91707302,66.27009464)(357.95207299,66.2100947)(357.99207275,66.15010061)
\curveto(358.0420729,66.07009484)(358.08207286,65.97509493)(358.11207275,65.86510061)
\curveto(358.1420728,65.75509515)(358.17207277,65.65009526)(358.20207275,65.55010061)
\curveto(358.2420727,65.44009547)(358.26707267,65.33009558)(358.27707275,65.22010061)
\curveto(358.28707265,65.1100958)(358.30207264,64.99509591)(358.32207275,64.87510061)
\curveto(358.33207261,64.83509607)(358.33207261,64.79009612)(358.32207275,64.74010061)
\curveto(358.32207262,64.70009621)(358.32707261,64.66009625)(358.33707275,64.62010061)
\curveto(358.34707259,64.58009633)(358.35207259,64.52509638)(358.35207275,64.45510061)
\curveto(358.35207259,64.38509652)(358.34707259,64.33509657)(358.33707275,64.30510061)
\curveto(358.31707262,64.25509665)(358.31207263,64.2100967)(358.32207275,64.17010061)
\curveto(358.33207261,64.13009678)(358.33207261,64.09509681)(358.32207275,64.06510061)
\lineto(358.32207275,63.97510061)
\curveto(358.30207264,63.91509699)(358.28707265,63.85009706)(358.27707275,63.78010061)
\curveto(358.27707266,63.72009719)(358.27207267,63.65509725)(358.26207275,63.58510061)
\curveto(358.21207273,63.41509749)(358.16207278,63.25509765)(358.11207275,63.10510061)
\curveto(358.06207288,62.95509795)(357.99707294,62.8100981)(357.91707275,62.67010061)
\curveto(357.87707306,62.62009829)(357.84707309,62.56509834)(357.82707275,62.50510061)
\curveto(357.79707314,62.45509845)(357.76207318,62.4050985)(357.72207275,62.35510061)
\curveto(357.5420734,62.11509879)(357.32207362,61.91509899)(357.06207275,61.75510061)
\curveto(356.80207414,61.59509931)(356.51707442,61.45509945)(356.20707275,61.33510061)
\curveto(356.06707487,61.27509963)(355.92707501,61.23009968)(355.78707275,61.20010061)
\curveto(355.6370753,61.17009974)(355.48207546,61.13509977)(355.32207275,61.09510061)
\curveto(355.21207573,61.07509983)(355.10207584,61.06009985)(354.99207275,61.05010061)
\curveto(354.88207606,61.04009987)(354.77207617,61.02509988)(354.66207275,61.00510061)
\curveto(354.62207632,60.99509991)(354.58207636,60.99009992)(354.54207275,60.99010061)
\curveto(354.50207644,61.00009991)(354.46207648,61.00009991)(354.42207275,60.99010061)
\curveto(354.37207657,60.98009993)(354.32207662,60.97509993)(354.27207275,60.97510061)
\lineto(354.10707275,60.97510061)
\curveto(354.05707688,60.95509995)(354.00707693,60.95009996)(353.95707275,60.96010061)
\curveto(353.89707704,60.97009994)(353.8420771,60.97009994)(353.79207275,60.96010061)
\curveto(353.75207719,60.95009996)(353.70707723,60.95009996)(353.65707275,60.96010061)
\curveto(353.60707733,60.97009994)(353.55707738,60.96509994)(353.50707275,60.94510061)
\curveto(353.4370775,60.92509998)(353.36207758,60.92009999)(353.28207275,60.93010061)
\curveto(353.19207775,60.94009997)(353.10707783,60.94509996)(353.02707275,60.94510061)
\curveto(352.937078,60.94509996)(352.8370781,60.94009997)(352.72707275,60.93010061)
\curveto(352.60707833,60.92009999)(352.50707843,60.92509998)(352.42707275,60.94510061)
\lineto(352.14207275,60.94510061)
\lineto(351.51207275,60.99010061)
\curveto(351.41207953,61.00009991)(351.31707962,61.0100999)(351.22707275,61.02010061)
\lineto(350.92707275,61.05010061)
\curveto(350.87708006,61.07009984)(350.82708011,61.07509983)(350.77707275,61.06510061)
\curveto(350.71708022,61.06509984)(350.66208028,61.07509983)(350.61207275,61.09510061)
\curveto(350.4420805,61.14509976)(350.27708066,61.18509972)(350.11707275,61.21510061)
\curveto(349.94708099,61.24509966)(349.78708115,61.29509961)(349.63707275,61.36510061)
\curveto(349.17708176,61.55509935)(348.80208214,61.77509913)(348.51207275,62.02510061)
\curveto(348.22208272,62.28509862)(347.97708296,62.64509826)(347.77707275,63.10510061)
\curveto(347.72708321,63.23509767)(347.69208325,63.36509754)(347.67207275,63.49510061)
\curveto(347.65208329,63.63509727)(347.62708331,63.77509713)(347.59707275,63.91510061)
\curveto(347.58708335,63.98509692)(347.58208336,64.05009686)(347.58207275,64.11010061)
\curveto(347.58208336,64.17009674)(347.57708336,64.23509667)(347.56707275,64.30510061)
\curveto(347.54708339,65.13509577)(347.69708324,65.8050951)(348.01707275,66.31510061)
\curveto(348.32708261,66.82509408)(348.76708217,67.2050937)(349.33707275,67.45510061)
\curveto(349.45708148,67.5050934)(349.58208136,67.55009336)(349.71207275,67.59010061)
\curveto(349.8420811,67.63009328)(349.97708096,67.67509323)(350.11707275,67.72510061)
\curveto(350.19708074,67.74509316)(350.28208066,67.76009315)(350.37207275,67.77010061)
\lineto(350.61207275,67.83010061)
\curveto(350.72208022,67.86009305)(350.83208011,67.87509303)(350.94207275,67.87510061)
\curveto(351.05207989,67.88509302)(351.16207978,67.90009301)(351.27207275,67.92010061)
\curveto(351.32207962,67.94009297)(351.36707957,67.94509296)(351.40707275,67.93510061)
\curveto(351.44707949,67.93509297)(351.48707945,67.94009297)(351.52707275,67.95010061)
\curveto(351.57707936,67.96009295)(351.63207931,67.96009295)(351.69207275,67.95010061)
\curveto(351.7420792,67.95009296)(351.79207915,67.95509295)(351.84207275,67.96510061)
\lineto(351.97707275,67.96510061)
\curveto(352.0370789,67.98509292)(352.10707883,67.98509292)(352.18707275,67.96510061)
\curveto(352.25707868,67.95509295)(352.32207862,67.96009295)(352.38207275,67.98010061)
\curveto(352.41207853,67.99009292)(352.45207849,67.99509291)(352.50207275,67.99510061)
\lineto(352.62207275,67.99510061)
\lineto(353.08707275,67.99510061)
\moveto(355.41207275,66.45010061)
\curveto(355.09207585,66.55009436)(354.72707621,66.6100943)(354.31707275,66.63010061)
\curveto(353.90707703,66.65009426)(353.49707744,66.66009425)(353.08707275,66.66010061)
\curveto(352.65707828,66.66009425)(352.2370787,66.65009426)(351.82707275,66.63010061)
\curveto(351.41707952,66.6100943)(351.03207991,66.56509434)(350.67207275,66.49510061)
\curveto(350.31208063,66.42509448)(349.99208095,66.31509459)(349.71207275,66.16510061)
\curveto(349.42208152,66.02509488)(349.18708175,65.83009508)(349.00707275,65.58010061)
\curveto(348.89708204,65.42009549)(348.81708212,65.24009567)(348.76707275,65.04010061)
\curveto(348.70708223,64.84009607)(348.67708226,64.59509631)(348.67707275,64.30510061)
\curveto(348.69708224,64.28509662)(348.70708223,64.25009666)(348.70707275,64.20010061)
\curveto(348.69708224,64.15009676)(348.69708224,64.1100968)(348.70707275,64.08010061)
\curveto(348.72708221,64.00009691)(348.74708219,63.92509698)(348.76707275,63.85510061)
\curveto(348.77708216,63.79509711)(348.79708214,63.73009718)(348.82707275,63.66010061)
\curveto(348.94708199,63.39009752)(349.11708182,63.17009774)(349.33707275,63.00010061)
\curveto(349.54708139,62.84009807)(349.79208115,62.7050982)(350.07207275,62.59510061)
\curveto(350.18208076,62.54509836)(350.30208064,62.5050984)(350.43207275,62.47510061)
\curveto(350.55208039,62.45509845)(350.67708026,62.43009848)(350.80707275,62.40010061)
\curveto(350.85708008,62.38009853)(350.91208003,62.37009854)(350.97207275,62.37010061)
\curveto(351.02207992,62.37009854)(351.07207987,62.36509854)(351.12207275,62.35510061)
\curveto(351.21207973,62.34509856)(351.30707963,62.33509857)(351.40707275,62.32510061)
\curveto(351.49707944,62.31509859)(351.59207935,62.3050986)(351.69207275,62.29510061)
\curveto(351.77207917,62.29509861)(351.85707908,62.29009862)(351.94707275,62.28010061)
\lineto(352.18707275,62.28010061)
\lineto(352.36707275,62.28010061)
\curveto(352.39707854,62.27009864)(352.43207851,62.26509864)(352.47207275,62.26510061)
\lineto(352.60707275,62.26510061)
\lineto(353.05707275,62.26510061)
\curveto(353.1370778,62.26509864)(353.22207772,62.26009865)(353.31207275,62.25010061)
\curveto(353.39207755,62.25009866)(353.46707747,62.26009865)(353.53707275,62.28010061)
\lineto(353.80707275,62.28010061)
\curveto(353.82707711,62.28009863)(353.85707708,62.27509863)(353.89707275,62.26510061)
\curveto(353.92707701,62.26509864)(353.95207699,62.27009864)(353.97207275,62.28010061)
\curveto(354.07207687,62.29009862)(354.17207677,62.29509861)(354.27207275,62.29510061)
\curveto(354.36207658,62.3050986)(354.46207648,62.31509859)(354.57207275,62.32510061)
\curveto(354.69207625,62.35509855)(354.81707612,62.37009854)(354.94707275,62.37010061)
\curveto(355.06707587,62.38009853)(355.18207576,62.4050985)(355.29207275,62.44510061)
\curveto(355.59207535,62.52509838)(355.85707508,62.6100983)(356.08707275,62.70010061)
\curveto(356.31707462,62.80009811)(356.53207441,62.94509796)(356.73207275,63.13510061)
\curveto(356.93207401,63.34509756)(357.08207386,63.6100973)(357.18207275,63.93010061)
\curveto(357.20207374,63.97009694)(357.21207373,64.0050969)(357.21207275,64.03510061)
\curveto(357.20207374,64.07509683)(357.20707373,64.12009679)(357.22707275,64.17010061)
\curveto(357.2370737,64.2100967)(357.24707369,64.28009663)(357.25707275,64.38010061)
\curveto(357.26707367,64.49009642)(357.26207368,64.57509633)(357.24207275,64.63510061)
\curveto(357.22207372,64.7050962)(357.21207373,64.77509613)(357.21207275,64.84510061)
\curveto(357.20207374,64.91509599)(357.18707375,64.98009593)(357.16707275,65.04010061)
\curveto(357.10707383,65.24009567)(357.02207392,65.42009549)(356.91207275,65.58010061)
\curveto(356.89207405,65.6100953)(356.87207407,65.63509527)(356.85207275,65.65510061)
\lineto(356.79207275,65.71510061)
\curveto(356.77207417,65.75509515)(356.73207421,65.8050951)(356.67207275,65.86510061)
\curveto(356.53207441,65.96509494)(356.40207454,66.05009486)(356.28207275,66.12010061)
\curveto(356.16207478,66.19009472)(356.01707492,66.26009465)(355.84707275,66.33010061)
\curveto(355.77707516,66.36009455)(355.70707523,66.38009453)(355.63707275,66.39010061)
\curveto(355.56707537,66.4100945)(355.49207545,66.43009448)(355.41207275,66.45010061)
}
}
{
\newrgbcolor{curcolor}{0 0 0}
\pscustom[linestyle=none,fillstyle=solid,fillcolor=curcolor]
{
\newpath
\moveto(347.76207275,69.80470998)
\lineto(347.76207275,74.60470998)
\lineto(347.76207275,75.60970998)
\curveto(347.76208318,75.74970288)(347.77208317,75.86970276)(347.79207275,75.96970998)
\curveto(347.80208314,76.07970255)(347.84708309,76.15970247)(347.92707275,76.20970998)
\curveto(347.96708297,76.2297024)(348.01708292,76.23970239)(348.07707275,76.23970998)
\curveto(348.1370828,76.24970238)(348.20208274,76.25470238)(348.27207275,76.25470998)
\lineto(348.54207275,76.25470998)
\curveto(348.63208231,76.25470238)(348.71208223,76.24470239)(348.78207275,76.22470998)
\curveto(348.86208208,76.18470245)(348.93208201,76.13970249)(348.99207275,76.08970998)
\lineto(349.17207275,75.93970998)
\curveto(349.22208172,75.90970272)(349.26208168,75.87470276)(349.29207275,75.83470998)
\curveto(349.32208162,75.79470284)(349.36208158,75.75470288)(349.41207275,75.71470998)
\curveto(349.52208142,75.634703)(349.63208131,75.54970308)(349.74207275,75.45970998)
\curveto(349.8420811,75.36970326)(349.94708099,75.28470335)(350.05707275,75.20470998)
\curveto(350.25708068,75.06470357)(350.46708047,74.92470371)(350.68707275,74.78470998)
\curveto(350.89708004,74.64470399)(351.11207983,74.50470413)(351.33207275,74.36470998)
\curveto(351.42207952,74.31470432)(351.51707942,74.26470437)(351.61707275,74.21470998)
\curveto(351.71707922,74.16470447)(351.81207913,74.10970452)(351.90207275,74.04970998)
\curveto(351.92207902,74.0297046)(351.94707899,74.01970461)(351.97707275,74.01970998)
\curveto(352.00707893,74.01970461)(352.03207891,74.00970462)(352.05207275,73.98970998)
\curveto(352.15207879,73.91970471)(352.26707867,73.85470478)(352.39707275,73.79470998)
\curveto(352.51707842,73.7347049)(352.63207831,73.67970495)(352.74207275,73.62970998)
\curveto(352.97207797,73.5297051)(353.20707773,73.4347052)(353.44707275,73.34470998)
\curveto(353.68707725,73.25470538)(353.92707701,73.15470548)(354.16707275,73.04470998)
\curveto(354.21707672,73.02470561)(354.26207668,73.00970562)(354.30207275,72.99970998)
\curveto(354.3420766,72.99970563)(354.38707655,72.98970564)(354.43707275,72.96970998)
\curveto(354.55707638,72.91970571)(354.68207626,72.87470576)(354.81207275,72.83470998)
\curveto(354.93207601,72.80470583)(355.05207589,72.76970586)(355.17207275,72.72970998)
\curveto(355.40207554,72.64970598)(355.6420753,72.58470605)(355.89207275,72.53470998)
\curveto(356.13207481,72.49470614)(356.37207457,72.44470619)(356.61207275,72.38470998)
\curveto(356.76207418,72.34470629)(356.91207403,72.31970631)(357.06207275,72.30970998)
\curveto(357.21207373,72.29970633)(357.36207358,72.27970635)(357.51207275,72.24970998)
\curveto(357.55207339,72.23970639)(357.61207333,72.2347064)(357.69207275,72.23470998)
\curveto(357.81207313,72.20470643)(357.91207303,72.17470646)(357.99207275,72.14470998)
\curveto(358.07207287,72.11470652)(358.12707281,72.04470659)(358.15707275,71.93470998)
\curveto(358.17707276,71.88470675)(358.18707275,71.8297068)(358.18707275,71.76970998)
\lineto(358.18707275,71.57470998)
\curveto(358.18707275,71.4347072)(358.18207276,71.29470734)(358.17207275,71.15470998)
\curveto(358.16207278,71.02470761)(358.11707282,70.9297077)(358.03707275,70.86970998)
\curveto(357.97707296,70.8297078)(357.89207305,70.80970782)(357.78207275,70.80970998)
\curveto(357.67207327,70.81970781)(357.57707336,70.8347078)(357.49707275,70.85470998)
\lineto(357.42207275,70.85470998)
\curveto(357.39207355,70.86470777)(357.36207358,70.86970776)(357.33207275,70.86970998)
\curveto(357.25207369,70.88970774)(357.17707376,70.89970773)(357.10707275,70.89970998)
\curveto(357.0370739,70.89970773)(356.96707397,70.90970772)(356.89707275,70.92970998)
\curveto(356.70707423,70.97970765)(356.52207442,71.01970761)(356.34207275,71.04970998)
\curveto(356.15207479,71.07970755)(355.97207497,71.11970751)(355.80207275,71.16970998)
\curveto(355.75207519,71.18970744)(355.71207523,71.19970743)(355.68207275,71.19970998)
\curveto(355.65207529,71.19970743)(355.61707532,71.20470743)(355.57707275,71.21470998)
\curveto(355.27707566,71.31470732)(354.98207596,71.40470723)(354.69207275,71.48470998)
\curveto(354.40207654,71.57470706)(354.12207682,71.67970695)(353.85207275,71.79970998)
\curveto(353.27207767,72.05970657)(352.72207822,72.3297063)(352.20207275,72.60970998)
\curveto(351.67207927,72.88970574)(351.16707977,73.19970543)(350.68707275,73.53970998)
\curveto(350.48708045,73.67970495)(350.29708064,73.8297048)(350.11707275,73.98970998)
\curveto(349.92708101,74.14970448)(349.7370812,74.29970433)(349.54707275,74.43970998)
\curveto(349.49708144,74.47970415)(349.45208149,74.51470412)(349.41207275,74.54470998)
\curveto(349.36208158,74.58470405)(349.31208163,74.61970401)(349.26207275,74.64970998)
\curveto(349.2420817,74.65970397)(349.21708172,74.66970396)(349.18707275,74.67970998)
\curveto(349.15708178,74.69970393)(349.12708181,74.69970393)(349.09707275,74.67970998)
\curveto(349.0370819,74.65970397)(349.00208194,74.62470401)(348.99207275,74.57470998)
\curveto(348.97208197,74.52470411)(348.95208199,74.47470416)(348.93207275,74.42470998)
\lineto(348.93207275,74.31970998)
\curveto(348.92208202,74.27970435)(348.92208202,74.2297044)(348.93207275,74.16970998)
\lineto(348.93207275,74.01970998)
\lineto(348.93207275,73.41970998)
\lineto(348.93207275,70.77970998)
\lineto(348.93207275,70.04470998)
\lineto(348.93207275,69.80470998)
\curveto(348.92208202,69.7347089)(348.90708203,69.67470896)(348.88707275,69.62470998)
\curveto(348.84708209,69.5347091)(348.78708215,69.47470916)(348.70707275,69.44470998)
\curveto(348.60708233,69.39470924)(348.46208248,69.37970925)(348.27207275,69.39970998)
\curveto(348.07208287,69.41970921)(347.937083,69.45470918)(347.86707275,69.50470998)
\curveto(347.84708309,69.52470911)(347.83208311,69.54970908)(347.82207275,69.57970998)
\lineto(347.76207275,69.69970998)
\curveto(347.76208318,69.71970891)(347.76708317,69.7347089)(347.77707275,69.74470998)
\curveto(347.77708316,69.76470887)(347.77208317,69.78470885)(347.76207275,69.80470998)
}
}
{
\newrgbcolor{curcolor}{0 0 0}
\pscustom[linestyle=none,fillstyle=solid,fillcolor=curcolor]
{
\newpath
\moveto(356.53707275,78.63431936)
\lineto(356.53707275,79.26431936)
\lineto(356.53707275,79.45931936)
\curveto(356.5370744,79.52931683)(356.54707439,79.58931677)(356.56707275,79.63931936)
\curveto(356.60707433,79.70931665)(356.64707429,79.7593166)(356.68707275,79.78931936)
\curveto(356.7370742,79.82931653)(356.80207414,79.84931651)(356.88207275,79.84931936)
\curveto(356.96207398,79.8593165)(357.04707389,79.86431649)(357.13707275,79.86431936)
\lineto(357.85707275,79.86431936)
\curveto(358.3370726,79.86431649)(358.74707219,79.80431655)(359.08707275,79.68431936)
\curveto(359.42707151,79.56431679)(359.70207124,79.36931699)(359.91207275,79.09931936)
\curveto(359.96207098,79.02931733)(360.00707093,78.9593174)(360.04707275,78.88931936)
\curveto(360.09707084,78.82931753)(360.1420708,78.7543176)(360.18207275,78.66431936)
\curveto(360.19207075,78.64431771)(360.20207074,78.61431774)(360.21207275,78.57431936)
\curveto(360.23207071,78.53431782)(360.2370707,78.48931787)(360.22707275,78.43931936)
\curveto(360.19707074,78.34931801)(360.12207082,78.29431806)(360.00207275,78.27431936)
\curveto(359.89207105,78.2543181)(359.79707114,78.26931809)(359.71707275,78.31931936)
\curveto(359.64707129,78.34931801)(359.58207136,78.39431796)(359.52207275,78.45431936)
\curveto(359.47207147,78.52431783)(359.42207152,78.58931777)(359.37207275,78.64931936)
\curveto(359.32207162,78.71931764)(359.24707169,78.77931758)(359.14707275,78.82931936)
\curveto(359.05707188,78.88931747)(358.96707197,78.93931742)(358.87707275,78.97931936)
\curveto(358.84707209,78.99931736)(358.78707215,79.02431733)(358.69707275,79.05431936)
\curveto(358.61707232,79.08431727)(358.54707239,79.08931727)(358.48707275,79.06931936)
\curveto(358.34707259,79.03931732)(358.25707268,78.97931738)(358.21707275,78.88931936)
\curveto(358.18707275,78.80931755)(358.17207277,78.71931764)(358.17207275,78.61931936)
\curveto(358.17207277,78.51931784)(358.14707279,78.43431792)(358.09707275,78.36431936)
\curveto(358.02707291,78.27431808)(357.88707305,78.22931813)(357.67707275,78.22931936)
\lineto(357.12207275,78.22931936)
\lineto(356.89707275,78.22931936)
\curveto(356.81707412,78.23931812)(356.75207419,78.2593181)(356.70207275,78.28931936)
\curveto(356.62207432,78.34931801)(356.57707436,78.41931794)(356.56707275,78.49931936)
\curveto(356.55707438,78.51931784)(356.55207439,78.53931782)(356.55207275,78.55931936)
\curveto(356.55207439,78.58931777)(356.54707439,78.61431774)(356.53707275,78.63431936)
}
}
{
\newrgbcolor{curcolor}{0 0 0}
\pscustom[linestyle=none,fillstyle=solid,fillcolor=curcolor]
{
}
}
{
\newrgbcolor{curcolor}{0 0 0}
\pscustom[linestyle=none,fillstyle=solid,fillcolor=curcolor]
{
\newpath
\moveto(347.56707275,89.26463186)
\curveto(347.55708338,89.95462722)(347.67708326,90.55462662)(347.92707275,91.06463186)
\curveto(348.17708276,91.58462559)(348.51208243,91.9796252)(348.93207275,92.24963186)
\curveto(349.01208193,92.29962488)(349.10208184,92.34462483)(349.20207275,92.38463186)
\curveto(349.29208165,92.42462475)(349.38708155,92.46962471)(349.48707275,92.51963186)
\curveto(349.58708135,92.55962462)(349.68708125,92.58962459)(349.78707275,92.60963186)
\curveto(349.88708105,92.62962455)(349.99208095,92.64962453)(350.10207275,92.66963186)
\curveto(350.15208079,92.68962449)(350.19708074,92.69462448)(350.23707275,92.68463186)
\curveto(350.27708066,92.6746245)(350.32208062,92.6796245)(350.37207275,92.69963186)
\curveto(350.42208052,92.70962447)(350.50708043,92.71462446)(350.62707275,92.71463186)
\curveto(350.7370802,92.71462446)(350.82208012,92.70962447)(350.88207275,92.69963186)
\curveto(350.94208,92.6796245)(351.00207994,92.66962451)(351.06207275,92.66963186)
\curveto(351.12207982,92.6796245)(351.18207976,92.6746245)(351.24207275,92.65463186)
\curveto(351.38207956,92.61462456)(351.51707942,92.5796246)(351.64707275,92.54963186)
\curveto(351.77707916,92.51962466)(351.90207904,92.4796247)(352.02207275,92.42963186)
\curveto(352.16207878,92.36962481)(352.28707865,92.29962488)(352.39707275,92.21963186)
\curveto(352.50707843,92.14962503)(352.61707832,92.0746251)(352.72707275,91.99463186)
\lineto(352.78707275,91.93463186)
\curveto(352.80707813,91.92462525)(352.82707811,91.90962527)(352.84707275,91.88963186)
\curveto(353.00707793,91.76962541)(353.15207779,91.63462554)(353.28207275,91.48463186)
\curveto(353.41207753,91.33462584)(353.5370774,91.174626)(353.65707275,91.00463186)
\curveto(353.87707706,90.69462648)(354.08207686,90.39962678)(354.27207275,90.11963186)
\curveto(354.41207653,89.88962729)(354.54707639,89.65962752)(354.67707275,89.42963186)
\curveto(354.80707613,89.20962797)(354.942076,88.98962819)(355.08207275,88.76963186)
\curveto(355.25207569,88.51962866)(355.43207551,88.2796289)(355.62207275,88.04963186)
\curveto(355.81207513,87.82962935)(356.0370749,87.63962954)(356.29707275,87.47963186)
\curveto(356.35707458,87.43962974)(356.41707452,87.40462977)(356.47707275,87.37463186)
\curveto(356.52707441,87.34462983)(356.59207435,87.31462986)(356.67207275,87.28463186)
\curveto(356.7420742,87.26462991)(356.80207414,87.25962992)(356.85207275,87.26963186)
\curveto(356.92207402,87.28962989)(356.97707396,87.32462985)(357.01707275,87.37463186)
\curveto(357.04707389,87.42462975)(357.06707387,87.48462969)(357.07707275,87.55463186)
\lineto(357.07707275,87.79463186)
\lineto(357.07707275,88.54463186)
\lineto(357.07707275,91.34963186)
\lineto(357.07707275,92.00963186)
\curveto(357.07707386,92.09962508)(357.08207386,92.18462499)(357.09207275,92.26463186)
\curveto(357.09207385,92.34462483)(357.11207383,92.40962477)(357.15207275,92.45963186)
\curveto(357.19207375,92.50962467)(357.26707367,92.54962463)(357.37707275,92.57963186)
\curveto(357.47707346,92.61962456)(357.57707336,92.62962455)(357.67707275,92.60963186)
\lineto(357.81207275,92.60963186)
\curveto(357.88207306,92.58962459)(357.942073,92.56962461)(357.99207275,92.54963186)
\curveto(358.0420729,92.52962465)(358.08207286,92.49462468)(358.11207275,92.44463186)
\curveto(358.15207279,92.39462478)(358.17207277,92.32462485)(358.17207275,92.23463186)
\lineto(358.17207275,91.96463186)
\lineto(358.17207275,91.06463186)
\lineto(358.17207275,87.55463186)
\lineto(358.17207275,86.48963186)
\curveto(358.17207277,86.40963077)(358.17707276,86.31963086)(358.18707275,86.21963186)
\curveto(358.18707275,86.11963106)(358.17707276,86.03463114)(358.15707275,85.96463186)
\curveto(358.08707285,85.75463142)(357.90707303,85.68963149)(357.61707275,85.76963186)
\curveto(357.57707336,85.7796314)(357.5420734,85.7796314)(357.51207275,85.76963186)
\curveto(357.47207347,85.76963141)(357.42707351,85.7796314)(357.37707275,85.79963186)
\curveto(357.29707364,85.81963136)(357.21207373,85.83963134)(357.12207275,85.85963186)
\curveto(357.03207391,85.8796313)(356.94707399,85.90463127)(356.86707275,85.93463186)
\curveto(356.37707456,86.09463108)(355.96207498,86.29463088)(355.62207275,86.53463186)
\curveto(355.37207557,86.71463046)(355.14707579,86.91963026)(354.94707275,87.14963186)
\curveto(354.7370762,87.3796298)(354.5420764,87.61962956)(354.36207275,87.86963186)
\curveto(354.18207676,88.12962905)(354.01207693,88.39462878)(353.85207275,88.66463186)
\curveto(353.68207726,88.94462823)(353.50707743,89.21462796)(353.32707275,89.47463186)
\curveto(353.24707769,89.58462759)(353.17207777,89.68962749)(353.10207275,89.78963186)
\curveto(353.03207791,89.89962728)(352.95707798,90.00962717)(352.87707275,90.11963186)
\curveto(352.84707809,90.15962702)(352.81707812,90.19462698)(352.78707275,90.22463186)
\curveto(352.74707819,90.26462691)(352.71707822,90.30462687)(352.69707275,90.34463186)
\curveto(352.58707835,90.48462669)(352.46207848,90.60962657)(352.32207275,90.71963186)
\curveto(352.29207865,90.73962644)(352.26707867,90.76462641)(352.24707275,90.79463186)
\curveto(352.21707872,90.82462635)(352.18707875,90.84962633)(352.15707275,90.86963186)
\curveto(352.05707888,90.94962623)(351.95707898,91.01462616)(351.85707275,91.06463186)
\curveto(351.75707918,91.12462605)(351.64707929,91.179626)(351.52707275,91.22963186)
\curveto(351.45707948,91.25962592)(351.38207956,91.2796259)(351.30207275,91.28963186)
\lineto(351.06207275,91.34963186)
\lineto(350.97207275,91.34963186)
\curveto(350.94208,91.35962582)(350.91208003,91.36462581)(350.88207275,91.36463186)
\curveto(350.81208013,91.38462579)(350.71708022,91.38962579)(350.59707275,91.37963186)
\curveto(350.46708047,91.3796258)(350.36708057,91.36962581)(350.29707275,91.34963186)
\curveto(350.21708072,91.32962585)(350.1420808,91.30962587)(350.07207275,91.28963186)
\curveto(349.99208095,91.2796259)(349.91208103,91.25962592)(349.83207275,91.22963186)
\curveto(349.59208135,91.11962606)(349.39208155,90.96962621)(349.23207275,90.77963186)
\curveto(349.06208188,90.59962658)(348.92208202,90.3796268)(348.81207275,90.11963186)
\curveto(348.79208215,90.04962713)(348.77708216,89.9796272)(348.76707275,89.90963186)
\curveto(348.74708219,89.83962734)(348.72708221,89.76462741)(348.70707275,89.68463186)
\curveto(348.68708225,89.60462757)(348.67708226,89.49462768)(348.67707275,89.35463186)
\curveto(348.67708226,89.22462795)(348.68708225,89.11962806)(348.70707275,89.03963186)
\curveto(348.71708222,88.9796282)(348.72208222,88.92462825)(348.72207275,88.87463186)
\curveto(348.72208222,88.82462835)(348.73208221,88.7746284)(348.75207275,88.72463186)
\curveto(348.79208215,88.62462855)(348.83208211,88.52962865)(348.87207275,88.43963186)
\curveto(348.91208203,88.35962882)(348.95708198,88.2796289)(349.00707275,88.19963186)
\curveto(349.02708191,88.16962901)(349.05208189,88.13962904)(349.08207275,88.10963186)
\curveto(349.11208183,88.08962909)(349.1370818,88.06462911)(349.15707275,88.03463186)
\lineto(349.23207275,87.95963186)
\curveto(349.25208169,87.92962925)(349.27208167,87.90462927)(349.29207275,87.88463186)
\lineto(349.50207275,87.73463186)
\curveto(349.56208138,87.69462948)(349.62708131,87.64962953)(349.69707275,87.59963186)
\curveto(349.78708115,87.53962964)(349.89208105,87.48962969)(350.01207275,87.44963186)
\curveto(350.12208082,87.41962976)(350.23208071,87.38462979)(350.34207275,87.34463186)
\curveto(350.45208049,87.30462987)(350.59708034,87.2796299)(350.77707275,87.26963186)
\curveto(350.94707999,87.25962992)(351.07207987,87.22962995)(351.15207275,87.17963186)
\curveto(351.23207971,87.12963005)(351.27707966,87.05463012)(351.28707275,86.95463186)
\curveto(351.29707964,86.85463032)(351.30207964,86.74463043)(351.30207275,86.62463186)
\curveto(351.30207964,86.58463059)(351.30707963,86.54463063)(351.31707275,86.50463186)
\curveto(351.31707962,86.46463071)(351.31207963,86.42963075)(351.30207275,86.39963186)
\curveto(351.28207966,86.34963083)(351.27207967,86.29963088)(351.27207275,86.24963186)
\curveto(351.27207967,86.20963097)(351.26207968,86.16963101)(351.24207275,86.12963186)
\curveto(351.18207976,86.03963114)(351.04707989,85.99463118)(350.83707275,85.99463186)
\lineto(350.71707275,85.99463186)
\curveto(350.65708028,86.00463117)(350.59708034,86.00963117)(350.53707275,86.00963186)
\curveto(350.46708047,86.01963116)(350.40208054,86.02963115)(350.34207275,86.03963186)
\curveto(350.23208071,86.05963112)(350.13208081,86.0796311)(350.04207275,86.09963186)
\curveto(349.942081,86.11963106)(349.84708109,86.14963103)(349.75707275,86.18963186)
\curveto(349.68708125,86.20963097)(349.62708131,86.22963095)(349.57707275,86.24963186)
\lineto(349.39707275,86.30963186)
\curveto(349.1370818,86.42963075)(348.89208205,86.58463059)(348.66207275,86.77463186)
\curveto(348.43208251,86.9746302)(348.24708269,87.18962999)(348.10707275,87.41963186)
\curveto(348.02708291,87.52962965)(347.96208298,87.64462953)(347.91207275,87.76463186)
\lineto(347.76207275,88.15463186)
\curveto(347.71208323,88.26462891)(347.68208326,88.3796288)(347.67207275,88.49963186)
\curveto(347.65208329,88.61962856)(347.62708331,88.74462843)(347.59707275,88.87463186)
\curveto(347.59708334,88.94462823)(347.59708334,89.00962817)(347.59707275,89.06963186)
\curveto(347.58708335,89.12962805)(347.57708336,89.19462798)(347.56707275,89.26463186)
}
}
{
\newrgbcolor{curcolor}{0 0 0}
\pscustom[linestyle=none,fillstyle=solid,fillcolor=curcolor]
{
\newpath
\moveto(353.08707275,101.36424123)
\lineto(353.34207275,101.36424123)
\curveto(353.42207752,101.37423353)(353.49707744,101.36923353)(353.56707275,101.34924123)
\lineto(353.80707275,101.34924123)
\lineto(353.97207275,101.34924123)
\curveto(354.07207687,101.32923357)(354.17707676,101.31923358)(354.28707275,101.31924123)
\curveto(354.38707655,101.31923358)(354.48707645,101.30923359)(354.58707275,101.28924123)
\lineto(354.73707275,101.28924123)
\curveto(354.87707606,101.25923364)(355.01707592,101.23923366)(355.15707275,101.22924123)
\curveto(355.28707565,101.21923368)(355.41707552,101.19423371)(355.54707275,101.15424123)
\curveto(355.62707531,101.13423377)(355.71207523,101.11423379)(355.80207275,101.09424123)
\lineto(356.04207275,101.03424123)
\lineto(356.34207275,100.91424123)
\curveto(356.43207451,100.88423402)(356.52207442,100.84923405)(356.61207275,100.80924123)
\curveto(356.83207411,100.70923419)(357.04707389,100.57423433)(357.25707275,100.40424123)
\curveto(357.46707347,100.24423466)(357.6370733,100.06923483)(357.76707275,99.87924123)
\curveto(357.80707313,99.82923507)(357.84707309,99.76923513)(357.88707275,99.69924123)
\curveto(357.91707302,99.63923526)(357.95207299,99.57923532)(357.99207275,99.51924123)
\curveto(358.0420729,99.43923546)(358.08207286,99.34423556)(358.11207275,99.23424123)
\curveto(358.1420728,99.12423578)(358.17207277,99.01923588)(358.20207275,98.91924123)
\curveto(358.2420727,98.80923609)(358.26707267,98.6992362)(358.27707275,98.58924123)
\curveto(358.28707265,98.47923642)(358.30207264,98.36423654)(358.32207275,98.24424123)
\curveto(358.33207261,98.2042367)(358.33207261,98.15923674)(358.32207275,98.10924123)
\curveto(358.32207262,98.06923683)(358.32707261,98.02923687)(358.33707275,97.98924123)
\curveto(358.34707259,97.94923695)(358.35207259,97.89423701)(358.35207275,97.82424123)
\curveto(358.35207259,97.75423715)(358.34707259,97.7042372)(358.33707275,97.67424123)
\curveto(358.31707262,97.62423728)(358.31207263,97.57923732)(358.32207275,97.53924123)
\curveto(358.33207261,97.4992374)(358.33207261,97.46423744)(358.32207275,97.43424123)
\lineto(358.32207275,97.34424123)
\curveto(358.30207264,97.28423762)(358.28707265,97.21923768)(358.27707275,97.14924123)
\curveto(358.27707266,97.08923781)(358.27207267,97.02423788)(358.26207275,96.95424123)
\curveto(358.21207273,96.78423812)(358.16207278,96.62423828)(358.11207275,96.47424123)
\curveto(358.06207288,96.32423858)(357.99707294,96.17923872)(357.91707275,96.03924123)
\curveto(357.87707306,95.98923891)(357.84707309,95.93423897)(357.82707275,95.87424123)
\curveto(357.79707314,95.82423908)(357.76207318,95.77423913)(357.72207275,95.72424123)
\curveto(357.5420734,95.48423942)(357.32207362,95.28423962)(357.06207275,95.12424123)
\curveto(356.80207414,94.96423994)(356.51707442,94.82424008)(356.20707275,94.70424123)
\curveto(356.06707487,94.64424026)(355.92707501,94.5992403)(355.78707275,94.56924123)
\curveto(355.6370753,94.53924036)(355.48207546,94.5042404)(355.32207275,94.46424123)
\curveto(355.21207573,94.44424046)(355.10207584,94.42924047)(354.99207275,94.41924123)
\curveto(354.88207606,94.40924049)(354.77207617,94.39424051)(354.66207275,94.37424123)
\curveto(354.62207632,94.36424054)(354.58207636,94.35924054)(354.54207275,94.35924123)
\curveto(354.50207644,94.36924053)(354.46207648,94.36924053)(354.42207275,94.35924123)
\curveto(354.37207657,94.34924055)(354.32207662,94.34424056)(354.27207275,94.34424123)
\lineto(354.10707275,94.34424123)
\curveto(354.05707688,94.32424058)(354.00707693,94.31924058)(353.95707275,94.32924123)
\curveto(353.89707704,94.33924056)(353.8420771,94.33924056)(353.79207275,94.32924123)
\curveto(353.75207719,94.31924058)(353.70707723,94.31924058)(353.65707275,94.32924123)
\curveto(353.60707733,94.33924056)(353.55707738,94.33424057)(353.50707275,94.31424123)
\curveto(353.4370775,94.29424061)(353.36207758,94.28924061)(353.28207275,94.29924123)
\curveto(353.19207775,94.30924059)(353.10707783,94.31424059)(353.02707275,94.31424123)
\curveto(352.937078,94.31424059)(352.8370781,94.30924059)(352.72707275,94.29924123)
\curveto(352.60707833,94.28924061)(352.50707843,94.29424061)(352.42707275,94.31424123)
\lineto(352.14207275,94.31424123)
\lineto(351.51207275,94.35924123)
\curveto(351.41207953,94.36924053)(351.31707962,94.37924052)(351.22707275,94.38924123)
\lineto(350.92707275,94.41924123)
\curveto(350.87708006,94.43924046)(350.82708011,94.44424046)(350.77707275,94.43424123)
\curveto(350.71708022,94.43424047)(350.66208028,94.44424046)(350.61207275,94.46424123)
\curveto(350.4420805,94.51424039)(350.27708066,94.55424035)(350.11707275,94.58424123)
\curveto(349.94708099,94.61424029)(349.78708115,94.66424024)(349.63707275,94.73424123)
\curveto(349.17708176,94.92423998)(348.80208214,95.14423976)(348.51207275,95.39424123)
\curveto(348.22208272,95.65423925)(347.97708296,96.01423889)(347.77707275,96.47424123)
\curveto(347.72708321,96.6042383)(347.69208325,96.73423817)(347.67207275,96.86424123)
\curveto(347.65208329,97.0042379)(347.62708331,97.14423776)(347.59707275,97.28424123)
\curveto(347.58708335,97.35423755)(347.58208336,97.41923748)(347.58207275,97.47924123)
\curveto(347.58208336,97.53923736)(347.57708336,97.6042373)(347.56707275,97.67424123)
\curveto(347.54708339,98.5042364)(347.69708324,99.17423573)(348.01707275,99.68424123)
\curveto(348.32708261,100.19423471)(348.76708217,100.57423433)(349.33707275,100.82424123)
\curveto(349.45708148,100.87423403)(349.58208136,100.91923398)(349.71207275,100.95924123)
\curveto(349.8420811,100.9992339)(349.97708096,101.04423386)(350.11707275,101.09424123)
\curveto(350.19708074,101.11423379)(350.28208066,101.12923377)(350.37207275,101.13924123)
\lineto(350.61207275,101.19924123)
\curveto(350.72208022,101.22923367)(350.83208011,101.24423366)(350.94207275,101.24424123)
\curveto(351.05207989,101.25423365)(351.16207978,101.26923363)(351.27207275,101.28924123)
\curveto(351.32207962,101.30923359)(351.36707957,101.31423359)(351.40707275,101.30424123)
\curveto(351.44707949,101.3042336)(351.48707945,101.30923359)(351.52707275,101.31924123)
\curveto(351.57707936,101.32923357)(351.63207931,101.32923357)(351.69207275,101.31924123)
\curveto(351.7420792,101.31923358)(351.79207915,101.32423358)(351.84207275,101.33424123)
\lineto(351.97707275,101.33424123)
\curveto(352.0370789,101.35423355)(352.10707883,101.35423355)(352.18707275,101.33424123)
\curveto(352.25707868,101.32423358)(352.32207862,101.32923357)(352.38207275,101.34924123)
\curveto(352.41207853,101.35923354)(352.45207849,101.36423354)(352.50207275,101.36424123)
\lineto(352.62207275,101.36424123)
\lineto(353.08707275,101.36424123)
\moveto(355.41207275,99.81924123)
\curveto(355.09207585,99.91923498)(354.72707621,99.97923492)(354.31707275,99.99924123)
\curveto(353.90707703,100.01923488)(353.49707744,100.02923487)(353.08707275,100.02924123)
\curveto(352.65707828,100.02923487)(352.2370787,100.01923488)(351.82707275,99.99924123)
\curveto(351.41707952,99.97923492)(351.03207991,99.93423497)(350.67207275,99.86424123)
\curveto(350.31208063,99.79423511)(349.99208095,99.68423522)(349.71207275,99.53424123)
\curveto(349.42208152,99.39423551)(349.18708175,99.1992357)(349.00707275,98.94924123)
\curveto(348.89708204,98.78923611)(348.81708212,98.60923629)(348.76707275,98.40924123)
\curveto(348.70708223,98.20923669)(348.67708226,97.96423694)(348.67707275,97.67424123)
\curveto(348.69708224,97.65423725)(348.70708223,97.61923728)(348.70707275,97.56924123)
\curveto(348.69708224,97.51923738)(348.69708224,97.47923742)(348.70707275,97.44924123)
\curveto(348.72708221,97.36923753)(348.74708219,97.29423761)(348.76707275,97.22424123)
\curveto(348.77708216,97.16423774)(348.79708214,97.0992378)(348.82707275,97.02924123)
\curveto(348.94708199,96.75923814)(349.11708182,96.53923836)(349.33707275,96.36924123)
\curveto(349.54708139,96.20923869)(349.79208115,96.07423883)(350.07207275,95.96424123)
\curveto(350.18208076,95.91423899)(350.30208064,95.87423903)(350.43207275,95.84424123)
\curveto(350.55208039,95.82423908)(350.67708026,95.7992391)(350.80707275,95.76924123)
\curveto(350.85708008,95.74923915)(350.91208003,95.73923916)(350.97207275,95.73924123)
\curveto(351.02207992,95.73923916)(351.07207987,95.73423917)(351.12207275,95.72424123)
\curveto(351.21207973,95.71423919)(351.30707963,95.7042392)(351.40707275,95.69424123)
\curveto(351.49707944,95.68423922)(351.59207935,95.67423923)(351.69207275,95.66424123)
\curveto(351.77207917,95.66423924)(351.85707908,95.65923924)(351.94707275,95.64924123)
\lineto(352.18707275,95.64924123)
\lineto(352.36707275,95.64924123)
\curveto(352.39707854,95.63923926)(352.43207851,95.63423927)(352.47207275,95.63424123)
\lineto(352.60707275,95.63424123)
\lineto(353.05707275,95.63424123)
\curveto(353.1370778,95.63423927)(353.22207772,95.62923927)(353.31207275,95.61924123)
\curveto(353.39207755,95.61923928)(353.46707747,95.62923927)(353.53707275,95.64924123)
\lineto(353.80707275,95.64924123)
\curveto(353.82707711,95.64923925)(353.85707708,95.64423926)(353.89707275,95.63424123)
\curveto(353.92707701,95.63423927)(353.95207699,95.63923926)(353.97207275,95.64924123)
\curveto(354.07207687,95.65923924)(354.17207677,95.66423924)(354.27207275,95.66424123)
\curveto(354.36207658,95.67423923)(354.46207648,95.68423922)(354.57207275,95.69424123)
\curveto(354.69207625,95.72423918)(354.81707612,95.73923916)(354.94707275,95.73924123)
\curveto(355.06707587,95.74923915)(355.18207576,95.77423913)(355.29207275,95.81424123)
\curveto(355.59207535,95.89423901)(355.85707508,95.97923892)(356.08707275,96.06924123)
\curveto(356.31707462,96.16923873)(356.53207441,96.31423859)(356.73207275,96.50424123)
\curveto(356.93207401,96.71423819)(357.08207386,96.97923792)(357.18207275,97.29924123)
\curveto(357.20207374,97.33923756)(357.21207373,97.37423753)(357.21207275,97.40424123)
\curveto(357.20207374,97.44423746)(357.20707373,97.48923741)(357.22707275,97.53924123)
\curveto(357.2370737,97.57923732)(357.24707369,97.64923725)(357.25707275,97.74924123)
\curveto(357.26707367,97.85923704)(357.26207368,97.94423696)(357.24207275,98.00424123)
\curveto(357.22207372,98.07423683)(357.21207373,98.14423676)(357.21207275,98.21424123)
\curveto(357.20207374,98.28423662)(357.18707375,98.34923655)(357.16707275,98.40924123)
\curveto(357.10707383,98.60923629)(357.02207392,98.78923611)(356.91207275,98.94924123)
\curveto(356.89207405,98.97923592)(356.87207407,99.0042359)(356.85207275,99.02424123)
\lineto(356.79207275,99.08424123)
\curveto(356.77207417,99.12423578)(356.73207421,99.17423573)(356.67207275,99.23424123)
\curveto(356.53207441,99.33423557)(356.40207454,99.41923548)(356.28207275,99.48924123)
\curveto(356.16207478,99.55923534)(356.01707492,99.62923527)(355.84707275,99.69924123)
\curveto(355.77707516,99.72923517)(355.70707523,99.74923515)(355.63707275,99.75924123)
\curveto(355.56707537,99.77923512)(355.49207545,99.7992351)(355.41207275,99.81924123)
}
}
{
\newrgbcolor{curcolor}{0 0 0}
\pscustom[linestyle=none,fillstyle=solid,fillcolor=curcolor]
{
\newpath
\moveto(347.56707275,106.77385061)
\curveto(347.56708337,106.87384575)(347.57708336,106.96884566)(347.59707275,107.05885061)
\curveto(347.60708333,107.14884548)(347.6370833,107.21384541)(347.68707275,107.25385061)
\curveto(347.76708317,107.31384531)(347.87208307,107.34384528)(348.00207275,107.34385061)
\lineto(348.39207275,107.34385061)
\lineto(349.89207275,107.34385061)
\lineto(356.28207275,107.34385061)
\lineto(357.45207275,107.34385061)
\lineto(357.76707275,107.34385061)
\curveto(357.86707307,107.35384527)(357.94707299,107.33884529)(358.00707275,107.29885061)
\curveto(358.08707285,107.24884538)(358.1370728,107.17384545)(358.15707275,107.07385061)
\curveto(358.16707277,106.98384564)(358.17207277,106.87384575)(358.17207275,106.74385061)
\lineto(358.17207275,106.51885061)
\curveto(358.15207279,106.43884619)(358.1370728,106.36884626)(358.12707275,106.30885061)
\curveto(358.10707283,106.24884638)(358.06707287,106.19884643)(358.00707275,106.15885061)
\curveto(357.94707299,106.11884651)(357.87207307,106.09884653)(357.78207275,106.09885061)
\lineto(357.48207275,106.09885061)
\lineto(356.38707275,106.09885061)
\lineto(351.04707275,106.09885061)
\curveto(350.95707998,106.07884655)(350.88208006,106.06384656)(350.82207275,106.05385061)
\curveto(350.75208019,106.05384657)(350.69208025,106.0238466)(350.64207275,105.96385061)
\curveto(350.59208035,105.89384673)(350.56708037,105.80384682)(350.56707275,105.69385061)
\curveto(350.55708038,105.59384703)(350.55208039,105.48384714)(350.55207275,105.36385061)
\lineto(350.55207275,104.22385061)
\lineto(350.55207275,103.72885061)
\curveto(350.5420804,103.56884906)(350.48208046,103.45884917)(350.37207275,103.39885061)
\curveto(350.3420806,103.37884925)(350.31208063,103.36884926)(350.28207275,103.36885061)
\curveto(350.2420807,103.36884926)(350.19708074,103.36384926)(350.14707275,103.35385061)
\curveto(350.02708091,103.33384929)(349.91708102,103.33884929)(349.81707275,103.36885061)
\curveto(349.71708122,103.40884922)(349.64708129,103.46384916)(349.60707275,103.53385061)
\curveto(349.55708138,103.61384901)(349.53208141,103.73384889)(349.53207275,103.89385061)
\curveto(349.53208141,104.05384857)(349.51708142,104.18884844)(349.48707275,104.29885061)
\curveto(349.47708146,104.34884828)(349.47208147,104.40384822)(349.47207275,104.46385061)
\curveto(349.46208148,104.5238481)(349.44708149,104.58384804)(349.42707275,104.64385061)
\curveto(349.37708156,104.79384783)(349.32708161,104.93884769)(349.27707275,105.07885061)
\curveto(349.21708172,105.21884741)(349.14708179,105.35384727)(349.06707275,105.48385061)
\curveto(348.97708196,105.623847)(348.87208207,105.74384688)(348.75207275,105.84385061)
\curveto(348.63208231,105.94384668)(348.50208244,106.03884659)(348.36207275,106.12885061)
\curveto(348.26208268,106.18884644)(348.15208279,106.23384639)(348.03207275,106.26385061)
\curveto(347.91208303,106.30384632)(347.80708313,106.35384627)(347.71707275,106.41385061)
\curveto(347.65708328,106.46384616)(347.61708332,106.53384609)(347.59707275,106.62385061)
\curveto(347.58708335,106.64384598)(347.58208336,106.66884596)(347.58207275,106.69885061)
\curveto(347.58208336,106.7288459)(347.57708336,106.75384587)(347.56707275,106.77385061)
}
}
{
\newrgbcolor{curcolor}{0 0 0}
\pscustom[linestyle=none,fillstyle=solid,fillcolor=curcolor]
{
\newpath
\moveto(347.56707275,115.12345998)
\curveto(347.56708337,115.22345513)(347.57708336,115.31845503)(347.59707275,115.40845998)
\curveto(347.60708333,115.49845485)(347.6370833,115.56345479)(347.68707275,115.60345998)
\curveto(347.76708317,115.66345469)(347.87208307,115.69345466)(348.00207275,115.69345998)
\lineto(348.39207275,115.69345998)
\lineto(349.89207275,115.69345998)
\lineto(356.28207275,115.69345998)
\lineto(357.45207275,115.69345998)
\lineto(357.76707275,115.69345998)
\curveto(357.86707307,115.70345465)(357.94707299,115.68845466)(358.00707275,115.64845998)
\curveto(358.08707285,115.59845475)(358.1370728,115.52345483)(358.15707275,115.42345998)
\curveto(358.16707277,115.33345502)(358.17207277,115.22345513)(358.17207275,115.09345998)
\lineto(358.17207275,114.86845998)
\curveto(358.15207279,114.78845556)(358.1370728,114.71845563)(358.12707275,114.65845998)
\curveto(358.10707283,114.59845575)(358.06707287,114.5484558)(358.00707275,114.50845998)
\curveto(357.94707299,114.46845588)(357.87207307,114.4484559)(357.78207275,114.44845998)
\lineto(357.48207275,114.44845998)
\lineto(356.38707275,114.44845998)
\lineto(351.04707275,114.44845998)
\curveto(350.95707998,114.42845592)(350.88208006,114.41345594)(350.82207275,114.40345998)
\curveto(350.75208019,114.40345595)(350.69208025,114.37345598)(350.64207275,114.31345998)
\curveto(350.59208035,114.24345611)(350.56708037,114.1534562)(350.56707275,114.04345998)
\curveto(350.55708038,113.94345641)(350.55208039,113.83345652)(350.55207275,113.71345998)
\lineto(350.55207275,112.57345998)
\lineto(350.55207275,112.07845998)
\curveto(350.5420804,111.91845843)(350.48208046,111.80845854)(350.37207275,111.74845998)
\curveto(350.3420806,111.72845862)(350.31208063,111.71845863)(350.28207275,111.71845998)
\curveto(350.2420807,111.71845863)(350.19708074,111.71345864)(350.14707275,111.70345998)
\curveto(350.02708091,111.68345867)(349.91708102,111.68845866)(349.81707275,111.71845998)
\curveto(349.71708122,111.75845859)(349.64708129,111.81345854)(349.60707275,111.88345998)
\curveto(349.55708138,111.96345839)(349.53208141,112.08345827)(349.53207275,112.24345998)
\curveto(349.53208141,112.40345795)(349.51708142,112.53845781)(349.48707275,112.64845998)
\curveto(349.47708146,112.69845765)(349.47208147,112.7534576)(349.47207275,112.81345998)
\curveto(349.46208148,112.87345748)(349.44708149,112.93345742)(349.42707275,112.99345998)
\curveto(349.37708156,113.14345721)(349.32708161,113.28845706)(349.27707275,113.42845998)
\curveto(349.21708172,113.56845678)(349.14708179,113.70345665)(349.06707275,113.83345998)
\curveto(348.97708196,113.97345638)(348.87208207,114.09345626)(348.75207275,114.19345998)
\curveto(348.63208231,114.29345606)(348.50208244,114.38845596)(348.36207275,114.47845998)
\curveto(348.26208268,114.53845581)(348.15208279,114.58345577)(348.03207275,114.61345998)
\curveto(347.91208303,114.6534557)(347.80708313,114.70345565)(347.71707275,114.76345998)
\curveto(347.65708328,114.81345554)(347.61708332,114.88345547)(347.59707275,114.97345998)
\curveto(347.58708335,114.99345536)(347.58208336,115.01845533)(347.58207275,115.04845998)
\curveto(347.58208336,115.07845527)(347.57708336,115.10345525)(347.56707275,115.12345998)
}
}
{
\newrgbcolor{curcolor}{0 0 0}
\pscustom[linestyle=none,fillstyle=solid,fillcolor=curcolor]
{
\newpath
\moveto(378.30341919,42.02236623)
\curveto(378.35341993,42.04235669)(378.41341987,42.06735666)(378.48341919,42.09736623)
\curveto(378.55341973,42.1273566)(378.62841966,42.14735658)(378.70841919,42.15736623)
\curveto(378.77841951,42.17735655)(378.84841944,42.17735655)(378.91841919,42.15736623)
\curveto(378.97841931,42.14735658)(379.02341926,42.10735662)(379.05341919,42.03736623)
\curveto(379.07341921,41.98735674)(379.0834192,41.9273568)(379.08341919,41.85736623)
\lineto(379.08341919,41.64736623)
\lineto(379.08341919,41.19736623)
\curveto(379.0834192,41.04735768)(379.05841923,40.9273578)(379.00841919,40.83736623)
\curveto(378.94841934,40.73735799)(378.84341944,40.66235807)(378.69341919,40.61236623)
\curveto(378.54341974,40.57235816)(378.40841988,40.5273582)(378.28841919,40.47736623)
\curveto(378.02842026,40.36735836)(377.75842053,40.26735846)(377.47841919,40.17736623)
\curveto(377.19842109,40.08735864)(376.92342136,39.98735874)(376.65341919,39.87736623)
\curveto(376.56342172,39.84735888)(376.47842181,39.81735891)(376.39841919,39.78736623)
\curveto(376.31842197,39.76735896)(376.24342204,39.73735899)(376.17341919,39.69736623)
\curveto(376.10342218,39.66735906)(376.04342224,39.62235911)(375.99341919,39.56236623)
\curveto(375.94342234,39.50235923)(375.90342238,39.42235931)(375.87341919,39.32236623)
\curveto(375.85342243,39.27235946)(375.84842244,39.21235952)(375.85841919,39.14236623)
\lineto(375.85841919,38.94736623)
\lineto(375.85841919,36.11236623)
\lineto(375.85841919,35.81236623)
\curveto(375.84842244,35.70236303)(375.84842244,35.59736313)(375.85841919,35.49736623)
\curveto(375.86842242,35.39736333)(375.8834224,35.30236343)(375.90341919,35.21236623)
\curveto(375.92342236,35.1323636)(375.96342232,35.07236366)(376.02341919,35.03236623)
\curveto(376.12342216,34.95236378)(376.23842205,34.89236384)(376.36841919,34.85236623)
\curveto(376.4884218,34.82236391)(376.61342167,34.78236395)(376.74341919,34.73236623)
\curveto(376.97342131,34.6323641)(377.21342107,34.53736419)(377.46341919,34.44736623)
\curveto(377.71342057,34.36736436)(377.95342033,34.27736445)(378.18341919,34.17736623)
\curveto(378.24342004,34.15736457)(378.31341997,34.1323646)(378.39341919,34.10236623)
\curveto(378.46341982,34.08236465)(378.53841975,34.05736467)(378.61841919,34.02736623)
\curveto(378.69841959,33.99736473)(378.77341951,33.96236477)(378.84341919,33.92236623)
\curveto(378.90341938,33.89236484)(378.94841934,33.85736487)(378.97841919,33.81736623)
\curveto(379.03841925,33.73736499)(379.07341921,33.6273651)(379.08341919,33.48736623)
\lineto(379.08341919,33.06736623)
\lineto(379.08341919,32.82736623)
\curveto(379.07341921,32.75736597)(379.04841924,32.69736603)(379.00841919,32.64736623)
\curveto(378.97841931,32.59736613)(378.93341935,32.56736616)(378.87341919,32.55736623)
\curveto(378.81341947,32.55736617)(378.75341953,32.56236617)(378.69341919,32.57236623)
\curveto(378.62341966,32.59236614)(378.55841973,32.61236612)(378.49841919,32.63236623)
\curveto(378.42841986,32.66236607)(378.37841991,32.68736604)(378.34841919,32.70736623)
\curveto(378.02842026,32.84736588)(377.71342057,32.97236576)(377.40341919,33.08236623)
\curveto(377.0834212,33.19236554)(376.76342152,33.31236542)(376.44341919,33.44236623)
\curveto(376.22342206,33.5323652)(376.00842228,33.61736511)(375.79841919,33.69736623)
\curveto(375.57842271,33.77736495)(375.35842293,33.86236487)(375.13841919,33.95236623)
\curveto(374.41842387,34.25236448)(373.69342459,34.53736419)(372.96341919,34.80736623)
\curveto(372.22342606,35.07736365)(371.4884268,35.36236337)(370.75841919,35.66236623)
\curveto(370.49842779,35.77236296)(370.23342805,35.87236286)(369.96341919,35.96236623)
\curveto(369.69342859,36.06236267)(369.42842886,36.16736256)(369.16841919,36.27736623)
\curveto(369.05842923,36.3273624)(368.93842935,36.37236236)(368.80841919,36.41236623)
\curveto(368.66842962,36.46236227)(368.56842972,36.5323622)(368.50841919,36.62236623)
\curveto(368.46842982,36.66236207)(368.43842985,36.727362)(368.41841919,36.81736623)
\curveto(368.40842988,36.83736189)(368.40842988,36.85736187)(368.41841919,36.87736623)
\curveto(368.41842987,36.90736182)(368.41342987,36.9323618)(368.40341919,36.95236623)
\curveto(368.40342988,37.1323616)(368.40342988,37.34236139)(368.40341919,37.58236623)
\curveto(368.39342989,37.82236091)(368.42842986,37.99736073)(368.50841919,38.10736623)
\curveto(368.56842972,38.18736054)(368.66842962,38.24736048)(368.80841919,38.28736623)
\curveto(368.93842935,38.33736039)(369.05842923,38.38736034)(369.16841919,38.43736623)
\curveto(369.39842889,38.53736019)(369.62842866,38.6273601)(369.85841919,38.70736623)
\curveto(370.0884282,38.78735994)(370.31842797,38.87735985)(370.54841919,38.97736623)
\curveto(370.74842754,39.05735967)(370.95342733,39.1323596)(371.16341919,39.20236623)
\curveto(371.37342691,39.28235945)(371.57842671,39.36735936)(371.77841919,39.45736623)
\curveto(372.50842578,39.75735897)(373.24842504,40.04235869)(373.99841919,40.31236623)
\curveto(374.73842355,40.59235814)(375.47342281,40.88735784)(376.20341919,41.19736623)
\curveto(376.29342199,41.23735749)(376.37842191,41.26735746)(376.45841919,41.28736623)
\curveto(376.53842175,41.31735741)(376.62342166,41.34735738)(376.71341919,41.37736623)
\curveto(376.97342131,41.48735724)(377.23842105,41.59235714)(377.50841919,41.69236623)
\curveto(377.77842051,41.80235693)(378.04342024,41.91235682)(378.30341919,42.02236623)
\moveto(374.65841919,38.81236623)
\curveto(374.62842366,38.90235983)(374.57842371,38.95735977)(374.50841919,38.97736623)
\curveto(374.43842385,39.00735972)(374.36342392,39.01235972)(374.28341919,38.99236623)
\curveto(374.19342409,38.98235975)(374.10842418,38.95735977)(374.02841919,38.91736623)
\curveto(373.93842435,38.88735984)(373.86342442,38.85735987)(373.80341919,38.82736623)
\curveto(373.76342452,38.80735992)(373.72842456,38.79735993)(373.69841919,38.79736623)
\curveto(373.66842462,38.79735993)(373.63342465,38.78735994)(373.59341919,38.76736623)
\lineto(373.35341919,38.67736623)
\curveto(373.26342502,38.65736007)(373.17342511,38.6273601)(373.08341919,38.58736623)
\curveto(372.72342556,38.43736029)(372.35842593,38.30236043)(371.98841919,38.18236623)
\curveto(371.60842668,38.07236066)(371.23842705,37.94236079)(370.87841919,37.79236623)
\curveto(370.76842752,37.74236099)(370.65842763,37.69736103)(370.54841919,37.65736623)
\curveto(370.43842785,37.6273611)(370.33342795,37.58736114)(370.23341919,37.53736623)
\curveto(370.1834281,37.51736121)(370.13842815,37.49236124)(370.09841919,37.46236623)
\curveto(370.04842824,37.44236129)(370.02342826,37.39236134)(370.02341919,37.31236623)
\curveto(370.04342824,37.29236144)(370.05842823,37.27236146)(370.06841919,37.25236623)
\curveto(370.07842821,37.2323615)(370.09342819,37.21236152)(370.11341919,37.19236623)
\curveto(370.16342812,37.15236158)(370.21842807,37.12236161)(370.27841919,37.10236623)
\curveto(370.32842796,37.08236165)(370.3834279,37.06236167)(370.44341919,37.04236623)
\curveto(370.55342773,36.99236174)(370.66342762,36.95236178)(370.77341919,36.92236623)
\curveto(370.8834274,36.89236184)(370.99342729,36.85236188)(371.10341919,36.80236623)
\curveto(371.49342679,36.6323621)(371.8884264,36.48236225)(372.28841919,36.35236623)
\curveto(372.6884256,36.2323625)(373.07842521,36.09236264)(373.45841919,35.93236623)
\lineto(373.60841919,35.87236623)
\curveto(373.65842463,35.86236287)(373.70842458,35.84736288)(373.75841919,35.82736623)
\lineto(373.99841919,35.73736623)
\curveto(374.07842421,35.70736302)(374.15842413,35.68236305)(374.23841919,35.66236623)
\curveto(374.288424,35.64236309)(374.34342394,35.6323631)(374.40341919,35.63236623)
\curveto(374.46342382,35.64236309)(374.51342377,35.65736307)(374.55341919,35.67736623)
\curveto(374.63342365,35.727363)(374.67842361,35.8323629)(374.68841919,35.99236623)
\lineto(374.68841919,36.44236623)
\lineto(374.68841919,38.04736623)
\curveto(374.6884236,38.15736057)(374.69342359,38.29236044)(374.70341919,38.45236623)
\curveto(374.70342358,38.61236012)(374.6884236,38.73236)(374.65841919,38.81236623)
}
}
{
\newrgbcolor{curcolor}{0 0 0}
\pscustom[linestyle=none,fillstyle=solid,fillcolor=curcolor]
{
\newpath
\moveto(375.04841919,50.56392873)
\curveto(375.09842319,50.57392038)(375.16842312,50.57892038)(375.25841919,50.57892873)
\curveto(375.33842295,50.57892038)(375.40342288,50.57392038)(375.45341919,50.56392873)
\curveto(375.49342279,50.56392039)(375.53342275,50.5589204)(375.57341919,50.54892873)
\lineto(375.69341919,50.54892873)
\curveto(375.77342251,50.52892043)(375.85342243,50.51892044)(375.93341919,50.51892873)
\curveto(376.01342227,50.51892044)(376.09342219,50.50892045)(376.17341919,50.48892873)
\curveto(376.21342207,50.47892048)(376.25342203,50.47392048)(376.29341919,50.47392873)
\curveto(376.32342196,50.47392048)(376.35842193,50.46892049)(376.39841919,50.45892873)
\curveto(376.50842178,50.42892053)(376.61342167,50.39892056)(376.71341919,50.36892873)
\curveto(376.81342147,50.34892061)(376.91342137,50.31892064)(377.01341919,50.27892873)
\curveto(377.36342092,50.13892082)(377.67842061,49.96892099)(377.95841919,49.76892873)
\curveto(378.23842005,49.56892139)(378.47841981,49.31892164)(378.67841919,49.01892873)
\curveto(378.77841951,48.86892209)(378.86341942,48.72392223)(378.93341919,48.58392873)
\curveto(378.9834193,48.47392248)(379.02341926,48.36392259)(379.05341919,48.25392873)
\curveto(379.0834192,48.1539228)(379.11341917,48.04892291)(379.14341919,47.93892873)
\curveto(379.16341912,47.86892309)(379.17341911,47.80392315)(379.17341919,47.74392873)
\curveto(379.1834191,47.68392327)(379.19841909,47.62392333)(379.21841919,47.56392873)
\lineto(379.21841919,47.41392873)
\curveto(379.23841905,47.36392359)(379.24841904,47.28892367)(379.24841919,47.18892873)
\curveto(379.25841903,47.08892387)(379.25341903,47.00892395)(379.23341919,46.94892873)
\lineto(379.23341919,46.79892873)
\curveto(379.22341906,46.7589242)(379.21841907,46.71392424)(379.21841919,46.66392873)
\curveto(379.21841907,46.62392433)(379.21341907,46.57892438)(379.20341919,46.52892873)
\curveto(379.16341912,46.37892458)(379.12841916,46.22892473)(379.09841919,46.07892873)
\curveto(379.06841922,45.93892502)(379.02341926,45.79892516)(378.96341919,45.65892873)
\curveto(378.8834194,45.4589255)(378.7834195,45.27892568)(378.66341919,45.11892873)
\lineto(378.51341919,44.93892873)
\curveto(378.45341983,44.87892608)(378.41341987,44.80892615)(378.39341919,44.72892873)
\curveto(378.3834199,44.66892629)(378.39841989,44.61892634)(378.43841919,44.57892873)
\curveto(378.46841982,44.54892641)(378.51341977,44.52392643)(378.57341919,44.50392873)
\curveto(378.63341965,44.49392646)(378.69841959,44.48392647)(378.76841919,44.47392873)
\curveto(378.82841946,44.47392648)(378.87341941,44.46392649)(378.90341919,44.44392873)
\curveto(378.95341933,44.40392655)(378.99841929,44.3589266)(379.03841919,44.30892873)
\curveto(379.05841923,44.2589267)(379.07341921,44.18892677)(379.08341919,44.09892873)
\lineto(379.08341919,43.82892873)
\curveto(379.0834192,43.73892722)(379.07841921,43.6539273)(379.06841919,43.57392873)
\curveto(379.04841924,43.49392746)(379.02841926,43.43392752)(379.00841919,43.39392873)
\curveto(378.9884193,43.37392758)(378.96341932,43.3539276)(378.93341919,43.33392873)
\lineto(378.84341919,43.27392873)
\curveto(378.76341952,43.24392771)(378.64341964,43.22892773)(378.48341919,43.22892873)
\curveto(378.32341996,43.23892772)(378.1884201,43.24392771)(378.07841919,43.24392873)
\lineto(369.27341919,43.24392873)
\curveto(369.15342913,43.24392771)(369.02842926,43.23892772)(368.89841919,43.22892873)
\curveto(368.75842953,43.22892773)(368.64842964,43.2539277)(368.56841919,43.30392873)
\curveto(368.50842978,43.34392761)(368.45842983,43.40892755)(368.41841919,43.49892873)
\curveto(368.41842987,43.51892744)(368.41842987,43.54392741)(368.41841919,43.57392873)
\curveto(368.40842988,43.60392735)(368.40342988,43.62892733)(368.40341919,43.64892873)
\curveto(368.39342989,43.78892717)(368.39342989,43.93392702)(368.40341919,44.08392873)
\curveto(368.40342988,44.24392671)(368.44342984,44.3539266)(368.52341919,44.41392873)
\curveto(368.60342968,44.46392649)(368.71842957,44.48892647)(368.86841919,44.48892873)
\lineto(369.27341919,44.48892873)
\lineto(371.02841919,44.48892873)
\lineto(371.28341919,44.48892873)
\lineto(371.56841919,44.48892873)
\curveto(371.65842663,44.49892646)(371.74342654,44.50892645)(371.82341919,44.51892873)
\curveto(371.89342639,44.53892642)(371.94342634,44.56892639)(371.97341919,44.60892873)
\curveto(372.00342628,44.64892631)(372.00842628,44.69392626)(371.98841919,44.74392873)
\curveto(371.96842632,44.79392616)(371.94842634,44.83392612)(371.92841919,44.86392873)
\curveto(371.8884264,44.91392604)(371.84842644,44.958926)(371.80841919,44.99892873)
\lineto(371.68841919,45.14892873)
\curveto(371.63842665,45.21892574)(371.59342669,45.28892567)(371.55341919,45.35892873)
\lineto(371.43341919,45.59892873)
\curveto(371.34342694,45.77892518)(371.27842701,45.99392496)(371.23841919,46.24392873)
\curveto(371.19842709,46.49392446)(371.17842711,46.74892421)(371.17841919,47.00892873)
\curveto(371.17842711,47.26892369)(371.20342708,47.52392343)(371.25341919,47.77392873)
\curveto(371.29342699,48.02392293)(371.35342693,48.24392271)(371.43341919,48.43392873)
\curveto(371.60342668,48.83392212)(371.83842645,49.17892178)(372.13841919,49.46892873)
\curveto(372.43842585,49.7589212)(372.7884255,49.98892097)(373.18841919,50.15892873)
\curveto(373.29842499,50.20892075)(373.40842488,50.24892071)(373.51841919,50.27892873)
\curveto(373.61842467,50.31892064)(373.72342456,50.3589206)(373.83341919,50.39892873)
\curveto(373.94342434,50.42892053)(374.05842423,50.44892051)(374.17841919,50.45892873)
\lineto(374.50841919,50.51892873)
\curveto(374.53842375,50.52892043)(374.57342371,50.53392042)(374.61341919,50.53392873)
\curveto(374.64342364,50.53392042)(374.67342361,50.53892042)(374.70341919,50.54892873)
\curveto(374.76342352,50.56892039)(374.82342346,50.56892039)(374.88341919,50.54892873)
\curveto(374.93342335,50.53892042)(374.9884233,50.54392041)(375.04841919,50.56392873)
\moveto(375.43841919,49.22892873)
\curveto(375.3884229,49.24892171)(375.32842296,49.2539217)(375.25841919,49.24392873)
\curveto(375.1884231,49.23392172)(375.12342316,49.22892173)(375.06341919,49.22892873)
\curveto(374.89342339,49.22892173)(374.73342355,49.21892174)(374.58341919,49.19892873)
\curveto(374.43342385,49.18892177)(374.29842399,49.1589218)(374.17841919,49.10892873)
\curveto(374.07842421,49.07892188)(373.9884243,49.0539219)(373.90841919,49.03392873)
\curveto(373.82842446,49.01392194)(373.74842454,48.98392197)(373.66841919,48.94392873)
\curveto(373.41842487,48.83392212)(373.1884251,48.68392227)(372.97841919,48.49392873)
\curveto(372.75842553,48.30392265)(372.59342569,48.08392287)(372.48341919,47.83392873)
\curveto(372.45342583,47.7539232)(372.42842586,47.67392328)(372.40841919,47.59392873)
\curveto(372.37842591,47.52392343)(372.35342593,47.44892351)(372.33341919,47.36892873)
\curveto(372.30342598,47.2589237)(372.288426,47.14892381)(372.28841919,47.03892873)
\curveto(372.27842601,46.92892403)(372.27342601,46.80892415)(372.27341919,46.67892873)
\curveto(372.283426,46.62892433)(372.29342599,46.58392437)(372.30341919,46.54392873)
\lineto(372.30341919,46.40892873)
\lineto(372.36341919,46.13892873)
\curveto(372.3834259,46.0589249)(372.41342587,45.97892498)(372.45341919,45.89892873)
\curveto(372.59342569,45.5589254)(372.80342548,45.28892567)(373.08341919,45.08892873)
\curveto(373.35342493,44.88892607)(373.67342461,44.72892623)(374.04341919,44.60892873)
\curveto(374.15342413,44.56892639)(374.26342402,44.54392641)(374.37341919,44.53392873)
\curveto(374.4834238,44.52392643)(374.59842369,44.50392645)(374.71841919,44.47392873)
\curveto(374.76842352,44.46392649)(374.81342347,44.46392649)(374.85341919,44.47392873)
\curveto(374.89342339,44.48392647)(374.93842335,44.47892648)(374.98841919,44.45892873)
\curveto(375.03842325,44.44892651)(375.11342317,44.44392651)(375.21341919,44.44392873)
\curveto(375.30342298,44.44392651)(375.37342291,44.44892651)(375.42341919,44.45892873)
\lineto(375.54341919,44.45892873)
\curveto(375.5834227,44.46892649)(375.62342266,44.47392648)(375.66341919,44.47392873)
\curveto(375.70342258,44.47392648)(375.73842255,44.47892648)(375.76841919,44.48892873)
\curveto(375.79842249,44.49892646)(375.83342245,44.50392645)(375.87341919,44.50392873)
\curveto(375.90342238,44.50392645)(375.93342235,44.50892645)(375.96341919,44.51892873)
\curveto(376.04342224,44.53892642)(376.12342216,44.5539264)(376.20341919,44.56392873)
\lineto(376.44341919,44.62392873)
\curveto(376.7834215,44.73392622)(377.07342121,44.88392607)(377.31341919,45.07392873)
\curveto(377.55342073,45.27392568)(377.75342053,45.51892544)(377.91341919,45.80892873)
\curveto(377.96342032,45.89892506)(378.00342028,45.99392496)(378.03341919,46.09392873)
\curveto(378.05342023,46.19392476)(378.07842021,46.29892466)(378.10841919,46.40892873)
\curveto(378.12842016,46.4589245)(378.13842015,46.50392445)(378.13841919,46.54392873)
\curveto(378.12842016,46.59392436)(378.12842016,46.64392431)(378.13841919,46.69392873)
\curveto(378.14842014,46.73392422)(378.15342013,46.77892418)(378.15341919,46.82892873)
\lineto(378.15341919,46.96392873)
\lineto(378.15341919,47.09892873)
\curveto(378.14342014,47.13892382)(378.13842015,47.17392378)(378.13841919,47.20392873)
\curveto(378.13842015,47.23392372)(378.13342015,47.26892369)(378.12341919,47.30892873)
\curveto(378.10342018,47.38892357)(378.0884202,47.46392349)(378.07841919,47.53392873)
\curveto(378.05842023,47.60392335)(378.03342025,47.67892328)(378.00341919,47.75892873)
\curveto(377.87342041,48.06892289)(377.70342058,48.31892264)(377.49341919,48.50892873)
\curveto(377.27342101,48.69892226)(377.00842128,48.8589221)(376.69841919,48.98892873)
\curveto(376.55842173,49.03892192)(376.41842187,49.07392188)(376.27841919,49.09392873)
\curveto(376.12842216,49.12392183)(375.97842231,49.1589218)(375.82841919,49.19892873)
\curveto(375.77842251,49.21892174)(375.73342255,49.22392173)(375.69341919,49.21392873)
\curveto(375.64342264,49.21392174)(375.59342269,49.21892174)(375.54341919,49.22892873)
\lineto(375.43841919,49.22892873)
}
}
{
\newrgbcolor{curcolor}{0 0 0}
\pscustom[linestyle=none,fillstyle=solid,fillcolor=curcolor]
{
\newpath
\moveto(371.17841919,55.69017873)
\curveto(371.17842711,55.92017394)(371.23842705,56.05017381)(371.35841919,56.08017873)
\curveto(371.46842682,56.11017375)(371.63342665,56.12517374)(371.85341919,56.12517873)
\lineto(372.13841919,56.12517873)
\curveto(372.22842606,56.12517374)(372.30342598,56.10017376)(372.36341919,56.05017873)
\curveto(372.44342584,55.99017387)(372.4884258,55.90517396)(372.49841919,55.79517873)
\curveto(372.49842579,55.68517418)(372.51342577,55.57517429)(372.54341919,55.46517873)
\curveto(372.57342571,55.32517454)(372.60342568,55.19017467)(372.63341919,55.06017873)
\curveto(372.66342562,54.94017492)(372.70342558,54.82517504)(372.75341919,54.71517873)
\curveto(372.8834254,54.42517544)(373.06342522,54.19017567)(373.29341919,54.01017873)
\curveto(373.51342477,53.83017603)(373.76842452,53.67517619)(374.05841919,53.54517873)
\curveto(374.16842412,53.50517636)(374.283424,53.47517639)(374.40341919,53.45517873)
\curveto(374.51342377,53.43517643)(374.62842366,53.41017645)(374.74841919,53.38017873)
\curveto(374.79842349,53.37017649)(374.84842344,53.3651765)(374.89841919,53.36517873)
\curveto(374.94842334,53.37517649)(374.99842329,53.37517649)(375.04841919,53.36517873)
\curveto(375.16842312,53.33517653)(375.30842298,53.32017654)(375.46841919,53.32017873)
\curveto(375.61842267,53.33017653)(375.76342252,53.33517653)(375.90341919,53.33517873)
\lineto(377.74841919,53.33517873)
\lineto(378.09341919,53.33517873)
\curveto(378.21342007,53.33517653)(378.32841996,53.33017653)(378.43841919,53.32017873)
\curveto(378.54841974,53.31017655)(378.64341964,53.30517656)(378.72341919,53.30517873)
\curveto(378.80341948,53.31517655)(378.87341941,53.29517657)(378.93341919,53.24517873)
\curveto(379.00341928,53.19517667)(379.04341924,53.11517675)(379.05341919,53.00517873)
\curveto(379.06341922,52.90517696)(379.06841922,52.79517707)(379.06841919,52.67517873)
\lineto(379.06841919,52.40517873)
\curveto(379.04841924,52.35517751)(379.03341925,52.30517756)(379.02341919,52.25517873)
\curveto(379.00341928,52.21517765)(378.97841931,52.18517768)(378.94841919,52.16517873)
\curveto(378.87841941,52.11517775)(378.79341949,52.08517778)(378.69341919,52.07517873)
\lineto(378.36341919,52.07517873)
\lineto(377.20841919,52.07517873)
\lineto(373.05341919,52.07517873)
\lineto(372.01841919,52.07517873)
\lineto(371.71841919,52.07517873)
\curveto(371.61842667,52.08517778)(371.53342675,52.11517775)(371.46341919,52.16517873)
\curveto(371.42342686,52.19517767)(371.39342689,52.24517762)(371.37341919,52.31517873)
\curveto(371.35342693,52.39517747)(371.34342694,52.48017738)(371.34341919,52.57017873)
\curveto(371.33342695,52.6601772)(371.33342695,52.75017711)(371.34341919,52.84017873)
\curveto(371.35342693,52.93017693)(371.36842692,53.00017686)(371.38841919,53.05017873)
\curveto(371.41842687,53.13017673)(371.47842681,53.18017668)(371.56841919,53.20017873)
\curveto(371.64842664,53.23017663)(371.73842655,53.24517662)(371.83841919,53.24517873)
\lineto(372.13841919,53.24517873)
\curveto(372.23842605,53.24517662)(372.32842596,53.2651766)(372.40841919,53.30517873)
\curveto(372.42842586,53.31517655)(372.44342584,53.32517654)(372.45341919,53.33517873)
\lineto(372.49841919,53.38017873)
\curveto(372.49842579,53.49017637)(372.45342583,53.58017628)(372.36341919,53.65017873)
\curveto(372.26342602,53.72017614)(372.1834261,53.78017608)(372.12341919,53.83017873)
\lineto(372.03341919,53.92017873)
\curveto(371.92342636,54.01017585)(371.80842648,54.13517573)(371.68841919,54.29517873)
\curveto(371.56842672,54.45517541)(371.47842681,54.60517526)(371.41841919,54.74517873)
\curveto(371.36842692,54.83517503)(371.33342695,54.93017493)(371.31341919,55.03017873)
\curveto(371.283427,55.13017473)(371.25342703,55.23517463)(371.22341919,55.34517873)
\curveto(371.21342707,55.40517446)(371.20842708,55.4651744)(371.20841919,55.52517873)
\curveto(371.19842709,55.58517428)(371.1884271,55.64017422)(371.17841919,55.69017873)
}
}
{
\newrgbcolor{curcolor}{0 0 0}
\pscustom[linestyle=none,fillstyle=solid,fillcolor=curcolor]
{
}
}
{
\newrgbcolor{curcolor}{0 0 0}
\pscustom[linestyle=none,fillstyle=solid,fillcolor=curcolor]
{
\newpath
\moveto(368.47841919,65.05510061)
\curveto(368.47842981,65.15509575)(368.4884298,65.25009566)(368.50841919,65.34010061)
\curveto(368.51842977,65.43009548)(368.54842974,65.49509541)(368.59841919,65.53510061)
\curveto(368.67842961,65.59509531)(368.7834295,65.62509528)(368.91341919,65.62510061)
\lineto(369.30341919,65.62510061)
\lineto(370.80341919,65.62510061)
\lineto(377.19341919,65.62510061)
\lineto(378.36341919,65.62510061)
\lineto(378.67841919,65.62510061)
\curveto(378.77841951,65.63509527)(378.85841943,65.62009529)(378.91841919,65.58010061)
\curveto(378.99841929,65.53009538)(379.04841924,65.45509545)(379.06841919,65.35510061)
\curveto(379.07841921,65.26509564)(379.0834192,65.15509575)(379.08341919,65.02510061)
\lineto(379.08341919,64.80010061)
\curveto(379.06341922,64.72009619)(379.04841924,64.65009626)(379.03841919,64.59010061)
\curveto(379.01841927,64.53009638)(378.97841931,64.48009643)(378.91841919,64.44010061)
\curveto(378.85841943,64.40009651)(378.7834195,64.38009653)(378.69341919,64.38010061)
\lineto(378.39341919,64.38010061)
\lineto(377.29841919,64.38010061)
\lineto(371.95841919,64.38010061)
\curveto(371.86842642,64.36009655)(371.79342649,64.34509656)(371.73341919,64.33510061)
\curveto(371.66342662,64.33509657)(371.60342668,64.3050966)(371.55341919,64.24510061)
\curveto(371.50342678,64.17509673)(371.47842681,64.08509682)(371.47841919,63.97510061)
\curveto(371.46842682,63.87509703)(371.46342682,63.76509714)(371.46341919,63.64510061)
\lineto(371.46341919,62.50510061)
\lineto(371.46341919,62.01010061)
\curveto(371.45342683,61.85009906)(371.39342689,61.74009917)(371.28341919,61.68010061)
\curveto(371.25342703,61.66009925)(371.22342706,61.65009926)(371.19341919,61.65010061)
\curveto(371.15342713,61.65009926)(371.10842718,61.64509926)(371.05841919,61.63510061)
\curveto(370.93842735,61.61509929)(370.82842746,61.62009929)(370.72841919,61.65010061)
\curveto(370.62842766,61.69009922)(370.55842773,61.74509916)(370.51841919,61.81510061)
\curveto(370.46842782,61.89509901)(370.44342784,62.01509889)(370.44341919,62.17510061)
\curveto(370.44342784,62.33509857)(370.42842786,62.47009844)(370.39841919,62.58010061)
\curveto(370.3884279,62.63009828)(370.3834279,62.68509822)(370.38341919,62.74510061)
\curveto(370.37342791,62.8050981)(370.35842793,62.86509804)(370.33841919,62.92510061)
\curveto(370.288428,63.07509783)(370.23842805,63.22009769)(370.18841919,63.36010061)
\curveto(370.12842816,63.50009741)(370.05842823,63.63509727)(369.97841919,63.76510061)
\curveto(369.8884284,63.905097)(369.7834285,64.02509688)(369.66341919,64.12510061)
\curveto(369.54342874,64.22509668)(369.41342887,64.32009659)(369.27341919,64.41010061)
\curveto(369.17342911,64.47009644)(369.06342922,64.51509639)(368.94341919,64.54510061)
\curveto(368.82342946,64.58509632)(368.71842957,64.63509627)(368.62841919,64.69510061)
\curveto(368.56842972,64.74509616)(368.52842976,64.81509609)(368.50841919,64.90510061)
\curveto(368.49842979,64.92509598)(368.49342979,64.95009596)(368.49341919,64.98010061)
\curveto(368.49342979,65.0100959)(368.4884298,65.03509587)(368.47841919,65.05510061)
}
}
{
\newrgbcolor{curcolor}{0 0 0}
\pscustom[linestyle=none,fillstyle=solid,fillcolor=curcolor]
{
\newpath
\moveto(368.47841919,73.40470998)
\curveto(368.47842981,73.50470513)(368.4884298,73.59970503)(368.50841919,73.68970998)
\curveto(368.51842977,73.77970485)(368.54842974,73.84470479)(368.59841919,73.88470998)
\curveto(368.67842961,73.94470469)(368.7834295,73.97470466)(368.91341919,73.97470998)
\lineto(369.30341919,73.97470998)
\lineto(370.80341919,73.97470998)
\lineto(377.19341919,73.97470998)
\lineto(378.36341919,73.97470998)
\lineto(378.67841919,73.97470998)
\curveto(378.77841951,73.98470465)(378.85841943,73.96970466)(378.91841919,73.92970998)
\curveto(378.99841929,73.87970475)(379.04841924,73.80470483)(379.06841919,73.70470998)
\curveto(379.07841921,73.61470502)(379.0834192,73.50470513)(379.08341919,73.37470998)
\lineto(379.08341919,73.14970998)
\curveto(379.06341922,73.06970556)(379.04841924,72.99970563)(379.03841919,72.93970998)
\curveto(379.01841927,72.87970575)(378.97841931,72.8297058)(378.91841919,72.78970998)
\curveto(378.85841943,72.74970588)(378.7834195,72.7297059)(378.69341919,72.72970998)
\lineto(378.39341919,72.72970998)
\lineto(377.29841919,72.72970998)
\lineto(371.95841919,72.72970998)
\curveto(371.86842642,72.70970592)(371.79342649,72.69470594)(371.73341919,72.68470998)
\curveto(371.66342662,72.68470595)(371.60342668,72.65470598)(371.55341919,72.59470998)
\curveto(371.50342678,72.52470611)(371.47842681,72.4347062)(371.47841919,72.32470998)
\curveto(371.46842682,72.22470641)(371.46342682,72.11470652)(371.46341919,71.99470998)
\lineto(371.46341919,70.85470998)
\lineto(371.46341919,70.35970998)
\curveto(371.45342683,70.19970843)(371.39342689,70.08970854)(371.28341919,70.02970998)
\curveto(371.25342703,70.00970862)(371.22342706,69.99970863)(371.19341919,69.99970998)
\curveto(371.15342713,69.99970863)(371.10842718,69.99470864)(371.05841919,69.98470998)
\curveto(370.93842735,69.96470867)(370.82842746,69.96970866)(370.72841919,69.99970998)
\curveto(370.62842766,70.03970859)(370.55842773,70.09470854)(370.51841919,70.16470998)
\curveto(370.46842782,70.24470839)(370.44342784,70.36470827)(370.44341919,70.52470998)
\curveto(370.44342784,70.68470795)(370.42842786,70.81970781)(370.39841919,70.92970998)
\curveto(370.3884279,70.97970765)(370.3834279,71.0347076)(370.38341919,71.09470998)
\curveto(370.37342791,71.15470748)(370.35842793,71.21470742)(370.33841919,71.27470998)
\curveto(370.288428,71.42470721)(370.23842805,71.56970706)(370.18841919,71.70970998)
\curveto(370.12842816,71.84970678)(370.05842823,71.98470665)(369.97841919,72.11470998)
\curveto(369.8884284,72.25470638)(369.7834285,72.37470626)(369.66341919,72.47470998)
\curveto(369.54342874,72.57470606)(369.41342887,72.66970596)(369.27341919,72.75970998)
\curveto(369.17342911,72.81970581)(369.06342922,72.86470577)(368.94341919,72.89470998)
\curveto(368.82342946,72.9347057)(368.71842957,72.98470565)(368.62841919,73.04470998)
\curveto(368.56842972,73.09470554)(368.52842976,73.16470547)(368.50841919,73.25470998)
\curveto(368.49842979,73.27470536)(368.49342979,73.29970533)(368.49341919,73.32970998)
\curveto(368.49342979,73.35970527)(368.4884298,73.38470525)(368.47841919,73.40470998)
}
}
{
\newrgbcolor{curcolor}{0 0 0}
\pscustom[linestyle=none,fillstyle=solid,fillcolor=curcolor]
{
\newpath
\moveto(377.44841919,78.63431936)
\lineto(377.44841919,79.26431936)
\lineto(377.44841919,79.45931936)
\curveto(377.44842084,79.52931683)(377.45842083,79.58931677)(377.47841919,79.63931936)
\curveto(377.51842077,79.70931665)(377.55842073,79.7593166)(377.59841919,79.78931936)
\curveto(377.64842064,79.82931653)(377.71342057,79.84931651)(377.79341919,79.84931936)
\curveto(377.87342041,79.8593165)(377.95842033,79.86431649)(378.04841919,79.86431936)
\lineto(378.76841919,79.86431936)
\curveto(379.24841904,79.86431649)(379.65841863,79.80431655)(379.99841919,79.68431936)
\curveto(380.33841795,79.56431679)(380.61341767,79.36931699)(380.82341919,79.09931936)
\curveto(380.87341741,79.02931733)(380.91841737,78.9593174)(380.95841919,78.88931936)
\curveto(381.00841728,78.82931753)(381.05341723,78.7543176)(381.09341919,78.66431936)
\curveto(381.10341718,78.64431771)(381.11341717,78.61431774)(381.12341919,78.57431936)
\curveto(381.14341714,78.53431782)(381.14841714,78.48931787)(381.13841919,78.43931936)
\curveto(381.10841718,78.34931801)(381.03341725,78.29431806)(380.91341919,78.27431936)
\curveto(380.80341748,78.2543181)(380.70841758,78.26931809)(380.62841919,78.31931936)
\curveto(380.55841773,78.34931801)(380.49341779,78.39431796)(380.43341919,78.45431936)
\curveto(380.3834179,78.52431783)(380.33341795,78.58931777)(380.28341919,78.64931936)
\curveto(380.23341805,78.71931764)(380.15841813,78.77931758)(380.05841919,78.82931936)
\curveto(379.96841832,78.88931747)(379.87841841,78.93931742)(379.78841919,78.97931936)
\curveto(379.75841853,78.99931736)(379.69841859,79.02431733)(379.60841919,79.05431936)
\curveto(379.52841876,79.08431727)(379.45841883,79.08931727)(379.39841919,79.06931936)
\curveto(379.25841903,79.03931732)(379.16841912,78.97931738)(379.12841919,78.88931936)
\curveto(379.09841919,78.80931755)(379.0834192,78.71931764)(379.08341919,78.61931936)
\curveto(379.0834192,78.51931784)(379.05841923,78.43431792)(379.00841919,78.36431936)
\curveto(378.93841935,78.27431808)(378.79841949,78.22931813)(378.58841919,78.22931936)
\lineto(378.03341919,78.22931936)
\lineto(377.80841919,78.22931936)
\curveto(377.72842056,78.23931812)(377.66342062,78.2593181)(377.61341919,78.28931936)
\curveto(377.53342075,78.34931801)(377.4884208,78.41931794)(377.47841919,78.49931936)
\curveto(377.46842082,78.51931784)(377.46342082,78.53931782)(377.46341919,78.55931936)
\curveto(377.46342082,78.58931777)(377.45842083,78.61431774)(377.44841919,78.63431936)
}
}
{
\newrgbcolor{curcolor}{0 0 0}
\pscustom[linestyle=none,fillstyle=solid,fillcolor=curcolor]
{
}
}
{
\newrgbcolor{curcolor}{0 0 0}
\pscustom[linestyle=none,fillstyle=solid,fillcolor=curcolor]
{
\newpath
\moveto(368.47841919,89.26463186)
\curveto(368.46842982,89.95462722)(368.5884297,90.55462662)(368.83841919,91.06463186)
\curveto(369.0884292,91.58462559)(369.42342886,91.9796252)(369.84341919,92.24963186)
\curveto(369.92342836,92.29962488)(370.01342827,92.34462483)(370.11341919,92.38463186)
\curveto(370.20342808,92.42462475)(370.29842799,92.46962471)(370.39841919,92.51963186)
\curveto(370.49842779,92.55962462)(370.59842769,92.58962459)(370.69841919,92.60963186)
\curveto(370.79842749,92.62962455)(370.90342738,92.64962453)(371.01341919,92.66963186)
\curveto(371.06342722,92.68962449)(371.10842718,92.69462448)(371.14841919,92.68463186)
\curveto(371.1884271,92.6746245)(371.23342705,92.6796245)(371.28341919,92.69963186)
\curveto(371.33342695,92.70962447)(371.41842687,92.71462446)(371.53841919,92.71463186)
\curveto(371.64842664,92.71462446)(371.73342655,92.70962447)(371.79341919,92.69963186)
\curveto(371.85342643,92.6796245)(371.91342637,92.66962451)(371.97341919,92.66963186)
\curveto(372.03342625,92.6796245)(372.09342619,92.6746245)(372.15341919,92.65463186)
\curveto(372.29342599,92.61462456)(372.42842586,92.5796246)(372.55841919,92.54963186)
\curveto(372.6884256,92.51962466)(372.81342547,92.4796247)(372.93341919,92.42963186)
\curveto(373.07342521,92.36962481)(373.19842509,92.29962488)(373.30841919,92.21963186)
\curveto(373.41842487,92.14962503)(373.52842476,92.0746251)(373.63841919,91.99463186)
\lineto(373.69841919,91.93463186)
\curveto(373.71842457,91.92462525)(373.73842455,91.90962527)(373.75841919,91.88963186)
\curveto(373.91842437,91.76962541)(374.06342422,91.63462554)(374.19341919,91.48463186)
\curveto(374.32342396,91.33462584)(374.44842384,91.174626)(374.56841919,91.00463186)
\curveto(374.7884235,90.69462648)(374.99342329,90.39962678)(375.18341919,90.11963186)
\curveto(375.32342296,89.88962729)(375.45842283,89.65962752)(375.58841919,89.42963186)
\curveto(375.71842257,89.20962797)(375.85342243,88.98962819)(375.99341919,88.76963186)
\curveto(376.16342212,88.51962866)(376.34342194,88.2796289)(376.53341919,88.04963186)
\curveto(376.72342156,87.82962935)(376.94842134,87.63962954)(377.20841919,87.47963186)
\curveto(377.26842102,87.43962974)(377.32842096,87.40462977)(377.38841919,87.37463186)
\curveto(377.43842085,87.34462983)(377.50342078,87.31462986)(377.58341919,87.28463186)
\curveto(377.65342063,87.26462991)(377.71342057,87.25962992)(377.76341919,87.26963186)
\curveto(377.83342045,87.28962989)(377.8884204,87.32462985)(377.92841919,87.37463186)
\curveto(377.95842033,87.42462975)(377.97842031,87.48462969)(377.98841919,87.55463186)
\lineto(377.98841919,87.79463186)
\lineto(377.98841919,88.54463186)
\lineto(377.98841919,91.34963186)
\lineto(377.98841919,92.00963186)
\curveto(377.9884203,92.09962508)(377.99342029,92.18462499)(378.00341919,92.26463186)
\curveto(378.00342028,92.34462483)(378.02342026,92.40962477)(378.06341919,92.45963186)
\curveto(378.10342018,92.50962467)(378.17842011,92.54962463)(378.28841919,92.57963186)
\curveto(378.3884199,92.61962456)(378.4884198,92.62962455)(378.58841919,92.60963186)
\lineto(378.72341919,92.60963186)
\curveto(378.79341949,92.58962459)(378.85341943,92.56962461)(378.90341919,92.54963186)
\curveto(378.95341933,92.52962465)(378.99341929,92.49462468)(379.02341919,92.44463186)
\curveto(379.06341922,92.39462478)(379.0834192,92.32462485)(379.08341919,92.23463186)
\lineto(379.08341919,91.96463186)
\lineto(379.08341919,91.06463186)
\lineto(379.08341919,87.55463186)
\lineto(379.08341919,86.48963186)
\curveto(379.0834192,86.40963077)(379.0884192,86.31963086)(379.09841919,86.21963186)
\curveto(379.09841919,86.11963106)(379.0884192,86.03463114)(379.06841919,85.96463186)
\curveto(378.99841929,85.75463142)(378.81841947,85.68963149)(378.52841919,85.76963186)
\curveto(378.4884198,85.7796314)(378.45341983,85.7796314)(378.42341919,85.76963186)
\curveto(378.3834199,85.76963141)(378.33841995,85.7796314)(378.28841919,85.79963186)
\curveto(378.20842008,85.81963136)(378.12342016,85.83963134)(378.03341919,85.85963186)
\curveto(377.94342034,85.8796313)(377.85842043,85.90463127)(377.77841919,85.93463186)
\curveto(377.288421,86.09463108)(376.87342141,86.29463088)(376.53341919,86.53463186)
\curveto(376.283422,86.71463046)(376.05842223,86.91963026)(375.85841919,87.14963186)
\curveto(375.64842264,87.3796298)(375.45342283,87.61962956)(375.27341919,87.86963186)
\curveto(375.09342319,88.12962905)(374.92342336,88.39462878)(374.76341919,88.66463186)
\curveto(374.59342369,88.94462823)(374.41842387,89.21462796)(374.23841919,89.47463186)
\curveto(374.15842413,89.58462759)(374.0834242,89.68962749)(374.01341919,89.78963186)
\curveto(373.94342434,89.89962728)(373.86842442,90.00962717)(373.78841919,90.11963186)
\curveto(373.75842453,90.15962702)(373.72842456,90.19462698)(373.69841919,90.22463186)
\curveto(373.65842463,90.26462691)(373.62842466,90.30462687)(373.60841919,90.34463186)
\curveto(373.49842479,90.48462669)(373.37342491,90.60962657)(373.23341919,90.71963186)
\curveto(373.20342508,90.73962644)(373.17842511,90.76462641)(373.15841919,90.79463186)
\curveto(373.12842516,90.82462635)(373.09842519,90.84962633)(373.06841919,90.86963186)
\curveto(372.96842532,90.94962623)(372.86842542,91.01462616)(372.76841919,91.06463186)
\curveto(372.66842562,91.12462605)(372.55842573,91.179626)(372.43841919,91.22963186)
\curveto(372.36842592,91.25962592)(372.29342599,91.2796259)(372.21341919,91.28963186)
\lineto(371.97341919,91.34963186)
\lineto(371.88341919,91.34963186)
\curveto(371.85342643,91.35962582)(371.82342646,91.36462581)(371.79341919,91.36463186)
\curveto(371.72342656,91.38462579)(371.62842666,91.38962579)(371.50841919,91.37963186)
\curveto(371.37842691,91.3796258)(371.27842701,91.36962581)(371.20841919,91.34963186)
\curveto(371.12842716,91.32962585)(371.05342723,91.30962587)(370.98341919,91.28963186)
\curveto(370.90342738,91.2796259)(370.82342746,91.25962592)(370.74341919,91.22963186)
\curveto(370.50342778,91.11962606)(370.30342798,90.96962621)(370.14341919,90.77963186)
\curveto(369.97342831,90.59962658)(369.83342845,90.3796268)(369.72341919,90.11963186)
\curveto(369.70342858,90.04962713)(369.6884286,89.9796272)(369.67841919,89.90963186)
\curveto(369.65842863,89.83962734)(369.63842865,89.76462741)(369.61841919,89.68463186)
\curveto(369.59842869,89.60462757)(369.5884287,89.49462768)(369.58841919,89.35463186)
\curveto(369.5884287,89.22462795)(369.59842869,89.11962806)(369.61841919,89.03963186)
\curveto(369.62842866,88.9796282)(369.63342865,88.92462825)(369.63341919,88.87463186)
\curveto(369.63342865,88.82462835)(369.64342864,88.7746284)(369.66341919,88.72463186)
\curveto(369.70342858,88.62462855)(369.74342854,88.52962865)(369.78341919,88.43963186)
\curveto(369.82342846,88.35962882)(369.86842842,88.2796289)(369.91841919,88.19963186)
\curveto(369.93842835,88.16962901)(369.96342832,88.13962904)(369.99341919,88.10963186)
\curveto(370.02342826,88.08962909)(370.04842824,88.06462911)(370.06841919,88.03463186)
\lineto(370.14341919,87.95963186)
\curveto(370.16342812,87.92962925)(370.1834281,87.90462927)(370.20341919,87.88463186)
\lineto(370.41341919,87.73463186)
\curveto(370.47342781,87.69462948)(370.53842775,87.64962953)(370.60841919,87.59963186)
\curveto(370.69842759,87.53962964)(370.80342748,87.48962969)(370.92341919,87.44963186)
\curveto(371.03342725,87.41962976)(371.14342714,87.38462979)(371.25341919,87.34463186)
\curveto(371.36342692,87.30462987)(371.50842678,87.2796299)(371.68841919,87.26963186)
\curveto(371.85842643,87.25962992)(371.9834263,87.22962995)(372.06341919,87.17963186)
\curveto(372.14342614,87.12963005)(372.1884261,87.05463012)(372.19841919,86.95463186)
\curveto(372.20842608,86.85463032)(372.21342607,86.74463043)(372.21341919,86.62463186)
\curveto(372.21342607,86.58463059)(372.21842607,86.54463063)(372.22841919,86.50463186)
\curveto(372.22842606,86.46463071)(372.22342606,86.42963075)(372.21341919,86.39963186)
\curveto(372.19342609,86.34963083)(372.1834261,86.29963088)(372.18341919,86.24963186)
\curveto(372.1834261,86.20963097)(372.17342611,86.16963101)(372.15341919,86.12963186)
\curveto(372.09342619,86.03963114)(371.95842633,85.99463118)(371.74841919,85.99463186)
\lineto(371.62841919,85.99463186)
\curveto(371.56842672,86.00463117)(371.50842678,86.00963117)(371.44841919,86.00963186)
\curveto(371.37842691,86.01963116)(371.31342697,86.02963115)(371.25341919,86.03963186)
\curveto(371.14342714,86.05963112)(371.04342724,86.0796311)(370.95341919,86.09963186)
\curveto(370.85342743,86.11963106)(370.75842753,86.14963103)(370.66841919,86.18963186)
\curveto(370.59842769,86.20963097)(370.53842775,86.22963095)(370.48841919,86.24963186)
\lineto(370.30841919,86.30963186)
\curveto(370.04842824,86.42963075)(369.80342848,86.58463059)(369.57341919,86.77463186)
\curveto(369.34342894,86.9746302)(369.15842913,87.18962999)(369.01841919,87.41963186)
\curveto(368.93842935,87.52962965)(368.87342941,87.64462953)(368.82341919,87.76463186)
\lineto(368.67341919,88.15463186)
\curveto(368.62342966,88.26462891)(368.59342969,88.3796288)(368.58341919,88.49963186)
\curveto(368.56342972,88.61962856)(368.53842975,88.74462843)(368.50841919,88.87463186)
\curveto(368.50842978,88.94462823)(368.50842978,89.00962817)(368.50841919,89.06963186)
\curveto(368.49842979,89.12962805)(368.4884298,89.19462798)(368.47841919,89.26463186)
}
}
{
\newrgbcolor{curcolor}{0 0 0}
\pscustom[linestyle=none,fillstyle=solid,fillcolor=curcolor]
{
\newpath
\moveto(373.99841919,101.36424123)
\lineto(374.25341919,101.36424123)
\curveto(374.33342395,101.37423353)(374.40842388,101.36923353)(374.47841919,101.34924123)
\lineto(374.71841919,101.34924123)
\lineto(374.88341919,101.34924123)
\curveto(374.9834233,101.32923357)(375.0884232,101.31923358)(375.19841919,101.31924123)
\curveto(375.29842299,101.31923358)(375.39842289,101.30923359)(375.49841919,101.28924123)
\lineto(375.64841919,101.28924123)
\curveto(375.7884225,101.25923364)(375.92842236,101.23923366)(376.06841919,101.22924123)
\curveto(376.19842209,101.21923368)(376.32842196,101.19423371)(376.45841919,101.15424123)
\curveto(376.53842175,101.13423377)(376.62342166,101.11423379)(376.71341919,101.09424123)
\lineto(376.95341919,101.03424123)
\lineto(377.25341919,100.91424123)
\curveto(377.34342094,100.88423402)(377.43342085,100.84923405)(377.52341919,100.80924123)
\curveto(377.74342054,100.70923419)(377.95842033,100.57423433)(378.16841919,100.40424123)
\curveto(378.37841991,100.24423466)(378.54841974,100.06923483)(378.67841919,99.87924123)
\curveto(378.71841957,99.82923507)(378.75841953,99.76923513)(378.79841919,99.69924123)
\curveto(378.82841946,99.63923526)(378.86341942,99.57923532)(378.90341919,99.51924123)
\curveto(378.95341933,99.43923546)(378.99341929,99.34423556)(379.02341919,99.23424123)
\curveto(379.05341923,99.12423578)(379.0834192,99.01923588)(379.11341919,98.91924123)
\curveto(379.15341913,98.80923609)(379.17841911,98.6992362)(379.18841919,98.58924123)
\curveto(379.19841909,98.47923642)(379.21341907,98.36423654)(379.23341919,98.24424123)
\curveto(379.24341904,98.2042367)(379.24341904,98.15923674)(379.23341919,98.10924123)
\curveto(379.23341905,98.06923683)(379.23841905,98.02923687)(379.24841919,97.98924123)
\curveto(379.25841903,97.94923695)(379.26341902,97.89423701)(379.26341919,97.82424123)
\curveto(379.26341902,97.75423715)(379.25841903,97.7042372)(379.24841919,97.67424123)
\curveto(379.22841906,97.62423728)(379.22341906,97.57923732)(379.23341919,97.53924123)
\curveto(379.24341904,97.4992374)(379.24341904,97.46423744)(379.23341919,97.43424123)
\lineto(379.23341919,97.34424123)
\curveto(379.21341907,97.28423762)(379.19841909,97.21923768)(379.18841919,97.14924123)
\curveto(379.1884191,97.08923781)(379.1834191,97.02423788)(379.17341919,96.95424123)
\curveto(379.12341916,96.78423812)(379.07341921,96.62423828)(379.02341919,96.47424123)
\curveto(378.97341931,96.32423858)(378.90841938,96.17923872)(378.82841919,96.03924123)
\curveto(378.7884195,95.98923891)(378.75841953,95.93423897)(378.73841919,95.87424123)
\curveto(378.70841958,95.82423908)(378.67341961,95.77423913)(378.63341919,95.72424123)
\curveto(378.45341983,95.48423942)(378.23342005,95.28423962)(377.97341919,95.12424123)
\curveto(377.71342057,94.96423994)(377.42842086,94.82424008)(377.11841919,94.70424123)
\curveto(376.97842131,94.64424026)(376.83842145,94.5992403)(376.69841919,94.56924123)
\curveto(376.54842174,94.53924036)(376.39342189,94.5042404)(376.23341919,94.46424123)
\curveto(376.12342216,94.44424046)(376.01342227,94.42924047)(375.90341919,94.41924123)
\curveto(375.79342249,94.40924049)(375.6834226,94.39424051)(375.57341919,94.37424123)
\curveto(375.53342275,94.36424054)(375.49342279,94.35924054)(375.45341919,94.35924123)
\curveto(375.41342287,94.36924053)(375.37342291,94.36924053)(375.33341919,94.35924123)
\curveto(375.283423,94.34924055)(375.23342305,94.34424056)(375.18341919,94.34424123)
\lineto(375.01841919,94.34424123)
\curveto(374.96842332,94.32424058)(374.91842337,94.31924058)(374.86841919,94.32924123)
\curveto(374.80842348,94.33924056)(374.75342353,94.33924056)(374.70341919,94.32924123)
\curveto(374.66342362,94.31924058)(374.61842367,94.31924058)(374.56841919,94.32924123)
\curveto(374.51842377,94.33924056)(374.46842382,94.33424057)(374.41841919,94.31424123)
\curveto(374.34842394,94.29424061)(374.27342401,94.28924061)(374.19341919,94.29924123)
\curveto(374.10342418,94.30924059)(374.01842427,94.31424059)(373.93841919,94.31424123)
\curveto(373.84842444,94.31424059)(373.74842454,94.30924059)(373.63841919,94.29924123)
\curveto(373.51842477,94.28924061)(373.41842487,94.29424061)(373.33841919,94.31424123)
\lineto(373.05341919,94.31424123)
\lineto(372.42341919,94.35924123)
\curveto(372.32342596,94.36924053)(372.22842606,94.37924052)(372.13841919,94.38924123)
\lineto(371.83841919,94.41924123)
\curveto(371.7884265,94.43924046)(371.73842655,94.44424046)(371.68841919,94.43424123)
\curveto(371.62842666,94.43424047)(371.57342671,94.44424046)(371.52341919,94.46424123)
\curveto(371.35342693,94.51424039)(371.1884271,94.55424035)(371.02841919,94.58424123)
\curveto(370.85842743,94.61424029)(370.69842759,94.66424024)(370.54841919,94.73424123)
\curveto(370.0884282,94.92423998)(369.71342857,95.14423976)(369.42341919,95.39424123)
\curveto(369.13342915,95.65423925)(368.8884294,96.01423889)(368.68841919,96.47424123)
\curveto(368.63842965,96.6042383)(368.60342968,96.73423817)(368.58341919,96.86424123)
\curveto(368.56342972,97.0042379)(368.53842975,97.14423776)(368.50841919,97.28424123)
\curveto(368.49842979,97.35423755)(368.49342979,97.41923748)(368.49341919,97.47924123)
\curveto(368.49342979,97.53923736)(368.4884298,97.6042373)(368.47841919,97.67424123)
\curveto(368.45842983,98.5042364)(368.60842968,99.17423573)(368.92841919,99.68424123)
\curveto(369.23842905,100.19423471)(369.67842861,100.57423433)(370.24841919,100.82424123)
\curveto(370.36842792,100.87423403)(370.49342779,100.91923398)(370.62341919,100.95924123)
\curveto(370.75342753,100.9992339)(370.8884274,101.04423386)(371.02841919,101.09424123)
\curveto(371.10842718,101.11423379)(371.19342709,101.12923377)(371.28341919,101.13924123)
\lineto(371.52341919,101.19924123)
\curveto(371.63342665,101.22923367)(371.74342654,101.24423366)(371.85341919,101.24424123)
\curveto(371.96342632,101.25423365)(372.07342621,101.26923363)(372.18341919,101.28924123)
\curveto(372.23342605,101.30923359)(372.27842601,101.31423359)(372.31841919,101.30424123)
\curveto(372.35842593,101.3042336)(372.39842589,101.30923359)(372.43841919,101.31924123)
\curveto(372.4884258,101.32923357)(372.54342574,101.32923357)(372.60341919,101.31924123)
\curveto(372.65342563,101.31923358)(372.70342558,101.32423358)(372.75341919,101.33424123)
\lineto(372.88841919,101.33424123)
\curveto(372.94842534,101.35423355)(373.01842527,101.35423355)(373.09841919,101.33424123)
\curveto(373.16842512,101.32423358)(373.23342505,101.32923357)(373.29341919,101.34924123)
\curveto(373.32342496,101.35923354)(373.36342492,101.36423354)(373.41341919,101.36424123)
\lineto(373.53341919,101.36424123)
\lineto(373.99841919,101.36424123)
\moveto(376.32341919,99.81924123)
\curveto(376.00342228,99.91923498)(375.63842265,99.97923492)(375.22841919,99.99924123)
\curveto(374.81842347,100.01923488)(374.40842388,100.02923487)(373.99841919,100.02924123)
\curveto(373.56842472,100.02923487)(373.14842514,100.01923488)(372.73841919,99.99924123)
\curveto(372.32842596,99.97923492)(371.94342634,99.93423497)(371.58341919,99.86424123)
\curveto(371.22342706,99.79423511)(370.90342738,99.68423522)(370.62341919,99.53424123)
\curveto(370.33342795,99.39423551)(370.09842819,99.1992357)(369.91841919,98.94924123)
\curveto(369.80842848,98.78923611)(369.72842856,98.60923629)(369.67841919,98.40924123)
\curveto(369.61842867,98.20923669)(369.5884287,97.96423694)(369.58841919,97.67424123)
\curveto(369.60842868,97.65423725)(369.61842867,97.61923728)(369.61841919,97.56924123)
\curveto(369.60842868,97.51923738)(369.60842868,97.47923742)(369.61841919,97.44924123)
\curveto(369.63842865,97.36923753)(369.65842863,97.29423761)(369.67841919,97.22424123)
\curveto(369.6884286,97.16423774)(369.70842858,97.0992378)(369.73841919,97.02924123)
\curveto(369.85842843,96.75923814)(370.02842826,96.53923836)(370.24841919,96.36924123)
\curveto(370.45842783,96.20923869)(370.70342758,96.07423883)(370.98341919,95.96424123)
\curveto(371.09342719,95.91423899)(371.21342707,95.87423903)(371.34341919,95.84424123)
\curveto(371.46342682,95.82423908)(371.5884267,95.7992391)(371.71841919,95.76924123)
\curveto(371.76842652,95.74923915)(371.82342646,95.73923916)(371.88341919,95.73924123)
\curveto(371.93342635,95.73923916)(371.9834263,95.73423917)(372.03341919,95.72424123)
\curveto(372.12342616,95.71423919)(372.21842607,95.7042392)(372.31841919,95.69424123)
\curveto(372.40842588,95.68423922)(372.50342578,95.67423923)(372.60341919,95.66424123)
\curveto(372.6834256,95.66423924)(372.76842552,95.65923924)(372.85841919,95.64924123)
\lineto(373.09841919,95.64924123)
\lineto(373.27841919,95.64924123)
\curveto(373.30842498,95.63923926)(373.34342494,95.63423927)(373.38341919,95.63424123)
\lineto(373.51841919,95.63424123)
\lineto(373.96841919,95.63424123)
\curveto(374.04842424,95.63423927)(374.13342415,95.62923927)(374.22341919,95.61924123)
\curveto(374.30342398,95.61923928)(374.37842391,95.62923927)(374.44841919,95.64924123)
\lineto(374.71841919,95.64924123)
\curveto(374.73842355,95.64923925)(374.76842352,95.64423926)(374.80841919,95.63424123)
\curveto(374.83842345,95.63423927)(374.86342342,95.63923926)(374.88341919,95.64924123)
\curveto(374.9834233,95.65923924)(375.0834232,95.66423924)(375.18341919,95.66424123)
\curveto(375.27342301,95.67423923)(375.37342291,95.68423922)(375.48341919,95.69424123)
\curveto(375.60342268,95.72423918)(375.72842256,95.73923916)(375.85841919,95.73924123)
\curveto(375.97842231,95.74923915)(376.09342219,95.77423913)(376.20341919,95.81424123)
\curveto(376.50342178,95.89423901)(376.76842152,95.97923892)(376.99841919,96.06924123)
\curveto(377.22842106,96.16923873)(377.44342084,96.31423859)(377.64341919,96.50424123)
\curveto(377.84342044,96.71423819)(377.99342029,96.97923792)(378.09341919,97.29924123)
\curveto(378.11342017,97.33923756)(378.12342016,97.37423753)(378.12341919,97.40424123)
\curveto(378.11342017,97.44423746)(378.11842017,97.48923741)(378.13841919,97.53924123)
\curveto(378.14842014,97.57923732)(378.15842013,97.64923725)(378.16841919,97.74924123)
\curveto(378.17842011,97.85923704)(378.17342011,97.94423696)(378.15341919,98.00424123)
\curveto(378.13342015,98.07423683)(378.12342016,98.14423676)(378.12341919,98.21424123)
\curveto(378.11342017,98.28423662)(378.09842019,98.34923655)(378.07841919,98.40924123)
\curveto(378.01842027,98.60923629)(377.93342035,98.78923611)(377.82341919,98.94924123)
\curveto(377.80342048,98.97923592)(377.7834205,99.0042359)(377.76341919,99.02424123)
\lineto(377.70341919,99.08424123)
\curveto(377.6834206,99.12423578)(377.64342064,99.17423573)(377.58341919,99.23424123)
\curveto(377.44342084,99.33423557)(377.31342097,99.41923548)(377.19341919,99.48924123)
\curveto(377.07342121,99.55923534)(376.92842136,99.62923527)(376.75841919,99.69924123)
\curveto(376.6884216,99.72923517)(376.61842167,99.74923515)(376.54841919,99.75924123)
\curveto(376.47842181,99.77923512)(376.40342188,99.7992351)(376.32341919,99.81924123)
}
}
{
\newrgbcolor{curcolor}{0 0 0}
\pscustom[linestyle=none,fillstyle=solid,fillcolor=curcolor]
{
\newpath
\moveto(368.47841919,106.77385061)
\curveto(368.47842981,106.87384575)(368.4884298,106.96884566)(368.50841919,107.05885061)
\curveto(368.51842977,107.14884548)(368.54842974,107.21384541)(368.59841919,107.25385061)
\curveto(368.67842961,107.31384531)(368.7834295,107.34384528)(368.91341919,107.34385061)
\lineto(369.30341919,107.34385061)
\lineto(370.80341919,107.34385061)
\lineto(377.19341919,107.34385061)
\lineto(378.36341919,107.34385061)
\lineto(378.67841919,107.34385061)
\curveto(378.77841951,107.35384527)(378.85841943,107.33884529)(378.91841919,107.29885061)
\curveto(378.99841929,107.24884538)(379.04841924,107.17384545)(379.06841919,107.07385061)
\curveto(379.07841921,106.98384564)(379.0834192,106.87384575)(379.08341919,106.74385061)
\lineto(379.08341919,106.51885061)
\curveto(379.06341922,106.43884619)(379.04841924,106.36884626)(379.03841919,106.30885061)
\curveto(379.01841927,106.24884638)(378.97841931,106.19884643)(378.91841919,106.15885061)
\curveto(378.85841943,106.11884651)(378.7834195,106.09884653)(378.69341919,106.09885061)
\lineto(378.39341919,106.09885061)
\lineto(377.29841919,106.09885061)
\lineto(371.95841919,106.09885061)
\curveto(371.86842642,106.07884655)(371.79342649,106.06384656)(371.73341919,106.05385061)
\curveto(371.66342662,106.05384657)(371.60342668,106.0238466)(371.55341919,105.96385061)
\curveto(371.50342678,105.89384673)(371.47842681,105.80384682)(371.47841919,105.69385061)
\curveto(371.46842682,105.59384703)(371.46342682,105.48384714)(371.46341919,105.36385061)
\lineto(371.46341919,104.22385061)
\lineto(371.46341919,103.72885061)
\curveto(371.45342683,103.56884906)(371.39342689,103.45884917)(371.28341919,103.39885061)
\curveto(371.25342703,103.37884925)(371.22342706,103.36884926)(371.19341919,103.36885061)
\curveto(371.15342713,103.36884926)(371.10842718,103.36384926)(371.05841919,103.35385061)
\curveto(370.93842735,103.33384929)(370.82842746,103.33884929)(370.72841919,103.36885061)
\curveto(370.62842766,103.40884922)(370.55842773,103.46384916)(370.51841919,103.53385061)
\curveto(370.46842782,103.61384901)(370.44342784,103.73384889)(370.44341919,103.89385061)
\curveto(370.44342784,104.05384857)(370.42842786,104.18884844)(370.39841919,104.29885061)
\curveto(370.3884279,104.34884828)(370.3834279,104.40384822)(370.38341919,104.46385061)
\curveto(370.37342791,104.5238481)(370.35842793,104.58384804)(370.33841919,104.64385061)
\curveto(370.288428,104.79384783)(370.23842805,104.93884769)(370.18841919,105.07885061)
\curveto(370.12842816,105.21884741)(370.05842823,105.35384727)(369.97841919,105.48385061)
\curveto(369.8884284,105.623847)(369.7834285,105.74384688)(369.66341919,105.84385061)
\curveto(369.54342874,105.94384668)(369.41342887,106.03884659)(369.27341919,106.12885061)
\curveto(369.17342911,106.18884644)(369.06342922,106.23384639)(368.94341919,106.26385061)
\curveto(368.82342946,106.30384632)(368.71842957,106.35384627)(368.62841919,106.41385061)
\curveto(368.56842972,106.46384616)(368.52842976,106.53384609)(368.50841919,106.62385061)
\curveto(368.49842979,106.64384598)(368.49342979,106.66884596)(368.49341919,106.69885061)
\curveto(368.49342979,106.7288459)(368.4884298,106.75384587)(368.47841919,106.77385061)
}
}
{
\newrgbcolor{curcolor}{0 0 0}
\pscustom[linestyle=none,fillstyle=solid,fillcolor=curcolor]
{
\newpath
\moveto(368.47841919,115.12345998)
\curveto(368.47842981,115.22345513)(368.4884298,115.31845503)(368.50841919,115.40845998)
\curveto(368.51842977,115.49845485)(368.54842974,115.56345479)(368.59841919,115.60345998)
\curveto(368.67842961,115.66345469)(368.7834295,115.69345466)(368.91341919,115.69345998)
\lineto(369.30341919,115.69345998)
\lineto(370.80341919,115.69345998)
\lineto(377.19341919,115.69345998)
\lineto(378.36341919,115.69345998)
\lineto(378.67841919,115.69345998)
\curveto(378.77841951,115.70345465)(378.85841943,115.68845466)(378.91841919,115.64845998)
\curveto(378.99841929,115.59845475)(379.04841924,115.52345483)(379.06841919,115.42345998)
\curveto(379.07841921,115.33345502)(379.0834192,115.22345513)(379.08341919,115.09345998)
\lineto(379.08341919,114.86845998)
\curveto(379.06341922,114.78845556)(379.04841924,114.71845563)(379.03841919,114.65845998)
\curveto(379.01841927,114.59845575)(378.97841931,114.5484558)(378.91841919,114.50845998)
\curveto(378.85841943,114.46845588)(378.7834195,114.4484559)(378.69341919,114.44845998)
\lineto(378.39341919,114.44845998)
\lineto(377.29841919,114.44845998)
\lineto(371.95841919,114.44845998)
\curveto(371.86842642,114.42845592)(371.79342649,114.41345594)(371.73341919,114.40345998)
\curveto(371.66342662,114.40345595)(371.60342668,114.37345598)(371.55341919,114.31345998)
\curveto(371.50342678,114.24345611)(371.47842681,114.1534562)(371.47841919,114.04345998)
\curveto(371.46842682,113.94345641)(371.46342682,113.83345652)(371.46341919,113.71345998)
\lineto(371.46341919,112.57345998)
\lineto(371.46341919,112.07845998)
\curveto(371.45342683,111.91845843)(371.39342689,111.80845854)(371.28341919,111.74845998)
\curveto(371.25342703,111.72845862)(371.22342706,111.71845863)(371.19341919,111.71845998)
\curveto(371.15342713,111.71845863)(371.10842718,111.71345864)(371.05841919,111.70345998)
\curveto(370.93842735,111.68345867)(370.82842746,111.68845866)(370.72841919,111.71845998)
\curveto(370.62842766,111.75845859)(370.55842773,111.81345854)(370.51841919,111.88345998)
\curveto(370.46842782,111.96345839)(370.44342784,112.08345827)(370.44341919,112.24345998)
\curveto(370.44342784,112.40345795)(370.42842786,112.53845781)(370.39841919,112.64845998)
\curveto(370.3884279,112.69845765)(370.3834279,112.7534576)(370.38341919,112.81345998)
\curveto(370.37342791,112.87345748)(370.35842793,112.93345742)(370.33841919,112.99345998)
\curveto(370.288428,113.14345721)(370.23842805,113.28845706)(370.18841919,113.42845998)
\curveto(370.12842816,113.56845678)(370.05842823,113.70345665)(369.97841919,113.83345998)
\curveto(369.8884284,113.97345638)(369.7834285,114.09345626)(369.66341919,114.19345998)
\curveto(369.54342874,114.29345606)(369.41342887,114.38845596)(369.27341919,114.47845998)
\curveto(369.17342911,114.53845581)(369.06342922,114.58345577)(368.94341919,114.61345998)
\curveto(368.82342946,114.6534557)(368.71842957,114.70345565)(368.62841919,114.76345998)
\curveto(368.56842972,114.81345554)(368.52842976,114.88345547)(368.50841919,114.97345998)
\curveto(368.49842979,114.99345536)(368.49342979,115.01845533)(368.49341919,115.04845998)
\curveto(368.49342979,115.07845527)(368.4884298,115.10345525)(368.47841919,115.12345998)
}
}
{
\newrgbcolor{curcolor}{0 0 0}
\pscustom[linestyle=none,fillstyle=solid,fillcolor=curcolor]
{
\newpath
\moveto(399.21476562,42.02236623)
\curveto(399.26476637,42.04235669)(399.32476631,42.06735666)(399.39476562,42.09736623)
\curveto(399.46476617,42.1273566)(399.5397661,42.14735658)(399.61976563,42.15736623)
\curveto(399.68976594,42.17735655)(399.75976588,42.17735655)(399.82976562,42.15736623)
\curveto(399.88976574,42.14735658)(399.9347657,42.10735662)(399.96476562,42.03736623)
\curveto(399.98476565,41.98735674)(399.99476564,41.9273568)(399.99476563,41.85736623)
\lineto(399.99476563,41.64736623)
\lineto(399.99476563,41.19736623)
\curveto(399.99476564,41.04735768)(399.96976567,40.9273578)(399.91976563,40.83736623)
\curveto(399.85976578,40.73735799)(399.75476588,40.66235807)(399.60476563,40.61236623)
\curveto(399.45476618,40.57235816)(399.31976631,40.5273582)(399.19976562,40.47736623)
\curveto(398.9397667,40.36735836)(398.66976697,40.26735846)(398.38976562,40.17736623)
\curveto(398.10976752,40.08735864)(397.8347678,39.98735874)(397.56476563,39.87736623)
\curveto(397.47476816,39.84735888)(397.38976825,39.81735891)(397.30976563,39.78736623)
\curveto(397.22976841,39.76735896)(397.15476848,39.73735899)(397.08476562,39.69736623)
\curveto(397.01476862,39.66735906)(396.95476868,39.62235911)(396.90476562,39.56236623)
\curveto(396.85476878,39.50235923)(396.81476882,39.42235931)(396.78476562,39.32236623)
\curveto(396.76476887,39.27235946)(396.75976888,39.21235952)(396.76976562,39.14236623)
\lineto(396.76976562,38.94736623)
\lineto(396.76976562,36.11236623)
\lineto(396.76976562,35.81236623)
\curveto(396.75976888,35.70236303)(396.75976888,35.59736313)(396.76976562,35.49736623)
\curveto(396.77976885,35.39736333)(396.79476884,35.30236343)(396.81476563,35.21236623)
\curveto(396.8347688,35.1323636)(396.87476876,35.07236366)(396.93476563,35.03236623)
\curveto(397.0347686,34.95236378)(397.14976848,34.89236384)(397.27976562,34.85236623)
\curveto(397.39976824,34.82236391)(397.52476811,34.78236395)(397.65476562,34.73236623)
\curveto(397.88476775,34.6323641)(398.12476751,34.53736419)(398.37476563,34.44736623)
\curveto(398.62476701,34.36736436)(398.86476677,34.27736445)(399.09476562,34.17736623)
\curveto(399.15476648,34.15736457)(399.22476641,34.1323646)(399.30476563,34.10236623)
\curveto(399.37476626,34.08236465)(399.44976618,34.05736467)(399.52976562,34.02736623)
\curveto(399.60976602,33.99736473)(399.68476595,33.96236477)(399.75476563,33.92236623)
\curveto(399.81476582,33.89236484)(399.85976578,33.85736487)(399.88976562,33.81736623)
\curveto(399.94976568,33.73736499)(399.98476565,33.6273651)(399.99476563,33.48736623)
\lineto(399.99476563,33.06736623)
\lineto(399.99476563,32.82736623)
\curveto(399.98476565,32.75736597)(399.95976568,32.69736603)(399.91976563,32.64736623)
\curveto(399.88976574,32.59736613)(399.84476579,32.56736616)(399.78476562,32.55736623)
\curveto(399.72476591,32.55736617)(399.66476597,32.56236617)(399.60476563,32.57236623)
\curveto(399.5347661,32.59236614)(399.46976617,32.61236612)(399.40976562,32.63236623)
\curveto(399.3397663,32.66236607)(399.28976634,32.68736604)(399.25976562,32.70736623)
\curveto(398.9397667,32.84736588)(398.62476701,32.97236576)(398.31476563,33.08236623)
\curveto(397.99476764,33.19236554)(397.67476796,33.31236542)(397.35476563,33.44236623)
\curveto(397.1347685,33.5323652)(396.91976871,33.61736511)(396.70976562,33.69736623)
\curveto(396.48976915,33.77736495)(396.26976937,33.86236487)(396.04976563,33.95236623)
\curveto(395.32977031,34.25236448)(394.60477103,34.53736419)(393.87476563,34.80736623)
\curveto(393.1347725,35.07736365)(392.39977323,35.36236337)(391.66976563,35.66236623)
\curveto(391.40977422,35.77236296)(391.14477449,35.87236286)(390.87476563,35.96236623)
\curveto(390.60477503,36.06236267)(390.33977529,36.16736256)(390.07976562,36.27736623)
\curveto(389.96977566,36.3273624)(389.84977579,36.37236236)(389.71976562,36.41236623)
\curveto(389.57977606,36.46236227)(389.47977616,36.5323622)(389.41976563,36.62236623)
\curveto(389.37977626,36.66236207)(389.34977629,36.727362)(389.32976562,36.81736623)
\curveto(389.31977632,36.83736189)(389.31977632,36.85736187)(389.32976562,36.87736623)
\curveto(389.3297763,36.90736182)(389.32477631,36.9323618)(389.31476563,36.95236623)
\curveto(389.31477632,37.1323616)(389.31477632,37.34236139)(389.31476563,37.58236623)
\curveto(389.30477633,37.82236091)(389.33977629,37.99736073)(389.41976563,38.10736623)
\curveto(389.47977616,38.18736054)(389.57977606,38.24736048)(389.71976562,38.28736623)
\curveto(389.84977579,38.33736039)(389.96977566,38.38736034)(390.07976562,38.43736623)
\curveto(390.30977532,38.53736019)(390.53977509,38.6273601)(390.76976562,38.70736623)
\curveto(390.99977463,38.78735994)(391.22977441,38.87735985)(391.45976562,38.97736623)
\curveto(391.65977398,39.05735967)(391.86477377,39.1323596)(392.07476562,39.20236623)
\curveto(392.28477335,39.28235945)(392.48977315,39.36735936)(392.68976563,39.45736623)
\curveto(393.41977222,39.75735897)(394.15977148,40.04235869)(394.90976562,40.31236623)
\curveto(395.64976998,40.59235814)(396.38476925,40.88735784)(397.11476563,41.19736623)
\curveto(397.20476843,41.23735749)(397.28976835,41.26735746)(397.36976563,41.28736623)
\curveto(397.44976818,41.31735741)(397.5347681,41.34735738)(397.62476563,41.37736623)
\curveto(397.88476775,41.48735724)(398.14976748,41.59235714)(398.41976563,41.69236623)
\curveto(398.68976694,41.80235693)(398.95476668,41.91235682)(399.21476562,42.02236623)
\moveto(395.56976563,38.81236623)
\curveto(395.53977009,38.90235983)(395.48977015,38.95735977)(395.41976563,38.97736623)
\curveto(395.34977028,39.00735972)(395.27477036,39.01235972)(395.19476563,38.99236623)
\curveto(395.10477053,38.98235975)(395.01977062,38.95735977)(394.93976563,38.91736623)
\curveto(394.84977078,38.88735984)(394.77477086,38.85735987)(394.71476562,38.82736623)
\curveto(394.67477096,38.80735992)(394.63977099,38.79735993)(394.60976563,38.79736623)
\curveto(394.57977105,38.79735993)(394.54477109,38.78735994)(394.50476563,38.76736623)
\lineto(394.26476562,38.67736623)
\curveto(394.17477146,38.65736007)(394.08477155,38.6273601)(393.99476563,38.58736623)
\curveto(393.634772,38.43736029)(393.26977236,38.30236043)(392.89976562,38.18236623)
\curveto(392.51977312,38.07236066)(392.14977349,37.94236079)(391.78976562,37.79236623)
\curveto(391.67977395,37.74236099)(391.56977406,37.69736103)(391.45976562,37.65736623)
\curveto(391.34977429,37.6273611)(391.24477439,37.58736114)(391.14476562,37.53736623)
\curveto(391.09477454,37.51736121)(391.04977459,37.49236124)(391.00976562,37.46236623)
\curveto(390.95977468,37.44236129)(390.9347747,37.39236134)(390.93476563,37.31236623)
\curveto(390.95477468,37.29236144)(390.96977466,37.27236146)(390.97976563,37.25236623)
\curveto(390.98977465,37.2323615)(391.00477463,37.21236152)(391.02476562,37.19236623)
\curveto(391.07477456,37.15236158)(391.12977451,37.12236161)(391.18976563,37.10236623)
\curveto(391.23977439,37.08236165)(391.29477434,37.06236167)(391.35476563,37.04236623)
\curveto(391.46477417,36.99236174)(391.57477406,36.95236178)(391.68476563,36.92236623)
\curveto(391.79477384,36.89236184)(391.90477373,36.85236188)(392.01476562,36.80236623)
\curveto(392.40477323,36.6323621)(392.79977283,36.48236225)(393.19976562,36.35236623)
\curveto(393.59977204,36.2323625)(393.98977165,36.09236264)(394.36976563,35.93236623)
\lineto(394.51976562,35.87236623)
\curveto(394.56977106,35.86236287)(394.61977102,35.84736288)(394.66976563,35.82736623)
\lineto(394.90976562,35.73736623)
\curveto(394.98977065,35.70736302)(395.06977056,35.68236305)(395.14976562,35.66236623)
\curveto(395.19977044,35.64236309)(395.25477038,35.6323631)(395.31476563,35.63236623)
\curveto(395.37477026,35.64236309)(395.42477021,35.65736307)(395.46476562,35.67736623)
\curveto(395.54477009,35.727363)(395.58977005,35.8323629)(395.59976562,35.99236623)
\lineto(395.59976562,36.44236623)
\lineto(395.59976562,38.04736623)
\curveto(395.59977004,38.15736057)(395.60477003,38.29236044)(395.61476563,38.45236623)
\curveto(395.61477002,38.61236012)(395.59977004,38.73236)(395.56976563,38.81236623)
}
}
{
\newrgbcolor{curcolor}{0 0 0}
\pscustom[linestyle=none,fillstyle=solid,fillcolor=curcolor]
{
\newpath
\moveto(395.95976562,50.56392873)
\curveto(396.00976962,50.57392038)(396.07976955,50.57892038)(396.16976563,50.57892873)
\curveto(396.24976938,50.57892038)(396.31476932,50.57392038)(396.36476563,50.56392873)
\curveto(396.40476923,50.56392039)(396.44476919,50.5589204)(396.48476563,50.54892873)
\lineto(396.60476563,50.54892873)
\curveto(396.68476895,50.52892043)(396.76476887,50.51892044)(396.84476562,50.51892873)
\curveto(396.92476871,50.51892044)(397.00476863,50.50892045)(397.08476562,50.48892873)
\curveto(397.12476851,50.47892048)(397.16476847,50.47392048)(397.20476562,50.47392873)
\curveto(397.2347684,50.47392048)(397.26976837,50.46892049)(397.30976563,50.45892873)
\curveto(397.41976821,50.42892053)(397.52476811,50.39892056)(397.62476563,50.36892873)
\curveto(397.72476791,50.34892061)(397.82476781,50.31892064)(397.92476563,50.27892873)
\curveto(398.27476736,50.13892082)(398.58976704,49.96892099)(398.86976563,49.76892873)
\curveto(399.14976648,49.56892139)(399.38976624,49.31892164)(399.58976562,49.01892873)
\curveto(399.68976594,48.86892209)(399.77476586,48.72392223)(399.84476562,48.58392873)
\curveto(399.89476574,48.47392248)(399.9347657,48.36392259)(399.96476562,48.25392873)
\curveto(399.99476564,48.1539228)(400.02476561,48.04892291)(400.05476563,47.93892873)
\curveto(400.07476556,47.86892309)(400.08476555,47.80392315)(400.08476562,47.74392873)
\curveto(400.09476554,47.68392327)(400.10976553,47.62392333)(400.12976563,47.56392873)
\lineto(400.12976563,47.41392873)
\curveto(400.14976548,47.36392359)(400.15976547,47.28892367)(400.15976562,47.18892873)
\curveto(400.16976547,47.08892387)(400.16476547,47.00892395)(400.14476562,46.94892873)
\lineto(400.14476562,46.79892873)
\curveto(400.1347655,46.7589242)(400.12976551,46.71392424)(400.12976563,46.66392873)
\curveto(400.12976551,46.62392433)(400.12476551,46.57892438)(400.11476563,46.52892873)
\curveto(400.07476556,46.37892458)(400.0397656,46.22892473)(400.00976562,46.07892873)
\curveto(399.97976565,45.93892502)(399.9347657,45.79892516)(399.87476563,45.65892873)
\curveto(399.79476584,45.4589255)(399.69476594,45.27892568)(399.57476562,45.11892873)
\lineto(399.42476563,44.93892873)
\curveto(399.36476627,44.87892608)(399.32476631,44.80892615)(399.30476563,44.72892873)
\curveto(399.29476634,44.66892629)(399.30976632,44.61892634)(399.34976562,44.57892873)
\curveto(399.37976625,44.54892641)(399.42476621,44.52392643)(399.48476563,44.50392873)
\curveto(399.54476609,44.49392646)(399.60976602,44.48392647)(399.67976563,44.47392873)
\curveto(399.7397659,44.47392648)(399.78476585,44.46392649)(399.81476563,44.44392873)
\curveto(399.86476577,44.40392655)(399.90976572,44.3589266)(399.94976562,44.30892873)
\curveto(399.96976567,44.2589267)(399.98476565,44.18892677)(399.99476563,44.09892873)
\lineto(399.99476563,43.82892873)
\curveto(399.99476564,43.73892722)(399.98976564,43.6539273)(399.97976563,43.57392873)
\curveto(399.95976568,43.49392746)(399.9397657,43.43392752)(399.91976563,43.39392873)
\curveto(399.89976574,43.37392758)(399.87476576,43.3539276)(399.84476562,43.33392873)
\lineto(399.75476563,43.27392873)
\curveto(399.67476596,43.24392771)(399.55476608,43.22892773)(399.39476562,43.22892873)
\curveto(399.2347664,43.23892772)(399.09976654,43.24392771)(398.98976563,43.24392873)
\lineto(390.18476563,43.24392873)
\curveto(390.06477557,43.24392771)(389.93977569,43.23892772)(389.80976563,43.22892873)
\curveto(389.66977596,43.22892773)(389.55977608,43.2539277)(389.47976563,43.30392873)
\curveto(389.41977622,43.34392761)(389.36977626,43.40892755)(389.32976562,43.49892873)
\curveto(389.3297763,43.51892744)(389.3297763,43.54392741)(389.32976562,43.57392873)
\curveto(389.31977632,43.60392735)(389.31477632,43.62892733)(389.31476563,43.64892873)
\curveto(389.30477633,43.78892717)(389.30477633,43.93392702)(389.31476563,44.08392873)
\curveto(389.31477632,44.24392671)(389.35477628,44.3539266)(389.43476563,44.41392873)
\curveto(389.51477612,44.46392649)(389.629776,44.48892647)(389.77976562,44.48892873)
\lineto(390.18476563,44.48892873)
\lineto(391.93976563,44.48892873)
\lineto(392.19476563,44.48892873)
\lineto(392.47976563,44.48892873)
\curveto(392.56977306,44.49892646)(392.65477298,44.50892645)(392.73476563,44.51892873)
\curveto(392.80477283,44.53892642)(392.85477278,44.56892639)(392.88476562,44.60892873)
\curveto(392.91477272,44.64892631)(392.91977272,44.69392626)(392.89976562,44.74392873)
\curveto(392.87977275,44.79392616)(392.85977278,44.83392612)(392.83976562,44.86392873)
\curveto(392.79977283,44.91392604)(392.75977288,44.958926)(392.71976562,44.99892873)
\lineto(392.59976562,45.14892873)
\curveto(392.54977309,45.21892574)(392.50477313,45.28892567)(392.46476562,45.35892873)
\lineto(392.34476562,45.59892873)
\curveto(392.25477338,45.77892518)(392.18977345,45.99392496)(392.14976562,46.24392873)
\curveto(392.10977352,46.49392446)(392.08977355,46.74892421)(392.08976562,47.00892873)
\curveto(392.08977355,47.26892369)(392.11477352,47.52392343)(392.16476562,47.77392873)
\curveto(392.20477343,48.02392293)(392.26477337,48.24392271)(392.34476562,48.43392873)
\curveto(392.51477312,48.83392212)(392.74977289,49.17892178)(393.04976563,49.46892873)
\curveto(393.34977228,49.7589212)(393.69977194,49.98892097)(394.09976562,50.15892873)
\curveto(394.20977142,50.20892075)(394.31977132,50.24892071)(394.42976563,50.27892873)
\curveto(394.52977111,50.31892064)(394.634771,50.3589206)(394.74476563,50.39892873)
\curveto(394.85477078,50.42892053)(394.96977066,50.44892051)(395.08976562,50.45892873)
\lineto(395.41976563,50.51892873)
\curveto(395.44977018,50.52892043)(395.48477015,50.53392042)(395.52476562,50.53392873)
\curveto(395.55477008,50.53392042)(395.58477005,50.53892042)(395.61476563,50.54892873)
\curveto(395.67476996,50.56892039)(395.7347699,50.56892039)(395.79476563,50.54892873)
\curveto(395.84476979,50.53892042)(395.89976974,50.54392041)(395.95976562,50.56392873)
\moveto(396.34976562,49.22892873)
\curveto(396.29976934,49.24892171)(396.23976939,49.2539217)(396.16976563,49.24392873)
\curveto(396.09976954,49.23392172)(396.0347696,49.22892173)(395.97476562,49.22892873)
\curveto(395.80476983,49.22892173)(395.64476999,49.21892174)(395.49476563,49.19892873)
\curveto(395.34477029,49.18892177)(395.20977042,49.1589218)(395.08976562,49.10892873)
\curveto(394.98977065,49.07892188)(394.89977074,49.0539219)(394.81976563,49.03392873)
\curveto(394.73977089,49.01392194)(394.65977098,48.98392197)(394.57976562,48.94392873)
\curveto(394.32977131,48.83392212)(394.09977154,48.68392227)(393.88976562,48.49392873)
\curveto(393.66977196,48.30392265)(393.50477213,48.08392287)(393.39476562,47.83392873)
\curveto(393.36477227,47.7539232)(393.33977229,47.67392328)(393.31976563,47.59392873)
\curveto(393.28977235,47.52392343)(393.26477237,47.44892351)(393.24476563,47.36892873)
\curveto(393.21477242,47.2589237)(393.19977243,47.14892381)(393.19976562,47.03892873)
\curveto(393.18977245,46.92892403)(393.18477245,46.80892415)(393.18476563,46.67892873)
\curveto(393.19477244,46.62892433)(393.20477243,46.58392437)(393.21476562,46.54392873)
\lineto(393.21476562,46.40892873)
\lineto(393.27476562,46.13892873)
\curveto(393.29477234,46.0589249)(393.32477231,45.97892498)(393.36476563,45.89892873)
\curveto(393.50477213,45.5589254)(393.71477192,45.28892567)(393.99476563,45.08892873)
\curveto(394.26477137,44.88892607)(394.58477105,44.72892623)(394.95476562,44.60892873)
\curveto(395.06477057,44.56892639)(395.17477046,44.54392641)(395.28476562,44.53392873)
\curveto(395.39477024,44.52392643)(395.50977012,44.50392645)(395.62976563,44.47392873)
\curveto(395.67976995,44.46392649)(395.72476991,44.46392649)(395.76476562,44.47392873)
\curveto(395.80476983,44.48392647)(395.84976978,44.47892648)(395.89976562,44.45892873)
\curveto(395.94976968,44.44892651)(396.02476961,44.44392651)(396.12476563,44.44392873)
\curveto(396.21476942,44.44392651)(396.28476935,44.44892651)(396.33476562,44.45892873)
\lineto(396.45476562,44.45892873)
\curveto(396.49476914,44.46892649)(396.5347691,44.47392648)(396.57476562,44.47392873)
\curveto(396.61476902,44.47392648)(396.64976898,44.47892648)(396.67976563,44.48892873)
\curveto(396.70976892,44.49892646)(396.74476889,44.50392645)(396.78476562,44.50392873)
\curveto(396.81476882,44.50392645)(396.84476879,44.50892645)(396.87476563,44.51892873)
\curveto(396.95476868,44.53892642)(397.0347686,44.5539264)(397.11476563,44.56392873)
\lineto(397.35476563,44.62392873)
\curveto(397.69476794,44.73392622)(397.98476765,44.88392607)(398.22476562,45.07392873)
\curveto(398.46476717,45.27392568)(398.66476697,45.51892544)(398.82476562,45.80892873)
\curveto(398.87476676,45.89892506)(398.91476672,45.99392496)(398.94476563,46.09392873)
\curveto(398.96476667,46.19392476)(398.98976664,46.29892466)(399.01976562,46.40892873)
\curveto(399.0397666,46.4589245)(399.04976658,46.50392445)(399.04976563,46.54392873)
\curveto(399.0397666,46.59392436)(399.0397666,46.64392431)(399.04976563,46.69392873)
\curveto(399.05976658,46.73392422)(399.06476657,46.77892418)(399.06476563,46.82892873)
\lineto(399.06476563,46.96392873)
\lineto(399.06476563,47.09892873)
\curveto(399.05476658,47.13892382)(399.04976658,47.17392378)(399.04976563,47.20392873)
\curveto(399.04976658,47.23392372)(399.04476659,47.26892369)(399.03476562,47.30892873)
\curveto(399.01476662,47.38892357)(398.99976664,47.46392349)(398.98976563,47.53392873)
\curveto(398.96976667,47.60392335)(398.94476669,47.67892328)(398.91476562,47.75892873)
\curveto(398.78476685,48.06892289)(398.61476702,48.31892264)(398.40476562,48.50892873)
\curveto(398.18476745,48.69892226)(397.91976771,48.8589221)(397.60976563,48.98892873)
\curveto(397.46976817,49.03892192)(397.32976831,49.07392188)(397.18976563,49.09392873)
\curveto(397.03976859,49.12392183)(396.88976875,49.1589218)(396.73976563,49.19892873)
\curveto(396.68976895,49.21892174)(396.64476899,49.22392173)(396.60476563,49.21392873)
\curveto(396.55476908,49.21392174)(396.50476913,49.21892174)(396.45476562,49.22892873)
\lineto(396.34976562,49.22892873)
}
}
{
\newrgbcolor{curcolor}{0 0 0}
\pscustom[linestyle=none,fillstyle=solid,fillcolor=curcolor]
{
\newpath
\moveto(392.08976562,55.69017873)
\curveto(392.08977355,55.92017394)(392.14977349,56.05017381)(392.26976562,56.08017873)
\curveto(392.37977325,56.11017375)(392.54477309,56.12517374)(392.76476562,56.12517873)
\lineto(393.04976563,56.12517873)
\curveto(393.13977249,56.12517374)(393.21477242,56.10017376)(393.27476562,56.05017873)
\curveto(393.35477228,55.99017387)(393.39977224,55.90517396)(393.40976562,55.79517873)
\curveto(393.40977222,55.68517418)(393.42477221,55.57517429)(393.45476562,55.46517873)
\curveto(393.48477215,55.32517454)(393.51477212,55.19017467)(393.54476563,55.06017873)
\curveto(393.57477206,54.94017492)(393.61477202,54.82517504)(393.66476562,54.71517873)
\curveto(393.79477184,54.42517544)(393.97477166,54.19017567)(394.20476562,54.01017873)
\curveto(394.42477121,53.83017603)(394.67977095,53.67517619)(394.96976562,53.54517873)
\curveto(395.07977055,53.50517636)(395.19477044,53.47517639)(395.31476563,53.45517873)
\curveto(395.42477021,53.43517643)(395.53977009,53.41017645)(395.65976562,53.38017873)
\curveto(395.70976992,53.37017649)(395.75976988,53.3651765)(395.80976563,53.36517873)
\curveto(395.85976978,53.37517649)(395.90976972,53.37517649)(395.95976562,53.36517873)
\curveto(396.07976955,53.33517653)(396.21976941,53.32017654)(396.37976563,53.32017873)
\curveto(396.52976911,53.33017653)(396.67476896,53.33517653)(396.81476563,53.33517873)
\lineto(398.65976562,53.33517873)
\lineto(399.00476563,53.33517873)
\curveto(399.12476651,53.33517653)(399.2397664,53.33017653)(399.34976562,53.32017873)
\curveto(399.45976618,53.31017655)(399.55476608,53.30517656)(399.63476562,53.30517873)
\curveto(399.71476592,53.31517655)(399.78476585,53.29517657)(399.84476562,53.24517873)
\curveto(399.91476572,53.19517667)(399.95476568,53.11517675)(399.96476562,53.00517873)
\curveto(399.97476566,52.90517696)(399.97976565,52.79517707)(399.97976563,52.67517873)
\lineto(399.97976563,52.40517873)
\curveto(399.95976568,52.35517751)(399.94476569,52.30517756)(399.93476563,52.25517873)
\curveto(399.91476572,52.21517765)(399.88976574,52.18517768)(399.85976563,52.16517873)
\curveto(399.78976584,52.11517775)(399.70476593,52.08517778)(399.60476563,52.07517873)
\lineto(399.27476562,52.07517873)
\lineto(398.11976563,52.07517873)
\lineto(393.96476562,52.07517873)
\lineto(392.92976563,52.07517873)
\lineto(392.62976563,52.07517873)
\curveto(392.52977311,52.08517778)(392.44477319,52.11517775)(392.37476563,52.16517873)
\curveto(392.3347733,52.19517767)(392.30477333,52.24517762)(392.28476562,52.31517873)
\curveto(392.26477337,52.39517747)(392.25477338,52.48017738)(392.25476563,52.57017873)
\curveto(392.24477339,52.6601772)(392.24477339,52.75017711)(392.25476563,52.84017873)
\curveto(392.26477337,52.93017693)(392.27977335,53.00017686)(392.29976563,53.05017873)
\curveto(392.32977331,53.13017673)(392.38977325,53.18017668)(392.47976563,53.20017873)
\curveto(392.55977308,53.23017663)(392.64977299,53.24517662)(392.74976563,53.24517873)
\lineto(393.04976563,53.24517873)
\curveto(393.14977248,53.24517662)(393.23977239,53.2651766)(393.31976563,53.30517873)
\curveto(393.33977229,53.31517655)(393.35477228,53.32517654)(393.36476563,53.33517873)
\lineto(393.40976562,53.38017873)
\curveto(393.40977222,53.49017637)(393.36477227,53.58017628)(393.27476562,53.65017873)
\curveto(393.17477246,53.72017614)(393.09477254,53.78017608)(393.03476562,53.83017873)
\lineto(392.94476563,53.92017873)
\curveto(392.8347728,54.01017585)(392.71977292,54.13517573)(392.59976562,54.29517873)
\curveto(392.47977315,54.45517541)(392.38977325,54.60517526)(392.32976562,54.74517873)
\curveto(392.27977335,54.83517503)(392.24477339,54.93017493)(392.22476562,55.03017873)
\curveto(392.19477344,55.13017473)(392.16477347,55.23517463)(392.13476562,55.34517873)
\curveto(392.12477351,55.40517446)(392.11977352,55.4651744)(392.11976563,55.52517873)
\curveto(392.10977352,55.58517428)(392.09977353,55.64017422)(392.08976562,55.69017873)
}
}
{
\newrgbcolor{curcolor}{0 0 0}
\pscustom[linestyle=none,fillstyle=solid,fillcolor=curcolor]
{
}
}
{
\newrgbcolor{curcolor}{0 0 0}
\pscustom[linestyle=none,fillstyle=solid,fillcolor=curcolor]
{
\newpath
\moveto(389.38976562,65.05510061)
\curveto(389.38977625,65.15509575)(389.39977623,65.25009566)(389.41976563,65.34010061)
\curveto(389.4297762,65.43009548)(389.45977618,65.49509541)(389.50976562,65.53510061)
\curveto(389.58977605,65.59509531)(389.69477594,65.62509528)(389.82476562,65.62510061)
\lineto(390.21476562,65.62510061)
\lineto(391.71476562,65.62510061)
\lineto(398.10476563,65.62510061)
\lineto(399.27476562,65.62510061)
\lineto(399.58976562,65.62510061)
\curveto(399.68976594,65.63509527)(399.76976587,65.62009529)(399.82976562,65.58010061)
\curveto(399.90976572,65.53009538)(399.95976568,65.45509545)(399.97976563,65.35510061)
\curveto(399.98976564,65.26509564)(399.99476564,65.15509575)(399.99476563,65.02510061)
\lineto(399.99476563,64.80010061)
\curveto(399.97476566,64.72009619)(399.95976568,64.65009626)(399.94976562,64.59010061)
\curveto(399.92976571,64.53009638)(399.88976574,64.48009643)(399.82976562,64.44010061)
\curveto(399.76976587,64.40009651)(399.69476594,64.38009653)(399.60476563,64.38010061)
\lineto(399.30476563,64.38010061)
\lineto(398.20976562,64.38010061)
\lineto(392.86976563,64.38010061)
\curveto(392.77977285,64.36009655)(392.70477293,64.34509656)(392.64476562,64.33510061)
\curveto(392.57477306,64.33509657)(392.51477312,64.3050966)(392.46476562,64.24510061)
\curveto(392.41477322,64.17509673)(392.38977325,64.08509682)(392.38976562,63.97510061)
\curveto(392.37977325,63.87509703)(392.37477326,63.76509714)(392.37476563,63.64510061)
\lineto(392.37476563,62.50510061)
\lineto(392.37476563,62.01010061)
\curveto(392.36477327,61.85009906)(392.30477333,61.74009917)(392.19476563,61.68010061)
\curveto(392.16477347,61.66009925)(392.1347735,61.65009926)(392.10476563,61.65010061)
\curveto(392.06477357,61.65009926)(392.01977362,61.64509926)(391.96976562,61.63510061)
\curveto(391.84977379,61.61509929)(391.73977389,61.62009929)(391.63976562,61.65010061)
\curveto(391.53977409,61.69009922)(391.46977416,61.74509916)(391.42976563,61.81510061)
\curveto(391.37977426,61.89509901)(391.35477428,62.01509889)(391.35476563,62.17510061)
\curveto(391.35477428,62.33509857)(391.33977429,62.47009844)(391.30976563,62.58010061)
\curveto(391.29977433,62.63009828)(391.29477434,62.68509822)(391.29476563,62.74510061)
\curveto(391.28477435,62.8050981)(391.26977436,62.86509804)(391.24976563,62.92510061)
\curveto(391.19977443,63.07509783)(391.14977449,63.22009769)(391.09976562,63.36010061)
\curveto(391.03977459,63.50009741)(390.96977466,63.63509727)(390.88976562,63.76510061)
\curveto(390.79977483,63.905097)(390.69477494,64.02509688)(390.57476562,64.12510061)
\curveto(390.45477518,64.22509668)(390.32477531,64.32009659)(390.18476563,64.41010061)
\curveto(390.08477555,64.47009644)(389.97477566,64.51509639)(389.85476563,64.54510061)
\curveto(389.7347759,64.58509632)(389.629776,64.63509627)(389.53976562,64.69510061)
\curveto(389.47977616,64.74509616)(389.43977619,64.81509609)(389.41976563,64.90510061)
\curveto(389.40977622,64.92509598)(389.40477623,64.95009596)(389.40476562,64.98010061)
\curveto(389.40477623,65.0100959)(389.39977623,65.03509587)(389.38976562,65.05510061)
}
}
{
\newrgbcolor{curcolor}{0 0 0}
\pscustom[linestyle=none,fillstyle=solid,fillcolor=curcolor]
{
\newpath
\moveto(389.58476562,70.85470998)
\lineto(389.58476562,74.45470998)
\lineto(389.58476562,75.09970998)
\curveto(389.58477605,75.17970345)(389.58977605,75.25470338)(389.59976562,75.32470998)
\curveto(389.59977603,75.39470324)(389.60977602,75.45470318)(389.62976563,75.50470998)
\curveto(389.65977598,75.57470306)(389.71977592,75.629703)(389.80976563,75.66970998)
\curveto(389.83977579,75.68970294)(389.87977576,75.69970293)(389.92976563,75.69970998)
\lineto(390.06476563,75.69970998)
\curveto(390.17477546,75.70970292)(390.27977536,75.70470293)(390.37976563,75.68470998)
\curveto(390.47977515,75.67470296)(390.54977509,75.63970299)(390.58976562,75.57970998)
\curveto(390.65977498,75.48970314)(390.69477494,75.35470328)(390.69476563,75.17470998)
\curveto(390.68477495,74.99470364)(390.67977495,74.8297038)(390.67976563,74.67970998)
\lineto(390.67976563,72.68470998)
\lineto(390.67976563,72.18970998)
\lineto(390.67976563,72.05470998)
\curveto(390.67977495,72.01470662)(390.68477495,71.97470666)(390.69476563,71.93470998)
\lineto(390.69476563,71.72470998)
\curveto(390.72477491,71.61470702)(390.76477487,71.5347071)(390.81476563,71.48470998)
\curveto(390.85477478,71.4347072)(390.90977472,71.39970723)(390.97976563,71.37970998)
\curveto(391.03977459,71.35970727)(391.10977452,71.34470729)(391.18976563,71.33470998)
\curveto(391.26977436,71.32470731)(391.35977428,71.30470733)(391.45976562,71.27470998)
\curveto(391.65977398,71.22470741)(391.86477377,71.18470745)(392.07476562,71.15470998)
\curveto(392.28477335,71.12470751)(392.48977315,71.08470755)(392.68976563,71.03470998)
\curveto(392.75977288,71.01470762)(392.82977281,71.00470763)(392.89976562,71.00470998)
\curveto(392.95977268,71.00470763)(393.02477261,70.99470764)(393.09476562,70.97470998)
\curveto(393.12477251,70.96470767)(393.16477247,70.95470768)(393.21476562,70.94470998)
\curveto(393.25477238,70.94470769)(393.29477234,70.94970768)(393.33476562,70.95970998)
\curveto(393.38477225,70.97970765)(393.42977221,71.00470763)(393.46976562,71.03470998)
\curveto(393.49977214,71.07470756)(393.50477213,71.1347075)(393.48476563,71.21470998)
\curveto(393.46477217,71.27470736)(393.43977219,71.3347073)(393.40976562,71.39470998)
\curveto(393.36977226,71.45470718)(393.3347723,71.51470712)(393.30476563,71.57470998)
\curveto(393.28477235,71.634707)(393.26977236,71.68470695)(393.25976562,71.72470998)
\curveto(393.17977245,71.91470672)(393.12477251,72.11970651)(393.09476562,72.33970998)
\curveto(393.06477257,72.56970606)(393.05477258,72.79970583)(393.06476563,73.02970998)
\curveto(393.06477257,73.26970536)(393.08977255,73.49970513)(393.13976562,73.71970998)
\curveto(393.17977245,73.93970469)(393.23977239,74.13970449)(393.31976563,74.31970998)
\curveto(393.33977229,74.36970426)(393.35977228,74.41470422)(393.37976563,74.45470998)
\curveto(393.39977224,74.50470413)(393.42477221,74.55470408)(393.45476562,74.60470998)
\curveto(393.66477197,74.95470368)(393.89477174,75.2347034)(394.14476562,75.44470998)
\curveto(394.39477124,75.66470297)(394.71977092,75.85970277)(395.11976563,76.02970998)
\curveto(395.22977041,76.07970255)(395.33977029,76.11470252)(395.44976562,76.13470998)
\curveto(395.55977008,76.15470248)(395.67476996,76.17970245)(395.79476563,76.20970998)
\curveto(395.82476981,76.21970241)(395.86976977,76.22470241)(395.92976563,76.22470998)
\curveto(395.98976965,76.24470239)(396.05976958,76.25470238)(396.13976562,76.25470998)
\curveto(396.20976942,76.25470238)(396.27476936,76.26470237)(396.33476562,76.28470998)
\lineto(396.49976563,76.28470998)
\curveto(396.54976908,76.29470234)(396.61976901,76.29970233)(396.70976562,76.29970998)
\curveto(396.79976884,76.29970233)(396.86976877,76.28970234)(396.91976563,76.26970998)
\curveto(396.97976865,76.24970238)(397.03976859,76.24470239)(397.09976562,76.25470998)
\curveto(397.14976848,76.26470237)(397.19976844,76.25970237)(397.24976563,76.23970998)
\curveto(397.40976822,76.19970243)(397.55976808,76.16470247)(397.69976562,76.13470998)
\curveto(397.8397678,76.10470253)(397.97476766,76.05970257)(398.10476563,75.99970998)
\curveto(398.47476716,75.83970279)(398.80976682,75.61970301)(399.10976563,75.33970998)
\curveto(399.40976622,75.05970357)(399.639766,74.73970389)(399.79976563,74.37970998)
\curveto(399.87976575,74.20970442)(399.95476568,74.00970462)(400.02476562,73.77970998)
\curveto(400.06476557,73.66970496)(400.08976554,73.55470508)(400.09976562,73.43470998)
\curveto(400.10976553,73.31470532)(400.12976551,73.19470544)(400.15976562,73.07470998)
\curveto(400.17976545,73.02470561)(400.17976545,72.96970566)(400.15976562,72.90970998)
\curveto(400.14976548,72.84970578)(400.15476548,72.78970584)(400.17476563,72.72970998)
\curveto(400.19476544,72.629706)(400.19476544,72.5297061)(400.17476563,72.42970998)
\lineto(400.17476563,72.29470998)
\curveto(400.15476548,72.24470639)(400.14476549,72.18470645)(400.14476562,72.11470998)
\curveto(400.15476548,72.05470658)(400.14976548,71.99970663)(400.12976563,71.94970998)
\curveto(400.11976551,71.90970672)(400.11476552,71.87470676)(400.11476563,71.84470998)
\curveto(400.11476552,71.81470682)(400.10976553,71.77970685)(400.09976562,71.73970998)
\lineto(400.03976562,71.46970998)
\curveto(400.01976561,71.37970725)(399.98976564,71.29470734)(399.94976562,71.21470998)
\curveto(399.80976582,70.87470776)(399.65476598,70.58470805)(399.48476563,70.34470998)
\curveto(399.30476633,70.10470853)(399.07476656,69.88470875)(398.79476563,69.68470998)
\curveto(398.56476707,69.5347091)(398.32476731,69.41970921)(398.07476562,69.33970998)
\curveto(398.02476761,69.31970931)(397.97976765,69.30970932)(397.93976563,69.30970998)
\curveto(397.88976774,69.30970932)(397.8397678,69.29970933)(397.78976562,69.27970998)
\curveto(397.72976791,69.25970937)(397.64976798,69.24470939)(397.54976563,69.23470998)
\curveto(397.44976818,69.2347094)(397.37476826,69.25470938)(397.32476562,69.29470998)
\curveto(397.24476839,69.34470929)(397.19976844,69.42470921)(397.18976563,69.53470998)
\curveto(397.17976845,69.64470899)(397.17476846,69.75970887)(397.17476563,69.87970998)
\lineto(397.17476563,70.04470998)
\curveto(397.17476846,70.10470853)(397.18476845,70.15970847)(397.20476562,70.20970998)
\curveto(397.22476841,70.29970833)(397.26476837,70.36970826)(397.32476562,70.41970998)
\curveto(397.41476822,70.48970814)(397.52476811,70.5347081)(397.65476562,70.55470998)
\curveto(397.77476786,70.58470805)(397.87976775,70.629708)(397.96976562,70.68970998)
\curveto(398.30976732,70.87970775)(398.57976705,71.13970749)(398.77976562,71.46970998)
\curveto(398.8397668,71.56970706)(398.88976674,71.67470696)(398.92976563,71.78470998)
\curveto(398.95976668,71.90470673)(398.99476664,72.02470661)(399.03476562,72.14470998)
\curveto(399.08476655,72.31470632)(399.10476653,72.51970611)(399.09476562,72.75970998)
\curveto(399.07476656,73.00970562)(399.0397666,73.20970542)(398.98976563,73.35970998)
\curveto(398.86976677,73.7297049)(398.70976692,74.01970461)(398.50976562,74.22970998)
\curveto(398.29976734,74.44970418)(398.01976761,74.629704)(397.66976563,74.76970998)
\curveto(397.56976807,74.81970381)(397.46476817,74.84970378)(397.35476563,74.85970998)
\curveto(397.24476839,74.87970375)(397.12976851,74.90470373)(397.00976562,74.93470998)
\lineto(396.90476562,74.93470998)
\curveto(396.86476877,74.94470369)(396.82476881,74.94970368)(396.78476562,74.94970998)
\curveto(396.75476888,74.95970367)(396.71976891,74.95970367)(396.67976563,74.94970998)
\lineto(396.55976563,74.94970998)
\curveto(396.29976934,74.94970368)(396.05476958,74.91970371)(395.82476562,74.85970998)
\curveto(395.47477016,74.74970388)(395.17977045,74.59470404)(394.93976563,74.39470998)
\curveto(394.68977095,74.19470444)(394.49477114,73.9347047)(394.35476563,73.61470998)
\lineto(394.29476563,73.43470998)
\curveto(394.27477136,73.38470525)(394.25477138,73.32470531)(394.23476563,73.25470998)
\curveto(394.21477142,73.20470543)(394.20477143,73.14470549)(394.20476562,73.07470998)
\curveto(394.19477144,73.01470562)(394.17977145,72.94970568)(394.15976562,72.87970998)
\lineto(394.15976562,72.72970998)
\curveto(394.13977149,72.68970594)(394.12977151,72.634706)(394.12976563,72.56470998)
\curveto(394.12977151,72.50470613)(394.13977149,72.44970618)(394.15976562,72.39970998)
\lineto(394.15976562,72.29470998)
\curveto(394.15977148,72.26470637)(394.16477147,72.2297064)(394.17476563,72.18970998)
\lineto(394.23476563,71.94970998)
\curveto(394.24477139,71.86970676)(394.26477137,71.78970684)(394.29476563,71.70970998)
\curveto(394.39477124,71.46970716)(394.52977111,71.23970739)(394.69976562,71.01970998)
\curveto(394.76977086,70.9297077)(394.84477079,70.84470779)(394.92476563,70.76470998)
\curveto(394.99477064,70.68470795)(395.04977058,70.58470805)(395.08976562,70.46470998)
\curveto(395.11977052,70.37470826)(395.12977051,70.2347084)(395.11976563,70.04470998)
\curveto(395.10977052,69.86470877)(395.08477055,69.74470889)(395.04476563,69.68470998)
\curveto(395.00477063,69.634709)(394.94477069,69.59470904)(394.86476563,69.56470998)
\curveto(394.78477085,69.54470909)(394.69977094,69.54470909)(394.60976563,69.56470998)
\curveto(394.48977115,69.59470904)(394.36977126,69.61470902)(394.24976563,69.62470998)
\curveto(394.11977152,69.64470899)(393.99477164,69.66970896)(393.87476563,69.69970998)
\curveto(393.8347718,69.71970891)(393.79977184,69.72470891)(393.76976562,69.71470998)
\curveto(393.72977191,69.71470892)(393.68477195,69.72470891)(393.63476562,69.74470998)
\curveto(393.54477209,69.76470887)(393.45477218,69.77970885)(393.36476563,69.78970998)
\curveto(393.26477237,69.79970883)(393.16977246,69.81970881)(393.07976562,69.84970998)
\curveto(393.01977262,69.85970877)(392.95977268,69.86470877)(392.89976562,69.86470998)
\curveto(392.83977279,69.87470876)(392.77977285,69.88970874)(392.71976562,69.90970998)
\curveto(392.51977312,69.95970867)(392.31477332,69.99470864)(392.10476563,70.01470998)
\curveto(391.88477375,70.04470859)(391.67477396,70.08470855)(391.47476562,70.13470998)
\curveto(391.37477426,70.16470847)(391.27477436,70.18470845)(391.17476563,70.19470998)
\curveto(391.07477456,70.20470843)(390.97477466,70.21970841)(390.87476563,70.23970998)
\curveto(390.84477479,70.24970838)(390.80477483,70.25470838)(390.75476563,70.25470998)
\curveto(390.64477499,70.28470835)(390.53977509,70.30470833)(390.43976563,70.31470998)
\curveto(390.3297753,70.3347083)(390.21977542,70.35970827)(390.10976563,70.38970998)
\curveto(390.0297756,70.40970822)(389.95977568,70.42470821)(389.89976562,70.43470998)
\curveto(389.8297758,70.44470819)(389.76977586,70.46970816)(389.71976562,70.50970998)
\curveto(389.68977595,70.5297081)(389.66977596,70.55970807)(389.65976562,70.59970998)
\curveto(389.63977599,70.63970799)(389.61977602,70.68470795)(389.59976562,70.73470998)
\curveto(389.59977603,70.79470784)(389.59477604,70.8347078)(389.58476562,70.85470998)
}
}
{
\newrgbcolor{curcolor}{0 0 0}
\pscustom[linestyle=none,fillstyle=solid,fillcolor=curcolor]
{
\newpath
\moveto(398.35976563,78.63431936)
\lineto(398.35976563,79.26431936)
\lineto(398.35976563,79.45931936)
\curveto(398.35976728,79.52931683)(398.36976727,79.58931677)(398.38976562,79.63931936)
\curveto(398.42976721,79.70931665)(398.46976717,79.7593166)(398.50976562,79.78931936)
\curveto(398.55976708,79.82931653)(398.62476701,79.84931651)(398.70476562,79.84931936)
\curveto(398.78476685,79.8593165)(398.86976677,79.86431649)(398.95976562,79.86431936)
\lineto(399.67976563,79.86431936)
\curveto(400.15976547,79.86431649)(400.56976507,79.80431655)(400.90976562,79.68431936)
\curveto(401.24976438,79.56431679)(401.52476411,79.36931699)(401.73476563,79.09931936)
\curveto(401.78476385,79.02931733)(401.82976381,78.9593174)(401.86976563,78.88931936)
\curveto(401.91976371,78.82931753)(401.96476367,78.7543176)(402.00476563,78.66431936)
\curveto(402.01476362,78.64431771)(402.02476361,78.61431774)(402.03476562,78.57431936)
\curveto(402.05476358,78.53431782)(402.05976357,78.48931787)(402.04976563,78.43931936)
\curveto(402.01976361,78.34931801)(401.94476369,78.29431806)(401.82476562,78.27431936)
\curveto(401.71476392,78.2543181)(401.61976401,78.26931809)(401.53976562,78.31931936)
\curveto(401.46976417,78.34931801)(401.40476423,78.39431796)(401.34476562,78.45431936)
\curveto(401.29476434,78.52431783)(401.24476439,78.58931777)(401.19476563,78.64931936)
\curveto(401.14476449,78.71931764)(401.06976457,78.77931758)(400.96976562,78.82931936)
\curveto(400.87976475,78.88931747)(400.78976484,78.93931742)(400.69976562,78.97931936)
\curveto(400.66976497,78.99931736)(400.60976503,79.02431733)(400.51976562,79.05431936)
\curveto(400.4397652,79.08431727)(400.36976527,79.08931727)(400.30976563,79.06931936)
\curveto(400.16976547,79.03931732)(400.07976555,78.97931738)(400.03976562,78.88931936)
\curveto(400.00976562,78.80931755)(399.99476564,78.71931764)(399.99476563,78.61931936)
\curveto(399.99476564,78.51931784)(399.96976567,78.43431792)(399.91976563,78.36431936)
\curveto(399.84976578,78.27431808)(399.70976592,78.22931813)(399.49976563,78.22931936)
\lineto(398.94476563,78.22931936)
\lineto(398.71976562,78.22931936)
\curveto(398.639767,78.23931812)(398.57476706,78.2593181)(398.52476562,78.28931936)
\curveto(398.44476719,78.34931801)(398.39976724,78.41931794)(398.38976562,78.49931936)
\curveto(398.37976725,78.51931784)(398.37476726,78.53931782)(398.37476563,78.55931936)
\curveto(398.37476726,78.58931777)(398.36976727,78.61431774)(398.35976563,78.63431936)
}
}
{
\newrgbcolor{curcolor}{0 0 0}
\pscustom[linestyle=none,fillstyle=solid,fillcolor=curcolor]
{
}
}
{
\newrgbcolor{curcolor}{0 0 0}
\pscustom[linestyle=none,fillstyle=solid,fillcolor=curcolor]
{
\newpath
\moveto(389.38976562,89.26463186)
\curveto(389.37977626,89.95462722)(389.49977613,90.55462662)(389.74976563,91.06463186)
\curveto(389.99977563,91.58462559)(390.3347753,91.9796252)(390.75476563,92.24963186)
\curveto(390.8347748,92.29962488)(390.92477471,92.34462483)(391.02476562,92.38463186)
\curveto(391.11477452,92.42462475)(391.20977442,92.46962471)(391.30976563,92.51963186)
\curveto(391.40977422,92.55962462)(391.50977412,92.58962459)(391.60976563,92.60963186)
\curveto(391.70977392,92.62962455)(391.81477382,92.64962453)(391.92476563,92.66963186)
\curveto(391.97477366,92.68962449)(392.01977362,92.69462448)(392.05976563,92.68463186)
\curveto(392.09977353,92.6746245)(392.14477349,92.6796245)(392.19476563,92.69963186)
\curveto(392.24477339,92.70962447)(392.32977331,92.71462446)(392.44976562,92.71463186)
\curveto(392.55977308,92.71462446)(392.64477299,92.70962447)(392.70476562,92.69963186)
\curveto(392.76477287,92.6796245)(392.82477281,92.66962451)(392.88476562,92.66963186)
\curveto(392.94477269,92.6796245)(393.00477263,92.6746245)(393.06476563,92.65463186)
\curveto(393.20477243,92.61462456)(393.33977229,92.5796246)(393.46976562,92.54963186)
\curveto(393.59977204,92.51962466)(393.72477191,92.4796247)(393.84476562,92.42963186)
\curveto(393.98477165,92.36962481)(394.10977152,92.29962488)(394.21976562,92.21963186)
\curveto(394.32977131,92.14962503)(394.43977119,92.0746251)(394.54976563,91.99463186)
\lineto(394.60976563,91.93463186)
\curveto(394.62977101,91.92462525)(394.64977098,91.90962527)(394.66976563,91.88963186)
\curveto(394.82977081,91.76962541)(394.97477066,91.63462554)(395.10476563,91.48463186)
\curveto(395.2347704,91.33462584)(395.35977028,91.174626)(395.47976563,91.00463186)
\curveto(395.69976994,90.69462648)(395.90476973,90.39962678)(396.09476562,90.11963186)
\curveto(396.2347694,89.88962729)(396.36976927,89.65962752)(396.49976563,89.42963186)
\curveto(396.62976901,89.20962797)(396.76476887,88.98962819)(396.90476562,88.76963186)
\curveto(397.07476856,88.51962866)(397.25476838,88.2796289)(397.44476563,88.04963186)
\curveto(397.634768,87.82962935)(397.85976778,87.63962954)(398.11976563,87.47963186)
\curveto(398.17976745,87.43962974)(398.2397674,87.40462977)(398.29976563,87.37463186)
\curveto(398.34976728,87.34462983)(398.41476722,87.31462986)(398.49476563,87.28463186)
\curveto(398.56476707,87.26462991)(398.62476701,87.25962992)(398.67476563,87.26963186)
\curveto(398.74476689,87.28962989)(398.79976684,87.32462985)(398.83976562,87.37463186)
\curveto(398.86976677,87.42462975)(398.88976674,87.48462969)(398.89976562,87.55463186)
\lineto(398.89976562,87.79463186)
\lineto(398.89976562,88.54463186)
\lineto(398.89976562,91.34963186)
\lineto(398.89976562,92.00963186)
\curveto(398.89976674,92.09962508)(398.90476673,92.18462499)(398.91476562,92.26463186)
\curveto(398.91476672,92.34462483)(398.9347667,92.40962477)(398.97476562,92.45963186)
\curveto(399.01476662,92.50962467)(399.08976654,92.54962463)(399.19976562,92.57963186)
\curveto(399.29976634,92.61962456)(399.39976624,92.62962455)(399.49976563,92.60963186)
\lineto(399.63476562,92.60963186)
\curveto(399.70476593,92.58962459)(399.76476587,92.56962461)(399.81476563,92.54963186)
\curveto(399.86476577,92.52962465)(399.90476573,92.49462468)(399.93476563,92.44463186)
\curveto(399.97476566,92.39462478)(399.99476564,92.32462485)(399.99476563,92.23463186)
\lineto(399.99476563,91.96463186)
\lineto(399.99476563,91.06463186)
\lineto(399.99476563,87.55463186)
\lineto(399.99476563,86.48963186)
\curveto(399.99476564,86.40963077)(399.99976564,86.31963086)(400.00976562,86.21963186)
\curveto(400.00976562,86.11963106)(399.99976564,86.03463114)(399.97976563,85.96463186)
\curveto(399.90976572,85.75463142)(399.72976591,85.68963149)(399.43976563,85.76963186)
\curveto(399.39976624,85.7796314)(399.36476627,85.7796314)(399.33476562,85.76963186)
\curveto(399.29476634,85.76963141)(399.24976638,85.7796314)(399.19976562,85.79963186)
\curveto(399.11976651,85.81963136)(399.0347666,85.83963134)(398.94476563,85.85963186)
\curveto(398.85476678,85.8796313)(398.76976687,85.90463127)(398.68976563,85.93463186)
\curveto(398.19976744,86.09463108)(397.78476785,86.29463088)(397.44476563,86.53463186)
\curveto(397.19476844,86.71463046)(396.96976867,86.91963026)(396.76976562,87.14963186)
\curveto(396.55976908,87.3796298)(396.36476927,87.61962956)(396.18476563,87.86963186)
\curveto(396.00476963,88.12962905)(395.8347698,88.39462878)(395.67476563,88.66463186)
\curveto(395.50477013,88.94462823)(395.32977031,89.21462796)(395.14976562,89.47463186)
\curveto(395.06977056,89.58462759)(394.99477064,89.68962749)(394.92476563,89.78963186)
\curveto(394.85477078,89.89962728)(394.77977085,90.00962717)(394.69976562,90.11963186)
\curveto(394.66977096,90.15962702)(394.63977099,90.19462698)(394.60976563,90.22463186)
\curveto(394.56977106,90.26462691)(394.53977109,90.30462687)(394.51976562,90.34463186)
\curveto(394.40977122,90.48462669)(394.28477135,90.60962657)(394.14476562,90.71963186)
\curveto(394.11477152,90.73962644)(394.08977155,90.76462641)(394.06976563,90.79463186)
\curveto(394.03977159,90.82462635)(394.00977162,90.84962633)(393.97976563,90.86963186)
\curveto(393.87977175,90.94962623)(393.77977185,91.01462616)(393.67976563,91.06463186)
\curveto(393.57977205,91.12462605)(393.46977216,91.179626)(393.34976562,91.22963186)
\curveto(393.27977235,91.25962592)(393.20477243,91.2796259)(393.12476563,91.28963186)
\lineto(392.88476562,91.34963186)
\lineto(392.79476563,91.34963186)
\curveto(392.76477287,91.35962582)(392.7347729,91.36462581)(392.70476562,91.36463186)
\curveto(392.634773,91.38462579)(392.53977309,91.38962579)(392.41976563,91.37963186)
\curveto(392.28977335,91.3796258)(392.18977345,91.36962581)(392.11976563,91.34963186)
\curveto(392.03977359,91.32962585)(391.96477367,91.30962587)(391.89476562,91.28963186)
\curveto(391.81477382,91.2796259)(391.7347739,91.25962592)(391.65476562,91.22963186)
\curveto(391.41477422,91.11962606)(391.21477442,90.96962621)(391.05476563,90.77963186)
\curveto(390.88477475,90.59962658)(390.74477489,90.3796268)(390.63476562,90.11963186)
\curveto(390.61477502,90.04962713)(390.59977503,89.9796272)(390.58976562,89.90963186)
\curveto(390.56977506,89.83962734)(390.54977509,89.76462741)(390.52976562,89.68463186)
\curveto(390.50977512,89.60462757)(390.49977513,89.49462768)(390.49976563,89.35463186)
\curveto(390.49977513,89.22462795)(390.50977512,89.11962806)(390.52976562,89.03963186)
\curveto(390.53977509,88.9796282)(390.54477509,88.92462825)(390.54476563,88.87463186)
\curveto(390.54477509,88.82462835)(390.55477508,88.7746284)(390.57476562,88.72463186)
\curveto(390.61477502,88.62462855)(390.65477498,88.52962865)(390.69476563,88.43963186)
\curveto(390.7347749,88.35962882)(390.77977486,88.2796289)(390.82976562,88.19963186)
\curveto(390.84977479,88.16962901)(390.87477476,88.13962904)(390.90476562,88.10963186)
\curveto(390.9347747,88.08962909)(390.95977468,88.06462911)(390.97976563,88.03463186)
\lineto(391.05476563,87.95963186)
\curveto(391.07477456,87.92962925)(391.09477454,87.90462927)(391.11476563,87.88463186)
\lineto(391.32476562,87.73463186)
\curveto(391.38477425,87.69462948)(391.44977419,87.64962953)(391.51976562,87.59963186)
\curveto(391.60977402,87.53962964)(391.71477392,87.48962969)(391.83476562,87.44963186)
\curveto(391.94477369,87.41962976)(392.05477358,87.38462979)(392.16476562,87.34463186)
\curveto(392.27477336,87.30462987)(392.41977322,87.2796299)(392.59976562,87.26963186)
\curveto(392.76977286,87.25962992)(392.89477274,87.22962995)(392.97476562,87.17963186)
\curveto(393.05477258,87.12963005)(393.09977253,87.05463012)(393.10976563,86.95463186)
\curveto(393.11977252,86.85463032)(393.12477251,86.74463043)(393.12476563,86.62463186)
\curveto(393.12477251,86.58463059)(393.12977251,86.54463063)(393.13976562,86.50463186)
\curveto(393.13977249,86.46463071)(393.1347725,86.42963075)(393.12476563,86.39963186)
\curveto(393.10477253,86.34963083)(393.09477254,86.29963088)(393.09476562,86.24963186)
\curveto(393.09477254,86.20963097)(393.08477255,86.16963101)(393.06476563,86.12963186)
\curveto(393.00477263,86.03963114)(392.86977276,85.99463118)(392.65976562,85.99463186)
\lineto(392.53976562,85.99463186)
\curveto(392.47977315,86.00463117)(392.41977322,86.00963117)(392.35976563,86.00963186)
\curveto(392.28977335,86.01963116)(392.22477341,86.02963115)(392.16476562,86.03963186)
\curveto(392.05477358,86.05963112)(391.95477368,86.0796311)(391.86476563,86.09963186)
\curveto(391.76477387,86.11963106)(391.66977396,86.14963103)(391.57976562,86.18963186)
\curveto(391.50977412,86.20963097)(391.44977419,86.22963095)(391.39976562,86.24963186)
\lineto(391.21976562,86.30963186)
\curveto(390.95977468,86.42963075)(390.71477492,86.58463059)(390.48476563,86.77463186)
\curveto(390.25477538,86.9746302)(390.06977556,87.18962999)(389.92976563,87.41963186)
\curveto(389.84977579,87.52962965)(389.78477585,87.64462953)(389.73476563,87.76463186)
\lineto(389.58476562,88.15463186)
\curveto(389.5347761,88.26462891)(389.50477613,88.3796288)(389.49476563,88.49963186)
\curveto(389.47477616,88.61962856)(389.44977619,88.74462843)(389.41976563,88.87463186)
\curveto(389.41977622,88.94462823)(389.41977622,89.00962817)(389.41976563,89.06963186)
\curveto(389.40977622,89.12962805)(389.39977623,89.19462798)(389.38976562,89.26463186)
}
}
{
\newrgbcolor{curcolor}{0 0 0}
\pscustom[linestyle=none,fillstyle=solid,fillcolor=curcolor]
{
\newpath
\moveto(394.90976562,101.36424123)
\lineto(395.16476562,101.36424123)
\curveto(395.24477039,101.37423353)(395.31977032,101.36923353)(395.38976562,101.34924123)
\lineto(395.62976563,101.34924123)
\lineto(395.79476563,101.34924123)
\curveto(395.89476974,101.32923357)(395.99976964,101.31923358)(396.10976563,101.31924123)
\curveto(396.20976942,101.31923358)(396.30976932,101.30923359)(396.40976562,101.28924123)
\lineto(396.55976563,101.28924123)
\curveto(396.69976894,101.25923364)(396.83976879,101.23923366)(396.97976563,101.22924123)
\curveto(397.10976852,101.21923368)(397.23976839,101.19423371)(397.36976563,101.15424123)
\curveto(397.44976818,101.13423377)(397.5347681,101.11423379)(397.62476563,101.09424123)
\lineto(397.86476563,101.03424123)
\lineto(398.16476562,100.91424123)
\curveto(398.25476738,100.88423402)(398.34476729,100.84923405)(398.43476563,100.80924123)
\curveto(398.65476698,100.70923419)(398.86976677,100.57423433)(399.07976562,100.40424123)
\curveto(399.28976634,100.24423466)(399.45976618,100.06923483)(399.58976562,99.87924123)
\curveto(399.62976601,99.82923507)(399.66976597,99.76923513)(399.70976562,99.69924123)
\curveto(399.7397659,99.63923526)(399.77476586,99.57923532)(399.81476563,99.51924123)
\curveto(399.86476577,99.43923546)(399.90476573,99.34423556)(399.93476563,99.23424123)
\curveto(399.96476567,99.12423578)(399.99476564,99.01923588)(400.02476562,98.91924123)
\curveto(400.06476557,98.80923609)(400.08976554,98.6992362)(400.09976562,98.58924123)
\curveto(400.10976553,98.47923642)(400.12476551,98.36423654)(400.14476562,98.24424123)
\curveto(400.15476548,98.2042367)(400.15476548,98.15923674)(400.14476562,98.10924123)
\curveto(400.14476549,98.06923683)(400.14976548,98.02923687)(400.15976562,97.98924123)
\curveto(400.16976547,97.94923695)(400.17476546,97.89423701)(400.17476563,97.82424123)
\curveto(400.17476546,97.75423715)(400.16976547,97.7042372)(400.15976562,97.67424123)
\curveto(400.1397655,97.62423728)(400.1347655,97.57923732)(400.14476562,97.53924123)
\curveto(400.15476548,97.4992374)(400.15476548,97.46423744)(400.14476562,97.43424123)
\lineto(400.14476562,97.34424123)
\curveto(400.12476551,97.28423762)(400.10976553,97.21923768)(400.09976562,97.14924123)
\curveto(400.09976554,97.08923781)(400.09476554,97.02423788)(400.08476562,96.95424123)
\curveto(400.0347656,96.78423812)(399.98476565,96.62423828)(399.93476563,96.47424123)
\curveto(399.88476575,96.32423858)(399.81976581,96.17923872)(399.73976563,96.03924123)
\curveto(399.69976594,95.98923891)(399.66976597,95.93423897)(399.64976562,95.87424123)
\curveto(399.61976601,95.82423908)(399.58476605,95.77423913)(399.54476563,95.72424123)
\curveto(399.36476627,95.48423942)(399.14476649,95.28423962)(398.88476562,95.12424123)
\curveto(398.62476701,94.96423994)(398.3397673,94.82424008)(398.02976562,94.70424123)
\curveto(397.88976774,94.64424026)(397.74976788,94.5992403)(397.60976563,94.56924123)
\curveto(397.45976818,94.53924036)(397.30476833,94.5042404)(397.14476562,94.46424123)
\curveto(397.0347686,94.44424046)(396.92476871,94.42924047)(396.81476563,94.41924123)
\curveto(396.70476893,94.40924049)(396.59476904,94.39424051)(396.48476563,94.37424123)
\curveto(396.44476919,94.36424054)(396.40476923,94.35924054)(396.36476563,94.35924123)
\curveto(396.32476931,94.36924053)(396.28476935,94.36924053)(396.24476563,94.35924123)
\curveto(396.19476944,94.34924055)(396.14476949,94.34424056)(396.09476562,94.34424123)
\lineto(395.92976563,94.34424123)
\curveto(395.87976975,94.32424058)(395.82976981,94.31924058)(395.77976562,94.32924123)
\curveto(395.71976991,94.33924056)(395.66476997,94.33924056)(395.61476563,94.32924123)
\curveto(395.57477006,94.31924058)(395.52977011,94.31924058)(395.47976563,94.32924123)
\curveto(395.42977021,94.33924056)(395.37977025,94.33424057)(395.32976562,94.31424123)
\curveto(395.25977038,94.29424061)(395.18477045,94.28924061)(395.10476563,94.29924123)
\curveto(395.01477062,94.30924059)(394.92977071,94.31424059)(394.84976562,94.31424123)
\curveto(394.75977088,94.31424059)(394.65977098,94.30924059)(394.54976563,94.29924123)
\curveto(394.42977121,94.28924061)(394.32977131,94.29424061)(394.24976563,94.31424123)
\lineto(393.96476562,94.31424123)
\lineto(393.33476562,94.35924123)
\curveto(393.2347724,94.36924053)(393.13977249,94.37924052)(393.04976563,94.38924123)
\lineto(392.74976563,94.41924123)
\curveto(392.69977293,94.43924046)(392.64977299,94.44424046)(392.59976562,94.43424123)
\curveto(392.53977309,94.43424047)(392.48477315,94.44424046)(392.43476563,94.46424123)
\curveto(392.26477337,94.51424039)(392.09977353,94.55424035)(391.93976563,94.58424123)
\curveto(391.76977386,94.61424029)(391.60977402,94.66424024)(391.45976562,94.73424123)
\curveto(390.99977463,94.92423998)(390.62477501,95.14423976)(390.33476562,95.39424123)
\curveto(390.04477559,95.65423925)(389.79977583,96.01423889)(389.59976562,96.47424123)
\curveto(389.54977609,96.6042383)(389.51477612,96.73423817)(389.49476563,96.86424123)
\curveto(389.47477616,97.0042379)(389.44977619,97.14423776)(389.41976563,97.28424123)
\curveto(389.40977622,97.35423755)(389.40477623,97.41923748)(389.40476562,97.47924123)
\curveto(389.40477623,97.53923736)(389.39977623,97.6042373)(389.38976562,97.67424123)
\curveto(389.36977626,98.5042364)(389.51977612,99.17423573)(389.83976562,99.68424123)
\curveto(390.14977549,100.19423471)(390.58977505,100.57423433)(391.15976562,100.82424123)
\curveto(391.27977435,100.87423403)(391.40477423,100.91923398)(391.53476562,100.95924123)
\curveto(391.66477397,100.9992339)(391.79977383,101.04423386)(391.93976563,101.09424123)
\curveto(392.01977362,101.11423379)(392.10477353,101.12923377)(392.19476563,101.13924123)
\lineto(392.43476563,101.19924123)
\curveto(392.54477309,101.22923367)(392.65477298,101.24423366)(392.76476562,101.24424123)
\curveto(392.87477276,101.25423365)(392.98477265,101.26923363)(393.09476562,101.28924123)
\curveto(393.14477249,101.30923359)(393.18977245,101.31423359)(393.22976563,101.30424123)
\curveto(393.26977236,101.3042336)(393.30977232,101.30923359)(393.34976562,101.31924123)
\curveto(393.39977224,101.32923357)(393.45477218,101.32923357)(393.51476562,101.31924123)
\curveto(393.56477207,101.31923358)(393.61477202,101.32423358)(393.66476562,101.33424123)
\lineto(393.79976563,101.33424123)
\curveto(393.85977178,101.35423355)(393.92977171,101.35423355)(394.00976562,101.33424123)
\curveto(394.07977155,101.32423358)(394.14477149,101.32923357)(394.20476562,101.34924123)
\curveto(394.2347714,101.35923354)(394.27477136,101.36423354)(394.32476562,101.36424123)
\lineto(394.44476563,101.36424123)
\lineto(394.90976562,101.36424123)
\moveto(397.23476563,99.81924123)
\curveto(396.91476872,99.91923498)(396.54976908,99.97923492)(396.13976562,99.99924123)
\curveto(395.72976991,100.01923488)(395.31977032,100.02923487)(394.90976562,100.02924123)
\curveto(394.47977115,100.02923487)(394.05977158,100.01923488)(393.64976562,99.99924123)
\curveto(393.23977239,99.97923492)(392.85477278,99.93423497)(392.49476563,99.86424123)
\curveto(392.1347735,99.79423511)(391.81477382,99.68423522)(391.53476562,99.53424123)
\curveto(391.24477439,99.39423551)(391.00977462,99.1992357)(390.82976562,98.94924123)
\curveto(390.71977492,98.78923611)(390.63977499,98.60923629)(390.58976562,98.40924123)
\curveto(390.5297751,98.20923669)(390.49977513,97.96423694)(390.49976563,97.67424123)
\curveto(390.51977512,97.65423725)(390.5297751,97.61923728)(390.52976562,97.56924123)
\curveto(390.51977512,97.51923738)(390.51977512,97.47923742)(390.52976562,97.44924123)
\curveto(390.54977509,97.36923753)(390.56977506,97.29423761)(390.58976562,97.22424123)
\curveto(390.59977503,97.16423774)(390.61977502,97.0992378)(390.64976562,97.02924123)
\curveto(390.76977486,96.75923814)(390.93977469,96.53923836)(391.15976562,96.36924123)
\curveto(391.36977426,96.20923869)(391.61477402,96.07423883)(391.89476562,95.96424123)
\curveto(392.00477363,95.91423899)(392.12477351,95.87423903)(392.25476563,95.84424123)
\curveto(392.37477326,95.82423908)(392.49977313,95.7992391)(392.62976563,95.76924123)
\curveto(392.67977295,95.74923915)(392.7347729,95.73923916)(392.79476563,95.73924123)
\curveto(392.84477279,95.73923916)(392.89477274,95.73423917)(392.94476563,95.72424123)
\curveto(393.0347726,95.71423919)(393.12977251,95.7042392)(393.22976563,95.69424123)
\curveto(393.31977232,95.68423922)(393.41477222,95.67423923)(393.51476562,95.66424123)
\curveto(393.59477204,95.66423924)(393.67977195,95.65923924)(393.76976562,95.64924123)
\lineto(394.00976562,95.64924123)
\lineto(394.18976563,95.64924123)
\curveto(394.21977142,95.63923926)(394.25477138,95.63423927)(394.29476563,95.63424123)
\lineto(394.42976563,95.63424123)
\lineto(394.87976563,95.63424123)
\curveto(394.95977068,95.63423927)(395.04477059,95.62923927)(395.13476562,95.61924123)
\curveto(395.21477042,95.61923928)(395.28977035,95.62923927)(395.35976563,95.64924123)
\lineto(395.62976563,95.64924123)
\curveto(395.64976998,95.64923925)(395.67976995,95.64423926)(395.71976562,95.63424123)
\curveto(395.74976988,95.63423927)(395.77476986,95.63923926)(395.79476563,95.64924123)
\curveto(395.89476974,95.65923924)(395.99476964,95.66423924)(396.09476562,95.66424123)
\curveto(396.18476945,95.67423923)(396.28476935,95.68423922)(396.39476562,95.69424123)
\curveto(396.51476912,95.72423918)(396.63976899,95.73923916)(396.76976562,95.73924123)
\curveto(396.88976875,95.74923915)(397.00476863,95.77423913)(397.11476563,95.81424123)
\curveto(397.41476822,95.89423901)(397.67976795,95.97923892)(397.90976562,96.06924123)
\curveto(398.1397675,96.16923873)(398.35476728,96.31423859)(398.55476563,96.50424123)
\curveto(398.75476688,96.71423819)(398.90476673,96.97923792)(399.00476563,97.29924123)
\curveto(399.02476661,97.33923756)(399.0347666,97.37423753)(399.03476562,97.40424123)
\curveto(399.02476661,97.44423746)(399.02976661,97.48923741)(399.04976563,97.53924123)
\curveto(399.05976658,97.57923732)(399.06976657,97.64923725)(399.07976562,97.74924123)
\curveto(399.08976654,97.85923704)(399.08476655,97.94423696)(399.06476563,98.00424123)
\curveto(399.04476659,98.07423683)(399.0347666,98.14423676)(399.03476562,98.21424123)
\curveto(399.02476661,98.28423662)(399.00976662,98.34923655)(398.98976563,98.40924123)
\curveto(398.92976671,98.60923629)(398.84476679,98.78923611)(398.73476563,98.94924123)
\curveto(398.71476692,98.97923592)(398.69476694,99.0042359)(398.67476563,99.02424123)
\lineto(398.61476563,99.08424123)
\curveto(398.59476704,99.12423578)(398.55476708,99.17423573)(398.49476563,99.23424123)
\curveto(398.35476728,99.33423557)(398.22476741,99.41923548)(398.10476563,99.48924123)
\curveto(397.98476765,99.55923534)(397.8397678,99.62923527)(397.66976563,99.69924123)
\curveto(397.59976804,99.72923517)(397.52976811,99.74923515)(397.45976562,99.75924123)
\curveto(397.38976825,99.77923512)(397.31476832,99.7992351)(397.23476563,99.81924123)
}
}
{
\newrgbcolor{curcolor}{0 0 0}
\pscustom[linestyle=none,fillstyle=solid,fillcolor=curcolor]
{
\newpath
\moveto(389.38976562,106.77385061)
\curveto(389.38977625,106.87384575)(389.39977623,106.96884566)(389.41976563,107.05885061)
\curveto(389.4297762,107.14884548)(389.45977618,107.21384541)(389.50976562,107.25385061)
\curveto(389.58977605,107.31384531)(389.69477594,107.34384528)(389.82476562,107.34385061)
\lineto(390.21476562,107.34385061)
\lineto(391.71476562,107.34385061)
\lineto(398.10476563,107.34385061)
\lineto(399.27476562,107.34385061)
\lineto(399.58976562,107.34385061)
\curveto(399.68976594,107.35384527)(399.76976587,107.33884529)(399.82976562,107.29885061)
\curveto(399.90976572,107.24884538)(399.95976568,107.17384545)(399.97976563,107.07385061)
\curveto(399.98976564,106.98384564)(399.99476564,106.87384575)(399.99476563,106.74385061)
\lineto(399.99476563,106.51885061)
\curveto(399.97476566,106.43884619)(399.95976568,106.36884626)(399.94976562,106.30885061)
\curveto(399.92976571,106.24884638)(399.88976574,106.19884643)(399.82976562,106.15885061)
\curveto(399.76976587,106.11884651)(399.69476594,106.09884653)(399.60476563,106.09885061)
\lineto(399.30476563,106.09885061)
\lineto(398.20976562,106.09885061)
\lineto(392.86976563,106.09885061)
\curveto(392.77977285,106.07884655)(392.70477293,106.06384656)(392.64476562,106.05385061)
\curveto(392.57477306,106.05384657)(392.51477312,106.0238466)(392.46476562,105.96385061)
\curveto(392.41477322,105.89384673)(392.38977325,105.80384682)(392.38976562,105.69385061)
\curveto(392.37977325,105.59384703)(392.37477326,105.48384714)(392.37476563,105.36385061)
\lineto(392.37476563,104.22385061)
\lineto(392.37476563,103.72885061)
\curveto(392.36477327,103.56884906)(392.30477333,103.45884917)(392.19476563,103.39885061)
\curveto(392.16477347,103.37884925)(392.1347735,103.36884926)(392.10476563,103.36885061)
\curveto(392.06477357,103.36884926)(392.01977362,103.36384926)(391.96976562,103.35385061)
\curveto(391.84977379,103.33384929)(391.73977389,103.33884929)(391.63976562,103.36885061)
\curveto(391.53977409,103.40884922)(391.46977416,103.46384916)(391.42976563,103.53385061)
\curveto(391.37977426,103.61384901)(391.35477428,103.73384889)(391.35476563,103.89385061)
\curveto(391.35477428,104.05384857)(391.33977429,104.18884844)(391.30976563,104.29885061)
\curveto(391.29977433,104.34884828)(391.29477434,104.40384822)(391.29476563,104.46385061)
\curveto(391.28477435,104.5238481)(391.26977436,104.58384804)(391.24976563,104.64385061)
\curveto(391.19977443,104.79384783)(391.14977449,104.93884769)(391.09976562,105.07885061)
\curveto(391.03977459,105.21884741)(390.96977466,105.35384727)(390.88976562,105.48385061)
\curveto(390.79977483,105.623847)(390.69477494,105.74384688)(390.57476562,105.84385061)
\curveto(390.45477518,105.94384668)(390.32477531,106.03884659)(390.18476563,106.12885061)
\curveto(390.08477555,106.18884644)(389.97477566,106.23384639)(389.85476563,106.26385061)
\curveto(389.7347759,106.30384632)(389.629776,106.35384627)(389.53976562,106.41385061)
\curveto(389.47977616,106.46384616)(389.43977619,106.53384609)(389.41976563,106.62385061)
\curveto(389.40977622,106.64384598)(389.40477623,106.66884596)(389.40476562,106.69885061)
\curveto(389.40477623,106.7288459)(389.39977623,106.75384587)(389.38976562,106.77385061)
}
}
{
\newrgbcolor{curcolor}{0 0 0}
\pscustom[linestyle=none,fillstyle=solid,fillcolor=curcolor]
{
\newpath
\moveto(389.38976562,115.12345998)
\curveto(389.38977625,115.22345513)(389.39977623,115.31845503)(389.41976563,115.40845998)
\curveto(389.4297762,115.49845485)(389.45977618,115.56345479)(389.50976562,115.60345998)
\curveto(389.58977605,115.66345469)(389.69477594,115.69345466)(389.82476562,115.69345998)
\lineto(390.21476562,115.69345998)
\lineto(391.71476562,115.69345998)
\lineto(398.10476563,115.69345998)
\lineto(399.27476562,115.69345998)
\lineto(399.58976562,115.69345998)
\curveto(399.68976594,115.70345465)(399.76976587,115.68845466)(399.82976562,115.64845998)
\curveto(399.90976572,115.59845475)(399.95976568,115.52345483)(399.97976563,115.42345998)
\curveto(399.98976564,115.33345502)(399.99476564,115.22345513)(399.99476563,115.09345998)
\lineto(399.99476563,114.86845998)
\curveto(399.97476566,114.78845556)(399.95976568,114.71845563)(399.94976562,114.65845998)
\curveto(399.92976571,114.59845575)(399.88976574,114.5484558)(399.82976562,114.50845998)
\curveto(399.76976587,114.46845588)(399.69476594,114.4484559)(399.60476563,114.44845998)
\lineto(399.30476563,114.44845998)
\lineto(398.20976562,114.44845998)
\lineto(392.86976563,114.44845998)
\curveto(392.77977285,114.42845592)(392.70477293,114.41345594)(392.64476562,114.40345998)
\curveto(392.57477306,114.40345595)(392.51477312,114.37345598)(392.46476562,114.31345998)
\curveto(392.41477322,114.24345611)(392.38977325,114.1534562)(392.38976562,114.04345998)
\curveto(392.37977325,113.94345641)(392.37477326,113.83345652)(392.37476563,113.71345998)
\lineto(392.37476563,112.57345998)
\lineto(392.37476563,112.07845998)
\curveto(392.36477327,111.91845843)(392.30477333,111.80845854)(392.19476563,111.74845998)
\curveto(392.16477347,111.72845862)(392.1347735,111.71845863)(392.10476563,111.71845998)
\curveto(392.06477357,111.71845863)(392.01977362,111.71345864)(391.96976562,111.70345998)
\curveto(391.84977379,111.68345867)(391.73977389,111.68845866)(391.63976562,111.71845998)
\curveto(391.53977409,111.75845859)(391.46977416,111.81345854)(391.42976563,111.88345998)
\curveto(391.37977426,111.96345839)(391.35477428,112.08345827)(391.35476563,112.24345998)
\curveto(391.35477428,112.40345795)(391.33977429,112.53845781)(391.30976563,112.64845998)
\curveto(391.29977433,112.69845765)(391.29477434,112.7534576)(391.29476563,112.81345998)
\curveto(391.28477435,112.87345748)(391.26977436,112.93345742)(391.24976563,112.99345998)
\curveto(391.19977443,113.14345721)(391.14977449,113.28845706)(391.09976562,113.42845998)
\curveto(391.03977459,113.56845678)(390.96977466,113.70345665)(390.88976562,113.83345998)
\curveto(390.79977483,113.97345638)(390.69477494,114.09345626)(390.57476562,114.19345998)
\curveto(390.45477518,114.29345606)(390.32477531,114.38845596)(390.18476563,114.47845998)
\curveto(390.08477555,114.53845581)(389.97477566,114.58345577)(389.85476563,114.61345998)
\curveto(389.7347759,114.6534557)(389.629776,114.70345565)(389.53976562,114.76345998)
\curveto(389.47977616,114.81345554)(389.43977619,114.88345547)(389.41976563,114.97345998)
\curveto(389.40977622,114.99345536)(389.40477623,115.01845533)(389.40476562,115.04845998)
\curveto(389.40477623,115.07845527)(389.39977623,115.10345525)(389.38976562,115.12345998)
}
}
{
\newrgbcolor{curcolor}{0 0 0}
\pscustom[linestyle=none,fillstyle=solid,fillcolor=curcolor]
{
\newpath
\moveto(420.12611206,42.02236623)
\curveto(420.17611281,42.04235669)(420.23611275,42.06735666)(420.30611206,42.09736623)
\curveto(420.37611261,42.1273566)(420.45111253,42.14735658)(420.53111206,42.15736623)
\curveto(420.60111238,42.17735655)(420.67111231,42.17735655)(420.74111206,42.15736623)
\curveto(420.80111218,42.14735658)(420.84611214,42.10735662)(420.87611206,42.03736623)
\curveto(420.89611209,41.98735674)(420.90611208,41.9273568)(420.90611206,41.85736623)
\lineto(420.90611206,41.64736623)
\lineto(420.90611206,41.19736623)
\curveto(420.90611208,41.04735768)(420.8811121,40.9273578)(420.83111206,40.83736623)
\curveto(420.77111221,40.73735799)(420.66611232,40.66235807)(420.51611206,40.61236623)
\curveto(420.36611262,40.57235816)(420.23111275,40.5273582)(420.11111206,40.47736623)
\curveto(419.85111313,40.36735836)(419.5811134,40.26735846)(419.30111206,40.17736623)
\curveto(419.02111396,40.08735864)(418.74611424,39.98735874)(418.47611206,39.87736623)
\curveto(418.3861146,39.84735888)(418.30111468,39.81735891)(418.22111206,39.78736623)
\curveto(418.14111484,39.76735896)(418.06611492,39.73735899)(417.99611206,39.69736623)
\curveto(417.92611506,39.66735906)(417.86611512,39.62235911)(417.81611206,39.56236623)
\curveto(417.76611522,39.50235923)(417.72611526,39.42235931)(417.69611206,39.32236623)
\curveto(417.67611531,39.27235946)(417.67111531,39.21235952)(417.68111206,39.14236623)
\lineto(417.68111206,38.94736623)
\lineto(417.68111206,36.11236623)
\lineto(417.68111206,35.81236623)
\curveto(417.67111531,35.70236303)(417.67111531,35.59736313)(417.68111206,35.49736623)
\curveto(417.69111529,35.39736333)(417.70611528,35.30236343)(417.72611206,35.21236623)
\curveto(417.74611524,35.1323636)(417.7861152,35.07236366)(417.84611206,35.03236623)
\curveto(417.94611504,34.95236378)(418.06111492,34.89236384)(418.19111206,34.85236623)
\curveto(418.31111467,34.82236391)(418.43611455,34.78236395)(418.56611206,34.73236623)
\curveto(418.79611419,34.6323641)(419.03611395,34.53736419)(419.28611206,34.44736623)
\curveto(419.53611345,34.36736436)(419.77611321,34.27736445)(420.00611206,34.17736623)
\curveto(420.06611292,34.15736457)(420.13611285,34.1323646)(420.21611206,34.10236623)
\curveto(420.2861127,34.08236465)(420.36111262,34.05736467)(420.44111206,34.02736623)
\curveto(420.52111246,33.99736473)(420.59611239,33.96236477)(420.66611206,33.92236623)
\curveto(420.72611226,33.89236484)(420.77111221,33.85736487)(420.80111206,33.81736623)
\curveto(420.86111212,33.73736499)(420.89611209,33.6273651)(420.90611206,33.48736623)
\lineto(420.90611206,33.06736623)
\lineto(420.90611206,32.82736623)
\curveto(420.89611209,32.75736597)(420.87111211,32.69736603)(420.83111206,32.64736623)
\curveto(420.80111218,32.59736613)(420.75611223,32.56736616)(420.69611206,32.55736623)
\curveto(420.63611235,32.55736617)(420.57611241,32.56236617)(420.51611206,32.57236623)
\curveto(420.44611254,32.59236614)(420.3811126,32.61236612)(420.32111206,32.63236623)
\curveto(420.25111273,32.66236607)(420.20111278,32.68736604)(420.17111206,32.70736623)
\curveto(419.85111313,32.84736588)(419.53611345,32.97236576)(419.22611206,33.08236623)
\curveto(418.90611408,33.19236554)(418.5861144,33.31236542)(418.26611206,33.44236623)
\curveto(418.04611494,33.5323652)(417.83111515,33.61736511)(417.62111206,33.69736623)
\curveto(417.40111558,33.77736495)(417.1811158,33.86236487)(416.96111206,33.95236623)
\curveto(416.24111674,34.25236448)(415.51611747,34.53736419)(414.78611206,34.80736623)
\curveto(414.04611894,35.07736365)(413.31111967,35.36236337)(412.58111206,35.66236623)
\curveto(412.32112066,35.77236296)(412.05612093,35.87236286)(411.78611206,35.96236623)
\curveto(411.51612147,36.06236267)(411.25112173,36.16736256)(410.99111206,36.27736623)
\curveto(410.8811221,36.3273624)(410.76112222,36.37236236)(410.63111206,36.41236623)
\curveto(410.49112249,36.46236227)(410.39112259,36.5323622)(410.33111206,36.62236623)
\curveto(410.29112269,36.66236207)(410.26112272,36.727362)(410.24111206,36.81736623)
\curveto(410.23112275,36.83736189)(410.23112275,36.85736187)(410.24111206,36.87736623)
\curveto(410.24112274,36.90736182)(410.23612275,36.9323618)(410.22611206,36.95236623)
\curveto(410.22612276,37.1323616)(410.22612276,37.34236139)(410.22611206,37.58236623)
\curveto(410.21612277,37.82236091)(410.25112273,37.99736073)(410.33111206,38.10736623)
\curveto(410.39112259,38.18736054)(410.49112249,38.24736048)(410.63111206,38.28736623)
\curveto(410.76112222,38.33736039)(410.8811221,38.38736034)(410.99111206,38.43736623)
\curveto(411.22112176,38.53736019)(411.45112153,38.6273601)(411.68111206,38.70736623)
\curveto(411.91112107,38.78735994)(412.14112084,38.87735985)(412.37111206,38.97736623)
\curveto(412.57112041,39.05735967)(412.77612021,39.1323596)(412.98611206,39.20236623)
\curveto(413.19611979,39.28235945)(413.40111958,39.36735936)(413.60111206,39.45736623)
\curveto(414.33111865,39.75735897)(415.07111791,40.04235869)(415.82111206,40.31236623)
\curveto(416.56111642,40.59235814)(417.29611569,40.88735784)(418.02611206,41.19736623)
\curveto(418.11611487,41.23735749)(418.20111478,41.26735746)(418.28111206,41.28736623)
\curveto(418.36111462,41.31735741)(418.44611454,41.34735738)(418.53611206,41.37736623)
\curveto(418.79611419,41.48735724)(419.06111392,41.59235714)(419.33111206,41.69236623)
\curveto(419.60111338,41.80235693)(419.86611312,41.91235682)(420.12611206,42.02236623)
\moveto(416.48111206,38.81236623)
\curveto(416.45111653,38.90235983)(416.40111658,38.95735977)(416.33111206,38.97736623)
\curveto(416.26111672,39.00735972)(416.1861168,39.01235972)(416.10611206,38.99236623)
\curveto(416.01611697,38.98235975)(415.93111705,38.95735977)(415.85111206,38.91736623)
\curveto(415.76111722,38.88735984)(415.6861173,38.85735987)(415.62611206,38.82736623)
\curveto(415.5861174,38.80735992)(415.55111743,38.79735993)(415.52111206,38.79736623)
\curveto(415.49111749,38.79735993)(415.45611753,38.78735994)(415.41611206,38.76736623)
\lineto(415.17611206,38.67736623)
\curveto(415.0861179,38.65736007)(414.99611799,38.6273601)(414.90611206,38.58736623)
\curveto(414.54611844,38.43736029)(414.1811188,38.30236043)(413.81111206,38.18236623)
\curveto(413.43111955,38.07236066)(413.06111992,37.94236079)(412.70111206,37.79236623)
\curveto(412.59112039,37.74236099)(412.4811205,37.69736103)(412.37111206,37.65736623)
\curveto(412.26112072,37.6273611)(412.15612083,37.58736114)(412.05611206,37.53736623)
\curveto(412.00612098,37.51736121)(411.96112102,37.49236124)(411.92111206,37.46236623)
\curveto(411.87112111,37.44236129)(411.84612114,37.39236134)(411.84611206,37.31236623)
\curveto(411.86612112,37.29236144)(411.8811211,37.27236146)(411.89111206,37.25236623)
\curveto(411.90112108,37.2323615)(411.91612107,37.21236152)(411.93611206,37.19236623)
\curveto(411.986121,37.15236158)(412.04112094,37.12236161)(412.10111206,37.10236623)
\curveto(412.15112083,37.08236165)(412.20612078,37.06236167)(412.26611206,37.04236623)
\curveto(412.37612061,36.99236174)(412.4861205,36.95236178)(412.59611206,36.92236623)
\curveto(412.70612028,36.89236184)(412.81612017,36.85236188)(412.92611206,36.80236623)
\curveto(413.31611967,36.6323621)(413.71111927,36.48236225)(414.11111206,36.35236623)
\curveto(414.51111847,36.2323625)(414.90111808,36.09236264)(415.28111206,35.93236623)
\lineto(415.43111206,35.87236623)
\curveto(415.4811175,35.86236287)(415.53111745,35.84736288)(415.58111206,35.82736623)
\lineto(415.82111206,35.73736623)
\curveto(415.90111708,35.70736302)(415.981117,35.68236305)(416.06111206,35.66236623)
\curveto(416.11111687,35.64236309)(416.16611682,35.6323631)(416.22611206,35.63236623)
\curveto(416.2861167,35.64236309)(416.33611665,35.65736307)(416.37611206,35.67736623)
\curveto(416.45611653,35.727363)(416.50111648,35.8323629)(416.51111206,35.99236623)
\lineto(416.51111206,36.44236623)
\lineto(416.51111206,38.04736623)
\curveto(416.51111647,38.15736057)(416.51611647,38.29236044)(416.52611206,38.45236623)
\curveto(416.52611646,38.61236012)(416.51111647,38.73236)(416.48111206,38.81236623)
}
}
{
\newrgbcolor{curcolor}{0 0 0}
\pscustom[linestyle=none,fillstyle=solid,fillcolor=curcolor]
{
\newpath
\moveto(416.87111206,50.56392873)
\curveto(416.92111606,50.57392038)(416.99111599,50.57892038)(417.08111206,50.57892873)
\curveto(417.16111582,50.57892038)(417.22611576,50.57392038)(417.27611206,50.56392873)
\curveto(417.31611567,50.56392039)(417.35611563,50.5589204)(417.39611206,50.54892873)
\lineto(417.51611206,50.54892873)
\curveto(417.59611539,50.52892043)(417.67611531,50.51892044)(417.75611206,50.51892873)
\curveto(417.83611515,50.51892044)(417.91611507,50.50892045)(417.99611206,50.48892873)
\curveto(418.03611495,50.47892048)(418.07611491,50.47392048)(418.11611206,50.47392873)
\curveto(418.14611484,50.47392048)(418.1811148,50.46892049)(418.22111206,50.45892873)
\curveto(418.33111465,50.42892053)(418.43611455,50.39892056)(418.53611206,50.36892873)
\curveto(418.63611435,50.34892061)(418.73611425,50.31892064)(418.83611206,50.27892873)
\curveto(419.1861138,50.13892082)(419.50111348,49.96892099)(419.78111206,49.76892873)
\curveto(420.06111292,49.56892139)(420.30111268,49.31892164)(420.50111206,49.01892873)
\curveto(420.60111238,48.86892209)(420.6861123,48.72392223)(420.75611206,48.58392873)
\curveto(420.80611218,48.47392248)(420.84611214,48.36392259)(420.87611206,48.25392873)
\curveto(420.90611208,48.1539228)(420.93611205,48.04892291)(420.96611206,47.93892873)
\curveto(420.986112,47.86892309)(420.99611199,47.80392315)(420.99611206,47.74392873)
\curveto(421.00611198,47.68392327)(421.02111196,47.62392333)(421.04111206,47.56392873)
\lineto(421.04111206,47.41392873)
\curveto(421.06111192,47.36392359)(421.07111191,47.28892367)(421.07111206,47.18892873)
\curveto(421.0811119,47.08892387)(421.07611191,47.00892395)(421.05611206,46.94892873)
\lineto(421.05611206,46.79892873)
\curveto(421.04611194,46.7589242)(421.04111194,46.71392424)(421.04111206,46.66392873)
\curveto(421.04111194,46.62392433)(421.03611195,46.57892438)(421.02611206,46.52892873)
\curveto(420.986112,46.37892458)(420.95111203,46.22892473)(420.92111206,46.07892873)
\curveto(420.89111209,45.93892502)(420.84611214,45.79892516)(420.78611206,45.65892873)
\curveto(420.70611228,45.4589255)(420.60611238,45.27892568)(420.48611206,45.11892873)
\lineto(420.33611206,44.93892873)
\curveto(420.27611271,44.87892608)(420.23611275,44.80892615)(420.21611206,44.72892873)
\curveto(420.20611278,44.66892629)(420.22111276,44.61892634)(420.26111206,44.57892873)
\curveto(420.29111269,44.54892641)(420.33611265,44.52392643)(420.39611206,44.50392873)
\curveto(420.45611253,44.49392646)(420.52111246,44.48392647)(420.59111206,44.47392873)
\curveto(420.65111233,44.47392648)(420.69611229,44.46392649)(420.72611206,44.44392873)
\curveto(420.77611221,44.40392655)(420.82111216,44.3589266)(420.86111206,44.30892873)
\curveto(420.8811121,44.2589267)(420.89611209,44.18892677)(420.90611206,44.09892873)
\lineto(420.90611206,43.82892873)
\curveto(420.90611208,43.73892722)(420.90111208,43.6539273)(420.89111206,43.57392873)
\curveto(420.87111211,43.49392746)(420.85111213,43.43392752)(420.83111206,43.39392873)
\curveto(420.81111217,43.37392758)(420.7861122,43.3539276)(420.75611206,43.33392873)
\lineto(420.66611206,43.27392873)
\curveto(420.5861124,43.24392771)(420.46611252,43.22892773)(420.30611206,43.22892873)
\curveto(420.14611284,43.23892772)(420.01111297,43.24392771)(419.90111206,43.24392873)
\lineto(411.09611206,43.24392873)
\curveto(410.97612201,43.24392771)(410.85112213,43.23892772)(410.72111206,43.22892873)
\curveto(410.5811224,43.22892773)(410.47112251,43.2539277)(410.39111206,43.30392873)
\curveto(410.33112265,43.34392761)(410.2811227,43.40892755)(410.24111206,43.49892873)
\curveto(410.24112274,43.51892744)(410.24112274,43.54392741)(410.24111206,43.57392873)
\curveto(410.23112275,43.60392735)(410.22612276,43.62892733)(410.22611206,43.64892873)
\curveto(410.21612277,43.78892717)(410.21612277,43.93392702)(410.22611206,44.08392873)
\curveto(410.22612276,44.24392671)(410.26612272,44.3539266)(410.34611206,44.41392873)
\curveto(410.42612256,44.46392649)(410.54112244,44.48892647)(410.69111206,44.48892873)
\lineto(411.09611206,44.48892873)
\lineto(412.85111206,44.48892873)
\lineto(413.10611206,44.48892873)
\lineto(413.39111206,44.48892873)
\curveto(413.4811195,44.49892646)(413.56611942,44.50892645)(413.64611206,44.51892873)
\curveto(413.71611927,44.53892642)(413.76611922,44.56892639)(413.79611206,44.60892873)
\curveto(413.82611916,44.64892631)(413.83111915,44.69392626)(413.81111206,44.74392873)
\curveto(413.79111919,44.79392616)(413.77111921,44.83392612)(413.75111206,44.86392873)
\curveto(413.71111927,44.91392604)(413.67111931,44.958926)(413.63111206,44.99892873)
\lineto(413.51111206,45.14892873)
\curveto(413.46111952,45.21892574)(413.41611957,45.28892567)(413.37611206,45.35892873)
\lineto(413.25611206,45.59892873)
\curveto(413.16611982,45.77892518)(413.10111988,45.99392496)(413.06111206,46.24392873)
\curveto(413.02111996,46.49392446)(413.00111998,46.74892421)(413.00111206,47.00892873)
\curveto(413.00111998,47.26892369)(413.02611996,47.52392343)(413.07611206,47.77392873)
\curveto(413.11611987,48.02392293)(413.17611981,48.24392271)(413.25611206,48.43392873)
\curveto(413.42611956,48.83392212)(413.66111932,49.17892178)(413.96111206,49.46892873)
\curveto(414.26111872,49.7589212)(414.61111837,49.98892097)(415.01111206,50.15892873)
\curveto(415.12111786,50.20892075)(415.23111775,50.24892071)(415.34111206,50.27892873)
\curveto(415.44111754,50.31892064)(415.54611744,50.3589206)(415.65611206,50.39892873)
\curveto(415.76611722,50.42892053)(415.8811171,50.44892051)(416.00111206,50.45892873)
\lineto(416.33111206,50.51892873)
\curveto(416.36111662,50.52892043)(416.39611659,50.53392042)(416.43611206,50.53392873)
\curveto(416.46611652,50.53392042)(416.49611649,50.53892042)(416.52611206,50.54892873)
\curveto(416.5861164,50.56892039)(416.64611634,50.56892039)(416.70611206,50.54892873)
\curveto(416.75611623,50.53892042)(416.81111617,50.54392041)(416.87111206,50.56392873)
\moveto(417.26111206,49.22892873)
\curveto(417.21111577,49.24892171)(417.15111583,49.2539217)(417.08111206,49.24392873)
\curveto(417.01111597,49.23392172)(416.94611604,49.22892173)(416.88611206,49.22892873)
\curveto(416.71611627,49.22892173)(416.55611643,49.21892174)(416.40611206,49.19892873)
\curveto(416.25611673,49.18892177)(416.12111686,49.1589218)(416.00111206,49.10892873)
\curveto(415.90111708,49.07892188)(415.81111717,49.0539219)(415.73111206,49.03392873)
\curveto(415.65111733,49.01392194)(415.57111741,48.98392197)(415.49111206,48.94392873)
\curveto(415.24111774,48.83392212)(415.01111797,48.68392227)(414.80111206,48.49392873)
\curveto(414.5811184,48.30392265)(414.41611857,48.08392287)(414.30611206,47.83392873)
\curveto(414.27611871,47.7539232)(414.25111873,47.67392328)(414.23111206,47.59392873)
\curveto(414.20111878,47.52392343)(414.17611881,47.44892351)(414.15611206,47.36892873)
\curveto(414.12611886,47.2589237)(414.11111887,47.14892381)(414.11111206,47.03892873)
\curveto(414.10111888,46.92892403)(414.09611889,46.80892415)(414.09611206,46.67892873)
\curveto(414.10611888,46.62892433)(414.11611887,46.58392437)(414.12611206,46.54392873)
\lineto(414.12611206,46.40892873)
\lineto(414.18611206,46.13892873)
\curveto(414.20611878,46.0589249)(414.23611875,45.97892498)(414.27611206,45.89892873)
\curveto(414.41611857,45.5589254)(414.62611836,45.28892567)(414.90611206,45.08892873)
\curveto(415.17611781,44.88892607)(415.49611749,44.72892623)(415.86611206,44.60892873)
\curveto(415.97611701,44.56892639)(416.0861169,44.54392641)(416.19611206,44.53392873)
\curveto(416.30611668,44.52392643)(416.42111656,44.50392645)(416.54111206,44.47392873)
\curveto(416.59111639,44.46392649)(416.63611635,44.46392649)(416.67611206,44.47392873)
\curveto(416.71611627,44.48392647)(416.76111622,44.47892648)(416.81111206,44.45892873)
\curveto(416.86111612,44.44892651)(416.93611605,44.44392651)(417.03611206,44.44392873)
\curveto(417.12611586,44.44392651)(417.19611579,44.44892651)(417.24611206,44.45892873)
\lineto(417.36611206,44.45892873)
\curveto(417.40611558,44.46892649)(417.44611554,44.47392648)(417.48611206,44.47392873)
\curveto(417.52611546,44.47392648)(417.56111542,44.47892648)(417.59111206,44.48892873)
\curveto(417.62111536,44.49892646)(417.65611533,44.50392645)(417.69611206,44.50392873)
\curveto(417.72611526,44.50392645)(417.75611523,44.50892645)(417.78611206,44.51892873)
\curveto(417.86611512,44.53892642)(417.94611504,44.5539264)(418.02611206,44.56392873)
\lineto(418.26611206,44.62392873)
\curveto(418.60611438,44.73392622)(418.89611409,44.88392607)(419.13611206,45.07392873)
\curveto(419.37611361,45.27392568)(419.57611341,45.51892544)(419.73611206,45.80892873)
\curveto(419.7861132,45.89892506)(419.82611316,45.99392496)(419.85611206,46.09392873)
\curveto(419.87611311,46.19392476)(419.90111308,46.29892466)(419.93111206,46.40892873)
\curveto(419.95111303,46.4589245)(419.96111302,46.50392445)(419.96111206,46.54392873)
\curveto(419.95111303,46.59392436)(419.95111303,46.64392431)(419.96111206,46.69392873)
\curveto(419.97111301,46.73392422)(419.97611301,46.77892418)(419.97611206,46.82892873)
\lineto(419.97611206,46.96392873)
\lineto(419.97611206,47.09892873)
\curveto(419.96611302,47.13892382)(419.96111302,47.17392378)(419.96111206,47.20392873)
\curveto(419.96111302,47.23392372)(419.95611303,47.26892369)(419.94611206,47.30892873)
\curveto(419.92611306,47.38892357)(419.91111307,47.46392349)(419.90111206,47.53392873)
\curveto(419.8811131,47.60392335)(419.85611313,47.67892328)(419.82611206,47.75892873)
\curveto(419.69611329,48.06892289)(419.52611346,48.31892264)(419.31611206,48.50892873)
\curveto(419.09611389,48.69892226)(418.83111415,48.8589221)(418.52111206,48.98892873)
\curveto(418.3811146,49.03892192)(418.24111474,49.07392188)(418.10111206,49.09392873)
\curveto(417.95111503,49.12392183)(417.80111518,49.1589218)(417.65111206,49.19892873)
\curveto(417.60111538,49.21892174)(417.55611543,49.22392173)(417.51611206,49.21392873)
\curveto(417.46611552,49.21392174)(417.41611557,49.21892174)(417.36611206,49.22892873)
\lineto(417.26111206,49.22892873)
}
}
{
\newrgbcolor{curcolor}{0 0 0}
\pscustom[linestyle=none,fillstyle=solid,fillcolor=curcolor]
{
\newpath
\moveto(413.00111206,55.69017873)
\curveto(413.00111998,55.92017394)(413.06111992,56.05017381)(413.18111206,56.08017873)
\curveto(413.29111969,56.11017375)(413.45611953,56.12517374)(413.67611206,56.12517873)
\lineto(413.96111206,56.12517873)
\curveto(414.05111893,56.12517374)(414.12611886,56.10017376)(414.18611206,56.05017873)
\curveto(414.26611872,55.99017387)(414.31111867,55.90517396)(414.32111206,55.79517873)
\curveto(414.32111866,55.68517418)(414.33611865,55.57517429)(414.36611206,55.46517873)
\curveto(414.39611859,55.32517454)(414.42611856,55.19017467)(414.45611206,55.06017873)
\curveto(414.4861185,54.94017492)(414.52611846,54.82517504)(414.57611206,54.71517873)
\curveto(414.70611828,54.42517544)(414.8861181,54.19017567)(415.11611206,54.01017873)
\curveto(415.33611765,53.83017603)(415.59111739,53.67517619)(415.88111206,53.54517873)
\curveto(415.99111699,53.50517636)(416.10611688,53.47517639)(416.22611206,53.45517873)
\curveto(416.33611665,53.43517643)(416.45111653,53.41017645)(416.57111206,53.38017873)
\curveto(416.62111636,53.37017649)(416.67111631,53.3651765)(416.72111206,53.36517873)
\curveto(416.77111621,53.37517649)(416.82111616,53.37517649)(416.87111206,53.36517873)
\curveto(416.99111599,53.33517653)(417.13111585,53.32017654)(417.29111206,53.32017873)
\curveto(417.44111554,53.33017653)(417.5861154,53.33517653)(417.72611206,53.33517873)
\lineto(419.57111206,53.33517873)
\lineto(419.91611206,53.33517873)
\curveto(420.03611295,53.33517653)(420.15111283,53.33017653)(420.26111206,53.32017873)
\curveto(420.37111261,53.31017655)(420.46611252,53.30517656)(420.54611206,53.30517873)
\curveto(420.62611236,53.31517655)(420.69611229,53.29517657)(420.75611206,53.24517873)
\curveto(420.82611216,53.19517667)(420.86611212,53.11517675)(420.87611206,53.00517873)
\curveto(420.8861121,52.90517696)(420.89111209,52.79517707)(420.89111206,52.67517873)
\lineto(420.89111206,52.40517873)
\curveto(420.87111211,52.35517751)(420.85611213,52.30517756)(420.84611206,52.25517873)
\curveto(420.82611216,52.21517765)(420.80111218,52.18517768)(420.77111206,52.16517873)
\curveto(420.70111228,52.11517775)(420.61611237,52.08517778)(420.51611206,52.07517873)
\lineto(420.18611206,52.07517873)
\lineto(419.03111206,52.07517873)
\lineto(414.87611206,52.07517873)
\lineto(413.84111206,52.07517873)
\lineto(413.54111206,52.07517873)
\curveto(413.44111954,52.08517778)(413.35611963,52.11517775)(413.28611206,52.16517873)
\curveto(413.24611974,52.19517767)(413.21611977,52.24517762)(413.19611206,52.31517873)
\curveto(413.17611981,52.39517747)(413.16611982,52.48017738)(413.16611206,52.57017873)
\curveto(413.15611983,52.6601772)(413.15611983,52.75017711)(413.16611206,52.84017873)
\curveto(413.17611981,52.93017693)(413.19111979,53.00017686)(413.21111206,53.05017873)
\curveto(413.24111974,53.13017673)(413.30111968,53.18017668)(413.39111206,53.20017873)
\curveto(413.47111951,53.23017663)(413.56111942,53.24517662)(413.66111206,53.24517873)
\lineto(413.96111206,53.24517873)
\curveto(414.06111892,53.24517662)(414.15111883,53.2651766)(414.23111206,53.30517873)
\curveto(414.25111873,53.31517655)(414.26611872,53.32517654)(414.27611206,53.33517873)
\lineto(414.32111206,53.38017873)
\curveto(414.32111866,53.49017637)(414.27611871,53.58017628)(414.18611206,53.65017873)
\curveto(414.0861189,53.72017614)(414.00611898,53.78017608)(413.94611206,53.83017873)
\lineto(413.85611206,53.92017873)
\curveto(413.74611924,54.01017585)(413.63111935,54.13517573)(413.51111206,54.29517873)
\curveto(413.39111959,54.45517541)(413.30111968,54.60517526)(413.24111206,54.74517873)
\curveto(413.19111979,54.83517503)(413.15611983,54.93017493)(413.13611206,55.03017873)
\curveto(413.10611988,55.13017473)(413.07611991,55.23517463)(413.04611206,55.34517873)
\curveto(413.03611995,55.40517446)(413.03111995,55.4651744)(413.03111206,55.52517873)
\curveto(413.02111996,55.58517428)(413.01111997,55.64017422)(413.00111206,55.69017873)
}
}
{
\newrgbcolor{curcolor}{0 0 0}
\pscustom[linestyle=none,fillstyle=solid,fillcolor=curcolor]
{
}
}
{
\newrgbcolor{curcolor}{0 0 0}
\pscustom[linestyle=none,fillstyle=solid,fillcolor=curcolor]
{
\newpath
\moveto(410.30111206,65.05510061)
\curveto(410.30112268,65.15509575)(410.31112267,65.25009566)(410.33111206,65.34010061)
\curveto(410.34112264,65.43009548)(410.37112261,65.49509541)(410.42111206,65.53510061)
\curveto(410.50112248,65.59509531)(410.60612238,65.62509528)(410.73611206,65.62510061)
\lineto(411.12611206,65.62510061)
\lineto(412.62611206,65.62510061)
\lineto(419.01611206,65.62510061)
\lineto(420.18611206,65.62510061)
\lineto(420.50111206,65.62510061)
\curveto(420.60111238,65.63509527)(420.6811123,65.62009529)(420.74111206,65.58010061)
\curveto(420.82111216,65.53009538)(420.87111211,65.45509545)(420.89111206,65.35510061)
\curveto(420.90111208,65.26509564)(420.90611208,65.15509575)(420.90611206,65.02510061)
\lineto(420.90611206,64.80010061)
\curveto(420.8861121,64.72009619)(420.87111211,64.65009626)(420.86111206,64.59010061)
\curveto(420.84111214,64.53009638)(420.80111218,64.48009643)(420.74111206,64.44010061)
\curveto(420.6811123,64.40009651)(420.60611238,64.38009653)(420.51611206,64.38010061)
\lineto(420.21611206,64.38010061)
\lineto(419.12111206,64.38010061)
\lineto(413.78111206,64.38010061)
\curveto(413.69111929,64.36009655)(413.61611937,64.34509656)(413.55611206,64.33510061)
\curveto(413.4861195,64.33509657)(413.42611956,64.3050966)(413.37611206,64.24510061)
\curveto(413.32611966,64.17509673)(413.30111968,64.08509682)(413.30111206,63.97510061)
\curveto(413.29111969,63.87509703)(413.2861197,63.76509714)(413.28611206,63.64510061)
\lineto(413.28611206,62.50510061)
\lineto(413.28611206,62.01010061)
\curveto(413.27611971,61.85009906)(413.21611977,61.74009917)(413.10611206,61.68010061)
\curveto(413.07611991,61.66009925)(413.04611994,61.65009926)(413.01611206,61.65010061)
\curveto(412.97612001,61.65009926)(412.93112005,61.64509926)(412.88111206,61.63510061)
\curveto(412.76112022,61.61509929)(412.65112033,61.62009929)(412.55111206,61.65010061)
\curveto(412.45112053,61.69009922)(412.3811206,61.74509916)(412.34111206,61.81510061)
\curveto(412.29112069,61.89509901)(412.26612072,62.01509889)(412.26611206,62.17510061)
\curveto(412.26612072,62.33509857)(412.25112073,62.47009844)(412.22111206,62.58010061)
\curveto(412.21112077,62.63009828)(412.20612078,62.68509822)(412.20611206,62.74510061)
\curveto(412.19612079,62.8050981)(412.1811208,62.86509804)(412.16111206,62.92510061)
\curveto(412.11112087,63.07509783)(412.06112092,63.22009769)(412.01111206,63.36010061)
\curveto(411.95112103,63.50009741)(411.8811211,63.63509727)(411.80111206,63.76510061)
\curveto(411.71112127,63.905097)(411.60612138,64.02509688)(411.48611206,64.12510061)
\curveto(411.36612162,64.22509668)(411.23612175,64.32009659)(411.09611206,64.41010061)
\curveto(410.99612199,64.47009644)(410.8861221,64.51509639)(410.76611206,64.54510061)
\curveto(410.64612234,64.58509632)(410.54112244,64.63509627)(410.45111206,64.69510061)
\curveto(410.39112259,64.74509616)(410.35112263,64.81509609)(410.33111206,64.90510061)
\curveto(410.32112266,64.92509598)(410.31612267,64.95009596)(410.31611206,64.98010061)
\curveto(410.31612267,65.0100959)(410.31112267,65.03509587)(410.30111206,65.05510061)
}
}
{
\newrgbcolor{curcolor}{0 0 0}
\pscustom[linestyle=none,fillstyle=solid,fillcolor=curcolor]
{
\newpath
\moveto(415.31111206,76.28470998)
\curveto(415.39111759,76.28470235)(415.47111751,76.28970234)(415.55111206,76.29970998)
\curveto(415.63111735,76.30970232)(415.70611728,76.30470233)(415.77611206,76.28470998)
\curveto(415.81611717,76.26470237)(415.86111712,76.25970237)(415.91111206,76.26970998)
\curveto(415.95111703,76.27970235)(415.99111699,76.27970235)(416.03111206,76.26970998)
\lineto(416.18111206,76.26970998)
\curveto(416.27111671,76.25970237)(416.36111662,76.25470238)(416.45111206,76.25470998)
\curveto(416.53111645,76.25470238)(416.61111637,76.24970238)(416.69111206,76.23970998)
\lineto(416.93111206,76.20970998)
\curveto(417.00111598,76.19970243)(417.07611591,76.18970244)(417.15611206,76.17970998)
\curveto(417.19611579,76.16970246)(417.23611575,76.16470247)(417.27611206,76.16470998)
\curveto(417.31611567,76.16470247)(417.36111562,76.15970247)(417.41111206,76.14970998)
\curveto(417.55111543,76.10970252)(417.69111529,76.07970255)(417.83111206,76.05970998)
\curveto(417.97111501,76.04970258)(418.10611488,76.01970261)(418.23611206,75.96970998)
\curveto(418.40611458,75.91970271)(418.57111441,75.86470277)(418.73111206,75.80470998)
\curveto(418.89111409,75.75470288)(419.04611394,75.69470294)(419.19611206,75.62470998)
\curveto(419.25611373,75.60470303)(419.31611367,75.57470306)(419.37611206,75.53470998)
\lineto(419.52611206,75.44470998)
\curveto(419.84611314,75.24470339)(420.11111287,75.0297036)(420.32111206,74.79970998)
\curveto(420.53111245,74.56970406)(420.71111227,74.27470436)(420.86111206,73.91470998)
\curveto(420.91111207,73.79470484)(420.94611204,73.66470497)(420.96611206,73.52470998)
\curveto(420.986112,73.39470524)(421.01111197,73.25970537)(421.04111206,73.11970998)
\curveto(421.05111193,73.05970557)(421.05611193,72.99970563)(421.05611206,72.93970998)
\curveto(421.05611193,72.87970575)(421.06111192,72.81470582)(421.07111206,72.74470998)
\curveto(421.0811119,72.71470592)(421.0811119,72.66470597)(421.07111206,72.59470998)
\lineto(421.07111206,72.44470998)
\lineto(421.07111206,72.29470998)
\curveto(421.05111193,72.21470642)(421.03611195,72.1297065)(421.02611206,72.03970998)
\curveto(421.02611196,71.95970667)(421.01611197,71.88470675)(420.99611206,71.81470998)
\curveto(420.986112,71.77470686)(420.981112,71.73970689)(420.98111206,71.70970998)
\curveto(420.99111199,71.68970694)(420.986112,71.66470697)(420.96611206,71.63470998)
\lineto(420.90611206,71.36470998)
\curveto(420.87611211,71.27470736)(420.84611214,71.18970744)(420.81611206,71.10970998)
\curveto(420.57611241,70.5297081)(420.20611278,70.09470854)(419.70611206,69.80470998)
\curveto(419.57611341,69.72470891)(419.44111354,69.65970897)(419.30111206,69.60970998)
\curveto(419.16111382,69.56970906)(419.01111397,69.52470911)(418.85111206,69.47470998)
\curveto(418.77111421,69.45470918)(418.69111429,69.44970918)(418.61111206,69.45970998)
\curveto(418.53111445,69.47970915)(418.47611451,69.51470912)(418.44611206,69.56470998)
\curveto(418.42611456,69.59470904)(418.41111457,69.64970898)(418.40111206,69.72970998)
\curveto(418.3811146,69.80970882)(418.37111461,69.89470874)(418.37111206,69.98470998)
\curveto(418.36111462,70.07470856)(418.36111462,70.15970847)(418.37111206,70.23970998)
\curveto(418.3811146,70.3297083)(418.39111459,70.39970823)(418.40111206,70.44970998)
\curveto(418.41111457,70.46970816)(418.42611456,70.49470814)(418.44611206,70.52470998)
\curveto(418.46611452,70.56470807)(418.4861145,70.59470804)(418.50611206,70.61470998)
\curveto(418.5861144,70.67470796)(418.6811143,70.71970791)(418.79111206,70.74970998)
\curveto(418.90111408,70.78970784)(419.00111398,70.8347078)(419.09111206,70.88470998)
\curveto(419.4811135,71.1347075)(419.75111323,71.50470713)(419.90111206,71.99470998)
\curveto(419.92111306,72.06470657)(419.93611305,72.1347065)(419.94611206,72.20470998)
\curveto(419.94611304,72.28470635)(419.95611303,72.36470627)(419.97611206,72.44470998)
\curveto(419.986113,72.48470615)(419.99111299,72.53970609)(419.99111206,72.60970998)
\curveto(419.99111299,72.68970594)(419.986113,72.74470589)(419.97611206,72.77470998)
\curveto(419.96611302,72.80470583)(419.96111302,72.8347058)(419.96111206,72.86470998)
\lineto(419.96111206,72.96970998)
\curveto(419.94111304,73.04970558)(419.92111306,73.12470551)(419.90111206,73.19470998)
\curveto(419.8811131,73.27470536)(419.85611313,73.34970528)(419.82611206,73.41970998)
\curveto(419.67611331,73.76970486)(419.46111352,74.03970459)(419.18111206,74.22970998)
\curveto(418.90111408,74.41970421)(418.57611441,74.57470406)(418.20611206,74.69470998)
\curveto(418.12611486,74.72470391)(418.05111493,74.74470389)(417.98111206,74.75470998)
\curveto(417.91111507,74.77470386)(417.83611515,74.79470384)(417.75611206,74.81470998)
\curveto(417.66611532,74.8347038)(417.57111541,74.84970378)(417.47111206,74.85970998)
\curveto(417.36111562,74.87970375)(417.25611573,74.89970373)(417.15611206,74.91970998)
\curveto(417.10611588,74.9297037)(417.05611593,74.9347037)(417.00611206,74.93470998)
\curveto(416.94611604,74.94470369)(416.89111609,74.94970368)(416.84111206,74.94970998)
\curveto(416.7811162,74.96970366)(416.70611628,74.97970365)(416.61611206,74.97970998)
\curveto(416.51611647,74.97970365)(416.43611655,74.96970366)(416.37611206,74.94970998)
\curveto(416.2861167,74.91970371)(416.24611674,74.86970376)(416.25611206,74.79970998)
\curveto(416.26611672,74.73970389)(416.29611669,74.68470395)(416.34611206,74.63470998)
\curveto(416.39611659,74.55470408)(416.45611653,74.48470415)(416.52611206,74.42470998)
\curveto(416.59611639,74.37470426)(416.65611633,74.30970432)(416.70611206,74.22970998)
\curveto(416.81611617,74.06970456)(416.91611607,73.90470473)(417.00611206,73.73470998)
\curveto(417.0861159,73.56470507)(417.15611583,73.36970526)(417.21611206,73.14970998)
\curveto(417.24611574,73.04970558)(417.26111572,72.94970568)(417.26111206,72.84970998)
\curveto(417.26111572,72.75970587)(417.27111571,72.65970597)(417.29111206,72.54970998)
\lineto(417.29111206,72.39970998)
\curveto(417.27111571,72.34970628)(417.26611572,72.29970633)(417.27611206,72.24970998)
\curveto(417.2861157,72.20970642)(417.2861157,72.16970646)(417.27611206,72.12970998)
\curveto(417.26611572,72.09970653)(417.26111572,72.05470658)(417.26111206,71.99470998)
\curveto(417.25111573,71.9347067)(417.24111574,71.86970676)(417.23111206,71.79970998)
\lineto(417.20111206,71.61970998)
\curveto(417.0811159,71.16970746)(416.91611607,70.78970784)(416.70611206,70.47970998)
\curveto(416.51611647,70.20970842)(416.2861167,69.97970865)(416.01611206,69.78970998)
\curveto(415.73611725,69.60970902)(415.42111756,69.46470917)(415.07111206,69.35470998)
\lineto(414.86111206,69.29470998)
\curveto(414.7811182,69.28470935)(414.70111828,69.26970936)(414.62111206,69.24970998)
\curveto(414.59111839,69.23970939)(414.56111842,69.2347094)(414.53111206,69.23470998)
\curveto(414.50111848,69.2347094)(414.47111851,69.2297094)(414.44111206,69.21970998)
\curveto(414.3811186,69.20970942)(414.32111866,69.20470943)(414.26111206,69.20470998)
\curveto(414.19111879,69.20470943)(414.13111885,69.19470944)(414.08111206,69.17470998)
\lineto(413.90111206,69.17470998)
\curveto(413.85111913,69.16470947)(413.7811192,69.15970947)(413.69111206,69.15970998)
\curveto(413.60111938,69.15970947)(413.53111945,69.16970946)(413.48111206,69.18970998)
\lineto(413.31611206,69.18970998)
\curveto(413.23611975,69.20970942)(413.16111982,69.21970941)(413.09111206,69.21970998)
\curveto(413.02111996,69.2297094)(412.95112003,69.24470939)(412.88111206,69.26470998)
\curveto(412.6811203,69.32470931)(412.49112049,69.38470925)(412.31111206,69.44470998)
\curveto(412.13112085,69.51470912)(411.96112102,69.60470903)(411.80111206,69.71470998)
\curveto(411.73112125,69.75470888)(411.66612132,69.79470884)(411.60611206,69.83470998)
\lineto(411.42611206,69.98470998)
\curveto(411.41612157,70.00470863)(411.40112158,70.02470861)(411.38111206,70.04470998)
\curveto(411.25112173,70.1347085)(411.14112184,70.24470839)(411.05111206,70.37470998)
\curveto(410.85112213,70.634708)(410.69612229,70.89970773)(410.58611206,71.16970998)
\curveto(410.54612244,71.24970738)(410.51612247,71.3297073)(410.49611206,71.40970998)
\curveto(410.46612252,71.49970713)(410.44112254,71.58970704)(410.42111206,71.67970998)
\curveto(410.39112259,71.77970685)(410.37112261,71.87970675)(410.36111206,71.97970998)
\curveto(410.35112263,72.07970655)(410.33612265,72.18470645)(410.31611206,72.29470998)
\curveto(410.30612268,72.32470631)(410.30612268,72.36470627)(410.31611206,72.41470998)
\curveto(410.32612266,72.47470616)(410.32112266,72.51470612)(410.30111206,72.53470998)
\curveto(410.2811227,73.25470538)(410.39612259,73.85470478)(410.64611206,74.33470998)
\curveto(410.89612209,74.81470382)(411.23612175,75.18970344)(411.66611206,75.45970998)
\curveto(411.80612118,75.54970308)(411.95112103,75.629703)(412.10111206,75.69970998)
\curveto(412.25112073,75.76970286)(412.41112057,75.83970279)(412.58111206,75.90970998)
\curveto(412.72112026,75.95970267)(412.87112011,75.99970263)(413.03111206,76.02970998)
\curveto(413.19111979,76.05970257)(413.35111963,76.09470254)(413.51111206,76.13470998)
\curveto(413.56111942,76.15470248)(413.61611937,76.16470247)(413.67611206,76.16470998)
\curveto(413.72611926,76.16470247)(413.77611921,76.16970246)(413.82611206,76.17970998)
\curveto(413.8861191,76.19970243)(413.95111903,76.20970242)(414.02111206,76.20970998)
\curveto(414.0811189,76.20970242)(414.13611885,76.21970241)(414.18611206,76.23970998)
\lineto(414.35111206,76.23970998)
\curveto(414.40111858,76.25970237)(414.45111853,76.26470237)(414.50111206,76.25470998)
\curveto(414.55111843,76.24470239)(414.60111838,76.24970238)(414.65111206,76.26970998)
\curveto(414.67111831,76.26970236)(414.69611829,76.26470237)(414.72611206,76.25470998)
\curveto(414.75611823,76.25470238)(414.7811182,76.25970237)(414.80111206,76.26970998)
\curveto(414.83111815,76.27970235)(414.86611812,76.27970235)(414.90611206,76.26970998)
\curveto(414.94611804,76.26970236)(414.986118,76.27470236)(415.02611206,76.28470998)
\curveto(415.06611792,76.29470234)(415.11111787,76.29470234)(415.16111206,76.28470998)
\lineto(415.31111206,76.28470998)
\moveto(414.00611206,74.78470998)
\curveto(413.95611903,74.79470384)(413.89611909,74.79970383)(413.82611206,74.79970998)
\curveto(413.75611923,74.79970383)(413.69611929,74.79470384)(413.64611206,74.78470998)
\curveto(413.59611939,74.77470386)(413.52111946,74.76970386)(413.42111206,74.76970998)
\curveto(413.34111964,74.74970388)(413.26611972,74.7297039)(413.19611206,74.70970998)
\curveto(413.12611986,74.69970393)(413.05611993,74.68470395)(412.98611206,74.66470998)
\curveto(412.55612043,74.52470411)(412.22112076,74.3297043)(411.98111206,74.07970998)
\curveto(411.74112124,73.83970479)(411.56112142,73.49470514)(411.44111206,73.04470998)
\curveto(411.42112156,72.95470568)(411.41112157,72.85470578)(411.41111206,72.74470998)
\lineto(411.41111206,72.41470998)
\curveto(411.43112155,72.39470624)(411.44112154,72.35970627)(411.44111206,72.30970998)
\curveto(411.43112155,72.25970637)(411.43112155,72.21470642)(411.44111206,72.17470998)
\curveto(411.46112152,72.09470654)(411.4811215,72.01970661)(411.50111206,71.94970998)
\lineto(411.56111206,71.73970998)
\curveto(411.69112129,71.44970718)(411.87112111,71.21970741)(412.10111206,71.04970998)
\curveto(412.32112066,70.87970775)(412.5811204,70.74470789)(412.88111206,70.64470998)
\curveto(412.97112001,70.61470802)(413.06611992,70.58970804)(413.16611206,70.56970998)
\curveto(413.25611973,70.55970807)(413.35111963,70.54470809)(413.45111206,70.52470998)
\lineto(413.58611206,70.52470998)
\curveto(413.69611929,70.49470814)(413.83611915,70.48470815)(414.00611206,70.49470998)
\curveto(414.16611882,70.51470812)(414.29611869,70.5347081)(414.39611206,70.55470998)
\curveto(414.45611853,70.57470806)(414.51611847,70.58970804)(414.57611206,70.59970998)
\curveto(414.62611836,70.60970802)(414.67611831,70.62470801)(414.72611206,70.64470998)
\curveto(414.92611806,70.72470791)(415.11611787,70.81970781)(415.29611206,70.92970998)
\curveto(415.47611751,71.04970758)(415.62111736,71.18970744)(415.73111206,71.34970998)
\curveto(415.7811172,71.39970723)(415.82111716,71.45470718)(415.85111206,71.51470998)
\curveto(415.8811171,71.57470706)(415.91611707,71.634707)(415.95611206,71.69470998)
\curveto(416.03611695,71.84470679)(416.10111688,72.0297066)(416.15111206,72.24970998)
\curveto(416.17111681,72.29970633)(416.17611681,72.33970629)(416.16611206,72.36970998)
\curveto(416.15611683,72.40970622)(416.16111682,72.45470618)(416.18111206,72.50470998)
\curveto(416.19111679,72.54470609)(416.19611679,72.59970603)(416.19611206,72.66970998)
\curveto(416.19611679,72.73970589)(416.19111679,72.79970583)(416.18111206,72.84970998)
\curveto(416.16111682,72.94970568)(416.14611684,73.04470559)(416.13611206,73.13470998)
\curveto(416.11611687,73.22470541)(416.0861169,73.31470532)(416.04611206,73.40470998)
\curveto(415.82611716,73.94470469)(415.43111755,74.33970429)(414.86111206,74.58970998)
\curveto(414.76111822,74.63970399)(414.66111832,74.67470396)(414.56111206,74.69470998)
\curveto(414.45111853,74.71470392)(414.34111864,74.73970389)(414.23111206,74.76970998)
\curveto(414.13111885,74.76970386)(414.05611893,74.77470386)(414.00611206,74.78470998)
}
}
{
\newrgbcolor{curcolor}{0 0 0}
\pscustom[linestyle=none,fillstyle=solid,fillcolor=curcolor]
{
\newpath
\moveto(419.27111206,78.63431936)
\lineto(419.27111206,79.26431936)
\lineto(419.27111206,79.45931936)
\curveto(419.27111371,79.52931683)(419.2811137,79.58931677)(419.30111206,79.63931936)
\curveto(419.34111364,79.70931665)(419.3811136,79.7593166)(419.42111206,79.78931936)
\curveto(419.47111351,79.82931653)(419.53611345,79.84931651)(419.61611206,79.84931936)
\curveto(419.69611329,79.8593165)(419.7811132,79.86431649)(419.87111206,79.86431936)
\lineto(420.59111206,79.86431936)
\curveto(421.07111191,79.86431649)(421.4811115,79.80431655)(421.82111206,79.68431936)
\curveto(422.16111082,79.56431679)(422.43611055,79.36931699)(422.64611206,79.09931936)
\curveto(422.69611029,79.02931733)(422.74111024,78.9593174)(422.78111206,78.88931936)
\curveto(422.83111015,78.82931753)(422.87611011,78.7543176)(422.91611206,78.66431936)
\curveto(422.92611006,78.64431771)(422.93611005,78.61431774)(422.94611206,78.57431936)
\curveto(422.96611002,78.53431782)(422.97111001,78.48931787)(422.96111206,78.43931936)
\curveto(422.93111005,78.34931801)(422.85611013,78.29431806)(422.73611206,78.27431936)
\curveto(422.62611036,78.2543181)(422.53111045,78.26931809)(422.45111206,78.31931936)
\curveto(422.3811106,78.34931801)(422.31611067,78.39431796)(422.25611206,78.45431936)
\curveto(422.20611078,78.52431783)(422.15611083,78.58931777)(422.10611206,78.64931936)
\curveto(422.05611093,78.71931764)(421.981111,78.77931758)(421.88111206,78.82931936)
\curveto(421.79111119,78.88931747)(421.70111128,78.93931742)(421.61111206,78.97931936)
\curveto(421.5811114,78.99931736)(421.52111146,79.02431733)(421.43111206,79.05431936)
\curveto(421.35111163,79.08431727)(421.2811117,79.08931727)(421.22111206,79.06931936)
\curveto(421.0811119,79.03931732)(420.99111199,78.97931738)(420.95111206,78.88931936)
\curveto(420.92111206,78.80931755)(420.90611208,78.71931764)(420.90611206,78.61931936)
\curveto(420.90611208,78.51931784)(420.8811121,78.43431792)(420.83111206,78.36431936)
\curveto(420.76111222,78.27431808)(420.62111236,78.22931813)(420.41111206,78.22931936)
\lineto(419.85611206,78.22931936)
\lineto(419.63111206,78.22931936)
\curveto(419.55111343,78.23931812)(419.4861135,78.2593181)(419.43611206,78.28931936)
\curveto(419.35611363,78.34931801)(419.31111367,78.41931794)(419.30111206,78.49931936)
\curveto(419.29111369,78.51931784)(419.2861137,78.53931782)(419.28611206,78.55931936)
\curveto(419.2861137,78.58931777)(419.2811137,78.61431774)(419.27111206,78.63431936)
}
}
{
\newrgbcolor{curcolor}{0 0 0}
\pscustom[linestyle=none,fillstyle=solid,fillcolor=curcolor]
{
}
}
{
\newrgbcolor{curcolor}{0 0 0}
\pscustom[linestyle=none,fillstyle=solid,fillcolor=curcolor]
{
\newpath
\moveto(410.30111206,89.26463186)
\curveto(410.29112269,89.95462722)(410.41112257,90.55462662)(410.66111206,91.06463186)
\curveto(410.91112207,91.58462559)(411.24612174,91.9796252)(411.66611206,92.24963186)
\curveto(411.74612124,92.29962488)(411.83612115,92.34462483)(411.93611206,92.38463186)
\curveto(412.02612096,92.42462475)(412.12112086,92.46962471)(412.22111206,92.51963186)
\curveto(412.32112066,92.55962462)(412.42112056,92.58962459)(412.52111206,92.60963186)
\curveto(412.62112036,92.62962455)(412.72612026,92.64962453)(412.83611206,92.66963186)
\curveto(412.8861201,92.68962449)(412.93112005,92.69462448)(412.97111206,92.68463186)
\curveto(413.01111997,92.6746245)(413.05611993,92.6796245)(413.10611206,92.69963186)
\curveto(413.15611983,92.70962447)(413.24111974,92.71462446)(413.36111206,92.71463186)
\curveto(413.47111951,92.71462446)(413.55611943,92.70962447)(413.61611206,92.69963186)
\curveto(413.67611931,92.6796245)(413.73611925,92.66962451)(413.79611206,92.66963186)
\curveto(413.85611913,92.6796245)(413.91611907,92.6746245)(413.97611206,92.65463186)
\curveto(414.11611887,92.61462456)(414.25111873,92.5796246)(414.38111206,92.54963186)
\curveto(414.51111847,92.51962466)(414.63611835,92.4796247)(414.75611206,92.42963186)
\curveto(414.89611809,92.36962481)(415.02111796,92.29962488)(415.13111206,92.21963186)
\curveto(415.24111774,92.14962503)(415.35111763,92.0746251)(415.46111206,91.99463186)
\lineto(415.52111206,91.93463186)
\curveto(415.54111744,91.92462525)(415.56111742,91.90962527)(415.58111206,91.88963186)
\curveto(415.74111724,91.76962541)(415.8861171,91.63462554)(416.01611206,91.48463186)
\curveto(416.14611684,91.33462584)(416.27111671,91.174626)(416.39111206,91.00463186)
\curveto(416.61111637,90.69462648)(416.81611617,90.39962678)(417.00611206,90.11963186)
\curveto(417.14611584,89.88962729)(417.2811157,89.65962752)(417.41111206,89.42963186)
\curveto(417.54111544,89.20962797)(417.67611531,88.98962819)(417.81611206,88.76963186)
\curveto(417.986115,88.51962866)(418.16611482,88.2796289)(418.35611206,88.04963186)
\curveto(418.54611444,87.82962935)(418.77111421,87.63962954)(419.03111206,87.47963186)
\curveto(419.09111389,87.43962974)(419.15111383,87.40462977)(419.21111206,87.37463186)
\curveto(419.26111372,87.34462983)(419.32611366,87.31462986)(419.40611206,87.28463186)
\curveto(419.47611351,87.26462991)(419.53611345,87.25962992)(419.58611206,87.26963186)
\curveto(419.65611333,87.28962989)(419.71111327,87.32462985)(419.75111206,87.37463186)
\curveto(419.7811132,87.42462975)(419.80111318,87.48462969)(419.81111206,87.55463186)
\lineto(419.81111206,87.79463186)
\lineto(419.81111206,88.54463186)
\lineto(419.81111206,91.34963186)
\lineto(419.81111206,92.00963186)
\curveto(419.81111317,92.09962508)(419.81611317,92.18462499)(419.82611206,92.26463186)
\curveto(419.82611316,92.34462483)(419.84611314,92.40962477)(419.88611206,92.45963186)
\curveto(419.92611306,92.50962467)(420.00111298,92.54962463)(420.11111206,92.57963186)
\curveto(420.21111277,92.61962456)(420.31111267,92.62962455)(420.41111206,92.60963186)
\lineto(420.54611206,92.60963186)
\curveto(420.61611237,92.58962459)(420.67611231,92.56962461)(420.72611206,92.54963186)
\curveto(420.77611221,92.52962465)(420.81611217,92.49462468)(420.84611206,92.44463186)
\curveto(420.8861121,92.39462478)(420.90611208,92.32462485)(420.90611206,92.23463186)
\lineto(420.90611206,91.96463186)
\lineto(420.90611206,91.06463186)
\lineto(420.90611206,87.55463186)
\lineto(420.90611206,86.48963186)
\curveto(420.90611208,86.40963077)(420.91111207,86.31963086)(420.92111206,86.21963186)
\curveto(420.92111206,86.11963106)(420.91111207,86.03463114)(420.89111206,85.96463186)
\curveto(420.82111216,85.75463142)(420.64111234,85.68963149)(420.35111206,85.76963186)
\curveto(420.31111267,85.7796314)(420.27611271,85.7796314)(420.24611206,85.76963186)
\curveto(420.20611278,85.76963141)(420.16111282,85.7796314)(420.11111206,85.79963186)
\curveto(420.03111295,85.81963136)(419.94611304,85.83963134)(419.85611206,85.85963186)
\curveto(419.76611322,85.8796313)(419.6811133,85.90463127)(419.60111206,85.93463186)
\curveto(419.11111387,86.09463108)(418.69611429,86.29463088)(418.35611206,86.53463186)
\curveto(418.10611488,86.71463046)(417.8811151,86.91963026)(417.68111206,87.14963186)
\curveto(417.47111551,87.3796298)(417.27611571,87.61962956)(417.09611206,87.86963186)
\curveto(416.91611607,88.12962905)(416.74611624,88.39462878)(416.58611206,88.66463186)
\curveto(416.41611657,88.94462823)(416.24111674,89.21462796)(416.06111206,89.47463186)
\curveto(415.981117,89.58462759)(415.90611708,89.68962749)(415.83611206,89.78963186)
\curveto(415.76611722,89.89962728)(415.69111729,90.00962717)(415.61111206,90.11963186)
\curveto(415.5811174,90.15962702)(415.55111743,90.19462698)(415.52111206,90.22463186)
\curveto(415.4811175,90.26462691)(415.45111753,90.30462687)(415.43111206,90.34463186)
\curveto(415.32111766,90.48462669)(415.19611779,90.60962657)(415.05611206,90.71963186)
\curveto(415.02611796,90.73962644)(415.00111798,90.76462641)(414.98111206,90.79463186)
\curveto(414.95111803,90.82462635)(414.92111806,90.84962633)(414.89111206,90.86963186)
\curveto(414.79111819,90.94962623)(414.69111829,91.01462616)(414.59111206,91.06463186)
\curveto(414.49111849,91.12462605)(414.3811186,91.179626)(414.26111206,91.22963186)
\curveto(414.19111879,91.25962592)(414.11611887,91.2796259)(414.03611206,91.28963186)
\lineto(413.79611206,91.34963186)
\lineto(413.70611206,91.34963186)
\curveto(413.67611931,91.35962582)(413.64611934,91.36462581)(413.61611206,91.36463186)
\curveto(413.54611944,91.38462579)(413.45111953,91.38962579)(413.33111206,91.37963186)
\curveto(413.20111978,91.3796258)(413.10111988,91.36962581)(413.03111206,91.34963186)
\curveto(412.95112003,91.32962585)(412.87612011,91.30962587)(412.80611206,91.28963186)
\curveto(412.72612026,91.2796259)(412.64612034,91.25962592)(412.56611206,91.22963186)
\curveto(412.32612066,91.11962606)(412.12612086,90.96962621)(411.96611206,90.77963186)
\curveto(411.79612119,90.59962658)(411.65612133,90.3796268)(411.54611206,90.11963186)
\curveto(411.52612146,90.04962713)(411.51112147,89.9796272)(411.50111206,89.90963186)
\curveto(411.4811215,89.83962734)(411.46112152,89.76462741)(411.44111206,89.68463186)
\curveto(411.42112156,89.60462757)(411.41112157,89.49462768)(411.41111206,89.35463186)
\curveto(411.41112157,89.22462795)(411.42112156,89.11962806)(411.44111206,89.03963186)
\curveto(411.45112153,88.9796282)(411.45612153,88.92462825)(411.45611206,88.87463186)
\curveto(411.45612153,88.82462835)(411.46612152,88.7746284)(411.48611206,88.72463186)
\curveto(411.52612146,88.62462855)(411.56612142,88.52962865)(411.60611206,88.43963186)
\curveto(411.64612134,88.35962882)(411.69112129,88.2796289)(411.74111206,88.19963186)
\curveto(411.76112122,88.16962901)(411.7861212,88.13962904)(411.81611206,88.10963186)
\curveto(411.84612114,88.08962909)(411.87112111,88.06462911)(411.89111206,88.03463186)
\lineto(411.96611206,87.95963186)
\curveto(411.986121,87.92962925)(412.00612098,87.90462927)(412.02611206,87.88463186)
\lineto(412.23611206,87.73463186)
\curveto(412.29612069,87.69462948)(412.36112062,87.64962953)(412.43111206,87.59963186)
\curveto(412.52112046,87.53962964)(412.62612036,87.48962969)(412.74611206,87.44963186)
\curveto(412.85612013,87.41962976)(412.96612002,87.38462979)(413.07611206,87.34463186)
\curveto(413.1861198,87.30462987)(413.33111965,87.2796299)(413.51111206,87.26963186)
\curveto(413.6811193,87.25962992)(413.80611918,87.22962995)(413.88611206,87.17963186)
\curveto(413.96611902,87.12963005)(414.01111897,87.05463012)(414.02111206,86.95463186)
\curveto(414.03111895,86.85463032)(414.03611895,86.74463043)(414.03611206,86.62463186)
\curveto(414.03611895,86.58463059)(414.04111894,86.54463063)(414.05111206,86.50463186)
\curveto(414.05111893,86.46463071)(414.04611894,86.42963075)(414.03611206,86.39963186)
\curveto(414.01611897,86.34963083)(414.00611898,86.29963088)(414.00611206,86.24963186)
\curveto(414.00611898,86.20963097)(413.99611899,86.16963101)(413.97611206,86.12963186)
\curveto(413.91611907,86.03963114)(413.7811192,85.99463118)(413.57111206,85.99463186)
\lineto(413.45111206,85.99463186)
\curveto(413.39111959,86.00463117)(413.33111965,86.00963117)(413.27111206,86.00963186)
\curveto(413.20111978,86.01963116)(413.13611985,86.02963115)(413.07611206,86.03963186)
\curveto(412.96612002,86.05963112)(412.86612012,86.0796311)(412.77611206,86.09963186)
\curveto(412.67612031,86.11963106)(412.5811204,86.14963103)(412.49111206,86.18963186)
\curveto(412.42112056,86.20963097)(412.36112062,86.22963095)(412.31111206,86.24963186)
\lineto(412.13111206,86.30963186)
\curveto(411.87112111,86.42963075)(411.62612136,86.58463059)(411.39611206,86.77463186)
\curveto(411.16612182,86.9746302)(410.981122,87.18962999)(410.84111206,87.41963186)
\curveto(410.76112222,87.52962965)(410.69612229,87.64462953)(410.64611206,87.76463186)
\lineto(410.49611206,88.15463186)
\curveto(410.44612254,88.26462891)(410.41612257,88.3796288)(410.40611206,88.49963186)
\curveto(410.3861226,88.61962856)(410.36112262,88.74462843)(410.33111206,88.87463186)
\curveto(410.33112265,88.94462823)(410.33112265,89.00962817)(410.33111206,89.06963186)
\curveto(410.32112266,89.12962805)(410.31112267,89.19462798)(410.30111206,89.26463186)
}
}
{
\newrgbcolor{curcolor}{0 0 0}
\pscustom[linestyle=none,fillstyle=solid,fillcolor=curcolor]
{
\newpath
\moveto(415.82111206,101.36424123)
\lineto(416.07611206,101.36424123)
\curveto(416.15611683,101.37423353)(416.23111675,101.36923353)(416.30111206,101.34924123)
\lineto(416.54111206,101.34924123)
\lineto(416.70611206,101.34924123)
\curveto(416.80611618,101.32923357)(416.91111607,101.31923358)(417.02111206,101.31924123)
\curveto(417.12111586,101.31923358)(417.22111576,101.30923359)(417.32111206,101.28924123)
\lineto(417.47111206,101.28924123)
\curveto(417.61111537,101.25923364)(417.75111523,101.23923366)(417.89111206,101.22924123)
\curveto(418.02111496,101.21923368)(418.15111483,101.19423371)(418.28111206,101.15424123)
\curveto(418.36111462,101.13423377)(418.44611454,101.11423379)(418.53611206,101.09424123)
\lineto(418.77611206,101.03424123)
\lineto(419.07611206,100.91424123)
\curveto(419.16611382,100.88423402)(419.25611373,100.84923405)(419.34611206,100.80924123)
\curveto(419.56611342,100.70923419)(419.7811132,100.57423433)(419.99111206,100.40424123)
\curveto(420.20111278,100.24423466)(420.37111261,100.06923483)(420.50111206,99.87924123)
\curveto(420.54111244,99.82923507)(420.5811124,99.76923513)(420.62111206,99.69924123)
\curveto(420.65111233,99.63923526)(420.6861123,99.57923532)(420.72611206,99.51924123)
\curveto(420.77611221,99.43923546)(420.81611217,99.34423556)(420.84611206,99.23424123)
\curveto(420.87611211,99.12423578)(420.90611208,99.01923588)(420.93611206,98.91924123)
\curveto(420.97611201,98.80923609)(421.00111198,98.6992362)(421.01111206,98.58924123)
\curveto(421.02111196,98.47923642)(421.03611195,98.36423654)(421.05611206,98.24424123)
\curveto(421.06611192,98.2042367)(421.06611192,98.15923674)(421.05611206,98.10924123)
\curveto(421.05611193,98.06923683)(421.06111192,98.02923687)(421.07111206,97.98924123)
\curveto(421.0811119,97.94923695)(421.0861119,97.89423701)(421.08611206,97.82424123)
\curveto(421.0861119,97.75423715)(421.0811119,97.7042372)(421.07111206,97.67424123)
\curveto(421.05111193,97.62423728)(421.04611194,97.57923732)(421.05611206,97.53924123)
\curveto(421.06611192,97.4992374)(421.06611192,97.46423744)(421.05611206,97.43424123)
\lineto(421.05611206,97.34424123)
\curveto(421.03611195,97.28423762)(421.02111196,97.21923768)(421.01111206,97.14924123)
\curveto(421.01111197,97.08923781)(421.00611198,97.02423788)(420.99611206,96.95424123)
\curveto(420.94611204,96.78423812)(420.89611209,96.62423828)(420.84611206,96.47424123)
\curveto(420.79611219,96.32423858)(420.73111225,96.17923872)(420.65111206,96.03924123)
\curveto(420.61111237,95.98923891)(420.5811124,95.93423897)(420.56111206,95.87424123)
\curveto(420.53111245,95.82423908)(420.49611249,95.77423913)(420.45611206,95.72424123)
\curveto(420.27611271,95.48423942)(420.05611293,95.28423962)(419.79611206,95.12424123)
\curveto(419.53611345,94.96423994)(419.25111373,94.82424008)(418.94111206,94.70424123)
\curveto(418.80111418,94.64424026)(418.66111432,94.5992403)(418.52111206,94.56924123)
\curveto(418.37111461,94.53924036)(418.21611477,94.5042404)(418.05611206,94.46424123)
\curveto(417.94611504,94.44424046)(417.83611515,94.42924047)(417.72611206,94.41924123)
\curveto(417.61611537,94.40924049)(417.50611548,94.39424051)(417.39611206,94.37424123)
\curveto(417.35611563,94.36424054)(417.31611567,94.35924054)(417.27611206,94.35924123)
\curveto(417.23611575,94.36924053)(417.19611579,94.36924053)(417.15611206,94.35924123)
\curveto(417.10611588,94.34924055)(417.05611593,94.34424056)(417.00611206,94.34424123)
\lineto(416.84111206,94.34424123)
\curveto(416.79111619,94.32424058)(416.74111624,94.31924058)(416.69111206,94.32924123)
\curveto(416.63111635,94.33924056)(416.57611641,94.33924056)(416.52611206,94.32924123)
\curveto(416.4861165,94.31924058)(416.44111654,94.31924058)(416.39111206,94.32924123)
\curveto(416.34111664,94.33924056)(416.29111669,94.33424057)(416.24111206,94.31424123)
\curveto(416.17111681,94.29424061)(416.09611689,94.28924061)(416.01611206,94.29924123)
\curveto(415.92611706,94.30924059)(415.84111714,94.31424059)(415.76111206,94.31424123)
\curveto(415.67111731,94.31424059)(415.57111741,94.30924059)(415.46111206,94.29924123)
\curveto(415.34111764,94.28924061)(415.24111774,94.29424061)(415.16111206,94.31424123)
\lineto(414.87611206,94.31424123)
\lineto(414.24611206,94.35924123)
\curveto(414.14611884,94.36924053)(414.05111893,94.37924052)(413.96111206,94.38924123)
\lineto(413.66111206,94.41924123)
\curveto(413.61111937,94.43924046)(413.56111942,94.44424046)(413.51111206,94.43424123)
\curveto(413.45111953,94.43424047)(413.39611959,94.44424046)(413.34611206,94.46424123)
\curveto(413.17611981,94.51424039)(413.01111997,94.55424035)(412.85111206,94.58424123)
\curveto(412.6811203,94.61424029)(412.52112046,94.66424024)(412.37111206,94.73424123)
\curveto(411.91112107,94.92423998)(411.53612145,95.14423976)(411.24611206,95.39424123)
\curveto(410.95612203,95.65423925)(410.71112227,96.01423889)(410.51111206,96.47424123)
\curveto(410.46112252,96.6042383)(410.42612256,96.73423817)(410.40611206,96.86424123)
\curveto(410.3861226,97.0042379)(410.36112262,97.14423776)(410.33111206,97.28424123)
\curveto(410.32112266,97.35423755)(410.31612267,97.41923748)(410.31611206,97.47924123)
\curveto(410.31612267,97.53923736)(410.31112267,97.6042373)(410.30111206,97.67424123)
\curveto(410.2811227,98.5042364)(410.43112255,99.17423573)(410.75111206,99.68424123)
\curveto(411.06112192,100.19423471)(411.50112148,100.57423433)(412.07111206,100.82424123)
\curveto(412.19112079,100.87423403)(412.31612067,100.91923398)(412.44611206,100.95924123)
\curveto(412.57612041,100.9992339)(412.71112027,101.04423386)(412.85111206,101.09424123)
\curveto(412.93112005,101.11423379)(413.01611997,101.12923377)(413.10611206,101.13924123)
\lineto(413.34611206,101.19924123)
\curveto(413.45611953,101.22923367)(413.56611942,101.24423366)(413.67611206,101.24424123)
\curveto(413.7861192,101.25423365)(413.89611909,101.26923363)(414.00611206,101.28924123)
\curveto(414.05611893,101.30923359)(414.10111888,101.31423359)(414.14111206,101.30424123)
\curveto(414.1811188,101.3042336)(414.22111876,101.30923359)(414.26111206,101.31924123)
\curveto(414.31111867,101.32923357)(414.36611862,101.32923357)(414.42611206,101.31924123)
\curveto(414.47611851,101.31923358)(414.52611846,101.32423358)(414.57611206,101.33424123)
\lineto(414.71111206,101.33424123)
\curveto(414.77111821,101.35423355)(414.84111814,101.35423355)(414.92111206,101.33424123)
\curveto(414.99111799,101.32423358)(415.05611793,101.32923357)(415.11611206,101.34924123)
\curveto(415.14611784,101.35923354)(415.1861178,101.36423354)(415.23611206,101.36424123)
\lineto(415.35611206,101.36424123)
\lineto(415.82111206,101.36424123)
\moveto(418.14611206,99.81924123)
\curveto(417.82611516,99.91923498)(417.46111552,99.97923492)(417.05111206,99.99924123)
\curveto(416.64111634,100.01923488)(416.23111675,100.02923487)(415.82111206,100.02924123)
\curveto(415.39111759,100.02923487)(414.97111801,100.01923488)(414.56111206,99.99924123)
\curveto(414.15111883,99.97923492)(413.76611922,99.93423497)(413.40611206,99.86424123)
\curveto(413.04611994,99.79423511)(412.72612026,99.68423522)(412.44611206,99.53424123)
\curveto(412.15612083,99.39423551)(411.92112106,99.1992357)(411.74111206,98.94924123)
\curveto(411.63112135,98.78923611)(411.55112143,98.60923629)(411.50111206,98.40924123)
\curveto(411.44112154,98.20923669)(411.41112157,97.96423694)(411.41111206,97.67424123)
\curveto(411.43112155,97.65423725)(411.44112154,97.61923728)(411.44111206,97.56924123)
\curveto(411.43112155,97.51923738)(411.43112155,97.47923742)(411.44111206,97.44924123)
\curveto(411.46112152,97.36923753)(411.4811215,97.29423761)(411.50111206,97.22424123)
\curveto(411.51112147,97.16423774)(411.53112145,97.0992378)(411.56111206,97.02924123)
\curveto(411.6811213,96.75923814)(411.85112113,96.53923836)(412.07111206,96.36924123)
\curveto(412.2811207,96.20923869)(412.52612046,96.07423883)(412.80611206,95.96424123)
\curveto(412.91612007,95.91423899)(413.03611995,95.87423903)(413.16611206,95.84424123)
\curveto(413.2861197,95.82423908)(413.41111957,95.7992391)(413.54111206,95.76924123)
\curveto(413.59111939,95.74923915)(413.64611934,95.73923916)(413.70611206,95.73924123)
\curveto(413.75611923,95.73923916)(413.80611918,95.73423917)(413.85611206,95.72424123)
\curveto(413.94611904,95.71423919)(414.04111894,95.7042392)(414.14111206,95.69424123)
\curveto(414.23111875,95.68423922)(414.32611866,95.67423923)(414.42611206,95.66424123)
\curveto(414.50611848,95.66423924)(414.59111839,95.65923924)(414.68111206,95.64924123)
\lineto(414.92111206,95.64924123)
\lineto(415.10111206,95.64924123)
\curveto(415.13111785,95.63923926)(415.16611782,95.63423927)(415.20611206,95.63424123)
\lineto(415.34111206,95.63424123)
\lineto(415.79111206,95.63424123)
\curveto(415.87111711,95.63423927)(415.95611703,95.62923927)(416.04611206,95.61924123)
\curveto(416.12611686,95.61923928)(416.20111678,95.62923927)(416.27111206,95.64924123)
\lineto(416.54111206,95.64924123)
\curveto(416.56111642,95.64923925)(416.59111639,95.64423926)(416.63111206,95.63424123)
\curveto(416.66111632,95.63423927)(416.6861163,95.63923926)(416.70611206,95.64924123)
\curveto(416.80611618,95.65923924)(416.90611608,95.66423924)(417.00611206,95.66424123)
\curveto(417.09611589,95.67423923)(417.19611579,95.68423922)(417.30611206,95.69424123)
\curveto(417.42611556,95.72423918)(417.55111543,95.73923916)(417.68111206,95.73924123)
\curveto(417.80111518,95.74923915)(417.91611507,95.77423913)(418.02611206,95.81424123)
\curveto(418.32611466,95.89423901)(418.59111439,95.97923892)(418.82111206,96.06924123)
\curveto(419.05111393,96.16923873)(419.26611372,96.31423859)(419.46611206,96.50424123)
\curveto(419.66611332,96.71423819)(419.81611317,96.97923792)(419.91611206,97.29924123)
\curveto(419.93611305,97.33923756)(419.94611304,97.37423753)(419.94611206,97.40424123)
\curveto(419.93611305,97.44423746)(419.94111304,97.48923741)(419.96111206,97.53924123)
\curveto(419.97111301,97.57923732)(419.981113,97.64923725)(419.99111206,97.74924123)
\curveto(420.00111298,97.85923704)(419.99611299,97.94423696)(419.97611206,98.00424123)
\curveto(419.95611303,98.07423683)(419.94611304,98.14423676)(419.94611206,98.21424123)
\curveto(419.93611305,98.28423662)(419.92111306,98.34923655)(419.90111206,98.40924123)
\curveto(419.84111314,98.60923629)(419.75611323,98.78923611)(419.64611206,98.94924123)
\curveto(419.62611336,98.97923592)(419.60611338,99.0042359)(419.58611206,99.02424123)
\lineto(419.52611206,99.08424123)
\curveto(419.50611348,99.12423578)(419.46611352,99.17423573)(419.40611206,99.23424123)
\curveto(419.26611372,99.33423557)(419.13611385,99.41923548)(419.01611206,99.48924123)
\curveto(418.89611409,99.55923534)(418.75111423,99.62923527)(418.58111206,99.69924123)
\curveto(418.51111447,99.72923517)(418.44111454,99.74923515)(418.37111206,99.75924123)
\curveto(418.30111468,99.77923512)(418.22611476,99.7992351)(418.14611206,99.81924123)
}
}
{
\newrgbcolor{curcolor}{0 0 0}
\pscustom[linestyle=none,fillstyle=solid,fillcolor=curcolor]
{
\newpath
\moveto(410.30111206,106.77385061)
\curveto(410.30112268,106.87384575)(410.31112267,106.96884566)(410.33111206,107.05885061)
\curveto(410.34112264,107.14884548)(410.37112261,107.21384541)(410.42111206,107.25385061)
\curveto(410.50112248,107.31384531)(410.60612238,107.34384528)(410.73611206,107.34385061)
\lineto(411.12611206,107.34385061)
\lineto(412.62611206,107.34385061)
\lineto(419.01611206,107.34385061)
\lineto(420.18611206,107.34385061)
\lineto(420.50111206,107.34385061)
\curveto(420.60111238,107.35384527)(420.6811123,107.33884529)(420.74111206,107.29885061)
\curveto(420.82111216,107.24884538)(420.87111211,107.17384545)(420.89111206,107.07385061)
\curveto(420.90111208,106.98384564)(420.90611208,106.87384575)(420.90611206,106.74385061)
\lineto(420.90611206,106.51885061)
\curveto(420.8861121,106.43884619)(420.87111211,106.36884626)(420.86111206,106.30885061)
\curveto(420.84111214,106.24884638)(420.80111218,106.19884643)(420.74111206,106.15885061)
\curveto(420.6811123,106.11884651)(420.60611238,106.09884653)(420.51611206,106.09885061)
\lineto(420.21611206,106.09885061)
\lineto(419.12111206,106.09885061)
\lineto(413.78111206,106.09885061)
\curveto(413.69111929,106.07884655)(413.61611937,106.06384656)(413.55611206,106.05385061)
\curveto(413.4861195,106.05384657)(413.42611956,106.0238466)(413.37611206,105.96385061)
\curveto(413.32611966,105.89384673)(413.30111968,105.80384682)(413.30111206,105.69385061)
\curveto(413.29111969,105.59384703)(413.2861197,105.48384714)(413.28611206,105.36385061)
\lineto(413.28611206,104.22385061)
\lineto(413.28611206,103.72885061)
\curveto(413.27611971,103.56884906)(413.21611977,103.45884917)(413.10611206,103.39885061)
\curveto(413.07611991,103.37884925)(413.04611994,103.36884926)(413.01611206,103.36885061)
\curveto(412.97612001,103.36884926)(412.93112005,103.36384926)(412.88111206,103.35385061)
\curveto(412.76112022,103.33384929)(412.65112033,103.33884929)(412.55111206,103.36885061)
\curveto(412.45112053,103.40884922)(412.3811206,103.46384916)(412.34111206,103.53385061)
\curveto(412.29112069,103.61384901)(412.26612072,103.73384889)(412.26611206,103.89385061)
\curveto(412.26612072,104.05384857)(412.25112073,104.18884844)(412.22111206,104.29885061)
\curveto(412.21112077,104.34884828)(412.20612078,104.40384822)(412.20611206,104.46385061)
\curveto(412.19612079,104.5238481)(412.1811208,104.58384804)(412.16111206,104.64385061)
\curveto(412.11112087,104.79384783)(412.06112092,104.93884769)(412.01111206,105.07885061)
\curveto(411.95112103,105.21884741)(411.8811211,105.35384727)(411.80111206,105.48385061)
\curveto(411.71112127,105.623847)(411.60612138,105.74384688)(411.48611206,105.84385061)
\curveto(411.36612162,105.94384668)(411.23612175,106.03884659)(411.09611206,106.12885061)
\curveto(410.99612199,106.18884644)(410.8861221,106.23384639)(410.76611206,106.26385061)
\curveto(410.64612234,106.30384632)(410.54112244,106.35384627)(410.45111206,106.41385061)
\curveto(410.39112259,106.46384616)(410.35112263,106.53384609)(410.33111206,106.62385061)
\curveto(410.32112266,106.64384598)(410.31612267,106.66884596)(410.31611206,106.69885061)
\curveto(410.31612267,106.7288459)(410.31112267,106.75384587)(410.30111206,106.77385061)
}
}
{
\newrgbcolor{curcolor}{0 0 0}
\pscustom[linestyle=none,fillstyle=solid,fillcolor=curcolor]
{
\newpath
\moveto(410.30111206,115.12345998)
\curveto(410.30112268,115.22345513)(410.31112267,115.31845503)(410.33111206,115.40845998)
\curveto(410.34112264,115.49845485)(410.37112261,115.56345479)(410.42111206,115.60345998)
\curveto(410.50112248,115.66345469)(410.60612238,115.69345466)(410.73611206,115.69345998)
\lineto(411.12611206,115.69345998)
\lineto(412.62611206,115.69345998)
\lineto(419.01611206,115.69345998)
\lineto(420.18611206,115.69345998)
\lineto(420.50111206,115.69345998)
\curveto(420.60111238,115.70345465)(420.6811123,115.68845466)(420.74111206,115.64845998)
\curveto(420.82111216,115.59845475)(420.87111211,115.52345483)(420.89111206,115.42345998)
\curveto(420.90111208,115.33345502)(420.90611208,115.22345513)(420.90611206,115.09345998)
\lineto(420.90611206,114.86845998)
\curveto(420.8861121,114.78845556)(420.87111211,114.71845563)(420.86111206,114.65845998)
\curveto(420.84111214,114.59845575)(420.80111218,114.5484558)(420.74111206,114.50845998)
\curveto(420.6811123,114.46845588)(420.60611238,114.4484559)(420.51611206,114.44845998)
\lineto(420.21611206,114.44845998)
\lineto(419.12111206,114.44845998)
\lineto(413.78111206,114.44845998)
\curveto(413.69111929,114.42845592)(413.61611937,114.41345594)(413.55611206,114.40345998)
\curveto(413.4861195,114.40345595)(413.42611956,114.37345598)(413.37611206,114.31345998)
\curveto(413.32611966,114.24345611)(413.30111968,114.1534562)(413.30111206,114.04345998)
\curveto(413.29111969,113.94345641)(413.2861197,113.83345652)(413.28611206,113.71345998)
\lineto(413.28611206,112.57345998)
\lineto(413.28611206,112.07845998)
\curveto(413.27611971,111.91845843)(413.21611977,111.80845854)(413.10611206,111.74845998)
\curveto(413.07611991,111.72845862)(413.04611994,111.71845863)(413.01611206,111.71845998)
\curveto(412.97612001,111.71845863)(412.93112005,111.71345864)(412.88111206,111.70345998)
\curveto(412.76112022,111.68345867)(412.65112033,111.68845866)(412.55111206,111.71845998)
\curveto(412.45112053,111.75845859)(412.3811206,111.81345854)(412.34111206,111.88345998)
\curveto(412.29112069,111.96345839)(412.26612072,112.08345827)(412.26611206,112.24345998)
\curveto(412.26612072,112.40345795)(412.25112073,112.53845781)(412.22111206,112.64845998)
\curveto(412.21112077,112.69845765)(412.20612078,112.7534576)(412.20611206,112.81345998)
\curveto(412.19612079,112.87345748)(412.1811208,112.93345742)(412.16111206,112.99345998)
\curveto(412.11112087,113.14345721)(412.06112092,113.28845706)(412.01111206,113.42845998)
\curveto(411.95112103,113.56845678)(411.8811211,113.70345665)(411.80111206,113.83345998)
\curveto(411.71112127,113.97345638)(411.60612138,114.09345626)(411.48611206,114.19345998)
\curveto(411.36612162,114.29345606)(411.23612175,114.38845596)(411.09611206,114.47845998)
\curveto(410.99612199,114.53845581)(410.8861221,114.58345577)(410.76611206,114.61345998)
\curveto(410.64612234,114.6534557)(410.54112244,114.70345565)(410.45111206,114.76345998)
\curveto(410.39112259,114.81345554)(410.35112263,114.88345547)(410.33111206,114.97345998)
\curveto(410.32112266,114.99345536)(410.31612267,115.01845533)(410.31611206,115.04845998)
\curveto(410.31612267,115.07845527)(410.31112267,115.10345525)(410.30111206,115.12345998)
}
}
{
\newrgbcolor{curcolor}{0 0 0}
\pscustom[linestyle=none,fillstyle=solid,fillcolor=curcolor]
{
\newpath
\moveto(441.0374585,42.02236623)
\curveto(441.08745924,42.04235669)(441.14745918,42.06735666)(441.2174585,42.09736623)
\curveto(441.28745904,42.1273566)(441.36245897,42.14735658)(441.4424585,42.15736623)
\curveto(441.51245882,42.17735655)(441.58245875,42.17735655)(441.6524585,42.15736623)
\curveto(441.71245862,42.14735658)(441.75745857,42.10735662)(441.7874585,42.03736623)
\curveto(441.80745852,41.98735674)(441.81745851,41.9273568)(441.8174585,41.85736623)
\lineto(441.8174585,41.64736623)
\lineto(441.8174585,41.19736623)
\curveto(441.81745851,41.04735768)(441.79245854,40.9273578)(441.7424585,40.83736623)
\curveto(441.68245865,40.73735799)(441.57745875,40.66235807)(441.4274585,40.61236623)
\curveto(441.27745905,40.57235816)(441.14245919,40.5273582)(441.0224585,40.47736623)
\curveto(440.76245957,40.36735836)(440.49245984,40.26735846)(440.2124585,40.17736623)
\curveto(439.9324604,40.08735864)(439.65746067,39.98735874)(439.3874585,39.87736623)
\curveto(439.29746103,39.84735888)(439.21246112,39.81735891)(439.1324585,39.78736623)
\curveto(439.05246128,39.76735896)(438.97746135,39.73735899)(438.9074585,39.69736623)
\curveto(438.83746149,39.66735906)(438.77746155,39.62235911)(438.7274585,39.56236623)
\curveto(438.67746165,39.50235923)(438.63746169,39.42235931)(438.6074585,39.32236623)
\curveto(438.58746174,39.27235946)(438.58246175,39.21235952)(438.5924585,39.14236623)
\lineto(438.5924585,38.94736623)
\lineto(438.5924585,36.11236623)
\lineto(438.5924585,35.81236623)
\curveto(438.58246175,35.70236303)(438.58246175,35.59736313)(438.5924585,35.49736623)
\curveto(438.60246173,35.39736333)(438.61746171,35.30236343)(438.6374585,35.21236623)
\curveto(438.65746167,35.1323636)(438.69746163,35.07236366)(438.7574585,35.03236623)
\curveto(438.85746147,34.95236378)(438.97246136,34.89236384)(439.1024585,34.85236623)
\curveto(439.22246111,34.82236391)(439.34746098,34.78236395)(439.4774585,34.73236623)
\curveto(439.70746062,34.6323641)(439.94746038,34.53736419)(440.1974585,34.44736623)
\curveto(440.44745988,34.36736436)(440.68745964,34.27736445)(440.9174585,34.17736623)
\curveto(440.97745935,34.15736457)(441.04745928,34.1323646)(441.1274585,34.10236623)
\curveto(441.19745913,34.08236465)(441.27245906,34.05736467)(441.3524585,34.02736623)
\curveto(441.4324589,33.99736473)(441.50745882,33.96236477)(441.5774585,33.92236623)
\curveto(441.63745869,33.89236484)(441.68245865,33.85736487)(441.7124585,33.81736623)
\curveto(441.77245856,33.73736499)(441.80745852,33.6273651)(441.8174585,33.48736623)
\lineto(441.8174585,33.06736623)
\lineto(441.8174585,32.82736623)
\curveto(441.80745852,32.75736597)(441.78245855,32.69736603)(441.7424585,32.64736623)
\curveto(441.71245862,32.59736613)(441.66745866,32.56736616)(441.6074585,32.55736623)
\curveto(441.54745878,32.55736617)(441.48745884,32.56236617)(441.4274585,32.57236623)
\curveto(441.35745897,32.59236614)(441.29245904,32.61236612)(441.2324585,32.63236623)
\curveto(441.16245917,32.66236607)(441.11245922,32.68736604)(441.0824585,32.70736623)
\curveto(440.76245957,32.84736588)(440.44745988,32.97236576)(440.1374585,33.08236623)
\curveto(439.81746051,33.19236554)(439.49746083,33.31236542)(439.1774585,33.44236623)
\curveto(438.95746137,33.5323652)(438.74246159,33.61736511)(438.5324585,33.69736623)
\curveto(438.31246202,33.77736495)(438.09246224,33.86236487)(437.8724585,33.95236623)
\curveto(437.15246318,34.25236448)(436.4274639,34.53736419)(435.6974585,34.80736623)
\curveto(434.95746537,35.07736365)(434.22246611,35.36236337)(433.4924585,35.66236623)
\curveto(433.2324671,35.77236296)(432.96746736,35.87236286)(432.6974585,35.96236623)
\curveto(432.4274679,36.06236267)(432.16246817,36.16736256)(431.9024585,36.27736623)
\curveto(431.79246854,36.3273624)(431.67246866,36.37236236)(431.5424585,36.41236623)
\curveto(431.40246893,36.46236227)(431.30246903,36.5323622)(431.2424585,36.62236623)
\curveto(431.20246913,36.66236207)(431.17246916,36.727362)(431.1524585,36.81736623)
\curveto(431.14246919,36.83736189)(431.14246919,36.85736187)(431.1524585,36.87736623)
\curveto(431.15246918,36.90736182)(431.14746918,36.9323618)(431.1374585,36.95236623)
\curveto(431.13746919,37.1323616)(431.13746919,37.34236139)(431.1374585,37.58236623)
\curveto(431.1274692,37.82236091)(431.16246917,37.99736073)(431.2424585,38.10736623)
\curveto(431.30246903,38.18736054)(431.40246893,38.24736048)(431.5424585,38.28736623)
\curveto(431.67246866,38.33736039)(431.79246854,38.38736034)(431.9024585,38.43736623)
\curveto(432.1324682,38.53736019)(432.36246797,38.6273601)(432.5924585,38.70736623)
\curveto(432.82246751,38.78735994)(433.05246728,38.87735985)(433.2824585,38.97736623)
\curveto(433.48246685,39.05735967)(433.68746664,39.1323596)(433.8974585,39.20236623)
\curveto(434.10746622,39.28235945)(434.31246602,39.36735936)(434.5124585,39.45736623)
\curveto(435.24246509,39.75735897)(435.98246435,40.04235869)(436.7324585,40.31236623)
\curveto(437.47246286,40.59235814)(438.20746212,40.88735784)(438.9374585,41.19736623)
\curveto(439.0274613,41.23735749)(439.11246122,41.26735746)(439.1924585,41.28736623)
\curveto(439.27246106,41.31735741)(439.35746097,41.34735738)(439.4474585,41.37736623)
\curveto(439.70746062,41.48735724)(439.97246036,41.59235714)(440.2424585,41.69236623)
\curveto(440.51245982,41.80235693)(440.77745955,41.91235682)(441.0374585,42.02236623)
\moveto(437.3924585,38.81236623)
\curveto(437.36246297,38.90235983)(437.31246302,38.95735977)(437.2424585,38.97736623)
\curveto(437.17246316,39.00735972)(437.09746323,39.01235972)(437.0174585,38.99236623)
\curveto(436.9274634,38.98235975)(436.84246349,38.95735977)(436.7624585,38.91736623)
\curveto(436.67246366,38.88735984)(436.59746373,38.85735987)(436.5374585,38.82736623)
\curveto(436.49746383,38.80735992)(436.46246387,38.79735993)(436.4324585,38.79736623)
\curveto(436.40246393,38.79735993)(436.36746396,38.78735994)(436.3274585,38.76736623)
\lineto(436.0874585,38.67736623)
\curveto(435.99746433,38.65736007)(435.90746442,38.6273601)(435.8174585,38.58736623)
\curveto(435.45746487,38.43736029)(435.09246524,38.30236043)(434.7224585,38.18236623)
\curveto(434.34246599,38.07236066)(433.97246636,37.94236079)(433.6124585,37.79236623)
\curveto(433.50246683,37.74236099)(433.39246694,37.69736103)(433.2824585,37.65736623)
\curveto(433.17246716,37.6273611)(433.06746726,37.58736114)(432.9674585,37.53736623)
\curveto(432.91746741,37.51736121)(432.87246746,37.49236124)(432.8324585,37.46236623)
\curveto(432.78246755,37.44236129)(432.75746757,37.39236134)(432.7574585,37.31236623)
\curveto(432.77746755,37.29236144)(432.79246754,37.27236146)(432.8024585,37.25236623)
\curveto(432.81246752,37.2323615)(432.8274675,37.21236152)(432.8474585,37.19236623)
\curveto(432.89746743,37.15236158)(432.95246738,37.12236161)(433.0124585,37.10236623)
\curveto(433.06246727,37.08236165)(433.11746721,37.06236167)(433.1774585,37.04236623)
\curveto(433.28746704,36.99236174)(433.39746693,36.95236178)(433.5074585,36.92236623)
\curveto(433.61746671,36.89236184)(433.7274666,36.85236188)(433.8374585,36.80236623)
\curveto(434.2274661,36.6323621)(434.62246571,36.48236225)(435.0224585,36.35236623)
\curveto(435.42246491,36.2323625)(435.81246452,36.09236264)(436.1924585,35.93236623)
\lineto(436.3424585,35.87236623)
\curveto(436.39246394,35.86236287)(436.44246389,35.84736288)(436.4924585,35.82736623)
\lineto(436.7324585,35.73736623)
\curveto(436.81246352,35.70736302)(436.89246344,35.68236305)(436.9724585,35.66236623)
\curveto(437.02246331,35.64236309)(437.07746325,35.6323631)(437.1374585,35.63236623)
\curveto(437.19746313,35.64236309)(437.24746308,35.65736307)(437.2874585,35.67736623)
\curveto(437.36746296,35.727363)(437.41246292,35.8323629)(437.4224585,35.99236623)
\lineto(437.4224585,36.44236623)
\lineto(437.4224585,38.04736623)
\curveto(437.42246291,38.15736057)(437.4274629,38.29236044)(437.4374585,38.45236623)
\curveto(437.43746289,38.61236012)(437.42246291,38.73236)(437.3924585,38.81236623)
}
}
{
\newrgbcolor{curcolor}{0 0 0}
\pscustom[linestyle=none,fillstyle=solid,fillcolor=curcolor]
{
\newpath
\moveto(437.7824585,50.56392873)
\curveto(437.8324625,50.57392038)(437.90246243,50.57892038)(437.9924585,50.57892873)
\curveto(438.07246226,50.57892038)(438.13746219,50.57392038)(438.1874585,50.56392873)
\curveto(438.2274621,50.56392039)(438.26746206,50.5589204)(438.3074585,50.54892873)
\lineto(438.4274585,50.54892873)
\curveto(438.50746182,50.52892043)(438.58746174,50.51892044)(438.6674585,50.51892873)
\curveto(438.74746158,50.51892044)(438.8274615,50.50892045)(438.9074585,50.48892873)
\curveto(438.94746138,50.47892048)(438.98746134,50.47392048)(439.0274585,50.47392873)
\curveto(439.05746127,50.47392048)(439.09246124,50.46892049)(439.1324585,50.45892873)
\curveto(439.24246109,50.42892053)(439.34746098,50.39892056)(439.4474585,50.36892873)
\curveto(439.54746078,50.34892061)(439.64746068,50.31892064)(439.7474585,50.27892873)
\curveto(440.09746023,50.13892082)(440.41245992,49.96892099)(440.6924585,49.76892873)
\curveto(440.97245936,49.56892139)(441.21245912,49.31892164)(441.4124585,49.01892873)
\curveto(441.51245882,48.86892209)(441.59745873,48.72392223)(441.6674585,48.58392873)
\curveto(441.71745861,48.47392248)(441.75745857,48.36392259)(441.7874585,48.25392873)
\curveto(441.81745851,48.1539228)(441.84745848,48.04892291)(441.8774585,47.93892873)
\curveto(441.89745843,47.86892309)(441.90745842,47.80392315)(441.9074585,47.74392873)
\curveto(441.91745841,47.68392327)(441.9324584,47.62392333)(441.9524585,47.56392873)
\lineto(441.9524585,47.41392873)
\curveto(441.97245836,47.36392359)(441.98245835,47.28892367)(441.9824585,47.18892873)
\curveto(441.99245834,47.08892387)(441.98745834,47.00892395)(441.9674585,46.94892873)
\lineto(441.9674585,46.79892873)
\curveto(441.95745837,46.7589242)(441.95245838,46.71392424)(441.9524585,46.66392873)
\curveto(441.95245838,46.62392433)(441.94745838,46.57892438)(441.9374585,46.52892873)
\curveto(441.89745843,46.37892458)(441.86245847,46.22892473)(441.8324585,46.07892873)
\curveto(441.80245853,45.93892502)(441.75745857,45.79892516)(441.6974585,45.65892873)
\curveto(441.61745871,45.4589255)(441.51745881,45.27892568)(441.3974585,45.11892873)
\lineto(441.2474585,44.93892873)
\curveto(441.18745914,44.87892608)(441.14745918,44.80892615)(441.1274585,44.72892873)
\curveto(441.11745921,44.66892629)(441.1324592,44.61892634)(441.1724585,44.57892873)
\curveto(441.20245913,44.54892641)(441.24745908,44.52392643)(441.3074585,44.50392873)
\curveto(441.36745896,44.49392646)(441.4324589,44.48392647)(441.5024585,44.47392873)
\curveto(441.56245877,44.47392648)(441.60745872,44.46392649)(441.6374585,44.44392873)
\curveto(441.68745864,44.40392655)(441.7324586,44.3589266)(441.7724585,44.30892873)
\curveto(441.79245854,44.2589267)(441.80745852,44.18892677)(441.8174585,44.09892873)
\lineto(441.8174585,43.82892873)
\curveto(441.81745851,43.73892722)(441.81245852,43.6539273)(441.8024585,43.57392873)
\curveto(441.78245855,43.49392746)(441.76245857,43.43392752)(441.7424585,43.39392873)
\curveto(441.72245861,43.37392758)(441.69745863,43.3539276)(441.6674585,43.33392873)
\lineto(441.5774585,43.27392873)
\curveto(441.49745883,43.24392771)(441.37745895,43.22892773)(441.2174585,43.22892873)
\curveto(441.05745927,43.23892772)(440.92245941,43.24392771)(440.8124585,43.24392873)
\lineto(432.0074585,43.24392873)
\curveto(431.88746844,43.24392771)(431.76246857,43.23892772)(431.6324585,43.22892873)
\curveto(431.49246884,43.22892773)(431.38246895,43.2539277)(431.3024585,43.30392873)
\curveto(431.24246909,43.34392761)(431.19246914,43.40892755)(431.1524585,43.49892873)
\curveto(431.15246918,43.51892744)(431.15246918,43.54392741)(431.1524585,43.57392873)
\curveto(431.14246919,43.60392735)(431.13746919,43.62892733)(431.1374585,43.64892873)
\curveto(431.1274692,43.78892717)(431.1274692,43.93392702)(431.1374585,44.08392873)
\curveto(431.13746919,44.24392671)(431.17746915,44.3539266)(431.2574585,44.41392873)
\curveto(431.33746899,44.46392649)(431.45246888,44.48892647)(431.6024585,44.48892873)
\lineto(432.0074585,44.48892873)
\lineto(433.7624585,44.48892873)
\lineto(434.0174585,44.48892873)
\lineto(434.3024585,44.48892873)
\curveto(434.39246594,44.49892646)(434.47746585,44.50892645)(434.5574585,44.51892873)
\curveto(434.6274657,44.53892642)(434.67746565,44.56892639)(434.7074585,44.60892873)
\curveto(434.73746559,44.64892631)(434.74246559,44.69392626)(434.7224585,44.74392873)
\curveto(434.70246563,44.79392616)(434.68246565,44.83392612)(434.6624585,44.86392873)
\curveto(434.62246571,44.91392604)(434.58246575,44.958926)(434.5424585,44.99892873)
\lineto(434.4224585,45.14892873)
\curveto(434.37246596,45.21892574)(434.327466,45.28892567)(434.2874585,45.35892873)
\lineto(434.1674585,45.59892873)
\curveto(434.07746625,45.77892518)(434.01246632,45.99392496)(433.9724585,46.24392873)
\curveto(433.9324664,46.49392446)(433.91246642,46.74892421)(433.9124585,47.00892873)
\curveto(433.91246642,47.26892369)(433.93746639,47.52392343)(433.9874585,47.77392873)
\curveto(434.0274663,48.02392293)(434.08746624,48.24392271)(434.1674585,48.43392873)
\curveto(434.33746599,48.83392212)(434.57246576,49.17892178)(434.8724585,49.46892873)
\curveto(435.17246516,49.7589212)(435.52246481,49.98892097)(435.9224585,50.15892873)
\curveto(436.0324643,50.20892075)(436.14246419,50.24892071)(436.2524585,50.27892873)
\curveto(436.35246398,50.31892064)(436.45746387,50.3589206)(436.5674585,50.39892873)
\curveto(436.67746365,50.42892053)(436.79246354,50.44892051)(436.9124585,50.45892873)
\lineto(437.2424585,50.51892873)
\curveto(437.27246306,50.52892043)(437.30746302,50.53392042)(437.3474585,50.53392873)
\curveto(437.37746295,50.53392042)(437.40746292,50.53892042)(437.4374585,50.54892873)
\curveto(437.49746283,50.56892039)(437.55746277,50.56892039)(437.6174585,50.54892873)
\curveto(437.66746266,50.53892042)(437.72246261,50.54392041)(437.7824585,50.56392873)
\moveto(438.1724585,49.22892873)
\curveto(438.12246221,49.24892171)(438.06246227,49.2539217)(437.9924585,49.24392873)
\curveto(437.92246241,49.23392172)(437.85746247,49.22892173)(437.7974585,49.22892873)
\curveto(437.6274627,49.22892173)(437.46746286,49.21892174)(437.3174585,49.19892873)
\curveto(437.16746316,49.18892177)(437.0324633,49.1589218)(436.9124585,49.10892873)
\curveto(436.81246352,49.07892188)(436.72246361,49.0539219)(436.6424585,49.03392873)
\curveto(436.56246377,49.01392194)(436.48246385,48.98392197)(436.4024585,48.94392873)
\curveto(436.15246418,48.83392212)(435.92246441,48.68392227)(435.7124585,48.49392873)
\curveto(435.49246484,48.30392265)(435.327465,48.08392287)(435.2174585,47.83392873)
\curveto(435.18746514,47.7539232)(435.16246517,47.67392328)(435.1424585,47.59392873)
\curveto(435.11246522,47.52392343)(435.08746524,47.44892351)(435.0674585,47.36892873)
\curveto(435.03746529,47.2589237)(435.02246531,47.14892381)(435.0224585,47.03892873)
\curveto(435.01246532,46.92892403)(435.00746532,46.80892415)(435.0074585,46.67892873)
\curveto(435.01746531,46.62892433)(435.0274653,46.58392437)(435.0374585,46.54392873)
\lineto(435.0374585,46.40892873)
\lineto(435.0974585,46.13892873)
\curveto(435.11746521,46.0589249)(435.14746518,45.97892498)(435.1874585,45.89892873)
\curveto(435.327465,45.5589254)(435.53746479,45.28892567)(435.8174585,45.08892873)
\curveto(436.08746424,44.88892607)(436.40746392,44.72892623)(436.7774585,44.60892873)
\curveto(436.88746344,44.56892639)(436.99746333,44.54392641)(437.1074585,44.53392873)
\curveto(437.21746311,44.52392643)(437.332463,44.50392645)(437.4524585,44.47392873)
\curveto(437.50246283,44.46392649)(437.54746278,44.46392649)(437.5874585,44.47392873)
\curveto(437.6274627,44.48392647)(437.67246266,44.47892648)(437.7224585,44.45892873)
\curveto(437.77246256,44.44892651)(437.84746248,44.44392651)(437.9474585,44.44392873)
\curveto(438.03746229,44.44392651)(438.10746222,44.44892651)(438.1574585,44.45892873)
\lineto(438.2774585,44.45892873)
\curveto(438.31746201,44.46892649)(438.35746197,44.47392648)(438.3974585,44.47392873)
\curveto(438.43746189,44.47392648)(438.47246186,44.47892648)(438.5024585,44.48892873)
\curveto(438.5324618,44.49892646)(438.56746176,44.50392645)(438.6074585,44.50392873)
\curveto(438.63746169,44.50392645)(438.66746166,44.50892645)(438.6974585,44.51892873)
\curveto(438.77746155,44.53892642)(438.85746147,44.5539264)(438.9374585,44.56392873)
\lineto(439.1774585,44.62392873)
\curveto(439.51746081,44.73392622)(439.80746052,44.88392607)(440.0474585,45.07392873)
\curveto(440.28746004,45.27392568)(440.48745984,45.51892544)(440.6474585,45.80892873)
\curveto(440.69745963,45.89892506)(440.73745959,45.99392496)(440.7674585,46.09392873)
\curveto(440.78745954,46.19392476)(440.81245952,46.29892466)(440.8424585,46.40892873)
\curveto(440.86245947,46.4589245)(440.87245946,46.50392445)(440.8724585,46.54392873)
\curveto(440.86245947,46.59392436)(440.86245947,46.64392431)(440.8724585,46.69392873)
\curveto(440.88245945,46.73392422)(440.88745944,46.77892418)(440.8874585,46.82892873)
\lineto(440.8874585,46.96392873)
\lineto(440.8874585,47.09892873)
\curveto(440.87745945,47.13892382)(440.87245946,47.17392378)(440.8724585,47.20392873)
\curveto(440.87245946,47.23392372)(440.86745946,47.26892369)(440.8574585,47.30892873)
\curveto(440.83745949,47.38892357)(440.82245951,47.46392349)(440.8124585,47.53392873)
\curveto(440.79245954,47.60392335)(440.76745956,47.67892328)(440.7374585,47.75892873)
\curveto(440.60745972,48.06892289)(440.43745989,48.31892264)(440.2274585,48.50892873)
\curveto(440.00746032,48.69892226)(439.74246059,48.8589221)(439.4324585,48.98892873)
\curveto(439.29246104,49.03892192)(439.15246118,49.07392188)(439.0124585,49.09392873)
\curveto(438.86246147,49.12392183)(438.71246162,49.1589218)(438.5624585,49.19892873)
\curveto(438.51246182,49.21892174)(438.46746186,49.22392173)(438.4274585,49.21392873)
\curveto(438.37746195,49.21392174)(438.327462,49.21892174)(438.2774585,49.22892873)
\lineto(438.1724585,49.22892873)
}
}
{
\newrgbcolor{curcolor}{0 0 0}
\pscustom[linestyle=none,fillstyle=solid,fillcolor=curcolor]
{
\newpath
\moveto(433.9124585,55.69017873)
\curveto(433.91246642,55.92017394)(433.97246636,56.05017381)(434.0924585,56.08017873)
\curveto(434.20246613,56.11017375)(434.36746596,56.12517374)(434.5874585,56.12517873)
\lineto(434.8724585,56.12517873)
\curveto(434.96246537,56.12517374)(435.03746529,56.10017376)(435.0974585,56.05017873)
\curveto(435.17746515,55.99017387)(435.22246511,55.90517396)(435.2324585,55.79517873)
\curveto(435.2324651,55.68517418)(435.24746508,55.57517429)(435.2774585,55.46517873)
\curveto(435.30746502,55.32517454)(435.33746499,55.19017467)(435.3674585,55.06017873)
\curveto(435.39746493,54.94017492)(435.43746489,54.82517504)(435.4874585,54.71517873)
\curveto(435.61746471,54.42517544)(435.79746453,54.19017567)(436.0274585,54.01017873)
\curveto(436.24746408,53.83017603)(436.50246383,53.67517619)(436.7924585,53.54517873)
\curveto(436.90246343,53.50517636)(437.01746331,53.47517639)(437.1374585,53.45517873)
\curveto(437.24746308,53.43517643)(437.36246297,53.41017645)(437.4824585,53.38017873)
\curveto(437.5324628,53.37017649)(437.58246275,53.3651765)(437.6324585,53.36517873)
\curveto(437.68246265,53.37517649)(437.7324626,53.37517649)(437.7824585,53.36517873)
\curveto(437.90246243,53.33517653)(438.04246229,53.32017654)(438.2024585,53.32017873)
\curveto(438.35246198,53.33017653)(438.49746183,53.33517653)(438.6374585,53.33517873)
\lineto(440.4824585,53.33517873)
\lineto(440.8274585,53.33517873)
\curveto(440.94745938,53.33517653)(441.06245927,53.33017653)(441.1724585,53.32017873)
\curveto(441.28245905,53.31017655)(441.37745895,53.30517656)(441.4574585,53.30517873)
\curveto(441.53745879,53.31517655)(441.60745872,53.29517657)(441.6674585,53.24517873)
\curveto(441.73745859,53.19517667)(441.77745855,53.11517675)(441.7874585,53.00517873)
\curveto(441.79745853,52.90517696)(441.80245853,52.79517707)(441.8024585,52.67517873)
\lineto(441.8024585,52.40517873)
\curveto(441.78245855,52.35517751)(441.76745856,52.30517756)(441.7574585,52.25517873)
\curveto(441.73745859,52.21517765)(441.71245862,52.18517768)(441.6824585,52.16517873)
\curveto(441.61245872,52.11517775)(441.5274588,52.08517778)(441.4274585,52.07517873)
\lineto(441.0974585,52.07517873)
\lineto(439.9424585,52.07517873)
\lineto(435.7874585,52.07517873)
\lineto(434.7524585,52.07517873)
\lineto(434.4524585,52.07517873)
\curveto(434.35246598,52.08517778)(434.26746606,52.11517775)(434.1974585,52.16517873)
\curveto(434.15746617,52.19517767)(434.1274662,52.24517762)(434.1074585,52.31517873)
\curveto(434.08746624,52.39517747)(434.07746625,52.48017738)(434.0774585,52.57017873)
\curveto(434.06746626,52.6601772)(434.06746626,52.75017711)(434.0774585,52.84017873)
\curveto(434.08746624,52.93017693)(434.10246623,53.00017686)(434.1224585,53.05017873)
\curveto(434.15246618,53.13017673)(434.21246612,53.18017668)(434.3024585,53.20017873)
\curveto(434.38246595,53.23017663)(434.47246586,53.24517662)(434.5724585,53.24517873)
\lineto(434.8724585,53.24517873)
\curveto(434.97246536,53.24517662)(435.06246527,53.2651766)(435.1424585,53.30517873)
\curveto(435.16246517,53.31517655)(435.17746515,53.32517654)(435.1874585,53.33517873)
\lineto(435.2324585,53.38017873)
\curveto(435.2324651,53.49017637)(435.18746514,53.58017628)(435.0974585,53.65017873)
\curveto(434.99746533,53.72017614)(434.91746541,53.78017608)(434.8574585,53.83017873)
\lineto(434.7674585,53.92017873)
\curveto(434.65746567,54.01017585)(434.54246579,54.13517573)(434.4224585,54.29517873)
\curveto(434.30246603,54.45517541)(434.21246612,54.60517526)(434.1524585,54.74517873)
\curveto(434.10246623,54.83517503)(434.06746626,54.93017493)(434.0474585,55.03017873)
\curveto(434.01746631,55.13017473)(433.98746634,55.23517463)(433.9574585,55.34517873)
\curveto(433.94746638,55.40517446)(433.94246639,55.4651744)(433.9424585,55.52517873)
\curveto(433.9324664,55.58517428)(433.92246641,55.64017422)(433.9124585,55.69017873)
}
}
{
\newrgbcolor{curcolor}{0 0 0}
\pscustom[linestyle=none,fillstyle=solid,fillcolor=curcolor]
{
}
}
{
\newrgbcolor{curcolor}{0 0 0}
\pscustom[linestyle=none,fillstyle=solid,fillcolor=curcolor]
{
\newpath
\moveto(431.2124585,64.24510061)
\curveto(431.20246913,64.93509597)(431.32246901,65.53509537)(431.5724585,66.04510061)
\curveto(431.82246851,66.56509434)(432.15746817,66.96009395)(432.5774585,67.23010061)
\curveto(432.65746767,67.28009363)(432.74746758,67.32509358)(432.8474585,67.36510061)
\curveto(432.93746739,67.4050935)(433.0324673,67.45009346)(433.1324585,67.50010061)
\curveto(433.2324671,67.54009337)(433.332467,67.57009334)(433.4324585,67.59010061)
\curveto(433.5324668,67.6100933)(433.63746669,67.63009328)(433.7474585,67.65010061)
\curveto(433.79746653,67.67009324)(433.84246649,67.67509323)(433.8824585,67.66510061)
\curveto(433.92246641,67.65509325)(433.96746636,67.66009325)(434.0174585,67.68010061)
\curveto(434.06746626,67.69009322)(434.15246618,67.69509321)(434.2724585,67.69510061)
\curveto(434.38246595,67.69509321)(434.46746586,67.69009322)(434.5274585,67.68010061)
\curveto(434.58746574,67.66009325)(434.64746568,67.65009326)(434.7074585,67.65010061)
\curveto(434.76746556,67.66009325)(434.8274655,67.65509325)(434.8874585,67.63510061)
\curveto(435.0274653,67.59509331)(435.16246517,67.56009335)(435.2924585,67.53010061)
\curveto(435.42246491,67.50009341)(435.54746478,67.46009345)(435.6674585,67.41010061)
\curveto(435.80746452,67.35009356)(435.9324644,67.28009363)(436.0424585,67.20010061)
\curveto(436.15246418,67.13009378)(436.26246407,67.05509385)(436.3724585,66.97510061)
\lineto(436.4324585,66.91510061)
\curveto(436.45246388,66.905094)(436.47246386,66.89009402)(436.4924585,66.87010061)
\curveto(436.65246368,66.75009416)(436.79746353,66.61509429)(436.9274585,66.46510061)
\curveto(437.05746327,66.31509459)(437.18246315,66.15509475)(437.3024585,65.98510061)
\curveto(437.52246281,65.67509523)(437.7274626,65.38009553)(437.9174585,65.10010061)
\curveto(438.05746227,64.87009604)(438.19246214,64.64009627)(438.3224585,64.41010061)
\curveto(438.45246188,64.19009672)(438.58746174,63.97009694)(438.7274585,63.75010061)
\curveto(438.89746143,63.50009741)(439.07746125,63.26009765)(439.2674585,63.03010061)
\curveto(439.45746087,62.8100981)(439.68246065,62.62009829)(439.9424585,62.46010061)
\curveto(440.00246033,62.42009849)(440.06246027,62.38509852)(440.1224585,62.35510061)
\curveto(440.17246016,62.32509858)(440.23746009,62.29509861)(440.3174585,62.26510061)
\curveto(440.38745994,62.24509866)(440.44745988,62.24009867)(440.4974585,62.25010061)
\curveto(440.56745976,62.27009864)(440.62245971,62.3050986)(440.6624585,62.35510061)
\curveto(440.69245964,62.4050985)(440.71245962,62.46509844)(440.7224585,62.53510061)
\lineto(440.7224585,62.77510061)
\lineto(440.7224585,63.52510061)
\lineto(440.7224585,66.33010061)
\lineto(440.7224585,66.99010061)
\curveto(440.72245961,67.08009383)(440.7274596,67.16509374)(440.7374585,67.24510061)
\curveto(440.73745959,67.32509358)(440.75745957,67.39009352)(440.7974585,67.44010061)
\curveto(440.83745949,67.49009342)(440.91245942,67.53009338)(441.0224585,67.56010061)
\curveto(441.12245921,67.60009331)(441.22245911,67.6100933)(441.3224585,67.59010061)
\lineto(441.4574585,67.59010061)
\curveto(441.5274588,67.57009334)(441.58745874,67.55009336)(441.6374585,67.53010061)
\curveto(441.68745864,67.5100934)(441.7274586,67.47509343)(441.7574585,67.42510061)
\curveto(441.79745853,67.37509353)(441.81745851,67.3050936)(441.8174585,67.21510061)
\lineto(441.8174585,66.94510061)
\lineto(441.8174585,66.04510061)
\lineto(441.8174585,62.53510061)
\lineto(441.8174585,61.47010061)
\curveto(441.81745851,61.39009952)(441.82245851,61.30009961)(441.8324585,61.20010061)
\curveto(441.8324585,61.10009981)(441.82245851,61.01509989)(441.8024585,60.94510061)
\curveto(441.7324586,60.73510017)(441.55245878,60.67010024)(441.2624585,60.75010061)
\curveto(441.22245911,60.76010015)(441.18745914,60.76010015)(441.1574585,60.75010061)
\curveto(441.11745921,60.75010016)(441.07245926,60.76010015)(441.0224585,60.78010061)
\curveto(440.94245939,60.80010011)(440.85745947,60.82010009)(440.7674585,60.84010061)
\curveto(440.67745965,60.86010005)(440.59245974,60.88510002)(440.5124585,60.91510061)
\curveto(440.02246031,61.07509983)(439.60746072,61.27509963)(439.2674585,61.51510061)
\curveto(439.01746131,61.69509921)(438.79246154,61.90009901)(438.5924585,62.13010061)
\curveto(438.38246195,62.36009855)(438.18746214,62.60009831)(438.0074585,62.85010061)
\curveto(437.8274625,63.1100978)(437.65746267,63.37509753)(437.4974585,63.64510061)
\curveto(437.327463,63.92509698)(437.15246318,64.19509671)(436.9724585,64.45510061)
\curveto(436.89246344,64.56509634)(436.81746351,64.67009624)(436.7474585,64.77010061)
\curveto(436.67746365,64.88009603)(436.60246373,64.99009592)(436.5224585,65.10010061)
\curveto(436.49246384,65.14009577)(436.46246387,65.17509573)(436.4324585,65.20510061)
\curveto(436.39246394,65.24509566)(436.36246397,65.28509562)(436.3424585,65.32510061)
\curveto(436.2324641,65.46509544)(436.10746422,65.59009532)(435.9674585,65.70010061)
\curveto(435.93746439,65.72009519)(435.91246442,65.74509516)(435.8924585,65.77510061)
\curveto(435.86246447,65.8050951)(435.8324645,65.83009508)(435.8024585,65.85010061)
\curveto(435.70246463,65.93009498)(435.60246473,65.99509491)(435.5024585,66.04510061)
\curveto(435.40246493,66.1050948)(435.29246504,66.16009475)(435.1724585,66.21010061)
\curveto(435.10246523,66.24009467)(435.0274653,66.26009465)(434.9474585,66.27010061)
\lineto(434.7074585,66.33010061)
\lineto(434.6174585,66.33010061)
\curveto(434.58746574,66.34009457)(434.55746577,66.34509456)(434.5274585,66.34510061)
\curveto(434.45746587,66.36509454)(434.36246597,66.37009454)(434.2424585,66.36010061)
\curveto(434.11246622,66.36009455)(434.01246632,66.35009456)(433.9424585,66.33010061)
\curveto(433.86246647,66.3100946)(433.78746654,66.29009462)(433.7174585,66.27010061)
\curveto(433.63746669,66.26009465)(433.55746677,66.24009467)(433.4774585,66.21010061)
\curveto(433.23746709,66.10009481)(433.03746729,65.95009496)(432.8774585,65.76010061)
\curveto(432.70746762,65.58009533)(432.56746776,65.36009555)(432.4574585,65.10010061)
\curveto(432.43746789,65.03009588)(432.42246791,64.96009595)(432.4124585,64.89010061)
\curveto(432.39246794,64.82009609)(432.37246796,64.74509616)(432.3524585,64.66510061)
\curveto(432.332468,64.58509632)(432.32246801,64.47509643)(432.3224585,64.33510061)
\curveto(432.32246801,64.2050967)(432.332468,64.10009681)(432.3524585,64.02010061)
\curveto(432.36246797,63.96009695)(432.36746796,63.905097)(432.3674585,63.85510061)
\curveto(432.36746796,63.8050971)(432.37746795,63.75509715)(432.3974585,63.70510061)
\curveto(432.43746789,63.6050973)(432.47746785,63.5100974)(432.5174585,63.42010061)
\curveto(432.55746777,63.34009757)(432.60246773,63.26009765)(432.6524585,63.18010061)
\curveto(432.67246766,63.15009776)(432.69746763,63.12009779)(432.7274585,63.09010061)
\curveto(432.75746757,63.07009784)(432.78246755,63.04509786)(432.8024585,63.01510061)
\lineto(432.8774585,62.94010061)
\curveto(432.89746743,62.910098)(432.91746741,62.88509802)(432.9374585,62.86510061)
\lineto(433.1474585,62.71510061)
\curveto(433.20746712,62.67509823)(433.27246706,62.63009828)(433.3424585,62.58010061)
\curveto(433.4324669,62.52009839)(433.53746679,62.47009844)(433.6574585,62.43010061)
\curveto(433.76746656,62.40009851)(433.87746645,62.36509854)(433.9874585,62.32510061)
\curveto(434.09746623,62.28509862)(434.24246609,62.26009865)(434.4224585,62.25010061)
\curveto(434.59246574,62.24009867)(434.71746561,62.2100987)(434.7974585,62.16010061)
\curveto(434.87746545,62.1100988)(434.92246541,62.03509887)(434.9324585,61.93510061)
\curveto(434.94246539,61.83509907)(434.94746538,61.72509918)(434.9474585,61.60510061)
\curveto(434.94746538,61.56509934)(434.95246538,61.52509938)(434.9624585,61.48510061)
\curveto(434.96246537,61.44509946)(434.95746537,61.4100995)(434.9474585,61.38010061)
\curveto(434.9274654,61.33009958)(434.91746541,61.28009963)(434.9174585,61.23010061)
\curveto(434.91746541,61.19009972)(434.90746542,61.15009976)(434.8874585,61.11010061)
\curveto(434.8274655,61.02009989)(434.69246564,60.97509993)(434.4824585,60.97510061)
\lineto(434.3624585,60.97510061)
\curveto(434.30246603,60.98509992)(434.24246609,60.99009992)(434.1824585,60.99010061)
\curveto(434.11246622,61.00009991)(434.04746628,61.0100999)(433.9874585,61.02010061)
\curveto(433.87746645,61.04009987)(433.77746655,61.06009985)(433.6874585,61.08010061)
\curveto(433.58746674,61.10009981)(433.49246684,61.13009978)(433.4024585,61.17010061)
\curveto(433.332467,61.19009972)(433.27246706,61.2100997)(433.2224585,61.23010061)
\lineto(433.0424585,61.29010061)
\curveto(432.78246755,61.4100995)(432.53746779,61.56509934)(432.3074585,61.75510061)
\curveto(432.07746825,61.95509895)(431.89246844,62.17009874)(431.7524585,62.40010061)
\curveto(431.67246866,62.5100984)(431.60746872,62.62509828)(431.5574585,62.74510061)
\lineto(431.4074585,63.13510061)
\curveto(431.35746897,63.24509766)(431.327469,63.36009755)(431.3174585,63.48010061)
\curveto(431.29746903,63.60009731)(431.27246906,63.72509718)(431.2424585,63.85510061)
\curveto(431.24246909,63.92509698)(431.24246909,63.99009692)(431.2424585,64.05010061)
\curveto(431.2324691,64.1100968)(431.22246911,64.17509673)(431.2124585,64.24510061)
}
}
{
\newrgbcolor{curcolor}{0 0 0}
\pscustom[linestyle=none,fillstyle=solid,fillcolor=curcolor]
{
\newpath
\moveto(431.2124585,72.45970998)
\curveto(431.18246915,74.08970454)(431.73746859,75.13970349)(432.8774585,75.60970998)
\curveto(433.10746722,75.70970292)(433.39746693,75.77470286)(433.7474585,75.80470998)
\curveto(434.08746624,75.84470279)(434.39746593,75.81970281)(434.6774585,75.72970998)
\curveto(434.93746539,75.63970299)(435.16246517,75.51970311)(435.3524585,75.36970998)
\curveto(435.39246494,75.34970328)(435.4274649,75.32470331)(435.4574585,75.29470998)
\curveto(435.47746485,75.26470337)(435.50246483,75.23970339)(435.5324585,75.21970998)
\lineto(435.6524585,75.12970998)
\curveto(435.68246465,75.09970353)(435.70746462,75.06470357)(435.7274585,75.02470998)
\curveto(435.77746455,74.97470366)(435.82246451,74.91970371)(435.8624585,74.85970998)
\curveto(435.90246443,74.80970382)(435.95246438,74.76470387)(436.0124585,74.72470998)
\curveto(436.05246428,74.68470395)(436.10246423,74.66970396)(436.1624585,74.67970998)
\curveto(436.21246412,74.68970394)(436.25746407,74.71970391)(436.2974585,74.76970998)
\curveto(436.33746399,74.81970381)(436.37746395,74.87470376)(436.4174585,74.93470998)
\curveto(436.44746388,75.00470363)(436.47746385,75.06970356)(436.5074585,75.12970998)
\curveto(436.53746379,75.18970344)(436.56746376,75.23970339)(436.5974585,75.27970998)
\curveto(436.81746351,75.59970303)(437.1274632,75.85470278)(437.5274585,76.04470998)
\curveto(437.61746271,76.08470255)(437.71246262,76.11470252)(437.8124585,76.13470998)
\curveto(437.90246243,76.16470247)(437.99246234,76.18970244)(438.0824585,76.20970998)
\curveto(438.1324622,76.21970241)(438.18246215,76.22470241)(438.2324585,76.22470998)
\curveto(438.27246206,76.2347024)(438.31746201,76.24470239)(438.3674585,76.25470998)
\curveto(438.41746191,76.26470237)(438.46746186,76.26470237)(438.5174585,76.25470998)
\curveto(438.56746176,76.24470239)(438.61746171,76.24970238)(438.6674585,76.26970998)
\curveto(438.71746161,76.27970235)(438.77746155,76.28470235)(438.8474585,76.28470998)
\curveto(438.91746141,76.28470235)(438.97746135,76.27470236)(439.0274585,76.25470998)
\lineto(439.2524585,76.25470998)
\lineto(439.4924585,76.19470998)
\curveto(439.56246077,76.18470245)(439.6324607,76.16970246)(439.7024585,76.14970998)
\curveto(439.79246054,76.11970251)(439.87746045,76.08970254)(439.9574585,76.05970998)
\curveto(440.03746029,76.03970259)(440.11746021,76.00970262)(440.1974585,75.96970998)
\curveto(440.25746007,75.94970268)(440.31746001,75.91970271)(440.3774585,75.87970998)
\curveto(440.4274599,75.84970278)(440.47745985,75.81470282)(440.5274585,75.77470998)
\curveto(440.83745949,75.57470306)(441.09745923,75.32470331)(441.3074585,75.02470998)
\curveto(441.50745882,74.72470391)(441.67245866,74.37970425)(441.8024585,73.98970998)
\curveto(441.84245849,73.86970476)(441.86745846,73.73970489)(441.8774585,73.59970998)
\curveto(441.89745843,73.46970516)(441.92245841,73.3347053)(441.9524585,73.19470998)
\curveto(441.96245837,73.12470551)(441.96745836,73.05470558)(441.9674585,72.98470998)
\curveto(441.96745836,72.92470571)(441.97245836,72.85970577)(441.9824585,72.78970998)
\curveto(441.99245834,72.74970588)(441.99745833,72.68970594)(441.9974585,72.60970998)
\curveto(441.99745833,72.53970609)(441.99245834,72.48970614)(441.9824585,72.45970998)
\curveto(441.97245836,72.40970622)(441.96745836,72.36470627)(441.9674585,72.32470998)
\lineto(441.9674585,72.20470998)
\curveto(441.94745838,72.10470653)(441.9324584,72.00470663)(441.9224585,71.90470998)
\curveto(441.91245842,71.80470683)(441.89745843,71.70970692)(441.8774585,71.61970998)
\curveto(441.84745848,71.50970712)(441.82245851,71.39970723)(441.8024585,71.28970998)
\curveto(441.77245856,71.18970744)(441.7324586,71.08470755)(441.6824585,70.97470998)
\curveto(441.52245881,70.60470803)(441.32245901,70.28970834)(441.0824585,70.02970998)
\curveto(440.8324595,69.76970886)(440.52245981,69.55970907)(440.1524585,69.39970998)
\curveto(440.06246027,69.35970927)(439.96746036,69.32470931)(439.8674585,69.29470998)
\curveto(439.76746056,69.26470937)(439.66246067,69.2347094)(439.5524585,69.20470998)
\curveto(439.50246083,69.18470945)(439.45246088,69.17470946)(439.4024585,69.17470998)
\curveto(439.34246099,69.17470946)(439.28246105,69.16470947)(439.2224585,69.14470998)
\curveto(439.16246117,69.12470951)(439.08246125,69.11470952)(438.9824585,69.11470998)
\curveto(438.88246145,69.11470952)(438.80746152,69.1297095)(438.7574585,69.15970998)
\curveto(438.7274616,69.16970946)(438.70246163,69.18470945)(438.6824585,69.20470998)
\lineto(438.6224585,69.26470998)
\curveto(438.60246173,69.30470933)(438.58746174,69.36470927)(438.5774585,69.44470998)
\curveto(438.56746176,69.5347091)(438.56246177,69.62470901)(438.5624585,69.71470998)
\curveto(438.56246177,69.80470883)(438.56746176,69.88970874)(438.5774585,69.96970998)
\curveto(438.58746174,70.05970857)(438.59746173,70.12470851)(438.6074585,70.16470998)
\curveto(438.6274617,70.18470845)(438.64246169,70.20470843)(438.6524585,70.22470998)
\curveto(438.65246168,70.24470839)(438.66246167,70.26470837)(438.6824585,70.28470998)
\curveto(438.77246156,70.35470828)(438.88746144,70.39470824)(439.0274585,70.40470998)
\curveto(439.16746116,70.42470821)(439.29246104,70.45470818)(439.4024585,70.49470998)
\lineto(439.7624585,70.64470998)
\curveto(439.87246046,70.69470794)(439.97746035,70.75970787)(440.0774585,70.83970998)
\curveto(440.10746022,70.85970777)(440.1324602,70.87970775)(440.1524585,70.89970998)
\curveto(440.17246016,70.9297077)(440.19746013,70.95470768)(440.2274585,70.97470998)
\curveto(440.28746004,71.01470762)(440.33246,71.04970758)(440.3624585,71.07970998)
\curveto(440.39245994,71.11970751)(440.42245991,71.15470748)(440.4524585,71.18470998)
\curveto(440.48245985,71.22470741)(440.51245982,71.26970736)(440.5424585,71.31970998)
\curveto(440.60245973,71.40970722)(440.65245968,71.50470713)(440.6924585,71.60470998)
\lineto(440.8124585,71.93470998)
\curveto(440.86245947,72.08470655)(440.89245944,72.28470635)(440.9024585,72.53470998)
\curveto(440.91245942,72.78470585)(440.89245944,72.99470564)(440.8424585,73.16470998)
\curveto(440.82245951,73.24470539)(440.80745952,73.31470532)(440.7974585,73.37470998)
\lineto(440.7374585,73.58470998)
\curveto(440.61745971,73.86470477)(440.46745986,74.10470453)(440.2874585,74.30470998)
\curveto(440.10746022,74.51470412)(439.87746045,74.67970395)(439.5974585,74.79970998)
\curveto(439.5274608,74.8297038)(439.45746087,74.84970378)(439.3874585,74.85970998)
\lineto(439.1474585,74.91970998)
\curveto(439.00746132,74.95970367)(438.84746148,74.96970366)(438.6674585,74.94970998)
\curveto(438.47746185,74.9297037)(438.327462,74.89970373)(438.2174585,74.85970998)
\curveto(437.83746249,74.7297039)(437.54746278,74.54470409)(437.3474585,74.30470998)
\curveto(437.14746318,74.07470456)(436.98746334,73.76470487)(436.8674585,73.37470998)
\curveto(436.83746349,73.26470537)(436.81746351,73.14470549)(436.8074585,73.01470998)
\curveto(436.79746353,72.89470574)(436.79246354,72.76970586)(436.7924585,72.63970998)
\curveto(436.79246354,72.47970615)(436.78746354,72.33970629)(436.7774585,72.21970998)
\curveto(436.76746356,72.09970653)(436.70746362,72.01470662)(436.5974585,71.96470998)
\curveto(436.56746376,71.94470669)(436.5324638,71.9347067)(436.4924585,71.93470998)
\lineto(436.3574585,71.93470998)
\curveto(436.25746407,71.92470671)(436.16246417,71.92470671)(436.0724585,71.93470998)
\curveto(435.98246435,71.95470668)(435.91746441,71.99470664)(435.8774585,72.05470998)
\curveto(435.84746448,72.09470654)(435.8274645,72.1347065)(435.8174585,72.17470998)
\curveto(435.80746452,72.22470641)(435.79746453,72.27970635)(435.7874585,72.33970998)
\curveto(435.77746455,72.35970627)(435.77746455,72.38470625)(435.7874585,72.41470998)
\curveto(435.78746454,72.44470619)(435.78246455,72.46970616)(435.7724585,72.48970998)
\lineto(435.7724585,72.62470998)
\curveto(435.75246458,72.7347059)(435.74246459,72.8347058)(435.7424585,72.92470998)
\curveto(435.7324646,73.02470561)(435.71246462,73.11970551)(435.6824585,73.20970998)
\curveto(435.57246476,73.5297051)(435.4274649,73.78470485)(435.2474585,73.97470998)
\curveto(435.06746526,74.16470447)(434.81746551,74.31470432)(434.4974585,74.42470998)
\curveto(434.39746593,74.45470418)(434.27246606,74.47470416)(434.1224585,74.48470998)
\curveto(433.96246637,74.50470413)(433.81746651,74.49970413)(433.6874585,74.46970998)
\curveto(433.61746671,74.44970418)(433.55246678,74.4297042)(433.4924585,74.40970998)
\curveto(433.42246691,74.39970423)(433.35746697,74.37970425)(433.2974585,74.34970998)
\curveto(433.05746727,74.24970438)(432.86746746,74.10470453)(432.7274585,73.91470998)
\curveto(432.58746774,73.72470491)(432.47746785,73.49970513)(432.3974585,73.23970998)
\curveto(432.37746795,73.17970545)(432.36746796,73.11970551)(432.3674585,73.05970998)
\curveto(432.36746796,72.99970563)(432.35746797,72.9347057)(432.3374585,72.86470998)
\curveto(432.31746801,72.78470585)(432.30746802,72.68970594)(432.3074585,72.57970998)
\curveto(432.30746802,72.46970616)(432.31746801,72.37470626)(432.3374585,72.29470998)
\curveto(432.35746797,72.24470639)(432.36746796,72.19470644)(432.3674585,72.14470998)
\curveto(432.36746796,72.10470653)(432.37746795,72.05970657)(432.3974585,72.00970998)
\curveto(432.44746788,71.8297068)(432.52246781,71.65970697)(432.6224585,71.49970998)
\curveto(432.71246762,71.34970728)(432.8274675,71.21970741)(432.9674585,71.10970998)
\curveto(433.08746724,71.01970761)(433.21746711,70.93970769)(433.3574585,70.86970998)
\curveto(433.49746683,70.79970783)(433.65246668,70.7347079)(433.8224585,70.67470998)
\curveto(433.9324664,70.64470799)(434.05246628,70.62470801)(434.1824585,70.61470998)
\curveto(434.30246603,70.60470803)(434.40246593,70.56970806)(434.4824585,70.50970998)
\curveto(434.52246581,70.48970814)(434.56246577,70.4297082)(434.6024585,70.32970998)
\curveto(434.61246572,70.28970834)(434.62246571,70.2297084)(434.6324585,70.14970998)
\lineto(434.6324585,69.89470998)
\curveto(434.62246571,69.80470883)(434.61246572,69.71970891)(434.6024585,69.63970998)
\curveto(434.59246574,69.56970906)(434.57746575,69.51970911)(434.5574585,69.48970998)
\curveto(434.5274658,69.44970918)(434.47246586,69.41470922)(434.3924585,69.38470998)
\curveto(434.31246602,69.35470928)(434.2274661,69.34970928)(434.1374585,69.36970998)
\curveto(434.08746624,69.37970925)(434.03746629,69.38470925)(433.9874585,69.38470998)
\lineto(433.8074585,69.41470998)
\curveto(433.70746662,69.44470919)(433.60746672,69.46970916)(433.5074585,69.48970998)
\curveto(433.40746692,69.51970911)(433.31746701,69.55470908)(433.2374585,69.59470998)
\curveto(433.1274672,69.64470899)(433.02246731,69.68970894)(432.9224585,69.72970998)
\curveto(432.81246752,69.76970886)(432.70746762,69.81970881)(432.6074585,69.87970998)
\curveto(432.06746826,70.20970842)(431.67246866,70.67970795)(431.4224585,71.28970998)
\curveto(431.37246896,71.40970722)(431.33746899,71.5347071)(431.3174585,71.66470998)
\curveto(431.29746903,71.80470683)(431.27246906,71.94470669)(431.2424585,72.08470998)
\curveto(431.2324691,72.14470649)(431.2274691,72.20470643)(431.2274585,72.26470998)
\curveto(431.2274691,72.3347063)(431.22246911,72.39970623)(431.2124585,72.45970998)
}
}
{
\newrgbcolor{curcolor}{0 0 0}
\pscustom[linestyle=none,fillstyle=solid,fillcolor=curcolor]
{
\newpath
\moveto(440.1824585,78.63431936)
\lineto(440.1824585,79.26431936)
\lineto(440.1824585,79.45931936)
\curveto(440.18246015,79.52931683)(440.19246014,79.58931677)(440.2124585,79.63931936)
\curveto(440.25246008,79.70931665)(440.29246004,79.7593166)(440.3324585,79.78931936)
\curveto(440.38245995,79.82931653)(440.44745988,79.84931651)(440.5274585,79.84931936)
\curveto(440.60745972,79.8593165)(440.69245964,79.86431649)(440.7824585,79.86431936)
\lineto(441.5024585,79.86431936)
\curveto(441.98245835,79.86431649)(442.39245794,79.80431655)(442.7324585,79.68431936)
\curveto(443.07245726,79.56431679)(443.34745698,79.36931699)(443.5574585,79.09931936)
\curveto(443.60745672,79.02931733)(443.65245668,78.9593174)(443.6924585,78.88931936)
\curveto(443.74245659,78.82931753)(443.78745654,78.7543176)(443.8274585,78.66431936)
\curveto(443.83745649,78.64431771)(443.84745648,78.61431774)(443.8574585,78.57431936)
\curveto(443.87745645,78.53431782)(443.88245645,78.48931787)(443.8724585,78.43931936)
\curveto(443.84245649,78.34931801)(443.76745656,78.29431806)(443.6474585,78.27431936)
\curveto(443.53745679,78.2543181)(443.44245689,78.26931809)(443.3624585,78.31931936)
\curveto(443.29245704,78.34931801)(443.2274571,78.39431796)(443.1674585,78.45431936)
\curveto(443.11745721,78.52431783)(443.06745726,78.58931777)(443.0174585,78.64931936)
\curveto(442.96745736,78.71931764)(442.89245744,78.77931758)(442.7924585,78.82931936)
\curveto(442.70245763,78.88931747)(442.61245772,78.93931742)(442.5224585,78.97931936)
\curveto(442.49245784,78.99931736)(442.4324579,79.02431733)(442.3424585,79.05431936)
\curveto(442.26245807,79.08431727)(442.19245814,79.08931727)(442.1324585,79.06931936)
\curveto(441.99245834,79.03931732)(441.90245843,78.97931738)(441.8624585,78.88931936)
\curveto(441.8324585,78.80931755)(441.81745851,78.71931764)(441.8174585,78.61931936)
\curveto(441.81745851,78.51931784)(441.79245854,78.43431792)(441.7424585,78.36431936)
\curveto(441.67245866,78.27431808)(441.5324588,78.22931813)(441.3224585,78.22931936)
\lineto(440.7674585,78.22931936)
\lineto(440.5424585,78.22931936)
\curveto(440.46245987,78.23931812)(440.39745993,78.2593181)(440.3474585,78.28931936)
\curveto(440.26746006,78.34931801)(440.22246011,78.41931794)(440.2124585,78.49931936)
\curveto(440.20246013,78.51931784)(440.19746013,78.53931782)(440.1974585,78.55931936)
\curveto(440.19746013,78.58931777)(440.19246014,78.61431774)(440.1824585,78.63431936)
}
}
{
\newrgbcolor{curcolor}{0 0 0}
\pscustom[linestyle=none,fillstyle=solid,fillcolor=curcolor]
{
}
}
{
\newrgbcolor{curcolor}{0 0 0}
\pscustom[linestyle=none,fillstyle=solid,fillcolor=curcolor]
{
\newpath
\moveto(431.2124585,89.26463186)
\curveto(431.20246913,89.95462722)(431.32246901,90.55462662)(431.5724585,91.06463186)
\curveto(431.82246851,91.58462559)(432.15746817,91.9796252)(432.5774585,92.24963186)
\curveto(432.65746767,92.29962488)(432.74746758,92.34462483)(432.8474585,92.38463186)
\curveto(432.93746739,92.42462475)(433.0324673,92.46962471)(433.1324585,92.51963186)
\curveto(433.2324671,92.55962462)(433.332467,92.58962459)(433.4324585,92.60963186)
\curveto(433.5324668,92.62962455)(433.63746669,92.64962453)(433.7474585,92.66963186)
\curveto(433.79746653,92.68962449)(433.84246649,92.69462448)(433.8824585,92.68463186)
\curveto(433.92246641,92.6746245)(433.96746636,92.6796245)(434.0174585,92.69963186)
\curveto(434.06746626,92.70962447)(434.15246618,92.71462446)(434.2724585,92.71463186)
\curveto(434.38246595,92.71462446)(434.46746586,92.70962447)(434.5274585,92.69963186)
\curveto(434.58746574,92.6796245)(434.64746568,92.66962451)(434.7074585,92.66963186)
\curveto(434.76746556,92.6796245)(434.8274655,92.6746245)(434.8874585,92.65463186)
\curveto(435.0274653,92.61462456)(435.16246517,92.5796246)(435.2924585,92.54963186)
\curveto(435.42246491,92.51962466)(435.54746478,92.4796247)(435.6674585,92.42963186)
\curveto(435.80746452,92.36962481)(435.9324644,92.29962488)(436.0424585,92.21963186)
\curveto(436.15246418,92.14962503)(436.26246407,92.0746251)(436.3724585,91.99463186)
\lineto(436.4324585,91.93463186)
\curveto(436.45246388,91.92462525)(436.47246386,91.90962527)(436.4924585,91.88963186)
\curveto(436.65246368,91.76962541)(436.79746353,91.63462554)(436.9274585,91.48463186)
\curveto(437.05746327,91.33462584)(437.18246315,91.174626)(437.3024585,91.00463186)
\curveto(437.52246281,90.69462648)(437.7274626,90.39962678)(437.9174585,90.11963186)
\curveto(438.05746227,89.88962729)(438.19246214,89.65962752)(438.3224585,89.42963186)
\curveto(438.45246188,89.20962797)(438.58746174,88.98962819)(438.7274585,88.76963186)
\curveto(438.89746143,88.51962866)(439.07746125,88.2796289)(439.2674585,88.04963186)
\curveto(439.45746087,87.82962935)(439.68246065,87.63962954)(439.9424585,87.47963186)
\curveto(440.00246033,87.43962974)(440.06246027,87.40462977)(440.1224585,87.37463186)
\curveto(440.17246016,87.34462983)(440.23746009,87.31462986)(440.3174585,87.28463186)
\curveto(440.38745994,87.26462991)(440.44745988,87.25962992)(440.4974585,87.26963186)
\curveto(440.56745976,87.28962989)(440.62245971,87.32462985)(440.6624585,87.37463186)
\curveto(440.69245964,87.42462975)(440.71245962,87.48462969)(440.7224585,87.55463186)
\lineto(440.7224585,87.79463186)
\lineto(440.7224585,88.54463186)
\lineto(440.7224585,91.34963186)
\lineto(440.7224585,92.00963186)
\curveto(440.72245961,92.09962508)(440.7274596,92.18462499)(440.7374585,92.26463186)
\curveto(440.73745959,92.34462483)(440.75745957,92.40962477)(440.7974585,92.45963186)
\curveto(440.83745949,92.50962467)(440.91245942,92.54962463)(441.0224585,92.57963186)
\curveto(441.12245921,92.61962456)(441.22245911,92.62962455)(441.3224585,92.60963186)
\lineto(441.4574585,92.60963186)
\curveto(441.5274588,92.58962459)(441.58745874,92.56962461)(441.6374585,92.54963186)
\curveto(441.68745864,92.52962465)(441.7274586,92.49462468)(441.7574585,92.44463186)
\curveto(441.79745853,92.39462478)(441.81745851,92.32462485)(441.8174585,92.23463186)
\lineto(441.8174585,91.96463186)
\lineto(441.8174585,91.06463186)
\lineto(441.8174585,87.55463186)
\lineto(441.8174585,86.48963186)
\curveto(441.81745851,86.40963077)(441.82245851,86.31963086)(441.8324585,86.21963186)
\curveto(441.8324585,86.11963106)(441.82245851,86.03463114)(441.8024585,85.96463186)
\curveto(441.7324586,85.75463142)(441.55245878,85.68963149)(441.2624585,85.76963186)
\curveto(441.22245911,85.7796314)(441.18745914,85.7796314)(441.1574585,85.76963186)
\curveto(441.11745921,85.76963141)(441.07245926,85.7796314)(441.0224585,85.79963186)
\curveto(440.94245939,85.81963136)(440.85745947,85.83963134)(440.7674585,85.85963186)
\curveto(440.67745965,85.8796313)(440.59245974,85.90463127)(440.5124585,85.93463186)
\curveto(440.02246031,86.09463108)(439.60746072,86.29463088)(439.2674585,86.53463186)
\curveto(439.01746131,86.71463046)(438.79246154,86.91963026)(438.5924585,87.14963186)
\curveto(438.38246195,87.3796298)(438.18746214,87.61962956)(438.0074585,87.86963186)
\curveto(437.8274625,88.12962905)(437.65746267,88.39462878)(437.4974585,88.66463186)
\curveto(437.327463,88.94462823)(437.15246318,89.21462796)(436.9724585,89.47463186)
\curveto(436.89246344,89.58462759)(436.81746351,89.68962749)(436.7474585,89.78963186)
\curveto(436.67746365,89.89962728)(436.60246373,90.00962717)(436.5224585,90.11963186)
\curveto(436.49246384,90.15962702)(436.46246387,90.19462698)(436.4324585,90.22463186)
\curveto(436.39246394,90.26462691)(436.36246397,90.30462687)(436.3424585,90.34463186)
\curveto(436.2324641,90.48462669)(436.10746422,90.60962657)(435.9674585,90.71963186)
\curveto(435.93746439,90.73962644)(435.91246442,90.76462641)(435.8924585,90.79463186)
\curveto(435.86246447,90.82462635)(435.8324645,90.84962633)(435.8024585,90.86963186)
\curveto(435.70246463,90.94962623)(435.60246473,91.01462616)(435.5024585,91.06463186)
\curveto(435.40246493,91.12462605)(435.29246504,91.179626)(435.1724585,91.22963186)
\curveto(435.10246523,91.25962592)(435.0274653,91.2796259)(434.9474585,91.28963186)
\lineto(434.7074585,91.34963186)
\lineto(434.6174585,91.34963186)
\curveto(434.58746574,91.35962582)(434.55746577,91.36462581)(434.5274585,91.36463186)
\curveto(434.45746587,91.38462579)(434.36246597,91.38962579)(434.2424585,91.37963186)
\curveto(434.11246622,91.3796258)(434.01246632,91.36962581)(433.9424585,91.34963186)
\curveto(433.86246647,91.32962585)(433.78746654,91.30962587)(433.7174585,91.28963186)
\curveto(433.63746669,91.2796259)(433.55746677,91.25962592)(433.4774585,91.22963186)
\curveto(433.23746709,91.11962606)(433.03746729,90.96962621)(432.8774585,90.77963186)
\curveto(432.70746762,90.59962658)(432.56746776,90.3796268)(432.4574585,90.11963186)
\curveto(432.43746789,90.04962713)(432.42246791,89.9796272)(432.4124585,89.90963186)
\curveto(432.39246794,89.83962734)(432.37246796,89.76462741)(432.3524585,89.68463186)
\curveto(432.332468,89.60462757)(432.32246801,89.49462768)(432.3224585,89.35463186)
\curveto(432.32246801,89.22462795)(432.332468,89.11962806)(432.3524585,89.03963186)
\curveto(432.36246797,88.9796282)(432.36746796,88.92462825)(432.3674585,88.87463186)
\curveto(432.36746796,88.82462835)(432.37746795,88.7746284)(432.3974585,88.72463186)
\curveto(432.43746789,88.62462855)(432.47746785,88.52962865)(432.5174585,88.43963186)
\curveto(432.55746777,88.35962882)(432.60246773,88.2796289)(432.6524585,88.19963186)
\curveto(432.67246766,88.16962901)(432.69746763,88.13962904)(432.7274585,88.10963186)
\curveto(432.75746757,88.08962909)(432.78246755,88.06462911)(432.8024585,88.03463186)
\lineto(432.8774585,87.95963186)
\curveto(432.89746743,87.92962925)(432.91746741,87.90462927)(432.9374585,87.88463186)
\lineto(433.1474585,87.73463186)
\curveto(433.20746712,87.69462948)(433.27246706,87.64962953)(433.3424585,87.59963186)
\curveto(433.4324669,87.53962964)(433.53746679,87.48962969)(433.6574585,87.44963186)
\curveto(433.76746656,87.41962976)(433.87746645,87.38462979)(433.9874585,87.34463186)
\curveto(434.09746623,87.30462987)(434.24246609,87.2796299)(434.4224585,87.26963186)
\curveto(434.59246574,87.25962992)(434.71746561,87.22962995)(434.7974585,87.17963186)
\curveto(434.87746545,87.12963005)(434.92246541,87.05463012)(434.9324585,86.95463186)
\curveto(434.94246539,86.85463032)(434.94746538,86.74463043)(434.9474585,86.62463186)
\curveto(434.94746538,86.58463059)(434.95246538,86.54463063)(434.9624585,86.50463186)
\curveto(434.96246537,86.46463071)(434.95746537,86.42963075)(434.9474585,86.39963186)
\curveto(434.9274654,86.34963083)(434.91746541,86.29963088)(434.9174585,86.24963186)
\curveto(434.91746541,86.20963097)(434.90746542,86.16963101)(434.8874585,86.12963186)
\curveto(434.8274655,86.03963114)(434.69246564,85.99463118)(434.4824585,85.99463186)
\lineto(434.3624585,85.99463186)
\curveto(434.30246603,86.00463117)(434.24246609,86.00963117)(434.1824585,86.00963186)
\curveto(434.11246622,86.01963116)(434.04746628,86.02963115)(433.9874585,86.03963186)
\curveto(433.87746645,86.05963112)(433.77746655,86.0796311)(433.6874585,86.09963186)
\curveto(433.58746674,86.11963106)(433.49246684,86.14963103)(433.4024585,86.18963186)
\curveto(433.332467,86.20963097)(433.27246706,86.22963095)(433.2224585,86.24963186)
\lineto(433.0424585,86.30963186)
\curveto(432.78246755,86.42963075)(432.53746779,86.58463059)(432.3074585,86.77463186)
\curveto(432.07746825,86.9746302)(431.89246844,87.18962999)(431.7524585,87.41963186)
\curveto(431.67246866,87.52962965)(431.60746872,87.64462953)(431.5574585,87.76463186)
\lineto(431.4074585,88.15463186)
\curveto(431.35746897,88.26462891)(431.327469,88.3796288)(431.3174585,88.49963186)
\curveto(431.29746903,88.61962856)(431.27246906,88.74462843)(431.2424585,88.87463186)
\curveto(431.24246909,88.94462823)(431.24246909,89.00962817)(431.2424585,89.06963186)
\curveto(431.2324691,89.12962805)(431.22246911,89.19462798)(431.2124585,89.26463186)
}
}
{
\newrgbcolor{curcolor}{0 0 0}
\pscustom[linestyle=none,fillstyle=solid,fillcolor=curcolor]
{
\newpath
\moveto(436.7324585,101.36424123)
\lineto(436.9874585,101.36424123)
\curveto(437.06746326,101.37423353)(437.14246319,101.36923353)(437.2124585,101.34924123)
\lineto(437.4524585,101.34924123)
\lineto(437.6174585,101.34924123)
\curveto(437.71746261,101.32923357)(437.82246251,101.31923358)(437.9324585,101.31924123)
\curveto(438.0324623,101.31923358)(438.1324622,101.30923359)(438.2324585,101.28924123)
\lineto(438.3824585,101.28924123)
\curveto(438.52246181,101.25923364)(438.66246167,101.23923366)(438.8024585,101.22924123)
\curveto(438.9324614,101.21923368)(439.06246127,101.19423371)(439.1924585,101.15424123)
\curveto(439.27246106,101.13423377)(439.35746097,101.11423379)(439.4474585,101.09424123)
\lineto(439.6874585,101.03424123)
\lineto(439.9874585,100.91424123)
\curveto(440.07746025,100.88423402)(440.16746016,100.84923405)(440.2574585,100.80924123)
\curveto(440.47745985,100.70923419)(440.69245964,100.57423433)(440.9024585,100.40424123)
\curveto(441.11245922,100.24423466)(441.28245905,100.06923483)(441.4124585,99.87924123)
\curveto(441.45245888,99.82923507)(441.49245884,99.76923513)(441.5324585,99.69924123)
\curveto(441.56245877,99.63923526)(441.59745873,99.57923532)(441.6374585,99.51924123)
\curveto(441.68745864,99.43923546)(441.7274586,99.34423556)(441.7574585,99.23424123)
\curveto(441.78745854,99.12423578)(441.81745851,99.01923588)(441.8474585,98.91924123)
\curveto(441.88745844,98.80923609)(441.91245842,98.6992362)(441.9224585,98.58924123)
\curveto(441.9324584,98.47923642)(441.94745838,98.36423654)(441.9674585,98.24424123)
\curveto(441.97745835,98.2042367)(441.97745835,98.15923674)(441.9674585,98.10924123)
\curveto(441.96745836,98.06923683)(441.97245836,98.02923687)(441.9824585,97.98924123)
\curveto(441.99245834,97.94923695)(441.99745833,97.89423701)(441.9974585,97.82424123)
\curveto(441.99745833,97.75423715)(441.99245834,97.7042372)(441.9824585,97.67424123)
\curveto(441.96245837,97.62423728)(441.95745837,97.57923732)(441.9674585,97.53924123)
\curveto(441.97745835,97.4992374)(441.97745835,97.46423744)(441.9674585,97.43424123)
\lineto(441.9674585,97.34424123)
\curveto(441.94745838,97.28423762)(441.9324584,97.21923768)(441.9224585,97.14924123)
\curveto(441.92245841,97.08923781)(441.91745841,97.02423788)(441.9074585,96.95424123)
\curveto(441.85745847,96.78423812)(441.80745852,96.62423828)(441.7574585,96.47424123)
\curveto(441.70745862,96.32423858)(441.64245869,96.17923872)(441.5624585,96.03924123)
\curveto(441.52245881,95.98923891)(441.49245884,95.93423897)(441.4724585,95.87424123)
\curveto(441.44245889,95.82423908)(441.40745892,95.77423913)(441.3674585,95.72424123)
\curveto(441.18745914,95.48423942)(440.96745936,95.28423962)(440.7074585,95.12424123)
\curveto(440.44745988,94.96423994)(440.16246017,94.82424008)(439.8524585,94.70424123)
\curveto(439.71246062,94.64424026)(439.57246076,94.5992403)(439.4324585,94.56924123)
\curveto(439.28246105,94.53924036)(439.1274612,94.5042404)(438.9674585,94.46424123)
\curveto(438.85746147,94.44424046)(438.74746158,94.42924047)(438.6374585,94.41924123)
\curveto(438.5274618,94.40924049)(438.41746191,94.39424051)(438.3074585,94.37424123)
\curveto(438.26746206,94.36424054)(438.2274621,94.35924054)(438.1874585,94.35924123)
\curveto(438.14746218,94.36924053)(438.10746222,94.36924053)(438.0674585,94.35924123)
\curveto(438.01746231,94.34924055)(437.96746236,94.34424056)(437.9174585,94.34424123)
\lineto(437.7524585,94.34424123)
\curveto(437.70246263,94.32424058)(437.65246268,94.31924058)(437.6024585,94.32924123)
\curveto(437.54246279,94.33924056)(437.48746284,94.33924056)(437.4374585,94.32924123)
\curveto(437.39746293,94.31924058)(437.35246298,94.31924058)(437.3024585,94.32924123)
\curveto(437.25246308,94.33924056)(437.20246313,94.33424057)(437.1524585,94.31424123)
\curveto(437.08246325,94.29424061)(437.00746332,94.28924061)(436.9274585,94.29924123)
\curveto(436.83746349,94.30924059)(436.75246358,94.31424059)(436.6724585,94.31424123)
\curveto(436.58246375,94.31424059)(436.48246385,94.30924059)(436.3724585,94.29924123)
\curveto(436.25246408,94.28924061)(436.15246418,94.29424061)(436.0724585,94.31424123)
\lineto(435.7874585,94.31424123)
\lineto(435.1574585,94.35924123)
\curveto(435.05746527,94.36924053)(434.96246537,94.37924052)(434.8724585,94.38924123)
\lineto(434.5724585,94.41924123)
\curveto(434.52246581,94.43924046)(434.47246586,94.44424046)(434.4224585,94.43424123)
\curveto(434.36246597,94.43424047)(434.30746602,94.44424046)(434.2574585,94.46424123)
\curveto(434.08746624,94.51424039)(433.92246641,94.55424035)(433.7624585,94.58424123)
\curveto(433.59246674,94.61424029)(433.4324669,94.66424024)(433.2824585,94.73424123)
\curveto(432.82246751,94.92423998)(432.44746788,95.14423976)(432.1574585,95.39424123)
\curveto(431.86746846,95.65423925)(431.62246871,96.01423889)(431.4224585,96.47424123)
\curveto(431.37246896,96.6042383)(431.33746899,96.73423817)(431.3174585,96.86424123)
\curveto(431.29746903,97.0042379)(431.27246906,97.14423776)(431.2424585,97.28424123)
\curveto(431.2324691,97.35423755)(431.2274691,97.41923748)(431.2274585,97.47924123)
\curveto(431.2274691,97.53923736)(431.22246911,97.6042373)(431.2124585,97.67424123)
\curveto(431.19246914,98.5042364)(431.34246899,99.17423573)(431.6624585,99.68424123)
\curveto(431.97246836,100.19423471)(432.41246792,100.57423433)(432.9824585,100.82424123)
\curveto(433.10246723,100.87423403)(433.2274671,100.91923398)(433.3574585,100.95924123)
\curveto(433.48746684,100.9992339)(433.62246671,101.04423386)(433.7624585,101.09424123)
\curveto(433.84246649,101.11423379)(433.9274664,101.12923377)(434.0174585,101.13924123)
\lineto(434.2574585,101.19924123)
\curveto(434.36746596,101.22923367)(434.47746585,101.24423366)(434.5874585,101.24424123)
\curveto(434.69746563,101.25423365)(434.80746552,101.26923363)(434.9174585,101.28924123)
\curveto(434.96746536,101.30923359)(435.01246532,101.31423359)(435.0524585,101.30424123)
\curveto(435.09246524,101.3042336)(435.1324652,101.30923359)(435.1724585,101.31924123)
\curveto(435.22246511,101.32923357)(435.27746505,101.32923357)(435.3374585,101.31924123)
\curveto(435.38746494,101.31923358)(435.43746489,101.32423358)(435.4874585,101.33424123)
\lineto(435.6224585,101.33424123)
\curveto(435.68246465,101.35423355)(435.75246458,101.35423355)(435.8324585,101.33424123)
\curveto(435.90246443,101.32423358)(435.96746436,101.32923357)(436.0274585,101.34924123)
\curveto(436.05746427,101.35923354)(436.09746423,101.36423354)(436.1474585,101.36424123)
\lineto(436.2674585,101.36424123)
\lineto(436.7324585,101.36424123)
\moveto(439.0574585,99.81924123)
\curveto(438.73746159,99.91923498)(438.37246196,99.97923492)(437.9624585,99.99924123)
\curveto(437.55246278,100.01923488)(437.14246319,100.02923487)(436.7324585,100.02924123)
\curveto(436.30246403,100.02923487)(435.88246445,100.01923488)(435.4724585,99.99924123)
\curveto(435.06246527,99.97923492)(434.67746565,99.93423497)(434.3174585,99.86424123)
\curveto(433.95746637,99.79423511)(433.63746669,99.68423522)(433.3574585,99.53424123)
\curveto(433.06746726,99.39423551)(432.8324675,99.1992357)(432.6524585,98.94924123)
\curveto(432.54246779,98.78923611)(432.46246787,98.60923629)(432.4124585,98.40924123)
\curveto(432.35246798,98.20923669)(432.32246801,97.96423694)(432.3224585,97.67424123)
\curveto(432.34246799,97.65423725)(432.35246798,97.61923728)(432.3524585,97.56924123)
\curveto(432.34246799,97.51923738)(432.34246799,97.47923742)(432.3524585,97.44924123)
\curveto(432.37246796,97.36923753)(432.39246794,97.29423761)(432.4124585,97.22424123)
\curveto(432.42246791,97.16423774)(432.44246789,97.0992378)(432.4724585,97.02924123)
\curveto(432.59246774,96.75923814)(432.76246757,96.53923836)(432.9824585,96.36924123)
\curveto(433.19246714,96.20923869)(433.43746689,96.07423883)(433.7174585,95.96424123)
\curveto(433.8274665,95.91423899)(433.94746638,95.87423903)(434.0774585,95.84424123)
\curveto(434.19746613,95.82423908)(434.32246601,95.7992391)(434.4524585,95.76924123)
\curveto(434.50246583,95.74923915)(434.55746577,95.73923916)(434.6174585,95.73924123)
\curveto(434.66746566,95.73923916)(434.71746561,95.73423917)(434.7674585,95.72424123)
\curveto(434.85746547,95.71423919)(434.95246538,95.7042392)(435.0524585,95.69424123)
\curveto(435.14246519,95.68423922)(435.23746509,95.67423923)(435.3374585,95.66424123)
\curveto(435.41746491,95.66423924)(435.50246483,95.65923924)(435.5924585,95.64924123)
\lineto(435.8324585,95.64924123)
\lineto(436.0124585,95.64924123)
\curveto(436.04246429,95.63923926)(436.07746425,95.63423927)(436.1174585,95.63424123)
\lineto(436.2524585,95.63424123)
\lineto(436.7024585,95.63424123)
\curveto(436.78246355,95.63423927)(436.86746346,95.62923927)(436.9574585,95.61924123)
\curveto(437.03746329,95.61923928)(437.11246322,95.62923927)(437.1824585,95.64924123)
\lineto(437.4524585,95.64924123)
\curveto(437.47246286,95.64923925)(437.50246283,95.64423926)(437.5424585,95.63424123)
\curveto(437.57246276,95.63423927)(437.59746273,95.63923926)(437.6174585,95.64924123)
\curveto(437.71746261,95.65923924)(437.81746251,95.66423924)(437.9174585,95.66424123)
\curveto(438.00746232,95.67423923)(438.10746222,95.68423922)(438.2174585,95.69424123)
\curveto(438.33746199,95.72423918)(438.46246187,95.73923916)(438.5924585,95.73924123)
\curveto(438.71246162,95.74923915)(438.8274615,95.77423913)(438.9374585,95.81424123)
\curveto(439.23746109,95.89423901)(439.50246083,95.97923892)(439.7324585,96.06924123)
\curveto(439.96246037,96.16923873)(440.17746015,96.31423859)(440.3774585,96.50424123)
\curveto(440.57745975,96.71423819)(440.7274596,96.97923792)(440.8274585,97.29924123)
\curveto(440.84745948,97.33923756)(440.85745947,97.37423753)(440.8574585,97.40424123)
\curveto(440.84745948,97.44423746)(440.85245948,97.48923741)(440.8724585,97.53924123)
\curveto(440.88245945,97.57923732)(440.89245944,97.64923725)(440.9024585,97.74924123)
\curveto(440.91245942,97.85923704)(440.90745942,97.94423696)(440.8874585,98.00424123)
\curveto(440.86745946,98.07423683)(440.85745947,98.14423676)(440.8574585,98.21424123)
\curveto(440.84745948,98.28423662)(440.8324595,98.34923655)(440.8124585,98.40924123)
\curveto(440.75245958,98.60923629)(440.66745966,98.78923611)(440.5574585,98.94924123)
\curveto(440.53745979,98.97923592)(440.51745981,99.0042359)(440.4974585,99.02424123)
\lineto(440.4374585,99.08424123)
\curveto(440.41745991,99.12423578)(440.37745995,99.17423573)(440.3174585,99.23424123)
\curveto(440.17746015,99.33423557)(440.04746028,99.41923548)(439.9274585,99.48924123)
\curveto(439.80746052,99.55923534)(439.66246067,99.62923527)(439.4924585,99.69924123)
\curveto(439.42246091,99.72923517)(439.35246098,99.74923515)(439.2824585,99.75924123)
\curveto(439.21246112,99.77923512)(439.13746119,99.7992351)(439.0574585,99.81924123)
}
}
{
\newrgbcolor{curcolor}{0 0 0}
\pscustom[linestyle=none,fillstyle=solid,fillcolor=curcolor]
{
\newpath
\moveto(431.2124585,106.77385061)
\curveto(431.21246912,106.87384575)(431.22246911,106.96884566)(431.2424585,107.05885061)
\curveto(431.25246908,107.14884548)(431.28246905,107.21384541)(431.3324585,107.25385061)
\curveto(431.41246892,107.31384531)(431.51746881,107.34384528)(431.6474585,107.34385061)
\lineto(432.0374585,107.34385061)
\lineto(433.5374585,107.34385061)
\lineto(439.9274585,107.34385061)
\lineto(441.0974585,107.34385061)
\lineto(441.4124585,107.34385061)
\curveto(441.51245882,107.35384527)(441.59245874,107.33884529)(441.6524585,107.29885061)
\curveto(441.7324586,107.24884538)(441.78245855,107.17384545)(441.8024585,107.07385061)
\curveto(441.81245852,106.98384564)(441.81745851,106.87384575)(441.8174585,106.74385061)
\lineto(441.8174585,106.51885061)
\curveto(441.79745853,106.43884619)(441.78245855,106.36884626)(441.7724585,106.30885061)
\curveto(441.75245858,106.24884638)(441.71245862,106.19884643)(441.6524585,106.15885061)
\curveto(441.59245874,106.11884651)(441.51745881,106.09884653)(441.4274585,106.09885061)
\lineto(441.1274585,106.09885061)
\lineto(440.0324585,106.09885061)
\lineto(434.6924585,106.09885061)
\curveto(434.60246573,106.07884655)(434.5274658,106.06384656)(434.4674585,106.05385061)
\curveto(434.39746593,106.05384657)(434.33746599,106.0238466)(434.2874585,105.96385061)
\curveto(434.23746609,105.89384673)(434.21246612,105.80384682)(434.2124585,105.69385061)
\curveto(434.20246613,105.59384703)(434.19746613,105.48384714)(434.1974585,105.36385061)
\lineto(434.1974585,104.22385061)
\lineto(434.1974585,103.72885061)
\curveto(434.18746614,103.56884906)(434.1274662,103.45884917)(434.0174585,103.39885061)
\curveto(433.98746634,103.37884925)(433.95746637,103.36884926)(433.9274585,103.36885061)
\curveto(433.88746644,103.36884926)(433.84246649,103.36384926)(433.7924585,103.35385061)
\curveto(433.67246666,103.33384929)(433.56246677,103.33884929)(433.4624585,103.36885061)
\curveto(433.36246697,103.40884922)(433.29246704,103.46384916)(433.2524585,103.53385061)
\curveto(433.20246713,103.61384901)(433.17746715,103.73384889)(433.1774585,103.89385061)
\curveto(433.17746715,104.05384857)(433.16246717,104.18884844)(433.1324585,104.29885061)
\curveto(433.12246721,104.34884828)(433.11746721,104.40384822)(433.1174585,104.46385061)
\curveto(433.10746722,104.5238481)(433.09246724,104.58384804)(433.0724585,104.64385061)
\curveto(433.02246731,104.79384783)(432.97246736,104.93884769)(432.9224585,105.07885061)
\curveto(432.86246747,105.21884741)(432.79246754,105.35384727)(432.7124585,105.48385061)
\curveto(432.62246771,105.623847)(432.51746781,105.74384688)(432.3974585,105.84385061)
\curveto(432.27746805,105.94384668)(432.14746818,106.03884659)(432.0074585,106.12885061)
\curveto(431.90746842,106.18884644)(431.79746853,106.23384639)(431.6774585,106.26385061)
\curveto(431.55746877,106.30384632)(431.45246888,106.35384627)(431.3624585,106.41385061)
\curveto(431.30246903,106.46384616)(431.26246907,106.53384609)(431.2424585,106.62385061)
\curveto(431.2324691,106.64384598)(431.2274691,106.66884596)(431.2274585,106.69885061)
\curveto(431.2274691,106.7288459)(431.22246911,106.75384587)(431.2124585,106.77385061)
}
}
{
\newrgbcolor{curcolor}{0 0 0}
\pscustom[linestyle=none,fillstyle=solid,fillcolor=curcolor]
{
\newpath
\moveto(431.2124585,115.12345998)
\curveto(431.21246912,115.22345513)(431.22246911,115.31845503)(431.2424585,115.40845998)
\curveto(431.25246908,115.49845485)(431.28246905,115.56345479)(431.3324585,115.60345998)
\curveto(431.41246892,115.66345469)(431.51746881,115.69345466)(431.6474585,115.69345998)
\lineto(432.0374585,115.69345998)
\lineto(433.5374585,115.69345998)
\lineto(439.9274585,115.69345998)
\lineto(441.0974585,115.69345998)
\lineto(441.4124585,115.69345998)
\curveto(441.51245882,115.70345465)(441.59245874,115.68845466)(441.6524585,115.64845998)
\curveto(441.7324586,115.59845475)(441.78245855,115.52345483)(441.8024585,115.42345998)
\curveto(441.81245852,115.33345502)(441.81745851,115.22345513)(441.8174585,115.09345998)
\lineto(441.8174585,114.86845998)
\curveto(441.79745853,114.78845556)(441.78245855,114.71845563)(441.7724585,114.65845998)
\curveto(441.75245858,114.59845575)(441.71245862,114.5484558)(441.6524585,114.50845998)
\curveto(441.59245874,114.46845588)(441.51745881,114.4484559)(441.4274585,114.44845998)
\lineto(441.1274585,114.44845998)
\lineto(440.0324585,114.44845998)
\lineto(434.6924585,114.44845998)
\curveto(434.60246573,114.42845592)(434.5274658,114.41345594)(434.4674585,114.40345998)
\curveto(434.39746593,114.40345595)(434.33746599,114.37345598)(434.2874585,114.31345998)
\curveto(434.23746609,114.24345611)(434.21246612,114.1534562)(434.2124585,114.04345998)
\curveto(434.20246613,113.94345641)(434.19746613,113.83345652)(434.1974585,113.71345998)
\lineto(434.1974585,112.57345998)
\lineto(434.1974585,112.07845998)
\curveto(434.18746614,111.91845843)(434.1274662,111.80845854)(434.0174585,111.74845998)
\curveto(433.98746634,111.72845862)(433.95746637,111.71845863)(433.9274585,111.71845998)
\curveto(433.88746644,111.71845863)(433.84246649,111.71345864)(433.7924585,111.70345998)
\curveto(433.67246666,111.68345867)(433.56246677,111.68845866)(433.4624585,111.71845998)
\curveto(433.36246697,111.75845859)(433.29246704,111.81345854)(433.2524585,111.88345998)
\curveto(433.20246713,111.96345839)(433.17746715,112.08345827)(433.1774585,112.24345998)
\curveto(433.17746715,112.40345795)(433.16246717,112.53845781)(433.1324585,112.64845998)
\curveto(433.12246721,112.69845765)(433.11746721,112.7534576)(433.1174585,112.81345998)
\curveto(433.10746722,112.87345748)(433.09246724,112.93345742)(433.0724585,112.99345998)
\curveto(433.02246731,113.14345721)(432.97246736,113.28845706)(432.9224585,113.42845998)
\curveto(432.86246747,113.56845678)(432.79246754,113.70345665)(432.7124585,113.83345998)
\curveto(432.62246771,113.97345638)(432.51746781,114.09345626)(432.3974585,114.19345998)
\curveto(432.27746805,114.29345606)(432.14746818,114.38845596)(432.0074585,114.47845998)
\curveto(431.90746842,114.53845581)(431.79746853,114.58345577)(431.6774585,114.61345998)
\curveto(431.55746877,114.6534557)(431.45246888,114.70345565)(431.3624585,114.76345998)
\curveto(431.30246903,114.81345554)(431.26246907,114.88345547)(431.2424585,114.97345998)
\curveto(431.2324691,114.99345536)(431.2274691,115.01845533)(431.2274585,115.04845998)
\curveto(431.2274691,115.07845527)(431.22246911,115.10345525)(431.2124585,115.12345998)
}
}
{
\newrgbcolor{curcolor}{0 0 0}
\pscustom[linestyle=none,fillstyle=solid,fillcolor=curcolor]
{
\newpath
\moveto(461.94880493,42.02236623)
\curveto(461.99880568,42.04235669)(462.05880562,42.06735666)(462.12880493,42.09736623)
\curveto(462.19880548,42.1273566)(462.2738054,42.14735658)(462.35380493,42.15736623)
\curveto(462.42380525,42.17735655)(462.49380518,42.17735655)(462.56380493,42.15736623)
\curveto(462.62380505,42.14735658)(462.66880501,42.10735662)(462.69880493,42.03736623)
\curveto(462.71880496,41.98735674)(462.72880495,41.9273568)(462.72880493,41.85736623)
\lineto(462.72880493,41.64736623)
\lineto(462.72880493,41.19736623)
\curveto(462.72880495,41.04735768)(462.70380497,40.9273578)(462.65380493,40.83736623)
\curveto(462.59380508,40.73735799)(462.48880519,40.66235807)(462.33880493,40.61236623)
\curveto(462.18880549,40.57235816)(462.05380562,40.5273582)(461.93380493,40.47736623)
\curveto(461.673806,40.36735836)(461.40380627,40.26735846)(461.12380493,40.17736623)
\curveto(460.84380683,40.08735864)(460.56880711,39.98735874)(460.29880493,39.87736623)
\curveto(460.20880747,39.84735888)(460.12380755,39.81735891)(460.04380493,39.78736623)
\curveto(459.96380771,39.76735896)(459.88880779,39.73735899)(459.81880493,39.69736623)
\curveto(459.74880793,39.66735906)(459.68880799,39.62235911)(459.63880493,39.56236623)
\curveto(459.58880809,39.50235923)(459.54880813,39.42235931)(459.51880493,39.32236623)
\curveto(459.49880818,39.27235946)(459.49380818,39.21235952)(459.50380493,39.14236623)
\lineto(459.50380493,38.94736623)
\lineto(459.50380493,36.11236623)
\lineto(459.50380493,35.81236623)
\curveto(459.49380818,35.70236303)(459.49380818,35.59736313)(459.50380493,35.49736623)
\curveto(459.51380816,35.39736333)(459.52880815,35.30236343)(459.54880493,35.21236623)
\curveto(459.56880811,35.1323636)(459.60880807,35.07236366)(459.66880493,35.03236623)
\curveto(459.76880791,34.95236378)(459.88380779,34.89236384)(460.01380493,34.85236623)
\curveto(460.13380754,34.82236391)(460.25880742,34.78236395)(460.38880493,34.73236623)
\curveto(460.61880706,34.6323641)(460.85880682,34.53736419)(461.10880493,34.44736623)
\curveto(461.35880632,34.36736436)(461.59880608,34.27736445)(461.82880493,34.17736623)
\curveto(461.88880579,34.15736457)(461.95880572,34.1323646)(462.03880493,34.10236623)
\curveto(462.10880557,34.08236465)(462.18380549,34.05736467)(462.26380493,34.02736623)
\curveto(462.34380533,33.99736473)(462.41880526,33.96236477)(462.48880493,33.92236623)
\curveto(462.54880513,33.89236484)(462.59380508,33.85736487)(462.62380493,33.81736623)
\curveto(462.68380499,33.73736499)(462.71880496,33.6273651)(462.72880493,33.48736623)
\lineto(462.72880493,33.06736623)
\lineto(462.72880493,32.82736623)
\curveto(462.71880496,32.75736597)(462.69380498,32.69736603)(462.65380493,32.64736623)
\curveto(462.62380505,32.59736613)(462.5788051,32.56736616)(462.51880493,32.55736623)
\curveto(462.45880522,32.55736617)(462.39880528,32.56236617)(462.33880493,32.57236623)
\curveto(462.26880541,32.59236614)(462.20380547,32.61236612)(462.14380493,32.63236623)
\curveto(462.0738056,32.66236607)(462.02380565,32.68736604)(461.99380493,32.70736623)
\curveto(461.673806,32.84736588)(461.35880632,32.97236576)(461.04880493,33.08236623)
\curveto(460.72880695,33.19236554)(460.40880727,33.31236542)(460.08880493,33.44236623)
\curveto(459.86880781,33.5323652)(459.65380802,33.61736511)(459.44380493,33.69736623)
\curveto(459.22380845,33.77736495)(459.00380867,33.86236487)(458.78380493,33.95236623)
\curveto(458.06380961,34.25236448)(457.33881034,34.53736419)(456.60880493,34.80736623)
\curveto(455.86881181,35.07736365)(455.13381254,35.36236337)(454.40380493,35.66236623)
\curveto(454.14381353,35.77236296)(453.8788138,35.87236286)(453.60880493,35.96236623)
\curveto(453.33881434,36.06236267)(453.0738146,36.16736256)(452.81380493,36.27736623)
\curveto(452.70381497,36.3273624)(452.58381509,36.37236236)(452.45380493,36.41236623)
\curveto(452.31381536,36.46236227)(452.21381546,36.5323622)(452.15380493,36.62236623)
\curveto(452.11381556,36.66236207)(452.08381559,36.727362)(452.06380493,36.81736623)
\curveto(452.05381562,36.83736189)(452.05381562,36.85736187)(452.06380493,36.87736623)
\curveto(452.06381561,36.90736182)(452.05881562,36.9323618)(452.04880493,36.95236623)
\curveto(452.04881563,37.1323616)(452.04881563,37.34236139)(452.04880493,37.58236623)
\curveto(452.03881564,37.82236091)(452.0738156,37.99736073)(452.15380493,38.10736623)
\curveto(452.21381546,38.18736054)(452.31381536,38.24736048)(452.45380493,38.28736623)
\curveto(452.58381509,38.33736039)(452.70381497,38.38736034)(452.81380493,38.43736623)
\curveto(453.04381463,38.53736019)(453.2738144,38.6273601)(453.50380493,38.70736623)
\curveto(453.73381394,38.78735994)(453.96381371,38.87735985)(454.19380493,38.97736623)
\curveto(454.39381328,39.05735967)(454.59881308,39.1323596)(454.80880493,39.20236623)
\curveto(455.01881266,39.28235945)(455.22381245,39.36735936)(455.42380493,39.45736623)
\curveto(456.15381152,39.75735897)(456.89381078,40.04235869)(457.64380493,40.31236623)
\curveto(458.38380929,40.59235814)(459.11880856,40.88735784)(459.84880493,41.19736623)
\curveto(459.93880774,41.23735749)(460.02380765,41.26735746)(460.10380493,41.28736623)
\curveto(460.18380749,41.31735741)(460.26880741,41.34735738)(460.35880493,41.37736623)
\curveto(460.61880706,41.48735724)(460.88380679,41.59235714)(461.15380493,41.69236623)
\curveto(461.42380625,41.80235693)(461.68880599,41.91235682)(461.94880493,42.02236623)
\moveto(458.30380493,38.81236623)
\curveto(458.2738094,38.90235983)(458.22380945,38.95735977)(458.15380493,38.97736623)
\curveto(458.08380959,39.00735972)(458.00880967,39.01235972)(457.92880493,38.99236623)
\curveto(457.83880984,38.98235975)(457.75380992,38.95735977)(457.67380493,38.91736623)
\curveto(457.58381009,38.88735984)(457.50881017,38.85735987)(457.44880493,38.82736623)
\curveto(457.40881027,38.80735992)(457.3738103,38.79735993)(457.34380493,38.79736623)
\curveto(457.31381036,38.79735993)(457.2788104,38.78735994)(457.23880493,38.76736623)
\lineto(456.99880493,38.67736623)
\curveto(456.90881077,38.65736007)(456.81881086,38.6273601)(456.72880493,38.58736623)
\curveto(456.36881131,38.43736029)(456.00381167,38.30236043)(455.63380493,38.18236623)
\curveto(455.25381242,38.07236066)(454.88381279,37.94236079)(454.52380493,37.79236623)
\curveto(454.41381326,37.74236099)(454.30381337,37.69736103)(454.19380493,37.65736623)
\curveto(454.08381359,37.6273611)(453.9788137,37.58736114)(453.87880493,37.53736623)
\curveto(453.82881385,37.51736121)(453.78381389,37.49236124)(453.74380493,37.46236623)
\curveto(453.69381398,37.44236129)(453.66881401,37.39236134)(453.66880493,37.31236623)
\curveto(453.68881399,37.29236144)(453.70381397,37.27236146)(453.71380493,37.25236623)
\curveto(453.72381395,37.2323615)(453.73881394,37.21236152)(453.75880493,37.19236623)
\curveto(453.80881387,37.15236158)(453.86381381,37.12236161)(453.92380493,37.10236623)
\curveto(453.9738137,37.08236165)(454.02881365,37.06236167)(454.08880493,37.04236623)
\curveto(454.19881348,36.99236174)(454.30881337,36.95236178)(454.41880493,36.92236623)
\curveto(454.52881315,36.89236184)(454.63881304,36.85236188)(454.74880493,36.80236623)
\curveto(455.13881254,36.6323621)(455.53381214,36.48236225)(455.93380493,36.35236623)
\curveto(456.33381134,36.2323625)(456.72381095,36.09236264)(457.10380493,35.93236623)
\lineto(457.25380493,35.87236623)
\curveto(457.30381037,35.86236287)(457.35381032,35.84736288)(457.40380493,35.82736623)
\lineto(457.64380493,35.73736623)
\curveto(457.72380995,35.70736302)(457.80380987,35.68236305)(457.88380493,35.66236623)
\curveto(457.93380974,35.64236309)(457.98880969,35.6323631)(458.04880493,35.63236623)
\curveto(458.10880957,35.64236309)(458.15880952,35.65736307)(458.19880493,35.67736623)
\curveto(458.2788094,35.727363)(458.32380935,35.8323629)(458.33380493,35.99236623)
\lineto(458.33380493,36.44236623)
\lineto(458.33380493,38.04736623)
\curveto(458.33380934,38.15736057)(458.33880934,38.29236044)(458.34880493,38.45236623)
\curveto(458.34880933,38.61236012)(458.33380934,38.73236)(458.30380493,38.81236623)
}
}
{
\newrgbcolor{curcolor}{0 0 0}
\pscustom[linestyle=none,fillstyle=solid,fillcolor=curcolor]
{
\newpath
\moveto(458.69380493,50.56392873)
\curveto(458.74380893,50.57392038)(458.81380886,50.57892038)(458.90380493,50.57892873)
\curveto(458.98380869,50.57892038)(459.04880863,50.57392038)(459.09880493,50.56392873)
\curveto(459.13880854,50.56392039)(459.1788085,50.5589204)(459.21880493,50.54892873)
\lineto(459.33880493,50.54892873)
\curveto(459.41880826,50.52892043)(459.49880818,50.51892044)(459.57880493,50.51892873)
\curveto(459.65880802,50.51892044)(459.73880794,50.50892045)(459.81880493,50.48892873)
\curveto(459.85880782,50.47892048)(459.89880778,50.47392048)(459.93880493,50.47392873)
\curveto(459.96880771,50.47392048)(460.00380767,50.46892049)(460.04380493,50.45892873)
\curveto(460.15380752,50.42892053)(460.25880742,50.39892056)(460.35880493,50.36892873)
\curveto(460.45880722,50.34892061)(460.55880712,50.31892064)(460.65880493,50.27892873)
\curveto(461.00880667,50.13892082)(461.32380635,49.96892099)(461.60380493,49.76892873)
\curveto(461.88380579,49.56892139)(462.12380555,49.31892164)(462.32380493,49.01892873)
\curveto(462.42380525,48.86892209)(462.50880517,48.72392223)(462.57880493,48.58392873)
\curveto(462.62880505,48.47392248)(462.66880501,48.36392259)(462.69880493,48.25392873)
\curveto(462.72880495,48.1539228)(462.75880492,48.04892291)(462.78880493,47.93892873)
\curveto(462.80880487,47.86892309)(462.81880486,47.80392315)(462.81880493,47.74392873)
\curveto(462.82880485,47.68392327)(462.84380483,47.62392333)(462.86380493,47.56392873)
\lineto(462.86380493,47.41392873)
\curveto(462.88380479,47.36392359)(462.89380478,47.28892367)(462.89380493,47.18892873)
\curveto(462.90380477,47.08892387)(462.89880478,47.00892395)(462.87880493,46.94892873)
\lineto(462.87880493,46.79892873)
\curveto(462.86880481,46.7589242)(462.86380481,46.71392424)(462.86380493,46.66392873)
\curveto(462.86380481,46.62392433)(462.85880482,46.57892438)(462.84880493,46.52892873)
\curveto(462.80880487,46.37892458)(462.7738049,46.22892473)(462.74380493,46.07892873)
\curveto(462.71380496,45.93892502)(462.66880501,45.79892516)(462.60880493,45.65892873)
\curveto(462.52880515,45.4589255)(462.42880525,45.27892568)(462.30880493,45.11892873)
\lineto(462.15880493,44.93892873)
\curveto(462.09880558,44.87892608)(462.05880562,44.80892615)(462.03880493,44.72892873)
\curveto(462.02880565,44.66892629)(462.04380563,44.61892634)(462.08380493,44.57892873)
\curveto(462.11380556,44.54892641)(462.15880552,44.52392643)(462.21880493,44.50392873)
\curveto(462.2788054,44.49392646)(462.34380533,44.48392647)(462.41380493,44.47392873)
\curveto(462.4738052,44.47392648)(462.51880516,44.46392649)(462.54880493,44.44392873)
\curveto(462.59880508,44.40392655)(462.64380503,44.3589266)(462.68380493,44.30892873)
\curveto(462.70380497,44.2589267)(462.71880496,44.18892677)(462.72880493,44.09892873)
\lineto(462.72880493,43.82892873)
\curveto(462.72880495,43.73892722)(462.72380495,43.6539273)(462.71380493,43.57392873)
\curveto(462.69380498,43.49392746)(462.673805,43.43392752)(462.65380493,43.39392873)
\curveto(462.63380504,43.37392758)(462.60880507,43.3539276)(462.57880493,43.33392873)
\lineto(462.48880493,43.27392873)
\curveto(462.40880527,43.24392771)(462.28880539,43.22892773)(462.12880493,43.22892873)
\curveto(461.96880571,43.23892772)(461.83380584,43.24392771)(461.72380493,43.24392873)
\lineto(452.91880493,43.24392873)
\curveto(452.79881488,43.24392771)(452.673815,43.23892772)(452.54380493,43.22892873)
\curveto(452.40381527,43.22892773)(452.29381538,43.2539277)(452.21380493,43.30392873)
\curveto(452.15381552,43.34392761)(452.10381557,43.40892755)(452.06380493,43.49892873)
\curveto(452.06381561,43.51892744)(452.06381561,43.54392741)(452.06380493,43.57392873)
\curveto(452.05381562,43.60392735)(452.04881563,43.62892733)(452.04880493,43.64892873)
\curveto(452.03881564,43.78892717)(452.03881564,43.93392702)(452.04880493,44.08392873)
\curveto(452.04881563,44.24392671)(452.08881559,44.3539266)(452.16880493,44.41392873)
\curveto(452.24881543,44.46392649)(452.36381531,44.48892647)(452.51380493,44.48892873)
\lineto(452.91880493,44.48892873)
\lineto(454.67380493,44.48892873)
\lineto(454.92880493,44.48892873)
\lineto(455.21380493,44.48892873)
\curveto(455.30381237,44.49892646)(455.38881229,44.50892645)(455.46880493,44.51892873)
\curveto(455.53881214,44.53892642)(455.58881209,44.56892639)(455.61880493,44.60892873)
\curveto(455.64881203,44.64892631)(455.65381202,44.69392626)(455.63380493,44.74392873)
\curveto(455.61381206,44.79392616)(455.59381208,44.83392612)(455.57380493,44.86392873)
\curveto(455.53381214,44.91392604)(455.49381218,44.958926)(455.45380493,44.99892873)
\lineto(455.33380493,45.14892873)
\curveto(455.28381239,45.21892574)(455.23881244,45.28892567)(455.19880493,45.35892873)
\lineto(455.07880493,45.59892873)
\curveto(454.98881269,45.77892518)(454.92381275,45.99392496)(454.88380493,46.24392873)
\curveto(454.84381283,46.49392446)(454.82381285,46.74892421)(454.82380493,47.00892873)
\curveto(454.82381285,47.26892369)(454.84881283,47.52392343)(454.89880493,47.77392873)
\curveto(454.93881274,48.02392293)(454.99881268,48.24392271)(455.07880493,48.43392873)
\curveto(455.24881243,48.83392212)(455.48381219,49.17892178)(455.78380493,49.46892873)
\curveto(456.08381159,49.7589212)(456.43381124,49.98892097)(456.83380493,50.15892873)
\curveto(456.94381073,50.20892075)(457.05381062,50.24892071)(457.16380493,50.27892873)
\curveto(457.26381041,50.31892064)(457.36881031,50.3589206)(457.47880493,50.39892873)
\curveto(457.58881009,50.42892053)(457.70380997,50.44892051)(457.82380493,50.45892873)
\lineto(458.15380493,50.51892873)
\curveto(458.18380949,50.52892043)(458.21880946,50.53392042)(458.25880493,50.53392873)
\curveto(458.28880939,50.53392042)(458.31880936,50.53892042)(458.34880493,50.54892873)
\curveto(458.40880927,50.56892039)(458.46880921,50.56892039)(458.52880493,50.54892873)
\curveto(458.5788091,50.53892042)(458.63380904,50.54392041)(458.69380493,50.56392873)
\moveto(459.08380493,49.22892873)
\curveto(459.03380864,49.24892171)(458.9738087,49.2539217)(458.90380493,49.24392873)
\curveto(458.83380884,49.23392172)(458.76880891,49.22892173)(458.70880493,49.22892873)
\curveto(458.53880914,49.22892173)(458.3788093,49.21892174)(458.22880493,49.19892873)
\curveto(458.0788096,49.18892177)(457.94380973,49.1589218)(457.82380493,49.10892873)
\curveto(457.72380995,49.07892188)(457.63381004,49.0539219)(457.55380493,49.03392873)
\curveto(457.4738102,49.01392194)(457.39381028,48.98392197)(457.31380493,48.94392873)
\curveto(457.06381061,48.83392212)(456.83381084,48.68392227)(456.62380493,48.49392873)
\curveto(456.40381127,48.30392265)(456.23881144,48.08392287)(456.12880493,47.83392873)
\curveto(456.09881158,47.7539232)(456.0738116,47.67392328)(456.05380493,47.59392873)
\curveto(456.02381165,47.52392343)(455.99881168,47.44892351)(455.97880493,47.36892873)
\curveto(455.94881173,47.2589237)(455.93381174,47.14892381)(455.93380493,47.03892873)
\curveto(455.92381175,46.92892403)(455.91881176,46.80892415)(455.91880493,46.67892873)
\curveto(455.92881175,46.62892433)(455.93881174,46.58392437)(455.94880493,46.54392873)
\lineto(455.94880493,46.40892873)
\lineto(456.00880493,46.13892873)
\curveto(456.02881165,46.0589249)(456.05881162,45.97892498)(456.09880493,45.89892873)
\curveto(456.23881144,45.5589254)(456.44881123,45.28892567)(456.72880493,45.08892873)
\curveto(456.99881068,44.88892607)(457.31881036,44.72892623)(457.68880493,44.60892873)
\curveto(457.79880988,44.56892639)(457.90880977,44.54392641)(458.01880493,44.53392873)
\curveto(458.12880955,44.52392643)(458.24380943,44.50392645)(458.36380493,44.47392873)
\curveto(458.41380926,44.46392649)(458.45880922,44.46392649)(458.49880493,44.47392873)
\curveto(458.53880914,44.48392647)(458.58380909,44.47892648)(458.63380493,44.45892873)
\curveto(458.68380899,44.44892651)(458.75880892,44.44392651)(458.85880493,44.44392873)
\curveto(458.94880873,44.44392651)(459.01880866,44.44892651)(459.06880493,44.45892873)
\lineto(459.18880493,44.45892873)
\curveto(459.22880845,44.46892649)(459.26880841,44.47392648)(459.30880493,44.47392873)
\curveto(459.34880833,44.47392648)(459.38380829,44.47892648)(459.41380493,44.48892873)
\curveto(459.44380823,44.49892646)(459.4788082,44.50392645)(459.51880493,44.50392873)
\curveto(459.54880813,44.50392645)(459.5788081,44.50892645)(459.60880493,44.51892873)
\curveto(459.68880799,44.53892642)(459.76880791,44.5539264)(459.84880493,44.56392873)
\lineto(460.08880493,44.62392873)
\curveto(460.42880725,44.73392622)(460.71880696,44.88392607)(460.95880493,45.07392873)
\curveto(461.19880648,45.27392568)(461.39880628,45.51892544)(461.55880493,45.80892873)
\curveto(461.60880607,45.89892506)(461.64880603,45.99392496)(461.67880493,46.09392873)
\curveto(461.69880598,46.19392476)(461.72380595,46.29892466)(461.75380493,46.40892873)
\curveto(461.7738059,46.4589245)(461.78380589,46.50392445)(461.78380493,46.54392873)
\curveto(461.7738059,46.59392436)(461.7738059,46.64392431)(461.78380493,46.69392873)
\curveto(461.79380588,46.73392422)(461.79880588,46.77892418)(461.79880493,46.82892873)
\lineto(461.79880493,46.96392873)
\lineto(461.79880493,47.09892873)
\curveto(461.78880589,47.13892382)(461.78380589,47.17392378)(461.78380493,47.20392873)
\curveto(461.78380589,47.23392372)(461.7788059,47.26892369)(461.76880493,47.30892873)
\curveto(461.74880593,47.38892357)(461.73380594,47.46392349)(461.72380493,47.53392873)
\curveto(461.70380597,47.60392335)(461.678806,47.67892328)(461.64880493,47.75892873)
\curveto(461.51880616,48.06892289)(461.34880633,48.31892264)(461.13880493,48.50892873)
\curveto(460.91880676,48.69892226)(460.65380702,48.8589221)(460.34380493,48.98892873)
\curveto(460.20380747,49.03892192)(460.06380761,49.07392188)(459.92380493,49.09392873)
\curveto(459.7738079,49.12392183)(459.62380805,49.1589218)(459.47380493,49.19892873)
\curveto(459.42380825,49.21892174)(459.3788083,49.22392173)(459.33880493,49.21392873)
\curveto(459.28880839,49.21392174)(459.23880844,49.21892174)(459.18880493,49.22892873)
\lineto(459.08380493,49.22892873)
}
}
{
\newrgbcolor{curcolor}{0 0 0}
\pscustom[linestyle=none,fillstyle=solid,fillcolor=curcolor]
{
\newpath
\moveto(454.82380493,55.69017873)
\curveto(454.82381285,55.92017394)(454.88381279,56.05017381)(455.00380493,56.08017873)
\curveto(455.11381256,56.11017375)(455.2788124,56.12517374)(455.49880493,56.12517873)
\lineto(455.78380493,56.12517873)
\curveto(455.8738118,56.12517374)(455.94881173,56.10017376)(456.00880493,56.05017873)
\curveto(456.08881159,55.99017387)(456.13381154,55.90517396)(456.14380493,55.79517873)
\curveto(456.14381153,55.68517418)(456.15881152,55.57517429)(456.18880493,55.46517873)
\curveto(456.21881146,55.32517454)(456.24881143,55.19017467)(456.27880493,55.06017873)
\curveto(456.30881137,54.94017492)(456.34881133,54.82517504)(456.39880493,54.71517873)
\curveto(456.52881115,54.42517544)(456.70881097,54.19017567)(456.93880493,54.01017873)
\curveto(457.15881052,53.83017603)(457.41381026,53.67517619)(457.70380493,53.54517873)
\curveto(457.81380986,53.50517636)(457.92880975,53.47517639)(458.04880493,53.45517873)
\curveto(458.15880952,53.43517643)(458.2738094,53.41017645)(458.39380493,53.38017873)
\curveto(458.44380923,53.37017649)(458.49380918,53.3651765)(458.54380493,53.36517873)
\curveto(458.59380908,53.37517649)(458.64380903,53.37517649)(458.69380493,53.36517873)
\curveto(458.81380886,53.33517653)(458.95380872,53.32017654)(459.11380493,53.32017873)
\curveto(459.26380841,53.33017653)(459.40880827,53.33517653)(459.54880493,53.33517873)
\lineto(461.39380493,53.33517873)
\lineto(461.73880493,53.33517873)
\curveto(461.85880582,53.33517653)(461.9738057,53.33017653)(462.08380493,53.32017873)
\curveto(462.19380548,53.31017655)(462.28880539,53.30517656)(462.36880493,53.30517873)
\curveto(462.44880523,53.31517655)(462.51880516,53.29517657)(462.57880493,53.24517873)
\curveto(462.64880503,53.19517667)(462.68880499,53.11517675)(462.69880493,53.00517873)
\curveto(462.70880497,52.90517696)(462.71380496,52.79517707)(462.71380493,52.67517873)
\lineto(462.71380493,52.40517873)
\curveto(462.69380498,52.35517751)(462.678805,52.30517756)(462.66880493,52.25517873)
\curveto(462.64880503,52.21517765)(462.62380505,52.18517768)(462.59380493,52.16517873)
\curveto(462.52380515,52.11517775)(462.43880524,52.08517778)(462.33880493,52.07517873)
\lineto(462.00880493,52.07517873)
\lineto(460.85380493,52.07517873)
\lineto(456.69880493,52.07517873)
\lineto(455.66380493,52.07517873)
\lineto(455.36380493,52.07517873)
\curveto(455.26381241,52.08517778)(455.1788125,52.11517775)(455.10880493,52.16517873)
\curveto(455.06881261,52.19517767)(455.03881264,52.24517762)(455.01880493,52.31517873)
\curveto(454.99881268,52.39517747)(454.98881269,52.48017738)(454.98880493,52.57017873)
\curveto(454.9788127,52.6601772)(454.9788127,52.75017711)(454.98880493,52.84017873)
\curveto(454.99881268,52.93017693)(455.01381266,53.00017686)(455.03380493,53.05017873)
\curveto(455.06381261,53.13017673)(455.12381255,53.18017668)(455.21380493,53.20017873)
\curveto(455.29381238,53.23017663)(455.38381229,53.24517662)(455.48380493,53.24517873)
\lineto(455.78380493,53.24517873)
\curveto(455.88381179,53.24517662)(455.9738117,53.2651766)(456.05380493,53.30517873)
\curveto(456.0738116,53.31517655)(456.08881159,53.32517654)(456.09880493,53.33517873)
\lineto(456.14380493,53.38017873)
\curveto(456.14381153,53.49017637)(456.09881158,53.58017628)(456.00880493,53.65017873)
\curveto(455.90881177,53.72017614)(455.82881185,53.78017608)(455.76880493,53.83017873)
\lineto(455.67880493,53.92017873)
\curveto(455.56881211,54.01017585)(455.45381222,54.13517573)(455.33380493,54.29517873)
\curveto(455.21381246,54.45517541)(455.12381255,54.60517526)(455.06380493,54.74517873)
\curveto(455.01381266,54.83517503)(454.9788127,54.93017493)(454.95880493,55.03017873)
\curveto(454.92881275,55.13017473)(454.89881278,55.23517463)(454.86880493,55.34517873)
\curveto(454.85881282,55.40517446)(454.85381282,55.4651744)(454.85380493,55.52517873)
\curveto(454.84381283,55.58517428)(454.83381284,55.64017422)(454.82380493,55.69017873)
}
}
{
\newrgbcolor{curcolor}{0 0 0}
\pscustom[linestyle=none,fillstyle=solid,fillcolor=curcolor]
{
}
}
{
\newrgbcolor{curcolor}{0 0 0}
\pscustom[linestyle=none,fillstyle=solid,fillcolor=curcolor]
{
\newpath
\moveto(452.12380493,64.24510061)
\curveto(452.11381556,64.93509597)(452.23381544,65.53509537)(452.48380493,66.04510061)
\curveto(452.73381494,66.56509434)(453.06881461,66.96009395)(453.48880493,67.23010061)
\curveto(453.56881411,67.28009363)(453.65881402,67.32509358)(453.75880493,67.36510061)
\curveto(453.84881383,67.4050935)(453.94381373,67.45009346)(454.04380493,67.50010061)
\curveto(454.14381353,67.54009337)(454.24381343,67.57009334)(454.34380493,67.59010061)
\curveto(454.44381323,67.6100933)(454.54881313,67.63009328)(454.65880493,67.65010061)
\curveto(454.70881297,67.67009324)(454.75381292,67.67509323)(454.79380493,67.66510061)
\curveto(454.83381284,67.65509325)(454.8788128,67.66009325)(454.92880493,67.68010061)
\curveto(454.9788127,67.69009322)(455.06381261,67.69509321)(455.18380493,67.69510061)
\curveto(455.29381238,67.69509321)(455.3788123,67.69009322)(455.43880493,67.68010061)
\curveto(455.49881218,67.66009325)(455.55881212,67.65009326)(455.61880493,67.65010061)
\curveto(455.678812,67.66009325)(455.73881194,67.65509325)(455.79880493,67.63510061)
\curveto(455.93881174,67.59509331)(456.0738116,67.56009335)(456.20380493,67.53010061)
\curveto(456.33381134,67.50009341)(456.45881122,67.46009345)(456.57880493,67.41010061)
\curveto(456.71881096,67.35009356)(456.84381083,67.28009363)(456.95380493,67.20010061)
\curveto(457.06381061,67.13009378)(457.1738105,67.05509385)(457.28380493,66.97510061)
\lineto(457.34380493,66.91510061)
\curveto(457.36381031,66.905094)(457.38381029,66.89009402)(457.40380493,66.87010061)
\curveto(457.56381011,66.75009416)(457.70880997,66.61509429)(457.83880493,66.46510061)
\curveto(457.96880971,66.31509459)(458.09380958,66.15509475)(458.21380493,65.98510061)
\curveto(458.43380924,65.67509523)(458.63880904,65.38009553)(458.82880493,65.10010061)
\curveto(458.96880871,64.87009604)(459.10380857,64.64009627)(459.23380493,64.41010061)
\curveto(459.36380831,64.19009672)(459.49880818,63.97009694)(459.63880493,63.75010061)
\curveto(459.80880787,63.50009741)(459.98880769,63.26009765)(460.17880493,63.03010061)
\curveto(460.36880731,62.8100981)(460.59380708,62.62009829)(460.85380493,62.46010061)
\curveto(460.91380676,62.42009849)(460.9738067,62.38509852)(461.03380493,62.35510061)
\curveto(461.08380659,62.32509858)(461.14880653,62.29509861)(461.22880493,62.26510061)
\curveto(461.29880638,62.24509866)(461.35880632,62.24009867)(461.40880493,62.25010061)
\curveto(461.4788062,62.27009864)(461.53380614,62.3050986)(461.57380493,62.35510061)
\curveto(461.60380607,62.4050985)(461.62380605,62.46509844)(461.63380493,62.53510061)
\lineto(461.63380493,62.77510061)
\lineto(461.63380493,63.52510061)
\lineto(461.63380493,66.33010061)
\lineto(461.63380493,66.99010061)
\curveto(461.63380604,67.08009383)(461.63880604,67.16509374)(461.64880493,67.24510061)
\curveto(461.64880603,67.32509358)(461.66880601,67.39009352)(461.70880493,67.44010061)
\curveto(461.74880593,67.49009342)(461.82380585,67.53009338)(461.93380493,67.56010061)
\curveto(462.03380564,67.60009331)(462.13380554,67.6100933)(462.23380493,67.59010061)
\lineto(462.36880493,67.59010061)
\curveto(462.43880524,67.57009334)(462.49880518,67.55009336)(462.54880493,67.53010061)
\curveto(462.59880508,67.5100934)(462.63880504,67.47509343)(462.66880493,67.42510061)
\curveto(462.70880497,67.37509353)(462.72880495,67.3050936)(462.72880493,67.21510061)
\lineto(462.72880493,66.94510061)
\lineto(462.72880493,66.04510061)
\lineto(462.72880493,62.53510061)
\lineto(462.72880493,61.47010061)
\curveto(462.72880495,61.39009952)(462.73380494,61.30009961)(462.74380493,61.20010061)
\curveto(462.74380493,61.10009981)(462.73380494,61.01509989)(462.71380493,60.94510061)
\curveto(462.64380503,60.73510017)(462.46380521,60.67010024)(462.17380493,60.75010061)
\curveto(462.13380554,60.76010015)(462.09880558,60.76010015)(462.06880493,60.75010061)
\curveto(462.02880565,60.75010016)(461.98380569,60.76010015)(461.93380493,60.78010061)
\curveto(461.85380582,60.80010011)(461.76880591,60.82010009)(461.67880493,60.84010061)
\curveto(461.58880609,60.86010005)(461.50380617,60.88510002)(461.42380493,60.91510061)
\curveto(460.93380674,61.07509983)(460.51880716,61.27509963)(460.17880493,61.51510061)
\curveto(459.92880775,61.69509921)(459.70380797,61.90009901)(459.50380493,62.13010061)
\curveto(459.29380838,62.36009855)(459.09880858,62.60009831)(458.91880493,62.85010061)
\curveto(458.73880894,63.1100978)(458.56880911,63.37509753)(458.40880493,63.64510061)
\curveto(458.23880944,63.92509698)(458.06380961,64.19509671)(457.88380493,64.45510061)
\curveto(457.80380987,64.56509634)(457.72880995,64.67009624)(457.65880493,64.77010061)
\curveto(457.58881009,64.88009603)(457.51381016,64.99009592)(457.43380493,65.10010061)
\curveto(457.40381027,65.14009577)(457.3738103,65.17509573)(457.34380493,65.20510061)
\curveto(457.30381037,65.24509566)(457.2738104,65.28509562)(457.25380493,65.32510061)
\curveto(457.14381053,65.46509544)(457.01881066,65.59009532)(456.87880493,65.70010061)
\curveto(456.84881083,65.72009519)(456.82381085,65.74509516)(456.80380493,65.77510061)
\curveto(456.7738109,65.8050951)(456.74381093,65.83009508)(456.71380493,65.85010061)
\curveto(456.61381106,65.93009498)(456.51381116,65.99509491)(456.41380493,66.04510061)
\curveto(456.31381136,66.1050948)(456.20381147,66.16009475)(456.08380493,66.21010061)
\curveto(456.01381166,66.24009467)(455.93881174,66.26009465)(455.85880493,66.27010061)
\lineto(455.61880493,66.33010061)
\lineto(455.52880493,66.33010061)
\curveto(455.49881218,66.34009457)(455.46881221,66.34509456)(455.43880493,66.34510061)
\curveto(455.36881231,66.36509454)(455.2738124,66.37009454)(455.15380493,66.36010061)
\curveto(455.02381265,66.36009455)(454.92381275,66.35009456)(454.85380493,66.33010061)
\curveto(454.7738129,66.3100946)(454.69881298,66.29009462)(454.62880493,66.27010061)
\curveto(454.54881313,66.26009465)(454.46881321,66.24009467)(454.38880493,66.21010061)
\curveto(454.14881353,66.10009481)(453.94881373,65.95009496)(453.78880493,65.76010061)
\curveto(453.61881406,65.58009533)(453.4788142,65.36009555)(453.36880493,65.10010061)
\curveto(453.34881433,65.03009588)(453.33381434,64.96009595)(453.32380493,64.89010061)
\curveto(453.30381437,64.82009609)(453.28381439,64.74509616)(453.26380493,64.66510061)
\curveto(453.24381443,64.58509632)(453.23381444,64.47509643)(453.23380493,64.33510061)
\curveto(453.23381444,64.2050967)(453.24381443,64.10009681)(453.26380493,64.02010061)
\curveto(453.2738144,63.96009695)(453.2788144,63.905097)(453.27880493,63.85510061)
\curveto(453.2788144,63.8050971)(453.28881439,63.75509715)(453.30880493,63.70510061)
\curveto(453.34881433,63.6050973)(453.38881429,63.5100974)(453.42880493,63.42010061)
\curveto(453.46881421,63.34009757)(453.51381416,63.26009765)(453.56380493,63.18010061)
\curveto(453.58381409,63.15009776)(453.60881407,63.12009779)(453.63880493,63.09010061)
\curveto(453.66881401,63.07009784)(453.69381398,63.04509786)(453.71380493,63.01510061)
\lineto(453.78880493,62.94010061)
\curveto(453.80881387,62.910098)(453.82881385,62.88509802)(453.84880493,62.86510061)
\lineto(454.05880493,62.71510061)
\curveto(454.11881356,62.67509823)(454.18381349,62.63009828)(454.25380493,62.58010061)
\curveto(454.34381333,62.52009839)(454.44881323,62.47009844)(454.56880493,62.43010061)
\curveto(454.678813,62.40009851)(454.78881289,62.36509854)(454.89880493,62.32510061)
\curveto(455.00881267,62.28509862)(455.15381252,62.26009865)(455.33380493,62.25010061)
\curveto(455.50381217,62.24009867)(455.62881205,62.2100987)(455.70880493,62.16010061)
\curveto(455.78881189,62.1100988)(455.83381184,62.03509887)(455.84380493,61.93510061)
\curveto(455.85381182,61.83509907)(455.85881182,61.72509918)(455.85880493,61.60510061)
\curveto(455.85881182,61.56509934)(455.86381181,61.52509938)(455.87380493,61.48510061)
\curveto(455.8738118,61.44509946)(455.86881181,61.4100995)(455.85880493,61.38010061)
\curveto(455.83881184,61.33009958)(455.82881185,61.28009963)(455.82880493,61.23010061)
\curveto(455.82881185,61.19009972)(455.81881186,61.15009976)(455.79880493,61.11010061)
\curveto(455.73881194,61.02009989)(455.60381207,60.97509993)(455.39380493,60.97510061)
\lineto(455.27380493,60.97510061)
\curveto(455.21381246,60.98509992)(455.15381252,60.99009992)(455.09380493,60.99010061)
\curveto(455.02381265,61.00009991)(454.95881272,61.0100999)(454.89880493,61.02010061)
\curveto(454.78881289,61.04009987)(454.68881299,61.06009985)(454.59880493,61.08010061)
\curveto(454.49881318,61.10009981)(454.40381327,61.13009978)(454.31380493,61.17010061)
\curveto(454.24381343,61.19009972)(454.18381349,61.2100997)(454.13380493,61.23010061)
\lineto(453.95380493,61.29010061)
\curveto(453.69381398,61.4100995)(453.44881423,61.56509934)(453.21880493,61.75510061)
\curveto(452.98881469,61.95509895)(452.80381487,62.17009874)(452.66380493,62.40010061)
\curveto(452.58381509,62.5100984)(452.51881516,62.62509828)(452.46880493,62.74510061)
\lineto(452.31880493,63.13510061)
\curveto(452.26881541,63.24509766)(452.23881544,63.36009755)(452.22880493,63.48010061)
\curveto(452.20881547,63.60009731)(452.18381549,63.72509718)(452.15380493,63.85510061)
\curveto(452.15381552,63.92509698)(452.15381552,63.99009692)(452.15380493,64.05010061)
\curveto(452.14381553,64.1100968)(452.13381554,64.17509673)(452.12380493,64.24510061)
}
}
{
\newrgbcolor{curcolor}{0 0 0}
\pscustom[linestyle=none,fillstyle=solid,fillcolor=curcolor]
{
\newpath
\moveto(452.31880493,69.80470998)
\lineto(452.31880493,74.60470998)
\lineto(452.31880493,75.60970998)
\curveto(452.31881536,75.74970288)(452.32881535,75.86970276)(452.34880493,75.96970998)
\curveto(452.35881532,76.07970255)(452.40381527,76.15970247)(452.48380493,76.20970998)
\curveto(452.52381515,76.2297024)(452.5738151,76.23970239)(452.63380493,76.23970998)
\curveto(452.69381498,76.24970238)(452.75881492,76.25470238)(452.82880493,76.25470998)
\lineto(453.09880493,76.25470998)
\curveto(453.18881449,76.25470238)(453.26881441,76.24470239)(453.33880493,76.22470998)
\curveto(453.41881426,76.18470245)(453.48881419,76.13970249)(453.54880493,76.08970998)
\lineto(453.72880493,75.93970998)
\curveto(453.7788139,75.90970272)(453.81881386,75.87470276)(453.84880493,75.83470998)
\curveto(453.8788138,75.79470284)(453.91881376,75.75470288)(453.96880493,75.71470998)
\curveto(454.0788136,75.634703)(454.18881349,75.54970308)(454.29880493,75.45970998)
\curveto(454.39881328,75.36970326)(454.50381317,75.28470335)(454.61380493,75.20470998)
\curveto(454.81381286,75.06470357)(455.02381265,74.92470371)(455.24380493,74.78470998)
\curveto(455.45381222,74.64470399)(455.66881201,74.50470413)(455.88880493,74.36470998)
\curveto(455.9788117,74.31470432)(456.0738116,74.26470437)(456.17380493,74.21470998)
\curveto(456.2738114,74.16470447)(456.36881131,74.10970452)(456.45880493,74.04970998)
\curveto(456.4788112,74.0297046)(456.50381117,74.01970461)(456.53380493,74.01970998)
\curveto(456.56381111,74.01970461)(456.58881109,74.00970462)(456.60880493,73.98970998)
\curveto(456.70881097,73.91970471)(456.82381085,73.85470478)(456.95380493,73.79470998)
\curveto(457.0738106,73.7347049)(457.18881049,73.67970495)(457.29880493,73.62970998)
\curveto(457.52881015,73.5297051)(457.76380991,73.4347052)(458.00380493,73.34470998)
\curveto(458.24380943,73.25470538)(458.48380919,73.15470548)(458.72380493,73.04470998)
\curveto(458.7738089,73.02470561)(458.81880886,73.00970562)(458.85880493,72.99970998)
\curveto(458.89880878,72.99970563)(458.94380873,72.98970564)(458.99380493,72.96970998)
\curveto(459.11380856,72.91970571)(459.23880844,72.87470576)(459.36880493,72.83470998)
\curveto(459.48880819,72.80470583)(459.60880807,72.76970586)(459.72880493,72.72970998)
\curveto(459.95880772,72.64970598)(460.19880748,72.58470605)(460.44880493,72.53470998)
\curveto(460.68880699,72.49470614)(460.92880675,72.44470619)(461.16880493,72.38470998)
\curveto(461.31880636,72.34470629)(461.46880621,72.31970631)(461.61880493,72.30970998)
\curveto(461.76880591,72.29970633)(461.91880576,72.27970635)(462.06880493,72.24970998)
\curveto(462.10880557,72.23970639)(462.16880551,72.2347064)(462.24880493,72.23470998)
\curveto(462.36880531,72.20470643)(462.46880521,72.17470646)(462.54880493,72.14470998)
\curveto(462.62880505,72.11470652)(462.68380499,72.04470659)(462.71380493,71.93470998)
\curveto(462.73380494,71.88470675)(462.74380493,71.8297068)(462.74380493,71.76970998)
\lineto(462.74380493,71.57470998)
\curveto(462.74380493,71.4347072)(462.73880494,71.29470734)(462.72880493,71.15470998)
\curveto(462.71880496,71.02470761)(462.673805,70.9297077)(462.59380493,70.86970998)
\curveto(462.53380514,70.8297078)(462.44880523,70.80970782)(462.33880493,70.80970998)
\curveto(462.22880545,70.81970781)(462.13380554,70.8347078)(462.05380493,70.85470998)
\lineto(461.97880493,70.85470998)
\curveto(461.94880573,70.86470777)(461.91880576,70.86970776)(461.88880493,70.86970998)
\curveto(461.80880587,70.88970774)(461.73380594,70.89970773)(461.66380493,70.89970998)
\curveto(461.59380608,70.89970773)(461.52380615,70.90970772)(461.45380493,70.92970998)
\curveto(461.26380641,70.97970765)(461.0788066,71.01970761)(460.89880493,71.04970998)
\curveto(460.70880697,71.07970755)(460.52880715,71.11970751)(460.35880493,71.16970998)
\curveto(460.30880737,71.18970744)(460.26880741,71.19970743)(460.23880493,71.19970998)
\curveto(460.20880747,71.19970743)(460.1738075,71.20470743)(460.13380493,71.21470998)
\curveto(459.83380784,71.31470732)(459.53880814,71.40470723)(459.24880493,71.48470998)
\curveto(458.95880872,71.57470706)(458.678809,71.67970695)(458.40880493,71.79970998)
\curveto(457.82880985,72.05970657)(457.2788104,72.3297063)(456.75880493,72.60970998)
\curveto(456.22881145,72.88970574)(455.72381195,73.19970543)(455.24380493,73.53970998)
\curveto(455.04381263,73.67970495)(454.85381282,73.8297048)(454.67380493,73.98970998)
\curveto(454.48381319,74.14970448)(454.29381338,74.29970433)(454.10380493,74.43970998)
\curveto(454.05381362,74.47970415)(454.00881367,74.51470412)(453.96880493,74.54470998)
\curveto(453.91881376,74.58470405)(453.86881381,74.61970401)(453.81880493,74.64970998)
\curveto(453.79881388,74.65970397)(453.7738139,74.66970396)(453.74380493,74.67970998)
\curveto(453.71381396,74.69970393)(453.68381399,74.69970393)(453.65380493,74.67970998)
\curveto(453.59381408,74.65970397)(453.55881412,74.62470401)(453.54880493,74.57470998)
\curveto(453.52881415,74.52470411)(453.50881417,74.47470416)(453.48880493,74.42470998)
\lineto(453.48880493,74.31970998)
\curveto(453.4788142,74.27970435)(453.4788142,74.2297044)(453.48880493,74.16970998)
\lineto(453.48880493,74.01970998)
\lineto(453.48880493,73.41970998)
\lineto(453.48880493,70.77970998)
\lineto(453.48880493,70.04470998)
\lineto(453.48880493,69.80470998)
\curveto(453.4788142,69.7347089)(453.46381421,69.67470896)(453.44380493,69.62470998)
\curveto(453.40381427,69.5347091)(453.34381433,69.47470916)(453.26380493,69.44470998)
\curveto(453.16381451,69.39470924)(453.01881466,69.37970925)(452.82880493,69.39970998)
\curveto(452.62881505,69.41970921)(452.49381518,69.45470918)(452.42380493,69.50470998)
\curveto(452.40381527,69.52470911)(452.38881529,69.54970908)(452.37880493,69.57970998)
\lineto(452.31880493,69.69970998)
\curveto(452.31881536,69.71970891)(452.32381535,69.7347089)(452.33380493,69.74470998)
\curveto(452.33381534,69.76470887)(452.32881535,69.78470885)(452.31880493,69.80470998)
}
}
{
\newrgbcolor{curcolor}{0 0 0}
\pscustom[linestyle=none,fillstyle=solid,fillcolor=curcolor]
{
\newpath
\moveto(461.09380493,78.63431936)
\lineto(461.09380493,79.26431936)
\lineto(461.09380493,79.45931936)
\curveto(461.09380658,79.52931683)(461.10380657,79.58931677)(461.12380493,79.63931936)
\curveto(461.16380651,79.70931665)(461.20380647,79.7593166)(461.24380493,79.78931936)
\curveto(461.29380638,79.82931653)(461.35880632,79.84931651)(461.43880493,79.84931936)
\curveto(461.51880616,79.8593165)(461.60380607,79.86431649)(461.69380493,79.86431936)
\lineto(462.41380493,79.86431936)
\curveto(462.89380478,79.86431649)(463.30380437,79.80431655)(463.64380493,79.68431936)
\curveto(463.98380369,79.56431679)(464.25880342,79.36931699)(464.46880493,79.09931936)
\curveto(464.51880316,79.02931733)(464.56380311,78.9593174)(464.60380493,78.88931936)
\curveto(464.65380302,78.82931753)(464.69880298,78.7543176)(464.73880493,78.66431936)
\curveto(464.74880293,78.64431771)(464.75880292,78.61431774)(464.76880493,78.57431936)
\curveto(464.78880289,78.53431782)(464.79380288,78.48931787)(464.78380493,78.43931936)
\curveto(464.75380292,78.34931801)(464.678803,78.29431806)(464.55880493,78.27431936)
\curveto(464.44880323,78.2543181)(464.35380332,78.26931809)(464.27380493,78.31931936)
\curveto(464.20380347,78.34931801)(464.13880354,78.39431796)(464.07880493,78.45431936)
\curveto(464.02880365,78.52431783)(463.9788037,78.58931777)(463.92880493,78.64931936)
\curveto(463.8788038,78.71931764)(463.80380387,78.77931758)(463.70380493,78.82931936)
\curveto(463.61380406,78.88931747)(463.52380415,78.93931742)(463.43380493,78.97931936)
\curveto(463.40380427,78.99931736)(463.34380433,79.02431733)(463.25380493,79.05431936)
\curveto(463.1738045,79.08431727)(463.10380457,79.08931727)(463.04380493,79.06931936)
\curveto(462.90380477,79.03931732)(462.81380486,78.97931738)(462.77380493,78.88931936)
\curveto(462.74380493,78.80931755)(462.72880495,78.71931764)(462.72880493,78.61931936)
\curveto(462.72880495,78.51931784)(462.70380497,78.43431792)(462.65380493,78.36431936)
\curveto(462.58380509,78.27431808)(462.44380523,78.22931813)(462.23380493,78.22931936)
\lineto(461.67880493,78.22931936)
\lineto(461.45380493,78.22931936)
\curveto(461.3738063,78.23931812)(461.30880637,78.2593181)(461.25880493,78.28931936)
\curveto(461.1788065,78.34931801)(461.13380654,78.41931794)(461.12380493,78.49931936)
\curveto(461.11380656,78.51931784)(461.10880657,78.53931782)(461.10880493,78.55931936)
\curveto(461.10880657,78.58931777)(461.10380657,78.61431774)(461.09380493,78.63431936)
}
}
{
\newrgbcolor{curcolor}{0 0 0}
\pscustom[linestyle=none,fillstyle=solid,fillcolor=curcolor]
{
}
}
{
\newrgbcolor{curcolor}{0 0 0}
\pscustom[linestyle=none,fillstyle=solid,fillcolor=curcolor]
{
\newpath
\moveto(452.12380493,89.26463186)
\curveto(452.11381556,89.95462722)(452.23381544,90.55462662)(452.48380493,91.06463186)
\curveto(452.73381494,91.58462559)(453.06881461,91.9796252)(453.48880493,92.24963186)
\curveto(453.56881411,92.29962488)(453.65881402,92.34462483)(453.75880493,92.38463186)
\curveto(453.84881383,92.42462475)(453.94381373,92.46962471)(454.04380493,92.51963186)
\curveto(454.14381353,92.55962462)(454.24381343,92.58962459)(454.34380493,92.60963186)
\curveto(454.44381323,92.62962455)(454.54881313,92.64962453)(454.65880493,92.66963186)
\curveto(454.70881297,92.68962449)(454.75381292,92.69462448)(454.79380493,92.68463186)
\curveto(454.83381284,92.6746245)(454.8788128,92.6796245)(454.92880493,92.69963186)
\curveto(454.9788127,92.70962447)(455.06381261,92.71462446)(455.18380493,92.71463186)
\curveto(455.29381238,92.71462446)(455.3788123,92.70962447)(455.43880493,92.69963186)
\curveto(455.49881218,92.6796245)(455.55881212,92.66962451)(455.61880493,92.66963186)
\curveto(455.678812,92.6796245)(455.73881194,92.6746245)(455.79880493,92.65463186)
\curveto(455.93881174,92.61462456)(456.0738116,92.5796246)(456.20380493,92.54963186)
\curveto(456.33381134,92.51962466)(456.45881122,92.4796247)(456.57880493,92.42963186)
\curveto(456.71881096,92.36962481)(456.84381083,92.29962488)(456.95380493,92.21963186)
\curveto(457.06381061,92.14962503)(457.1738105,92.0746251)(457.28380493,91.99463186)
\lineto(457.34380493,91.93463186)
\curveto(457.36381031,91.92462525)(457.38381029,91.90962527)(457.40380493,91.88963186)
\curveto(457.56381011,91.76962541)(457.70880997,91.63462554)(457.83880493,91.48463186)
\curveto(457.96880971,91.33462584)(458.09380958,91.174626)(458.21380493,91.00463186)
\curveto(458.43380924,90.69462648)(458.63880904,90.39962678)(458.82880493,90.11963186)
\curveto(458.96880871,89.88962729)(459.10380857,89.65962752)(459.23380493,89.42963186)
\curveto(459.36380831,89.20962797)(459.49880818,88.98962819)(459.63880493,88.76963186)
\curveto(459.80880787,88.51962866)(459.98880769,88.2796289)(460.17880493,88.04963186)
\curveto(460.36880731,87.82962935)(460.59380708,87.63962954)(460.85380493,87.47963186)
\curveto(460.91380676,87.43962974)(460.9738067,87.40462977)(461.03380493,87.37463186)
\curveto(461.08380659,87.34462983)(461.14880653,87.31462986)(461.22880493,87.28463186)
\curveto(461.29880638,87.26462991)(461.35880632,87.25962992)(461.40880493,87.26963186)
\curveto(461.4788062,87.28962989)(461.53380614,87.32462985)(461.57380493,87.37463186)
\curveto(461.60380607,87.42462975)(461.62380605,87.48462969)(461.63380493,87.55463186)
\lineto(461.63380493,87.79463186)
\lineto(461.63380493,88.54463186)
\lineto(461.63380493,91.34963186)
\lineto(461.63380493,92.00963186)
\curveto(461.63380604,92.09962508)(461.63880604,92.18462499)(461.64880493,92.26463186)
\curveto(461.64880603,92.34462483)(461.66880601,92.40962477)(461.70880493,92.45963186)
\curveto(461.74880593,92.50962467)(461.82380585,92.54962463)(461.93380493,92.57963186)
\curveto(462.03380564,92.61962456)(462.13380554,92.62962455)(462.23380493,92.60963186)
\lineto(462.36880493,92.60963186)
\curveto(462.43880524,92.58962459)(462.49880518,92.56962461)(462.54880493,92.54963186)
\curveto(462.59880508,92.52962465)(462.63880504,92.49462468)(462.66880493,92.44463186)
\curveto(462.70880497,92.39462478)(462.72880495,92.32462485)(462.72880493,92.23463186)
\lineto(462.72880493,91.96463186)
\lineto(462.72880493,91.06463186)
\lineto(462.72880493,87.55463186)
\lineto(462.72880493,86.48963186)
\curveto(462.72880495,86.40963077)(462.73380494,86.31963086)(462.74380493,86.21963186)
\curveto(462.74380493,86.11963106)(462.73380494,86.03463114)(462.71380493,85.96463186)
\curveto(462.64380503,85.75463142)(462.46380521,85.68963149)(462.17380493,85.76963186)
\curveto(462.13380554,85.7796314)(462.09880558,85.7796314)(462.06880493,85.76963186)
\curveto(462.02880565,85.76963141)(461.98380569,85.7796314)(461.93380493,85.79963186)
\curveto(461.85380582,85.81963136)(461.76880591,85.83963134)(461.67880493,85.85963186)
\curveto(461.58880609,85.8796313)(461.50380617,85.90463127)(461.42380493,85.93463186)
\curveto(460.93380674,86.09463108)(460.51880716,86.29463088)(460.17880493,86.53463186)
\curveto(459.92880775,86.71463046)(459.70380797,86.91963026)(459.50380493,87.14963186)
\curveto(459.29380838,87.3796298)(459.09880858,87.61962956)(458.91880493,87.86963186)
\curveto(458.73880894,88.12962905)(458.56880911,88.39462878)(458.40880493,88.66463186)
\curveto(458.23880944,88.94462823)(458.06380961,89.21462796)(457.88380493,89.47463186)
\curveto(457.80380987,89.58462759)(457.72880995,89.68962749)(457.65880493,89.78963186)
\curveto(457.58881009,89.89962728)(457.51381016,90.00962717)(457.43380493,90.11963186)
\curveto(457.40381027,90.15962702)(457.3738103,90.19462698)(457.34380493,90.22463186)
\curveto(457.30381037,90.26462691)(457.2738104,90.30462687)(457.25380493,90.34463186)
\curveto(457.14381053,90.48462669)(457.01881066,90.60962657)(456.87880493,90.71963186)
\curveto(456.84881083,90.73962644)(456.82381085,90.76462641)(456.80380493,90.79463186)
\curveto(456.7738109,90.82462635)(456.74381093,90.84962633)(456.71380493,90.86963186)
\curveto(456.61381106,90.94962623)(456.51381116,91.01462616)(456.41380493,91.06463186)
\curveto(456.31381136,91.12462605)(456.20381147,91.179626)(456.08380493,91.22963186)
\curveto(456.01381166,91.25962592)(455.93881174,91.2796259)(455.85880493,91.28963186)
\lineto(455.61880493,91.34963186)
\lineto(455.52880493,91.34963186)
\curveto(455.49881218,91.35962582)(455.46881221,91.36462581)(455.43880493,91.36463186)
\curveto(455.36881231,91.38462579)(455.2738124,91.38962579)(455.15380493,91.37963186)
\curveto(455.02381265,91.3796258)(454.92381275,91.36962581)(454.85380493,91.34963186)
\curveto(454.7738129,91.32962585)(454.69881298,91.30962587)(454.62880493,91.28963186)
\curveto(454.54881313,91.2796259)(454.46881321,91.25962592)(454.38880493,91.22963186)
\curveto(454.14881353,91.11962606)(453.94881373,90.96962621)(453.78880493,90.77963186)
\curveto(453.61881406,90.59962658)(453.4788142,90.3796268)(453.36880493,90.11963186)
\curveto(453.34881433,90.04962713)(453.33381434,89.9796272)(453.32380493,89.90963186)
\curveto(453.30381437,89.83962734)(453.28381439,89.76462741)(453.26380493,89.68463186)
\curveto(453.24381443,89.60462757)(453.23381444,89.49462768)(453.23380493,89.35463186)
\curveto(453.23381444,89.22462795)(453.24381443,89.11962806)(453.26380493,89.03963186)
\curveto(453.2738144,88.9796282)(453.2788144,88.92462825)(453.27880493,88.87463186)
\curveto(453.2788144,88.82462835)(453.28881439,88.7746284)(453.30880493,88.72463186)
\curveto(453.34881433,88.62462855)(453.38881429,88.52962865)(453.42880493,88.43963186)
\curveto(453.46881421,88.35962882)(453.51381416,88.2796289)(453.56380493,88.19963186)
\curveto(453.58381409,88.16962901)(453.60881407,88.13962904)(453.63880493,88.10963186)
\curveto(453.66881401,88.08962909)(453.69381398,88.06462911)(453.71380493,88.03463186)
\lineto(453.78880493,87.95963186)
\curveto(453.80881387,87.92962925)(453.82881385,87.90462927)(453.84880493,87.88463186)
\lineto(454.05880493,87.73463186)
\curveto(454.11881356,87.69462948)(454.18381349,87.64962953)(454.25380493,87.59963186)
\curveto(454.34381333,87.53962964)(454.44881323,87.48962969)(454.56880493,87.44963186)
\curveto(454.678813,87.41962976)(454.78881289,87.38462979)(454.89880493,87.34463186)
\curveto(455.00881267,87.30462987)(455.15381252,87.2796299)(455.33380493,87.26963186)
\curveto(455.50381217,87.25962992)(455.62881205,87.22962995)(455.70880493,87.17963186)
\curveto(455.78881189,87.12963005)(455.83381184,87.05463012)(455.84380493,86.95463186)
\curveto(455.85381182,86.85463032)(455.85881182,86.74463043)(455.85880493,86.62463186)
\curveto(455.85881182,86.58463059)(455.86381181,86.54463063)(455.87380493,86.50463186)
\curveto(455.8738118,86.46463071)(455.86881181,86.42963075)(455.85880493,86.39963186)
\curveto(455.83881184,86.34963083)(455.82881185,86.29963088)(455.82880493,86.24963186)
\curveto(455.82881185,86.20963097)(455.81881186,86.16963101)(455.79880493,86.12963186)
\curveto(455.73881194,86.03963114)(455.60381207,85.99463118)(455.39380493,85.99463186)
\lineto(455.27380493,85.99463186)
\curveto(455.21381246,86.00463117)(455.15381252,86.00963117)(455.09380493,86.00963186)
\curveto(455.02381265,86.01963116)(454.95881272,86.02963115)(454.89880493,86.03963186)
\curveto(454.78881289,86.05963112)(454.68881299,86.0796311)(454.59880493,86.09963186)
\curveto(454.49881318,86.11963106)(454.40381327,86.14963103)(454.31380493,86.18963186)
\curveto(454.24381343,86.20963097)(454.18381349,86.22963095)(454.13380493,86.24963186)
\lineto(453.95380493,86.30963186)
\curveto(453.69381398,86.42963075)(453.44881423,86.58463059)(453.21880493,86.77463186)
\curveto(452.98881469,86.9746302)(452.80381487,87.18962999)(452.66380493,87.41963186)
\curveto(452.58381509,87.52962965)(452.51881516,87.64462953)(452.46880493,87.76463186)
\lineto(452.31880493,88.15463186)
\curveto(452.26881541,88.26462891)(452.23881544,88.3796288)(452.22880493,88.49963186)
\curveto(452.20881547,88.61962856)(452.18381549,88.74462843)(452.15380493,88.87463186)
\curveto(452.15381552,88.94462823)(452.15381552,89.00962817)(452.15380493,89.06963186)
\curveto(452.14381553,89.12962805)(452.13381554,89.19462798)(452.12380493,89.26463186)
}
}
{
\newrgbcolor{curcolor}{0 0 0}
\pscustom[linestyle=none,fillstyle=solid,fillcolor=curcolor]
{
\newpath
\moveto(457.64380493,101.36424123)
\lineto(457.89880493,101.36424123)
\curveto(457.9788097,101.37423353)(458.05380962,101.36923353)(458.12380493,101.34924123)
\lineto(458.36380493,101.34924123)
\lineto(458.52880493,101.34924123)
\curveto(458.62880905,101.32923357)(458.73380894,101.31923358)(458.84380493,101.31924123)
\curveto(458.94380873,101.31923358)(459.04380863,101.30923359)(459.14380493,101.28924123)
\lineto(459.29380493,101.28924123)
\curveto(459.43380824,101.25923364)(459.5738081,101.23923366)(459.71380493,101.22924123)
\curveto(459.84380783,101.21923368)(459.9738077,101.19423371)(460.10380493,101.15424123)
\curveto(460.18380749,101.13423377)(460.26880741,101.11423379)(460.35880493,101.09424123)
\lineto(460.59880493,101.03424123)
\lineto(460.89880493,100.91424123)
\curveto(460.98880669,100.88423402)(461.0788066,100.84923405)(461.16880493,100.80924123)
\curveto(461.38880629,100.70923419)(461.60380607,100.57423433)(461.81380493,100.40424123)
\curveto(462.02380565,100.24423466)(462.19380548,100.06923483)(462.32380493,99.87924123)
\curveto(462.36380531,99.82923507)(462.40380527,99.76923513)(462.44380493,99.69924123)
\curveto(462.4738052,99.63923526)(462.50880517,99.57923532)(462.54880493,99.51924123)
\curveto(462.59880508,99.43923546)(462.63880504,99.34423556)(462.66880493,99.23424123)
\curveto(462.69880498,99.12423578)(462.72880495,99.01923588)(462.75880493,98.91924123)
\curveto(462.79880488,98.80923609)(462.82380485,98.6992362)(462.83380493,98.58924123)
\curveto(462.84380483,98.47923642)(462.85880482,98.36423654)(462.87880493,98.24424123)
\curveto(462.88880479,98.2042367)(462.88880479,98.15923674)(462.87880493,98.10924123)
\curveto(462.8788048,98.06923683)(462.88380479,98.02923687)(462.89380493,97.98924123)
\curveto(462.90380477,97.94923695)(462.90880477,97.89423701)(462.90880493,97.82424123)
\curveto(462.90880477,97.75423715)(462.90380477,97.7042372)(462.89380493,97.67424123)
\curveto(462.8738048,97.62423728)(462.86880481,97.57923732)(462.87880493,97.53924123)
\curveto(462.88880479,97.4992374)(462.88880479,97.46423744)(462.87880493,97.43424123)
\lineto(462.87880493,97.34424123)
\curveto(462.85880482,97.28423762)(462.84380483,97.21923768)(462.83380493,97.14924123)
\curveto(462.83380484,97.08923781)(462.82880485,97.02423788)(462.81880493,96.95424123)
\curveto(462.76880491,96.78423812)(462.71880496,96.62423828)(462.66880493,96.47424123)
\curveto(462.61880506,96.32423858)(462.55380512,96.17923872)(462.47380493,96.03924123)
\curveto(462.43380524,95.98923891)(462.40380527,95.93423897)(462.38380493,95.87424123)
\curveto(462.35380532,95.82423908)(462.31880536,95.77423913)(462.27880493,95.72424123)
\curveto(462.09880558,95.48423942)(461.8788058,95.28423962)(461.61880493,95.12424123)
\curveto(461.35880632,94.96423994)(461.0738066,94.82424008)(460.76380493,94.70424123)
\curveto(460.62380705,94.64424026)(460.48380719,94.5992403)(460.34380493,94.56924123)
\curveto(460.19380748,94.53924036)(460.03880764,94.5042404)(459.87880493,94.46424123)
\curveto(459.76880791,94.44424046)(459.65880802,94.42924047)(459.54880493,94.41924123)
\curveto(459.43880824,94.40924049)(459.32880835,94.39424051)(459.21880493,94.37424123)
\curveto(459.1788085,94.36424054)(459.13880854,94.35924054)(459.09880493,94.35924123)
\curveto(459.05880862,94.36924053)(459.01880866,94.36924053)(458.97880493,94.35924123)
\curveto(458.92880875,94.34924055)(458.8788088,94.34424056)(458.82880493,94.34424123)
\lineto(458.66380493,94.34424123)
\curveto(458.61380906,94.32424058)(458.56380911,94.31924058)(458.51380493,94.32924123)
\curveto(458.45380922,94.33924056)(458.39880928,94.33924056)(458.34880493,94.32924123)
\curveto(458.30880937,94.31924058)(458.26380941,94.31924058)(458.21380493,94.32924123)
\curveto(458.16380951,94.33924056)(458.11380956,94.33424057)(458.06380493,94.31424123)
\curveto(457.99380968,94.29424061)(457.91880976,94.28924061)(457.83880493,94.29924123)
\curveto(457.74880993,94.30924059)(457.66381001,94.31424059)(457.58380493,94.31424123)
\curveto(457.49381018,94.31424059)(457.39381028,94.30924059)(457.28380493,94.29924123)
\curveto(457.16381051,94.28924061)(457.06381061,94.29424061)(456.98380493,94.31424123)
\lineto(456.69880493,94.31424123)
\lineto(456.06880493,94.35924123)
\curveto(455.96881171,94.36924053)(455.8738118,94.37924052)(455.78380493,94.38924123)
\lineto(455.48380493,94.41924123)
\curveto(455.43381224,94.43924046)(455.38381229,94.44424046)(455.33380493,94.43424123)
\curveto(455.2738124,94.43424047)(455.21881246,94.44424046)(455.16880493,94.46424123)
\curveto(454.99881268,94.51424039)(454.83381284,94.55424035)(454.67380493,94.58424123)
\curveto(454.50381317,94.61424029)(454.34381333,94.66424024)(454.19380493,94.73424123)
\curveto(453.73381394,94.92423998)(453.35881432,95.14423976)(453.06880493,95.39424123)
\curveto(452.7788149,95.65423925)(452.53381514,96.01423889)(452.33380493,96.47424123)
\curveto(452.28381539,96.6042383)(452.24881543,96.73423817)(452.22880493,96.86424123)
\curveto(452.20881547,97.0042379)(452.18381549,97.14423776)(452.15380493,97.28424123)
\curveto(452.14381553,97.35423755)(452.13881554,97.41923748)(452.13880493,97.47924123)
\curveto(452.13881554,97.53923736)(452.13381554,97.6042373)(452.12380493,97.67424123)
\curveto(452.10381557,98.5042364)(452.25381542,99.17423573)(452.57380493,99.68424123)
\curveto(452.88381479,100.19423471)(453.32381435,100.57423433)(453.89380493,100.82424123)
\curveto(454.01381366,100.87423403)(454.13881354,100.91923398)(454.26880493,100.95924123)
\curveto(454.39881328,100.9992339)(454.53381314,101.04423386)(454.67380493,101.09424123)
\curveto(454.75381292,101.11423379)(454.83881284,101.12923377)(454.92880493,101.13924123)
\lineto(455.16880493,101.19924123)
\curveto(455.2788124,101.22923367)(455.38881229,101.24423366)(455.49880493,101.24424123)
\curveto(455.60881207,101.25423365)(455.71881196,101.26923363)(455.82880493,101.28924123)
\curveto(455.8788118,101.30923359)(455.92381175,101.31423359)(455.96380493,101.30424123)
\curveto(456.00381167,101.3042336)(456.04381163,101.30923359)(456.08380493,101.31924123)
\curveto(456.13381154,101.32923357)(456.18881149,101.32923357)(456.24880493,101.31924123)
\curveto(456.29881138,101.31923358)(456.34881133,101.32423358)(456.39880493,101.33424123)
\lineto(456.53380493,101.33424123)
\curveto(456.59381108,101.35423355)(456.66381101,101.35423355)(456.74380493,101.33424123)
\curveto(456.81381086,101.32423358)(456.8788108,101.32923357)(456.93880493,101.34924123)
\curveto(456.96881071,101.35923354)(457.00881067,101.36423354)(457.05880493,101.36424123)
\lineto(457.17880493,101.36424123)
\lineto(457.64380493,101.36424123)
\moveto(459.96880493,99.81924123)
\curveto(459.64880803,99.91923498)(459.28380839,99.97923492)(458.87380493,99.99924123)
\curveto(458.46380921,100.01923488)(458.05380962,100.02923487)(457.64380493,100.02924123)
\curveto(457.21381046,100.02923487)(456.79381088,100.01923488)(456.38380493,99.99924123)
\curveto(455.9738117,99.97923492)(455.58881209,99.93423497)(455.22880493,99.86424123)
\curveto(454.86881281,99.79423511)(454.54881313,99.68423522)(454.26880493,99.53424123)
\curveto(453.9788137,99.39423551)(453.74381393,99.1992357)(453.56380493,98.94924123)
\curveto(453.45381422,98.78923611)(453.3738143,98.60923629)(453.32380493,98.40924123)
\curveto(453.26381441,98.20923669)(453.23381444,97.96423694)(453.23380493,97.67424123)
\curveto(453.25381442,97.65423725)(453.26381441,97.61923728)(453.26380493,97.56924123)
\curveto(453.25381442,97.51923738)(453.25381442,97.47923742)(453.26380493,97.44924123)
\curveto(453.28381439,97.36923753)(453.30381437,97.29423761)(453.32380493,97.22424123)
\curveto(453.33381434,97.16423774)(453.35381432,97.0992378)(453.38380493,97.02924123)
\curveto(453.50381417,96.75923814)(453.673814,96.53923836)(453.89380493,96.36924123)
\curveto(454.10381357,96.20923869)(454.34881333,96.07423883)(454.62880493,95.96424123)
\curveto(454.73881294,95.91423899)(454.85881282,95.87423903)(454.98880493,95.84424123)
\curveto(455.10881257,95.82423908)(455.23381244,95.7992391)(455.36380493,95.76924123)
\curveto(455.41381226,95.74923915)(455.46881221,95.73923916)(455.52880493,95.73924123)
\curveto(455.5788121,95.73923916)(455.62881205,95.73423917)(455.67880493,95.72424123)
\curveto(455.76881191,95.71423919)(455.86381181,95.7042392)(455.96380493,95.69424123)
\curveto(456.05381162,95.68423922)(456.14881153,95.67423923)(456.24880493,95.66424123)
\curveto(456.32881135,95.66423924)(456.41381126,95.65923924)(456.50380493,95.64924123)
\lineto(456.74380493,95.64924123)
\lineto(456.92380493,95.64924123)
\curveto(456.95381072,95.63923926)(456.98881069,95.63423927)(457.02880493,95.63424123)
\lineto(457.16380493,95.63424123)
\lineto(457.61380493,95.63424123)
\curveto(457.69380998,95.63423927)(457.7788099,95.62923927)(457.86880493,95.61924123)
\curveto(457.94880973,95.61923928)(458.02380965,95.62923927)(458.09380493,95.64924123)
\lineto(458.36380493,95.64924123)
\curveto(458.38380929,95.64923925)(458.41380926,95.64423926)(458.45380493,95.63424123)
\curveto(458.48380919,95.63423927)(458.50880917,95.63923926)(458.52880493,95.64924123)
\curveto(458.62880905,95.65923924)(458.72880895,95.66423924)(458.82880493,95.66424123)
\curveto(458.91880876,95.67423923)(459.01880866,95.68423922)(459.12880493,95.69424123)
\curveto(459.24880843,95.72423918)(459.3738083,95.73923916)(459.50380493,95.73924123)
\curveto(459.62380805,95.74923915)(459.73880794,95.77423913)(459.84880493,95.81424123)
\curveto(460.14880753,95.89423901)(460.41380726,95.97923892)(460.64380493,96.06924123)
\curveto(460.8738068,96.16923873)(461.08880659,96.31423859)(461.28880493,96.50424123)
\curveto(461.48880619,96.71423819)(461.63880604,96.97923792)(461.73880493,97.29924123)
\curveto(461.75880592,97.33923756)(461.76880591,97.37423753)(461.76880493,97.40424123)
\curveto(461.75880592,97.44423746)(461.76380591,97.48923741)(461.78380493,97.53924123)
\curveto(461.79380588,97.57923732)(461.80380587,97.64923725)(461.81380493,97.74924123)
\curveto(461.82380585,97.85923704)(461.81880586,97.94423696)(461.79880493,98.00424123)
\curveto(461.7788059,98.07423683)(461.76880591,98.14423676)(461.76880493,98.21424123)
\curveto(461.75880592,98.28423662)(461.74380593,98.34923655)(461.72380493,98.40924123)
\curveto(461.66380601,98.60923629)(461.5788061,98.78923611)(461.46880493,98.94924123)
\curveto(461.44880623,98.97923592)(461.42880625,99.0042359)(461.40880493,99.02424123)
\lineto(461.34880493,99.08424123)
\curveto(461.32880635,99.12423578)(461.28880639,99.17423573)(461.22880493,99.23424123)
\curveto(461.08880659,99.33423557)(460.95880672,99.41923548)(460.83880493,99.48924123)
\curveto(460.71880696,99.55923534)(460.5738071,99.62923527)(460.40380493,99.69924123)
\curveto(460.33380734,99.72923517)(460.26380741,99.74923515)(460.19380493,99.75924123)
\curveto(460.12380755,99.77923512)(460.04880763,99.7992351)(459.96880493,99.81924123)
}
}
{
\newrgbcolor{curcolor}{0 0 0}
\pscustom[linestyle=none,fillstyle=solid,fillcolor=curcolor]
{
\newpath
\moveto(452.12380493,106.77385061)
\curveto(452.12381555,106.87384575)(452.13381554,106.96884566)(452.15380493,107.05885061)
\curveto(452.16381551,107.14884548)(452.19381548,107.21384541)(452.24380493,107.25385061)
\curveto(452.32381535,107.31384531)(452.42881525,107.34384528)(452.55880493,107.34385061)
\lineto(452.94880493,107.34385061)
\lineto(454.44880493,107.34385061)
\lineto(460.83880493,107.34385061)
\lineto(462.00880493,107.34385061)
\lineto(462.32380493,107.34385061)
\curveto(462.42380525,107.35384527)(462.50380517,107.33884529)(462.56380493,107.29885061)
\curveto(462.64380503,107.24884538)(462.69380498,107.17384545)(462.71380493,107.07385061)
\curveto(462.72380495,106.98384564)(462.72880495,106.87384575)(462.72880493,106.74385061)
\lineto(462.72880493,106.51885061)
\curveto(462.70880497,106.43884619)(462.69380498,106.36884626)(462.68380493,106.30885061)
\curveto(462.66380501,106.24884638)(462.62380505,106.19884643)(462.56380493,106.15885061)
\curveto(462.50380517,106.11884651)(462.42880525,106.09884653)(462.33880493,106.09885061)
\lineto(462.03880493,106.09885061)
\lineto(460.94380493,106.09885061)
\lineto(455.60380493,106.09885061)
\curveto(455.51381216,106.07884655)(455.43881224,106.06384656)(455.37880493,106.05385061)
\curveto(455.30881237,106.05384657)(455.24881243,106.0238466)(455.19880493,105.96385061)
\curveto(455.14881253,105.89384673)(455.12381255,105.80384682)(455.12380493,105.69385061)
\curveto(455.11381256,105.59384703)(455.10881257,105.48384714)(455.10880493,105.36385061)
\lineto(455.10880493,104.22385061)
\lineto(455.10880493,103.72885061)
\curveto(455.09881258,103.56884906)(455.03881264,103.45884917)(454.92880493,103.39885061)
\curveto(454.89881278,103.37884925)(454.86881281,103.36884926)(454.83880493,103.36885061)
\curveto(454.79881288,103.36884926)(454.75381292,103.36384926)(454.70380493,103.35385061)
\curveto(454.58381309,103.33384929)(454.4738132,103.33884929)(454.37380493,103.36885061)
\curveto(454.2738134,103.40884922)(454.20381347,103.46384916)(454.16380493,103.53385061)
\curveto(454.11381356,103.61384901)(454.08881359,103.73384889)(454.08880493,103.89385061)
\curveto(454.08881359,104.05384857)(454.0738136,104.18884844)(454.04380493,104.29885061)
\curveto(454.03381364,104.34884828)(454.02881365,104.40384822)(454.02880493,104.46385061)
\curveto(454.01881366,104.5238481)(454.00381367,104.58384804)(453.98380493,104.64385061)
\curveto(453.93381374,104.79384783)(453.88381379,104.93884769)(453.83380493,105.07885061)
\curveto(453.7738139,105.21884741)(453.70381397,105.35384727)(453.62380493,105.48385061)
\curveto(453.53381414,105.623847)(453.42881425,105.74384688)(453.30880493,105.84385061)
\curveto(453.18881449,105.94384668)(453.05881462,106.03884659)(452.91880493,106.12885061)
\curveto(452.81881486,106.18884644)(452.70881497,106.23384639)(452.58880493,106.26385061)
\curveto(452.46881521,106.30384632)(452.36381531,106.35384627)(452.27380493,106.41385061)
\curveto(452.21381546,106.46384616)(452.1738155,106.53384609)(452.15380493,106.62385061)
\curveto(452.14381553,106.64384598)(452.13881554,106.66884596)(452.13880493,106.69885061)
\curveto(452.13881554,106.7288459)(452.13381554,106.75384587)(452.12380493,106.77385061)
}
}
{
\newrgbcolor{curcolor}{0 0 0}
\pscustom[linestyle=none,fillstyle=solid,fillcolor=curcolor]
{
\newpath
\moveto(452.12380493,115.12345998)
\curveto(452.12381555,115.22345513)(452.13381554,115.31845503)(452.15380493,115.40845998)
\curveto(452.16381551,115.49845485)(452.19381548,115.56345479)(452.24380493,115.60345998)
\curveto(452.32381535,115.66345469)(452.42881525,115.69345466)(452.55880493,115.69345998)
\lineto(452.94880493,115.69345998)
\lineto(454.44880493,115.69345998)
\lineto(460.83880493,115.69345998)
\lineto(462.00880493,115.69345998)
\lineto(462.32380493,115.69345998)
\curveto(462.42380525,115.70345465)(462.50380517,115.68845466)(462.56380493,115.64845998)
\curveto(462.64380503,115.59845475)(462.69380498,115.52345483)(462.71380493,115.42345998)
\curveto(462.72380495,115.33345502)(462.72880495,115.22345513)(462.72880493,115.09345998)
\lineto(462.72880493,114.86845998)
\curveto(462.70880497,114.78845556)(462.69380498,114.71845563)(462.68380493,114.65845998)
\curveto(462.66380501,114.59845575)(462.62380505,114.5484558)(462.56380493,114.50845998)
\curveto(462.50380517,114.46845588)(462.42880525,114.4484559)(462.33880493,114.44845998)
\lineto(462.03880493,114.44845998)
\lineto(460.94380493,114.44845998)
\lineto(455.60380493,114.44845998)
\curveto(455.51381216,114.42845592)(455.43881224,114.41345594)(455.37880493,114.40345998)
\curveto(455.30881237,114.40345595)(455.24881243,114.37345598)(455.19880493,114.31345998)
\curveto(455.14881253,114.24345611)(455.12381255,114.1534562)(455.12380493,114.04345998)
\curveto(455.11381256,113.94345641)(455.10881257,113.83345652)(455.10880493,113.71345998)
\lineto(455.10880493,112.57345998)
\lineto(455.10880493,112.07845998)
\curveto(455.09881258,111.91845843)(455.03881264,111.80845854)(454.92880493,111.74845998)
\curveto(454.89881278,111.72845862)(454.86881281,111.71845863)(454.83880493,111.71845998)
\curveto(454.79881288,111.71845863)(454.75381292,111.71345864)(454.70380493,111.70345998)
\curveto(454.58381309,111.68345867)(454.4738132,111.68845866)(454.37380493,111.71845998)
\curveto(454.2738134,111.75845859)(454.20381347,111.81345854)(454.16380493,111.88345998)
\curveto(454.11381356,111.96345839)(454.08881359,112.08345827)(454.08880493,112.24345998)
\curveto(454.08881359,112.40345795)(454.0738136,112.53845781)(454.04380493,112.64845998)
\curveto(454.03381364,112.69845765)(454.02881365,112.7534576)(454.02880493,112.81345998)
\curveto(454.01881366,112.87345748)(454.00381367,112.93345742)(453.98380493,112.99345998)
\curveto(453.93381374,113.14345721)(453.88381379,113.28845706)(453.83380493,113.42845998)
\curveto(453.7738139,113.56845678)(453.70381397,113.70345665)(453.62380493,113.83345998)
\curveto(453.53381414,113.97345638)(453.42881425,114.09345626)(453.30880493,114.19345998)
\curveto(453.18881449,114.29345606)(453.05881462,114.38845596)(452.91880493,114.47845998)
\curveto(452.81881486,114.53845581)(452.70881497,114.58345577)(452.58880493,114.61345998)
\curveto(452.46881521,114.6534557)(452.36381531,114.70345565)(452.27380493,114.76345998)
\curveto(452.21381546,114.81345554)(452.1738155,114.88345547)(452.15380493,114.97345998)
\curveto(452.14381553,114.99345536)(452.13881554,115.01845533)(452.13880493,115.04845998)
\curveto(452.13881554,115.07845527)(452.13381554,115.10345525)(452.12380493,115.12345998)
}
}
{
\newrgbcolor{curcolor}{0 0 0}
\pscustom[linestyle=none,fillstyle=solid,fillcolor=curcolor]
{
\newpath
\moveto(472.96015137,29.18119436)
\lineto(472.96015137,30.09619436)
\curveto(472.96016206,30.19619171)(472.96016206,30.29119161)(472.96015137,30.38119436)
\curveto(472.96016206,30.47119143)(472.98016204,30.54619136)(473.02015137,30.60619436)
\curveto(473.08016194,30.69619121)(473.16016186,30.75619115)(473.26015137,30.78619436)
\curveto(473.36016166,30.82619108)(473.46516156,30.87119103)(473.57515137,30.92119436)
\curveto(473.76516126,31.0011909)(473.95516107,31.07119083)(474.14515137,31.13119436)
\curveto(474.33516069,31.2011907)(474.5251605,31.27619063)(474.71515137,31.35619436)
\curveto(474.89516013,31.42619048)(475.08015994,31.49119041)(475.27015137,31.55119436)
\curveto(475.45015957,31.61119029)(475.63015939,31.68119022)(475.81015137,31.76119436)
\curveto(475.95015907,31.82119008)(476.09515893,31.87619003)(476.24515137,31.92619436)
\curveto(476.39515863,31.97618993)(476.54015848,32.03118987)(476.68015137,32.09119436)
\curveto(477.13015789,32.27118963)(477.58515744,32.44118946)(478.04515137,32.60119436)
\curveto(478.49515653,32.76118914)(478.94515608,32.93118897)(479.39515137,33.11119436)
\curveto(479.44515558,33.13118877)(479.49515553,33.14618876)(479.54515137,33.15619436)
\lineto(479.69515137,33.21619436)
\curveto(479.91515511,33.3061886)(480.14015488,33.39118851)(480.37015137,33.47119436)
\curveto(480.59015443,33.55118835)(480.81015421,33.63618827)(481.03015137,33.72619436)
\curveto(481.1201539,33.76618814)(481.23015379,33.8061881)(481.36015137,33.84619436)
\curveto(481.48015354,33.88618802)(481.55015347,33.95118795)(481.57015137,34.04119436)
\curveto(481.58015344,34.08118782)(481.58015344,34.11118779)(481.57015137,34.13119436)
\lineto(481.51015137,34.19119436)
\curveto(481.46015356,34.24118766)(481.40515362,34.27618763)(481.34515137,34.29619436)
\curveto(481.28515374,34.32618758)(481.2201538,34.35618755)(481.15015137,34.38619436)
\lineto(480.52015137,34.62619436)
\curveto(480.30015472,34.7061872)(480.08515494,34.78618712)(479.87515137,34.86619436)
\lineto(479.72515137,34.92619436)
\lineto(479.54515137,34.98619436)
\curveto(479.35515567,35.06618684)(479.16515586,35.13618677)(478.97515137,35.19619436)
\curveto(478.77515625,35.26618664)(478.57515645,35.34118656)(478.37515137,35.42119436)
\curveto(477.79515723,35.66118624)(477.21015781,35.88118602)(476.62015137,36.08119436)
\curveto(476.03015899,36.29118561)(475.44515958,36.51618539)(474.86515137,36.75619436)
\curveto(474.66516036,36.83618507)(474.46016056,36.91118499)(474.25015137,36.98119436)
\curveto(474.04016098,37.06118484)(473.83516119,37.14118476)(473.63515137,37.22119436)
\curveto(473.55516147,37.26118464)(473.45516157,37.29618461)(473.33515137,37.32619436)
\curveto(473.21516181,37.36618454)(473.13016189,37.42118448)(473.08015137,37.49119436)
\curveto(473.04016198,37.55118435)(473.01016201,37.62618428)(472.99015137,37.71619436)
\curveto(472.97016205,37.81618409)(472.96016206,37.92618398)(472.96015137,38.04619436)
\curveto(472.95016207,38.16618374)(472.95016207,38.28618362)(472.96015137,38.40619436)
\curveto(472.96016206,38.52618338)(472.96016206,38.63618327)(472.96015137,38.73619436)
\curveto(472.96016206,38.82618308)(472.96016206,38.91618299)(472.96015137,39.00619436)
\curveto(472.96016206,39.1061828)(472.98016204,39.18118272)(473.02015137,39.23119436)
\curveto(473.07016195,39.32118258)(473.16016186,39.37118253)(473.29015137,39.38119436)
\curveto(473.4201616,39.39118251)(473.56016146,39.39618251)(473.71015137,39.39619436)
\lineto(475.36015137,39.39619436)
\lineto(481.63015137,39.39619436)
\lineto(482.89015137,39.39619436)
\curveto(483.00015202,39.39618251)(483.11015191,39.39618251)(483.22015137,39.39619436)
\curveto(483.33015169,39.4061825)(483.41515161,39.38618252)(483.47515137,39.33619436)
\curveto(483.53515149,39.3061826)(483.57515145,39.26118264)(483.59515137,39.20119436)
\curveto(483.60515142,39.14118276)(483.6201514,39.07118283)(483.64015137,38.99119436)
\lineto(483.64015137,38.75119436)
\lineto(483.64015137,38.39119436)
\curveto(483.63015139,38.28118362)(483.58515144,38.2011837)(483.50515137,38.15119436)
\curveto(483.47515155,38.13118377)(483.44515158,38.11618379)(483.41515137,38.10619436)
\curveto(483.37515165,38.1061838)(483.33015169,38.09618381)(483.28015137,38.07619436)
\lineto(483.11515137,38.07619436)
\curveto(483.05515197,38.06618384)(482.98515204,38.06118384)(482.90515137,38.06119436)
\curveto(482.8251522,38.07118383)(482.75015227,38.07618383)(482.68015137,38.07619436)
\lineto(481.84015137,38.07619436)
\lineto(477.41515137,38.07619436)
\curveto(477.16515786,38.07618383)(476.91515811,38.07618383)(476.66515137,38.07619436)
\curveto(476.40515862,38.07618383)(476.15515887,38.07118383)(475.91515137,38.06119436)
\curveto(475.81515921,38.06118384)(475.70515932,38.05618385)(475.58515137,38.04619436)
\curveto(475.46515956,38.03618387)(475.40515962,37.98118392)(475.40515137,37.88119436)
\lineto(475.42015137,37.88119436)
\curveto(475.44015958,37.81118409)(475.50515952,37.75118415)(475.61515137,37.70119436)
\curveto(475.7251593,37.66118424)(475.8201592,37.62618428)(475.90015137,37.59619436)
\curveto(476.07015895,37.52618438)(476.24515878,37.46118444)(476.42515137,37.40119436)
\curveto(476.59515843,37.34118456)(476.76515826,37.27118463)(476.93515137,37.19119436)
\curveto(476.98515804,37.17118473)(477.03015799,37.15618475)(477.07015137,37.14619436)
\curveto(477.11015791,37.13618477)(477.15515787,37.12118478)(477.20515137,37.10119436)
\curveto(477.38515764,37.02118488)(477.57015745,36.95118495)(477.76015137,36.89119436)
\curveto(477.94015708,36.84118506)(478.1201569,36.77618513)(478.30015137,36.69619436)
\curveto(478.45015657,36.62618528)(478.60515642,36.56618534)(478.76515137,36.51619436)
\curveto(478.91515611,36.46618544)(479.06515596,36.41118549)(479.21515137,36.35119436)
\curveto(479.68515534,36.15118575)(480.16015486,35.97118593)(480.64015137,35.81119436)
\curveto(481.11015391,35.65118625)(481.57515345,35.47618643)(482.03515137,35.28619436)
\curveto(482.21515281,35.2061867)(482.39515263,35.13618677)(482.57515137,35.07619436)
\curveto(482.75515227,35.01618689)(482.93515209,34.95118695)(483.11515137,34.88119436)
\curveto(483.2251518,34.83118707)(483.33015169,34.78118712)(483.43015137,34.73119436)
\curveto(483.5201515,34.69118721)(483.58515144,34.6061873)(483.62515137,34.47619436)
\curveto(483.63515139,34.45618745)(483.64015138,34.43118747)(483.64015137,34.40119436)
\curveto(483.63015139,34.38118752)(483.63015139,34.35618755)(483.64015137,34.32619436)
\curveto(483.65015137,34.29618761)(483.65515137,34.26118764)(483.65515137,34.22119436)
\curveto(483.64515138,34.18118772)(483.64015138,34.14118776)(483.64015137,34.10119436)
\lineto(483.64015137,33.80119436)
\curveto(483.64015138,33.7011882)(483.61515141,33.62118828)(483.56515137,33.56119436)
\curveto(483.51515151,33.48118842)(483.44515158,33.42118848)(483.35515137,33.38119436)
\curveto(483.25515177,33.35118855)(483.15515187,33.31118859)(483.05515137,33.26119436)
\curveto(482.85515217,33.18118872)(482.65015237,33.1011888)(482.44015137,33.02119436)
\curveto(482.2201528,32.95118895)(482.01015301,32.87618903)(481.81015137,32.79619436)
\curveto(481.63015339,32.71618919)(481.45015357,32.64618926)(481.27015137,32.58619436)
\curveto(481.08015394,32.53618937)(480.89515413,32.47118943)(480.71515137,32.39119436)
\curveto(480.15515487,32.16118974)(479.59015543,31.94618996)(479.02015137,31.74619436)
\curveto(478.45015657,31.54619036)(477.88515714,31.33119057)(477.32515137,31.10119436)
\lineto(476.69515137,30.86119436)
\curveto(476.47515855,30.79119111)(476.26515876,30.71619119)(476.06515137,30.63619436)
\curveto(475.95515907,30.58619132)(475.85015917,30.54119136)(475.75015137,30.50119436)
\curveto(475.64015938,30.47119143)(475.54515948,30.42119148)(475.46515137,30.35119436)
\curveto(475.44515958,30.34119156)(475.43515959,30.33119157)(475.43515137,30.32119436)
\lineto(475.40515137,30.29119436)
\lineto(475.40515137,30.21619436)
\lineto(475.43515137,30.18619436)
\curveto(475.43515959,30.17619173)(475.44015958,30.16619174)(475.45015137,30.15619436)
\curveto(475.50015952,30.13619177)(475.55515947,30.12619178)(475.61515137,30.12619436)
\curveto(475.67515935,30.12619178)(475.73515929,30.11619179)(475.79515137,30.09619436)
\lineto(475.96015137,30.09619436)
\curveto(476.020159,30.07619183)(476.08515894,30.07119183)(476.15515137,30.08119436)
\curveto(476.2251588,30.09119181)(476.29515873,30.09619181)(476.36515137,30.09619436)
\lineto(477.17515137,30.09619436)
\lineto(481.73515137,30.09619436)
\lineto(482.92015137,30.09619436)
\curveto(483.03015199,30.09619181)(483.14015188,30.09119181)(483.25015137,30.08119436)
\curveto(483.36015166,30.08119182)(483.44515158,30.05619185)(483.50515137,30.00619436)
\curveto(483.58515144,29.95619195)(483.63015139,29.86619204)(483.64015137,29.73619436)
\lineto(483.64015137,29.34619436)
\lineto(483.64015137,29.15119436)
\curveto(483.64015138,29.1011928)(483.63015139,29.05119285)(483.61015137,29.00119436)
\curveto(483.57015145,28.87119303)(483.48515154,28.79619311)(483.35515137,28.77619436)
\curveto(483.2251518,28.76619314)(483.07515195,28.76119314)(482.90515137,28.76119436)
\lineto(481.16515137,28.76119436)
\lineto(475.16515137,28.76119436)
\lineto(473.75515137,28.76119436)
\curveto(473.64516138,28.76119314)(473.53016149,28.75619315)(473.41015137,28.74619436)
\curveto(473.29016173,28.74619316)(473.19516183,28.77119313)(473.12515137,28.82119436)
\curveto(473.06516196,28.86119304)(473.01516201,28.93619297)(472.97515137,29.04619436)
\curveto(472.96516206,29.06619284)(472.96516206,29.08619282)(472.97515137,29.10619436)
\curveto(472.97516205,29.13619277)(472.97016205,29.16119274)(472.96015137,29.18119436)
}
}
{
\newrgbcolor{curcolor}{0 0 0}
\pscustom[linestyle=none,fillstyle=solid,fillcolor=curcolor]
{
\newpath
\moveto(483.08515137,48.38330373)
\curveto(483.24515178,48.4132959)(483.38015164,48.39829592)(483.49015137,48.33830373)
\curveto(483.59015143,48.27829604)(483.66515136,48.19829612)(483.71515137,48.09830373)
\curveto(483.73515129,48.04829627)(483.74515128,47.99329632)(483.74515137,47.93330373)
\curveto(483.74515128,47.88329643)(483.75515127,47.82829649)(483.77515137,47.76830373)
\curveto(483.8251512,47.54829677)(483.81015121,47.32829699)(483.73015137,47.10830373)
\curveto(483.66015136,46.89829742)(483.57015145,46.75329756)(483.46015137,46.67330373)
\curveto(483.39015163,46.62329769)(483.31015171,46.57829774)(483.22015137,46.53830373)
\curveto(483.1201519,46.49829782)(483.04015198,46.44829787)(482.98015137,46.38830373)
\curveto(482.96015206,46.36829795)(482.94015208,46.34329797)(482.92015137,46.31330373)
\curveto(482.90015212,46.29329802)(482.89515213,46.26329805)(482.90515137,46.22330373)
\curveto(482.93515209,46.1132982)(482.99015203,46.00829831)(483.07015137,45.90830373)
\curveto(483.15015187,45.8182985)(483.2201518,45.72829859)(483.28015137,45.63830373)
\curveto(483.36015166,45.50829881)(483.43515159,45.36829895)(483.50515137,45.21830373)
\curveto(483.56515146,45.06829925)(483.6201514,44.90829941)(483.67015137,44.73830373)
\curveto(483.70015132,44.63829968)(483.7201513,44.52829979)(483.73015137,44.40830373)
\curveto(483.74015128,44.29830002)(483.75515127,44.18830013)(483.77515137,44.07830373)
\curveto(483.78515124,44.02830029)(483.79015123,43.98330033)(483.79015137,43.94330373)
\lineto(483.79015137,43.83830373)
\curveto(483.81015121,43.72830059)(483.81015121,43.62330069)(483.79015137,43.52330373)
\lineto(483.79015137,43.38830373)
\curveto(483.78015124,43.33830098)(483.77515125,43.28830103)(483.77515137,43.23830373)
\curveto(483.77515125,43.18830113)(483.76515126,43.14330117)(483.74515137,43.10330373)
\curveto(483.73515129,43.06330125)(483.73015129,43.02830129)(483.73015137,42.99830373)
\curveto(483.74015128,42.97830134)(483.74015128,42.95330136)(483.73015137,42.92330373)
\lineto(483.67015137,42.68330373)
\curveto(483.66015136,42.60330171)(483.64015138,42.52830179)(483.61015137,42.45830373)
\curveto(483.48015154,42.15830216)(483.33515169,41.9133024)(483.17515137,41.72330373)
\curveto(483.00515202,41.54330277)(482.77015225,41.39330292)(482.47015137,41.27330373)
\curveto(482.25015277,41.18330313)(481.98515304,41.13830318)(481.67515137,41.13830373)
\lineto(481.36015137,41.13830373)
\curveto(481.31015371,41.14830317)(481.26015376,41.15330316)(481.21015137,41.15330373)
\lineto(481.03015137,41.18330373)
\lineto(480.70015137,41.30330373)
\curveto(480.59015443,41.34330297)(480.49015453,41.39330292)(480.40015137,41.45330373)
\curveto(480.11015491,41.63330268)(479.89515513,41.87830244)(479.75515137,42.18830373)
\curveto(479.61515541,42.49830182)(479.49015553,42.83830148)(479.38015137,43.20830373)
\curveto(479.34015568,43.34830097)(479.31015571,43.49330082)(479.29015137,43.64330373)
\curveto(479.27015575,43.79330052)(479.24515578,43.94330037)(479.21515137,44.09330373)
\curveto(479.19515583,44.16330015)(479.18515584,44.22830009)(479.18515137,44.28830373)
\curveto(479.18515584,44.35829996)(479.17515585,44.43329988)(479.15515137,44.51330373)
\curveto(479.13515589,44.58329973)(479.1251559,44.65329966)(479.12515137,44.72330373)
\curveto(479.11515591,44.79329952)(479.10015592,44.86829945)(479.08015137,44.94830373)
\curveto(479.020156,45.19829912)(478.97015605,45.43329888)(478.93015137,45.65330373)
\curveto(478.88015614,45.87329844)(478.76515626,46.04829827)(478.58515137,46.17830373)
\curveto(478.50515652,46.23829808)(478.40515662,46.28829803)(478.28515137,46.32830373)
\curveto(478.15515687,46.36829795)(478.01515701,46.36829795)(477.86515137,46.32830373)
\curveto(477.6251574,46.26829805)(477.43515759,46.17829814)(477.29515137,46.05830373)
\curveto(477.15515787,45.94829837)(477.04515798,45.78829853)(476.96515137,45.57830373)
\curveto(476.91515811,45.45829886)(476.88015814,45.313299)(476.86015137,45.14330373)
\curveto(476.84015818,44.98329933)(476.83015819,44.8132995)(476.83015137,44.63330373)
\curveto(476.83015819,44.45329986)(476.84015818,44.27830004)(476.86015137,44.10830373)
\curveto(476.88015814,43.93830038)(476.91015811,43.79330052)(476.95015137,43.67330373)
\curveto(477.01015801,43.50330081)(477.09515793,43.33830098)(477.20515137,43.17830373)
\curveto(477.26515776,43.09830122)(477.34515768,43.02330129)(477.44515137,42.95330373)
\curveto(477.53515749,42.89330142)(477.63515739,42.83830148)(477.74515137,42.78830373)
\curveto(477.8251572,42.75830156)(477.91015711,42.72830159)(478.00015137,42.69830373)
\curveto(478.09015693,42.67830164)(478.16015686,42.63330168)(478.21015137,42.56330373)
\curveto(478.24015678,42.52330179)(478.26515676,42.45330186)(478.28515137,42.35330373)
\curveto(478.29515673,42.26330205)(478.30015672,42.16830215)(478.30015137,42.06830373)
\curveto(478.30015672,41.96830235)(478.29515673,41.86830245)(478.28515137,41.76830373)
\curveto(478.26515676,41.67830264)(478.24015678,41.6133027)(478.21015137,41.57330373)
\curveto(478.18015684,41.53330278)(478.13015689,41.50330281)(478.06015137,41.48330373)
\curveto(477.99015703,41.46330285)(477.91515711,41.46330285)(477.83515137,41.48330373)
\curveto(477.70515732,41.5133028)(477.58515744,41.54330277)(477.47515137,41.57330373)
\curveto(477.35515767,41.6133027)(477.24015778,41.65830266)(477.13015137,41.70830373)
\curveto(476.78015824,41.89830242)(476.51015851,42.13830218)(476.32015137,42.42830373)
\curveto(476.1201589,42.7183016)(475.96015906,43.07830124)(475.84015137,43.50830373)
\curveto(475.8201592,43.60830071)(475.80515922,43.70830061)(475.79515137,43.80830373)
\curveto(475.78515924,43.9183004)(475.77015925,44.02830029)(475.75015137,44.13830373)
\curveto(475.74015928,44.17830014)(475.74015928,44.24330007)(475.75015137,44.33330373)
\curveto(475.75015927,44.42329989)(475.74015928,44.47829984)(475.72015137,44.49830373)
\curveto(475.71015931,45.19829912)(475.79015923,45.80829851)(475.96015137,46.32830373)
\curveto(476.13015889,46.84829747)(476.45515857,47.2132971)(476.93515137,47.42330373)
\curveto(477.13515789,47.5132968)(477.37015765,47.56329675)(477.64015137,47.57330373)
\curveto(477.90015712,47.59329672)(478.17515685,47.60329671)(478.46515137,47.60330373)
\lineto(481.78015137,47.60330373)
\curveto(481.9201531,47.60329671)(482.05515297,47.60829671)(482.18515137,47.61830373)
\curveto(482.31515271,47.62829669)(482.4201526,47.65829666)(482.50015137,47.70830373)
\curveto(482.57015245,47.75829656)(482.6201524,47.82329649)(482.65015137,47.90330373)
\curveto(482.69015233,47.99329632)(482.7201523,48.07829624)(482.74015137,48.15830373)
\curveto(482.75015227,48.23829608)(482.79515223,48.29829602)(482.87515137,48.33830373)
\curveto(482.90515212,48.35829596)(482.93515209,48.36829595)(482.96515137,48.36830373)
\curveto(482.99515203,48.36829595)(483.03515199,48.37329594)(483.08515137,48.38330373)
\moveto(481.42015137,46.23830373)
\curveto(481.28015374,46.29829802)(481.1201539,46.32829799)(480.94015137,46.32830373)
\curveto(480.75015427,46.33829798)(480.55515447,46.34329797)(480.35515137,46.34330373)
\curveto(480.24515478,46.34329797)(480.14515488,46.33829798)(480.05515137,46.32830373)
\curveto(479.96515506,46.318298)(479.89515513,46.27829804)(479.84515137,46.20830373)
\curveto(479.8251552,46.17829814)(479.81515521,46.10829821)(479.81515137,45.99830373)
\curveto(479.83515519,45.97829834)(479.84515518,45.94329837)(479.84515137,45.89330373)
\curveto(479.84515518,45.84329847)(479.85515517,45.79829852)(479.87515137,45.75830373)
\curveto(479.89515513,45.67829864)(479.91515511,45.58829873)(479.93515137,45.48830373)
\lineto(479.99515137,45.18830373)
\curveto(479.99515503,45.15829916)(480.00015502,45.12329919)(480.01015137,45.08330373)
\lineto(480.01015137,44.97830373)
\curveto(480.05015497,44.82829949)(480.07515495,44.66329965)(480.08515137,44.48330373)
\curveto(480.08515494,44.3133)(480.10515492,44.15330016)(480.14515137,44.00330373)
\curveto(480.16515486,43.92330039)(480.18515484,43.84830047)(480.20515137,43.77830373)
\curveto(480.21515481,43.7183006)(480.23015479,43.64830067)(480.25015137,43.56830373)
\curveto(480.30015472,43.40830091)(480.36515466,43.25830106)(480.44515137,43.11830373)
\curveto(480.51515451,42.97830134)(480.60515442,42.85830146)(480.71515137,42.75830373)
\curveto(480.8251542,42.65830166)(480.96015406,42.58330173)(481.12015137,42.53330373)
\curveto(481.27015375,42.48330183)(481.45515357,42.46330185)(481.67515137,42.47330373)
\curveto(481.77515325,42.47330184)(481.87015315,42.48830183)(481.96015137,42.51830373)
\curveto(482.04015298,42.55830176)(482.11515291,42.60330171)(482.18515137,42.65330373)
\curveto(482.29515273,42.73330158)(482.39015263,42.83830148)(482.47015137,42.96830373)
\curveto(482.54015248,43.09830122)(482.60015242,43.23830108)(482.65015137,43.38830373)
\curveto(482.66015236,43.43830088)(482.66515236,43.48830083)(482.66515137,43.53830373)
\curveto(482.66515236,43.58830073)(482.67015235,43.63830068)(482.68015137,43.68830373)
\curveto(482.70015232,43.75830056)(482.71515231,43.84330047)(482.72515137,43.94330373)
\curveto(482.7251523,44.05330026)(482.71515231,44.14330017)(482.69515137,44.21330373)
\curveto(482.67515235,44.27330004)(482.67015235,44.33329998)(482.68015137,44.39330373)
\curveto(482.68015234,44.45329986)(482.67015235,44.5132998)(482.65015137,44.57330373)
\curveto(482.63015239,44.65329966)(482.61515241,44.72829959)(482.60515137,44.79830373)
\curveto(482.59515243,44.87829944)(482.57515245,44.95329936)(482.54515137,45.02330373)
\curveto(482.4251526,45.313299)(482.28015274,45.55829876)(482.11015137,45.75830373)
\curveto(481.94015308,45.96829835)(481.71015331,46.12829819)(481.42015137,46.23830373)
}
}
{
\newrgbcolor{curcolor}{0 0 0}
\pscustom[linestyle=none,fillstyle=solid,fillcolor=curcolor]
{
\newpath
\moveto(475.91515137,49.26994436)
\lineto(475.91515137,49.71994436)
\curveto(475.90515912,49.88994311)(475.9251591,50.01494298)(475.97515137,50.09494436)
\curveto(476.025159,50.17494282)(476.09015893,50.22994277)(476.17015137,50.25994436)
\curveto(476.25015877,50.2999427)(476.33515869,50.33994266)(476.42515137,50.37994436)
\curveto(476.55515847,50.42994257)(476.68515834,50.47494252)(476.81515137,50.51494436)
\curveto(476.94515808,50.55494244)(477.07515795,50.5999424)(477.20515137,50.64994436)
\curveto(477.3251577,50.6999423)(477.45015757,50.74494225)(477.58015137,50.78494436)
\curveto(477.70015732,50.82494217)(477.8201572,50.86994213)(477.94015137,50.91994436)
\curveto(478.05015697,50.96994203)(478.16515686,51.00994199)(478.28515137,51.03994436)
\curveto(478.40515662,51.06994193)(478.5251565,51.10994189)(478.64515137,51.15994436)
\curveto(478.93515609,51.27994172)(479.23515579,51.38994161)(479.54515137,51.48994436)
\curveto(479.85515517,51.58994141)(480.15515487,51.6999413)(480.44515137,51.81994436)
\curveto(480.48515454,51.83994116)(480.5251545,51.84994115)(480.56515137,51.84994436)
\curveto(480.59515443,51.84994115)(480.6251544,51.85994114)(480.65515137,51.87994436)
\curveto(480.79515423,51.93994106)(480.94015408,51.994941)(481.09015137,52.04494436)
\lineto(481.51015137,52.19494436)
\curveto(481.58015344,52.22494077)(481.65515337,52.25494074)(481.73515137,52.28494436)
\curveto(481.80515322,52.31494068)(481.85015317,52.36494063)(481.87015137,52.43494436)
\curveto(481.90015312,52.51494048)(481.87515315,52.57494042)(481.79515137,52.61494436)
\curveto(481.70515332,52.66494033)(481.63515339,52.6999403)(481.58515137,52.71994436)
\curveto(481.41515361,52.7999402)(481.23515379,52.86494013)(481.04515137,52.91494436)
\curveto(480.85515417,52.96494003)(480.67015435,53.02493997)(480.49015137,53.09494436)
\curveto(480.26015476,53.18493981)(480.03015499,53.26493973)(479.80015137,53.33494436)
\curveto(479.56015546,53.40493959)(479.33015569,53.48993951)(479.11015137,53.58994436)
\curveto(479.06015596,53.5999394)(478.99515603,53.61493938)(478.91515137,53.63494436)
\curveto(478.8251562,53.67493932)(478.73515629,53.70993929)(478.64515137,53.73994436)
\curveto(478.54515648,53.76993923)(478.45515657,53.7999392)(478.37515137,53.82994436)
\curveto(478.3251567,53.84993915)(478.28015674,53.86493913)(478.24015137,53.87494436)
\curveto(478.20015682,53.88493911)(478.15515687,53.8999391)(478.10515137,53.91994436)
\curveto(477.98515704,53.96993903)(477.86515716,54.00993899)(477.74515137,54.03994436)
\curveto(477.61515741,54.07993892)(477.49015753,54.12493887)(477.37015137,54.17494436)
\curveto(477.3201577,54.1949388)(477.27515775,54.20993879)(477.23515137,54.21994436)
\curveto(477.19515783,54.22993877)(477.15015787,54.24493875)(477.10015137,54.26494436)
\curveto(477.01015801,54.30493869)(476.9201581,54.33993866)(476.83015137,54.36994436)
\curveto(476.73015829,54.3999386)(476.63515839,54.42993857)(476.54515137,54.45994436)
\curveto(476.46515856,54.48993851)(476.38515864,54.51493848)(476.30515137,54.53494436)
\curveto(476.21515881,54.56493843)(476.14015888,54.60493839)(476.08015137,54.65494436)
\curveto(475.99015903,54.72493827)(475.94015908,54.81993818)(475.93015137,54.93994436)
\curveto(475.9201591,55.06993793)(475.91515911,55.20993779)(475.91515137,55.35994436)
\curveto(475.91515911,55.43993756)(475.9201591,55.51493748)(475.93015137,55.58494436)
\curveto(475.93015909,55.66493733)(475.94515908,55.72993727)(475.97515137,55.77994436)
\curveto(476.03515899,55.86993713)(476.13015889,55.8949371)(476.26015137,55.85494436)
\curveto(476.39015863,55.81493718)(476.49015853,55.77993722)(476.56015137,55.74994436)
\lineto(476.62015137,55.71994436)
\curveto(476.64015838,55.71993728)(476.66015836,55.71493728)(476.68015137,55.70494436)
\curveto(476.96015806,55.5949374)(477.24515778,55.48493751)(477.53515137,55.37494436)
\lineto(478.37515137,55.04494436)
\curveto(478.45515657,55.01493798)(478.53015649,54.98993801)(478.60015137,54.96994436)
\curveto(478.66015636,54.94993805)(478.7251563,54.92493807)(478.79515137,54.89494436)
\curveto(478.99515603,54.81493818)(479.20015582,54.73493826)(479.41015137,54.65494436)
\curveto(479.61015541,54.58493841)(479.81015521,54.50993849)(480.01015137,54.42994436)
\curveto(480.70015432,54.13993886)(481.39515363,53.86993913)(482.09515137,53.61994436)
\curveto(482.79515223,53.36993963)(483.49015153,53.0999399)(484.18015137,52.80994436)
\lineto(484.33015137,52.74994436)
\curveto(484.39015063,52.73994026)(484.45015057,52.72494027)(484.51015137,52.70494436)
\curveto(484.88015014,52.54494045)(485.24514978,52.37494062)(485.60515137,52.19494436)
\curveto(485.97514905,52.01494098)(486.26014876,51.76494123)(486.46015137,51.44494436)
\curveto(486.5201485,51.33494166)(486.56514846,51.22494177)(486.59515137,51.11494436)
\curveto(486.63514839,51.00494199)(486.67014835,50.87994212)(486.70015137,50.73994436)
\curveto(486.7201483,50.68994231)(486.7251483,50.63494236)(486.71515137,50.57494436)
\curveto(486.70514832,50.52494247)(486.70514832,50.46994253)(486.71515137,50.40994436)
\curveto(486.73514829,50.32994267)(486.73514829,50.24994275)(486.71515137,50.16994436)
\curveto(486.70514832,50.12994287)(486.70014832,50.07994292)(486.70015137,50.01994436)
\lineto(486.64015137,49.77994436)
\curveto(486.6201484,49.70994329)(486.58014844,49.65494334)(486.52015137,49.61494436)
\curveto(486.46014856,49.56494343)(486.38514864,49.53494346)(486.29515137,49.52494436)
\lineto(486.02515137,49.52494436)
\lineto(485.81515137,49.52494436)
\curveto(485.75514927,49.53494346)(485.70514932,49.55494344)(485.66515137,49.58494436)
\curveto(485.55514947,49.65494334)(485.5251495,49.77494322)(485.57515137,49.94494436)
\curveto(485.59514943,50.05494294)(485.60514942,50.17494282)(485.60515137,50.30494436)
\curveto(485.60514942,50.43494256)(485.58514944,50.54994245)(485.54515137,50.64994436)
\curveto(485.49514953,50.7999422)(485.4201496,50.91994208)(485.32015137,51.00994436)
\curveto(485.2201498,51.10994189)(485.10514992,51.1949418)(484.97515137,51.26494436)
\curveto(484.85515017,51.33494166)(484.7251503,51.3949416)(484.58515137,51.44494436)
\lineto(484.16515137,51.62494436)
\curveto(484.07515095,51.66494133)(483.96515106,51.70494129)(483.83515137,51.74494436)
\curveto(483.70515132,51.7949412)(483.57015145,51.7999412)(483.43015137,51.75994436)
\curveto(483.27015175,51.70994129)(483.1201519,51.65494134)(482.98015137,51.59494436)
\curveto(482.84015218,51.54494145)(482.70015232,51.48994151)(482.56015137,51.42994436)
\curveto(482.35015267,51.33994166)(482.14015288,51.25494174)(481.93015137,51.17494436)
\curveto(481.7201533,51.0949419)(481.51515351,51.01494198)(481.31515137,50.93494436)
\curveto(481.17515385,50.87494212)(481.04015398,50.81994218)(480.91015137,50.76994436)
\curveto(480.78015424,50.71994228)(480.64515438,50.66994233)(480.50515137,50.61994436)
\lineto(479.18515137,50.07994436)
\curveto(478.74515628,49.90994309)(478.30515672,49.73494326)(477.86515137,49.55494436)
\curveto(477.63515739,49.45494354)(477.41515761,49.36494363)(477.20515137,49.28494436)
\curveto(476.98515804,49.20494379)(476.76515826,49.11994388)(476.54515137,49.02994436)
\curveto(476.48515854,49.00994399)(476.40515862,48.97994402)(476.30515137,48.93994436)
\curveto(476.19515883,48.8999441)(476.10515892,48.90494409)(476.03515137,48.95494436)
\curveto(475.98515904,48.98494401)(475.95015907,49.04494395)(475.93015137,49.13494436)
\curveto(475.9201591,49.15494384)(475.9201591,49.17494382)(475.93015137,49.19494436)
\curveto(475.93015909,49.22494377)(475.9251591,49.24994375)(475.91515137,49.26994436)
}
}
{
\newrgbcolor{curcolor}{0 0 0}
\pscustom[linestyle=none,fillstyle=solid,fillcolor=curcolor]
{
}
}
{
\newrgbcolor{curcolor}{0 0 0}
\pscustom[linestyle=none,fillstyle=solid,fillcolor=curcolor]
{
\newpath
\moveto(478.55515137,67.99510061)
\lineto(478.81015137,67.99510061)
\curveto(478.89015613,68.0050929)(478.96515606,68.00009291)(479.03515137,67.98010061)
\lineto(479.27515137,67.98010061)
\lineto(479.44015137,67.98010061)
\curveto(479.54015548,67.96009295)(479.64515538,67.95009296)(479.75515137,67.95010061)
\curveto(479.85515517,67.95009296)(479.95515507,67.94009297)(480.05515137,67.92010061)
\lineto(480.20515137,67.92010061)
\curveto(480.34515468,67.89009302)(480.48515454,67.87009304)(480.62515137,67.86010061)
\curveto(480.75515427,67.85009306)(480.88515414,67.82509308)(481.01515137,67.78510061)
\curveto(481.09515393,67.76509314)(481.18015384,67.74509316)(481.27015137,67.72510061)
\lineto(481.51015137,67.66510061)
\lineto(481.81015137,67.54510061)
\curveto(481.90015312,67.51509339)(481.99015303,67.48009343)(482.08015137,67.44010061)
\curveto(482.30015272,67.34009357)(482.51515251,67.2050937)(482.72515137,67.03510061)
\curveto(482.93515209,66.87509403)(483.10515192,66.70009421)(483.23515137,66.51010061)
\curveto(483.27515175,66.46009445)(483.31515171,66.40009451)(483.35515137,66.33010061)
\curveto(483.38515164,66.27009464)(483.4201516,66.2100947)(483.46015137,66.15010061)
\curveto(483.51015151,66.07009484)(483.55015147,65.97509493)(483.58015137,65.86510061)
\curveto(483.61015141,65.75509515)(483.64015138,65.65009526)(483.67015137,65.55010061)
\curveto(483.71015131,65.44009547)(483.73515129,65.33009558)(483.74515137,65.22010061)
\curveto(483.75515127,65.1100958)(483.77015125,64.99509591)(483.79015137,64.87510061)
\curveto(483.80015122,64.83509607)(483.80015122,64.79009612)(483.79015137,64.74010061)
\curveto(483.79015123,64.70009621)(483.79515123,64.66009625)(483.80515137,64.62010061)
\curveto(483.81515121,64.58009633)(483.8201512,64.52509638)(483.82015137,64.45510061)
\curveto(483.8201512,64.38509652)(483.81515121,64.33509657)(483.80515137,64.30510061)
\curveto(483.78515124,64.25509665)(483.78015124,64.2100967)(483.79015137,64.17010061)
\curveto(483.80015122,64.13009678)(483.80015122,64.09509681)(483.79015137,64.06510061)
\lineto(483.79015137,63.97510061)
\curveto(483.77015125,63.91509699)(483.75515127,63.85009706)(483.74515137,63.78010061)
\curveto(483.74515128,63.72009719)(483.74015128,63.65509725)(483.73015137,63.58510061)
\curveto(483.68015134,63.41509749)(483.63015139,63.25509765)(483.58015137,63.10510061)
\curveto(483.53015149,62.95509795)(483.46515156,62.8100981)(483.38515137,62.67010061)
\curveto(483.34515168,62.62009829)(483.31515171,62.56509834)(483.29515137,62.50510061)
\curveto(483.26515176,62.45509845)(483.23015179,62.4050985)(483.19015137,62.35510061)
\curveto(483.01015201,62.11509879)(482.79015223,61.91509899)(482.53015137,61.75510061)
\curveto(482.27015275,61.59509931)(481.98515304,61.45509945)(481.67515137,61.33510061)
\curveto(481.53515349,61.27509963)(481.39515363,61.23009968)(481.25515137,61.20010061)
\curveto(481.10515392,61.17009974)(480.95015407,61.13509977)(480.79015137,61.09510061)
\curveto(480.68015434,61.07509983)(480.57015445,61.06009985)(480.46015137,61.05010061)
\curveto(480.35015467,61.04009987)(480.24015478,61.02509988)(480.13015137,61.00510061)
\curveto(480.09015493,60.99509991)(480.05015497,60.99009992)(480.01015137,60.99010061)
\curveto(479.97015505,61.00009991)(479.93015509,61.00009991)(479.89015137,60.99010061)
\curveto(479.84015518,60.98009993)(479.79015523,60.97509993)(479.74015137,60.97510061)
\lineto(479.57515137,60.97510061)
\curveto(479.5251555,60.95509995)(479.47515555,60.95009996)(479.42515137,60.96010061)
\curveto(479.36515566,60.97009994)(479.31015571,60.97009994)(479.26015137,60.96010061)
\curveto(479.2201558,60.95009996)(479.17515585,60.95009996)(479.12515137,60.96010061)
\curveto(479.07515595,60.97009994)(479.025156,60.96509994)(478.97515137,60.94510061)
\curveto(478.90515612,60.92509998)(478.83015619,60.92009999)(478.75015137,60.93010061)
\curveto(478.66015636,60.94009997)(478.57515645,60.94509996)(478.49515137,60.94510061)
\curveto(478.40515662,60.94509996)(478.30515672,60.94009997)(478.19515137,60.93010061)
\curveto(478.07515695,60.92009999)(477.97515705,60.92509998)(477.89515137,60.94510061)
\lineto(477.61015137,60.94510061)
\lineto(476.98015137,60.99010061)
\curveto(476.88015814,61.00009991)(476.78515824,61.0100999)(476.69515137,61.02010061)
\lineto(476.39515137,61.05010061)
\curveto(476.34515868,61.07009984)(476.29515873,61.07509983)(476.24515137,61.06510061)
\curveto(476.18515884,61.06509984)(476.13015889,61.07509983)(476.08015137,61.09510061)
\curveto(475.91015911,61.14509976)(475.74515928,61.18509972)(475.58515137,61.21510061)
\curveto(475.41515961,61.24509966)(475.25515977,61.29509961)(475.10515137,61.36510061)
\curveto(474.64516038,61.55509935)(474.27016075,61.77509913)(473.98015137,62.02510061)
\curveto(473.69016133,62.28509862)(473.44516158,62.64509826)(473.24515137,63.10510061)
\curveto(473.19516183,63.23509767)(473.16016186,63.36509754)(473.14015137,63.49510061)
\curveto(473.1201619,63.63509727)(473.09516193,63.77509713)(473.06515137,63.91510061)
\curveto(473.05516197,63.98509692)(473.05016197,64.05009686)(473.05015137,64.11010061)
\curveto(473.05016197,64.17009674)(473.04516198,64.23509667)(473.03515137,64.30510061)
\curveto(473.01516201,65.13509577)(473.16516186,65.8050951)(473.48515137,66.31510061)
\curveto(473.79516123,66.82509408)(474.23516079,67.2050937)(474.80515137,67.45510061)
\curveto(474.9251601,67.5050934)(475.05015997,67.55009336)(475.18015137,67.59010061)
\curveto(475.31015971,67.63009328)(475.44515958,67.67509323)(475.58515137,67.72510061)
\curveto(475.66515936,67.74509316)(475.75015927,67.76009315)(475.84015137,67.77010061)
\lineto(476.08015137,67.83010061)
\curveto(476.19015883,67.86009305)(476.30015872,67.87509303)(476.41015137,67.87510061)
\curveto(476.5201585,67.88509302)(476.63015839,67.90009301)(476.74015137,67.92010061)
\curveto(476.79015823,67.94009297)(476.83515819,67.94509296)(476.87515137,67.93510061)
\curveto(476.91515811,67.93509297)(476.95515807,67.94009297)(476.99515137,67.95010061)
\curveto(477.04515798,67.96009295)(477.10015792,67.96009295)(477.16015137,67.95010061)
\curveto(477.21015781,67.95009296)(477.26015776,67.95509295)(477.31015137,67.96510061)
\lineto(477.44515137,67.96510061)
\curveto(477.50515752,67.98509292)(477.57515745,67.98509292)(477.65515137,67.96510061)
\curveto(477.7251573,67.95509295)(477.79015723,67.96009295)(477.85015137,67.98010061)
\curveto(477.88015714,67.99009292)(477.9201571,67.99509291)(477.97015137,67.99510061)
\lineto(478.09015137,67.99510061)
\lineto(478.55515137,67.99510061)
\moveto(480.88015137,66.45010061)
\curveto(480.56015446,66.55009436)(480.19515483,66.6100943)(479.78515137,66.63010061)
\curveto(479.37515565,66.65009426)(478.96515606,66.66009425)(478.55515137,66.66010061)
\curveto(478.1251569,66.66009425)(477.70515732,66.65009426)(477.29515137,66.63010061)
\curveto(476.88515814,66.6100943)(476.50015852,66.56509434)(476.14015137,66.49510061)
\curveto(475.78015924,66.42509448)(475.46015956,66.31509459)(475.18015137,66.16510061)
\curveto(474.89016013,66.02509488)(474.65516037,65.83009508)(474.47515137,65.58010061)
\curveto(474.36516066,65.42009549)(474.28516074,65.24009567)(474.23515137,65.04010061)
\curveto(474.17516085,64.84009607)(474.14516088,64.59509631)(474.14515137,64.30510061)
\curveto(474.16516086,64.28509662)(474.17516085,64.25009666)(474.17515137,64.20010061)
\curveto(474.16516086,64.15009676)(474.16516086,64.1100968)(474.17515137,64.08010061)
\curveto(474.19516083,64.00009691)(474.21516081,63.92509698)(474.23515137,63.85510061)
\curveto(474.24516078,63.79509711)(474.26516076,63.73009718)(474.29515137,63.66010061)
\curveto(474.41516061,63.39009752)(474.58516044,63.17009774)(474.80515137,63.00010061)
\curveto(475.01516001,62.84009807)(475.26015976,62.7050982)(475.54015137,62.59510061)
\curveto(475.65015937,62.54509836)(475.77015925,62.5050984)(475.90015137,62.47510061)
\curveto(476.020159,62.45509845)(476.14515888,62.43009848)(476.27515137,62.40010061)
\curveto(476.3251587,62.38009853)(476.38015864,62.37009854)(476.44015137,62.37010061)
\curveto(476.49015853,62.37009854)(476.54015848,62.36509854)(476.59015137,62.35510061)
\curveto(476.68015834,62.34509856)(476.77515825,62.33509857)(476.87515137,62.32510061)
\curveto(476.96515806,62.31509859)(477.06015796,62.3050986)(477.16015137,62.29510061)
\curveto(477.24015778,62.29509861)(477.3251577,62.29009862)(477.41515137,62.28010061)
\lineto(477.65515137,62.28010061)
\lineto(477.83515137,62.28010061)
\curveto(477.86515716,62.27009864)(477.90015712,62.26509864)(477.94015137,62.26510061)
\lineto(478.07515137,62.26510061)
\lineto(478.52515137,62.26510061)
\curveto(478.60515642,62.26509864)(478.69015633,62.26009865)(478.78015137,62.25010061)
\curveto(478.86015616,62.25009866)(478.93515609,62.26009865)(479.00515137,62.28010061)
\lineto(479.27515137,62.28010061)
\curveto(479.29515573,62.28009863)(479.3251557,62.27509863)(479.36515137,62.26510061)
\curveto(479.39515563,62.26509864)(479.4201556,62.27009864)(479.44015137,62.28010061)
\curveto(479.54015548,62.29009862)(479.64015538,62.29509861)(479.74015137,62.29510061)
\curveto(479.83015519,62.3050986)(479.93015509,62.31509859)(480.04015137,62.32510061)
\curveto(480.16015486,62.35509855)(480.28515474,62.37009854)(480.41515137,62.37010061)
\curveto(480.53515449,62.38009853)(480.65015437,62.4050985)(480.76015137,62.44510061)
\curveto(481.06015396,62.52509838)(481.3251537,62.6100983)(481.55515137,62.70010061)
\curveto(481.78515324,62.80009811)(482.00015302,62.94509796)(482.20015137,63.13510061)
\curveto(482.40015262,63.34509756)(482.55015247,63.6100973)(482.65015137,63.93010061)
\curveto(482.67015235,63.97009694)(482.68015234,64.0050969)(482.68015137,64.03510061)
\curveto(482.67015235,64.07509683)(482.67515235,64.12009679)(482.69515137,64.17010061)
\curveto(482.70515232,64.2100967)(482.71515231,64.28009663)(482.72515137,64.38010061)
\curveto(482.73515229,64.49009642)(482.73015229,64.57509633)(482.71015137,64.63510061)
\curveto(482.69015233,64.7050962)(482.68015234,64.77509613)(482.68015137,64.84510061)
\curveto(482.67015235,64.91509599)(482.65515237,64.98009593)(482.63515137,65.04010061)
\curveto(482.57515245,65.24009567)(482.49015253,65.42009549)(482.38015137,65.58010061)
\curveto(482.36015266,65.6100953)(482.34015268,65.63509527)(482.32015137,65.65510061)
\lineto(482.26015137,65.71510061)
\curveto(482.24015278,65.75509515)(482.20015282,65.8050951)(482.14015137,65.86510061)
\curveto(482.00015302,65.96509494)(481.87015315,66.05009486)(481.75015137,66.12010061)
\curveto(481.63015339,66.19009472)(481.48515354,66.26009465)(481.31515137,66.33010061)
\curveto(481.24515378,66.36009455)(481.17515385,66.38009453)(481.10515137,66.39010061)
\curveto(481.03515399,66.4100945)(480.96015406,66.43009448)(480.88015137,66.45010061)
}
}
{
\newrgbcolor{curcolor}{0 0 0}
\pscustom[linestyle=none,fillstyle=solid,fillcolor=curcolor]
{
\newpath
\moveto(473.03515137,73.40470998)
\curveto(473.03516199,73.50470513)(473.04516198,73.59970503)(473.06515137,73.68970998)
\curveto(473.07516195,73.77970485)(473.10516192,73.84470479)(473.15515137,73.88470998)
\curveto(473.23516179,73.94470469)(473.34016168,73.97470466)(473.47015137,73.97470998)
\lineto(473.86015137,73.97470998)
\lineto(475.36015137,73.97470998)
\lineto(481.75015137,73.97470998)
\lineto(482.92015137,73.97470998)
\lineto(483.23515137,73.97470998)
\curveto(483.33515169,73.98470465)(483.41515161,73.96970466)(483.47515137,73.92970998)
\curveto(483.55515147,73.87970475)(483.60515142,73.80470483)(483.62515137,73.70470998)
\curveto(483.63515139,73.61470502)(483.64015138,73.50470513)(483.64015137,73.37470998)
\lineto(483.64015137,73.14970998)
\curveto(483.6201514,73.06970556)(483.60515142,72.99970563)(483.59515137,72.93970998)
\curveto(483.57515145,72.87970575)(483.53515149,72.8297058)(483.47515137,72.78970998)
\curveto(483.41515161,72.74970588)(483.34015168,72.7297059)(483.25015137,72.72970998)
\lineto(482.95015137,72.72970998)
\lineto(481.85515137,72.72970998)
\lineto(476.51515137,72.72970998)
\curveto(476.4251586,72.70970592)(476.35015867,72.69470594)(476.29015137,72.68470998)
\curveto(476.2201588,72.68470595)(476.16015886,72.65470598)(476.11015137,72.59470998)
\curveto(476.06015896,72.52470611)(476.03515899,72.4347062)(476.03515137,72.32470998)
\curveto(476.025159,72.22470641)(476.020159,72.11470652)(476.02015137,71.99470998)
\lineto(476.02015137,70.85470998)
\lineto(476.02015137,70.35970998)
\curveto(476.01015901,70.19970843)(475.95015907,70.08970854)(475.84015137,70.02970998)
\curveto(475.81015921,70.00970862)(475.78015924,69.99970863)(475.75015137,69.99970998)
\curveto(475.71015931,69.99970863)(475.66515936,69.99470864)(475.61515137,69.98470998)
\curveto(475.49515953,69.96470867)(475.38515964,69.96970866)(475.28515137,69.99970998)
\curveto(475.18515984,70.03970859)(475.11515991,70.09470854)(475.07515137,70.16470998)
\curveto(475.02516,70.24470839)(475.00016002,70.36470827)(475.00015137,70.52470998)
\curveto(475.00016002,70.68470795)(474.98516004,70.81970781)(474.95515137,70.92970998)
\curveto(474.94516008,70.97970765)(474.94016008,71.0347076)(474.94015137,71.09470998)
\curveto(474.93016009,71.15470748)(474.91516011,71.21470742)(474.89515137,71.27470998)
\curveto(474.84516018,71.42470721)(474.79516023,71.56970706)(474.74515137,71.70970998)
\curveto(474.68516034,71.84970678)(474.61516041,71.98470665)(474.53515137,72.11470998)
\curveto(474.44516058,72.25470638)(474.34016068,72.37470626)(474.22015137,72.47470998)
\curveto(474.10016092,72.57470606)(473.97016105,72.66970596)(473.83015137,72.75970998)
\curveto(473.73016129,72.81970581)(473.6201614,72.86470577)(473.50015137,72.89470998)
\curveto(473.38016164,72.9347057)(473.27516175,72.98470565)(473.18515137,73.04470998)
\curveto(473.1251619,73.09470554)(473.08516194,73.16470547)(473.06515137,73.25470998)
\curveto(473.05516197,73.27470536)(473.05016197,73.29970533)(473.05015137,73.32970998)
\curveto(473.05016197,73.35970527)(473.04516198,73.38470525)(473.03515137,73.40470998)
}
}
{
\newrgbcolor{curcolor}{0 0 0}
\pscustom[linestyle=none,fillstyle=solid,fillcolor=curcolor]
{
\newpath
\moveto(482.00515137,78.63431936)
\lineto(482.00515137,79.26431936)
\lineto(482.00515137,79.45931936)
\curveto(482.00515302,79.52931683)(482.01515301,79.58931677)(482.03515137,79.63931936)
\curveto(482.07515295,79.70931665)(482.11515291,79.7593166)(482.15515137,79.78931936)
\curveto(482.20515282,79.82931653)(482.27015275,79.84931651)(482.35015137,79.84931936)
\curveto(482.43015259,79.8593165)(482.51515251,79.86431649)(482.60515137,79.86431936)
\lineto(483.32515137,79.86431936)
\curveto(483.80515122,79.86431649)(484.21515081,79.80431655)(484.55515137,79.68431936)
\curveto(484.89515013,79.56431679)(485.17014985,79.36931699)(485.38015137,79.09931936)
\curveto(485.43014959,79.02931733)(485.47514955,78.9593174)(485.51515137,78.88931936)
\curveto(485.56514946,78.82931753)(485.61014941,78.7543176)(485.65015137,78.66431936)
\curveto(485.66014936,78.64431771)(485.67014935,78.61431774)(485.68015137,78.57431936)
\curveto(485.70014932,78.53431782)(485.70514932,78.48931787)(485.69515137,78.43931936)
\curveto(485.66514936,78.34931801)(485.59014943,78.29431806)(485.47015137,78.27431936)
\curveto(485.36014966,78.2543181)(485.26514976,78.26931809)(485.18515137,78.31931936)
\curveto(485.11514991,78.34931801)(485.05014997,78.39431796)(484.99015137,78.45431936)
\curveto(484.94015008,78.52431783)(484.89015013,78.58931777)(484.84015137,78.64931936)
\curveto(484.79015023,78.71931764)(484.71515031,78.77931758)(484.61515137,78.82931936)
\curveto(484.5251505,78.88931747)(484.43515059,78.93931742)(484.34515137,78.97931936)
\curveto(484.31515071,78.99931736)(484.25515077,79.02431733)(484.16515137,79.05431936)
\curveto(484.08515094,79.08431727)(484.01515101,79.08931727)(483.95515137,79.06931936)
\curveto(483.81515121,79.03931732)(483.7251513,78.97931738)(483.68515137,78.88931936)
\curveto(483.65515137,78.80931755)(483.64015138,78.71931764)(483.64015137,78.61931936)
\curveto(483.64015138,78.51931784)(483.61515141,78.43431792)(483.56515137,78.36431936)
\curveto(483.49515153,78.27431808)(483.35515167,78.22931813)(483.14515137,78.22931936)
\lineto(482.59015137,78.22931936)
\lineto(482.36515137,78.22931936)
\curveto(482.28515274,78.23931812)(482.2201528,78.2593181)(482.17015137,78.28931936)
\curveto(482.09015293,78.34931801)(482.04515298,78.41931794)(482.03515137,78.49931936)
\curveto(482.025153,78.51931784)(482.020153,78.53931782)(482.02015137,78.55931936)
\curveto(482.020153,78.58931777)(482.01515301,78.61431774)(482.00515137,78.63431936)
}
}
{
\newrgbcolor{curcolor}{0 0 0}
\pscustom[linestyle=none,fillstyle=solid,fillcolor=curcolor]
{
}
}
{
\newrgbcolor{curcolor}{0 0 0}
\pscustom[linestyle=none,fillstyle=solid,fillcolor=curcolor]
{
\newpath
\moveto(473.03515137,89.26463186)
\curveto(473.025162,89.95462722)(473.14516188,90.55462662)(473.39515137,91.06463186)
\curveto(473.64516138,91.58462559)(473.98016104,91.9796252)(474.40015137,92.24963186)
\curveto(474.48016054,92.29962488)(474.57016045,92.34462483)(474.67015137,92.38463186)
\curveto(474.76016026,92.42462475)(474.85516017,92.46962471)(474.95515137,92.51963186)
\curveto(475.05515997,92.55962462)(475.15515987,92.58962459)(475.25515137,92.60963186)
\curveto(475.35515967,92.62962455)(475.46015956,92.64962453)(475.57015137,92.66963186)
\curveto(475.6201594,92.68962449)(475.66515936,92.69462448)(475.70515137,92.68463186)
\curveto(475.74515928,92.6746245)(475.79015923,92.6796245)(475.84015137,92.69963186)
\curveto(475.89015913,92.70962447)(475.97515905,92.71462446)(476.09515137,92.71463186)
\curveto(476.20515882,92.71462446)(476.29015873,92.70962447)(476.35015137,92.69963186)
\curveto(476.41015861,92.6796245)(476.47015855,92.66962451)(476.53015137,92.66963186)
\curveto(476.59015843,92.6796245)(476.65015837,92.6746245)(476.71015137,92.65463186)
\curveto(476.85015817,92.61462456)(476.98515804,92.5796246)(477.11515137,92.54963186)
\curveto(477.24515778,92.51962466)(477.37015765,92.4796247)(477.49015137,92.42963186)
\curveto(477.63015739,92.36962481)(477.75515727,92.29962488)(477.86515137,92.21963186)
\curveto(477.97515705,92.14962503)(478.08515694,92.0746251)(478.19515137,91.99463186)
\lineto(478.25515137,91.93463186)
\curveto(478.27515675,91.92462525)(478.29515673,91.90962527)(478.31515137,91.88963186)
\curveto(478.47515655,91.76962541)(478.6201564,91.63462554)(478.75015137,91.48463186)
\curveto(478.88015614,91.33462584)(479.00515602,91.174626)(479.12515137,91.00463186)
\curveto(479.34515568,90.69462648)(479.55015547,90.39962678)(479.74015137,90.11963186)
\curveto(479.88015514,89.88962729)(480.01515501,89.65962752)(480.14515137,89.42963186)
\curveto(480.27515475,89.20962797)(480.41015461,88.98962819)(480.55015137,88.76963186)
\curveto(480.7201543,88.51962866)(480.90015412,88.2796289)(481.09015137,88.04963186)
\curveto(481.28015374,87.82962935)(481.50515352,87.63962954)(481.76515137,87.47963186)
\curveto(481.8251532,87.43962974)(481.88515314,87.40462977)(481.94515137,87.37463186)
\curveto(481.99515303,87.34462983)(482.06015296,87.31462986)(482.14015137,87.28463186)
\curveto(482.21015281,87.26462991)(482.27015275,87.25962992)(482.32015137,87.26963186)
\curveto(482.39015263,87.28962989)(482.44515258,87.32462985)(482.48515137,87.37463186)
\curveto(482.51515251,87.42462975)(482.53515249,87.48462969)(482.54515137,87.55463186)
\lineto(482.54515137,87.79463186)
\lineto(482.54515137,88.54463186)
\lineto(482.54515137,91.34963186)
\lineto(482.54515137,92.00963186)
\curveto(482.54515248,92.09962508)(482.55015247,92.18462499)(482.56015137,92.26463186)
\curveto(482.56015246,92.34462483)(482.58015244,92.40962477)(482.62015137,92.45963186)
\curveto(482.66015236,92.50962467)(482.73515229,92.54962463)(482.84515137,92.57963186)
\curveto(482.94515208,92.61962456)(483.04515198,92.62962455)(483.14515137,92.60963186)
\lineto(483.28015137,92.60963186)
\curveto(483.35015167,92.58962459)(483.41015161,92.56962461)(483.46015137,92.54963186)
\curveto(483.51015151,92.52962465)(483.55015147,92.49462468)(483.58015137,92.44463186)
\curveto(483.6201514,92.39462478)(483.64015138,92.32462485)(483.64015137,92.23463186)
\lineto(483.64015137,91.96463186)
\lineto(483.64015137,91.06463186)
\lineto(483.64015137,87.55463186)
\lineto(483.64015137,86.48963186)
\curveto(483.64015138,86.40963077)(483.64515138,86.31963086)(483.65515137,86.21963186)
\curveto(483.65515137,86.11963106)(483.64515138,86.03463114)(483.62515137,85.96463186)
\curveto(483.55515147,85.75463142)(483.37515165,85.68963149)(483.08515137,85.76963186)
\curveto(483.04515198,85.7796314)(483.01015201,85.7796314)(482.98015137,85.76963186)
\curveto(482.94015208,85.76963141)(482.89515213,85.7796314)(482.84515137,85.79963186)
\curveto(482.76515226,85.81963136)(482.68015234,85.83963134)(482.59015137,85.85963186)
\curveto(482.50015252,85.8796313)(482.41515261,85.90463127)(482.33515137,85.93463186)
\curveto(481.84515318,86.09463108)(481.43015359,86.29463088)(481.09015137,86.53463186)
\curveto(480.84015418,86.71463046)(480.61515441,86.91963026)(480.41515137,87.14963186)
\curveto(480.20515482,87.3796298)(480.01015501,87.61962956)(479.83015137,87.86963186)
\curveto(479.65015537,88.12962905)(479.48015554,88.39462878)(479.32015137,88.66463186)
\curveto(479.15015587,88.94462823)(478.97515605,89.21462796)(478.79515137,89.47463186)
\curveto(478.71515631,89.58462759)(478.64015638,89.68962749)(478.57015137,89.78963186)
\curveto(478.50015652,89.89962728)(478.4251566,90.00962717)(478.34515137,90.11963186)
\curveto(478.31515671,90.15962702)(478.28515674,90.19462698)(478.25515137,90.22463186)
\curveto(478.21515681,90.26462691)(478.18515684,90.30462687)(478.16515137,90.34463186)
\curveto(478.05515697,90.48462669)(477.93015709,90.60962657)(477.79015137,90.71963186)
\curveto(477.76015726,90.73962644)(477.73515729,90.76462641)(477.71515137,90.79463186)
\curveto(477.68515734,90.82462635)(477.65515737,90.84962633)(477.62515137,90.86963186)
\curveto(477.5251575,90.94962623)(477.4251576,91.01462616)(477.32515137,91.06463186)
\curveto(477.2251578,91.12462605)(477.11515791,91.179626)(476.99515137,91.22963186)
\curveto(476.9251581,91.25962592)(476.85015817,91.2796259)(476.77015137,91.28963186)
\lineto(476.53015137,91.34963186)
\lineto(476.44015137,91.34963186)
\curveto(476.41015861,91.35962582)(476.38015864,91.36462581)(476.35015137,91.36463186)
\curveto(476.28015874,91.38462579)(476.18515884,91.38962579)(476.06515137,91.37963186)
\curveto(475.93515909,91.3796258)(475.83515919,91.36962581)(475.76515137,91.34963186)
\curveto(475.68515934,91.32962585)(475.61015941,91.30962587)(475.54015137,91.28963186)
\curveto(475.46015956,91.2796259)(475.38015964,91.25962592)(475.30015137,91.22963186)
\curveto(475.06015996,91.11962606)(474.86016016,90.96962621)(474.70015137,90.77963186)
\curveto(474.53016049,90.59962658)(474.39016063,90.3796268)(474.28015137,90.11963186)
\curveto(474.26016076,90.04962713)(474.24516078,89.9796272)(474.23515137,89.90963186)
\curveto(474.21516081,89.83962734)(474.19516083,89.76462741)(474.17515137,89.68463186)
\curveto(474.15516087,89.60462757)(474.14516088,89.49462768)(474.14515137,89.35463186)
\curveto(474.14516088,89.22462795)(474.15516087,89.11962806)(474.17515137,89.03963186)
\curveto(474.18516084,88.9796282)(474.19016083,88.92462825)(474.19015137,88.87463186)
\curveto(474.19016083,88.82462835)(474.20016082,88.7746284)(474.22015137,88.72463186)
\curveto(474.26016076,88.62462855)(474.30016072,88.52962865)(474.34015137,88.43963186)
\curveto(474.38016064,88.35962882)(474.4251606,88.2796289)(474.47515137,88.19963186)
\curveto(474.49516053,88.16962901)(474.5201605,88.13962904)(474.55015137,88.10963186)
\curveto(474.58016044,88.08962909)(474.60516042,88.06462911)(474.62515137,88.03463186)
\lineto(474.70015137,87.95963186)
\curveto(474.7201603,87.92962925)(474.74016028,87.90462927)(474.76015137,87.88463186)
\lineto(474.97015137,87.73463186)
\curveto(475.03015999,87.69462948)(475.09515993,87.64962953)(475.16515137,87.59963186)
\curveto(475.25515977,87.53962964)(475.36015966,87.48962969)(475.48015137,87.44963186)
\curveto(475.59015943,87.41962976)(475.70015932,87.38462979)(475.81015137,87.34463186)
\curveto(475.9201591,87.30462987)(476.06515896,87.2796299)(476.24515137,87.26963186)
\curveto(476.41515861,87.25962992)(476.54015848,87.22962995)(476.62015137,87.17963186)
\curveto(476.70015832,87.12963005)(476.74515828,87.05463012)(476.75515137,86.95463186)
\curveto(476.76515826,86.85463032)(476.77015825,86.74463043)(476.77015137,86.62463186)
\curveto(476.77015825,86.58463059)(476.77515825,86.54463063)(476.78515137,86.50463186)
\curveto(476.78515824,86.46463071)(476.78015824,86.42963075)(476.77015137,86.39963186)
\curveto(476.75015827,86.34963083)(476.74015828,86.29963088)(476.74015137,86.24963186)
\curveto(476.74015828,86.20963097)(476.73015829,86.16963101)(476.71015137,86.12963186)
\curveto(476.65015837,86.03963114)(476.51515851,85.99463118)(476.30515137,85.99463186)
\lineto(476.18515137,85.99463186)
\curveto(476.1251589,86.00463117)(476.06515896,86.00963117)(476.00515137,86.00963186)
\curveto(475.93515909,86.01963116)(475.87015915,86.02963115)(475.81015137,86.03963186)
\curveto(475.70015932,86.05963112)(475.60015942,86.0796311)(475.51015137,86.09963186)
\curveto(475.41015961,86.11963106)(475.31515971,86.14963103)(475.22515137,86.18963186)
\curveto(475.15515987,86.20963097)(475.09515993,86.22963095)(475.04515137,86.24963186)
\lineto(474.86515137,86.30963186)
\curveto(474.60516042,86.42963075)(474.36016066,86.58463059)(474.13015137,86.77463186)
\curveto(473.90016112,86.9746302)(473.71516131,87.18962999)(473.57515137,87.41963186)
\curveto(473.49516153,87.52962965)(473.43016159,87.64462953)(473.38015137,87.76463186)
\lineto(473.23015137,88.15463186)
\curveto(473.18016184,88.26462891)(473.15016187,88.3796288)(473.14015137,88.49963186)
\curveto(473.1201619,88.61962856)(473.09516193,88.74462843)(473.06515137,88.87463186)
\curveto(473.06516196,88.94462823)(473.06516196,89.00962817)(473.06515137,89.06963186)
\curveto(473.05516197,89.12962805)(473.04516198,89.19462798)(473.03515137,89.26463186)
}
}
{
\newrgbcolor{curcolor}{0 0 0}
\pscustom[linestyle=none,fillstyle=solid,fillcolor=curcolor]
{
\newpath
\moveto(478.55515137,101.36424123)
\lineto(478.81015137,101.36424123)
\curveto(478.89015613,101.37423353)(478.96515606,101.36923353)(479.03515137,101.34924123)
\lineto(479.27515137,101.34924123)
\lineto(479.44015137,101.34924123)
\curveto(479.54015548,101.32923357)(479.64515538,101.31923358)(479.75515137,101.31924123)
\curveto(479.85515517,101.31923358)(479.95515507,101.30923359)(480.05515137,101.28924123)
\lineto(480.20515137,101.28924123)
\curveto(480.34515468,101.25923364)(480.48515454,101.23923366)(480.62515137,101.22924123)
\curveto(480.75515427,101.21923368)(480.88515414,101.19423371)(481.01515137,101.15424123)
\curveto(481.09515393,101.13423377)(481.18015384,101.11423379)(481.27015137,101.09424123)
\lineto(481.51015137,101.03424123)
\lineto(481.81015137,100.91424123)
\curveto(481.90015312,100.88423402)(481.99015303,100.84923405)(482.08015137,100.80924123)
\curveto(482.30015272,100.70923419)(482.51515251,100.57423433)(482.72515137,100.40424123)
\curveto(482.93515209,100.24423466)(483.10515192,100.06923483)(483.23515137,99.87924123)
\curveto(483.27515175,99.82923507)(483.31515171,99.76923513)(483.35515137,99.69924123)
\curveto(483.38515164,99.63923526)(483.4201516,99.57923532)(483.46015137,99.51924123)
\curveto(483.51015151,99.43923546)(483.55015147,99.34423556)(483.58015137,99.23424123)
\curveto(483.61015141,99.12423578)(483.64015138,99.01923588)(483.67015137,98.91924123)
\curveto(483.71015131,98.80923609)(483.73515129,98.6992362)(483.74515137,98.58924123)
\curveto(483.75515127,98.47923642)(483.77015125,98.36423654)(483.79015137,98.24424123)
\curveto(483.80015122,98.2042367)(483.80015122,98.15923674)(483.79015137,98.10924123)
\curveto(483.79015123,98.06923683)(483.79515123,98.02923687)(483.80515137,97.98924123)
\curveto(483.81515121,97.94923695)(483.8201512,97.89423701)(483.82015137,97.82424123)
\curveto(483.8201512,97.75423715)(483.81515121,97.7042372)(483.80515137,97.67424123)
\curveto(483.78515124,97.62423728)(483.78015124,97.57923732)(483.79015137,97.53924123)
\curveto(483.80015122,97.4992374)(483.80015122,97.46423744)(483.79015137,97.43424123)
\lineto(483.79015137,97.34424123)
\curveto(483.77015125,97.28423762)(483.75515127,97.21923768)(483.74515137,97.14924123)
\curveto(483.74515128,97.08923781)(483.74015128,97.02423788)(483.73015137,96.95424123)
\curveto(483.68015134,96.78423812)(483.63015139,96.62423828)(483.58015137,96.47424123)
\curveto(483.53015149,96.32423858)(483.46515156,96.17923872)(483.38515137,96.03924123)
\curveto(483.34515168,95.98923891)(483.31515171,95.93423897)(483.29515137,95.87424123)
\curveto(483.26515176,95.82423908)(483.23015179,95.77423913)(483.19015137,95.72424123)
\curveto(483.01015201,95.48423942)(482.79015223,95.28423962)(482.53015137,95.12424123)
\curveto(482.27015275,94.96423994)(481.98515304,94.82424008)(481.67515137,94.70424123)
\curveto(481.53515349,94.64424026)(481.39515363,94.5992403)(481.25515137,94.56924123)
\curveto(481.10515392,94.53924036)(480.95015407,94.5042404)(480.79015137,94.46424123)
\curveto(480.68015434,94.44424046)(480.57015445,94.42924047)(480.46015137,94.41924123)
\curveto(480.35015467,94.40924049)(480.24015478,94.39424051)(480.13015137,94.37424123)
\curveto(480.09015493,94.36424054)(480.05015497,94.35924054)(480.01015137,94.35924123)
\curveto(479.97015505,94.36924053)(479.93015509,94.36924053)(479.89015137,94.35924123)
\curveto(479.84015518,94.34924055)(479.79015523,94.34424056)(479.74015137,94.34424123)
\lineto(479.57515137,94.34424123)
\curveto(479.5251555,94.32424058)(479.47515555,94.31924058)(479.42515137,94.32924123)
\curveto(479.36515566,94.33924056)(479.31015571,94.33924056)(479.26015137,94.32924123)
\curveto(479.2201558,94.31924058)(479.17515585,94.31924058)(479.12515137,94.32924123)
\curveto(479.07515595,94.33924056)(479.025156,94.33424057)(478.97515137,94.31424123)
\curveto(478.90515612,94.29424061)(478.83015619,94.28924061)(478.75015137,94.29924123)
\curveto(478.66015636,94.30924059)(478.57515645,94.31424059)(478.49515137,94.31424123)
\curveto(478.40515662,94.31424059)(478.30515672,94.30924059)(478.19515137,94.29924123)
\curveto(478.07515695,94.28924061)(477.97515705,94.29424061)(477.89515137,94.31424123)
\lineto(477.61015137,94.31424123)
\lineto(476.98015137,94.35924123)
\curveto(476.88015814,94.36924053)(476.78515824,94.37924052)(476.69515137,94.38924123)
\lineto(476.39515137,94.41924123)
\curveto(476.34515868,94.43924046)(476.29515873,94.44424046)(476.24515137,94.43424123)
\curveto(476.18515884,94.43424047)(476.13015889,94.44424046)(476.08015137,94.46424123)
\curveto(475.91015911,94.51424039)(475.74515928,94.55424035)(475.58515137,94.58424123)
\curveto(475.41515961,94.61424029)(475.25515977,94.66424024)(475.10515137,94.73424123)
\curveto(474.64516038,94.92423998)(474.27016075,95.14423976)(473.98015137,95.39424123)
\curveto(473.69016133,95.65423925)(473.44516158,96.01423889)(473.24515137,96.47424123)
\curveto(473.19516183,96.6042383)(473.16016186,96.73423817)(473.14015137,96.86424123)
\curveto(473.1201619,97.0042379)(473.09516193,97.14423776)(473.06515137,97.28424123)
\curveto(473.05516197,97.35423755)(473.05016197,97.41923748)(473.05015137,97.47924123)
\curveto(473.05016197,97.53923736)(473.04516198,97.6042373)(473.03515137,97.67424123)
\curveto(473.01516201,98.5042364)(473.16516186,99.17423573)(473.48515137,99.68424123)
\curveto(473.79516123,100.19423471)(474.23516079,100.57423433)(474.80515137,100.82424123)
\curveto(474.9251601,100.87423403)(475.05015997,100.91923398)(475.18015137,100.95924123)
\curveto(475.31015971,100.9992339)(475.44515958,101.04423386)(475.58515137,101.09424123)
\curveto(475.66515936,101.11423379)(475.75015927,101.12923377)(475.84015137,101.13924123)
\lineto(476.08015137,101.19924123)
\curveto(476.19015883,101.22923367)(476.30015872,101.24423366)(476.41015137,101.24424123)
\curveto(476.5201585,101.25423365)(476.63015839,101.26923363)(476.74015137,101.28924123)
\curveto(476.79015823,101.30923359)(476.83515819,101.31423359)(476.87515137,101.30424123)
\curveto(476.91515811,101.3042336)(476.95515807,101.30923359)(476.99515137,101.31924123)
\curveto(477.04515798,101.32923357)(477.10015792,101.32923357)(477.16015137,101.31924123)
\curveto(477.21015781,101.31923358)(477.26015776,101.32423358)(477.31015137,101.33424123)
\lineto(477.44515137,101.33424123)
\curveto(477.50515752,101.35423355)(477.57515745,101.35423355)(477.65515137,101.33424123)
\curveto(477.7251573,101.32423358)(477.79015723,101.32923357)(477.85015137,101.34924123)
\curveto(477.88015714,101.35923354)(477.9201571,101.36423354)(477.97015137,101.36424123)
\lineto(478.09015137,101.36424123)
\lineto(478.55515137,101.36424123)
\moveto(480.88015137,99.81924123)
\curveto(480.56015446,99.91923498)(480.19515483,99.97923492)(479.78515137,99.99924123)
\curveto(479.37515565,100.01923488)(478.96515606,100.02923487)(478.55515137,100.02924123)
\curveto(478.1251569,100.02923487)(477.70515732,100.01923488)(477.29515137,99.99924123)
\curveto(476.88515814,99.97923492)(476.50015852,99.93423497)(476.14015137,99.86424123)
\curveto(475.78015924,99.79423511)(475.46015956,99.68423522)(475.18015137,99.53424123)
\curveto(474.89016013,99.39423551)(474.65516037,99.1992357)(474.47515137,98.94924123)
\curveto(474.36516066,98.78923611)(474.28516074,98.60923629)(474.23515137,98.40924123)
\curveto(474.17516085,98.20923669)(474.14516088,97.96423694)(474.14515137,97.67424123)
\curveto(474.16516086,97.65423725)(474.17516085,97.61923728)(474.17515137,97.56924123)
\curveto(474.16516086,97.51923738)(474.16516086,97.47923742)(474.17515137,97.44924123)
\curveto(474.19516083,97.36923753)(474.21516081,97.29423761)(474.23515137,97.22424123)
\curveto(474.24516078,97.16423774)(474.26516076,97.0992378)(474.29515137,97.02924123)
\curveto(474.41516061,96.75923814)(474.58516044,96.53923836)(474.80515137,96.36924123)
\curveto(475.01516001,96.20923869)(475.26015976,96.07423883)(475.54015137,95.96424123)
\curveto(475.65015937,95.91423899)(475.77015925,95.87423903)(475.90015137,95.84424123)
\curveto(476.020159,95.82423908)(476.14515888,95.7992391)(476.27515137,95.76924123)
\curveto(476.3251587,95.74923915)(476.38015864,95.73923916)(476.44015137,95.73924123)
\curveto(476.49015853,95.73923916)(476.54015848,95.73423917)(476.59015137,95.72424123)
\curveto(476.68015834,95.71423919)(476.77515825,95.7042392)(476.87515137,95.69424123)
\curveto(476.96515806,95.68423922)(477.06015796,95.67423923)(477.16015137,95.66424123)
\curveto(477.24015778,95.66423924)(477.3251577,95.65923924)(477.41515137,95.64924123)
\lineto(477.65515137,95.64924123)
\lineto(477.83515137,95.64924123)
\curveto(477.86515716,95.63923926)(477.90015712,95.63423927)(477.94015137,95.63424123)
\lineto(478.07515137,95.63424123)
\lineto(478.52515137,95.63424123)
\curveto(478.60515642,95.63423927)(478.69015633,95.62923927)(478.78015137,95.61924123)
\curveto(478.86015616,95.61923928)(478.93515609,95.62923927)(479.00515137,95.64924123)
\lineto(479.27515137,95.64924123)
\curveto(479.29515573,95.64923925)(479.3251557,95.64423926)(479.36515137,95.63424123)
\curveto(479.39515563,95.63423927)(479.4201556,95.63923926)(479.44015137,95.64924123)
\curveto(479.54015548,95.65923924)(479.64015538,95.66423924)(479.74015137,95.66424123)
\curveto(479.83015519,95.67423923)(479.93015509,95.68423922)(480.04015137,95.69424123)
\curveto(480.16015486,95.72423918)(480.28515474,95.73923916)(480.41515137,95.73924123)
\curveto(480.53515449,95.74923915)(480.65015437,95.77423913)(480.76015137,95.81424123)
\curveto(481.06015396,95.89423901)(481.3251537,95.97923892)(481.55515137,96.06924123)
\curveto(481.78515324,96.16923873)(482.00015302,96.31423859)(482.20015137,96.50424123)
\curveto(482.40015262,96.71423819)(482.55015247,96.97923792)(482.65015137,97.29924123)
\curveto(482.67015235,97.33923756)(482.68015234,97.37423753)(482.68015137,97.40424123)
\curveto(482.67015235,97.44423746)(482.67515235,97.48923741)(482.69515137,97.53924123)
\curveto(482.70515232,97.57923732)(482.71515231,97.64923725)(482.72515137,97.74924123)
\curveto(482.73515229,97.85923704)(482.73015229,97.94423696)(482.71015137,98.00424123)
\curveto(482.69015233,98.07423683)(482.68015234,98.14423676)(482.68015137,98.21424123)
\curveto(482.67015235,98.28423662)(482.65515237,98.34923655)(482.63515137,98.40924123)
\curveto(482.57515245,98.60923629)(482.49015253,98.78923611)(482.38015137,98.94924123)
\curveto(482.36015266,98.97923592)(482.34015268,99.0042359)(482.32015137,99.02424123)
\lineto(482.26015137,99.08424123)
\curveto(482.24015278,99.12423578)(482.20015282,99.17423573)(482.14015137,99.23424123)
\curveto(482.00015302,99.33423557)(481.87015315,99.41923548)(481.75015137,99.48924123)
\curveto(481.63015339,99.55923534)(481.48515354,99.62923527)(481.31515137,99.69924123)
\curveto(481.24515378,99.72923517)(481.17515385,99.74923515)(481.10515137,99.75924123)
\curveto(481.03515399,99.77923512)(480.96015406,99.7992351)(480.88015137,99.81924123)
}
}
{
\newrgbcolor{curcolor}{0 0 0}
\pscustom[linestyle=none,fillstyle=solid,fillcolor=curcolor]
{
\newpath
\moveto(473.03515137,106.77385061)
\curveto(473.03516199,106.87384575)(473.04516198,106.96884566)(473.06515137,107.05885061)
\curveto(473.07516195,107.14884548)(473.10516192,107.21384541)(473.15515137,107.25385061)
\curveto(473.23516179,107.31384531)(473.34016168,107.34384528)(473.47015137,107.34385061)
\lineto(473.86015137,107.34385061)
\lineto(475.36015137,107.34385061)
\lineto(481.75015137,107.34385061)
\lineto(482.92015137,107.34385061)
\lineto(483.23515137,107.34385061)
\curveto(483.33515169,107.35384527)(483.41515161,107.33884529)(483.47515137,107.29885061)
\curveto(483.55515147,107.24884538)(483.60515142,107.17384545)(483.62515137,107.07385061)
\curveto(483.63515139,106.98384564)(483.64015138,106.87384575)(483.64015137,106.74385061)
\lineto(483.64015137,106.51885061)
\curveto(483.6201514,106.43884619)(483.60515142,106.36884626)(483.59515137,106.30885061)
\curveto(483.57515145,106.24884638)(483.53515149,106.19884643)(483.47515137,106.15885061)
\curveto(483.41515161,106.11884651)(483.34015168,106.09884653)(483.25015137,106.09885061)
\lineto(482.95015137,106.09885061)
\lineto(481.85515137,106.09885061)
\lineto(476.51515137,106.09885061)
\curveto(476.4251586,106.07884655)(476.35015867,106.06384656)(476.29015137,106.05385061)
\curveto(476.2201588,106.05384657)(476.16015886,106.0238466)(476.11015137,105.96385061)
\curveto(476.06015896,105.89384673)(476.03515899,105.80384682)(476.03515137,105.69385061)
\curveto(476.025159,105.59384703)(476.020159,105.48384714)(476.02015137,105.36385061)
\lineto(476.02015137,104.22385061)
\lineto(476.02015137,103.72885061)
\curveto(476.01015901,103.56884906)(475.95015907,103.45884917)(475.84015137,103.39885061)
\curveto(475.81015921,103.37884925)(475.78015924,103.36884926)(475.75015137,103.36885061)
\curveto(475.71015931,103.36884926)(475.66515936,103.36384926)(475.61515137,103.35385061)
\curveto(475.49515953,103.33384929)(475.38515964,103.33884929)(475.28515137,103.36885061)
\curveto(475.18515984,103.40884922)(475.11515991,103.46384916)(475.07515137,103.53385061)
\curveto(475.02516,103.61384901)(475.00016002,103.73384889)(475.00015137,103.89385061)
\curveto(475.00016002,104.05384857)(474.98516004,104.18884844)(474.95515137,104.29885061)
\curveto(474.94516008,104.34884828)(474.94016008,104.40384822)(474.94015137,104.46385061)
\curveto(474.93016009,104.5238481)(474.91516011,104.58384804)(474.89515137,104.64385061)
\curveto(474.84516018,104.79384783)(474.79516023,104.93884769)(474.74515137,105.07885061)
\curveto(474.68516034,105.21884741)(474.61516041,105.35384727)(474.53515137,105.48385061)
\curveto(474.44516058,105.623847)(474.34016068,105.74384688)(474.22015137,105.84385061)
\curveto(474.10016092,105.94384668)(473.97016105,106.03884659)(473.83015137,106.12885061)
\curveto(473.73016129,106.18884644)(473.6201614,106.23384639)(473.50015137,106.26385061)
\curveto(473.38016164,106.30384632)(473.27516175,106.35384627)(473.18515137,106.41385061)
\curveto(473.1251619,106.46384616)(473.08516194,106.53384609)(473.06515137,106.62385061)
\curveto(473.05516197,106.64384598)(473.05016197,106.66884596)(473.05015137,106.69885061)
\curveto(473.05016197,106.7288459)(473.04516198,106.75384587)(473.03515137,106.77385061)
}
}
{
\newrgbcolor{curcolor}{0 0 0}
\pscustom[linestyle=none,fillstyle=solid,fillcolor=curcolor]
{
\newpath
\moveto(473.03515137,115.12345998)
\curveto(473.03516199,115.22345513)(473.04516198,115.31845503)(473.06515137,115.40845998)
\curveto(473.07516195,115.49845485)(473.10516192,115.56345479)(473.15515137,115.60345998)
\curveto(473.23516179,115.66345469)(473.34016168,115.69345466)(473.47015137,115.69345998)
\lineto(473.86015137,115.69345998)
\lineto(475.36015137,115.69345998)
\lineto(481.75015137,115.69345998)
\lineto(482.92015137,115.69345998)
\lineto(483.23515137,115.69345998)
\curveto(483.33515169,115.70345465)(483.41515161,115.68845466)(483.47515137,115.64845998)
\curveto(483.55515147,115.59845475)(483.60515142,115.52345483)(483.62515137,115.42345998)
\curveto(483.63515139,115.33345502)(483.64015138,115.22345513)(483.64015137,115.09345998)
\lineto(483.64015137,114.86845998)
\curveto(483.6201514,114.78845556)(483.60515142,114.71845563)(483.59515137,114.65845998)
\curveto(483.57515145,114.59845575)(483.53515149,114.5484558)(483.47515137,114.50845998)
\curveto(483.41515161,114.46845588)(483.34015168,114.4484559)(483.25015137,114.44845998)
\lineto(482.95015137,114.44845998)
\lineto(481.85515137,114.44845998)
\lineto(476.51515137,114.44845998)
\curveto(476.4251586,114.42845592)(476.35015867,114.41345594)(476.29015137,114.40345998)
\curveto(476.2201588,114.40345595)(476.16015886,114.37345598)(476.11015137,114.31345998)
\curveto(476.06015896,114.24345611)(476.03515899,114.1534562)(476.03515137,114.04345998)
\curveto(476.025159,113.94345641)(476.020159,113.83345652)(476.02015137,113.71345998)
\lineto(476.02015137,112.57345998)
\lineto(476.02015137,112.07845998)
\curveto(476.01015901,111.91845843)(475.95015907,111.80845854)(475.84015137,111.74845998)
\curveto(475.81015921,111.72845862)(475.78015924,111.71845863)(475.75015137,111.71845998)
\curveto(475.71015931,111.71845863)(475.66515936,111.71345864)(475.61515137,111.70345998)
\curveto(475.49515953,111.68345867)(475.38515964,111.68845866)(475.28515137,111.71845998)
\curveto(475.18515984,111.75845859)(475.11515991,111.81345854)(475.07515137,111.88345998)
\curveto(475.02516,111.96345839)(475.00016002,112.08345827)(475.00015137,112.24345998)
\curveto(475.00016002,112.40345795)(474.98516004,112.53845781)(474.95515137,112.64845998)
\curveto(474.94516008,112.69845765)(474.94016008,112.7534576)(474.94015137,112.81345998)
\curveto(474.93016009,112.87345748)(474.91516011,112.93345742)(474.89515137,112.99345998)
\curveto(474.84516018,113.14345721)(474.79516023,113.28845706)(474.74515137,113.42845998)
\curveto(474.68516034,113.56845678)(474.61516041,113.70345665)(474.53515137,113.83345998)
\curveto(474.44516058,113.97345638)(474.34016068,114.09345626)(474.22015137,114.19345998)
\curveto(474.10016092,114.29345606)(473.97016105,114.38845596)(473.83015137,114.47845998)
\curveto(473.73016129,114.53845581)(473.6201614,114.58345577)(473.50015137,114.61345998)
\curveto(473.38016164,114.6534557)(473.27516175,114.70345565)(473.18515137,114.76345998)
\curveto(473.1251619,114.81345554)(473.08516194,114.88345547)(473.06515137,114.97345998)
\curveto(473.05516197,114.99345536)(473.05016197,115.01845533)(473.05015137,115.04845998)
\curveto(473.05016197,115.07845527)(473.04516198,115.10345525)(473.03515137,115.12345998)
}
}
{
\newrgbcolor{curcolor}{0 0 0}
\pscustom[linestyle=none,fillstyle=solid,fillcolor=curcolor]
{
\newpath
\moveto(494.90647095,29.18119436)
\lineto(494.90647095,30.09619436)
\curveto(494.90648164,30.19619171)(494.90648164,30.29119161)(494.90647095,30.38119436)
\curveto(494.90648164,30.47119143)(494.92648162,30.54619136)(494.96647095,30.60619436)
\curveto(495.02648152,30.69619121)(495.10648144,30.75619115)(495.20647095,30.78619436)
\curveto(495.30648124,30.82619108)(495.41148114,30.87119103)(495.52147095,30.92119436)
\curveto(495.71148084,31.0011909)(495.90148065,31.07119083)(496.09147095,31.13119436)
\curveto(496.28148027,31.2011907)(496.47148008,31.27619063)(496.66147095,31.35619436)
\curveto(496.84147971,31.42619048)(497.02647952,31.49119041)(497.21647095,31.55119436)
\curveto(497.39647915,31.61119029)(497.57647897,31.68119022)(497.75647095,31.76119436)
\curveto(497.89647865,31.82119008)(498.04147851,31.87619003)(498.19147095,31.92619436)
\curveto(498.34147821,31.97618993)(498.48647806,32.03118987)(498.62647095,32.09119436)
\curveto(499.07647747,32.27118963)(499.53147702,32.44118946)(499.99147095,32.60119436)
\curveto(500.44147611,32.76118914)(500.89147566,32.93118897)(501.34147095,33.11119436)
\curveto(501.39147516,33.13118877)(501.44147511,33.14618876)(501.49147095,33.15619436)
\lineto(501.64147095,33.21619436)
\curveto(501.86147469,33.3061886)(502.08647446,33.39118851)(502.31647095,33.47119436)
\curveto(502.53647401,33.55118835)(502.75647379,33.63618827)(502.97647095,33.72619436)
\curveto(503.06647348,33.76618814)(503.17647337,33.8061881)(503.30647095,33.84619436)
\curveto(503.42647312,33.88618802)(503.49647305,33.95118795)(503.51647095,34.04119436)
\curveto(503.52647302,34.08118782)(503.52647302,34.11118779)(503.51647095,34.13119436)
\lineto(503.45647095,34.19119436)
\curveto(503.40647314,34.24118766)(503.3514732,34.27618763)(503.29147095,34.29619436)
\curveto(503.23147332,34.32618758)(503.16647338,34.35618755)(503.09647095,34.38619436)
\lineto(502.46647095,34.62619436)
\curveto(502.2464743,34.7061872)(502.03147452,34.78618712)(501.82147095,34.86619436)
\lineto(501.67147095,34.92619436)
\lineto(501.49147095,34.98619436)
\curveto(501.30147525,35.06618684)(501.11147544,35.13618677)(500.92147095,35.19619436)
\curveto(500.72147583,35.26618664)(500.52147603,35.34118656)(500.32147095,35.42119436)
\curveto(499.74147681,35.66118624)(499.15647739,35.88118602)(498.56647095,36.08119436)
\curveto(497.97647857,36.29118561)(497.39147916,36.51618539)(496.81147095,36.75619436)
\curveto(496.61147994,36.83618507)(496.40648014,36.91118499)(496.19647095,36.98119436)
\curveto(495.98648056,37.06118484)(495.78148077,37.14118476)(495.58147095,37.22119436)
\curveto(495.50148105,37.26118464)(495.40148115,37.29618461)(495.28147095,37.32619436)
\curveto(495.16148139,37.36618454)(495.07648147,37.42118448)(495.02647095,37.49119436)
\curveto(494.98648156,37.55118435)(494.95648159,37.62618428)(494.93647095,37.71619436)
\curveto(494.91648163,37.81618409)(494.90648164,37.92618398)(494.90647095,38.04619436)
\curveto(494.89648165,38.16618374)(494.89648165,38.28618362)(494.90647095,38.40619436)
\curveto(494.90648164,38.52618338)(494.90648164,38.63618327)(494.90647095,38.73619436)
\curveto(494.90648164,38.82618308)(494.90648164,38.91618299)(494.90647095,39.00619436)
\curveto(494.90648164,39.1061828)(494.92648162,39.18118272)(494.96647095,39.23119436)
\curveto(495.01648153,39.32118258)(495.10648144,39.37118253)(495.23647095,39.38119436)
\curveto(495.36648118,39.39118251)(495.50648104,39.39618251)(495.65647095,39.39619436)
\lineto(497.30647095,39.39619436)
\lineto(503.57647095,39.39619436)
\lineto(504.83647095,39.39619436)
\curveto(504.9464716,39.39618251)(505.05647149,39.39618251)(505.16647095,39.39619436)
\curveto(505.27647127,39.4061825)(505.36147119,39.38618252)(505.42147095,39.33619436)
\curveto(505.48147107,39.3061826)(505.52147103,39.26118264)(505.54147095,39.20119436)
\curveto(505.551471,39.14118276)(505.56647098,39.07118283)(505.58647095,38.99119436)
\lineto(505.58647095,38.75119436)
\lineto(505.58647095,38.39119436)
\curveto(505.57647097,38.28118362)(505.53147102,38.2011837)(505.45147095,38.15119436)
\curveto(505.42147113,38.13118377)(505.39147116,38.11618379)(505.36147095,38.10619436)
\curveto(505.32147123,38.1061838)(505.27647127,38.09618381)(505.22647095,38.07619436)
\lineto(505.06147095,38.07619436)
\curveto(505.00147155,38.06618384)(504.93147162,38.06118384)(504.85147095,38.06119436)
\curveto(504.77147178,38.07118383)(504.69647185,38.07618383)(504.62647095,38.07619436)
\lineto(503.78647095,38.07619436)
\lineto(499.36147095,38.07619436)
\curveto(499.11147744,38.07618383)(498.86147769,38.07618383)(498.61147095,38.07619436)
\curveto(498.3514782,38.07618383)(498.10147845,38.07118383)(497.86147095,38.06119436)
\curveto(497.76147879,38.06118384)(497.6514789,38.05618385)(497.53147095,38.04619436)
\curveto(497.41147914,38.03618387)(497.3514792,37.98118392)(497.35147095,37.88119436)
\lineto(497.36647095,37.88119436)
\curveto(497.38647916,37.81118409)(497.4514791,37.75118415)(497.56147095,37.70119436)
\curveto(497.67147888,37.66118424)(497.76647878,37.62618428)(497.84647095,37.59619436)
\curveto(498.01647853,37.52618438)(498.19147836,37.46118444)(498.37147095,37.40119436)
\curveto(498.54147801,37.34118456)(498.71147784,37.27118463)(498.88147095,37.19119436)
\curveto(498.93147762,37.17118473)(498.97647757,37.15618475)(499.01647095,37.14619436)
\curveto(499.05647749,37.13618477)(499.10147745,37.12118478)(499.15147095,37.10119436)
\curveto(499.33147722,37.02118488)(499.51647703,36.95118495)(499.70647095,36.89119436)
\curveto(499.88647666,36.84118506)(500.06647648,36.77618513)(500.24647095,36.69619436)
\curveto(500.39647615,36.62618528)(500.551476,36.56618534)(500.71147095,36.51619436)
\curveto(500.86147569,36.46618544)(501.01147554,36.41118549)(501.16147095,36.35119436)
\curveto(501.63147492,36.15118575)(502.10647444,35.97118593)(502.58647095,35.81119436)
\curveto(503.05647349,35.65118625)(503.52147303,35.47618643)(503.98147095,35.28619436)
\curveto(504.16147239,35.2061867)(504.34147221,35.13618677)(504.52147095,35.07619436)
\curveto(504.70147185,35.01618689)(504.88147167,34.95118695)(505.06147095,34.88119436)
\curveto(505.17147138,34.83118707)(505.27647127,34.78118712)(505.37647095,34.73119436)
\curveto(505.46647108,34.69118721)(505.53147102,34.6061873)(505.57147095,34.47619436)
\curveto(505.58147097,34.45618745)(505.58647096,34.43118747)(505.58647095,34.40119436)
\curveto(505.57647097,34.38118752)(505.57647097,34.35618755)(505.58647095,34.32619436)
\curveto(505.59647095,34.29618761)(505.60147095,34.26118764)(505.60147095,34.22119436)
\curveto(505.59147096,34.18118772)(505.58647096,34.14118776)(505.58647095,34.10119436)
\lineto(505.58647095,33.80119436)
\curveto(505.58647096,33.7011882)(505.56147099,33.62118828)(505.51147095,33.56119436)
\curveto(505.46147109,33.48118842)(505.39147116,33.42118848)(505.30147095,33.38119436)
\curveto(505.20147135,33.35118855)(505.10147145,33.31118859)(505.00147095,33.26119436)
\curveto(504.80147175,33.18118872)(504.59647195,33.1011888)(504.38647095,33.02119436)
\curveto(504.16647238,32.95118895)(503.95647259,32.87618903)(503.75647095,32.79619436)
\curveto(503.57647297,32.71618919)(503.39647315,32.64618926)(503.21647095,32.58619436)
\curveto(503.02647352,32.53618937)(502.84147371,32.47118943)(502.66147095,32.39119436)
\curveto(502.10147445,32.16118974)(501.53647501,31.94618996)(500.96647095,31.74619436)
\curveto(500.39647615,31.54619036)(499.83147672,31.33119057)(499.27147095,31.10119436)
\lineto(498.64147095,30.86119436)
\curveto(498.42147813,30.79119111)(498.21147834,30.71619119)(498.01147095,30.63619436)
\curveto(497.90147865,30.58619132)(497.79647875,30.54119136)(497.69647095,30.50119436)
\curveto(497.58647896,30.47119143)(497.49147906,30.42119148)(497.41147095,30.35119436)
\curveto(497.39147916,30.34119156)(497.38147917,30.33119157)(497.38147095,30.32119436)
\lineto(497.35147095,30.29119436)
\lineto(497.35147095,30.21619436)
\lineto(497.38147095,30.18619436)
\curveto(497.38147917,30.17619173)(497.38647916,30.16619174)(497.39647095,30.15619436)
\curveto(497.4464791,30.13619177)(497.50147905,30.12619178)(497.56147095,30.12619436)
\curveto(497.62147893,30.12619178)(497.68147887,30.11619179)(497.74147095,30.09619436)
\lineto(497.90647095,30.09619436)
\curveto(497.96647858,30.07619183)(498.03147852,30.07119183)(498.10147095,30.08119436)
\curveto(498.17147838,30.09119181)(498.24147831,30.09619181)(498.31147095,30.09619436)
\lineto(499.12147095,30.09619436)
\lineto(503.68147095,30.09619436)
\lineto(504.86647095,30.09619436)
\curveto(504.97647157,30.09619181)(505.08647146,30.09119181)(505.19647095,30.08119436)
\curveto(505.30647124,30.08119182)(505.39147116,30.05619185)(505.45147095,30.00619436)
\curveto(505.53147102,29.95619195)(505.57647097,29.86619204)(505.58647095,29.73619436)
\lineto(505.58647095,29.34619436)
\lineto(505.58647095,29.15119436)
\curveto(505.58647096,29.1011928)(505.57647097,29.05119285)(505.55647095,29.00119436)
\curveto(505.51647103,28.87119303)(505.43147112,28.79619311)(505.30147095,28.77619436)
\curveto(505.17147138,28.76619314)(505.02147153,28.76119314)(504.85147095,28.76119436)
\lineto(503.11147095,28.76119436)
\lineto(497.11147095,28.76119436)
\lineto(495.70147095,28.76119436)
\curveto(495.59148096,28.76119314)(495.47648107,28.75619315)(495.35647095,28.74619436)
\curveto(495.23648131,28.74619316)(495.14148141,28.77119313)(495.07147095,28.82119436)
\curveto(495.01148154,28.86119304)(494.96148159,28.93619297)(494.92147095,29.04619436)
\curveto(494.91148164,29.06619284)(494.91148164,29.08619282)(494.92147095,29.10619436)
\curveto(494.92148163,29.13619277)(494.91648163,29.16119274)(494.90647095,29.18119436)
}
}
{
\newrgbcolor{curcolor}{0 0 0}
\pscustom[linestyle=none,fillstyle=solid,fillcolor=curcolor]
{
\newpath
\moveto(505.03147095,48.38330373)
\curveto(505.19147136,48.4132959)(505.32647122,48.39829592)(505.43647095,48.33830373)
\curveto(505.53647101,48.27829604)(505.61147094,48.19829612)(505.66147095,48.09830373)
\curveto(505.68147087,48.04829627)(505.69147086,47.99329632)(505.69147095,47.93330373)
\curveto(505.69147086,47.88329643)(505.70147085,47.82829649)(505.72147095,47.76830373)
\curveto(505.77147078,47.54829677)(505.75647079,47.32829699)(505.67647095,47.10830373)
\curveto(505.60647094,46.89829742)(505.51647103,46.75329756)(505.40647095,46.67330373)
\curveto(505.33647121,46.62329769)(505.25647129,46.57829774)(505.16647095,46.53830373)
\curveto(505.06647148,46.49829782)(504.98647156,46.44829787)(504.92647095,46.38830373)
\curveto(504.90647164,46.36829795)(504.88647166,46.34329797)(504.86647095,46.31330373)
\curveto(504.8464717,46.29329802)(504.84147171,46.26329805)(504.85147095,46.22330373)
\curveto(504.88147167,46.1132982)(504.93647161,46.00829831)(505.01647095,45.90830373)
\curveto(505.09647145,45.8182985)(505.16647138,45.72829859)(505.22647095,45.63830373)
\curveto(505.30647124,45.50829881)(505.38147117,45.36829895)(505.45147095,45.21830373)
\curveto(505.51147104,45.06829925)(505.56647098,44.90829941)(505.61647095,44.73830373)
\curveto(505.6464709,44.63829968)(505.66647088,44.52829979)(505.67647095,44.40830373)
\curveto(505.68647086,44.29830002)(505.70147085,44.18830013)(505.72147095,44.07830373)
\curveto(505.73147082,44.02830029)(505.73647081,43.98330033)(505.73647095,43.94330373)
\lineto(505.73647095,43.83830373)
\curveto(505.75647079,43.72830059)(505.75647079,43.62330069)(505.73647095,43.52330373)
\lineto(505.73647095,43.38830373)
\curveto(505.72647082,43.33830098)(505.72147083,43.28830103)(505.72147095,43.23830373)
\curveto(505.72147083,43.18830113)(505.71147084,43.14330117)(505.69147095,43.10330373)
\curveto(505.68147087,43.06330125)(505.67647087,43.02830129)(505.67647095,42.99830373)
\curveto(505.68647086,42.97830134)(505.68647086,42.95330136)(505.67647095,42.92330373)
\lineto(505.61647095,42.68330373)
\curveto(505.60647094,42.60330171)(505.58647096,42.52830179)(505.55647095,42.45830373)
\curveto(505.42647112,42.15830216)(505.28147127,41.9133024)(505.12147095,41.72330373)
\curveto(504.9514716,41.54330277)(504.71647183,41.39330292)(504.41647095,41.27330373)
\curveto(504.19647235,41.18330313)(503.93147262,41.13830318)(503.62147095,41.13830373)
\lineto(503.30647095,41.13830373)
\curveto(503.25647329,41.14830317)(503.20647334,41.15330316)(503.15647095,41.15330373)
\lineto(502.97647095,41.18330373)
\lineto(502.64647095,41.30330373)
\curveto(502.53647401,41.34330297)(502.43647411,41.39330292)(502.34647095,41.45330373)
\curveto(502.05647449,41.63330268)(501.84147471,41.87830244)(501.70147095,42.18830373)
\curveto(501.56147499,42.49830182)(501.43647511,42.83830148)(501.32647095,43.20830373)
\curveto(501.28647526,43.34830097)(501.25647529,43.49330082)(501.23647095,43.64330373)
\curveto(501.21647533,43.79330052)(501.19147536,43.94330037)(501.16147095,44.09330373)
\curveto(501.14147541,44.16330015)(501.13147542,44.22830009)(501.13147095,44.28830373)
\curveto(501.13147542,44.35829996)(501.12147543,44.43329988)(501.10147095,44.51330373)
\curveto(501.08147547,44.58329973)(501.07147548,44.65329966)(501.07147095,44.72330373)
\curveto(501.06147549,44.79329952)(501.0464755,44.86829945)(501.02647095,44.94830373)
\curveto(500.96647558,45.19829912)(500.91647563,45.43329888)(500.87647095,45.65330373)
\curveto(500.82647572,45.87329844)(500.71147584,46.04829827)(500.53147095,46.17830373)
\curveto(500.4514761,46.23829808)(500.3514762,46.28829803)(500.23147095,46.32830373)
\curveto(500.10147645,46.36829795)(499.96147659,46.36829795)(499.81147095,46.32830373)
\curveto(499.57147698,46.26829805)(499.38147717,46.17829814)(499.24147095,46.05830373)
\curveto(499.10147745,45.94829837)(498.99147756,45.78829853)(498.91147095,45.57830373)
\curveto(498.86147769,45.45829886)(498.82647772,45.313299)(498.80647095,45.14330373)
\curveto(498.78647776,44.98329933)(498.77647777,44.8132995)(498.77647095,44.63330373)
\curveto(498.77647777,44.45329986)(498.78647776,44.27830004)(498.80647095,44.10830373)
\curveto(498.82647772,43.93830038)(498.85647769,43.79330052)(498.89647095,43.67330373)
\curveto(498.95647759,43.50330081)(499.04147751,43.33830098)(499.15147095,43.17830373)
\curveto(499.21147734,43.09830122)(499.29147726,43.02330129)(499.39147095,42.95330373)
\curveto(499.48147707,42.89330142)(499.58147697,42.83830148)(499.69147095,42.78830373)
\curveto(499.77147678,42.75830156)(499.85647669,42.72830159)(499.94647095,42.69830373)
\curveto(500.03647651,42.67830164)(500.10647644,42.63330168)(500.15647095,42.56330373)
\curveto(500.18647636,42.52330179)(500.21147634,42.45330186)(500.23147095,42.35330373)
\curveto(500.24147631,42.26330205)(500.2464763,42.16830215)(500.24647095,42.06830373)
\curveto(500.2464763,41.96830235)(500.24147631,41.86830245)(500.23147095,41.76830373)
\curveto(500.21147634,41.67830264)(500.18647636,41.6133027)(500.15647095,41.57330373)
\curveto(500.12647642,41.53330278)(500.07647647,41.50330281)(500.00647095,41.48330373)
\curveto(499.93647661,41.46330285)(499.86147669,41.46330285)(499.78147095,41.48330373)
\curveto(499.6514769,41.5133028)(499.53147702,41.54330277)(499.42147095,41.57330373)
\curveto(499.30147725,41.6133027)(499.18647736,41.65830266)(499.07647095,41.70830373)
\curveto(498.72647782,41.89830242)(498.45647809,42.13830218)(498.26647095,42.42830373)
\curveto(498.06647848,42.7183016)(497.90647864,43.07830124)(497.78647095,43.50830373)
\curveto(497.76647878,43.60830071)(497.7514788,43.70830061)(497.74147095,43.80830373)
\curveto(497.73147882,43.9183004)(497.71647883,44.02830029)(497.69647095,44.13830373)
\curveto(497.68647886,44.17830014)(497.68647886,44.24330007)(497.69647095,44.33330373)
\curveto(497.69647885,44.42329989)(497.68647886,44.47829984)(497.66647095,44.49830373)
\curveto(497.65647889,45.19829912)(497.73647881,45.80829851)(497.90647095,46.32830373)
\curveto(498.07647847,46.84829747)(498.40147815,47.2132971)(498.88147095,47.42330373)
\curveto(499.08147747,47.5132968)(499.31647723,47.56329675)(499.58647095,47.57330373)
\curveto(499.8464767,47.59329672)(500.12147643,47.60329671)(500.41147095,47.60330373)
\lineto(503.72647095,47.60330373)
\curveto(503.86647268,47.60329671)(504.00147255,47.60829671)(504.13147095,47.61830373)
\curveto(504.26147229,47.62829669)(504.36647218,47.65829666)(504.44647095,47.70830373)
\curveto(504.51647203,47.75829656)(504.56647198,47.82329649)(504.59647095,47.90330373)
\curveto(504.63647191,47.99329632)(504.66647188,48.07829624)(504.68647095,48.15830373)
\curveto(504.69647185,48.23829608)(504.74147181,48.29829602)(504.82147095,48.33830373)
\curveto(504.8514717,48.35829596)(504.88147167,48.36829595)(504.91147095,48.36830373)
\curveto(504.94147161,48.36829595)(504.98147157,48.37329594)(505.03147095,48.38330373)
\moveto(503.36647095,46.23830373)
\curveto(503.22647332,46.29829802)(503.06647348,46.32829799)(502.88647095,46.32830373)
\curveto(502.69647385,46.33829798)(502.50147405,46.34329797)(502.30147095,46.34330373)
\curveto(502.19147436,46.34329797)(502.09147446,46.33829798)(502.00147095,46.32830373)
\curveto(501.91147464,46.318298)(501.84147471,46.27829804)(501.79147095,46.20830373)
\curveto(501.77147478,46.17829814)(501.76147479,46.10829821)(501.76147095,45.99830373)
\curveto(501.78147477,45.97829834)(501.79147476,45.94329837)(501.79147095,45.89330373)
\curveto(501.79147476,45.84329847)(501.80147475,45.79829852)(501.82147095,45.75830373)
\curveto(501.84147471,45.67829864)(501.86147469,45.58829873)(501.88147095,45.48830373)
\lineto(501.94147095,45.18830373)
\curveto(501.94147461,45.15829916)(501.9464746,45.12329919)(501.95647095,45.08330373)
\lineto(501.95647095,44.97830373)
\curveto(501.99647455,44.82829949)(502.02147453,44.66329965)(502.03147095,44.48330373)
\curveto(502.03147452,44.3133)(502.0514745,44.15330016)(502.09147095,44.00330373)
\curveto(502.11147444,43.92330039)(502.13147442,43.84830047)(502.15147095,43.77830373)
\curveto(502.16147439,43.7183006)(502.17647437,43.64830067)(502.19647095,43.56830373)
\curveto(502.2464743,43.40830091)(502.31147424,43.25830106)(502.39147095,43.11830373)
\curveto(502.46147409,42.97830134)(502.551474,42.85830146)(502.66147095,42.75830373)
\curveto(502.77147378,42.65830166)(502.90647364,42.58330173)(503.06647095,42.53330373)
\curveto(503.21647333,42.48330183)(503.40147315,42.46330185)(503.62147095,42.47330373)
\curveto(503.72147283,42.47330184)(503.81647273,42.48830183)(503.90647095,42.51830373)
\curveto(503.98647256,42.55830176)(504.06147249,42.60330171)(504.13147095,42.65330373)
\curveto(504.24147231,42.73330158)(504.33647221,42.83830148)(504.41647095,42.96830373)
\curveto(504.48647206,43.09830122)(504.546472,43.23830108)(504.59647095,43.38830373)
\curveto(504.60647194,43.43830088)(504.61147194,43.48830083)(504.61147095,43.53830373)
\curveto(504.61147194,43.58830073)(504.61647193,43.63830068)(504.62647095,43.68830373)
\curveto(504.6464719,43.75830056)(504.66147189,43.84330047)(504.67147095,43.94330373)
\curveto(504.67147188,44.05330026)(504.66147189,44.14330017)(504.64147095,44.21330373)
\curveto(504.62147193,44.27330004)(504.61647193,44.33329998)(504.62647095,44.39330373)
\curveto(504.62647192,44.45329986)(504.61647193,44.5132998)(504.59647095,44.57330373)
\curveto(504.57647197,44.65329966)(504.56147199,44.72829959)(504.55147095,44.79830373)
\curveto(504.54147201,44.87829944)(504.52147203,44.95329936)(504.49147095,45.02330373)
\curveto(504.37147218,45.313299)(504.22647232,45.55829876)(504.05647095,45.75830373)
\curveto(503.88647266,45.96829835)(503.65647289,46.12829819)(503.36647095,46.23830373)
}
}
{
\newrgbcolor{curcolor}{0 0 0}
\pscustom[linestyle=none,fillstyle=solid,fillcolor=curcolor]
{
\newpath
\moveto(497.86147095,49.26994436)
\lineto(497.86147095,49.71994436)
\curveto(497.8514787,49.88994311)(497.87147868,50.01494298)(497.92147095,50.09494436)
\curveto(497.97147858,50.17494282)(498.03647851,50.22994277)(498.11647095,50.25994436)
\curveto(498.19647835,50.2999427)(498.28147827,50.33994266)(498.37147095,50.37994436)
\curveto(498.50147805,50.42994257)(498.63147792,50.47494252)(498.76147095,50.51494436)
\curveto(498.89147766,50.55494244)(499.02147753,50.5999424)(499.15147095,50.64994436)
\curveto(499.27147728,50.6999423)(499.39647715,50.74494225)(499.52647095,50.78494436)
\curveto(499.6464769,50.82494217)(499.76647678,50.86994213)(499.88647095,50.91994436)
\curveto(499.99647655,50.96994203)(500.11147644,51.00994199)(500.23147095,51.03994436)
\curveto(500.3514762,51.06994193)(500.47147608,51.10994189)(500.59147095,51.15994436)
\curveto(500.88147567,51.27994172)(501.18147537,51.38994161)(501.49147095,51.48994436)
\curveto(501.80147475,51.58994141)(502.10147445,51.6999413)(502.39147095,51.81994436)
\curveto(502.43147412,51.83994116)(502.47147408,51.84994115)(502.51147095,51.84994436)
\curveto(502.54147401,51.84994115)(502.57147398,51.85994114)(502.60147095,51.87994436)
\curveto(502.74147381,51.93994106)(502.88647366,51.994941)(503.03647095,52.04494436)
\lineto(503.45647095,52.19494436)
\curveto(503.52647302,52.22494077)(503.60147295,52.25494074)(503.68147095,52.28494436)
\curveto(503.7514728,52.31494068)(503.79647275,52.36494063)(503.81647095,52.43494436)
\curveto(503.8464727,52.51494048)(503.82147273,52.57494042)(503.74147095,52.61494436)
\curveto(503.6514729,52.66494033)(503.58147297,52.6999403)(503.53147095,52.71994436)
\curveto(503.36147319,52.7999402)(503.18147337,52.86494013)(502.99147095,52.91494436)
\curveto(502.80147375,52.96494003)(502.61647393,53.02493997)(502.43647095,53.09494436)
\curveto(502.20647434,53.18493981)(501.97647457,53.26493973)(501.74647095,53.33494436)
\curveto(501.50647504,53.40493959)(501.27647527,53.48993951)(501.05647095,53.58994436)
\curveto(501.00647554,53.5999394)(500.94147561,53.61493938)(500.86147095,53.63494436)
\curveto(500.77147578,53.67493932)(500.68147587,53.70993929)(500.59147095,53.73994436)
\curveto(500.49147606,53.76993923)(500.40147615,53.7999392)(500.32147095,53.82994436)
\curveto(500.27147628,53.84993915)(500.22647632,53.86493913)(500.18647095,53.87494436)
\curveto(500.1464764,53.88493911)(500.10147645,53.8999391)(500.05147095,53.91994436)
\curveto(499.93147662,53.96993903)(499.81147674,54.00993899)(499.69147095,54.03994436)
\curveto(499.56147699,54.07993892)(499.43647711,54.12493887)(499.31647095,54.17494436)
\curveto(499.26647728,54.1949388)(499.22147733,54.20993879)(499.18147095,54.21994436)
\curveto(499.14147741,54.22993877)(499.09647745,54.24493875)(499.04647095,54.26494436)
\curveto(498.95647759,54.30493869)(498.86647768,54.33993866)(498.77647095,54.36994436)
\curveto(498.67647787,54.3999386)(498.58147797,54.42993857)(498.49147095,54.45994436)
\curveto(498.41147814,54.48993851)(498.33147822,54.51493848)(498.25147095,54.53494436)
\curveto(498.16147839,54.56493843)(498.08647846,54.60493839)(498.02647095,54.65494436)
\curveto(497.93647861,54.72493827)(497.88647866,54.81993818)(497.87647095,54.93994436)
\curveto(497.86647868,55.06993793)(497.86147869,55.20993779)(497.86147095,55.35994436)
\curveto(497.86147869,55.43993756)(497.86647868,55.51493748)(497.87647095,55.58494436)
\curveto(497.87647867,55.66493733)(497.89147866,55.72993727)(497.92147095,55.77994436)
\curveto(497.98147857,55.86993713)(498.07647847,55.8949371)(498.20647095,55.85494436)
\curveto(498.33647821,55.81493718)(498.43647811,55.77993722)(498.50647095,55.74994436)
\lineto(498.56647095,55.71994436)
\curveto(498.58647796,55.71993728)(498.60647794,55.71493728)(498.62647095,55.70494436)
\curveto(498.90647764,55.5949374)(499.19147736,55.48493751)(499.48147095,55.37494436)
\lineto(500.32147095,55.04494436)
\curveto(500.40147615,55.01493798)(500.47647607,54.98993801)(500.54647095,54.96994436)
\curveto(500.60647594,54.94993805)(500.67147588,54.92493807)(500.74147095,54.89494436)
\curveto(500.94147561,54.81493818)(501.1464754,54.73493826)(501.35647095,54.65494436)
\curveto(501.55647499,54.58493841)(501.75647479,54.50993849)(501.95647095,54.42994436)
\curveto(502.6464739,54.13993886)(503.34147321,53.86993913)(504.04147095,53.61994436)
\curveto(504.74147181,53.36993963)(505.43647111,53.0999399)(506.12647095,52.80994436)
\lineto(506.27647095,52.74994436)
\curveto(506.33647021,52.73994026)(506.39647015,52.72494027)(506.45647095,52.70494436)
\curveto(506.82646972,52.54494045)(507.19146936,52.37494062)(507.55147095,52.19494436)
\curveto(507.92146863,52.01494098)(508.20646834,51.76494123)(508.40647095,51.44494436)
\curveto(508.46646808,51.33494166)(508.51146804,51.22494177)(508.54147095,51.11494436)
\curveto(508.58146797,51.00494199)(508.61646793,50.87994212)(508.64647095,50.73994436)
\curveto(508.66646788,50.68994231)(508.67146788,50.63494236)(508.66147095,50.57494436)
\curveto(508.6514679,50.52494247)(508.6514679,50.46994253)(508.66147095,50.40994436)
\curveto(508.68146787,50.32994267)(508.68146787,50.24994275)(508.66147095,50.16994436)
\curveto(508.6514679,50.12994287)(508.6464679,50.07994292)(508.64647095,50.01994436)
\lineto(508.58647095,49.77994436)
\curveto(508.56646798,49.70994329)(508.52646802,49.65494334)(508.46647095,49.61494436)
\curveto(508.40646814,49.56494343)(508.33146822,49.53494346)(508.24147095,49.52494436)
\lineto(507.97147095,49.52494436)
\lineto(507.76147095,49.52494436)
\curveto(507.70146885,49.53494346)(507.6514689,49.55494344)(507.61147095,49.58494436)
\curveto(507.50146905,49.65494334)(507.47146908,49.77494322)(507.52147095,49.94494436)
\curveto(507.54146901,50.05494294)(507.551469,50.17494282)(507.55147095,50.30494436)
\curveto(507.551469,50.43494256)(507.53146902,50.54994245)(507.49147095,50.64994436)
\curveto(507.44146911,50.7999422)(507.36646918,50.91994208)(507.26647095,51.00994436)
\curveto(507.16646938,51.10994189)(507.0514695,51.1949418)(506.92147095,51.26494436)
\curveto(506.80146975,51.33494166)(506.67146988,51.3949416)(506.53147095,51.44494436)
\lineto(506.11147095,51.62494436)
\curveto(506.02147053,51.66494133)(505.91147064,51.70494129)(505.78147095,51.74494436)
\curveto(505.6514709,51.7949412)(505.51647103,51.7999412)(505.37647095,51.75994436)
\curveto(505.21647133,51.70994129)(505.06647148,51.65494134)(504.92647095,51.59494436)
\curveto(504.78647176,51.54494145)(504.6464719,51.48994151)(504.50647095,51.42994436)
\curveto(504.29647225,51.33994166)(504.08647246,51.25494174)(503.87647095,51.17494436)
\curveto(503.66647288,51.0949419)(503.46147309,51.01494198)(503.26147095,50.93494436)
\curveto(503.12147343,50.87494212)(502.98647356,50.81994218)(502.85647095,50.76994436)
\curveto(502.72647382,50.71994228)(502.59147396,50.66994233)(502.45147095,50.61994436)
\lineto(501.13147095,50.07994436)
\curveto(500.69147586,49.90994309)(500.2514763,49.73494326)(499.81147095,49.55494436)
\curveto(499.58147697,49.45494354)(499.36147719,49.36494363)(499.15147095,49.28494436)
\curveto(498.93147762,49.20494379)(498.71147784,49.11994388)(498.49147095,49.02994436)
\curveto(498.43147812,49.00994399)(498.3514782,48.97994402)(498.25147095,48.93994436)
\curveto(498.14147841,48.8999441)(498.0514785,48.90494409)(497.98147095,48.95494436)
\curveto(497.93147862,48.98494401)(497.89647865,49.04494395)(497.87647095,49.13494436)
\curveto(497.86647868,49.15494384)(497.86647868,49.17494382)(497.87647095,49.19494436)
\curveto(497.87647867,49.22494377)(497.87147868,49.24994375)(497.86147095,49.26994436)
}
}
{
\newrgbcolor{curcolor}{0 0 0}
\pscustom[linestyle=none,fillstyle=solid,fillcolor=curcolor]
{
}
}
{
\newrgbcolor{curcolor}{0 0 0}
\pscustom[linestyle=none,fillstyle=solid,fillcolor=curcolor]
{
\newpath
\moveto(500.50147095,67.99510061)
\lineto(500.75647095,67.99510061)
\curveto(500.83647571,68.0050929)(500.91147564,68.00009291)(500.98147095,67.98010061)
\lineto(501.22147095,67.98010061)
\lineto(501.38647095,67.98010061)
\curveto(501.48647506,67.96009295)(501.59147496,67.95009296)(501.70147095,67.95010061)
\curveto(501.80147475,67.95009296)(501.90147465,67.94009297)(502.00147095,67.92010061)
\lineto(502.15147095,67.92010061)
\curveto(502.29147426,67.89009302)(502.43147412,67.87009304)(502.57147095,67.86010061)
\curveto(502.70147385,67.85009306)(502.83147372,67.82509308)(502.96147095,67.78510061)
\curveto(503.04147351,67.76509314)(503.12647342,67.74509316)(503.21647095,67.72510061)
\lineto(503.45647095,67.66510061)
\lineto(503.75647095,67.54510061)
\curveto(503.8464727,67.51509339)(503.93647261,67.48009343)(504.02647095,67.44010061)
\curveto(504.2464723,67.34009357)(504.46147209,67.2050937)(504.67147095,67.03510061)
\curveto(504.88147167,66.87509403)(505.0514715,66.70009421)(505.18147095,66.51010061)
\curveto(505.22147133,66.46009445)(505.26147129,66.40009451)(505.30147095,66.33010061)
\curveto(505.33147122,66.27009464)(505.36647118,66.2100947)(505.40647095,66.15010061)
\curveto(505.45647109,66.07009484)(505.49647105,65.97509493)(505.52647095,65.86510061)
\curveto(505.55647099,65.75509515)(505.58647096,65.65009526)(505.61647095,65.55010061)
\curveto(505.65647089,65.44009547)(505.68147087,65.33009558)(505.69147095,65.22010061)
\curveto(505.70147085,65.1100958)(505.71647083,64.99509591)(505.73647095,64.87510061)
\curveto(505.7464708,64.83509607)(505.7464708,64.79009612)(505.73647095,64.74010061)
\curveto(505.73647081,64.70009621)(505.74147081,64.66009625)(505.75147095,64.62010061)
\curveto(505.76147079,64.58009633)(505.76647078,64.52509638)(505.76647095,64.45510061)
\curveto(505.76647078,64.38509652)(505.76147079,64.33509657)(505.75147095,64.30510061)
\curveto(505.73147082,64.25509665)(505.72647082,64.2100967)(505.73647095,64.17010061)
\curveto(505.7464708,64.13009678)(505.7464708,64.09509681)(505.73647095,64.06510061)
\lineto(505.73647095,63.97510061)
\curveto(505.71647083,63.91509699)(505.70147085,63.85009706)(505.69147095,63.78010061)
\curveto(505.69147086,63.72009719)(505.68647086,63.65509725)(505.67647095,63.58510061)
\curveto(505.62647092,63.41509749)(505.57647097,63.25509765)(505.52647095,63.10510061)
\curveto(505.47647107,62.95509795)(505.41147114,62.8100981)(505.33147095,62.67010061)
\curveto(505.29147126,62.62009829)(505.26147129,62.56509834)(505.24147095,62.50510061)
\curveto(505.21147134,62.45509845)(505.17647137,62.4050985)(505.13647095,62.35510061)
\curveto(504.95647159,62.11509879)(504.73647181,61.91509899)(504.47647095,61.75510061)
\curveto(504.21647233,61.59509931)(503.93147262,61.45509945)(503.62147095,61.33510061)
\curveto(503.48147307,61.27509963)(503.34147321,61.23009968)(503.20147095,61.20010061)
\curveto(503.0514735,61.17009974)(502.89647365,61.13509977)(502.73647095,61.09510061)
\curveto(502.62647392,61.07509983)(502.51647403,61.06009985)(502.40647095,61.05010061)
\curveto(502.29647425,61.04009987)(502.18647436,61.02509988)(502.07647095,61.00510061)
\curveto(502.03647451,60.99509991)(501.99647455,60.99009992)(501.95647095,60.99010061)
\curveto(501.91647463,61.00009991)(501.87647467,61.00009991)(501.83647095,60.99010061)
\curveto(501.78647476,60.98009993)(501.73647481,60.97509993)(501.68647095,60.97510061)
\lineto(501.52147095,60.97510061)
\curveto(501.47147508,60.95509995)(501.42147513,60.95009996)(501.37147095,60.96010061)
\curveto(501.31147524,60.97009994)(501.25647529,60.97009994)(501.20647095,60.96010061)
\curveto(501.16647538,60.95009996)(501.12147543,60.95009996)(501.07147095,60.96010061)
\curveto(501.02147553,60.97009994)(500.97147558,60.96509994)(500.92147095,60.94510061)
\curveto(500.8514757,60.92509998)(500.77647577,60.92009999)(500.69647095,60.93010061)
\curveto(500.60647594,60.94009997)(500.52147603,60.94509996)(500.44147095,60.94510061)
\curveto(500.3514762,60.94509996)(500.2514763,60.94009997)(500.14147095,60.93010061)
\curveto(500.02147653,60.92009999)(499.92147663,60.92509998)(499.84147095,60.94510061)
\lineto(499.55647095,60.94510061)
\lineto(498.92647095,60.99010061)
\curveto(498.82647772,61.00009991)(498.73147782,61.0100999)(498.64147095,61.02010061)
\lineto(498.34147095,61.05010061)
\curveto(498.29147826,61.07009984)(498.24147831,61.07509983)(498.19147095,61.06510061)
\curveto(498.13147842,61.06509984)(498.07647847,61.07509983)(498.02647095,61.09510061)
\curveto(497.85647869,61.14509976)(497.69147886,61.18509972)(497.53147095,61.21510061)
\curveto(497.36147919,61.24509966)(497.20147935,61.29509961)(497.05147095,61.36510061)
\curveto(496.59147996,61.55509935)(496.21648033,61.77509913)(495.92647095,62.02510061)
\curveto(495.63648091,62.28509862)(495.39148116,62.64509826)(495.19147095,63.10510061)
\curveto(495.14148141,63.23509767)(495.10648144,63.36509754)(495.08647095,63.49510061)
\curveto(495.06648148,63.63509727)(495.04148151,63.77509713)(495.01147095,63.91510061)
\curveto(495.00148155,63.98509692)(494.99648155,64.05009686)(494.99647095,64.11010061)
\curveto(494.99648155,64.17009674)(494.99148156,64.23509667)(494.98147095,64.30510061)
\curveto(494.96148159,65.13509577)(495.11148144,65.8050951)(495.43147095,66.31510061)
\curveto(495.74148081,66.82509408)(496.18148037,67.2050937)(496.75147095,67.45510061)
\curveto(496.87147968,67.5050934)(496.99647955,67.55009336)(497.12647095,67.59010061)
\curveto(497.25647929,67.63009328)(497.39147916,67.67509323)(497.53147095,67.72510061)
\curveto(497.61147894,67.74509316)(497.69647885,67.76009315)(497.78647095,67.77010061)
\lineto(498.02647095,67.83010061)
\curveto(498.13647841,67.86009305)(498.2464783,67.87509303)(498.35647095,67.87510061)
\curveto(498.46647808,67.88509302)(498.57647797,67.90009301)(498.68647095,67.92010061)
\curveto(498.73647781,67.94009297)(498.78147777,67.94509296)(498.82147095,67.93510061)
\curveto(498.86147769,67.93509297)(498.90147765,67.94009297)(498.94147095,67.95010061)
\curveto(498.99147756,67.96009295)(499.0464775,67.96009295)(499.10647095,67.95010061)
\curveto(499.15647739,67.95009296)(499.20647734,67.95509295)(499.25647095,67.96510061)
\lineto(499.39147095,67.96510061)
\curveto(499.4514771,67.98509292)(499.52147703,67.98509292)(499.60147095,67.96510061)
\curveto(499.67147688,67.95509295)(499.73647681,67.96009295)(499.79647095,67.98010061)
\curveto(499.82647672,67.99009292)(499.86647668,67.99509291)(499.91647095,67.99510061)
\lineto(500.03647095,67.99510061)
\lineto(500.50147095,67.99510061)
\moveto(502.82647095,66.45010061)
\curveto(502.50647404,66.55009436)(502.14147441,66.6100943)(501.73147095,66.63010061)
\curveto(501.32147523,66.65009426)(500.91147564,66.66009425)(500.50147095,66.66010061)
\curveto(500.07147648,66.66009425)(499.6514769,66.65009426)(499.24147095,66.63010061)
\curveto(498.83147772,66.6100943)(498.4464781,66.56509434)(498.08647095,66.49510061)
\curveto(497.72647882,66.42509448)(497.40647914,66.31509459)(497.12647095,66.16510061)
\curveto(496.83647971,66.02509488)(496.60147995,65.83009508)(496.42147095,65.58010061)
\curveto(496.31148024,65.42009549)(496.23148032,65.24009567)(496.18147095,65.04010061)
\curveto(496.12148043,64.84009607)(496.09148046,64.59509631)(496.09147095,64.30510061)
\curveto(496.11148044,64.28509662)(496.12148043,64.25009666)(496.12147095,64.20010061)
\curveto(496.11148044,64.15009676)(496.11148044,64.1100968)(496.12147095,64.08010061)
\curveto(496.14148041,64.00009691)(496.16148039,63.92509698)(496.18147095,63.85510061)
\curveto(496.19148036,63.79509711)(496.21148034,63.73009718)(496.24147095,63.66010061)
\curveto(496.36148019,63.39009752)(496.53148002,63.17009774)(496.75147095,63.00010061)
\curveto(496.96147959,62.84009807)(497.20647934,62.7050982)(497.48647095,62.59510061)
\curveto(497.59647895,62.54509836)(497.71647883,62.5050984)(497.84647095,62.47510061)
\curveto(497.96647858,62.45509845)(498.09147846,62.43009848)(498.22147095,62.40010061)
\curveto(498.27147828,62.38009853)(498.32647822,62.37009854)(498.38647095,62.37010061)
\curveto(498.43647811,62.37009854)(498.48647806,62.36509854)(498.53647095,62.35510061)
\curveto(498.62647792,62.34509856)(498.72147783,62.33509857)(498.82147095,62.32510061)
\curveto(498.91147764,62.31509859)(499.00647754,62.3050986)(499.10647095,62.29510061)
\curveto(499.18647736,62.29509861)(499.27147728,62.29009862)(499.36147095,62.28010061)
\lineto(499.60147095,62.28010061)
\lineto(499.78147095,62.28010061)
\curveto(499.81147674,62.27009864)(499.8464767,62.26509864)(499.88647095,62.26510061)
\lineto(500.02147095,62.26510061)
\lineto(500.47147095,62.26510061)
\curveto(500.551476,62.26509864)(500.63647591,62.26009865)(500.72647095,62.25010061)
\curveto(500.80647574,62.25009866)(500.88147567,62.26009865)(500.95147095,62.28010061)
\lineto(501.22147095,62.28010061)
\curveto(501.24147531,62.28009863)(501.27147528,62.27509863)(501.31147095,62.26510061)
\curveto(501.34147521,62.26509864)(501.36647518,62.27009864)(501.38647095,62.28010061)
\curveto(501.48647506,62.29009862)(501.58647496,62.29509861)(501.68647095,62.29510061)
\curveto(501.77647477,62.3050986)(501.87647467,62.31509859)(501.98647095,62.32510061)
\curveto(502.10647444,62.35509855)(502.23147432,62.37009854)(502.36147095,62.37010061)
\curveto(502.48147407,62.38009853)(502.59647395,62.4050985)(502.70647095,62.44510061)
\curveto(503.00647354,62.52509838)(503.27147328,62.6100983)(503.50147095,62.70010061)
\curveto(503.73147282,62.80009811)(503.9464726,62.94509796)(504.14647095,63.13510061)
\curveto(504.3464722,63.34509756)(504.49647205,63.6100973)(504.59647095,63.93010061)
\curveto(504.61647193,63.97009694)(504.62647192,64.0050969)(504.62647095,64.03510061)
\curveto(504.61647193,64.07509683)(504.62147193,64.12009679)(504.64147095,64.17010061)
\curveto(504.6514719,64.2100967)(504.66147189,64.28009663)(504.67147095,64.38010061)
\curveto(504.68147187,64.49009642)(504.67647187,64.57509633)(504.65647095,64.63510061)
\curveto(504.63647191,64.7050962)(504.62647192,64.77509613)(504.62647095,64.84510061)
\curveto(504.61647193,64.91509599)(504.60147195,64.98009593)(504.58147095,65.04010061)
\curveto(504.52147203,65.24009567)(504.43647211,65.42009549)(504.32647095,65.58010061)
\curveto(504.30647224,65.6100953)(504.28647226,65.63509527)(504.26647095,65.65510061)
\lineto(504.20647095,65.71510061)
\curveto(504.18647236,65.75509515)(504.1464724,65.8050951)(504.08647095,65.86510061)
\curveto(503.9464726,65.96509494)(503.81647273,66.05009486)(503.69647095,66.12010061)
\curveto(503.57647297,66.19009472)(503.43147312,66.26009465)(503.26147095,66.33010061)
\curveto(503.19147336,66.36009455)(503.12147343,66.38009453)(503.05147095,66.39010061)
\curveto(502.98147357,66.4100945)(502.90647364,66.43009448)(502.82647095,66.45010061)
}
}
{
\newrgbcolor{curcolor}{0 0 0}
\pscustom[linestyle=none,fillstyle=solid,fillcolor=curcolor]
{
\newpath
\moveto(495.17647095,70.85470998)
\lineto(495.17647095,74.45470998)
\lineto(495.17647095,75.09970998)
\curveto(495.17648137,75.17970345)(495.18148137,75.25470338)(495.19147095,75.32470998)
\curveto(495.19148136,75.39470324)(495.20148135,75.45470318)(495.22147095,75.50470998)
\curveto(495.2514813,75.57470306)(495.31148124,75.629703)(495.40147095,75.66970998)
\curveto(495.43148112,75.68970294)(495.47148108,75.69970293)(495.52147095,75.69970998)
\lineto(495.65647095,75.69970998)
\curveto(495.76648078,75.70970292)(495.87148068,75.70470293)(495.97147095,75.68470998)
\curveto(496.07148048,75.67470296)(496.14148041,75.63970299)(496.18147095,75.57970998)
\curveto(496.2514803,75.48970314)(496.28648026,75.35470328)(496.28647095,75.17470998)
\curveto(496.27648027,74.99470364)(496.27148028,74.8297038)(496.27147095,74.67970998)
\lineto(496.27147095,72.68470998)
\lineto(496.27147095,72.18970998)
\lineto(496.27147095,72.05470998)
\curveto(496.27148028,72.01470662)(496.27648027,71.97470666)(496.28647095,71.93470998)
\lineto(496.28647095,71.72470998)
\curveto(496.31648023,71.61470702)(496.35648019,71.5347071)(496.40647095,71.48470998)
\curveto(496.4464801,71.4347072)(496.50148005,71.39970723)(496.57147095,71.37970998)
\curveto(496.63147992,71.35970727)(496.70147985,71.34470729)(496.78147095,71.33470998)
\curveto(496.86147969,71.32470731)(496.9514796,71.30470733)(497.05147095,71.27470998)
\curveto(497.2514793,71.22470741)(497.45647909,71.18470745)(497.66647095,71.15470998)
\curveto(497.87647867,71.12470751)(498.08147847,71.08470755)(498.28147095,71.03470998)
\curveto(498.3514782,71.01470762)(498.42147813,71.00470763)(498.49147095,71.00470998)
\curveto(498.551478,71.00470763)(498.61647793,70.99470764)(498.68647095,70.97470998)
\curveto(498.71647783,70.96470767)(498.75647779,70.95470768)(498.80647095,70.94470998)
\curveto(498.8464777,70.94470769)(498.88647766,70.94970768)(498.92647095,70.95970998)
\curveto(498.97647757,70.97970765)(499.02147753,71.00470763)(499.06147095,71.03470998)
\curveto(499.09147746,71.07470756)(499.09647745,71.1347075)(499.07647095,71.21470998)
\curveto(499.05647749,71.27470736)(499.03147752,71.3347073)(499.00147095,71.39470998)
\curveto(498.96147759,71.45470718)(498.92647762,71.51470712)(498.89647095,71.57470998)
\curveto(498.87647767,71.634707)(498.86147769,71.68470695)(498.85147095,71.72470998)
\curveto(498.77147778,71.91470672)(498.71647783,72.11970651)(498.68647095,72.33970998)
\curveto(498.65647789,72.56970606)(498.6464779,72.79970583)(498.65647095,73.02970998)
\curveto(498.65647789,73.26970536)(498.68147787,73.49970513)(498.73147095,73.71970998)
\curveto(498.77147778,73.93970469)(498.83147772,74.13970449)(498.91147095,74.31970998)
\curveto(498.93147762,74.36970426)(498.9514776,74.41470422)(498.97147095,74.45470998)
\curveto(498.99147756,74.50470413)(499.01647753,74.55470408)(499.04647095,74.60470998)
\curveto(499.25647729,74.95470368)(499.48647706,75.2347034)(499.73647095,75.44470998)
\curveto(499.98647656,75.66470297)(500.31147624,75.85970277)(500.71147095,76.02970998)
\curveto(500.82147573,76.07970255)(500.93147562,76.11470252)(501.04147095,76.13470998)
\curveto(501.1514754,76.15470248)(501.26647528,76.17970245)(501.38647095,76.20970998)
\curveto(501.41647513,76.21970241)(501.46147509,76.22470241)(501.52147095,76.22470998)
\curveto(501.58147497,76.24470239)(501.6514749,76.25470238)(501.73147095,76.25470998)
\curveto(501.80147475,76.25470238)(501.86647468,76.26470237)(501.92647095,76.28470998)
\lineto(502.09147095,76.28470998)
\curveto(502.14147441,76.29470234)(502.21147434,76.29970233)(502.30147095,76.29970998)
\curveto(502.39147416,76.29970233)(502.46147409,76.28970234)(502.51147095,76.26970998)
\curveto(502.57147398,76.24970238)(502.63147392,76.24470239)(502.69147095,76.25470998)
\curveto(502.74147381,76.26470237)(502.79147376,76.25970237)(502.84147095,76.23970998)
\curveto(503.00147355,76.19970243)(503.1514734,76.16470247)(503.29147095,76.13470998)
\curveto(503.43147312,76.10470253)(503.56647298,76.05970257)(503.69647095,75.99970998)
\curveto(504.06647248,75.83970279)(504.40147215,75.61970301)(504.70147095,75.33970998)
\curveto(505.00147155,75.05970357)(505.23147132,74.73970389)(505.39147095,74.37970998)
\curveto(505.47147108,74.20970442)(505.546471,74.00970462)(505.61647095,73.77970998)
\curveto(505.65647089,73.66970496)(505.68147087,73.55470508)(505.69147095,73.43470998)
\curveto(505.70147085,73.31470532)(505.72147083,73.19470544)(505.75147095,73.07470998)
\curveto(505.77147078,73.02470561)(505.77147078,72.96970566)(505.75147095,72.90970998)
\curveto(505.74147081,72.84970578)(505.7464708,72.78970584)(505.76647095,72.72970998)
\curveto(505.78647076,72.629706)(505.78647076,72.5297061)(505.76647095,72.42970998)
\lineto(505.76647095,72.29470998)
\curveto(505.7464708,72.24470639)(505.73647081,72.18470645)(505.73647095,72.11470998)
\curveto(505.7464708,72.05470658)(505.74147081,71.99970663)(505.72147095,71.94970998)
\curveto(505.71147084,71.90970672)(505.70647084,71.87470676)(505.70647095,71.84470998)
\curveto(505.70647084,71.81470682)(505.70147085,71.77970685)(505.69147095,71.73970998)
\lineto(505.63147095,71.46970998)
\curveto(505.61147094,71.37970725)(505.58147097,71.29470734)(505.54147095,71.21470998)
\curveto(505.40147115,70.87470776)(505.2464713,70.58470805)(505.07647095,70.34470998)
\curveto(504.89647165,70.10470853)(504.66647188,69.88470875)(504.38647095,69.68470998)
\curveto(504.15647239,69.5347091)(503.91647263,69.41970921)(503.66647095,69.33970998)
\curveto(503.61647293,69.31970931)(503.57147298,69.30970932)(503.53147095,69.30970998)
\curveto(503.48147307,69.30970932)(503.43147312,69.29970933)(503.38147095,69.27970998)
\curveto(503.32147323,69.25970937)(503.24147331,69.24470939)(503.14147095,69.23470998)
\curveto(503.04147351,69.2347094)(502.96647358,69.25470938)(502.91647095,69.29470998)
\curveto(502.83647371,69.34470929)(502.79147376,69.42470921)(502.78147095,69.53470998)
\curveto(502.77147378,69.64470899)(502.76647378,69.75970887)(502.76647095,69.87970998)
\lineto(502.76647095,70.04470998)
\curveto(502.76647378,70.10470853)(502.77647377,70.15970847)(502.79647095,70.20970998)
\curveto(502.81647373,70.29970833)(502.85647369,70.36970826)(502.91647095,70.41970998)
\curveto(503.00647354,70.48970814)(503.11647343,70.5347081)(503.24647095,70.55470998)
\curveto(503.36647318,70.58470805)(503.47147308,70.629708)(503.56147095,70.68970998)
\curveto(503.90147265,70.87970775)(504.17147238,71.13970749)(504.37147095,71.46970998)
\curveto(504.43147212,71.56970706)(504.48147207,71.67470696)(504.52147095,71.78470998)
\curveto(504.551472,71.90470673)(504.58647196,72.02470661)(504.62647095,72.14470998)
\curveto(504.67647187,72.31470632)(504.69647185,72.51970611)(504.68647095,72.75970998)
\curveto(504.66647188,73.00970562)(504.63147192,73.20970542)(504.58147095,73.35970998)
\curveto(504.46147209,73.7297049)(504.30147225,74.01970461)(504.10147095,74.22970998)
\curveto(503.89147266,74.44970418)(503.61147294,74.629704)(503.26147095,74.76970998)
\curveto(503.16147339,74.81970381)(503.05647349,74.84970378)(502.94647095,74.85970998)
\curveto(502.83647371,74.87970375)(502.72147383,74.90470373)(502.60147095,74.93470998)
\lineto(502.49647095,74.93470998)
\curveto(502.45647409,74.94470369)(502.41647413,74.94970368)(502.37647095,74.94970998)
\curveto(502.3464742,74.95970367)(502.31147424,74.95970367)(502.27147095,74.94970998)
\lineto(502.15147095,74.94970998)
\curveto(501.89147466,74.94970368)(501.6464749,74.91970371)(501.41647095,74.85970998)
\curveto(501.06647548,74.74970388)(500.77147578,74.59470404)(500.53147095,74.39470998)
\curveto(500.28147627,74.19470444)(500.08647646,73.9347047)(499.94647095,73.61470998)
\lineto(499.88647095,73.43470998)
\curveto(499.86647668,73.38470525)(499.8464767,73.32470531)(499.82647095,73.25470998)
\curveto(499.80647674,73.20470543)(499.79647675,73.14470549)(499.79647095,73.07470998)
\curveto(499.78647676,73.01470562)(499.77147678,72.94970568)(499.75147095,72.87970998)
\lineto(499.75147095,72.72970998)
\curveto(499.73147682,72.68970594)(499.72147683,72.634706)(499.72147095,72.56470998)
\curveto(499.72147683,72.50470613)(499.73147682,72.44970618)(499.75147095,72.39970998)
\lineto(499.75147095,72.29470998)
\curveto(499.7514768,72.26470637)(499.75647679,72.2297064)(499.76647095,72.18970998)
\lineto(499.82647095,71.94970998)
\curveto(499.83647671,71.86970676)(499.85647669,71.78970684)(499.88647095,71.70970998)
\curveto(499.98647656,71.46970716)(500.12147643,71.23970739)(500.29147095,71.01970998)
\curveto(500.36147619,70.9297077)(500.43647611,70.84470779)(500.51647095,70.76470998)
\curveto(500.58647596,70.68470795)(500.64147591,70.58470805)(500.68147095,70.46470998)
\curveto(500.71147584,70.37470826)(500.72147583,70.2347084)(500.71147095,70.04470998)
\curveto(500.70147585,69.86470877)(500.67647587,69.74470889)(500.63647095,69.68470998)
\curveto(500.59647595,69.634709)(500.53647601,69.59470904)(500.45647095,69.56470998)
\curveto(500.37647617,69.54470909)(500.29147626,69.54470909)(500.20147095,69.56470998)
\curveto(500.08147647,69.59470904)(499.96147659,69.61470902)(499.84147095,69.62470998)
\curveto(499.71147684,69.64470899)(499.58647696,69.66970896)(499.46647095,69.69970998)
\curveto(499.42647712,69.71970891)(499.39147716,69.72470891)(499.36147095,69.71470998)
\curveto(499.32147723,69.71470892)(499.27647727,69.72470891)(499.22647095,69.74470998)
\curveto(499.13647741,69.76470887)(499.0464775,69.77970885)(498.95647095,69.78970998)
\curveto(498.85647769,69.79970883)(498.76147779,69.81970881)(498.67147095,69.84970998)
\curveto(498.61147794,69.85970877)(498.551478,69.86470877)(498.49147095,69.86470998)
\curveto(498.43147812,69.87470876)(498.37147818,69.88970874)(498.31147095,69.90970998)
\curveto(498.11147844,69.95970867)(497.90647864,69.99470864)(497.69647095,70.01470998)
\curveto(497.47647907,70.04470859)(497.26647928,70.08470855)(497.06647095,70.13470998)
\curveto(496.96647958,70.16470847)(496.86647968,70.18470845)(496.76647095,70.19470998)
\curveto(496.66647988,70.20470843)(496.56647998,70.21970841)(496.46647095,70.23970998)
\curveto(496.43648011,70.24970838)(496.39648015,70.25470838)(496.34647095,70.25470998)
\curveto(496.23648031,70.28470835)(496.13148042,70.30470833)(496.03147095,70.31470998)
\curveto(495.92148063,70.3347083)(495.81148074,70.35970827)(495.70147095,70.38970998)
\curveto(495.62148093,70.40970822)(495.551481,70.42470821)(495.49147095,70.43470998)
\curveto(495.42148113,70.44470819)(495.36148119,70.46970816)(495.31147095,70.50970998)
\curveto(495.28148127,70.5297081)(495.26148129,70.55970807)(495.25147095,70.59970998)
\curveto(495.23148132,70.63970799)(495.21148134,70.68470795)(495.19147095,70.73470998)
\curveto(495.19148136,70.79470784)(495.18648136,70.8347078)(495.17647095,70.85470998)
}
}
{
\newrgbcolor{curcolor}{0 0 0}
\pscustom[linestyle=none,fillstyle=solid,fillcolor=curcolor]
{
\newpath
\moveto(503.95147095,78.63431936)
\lineto(503.95147095,79.26431936)
\lineto(503.95147095,79.45931936)
\curveto(503.9514726,79.52931683)(503.96147259,79.58931677)(503.98147095,79.63931936)
\curveto(504.02147253,79.70931665)(504.06147249,79.7593166)(504.10147095,79.78931936)
\curveto(504.1514724,79.82931653)(504.21647233,79.84931651)(504.29647095,79.84931936)
\curveto(504.37647217,79.8593165)(504.46147209,79.86431649)(504.55147095,79.86431936)
\lineto(505.27147095,79.86431936)
\curveto(505.7514708,79.86431649)(506.16147039,79.80431655)(506.50147095,79.68431936)
\curveto(506.84146971,79.56431679)(507.11646943,79.36931699)(507.32647095,79.09931936)
\curveto(507.37646917,79.02931733)(507.42146913,78.9593174)(507.46147095,78.88931936)
\curveto(507.51146904,78.82931753)(507.55646899,78.7543176)(507.59647095,78.66431936)
\curveto(507.60646894,78.64431771)(507.61646893,78.61431774)(507.62647095,78.57431936)
\curveto(507.6464689,78.53431782)(507.6514689,78.48931787)(507.64147095,78.43931936)
\curveto(507.61146894,78.34931801)(507.53646901,78.29431806)(507.41647095,78.27431936)
\curveto(507.30646924,78.2543181)(507.21146934,78.26931809)(507.13147095,78.31931936)
\curveto(507.06146949,78.34931801)(506.99646955,78.39431796)(506.93647095,78.45431936)
\curveto(506.88646966,78.52431783)(506.83646971,78.58931777)(506.78647095,78.64931936)
\curveto(506.73646981,78.71931764)(506.66146989,78.77931758)(506.56147095,78.82931936)
\curveto(506.47147008,78.88931747)(506.38147017,78.93931742)(506.29147095,78.97931936)
\curveto(506.26147029,78.99931736)(506.20147035,79.02431733)(506.11147095,79.05431936)
\curveto(506.03147052,79.08431727)(505.96147059,79.08931727)(505.90147095,79.06931936)
\curveto(505.76147079,79.03931732)(505.67147088,78.97931738)(505.63147095,78.88931936)
\curveto(505.60147095,78.80931755)(505.58647096,78.71931764)(505.58647095,78.61931936)
\curveto(505.58647096,78.51931784)(505.56147099,78.43431792)(505.51147095,78.36431936)
\curveto(505.44147111,78.27431808)(505.30147125,78.22931813)(505.09147095,78.22931936)
\lineto(504.53647095,78.22931936)
\lineto(504.31147095,78.22931936)
\curveto(504.23147232,78.23931812)(504.16647238,78.2593181)(504.11647095,78.28931936)
\curveto(504.03647251,78.34931801)(503.99147256,78.41931794)(503.98147095,78.49931936)
\curveto(503.97147258,78.51931784)(503.96647258,78.53931782)(503.96647095,78.55931936)
\curveto(503.96647258,78.58931777)(503.96147259,78.61431774)(503.95147095,78.63431936)
}
}
{
\newrgbcolor{curcolor}{0 0 0}
\pscustom[linestyle=none,fillstyle=solid,fillcolor=curcolor]
{
}
}
{
\newrgbcolor{curcolor}{0 0 0}
\pscustom[linestyle=none,fillstyle=solid,fillcolor=curcolor]
{
\newpath
\moveto(494.98147095,89.26463186)
\curveto(494.97148158,89.95462722)(495.09148146,90.55462662)(495.34147095,91.06463186)
\curveto(495.59148096,91.58462559)(495.92648062,91.9796252)(496.34647095,92.24963186)
\curveto(496.42648012,92.29962488)(496.51648003,92.34462483)(496.61647095,92.38463186)
\curveto(496.70647984,92.42462475)(496.80147975,92.46962471)(496.90147095,92.51963186)
\curveto(497.00147955,92.55962462)(497.10147945,92.58962459)(497.20147095,92.60963186)
\curveto(497.30147925,92.62962455)(497.40647914,92.64962453)(497.51647095,92.66963186)
\curveto(497.56647898,92.68962449)(497.61147894,92.69462448)(497.65147095,92.68463186)
\curveto(497.69147886,92.6746245)(497.73647881,92.6796245)(497.78647095,92.69963186)
\curveto(497.83647871,92.70962447)(497.92147863,92.71462446)(498.04147095,92.71463186)
\curveto(498.1514784,92.71462446)(498.23647831,92.70962447)(498.29647095,92.69963186)
\curveto(498.35647819,92.6796245)(498.41647813,92.66962451)(498.47647095,92.66963186)
\curveto(498.53647801,92.6796245)(498.59647795,92.6746245)(498.65647095,92.65463186)
\curveto(498.79647775,92.61462456)(498.93147762,92.5796246)(499.06147095,92.54963186)
\curveto(499.19147736,92.51962466)(499.31647723,92.4796247)(499.43647095,92.42963186)
\curveto(499.57647697,92.36962481)(499.70147685,92.29962488)(499.81147095,92.21963186)
\curveto(499.92147663,92.14962503)(500.03147652,92.0746251)(500.14147095,91.99463186)
\lineto(500.20147095,91.93463186)
\curveto(500.22147633,91.92462525)(500.24147631,91.90962527)(500.26147095,91.88963186)
\curveto(500.42147613,91.76962541)(500.56647598,91.63462554)(500.69647095,91.48463186)
\curveto(500.82647572,91.33462584)(500.9514756,91.174626)(501.07147095,91.00463186)
\curveto(501.29147526,90.69462648)(501.49647505,90.39962678)(501.68647095,90.11963186)
\curveto(501.82647472,89.88962729)(501.96147459,89.65962752)(502.09147095,89.42963186)
\curveto(502.22147433,89.20962797)(502.35647419,88.98962819)(502.49647095,88.76963186)
\curveto(502.66647388,88.51962866)(502.8464737,88.2796289)(503.03647095,88.04963186)
\curveto(503.22647332,87.82962935)(503.4514731,87.63962954)(503.71147095,87.47963186)
\curveto(503.77147278,87.43962974)(503.83147272,87.40462977)(503.89147095,87.37463186)
\curveto(503.94147261,87.34462983)(504.00647254,87.31462986)(504.08647095,87.28463186)
\curveto(504.15647239,87.26462991)(504.21647233,87.25962992)(504.26647095,87.26963186)
\curveto(504.33647221,87.28962989)(504.39147216,87.32462985)(504.43147095,87.37463186)
\curveto(504.46147209,87.42462975)(504.48147207,87.48462969)(504.49147095,87.55463186)
\lineto(504.49147095,87.79463186)
\lineto(504.49147095,88.54463186)
\lineto(504.49147095,91.34963186)
\lineto(504.49147095,92.00963186)
\curveto(504.49147206,92.09962508)(504.49647205,92.18462499)(504.50647095,92.26463186)
\curveto(504.50647204,92.34462483)(504.52647202,92.40962477)(504.56647095,92.45963186)
\curveto(504.60647194,92.50962467)(504.68147187,92.54962463)(504.79147095,92.57963186)
\curveto(504.89147166,92.61962456)(504.99147156,92.62962455)(505.09147095,92.60963186)
\lineto(505.22647095,92.60963186)
\curveto(505.29647125,92.58962459)(505.35647119,92.56962461)(505.40647095,92.54963186)
\curveto(505.45647109,92.52962465)(505.49647105,92.49462468)(505.52647095,92.44463186)
\curveto(505.56647098,92.39462478)(505.58647096,92.32462485)(505.58647095,92.23463186)
\lineto(505.58647095,91.96463186)
\lineto(505.58647095,91.06463186)
\lineto(505.58647095,87.55463186)
\lineto(505.58647095,86.48963186)
\curveto(505.58647096,86.40963077)(505.59147096,86.31963086)(505.60147095,86.21963186)
\curveto(505.60147095,86.11963106)(505.59147096,86.03463114)(505.57147095,85.96463186)
\curveto(505.50147105,85.75463142)(505.32147123,85.68963149)(505.03147095,85.76963186)
\curveto(504.99147156,85.7796314)(504.95647159,85.7796314)(504.92647095,85.76963186)
\curveto(504.88647166,85.76963141)(504.84147171,85.7796314)(504.79147095,85.79963186)
\curveto(504.71147184,85.81963136)(504.62647192,85.83963134)(504.53647095,85.85963186)
\curveto(504.4464721,85.8796313)(504.36147219,85.90463127)(504.28147095,85.93463186)
\curveto(503.79147276,86.09463108)(503.37647317,86.29463088)(503.03647095,86.53463186)
\curveto(502.78647376,86.71463046)(502.56147399,86.91963026)(502.36147095,87.14963186)
\curveto(502.1514744,87.3796298)(501.95647459,87.61962956)(501.77647095,87.86963186)
\curveto(501.59647495,88.12962905)(501.42647512,88.39462878)(501.26647095,88.66463186)
\curveto(501.09647545,88.94462823)(500.92147563,89.21462796)(500.74147095,89.47463186)
\curveto(500.66147589,89.58462759)(500.58647596,89.68962749)(500.51647095,89.78963186)
\curveto(500.4464761,89.89962728)(500.37147618,90.00962717)(500.29147095,90.11963186)
\curveto(500.26147629,90.15962702)(500.23147632,90.19462698)(500.20147095,90.22463186)
\curveto(500.16147639,90.26462691)(500.13147642,90.30462687)(500.11147095,90.34463186)
\curveto(500.00147655,90.48462669)(499.87647667,90.60962657)(499.73647095,90.71963186)
\curveto(499.70647684,90.73962644)(499.68147687,90.76462641)(499.66147095,90.79463186)
\curveto(499.63147692,90.82462635)(499.60147695,90.84962633)(499.57147095,90.86963186)
\curveto(499.47147708,90.94962623)(499.37147718,91.01462616)(499.27147095,91.06463186)
\curveto(499.17147738,91.12462605)(499.06147749,91.179626)(498.94147095,91.22963186)
\curveto(498.87147768,91.25962592)(498.79647775,91.2796259)(498.71647095,91.28963186)
\lineto(498.47647095,91.34963186)
\lineto(498.38647095,91.34963186)
\curveto(498.35647819,91.35962582)(498.32647822,91.36462581)(498.29647095,91.36463186)
\curveto(498.22647832,91.38462579)(498.13147842,91.38962579)(498.01147095,91.37963186)
\curveto(497.88147867,91.3796258)(497.78147877,91.36962581)(497.71147095,91.34963186)
\curveto(497.63147892,91.32962585)(497.55647899,91.30962587)(497.48647095,91.28963186)
\curveto(497.40647914,91.2796259)(497.32647922,91.25962592)(497.24647095,91.22963186)
\curveto(497.00647954,91.11962606)(496.80647974,90.96962621)(496.64647095,90.77963186)
\curveto(496.47648007,90.59962658)(496.33648021,90.3796268)(496.22647095,90.11963186)
\curveto(496.20648034,90.04962713)(496.19148036,89.9796272)(496.18147095,89.90963186)
\curveto(496.16148039,89.83962734)(496.14148041,89.76462741)(496.12147095,89.68463186)
\curveto(496.10148045,89.60462757)(496.09148046,89.49462768)(496.09147095,89.35463186)
\curveto(496.09148046,89.22462795)(496.10148045,89.11962806)(496.12147095,89.03963186)
\curveto(496.13148042,88.9796282)(496.13648041,88.92462825)(496.13647095,88.87463186)
\curveto(496.13648041,88.82462835)(496.1464804,88.7746284)(496.16647095,88.72463186)
\curveto(496.20648034,88.62462855)(496.2464803,88.52962865)(496.28647095,88.43963186)
\curveto(496.32648022,88.35962882)(496.37148018,88.2796289)(496.42147095,88.19963186)
\curveto(496.44148011,88.16962901)(496.46648008,88.13962904)(496.49647095,88.10963186)
\curveto(496.52648002,88.08962909)(496.55148,88.06462911)(496.57147095,88.03463186)
\lineto(496.64647095,87.95963186)
\curveto(496.66647988,87.92962925)(496.68647986,87.90462927)(496.70647095,87.88463186)
\lineto(496.91647095,87.73463186)
\curveto(496.97647957,87.69462948)(497.04147951,87.64962953)(497.11147095,87.59963186)
\curveto(497.20147935,87.53962964)(497.30647924,87.48962969)(497.42647095,87.44963186)
\curveto(497.53647901,87.41962976)(497.6464789,87.38462979)(497.75647095,87.34463186)
\curveto(497.86647868,87.30462987)(498.01147854,87.2796299)(498.19147095,87.26963186)
\curveto(498.36147819,87.25962992)(498.48647806,87.22962995)(498.56647095,87.17963186)
\curveto(498.6464779,87.12963005)(498.69147786,87.05463012)(498.70147095,86.95463186)
\curveto(498.71147784,86.85463032)(498.71647783,86.74463043)(498.71647095,86.62463186)
\curveto(498.71647783,86.58463059)(498.72147783,86.54463063)(498.73147095,86.50463186)
\curveto(498.73147782,86.46463071)(498.72647782,86.42963075)(498.71647095,86.39963186)
\curveto(498.69647785,86.34963083)(498.68647786,86.29963088)(498.68647095,86.24963186)
\curveto(498.68647786,86.20963097)(498.67647787,86.16963101)(498.65647095,86.12963186)
\curveto(498.59647795,86.03963114)(498.46147809,85.99463118)(498.25147095,85.99463186)
\lineto(498.13147095,85.99463186)
\curveto(498.07147848,86.00463117)(498.01147854,86.00963117)(497.95147095,86.00963186)
\curveto(497.88147867,86.01963116)(497.81647873,86.02963115)(497.75647095,86.03963186)
\curveto(497.6464789,86.05963112)(497.546479,86.0796311)(497.45647095,86.09963186)
\curveto(497.35647919,86.11963106)(497.26147929,86.14963103)(497.17147095,86.18963186)
\curveto(497.10147945,86.20963097)(497.04147951,86.22963095)(496.99147095,86.24963186)
\lineto(496.81147095,86.30963186)
\curveto(496.55148,86.42963075)(496.30648024,86.58463059)(496.07647095,86.77463186)
\curveto(495.8464807,86.9746302)(495.66148089,87.18962999)(495.52147095,87.41963186)
\curveto(495.44148111,87.52962965)(495.37648117,87.64462953)(495.32647095,87.76463186)
\lineto(495.17647095,88.15463186)
\curveto(495.12648142,88.26462891)(495.09648145,88.3796288)(495.08647095,88.49963186)
\curveto(495.06648148,88.61962856)(495.04148151,88.74462843)(495.01147095,88.87463186)
\curveto(495.01148154,88.94462823)(495.01148154,89.00962817)(495.01147095,89.06963186)
\curveto(495.00148155,89.12962805)(494.99148156,89.19462798)(494.98147095,89.26463186)
}
}
{
\newrgbcolor{curcolor}{0 0 0}
\pscustom[linestyle=none,fillstyle=solid,fillcolor=curcolor]
{
\newpath
\moveto(500.50147095,101.36424123)
\lineto(500.75647095,101.36424123)
\curveto(500.83647571,101.37423353)(500.91147564,101.36923353)(500.98147095,101.34924123)
\lineto(501.22147095,101.34924123)
\lineto(501.38647095,101.34924123)
\curveto(501.48647506,101.32923357)(501.59147496,101.31923358)(501.70147095,101.31924123)
\curveto(501.80147475,101.31923358)(501.90147465,101.30923359)(502.00147095,101.28924123)
\lineto(502.15147095,101.28924123)
\curveto(502.29147426,101.25923364)(502.43147412,101.23923366)(502.57147095,101.22924123)
\curveto(502.70147385,101.21923368)(502.83147372,101.19423371)(502.96147095,101.15424123)
\curveto(503.04147351,101.13423377)(503.12647342,101.11423379)(503.21647095,101.09424123)
\lineto(503.45647095,101.03424123)
\lineto(503.75647095,100.91424123)
\curveto(503.8464727,100.88423402)(503.93647261,100.84923405)(504.02647095,100.80924123)
\curveto(504.2464723,100.70923419)(504.46147209,100.57423433)(504.67147095,100.40424123)
\curveto(504.88147167,100.24423466)(505.0514715,100.06923483)(505.18147095,99.87924123)
\curveto(505.22147133,99.82923507)(505.26147129,99.76923513)(505.30147095,99.69924123)
\curveto(505.33147122,99.63923526)(505.36647118,99.57923532)(505.40647095,99.51924123)
\curveto(505.45647109,99.43923546)(505.49647105,99.34423556)(505.52647095,99.23424123)
\curveto(505.55647099,99.12423578)(505.58647096,99.01923588)(505.61647095,98.91924123)
\curveto(505.65647089,98.80923609)(505.68147087,98.6992362)(505.69147095,98.58924123)
\curveto(505.70147085,98.47923642)(505.71647083,98.36423654)(505.73647095,98.24424123)
\curveto(505.7464708,98.2042367)(505.7464708,98.15923674)(505.73647095,98.10924123)
\curveto(505.73647081,98.06923683)(505.74147081,98.02923687)(505.75147095,97.98924123)
\curveto(505.76147079,97.94923695)(505.76647078,97.89423701)(505.76647095,97.82424123)
\curveto(505.76647078,97.75423715)(505.76147079,97.7042372)(505.75147095,97.67424123)
\curveto(505.73147082,97.62423728)(505.72647082,97.57923732)(505.73647095,97.53924123)
\curveto(505.7464708,97.4992374)(505.7464708,97.46423744)(505.73647095,97.43424123)
\lineto(505.73647095,97.34424123)
\curveto(505.71647083,97.28423762)(505.70147085,97.21923768)(505.69147095,97.14924123)
\curveto(505.69147086,97.08923781)(505.68647086,97.02423788)(505.67647095,96.95424123)
\curveto(505.62647092,96.78423812)(505.57647097,96.62423828)(505.52647095,96.47424123)
\curveto(505.47647107,96.32423858)(505.41147114,96.17923872)(505.33147095,96.03924123)
\curveto(505.29147126,95.98923891)(505.26147129,95.93423897)(505.24147095,95.87424123)
\curveto(505.21147134,95.82423908)(505.17647137,95.77423913)(505.13647095,95.72424123)
\curveto(504.95647159,95.48423942)(504.73647181,95.28423962)(504.47647095,95.12424123)
\curveto(504.21647233,94.96423994)(503.93147262,94.82424008)(503.62147095,94.70424123)
\curveto(503.48147307,94.64424026)(503.34147321,94.5992403)(503.20147095,94.56924123)
\curveto(503.0514735,94.53924036)(502.89647365,94.5042404)(502.73647095,94.46424123)
\curveto(502.62647392,94.44424046)(502.51647403,94.42924047)(502.40647095,94.41924123)
\curveto(502.29647425,94.40924049)(502.18647436,94.39424051)(502.07647095,94.37424123)
\curveto(502.03647451,94.36424054)(501.99647455,94.35924054)(501.95647095,94.35924123)
\curveto(501.91647463,94.36924053)(501.87647467,94.36924053)(501.83647095,94.35924123)
\curveto(501.78647476,94.34924055)(501.73647481,94.34424056)(501.68647095,94.34424123)
\lineto(501.52147095,94.34424123)
\curveto(501.47147508,94.32424058)(501.42147513,94.31924058)(501.37147095,94.32924123)
\curveto(501.31147524,94.33924056)(501.25647529,94.33924056)(501.20647095,94.32924123)
\curveto(501.16647538,94.31924058)(501.12147543,94.31924058)(501.07147095,94.32924123)
\curveto(501.02147553,94.33924056)(500.97147558,94.33424057)(500.92147095,94.31424123)
\curveto(500.8514757,94.29424061)(500.77647577,94.28924061)(500.69647095,94.29924123)
\curveto(500.60647594,94.30924059)(500.52147603,94.31424059)(500.44147095,94.31424123)
\curveto(500.3514762,94.31424059)(500.2514763,94.30924059)(500.14147095,94.29924123)
\curveto(500.02147653,94.28924061)(499.92147663,94.29424061)(499.84147095,94.31424123)
\lineto(499.55647095,94.31424123)
\lineto(498.92647095,94.35924123)
\curveto(498.82647772,94.36924053)(498.73147782,94.37924052)(498.64147095,94.38924123)
\lineto(498.34147095,94.41924123)
\curveto(498.29147826,94.43924046)(498.24147831,94.44424046)(498.19147095,94.43424123)
\curveto(498.13147842,94.43424047)(498.07647847,94.44424046)(498.02647095,94.46424123)
\curveto(497.85647869,94.51424039)(497.69147886,94.55424035)(497.53147095,94.58424123)
\curveto(497.36147919,94.61424029)(497.20147935,94.66424024)(497.05147095,94.73424123)
\curveto(496.59147996,94.92423998)(496.21648033,95.14423976)(495.92647095,95.39424123)
\curveto(495.63648091,95.65423925)(495.39148116,96.01423889)(495.19147095,96.47424123)
\curveto(495.14148141,96.6042383)(495.10648144,96.73423817)(495.08647095,96.86424123)
\curveto(495.06648148,97.0042379)(495.04148151,97.14423776)(495.01147095,97.28424123)
\curveto(495.00148155,97.35423755)(494.99648155,97.41923748)(494.99647095,97.47924123)
\curveto(494.99648155,97.53923736)(494.99148156,97.6042373)(494.98147095,97.67424123)
\curveto(494.96148159,98.5042364)(495.11148144,99.17423573)(495.43147095,99.68424123)
\curveto(495.74148081,100.19423471)(496.18148037,100.57423433)(496.75147095,100.82424123)
\curveto(496.87147968,100.87423403)(496.99647955,100.91923398)(497.12647095,100.95924123)
\curveto(497.25647929,100.9992339)(497.39147916,101.04423386)(497.53147095,101.09424123)
\curveto(497.61147894,101.11423379)(497.69647885,101.12923377)(497.78647095,101.13924123)
\lineto(498.02647095,101.19924123)
\curveto(498.13647841,101.22923367)(498.2464783,101.24423366)(498.35647095,101.24424123)
\curveto(498.46647808,101.25423365)(498.57647797,101.26923363)(498.68647095,101.28924123)
\curveto(498.73647781,101.30923359)(498.78147777,101.31423359)(498.82147095,101.30424123)
\curveto(498.86147769,101.3042336)(498.90147765,101.30923359)(498.94147095,101.31924123)
\curveto(498.99147756,101.32923357)(499.0464775,101.32923357)(499.10647095,101.31924123)
\curveto(499.15647739,101.31923358)(499.20647734,101.32423358)(499.25647095,101.33424123)
\lineto(499.39147095,101.33424123)
\curveto(499.4514771,101.35423355)(499.52147703,101.35423355)(499.60147095,101.33424123)
\curveto(499.67147688,101.32423358)(499.73647681,101.32923357)(499.79647095,101.34924123)
\curveto(499.82647672,101.35923354)(499.86647668,101.36423354)(499.91647095,101.36424123)
\lineto(500.03647095,101.36424123)
\lineto(500.50147095,101.36424123)
\moveto(502.82647095,99.81924123)
\curveto(502.50647404,99.91923498)(502.14147441,99.97923492)(501.73147095,99.99924123)
\curveto(501.32147523,100.01923488)(500.91147564,100.02923487)(500.50147095,100.02924123)
\curveto(500.07147648,100.02923487)(499.6514769,100.01923488)(499.24147095,99.99924123)
\curveto(498.83147772,99.97923492)(498.4464781,99.93423497)(498.08647095,99.86424123)
\curveto(497.72647882,99.79423511)(497.40647914,99.68423522)(497.12647095,99.53424123)
\curveto(496.83647971,99.39423551)(496.60147995,99.1992357)(496.42147095,98.94924123)
\curveto(496.31148024,98.78923611)(496.23148032,98.60923629)(496.18147095,98.40924123)
\curveto(496.12148043,98.20923669)(496.09148046,97.96423694)(496.09147095,97.67424123)
\curveto(496.11148044,97.65423725)(496.12148043,97.61923728)(496.12147095,97.56924123)
\curveto(496.11148044,97.51923738)(496.11148044,97.47923742)(496.12147095,97.44924123)
\curveto(496.14148041,97.36923753)(496.16148039,97.29423761)(496.18147095,97.22424123)
\curveto(496.19148036,97.16423774)(496.21148034,97.0992378)(496.24147095,97.02924123)
\curveto(496.36148019,96.75923814)(496.53148002,96.53923836)(496.75147095,96.36924123)
\curveto(496.96147959,96.20923869)(497.20647934,96.07423883)(497.48647095,95.96424123)
\curveto(497.59647895,95.91423899)(497.71647883,95.87423903)(497.84647095,95.84424123)
\curveto(497.96647858,95.82423908)(498.09147846,95.7992391)(498.22147095,95.76924123)
\curveto(498.27147828,95.74923915)(498.32647822,95.73923916)(498.38647095,95.73924123)
\curveto(498.43647811,95.73923916)(498.48647806,95.73423917)(498.53647095,95.72424123)
\curveto(498.62647792,95.71423919)(498.72147783,95.7042392)(498.82147095,95.69424123)
\curveto(498.91147764,95.68423922)(499.00647754,95.67423923)(499.10647095,95.66424123)
\curveto(499.18647736,95.66423924)(499.27147728,95.65923924)(499.36147095,95.64924123)
\lineto(499.60147095,95.64924123)
\lineto(499.78147095,95.64924123)
\curveto(499.81147674,95.63923926)(499.8464767,95.63423927)(499.88647095,95.63424123)
\lineto(500.02147095,95.63424123)
\lineto(500.47147095,95.63424123)
\curveto(500.551476,95.63423927)(500.63647591,95.62923927)(500.72647095,95.61924123)
\curveto(500.80647574,95.61923928)(500.88147567,95.62923927)(500.95147095,95.64924123)
\lineto(501.22147095,95.64924123)
\curveto(501.24147531,95.64923925)(501.27147528,95.64423926)(501.31147095,95.63424123)
\curveto(501.34147521,95.63423927)(501.36647518,95.63923926)(501.38647095,95.64924123)
\curveto(501.48647506,95.65923924)(501.58647496,95.66423924)(501.68647095,95.66424123)
\curveto(501.77647477,95.67423923)(501.87647467,95.68423922)(501.98647095,95.69424123)
\curveto(502.10647444,95.72423918)(502.23147432,95.73923916)(502.36147095,95.73924123)
\curveto(502.48147407,95.74923915)(502.59647395,95.77423913)(502.70647095,95.81424123)
\curveto(503.00647354,95.89423901)(503.27147328,95.97923892)(503.50147095,96.06924123)
\curveto(503.73147282,96.16923873)(503.9464726,96.31423859)(504.14647095,96.50424123)
\curveto(504.3464722,96.71423819)(504.49647205,96.97923792)(504.59647095,97.29924123)
\curveto(504.61647193,97.33923756)(504.62647192,97.37423753)(504.62647095,97.40424123)
\curveto(504.61647193,97.44423746)(504.62147193,97.48923741)(504.64147095,97.53924123)
\curveto(504.6514719,97.57923732)(504.66147189,97.64923725)(504.67147095,97.74924123)
\curveto(504.68147187,97.85923704)(504.67647187,97.94423696)(504.65647095,98.00424123)
\curveto(504.63647191,98.07423683)(504.62647192,98.14423676)(504.62647095,98.21424123)
\curveto(504.61647193,98.28423662)(504.60147195,98.34923655)(504.58147095,98.40924123)
\curveto(504.52147203,98.60923629)(504.43647211,98.78923611)(504.32647095,98.94924123)
\curveto(504.30647224,98.97923592)(504.28647226,99.0042359)(504.26647095,99.02424123)
\lineto(504.20647095,99.08424123)
\curveto(504.18647236,99.12423578)(504.1464724,99.17423573)(504.08647095,99.23424123)
\curveto(503.9464726,99.33423557)(503.81647273,99.41923548)(503.69647095,99.48924123)
\curveto(503.57647297,99.55923534)(503.43147312,99.62923527)(503.26147095,99.69924123)
\curveto(503.19147336,99.72923517)(503.12147343,99.74923515)(503.05147095,99.75924123)
\curveto(502.98147357,99.77923512)(502.90647364,99.7992351)(502.82647095,99.81924123)
}
}
{
\newrgbcolor{curcolor}{0 0 0}
\pscustom[linestyle=none,fillstyle=solid,fillcolor=curcolor]
{
\newpath
\moveto(494.98147095,106.77385061)
\curveto(494.98148157,106.87384575)(494.99148156,106.96884566)(495.01147095,107.05885061)
\curveto(495.02148153,107.14884548)(495.0514815,107.21384541)(495.10147095,107.25385061)
\curveto(495.18148137,107.31384531)(495.28648126,107.34384528)(495.41647095,107.34385061)
\lineto(495.80647095,107.34385061)
\lineto(497.30647095,107.34385061)
\lineto(503.69647095,107.34385061)
\lineto(504.86647095,107.34385061)
\lineto(505.18147095,107.34385061)
\curveto(505.28147127,107.35384527)(505.36147119,107.33884529)(505.42147095,107.29885061)
\curveto(505.50147105,107.24884538)(505.551471,107.17384545)(505.57147095,107.07385061)
\curveto(505.58147097,106.98384564)(505.58647096,106.87384575)(505.58647095,106.74385061)
\lineto(505.58647095,106.51885061)
\curveto(505.56647098,106.43884619)(505.551471,106.36884626)(505.54147095,106.30885061)
\curveto(505.52147103,106.24884638)(505.48147107,106.19884643)(505.42147095,106.15885061)
\curveto(505.36147119,106.11884651)(505.28647126,106.09884653)(505.19647095,106.09885061)
\lineto(504.89647095,106.09885061)
\lineto(503.80147095,106.09885061)
\lineto(498.46147095,106.09885061)
\curveto(498.37147818,106.07884655)(498.29647825,106.06384656)(498.23647095,106.05385061)
\curveto(498.16647838,106.05384657)(498.10647844,106.0238466)(498.05647095,105.96385061)
\curveto(498.00647854,105.89384673)(497.98147857,105.80384682)(497.98147095,105.69385061)
\curveto(497.97147858,105.59384703)(497.96647858,105.48384714)(497.96647095,105.36385061)
\lineto(497.96647095,104.22385061)
\lineto(497.96647095,103.72885061)
\curveto(497.95647859,103.56884906)(497.89647865,103.45884917)(497.78647095,103.39885061)
\curveto(497.75647879,103.37884925)(497.72647882,103.36884926)(497.69647095,103.36885061)
\curveto(497.65647889,103.36884926)(497.61147894,103.36384926)(497.56147095,103.35385061)
\curveto(497.44147911,103.33384929)(497.33147922,103.33884929)(497.23147095,103.36885061)
\curveto(497.13147942,103.40884922)(497.06147949,103.46384916)(497.02147095,103.53385061)
\curveto(496.97147958,103.61384901)(496.9464796,103.73384889)(496.94647095,103.89385061)
\curveto(496.9464796,104.05384857)(496.93147962,104.18884844)(496.90147095,104.29885061)
\curveto(496.89147966,104.34884828)(496.88647966,104.40384822)(496.88647095,104.46385061)
\curveto(496.87647967,104.5238481)(496.86147969,104.58384804)(496.84147095,104.64385061)
\curveto(496.79147976,104.79384783)(496.74147981,104.93884769)(496.69147095,105.07885061)
\curveto(496.63147992,105.21884741)(496.56147999,105.35384727)(496.48147095,105.48385061)
\curveto(496.39148016,105.623847)(496.28648026,105.74384688)(496.16647095,105.84385061)
\curveto(496.0464805,105.94384668)(495.91648063,106.03884659)(495.77647095,106.12885061)
\curveto(495.67648087,106.18884644)(495.56648098,106.23384639)(495.44647095,106.26385061)
\curveto(495.32648122,106.30384632)(495.22148133,106.35384627)(495.13147095,106.41385061)
\curveto(495.07148148,106.46384616)(495.03148152,106.53384609)(495.01147095,106.62385061)
\curveto(495.00148155,106.64384598)(494.99648155,106.66884596)(494.99647095,106.69885061)
\curveto(494.99648155,106.7288459)(494.99148156,106.75384587)(494.98147095,106.77385061)
}
}
{
\newrgbcolor{curcolor}{0 0 0}
\pscustom[linestyle=none,fillstyle=solid,fillcolor=curcolor]
{
\newpath
\moveto(494.98147095,115.12345998)
\curveto(494.98148157,115.22345513)(494.99148156,115.31845503)(495.01147095,115.40845998)
\curveto(495.02148153,115.49845485)(495.0514815,115.56345479)(495.10147095,115.60345998)
\curveto(495.18148137,115.66345469)(495.28648126,115.69345466)(495.41647095,115.69345998)
\lineto(495.80647095,115.69345998)
\lineto(497.30647095,115.69345998)
\lineto(503.69647095,115.69345998)
\lineto(504.86647095,115.69345998)
\lineto(505.18147095,115.69345998)
\curveto(505.28147127,115.70345465)(505.36147119,115.68845466)(505.42147095,115.64845998)
\curveto(505.50147105,115.59845475)(505.551471,115.52345483)(505.57147095,115.42345998)
\curveto(505.58147097,115.33345502)(505.58647096,115.22345513)(505.58647095,115.09345998)
\lineto(505.58647095,114.86845998)
\curveto(505.56647098,114.78845556)(505.551471,114.71845563)(505.54147095,114.65845998)
\curveto(505.52147103,114.59845575)(505.48147107,114.5484558)(505.42147095,114.50845998)
\curveto(505.36147119,114.46845588)(505.28647126,114.4484559)(505.19647095,114.44845998)
\lineto(504.89647095,114.44845998)
\lineto(503.80147095,114.44845998)
\lineto(498.46147095,114.44845998)
\curveto(498.37147818,114.42845592)(498.29647825,114.41345594)(498.23647095,114.40345998)
\curveto(498.16647838,114.40345595)(498.10647844,114.37345598)(498.05647095,114.31345998)
\curveto(498.00647854,114.24345611)(497.98147857,114.1534562)(497.98147095,114.04345998)
\curveto(497.97147858,113.94345641)(497.96647858,113.83345652)(497.96647095,113.71345998)
\lineto(497.96647095,112.57345998)
\lineto(497.96647095,112.07845998)
\curveto(497.95647859,111.91845843)(497.89647865,111.80845854)(497.78647095,111.74845998)
\curveto(497.75647879,111.72845862)(497.72647882,111.71845863)(497.69647095,111.71845998)
\curveto(497.65647889,111.71845863)(497.61147894,111.71345864)(497.56147095,111.70345998)
\curveto(497.44147911,111.68345867)(497.33147922,111.68845866)(497.23147095,111.71845998)
\curveto(497.13147942,111.75845859)(497.06147949,111.81345854)(497.02147095,111.88345998)
\curveto(496.97147958,111.96345839)(496.9464796,112.08345827)(496.94647095,112.24345998)
\curveto(496.9464796,112.40345795)(496.93147962,112.53845781)(496.90147095,112.64845998)
\curveto(496.89147966,112.69845765)(496.88647966,112.7534576)(496.88647095,112.81345998)
\curveto(496.87647967,112.87345748)(496.86147969,112.93345742)(496.84147095,112.99345998)
\curveto(496.79147976,113.14345721)(496.74147981,113.28845706)(496.69147095,113.42845998)
\curveto(496.63147992,113.56845678)(496.56147999,113.70345665)(496.48147095,113.83345998)
\curveto(496.39148016,113.97345638)(496.28648026,114.09345626)(496.16647095,114.19345998)
\curveto(496.0464805,114.29345606)(495.91648063,114.38845596)(495.77647095,114.47845998)
\curveto(495.67648087,114.53845581)(495.56648098,114.58345577)(495.44647095,114.61345998)
\curveto(495.32648122,114.6534557)(495.22148133,114.70345565)(495.13147095,114.76345998)
\curveto(495.07148148,114.81345554)(495.03148152,114.88345547)(495.01147095,114.97345998)
\curveto(495.00148155,114.99345536)(494.99648155,115.01845533)(494.99647095,115.04845998)
\curveto(494.99648155,115.07845527)(494.99148156,115.10345525)(494.98147095,115.12345998)
}
}
{
\newrgbcolor{curcolor}{0 0 0}
\pscustom[linestyle=none,fillstyle=solid,fillcolor=curcolor]
{
\newpath
\moveto(516.85279053,29.18119436)
\lineto(516.85279053,30.09619436)
\curveto(516.85280122,30.19619171)(516.85280122,30.29119161)(516.85279053,30.38119436)
\curveto(516.85280122,30.47119143)(516.8728012,30.54619136)(516.91279053,30.60619436)
\curveto(516.9728011,30.69619121)(517.05280102,30.75619115)(517.15279053,30.78619436)
\curveto(517.25280082,30.82619108)(517.35780072,30.87119103)(517.46779053,30.92119436)
\curveto(517.65780042,31.0011909)(517.84780023,31.07119083)(518.03779053,31.13119436)
\curveto(518.22779985,31.2011907)(518.41779966,31.27619063)(518.60779053,31.35619436)
\curveto(518.78779929,31.42619048)(518.9727991,31.49119041)(519.16279053,31.55119436)
\curveto(519.34279873,31.61119029)(519.52279855,31.68119022)(519.70279053,31.76119436)
\curveto(519.84279823,31.82119008)(519.98779809,31.87619003)(520.13779053,31.92619436)
\curveto(520.28779779,31.97618993)(520.43279764,32.03118987)(520.57279053,32.09119436)
\curveto(521.02279705,32.27118963)(521.4777966,32.44118946)(521.93779053,32.60119436)
\curveto(522.38779569,32.76118914)(522.83779524,32.93118897)(523.28779053,33.11119436)
\curveto(523.33779474,33.13118877)(523.38779469,33.14618876)(523.43779053,33.15619436)
\lineto(523.58779053,33.21619436)
\curveto(523.80779427,33.3061886)(524.03279404,33.39118851)(524.26279053,33.47119436)
\curveto(524.48279359,33.55118835)(524.70279337,33.63618827)(524.92279053,33.72619436)
\curveto(525.01279306,33.76618814)(525.12279295,33.8061881)(525.25279053,33.84619436)
\curveto(525.3727927,33.88618802)(525.44279263,33.95118795)(525.46279053,34.04119436)
\curveto(525.4727926,34.08118782)(525.4727926,34.11118779)(525.46279053,34.13119436)
\lineto(525.40279053,34.19119436)
\curveto(525.35279272,34.24118766)(525.29779278,34.27618763)(525.23779053,34.29619436)
\curveto(525.1777929,34.32618758)(525.11279296,34.35618755)(525.04279053,34.38619436)
\lineto(524.41279053,34.62619436)
\curveto(524.19279388,34.7061872)(523.9777941,34.78618712)(523.76779053,34.86619436)
\lineto(523.61779053,34.92619436)
\lineto(523.43779053,34.98619436)
\curveto(523.24779483,35.06618684)(523.05779502,35.13618677)(522.86779053,35.19619436)
\curveto(522.66779541,35.26618664)(522.46779561,35.34118656)(522.26779053,35.42119436)
\curveto(521.68779639,35.66118624)(521.10279697,35.88118602)(520.51279053,36.08119436)
\curveto(519.92279815,36.29118561)(519.33779874,36.51618539)(518.75779053,36.75619436)
\curveto(518.55779952,36.83618507)(518.35279972,36.91118499)(518.14279053,36.98119436)
\curveto(517.93280014,37.06118484)(517.72780035,37.14118476)(517.52779053,37.22119436)
\curveto(517.44780063,37.26118464)(517.34780073,37.29618461)(517.22779053,37.32619436)
\curveto(517.10780097,37.36618454)(517.02280105,37.42118448)(516.97279053,37.49119436)
\curveto(516.93280114,37.55118435)(516.90280117,37.62618428)(516.88279053,37.71619436)
\curveto(516.86280121,37.81618409)(516.85280122,37.92618398)(516.85279053,38.04619436)
\curveto(516.84280123,38.16618374)(516.84280123,38.28618362)(516.85279053,38.40619436)
\curveto(516.85280122,38.52618338)(516.85280122,38.63618327)(516.85279053,38.73619436)
\curveto(516.85280122,38.82618308)(516.85280122,38.91618299)(516.85279053,39.00619436)
\curveto(516.85280122,39.1061828)(516.8728012,39.18118272)(516.91279053,39.23119436)
\curveto(516.96280111,39.32118258)(517.05280102,39.37118253)(517.18279053,39.38119436)
\curveto(517.31280076,39.39118251)(517.45280062,39.39618251)(517.60279053,39.39619436)
\lineto(519.25279053,39.39619436)
\lineto(525.52279053,39.39619436)
\lineto(526.78279053,39.39619436)
\curveto(526.89279118,39.39618251)(527.00279107,39.39618251)(527.11279053,39.39619436)
\curveto(527.22279085,39.4061825)(527.30779077,39.38618252)(527.36779053,39.33619436)
\curveto(527.42779065,39.3061826)(527.46779061,39.26118264)(527.48779053,39.20119436)
\curveto(527.49779058,39.14118276)(527.51279056,39.07118283)(527.53279053,38.99119436)
\lineto(527.53279053,38.75119436)
\lineto(527.53279053,38.39119436)
\curveto(527.52279055,38.28118362)(527.4777906,38.2011837)(527.39779053,38.15119436)
\curveto(527.36779071,38.13118377)(527.33779074,38.11618379)(527.30779053,38.10619436)
\curveto(527.26779081,38.1061838)(527.22279085,38.09618381)(527.17279053,38.07619436)
\lineto(527.00779053,38.07619436)
\curveto(526.94779113,38.06618384)(526.8777912,38.06118384)(526.79779053,38.06119436)
\curveto(526.71779136,38.07118383)(526.64279143,38.07618383)(526.57279053,38.07619436)
\lineto(525.73279053,38.07619436)
\lineto(521.30779053,38.07619436)
\curveto(521.05779702,38.07618383)(520.80779727,38.07618383)(520.55779053,38.07619436)
\curveto(520.29779778,38.07618383)(520.04779803,38.07118383)(519.80779053,38.06119436)
\curveto(519.70779837,38.06118384)(519.59779848,38.05618385)(519.47779053,38.04619436)
\curveto(519.35779872,38.03618387)(519.29779878,37.98118392)(519.29779053,37.88119436)
\lineto(519.31279053,37.88119436)
\curveto(519.33279874,37.81118409)(519.39779868,37.75118415)(519.50779053,37.70119436)
\curveto(519.61779846,37.66118424)(519.71279836,37.62618428)(519.79279053,37.59619436)
\curveto(519.96279811,37.52618438)(520.13779794,37.46118444)(520.31779053,37.40119436)
\curveto(520.48779759,37.34118456)(520.65779742,37.27118463)(520.82779053,37.19119436)
\curveto(520.8777972,37.17118473)(520.92279715,37.15618475)(520.96279053,37.14619436)
\curveto(521.00279707,37.13618477)(521.04779703,37.12118478)(521.09779053,37.10119436)
\curveto(521.2777968,37.02118488)(521.46279661,36.95118495)(521.65279053,36.89119436)
\curveto(521.83279624,36.84118506)(522.01279606,36.77618513)(522.19279053,36.69619436)
\curveto(522.34279573,36.62618528)(522.49779558,36.56618534)(522.65779053,36.51619436)
\curveto(522.80779527,36.46618544)(522.95779512,36.41118549)(523.10779053,36.35119436)
\curveto(523.5777945,36.15118575)(524.05279402,35.97118593)(524.53279053,35.81119436)
\curveto(525.00279307,35.65118625)(525.46779261,35.47618643)(525.92779053,35.28619436)
\curveto(526.10779197,35.2061867)(526.28779179,35.13618677)(526.46779053,35.07619436)
\curveto(526.64779143,35.01618689)(526.82779125,34.95118695)(527.00779053,34.88119436)
\curveto(527.11779096,34.83118707)(527.22279085,34.78118712)(527.32279053,34.73119436)
\curveto(527.41279066,34.69118721)(527.4777906,34.6061873)(527.51779053,34.47619436)
\curveto(527.52779055,34.45618745)(527.53279054,34.43118747)(527.53279053,34.40119436)
\curveto(527.52279055,34.38118752)(527.52279055,34.35618755)(527.53279053,34.32619436)
\curveto(527.54279053,34.29618761)(527.54779053,34.26118764)(527.54779053,34.22119436)
\curveto(527.53779054,34.18118772)(527.53279054,34.14118776)(527.53279053,34.10119436)
\lineto(527.53279053,33.80119436)
\curveto(527.53279054,33.7011882)(527.50779057,33.62118828)(527.45779053,33.56119436)
\curveto(527.40779067,33.48118842)(527.33779074,33.42118848)(527.24779053,33.38119436)
\curveto(527.14779093,33.35118855)(527.04779103,33.31118859)(526.94779053,33.26119436)
\curveto(526.74779133,33.18118872)(526.54279153,33.1011888)(526.33279053,33.02119436)
\curveto(526.11279196,32.95118895)(525.90279217,32.87618903)(525.70279053,32.79619436)
\curveto(525.52279255,32.71618919)(525.34279273,32.64618926)(525.16279053,32.58619436)
\curveto(524.9727931,32.53618937)(524.78779329,32.47118943)(524.60779053,32.39119436)
\curveto(524.04779403,32.16118974)(523.48279459,31.94618996)(522.91279053,31.74619436)
\curveto(522.34279573,31.54619036)(521.7777963,31.33119057)(521.21779053,31.10119436)
\lineto(520.58779053,30.86119436)
\curveto(520.36779771,30.79119111)(520.15779792,30.71619119)(519.95779053,30.63619436)
\curveto(519.84779823,30.58619132)(519.74279833,30.54119136)(519.64279053,30.50119436)
\curveto(519.53279854,30.47119143)(519.43779864,30.42119148)(519.35779053,30.35119436)
\curveto(519.33779874,30.34119156)(519.32779875,30.33119157)(519.32779053,30.32119436)
\lineto(519.29779053,30.29119436)
\lineto(519.29779053,30.21619436)
\lineto(519.32779053,30.18619436)
\curveto(519.32779875,30.17619173)(519.33279874,30.16619174)(519.34279053,30.15619436)
\curveto(519.39279868,30.13619177)(519.44779863,30.12619178)(519.50779053,30.12619436)
\curveto(519.56779851,30.12619178)(519.62779845,30.11619179)(519.68779053,30.09619436)
\lineto(519.85279053,30.09619436)
\curveto(519.91279816,30.07619183)(519.9777981,30.07119183)(520.04779053,30.08119436)
\curveto(520.11779796,30.09119181)(520.18779789,30.09619181)(520.25779053,30.09619436)
\lineto(521.06779053,30.09619436)
\lineto(525.62779053,30.09619436)
\lineto(526.81279053,30.09619436)
\curveto(526.92279115,30.09619181)(527.03279104,30.09119181)(527.14279053,30.08119436)
\curveto(527.25279082,30.08119182)(527.33779074,30.05619185)(527.39779053,30.00619436)
\curveto(527.4777906,29.95619195)(527.52279055,29.86619204)(527.53279053,29.73619436)
\lineto(527.53279053,29.34619436)
\lineto(527.53279053,29.15119436)
\curveto(527.53279054,29.1011928)(527.52279055,29.05119285)(527.50279053,29.00119436)
\curveto(527.46279061,28.87119303)(527.3777907,28.79619311)(527.24779053,28.77619436)
\curveto(527.11779096,28.76619314)(526.96779111,28.76119314)(526.79779053,28.76119436)
\lineto(525.05779053,28.76119436)
\lineto(519.05779053,28.76119436)
\lineto(517.64779053,28.76119436)
\curveto(517.53780054,28.76119314)(517.42280065,28.75619315)(517.30279053,28.74619436)
\curveto(517.18280089,28.74619316)(517.08780099,28.77119313)(517.01779053,28.82119436)
\curveto(516.95780112,28.86119304)(516.90780117,28.93619297)(516.86779053,29.04619436)
\curveto(516.85780122,29.06619284)(516.85780122,29.08619282)(516.86779053,29.10619436)
\curveto(516.86780121,29.13619277)(516.86280121,29.16119274)(516.85279053,29.18119436)
}
}
{
\newrgbcolor{curcolor}{0 0 0}
\pscustom[linestyle=none,fillstyle=solid,fillcolor=curcolor]
{
\newpath
\moveto(526.97779053,48.38330373)
\curveto(527.13779094,48.4132959)(527.2727908,48.39829592)(527.38279053,48.33830373)
\curveto(527.48279059,48.27829604)(527.55779052,48.19829612)(527.60779053,48.09830373)
\curveto(527.62779045,48.04829627)(527.63779044,47.99329632)(527.63779053,47.93330373)
\curveto(527.63779044,47.88329643)(527.64779043,47.82829649)(527.66779053,47.76830373)
\curveto(527.71779036,47.54829677)(527.70279037,47.32829699)(527.62279053,47.10830373)
\curveto(527.55279052,46.89829742)(527.46279061,46.75329756)(527.35279053,46.67330373)
\curveto(527.28279079,46.62329769)(527.20279087,46.57829774)(527.11279053,46.53830373)
\curveto(527.01279106,46.49829782)(526.93279114,46.44829787)(526.87279053,46.38830373)
\curveto(526.85279122,46.36829795)(526.83279124,46.34329797)(526.81279053,46.31330373)
\curveto(526.79279128,46.29329802)(526.78779129,46.26329805)(526.79779053,46.22330373)
\curveto(526.82779125,46.1132982)(526.88279119,46.00829831)(526.96279053,45.90830373)
\curveto(527.04279103,45.8182985)(527.11279096,45.72829859)(527.17279053,45.63830373)
\curveto(527.25279082,45.50829881)(527.32779075,45.36829895)(527.39779053,45.21830373)
\curveto(527.45779062,45.06829925)(527.51279056,44.90829941)(527.56279053,44.73830373)
\curveto(527.59279048,44.63829968)(527.61279046,44.52829979)(527.62279053,44.40830373)
\curveto(527.63279044,44.29830002)(527.64779043,44.18830013)(527.66779053,44.07830373)
\curveto(527.6777904,44.02830029)(527.68279039,43.98330033)(527.68279053,43.94330373)
\lineto(527.68279053,43.83830373)
\curveto(527.70279037,43.72830059)(527.70279037,43.62330069)(527.68279053,43.52330373)
\lineto(527.68279053,43.38830373)
\curveto(527.6727904,43.33830098)(527.66779041,43.28830103)(527.66779053,43.23830373)
\curveto(527.66779041,43.18830113)(527.65779042,43.14330117)(527.63779053,43.10330373)
\curveto(527.62779045,43.06330125)(527.62279045,43.02830129)(527.62279053,42.99830373)
\curveto(527.63279044,42.97830134)(527.63279044,42.95330136)(527.62279053,42.92330373)
\lineto(527.56279053,42.68330373)
\curveto(527.55279052,42.60330171)(527.53279054,42.52830179)(527.50279053,42.45830373)
\curveto(527.3727907,42.15830216)(527.22779085,41.9133024)(527.06779053,41.72330373)
\curveto(526.89779118,41.54330277)(526.66279141,41.39330292)(526.36279053,41.27330373)
\curveto(526.14279193,41.18330313)(525.8777922,41.13830318)(525.56779053,41.13830373)
\lineto(525.25279053,41.13830373)
\curveto(525.20279287,41.14830317)(525.15279292,41.15330316)(525.10279053,41.15330373)
\lineto(524.92279053,41.18330373)
\lineto(524.59279053,41.30330373)
\curveto(524.48279359,41.34330297)(524.38279369,41.39330292)(524.29279053,41.45330373)
\curveto(524.00279407,41.63330268)(523.78779429,41.87830244)(523.64779053,42.18830373)
\curveto(523.50779457,42.49830182)(523.38279469,42.83830148)(523.27279053,43.20830373)
\curveto(523.23279484,43.34830097)(523.20279487,43.49330082)(523.18279053,43.64330373)
\curveto(523.16279491,43.79330052)(523.13779494,43.94330037)(523.10779053,44.09330373)
\curveto(523.08779499,44.16330015)(523.077795,44.22830009)(523.07779053,44.28830373)
\curveto(523.077795,44.35829996)(523.06779501,44.43329988)(523.04779053,44.51330373)
\curveto(523.02779505,44.58329973)(523.01779506,44.65329966)(523.01779053,44.72330373)
\curveto(523.00779507,44.79329952)(522.99279508,44.86829945)(522.97279053,44.94830373)
\curveto(522.91279516,45.19829912)(522.86279521,45.43329888)(522.82279053,45.65330373)
\curveto(522.7727953,45.87329844)(522.65779542,46.04829827)(522.47779053,46.17830373)
\curveto(522.39779568,46.23829808)(522.29779578,46.28829803)(522.17779053,46.32830373)
\curveto(522.04779603,46.36829795)(521.90779617,46.36829795)(521.75779053,46.32830373)
\curveto(521.51779656,46.26829805)(521.32779675,46.17829814)(521.18779053,46.05830373)
\curveto(521.04779703,45.94829837)(520.93779714,45.78829853)(520.85779053,45.57830373)
\curveto(520.80779727,45.45829886)(520.7727973,45.313299)(520.75279053,45.14330373)
\curveto(520.73279734,44.98329933)(520.72279735,44.8132995)(520.72279053,44.63330373)
\curveto(520.72279735,44.45329986)(520.73279734,44.27830004)(520.75279053,44.10830373)
\curveto(520.7727973,43.93830038)(520.80279727,43.79330052)(520.84279053,43.67330373)
\curveto(520.90279717,43.50330081)(520.98779709,43.33830098)(521.09779053,43.17830373)
\curveto(521.15779692,43.09830122)(521.23779684,43.02330129)(521.33779053,42.95330373)
\curveto(521.42779665,42.89330142)(521.52779655,42.83830148)(521.63779053,42.78830373)
\curveto(521.71779636,42.75830156)(521.80279627,42.72830159)(521.89279053,42.69830373)
\curveto(521.98279609,42.67830164)(522.05279602,42.63330168)(522.10279053,42.56330373)
\curveto(522.13279594,42.52330179)(522.15779592,42.45330186)(522.17779053,42.35330373)
\curveto(522.18779589,42.26330205)(522.19279588,42.16830215)(522.19279053,42.06830373)
\curveto(522.19279588,41.96830235)(522.18779589,41.86830245)(522.17779053,41.76830373)
\curveto(522.15779592,41.67830264)(522.13279594,41.6133027)(522.10279053,41.57330373)
\curveto(522.072796,41.53330278)(522.02279605,41.50330281)(521.95279053,41.48330373)
\curveto(521.88279619,41.46330285)(521.80779627,41.46330285)(521.72779053,41.48330373)
\curveto(521.59779648,41.5133028)(521.4777966,41.54330277)(521.36779053,41.57330373)
\curveto(521.24779683,41.6133027)(521.13279694,41.65830266)(521.02279053,41.70830373)
\curveto(520.6727974,41.89830242)(520.40279767,42.13830218)(520.21279053,42.42830373)
\curveto(520.01279806,42.7183016)(519.85279822,43.07830124)(519.73279053,43.50830373)
\curveto(519.71279836,43.60830071)(519.69779838,43.70830061)(519.68779053,43.80830373)
\curveto(519.6777984,43.9183004)(519.66279841,44.02830029)(519.64279053,44.13830373)
\curveto(519.63279844,44.17830014)(519.63279844,44.24330007)(519.64279053,44.33330373)
\curveto(519.64279843,44.42329989)(519.63279844,44.47829984)(519.61279053,44.49830373)
\curveto(519.60279847,45.19829912)(519.68279839,45.80829851)(519.85279053,46.32830373)
\curveto(520.02279805,46.84829747)(520.34779773,47.2132971)(520.82779053,47.42330373)
\curveto(521.02779705,47.5132968)(521.26279681,47.56329675)(521.53279053,47.57330373)
\curveto(521.79279628,47.59329672)(522.06779601,47.60329671)(522.35779053,47.60330373)
\lineto(525.67279053,47.60330373)
\curveto(525.81279226,47.60329671)(525.94779213,47.60829671)(526.07779053,47.61830373)
\curveto(526.20779187,47.62829669)(526.31279176,47.65829666)(526.39279053,47.70830373)
\curveto(526.46279161,47.75829656)(526.51279156,47.82329649)(526.54279053,47.90330373)
\curveto(526.58279149,47.99329632)(526.61279146,48.07829624)(526.63279053,48.15830373)
\curveto(526.64279143,48.23829608)(526.68779139,48.29829602)(526.76779053,48.33830373)
\curveto(526.79779128,48.35829596)(526.82779125,48.36829595)(526.85779053,48.36830373)
\curveto(526.88779119,48.36829595)(526.92779115,48.37329594)(526.97779053,48.38330373)
\moveto(525.31279053,46.23830373)
\curveto(525.1727929,46.29829802)(525.01279306,46.32829799)(524.83279053,46.32830373)
\curveto(524.64279343,46.33829798)(524.44779363,46.34329797)(524.24779053,46.34330373)
\curveto(524.13779394,46.34329797)(524.03779404,46.33829798)(523.94779053,46.32830373)
\curveto(523.85779422,46.318298)(523.78779429,46.27829804)(523.73779053,46.20830373)
\curveto(523.71779436,46.17829814)(523.70779437,46.10829821)(523.70779053,45.99830373)
\curveto(523.72779435,45.97829834)(523.73779434,45.94329837)(523.73779053,45.89330373)
\curveto(523.73779434,45.84329847)(523.74779433,45.79829852)(523.76779053,45.75830373)
\curveto(523.78779429,45.67829864)(523.80779427,45.58829873)(523.82779053,45.48830373)
\lineto(523.88779053,45.18830373)
\curveto(523.88779419,45.15829916)(523.89279418,45.12329919)(523.90279053,45.08330373)
\lineto(523.90279053,44.97830373)
\curveto(523.94279413,44.82829949)(523.96779411,44.66329965)(523.97779053,44.48330373)
\curveto(523.9777941,44.3133)(523.99779408,44.15330016)(524.03779053,44.00330373)
\curveto(524.05779402,43.92330039)(524.077794,43.84830047)(524.09779053,43.77830373)
\curveto(524.10779397,43.7183006)(524.12279395,43.64830067)(524.14279053,43.56830373)
\curveto(524.19279388,43.40830091)(524.25779382,43.25830106)(524.33779053,43.11830373)
\curveto(524.40779367,42.97830134)(524.49779358,42.85830146)(524.60779053,42.75830373)
\curveto(524.71779336,42.65830166)(524.85279322,42.58330173)(525.01279053,42.53330373)
\curveto(525.16279291,42.48330183)(525.34779273,42.46330185)(525.56779053,42.47330373)
\curveto(525.66779241,42.47330184)(525.76279231,42.48830183)(525.85279053,42.51830373)
\curveto(525.93279214,42.55830176)(526.00779207,42.60330171)(526.07779053,42.65330373)
\curveto(526.18779189,42.73330158)(526.28279179,42.83830148)(526.36279053,42.96830373)
\curveto(526.43279164,43.09830122)(526.49279158,43.23830108)(526.54279053,43.38830373)
\curveto(526.55279152,43.43830088)(526.55779152,43.48830083)(526.55779053,43.53830373)
\curveto(526.55779152,43.58830073)(526.56279151,43.63830068)(526.57279053,43.68830373)
\curveto(526.59279148,43.75830056)(526.60779147,43.84330047)(526.61779053,43.94330373)
\curveto(526.61779146,44.05330026)(526.60779147,44.14330017)(526.58779053,44.21330373)
\curveto(526.56779151,44.27330004)(526.56279151,44.33329998)(526.57279053,44.39330373)
\curveto(526.5727915,44.45329986)(526.56279151,44.5132998)(526.54279053,44.57330373)
\curveto(526.52279155,44.65329966)(526.50779157,44.72829959)(526.49779053,44.79830373)
\curveto(526.48779159,44.87829944)(526.46779161,44.95329936)(526.43779053,45.02330373)
\curveto(526.31779176,45.313299)(526.1727919,45.55829876)(526.00279053,45.75830373)
\curveto(525.83279224,45.96829835)(525.60279247,46.12829819)(525.31279053,46.23830373)
}
}
{
\newrgbcolor{curcolor}{0 0 0}
\pscustom[linestyle=none,fillstyle=solid,fillcolor=curcolor]
{
\newpath
\moveto(519.80779053,49.26994436)
\lineto(519.80779053,49.71994436)
\curveto(519.79779828,49.88994311)(519.81779826,50.01494298)(519.86779053,50.09494436)
\curveto(519.91779816,50.17494282)(519.98279809,50.22994277)(520.06279053,50.25994436)
\curveto(520.14279793,50.2999427)(520.22779785,50.33994266)(520.31779053,50.37994436)
\curveto(520.44779763,50.42994257)(520.5777975,50.47494252)(520.70779053,50.51494436)
\curveto(520.83779724,50.55494244)(520.96779711,50.5999424)(521.09779053,50.64994436)
\curveto(521.21779686,50.6999423)(521.34279673,50.74494225)(521.47279053,50.78494436)
\curveto(521.59279648,50.82494217)(521.71279636,50.86994213)(521.83279053,50.91994436)
\curveto(521.94279613,50.96994203)(522.05779602,51.00994199)(522.17779053,51.03994436)
\curveto(522.29779578,51.06994193)(522.41779566,51.10994189)(522.53779053,51.15994436)
\curveto(522.82779525,51.27994172)(523.12779495,51.38994161)(523.43779053,51.48994436)
\curveto(523.74779433,51.58994141)(524.04779403,51.6999413)(524.33779053,51.81994436)
\curveto(524.3777937,51.83994116)(524.41779366,51.84994115)(524.45779053,51.84994436)
\curveto(524.48779359,51.84994115)(524.51779356,51.85994114)(524.54779053,51.87994436)
\curveto(524.68779339,51.93994106)(524.83279324,51.994941)(524.98279053,52.04494436)
\lineto(525.40279053,52.19494436)
\curveto(525.4727926,52.22494077)(525.54779253,52.25494074)(525.62779053,52.28494436)
\curveto(525.69779238,52.31494068)(525.74279233,52.36494063)(525.76279053,52.43494436)
\curveto(525.79279228,52.51494048)(525.76779231,52.57494042)(525.68779053,52.61494436)
\curveto(525.59779248,52.66494033)(525.52779255,52.6999403)(525.47779053,52.71994436)
\curveto(525.30779277,52.7999402)(525.12779295,52.86494013)(524.93779053,52.91494436)
\curveto(524.74779333,52.96494003)(524.56279351,53.02493997)(524.38279053,53.09494436)
\curveto(524.15279392,53.18493981)(523.92279415,53.26493973)(523.69279053,53.33494436)
\curveto(523.45279462,53.40493959)(523.22279485,53.48993951)(523.00279053,53.58994436)
\curveto(522.95279512,53.5999394)(522.88779519,53.61493938)(522.80779053,53.63494436)
\curveto(522.71779536,53.67493932)(522.62779545,53.70993929)(522.53779053,53.73994436)
\curveto(522.43779564,53.76993923)(522.34779573,53.7999392)(522.26779053,53.82994436)
\curveto(522.21779586,53.84993915)(522.1727959,53.86493913)(522.13279053,53.87494436)
\curveto(522.09279598,53.88493911)(522.04779603,53.8999391)(521.99779053,53.91994436)
\curveto(521.8777962,53.96993903)(521.75779632,54.00993899)(521.63779053,54.03994436)
\curveto(521.50779657,54.07993892)(521.38279669,54.12493887)(521.26279053,54.17494436)
\curveto(521.21279686,54.1949388)(521.16779691,54.20993879)(521.12779053,54.21994436)
\curveto(521.08779699,54.22993877)(521.04279703,54.24493875)(520.99279053,54.26494436)
\curveto(520.90279717,54.30493869)(520.81279726,54.33993866)(520.72279053,54.36994436)
\curveto(520.62279745,54.3999386)(520.52779755,54.42993857)(520.43779053,54.45994436)
\curveto(520.35779772,54.48993851)(520.2777978,54.51493848)(520.19779053,54.53494436)
\curveto(520.10779797,54.56493843)(520.03279804,54.60493839)(519.97279053,54.65494436)
\curveto(519.88279819,54.72493827)(519.83279824,54.81993818)(519.82279053,54.93994436)
\curveto(519.81279826,55.06993793)(519.80779827,55.20993779)(519.80779053,55.35994436)
\curveto(519.80779827,55.43993756)(519.81279826,55.51493748)(519.82279053,55.58494436)
\curveto(519.82279825,55.66493733)(519.83779824,55.72993727)(519.86779053,55.77994436)
\curveto(519.92779815,55.86993713)(520.02279805,55.8949371)(520.15279053,55.85494436)
\curveto(520.28279779,55.81493718)(520.38279769,55.77993722)(520.45279053,55.74994436)
\lineto(520.51279053,55.71994436)
\curveto(520.53279754,55.71993728)(520.55279752,55.71493728)(520.57279053,55.70494436)
\curveto(520.85279722,55.5949374)(521.13779694,55.48493751)(521.42779053,55.37494436)
\lineto(522.26779053,55.04494436)
\curveto(522.34779573,55.01493798)(522.42279565,54.98993801)(522.49279053,54.96994436)
\curveto(522.55279552,54.94993805)(522.61779546,54.92493807)(522.68779053,54.89494436)
\curveto(522.88779519,54.81493818)(523.09279498,54.73493826)(523.30279053,54.65494436)
\curveto(523.50279457,54.58493841)(523.70279437,54.50993849)(523.90279053,54.42994436)
\curveto(524.59279348,54.13993886)(525.28779279,53.86993913)(525.98779053,53.61994436)
\curveto(526.68779139,53.36993963)(527.38279069,53.0999399)(528.07279053,52.80994436)
\lineto(528.22279053,52.74994436)
\curveto(528.28278979,52.73994026)(528.34278973,52.72494027)(528.40279053,52.70494436)
\curveto(528.7727893,52.54494045)(529.13778894,52.37494062)(529.49779053,52.19494436)
\curveto(529.86778821,52.01494098)(530.15278792,51.76494123)(530.35279053,51.44494436)
\curveto(530.41278766,51.33494166)(530.45778762,51.22494177)(530.48779053,51.11494436)
\curveto(530.52778755,51.00494199)(530.56278751,50.87994212)(530.59279053,50.73994436)
\curveto(530.61278746,50.68994231)(530.61778746,50.63494236)(530.60779053,50.57494436)
\curveto(530.59778748,50.52494247)(530.59778748,50.46994253)(530.60779053,50.40994436)
\curveto(530.62778745,50.32994267)(530.62778745,50.24994275)(530.60779053,50.16994436)
\curveto(530.59778748,50.12994287)(530.59278748,50.07994292)(530.59279053,50.01994436)
\lineto(530.53279053,49.77994436)
\curveto(530.51278756,49.70994329)(530.4727876,49.65494334)(530.41279053,49.61494436)
\curveto(530.35278772,49.56494343)(530.2777878,49.53494346)(530.18779053,49.52494436)
\lineto(529.91779053,49.52494436)
\lineto(529.70779053,49.52494436)
\curveto(529.64778843,49.53494346)(529.59778848,49.55494344)(529.55779053,49.58494436)
\curveto(529.44778863,49.65494334)(529.41778866,49.77494322)(529.46779053,49.94494436)
\curveto(529.48778859,50.05494294)(529.49778858,50.17494282)(529.49779053,50.30494436)
\curveto(529.49778858,50.43494256)(529.4777886,50.54994245)(529.43779053,50.64994436)
\curveto(529.38778869,50.7999422)(529.31278876,50.91994208)(529.21279053,51.00994436)
\curveto(529.11278896,51.10994189)(528.99778908,51.1949418)(528.86779053,51.26494436)
\curveto(528.74778933,51.33494166)(528.61778946,51.3949416)(528.47779053,51.44494436)
\lineto(528.05779053,51.62494436)
\curveto(527.96779011,51.66494133)(527.85779022,51.70494129)(527.72779053,51.74494436)
\curveto(527.59779048,51.7949412)(527.46279061,51.7999412)(527.32279053,51.75994436)
\curveto(527.16279091,51.70994129)(527.01279106,51.65494134)(526.87279053,51.59494436)
\curveto(526.73279134,51.54494145)(526.59279148,51.48994151)(526.45279053,51.42994436)
\curveto(526.24279183,51.33994166)(526.03279204,51.25494174)(525.82279053,51.17494436)
\curveto(525.61279246,51.0949419)(525.40779267,51.01494198)(525.20779053,50.93494436)
\curveto(525.06779301,50.87494212)(524.93279314,50.81994218)(524.80279053,50.76994436)
\curveto(524.6727934,50.71994228)(524.53779354,50.66994233)(524.39779053,50.61994436)
\lineto(523.07779053,50.07994436)
\curveto(522.63779544,49.90994309)(522.19779588,49.73494326)(521.75779053,49.55494436)
\curveto(521.52779655,49.45494354)(521.30779677,49.36494363)(521.09779053,49.28494436)
\curveto(520.8777972,49.20494379)(520.65779742,49.11994388)(520.43779053,49.02994436)
\curveto(520.3777977,49.00994399)(520.29779778,48.97994402)(520.19779053,48.93994436)
\curveto(520.08779799,48.8999441)(519.99779808,48.90494409)(519.92779053,48.95494436)
\curveto(519.8777982,48.98494401)(519.84279823,49.04494395)(519.82279053,49.13494436)
\curveto(519.81279826,49.15494384)(519.81279826,49.17494382)(519.82279053,49.19494436)
\curveto(519.82279825,49.22494377)(519.81779826,49.24994375)(519.80779053,49.26994436)
}
}
{
\newrgbcolor{curcolor}{0 0 0}
\pscustom[linestyle=none,fillstyle=solid,fillcolor=curcolor]
{
}
}
{
\newrgbcolor{curcolor}{0 0 0}
\pscustom[linestyle=none,fillstyle=solid,fillcolor=curcolor]
{
\newpath
\moveto(522.44779053,67.99510061)
\lineto(522.70279053,67.99510061)
\curveto(522.78279529,68.0050929)(522.85779522,68.00009291)(522.92779053,67.98010061)
\lineto(523.16779053,67.98010061)
\lineto(523.33279053,67.98010061)
\curveto(523.43279464,67.96009295)(523.53779454,67.95009296)(523.64779053,67.95010061)
\curveto(523.74779433,67.95009296)(523.84779423,67.94009297)(523.94779053,67.92010061)
\lineto(524.09779053,67.92010061)
\curveto(524.23779384,67.89009302)(524.3777937,67.87009304)(524.51779053,67.86010061)
\curveto(524.64779343,67.85009306)(524.7777933,67.82509308)(524.90779053,67.78510061)
\curveto(524.98779309,67.76509314)(525.072793,67.74509316)(525.16279053,67.72510061)
\lineto(525.40279053,67.66510061)
\lineto(525.70279053,67.54510061)
\curveto(525.79279228,67.51509339)(525.88279219,67.48009343)(525.97279053,67.44010061)
\curveto(526.19279188,67.34009357)(526.40779167,67.2050937)(526.61779053,67.03510061)
\curveto(526.82779125,66.87509403)(526.99779108,66.70009421)(527.12779053,66.51010061)
\curveto(527.16779091,66.46009445)(527.20779087,66.40009451)(527.24779053,66.33010061)
\curveto(527.2777908,66.27009464)(527.31279076,66.2100947)(527.35279053,66.15010061)
\curveto(527.40279067,66.07009484)(527.44279063,65.97509493)(527.47279053,65.86510061)
\curveto(527.50279057,65.75509515)(527.53279054,65.65009526)(527.56279053,65.55010061)
\curveto(527.60279047,65.44009547)(527.62779045,65.33009558)(527.63779053,65.22010061)
\curveto(527.64779043,65.1100958)(527.66279041,64.99509591)(527.68279053,64.87510061)
\curveto(527.69279038,64.83509607)(527.69279038,64.79009612)(527.68279053,64.74010061)
\curveto(527.68279039,64.70009621)(527.68779039,64.66009625)(527.69779053,64.62010061)
\curveto(527.70779037,64.58009633)(527.71279036,64.52509638)(527.71279053,64.45510061)
\curveto(527.71279036,64.38509652)(527.70779037,64.33509657)(527.69779053,64.30510061)
\curveto(527.6777904,64.25509665)(527.6727904,64.2100967)(527.68279053,64.17010061)
\curveto(527.69279038,64.13009678)(527.69279038,64.09509681)(527.68279053,64.06510061)
\lineto(527.68279053,63.97510061)
\curveto(527.66279041,63.91509699)(527.64779043,63.85009706)(527.63779053,63.78010061)
\curveto(527.63779044,63.72009719)(527.63279044,63.65509725)(527.62279053,63.58510061)
\curveto(527.5727905,63.41509749)(527.52279055,63.25509765)(527.47279053,63.10510061)
\curveto(527.42279065,62.95509795)(527.35779072,62.8100981)(527.27779053,62.67010061)
\curveto(527.23779084,62.62009829)(527.20779087,62.56509834)(527.18779053,62.50510061)
\curveto(527.15779092,62.45509845)(527.12279095,62.4050985)(527.08279053,62.35510061)
\curveto(526.90279117,62.11509879)(526.68279139,61.91509899)(526.42279053,61.75510061)
\curveto(526.16279191,61.59509931)(525.8777922,61.45509945)(525.56779053,61.33510061)
\curveto(525.42779265,61.27509963)(525.28779279,61.23009968)(525.14779053,61.20010061)
\curveto(524.99779308,61.17009974)(524.84279323,61.13509977)(524.68279053,61.09510061)
\curveto(524.5727935,61.07509983)(524.46279361,61.06009985)(524.35279053,61.05010061)
\curveto(524.24279383,61.04009987)(524.13279394,61.02509988)(524.02279053,61.00510061)
\curveto(523.98279409,60.99509991)(523.94279413,60.99009992)(523.90279053,60.99010061)
\curveto(523.86279421,61.00009991)(523.82279425,61.00009991)(523.78279053,60.99010061)
\curveto(523.73279434,60.98009993)(523.68279439,60.97509993)(523.63279053,60.97510061)
\lineto(523.46779053,60.97510061)
\curveto(523.41779466,60.95509995)(523.36779471,60.95009996)(523.31779053,60.96010061)
\curveto(523.25779482,60.97009994)(523.20279487,60.97009994)(523.15279053,60.96010061)
\curveto(523.11279496,60.95009996)(523.06779501,60.95009996)(523.01779053,60.96010061)
\curveto(522.96779511,60.97009994)(522.91779516,60.96509994)(522.86779053,60.94510061)
\curveto(522.79779528,60.92509998)(522.72279535,60.92009999)(522.64279053,60.93010061)
\curveto(522.55279552,60.94009997)(522.46779561,60.94509996)(522.38779053,60.94510061)
\curveto(522.29779578,60.94509996)(522.19779588,60.94009997)(522.08779053,60.93010061)
\curveto(521.96779611,60.92009999)(521.86779621,60.92509998)(521.78779053,60.94510061)
\lineto(521.50279053,60.94510061)
\lineto(520.87279053,60.99010061)
\curveto(520.7727973,61.00009991)(520.6777974,61.0100999)(520.58779053,61.02010061)
\lineto(520.28779053,61.05010061)
\curveto(520.23779784,61.07009984)(520.18779789,61.07509983)(520.13779053,61.06510061)
\curveto(520.077798,61.06509984)(520.02279805,61.07509983)(519.97279053,61.09510061)
\curveto(519.80279827,61.14509976)(519.63779844,61.18509972)(519.47779053,61.21510061)
\curveto(519.30779877,61.24509966)(519.14779893,61.29509961)(518.99779053,61.36510061)
\curveto(518.53779954,61.55509935)(518.16279991,61.77509913)(517.87279053,62.02510061)
\curveto(517.58280049,62.28509862)(517.33780074,62.64509826)(517.13779053,63.10510061)
\curveto(517.08780099,63.23509767)(517.05280102,63.36509754)(517.03279053,63.49510061)
\curveto(517.01280106,63.63509727)(516.98780109,63.77509713)(516.95779053,63.91510061)
\curveto(516.94780113,63.98509692)(516.94280113,64.05009686)(516.94279053,64.11010061)
\curveto(516.94280113,64.17009674)(516.93780114,64.23509667)(516.92779053,64.30510061)
\curveto(516.90780117,65.13509577)(517.05780102,65.8050951)(517.37779053,66.31510061)
\curveto(517.68780039,66.82509408)(518.12779995,67.2050937)(518.69779053,67.45510061)
\curveto(518.81779926,67.5050934)(518.94279913,67.55009336)(519.07279053,67.59010061)
\curveto(519.20279887,67.63009328)(519.33779874,67.67509323)(519.47779053,67.72510061)
\curveto(519.55779852,67.74509316)(519.64279843,67.76009315)(519.73279053,67.77010061)
\lineto(519.97279053,67.83010061)
\curveto(520.08279799,67.86009305)(520.19279788,67.87509303)(520.30279053,67.87510061)
\curveto(520.41279766,67.88509302)(520.52279755,67.90009301)(520.63279053,67.92010061)
\curveto(520.68279739,67.94009297)(520.72779735,67.94509296)(520.76779053,67.93510061)
\curveto(520.80779727,67.93509297)(520.84779723,67.94009297)(520.88779053,67.95010061)
\curveto(520.93779714,67.96009295)(520.99279708,67.96009295)(521.05279053,67.95010061)
\curveto(521.10279697,67.95009296)(521.15279692,67.95509295)(521.20279053,67.96510061)
\lineto(521.33779053,67.96510061)
\curveto(521.39779668,67.98509292)(521.46779661,67.98509292)(521.54779053,67.96510061)
\curveto(521.61779646,67.95509295)(521.68279639,67.96009295)(521.74279053,67.98010061)
\curveto(521.7727963,67.99009292)(521.81279626,67.99509291)(521.86279053,67.99510061)
\lineto(521.98279053,67.99510061)
\lineto(522.44779053,67.99510061)
\moveto(524.77279053,66.45010061)
\curveto(524.45279362,66.55009436)(524.08779399,66.6100943)(523.67779053,66.63010061)
\curveto(523.26779481,66.65009426)(522.85779522,66.66009425)(522.44779053,66.66010061)
\curveto(522.01779606,66.66009425)(521.59779648,66.65009426)(521.18779053,66.63010061)
\curveto(520.7777973,66.6100943)(520.39279768,66.56509434)(520.03279053,66.49510061)
\curveto(519.6727984,66.42509448)(519.35279872,66.31509459)(519.07279053,66.16510061)
\curveto(518.78279929,66.02509488)(518.54779953,65.83009508)(518.36779053,65.58010061)
\curveto(518.25779982,65.42009549)(518.1777999,65.24009567)(518.12779053,65.04010061)
\curveto(518.06780001,64.84009607)(518.03780004,64.59509631)(518.03779053,64.30510061)
\curveto(518.05780002,64.28509662)(518.06780001,64.25009666)(518.06779053,64.20010061)
\curveto(518.05780002,64.15009676)(518.05780002,64.1100968)(518.06779053,64.08010061)
\curveto(518.08779999,64.00009691)(518.10779997,63.92509698)(518.12779053,63.85510061)
\curveto(518.13779994,63.79509711)(518.15779992,63.73009718)(518.18779053,63.66010061)
\curveto(518.30779977,63.39009752)(518.4777996,63.17009774)(518.69779053,63.00010061)
\curveto(518.90779917,62.84009807)(519.15279892,62.7050982)(519.43279053,62.59510061)
\curveto(519.54279853,62.54509836)(519.66279841,62.5050984)(519.79279053,62.47510061)
\curveto(519.91279816,62.45509845)(520.03779804,62.43009848)(520.16779053,62.40010061)
\curveto(520.21779786,62.38009853)(520.2727978,62.37009854)(520.33279053,62.37010061)
\curveto(520.38279769,62.37009854)(520.43279764,62.36509854)(520.48279053,62.35510061)
\curveto(520.5727975,62.34509856)(520.66779741,62.33509857)(520.76779053,62.32510061)
\curveto(520.85779722,62.31509859)(520.95279712,62.3050986)(521.05279053,62.29510061)
\curveto(521.13279694,62.29509861)(521.21779686,62.29009862)(521.30779053,62.28010061)
\lineto(521.54779053,62.28010061)
\lineto(521.72779053,62.28010061)
\curveto(521.75779632,62.27009864)(521.79279628,62.26509864)(521.83279053,62.26510061)
\lineto(521.96779053,62.26510061)
\lineto(522.41779053,62.26510061)
\curveto(522.49779558,62.26509864)(522.58279549,62.26009865)(522.67279053,62.25010061)
\curveto(522.75279532,62.25009866)(522.82779525,62.26009865)(522.89779053,62.28010061)
\lineto(523.16779053,62.28010061)
\curveto(523.18779489,62.28009863)(523.21779486,62.27509863)(523.25779053,62.26510061)
\curveto(523.28779479,62.26509864)(523.31279476,62.27009864)(523.33279053,62.28010061)
\curveto(523.43279464,62.29009862)(523.53279454,62.29509861)(523.63279053,62.29510061)
\curveto(523.72279435,62.3050986)(523.82279425,62.31509859)(523.93279053,62.32510061)
\curveto(524.05279402,62.35509855)(524.1777939,62.37009854)(524.30779053,62.37010061)
\curveto(524.42779365,62.38009853)(524.54279353,62.4050985)(524.65279053,62.44510061)
\curveto(524.95279312,62.52509838)(525.21779286,62.6100983)(525.44779053,62.70010061)
\curveto(525.6777924,62.80009811)(525.89279218,62.94509796)(526.09279053,63.13510061)
\curveto(526.29279178,63.34509756)(526.44279163,63.6100973)(526.54279053,63.93010061)
\curveto(526.56279151,63.97009694)(526.5727915,64.0050969)(526.57279053,64.03510061)
\curveto(526.56279151,64.07509683)(526.56779151,64.12009679)(526.58779053,64.17010061)
\curveto(526.59779148,64.2100967)(526.60779147,64.28009663)(526.61779053,64.38010061)
\curveto(526.62779145,64.49009642)(526.62279145,64.57509633)(526.60279053,64.63510061)
\curveto(526.58279149,64.7050962)(526.5727915,64.77509613)(526.57279053,64.84510061)
\curveto(526.56279151,64.91509599)(526.54779153,64.98009593)(526.52779053,65.04010061)
\curveto(526.46779161,65.24009567)(526.38279169,65.42009549)(526.27279053,65.58010061)
\curveto(526.25279182,65.6100953)(526.23279184,65.63509527)(526.21279053,65.65510061)
\lineto(526.15279053,65.71510061)
\curveto(526.13279194,65.75509515)(526.09279198,65.8050951)(526.03279053,65.86510061)
\curveto(525.89279218,65.96509494)(525.76279231,66.05009486)(525.64279053,66.12010061)
\curveto(525.52279255,66.19009472)(525.3777927,66.26009465)(525.20779053,66.33010061)
\curveto(525.13779294,66.36009455)(525.06779301,66.38009453)(524.99779053,66.39010061)
\curveto(524.92779315,66.4100945)(524.85279322,66.43009448)(524.77279053,66.45010061)
}
}
{
\newrgbcolor{curcolor}{0 0 0}
\pscustom[linestyle=none,fillstyle=solid,fillcolor=curcolor]
{
\newpath
\moveto(521.93779053,76.28470998)
\curveto(522.01779606,76.28470235)(522.09779598,76.28970234)(522.17779053,76.29970998)
\curveto(522.25779582,76.30970232)(522.33279574,76.30470233)(522.40279053,76.28470998)
\curveto(522.44279563,76.26470237)(522.48779559,76.25970237)(522.53779053,76.26970998)
\curveto(522.5777955,76.27970235)(522.61779546,76.27970235)(522.65779053,76.26970998)
\lineto(522.80779053,76.26970998)
\curveto(522.89779518,76.25970237)(522.98779509,76.25470238)(523.07779053,76.25470998)
\curveto(523.15779492,76.25470238)(523.23779484,76.24970238)(523.31779053,76.23970998)
\lineto(523.55779053,76.20970998)
\curveto(523.62779445,76.19970243)(523.70279437,76.18970244)(523.78279053,76.17970998)
\curveto(523.82279425,76.16970246)(523.86279421,76.16470247)(523.90279053,76.16470998)
\curveto(523.94279413,76.16470247)(523.98779409,76.15970247)(524.03779053,76.14970998)
\curveto(524.1777939,76.10970252)(524.31779376,76.07970255)(524.45779053,76.05970998)
\curveto(524.59779348,76.04970258)(524.73279334,76.01970261)(524.86279053,75.96970998)
\curveto(525.03279304,75.91970271)(525.19779288,75.86470277)(525.35779053,75.80470998)
\curveto(525.51779256,75.75470288)(525.6727924,75.69470294)(525.82279053,75.62470998)
\curveto(525.88279219,75.60470303)(525.94279213,75.57470306)(526.00279053,75.53470998)
\lineto(526.15279053,75.44470998)
\curveto(526.4727916,75.24470339)(526.73779134,75.0297036)(526.94779053,74.79970998)
\curveto(527.15779092,74.56970406)(527.33779074,74.27470436)(527.48779053,73.91470998)
\curveto(527.53779054,73.79470484)(527.5727905,73.66470497)(527.59279053,73.52470998)
\curveto(527.61279046,73.39470524)(527.63779044,73.25970537)(527.66779053,73.11970998)
\curveto(527.6777904,73.05970557)(527.68279039,72.99970563)(527.68279053,72.93970998)
\curveto(527.68279039,72.87970575)(527.68779039,72.81470582)(527.69779053,72.74470998)
\curveto(527.70779037,72.71470592)(527.70779037,72.66470597)(527.69779053,72.59470998)
\lineto(527.69779053,72.44470998)
\lineto(527.69779053,72.29470998)
\curveto(527.6777904,72.21470642)(527.66279041,72.1297065)(527.65279053,72.03970998)
\curveto(527.65279042,71.95970667)(527.64279043,71.88470675)(527.62279053,71.81470998)
\curveto(527.61279046,71.77470686)(527.60779047,71.73970689)(527.60779053,71.70970998)
\curveto(527.61779046,71.68970694)(527.61279046,71.66470697)(527.59279053,71.63470998)
\lineto(527.53279053,71.36470998)
\curveto(527.50279057,71.27470736)(527.4727906,71.18970744)(527.44279053,71.10970998)
\curveto(527.20279087,70.5297081)(526.83279124,70.09470854)(526.33279053,69.80470998)
\curveto(526.20279187,69.72470891)(526.06779201,69.65970897)(525.92779053,69.60970998)
\curveto(525.78779229,69.56970906)(525.63779244,69.52470911)(525.47779053,69.47470998)
\curveto(525.39779268,69.45470918)(525.31779276,69.44970918)(525.23779053,69.45970998)
\curveto(525.15779292,69.47970915)(525.10279297,69.51470912)(525.07279053,69.56470998)
\curveto(525.05279302,69.59470904)(525.03779304,69.64970898)(525.02779053,69.72970998)
\curveto(525.00779307,69.80970882)(524.99779308,69.89470874)(524.99779053,69.98470998)
\curveto(524.98779309,70.07470856)(524.98779309,70.15970847)(524.99779053,70.23970998)
\curveto(525.00779307,70.3297083)(525.01779306,70.39970823)(525.02779053,70.44970998)
\curveto(525.03779304,70.46970816)(525.05279302,70.49470814)(525.07279053,70.52470998)
\curveto(525.09279298,70.56470807)(525.11279296,70.59470804)(525.13279053,70.61470998)
\curveto(525.21279286,70.67470796)(525.30779277,70.71970791)(525.41779053,70.74970998)
\curveto(525.52779255,70.78970784)(525.62779245,70.8347078)(525.71779053,70.88470998)
\curveto(526.10779197,71.1347075)(526.3777917,71.50470713)(526.52779053,71.99470998)
\curveto(526.54779153,72.06470657)(526.56279151,72.1347065)(526.57279053,72.20470998)
\curveto(526.5727915,72.28470635)(526.58279149,72.36470627)(526.60279053,72.44470998)
\curveto(526.61279146,72.48470615)(526.61779146,72.53970609)(526.61779053,72.60970998)
\curveto(526.61779146,72.68970594)(526.61279146,72.74470589)(526.60279053,72.77470998)
\curveto(526.59279148,72.80470583)(526.58779149,72.8347058)(526.58779053,72.86470998)
\lineto(526.58779053,72.96970998)
\curveto(526.56779151,73.04970558)(526.54779153,73.12470551)(526.52779053,73.19470998)
\curveto(526.50779157,73.27470536)(526.48279159,73.34970528)(526.45279053,73.41970998)
\curveto(526.30279177,73.76970486)(526.08779199,74.03970459)(525.80779053,74.22970998)
\curveto(525.52779255,74.41970421)(525.20279287,74.57470406)(524.83279053,74.69470998)
\curveto(524.75279332,74.72470391)(524.6777934,74.74470389)(524.60779053,74.75470998)
\curveto(524.53779354,74.77470386)(524.46279361,74.79470384)(524.38279053,74.81470998)
\curveto(524.29279378,74.8347038)(524.19779388,74.84970378)(524.09779053,74.85970998)
\curveto(523.98779409,74.87970375)(523.88279419,74.89970373)(523.78279053,74.91970998)
\curveto(523.73279434,74.9297037)(523.68279439,74.9347037)(523.63279053,74.93470998)
\curveto(523.5727945,74.94470369)(523.51779456,74.94970368)(523.46779053,74.94970998)
\curveto(523.40779467,74.96970366)(523.33279474,74.97970365)(523.24279053,74.97970998)
\curveto(523.14279493,74.97970365)(523.06279501,74.96970366)(523.00279053,74.94970998)
\curveto(522.91279516,74.91970371)(522.8727952,74.86970376)(522.88279053,74.79970998)
\curveto(522.89279518,74.73970389)(522.92279515,74.68470395)(522.97279053,74.63470998)
\curveto(523.02279505,74.55470408)(523.08279499,74.48470415)(523.15279053,74.42470998)
\curveto(523.22279485,74.37470426)(523.28279479,74.30970432)(523.33279053,74.22970998)
\curveto(523.44279463,74.06970456)(523.54279453,73.90470473)(523.63279053,73.73470998)
\curveto(523.71279436,73.56470507)(523.78279429,73.36970526)(523.84279053,73.14970998)
\curveto(523.8727942,73.04970558)(523.88779419,72.94970568)(523.88779053,72.84970998)
\curveto(523.88779419,72.75970587)(523.89779418,72.65970597)(523.91779053,72.54970998)
\lineto(523.91779053,72.39970998)
\curveto(523.89779418,72.34970628)(523.89279418,72.29970633)(523.90279053,72.24970998)
\curveto(523.91279416,72.20970642)(523.91279416,72.16970646)(523.90279053,72.12970998)
\curveto(523.89279418,72.09970653)(523.88779419,72.05470658)(523.88779053,71.99470998)
\curveto(523.8777942,71.9347067)(523.86779421,71.86970676)(523.85779053,71.79970998)
\lineto(523.82779053,71.61970998)
\curveto(523.70779437,71.16970746)(523.54279453,70.78970784)(523.33279053,70.47970998)
\curveto(523.14279493,70.20970842)(522.91279516,69.97970865)(522.64279053,69.78970998)
\curveto(522.36279571,69.60970902)(522.04779603,69.46470917)(521.69779053,69.35470998)
\lineto(521.48779053,69.29470998)
\curveto(521.40779667,69.28470935)(521.32779675,69.26970936)(521.24779053,69.24970998)
\curveto(521.21779686,69.23970939)(521.18779689,69.2347094)(521.15779053,69.23470998)
\curveto(521.12779695,69.2347094)(521.09779698,69.2297094)(521.06779053,69.21970998)
\curveto(521.00779707,69.20970942)(520.94779713,69.20470943)(520.88779053,69.20470998)
\curveto(520.81779726,69.20470943)(520.75779732,69.19470944)(520.70779053,69.17470998)
\lineto(520.52779053,69.17470998)
\curveto(520.4777976,69.16470947)(520.40779767,69.15970947)(520.31779053,69.15970998)
\curveto(520.22779785,69.15970947)(520.15779792,69.16970946)(520.10779053,69.18970998)
\lineto(519.94279053,69.18970998)
\curveto(519.86279821,69.20970942)(519.78779829,69.21970941)(519.71779053,69.21970998)
\curveto(519.64779843,69.2297094)(519.5777985,69.24470939)(519.50779053,69.26470998)
\curveto(519.30779877,69.32470931)(519.11779896,69.38470925)(518.93779053,69.44470998)
\curveto(518.75779932,69.51470912)(518.58779949,69.60470903)(518.42779053,69.71470998)
\curveto(518.35779972,69.75470888)(518.29279978,69.79470884)(518.23279053,69.83470998)
\lineto(518.05279053,69.98470998)
\curveto(518.04280003,70.00470863)(518.02780005,70.02470861)(518.00779053,70.04470998)
\curveto(517.8778002,70.1347085)(517.76780031,70.24470839)(517.67779053,70.37470998)
\curveto(517.4778006,70.634708)(517.32280075,70.89970773)(517.21279053,71.16970998)
\curveto(517.1728009,71.24970738)(517.14280093,71.3297073)(517.12279053,71.40970998)
\curveto(517.09280098,71.49970713)(517.06780101,71.58970704)(517.04779053,71.67970998)
\curveto(517.01780106,71.77970685)(516.99780108,71.87970675)(516.98779053,71.97970998)
\curveto(516.9778011,72.07970655)(516.96280111,72.18470645)(516.94279053,72.29470998)
\curveto(516.93280114,72.32470631)(516.93280114,72.36470627)(516.94279053,72.41470998)
\curveto(516.95280112,72.47470616)(516.94780113,72.51470612)(516.92779053,72.53470998)
\curveto(516.90780117,73.25470538)(517.02280105,73.85470478)(517.27279053,74.33470998)
\curveto(517.52280055,74.81470382)(517.86280021,75.18970344)(518.29279053,75.45970998)
\curveto(518.43279964,75.54970308)(518.5777995,75.629703)(518.72779053,75.69970998)
\curveto(518.8777992,75.76970286)(519.03779904,75.83970279)(519.20779053,75.90970998)
\curveto(519.34779873,75.95970267)(519.49779858,75.99970263)(519.65779053,76.02970998)
\curveto(519.81779826,76.05970257)(519.9777981,76.09470254)(520.13779053,76.13470998)
\curveto(520.18779789,76.15470248)(520.24279783,76.16470247)(520.30279053,76.16470998)
\curveto(520.35279772,76.16470247)(520.40279767,76.16970246)(520.45279053,76.17970998)
\curveto(520.51279756,76.19970243)(520.5777975,76.20970242)(520.64779053,76.20970998)
\curveto(520.70779737,76.20970242)(520.76279731,76.21970241)(520.81279053,76.23970998)
\lineto(520.97779053,76.23970998)
\curveto(521.02779705,76.25970237)(521.077797,76.26470237)(521.12779053,76.25470998)
\curveto(521.1777969,76.24470239)(521.22779685,76.24970238)(521.27779053,76.26970998)
\curveto(521.29779678,76.26970236)(521.32279675,76.26470237)(521.35279053,76.25470998)
\curveto(521.38279669,76.25470238)(521.40779667,76.25970237)(521.42779053,76.26970998)
\curveto(521.45779662,76.27970235)(521.49279658,76.27970235)(521.53279053,76.26970998)
\curveto(521.5727965,76.26970236)(521.61279646,76.27470236)(521.65279053,76.28470998)
\curveto(521.69279638,76.29470234)(521.73779634,76.29470234)(521.78779053,76.28470998)
\lineto(521.93779053,76.28470998)
\moveto(520.63279053,74.78470998)
\curveto(520.58279749,74.79470384)(520.52279755,74.79970383)(520.45279053,74.79970998)
\curveto(520.38279769,74.79970383)(520.32279775,74.79470384)(520.27279053,74.78470998)
\curveto(520.22279785,74.77470386)(520.14779793,74.76970386)(520.04779053,74.76970998)
\curveto(519.96779811,74.74970388)(519.89279818,74.7297039)(519.82279053,74.70970998)
\curveto(519.75279832,74.69970393)(519.68279839,74.68470395)(519.61279053,74.66470998)
\curveto(519.18279889,74.52470411)(518.84779923,74.3297043)(518.60779053,74.07970998)
\curveto(518.36779971,73.83970479)(518.18779989,73.49470514)(518.06779053,73.04470998)
\curveto(518.04780003,72.95470568)(518.03780004,72.85470578)(518.03779053,72.74470998)
\lineto(518.03779053,72.41470998)
\curveto(518.05780002,72.39470624)(518.06780001,72.35970627)(518.06779053,72.30970998)
\curveto(518.05780002,72.25970637)(518.05780002,72.21470642)(518.06779053,72.17470998)
\curveto(518.08779999,72.09470654)(518.10779997,72.01970661)(518.12779053,71.94970998)
\lineto(518.18779053,71.73970998)
\curveto(518.31779976,71.44970718)(518.49779958,71.21970741)(518.72779053,71.04970998)
\curveto(518.94779913,70.87970775)(519.20779887,70.74470789)(519.50779053,70.64470998)
\curveto(519.59779848,70.61470802)(519.69279838,70.58970804)(519.79279053,70.56970998)
\curveto(519.88279819,70.55970807)(519.9777981,70.54470809)(520.07779053,70.52470998)
\lineto(520.21279053,70.52470998)
\curveto(520.32279775,70.49470814)(520.46279761,70.48470815)(520.63279053,70.49470998)
\curveto(520.79279728,70.51470812)(520.92279715,70.5347081)(521.02279053,70.55470998)
\curveto(521.08279699,70.57470806)(521.14279693,70.58970804)(521.20279053,70.59970998)
\curveto(521.25279682,70.60970802)(521.30279677,70.62470801)(521.35279053,70.64470998)
\curveto(521.55279652,70.72470791)(521.74279633,70.81970781)(521.92279053,70.92970998)
\curveto(522.10279597,71.04970758)(522.24779583,71.18970744)(522.35779053,71.34970998)
\curveto(522.40779567,71.39970723)(522.44779563,71.45470718)(522.47779053,71.51470998)
\curveto(522.50779557,71.57470706)(522.54279553,71.634707)(522.58279053,71.69470998)
\curveto(522.66279541,71.84470679)(522.72779535,72.0297066)(522.77779053,72.24970998)
\curveto(522.79779528,72.29970633)(522.80279527,72.33970629)(522.79279053,72.36970998)
\curveto(522.78279529,72.40970622)(522.78779529,72.45470618)(522.80779053,72.50470998)
\curveto(522.81779526,72.54470609)(522.82279525,72.59970603)(522.82279053,72.66970998)
\curveto(522.82279525,72.73970589)(522.81779526,72.79970583)(522.80779053,72.84970998)
\curveto(522.78779529,72.94970568)(522.7727953,73.04470559)(522.76279053,73.13470998)
\curveto(522.74279533,73.22470541)(522.71279536,73.31470532)(522.67279053,73.40470998)
\curveto(522.45279562,73.94470469)(522.05779602,74.33970429)(521.48779053,74.58970998)
\curveto(521.38779669,74.63970399)(521.28779679,74.67470396)(521.18779053,74.69470998)
\curveto(521.077797,74.71470392)(520.96779711,74.73970389)(520.85779053,74.76970998)
\curveto(520.75779732,74.76970386)(520.68279739,74.77470386)(520.63279053,74.78470998)
}
}
{
\newrgbcolor{curcolor}{0 0 0}
\pscustom[linestyle=none,fillstyle=solid,fillcolor=curcolor]
{
\newpath
\moveto(525.89779053,78.63431936)
\lineto(525.89779053,79.26431936)
\lineto(525.89779053,79.45931936)
\curveto(525.89779218,79.52931683)(525.90779217,79.58931677)(525.92779053,79.63931936)
\curveto(525.96779211,79.70931665)(526.00779207,79.7593166)(526.04779053,79.78931936)
\curveto(526.09779198,79.82931653)(526.16279191,79.84931651)(526.24279053,79.84931936)
\curveto(526.32279175,79.8593165)(526.40779167,79.86431649)(526.49779053,79.86431936)
\lineto(527.21779053,79.86431936)
\curveto(527.69779038,79.86431649)(528.10778997,79.80431655)(528.44779053,79.68431936)
\curveto(528.78778929,79.56431679)(529.06278901,79.36931699)(529.27279053,79.09931936)
\curveto(529.32278875,79.02931733)(529.36778871,78.9593174)(529.40779053,78.88931936)
\curveto(529.45778862,78.82931753)(529.50278857,78.7543176)(529.54279053,78.66431936)
\curveto(529.55278852,78.64431771)(529.56278851,78.61431774)(529.57279053,78.57431936)
\curveto(529.59278848,78.53431782)(529.59778848,78.48931787)(529.58779053,78.43931936)
\curveto(529.55778852,78.34931801)(529.48278859,78.29431806)(529.36279053,78.27431936)
\curveto(529.25278882,78.2543181)(529.15778892,78.26931809)(529.07779053,78.31931936)
\curveto(529.00778907,78.34931801)(528.94278913,78.39431796)(528.88279053,78.45431936)
\curveto(528.83278924,78.52431783)(528.78278929,78.58931777)(528.73279053,78.64931936)
\curveto(528.68278939,78.71931764)(528.60778947,78.77931758)(528.50779053,78.82931936)
\curveto(528.41778966,78.88931747)(528.32778975,78.93931742)(528.23779053,78.97931936)
\curveto(528.20778987,78.99931736)(528.14778993,79.02431733)(528.05779053,79.05431936)
\curveto(527.9777901,79.08431727)(527.90779017,79.08931727)(527.84779053,79.06931936)
\curveto(527.70779037,79.03931732)(527.61779046,78.97931738)(527.57779053,78.88931936)
\curveto(527.54779053,78.80931755)(527.53279054,78.71931764)(527.53279053,78.61931936)
\curveto(527.53279054,78.51931784)(527.50779057,78.43431792)(527.45779053,78.36431936)
\curveto(527.38779069,78.27431808)(527.24779083,78.22931813)(527.03779053,78.22931936)
\lineto(526.48279053,78.22931936)
\lineto(526.25779053,78.22931936)
\curveto(526.1777919,78.23931812)(526.11279196,78.2593181)(526.06279053,78.28931936)
\curveto(525.98279209,78.34931801)(525.93779214,78.41931794)(525.92779053,78.49931936)
\curveto(525.91779216,78.51931784)(525.91279216,78.53931782)(525.91279053,78.55931936)
\curveto(525.91279216,78.58931777)(525.90779217,78.61431774)(525.89779053,78.63431936)
}
}
{
\newrgbcolor{curcolor}{0 0 0}
\pscustom[linestyle=none,fillstyle=solid,fillcolor=curcolor]
{
}
}
{
\newrgbcolor{curcolor}{0 0 0}
\pscustom[linestyle=none,fillstyle=solid,fillcolor=curcolor]
{
\newpath
\moveto(516.92779053,89.26463186)
\curveto(516.91780116,89.95462722)(517.03780104,90.55462662)(517.28779053,91.06463186)
\curveto(517.53780054,91.58462559)(517.8728002,91.9796252)(518.29279053,92.24963186)
\curveto(518.3727997,92.29962488)(518.46279961,92.34462483)(518.56279053,92.38463186)
\curveto(518.65279942,92.42462475)(518.74779933,92.46962471)(518.84779053,92.51963186)
\curveto(518.94779913,92.55962462)(519.04779903,92.58962459)(519.14779053,92.60963186)
\curveto(519.24779883,92.62962455)(519.35279872,92.64962453)(519.46279053,92.66963186)
\curveto(519.51279856,92.68962449)(519.55779852,92.69462448)(519.59779053,92.68463186)
\curveto(519.63779844,92.6746245)(519.68279839,92.6796245)(519.73279053,92.69963186)
\curveto(519.78279829,92.70962447)(519.86779821,92.71462446)(519.98779053,92.71463186)
\curveto(520.09779798,92.71462446)(520.18279789,92.70962447)(520.24279053,92.69963186)
\curveto(520.30279777,92.6796245)(520.36279771,92.66962451)(520.42279053,92.66963186)
\curveto(520.48279759,92.6796245)(520.54279753,92.6746245)(520.60279053,92.65463186)
\curveto(520.74279733,92.61462456)(520.8777972,92.5796246)(521.00779053,92.54963186)
\curveto(521.13779694,92.51962466)(521.26279681,92.4796247)(521.38279053,92.42963186)
\curveto(521.52279655,92.36962481)(521.64779643,92.29962488)(521.75779053,92.21963186)
\curveto(521.86779621,92.14962503)(521.9777961,92.0746251)(522.08779053,91.99463186)
\lineto(522.14779053,91.93463186)
\curveto(522.16779591,91.92462525)(522.18779589,91.90962527)(522.20779053,91.88963186)
\curveto(522.36779571,91.76962541)(522.51279556,91.63462554)(522.64279053,91.48463186)
\curveto(522.7727953,91.33462584)(522.89779518,91.174626)(523.01779053,91.00463186)
\curveto(523.23779484,90.69462648)(523.44279463,90.39962678)(523.63279053,90.11963186)
\curveto(523.7727943,89.88962729)(523.90779417,89.65962752)(524.03779053,89.42963186)
\curveto(524.16779391,89.20962797)(524.30279377,88.98962819)(524.44279053,88.76963186)
\curveto(524.61279346,88.51962866)(524.79279328,88.2796289)(524.98279053,88.04963186)
\curveto(525.1727929,87.82962935)(525.39779268,87.63962954)(525.65779053,87.47963186)
\curveto(525.71779236,87.43962974)(525.7777923,87.40462977)(525.83779053,87.37463186)
\curveto(525.88779219,87.34462983)(525.95279212,87.31462986)(526.03279053,87.28463186)
\curveto(526.10279197,87.26462991)(526.16279191,87.25962992)(526.21279053,87.26963186)
\curveto(526.28279179,87.28962989)(526.33779174,87.32462985)(526.37779053,87.37463186)
\curveto(526.40779167,87.42462975)(526.42779165,87.48462969)(526.43779053,87.55463186)
\lineto(526.43779053,87.79463186)
\lineto(526.43779053,88.54463186)
\lineto(526.43779053,91.34963186)
\lineto(526.43779053,92.00963186)
\curveto(526.43779164,92.09962508)(526.44279163,92.18462499)(526.45279053,92.26463186)
\curveto(526.45279162,92.34462483)(526.4727916,92.40962477)(526.51279053,92.45963186)
\curveto(526.55279152,92.50962467)(526.62779145,92.54962463)(526.73779053,92.57963186)
\curveto(526.83779124,92.61962456)(526.93779114,92.62962455)(527.03779053,92.60963186)
\lineto(527.17279053,92.60963186)
\curveto(527.24279083,92.58962459)(527.30279077,92.56962461)(527.35279053,92.54963186)
\curveto(527.40279067,92.52962465)(527.44279063,92.49462468)(527.47279053,92.44463186)
\curveto(527.51279056,92.39462478)(527.53279054,92.32462485)(527.53279053,92.23463186)
\lineto(527.53279053,91.96463186)
\lineto(527.53279053,91.06463186)
\lineto(527.53279053,87.55463186)
\lineto(527.53279053,86.48963186)
\curveto(527.53279054,86.40963077)(527.53779054,86.31963086)(527.54779053,86.21963186)
\curveto(527.54779053,86.11963106)(527.53779054,86.03463114)(527.51779053,85.96463186)
\curveto(527.44779063,85.75463142)(527.26779081,85.68963149)(526.97779053,85.76963186)
\curveto(526.93779114,85.7796314)(526.90279117,85.7796314)(526.87279053,85.76963186)
\curveto(526.83279124,85.76963141)(526.78779129,85.7796314)(526.73779053,85.79963186)
\curveto(526.65779142,85.81963136)(526.5727915,85.83963134)(526.48279053,85.85963186)
\curveto(526.39279168,85.8796313)(526.30779177,85.90463127)(526.22779053,85.93463186)
\curveto(525.73779234,86.09463108)(525.32279275,86.29463088)(524.98279053,86.53463186)
\curveto(524.73279334,86.71463046)(524.50779357,86.91963026)(524.30779053,87.14963186)
\curveto(524.09779398,87.3796298)(523.90279417,87.61962956)(523.72279053,87.86963186)
\curveto(523.54279453,88.12962905)(523.3727947,88.39462878)(523.21279053,88.66463186)
\curveto(523.04279503,88.94462823)(522.86779521,89.21462796)(522.68779053,89.47463186)
\curveto(522.60779547,89.58462759)(522.53279554,89.68962749)(522.46279053,89.78963186)
\curveto(522.39279568,89.89962728)(522.31779576,90.00962717)(522.23779053,90.11963186)
\curveto(522.20779587,90.15962702)(522.1777959,90.19462698)(522.14779053,90.22463186)
\curveto(522.10779597,90.26462691)(522.077796,90.30462687)(522.05779053,90.34463186)
\curveto(521.94779613,90.48462669)(521.82279625,90.60962657)(521.68279053,90.71963186)
\curveto(521.65279642,90.73962644)(521.62779645,90.76462641)(521.60779053,90.79463186)
\curveto(521.5777965,90.82462635)(521.54779653,90.84962633)(521.51779053,90.86963186)
\curveto(521.41779666,90.94962623)(521.31779676,91.01462616)(521.21779053,91.06463186)
\curveto(521.11779696,91.12462605)(521.00779707,91.179626)(520.88779053,91.22963186)
\curveto(520.81779726,91.25962592)(520.74279733,91.2796259)(520.66279053,91.28963186)
\lineto(520.42279053,91.34963186)
\lineto(520.33279053,91.34963186)
\curveto(520.30279777,91.35962582)(520.2727978,91.36462581)(520.24279053,91.36463186)
\curveto(520.1727979,91.38462579)(520.077798,91.38962579)(519.95779053,91.37963186)
\curveto(519.82779825,91.3796258)(519.72779835,91.36962581)(519.65779053,91.34963186)
\curveto(519.5777985,91.32962585)(519.50279857,91.30962587)(519.43279053,91.28963186)
\curveto(519.35279872,91.2796259)(519.2727988,91.25962592)(519.19279053,91.22963186)
\curveto(518.95279912,91.11962606)(518.75279932,90.96962621)(518.59279053,90.77963186)
\curveto(518.42279965,90.59962658)(518.28279979,90.3796268)(518.17279053,90.11963186)
\curveto(518.15279992,90.04962713)(518.13779994,89.9796272)(518.12779053,89.90963186)
\curveto(518.10779997,89.83962734)(518.08779999,89.76462741)(518.06779053,89.68463186)
\curveto(518.04780003,89.60462757)(518.03780004,89.49462768)(518.03779053,89.35463186)
\curveto(518.03780004,89.22462795)(518.04780003,89.11962806)(518.06779053,89.03963186)
\curveto(518.0778,88.9796282)(518.08279999,88.92462825)(518.08279053,88.87463186)
\curveto(518.08279999,88.82462835)(518.09279998,88.7746284)(518.11279053,88.72463186)
\curveto(518.15279992,88.62462855)(518.19279988,88.52962865)(518.23279053,88.43963186)
\curveto(518.2727998,88.35962882)(518.31779976,88.2796289)(518.36779053,88.19963186)
\curveto(518.38779969,88.16962901)(518.41279966,88.13962904)(518.44279053,88.10963186)
\curveto(518.4727996,88.08962909)(518.49779958,88.06462911)(518.51779053,88.03463186)
\lineto(518.59279053,87.95963186)
\curveto(518.61279946,87.92962925)(518.63279944,87.90462927)(518.65279053,87.88463186)
\lineto(518.86279053,87.73463186)
\curveto(518.92279915,87.69462948)(518.98779909,87.64962953)(519.05779053,87.59963186)
\curveto(519.14779893,87.53962964)(519.25279882,87.48962969)(519.37279053,87.44963186)
\curveto(519.48279859,87.41962976)(519.59279848,87.38462979)(519.70279053,87.34463186)
\curveto(519.81279826,87.30462987)(519.95779812,87.2796299)(520.13779053,87.26963186)
\curveto(520.30779777,87.25962992)(520.43279764,87.22962995)(520.51279053,87.17963186)
\curveto(520.59279748,87.12963005)(520.63779744,87.05463012)(520.64779053,86.95463186)
\curveto(520.65779742,86.85463032)(520.66279741,86.74463043)(520.66279053,86.62463186)
\curveto(520.66279741,86.58463059)(520.66779741,86.54463063)(520.67779053,86.50463186)
\curveto(520.6777974,86.46463071)(520.6727974,86.42963075)(520.66279053,86.39963186)
\curveto(520.64279743,86.34963083)(520.63279744,86.29963088)(520.63279053,86.24963186)
\curveto(520.63279744,86.20963097)(520.62279745,86.16963101)(520.60279053,86.12963186)
\curveto(520.54279753,86.03963114)(520.40779767,85.99463118)(520.19779053,85.99463186)
\lineto(520.07779053,85.99463186)
\curveto(520.01779806,86.00463117)(519.95779812,86.00963117)(519.89779053,86.00963186)
\curveto(519.82779825,86.01963116)(519.76279831,86.02963115)(519.70279053,86.03963186)
\curveto(519.59279848,86.05963112)(519.49279858,86.0796311)(519.40279053,86.09963186)
\curveto(519.30279877,86.11963106)(519.20779887,86.14963103)(519.11779053,86.18963186)
\curveto(519.04779903,86.20963097)(518.98779909,86.22963095)(518.93779053,86.24963186)
\lineto(518.75779053,86.30963186)
\curveto(518.49779958,86.42963075)(518.25279982,86.58463059)(518.02279053,86.77463186)
\curveto(517.79280028,86.9746302)(517.60780047,87.18962999)(517.46779053,87.41963186)
\curveto(517.38780069,87.52962965)(517.32280075,87.64462953)(517.27279053,87.76463186)
\lineto(517.12279053,88.15463186)
\curveto(517.072801,88.26462891)(517.04280103,88.3796288)(517.03279053,88.49963186)
\curveto(517.01280106,88.61962856)(516.98780109,88.74462843)(516.95779053,88.87463186)
\curveto(516.95780112,88.94462823)(516.95780112,89.00962817)(516.95779053,89.06963186)
\curveto(516.94780113,89.12962805)(516.93780114,89.19462798)(516.92779053,89.26463186)
}
}
{
\newrgbcolor{curcolor}{0 0 0}
\pscustom[linestyle=none,fillstyle=solid,fillcolor=curcolor]
{
\newpath
\moveto(522.44779053,101.36424123)
\lineto(522.70279053,101.36424123)
\curveto(522.78279529,101.37423353)(522.85779522,101.36923353)(522.92779053,101.34924123)
\lineto(523.16779053,101.34924123)
\lineto(523.33279053,101.34924123)
\curveto(523.43279464,101.32923357)(523.53779454,101.31923358)(523.64779053,101.31924123)
\curveto(523.74779433,101.31923358)(523.84779423,101.30923359)(523.94779053,101.28924123)
\lineto(524.09779053,101.28924123)
\curveto(524.23779384,101.25923364)(524.3777937,101.23923366)(524.51779053,101.22924123)
\curveto(524.64779343,101.21923368)(524.7777933,101.19423371)(524.90779053,101.15424123)
\curveto(524.98779309,101.13423377)(525.072793,101.11423379)(525.16279053,101.09424123)
\lineto(525.40279053,101.03424123)
\lineto(525.70279053,100.91424123)
\curveto(525.79279228,100.88423402)(525.88279219,100.84923405)(525.97279053,100.80924123)
\curveto(526.19279188,100.70923419)(526.40779167,100.57423433)(526.61779053,100.40424123)
\curveto(526.82779125,100.24423466)(526.99779108,100.06923483)(527.12779053,99.87924123)
\curveto(527.16779091,99.82923507)(527.20779087,99.76923513)(527.24779053,99.69924123)
\curveto(527.2777908,99.63923526)(527.31279076,99.57923532)(527.35279053,99.51924123)
\curveto(527.40279067,99.43923546)(527.44279063,99.34423556)(527.47279053,99.23424123)
\curveto(527.50279057,99.12423578)(527.53279054,99.01923588)(527.56279053,98.91924123)
\curveto(527.60279047,98.80923609)(527.62779045,98.6992362)(527.63779053,98.58924123)
\curveto(527.64779043,98.47923642)(527.66279041,98.36423654)(527.68279053,98.24424123)
\curveto(527.69279038,98.2042367)(527.69279038,98.15923674)(527.68279053,98.10924123)
\curveto(527.68279039,98.06923683)(527.68779039,98.02923687)(527.69779053,97.98924123)
\curveto(527.70779037,97.94923695)(527.71279036,97.89423701)(527.71279053,97.82424123)
\curveto(527.71279036,97.75423715)(527.70779037,97.7042372)(527.69779053,97.67424123)
\curveto(527.6777904,97.62423728)(527.6727904,97.57923732)(527.68279053,97.53924123)
\curveto(527.69279038,97.4992374)(527.69279038,97.46423744)(527.68279053,97.43424123)
\lineto(527.68279053,97.34424123)
\curveto(527.66279041,97.28423762)(527.64779043,97.21923768)(527.63779053,97.14924123)
\curveto(527.63779044,97.08923781)(527.63279044,97.02423788)(527.62279053,96.95424123)
\curveto(527.5727905,96.78423812)(527.52279055,96.62423828)(527.47279053,96.47424123)
\curveto(527.42279065,96.32423858)(527.35779072,96.17923872)(527.27779053,96.03924123)
\curveto(527.23779084,95.98923891)(527.20779087,95.93423897)(527.18779053,95.87424123)
\curveto(527.15779092,95.82423908)(527.12279095,95.77423913)(527.08279053,95.72424123)
\curveto(526.90279117,95.48423942)(526.68279139,95.28423962)(526.42279053,95.12424123)
\curveto(526.16279191,94.96423994)(525.8777922,94.82424008)(525.56779053,94.70424123)
\curveto(525.42779265,94.64424026)(525.28779279,94.5992403)(525.14779053,94.56924123)
\curveto(524.99779308,94.53924036)(524.84279323,94.5042404)(524.68279053,94.46424123)
\curveto(524.5727935,94.44424046)(524.46279361,94.42924047)(524.35279053,94.41924123)
\curveto(524.24279383,94.40924049)(524.13279394,94.39424051)(524.02279053,94.37424123)
\curveto(523.98279409,94.36424054)(523.94279413,94.35924054)(523.90279053,94.35924123)
\curveto(523.86279421,94.36924053)(523.82279425,94.36924053)(523.78279053,94.35924123)
\curveto(523.73279434,94.34924055)(523.68279439,94.34424056)(523.63279053,94.34424123)
\lineto(523.46779053,94.34424123)
\curveto(523.41779466,94.32424058)(523.36779471,94.31924058)(523.31779053,94.32924123)
\curveto(523.25779482,94.33924056)(523.20279487,94.33924056)(523.15279053,94.32924123)
\curveto(523.11279496,94.31924058)(523.06779501,94.31924058)(523.01779053,94.32924123)
\curveto(522.96779511,94.33924056)(522.91779516,94.33424057)(522.86779053,94.31424123)
\curveto(522.79779528,94.29424061)(522.72279535,94.28924061)(522.64279053,94.29924123)
\curveto(522.55279552,94.30924059)(522.46779561,94.31424059)(522.38779053,94.31424123)
\curveto(522.29779578,94.31424059)(522.19779588,94.30924059)(522.08779053,94.29924123)
\curveto(521.96779611,94.28924061)(521.86779621,94.29424061)(521.78779053,94.31424123)
\lineto(521.50279053,94.31424123)
\lineto(520.87279053,94.35924123)
\curveto(520.7727973,94.36924053)(520.6777974,94.37924052)(520.58779053,94.38924123)
\lineto(520.28779053,94.41924123)
\curveto(520.23779784,94.43924046)(520.18779789,94.44424046)(520.13779053,94.43424123)
\curveto(520.077798,94.43424047)(520.02279805,94.44424046)(519.97279053,94.46424123)
\curveto(519.80279827,94.51424039)(519.63779844,94.55424035)(519.47779053,94.58424123)
\curveto(519.30779877,94.61424029)(519.14779893,94.66424024)(518.99779053,94.73424123)
\curveto(518.53779954,94.92423998)(518.16279991,95.14423976)(517.87279053,95.39424123)
\curveto(517.58280049,95.65423925)(517.33780074,96.01423889)(517.13779053,96.47424123)
\curveto(517.08780099,96.6042383)(517.05280102,96.73423817)(517.03279053,96.86424123)
\curveto(517.01280106,97.0042379)(516.98780109,97.14423776)(516.95779053,97.28424123)
\curveto(516.94780113,97.35423755)(516.94280113,97.41923748)(516.94279053,97.47924123)
\curveto(516.94280113,97.53923736)(516.93780114,97.6042373)(516.92779053,97.67424123)
\curveto(516.90780117,98.5042364)(517.05780102,99.17423573)(517.37779053,99.68424123)
\curveto(517.68780039,100.19423471)(518.12779995,100.57423433)(518.69779053,100.82424123)
\curveto(518.81779926,100.87423403)(518.94279913,100.91923398)(519.07279053,100.95924123)
\curveto(519.20279887,100.9992339)(519.33779874,101.04423386)(519.47779053,101.09424123)
\curveto(519.55779852,101.11423379)(519.64279843,101.12923377)(519.73279053,101.13924123)
\lineto(519.97279053,101.19924123)
\curveto(520.08279799,101.22923367)(520.19279788,101.24423366)(520.30279053,101.24424123)
\curveto(520.41279766,101.25423365)(520.52279755,101.26923363)(520.63279053,101.28924123)
\curveto(520.68279739,101.30923359)(520.72779735,101.31423359)(520.76779053,101.30424123)
\curveto(520.80779727,101.3042336)(520.84779723,101.30923359)(520.88779053,101.31924123)
\curveto(520.93779714,101.32923357)(520.99279708,101.32923357)(521.05279053,101.31924123)
\curveto(521.10279697,101.31923358)(521.15279692,101.32423358)(521.20279053,101.33424123)
\lineto(521.33779053,101.33424123)
\curveto(521.39779668,101.35423355)(521.46779661,101.35423355)(521.54779053,101.33424123)
\curveto(521.61779646,101.32423358)(521.68279639,101.32923357)(521.74279053,101.34924123)
\curveto(521.7727963,101.35923354)(521.81279626,101.36423354)(521.86279053,101.36424123)
\lineto(521.98279053,101.36424123)
\lineto(522.44779053,101.36424123)
\moveto(524.77279053,99.81924123)
\curveto(524.45279362,99.91923498)(524.08779399,99.97923492)(523.67779053,99.99924123)
\curveto(523.26779481,100.01923488)(522.85779522,100.02923487)(522.44779053,100.02924123)
\curveto(522.01779606,100.02923487)(521.59779648,100.01923488)(521.18779053,99.99924123)
\curveto(520.7777973,99.97923492)(520.39279768,99.93423497)(520.03279053,99.86424123)
\curveto(519.6727984,99.79423511)(519.35279872,99.68423522)(519.07279053,99.53424123)
\curveto(518.78279929,99.39423551)(518.54779953,99.1992357)(518.36779053,98.94924123)
\curveto(518.25779982,98.78923611)(518.1777999,98.60923629)(518.12779053,98.40924123)
\curveto(518.06780001,98.20923669)(518.03780004,97.96423694)(518.03779053,97.67424123)
\curveto(518.05780002,97.65423725)(518.06780001,97.61923728)(518.06779053,97.56924123)
\curveto(518.05780002,97.51923738)(518.05780002,97.47923742)(518.06779053,97.44924123)
\curveto(518.08779999,97.36923753)(518.10779997,97.29423761)(518.12779053,97.22424123)
\curveto(518.13779994,97.16423774)(518.15779992,97.0992378)(518.18779053,97.02924123)
\curveto(518.30779977,96.75923814)(518.4777996,96.53923836)(518.69779053,96.36924123)
\curveto(518.90779917,96.20923869)(519.15279892,96.07423883)(519.43279053,95.96424123)
\curveto(519.54279853,95.91423899)(519.66279841,95.87423903)(519.79279053,95.84424123)
\curveto(519.91279816,95.82423908)(520.03779804,95.7992391)(520.16779053,95.76924123)
\curveto(520.21779786,95.74923915)(520.2727978,95.73923916)(520.33279053,95.73924123)
\curveto(520.38279769,95.73923916)(520.43279764,95.73423917)(520.48279053,95.72424123)
\curveto(520.5727975,95.71423919)(520.66779741,95.7042392)(520.76779053,95.69424123)
\curveto(520.85779722,95.68423922)(520.95279712,95.67423923)(521.05279053,95.66424123)
\curveto(521.13279694,95.66423924)(521.21779686,95.65923924)(521.30779053,95.64924123)
\lineto(521.54779053,95.64924123)
\lineto(521.72779053,95.64924123)
\curveto(521.75779632,95.63923926)(521.79279628,95.63423927)(521.83279053,95.63424123)
\lineto(521.96779053,95.63424123)
\lineto(522.41779053,95.63424123)
\curveto(522.49779558,95.63423927)(522.58279549,95.62923927)(522.67279053,95.61924123)
\curveto(522.75279532,95.61923928)(522.82779525,95.62923927)(522.89779053,95.64924123)
\lineto(523.16779053,95.64924123)
\curveto(523.18779489,95.64923925)(523.21779486,95.64423926)(523.25779053,95.63424123)
\curveto(523.28779479,95.63423927)(523.31279476,95.63923926)(523.33279053,95.64924123)
\curveto(523.43279464,95.65923924)(523.53279454,95.66423924)(523.63279053,95.66424123)
\curveto(523.72279435,95.67423923)(523.82279425,95.68423922)(523.93279053,95.69424123)
\curveto(524.05279402,95.72423918)(524.1777939,95.73923916)(524.30779053,95.73924123)
\curveto(524.42779365,95.74923915)(524.54279353,95.77423913)(524.65279053,95.81424123)
\curveto(524.95279312,95.89423901)(525.21779286,95.97923892)(525.44779053,96.06924123)
\curveto(525.6777924,96.16923873)(525.89279218,96.31423859)(526.09279053,96.50424123)
\curveto(526.29279178,96.71423819)(526.44279163,96.97923792)(526.54279053,97.29924123)
\curveto(526.56279151,97.33923756)(526.5727915,97.37423753)(526.57279053,97.40424123)
\curveto(526.56279151,97.44423746)(526.56779151,97.48923741)(526.58779053,97.53924123)
\curveto(526.59779148,97.57923732)(526.60779147,97.64923725)(526.61779053,97.74924123)
\curveto(526.62779145,97.85923704)(526.62279145,97.94423696)(526.60279053,98.00424123)
\curveto(526.58279149,98.07423683)(526.5727915,98.14423676)(526.57279053,98.21424123)
\curveto(526.56279151,98.28423662)(526.54779153,98.34923655)(526.52779053,98.40924123)
\curveto(526.46779161,98.60923629)(526.38279169,98.78923611)(526.27279053,98.94924123)
\curveto(526.25279182,98.97923592)(526.23279184,99.0042359)(526.21279053,99.02424123)
\lineto(526.15279053,99.08424123)
\curveto(526.13279194,99.12423578)(526.09279198,99.17423573)(526.03279053,99.23424123)
\curveto(525.89279218,99.33423557)(525.76279231,99.41923548)(525.64279053,99.48924123)
\curveto(525.52279255,99.55923534)(525.3777927,99.62923527)(525.20779053,99.69924123)
\curveto(525.13779294,99.72923517)(525.06779301,99.74923515)(524.99779053,99.75924123)
\curveto(524.92779315,99.77923512)(524.85279322,99.7992351)(524.77279053,99.81924123)
}
}
{
\newrgbcolor{curcolor}{0 0 0}
\pscustom[linestyle=none,fillstyle=solid,fillcolor=curcolor]
{
\newpath
\moveto(516.92779053,106.77385061)
\curveto(516.92780115,106.87384575)(516.93780114,106.96884566)(516.95779053,107.05885061)
\curveto(516.96780111,107.14884548)(516.99780108,107.21384541)(517.04779053,107.25385061)
\curveto(517.12780095,107.31384531)(517.23280084,107.34384528)(517.36279053,107.34385061)
\lineto(517.75279053,107.34385061)
\lineto(519.25279053,107.34385061)
\lineto(525.64279053,107.34385061)
\lineto(526.81279053,107.34385061)
\lineto(527.12779053,107.34385061)
\curveto(527.22779085,107.35384527)(527.30779077,107.33884529)(527.36779053,107.29885061)
\curveto(527.44779063,107.24884538)(527.49779058,107.17384545)(527.51779053,107.07385061)
\curveto(527.52779055,106.98384564)(527.53279054,106.87384575)(527.53279053,106.74385061)
\lineto(527.53279053,106.51885061)
\curveto(527.51279056,106.43884619)(527.49779058,106.36884626)(527.48779053,106.30885061)
\curveto(527.46779061,106.24884638)(527.42779065,106.19884643)(527.36779053,106.15885061)
\curveto(527.30779077,106.11884651)(527.23279084,106.09884653)(527.14279053,106.09885061)
\lineto(526.84279053,106.09885061)
\lineto(525.74779053,106.09885061)
\lineto(520.40779053,106.09885061)
\curveto(520.31779776,106.07884655)(520.24279783,106.06384656)(520.18279053,106.05385061)
\curveto(520.11279796,106.05384657)(520.05279802,106.0238466)(520.00279053,105.96385061)
\curveto(519.95279812,105.89384673)(519.92779815,105.80384682)(519.92779053,105.69385061)
\curveto(519.91779816,105.59384703)(519.91279816,105.48384714)(519.91279053,105.36385061)
\lineto(519.91279053,104.22385061)
\lineto(519.91279053,103.72885061)
\curveto(519.90279817,103.56884906)(519.84279823,103.45884917)(519.73279053,103.39885061)
\curveto(519.70279837,103.37884925)(519.6727984,103.36884926)(519.64279053,103.36885061)
\curveto(519.60279847,103.36884926)(519.55779852,103.36384926)(519.50779053,103.35385061)
\curveto(519.38779869,103.33384929)(519.2777988,103.33884929)(519.17779053,103.36885061)
\curveto(519.077799,103.40884922)(519.00779907,103.46384916)(518.96779053,103.53385061)
\curveto(518.91779916,103.61384901)(518.89279918,103.73384889)(518.89279053,103.89385061)
\curveto(518.89279918,104.05384857)(518.8777992,104.18884844)(518.84779053,104.29885061)
\curveto(518.83779924,104.34884828)(518.83279924,104.40384822)(518.83279053,104.46385061)
\curveto(518.82279925,104.5238481)(518.80779927,104.58384804)(518.78779053,104.64385061)
\curveto(518.73779934,104.79384783)(518.68779939,104.93884769)(518.63779053,105.07885061)
\curveto(518.5777995,105.21884741)(518.50779957,105.35384727)(518.42779053,105.48385061)
\curveto(518.33779974,105.623847)(518.23279984,105.74384688)(518.11279053,105.84385061)
\curveto(517.99280008,105.94384668)(517.86280021,106.03884659)(517.72279053,106.12885061)
\curveto(517.62280045,106.18884644)(517.51280056,106.23384639)(517.39279053,106.26385061)
\curveto(517.2728008,106.30384632)(517.16780091,106.35384627)(517.07779053,106.41385061)
\curveto(517.01780106,106.46384616)(516.9778011,106.53384609)(516.95779053,106.62385061)
\curveto(516.94780113,106.64384598)(516.94280113,106.66884596)(516.94279053,106.69885061)
\curveto(516.94280113,106.7288459)(516.93780114,106.75384587)(516.92779053,106.77385061)
}
}
{
\newrgbcolor{curcolor}{0 0 0}
\pscustom[linestyle=none,fillstyle=solid,fillcolor=curcolor]
{
\newpath
\moveto(516.92779053,115.12345998)
\curveto(516.92780115,115.22345513)(516.93780114,115.31845503)(516.95779053,115.40845998)
\curveto(516.96780111,115.49845485)(516.99780108,115.56345479)(517.04779053,115.60345998)
\curveto(517.12780095,115.66345469)(517.23280084,115.69345466)(517.36279053,115.69345998)
\lineto(517.75279053,115.69345998)
\lineto(519.25279053,115.69345998)
\lineto(525.64279053,115.69345998)
\lineto(526.81279053,115.69345998)
\lineto(527.12779053,115.69345998)
\curveto(527.22779085,115.70345465)(527.30779077,115.68845466)(527.36779053,115.64845998)
\curveto(527.44779063,115.59845475)(527.49779058,115.52345483)(527.51779053,115.42345998)
\curveto(527.52779055,115.33345502)(527.53279054,115.22345513)(527.53279053,115.09345998)
\lineto(527.53279053,114.86845998)
\curveto(527.51279056,114.78845556)(527.49779058,114.71845563)(527.48779053,114.65845998)
\curveto(527.46779061,114.59845575)(527.42779065,114.5484558)(527.36779053,114.50845998)
\curveto(527.30779077,114.46845588)(527.23279084,114.4484559)(527.14279053,114.44845998)
\lineto(526.84279053,114.44845998)
\lineto(525.74779053,114.44845998)
\lineto(520.40779053,114.44845998)
\curveto(520.31779776,114.42845592)(520.24279783,114.41345594)(520.18279053,114.40345998)
\curveto(520.11279796,114.40345595)(520.05279802,114.37345598)(520.00279053,114.31345998)
\curveto(519.95279812,114.24345611)(519.92779815,114.1534562)(519.92779053,114.04345998)
\curveto(519.91779816,113.94345641)(519.91279816,113.83345652)(519.91279053,113.71345998)
\lineto(519.91279053,112.57345998)
\lineto(519.91279053,112.07845998)
\curveto(519.90279817,111.91845843)(519.84279823,111.80845854)(519.73279053,111.74845998)
\curveto(519.70279837,111.72845862)(519.6727984,111.71845863)(519.64279053,111.71845998)
\curveto(519.60279847,111.71845863)(519.55779852,111.71345864)(519.50779053,111.70345998)
\curveto(519.38779869,111.68345867)(519.2777988,111.68845866)(519.17779053,111.71845998)
\curveto(519.077799,111.75845859)(519.00779907,111.81345854)(518.96779053,111.88345998)
\curveto(518.91779916,111.96345839)(518.89279918,112.08345827)(518.89279053,112.24345998)
\curveto(518.89279918,112.40345795)(518.8777992,112.53845781)(518.84779053,112.64845998)
\curveto(518.83779924,112.69845765)(518.83279924,112.7534576)(518.83279053,112.81345998)
\curveto(518.82279925,112.87345748)(518.80779927,112.93345742)(518.78779053,112.99345998)
\curveto(518.73779934,113.14345721)(518.68779939,113.28845706)(518.63779053,113.42845998)
\curveto(518.5777995,113.56845678)(518.50779957,113.70345665)(518.42779053,113.83345998)
\curveto(518.33779974,113.97345638)(518.23279984,114.09345626)(518.11279053,114.19345998)
\curveto(517.99280008,114.29345606)(517.86280021,114.38845596)(517.72279053,114.47845998)
\curveto(517.62280045,114.53845581)(517.51280056,114.58345577)(517.39279053,114.61345998)
\curveto(517.2728008,114.6534557)(517.16780091,114.70345565)(517.07779053,114.76345998)
\curveto(517.01780106,114.81345554)(516.9778011,114.88345547)(516.95779053,114.97345998)
\curveto(516.94780113,114.99345536)(516.94280113,115.01845533)(516.94279053,115.04845998)
\curveto(516.94280113,115.07845527)(516.93780114,115.10345525)(516.92779053,115.12345998)
}
}
{
\newrgbcolor{curcolor}{0 0 0}
\pscustom[linestyle=none,fillstyle=solid,fillcolor=curcolor]
{
\newpath
\moveto(538.79907959,29.18119436)
\lineto(538.79907959,30.09619436)
\curveto(538.79909028,30.19619171)(538.79909028,30.29119161)(538.79907959,30.38119436)
\curveto(538.79909028,30.47119143)(538.81909026,30.54619136)(538.85907959,30.60619436)
\curveto(538.91909016,30.69619121)(538.99909008,30.75619115)(539.09907959,30.78619436)
\curveto(539.19908988,30.82619108)(539.30408978,30.87119103)(539.41407959,30.92119436)
\curveto(539.60408948,31.0011909)(539.79408929,31.07119083)(539.98407959,31.13119436)
\curveto(540.17408891,31.2011907)(540.36408872,31.27619063)(540.55407959,31.35619436)
\curveto(540.73408835,31.42619048)(540.91908816,31.49119041)(541.10907959,31.55119436)
\curveto(541.28908779,31.61119029)(541.46908761,31.68119022)(541.64907959,31.76119436)
\curveto(541.78908729,31.82119008)(541.93408715,31.87619003)(542.08407959,31.92619436)
\curveto(542.23408685,31.97618993)(542.3790867,32.03118987)(542.51907959,32.09119436)
\curveto(542.96908611,32.27118963)(543.42408566,32.44118946)(543.88407959,32.60119436)
\curveto(544.33408475,32.76118914)(544.7840843,32.93118897)(545.23407959,33.11119436)
\curveto(545.2840838,33.13118877)(545.33408375,33.14618876)(545.38407959,33.15619436)
\lineto(545.53407959,33.21619436)
\curveto(545.75408333,33.3061886)(545.9790831,33.39118851)(546.20907959,33.47119436)
\curveto(546.42908265,33.55118835)(546.64908243,33.63618827)(546.86907959,33.72619436)
\curveto(546.95908212,33.76618814)(547.06908201,33.8061881)(547.19907959,33.84619436)
\curveto(547.31908176,33.88618802)(547.38908169,33.95118795)(547.40907959,34.04119436)
\curveto(547.41908166,34.08118782)(547.41908166,34.11118779)(547.40907959,34.13119436)
\lineto(547.34907959,34.19119436)
\curveto(547.29908178,34.24118766)(547.24408184,34.27618763)(547.18407959,34.29619436)
\curveto(547.12408196,34.32618758)(547.05908202,34.35618755)(546.98907959,34.38619436)
\lineto(546.35907959,34.62619436)
\curveto(546.13908294,34.7061872)(545.92408316,34.78618712)(545.71407959,34.86619436)
\lineto(545.56407959,34.92619436)
\lineto(545.38407959,34.98619436)
\curveto(545.19408389,35.06618684)(545.00408408,35.13618677)(544.81407959,35.19619436)
\curveto(544.61408447,35.26618664)(544.41408467,35.34118656)(544.21407959,35.42119436)
\curveto(543.63408545,35.66118624)(543.04908603,35.88118602)(542.45907959,36.08119436)
\curveto(541.86908721,36.29118561)(541.2840878,36.51618539)(540.70407959,36.75619436)
\curveto(540.50408858,36.83618507)(540.29908878,36.91118499)(540.08907959,36.98119436)
\curveto(539.8790892,37.06118484)(539.67408941,37.14118476)(539.47407959,37.22119436)
\curveto(539.39408969,37.26118464)(539.29408979,37.29618461)(539.17407959,37.32619436)
\curveto(539.05409003,37.36618454)(538.96909011,37.42118448)(538.91907959,37.49119436)
\curveto(538.8790902,37.55118435)(538.84909023,37.62618428)(538.82907959,37.71619436)
\curveto(538.80909027,37.81618409)(538.79909028,37.92618398)(538.79907959,38.04619436)
\curveto(538.78909029,38.16618374)(538.78909029,38.28618362)(538.79907959,38.40619436)
\curveto(538.79909028,38.52618338)(538.79909028,38.63618327)(538.79907959,38.73619436)
\curveto(538.79909028,38.82618308)(538.79909028,38.91618299)(538.79907959,39.00619436)
\curveto(538.79909028,39.1061828)(538.81909026,39.18118272)(538.85907959,39.23119436)
\curveto(538.90909017,39.32118258)(538.99909008,39.37118253)(539.12907959,39.38119436)
\curveto(539.25908982,39.39118251)(539.39908968,39.39618251)(539.54907959,39.39619436)
\lineto(541.19907959,39.39619436)
\lineto(547.46907959,39.39619436)
\lineto(548.72907959,39.39619436)
\curveto(548.83908024,39.39618251)(548.94908013,39.39618251)(549.05907959,39.39619436)
\curveto(549.16907991,39.4061825)(549.25407983,39.38618252)(549.31407959,39.33619436)
\curveto(549.37407971,39.3061826)(549.41407967,39.26118264)(549.43407959,39.20119436)
\curveto(549.44407964,39.14118276)(549.45907962,39.07118283)(549.47907959,38.99119436)
\lineto(549.47907959,38.75119436)
\lineto(549.47907959,38.39119436)
\curveto(549.46907961,38.28118362)(549.42407966,38.2011837)(549.34407959,38.15119436)
\curveto(549.31407977,38.13118377)(549.2840798,38.11618379)(549.25407959,38.10619436)
\curveto(549.21407987,38.1061838)(549.16907991,38.09618381)(549.11907959,38.07619436)
\lineto(548.95407959,38.07619436)
\curveto(548.89408019,38.06618384)(548.82408026,38.06118384)(548.74407959,38.06119436)
\curveto(548.66408042,38.07118383)(548.58908049,38.07618383)(548.51907959,38.07619436)
\lineto(547.67907959,38.07619436)
\lineto(543.25407959,38.07619436)
\curveto(543.00408608,38.07618383)(542.75408633,38.07618383)(542.50407959,38.07619436)
\curveto(542.24408684,38.07618383)(541.99408709,38.07118383)(541.75407959,38.06119436)
\curveto(541.65408743,38.06118384)(541.54408754,38.05618385)(541.42407959,38.04619436)
\curveto(541.30408778,38.03618387)(541.24408784,37.98118392)(541.24407959,37.88119436)
\lineto(541.25907959,37.88119436)
\curveto(541.2790878,37.81118409)(541.34408774,37.75118415)(541.45407959,37.70119436)
\curveto(541.56408752,37.66118424)(541.65908742,37.62618428)(541.73907959,37.59619436)
\curveto(541.90908717,37.52618438)(542.084087,37.46118444)(542.26407959,37.40119436)
\curveto(542.43408665,37.34118456)(542.60408648,37.27118463)(542.77407959,37.19119436)
\curveto(542.82408626,37.17118473)(542.86908621,37.15618475)(542.90907959,37.14619436)
\curveto(542.94908613,37.13618477)(542.99408609,37.12118478)(543.04407959,37.10119436)
\curveto(543.22408586,37.02118488)(543.40908567,36.95118495)(543.59907959,36.89119436)
\curveto(543.7790853,36.84118506)(543.95908512,36.77618513)(544.13907959,36.69619436)
\curveto(544.28908479,36.62618528)(544.44408464,36.56618534)(544.60407959,36.51619436)
\curveto(544.75408433,36.46618544)(544.90408418,36.41118549)(545.05407959,36.35119436)
\curveto(545.52408356,36.15118575)(545.99908308,35.97118593)(546.47907959,35.81119436)
\curveto(546.94908213,35.65118625)(547.41408167,35.47618643)(547.87407959,35.28619436)
\curveto(548.05408103,35.2061867)(548.23408085,35.13618677)(548.41407959,35.07619436)
\curveto(548.59408049,35.01618689)(548.77408031,34.95118695)(548.95407959,34.88119436)
\curveto(549.06408002,34.83118707)(549.16907991,34.78118712)(549.26907959,34.73119436)
\curveto(549.35907972,34.69118721)(549.42407966,34.6061873)(549.46407959,34.47619436)
\curveto(549.47407961,34.45618745)(549.4790796,34.43118747)(549.47907959,34.40119436)
\curveto(549.46907961,34.38118752)(549.46907961,34.35618755)(549.47907959,34.32619436)
\curveto(549.48907959,34.29618761)(549.49407959,34.26118764)(549.49407959,34.22119436)
\curveto(549.4840796,34.18118772)(549.4790796,34.14118776)(549.47907959,34.10119436)
\lineto(549.47907959,33.80119436)
\curveto(549.4790796,33.7011882)(549.45407963,33.62118828)(549.40407959,33.56119436)
\curveto(549.35407973,33.48118842)(549.2840798,33.42118848)(549.19407959,33.38119436)
\curveto(549.09407999,33.35118855)(548.99408009,33.31118859)(548.89407959,33.26119436)
\curveto(548.69408039,33.18118872)(548.48908059,33.1011888)(548.27907959,33.02119436)
\curveto(548.05908102,32.95118895)(547.84908123,32.87618903)(547.64907959,32.79619436)
\curveto(547.46908161,32.71618919)(547.28908179,32.64618926)(547.10907959,32.58619436)
\curveto(546.91908216,32.53618937)(546.73408235,32.47118943)(546.55407959,32.39119436)
\curveto(545.99408309,32.16118974)(545.42908365,31.94618996)(544.85907959,31.74619436)
\curveto(544.28908479,31.54619036)(543.72408536,31.33119057)(543.16407959,31.10119436)
\lineto(542.53407959,30.86119436)
\curveto(542.31408677,30.79119111)(542.10408698,30.71619119)(541.90407959,30.63619436)
\curveto(541.79408729,30.58619132)(541.68908739,30.54119136)(541.58907959,30.50119436)
\curveto(541.4790876,30.47119143)(541.3840877,30.42119148)(541.30407959,30.35119436)
\curveto(541.2840878,30.34119156)(541.27408781,30.33119157)(541.27407959,30.32119436)
\lineto(541.24407959,30.29119436)
\lineto(541.24407959,30.21619436)
\lineto(541.27407959,30.18619436)
\curveto(541.27408781,30.17619173)(541.2790878,30.16619174)(541.28907959,30.15619436)
\curveto(541.33908774,30.13619177)(541.39408769,30.12619178)(541.45407959,30.12619436)
\curveto(541.51408757,30.12619178)(541.57408751,30.11619179)(541.63407959,30.09619436)
\lineto(541.79907959,30.09619436)
\curveto(541.85908722,30.07619183)(541.92408716,30.07119183)(541.99407959,30.08119436)
\curveto(542.06408702,30.09119181)(542.13408695,30.09619181)(542.20407959,30.09619436)
\lineto(543.01407959,30.09619436)
\lineto(547.57407959,30.09619436)
\lineto(548.75907959,30.09619436)
\curveto(548.86908021,30.09619181)(548.9790801,30.09119181)(549.08907959,30.08119436)
\curveto(549.19907988,30.08119182)(549.2840798,30.05619185)(549.34407959,30.00619436)
\curveto(549.42407966,29.95619195)(549.46907961,29.86619204)(549.47907959,29.73619436)
\lineto(549.47907959,29.34619436)
\lineto(549.47907959,29.15119436)
\curveto(549.4790796,29.1011928)(549.46907961,29.05119285)(549.44907959,29.00119436)
\curveto(549.40907967,28.87119303)(549.32407976,28.79619311)(549.19407959,28.77619436)
\curveto(549.06408002,28.76619314)(548.91408017,28.76119314)(548.74407959,28.76119436)
\lineto(547.00407959,28.76119436)
\lineto(541.00407959,28.76119436)
\lineto(539.59407959,28.76119436)
\curveto(539.4840896,28.76119314)(539.36908971,28.75619315)(539.24907959,28.74619436)
\curveto(539.12908995,28.74619316)(539.03409005,28.77119313)(538.96407959,28.82119436)
\curveto(538.90409018,28.86119304)(538.85409023,28.93619297)(538.81407959,29.04619436)
\curveto(538.80409028,29.06619284)(538.80409028,29.08619282)(538.81407959,29.10619436)
\curveto(538.81409027,29.13619277)(538.80909027,29.16119274)(538.79907959,29.18119436)
}
}
{
\newrgbcolor{curcolor}{0 0 0}
\pscustom[linestyle=none,fillstyle=solid,fillcolor=curcolor]
{
\newpath
\moveto(548.92407959,48.38330373)
\curveto(549.08408,48.4132959)(549.21907986,48.39829592)(549.32907959,48.33830373)
\curveto(549.42907965,48.27829604)(549.50407958,48.19829612)(549.55407959,48.09830373)
\curveto(549.57407951,48.04829627)(549.5840795,47.99329632)(549.58407959,47.93330373)
\curveto(549.5840795,47.88329643)(549.59407949,47.82829649)(549.61407959,47.76830373)
\curveto(549.66407942,47.54829677)(549.64907943,47.32829699)(549.56907959,47.10830373)
\curveto(549.49907958,46.89829742)(549.40907967,46.75329756)(549.29907959,46.67330373)
\curveto(549.22907985,46.62329769)(549.14907993,46.57829774)(549.05907959,46.53830373)
\curveto(548.95908012,46.49829782)(548.8790802,46.44829787)(548.81907959,46.38830373)
\curveto(548.79908028,46.36829795)(548.7790803,46.34329797)(548.75907959,46.31330373)
\curveto(548.73908034,46.29329802)(548.73408035,46.26329805)(548.74407959,46.22330373)
\curveto(548.77408031,46.1132982)(548.82908025,46.00829831)(548.90907959,45.90830373)
\curveto(548.98908009,45.8182985)(549.05908002,45.72829859)(549.11907959,45.63830373)
\curveto(549.19907988,45.50829881)(549.27407981,45.36829895)(549.34407959,45.21830373)
\curveto(549.40407968,45.06829925)(549.45907962,44.90829941)(549.50907959,44.73830373)
\curveto(549.53907954,44.63829968)(549.55907952,44.52829979)(549.56907959,44.40830373)
\curveto(549.5790795,44.29830002)(549.59407949,44.18830013)(549.61407959,44.07830373)
\curveto(549.62407946,44.02830029)(549.62907945,43.98330033)(549.62907959,43.94330373)
\lineto(549.62907959,43.83830373)
\curveto(549.64907943,43.72830059)(549.64907943,43.62330069)(549.62907959,43.52330373)
\lineto(549.62907959,43.38830373)
\curveto(549.61907946,43.33830098)(549.61407947,43.28830103)(549.61407959,43.23830373)
\curveto(549.61407947,43.18830113)(549.60407948,43.14330117)(549.58407959,43.10330373)
\curveto(549.57407951,43.06330125)(549.56907951,43.02830129)(549.56907959,42.99830373)
\curveto(549.5790795,42.97830134)(549.5790795,42.95330136)(549.56907959,42.92330373)
\lineto(549.50907959,42.68330373)
\curveto(549.49907958,42.60330171)(549.4790796,42.52830179)(549.44907959,42.45830373)
\curveto(549.31907976,42.15830216)(549.17407991,41.9133024)(549.01407959,41.72330373)
\curveto(548.84408024,41.54330277)(548.60908047,41.39330292)(548.30907959,41.27330373)
\curveto(548.08908099,41.18330313)(547.82408126,41.13830318)(547.51407959,41.13830373)
\lineto(547.19907959,41.13830373)
\curveto(547.14908193,41.14830317)(547.09908198,41.15330316)(547.04907959,41.15330373)
\lineto(546.86907959,41.18330373)
\lineto(546.53907959,41.30330373)
\curveto(546.42908265,41.34330297)(546.32908275,41.39330292)(546.23907959,41.45330373)
\curveto(545.94908313,41.63330268)(545.73408335,41.87830244)(545.59407959,42.18830373)
\curveto(545.45408363,42.49830182)(545.32908375,42.83830148)(545.21907959,43.20830373)
\curveto(545.1790839,43.34830097)(545.14908393,43.49330082)(545.12907959,43.64330373)
\curveto(545.10908397,43.79330052)(545.084084,43.94330037)(545.05407959,44.09330373)
\curveto(545.03408405,44.16330015)(545.02408406,44.22830009)(545.02407959,44.28830373)
\curveto(545.02408406,44.35829996)(545.01408407,44.43329988)(544.99407959,44.51330373)
\curveto(544.97408411,44.58329973)(544.96408412,44.65329966)(544.96407959,44.72330373)
\curveto(544.95408413,44.79329952)(544.93908414,44.86829945)(544.91907959,44.94830373)
\curveto(544.85908422,45.19829912)(544.80908427,45.43329888)(544.76907959,45.65330373)
\curveto(544.71908436,45.87329844)(544.60408448,46.04829827)(544.42407959,46.17830373)
\curveto(544.34408474,46.23829808)(544.24408484,46.28829803)(544.12407959,46.32830373)
\curveto(543.99408509,46.36829795)(543.85408523,46.36829795)(543.70407959,46.32830373)
\curveto(543.46408562,46.26829805)(543.27408581,46.17829814)(543.13407959,46.05830373)
\curveto(542.99408609,45.94829837)(542.8840862,45.78829853)(542.80407959,45.57830373)
\curveto(542.75408633,45.45829886)(542.71908636,45.313299)(542.69907959,45.14330373)
\curveto(542.6790864,44.98329933)(542.66908641,44.8132995)(542.66907959,44.63330373)
\curveto(542.66908641,44.45329986)(542.6790864,44.27830004)(542.69907959,44.10830373)
\curveto(542.71908636,43.93830038)(542.74908633,43.79330052)(542.78907959,43.67330373)
\curveto(542.84908623,43.50330081)(542.93408615,43.33830098)(543.04407959,43.17830373)
\curveto(543.10408598,43.09830122)(543.1840859,43.02330129)(543.28407959,42.95330373)
\curveto(543.37408571,42.89330142)(543.47408561,42.83830148)(543.58407959,42.78830373)
\curveto(543.66408542,42.75830156)(543.74908533,42.72830159)(543.83907959,42.69830373)
\curveto(543.92908515,42.67830164)(543.99908508,42.63330168)(544.04907959,42.56330373)
\curveto(544.079085,42.52330179)(544.10408498,42.45330186)(544.12407959,42.35330373)
\curveto(544.13408495,42.26330205)(544.13908494,42.16830215)(544.13907959,42.06830373)
\curveto(544.13908494,41.96830235)(544.13408495,41.86830245)(544.12407959,41.76830373)
\curveto(544.10408498,41.67830264)(544.079085,41.6133027)(544.04907959,41.57330373)
\curveto(544.01908506,41.53330278)(543.96908511,41.50330281)(543.89907959,41.48330373)
\curveto(543.82908525,41.46330285)(543.75408533,41.46330285)(543.67407959,41.48330373)
\curveto(543.54408554,41.5133028)(543.42408566,41.54330277)(543.31407959,41.57330373)
\curveto(543.19408589,41.6133027)(543.079086,41.65830266)(542.96907959,41.70830373)
\curveto(542.61908646,41.89830242)(542.34908673,42.13830218)(542.15907959,42.42830373)
\curveto(541.95908712,42.7183016)(541.79908728,43.07830124)(541.67907959,43.50830373)
\curveto(541.65908742,43.60830071)(541.64408744,43.70830061)(541.63407959,43.80830373)
\curveto(541.62408746,43.9183004)(541.60908747,44.02830029)(541.58907959,44.13830373)
\curveto(541.5790875,44.17830014)(541.5790875,44.24330007)(541.58907959,44.33330373)
\curveto(541.58908749,44.42329989)(541.5790875,44.47829984)(541.55907959,44.49830373)
\curveto(541.54908753,45.19829912)(541.62908745,45.80829851)(541.79907959,46.32830373)
\curveto(541.96908711,46.84829747)(542.29408679,47.2132971)(542.77407959,47.42330373)
\curveto(542.97408611,47.5132968)(543.20908587,47.56329675)(543.47907959,47.57330373)
\curveto(543.73908534,47.59329672)(544.01408507,47.60329671)(544.30407959,47.60330373)
\lineto(547.61907959,47.60330373)
\curveto(547.75908132,47.60329671)(547.89408119,47.60829671)(548.02407959,47.61830373)
\curveto(548.15408093,47.62829669)(548.25908082,47.65829666)(548.33907959,47.70830373)
\curveto(548.40908067,47.75829656)(548.45908062,47.82329649)(548.48907959,47.90330373)
\curveto(548.52908055,47.99329632)(548.55908052,48.07829624)(548.57907959,48.15830373)
\curveto(548.58908049,48.23829608)(548.63408045,48.29829602)(548.71407959,48.33830373)
\curveto(548.74408034,48.35829596)(548.77408031,48.36829595)(548.80407959,48.36830373)
\curveto(548.83408025,48.36829595)(548.87408021,48.37329594)(548.92407959,48.38330373)
\moveto(547.25907959,46.23830373)
\curveto(547.11908196,46.29829802)(546.95908212,46.32829799)(546.77907959,46.32830373)
\curveto(546.58908249,46.33829798)(546.39408269,46.34329797)(546.19407959,46.34330373)
\curveto(546.084083,46.34329797)(545.9840831,46.33829798)(545.89407959,46.32830373)
\curveto(545.80408328,46.318298)(545.73408335,46.27829804)(545.68407959,46.20830373)
\curveto(545.66408342,46.17829814)(545.65408343,46.10829821)(545.65407959,45.99830373)
\curveto(545.67408341,45.97829834)(545.6840834,45.94329837)(545.68407959,45.89330373)
\curveto(545.6840834,45.84329847)(545.69408339,45.79829852)(545.71407959,45.75830373)
\curveto(545.73408335,45.67829864)(545.75408333,45.58829873)(545.77407959,45.48830373)
\lineto(545.83407959,45.18830373)
\curveto(545.83408325,45.15829916)(545.83908324,45.12329919)(545.84907959,45.08330373)
\lineto(545.84907959,44.97830373)
\curveto(545.88908319,44.82829949)(545.91408317,44.66329965)(545.92407959,44.48330373)
\curveto(545.92408316,44.3133)(545.94408314,44.15330016)(545.98407959,44.00330373)
\curveto(546.00408308,43.92330039)(546.02408306,43.84830047)(546.04407959,43.77830373)
\curveto(546.05408303,43.7183006)(546.06908301,43.64830067)(546.08907959,43.56830373)
\curveto(546.13908294,43.40830091)(546.20408288,43.25830106)(546.28407959,43.11830373)
\curveto(546.35408273,42.97830134)(546.44408264,42.85830146)(546.55407959,42.75830373)
\curveto(546.66408242,42.65830166)(546.79908228,42.58330173)(546.95907959,42.53330373)
\curveto(547.10908197,42.48330183)(547.29408179,42.46330185)(547.51407959,42.47330373)
\curveto(547.61408147,42.47330184)(547.70908137,42.48830183)(547.79907959,42.51830373)
\curveto(547.8790812,42.55830176)(547.95408113,42.60330171)(548.02407959,42.65330373)
\curveto(548.13408095,42.73330158)(548.22908085,42.83830148)(548.30907959,42.96830373)
\curveto(548.3790807,43.09830122)(548.43908064,43.23830108)(548.48907959,43.38830373)
\curveto(548.49908058,43.43830088)(548.50408058,43.48830083)(548.50407959,43.53830373)
\curveto(548.50408058,43.58830073)(548.50908057,43.63830068)(548.51907959,43.68830373)
\curveto(548.53908054,43.75830056)(548.55408053,43.84330047)(548.56407959,43.94330373)
\curveto(548.56408052,44.05330026)(548.55408053,44.14330017)(548.53407959,44.21330373)
\curveto(548.51408057,44.27330004)(548.50908057,44.33329998)(548.51907959,44.39330373)
\curveto(548.51908056,44.45329986)(548.50908057,44.5132998)(548.48907959,44.57330373)
\curveto(548.46908061,44.65329966)(548.45408063,44.72829959)(548.44407959,44.79830373)
\curveto(548.43408065,44.87829944)(548.41408067,44.95329936)(548.38407959,45.02330373)
\curveto(548.26408082,45.313299)(548.11908096,45.55829876)(547.94907959,45.75830373)
\curveto(547.7790813,45.96829835)(547.54908153,46.12829819)(547.25907959,46.23830373)
}
}
{
\newrgbcolor{curcolor}{0 0 0}
\pscustom[linestyle=none,fillstyle=solid,fillcolor=curcolor]
{
\newpath
\moveto(541.75407959,49.26994436)
\lineto(541.75407959,49.71994436)
\curveto(541.74408734,49.88994311)(541.76408732,50.01494298)(541.81407959,50.09494436)
\curveto(541.86408722,50.17494282)(541.92908715,50.22994277)(542.00907959,50.25994436)
\curveto(542.08908699,50.2999427)(542.17408691,50.33994266)(542.26407959,50.37994436)
\curveto(542.39408669,50.42994257)(542.52408656,50.47494252)(542.65407959,50.51494436)
\curveto(542.7840863,50.55494244)(542.91408617,50.5999424)(543.04407959,50.64994436)
\curveto(543.16408592,50.6999423)(543.28908579,50.74494225)(543.41907959,50.78494436)
\curveto(543.53908554,50.82494217)(543.65908542,50.86994213)(543.77907959,50.91994436)
\curveto(543.88908519,50.96994203)(544.00408508,51.00994199)(544.12407959,51.03994436)
\curveto(544.24408484,51.06994193)(544.36408472,51.10994189)(544.48407959,51.15994436)
\curveto(544.77408431,51.27994172)(545.07408401,51.38994161)(545.38407959,51.48994436)
\curveto(545.69408339,51.58994141)(545.99408309,51.6999413)(546.28407959,51.81994436)
\curveto(546.32408276,51.83994116)(546.36408272,51.84994115)(546.40407959,51.84994436)
\curveto(546.43408265,51.84994115)(546.46408262,51.85994114)(546.49407959,51.87994436)
\curveto(546.63408245,51.93994106)(546.7790823,51.994941)(546.92907959,52.04494436)
\lineto(547.34907959,52.19494436)
\curveto(547.41908166,52.22494077)(547.49408159,52.25494074)(547.57407959,52.28494436)
\curveto(547.64408144,52.31494068)(547.68908139,52.36494063)(547.70907959,52.43494436)
\curveto(547.73908134,52.51494048)(547.71408137,52.57494042)(547.63407959,52.61494436)
\curveto(547.54408154,52.66494033)(547.47408161,52.6999403)(547.42407959,52.71994436)
\curveto(547.25408183,52.7999402)(547.07408201,52.86494013)(546.88407959,52.91494436)
\curveto(546.69408239,52.96494003)(546.50908257,53.02493997)(546.32907959,53.09494436)
\curveto(546.09908298,53.18493981)(545.86908321,53.26493973)(545.63907959,53.33494436)
\curveto(545.39908368,53.40493959)(545.16908391,53.48993951)(544.94907959,53.58994436)
\curveto(544.89908418,53.5999394)(544.83408425,53.61493938)(544.75407959,53.63494436)
\curveto(544.66408442,53.67493932)(544.57408451,53.70993929)(544.48407959,53.73994436)
\curveto(544.3840847,53.76993923)(544.29408479,53.7999392)(544.21407959,53.82994436)
\curveto(544.16408492,53.84993915)(544.11908496,53.86493913)(544.07907959,53.87494436)
\curveto(544.03908504,53.88493911)(543.99408509,53.8999391)(543.94407959,53.91994436)
\curveto(543.82408526,53.96993903)(543.70408538,54.00993899)(543.58407959,54.03994436)
\curveto(543.45408563,54.07993892)(543.32908575,54.12493887)(543.20907959,54.17494436)
\curveto(543.15908592,54.1949388)(543.11408597,54.20993879)(543.07407959,54.21994436)
\curveto(543.03408605,54.22993877)(542.98908609,54.24493875)(542.93907959,54.26494436)
\curveto(542.84908623,54.30493869)(542.75908632,54.33993866)(542.66907959,54.36994436)
\curveto(542.56908651,54.3999386)(542.47408661,54.42993857)(542.38407959,54.45994436)
\curveto(542.30408678,54.48993851)(542.22408686,54.51493848)(542.14407959,54.53494436)
\curveto(542.05408703,54.56493843)(541.9790871,54.60493839)(541.91907959,54.65494436)
\curveto(541.82908725,54.72493827)(541.7790873,54.81993818)(541.76907959,54.93994436)
\curveto(541.75908732,55.06993793)(541.75408733,55.20993779)(541.75407959,55.35994436)
\curveto(541.75408733,55.43993756)(541.75908732,55.51493748)(541.76907959,55.58494436)
\curveto(541.76908731,55.66493733)(541.7840873,55.72993727)(541.81407959,55.77994436)
\curveto(541.87408721,55.86993713)(541.96908711,55.8949371)(542.09907959,55.85494436)
\curveto(542.22908685,55.81493718)(542.32908675,55.77993722)(542.39907959,55.74994436)
\lineto(542.45907959,55.71994436)
\curveto(542.4790866,55.71993728)(542.49908658,55.71493728)(542.51907959,55.70494436)
\curveto(542.79908628,55.5949374)(543.084086,55.48493751)(543.37407959,55.37494436)
\lineto(544.21407959,55.04494436)
\curveto(544.29408479,55.01493798)(544.36908471,54.98993801)(544.43907959,54.96994436)
\curveto(544.49908458,54.94993805)(544.56408452,54.92493807)(544.63407959,54.89494436)
\curveto(544.83408425,54.81493818)(545.03908404,54.73493826)(545.24907959,54.65494436)
\curveto(545.44908363,54.58493841)(545.64908343,54.50993849)(545.84907959,54.42994436)
\curveto(546.53908254,54.13993886)(547.23408185,53.86993913)(547.93407959,53.61994436)
\curveto(548.63408045,53.36993963)(549.32907975,53.0999399)(550.01907959,52.80994436)
\lineto(550.16907959,52.74994436)
\curveto(550.22907885,52.73994026)(550.28907879,52.72494027)(550.34907959,52.70494436)
\curveto(550.71907836,52.54494045)(551.084078,52.37494062)(551.44407959,52.19494436)
\curveto(551.81407727,52.01494098)(552.09907698,51.76494123)(552.29907959,51.44494436)
\curveto(552.35907672,51.33494166)(552.40407668,51.22494177)(552.43407959,51.11494436)
\curveto(552.47407661,51.00494199)(552.50907657,50.87994212)(552.53907959,50.73994436)
\curveto(552.55907652,50.68994231)(552.56407652,50.63494236)(552.55407959,50.57494436)
\curveto(552.54407654,50.52494247)(552.54407654,50.46994253)(552.55407959,50.40994436)
\curveto(552.57407651,50.32994267)(552.57407651,50.24994275)(552.55407959,50.16994436)
\curveto(552.54407654,50.12994287)(552.53907654,50.07994292)(552.53907959,50.01994436)
\lineto(552.47907959,49.77994436)
\curveto(552.45907662,49.70994329)(552.41907666,49.65494334)(552.35907959,49.61494436)
\curveto(552.29907678,49.56494343)(552.22407686,49.53494346)(552.13407959,49.52494436)
\lineto(551.86407959,49.52494436)
\lineto(551.65407959,49.52494436)
\curveto(551.59407749,49.53494346)(551.54407754,49.55494344)(551.50407959,49.58494436)
\curveto(551.39407769,49.65494334)(551.36407772,49.77494322)(551.41407959,49.94494436)
\curveto(551.43407765,50.05494294)(551.44407764,50.17494282)(551.44407959,50.30494436)
\curveto(551.44407764,50.43494256)(551.42407766,50.54994245)(551.38407959,50.64994436)
\curveto(551.33407775,50.7999422)(551.25907782,50.91994208)(551.15907959,51.00994436)
\curveto(551.05907802,51.10994189)(550.94407814,51.1949418)(550.81407959,51.26494436)
\curveto(550.69407839,51.33494166)(550.56407852,51.3949416)(550.42407959,51.44494436)
\lineto(550.00407959,51.62494436)
\curveto(549.91407917,51.66494133)(549.80407928,51.70494129)(549.67407959,51.74494436)
\curveto(549.54407954,51.7949412)(549.40907967,51.7999412)(549.26907959,51.75994436)
\curveto(549.10907997,51.70994129)(548.95908012,51.65494134)(548.81907959,51.59494436)
\curveto(548.6790804,51.54494145)(548.53908054,51.48994151)(548.39907959,51.42994436)
\curveto(548.18908089,51.33994166)(547.9790811,51.25494174)(547.76907959,51.17494436)
\curveto(547.55908152,51.0949419)(547.35408173,51.01494198)(547.15407959,50.93494436)
\curveto(547.01408207,50.87494212)(546.8790822,50.81994218)(546.74907959,50.76994436)
\curveto(546.61908246,50.71994228)(546.4840826,50.66994233)(546.34407959,50.61994436)
\lineto(545.02407959,50.07994436)
\curveto(544.5840845,49.90994309)(544.14408494,49.73494326)(543.70407959,49.55494436)
\curveto(543.47408561,49.45494354)(543.25408583,49.36494363)(543.04407959,49.28494436)
\curveto(542.82408626,49.20494379)(542.60408648,49.11994388)(542.38407959,49.02994436)
\curveto(542.32408676,49.00994399)(542.24408684,48.97994402)(542.14407959,48.93994436)
\curveto(542.03408705,48.8999441)(541.94408714,48.90494409)(541.87407959,48.95494436)
\curveto(541.82408726,48.98494401)(541.78908729,49.04494395)(541.76907959,49.13494436)
\curveto(541.75908732,49.15494384)(541.75908732,49.17494382)(541.76907959,49.19494436)
\curveto(541.76908731,49.22494377)(541.76408732,49.24994375)(541.75407959,49.26994436)
}
}
{
\newrgbcolor{curcolor}{0 0 0}
\pscustom[linestyle=none,fillstyle=solid,fillcolor=curcolor]
{
}
}
{
\newrgbcolor{curcolor}{0 0 0}
\pscustom[linestyle=none,fillstyle=solid,fillcolor=curcolor]
{
\newpath
\moveto(538.87407959,65.05510061)
\curveto(538.87409021,65.15509575)(538.8840902,65.25009566)(538.90407959,65.34010061)
\curveto(538.91409017,65.43009548)(538.94409014,65.49509541)(538.99407959,65.53510061)
\curveto(539.07409001,65.59509531)(539.1790899,65.62509528)(539.30907959,65.62510061)
\lineto(539.69907959,65.62510061)
\lineto(541.19907959,65.62510061)
\lineto(547.58907959,65.62510061)
\lineto(548.75907959,65.62510061)
\lineto(549.07407959,65.62510061)
\curveto(549.17407991,65.63509527)(549.25407983,65.62009529)(549.31407959,65.58010061)
\curveto(549.39407969,65.53009538)(549.44407964,65.45509545)(549.46407959,65.35510061)
\curveto(549.47407961,65.26509564)(549.4790796,65.15509575)(549.47907959,65.02510061)
\lineto(549.47907959,64.80010061)
\curveto(549.45907962,64.72009619)(549.44407964,64.65009626)(549.43407959,64.59010061)
\curveto(549.41407967,64.53009638)(549.37407971,64.48009643)(549.31407959,64.44010061)
\curveto(549.25407983,64.40009651)(549.1790799,64.38009653)(549.08907959,64.38010061)
\lineto(548.78907959,64.38010061)
\lineto(547.69407959,64.38010061)
\lineto(542.35407959,64.38010061)
\curveto(542.26408682,64.36009655)(542.18908689,64.34509656)(542.12907959,64.33510061)
\curveto(542.05908702,64.33509657)(541.99908708,64.3050966)(541.94907959,64.24510061)
\curveto(541.89908718,64.17509673)(541.87408721,64.08509682)(541.87407959,63.97510061)
\curveto(541.86408722,63.87509703)(541.85908722,63.76509714)(541.85907959,63.64510061)
\lineto(541.85907959,62.50510061)
\lineto(541.85907959,62.01010061)
\curveto(541.84908723,61.85009906)(541.78908729,61.74009917)(541.67907959,61.68010061)
\curveto(541.64908743,61.66009925)(541.61908746,61.65009926)(541.58907959,61.65010061)
\curveto(541.54908753,61.65009926)(541.50408758,61.64509926)(541.45407959,61.63510061)
\curveto(541.33408775,61.61509929)(541.22408786,61.62009929)(541.12407959,61.65010061)
\curveto(541.02408806,61.69009922)(540.95408813,61.74509916)(540.91407959,61.81510061)
\curveto(540.86408822,61.89509901)(540.83908824,62.01509889)(540.83907959,62.17510061)
\curveto(540.83908824,62.33509857)(540.82408826,62.47009844)(540.79407959,62.58010061)
\curveto(540.7840883,62.63009828)(540.7790883,62.68509822)(540.77907959,62.74510061)
\curveto(540.76908831,62.8050981)(540.75408833,62.86509804)(540.73407959,62.92510061)
\curveto(540.6840884,63.07509783)(540.63408845,63.22009769)(540.58407959,63.36010061)
\curveto(540.52408856,63.50009741)(540.45408863,63.63509727)(540.37407959,63.76510061)
\curveto(540.2840888,63.905097)(540.1790889,64.02509688)(540.05907959,64.12510061)
\curveto(539.93908914,64.22509668)(539.80908927,64.32009659)(539.66907959,64.41010061)
\curveto(539.56908951,64.47009644)(539.45908962,64.51509639)(539.33907959,64.54510061)
\curveto(539.21908986,64.58509632)(539.11408997,64.63509627)(539.02407959,64.69510061)
\curveto(538.96409012,64.74509616)(538.92409016,64.81509609)(538.90407959,64.90510061)
\curveto(538.89409019,64.92509598)(538.88909019,64.95009596)(538.88907959,64.98010061)
\curveto(538.88909019,65.0100959)(538.8840902,65.03509587)(538.87407959,65.05510061)
}
}
{
\newrgbcolor{curcolor}{0 0 0}
\pscustom[linestyle=none,fillstyle=solid,fillcolor=curcolor]
{
\newpath
\moveto(538.87407959,72.45970998)
\curveto(538.84409024,74.08970454)(539.39908968,75.13970349)(540.53907959,75.60970998)
\curveto(540.76908831,75.70970292)(541.05908802,75.77470286)(541.40907959,75.80470998)
\curveto(541.74908733,75.84470279)(542.05908702,75.81970281)(542.33907959,75.72970998)
\curveto(542.59908648,75.63970299)(542.82408626,75.51970311)(543.01407959,75.36970998)
\curveto(543.05408603,75.34970328)(543.08908599,75.32470331)(543.11907959,75.29470998)
\curveto(543.13908594,75.26470337)(543.16408592,75.23970339)(543.19407959,75.21970998)
\lineto(543.31407959,75.12970998)
\curveto(543.34408574,75.09970353)(543.36908571,75.06470357)(543.38907959,75.02470998)
\curveto(543.43908564,74.97470366)(543.4840856,74.91970371)(543.52407959,74.85970998)
\curveto(543.56408552,74.80970382)(543.61408547,74.76470387)(543.67407959,74.72470998)
\curveto(543.71408537,74.68470395)(543.76408532,74.66970396)(543.82407959,74.67970998)
\curveto(543.87408521,74.68970394)(543.91908516,74.71970391)(543.95907959,74.76970998)
\curveto(543.99908508,74.81970381)(544.03908504,74.87470376)(544.07907959,74.93470998)
\curveto(544.10908497,75.00470363)(544.13908494,75.06970356)(544.16907959,75.12970998)
\curveto(544.19908488,75.18970344)(544.22908485,75.23970339)(544.25907959,75.27970998)
\curveto(544.4790846,75.59970303)(544.78908429,75.85470278)(545.18907959,76.04470998)
\curveto(545.2790838,76.08470255)(545.37408371,76.11470252)(545.47407959,76.13470998)
\curveto(545.56408352,76.16470247)(545.65408343,76.18970244)(545.74407959,76.20970998)
\curveto(545.79408329,76.21970241)(545.84408324,76.22470241)(545.89407959,76.22470998)
\curveto(545.93408315,76.2347024)(545.9790831,76.24470239)(546.02907959,76.25470998)
\curveto(546.079083,76.26470237)(546.12908295,76.26470237)(546.17907959,76.25470998)
\curveto(546.22908285,76.24470239)(546.2790828,76.24970238)(546.32907959,76.26970998)
\curveto(546.3790827,76.27970235)(546.43908264,76.28470235)(546.50907959,76.28470998)
\curveto(546.5790825,76.28470235)(546.63908244,76.27470236)(546.68907959,76.25470998)
\lineto(546.91407959,76.25470998)
\lineto(547.15407959,76.19470998)
\curveto(547.22408186,76.18470245)(547.29408179,76.16970246)(547.36407959,76.14970998)
\curveto(547.45408163,76.11970251)(547.53908154,76.08970254)(547.61907959,76.05970998)
\curveto(547.69908138,76.03970259)(547.7790813,76.00970262)(547.85907959,75.96970998)
\curveto(547.91908116,75.94970268)(547.9790811,75.91970271)(548.03907959,75.87970998)
\curveto(548.08908099,75.84970278)(548.13908094,75.81470282)(548.18907959,75.77470998)
\curveto(548.49908058,75.57470306)(548.75908032,75.32470331)(548.96907959,75.02470998)
\curveto(549.16907991,74.72470391)(549.33407975,74.37970425)(549.46407959,73.98970998)
\curveto(549.50407958,73.86970476)(549.52907955,73.73970489)(549.53907959,73.59970998)
\curveto(549.55907952,73.46970516)(549.5840795,73.3347053)(549.61407959,73.19470998)
\curveto(549.62407946,73.12470551)(549.62907945,73.05470558)(549.62907959,72.98470998)
\curveto(549.62907945,72.92470571)(549.63407945,72.85970577)(549.64407959,72.78970998)
\curveto(549.65407943,72.74970588)(549.65907942,72.68970594)(549.65907959,72.60970998)
\curveto(549.65907942,72.53970609)(549.65407943,72.48970614)(549.64407959,72.45970998)
\curveto(549.63407945,72.40970622)(549.62907945,72.36470627)(549.62907959,72.32470998)
\lineto(549.62907959,72.20470998)
\curveto(549.60907947,72.10470653)(549.59407949,72.00470663)(549.58407959,71.90470998)
\curveto(549.57407951,71.80470683)(549.55907952,71.70970692)(549.53907959,71.61970998)
\curveto(549.50907957,71.50970712)(549.4840796,71.39970723)(549.46407959,71.28970998)
\curveto(549.43407965,71.18970744)(549.39407969,71.08470755)(549.34407959,70.97470998)
\curveto(549.1840799,70.60470803)(548.9840801,70.28970834)(548.74407959,70.02970998)
\curveto(548.49408059,69.76970886)(548.1840809,69.55970907)(547.81407959,69.39970998)
\curveto(547.72408136,69.35970927)(547.62908145,69.32470931)(547.52907959,69.29470998)
\curveto(547.42908165,69.26470937)(547.32408176,69.2347094)(547.21407959,69.20470998)
\curveto(547.16408192,69.18470945)(547.11408197,69.17470946)(547.06407959,69.17470998)
\curveto(547.00408208,69.17470946)(546.94408214,69.16470947)(546.88407959,69.14470998)
\curveto(546.82408226,69.12470951)(546.74408234,69.11470952)(546.64407959,69.11470998)
\curveto(546.54408254,69.11470952)(546.46908261,69.1297095)(546.41907959,69.15970998)
\curveto(546.38908269,69.16970946)(546.36408272,69.18470945)(546.34407959,69.20470998)
\lineto(546.28407959,69.26470998)
\curveto(546.26408282,69.30470933)(546.24908283,69.36470927)(546.23907959,69.44470998)
\curveto(546.22908285,69.5347091)(546.22408286,69.62470901)(546.22407959,69.71470998)
\curveto(546.22408286,69.80470883)(546.22908285,69.88970874)(546.23907959,69.96970998)
\curveto(546.24908283,70.05970857)(546.25908282,70.12470851)(546.26907959,70.16470998)
\curveto(546.28908279,70.18470845)(546.30408278,70.20470843)(546.31407959,70.22470998)
\curveto(546.31408277,70.24470839)(546.32408276,70.26470837)(546.34407959,70.28470998)
\curveto(546.43408265,70.35470828)(546.54908253,70.39470824)(546.68907959,70.40470998)
\curveto(546.82908225,70.42470821)(546.95408213,70.45470818)(547.06407959,70.49470998)
\lineto(547.42407959,70.64470998)
\curveto(547.53408155,70.69470794)(547.63908144,70.75970787)(547.73907959,70.83970998)
\curveto(547.76908131,70.85970777)(547.79408129,70.87970775)(547.81407959,70.89970998)
\curveto(547.83408125,70.9297077)(547.85908122,70.95470768)(547.88907959,70.97470998)
\curveto(547.94908113,71.01470762)(547.99408109,71.04970758)(548.02407959,71.07970998)
\curveto(548.05408103,71.11970751)(548.084081,71.15470748)(548.11407959,71.18470998)
\curveto(548.14408094,71.22470741)(548.17408091,71.26970736)(548.20407959,71.31970998)
\curveto(548.26408082,71.40970722)(548.31408077,71.50470713)(548.35407959,71.60470998)
\lineto(548.47407959,71.93470998)
\curveto(548.52408056,72.08470655)(548.55408053,72.28470635)(548.56407959,72.53470998)
\curveto(548.57408051,72.78470585)(548.55408053,72.99470564)(548.50407959,73.16470998)
\curveto(548.4840806,73.24470539)(548.46908061,73.31470532)(548.45907959,73.37470998)
\lineto(548.39907959,73.58470998)
\curveto(548.2790808,73.86470477)(548.12908095,74.10470453)(547.94907959,74.30470998)
\curveto(547.76908131,74.51470412)(547.53908154,74.67970395)(547.25907959,74.79970998)
\curveto(547.18908189,74.8297038)(547.11908196,74.84970378)(547.04907959,74.85970998)
\lineto(546.80907959,74.91970998)
\curveto(546.66908241,74.95970367)(546.50908257,74.96970366)(546.32907959,74.94970998)
\curveto(546.13908294,74.9297037)(545.98908309,74.89970373)(545.87907959,74.85970998)
\curveto(545.49908358,74.7297039)(545.20908387,74.54470409)(545.00907959,74.30470998)
\curveto(544.80908427,74.07470456)(544.64908443,73.76470487)(544.52907959,73.37470998)
\curveto(544.49908458,73.26470537)(544.4790846,73.14470549)(544.46907959,73.01470998)
\curveto(544.45908462,72.89470574)(544.45408463,72.76970586)(544.45407959,72.63970998)
\curveto(544.45408463,72.47970615)(544.44908463,72.33970629)(544.43907959,72.21970998)
\curveto(544.42908465,72.09970653)(544.36908471,72.01470662)(544.25907959,71.96470998)
\curveto(544.22908485,71.94470669)(544.19408489,71.9347067)(544.15407959,71.93470998)
\lineto(544.01907959,71.93470998)
\curveto(543.91908516,71.92470671)(543.82408526,71.92470671)(543.73407959,71.93470998)
\curveto(543.64408544,71.95470668)(543.5790855,71.99470664)(543.53907959,72.05470998)
\curveto(543.50908557,72.09470654)(543.48908559,72.1347065)(543.47907959,72.17470998)
\curveto(543.46908561,72.22470641)(543.45908562,72.27970635)(543.44907959,72.33970998)
\curveto(543.43908564,72.35970627)(543.43908564,72.38470625)(543.44907959,72.41470998)
\curveto(543.44908563,72.44470619)(543.44408564,72.46970616)(543.43407959,72.48970998)
\lineto(543.43407959,72.62470998)
\curveto(543.41408567,72.7347059)(543.40408568,72.8347058)(543.40407959,72.92470998)
\curveto(543.39408569,73.02470561)(543.37408571,73.11970551)(543.34407959,73.20970998)
\curveto(543.23408585,73.5297051)(543.08908599,73.78470485)(542.90907959,73.97470998)
\curveto(542.72908635,74.16470447)(542.4790866,74.31470432)(542.15907959,74.42470998)
\curveto(542.05908702,74.45470418)(541.93408715,74.47470416)(541.78407959,74.48470998)
\curveto(541.62408746,74.50470413)(541.4790876,74.49970413)(541.34907959,74.46970998)
\curveto(541.2790878,74.44970418)(541.21408787,74.4297042)(541.15407959,74.40970998)
\curveto(541.084088,74.39970423)(541.01908806,74.37970425)(540.95907959,74.34970998)
\curveto(540.71908836,74.24970438)(540.52908855,74.10470453)(540.38907959,73.91470998)
\curveto(540.24908883,73.72470491)(540.13908894,73.49970513)(540.05907959,73.23970998)
\curveto(540.03908904,73.17970545)(540.02908905,73.11970551)(540.02907959,73.05970998)
\curveto(540.02908905,72.99970563)(540.01908906,72.9347057)(539.99907959,72.86470998)
\curveto(539.9790891,72.78470585)(539.96908911,72.68970594)(539.96907959,72.57970998)
\curveto(539.96908911,72.46970616)(539.9790891,72.37470626)(539.99907959,72.29470998)
\curveto(540.01908906,72.24470639)(540.02908905,72.19470644)(540.02907959,72.14470998)
\curveto(540.02908905,72.10470653)(540.03908904,72.05970657)(540.05907959,72.00970998)
\curveto(540.10908897,71.8297068)(540.1840889,71.65970697)(540.28407959,71.49970998)
\curveto(540.37408871,71.34970728)(540.48908859,71.21970741)(540.62907959,71.10970998)
\curveto(540.74908833,71.01970761)(540.8790882,70.93970769)(541.01907959,70.86970998)
\curveto(541.15908792,70.79970783)(541.31408777,70.7347079)(541.48407959,70.67470998)
\curveto(541.59408749,70.64470799)(541.71408737,70.62470801)(541.84407959,70.61470998)
\curveto(541.96408712,70.60470803)(542.06408702,70.56970806)(542.14407959,70.50970998)
\curveto(542.1840869,70.48970814)(542.22408686,70.4297082)(542.26407959,70.32970998)
\curveto(542.27408681,70.28970834)(542.2840868,70.2297084)(542.29407959,70.14970998)
\lineto(542.29407959,69.89470998)
\curveto(542.2840868,69.80470883)(542.27408681,69.71970891)(542.26407959,69.63970998)
\curveto(542.25408683,69.56970906)(542.23908684,69.51970911)(542.21907959,69.48970998)
\curveto(542.18908689,69.44970918)(542.13408695,69.41470922)(542.05407959,69.38470998)
\curveto(541.97408711,69.35470928)(541.88908719,69.34970928)(541.79907959,69.36970998)
\curveto(541.74908733,69.37970925)(541.69908738,69.38470925)(541.64907959,69.38470998)
\lineto(541.46907959,69.41470998)
\curveto(541.36908771,69.44470919)(541.26908781,69.46970916)(541.16907959,69.48970998)
\curveto(541.06908801,69.51970911)(540.9790881,69.55470908)(540.89907959,69.59470998)
\curveto(540.78908829,69.64470899)(540.6840884,69.68970894)(540.58407959,69.72970998)
\curveto(540.47408861,69.76970886)(540.36908871,69.81970881)(540.26907959,69.87970998)
\curveto(539.72908935,70.20970842)(539.33408975,70.67970795)(539.08407959,71.28970998)
\curveto(539.03409005,71.40970722)(538.99909008,71.5347071)(538.97907959,71.66470998)
\curveto(538.95909012,71.80470683)(538.93409015,71.94470669)(538.90407959,72.08470998)
\curveto(538.89409019,72.14470649)(538.88909019,72.20470643)(538.88907959,72.26470998)
\curveto(538.88909019,72.3347063)(538.8840902,72.39970623)(538.87407959,72.45970998)
}
}
{
\newrgbcolor{curcolor}{0 0 0}
\pscustom[linestyle=none,fillstyle=solid,fillcolor=curcolor]
{
\newpath
\moveto(547.84407959,78.63431936)
\lineto(547.84407959,79.26431936)
\lineto(547.84407959,79.45931936)
\curveto(547.84408124,79.52931683)(547.85408123,79.58931677)(547.87407959,79.63931936)
\curveto(547.91408117,79.70931665)(547.95408113,79.7593166)(547.99407959,79.78931936)
\curveto(548.04408104,79.82931653)(548.10908097,79.84931651)(548.18907959,79.84931936)
\curveto(548.26908081,79.8593165)(548.35408073,79.86431649)(548.44407959,79.86431936)
\lineto(549.16407959,79.86431936)
\curveto(549.64407944,79.86431649)(550.05407903,79.80431655)(550.39407959,79.68431936)
\curveto(550.73407835,79.56431679)(551.00907807,79.36931699)(551.21907959,79.09931936)
\curveto(551.26907781,79.02931733)(551.31407777,78.9593174)(551.35407959,78.88931936)
\curveto(551.40407768,78.82931753)(551.44907763,78.7543176)(551.48907959,78.66431936)
\curveto(551.49907758,78.64431771)(551.50907757,78.61431774)(551.51907959,78.57431936)
\curveto(551.53907754,78.53431782)(551.54407754,78.48931787)(551.53407959,78.43931936)
\curveto(551.50407758,78.34931801)(551.42907765,78.29431806)(551.30907959,78.27431936)
\curveto(551.19907788,78.2543181)(551.10407798,78.26931809)(551.02407959,78.31931936)
\curveto(550.95407813,78.34931801)(550.88907819,78.39431796)(550.82907959,78.45431936)
\curveto(550.7790783,78.52431783)(550.72907835,78.58931777)(550.67907959,78.64931936)
\curveto(550.62907845,78.71931764)(550.55407853,78.77931758)(550.45407959,78.82931936)
\curveto(550.36407872,78.88931747)(550.27407881,78.93931742)(550.18407959,78.97931936)
\curveto(550.15407893,78.99931736)(550.09407899,79.02431733)(550.00407959,79.05431936)
\curveto(549.92407916,79.08431727)(549.85407923,79.08931727)(549.79407959,79.06931936)
\curveto(549.65407943,79.03931732)(549.56407952,78.97931738)(549.52407959,78.88931936)
\curveto(549.49407959,78.80931755)(549.4790796,78.71931764)(549.47907959,78.61931936)
\curveto(549.4790796,78.51931784)(549.45407963,78.43431792)(549.40407959,78.36431936)
\curveto(549.33407975,78.27431808)(549.19407989,78.22931813)(548.98407959,78.22931936)
\lineto(548.42907959,78.22931936)
\lineto(548.20407959,78.22931936)
\curveto(548.12408096,78.23931812)(548.05908102,78.2593181)(548.00907959,78.28931936)
\curveto(547.92908115,78.34931801)(547.8840812,78.41931794)(547.87407959,78.49931936)
\curveto(547.86408122,78.51931784)(547.85908122,78.53931782)(547.85907959,78.55931936)
\curveto(547.85908122,78.58931777)(547.85408123,78.61431774)(547.84407959,78.63431936)
}
}
{
\newrgbcolor{curcolor}{0 0 0}
\pscustom[linestyle=none,fillstyle=solid,fillcolor=curcolor]
{
}
}
{
\newrgbcolor{curcolor}{0 0 0}
\pscustom[linestyle=none,fillstyle=solid,fillcolor=curcolor]
{
\newpath
\moveto(538.87407959,89.26463186)
\curveto(538.86409022,89.95462722)(538.9840901,90.55462662)(539.23407959,91.06463186)
\curveto(539.4840896,91.58462559)(539.81908926,91.9796252)(540.23907959,92.24963186)
\curveto(540.31908876,92.29962488)(540.40908867,92.34462483)(540.50907959,92.38463186)
\curveto(540.59908848,92.42462475)(540.69408839,92.46962471)(540.79407959,92.51963186)
\curveto(540.89408819,92.55962462)(540.99408809,92.58962459)(541.09407959,92.60963186)
\curveto(541.19408789,92.62962455)(541.29908778,92.64962453)(541.40907959,92.66963186)
\curveto(541.45908762,92.68962449)(541.50408758,92.69462448)(541.54407959,92.68463186)
\curveto(541.5840875,92.6746245)(541.62908745,92.6796245)(541.67907959,92.69963186)
\curveto(541.72908735,92.70962447)(541.81408727,92.71462446)(541.93407959,92.71463186)
\curveto(542.04408704,92.71462446)(542.12908695,92.70962447)(542.18907959,92.69963186)
\curveto(542.24908683,92.6796245)(542.30908677,92.66962451)(542.36907959,92.66963186)
\curveto(542.42908665,92.6796245)(542.48908659,92.6746245)(542.54907959,92.65463186)
\curveto(542.68908639,92.61462456)(542.82408626,92.5796246)(542.95407959,92.54963186)
\curveto(543.084086,92.51962466)(543.20908587,92.4796247)(543.32907959,92.42963186)
\curveto(543.46908561,92.36962481)(543.59408549,92.29962488)(543.70407959,92.21963186)
\curveto(543.81408527,92.14962503)(543.92408516,92.0746251)(544.03407959,91.99463186)
\lineto(544.09407959,91.93463186)
\curveto(544.11408497,91.92462525)(544.13408495,91.90962527)(544.15407959,91.88963186)
\curveto(544.31408477,91.76962541)(544.45908462,91.63462554)(544.58907959,91.48463186)
\curveto(544.71908436,91.33462584)(544.84408424,91.174626)(544.96407959,91.00463186)
\curveto(545.1840839,90.69462648)(545.38908369,90.39962678)(545.57907959,90.11963186)
\curveto(545.71908336,89.88962729)(545.85408323,89.65962752)(545.98407959,89.42963186)
\curveto(546.11408297,89.20962797)(546.24908283,88.98962819)(546.38907959,88.76963186)
\curveto(546.55908252,88.51962866)(546.73908234,88.2796289)(546.92907959,88.04963186)
\curveto(547.11908196,87.82962935)(547.34408174,87.63962954)(547.60407959,87.47963186)
\curveto(547.66408142,87.43962974)(547.72408136,87.40462977)(547.78407959,87.37463186)
\curveto(547.83408125,87.34462983)(547.89908118,87.31462986)(547.97907959,87.28463186)
\curveto(548.04908103,87.26462991)(548.10908097,87.25962992)(548.15907959,87.26963186)
\curveto(548.22908085,87.28962989)(548.2840808,87.32462985)(548.32407959,87.37463186)
\curveto(548.35408073,87.42462975)(548.37408071,87.48462969)(548.38407959,87.55463186)
\lineto(548.38407959,87.79463186)
\lineto(548.38407959,88.54463186)
\lineto(548.38407959,91.34963186)
\lineto(548.38407959,92.00963186)
\curveto(548.3840807,92.09962508)(548.38908069,92.18462499)(548.39907959,92.26463186)
\curveto(548.39908068,92.34462483)(548.41908066,92.40962477)(548.45907959,92.45963186)
\curveto(548.49908058,92.50962467)(548.57408051,92.54962463)(548.68407959,92.57963186)
\curveto(548.7840803,92.61962456)(548.8840802,92.62962455)(548.98407959,92.60963186)
\lineto(549.11907959,92.60963186)
\curveto(549.18907989,92.58962459)(549.24907983,92.56962461)(549.29907959,92.54963186)
\curveto(549.34907973,92.52962465)(549.38907969,92.49462468)(549.41907959,92.44463186)
\curveto(549.45907962,92.39462478)(549.4790796,92.32462485)(549.47907959,92.23463186)
\lineto(549.47907959,91.96463186)
\lineto(549.47907959,91.06463186)
\lineto(549.47907959,87.55463186)
\lineto(549.47907959,86.48963186)
\curveto(549.4790796,86.40963077)(549.4840796,86.31963086)(549.49407959,86.21963186)
\curveto(549.49407959,86.11963106)(549.4840796,86.03463114)(549.46407959,85.96463186)
\curveto(549.39407969,85.75463142)(549.21407987,85.68963149)(548.92407959,85.76963186)
\curveto(548.8840802,85.7796314)(548.84908023,85.7796314)(548.81907959,85.76963186)
\curveto(548.7790803,85.76963141)(548.73408035,85.7796314)(548.68407959,85.79963186)
\curveto(548.60408048,85.81963136)(548.51908056,85.83963134)(548.42907959,85.85963186)
\curveto(548.33908074,85.8796313)(548.25408083,85.90463127)(548.17407959,85.93463186)
\curveto(547.6840814,86.09463108)(547.26908181,86.29463088)(546.92907959,86.53463186)
\curveto(546.6790824,86.71463046)(546.45408263,86.91963026)(546.25407959,87.14963186)
\curveto(546.04408304,87.3796298)(545.84908323,87.61962956)(545.66907959,87.86963186)
\curveto(545.48908359,88.12962905)(545.31908376,88.39462878)(545.15907959,88.66463186)
\curveto(544.98908409,88.94462823)(544.81408427,89.21462796)(544.63407959,89.47463186)
\curveto(544.55408453,89.58462759)(544.4790846,89.68962749)(544.40907959,89.78963186)
\curveto(544.33908474,89.89962728)(544.26408482,90.00962717)(544.18407959,90.11963186)
\curveto(544.15408493,90.15962702)(544.12408496,90.19462698)(544.09407959,90.22463186)
\curveto(544.05408503,90.26462691)(544.02408506,90.30462687)(544.00407959,90.34463186)
\curveto(543.89408519,90.48462669)(543.76908531,90.60962657)(543.62907959,90.71963186)
\curveto(543.59908548,90.73962644)(543.57408551,90.76462641)(543.55407959,90.79463186)
\curveto(543.52408556,90.82462635)(543.49408559,90.84962633)(543.46407959,90.86963186)
\curveto(543.36408572,90.94962623)(543.26408582,91.01462616)(543.16407959,91.06463186)
\curveto(543.06408602,91.12462605)(542.95408613,91.179626)(542.83407959,91.22963186)
\curveto(542.76408632,91.25962592)(542.68908639,91.2796259)(542.60907959,91.28963186)
\lineto(542.36907959,91.34963186)
\lineto(542.27907959,91.34963186)
\curveto(542.24908683,91.35962582)(542.21908686,91.36462581)(542.18907959,91.36463186)
\curveto(542.11908696,91.38462579)(542.02408706,91.38962579)(541.90407959,91.37963186)
\curveto(541.77408731,91.3796258)(541.67408741,91.36962581)(541.60407959,91.34963186)
\curveto(541.52408756,91.32962585)(541.44908763,91.30962587)(541.37907959,91.28963186)
\curveto(541.29908778,91.2796259)(541.21908786,91.25962592)(541.13907959,91.22963186)
\curveto(540.89908818,91.11962606)(540.69908838,90.96962621)(540.53907959,90.77963186)
\curveto(540.36908871,90.59962658)(540.22908885,90.3796268)(540.11907959,90.11963186)
\curveto(540.09908898,90.04962713)(540.084089,89.9796272)(540.07407959,89.90963186)
\curveto(540.05408903,89.83962734)(540.03408905,89.76462741)(540.01407959,89.68463186)
\curveto(539.99408909,89.60462757)(539.9840891,89.49462768)(539.98407959,89.35463186)
\curveto(539.9840891,89.22462795)(539.99408909,89.11962806)(540.01407959,89.03963186)
\curveto(540.02408906,88.9796282)(540.02908905,88.92462825)(540.02907959,88.87463186)
\curveto(540.02908905,88.82462835)(540.03908904,88.7746284)(540.05907959,88.72463186)
\curveto(540.09908898,88.62462855)(540.13908894,88.52962865)(540.17907959,88.43963186)
\curveto(540.21908886,88.35962882)(540.26408882,88.2796289)(540.31407959,88.19963186)
\curveto(540.33408875,88.16962901)(540.35908872,88.13962904)(540.38907959,88.10963186)
\curveto(540.41908866,88.08962909)(540.44408864,88.06462911)(540.46407959,88.03463186)
\lineto(540.53907959,87.95963186)
\curveto(540.55908852,87.92962925)(540.5790885,87.90462927)(540.59907959,87.88463186)
\lineto(540.80907959,87.73463186)
\curveto(540.86908821,87.69462948)(540.93408815,87.64962953)(541.00407959,87.59963186)
\curveto(541.09408799,87.53962964)(541.19908788,87.48962969)(541.31907959,87.44963186)
\curveto(541.42908765,87.41962976)(541.53908754,87.38462979)(541.64907959,87.34463186)
\curveto(541.75908732,87.30462987)(541.90408718,87.2796299)(542.08407959,87.26963186)
\curveto(542.25408683,87.25962992)(542.3790867,87.22962995)(542.45907959,87.17963186)
\curveto(542.53908654,87.12963005)(542.5840865,87.05463012)(542.59407959,86.95463186)
\curveto(542.60408648,86.85463032)(542.60908647,86.74463043)(542.60907959,86.62463186)
\curveto(542.60908647,86.58463059)(542.61408647,86.54463063)(542.62407959,86.50463186)
\curveto(542.62408646,86.46463071)(542.61908646,86.42963075)(542.60907959,86.39963186)
\curveto(542.58908649,86.34963083)(542.5790865,86.29963088)(542.57907959,86.24963186)
\curveto(542.5790865,86.20963097)(542.56908651,86.16963101)(542.54907959,86.12963186)
\curveto(542.48908659,86.03963114)(542.35408673,85.99463118)(542.14407959,85.99463186)
\lineto(542.02407959,85.99463186)
\curveto(541.96408712,86.00463117)(541.90408718,86.00963117)(541.84407959,86.00963186)
\curveto(541.77408731,86.01963116)(541.70908737,86.02963115)(541.64907959,86.03963186)
\curveto(541.53908754,86.05963112)(541.43908764,86.0796311)(541.34907959,86.09963186)
\curveto(541.24908783,86.11963106)(541.15408793,86.14963103)(541.06407959,86.18963186)
\curveto(540.99408809,86.20963097)(540.93408815,86.22963095)(540.88407959,86.24963186)
\lineto(540.70407959,86.30963186)
\curveto(540.44408864,86.42963075)(540.19908888,86.58463059)(539.96907959,86.77463186)
\curveto(539.73908934,86.9746302)(539.55408953,87.18962999)(539.41407959,87.41963186)
\curveto(539.33408975,87.52962965)(539.26908981,87.64462953)(539.21907959,87.76463186)
\lineto(539.06907959,88.15463186)
\curveto(539.01909006,88.26462891)(538.98909009,88.3796288)(538.97907959,88.49963186)
\curveto(538.95909012,88.61962856)(538.93409015,88.74462843)(538.90407959,88.87463186)
\curveto(538.90409018,88.94462823)(538.90409018,89.00962817)(538.90407959,89.06963186)
\curveto(538.89409019,89.12962805)(538.8840902,89.19462798)(538.87407959,89.26463186)
}
}
{
\newrgbcolor{curcolor}{0 0 0}
\pscustom[linestyle=none,fillstyle=solid,fillcolor=curcolor]
{
\newpath
\moveto(544.39407959,101.36424123)
\lineto(544.64907959,101.36424123)
\curveto(544.72908435,101.37423353)(544.80408428,101.36923353)(544.87407959,101.34924123)
\lineto(545.11407959,101.34924123)
\lineto(545.27907959,101.34924123)
\curveto(545.3790837,101.32923357)(545.4840836,101.31923358)(545.59407959,101.31924123)
\curveto(545.69408339,101.31923358)(545.79408329,101.30923359)(545.89407959,101.28924123)
\lineto(546.04407959,101.28924123)
\curveto(546.1840829,101.25923364)(546.32408276,101.23923366)(546.46407959,101.22924123)
\curveto(546.59408249,101.21923368)(546.72408236,101.19423371)(546.85407959,101.15424123)
\curveto(546.93408215,101.13423377)(547.01908206,101.11423379)(547.10907959,101.09424123)
\lineto(547.34907959,101.03424123)
\lineto(547.64907959,100.91424123)
\curveto(547.73908134,100.88423402)(547.82908125,100.84923405)(547.91907959,100.80924123)
\curveto(548.13908094,100.70923419)(548.35408073,100.57423433)(548.56407959,100.40424123)
\curveto(548.77408031,100.24423466)(548.94408014,100.06923483)(549.07407959,99.87924123)
\curveto(549.11407997,99.82923507)(549.15407993,99.76923513)(549.19407959,99.69924123)
\curveto(549.22407986,99.63923526)(549.25907982,99.57923532)(549.29907959,99.51924123)
\curveto(549.34907973,99.43923546)(549.38907969,99.34423556)(549.41907959,99.23424123)
\curveto(549.44907963,99.12423578)(549.4790796,99.01923588)(549.50907959,98.91924123)
\curveto(549.54907953,98.80923609)(549.57407951,98.6992362)(549.58407959,98.58924123)
\curveto(549.59407949,98.47923642)(549.60907947,98.36423654)(549.62907959,98.24424123)
\curveto(549.63907944,98.2042367)(549.63907944,98.15923674)(549.62907959,98.10924123)
\curveto(549.62907945,98.06923683)(549.63407945,98.02923687)(549.64407959,97.98924123)
\curveto(549.65407943,97.94923695)(549.65907942,97.89423701)(549.65907959,97.82424123)
\curveto(549.65907942,97.75423715)(549.65407943,97.7042372)(549.64407959,97.67424123)
\curveto(549.62407946,97.62423728)(549.61907946,97.57923732)(549.62907959,97.53924123)
\curveto(549.63907944,97.4992374)(549.63907944,97.46423744)(549.62907959,97.43424123)
\lineto(549.62907959,97.34424123)
\curveto(549.60907947,97.28423762)(549.59407949,97.21923768)(549.58407959,97.14924123)
\curveto(549.5840795,97.08923781)(549.5790795,97.02423788)(549.56907959,96.95424123)
\curveto(549.51907956,96.78423812)(549.46907961,96.62423828)(549.41907959,96.47424123)
\curveto(549.36907971,96.32423858)(549.30407978,96.17923872)(549.22407959,96.03924123)
\curveto(549.1840799,95.98923891)(549.15407993,95.93423897)(549.13407959,95.87424123)
\curveto(549.10407998,95.82423908)(549.06908001,95.77423913)(549.02907959,95.72424123)
\curveto(548.84908023,95.48423942)(548.62908045,95.28423962)(548.36907959,95.12424123)
\curveto(548.10908097,94.96423994)(547.82408126,94.82424008)(547.51407959,94.70424123)
\curveto(547.37408171,94.64424026)(547.23408185,94.5992403)(547.09407959,94.56924123)
\curveto(546.94408214,94.53924036)(546.78908229,94.5042404)(546.62907959,94.46424123)
\curveto(546.51908256,94.44424046)(546.40908267,94.42924047)(546.29907959,94.41924123)
\curveto(546.18908289,94.40924049)(546.079083,94.39424051)(545.96907959,94.37424123)
\curveto(545.92908315,94.36424054)(545.88908319,94.35924054)(545.84907959,94.35924123)
\curveto(545.80908327,94.36924053)(545.76908331,94.36924053)(545.72907959,94.35924123)
\curveto(545.6790834,94.34924055)(545.62908345,94.34424056)(545.57907959,94.34424123)
\lineto(545.41407959,94.34424123)
\curveto(545.36408372,94.32424058)(545.31408377,94.31924058)(545.26407959,94.32924123)
\curveto(545.20408388,94.33924056)(545.14908393,94.33924056)(545.09907959,94.32924123)
\curveto(545.05908402,94.31924058)(545.01408407,94.31924058)(544.96407959,94.32924123)
\curveto(544.91408417,94.33924056)(544.86408422,94.33424057)(544.81407959,94.31424123)
\curveto(544.74408434,94.29424061)(544.66908441,94.28924061)(544.58907959,94.29924123)
\curveto(544.49908458,94.30924059)(544.41408467,94.31424059)(544.33407959,94.31424123)
\curveto(544.24408484,94.31424059)(544.14408494,94.30924059)(544.03407959,94.29924123)
\curveto(543.91408517,94.28924061)(543.81408527,94.29424061)(543.73407959,94.31424123)
\lineto(543.44907959,94.31424123)
\lineto(542.81907959,94.35924123)
\curveto(542.71908636,94.36924053)(542.62408646,94.37924052)(542.53407959,94.38924123)
\lineto(542.23407959,94.41924123)
\curveto(542.1840869,94.43924046)(542.13408695,94.44424046)(542.08407959,94.43424123)
\curveto(542.02408706,94.43424047)(541.96908711,94.44424046)(541.91907959,94.46424123)
\curveto(541.74908733,94.51424039)(541.5840875,94.55424035)(541.42407959,94.58424123)
\curveto(541.25408783,94.61424029)(541.09408799,94.66424024)(540.94407959,94.73424123)
\curveto(540.4840886,94.92423998)(540.10908897,95.14423976)(539.81907959,95.39424123)
\curveto(539.52908955,95.65423925)(539.2840898,96.01423889)(539.08407959,96.47424123)
\curveto(539.03409005,96.6042383)(538.99909008,96.73423817)(538.97907959,96.86424123)
\curveto(538.95909012,97.0042379)(538.93409015,97.14423776)(538.90407959,97.28424123)
\curveto(538.89409019,97.35423755)(538.88909019,97.41923748)(538.88907959,97.47924123)
\curveto(538.88909019,97.53923736)(538.8840902,97.6042373)(538.87407959,97.67424123)
\curveto(538.85409023,98.5042364)(539.00409008,99.17423573)(539.32407959,99.68424123)
\curveto(539.63408945,100.19423471)(540.07408901,100.57423433)(540.64407959,100.82424123)
\curveto(540.76408832,100.87423403)(540.88908819,100.91923398)(541.01907959,100.95924123)
\curveto(541.14908793,100.9992339)(541.2840878,101.04423386)(541.42407959,101.09424123)
\curveto(541.50408758,101.11423379)(541.58908749,101.12923377)(541.67907959,101.13924123)
\lineto(541.91907959,101.19924123)
\curveto(542.02908705,101.22923367)(542.13908694,101.24423366)(542.24907959,101.24424123)
\curveto(542.35908672,101.25423365)(542.46908661,101.26923363)(542.57907959,101.28924123)
\curveto(542.62908645,101.30923359)(542.67408641,101.31423359)(542.71407959,101.30424123)
\curveto(542.75408633,101.3042336)(542.79408629,101.30923359)(542.83407959,101.31924123)
\curveto(542.8840862,101.32923357)(542.93908614,101.32923357)(542.99907959,101.31924123)
\curveto(543.04908603,101.31923358)(543.09908598,101.32423358)(543.14907959,101.33424123)
\lineto(543.28407959,101.33424123)
\curveto(543.34408574,101.35423355)(543.41408567,101.35423355)(543.49407959,101.33424123)
\curveto(543.56408552,101.32423358)(543.62908545,101.32923357)(543.68907959,101.34924123)
\curveto(543.71908536,101.35923354)(543.75908532,101.36423354)(543.80907959,101.36424123)
\lineto(543.92907959,101.36424123)
\lineto(544.39407959,101.36424123)
\moveto(546.71907959,99.81924123)
\curveto(546.39908268,99.91923498)(546.03408305,99.97923492)(545.62407959,99.99924123)
\curveto(545.21408387,100.01923488)(544.80408428,100.02923487)(544.39407959,100.02924123)
\curveto(543.96408512,100.02923487)(543.54408554,100.01923488)(543.13407959,99.99924123)
\curveto(542.72408636,99.97923492)(542.33908674,99.93423497)(541.97907959,99.86424123)
\curveto(541.61908746,99.79423511)(541.29908778,99.68423522)(541.01907959,99.53424123)
\curveto(540.72908835,99.39423551)(540.49408859,99.1992357)(540.31407959,98.94924123)
\curveto(540.20408888,98.78923611)(540.12408896,98.60923629)(540.07407959,98.40924123)
\curveto(540.01408907,98.20923669)(539.9840891,97.96423694)(539.98407959,97.67424123)
\curveto(540.00408908,97.65423725)(540.01408907,97.61923728)(540.01407959,97.56924123)
\curveto(540.00408908,97.51923738)(540.00408908,97.47923742)(540.01407959,97.44924123)
\curveto(540.03408905,97.36923753)(540.05408903,97.29423761)(540.07407959,97.22424123)
\curveto(540.084089,97.16423774)(540.10408898,97.0992378)(540.13407959,97.02924123)
\curveto(540.25408883,96.75923814)(540.42408866,96.53923836)(540.64407959,96.36924123)
\curveto(540.85408823,96.20923869)(541.09908798,96.07423883)(541.37907959,95.96424123)
\curveto(541.48908759,95.91423899)(541.60908747,95.87423903)(541.73907959,95.84424123)
\curveto(541.85908722,95.82423908)(541.9840871,95.7992391)(542.11407959,95.76924123)
\curveto(542.16408692,95.74923915)(542.21908686,95.73923916)(542.27907959,95.73924123)
\curveto(542.32908675,95.73923916)(542.3790867,95.73423917)(542.42907959,95.72424123)
\curveto(542.51908656,95.71423919)(542.61408647,95.7042392)(542.71407959,95.69424123)
\curveto(542.80408628,95.68423922)(542.89908618,95.67423923)(542.99907959,95.66424123)
\curveto(543.079086,95.66423924)(543.16408592,95.65923924)(543.25407959,95.64924123)
\lineto(543.49407959,95.64924123)
\lineto(543.67407959,95.64924123)
\curveto(543.70408538,95.63923926)(543.73908534,95.63423927)(543.77907959,95.63424123)
\lineto(543.91407959,95.63424123)
\lineto(544.36407959,95.63424123)
\curveto(544.44408464,95.63423927)(544.52908455,95.62923927)(544.61907959,95.61924123)
\curveto(544.69908438,95.61923928)(544.77408431,95.62923927)(544.84407959,95.64924123)
\lineto(545.11407959,95.64924123)
\curveto(545.13408395,95.64923925)(545.16408392,95.64423926)(545.20407959,95.63424123)
\curveto(545.23408385,95.63423927)(545.25908382,95.63923926)(545.27907959,95.64924123)
\curveto(545.3790837,95.65923924)(545.4790836,95.66423924)(545.57907959,95.66424123)
\curveto(545.66908341,95.67423923)(545.76908331,95.68423922)(545.87907959,95.69424123)
\curveto(545.99908308,95.72423918)(546.12408296,95.73923916)(546.25407959,95.73924123)
\curveto(546.37408271,95.74923915)(546.48908259,95.77423913)(546.59907959,95.81424123)
\curveto(546.89908218,95.89423901)(547.16408192,95.97923892)(547.39407959,96.06924123)
\curveto(547.62408146,96.16923873)(547.83908124,96.31423859)(548.03907959,96.50424123)
\curveto(548.23908084,96.71423819)(548.38908069,96.97923792)(548.48907959,97.29924123)
\curveto(548.50908057,97.33923756)(548.51908056,97.37423753)(548.51907959,97.40424123)
\curveto(548.50908057,97.44423746)(548.51408057,97.48923741)(548.53407959,97.53924123)
\curveto(548.54408054,97.57923732)(548.55408053,97.64923725)(548.56407959,97.74924123)
\curveto(548.57408051,97.85923704)(548.56908051,97.94423696)(548.54907959,98.00424123)
\curveto(548.52908055,98.07423683)(548.51908056,98.14423676)(548.51907959,98.21424123)
\curveto(548.50908057,98.28423662)(548.49408059,98.34923655)(548.47407959,98.40924123)
\curveto(548.41408067,98.60923629)(548.32908075,98.78923611)(548.21907959,98.94924123)
\curveto(548.19908088,98.97923592)(548.1790809,99.0042359)(548.15907959,99.02424123)
\lineto(548.09907959,99.08424123)
\curveto(548.079081,99.12423578)(548.03908104,99.17423573)(547.97907959,99.23424123)
\curveto(547.83908124,99.33423557)(547.70908137,99.41923548)(547.58907959,99.48924123)
\curveto(547.46908161,99.55923534)(547.32408176,99.62923527)(547.15407959,99.69924123)
\curveto(547.084082,99.72923517)(547.01408207,99.74923515)(546.94407959,99.75924123)
\curveto(546.87408221,99.77923512)(546.79908228,99.7992351)(546.71907959,99.81924123)
}
}
{
\newrgbcolor{curcolor}{0 0 0}
\pscustom[linestyle=none,fillstyle=solid,fillcolor=curcolor]
{
\newpath
\moveto(538.87407959,106.77385061)
\curveto(538.87409021,106.87384575)(538.8840902,106.96884566)(538.90407959,107.05885061)
\curveto(538.91409017,107.14884548)(538.94409014,107.21384541)(538.99407959,107.25385061)
\curveto(539.07409001,107.31384531)(539.1790899,107.34384528)(539.30907959,107.34385061)
\lineto(539.69907959,107.34385061)
\lineto(541.19907959,107.34385061)
\lineto(547.58907959,107.34385061)
\lineto(548.75907959,107.34385061)
\lineto(549.07407959,107.34385061)
\curveto(549.17407991,107.35384527)(549.25407983,107.33884529)(549.31407959,107.29885061)
\curveto(549.39407969,107.24884538)(549.44407964,107.17384545)(549.46407959,107.07385061)
\curveto(549.47407961,106.98384564)(549.4790796,106.87384575)(549.47907959,106.74385061)
\lineto(549.47907959,106.51885061)
\curveto(549.45907962,106.43884619)(549.44407964,106.36884626)(549.43407959,106.30885061)
\curveto(549.41407967,106.24884638)(549.37407971,106.19884643)(549.31407959,106.15885061)
\curveto(549.25407983,106.11884651)(549.1790799,106.09884653)(549.08907959,106.09885061)
\lineto(548.78907959,106.09885061)
\lineto(547.69407959,106.09885061)
\lineto(542.35407959,106.09885061)
\curveto(542.26408682,106.07884655)(542.18908689,106.06384656)(542.12907959,106.05385061)
\curveto(542.05908702,106.05384657)(541.99908708,106.0238466)(541.94907959,105.96385061)
\curveto(541.89908718,105.89384673)(541.87408721,105.80384682)(541.87407959,105.69385061)
\curveto(541.86408722,105.59384703)(541.85908722,105.48384714)(541.85907959,105.36385061)
\lineto(541.85907959,104.22385061)
\lineto(541.85907959,103.72885061)
\curveto(541.84908723,103.56884906)(541.78908729,103.45884917)(541.67907959,103.39885061)
\curveto(541.64908743,103.37884925)(541.61908746,103.36884926)(541.58907959,103.36885061)
\curveto(541.54908753,103.36884926)(541.50408758,103.36384926)(541.45407959,103.35385061)
\curveto(541.33408775,103.33384929)(541.22408786,103.33884929)(541.12407959,103.36885061)
\curveto(541.02408806,103.40884922)(540.95408813,103.46384916)(540.91407959,103.53385061)
\curveto(540.86408822,103.61384901)(540.83908824,103.73384889)(540.83907959,103.89385061)
\curveto(540.83908824,104.05384857)(540.82408826,104.18884844)(540.79407959,104.29885061)
\curveto(540.7840883,104.34884828)(540.7790883,104.40384822)(540.77907959,104.46385061)
\curveto(540.76908831,104.5238481)(540.75408833,104.58384804)(540.73407959,104.64385061)
\curveto(540.6840884,104.79384783)(540.63408845,104.93884769)(540.58407959,105.07885061)
\curveto(540.52408856,105.21884741)(540.45408863,105.35384727)(540.37407959,105.48385061)
\curveto(540.2840888,105.623847)(540.1790889,105.74384688)(540.05907959,105.84385061)
\curveto(539.93908914,105.94384668)(539.80908927,106.03884659)(539.66907959,106.12885061)
\curveto(539.56908951,106.18884644)(539.45908962,106.23384639)(539.33907959,106.26385061)
\curveto(539.21908986,106.30384632)(539.11408997,106.35384627)(539.02407959,106.41385061)
\curveto(538.96409012,106.46384616)(538.92409016,106.53384609)(538.90407959,106.62385061)
\curveto(538.89409019,106.64384598)(538.88909019,106.66884596)(538.88907959,106.69885061)
\curveto(538.88909019,106.7288459)(538.8840902,106.75384587)(538.87407959,106.77385061)
}
}
{
\newrgbcolor{curcolor}{0 0 0}
\pscustom[linestyle=none,fillstyle=solid,fillcolor=curcolor]
{
\newpath
\moveto(538.87407959,115.12345998)
\curveto(538.87409021,115.22345513)(538.8840902,115.31845503)(538.90407959,115.40845998)
\curveto(538.91409017,115.49845485)(538.94409014,115.56345479)(538.99407959,115.60345998)
\curveto(539.07409001,115.66345469)(539.1790899,115.69345466)(539.30907959,115.69345998)
\lineto(539.69907959,115.69345998)
\lineto(541.19907959,115.69345998)
\lineto(547.58907959,115.69345998)
\lineto(548.75907959,115.69345998)
\lineto(549.07407959,115.69345998)
\curveto(549.17407991,115.70345465)(549.25407983,115.68845466)(549.31407959,115.64845998)
\curveto(549.39407969,115.59845475)(549.44407964,115.52345483)(549.46407959,115.42345998)
\curveto(549.47407961,115.33345502)(549.4790796,115.22345513)(549.47907959,115.09345998)
\lineto(549.47907959,114.86845998)
\curveto(549.45907962,114.78845556)(549.44407964,114.71845563)(549.43407959,114.65845998)
\curveto(549.41407967,114.59845575)(549.37407971,114.5484558)(549.31407959,114.50845998)
\curveto(549.25407983,114.46845588)(549.1790799,114.4484559)(549.08907959,114.44845998)
\lineto(548.78907959,114.44845998)
\lineto(547.69407959,114.44845998)
\lineto(542.35407959,114.44845998)
\curveto(542.26408682,114.42845592)(542.18908689,114.41345594)(542.12907959,114.40345998)
\curveto(542.05908702,114.40345595)(541.99908708,114.37345598)(541.94907959,114.31345998)
\curveto(541.89908718,114.24345611)(541.87408721,114.1534562)(541.87407959,114.04345998)
\curveto(541.86408722,113.94345641)(541.85908722,113.83345652)(541.85907959,113.71345998)
\lineto(541.85907959,112.57345998)
\lineto(541.85907959,112.07845998)
\curveto(541.84908723,111.91845843)(541.78908729,111.80845854)(541.67907959,111.74845998)
\curveto(541.64908743,111.72845862)(541.61908746,111.71845863)(541.58907959,111.71845998)
\curveto(541.54908753,111.71845863)(541.50408758,111.71345864)(541.45407959,111.70345998)
\curveto(541.33408775,111.68345867)(541.22408786,111.68845866)(541.12407959,111.71845998)
\curveto(541.02408806,111.75845859)(540.95408813,111.81345854)(540.91407959,111.88345998)
\curveto(540.86408822,111.96345839)(540.83908824,112.08345827)(540.83907959,112.24345998)
\curveto(540.83908824,112.40345795)(540.82408826,112.53845781)(540.79407959,112.64845998)
\curveto(540.7840883,112.69845765)(540.7790883,112.7534576)(540.77907959,112.81345998)
\curveto(540.76908831,112.87345748)(540.75408833,112.93345742)(540.73407959,112.99345998)
\curveto(540.6840884,113.14345721)(540.63408845,113.28845706)(540.58407959,113.42845998)
\curveto(540.52408856,113.56845678)(540.45408863,113.70345665)(540.37407959,113.83345998)
\curveto(540.2840888,113.97345638)(540.1790889,114.09345626)(540.05907959,114.19345998)
\curveto(539.93908914,114.29345606)(539.80908927,114.38845596)(539.66907959,114.47845998)
\curveto(539.56908951,114.53845581)(539.45908962,114.58345577)(539.33907959,114.61345998)
\curveto(539.21908986,114.6534557)(539.11408997,114.70345565)(539.02407959,114.76345998)
\curveto(538.96409012,114.81345554)(538.92409016,114.88345547)(538.90407959,114.97345998)
\curveto(538.89409019,114.99345536)(538.88909019,115.01845533)(538.88907959,115.04845998)
\curveto(538.88909019,115.07845527)(538.8840902,115.10345525)(538.87407959,115.12345998)
}
}
{
\newrgbcolor{curcolor}{0 0 0}
\pscustom[linestyle=none,fillstyle=solid,fillcolor=curcolor]
{
\newpath
\moveto(560.74536865,29.18119436)
\lineto(560.74536865,30.09619436)
\curveto(560.74537935,30.19619171)(560.74537935,30.29119161)(560.74536865,30.38119436)
\curveto(560.74537935,30.47119143)(560.76537933,30.54619136)(560.80536865,30.60619436)
\curveto(560.86537923,30.69619121)(560.94537915,30.75619115)(561.04536865,30.78619436)
\curveto(561.14537895,30.82619108)(561.25037884,30.87119103)(561.36036865,30.92119436)
\curveto(561.55037854,31.0011909)(561.74037835,31.07119083)(561.93036865,31.13119436)
\curveto(562.12037797,31.2011907)(562.31037778,31.27619063)(562.50036865,31.35619436)
\curveto(562.68037741,31.42619048)(562.86537723,31.49119041)(563.05536865,31.55119436)
\curveto(563.23537686,31.61119029)(563.41537668,31.68119022)(563.59536865,31.76119436)
\curveto(563.73537636,31.82119008)(563.88037621,31.87619003)(564.03036865,31.92619436)
\curveto(564.18037591,31.97618993)(564.32537577,32.03118987)(564.46536865,32.09119436)
\curveto(564.91537518,32.27118963)(565.37037472,32.44118946)(565.83036865,32.60119436)
\curveto(566.28037381,32.76118914)(566.73037336,32.93118897)(567.18036865,33.11119436)
\curveto(567.23037286,33.13118877)(567.28037281,33.14618876)(567.33036865,33.15619436)
\lineto(567.48036865,33.21619436)
\curveto(567.70037239,33.3061886)(567.92537217,33.39118851)(568.15536865,33.47119436)
\curveto(568.37537172,33.55118835)(568.5953715,33.63618827)(568.81536865,33.72619436)
\curveto(568.90537119,33.76618814)(569.01537108,33.8061881)(569.14536865,33.84619436)
\curveto(569.26537083,33.88618802)(569.33537076,33.95118795)(569.35536865,34.04119436)
\curveto(569.36537073,34.08118782)(569.36537073,34.11118779)(569.35536865,34.13119436)
\lineto(569.29536865,34.19119436)
\curveto(569.24537085,34.24118766)(569.1903709,34.27618763)(569.13036865,34.29619436)
\curveto(569.07037102,34.32618758)(569.00537109,34.35618755)(568.93536865,34.38619436)
\lineto(568.30536865,34.62619436)
\curveto(568.08537201,34.7061872)(567.87037222,34.78618712)(567.66036865,34.86619436)
\lineto(567.51036865,34.92619436)
\lineto(567.33036865,34.98619436)
\curveto(567.14037295,35.06618684)(566.95037314,35.13618677)(566.76036865,35.19619436)
\curveto(566.56037353,35.26618664)(566.36037373,35.34118656)(566.16036865,35.42119436)
\curveto(565.58037451,35.66118624)(564.9953751,35.88118602)(564.40536865,36.08119436)
\curveto(563.81537628,36.29118561)(563.23037686,36.51618539)(562.65036865,36.75619436)
\curveto(562.45037764,36.83618507)(562.24537785,36.91118499)(562.03536865,36.98119436)
\curveto(561.82537827,37.06118484)(561.62037847,37.14118476)(561.42036865,37.22119436)
\curveto(561.34037875,37.26118464)(561.24037885,37.29618461)(561.12036865,37.32619436)
\curveto(561.00037909,37.36618454)(560.91537918,37.42118448)(560.86536865,37.49119436)
\curveto(560.82537927,37.55118435)(560.7953793,37.62618428)(560.77536865,37.71619436)
\curveto(560.75537934,37.81618409)(560.74537935,37.92618398)(560.74536865,38.04619436)
\curveto(560.73537936,38.16618374)(560.73537936,38.28618362)(560.74536865,38.40619436)
\curveto(560.74537935,38.52618338)(560.74537935,38.63618327)(560.74536865,38.73619436)
\curveto(560.74537935,38.82618308)(560.74537935,38.91618299)(560.74536865,39.00619436)
\curveto(560.74537935,39.1061828)(560.76537933,39.18118272)(560.80536865,39.23119436)
\curveto(560.85537924,39.32118258)(560.94537915,39.37118253)(561.07536865,39.38119436)
\curveto(561.20537889,39.39118251)(561.34537875,39.39618251)(561.49536865,39.39619436)
\lineto(563.14536865,39.39619436)
\lineto(569.41536865,39.39619436)
\lineto(570.67536865,39.39619436)
\curveto(570.78536931,39.39618251)(570.8953692,39.39618251)(571.00536865,39.39619436)
\curveto(571.11536898,39.4061825)(571.20036889,39.38618252)(571.26036865,39.33619436)
\curveto(571.32036877,39.3061826)(571.36036873,39.26118264)(571.38036865,39.20119436)
\curveto(571.3903687,39.14118276)(571.40536869,39.07118283)(571.42536865,38.99119436)
\lineto(571.42536865,38.75119436)
\lineto(571.42536865,38.39119436)
\curveto(571.41536868,38.28118362)(571.37036872,38.2011837)(571.29036865,38.15119436)
\curveto(571.26036883,38.13118377)(571.23036886,38.11618379)(571.20036865,38.10619436)
\curveto(571.16036893,38.1061838)(571.11536898,38.09618381)(571.06536865,38.07619436)
\lineto(570.90036865,38.07619436)
\curveto(570.84036925,38.06618384)(570.77036932,38.06118384)(570.69036865,38.06119436)
\curveto(570.61036948,38.07118383)(570.53536956,38.07618383)(570.46536865,38.07619436)
\lineto(569.62536865,38.07619436)
\lineto(565.20036865,38.07619436)
\curveto(564.95037514,38.07618383)(564.70037539,38.07618383)(564.45036865,38.07619436)
\curveto(564.1903759,38.07618383)(563.94037615,38.07118383)(563.70036865,38.06119436)
\curveto(563.60037649,38.06118384)(563.4903766,38.05618385)(563.37036865,38.04619436)
\curveto(563.25037684,38.03618387)(563.1903769,37.98118392)(563.19036865,37.88119436)
\lineto(563.20536865,37.88119436)
\curveto(563.22537687,37.81118409)(563.2903768,37.75118415)(563.40036865,37.70119436)
\curveto(563.51037658,37.66118424)(563.60537649,37.62618428)(563.68536865,37.59619436)
\curveto(563.85537624,37.52618438)(564.03037606,37.46118444)(564.21036865,37.40119436)
\curveto(564.38037571,37.34118456)(564.55037554,37.27118463)(564.72036865,37.19119436)
\curveto(564.77037532,37.17118473)(564.81537528,37.15618475)(564.85536865,37.14619436)
\curveto(564.8953752,37.13618477)(564.94037515,37.12118478)(564.99036865,37.10119436)
\curveto(565.17037492,37.02118488)(565.35537474,36.95118495)(565.54536865,36.89119436)
\curveto(565.72537437,36.84118506)(565.90537419,36.77618513)(566.08536865,36.69619436)
\curveto(566.23537386,36.62618528)(566.3903737,36.56618534)(566.55036865,36.51619436)
\curveto(566.70037339,36.46618544)(566.85037324,36.41118549)(567.00036865,36.35119436)
\curveto(567.47037262,36.15118575)(567.94537215,35.97118593)(568.42536865,35.81119436)
\curveto(568.8953712,35.65118625)(569.36037073,35.47618643)(569.82036865,35.28619436)
\curveto(570.00037009,35.2061867)(570.18036991,35.13618677)(570.36036865,35.07619436)
\curveto(570.54036955,35.01618689)(570.72036937,34.95118695)(570.90036865,34.88119436)
\curveto(571.01036908,34.83118707)(571.11536898,34.78118712)(571.21536865,34.73119436)
\curveto(571.30536879,34.69118721)(571.37036872,34.6061873)(571.41036865,34.47619436)
\curveto(571.42036867,34.45618745)(571.42536867,34.43118747)(571.42536865,34.40119436)
\curveto(571.41536868,34.38118752)(571.41536868,34.35618755)(571.42536865,34.32619436)
\curveto(571.43536866,34.29618761)(571.44036865,34.26118764)(571.44036865,34.22119436)
\curveto(571.43036866,34.18118772)(571.42536867,34.14118776)(571.42536865,34.10119436)
\lineto(571.42536865,33.80119436)
\curveto(571.42536867,33.7011882)(571.40036869,33.62118828)(571.35036865,33.56119436)
\curveto(571.30036879,33.48118842)(571.23036886,33.42118848)(571.14036865,33.38119436)
\curveto(571.04036905,33.35118855)(570.94036915,33.31118859)(570.84036865,33.26119436)
\curveto(570.64036945,33.18118872)(570.43536966,33.1011888)(570.22536865,33.02119436)
\curveto(570.00537009,32.95118895)(569.7953703,32.87618903)(569.59536865,32.79619436)
\curveto(569.41537068,32.71618919)(569.23537086,32.64618926)(569.05536865,32.58619436)
\curveto(568.86537123,32.53618937)(568.68037141,32.47118943)(568.50036865,32.39119436)
\curveto(567.94037215,32.16118974)(567.37537272,31.94618996)(566.80536865,31.74619436)
\curveto(566.23537386,31.54619036)(565.67037442,31.33119057)(565.11036865,31.10119436)
\lineto(564.48036865,30.86119436)
\curveto(564.26037583,30.79119111)(564.05037604,30.71619119)(563.85036865,30.63619436)
\curveto(563.74037635,30.58619132)(563.63537646,30.54119136)(563.53536865,30.50119436)
\curveto(563.42537667,30.47119143)(563.33037676,30.42119148)(563.25036865,30.35119436)
\curveto(563.23037686,30.34119156)(563.22037687,30.33119157)(563.22036865,30.32119436)
\lineto(563.19036865,30.29119436)
\lineto(563.19036865,30.21619436)
\lineto(563.22036865,30.18619436)
\curveto(563.22037687,30.17619173)(563.22537687,30.16619174)(563.23536865,30.15619436)
\curveto(563.28537681,30.13619177)(563.34037675,30.12619178)(563.40036865,30.12619436)
\curveto(563.46037663,30.12619178)(563.52037657,30.11619179)(563.58036865,30.09619436)
\lineto(563.74536865,30.09619436)
\curveto(563.80537629,30.07619183)(563.87037622,30.07119183)(563.94036865,30.08119436)
\curveto(564.01037608,30.09119181)(564.08037601,30.09619181)(564.15036865,30.09619436)
\lineto(564.96036865,30.09619436)
\lineto(569.52036865,30.09619436)
\lineto(570.70536865,30.09619436)
\curveto(570.81536928,30.09619181)(570.92536917,30.09119181)(571.03536865,30.08119436)
\curveto(571.14536895,30.08119182)(571.23036886,30.05619185)(571.29036865,30.00619436)
\curveto(571.37036872,29.95619195)(571.41536868,29.86619204)(571.42536865,29.73619436)
\lineto(571.42536865,29.34619436)
\lineto(571.42536865,29.15119436)
\curveto(571.42536867,29.1011928)(571.41536868,29.05119285)(571.39536865,29.00119436)
\curveto(571.35536874,28.87119303)(571.27036882,28.79619311)(571.14036865,28.77619436)
\curveto(571.01036908,28.76619314)(570.86036923,28.76119314)(570.69036865,28.76119436)
\lineto(568.95036865,28.76119436)
\lineto(562.95036865,28.76119436)
\lineto(561.54036865,28.76119436)
\curveto(561.43037866,28.76119314)(561.31537878,28.75619315)(561.19536865,28.74619436)
\curveto(561.07537902,28.74619316)(560.98037911,28.77119313)(560.91036865,28.82119436)
\curveto(560.85037924,28.86119304)(560.80037929,28.93619297)(560.76036865,29.04619436)
\curveto(560.75037934,29.06619284)(560.75037934,29.08619282)(560.76036865,29.10619436)
\curveto(560.76037933,29.13619277)(560.75537934,29.16119274)(560.74536865,29.18119436)
}
}
{
\newrgbcolor{curcolor}{0 0 0}
\pscustom[linestyle=none,fillstyle=solid,fillcolor=curcolor]
{
\newpath
\moveto(570.87036865,48.38330373)
\curveto(571.03036906,48.4132959)(571.16536893,48.39829592)(571.27536865,48.33830373)
\curveto(571.37536872,48.27829604)(571.45036864,48.19829612)(571.50036865,48.09830373)
\curveto(571.52036857,48.04829627)(571.53036856,47.99329632)(571.53036865,47.93330373)
\curveto(571.53036856,47.88329643)(571.54036855,47.82829649)(571.56036865,47.76830373)
\curveto(571.61036848,47.54829677)(571.5953685,47.32829699)(571.51536865,47.10830373)
\curveto(571.44536865,46.89829742)(571.35536874,46.75329756)(571.24536865,46.67330373)
\curveto(571.17536892,46.62329769)(571.095369,46.57829774)(571.00536865,46.53830373)
\curveto(570.90536919,46.49829782)(570.82536927,46.44829787)(570.76536865,46.38830373)
\curveto(570.74536935,46.36829795)(570.72536937,46.34329797)(570.70536865,46.31330373)
\curveto(570.68536941,46.29329802)(570.68036941,46.26329805)(570.69036865,46.22330373)
\curveto(570.72036937,46.1132982)(570.77536932,46.00829831)(570.85536865,45.90830373)
\curveto(570.93536916,45.8182985)(571.00536909,45.72829859)(571.06536865,45.63830373)
\curveto(571.14536895,45.50829881)(571.22036887,45.36829895)(571.29036865,45.21830373)
\curveto(571.35036874,45.06829925)(571.40536869,44.90829941)(571.45536865,44.73830373)
\curveto(571.48536861,44.63829968)(571.50536859,44.52829979)(571.51536865,44.40830373)
\curveto(571.52536857,44.29830002)(571.54036855,44.18830013)(571.56036865,44.07830373)
\curveto(571.57036852,44.02830029)(571.57536852,43.98330033)(571.57536865,43.94330373)
\lineto(571.57536865,43.83830373)
\curveto(571.5953685,43.72830059)(571.5953685,43.62330069)(571.57536865,43.52330373)
\lineto(571.57536865,43.38830373)
\curveto(571.56536853,43.33830098)(571.56036853,43.28830103)(571.56036865,43.23830373)
\curveto(571.56036853,43.18830113)(571.55036854,43.14330117)(571.53036865,43.10330373)
\curveto(571.52036857,43.06330125)(571.51536858,43.02830129)(571.51536865,42.99830373)
\curveto(571.52536857,42.97830134)(571.52536857,42.95330136)(571.51536865,42.92330373)
\lineto(571.45536865,42.68330373)
\curveto(571.44536865,42.60330171)(571.42536867,42.52830179)(571.39536865,42.45830373)
\curveto(571.26536883,42.15830216)(571.12036897,41.9133024)(570.96036865,41.72330373)
\curveto(570.7903693,41.54330277)(570.55536954,41.39330292)(570.25536865,41.27330373)
\curveto(570.03537006,41.18330313)(569.77037032,41.13830318)(569.46036865,41.13830373)
\lineto(569.14536865,41.13830373)
\curveto(569.095371,41.14830317)(569.04537105,41.15330316)(568.99536865,41.15330373)
\lineto(568.81536865,41.18330373)
\lineto(568.48536865,41.30330373)
\curveto(568.37537172,41.34330297)(568.27537182,41.39330292)(568.18536865,41.45330373)
\curveto(567.8953722,41.63330268)(567.68037241,41.87830244)(567.54036865,42.18830373)
\curveto(567.40037269,42.49830182)(567.27537282,42.83830148)(567.16536865,43.20830373)
\curveto(567.12537297,43.34830097)(567.095373,43.49330082)(567.07536865,43.64330373)
\curveto(567.05537304,43.79330052)(567.03037306,43.94330037)(567.00036865,44.09330373)
\curveto(566.98037311,44.16330015)(566.97037312,44.22830009)(566.97036865,44.28830373)
\curveto(566.97037312,44.35829996)(566.96037313,44.43329988)(566.94036865,44.51330373)
\curveto(566.92037317,44.58329973)(566.91037318,44.65329966)(566.91036865,44.72330373)
\curveto(566.90037319,44.79329952)(566.88537321,44.86829945)(566.86536865,44.94830373)
\curveto(566.80537329,45.19829912)(566.75537334,45.43329888)(566.71536865,45.65330373)
\curveto(566.66537343,45.87329844)(566.55037354,46.04829827)(566.37036865,46.17830373)
\curveto(566.2903738,46.23829808)(566.1903739,46.28829803)(566.07036865,46.32830373)
\curveto(565.94037415,46.36829795)(565.80037429,46.36829795)(565.65036865,46.32830373)
\curveto(565.41037468,46.26829805)(565.22037487,46.17829814)(565.08036865,46.05830373)
\curveto(564.94037515,45.94829837)(564.83037526,45.78829853)(564.75036865,45.57830373)
\curveto(564.70037539,45.45829886)(564.66537543,45.313299)(564.64536865,45.14330373)
\curveto(564.62537547,44.98329933)(564.61537548,44.8132995)(564.61536865,44.63330373)
\curveto(564.61537548,44.45329986)(564.62537547,44.27830004)(564.64536865,44.10830373)
\curveto(564.66537543,43.93830038)(564.6953754,43.79330052)(564.73536865,43.67330373)
\curveto(564.7953753,43.50330081)(564.88037521,43.33830098)(564.99036865,43.17830373)
\curveto(565.05037504,43.09830122)(565.13037496,43.02330129)(565.23036865,42.95330373)
\curveto(565.32037477,42.89330142)(565.42037467,42.83830148)(565.53036865,42.78830373)
\curveto(565.61037448,42.75830156)(565.6953744,42.72830159)(565.78536865,42.69830373)
\curveto(565.87537422,42.67830164)(565.94537415,42.63330168)(565.99536865,42.56330373)
\curveto(566.02537407,42.52330179)(566.05037404,42.45330186)(566.07036865,42.35330373)
\curveto(566.08037401,42.26330205)(566.08537401,42.16830215)(566.08536865,42.06830373)
\curveto(566.08537401,41.96830235)(566.08037401,41.86830245)(566.07036865,41.76830373)
\curveto(566.05037404,41.67830264)(566.02537407,41.6133027)(565.99536865,41.57330373)
\curveto(565.96537413,41.53330278)(565.91537418,41.50330281)(565.84536865,41.48330373)
\curveto(565.77537432,41.46330285)(565.70037439,41.46330285)(565.62036865,41.48330373)
\curveto(565.4903746,41.5133028)(565.37037472,41.54330277)(565.26036865,41.57330373)
\curveto(565.14037495,41.6133027)(565.02537507,41.65830266)(564.91536865,41.70830373)
\curveto(564.56537553,41.89830242)(564.2953758,42.13830218)(564.10536865,42.42830373)
\curveto(563.90537619,42.7183016)(563.74537635,43.07830124)(563.62536865,43.50830373)
\curveto(563.60537649,43.60830071)(563.5903765,43.70830061)(563.58036865,43.80830373)
\curveto(563.57037652,43.9183004)(563.55537654,44.02830029)(563.53536865,44.13830373)
\curveto(563.52537657,44.17830014)(563.52537657,44.24330007)(563.53536865,44.33330373)
\curveto(563.53537656,44.42329989)(563.52537657,44.47829984)(563.50536865,44.49830373)
\curveto(563.4953766,45.19829912)(563.57537652,45.80829851)(563.74536865,46.32830373)
\curveto(563.91537618,46.84829747)(564.24037585,47.2132971)(564.72036865,47.42330373)
\curveto(564.92037517,47.5132968)(565.15537494,47.56329675)(565.42536865,47.57330373)
\curveto(565.68537441,47.59329672)(565.96037413,47.60329671)(566.25036865,47.60330373)
\lineto(569.56536865,47.60330373)
\curveto(569.70537039,47.60329671)(569.84037025,47.60829671)(569.97036865,47.61830373)
\curveto(570.10036999,47.62829669)(570.20536989,47.65829666)(570.28536865,47.70830373)
\curveto(570.35536974,47.75829656)(570.40536969,47.82329649)(570.43536865,47.90330373)
\curveto(570.47536962,47.99329632)(570.50536959,48.07829624)(570.52536865,48.15830373)
\curveto(570.53536956,48.23829608)(570.58036951,48.29829602)(570.66036865,48.33830373)
\curveto(570.6903694,48.35829596)(570.72036937,48.36829595)(570.75036865,48.36830373)
\curveto(570.78036931,48.36829595)(570.82036927,48.37329594)(570.87036865,48.38330373)
\moveto(569.20536865,46.23830373)
\curveto(569.06537103,46.29829802)(568.90537119,46.32829799)(568.72536865,46.32830373)
\curveto(568.53537156,46.33829798)(568.34037175,46.34329797)(568.14036865,46.34330373)
\curveto(568.03037206,46.34329797)(567.93037216,46.33829798)(567.84036865,46.32830373)
\curveto(567.75037234,46.318298)(567.68037241,46.27829804)(567.63036865,46.20830373)
\curveto(567.61037248,46.17829814)(567.60037249,46.10829821)(567.60036865,45.99830373)
\curveto(567.62037247,45.97829834)(567.63037246,45.94329837)(567.63036865,45.89330373)
\curveto(567.63037246,45.84329847)(567.64037245,45.79829852)(567.66036865,45.75830373)
\curveto(567.68037241,45.67829864)(567.70037239,45.58829873)(567.72036865,45.48830373)
\lineto(567.78036865,45.18830373)
\curveto(567.78037231,45.15829916)(567.78537231,45.12329919)(567.79536865,45.08330373)
\lineto(567.79536865,44.97830373)
\curveto(567.83537226,44.82829949)(567.86037223,44.66329965)(567.87036865,44.48330373)
\curveto(567.87037222,44.3133)(567.8903722,44.15330016)(567.93036865,44.00330373)
\curveto(567.95037214,43.92330039)(567.97037212,43.84830047)(567.99036865,43.77830373)
\curveto(568.00037209,43.7183006)(568.01537208,43.64830067)(568.03536865,43.56830373)
\curveto(568.08537201,43.40830091)(568.15037194,43.25830106)(568.23036865,43.11830373)
\curveto(568.30037179,42.97830134)(568.3903717,42.85830146)(568.50036865,42.75830373)
\curveto(568.61037148,42.65830166)(568.74537135,42.58330173)(568.90536865,42.53330373)
\curveto(569.05537104,42.48330183)(569.24037085,42.46330185)(569.46036865,42.47330373)
\curveto(569.56037053,42.47330184)(569.65537044,42.48830183)(569.74536865,42.51830373)
\curveto(569.82537027,42.55830176)(569.90037019,42.60330171)(569.97036865,42.65330373)
\curveto(570.08037001,42.73330158)(570.17536992,42.83830148)(570.25536865,42.96830373)
\curveto(570.32536977,43.09830122)(570.38536971,43.23830108)(570.43536865,43.38830373)
\curveto(570.44536965,43.43830088)(570.45036964,43.48830083)(570.45036865,43.53830373)
\curveto(570.45036964,43.58830073)(570.45536964,43.63830068)(570.46536865,43.68830373)
\curveto(570.48536961,43.75830056)(570.50036959,43.84330047)(570.51036865,43.94330373)
\curveto(570.51036958,44.05330026)(570.50036959,44.14330017)(570.48036865,44.21330373)
\curveto(570.46036963,44.27330004)(570.45536964,44.33329998)(570.46536865,44.39330373)
\curveto(570.46536963,44.45329986)(570.45536964,44.5132998)(570.43536865,44.57330373)
\curveto(570.41536968,44.65329966)(570.40036969,44.72829959)(570.39036865,44.79830373)
\curveto(570.38036971,44.87829944)(570.36036973,44.95329936)(570.33036865,45.02330373)
\curveto(570.21036988,45.313299)(570.06537003,45.55829876)(569.89536865,45.75830373)
\curveto(569.72537037,45.96829835)(569.4953706,46.12829819)(569.20536865,46.23830373)
}
}
{
\newrgbcolor{curcolor}{0 0 0}
\pscustom[linestyle=none,fillstyle=solid,fillcolor=curcolor]
{
\newpath
\moveto(563.70036865,49.26994436)
\lineto(563.70036865,49.71994436)
\curveto(563.6903764,49.88994311)(563.71037638,50.01494298)(563.76036865,50.09494436)
\curveto(563.81037628,50.17494282)(563.87537622,50.22994277)(563.95536865,50.25994436)
\curveto(564.03537606,50.2999427)(564.12037597,50.33994266)(564.21036865,50.37994436)
\curveto(564.34037575,50.42994257)(564.47037562,50.47494252)(564.60036865,50.51494436)
\curveto(564.73037536,50.55494244)(564.86037523,50.5999424)(564.99036865,50.64994436)
\curveto(565.11037498,50.6999423)(565.23537486,50.74494225)(565.36536865,50.78494436)
\curveto(565.48537461,50.82494217)(565.60537449,50.86994213)(565.72536865,50.91994436)
\curveto(565.83537426,50.96994203)(565.95037414,51.00994199)(566.07036865,51.03994436)
\curveto(566.1903739,51.06994193)(566.31037378,51.10994189)(566.43036865,51.15994436)
\curveto(566.72037337,51.27994172)(567.02037307,51.38994161)(567.33036865,51.48994436)
\curveto(567.64037245,51.58994141)(567.94037215,51.6999413)(568.23036865,51.81994436)
\curveto(568.27037182,51.83994116)(568.31037178,51.84994115)(568.35036865,51.84994436)
\curveto(568.38037171,51.84994115)(568.41037168,51.85994114)(568.44036865,51.87994436)
\curveto(568.58037151,51.93994106)(568.72537137,51.994941)(568.87536865,52.04494436)
\lineto(569.29536865,52.19494436)
\curveto(569.36537073,52.22494077)(569.44037065,52.25494074)(569.52036865,52.28494436)
\curveto(569.5903705,52.31494068)(569.63537046,52.36494063)(569.65536865,52.43494436)
\curveto(569.68537041,52.51494048)(569.66037043,52.57494042)(569.58036865,52.61494436)
\curveto(569.4903706,52.66494033)(569.42037067,52.6999403)(569.37036865,52.71994436)
\curveto(569.20037089,52.7999402)(569.02037107,52.86494013)(568.83036865,52.91494436)
\curveto(568.64037145,52.96494003)(568.45537164,53.02493997)(568.27536865,53.09494436)
\curveto(568.04537205,53.18493981)(567.81537228,53.26493973)(567.58536865,53.33494436)
\curveto(567.34537275,53.40493959)(567.11537298,53.48993951)(566.89536865,53.58994436)
\curveto(566.84537325,53.5999394)(566.78037331,53.61493938)(566.70036865,53.63494436)
\curveto(566.61037348,53.67493932)(566.52037357,53.70993929)(566.43036865,53.73994436)
\curveto(566.33037376,53.76993923)(566.24037385,53.7999392)(566.16036865,53.82994436)
\curveto(566.11037398,53.84993915)(566.06537403,53.86493913)(566.02536865,53.87494436)
\curveto(565.98537411,53.88493911)(565.94037415,53.8999391)(565.89036865,53.91994436)
\curveto(565.77037432,53.96993903)(565.65037444,54.00993899)(565.53036865,54.03994436)
\curveto(565.40037469,54.07993892)(565.27537482,54.12493887)(565.15536865,54.17494436)
\curveto(565.10537499,54.1949388)(565.06037503,54.20993879)(565.02036865,54.21994436)
\curveto(564.98037511,54.22993877)(564.93537516,54.24493875)(564.88536865,54.26494436)
\curveto(564.7953753,54.30493869)(564.70537539,54.33993866)(564.61536865,54.36994436)
\curveto(564.51537558,54.3999386)(564.42037567,54.42993857)(564.33036865,54.45994436)
\curveto(564.25037584,54.48993851)(564.17037592,54.51493848)(564.09036865,54.53494436)
\curveto(564.00037609,54.56493843)(563.92537617,54.60493839)(563.86536865,54.65494436)
\curveto(563.77537632,54.72493827)(563.72537637,54.81993818)(563.71536865,54.93994436)
\curveto(563.70537639,55.06993793)(563.70037639,55.20993779)(563.70036865,55.35994436)
\curveto(563.70037639,55.43993756)(563.70537639,55.51493748)(563.71536865,55.58494436)
\curveto(563.71537638,55.66493733)(563.73037636,55.72993727)(563.76036865,55.77994436)
\curveto(563.82037627,55.86993713)(563.91537618,55.8949371)(564.04536865,55.85494436)
\curveto(564.17537592,55.81493718)(564.27537582,55.77993722)(564.34536865,55.74994436)
\lineto(564.40536865,55.71994436)
\curveto(564.42537567,55.71993728)(564.44537565,55.71493728)(564.46536865,55.70494436)
\curveto(564.74537535,55.5949374)(565.03037506,55.48493751)(565.32036865,55.37494436)
\lineto(566.16036865,55.04494436)
\curveto(566.24037385,55.01493798)(566.31537378,54.98993801)(566.38536865,54.96994436)
\curveto(566.44537365,54.94993805)(566.51037358,54.92493807)(566.58036865,54.89494436)
\curveto(566.78037331,54.81493818)(566.98537311,54.73493826)(567.19536865,54.65494436)
\curveto(567.3953727,54.58493841)(567.5953725,54.50993849)(567.79536865,54.42994436)
\curveto(568.48537161,54.13993886)(569.18037091,53.86993913)(569.88036865,53.61994436)
\curveto(570.58036951,53.36993963)(571.27536882,53.0999399)(571.96536865,52.80994436)
\lineto(572.11536865,52.74994436)
\curveto(572.17536792,52.73994026)(572.23536786,52.72494027)(572.29536865,52.70494436)
\curveto(572.66536743,52.54494045)(573.03036706,52.37494062)(573.39036865,52.19494436)
\curveto(573.76036633,52.01494098)(574.04536605,51.76494123)(574.24536865,51.44494436)
\curveto(574.30536579,51.33494166)(574.35036574,51.22494177)(574.38036865,51.11494436)
\curveto(574.42036567,51.00494199)(574.45536564,50.87994212)(574.48536865,50.73994436)
\curveto(574.50536559,50.68994231)(574.51036558,50.63494236)(574.50036865,50.57494436)
\curveto(574.4903656,50.52494247)(574.4903656,50.46994253)(574.50036865,50.40994436)
\curveto(574.52036557,50.32994267)(574.52036557,50.24994275)(574.50036865,50.16994436)
\curveto(574.4903656,50.12994287)(574.48536561,50.07994292)(574.48536865,50.01994436)
\lineto(574.42536865,49.77994436)
\curveto(574.40536569,49.70994329)(574.36536573,49.65494334)(574.30536865,49.61494436)
\curveto(574.24536585,49.56494343)(574.17036592,49.53494346)(574.08036865,49.52494436)
\lineto(573.81036865,49.52494436)
\lineto(573.60036865,49.52494436)
\curveto(573.54036655,49.53494346)(573.4903666,49.55494344)(573.45036865,49.58494436)
\curveto(573.34036675,49.65494334)(573.31036678,49.77494322)(573.36036865,49.94494436)
\curveto(573.38036671,50.05494294)(573.3903667,50.17494282)(573.39036865,50.30494436)
\curveto(573.3903667,50.43494256)(573.37036672,50.54994245)(573.33036865,50.64994436)
\curveto(573.28036681,50.7999422)(573.20536689,50.91994208)(573.10536865,51.00994436)
\curveto(573.00536709,51.10994189)(572.8903672,51.1949418)(572.76036865,51.26494436)
\curveto(572.64036745,51.33494166)(572.51036758,51.3949416)(572.37036865,51.44494436)
\lineto(571.95036865,51.62494436)
\curveto(571.86036823,51.66494133)(571.75036834,51.70494129)(571.62036865,51.74494436)
\curveto(571.4903686,51.7949412)(571.35536874,51.7999412)(571.21536865,51.75994436)
\curveto(571.05536904,51.70994129)(570.90536919,51.65494134)(570.76536865,51.59494436)
\curveto(570.62536947,51.54494145)(570.48536961,51.48994151)(570.34536865,51.42994436)
\curveto(570.13536996,51.33994166)(569.92537017,51.25494174)(569.71536865,51.17494436)
\curveto(569.50537059,51.0949419)(569.30037079,51.01494198)(569.10036865,50.93494436)
\curveto(568.96037113,50.87494212)(568.82537127,50.81994218)(568.69536865,50.76994436)
\curveto(568.56537153,50.71994228)(568.43037166,50.66994233)(568.29036865,50.61994436)
\lineto(566.97036865,50.07994436)
\curveto(566.53037356,49.90994309)(566.090374,49.73494326)(565.65036865,49.55494436)
\curveto(565.42037467,49.45494354)(565.20037489,49.36494363)(564.99036865,49.28494436)
\curveto(564.77037532,49.20494379)(564.55037554,49.11994388)(564.33036865,49.02994436)
\curveto(564.27037582,49.00994399)(564.1903759,48.97994402)(564.09036865,48.93994436)
\curveto(563.98037611,48.8999441)(563.8903762,48.90494409)(563.82036865,48.95494436)
\curveto(563.77037632,48.98494401)(563.73537636,49.04494395)(563.71536865,49.13494436)
\curveto(563.70537639,49.15494384)(563.70537639,49.17494382)(563.71536865,49.19494436)
\curveto(563.71537638,49.22494377)(563.71037638,49.24994375)(563.70036865,49.26994436)
}
}
{
\newrgbcolor{curcolor}{0 0 0}
\pscustom[linestyle=none,fillstyle=solid,fillcolor=curcolor]
{
}
}
{
\newrgbcolor{curcolor}{0 0 0}
\pscustom[linestyle=none,fillstyle=solid,fillcolor=curcolor]
{
\newpath
\moveto(560.82036865,65.05510061)
\curveto(560.82037927,65.15509575)(560.83037926,65.25009566)(560.85036865,65.34010061)
\curveto(560.86037923,65.43009548)(560.8903792,65.49509541)(560.94036865,65.53510061)
\curveto(561.02037907,65.59509531)(561.12537897,65.62509528)(561.25536865,65.62510061)
\lineto(561.64536865,65.62510061)
\lineto(563.14536865,65.62510061)
\lineto(569.53536865,65.62510061)
\lineto(570.70536865,65.62510061)
\lineto(571.02036865,65.62510061)
\curveto(571.12036897,65.63509527)(571.20036889,65.62009529)(571.26036865,65.58010061)
\curveto(571.34036875,65.53009538)(571.3903687,65.45509545)(571.41036865,65.35510061)
\curveto(571.42036867,65.26509564)(571.42536867,65.15509575)(571.42536865,65.02510061)
\lineto(571.42536865,64.80010061)
\curveto(571.40536869,64.72009619)(571.3903687,64.65009626)(571.38036865,64.59010061)
\curveto(571.36036873,64.53009638)(571.32036877,64.48009643)(571.26036865,64.44010061)
\curveto(571.20036889,64.40009651)(571.12536897,64.38009653)(571.03536865,64.38010061)
\lineto(570.73536865,64.38010061)
\lineto(569.64036865,64.38010061)
\lineto(564.30036865,64.38010061)
\curveto(564.21037588,64.36009655)(564.13537596,64.34509656)(564.07536865,64.33510061)
\curveto(564.00537609,64.33509657)(563.94537615,64.3050966)(563.89536865,64.24510061)
\curveto(563.84537625,64.17509673)(563.82037627,64.08509682)(563.82036865,63.97510061)
\curveto(563.81037628,63.87509703)(563.80537629,63.76509714)(563.80536865,63.64510061)
\lineto(563.80536865,62.50510061)
\lineto(563.80536865,62.01010061)
\curveto(563.7953763,61.85009906)(563.73537636,61.74009917)(563.62536865,61.68010061)
\curveto(563.5953765,61.66009925)(563.56537653,61.65009926)(563.53536865,61.65010061)
\curveto(563.4953766,61.65009926)(563.45037664,61.64509926)(563.40036865,61.63510061)
\curveto(563.28037681,61.61509929)(563.17037692,61.62009929)(563.07036865,61.65010061)
\curveto(562.97037712,61.69009922)(562.90037719,61.74509916)(562.86036865,61.81510061)
\curveto(562.81037728,61.89509901)(562.78537731,62.01509889)(562.78536865,62.17510061)
\curveto(562.78537731,62.33509857)(562.77037732,62.47009844)(562.74036865,62.58010061)
\curveto(562.73037736,62.63009828)(562.72537737,62.68509822)(562.72536865,62.74510061)
\curveto(562.71537738,62.8050981)(562.70037739,62.86509804)(562.68036865,62.92510061)
\curveto(562.63037746,63.07509783)(562.58037751,63.22009769)(562.53036865,63.36010061)
\curveto(562.47037762,63.50009741)(562.40037769,63.63509727)(562.32036865,63.76510061)
\curveto(562.23037786,63.905097)(562.12537797,64.02509688)(562.00536865,64.12510061)
\curveto(561.88537821,64.22509668)(561.75537834,64.32009659)(561.61536865,64.41010061)
\curveto(561.51537858,64.47009644)(561.40537869,64.51509639)(561.28536865,64.54510061)
\curveto(561.16537893,64.58509632)(561.06037903,64.63509627)(560.97036865,64.69510061)
\curveto(560.91037918,64.74509616)(560.87037922,64.81509609)(560.85036865,64.90510061)
\curveto(560.84037925,64.92509598)(560.83537926,64.95009596)(560.83536865,64.98010061)
\curveto(560.83537926,65.0100959)(560.83037926,65.03509587)(560.82036865,65.05510061)
}
}
{
\newrgbcolor{curcolor}{0 0 0}
\pscustom[linestyle=none,fillstyle=solid,fillcolor=curcolor]
{
\newpath
\moveto(561.01536865,69.80470998)
\lineto(561.01536865,74.60470998)
\lineto(561.01536865,75.60970998)
\curveto(561.01537908,75.74970288)(561.02537907,75.86970276)(561.04536865,75.96970998)
\curveto(561.05537904,76.07970255)(561.10037899,76.15970247)(561.18036865,76.20970998)
\curveto(561.22037887,76.2297024)(561.27037882,76.23970239)(561.33036865,76.23970998)
\curveto(561.3903787,76.24970238)(561.45537864,76.25470238)(561.52536865,76.25470998)
\lineto(561.79536865,76.25470998)
\curveto(561.88537821,76.25470238)(561.96537813,76.24470239)(562.03536865,76.22470998)
\curveto(562.11537798,76.18470245)(562.18537791,76.13970249)(562.24536865,76.08970998)
\lineto(562.42536865,75.93970998)
\curveto(562.47537762,75.90970272)(562.51537758,75.87470276)(562.54536865,75.83470998)
\curveto(562.57537752,75.79470284)(562.61537748,75.75470288)(562.66536865,75.71470998)
\curveto(562.77537732,75.634703)(562.88537721,75.54970308)(562.99536865,75.45970998)
\curveto(563.095377,75.36970326)(563.20037689,75.28470335)(563.31036865,75.20470998)
\curveto(563.51037658,75.06470357)(563.72037637,74.92470371)(563.94036865,74.78470998)
\curveto(564.15037594,74.64470399)(564.36537573,74.50470413)(564.58536865,74.36470998)
\curveto(564.67537542,74.31470432)(564.77037532,74.26470437)(564.87036865,74.21470998)
\curveto(564.97037512,74.16470447)(565.06537503,74.10970452)(565.15536865,74.04970998)
\curveto(565.17537492,74.0297046)(565.20037489,74.01970461)(565.23036865,74.01970998)
\curveto(565.26037483,74.01970461)(565.28537481,74.00970462)(565.30536865,73.98970998)
\curveto(565.40537469,73.91970471)(565.52037457,73.85470478)(565.65036865,73.79470998)
\curveto(565.77037432,73.7347049)(565.88537421,73.67970495)(565.99536865,73.62970998)
\curveto(566.22537387,73.5297051)(566.46037363,73.4347052)(566.70036865,73.34470998)
\curveto(566.94037315,73.25470538)(567.18037291,73.15470548)(567.42036865,73.04470998)
\curveto(567.47037262,73.02470561)(567.51537258,73.00970562)(567.55536865,72.99970998)
\curveto(567.5953725,72.99970563)(567.64037245,72.98970564)(567.69036865,72.96970998)
\curveto(567.81037228,72.91970571)(567.93537216,72.87470576)(568.06536865,72.83470998)
\curveto(568.18537191,72.80470583)(568.30537179,72.76970586)(568.42536865,72.72970998)
\curveto(568.65537144,72.64970598)(568.8953712,72.58470605)(569.14536865,72.53470998)
\curveto(569.38537071,72.49470614)(569.62537047,72.44470619)(569.86536865,72.38470998)
\curveto(570.01537008,72.34470629)(570.16536993,72.31970631)(570.31536865,72.30970998)
\curveto(570.46536963,72.29970633)(570.61536948,72.27970635)(570.76536865,72.24970998)
\curveto(570.80536929,72.23970639)(570.86536923,72.2347064)(570.94536865,72.23470998)
\curveto(571.06536903,72.20470643)(571.16536893,72.17470646)(571.24536865,72.14470998)
\curveto(571.32536877,72.11470652)(571.38036871,72.04470659)(571.41036865,71.93470998)
\curveto(571.43036866,71.88470675)(571.44036865,71.8297068)(571.44036865,71.76970998)
\lineto(571.44036865,71.57470998)
\curveto(571.44036865,71.4347072)(571.43536866,71.29470734)(571.42536865,71.15470998)
\curveto(571.41536868,71.02470761)(571.37036872,70.9297077)(571.29036865,70.86970998)
\curveto(571.23036886,70.8297078)(571.14536895,70.80970782)(571.03536865,70.80970998)
\curveto(570.92536917,70.81970781)(570.83036926,70.8347078)(570.75036865,70.85470998)
\lineto(570.67536865,70.85470998)
\curveto(570.64536945,70.86470777)(570.61536948,70.86970776)(570.58536865,70.86970998)
\curveto(570.50536959,70.88970774)(570.43036966,70.89970773)(570.36036865,70.89970998)
\curveto(570.2903698,70.89970773)(570.22036987,70.90970772)(570.15036865,70.92970998)
\curveto(569.96037013,70.97970765)(569.77537032,71.01970761)(569.59536865,71.04970998)
\curveto(569.40537069,71.07970755)(569.22537087,71.11970751)(569.05536865,71.16970998)
\curveto(569.00537109,71.18970744)(568.96537113,71.19970743)(568.93536865,71.19970998)
\curveto(568.90537119,71.19970743)(568.87037122,71.20470743)(568.83036865,71.21470998)
\curveto(568.53037156,71.31470732)(568.23537186,71.40470723)(567.94536865,71.48470998)
\curveto(567.65537244,71.57470706)(567.37537272,71.67970695)(567.10536865,71.79970998)
\curveto(566.52537357,72.05970657)(565.97537412,72.3297063)(565.45536865,72.60970998)
\curveto(564.92537517,72.88970574)(564.42037567,73.19970543)(563.94036865,73.53970998)
\curveto(563.74037635,73.67970495)(563.55037654,73.8297048)(563.37036865,73.98970998)
\curveto(563.18037691,74.14970448)(562.9903771,74.29970433)(562.80036865,74.43970998)
\curveto(562.75037734,74.47970415)(562.70537739,74.51470412)(562.66536865,74.54470998)
\curveto(562.61537748,74.58470405)(562.56537753,74.61970401)(562.51536865,74.64970998)
\curveto(562.4953776,74.65970397)(562.47037762,74.66970396)(562.44036865,74.67970998)
\curveto(562.41037768,74.69970393)(562.38037771,74.69970393)(562.35036865,74.67970998)
\curveto(562.2903778,74.65970397)(562.25537784,74.62470401)(562.24536865,74.57470998)
\curveto(562.22537787,74.52470411)(562.20537789,74.47470416)(562.18536865,74.42470998)
\lineto(562.18536865,74.31970998)
\curveto(562.17537792,74.27970435)(562.17537792,74.2297044)(562.18536865,74.16970998)
\lineto(562.18536865,74.01970998)
\lineto(562.18536865,73.41970998)
\lineto(562.18536865,70.77970998)
\lineto(562.18536865,70.04470998)
\lineto(562.18536865,69.80470998)
\curveto(562.17537792,69.7347089)(562.16037793,69.67470896)(562.14036865,69.62470998)
\curveto(562.10037799,69.5347091)(562.04037805,69.47470916)(561.96036865,69.44470998)
\curveto(561.86037823,69.39470924)(561.71537838,69.37970925)(561.52536865,69.39970998)
\curveto(561.32537877,69.41970921)(561.1903789,69.45470918)(561.12036865,69.50470998)
\curveto(561.10037899,69.52470911)(561.08537901,69.54970908)(561.07536865,69.57970998)
\lineto(561.01536865,69.69970998)
\curveto(561.01537908,69.71970891)(561.02037907,69.7347089)(561.03036865,69.74470998)
\curveto(561.03037906,69.76470887)(561.02537907,69.78470885)(561.01536865,69.80470998)
}
}
{
\newrgbcolor{curcolor}{0 0 0}
\pscustom[linestyle=none,fillstyle=solid,fillcolor=curcolor]
{
\newpath
\moveto(569.79036865,78.63431936)
\lineto(569.79036865,79.26431936)
\lineto(569.79036865,79.45931936)
\curveto(569.7903703,79.52931683)(569.80037029,79.58931677)(569.82036865,79.63931936)
\curveto(569.86037023,79.70931665)(569.90037019,79.7593166)(569.94036865,79.78931936)
\curveto(569.9903701,79.82931653)(570.05537004,79.84931651)(570.13536865,79.84931936)
\curveto(570.21536988,79.8593165)(570.30036979,79.86431649)(570.39036865,79.86431936)
\lineto(571.11036865,79.86431936)
\curveto(571.5903685,79.86431649)(572.00036809,79.80431655)(572.34036865,79.68431936)
\curveto(572.68036741,79.56431679)(572.95536714,79.36931699)(573.16536865,79.09931936)
\curveto(573.21536688,79.02931733)(573.26036683,78.9593174)(573.30036865,78.88931936)
\curveto(573.35036674,78.82931753)(573.3953667,78.7543176)(573.43536865,78.66431936)
\curveto(573.44536665,78.64431771)(573.45536664,78.61431774)(573.46536865,78.57431936)
\curveto(573.48536661,78.53431782)(573.4903666,78.48931787)(573.48036865,78.43931936)
\curveto(573.45036664,78.34931801)(573.37536672,78.29431806)(573.25536865,78.27431936)
\curveto(573.14536695,78.2543181)(573.05036704,78.26931809)(572.97036865,78.31931936)
\curveto(572.90036719,78.34931801)(572.83536726,78.39431796)(572.77536865,78.45431936)
\curveto(572.72536737,78.52431783)(572.67536742,78.58931777)(572.62536865,78.64931936)
\curveto(572.57536752,78.71931764)(572.50036759,78.77931758)(572.40036865,78.82931936)
\curveto(572.31036778,78.88931747)(572.22036787,78.93931742)(572.13036865,78.97931936)
\curveto(572.10036799,78.99931736)(572.04036805,79.02431733)(571.95036865,79.05431936)
\curveto(571.87036822,79.08431727)(571.80036829,79.08931727)(571.74036865,79.06931936)
\curveto(571.60036849,79.03931732)(571.51036858,78.97931738)(571.47036865,78.88931936)
\curveto(571.44036865,78.80931755)(571.42536867,78.71931764)(571.42536865,78.61931936)
\curveto(571.42536867,78.51931784)(571.40036869,78.43431792)(571.35036865,78.36431936)
\curveto(571.28036881,78.27431808)(571.14036895,78.22931813)(570.93036865,78.22931936)
\lineto(570.37536865,78.22931936)
\lineto(570.15036865,78.22931936)
\curveto(570.07037002,78.23931812)(570.00537009,78.2593181)(569.95536865,78.28931936)
\curveto(569.87537022,78.34931801)(569.83037026,78.41931794)(569.82036865,78.49931936)
\curveto(569.81037028,78.51931784)(569.80537029,78.53931782)(569.80536865,78.55931936)
\curveto(569.80537029,78.58931777)(569.80037029,78.61431774)(569.79036865,78.63431936)
}
}
{
\newrgbcolor{curcolor}{0 0 0}
\pscustom[linestyle=none,fillstyle=solid,fillcolor=curcolor]
{
}
}
{
\newrgbcolor{curcolor}{0 0 0}
\pscustom[linestyle=none,fillstyle=solid,fillcolor=curcolor]
{
\newpath
\moveto(560.82036865,89.26463186)
\curveto(560.81037928,89.95462722)(560.93037916,90.55462662)(561.18036865,91.06463186)
\curveto(561.43037866,91.58462559)(561.76537833,91.9796252)(562.18536865,92.24963186)
\curveto(562.26537783,92.29962488)(562.35537774,92.34462483)(562.45536865,92.38463186)
\curveto(562.54537755,92.42462475)(562.64037745,92.46962471)(562.74036865,92.51963186)
\curveto(562.84037725,92.55962462)(562.94037715,92.58962459)(563.04036865,92.60963186)
\curveto(563.14037695,92.62962455)(563.24537685,92.64962453)(563.35536865,92.66963186)
\curveto(563.40537669,92.68962449)(563.45037664,92.69462448)(563.49036865,92.68463186)
\curveto(563.53037656,92.6746245)(563.57537652,92.6796245)(563.62536865,92.69963186)
\curveto(563.67537642,92.70962447)(563.76037633,92.71462446)(563.88036865,92.71463186)
\curveto(563.9903761,92.71462446)(564.07537602,92.70962447)(564.13536865,92.69963186)
\curveto(564.1953759,92.6796245)(564.25537584,92.66962451)(564.31536865,92.66963186)
\curveto(564.37537572,92.6796245)(564.43537566,92.6746245)(564.49536865,92.65463186)
\curveto(564.63537546,92.61462456)(564.77037532,92.5796246)(564.90036865,92.54963186)
\curveto(565.03037506,92.51962466)(565.15537494,92.4796247)(565.27536865,92.42963186)
\curveto(565.41537468,92.36962481)(565.54037455,92.29962488)(565.65036865,92.21963186)
\curveto(565.76037433,92.14962503)(565.87037422,92.0746251)(565.98036865,91.99463186)
\lineto(566.04036865,91.93463186)
\curveto(566.06037403,91.92462525)(566.08037401,91.90962527)(566.10036865,91.88963186)
\curveto(566.26037383,91.76962541)(566.40537369,91.63462554)(566.53536865,91.48463186)
\curveto(566.66537343,91.33462584)(566.7903733,91.174626)(566.91036865,91.00463186)
\curveto(567.13037296,90.69462648)(567.33537276,90.39962678)(567.52536865,90.11963186)
\curveto(567.66537243,89.88962729)(567.80037229,89.65962752)(567.93036865,89.42963186)
\curveto(568.06037203,89.20962797)(568.1953719,88.98962819)(568.33536865,88.76963186)
\curveto(568.50537159,88.51962866)(568.68537141,88.2796289)(568.87536865,88.04963186)
\curveto(569.06537103,87.82962935)(569.2903708,87.63962954)(569.55036865,87.47963186)
\curveto(569.61037048,87.43962974)(569.67037042,87.40462977)(569.73036865,87.37463186)
\curveto(569.78037031,87.34462983)(569.84537025,87.31462986)(569.92536865,87.28463186)
\curveto(569.9953701,87.26462991)(570.05537004,87.25962992)(570.10536865,87.26963186)
\curveto(570.17536992,87.28962989)(570.23036986,87.32462985)(570.27036865,87.37463186)
\curveto(570.30036979,87.42462975)(570.32036977,87.48462969)(570.33036865,87.55463186)
\lineto(570.33036865,87.79463186)
\lineto(570.33036865,88.54463186)
\lineto(570.33036865,91.34963186)
\lineto(570.33036865,92.00963186)
\curveto(570.33036976,92.09962508)(570.33536976,92.18462499)(570.34536865,92.26463186)
\curveto(570.34536975,92.34462483)(570.36536973,92.40962477)(570.40536865,92.45963186)
\curveto(570.44536965,92.50962467)(570.52036957,92.54962463)(570.63036865,92.57963186)
\curveto(570.73036936,92.61962456)(570.83036926,92.62962455)(570.93036865,92.60963186)
\lineto(571.06536865,92.60963186)
\curveto(571.13536896,92.58962459)(571.1953689,92.56962461)(571.24536865,92.54963186)
\curveto(571.2953688,92.52962465)(571.33536876,92.49462468)(571.36536865,92.44463186)
\curveto(571.40536869,92.39462478)(571.42536867,92.32462485)(571.42536865,92.23463186)
\lineto(571.42536865,91.96463186)
\lineto(571.42536865,91.06463186)
\lineto(571.42536865,87.55463186)
\lineto(571.42536865,86.48963186)
\curveto(571.42536867,86.40963077)(571.43036866,86.31963086)(571.44036865,86.21963186)
\curveto(571.44036865,86.11963106)(571.43036866,86.03463114)(571.41036865,85.96463186)
\curveto(571.34036875,85.75463142)(571.16036893,85.68963149)(570.87036865,85.76963186)
\curveto(570.83036926,85.7796314)(570.7953693,85.7796314)(570.76536865,85.76963186)
\curveto(570.72536937,85.76963141)(570.68036941,85.7796314)(570.63036865,85.79963186)
\curveto(570.55036954,85.81963136)(570.46536963,85.83963134)(570.37536865,85.85963186)
\curveto(570.28536981,85.8796313)(570.20036989,85.90463127)(570.12036865,85.93463186)
\curveto(569.63037046,86.09463108)(569.21537088,86.29463088)(568.87536865,86.53463186)
\curveto(568.62537147,86.71463046)(568.40037169,86.91963026)(568.20036865,87.14963186)
\curveto(567.9903721,87.3796298)(567.7953723,87.61962956)(567.61536865,87.86963186)
\curveto(567.43537266,88.12962905)(567.26537283,88.39462878)(567.10536865,88.66463186)
\curveto(566.93537316,88.94462823)(566.76037333,89.21462796)(566.58036865,89.47463186)
\curveto(566.50037359,89.58462759)(566.42537367,89.68962749)(566.35536865,89.78963186)
\curveto(566.28537381,89.89962728)(566.21037388,90.00962717)(566.13036865,90.11963186)
\curveto(566.10037399,90.15962702)(566.07037402,90.19462698)(566.04036865,90.22463186)
\curveto(566.00037409,90.26462691)(565.97037412,90.30462687)(565.95036865,90.34463186)
\curveto(565.84037425,90.48462669)(565.71537438,90.60962657)(565.57536865,90.71963186)
\curveto(565.54537455,90.73962644)(565.52037457,90.76462641)(565.50036865,90.79463186)
\curveto(565.47037462,90.82462635)(565.44037465,90.84962633)(565.41036865,90.86963186)
\curveto(565.31037478,90.94962623)(565.21037488,91.01462616)(565.11036865,91.06463186)
\curveto(565.01037508,91.12462605)(564.90037519,91.179626)(564.78036865,91.22963186)
\curveto(564.71037538,91.25962592)(564.63537546,91.2796259)(564.55536865,91.28963186)
\lineto(564.31536865,91.34963186)
\lineto(564.22536865,91.34963186)
\curveto(564.1953759,91.35962582)(564.16537593,91.36462581)(564.13536865,91.36463186)
\curveto(564.06537603,91.38462579)(563.97037612,91.38962579)(563.85036865,91.37963186)
\curveto(563.72037637,91.3796258)(563.62037647,91.36962581)(563.55036865,91.34963186)
\curveto(563.47037662,91.32962585)(563.3953767,91.30962587)(563.32536865,91.28963186)
\curveto(563.24537685,91.2796259)(563.16537693,91.25962592)(563.08536865,91.22963186)
\curveto(562.84537725,91.11962606)(562.64537745,90.96962621)(562.48536865,90.77963186)
\curveto(562.31537778,90.59962658)(562.17537792,90.3796268)(562.06536865,90.11963186)
\curveto(562.04537805,90.04962713)(562.03037806,89.9796272)(562.02036865,89.90963186)
\curveto(562.00037809,89.83962734)(561.98037811,89.76462741)(561.96036865,89.68463186)
\curveto(561.94037815,89.60462757)(561.93037816,89.49462768)(561.93036865,89.35463186)
\curveto(561.93037816,89.22462795)(561.94037815,89.11962806)(561.96036865,89.03963186)
\curveto(561.97037812,88.9796282)(561.97537812,88.92462825)(561.97536865,88.87463186)
\curveto(561.97537812,88.82462835)(561.98537811,88.7746284)(562.00536865,88.72463186)
\curveto(562.04537805,88.62462855)(562.08537801,88.52962865)(562.12536865,88.43963186)
\curveto(562.16537793,88.35962882)(562.21037788,88.2796289)(562.26036865,88.19963186)
\curveto(562.28037781,88.16962901)(562.30537779,88.13962904)(562.33536865,88.10963186)
\curveto(562.36537773,88.08962909)(562.3903777,88.06462911)(562.41036865,88.03463186)
\lineto(562.48536865,87.95963186)
\curveto(562.50537759,87.92962925)(562.52537757,87.90462927)(562.54536865,87.88463186)
\lineto(562.75536865,87.73463186)
\curveto(562.81537728,87.69462948)(562.88037721,87.64962953)(562.95036865,87.59963186)
\curveto(563.04037705,87.53962964)(563.14537695,87.48962969)(563.26536865,87.44963186)
\curveto(563.37537672,87.41962976)(563.48537661,87.38462979)(563.59536865,87.34463186)
\curveto(563.70537639,87.30462987)(563.85037624,87.2796299)(564.03036865,87.26963186)
\curveto(564.20037589,87.25962992)(564.32537577,87.22962995)(564.40536865,87.17963186)
\curveto(564.48537561,87.12963005)(564.53037556,87.05463012)(564.54036865,86.95463186)
\curveto(564.55037554,86.85463032)(564.55537554,86.74463043)(564.55536865,86.62463186)
\curveto(564.55537554,86.58463059)(564.56037553,86.54463063)(564.57036865,86.50463186)
\curveto(564.57037552,86.46463071)(564.56537553,86.42963075)(564.55536865,86.39963186)
\curveto(564.53537556,86.34963083)(564.52537557,86.29963088)(564.52536865,86.24963186)
\curveto(564.52537557,86.20963097)(564.51537558,86.16963101)(564.49536865,86.12963186)
\curveto(564.43537566,86.03963114)(564.30037579,85.99463118)(564.09036865,85.99463186)
\lineto(563.97036865,85.99463186)
\curveto(563.91037618,86.00463117)(563.85037624,86.00963117)(563.79036865,86.00963186)
\curveto(563.72037637,86.01963116)(563.65537644,86.02963115)(563.59536865,86.03963186)
\curveto(563.48537661,86.05963112)(563.38537671,86.0796311)(563.29536865,86.09963186)
\curveto(563.1953769,86.11963106)(563.10037699,86.14963103)(563.01036865,86.18963186)
\curveto(562.94037715,86.20963097)(562.88037721,86.22963095)(562.83036865,86.24963186)
\lineto(562.65036865,86.30963186)
\curveto(562.3903777,86.42963075)(562.14537795,86.58463059)(561.91536865,86.77463186)
\curveto(561.68537841,86.9746302)(561.50037859,87.18962999)(561.36036865,87.41963186)
\curveto(561.28037881,87.52962965)(561.21537888,87.64462953)(561.16536865,87.76463186)
\lineto(561.01536865,88.15463186)
\curveto(560.96537913,88.26462891)(560.93537916,88.3796288)(560.92536865,88.49963186)
\curveto(560.90537919,88.61962856)(560.88037921,88.74462843)(560.85036865,88.87463186)
\curveto(560.85037924,88.94462823)(560.85037924,89.00962817)(560.85036865,89.06963186)
\curveto(560.84037925,89.12962805)(560.83037926,89.19462798)(560.82036865,89.26463186)
}
}
{
\newrgbcolor{curcolor}{0 0 0}
\pscustom[linestyle=none,fillstyle=solid,fillcolor=curcolor]
{
\newpath
\moveto(566.34036865,101.36424123)
\lineto(566.59536865,101.36424123)
\curveto(566.67537342,101.37423353)(566.75037334,101.36923353)(566.82036865,101.34924123)
\lineto(567.06036865,101.34924123)
\lineto(567.22536865,101.34924123)
\curveto(567.32537277,101.32923357)(567.43037266,101.31923358)(567.54036865,101.31924123)
\curveto(567.64037245,101.31923358)(567.74037235,101.30923359)(567.84036865,101.28924123)
\lineto(567.99036865,101.28924123)
\curveto(568.13037196,101.25923364)(568.27037182,101.23923366)(568.41036865,101.22924123)
\curveto(568.54037155,101.21923368)(568.67037142,101.19423371)(568.80036865,101.15424123)
\curveto(568.88037121,101.13423377)(568.96537113,101.11423379)(569.05536865,101.09424123)
\lineto(569.29536865,101.03424123)
\lineto(569.59536865,100.91424123)
\curveto(569.68537041,100.88423402)(569.77537032,100.84923405)(569.86536865,100.80924123)
\curveto(570.08537001,100.70923419)(570.30036979,100.57423433)(570.51036865,100.40424123)
\curveto(570.72036937,100.24423466)(570.8903692,100.06923483)(571.02036865,99.87924123)
\curveto(571.06036903,99.82923507)(571.10036899,99.76923513)(571.14036865,99.69924123)
\curveto(571.17036892,99.63923526)(571.20536889,99.57923532)(571.24536865,99.51924123)
\curveto(571.2953688,99.43923546)(571.33536876,99.34423556)(571.36536865,99.23424123)
\curveto(571.3953687,99.12423578)(571.42536867,99.01923588)(571.45536865,98.91924123)
\curveto(571.4953686,98.80923609)(571.52036857,98.6992362)(571.53036865,98.58924123)
\curveto(571.54036855,98.47923642)(571.55536854,98.36423654)(571.57536865,98.24424123)
\curveto(571.58536851,98.2042367)(571.58536851,98.15923674)(571.57536865,98.10924123)
\curveto(571.57536852,98.06923683)(571.58036851,98.02923687)(571.59036865,97.98924123)
\curveto(571.60036849,97.94923695)(571.60536849,97.89423701)(571.60536865,97.82424123)
\curveto(571.60536849,97.75423715)(571.60036849,97.7042372)(571.59036865,97.67424123)
\curveto(571.57036852,97.62423728)(571.56536853,97.57923732)(571.57536865,97.53924123)
\curveto(571.58536851,97.4992374)(571.58536851,97.46423744)(571.57536865,97.43424123)
\lineto(571.57536865,97.34424123)
\curveto(571.55536854,97.28423762)(571.54036855,97.21923768)(571.53036865,97.14924123)
\curveto(571.53036856,97.08923781)(571.52536857,97.02423788)(571.51536865,96.95424123)
\curveto(571.46536863,96.78423812)(571.41536868,96.62423828)(571.36536865,96.47424123)
\curveto(571.31536878,96.32423858)(571.25036884,96.17923872)(571.17036865,96.03924123)
\curveto(571.13036896,95.98923891)(571.10036899,95.93423897)(571.08036865,95.87424123)
\curveto(571.05036904,95.82423908)(571.01536908,95.77423913)(570.97536865,95.72424123)
\curveto(570.7953693,95.48423942)(570.57536952,95.28423962)(570.31536865,95.12424123)
\curveto(570.05537004,94.96423994)(569.77037032,94.82424008)(569.46036865,94.70424123)
\curveto(569.32037077,94.64424026)(569.18037091,94.5992403)(569.04036865,94.56924123)
\curveto(568.8903712,94.53924036)(568.73537136,94.5042404)(568.57536865,94.46424123)
\curveto(568.46537163,94.44424046)(568.35537174,94.42924047)(568.24536865,94.41924123)
\curveto(568.13537196,94.40924049)(568.02537207,94.39424051)(567.91536865,94.37424123)
\curveto(567.87537222,94.36424054)(567.83537226,94.35924054)(567.79536865,94.35924123)
\curveto(567.75537234,94.36924053)(567.71537238,94.36924053)(567.67536865,94.35924123)
\curveto(567.62537247,94.34924055)(567.57537252,94.34424056)(567.52536865,94.34424123)
\lineto(567.36036865,94.34424123)
\curveto(567.31037278,94.32424058)(567.26037283,94.31924058)(567.21036865,94.32924123)
\curveto(567.15037294,94.33924056)(567.095373,94.33924056)(567.04536865,94.32924123)
\curveto(567.00537309,94.31924058)(566.96037313,94.31924058)(566.91036865,94.32924123)
\curveto(566.86037323,94.33924056)(566.81037328,94.33424057)(566.76036865,94.31424123)
\curveto(566.6903734,94.29424061)(566.61537348,94.28924061)(566.53536865,94.29924123)
\curveto(566.44537365,94.30924059)(566.36037373,94.31424059)(566.28036865,94.31424123)
\curveto(566.1903739,94.31424059)(566.090374,94.30924059)(565.98036865,94.29924123)
\curveto(565.86037423,94.28924061)(565.76037433,94.29424061)(565.68036865,94.31424123)
\lineto(565.39536865,94.31424123)
\lineto(564.76536865,94.35924123)
\curveto(564.66537543,94.36924053)(564.57037552,94.37924052)(564.48036865,94.38924123)
\lineto(564.18036865,94.41924123)
\curveto(564.13037596,94.43924046)(564.08037601,94.44424046)(564.03036865,94.43424123)
\curveto(563.97037612,94.43424047)(563.91537618,94.44424046)(563.86536865,94.46424123)
\curveto(563.6953764,94.51424039)(563.53037656,94.55424035)(563.37036865,94.58424123)
\curveto(563.20037689,94.61424029)(563.04037705,94.66424024)(562.89036865,94.73424123)
\curveto(562.43037766,94.92423998)(562.05537804,95.14423976)(561.76536865,95.39424123)
\curveto(561.47537862,95.65423925)(561.23037886,96.01423889)(561.03036865,96.47424123)
\curveto(560.98037911,96.6042383)(560.94537915,96.73423817)(560.92536865,96.86424123)
\curveto(560.90537919,97.0042379)(560.88037921,97.14423776)(560.85036865,97.28424123)
\curveto(560.84037925,97.35423755)(560.83537926,97.41923748)(560.83536865,97.47924123)
\curveto(560.83537926,97.53923736)(560.83037926,97.6042373)(560.82036865,97.67424123)
\curveto(560.80037929,98.5042364)(560.95037914,99.17423573)(561.27036865,99.68424123)
\curveto(561.58037851,100.19423471)(562.02037807,100.57423433)(562.59036865,100.82424123)
\curveto(562.71037738,100.87423403)(562.83537726,100.91923398)(562.96536865,100.95924123)
\curveto(563.095377,100.9992339)(563.23037686,101.04423386)(563.37036865,101.09424123)
\curveto(563.45037664,101.11423379)(563.53537656,101.12923377)(563.62536865,101.13924123)
\lineto(563.86536865,101.19924123)
\curveto(563.97537612,101.22923367)(564.08537601,101.24423366)(564.19536865,101.24424123)
\curveto(564.30537579,101.25423365)(564.41537568,101.26923363)(564.52536865,101.28924123)
\curveto(564.57537552,101.30923359)(564.62037547,101.31423359)(564.66036865,101.30424123)
\curveto(564.70037539,101.3042336)(564.74037535,101.30923359)(564.78036865,101.31924123)
\curveto(564.83037526,101.32923357)(564.88537521,101.32923357)(564.94536865,101.31924123)
\curveto(564.9953751,101.31923358)(565.04537505,101.32423358)(565.09536865,101.33424123)
\lineto(565.23036865,101.33424123)
\curveto(565.2903748,101.35423355)(565.36037473,101.35423355)(565.44036865,101.33424123)
\curveto(565.51037458,101.32423358)(565.57537452,101.32923357)(565.63536865,101.34924123)
\curveto(565.66537443,101.35923354)(565.70537439,101.36423354)(565.75536865,101.36424123)
\lineto(565.87536865,101.36424123)
\lineto(566.34036865,101.36424123)
\moveto(568.66536865,99.81924123)
\curveto(568.34537175,99.91923498)(567.98037211,99.97923492)(567.57036865,99.99924123)
\curveto(567.16037293,100.01923488)(566.75037334,100.02923487)(566.34036865,100.02924123)
\curveto(565.91037418,100.02923487)(565.4903746,100.01923488)(565.08036865,99.99924123)
\curveto(564.67037542,99.97923492)(564.28537581,99.93423497)(563.92536865,99.86424123)
\curveto(563.56537653,99.79423511)(563.24537685,99.68423522)(562.96536865,99.53424123)
\curveto(562.67537742,99.39423551)(562.44037765,99.1992357)(562.26036865,98.94924123)
\curveto(562.15037794,98.78923611)(562.07037802,98.60923629)(562.02036865,98.40924123)
\curveto(561.96037813,98.20923669)(561.93037816,97.96423694)(561.93036865,97.67424123)
\curveto(561.95037814,97.65423725)(561.96037813,97.61923728)(561.96036865,97.56924123)
\curveto(561.95037814,97.51923738)(561.95037814,97.47923742)(561.96036865,97.44924123)
\curveto(561.98037811,97.36923753)(562.00037809,97.29423761)(562.02036865,97.22424123)
\curveto(562.03037806,97.16423774)(562.05037804,97.0992378)(562.08036865,97.02924123)
\curveto(562.20037789,96.75923814)(562.37037772,96.53923836)(562.59036865,96.36924123)
\curveto(562.80037729,96.20923869)(563.04537705,96.07423883)(563.32536865,95.96424123)
\curveto(563.43537666,95.91423899)(563.55537654,95.87423903)(563.68536865,95.84424123)
\curveto(563.80537629,95.82423908)(563.93037616,95.7992391)(564.06036865,95.76924123)
\curveto(564.11037598,95.74923915)(564.16537593,95.73923916)(564.22536865,95.73924123)
\curveto(564.27537582,95.73923916)(564.32537577,95.73423917)(564.37536865,95.72424123)
\curveto(564.46537563,95.71423919)(564.56037553,95.7042392)(564.66036865,95.69424123)
\curveto(564.75037534,95.68423922)(564.84537525,95.67423923)(564.94536865,95.66424123)
\curveto(565.02537507,95.66423924)(565.11037498,95.65923924)(565.20036865,95.64924123)
\lineto(565.44036865,95.64924123)
\lineto(565.62036865,95.64924123)
\curveto(565.65037444,95.63923926)(565.68537441,95.63423927)(565.72536865,95.63424123)
\lineto(565.86036865,95.63424123)
\lineto(566.31036865,95.63424123)
\curveto(566.3903737,95.63423927)(566.47537362,95.62923927)(566.56536865,95.61924123)
\curveto(566.64537345,95.61923928)(566.72037337,95.62923927)(566.79036865,95.64924123)
\lineto(567.06036865,95.64924123)
\curveto(567.08037301,95.64923925)(567.11037298,95.64423926)(567.15036865,95.63424123)
\curveto(567.18037291,95.63423927)(567.20537289,95.63923926)(567.22536865,95.64924123)
\curveto(567.32537277,95.65923924)(567.42537267,95.66423924)(567.52536865,95.66424123)
\curveto(567.61537248,95.67423923)(567.71537238,95.68423922)(567.82536865,95.69424123)
\curveto(567.94537215,95.72423918)(568.07037202,95.73923916)(568.20036865,95.73924123)
\curveto(568.32037177,95.74923915)(568.43537166,95.77423913)(568.54536865,95.81424123)
\curveto(568.84537125,95.89423901)(569.11037098,95.97923892)(569.34036865,96.06924123)
\curveto(569.57037052,96.16923873)(569.78537031,96.31423859)(569.98536865,96.50424123)
\curveto(570.18536991,96.71423819)(570.33536976,96.97923792)(570.43536865,97.29924123)
\curveto(570.45536964,97.33923756)(570.46536963,97.37423753)(570.46536865,97.40424123)
\curveto(570.45536964,97.44423746)(570.46036963,97.48923741)(570.48036865,97.53924123)
\curveto(570.4903696,97.57923732)(570.50036959,97.64923725)(570.51036865,97.74924123)
\curveto(570.52036957,97.85923704)(570.51536958,97.94423696)(570.49536865,98.00424123)
\curveto(570.47536962,98.07423683)(570.46536963,98.14423676)(570.46536865,98.21424123)
\curveto(570.45536964,98.28423662)(570.44036965,98.34923655)(570.42036865,98.40924123)
\curveto(570.36036973,98.60923629)(570.27536982,98.78923611)(570.16536865,98.94924123)
\curveto(570.14536995,98.97923592)(570.12536997,99.0042359)(570.10536865,99.02424123)
\lineto(570.04536865,99.08424123)
\curveto(570.02537007,99.12423578)(569.98537011,99.17423573)(569.92536865,99.23424123)
\curveto(569.78537031,99.33423557)(569.65537044,99.41923548)(569.53536865,99.48924123)
\curveto(569.41537068,99.55923534)(569.27037082,99.62923527)(569.10036865,99.69924123)
\curveto(569.03037106,99.72923517)(568.96037113,99.74923515)(568.89036865,99.75924123)
\curveto(568.82037127,99.77923512)(568.74537135,99.7992351)(568.66536865,99.81924123)
}
}
{
\newrgbcolor{curcolor}{0 0 0}
\pscustom[linestyle=none,fillstyle=solid,fillcolor=curcolor]
{
\newpath
\moveto(560.82036865,106.77385061)
\curveto(560.82037927,106.87384575)(560.83037926,106.96884566)(560.85036865,107.05885061)
\curveto(560.86037923,107.14884548)(560.8903792,107.21384541)(560.94036865,107.25385061)
\curveto(561.02037907,107.31384531)(561.12537897,107.34384528)(561.25536865,107.34385061)
\lineto(561.64536865,107.34385061)
\lineto(563.14536865,107.34385061)
\lineto(569.53536865,107.34385061)
\lineto(570.70536865,107.34385061)
\lineto(571.02036865,107.34385061)
\curveto(571.12036897,107.35384527)(571.20036889,107.33884529)(571.26036865,107.29885061)
\curveto(571.34036875,107.24884538)(571.3903687,107.17384545)(571.41036865,107.07385061)
\curveto(571.42036867,106.98384564)(571.42536867,106.87384575)(571.42536865,106.74385061)
\lineto(571.42536865,106.51885061)
\curveto(571.40536869,106.43884619)(571.3903687,106.36884626)(571.38036865,106.30885061)
\curveto(571.36036873,106.24884638)(571.32036877,106.19884643)(571.26036865,106.15885061)
\curveto(571.20036889,106.11884651)(571.12536897,106.09884653)(571.03536865,106.09885061)
\lineto(570.73536865,106.09885061)
\lineto(569.64036865,106.09885061)
\lineto(564.30036865,106.09885061)
\curveto(564.21037588,106.07884655)(564.13537596,106.06384656)(564.07536865,106.05385061)
\curveto(564.00537609,106.05384657)(563.94537615,106.0238466)(563.89536865,105.96385061)
\curveto(563.84537625,105.89384673)(563.82037627,105.80384682)(563.82036865,105.69385061)
\curveto(563.81037628,105.59384703)(563.80537629,105.48384714)(563.80536865,105.36385061)
\lineto(563.80536865,104.22385061)
\lineto(563.80536865,103.72885061)
\curveto(563.7953763,103.56884906)(563.73537636,103.45884917)(563.62536865,103.39885061)
\curveto(563.5953765,103.37884925)(563.56537653,103.36884926)(563.53536865,103.36885061)
\curveto(563.4953766,103.36884926)(563.45037664,103.36384926)(563.40036865,103.35385061)
\curveto(563.28037681,103.33384929)(563.17037692,103.33884929)(563.07036865,103.36885061)
\curveto(562.97037712,103.40884922)(562.90037719,103.46384916)(562.86036865,103.53385061)
\curveto(562.81037728,103.61384901)(562.78537731,103.73384889)(562.78536865,103.89385061)
\curveto(562.78537731,104.05384857)(562.77037732,104.18884844)(562.74036865,104.29885061)
\curveto(562.73037736,104.34884828)(562.72537737,104.40384822)(562.72536865,104.46385061)
\curveto(562.71537738,104.5238481)(562.70037739,104.58384804)(562.68036865,104.64385061)
\curveto(562.63037746,104.79384783)(562.58037751,104.93884769)(562.53036865,105.07885061)
\curveto(562.47037762,105.21884741)(562.40037769,105.35384727)(562.32036865,105.48385061)
\curveto(562.23037786,105.623847)(562.12537797,105.74384688)(562.00536865,105.84385061)
\curveto(561.88537821,105.94384668)(561.75537834,106.03884659)(561.61536865,106.12885061)
\curveto(561.51537858,106.18884644)(561.40537869,106.23384639)(561.28536865,106.26385061)
\curveto(561.16537893,106.30384632)(561.06037903,106.35384627)(560.97036865,106.41385061)
\curveto(560.91037918,106.46384616)(560.87037922,106.53384609)(560.85036865,106.62385061)
\curveto(560.84037925,106.64384598)(560.83537926,106.66884596)(560.83536865,106.69885061)
\curveto(560.83537926,106.7288459)(560.83037926,106.75384587)(560.82036865,106.77385061)
}
}
{
\newrgbcolor{curcolor}{0 0 0}
\pscustom[linestyle=none,fillstyle=solid,fillcolor=curcolor]
{
\newpath
\moveto(560.82036865,115.12345998)
\curveto(560.82037927,115.22345513)(560.83037926,115.31845503)(560.85036865,115.40845998)
\curveto(560.86037923,115.49845485)(560.8903792,115.56345479)(560.94036865,115.60345998)
\curveto(561.02037907,115.66345469)(561.12537897,115.69345466)(561.25536865,115.69345998)
\lineto(561.64536865,115.69345998)
\lineto(563.14536865,115.69345998)
\lineto(569.53536865,115.69345998)
\lineto(570.70536865,115.69345998)
\lineto(571.02036865,115.69345998)
\curveto(571.12036897,115.70345465)(571.20036889,115.68845466)(571.26036865,115.64845998)
\curveto(571.34036875,115.59845475)(571.3903687,115.52345483)(571.41036865,115.42345998)
\curveto(571.42036867,115.33345502)(571.42536867,115.22345513)(571.42536865,115.09345998)
\lineto(571.42536865,114.86845998)
\curveto(571.40536869,114.78845556)(571.3903687,114.71845563)(571.38036865,114.65845998)
\curveto(571.36036873,114.59845575)(571.32036877,114.5484558)(571.26036865,114.50845998)
\curveto(571.20036889,114.46845588)(571.12536897,114.4484559)(571.03536865,114.44845998)
\lineto(570.73536865,114.44845998)
\lineto(569.64036865,114.44845998)
\lineto(564.30036865,114.44845998)
\curveto(564.21037588,114.42845592)(564.13537596,114.41345594)(564.07536865,114.40345998)
\curveto(564.00537609,114.40345595)(563.94537615,114.37345598)(563.89536865,114.31345998)
\curveto(563.84537625,114.24345611)(563.82037627,114.1534562)(563.82036865,114.04345998)
\curveto(563.81037628,113.94345641)(563.80537629,113.83345652)(563.80536865,113.71345998)
\lineto(563.80536865,112.57345998)
\lineto(563.80536865,112.07845998)
\curveto(563.7953763,111.91845843)(563.73537636,111.80845854)(563.62536865,111.74845998)
\curveto(563.5953765,111.72845862)(563.56537653,111.71845863)(563.53536865,111.71845998)
\curveto(563.4953766,111.71845863)(563.45037664,111.71345864)(563.40036865,111.70345998)
\curveto(563.28037681,111.68345867)(563.17037692,111.68845866)(563.07036865,111.71845998)
\curveto(562.97037712,111.75845859)(562.90037719,111.81345854)(562.86036865,111.88345998)
\curveto(562.81037728,111.96345839)(562.78537731,112.08345827)(562.78536865,112.24345998)
\curveto(562.78537731,112.40345795)(562.77037732,112.53845781)(562.74036865,112.64845998)
\curveto(562.73037736,112.69845765)(562.72537737,112.7534576)(562.72536865,112.81345998)
\curveto(562.71537738,112.87345748)(562.70037739,112.93345742)(562.68036865,112.99345998)
\curveto(562.63037746,113.14345721)(562.58037751,113.28845706)(562.53036865,113.42845998)
\curveto(562.47037762,113.56845678)(562.40037769,113.70345665)(562.32036865,113.83345998)
\curveto(562.23037786,113.97345638)(562.12537797,114.09345626)(562.00536865,114.19345998)
\curveto(561.88537821,114.29345606)(561.75537834,114.38845596)(561.61536865,114.47845998)
\curveto(561.51537858,114.53845581)(561.40537869,114.58345577)(561.28536865,114.61345998)
\curveto(561.16537893,114.6534557)(561.06037903,114.70345565)(560.97036865,114.76345998)
\curveto(560.91037918,114.81345554)(560.87037922,114.88345547)(560.85036865,114.97345998)
\curveto(560.84037925,114.99345536)(560.83537926,115.01845533)(560.83536865,115.04845998)
\curveto(560.83537926,115.07845527)(560.83037926,115.10345525)(560.82036865,115.12345998)
}
}
{
\newrgbcolor{curcolor}{0 0 0}
\pscustom[linestyle=none,fillstyle=solid,fillcolor=curcolor]
{
\newpath
\moveto(582.69165771,29.18119436)
\lineto(582.69165771,30.09619436)
\curveto(582.69166841,30.19619171)(582.69166841,30.29119161)(582.69165771,30.38119436)
\curveto(582.69166841,30.47119143)(582.71166839,30.54619136)(582.75165771,30.60619436)
\curveto(582.81166829,30.69619121)(582.89166821,30.75619115)(582.99165771,30.78619436)
\curveto(583.09166801,30.82619108)(583.1966679,30.87119103)(583.30665771,30.92119436)
\curveto(583.4966676,31.0011909)(583.68666741,31.07119083)(583.87665771,31.13119436)
\curveto(584.06666703,31.2011907)(584.25666684,31.27619063)(584.44665771,31.35619436)
\curveto(584.62666647,31.42619048)(584.81166629,31.49119041)(585.00165771,31.55119436)
\curveto(585.18166592,31.61119029)(585.36166574,31.68119022)(585.54165771,31.76119436)
\curveto(585.68166542,31.82119008)(585.82666527,31.87619003)(585.97665771,31.92619436)
\curveto(586.12666497,31.97618993)(586.27166483,32.03118987)(586.41165771,32.09119436)
\curveto(586.86166424,32.27118963)(587.31666378,32.44118946)(587.77665771,32.60119436)
\curveto(588.22666287,32.76118914)(588.67666242,32.93118897)(589.12665771,33.11119436)
\curveto(589.17666192,33.13118877)(589.22666187,33.14618876)(589.27665771,33.15619436)
\lineto(589.42665771,33.21619436)
\curveto(589.64666145,33.3061886)(589.87166123,33.39118851)(590.10165771,33.47119436)
\curveto(590.32166078,33.55118835)(590.54166056,33.63618827)(590.76165771,33.72619436)
\curveto(590.85166025,33.76618814)(590.96166014,33.8061881)(591.09165771,33.84619436)
\curveto(591.21165989,33.88618802)(591.28165982,33.95118795)(591.30165771,34.04119436)
\curveto(591.31165979,34.08118782)(591.31165979,34.11118779)(591.30165771,34.13119436)
\lineto(591.24165771,34.19119436)
\curveto(591.19165991,34.24118766)(591.13665996,34.27618763)(591.07665771,34.29619436)
\curveto(591.01666008,34.32618758)(590.95166015,34.35618755)(590.88165771,34.38619436)
\lineto(590.25165771,34.62619436)
\curveto(590.03166107,34.7061872)(589.81666128,34.78618712)(589.60665771,34.86619436)
\lineto(589.45665771,34.92619436)
\lineto(589.27665771,34.98619436)
\curveto(589.08666201,35.06618684)(588.8966622,35.13618677)(588.70665771,35.19619436)
\curveto(588.50666259,35.26618664)(588.30666279,35.34118656)(588.10665771,35.42119436)
\curveto(587.52666357,35.66118624)(586.94166416,35.88118602)(586.35165771,36.08119436)
\curveto(585.76166534,36.29118561)(585.17666592,36.51618539)(584.59665771,36.75619436)
\curveto(584.3966667,36.83618507)(584.19166691,36.91118499)(583.98165771,36.98119436)
\curveto(583.77166733,37.06118484)(583.56666753,37.14118476)(583.36665771,37.22119436)
\curveto(583.28666781,37.26118464)(583.18666791,37.29618461)(583.06665771,37.32619436)
\curveto(582.94666815,37.36618454)(582.86166824,37.42118448)(582.81165771,37.49119436)
\curveto(582.77166833,37.55118435)(582.74166836,37.62618428)(582.72165771,37.71619436)
\curveto(582.7016684,37.81618409)(582.69166841,37.92618398)(582.69165771,38.04619436)
\curveto(582.68166842,38.16618374)(582.68166842,38.28618362)(582.69165771,38.40619436)
\curveto(582.69166841,38.52618338)(582.69166841,38.63618327)(582.69165771,38.73619436)
\curveto(582.69166841,38.82618308)(582.69166841,38.91618299)(582.69165771,39.00619436)
\curveto(582.69166841,39.1061828)(582.71166839,39.18118272)(582.75165771,39.23119436)
\curveto(582.8016683,39.32118258)(582.89166821,39.37118253)(583.02165771,39.38119436)
\curveto(583.15166795,39.39118251)(583.29166781,39.39618251)(583.44165771,39.39619436)
\lineto(585.09165771,39.39619436)
\lineto(591.36165771,39.39619436)
\lineto(592.62165771,39.39619436)
\curveto(592.73165837,39.39618251)(592.84165826,39.39618251)(592.95165771,39.39619436)
\curveto(593.06165804,39.4061825)(593.14665795,39.38618252)(593.20665771,39.33619436)
\curveto(593.26665783,39.3061826)(593.30665779,39.26118264)(593.32665771,39.20119436)
\curveto(593.33665776,39.14118276)(593.35165775,39.07118283)(593.37165771,38.99119436)
\lineto(593.37165771,38.75119436)
\lineto(593.37165771,38.39119436)
\curveto(593.36165774,38.28118362)(593.31665778,38.2011837)(593.23665771,38.15119436)
\curveto(593.20665789,38.13118377)(593.17665792,38.11618379)(593.14665771,38.10619436)
\curveto(593.10665799,38.1061838)(593.06165804,38.09618381)(593.01165771,38.07619436)
\lineto(592.84665771,38.07619436)
\curveto(592.78665831,38.06618384)(592.71665838,38.06118384)(592.63665771,38.06119436)
\curveto(592.55665854,38.07118383)(592.48165862,38.07618383)(592.41165771,38.07619436)
\lineto(591.57165771,38.07619436)
\lineto(587.14665771,38.07619436)
\curveto(586.8966642,38.07618383)(586.64666445,38.07618383)(586.39665771,38.07619436)
\curveto(586.13666496,38.07618383)(585.88666521,38.07118383)(585.64665771,38.06119436)
\curveto(585.54666555,38.06118384)(585.43666566,38.05618385)(585.31665771,38.04619436)
\curveto(585.1966659,38.03618387)(585.13666596,37.98118392)(585.13665771,37.88119436)
\lineto(585.15165771,37.88119436)
\curveto(585.17166593,37.81118409)(585.23666586,37.75118415)(585.34665771,37.70119436)
\curveto(585.45666564,37.66118424)(585.55166555,37.62618428)(585.63165771,37.59619436)
\curveto(585.8016653,37.52618438)(585.97666512,37.46118444)(586.15665771,37.40119436)
\curveto(586.32666477,37.34118456)(586.4966646,37.27118463)(586.66665771,37.19119436)
\curveto(586.71666438,37.17118473)(586.76166434,37.15618475)(586.80165771,37.14619436)
\curveto(586.84166426,37.13618477)(586.88666421,37.12118478)(586.93665771,37.10119436)
\curveto(587.11666398,37.02118488)(587.3016638,36.95118495)(587.49165771,36.89119436)
\curveto(587.67166343,36.84118506)(587.85166325,36.77618513)(588.03165771,36.69619436)
\curveto(588.18166292,36.62618528)(588.33666276,36.56618534)(588.49665771,36.51619436)
\curveto(588.64666245,36.46618544)(588.7966623,36.41118549)(588.94665771,36.35119436)
\curveto(589.41666168,36.15118575)(589.89166121,35.97118593)(590.37165771,35.81119436)
\curveto(590.84166026,35.65118625)(591.30665979,35.47618643)(591.76665771,35.28619436)
\curveto(591.94665915,35.2061867)(592.12665897,35.13618677)(592.30665771,35.07619436)
\curveto(592.48665861,35.01618689)(592.66665843,34.95118695)(592.84665771,34.88119436)
\curveto(592.95665814,34.83118707)(593.06165804,34.78118712)(593.16165771,34.73119436)
\curveto(593.25165785,34.69118721)(593.31665778,34.6061873)(593.35665771,34.47619436)
\curveto(593.36665773,34.45618745)(593.37165773,34.43118747)(593.37165771,34.40119436)
\curveto(593.36165774,34.38118752)(593.36165774,34.35618755)(593.37165771,34.32619436)
\curveto(593.38165772,34.29618761)(593.38665771,34.26118764)(593.38665771,34.22119436)
\curveto(593.37665772,34.18118772)(593.37165773,34.14118776)(593.37165771,34.10119436)
\lineto(593.37165771,33.80119436)
\curveto(593.37165773,33.7011882)(593.34665775,33.62118828)(593.29665771,33.56119436)
\curveto(593.24665785,33.48118842)(593.17665792,33.42118848)(593.08665771,33.38119436)
\curveto(592.98665811,33.35118855)(592.88665821,33.31118859)(592.78665771,33.26119436)
\curveto(592.58665851,33.18118872)(592.38165872,33.1011888)(592.17165771,33.02119436)
\curveto(591.95165915,32.95118895)(591.74165936,32.87618903)(591.54165771,32.79619436)
\curveto(591.36165974,32.71618919)(591.18165992,32.64618926)(591.00165771,32.58619436)
\curveto(590.81166029,32.53618937)(590.62666047,32.47118943)(590.44665771,32.39119436)
\curveto(589.88666121,32.16118974)(589.32166178,31.94618996)(588.75165771,31.74619436)
\curveto(588.18166292,31.54619036)(587.61666348,31.33119057)(587.05665771,31.10119436)
\lineto(586.42665771,30.86119436)
\curveto(586.20666489,30.79119111)(585.9966651,30.71619119)(585.79665771,30.63619436)
\curveto(585.68666541,30.58619132)(585.58166552,30.54119136)(585.48165771,30.50119436)
\curveto(585.37166573,30.47119143)(585.27666582,30.42119148)(585.19665771,30.35119436)
\curveto(585.17666592,30.34119156)(585.16666593,30.33119157)(585.16665771,30.32119436)
\lineto(585.13665771,30.29119436)
\lineto(585.13665771,30.21619436)
\lineto(585.16665771,30.18619436)
\curveto(585.16666593,30.17619173)(585.17166593,30.16619174)(585.18165771,30.15619436)
\curveto(585.23166587,30.13619177)(585.28666581,30.12619178)(585.34665771,30.12619436)
\curveto(585.40666569,30.12619178)(585.46666563,30.11619179)(585.52665771,30.09619436)
\lineto(585.69165771,30.09619436)
\curveto(585.75166535,30.07619183)(585.81666528,30.07119183)(585.88665771,30.08119436)
\curveto(585.95666514,30.09119181)(586.02666507,30.09619181)(586.09665771,30.09619436)
\lineto(586.90665771,30.09619436)
\lineto(591.46665771,30.09619436)
\lineto(592.65165771,30.09619436)
\curveto(592.76165834,30.09619181)(592.87165823,30.09119181)(592.98165771,30.08119436)
\curveto(593.09165801,30.08119182)(593.17665792,30.05619185)(593.23665771,30.00619436)
\curveto(593.31665778,29.95619195)(593.36165774,29.86619204)(593.37165771,29.73619436)
\lineto(593.37165771,29.34619436)
\lineto(593.37165771,29.15119436)
\curveto(593.37165773,29.1011928)(593.36165774,29.05119285)(593.34165771,29.00119436)
\curveto(593.3016578,28.87119303)(593.21665788,28.79619311)(593.08665771,28.77619436)
\curveto(592.95665814,28.76619314)(592.80665829,28.76119314)(592.63665771,28.76119436)
\lineto(590.89665771,28.76119436)
\lineto(584.89665771,28.76119436)
\lineto(583.48665771,28.76119436)
\curveto(583.37666772,28.76119314)(583.26166784,28.75619315)(583.14165771,28.74619436)
\curveto(583.02166808,28.74619316)(582.92666817,28.77119313)(582.85665771,28.82119436)
\curveto(582.7966683,28.86119304)(582.74666835,28.93619297)(582.70665771,29.04619436)
\curveto(582.6966684,29.06619284)(582.6966684,29.08619282)(582.70665771,29.10619436)
\curveto(582.70666839,29.13619277)(582.7016684,29.16119274)(582.69165771,29.18119436)
}
}
{
\newrgbcolor{curcolor}{0 0 0}
\pscustom[linestyle=none,fillstyle=solid,fillcolor=curcolor]
{
\newpath
\moveto(592.81665771,48.38330373)
\curveto(592.97665812,48.4132959)(593.11165799,48.39829592)(593.22165771,48.33830373)
\curveto(593.32165778,48.27829604)(593.3966577,48.19829612)(593.44665771,48.09830373)
\curveto(593.46665763,48.04829627)(593.47665762,47.99329632)(593.47665771,47.93330373)
\curveto(593.47665762,47.88329643)(593.48665761,47.82829649)(593.50665771,47.76830373)
\curveto(593.55665754,47.54829677)(593.54165756,47.32829699)(593.46165771,47.10830373)
\curveto(593.39165771,46.89829742)(593.3016578,46.75329756)(593.19165771,46.67330373)
\curveto(593.12165798,46.62329769)(593.04165806,46.57829774)(592.95165771,46.53830373)
\curveto(592.85165825,46.49829782)(592.77165833,46.44829787)(592.71165771,46.38830373)
\curveto(592.69165841,46.36829795)(592.67165843,46.34329797)(592.65165771,46.31330373)
\curveto(592.63165847,46.29329802)(592.62665847,46.26329805)(592.63665771,46.22330373)
\curveto(592.66665843,46.1132982)(592.72165838,46.00829831)(592.80165771,45.90830373)
\curveto(592.88165822,45.8182985)(592.95165815,45.72829859)(593.01165771,45.63830373)
\curveto(593.09165801,45.50829881)(593.16665793,45.36829895)(593.23665771,45.21830373)
\curveto(593.2966578,45.06829925)(593.35165775,44.90829941)(593.40165771,44.73830373)
\curveto(593.43165767,44.63829968)(593.45165765,44.52829979)(593.46165771,44.40830373)
\curveto(593.47165763,44.29830002)(593.48665761,44.18830013)(593.50665771,44.07830373)
\curveto(593.51665758,44.02830029)(593.52165758,43.98330033)(593.52165771,43.94330373)
\lineto(593.52165771,43.83830373)
\curveto(593.54165756,43.72830059)(593.54165756,43.62330069)(593.52165771,43.52330373)
\lineto(593.52165771,43.38830373)
\curveto(593.51165759,43.33830098)(593.50665759,43.28830103)(593.50665771,43.23830373)
\curveto(593.50665759,43.18830113)(593.4966576,43.14330117)(593.47665771,43.10330373)
\curveto(593.46665763,43.06330125)(593.46165764,43.02830129)(593.46165771,42.99830373)
\curveto(593.47165763,42.97830134)(593.47165763,42.95330136)(593.46165771,42.92330373)
\lineto(593.40165771,42.68330373)
\curveto(593.39165771,42.60330171)(593.37165773,42.52830179)(593.34165771,42.45830373)
\curveto(593.21165789,42.15830216)(593.06665803,41.9133024)(592.90665771,41.72330373)
\curveto(592.73665836,41.54330277)(592.5016586,41.39330292)(592.20165771,41.27330373)
\curveto(591.98165912,41.18330313)(591.71665938,41.13830318)(591.40665771,41.13830373)
\lineto(591.09165771,41.13830373)
\curveto(591.04166006,41.14830317)(590.99166011,41.15330316)(590.94165771,41.15330373)
\lineto(590.76165771,41.18330373)
\lineto(590.43165771,41.30330373)
\curveto(590.32166078,41.34330297)(590.22166088,41.39330292)(590.13165771,41.45330373)
\curveto(589.84166126,41.63330268)(589.62666147,41.87830244)(589.48665771,42.18830373)
\curveto(589.34666175,42.49830182)(589.22166188,42.83830148)(589.11165771,43.20830373)
\curveto(589.07166203,43.34830097)(589.04166206,43.49330082)(589.02165771,43.64330373)
\curveto(589.0016621,43.79330052)(588.97666212,43.94330037)(588.94665771,44.09330373)
\curveto(588.92666217,44.16330015)(588.91666218,44.22830009)(588.91665771,44.28830373)
\curveto(588.91666218,44.35829996)(588.90666219,44.43329988)(588.88665771,44.51330373)
\curveto(588.86666223,44.58329973)(588.85666224,44.65329966)(588.85665771,44.72330373)
\curveto(588.84666225,44.79329952)(588.83166227,44.86829945)(588.81165771,44.94830373)
\curveto(588.75166235,45.19829912)(588.7016624,45.43329888)(588.66165771,45.65330373)
\curveto(588.61166249,45.87329844)(588.4966626,46.04829827)(588.31665771,46.17830373)
\curveto(588.23666286,46.23829808)(588.13666296,46.28829803)(588.01665771,46.32830373)
\curveto(587.88666321,46.36829795)(587.74666335,46.36829795)(587.59665771,46.32830373)
\curveto(587.35666374,46.26829805)(587.16666393,46.17829814)(587.02665771,46.05830373)
\curveto(586.88666421,45.94829837)(586.77666432,45.78829853)(586.69665771,45.57830373)
\curveto(586.64666445,45.45829886)(586.61166449,45.313299)(586.59165771,45.14330373)
\curveto(586.57166453,44.98329933)(586.56166454,44.8132995)(586.56165771,44.63330373)
\curveto(586.56166454,44.45329986)(586.57166453,44.27830004)(586.59165771,44.10830373)
\curveto(586.61166449,43.93830038)(586.64166446,43.79330052)(586.68165771,43.67330373)
\curveto(586.74166436,43.50330081)(586.82666427,43.33830098)(586.93665771,43.17830373)
\curveto(586.9966641,43.09830122)(587.07666402,43.02330129)(587.17665771,42.95330373)
\curveto(587.26666383,42.89330142)(587.36666373,42.83830148)(587.47665771,42.78830373)
\curveto(587.55666354,42.75830156)(587.64166346,42.72830159)(587.73165771,42.69830373)
\curveto(587.82166328,42.67830164)(587.89166321,42.63330168)(587.94165771,42.56330373)
\curveto(587.97166313,42.52330179)(587.9966631,42.45330186)(588.01665771,42.35330373)
\curveto(588.02666307,42.26330205)(588.03166307,42.16830215)(588.03165771,42.06830373)
\curveto(588.03166307,41.96830235)(588.02666307,41.86830245)(588.01665771,41.76830373)
\curveto(587.9966631,41.67830264)(587.97166313,41.6133027)(587.94165771,41.57330373)
\curveto(587.91166319,41.53330278)(587.86166324,41.50330281)(587.79165771,41.48330373)
\curveto(587.72166338,41.46330285)(587.64666345,41.46330285)(587.56665771,41.48330373)
\curveto(587.43666366,41.5133028)(587.31666378,41.54330277)(587.20665771,41.57330373)
\curveto(587.08666401,41.6133027)(586.97166413,41.65830266)(586.86165771,41.70830373)
\curveto(586.51166459,41.89830242)(586.24166486,42.13830218)(586.05165771,42.42830373)
\curveto(585.85166525,42.7183016)(585.69166541,43.07830124)(585.57165771,43.50830373)
\curveto(585.55166555,43.60830071)(585.53666556,43.70830061)(585.52665771,43.80830373)
\curveto(585.51666558,43.9183004)(585.5016656,44.02830029)(585.48165771,44.13830373)
\curveto(585.47166563,44.17830014)(585.47166563,44.24330007)(585.48165771,44.33330373)
\curveto(585.48166562,44.42329989)(585.47166563,44.47829984)(585.45165771,44.49830373)
\curveto(585.44166566,45.19829912)(585.52166558,45.80829851)(585.69165771,46.32830373)
\curveto(585.86166524,46.84829747)(586.18666491,47.2132971)(586.66665771,47.42330373)
\curveto(586.86666423,47.5132968)(587.101664,47.56329675)(587.37165771,47.57330373)
\curveto(587.63166347,47.59329672)(587.90666319,47.60329671)(588.19665771,47.60330373)
\lineto(591.51165771,47.60330373)
\curveto(591.65165945,47.60329671)(591.78665931,47.60829671)(591.91665771,47.61830373)
\curveto(592.04665905,47.62829669)(592.15165895,47.65829666)(592.23165771,47.70830373)
\curveto(592.3016588,47.75829656)(592.35165875,47.82329649)(592.38165771,47.90330373)
\curveto(592.42165868,47.99329632)(592.45165865,48.07829624)(592.47165771,48.15830373)
\curveto(592.48165862,48.23829608)(592.52665857,48.29829602)(592.60665771,48.33830373)
\curveto(592.63665846,48.35829596)(592.66665843,48.36829595)(592.69665771,48.36830373)
\curveto(592.72665837,48.36829595)(592.76665833,48.37329594)(592.81665771,48.38330373)
\moveto(591.15165771,46.23830373)
\curveto(591.01166009,46.29829802)(590.85166025,46.32829799)(590.67165771,46.32830373)
\curveto(590.48166062,46.33829798)(590.28666081,46.34329797)(590.08665771,46.34330373)
\curveto(589.97666112,46.34329797)(589.87666122,46.33829798)(589.78665771,46.32830373)
\curveto(589.6966614,46.318298)(589.62666147,46.27829804)(589.57665771,46.20830373)
\curveto(589.55666154,46.17829814)(589.54666155,46.10829821)(589.54665771,45.99830373)
\curveto(589.56666153,45.97829834)(589.57666152,45.94329837)(589.57665771,45.89330373)
\curveto(589.57666152,45.84329847)(589.58666151,45.79829852)(589.60665771,45.75830373)
\curveto(589.62666147,45.67829864)(589.64666145,45.58829873)(589.66665771,45.48830373)
\lineto(589.72665771,45.18830373)
\curveto(589.72666137,45.15829916)(589.73166137,45.12329919)(589.74165771,45.08330373)
\lineto(589.74165771,44.97830373)
\curveto(589.78166132,44.82829949)(589.80666129,44.66329965)(589.81665771,44.48330373)
\curveto(589.81666128,44.3133)(589.83666126,44.15330016)(589.87665771,44.00330373)
\curveto(589.8966612,43.92330039)(589.91666118,43.84830047)(589.93665771,43.77830373)
\curveto(589.94666115,43.7183006)(589.96166114,43.64830067)(589.98165771,43.56830373)
\curveto(590.03166107,43.40830091)(590.096661,43.25830106)(590.17665771,43.11830373)
\curveto(590.24666085,42.97830134)(590.33666076,42.85830146)(590.44665771,42.75830373)
\curveto(590.55666054,42.65830166)(590.69166041,42.58330173)(590.85165771,42.53330373)
\curveto(591.0016601,42.48330183)(591.18665991,42.46330185)(591.40665771,42.47330373)
\curveto(591.50665959,42.47330184)(591.6016595,42.48830183)(591.69165771,42.51830373)
\curveto(591.77165933,42.55830176)(591.84665925,42.60330171)(591.91665771,42.65330373)
\curveto(592.02665907,42.73330158)(592.12165898,42.83830148)(592.20165771,42.96830373)
\curveto(592.27165883,43.09830122)(592.33165877,43.23830108)(592.38165771,43.38830373)
\curveto(592.39165871,43.43830088)(592.3966587,43.48830083)(592.39665771,43.53830373)
\curveto(592.3966587,43.58830073)(592.4016587,43.63830068)(592.41165771,43.68830373)
\curveto(592.43165867,43.75830056)(592.44665865,43.84330047)(592.45665771,43.94330373)
\curveto(592.45665864,44.05330026)(592.44665865,44.14330017)(592.42665771,44.21330373)
\curveto(592.40665869,44.27330004)(592.4016587,44.33329998)(592.41165771,44.39330373)
\curveto(592.41165869,44.45329986)(592.4016587,44.5132998)(592.38165771,44.57330373)
\curveto(592.36165874,44.65329966)(592.34665875,44.72829959)(592.33665771,44.79830373)
\curveto(592.32665877,44.87829944)(592.30665879,44.95329936)(592.27665771,45.02330373)
\curveto(592.15665894,45.313299)(592.01165909,45.55829876)(591.84165771,45.75830373)
\curveto(591.67165943,45.96829835)(591.44165966,46.12829819)(591.15165771,46.23830373)
}
}
{
\newrgbcolor{curcolor}{0 0 0}
\pscustom[linestyle=none,fillstyle=solid,fillcolor=curcolor]
{
\newpath
\moveto(585.64665771,49.26994436)
\lineto(585.64665771,49.71994436)
\curveto(585.63666546,49.88994311)(585.65666544,50.01494298)(585.70665771,50.09494436)
\curveto(585.75666534,50.17494282)(585.82166528,50.22994277)(585.90165771,50.25994436)
\curveto(585.98166512,50.2999427)(586.06666503,50.33994266)(586.15665771,50.37994436)
\curveto(586.28666481,50.42994257)(586.41666468,50.47494252)(586.54665771,50.51494436)
\curveto(586.67666442,50.55494244)(586.80666429,50.5999424)(586.93665771,50.64994436)
\curveto(587.05666404,50.6999423)(587.18166392,50.74494225)(587.31165771,50.78494436)
\curveto(587.43166367,50.82494217)(587.55166355,50.86994213)(587.67165771,50.91994436)
\curveto(587.78166332,50.96994203)(587.8966632,51.00994199)(588.01665771,51.03994436)
\curveto(588.13666296,51.06994193)(588.25666284,51.10994189)(588.37665771,51.15994436)
\curveto(588.66666243,51.27994172)(588.96666213,51.38994161)(589.27665771,51.48994436)
\curveto(589.58666151,51.58994141)(589.88666121,51.6999413)(590.17665771,51.81994436)
\curveto(590.21666088,51.83994116)(590.25666084,51.84994115)(590.29665771,51.84994436)
\curveto(590.32666077,51.84994115)(590.35666074,51.85994114)(590.38665771,51.87994436)
\curveto(590.52666057,51.93994106)(590.67166043,51.994941)(590.82165771,52.04494436)
\lineto(591.24165771,52.19494436)
\curveto(591.31165979,52.22494077)(591.38665971,52.25494074)(591.46665771,52.28494436)
\curveto(591.53665956,52.31494068)(591.58165952,52.36494063)(591.60165771,52.43494436)
\curveto(591.63165947,52.51494048)(591.60665949,52.57494042)(591.52665771,52.61494436)
\curveto(591.43665966,52.66494033)(591.36665973,52.6999403)(591.31665771,52.71994436)
\curveto(591.14665995,52.7999402)(590.96666013,52.86494013)(590.77665771,52.91494436)
\curveto(590.58666051,52.96494003)(590.4016607,53.02493997)(590.22165771,53.09494436)
\curveto(589.99166111,53.18493981)(589.76166134,53.26493973)(589.53165771,53.33494436)
\curveto(589.29166181,53.40493959)(589.06166204,53.48993951)(588.84165771,53.58994436)
\curveto(588.79166231,53.5999394)(588.72666237,53.61493938)(588.64665771,53.63494436)
\curveto(588.55666254,53.67493932)(588.46666263,53.70993929)(588.37665771,53.73994436)
\curveto(588.27666282,53.76993923)(588.18666291,53.7999392)(588.10665771,53.82994436)
\curveto(588.05666304,53.84993915)(588.01166309,53.86493913)(587.97165771,53.87494436)
\curveto(587.93166317,53.88493911)(587.88666321,53.8999391)(587.83665771,53.91994436)
\curveto(587.71666338,53.96993903)(587.5966635,54.00993899)(587.47665771,54.03994436)
\curveto(587.34666375,54.07993892)(587.22166388,54.12493887)(587.10165771,54.17494436)
\curveto(587.05166405,54.1949388)(587.00666409,54.20993879)(586.96665771,54.21994436)
\curveto(586.92666417,54.22993877)(586.88166422,54.24493875)(586.83165771,54.26494436)
\curveto(586.74166436,54.30493869)(586.65166445,54.33993866)(586.56165771,54.36994436)
\curveto(586.46166464,54.3999386)(586.36666473,54.42993857)(586.27665771,54.45994436)
\curveto(586.1966649,54.48993851)(586.11666498,54.51493848)(586.03665771,54.53494436)
\curveto(585.94666515,54.56493843)(585.87166523,54.60493839)(585.81165771,54.65494436)
\curveto(585.72166538,54.72493827)(585.67166543,54.81993818)(585.66165771,54.93994436)
\curveto(585.65166545,55.06993793)(585.64666545,55.20993779)(585.64665771,55.35994436)
\curveto(585.64666545,55.43993756)(585.65166545,55.51493748)(585.66165771,55.58494436)
\curveto(585.66166544,55.66493733)(585.67666542,55.72993727)(585.70665771,55.77994436)
\curveto(585.76666533,55.86993713)(585.86166524,55.8949371)(585.99165771,55.85494436)
\curveto(586.12166498,55.81493718)(586.22166488,55.77993722)(586.29165771,55.74994436)
\lineto(586.35165771,55.71994436)
\curveto(586.37166473,55.71993728)(586.39166471,55.71493728)(586.41165771,55.70494436)
\curveto(586.69166441,55.5949374)(586.97666412,55.48493751)(587.26665771,55.37494436)
\lineto(588.10665771,55.04494436)
\curveto(588.18666291,55.01493798)(588.26166284,54.98993801)(588.33165771,54.96994436)
\curveto(588.39166271,54.94993805)(588.45666264,54.92493807)(588.52665771,54.89494436)
\curveto(588.72666237,54.81493818)(588.93166217,54.73493826)(589.14165771,54.65494436)
\curveto(589.34166176,54.58493841)(589.54166156,54.50993849)(589.74165771,54.42994436)
\curveto(590.43166067,54.13993886)(591.12665997,53.86993913)(591.82665771,53.61994436)
\curveto(592.52665857,53.36993963)(593.22165788,53.0999399)(593.91165771,52.80994436)
\lineto(594.06165771,52.74994436)
\curveto(594.12165698,52.73994026)(594.18165692,52.72494027)(594.24165771,52.70494436)
\curveto(594.61165649,52.54494045)(594.97665612,52.37494062)(595.33665771,52.19494436)
\curveto(595.70665539,52.01494098)(595.99165511,51.76494123)(596.19165771,51.44494436)
\curveto(596.25165485,51.33494166)(596.2966548,51.22494177)(596.32665771,51.11494436)
\curveto(596.36665473,51.00494199)(596.4016547,50.87994212)(596.43165771,50.73994436)
\curveto(596.45165465,50.68994231)(596.45665464,50.63494236)(596.44665771,50.57494436)
\curveto(596.43665466,50.52494247)(596.43665466,50.46994253)(596.44665771,50.40994436)
\curveto(596.46665463,50.32994267)(596.46665463,50.24994275)(596.44665771,50.16994436)
\curveto(596.43665466,50.12994287)(596.43165467,50.07994292)(596.43165771,50.01994436)
\lineto(596.37165771,49.77994436)
\curveto(596.35165475,49.70994329)(596.31165479,49.65494334)(596.25165771,49.61494436)
\curveto(596.19165491,49.56494343)(596.11665498,49.53494346)(596.02665771,49.52494436)
\lineto(595.75665771,49.52494436)
\lineto(595.54665771,49.52494436)
\curveto(595.48665561,49.53494346)(595.43665566,49.55494344)(595.39665771,49.58494436)
\curveto(595.28665581,49.65494334)(595.25665584,49.77494322)(595.30665771,49.94494436)
\curveto(595.32665577,50.05494294)(595.33665576,50.17494282)(595.33665771,50.30494436)
\curveto(595.33665576,50.43494256)(595.31665578,50.54994245)(595.27665771,50.64994436)
\curveto(595.22665587,50.7999422)(595.15165595,50.91994208)(595.05165771,51.00994436)
\curveto(594.95165615,51.10994189)(594.83665626,51.1949418)(594.70665771,51.26494436)
\curveto(594.58665651,51.33494166)(594.45665664,51.3949416)(594.31665771,51.44494436)
\lineto(593.89665771,51.62494436)
\curveto(593.80665729,51.66494133)(593.6966574,51.70494129)(593.56665771,51.74494436)
\curveto(593.43665766,51.7949412)(593.3016578,51.7999412)(593.16165771,51.75994436)
\curveto(593.0016581,51.70994129)(592.85165825,51.65494134)(592.71165771,51.59494436)
\curveto(592.57165853,51.54494145)(592.43165867,51.48994151)(592.29165771,51.42994436)
\curveto(592.08165902,51.33994166)(591.87165923,51.25494174)(591.66165771,51.17494436)
\curveto(591.45165965,51.0949419)(591.24665985,51.01494198)(591.04665771,50.93494436)
\curveto(590.90666019,50.87494212)(590.77166033,50.81994218)(590.64165771,50.76994436)
\curveto(590.51166059,50.71994228)(590.37666072,50.66994233)(590.23665771,50.61994436)
\lineto(588.91665771,50.07994436)
\curveto(588.47666262,49.90994309)(588.03666306,49.73494326)(587.59665771,49.55494436)
\curveto(587.36666373,49.45494354)(587.14666395,49.36494363)(586.93665771,49.28494436)
\curveto(586.71666438,49.20494379)(586.4966646,49.11994388)(586.27665771,49.02994436)
\curveto(586.21666488,49.00994399)(586.13666496,48.97994402)(586.03665771,48.93994436)
\curveto(585.92666517,48.8999441)(585.83666526,48.90494409)(585.76665771,48.95494436)
\curveto(585.71666538,48.98494401)(585.68166542,49.04494395)(585.66165771,49.13494436)
\curveto(585.65166545,49.15494384)(585.65166545,49.17494382)(585.66165771,49.19494436)
\curveto(585.66166544,49.22494377)(585.65666544,49.24994375)(585.64665771,49.26994436)
}
}
{
\newrgbcolor{curcolor}{0 0 0}
\pscustom[linestyle=none,fillstyle=solid,fillcolor=curcolor]
{
}
}
{
\newrgbcolor{curcolor}{0 0 0}
\pscustom[linestyle=none,fillstyle=solid,fillcolor=curcolor]
{
\newpath
\moveto(582.76665771,64.24510061)
\curveto(582.75666834,64.93509597)(582.87666822,65.53509537)(583.12665771,66.04510061)
\curveto(583.37666772,66.56509434)(583.71166739,66.96009395)(584.13165771,67.23010061)
\curveto(584.21166689,67.28009363)(584.3016668,67.32509358)(584.40165771,67.36510061)
\curveto(584.49166661,67.4050935)(584.58666651,67.45009346)(584.68665771,67.50010061)
\curveto(584.78666631,67.54009337)(584.88666621,67.57009334)(584.98665771,67.59010061)
\curveto(585.08666601,67.6100933)(585.19166591,67.63009328)(585.30165771,67.65010061)
\curveto(585.35166575,67.67009324)(585.3966657,67.67509323)(585.43665771,67.66510061)
\curveto(585.47666562,67.65509325)(585.52166558,67.66009325)(585.57165771,67.68010061)
\curveto(585.62166548,67.69009322)(585.70666539,67.69509321)(585.82665771,67.69510061)
\curveto(585.93666516,67.69509321)(586.02166508,67.69009322)(586.08165771,67.68010061)
\curveto(586.14166496,67.66009325)(586.2016649,67.65009326)(586.26165771,67.65010061)
\curveto(586.32166478,67.66009325)(586.38166472,67.65509325)(586.44165771,67.63510061)
\curveto(586.58166452,67.59509331)(586.71666438,67.56009335)(586.84665771,67.53010061)
\curveto(586.97666412,67.50009341)(587.101664,67.46009345)(587.22165771,67.41010061)
\curveto(587.36166374,67.35009356)(587.48666361,67.28009363)(587.59665771,67.20010061)
\curveto(587.70666339,67.13009378)(587.81666328,67.05509385)(587.92665771,66.97510061)
\lineto(587.98665771,66.91510061)
\curveto(588.00666309,66.905094)(588.02666307,66.89009402)(588.04665771,66.87010061)
\curveto(588.20666289,66.75009416)(588.35166275,66.61509429)(588.48165771,66.46510061)
\curveto(588.61166249,66.31509459)(588.73666236,66.15509475)(588.85665771,65.98510061)
\curveto(589.07666202,65.67509523)(589.28166182,65.38009553)(589.47165771,65.10010061)
\curveto(589.61166149,64.87009604)(589.74666135,64.64009627)(589.87665771,64.41010061)
\curveto(590.00666109,64.19009672)(590.14166096,63.97009694)(590.28165771,63.75010061)
\curveto(590.45166065,63.50009741)(590.63166047,63.26009765)(590.82165771,63.03010061)
\curveto(591.01166009,62.8100981)(591.23665986,62.62009829)(591.49665771,62.46010061)
\curveto(591.55665954,62.42009849)(591.61665948,62.38509852)(591.67665771,62.35510061)
\curveto(591.72665937,62.32509858)(591.79165931,62.29509861)(591.87165771,62.26510061)
\curveto(591.94165916,62.24509866)(592.0016591,62.24009867)(592.05165771,62.25010061)
\curveto(592.12165898,62.27009864)(592.17665892,62.3050986)(592.21665771,62.35510061)
\curveto(592.24665885,62.4050985)(592.26665883,62.46509844)(592.27665771,62.53510061)
\lineto(592.27665771,62.77510061)
\lineto(592.27665771,63.52510061)
\lineto(592.27665771,66.33010061)
\lineto(592.27665771,66.99010061)
\curveto(592.27665882,67.08009383)(592.28165882,67.16509374)(592.29165771,67.24510061)
\curveto(592.29165881,67.32509358)(592.31165879,67.39009352)(592.35165771,67.44010061)
\curveto(592.39165871,67.49009342)(592.46665863,67.53009338)(592.57665771,67.56010061)
\curveto(592.67665842,67.60009331)(592.77665832,67.6100933)(592.87665771,67.59010061)
\lineto(593.01165771,67.59010061)
\curveto(593.08165802,67.57009334)(593.14165796,67.55009336)(593.19165771,67.53010061)
\curveto(593.24165786,67.5100934)(593.28165782,67.47509343)(593.31165771,67.42510061)
\curveto(593.35165775,67.37509353)(593.37165773,67.3050936)(593.37165771,67.21510061)
\lineto(593.37165771,66.94510061)
\lineto(593.37165771,66.04510061)
\lineto(593.37165771,62.53510061)
\lineto(593.37165771,61.47010061)
\curveto(593.37165773,61.39009952)(593.37665772,61.30009961)(593.38665771,61.20010061)
\curveto(593.38665771,61.10009981)(593.37665772,61.01509989)(593.35665771,60.94510061)
\curveto(593.28665781,60.73510017)(593.10665799,60.67010024)(592.81665771,60.75010061)
\curveto(592.77665832,60.76010015)(592.74165836,60.76010015)(592.71165771,60.75010061)
\curveto(592.67165843,60.75010016)(592.62665847,60.76010015)(592.57665771,60.78010061)
\curveto(592.4966586,60.80010011)(592.41165869,60.82010009)(592.32165771,60.84010061)
\curveto(592.23165887,60.86010005)(592.14665895,60.88510002)(592.06665771,60.91510061)
\curveto(591.57665952,61.07509983)(591.16165994,61.27509963)(590.82165771,61.51510061)
\curveto(590.57166053,61.69509921)(590.34666075,61.90009901)(590.14665771,62.13010061)
\curveto(589.93666116,62.36009855)(589.74166136,62.60009831)(589.56165771,62.85010061)
\curveto(589.38166172,63.1100978)(589.21166189,63.37509753)(589.05165771,63.64510061)
\curveto(588.88166222,63.92509698)(588.70666239,64.19509671)(588.52665771,64.45510061)
\curveto(588.44666265,64.56509634)(588.37166273,64.67009624)(588.30165771,64.77010061)
\curveto(588.23166287,64.88009603)(588.15666294,64.99009592)(588.07665771,65.10010061)
\curveto(588.04666305,65.14009577)(588.01666308,65.17509573)(587.98665771,65.20510061)
\curveto(587.94666315,65.24509566)(587.91666318,65.28509562)(587.89665771,65.32510061)
\curveto(587.78666331,65.46509544)(587.66166344,65.59009532)(587.52165771,65.70010061)
\curveto(587.49166361,65.72009519)(587.46666363,65.74509516)(587.44665771,65.77510061)
\curveto(587.41666368,65.8050951)(587.38666371,65.83009508)(587.35665771,65.85010061)
\curveto(587.25666384,65.93009498)(587.15666394,65.99509491)(587.05665771,66.04510061)
\curveto(586.95666414,66.1050948)(586.84666425,66.16009475)(586.72665771,66.21010061)
\curveto(586.65666444,66.24009467)(586.58166452,66.26009465)(586.50165771,66.27010061)
\lineto(586.26165771,66.33010061)
\lineto(586.17165771,66.33010061)
\curveto(586.14166496,66.34009457)(586.11166499,66.34509456)(586.08165771,66.34510061)
\curveto(586.01166509,66.36509454)(585.91666518,66.37009454)(585.79665771,66.36010061)
\curveto(585.66666543,66.36009455)(585.56666553,66.35009456)(585.49665771,66.33010061)
\curveto(585.41666568,66.3100946)(585.34166576,66.29009462)(585.27165771,66.27010061)
\curveto(585.19166591,66.26009465)(585.11166599,66.24009467)(585.03165771,66.21010061)
\curveto(584.79166631,66.10009481)(584.59166651,65.95009496)(584.43165771,65.76010061)
\curveto(584.26166684,65.58009533)(584.12166698,65.36009555)(584.01165771,65.10010061)
\curveto(583.99166711,65.03009588)(583.97666712,64.96009595)(583.96665771,64.89010061)
\curveto(583.94666715,64.82009609)(583.92666717,64.74509616)(583.90665771,64.66510061)
\curveto(583.88666721,64.58509632)(583.87666722,64.47509643)(583.87665771,64.33510061)
\curveto(583.87666722,64.2050967)(583.88666721,64.10009681)(583.90665771,64.02010061)
\curveto(583.91666718,63.96009695)(583.92166718,63.905097)(583.92165771,63.85510061)
\curveto(583.92166718,63.8050971)(583.93166717,63.75509715)(583.95165771,63.70510061)
\curveto(583.99166711,63.6050973)(584.03166707,63.5100974)(584.07165771,63.42010061)
\curveto(584.11166699,63.34009757)(584.15666694,63.26009765)(584.20665771,63.18010061)
\curveto(584.22666687,63.15009776)(584.25166685,63.12009779)(584.28165771,63.09010061)
\curveto(584.31166679,63.07009784)(584.33666676,63.04509786)(584.35665771,63.01510061)
\lineto(584.43165771,62.94010061)
\curveto(584.45166665,62.910098)(584.47166663,62.88509802)(584.49165771,62.86510061)
\lineto(584.70165771,62.71510061)
\curveto(584.76166634,62.67509823)(584.82666627,62.63009828)(584.89665771,62.58010061)
\curveto(584.98666611,62.52009839)(585.09166601,62.47009844)(585.21165771,62.43010061)
\curveto(585.32166578,62.40009851)(585.43166567,62.36509854)(585.54165771,62.32510061)
\curveto(585.65166545,62.28509862)(585.7966653,62.26009865)(585.97665771,62.25010061)
\curveto(586.14666495,62.24009867)(586.27166483,62.2100987)(586.35165771,62.16010061)
\curveto(586.43166467,62.1100988)(586.47666462,62.03509887)(586.48665771,61.93510061)
\curveto(586.4966646,61.83509907)(586.5016646,61.72509918)(586.50165771,61.60510061)
\curveto(586.5016646,61.56509934)(586.50666459,61.52509938)(586.51665771,61.48510061)
\curveto(586.51666458,61.44509946)(586.51166459,61.4100995)(586.50165771,61.38010061)
\curveto(586.48166462,61.33009958)(586.47166463,61.28009963)(586.47165771,61.23010061)
\curveto(586.47166463,61.19009972)(586.46166464,61.15009976)(586.44165771,61.11010061)
\curveto(586.38166472,61.02009989)(586.24666485,60.97509993)(586.03665771,60.97510061)
\lineto(585.91665771,60.97510061)
\curveto(585.85666524,60.98509992)(585.7966653,60.99009992)(585.73665771,60.99010061)
\curveto(585.66666543,61.00009991)(585.6016655,61.0100999)(585.54165771,61.02010061)
\curveto(585.43166567,61.04009987)(585.33166577,61.06009985)(585.24165771,61.08010061)
\curveto(585.14166596,61.10009981)(585.04666605,61.13009978)(584.95665771,61.17010061)
\curveto(584.88666621,61.19009972)(584.82666627,61.2100997)(584.77665771,61.23010061)
\lineto(584.59665771,61.29010061)
\curveto(584.33666676,61.4100995)(584.09166701,61.56509934)(583.86165771,61.75510061)
\curveto(583.63166747,61.95509895)(583.44666765,62.17009874)(583.30665771,62.40010061)
\curveto(583.22666787,62.5100984)(583.16166794,62.62509828)(583.11165771,62.74510061)
\lineto(582.96165771,63.13510061)
\curveto(582.91166819,63.24509766)(582.88166822,63.36009755)(582.87165771,63.48010061)
\curveto(582.85166825,63.60009731)(582.82666827,63.72509718)(582.79665771,63.85510061)
\curveto(582.7966683,63.92509698)(582.7966683,63.99009692)(582.79665771,64.05010061)
\curveto(582.78666831,64.1100968)(582.77666832,64.17509673)(582.76665771,64.24510061)
}
}
{
\newrgbcolor{curcolor}{0 0 0}
\pscustom[linestyle=none,fillstyle=solid,fillcolor=curcolor]
{
\newpath
\moveto(582.76665771,73.40470998)
\curveto(582.76666833,73.50470513)(582.77666832,73.59970503)(582.79665771,73.68970998)
\curveto(582.80666829,73.77970485)(582.83666826,73.84470479)(582.88665771,73.88470998)
\curveto(582.96666813,73.94470469)(583.07166803,73.97470466)(583.20165771,73.97470998)
\lineto(583.59165771,73.97470998)
\lineto(585.09165771,73.97470998)
\lineto(591.48165771,73.97470998)
\lineto(592.65165771,73.97470998)
\lineto(592.96665771,73.97470998)
\curveto(593.06665803,73.98470465)(593.14665795,73.96970466)(593.20665771,73.92970998)
\curveto(593.28665781,73.87970475)(593.33665776,73.80470483)(593.35665771,73.70470998)
\curveto(593.36665773,73.61470502)(593.37165773,73.50470513)(593.37165771,73.37470998)
\lineto(593.37165771,73.14970998)
\curveto(593.35165775,73.06970556)(593.33665776,72.99970563)(593.32665771,72.93970998)
\curveto(593.30665779,72.87970575)(593.26665783,72.8297058)(593.20665771,72.78970998)
\curveto(593.14665795,72.74970588)(593.07165803,72.7297059)(592.98165771,72.72970998)
\lineto(592.68165771,72.72970998)
\lineto(591.58665771,72.72970998)
\lineto(586.24665771,72.72970998)
\curveto(586.15666494,72.70970592)(586.08166502,72.69470594)(586.02165771,72.68470998)
\curveto(585.95166515,72.68470595)(585.89166521,72.65470598)(585.84165771,72.59470998)
\curveto(585.79166531,72.52470611)(585.76666533,72.4347062)(585.76665771,72.32470998)
\curveto(585.75666534,72.22470641)(585.75166535,72.11470652)(585.75165771,71.99470998)
\lineto(585.75165771,70.85470998)
\lineto(585.75165771,70.35970998)
\curveto(585.74166536,70.19970843)(585.68166542,70.08970854)(585.57165771,70.02970998)
\curveto(585.54166556,70.00970862)(585.51166559,69.99970863)(585.48165771,69.99970998)
\curveto(585.44166566,69.99970863)(585.3966657,69.99470864)(585.34665771,69.98470998)
\curveto(585.22666587,69.96470867)(585.11666598,69.96970866)(585.01665771,69.99970998)
\curveto(584.91666618,70.03970859)(584.84666625,70.09470854)(584.80665771,70.16470998)
\curveto(584.75666634,70.24470839)(584.73166637,70.36470827)(584.73165771,70.52470998)
\curveto(584.73166637,70.68470795)(584.71666638,70.81970781)(584.68665771,70.92970998)
\curveto(584.67666642,70.97970765)(584.67166643,71.0347076)(584.67165771,71.09470998)
\curveto(584.66166644,71.15470748)(584.64666645,71.21470742)(584.62665771,71.27470998)
\curveto(584.57666652,71.42470721)(584.52666657,71.56970706)(584.47665771,71.70970998)
\curveto(584.41666668,71.84970678)(584.34666675,71.98470665)(584.26665771,72.11470998)
\curveto(584.17666692,72.25470638)(584.07166703,72.37470626)(583.95165771,72.47470998)
\curveto(583.83166727,72.57470606)(583.7016674,72.66970596)(583.56165771,72.75970998)
\curveto(583.46166764,72.81970581)(583.35166775,72.86470577)(583.23165771,72.89470998)
\curveto(583.11166799,72.9347057)(583.00666809,72.98470565)(582.91665771,73.04470998)
\curveto(582.85666824,73.09470554)(582.81666828,73.16470547)(582.79665771,73.25470998)
\curveto(582.78666831,73.27470536)(582.78166832,73.29970533)(582.78165771,73.32970998)
\curveto(582.78166832,73.35970527)(582.77666832,73.38470525)(582.76665771,73.40470998)
}
}
{
\newrgbcolor{curcolor}{0 0 0}
\pscustom[linestyle=none,fillstyle=solid,fillcolor=curcolor]
{
\newpath
\moveto(591.73665771,78.63431936)
\lineto(591.73665771,79.26431936)
\lineto(591.73665771,79.45931936)
\curveto(591.73665936,79.52931683)(591.74665935,79.58931677)(591.76665771,79.63931936)
\curveto(591.80665929,79.70931665)(591.84665925,79.7593166)(591.88665771,79.78931936)
\curveto(591.93665916,79.82931653)(592.0016591,79.84931651)(592.08165771,79.84931936)
\curveto(592.16165894,79.8593165)(592.24665885,79.86431649)(592.33665771,79.86431936)
\lineto(593.05665771,79.86431936)
\curveto(593.53665756,79.86431649)(593.94665715,79.80431655)(594.28665771,79.68431936)
\curveto(594.62665647,79.56431679)(594.9016562,79.36931699)(595.11165771,79.09931936)
\curveto(595.16165594,79.02931733)(595.20665589,78.9593174)(595.24665771,78.88931936)
\curveto(595.2966558,78.82931753)(595.34165576,78.7543176)(595.38165771,78.66431936)
\curveto(595.39165571,78.64431771)(595.4016557,78.61431774)(595.41165771,78.57431936)
\curveto(595.43165567,78.53431782)(595.43665566,78.48931787)(595.42665771,78.43931936)
\curveto(595.3966557,78.34931801)(595.32165578,78.29431806)(595.20165771,78.27431936)
\curveto(595.09165601,78.2543181)(594.9966561,78.26931809)(594.91665771,78.31931936)
\curveto(594.84665625,78.34931801)(594.78165632,78.39431796)(594.72165771,78.45431936)
\curveto(594.67165643,78.52431783)(594.62165648,78.58931777)(594.57165771,78.64931936)
\curveto(594.52165658,78.71931764)(594.44665665,78.77931758)(594.34665771,78.82931936)
\curveto(594.25665684,78.88931747)(594.16665693,78.93931742)(594.07665771,78.97931936)
\curveto(594.04665705,78.99931736)(593.98665711,79.02431733)(593.89665771,79.05431936)
\curveto(593.81665728,79.08431727)(593.74665735,79.08931727)(593.68665771,79.06931936)
\curveto(593.54665755,79.03931732)(593.45665764,78.97931738)(593.41665771,78.88931936)
\curveto(593.38665771,78.80931755)(593.37165773,78.71931764)(593.37165771,78.61931936)
\curveto(593.37165773,78.51931784)(593.34665775,78.43431792)(593.29665771,78.36431936)
\curveto(593.22665787,78.27431808)(593.08665801,78.22931813)(592.87665771,78.22931936)
\lineto(592.32165771,78.22931936)
\lineto(592.09665771,78.22931936)
\curveto(592.01665908,78.23931812)(591.95165915,78.2593181)(591.90165771,78.28931936)
\curveto(591.82165928,78.34931801)(591.77665932,78.41931794)(591.76665771,78.49931936)
\curveto(591.75665934,78.51931784)(591.75165935,78.53931782)(591.75165771,78.55931936)
\curveto(591.75165935,78.58931777)(591.74665935,78.61431774)(591.73665771,78.63431936)
}
}
{
\newrgbcolor{curcolor}{0 0 0}
\pscustom[linestyle=none,fillstyle=solid,fillcolor=curcolor]
{
}
}
{
\newrgbcolor{curcolor}{0 0 0}
\pscustom[linestyle=none,fillstyle=solid,fillcolor=curcolor]
{
\newpath
\moveto(582.76665771,89.26463186)
\curveto(582.75666834,89.95462722)(582.87666822,90.55462662)(583.12665771,91.06463186)
\curveto(583.37666772,91.58462559)(583.71166739,91.9796252)(584.13165771,92.24963186)
\curveto(584.21166689,92.29962488)(584.3016668,92.34462483)(584.40165771,92.38463186)
\curveto(584.49166661,92.42462475)(584.58666651,92.46962471)(584.68665771,92.51963186)
\curveto(584.78666631,92.55962462)(584.88666621,92.58962459)(584.98665771,92.60963186)
\curveto(585.08666601,92.62962455)(585.19166591,92.64962453)(585.30165771,92.66963186)
\curveto(585.35166575,92.68962449)(585.3966657,92.69462448)(585.43665771,92.68463186)
\curveto(585.47666562,92.6746245)(585.52166558,92.6796245)(585.57165771,92.69963186)
\curveto(585.62166548,92.70962447)(585.70666539,92.71462446)(585.82665771,92.71463186)
\curveto(585.93666516,92.71462446)(586.02166508,92.70962447)(586.08165771,92.69963186)
\curveto(586.14166496,92.6796245)(586.2016649,92.66962451)(586.26165771,92.66963186)
\curveto(586.32166478,92.6796245)(586.38166472,92.6746245)(586.44165771,92.65463186)
\curveto(586.58166452,92.61462456)(586.71666438,92.5796246)(586.84665771,92.54963186)
\curveto(586.97666412,92.51962466)(587.101664,92.4796247)(587.22165771,92.42963186)
\curveto(587.36166374,92.36962481)(587.48666361,92.29962488)(587.59665771,92.21963186)
\curveto(587.70666339,92.14962503)(587.81666328,92.0746251)(587.92665771,91.99463186)
\lineto(587.98665771,91.93463186)
\curveto(588.00666309,91.92462525)(588.02666307,91.90962527)(588.04665771,91.88963186)
\curveto(588.20666289,91.76962541)(588.35166275,91.63462554)(588.48165771,91.48463186)
\curveto(588.61166249,91.33462584)(588.73666236,91.174626)(588.85665771,91.00463186)
\curveto(589.07666202,90.69462648)(589.28166182,90.39962678)(589.47165771,90.11963186)
\curveto(589.61166149,89.88962729)(589.74666135,89.65962752)(589.87665771,89.42963186)
\curveto(590.00666109,89.20962797)(590.14166096,88.98962819)(590.28165771,88.76963186)
\curveto(590.45166065,88.51962866)(590.63166047,88.2796289)(590.82165771,88.04963186)
\curveto(591.01166009,87.82962935)(591.23665986,87.63962954)(591.49665771,87.47963186)
\curveto(591.55665954,87.43962974)(591.61665948,87.40462977)(591.67665771,87.37463186)
\curveto(591.72665937,87.34462983)(591.79165931,87.31462986)(591.87165771,87.28463186)
\curveto(591.94165916,87.26462991)(592.0016591,87.25962992)(592.05165771,87.26963186)
\curveto(592.12165898,87.28962989)(592.17665892,87.32462985)(592.21665771,87.37463186)
\curveto(592.24665885,87.42462975)(592.26665883,87.48462969)(592.27665771,87.55463186)
\lineto(592.27665771,87.79463186)
\lineto(592.27665771,88.54463186)
\lineto(592.27665771,91.34963186)
\lineto(592.27665771,92.00963186)
\curveto(592.27665882,92.09962508)(592.28165882,92.18462499)(592.29165771,92.26463186)
\curveto(592.29165881,92.34462483)(592.31165879,92.40962477)(592.35165771,92.45963186)
\curveto(592.39165871,92.50962467)(592.46665863,92.54962463)(592.57665771,92.57963186)
\curveto(592.67665842,92.61962456)(592.77665832,92.62962455)(592.87665771,92.60963186)
\lineto(593.01165771,92.60963186)
\curveto(593.08165802,92.58962459)(593.14165796,92.56962461)(593.19165771,92.54963186)
\curveto(593.24165786,92.52962465)(593.28165782,92.49462468)(593.31165771,92.44463186)
\curveto(593.35165775,92.39462478)(593.37165773,92.32462485)(593.37165771,92.23463186)
\lineto(593.37165771,91.96463186)
\lineto(593.37165771,91.06463186)
\lineto(593.37165771,87.55463186)
\lineto(593.37165771,86.48963186)
\curveto(593.37165773,86.40963077)(593.37665772,86.31963086)(593.38665771,86.21963186)
\curveto(593.38665771,86.11963106)(593.37665772,86.03463114)(593.35665771,85.96463186)
\curveto(593.28665781,85.75463142)(593.10665799,85.68963149)(592.81665771,85.76963186)
\curveto(592.77665832,85.7796314)(592.74165836,85.7796314)(592.71165771,85.76963186)
\curveto(592.67165843,85.76963141)(592.62665847,85.7796314)(592.57665771,85.79963186)
\curveto(592.4966586,85.81963136)(592.41165869,85.83963134)(592.32165771,85.85963186)
\curveto(592.23165887,85.8796313)(592.14665895,85.90463127)(592.06665771,85.93463186)
\curveto(591.57665952,86.09463108)(591.16165994,86.29463088)(590.82165771,86.53463186)
\curveto(590.57166053,86.71463046)(590.34666075,86.91963026)(590.14665771,87.14963186)
\curveto(589.93666116,87.3796298)(589.74166136,87.61962956)(589.56165771,87.86963186)
\curveto(589.38166172,88.12962905)(589.21166189,88.39462878)(589.05165771,88.66463186)
\curveto(588.88166222,88.94462823)(588.70666239,89.21462796)(588.52665771,89.47463186)
\curveto(588.44666265,89.58462759)(588.37166273,89.68962749)(588.30165771,89.78963186)
\curveto(588.23166287,89.89962728)(588.15666294,90.00962717)(588.07665771,90.11963186)
\curveto(588.04666305,90.15962702)(588.01666308,90.19462698)(587.98665771,90.22463186)
\curveto(587.94666315,90.26462691)(587.91666318,90.30462687)(587.89665771,90.34463186)
\curveto(587.78666331,90.48462669)(587.66166344,90.60962657)(587.52165771,90.71963186)
\curveto(587.49166361,90.73962644)(587.46666363,90.76462641)(587.44665771,90.79463186)
\curveto(587.41666368,90.82462635)(587.38666371,90.84962633)(587.35665771,90.86963186)
\curveto(587.25666384,90.94962623)(587.15666394,91.01462616)(587.05665771,91.06463186)
\curveto(586.95666414,91.12462605)(586.84666425,91.179626)(586.72665771,91.22963186)
\curveto(586.65666444,91.25962592)(586.58166452,91.2796259)(586.50165771,91.28963186)
\lineto(586.26165771,91.34963186)
\lineto(586.17165771,91.34963186)
\curveto(586.14166496,91.35962582)(586.11166499,91.36462581)(586.08165771,91.36463186)
\curveto(586.01166509,91.38462579)(585.91666518,91.38962579)(585.79665771,91.37963186)
\curveto(585.66666543,91.3796258)(585.56666553,91.36962581)(585.49665771,91.34963186)
\curveto(585.41666568,91.32962585)(585.34166576,91.30962587)(585.27165771,91.28963186)
\curveto(585.19166591,91.2796259)(585.11166599,91.25962592)(585.03165771,91.22963186)
\curveto(584.79166631,91.11962606)(584.59166651,90.96962621)(584.43165771,90.77963186)
\curveto(584.26166684,90.59962658)(584.12166698,90.3796268)(584.01165771,90.11963186)
\curveto(583.99166711,90.04962713)(583.97666712,89.9796272)(583.96665771,89.90963186)
\curveto(583.94666715,89.83962734)(583.92666717,89.76462741)(583.90665771,89.68463186)
\curveto(583.88666721,89.60462757)(583.87666722,89.49462768)(583.87665771,89.35463186)
\curveto(583.87666722,89.22462795)(583.88666721,89.11962806)(583.90665771,89.03963186)
\curveto(583.91666718,88.9796282)(583.92166718,88.92462825)(583.92165771,88.87463186)
\curveto(583.92166718,88.82462835)(583.93166717,88.7746284)(583.95165771,88.72463186)
\curveto(583.99166711,88.62462855)(584.03166707,88.52962865)(584.07165771,88.43963186)
\curveto(584.11166699,88.35962882)(584.15666694,88.2796289)(584.20665771,88.19963186)
\curveto(584.22666687,88.16962901)(584.25166685,88.13962904)(584.28165771,88.10963186)
\curveto(584.31166679,88.08962909)(584.33666676,88.06462911)(584.35665771,88.03463186)
\lineto(584.43165771,87.95963186)
\curveto(584.45166665,87.92962925)(584.47166663,87.90462927)(584.49165771,87.88463186)
\lineto(584.70165771,87.73463186)
\curveto(584.76166634,87.69462948)(584.82666627,87.64962953)(584.89665771,87.59963186)
\curveto(584.98666611,87.53962964)(585.09166601,87.48962969)(585.21165771,87.44963186)
\curveto(585.32166578,87.41962976)(585.43166567,87.38462979)(585.54165771,87.34463186)
\curveto(585.65166545,87.30462987)(585.7966653,87.2796299)(585.97665771,87.26963186)
\curveto(586.14666495,87.25962992)(586.27166483,87.22962995)(586.35165771,87.17963186)
\curveto(586.43166467,87.12963005)(586.47666462,87.05463012)(586.48665771,86.95463186)
\curveto(586.4966646,86.85463032)(586.5016646,86.74463043)(586.50165771,86.62463186)
\curveto(586.5016646,86.58463059)(586.50666459,86.54463063)(586.51665771,86.50463186)
\curveto(586.51666458,86.46463071)(586.51166459,86.42963075)(586.50165771,86.39963186)
\curveto(586.48166462,86.34963083)(586.47166463,86.29963088)(586.47165771,86.24963186)
\curveto(586.47166463,86.20963097)(586.46166464,86.16963101)(586.44165771,86.12963186)
\curveto(586.38166472,86.03963114)(586.24666485,85.99463118)(586.03665771,85.99463186)
\lineto(585.91665771,85.99463186)
\curveto(585.85666524,86.00463117)(585.7966653,86.00963117)(585.73665771,86.00963186)
\curveto(585.66666543,86.01963116)(585.6016655,86.02963115)(585.54165771,86.03963186)
\curveto(585.43166567,86.05963112)(585.33166577,86.0796311)(585.24165771,86.09963186)
\curveto(585.14166596,86.11963106)(585.04666605,86.14963103)(584.95665771,86.18963186)
\curveto(584.88666621,86.20963097)(584.82666627,86.22963095)(584.77665771,86.24963186)
\lineto(584.59665771,86.30963186)
\curveto(584.33666676,86.42963075)(584.09166701,86.58463059)(583.86165771,86.77463186)
\curveto(583.63166747,86.9746302)(583.44666765,87.18962999)(583.30665771,87.41963186)
\curveto(583.22666787,87.52962965)(583.16166794,87.64462953)(583.11165771,87.76463186)
\lineto(582.96165771,88.15463186)
\curveto(582.91166819,88.26462891)(582.88166822,88.3796288)(582.87165771,88.49963186)
\curveto(582.85166825,88.61962856)(582.82666827,88.74462843)(582.79665771,88.87463186)
\curveto(582.7966683,88.94462823)(582.7966683,89.00962817)(582.79665771,89.06963186)
\curveto(582.78666831,89.12962805)(582.77666832,89.19462798)(582.76665771,89.26463186)
}
}
{
\newrgbcolor{curcolor}{0 0 0}
\pscustom[linestyle=none,fillstyle=solid,fillcolor=curcolor]
{
\newpath
\moveto(588.28665771,101.36424123)
\lineto(588.54165771,101.36424123)
\curveto(588.62166248,101.37423353)(588.6966624,101.36923353)(588.76665771,101.34924123)
\lineto(589.00665771,101.34924123)
\lineto(589.17165771,101.34924123)
\curveto(589.27166183,101.32923357)(589.37666172,101.31923358)(589.48665771,101.31924123)
\curveto(589.58666151,101.31923358)(589.68666141,101.30923359)(589.78665771,101.28924123)
\lineto(589.93665771,101.28924123)
\curveto(590.07666102,101.25923364)(590.21666088,101.23923366)(590.35665771,101.22924123)
\curveto(590.48666061,101.21923368)(590.61666048,101.19423371)(590.74665771,101.15424123)
\curveto(590.82666027,101.13423377)(590.91166019,101.11423379)(591.00165771,101.09424123)
\lineto(591.24165771,101.03424123)
\lineto(591.54165771,100.91424123)
\curveto(591.63165947,100.88423402)(591.72165938,100.84923405)(591.81165771,100.80924123)
\curveto(592.03165907,100.70923419)(592.24665885,100.57423433)(592.45665771,100.40424123)
\curveto(592.66665843,100.24423466)(592.83665826,100.06923483)(592.96665771,99.87924123)
\curveto(593.00665809,99.82923507)(593.04665805,99.76923513)(593.08665771,99.69924123)
\curveto(593.11665798,99.63923526)(593.15165795,99.57923532)(593.19165771,99.51924123)
\curveto(593.24165786,99.43923546)(593.28165782,99.34423556)(593.31165771,99.23424123)
\curveto(593.34165776,99.12423578)(593.37165773,99.01923588)(593.40165771,98.91924123)
\curveto(593.44165766,98.80923609)(593.46665763,98.6992362)(593.47665771,98.58924123)
\curveto(593.48665761,98.47923642)(593.5016576,98.36423654)(593.52165771,98.24424123)
\curveto(593.53165757,98.2042367)(593.53165757,98.15923674)(593.52165771,98.10924123)
\curveto(593.52165758,98.06923683)(593.52665757,98.02923687)(593.53665771,97.98924123)
\curveto(593.54665755,97.94923695)(593.55165755,97.89423701)(593.55165771,97.82424123)
\curveto(593.55165755,97.75423715)(593.54665755,97.7042372)(593.53665771,97.67424123)
\curveto(593.51665758,97.62423728)(593.51165759,97.57923732)(593.52165771,97.53924123)
\curveto(593.53165757,97.4992374)(593.53165757,97.46423744)(593.52165771,97.43424123)
\lineto(593.52165771,97.34424123)
\curveto(593.5016576,97.28423762)(593.48665761,97.21923768)(593.47665771,97.14924123)
\curveto(593.47665762,97.08923781)(593.47165763,97.02423788)(593.46165771,96.95424123)
\curveto(593.41165769,96.78423812)(593.36165774,96.62423828)(593.31165771,96.47424123)
\curveto(593.26165784,96.32423858)(593.1966579,96.17923872)(593.11665771,96.03924123)
\curveto(593.07665802,95.98923891)(593.04665805,95.93423897)(593.02665771,95.87424123)
\curveto(592.9966581,95.82423908)(592.96165814,95.77423913)(592.92165771,95.72424123)
\curveto(592.74165836,95.48423942)(592.52165858,95.28423962)(592.26165771,95.12424123)
\curveto(592.0016591,94.96423994)(591.71665938,94.82424008)(591.40665771,94.70424123)
\curveto(591.26665983,94.64424026)(591.12665997,94.5992403)(590.98665771,94.56924123)
\curveto(590.83666026,94.53924036)(590.68166042,94.5042404)(590.52165771,94.46424123)
\curveto(590.41166069,94.44424046)(590.3016608,94.42924047)(590.19165771,94.41924123)
\curveto(590.08166102,94.40924049)(589.97166113,94.39424051)(589.86165771,94.37424123)
\curveto(589.82166128,94.36424054)(589.78166132,94.35924054)(589.74165771,94.35924123)
\curveto(589.7016614,94.36924053)(589.66166144,94.36924053)(589.62165771,94.35924123)
\curveto(589.57166153,94.34924055)(589.52166158,94.34424056)(589.47165771,94.34424123)
\lineto(589.30665771,94.34424123)
\curveto(589.25666184,94.32424058)(589.20666189,94.31924058)(589.15665771,94.32924123)
\curveto(589.096662,94.33924056)(589.04166206,94.33924056)(588.99165771,94.32924123)
\curveto(588.95166215,94.31924058)(588.90666219,94.31924058)(588.85665771,94.32924123)
\curveto(588.80666229,94.33924056)(588.75666234,94.33424057)(588.70665771,94.31424123)
\curveto(588.63666246,94.29424061)(588.56166254,94.28924061)(588.48165771,94.29924123)
\curveto(588.39166271,94.30924059)(588.30666279,94.31424059)(588.22665771,94.31424123)
\curveto(588.13666296,94.31424059)(588.03666306,94.30924059)(587.92665771,94.29924123)
\curveto(587.80666329,94.28924061)(587.70666339,94.29424061)(587.62665771,94.31424123)
\lineto(587.34165771,94.31424123)
\lineto(586.71165771,94.35924123)
\curveto(586.61166449,94.36924053)(586.51666458,94.37924052)(586.42665771,94.38924123)
\lineto(586.12665771,94.41924123)
\curveto(586.07666502,94.43924046)(586.02666507,94.44424046)(585.97665771,94.43424123)
\curveto(585.91666518,94.43424047)(585.86166524,94.44424046)(585.81165771,94.46424123)
\curveto(585.64166546,94.51424039)(585.47666562,94.55424035)(585.31665771,94.58424123)
\curveto(585.14666595,94.61424029)(584.98666611,94.66424024)(584.83665771,94.73424123)
\curveto(584.37666672,94.92423998)(584.0016671,95.14423976)(583.71165771,95.39424123)
\curveto(583.42166768,95.65423925)(583.17666792,96.01423889)(582.97665771,96.47424123)
\curveto(582.92666817,96.6042383)(582.89166821,96.73423817)(582.87165771,96.86424123)
\curveto(582.85166825,97.0042379)(582.82666827,97.14423776)(582.79665771,97.28424123)
\curveto(582.78666831,97.35423755)(582.78166832,97.41923748)(582.78165771,97.47924123)
\curveto(582.78166832,97.53923736)(582.77666832,97.6042373)(582.76665771,97.67424123)
\curveto(582.74666835,98.5042364)(582.8966682,99.17423573)(583.21665771,99.68424123)
\curveto(583.52666757,100.19423471)(583.96666713,100.57423433)(584.53665771,100.82424123)
\curveto(584.65666644,100.87423403)(584.78166632,100.91923398)(584.91165771,100.95924123)
\curveto(585.04166606,100.9992339)(585.17666592,101.04423386)(585.31665771,101.09424123)
\curveto(585.3966657,101.11423379)(585.48166562,101.12923377)(585.57165771,101.13924123)
\lineto(585.81165771,101.19924123)
\curveto(585.92166518,101.22923367)(586.03166507,101.24423366)(586.14165771,101.24424123)
\curveto(586.25166485,101.25423365)(586.36166474,101.26923363)(586.47165771,101.28924123)
\curveto(586.52166458,101.30923359)(586.56666453,101.31423359)(586.60665771,101.30424123)
\curveto(586.64666445,101.3042336)(586.68666441,101.30923359)(586.72665771,101.31924123)
\curveto(586.77666432,101.32923357)(586.83166427,101.32923357)(586.89165771,101.31924123)
\curveto(586.94166416,101.31923358)(586.99166411,101.32423358)(587.04165771,101.33424123)
\lineto(587.17665771,101.33424123)
\curveto(587.23666386,101.35423355)(587.30666379,101.35423355)(587.38665771,101.33424123)
\curveto(587.45666364,101.32423358)(587.52166358,101.32923357)(587.58165771,101.34924123)
\curveto(587.61166349,101.35923354)(587.65166345,101.36423354)(587.70165771,101.36424123)
\lineto(587.82165771,101.36424123)
\lineto(588.28665771,101.36424123)
\moveto(590.61165771,99.81924123)
\curveto(590.29166081,99.91923498)(589.92666117,99.97923492)(589.51665771,99.99924123)
\curveto(589.10666199,100.01923488)(588.6966624,100.02923487)(588.28665771,100.02924123)
\curveto(587.85666324,100.02923487)(587.43666366,100.01923488)(587.02665771,99.99924123)
\curveto(586.61666448,99.97923492)(586.23166487,99.93423497)(585.87165771,99.86424123)
\curveto(585.51166559,99.79423511)(585.19166591,99.68423522)(584.91165771,99.53424123)
\curveto(584.62166648,99.39423551)(584.38666671,99.1992357)(584.20665771,98.94924123)
\curveto(584.096667,98.78923611)(584.01666708,98.60923629)(583.96665771,98.40924123)
\curveto(583.90666719,98.20923669)(583.87666722,97.96423694)(583.87665771,97.67424123)
\curveto(583.8966672,97.65423725)(583.90666719,97.61923728)(583.90665771,97.56924123)
\curveto(583.8966672,97.51923738)(583.8966672,97.47923742)(583.90665771,97.44924123)
\curveto(583.92666717,97.36923753)(583.94666715,97.29423761)(583.96665771,97.22424123)
\curveto(583.97666712,97.16423774)(583.9966671,97.0992378)(584.02665771,97.02924123)
\curveto(584.14666695,96.75923814)(584.31666678,96.53923836)(584.53665771,96.36924123)
\curveto(584.74666635,96.20923869)(584.99166611,96.07423883)(585.27165771,95.96424123)
\curveto(585.38166572,95.91423899)(585.5016656,95.87423903)(585.63165771,95.84424123)
\curveto(585.75166535,95.82423908)(585.87666522,95.7992391)(586.00665771,95.76924123)
\curveto(586.05666504,95.74923915)(586.11166499,95.73923916)(586.17165771,95.73924123)
\curveto(586.22166488,95.73923916)(586.27166483,95.73423917)(586.32165771,95.72424123)
\curveto(586.41166469,95.71423919)(586.50666459,95.7042392)(586.60665771,95.69424123)
\curveto(586.6966644,95.68423922)(586.79166431,95.67423923)(586.89165771,95.66424123)
\curveto(586.97166413,95.66423924)(587.05666404,95.65923924)(587.14665771,95.64924123)
\lineto(587.38665771,95.64924123)
\lineto(587.56665771,95.64924123)
\curveto(587.5966635,95.63923926)(587.63166347,95.63423927)(587.67165771,95.63424123)
\lineto(587.80665771,95.63424123)
\lineto(588.25665771,95.63424123)
\curveto(588.33666276,95.63423927)(588.42166268,95.62923927)(588.51165771,95.61924123)
\curveto(588.59166251,95.61923928)(588.66666243,95.62923927)(588.73665771,95.64924123)
\lineto(589.00665771,95.64924123)
\curveto(589.02666207,95.64923925)(589.05666204,95.64423926)(589.09665771,95.63424123)
\curveto(589.12666197,95.63423927)(589.15166195,95.63923926)(589.17165771,95.64924123)
\curveto(589.27166183,95.65923924)(589.37166173,95.66423924)(589.47165771,95.66424123)
\curveto(589.56166154,95.67423923)(589.66166144,95.68423922)(589.77165771,95.69424123)
\curveto(589.89166121,95.72423918)(590.01666108,95.73923916)(590.14665771,95.73924123)
\curveto(590.26666083,95.74923915)(590.38166072,95.77423913)(590.49165771,95.81424123)
\curveto(590.79166031,95.89423901)(591.05666004,95.97923892)(591.28665771,96.06924123)
\curveto(591.51665958,96.16923873)(591.73165937,96.31423859)(591.93165771,96.50424123)
\curveto(592.13165897,96.71423819)(592.28165882,96.97923792)(592.38165771,97.29924123)
\curveto(592.4016587,97.33923756)(592.41165869,97.37423753)(592.41165771,97.40424123)
\curveto(592.4016587,97.44423746)(592.40665869,97.48923741)(592.42665771,97.53924123)
\curveto(592.43665866,97.57923732)(592.44665865,97.64923725)(592.45665771,97.74924123)
\curveto(592.46665863,97.85923704)(592.46165864,97.94423696)(592.44165771,98.00424123)
\curveto(592.42165868,98.07423683)(592.41165869,98.14423676)(592.41165771,98.21424123)
\curveto(592.4016587,98.28423662)(592.38665871,98.34923655)(592.36665771,98.40924123)
\curveto(592.30665879,98.60923629)(592.22165888,98.78923611)(592.11165771,98.94924123)
\curveto(592.09165901,98.97923592)(592.07165903,99.0042359)(592.05165771,99.02424123)
\lineto(591.99165771,99.08424123)
\curveto(591.97165913,99.12423578)(591.93165917,99.17423573)(591.87165771,99.23424123)
\curveto(591.73165937,99.33423557)(591.6016595,99.41923548)(591.48165771,99.48924123)
\curveto(591.36165974,99.55923534)(591.21665988,99.62923527)(591.04665771,99.69924123)
\curveto(590.97666012,99.72923517)(590.90666019,99.74923515)(590.83665771,99.75924123)
\curveto(590.76666033,99.77923512)(590.69166041,99.7992351)(590.61165771,99.81924123)
}
}
{
\newrgbcolor{curcolor}{0 0 0}
\pscustom[linestyle=none,fillstyle=solid,fillcolor=curcolor]
{
\newpath
\moveto(582.76665771,106.77385061)
\curveto(582.76666833,106.87384575)(582.77666832,106.96884566)(582.79665771,107.05885061)
\curveto(582.80666829,107.14884548)(582.83666826,107.21384541)(582.88665771,107.25385061)
\curveto(582.96666813,107.31384531)(583.07166803,107.34384528)(583.20165771,107.34385061)
\lineto(583.59165771,107.34385061)
\lineto(585.09165771,107.34385061)
\lineto(591.48165771,107.34385061)
\lineto(592.65165771,107.34385061)
\lineto(592.96665771,107.34385061)
\curveto(593.06665803,107.35384527)(593.14665795,107.33884529)(593.20665771,107.29885061)
\curveto(593.28665781,107.24884538)(593.33665776,107.17384545)(593.35665771,107.07385061)
\curveto(593.36665773,106.98384564)(593.37165773,106.87384575)(593.37165771,106.74385061)
\lineto(593.37165771,106.51885061)
\curveto(593.35165775,106.43884619)(593.33665776,106.36884626)(593.32665771,106.30885061)
\curveto(593.30665779,106.24884638)(593.26665783,106.19884643)(593.20665771,106.15885061)
\curveto(593.14665795,106.11884651)(593.07165803,106.09884653)(592.98165771,106.09885061)
\lineto(592.68165771,106.09885061)
\lineto(591.58665771,106.09885061)
\lineto(586.24665771,106.09885061)
\curveto(586.15666494,106.07884655)(586.08166502,106.06384656)(586.02165771,106.05385061)
\curveto(585.95166515,106.05384657)(585.89166521,106.0238466)(585.84165771,105.96385061)
\curveto(585.79166531,105.89384673)(585.76666533,105.80384682)(585.76665771,105.69385061)
\curveto(585.75666534,105.59384703)(585.75166535,105.48384714)(585.75165771,105.36385061)
\lineto(585.75165771,104.22385061)
\lineto(585.75165771,103.72885061)
\curveto(585.74166536,103.56884906)(585.68166542,103.45884917)(585.57165771,103.39885061)
\curveto(585.54166556,103.37884925)(585.51166559,103.36884926)(585.48165771,103.36885061)
\curveto(585.44166566,103.36884926)(585.3966657,103.36384926)(585.34665771,103.35385061)
\curveto(585.22666587,103.33384929)(585.11666598,103.33884929)(585.01665771,103.36885061)
\curveto(584.91666618,103.40884922)(584.84666625,103.46384916)(584.80665771,103.53385061)
\curveto(584.75666634,103.61384901)(584.73166637,103.73384889)(584.73165771,103.89385061)
\curveto(584.73166637,104.05384857)(584.71666638,104.18884844)(584.68665771,104.29885061)
\curveto(584.67666642,104.34884828)(584.67166643,104.40384822)(584.67165771,104.46385061)
\curveto(584.66166644,104.5238481)(584.64666645,104.58384804)(584.62665771,104.64385061)
\curveto(584.57666652,104.79384783)(584.52666657,104.93884769)(584.47665771,105.07885061)
\curveto(584.41666668,105.21884741)(584.34666675,105.35384727)(584.26665771,105.48385061)
\curveto(584.17666692,105.623847)(584.07166703,105.74384688)(583.95165771,105.84385061)
\curveto(583.83166727,105.94384668)(583.7016674,106.03884659)(583.56165771,106.12885061)
\curveto(583.46166764,106.18884644)(583.35166775,106.23384639)(583.23165771,106.26385061)
\curveto(583.11166799,106.30384632)(583.00666809,106.35384627)(582.91665771,106.41385061)
\curveto(582.85666824,106.46384616)(582.81666828,106.53384609)(582.79665771,106.62385061)
\curveto(582.78666831,106.64384598)(582.78166832,106.66884596)(582.78165771,106.69885061)
\curveto(582.78166832,106.7288459)(582.77666832,106.75384587)(582.76665771,106.77385061)
}
}
{
\newrgbcolor{curcolor}{0 0 0}
\pscustom[linestyle=none,fillstyle=solid,fillcolor=curcolor]
{
\newpath
\moveto(582.76665771,115.12345998)
\curveto(582.76666833,115.22345513)(582.77666832,115.31845503)(582.79665771,115.40845998)
\curveto(582.80666829,115.49845485)(582.83666826,115.56345479)(582.88665771,115.60345998)
\curveto(582.96666813,115.66345469)(583.07166803,115.69345466)(583.20165771,115.69345998)
\lineto(583.59165771,115.69345998)
\lineto(585.09165771,115.69345998)
\lineto(591.48165771,115.69345998)
\lineto(592.65165771,115.69345998)
\lineto(592.96665771,115.69345998)
\curveto(593.06665803,115.70345465)(593.14665795,115.68845466)(593.20665771,115.64845998)
\curveto(593.28665781,115.59845475)(593.33665776,115.52345483)(593.35665771,115.42345998)
\curveto(593.36665773,115.33345502)(593.37165773,115.22345513)(593.37165771,115.09345998)
\lineto(593.37165771,114.86845998)
\curveto(593.35165775,114.78845556)(593.33665776,114.71845563)(593.32665771,114.65845998)
\curveto(593.30665779,114.59845575)(593.26665783,114.5484558)(593.20665771,114.50845998)
\curveto(593.14665795,114.46845588)(593.07165803,114.4484559)(592.98165771,114.44845998)
\lineto(592.68165771,114.44845998)
\lineto(591.58665771,114.44845998)
\lineto(586.24665771,114.44845998)
\curveto(586.15666494,114.42845592)(586.08166502,114.41345594)(586.02165771,114.40345998)
\curveto(585.95166515,114.40345595)(585.89166521,114.37345598)(585.84165771,114.31345998)
\curveto(585.79166531,114.24345611)(585.76666533,114.1534562)(585.76665771,114.04345998)
\curveto(585.75666534,113.94345641)(585.75166535,113.83345652)(585.75165771,113.71345998)
\lineto(585.75165771,112.57345998)
\lineto(585.75165771,112.07845998)
\curveto(585.74166536,111.91845843)(585.68166542,111.80845854)(585.57165771,111.74845998)
\curveto(585.54166556,111.72845862)(585.51166559,111.71845863)(585.48165771,111.71845998)
\curveto(585.44166566,111.71845863)(585.3966657,111.71345864)(585.34665771,111.70345998)
\curveto(585.22666587,111.68345867)(585.11666598,111.68845866)(585.01665771,111.71845998)
\curveto(584.91666618,111.75845859)(584.84666625,111.81345854)(584.80665771,111.88345998)
\curveto(584.75666634,111.96345839)(584.73166637,112.08345827)(584.73165771,112.24345998)
\curveto(584.73166637,112.40345795)(584.71666638,112.53845781)(584.68665771,112.64845998)
\curveto(584.67666642,112.69845765)(584.67166643,112.7534576)(584.67165771,112.81345998)
\curveto(584.66166644,112.87345748)(584.64666645,112.93345742)(584.62665771,112.99345998)
\curveto(584.57666652,113.14345721)(584.52666657,113.28845706)(584.47665771,113.42845998)
\curveto(584.41666668,113.56845678)(584.34666675,113.70345665)(584.26665771,113.83345998)
\curveto(584.17666692,113.97345638)(584.07166703,114.09345626)(583.95165771,114.19345998)
\curveto(583.83166727,114.29345606)(583.7016674,114.38845596)(583.56165771,114.47845998)
\curveto(583.46166764,114.53845581)(583.35166775,114.58345577)(583.23165771,114.61345998)
\curveto(583.11166799,114.6534557)(583.00666809,114.70345565)(582.91665771,114.76345998)
\curveto(582.85666824,114.81345554)(582.81666828,114.88345547)(582.79665771,114.97345998)
\curveto(582.78666831,114.99345536)(582.78166832,115.01845533)(582.78165771,115.04845998)
\curveto(582.78166832,115.07845527)(582.77666832,115.10345525)(582.76665771,115.12345998)
}
}
{
\newrgbcolor{curcolor}{0 0 0}
\pscustom[linestyle=none,fillstyle=solid,fillcolor=curcolor]
{
\newpath
\moveto(604.63794678,29.18119436)
\lineto(604.63794678,30.09619436)
\curveto(604.63795747,30.19619171)(604.63795747,30.29119161)(604.63794678,30.38119436)
\curveto(604.63795747,30.47119143)(604.65795745,30.54619136)(604.69794678,30.60619436)
\curveto(604.75795735,30.69619121)(604.83795727,30.75619115)(604.93794678,30.78619436)
\curveto(605.03795707,30.82619108)(605.14295697,30.87119103)(605.25294678,30.92119436)
\curveto(605.44295667,31.0011909)(605.63295648,31.07119083)(605.82294678,31.13119436)
\curveto(606.0129561,31.2011907)(606.20295591,31.27619063)(606.39294678,31.35619436)
\curveto(606.57295554,31.42619048)(606.75795535,31.49119041)(606.94794678,31.55119436)
\curveto(607.12795498,31.61119029)(607.3079548,31.68119022)(607.48794678,31.76119436)
\curveto(607.62795448,31.82119008)(607.77295434,31.87619003)(607.92294678,31.92619436)
\curveto(608.07295404,31.97618993)(608.21795389,32.03118987)(608.35794678,32.09119436)
\curveto(608.8079533,32.27118963)(609.26295285,32.44118946)(609.72294678,32.60119436)
\curveto(610.17295194,32.76118914)(610.62295149,32.93118897)(611.07294678,33.11119436)
\curveto(611.12295099,33.13118877)(611.17295094,33.14618876)(611.22294678,33.15619436)
\lineto(611.37294678,33.21619436)
\curveto(611.59295052,33.3061886)(611.81795029,33.39118851)(612.04794678,33.47119436)
\curveto(612.26794984,33.55118835)(612.48794962,33.63618827)(612.70794678,33.72619436)
\curveto(612.79794931,33.76618814)(612.9079492,33.8061881)(613.03794678,33.84619436)
\curveto(613.15794895,33.88618802)(613.22794888,33.95118795)(613.24794678,34.04119436)
\curveto(613.25794885,34.08118782)(613.25794885,34.11118779)(613.24794678,34.13119436)
\lineto(613.18794678,34.19119436)
\curveto(613.13794897,34.24118766)(613.08294903,34.27618763)(613.02294678,34.29619436)
\curveto(612.96294915,34.32618758)(612.89794921,34.35618755)(612.82794678,34.38619436)
\lineto(612.19794678,34.62619436)
\curveto(611.97795013,34.7061872)(611.76295035,34.78618712)(611.55294678,34.86619436)
\lineto(611.40294678,34.92619436)
\lineto(611.22294678,34.98619436)
\curveto(611.03295108,35.06618684)(610.84295127,35.13618677)(610.65294678,35.19619436)
\curveto(610.45295166,35.26618664)(610.25295186,35.34118656)(610.05294678,35.42119436)
\curveto(609.47295264,35.66118624)(608.88795322,35.88118602)(608.29794678,36.08119436)
\curveto(607.7079544,36.29118561)(607.12295499,36.51618539)(606.54294678,36.75619436)
\curveto(606.34295577,36.83618507)(606.13795597,36.91118499)(605.92794678,36.98119436)
\curveto(605.71795639,37.06118484)(605.5129566,37.14118476)(605.31294678,37.22119436)
\curveto(605.23295688,37.26118464)(605.13295698,37.29618461)(605.01294678,37.32619436)
\curveto(604.89295722,37.36618454)(604.8079573,37.42118448)(604.75794678,37.49119436)
\curveto(604.71795739,37.55118435)(604.68795742,37.62618428)(604.66794678,37.71619436)
\curveto(604.64795746,37.81618409)(604.63795747,37.92618398)(604.63794678,38.04619436)
\curveto(604.62795748,38.16618374)(604.62795748,38.28618362)(604.63794678,38.40619436)
\curveto(604.63795747,38.52618338)(604.63795747,38.63618327)(604.63794678,38.73619436)
\curveto(604.63795747,38.82618308)(604.63795747,38.91618299)(604.63794678,39.00619436)
\curveto(604.63795747,39.1061828)(604.65795745,39.18118272)(604.69794678,39.23119436)
\curveto(604.74795736,39.32118258)(604.83795727,39.37118253)(604.96794678,39.38119436)
\curveto(605.09795701,39.39118251)(605.23795687,39.39618251)(605.38794678,39.39619436)
\lineto(607.03794678,39.39619436)
\lineto(613.30794678,39.39619436)
\lineto(614.56794678,39.39619436)
\curveto(614.67794743,39.39618251)(614.78794732,39.39618251)(614.89794678,39.39619436)
\curveto(615.0079471,39.4061825)(615.09294702,39.38618252)(615.15294678,39.33619436)
\curveto(615.2129469,39.3061826)(615.25294686,39.26118264)(615.27294678,39.20119436)
\curveto(615.28294683,39.14118276)(615.29794681,39.07118283)(615.31794678,38.99119436)
\lineto(615.31794678,38.75119436)
\lineto(615.31794678,38.39119436)
\curveto(615.3079468,38.28118362)(615.26294685,38.2011837)(615.18294678,38.15119436)
\curveto(615.15294696,38.13118377)(615.12294699,38.11618379)(615.09294678,38.10619436)
\curveto(615.05294706,38.1061838)(615.0079471,38.09618381)(614.95794678,38.07619436)
\lineto(614.79294678,38.07619436)
\curveto(614.73294738,38.06618384)(614.66294745,38.06118384)(614.58294678,38.06119436)
\curveto(614.50294761,38.07118383)(614.42794768,38.07618383)(614.35794678,38.07619436)
\lineto(613.51794678,38.07619436)
\lineto(609.09294678,38.07619436)
\curveto(608.84295327,38.07618383)(608.59295352,38.07618383)(608.34294678,38.07619436)
\curveto(608.08295403,38.07618383)(607.83295428,38.07118383)(607.59294678,38.06119436)
\curveto(607.49295462,38.06118384)(607.38295473,38.05618385)(607.26294678,38.04619436)
\curveto(607.14295497,38.03618387)(607.08295503,37.98118392)(607.08294678,37.88119436)
\lineto(607.09794678,37.88119436)
\curveto(607.11795499,37.81118409)(607.18295493,37.75118415)(607.29294678,37.70119436)
\curveto(607.40295471,37.66118424)(607.49795461,37.62618428)(607.57794678,37.59619436)
\curveto(607.74795436,37.52618438)(607.92295419,37.46118444)(608.10294678,37.40119436)
\curveto(608.27295384,37.34118456)(608.44295367,37.27118463)(608.61294678,37.19119436)
\curveto(608.66295345,37.17118473)(608.7079534,37.15618475)(608.74794678,37.14619436)
\curveto(608.78795332,37.13618477)(608.83295328,37.12118478)(608.88294678,37.10119436)
\curveto(609.06295305,37.02118488)(609.24795286,36.95118495)(609.43794678,36.89119436)
\curveto(609.61795249,36.84118506)(609.79795231,36.77618513)(609.97794678,36.69619436)
\curveto(610.12795198,36.62618528)(610.28295183,36.56618534)(610.44294678,36.51619436)
\curveto(610.59295152,36.46618544)(610.74295137,36.41118549)(610.89294678,36.35119436)
\curveto(611.36295075,36.15118575)(611.83795027,35.97118593)(612.31794678,35.81119436)
\curveto(612.78794932,35.65118625)(613.25294886,35.47618643)(613.71294678,35.28619436)
\curveto(613.89294822,35.2061867)(614.07294804,35.13618677)(614.25294678,35.07619436)
\curveto(614.43294768,35.01618689)(614.6129475,34.95118695)(614.79294678,34.88119436)
\curveto(614.90294721,34.83118707)(615.0079471,34.78118712)(615.10794678,34.73119436)
\curveto(615.19794691,34.69118721)(615.26294685,34.6061873)(615.30294678,34.47619436)
\curveto(615.3129468,34.45618745)(615.31794679,34.43118747)(615.31794678,34.40119436)
\curveto(615.3079468,34.38118752)(615.3079468,34.35618755)(615.31794678,34.32619436)
\curveto(615.32794678,34.29618761)(615.33294678,34.26118764)(615.33294678,34.22119436)
\curveto(615.32294679,34.18118772)(615.31794679,34.14118776)(615.31794678,34.10119436)
\lineto(615.31794678,33.80119436)
\curveto(615.31794679,33.7011882)(615.29294682,33.62118828)(615.24294678,33.56119436)
\curveto(615.19294692,33.48118842)(615.12294699,33.42118848)(615.03294678,33.38119436)
\curveto(614.93294718,33.35118855)(614.83294728,33.31118859)(614.73294678,33.26119436)
\curveto(614.53294758,33.18118872)(614.32794778,33.1011888)(614.11794678,33.02119436)
\curveto(613.89794821,32.95118895)(613.68794842,32.87618903)(613.48794678,32.79619436)
\curveto(613.3079488,32.71618919)(613.12794898,32.64618926)(612.94794678,32.58619436)
\curveto(612.75794935,32.53618937)(612.57294954,32.47118943)(612.39294678,32.39119436)
\curveto(611.83295028,32.16118974)(611.26795084,31.94618996)(610.69794678,31.74619436)
\curveto(610.12795198,31.54619036)(609.56295255,31.33119057)(609.00294678,31.10119436)
\lineto(608.37294678,30.86119436)
\curveto(608.15295396,30.79119111)(607.94295417,30.71619119)(607.74294678,30.63619436)
\curveto(607.63295448,30.58619132)(607.52795458,30.54119136)(607.42794678,30.50119436)
\curveto(607.31795479,30.47119143)(607.22295489,30.42119148)(607.14294678,30.35119436)
\curveto(607.12295499,30.34119156)(607.112955,30.33119157)(607.11294678,30.32119436)
\lineto(607.08294678,30.29119436)
\lineto(607.08294678,30.21619436)
\lineto(607.11294678,30.18619436)
\curveto(607.112955,30.17619173)(607.11795499,30.16619174)(607.12794678,30.15619436)
\curveto(607.17795493,30.13619177)(607.23295488,30.12619178)(607.29294678,30.12619436)
\curveto(607.35295476,30.12619178)(607.4129547,30.11619179)(607.47294678,30.09619436)
\lineto(607.63794678,30.09619436)
\curveto(607.69795441,30.07619183)(607.76295435,30.07119183)(607.83294678,30.08119436)
\curveto(607.90295421,30.09119181)(607.97295414,30.09619181)(608.04294678,30.09619436)
\lineto(608.85294678,30.09619436)
\lineto(613.41294678,30.09619436)
\lineto(614.59794678,30.09619436)
\curveto(614.7079474,30.09619181)(614.81794729,30.09119181)(614.92794678,30.08119436)
\curveto(615.03794707,30.08119182)(615.12294699,30.05619185)(615.18294678,30.00619436)
\curveto(615.26294685,29.95619195)(615.3079468,29.86619204)(615.31794678,29.73619436)
\lineto(615.31794678,29.34619436)
\lineto(615.31794678,29.15119436)
\curveto(615.31794679,29.1011928)(615.3079468,29.05119285)(615.28794678,29.00119436)
\curveto(615.24794686,28.87119303)(615.16294695,28.79619311)(615.03294678,28.77619436)
\curveto(614.90294721,28.76619314)(614.75294736,28.76119314)(614.58294678,28.76119436)
\lineto(612.84294678,28.76119436)
\lineto(606.84294678,28.76119436)
\lineto(605.43294678,28.76119436)
\curveto(605.32295679,28.76119314)(605.2079569,28.75619315)(605.08794678,28.74619436)
\curveto(604.96795714,28.74619316)(604.87295724,28.77119313)(604.80294678,28.82119436)
\curveto(604.74295737,28.86119304)(604.69295742,28.93619297)(604.65294678,29.04619436)
\curveto(604.64295747,29.06619284)(604.64295747,29.08619282)(604.65294678,29.10619436)
\curveto(604.65295746,29.13619277)(604.64795746,29.16119274)(604.63794678,29.18119436)
}
}
{
\newrgbcolor{curcolor}{0 0 0}
\pscustom[linestyle=none,fillstyle=solid,fillcolor=curcolor]
{
\newpath
\moveto(614.76294678,48.38330373)
\curveto(614.92294719,48.4132959)(615.05794705,48.39829592)(615.16794678,48.33830373)
\curveto(615.26794684,48.27829604)(615.34294677,48.19829612)(615.39294678,48.09830373)
\curveto(615.4129467,48.04829627)(615.42294669,47.99329632)(615.42294678,47.93330373)
\curveto(615.42294669,47.88329643)(615.43294668,47.82829649)(615.45294678,47.76830373)
\curveto(615.50294661,47.54829677)(615.48794662,47.32829699)(615.40794678,47.10830373)
\curveto(615.33794677,46.89829742)(615.24794686,46.75329756)(615.13794678,46.67330373)
\curveto(615.06794704,46.62329769)(614.98794712,46.57829774)(614.89794678,46.53830373)
\curveto(614.79794731,46.49829782)(614.71794739,46.44829787)(614.65794678,46.38830373)
\curveto(614.63794747,46.36829795)(614.61794749,46.34329797)(614.59794678,46.31330373)
\curveto(614.57794753,46.29329802)(614.57294754,46.26329805)(614.58294678,46.22330373)
\curveto(614.6129475,46.1132982)(614.66794744,46.00829831)(614.74794678,45.90830373)
\curveto(614.82794728,45.8182985)(614.89794721,45.72829859)(614.95794678,45.63830373)
\curveto(615.03794707,45.50829881)(615.112947,45.36829895)(615.18294678,45.21830373)
\curveto(615.24294687,45.06829925)(615.29794681,44.90829941)(615.34794678,44.73830373)
\curveto(615.37794673,44.63829968)(615.39794671,44.52829979)(615.40794678,44.40830373)
\curveto(615.41794669,44.29830002)(615.43294668,44.18830013)(615.45294678,44.07830373)
\curveto(615.46294665,44.02830029)(615.46794664,43.98330033)(615.46794678,43.94330373)
\lineto(615.46794678,43.83830373)
\curveto(615.48794662,43.72830059)(615.48794662,43.62330069)(615.46794678,43.52330373)
\lineto(615.46794678,43.38830373)
\curveto(615.45794665,43.33830098)(615.45294666,43.28830103)(615.45294678,43.23830373)
\curveto(615.45294666,43.18830113)(615.44294667,43.14330117)(615.42294678,43.10330373)
\curveto(615.4129467,43.06330125)(615.4079467,43.02830129)(615.40794678,42.99830373)
\curveto(615.41794669,42.97830134)(615.41794669,42.95330136)(615.40794678,42.92330373)
\lineto(615.34794678,42.68330373)
\curveto(615.33794677,42.60330171)(615.31794679,42.52830179)(615.28794678,42.45830373)
\curveto(615.15794695,42.15830216)(615.0129471,41.9133024)(614.85294678,41.72330373)
\curveto(614.68294743,41.54330277)(614.44794766,41.39330292)(614.14794678,41.27330373)
\curveto(613.92794818,41.18330313)(613.66294845,41.13830318)(613.35294678,41.13830373)
\lineto(613.03794678,41.13830373)
\curveto(612.98794912,41.14830317)(612.93794917,41.15330316)(612.88794678,41.15330373)
\lineto(612.70794678,41.18330373)
\lineto(612.37794678,41.30330373)
\curveto(612.26794984,41.34330297)(612.16794994,41.39330292)(612.07794678,41.45330373)
\curveto(611.78795032,41.63330268)(611.57295054,41.87830244)(611.43294678,42.18830373)
\curveto(611.29295082,42.49830182)(611.16795094,42.83830148)(611.05794678,43.20830373)
\curveto(611.01795109,43.34830097)(610.98795112,43.49330082)(610.96794678,43.64330373)
\curveto(610.94795116,43.79330052)(610.92295119,43.94330037)(610.89294678,44.09330373)
\curveto(610.87295124,44.16330015)(610.86295125,44.22830009)(610.86294678,44.28830373)
\curveto(610.86295125,44.35829996)(610.85295126,44.43329988)(610.83294678,44.51330373)
\curveto(610.8129513,44.58329973)(610.80295131,44.65329966)(610.80294678,44.72330373)
\curveto(610.79295132,44.79329952)(610.77795133,44.86829945)(610.75794678,44.94830373)
\curveto(610.69795141,45.19829912)(610.64795146,45.43329888)(610.60794678,45.65330373)
\curveto(610.55795155,45.87329844)(610.44295167,46.04829827)(610.26294678,46.17830373)
\curveto(610.18295193,46.23829808)(610.08295203,46.28829803)(609.96294678,46.32830373)
\curveto(609.83295228,46.36829795)(609.69295242,46.36829795)(609.54294678,46.32830373)
\curveto(609.30295281,46.26829805)(609.112953,46.17829814)(608.97294678,46.05830373)
\curveto(608.83295328,45.94829837)(608.72295339,45.78829853)(608.64294678,45.57830373)
\curveto(608.59295352,45.45829886)(608.55795355,45.313299)(608.53794678,45.14330373)
\curveto(608.51795359,44.98329933)(608.5079536,44.8132995)(608.50794678,44.63330373)
\curveto(608.5079536,44.45329986)(608.51795359,44.27830004)(608.53794678,44.10830373)
\curveto(608.55795355,43.93830038)(608.58795352,43.79330052)(608.62794678,43.67330373)
\curveto(608.68795342,43.50330081)(608.77295334,43.33830098)(608.88294678,43.17830373)
\curveto(608.94295317,43.09830122)(609.02295309,43.02330129)(609.12294678,42.95330373)
\curveto(609.2129529,42.89330142)(609.3129528,42.83830148)(609.42294678,42.78830373)
\curveto(609.50295261,42.75830156)(609.58795252,42.72830159)(609.67794678,42.69830373)
\curveto(609.76795234,42.67830164)(609.83795227,42.63330168)(609.88794678,42.56330373)
\curveto(609.91795219,42.52330179)(609.94295217,42.45330186)(609.96294678,42.35330373)
\curveto(609.97295214,42.26330205)(609.97795213,42.16830215)(609.97794678,42.06830373)
\curveto(609.97795213,41.96830235)(609.97295214,41.86830245)(609.96294678,41.76830373)
\curveto(609.94295217,41.67830264)(609.91795219,41.6133027)(609.88794678,41.57330373)
\curveto(609.85795225,41.53330278)(609.8079523,41.50330281)(609.73794678,41.48330373)
\curveto(609.66795244,41.46330285)(609.59295252,41.46330285)(609.51294678,41.48330373)
\curveto(609.38295273,41.5133028)(609.26295285,41.54330277)(609.15294678,41.57330373)
\curveto(609.03295308,41.6133027)(608.91795319,41.65830266)(608.80794678,41.70830373)
\curveto(608.45795365,41.89830242)(608.18795392,42.13830218)(607.99794678,42.42830373)
\curveto(607.79795431,42.7183016)(607.63795447,43.07830124)(607.51794678,43.50830373)
\curveto(607.49795461,43.60830071)(607.48295463,43.70830061)(607.47294678,43.80830373)
\curveto(607.46295465,43.9183004)(607.44795466,44.02830029)(607.42794678,44.13830373)
\curveto(607.41795469,44.17830014)(607.41795469,44.24330007)(607.42794678,44.33330373)
\curveto(607.42795468,44.42329989)(607.41795469,44.47829984)(607.39794678,44.49830373)
\curveto(607.38795472,45.19829912)(607.46795464,45.80829851)(607.63794678,46.32830373)
\curveto(607.8079543,46.84829747)(608.13295398,47.2132971)(608.61294678,47.42330373)
\curveto(608.8129533,47.5132968)(609.04795306,47.56329675)(609.31794678,47.57330373)
\curveto(609.57795253,47.59329672)(609.85295226,47.60329671)(610.14294678,47.60330373)
\lineto(613.45794678,47.60330373)
\curveto(613.59794851,47.60329671)(613.73294838,47.60829671)(613.86294678,47.61830373)
\curveto(613.99294812,47.62829669)(614.09794801,47.65829666)(614.17794678,47.70830373)
\curveto(614.24794786,47.75829656)(614.29794781,47.82329649)(614.32794678,47.90330373)
\curveto(614.36794774,47.99329632)(614.39794771,48.07829624)(614.41794678,48.15830373)
\curveto(614.42794768,48.23829608)(614.47294764,48.29829602)(614.55294678,48.33830373)
\curveto(614.58294753,48.35829596)(614.6129475,48.36829595)(614.64294678,48.36830373)
\curveto(614.67294744,48.36829595)(614.7129474,48.37329594)(614.76294678,48.38330373)
\moveto(613.09794678,46.23830373)
\curveto(612.95794915,46.29829802)(612.79794931,46.32829799)(612.61794678,46.32830373)
\curveto(612.42794968,46.33829798)(612.23294988,46.34329797)(612.03294678,46.34330373)
\curveto(611.92295019,46.34329797)(611.82295029,46.33829798)(611.73294678,46.32830373)
\curveto(611.64295047,46.318298)(611.57295054,46.27829804)(611.52294678,46.20830373)
\curveto(611.50295061,46.17829814)(611.49295062,46.10829821)(611.49294678,45.99830373)
\curveto(611.5129506,45.97829834)(611.52295059,45.94329837)(611.52294678,45.89330373)
\curveto(611.52295059,45.84329847)(611.53295058,45.79829852)(611.55294678,45.75830373)
\curveto(611.57295054,45.67829864)(611.59295052,45.58829873)(611.61294678,45.48830373)
\lineto(611.67294678,45.18830373)
\curveto(611.67295044,45.15829916)(611.67795043,45.12329919)(611.68794678,45.08330373)
\lineto(611.68794678,44.97830373)
\curveto(611.72795038,44.82829949)(611.75295036,44.66329965)(611.76294678,44.48330373)
\curveto(611.76295035,44.3133)(611.78295033,44.15330016)(611.82294678,44.00330373)
\curveto(611.84295027,43.92330039)(611.86295025,43.84830047)(611.88294678,43.77830373)
\curveto(611.89295022,43.7183006)(611.9079502,43.64830067)(611.92794678,43.56830373)
\curveto(611.97795013,43.40830091)(612.04295007,43.25830106)(612.12294678,43.11830373)
\curveto(612.19294992,42.97830134)(612.28294983,42.85830146)(612.39294678,42.75830373)
\curveto(612.50294961,42.65830166)(612.63794947,42.58330173)(612.79794678,42.53330373)
\curveto(612.94794916,42.48330183)(613.13294898,42.46330185)(613.35294678,42.47330373)
\curveto(613.45294866,42.47330184)(613.54794856,42.48830183)(613.63794678,42.51830373)
\curveto(613.71794839,42.55830176)(613.79294832,42.60330171)(613.86294678,42.65330373)
\curveto(613.97294814,42.73330158)(614.06794804,42.83830148)(614.14794678,42.96830373)
\curveto(614.21794789,43.09830122)(614.27794783,43.23830108)(614.32794678,43.38830373)
\curveto(614.33794777,43.43830088)(614.34294777,43.48830083)(614.34294678,43.53830373)
\curveto(614.34294777,43.58830073)(614.34794776,43.63830068)(614.35794678,43.68830373)
\curveto(614.37794773,43.75830056)(614.39294772,43.84330047)(614.40294678,43.94330373)
\curveto(614.40294771,44.05330026)(614.39294772,44.14330017)(614.37294678,44.21330373)
\curveto(614.35294776,44.27330004)(614.34794776,44.33329998)(614.35794678,44.39330373)
\curveto(614.35794775,44.45329986)(614.34794776,44.5132998)(614.32794678,44.57330373)
\curveto(614.3079478,44.65329966)(614.29294782,44.72829959)(614.28294678,44.79830373)
\curveto(614.27294784,44.87829944)(614.25294786,44.95329936)(614.22294678,45.02330373)
\curveto(614.10294801,45.313299)(613.95794815,45.55829876)(613.78794678,45.75830373)
\curveto(613.61794849,45.96829835)(613.38794872,46.12829819)(613.09794678,46.23830373)
}
}
{
\newrgbcolor{curcolor}{0 0 0}
\pscustom[linestyle=none,fillstyle=solid,fillcolor=curcolor]
{
\newpath
\moveto(607.59294678,49.26994436)
\lineto(607.59294678,49.71994436)
\curveto(607.58295453,49.88994311)(607.60295451,50.01494298)(607.65294678,50.09494436)
\curveto(607.70295441,50.17494282)(607.76795434,50.22994277)(607.84794678,50.25994436)
\curveto(607.92795418,50.2999427)(608.0129541,50.33994266)(608.10294678,50.37994436)
\curveto(608.23295388,50.42994257)(608.36295375,50.47494252)(608.49294678,50.51494436)
\curveto(608.62295349,50.55494244)(608.75295336,50.5999424)(608.88294678,50.64994436)
\curveto(609.00295311,50.6999423)(609.12795298,50.74494225)(609.25794678,50.78494436)
\curveto(609.37795273,50.82494217)(609.49795261,50.86994213)(609.61794678,50.91994436)
\curveto(609.72795238,50.96994203)(609.84295227,51.00994199)(609.96294678,51.03994436)
\curveto(610.08295203,51.06994193)(610.20295191,51.10994189)(610.32294678,51.15994436)
\curveto(610.6129515,51.27994172)(610.9129512,51.38994161)(611.22294678,51.48994436)
\curveto(611.53295058,51.58994141)(611.83295028,51.6999413)(612.12294678,51.81994436)
\curveto(612.16294995,51.83994116)(612.20294991,51.84994115)(612.24294678,51.84994436)
\curveto(612.27294984,51.84994115)(612.30294981,51.85994114)(612.33294678,51.87994436)
\curveto(612.47294964,51.93994106)(612.61794949,51.994941)(612.76794678,52.04494436)
\lineto(613.18794678,52.19494436)
\curveto(613.25794885,52.22494077)(613.33294878,52.25494074)(613.41294678,52.28494436)
\curveto(613.48294863,52.31494068)(613.52794858,52.36494063)(613.54794678,52.43494436)
\curveto(613.57794853,52.51494048)(613.55294856,52.57494042)(613.47294678,52.61494436)
\curveto(613.38294873,52.66494033)(613.3129488,52.6999403)(613.26294678,52.71994436)
\curveto(613.09294902,52.7999402)(612.9129492,52.86494013)(612.72294678,52.91494436)
\curveto(612.53294958,52.96494003)(612.34794976,53.02493997)(612.16794678,53.09494436)
\curveto(611.93795017,53.18493981)(611.7079504,53.26493973)(611.47794678,53.33494436)
\curveto(611.23795087,53.40493959)(611.0079511,53.48993951)(610.78794678,53.58994436)
\curveto(610.73795137,53.5999394)(610.67295144,53.61493938)(610.59294678,53.63494436)
\curveto(610.50295161,53.67493932)(610.4129517,53.70993929)(610.32294678,53.73994436)
\curveto(610.22295189,53.76993923)(610.13295198,53.7999392)(610.05294678,53.82994436)
\curveto(610.00295211,53.84993915)(609.95795215,53.86493913)(609.91794678,53.87494436)
\curveto(609.87795223,53.88493911)(609.83295228,53.8999391)(609.78294678,53.91994436)
\curveto(609.66295245,53.96993903)(609.54295257,54.00993899)(609.42294678,54.03994436)
\curveto(609.29295282,54.07993892)(609.16795294,54.12493887)(609.04794678,54.17494436)
\curveto(608.99795311,54.1949388)(608.95295316,54.20993879)(608.91294678,54.21994436)
\curveto(608.87295324,54.22993877)(608.82795328,54.24493875)(608.77794678,54.26494436)
\curveto(608.68795342,54.30493869)(608.59795351,54.33993866)(608.50794678,54.36994436)
\curveto(608.4079537,54.3999386)(608.3129538,54.42993857)(608.22294678,54.45994436)
\curveto(608.14295397,54.48993851)(608.06295405,54.51493848)(607.98294678,54.53494436)
\curveto(607.89295422,54.56493843)(607.81795429,54.60493839)(607.75794678,54.65494436)
\curveto(607.66795444,54.72493827)(607.61795449,54.81993818)(607.60794678,54.93994436)
\curveto(607.59795451,55.06993793)(607.59295452,55.20993779)(607.59294678,55.35994436)
\curveto(607.59295452,55.43993756)(607.59795451,55.51493748)(607.60794678,55.58494436)
\curveto(607.6079545,55.66493733)(607.62295449,55.72993727)(607.65294678,55.77994436)
\curveto(607.7129544,55.86993713)(607.8079543,55.8949371)(607.93794678,55.85494436)
\curveto(608.06795404,55.81493718)(608.16795394,55.77993722)(608.23794678,55.74994436)
\lineto(608.29794678,55.71994436)
\curveto(608.31795379,55.71993728)(608.33795377,55.71493728)(608.35794678,55.70494436)
\curveto(608.63795347,55.5949374)(608.92295319,55.48493751)(609.21294678,55.37494436)
\lineto(610.05294678,55.04494436)
\curveto(610.13295198,55.01493798)(610.2079519,54.98993801)(610.27794678,54.96994436)
\curveto(610.33795177,54.94993805)(610.40295171,54.92493807)(610.47294678,54.89494436)
\curveto(610.67295144,54.81493818)(610.87795123,54.73493826)(611.08794678,54.65494436)
\curveto(611.28795082,54.58493841)(611.48795062,54.50993849)(611.68794678,54.42994436)
\curveto(612.37794973,54.13993886)(613.07294904,53.86993913)(613.77294678,53.61994436)
\curveto(614.47294764,53.36993963)(615.16794694,53.0999399)(615.85794678,52.80994436)
\lineto(616.00794678,52.74994436)
\curveto(616.06794604,52.73994026)(616.12794598,52.72494027)(616.18794678,52.70494436)
\curveto(616.55794555,52.54494045)(616.92294519,52.37494062)(617.28294678,52.19494436)
\curveto(617.65294446,52.01494098)(617.93794417,51.76494123)(618.13794678,51.44494436)
\curveto(618.19794391,51.33494166)(618.24294387,51.22494177)(618.27294678,51.11494436)
\curveto(618.3129438,51.00494199)(618.34794376,50.87994212)(618.37794678,50.73994436)
\curveto(618.39794371,50.68994231)(618.40294371,50.63494236)(618.39294678,50.57494436)
\curveto(618.38294373,50.52494247)(618.38294373,50.46994253)(618.39294678,50.40994436)
\curveto(618.4129437,50.32994267)(618.4129437,50.24994275)(618.39294678,50.16994436)
\curveto(618.38294373,50.12994287)(618.37794373,50.07994292)(618.37794678,50.01994436)
\lineto(618.31794678,49.77994436)
\curveto(618.29794381,49.70994329)(618.25794385,49.65494334)(618.19794678,49.61494436)
\curveto(618.13794397,49.56494343)(618.06294405,49.53494346)(617.97294678,49.52494436)
\lineto(617.70294678,49.52494436)
\lineto(617.49294678,49.52494436)
\curveto(617.43294468,49.53494346)(617.38294473,49.55494344)(617.34294678,49.58494436)
\curveto(617.23294488,49.65494334)(617.20294491,49.77494322)(617.25294678,49.94494436)
\curveto(617.27294484,50.05494294)(617.28294483,50.17494282)(617.28294678,50.30494436)
\curveto(617.28294483,50.43494256)(617.26294485,50.54994245)(617.22294678,50.64994436)
\curveto(617.17294494,50.7999422)(617.09794501,50.91994208)(616.99794678,51.00994436)
\curveto(616.89794521,51.10994189)(616.78294533,51.1949418)(616.65294678,51.26494436)
\curveto(616.53294558,51.33494166)(616.40294571,51.3949416)(616.26294678,51.44494436)
\lineto(615.84294678,51.62494436)
\curveto(615.75294636,51.66494133)(615.64294647,51.70494129)(615.51294678,51.74494436)
\curveto(615.38294673,51.7949412)(615.24794686,51.7999412)(615.10794678,51.75994436)
\curveto(614.94794716,51.70994129)(614.79794731,51.65494134)(614.65794678,51.59494436)
\curveto(614.51794759,51.54494145)(614.37794773,51.48994151)(614.23794678,51.42994436)
\curveto(614.02794808,51.33994166)(613.81794829,51.25494174)(613.60794678,51.17494436)
\curveto(613.39794871,51.0949419)(613.19294892,51.01494198)(612.99294678,50.93494436)
\curveto(612.85294926,50.87494212)(612.71794939,50.81994218)(612.58794678,50.76994436)
\curveto(612.45794965,50.71994228)(612.32294979,50.66994233)(612.18294678,50.61994436)
\lineto(610.86294678,50.07994436)
\curveto(610.42295169,49.90994309)(609.98295213,49.73494326)(609.54294678,49.55494436)
\curveto(609.3129528,49.45494354)(609.09295302,49.36494363)(608.88294678,49.28494436)
\curveto(608.66295345,49.20494379)(608.44295367,49.11994388)(608.22294678,49.02994436)
\curveto(608.16295395,49.00994399)(608.08295403,48.97994402)(607.98294678,48.93994436)
\curveto(607.87295424,48.8999441)(607.78295433,48.90494409)(607.71294678,48.95494436)
\curveto(607.66295445,48.98494401)(607.62795448,49.04494395)(607.60794678,49.13494436)
\curveto(607.59795451,49.15494384)(607.59795451,49.17494382)(607.60794678,49.19494436)
\curveto(607.6079545,49.22494377)(607.60295451,49.24994375)(607.59294678,49.26994436)
}
}
{
\newrgbcolor{curcolor}{0 0 0}
\pscustom[linestyle=none,fillstyle=solid,fillcolor=curcolor]
{
}
}
{
\newrgbcolor{curcolor}{0 0 0}
\pscustom[linestyle=none,fillstyle=solid,fillcolor=curcolor]
{
\newpath
\moveto(604.71294678,64.24510061)
\curveto(604.70295741,64.93509597)(604.82295729,65.53509537)(605.07294678,66.04510061)
\curveto(605.32295679,66.56509434)(605.65795645,66.96009395)(606.07794678,67.23010061)
\curveto(606.15795595,67.28009363)(606.24795586,67.32509358)(606.34794678,67.36510061)
\curveto(606.43795567,67.4050935)(606.53295558,67.45009346)(606.63294678,67.50010061)
\curveto(606.73295538,67.54009337)(606.83295528,67.57009334)(606.93294678,67.59010061)
\curveto(607.03295508,67.6100933)(607.13795497,67.63009328)(607.24794678,67.65010061)
\curveto(607.29795481,67.67009324)(607.34295477,67.67509323)(607.38294678,67.66510061)
\curveto(607.42295469,67.65509325)(607.46795464,67.66009325)(607.51794678,67.68010061)
\curveto(607.56795454,67.69009322)(607.65295446,67.69509321)(607.77294678,67.69510061)
\curveto(607.88295423,67.69509321)(607.96795414,67.69009322)(608.02794678,67.68010061)
\curveto(608.08795402,67.66009325)(608.14795396,67.65009326)(608.20794678,67.65010061)
\curveto(608.26795384,67.66009325)(608.32795378,67.65509325)(608.38794678,67.63510061)
\curveto(608.52795358,67.59509331)(608.66295345,67.56009335)(608.79294678,67.53010061)
\curveto(608.92295319,67.50009341)(609.04795306,67.46009345)(609.16794678,67.41010061)
\curveto(609.3079528,67.35009356)(609.43295268,67.28009363)(609.54294678,67.20010061)
\curveto(609.65295246,67.13009378)(609.76295235,67.05509385)(609.87294678,66.97510061)
\lineto(609.93294678,66.91510061)
\curveto(609.95295216,66.905094)(609.97295214,66.89009402)(609.99294678,66.87010061)
\curveto(610.15295196,66.75009416)(610.29795181,66.61509429)(610.42794678,66.46510061)
\curveto(610.55795155,66.31509459)(610.68295143,66.15509475)(610.80294678,65.98510061)
\curveto(611.02295109,65.67509523)(611.22795088,65.38009553)(611.41794678,65.10010061)
\curveto(611.55795055,64.87009604)(611.69295042,64.64009627)(611.82294678,64.41010061)
\curveto(611.95295016,64.19009672)(612.08795002,63.97009694)(612.22794678,63.75010061)
\curveto(612.39794971,63.50009741)(612.57794953,63.26009765)(612.76794678,63.03010061)
\curveto(612.95794915,62.8100981)(613.18294893,62.62009829)(613.44294678,62.46010061)
\curveto(613.50294861,62.42009849)(613.56294855,62.38509852)(613.62294678,62.35510061)
\curveto(613.67294844,62.32509858)(613.73794837,62.29509861)(613.81794678,62.26510061)
\curveto(613.88794822,62.24509866)(613.94794816,62.24009867)(613.99794678,62.25010061)
\curveto(614.06794804,62.27009864)(614.12294799,62.3050986)(614.16294678,62.35510061)
\curveto(614.19294792,62.4050985)(614.2129479,62.46509844)(614.22294678,62.53510061)
\lineto(614.22294678,62.77510061)
\lineto(614.22294678,63.52510061)
\lineto(614.22294678,66.33010061)
\lineto(614.22294678,66.99010061)
\curveto(614.22294789,67.08009383)(614.22794788,67.16509374)(614.23794678,67.24510061)
\curveto(614.23794787,67.32509358)(614.25794785,67.39009352)(614.29794678,67.44010061)
\curveto(614.33794777,67.49009342)(614.4129477,67.53009338)(614.52294678,67.56010061)
\curveto(614.62294749,67.60009331)(614.72294739,67.6100933)(614.82294678,67.59010061)
\lineto(614.95794678,67.59010061)
\curveto(615.02794708,67.57009334)(615.08794702,67.55009336)(615.13794678,67.53010061)
\curveto(615.18794692,67.5100934)(615.22794688,67.47509343)(615.25794678,67.42510061)
\curveto(615.29794681,67.37509353)(615.31794679,67.3050936)(615.31794678,67.21510061)
\lineto(615.31794678,66.94510061)
\lineto(615.31794678,66.04510061)
\lineto(615.31794678,62.53510061)
\lineto(615.31794678,61.47010061)
\curveto(615.31794679,61.39009952)(615.32294679,61.30009961)(615.33294678,61.20010061)
\curveto(615.33294678,61.10009981)(615.32294679,61.01509989)(615.30294678,60.94510061)
\curveto(615.23294688,60.73510017)(615.05294706,60.67010024)(614.76294678,60.75010061)
\curveto(614.72294739,60.76010015)(614.68794742,60.76010015)(614.65794678,60.75010061)
\curveto(614.61794749,60.75010016)(614.57294754,60.76010015)(614.52294678,60.78010061)
\curveto(614.44294767,60.80010011)(614.35794775,60.82010009)(614.26794678,60.84010061)
\curveto(614.17794793,60.86010005)(614.09294802,60.88510002)(614.01294678,60.91510061)
\curveto(613.52294859,61.07509983)(613.107949,61.27509963)(612.76794678,61.51510061)
\curveto(612.51794959,61.69509921)(612.29294982,61.90009901)(612.09294678,62.13010061)
\curveto(611.88295023,62.36009855)(611.68795042,62.60009831)(611.50794678,62.85010061)
\curveto(611.32795078,63.1100978)(611.15795095,63.37509753)(610.99794678,63.64510061)
\curveto(610.82795128,63.92509698)(610.65295146,64.19509671)(610.47294678,64.45510061)
\curveto(610.39295172,64.56509634)(610.31795179,64.67009624)(610.24794678,64.77010061)
\curveto(610.17795193,64.88009603)(610.10295201,64.99009592)(610.02294678,65.10010061)
\curveto(609.99295212,65.14009577)(609.96295215,65.17509573)(609.93294678,65.20510061)
\curveto(609.89295222,65.24509566)(609.86295225,65.28509562)(609.84294678,65.32510061)
\curveto(609.73295238,65.46509544)(609.6079525,65.59009532)(609.46794678,65.70010061)
\curveto(609.43795267,65.72009519)(609.4129527,65.74509516)(609.39294678,65.77510061)
\curveto(609.36295275,65.8050951)(609.33295278,65.83009508)(609.30294678,65.85010061)
\curveto(609.20295291,65.93009498)(609.10295301,65.99509491)(609.00294678,66.04510061)
\curveto(608.90295321,66.1050948)(608.79295332,66.16009475)(608.67294678,66.21010061)
\curveto(608.60295351,66.24009467)(608.52795358,66.26009465)(608.44794678,66.27010061)
\lineto(608.20794678,66.33010061)
\lineto(608.11794678,66.33010061)
\curveto(608.08795402,66.34009457)(608.05795405,66.34509456)(608.02794678,66.34510061)
\curveto(607.95795415,66.36509454)(607.86295425,66.37009454)(607.74294678,66.36010061)
\curveto(607.6129545,66.36009455)(607.5129546,66.35009456)(607.44294678,66.33010061)
\curveto(607.36295475,66.3100946)(607.28795482,66.29009462)(607.21794678,66.27010061)
\curveto(607.13795497,66.26009465)(607.05795505,66.24009467)(606.97794678,66.21010061)
\curveto(606.73795537,66.10009481)(606.53795557,65.95009496)(606.37794678,65.76010061)
\curveto(606.2079559,65.58009533)(606.06795604,65.36009555)(605.95794678,65.10010061)
\curveto(605.93795617,65.03009588)(605.92295619,64.96009595)(605.91294678,64.89010061)
\curveto(605.89295622,64.82009609)(605.87295624,64.74509616)(605.85294678,64.66510061)
\curveto(605.83295628,64.58509632)(605.82295629,64.47509643)(605.82294678,64.33510061)
\curveto(605.82295629,64.2050967)(605.83295628,64.10009681)(605.85294678,64.02010061)
\curveto(605.86295625,63.96009695)(605.86795624,63.905097)(605.86794678,63.85510061)
\curveto(605.86795624,63.8050971)(605.87795623,63.75509715)(605.89794678,63.70510061)
\curveto(605.93795617,63.6050973)(605.97795613,63.5100974)(606.01794678,63.42010061)
\curveto(606.05795605,63.34009757)(606.10295601,63.26009765)(606.15294678,63.18010061)
\curveto(606.17295594,63.15009776)(606.19795591,63.12009779)(606.22794678,63.09010061)
\curveto(606.25795585,63.07009784)(606.28295583,63.04509786)(606.30294678,63.01510061)
\lineto(606.37794678,62.94010061)
\curveto(606.39795571,62.910098)(606.41795569,62.88509802)(606.43794678,62.86510061)
\lineto(606.64794678,62.71510061)
\curveto(606.7079554,62.67509823)(606.77295534,62.63009828)(606.84294678,62.58010061)
\curveto(606.93295518,62.52009839)(607.03795507,62.47009844)(607.15794678,62.43010061)
\curveto(607.26795484,62.40009851)(607.37795473,62.36509854)(607.48794678,62.32510061)
\curveto(607.59795451,62.28509862)(607.74295437,62.26009865)(607.92294678,62.25010061)
\curveto(608.09295402,62.24009867)(608.21795389,62.2100987)(608.29794678,62.16010061)
\curveto(608.37795373,62.1100988)(608.42295369,62.03509887)(608.43294678,61.93510061)
\curveto(608.44295367,61.83509907)(608.44795366,61.72509918)(608.44794678,61.60510061)
\curveto(608.44795366,61.56509934)(608.45295366,61.52509938)(608.46294678,61.48510061)
\curveto(608.46295365,61.44509946)(608.45795365,61.4100995)(608.44794678,61.38010061)
\curveto(608.42795368,61.33009958)(608.41795369,61.28009963)(608.41794678,61.23010061)
\curveto(608.41795369,61.19009972)(608.4079537,61.15009976)(608.38794678,61.11010061)
\curveto(608.32795378,61.02009989)(608.19295392,60.97509993)(607.98294678,60.97510061)
\lineto(607.86294678,60.97510061)
\curveto(607.80295431,60.98509992)(607.74295437,60.99009992)(607.68294678,60.99010061)
\curveto(607.6129545,61.00009991)(607.54795456,61.0100999)(607.48794678,61.02010061)
\curveto(607.37795473,61.04009987)(607.27795483,61.06009985)(607.18794678,61.08010061)
\curveto(607.08795502,61.10009981)(606.99295512,61.13009978)(606.90294678,61.17010061)
\curveto(606.83295528,61.19009972)(606.77295534,61.2100997)(606.72294678,61.23010061)
\lineto(606.54294678,61.29010061)
\curveto(606.28295583,61.4100995)(606.03795607,61.56509934)(605.80794678,61.75510061)
\curveto(605.57795653,61.95509895)(605.39295672,62.17009874)(605.25294678,62.40010061)
\curveto(605.17295694,62.5100984)(605.107957,62.62509828)(605.05794678,62.74510061)
\lineto(604.90794678,63.13510061)
\curveto(604.85795725,63.24509766)(604.82795728,63.36009755)(604.81794678,63.48010061)
\curveto(604.79795731,63.60009731)(604.77295734,63.72509718)(604.74294678,63.85510061)
\curveto(604.74295737,63.92509698)(604.74295737,63.99009692)(604.74294678,64.05010061)
\curveto(604.73295738,64.1100968)(604.72295739,64.17509673)(604.71294678,64.24510061)
}
}
{
\newrgbcolor{curcolor}{0 0 0}
\pscustom[linestyle=none,fillstyle=solid,fillcolor=curcolor]
{
\newpath
\moveto(604.90794678,70.85470998)
\lineto(604.90794678,74.45470998)
\lineto(604.90794678,75.09970998)
\curveto(604.9079572,75.17970345)(604.9129572,75.25470338)(604.92294678,75.32470998)
\curveto(604.92295719,75.39470324)(604.93295718,75.45470318)(604.95294678,75.50470998)
\curveto(604.98295713,75.57470306)(605.04295707,75.629703)(605.13294678,75.66970998)
\curveto(605.16295695,75.68970294)(605.20295691,75.69970293)(605.25294678,75.69970998)
\lineto(605.38794678,75.69970998)
\curveto(605.49795661,75.70970292)(605.60295651,75.70470293)(605.70294678,75.68470998)
\curveto(605.80295631,75.67470296)(605.87295624,75.63970299)(605.91294678,75.57970998)
\curveto(605.98295613,75.48970314)(606.01795609,75.35470328)(606.01794678,75.17470998)
\curveto(606.0079561,74.99470364)(606.00295611,74.8297038)(606.00294678,74.67970998)
\lineto(606.00294678,72.68470998)
\lineto(606.00294678,72.18970998)
\lineto(606.00294678,72.05470998)
\curveto(606.00295611,72.01470662)(606.0079561,71.97470666)(606.01794678,71.93470998)
\lineto(606.01794678,71.72470998)
\curveto(606.04795606,71.61470702)(606.08795602,71.5347071)(606.13794678,71.48470998)
\curveto(606.17795593,71.4347072)(606.23295588,71.39970723)(606.30294678,71.37970998)
\curveto(606.36295575,71.35970727)(606.43295568,71.34470729)(606.51294678,71.33470998)
\curveto(606.59295552,71.32470731)(606.68295543,71.30470733)(606.78294678,71.27470998)
\curveto(606.98295513,71.22470741)(607.18795492,71.18470745)(607.39794678,71.15470998)
\curveto(607.6079545,71.12470751)(607.8129543,71.08470755)(608.01294678,71.03470998)
\curveto(608.08295403,71.01470762)(608.15295396,71.00470763)(608.22294678,71.00470998)
\curveto(608.28295383,71.00470763)(608.34795376,70.99470764)(608.41794678,70.97470998)
\curveto(608.44795366,70.96470767)(608.48795362,70.95470768)(608.53794678,70.94470998)
\curveto(608.57795353,70.94470769)(608.61795349,70.94970768)(608.65794678,70.95970998)
\curveto(608.7079534,70.97970765)(608.75295336,71.00470763)(608.79294678,71.03470998)
\curveto(608.82295329,71.07470756)(608.82795328,71.1347075)(608.80794678,71.21470998)
\curveto(608.78795332,71.27470736)(608.76295335,71.3347073)(608.73294678,71.39470998)
\curveto(608.69295342,71.45470718)(608.65795345,71.51470712)(608.62794678,71.57470998)
\curveto(608.6079535,71.634707)(608.59295352,71.68470695)(608.58294678,71.72470998)
\curveto(608.50295361,71.91470672)(608.44795366,72.11970651)(608.41794678,72.33970998)
\curveto(608.38795372,72.56970606)(608.37795373,72.79970583)(608.38794678,73.02970998)
\curveto(608.38795372,73.26970536)(608.4129537,73.49970513)(608.46294678,73.71970998)
\curveto(608.50295361,73.93970469)(608.56295355,74.13970449)(608.64294678,74.31970998)
\curveto(608.66295345,74.36970426)(608.68295343,74.41470422)(608.70294678,74.45470998)
\curveto(608.72295339,74.50470413)(608.74795336,74.55470408)(608.77794678,74.60470998)
\curveto(608.98795312,74.95470368)(609.21795289,75.2347034)(609.46794678,75.44470998)
\curveto(609.71795239,75.66470297)(610.04295207,75.85970277)(610.44294678,76.02970998)
\curveto(610.55295156,76.07970255)(610.66295145,76.11470252)(610.77294678,76.13470998)
\curveto(610.88295123,76.15470248)(610.99795111,76.17970245)(611.11794678,76.20970998)
\curveto(611.14795096,76.21970241)(611.19295092,76.22470241)(611.25294678,76.22470998)
\curveto(611.3129508,76.24470239)(611.38295073,76.25470238)(611.46294678,76.25470998)
\curveto(611.53295058,76.25470238)(611.59795051,76.26470237)(611.65794678,76.28470998)
\lineto(611.82294678,76.28470998)
\curveto(611.87295024,76.29470234)(611.94295017,76.29970233)(612.03294678,76.29970998)
\curveto(612.12294999,76.29970233)(612.19294992,76.28970234)(612.24294678,76.26970998)
\curveto(612.30294981,76.24970238)(612.36294975,76.24470239)(612.42294678,76.25470998)
\curveto(612.47294964,76.26470237)(612.52294959,76.25970237)(612.57294678,76.23970998)
\curveto(612.73294938,76.19970243)(612.88294923,76.16470247)(613.02294678,76.13470998)
\curveto(613.16294895,76.10470253)(613.29794881,76.05970257)(613.42794678,75.99970998)
\curveto(613.79794831,75.83970279)(614.13294798,75.61970301)(614.43294678,75.33970998)
\curveto(614.73294738,75.05970357)(614.96294715,74.73970389)(615.12294678,74.37970998)
\curveto(615.20294691,74.20970442)(615.27794683,74.00970462)(615.34794678,73.77970998)
\curveto(615.38794672,73.66970496)(615.4129467,73.55470508)(615.42294678,73.43470998)
\curveto(615.43294668,73.31470532)(615.45294666,73.19470544)(615.48294678,73.07470998)
\curveto(615.50294661,73.02470561)(615.50294661,72.96970566)(615.48294678,72.90970998)
\curveto(615.47294664,72.84970578)(615.47794663,72.78970584)(615.49794678,72.72970998)
\curveto(615.51794659,72.629706)(615.51794659,72.5297061)(615.49794678,72.42970998)
\lineto(615.49794678,72.29470998)
\curveto(615.47794663,72.24470639)(615.46794664,72.18470645)(615.46794678,72.11470998)
\curveto(615.47794663,72.05470658)(615.47294664,71.99970663)(615.45294678,71.94970998)
\curveto(615.44294667,71.90970672)(615.43794667,71.87470676)(615.43794678,71.84470998)
\curveto(615.43794667,71.81470682)(615.43294668,71.77970685)(615.42294678,71.73970998)
\lineto(615.36294678,71.46970998)
\curveto(615.34294677,71.37970725)(615.3129468,71.29470734)(615.27294678,71.21470998)
\curveto(615.13294698,70.87470776)(614.97794713,70.58470805)(614.80794678,70.34470998)
\curveto(614.62794748,70.10470853)(614.39794771,69.88470875)(614.11794678,69.68470998)
\curveto(613.88794822,69.5347091)(613.64794846,69.41970921)(613.39794678,69.33970998)
\curveto(613.34794876,69.31970931)(613.30294881,69.30970932)(613.26294678,69.30970998)
\curveto(613.2129489,69.30970932)(613.16294895,69.29970933)(613.11294678,69.27970998)
\curveto(613.05294906,69.25970937)(612.97294914,69.24470939)(612.87294678,69.23470998)
\curveto(612.77294934,69.2347094)(612.69794941,69.25470938)(612.64794678,69.29470998)
\curveto(612.56794954,69.34470929)(612.52294959,69.42470921)(612.51294678,69.53470998)
\curveto(612.50294961,69.64470899)(612.49794961,69.75970887)(612.49794678,69.87970998)
\lineto(612.49794678,70.04470998)
\curveto(612.49794961,70.10470853)(612.5079496,70.15970847)(612.52794678,70.20970998)
\curveto(612.54794956,70.29970833)(612.58794952,70.36970826)(612.64794678,70.41970998)
\curveto(612.73794937,70.48970814)(612.84794926,70.5347081)(612.97794678,70.55470998)
\curveto(613.09794901,70.58470805)(613.20294891,70.629708)(613.29294678,70.68970998)
\curveto(613.63294848,70.87970775)(613.90294821,71.13970749)(614.10294678,71.46970998)
\curveto(614.16294795,71.56970706)(614.2129479,71.67470696)(614.25294678,71.78470998)
\curveto(614.28294783,71.90470673)(614.31794779,72.02470661)(614.35794678,72.14470998)
\curveto(614.4079477,72.31470632)(614.42794768,72.51970611)(614.41794678,72.75970998)
\curveto(614.39794771,73.00970562)(614.36294775,73.20970542)(614.31294678,73.35970998)
\curveto(614.19294792,73.7297049)(614.03294808,74.01970461)(613.83294678,74.22970998)
\curveto(613.62294849,74.44970418)(613.34294877,74.629704)(612.99294678,74.76970998)
\curveto(612.89294922,74.81970381)(612.78794932,74.84970378)(612.67794678,74.85970998)
\curveto(612.56794954,74.87970375)(612.45294966,74.90470373)(612.33294678,74.93470998)
\lineto(612.22794678,74.93470998)
\curveto(612.18794992,74.94470369)(612.14794996,74.94970368)(612.10794678,74.94970998)
\curveto(612.07795003,74.95970367)(612.04295007,74.95970367)(612.00294678,74.94970998)
\lineto(611.88294678,74.94970998)
\curveto(611.62295049,74.94970368)(611.37795073,74.91970371)(611.14794678,74.85970998)
\curveto(610.79795131,74.74970388)(610.50295161,74.59470404)(610.26294678,74.39470998)
\curveto(610.0129521,74.19470444)(609.81795229,73.9347047)(609.67794678,73.61470998)
\lineto(609.61794678,73.43470998)
\curveto(609.59795251,73.38470525)(609.57795253,73.32470531)(609.55794678,73.25470998)
\curveto(609.53795257,73.20470543)(609.52795258,73.14470549)(609.52794678,73.07470998)
\curveto(609.51795259,73.01470562)(609.50295261,72.94970568)(609.48294678,72.87970998)
\lineto(609.48294678,72.72970998)
\curveto(609.46295265,72.68970594)(609.45295266,72.634706)(609.45294678,72.56470998)
\curveto(609.45295266,72.50470613)(609.46295265,72.44970618)(609.48294678,72.39970998)
\lineto(609.48294678,72.29470998)
\curveto(609.48295263,72.26470637)(609.48795262,72.2297064)(609.49794678,72.18970998)
\lineto(609.55794678,71.94970998)
\curveto(609.56795254,71.86970676)(609.58795252,71.78970684)(609.61794678,71.70970998)
\curveto(609.71795239,71.46970716)(609.85295226,71.23970739)(610.02294678,71.01970998)
\curveto(610.09295202,70.9297077)(610.16795194,70.84470779)(610.24794678,70.76470998)
\curveto(610.31795179,70.68470795)(610.37295174,70.58470805)(610.41294678,70.46470998)
\curveto(610.44295167,70.37470826)(610.45295166,70.2347084)(610.44294678,70.04470998)
\curveto(610.43295168,69.86470877)(610.4079517,69.74470889)(610.36794678,69.68470998)
\curveto(610.32795178,69.634709)(610.26795184,69.59470904)(610.18794678,69.56470998)
\curveto(610.107952,69.54470909)(610.02295209,69.54470909)(609.93294678,69.56470998)
\curveto(609.8129523,69.59470904)(609.69295242,69.61470902)(609.57294678,69.62470998)
\curveto(609.44295267,69.64470899)(609.31795279,69.66970896)(609.19794678,69.69970998)
\curveto(609.15795295,69.71970891)(609.12295299,69.72470891)(609.09294678,69.71470998)
\curveto(609.05295306,69.71470892)(609.0079531,69.72470891)(608.95794678,69.74470998)
\curveto(608.86795324,69.76470887)(608.77795333,69.77970885)(608.68794678,69.78970998)
\curveto(608.58795352,69.79970883)(608.49295362,69.81970881)(608.40294678,69.84970998)
\curveto(608.34295377,69.85970877)(608.28295383,69.86470877)(608.22294678,69.86470998)
\curveto(608.16295395,69.87470876)(608.10295401,69.88970874)(608.04294678,69.90970998)
\curveto(607.84295427,69.95970867)(607.63795447,69.99470864)(607.42794678,70.01470998)
\curveto(607.2079549,70.04470859)(606.99795511,70.08470855)(606.79794678,70.13470998)
\curveto(606.69795541,70.16470847)(606.59795551,70.18470845)(606.49794678,70.19470998)
\curveto(606.39795571,70.20470843)(606.29795581,70.21970841)(606.19794678,70.23970998)
\curveto(606.16795594,70.24970838)(606.12795598,70.25470838)(606.07794678,70.25470998)
\curveto(605.96795614,70.28470835)(605.86295625,70.30470833)(605.76294678,70.31470998)
\curveto(605.65295646,70.3347083)(605.54295657,70.35970827)(605.43294678,70.38970998)
\curveto(605.35295676,70.40970822)(605.28295683,70.42470821)(605.22294678,70.43470998)
\curveto(605.15295696,70.44470819)(605.09295702,70.46970816)(605.04294678,70.50970998)
\curveto(605.0129571,70.5297081)(604.99295712,70.55970807)(604.98294678,70.59970998)
\curveto(604.96295715,70.63970799)(604.94295717,70.68470795)(604.92294678,70.73470998)
\curveto(604.92295719,70.79470784)(604.91795719,70.8347078)(604.90794678,70.85470998)
}
}
{
\newrgbcolor{curcolor}{0 0 0}
\pscustom[linestyle=none,fillstyle=solid,fillcolor=curcolor]
{
\newpath
\moveto(613.68294678,78.63431936)
\lineto(613.68294678,79.26431936)
\lineto(613.68294678,79.45931936)
\curveto(613.68294843,79.52931683)(613.69294842,79.58931677)(613.71294678,79.63931936)
\curveto(613.75294836,79.70931665)(613.79294832,79.7593166)(613.83294678,79.78931936)
\curveto(613.88294823,79.82931653)(613.94794816,79.84931651)(614.02794678,79.84931936)
\curveto(614.107948,79.8593165)(614.19294792,79.86431649)(614.28294678,79.86431936)
\lineto(615.00294678,79.86431936)
\curveto(615.48294663,79.86431649)(615.89294622,79.80431655)(616.23294678,79.68431936)
\curveto(616.57294554,79.56431679)(616.84794526,79.36931699)(617.05794678,79.09931936)
\curveto(617.107945,79.02931733)(617.15294496,78.9593174)(617.19294678,78.88931936)
\curveto(617.24294487,78.82931753)(617.28794482,78.7543176)(617.32794678,78.66431936)
\curveto(617.33794477,78.64431771)(617.34794476,78.61431774)(617.35794678,78.57431936)
\curveto(617.37794473,78.53431782)(617.38294473,78.48931787)(617.37294678,78.43931936)
\curveto(617.34294477,78.34931801)(617.26794484,78.29431806)(617.14794678,78.27431936)
\curveto(617.03794507,78.2543181)(616.94294517,78.26931809)(616.86294678,78.31931936)
\curveto(616.79294532,78.34931801)(616.72794538,78.39431796)(616.66794678,78.45431936)
\curveto(616.61794549,78.52431783)(616.56794554,78.58931777)(616.51794678,78.64931936)
\curveto(616.46794564,78.71931764)(616.39294572,78.77931758)(616.29294678,78.82931936)
\curveto(616.20294591,78.88931747)(616.112946,78.93931742)(616.02294678,78.97931936)
\curveto(615.99294612,78.99931736)(615.93294618,79.02431733)(615.84294678,79.05431936)
\curveto(615.76294635,79.08431727)(615.69294642,79.08931727)(615.63294678,79.06931936)
\curveto(615.49294662,79.03931732)(615.40294671,78.97931738)(615.36294678,78.88931936)
\curveto(615.33294678,78.80931755)(615.31794679,78.71931764)(615.31794678,78.61931936)
\curveto(615.31794679,78.51931784)(615.29294682,78.43431792)(615.24294678,78.36431936)
\curveto(615.17294694,78.27431808)(615.03294708,78.22931813)(614.82294678,78.22931936)
\lineto(614.26794678,78.22931936)
\lineto(614.04294678,78.22931936)
\curveto(613.96294815,78.23931812)(613.89794821,78.2593181)(613.84794678,78.28931936)
\curveto(613.76794834,78.34931801)(613.72294839,78.41931794)(613.71294678,78.49931936)
\curveto(613.70294841,78.51931784)(613.69794841,78.53931782)(613.69794678,78.55931936)
\curveto(613.69794841,78.58931777)(613.69294842,78.61431774)(613.68294678,78.63431936)
}
}
{
\newrgbcolor{curcolor}{0 0 0}
\pscustom[linestyle=none,fillstyle=solid,fillcolor=curcolor]
{
}
}
{
\newrgbcolor{curcolor}{0 0 0}
\pscustom[linestyle=none,fillstyle=solid,fillcolor=curcolor]
{
\newpath
\moveto(604.71294678,89.26463186)
\curveto(604.70295741,89.95462722)(604.82295729,90.55462662)(605.07294678,91.06463186)
\curveto(605.32295679,91.58462559)(605.65795645,91.9796252)(606.07794678,92.24963186)
\curveto(606.15795595,92.29962488)(606.24795586,92.34462483)(606.34794678,92.38463186)
\curveto(606.43795567,92.42462475)(606.53295558,92.46962471)(606.63294678,92.51963186)
\curveto(606.73295538,92.55962462)(606.83295528,92.58962459)(606.93294678,92.60963186)
\curveto(607.03295508,92.62962455)(607.13795497,92.64962453)(607.24794678,92.66963186)
\curveto(607.29795481,92.68962449)(607.34295477,92.69462448)(607.38294678,92.68463186)
\curveto(607.42295469,92.6746245)(607.46795464,92.6796245)(607.51794678,92.69963186)
\curveto(607.56795454,92.70962447)(607.65295446,92.71462446)(607.77294678,92.71463186)
\curveto(607.88295423,92.71462446)(607.96795414,92.70962447)(608.02794678,92.69963186)
\curveto(608.08795402,92.6796245)(608.14795396,92.66962451)(608.20794678,92.66963186)
\curveto(608.26795384,92.6796245)(608.32795378,92.6746245)(608.38794678,92.65463186)
\curveto(608.52795358,92.61462456)(608.66295345,92.5796246)(608.79294678,92.54963186)
\curveto(608.92295319,92.51962466)(609.04795306,92.4796247)(609.16794678,92.42963186)
\curveto(609.3079528,92.36962481)(609.43295268,92.29962488)(609.54294678,92.21963186)
\curveto(609.65295246,92.14962503)(609.76295235,92.0746251)(609.87294678,91.99463186)
\lineto(609.93294678,91.93463186)
\curveto(609.95295216,91.92462525)(609.97295214,91.90962527)(609.99294678,91.88963186)
\curveto(610.15295196,91.76962541)(610.29795181,91.63462554)(610.42794678,91.48463186)
\curveto(610.55795155,91.33462584)(610.68295143,91.174626)(610.80294678,91.00463186)
\curveto(611.02295109,90.69462648)(611.22795088,90.39962678)(611.41794678,90.11963186)
\curveto(611.55795055,89.88962729)(611.69295042,89.65962752)(611.82294678,89.42963186)
\curveto(611.95295016,89.20962797)(612.08795002,88.98962819)(612.22794678,88.76963186)
\curveto(612.39794971,88.51962866)(612.57794953,88.2796289)(612.76794678,88.04963186)
\curveto(612.95794915,87.82962935)(613.18294893,87.63962954)(613.44294678,87.47963186)
\curveto(613.50294861,87.43962974)(613.56294855,87.40462977)(613.62294678,87.37463186)
\curveto(613.67294844,87.34462983)(613.73794837,87.31462986)(613.81794678,87.28463186)
\curveto(613.88794822,87.26462991)(613.94794816,87.25962992)(613.99794678,87.26963186)
\curveto(614.06794804,87.28962989)(614.12294799,87.32462985)(614.16294678,87.37463186)
\curveto(614.19294792,87.42462975)(614.2129479,87.48462969)(614.22294678,87.55463186)
\lineto(614.22294678,87.79463186)
\lineto(614.22294678,88.54463186)
\lineto(614.22294678,91.34963186)
\lineto(614.22294678,92.00963186)
\curveto(614.22294789,92.09962508)(614.22794788,92.18462499)(614.23794678,92.26463186)
\curveto(614.23794787,92.34462483)(614.25794785,92.40962477)(614.29794678,92.45963186)
\curveto(614.33794777,92.50962467)(614.4129477,92.54962463)(614.52294678,92.57963186)
\curveto(614.62294749,92.61962456)(614.72294739,92.62962455)(614.82294678,92.60963186)
\lineto(614.95794678,92.60963186)
\curveto(615.02794708,92.58962459)(615.08794702,92.56962461)(615.13794678,92.54963186)
\curveto(615.18794692,92.52962465)(615.22794688,92.49462468)(615.25794678,92.44463186)
\curveto(615.29794681,92.39462478)(615.31794679,92.32462485)(615.31794678,92.23463186)
\lineto(615.31794678,91.96463186)
\lineto(615.31794678,91.06463186)
\lineto(615.31794678,87.55463186)
\lineto(615.31794678,86.48963186)
\curveto(615.31794679,86.40963077)(615.32294679,86.31963086)(615.33294678,86.21963186)
\curveto(615.33294678,86.11963106)(615.32294679,86.03463114)(615.30294678,85.96463186)
\curveto(615.23294688,85.75463142)(615.05294706,85.68963149)(614.76294678,85.76963186)
\curveto(614.72294739,85.7796314)(614.68794742,85.7796314)(614.65794678,85.76963186)
\curveto(614.61794749,85.76963141)(614.57294754,85.7796314)(614.52294678,85.79963186)
\curveto(614.44294767,85.81963136)(614.35794775,85.83963134)(614.26794678,85.85963186)
\curveto(614.17794793,85.8796313)(614.09294802,85.90463127)(614.01294678,85.93463186)
\curveto(613.52294859,86.09463108)(613.107949,86.29463088)(612.76794678,86.53463186)
\curveto(612.51794959,86.71463046)(612.29294982,86.91963026)(612.09294678,87.14963186)
\curveto(611.88295023,87.3796298)(611.68795042,87.61962956)(611.50794678,87.86963186)
\curveto(611.32795078,88.12962905)(611.15795095,88.39462878)(610.99794678,88.66463186)
\curveto(610.82795128,88.94462823)(610.65295146,89.21462796)(610.47294678,89.47463186)
\curveto(610.39295172,89.58462759)(610.31795179,89.68962749)(610.24794678,89.78963186)
\curveto(610.17795193,89.89962728)(610.10295201,90.00962717)(610.02294678,90.11963186)
\curveto(609.99295212,90.15962702)(609.96295215,90.19462698)(609.93294678,90.22463186)
\curveto(609.89295222,90.26462691)(609.86295225,90.30462687)(609.84294678,90.34463186)
\curveto(609.73295238,90.48462669)(609.6079525,90.60962657)(609.46794678,90.71963186)
\curveto(609.43795267,90.73962644)(609.4129527,90.76462641)(609.39294678,90.79463186)
\curveto(609.36295275,90.82462635)(609.33295278,90.84962633)(609.30294678,90.86963186)
\curveto(609.20295291,90.94962623)(609.10295301,91.01462616)(609.00294678,91.06463186)
\curveto(608.90295321,91.12462605)(608.79295332,91.179626)(608.67294678,91.22963186)
\curveto(608.60295351,91.25962592)(608.52795358,91.2796259)(608.44794678,91.28963186)
\lineto(608.20794678,91.34963186)
\lineto(608.11794678,91.34963186)
\curveto(608.08795402,91.35962582)(608.05795405,91.36462581)(608.02794678,91.36463186)
\curveto(607.95795415,91.38462579)(607.86295425,91.38962579)(607.74294678,91.37963186)
\curveto(607.6129545,91.3796258)(607.5129546,91.36962581)(607.44294678,91.34963186)
\curveto(607.36295475,91.32962585)(607.28795482,91.30962587)(607.21794678,91.28963186)
\curveto(607.13795497,91.2796259)(607.05795505,91.25962592)(606.97794678,91.22963186)
\curveto(606.73795537,91.11962606)(606.53795557,90.96962621)(606.37794678,90.77963186)
\curveto(606.2079559,90.59962658)(606.06795604,90.3796268)(605.95794678,90.11963186)
\curveto(605.93795617,90.04962713)(605.92295619,89.9796272)(605.91294678,89.90963186)
\curveto(605.89295622,89.83962734)(605.87295624,89.76462741)(605.85294678,89.68463186)
\curveto(605.83295628,89.60462757)(605.82295629,89.49462768)(605.82294678,89.35463186)
\curveto(605.82295629,89.22462795)(605.83295628,89.11962806)(605.85294678,89.03963186)
\curveto(605.86295625,88.9796282)(605.86795624,88.92462825)(605.86794678,88.87463186)
\curveto(605.86795624,88.82462835)(605.87795623,88.7746284)(605.89794678,88.72463186)
\curveto(605.93795617,88.62462855)(605.97795613,88.52962865)(606.01794678,88.43963186)
\curveto(606.05795605,88.35962882)(606.10295601,88.2796289)(606.15294678,88.19963186)
\curveto(606.17295594,88.16962901)(606.19795591,88.13962904)(606.22794678,88.10963186)
\curveto(606.25795585,88.08962909)(606.28295583,88.06462911)(606.30294678,88.03463186)
\lineto(606.37794678,87.95963186)
\curveto(606.39795571,87.92962925)(606.41795569,87.90462927)(606.43794678,87.88463186)
\lineto(606.64794678,87.73463186)
\curveto(606.7079554,87.69462948)(606.77295534,87.64962953)(606.84294678,87.59963186)
\curveto(606.93295518,87.53962964)(607.03795507,87.48962969)(607.15794678,87.44963186)
\curveto(607.26795484,87.41962976)(607.37795473,87.38462979)(607.48794678,87.34463186)
\curveto(607.59795451,87.30462987)(607.74295437,87.2796299)(607.92294678,87.26963186)
\curveto(608.09295402,87.25962992)(608.21795389,87.22962995)(608.29794678,87.17963186)
\curveto(608.37795373,87.12963005)(608.42295369,87.05463012)(608.43294678,86.95463186)
\curveto(608.44295367,86.85463032)(608.44795366,86.74463043)(608.44794678,86.62463186)
\curveto(608.44795366,86.58463059)(608.45295366,86.54463063)(608.46294678,86.50463186)
\curveto(608.46295365,86.46463071)(608.45795365,86.42963075)(608.44794678,86.39963186)
\curveto(608.42795368,86.34963083)(608.41795369,86.29963088)(608.41794678,86.24963186)
\curveto(608.41795369,86.20963097)(608.4079537,86.16963101)(608.38794678,86.12963186)
\curveto(608.32795378,86.03963114)(608.19295392,85.99463118)(607.98294678,85.99463186)
\lineto(607.86294678,85.99463186)
\curveto(607.80295431,86.00463117)(607.74295437,86.00963117)(607.68294678,86.00963186)
\curveto(607.6129545,86.01963116)(607.54795456,86.02963115)(607.48794678,86.03963186)
\curveto(607.37795473,86.05963112)(607.27795483,86.0796311)(607.18794678,86.09963186)
\curveto(607.08795502,86.11963106)(606.99295512,86.14963103)(606.90294678,86.18963186)
\curveto(606.83295528,86.20963097)(606.77295534,86.22963095)(606.72294678,86.24963186)
\lineto(606.54294678,86.30963186)
\curveto(606.28295583,86.42963075)(606.03795607,86.58463059)(605.80794678,86.77463186)
\curveto(605.57795653,86.9746302)(605.39295672,87.18962999)(605.25294678,87.41963186)
\curveto(605.17295694,87.52962965)(605.107957,87.64462953)(605.05794678,87.76463186)
\lineto(604.90794678,88.15463186)
\curveto(604.85795725,88.26462891)(604.82795728,88.3796288)(604.81794678,88.49963186)
\curveto(604.79795731,88.61962856)(604.77295734,88.74462843)(604.74294678,88.87463186)
\curveto(604.74295737,88.94462823)(604.74295737,89.00962817)(604.74294678,89.06963186)
\curveto(604.73295738,89.12962805)(604.72295739,89.19462798)(604.71294678,89.26463186)
}
}
{
\newrgbcolor{curcolor}{0 0 0}
\pscustom[linestyle=none,fillstyle=solid,fillcolor=curcolor]
{
\newpath
\moveto(610.23294678,101.36424123)
\lineto(610.48794678,101.36424123)
\curveto(610.56795154,101.37423353)(610.64295147,101.36923353)(610.71294678,101.34924123)
\lineto(610.95294678,101.34924123)
\lineto(611.11794678,101.34924123)
\curveto(611.21795089,101.32923357)(611.32295079,101.31923358)(611.43294678,101.31924123)
\curveto(611.53295058,101.31923358)(611.63295048,101.30923359)(611.73294678,101.28924123)
\lineto(611.88294678,101.28924123)
\curveto(612.02295009,101.25923364)(612.16294995,101.23923366)(612.30294678,101.22924123)
\curveto(612.43294968,101.21923368)(612.56294955,101.19423371)(612.69294678,101.15424123)
\curveto(612.77294934,101.13423377)(612.85794925,101.11423379)(612.94794678,101.09424123)
\lineto(613.18794678,101.03424123)
\lineto(613.48794678,100.91424123)
\curveto(613.57794853,100.88423402)(613.66794844,100.84923405)(613.75794678,100.80924123)
\curveto(613.97794813,100.70923419)(614.19294792,100.57423433)(614.40294678,100.40424123)
\curveto(614.6129475,100.24423466)(614.78294733,100.06923483)(614.91294678,99.87924123)
\curveto(614.95294716,99.82923507)(614.99294712,99.76923513)(615.03294678,99.69924123)
\curveto(615.06294705,99.63923526)(615.09794701,99.57923532)(615.13794678,99.51924123)
\curveto(615.18794692,99.43923546)(615.22794688,99.34423556)(615.25794678,99.23424123)
\curveto(615.28794682,99.12423578)(615.31794679,99.01923588)(615.34794678,98.91924123)
\curveto(615.38794672,98.80923609)(615.4129467,98.6992362)(615.42294678,98.58924123)
\curveto(615.43294668,98.47923642)(615.44794666,98.36423654)(615.46794678,98.24424123)
\curveto(615.47794663,98.2042367)(615.47794663,98.15923674)(615.46794678,98.10924123)
\curveto(615.46794664,98.06923683)(615.47294664,98.02923687)(615.48294678,97.98924123)
\curveto(615.49294662,97.94923695)(615.49794661,97.89423701)(615.49794678,97.82424123)
\curveto(615.49794661,97.75423715)(615.49294662,97.7042372)(615.48294678,97.67424123)
\curveto(615.46294665,97.62423728)(615.45794665,97.57923732)(615.46794678,97.53924123)
\curveto(615.47794663,97.4992374)(615.47794663,97.46423744)(615.46794678,97.43424123)
\lineto(615.46794678,97.34424123)
\curveto(615.44794666,97.28423762)(615.43294668,97.21923768)(615.42294678,97.14924123)
\curveto(615.42294669,97.08923781)(615.41794669,97.02423788)(615.40794678,96.95424123)
\curveto(615.35794675,96.78423812)(615.3079468,96.62423828)(615.25794678,96.47424123)
\curveto(615.2079469,96.32423858)(615.14294697,96.17923872)(615.06294678,96.03924123)
\curveto(615.02294709,95.98923891)(614.99294712,95.93423897)(614.97294678,95.87424123)
\curveto(614.94294717,95.82423908)(614.9079472,95.77423913)(614.86794678,95.72424123)
\curveto(614.68794742,95.48423942)(614.46794764,95.28423962)(614.20794678,95.12424123)
\curveto(613.94794816,94.96423994)(613.66294845,94.82424008)(613.35294678,94.70424123)
\curveto(613.2129489,94.64424026)(613.07294904,94.5992403)(612.93294678,94.56924123)
\curveto(612.78294933,94.53924036)(612.62794948,94.5042404)(612.46794678,94.46424123)
\curveto(612.35794975,94.44424046)(612.24794986,94.42924047)(612.13794678,94.41924123)
\curveto(612.02795008,94.40924049)(611.91795019,94.39424051)(611.80794678,94.37424123)
\curveto(611.76795034,94.36424054)(611.72795038,94.35924054)(611.68794678,94.35924123)
\curveto(611.64795046,94.36924053)(611.6079505,94.36924053)(611.56794678,94.35924123)
\curveto(611.51795059,94.34924055)(611.46795064,94.34424056)(611.41794678,94.34424123)
\lineto(611.25294678,94.34424123)
\curveto(611.20295091,94.32424058)(611.15295096,94.31924058)(611.10294678,94.32924123)
\curveto(611.04295107,94.33924056)(610.98795112,94.33924056)(610.93794678,94.32924123)
\curveto(610.89795121,94.31924058)(610.85295126,94.31924058)(610.80294678,94.32924123)
\curveto(610.75295136,94.33924056)(610.70295141,94.33424057)(610.65294678,94.31424123)
\curveto(610.58295153,94.29424061)(610.5079516,94.28924061)(610.42794678,94.29924123)
\curveto(610.33795177,94.30924059)(610.25295186,94.31424059)(610.17294678,94.31424123)
\curveto(610.08295203,94.31424059)(609.98295213,94.30924059)(609.87294678,94.29924123)
\curveto(609.75295236,94.28924061)(609.65295246,94.29424061)(609.57294678,94.31424123)
\lineto(609.28794678,94.31424123)
\lineto(608.65794678,94.35924123)
\curveto(608.55795355,94.36924053)(608.46295365,94.37924052)(608.37294678,94.38924123)
\lineto(608.07294678,94.41924123)
\curveto(608.02295409,94.43924046)(607.97295414,94.44424046)(607.92294678,94.43424123)
\curveto(607.86295425,94.43424047)(607.8079543,94.44424046)(607.75794678,94.46424123)
\curveto(607.58795452,94.51424039)(607.42295469,94.55424035)(607.26294678,94.58424123)
\curveto(607.09295502,94.61424029)(606.93295518,94.66424024)(606.78294678,94.73424123)
\curveto(606.32295579,94.92423998)(605.94795616,95.14423976)(605.65794678,95.39424123)
\curveto(605.36795674,95.65423925)(605.12295699,96.01423889)(604.92294678,96.47424123)
\curveto(604.87295724,96.6042383)(604.83795727,96.73423817)(604.81794678,96.86424123)
\curveto(604.79795731,97.0042379)(604.77295734,97.14423776)(604.74294678,97.28424123)
\curveto(604.73295738,97.35423755)(604.72795738,97.41923748)(604.72794678,97.47924123)
\curveto(604.72795738,97.53923736)(604.72295739,97.6042373)(604.71294678,97.67424123)
\curveto(604.69295742,98.5042364)(604.84295727,99.17423573)(605.16294678,99.68424123)
\curveto(605.47295664,100.19423471)(605.9129562,100.57423433)(606.48294678,100.82424123)
\curveto(606.60295551,100.87423403)(606.72795538,100.91923398)(606.85794678,100.95924123)
\curveto(606.98795512,100.9992339)(607.12295499,101.04423386)(607.26294678,101.09424123)
\curveto(607.34295477,101.11423379)(607.42795468,101.12923377)(607.51794678,101.13924123)
\lineto(607.75794678,101.19924123)
\curveto(607.86795424,101.22923367)(607.97795413,101.24423366)(608.08794678,101.24424123)
\curveto(608.19795391,101.25423365)(608.3079538,101.26923363)(608.41794678,101.28924123)
\curveto(608.46795364,101.30923359)(608.5129536,101.31423359)(608.55294678,101.30424123)
\curveto(608.59295352,101.3042336)(608.63295348,101.30923359)(608.67294678,101.31924123)
\curveto(608.72295339,101.32923357)(608.77795333,101.32923357)(608.83794678,101.31924123)
\curveto(608.88795322,101.31923358)(608.93795317,101.32423358)(608.98794678,101.33424123)
\lineto(609.12294678,101.33424123)
\curveto(609.18295293,101.35423355)(609.25295286,101.35423355)(609.33294678,101.33424123)
\curveto(609.40295271,101.32423358)(609.46795264,101.32923357)(609.52794678,101.34924123)
\curveto(609.55795255,101.35923354)(609.59795251,101.36423354)(609.64794678,101.36424123)
\lineto(609.76794678,101.36424123)
\lineto(610.23294678,101.36424123)
\moveto(612.55794678,99.81924123)
\curveto(612.23794987,99.91923498)(611.87295024,99.97923492)(611.46294678,99.99924123)
\curveto(611.05295106,100.01923488)(610.64295147,100.02923487)(610.23294678,100.02924123)
\curveto(609.80295231,100.02923487)(609.38295273,100.01923488)(608.97294678,99.99924123)
\curveto(608.56295355,99.97923492)(608.17795393,99.93423497)(607.81794678,99.86424123)
\curveto(607.45795465,99.79423511)(607.13795497,99.68423522)(606.85794678,99.53424123)
\curveto(606.56795554,99.39423551)(606.33295578,99.1992357)(606.15294678,98.94924123)
\curveto(606.04295607,98.78923611)(605.96295615,98.60923629)(605.91294678,98.40924123)
\curveto(605.85295626,98.20923669)(605.82295629,97.96423694)(605.82294678,97.67424123)
\curveto(605.84295627,97.65423725)(605.85295626,97.61923728)(605.85294678,97.56924123)
\curveto(605.84295627,97.51923738)(605.84295627,97.47923742)(605.85294678,97.44924123)
\curveto(605.87295624,97.36923753)(605.89295622,97.29423761)(605.91294678,97.22424123)
\curveto(605.92295619,97.16423774)(605.94295617,97.0992378)(605.97294678,97.02924123)
\curveto(606.09295602,96.75923814)(606.26295585,96.53923836)(606.48294678,96.36924123)
\curveto(606.69295542,96.20923869)(606.93795517,96.07423883)(607.21794678,95.96424123)
\curveto(607.32795478,95.91423899)(607.44795466,95.87423903)(607.57794678,95.84424123)
\curveto(607.69795441,95.82423908)(607.82295429,95.7992391)(607.95294678,95.76924123)
\curveto(608.00295411,95.74923915)(608.05795405,95.73923916)(608.11794678,95.73924123)
\curveto(608.16795394,95.73923916)(608.21795389,95.73423917)(608.26794678,95.72424123)
\curveto(608.35795375,95.71423919)(608.45295366,95.7042392)(608.55294678,95.69424123)
\curveto(608.64295347,95.68423922)(608.73795337,95.67423923)(608.83794678,95.66424123)
\curveto(608.91795319,95.66423924)(609.00295311,95.65923924)(609.09294678,95.64924123)
\lineto(609.33294678,95.64924123)
\lineto(609.51294678,95.64924123)
\curveto(609.54295257,95.63923926)(609.57795253,95.63423927)(609.61794678,95.63424123)
\lineto(609.75294678,95.63424123)
\lineto(610.20294678,95.63424123)
\curveto(610.28295183,95.63423927)(610.36795174,95.62923927)(610.45794678,95.61924123)
\curveto(610.53795157,95.61923928)(610.6129515,95.62923927)(610.68294678,95.64924123)
\lineto(610.95294678,95.64924123)
\curveto(610.97295114,95.64923925)(611.00295111,95.64423926)(611.04294678,95.63424123)
\curveto(611.07295104,95.63423927)(611.09795101,95.63923926)(611.11794678,95.64924123)
\curveto(611.21795089,95.65923924)(611.31795079,95.66423924)(611.41794678,95.66424123)
\curveto(611.5079506,95.67423923)(611.6079505,95.68423922)(611.71794678,95.69424123)
\curveto(611.83795027,95.72423918)(611.96295015,95.73923916)(612.09294678,95.73924123)
\curveto(612.2129499,95.74923915)(612.32794978,95.77423913)(612.43794678,95.81424123)
\curveto(612.73794937,95.89423901)(613.00294911,95.97923892)(613.23294678,96.06924123)
\curveto(613.46294865,96.16923873)(613.67794843,96.31423859)(613.87794678,96.50424123)
\curveto(614.07794803,96.71423819)(614.22794788,96.97923792)(614.32794678,97.29924123)
\curveto(614.34794776,97.33923756)(614.35794775,97.37423753)(614.35794678,97.40424123)
\curveto(614.34794776,97.44423746)(614.35294776,97.48923741)(614.37294678,97.53924123)
\curveto(614.38294773,97.57923732)(614.39294772,97.64923725)(614.40294678,97.74924123)
\curveto(614.4129477,97.85923704)(614.4079477,97.94423696)(614.38794678,98.00424123)
\curveto(614.36794774,98.07423683)(614.35794775,98.14423676)(614.35794678,98.21424123)
\curveto(614.34794776,98.28423662)(614.33294778,98.34923655)(614.31294678,98.40924123)
\curveto(614.25294786,98.60923629)(614.16794794,98.78923611)(614.05794678,98.94924123)
\curveto(614.03794807,98.97923592)(614.01794809,99.0042359)(613.99794678,99.02424123)
\lineto(613.93794678,99.08424123)
\curveto(613.91794819,99.12423578)(613.87794823,99.17423573)(613.81794678,99.23424123)
\curveto(613.67794843,99.33423557)(613.54794856,99.41923548)(613.42794678,99.48924123)
\curveto(613.3079488,99.55923534)(613.16294895,99.62923527)(612.99294678,99.69924123)
\curveto(612.92294919,99.72923517)(612.85294926,99.74923515)(612.78294678,99.75924123)
\curveto(612.7129494,99.77923512)(612.63794947,99.7992351)(612.55794678,99.81924123)
}
}
{
\newrgbcolor{curcolor}{0 0 0}
\pscustom[linestyle=none,fillstyle=solid,fillcolor=curcolor]
{
\newpath
\moveto(604.71294678,106.77385061)
\curveto(604.7129574,106.87384575)(604.72295739,106.96884566)(604.74294678,107.05885061)
\curveto(604.75295736,107.14884548)(604.78295733,107.21384541)(604.83294678,107.25385061)
\curveto(604.9129572,107.31384531)(605.01795709,107.34384528)(605.14794678,107.34385061)
\lineto(605.53794678,107.34385061)
\lineto(607.03794678,107.34385061)
\lineto(613.42794678,107.34385061)
\lineto(614.59794678,107.34385061)
\lineto(614.91294678,107.34385061)
\curveto(615.0129471,107.35384527)(615.09294702,107.33884529)(615.15294678,107.29885061)
\curveto(615.23294688,107.24884538)(615.28294683,107.17384545)(615.30294678,107.07385061)
\curveto(615.3129468,106.98384564)(615.31794679,106.87384575)(615.31794678,106.74385061)
\lineto(615.31794678,106.51885061)
\curveto(615.29794681,106.43884619)(615.28294683,106.36884626)(615.27294678,106.30885061)
\curveto(615.25294686,106.24884638)(615.2129469,106.19884643)(615.15294678,106.15885061)
\curveto(615.09294702,106.11884651)(615.01794709,106.09884653)(614.92794678,106.09885061)
\lineto(614.62794678,106.09885061)
\lineto(613.53294678,106.09885061)
\lineto(608.19294678,106.09885061)
\curveto(608.10295401,106.07884655)(608.02795408,106.06384656)(607.96794678,106.05385061)
\curveto(607.89795421,106.05384657)(607.83795427,106.0238466)(607.78794678,105.96385061)
\curveto(607.73795437,105.89384673)(607.7129544,105.80384682)(607.71294678,105.69385061)
\curveto(607.70295441,105.59384703)(607.69795441,105.48384714)(607.69794678,105.36385061)
\lineto(607.69794678,104.22385061)
\lineto(607.69794678,103.72885061)
\curveto(607.68795442,103.56884906)(607.62795448,103.45884917)(607.51794678,103.39885061)
\curveto(607.48795462,103.37884925)(607.45795465,103.36884926)(607.42794678,103.36885061)
\curveto(607.38795472,103.36884926)(607.34295477,103.36384926)(607.29294678,103.35385061)
\curveto(607.17295494,103.33384929)(607.06295505,103.33884929)(606.96294678,103.36885061)
\curveto(606.86295525,103.40884922)(606.79295532,103.46384916)(606.75294678,103.53385061)
\curveto(606.70295541,103.61384901)(606.67795543,103.73384889)(606.67794678,103.89385061)
\curveto(606.67795543,104.05384857)(606.66295545,104.18884844)(606.63294678,104.29885061)
\curveto(606.62295549,104.34884828)(606.61795549,104.40384822)(606.61794678,104.46385061)
\curveto(606.6079555,104.5238481)(606.59295552,104.58384804)(606.57294678,104.64385061)
\curveto(606.52295559,104.79384783)(606.47295564,104.93884769)(606.42294678,105.07885061)
\curveto(606.36295575,105.21884741)(606.29295582,105.35384727)(606.21294678,105.48385061)
\curveto(606.12295599,105.623847)(606.01795609,105.74384688)(605.89794678,105.84385061)
\curveto(605.77795633,105.94384668)(605.64795646,106.03884659)(605.50794678,106.12885061)
\curveto(605.4079567,106.18884644)(605.29795681,106.23384639)(605.17794678,106.26385061)
\curveto(605.05795705,106.30384632)(604.95295716,106.35384627)(604.86294678,106.41385061)
\curveto(604.80295731,106.46384616)(604.76295735,106.53384609)(604.74294678,106.62385061)
\curveto(604.73295738,106.64384598)(604.72795738,106.66884596)(604.72794678,106.69885061)
\curveto(604.72795738,106.7288459)(604.72295739,106.75384587)(604.71294678,106.77385061)
}
}
{
\newrgbcolor{curcolor}{0 0 0}
\pscustom[linestyle=none,fillstyle=solid,fillcolor=curcolor]
{
\newpath
\moveto(604.71294678,115.12345998)
\curveto(604.7129574,115.22345513)(604.72295739,115.31845503)(604.74294678,115.40845998)
\curveto(604.75295736,115.49845485)(604.78295733,115.56345479)(604.83294678,115.60345998)
\curveto(604.9129572,115.66345469)(605.01795709,115.69345466)(605.14794678,115.69345998)
\lineto(605.53794678,115.69345998)
\lineto(607.03794678,115.69345998)
\lineto(613.42794678,115.69345998)
\lineto(614.59794678,115.69345998)
\lineto(614.91294678,115.69345998)
\curveto(615.0129471,115.70345465)(615.09294702,115.68845466)(615.15294678,115.64845998)
\curveto(615.23294688,115.59845475)(615.28294683,115.52345483)(615.30294678,115.42345998)
\curveto(615.3129468,115.33345502)(615.31794679,115.22345513)(615.31794678,115.09345998)
\lineto(615.31794678,114.86845998)
\curveto(615.29794681,114.78845556)(615.28294683,114.71845563)(615.27294678,114.65845998)
\curveto(615.25294686,114.59845575)(615.2129469,114.5484558)(615.15294678,114.50845998)
\curveto(615.09294702,114.46845588)(615.01794709,114.4484559)(614.92794678,114.44845998)
\lineto(614.62794678,114.44845998)
\lineto(613.53294678,114.44845998)
\lineto(608.19294678,114.44845998)
\curveto(608.10295401,114.42845592)(608.02795408,114.41345594)(607.96794678,114.40345998)
\curveto(607.89795421,114.40345595)(607.83795427,114.37345598)(607.78794678,114.31345998)
\curveto(607.73795437,114.24345611)(607.7129544,114.1534562)(607.71294678,114.04345998)
\curveto(607.70295441,113.94345641)(607.69795441,113.83345652)(607.69794678,113.71345998)
\lineto(607.69794678,112.57345998)
\lineto(607.69794678,112.07845998)
\curveto(607.68795442,111.91845843)(607.62795448,111.80845854)(607.51794678,111.74845998)
\curveto(607.48795462,111.72845862)(607.45795465,111.71845863)(607.42794678,111.71845998)
\curveto(607.38795472,111.71845863)(607.34295477,111.71345864)(607.29294678,111.70345998)
\curveto(607.17295494,111.68345867)(607.06295505,111.68845866)(606.96294678,111.71845998)
\curveto(606.86295525,111.75845859)(606.79295532,111.81345854)(606.75294678,111.88345998)
\curveto(606.70295541,111.96345839)(606.67795543,112.08345827)(606.67794678,112.24345998)
\curveto(606.67795543,112.40345795)(606.66295545,112.53845781)(606.63294678,112.64845998)
\curveto(606.62295549,112.69845765)(606.61795549,112.7534576)(606.61794678,112.81345998)
\curveto(606.6079555,112.87345748)(606.59295552,112.93345742)(606.57294678,112.99345998)
\curveto(606.52295559,113.14345721)(606.47295564,113.28845706)(606.42294678,113.42845998)
\curveto(606.36295575,113.56845678)(606.29295582,113.70345665)(606.21294678,113.83345998)
\curveto(606.12295599,113.97345638)(606.01795609,114.09345626)(605.89794678,114.19345998)
\curveto(605.77795633,114.29345606)(605.64795646,114.38845596)(605.50794678,114.47845998)
\curveto(605.4079567,114.53845581)(605.29795681,114.58345577)(605.17794678,114.61345998)
\curveto(605.05795705,114.6534557)(604.95295716,114.70345565)(604.86294678,114.76345998)
\curveto(604.80295731,114.81345554)(604.76295735,114.88345547)(604.74294678,114.97345998)
\curveto(604.73295738,114.99345536)(604.72795738,115.01845533)(604.72794678,115.04845998)
\curveto(604.72795738,115.07845527)(604.72295739,115.10345525)(604.71294678,115.12345998)
}
}
{
\newrgbcolor{curcolor}{0 0 0}
\pscustom[linestyle=none,fillstyle=solid,fillcolor=curcolor]
{
\newpath
\moveto(626.58423584,29.18119436)
\lineto(626.58423584,30.09619436)
\curveto(626.58424653,30.19619171)(626.58424653,30.29119161)(626.58423584,30.38119436)
\curveto(626.58424653,30.47119143)(626.60424651,30.54619136)(626.64423584,30.60619436)
\curveto(626.70424641,30.69619121)(626.78424633,30.75619115)(626.88423584,30.78619436)
\curveto(626.98424613,30.82619108)(627.08924603,30.87119103)(627.19923584,30.92119436)
\curveto(627.38924573,31.0011909)(627.57924554,31.07119083)(627.76923584,31.13119436)
\curveto(627.95924516,31.2011907)(628.14924497,31.27619063)(628.33923584,31.35619436)
\curveto(628.5192446,31.42619048)(628.70424441,31.49119041)(628.89423584,31.55119436)
\curveto(629.07424404,31.61119029)(629.25424386,31.68119022)(629.43423584,31.76119436)
\curveto(629.57424354,31.82119008)(629.7192434,31.87619003)(629.86923584,31.92619436)
\curveto(630.0192431,31.97618993)(630.16424295,32.03118987)(630.30423584,32.09119436)
\curveto(630.75424236,32.27118963)(631.20924191,32.44118946)(631.66923584,32.60119436)
\curveto(632.119241,32.76118914)(632.56924055,32.93118897)(633.01923584,33.11119436)
\curveto(633.06924005,33.13118877)(633.11924,33.14618876)(633.16923584,33.15619436)
\lineto(633.31923584,33.21619436)
\curveto(633.53923958,33.3061886)(633.76423935,33.39118851)(633.99423584,33.47119436)
\curveto(634.2142389,33.55118835)(634.43423868,33.63618827)(634.65423584,33.72619436)
\curveto(634.74423837,33.76618814)(634.85423826,33.8061881)(634.98423584,33.84619436)
\curveto(635.10423801,33.88618802)(635.17423794,33.95118795)(635.19423584,34.04119436)
\curveto(635.20423791,34.08118782)(635.20423791,34.11118779)(635.19423584,34.13119436)
\lineto(635.13423584,34.19119436)
\curveto(635.08423803,34.24118766)(635.02923809,34.27618763)(634.96923584,34.29619436)
\curveto(634.90923821,34.32618758)(634.84423827,34.35618755)(634.77423584,34.38619436)
\lineto(634.14423584,34.62619436)
\curveto(633.92423919,34.7061872)(633.70923941,34.78618712)(633.49923584,34.86619436)
\lineto(633.34923584,34.92619436)
\lineto(633.16923584,34.98619436)
\curveto(632.97924014,35.06618684)(632.78924033,35.13618677)(632.59923584,35.19619436)
\curveto(632.39924072,35.26618664)(632.19924092,35.34118656)(631.99923584,35.42119436)
\curveto(631.4192417,35.66118624)(630.83424228,35.88118602)(630.24423584,36.08119436)
\curveto(629.65424346,36.29118561)(629.06924405,36.51618539)(628.48923584,36.75619436)
\curveto(628.28924483,36.83618507)(628.08424503,36.91118499)(627.87423584,36.98119436)
\curveto(627.66424545,37.06118484)(627.45924566,37.14118476)(627.25923584,37.22119436)
\curveto(627.17924594,37.26118464)(627.07924604,37.29618461)(626.95923584,37.32619436)
\curveto(626.83924628,37.36618454)(626.75424636,37.42118448)(626.70423584,37.49119436)
\curveto(626.66424645,37.55118435)(626.63424648,37.62618428)(626.61423584,37.71619436)
\curveto(626.59424652,37.81618409)(626.58424653,37.92618398)(626.58423584,38.04619436)
\curveto(626.57424654,38.16618374)(626.57424654,38.28618362)(626.58423584,38.40619436)
\curveto(626.58424653,38.52618338)(626.58424653,38.63618327)(626.58423584,38.73619436)
\curveto(626.58424653,38.82618308)(626.58424653,38.91618299)(626.58423584,39.00619436)
\curveto(626.58424653,39.1061828)(626.60424651,39.18118272)(626.64423584,39.23119436)
\curveto(626.69424642,39.32118258)(626.78424633,39.37118253)(626.91423584,39.38119436)
\curveto(627.04424607,39.39118251)(627.18424593,39.39618251)(627.33423584,39.39619436)
\lineto(628.98423584,39.39619436)
\lineto(635.25423584,39.39619436)
\lineto(636.51423584,39.39619436)
\curveto(636.62423649,39.39618251)(636.73423638,39.39618251)(636.84423584,39.39619436)
\curveto(636.95423616,39.4061825)(637.03923608,39.38618252)(637.09923584,39.33619436)
\curveto(637.15923596,39.3061826)(637.19923592,39.26118264)(637.21923584,39.20119436)
\curveto(637.22923589,39.14118276)(637.24423587,39.07118283)(637.26423584,38.99119436)
\lineto(637.26423584,38.75119436)
\lineto(637.26423584,38.39119436)
\curveto(637.25423586,38.28118362)(637.20923591,38.2011837)(637.12923584,38.15119436)
\curveto(637.09923602,38.13118377)(637.06923605,38.11618379)(637.03923584,38.10619436)
\curveto(636.99923612,38.1061838)(636.95423616,38.09618381)(636.90423584,38.07619436)
\lineto(636.73923584,38.07619436)
\curveto(636.67923644,38.06618384)(636.60923651,38.06118384)(636.52923584,38.06119436)
\curveto(636.44923667,38.07118383)(636.37423674,38.07618383)(636.30423584,38.07619436)
\lineto(635.46423584,38.07619436)
\lineto(631.03923584,38.07619436)
\curveto(630.78924233,38.07618383)(630.53924258,38.07618383)(630.28923584,38.07619436)
\curveto(630.02924309,38.07618383)(629.77924334,38.07118383)(629.53923584,38.06119436)
\curveto(629.43924368,38.06118384)(629.32924379,38.05618385)(629.20923584,38.04619436)
\curveto(629.08924403,38.03618387)(629.02924409,37.98118392)(629.02923584,37.88119436)
\lineto(629.04423584,37.88119436)
\curveto(629.06424405,37.81118409)(629.12924399,37.75118415)(629.23923584,37.70119436)
\curveto(629.34924377,37.66118424)(629.44424367,37.62618428)(629.52423584,37.59619436)
\curveto(629.69424342,37.52618438)(629.86924325,37.46118444)(630.04923584,37.40119436)
\curveto(630.2192429,37.34118456)(630.38924273,37.27118463)(630.55923584,37.19119436)
\curveto(630.60924251,37.17118473)(630.65424246,37.15618475)(630.69423584,37.14619436)
\curveto(630.73424238,37.13618477)(630.77924234,37.12118478)(630.82923584,37.10119436)
\curveto(631.00924211,37.02118488)(631.19424192,36.95118495)(631.38423584,36.89119436)
\curveto(631.56424155,36.84118506)(631.74424137,36.77618513)(631.92423584,36.69619436)
\curveto(632.07424104,36.62618528)(632.22924089,36.56618534)(632.38923584,36.51619436)
\curveto(632.53924058,36.46618544)(632.68924043,36.41118549)(632.83923584,36.35119436)
\curveto(633.30923981,36.15118575)(633.78423933,35.97118593)(634.26423584,35.81119436)
\curveto(634.73423838,35.65118625)(635.19923792,35.47618643)(635.65923584,35.28619436)
\curveto(635.83923728,35.2061867)(636.0192371,35.13618677)(636.19923584,35.07619436)
\curveto(636.37923674,35.01618689)(636.55923656,34.95118695)(636.73923584,34.88119436)
\curveto(636.84923627,34.83118707)(636.95423616,34.78118712)(637.05423584,34.73119436)
\curveto(637.14423597,34.69118721)(637.20923591,34.6061873)(637.24923584,34.47619436)
\curveto(637.25923586,34.45618745)(637.26423585,34.43118747)(637.26423584,34.40119436)
\curveto(637.25423586,34.38118752)(637.25423586,34.35618755)(637.26423584,34.32619436)
\curveto(637.27423584,34.29618761)(637.27923584,34.26118764)(637.27923584,34.22119436)
\curveto(637.26923585,34.18118772)(637.26423585,34.14118776)(637.26423584,34.10119436)
\lineto(637.26423584,33.80119436)
\curveto(637.26423585,33.7011882)(637.23923588,33.62118828)(637.18923584,33.56119436)
\curveto(637.13923598,33.48118842)(637.06923605,33.42118848)(636.97923584,33.38119436)
\curveto(636.87923624,33.35118855)(636.77923634,33.31118859)(636.67923584,33.26119436)
\curveto(636.47923664,33.18118872)(636.27423684,33.1011888)(636.06423584,33.02119436)
\curveto(635.84423727,32.95118895)(635.63423748,32.87618903)(635.43423584,32.79619436)
\curveto(635.25423786,32.71618919)(635.07423804,32.64618926)(634.89423584,32.58619436)
\curveto(634.70423841,32.53618937)(634.5192386,32.47118943)(634.33923584,32.39119436)
\curveto(633.77923934,32.16118974)(633.2142399,31.94618996)(632.64423584,31.74619436)
\curveto(632.07424104,31.54619036)(631.50924161,31.33119057)(630.94923584,31.10119436)
\lineto(630.31923584,30.86119436)
\curveto(630.09924302,30.79119111)(629.88924323,30.71619119)(629.68923584,30.63619436)
\curveto(629.57924354,30.58619132)(629.47424364,30.54119136)(629.37423584,30.50119436)
\curveto(629.26424385,30.47119143)(629.16924395,30.42119148)(629.08923584,30.35119436)
\curveto(629.06924405,30.34119156)(629.05924406,30.33119157)(629.05923584,30.32119436)
\lineto(629.02923584,30.29119436)
\lineto(629.02923584,30.21619436)
\lineto(629.05923584,30.18619436)
\curveto(629.05924406,30.17619173)(629.06424405,30.16619174)(629.07423584,30.15619436)
\curveto(629.12424399,30.13619177)(629.17924394,30.12619178)(629.23923584,30.12619436)
\curveto(629.29924382,30.12619178)(629.35924376,30.11619179)(629.41923584,30.09619436)
\lineto(629.58423584,30.09619436)
\curveto(629.64424347,30.07619183)(629.70924341,30.07119183)(629.77923584,30.08119436)
\curveto(629.84924327,30.09119181)(629.9192432,30.09619181)(629.98923584,30.09619436)
\lineto(630.79923584,30.09619436)
\lineto(635.35923584,30.09619436)
\lineto(636.54423584,30.09619436)
\curveto(636.65423646,30.09619181)(636.76423635,30.09119181)(636.87423584,30.08119436)
\curveto(636.98423613,30.08119182)(637.06923605,30.05619185)(637.12923584,30.00619436)
\curveto(637.20923591,29.95619195)(637.25423586,29.86619204)(637.26423584,29.73619436)
\lineto(637.26423584,29.34619436)
\lineto(637.26423584,29.15119436)
\curveto(637.26423585,29.1011928)(637.25423586,29.05119285)(637.23423584,29.00119436)
\curveto(637.19423592,28.87119303)(637.10923601,28.79619311)(636.97923584,28.77619436)
\curveto(636.84923627,28.76619314)(636.69923642,28.76119314)(636.52923584,28.76119436)
\lineto(634.78923584,28.76119436)
\lineto(628.78923584,28.76119436)
\lineto(627.37923584,28.76119436)
\curveto(627.26924585,28.76119314)(627.15424596,28.75619315)(627.03423584,28.74619436)
\curveto(626.9142462,28.74619316)(626.8192463,28.77119313)(626.74923584,28.82119436)
\curveto(626.68924643,28.86119304)(626.63924648,28.93619297)(626.59923584,29.04619436)
\curveto(626.58924653,29.06619284)(626.58924653,29.08619282)(626.59923584,29.10619436)
\curveto(626.59924652,29.13619277)(626.59424652,29.16119274)(626.58423584,29.18119436)
}
}
{
\newrgbcolor{curcolor}{0 0 0}
\pscustom[linestyle=none,fillstyle=solid,fillcolor=curcolor]
{
\newpath
\moveto(636.70923584,48.38330373)
\curveto(636.86923625,48.4132959)(637.00423611,48.39829592)(637.11423584,48.33830373)
\curveto(637.2142359,48.27829604)(637.28923583,48.19829612)(637.33923584,48.09830373)
\curveto(637.35923576,48.04829627)(637.36923575,47.99329632)(637.36923584,47.93330373)
\curveto(637.36923575,47.88329643)(637.37923574,47.82829649)(637.39923584,47.76830373)
\curveto(637.44923567,47.54829677)(637.43423568,47.32829699)(637.35423584,47.10830373)
\curveto(637.28423583,46.89829742)(637.19423592,46.75329756)(637.08423584,46.67330373)
\curveto(637.0142361,46.62329769)(636.93423618,46.57829774)(636.84423584,46.53830373)
\curveto(636.74423637,46.49829782)(636.66423645,46.44829787)(636.60423584,46.38830373)
\curveto(636.58423653,46.36829795)(636.56423655,46.34329797)(636.54423584,46.31330373)
\curveto(636.52423659,46.29329802)(636.5192366,46.26329805)(636.52923584,46.22330373)
\curveto(636.55923656,46.1132982)(636.6142365,46.00829831)(636.69423584,45.90830373)
\curveto(636.77423634,45.8182985)(636.84423627,45.72829859)(636.90423584,45.63830373)
\curveto(636.98423613,45.50829881)(637.05923606,45.36829895)(637.12923584,45.21830373)
\curveto(637.18923593,45.06829925)(637.24423587,44.90829941)(637.29423584,44.73830373)
\curveto(637.32423579,44.63829968)(637.34423577,44.52829979)(637.35423584,44.40830373)
\curveto(637.36423575,44.29830002)(637.37923574,44.18830013)(637.39923584,44.07830373)
\curveto(637.40923571,44.02830029)(637.4142357,43.98330033)(637.41423584,43.94330373)
\lineto(637.41423584,43.83830373)
\curveto(637.43423568,43.72830059)(637.43423568,43.62330069)(637.41423584,43.52330373)
\lineto(637.41423584,43.38830373)
\curveto(637.40423571,43.33830098)(637.39923572,43.28830103)(637.39923584,43.23830373)
\curveto(637.39923572,43.18830113)(637.38923573,43.14330117)(637.36923584,43.10330373)
\curveto(637.35923576,43.06330125)(637.35423576,43.02830129)(637.35423584,42.99830373)
\curveto(637.36423575,42.97830134)(637.36423575,42.95330136)(637.35423584,42.92330373)
\lineto(637.29423584,42.68330373)
\curveto(637.28423583,42.60330171)(637.26423585,42.52830179)(637.23423584,42.45830373)
\curveto(637.10423601,42.15830216)(636.95923616,41.9133024)(636.79923584,41.72330373)
\curveto(636.62923649,41.54330277)(636.39423672,41.39330292)(636.09423584,41.27330373)
\curveto(635.87423724,41.18330313)(635.60923751,41.13830318)(635.29923584,41.13830373)
\lineto(634.98423584,41.13830373)
\curveto(634.93423818,41.14830317)(634.88423823,41.15330316)(634.83423584,41.15330373)
\lineto(634.65423584,41.18330373)
\lineto(634.32423584,41.30330373)
\curveto(634.2142389,41.34330297)(634.114239,41.39330292)(634.02423584,41.45330373)
\curveto(633.73423938,41.63330268)(633.5192396,41.87830244)(633.37923584,42.18830373)
\curveto(633.23923988,42.49830182)(633.11424,42.83830148)(633.00423584,43.20830373)
\curveto(632.96424015,43.34830097)(632.93424018,43.49330082)(632.91423584,43.64330373)
\curveto(632.89424022,43.79330052)(632.86924025,43.94330037)(632.83923584,44.09330373)
\curveto(632.8192403,44.16330015)(632.80924031,44.22830009)(632.80923584,44.28830373)
\curveto(632.80924031,44.35829996)(632.79924032,44.43329988)(632.77923584,44.51330373)
\curveto(632.75924036,44.58329973)(632.74924037,44.65329966)(632.74923584,44.72330373)
\curveto(632.73924038,44.79329952)(632.72424039,44.86829945)(632.70423584,44.94830373)
\curveto(632.64424047,45.19829912)(632.59424052,45.43329888)(632.55423584,45.65330373)
\curveto(632.50424061,45.87329844)(632.38924073,46.04829827)(632.20923584,46.17830373)
\curveto(632.12924099,46.23829808)(632.02924109,46.28829803)(631.90923584,46.32830373)
\curveto(631.77924134,46.36829795)(631.63924148,46.36829795)(631.48923584,46.32830373)
\curveto(631.24924187,46.26829805)(631.05924206,46.17829814)(630.91923584,46.05830373)
\curveto(630.77924234,45.94829837)(630.66924245,45.78829853)(630.58923584,45.57830373)
\curveto(630.53924258,45.45829886)(630.50424261,45.313299)(630.48423584,45.14330373)
\curveto(630.46424265,44.98329933)(630.45424266,44.8132995)(630.45423584,44.63330373)
\curveto(630.45424266,44.45329986)(630.46424265,44.27830004)(630.48423584,44.10830373)
\curveto(630.50424261,43.93830038)(630.53424258,43.79330052)(630.57423584,43.67330373)
\curveto(630.63424248,43.50330081)(630.7192424,43.33830098)(630.82923584,43.17830373)
\curveto(630.88924223,43.09830122)(630.96924215,43.02330129)(631.06923584,42.95330373)
\curveto(631.15924196,42.89330142)(631.25924186,42.83830148)(631.36923584,42.78830373)
\curveto(631.44924167,42.75830156)(631.53424158,42.72830159)(631.62423584,42.69830373)
\curveto(631.7142414,42.67830164)(631.78424133,42.63330168)(631.83423584,42.56330373)
\curveto(631.86424125,42.52330179)(631.88924123,42.45330186)(631.90923584,42.35330373)
\curveto(631.9192412,42.26330205)(631.92424119,42.16830215)(631.92423584,42.06830373)
\curveto(631.92424119,41.96830235)(631.9192412,41.86830245)(631.90923584,41.76830373)
\curveto(631.88924123,41.67830264)(631.86424125,41.6133027)(631.83423584,41.57330373)
\curveto(631.80424131,41.53330278)(631.75424136,41.50330281)(631.68423584,41.48330373)
\curveto(631.6142415,41.46330285)(631.53924158,41.46330285)(631.45923584,41.48330373)
\curveto(631.32924179,41.5133028)(631.20924191,41.54330277)(631.09923584,41.57330373)
\curveto(630.97924214,41.6133027)(630.86424225,41.65830266)(630.75423584,41.70830373)
\curveto(630.40424271,41.89830242)(630.13424298,42.13830218)(629.94423584,42.42830373)
\curveto(629.74424337,42.7183016)(629.58424353,43.07830124)(629.46423584,43.50830373)
\curveto(629.44424367,43.60830071)(629.42924369,43.70830061)(629.41923584,43.80830373)
\curveto(629.40924371,43.9183004)(629.39424372,44.02830029)(629.37423584,44.13830373)
\curveto(629.36424375,44.17830014)(629.36424375,44.24330007)(629.37423584,44.33330373)
\curveto(629.37424374,44.42329989)(629.36424375,44.47829984)(629.34423584,44.49830373)
\curveto(629.33424378,45.19829912)(629.4142437,45.80829851)(629.58423584,46.32830373)
\curveto(629.75424336,46.84829747)(630.07924304,47.2132971)(630.55923584,47.42330373)
\curveto(630.75924236,47.5132968)(630.99424212,47.56329675)(631.26423584,47.57330373)
\curveto(631.52424159,47.59329672)(631.79924132,47.60329671)(632.08923584,47.60330373)
\lineto(635.40423584,47.60330373)
\curveto(635.54423757,47.60329671)(635.67923744,47.60829671)(635.80923584,47.61830373)
\curveto(635.93923718,47.62829669)(636.04423707,47.65829666)(636.12423584,47.70830373)
\curveto(636.19423692,47.75829656)(636.24423687,47.82329649)(636.27423584,47.90330373)
\curveto(636.3142368,47.99329632)(636.34423677,48.07829624)(636.36423584,48.15830373)
\curveto(636.37423674,48.23829608)(636.4192367,48.29829602)(636.49923584,48.33830373)
\curveto(636.52923659,48.35829596)(636.55923656,48.36829595)(636.58923584,48.36830373)
\curveto(636.6192365,48.36829595)(636.65923646,48.37329594)(636.70923584,48.38330373)
\moveto(635.04423584,46.23830373)
\curveto(634.90423821,46.29829802)(634.74423837,46.32829799)(634.56423584,46.32830373)
\curveto(634.37423874,46.33829798)(634.17923894,46.34329797)(633.97923584,46.34330373)
\curveto(633.86923925,46.34329797)(633.76923935,46.33829798)(633.67923584,46.32830373)
\curveto(633.58923953,46.318298)(633.5192396,46.27829804)(633.46923584,46.20830373)
\curveto(633.44923967,46.17829814)(633.43923968,46.10829821)(633.43923584,45.99830373)
\curveto(633.45923966,45.97829834)(633.46923965,45.94329837)(633.46923584,45.89330373)
\curveto(633.46923965,45.84329847)(633.47923964,45.79829852)(633.49923584,45.75830373)
\curveto(633.5192396,45.67829864)(633.53923958,45.58829873)(633.55923584,45.48830373)
\lineto(633.61923584,45.18830373)
\curveto(633.6192395,45.15829916)(633.62423949,45.12329919)(633.63423584,45.08330373)
\lineto(633.63423584,44.97830373)
\curveto(633.67423944,44.82829949)(633.69923942,44.66329965)(633.70923584,44.48330373)
\curveto(633.70923941,44.3133)(633.72923939,44.15330016)(633.76923584,44.00330373)
\curveto(633.78923933,43.92330039)(633.80923931,43.84830047)(633.82923584,43.77830373)
\curveto(633.83923928,43.7183006)(633.85423926,43.64830067)(633.87423584,43.56830373)
\curveto(633.92423919,43.40830091)(633.98923913,43.25830106)(634.06923584,43.11830373)
\curveto(634.13923898,42.97830134)(634.22923889,42.85830146)(634.33923584,42.75830373)
\curveto(634.44923867,42.65830166)(634.58423853,42.58330173)(634.74423584,42.53330373)
\curveto(634.89423822,42.48330183)(635.07923804,42.46330185)(635.29923584,42.47330373)
\curveto(635.39923772,42.47330184)(635.49423762,42.48830183)(635.58423584,42.51830373)
\curveto(635.66423745,42.55830176)(635.73923738,42.60330171)(635.80923584,42.65330373)
\curveto(635.9192372,42.73330158)(636.0142371,42.83830148)(636.09423584,42.96830373)
\curveto(636.16423695,43.09830122)(636.22423689,43.23830108)(636.27423584,43.38830373)
\curveto(636.28423683,43.43830088)(636.28923683,43.48830083)(636.28923584,43.53830373)
\curveto(636.28923683,43.58830073)(636.29423682,43.63830068)(636.30423584,43.68830373)
\curveto(636.32423679,43.75830056)(636.33923678,43.84330047)(636.34923584,43.94330373)
\curveto(636.34923677,44.05330026)(636.33923678,44.14330017)(636.31923584,44.21330373)
\curveto(636.29923682,44.27330004)(636.29423682,44.33329998)(636.30423584,44.39330373)
\curveto(636.30423681,44.45329986)(636.29423682,44.5132998)(636.27423584,44.57330373)
\curveto(636.25423686,44.65329966)(636.23923688,44.72829959)(636.22923584,44.79830373)
\curveto(636.2192369,44.87829944)(636.19923692,44.95329936)(636.16923584,45.02330373)
\curveto(636.04923707,45.313299)(635.90423721,45.55829876)(635.73423584,45.75830373)
\curveto(635.56423755,45.96829835)(635.33423778,46.12829819)(635.04423584,46.23830373)
}
}
{
\newrgbcolor{curcolor}{0 0 0}
\pscustom[linestyle=none,fillstyle=solid,fillcolor=curcolor]
{
\newpath
\moveto(629.53923584,49.26994436)
\lineto(629.53923584,49.71994436)
\curveto(629.52924359,49.88994311)(629.54924357,50.01494298)(629.59923584,50.09494436)
\curveto(629.64924347,50.17494282)(629.7142434,50.22994277)(629.79423584,50.25994436)
\curveto(629.87424324,50.2999427)(629.95924316,50.33994266)(630.04923584,50.37994436)
\curveto(630.17924294,50.42994257)(630.30924281,50.47494252)(630.43923584,50.51494436)
\curveto(630.56924255,50.55494244)(630.69924242,50.5999424)(630.82923584,50.64994436)
\curveto(630.94924217,50.6999423)(631.07424204,50.74494225)(631.20423584,50.78494436)
\curveto(631.32424179,50.82494217)(631.44424167,50.86994213)(631.56423584,50.91994436)
\curveto(631.67424144,50.96994203)(631.78924133,51.00994199)(631.90923584,51.03994436)
\curveto(632.02924109,51.06994193)(632.14924097,51.10994189)(632.26923584,51.15994436)
\curveto(632.55924056,51.27994172)(632.85924026,51.38994161)(633.16923584,51.48994436)
\curveto(633.47923964,51.58994141)(633.77923934,51.6999413)(634.06923584,51.81994436)
\curveto(634.10923901,51.83994116)(634.14923897,51.84994115)(634.18923584,51.84994436)
\curveto(634.2192389,51.84994115)(634.24923887,51.85994114)(634.27923584,51.87994436)
\curveto(634.4192387,51.93994106)(634.56423855,51.994941)(634.71423584,52.04494436)
\lineto(635.13423584,52.19494436)
\curveto(635.20423791,52.22494077)(635.27923784,52.25494074)(635.35923584,52.28494436)
\curveto(635.42923769,52.31494068)(635.47423764,52.36494063)(635.49423584,52.43494436)
\curveto(635.52423759,52.51494048)(635.49923762,52.57494042)(635.41923584,52.61494436)
\curveto(635.32923779,52.66494033)(635.25923786,52.6999403)(635.20923584,52.71994436)
\curveto(635.03923808,52.7999402)(634.85923826,52.86494013)(634.66923584,52.91494436)
\curveto(634.47923864,52.96494003)(634.29423882,53.02493997)(634.11423584,53.09494436)
\curveto(633.88423923,53.18493981)(633.65423946,53.26493973)(633.42423584,53.33494436)
\curveto(633.18423993,53.40493959)(632.95424016,53.48993951)(632.73423584,53.58994436)
\curveto(632.68424043,53.5999394)(632.6192405,53.61493938)(632.53923584,53.63494436)
\curveto(632.44924067,53.67493932)(632.35924076,53.70993929)(632.26923584,53.73994436)
\curveto(632.16924095,53.76993923)(632.07924104,53.7999392)(631.99923584,53.82994436)
\curveto(631.94924117,53.84993915)(631.90424121,53.86493913)(631.86423584,53.87494436)
\curveto(631.82424129,53.88493911)(631.77924134,53.8999391)(631.72923584,53.91994436)
\curveto(631.60924151,53.96993903)(631.48924163,54.00993899)(631.36923584,54.03994436)
\curveto(631.23924188,54.07993892)(631.114242,54.12493887)(630.99423584,54.17494436)
\curveto(630.94424217,54.1949388)(630.89924222,54.20993879)(630.85923584,54.21994436)
\curveto(630.8192423,54.22993877)(630.77424234,54.24493875)(630.72423584,54.26494436)
\curveto(630.63424248,54.30493869)(630.54424257,54.33993866)(630.45423584,54.36994436)
\curveto(630.35424276,54.3999386)(630.25924286,54.42993857)(630.16923584,54.45994436)
\curveto(630.08924303,54.48993851)(630.00924311,54.51493848)(629.92923584,54.53494436)
\curveto(629.83924328,54.56493843)(629.76424335,54.60493839)(629.70423584,54.65494436)
\curveto(629.6142435,54.72493827)(629.56424355,54.81993818)(629.55423584,54.93994436)
\curveto(629.54424357,55.06993793)(629.53924358,55.20993779)(629.53923584,55.35994436)
\curveto(629.53924358,55.43993756)(629.54424357,55.51493748)(629.55423584,55.58494436)
\curveto(629.55424356,55.66493733)(629.56924355,55.72993727)(629.59923584,55.77994436)
\curveto(629.65924346,55.86993713)(629.75424336,55.8949371)(629.88423584,55.85494436)
\curveto(630.0142431,55.81493718)(630.114243,55.77993722)(630.18423584,55.74994436)
\lineto(630.24423584,55.71994436)
\curveto(630.26424285,55.71993728)(630.28424283,55.71493728)(630.30423584,55.70494436)
\curveto(630.58424253,55.5949374)(630.86924225,55.48493751)(631.15923584,55.37494436)
\lineto(631.99923584,55.04494436)
\curveto(632.07924104,55.01493798)(632.15424096,54.98993801)(632.22423584,54.96994436)
\curveto(632.28424083,54.94993805)(632.34924077,54.92493807)(632.41923584,54.89494436)
\curveto(632.6192405,54.81493818)(632.82424029,54.73493826)(633.03423584,54.65494436)
\curveto(633.23423988,54.58493841)(633.43423968,54.50993849)(633.63423584,54.42994436)
\curveto(634.32423879,54.13993886)(635.0192381,53.86993913)(635.71923584,53.61994436)
\curveto(636.4192367,53.36993963)(637.114236,53.0999399)(637.80423584,52.80994436)
\lineto(637.95423584,52.74994436)
\curveto(638.0142351,52.73994026)(638.07423504,52.72494027)(638.13423584,52.70494436)
\curveto(638.50423461,52.54494045)(638.86923425,52.37494062)(639.22923584,52.19494436)
\curveto(639.59923352,52.01494098)(639.88423323,51.76494123)(640.08423584,51.44494436)
\curveto(640.14423297,51.33494166)(640.18923293,51.22494177)(640.21923584,51.11494436)
\curveto(640.25923286,51.00494199)(640.29423282,50.87994212)(640.32423584,50.73994436)
\curveto(640.34423277,50.68994231)(640.34923277,50.63494236)(640.33923584,50.57494436)
\curveto(640.32923279,50.52494247)(640.32923279,50.46994253)(640.33923584,50.40994436)
\curveto(640.35923276,50.32994267)(640.35923276,50.24994275)(640.33923584,50.16994436)
\curveto(640.32923279,50.12994287)(640.32423279,50.07994292)(640.32423584,50.01994436)
\lineto(640.26423584,49.77994436)
\curveto(640.24423287,49.70994329)(640.20423291,49.65494334)(640.14423584,49.61494436)
\curveto(640.08423303,49.56494343)(640.00923311,49.53494346)(639.91923584,49.52494436)
\lineto(639.64923584,49.52494436)
\lineto(639.43923584,49.52494436)
\curveto(639.37923374,49.53494346)(639.32923379,49.55494344)(639.28923584,49.58494436)
\curveto(639.17923394,49.65494334)(639.14923397,49.77494322)(639.19923584,49.94494436)
\curveto(639.2192339,50.05494294)(639.22923389,50.17494282)(639.22923584,50.30494436)
\curveto(639.22923389,50.43494256)(639.20923391,50.54994245)(639.16923584,50.64994436)
\curveto(639.119234,50.7999422)(639.04423407,50.91994208)(638.94423584,51.00994436)
\curveto(638.84423427,51.10994189)(638.72923439,51.1949418)(638.59923584,51.26494436)
\curveto(638.47923464,51.33494166)(638.34923477,51.3949416)(638.20923584,51.44494436)
\lineto(637.78923584,51.62494436)
\curveto(637.69923542,51.66494133)(637.58923553,51.70494129)(637.45923584,51.74494436)
\curveto(637.32923579,51.7949412)(637.19423592,51.7999412)(637.05423584,51.75994436)
\curveto(636.89423622,51.70994129)(636.74423637,51.65494134)(636.60423584,51.59494436)
\curveto(636.46423665,51.54494145)(636.32423679,51.48994151)(636.18423584,51.42994436)
\curveto(635.97423714,51.33994166)(635.76423735,51.25494174)(635.55423584,51.17494436)
\curveto(635.34423777,51.0949419)(635.13923798,51.01494198)(634.93923584,50.93494436)
\curveto(634.79923832,50.87494212)(634.66423845,50.81994218)(634.53423584,50.76994436)
\curveto(634.40423871,50.71994228)(634.26923885,50.66994233)(634.12923584,50.61994436)
\lineto(632.80923584,50.07994436)
\curveto(632.36924075,49.90994309)(631.92924119,49.73494326)(631.48923584,49.55494436)
\curveto(631.25924186,49.45494354)(631.03924208,49.36494363)(630.82923584,49.28494436)
\curveto(630.60924251,49.20494379)(630.38924273,49.11994388)(630.16923584,49.02994436)
\curveto(630.10924301,49.00994399)(630.02924309,48.97994402)(629.92923584,48.93994436)
\curveto(629.8192433,48.8999441)(629.72924339,48.90494409)(629.65923584,48.95494436)
\curveto(629.60924351,48.98494401)(629.57424354,49.04494395)(629.55423584,49.13494436)
\curveto(629.54424357,49.15494384)(629.54424357,49.17494382)(629.55423584,49.19494436)
\curveto(629.55424356,49.22494377)(629.54924357,49.24994375)(629.53923584,49.26994436)
}
}
{
\newrgbcolor{curcolor}{0 0 0}
\pscustom[linestyle=none,fillstyle=solid,fillcolor=curcolor]
{
}
}
{
\newrgbcolor{curcolor}{0 0 0}
\pscustom[linestyle=none,fillstyle=solid,fillcolor=curcolor]
{
\newpath
\moveto(626.65923584,64.24510061)
\curveto(626.64924647,64.93509597)(626.76924635,65.53509537)(627.01923584,66.04510061)
\curveto(627.26924585,66.56509434)(627.60424551,66.96009395)(628.02423584,67.23010061)
\curveto(628.10424501,67.28009363)(628.19424492,67.32509358)(628.29423584,67.36510061)
\curveto(628.38424473,67.4050935)(628.47924464,67.45009346)(628.57923584,67.50010061)
\curveto(628.67924444,67.54009337)(628.77924434,67.57009334)(628.87923584,67.59010061)
\curveto(628.97924414,67.6100933)(629.08424403,67.63009328)(629.19423584,67.65010061)
\curveto(629.24424387,67.67009324)(629.28924383,67.67509323)(629.32923584,67.66510061)
\curveto(629.36924375,67.65509325)(629.4142437,67.66009325)(629.46423584,67.68010061)
\curveto(629.5142436,67.69009322)(629.59924352,67.69509321)(629.71923584,67.69510061)
\curveto(629.82924329,67.69509321)(629.9142432,67.69009322)(629.97423584,67.68010061)
\curveto(630.03424308,67.66009325)(630.09424302,67.65009326)(630.15423584,67.65010061)
\curveto(630.2142429,67.66009325)(630.27424284,67.65509325)(630.33423584,67.63510061)
\curveto(630.47424264,67.59509331)(630.60924251,67.56009335)(630.73923584,67.53010061)
\curveto(630.86924225,67.50009341)(630.99424212,67.46009345)(631.11423584,67.41010061)
\curveto(631.25424186,67.35009356)(631.37924174,67.28009363)(631.48923584,67.20010061)
\curveto(631.59924152,67.13009378)(631.70924141,67.05509385)(631.81923584,66.97510061)
\lineto(631.87923584,66.91510061)
\curveto(631.89924122,66.905094)(631.9192412,66.89009402)(631.93923584,66.87010061)
\curveto(632.09924102,66.75009416)(632.24424087,66.61509429)(632.37423584,66.46510061)
\curveto(632.50424061,66.31509459)(632.62924049,66.15509475)(632.74923584,65.98510061)
\curveto(632.96924015,65.67509523)(633.17423994,65.38009553)(633.36423584,65.10010061)
\curveto(633.50423961,64.87009604)(633.63923948,64.64009627)(633.76923584,64.41010061)
\curveto(633.89923922,64.19009672)(634.03423908,63.97009694)(634.17423584,63.75010061)
\curveto(634.34423877,63.50009741)(634.52423859,63.26009765)(634.71423584,63.03010061)
\curveto(634.90423821,62.8100981)(635.12923799,62.62009829)(635.38923584,62.46010061)
\curveto(635.44923767,62.42009849)(635.50923761,62.38509852)(635.56923584,62.35510061)
\curveto(635.6192375,62.32509858)(635.68423743,62.29509861)(635.76423584,62.26510061)
\curveto(635.83423728,62.24509866)(635.89423722,62.24009867)(635.94423584,62.25010061)
\curveto(636.0142371,62.27009864)(636.06923705,62.3050986)(636.10923584,62.35510061)
\curveto(636.13923698,62.4050985)(636.15923696,62.46509844)(636.16923584,62.53510061)
\lineto(636.16923584,62.77510061)
\lineto(636.16923584,63.52510061)
\lineto(636.16923584,66.33010061)
\lineto(636.16923584,66.99010061)
\curveto(636.16923695,67.08009383)(636.17423694,67.16509374)(636.18423584,67.24510061)
\curveto(636.18423693,67.32509358)(636.20423691,67.39009352)(636.24423584,67.44010061)
\curveto(636.28423683,67.49009342)(636.35923676,67.53009338)(636.46923584,67.56010061)
\curveto(636.56923655,67.60009331)(636.66923645,67.6100933)(636.76923584,67.59010061)
\lineto(636.90423584,67.59010061)
\curveto(636.97423614,67.57009334)(637.03423608,67.55009336)(637.08423584,67.53010061)
\curveto(637.13423598,67.5100934)(637.17423594,67.47509343)(637.20423584,67.42510061)
\curveto(637.24423587,67.37509353)(637.26423585,67.3050936)(637.26423584,67.21510061)
\lineto(637.26423584,66.94510061)
\lineto(637.26423584,66.04510061)
\lineto(637.26423584,62.53510061)
\lineto(637.26423584,61.47010061)
\curveto(637.26423585,61.39009952)(637.26923585,61.30009961)(637.27923584,61.20010061)
\curveto(637.27923584,61.10009981)(637.26923585,61.01509989)(637.24923584,60.94510061)
\curveto(637.17923594,60.73510017)(636.99923612,60.67010024)(636.70923584,60.75010061)
\curveto(636.66923645,60.76010015)(636.63423648,60.76010015)(636.60423584,60.75010061)
\curveto(636.56423655,60.75010016)(636.5192366,60.76010015)(636.46923584,60.78010061)
\curveto(636.38923673,60.80010011)(636.30423681,60.82010009)(636.21423584,60.84010061)
\curveto(636.12423699,60.86010005)(636.03923708,60.88510002)(635.95923584,60.91510061)
\curveto(635.46923765,61.07509983)(635.05423806,61.27509963)(634.71423584,61.51510061)
\curveto(634.46423865,61.69509921)(634.23923888,61.90009901)(634.03923584,62.13010061)
\curveto(633.82923929,62.36009855)(633.63423948,62.60009831)(633.45423584,62.85010061)
\curveto(633.27423984,63.1100978)(633.10424001,63.37509753)(632.94423584,63.64510061)
\curveto(632.77424034,63.92509698)(632.59924052,64.19509671)(632.41923584,64.45510061)
\curveto(632.33924078,64.56509634)(632.26424085,64.67009624)(632.19423584,64.77010061)
\curveto(632.12424099,64.88009603)(632.04924107,64.99009592)(631.96923584,65.10010061)
\curveto(631.93924118,65.14009577)(631.90924121,65.17509573)(631.87923584,65.20510061)
\curveto(631.83924128,65.24509566)(631.80924131,65.28509562)(631.78923584,65.32510061)
\curveto(631.67924144,65.46509544)(631.55424156,65.59009532)(631.41423584,65.70010061)
\curveto(631.38424173,65.72009519)(631.35924176,65.74509516)(631.33923584,65.77510061)
\curveto(631.30924181,65.8050951)(631.27924184,65.83009508)(631.24923584,65.85010061)
\curveto(631.14924197,65.93009498)(631.04924207,65.99509491)(630.94923584,66.04510061)
\curveto(630.84924227,66.1050948)(630.73924238,66.16009475)(630.61923584,66.21010061)
\curveto(630.54924257,66.24009467)(630.47424264,66.26009465)(630.39423584,66.27010061)
\lineto(630.15423584,66.33010061)
\lineto(630.06423584,66.33010061)
\curveto(630.03424308,66.34009457)(630.00424311,66.34509456)(629.97423584,66.34510061)
\curveto(629.90424321,66.36509454)(629.80924331,66.37009454)(629.68923584,66.36010061)
\curveto(629.55924356,66.36009455)(629.45924366,66.35009456)(629.38923584,66.33010061)
\curveto(629.30924381,66.3100946)(629.23424388,66.29009462)(629.16423584,66.27010061)
\curveto(629.08424403,66.26009465)(629.00424411,66.24009467)(628.92423584,66.21010061)
\curveto(628.68424443,66.10009481)(628.48424463,65.95009496)(628.32423584,65.76010061)
\curveto(628.15424496,65.58009533)(628.0142451,65.36009555)(627.90423584,65.10010061)
\curveto(627.88424523,65.03009588)(627.86924525,64.96009595)(627.85923584,64.89010061)
\curveto(627.83924528,64.82009609)(627.8192453,64.74509616)(627.79923584,64.66510061)
\curveto(627.77924534,64.58509632)(627.76924535,64.47509643)(627.76923584,64.33510061)
\curveto(627.76924535,64.2050967)(627.77924534,64.10009681)(627.79923584,64.02010061)
\curveto(627.80924531,63.96009695)(627.8142453,63.905097)(627.81423584,63.85510061)
\curveto(627.8142453,63.8050971)(627.82424529,63.75509715)(627.84423584,63.70510061)
\curveto(627.88424523,63.6050973)(627.92424519,63.5100974)(627.96423584,63.42010061)
\curveto(628.00424511,63.34009757)(628.04924507,63.26009765)(628.09923584,63.18010061)
\curveto(628.119245,63.15009776)(628.14424497,63.12009779)(628.17423584,63.09010061)
\curveto(628.20424491,63.07009784)(628.22924489,63.04509786)(628.24923584,63.01510061)
\lineto(628.32423584,62.94010061)
\curveto(628.34424477,62.910098)(628.36424475,62.88509802)(628.38423584,62.86510061)
\lineto(628.59423584,62.71510061)
\curveto(628.65424446,62.67509823)(628.7192444,62.63009828)(628.78923584,62.58010061)
\curveto(628.87924424,62.52009839)(628.98424413,62.47009844)(629.10423584,62.43010061)
\curveto(629.2142439,62.40009851)(629.32424379,62.36509854)(629.43423584,62.32510061)
\curveto(629.54424357,62.28509862)(629.68924343,62.26009865)(629.86923584,62.25010061)
\curveto(630.03924308,62.24009867)(630.16424295,62.2100987)(630.24423584,62.16010061)
\curveto(630.32424279,62.1100988)(630.36924275,62.03509887)(630.37923584,61.93510061)
\curveto(630.38924273,61.83509907)(630.39424272,61.72509918)(630.39423584,61.60510061)
\curveto(630.39424272,61.56509934)(630.39924272,61.52509938)(630.40923584,61.48510061)
\curveto(630.40924271,61.44509946)(630.40424271,61.4100995)(630.39423584,61.38010061)
\curveto(630.37424274,61.33009958)(630.36424275,61.28009963)(630.36423584,61.23010061)
\curveto(630.36424275,61.19009972)(630.35424276,61.15009976)(630.33423584,61.11010061)
\curveto(630.27424284,61.02009989)(630.13924298,60.97509993)(629.92923584,60.97510061)
\lineto(629.80923584,60.97510061)
\curveto(629.74924337,60.98509992)(629.68924343,60.99009992)(629.62923584,60.99010061)
\curveto(629.55924356,61.00009991)(629.49424362,61.0100999)(629.43423584,61.02010061)
\curveto(629.32424379,61.04009987)(629.22424389,61.06009985)(629.13423584,61.08010061)
\curveto(629.03424408,61.10009981)(628.93924418,61.13009978)(628.84923584,61.17010061)
\curveto(628.77924434,61.19009972)(628.7192444,61.2100997)(628.66923584,61.23010061)
\lineto(628.48923584,61.29010061)
\curveto(628.22924489,61.4100995)(627.98424513,61.56509934)(627.75423584,61.75510061)
\curveto(627.52424559,61.95509895)(627.33924578,62.17009874)(627.19923584,62.40010061)
\curveto(627.119246,62.5100984)(627.05424606,62.62509828)(627.00423584,62.74510061)
\lineto(626.85423584,63.13510061)
\curveto(626.80424631,63.24509766)(626.77424634,63.36009755)(626.76423584,63.48010061)
\curveto(626.74424637,63.60009731)(626.7192464,63.72509718)(626.68923584,63.85510061)
\curveto(626.68924643,63.92509698)(626.68924643,63.99009692)(626.68923584,64.05010061)
\curveto(626.67924644,64.1100968)(626.66924645,64.17509673)(626.65923584,64.24510061)
}
}
{
\newrgbcolor{curcolor}{0 0 0}
\pscustom[linestyle=none,fillstyle=solid,fillcolor=curcolor]
{
\newpath
\moveto(631.66923584,76.28470998)
\curveto(631.74924137,76.28470235)(631.82924129,76.28970234)(631.90923584,76.29970998)
\curveto(631.98924113,76.30970232)(632.06424105,76.30470233)(632.13423584,76.28470998)
\curveto(632.17424094,76.26470237)(632.2192409,76.25970237)(632.26923584,76.26970998)
\curveto(632.30924081,76.27970235)(632.34924077,76.27970235)(632.38923584,76.26970998)
\lineto(632.53923584,76.26970998)
\curveto(632.62924049,76.25970237)(632.7192404,76.25470238)(632.80923584,76.25470998)
\curveto(632.88924023,76.25470238)(632.96924015,76.24970238)(633.04923584,76.23970998)
\lineto(633.28923584,76.20970998)
\curveto(633.35923976,76.19970243)(633.43423968,76.18970244)(633.51423584,76.17970998)
\curveto(633.55423956,76.16970246)(633.59423952,76.16470247)(633.63423584,76.16470998)
\curveto(633.67423944,76.16470247)(633.7192394,76.15970247)(633.76923584,76.14970998)
\curveto(633.90923921,76.10970252)(634.04923907,76.07970255)(634.18923584,76.05970998)
\curveto(634.32923879,76.04970258)(634.46423865,76.01970261)(634.59423584,75.96970998)
\curveto(634.76423835,75.91970271)(634.92923819,75.86470277)(635.08923584,75.80470998)
\curveto(635.24923787,75.75470288)(635.40423771,75.69470294)(635.55423584,75.62470998)
\curveto(635.6142375,75.60470303)(635.67423744,75.57470306)(635.73423584,75.53470998)
\lineto(635.88423584,75.44470998)
\curveto(636.20423691,75.24470339)(636.46923665,75.0297036)(636.67923584,74.79970998)
\curveto(636.88923623,74.56970406)(637.06923605,74.27470436)(637.21923584,73.91470998)
\curveto(637.26923585,73.79470484)(637.30423581,73.66470497)(637.32423584,73.52470998)
\curveto(637.34423577,73.39470524)(637.36923575,73.25970537)(637.39923584,73.11970998)
\curveto(637.40923571,73.05970557)(637.4142357,72.99970563)(637.41423584,72.93970998)
\curveto(637.4142357,72.87970575)(637.4192357,72.81470582)(637.42923584,72.74470998)
\curveto(637.43923568,72.71470592)(637.43923568,72.66470597)(637.42923584,72.59470998)
\lineto(637.42923584,72.44470998)
\lineto(637.42923584,72.29470998)
\curveto(637.40923571,72.21470642)(637.39423572,72.1297065)(637.38423584,72.03970998)
\curveto(637.38423573,71.95970667)(637.37423574,71.88470675)(637.35423584,71.81470998)
\curveto(637.34423577,71.77470686)(637.33923578,71.73970689)(637.33923584,71.70970998)
\curveto(637.34923577,71.68970694)(637.34423577,71.66470697)(637.32423584,71.63470998)
\lineto(637.26423584,71.36470998)
\curveto(637.23423588,71.27470736)(637.20423591,71.18970744)(637.17423584,71.10970998)
\curveto(636.93423618,70.5297081)(636.56423655,70.09470854)(636.06423584,69.80470998)
\curveto(635.93423718,69.72470891)(635.79923732,69.65970897)(635.65923584,69.60970998)
\curveto(635.5192376,69.56970906)(635.36923775,69.52470911)(635.20923584,69.47470998)
\curveto(635.12923799,69.45470918)(635.04923807,69.44970918)(634.96923584,69.45970998)
\curveto(634.88923823,69.47970915)(634.83423828,69.51470912)(634.80423584,69.56470998)
\curveto(634.78423833,69.59470904)(634.76923835,69.64970898)(634.75923584,69.72970998)
\curveto(634.73923838,69.80970882)(634.72923839,69.89470874)(634.72923584,69.98470998)
\curveto(634.7192384,70.07470856)(634.7192384,70.15970847)(634.72923584,70.23970998)
\curveto(634.73923838,70.3297083)(634.74923837,70.39970823)(634.75923584,70.44970998)
\curveto(634.76923835,70.46970816)(634.78423833,70.49470814)(634.80423584,70.52470998)
\curveto(634.82423829,70.56470807)(634.84423827,70.59470804)(634.86423584,70.61470998)
\curveto(634.94423817,70.67470796)(635.03923808,70.71970791)(635.14923584,70.74970998)
\curveto(635.25923786,70.78970784)(635.35923776,70.8347078)(635.44923584,70.88470998)
\curveto(635.83923728,71.1347075)(636.10923701,71.50470713)(636.25923584,71.99470998)
\curveto(636.27923684,72.06470657)(636.29423682,72.1347065)(636.30423584,72.20470998)
\curveto(636.30423681,72.28470635)(636.3142368,72.36470627)(636.33423584,72.44470998)
\curveto(636.34423677,72.48470615)(636.34923677,72.53970609)(636.34923584,72.60970998)
\curveto(636.34923677,72.68970594)(636.34423677,72.74470589)(636.33423584,72.77470998)
\curveto(636.32423679,72.80470583)(636.3192368,72.8347058)(636.31923584,72.86470998)
\lineto(636.31923584,72.96970998)
\curveto(636.29923682,73.04970558)(636.27923684,73.12470551)(636.25923584,73.19470998)
\curveto(636.23923688,73.27470536)(636.2142369,73.34970528)(636.18423584,73.41970998)
\curveto(636.03423708,73.76970486)(635.8192373,74.03970459)(635.53923584,74.22970998)
\curveto(635.25923786,74.41970421)(634.93423818,74.57470406)(634.56423584,74.69470998)
\curveto(634.48423863,74.72470391)(634.40923871,74.74470389)(634.33923584,74.75470998)
\curveto(634.26923885,74.77470386)(634.19423892,74.79470384)(634.11423584,74.81470998)
\curveto(634.02423909,74.8347038)(633.92923919,74.84970378)(633.82923584,74.85970998)
\curveto(633.7192394,74.87970375)(633.6142395,74.89970373)(633.51423584,74.91970998)
\curveto(633.46423965,74.9297037)(633.4142397,74.9347037)(633.36423584,74.93470998)
\curveto(633.30423981,74.94470369)(633.24923987,74.94970368)(633.19923584,74.94970998)
\curveto(633.13923998,74.96970366)(633.06424005,74.97970365)(632.97423584,74.97970998)
\curveto(632.87424024,74.97970365)(632.79424032,74.96970366)(632.73423584,74.94970998)
\curveto(632.64424047,74.91970371)(632.60424051,74.86970376)(632.61423584,74.79970998)
\curveto(632.62424049,74.73970389)(632.65424046,74.68470395)(632.70423584,74.63470998)
\curveto(632.75424036,74.55470408)(632.8142403,74.48470415)(632.88423584,74.42470998)
\curveto(632.95424016,74.37470426)(633.0142401,74.30970432)(633.06423584,74.22970998)
\curveto(633.17423994,74.06970456)(633.27423984,73.90470473)(633.36423584,73.73470998)
\curveto(633.44423967,73.56470507)(633.5142396,73.36970526)(633.57423584,73.14970998)
\curveto(633.60423951,73.04970558)(633.6192395,72.94970568)(633.61923584,72.84970998)
\curveto(633.6192395,72.75970587)(633.62923949,72.65970597)(633.64923584,72.54970998)
\lineto(633.64923584,72.39970998)
\curveto(633.62923949,72.34970628)(633.62423949,72.29970633)(633.63423584,72.24970998)
\curveto(633.64423947,72.20970642)(633.64423947,72.16970646)(633.63423584,72.12970998)
\curveto(633.62423949,72.09970653)(633.6192395,72.05470658)(633.61923584,71.99470998)
\curveto(633.60923951,71.9347067)(633.59923952,71.86970676)(633.58923584,71.79970998)
\lineto(633.55923584,71.61970998)
\curveto(633.43923968,71.16970746)(633.27423984,70.78970784)(633.06423584,70.47970998)
\curveto(632.87424024,70.20970842)(632.64424047,69.97970865)(632.37423584,69.78970998)
\curveto(632.09424102,69.60970902)(631.77924134,69.46470917)(631.42923584,69.35470998)
\lineto(631.21923584,69.29470998)
\curveto(631.13924198,69.28470935)(631.05924206,69.26970936)(630.97923584,69.24970998)
\curveto(630.94924217,69.23970939)(630.9192422,69.2347094)(630.88923584,69.23470998)
\curveto(630.85924226,69.2347094)(630.82924229,69.2297094)(630.79923584,69.21970998)
\curveto(630.73924238,69.20970942)(630.67924244,69.20470943)(630.61923584,69.20470998)
\curveto(630.54924257,69.20470943)(630.48924263,69.19470944)(630.43923584,69.17470998)
\lineto(630.25923584,69.17470998)
\curveto(630.20924291,69.16470947)(630.13924298,69.15970947)(630.04923584,69.15970998)
\curveto(629.95924316,69.15970947)(629.88924323,69.16970946)(629.83923584,69.18970998)
\lineto(629.67423584,69.18970998)
\curveto(629.59424352,69.20970942)(629.5192436,69.21970941)(629.44923584,69.21970998)
\curveto(629.37924374,69.2297094)(629.30924381,69.24470939)(629.23923584,69.26470998)
\curveto(629.03924408,69.32470931)(628.84924427,69.38470925)(628.66923584,69.44470998)
\curveto(628.48924463,69.51470912)(628.3192448,69.60470903)(628.15923584,69.71470998)
\curveto(628.08924503,69.75470888)(628.02424509,69.79470884)(627.96423584,69.83470998)
\lineto(627.78423584,69.98470998)
\curveto(627.77424534,70.00470863)(627.75924536,70.02470861)(627.73923584,70.04470998)
\curveto(627.60924551,70.1347085)(627.49924562,70.24470839)(627.40923584,70.37470998)
\curveto(627.20924591,70.634708)(627.05424606,70.89970773)(626.94423584,71.16970998)
\curveto(626.90424621,71.24970738)(626.87424624,71.3297073)(626.85423584,71.40970998)
\curveto(626.82424629,71.49970713)(626.79924632,71.58970704)(626.77923584,71.67970998)
\curveto(626.74924637,71.77970685)(626.72924639,71.87970675)(626.71923584,71.97970998)
\curveto(626.70924641,72.07970655)(626.69424642,72.18470645)(626.67423584,72.29470998)
\curveto(626.66424645,72.32470631)(626.66424645,72.36470627)(626.67423584,72.41470998)
\curveto(626.68424643,72.47470616)(626.67924644,72.51470612)(626.65923584,72.53470998)
\curveto(626.63924648,73.25470538)(626.75424636,73.85470478)(627.00423584,74.33470998)
\curveto(627.25424586,74.81470382)(627.59424552,75.18970344)(628.02423584,75.45970998)
\curveto(628.16424495,75.54970308)(628.30924481,75.629703)(628.45923584,75.69970998)
\curveto(628.60924451,75.76970286)(628.76924435,75.83970279)(628.93923584,75.90970998)
\curveto(629.07924404,75.95970267)(629.22924389,75.99970263)(629.38923584,76.02970998)
\curveto(629.54924357,76.05970257)(629.70924341,76.09470254)(629.86923584,76.13470998)
\curveto(629.9192432,76.15470248)(629.97424314,76.16470247)(630.03423584,76.16470998)
\curveto(630.08424303,76.16470247)(630.13424298,76.16970246)(630.18423584,76.17970998)
\curveto(630.24424287,76.19970243)(630.30924281,76.20970242)(630.37923584,76.20970998)
\curveto(630.43924268,76.20970242)(630.49424262,76.21970241)(630.54423584,76.23970998)
\lineto(630.70923584,76.23970998)
\curveto(630.75924236,76.25970237)(630.80924231,76.26470237)(630.85923584,76.25470998)
\curveto(630.90924221,76.24470239)(630.95924216,76.24970238)(631.00923584,76.26970998)
\curveto(631.02924209,76.26970236)(631.05424206,76.26470237)(631.08423584,76.25470998)
\curveto(631.114242,76.25470238)(631.13924198,76.25970237)(631.15923584,76.26970998)
\curveto(631.18924193,76.27970235)(631.22424189,76.27970235)(631.26423584,76.26970998)
\curveto(631.30424181,76.26970236)(631.34424177,76.27470236)(631.38423584,76.28470998)
\curveto(631.42424169,76.29470234)(631.46924165,76.29470234)(631.51923584,76.28470998)
\lineto(631.66923584,76.28470998)
\moveto(630.36423584,74.78470998)
\curveto(630.3142428,74.79470384)(630.25424286,74.79970383)(630.18423584,74.79970998)
\curveto(630.114243,74.79970383)(630.05424306,74.79470384)(630.00423584,74.78470998)
\curveto(629.95424316,74.77470386)(629.87924324,74.76970386)(629.77923584,74.76970998)
\curveto(629.69924342,74.74970388)(629.62424349,74.7297039)(629.55423584,74.70970998)
\curveto(629.48424363,74.69970393)(629.4142437,74.68470395)(629.34423584,74.66470998)
\curveto(628.9142442,74.52470411)(628.57924454,74.3297043)(628.33923584,74.07970998)
\curveto(628.09924502,73.83970479)(627.9192452,73.49470514)(627.79923584,73.04470998)
\curveto(627.77924534,72.95470568)(627.76924535,72.85470578)(627.76923584,72.74470998)
\lineto(627.76923584,72.41470998)
\curveto(627.78924533,72.39470624)(627.79924532,72.35970627)(627.79923584,72.30970998)
\curveto(627.78924533,72.25970637)(627.78924533,72.21470642)(627.79923584,72.17470998)
\curveto(627.8192453,72.09470654)(627.83924528,72.01970661)(627.85923584,71.94970998)
\lineto(627.91923584,71.73970998)
\curveto(628.04924507,71.44970718)(628.22924489,71.21970741)(628.45923584,71.04970998)
\curveto(628.67924444,70.87970775)(628.93924418,70.74470789)(629.23923584,70.64470998)
\curveto(629.32924379,70.61470802)(629.42424369,70.58970804)(629.52423584,70.56970998)
\curveto(629.6142435,70.55970807)(629.70924341,70.54470809)(629.80923584,70.52470998)
\lineto(629.94423584,70.52470998)
\curveto(630.05424306,70.49470814)(630.19424292,70.48470815)(630.36423584,70.49470998)
\curveto(630.52424259,70.51470812)(630.65424246,70.5347081)(630.75423584,70.55470998)
\curveto(630.8142423,70.57470806)(630.87424224,70.58970804)(630.93423584,70.59970998)
\curveto(630.98424213,70.60970802)(631.03424208,70.62470801)(631.08423584,70.64470998)
\curveto(631.28424183,70.72470791)(631.47424164,70.81970781)(631.65423584,70.92970998)
\curveto(631.83424128,71.04970758)(631.97924114,71.18970744)(632.08923584,71.34970998)
\curveto(632.13924098,71.39970723)(632.17924094,71.45470718)(632.20923584,71.51470998)
\curveto(632.23924088,71.57470706)(632.27424084,71.634707)(632.31423584,71.69470998)
\curveto(632.39424072,71.84470679)(632.45924066,72.0297066)(632.50923584,72.24970998)
\curveto(632.52924059,72.29970633)(632.53424058,72.33970629)(632.52423584,72.36970998)
\curveto(632.5142406,72.40970622)(632.5192406,72.45470618)(632.53923584,72.50470998)
\curveto(632.54924057,72.54470609)(632.55424056,72.59970603)(632.55423584,72.66970998)
\curveto(632.55424056,72.73970589)(632.54924057,72.79970583)(632.53923584,72.84970998)
\curveto(632.5192406,72.94970568)(632.50424061,73.04470559)(632.49423584,73.13470998)
\curveto(632.47424064,73.22470541)(632.44424067,73.31470532)(632.40423584,73.40470998)
\curveto(632.18424093,73.94470469)(631.78924133,74.33970429)(631.21923584,74.58970998)
\curveto(631.119242,74.63970399)(631.0192421,74.67470396)(630.91923584,74.69470998)
\curveto(630.80924231,74.71470392)(630.69924242,74.73970389)(630.58923584,74.76970998)
\curveto(630.48924263,74.76970386)(630.4142427,74.77470386)(630.36423584,74.78470998)
}
}
{
\newrgbcolor{curcolor}{0 0 0}
\pscustom[linestyle=none,fillstyle=solid,fillcolor=curcolor]
{
\newpath
\moveto(635.62923584,78.63431936)
\lineto(635.62923584,79.26431936)
\lineto(635.62923584,79.45931936)
\curveto(635.62923749,79.52931683)(635.63923748,79.58931677)(635.65923584,79.63931936)
\curveto(635.69923742,79.70931665)(635.73923738,79.7593166)(635.77923584,79.78931936)
\curveto(635.82923729,79.82931653)(635.89423722,79.84931651)(635.97423584,79.84931936)
\curveto(636.05423706,79.8593165)(636.13923698,79.86431649)(636.22923584,79.86431936)
\lineto(636.94923584,79.86431936)
\curveto(637.42923569,79.86431649)(637.83923528,79.80431655)(638.17923584,79.68431936)
\curveto(638.5192346,79.56431679)(638.79423432,79.36931699)(639.00423584,79.09931936)
\curveto(639.05423406,79.02931733)(639.09923402,78.9593174)(639.13923584,78.88931936)
\curveto(639.18923393,78.82931753)(639.23423388,78.7543176)(639.27423584,78.66431936)
\curveto(639.28423383,78.64431771)(639.29423382,78.61431774)(639.30423584,78.57431936)
\curveto(639.32423379,78.53431782)(639.32923379,78.48931787)(639.31923584,78.43931936)
\curveto(639.28923383,78.34931801)(639.2142339,78.29431806)(639.09423584,78.27431936)
\curveto(638.98423413,78.2543181)(638.88923423,78.26931809)(638.80923584,78.31931936)
\curveto(638.73923438,78.34931801)(638.67423444,78.39431796)(638.61423584,78.45431936)
\curveto(638.56423455,78.52431783)(638.5142346,78.58931777)(638.46423584,78.64931936)
\curveto(638.4142347,78.71931764)(638.33923478,78.77931758)(638.23923584,78.82931936)
\curveto(638.14923497,78.88931747)(638.05923506,78.93931742)(637.96923584,78.97931936)
\curveto(637.93923518,78.99931736)(637.87923524,79.02431733)(637.78923584,79.05431936)
\curveto(637.70923541,79.08431727)(637.63923548,79.08931727)(637.57923584,79.06931936)
\curveto(637.43923568,79.03931732)(637.34923577,78.97931738)(637.30923584,78.88931936)
\curveto(637.27923584,78.80931755)(637.26423585,78.71931764)(637.26423584,78.61931936)
\curveto(637.26423585,78.51931784)(637.23923588,78.43431792)(637.18923584,78.36431936)
\curveto(637.119236,78.27431808)(636.97923614,78.22931813)(636.76923584,78.22931936)
\lineto(636.21423584,78.22931936)
\lineto(635.98923584,78.22931936)
\curveto(635.90923721,78.23931812)(635.84423727,78.2593181)(635.79423584,78.28931936)
\curveto(635.7142374,78.34931801)(635.66923745,78.41931794)(635.65923584,78.49931936)
\curveto(635.64923747,78.51931784)(635.64423747,78.53931782)(635.64423584,78.55931936)
\curveto(635.64423747,78.58931777)(635.63923748,78.61431774)(635.62923584,78.63431936)
}
}
{
\newrgbcolor{curcolor}{0 0 0}
\pscustom[linestyle=none,fillstyle=solid,fillcolor=curcolor]
{
}
}
{
\newrgbcolor{curcolor}{0 0 0}
\pscustom[linestyle=none,fillstyle=solid,fillcolor=curcolor]
{
\newpath
\moveto(626.65923584,89.26463186)
\curveto(626.64924647,89.95462722)(626.76924635,90.55462662)(627.01923584,91.06463186)
\curveto(627.26924585,91.58462559)(627.60424551,91.9796252)(628.02423584,92.24963186)
\curveto(628.10424501,92.29962488)(628.19424492,92.34462483)(628.29423584,92.38463186)
\curveto(628.38424473,92.42462475)(628.47924464,92.46962471)(628.57923584,92.51963186)
\curveto(628.67924444,92.55962462)(628.77924434,92.58962459)(628.87923584,92.60963186)
\curveto(628.97924414,92.62962455)(629.08424403,92.64962453)(629.19423584,92.66963186)
\curveto(629.24424387,92.68962449)(629.28924383,92.69462448)(629.32923584,92.68463186)
\curveto(629.36924375,92.6746245)(629.4142437,92.6796245)(629.46423584,92.69963186)
\curveto(629.5142436,92.70962447)(629.59924352,92.71462446)(629.71923584,92.71463186)
\curveto(629.82924329,92.71462446)(629.9142432,92.70962447)(629.97423584,92.69963186)
\curveto(630.03424308,92.6796245)(630.09424302,92.66962451)(630.15423584,92.66963186)
\curveto(630.2142429,92.6796245)(630.27424284,92.6746245)(630.33423584,92.65463186)
\curveto(630.47424264,92.61462456)(630.60924251,92.5796246)(630.73923584,92.54963186)
\curveto(630.86924225,92.51962466)(630.99424212,92.4796247)(631.11423584,92.42963186)
\curveto(631.25424186,92.36962481)(631.37924174,92.29962488)(631.48923584,92.21963186)
\curveto(631.59924152,92.14962503)(631.70924141,92.0746251)(631.81923584,91.99463186)
\lineto(631.87923584,91.93463186)
\curveto(631.89924122,91.92462525)(631.9192412,91.90962527)(631.93923584,91.88963186)
\curveto(632.09924102,91.76962541)(632.24424087,91.63462554)(632.37423584,91.48463186)
\curveto(632.50424061,91.33462584)(632.62924049,91.174626)(632.74923584,91.00463186)
\curveto(632.96924015,90.69462648)(633.17423994,90.39962678)(633.36423584,90.11963186)
\curveto(633.50423961,89.88962729)(633.63923948,89.65962752)(633.76923584,89.42963186)
\curveto(633.89923922,89.20962797)(634.03423908,88.98962819)(634.17423584,88.76963186)
\curveto(634.34423877,88.51962866)(634.52423859,88.2796289)(634.71423584,88.04963186)
\curveto(634.90423821,87.82962935)(635.12923799,87.63962954)(635.38923584,87.47963186)
\curveto(635.44923767,87.43962974)(635.50923761,87.40462977)(635.56923584,87.37463186)
\curveto(635.6192375,87.34462983)(635.68423743,87.31462986)(635.76423584,87.28463186)
\curveto(635.83423728,87.26462991)(635.89423722,87.25962992)(635.94423584,87.26963186)
\curveto(636.0142371,87.28962989)(636.06923705,87.32462985)(636.10923584,87.37463186)
\curveto(636.13923698,87.42462975)(636.15923696,87.48462969)(636.16923584,87.55463186)
\lineto(636.16923584,87.79463186)
\lineto(636.16923584,88.54463186)
\lineto(636.16923584,91.34963186)
\lineto(636.16923584,92.00963186)
\curveto(636.16923695,92.09962508)(636.17423694,92.18462499)(636.18423584,92.26463186)
\curveto(636.18423693,92.34462483)(636.20423691,92.40962477)(636.24423584,92.45963186)
\curveto(636.28423683,92.50962467)(636.35923676,92.54962463)(636.46923584,92.57963186)
\curveto(636.56923655,92.61962456)(636.66923645,92.62962455)(636.76923584,92.60963186)
\lineto(636.90423584,92.60963186)
\curveto(636.97423614,92.58962459)(637.03423608,92.56962461)(637.08423584,92.54963186)
\curveto(637.13423598,92.52962465)(637.17423594,92.49462468)(637.20423584,92.44463186)
\curveto(637.24423587,92.39462478)(637.26423585,92.32462485)(637.26423584,92.23463186)
\lineto(637.26423584,91.96463186)
\lineto(637.26423584,91.06463186)
\lineto(637.26423584,87.55463186)
\lineto(637.26423584,86.48963186)
\curveto(637.26423585,86.40963077)(637.26923585,86.31963086)(637.27923584,86.21963186)
\curveto(637.27923584,86.11963106)(637.26923585,86.03463114)(637.24923584,85.96463186)
\curveto(637.17923594,85.75463142)(636.99923612,85.68963149)(636.70923584,85.76963186)
\curveto(636.66923645,85.7796314)(636.63423648,85.7796314)(636.60423584,85.76963186)
\curveto(636.56423655,85.76963141)(636.5192366,85.7796314)(636.46923584,85.79963186)
\curveto(636.38923673,85.81963136)(636.30423681,85.83963134)(636.21423584,85.85963186)
\curveto(636.12423699,85.8796313)(636.03923708,85.90463127)(635.95923584,85.93463186)
\curveto(635.46923765,86.09463108)(635.05423806,86.29463088)(634.71423584,86.53463186)
\curveto(634.46423865,86.71463046)(634.23923888,86.91963026)(634.03923584,87.14963186)
\curveto(633.82923929,87.3796298)(633.63423948,87.61962956)(633.45423584,87.86963186)
\curveto(633.27423984,88.12962905)(633.10424001,88.39462878)(632.94423584,88.66463186)
\curveto(632.77424034,88.94462823)(632.59924052,89.21462796)(632.41923584,89.47463186)
\curveto(632.33924078,89.58462759)(632.26424085,89.68962749)(632.19423584,89.78963186)
\curveto(632.12424099,89.89962728)(632.04924107,90.00962717)(631.96923584,90.11963186)
\curveto(631.93924118,90.15962702)(631.90924121,90.19462698)(631.87923584,90.22463186)
\curveto(631.83924128,90.26462691)(631.80924131,90.30462687)(631.78923584,90.34463186)
\curveto(631.67924144,90.48462669)(631.55424156,90.60962657)(631.41423584,90.71963186)
\curveto(631.38424173,90.73962644)(631.35924176,90.76462641)(631.33923584,90.79463186)
\curveto(631.30924181,90.82462635)(631.27924184,90.84962633)(631.24923584,90.86963186)
\curveto(631.14924197,90.94962623)(631.04924207,91.01462616)(630.94923584,91.06463186)
\curveto(630.84924227,91.12462605)(630.73924238,91.179626)(630.61923584,91.22963186)
\curveto(630.54924257,91.25962592)(630.47424264,91.2796259)(630.39423584,91.28963186)
\lineto(630.15423584,91.34963186)
\lineto(630.06423584,91.34963186)
\curveto(630.03424308,91.35962582)(630.00424311,91.36462581)(629.97423584,91.36463186)
\curveto(629.90424321,91.38462579)(629.80924331,91.38962579)(629.68923584,91.37963186)
\curveto(629.55924356,91.3796258)(629.45924366,91.36962581)(629.38923584,91.34963186)
\curveto(629.30924381,91.32962585)(629.23424388,91.30962587)(629.16423584,91.28963186)
\curveto(629.08424403,91.2796259)(629.00424411,91.25962592)(628.92423584,91.22963186)
\curveto(628.68424443,91.11962606)(628.48424463,90.96962621)(628.32423584,90.77963186)
\curveto(628.15424496,90.59962658)(628.0142451,90.3796268)(627.90423584,90.11963186)
\curveto(627.88424523,90.04962713)(627.86924525,89.9796272)(627.85923584,89.90963186)
\curveto(627.83924528,89.83962734)(627.8192453,89.76462741)(627.79923584,89.68463186)
\curveto(627.77924534,89.60462757)(627.76924535,89.49462768)(627.76923584,89.35463186)
\curveto(627.76924535,89.22462795)(627.77924534,89.11962806)(627.79923584,89.03963186)
\curveto(627.80924531,88.9796282)(627.8142453,88.92462825)(627.81423584,88.87463186)
\curveto(627.8142453,88.82462835)(627.82424529,88.7746284)(627.84423584,88.72463186)
\curveto(627.88424523,88.62462855)(627.92424519,88.52962865)(627.96423584,88.43963186)
\curveto(628.00424511,88.35962882)(628.04924507,88.2796289)(628.09923584,88.19963186)
\curveto(628.119245,88.16962901)(628.14424497,88.13962904)(628.17423584,88.10963186)
\curveto(628.20424491,88.08962909)(628.22924489,88.06462911)(628.24923584,88.03463186)
\lineto(628.32423584,87.95963186)
\curveto(628.34424477,87.92962925)(628.36424475,87.90462927)(628.38423584,87.88463186)
\lineto(628.59423584,87.73463186)
\curveto(628.65424446,87.69462948)(628.7192444,87.64962953)(628.78923584,87.59963186)
\curveto(628.87924424,87.53962964)(628.98424413,87.48962969)(629.10423584,87.44963186)
\curveto(629.2142439,87.41962976)(629.32424379,87.38462979)(629.43423584,87.34463186)
\curveto(629.54424357,87.30462987)(629.68924343,87.2796299)(629.86923584,87.26963186)
\curveto(630.03924308,87.25962992)(630.16424295,87.22962995)(630.24423584,87.17963186)
\curveto(630.32424279,87.12963005)(630.36924275,87.05463012)(630.37923584,86.95463186)
\curveto(630.38924273,86.85463032)(630.39424272,86.74463043)(630.39423584,86.62463186)
\curveto(630.39424272,86.58463059)(630.39924272,86.54463063)(630.40923584,86.50463186)
\curveto(630.40924271,86.46463071)(630.40424271,86.42963075)(630.39423584,86.39963186)
\curveto(630.37424274,86.34963083)(630.36424275,86.29963088)(630.36423584,86.24963186)
\curveto(630.36424275,86.20963097)(630.35424276,86.16963101)(630.33423584,86.12963186)
\curveto(630.27424284,86.03963114)(630.13924298,85.99463118)(629.92923584,85.99463186)
\lineto(629.80923584,85.99463186)
\curveto(629.74924337,86.00463117)(629.68924343,86.00963117)(629.62923584,86.00963186)
\curveto(629.55924356,86.01963116)(629.49424362,86.02963115)(629.43423584,86.03963186)
\curveto(629.32424379,86.05963112)(629.22424389,86.0796311)(629.13423584,86.09963186)
\curveto(629.03424408,86.11963106)(628.93924418,86.14963103)(628.84923584,86.18963186)
\curveto(628.77924434,86.20963097)(628.7192444,86.22963095)(628.66923584,86.24963186)
\lineto(628.48923584,86.30963186)
\curveto(628.22924489,86.42963075)(627.98424513,86.58463059)(627.75423584,86.77463186)
\curveto(627.52424559,86.9746302)(627.33924578,87.18962999)(627.19923584,87.41963186)
\curveto(627.119246,87.52962965)(627.05424606,87.64462953)(627.00423584,87.76463186)
\lineto(626.85423584,88.15463186)
\curveto(626.80424631,88.26462891)(626.77424634,88.3796288)(626.76423584,88.49963186)
\curveto(626.74424637,88.61962856)(626.7192464,88.74462843)(626.68923584,88.87463186)
\curveto(626.68924643,88.94462823)(626.68924643,89.00962817)(626.68923584,89.06963186)
\curveto(626.67924644,89.12962805)(626.66924645,89.19462798)(626.65923584,89.26463186)
}
}
{
\newrgbcolor{curcolor}{0 0 0}
\pscustom[linestyle=none,fillstyle=solid,fillcolor=curcolor]
{
\newpath
\moveto(632.17923584,101.36424123)
\lineto(632.43423584,101.36424123)
\curveto(632.5142406,101.37423353)(632.58924053,101.36923353)(632.65923584,101.34924123)
\lineto(632.89923584,101.34924123)
\lineto(633.06423584,101.34924123)
\curveto(633.16423995,101.32923357)(633.26923985,101.31923358)(633.37923584,101.31924123)
\curveto(633.47923964,101.31923358)(633.57923954,101.30923359)(633.67923584,101.28924123)
\lineto(633.82923584,101.28924123)
\curveto(633.96923915,101.25923364)(634.10923901,101.23923366)(634.24923584,101.22924123)
\curveto(634.37923874,101.21923368)(634.50923861,101.19423371)(634.63923584,101.15424123)
\curveto(634.7192384,101.13423377)(634.80423831,101.11423379)(634.89423584,101.09424123)
\lineto(635.13423584,101.03424123)
\lineto(635.43423584,100.91424123)
\curveto(635.52423759,100.88423402)(635.6142375,100.84923405)(635.70423584,100.80924123)
\curveto(635.92423719,100.70923419)(636.13923698,100.57423433)(636.34923584,100.40424123)
\curveto(636.55923656,100.24423466)(636.72923639,100.06923483)(636.85923584,99.87924123)
\curveto(636.89923622,99.82923507)(636.93923618,99.76923513)(636.97923584,99.69924123)
\curveto(637.00923611,99.63923526)(637.04423607,99.57923532)(637.08423584,99.51924123)
\curveto(637.13423598,99.43923546)(637.17423594,99.34423556)(637.20423584,99.23424123)
\curveto(637.23423588,99.12423578)(637.26423585,99.01923588)(637.29423584,98.91924123)
\curveto(637.33423578,98.80923609)(637.35923576,98.6992362)(637.36923584,98.58924123)
\curveto(637.37923574,98.47923642)(637.39423572,98.36423654)(637.41423584,98.24424123)
\curveto(637.42423569,98.2042367)(637.42423569,98.15923674)(637.41423584,98.10924123)
\curveto(637.4142357,98.06923683)(637.4192357,98.02923687)(637.42923584,97.98924123)
\curveto(637.43923568,97.94923695)(637.44423567,97.89423701)(637.44423584,97.82424123)
\curveto(637.44423567,97.75423715)(637.43923568,97.7042372)(637.42923584,97.67424123)
\curveto(637.40923571,97.62423728)(637.40423571,97.57923732)(637.41423584,97.53924123)
\curveto(637.42423569,97.4992374)(637.42423569,97.46423744)(637.41423584,97.43424123)
\lineto(637.41423584,97.34424123)
\curveto(637.39423572,97.28423762)(637.37923574,97.21923768)(637.36923584,97.14924123)
\curveto(637.36923575,97.08923781)(637.36423575,97.02423788)(637.35423584,96.95424123)
\curveto(637.30423581,96.78423812)(637.25423586,96.62423828)(637.20423584,96.47424123)
\curveto(637.15423596,96.32423858)(637.08923603,96.17923872)(637.00923584,96.03924123)
\curveto(636.96923615,95.98923891)(636.93923618,95.93423897)(636.91923584,95.87424123)
\curveto(636.88923623,95.82423908)(636.85423626,95.77423913)(636.81423584,95.72424123)
\curveto(636.63423648,95.48423942)(636.4142367,95.28423962)(636.15423584,95.12424123)
\curveto(635.89423722,94.96423994)(635.60923751,94.82424008)(635.29923584,94.70424123)
\curveto(635.15923796,94.64424026)(635.0192381,94.5992403)(634.87923584,94.56924123)
\curveto(634.72923839,94.53924036)(634.57423854,94.5042404)(634.41423584,94.46424123)
\curveto(634.30423881,94.44424046)(634.19423892,94.42924047)(634.08423584,94.41924123)
\curveto(633.97423914,94.40924049)(633.86423925,94.39424051)(633.75423584,94.37424123)
\curveto(633.7142394,94.36424054)(633.67423944,94.35924054)(633.63423584,94.35924123)
\curveto(633.59423952,94.36924053)(633.55423956,94.36924053)(633.51423584,94.35924123)
\curveto(633.46423965,94.34924055)(633.4142397,94.34424056)(633.36423584,94.34424123)
\lineto(633.19923584,94.34424123)
\curveto(633.14923997,94.32424058)(633.09924002,94.31924058)(633.04923584,94.32924123)
\curveto(632.98924013,94.33924056)(632.93424018,94.33924056)(632.88423584,94.32924123)
\curveto(632.84424027,94.31924058)(632.79924032,94.31924058)(632.74923584,94.32924123)
\curveto(632.69924042,94.33924056)(632.64924047,94.33424057)(632.59923584,94.31424123)
\curveto(632.52924059,94.29424061)(632.45424066,94.28924061)(632.37423584,94.29924123)
\curveto(632.28424083,94.30924059)(632.19924092,94.31424059)(632.11923584,94.31424123)
\curveto(632.02924109,94.31424059)(631.92924119,94.30924059)(631.81923584,94.29924123)
\curveto(631.69924142,94.28924061)(631.59924152,94.29424061)(631.51923584,94.31424123)
\lineto(631.23423584,94.31424123)
\lineto(630.60423584,94.35924123)
\curveto(630.50424261,94.36924053)(630.40924271,94.37924052)(630.31923584,94.38924123)
\lineto(630.01923584,94.41924123)
\curveto(629.96924315,94.43924046)(629.9192432,94.44424046)(629.86923584,94.43424123)
\curveto(629.80924331,94.43424047)(629.75424336,94.44424046)(629.70423584,94.46424123)
\curveto(629.53424358,94.51424039)(629.36924375,94.55424035)(629.20923584,94.58424123)
\curveto(629.03924408,94.61424029)(628.87924424,94.66424024)(628.72923584,94.73424123)
\curveto(628.26924485,94.92423998)(627.89424522,95.14423976)(627.60423584,95.39424123)
\curveto(627.3142458,95.65423925)(627.06924605,96.01423889)(626.86923584,96.47424123)
\curveto(626.8192463,96.6042383)(626.78424633,96.73423817)(626.76423584,96.86424123)
\curveto(626.74424637,97.0042379)(626.7192464,97.14423776)(626.68923584,97.28424123)
\curveto(626.67924644,97.35423755)(626.67424644,97.41923748)(626.67423584,97.47924123)
\curveto(626.67424644,97.53923736)(626.66924645,97.6042373)(626.65923584,97.67424123)
\curveto(626.63924648,98.5042364)(626.78924633,99.17423573)(627.10923584,99.68424123)
\curveto(627.4192457,100.19423471)(627.85924526,100.57423433)(628.42923584,100.82424123)
\curveto(628.54924457,100.87423403)(628.67424444,100.91923398)(628.80423584,100.95924123)
\curveto(628.93424418,100.9992339)(629.06924405,101.04423386)(629.20923584,101.09424123)
\curveto(629.28924383,101.11423379)(629.37424374,101.12923377)(629.46423584,101.13924123)
\lineto(629.70423584,101.19924123)
\curveto(629.8142433,101.22923367)(629.92424319,101.24423366)(630.03423584,101.24424123)
\curveto(630.14424297,101.25423365)(630.25424286,101.26923363)(630.36423584,101.28924123)
\curveto(630.4142427,101.30923359)(630.45924266,101.31423359)(630.49923584,101.30424123)
\curveto(630.53924258,101.3042336)(630.57924254,101.30923359)(630.61923584,101.31924123)
\curveto(630.66924245,101.32923357)(630.72424239,101.32923357)(630.78423584,101.31924123)
\curveto(630.83424228,101.31923358)(630.88424223,101.32423358)(630.93423584,101.33424123)
\lineto(631.06923584,101.33424123)
\curveto(631.12924199,101.35423355)(631.19924192,101.35423355)(631.27923584,101.33424123)
\curveto(631.34924177,101.32423358)(631.4142417,101.32923357)(631.47423584,101.34924123)
\curveto(631.50424161,101.35923354)(631.54424157,101.36423354)(631.59423584,101.36424123)
\lineto(631.71423584,101.36424123)
\lineto(632.17923584,101.36424123)
\moveto(634.50423584,99.81924123)
\curveto(634.18423893,99.91923498)(633.8192393,99.97923492)(633.40923584,99.99924123)
\curveto(632.99924012,100.01923488)(632.58924053,100.02923487)(632.17923584,100.02924123)
\curveto(631.74924137,100.02923487)(631.32924179,100.01923488)(630.91923584,99.99924123)
\curveto(630.50924261,99.97923492)(630.12424299,99.93423497)(629.76423584,99.86424123)
\curveto(629.40424371,99.79423511)(629.08424403,99.68423522)(628.80423584,99.53424123)
\curveto(628.5142446,99.39423551)(628.27924484,99.1992357)(628.09923584,98.94924123)
\curveto(627.98924513,98.78923611)(627.90924521,98.60923629)(627.85923584,98.40924123)
\curveto(627.79924532,98.20923669)(627.76924535,97.96423694)(627.76923584,97.67424123)
\curveto(627.78924533,97.65423725)(627.79924532,97.61923728)(627.79923584,97.56924123)
\curveto(627.78924533,97.51923738)(627.78924533,97.47923742)(627.79923584,97.44924123)
\curveto(627.8192453,97.36923753)(627.83924528,97.29423761)(627.85923584,97.22424123)
\curveto(627.86924525,97.16423774)(627.88924523,97.0992378)(627.91923584,97.02924123)
\curveto(628.03924508,96.75923814)(628.20924491,96.53923836)(628.42923584,96.36924123)
\curveto(628.63924448,96.20923869)(628.88424423,96.07423883)(629.16423584,95.96424123)
\curveto(629.27424384,95.91423899)(629.39424372,95.87423903)(629.52423584,95.84424123)
\curveto(629.64424347,95.82423908)(629.76924335,95.7992391)(629.89923584,95.76924123)
\curveto(629.94924317,95.74923915)(630.00424311,95.73923916)(630.06423584,95.73924123)
\curveto(630.114243,95.73923916)(630.16424295,95.73423917)(630.21423584,95.72424123)
\curveto(630.30424281,95.71423919)(630.39924272,95.7042392)(630.49923584,95.69424123)
\curveto(630.58924253,95.68423922)(630.68424243,95.67423923)(630.78423584,95.66424123)
\curveto(630.86424225,95.66423924)(630.94924217,95.65923924)(631.03923584,95.64924123)
\lineto(631.27923584,95.64924123)
\lineto(631.45923584,95.64924123)
\curveto(631.48924163,95.63923926)(631.52424159,95.63423927)(631.56423584,95.63424123)
\lineto(631.69923584,95.63424123)
\lineto(632.14923584,95.63424123)
\curveto(632.22924089,95.63423927)(632.3142408,95.62923927)(632.40423584,95.61924123)
\curveto(632.48424063,95.61923928)(632.55924056,95.62923927)(632.62923584,95.64924123)
\lineto(632.89923584,95.64924123)
\curveto(632.9192402,95.64923925)(632.94924017,95.64423926)(632.98923584,95.63424123)
\curveto(633.0192401,95.63423927)(633.04424007,95.63923926)(633.06423584,95.64924123)
\curveto(633.16423995,95.65923924)(633.26423985,95.66423924)(633.36423584,95.66424123)
\curveto(633.45423966,95.67423923)(633.55423956,95.68423922)(633.66423584,95.69424123)
\curveto(633.78423933,95.72423918)(633.90923921,95.73923916)(634.03923584,95.73924123)
\curveto(634.15923896,95.74923915)(634.27423884,95.77423913)(634.38423584,95.81424123)
\curveto(634.68423843,95.89423901)(634.94923817,95.97923892)(635.17923584,96.06924123)
\curveto(635.40923771,96.16923873)(635.62423749,96.31423859)(635.82423584,96.50424123)
\curveto(636.02423709,96.71423819)(636.17423694,96.97923792)(636.27423584,97.29924123)
\curveto(636.29423682,97.33923756)(636.30423681,97.37423753)(636.30423584,97.40424123)
\curveto(636.29423682,97.44423746)(636.29923682,97.48923741)(636.31923584,97.53924123)
\curveto(636.32923679,97.57923732)(636.33923678,97.64923725)(636.34923584,97.74924123)
\curveto(636.35923676,97.85923704)(636.35423676,97.94423696)(636.33423584,98.00424123)
\curveto(636.3142368,98.07423683)(636.30423681,98.14423676)(636.30423584,98.21424123)
\curveto(636.29423682,98.28423662)(636.27923684,98.34923655)(636.25923584,98.40924123)
\curveto(636.19923692,98.60923629)(636.114237,98.78923611)(636.00423584,98.94924123)
\curveto(635.98423713,98.97923592)(635.96423715,99.0042359)(635.94423584,99.02424123)
\lineto(635.88423584,99.08424123)
\curveto(635.86423725,99.12423578)(635.82423729,99.17423573)(635.76423584,99.23424123)
\curveto(635.62423749,99.33423557)(635.49423762,99.41923548)(635.37423584,99.48924123)
\curveto(635.25423786,99.55923534)(635.10923801,99.62923527)(634.93923584,99.69924123)
\curveto(634.86923825,99.72923517)(634.79923832,99.74923515)(634.72923584,99.75924123)
\curveto(634.65923846,99.77923512)(634.58423853,99.7992351)(634.50423584,99.81924123)
}
}
{
\newrgbcolor{curcolor}{0 0 0}
\pscustom[linestyle=none,fillstyle=solid,fillcolor=curcolor]
{
\newpath
\moveto(626.65923584,106.77385061)
\curveto(626.65924646,106.87384575)(626.66924645,106.96884566)(626.68923584,107.05885061)
\curveto(626.69924642,107.14884548)(626.72924639,107.21384541)(626.77923584,107.25385061)
\curveto(626.85924626,107.31384531)(626.96424615,107.34384528)(627.09423584,107.34385061)
\lineto(627.48423584,107.34385061)
\lineto(628.98423584,107.34385061)
\lineto(635.37423584,107.34385061)
\lineto(636.54423584,107.34385061)
\lineto(636.85923584,107.34385061)
\curveto(636.95923616,107.35384527)(637.03923608,107.33884529)(637.09923584,107.29885061)
\curveto(637.17923594,107.24884538)(637.22923589,107.17384545)(637.24923584,107.07385061)
\curveto(637.25923586,106.98384564)(637.26423585,106.87384575)(637.26423584,106.74385061)
\lineto(637.26423584,106.51885061)
\curveto(637.24423587,106.43884619)(637.22923589,106.36884626)(637.21923584,106.30885061)
\curveto(637.19923592,106.24884638)(637.15923596,106.19884643)(637.09923584,106.15885061)
\curveto(637.03923608,106.11884651)(636.96423615,106.09884653)(636.87423584,106.09885061)
\lineto(636.57423584,106.09885061)
\lineto(635.47923584,106.09885061)
\lineto(630.13923584,106.09885061)
\curveto(630.04924307,106.07884655)(629.97424314,106.06384656)(629.91423584,106.05385061)
\curveto(629.84424327,106.05384657)(629.78424333,106.0238466)(629.73423584,105.96385061)
\curveto(629.68424343,105.89384673)(629.65924346,105.80384682)(629.65923584,105.69385061)
\curveto(629.64924347,105.59384703)(629.64424347,105.48384714)(629.64423584,105.36385061)
\lineto(629.64423584,104.22385061)
\lineto(629.64423584,103.72885061)
\curveto(629.63424348,103.56884906)(629.57424354,103.45884917)(629.46423584,103.39885061)
\curveto(629.43424368,103.37884925)(629.40424371,103.36884926)(629.37423584,103.36885061)
\curveto(629.33424378,103.36884926)(629.28924383,103.36384926)(629.23923584,103.35385061)
\curveto(629.119244,103.33384929)(629.00924411,103.33884929)(628.90923584,103.36885061)
\curveto(628.80924431,103.40884922)(628.73924438,103.46384916)(628.69923584,103.53385061)
\curveto(628.64924447,103.61384901)(628.62424449,103.73384889)(628.62423584,103.89385061)
\curveto(628.62424449,104.05384857)(628.60924451,104.18884844)(628.57923584,104.29885061)
\curveto(628.56924455,104.34884828)(628.56424455,104.40384822)(628.56423584,104.46385061)
\curveto(628.55424456,104.5238481)(628.53924458,104.58384804)(628.51923584,104.64385061)
\curveto(628.46924465,104.79384783)(628.4192447,104.93884769)(628.36923584,105.07885061)
\curveto(628.30924481,105.21884741)(628.23924488,105.35384727)(628.15923584,105.48385061)
\curveto(628.06924505,105.623847)(627.96424515,105.74384688)(627.84423584,105.84385061)
\curveto(627.72424539,105.94384668)(627.59424552,106.03884659)(627.45423584,106.12885061)
\curveto(627.35424576,106.18884644)(627.24424587,106.23384639)(627.12423584,106.26385061)
\curveto(627.00424611,106.30384632)(626.89924622,106.35384627)(626.80923584,106.41385061)
\curveto(626.74924637,106.46384616)(626.70924641,106.53384609)(626.68923584,106.62385061)
\curveto(626.67924644,106.64384598)(626.67424644,106.66884596)(626.67423584,106.69885061)
\curveto(626.67424644,106.7288459)(626.66924645,106.75384587)(626.65923584,106.77385061)
}
}
{
\newrgbcolor{curcolor}{0 0 0}
\pscustom[linestyle=none,fillstyle=solid,fillcolor=curcolor]
{
\newpath
\moveto(626.65923584,115.12345998)
\curveto(626.65924646,115.22345513)(626.66924645,115.31845503)(626.68923584,115.40845998)
\curveto(626.69924642,115.49845485)(626.72924639,115.56345479)(626.77923584,115.60345998)
\curveto(626.85924626,115.66345469)(626.96424615,115.69345466)(627.09423584,115.69345998)
\lineto(627.48423584,115.69345998)
\lineto(628.98423584,115.69345998)
\lineto(635.37423584,115.69345998)
\lineto(636.54423584,115.69345998)
\lineto(636.85923584,115.69345998)
\curveto(636.95923616,115.70345465)(637.03923608,115.68845466)(637.09923584,115.64845998)
\curveto(637.17923594,115.59845475)(637.22923589,115.52345483)(637.24923584,115.42345998)
\curveto(637.25923586,115.33345502)(637.26423585,115.22345513)(637.26423584,115.09345998)
\lineto(637.26423584,114.86845998)
\curveto(637.24423587,114.78845556)(637.22923589,114.71845563)(637.21923584,114.65845998)
\curveto(637.19923592,114.59845575)(637.15923596,114.5484558)(637.09923584,114.50845998)
\curveto(637.03923608,114.46845588)(636.96423615,114.4484559)(636.87423584,114.44845998)
\lineto(636.57423584,114.44845998)
\lineto(635.47923584,114.44845998)
\lineto(630.13923584,114.44845998)
\curveto(630.04924307,114.42845592)(629.97424314,114.41345594)(629.91423584,114.40345998)
\curveto(629.84424327,114.40345595)(629.78424333,114.37345598)(629.73423584,114.31345998)
\curveto(629.68424343,114.24345611)(629.65924346,114.1534562)(629.65923584,114.04345998)
\curveto(629.64924347,113.94345641)(629.64424347,113.83345652)(629.64423584,113.71345998)
\lineto(629.64423584,112.57345998)
\lineto(629.64423584,112.07845998)
\curveto(629.63424348,111.91845843)(629.57424354,111.80845854)(629.46423584,111.74845998)
\curveto(629.43424368,111.72845862)(629.40424371,111.71845863)(629.37423584,111.71845998)
\curveto(629.33424378,111.71845863)(629.28924383,111.71345864)(629.23923584,111.70345998)
\curveto(629.119244,111.68345867)(629.00924411,111.68845866)(628.90923584,111.71845998)
\curveto(628.80924431,111.75845859)(628.73924438,111.81345854)(628.69923584,111.88345998)
\curveto(628.64924447,111.96345839)(628.62424449,112.08345827)(628.62423584,112.24345998)
\curveto(628.62424449,112.40345795)(628.60924451,112.53845781)(628.57923584,112.64845998)
\curveto(628.56924455,112.69845765)(628.56424455,112.7534576)(628.56423584,112.81345998)
\curveto(628.55424456,112.87345748)(628.53924458,112.93345742)(628.51923584,112.99345998)
\curveto(628.46924465,113.14345721)(628.4192447,113.28845706)(628.36923584,113.42845998)
\curveto(628.30924481,113.56845678)(628.23924488,113.70345665)(628.15923584,113.83345998)
\curveto(628.06924505,113.97345638)(627.96424515,114.09345626)(627.84423584,114.19345998)
\curveto(627.72424539,114.29345606)(627.59424552,114.38845596)(627.45423584,114.47845998)
\curveto(627.35424576,114.53845581)(627.24424587,114.58345577)(627.12423584,114.61345998)
\curveto(627.00424611,114.6534557)(626.89924622,114.70345565)(626.80923584,114.76345998)
\curveto(626.74924637,114.81345554)(626.70924641,114.88345547)(626.68923584,114.97345998)
\curveto(626.67924644,114.99345536)(626.67424644,115.01845533)(626.67423584,115.04845998)
\curveto(626.67424644,115.07845527)(626.66924645,115.10345525)(626.65923584,115.12345998)
}
}
{
\newrgbcolor{curcolor}{0 0 0}
\pscustom[linestyle=none,fillstyle=solid,fillcolor=curcolor]
{
\newpath
\moveto(648.5305249,37.28705373)
\curveto(648.5305356,37.35704805)(648.5305356,37.43704797)(648.5305249,37.52705373)
\curveto(648.52053561,37.61704779)(648.52053561,37.70204771)(648.5305249,37.78205373)
\curveto(648.5305356,37.87204754)(648.54053559,37.95204746)(648.5605249,38.02205373)
\curveto(648.58053555,38.10204731)(648.61053552,38.15704725)(648.6505249,38.18705373)
\curveto(648.70053543,38.21704719)(648.77553535,38.23704717)(648.8755249,38.24705373)
\curveto(648.96553516,38.26704714)(649.07053506,38.27704713)(649.1905249,38.27705373)
\curveto(649.30053483,38.28704712)(649.41553471,38.28704712)(649.5355249,38.27705373)
\lineto(649.8355249,38.27705373)
\lineto(652.8505249,38.27705373)
\lineto(655.7455249,38.27705373)
\curveto(656.07552805,38.27704713)(656.40052773,38.27204714)(656.7205249,38.26205373)
\curveto(657.0305271,38.26204715)(657.31052682,38.22204719)(657.5605249,38.14205373)
\curveto(657.91052622,38.02204739)(658.20552592,37.86704754)(658.4455249,37.67705373)
\curveto(658.67552545,37.48704792)(658.87552525,37.24704816)(659.0455249,36.95705373)
\curveto(659.09552503,36.89704851)(659.130525,36.83204858)(659.1505249,36.76205373)
\curveto(659.17052496,36.70204871)(659.19552493,36.63204878)(659.2255249,36.55205373)
\curveto(659.27552485,36.43204898)(659.31052482,36.30204911)(659.3305249,36.16205373)
\curveto(659.36052477,36.03204938)(659.39052474,35.89704951)(659.4205249,35.75705373)
\curveto(659.44052469,35.7070497)(659.44552468,35.65704975)(659.4355249,35.60705373)
\curveto(659.4255247,35.55704985)(659.4255247,35.50204991)(659.4355249,35.44205373)
\curveto(659.44552468,35.42204999)(659.44552468,35.39705001)(659.4355249,35.36705373)
\curveto(659.43552469,35.33705007)(659.44052469,35.3120501)(659.4505249,35.29205373)
\curveto(659.46052467,35.25205016)(659.46552466,35.19705021)(659.4655249,35.12705373)
\curveto(659.46552466,35.05705035)(659.46052467,35.00205041)(659.4505249,34.96205373)
\curveto(659.44052469,34.9120505)(659.44052469,34.85705055)(659.4505249,34.79705373)
\curveto(659.46052467,34.73705067)(659.45552467,34.68205073)(659.4355249,34.63205373)
\curveto(659.40552472,34.50205091)(659.38552474,34.37705103)(659.3755249,34.25705373)
\curveto(659.36552476,34.13705127)(659.34052479,34.02205139)(659.3005249,33.91205373)
\curveto(659.18052495,33.54205187)(659.01052512,33.22205219)(658.7905249,32.95205373)
\curveto(658.57052556,32.68205273)(658.29052584,32.47205294)(657.9505249,32.32205373)
\curveto(657.8305263,32.27205314)(657.70552642,32.22705318)(657.5755249,32.18705373)
\curveto(657.44552668,32.15705325)(657.31052682,32.12205329)(657.1705249,32.08205373)
\curveto(657.12052701,32.07205334)(657.08052705,32.06705334)(657.0505249,32.06705373)
\curveto(657.01052712,32.06705334)(656.96552716,32.06205335)(656.9155249,32.05205373)
\curveto(656.88552724,32.04205337)(656.85052728,32.03705337)(656.8105249,32.03705373)
\curveto(656.76052737,32.03705337)(656.72052741,32.03205338)(656.6905249,32.02205373)
\lineto(656.5255249,32.02205373)
\curveto(656.44552768,32.00205341)(656.34552778,31.99705341)(656.2255249,32.00705373)
\curveto(656.09552803,32.01705339)(656.00552812,32.03205338)(655.9555249,32.05205373)
\curveto(655.86552826,32.07205334)(655.80052833,32.12705328)(655.7605249,32.21705373)
\curveto(655.74052839,32.24705316)(655.73552839,32.27705313)(655.7455249,32.30705373)
\curveto(655.74552838,32.33705307)(655.74052839,32.37705303)(655.7305249,32.42705373)
\curveto(655.72052841,32.46705294)(655.71552841,32.5070529)(655.7155249,32.54705373)
\lineto(655.7155249,32.69705373)
\curveto(655.71552841,32.81705259)(655.72052841,32.93705247)(655.7305249,33.05705373)
\curveto(655.7305284,33.18705222)(655.76552836,33.27705213)(655.8355249,33.32705373)
\curveto(655.89552823,33.36705204)(655.95552817,33.38705202)(656.0155249,33.38705373)
\curveto(656.07552805,33.38705202)(656.14552798,33.39705201)(656.2255249,33.41705373)
\curveto(656.25552787,33.42705198)(656.29052784,33.42705198)(656.3305249,33.41705373)
\curveto(656.36052777,33.41705199)(656.38552774,33.42205199)(656.4055249,33.43205373)
\lineto(656.6155249,33.43205373)
\curveto(656.66552746,33.45205196)(656.71552741,33.45705195)(656.7655249,33.44705373)
\curveto(656.80552732,33.44705196)(656.85052728,33.45705195)(656.9005249,33.47705373)
\curveto(657.0305271,33.5070519)(657.15552697,33.53705187)(657.2755249,33.56705373)
\curveto(657.38552674,33.59705181)(657.49052664,33.64205177)(657.5905249,33.70205373)
\curveto(657.88052625,33.87205154)(658.08552604,34.14205127)(658.2055249,34.51205373)
\curveto(658.2255259,34.56205085)(658.24052589,34.6120508)(658.2505249,34.66205373)
\curveto(658.25052588,34.72205069)(658.26052587,34.77705063)(658.2805249,34.82705373)
\lineto(658.2805249,34.90205373)
\curveto(658.29052584,34.97205044)(658.30052583,35.06705034)(658.3105249,35.18705373)
\curveto(658.31052582,35.31705009)(658.30052583,35.41704999)(658.2805249,35.48705373)
\curveto(658.26052587,35.55704985)(658.24552588,35.62704978)(658.2355249,35.69705373)
\curveto(658.21552591,35.77704963)(658.19552593,35.84704956)(658.1755249,35.90705373)
\curveto(658.01552611,36.28704912)(657.74052639,36.56204885)(657.3505249,36.73205373)
\curveto(657.22052691,36.78204863)(657.06552706,36.81704859)(656.8855249,36.83705373)
\curveto(656.70552742,36.86704854)(656.52052761,36.88204853)(656.3305249,36.88205373)
\curveto(656.130528,36.89204852)(655.9305282,36.89204852)(655.7305249,36.88205373)
\lineto(655.1605249,36.88205373)
\lineto(650.9155249,36.88205373)
\lineto(649.3705249,36.88205373)
\curveto(649.26053487,36.88204853)(649.14053499,36.87704853)(649.0105249,36.86705373)
\curveto(648.88053525,36.85704855)(648.77553535,36.87704853)(648.6955249,36.92705373)
\curveto(648.6255355,36.98704842)(648.57553555,37.06704834)(648.5455249,37.16705373)
\curveto(648.54553558,37.18704822)(648.54553558,37.2070482)(648.5455249,37.22705373)
\curveto(648.54553558,37.24704816)(648.54053559,37.26704814)(648.5305249,37.28705373)
}
}
{
\newrgbcolor{curcolor}{0 0 0}
\pscustom[linestyle=none,fillstyle=solid,fillcolor=curcolor]
{
\newpath
\moveto(651.4855249,40.82072561)
\lineto(651.4855249,41.25572561)
\curveto(651.48553264,41.40572364)(651.5255326,41.51072354)(651.6055249,41.57072561)
\curveto(651.68553244,41.62072343)(651.78553234,41.6457234)(651.9055249,41.64572561)
\curveto(652.0255321,41.65572339)(652.14553198,41.66072339)(652.2655249,41.66072561)
\lineto(653.6905249,41.66072561)
\lineto(655.9555249,41.66072561)
\lineto(656.6455249,41.66072561)
\curveto(656.87552725,41.66072339)(657.07552705,41.68572336)(657.2455249,41.73572561)
\curveto(657.69552643,41.89572315)(658.01052612,42.19572285)(658.1905249,42.63572561)
\curveto(658.28052585,42.85572219)(658.31552581,43.12072193)(658.2955249,43.43072561)
\curveto(658.26552586,43.74072131)(658.21052592,43.99072106)(658.1305249,44.18072561)
\curveto(657.99052614,44.51072054)(657.81552631,44.77072028)(657.6055249,44.96072561)
\curveto(657.38552674,45.16071989)(657.10052703,45.31571973)(656.7505249,45.42572561)
\curveto(656.67052746,45.45571959)(656.59052754,45.47571957)(656.5105249,45.48572561)
\curveto(656.4305277,45.49571955)(656.34552778,45.51071954)(656.2555249,45.53072561)
\curveto(656.20552792,45.54071951)(656.16052797,45.54071951)(656.1205249,45.53072561)
\curveto(656.08052805,45.53071952)(656.03552809,45.54071951)(655.9855249,45.56072561)
\lineto(655.6705249,45.56072561)
\curveto(655.59052854,45.58071947)(655.50052863,45.58571946)(655.4005249,45.57572561)
\curveto(655.29052884,45.56571948)(655.19052894,45.56071949)(655.1005249,45.56072561)
\lineto(653.9305249,45.56072561)
\lineto(652.3405249,45.56072561)
\curveto(652.22053191,45.56071949)(652.09553203,45.55571949)(651.9655249,45.54572561)
\curveto(651.8255323,45.5457195)(651.71553241,45.57071948)(651.6355249,45.62072561)
\curveto(651.58553254,45.66071939)(651.55553257,45.70571934)(651.5455249,45.75572561)
\curveto(651.5255326,45.81571923)(651.50553262,45.88571916)(651.4855249,45.96572561)
\lineto(651.4855249,46.19072561)
\curveto(651.48553264,46.31071874)(651.49053264,46.41571863)(651.5005249,46.50572561)
\curveto(651.51053262,46.60571844)(651.55553257,46.68071837)(651.6355249,46.73072561)
\curveto(651.68553244,46.78071827)(651.76053237,46.80571824)(651.8605249,46.80572561)
\lineto(652.1455249,46.80572561)
\lineto(653.1655249,46.80572561)
\lineto(657.2005249,46.80572561)
\lineto(658.5505249,46.80572561)
\curveto(658.67052546,46.80571824)(658.78552534,46.80071825)(658.8955249,46.79072561)
\curveto(658.99552513,46.79071826)(659.07052506,46.75571829)(659.1205249,46.68572561)
\curveto(659.15052498,46.6457184)(659.17552495,46.58571846)(659.1955249,46.50572561)
\curveto(659.20552492,46.42571862)(659.21552491,46.33571871)(659.2255249,46.23572561)
\curveto(659.2255249,46.1457189)(659.22052491,46.05571899)(659.2105249,45.96572561)
\curveto(659.20052493,45.88571916)(659.18052495,45.82571922)(659.1505249,45.78572561)
\curveto(659.11052502,45.73571931)(659.04552508,45.69071936)(658.9555249,45.65072561)
\curveto(658.91552521,45.64071941)(658.86052527,45.63071942)(658.7905249,45.62072561)
\curveto(658.72052541,45.62071943)(658.65552547,45.61571943)(658.5955249,45.60572561)
\curveto(658.5255256,45.59571945)(658.47052566,45.57571947)(658.4305249,45.54572561)
\curveto(658.39052574,45.51571953)(658.37552575,45.47071958)(658.3855249,45.41072561)
\curveto(658.40552572,45.33071972)(658.46552566,45.2507198)(658.5655249,45.17072561)
\curveto(658.65552547,45.09071996)(658.7255254,45.01572003)(658.7755249,44.94572561)
\curveto(658.93552519,44.72572032)(659.07552505,44.47572057)(659.1955249,44.19572561)
\curveto(659.24552488,44.08572096)(659.27552485,43.97072108)(659.2855249,43.85072561)
\curveto(659.30552482,43.74072131)(659.3305248,43.62572142)(659.3605249,43.50572561)
\curveto(659.37052476,43.45572159)(659.37052476,43.40072165)(659.3605249,43.34072561)
\curveto(659.35052478,43.29072176)(659.35552477,43.24072181)(659.3755249,43.19072561)
\curveto(659.39552473,43.09072196)(659.39552473,43.00072205)(659.3755249,42.92072561)
\lineto(659.3755249,42.77072561)
\curveto(659.35552477,42.72072233)(659.34552478,42.66072239)(659.3455249,42.59072561)
\curveto(659.34552478,42.53072252)(659.34052479,42.47572257)(659.3305249,42.42572561)
\curveto(659.31052482,42.38572266)(659.30052483,42.3457227)(659.3005249,42.30572561)
\curveto(659.31052482,42.27572277)(659.30552482,42.23572281)(659.2855249,42.18572561)
\lineto(659.2255249,41.94572561)
\curveto(659.20552492,41.87572317)(659.17552495,41.80072325)(659.1355249,41.72072561)
\curveto(659.0255251,41.46072359)(658.88052525,41.24072381)(658.7005249,41.06072561)
\curveto(658.51052562,40.89072416)(658.28552584,40.7507243)(658.0255249,40.64072561)
\curveto(657.93552619,40.60072445)(657.84552628,40.57072448)(657.7555249,40.55072561)
\lineto(657.4555249,40.49072561)
\curveto(657.39552673,40.47072458)(657.34052679,40.46072459)(657.2905249,40.46072561)
\curveto(657.2305269,40.47072458)(657.16552696,40.46572458)(657.0955249,40.44572561)
\curveto(657.07552705,40.43572461)(657.05052708,40.43072462)(657.0205249,40.43072561)
\curveto(656.98052715,40.43072462)(656.94552718,40.42572462)(656.9155249,40.41572561)
\lineto(656.7655249,40.41572561)
\curveto(656.7255274,40.40572464)(656.68052745,40.40072465)(656.6305249,40.40072561)
\curveto(656.57052756,40.41072464)(656.51552761,40.41572463)(656.4655249,40.41572561)
\lineto(655.8655249,40.41572561)
\lineto(653.1055249,40.41572561)
\lineto(652.1455249,40.41572561)
\lineto(651.8755249,40.41572561)
\curveto(651.78553234,40.41572463)(651.71053242,40.43572461)(651.6505249,40.47572561)
\curveto(651.58053255,40.51572453)(651.5305326,40.59072446)(651.5005249,40.70072561)
\curveto(651.49053264,40.72072433)(651.49053264,40.74072431)(651.5005249,40.76072561)
\curveto(651.50053263,40.78072427)(651.49553263,40.80072425)(651.4855249,40.82072561)
}
}
{
\newrgbcolor{curcolor}{0 0 0}
\pscustom[linestyle=none,fillstyle=solid,fillcolor=curcolor]
{
\newpath
\moveto(651.3355249,52.39533498)
\curveto(651.31553281,53.02532975)(651.40053273,53.53032924)(651.5905249,53.91033498)
\curveto(651.78053235,54.29032848)(652.06553206,54.59532818)(652.4455249,54.82533498)
\curveto(652.54553158,54.88532789)(652.65553147,54.93032784)(652.7755249,54.96033498)
\curveto(652.88553124,55.00032777)(653.00053113,55.03532774)(653.1205249,55.06533498)
\curveto(653.31053082,55.11532766)(653.51553061,55.14532763)(653.7355249,55.15533498)
\curveto(653.95553017,55.16532761)(654.18052995,55.1703276)(654.4105249,55.17033498)
\lineto(656.0155249,55.17033498)
\lineto(658.3555249,55.17033498)
\curveto(658.5255256,55.1703276)(658.69552543,55.16532761)(658.8655249,55.15533498)
\curveto(659.03552509,55.15532762)(659.14552498,55.09032768)(659.1955249,54.96033498)
\curveto(659.21552491,54.91032786)(659.2255249,54.85532792)(659.2255249,54.79533498)
\curveto(659.23552489,54.74532803)(659.24052489,54.69032808)(659.2405249,54.63033498)
\curveto(659.24052489,54.50032827)(659.23552489,54.3753284)(659.2255249,54.25533498)
\curveto(659.2255249,54.13532864)(659.18552494,54.05032872)(659.1055249,54.00033498)
\curveto(659.03552509,53.95032882)(658.94552518,53.92532885)(658.8355249,53.92533498)
\lineto(658.5055249,53.92533498)
\lineto(657.2155249,53.92533498)
\lineto(654.7705249,53.92533498)
\curveto(654.50052963,53.92532885)(654.23552989,53.92032885)(653.9755249,53.91033498)
\curveto(653.70553042,53.90032887)(653.47553065,53.85532892)(653.2855249,53.77533498)
\curveto(653.08553104,53.69532908)(652.9255312,53.5753292)(652.8055249,53.41533498)
\curveto(652.67553145,53.25532952)(652.57553155,53.0703297)(652.5055249,52.86033498)
\curveto(652.48553164,52.80032997)(652.47553165,52.73533004)(652.4755249,52.66533498)
\curveto(652.46553166,52.60533017)(652.45053168,52.54533023)(652.4305249,52.48533498)
\curveto(652.42053171,52.43533034)(652.42053171,52.35533042)(652.4305249,52.24533498)
\curveto(652.4305317,52.14533063)(652.43553169,52.0753307)(652.4455249,52.03533498)
\curveto(652.46553166,51.99533078)(652.47553165,51.96033081)(652.4755249,51.93033498)
\curveto(652.46553166,51.90033087)(652.46553166,51.86533091)(652.4755249,51.82533498)
\curveto(652.50553162,51.69533108)(652.54053159,51.5703312)(652.5805249,51.45033498)
\curveto(652.61053152,51.34033143)(652.65553147,51.23533154)(652.7155249,51.13533498)
\curveto(652.73553139,51.09533168)(652.75553137,51.06033171)(652.7755249,51.03033498)
\curveto(652.79553133,51.00033177)(652.81553131,50.96533181)(652.8355249,50.92533498)
\curveto(653.08553104,50.5753322)(653.46053067,50.32033245)(653.9605249,50.16033498)
\curveto(654.04053009,50.13033264)(654.12553,50.11033266)(654.2155249,50.10033498)
\curveto(654.29552983,50.09033268)(654.37552975,50.0753327)(654.4555249,50.05533498)
\curveto(654.50552962,50.03533274)(654.55552957,50.03033274)(654.6055249,50.04033498)
\curveto(654.64552948,50.05033272)(654.68552944,50.04533273)(654.7255249,50.02533498)
\lineto(655.0405249,50.02533498)
\curveto(655.07052906,50.01533276)(655.10552902,50.01033276)(655.1455249,50.01033498)
\curveto(655.18552894,50.02033275)(655.2305289,50.02533275)(655.2805249,50.02533498)
\lineto(655.7305249,50.02533498)
\lineto(657.1705249,50.02533498)
\lineto(658.4905249,50.02533498)
\lineto(658.8355249,50.02533498)
\curveto(658.94552518,50.02533275)(659.03552509,50.00033277)(659.1055249,49.95033498)
\curveto(659.18552494,49.90033287)(659.2255249,49.81033296)(659.2255249,49.68033498)
\curveto(659.23552489,49.56033321)(659.24052489,49.43533334)(659.2405249,49.30533498)
\curveto(659.24052489,49.22533355)(659.23552489,49.15033362)(659.2255249,49.08033498)
\curveto(659.21552491,49.01033376)(659.19052494,48.95033382)(659.1505249,48.90033498)
\curveto(659.10052503,48.82033395)(659.00552512,48.78033399)(658.8655249,48.78033498)
\lineto(658.4605249,48.78033498)
\lineto(656.6905249,48.78033498)
\lineto(653.0605249,48.78033498)
\lineto(652.1455249,48.78033498)
\lineto(651.8755249,48.78033498)
\curveto(651.78553234,48.78033399)(651.71553241,48.80033397)(651.6655249,48.84033498)
\curveto(651.60553252,48.8703339)(651.56553256,48.92033385)(651.5455249,48.99033498)
\curveto(651.53553259,49.03033374)(651.5255326,49.08533369)(651.5155249,49.15533498)
\curveto(651.50553262,49.23533354)(651.50053263,49.31533346)(651.5005249,49.39533498)
\curveto(651.50053263,49.4753333)(651.50553262,49.55033322)(651.5155249,49.62033498)
\curveto(651.5255326,49.70033307)(651.54053259,49.75533302)(651.5605249,49.78533498)
\curveto(651.6305325,49.89533288)(651.72053241,49.94533283)(651.8305249,49.93533498)
\curveto(651.9305322,49.92533285)(652.04553208,49.94033283)(652.1755249,49.98033498)
\curveto(652.23553189,50.00033277)(652.28553184,50.04033273)(652.3255249,50.10033498)
\curveto(652.33553179,50.22033255)(652.29053184,50.31533246)(652.1905249,50.38533498)
\curveto(652.09053204,50.46533231)(652.01053212,50.54533223)(651.9505249,50.62533498)
\curveto(651.85053228,50.76533201)(651.76053237,50.90533187)(651.6805249,51.04533498)
\curveto(651.59053254,51.19533158)(651.51553261,51.36533141)(651.4555249,51.55533498)
\curveto(651.4255327,51.63533114)(651.40553272,51.72033105)(651.3955249,51.81033498)
\curveto(651.38553274,51.91033086)(651.37053276,52.00533077)(651.3505249,52.09533498)
\curveto(651.34053279,52.14533063)(651.33553279,52.19533058)(651.3355249,52.24533498)
\lineto(651.3355249,52.39533498)
}
}
{
\newrgbcolor{curcolor}{0 0 0}
\pscustom[linestyle=none,fillstyle=solid,fillcolor=curcolor]
{
}
}
{
\newrgbcolor{curcolor}{0 0 0}
\pscustom[linestyle=none,fillstyle=solid,fillcolor=curcolor]
{
\newpath
\moveto(654.1255249,67.99510061)
\lineto(654.3805249,67.99510061)
\curveto(654.46052967,68.0050929)(654.53552959,68.00009291)(654.6055249,67.98010061)
\lineto(654.8455249,67.98010061)
\lineto(655.0105249,67.98010061)
\curveto(655.11052902,67.96009295)(655.21552891,67.95009296)(655.3255249,67.95010061)
\curveto(655.4255287,67.95009296)(655.5255286,67.94009297)(655.6255249,67.92010061)
\lineto(655.7755249,67.92010061)
\curveto(655.91552821,67.89009302)(656.05552807,67.87009304)(656.1955249,67.86010061)
\curveto(656.3255278,67.85009306)(656.45552767,67.82509308)(656.5855249,67.78510061)
\curveto(656.66552746,67.76509314)(656.75052738,67.74509316)(656.8405249,67.72510061)
\lineto(657.0805249,67.66510061)
\lineto(657.3805249,67.54510061)
\curveto(657.47052666,67.51509339)(657.56052657,67.48009343)(657.6505249,67.44010061)
\curveto(657.87052626,67.34009357)(658.08552604,67.2050937)(658.2955249,67.03510061)
\curveto(658.50552562,66.87509403)(658.67552545,66.70009421)(658.8055249,66.51010061)
\curveto(658.84552528,66.46009445)(658.88552524,66.40009451)(658.9255249,66.33010061)
\curveto(658.95552517,66.27009464)(658.99052514,66.2100947)(659.0305249,66.15010061)
\curveto(659.08052505,66.07009484)(659.12052501,65.97509493)(659.1505249,65.86510061)
\curveto(659.18052495,65.75509515)(659.21052492,65.65009526)(659.2405249,65.55010061)
\curveto(659.28052485,65.44009547)(659.30552482,65.33009558)(659.3155249,65.22010061)
\curveto(659.3255248,65.1100958)(659.34052479,64.99509591)(659.3605249,64.87510061)
\curveto(659.37052476,64.83509607)(659.37052476,64.79009612)(659.3605249,64.74010061)
\curveto(659.36052477,64.70009621)(659.36552476,64.66009625)(659.3755249,64.62010061)
\curveto(659.38552474,64.58009633)(659.39052474,64.52509638)(659.3905249,64.45510061)
\curveto(659.39052474,64.38509652)(659.38552474,64.33509657)(659.3755249,64.30510061)
\curveto(659.35552477,64.25509665)(659.35052478,64.2100967)(659.3605249,64.17010061)
\curveto(659.37052476,64.13009678)(659.37052476,64.09509681)(659.3605249,64.06510061)
\lineto(659.3605249,63.97510061)
\curveto(659.34052479,63.91509699)(659.3255248,63.85009706)(659.3155249,63.78010061)
\curveto(659.31552481,63.72009719)(659.31052482,63.65509725)(659.3005249,63.58510061)
\curveto(659.25052488,63.41509749)(659.20052493,63.25509765)(659.1505249,63.10510061)
\curveto(659.10052503,62.95509795)(659.03552509,62.8100981)(658.9555249,62.67010061)
\curveto(658.91552521,62.62009829)(658.88552524,62.56509834)(658.8655249,62.50510061)
\curveto(658.83552529,62.45509845)(658.80052533,62.4050985)(658.7605249,62.35510061)
\curveto(658.58052555,62.11509879)(658.36052577,61.91509899)(658.1005249,61.75510061)
\curveto(657.84052629,61.59509931)(657.55552657,61.45509945)(657.2455249,61.33510061)
\curveto(657.10552702,61.27509963)(656.96552716,61.23009968)(656.8255249,61.20010061)
\curveto(656.67552745,61.17009974)(656.52052761,61.13509977)(656.3605249,61.09510061)
\curveto(656.25052788,61.07509983)(656.14052799,61.06009985)(656.0305249,61.05010061)
\curveto(655.92052821,61.04009987)(655.81052832,61.02509988)(655.7005249,61.00510061)
\curveto(655.66052847,60.99509991)(655.62052851,60.99009992)(655.5805249,60.99010061)
\curveto(655.54052859,61.00009991)(655.50052863,61.00009991)(655.4605249,60.99010061)
\curveto(655.41052872,60.98009993)(655.36052877,60.97509993)(655.3105249,60.97510061)
\lineto(655.1455249,60.97510061)
\curveto(655.09552903,60.95509995)(655.04552908,60.95009996)(654.9955249,60.96010061)
\curveto(654.93552919,60.97009994)(654.88052925,60.97009994)(654.8305249,60.96010061)
\curveto(654.79052934,60.95009996)(654.74552938,60.95009996)(654.6955249,60.96010061)
\curveto(654.64552948,60.97009994)(654.59552953,60.96509994)(654.5455249,60.94510061)
\curveto(654.47552965,60.92509998)(654.40052973,60.92009999)(654.3205249,60.93010061)
\curveto(654.2305299,60.94009997)(654.14552998,60.94509996)(654.0655249,60.94510061)
\curveto(653.97553015,60.94509996)(653.87553025,60.94009997)(653.7655249,60.93010061)
\curveto(653.64553048,60.92009999)(653.54553058,60.92509998)(653.4655249,60.94510061)
\lineto(653.1805249,60.94510061)
\lineto(652.5505249,60.99010061)
\curveto(652.45053168,61.00009991)(652.35553177,61.0100999)(652.2655249,61.02010061)
\lineto(651.9655249,61.05010061)
\curveto(651.91553221,61.07009984)(651.86553226,61.07509983)(651.8155249,61.06510061)
\curveto(651.75553237,61.06509984)(651.70053243,61.07509983)(651.6505249,61.09510061)
\curveto(651.48053265,61.14509976)(651.31553281,61.18509972)(651.1555249,61.21510061)
\curveto(650.98553314,61.24509966)(650.8255333,61.29509961)(650.6755249,61.36510061)
\curveto(650.21553391,61.55509935)(649.84053429,61.77509913)(649.5505249,62.02510061)
\curveto(649.26053487,62.28509862)(649.01553511,62.64509826)(648.8155249,63.10510061)
\curveto(648.76553536,63.23509767)(648.7305354,63.36509754)(648.7105249,63.49510061)
\curveto(648.69053544,63.63509727)(648.66553546,63.77509713)(648.6355249,63.91510061)
\curveto(648.6255355,63.98509692)(648.62053551,64.05009686)(648.6205249,64.11010061)
\curveto(648.62053551,64.17009674)(648.61553551,64.23509667)(648.6055249,64.30510061)
\curveto(648.58553554,65.13509577)(648.73553539,65.8050951)(649.0555249,66.31510061)
\curveto(649.36553476,66.82509408)(649.80553432,67.2050937)(650.3755249,67.45510061)
\curveto(650.49553363,67.5050934)(650.62053351,67.55009336)(650.7505249,67.59010061)
\curveto(650.88053325,67.63009328)(651.01553311,67.67509323)(651.1555249,67.72510061)
\curveto(651.23553289,67.74509316)(651.32053281,67.76009315)(651.4105249,67.77010061)
\lineto(651.6505249,67.83010061)
\curveto(651.76053237,67.86009305)(651.87053226,67.87509303)(651.9805249,67.87510061)
\curveto(652.09053204,67.88509302)(652.20053193,67.90009301)(652.3105249,67.92010061)
\curveto(652.36053177,67.94009297)(652.40553172,67.94509296)(652.4455249,67.93510061)
\curveto(652.48553164,67.93509297)(652.5255316,67.94009297)(652.5655249,67.95010061)
\curveto(652.61553151,67.96009295)(652.67053146,67.96009295)(652.7305249,67.95010061)
\curveto(652.78053135,67.95009296)(652.8305313,67.95509295)(652.8805249,67.96510061)
\lineto(653.0155249,67.96510061)
\curveto(653.07553105,67.98509292)(653.14553098,67.98509292)(653.2255249,67.96510061)
\curveto(653.29553083,67.95509295)(653.36053077,67.96009295)(653.4205249,67.98010061)
\curveto(653.45053068,67.99009292)(653.49053064,67.99509291)(653.5405249,67.99510061)
\lineto(653.6605249,67.99510061)
\lineto(654.1255249,67.99510061)
\moveto(656.4505249,66.45010061)
\curveto(656.130528,66.55009436)(655.76552836,66.6100943)(655.3555249,66.63010061)
\curveto(654.94552918,66.65009426)(654.53552959,66.66009425)(654.1255249,66.66010061)
\curveto(653.69553043,66.66009425)(653.27553085,66.65009426)(652.8655249,66.63010061)
\curveto(652.45553167,66.6100943)(652.07053206,66.56509434)(651.7105249,66.49510061)
\curveto(651.35053278,66.42509448)(651.0305331,66.31509459)(650.7505249,66.16510061)
\curveto(650.46053367,66.02509488)(650.2255339,65.83009508)(650.0455249,65.58010061)
\curveto(649.93553419,65.42009549)(649.85553427,65.24009567)(649.8055249,65.04010061)
\curveto(649.74553438,64.84009607)(649.71553441,64.59509631)(649.7155249,64.30510061)
\curveto(649.73553439,64.28509662)(649.74553438,64.25009666)(649.7455249,64.20010061)
\curveto(649.73553439,64.15009676)(649.73553439,64.1100968)(649.7455249,64.08010061)
\curveto(649.76553436,64.00009691)(649.78553434,63.92509698)(649.8055249,63.85510061)
\curveto(649.81553431,63.79509711)(649.83553429,63.73009718)(649.8655249,63.66010061)
\curveto(649.98553414,63.39009752)(650.15553397,63.17009774)(650.3755249,63.00010061)
\curveto(650.58553354,62.84009807)(650.8305333,62.7050982)(651.1105249,62.59510061)
\curveto(651.22053291,62.54509836)(651.34053279,62.5050984)(651.4705249,62.47510061)
\curveto(651.59053254,62.45509845)(651.71553241,62.43009848)(651.8455249,62.40010061)
\curveto(651.89553223,62.38009853)(651.95053218,62.37009854)(652.0105249,62.37010061)
\curveto(652.06053207,62.37009854)(652.11053202,62.36509854)(652.1605249,62.35510061)
\curveto(652.25053188,62.34509856)(652.34553178,62.33509857)(652.4455249,62.32510061)
\curveto(652.53553159,62.31509859)(652.6305315,62.3050986)(652.7305249,62.29510061)
\curveto(652.81053132,62.29509861)(652.89553123,62.29009862)(652.9855249,62.28010061)
\lineto(653.2255249,62.28010061)
\lineto(653.4055249,62.28010061)
\curveto(653.43553069,62.27009864)(653.47053066,62.26509864)(653.5105249,62.26510061)
\lineto(653.6455249,62.26510061)
\lineto(654.0955249,62.26510061)
\curveto(654.17552995,62.26509864)(654.26052987,62.26009865)(654.3505249,62.25010061)
\curveto(654.4305297,62.25009866)(654.50552962,62.26009865)(654.5755249,62.28010061)
\lineto(654.8455249,62.28010061)
\curveto(654.86552926,62.28009863)(654.89552923,62.27509863)(654.9355249,62.26510061)
\curveto(654.96552916,62.26509864)(654.99052914,62.27009864)(655.0105249,62.28010061)
\curveto(655.11052902,62.29009862)(655.21052892,62.29509861)(655.3105249,62.29510061)
\curveto(655.40052873,62.3050986)(655.50052863,62.31509859)(655.6105249,62.32510061)
\curveto(655.7305284,62.35509855)(655.85552827,62.37009854)(655.9855249,62.37010061)
\curveto(656.10552802,62.38009853)(656.22052791,62.4050985)(656.3305249,62.44510061)
\curveto(656.6305275,62.52509838)(656.89552723,62.6100983)(657.1255249,62.70010061)
\curveto(657.35552677,62.80009811)(657.57052656,62.94509796)(657.7705249,63.13510061)
\curveto(657.97052616,63.34509756)(658.12052601,63.6100973)(658.2205249,63.93010061)
\curveto(658.24052589,63.97009694)(658.25052588,64.0050969)(658.2505249,64.03510061)
\curveto(658.24052589,64.07509683)(658.24552588,64.12009679)(658.2655249,64.17010061)
\curveto(658.27552585,64.2100967)(658.28552584,64.28009663)(658.2955249,64.38010061)
\curveto(658.30552582,64.49009642)(658.30052583,64.57509633)(658.2805249,64.63510061)
\curveto(658.26052587,64.7050962)(658.25052588,64.77509613)(658.2505249,64.84510061)
\curveto(658.24052589,64.91509599)(658.2255259,64.98009593)(658.2055249,65.04010061)
\curveto(658.14552598,65.24009567)(658.06052607,65.42009549)(657.9505249,65.58010061)
\curveto(657.9305262,65.6100953)(657.91052622,65.63509527)(657.8905249,65.65510061)
\lineto(657.8305249,65.71510061)
\curveto(657.81052632,65.75509515)(657.77052636,65.8050951)(657.7105249,65.86510061)
\curveto(657.57052656,65.96509494)(657.44052669,66.05009486)(657.3205249,66.12010061)
\curveto(657.20052693,66.19009472)(657.05552707,66.26009465)(656.8855249,66.33010061)
\curveto(656.81552731,66.36009455)(656.74552738,66.38009453)(656.6755249,66.39010061)
\curveto(656.60552752,66.4100945)(656.5305276,66.43009448)(656.4505249,66.45010061)
}
}
{
\newrgbcolor{curcolor}{0 0 0}
\pscustom[linestyle=none,fillstyle=solid,fillcolor=curcolor]
{
\newpath
\moveto(648.6055249,72.59470998)
\curveto(648.59553553,73.28470535)(648.71553541,73.88470475)(648.9655249,74.39470998)
\curveto(649.21553491,74.91470372)(649.55053458,75.30970332)(649.9705249,75.57970998)
\curveto(650.05053408,75.629703)(650.14053399,75.67470296)(650.2405249,75.71470998)
\curveto(650.3305338,75.75470288)(650.4255337,75.79970283)(650.5255249,75.84970998)
\curveto(650.6255335,75.88970274)(650.7255334,75.91970271)(650.8255249,75.93970998)
\curveto(650.9255332,75.95970267)(651.0305331,75.97970265)(651.1405249,75.99970998)
\curveto(651.19053294,76.01970261)(651.23553289,76.02470261)(651.2755249,76.01470998)
\curveto(651.31553281,76.00470263)(651.36053277,76.00970262)(651.4105249,76.02970998)
\curveto(651.46053267,76.03970259)(651.54553258,76.04470259)(651.6655249,76.04470998)
\curveto(651.77553235,76.04470259)(651.86053227,76.03970259)(651.9205249,76.02970998)
\curveto(651.98053215,76.00970262)(652.04053209,75.99970263)(652.1005249,75.99970998)
\curveto(652.16053197,76.00970262)(652.22053191,76.00470263)(652.2805249,75.98470998)
\curveto(652.42053171,75.94470269)(652.55553157,75.90970272)(652.6855249,75.87970998)
\curveto(652.81553131,75.84970278)(652.94053119,75.80970282)(653.0605249,75.75970998)
\curveto(653.20053093,75.69970293)(653.3255308,75.629703)(653.4355249,75.54970998)
\curveto(653.54553058,75.47970315)(653.65553047,75.40470323)(653.7655249,75.32470998)
\lineto(653.8255249,75.26470998)
\curveto(653.84553028,75.25470338)(653.86553026,75.23970339)(653.8855249,75.21970998)
\curveto(654.04553008,75.09970353)(654.19052994,74.96470367)(654.3205249,74.81470998)
\curveto(654.45052968,74.66470397)(654.57552955,74.50470413)(654.6955249,74.33470998)
\curveto(654.91552921,74.02470461)(655.12052901,73.7297049)(655.3105249,73.44970998)
\curveto(655.45052868,73.21970541)(655.58552854,72.98970564)(655.7155249,72.75970998)
\curveto(655.84552828,72.53970609)(655.98052815,72.31970631)(656.1205249,72.09970998)
\curveto(656.29052784,71.84970678)(656.47052766,71.60970702)(656.6605249,71.37970998)
\curveto(656.85052728,71.15970747)(657.07552705,70.96970766)(657.3355249,70.80970998)
\curveto(657.39552673,70.76970786)(657.45552667,70.7347079)(657.5155249,70.70470998)
\curveto(657.56552656,70.67470796)(657.6305265,70.64470799)(657.7105249,70.61470998)
\curveto(657.78052635,70.59470804)(657.84052629,70.58970804)(657.8905249,70.59970998)
\curveto(657.96052617,70.61970801)(658.01552611,70.65470798)(658.0555249,70.70470998)
\curveto(658.08552604,70.75470788)(658.10552602,70.81470782)(658.1155249,70.88470998)
\lineto(658.1155249,71.12470998)
\lineto(658.1155249,71.87470998)
\lineto(658.1155249,74.67970998)
\lineto(658.1155249,75.33970998)
\curveto(658.11552601,75.4297032)(658.12052601,75.51470312)(658.1305249,75.59470998)
\curveto(658.130526,75.67470296)(658.15052598,75.73970289)(658.1905249,75.78970998)
\curveto(658.2305259,75.83970279)(658.30552582,75.87970275)(658.4155249,75.90970998)
\curveto(658.51552561,75.94970268)(658.61552551,75.95970267)(658.7155249,75.93970998)
\lineto(658.8505249,75.93970998)
\curveto(658.92052521,75.91970271)(658.98052515,75.89970273)(659.0305249,75.87970998)
\curveto(659.08052505,75.85970277)(659.12052501,75.82470281)(659.1505249,75.77470998)
\curveto(659.19052494,75.72470291)(659.21052492,75.65470298)(659.2105249,75.56470998)
\lineto(659.2105249,75.29470998)
\lineto(659.2105249,74.39470998)
\lineto(659.2105249,70.88470998)
\lineto(659.2105249,69.81970998)
\curveto(659.21052492,69.73970889)(659.21552491,69.64970898)(659.2255249,69.54970998)
\curveto(659.2255249,69.44970918)(659.21552491,69.36470927)(659.1955249,69.29470998)
\curveto(659.125525,69.08470955)(658.94552518,69.01970961)(658.6555249,69.09970998)
\curveto(658.61552551,69.10970952)(658.58052555,69.10970952)(658.5505249,69.09970998)
\curveto(658.51052562,69.09970953)(658.46552566,69.10970952)(658.4155249,69.12970998)
\curveto(658.33552579,69.14970948)(658.25052588,69.16970946)(658.1605249,69.18970998)
\curveto(658.07052606,69.20970942)(657.98552614,69.2347094)(657.9055249,69.26470998)
\curveto(657.41552671,69.42470921)(657.00052713,69.62470901)(656.6605249,69.86470998)
\curveto(656.41052772,70.04470859)(656.18552794,70.24970838)(655.9855249,70.47970998)
\curveto(655.77552835,70.70970792)(655.58052855,70.94970768)(655.4005249,71.19970998)
\curveto(655.22052891,71.45970717)(655.05052908,71.72470691)(654.8905249,71.99470998)
\curveto(654.72052941,72.27470636)(654.54552958,72.54470609)(654.3655249,72.80470998)
\curveto(654.28552984,72.91470572)(654.21052992,73.01970561)(654.1405249,73.11970998)
\curveto(654.07053006,73.2297054)(653.99553013,73.33970529)(653.9155249,73.44970998)
\curveto(653.88553024,73.48970514)(653.85553027,73.52470511)(653.8255249,73.55470998)
\curveto(653.78553034,73.59470504)(653.75553037,73.634705)(653.7355249,73.67470998)
\curveto(653.6255305,73.81470482)(653.50053063,73.93970469)(653.3605249,74.04970998)
\curveto(653.3305308,74.06970456)(653.30553082,74.09470454)(653.2855249,74.12470998)
\curveto(653.25553087,74.15470448)(653.2255309,74.17970445)(653.1955249,74.19970998)
\curveto(653.09553103,74.27970435)(652.99553113,74.34470429)(652.8955249,74.39470998)
\curveto(652.79553133,74.45470418)(652.68553144,74.50970412)(652.5655249,74.55970998)
\curveto(652.49553163,74.58970404)(652.42053171,74.60970402)(652.3405249,74.61970998)
\lineto(652.1005249,74.67970998)
\lineto(652.0105249,74.67970998)
\curveto(651.98053215,74.68970394)(651.95053218,74.69470394)(651.9205249,74.69470998)
\curveto(651.85053228,74.71470392)(651.75553237,74.71970391)(651.6355249,74.70970998)
\curveto(651.50553262,74.70970392)(651.40553272,74.69970393)(651.3355249,74.67970998)
\curveto(651.25553287,74.65970397)(651.18053295,74.63970399)(651.1105249,74.61970998)
\curveto(651.0305331,74.60970402)(650.95053318,74.58970404)(650.8705249,74.55970998)
\curveto(650.6305335,74.44970418)(650.4305337,74.29970433)(650.2705249,74.10970998)
\curveto(650.10053403,73.9297047)(649.96053417,73.70970492)(649.8505249,73.44970998)
\curveto(649.8305343,73.37970525)(649.81553431,73.30970532)(649.8055249,73.23970998)
\curveto(649.78553434,73.16970546)(649.76553436,73.09470554)(649.7455249,73.01470998)
\curveto(649.7255344,72.9347057)(649.71553441,72.82470581)(649.7155249,72.68470998)
\curveto(649.71553441,72.55470608)(649.7255344,72.44970618)(649.7455249,72.36970998)
\curveto(649.75553437,72.30970632)(649.76053437,72.25470638)(649.7605249,72.20470998)
\curveto(649.76053437,72.15470648)(649.77053436,72.10470653)(649.7905249,72.05470998)
\curveto(649.8305343,71.95470668)(649.87053426,71.85970677)(649.9105249,71.76970998)
\curveto(649.95053418,71.68970694)(649.99553413,71.60970702)(650.0455249,71.52970998)
\curveto(650.06553406,71.49970713)(650.09053404,71.46970716)(650.1205249,71.43970998)
\curveto(650.15053398,71.41970721)(650.17553395,71.39470724)(650.1955249,71.36470998)
\lineto(650.2705249,71.28970998)
\curveto(650.29053384,71.25970737)(650.31053382,71.2347074)(650.3305249,71.21470998)
\lineto(650.5405249,71.06470998)
\curveto(650.60053353,71.02470761)(650.66553346,70.97970765)(650.7355249,70.92970998)
\curveto(650.8255333,70.86970776)(650.9305332,70.81970781)(651.0505249,70.77970998)
\curveto(651.16053297,70.74970788)(651.27053286,70.71470792)(651.3805249,70.67470998)
\curveto(651.49053264,70.634708)(651.63553249,70.60970802)(651.8155249,70.59970998)
\curveto(651.98553214,70.58970804)(652.11053202,70.55970807)(652.1905249,70.50970998)
\curveto(652.27053186,70.45970817)(652.31553181,70.38470825)(652.3255249,70.28470998)
\curveto(652.33553179,70.18470845)(652.34053179,70.07470856)(652.3405249,69.95470998)
\curveto(652.34053179,69.91470872)(652.34553178,69.87470876)(652.3555249,69.83470998)
\curveto(652.35553177,69.79470884)(652.35053178,69.75970887)(652.3405249,69.72970998)
\curveto(652.32053181,69.67970895)(652.31053182,69.629709)(652.3105249,69.57970998)
\curveto(652.31053182,69.53970909)(652.30053183,69.49970913)(652.2805249,69.45970998)
\curveto(652.22053191,69.36970926)(652.08553204,69.32470931)(651.8755249,69.32470998)
\lineto(651.7555249,69.32470998)
\curveto(651.69553243,69.3347093)(651.63553249,69.33970929)(651.5755249,69.33970998)
\curveto(651.50553262,69.34970928)(651.44053269,69.35970927)(651.3805249,69.36970998)
\curveto(651.27053286,69.38970924)(651.17053296,69.40970922)(651.0805249,69.42970998)
\curveto(650.98053315,69.44970918)(650.88553324,69.47970915)(650.7955249,69.51970998)
\curveto(650.7255334,69.53970909)(650.66553346,69.55970907)(650.6155249,69.57970998)
\lineto(650.4355249,69.63970998)
\curveto(650.17553395,69.75970887)(649.9305342,69.91470872)(649.7005249,70.10470998)
\curveto(649.47053466,70.30470833)(649.28553484,70.51970811)(649.1455249,70.74970998)
\curveto(649.06553506,70.85970777)(649.00053513,70.97470766)(648.9505249,71.09470998)
\lineto(648.8005249,71.48470998)
\curveto(648.75053538,71.59470704)(648.72053541,71.70970692)(648.7105249,71.82970998)
\curveto(648.69053544,71.94970668)(648.66553546,72.07470656)(648.6355249,72.20470998)
\curveto(648.63553549,72.27470636)(648.63553549,72.33970629)(648.6355249,72.39970998)
\curveto(648.6255355,72.45970617)(648.61553551,72.52470611)(648.6055249,72.59470998)
}
}
{
\newrgbcolor{curcolor}{0 0 0}
\pscustom[linestyle=none,fillstyle=solid,fillcolor=curcolor]
{
\newpath
\moveto(657.5755249,78.63431936)
\lineto(657.5755249,79.26431936)
\lineto(657.5755249,79.45931936)
\curveto(657.57552655,79.52931683)(657.58552654,79.58931677)(657.6055249,79.63931936)
\curveto(657.64552648,79.70931665)(657.68552644,79.7593166)(657.7255249,79.78931936)
\curveto(657.77552635,79.82931653)(657.84052629,79.84931651)(657.9205249,79.84931936)
\curveto(658.00052613,79.8593165)(658.08552604,79.86431649)(658.1755249,79.86431936)
\lineto(658.8955249,79.86431936)
\curveto(659.37552475,79.86431649)(659.78552434,79.80431655)(660.1255249,79.68431936)
\curveto(660.46552366,79.56431679)(660.74052339,79.36931699)(660.9505249,79.09931936)
\curveto(661.00052313,79.02931733)(661.04552308,78.9593174)(661.0855249,78.88931936)
\curveto(661.13552299,78.82931753)(661.18052295,78.7543176)(661.2205249,78.66431936)
\curveto(661.2305229,78.64431771)(661.24052289,78.61431774)(661.2505249,78.57431936)
\curveto(661.27052286,78.53431782)(661.27552285,78.48931787)(661.2655249,78.43931936)
\curveto(661.23552289,78.34931801)(661.16052297,78.29431806)(661.0405249,78.27431936)
\curveto(660.9305232,78.2543181)(660.83552329,78.26931809)(660.7555249,78.31931936)
\curveto(660.68552344,78.34931801)(660.62052351,78.39431796)(660.5605249,78.45431936)
\curveto(660.51052362,78.52431783)(660.46052367,78.58931777)(660.4105249,78.64931936)
\curveto(660.36052377,78.71931764)(660.28552384,78.77931758)(660.1855249,78.82931936)
\curveto(660.09552403,78.88931747)(660.00552412,78.93931742)(659.9155249,78.97931936)
\curveto(659.88552424,78.99931736)(659.8255243,79.02431733)(659.7355249,79.05431936)
\curveto(659.65552447,79.08431727)(659.58552454,79.08931727)(659.5255249,79.06931936)
\curveto(659.38552474,79.03931732)(659.29552483,78.97931738)(659.2555249,78.88931936)
\curveto(659.2255249,78.80931755)(659.21052492,78.71931764)(659.2105249,78.61931936)
\curveto(659.21052492,78.51931784)(659.18552494,78.43431792)(659.1355249,78.36431936)
\curveto(659.06552506,78.27431808)(658.9255252,78.22931813)(658.7155249,78.22931936)
\lineto(658.1605249,78.22931936)
\lineto(657.9355249,78.22931936)
\curveto(657.85552627,78.23931812)(657.79052634,78.2593181)(657.7405249,78.28931936)
\curveto(657.66052647,78.34931801)(657.61552651,78.41931794)(657.6055249,78.49931936)
\curveto(657.59552653,78.51931784)(657.59052654,78.53931782)(657.5905249,78.55931936)
\curveto(657.59052654,78.58931777)(657.58552654,78.61431774)(657.5755249,78.63431936)
}
}
{
\newrgbcolor{curcolor}{0 0 0}
\pscustom[linestyle=none,fillstyle=solid,fillcolor=curcolor]
{
}
}
{
\newrgbcolor{curcolor}{0 0 0}
\pscustom[linestyle=none,fillstyle=solid,fillcolor=curcolor]
{
\newpath
\moveto(648.6055249,89.26463186)
\curveto(648.59553553,89.95462722)(648.71553541,90.55462662)(648.9655249,91.06463186)
\curveto(649.21553491,91.58462559)(649.55053458,91.9796252)(649.9705249,92.24963186)
\curveto(650.05053408,92.29962488)(650.14053399,92.34462483)(650.2405249,92.38463186)
\curveto(650.3305338,92.42462475)(650.4255337,92.46962471)(650.5255249,92.51963186)
\curveto(650.6255335,92.55962462)(650.7255334,92.58962459)(650.8255249,92.60963186)
\curveto(650.9255332,92.62962455)(651.0305331,92.64962453)(651.1405249,92.66963186)
\curveto(651.19053294,92.68962449)(651.23553289,92.69462448)(651.2755249,92.68463186)
\curveto(651.31553281,92.6746245)(651.36053277,92.6796245)(651.4105249,92.69963186)
\curveto(651.46053267,92.70962447)(651.54553258,92.71462446)(651.6655249,92.71463186)
\curveto(651.77553235,92.71462446)(651.86053227,92.70962447)(651.9205249,92.69963186)
\curveto(651.98053215,92.6796245)(652.04053209,92.66962451)(652.1005249,92.66963186)
\curveto(652.16053197,92.6796245)(652.22053191,92.6746245)(652.2805249,92.65463186)
\curveto(652.42053171,92.61462456)(652.55553157,92.5796246)(652.6855249,92.54963186)
\curveto(652.81553131,92.51962466)(652.94053119,92.4796247)(653.0605249,92.42963186)
\curveto(653.20053093,92.36962481)(653.3255308,92.29962488)(653.4355249,92.21963186)
\curveto(653.54553058,92.14962503)(653.65553047,92.0746251)(653.7655249,91.99463186)
\lineto(653.8255249,91.93463186)
\curveto(653.84553028,91.92462525)(653.86553026,91.90962527)(653.8855249,91.88963186)
\curveto(654.04553008,91.76962541)(654.19052994,91.63462554)(654.3205249,91.48463186)
\curveto(654.45052968,91.33462584)(654.57552955,91.174626)(654.6955249,91.00463186)
\curveto(654.91552921,90.69462648)(655.12052901,90.39962678)(655.3105249,90.11963186)
\curveto(655.45052868,89.88962729)(655.58552854,89.65962752)(655.7155249,89.42963186)
\curveto(655.84552828,89.20962797)(655.98052815,88.98962819)(656.1205249,88.76963186)
\curveto(656.29052784,88.51962866)(656.47052766,88.2796289)(656.6605249,88.04963186)
\curveto(656.85052728,87.82962935)(657.07552705,87.63962954)(657.3355249,87.47963186)
\curveto(657.39552673,87.43962974)(657.45552667,87.40462977)(657.5155249,87.37463186)
\curveto(657.56552656,87.34462983)(657.6305265,87.31462986)(657.7105249,87.28463186)
\curveto(657.78052635,87.26462991)(657.84052629,87.25962992)(657.8905249,87.26963186)
\curveto(657.96052617,87.28962989)(658.01552611,87.32462985)(658.0555249,87.37463186)
\curveto(658.08552604,87.42462975)(658.10552602,87.48462969)(658.1155249,87.55463186)
\lineto(658.1155249,87.79463186)
\lineto(658.1155249,88.54463186)
\lineto(658.1155249,91.34963186)
\lineto(658.1155249,92.00963186)
\curveto(658.11552601,92.09962508)(658.12052601,92.18462499)(658.1305249,92.26463186)
\curveto(658.130526,92.34462483)(658.15052598,92.40962477)(658.1905249,92.45963186)
\curveto(658.2305259,92.50962467)(658.30552582,92.54962463)(658.4155249,92.57963186)
\curveto(658.51552561,92.61962456)(658.61552551,92.62962455)(658.7155249,92.60963186)
\lineto(658.8505249,92.60963186)
\curveto(658.92052521,92.58962459)(658.98052515,92.56962461)(659.0305249,92.54963186)
\curveto(659.08052505,92.52962465)(659.12052501,92.49462468)(659.1505249,92.44463186)
\curveto(659.19052494,92.39462478)(659.21052492,92.32462485)(659.2105249,92.23463186)
\lineto(659.2105249,91.96463186)
\lineto(659.2105249,91.06463186)
\lineto(659.2105249,87.55463186)
\lineto(659.2105249,86.48963186)
\curveto(659.21052492,86.40963077)(659.21552491,86.31963086)(659.2255249,86.21963186)
\curveto(659.2255249,86.11963106)(659.21552491,86.03463114)(659.1955249,85.96463186)
\curveto(659.125525,85.75463142)(658.94552518,85.68963149)(658.6555249,85.76963186)
\curveto(658.61552551,85.7796314)(658.58052555,85.7796314)(658.5505249,85.76963186)
\curveto(658.51052562,85.76963141)(658.46552566,85.7796314)(658.4155249,85.79963186)
\curveto(658.33552579,85.81963136)(658.25052588,85.83963134)(658.1605249,85.85963186)
\curveto(658.07052606,85.8796313)(657.98552614,85.90463127)(657.9055249,85.93463186)
\curveto(657.41552671,86.09463108)(657.00052713,86.29463088)(656.6605249,86.53463186)
\curveto(656.41052772,86.71463046)(656.18552794,86.91963026)(655.9855249,87.14963186)
\curveto(655.77552835,87.3796298)(655.58052855,87.61962956)(655.4005249,87.86963186)
\curveto(655.22052891,88.12962905)(655.05052908,88.39462878)(654.8905249,88.66463186)
\curveto(654.72052941,88.94462823)(654.54552958,89.21462796)(654.3655249,89.47463186)
\curveto(654.28552984,89.58462759)(654.21052992,89.68962749)(654.1405249,89.78963186)
\curveto(654.07053006,89.89962728)(653.99553013,90.00962717)(653.9155249,90.11963186)
\curveto(653.88553024,90.15962702)(653.85553027,90.19462698)(653.8255249,90.22463186)
\curveto(653.78553034,90.26462691)(653.75553037,90.30462687)(653.7355249,90.34463186)
\curveto(653.6255305,90.48462669)(653.50053063,90.60962657)(653.3605249,90.71963186)
\curveto(653.3305308,90.73962644)(653.30553082,90.76462641)(653.2855249,90.79463186)
\curveto(653.25553087,90.82462635)(653.2255309,90.84962633)(653.1955249,90.86963186)
\curveto(653.09553103,90.94962623)(652.99553113,91.01462616)(652.8955249,91.06463186)
\curveto(652.79553133,91.12462605)(652.68553144,91.179626)(652.5655249,91.22963186)
\curveto(652.49553163,91.25962592)(652.42053171,91.2796259)(652.3405249,91.28963186)
\lineto(652.1005249,91.34963186)
\lineto(652.0105249,91.34963186)
\curveto(651.98053215,91.35962582)(651.95053218,91.36462581)(651.9205249,91.36463186)
\curveto(651.85053228,91.38462579)(651.75553237,91.38962579)(651.6355249,91.37963186)
\curveto(651.50553262,91.3796258)(651.40553272,91.36962581)(651.3355249,91.34963186)
\curveto(651.25553287,91.32962585)(651.18053295,91.30962587)(651.1105249,91.28963186)
\curveto(651.0305331,91.2796259)(650.95053318,91.25962592)(650.8705249,91.22963186)
\curveto(650.6305335,91.11962606)(650.4305337,90.96962621)(650.2705249,90.77963186)
\curveto(650.10053403,90.59962658)(649.96053417,90.3796268)(649.8505249,90.11963186)
\curveto(649.8305343,90.04962713)(649.81553431,89.9796272)(649.8055249,89.90963186)
\curveto(649.78553434,89.83962734)(649.76553436,89.76462741)(649.7455249,89.68463186)
\curveto(649.7255344,89.60462757)(649.71553441,89.49462768)(649.7155249,89.35463186)
\curveto(649.71553441,89.22462795)(649.7255344,89.11962806)(649.7455249,89.03963186)
\curveto(649.75553437,88.9796282)(649.76053437,88.92462825)(649.7605249,88.87463186)
\curveto(649.76053437,88.82462835)(649.77053436,88.7746284)(649.7905249,88.72463186)
\curveto(649.8305343,88.62462855)(649.87053426,88.52962865)(649.9105249,88.43963186)
\curveto(649.95053418,88.35962882)(649.99553413,88.2796289)(650.0455249,88.19963186)
\curveto(650.06553406,88.16962901)(650.09053404,88.13962904)(650.1205249,88.10963186)
\curveto(650.15053398,88.08962909)(650.17553395,88.06462911)(650.1955249,88.03463186)
\lineto(650.2705249,87.95963186)
\curveto(650.29053384,87.92962925)(650.31053382,87.90462927)(650.3305249,87.88463186)
\lineto(650.5405249,87.73463186)
\curveto(650.60053353,87.69462948)(650.66553346,87.64962953)(650.7355249,87.59963186)
\curveto(650.8255333,87.53962964)(650.9305332,87.48962969)(651.0505249,87.44963186)
\curveto(651.16053297,87.41962976)(651.27053286,87.38462979)(651.3805249,87.34463186)
\curveto(651.49053264,87.30462987)(651.63553249,87.2796299)(651.8155249,87.26963186)
\curveto(651.98553214,87.25962992)(652.11053202,87.22962995)(652.1905249,87.17963186)
\curveto(652.27053186,87.12963005)(652.31553181,87.05463012)(652.3255249,86.95463186)
\curveto(652.33553179,86.85463032)(652.34053179,86.74463043)(652.3405249,86.62463186)
\curveto(652.34053179,86.58463059)(652.34553178,86.54463063)(652.3555249,86.50463186)
\curveto(652.35553177,86.46463071)(652.35053178,86.42963075)(652.3405249,86.39963186)
\curveto(652.32053181,86.34963083)(652.31053182,86.29963088)(652.3105249,86.24963186)
\curveto(652.31053182,86.20963097)(652.30053183,86.16963101)(652.2805249,86.12963186)
\curveto(652.22053191,86.03963114)(652.08553204,85.99463118)(651.8755249,85.99463186)
\lineto(651.7555249,85.99463186)
\curveto(651.69553243,86.00463117)(651.63553249,86.00963117)(651.5755249,86.00963186)
\curveto(651.50553262,86.01963116)(651.44053269,86.02963115)(651.3805249,86.03963186)
\curveto(651.27053286,86.05963112)(651.17053296,86.0796311)(651.0805249,86.09963186)
\curveto(650.98053315,86.11963106)(650.88553324,86.14963103)(650.7955249,86.18963186)
\curveto(650.7255334,86.20963097)(650.66553346,86.22963095)(650.6155249,86.24963186)
\lineto(650.4355249,86.30963186)
\curveto(650.17553395,86.42963075)(649.9305342,86.58463059)(649.7005249,86.77463186)
\curveto(649.47053466,86.9746302)(649.28553484,87.18962999)(649.1455249,87.41963186)
\curveto(649.06553506,87.52962965)(649.00053513,87.64462953)(648.9505249,87.76463186)
\lineto(648.8005249,88.15463186)
\curveto(648.75053538,88.26462891)(648.72053541,88.3796288)(648.7105249,88.49963186)
\curveto(648.69053544,88.61962856)(648.66553546,88.74462843)(648.6355249,88.87463186)
\curveto(648.63553549,88.94462823)(648.63553549,89.00962817)(648.6355249,89.06963186)
\curveto(648.6255355,89.12962805)(648.61553551,89.19462798)(648.6055249,89.26463186)
}
}
{
\newrgbcolor{curcolor}{0 0 0}
\pscustom[linestyle=none,fillstyle=solid,fillcolor=curcolor]
{
\newpath
\moveto(654.1255249,101.36424123)
\lineto(654.3805249,101.36424123)
\curveto(654.46052967,101.37423353)(654.53552959,101.36923353)(654.6055249,101.34924123)
\lineto(654.8455249,101.34924123)
\lineto(655.0105249,101.34924123)
\curveto(655.11052902,101.32923357)(655.21552891,101.31923358)(655.3255249,101.31924123)
\curveto(655.4255287,101.31923358)(655.5255286,101.30923359)(655.6255249,101.28924123)
\lineto(655.7755249,101.28924123)
\curveto(655.91552821,101.25923364)(656.05552807,101.23923366)(656.1955249,101.22924123)
\curveto(656.3255278,101.21923368)(656.45552767,101.19423371)(656.5855249,101.15424123)
\curveto(656.66552746,101.13423377)(656.75052738,101.11423379)(656.8405249,101.09424123)
\lineto(657.0805249,101.03424123)
\lineto(657.3805249,100.91424123)
\curveto(657.47052666,100.88423402)(657.56052657,100.84923405)(657.6505249,100.80924123)
\curveto(657.87052626,100.70923419)(658.08552604,100.57423433)(658.2955249,100.40424123)
\curveto(658.50552562,100.24423466)(658.67552545,100.06923483)(658.8055249,99.87924123)
\curveto(658.84552528,99.82923507)(658.88552524,99.76923513)(658.9255249,99.69924123)
\curveto(658.95552517,99.63923526)(658.99052514,99.57923532)(659.0305249,99.51924123)
\curveto(659.08052505,99.43923546)(659.12052501,99.34423556)(659.1505249,99.23424123)
\curveto(659.18052495,99.12423578)(659.21052492,99.01923588)(659.2405249,98.91924123)
\curveto(659.28052485,98.80923609)(659.30552482,98.6992362)(659.3155249,98.58924123)
\curveto(659.3255248,98.47923642)(659.34052479,98.36423654)(659.3605249,98.24424123)
\curveto(659.37052476,98.2042367)(659.37052476,98.15923674)(659.3605249,98.10924123)
\curveto(659.36052477,98.06923683)(659.36552476,98.02923687)(659.3755249,97.98924123)
\curveto(659.38552474,97.94923695)(659.39052474,97.89423701)(659.3905249,97.82424123)
\curveto(659.39052474,97.75423715)(659.38552474,97.7042372)(659.3755249,97.67424123)
\curveto(659.35552477,97.62423728)(659.35052478,97.57923732)(659.3605249,97.53924123)
\curveto(659.37052476,97.4992374)(659.37052476,97.46423744)(659.3605249,97.43424123)
\lineto(659.3605249,97.34424123)
\curveto(659.34052479,97.28423762)(659.3255248,97.21923768)(659.3155249,97.14924123)
\curveto(659.31552481,97.08923781)(659.31052482,97.02423788)(659.3005249,96.95424123)
\curveto(659.25052488,96.78423812)(659.20052493,96.62423828)(659.1505249,96.47424123)
\curveto(659.10052503,96.32423858)(659.03552509,96.17923872)(658.9555249,96.03924123)
\curveto(658.91552521,95.98923891)(658.88552524,95.93423897)(658.8655249,95.87424123)
\curveto(658.83552529,95.82423908)(658.80052533,95.77423913)(658.7605249,95.72424123)
\curveto(658.58052555,95.48423942)(658.36052577,95.28423962)(658.1005249,95.12424123)
\curveto(657.84052629,94.96423994)(657.55552657,94.82424008)(657.2455249,94.70424123)
\curveto(657.10552702,94.64424026)(656.96552716,94.5992403)(656.8255249,94.56924123)
\curveto(656.67552745,94.53924036)(656.52052761,94.5042404)(656.3605249,94.46424123)
\curveto(656.25052788,94.44424046)(656.14052799,94.42924047)(656.0305249,94.41924123)
\curveto(655.92052821,94.40924049)(655.81052832,94.39424051)(655.7005249,94.37424123)
\curveto(655.66052847,94.36424054)(655.62052851,94.35924054)(655.5805249,94.35924123)
\curveto(655.54052859,94.36924053)(655.50052863,94.36924053)(655.4605249,94.35924123)
\curveto(655.41052872,94.34924055)(655.36052877,94.34424056)(655.3105249,94.34424123)
\lineto(655.1455249,94.34424123)
\curveto(655.09552903,94.32424058)(655.04552908,94.31924058)(654.9955249,94.32924123)
\curveto(654.93552919,94.33924056)(654.88052925,94.33924056)(654.8305249,94.32924123)
\curveto(654.79052934,94.31924058)(654.74552938,94.31924058)(654.6955249,94.32924123)
\curveto(654.64552948,94.33924056)(654.59552953,94.33424057)(654.5455249,94.31424123)
\curveto(654.47552965,94.29424061)(654.40052973,94.28924061)(654.3205249,94.29924123)
\curveto(654.2305299,94.30924059)(654.14552998,94.31424059)(654.0655249,94.31424123)
\curveto(653.97553015,94.31424059)(653.87553025,94.30924059)(653.7655249,94.29924123)
\curveto(653.64553048,94.28924061)(653.54553058,94.29424061)(653.4655249,94.31424123)
\lineto(653.1805249,94.31424123)
\lineto(652.5505249,94.35924123)
\curveto(652.45053168,94.36924053)(652.35553177,94.37924052)(652.2655249,94.38924123)
\lineto(651.9655249,94.41924123)
\curveto(651.91553221,94.43924046)(651.86553226,94.44424046)(651.8155249,94.43424123)
\curveto(651.75553237,94.43424047)(651.70053243,94.44424046)(651.6505249,94.46424123)
\curveto(651.48053265,94.51424039)(651.31553281,94.55424035)(651.1555249,94.58424123)
\curveto(650.98553314,94.61424029)(650.8255333,94.66424024)(650.6755249,94.73424123)
\curveto(650.21553391,94.92423998)(649.84053429,95.14423976)(649.5505249,95.39424123)
\curveto(649.26053487,95.65423925)(649.01553511,96.01423889)(648.8155249,96.47424123)
\curveto(648.76553536,96.6042383)(648.7305354,96.73423817)(648.7105249,96.86424123)
\curveto(648.69053544,97.0042379)(648.66553546,97.14423776)(648.6355249,97.28424123)
\curveto(648.6255355,97.35423755)(648.62053551,97.41923748)(648.6205249,97.47924123)
\curveto(648.62053551,97.53923736)(648.61553551,97.6042373)(648.6055249,97.67424123)
\curveto(648.58553554,98.5042364)(648.73553539,99.17423573)(649.0555249,99.68424123)
\curveto(649.36553476,100.19423471)(649.80553432,100.57423433)(650.3755249,100.82424123)
\curveto(650.49553363,100.87423403)(650.62053351,100.91923398)(650.7505249,100.95924123)
\curveto(650.88053325,100.9992339)(651.01553311,101.04423386)(651.1555249,101.09424123)
\curveto(651.23553289,101.11423379)(651.32053281,101.12923377)(651.4105249,101.13924123)
\lineto(651.6505249,101.19924123)
\curveto(651.76053237,101.22923367)(651.87053226,101.24423366)(651.9805249,101.24424123)
\curveto(652.09053204,101.25423365)(652.20053193,101.26923363)(652.3105249,101.28924123)
\curveto(652.36053177,101.30923359)(652.40553172,101.31423359)(652.4455249,101.30424123)
\curveto(652.48553164,101.3042336)(652.5255316,101.30923359)(652.5655249,101.31924123)
\curveto(652.61553151,101.32923357)(652.67053146,101.32923357)(652.7305249,101.31924123)
\curveto(652.78053135,101.31923358)(652.8305313,101.32423358)(652.8805249,101.33424123)
\lineto(653.0155249,101.33424123)
\curveto(653.07553105,101.35423355)(653.14553098,101.35423355)(653.2255249,101.33424123)
\curveto(653.29553083,101.32423358)(653.36053077,101.32923357)(653.4205249,101.34924123)
\curveto(653.45053068,101.35923354)(653.49053064,101.36423354)(653.5405249,101.36424123)
\lineto(653.6605249,101.36424123)
\lineto(654.1255249,101.36424123)
\moveto(656.4505249,99.81924123)
\curveto(656.130528,99.91923498)(655.76552836,99.97923492)(655.3555249,99.99924123)
\curveto(654.94552918,100.01923488)(654.53552959,100.02923487)(654.1255249,100.02924123)
\curveto(653.69553043,100.02923487)(653.27553085,100.01923488)(652.8655249,99.99924123)
\curveto(652.45553167,99.97923492)(652.07053206,99.93423497)(651.7105249,99.86424123)
\curveto(651.35053278,99.79423511)(651.0305331,99.68423522)(650.7505249,99.53424123)
\curveto(650.46053367,99.39423551)(650.2255339,99.1992357)(650.0455249,98.94924123)
\curveto(649.93553419,98.78923611)(649.85553427,98.60923629)(649.8055249,98.40924123)
\curveto(649.74553438,98.20923669)(649.71553441,97.96423694)(649.7155249,97.67424123)
\curveto(649.73553439,97.65423725)(649.74553438,97.61923728)(649.7455249,97.56924123)
\curveto(649.73553439,97.51923738)(649.73553439,97.47923742)(649.7455249,97.44924123)
\curveto(649.76553436,97.36923753)(649.78553434,97.29423761)(649.8055249,97.22424123)
\curveto(649.81553431,97.16423774)(649.83553429,97.0992378)(649.8655249,97.02924123)
\curveto(649.98553414,96.75923814)(650.15553397,96.53923836)(650.3755249,96.36924123)
\curveto(650.58553354,96.20923869)(650.8305333,96.07423883)(651.1105249,95.96424123)
\curveto(651.22053291,95.91423899)(651.34053279,95.87423903)(651.4705249,95.84424123)
\curveto(651.59053254,95.82423908)(651.71553241,95.7992391)(651.8455249,95.76924123)
\curveto(651.89553223,95.74923915)(651.95053218,95.73923916)(652.0105249,95.73924123)
\curveto(652.06053207,95.73923916)(652.11053202,95.73423917)(652.1605249,95.72424123)
\curveto(652.25053188,95.71423919)(652.34553178,95.7042392)(652.4455249,95.69424123)
\curveto(652.53553159,95.68423922)(652.6305315,95.67423923)(652.7305249,95.66424123)
\curveto(652.81053132,95.66423924)(652.89553123,95.65923924)(652.9855249,95.64924123)
\lineto(653.2255249,95.64924123)
\lineto(653.4055249,95.64924123)
\curveto(653.43553069,95.63923926)(653.47053066,95.63423927)(653.5105249,95.63424123)
\lineto(653.6455249,95.63424123)
\lineto(654.0955249,95.63424123)
\curveto(654.17552995,95.63423927)(654.26052987,95.62923927)(654.3505249,95.61924123)
\curveto(654.4305297,95.61923928)(654.50552962,95.62923927)(654.5755249,95.64924123)
\lineto(654.8455249,95.64924123)
\curveto(654.86552926,95.64923925)(654.89552923,95.64423926)(654.9355249,95.63424123)
\curveto(654.96552916,95.63423927)(654.99052914,95.63923926)(655.0105249,95.64924123)
\curveto(655.11052902,95.65923924)(655.21052892,95.66423924)(655.3105249,95.66424123)
\curveto(655.40052873,95.67423923)(655.50052863,95.68423922)(655.6105249,95.69424123)
\curveto(655.7305284,95.72423918)(655.85552827,95.73923916)(655.9855249,95.73924123)
\curveto(656.10552802,95.74923915)(656.22052791,95.77423913)(656.3305249,95.81424123)
\curveto(656.6305275,95.89423901)(656.89552723,95.97923892)(657.1255249,96.06924123)
\curveto(657.35552677,96.16923873)(657.57052656,96.31423859)(657.7705249,96.50424123)
\curveto(657.97052616,96.71423819)(658.12052601,96.97923792)(658.2205249,97.29924123)
\curveto(658.24052589,97.33923756)(658.25052588,97.37423753)(658.2505249,97.40424123)
\curveto(658.24052589,97.44423746)(658.24552588,97.48923741)(658.2655249,97.53924123)
\curveto(658.27552585,97.57923732)(658.28552584,97.64923725)(658.2955249,97.74924123)
\curveto(658.30552582,97.85923704)(658.30052583,97.94423696)(658.2805249,98.00424123)
\curveto(658.26052587,98.07423683)(658.25052588,98.14423676)(658.2505249,98.21424123)
\curveto(658.24052589,98.28423662)(658.2255259,98.34923655)(658.2055249,98.40924123)
\curveto(658.14552598,98.60923629)(658.06052607,98.78923611)(657.9505249,98.94924123)
\curveto(657.9305262,98.97923592)(657.91052622,99.0042359)(657.8905249,99.02424123)
\lineto(657.8305249,99.08424123)
\curveto(657.81052632,99.12423578)(657.77052636,99.17423573)(657.7105249,99.23424123)
\curveto(657.57052656,99.33423557)(657.44052669,99.41923548)(657.3205249,99.48924123)
\curveto(657.20052693,99.55923534)(657.05552707,99.62923527)(656.8855249,99.69924123)
\curveto(656.81552731,99.72923517)(656.74552738,99.74923515)(656.6755249,99.75924123)
\curveto(656.60552752,99.77923512)(656.5305276,99.7992351)(656.4505249,99.81924123)
}
}
{
\newrgbcolor{curcolor}{0 0 0}
\pscustom[linestyle=none,fillstyle=solid,fillcolor=curcolor]
{
\newpath
\moveto(648.6055249,106.77385061)
\curveto(648.60553552,106.87384575)(648.61553551,106.96884566)(648.6355249,107.05885061)
\curveto(648.64553548,107.14884548)(648.67553545,107.21384541)(648.7255249,107.25385061)
\curveto(648.80553532,107.31384531)(648.91053522,107.34384528)(649.0405249,107.34385061)
\lineto(649.4305249,107.34385061)
\lineto(650.9305249,107.34385061)
\lineto(657.3205249,107.34385061)
\lineto(658.4905249,107.34385061)
\lineto(658.8055249,107.34385061)
\curveto(658.90552522,107.35384527)(658.98552514,107.33884529)(659.0455249,107.29885061)
\curveto(659.125525,107.24884538)(659.17552495,107.17384545)(659.1955249,107.07385061)
\curveto(659.20552492,106.98384564)(659.21052492,106.87384575)(659.2105249,106.74385061)
\lineto(659.2105249,106.51885061)
\curveto(659.19052494,106.43884619)(659.17552495,106.36884626)(659.1655249,106.30885061)
\curveto(659.14552498,106.24884638)(659.10552502,106.19884643)(659.0455249,106.15885061)
\curveto(658.98552514,106.11884651)(658.91052522,106.09884653)(658.8205249,106.09885061)
\lineto(658.5205249,106.09885061)
\lineto(657.4255249,106.09885061)
\lineto(652.0855249,106.09885061)
\curveto(651.99553213,106.07884655)(651.92053221,106.06384656)(651.8605249,106.05385061)
\curveto(651.79053234,106.05384657)(651.7305324,106.0238466)(651.6805249,105.96385061)
\curveto(651.6305325,105.89384673)(651.60553252,105.80384682)(651.6055249,105.69385061)
\curveto(651.59553253,105.59384703)(651.59053254,105.48384714)(651.5905249,105.36385061)
\lineto(651.5905249,104.22385061)
\lineto(651.5905249,103.72885061)
\curveto(651.58053255,103.56884906)(651.52053261,103.45884917)(651.4105249,103.39885061)
\curveto(651.38053275,103.37884925)(651.35053278,103.36884926)(651.3205249,103.36885061)
\curveto(651.28053285,103.36884926)(651.23553289,103.36384926)(651.1855249,103.35385061)
\curveto(651.06553306,103.33384929)(650.95553317,103.33884929)(650.8555249,103.36885061)
\curveto(650.75553337,103.40884922)(650.68553344,103.46384916)(650.6455249,103.53385061)
\curveto(650.59553353,103.61384901)(650.57053356,103.73384889)(650.5705249,103.89385061)
\curveto(650.57053356,104.05384857)(650.55553357,104.18884844)(650.5255249,104.29885061)
\curveto(650.51553361,104.34884828)(650.51053362,104.40384822)(650.5105249,104.46385061)
\curveto(650.50053363,104.5238481)(650.48553364,104.58384804)(650.4655249,104.64385061)
\curveto(650.41553371,104.79384783)(650.36553376,104.93884769)(650.3155249,105.07885061)
\curveto(650.25553387,105.21884741)(650.18553394,105.35384727)(650.1055249,105.48385061)
\curveto(650.01553411,105.623847)(649.91053422,105.74384688)(649.7905249,105.84385061)
\curveto(649.67053446,105.94384668)(649.54053459,106.03884659)(649.4005249,106.12885061)
\curveto(649.30053483,106.18884644)(649.19053494,106.23384639)(649.0705249,106.26385061)
\curveto(648.95053518,106.30384632)(648.84553528,106.35384627)(648.7555249,106.41385061)
\curveto(648.69553543,106.46384616)(648.65553547,106.53384609)(648.6355249,106.62385061)
\curveto(648.6255355,106.64384598)(648.62053551,106.66884596)(648.6205249,106.69885061)
\curveto(648.62053551,106.7288459)(648.61553551,106.75384587)(648.6055249,106.77385061)
}
}
{
\newrgbcolor{curcolor}{0 0 0}
\pscustom[linestyle=none,fillstyle=solid,fillcolor=curcolor]
{
\newpath
\moveto(648.6055249,115.12345998)
\curveto(648.60553552,115.22345513)(648.61553551,115.31845503)(648.6355249,115.40845998)
\curveto(648.64553548,115.49845485)(648.67553545,115.56345479)(648.7255249,115.60345998)
\curveto(648.80553532,115.66345469)(648.91053522,115.69345466)(649.0405249,115.69345998)
\lineto(649.4305249,115.69345998)
\lineto(650.9305249,115.69345998)
\lineto(657.3205249,115.69345998)
\lineto(658.4905249,115.69345998)
\lineto(658.8055249,115.69345998)
\curveto(658.90552522,115.70345465)(658.98552514,115.68845466)(659.0455249,115.64845998)
\curveto(659.125525,115.59845475)(659.17552495,115.52345483)(659.1955249,115.42345998)
\curveto(659.20552492,115.33345502)(659.21052492,115.22345513)(659.2105249,115.09345998)
\lineto(659.2105249,114.86845998)
\curveto(659.19052494,114.78845556)(659.17552495,114.71845563)(659.1655249,114.65845998)
\curveto(659.14552498,114.59845575)(659.10552502,114.5484558)(659.0455249,114.50845998)
\curveto(658.98552514,114.46845588)(658.91052522,114.4484559)(658.8205249,114.44845998)
\lineto(658.5205249,114.44845998)
\lineto(657.4255249,114.44845998)
\lineto(652.0855249,114.44845998)
\curveto(651.99553213,114.42845592)(651.92053221,114.41345594)(651.8605249,114.40345998)
\curveto(651.79053234,114.40345595)(651.7305324,114.37345598)(651.6805249,114.31345998)
\curveto(651.6305325,114.24345611)(651.60553252,114.1534562)(651.6055249,114.04345998)
\curveto(651.59553253,113.94345641)(651.59053254,113.83345652)(651.5905249,113.71345998)
\lineto(651.5905249,112.57345998)
\lineto(651.5905249,112.07845998)
\curveto(651.58053255,111.91845843)(651.52053261,111.80845854)(651.4105249,111.74845998)
\curveto(651.38053275,111.72845862)(651.35053278,111.71845863)(651.3205249,111.71845998)
\curveto(651.28053285,111.71845863)(651.23553289,111.71345864)(651.1855249,111.70345998)
\curveto(651.06553306,111.68345867)(650.95553317,111.68845866)(650.8555249,111.71845998)
\curveto(650.75553337,111.75845859)(650.68553344,111.81345854)(650.6455249,111.88345998)
\curveto(650.59553353,111.96345839)(650.57053356,112.08345827)(650.5705249,112.24345998)
\curveto(650.57053356,112.40345795)(650.55553357,112.53845781)(650.5255249,112.64845998)
\curveto(650.51553361,112.69845765)(650.51053362,112.7534576)(650.5105249,112.81345998)
\curveto(650.50053363,112.87345748)(650.48553364,112.93345742)(650.4655249,112.99345998)
\curveto(650.41553371,113.14345721)(650.36553376,113.28845706)(650.3155249,113.42845998)
\curveto(650.25553387,113.56845678)(650.18553394,113.70345665)(650.1055249,113.83345998)
\curveto(650.01553411,113.97345638)(649.91053422,114.09345626)(649.7905249,114.19345998)
\curveto(649.67053446,114.29345606)(649.54053459,114.38845596)(649.4005249,114.47845998)
\curveto(649.30053483,114.53845581)(649.19053494,114.58345577)(649.0705249,114.61345998)
\curveto(648.95053518,114.6534557)(648.84553528,114.70345565)(648.7555249,114.76345998)
\curveto(648.69553543,114.81345554)(648.65553547,114.88345547)(648.6355249,114.97345998)
\curveto(648.6255355,114.99345536)(648.62053551,115.01845533)(648.6205249,115.04845998)
\curveto(648.62053551,115.07845527)(648.61553551,115.10345525)(648.6055249,115.12345998)
}
}
{
\newrgbcolor{curcolor}{0 0 0}
\pscustom[linestyle=none,fillstyle=solid,fillcolor=curcolor]
{
\newpath
\moveto(669.44184082,37.28705373)
\curveto(669.44185152,37.35704805)(669.44185152,37.43704797)(669.44184082,37.52705373)
\curveto(669.43185153,37.61704779)(669.43185153,37.70204771)(669.44184082,37.78205373)
\curveto(669.44185152,37.87204754)(669.45185151,37.95204746)(669.47184082,38.02205373)
\curveto(669.49185147,38.10204731)(669.52185144,38.15704725)(669.56184082,38.18705373)
\curveto(669.61185135,38.21704719)(669.68685127,38.23704717)(669.78684082,38.24705373)
\curveto(669.87685108,38.26704714)(669.98185098,38.27704713)(670.10184082,38.27705373)
\curveto(670.21185075,38.28704712)(670.32685063,38.28704712)(670.44684082,38.27705373)
\lineto(670.74684082,38.27705373)
\lineto(673.76184082,38.27705373)
\lineto(676.65684082,38.27705373)
\curveto(676.98684397,38.27704713)(677.31184365,38.27204714)(677.63184082,38.26205373)
\curveto(677.94184302,38.26204715)(678.22184274,38.22204719)(678.47184082,38.14205373)
\curveto(678.82184214,38.02204739)(679.11684184,37.86704754)(679.35684082,37.67705373)
\curveto(679.58684137,37.48704792)(679.78684117,37.24704816)(679.95684082,36.95705373)
\curveto(680.00684095,36.89704851)(680.04184092,36.83204858)(680.06184082,36.76205373)
\curveto(680.08184088,36.70204871)(680.10684085,36.63204878)(680.13684082,36.55205373)
\curveto(680.18684077,36.43204898)(680.22184074,36.30204911)(680.24184082,36.16205373)
\curveto(680.27184069,36.03204938)(680.30184066,35.89704951)(680.33184082,35.75705373)
\curveto(680.35184061,35.7070497)(680.3568406,35.65704975)(680.34684082,35.60705373)
\curveto(680.33684062,35.55704985)(680.33684062,35.50204991)(680.34684082,35.44205373)
\curveto(680.3568406,35.42204999)(680.3568406,35.39705001)(680.34684082,35.36705373)
\curveto(680.34684061,35.33705007)(680.35184061,35.3120501)(680.36184082,35.29205373)
\curveto(680.37184059,35.25205016)(680.37684058,35.19705021)(680.37684082,35.12705373)
\curveto(680.37684058,35.05705035)(680.37184059,35.00205041)(680.36184082,34.96205373)
\curveto(680.35184061,34.9120505)(680.35184061,34.85705055)(680.36184082,34.79705373)
\curveto(680.37184059,34.73705067)(680.36684059,34.68205073)(680.34684082,34.63205373)
\curveto(680.31684064,34.50205091)(680.29684066,34.37705103)(680.28684082,34.25705373)
\curveto(680.27684068,34.13705127)(680.25184071,34.02205139)(680.21184082,33.91205373)
\curveto(680.09184087,33.54205187)(679.92184104,33.22205219)(679.70184082,32.95205373)
\curveto(679.48184148,32.68205273)(679.20184176,32.47205294)(678.86184082,32.32205373)
\curveto(678.74184222,32.27205314)(678.61684234,32.22705318)(678.48684082,32.18705373)
\curveto(678.3568426,32.15705325)(678.22184274,32.12205329)(678.08184082,32.08205373)
\curveto(678.03184293,32.07205334)(677.99184297,32.06705334)(677.96184082,32.06705373)
\curveto(677.92184304,32.06705334)(677.87684308,32.06205335)(677.82684082,32.05205373)
\curveto(677.79684316,32.04205337)(677.7618432,32.03705337)(677.72184082,32.03705373)
\curveto(677.67184329,32.03705337)(677.63184333,32.03205338)(677.60184082,32.02205373)
\lineto(677.43684082,32.02205373)
\curveto(677.3568436,32.00205341)(677.2568437,31.99705341)(677.13684082,32.00705373)
\curveto(677.00684395,32.01705339)(676.91684404,32.03205338)(676.86684082,32.05205373)
\curveto(676.77684418,32.07205334)(676.71184425,32.12705328)(676.67184082,32.21705373)
\curveto(676.65184431,32.24705316)(676.64684431,32.27705313)(676.65684082,32.30705373)
\curveto(676.6568443,32.33705307)(676.65184431,32.37705303)(676.64184082,32.42705373)
\curveto(676.63184433,32.46705294)(676.62684433,32.5070529)(676.62684082,32.54705373)
\lineto(676.62684082,32.69705373)
\curveto(676.62684433,32.81705259)(676.63184433,32.93705247)(676.64184082,33.05705373)
\curveto(676.64184432,33.18705222)(676.67684428,33.27705213)(676.74684082,33.32705373)
\curveto(676.80684415,33.36705204)(676.86684409,33.38705202)(676.92684082,33.38705373)
\curveto(676.98684397,33.38705202)(677.0568439,33.39705201)(677.13684082,33.41705373)
\curveto(677.16684379,33.42705198)(677.20184376,33.42705198)(677.24184082,33.41705373)
\curveto(677.27184369,33.41705199)(677.29684366,33.42205199)(677.31684082,33.43205373)
\lineto(677.52684082,33.43205373)
\curveto(677.57684338,33.45205196)(677.62684333,33.45705195)(677.67684082,33.44705373)
\curveto(677.71684324,33.44705196)(677.7618432,33.45705195)(677.81184082,33.47705373)
\curveto(677.94184302,33.5070519)(678.06684289,33.53705187)(678.18684082,33.56705373)
\curveto(678.29684266,33.59705181)(678.40184256,33.64205177)(678.50184082,33.70205373)
\curveto(678.79184217,33.87205154)(678.99684196,34.14205127)(679.11684082,34.51205373)
\curveto(679.13684182,34.56205085)(679.15184181,34.6120508)(679.16184082,34.66205373)
\curveto(679.1618418,34.72205069)(679.17184179,34.77705063)(679.19184082,34.82705373)
\lineto(679.19184082,34.90205373)
\curveto(679.20184176,34.97205044)(679.21184175,35.06705034)(679.22184082,35.18705373)
\curveto(679.22184174,35.31705009)(679.21184175,35.41704999)(679.19184082,35.48705373)
\curveto(679.17184179,35.55704985)(679.1568418,35.62704978)(679.14684082,35.69705373)
\curveto(679.12684183,35.77704963)(679.10684185,35.84704956)(679.08684082,35.90705373)
\curveto(678.92684203,36.28704912)(678.65184231,36.56204885)(678.26184082,36.73205373)
\curveto(678.13184283,36.78204863)(677.97684298,36.81704859)(677.79684082,36.83705373)
\curveto(677.61684334,36.86704854)(677.43184353,36.88204853)(677.24184082,36.88205373)
\curveto(677.04184392,36.89204852)(676.84184412,36.89204852)(676.64184082,36.88205373)
\lineto(676.07184082,36.88205373)
\lineto(671.82684082,36.88205373)
\lineto(670.28184082,36.88205373)
\curveto(670.17185079,36.88204853)(670.05185091,36.87704853)(669.92184082,36.86705373)
\curveto(669.79185117,36.85704855)(669.68685127,36.87704853)(669.60684082,36.92705373)
\curveto(669.53685142,36.98704842)(669.48685147,37.06704834)(669.45684082,37.16705373)
\curveto(669.4568515,37.18704822)(669.4568515,37.2070482)(669.45684082,37.22705373)
\curveto(669.4568515,37.24704816)(669.45185151,37.26704814)(669.44184082,37.28705373)
}
}
{
\newrgbcolor{curcolor}{0 0 0}
\pscustom[linestyle=none,fillstyle=solid,fillcolor=curcolor]
{
\newpath
\moveto(672.39684082,40.82072561)
\lineto(672.39684082,41.25572561)
\curveto(672.39684856,41.40572364)(672.43684852,41.51072354)(672.51684082,41.57072561)
\curveto(672.59684836,41.62072343)(672.69684826,41.6457234)(672.81684082,41.64572561)
\curveto(672.93684802,41.65572339)(673.0568479,41.66072339)(673.17684082,41.66072561)
\lineto(674.60184082,41.66072561)
\lineto(676.86684082,41.66072561)
\lineto(677.55684082,41.66072561)
\curveto(677.78684317,41.66072339)(677.98684297,41.68572336)(678.15684082,41.73572561)
\curveto(678.60684235,41.89572315)(678.92184204,42.19572285)(679.10184082,42.63572561)
\curveto(679.19184177,42.85572219)(679.22684173,43.12072193)(679.20684082,43.43072561)
\curveto(679.17684178,43.74072131)(679.12184184,43.99072106)(679.04184082,44.18072561)
\curveto(678.90184206,44.51072054)(678.72684223,44.77072028)(678.51684082,44.96072561)
\curveto(678.29684266,45.16071989)(678.01184295,45.31571973)(677.66184082,45.42572561)
\curveto(677.58184338,45.45571959)(677.50184346,45.47571957)(677.42184082,45.48572561)
\curveto(677.34184362,45.49571955)(677.2568437,45.51071954)(677.16684082,45.53072561)
\curveto(677.11684384,45.54071951)(677.07184389,45.54071951)(677.03184082,45.53072561)
\curveto(676.99184397,45.53071952)(676.94684401,45.54071951)(676.89684082,45.56072561)
\lineto(676.58184082,45.56072561)
\curveto(676.50184446,45.58071947)(676.41184455,45.58571946)(676.31184082,45.57572561)
\curveto(676.20184476,45.56571948)(676.10184486,45.56071949)(676.01184082,45.56072561)
\lineto(674.84184082,45.56072561)
\lineto(673.25184082,45.56072561)
\curveto(673.13184783,45.56071949)(673.00684795,45.55571949)(672.87684082,45.54572561)
\curveto(672.73684822,45.5457195)(672.62684833,45.57071948)(672.54684082,45.62072561)
\curveto(672.49684846,45.66071939)(672.46684849,45.70571934)(672.45684082,45.75572561)
\curveto(672.43684852,45.81571923)(672.41684854,45.88571916)(672.39684082,45.96572561)
\lineto(672.39684082,46.19072561)
\curveto(672.39684856,46.31071874)(672.40184856,46.41571863)(672.41184082,46.50572561)
\curveto(672.42184854,46.60571844)(672.46684849,46.68071837)(672.54684082,46.73072561)
\curveto(672.59684836,46.78071827)(672.67184829,46.80571824)(672.77184082,46.80572561)
\lineto(673.05684082,46.80572561)
\lineto(674.07684082,46.80572561)
\lineto(678.11184082,46.80572561)
\lineto(679.46184082,46.80572561)
\curveto(679.58184138,46.80571824)(679.69684126,46.80071825)(679.80684082,46.79072561)
\curveto(679.90684105,46.79071826)(679.98184098,46.75571829)(680.03184082,46.68572561)
\curveto(680.0618409,46.6457184)(680.08684087,46.58571846)(680.10684082,46.50572561)
\curveto(680.11684084,46.42571862)(680.12684083,46.33571871)(680.13684082,46.23572561)
\curveto(680.13684082,46.1457189)(680.13184083,46.05571899)(680.12184082,45.96572561)
\curveto(680.11184085,45.88571916)(680.09184087,45.82571922)(680.06184082,45.78572561)
\curveto(680.02184094,45.73571931)(679.956841,45.69071936)(679.86684082,45.65072561)
\curveto(679.82684113,45.64071941)(679.77184119,45.63071942)(679.70184082,45.62072561)
\curveto(679.63184133,45.62071943)(679.56684139,45.61571943)(679.50684082,45.60572561)
\curveto(679.43684152,45.59571945)(679.38184158,45.57571947)(679.34184082,45.54572561)
\curveto(679.30184166,45.51571953)(679.28684167,45.47071958)(679.29684082,45.41072561)
\curveto(679.31684164,45.33071972)(679.37684158,45.2507198)(679.47684082,45.17072561)
\curveto(679.56684139,45.09071996)(679.63684132,45.01572003)(679.68684082,44.94572561)
\curveto(679.84684111,44.72572032)(679.98684097,44.47572057)(680.10684082,44.19572561)
\curveto(680.1568408,44.08572096)(680.18684077,43.97072108)(680.19684082,43.85072561)
\curveto(680.21684074,43.74072131)(680.24184072,43.62572142)(680.27184082,43.50572561)
\curveto(680.28184068,43.45572159)(680.28184068,43.40072165)(680.27184082,43.34072561)
\curveto(680.2618407,43.29072176)(680.26684069,43.24072181)(680.28684082,43.19072561)
\curveto(680.30684065,43.09072196)(680.30684065,43.00072205)(680.28684082,42.92072561)
\lineto(680.28684082,42.77072561)
\curveto(680.26684069,42.72072233)(680.2568407,42.66072239)(680.25684082,42.59072561)
\curveto(680.2568407,42.53072252)(680.25184071,42.47572257)(680.24184082,42.42572561)
\curveto(680.22184074,42.38572266)(680.21184075,42.3457227)(680.21184082,42.30572561)
\curveto(680.22184074,42.27572277)(680.21684074,42.23572281)(680.19684082,42.18572561)
\lineto(680.13684082,41.94572561)
\curveto(680.11684084,41.87572317)(680.08684087,41.80072325)(680.04684082,41.72072561)
\curveto(679.93684102,41.46072359)(679.79184117,41.24072381)(679.61184082,41.06072561)
\curveto(679.42184154,40.89072416)(679.19684176,40.7507243)(678.93684082,40.64072561)
\curveto(678.84684211,40.60072445)(678.7568422,40.57072448)(678.66684082,40.55072561)
\lineto(678.36684082,40.49072561)
\curveto(678.30684265,40.47072458)(678.25184271,40.46072459)(678.20184082,40.46072561)
\curveto(678.14184282,40.47072458)(678.07684288,40.46572458)(678.00684082,40.44572561)
\curveto(677.98684297,40.43572461)(677.961843,40.43072462)(677.93184082,40.43072561)
\curveto(677.89184307,40.43072462)(677.8568431,40.42572462)(677.82684082,40.41572561)
\lineto(677.67684082,40.41572561)
\curveto(677.63684332,40.40572464)(677.59184337,40.40072465)(677.54184082,40.40072561)
\curveto(677.48184348,40.41072464)(677.42684353,40.41572463)(677.37684082,40.41572561)
\lineto(676.77684082,40.41572561)
\lineto(674.01684082,40.41572561)
\lineto(673.05684082,40.41572561)
\lineto(672.78684082,40.41572561)
\curveto(672.69684826,40.41572463)(672.62184834,40.43572461)(672.56184082,40.47572561)
\curveto(672.49184847,40.51572453)(672.44184852,40.59072446)(672.41184082,40.70072561)
\curveto(672.40184856,40.72072433)(672.40184856,40.74072431)(672.41184082,40.76072561)
\curveto(672.41184855,40.78072427)(672.40684855,40.80072425)(672.39684082,40.82072561)
}
}
{
\newrgbcolor{curcolor}{0 0 0}
\pscustom[linestyle=none,fillstyle=solid,fillcolor=curcolor]
{
\newpath
\moveto(672.24684082,52.39533498)
\curveto(672.22684873,53.02532975)(672.31184865,53.53032924)(672.50184082,53.91033498)
\curveto(672.69184827,54.29032848)(672.97684798,54.59532818)(673.35684082,54.82533498)
\curveto(673.4568475,54.88532789)(673.56684739,54.93032784)(673.68684082,54.96033498)
\curveto(673.79684716,55.00032777)(673.91184705,55.03532774)(674.03184082,55.06533498)
\curveto(674.22184674,55.11532766)(674.42684653,55.14532763)(674.64684082,55.15533498)
\curveto(674.86684609,55.16532761)(675.09184587,55.1703276)(675.32184082,55.17033498)
\lineto(676.92684082,55.17033498)
\lineto(679.26684082,55.17033498)
\curveto(679.43684152,55.1703276)(679.60684135,55.16532761)(679.77684082,55.15533498)
\curveto(679.94684101,55.15532762)(680.0568409,55.09032768)(680.10684082,54.96033498)
\curveto(680.12684083,54.91032786)(680.13684082,54.85532792)(680.13684082,54.79533498)
\curveto(680.14684081,54.74532803)(680.15184081,54.69032808)(680.15184082,54.63033498)
\curveto(680.15184081,54.50032827)(680.14684081,54.3753284)(680.13684082,54.25533498)
\curveto(680.13684082,54.13532864)(680.09684086,54.05032872)(680.01684082,54.00033498)
\curveto(679.94684101,53.95032882)(679.8568411,53.92532885)(679.74684082,53.92533498)
\lineto(679.41684082,53.92533498)
\lineto(678.12684082,53.92533498)
\lineto(675.68184082,53.92533498)
\curveto(675.41184555,53.92532885)(675.14684581,53.92032885)(674.88684082,53.91033498)
\curveto(674.61684634,53.90032887)(674.38684657,53.85532892)(674.19684082,53.77533498)
\curveto(673.99684696,53.69532908)(673.83684712,53.5753292)(673.71684082,53.41533498)
\curveto(673.58684737,53.25532952)(673.48684747,53.0703297)(673.41684082,52.86033498)
\curveto(673.39684756,52.80032997)(673.38684757,52.73533004)(673.38684082,52.66533498)
\curveto(673.37684758,52.60533017)(673.3618476,52.54533023)(673.34184082,52.48533498)
\curveto(673.33184763,52.43533034)(673.33184763,52.35533042)(673.34184082,52.24533498)
\curveto(673.34184762,52.14533063)(673.34684761,52.0753307)(673.35684082,52.03533498)
\curveto(673.37684758,51.99533078)(673.38684757,51.96033081)(673.38684082,51.93033498)
\curveto(673.37684758,51.90033087)(673.37684758,51.86533091)(673.38684082,51.82533498)
\curveto(673.41684754,51.69533108)(673.45184751,51.5703312)(673.49184082,51.45033498)
\curveto(673.52184744,51.34033143)(673.56684739,51.23533154)(673.62684082,51.13533498)
\curveto(673.64684731,51.09533168)(673.66684729,51.06033171)(673.68684082,51.03033498)
\curveto(673.70684725,51.00033177)(673.72684723,50.96533181)(673.74684082,50.92533498)
\curveto(673.99684696,50.5753322)(674.37184659,50.32033245)(674.87184082,50.16033498)
\curveto(674.95184601,50.13033264)(675.03684592,50.11033266)(675.12684082,50.10033498)
\curveto(675.20684575,50.09033268)(675.28684567,50.0753327)(675.36684082,50.05533498)
\curveto(675.41684554,50.03533274)(675.46684549,50.03033274)(675.51684082,50.04033498)
\curveto(675.5568454,50.05033272)(675.59684536,50.04533273)(675.63684082,50.02533498)
\lineto(675.95184082,50.02533498)
\curveto(675.98184498,50.01533276)(676.01684494,50.01033276)(676.05684082,50.01033498)
\curveto(676.09684486,50.02033275)(676.14184482,50.02533275)(676.19184082,50.02533498)
\lineto(676.64184082,50.02533498)
\lineto(678.08184082,50.02533498)
\lineto(679.40184082,50.02533498)
\lineto(679.74684082,50.02533498)
\curveto(679.8568411,50.02533275)(679.94684101,50.00033277)(680.01684082,49.95033498)
\curveto(680.09684086,49.90033287)(680.13684082,49.81033296)(680.13684082,49.68033498)
\curveto(680.14684081,49.56033321)(680.15184081,49.43533334)(680.15184082,49.30533498)
\curveto(680.15184081,49.22533355)(680.14684081,49.15033362)(680.13684082,49.08033498)
\curveto(680.12684083,49.01033376)(680.10184086,48.95033382)(680.06184082,48.90033498)
\curveto(680.01184095,48.82033395)(679.91684104,48.78033399)(679.77684082,48.78033498)
\lineto(679.37184082,48.78033498)
\lineto(677.60184082,48.78033498)
\lineto(673.97184082,48.78033498)
\lineto(673.05684082,48.78033498)
\lineto(672.78684082,48.78033498)
\curveto(672.69684826,48.78033399)(672.62684833,48.80033397)(672.57684082,48.84033498)
\curveto(672.51684844,48.8703339)(672.47684848,48.92033385)(672.45684082,48.99033498)
\curveto(672.44684851,49.03033374)(672.43684852,49.08533369)(672.42684082,49.15533498)
\curveto(672.41684854,49.23533354)(672.41184855,49.31533346)(672.41184082,49.39533498)
\curveto(672.41184855,49.4753333)(672.41684854,49.55033322)(672.42684082,49.62033498)
\curveto(672.43684852,49.70033307)(672.45184851,49.75533302)(672.47184082,49.78533498)
\curveto(672.54184842,49.89533288)(672.63184833,49.94533283)(672.74184082,49.93533498)
\curveto(672.84184812,49.92533285)(672.956848,49.94033283)(673.08684082,49.98033498)
\curveto(673.14684781,50.00033277)(673.19684776,50.04033273)(673.23684082,50.10033498)
\curveto(673.24684771,50.22033255)(673.20184776,50.31533246)(673.10184082,50.38533498)
\curveto(673.00184796,50.46533231)(672.92184804,50.54533223)(672.86184082,50.62533498)
\curveto(672.7618482,50.76533201)(672.67184829,50.90533187)(672.59184082,51.04533498)
\curveto(672.50184846,51.19533158)(672.42684853,51.36533141)(672.36684082,51.55533498)
\curveto(672.33684862,51.63533114)(672.31684864,51.72033105)(672.30684082,51.81033498)
\curveto(672.29684866,51.91033086)(672.28184868,52.00533077)(672.26184082,52.09533498)
\curveto(672.25184871,52.14533063)(672.24684871,52.19533058)(672.24684082,52.24533498)
\lineto(672.24684082,52.39533498)
}
}
{
\newrgbcolor{curcolor}{0 0 0}
\pscustom[linestyle=none,fillstyle=solid,fillcolor=curcolor]
{
}
}
{
\newrgbcolor{curcolor}{0 0 0}
\pscustom[linestyle=none,fillstyle=solid,fillcolor=curcolor]
{
\newpath
\moveto(675.03684082,67.99510061)
\lineto(675.29184082,67.99510061)
\curveto(675.37184559,68.0050929)(675.44684551,68.00009291)(675.51684082,67.98010061)
\lineto(675.75684082,67.98010061)
\lineto(675.92184082,67.98010061)
\curveto(676.02184494,67.96009295)(676.12684483,67.95009296)(676.23684082,67.95010061)
\curveto(676.33684462,67.95009296)(676.43684452,67.94009297)(676.53684082,67.92010061)
\lineto(676.68684082,67.92010061)
\curveto(676.82684413,67.89009302)(676.96684399,67.87009304)(677.10684082,67.86010061)
\curveto(677.23684372,67.85009306)(677.36684359,67.82509308)(677.49684082,67.78510061)
\curveto(677.57684338,67.76509314)(677.6618433,67.74509316)(677.75184082,67.72510061)
\lineto(677.99184082,67.66510061)
\lineto(678.29184082,67.54510061)
\curveto(678.38184258,67.51509339)(678.47184249,67.48009343)(678.56184082,67.44010061)
\curveto(678.78184218,67.34009357)(678.99684196,67.2050937)(679.20684082,67.03510061)
\curveto(679.41684154,66.87509403)(679.58684137,66.70009421)(679.71684082,66.51010061)
\curveto(679.7568412,66.46009445)(679.79684116,66.40009451)(679.83684082,66.33010061)
\curveto(679.86684109,66.27009464)(679.90184106,66.2100947)(679.94184082,66.15010061)
\curveto(679.99184097,66.07009484)(680.03184093,65.97509493)(680.06184082,65.86510061)
\curveto(680.09184087,65.75509515)(680.12184084,65.65009526)(680.15184082,65.55010061)
\curveto(680.19184077,65.44009547)(680.21684074,65.33009558)(680.22684082,65.22010061)
\curveto(680.23684072,65.1100958)(680.25184071,64.99509591)(680.27184082,64.87510061)
\curveto(680.28184068,64.83509607)(680.28184068,64.79009612)(680.27184082,64.74010061)
\curveto(680.27184069,64.70009621)(680.27684068,64.66009625)(680.28684082,64.62010061)
\curveto(680.29684066,64.58009633)(680.30184066,64.52509638)(680.30184082,64.45510061)
\curveto(680.30184066,64.38509652)(680.29684066,64.33509657)(680.28684082,64.30510061)
\curveto(680.26684069,64.25509665)(680.2618407,64.2100967)(680.27184082,64.17010061)
\curveto(680.28184068,64.13009678)(680.28184068,64.09509681)(680.27184082,64.06510061)
\lineto(680.27184082,63.97510061)
\curveto(680.25184071,63.91509699)(680.23684072,63.85009706)(680.22684082,63.78010061)
\curveto(680.22684073,63.72009719)(680.22184074,63.65509725)(680.21184082,63.58510061)
\curveto(680.1618408,63.41509749)(680.11184085,63.25509765)(680.06184082,63.10510061)
\curveto(680.01184095,62.95509795)(679.94684101,62.8100981)(679.86684082,62.67010061)
\curveto(679.82684113,62.62009829)(679.79684116,62.56509834)(679.77684082,62.50510061)
\curveto(679.74684121,62.45509845)(679.71184125,62.4050985)(679.67184082,62.35510061)
\curveto(679.49184147,62.11509879)(679.27184169,61.91509899)(679.01184082,61.75510061)
\curveto(678.75184221,61.59509931)(678.46684249,61.45509945)(678.15684082,61.33510061)
\curveto(678.01684294,61.27509963)(677.87684308,61.23009968)(677.73684082,61.20010061)
\curveto(677.58684337,61.17009974)(677.43184353,61.13509977)(677.27184082,61.09510061)
\curveto(677.1618438,61.07509983)(677.05184391,61.06009985)(676.94184082,61.05010061)
\curveto(676.83184413,61.04009987)(676.72184424,61.02509988)(676.61184082,61.00510061)
\curveto(676.57184439,60.99509991)(676.53184443,60.99009992)(676.49184082,60.99010061)
\curveto(676.45184451,61.00009991)(676.41184455,61.00009991)(676.37184082,60.99010061)
\curveto(676.32184464,60.98009993)(676.27184469,60.97509993)(676.22184082,60.97510061)
\lineto(676.05684082,60.97510061)
\curveto(676.00684495,60.95509995)(675.956845,60.95009996)(675.90684082,60.96010061)
\curveto(675.84684511,60.97009994)(675.79184517,60.97009994)(675.74184082,60.96010061)
\curveto(675.70184526,60.95009996)(675.6568453,60.95009996)(675.60684082,60.96010061)
\curveto(675.5568454,60.97009994)(675.50684545,60.96509994)(675.45684082,60.94510061)
\curveto(675.38684557,60.92509998)(675.31184565,60.92009999)(675.23184082,60.93010061)
\curveto(675.14184582,60.94009997)(675.0568459,60.94509996)(674.97684082,60.94510061)
\curveto(674.88684607,60.94509996)(674.78684617,60.94009997)(674.67684082,60.93010061)
\curveto(674.5568464,60.92009999)(674.4568465,60.92509998)(674.37684082,60.94510061)
\lineto(674.09184082,60.94510061)
\lineto(673.46184082,60.99010061)
\curveto(673.3618476,61.00009991)(673.26684769,61.0100999)(673.17684082,61.02010061)
\lineto(672.87684082,61.05010061)
\curveto(672.82684813,61.07009984)(672.77684818,61.07509983)(672.72684082,61.06510061)
\curveto(672.66684829,61.06509984)(672.61184835,61.07509983)(672.56184082,61.09510061)
\curveto(672.39184857,61.14509976)(672.22684873,61.18509972)(672.06684082,61.21510061)
\curveto(671.89684906,61.24509966)(671.73684922,61.29509961)(671.58684082,61.36510061)
\curveto(671.12684983,61.55509935)(670.75185021,61.77509913)(670.46184082,62.02510061)
\curveto(670.17185079,62.28509862)(669.92685103,62.64509826)(669.72684082,63.10510061)
\curveto(669.67685128,63.23509767)(669.64185132,63.36509754)(669.62184082,63.49510061)
\curveto(669.60185136,63.63509727)(669.57685138,63.77509713)(669.54684082,63.91510061)
\curveto(669.53685142,63.98509692)(669.53185143,64.05009686)(669.53184082,64.11010061)
\curveto(669.53185143,64.17009674)(669.52685143,64.23509667)(669.51684082,64.30510061)
\curveto(669.49685146,65.13509577)(669.64685131,65.8050951)(669.96684082,66.31510061)
\curveto(670.27685068,66.82509408)(670.71685024,67.2050937)(671.28684082,67.45510061)
\curveto(671.40684955,67.5050934)(671.53184943,67.55009336)(671.66184082,67.59010061)
\curveto(671.79184917,67.63009328)(671.92684903,67.67509323)(672.06684082,67.72510061)
\curveto(672.14684881,67.74509316)(672.23184873,67.76009315)(672.32184082,67.77010061)
\lineto(672.56184082,67.83010061)
\curveto(672.67184829,67.86009305)(672.78184818,67.87509303)(672.89184082,67.87510061)
\curveto(673.00184796,67.88509302)(673.11184785,67.90009301)(673.22184082,67.92010061)
\curveto(673.27184769,67.94009297)(673.31684764,67.94509296)(673.35684082,67.93510061)
\curveto(673.39684756,67.93509297)(673.43684752,67.94009297)(673.47684082,67.95010061)
\curveto(673.52684743,67.96009295)(673.58184738,67.96009295)(673.64184082,67.95010061)
\curveto(673.69184727,67.95009296)(673.74184722,67.95509295)(673.79184082,67.96510061)
\lineto(673.92684082,67.96510061)
\curveto(673.98684697,67.98509292)(674.0568469,67.98509292)(674.13684082,67.96510061)
\curveto(674.20684675,67.95509295)(674.27184669,67.96009295)(674.33184082,67.98010061)
\curveto(674.3618466,67.99009292)(674.40184656,67.99509291)(674.45184082,67.99510061)
\lineto(674.57184082,67.99510061)
\lineto(675.03684082,67.99510061)
\moveto(677.36184082,66.45010061)
\curveto(677.04184392,66.55009436)(676.67684428,66.6100943)(676.26684082,66.63010061)
\curveto(675.8568451,66.65009426)(675.44684551,66.66009425)(675.03684082,66.66010061)
\curveto(674.60684635,66.66009425)(674.18684677,66.65009426)(673.77684082,66.63010061)
\curveto(673.36684759,66.6100943)(672.98184798,66.56509434)(672.62184082,66.49510061)
\curveto(672.2618487,66.42509448)(671.94184902,66.31509459)(671.66184082,66.16510061)
\curveto(671.37184959,66.02509488)(671.13684982,65.83009508)(670.95684082,65.58010061)
\curveto(670.84685011,65.42009549)(670.76685019,65.24009567)(670.71684082,65.04010061)
\curveto(670.6568503,64.84009607)(670.62685033,64.59509631)(670.62684082,64.30510061)
\curveto(670.64685031,64.28509662)(670.6568503,64.25009666)(670.65684082,64.20010061)
\curveto(670.64685031,64.15009676)(670.64685031,64.1100968)(670.65684082,64.08010061)
\curveto(670.67685028,64.00009691)(670.69685026,63.92509698)(670.71684082,63.85510061)
\curveto(670.72685023,63.79509711)(670.74685021,63.73009718)(670.77684082,63.66010061)
\curveto(670.89685006,63.39009752)(671.06684989,63.17009774)(671.28684082,63.00010061)
\curveto(671.49684946,62.84009807)(671.74184922,62.7050982)(672.02184082,62.59510061)
\curveto(672.13184883,62.54509836)(672.25184871,62.5050984)(672.38184082,62.47510061)
\curveto(672.50184846,62.45509845)(672.62684833,62.43009848)(672.75684082,62.40010061)
\curveto(672.80684815,62.38009853)(672.8618481,62.37009854)(672.92184082,62.37010061)
\curveto(672.97184799,62.37009854)(673.02184794,62.36509854)(673.07184082,62.35510061)
\curveto(673.1618478,62.34509856)(673.2568477,62.33509857)(673.35684082,62.32510061)
\curveto(673.44684751,62.31509859)(673.54184742,62.3050986)(673.64184082,62.29510061)
\curveto(673.72184724,62.29509861)(673.80684715,62.29009862)(673.89684082,62.28010061)
\lineto(674.13684082,62.28010061)
\lineto(674.31684082,62.28010061)
\curveto(674.34684661,62.27009864)(674.38184658,62.26509864)(674.42184082,62.26510061)
\lineto(674.55684082,62.26510061)
\lineto(675.00684082,62.26510061)
\curveto(675.08684587,62.26509864)(675.17184579,62.26009865)(675.26184082,62.25010061)
\curveto(675.34184562,62.25009866)(675.41684554,62.26009865)(675.48684082,62.28010061)
\lineto(675.75684082,62.28010061)
\curveto(675.77684518,62.28009863)(675.80684515,62.27509863)(675.84684082,62.26510061)
\curveto(675.87684508,62.26509864)(675.90184506,62.27009864)(675.92184082,62.28010061)
\curveto(676.02184494,62.29009862)(676.12184484,62.29509861)(676.22184082,62.29510061)
\curveto(676.31184465,62.3050986)(676.41184455,62.31509859)(676.52184082,62.32510061)
\curveto(676.64184432,62.35509855)(676.76684419,62.37009854)(676.89684082,62.37010061)
\curveto(677.01684394,62.38009853)(677.13184383,62.4050985)(677.24184082,62.44510061)
\curveto(677.54184342,62.52509838)(677.80684315,62.6100983)(678.03684082,62.70010061)
\curveto(678.26684269,62.80009811)(678.48184248,62.94509796)(678.68184082,63.13510061)
\curveto(678.88184208,63.34509756)(679.03184193,63.6100973)(679.13184082,63.93010061)
\curveto(679.15184181,63.97009694)(679.1618418,64.0050969)(679.16184082,64.03510061)
\curveto(679.15184181,64.07509683)(679.1568418,64.12009679)(679.17684082,64.17010061)
\curveto(679.18684177,64.2100967)(679.19684176,64.28009663)(679.20684082,64.38010061)
\curveto(679.21684174,64.49009642)(679.21184175,64.57509633)(679.19184082,64.63510061)
\curveto(679.17184179,64.7050962)(679.1618418,64.77509613)(679.16184082,64.84510061)
\curveto(679.15184181,64.91509599)(679.13684182,64.98009593)(679.11684082,65.04010061)
\curveto(679.0568419,65.24009567)(678.97184199,65.42009549)(678.86184082,65.58010061)
\curveto(678.84184212,65.6100953)(678.82184214,65.63509527)(678.80184082,65.65510061)
\lineto(678.74184082,65.71510061)
\curveto(678.72184224,65.75509515)(678.68184228,65.8050951)(678.62184082,65.86510061)
\curveto(678.48184248,65.96509494)(678.35184261,66.05009486)(678.23184082,66.12010061)
\curveto(678.11184285,66.19009472)(677.96684299,66.26009465)(677.79684082,66.33010061)
\curveto(677.72684323,66.36009455)(677.6568433,66.38009453)(677.58684082,66.39010061)
\curveto(677.51684344,66.4100945)(677.44184352,66.43009448)(677.36184082,66.45010061)
}
}
{
\newrgbcolor{curcolor}{0 0 0}
\pscustom[linestyle=none,fillstyle=solid,fillcolor=curcolor]
{
\newpath
\moveto(675.80184082,76.34470998)
\curveto(675.92184504,76.37470226)(676.0618449,76.39970223)(676.22184082,76.41970998)
\curveto(676.38184458,76.43970219)(676.54684441,76.44970218)(676.71684082,76.44970998)
\curveto(676.88684407,76.44970218)(677.05184391,76.43970219)(677.21184082,76.41970998)
\curveto(677.37184359,76.39970223)(677.51184345,76.37470226)(677.63184082,76.34470998)
\curveto(677.77184319,76.30470233)(677.89684306,76.26970236)(678.00684082,76.23970998)
\curveto(678.11684284,76.20970242)(678.22684273,76.16970246)(678.33684082,76.11970998)
\curveto(678.97684198,75.84970278)(679.4618415,75.4347032)(679.79184082,74.87470998)
\curveto(679.85184111,74.79470384)(679.90184106,74.70970392)(679.94184082,74.61970998)
\curveto(679.97184099,74.5297041)(680.00684095,74.4297042)(680.04684082,74.31970998)
\curveto(680.09684086,74.20970442)(680.13184083,74.08970454)(680.15184082,73.95970998)
\curveto(680.18184078,73.83970479)(680.21184075,73.70970492)(680.24184082,73.56970998)
\curveto(680.2618407,73.50970512)(680.26684069,73.44970518)(680.25684082,73.38970998)
\curveto(680.24684071,73.33970529)(680.25184071,73.27970535)(680.27184082,73.20970998)
\curveto(680.28184068,73.18970544)(680.28184068,73.16470547)(680.27184082,73.13470998)
\curveto(680.27184069,73.10470553)(680.27684068,73.07970555)(680.28684082,73.05970998)
\lineto(680.28684082,72.90970998)
\curveto(680.29684066,72.83970579)(680.29684066,72.78970584)(680.28684082,72.75970998)
\curveto(680.27684068,72.71970591)(680.27184069,72.67470596)(680.27184082,72.62470998)
\curveto(680.28184068,72.58470605)(680.28184068,72.54470609)(680.27184082,72.50470998)
\curveto(680.25184071,72.41470622)(680.23684072,72.32470631)(680.22684082,72.23470998)
\curveto(680.22684073,72.14470649)(680.21684074,72.05470658)(680.19684082,71.96470998)
\curveto(680.16684079,71.87470676)(680.14184082,71.78470685)(680.12184082,71.69470998)
\curveto(680.10184086,71.60470703)(680.07184089,71.51970711)(680.03184082,71.43970998)
\curveto(679.92184104,71.19970743)(679.79184117,70.97470766)(679.64184082,70.76470998)
\curveto(679.48184148,70.55470808)(679.30184166,70.37470826)(679.10184082,70.22470998)
\curveto(678.93184203,70.10470853)(678.7568422,69.99970863)(678.57684082,69.90970998)
\curveto(678.39684256,69.81970881)(678.20684275,69.7297089)(678.00684082,69.63970998)
\curveto(677.90684305,69.59970903)(677.80684315,69.56470907)(677.70684082,69.53470998)
\curveto(677.59684336,69.51470912)(677.48684347,69.48970914)(677.37684082,69.45970998)
\curveto(677.23684372,69.41970921)(677.09684386,69.39470924)(676.95684082,69.38470998)
\curveto(676.81684414,69.37470926)(676.67684428,69.35470928)(676.53684082,69.32470998)
\curveto(676.42684453,69.31470932)(676.32684463,69.30470933)(676.23684082,69.29470998)
\curveto(676.13684482,69.29470934)(676.03684492,69.28470935)(675.93684082,69.26470998)
\lineto(675.84684082,69.26470998)
\curveto(675.81684514,69.27470936)(675.79184517,69.27470936)(675.77184082,69.26470998)
\lineto(675.56184082,69.26470998)
\curveto(675.50184546,69.24470939)(675.43684552,69.2347094)(675.36684082,69.23470998)
\curveto(675.28684567,69.24470939)(675.21184575,69.24970938)(675.14184082,69.24970998)
\lineto(674.99184082,69.24970998)
\curveto(674.94184602,69.24970938)(674.89184607,69.25470938)(674.84184082,69.26470998)
\lineto(674.46684082,69.26470998)
\curveto(674.43684652,69.27470936)(674.40184656,69.27470936)(674.36184082,69.26470998)
\curveto(674.32184664,69.26470937)(674.28184668,69.26970936)(674.24184082,69.27970998)
\curveto(674.13184683,69.29970933)(674.02184694,69.31470932)(673.91184082,69.32470998)
\curveto(673.79184717,69.3347093)(673.67684728,69.34470929)(673.56684082,69.35470998)
\curveto(673.41684754,69.39470924)(673.27184769,69.41970921)(673.13184082,69.42970998)
\curveto(672.98184798,69.44970918)(672.83684812,69.47970915)(672.69684082,69.51970998)
\curveto(672.39684856,69.60970902)(672.11184885,69.70470893)(671.84184082,69.80470998)
\curveto(671.57184939,69.90470873)(671.32184964,70.0297086)(671.09184082,70.17970998)
\curveto(670.77185019,70.37970825)(670.49185047,70.62470801)(670.25184082,70.91470998)
\curveto(670.01185095,71.20470743)(669.82685113,71.54470709)(669.69684082,71.93470998)
\curveto(669.6568513,72.04470659)(669.63185133,72.15470648)(669.62184082,72.26470998)
\curveto(669.60185136,72.38470625)(669.57685138,72.50470613)(669.54684082,72.62470998)
\curveto(669.53685142,72.69470594)(669.53185143,72.75970587)(669.53184082,72.81970998)
\curveto(669.53185143,72.87970575)(669.52685143,72.94470569)(669.51684082,73.01470998)
\curveto(669.49685146,73.71470492)(669.61185135,74.28970434)(669.86184082,74.73970998)
\curveto(670.11185085,75.18970344)(670.4618505,75.5347031)(670.91184082,75.77470998)
\curveto(671.14184982,75.88470275)(671.41684954,75.98470265)(671.73684082,76.07470998)
\curveto(671.80684915,76.09470254)(671.88184908,76.09470254)(671.96184082,76.07470998)
\curveto(672.03184893,76.06470257)(672.08184888,76.03970259)(672.11184082,75.99970998)
\curveto(672.14184882,75.96970266)(672.16684879,75.90970272)(672.18684082,75.81970998)
\curveto(672.19684876,75.7297029)(672.20684875,75.629703)(672.21684082,75.51970998)
\curveto(672.21684874,75.41970321)(672.21184875,75.31970331)(672.20184082,75.21970998)
\curveto(672.19184877,75.1297035)(672.17184879,75.06470357)(672.14184082,75.02470998)
\curveto(672.07184889,74.91470372)(671.961849,74.8347038)(671.81184082,74.78470998)
\curveto(671.6618493,74.74470389)(671.53184943,74.68970394)(671.42184082,74.61970998)
\curveto(671.11184985,74.4297042)(670.88185008,74.14970448)(670.73184082,73.77970998)
\curveto(670.70185026,73.70970492)(670.68185028,73.634705)(670.67184082,73.55470998)
\curveto(670.6618503,73.48470515)(670.64685031,73.40970522)(670.62684082,73.32970998)
\curveto(670.61685034,73.27970535)(670.61185035,73.20970542)(670.61184082,73.11970998)
\curveto(670.61185035,73.03970559)(670.61685034,72.97470566)(670.62684082,72.92470998)
\curveto(670.64685031,72.88470575)(670.65185031,72.84970578)(670.64184082,72.81970998)
\curveto(670.63185033,72.78970584)(670.63185033,72.75470588)(670.64184082,72.71470998)
\lineto(670.70184082,72.47470998)
\curveto(670.72185024,72.40470623)(670.74685021,72.3347063)(670.77684082,72.26470998)
\curveto(670.93685002,71.88470675)(671.14684981,71.59470704)(671.40684082,71.39470998)
\curveto(671.66684929,71.20470743)(671.98184898,71.0297076)(672.35184082,70.86970998)
\curveto(672.43184853,70.83970779)(672.51184845,70.81470782)(672.59184082,70.79470998)
\curveto(672.67184829,70.78470785)(672.75184821,70.76470787)(672.83184082,70.73470998)
\curveto(672.94184802,70.70470793)(673.0568479,70.67970795)(673.17684082,70.65970998)
\curveto(673.29684766,70.64970798)(673.41684754,70.629708)(673.53684082,70.59970998)
\curveto(673.58684737,70.57970805)(673.63684732,70.56970806)(673.68684082,70.56970998)
\curveto(673.73684722,70.57970805)(673.78684717,70.57470806)(673.83684082,70.55470998)
\curveto(673.89684706,70.54470809)(673.97684698,70.54470809)(674.07684082,70.55470998)
\curveto(674.16684679,70.56470807)(674.22184674,70.57970805)(674.24184082,70.59970998)
\curveto(674.28184668,70.61970801)(674.30184666,70.64970798)(674.30184082,70.68970998)
\curveto(674.30184666,70.73970789)(674.29184667,70.78470785)(674.27184082,70.82470998)
\curveto(674.23184673,70.89470774)(674.18684677,70.95470768)(674.13684082,71.00470998)
\curveto(674.08684687,71.05470758)(674.03684692,71.11470752)(673.98684082,71.18470998)
\lineto(673.92684082,71.24470998)
\curveto(673.89684706,71.27470736)(673.87184709,71.30470733)(673.85184082,71.33470998)
\curveto(673.69184727,71.56470707)(673.5568474,71.83970679)(673.44684082,72.15970998)
\curveto(673.42684753,72.2297064)(673.41184755,72.29970633)(673.40184082,72.36970998)
\curveto(673.39184757,72.43970619)(673.37684758,72.51470612)(673.35684082,72.59470998)
\curveto(673.3568476,72.634706)(673.35184761,72.66970596)(673.34184082,72.69970998)
\curveto(673.33184763,72.7297059)(673.33184763,72.76470587)(673.34184082,72.80470998)
\curveto(673.34184762,72.85470578)(673.33184763,72.89470574)(673.31184082,72.92470998)
\lineto(673.31184082,73.08970998)
\lineto(673.31184082,73.17970998)
\curveto(673.30184766,73.2297054)(673.30184766,73.26970536)(673.31184082,73.29970998)
\curveto(673.32184764,73.34970528)(673.32684763,73.39970523)(673.32684082,73.44970998)
\curveto(673.31684764,73.50970512)(673.31684764,73.56470507)(673.32684082,73.61470998)
\curveto(673.3568476,73.72470491)(673.37684758,73.8297048)(673.38684082,73.92970998)
\curveto(673.39684756,74.03970459)(673.42184754,74.14470449)(673.46184082,74.24470998)
\curveto(673.60184736,74.66470397)(673.78684717,75.00970362)(674.01684082,75.27970998)
\curveto(674.23684672,75.54970308)(674.52184644,75.78970284)(674.87184082,75.99970998)
\curveto(675.01184595,76.07970255)(675.1618458,76.14470249)(675.32184082,76.19470998)
\curveto(675.47184549,76.24470239)(675.63184533,76.29470234)(675.80184082,76.34470998)
\moveto(677.10684082,75.09970998)
\curveto(677.0568439,75.10970352)(677.01184395,75.11470352)(676.97184082,75.11470998)
\lineto(676.82184082,75.11470998)
\curveto(676.51184445,75.11470352)(676.22684473,75.07470356)(675.96684082,74.99470998)
\curveto(675.90684505,74.97470366)(675.85184511,74.95470368)(675.80184082,74.93470998)
\curveto(675.74184522,74.92470371)(675.68684527,74.90970372)(675.63684082,74.88970998)
\curveto(675.14684581,74.66970396)(674.79684616,74.32470431)(674.58684082,73.85470998)
\curveto(674.5568464,73.77470486)(674.53184643,73.69470494)(674.51184082,73.61470998)
\lineto(674.45184082,73.37470998)
\curveto(674.43184653,73.29470534)(674.42184654,73.20470543)(674.42184082,73.10470998)
\lineto(674.42184082,72.78970998)
\curveto(674.44184652,72.76970586)(674.45184651,72.7297059)(674.45184082,72.66970998)
\curveto(674.44184652,72.61970601)(674.44184652,72.57470606)(674.45184082,72.53470998)
\lineto(674.51184082,72.29470998)
\curveto(674.52184644,72.22470641)(674.54184642,72.15470648)(674.57184082,72.08470998)
\curveto(674.83184613,71.48470715)(675.29684566,71.07970755)(675.96684082,70.86970998)
\curveto(676.04684491,70.83970779)(676.12684483,70.81970781)(676.20684082,70.80970998)
\curveto(676.28684467,70.79970783)(676.37184459,70.78470785)(676.46184082,70.76470998)
\lineto(676.61184082,70.76470998)
\curveto(676.65184431,70.75470788)(676.72184424,70.74970788)(676.82184082,70.74970998)
\curveto(677.05184391,70.74970788)(677.24684371,70.76970786)(677.40684082,70.80970998)
\curveto(677.47684348,70.8297078)(677.54184342,70.84470779)(677.60184082,70.85470998)
\curveto(677.6618433,70.86470777)(677.72684323,70.88470775)(677.79684082,70.91470998)
\curveto(678.07684288,71.02470761)(678.32184264,71.16970746)(678.53184082,71.34970998)
\curveto(678.73184223,71.5297071)(678.89184207,71.76470687)(679.01184082,72.05470998)
\lineto(679.10184082,72.29470998)
\lineto(679.16184082,72.53470998)
\curveto(679.18184178,72.58470605)(679.18684177,72.62470601)(679.17684082,72.65470998)
\curveto(679.16684179,72.69470594)(679.17184179,72.73970589)(679.19184082,72.78970998)
\curveto(679.20184176,72.81970581)(679.20684175,72.87470576)(679.20684082,72.95470998)
\curveto(679.20684175,73.0347056)(679.20184176,73.09470554)(679.19184082,73.13470998)
\curveto(679.17184179,73.24470539)(679.1568418,73.34970528)(679.14684082,73.44970998)
\curveto(679.13684182,73.54970508)(679.10684185,73.64470499)(679.05684082,73.73470998)
\curveto(678.8568421,74.26470437)(678.48184248,74.65470398)(677.93184082,74.90470998)
\curveto(677.83184313,74.94470369)(677.72684323,74.97470366)(677.61684082,74.99470998)
\lineto(677.28684082,75.08470998)
\curveto(677.20684375,75.08470355)(677.14684381,75.08970354)(677.10684082,75.09970998)
}
}
{
\newrgbcolor{curcolor}{0 0 0}
\pscustom[linestyle=none,fillstyle=solid,fillcolor=curcolor]
{
\newpath
\moveto(678.48684082,78.63431936)
\lineto(678.48684082,79.26431936)
\lineto(678.48684082,79.45931936)
\curveto(678.48684247,79.52931683)(678.49684246,79.58931677)(678.51684082,79.63931936)
\curveto(678.5568424,79.70931665)(678.59684236,79.7593166)(678.63684082,79.78931936)
\curveto(678.68684227,79.82931653)(678.75184221,79.84931651)(678.83184082,79.84931936)
\curveto(678.91184205,79.8593165)(678.99684196,79.86431649)(679.08684082,79.86431936)
\lineto(679.80684082,79.86431936)
\curveto(680.28684067,79.86431649)(680.69684026,79.80431655)(681.03684082,79.68431936)
\curveto(681.37683958,79.56431679)(681.65183931,79.36931699)(681.86184082,79.09931936)
\curveto(681.91183905,79.02931733)(681.956839,78.9593174)(681.99684082,78.88931936)
\curveto(682.04683891,78.82931753)(682.09183887,78.7543176)(682.13184082,78.66431936)
\curveto(682.14183882,78.64431771)(682.15183881,78.61431774)(682.16184082,78.57431936)
\curveto(682.18183878,78.53431782)(682.18683877,78.48931787)(682.17684082,78.43931936)
\curveto(682.14683881,78.34931801)(682.07183889,78.29431806)(681.95184082,78.27431936)
\curveto(681.84183912,78.2543181)(681.74683921,78.26931809)(681.66684082,78.31931936)
\curveto(681.59683936,78.34931801)(681.53183943,78.39431796)(681.47184082,78.45431936)
\curveto(681.42183954,78.52431783)(681.37183959,78.58931777)(681.32184082,78.64931936)
\curveto(681.27183969,78.71931764)(681.19683976,78.77931758)(681.09684082,78.82931936)
\curveto(681.00683995,78.88931747)(680.91684004,78.93931742)(680.82684082,78.97931936)
\curveto(680.79684016,78.99931736)(680.73684022,79.02431733)(680.64684082,79.05431936)
\curveto(680.56684039,79.08431727)(680.49684046,79.08931727)(680.43684082,79.06931936)
\curveto(680.29684066,79.03931732)(680.20684075,78.97931738)(680.16684082,78.88931936)
\curveto(680.13684082,78.80931755)(680.12184084,78.71931764)(680.12184082,78.61931936)
\curveto(680.12184084,78.51931784)(680.09684086,78.43431792)(680.04684082,78.36431936)
\curveto(679.97684098,78.27431808)(679.83684112,78.22931813)(679.62684082,78.22931936)
\lineto(679.07184082,78.22931936)
\lineto(678.84684082,78.22931936)
\curveto(678.76684219,78.23931812)(678.70184226,78.2593181)(678.65184082,78.28931936)
\curveto(678.57184239,78.34931801)(678.52684243,78.41931794)(678.51684082,78.49931936)
\curveto(678.50684245,78.51931784)(678.50184246,78.53931782)(678.50184082,78.55931936)
\curveto(678.50184246,78.58931777)(678.49684246,78.61431774)(678.48684082,78.63431936)
}
}
{
\newrgbcolor{curcolor}{0 0 0}
\pscustom[linestyle=none,fillstyle=solid,fillcolor=curcolor]
{
}
}
{
\newrgbcolor{curcolor}{0 0 0}
\pscustom[linestyle=none,fillstyle=solid,fillcolor=curcolor]
{
\newpath
\moveto(669.51684082,89.26463186)
\curveto(669.50685145,89.95462722)(669.62685133,90.55462662)(669.87684082,91.06463186)
\curveto(670.12685083,91.58462559)(670.4618505,91.9796252)(670.88184082,92.24963186)
\curveto(670.96185,92.29962488)(671.05184991,92.34462483)(671.15184082,92.38463186)
\curveto(671.24184972,92.42462475)(671.33684962,92.46962471)(671.43684082,92.51963186)
\curveto(671.53684942,92.55962462)(671.63684932,92.58962459)(671.73684082,92.60963186)
\curveto(671.83684912,92.62962455)(671.94184902,92.64962453)(672.05184082,92.66963186)
\curveto(672.10184886,92.68962449)(672.14684881,92.69462448)(672.18684082,92.68463186)
\curveto(672.22684873,92.6746245)(672.27184869,92.6796245)(672.32184082,92.69963186)
\curveto(672.37184859,92.70962447)(672.4568485,92.71462446)(672.57684082,92.71463186)
\curveto(672.68684827,92.71462446)(672.77184819,92.70962447)(672.83184082,92.69963186)
\curveto(672.89184807,92.6796245)(672.95184801,92.66962451)(673.01184082,92.66963186)
\curveto(673.07184789,92.6796245)(673.13184783,92.6746245)(673.19184082,92.65463186)
\curveto(673.33184763,92.61462456)(673.46684749,92.5796246)(673.59684082,92.54963186)
\curveto(673.72684723,92.51962466)(673.85184711,92.4796247)(673.97184082,92.42963186)
\curveto(674.11184685,92.36962481)(674.23684672,92.29962488)(674.34684082,92.21963186)
\curveto(674.4568465,92.14962503)(674.56684639,92.0746251)(674.67684082,91.99463186)
\lineto(674.73684082,91.93463186)
\curveto(674.7568462,91.92462525)(674.77684618,91.90962527)(674.79684082,91.88963186)
\curveto(674.956846,91.76962541)(675.10184586,91.63462554)(675.23184082,91.48463186)
\curveto(675.3618456,91.33462584)(675.48684547,91.174626)(675.60684082,91.00463186)
\curveto(675.82684513,90.69462648)(676.03184493,90.39962678)(676.22184082,90.11963186)
\curveto(676.3618446,89.88962729)(676.49684446,89.65962752)(676.62684082,89.42963186)
\curveto(676.7568442,89.20962797)(676.89184407,88.98962819)(677.03184082,88.76963186)
\curveto(677.20184376,88.51962866)(677.38184358,88.2796289)(677.57184082,88.04963186)
\curveto(677.7618432,87.82962935)(677.98684297,87.63962954)(678.24684082,87.47963186)
\curveto(678.30684265,87.43962974)(678.36684259,87.40462977)(678.42684082,87.37463186)
\curveto(678.47684248,87.34462983)(678.54184242,87.31462986)(678.62184082,87.28463186)
\curveto(678.69184227,87.26462991)(678.75184221,87.25962992)(678.80184082,87.26963186)
\curveto(678.87184209,87.28962989)(678.92684203,87.32462985)(678.96684082,87.37463186)
\curveto(678.99684196,87.42462975)(679.01684194,87.48462969)(679.02684082,87.55463186)
\lineto(679.02684082,87.79463186)
\lineto(679.02684082,88.54463186)
\lineto(679.02684082,91.34963186)
\lineto(679.02684082,92.00963186)
\curveto(679.02684193,92.09962508)(679.03184193,92.18462499)(679.04184082,92.26463186)
\curveto(679.04184192,92.34462483)(679.0618419,92.40962477)(679.10184082,92.45963186)
\curveto(679.14184182,92.50962467)(679.21684174,92.54962463)(679.32684082,92.57963186)
\curveto(679.42684153,92.61962456)(679.52684143,92.62962455)(679.62684082,92.60963186)
\lineto(679.76184082,92.60963186)
\curveto(679.83184113,92.58962459)(679.89184107,92.56962461)(679.94184082,92.54963186)
\curveto(679.99184097,92.52962465)(680.03184093,92.49462468)(680.06184082,92.44463186)
\curveto(680.10184086,92.39462478)(680.12184084,92.32462485)(680.12184082,92.23463186)
\lineto(680.12184082,91.96463186)
\lineto(680.12184082,91.06463186)
\lineto(680.12184082,87.55463186)
\lineto(680.12184082,86.48963186)
\curveto(680.12184084,86.40963077)(680.12684083,86.31963086)(680.13684082,86.21963186)
\curveto(680.13684082,86.11963106)(680.12684083,86.03463114)(680.10684082,85.96463186)
\curveto(680.03684092,85.75463142)(679.8568411,85.68963149)(679.56684082,85.76963186)
\curveto(679.52684143,85.7796314)(679.49184147,85.7796314)(679.46184082,85.76963186)
\curveto(679.42184154,85.76963141)(679.37684158,85.7796314)(679.32684082,85.79963186)
\curveto(679.24684171,85.81963136)(679.1618418,85.83963134)(679.07184082,85.85963186)
\curveto(678.98184198,85.8796313)(678.89684206,85.90463127)(678.81684082,85.93463186)
\curveto(678.32684263,86.09463108)(677.91184305,86.29463088)(677.57184082,86.53463186)
\curveto(677.32184364,86.71463046)(677.09684386,86.91963026)(676.89684082,87.14963186)
\curveto(676.68684427,87.3796298)(676.49184447,87.61962956)(676.31184082,87.86963186)
\curveto(676.13184483,88.12962905)(675.961845,88.39462878)(675.80184082,88.66463186)
\curveto(675.63184533,88.94462823)(675.4568455,89.21462796)(675.27684082,89.47463186)
\curveto(675.19684576,89.58462759)(675.12184584,89.68962749)(675.05184082,89.78963186)
\curveto(674.98184598,89.89962728)(674.90684605,90.00962717)(674.82684082,90.11963186)
\curveto(674.79684616,90.15962702)(674.76684619,90.19462698)(674.73684082,90.22463186)
\curveto(674.69684626,90.26462691)(674.66684629,90.30462687)(674.64684082,90.34463186)
\curveto(674.53684642,90.48462669)(674.41184655,90.60962657)(674.27184082,90.71963186)
\curveto(674.24184672,90.73962644)(674.21684674,90.76462641)(674.19684082,90.79463186)
\curveto(674.16684679,90.82462635)(674.13684682,90.84962633)(674.10684082,90.86963186)
\curveto(674.00684695,90.94962623)(673.90684705,91.01462616)(673.80684082,91.06463186)
\curveto(673.70684725,91.12462605)(673.59684736,91.179626)(673.47684082,91.22963186)
\curveto(673.40684755,91.25962592)(673.33184763,91.2796259)(673.25184082,91.28963186)
\lineto(673.01184082,91.34963186)
\lineto(672.92184082,91.34963186)
\curveto(672.89184807,91.35962582)(672.8618481,91.36462581)(672.83184082,91.36463186)
\curveto(672.7618482,91.38462579)(672.66684829,91.38962579)(672.54684082,91.37963186)
\curveto(672.41684854,91.3796258)(672.31684864,91.36962581)(672.24684082,91.34963186)
\curveto(672.16684879,91.32962585)(672.09184887,91.30962587)(672.02184082,91.28963186)
\curveto(671.94184902,91.2796259)(671.8618491,91.25962592)(671.78184082,91.22963186)
\curveto(671.54184942,91.11962606)(671.34184962,90.96962621)(671.18184082,90.77963186)
\curveto(671.01184995,90.59962658)(670.87185009,90.3796268)(670.76184082,90.11963186)
\curveto(670.74185022,90.04962713)(670.72685023,89.9796272)(670.71684082,89.90963186)
\curveto(670.69685026,89.83962734)(670.67685028,89.76462741)(670.65684082,89.68463186)
\curveto(670.63685032,89.60462757)(670.62685033,89.49462768)(670.62684082,89.35463186)
\curveto(670.62685033,89.22462795)(670.63685032,89.11962806)(670.65684082,89.03963186)
\curveto(670.66685029,88.9796282)(670.67185029,88.92462825)(670.67184082,88.87463186)
\curveto(670.67185029,88.82462835)(670.68185028,88.7746284)(670.70184082,88.72463186)
\curveto(670.74185022,88.62462855)(670.78185018,88.52962865)(670.82184082,88.43963186)
\curveto(670.8618501,88.35962882)(670.90685005,88.2796289)(670.95684082,88.19963186)
\curveto(670.97684998,88.16962901)(671.00184996,88.13962904)(671.03184082,88.10963186)
\curveto(671.0618499,88.08962909)(671.08684987,88.06462911)(671.10684082,88.03463186)
\lineto(671.18184082,87.95963186)
\curveto(671.20184976,87.92962925)(671.22184974,87.90462927)(671.24184082,87.88463186)
\lineto(671.45184082,87.73463186)
\curveto(671.51184945,87.69462948)(671.57684938,87.64962953)(671.64684082,87.59963186)
\curveto(671.73684922,87.53962964)(671.84184912,87.48962969)(671.96184082,87.44963186)
\curveto(672.07184889,87.41962976)(672.18184878,87.38462979)(672.29184082,87.34463186)
\curveto(672.40184856,87.30462987)(672.54684841,87.2796299)(672.72684082,87.26963186)
\curveto(672.89684806,87.25962992)(673.02184794,87.22962995)(673.10184082,87.17963186)
\curveto(673.18184778,87.12963005)(673.22684773,87.05463012)(673.23684082,86.95463186)
\curveto(673.24684771,86.85463032)(673.25184771,86.74463043)(673.25184082,86.62463186)
\curveto(673.25184771,86.58463059)(673.2568477,86.54463063)(673.26684082,86.50463186)
\curveto(673.26684769,86.46463071)(673.2618477,86.42963075)(673.25184082,86.39963186)
\curveto(673.23184773,86.34963083)(673.22184774,86.29963088)(673.22184082,86.24963186)
\curveto(673.22184774,86.20963097)(673.21184775,86.16963101)(673.19184082,86.12963186)
\curveto(673.13184783,86.03963114)(672.99684796,85.99463118)(672.78684082,85.99463186)
\lineto(672.66684082,85.99463186)
\curveto(672.60684835,86.00463117)(672.54684841,86.00963117)(672.48684082,86.00963186)
\curveto(672.41684854,86.01963116)(672.35184861,86.02963115)(672.29184082,86.03963186)
\curveto(672.18184878,86.05963112)(672.08184888,86.0796311)(671.99184082,86.09963186)
\curveto(671.89184907,86.11963106)(671.79684916,86.14963103)(671.70684082,86.18963186)
\curveto(671.63684932,86.20963097)(671.57684938,86.22963095)(671.52684082,86.24963186)
\lineto(671.34684082,86.30963186)
\curveto(671.08684987,86.42963075)(670.84185012,86.58463059)(670.61184082,86.77463186)
\curveto(670.38185058,86.9746302)(670.19685076,87.18962999)(670.05684082,87.41963186)
\curveto(669.97685098,87.52962965)(669.91185105,87.64462953)(669.86184082,87.76463186)
\lineto(669.71184082,88.15463186)
\curveto(669.6618513,88.26462891)(669.63185133,88.3796288)(669.62184082,88.49963186)
\curveto(669.60185136,88.61962856)(669.57685138,88.74462843)(669.54684082,88.87463186)
\curveto(669.54685141,88.94462823)(669.54685141,89.00962817)(669.54684082,89.06963186)
\curveto(669.53685142,89.12962805)(669.52685143,89.19462798)(669.51684082,89.26463186)
}
}
{
\newrgbcolor{curcolor}{0 0 0}
\pscustom[linestyle=none,fillstyle=solid,fillcolor=curcolor]
{
\newpath
\moveto(675.03684082,101.36424123)
\lineto(675.29184082,101.36424123)
\curveto(675.37184559,101.37423353)(675.44684551,101.36923353)(675.51684082,101.34924123)
\lineto(675.75684082,101.34924123)
\lineto(675.92184082,101.34924123)
\curveto(676.02184494,101.32923357)(676.12684483,101.31923358)(676.23684082,101.31924123)
\curveto(676.33684462,101.31923358)(676.43684452,101.30923359)(676.53684082,101.28924123)
\lineto(676.68684082,101.28924123)
\curveto(676.82684413,101.25923364)(676.96684399,101.23923366)(677.10684082,101.22924123)
\curveto(677.23684372,101.21923368)(677.36684359,101.19423371)(677.49684082,101.15424123)
\curveto(677.57684338,101.13423377)(677.6618433,101.11423379)(677.75184082,101.09424123)
\lineto(677.99184082,101.03424123)
\lineto(678.29184082,100.91424123)
\curveto(678.38184258,100.88423402)(678.47184249,100.84923405)(678.56184082,100.80924123)
\curveto(678.78184218,100.70923419)(678.99684196,100.57423433)(679.20684082,100.40424123)
\curveto(679.41684154,100.24423466)(679.58684137,100.06923483)(679.71684082,99.87924123)
\curveto(679.7568412,99.82923507)(679.79684116,99.76923513)(679.83684082,99.69924123)
\curveto(679.86684109,99.63923526)(679.90184106,99.57923532)(679.94184082,99.51924123)
\curveto(679.99184097,99.43923546)(680.03184093,99.34423556)(680.06184082,99.23424123)
\curveto(680.09184087,99.12423578)(680.12184084,99.01923588)(680.15184082,98.91924123)
\curveto(680.19184077,98.80923609)(680.21684074,98.6992362)(680.22684082,98.58924123)
\curveto(680.23684072,98.47923642)(680.25184071,98.36423654)(680.27184082,98.24424123)
\curveto(680.28184068,98.2042367)(680.28184068,98.15923674)(680.27184082,98.10924123)
\curveto(680.27184069,98.06923683)(680.27684068,98.02923687)(680.28684082,97.98924123)
\curveto(680.29684066,97.94923695)(680.30184066,97.89423701)(680.30184082,97.82424123)
\curveto(680.30184066,97.75423715)(680.29684066,97.7042372)(680.28684082,97.67424123)
\curveto(680.26684069,97.62423728)(680.2618407,97.57923732)(680.27184082,97.53924123)
\curveto(680.28184068,97.4992374)(680.28184068,97.46423744)(680.27184082,97.43424123)
\lineto(680.27184082,97.34424123)
\curveto(680.25184071,97.28423762)(680.23684072,97.21923768)(680.22684082,97.14924123)
\curveto(680.22684073,97.08923781)(680.22184074,97.02423788)(680.21184082,96.95424123)
\curveto(680.1618408,96.78423812)(680.11184085,96.62423828)(680.06184082,96.47424123)
\curveto(680.01184095,96.32423858)(679.94684101,96.17923872)(679.86684082,96.03924123)
\curveto(679.82684113,95.98923891)(679.79684116,95.93423897)(679.77684082,95.87424123)
\curveto(679.74684121,95.82423908)(679.71184125,95.77423913)(679.67184082,95.72424123)
\curveto(679.49184147,95.48423942)(679.27184169,95.28423962)(679.01184082,95.12424123)
\curveto(678.75184221,94.96423994)(678.46684249,94.82424008)(678.15684082,94.70424123)
\curveto(678.01684294,94.64424026)(677.87684308,94.5992403)(677.73684082,94.56924123)
\curveto(677.58684337,94.53924036)(677.43184353,94.5042404)(677.27184082,94.46424123)
\curveto(677.1618438,94.44424046)(677.05184391,94.42924047)(676.94184082,94.41924123)
\curveto(676.83184413,94.40924049)(676.72184424,94.39424051)(676.61184082,94.37424123)
\curveto(676.57184439,94.36424054)(676.53184443,94.35924054)(676.49184082,94.35924123)
\curveto(676.45184451,94.36924053)(676.41184455,94.36924053)(676.37184082,94.35924123)
\curveto(676.32184464,94.34924055)(676.27184469,94.34424056)(676.22184082,94.34424123)
\lineto(676.05684082,94.34424123)
\curveto(676.00684495,94.32424058)(675.956845,94.31924058)(675.90684082,94.32924123)
\curveto(675.84684511,94.33924056)(675.79184517,94.33924056)(675.74184082,94.32924123)
\curveto(675.70184526,94.31924058)(675.6568453,94.31924058)(675.60684082,94.32924123)
\curveto(675.5568454,94.33924056)(675.50684545,94.33424057)(675.45684082,94.31424123)
\curveto(675.38684557,94.29424061)(675.31184565,94.28924061)(675.23184082,94.29924123)
\curveto(675.14184582,94.30924059)(675.0568459,94.31424059)(674.97684082,94.31424123)
\curveto(674.88684607,94.31424059)(674.78684617,94.30924059)(674.67684082,94.29924123)
\curveto(674.5568464,94.28924061)(674.4568465,94.29424061)(674.37684082,94.31424123)
\lineto(674.09184082,94.31424123)
\lineto(673.46184082,94.35924123)
\curveto(673.3618476,94.36924053)(673.26684769,94.37924052)(673.17684082,94.38924123)
\lineto(672.87684082,94.41924123)
\curveto(672.82684813,94.43924046)(672.77684818,94.44424046)(672.72684082,94.43424123)
\curveto(672.66684829,94.43424047)(672.61184835,94.44424046)(672.56184082,94.46424123)
\curveto(672.39184857,94.51424039)(672.22684873,94.55424035)(672.06684082,94.58424123)
\curveto(671.89684906,94.61424029)(671.73684922,94.66424024)(671.58684082,94.73424123)
\curveto(671.12684983,94.92423998)(670.75185021,95.14423976)(670.46184082,95.39424123)
\curveto(670.17185079,95.65423925)(669.92685103,96.01423889)(669.72684082,96.47424123)
\curveto(669.67685128,96.6042383)(669.64185132,96.73423817)(669.62184082,96.86424123)
\curveto(669.60185136,97.0042379)(669.57685138,97.14423776)(669.54684082,97.28424123)
\curveto(669.53685142,97.35423755)(669.53185143,97.41923748)(669.53184082,97.47924123)
\curveto(669.53185143,97.53923736)(669.52685143,97.6042373)(669.51684082,97.67424123)
\curveto(669.49685146,98.5042364)(669.64685131,99.17423573)(669.96684082,99.68424123)
\curveto(670.27685068,100.19423471)(670.71685024,100.57423433)(671.28684082,100.82424123)
\curveto(671.40684955,100.87423403)(671.53184943,100.91923398)(671.66184082,100.95924123)
\curveto(671.79184917,100.9992339)(671.92684903,101.04423386)(672.06684082,101.09424123)
\curveto(672.14684881,101.11423379)(672.23184873,101.12923377)(672.32184082,101.13924123)
\lineto(672.56184082,101.19924123)
\curveto(672.67184829,101.22923367)(672.78184818,101.24423366)(672.89184082,101.24424123)
\curveto(673.00184796,101.25423365)(673.11184785,101.26923363)(673.22184082,101.28924123)
\curveto(673.27184769,101.30923359)(673.31684764,101.31423359)(673.35684082,101.30424123)
\curveto(673.39684756,101.3042336)(673.43684752,101.30923359)(673.47684082,101.31924123)
\curveto(673.52684743,101.32923357)(673.58184738,101.32923357)(673.64184082,101.31924123)
\curveto(673.69184727,101.31923358)(673.74184722,101.32423358)(673.79184082,101.33424123)
\lineto(673.92684082,101.33424123)
\curveto(673.98684697,101.35423355)(674.0568469,101.35423355)(674.13684082,101.33424123)
\curveto(674.20684675,101.32423358)(674.27184669,101.32923357)(674.33184082,101.34924123)
\curveto(674.3618466,101.35923354)(674.40184656,101.36423354)(674.45184082,101.36424123)
\lineto(674.57184082,101.36424123)
\lineto(675.03684082,101.36424123)
\moveto(677.36184082,99.81924123)
\curveto(677.04184392,99.91923498)(676.67684428,99.97923492)(676.26684082,99.99924123)
\curveto(675.8568451,100.01923488)(675.44684551,100.02923487)(675.03684082,100.02924123)
\curveto(674.60684635,100.02923487)(674.18684677,100.01923488)(673.77684082,99.99924123)
\curveto(673.36684759,99.97923492)(672.98184798,99.93423497)(672.62184082,99.86424123)
\curveto(672.2618487,99.79423511)(671.94184902,99.68423522)(671.66184082,99.53424123)
\curveto(671.37184959,99.39423551)(671.13684982,99.1992357)(670.95684082,98.94924123)
\curveto(670.84685011,98.78923611)(670.76685019,98.60923629)(670.71684082,98.40924123)
\curveto(670.6568503,98.20923669)(670.62685033,97.96423694)(670.62684082,97.67424123)
\curveto(670.64685031,97.65423725)(670.6568503,97.61923728)(670.65684082,97.56924123)
\curveto(670.64685031,97.51923738)(670.64685031,97.47923742)(670.65684082,97.44924123)
\curveto(670.67685028,97.36923753)(670.69685026,97.29423761)(670.71684082,97.22424123)
\curveto(670.72685023,97.16423774)(670.74685021,97.0992378)(670.77684082,97.02924123)
\curveto(670.89685006,96.75923814)(671.06684989,96.53923836)(671.28684082,96.36924123)
\curveto(671.49684946,96.20923869)(671.74184922,96.07423883)(672.02184082,95.96424123)
\curveto(672.13184883,95.91423899)(672.25184871,95.87423903)(672.38184082,95.84424123)
\curveto(672.50184846,95.82423908)(672.62684833,95.7992391)(672.75684082,95.76924123)
\curveto(672.80684815,95.74923915)(672.8618481,95.73923916)(672.92184082,95.73924123)
\curveto(672.97184799,95.73923916)(673.02184794,95.73423917)(673.07184082,95.72424123)
\curveto(673.1618478,95.71423919)(673.2568477,95.7042392)(673.35684082,95.69424123)
\curveto(673.44684751,95.68423922)(673.54184742,95.67423923)(673.64184082,95.66424123)
\curveto(673.72184724,95.66423924)(673.80684715,95.65923924)(673.89684082,95.64924123)
\lineto(674.13684082,95.64924123)
\lineto(674.31684082,95.64924123)
\curveto(674.34684661,95.63923926)(674.38184658,95.63423927)(674.42184082,95.63424123)
\lineto(674.55684082,95.63424123)
\lineto(675.00684082,95.63424123)
\curveto(675.08684587,95.63423927)(675.17184579,95.62923927)(675.26184082,95.61924123)
\curveto(675.34184562,95.61923928)(675.41684554,95.62923927)(675.48684082,95.64924123)
\lineto(675.75684082,95.64924123)
\curveto(675.77684518,95.64923925)(675.80684515,95.64423926)(675.84684082,95.63424123)
\curveto(675.87684508,95.63423927)(675.90184506,95.63923926)(675.92184082,95.64924123)
\curveto(676.02184494,95.65923924)(676.12184484,95.66423924)(676.22184082,95.66424123)
\curveto(676.31184465,95.67423923)(676.41184455,95.68423922)(676.52184082,95.69424123)
\curveto(676.64184432,95.72423918)(676.76684419,95.73923916)(676.89684082,95.73924123)
\curveto(677.01684394,95.74923915)(677.13184383,95.77423913)(677.24184082,95.81424123)
\curveto(677.54184342,95.89423901)(677.80684315,95.97923892)(678.03684082,96.06924123)
\curveto(678.26684269,96.16923873)(678.48184248,96.31423859)(678.68184082,96.50424123)
\curveto(678.88184208,96.71423819)(679.03184193,96.97923792)(679.13184082,97.29924123)
\curveto(679.15184181,97.33923756)(679.1618418,97.37423753)(679.16184082,97.40424123)
\curveto(679.15184181,97.44423746)(679.1568418,97.48923741)(679.17684082,97.53924123)
\curveto(679.18684177,97.57923732)(679.19684176,97.64923725)(679.20684082,97.74924123)
\curveto(679.21684174,97.85923704)(679.21184175,97.94423696)(679.19184082,98.00424123)
\curveto(679.17184179,98.07423683)(679.1618418,98.14423676)(679.16184082,98.21424123)
\curveto(679.15184181,98.28423662)(679.13684182,98.34923655)(679.11684082,98.40924123)
\curveto(679.0568419,98.60923629)(678.97184199,98.78923611)(678.86184082,98.94924123)
\curveto(678.84184212,98.97923592)(678.82184214,99.0042359)(678.80184082,99.02424123)
\lineto(678.74184082,99.08424123)
\curveto(678.72184224,99.12423578)(678.68184228,99.17423573)(678.62184082,99.23424123)
\curveto(678.48184248,99.33423557)(678.35184261,99.41923548)(678.23184082,99.48924123)
\curveto(678.11184285,99.55923534)(677.96684299,99.62923527)(677.79684082,99.69924123)
\curveto(677.72684323,99.72923517)(677.6568433,99.74923515)(677.58684082,99.75924123)
\curveto(677.51684344,99.77923512)(677.44184352,99.7992351)(677.36184082,99.81924123)
}
}
{
\newrgbcolor{curcolor}{0 0 0}
\pscustom[linestyle=none,fillstyle=solid,fillcolor=curcolor]
{
\newpath
\moveto(669.51684082,106.77385061)
\curveto(669.51685144,106.87384575)(669.52685143,106.96884566)(669.54684082,107.05885061)
\curveto(669.5568514,107.14884548)(669.58685137,107.21384541)(669.63684082,107.25385061)
\curveto(669.71685124,107.31384531)(669.82185114,107.34384528)(669.95184082,107.34385061)
\lineto(670.34184082,107.34385061)
\lineto(671.84184082,107.34385061)
\lineto(678.23184082,107.34385061)
\lineto(679.40184082,107.34385061)
\lineto(679.71684082,107.34385061)
\curveto(679.81684114,107.35384527)(679.89684106,107.33884529)(679.95684082,107.29885061)
\curveto(680.03684092,107.24884538)(680.08684087,107.17384545)(680.10684082,107.07385061)
\curveto(680.11684084,106.98384564)(680.12184084,106.87384575)(680.12184082,106.74385061)
\lineto(680.12184082,106.51885061)
\curveto(680.10184086,106.43884619)(680.08684087,106.36884626)(680.07684082,106.30885061)
\curveto(680.0568409,106.24884638)(680.01684094,106.19884643)(679.95684082,106.15885061)
\curveto(679.89684106,106.11884651)(679.82184114,106.09884653)(679.73184082,106.09885061)
\lineto(679.43184082,106.09885061)
\lineto(678.33684082,106.09885061)
\lineto(672.99684082,106.09885061)
\curveto(672.90684805,106.07884655)(672.83184813,106.06384656)(672.77184082,106.05385061)
\curveto(672.70184826,106.05384657)(672.64184832,106.0238466)(672.59184082,105.96385061)
\curveto(672.54184842,105.89384673)(672.51684844,105.80384682)(672.51684082,105.69385061)
\curveto(672.50684845,105.59384703)(672.50184846,105.48384714)(672.50184082,105.36385061)
\lineto(672.50184082,104.22385061)
\lineto(672.50184082,103.72885061)
\curveto(672.49184847,103.56884906)(672.43184853,103.45884917)(672.32184082,103.39885061)
\curveto(672.29184867,103.37884925)(672.2618487,103.36884926)(672.23184082,103.36885061)
\curveto(672.19184877,103.36884926)(672.14684881,103.36384926)(672.09684082,103.35385061)
\curveto(671.97684898,103.33384929)(671.86684909,103.33884929)(671.76684082,103.36885061)
\curveto(671.66684929,103.40884922)(671.59684936,103.46384916)(671.55684082,103.53385061)
\curveto(671.50684945,103.61384901)(671.48184948,103.73384889)(671.48184082,103.89385061)
\curveto(671.48184948,104.05384857)(671.46684949,104.18884844)(671.43684082,104.29885061)
\curveto(671.42684953,104.34884828)(671.42184954,104.40384822)(671.42184082,104.46385061)
\curveto(671.41184955,104.5238481)(671.39684956,104.58384804)(671.37684082,104.64385061)
\curveto(671.32684963,104.79384783)(671.27684968,104.93884769)(671.22684082,105.07885061)
\curveto(671.16684979,105.21884741)(671.09684986,105.35384727)(671.01684082,105.48385061)
\curveto(670.92685003,105.623847)(670.82185014,105.74384688)(670.70184082,105.84385061)
\curveto(670.58185038,105.94384668)(670.45185051,106.03884659)(670.31184082,106.12885061)
\curveto(670.21185075,106.18884644)(670.10185086,106.23384639)(669.98184082,106.26385061)
\curveto(669.8618511,106.30384632)(669.7568512,106.35384627)(669.66684082,106.41385061)
\curveto(669.60685135,106.46384616)(669.56685139,106.53384609)(669.54684082,106.62385061)
\curveto(669.53685142,106.64384598)(669.53185143,106.66884596)(669.53184082,106.69885061)
\curveto(669.53185143,106.7288459)(669.52685143,106.75384587)(669.51684082,106.77385061)
}
}
{
\newrgbcolor{curcolor}{0 0 0}
\pscustom[linestyle=none,fillstyle=solid,fillcolor=curcolor]
{
\newpath
\moveto(669.51684082,115.12345998)
\curveto(669.51685144,115.22345513)(669.52685143,115.31845503)(669.54684082,115.40845998)
\curveto(669.5568514,115.49845485)(669.58685137,115.56345479)(669.63684082,115.60345998)
\curveto(669.71685124,115.66345469)(669.82185114,115.69345466)(669.95184082,115.69345998)
\lineto(670.34184082,115.69345998)
\lineto(671.84184082,115.69345998)
\lineto(678.23184082,115.69345998)
\lineto(679.40184082,115.69345998)
\lineto(679.71684082,115.69345998)
\curveto(679.81684114,115.70345465)(679.89684106,115.68845466)(679.95684082,115.64845998)
\curveto(680.03684092,115.59845475)(680.08684087,115.52345483)(680.10684082,115.42345998)
\curveto(680.11684084,115.33345502)(680.12184084,115.22345513)(680.12184082,115.09345998)
\lineto(680.12184082,114.86845998)
\curveto(680.10184086,114.78845556)(680.08684087,114.71845563)(680.07684082,114.65845998)
\curveto(680.0568409,114.59845575)(680.01684094,114.5484558)(679.95684082,114.50845998)
\curveto(679.89684106,114.46845588)(679.82184114,114.4484559)(679.73184082,114.44845998)
\lineto(679.43184082,114.44845998)
\lineto(678.33684082,114.44845998)
\lineto(672.99684082,114.44845998)
\curveto(672.90684805,114.42845592)(672.83184813,114.41345594)(672.77184082,114.40345998)
\curveto(672.70184826,114.40345595)(672.64184832,114.37345598)(672.59184082,114.31345998)
\curveto(672.54184842,114.24345611)(672.51684844,114.1534562)(672.51684082,114.04345998)
\curveto(672.50684845,113.94345641)(672.50184846,113.83345652)(672.50184082,113.71345998)
\lineto(672.50184082,112.57345998)
\lineto(672.50184082,112.07845998)
\curveto(672.49184847,111.91845843)(672.43184853,111.80845854)(672.32184082,111.74845998)
\curveto(672.29184867,111.72845862)(672.2618487,111.71845863)(672.23184082,111.71845998)
\curveto(672.19184877,111.71845863)(672.14684881,111.71345864)(672.09684082,111.70345998)
\curveto(671.97684898,111.68345867)(671.86684909,111.68845866)(671.76684082,111.71845998)
\curveto(671.66684929,111.75845859)(671.59684936,111.81345854)(671.55684082,111.88345998)
\curveto(671.50684945,111.96345839)(671.48184948,112.08345827)(671.48184082,112.24345998)
\curveto(671.48184948,112.40345795)(671.46684949,112.53845781)(671.43684082,112.64845998)
\curveto(671.42684953,112.69845765)(671.42184954,112.7534576)(671.42184082,112.81345998)
\curveto(671.41184955,112.87345748)(671.39684956,112.93345742)(671.37684082,112.99345998)
\curveto(671.32684963,113.14345721)(671.27684968,113.28845706)(671.22684082,113.42845998)
\curveto(671.16684979,113.56845678)(671.09684986,113.70345665)(671.01684082,113.83345998)
\curveto(670.92685003,113.97345638)(670.82185014,114.09345626)(670.70184082,114.19345998)
\curveto(670.58185038,114.29345606)(670.45185051,114.38845596)(670.31184082,114.47845998)
\curveto(670.21185075,114.53845581)(670.10185086,114.58345577)(669.98184082,114.61345998)
\curveto(669.8618511,114.6534557)(669.7568512,114.70345565)(669.66684082,114.76345998)
\curveto(669.60685135,114.81345554)(669.56685139,114.88345547)(669.54684082,114.97345998)
\curveto(669.53685142,114.99345536)(669.53185143,115.01845533)(669.53184082,115.04845998)
\curveto(669.53185143,115.07845527)(669.52685143,115.10345525)(669.51684082,115.12345998)
}
}
{
\newrgbcolor{curcolor}{0 0 0}
\pscustom[linestyle=none,fillstyle=solid,fillcolor=curcolor]
{
\newpath
\moveto(690.35315674,37.28705373)
\curveto(690.35316743,37.35704805)(690.35316743,37.43704797)(690.35315674,37.52705373)
\curveto(690.34316744,37.61704779)(690.34316744,37.70204771)(690.35315674,37.78205373)
\curveto(690.35316743,37.87204754)(690.36316742,37.95204746)(690.38315674,38.02205373)
\curveto(690.40316738,38.10204731)(690.43316735,38.15704725)(690.47315674,38.18705373)
\curveto(690.52316726,38.21704719)(690.59816719,38.23704717)(690.69815674,38.24705373)
\curveto(690.788167,38.26704714)(690.89316689,38.27704713)(691.01315674,38.27705373)
\curveto(691.12316666,38.28704712)(691.23816655,38.28704712)(691.35815674,38.27705373)
\lineto(691.65815674,38.27705373)
\lineto(694.67315674,38.27705373)
\lineto(697.56815674,38.27705373)
\curveto(697.89815989,38.27704713)(698.22315956,38.27204714)(698.54315674,38.26205373)
\curveto(698.85315893,38.26204715)(699.13315865,38.22204719)(699.38315674,38.14205373)
\curveto(699.73315805,38.02204739)(700.02815776,37.86704754)(700.26815674,37.67705373)
\curveto(700.49815729,37.48704792)(700.69815709,37.24704816)(700.86815674,36.95705373)
\curveto(700.91815687,36.89704851)(700.95315683,36.83204858)(700.97315674,36.76205373)
\curveto(700.99315679,36.70204871)(701.01815677,36.63204878)(701.04815674,36.55205373)
\curveto(701.09815669,36.43204898)(701.13315665,36.30204911)(701.15315674,36.16205373)
\curveto(701.1831566,36.03204938)(701.21315657,35.89704951)(701.24315674,35.75705373)
\curveto(701.26315652,35.7070497)(701.26815652,35.65704975)(701.25815674,35.60705373)
\curveto(701.24815654,35.55704985)(701.24815654,35.50204991)(701.25815674,35.44205373)
\curveto(701.26815652,35.42204999)(701.26815652,35.39705001)(701.25815674,35.36705373)
\curveto(701.25815653,35.33705007)(701.26315652,35.3120501)(701.27315674,35.29205373)
\curveto(701.2831565,35.25205016)(701.2881565,35.19705021)(701.28815674,35.12705373)
\curveto(701.2881565,35.05705035)(701.2831565,35.00205041)(701.27315674,34.96205373)
\curveto(701.26315652,34.9120505)(701.26315652,34.85705055)(701.27315674,34.79705373)
\curveto(701.2831565,34.73705067)(701.27815651,34.68205073)(701.25815674,34.63205373)
\curveto(701.22815656,34.50205091)(701.20815658,34.37705103)(701.19815674,34.25705373)
\curveto(701.1881566,34.13705127)(701.16315662,34.02205139)(701.12315674,33.91205373)
\curveto(701.00315678,33.54205187)(700.83315695,33.22205219)(700.61315674,32.95205373)
\curveto(700.39315739,32.68205273)(700.11315767,32.47205294)(699.77315674,32.32205373)
\curveto(699.65315813,32.27205314)(699.52815826,32.22705318)(699.39815674,32.18705373)
\curveto(699.26815852,32.15705325)(699.13315865,32.12205329)(698.99315674,32.08205373)
\curveto(698.94315884,32.07205334)(698.90315888,32.06705334)(698.87315674,32.06705373)
\curveto(698.83315895,32.06705334)(698.788159,32.06205335)(698.73815674,32.05205373)
\curveto(698.70815908,32.04205337)(698.67315911,32.03705337)(698.63315674,32.03705373)
\curveto(698.5831592,32.03705337)(698.54315924,32.03205338)(698.51315674,32.02205373)
\lineto(698.34815674,32.02205373)
\curveto(698.26815952,32.00205341)(698.16815962,31.99705341)(698.04815674,32.00705373)
\curveto(697.91815987,32.01705339)(697.82815996,32.03205338)(697.77815674,32.05205373)
\curveto(697.6881601,32.07205334)(697.62316016,32.12705328)(697.58315674,32.21705373)
\curveto(697.56316022,32.24705316)(697.55816023,32.27705313)(697.56815674,32.30705373)
\curveto(697.56816022,32.33705307)(697.56316022,32.37705303)(697.55315674,32.42705373)
\curveto(697.54316024,32.46705294)(697.53816025,32.5070529)(697.53815674,32.54705373)
\lineto(697.53815674,32.69705373)
\curveto(697.53816025,32.81705259)(697.54316024,32.93705247)(697.55315674,33.05705373)
\curveto(697.55316023,33.18705222)(697.5881602,33.27705213)(697.65815674,33.32705373)
\curveto(697.71816007,33.36705204)(697.77816001,33.38705202)(697.83815674,33.38705373)
\curveto(697.89815989,33.38705202)(697.96815982,33.39705201)(698.04815674,33.41705373)
\curveto(698.07815971,33.42705198)(698.11315967,33.42705198)(698.15315674,33.41705373)
\curveto(698.1831596,33.41705199)(698.20815958,33.42205199)(698.22815674,33.43205373)
\lineto(698.43815674,33.43205373)
\curveto(698.4881593,33.45205196)(698.53815925,33.45705195)(698.58815674,33.44705373)
\curveto(698.62815916,33.44705196)(698.67315911,33.45705195)(698.72315674,33.47705373)
\curveto(698.85315893,33.5070519)(698.97815881,33.53705187)(699.09815674,33.56705373)
\curveto(699.20815858,33.59705181)(699.31315847,33.64205177)(699.41315674,33.70205373)
\curveto(699.70315808,33.87205154)(699.90815788,34.14205127)(700.02815674,34.51205373)
\curveto(700.04815774,34.56205085)(700.06315772,34.6120508)(700.07315674,34.66205373)
\curveto(700.07315771,34.72205069)(700.0831577,34.77705063)(700.10315674,34.82705373)
\lineto(700.10315674,34.90205373)
\curveto(700.11315767,34.97205044)(700.12315766,35.06705034)(700.13315674,35.18705373)
\curveto(700.13315765,35.31705009)(700.12315766,35.41704999)(700.10315674,35.48705373)
\curveto(700.0831577,35.55704985)(700.06815772,35.62704978)(700.05815674,35.69705373)
\curveto(700.03815775,35.77704963)(700.01815777,35.84704956)(699.99815674,35.90705373)
\curveto(699.83815795,36.28704912)(699.56315822,36.56204885)(699.17315674,36.73205373)
\curveto(699.04315874,36.78204863)(698.8881589,36.81704859)(698.70815674,36.83705373)
\curveto(698.52815926,36.86704854)(698.34315944,36.88204853)(698.15315674,36.88205373)
\curveto(697.95315983,36.89204852)(697.75316003,36.89204852)(697.55315674,36.88205373)
\lineto(696.98315674,36.88205373)
\lineto(692.73815674,36.88205373)
\lineto(691.19315674,36.88205373)
\curveto(691.0831667,36.88204853)(690.96316682,36.87704853)(690.83315674,36.86705373)
\curveto(690.70316708,36.85704855)(690.59816719,36.87704853)(690.51815674,36.92705373)
\curveto(690.44816734,36.98704842)(690.39816739,37.06704834)(690.36815674,37.16705373)
\curveto(690.36816742,37.18704822)(690.36816742,37.2070482)(690.36815674,37.22705373)
\curveto(690.36816742,37.24704816)(690.36316742,37.26704814)(690.35315674,37.28705373)
}
}
{
\newrgbcolor{curcolor}{0 0 0}
\pscustom[linestyle=none,fillstyle=solid,fillcolor=curcolor]
{
\newpath
\moveto(693.30815674,40.82072561)
\lineto(693.30815674,41.25572561)
\curveto(693.30816448,41.40572364)(693.34816444,41.51072354)(693.42815674,41.57072561)
\curveto(693.50816428,41.62072343)(693.60816418,41.6457234)(693.72815674,41.64572561)
\curveto(693.84816394,41.65572339)(693.96816382,41.66072339)(694.08815674,41.66072561)
\lineto(695.51315674,41.66072561)
\lineto(697.77815674,41.66072561)
\lineto(698.46815674,41.66072561)
\curveto(698.69815909,41.66072339)(698.89815889,41.68572336)(699.06815674,41.73572561)
\curveto(699.51815827,41.89572315)(699.83315795,42.19572285)(700.01315674,42.63572561)
\curveto(700.10315768,42.85572219)(700.13815765,43.12072193)(700.11815674,43.43072561)
\curveto(700.0881577,43.74072131)(700.03315775,43.99072106)(699.95315674,44.18072561)
\curveto(699.81315797,44.51072054)(699.63815815,44.77072028)(699.42815674,44.96072561)
\curveto(699.20815858,45.16071989)(698.92315886,45.31571973)(698.57315674,45.42572561)
\curveto(698.49315929,45.45571959)(698.41315937,45.47571957)(698.33315674,45.48572561)
\curveto(698.25315953,45.49571955)(698.16815962,45.51071954)(698.07815674,45.53072561)
\curveto(698.02815976,45.54071951)(697.9831598,45.54071951)(697.94315674,45.53072561)
\curveto(697.90315988,45.53071952)(697.85815993,45.54071951)(697.80815674,45.56072561)
\lineto(697.49315674,45.56072561)
\curveto(697.41316037,45.58071947)(697.32316046,45.58571946)(697.22315674,45.57572561)
\curveto(697.11316067,45.56571948)(697.01316077,45.56071949)(696.92315674,45.56072561)
\lineto(695.75315674,45.56072561)
\lineto(694.16315674,45.56072561)
\curveto(694.04316374,45.56071949)(693.91816387,45.55571949)(693.78815674,45.54572561)
\curveto(693.64816414,45.5457195)(693.53816425,45.57071948)(693.45815674,45.62072561)
\curveto(693.40816438,45.66071939)(693.37816441,45.70571934)(693.36815674,45.75572561)
\curveto(693.34816444,45.81571923)(693.32816446,45.88571916)(693.30815674,45.96572561)
\lineto(693.30815674,46.19072561)
\curveto(693.30816448,46.31071874)(693.31316447,46.41571863)(693.32315674,46.50572561)
\curveto(693.33316445,46.60571844)(693.37816441,46.68071837)(693.45815674,46.73072561)
\curveto(693.50816428,46.78071827)(693.5831642,46.80571824)(693.68315674,46.80572561)
\lineto(693.96815674,46.80572561)
\lineto(694.98815674,46.80572561)
\lineto(699.02315674,46.80572561)
\lineto(700.37315674,46.80572561)
\curveto(700.49315729,46.80571824)(700.60815718,46.80071825)(700.71815674,46.79072561)
\curveto(700.81815697,46.79071826)(700.89315689,46.75571829)(700.94315674,46.68572561)
\curveto(700.97315681,46.6457184)(700.99815679,46.58571846)(701.01815674,46.50572561)
\curveto(701.02815676,46.42571862)(701.03815675,46.33571871)(701.04815674,46.23572561)
\curveto(701.04815674,46.1457189)(701.04315674,46.05571899)(701.03315674,45.96572561)
\curveto(701.02315676,45.88571916)(701.00315678,45.82571922)(700.97315674,45.78572561)
\curveto(700.93315685,45.73571931)(700.86815692,45.69071936)(700.77815674,45.65072561)
\curveto(700.73815705,45.64071941)(700.6831571,45.63071942)(700.61315674,45.62072561)
\curveto(700.54315724,45.62071943)(700.47815731,45.61571943)(700.41815674,45.60572561)
\curveto(700.34815744,45.59571945)(700.29315749,45.57571947)(700.25315674,45.54572561)
\curveto(700.21315757,45.51571953)(700.19815759,45.47071958)(700.20815674,45.41072561)
\curveto(700.22815756,45.33071972)(700.2881575,45.2507198)(700.38815674,45.17072561)
\curveto(700.47815731,45.09071996)(700.54815724,45.01572003)(700.59815674,44.94572561)
\curveto(700.75815703,44.72572032)(700.89815689,44.47572057)(701.01815674,44.19572561)
\curveto(701.06815672,44.08572096)(701.09815669,43.97072108)(701.10815674,43.85072561)
\curveto(701.12815666,43.74072131)(701.15315663,43.62572142)(701.18315674,43.50572561)
\curveto(701.19315659,43.45572159)(701.19315659,43.40072165)(701.18315674,43.34072561)
\curveto(701.17315661,43.29072176)(701.17815661,43.24072181)(701.19815674,43.19072561)
\curveto(701.21815657,43.09072196)(701.21815657,43.00072205)(701.19815674,42.92072561)
\lineto(701.19815674,42.77072561)
\curveto(701.17815661,42.72072233)(701.16815662,42.66072239)(701.16815674,42.59072561)
\curveto(701.16815662,42.53072252)(701.16315662,42.47572257)(701.15315674,42.42572561)
\curveto(701.13315665,42.38572266)(701.12315666,42.3457227)(701.12315674,42.30572561)
\curveto(701.13315665,42.27572277)(701.12815666,42.23572281)(701.10815674,42.18572561)
\lineto(701.04815674,41.94572561)
\curveto(701.02815676,41.87572317)(700.99815679,41.80072325)(700.95815674,41.72072561)
\curveto(700.84815694,41.46072359)(700.70315708,41.24072381)(700.52315674,41.06072561)
\curveto(700.33315745,40.89072416)(700.10815768,40.7507243)(699.84815674,40.64072561)
\curveto(699.75815803,40.60072445)(699.66815812,40.57072448)(699.57815674,40.55072561)
\lineto(699.27815674,40.49072561)
\curveto(699.21815857,40.47072458)(699.16315862,40.46072459)(699.11315674,40.46072561)
\curveto(699.05315873,40.47072458)(698.9881588,40.46572458)(698.91815674,40.44572561)
\curveto(698.89815889,40.43572461)(698.87315891,40.43072462)(698.84315674,40.43072561)
\curveto(698.80315898,40.43072462)(698.76815902,40.42572462)(698.73815674,40.41572561)
\lineto(698.58815674,40.41572561)
\curveto(698.54815924,40.40572464)(698.50315928,40.40072465)(698.45315674,40.40072561)
\curveto(698.39315939,40.41072464)(698.33815945,40.41572463)(698.28815674,40.41572561)
\lineto(697.68815674,40.41572561)
\lineto(694.92815674,40.41572561)
\lineto(693.96815674,40.41572561)
\lineto(693.69815674,40.41572561)
\curveto(693.60816418,40.41572463)(693.53316425,40.43572461)(693.47315674,40.47572561)
\curveto(693.40316438,40.51572453)(693.35316443,40.59072446)(693.32315674,40.70072561)
\curveto(693.31316447,40.72072433)(693.31316447,40.74072431)(693.32315674,40.76072561)
\curveto(693.32316446,40.78072427)(693.31816447,40.80072425)(693.30815674,40.82072561)
}
}
{
\newrgbcolor{curcolor}{0 0 0}
\pscustom[linestyle=none,fillstyle=solid,fillcolor=curcolor]
{
\newpath
\moveto(693.15815674,52.39533498)
\curveto(693.13816465,53.02532975)(693.22316456,53.53032924)(693.41315674,53.91033498)
\curveto(693.60316418,54.29032848)(693.8881639,54.59532818)(694.26815674,54.82533498)
\curveto(694.36816342,54.88532789)(694.47816331,54.93032784)(694.59815674,54.96033498)
\curveto(694.70816308,55.00032777)(694.82316296,55.03532774)(694.94315674,55.06533498)
\curveto(695.13316265,55.11532766)(695.33816245,55.14532763)(695.55815674,55.15533498)
\curveto(695.77816201,55.16532761)(696.00316178,55.1703276)(696.23315674,55.17033498)
\lineto(697.83815674,55.17033498)
\lineto(700.17815674,55.17033498)
\curveto(700.34815744,55.1703276)(700.51815727,55.16532761)(700.68815674,55.15533498)
\curveto(700.85815693,55.15532762)(700.96815682,55.09032768)(701.01815674,54.96033498)
\curveto(701.03815675,54.91032786)(701.04815674,54.85532792)(701.04815674,54.79533498)
\curveto(701.05815673,54.74532803)(701.06315672,54.69032808)(701.06315674,54.63033498)
\curveto(701.06315672,54.50032827)(701.05815673,54.3753284)(701.04815674,54.25533498)
\curveto(701.04815674,54.13532864)(701.00815678,54.05032872)(700.92815674,54.00033498)
\curveto(700.85815693,53.95032882)(700.76815702,53.92532885)(700.65815674,53.92533498)
\lineto(700.32815674,53.92533498)
\lineto(699.03815674,53.92533498)
\lineto(696.59315674,53.92533498)
\curveto(696.32316146,53.92532885)(696.05816173,53.92032885)(695.79815674,53.91033498)
\curveto(695.52816226,53.90032887)(695.29816249,53.85532892)(695.10815674,53.77533498)
\curveto(694.90816288,53.69532908)(694.74816304,53.5753292)(694.62815674,53.41533498)
\curveto(694.49816329,53.25532952)(694.39816339,53.0703297)(694.32815674,52.86033498)
\curveto(694.30816348,52.80032997)(694.29816349,52.73533004)(694.29815674,52.66533498)
\curveto(694.2881635,52.60533017)(694.27316351,52.54533023)(694.25315674,52.48533498)
\curveto(694.24316354,52.43533034)(694.24316354,52.35533042)(694.25315674,52.24533498)
\curveto(694.25316353,52.14533063)(694.25816353,52.0753307)(694.26815674,52.03533498)
\curveto(694.2881635,51.99533078)(694.29816349,51.96033081)(694.29815674,51.93033498)
\curveto(694.2881635,51.90033087)(694.2881635,51.86533091)(694.29815674,51.82533498)
\curveto(694.32816346,51.69533108)(694.36316342,51.5703312)(694.40315674,51.45033498)
\curveto(694.43316335,51.34033143)(694.47816331,51.23533154)(694.53815674,51.13533498)
\curveto(694.55816323,51.09533168)(694.57816321,51.06033171)(694.59815674,51.03033498)
\curveto(694.61816317,51.00033177)(694.63816315,50.96533181)(694.65815674,50.92533498)
\curveto(694.90816288,50.5753322)(695.2831625,50.32033245)(695.78315674,50.16033498)
\curveto(695.86316192,50.13033264)(695.94816184,50.11033266)(696.03815674,50.10033498)
\curveto(696.11816167,50.09033268)(696.19816159,50.0753327)(696.27815674,50.05533498)
\curveto(696.32816146,50.03533274)(696.37816141,50.03033274)(696.42815674,50.04033498)
\curveto(696.46816132,50.05033272)(696.50816128,50.04533273)(696.54815674,50.02533498)
\lineto(696.86315674,50.02533498)
\curveto(696.89316089,50.01533276)(696.92816086,50.01033276)(696.96815674,50.01033498)
\curveto(697.00816078,50.02033275)(697.05316073,50.02533275)(697.10315674,50.02533498)
\lineto(697.55315674,50.02533498)
\lineto(698.99315674,50.02533498)
\lineto(700.31315674,50.02533498)
\lineto(700.65815674,50.02533498)
\curveto(700.76815702,50.02533275)(700.85815693,50.00033277)(700.92815674,49.95033498)
\curveto(701.00815678,49.90033287)(701.04815674,49.81033296)(701.04815674,49.68033498)
\curveto(701.05815673,49.56033321)(701.06315672,49.43533334)(701.06315674,49.30533498)
\curveto(701.06315672,49.22533355)(701.05815673,49.15033362)(701.04815674,49.08033498)
\curveto(701.03815675,49.01033376)(701.01315677,48.95033382)(700.97315674,48.90033498)
\curveto(700.92315686,48.82033395)(700.82815696,48.78033399)(700.68815674,48.78033498)
\lineto(700.28315674,48.78033498)
\lineto(698.51315674,48.78033498)
\lineto(694.88315674,48.78033498)
\lineto(693.96815674,48.78033498)
\lineto(693.69815674,48.78033498)
\curveto(693.60816418,48.78033399)(693.53816425,48.80033397)(693.48815674,48.84033498)
\curveto(693.42816436,48.8703339)(693.3881644,48.92033385)(693.36815674,48.99033498)
\curveto(693.35816443,49.03033374)(693.34816444,49.08533369)(693.33815674,49.15533498)
\curveto(693.32816446,49.23533354)(693.32316446,49.31533346)(693.32315674,49.39533498)
\curveto(693.32316446,49.4753333)(693.32816446,49.55033322)(693.33815674,49.62033498)
\curveto(693.34816444,49.70033307)(693.36316442,49.75533302)(693.38315674,49.78533498)
\curveto(693.45316433,49.89533288)(693.54316424,49.94533283)(693.65315674,49.93533498)
\curveto(693.75316403,49.92533285)(693.86816392,49.94033283)(693.99815674,49.98033498)
\curveto(694.05816373,50.00033277)(694.10816368,50.04033273)(694.14815674,50.10033498)
\curveto(694.15816363,50.22033255)(694.11316367,50.31533246)(694.01315674,50.38533498)
\curveto(693.91316387,50.46533231)(693.83316395,50.54533223)(693.77315674,50.62533498)
\curveto(693.67316411,50.76533201)(693.5831642,50.90533187)(693.50315674,51.04533498)
\curveto(693.41316437,51.19533158)(693.33816445,51.36533141)(693.27815674,51.55533498)
\curveto(693.24816454,51.63533114)(693.22816456,51.72033105)(693.21815674,51.81033498)
\curveto(693.20816458,51.91033086)(693.19316459,52.00533077)(693.17315674,52.09533498)
\curveto(693.16316462,52.14533063)(693.15816463,52.19533058)(693.15815674,52.24533498)
\lineto(693.15815674,52.39533498)
}
}
{
\newrgbcolor{curcolor}{0 0 0}
\pscustom[linestyle=none,fillstyle=solid,fillcolor=curcolor]
{
}
}
{
\newrgbcolor{curcolor}{0 0 0}
\pscustom[linestyle=none,fillstyle=solid,fillcolor=curcolor]
{
\newpath
\moveto(690.42815674,65.05510061)
\curveto(690.42816736,65.15509575)(690.43816735,65.25009566)(690.45815674,65.34010061)
\curveto(690.46816732,65.43009548)(690.49816729,65.49509541)(690.54815674,65.53510061)
\curveto(690.62816716,65.59509531)(690.73316705,65.62509528)(690.86315674,65.62510061)
\lineto(691.25315674,65.62510061)
\lineto(692.75315674,65.62510061)
\lineto(699.14315674,65.62510061)
\lineto(700.31315674,65.62510061)
\lineto(700.62815674,65.62510061)
\curveto(700.72815706,65.63509527)(700.80815698,65.62009529)(700.86815674,65.58010061)
\curveto(700.94815684,65.53009538)(700.99815679,65.45509545)(701.01815674,65.35510061)
\curveto(701.02815676,65.26509564)(701.03315675,65.15509575)(701.03315674,65.02510061)
\lineto(701.03315674,64.80010061)
\curveto(701.01315677,64.72009619)(700.99815679,64.65009626)(700.98815674,64.59010061)
\curveto(700.96815682,64.53009638)(700.92815686,64.48009643)(700.86815674,64.44010061)
\curveto(700.80815698,64.40009651)(700.73315705,64.38009653)(700.64315674,64.38010061)
\lineto(700.34315674,64.38010061)
\lineto(699.24815674,64.38010061)
\lineto(693.90815674,64.38010061)
\curveto(693.81816397,64.36009655)(693.74316404,64.34509656)(693.68315674,64.33510061)
\curveto(693.61316417,64.33509657)(693.55316423,64.3050966)(693.50315674,64.24510061)
\curveto(693.45316433,64.17509673)(693.42816436,64.08509682)(693.42815674,63.97510061)
\curveto(693.41816437,63.87509703)(693.41316437,63.76509714)(693.41315674,63.64510061)
\lineto(693.41315674,62.50510061)
\lineto(693.41315674,62.01010061)
\curveto(693.40316438,61.85009906)(693.34316444,61.74009917)(693.23315674,61.68010061)
\curveto(693.20316458,61.66009925)(693.17316461,61.65009926)(693.14315674,61.65010061)
\curveto(693.10316468,61.65009926)(693.05816473,61.64509926)(693.00815674,61.63510061)
\curveto(692.8881649,61.61509929)(692.77816501,61.62009929)(692.67815674,61.65010061)
\curveto(692.57816521,61.69009922)(692.50816528,61.74509916)(692.46815674,61.81510061)
\curveto(692.41816537,61.89509901)(692.39316539,62.01509889)(692.39315674,62.17510061)
\curveto(692.39316539,62.33509857)(692.37816541,62.47009844)(692.34815674,62.58010061)
\curveto(692.33816545,62.63009828)(692.33316545,62.68509822)(692.33315674,62.74510061)
\curveto(692.32316546,62.8050981)(692.30816548,62.86509804)(692.28815674,62.92510061)
\curveto(692.23816555,63.07509783)(692.1881656,63.22009769)(692.13815674,63.36010061)
\curveto(692.07816571,63.50009741)(692.00816578,63.63509727)(691.92815674,63.76510061)
\curveto(691.83816595,63.905097)(691.73316605,64.02509688)(691.61315674,64.12510061)
\curveto(691.49316629,64.22509668)(691.36316642,64.32009659)(691.22315674,64.41010061)
\curveto(691.12316666,64.47009644)(691.01316677,64.51509639)(690.89315674,64.54510061)
\curveto(690.77316701,64.58509632)(690.66816712,64.63509627)(690.57815674,64.69510061)
\curveto(690.51816727,64.74509616)(690.47816731,64.81509609)(690.45815674,64.90510061)
\curveto(690.44816734,64.92509598)(690.44316734,64.95009596)(690.44315674,64.98010061)
\curveto(690.44316734,65.0100959)(690.43816735,65.03509587)(690.42815674,65.05510061)
}
}
{
\newrgbcolor{curcolor}{0 0 0}
\pscustom[linestyle=none,fillstyle=solid,fillcolor=curcolor]
{
\newpath
\moveto(695.94815674,76.34470998)
\lineto(696.20315674,76.34470998)
\curveto(696.2831615,76.35470228)(696.35816143,76.34970228)(696.42815674,76.32970998)
\lineto(696.66815674,76.32970998)
\lineto(696.83315674,76.32970998)
\curveto(696.93316085,76.30970232)(697.03816075,76.29970233)(697.14815674,76.29970998)
\curveto(697.24816054,76.29970233)(697.34816044,76.28970234)(697.44815674,76.26970998)
\lineto(697.59815674,76.26970998)
\curveto(697.73816005,76.23970239)(697.87815991,76.21970241)(698.01815674,76.20970998)
\curveto(698.14815964,76.19970243)(698.27815951,76.17470246)(698.40815674,76.13470998)
\curveto(698.4881593,76.11470252)(698.57315921,76.09470254)(698.66315674,76.07470998)
\lineto(698.90315674,76.01470998)
\lineto(699.20315674,75.89470998)
\curveto(699.29315849,75.86470277)(699.3831584,75.8297028)(699.47315674,75.78970998)
\curveto(699.69315809,75.68970294)(699.90815788,75.55470308)(700.11815674,75.38470998)
\curveto(700.32815746,75.22470341)(700.49815729,75.04970358)(700.62815674,74.85970998)
\curveto(700.66815712,74.80970382)(700.70815708,74.74970388)(700.74815674,74.67970998)
\curveto(700.77815701,74.61970401)(700.81315697,74.55970407)(700.85315674,74.49970998)
\curveto(700.90315688,74.41970421)(700.94315684,74.32470431)(700.97315674,74.21470998)
\curveto(701.00315678,74.10470453)(701.03315675,73.99970463)(701.06315674,73.89970998)
\curveto(701.10315668,73.78970484)(701.12815666,73.67970495)(701.13815674,73.56970998)
\curveto(701.14815664,73.45970517)(701.16315662,73.34470529)(701.18315674,73.22470998)
\curveto(701.19315659,73.18470545)(701.19315659,73.13970549)(701.18315674,73.08970998)
\curveto(701.1831566,73.04970558)(701.1881566,73.00970562)(701.19815674,72.96970998)
\curveto(701.20815658,72.9297057)(701.21315657,72.87470576)(701.21315674,72.80470998)
\curveto(701.21315657,72.7347059)(701.20815658,72.68470595)(701.19815674,72.65470998)
\curveto(701.17815661,72.60470603)(701.17315661,72.55970607)(701.18315674,72.51970998)
\curveto(701.19315659,72.47970615)(701.19315659,72.44470619)(701.18315674,72.41470998)
\lineto(701.18315674,72.32470998)
\curveto(701.16315662,72.26470637)(701.14815664,72.19970643)(701.13815674,72.12970998)
\curveto(701.13815665,72.06970656)(701.13315665,72.00470663)(701.12315674,71.93470998)
\curveto(701.07315671,71.76470687)(701.02315676,71.60470703)(700.97315674,71.45470998)
\curveto(700.92315686,71.30470733)(700.85815693,71.15970747)(700.77815674,71.01970998)
\curveto(700.73815705,70.96970766)(700.70815708,70.91470772)(700.68815674,70.85470998)
\curveto(700.65815713,70.80470783)(700.62315716,70.75470788)(700.58315674,70.70470998)
\curveto(700.40315738,70.46470817)(700.1831576,70.26470837)(699.92315674,70.10470998)
\curveto(699.66315812,69.94470869)(699.37815841,69.80470883)(699.06815674,69.68470998)
\curveto(698.92815886,69.62470901)(698.788159,69.57970905)(698.64815674,69.54970998)
\curveto(698.49815929,69.51970911)(698.34315944,69.48470915)(698.18315674,69.44470998)
\curveto(698.07315971,69.42470921)(697.96315982,69.40970922)(697.85315674,69.39970998)
\curveto(697.74316004,69.38970924)(697.63316015,69.37470926)(697.52315674,69.35470998)
\curveto(697.4831603,69.34470929)(697.44316034,69.33970929)(697.40315674,69.33970998)
\curveto(697.36316042,69.34970928)(697.32316046,69.34970928)(697.28315674,69.33970998)
\curveto(697.23316055,69.3297093)(697.1831606,69.32470931)(697.13315674,69.32470998)
\lineto(696.96815674,69.32470998)
\curveto(696.91816087,69.30470933)(696.86816092,69.29970933)(696.81815674,69.30970998)
\curveto(696.75816103,69.31970931)(696.70316108,69.31970931)(696.65315674,69.30970998)
\curveto(696.61316117,69.29970933)(696.56816122,69.29970933)(696.51815674,69.30970998)
\curveto(696.46816132,69.31970931)(696.41816137,69.31470932)(696.36815674,69.29470998)
\curveto(696.29816149,69.27470936)(696.22316156,69.26970936)(696.14315674,69.27970998)
\curveto(696.05316173,69.28970934)(695.96816182,69.29470934)(695.88815674,69.29470998)
\curveto(695.79816199,69.29470934)(695.69816209,69.28970934)(695.58815674,69.27970998)
\curveto(695.46816232,69.26970936)(695.36816242,69.27470936)(695.28815674,69.29470998)
\lineto(695.00315674,69.29470998)
\lineto(694.37315674,69.33970998)
\curveto(694.27316351,69.34970928)(694.17816361,69.35970927)(694.08815674,69.36970998)
\lineto(693.78815674,69.39970998)
\curveto(693.73816405,69.41970921)(693.6881641,69.42470921)(693.63815674,69.41470998)
\curveto(693.57816421,69.41470922)(693.52316426,69.42470921)(693.47315674,69.44470998)
\curveto(693.30316448,69.49470914)(693.13816465,69.5347091)(692.97815674,69.56470998)
\curveto(692.80816498,69.59470904)(692.64816514,69.64470899)(692.49815674,69.71470998)
\curveto(692.03816575,69.90470873)(691.66316612,70.12470851)(691.37315674,70.37470998)
\curveto(691.0831667,70.634708)(690.83816695,70.99470764)(690.63815674,71.45470998)
\curveto(690.5881672,71.58470705)(690.55316723,71.71470692)(690.53315674,71.84470998)
\curveto(690.51316727,71.98470665)(690.4881673,72.12470651)(690.45815674,72.26470998)
\curveto(690.44816734,72.3347063)(690.44316734,72.39970623)(690.44315674,72.45970998)
\curveto(690.44316734,72.51970611)(690.43816735,72.58470605)(690.42815674,72.65470998)
\curveto(690.40816738,73.48470515)(690.55816723,74.15470448)(690.87815674,74.66470998)
\curveto(691.1881666,75.17470346)(691.62816616,75.55470308)(692.19815674,75.80470998)
\curveto(692.31816547,75.85470278)(692.44316534,75.89970273)(692.57315674,75.93970998)
\curveto(692.70316508,75.97970265)(692.83816495,76.02470261)(692.97815674,76.07470998)
\curveto(693.05816473,76.09470254)(693.14316464,76.10970252)(693.23315674,76.11970998)
\lineto(693.47315674,76.17970998)
\curveto(693.5831642,76.20970242)(693.69316409,76.22470241)(693.80315674,76.22470998)
\curveto(693.91316387,76.2347024)(694.02316376,76.24970238)(694.13315674,76.26970998)
\curveto(694.1831636,76.28970234)(694.22816356,76.29470234)(694.26815674,76.28470998)
\curveto(694.30816348,76.28470235)(694.34816344,76.28970234)(694.38815674,76.29970998)
\curveto(694.43816335,76.30970232)(694.49316329,76.30970232)(694.55315674,76.29970998)
\curveto(694.60316318,76.29970233)(694.65316313,76.30470233)(694.70315674,76.31470998)
\lineto(694.83815674,76.31470998)
\curveto(694.89816289,76.3347023)(694.96816282,76.3347023)(695.04815674,76.31470998)
\curveto(695.11816267,76.30470233)(695.1831626,76.30970232)(695.24315674,76.32970998)
\curveto(695.27316251,76.33970229)(695.31316247,76.34470229)(695.36315674,76.34470998)
\lineto(695.48315674,76.34470998)
\lineto(695.94815674,76.34470998)
\moveto(698.27315674,74.79970998)
\curveto(697.95315983,74.89970373)(697.5881602,74.95970367)(697.17815674,74.97970998)
\curveto(696.76816102,74.99970363)(696.35816143,75.00970362)(695.94815674,75.00970998)
\curveto(695.51816227,75.00970362)(695.09816269,74.99970363)(694.68815674,74.97970998)
\curveto(694.27816351,74.95970367)(693.89316389,74.91470372)(693.53315674,74.84470998)
\curveto(693.17316461,74.77470386)(692.85316493,74.66470397)(692.57315674,74.51470998)
\curveto(692.2831655,74.37470426)(692.04816574,74.17970445)(691.86815674,73.92970998)
\curveto(691.75816603,73.76970486)(691.67816611,73.58970504)(691.62815674,73.38970998)
\curveto(691.56816622,73.18970544)(691.53816625,72.94470569)(691.53815674,72.65470998)
\curveto(691.55816623,72.634706)(691.56816622,72.59970603)(691.56815674,72.54970998)
\curveto(691.55816623,72.49970613)(691.55816623,72.45970617)(691.56815674,72.42970998)
\curveto(691.5881662,72.34970628)(691.60816618,72.27470636)(691.62815674,72.20470998)
\curveto(691.63816615,72.14470649)(691.65816613,72.07970655)(691.68815674,72.00970998)
\curveto(691.80816598,71.73970689)(691.97816581,71.51970711)(692.19815674,71.34970998)
\curveto(692.40816538,71.18970744)(692.65316513,71.05470758)(692.93315674,70.94470998)
\curveto(693.04316474,70.89470774)(693.16316462,70.85470778)(693.29315674,70.82470998)
\curveto(693.41316437,70.80470783)(693.53816425,70.77970785)(693.66815674,70.74970998)
\curveto(693.71816407,70.7297079)(693.77316401,70.71970791)(693.83315674,70.71970998)
\curveto(693.8831639,70.71970791)(693.93316385,70.71470792)(693.98315674,70.70470998)
\curveto(694.07316371,70.69470794)(694.16816362,70.68470795)(694.26815674,70.67470998)
\curveto(694.35816343,70.66470797)(694.45316333,70.65470798)(694.55315674,70.64470998)
\curveto(694.63316315,70.64470799)(694.71816307,70.63970799)(694.80815674,70.62970998)
\lineto(695.04815674,70.62970998)
\lineto(695.22815674,70.62970998)
\curveto(695.25816253,70.61970801)(695.29316249,70.61470802)(695.33315674,70.61470998)
\lineto(695.46815674,70.61470998)
\lineto(695.91815674,70.61470998)
\curveto(695.99816179,70.61470802)(696.0831617,70.60970802)(696.17315674,70.59970998)
\curveto(696.25316153,70.59970803)(696.32816146,70.60970802)(696.39815674,70.62970998)
\lineto(696.66815674,70.62970998)
\curveto(696.6881611,70.629708)(696.71816107,70.62470801)(696.75815674,70.61470998)
\curveto(696.788161,70.61470802)(696.81316097,70.61970801)(696.83315674,70.62970998)
\curveto(696.93316085,70.63970799)(697.03316075,70.64470799)(697.13315674,70.64470998)
\curveto(697.22316056,70.65470798)(697.32316046,70.66470797)(697.43315674,70.67470998)
\curveto(697.55316023,70.70470793)(697.67816011,70.71970791)(697.80815674,70.71970998)
\curveto(697.92815986,70.7297079)(698.04315974,70.75470788)(698.15315674,70.79470998)
\curveto(698.45315933,70.87470776)(698.71815907,70.95970767)(698.94815674,71.04970998)
\curveto(699.17815861,71.14970748)(699.39315839,71.29470734)(699.59315674,71.48470998)
\curveto(699.79315799,71.69470694)(699.94315784,71.95970667)(700.04315674,72.27970998)
\curveto(700.06315772,72.31970631)(700.07315771,72.35470628)(700.07315674,72.38470998)
\curveto(700.06315772,72.42470621)(700.06815772,72.46970616)(700.08815674,72.51970998)
\curveto(700.09815769,72.55970607)(700.10815768,72.629706)(700.11815674,72.72970998)
\curveto(700.12815766,72.83970579)(700.12315766,72.92470571)(700.10315674,72.98470998)
\curveto(700.0831577,73.05470558)(700.07315771,73.12470551)(700.07315674,73.19470998)
\curveto(700.06315772,73.26470537)(700.04815774,73.3297053)(700.02815674,73.38970998)
\curveto(699.96815782,73.58970504)(699.8831579,73.76970486)(699.77315674,73.92970998)
\curveto(699.75315803,73.95970467)(699.73315805,73.98470465)(699.71315674,74.00470998)
\lineto(699.65315674,74.06470998)
\curveto(699.63315815,74.10470453)(699.59315819,74.15470448)(699.53315674,74.21470998)
\curveto(699.39315839,74.31470432)(699.26315852,74.39970423)(699.14315674,74.46970998)
\curveto(699.02315876,74.53970409)(698.87815891,74.60970402)(698.70815674,74.67970998)
\curveto(698.63815915,74.70970392)(698.56815922,74.7297039)(698.49815674,74.73970998)
\curveto(698.42815936,74.75970387)(698.35315943,74.77970385)(698.27315674,74.79970998)
}
}
{
\newrgbcolor{curcolor}{0 0 0}
\pscustom[linestyle=none,fillstyle=solid,fillcolor=curcolor]
{
\newpath
\moveto(699.39815674,78.63431936)
\lineto(699.39815674,79.26431936)
\lineto(699.39815674,79.45931936)
\curveto(699.39815839,79.52931683)(699.40815838,79.58931677)(699.42815674,79.63931936)
\curveto(699.46815832,79.70931665)(699.50815828,79.7593166)(699.54815674,79.78931936)
\curveto(699.59815819,79.82931653)(699.66315812,79.84931651)(699.74315674,79.84931936)
\curveto(699.82315796,79.8593165)(699.90815788,79.86431649)(699.99815674,79.86431936)
\lineto(700.71815674,79.86431936)
\curveto(701.19815659,79.86431649)(701.60815618,79.80431655)(701.94815674,79.68431936)
\curveto(702.2881555,79.56431679)(702.56315522,79.36931699)(702.77315674,79.09931936)
\curveto(702.82315496,79.02931733)(702.86815492,78.9593174)(702.90815674,78.88931936)
\curveto(702.95815483,78.82931753)(703.00315478,78.7543176)(703.04315674,78.66431936)
\curveto(703.05315473,78.64431771)(703.06315472,78.61431774)(703.07315674,78.57431936)
\curveto(703.09315469,78.53431782)(703.09815469,78.48931787)(703.08815674,78.43931936)
\curveto(703.05815473,78.34931801)(702.9831548,78.29431806)(702.86315674,78.27431936)
\curveto(702.75315503,78.2543181)(702.65815513,78.26931809)(702.57815674,78.31931936)
\curveto(702.50815528,78.34931801)(702.44315534,78.39431796)(702.38315674,78.45431936)
\curveto(702.33315545,78.52431783)(702.2831555,78.58931777)(702.23315674,78.64931936)
\curveto(702.1831556,78.71931764)(702.10815568,78.77931758)(702.00815674,78.82931936)
\curveto(701.91815587,78.88931747)(701.82815596,78.93931742)(701.73815674,78.97931936)
\curveto(701.70815608,78.99931736)(701.64815614,79.02431733)(701.55815674,79.05431936)
\curveto(701.47815631,79.08431727)(701.40815638,79.08931727)(701.34815674,79.06931936)
\curveto(701.20815658,79.03931732)(701.11815667,78.97931738)(701.07815674,78.88931936)
\curveto(701.04815674,78.80931755)(701.03315675,78.71931764)(701.03315674,78.61931936)
\curveto(701.03315675,78.51931784)(701.00815678,78.43431792)(700.95815674,78.36431936)
\curveto(700.8881569,78.27431808)(700.74815704,78.22931813)(700.53815674,78.22931936)
\lineto(699.98315674,78.22931936)
\lineto(699.75815674,78.22931936)
\curveto(699.67815811,78.23931812)(699.61315817,78.2593181)(699.56315674,78.28931936)
\curveto(699.4831583,78.34931801)(699.43815835,78.41931794)(699.42815674,78.49931936)
\curveto(699.41815837,78.51931784)(699.41315837,78.53931782)(699.41315674,78.55931936)
\curveto(699.41315837,78.58931777)(699.40815838,78.61431774)(699.39815674,78.63431936)
}
}
{
\newrgbcolor{curcolor}{0 0 0}
\pscustom[linestyle=none,fillstyle=solid,fillcolor=curcolor]
{
}
}
{
\newrgbcolor{curcolor}{0 0 0}
\pscustom[linestyle=none,fillstyle=solid,fillcolor=curcolor]
{
\newpath
\moveto(690.42815674,89.26463186)
\curveto(690.41816737,89.95462722)(690.53816725,90.55462662)(690.78815674,91.06463186)
\curveto(691.03816675,91.58462559)(691.37316641,91.9796252)(691.79315674,92.24963186)
\curveto(691.87316591,92.29962488)(691.96316582,92.34462483)(692.06315674,92.38463186)
\curveto(692.15316563,92.42462475)(692.24816554,92.46962471)(692.34815674,92.51963186)
\curveto(692.44816534,92.55962462)(692.54816524,92.58962459)(692.64815674,92.60963186)
\curveto(692.74816504,92.62962455)(692.85316493,92.64962453)(692.96315674,92.66963186)
\curveto(693.01316477,92.68962449)(693.05816473,92.69462448)(693.09815674,92.68463186)
\curveto(693.13816465,92.6746245)(693.1831646,92.6796245)(693.23315674,92.69963186)
\curveto(693.2831645,92.70962447)(693.36816442,92.71462446)(693.48815674,92.71463186)
\curveto(693.59816419,92.71462446)(693.6831641,92.70962447)(693.74315674,92.69963186)
\curveto(693.80316398,92.6796245)(693.86316392,92.66962451)(693.92315674,92.66963186)
\curveto(693.9831638,92.6796245)(694.04316374,92.6746245)(694.10315674,92.65463186)
\curveto(694.24316354,92.61462456)(694.37816341,92.5796246)(694.50815674,92.54963186)
\curveto(694.63816315,92.51962466)(694.76316302,92.4796247)(694.88315674,92.42963186)
\curveto(695.02316276,92.36962481)(695.14816264,92.29962488)(695.25815674,92.21963186)
\curveto(695.36816242,92.14962503)(695.47816231,92.0746251)(695.58815674,91.99463186)
\lineto(695.64815674,91.93463186)
\curveto(695.66816212,91.92462525)(695.6881621,91.90962527)(695.70815674,91.88963186)
\curveto(695.86816192,91.76962541)(696.01316177,91.63462554)(696.14315674,91.48463186)
\curveto(696.27316151,91.33462584)(696.39816139,91.174626)(696.51815674,91.00463186)
\curveto(696.73816105,90.69462648)(696.94316084,90.39962678)(697.13315674,90.11963186)
\curveto(697.27316051,89.88962729)(697.40816038,89.65962752)(697.53815674,89.42963186)
\curveto(697.66816012,89.20962797)(697.80315998,88.98962819)(697.94315674,88.76963186)
\curveto(698.11315967,88.51962866)(698.29315949,88.2796289)(698.48315674,88.04963186)
\curveto(698.67315911,87.82962935)(698.89815889,87.63962954)(699.15815674,87.47963186)
\curveto(699.21815857,87.43962974)(699.27815851,87.40462977)(699.33815674,87.37463186)
\curveto(699.3881584,87.34462983)(699.45315833,87.31462986)(699.53315674,87.28463186)
\curveto(699.60315818,87.26462991)(699.66315812,87.25962992)(699.71315674,87.26963186)
\curveto(699.783158,87.28962989)(699.83815795,87.32462985)(699.87815674,87.37463186)
\curveto(699.90815788,87.42462975)(699.92815786,87.48462969)(699.93815674,87.55463186)
\lineto(699.93815674,87.79463186)
\lineto(699.93815674,88.54463186)
\lineto(699.93815674,91.34963186)
\lineto(699.93815674,92.00963186)
\curveto(699.93815785,92.09962508)(699.94315784,92.18462499)(699.95315674,92.26463186)
\curveto(699.95315783,92.34462483)(699.97315781,92.40962477)(700.01315674,92.45963186)
\curveto(700.05315773,92.50962467)(700.12815766,92.54962463)(700.23815674,92.57963186)
\curveto(700.33815745,92.61962456)(700.43815735,92.62962455)(700.53815674,92.60963186)
\lineto(700.67315674,92.60963186)
\curveto(700.74315704,92.58962459)(700.80315698,92.56962461)(700.85315674,92.54963186)
\curveto(700.90315688,92.52962465)(700.94315684,92.49462468)(700.97315674,92.44463186)
\curveto(701.01315677,92.39462478)(701.03315675,92.32462485)(701.03315674,92.23463186)
\lineto(701.03315674,91.96463186)
\lineto(701.03315674,91.06463186)
\lineto(701.03315674,87.55463186)
\lineto(701.03315674,86.48963186)
\curveto(701.03315675,86.40963077)(701.03815675,86.31963086)(701.04815674,86.21963186)
\curveto(701.04815674,86.11963106)(701.03815675,86.03463114)(701.01815674,85.96463186)
\curveto(700.94815684,85.75463142)(700.76815702,85.68963149)(700.47815674,85.76963186)
\curveto(700.43815735,85.7796314)(700.40315738,85.7796314)(700.37315674,85.76963186)
\curveto(700.33315745,85.76963141)(700.2881575,85.7796314)(700.23815674,85.79963186)
\curveto(700.15815763,85.81963136)(700.07315771,85.83963134)(699.98315674,85.85963186)
\curveto(699.89315789,85.8796313)(699.80815798,85.90463127)(699.72815674,85.93463186)
\curveto(699.23815855,86.09463108)(698.82315896,86.29463088)(698.48315674,86.53463186)
\curveto(698.23315955,86.71463046)(698.00815978,86.91963026)(697.80815674,87.14963186)
\curveto(697.59816019,87.3796298)(697.40316038,87.61962956)(697.22315674,87.86963186)
\curveto(697.04316074,88.12962905)(696.87316091,88.39462878)(696.71315674,88.66463186)
\curveto(696.54316124,88.94462823)(696.36816142,89.21462796)(696.18815674,89.47463186)
\curveto(696.10816168,89.58462759)(696.03316175,89.68962749)(695.96315674,89.78963186)
\curveto(695.89316189,89.89962728)(695.81816197,90.00962717)(695.73815674,90.11963186)
\curveto(695.70816208,90.15962702)(695.67816211,90.19462698)(695.64815674,90.22463186)
\curveto(695.60816218,90.26462691)(695.57816221,90.30462687)(695.55815674,90.34463186)
\curveto(695.44816234,90.48462669)(695.32316246,90.60962657)(695.18315674,90.71963186)
\curveto(695.15316263,90.73962644)(695.12816266,90.76462641)(695.10815674,90.79463186)
\curveto(695.07816271,90.82462635)(695.04816274,90.84962633)(695.01815674,90.86963186)
\curveto(694.91816287,90.94962623)(694.81816297,91.01462616)(694.71815674,91.06463186)
\curveto(694.61816317,91.12462605)(694.50816328,91.179626)(694.38815674,91.22963186)
\curveto(694.31816347,91.25962592)(694.24316354,91.2796259)(694.16315674,91.28963186)
\lineto(693.92315674,91.34963186)
\lineto(693.83315674,91.34963186)
\curveto(693.80316398,91.35962582)(693.77316401,91.36462581)(693.74315674,91.36463186)
\curveto(693.67316411,91.38462579)(693.57816421,91.38962579)(693.45815674,91.37963186)
\curveto(693.32816446,91.3796258)(693.22816456,91.36962581)(693.15815674,91.34963186)
\curveto(693.07816471,91.32962585)(693.00316478,91.30962587)(692.93315674,91.28963186)
\curveto(692.85316493,91.2796259)(692.77316501,91.25962592)(692.69315674,91.22963186)
\curveto(692.45316533,91.11962606)(692.25316553,90.96962621)(692.09315674,90.77963186)
\curveto(691.92316586,90.59962658)(691.783166,90.3796268)(691.67315674,90.11963186)
\curveto(691.65316613,90.04962713)(691.63816615,89.9796272)(691.62815674,89.90963186)
\curveto(691.60816618,89.83962734)(691.5881662,89.76462741)(691.56815674,89.68463186)
\curveto(691.54816624,89.60462757)(691.53816625,89.49462768)(691.53815674,89.35463186)
\curveto(691.53816625,89.22462795)(691.54816624,89.11962806)(691.56815674,89.03963186)
\curveto(691.57816621,88.9796282)(691.5831662,88.92462825)(691.58315674,88.87463186)
\curveto(691.5831662,88.82462835)(691.59316619,88.7746284)(691.61315674,88.72463186)
\curveto(691.65316613,88.62462855)(691.69316609,88.52962865)(691.73315674,88.43963186)
\curveto(691.77316601,88.35962882)(691.81816597,88.2796289)(691.86815674,88.19963186)
\curveto(691.8881659,88.16962901)(691.91316587,88.13962904)(691.94315674,88.10963186)
\curveto(691.97316581,88.08962909)(691.99816579,88.06462911)(692.01815674,88.03463186)
\lineto(692.09315674,87.95963186)
\curveto(692.11316567,87.92962925)(692.13316565,87.90462927)(692.15315674,87.88463186)
\lineto(692.36315674,87.73463186)
\curveto(692.42316536,87.69462948)(692.4881653,87.64962953)(692.55815674,87.59963186)
\curveto(692.64816514,87.53962964)(692.75316503,87.48962969)(692.87315674,87.44963186)
\curveto(692.9831648,87.41962976)(693.09316469,87.38462979)(693.20315674,87.34463186)
\curveto(693.31316447,87.30462987)(693.45816433,87.2796299)(693.63815674,87.26963186)
\curveto(693.80816398,87.25962992)(693.93316385,87.22962995)(694.01315674,87.17963186)
\curveto(694.09316369,87.12963005)(694.13816365,87.05463012)(694.14815674,86.95463186)
\curveto(694.15816363,86.85463032)(694.16316362,86.74463043)(694.16315674,86.62463186)
\curveto(694.16316362,86.58463059)(694.16816362,86.54463063)(694.17815674,86.50463186)
\curveto(694.17816361,86.46463071)(694.17316361,86.42963075)(694.16315674,86.39963186)
\curveto(694.14316364,86.34963083)(694.13316365,86.29963088)(694.13315674,86.24963186)
\curveto(694.13316365,86.20963097)(694.12316366,86.16963101)(694.10315674,86.12963186)
\curveto(694.04316374,86.03963114)(693.90816388,85.99463118)(693.69815674,85.99463186)
\lineto(693.57815674,85.99463186)
\curveto(693.51816427,86.00463117)(693.45816433,86.00963117)(693.39815674,86.00963186)
\curveto(693.32816446,86.01963116)(693.26316452,86.02963115)(693.20315674,86.03963186)
\curveto(693.09316469,86.05963112)(692.99316479,86.0796311)(692.90315674,86.09963186)
\curveto(692.80316498,86.11963106)(692.70816508,86.14963103)(692.61815674,86.18963186)
\curveto(692.54816524,86.20963097)(692.4881653,86.22963095)(692.43815674,86.24963186)
\lineto(692.25815674,86.30963186)
\curveto(691.99816579,86.42963075)(691.75316603,86.58463059)(691.52315674,86.77463186)
\curveto(691.29316649,86.9746302)(691.10816668,87.18962999)(690.96815674,87.41963186)
\curveto(690.8881669,87.52962965)(690.82316696,87.64462953)(690.77315674,87.76463186)
\lineto(690.62315674,88.15463186)
\curveto(690.57316721,88.26462891)(690.54316724,88.3796288)(690.53315674,88.49963186)
\curveto(690.51316727,88.61962856)(690.4881673,88.74462843)(690.45815674,88.87463186)
\curveto(690.45816733,88.94462823)(690.45816733,89.00962817)(690.45815674,89.06963186)
\curveto(690.44816734,89.12962805)(690.43816735,89.19462798)(690.42815674,89.26463186)
}
}
{
\newrgbcolor{curcolor}{0 0 0}
\pscustom[linestyle=none,fillstyle=solid,fillcolor=curcolor]
{
\newpath
\moveto(695.94815674,101.36424123)
\lineto(696.20315674,101.36424123)
\curveto(696.2831615,101.37423353)(696.35816143,101.36923353)(696.42815674,101.34924123)
\lineto(696.66815674,101.34924123)
\lineto(696.83315674,101.34924123)
\curveto(696.93316085,101.32923357)(697.03816075,101.31923358)(697.14815674,101.31924123)
\curveto(697.24816054,101.31923358)(697.34816044,101.30923359)(697.44815674,101.28924123)
\lineto(697.59815674,101.28924123)
\curveto(697.73816005,101.25923364)(697.87815991,101.23923366)(698.01815674,101.22924123)
\curveto(698.14815964,101.21923368)(698.27815951,101.19423371)(698.40815674,101.15424123)
\curveto(698.4881593,101.13423377)(698.57315921,101.11423379)(698.66315674,101.09424123)
\lineto(698.90315674,101.03424123)
\lineto(699.20315674,100.91424123)
\curveto(699.29315849,100.88423402)(699.3831584,100.84923405)(699.47315674,100.80924123)
\curveto(699.69315809,100.70923419)(699.90815788,100.57423433)(700.11815674,100.40424123)
\curveto(700.32815746,100.24423466)(700.49815729,100.06923483)(700.62815674,99.87924123)
\curveto(700.66815712,99.82923507)(700.70815708,99.76923513)(700.74815674,99.69924123)
\curveto(700.77815701,99.63923526)(700.81315697,99.57923532)(700.85315674,99.51924123)
\curveto(700.90315688,99.43923546)(700.94315684,99.34423556)(700.97315674,99.23424123)
\curveto(701.00315678,99.12423578)(701.03315675,99.01923588)(701.06315674,98.91924123)
\curveto(701.10315668,98.80923609)(701.12815666,98.6992362)(701.13815674,98.58924123)
\curveto(701.14815664,98.47923642)(701.16315662,98.36423654)(701.18315674,98.24424123)
\curveto(701.19315659,98.2042367)(701.19315659,98.15923674)(701.18315674,98.10924123)
\curveto(701.1831566,98.06923683)(701.1881566,98.02923687)(701.19815674,97.98924123)
\curveto(701.20815658,97.94923695)(701.21315657,97.89423701)(701.21315674,97.82424123)
\curveto(701.21315657,97.75423715)(701.20815658,97.7042372)(701.19815674,97.67424123)
\curveto(701.17815661,97.62423728)(701.17315661,97.57923732)(701.18315674,97.53924123)
\curveto(701.19315659,97.4992374)(701.19315659,97.46423744)(701.18315674,97.43424123)
\lineto(701.18315674,97.34424123)
\curveto(701.16315662,97.28423762)(701.14815664,97.21923768)(701.13815674,97.14924123)
\curveto(701.13815665,97.08923781)(701.13315665,97.02423788)(701.12315674,96.95424123)
\curveto(701.07315671,96.78423812)(701.02315676,96.62423828)(700.97315674,96.47424123)
\curveto(700.92315686,96.32423858)(700.85815693,96.17923872)(700.77815674,96.03924123)
\curveto(700.73815705,95.98923891)(700.70815708,95.93423897)(700.68815674,95.87424123)
\curveto(700.65815713,95.82423908)(700.62315716,95.77423913)(700.58315674,95.72424123)
\curveto(700.40315738,95.48423942)(700.1831576,95.28423962)(699.92315674,95.12424123)
\curveto(699.66315812,94.96423994)(699.37815841,94.82424008)(699.06815674,94.70424123)
\curveto(698.92815886,94.64424026)(698.788159,94.5992403)(698.64815674,94.56924123)
\curveto(698.49815929,94.53924036)(698.34315944,94.5042404)(698.18315674,94.46424123)
\curveto(698.07315971,94.44424046)(697.96315982,94.42924047)(697.85315674,94.41924123)
\curveto(697.74316004,94.40924049)(697.63316015,94.39424051)(697.52315674,94.37424123)
\curveto(697.4831603,94.36424054)(697.44316034,94.35924054)(697.40315674,94.35924123)
\curveto(697.36316042,94.36924053)(697.32316046,94.36924053)(697.28315674,94.35924123)
\curveto(697.23316055,94.34924055)(697.1831606,94.34424056)(697.13315674,94.34424123)
\lineto(696.96815674,94.34424123)
\curveto(696.91816087,94.32424058)(696.86816092,94.31924058)(696.81815674,94.32924123)
\curveto(696.75816103,94.33924056)(696.70316108,94.33924056)(696.65315674,94.32924123)
\curveto(696.61316117,94.31924058)(696.56816122,94.31924058)(696.51815674,94.32924123)
\curveto(696.46816132,94.33924056)(696.41816137,94.33424057)(696.36815674,94.31424123)
\curveto(696.29816149,94.29424061)(696.22316156,94.28924061)(696.14315674,94.29924123)
\curveto(696.05316173,94.30924059)(695.96816182,94.31424059)(695.88815674,94.31424123)
\curveto(695.79816199,94.31424059)(695.69816209,94.30924059)(695.58815674,94.29924123)
\curveto(695.46816232,94.28924061)(695.36816242,94.29424061)(695.28815674,94.31424123)
\lineto(695.00315674,94.31424123)
\lineto(694.37315674,94.35924123)
\curveto(694.27316351,94.36924053)(694.17816361,94.37924052)(694.08815674,94.38924123)
\lineto(693.78815674,94.41924123)
\curveto(693.73816405,94.43924046)(693.6881641,94.44424046)(693.63815674,94.43424123)
\curveto(693.57816421,94.43424047)(693.52316426,94.44424046)(693.47315674,94.46424123)
\curveto(693.30316448,94.51424039)(693.13816465,94.55424035)(692.97815674,94.58424123)
\curveto(692.80816498,94.61424029)(692.64816514,94.66424024)(692.49815674,94.73424123)
\curveto(692.03816575,94.92423998)(691.66316612,95.14423976)(691.37315674,95.39424123)
\curveto(691.0831667,95.65423925)(690.83816695,96.01423889)(690.63815674,96.47424123)
\curveto(690.5881672,96.6042383)(690.55316723,96.73423817)(690.53315674,96.86424123)
\curveto(690.51316727,97.0042379)(690.4881673,97.14423776)(690.45815674,97.28424123)
\curveto(690.44816734,97.35423755)(690.44316734,97.41923748)(690.44315674,97.47924123)
\curveto(690.44316734,97.53923736)(690.43816735,97.6042373)(690.42815674,97.67424123)
\curveto(690.40816738,98.5042364)(690.55816723,99.17423573)(690.87815674,99.68424123)
\curveto(691.1881666,100.19423471)(691.62816616,100.57423433)(692.19815674,100.82424123)
\curveto(692.31816547,100.87423403)(692.44316534,100.91923398)(692.57315674,100.95924123)
\curveto(692.70316508,100.9992339)(692.83816495,101.04423386)(692.97815674,101.09424123)
\curveto(693.05816473,101.11423379)(693.14316464,101.12923377)(693.23315674,101.13924123)
\lineto(693.47315674,101.19924123)
\curveto(693.5831642,101.22923367)(693.69316409,101.24423366)(693.80315674,101.24424123)
\curveto(693.91316387,101.25423365)(694.02316376,101.26923363)(694.13315674,101.28924123)
\curveto(694.1831636,101.30923359)(694.22816356,101.31423359)(694.26815674,101.30424123)
\curveto(694.30816348,101.3042336)(694.34816344,101.30923359)(694.38815674,101.31924123)
\curveto(694.43816335,101.32923357)(694.49316329,101.32923357)(694.55315674,101.31924123)
\curveto(694.60316318,101.31923358)(694.65316313,101.32423358)(694.70315674,101.33424123)
\lineto(694.83815674,101.33424123)
\curveto(694.89816289,101.35423355)(694.96816282,101.35423355)(695.04815674,101.33424123)
\curveto(695.11816267,101.32423358)(695.1831626,101.32923357)(695.24315674,101.34924123)
\curveto(695.27316251,101.35923354)(695.31316247,101.36423354)(695.36315674,101.36424123)
\lineto(695.48315674,101.36424123)
\lineto(695.94815674,101.36424123)
\moveto(698.27315674,99.81924123)
\curveto(697.95315983,99.91923498)(697.5881602,99.97923492)(697.17815674,99.99924123)
\curveto(696.76816102,100.01923488)(696.35816143,100.02923487)(695.94815674,100.02924123)
\curveto(695.51816227,100.02923487)(695.09816269,100.01923488)(694.68815674,99.99924123)
\curveto(694.27816351,99.97923492)(693.89316389,99.93423497)(693.53315674,99.86424123)
\curveto(693.17316461,99.79423511)(692.85316493,99.68423522)(692.57315674,99.53424123)
\curveto(692.2831655,99.39423551)(692.04816574,99.1992357)(691.86815674,98.94924123)
\curveto(691.75816603,98.78923611)(691.67816611,98.60923629)(691.62815674,98.40924123)
\curveto(691.56816622,98.20923669)(691.53816625,97.96423694)(691.53815674,97.67424123)
\curveto(691.55816623,97.65423725)(691.56816622,97.61923728)(691.56815674,97.56924123)
\curveto(691.55816623,97.51923738)(691.55816623,97.47923742)(691.56815674,97.44924123)
\curveto(691.5881662,97.36923753)(691.60816618,97.29423761)(691.62815674,97.22424123)
\curveto(691.63816615,97.16423774)(691.65816613,97.0992378)(691.68815674,97.02924123)
\curveto(691.80816598,96.75923814)(691.97816581,96.53923836)(692.19815674,96.36924123)
\curveto(692.40816538,96.20923869)(692.65316513,96.07423883)(692.93315674,95.96424123)
\curveto(693.04316474,95.91423899)(693.16316462,95.87423903)(693.29315674,95.84424123)
\curveto(693.41316437,95.82423908)(693.53816425,95.7992391)(693.66815674,95.76924123)
\curveto(693.71816407,95.74923915)(693.77316401,95.73923916)(693.83315674,95.73924123)
\curveto(693.8831639,95.73923916)(693.93316385,95.73423917)(693.98315674,95.72424123)
\curveto(694.07316371,95.71423919)(694.16816362,95.7042392)(694.26815674,95.69424123)
\curveto(694.35816343,95.68423922)(694.45316333,95.67423923)(694.55315674,95.66424123)
\curveto(694.63316315,95.66423924)(694.71816307,95.65923924)(694.80815674,95.64924123)
\lineto(695.04815674,95.64924123)
\lineto(695.22815674,95.64924123)
\curveto(695.25816253,95.63923926)(695.29316249,95.63423927)(695.33315674,95.63424123)
\lineto(695.46815674,95.63424123)
\lineto(695.91815674,95.63424123)
\curveto(695.99816179,95.63423927)(696.0831617,95.62923927)(696.17315674,95.61924123)
\curveto(696.25316153,95.61923928)(696.32816146,95.62923927)(696.39815674,95.64924123)
\lineto(696.66815674,95.64924123)
\curveto(696.6881611,95.64923925)(696.71816107,95.64423926)(696.75815674,95.63424123)
\curveto(696.788161,95.63423927)(696.81316097,95.63923926)(696.83315674,95.64924123)
\curveto(696.93316085,95.65923924)(697.03316075,95.66423924)(697.13315674,95.66424123)
\curveto(697.22316056,95.67423923)(697.32316046,95.68423922)(697.43315674,95.69424123)
\curveto(697.55316023,95.72423918)(697.67816011,95.73923916)(697.80815674,95.73924123)
\curveto(697.92815986,95.74923915)(698.04315974,95.77423913)(698.15315674,95.81424123)
\curveto(698.45315933,95.89423901)(698.71815907,95.97923892)(698.94815674,96.06924123)
\curveto(699.17815861,96.16923873)(699.39315839,96.31423859)(699.59315674,96.50424123)
\curveto(699.79315799,96.71423819)(699.94315784,96.97923792)(700.04315674,97.29924123)
\curveto(700.06315772,97.33923756)(700.07315771,97.37423753)(700.07315674,97.40424123)
\curveto(700.06315772,97.44423746)(700.06815772,97.48923741)(700.08815674,97.53924123)
\curveto(700.09815769,97.57923732)(700.10815768,97.64923725)(700.11815674,97.74924123)
\curveto(700.12815766,97.85923704)(700.12315766,97.94423696)(700.10315674,98.00424123)
\curveto(700.0831577,98.07423683)(700.07315771,98.14423676)(700.07315674,98.21424123)
\curveto(700.06315772,98.28423662)(700.04815774,98.34923655)(700.02815674,98.40924123)
\curveto(699.96815782,98.60923629)(699.8831579,98.78923611)(699.77315674,98.94924123)
\curveto(699.75315803,98.97923592)(699.73315805,99.0042359)(699.71315674,99.02424123)
\lineto(699.65315674,99.08424123)
\curveto(699.63315815,99.12423578)(699.59315819,99.17423573)(699.53315674,99.23424123)
\curveto(699.39315839,99.33423557)(699.26315852,99.41923548)(699.14315674,99.48924123)
\curveto(699.02315876,99.55923534)(698.87815891,99.62923527)(698.70815674,99.69924123)
\curveto(698.63815915,99.72923517)(698.56815922,99.74923515)(698.49815674,99.75924123)
\curveto(698.42815936,99.77923512)(698.35315943,99.7992351)(698.27315674,99.81924123)
}
}
{
\newrgbcolor{curcolor}{0 0 0}
\pscustom[linestyle=none,fillstyle=solid,fillcolor=curcolor]
{
\newpath
\moveto(690.42815674,106.77385061)
\curveto(690.42816736,106.87384575)(690.43816735,106.96884566)(690.45815674,107.05885061)
\curveto(690.46816732,107.14884548)(690.49816729,107.21384541)(690.54815674,107.25385061)
\curveto(690.62816716,107.31384531)(690.73316705,107.34384528)(690.86315674,107.34385061)
\lineto(691.25315674,107.34385061)
\lineto(692.75315674,107.34385061)
\lineto(699.14315674,107.34385061)
\lineto(700.31315674,107.34385061)
\lineto(700.62815674,107.34385061)
\curveto(700.72815706,107.35384527)(700.80815698,107.33884529)(700.86815674,107.29885061)
\curveto(700.94815684,107.24884538)(700.99815679,107.17384545)(701.01815674,107.07385061)
\curveto(701.02815676,106.98384564)(701.03315675,106.87384575)(701.03315674,106.74385061)
\lineto(701.03315674,106.51885061)
\curveto(701.01315677,106.43884619)(700.99815679,106.36884626)(700.98815674,106.30885061)
\curveto(700.96815682,106.24884638)(700.92815686,106.19884643)(700.86815674,106.15885061)
\curveto(700.80815698,106.11884651)(700.73315705,106.09884653)(700.64315674,106.09885061)
\lineto(700.34315674,106.09885061)
\lineto(699.24815674,106.09885061)
\lineto(693.90815674,106.09885061)
\curveto(693.81816397,106.07884655)(693.74316404,106.06384656)(693.68315674,106.05385061)
\curveto(693.61316417,106.05384657)(693.55316423,106.0238466)(693.50315674,105.96385061)
\curveto(693.45316433,105.89384673)(693.42816436,105.80384682)(693.42815674,105.69385061)
\curveto(693.41816437,105.59384703)(693.41316437,105.48384714)(693.41315674,105.36385061)
\lineto(693.41315674,104.22385061)
\lineto(693.41315674,103.72885061)
\curveto(693.40316438,103.56884906)(693.34316444,103.45884917)(693.23315674,103.39885061)
\curveto(693.20316458,103.37884925)(693.17316461,103.36884926)(693.14315674,103.36885061)
\curveto(693.10316468,103.36884926)(693.05816473,103.36384926)(693.00815674,103.35385061)
\curveto(692.8881649,103.33384929)(692.77816501,103.33884929)(692.67815674,103.36885061)
\curveto(692.57816521,103.40884922)(692.50816528,103.46384916)(692.46815674,103.53385061)
\curveto(692.41816537,103.61384901)(692.39316539,103.73384889)(692.39315674,103.89385061)
\curveto(692.39316539,104.05384857)(692.37816541,104.18884844)(692.34815674,104.29885061)
\curveto(692.33816545,104.34884828)(692.33316545,104.40384822)(692.33315674,104.46385061)
\curveto(692.32316546,104.5238481)(692.30816548,104.58384804)(692.28815674,104.64385061)
\curveto(692.23816555,104.79384783)(692.1881656,104.93884769)(692.13815674,105.07885061)
\curveto(692.07816571,105.21884741)(692.00816578,105.35384727)(691.92815674,105.48385061)
\curveto(691.83816595,105.623847)(691.73316605,105.74384688)(691.61315674,105.84385061)
\curveto(691.49316629,105.94384668)(691.36316642,106.03884659)(691.22315674,106.12885061)
\curveto(691.12316666,106.18884644)(691.01316677,106.23384639)(690.89315674,106.26385061)
\curveto(690.77316701,106.30384632)(690.66816712,106.35384627)(690.57815674,106.41385061)
\curveto(690.51816727,106.46384616)(690.47816731,106.53384609)(690.45815674,106.62385061)
\curveto(690.44816734,106.64384598)(690.44316734,106.66884596)(690.44315674,106.69885061)
\curveto(690.44316734,106.7288459)(690.43816735,106.75384587)(690.42815674,106.77385061)
}
}
{
\newrgbcolor{curcolor}{0 0 0}
\pscustom[linestyle=none,fillstyle=solid,fillcolor=curcolor]
{
\newpath
\moveto(690.42815674,115.12345998)
\curveto(690.42816736,115.22345513)(690.43816735,115.31845503)(690.45815674,115.40845998)
\curveto(690.46816732,115.49845485)(690.49816729,115.56345479)(690.54815674,115.60345998)
\curveto(690.62816716,115.66345469)(690.73316705,115.69345466)(690.86315674,115.69345998)
\lineto(691.25315674,115.69345998)
\lineto(692.75315674,115.69345998)
\lineto(699.14315674,115.69345998)
\lineto(700.31315674,115.69345998)
\lineto(700.62815674,115.69345998)
\curveto(700.72815706,115.70345465)(700.80815698,115.68845466)(700.86815674,115.64845998)
\curveto(700.94815684,115.59845475)(700.99815679,115.52345483)(701.01815674,115.42345998)
\curveto(701.02815676,115.33345502)(701.03315675,115.22345513)(701.03315674,115.09345998)
\lineto(701.03315674,114.86845998)
\curveto(701.01315677,114.78845556)(700.99815679,114.71845563)(700.98815674,114.65845998)
\curveto(700.96815682,114.59845575)(700.92815686,114.5484558)(700.86815674,114.50845998)
\curveto(700.80815698,114.46845588)(700.73315705,114.4484559)(700.64315674,114.44845998)
\lineto(700.34315674,114.44845998)
\lineto(699.24815674,114.44845998)
\lineto(693.90815674,114.44845998)
\curveto(693.81816397,114.42845592)(693.74316404,114.41345594)(693.68315674,114.40345998)
\curveto(693.61316417,114.40345595)(693.55316423,114.37345598)(693.50315674,114.31345998)
\curveto(693.45316433,114.24345611)(693.42816436,114.1534562)(693.42815674,114.04345998)
\curveto(693.41816437,113.94345641)(693.41316437,113.83345652)(693.41315674,113.71345998)
\lineto(693.41315674,112.57345998)
\lineto(693.41315674,112.07845998)
\curveto(693.40316438,111.91845843)(693.34316444,111.80845854)(693.23315674,111.74845998)
\curveto(693.20316458,111.72845862)(693.17316461,111.71845863)(693.14315674,111.71845998)
\curveto(693.10316468,111.71845863)(693.05816473,111.71345864)(693.00815674,111.70345998)
\curveto(692.8881649,111.68345867)(692.77816501,111.68845866)(692.67815674,111.71845998)
\curveto(692.57816521,111.75845859)(692.50816528,111.81345854)(692.46815674,111.88345998)
\curveto(692.41816537,111.96345839)(692.39316539,112.08345827)(692.39315674,112.24345998)
\curveto(692.39316539,112.40345795)(692.37816541,112.53845781)(692.34815674,112.64845998)
\curveto(692.33816545,112.69845765)(692.33316545,112.7534576)(692.33315674,112.81345998)
\curveto(692.32316546,112.87345748)(692.30816548,112.93345742)(692.28815674,112.99345998)
\curveto(692.23816555,113.14345721)(692.1881656,113.28845706)(692.13815674,113.42845998)
\curveto(692.07816571,113.56845678)(692.00816578,113.70345665)(691.92815674,113.83345998)
\curveto(691.83816595,113.97345638)(691.73316605,114.09345626)(691.61315674,114.19345998)
\curveto(691.49316629,114.29345606)(691.36316642,114.38845596)(691.22315674,114.47845998)
\curveto(691.12316666,114.53845581)(691.01316677,114.58345577)(690.89315674,114.61345998)
\curveto(690.77316701,114.6534557)(690.66816712,114.70345565)(690.57815674,114.76345998)
\curveto(690.51816727,114.81345554)(690.47816731,114.88345547)(690.45815674,114.97345998)
\curveto(690.44816734,114.99345536)(690.44316734,115.01845533)(690.44315674,115.04845998)
\curveto(690.44316734,115.07845527)(690.43816735,115.10345525)(690.42815674,115.12345998)
}
}
{
\newrgbcolor{curcolor}{0 0 0}
\pscustom[linestyle=none,fillstyle=solid,fillcolor=curcolor]
{
\newpath
\moveto(711.26447266,37.28705373)
\curveto(711.26448335,37.35704805)(711.26448335,37.43704797)(711.26447266,37.52705373)
\curveto(711.25448336,37.61704779)(711.25448336,37.70204771)(711.26447266,37.78205373)
\curveto(711.26448335,37.87204754)(711.27448334,37.95204746)(711.29447266,38.02205373)
\curveto(711.3144833,38.10204731)(711.34448327,38.15704725)(711.38447266,38.18705373)
\curveto(711.43448318,38.21704719)(711.50948311,38.23704717)(711.60947266,38.24705373)
\curveto(711.69948292,38.26704714)(711.80448281,38.27704713)(711.92447266,38.27705373)
\curveto(712.03448258,38.28704712)(712.14948247,38.28704712)(712.26947266,38.27705373)
\lineto(712.56947266,38.27705373)
\lineto(715.58447266,38.27705373)
\lineto(718.47947266,38.27705373)
\curveto(718.80947581,38.27704713)(719.13447548,38.27204714)(719.45447266,38.26205373)
\curveto(719.76447485,38.26204715)(720.04447457,38.22204719)(720.29447266,38.14205373)
\curveto(720.64447397,38.02204739)(720.93947368,37.86704754)(721.17947266,37.67705373)
\curveto(721.40947321,37.48704792)(721.60947301,37.24704816)(721.77947266,36.95705373)
\curveto(721.82947279,36.89704851)(721.86447275,36.83204858)(721.88447266,36.76205373)
\curveto(721.90447271,36.70204871)(721.92947269,36.63204878)(721.95947266,36.55205373)
\curveto(722.00947261,36.43204898)(722.04447257,36.30204911)(722.06447266,36.16205373)
\curveto(722.09447252,36.03204938)(722.12447249,35.89704951)(722.15447266,35.75705373)
\curveto(722.17447244,35.7070497)(722.17947244,35.65704975)(722.16947266,35.60705373)
\curveto(722.15947246,35.55704985)(722.15947246,35.50204991)(722.16947266,35.44205373)
\curveto(722.17947244,35.42204999)(722.17947244,35.39705001)(722.16947266,35.36705373)
\curveto(722.16947245,35.33705007)(722.17447244,35.3120501)(722.18447266,35.29205373)
\curveto(722.19447242,35.25205016)(722.19947242,35.19705021)(722.19947266,35.12705373)
\curveto(722.19947242,35.05705035)(722.19447242,35.00205041)(722.18447266,34.96205373)
\curveto(722.17447244,34.9120505)(722.17447244,34.85705055)(722.18447266,34.79705373)
\curveto(722.19447242,34.73705067)(722.18947243,34.68205073)(722.16947266,34.63205373)
\curveto(722.13947248,34.50205091)(722.1194725,34.37705103)(722.10947266,34.25705373)
\curveto(722.09947252,34.13705127)(722.07447254,34.02205139)(722.03447266,33.91205373)
\curveto(721.9144727,33.54205187)(721.74447287,33.22205219)(721.52447266,32.95205373)
\curveto(721.30447331,32.68205273)(721.02447359,32.47205294)(720.68447266,32.32205373)
\curveto(720.56447405,32.27205314)(720.43947418,32.22705318)(720.30947266,32.18705373)
\curveto(720.17947444,32.15705325)(720.04447457,32.12205329)(719.90447266,32.08205373)
\curveto(719.85447476,32.07205334)(719.8144748,32.06705334)(719.78447266,32.06705373)
\curveto(719.74447487,32.06705334)(719.69947492,32.06205335)(719.64947266,32.05205373)
\curveto(719.619475,32.04205337)(719.58447503,32.03705337)(719.54447266,32.03705373)
\curveto(719.49447512,32.03705337)(719.45447516,32.03205338)(719.42447266,32.02205373)
\lineto(719.25947266,32.02205373)
\curveto(719.17947544,32.00205341)(719.07947554,31.99705341)(718.95947266,32.00705373)
\curveto(718.82947579,32.01705339)(718.73947588,32.03205338)(718.68947266,32.05205373)
\curveto(718.59947602,32.07205334)(718.53447608,32.12705328)(718.49447266,32.21705373)
\curveto(718.47447614,32.24705316)(718.46947615,32.27705313)(718.47947266,32.30705373)
\curveto(718.47947614,32.33705307)(718.47447614,32.37705303)(718.46447266,32.42705373)
\curveto(718.45447616,32.46705294)(718.44947617,32.5070529)(718.44947266,32.54705373)
\lineto(718.44947266,32.69705373)
\curveto(718.44947617,32.81705259)(718.45447616,32.93705247)(718.46447266,33.05705373)
\curveto(718.46447615,33.18705222)(718.49947612,33.27705213)(718.56947266,33.32705373)
\curveto(718.62947599,33.36705204)(718.68947593,33.38705202)(718.74947266,33.38705373)
\curveto(718.80947581,33.38705202)(718.87947574,33.39705201)(718.95947266,33.41705373)
\curveto(718.98947563,33.42705198)(719.02447559,33.42705198)(719.06447266,33.41705373)
\curveto(719.09447552,33.41705199)(719.1194755,33.42205199)(719.13947266,33.43205373)
\lineto(719.34947266,33.43205373)
\curveto(719.39947522,33.45205196)(719.44947517,33.45705195)(719.49947266,33.44705373)
\curveto(719.53947508,33.44705196)(719.58447503,33.45705195)(719.63447266,33.47705373)
\curveto(719.76447485,33.5070519)(719.88947473,33.53705187)(720.00947266,33.56705373)
\curveto(720.1194745,33.59705181)(720.22447439,33.64205177)(720.32447266,33.70205373)
\curveto(720.614474,33.87205154)(720.8194738,34.14205127)(720.93947266,34.51205373)
\curveto(720.95947366,34.56205085)(720.97447364,34.6120508)(720.98447266,34.66205373)
\curveto(720.98447363,34.72205069)(720.99447362,34.77705063)(721.01447266,34.82705373)
\lineto(721.01447266,34.90205373)
\curveto(721.02447359,34.97205044)(721.03447358,35.06705034)(721.04447266,35.18705373)
\curveto(721.04447357,35.31705009)(721.03447358,35.41704999)(721.01447266,35.48705373)
\curveto(720.99447362,35.55704985)(720.97947364,35.62704978)(720.96947266,35.69705373)
\curveto(720.94947367,35.77704963)(720.92947369,35.84704956)(720.90947266,35.90705373)
\curveto(720.74947387,36.28704912)(720.47447414,36.56204885)(720.08447266,36.73205373)
\curveto(719.95447466,36.78204863)(719.79947482,36.81704859)(719.61947266,36.83705373)
\curveto(719.43947518,36.86704854)(719.25447536,36.88204853)(719.06447266,36.88205373)
\curveto(718.86447575,36.89204852)(718.66447595,36.89204852)(718.46447266,36.88205373)
\lineto(717.89447266,36.88205373)
\lineto(713.64947266,36.88205373)
\lineto(712.10447266,36.88205373)
\curveto(711.99448262,36.88204853)(711.87448274,36.87704853)(711.74447266,36.86705373)
\curveto(711.614483,36.85704855)(711.50948311,36.87704853)(711.42947266,36.92705373)
\curveto(711.35948326,36.98704842)(711.30948331,37.06704834)(711.27947266,37.16705373)
\curveto(711.27948334,37.18704822)(711.27948334,37.2070482)(711.27947266,37.22705373)
\curveto(711.27948334,37.24704816)(711.27448334,37.26704814)(711.26447266,37.28705373)
}
}
{
\newrgbcolor{curcolor}{0 0 0}
\pscustom[linestyle=none,fillstyle=solid,fillcolor=curcolor]
{
\newpath
\moveto(714.21947266,40.82072561)
\lineto(714.21947266,41.25572561)
\curveto(714.2194804,41.40572364)(714.25948036,41.51072354)(714.33947266,41.57072561)
\curveto(714.4194802,41.62072343)(714.5194801,41.6457234)(714.63947266,41.64572561)
\curveto(714.75947986,41.65572339)(714.87947974,41.66072339)(714.99947266,41.66072561)
\lineto(716.42447266,41.66072561)
\lineto(718.68947266,41.66072561)
\lineto(719.37947266,41.66072561)
\curveto(719.60947501,41.66072339)(719.80947481,41.68572336)(719.97947266,41.73572561)
\curveto(720.42947419,41.89572315)(720.74447387,42.19572285)(720.92447266,42.63572561)
\curveto(721.0144736,42.85572219)(721.04947357,43.12072193)(721.02947266,43.43072561)
\curveto(720.99947362,43.74072131)(720.94447367,43.99072106)(720.86447266,44.18072561)
\curveto(720.72447389,44.51072054)(720.54947407,44.77072028)(720.33947266,44.96072561)
\curveto(720.1194745,45.16071989)(719.83447478,45.31571973)(719.48447266,45.42572561)
\curveto(719.40447521,45.45571959)(719.32447529,45.47571957)(719.24447266,45.48572561)
\curveto(719.16447545,45.49571955)(719.07947554,45.51071954)(718.98947266,45.53072561)
\curveto(718.93947568,45.54071951)(718.89447572,45.54071951)(718.85447266,45.53072561)
\curveto(718.8144758,45.53071952)(718.76947585,45.54071951)(718.71947266,45.56072561)
\lineto(718.40447266,45.56072561)
\curveto(718.32447629,45.58071947)(718.23447638,45.58571946)(718.13447266,45.57572561)
\curveto(718.02447659,45.56571948)(717.92447669,45.56071949)(717.83447266,45.56072561)
\lineto(716.66447266,45.56072561)
\lineto(715.07447266,45.56072561)
\curveto(714.95447966,45.56071949)(714.82947979,45.55571949)(714.69947266,45.54572561)
\curveto(714.55948006,45.5457195)(714.44948017,45.57071948)(714.36947266,45.62072561)
\curveto(714.3194803,45.66071939)(714.28948033,45.70571934)(714.27947266,45.75572561)
\curveto(714.25948036,45.81571923)(714.23948038,45.88571916)(714.21947266,45.96572561)
\lineto(714.21947266,46.19072561)
\curveto(714.2194804,46.31071874)(714.22448039,46.41571863)(714.23447266,46.50572561)
\curveto(714.24448037,46.60571844)(714.28948033,46.68071837)(714.36947266,46.73072561)
\curveto(714.4194802,46.78071827)(714.49448012,46.80571824)(714.59447266,46.80572561)
\lineto(714.87947266,46.80572561)
\lineto(715.89947266,46.80572561)
\lineto(719.93447266,46.80572561)
\lineto(721.28447266,46.80572561)
\curveto(721.40447321,46.80571824)(721.5194731,46.80071825)(721.62947266,46.79072561)
\curveto(721.72947289,46.79071826)(721.80447281,46.75571829)(721.85447266,46.68572561)
\curveto(721.88447273,46.6457184)(721.90947271,46.58571846)(721.92947266,46.50572561)
\curveto(721.93947268,46.42571862)(721.94947267,46.33571871)(721.95947266,46.23572561)
\curveto(721.95947266,46.1457189)(721.95447266,46.05571899)(721.94447266,45.96572561)
\curveto(721.93447268,45.88571916)(721.9144727,45.82571922)(721.88447266,45.78572561)
\curveto(721.84447277,45.73571931)(721.77947284,45.69071936)(721.68947266,45.65072561)
\curveto(721.64947297,45.64071941)(721.59447302,45.63071942)(721.52447266,45.62072561)
\curveto(721.45447316,45.62071943)(721.38947323,45.61571943)(721.32947266,45.60572561)
\curveto(721.25947336,45.59571945)(721.20447341,45.57571947)(721.16447266,45.54572561)
\curveto(721.12447349,45.51571953)(721.10947351,45.47071958)(721.11947266,45.41072561)
\curveto(721.13947348,45.33071972)(721.19947342,45.2507198)(721.29947266,45.17072561)
\curveto(721.38947323,45.09071996)(721.45947316,45.01572003)(721.50947266,44.94572561)
\curveto(721.66947295,44.72572032)(721.80947281,44.47572057)(721.92947266,44.19572561)
\curveto(721.97947264,44.08572096)(722.00947261,43.97072108)(722.01947266,43.85072561)
\curveto(722.03947258,43.74072131)(722.06447255,43.62572142)(722.09447266,43.50572561)
\curveto(722.10447251,43.45572159)(722.10447251,43.40072165)(722.09447266,43.34072561)
\curveto(722.08447253,43.29072176)(722.08947253,43.24072181)(722.10947266,43.19072561)
\curveto(722.12947249,43.09072196)(722.12947249,43.00072205)(722.10947266,42.92072561)
\lineto(722.10947266,42.77072561)
\curveto(722.08947253,42.72072233)(722.07947254,42.66072239)(722.07947266,42.59072561)
\curveto(722.07947254,42.53072252)(722.07447254,42.47572257)(722.06447266,42.42572561)
\curveto(722.04447257,42.38572266)(722.03447258,42.3457227)(722.03447266,42.30572561)
\curveto(722.04447257,42.27572277)(722.03947258,42.23572281)(722.01947266,42.18572561)
\lineto(721.95947266,41.94572561)
\curveto(721.93947268,41.87572317)(721.90947271,41.80072325)(721.86947266,41.72072561)
\curveto(721.75947286,41.46072359)(721.614473,41.24072381)(721.43447266,41.06072561)
\curveto(721.24447337,40.89072416)(721.0194736,40.7507243)(720.75947266,40.64072561)
\curveto(720.66947395,40.60072445)(720.57947404,40.57072448)(720.48947266,40.55072561)
\lineto(720.18947266,40.49072561)
\curveto(720.12947449,40.47072458)(720.07447454,40.46072459)(720.02447266,40.46072561)
\curveto(719.96447465,40.47072458)(719.89947472,40.46572458)(719.82947266,40.44572561)
\curveto(719.80947481,40.43572461)(719.78447483,40.43072462)(719.75447266,40.43072561)
\curveto(719.7144749,40.43072462)(719.67947494,40.42572462)(719.64947266,40.41572561)
\lineto(719.49947266,40.41572561)
\curveto(719.45947516,40.40572464)(719.4144752,40.40072465)(719.36447266,40.40072561)
\curveto(719.30447531,40.41072464)(719.24947537,40.41572463)(719.19947266,40.41572561)
\lineto(718.59947266,40.41572561)
\lineto(715.83947266,40.41572561)
\lineto(714.87947266,40.41572561)
\lineto(714.60947266,40.41572561)
\curveto(714.5194801,40.41572463)(714.44448017,40.43572461)(714.38447266,40.47572561)
\curveto(714.3144803,40.51572453)(714.26448035,40.59072446)(714.23447266,40.70072561)
\curveto(714.22448039,40.72072433)(714.22448039,40.74072431)(714.23447266,40.76072561)
\curveto(714.23448038,40.78072427)(714.22948039,40.80072425)(714.21947266,40.82072561)
}
}
{
\newrgbcolor{curcolor}{0 0 0}
\pscustom[linestyle=none,fillstyle=solid,fillcolor=curcolor]
{
\newpath
\moveto(714.06947266,52.39533498)
\curveto(714.04948057,53.02532975)(714.13448048,53.53032924)(714.32447266,53.91033498)
\curveto(714.5144801,54.29032848)(714.79947982,54.59532818)(715.17947266,54.82533498)
\curveto(715.27947934,54.88532789)(715.38947923,54.93032784)(715.50947266,54.96033498)
\curveto(715.619479,55.00032777)(715.73447888,55.03532774)(715.85447266,55.06533498)
\curveto(716.04447857,55.11532766)(716.24947837,55.14532763)(716.46947266,55.15533498)
\curveto(716.68947793,55.16532761)(716.9144777,55.1703276)(717.14447266,55.17033498)
\lineto(718.74947266,55.17033498)
\lineto(721.08947266,55.17033498)
\curveto(721.25947336,55.1703276)(721.42947319,55.16532761)(721.59947266,55.15533498)
\curveto(721.76947285,55.15532762)(721.87947274,55.09032768)(721.92947266,54.96033498)
\curveto(721.94947267,54.91032786)(721.95947266,54.85532792)(721.95947266,54.79533498)
\curveto(721.96947265,54.74532803)(721.97447264,54.69032808)(721.97447266,54.63033498)
\curveto(721.97447264,54.50032827)(721.96947265,54.3753284)(721.95947266,54.25533498)
\curveto(721.95947266,54.13532864)(721.9194727,54.05032872)(721.83947266,54.00033498)
\curveto(721.76947285,53.95032882)(721.67947294,53.92532885)(721.56947266,53.92533498)
\lineto(721.23947266,53.92533498)
\lineto(719.94947266,53.92533498)
\lineto(717.50447266,53.92533498)
\curveto(717.23447738,53.92532885)(716.96947765,53.92032885)(716.70947266,53.91033498)
\curveto(716.43947818,53.90032887)(716.20947841,53.85532892)(716.01947266,53.77533498)
\curveto(715.8194788,53.69532908)(715.65947896,53.5753292)(715.53947266,53.41533498)
\curveto(715.40947921,53.25532952)(715.30947931,53.0703297)(715.23947266,52.86033498)
\curveto(715.2194794,52.80032997)(715.20947941,52.73533004)(715.20947266,52.66533498)
\curveto(715.19947942,52.60533017)(715.18447943,52.54533023)(715.16447266,52.48533498)
\curveto(715.15447946,52.43533034)(715.15447946,52.35533042)(715.16447266,52.24533498)
\curveto(715.16447945,52.14533063)(715.16947945,52.0753307)(715.17947266,52.03533498)
\curveto(715.19947942,51.99533078)(715.20947941,51.96033081)(715.20947266,51.93033498)
\curveto(715.19947942,51.90033087)(715.19947942,51.86533091)(715.20947266,51.82533498)
\curveto(715.23947938,51.69533108)(715.27447934,51.5703312)(715.31447266,51.45033498)
\curveto(715.34447927,51.34033143)(715.38947923,51.23533154)(715.44947266,51.13533498)
\curveto(715.46947915,51.09533168)(715.48947913,51.06033171)(715.50947266,51.03033498)
\curveto(715.52947909,51.00033177)(715.54947907,50.96533181)(715.56947266,50.92533498)
\curveto(715.8194788,50.5753322)(716.19447842,50.32033245)(716.69447266,50.16033498)
\curveto(716.77447784,50.13033264)(716.85947776,50.11033266)(716.94947266,50.10033498)
\curveto(717.02947759,50.09033268)(717.10947751,50.0753327)(717.18947266,50.05533498)
\curveto(717.23947738,50.03533274)(717.28947733,50.03033274)(717.33947266,50.04033498)
\curveto(717.37947724,50.05033272)(717.4194772,50.04533273)(717.45947266,50.02533498)
\lineto(717.77447266,50.02533498)
\curveto(717.80447681,50.01533276)(717.83947678,50.01033276)(717.87947266,50.01033498)
\curveto(717.9194767,50.02033275)(717.96447665,50.02533275)(718.01447266,50.02533498)
\lineto(718.46447266,50.02533498)
\lineto(719.90447266,50.02533498)
\lineto(721.22447266,50.02533498)
\lineto(721.56947266,50.02533498)
\curveto(721.67947294,50.02533275)(721.76947285,50.00033277)(721.83947266,49.95033498)
\curveto(721.9194727,49.90033287)(721.95947266,49.81033296)(721.95947266,49.68033498)
\curveto(721.96947265,49.56033321)(721.97447264,49.43533334)(721.97447266,49.30533498)
\curveto(721.97447264,49.22533355)(721.96947265,49.15033362)(721.95947266,49.08033498)
\curveto(721.94947267,49.01033376)(721.92447269,48.95033382)(721.88447266,48.90033498)
\curveto(721.83447278,48.82033395)(721.73947288,48.78033399)(721.59947266,48.78033498)
\lineto(721.19447266,48.78033498)
\lineto(719.42447266,48.78033498)
\lineto(715.79447266,48.78033498)
\lineto(714.87947266,48.78033498)
\lineto(714.60947266,48.78033498)
\curveto(714.5194801,48.78033399)(714.44948017,48.80033397)(714.39947266,48.84033498)
\curveto(714.33948028,48.8703339)(714.29948032,48.92033385)(714.27947266,48.99033498)
\curveto(714.26948035,49.03033374)(714.25948036,49.08533369)(714.24947266,49.15533498)
\curveto(714.23948038,49.23533354)(714.23448038,49.31533346)(714.23447266,49.39533498)
\curveto(714.23448038,49.4753333)(714.23948038,49.55033322)(714.24947266,49.62033498)
\curveto(714.25948036,49.70033307)(714.27448034,49.75533302)(714.29447266,49.78533498)
\curveto(714.36448025,49.89533288)(714.45448016,49.94533283)(714.56447266,49.93533498)
\curveto(714.66447995,49.92533285)(714.77947984,49.94033283)(714.90947266,49.98033498)
\curveto(714.96947965,50.00033277)(715.0194796,50.04033273)(715.05947266,50.10033498)
\curveto(715.06947955,50.22033255)(715.02447959,50.31533246)(714.92447266,50.38533498)
\curveto(714.82447979,50.46533231)(714.74447987,50.54533223)(714.68447266,50.62533498)
\curveto(714.58448003,50.76533201)(714.49448012,50.90533187)(714.41447266,51.04533498)
\curveto(714.32448029,51.19533158)(714.24948037,51.36533141)(714.18947266,51.55533498)
\curveto(714.15948046,51.63533114)(714.13948048,51.72033105)(714.12947266,51.81033498)
\curveto(714.1194805,51.91033086)(714.10448051,52.00533077)(714.08447266,52.09533498)
\curveto(714.07448054,52.14533063)(714.06948055,52.19533058)(714.06947266,52.24533498)
\lineto(714.06947266,52.39533498)
}
}
{
\newrgbcolor{curcolor}{0 0 0}
\pscustom[linestyle=none,fillstyle=solid,fillcolor=curcolor]
{
}
}
{
\newrgbcolor{curcolor}{0 0 0}
\pscustom[linestyle=none,fillstyle=solid,fillcolor=curcolor]
{
\newpath
\moveto(711.33947266,65.05510061)
\curveto(711.33948328,65.15509575)(711.34948327,65.25009566)(711.36947266,65.34010061)
\curveto(711.37948324,65.43009548)(711.40948321,65.49509541)(711.45947266,65.53510061)
\curveto(711.53948308,65.59509531)(711.64448297,65.62509528)(711.77447266,65.62510061)
\lineto(712.16447266,65.62510061)
\lineto(713.66447266,65.62510061)
\lineto(720.05447266,65.62510061)
\lineto(721.22447266,65.62510061)
\lineto(721.53947266,65.62510061)
\curveto(721.63947298,65.63509527)(721.7194729,65.62009529)(721.77947266,65.58010061)
\curveto(721.85947276,65.53009538)(721.90947271,65.45509545)(721.92947266,65.35510061)
\curveto(721.93947268,65.26509564)(721.94447267,65.15509575)(721.94447266,65.02510061)
\lineto(721.94447266,64.80010061)
\curveto(721.92447269,64.72009619)(721.90947271,64.65009626)(721.89947266,64.59010061)
\curveto(721.87947274,64.53009638)(721.83947278,64.48009643)(721.77947266,64.44010061)
\curveto(721.7194729,64.40009651)(721.64447297,64.38009653)(721.55447266,64.38010061)
\lineto(721.25447266,64.38010061)
\lineto(720.15947266,64.38010061)
\lineto(714.81947266,64.38010061)
\curveto(714.72947989,64.36009655)(714.65447996,64.34509656)(714.59447266,64.33510061)
\curveto(714.52448009,64.33509657)(714.46448015,64.3050966)(714.41447266,64.24510061)
\curveto(714.36448025,64.17509673)(714.33948028,64.08509682)(714.33947266,63.97510061)
\curveto(714.32948029,63.87509703)(714.32448029,63.76509714)(714.32447266,63.64510061)
\lineto(714.32447266,62.50510061)
\lineto(714.32447266,62.01010061)
\curveto(714.3144803,61.85009906)(714.25448036,61.74009917)(714.14447266,61.68010061)
\curveto(714.1144805,61.66009925)(714.08448053,61.65009926)(714.05447266,61.65010061)
\curveto(714.0144806,61.65009926)(713.96948065,61.64509926)(713.91947266,61.63510061)
\curveto(713.79948082,61.61509929)(713.68948093,61.62009929)(713.58947266,61.65010061)
\curveto(713.48948113,61.69009922)(713.4194812,61.74509916)(713.37947266,61.81510061)
\curveto(713.32948129,61.89509901)(713.30448131,62.01509889)(713.30447266,62.17510061)
\curveto(713.30448131,62.33509857)(713.28948133,62.47009844)(713.25947266,62.58010061)
\curveto(713.24948137,62.63009828)(713.24448137,62.68509822)(713.24447266,62.74510061)
\curveto(713.23448138,62.8050981)(713.2194814,62.86509804)(713.19947266,62.92510061)
\curveto(713.14948147,63.07509783)(713.09948152,63.22009769)(713.04947266,63.36010061)
\curveto(712.98948163,63.50009741)(712.9194817,63.63509727)(712.83947266,63.76510061)
\curveto(712.74948187,63.905097)(712.64448197,64.02509688)(712.52447266,64.12510061)
\curveto(712.40448221,64.22509668)(712.27448234,64.32009659)(712.13447266,64.41010061)
\curveto(712.03448258,64.47009644)(711.92448269,64.51509639)(711.80447266,64.54510061)
\curveto(711.68448293,64.58509632)(711.57948304,64.63509627)(711.48947266,64.69510061)
\curveto(711.42948319,64.74509616)(711.38948323,64.81509609)(711.36947266,64.90510061)
\curveto(711.35948326,64.92509598)(711.35448326,64.95009596)(711.35447266,64.98010061)
\curveto(711.35448326,65.0100959)(711.34948327,65.03509587)(711.33947266,65.05510061)
}
}
{
\newrgbcolor{curcolor}{0 0 0}
\pscustom[linestyle=none,fillstyle=solid,fillcolor=curcolor]
{
\newpath
\moveto(718.44947266,76.22470998)
\curveto(718.49947612,76.29470234)(718.56947605,76.3347023)(718.65947266,76.34470998)
\curveto(718.74947587,76.36470227)(718.85447576,76.37470226)(718.97447266,76.37470998)
\curveto(719.02447559,76.37470226)(719.07447554,76.36970226)(719.12447266,76.35970998)
\curveto(719.17447544,76.35970227)(719.2194754,76.34970228)(719.25947266,76.32970998)
\curveto(719.34947527,76.29970233)(719.40947521,76.23970239)(719.43947266,76.14970998)
\curveto(719.45947516,76.06970256)(719.46947515,75.97470266)(719.46947266,75.86470998)
\lineto(719.46947266,75.54970998)
\curveto(719.45947516,75.43970319)(719.46947515,75.3347033)(719.49947266,75.23470998)
\curveto(719.52947509,75.09470354)(719.60947501,75.00470363)(719.73947266,74.96470998)
\curveto(719.80947481,74.94470369)(719.89447472,74.9347037)(719.99447266,74.93470998)
\lineto(720.26447266,74.93470998)
\lineto(721.20947266,74.93470998)
\lineto(721.53947266,74.93470998)
\curveto(721.64947297,74.9347037)(721.73447288,74.91470372)(721.79447266,74.87470998)
\curveto(721.85447276,74.8347038)(721.89447272,74.78470385)(721.91447266,74.72470998)
\curveto(721.92447269,74.67470396)(721.93947268,74.60970402)(721.95947266,74.52970998)
\lineto(721.95947266,74.33470998)
\curveto(721.95947266,74.21470442)(721.95447266,74.10970452)(721.94447266,74.01970998)
\curveto(721.92447269,73.9297047)(721.87447274,73.85970477)(721.79447266,73.80970998)
\curveto(721.74447287,73.77970485)(721.67447294,73.76470487)(721.58447266,73.76470998)
\lineto(721.28447266,73.76470998)
\lineto(720.24947266,73.76470998)
\curveto(720.08947453,73.76470487)(719.94447467,73.75470488)(719.81447266,73.73470998)
\curveto(719.67447494,73.72470491)(719.57947504,73.66970496)(719.52947266,73.56970998)
\curveto(719.50947511,73.51970511)(719.49447512,73.44970518)(719.48447266,73.35970998)
\curveto(719.47447514,73.27970535)(719.46947515,73.18970544)(719.46947266,73.08970998)
\lineto(719.46947266,72.80470998)
\lineto(719.46947266,72.56470998)
\lineto(719.46947266,70.29970998)
\curveto(719.46947515,70.20970842)(719.47447514,70.10470853)(719.48447266,69.98470998)
\lineto(719.48447266,69.65470998)
\curveto(719.48447513,69.54470909)(719.47447514,69.44470919)(719.45447266,69.35470998)
\curveto(719.43447518,69.26470937)(719.39947522,69.20470943)(719.34947266,69.17470998)
\curveto(719.27947534,69.12470951)(719.18447543,69.09970953)(719.06447266,69.09970998)
\lineto(718.71947266,69.09970998)
\lineto(718.44947266,69.09970998)
\curveto(718.27947634,69.13970949)(718.13947648,69.19470944)(718.02947266,69.26470998)
\curveto(717.9194767,69.3347093)(717.80447681,69.41470922)(717.68447266,69.50470998)
\lineto(717.14447266,69.86470998)
\curveto(716.5144781,70.30470833)(715.89447872,70.73970789)(715.28447266,71.16970998)
\lineto(713.42447266,72.48970998)
\curveto(713.19448142,72.64970598)(712.97448164,72.80470583)(712.76447266,72.95470998)
\curveto(712.54448207,73.10470553)(712.3194823,73.25970537)(712.08947266,73.41970998)
\curveto(712.0194826,73.46970516)(711.95448266,73.51970511)(711.89447266,73.56970998)
\curveto(711.82448279,73.61970501)(711.74948287,73.66970496)(711.66947266,73.71970998)
\lineto(711.57947266,73.77970998)
\curveto(711.53948308,73.80970482)(711.50948311,73.83970479)(711.48947266,73.86970998)
\curveto(711.45948316,73.90970472)(711.43948318,73.94970468)(711.42947266,73.98970998)
\curveto(711.40948321,74.0297046)(711.38948323,74.07470456)(711.36947266,74.12470998)
\curveto(711.36948325,74.14470449)(711.37448324,74.16470447)(711.38447266,74.18470998)
\curveto(711.38448323,74.21470442)(711.37448324,74.23970439)(711.35447266,74.25970998)
\curveto(711.35448326,74.38970424)(711.35948326,74.50970412)(711.36947266,74.61970998)
\curveto(711.37948324,74.7297039)(711.42448319,74.80970382)(711.50447266,74.85970998)
\curveto(711.55448306,74.89970373)(711.62448299,74.91970371)(711.71447266,74.91970998)
\curveto(711.80448281,74.9297037)(711.89948272,74.9347037)(711.99947266,74.93470998)
\lineto(717.45947266,74.93470998)
\curveto(717.52947709,74.9347037)(717.60447701,74.9297037)(717.68447266,74.91970998)
\curveto(717.75447686,74.91970371)(717.82447679,74.92470371)(717.89447266,74.93470998)
\lineto(717.99947266,74.93470998)
\curveto(718.04947657,74.95470368)(718.10447651,74.96970366)(718.16447266,74.97970998)
\curveto(718.2144764,74.98970364)(718.25447636,75.01470362)(718.28447266,75.05470998)
\curveto(718.33447628,75.12470351)(718.36447625,75.20970342)(718.37447266,75.30970998)
\lineto(718.37447266,75.63970998)
\curveto(718.37447624,75.74970288)(718.37947624,75.85470278)(718.38947266,75.95470998)
\curveto(718.38947623,76.06470257)(718.40947621,76.15470248)(718.44947266,76.22470998)
\moveto(718.25447266,73.65970998)
\curveto(718.14447647,73.73970489)(717.97447664,73.77470486)(717.74447266,73.76470998)
\lineto(717.12947266,73.76470998)
\lineto(714.65447266,73.76470998)
\lineto(714.33947266,73.76470998)
\curveto(714.2194804,73.77470486)(714.1194805,73.76970486)(714.03947266,73.74970998)
\lineto(713.88947266,73.74970998)
\curveto(713.79948082,73.74970488)(713.7144809,73.7347049)(713.63447266,73.70470998)
\curveto(713.614481,73.69470494)(713.60448101,73.68470495)(713.60447266,73.67470998)
\lineto(713.55947266,73.62970998)
\curveto(713.54948107,73.60970502)(713.54448107,73.57970505)(713.54447266,73.53970998)
\curveto(713.56448105,73.51970511)(713.57948104,73.49970513)(713.58947266,73.47970998)
\curveto(713.58948103,73.46970516)(713.59448102,73.45470518)(713.60447266,73.43470998)
\curveto(713.65448096,73.37470526)(713.72448089,73.31470532)(713.81447266,73.25470998)
\curveto(713.90448071,73.19470544)(713.98448063,73.13970549)(714.05447266,73.08970998)
\curveto(714.19448042,72.98970564)(714.33948028,72.89470574)(714.48947266,72.80470998)
\curveto(714.62947999,72.71470592)(714.76947985,72.61970601)(714.90947266,72.51970998)
\lineto(715.68947266,71.97970998)
\curveto(715.94947867,71.80970682)(716.20947841,71.634707)(716.46947266,71.45470998)
\curveto(716.57947804,71.37470726)(716.68447793,71.29970733)(716.78447266,71.22970998)
\lineto(717.08447266,71.01970998)
\curveto(717.16447745,70.96970766)(717.23947738,70.91970771)(717.30947266,70.86970998)
\curveto(717.37947724,70.8297078)(717.45447716,70.78470785)(717.53447266,70.73470998)
\curveto(717.59447702,70.68470795)(717.65947696,70.634708)(717.72947266,70.58470998)
\curveto(717.78947683,70.54470809)(717.85947676,70.50470813)(717.93947266,70.46470998)
\curveto(717.99947662,70.42470821)(718.06947655,70.39970823)(718.14947266,70.38970998)
\curveto(718.2194764,70.37970825)(718.27447634,70.41470822)(718.31447266,70.49470998)
\curveto(718.36447625,70.56470807)(718.38947623,70.67470796)(718.38947266,70.82470998)
\curveto(718.37947624,70.98470765)(718.37447624,71.11970751)(718.37447266,71.22970998)
\lineto(718.37447266,72.90970998)
\lineto(718.37447266,73.34470998)
\curveto(718.37447624,73.49470514)(718.33447628,73.59970503)(718.25447266,73.65970998)
}
}
{
\newrgbcolor{curcolor}{0 0 0}
\pscustom[linestyle=none,fillstyle=solid,fillcolor=curcolor]
{
\newpath
\moveto(720.30947266,78.63431936)
\lineto(720.30947266,79.26431936)
\lineto(720.30947266,79.45931936)
\curveto(720.30947431,79.52931683)(720.3194743,79.58931677)(720.33947266,79.63931936)
\curveto(720.37947424,79.70931665)(720.4194742,79.7593166)(720.45947266,79.78931936)
\curveto(720.50947411,79.82931653)(720.57447404,79.84931651)(720.65447266,79.84931936)
\curveto(720.73447388,79.8593165)(720.8194738,79.86431649)(720.90947266,79.86431936)
\lineto(721.62947266,79.86431936)
\curveto(722.10947251,79.86431649)(722.5194721,79.80431655)(722.85947266,79.68431936)
\curveto(723.19947142,79.56431679)(723.47447114,79.36931699)(723.68447266,79.09931936)
\curveto(723.73447088,79.02931733)(723.77947084,78.9593174)(723.81947266,78.88931936)
\curveto(723.86947075,78.82931753)(723.9144707,78.7543176)(723.95447266,78.66431936)
\curveto(723.96447065,78.64431771)(723.97447064,78.61431774)(723.98447266,78.57431936)
\curveto(724.00447061,78.53431782)(724.00947061,78.48931787)(723.99947266,78.43931936)
\curveto(723.96947065,78.34931801)(723.89447072,78.29431806)(723.77447266,78.27431936)
\curveto(723.66447095,78.2543181)(723.56947105,78.26931809)(723.48947266,78.31931936)
\curveto(723.4194712,78.34931801)(723.35447126,78.39431796)(723.29447266,78.45431936)
\curveto(723.24447137,78.52431783)(723.19447142,78.58931777)(723.14447266,78.64931936)
\curveto(723.09447152,78.71931764)(723.0194716,78.77931758)(722.91947266,78.82931936)
\curveto(722.82947179,78.88931747)(722.73947188,78.93931742)(722.64947266,78.97931936)
\curveto(722.619472,78.99931736)(722.55947206,79.02431733)(722.46947266,79.05431936)
\curveto(722.38947223,79.08431727)(722.3194723,79.08931727)(722.25947266,79.06931936)
\curveto(722.1194725,79.03931732)(722.02947259,78.97931738)(721.98947266,78.88931936)
\curveto(721.95947266,78.80931755)(721.94447267,78.71931764)(721.94447266,78.61931936)
\curveto(721.94447267,78.51931784)(721.9194727,78.43431792)(721.86947266,78.36431936)
\curveto(721.79947282,78.27431808)(721.65947296,78.22931813)(721.44947266,78.22931936)
\lineto(720.89447266,78.22931936)
\lineto(720.66947266,78.22931936)
\curveto(720.58947403,78.23931812)(720.52447409,78.2593181)(720.47447266,78.28931936)
\curveto(720.39447422,78.34931801)(720.34947427,78.41931794)(720.33947266,78.49931936)
\curveto(720.32947429,78.51931784)(720.32447429,78.53931782)(720.32447266,78.55931936)
\curveto(720.32447429,78.58931777)(720.3194743,78.61431774)(720.30947266,78.63431936)
}
}
{
\newrgbcolor{curcolor}{0 0 0}
\pscustom[linestyle=none,fillstyle=solid,fillcolor=curcolor]
{
}
}
{
\newrgbcolor{curcolor}{0 0 0}
\pscustom[linestyle=none,fillstyle=solid,fillcolor=curcolor]
{
\newpath
\moveto(711.33947266,89.26463186)
\curveto(711.32948329,89.95462722)(711.44948317,90.55462662)(711.69947266,91.06463186)
\curveto(711.94948267,91.58462559)(712.28448233,91.9796252)(712.70447266,92.24963186)
\curveto(712.78448183,92.29962488)(712.87448174,92.34462483)(712.97447266,92.38463186)
\curveto(713.06448155,92.42462475)(713.15948146,92.46962471)(713.25947266,92.51963186)
\curveto(713.35948126,92.55962462)(713.45948116,92.58962459)(713.55947266,92.60963186)
\curveto(713.65948096,92.62962455)(713.76448085,92.64962453)(713.87447266,92.66963186)
\curveto(713.92448069,92.68962449)(713.96948065,92.69462448)(714.00947266,92.68463186)
\curveto(714.04948057,92.6746245)(714.09448052,92.6796245)(714.14447266,92.69963186)
\curveto(714.19448042,92.70962447)(714.27948034,92.71462446)(714.39947266,92.71463186)
\curveto(714.50948011,92.71462446)(714.59448002,92.70962447)(714.65447266,92.69963186)
\curveto(714.7144799,92.6796245)(714.77447984,92.66962451)(714.83447266,92.66963186)
\curveto(714.89447972,92.6796245)(714.95447966,92.6746245)(715.01447266,92.65463186)
\curveto(715.15447946,92.61462456)(715.28947933,92.5796246)(715.41947266,92.54963186)
\curveto(715.54947907,92.51962466)(715.67447894,92.4796247)(715.79447266,92.42963186)
\curveto(715.93447868,92.36962481)(716.05947856,92.29962488)(716.16947266,92.21963186)
\curveto(716.27947834,92.14962503)(716.38947823,92.0746251)(716.49947266,91.99463186)
\lineto(716.55947266,91.93463186)
\curveto(716.57947804,91.92462525)(716.59947802,91.90962527)(716.61947266,91.88963186)
\curveto(716.77947784,91.76962541)(716.92447769,91.63462554)(717.05447266,91.48463186)
\curveto(717.18447743,91.33462584)(717.30947731,91.174626)(717.42947266,91.00463186)
\curveto(717.64947697,90.69462648)(717.85447676,90.39962678)(718.04447266,90.11963186)
\curveto(718.18447643,89.88962729)(718.3194763,89.65962752)(718.44947266,89.42963186)
\curveto(718.57947604,89.20962797)(718.7144759,88.98962819)(718.85447266,88.76963186)
\curveto(719.02447559,88.51962866)(719.20447541,88.2796289)(719.39447266,88.04963186)
\curveto(719.58447503,87.82962935)(719.80947481,87.63962954)(720.06947266,87.47963186)
\curveto(720.12947449,87.43962974)(720.18947443,87.40462977)(720.24947266,87.37463186)
\curveto(720.29947432,87.34462983)(720.36447425,87.31462986)(720.44447266,87.28463186)
\curveto(720.5144741,87.26462991)(720.57447404,87.25962992)(720.62447266,87.26963186)
\curveto(720.69447392,87.28962989)(720.74947387,87.32462985)(720.78947266,87.37463186)
\curveto(720.8194738,87.42462975)(720.83947378,87.48462969)(720.84947266,87.55463186)
\lineto(720.84947266,87.79463186)
\lineto(720.84947266,88.54463186)
\lineto(720.84947266,91.34963186)
\lineto(720.84947266,92.00963186)
\curveto(720.84947377,92.09962508)(720.85447376,92.18462499)(720.86447266,92.26463186)
\curveto(720.86447375,92.34462483)(720.88447373,92.40962477)(720.92447266,92.45963186)
\curveto(720.96447365,92.50962467)(721.03947358,92.54962463)(721.14947266,92.57963186)
\curveto(721.24947337,92.61962456)(721.34947327,92.62962455)(721.44947266,92.60963186)
\lineto(721.58447266,92.60963186)
\curveto(721.65447296,92.58962459)(721.7144729,92.56962461)(721.76447266,92.54963186)
\curveto(721.8144728,92.52962465)(721.85447276,92.49462468)(721.88447266,92.44463186)
\curveto(721.92447269,92.39462478)(721.94447267,92.32462485)(721.94447266,92.23463186)
\lineto(721.94447266,91.96463186)
\lineto(721.94447266,91.06463186)
\lineto(721.94447266,87.55463186)
\lineto(721.94447266,86.48963186)
\curveto(721.94447267,86.40963077)(721.94947267,86.31963086)(721.95947266,86.21963186)
\curveto(721.95947266,86.11963106)(721.94947267,86.03463114)(721.92947266,85.96463186)
\curveto(721.85947276,85.75463142)(721.67947294,85.68963149)(721.38947266,85.76963186)
\curveto(721.34947327,85.7796314)(721.3144733,85.7796314)(721.28447266,85.76963186)
\curveto(721.24447337,85.76963141)(721.19947342,85.7796314)(721.14947266,85.79963186)
\curveto(721.06947355,85.81963136)(720.98447363,85.83963134)(720.89447266,85.85963186)
\curveto(720.80447381,85.8796313)(720.7194739,85.90463127)(720.63947266,85.93463186)
\curveto(720.14947447,86.09463108)(719.73447488,86.29463088)(719.39447266,86.53463186)
\curveto(719.14447547,86.71463046)(718.9194757,86.91963026)(718.71947266,87.14963186)
\curveto(718.50947611,87.3796298)(718.3144763,87.61962956)(718.13447266,87.86963186)
\curveto(717.95447666,88.12962905)(717.78447683,88.39462878)(717.62447266,88.66463186)
\curveto(717.45447716,88.94462823)(717.27947734,89.21462796)(717.09947266,89.47463186)
\curveto(717.0194776,89.58462759)(716.94447767,89.68962749)(716.87447266,89.78963186)
\curveto(716.80447781,89.89962728)(716.72947789,90.00962717)(716.64947266,90.11963186)
\curveto(716.619478,90.15962702)(716.58947803,90.19462698)(716.55947266,90.22463186)
\curveto(716.5194781,90.26462691)(716.48947813,90.30462687)(716.46947266,90.34463186)
\curveto(716.35947826,90.48462669)(716.23447838,90.60962657)(716.09447266,90.71963186)
\curveto(716.06447855,90.73962644)(716.03947858,90.76462641)(716.01947266,90.79463186)
\curveto(715.98947863,90.82462635)(715.95947866,90.84962633)(715.92947266,90.86963186)
\curveto(715.82947879,90.94962623)(715.72947889,91.01462616)(715.62947266,91.06463186)
\curveto(715.52947909,91.12462605)(715.4194792,91.179626)(715.29947266,91.22963186)
\curveto(715.22947939,91.25962592)(715.15447946,91.2796259)(715.07447266,91.28963186)
\lineto(714.83447266,91.34963186)
\lineto(714.74447266,91.34963186)
\curveto(714.7144799,91.35962582)(714.68447993,91.36462581)(714.65447266,91.36463186)
\curveto(714.58448003,91.38462579)(714.48948013,91.38962579)(714.36947266,91.37963186)
\curveto(714.23948038,91.3796258)(714.13948048,91.36962581)(714.06947266,91.34963186)
\curveto(713.98948063,91.32962585)(713.9144807,91.30962587)(713.84447266,91.28963186)
\curveto(713.76448085,91.2796259)(713.68448093,91.25962592)(713.60447266,91.22963186)
\curveto(713.36448125,91.11962606)(713.16448145,90.96962621)(713.00447266,90.77963186)
\curveto(712.83448178,90.59962658)(712.69448192,90.3796268)(712.58447266,90.11963186)
\curveto(712.56448205,90.04962713)(712.54948207,89.9796272)(712.53947266,89.90963186)
\curveto(712.5194821,89.83962734)(712.49948212,89.76462741)(712.47947266,89.68463186)
\curveto(712.45948216,89.60462757)(712.44948217,89.49462768)(712.44947266,89.35463186)
\curveto(712.44948217,89.22462795)(712.45948216,89.11962806)(712.47947266,89.03963186)
\curveto(712.48948213,88.9796282)(712.49448212,88.92462825)(712.49447266,88.87463186)
\curveto(712.49448212,88.82462835)(712.50448211,88.7746284)(712.52447266,88.72463186)
\curveto(712.56448205,88.62462855)(712.60448201,88.52962865)(712.64447266,88.43963186)
\curveto(712.68448193,88.35962882)(712.72948189,88.2796289)(712.77947266,88.19963186)
\curveto(712.79948182,88.16962901)(712.82448179,88.13962904)(712.85447266,88.10963186)
\curveto(712.88448173,88.08962909)(712.90948171,88.06462911)(712.92947266,88.03463186)
\lineto(713.00447266,87.95963186)
\curveto(713.02448159,87.92962925)(713.04448157,87.90462927)(713.06447266,87.88463186)
\lineto(713.27447266,87.73463186)
\curveto(713.33448128,87.69462948)(713.39948122,87.64962953)(713.46947266,87.59963186)
\curveto(713.55948106,87.53962964)(713.66448095,87.48962969)(713.78447266,87.44963186)
\curveto(713.89448072,87.41962976)(714.00448061,87.38462979)(714.11447266,87.34463186)
\curveto(714.22448039,87.30462987)(714.36948025,87.2796299)(714.54947266,87.26963186)
\curveto(714.7194799,87.25962992)(714.84447977,87.22962995)(714.92447266,87.17963186)
\curveto(715.00447961,87.12963005)(715.04947957,87.05463012)(715.05947266,86.95463186)
\curveto(715.06947955,86.85463032)(715.07447954,86.74463043)(715.07447266,86.62463186)
\curveto(715.07447954,86.58463059)(715.07947954,86.54463063)(715.08947266,86.50463186)
\curveto(715.08947953,86.46463071)(715.08447953,86.42963075)(715.07447266,86.39963186)
\curveto(715.05447956,86.34963083)(715.04447957,86.29963088)(715.04447266,86.24963186)
\curveto(715.04447957,86.20963097)(715.03447958,86.16963101)(715.01447266,86.12963186)
\curveto(714.95447966,86.03963114)(714.8194798,85.99463118)(714.60947266,85.99463186)
\lineto(714.48947266,85.99463186)
\curveto(714.42948019,86.00463117)(714.36948025,86.00963117)(714.30947266,86.00963186)
\curveto(714.23948038,86.01963116)(714.17448044,86.02963115)(714.11447266,86.03963186)
\curveto(714.00448061,86.05963112)(713.90448071,86.0796311)(713.81447266,86.09963186)
\curveto(713.7144809,86.11963106)(713.619481,86.14963103)(713.52947266,86.18963186)
\curveto(713.45948116,86.20963097)(713.39948122,86.22963095)(713.34947266,86.24963186)
\lineto(713.16947266,86.30963186)
\curveto(712.90948171,86.42963075)(712.66448195,86.58463059)(712.43447266,86.77463186)
\curveto(712.20448241,86.9746302)(712.0194826,87.18962999)(711.87947266,87.41963186)
\curveto(711.79948282,87.52962965)(711.73448288,87.64462953)(711.68447266,87.76463186)
\lineto(711.53447266,88.15463186)
\curveto(711.48448313,88.26462891)(711.45448316,88.3796288)(711.44447266,88.49963186)
\curveto(711.42448319,88.61962856)(711.39948322,88.74462843)(711.36947266,88.87463186)
\curveto(711.36948325,88.94462823)(711.36948325,89.00962817)(711.36947266,89.06963186)
\curveto(711.35948326,89.12962805)(711.34948327,89.19462798)(711.33947266,89.26463186)
}
}
{
\newrgbcolor{curcolor}{0 0 0}
\pscustom[linestyle=none,fillstyle=solid,fillcolor=curcolor]
{
\newpath
\moveto(716.85947266,101.36424123)
\lineto(717.11447266,101.36424123)
\curveto(717.19447742,101.37423353)(717.26947735,101.36923353)(717.33947266,101.34924123)
\lineto(717.57947266,101.34924123)
\lineto(717.74447266,101.34924123)
\curveto(717.84447677,101.32923357)(717.94947667,101.31923358)(718.05947266,101.31924123)
\curveto(718.15947646,101.31923358)(718.25947636,101.30923359)(718.35947266,101.28924123)
\lineto(718.50947266,101.28924123)
\curveto(718.64947597,101.25923364)(718.78947583,101.23923366)(718.92947266,101.22924123)
\curveto(719.05947556,101.21923368)(719.18947543,101.19423371)(719.31947266,101.15424123)
\curveto(719.39947522,101.13423377)(719.48447513,101.11423379)(719.57447266,101.09424123)
\lineto(719.81447266,101.03424123)
\lineto(720.11447266,100.91424123)
\curveto(720.20447441,100.88423402)(720.29447432,100.84923405)(720.38447266,100.80924123)
\curveto(720.60447401,100.70923419)(720.8194738,100.57423433)(721.02947266,100.40424123)
\curveto(721.23947338,100.24423466)(721.40947321,100.06923483)(721.53947266,99.87924123)
\curveto(721.57947304,99.82923507)(721.619473,99.76923513)(721.65947266,99.69924123)
\curveto(721.68947293,99.63923526)(721.72447289,99.57923532)(721.76447266,99.51924123)
\curveto(721.8144728,99.43923546)(721.85447276,99.34423556)(721.88447266,99.23424123)
\curveto(721.9144727,99.12423578)(721.94447267,99.01923588)(721.97447266,98.91924123)
\curveto(722.0144726,98.80923609)(722.03947258,98.6992362)(722.04947266,98.58924123)
\curveto(722.05947256,98.47923642)(722.07447254,98.36423654)(722.09447266,98.24424123)
\curveto(722.10447251,98.2042367)(722.10447251,98.15923674)(722.09447266,98.10924123)
\curveto(722.09447252,98.06923683)(722.09947252,98.02923687)(722.10947266,97.98924123)
\curveto(722.1194725,97.94923695)(722.12447249,97.89423701)(722.12447266,97.82424123)
\curveto(722.12447249,97.75423715)(722.1194725,97.7042372)(722.10947266,97.67424123)
\curveto(722.08947253,97.62423728)(722.08447253,97.57923732)(722.09447266,97.53924123)
\curveto(722.10447251,97.4992374)(722.10447251,97.46423744)(722.09447266,97.43424123)
\lineto(722.09447266,97.34424123)
\curveto(722.07447254,97.28423762)(722.05947256,97.21923768)(722.04947266,97.14924123)
\curveto(722.04947257,97.08923781)(722.04447257,97.02423788)(722.03447266,96.95424123)
\curveto(721.98447263,96.78423812)(721.93447268,96.62423828)(721.88447266,96.47424123)
\curveto(721.83447278,96.32423858)(721.76947285,96.17923872)(721.68947266,96.03924123)
\curveto(721.64947297,95.98923891)(721.619473,95.93423897)(721.59947266,95.87424123)
\curveto(721.56947305,95.82423908)(721.53447308,95.77423913)(721.49447266,95.72424123)
\curveto(721.3144733,95.48423942)(721.09447352,95.28423962)(720.83447266,95.12424123)
\curveto(720.57447404,94.96423994)(720.28947433,94.82424008)(719.97947266,94.70424123)
\curveto(719.83947478,94.64424026)(719.69947492,94.5992403)(719.55947266,94.56924123)
\curveto(719.40947521,94.53924036)(719.25447536,94.5042404)(719.09447266,94.46424123)
\curveto(718.98447563,94.44424046)(718.87447574,94.42924047)(718.76447266,94.41924123)
\curveto(718.65447596,94.40924049)(718.54447607,94.39424051)(718.43447266,94.37424123)
\curveto(718.39447622,94.36424054)(718.35447626,94.35924054)(718.31447266,94.35924123)
\curveto(718.27447634,94.36924053)(718.23447638,94.36924053)(718.19447266,94.35924123)
\curveto(718.14447647,94.34924055)(718.09447652,94.34424056)(718.04447266,94.34424123)
\lineto(717.87947266,94.34424123)
\curveto(717.82947679,94.32424058)(717.77947684,94.31924058)(717.72947266,94.32924123)
\curveto(717.66947695,94.33924056)(717.614477,94.33924056)(717.56447266,94.32924123)
\curveto(717.52447709,94.31924058)(717.47947714,94.31924058)(717.42947266,94.32924123)
\curveto(717.37947724,94.33924056)(717.32947729,94.33424057)(717.27947266,94.31424123)
\curveto(717.20947741,94.29424061)(717.13447748,94.28924061)(717.05447266,94.29924123)
\curveto(716.96447765,94.30924059)(716.87947774,94.31424059)(716.79947266,94.31424123)
\curveto(716.70947791,94.31424059)(716.60947801,94.30924059)(716.49947266,94.29924123)
\curveto(716.37947824,94.28924061)(716.27947834,94.29424061)(716.19947266,94.31424123)
\lineto(715.91447266,94.31424123)
\lineto(715.28447266,94.35924123)
\curveto(715.18447943,94.36924053)(715.08947953,94.37924052)(714.99947266,94.38924123)
\lineto(714.69947266,94.41924123)
\curveto(714.64947997,94.43924046)(714.59948002,94.44424046)(714.54947266,94.43424123)
\curveto(714.48948013,94.43424047)(714.43448018,94.44424046)(714.38447266,94.46424123)
\curveto(714.2144804,94.51424039)(714.04948057,94.55424035)(713.88947266,94.58424123)
\curveto(713.7194809,94.61424029)(713.55948106,94.66424024)(713.40947266,94.73424123)
\curveto(712.94948167,94.92423998)(712.57448204,95.14423976)(712.28447266,95.39424123)
\curveto(711.99448262,95.65423925)(711.74948287,96.01423889)(711.54947266,96.47424123)
\curveto(711.49948312,96.6042383)(711.46448315,96.73423817)(711.44447266,96.86424123)
\curveto(711.42448319,97.0042379)(711.39948322,97.14423776)(711.36947266,97.28424123)
\curveto(711.35948326,97.35423755)(711.35448326,97.41923748)(711.35447266,97.47924123)
\curveto(711.35448326,97.53923736)(711.34948327,97.6042373)(711.33947266,97.67424123)
\curveto(711.3194833,98.5042364)(711.46948315,99.17423573)(711.78947266,99.68424123)
\curveto(712.09948252,100.19423471)(712.53948208,100.57423433)(713.10947266,100.82424123)
\curveto(713.22948139,100.87423403)(713.35448126,100.91923398)(713.48447266,100.95924123)
\curveto(713.614481,100.9992339)(713.74948087,101.04423386)(713.88947266,101.09424123)
\curveto(713.96948065,101.11423379)(714.05448056,101.12923377)(714.14447266,101.13924123)
\lineto(714.38447266,101.19924123)
\curveto(714.49448012,101.22923367)(714.60448001,101.24423366)(714.71447266,101.24424123)
\curveto(714.82447979,101.25423365)(714.93447968,101.26923363)(715.04447266,101.28924123)
\curveto(715.09447952,101.30923359)(715.13947948,101.31423359)(715.17947266,101.30424123)
\curveto(715.2194794,101.3042336)(715.25947936,101.30923359)(715.29947266,101.31924123)
\curveto(715.34947927,101.32923357)(715.40447921,101.32923357)(715.46447266,101.31924123)
\curveto(715.5144791,101.31923358)(715.56447905,101.32423358)(715.61447266,101.33424123)
\lineto(715.74947266,101.33424123)
\curveto(715.80947881,101.35423355)(715.87947874,101.35423355)(715.95947266,101.33424123)
\curveto(716.02947859,101.32423358)(716.09447852,101.32923357)(716.15447266,101.34924123)
\curveto(716.18447843,101.35923354)(716.22447839,101.36423354)(716.27447266,101.36424123)
\lineto(716.39447266,101.36424123)
\lineto(716.85947266,101.36424123)
\moveto(719.18447266,99.81924123)
\curveto(718.86447575,99.91923498)(718.49947612,99.97923492)(718.08947266,99.99924123)
\curveto(717.67947694,100.01923488)(717.26947735,100.02923487)(716.85947266,100.02924123)
\curveto(716.42947819,100.02923487)(716.00947861,100.01923488)(715.59947266,99.99924123)
\curveto(715.18947943,99.97923492)(714.80447981,99.93423497)(714.44447266,99.86424123)
\curveto(714.08448053,99.79423511)(713.76448085,99.68423522)(713.48447266,99.53424123)
\curveto(713.19448142,99.39423551)(712.95948166,99.1992357)(712.77947266,98.94924123)
\curveto(712.66948195,98.78923611)(712.58948203,98.60923629)(712.53947266,98.40924123)
\curveto(712.47948214,98.20923669)(712.44948217,97.96423694)(712.44947266,97.67424123)
\curveto(712.46948215,97.65423725)(712.47948214,97.61923728)(712.47947266,97.56924123)
\curveto(712.46948215,97.51923738)(712.46948215,97.47923742)(712.47947266,97.44924123)
\curveto(712.49948212,97.36923753)(712.5194821,97.29423761)(712.53947266,97.22424123)
\curveto(712.54948207,97.16423774)(712.56948205,97.0992378)(712.59947266,97.02924123)
\curveto(712.7194819,96.75923814)(712.88948173,96.53923836)(713.10947266,96.36924123)
\curveto(713.3194813,96.20923869)(713.56448105,96.07423883)(713.84447266,95.96424123)
\curveto(713.95448066,95.91423899)(714.07448054,95.87423903)(714.20447266,95.84424123)
\curveto(714.32448029,95.82423908)(714.44948017,95.7992391)(714.57947266,95.76924123)
\curveto(714.62947999,95.74923915)(714.68447993,95.73923916)(714.74447266,95.73924123)
\curveto(714.79447982,95.73923916)(714.84447977,95.73423917)(714.89447266,95.72424123)
\curveto(714.98447963,95.71423919)(715.07947954,95.7042392)(715.17947266,95.69424123)
\curveto(715.26947935,95.68423922)(715.36447925,95.67423923)(715.46447266,95.66424123)
\curveto(715.54447907,95.66423924)(715.62947899,95.65923924)(715.71947266,95.64924123)
\lineto(715.95947266,95.64924123)
\lineto(716.13947266,95.64924123)
\curveto(716.16947845,95.63923926)(716.20447841,95.63423927)(716.24447266,95.63424123)
\lineto(716.37947266,95.63424123)
\lineto(716.82947266,95.63424123)
\curveto(716.90947771,95.63423927)(716.99447762,95.62923927)(717.08447266,95.61924123)
\curveto(717.16447745,95.61923928)(717.23947738,95.62923927)(717.30947266,95.64924123)
\lineto(717.57947266,95.64924123)
\curveto(717.59947702,95.64923925)(717.62947699,95.64423926)(717.66947266,95.63424123)
\curveto(717.69947692,95.63423927)(717.72447689,95.63923926)(717.74447266,95.64924123)
\curveto(717.84447677,95.65923924)(717.94447667,95.66423924)(718.04447266,95.66424123)
\curveto(718.13447648,95.67423923)(718.23447638,95.68423922)(718.34447266,95.69424123)
\curveto(718.46447615,95.72423918)(718.58947603,95.73923916)(718.71947266,95.73924123)
\curveto(718.83947578,95.74923915)(718.95447566,95.77423913)(719.06447266,95.81424123)
\curveto(719.36447525,95.89423901)(719.62947499,95.97923892)(719.85947266,96.06924123)
\curveto(720.08947453,96.16923873)(720.30447431,96.31423859)(720.50447266,96.50424123)
\curveto(720.70447391,96.71423819)(720.85447376,96.97923792)(720.95447266,97.29924123)
\curveto(720.97447364,97.33923756)(720.98447363,97.37423753)(720.98447266,97.40424123)
\curveto(720.97447364,97.44423746)(720.97947364,97.48923741)(720.99947266,97.53924123)
\curveto(721.00947361,97.57923732)(721.0194736,97.64923725)(721.02947266,97.74924123)
\curveto(721.03947358,97.85923704)(721.03447358,97.94423696)(721.01447266,98.00424123)
\curveto(720.99447362,98.07423683)(720.98447363,98.14423676)(720.98447266,98.21424123)
\curveto(720.97447364,98.28423662)(720.95947366,98.34923655)(720.93947266,98.40924123)
\curveto(720.87947374,98.60923629)(720.79447382,98.78923611)(720.68447266,98.94924123)
\curveto(720.66447395,98.97923592)(720.64447397,99.0042359)(720.62447266,99.02424123)
\lineto(720.56447266,99.08424123)
\curveto(720.54447407,99.12423578)(720.50447411,99.17423573)(720.44447266,99.23424123)
\curveto(720.30447431,99.33423557)(720.17447444,99.41923548)(720.05447266,99.48924123)
\curveto(719.93447468,99.55923534)(719.78947483,99.62923527)(719.61947266,99.69924123)
\curveto(719.54947507,99.72923517)(719.47947514,99.74923515)(719.40947266,99.75924123)
\curveto(719.33947528,99.77923512)(719.26447535,99.7992351)(719.18447266,99.81924123)
}
}
{
\newrgbcolor{curcolor}{0 0 0}
\pscustom[linestyle=none,fillstyle=solid,fillcolor=curcolor]
{
\newpath
\moveto(711.33947266,106.77385061)
\curveto(711.33948328,106.87384575)(711.34948327,106.96884566)(711.36947266,107.05885061)
\curveto(711.37948324,107.14884548)(711.40948321,107.21384541)(711.45947266,107.25385061)
\curveto(711.53948308,107.31384531)(711.64448297,107.34384528)(711.77447266,107.34385061)
\lineto(712.16447266,107.34385061)
\lineto(713.66447266,107.34385061)
\lineto(720.05447266,107.34385061)
\lineto(721.22447266,107.34385061)
\lineto(721.53947266,107.34385061)
\curveto(721.63947298,107.35384527)(721.7194729,107.33884529)(721.77947266,107.29885061)
\curveto(721.85947276,107.24884538)(721.90947271,107.17384545)(721.92947266,107.07385061)
\curveto(721.93947268,106.98384564)(721.94447267,106.87384575)(721.94447266,106.74385061)
\lineto(721.94447266,106.51885061)
\curveto(721.92447269,106.43884619)(721.90947271,106.36884626)(721.89947266,106.30885061)
\curveto(721.87947274,106.24884638)(721.83947278,106.19884643)(721.77947266,106.15885061)
\curveto(721.7194729,106.11884651)(721.64447297,106.09884653)(721.55447266,106.09885061)
\lineto(721.25447266,106.09885061)
\lineto(720.15947266,106.09885061)
\lineto(714.81947266,106.09885061)
\curveto(714.72947989,106.07884655)(714.65447996,106.06384656)(714.59447266,106.05385061)
\curveto(714.52448009,106.05384657)(714.46448015,106.0238466)(714.41447266,105.96385061)
\curveto(714.36448025,105.89384673)(714.33948028,105.80384682)(714.33947266,105.69385061)
\curveto(714.32948029,105.59384703)(714.32448029,105.48384714)(714.32447266,105.36385061)
\lineto(714.32447266,104.22385061)
\lineto(714.32447266,103.72885061)
\curveto(714.3144803,103.56884906)(714.25448036,103.45884917)(714.14447266,103.39885061)
\curveto(714.1144805,103.37884925)(714.08448053,103.36884926)(714.05447266,103.36885061)
\curveto(714.0144806,103.36884926)(713.96948065,103.36384926)(713.91947266,103.35385061)
\curveto(713.79948082,103.33384929)(713.68948093,103.33884929)(713.58947266,103.36885061)
\curveto(713.48948113,103.40884922)(713.4194812,103.46384916)(713.37947266,103.53385061)
\curveto(713.32948129,103.61384901)(713.30448131,103.73384889)(713.30447266,103.89385061)
\curveto(713.30448131,104.05384857)(713.28948133,104.18884844)(713.25947266,104.29885061)
\curveto(713.24948137,104.34884828)(713.24448137,104.40384822)(713.24447266,104.46385061)
\curveto(713.23448138,104.5238481)(713.2194814,104.58384804)(713.19947266,104.64385061)
\curveto(713.14948147,104.79384783)(713.09948152,104.93884769)(713.04947266,105.07885061)
\curveto(712.98948163,105.21884741)(712.9194817,105.35384727)(712.83947266,105.48385061)
\curveto(712.74948187,105.623847)(712.64448197,105.74384688)(712.52447266,105.84385061)
\curveto(712.40448221,105.94384668)(712.27448234,106.03884659)(712.13447266,106.12885061)
\curveto(712.03448258,106.18884644)(711.92448269,106.23384639)(711.80447266,106.26385061)
\curveto(711.68448293,106.30384632)(711.57948304,106.35384627)(711.48947266,106.41385061)
\curveto(711.42948319,106.46384616)(711.38948323,106.53384609)(711.36947266,106.62385061)
\curveto(711.35948326,106.64384598)(711.35448326,106.66884596)(711.35447266,106.69885061)
\curveto(711.35448326,106.7288459)(711.34948327,106.75384587)(711.33947266,106.77385061)
}
}
{
\newrgbcolor{curcolor}{0 0 0}
\pscustom[linestyle=none,fillstyle=solid,fillcolor=curcolor]
{
\newpath
\moveto(711.33947266,115.12345998)
\curveto(711.33948328,115.22345513)(711.34948327,115.31845503)(711.36947266,115.40845998)
\curveto(711.37948324,115.49845485)(711.40948321,115.56345479)(711.45947266,115.60345998)
\curveto(711.53948308,115.66345469)(711.64448297,115.69345466)(711.77447266,115.69345998)
\lineto(712.16447266,115.69345998)
\lineto(713.66447266,115.69345998)
\lineto(720.05447266,115.69345998)
\lineto(721.22447266,115.69345998)
\lineto(721.53947266,115.69345998)
\curveto(721.63947298,115.70345465)(721.7194729,115.68845466)(721.77947266,115.64845998)
\curveto(721.85947276,115.59845475)(721.90947271,115.52345483)(721.92947266,115.42345998)
\curveto(721.93947268,115.33345502)(721.94447267,115.22345513)(721.94447266,115.09345998)
\lineto(721.94447266,114.86845998)
\curveto(721.92447269,114.78845556)(721.90947271,114.71845563)(721.89947266,114.65845998)
\curveto(721.87947274,114.59845575)(721.83947278,114.5484558)(721.77947266,114.50845998)
\curveto(721.7194729,114.46845588)(721.64447297,114.4484559)(721.55447266,114.44845998)
\lineto(721.25447266,114.44845998)
\lineto(720.15947266,114.44845998)
\lineto(714.81947266,114.44845998)
\curveto(714.72947989,114.42845592)(714.65447996,114.41345594)(714.59447266,114.40345998)
\curveto(714.52448009,114.40345595)(714.46448015,114.37345598)(714.41447266,114.31345998)
\curveto(714.36448025,114.24345611)(714.33948028,114.1534562)(714.33947266,114.04345998)
\curveto(714.32948029,113.94345641)(714.32448029,113.83345652)(714.32447266,113.71345998)
\lineto(714.32447266,112.57345998)
\lineto(714.32447266,112.07845998)
\curveto(714.3144803,111.91845843)(714.25448036,111.80845854)(714.14447266,111.74845998)
\curveto(714.1144805,111.72845862)(714.08448053,111.71845863)(714.05447266,111.71845998)
\curveto(714.0144806,111.71845863)(713.96948065,111.71345864)(713.91947266,111.70345998)
\curveto(713.79948082,111.68345867)(713.68948093,111.68845866)(713.58947266,111.71845998)
\curveto(713.48948113,111.75845859)(713.4194812,111.81345854)(713.37947266,111.88345998)
\curveto(713.32948129,111.96345839)(713.30448131,112.08345827)(713.30447266,112.24345998)
\curveto(713.30448131,112.40345795)(713.28948133,112.53845781)(713.25947266,112.64845998)
\curveto(713.24948137,112.69845765)(713.24448137,112.7534576)(713.24447266,112.81345998)
\curveto(713.23448138,112.87345748)(713.2194814,112.93345742)(713.19947266,112.99345998)
\curveto(713.14948147,113.14345721)(713.09948152,113.28845706)(713.04947266,113.42845998)
\curveto(712.98948163,113.56845678)(712.9194817,113.70345665)(712.83947266,113.83345998)
\curveto(712.74948187,113.97345638)(712.64448197,114.09345626)(712.52447266,114.19345998)
\curveto(712.40448221,114.29345606)(712.27448234,114.38845596)(712.13447266,114.47845998)
\curveto(712.03448258,114.53845581)(711.92448269,114.58345577)(711.80447266,114.61345998)
\curveto(711.68448293,114.6534557)(711.57948304,114.70345565)(711.48947266,114.76345998)
\curveto(711.42948319,114.81345554)(711.38948323,114.88345547)(711.36947266,114.97345998)
\curveto(711.35948326,114.99345536)(711.35448326,115.01845533)(711.35447266,115.04845998)
\curveto(711.35448326,115.07845527)(711.34948327,115.10345525)(711.33947266,115.12345998)
}
}
{
\newrgbcolor{curcolor}{0 0 0}
\pscustom[linestyle=none,fillstyle=solid,fillcolor=curcolor]
{
\newpath
\moveto(732.17578857,37.28705373)
\curveto(732.17579927,37.35704805)(732.17579927,37.43704797)(732.17578857,37.52705373)
\curveto(732.16579928,37.61704779)(732.16579928,37.70204771)(732.17578857,37.78205373)
\curveto(732.17579927,37.87204754)(732.18579926,37.95204746)(732.20578857,38.02205373)
\curveto(732.22579922,38.10204731)(732.25579919,38.15704725)(732.29578857,38.18705373)
\curveto(732.3457991,38.21704719)(732.42079902,38.23704717)(732.52078857,38.24705373)
\curveto(732.61079883,38.26704714)(732.71579873,38.27704713)(732.83578857,38.27705373)
\curveto(732.9457985,38.28704712)(733.06079838,38.28704712)(733.18078857,38.27705373)
\lineto(733.48078857,38.27705373)
\lineto(736.49578857,38.27705373)
\lineto(739.39078857,38.27705373)
\curveto(739.72079172,38.27704713)(740.0457914,38.27204714)(740.36578857,38.26205373)
\curveto(740.67579077,38.26204715)(740.95579049,38.22204719)(741.20578857,38.14205373)
\curveto(741.55578989,38.02204739)(741.85078959,37.86704754)(742.09078857,37.67705373)
\curveto(742.32078912,37.48704792)(742.52078892,37.24704816)(742.69078857,36.95705373)
\curveto(742.7407887,36.89704851)(742.77578867,36.83204858)(742.79578857,36.76205373)
\curveto(742.81578863,36.70204871)(742.8407886,36.63204878)(742.87078857,36.55205373)
\curveto(742.92078852,36.43204898)(742.95578849,36.30204911)(742.97578857,36.16205373)
\curveto(743.00578844,36.03204938)(743.03578841,35.89704951)(743.06578857,35.75705373)
\curveto(743.08578836,35.7070497)(743.09078835,35.65704975)(743.08078857,35.60705373)
\curveto(743.07078837,35.55704985)(743.07078837,35.50204991)(743.08078857,35.44205373)
\curveto(743.09078835,35.42204999)(743.09078835,35.39705001)(743.08078857,35.36705373)
\curveto(743.08078836,35.33705007)(743.08578836,35.3120501)(743.09578857,35.29205373)
\curveto(743.10578834,35.25205016)(743.11078833,35.19705021)(743.11078857,35.12705373)
\curveto(743.11078833,35.05705035)(743.10578834,35.00205041)(743.09578857,34.96205373)
\curveto(743.08578836,34.9120505)(743.08578836,34.85705055)(743.09578857,34.79705373)
\curveto(743.10578834,34.73705067)(743.10078834,34.68205073)(743.08078857,34.63205373)
\curveto(743.05078839,34.50205091)(743.03078841,34.37705103)(743.02078857,34.25705373)
\curveto(743.01078843,34.13705127)(742.98578846,34.02205139)(742.94578857,33.91205373)
\curveto(742.82578862,33.54205187)(742.65578879,33.22205219)(742.43578857,32.95205373)
\curveto(742.21578923,32.68205273)(741.93578951,32.47205294)(741.59578857,32.32205373)
\curveto(741.47578997,32.27205314)(741.35079009,32.22705318)(741.22078857,32.18705373)
\curveto(741.09079035,32.15705325)(740.95579049,32.12205329)(740.81578857,32.08205373)
\curveto(740.76579068,32.07205334)(740.72579072,32.06705334)(740.69578857,32.06705373)
\curveto(740.65579079,32.06705334)(740.61079083,32.06205335)(740.56078857,32.05205373)
\curveto(740.53079091,32.04205337)(740.49579095,32.03705337)(740.45578857,32.03705373)
\curveto(740.40579104,32.03705337)(740.36579108,32.03205338)(740.33578857,32.02205373)
\lineto(740.17078857,32.02205373)
\curveto(740.09079135,32.00205341)(739.99079145,31.99705341)(739.87078857,32.00705373)
\curveto(739.7407917,32.01705339)(739.65079179,32.03205338)(739.60078857,32.05205373)
\curveto(739.51079193,32.07205334)(739.445792,32.12705328)(739.40578857,32.21705373)
\curveto(739.38579206,32.24705316)(739.38079206,32.27705313)(739.39078857,32.30705373)
\curveto(739.39079205,32.33705307)(739.38579206,32.37705303)(739.37578857,32.42705373)
\curveto(739.36579208,32.46705294)(739.36079208,32.5070529)(739.36078857,32.54705373)
\lineto(739.36078857,32.69705373)
\curveto(739.36079208,32.81705259)(739.36579208,32.93705247)(739.37578857,33.05705373)
\curveto(739.37579207,33.18705222)(739.41079203,33.27705213)(739.48078857,33.32705373)
\curveto(739.5407919,33.36705204)(739.60079184,33.38705202)(739.66078857,33.38705373)
\curveto(739.72079172,33.38705202)(739.79079165,33.39705201)(739.87078857,33.41705373)
\curveto(739.90079154,33.42705198)(739.93579151,33.42705198)(739.97578857,33.41705373)
\curveto(740.00579144,33.41705199)(740.03079141,33.42205199)(740.05078857,33.43205373)
\lineto(740.26078857,33.43205373)
\curveto(740.31079113,33.45205196)(740.36079108,33.45705195)(740.41078857,33.44705373)
\curveto(740.45079099,33.44705196)(740.49579095,33.45705195)(740.54578857,33.47705373)
\curveto(740.67579077,33.5070519)(740.80079064,33.53705187)(740.92078857,33.56705373)
\curveto(741.03079041,33.59705181)(741.13579031,33.64205177)(741.23578857,33.70205373)
\curveto(741.52578992,33.87205154)(741.73078971,34.14205127)(741.85078857,34.51205373)
\curveto(741.87078957,34.56205085)(741.88578956,34.6120508)(741.89578857,34.66205373)
\curveto(741.89578955,34.72205069)(741.90578954,34.77705063)(741.92578857,34.82705373)
\lineto(741.92578857,34.90205373)
\curveto(741.93578951,34.97205044)(741.9457895,35.06705034)(741.95578857,35.18705373)
\curveto(741.95578949,35.31705009)(741.9457895,35.41704999)(741.92578857,35.48705373)
\curveto(741.90578954,35.55704985)(741.89078955,35.62704978)(741.88078857,35.69705373)
\curveto(741.86078958,35.77704963)(741.8407896,35.84704956)(741.82078857,35.90705373)
\curveto(741.66078978,36.28704912)(741.38579006,36.56204885)(740.99578857,36.73205373)
\curveto(740.86579058,36.78204863)(740.71079073,36.81704859)(740.53078857,36.83705373)
\curveto(740.35079109,36.86704854)(740.16579128,36.88204853)(739.97578857,36.88205373)
\curveto(739.77579167,36.89204852)(739.57579187,36.89204852)(739.37578857,36.88205373)
\lineto(738.80578857,36.88205373)
\lineto(734.56078857,36.88205373)
\lineto(733.01578857,36.88205373)
\curveto(732.90579854,36.88204853)(732.78579866,36.87704853)(732.65578857,36.86705373)
\curveto(732.52579892,36.85704855)(732.42079902,36.87704853)(732.34078857,36.92705373)
\curveto(732.27079917,36.98704842)(732.22079922,37.06704834)(732.19078857,37.16705373)
\curveto(732.19079925,37.18704822)(732.19079925,37.2070482)(732.19078857,37.22705373)
\curveto(732.19079925,37.24704816)(732.18579926,37.26704814)(732.17578857,37.28705373)
}
}
{
\newrgbcolor{curcolor}{0 0 0}
\pscustom[linestyle=none,fillstyle=solid,fillcolor=curcolor]
{
\newpath
\moveto(735.13078857,40.82072561)
\lineto(735.13078857,41.25572561)
\curveto(735.13079631,41.40572364)(735.17079627,41.51072354)(735.25078857,41.57072561)
\curveto(735.33079611,41.62072343)(735.43079601,41.6457234)(735.55078857,41.64572561)
\curveto(735.67079577,41.65572339)(735.79079565,41.66072339)(735.91078857,41.66072561)
\lineto(737.33578857,41.66072561)
\lineto(739.60078857,41.66072561)
\lineto(740.29078857,41.66072561)
\curveto(740.52079092,41.66072339)(740.72079072,41.68572336)(740.89078857,41.73572561)
\curveto(741.3407901,41.89572315)(741.65578979,42.19572285)(741.83578857,42.63572561)
\curveto(741.92578952,42.85572219)(741.96078948,43.12072193)(741.94078857,43.43072561)
\curveto(741.91078953,43.74072131)(741.85578959,43.99072106)(741.77578857,44.18072561)
\curveto(741.63578981,44.51072054)(741.46078998,44.77072028)(741.25078857,44.96072561)
\curveto(741.03079041,45.16071989)(740.7457907,45.31571973)(740.39578857,45.42572561)
\curveto(740.31579113,45.45571959)(740.23579121,45.47571957)(740.15578857,45.48572561)
\curveto(740.07579137,45.49571955)(739.99079145,45.51071954)(739.90078857,45.53072561)
\curveto(739.85079159,45.54071951)(739.80579164,45.54071951)(739.76578857,45.53072561)
\curveto(739.72579172,45.53071952)(739.68079176,45.54071951)(739.63078857,45.56072561)
\lineto(739.31578857,45.56072561)
\curveto(739.23579221,45.58071947)(739.1457923,45.58571946)(739.04578857,45.57572561)
\curveto(738.93579251,45.56571948)(738.83579261,45.56071949)(738.74578857,45.56072561)
\lineto(737.57578857,45.56072561)
\lineto(735.98578857,45.56072561)
\curveto(735.86579558,45.56071949)(735.7407957,45.55571949)(735.61078857,45.54572561)
\curveto(735.47079597,45.5457195)(735.36079608,45.57071948)(735.28078857,45.62072561)
\curveto(735.23079621,45.66071939)(735.20079624,45.70571934)(735.19078857,45.75572561)
\curveto(735.17079627,45.81571923)(735.15079629,45.88571916)(735.13078857,45.96572561)
\lineto(735.13078857,46.19072561)
\curveto(735.13079631,46.31071874)(735.13579631,46.41571863)(735.14578857,46.50572561)
\curveto(735.15579629,46.60571844)(735.20079624,46.68071837)(735.28078857,46.73072561)
\curveto(735.33079611,46.78071827)(735.40579604,46.80571824)(735.50578857,46.80572561)
\lineto(735.79078857,46.80572561)
\lineto(736.81078857,46.80572561)
\lineto(740.84578857,46.80572561)
\lineto(742.19578857,46.80572561)
\curveto(742.31578913,46.80571824)(742.43078901,46.80071825)(742.54078857,46.79072561)
\curveto(742.6407888,46.79071826)(742.71578873,46.75571829)(742.76578857,46.68572561)
\curveto(742.79578865,46.6457184)(742.82078862,46.58571846)(742.84078857,46.50572561)
\curveto(742.85078859,46.42571862)(742.86078858,46.33571871)(742.87078857,46.23572561)
\curveto(742.87078857,46.1457189)(742.86578858,46.05571899)(742.85578857,45.96572561)
\curveto(742.8457886,45.88571916)(742.82578862,45.82571922)(742.79578857,45.78572561)
\curveto(742.75578869,45.73571931)(742.69078875,45.69071936)(742.60078857,45.65072561)
\curveto(742.56078888,45.64071941)(742.50578894,45.63071942)(742.43578857,45.62072561)
\curveto(742.36578908,45.62071943)(742.30078914,45.61571943)(742.24078857,45.60572561)
\curveto(742.17078927,45.59571945)(742.11578933,45.57571947)(742.07578857,45.54572561)
\curveto(742.03578941,45.51571953)(742.02078942,45.47071958)(742.03078857,45.41072561)
\curveto(742.05078939,45.33071972)(742.11078933,45.2507198)(742.21078857,45.17072561)
\curveto(742.30078914,45.09071996)(742.37078907,45.01572003)(742.42078857,44.94572561)
\curveto(742.58078886,44.72572032)(742.72078872,44.47572057)(742.84078857,44.19572561)
\curveto(742.89078855,44.08572096)(742.92078852,43.97072108)(742.93078857,43.85072561)
\curveto(742.95078849,43.74072131)(742.97578847,43.62572142)(743.00578857,43.50572561)
\curveto(743.01578843,43.45572159)(743.01578843,43.40072165)(743.00578857,43.34072561)
\curveto(742.99578845,43.29072176)(743.00078844,43.24072181)(743.02078857,43.19072561)
\curveto(743.0407884,43.09072196)(743.0407884,43.00072205)(743.02078857,42.92072561)
\lineto(743.02078857,42.77072561)
\curveto(743.00078844,42.72072233)(742.99078845,42.66072239)(742.99078857,42.59072561)
\curveto(742.99078845,42.53072252)(742.98578846,42.47572257)(742.97578857,42.42572561)
\curveto(742.95578849,42.38572266)(742.9457885,42.3457227)(742.94578857,42.30572561)
\curveto(742.95578849,42.27572277)(742.95078849,42.23572281)(742.93078857,42.18572561)
\lineto(742.87078857,41.94572561)
\curveto(742.85078859,41.87572317)(742.82078862,41.80072325)(742.78078857,41.72072561)
\curveto(742.67078877,41.46072359)(742.52578892,41.24072381)(742.34578857,41.06072561)
\curveto(742.15578929,40.89072416)(741.93078951,40.7507243)(741.67078857,40.64072561)
\curveto(741.58078986,40.60072445)(741.49078995,40.57072448)(741.40078857,40.55072561)
\lineto(741.10078857,40.49072561)
\curveto(741.0407904,40.47072458)(740.98579046,40.46072459)(740.93578857,40.46072561)
\curveto(740.87579057,40.47072458)(740.81079063,40.46572458)(740.74078857,40.44572561)
\curveto(740.72079072,40.43572461)(740.69579075,40.43072462)(740.66578857,40.43072561)
\curveto(740.62579082,40.43072462)(740.59079085,40.42572462)(740.56078857,40.41572561)
\lineto(740.41078857,40.41572561)
\curveto(740.37079107,40.40572464)(740.32579112,40.40072465)(740.27578857,40.40072561)
\curveto(740.21579123,40.41072464)(740.16079128,40.41572463)(740.11078857,40.41572561)
\lineto(739.51078857,40.41572561)
\lineto(736.75078857,40.41572561)
\lineto(735.79078857,40.41572561)
\lineto(735.52078857,40.41572561)
\curveto(735.43079601,40.41572463)(735.35579609,40.43572461)(735.29578857,40.47572561)
\curveto(735.22579622,40.51572453)(735.17579627,40.59072446)(735.14578857,40.70072561)
\curveto(735.13579631,40.72072433)(735.13579631,40.74072431)(735.14578857,40.76072561)
\curveto(735.1457963,40.78072427)(735.1407963,40.80072425)(735.13078857,40.82072561)
}
}
{
\newrgbcolor{curcolor}{0 0 0}
\pscustom[linestyle=none,fillstyle=solid,fillcolor=curcolor]
{
\newpath
\moveto(734.98078857,52.39533498)
\curveto(734.96079648,53.02532975)(735.0457964,53.53032924)(735.23578857,53.91033498)
\curveto(735.42579602,54.29032848)(735.71079573,54.59532818)(736.09078857,54.82533498)
\curveto(736.19079525,54.88532789)(736.30079514,54.93032784)(736.42078857,54.96033498)
\curveto(736.53079491,55.00032777)(736.6457948,55.03532774)(736.76578857,55.06533498)
\curveto(736.95579449,55.11532766)(737.16079428,55.14532763)(737.38078857,55.15533498)
\curveto(737.60079384,55.16532761)(737.82579362,55.1703276)(738.05578857,55.17033498)
\lineto(739.66078857,55.17033498)
\lineto(742.00078857,55.17033498)
\curveto(742.17078927,55.1703276)(742.3407891,55.16532761)(742.51078857,55.15533498)
\curveto(742.68078876,55.15532762)(742.79078865,55.09032768)(742.84078857,54.96033498)
\curveto(742.86078858,54.91032786)(742.87078857,54.85532792)(742.87078857,54.79533498)
\curveto(742.88078856,54.74532803)(742.88578856,54.69032808)(742.88578857,54.63033498)
\curveto(742.88578856,54.50032827)(742.88078856,54.3753284)(742.87078857,54.25533498)
\curveto(742.87078857,54.13532864)(742.83078861,54.05032872)(742.75078857,54.00033498)
\curveto(742.68078876,53.95032882)(742.59078885,53.92532885)(742.48078857,53.92533498)
\lineto(742.15078857,53.92533498)
\lineto(740.86078857,53.92533498)
\lineto(738.41578857,53.92533498)
\curveto(738.1457933,53.92532885)(737.88079356,53.92032885)(737.62078857,53.91033498)
\curveto(737.35079409,53.90032887)(737.12079432,53.85532892)(736.93078857,53.77533498)
\curveto(736.73079471,53.69532908)(736.57079487,53.5753292)(736.45078857,53.41533498)
\curveto(736.32079512,53.25532952)(736.22079522,53.0703297)(736.15078857,52.86033498)
\curveto(736.13079531,52.80032997)(736.12079532,52.73533004)(736.12078857,52.66533498)
\curveto(736.11079533,52.60533017)(736.09579535,52.54533023)(736.07578857,52.48533498)
\curveto(736.06579538,52.43533034)(736.06579538,52.35533042)(736.07578857,52.24533498)
\curveto(736.07579537,52.14533063)(736.08079536,52.0753307)(736.09078857,52.03533498)
\curveto(736.11079533,51.99533078)(736.12079532,51.96033081)(736.12078857,51.93033498)
\curveto(736.11079533,51.90033087)(736.11079533,51.86533091)(736.12078857,51.82533498)
\curveto(736.15079529,51.69533108)(736.18579526,51.5703312)(736.22578857,51.45033498)
\curveto(736.25579519,51.34033143)(736.30079514,51.23533154)(736.36078857,51.13533498)
\curveto(736.38079506,51.09533168)(736.40079504,51.06033171)(736.42078857,51.03033498)
\curveto(736.440795,51.00033177)(736.46079498,50.96533181)(736.48078857,50.92533498)
\curveto(736.73079471,50.5753322)(737.10579434,50.32033245)(737.60578857,50.16033498)
\curveto(737.68579376,50.13033264)(737.77079367,50.11033266)(737.86078857,50.10033498)
\curveto(737.9407935,50.09033268)(738.02079342,50.0753327)(738.10078857,50.05533498)
\curveto(738.15079329,50.03533274)(738.20079324,50.03033274)(738.25078857,50.04033498)
\curveto(738.29079315,50.05033272)(738.33079311,50.04533273)(738.37078857,50.02533498)
\lineto(738.68578857,50.02533498)
\curveto(738.71579273,50.01533276)(738.75079269,50.01033276)(738.79078857,50.01033498)
\curveto(738.83079261,50.02033275)(738.87579257,50.02533275)(738.92578857,50.02533498)
\lineto(739.37578857,50.02533498)
\lineto(740.81578857,50.02533498)
\lineto(742.13578857,50.02533498)
\lineto(742.48078857,50.02533498)
\curveto(742.59078885,50.02533275)(742.68078876,50.00033277)(742.75078857,49.95033498)
\curveto(742.83078861,49.90033287)(742.87078857,49.81033296)(742.87078857,49.68033498)
\curveto(742.88078856,49.56033321)(742.88578856,49.43533334)(742.88578857,49.30533498)
\curveto(742.88578856,49.22533355)(742.88078856,49.15033362)(742.87078857,49.08033498)
\curveto(742.86078858,49.01033376)(742.83578861,48.95033382)(742.79578857,48.90033498)
\curveto(742.7457887,48.82033395)(742.65078879,48.78033399)(742.51078857,48.78033498)
\lineto(742.10578857,48.78033498)
\lineto(740.33578857,48.78033498)
\lineto(736.70578857,48.78033498)
\lineto(735.79078857,48.78033498)
\lineto(735.52078857,48.78033498)
\curveto(735.43079601,48.78033399)(735.36079608,48.80033397)(735.31078857,48.84033498)
\curveto(735.25079619,48.8703339)(735.21079623,48.92033385)(735.19078857,48.99033498)
\curveto(735.18079626,49.03033374)(735.17079627,49.08533369)(735.16078857,49.15533498)
\curveto(735.15079629,49.23533354)(735.1457963,49.31533346)(735.14578857,49.39533498)
\curveto(735.1457963,49.4753333)(735.15079629,49.55033322)(735.16078857,49.62033498)
\curveto(735.17079627,49.70033307)(735.18579626,49.75533302)(735.20578857,49.78533498)
\curveto(735.27579617,49.89533288)(735.36579608,49.94533283)(735.47578857,49.93533498)
\curveto(735.57579587,49.92533285)(735.69079575,49.94033283)(735.82078857,49.98033498)
\curveto(735.88079556,50.00033277)(735.93079551,50.04033273)(735.97078857,50.10033498)
\curveto(735.98079546,50.22033255)(735.93579551,50.31533246)(735.83578857,50.38533498)
\curveto(735.73579571,50.46533231)(735.65579579,50.54533223)(735.59578857,50.62533498)
\curveto(735.49579595,50.76533201)(735.40579604,50.90533187)(735.32578857,51.04533498)
\curveto(735.23579621,51.19533158)(735.16079628,51.36533141)(735.10078857,51.55533498)
\curveto(735.07079637,51.63533114)(735.05079639,51.72033105)(735.04078857,51.81033498)
\curveto(735.03079641,51.91033086)(735.01579643,52.00533077)(734.99578857,52.09533498)
\curveto(734.98579646,52.14533063)(734.98079646,52.19533058)(734.98078857,52.24533498)
\lineto(734.98078857,52.39533498)
}
}
{
\newrgbcolor{curcolor}{0 0 0}
\pscustom[linestyle=none,fillstyle=solid,fillcolor=curcolor]
{
}
}
{
\newrgbcolor{curcolor}{0 0 0}
\pscustom[linestyle=none,fillstyle=solid,fillcolor=curcolor]
{
\newpath
\moveto(732.25078857,65.05510061)
\curveto(732.25079919,65.15509575)(732.26079918,65.25009566)(732.28078857,65.34010061)
\curveto(732.29079915,65.43009548)(732.32079912,65.49509541)(732.37078857,65.53510061)
\curveto(732.45079899,65.59509531)(732.55579889,65.62509528)(732.68578857,65.62510061)
\lineto(733.07578857,65.62510061)
\lineto(734.57578857,65.62510061)
\lineto(740.96578857,65.62510061)
\lineto(742.13578857,65.62510061)
\lineto(742.45078857,65.62510061)
\curveto(742.55078889,65.63509527)(742.63078881,65.62009529)(742.69078857,65.58010061)
\curveto(742.77078867,65.53009538)(742.82078862,65.45509545)(742.84078857,65.35510061)
\curveto(742.85078859,65.26509564)(742.85578859,65.15509575)(742.85578857,65.02510061)
\lineto(742.85578857,64.80010061)
\curveto(742.83578861,64.72009619)(742.82078862,64.65009626)(742.81078857,64.59010061)
\curveto(742.79078865,64.53009638)(742.75078869,64.48009643)(742.69078857,64.44010061)
\curveto(742.63078881,64.40009651)(742.55578889,64.38009653)(742.46578857,64.38010061)
\lineto(742.16578857,64.38010061)
\lineto(741.07078857,64.38010061)
\lineto(735.73078857,64.38010061)
\curveto(735.6407958,64.36009655)(735.56579588,64.34509656)(735.50578857,64.33510061)
\curveto(735.43579601,64.33509657)(735.37579607,64.3050966)(735.32578857,64.24510061)
\curveto(735.27579617,64.17509673)(735.25079619,64.08509682)(735.25078857,63.97510061)
\curveto(735.2407962,63.87509703)(735.23579621,63.76509714)(735.23578857,63.64510061)
\lineto(735.23578857,62.50510061)
\lineto(735.23578857,62.01010061)
\curveto(735.22579622,61.85009906)(735.16579628,61.74009917)(735.05578857,61.68010061)
\curveto(735.02579642,61.66009925)(734.99579645,61.65009926)(734.96578857,61.65010061)
\curveto(734.92579652,61.65009926)(734.88079656,61.64509926)(734.83078857,61.63510061)
\curveto(734.71079673,61.61509929)(734.60079684,61.62009929)(734.50078857,61.65010061)
\curveto(734.40079704,61.69009922)(734.33079711,61.74509916)(734.29078857,61.81510061)
\curveto(734.2407972,61.89509901)(734.21579723,62.01509889)(734.21578857,62.17510061)
\curveto(734.21579723,62.33509857)(734.20079724,62.47009844)(734.17078857,62.58010061)
\curveto(734.16079728,62.63009828)(734.15579729,62.68509822)(734.15578857,62.74510061)
\curveto(734.1457973,62.8050981)(734.13079731,62.86509804)(734.11078857,62.92510061)
\curveto(734.06079738,63.07509783)(734.01079743,63.22009769)(733.96078857,63.36010061)
\curveto(733.90079754,63.50009741)(733.83079761,63.63509727)(733.75078857,63.76510061)
\curveto(733.66079778,63.905097)(733.55579789,64.02509688)(733.43578857,64.12510061)
\curveto(733.31579813,64.22509668)(733.18579826,64.32009659)(733.04578857,64.41010061)
\curveto(732.9457985,64.47009644)(732.83579861,64.51509639)(732.71578857,64.54510061)
\curveto(732.59579885,64.58509632)(732.49079895,64.63509627)(732.40078857,64.69510061)
\curveto(732.3407991,64.74509616)(732.30079914,64.81509609)(732.28078857,64.90510061)
\curveto(732.27079917,64.92509598)(732.26579918,64.95009596)(732.26578857,64.98010061)
\curveto(732.26579918,65.0100959)(732.26079918,65.03509587)(732.25078857,65.05510061)
}
}
{
\newrgbcolor{curcolor}{0 0 0}
\pscustom[linestyle=none,fillstyle=solid,fillcolor=curcolor]
{
\newpath
\moveto(739.78078857,76.35970998)
\curveto(739.82079162,76.36970226)(739.87079157,76.36970226)(739.93078857,76.35970998)
\curveto(739.99079145,76.35970227)(740.0407914,76.35470228)(740.08078857,76.34470998)
\curveto(740.12079132,76.34470229)(740.16079128,76.33970229)(740.20078857,76.32970998)
\lineto(740.30578857,76.32970998)
\curveto(740.38579106,76.30970232)(740.46579098,76.29470234)(740.54578857,76.28470998)
\curveto(740.62579082,76.27470236)(740.70079074,76.25470238)(740.77078857,76.22470998)
\curveto(740.85079059,76.20470243)(740.92579052,76.18470245)(740.99578857,76.16470998)
\curveto(741.06579038,76.14470249)(741.1407903,76.11470252)(741.22078857,76.07470998)
\curveto(741.6407898,75.89470274)(741.98078946,75.63970299)(742.24078857,75.30970998)
\curveto(742.50078894,74.97970365)(742.70578874,74.58970404)(742.85578857,74.13970998)
\curveto(742.89578855,74.01970461)(742.92078852,73.89470474)(742.93078857,73.76470998)
\curveto(742.95078849,73.64470499)(742.97578847,73.51970511)(743.00578857,73.38970998)
\curveto(743.01578843,73.3297053)(743.02078842,73.26470537)(743.02078857,73.19470998)
\curveto(743.02078842,73.1347055)(743.02578842,73.06970556)(743.03578857,72.99970998)
\lineto(743.03578857,72.87970998)
\lineto(743.03578857,72.68470998)
\curveto(743.0457884,72.62470601)(743.0407884,72.56970606)(743.02078857,72.51970998)
\curveto(743.00078844,72.44970618)(742.99578845,72.38470625)(743.00578857,72.32470998)
\curveto(743.01578843,72.26470637)(743.01078843,72.20470643)(742.99078857,72.14470998)
\curveto(742.98078846,72.09470654)(742.97578847,72.04970658)(742.97578857,72.00970998)
\curveto(742.97578847,71.96970666)(742.96578848,71.92470671)(742.94578857,71.87470998)
\curveto(742.92578852,71.79470684)(742.90578854,71.71970691)(742.88578857,71.64970998)
\curveto(742.87578857,71.57970705)(742.86078858,71.50970712)(742.84078857,71.43970998)
\curveto(742.67078877,70.95970767)(742.46078898,70.55970807)(742.21078857,70.23970998)
\curveto(741.95078949,69.9297087)(741.59578985,69.67970895)(741.14578857,69.48970998)
\curveto(741.08579036,69.45970917)(741.02579042,69.4347092)(740.96578857,69.41470998)
\curveto(740.89579055,69.40470923)(740.82079062,69.38970924)(740.74078857,69.36970998)
\curveto(740.68079076,69.34970928)(740.61579083,69.3347093)(740.54578857,69.32470998)
\curveto(740.47579097,69.31470932)(740.40579104,69.29970933)(740.33578857,69.27970998)
\curveto(740.28579116,69.26970936)(740.2457912,69.26470937)(740.21578857,69.26470998)
\lineto(740.09578857,69.26470998)
\curveto(740.05579139,69.25470938)(740.00579144,69.24470939)(739.94578857,69.23470998)
\curveto(739.88579156,69.2347094)(739.83579161,69.23970939)(739.79578857,69.24970998)
\lineto(739.66078857,69.24970998)
\curveto(739.61079183,69.25970937)(739.56079188,69.26470937)(739.51078857,69.26470998)
\curveto(739.41079203,69.28470935)(739.31579213,69.29970933)(739.22578857,69.30970998)
\curveto(739.12579232,69.31970931)(739.03079241,69.33970929)(738.94078857,69.36970998)
\curveto(738.79079265,69.41970921)(738.65079279,69.47470916)(738.52078857,69.53470998)
\curveto(738.39079305,69.59470904)(738.27079317,69.66470897)(738.16078857,69.74470998)
\curveto(738.11079333,69.77470886)(738.07079337,69.80470883)(738.04078857,69.83470998)
\curveto(738.01079343,69.87470876)(737.97579347,69.90970872)(737.93578857,69.93970998)
\curveto(737.85579359,69.99970863)(737.78579366,70.06970856)(737.72578857,70.14970998)
\curveto(737.67579377,70.20970842)(737.63079381,70.26970836)(737.59078857,70.32970998)
\lineto(737.44078857,70.53970998)
\curveto(737.40079404,70.58970804)(737.36579408,70.63970799)(737.33578857,70.68970998)
\curveto(737.29579415,70.73970789)(737.2407942,70.77470786)(737.17078857,70.79470998)
\curveto(737.1407943,70.79470784)(737.11579433,70.78470785)(737.09578857,70.76470998)
\curveto(737.06579438,70.75470788)(737.0407944,70.74470789)(737.02078857,70.73470998)
\curveto(736.97079447,70.69470794)(736.92579452,70.64470799)(736.88578857,70.58470998)
\curveto(736.83579461,70.5347081)(736.79079465,70.48470815)(736.75078857,70.43470998)
\curveto(736.72079472,70.39470824)(736.66579478,70.34470829)(736.58578857,70.28470998)
\curveto(736.55579489,70.26470837)(736.53079491,70.2347084)(736.51078857,70.19470998)
\curveto(736.48079496,70.16470847)(736.445795,70.13970849)(736.40578857,70.11970998)
\curveto(736.19579525,69.94970868)(735.95079549,69.81970881)(735.67078857,69.72970998)
\curveto(735.59079585,69.70970892)(735.51079593,69.69470894)(735.43078857,69.68470998)
\curveto(735.35079609,69.67470896)(735.27079617,69.65970897)(735.19078857,69.63970998)
\curveto(735.1407963,69.61970901)(735.07579637,69.60970902)(734.99578857,69.60970998)
\curveto(734.90579654,69.60970902)(734.83579661,69.61970901)(734.78578857,69.63970998)
\curveto(734.68579676,69.63970899)(734.61579683,69.64470899)(734.57578857,69.65470998)
\curveto(734.49579695,69.67470896)(734.42579702,69.68970894)(734.36578857,69.69970998)
\curveto(734.29579715,69.70970892)(734.22579722,69.72470891)(734.15578857,69.74470998)
\curveto(733.72579772,69.89470874)(733.38079806,70.10970852)(733.12078857,70.38970998)
\curveto(732.86079858,70.67970795)(732.6457988,71.0297076)(732.47578857,71.43970998)
\curveto(732.42579902,71.54970708)(732.39579905,71.66470697)(732.38578857,71.78470998)
\curveto(732.36579908,71.91470672)(732.33579911,72.04470659)(732.29578857,72.17470998)
\curveto(732.29579915,72.25470638)(732.29579915,72.32470631)(732.29578857,72.38470998)
\curveto(732.28579916,72.45470618)(732.27579917,72.5297061)(732.26578857,72.60970998)
\curveto(732.2457992,73.39970523)(732.37579907,74.05470458)(732.65578857,74.57470998)
\curveto(732.93579851,75.10470353)(733.3457981,75.48470315)(733.88578857,75.71470998)
\curveto(734.11579733,75.82470281)(734.40079704,75.89470274)(734.74078857,75.92470998)
\curveto(735.07079637,75.96470267)(735.37579607,75.9347027)(735.65578857,75.83470998)
\curveto(735.78579566,75.79470284)(735.90579554,75.74470289)(736.01578857,75.68470998)
\curveto(736.12579532,75.634703)(736.23079521,75.57470306)(736.33078857,75.50470998)
\curveto(736.37079507,75.48470315)(736.40579504,75.45470318)(736.43578857,75.41470998)
\lineto(736.52578857,75.32470998)
\curveto(736.61579483,75.27470336)(736.68079476,75.21470342)(736.72078857,75.14470998)
\curveto(736.77079467,75.09470354)(736.82079462,75.03970359)(736.87078857,74.97970998)
\curveto(736.91079453,74.9297037)(736.95579449,74.88470375)(737.00578857,74.84470998)
\curveto(737.02579442,74.82470381)(737.05079439,74.80470383)(737.08078857,74.78470998)
\curveto(737.10079434,74.77470386)(737.12579432,74.77470386)(737.15578857,74.78470998)
\curveto(737.20579424,74.79470384)(737.25579419,74.82470381)(737.30578857,74.87470998)
\curveto(737.3457941,74.92470371)(737.38579406,74.97970365)(737.42578857,75.03970998)
\lineto(737.54578857,75.21970998)
\curveto(737.57579387,75.27970335)(737.60579384,75.3297033)(737.63578857,75.36970998)
\curveto(737.87579357,75.69970293)(738.18579326,75.94970268)(738.56578857,76.11970998)
\curveto(738.6457928,76.15970247)(738.73079271,76.18970244)(738.82078857,76.20970998)
\curveto(738.91079253,76.23970239)(739.00079244,76.26470237)(739.09078857,76.28470998)
\curveto(739.1407923,76.29470234)(739.19579225,76.30470233)(739.25578857,76.31470998)
\lineto(739.40578857,76.34470998)
\curveto(739.46579198,76.35470228)(739.53079191,76.35470228)(739.60078857,76.34470998)
\curveto(739.66079178,76.3347023)(739.72079172,76.33970229)(739.78078857,76.35970998)
\moveto(734.74078857,70.97470998)
\curveto(734.85079659,70.94470769)(734.99079645,70.93970769)(735.16078857,70.95970998)
\curveto(735.32079612,70.97970765)(735.445796,71.00470763)(735.53578857,71.03470998)
\curveto(735.85579559,71.14470749)(736.10079534,71.29470734)(736.27078857,71.48470998)
\curveto(736.43079501,71.67470696)(736.56079488,71.93970669)(736.66078857,72.27970998)
\curveto(736.69079475,72.40970622)(736.71579473,72.57470606)(736.73578857,72.77470998)
\curveto(736.7457947,72.97470566)(736.73079471,73.14470549)(736.69078857,73.28470998)
\curveto(736.61079483,73.57470506)(736.50079494,73.81470482)(736.36078857,74.00470998)
\curveto(736.21079523,74.20470443)(736.01079543,74.35970427)(735.76078857,74.46970998)
\curveto(735.71079573,74.48970414)(735.66579578,74.49970413)(735.62578857,74.49970998)
\curveto(735.58579586,74.50970412)(735.5407959,74.52470411)(735.49078857,74.54470998)
\curveto(735.38079606,74.57470406)(735.2407962,74.59470404)(735.07078857,74.60470998)
\curveto(734.90079654,74.61470402)(734.75579669,74.60470403)(734.63578857,74.57470998)
\curveto(734.5457969,74.55470408)(734.46079698,74.5297041)(734.38078857,74.49970998)
\curveto(734.30079714,74.47970415)(734.22079722,74.44470419)(734.14078857,74.39470998)
\curveto(733.87079757,74.22470441)(733.67579777,73.99970463)(733.55578857,73.71970998)
\curveto(733.43579801,73.44970518)(733.37579807,73.08970554)(733.37578857,72.63970998)
\curveto(733.39579805,72.61970601)(733.40079804,72.58970604)(733.39078857,72.54970998)
\curveto(733.38079806,72.50970612)(733.38079806,72.47470616)(733.39078857,72.44470998)
\curveto(733.41079803,72.39470624)(733.42579802,72.33970629)(733.43578857,72.27970998)
\curveto(733.43579801,72.2297064)(733.445798,72.17970645)(733.46578857,72.12970998)
\curveto(733.55579789,71.88970674)(733.67079777,71.67970695)(733.81078857,71.49970998)
\curveto(733.9407975,71.31970731)(734.12079732,71.17970745)(734.35078857,71.07970998)
\curveto(734.41079703,71.05970757)(734.47579697,71.03970759)(734.54578857,71.01970998)
\curveto(734.60579684,71.00970762)(734.67079677,70.99470764)(734.74078857,70.97470998)
\moveto(740.27578857,74.99470998)
\curveto(740.08579136,75.04470359)(739.88079156,75.04970358)(739.66078857,75.00970998)
\curveto(739.440792,74.97970365)(739.26079218,74.9347037)(739.12078857,74.87470998)
\curveto(738.75079269,74.70470393)(738.445793,74.44470419)(738.20578857,74.09470998)
\curveto(737.96579348,73.75470488)(737.8457936,73.31970531)(737.84578857,72.78970998)
\curveto(737.86579358,72.75970587)(737.87079357,72.71970591)(737.86078857,72.66970998)
\curveto(737.8407936,72.61970601)(737.83579361,72.57970605)(737.84578857,72.54970998)
\lineto(737.90578857,72.27970998)
\curveto(737.91579353,72.19970643)(737.93079351,72.11970651)(737.95078857,72.03970998)
\curveto(738.06079338,71.73970689)(738.20579324,71.47470716)(738.38578857,71.24470998)
\curveto(738.56579288,71.02470761)(738.79579265,70.85470778)(739.07578857,70.73470998)
\curveto(739.15579229,70.70470793)(739.23579221,70.67970795)(739.31578857,70.65970998)
\curveto(739.39579205,70.63970799)(739.48079196,70.61970801)(739.57078857,70.59970998)
\curveto(739.69079175,70.56970806)(739.8407916,70.55970807)(740.02078857,70.56970998)
\curveto(740.20079124,70.58970804)(740.3407911,70.61470802)(740.44078857,70.64470998)
\curveto(740.49079095,70.66470797)(740.53579091,70.67470796)(740.57578857,70.67470998)
\curveto(740.60579084,70.68470795)(740.6457908,70.69970793)(740.69578857,70.71970998)
\curveto(740.91579053,70.81970781)(741.11579033,70.94970768)(741.29578857,71.10970998)
\curveto(741.47578997,71.27970735)(741.61078983,71.47470716)(741.70078857,71.69470998)
\curveto(741.7407897,71.76470687)(741.77578967,71.85970677)(741.80578857,71.97970998)
\curveto(741.89578955,72.19970643)(741.9407895,72.45470618)(741.94078857,72.74470998)
\lineto(741.94078857,73.02970998)
\curveto(741.92078952,73.1297055)(741.90578954,73.22470541)(741.89578857,73.31470998)
\curveto(741.88578956,73.40470523)(741.86578958,73.49470514)(741.83578857,73.58470998)
\curveto(741.75578969,73.84470479)(741.62578982,74.08470455)(741.44578857,74.30470998)
\curveto(741.25579019,74.5347041)(741.0407904,74.70470393)(740.80078857,74.81470998)
\curveto(740.72079072,74.85470378)(740.6407908,74.88470375)(740.56078857,74.90470998)
\curveto(740.47079097,74.9347037)(740.37579107,74.96470367)(740.27578857,74.99470998)
}
}
{
\newrgbcolor{curcolor}{0 0 0}
\pscustom[linestyle=none,fillstyle=solid,fillcolor=curcolor]
{
\newpath
\moveto(741.22078857,78.63431936)
\lineto(741.22078857,79.26431936)
\lineto(741.22078857,79.45931936)
\curveto(741.22079022,79.52931683)(741.23079021,79.58931677)(741.25078857,79.63931936)
\curveto(741.29079015,79.70931665)(741.33079011,79.7593166)(741.37078857,79.78931936)
\curveto(741.42079002,79.82931653)(741.48578996,79.84931651)(741.56578857,79.84931936)
\curveto(741.6457898,79.8593165)(741.73078971,79.86431649)(741.82078857,79.86431936)
\lineto(742.54078857,79.86431936)
\curveto(743.02078842,79.86431649)(743.43078801,79.80431655)(743.77078857,79.68431936)
\curveto(744.11078733,79.56431679)(744.38578706,79.36931699)(744.59578857,79.09931936)
\curveto(744.6457868,79.02931733)(744.69078675,78.9593174)(744.73078857,78.88931936)
\curveto(744.78078666,78.82931753)(744.82578662,78.7543176)(744.86578857,78.66431936)
\curveto(744.87578657,78.64431771)(744.88578656,78.61431774)(744.89578857,78.57431936)
\curveto(744.91578653,78.53431782)(744.92078652,78.48931787)(744.91078857,78.43931936)
\curveto(744.88078656,78.34931801)(744.80578664,78.29431806)(744.68578857,78.27431936)
\curveto(744.57578687,78.2543181)(744.48078696,78.26931809)(744.40078857,78.31931936)
\curveto(744.33078711,78.34931801)(744.26578718,78.39431796)(744.20578857,78.45431936)
\curveto(744.15578729,78.52431783)(744.10578734,78.58931777)(744.05578857,78.64931936)
\curveto(744.00578744,78.71931764)(743.93078751,78.77931758)(743.83078857,78.82931936)
\curveto(743.7407877,78.88931747)(743.65078779,78.93931742)(743.56078857,78.97931936)
\curveto(743.53078791,78.99931736)(743.47078797,79.02431733)(743.38078857,79.05431936)
\curveto(743.30078814,79.08431727)(743.23078821,79.08931727)(743.17078857,79.06931936)
\curveto(743.03078841,79.03931732)(742.9407885,78.97931738)(742.90078857,78.88931936)
\curveto(742.87078857,78.80931755)(742.85578859,78.71931764)(742.85578857,78.61931936)
\curveto(742.85578859,78.51931784)(742.83078861,78.43431792)(742.78078857,78.36431936)
\curveto(742.71078873,78.27431808)(742.57078887,78.22931813)(742.36078857,78.22931936)
\lineto(741.80578857,78.22931936)
\lineto(741.58078857,78.22931936)
\curveto(741.50078994,78.23931812)(741.43579001,78.2593181)(741.38578857,78.28931936)
\curveto(741.30579014,78.34931801)(741.26079018,78.41931794)(741.25078857,78.49931936)
\curveto(741.2407902,78.51931784)(741.23579021,78.53931782)(741.23578857,78.55931936)
\curveto(741.23579021,78.58931777)(741.23079021,78.61431774)(741.22078857,78.63431936)
}
}
{
\newrgbcolor{curcolor}{0 0 0}
\pscustom[linestyle=none,fillstyle=solid,fillcolor=curcolor]
{
}
}
{
\newrgbcolor{curcolor}{0 0 0}
\pscustom[linestyle=none,fillstyle=solid,fillcolor=curcolor]
{
\newpath
\moveto(732.25078857,89.26463186)
\curveto(732.2407992,89.95462722)(732.36079908,90.55462662)(732.61078857,91.06463186)
\curveto(732.86079858,91.58462559)(733.19579825,91.9796252)(733.61578857,92.24963186)
\curveto(733.69579775,92.29962488)(733.78579766,92.34462483)(733.88578857,92.38463186)
\curveto(733.97579747,92.42462475)(734.07079737,92.46962471)(734.17078857,92.51963186)
\curveto(734.27079717,92.55962462)(734.37079707,92.58962459)(734.47078857,92.60963186)
\curveto(734.57079687,92.62962455)(734.67579677,92.64962453)(734.78578857,92.66963186)
\curveto(734.83579661,92.68962449)(734.88079656,92.69462448)(734.92078857,92.68463186)
\curveto(734.96079648,92.6746245)(735.00579644,92.6796245)(735.05578857,92.69963186)
\curveto(735.10579634,92.70962447)(735.19079625,92.71462446)(735.31078857,92.71463186)
\curveto(735.42079602,92.71462446)(735.50579594,92.70962447)(735.56578857,92.69963186)
\curveto(735.62579582,92.6796245)(735.68579576,92.66962451)(735.74578857,92.66963186)
\curveto(735.80579564,92.6796245)(735.86579558,92.6746245)(735.92578857,92.65463186)
\curveto(736.06579538,92.61462456)(736.20079524,92.5796246)(736.33078857,92.54963186)
\curveto(736.46079498,92.51962466)(736.58579486,92.4796247)(736.70578857,92.42963186)
\curveto(736.8457946,92.36962481)(736.97079447,92.29962488)(737.08078857,92.21963186)
\curveto(737.19079425,92.14962503)(737.30079414,92.0746251)(737.41078857,91.99463186)
\lineto(737.47078857,91.93463186)
\curveto(737.49079395,91.92462525)(737.51079393,91.90962527)(737.53078857,91.88963186)
\curveto(737.69079375,91.76962541)(737.83579361,91.63462554)(737.96578857,91.48463186)
\curveto(738.09579335,91.33462584)(738.22079322,91.174626)(738.34078857,91.00463186)
\curveto(738.56079288,90.69462648)(738.76579268,90.39962678)(738.95578857,90.11963186)
\curveto(739.09579235,89.88962729)(739.23079221,89.65962752)(739.36078857,89.42963186)
\curveto(739.49079195,89.20962797)(739.62579182,88.98962819)(739.76578857,88.76963186)
\curveto(739.93579151,88.51962866)(740.11579133,88.2796289)(740.30578857,88.04963186)
\curveto(740.49579095,87.82962935)(740.72079072,87.63962954)(740.98078857,87.47963186)
\curveto(741.0407904,87.43962974)(741.10079034,87.40462977)(741.16078857,87.37463186)
\curveto(741.21079023,87.34462983)(741.27579017,87.31462986)(741.35578857,87.28463186)
\curveto(741.42579002,87.26462991)(741.48578996,87.25962992)(741.53578857,87.26963186)
\curveto(741.60578984,87.28962989)(741.66078978,87.32462985)(741.70078857,87.37463186)
\curveto(741.73078971,87.42462975)(741.75078969,87.48462969)(741.76078857,87.55463186)
\lineto(741.76078857,87.79463186)
\lineto(741.76078857,88.54463186)
\lineto(741.76078857,91.34963186)
\lineto(741.76078857,92.00963186)
\curveto(741.76078968,92.09962508)(741.76578968,92.18462499)(741.77578857,92.26463186)
\curveto(741.77578967,92.34462483)(741.79578965,92.40962477)(741.83578857,92.45963186)
\curveto(741.87578957,92.50962467)(741.95078949,92.54962463)(742.06078857,92.57963186)
\curveto(742.16078928,92.61962456)(742.26078918,92.62962455)(742.36078857,92.60963186)
\lineto(742.49578857,92.60963186)
\curveto(742.56578888,92.58962459)(742.62578882,92.56962461)(742.67578857,92.54963186)
\curveto(742.72578872,92.52962465)(742.76578868,92.49462468)(742.79578857,92.44463186)
\curveto(742.83578861,92.39462478)(742.85578859,92.32462485)(742.85578857,92.23463186)
\lineto(742.85578857,91.96463186)
\lineto(742.85578857,91.06463186)
\lineto(742.85578857,87.55463186)
\lineto(742.85578857,86.48963186)
\curveto(742.85578859,86.40963077)(742.86078858,86.31963086)(742.87078857,86.21963186)
\curveto(742.87078857,86.11963106)(742.86078858,86.03463114)(742.84078857,85.96463186)
\curveto(742.77078867,85.75463142)(742.59078885,85.68963149)(742.30078857,85.76963186)
\curveto(742.26078918,85.7796314)(742.22578922,85.7796314)(742.19578857,85.76963186)
\curveto(742.15578929,85.76963141)(742.11078933,85.7796314)(742.06078857,85.79963186)
\curveto(741.98078946,85.81963136)(741.89578955,85.83963134)(741.80578857,85.85963186)
\curveto(741.71578973,85.8796313)(741.63078981,85.90463127)(741.55078857,85.93463186)
\curveto(741.06079038,86.09463108)(740.6457908,86.29463088)(740.30578857,86.53463186)
\curveto(740.05579139,86.71463046)(739.83079161,86.91963026)(739.63078857,87.14963186)
\curveto(739.42079202,87.3796298)(739.22579222,87.61962956)(739.04578857,87.86963186)
\curveto(738.86579258,88.12962905)(738.69579275,88.39462878)(738.53578857,88.66463186)
\curveto(738.36579308,88.94462823)(738.19079325,89.21462796)(738.01078857,89.47463186)
\curveto(737.93079351,89.58462759)(737.85579359,89.68962749)(737.78578857,89.78963186)
\curveto(737.71579373,89.89962728)(737.6407938,90.00962717)(737.56078857,90.11963186)
\curveto(737.53079391,90.15962702)(737.50079394,90.19462698)(737.47078857,90.22463186)
\curveto(737.43079401,90.26462691)(737.40079404,90.30462687)(737.38078857,90.34463186)
\curveto(737.27079417,90.48462669)(737.1457943,90.60962657)(737.00578857,90.71963186)
\curveto(736.97579447,90.73962644)(736.95079449,90.76462641)(736.93078857,90.79463186)
\curveto(736.90079454,90.82462635)(736.87079457,90.84962633)(736.84078857,90.86963186)
\curveto(736.7407947,90.94962623)(736.6407948,91.01462616)(736.54078857,91.06463186)
\curveto(736.440795,91.12462605)(736.33079511,91.179626)(736.21078857,91.22963186)
\curveto(736.1407953,91.25962592)(736.06579538,91.2796259)(735.98578857,91.28963186)
\lineto(735.74578857,91.34963186)
\lineto(735.65578857,91.34963186)
\curveto(735.62579582,91.35962582)(735.59579585,91.36462581)(735.56578857,91.36463186)
\curveto(735.49579595,91.38462579)(735.40079604,91.38962579)(735.28078857,91.37963186)
\curveto(735.15079629,91.3796258)(735.05079639,91.36962581)(734.98078857,91.34963186)
\curveto(734.90079654,91.32962585)(734.82579662,91.30962587)(734.75578857,91.28963186)
\curveto(734.67579677,91.2796259)(734.59579685,91.25962592)(734.51578857,91.22963186)
\curveto(734.27579717,91.11962606)(734.07579737,90.96962621)(733.91578857,90.77963186)
\curveto(733.7457977,90.59962658)(733.60579784,90.3796268)(733.49578857,90.11963186)
\curveto(733.47579797,90.04962713)(733.46079798,89.9796272)(733.45078857,89.90963186)
\curveto(733.43079801,89.83962734)(733.41079803,89.76462741)(733.39078857,89.68463186)
\curveto(733.37079807,89.60462757)(733.36079808,89.49462768)(733.36078857,89.35463186)
\curveto(733.36079808,89.22462795)(733.37079807,89.11962806)(733.39078857,89.03963186)
\curveto(733.40079804,88.9796282)(733.40579804,88.92462825)(733.40578857,88.87463186)
\curveto(733.40579804,88.82462835)(733.41579803,88.7746284)(733.43578857,88.72463186)
\curveto(733.47579797,88.62462855)(733.51579793,88.52962865)(733.55578857,88.43963186)
\curveto(733.59579785,88.35962882)(733.6407978,88.2796289)(733.69078857,88.19963186)
\curveto(733.71079773,88.16962901)(733.73579771,88.13962904)(733.76578857,88.10963186)
\curveto(733.79579765,88.08962909)(733.82079762,88.06462911)(733.84078857,88.03463186)
\lineto(733.91578857,87.95963186)
\curveto(733.93579751,87.92962925)(733.95579749,87.90462927)(733.97578857,87.88463186)
\lineto(734.18578857,87.73463186)
\curveto(734.2457972,87.69462948)(734.31079713,87.64962953)(734.38078857,87.59963186)
\curveto(734.47079697,87.53962964)(734.57579687,87.48962969)(734.69578857,87.44963186)
\curveto(734.80579664,87.41962976)(734.91579653,87.38462979)(735.02578857,87.34463186)
\curveto(735.13579631,87.30462987)(735.28079616,87.2796299)(735.46078857,87.26963186)
\curveto(735.63079581,87.25962992)(735.75579569,87.22962995)(735.83578857,87.17963186)
\curveto(735.91579553,87.12963005)(735.96079548,87.05463012)(735.97078857,86.95463186)
\curveto(735.98079546,86.85463032)(735.98579546,86.74463043)(735.98578857,86.62463186)
\curveto(735.98579546,86.58463059)(735.99079545,86.54463063)(736.00078857,86.50463186)
\curveto(736.00079544,86.46463071)(735.99579545,86.42963075)(735.98578857,86.39963186)
\curveto(735.96579548,86.34963083)(735.95579549,86.29963088)(735.95578857,86.24963186)
\curveto(735.95579549,86.20963097)(735.9457955,86.16963101)(735.92578857,86.12963186)
\curveto(735.86579558,86.03963114)(735.73079571,85.99463118)(735.52078857,85.99463186)
\lineto(735.40078857,85.99463186)
\curveto(735.3407961,86.00463117)(735.28079616,86.00963117)(735.22078857,86.00963186)
\curveto(735.15079629,86.01963116)(735.08579636,86.02963115)(735.02578857,86.03963186)
\curveto(734.91579653,86.05963112)(734.81579663,86.0796311)(734.72578857,86.09963186)
\curveto(734.62579682,86.11963106)(734.53079691,86.14963103)(734.44078857,86.18963186)
\curveto(734.37079707,86.20963097)(734.31079713,86.22963095)(734.26078857,86.24963186)
\lineto(734.08078857,86.30963186)
\curveto(733.82079762,86.42963075)(733.57579787,86.58463059)(733.34578857,86.77463186)
\curveto(733.11579833,86.9746302)(732.93079851,87.18962999)(732.79078857,87.41963186)
\curveto(732.71079873,87.52962965)(732.6457988,87.64462953)(732.59578857,87.76463186)
\lineto(732.44578857,88.15463186)
\curveto(732.39579905,88.26462891)(732.36579908,88.3796288)(732.35578857,88.49963186)
\curveto(732.33579911,88.61962856)(732.31079913,88.74462843)(732.28078857,88.87463186)
\curveto(732.28079916,88.94462823)(732.28079916,89.00962817)(732.28078857,89.06963186)
\curveto(732.27079917,89.12962805)(732.26079918,89.19462798)(732.25078857,89.26463186)
}
}
{
\newrgbcolor{curcolor}{0 0 0}
\pscustom[linestyle=none,fillstyle=solid,fillcolor=curcolor]
{
\newpath
\moveto(737.77078857,101.36424123)
\lineto(738.02578857,101.36424123)
\curveto(738.10579334,101.37423353)(738.18079326,101.36923353)(738.25078857,101.34924123)
\lineto(738.49078857,101.34924123)
\lineto(738.65578857,101.34924123)
\curveto(738.75579269,101.32923357)(738.86079258,101.31923358)(738.97078857,101.31924123)
\curveto(739.07079237,101.31923358)(739.17079227,101.30923359)(739.27078857,101.28924123)
\lineto(739.42078857,101.28924123)
\curveto(739.56079188,101.25923364)(739.70079174,101.23923366)(739.84078857,101.22924123)
\curveto(739.97079147,101.21923368)(740.10079134,101.19423371)(740.23078857,101.15424123)
\curveto(740.31079113,101.13423377)(740.39579105,101.11423379)(740.48578857,101.09424123)
\lineto(740.72578857,101.03424123)
\lineto(741.02578857,100.91424123)
\curveto(741.11579033,100.88423402)(741.20579024,100.84923405)(741.29578857,100.80924123)
\curveto(741.51578993,100.70923419)(741.73078971,100.57423433)(741.94078857,100.40424123)
\curveto(742.15078929,100.24423466)(742.32078912,100.06923483)(742.45078857,99.87924123)
\curveto(742.49078895,99.82923507)(742.53078891,99.76923513)(742.57078857,99.69924123)
\curveto(742.60078884,99.63923526)(742.63578881,99.57923532)(742.67578857,99.51924123)
\curveto(742.72578872,99.43923546)(742.76578868,99.34423556)(742.79578857,99.23424123)
\curveto(742.82578862,99.12423578)(742.85578859,99.01923588)(742.88578857,98.91924123)
\curveto(742.92578852,98.80923609)(742.95078849,98.6992362)(742.96078857,98.58924123)
\curveto(742.97078847,98.47923642)(742.98578846,98.36423654)(743.00578857,98.24424123)
\curveto(743.01578843,98.2042367)(743.01578843,98.15923674)(743.00578857,98.10924123)
\curveto(743.00578844,98.06923683)(743.01078843,98.02923687)(743.02078857,97.98924123)
\curveto(743.03078841,97.94923695)(743.03578841,97.89423701)(743.03578857,97.82424123)
\curveto(743.03578841,97.75423715)(743.03078841,97.7042372)(743.02078857,97.67424123)
\curveto(743.00078844,97.62423728)(742.99578845,97.57923732)(743.00578857,97.53924123)
\curveto(743.01578843,97.4992374)(743.01578843,97.46423744)(743.00578857,97.43424123)
\lineto(743.00578857,97.34424123)
\curveto(742.98578846,97.28423762)(742.97078847,97.21923768)(742.96078857,97.14924123)
\curveto(742.96078848,97.08923781)(742.95578849,97.02423788)(742.94578857,96.95424123)
\curveto(742.89578855,96.78423812)(742.8457886,96.62423828)(742.79578857,96.47424123)
\curveto(742.7457887,96.32423858)(742.68078876,96.17923872)(742.60078857,96.03924123)
\curveto(742.56078888,95.98923891)(742.53078891,95.93423897)(742.51078857,95.87424123)
\curveto(742.48078896,95.82423908)(742.445789,95.77423913)(742.40578857,95.72424123)
\curveto(742.22578922,95.48423942)(742.00578944,95.28423962)(741.74578857,95.12424123)
\curveto(741.48578996,94.96423994)(741.20079024,94.82424008)(740.89078857,94.70424123)
\curveto(740.75079069,94.64424026)(740.61079083,94.5992403)(740.47078857,94.56924123)
\curveto(740.32079112,94.53924036)(740.16579128,94.5042404)(740.00578857,94.46424123)
\curveto(739.89579155,94.44424046)(739.78579166,94.42924047)(739.67578857,94.41924123)
\curveto(739.56579188,94.40924049)(739.45579199,94.39424051)(739.34578857,94.37424123)
\curveto(739.30579214,94.36424054)(739.26579218,94.35924054)(739.22578857,94.35924123)
\curveto(739.18579226,94.36924053)(739.1457923,94.36924053)(739.10578857,94.35924123)
\curveto(739.05579239,94.34924055)(739.00579244,94.34424056)(738.95578857,94.34424123)
\lineto(738.79078857,94.34424123)
\curveto(738.7407927,94.32424058)(738.69079275,94.31924058)(738.64078857,94.32924123)
\curveto(738.58079286,94.33924056)(738.52579292,94.33924056)(738.47578857,94.32924123)
\curveto(738.43579301,94.31924058)(738.39079305,94.31924058)(738.34078857,94.32924123)
\curveto(738.29079315,94.33924056)(738.2407932,94.33424057)(738.19078857,94.31424123)
\curveto(738.12079332,94.29424061)(738.0457934,94.28924061)(737.96578857,94.29924123)
\curveto(737.87579357,94.30924059)(737.79079365,94.31424059)(737.71078857,94.31424123)
\curveto(737.62079382,94.31424059)(737.52079392,94.30924059)(737.41078857,94.29924123)
\curveto(737.29079415,94.28924061)(737.19079425,94.29424061)(737.11078857,94.31424123)
\lineto(736.82578857,94.31424123)
\lineto(736.19578857,94.35924123)
\curveto(736.09579535,94.36924053)(736.00079544,94.37924052)(735.91078857,94.38924123)
\lineto(735.61078857,94.41924123)
\curveto(735.56079588,94.43924046)(735.51079593,94.44424046)(735.46078857,94.43424123)
\curveto(735.40079604,94.43424047)(735.3457961,94.44424046)(735.29578857,94.46424123)
\curveto(735.12579632,94.51424039)(734.96079648,94.55424035)(734.80078857,94.58424123)
\curveto(734.63079681,94.61424029)(734.47079697,94.66424024)(734.32078857,94.73424123)
\curveto(733.86079758,94.92423998)(733.48579796,95.14423976)(733.19578857,95.39424123)
\curveto(732.90579854,95.65423925)(732.66079878,96.01423889)(732.46078857,96.47424123)
\curveto(732.41079903,96.6042383)(732.37579907,96.73423817)(732.35578857,96.86424123)
\curveto(732.33579911,97.0042379)(732.31079913,97.14423776)(732.28078857,97.28424123)
\curveto(732.27079917,97.35423755)(732.26579918,97.41923748)(732.26578857,97.47924123)
\curveto(732.26579918,97.53923736)(732.26079918,97.6042373)(732.25078857,97.67424123)
\curveto(732.23079921,98.5042364)(732.38079906,99.17423573)(732.70078857,99.68424123)
\curveto(733.01079843,100.19423471)(733.45079799,100.57423433)(734.02078857,100.82424123)
\curveto(734.1407973,100.87423403)(734.26579718,100.91923398)(734.39578857,100.95924123)
\curveto(734.52579692,100.9992339)(734.66079678,101.04423386)(734.80078857,101.09424123)
\curveto(734.88079656,101.11423379)(734.96579648,101.12923377)(735.05578857,101.13924123)
\lineto(735.29578857,101.19924123)
\curveto(735.40579604,101.22923367)(735.51579593,101.24423366)(735.62578857,101.24424123)
\curveto(735.73579571,101.25423365)(735.8457956,101.26923363)(735.95578857,101.28924123)
\curveto(736.00579544,101.30923359)(736.05079539,101.31423359)(736.09078857,101.30424123)
\curveto(736.13079531,101.3042336)(736.17079527,101.30923359)(736.21078857,101.31924123)
\curveto(736.26079518,101.32923357)(736.31579513,101.32923357)(736.37578857,101.31924123)
\curveto(736.42579502,101.31923358)(736.47579497,101.32423358)(736.52578857,101.33424123)
\lineto(736.66078857,101.33424123)
\curveto(736.72079472,101.35423355)(736.79079465,101.35423355)(736.87078857,101.33424123)
\curveto(736.9407945,101.32423358)(737.00579444,101.32923357)(737.06578857,101.34924123)
\curveto(737.09579435,101.35923354)(737.13579431,101.36423354)(737.18578857,101.36424123)
\lineto(737.30578857,101.36424123)
\lineto(737.77078857,101.36424123)
\moveto(740.09578857,99.81924123)
\curveto(739.77579167,99.91923498)(739.41079203,99.97923492)(739.00078857,99.99924123)
\curveto(738.59079285,100.01923488)(738.18079326,100.02923487)(737.77078857,100.02924123)
\curveto(737.3407941,100.02923487)(736.92079452,100.01923488)(736.51078857,99.99924123)
\curveto(736.10079534,99.97923492)(735.71579573,99.93423497)(735.35578857,99.86424123)
\curveto(734.99579645,99.79423511)(734.67579677,99.68423522)(734.39578857,99.53424123)
\curveto(734.10579734,99.39423551)(733.87079757,99.1992357)(733.69078857,98.94924123)
\curveto(733.58079786,98.78923611)(733.50079794,98.60923629)(733.45078857,98.40924123)
\curveto(733.39079805,98.20923669)(733.36079808,97.96423694)(733.36078857,97.67424123)
\curveto(733.38079806,97.65423725)(733.39079805,97.61923728)(733.39078857,97.56924123)
\curveto(733.38079806,97.51923738)(733.38079806,97.47923742)(733.39078857,97.44924123)
\curveto(733.41079803,97.36923753)(733.43079801,97.29423761)(733.45078857,97.22424123)
\curveto(733.46079798,97.16423774)(733.48079796,97.0992378)(733.51078857,97.02924123)
\curveto(733.63079781,96.75923814)(733.80079764,96.53923836)(734.02078857,96.36924123)
\curveto(734.23079721,96.20923869)(734.47579697,96.07423883)(734.75578857,95.96424123)
\curveto(734.86579658,95.91423899)(734.98579646,95.87423903)(735.11578857,95.84424123)
\curveto(735.23579621,95.82423908)(735.36079608,95.7992391)(735.49078857,95.76924123)
\curveto(735.5407959,95.74923915)(735.59579585,95.73923916)(735.65578857,95.73924123)
\curveto(735.70579574,95.73923916)(735.75579569,95.73423917)(735.80578857,95.72424123)
\curveto(735.89579555,95.71423919)(735.99079545,95.7042392)(736.09078857,95.69424123)
\curveto(736.18079526,95.68423922)(736.27579517,95.67423923)(736.37578857,95.66424123)
\curveto(736.45579499,95.66423924)(736.5407949,95.65923924)(736.63078857,95.64924123)
\lineto(736.87078857,95.64924123)
\lineto(737.05078857,95.64924123)
\curveto(737.08079436,95.63923926)(737.11579433,95.63423927)(737.15578857,95.63424123)
\lineto(737.29078857,95.63424123)
\lineto(737.74078857,95.63424123)
\curveto(737.82079362,95.63423927)(737.90579354,95.62923927)(737.99578857,95.61924123)
\curveto(738.07579337,95.61923928)(738.15079329,95.62923927)(738.22078857,95.64924123)
\lineto(738.49078857,95.64924123)
\curveto(738.51079293,95.64923925)(738.5407929,95.64423926)(738.58078857,95.63424123)
\curveto(738.61079283,95.63423927)(738.63579281,95.63923926)(738.65578857,95.64924123)
\curveto(738.75579269,95.65923924)(738.85579259,95.66423924)(738.95578857,95.66424123)
\curveto(739.0457924,95.67423923)(739.1457923,95.68423922)(739.25578857,95.69424123)
\curveto(739.37579207,95.72423918)(739.50079194,95.73923916)(739.63078857,95.73924123)
\curveto(739.75079169,95.74923915)(739.86579158,95.77423913)(739.97578857,95.81424123)
\curveto(740.27579117,95.89423901)(740.5407909,95.97923892)(740.77078857,96.06924123)
\curveto(741.00079044,96.16923873)(741.21579023,96.31423859)(741.41578857,96.50424123)
\curveto(741.61578983,96.71423819)(741.76578968,96.97923792)(741.86578857,97.29924123)
\curveto(741.88578956,97.33923756)(741.89578955,97.37423753)(741.89578857,97.40424123)
\curveto(741.88578956,97.44423746)(741.89078955,97.48923741)(741.91078857,97.53924123)
\curveto(741.92078952,97.57923732)(741.93078951,97.64923725)(741.94078857,97.74924123)
\curveto(741.95078949,97.85923704)(741.9457895,97.94423696)(741.92578857,98.00424123)
\curveto(741.90578954,98.07423683)(741.89578955,98.14423676)(741.89578857,98.21424123)
\curveto(741.88578956,98.28423662)(741.87078957,98.34923655)(741.85078857,98.40924123)
\curveto(741.79078965,98.60923629)(741.70578974,98.78923611)(741.59578857,98.94924123)
\curveto(741.57578987,98.97923592)(741.55578989,99.0042359)(741.53578857,99.02424123)
\lineto(741.47578857,99.08424123)
\curveto(741.45578999,99.12423578)(741.41579003,99.17423573)(741.35578857,99.23424123)
\curveto(741.21579023,99.33423557)(741.08579036,99.41923548)(740.96578857,99.48924123)
\curveto(740.8457906,99.55923534)(740.70079074,99.62923527)(740.53078857,99.69924123)
\curveto(740.46079098,99.72923517)(740.39079105,99.74923515)(740.32078857,99.75924123)
\curveto(740.25079119,99.77923512)(740.17579127,99.7992351)(740.09578857,99.81924123)
}
}
{
\newrgbcolor{curcolor}{0 0 0}
\pscustom[linestyle=none,fillstyle=solid,fillcolor=curcolor]
{
\newpath
\moveto(732.25078857,106.77385061)
\curveto(732.25079919,106.87384575)(732.26079918,106.96884566)(732.28078857,107.05885061)
\curveto(732.29079915,107.14884548)(732.32079912,107.21384541)(732.37078857,107.25385061)
\curveto(732.45079899,107.31384531)(732.55579889,107.34384528)(732.68578857,107.34385061)
\lineto(733.07578857,107.34385061)
\lineto(734.57578857,107.34385061)
\lineto(740.96578857,107.34385061)
\lineto(742.13578857,107.34385061)
\lineto(742.45078857,107.34385061)
\curveto(742.55078889,107.35384527)(742.63078881,107.33884529)(742.69078857,107.29885061)
\curveto(742.77078867,107.24884538)(742.82078862,107.17384545)(742.84078857,107.07385061)
\curveto(742.85078859,106.98384564)(742.85578859,106.87384575)(742.85578857,106.74385061)
\lineto(742.85578857,106.51885061)
\curveto(742.83578861,106.43884619)(742.82078862,106.36884626)(742.81078857,106.30885061)
\curveto(742.79078865,106.24884638)(742.75078869,106.19884643)(742.69078857,106.15885061)
\curveto(742.63078881,106.11884651)(742.55578889,106.09884653)(742.46578857,106.09885061)
\lineto(742.16578857,106.09885061)
\lineto(741.07078857,106.09885061)
\lineto(735.73078857,106.09885061)
\curveto(735.6407958,106.07884655)(735.56579588,106.06384656)(735.50578857,106.05385061)
\curveto(735.43579601,106.05384657)(735.37579607,106.0238466)(735.32578857,105.96385061)
\curveto(735.27579617,105.89384673)(735.25079619,105.80384682)(735.25078857,105.69385061)
\curveto(735.2407962,105.59384703)(735.23579621,105.48384714)(735.23578857,105.36385061)
\lineto(735.23578857,104.22385061)
\lineto(735.23578857,103.72885061)
\curveto(735.22579622,103.56884906)(735.16579628,103.45884917)(735.05578857,103.39885061)
\curveto(735.02579642,103.37884925)(734.99579645,103.36884926)(734.96578857,103.36885061)
\curveto(734.92579652,103.36884926)(734.88079656,103.36384926)(734.83078857,103.35385061)
\curveto(734.71079673,103.33384929)(734.60079684,103.33884929)(734.50078857,103.36885061)
\curveto(734.40079704,103.40884922)(734.33079711,103.46384916)(734.29078857,103.53385061)
\curveto(734.2407972,103.61384901)(734.21579723,103.73384889)(734.21578857,103.89385061)
\curveto(734.21579723,104.05384857)(734.20079724,104.18884844)(734.17078857,104.29885061)
\curveto(734.16079728,104.34884828)(734.15579729,104.40384822)(734.15578857,104.46385061)
\curveto(734.1457973,104.5238481)(734.13079731,104.58384804)(734.11078857,104.64385061)
\curveto(734.06079738,104.79384783)(734.01079743,104.93884769)(733.96078857,105.07885061)
\curveto(733.90079754,105.21884741)(733.83079761,105.35384727)(733.75078857,105.48385061)
\curveto(733.66079778,105.623847)(733.55579789,105.74384688)(733.43578857,105.84385061)
\curveto(733.31579813,105.94384668)(733.18579826,106.03884659)(733.04578857,106.12885061)
\curveto(732.9457985,106.18884644)(732.83579861,106.23384639)(732.71578857,106.26385061)
\curveto(732.59579885,106.30384632)(732.49079895,106.35384627)(732.40078857,106.41385061)
\curveto(732.3407991,106.46384616)(732.30079914,106.53384609)(732.28078857,106.62385061)
\curveto(732.27079917,106.64384598)(732.26579918,106.66884596)(732.26578857,106.69885061)
\curveto(732.26579918,106.7288459)(732.26079918,106.75384587)(732.25078857,106.77385061)
}
}
{
\newrgbcolor{curcolor}{0 0 0}
\pscustom[linestyle=none,fillstyle=solid,fillcolor=curcolor]
{
\newpath
\moveto(732.25078857,115.12345998)
\curveto(732.25079919,115.22345513)(732.26079918,115.31845503)(732.28078857,115.40845998)
\curveto(732.29079915,115.49845485)(732.32079912,115.56345479)(732.37078857,115.60345998)
\curveto(732.45079899,115.66345469)(732.55579889,115.69345466)(732.68578857,115.69345998)
\lineto(733.07578857,115.69345998)
\lineto(734.57578857,115.69345998)
\lineto(740.96578857,115.69345998)
\lineto(742.13578857,115.69345998)
\lineto(742.45078857,115.69345998)
\curveto(742.55078889,115.70345465)(742.63078881,115.68845466)(742.69078857,115.64845998)
\curveto(742.77078867,115.59845475)(742.82078862,115.52345483)(742.84078857,115.42345998)
\curveto(742.85078859,115.33345502)(742.85578859,115.22345513)(742.85578857,115.09345998)
\lineto(742.85578857,114.86845998)
\curveto(742.83578861,114.78845556)(742.82078862,114.71845563)(742.81078857,114.65845998)
\curveto(742.79078865,114.59845575)(742.75078869,114.5484558)(742.69078857,114.50845998)
\curveto(742.63078881,114.46845588)(742.55578889,114.4484559)(742.46578857,114.44845998)
\lineto(742.16578857,114.44845998)
\lineto(741.07078857,114.44845998)
\lineto(735.73078857,114.44845998)
\curveto(735.6407958,114.42845592)(735.56579588,114.41345594)(735.50578857,114.40345998)
\curveto(735.43579601,114.40345595)(735.37579607,114.37345598)(735.32578857,114.31345998)
\curveto(735.27579617,114.24345611)(735.25079619,114.1534562)(735.25078857,114.04345998)
\curveto(735.2407962,113.94345641)(735.23579621,113.83345652)(735.23578857,113.71345998)
\lineto(735.23578857,112.57345998)
\lineto(735.23578857,112.07845998)
\curveto(735.22579622,111.91845843)(735.16579628,111.80845854)(735.05578857,111.74845998)
\curveto(735.02579642,111.72845862)(734.99579645,111.71845863)(734.96578857,111.71845998)
\curveto(734.92579652,111.71845863)(734.88079656,111.71345864)(734.83078857,111.70345998)
\curveto(734.71079673,111.68345867)(734.60079684,111.68845866)(734.50078857,111.71845998)
\curveto(734.40079704,111.75845859)(734.33079711,111.81345854)(734.29078857,111.88345998)
\curveto(734.2407972,111.96345839)(734.21579723,112.08345827)(734.21578857,112.24345998)
\curveto(734.21579723,112.40345795)(734.20079724,112.53845781)(734.17078857,112.64845998)
\curveto(734.16079728,112.69845765)(734.15579729,112.7534576)(734.15578857,112.81345998)
\curveto(734.1457973,112.87345748)(734.13079731,112.93345742)(734.11078857,112.99345998)
\curveto(734.06079738,113.14345721)(734.01079743,113.28845706)(733.96078857,113.42845998)
\curveto(733.90079754,113.56845678)(733.83079761,113.70345665)(733.75078857,113.83345998)
\curveto(733.66079778,113.97345638)(733.55579789,114.09345626)(733.43578857,114.19345998)
\curveto(733.31579813,114.29345606)(733.18579826,114.38845596)(733.04578857,114.47845998)
\curveto(732.9457985,114.53845581)(732.83579861,114.58345577)(732.71578857,114.61345998)
\curveto(732.59579885,114.6534557)(732.49079895,114.70345565)(732.40078857,114.76345998)
\curveto(732.3407991,114.81345554)(732.30079914,114.88345547)(732.28078857,114.97345998)
\curveto(732.27079917,114.99345536)(732.26579918,115.01845533)(732.26578857,115.04845998)
\curveto(732.26579918,115.07845527)(732.26079918,115.10345525)(732.25078857,115.12345998)
}
}
{
\newrgbcolor{curcolor}{0 0 0}
\pscustom[linestyle=none,fillstyle=solid,fillcolor=curcolor]
{
\newpath
\moveto(753.08710449,37.28705373)
\curveto(753.08711519,37.35704805)(753.08711519,37.43704797)(753.08710449,37.52705373)
\curveto(753.0771152,37.61704779)(753.0771152,37.70204771)(753.08710449,37.78205373)
\curveto(753.08711519,37.87204754)(753.09711518,37.95204746)(753.11710449,38.02205373)
\curveto(753.13711514,38.10204731)(753.16711511,38.15704725)(753.20710449,38.18705373)
\curveto(753.25711502,38.21704719)(753.33211494,38.23704717)(753.43210449,38.24705373)
\curveto(753.52211475,38.26704714)(753.62711465,38.27704713)(753.74710449,38.27705373)
\curveto(753.85711442,38.28704712)(753.9721143,38.28704712)(754.09210449,38.27705373)
\lineto(754.39210449,38.27705373)
\lineto(757.40710449,38.27705373)
\lineto(760.30210449,38.27705373)
\curveto(760.63210764,38.27704713)(760.95710732,38.27204714)(761.27710449,38.26205373)
\curveto(761.58710669,38.26204715)(761.86710641,38.22204719)(762.11710449,38.14205373)
\curveto(762.46710581,38.02204739)(762.76210551,37.86704754)(763.00210449,37.67705373)
\curveto(763.23210504,37.48704792)(763.43210484,37.24704816)(763.60210449,36.95705373)
\curveto(763.65210462,36.89704851)(763.68710459,36.83204858)(763.70710449,36.76205373)
\curveto(763.72710455,36.70204871)(763.75210452,36.63204878)(763.78210449,36.55205373)
\curveto(763.83210444,36.43204898)(763.86710441,36.30204911)(763.88710449,36.16205373)
\curveto(763.91710436,36.03204938)(763.94710433,35.89704951)(763.97710449,35.75705373)
\curveto(763.99710428,35.7070497)(764.00210427,35.65704975)(763.99210449,35.60705373)
\curveto(763.98210429,35.55704985)(763.98210429,35.50204991)(763.99210449,35.44205373)
\curveto(764.00210427,35.42204999)(764.00210427,35.39705001)(763.99210449,35.36705373)
\curveto(763.99210428,35.33705007)(763.99710428,35.3120501)(764.00710449,35.29205373)
\curveto(764.01710426,35.25205016)(764.02210425,35.19705021)(764.02210449,35.12705373)
\curveto(764.02210425,35.05705035)(764.01710426,35.00205041)(764.00710449,34.96205373)
\curveto(763.99710428,34.9120505)(763.99710428,34.85705055)(764.00710449,34.79705373)
\curveto(764.01710426,34.73705067)(764.01210426,34.68205073)(763.99210449,34.63205373)
\curveto(763.96210431,34.50205091)(763.94210433,34.37705103)(763.93210449,34.25705373)
\curveto(763.92210435,34.13705127)(763.89710438,34.02205139)(763.85710449,33.91205373)
\curveto(763.73710454,33.54205187)(763.56710471,33.22205219)(763.34710449,32.95205373)
\curveto(763.12710515,32.68205273)(762.84710543,32.47205294)(762.50710449,32.32205373)
\curveto(762.38710589,32.27205314)(762.26210601,32.22705318)(762.13210449,32.18705373)
\curveto(762.00210627,32.15705325)(761.86710641,32.12205329)(761.72710449,32.08205373)
\curveto(761.6771066,32.07205334)(761.63710664,32.06705334)(761.60710449,32.06705373)
\curveto(761.56710671,32.06705334)(761.52210675,32.06205335)(761.47210449,32.05205373)
\curveto(761.44210683,32.04205337)(761.40710687,32.03705337)(761.36710449,32.03705373)
\curveto(761.31710696,32.03705337)(761.277107,32.03205338)(761.24710449,32.02205373)
\lineto(761.08210449,32.02205373)
\curveto(761.00210727,32.00205341)(760.90210737,31.99705341)(760.78210449,32.00705373)
\curveto(760.65210762,32.01705339)(760.56210771,32.03205338)(760.51210449,32.05205373)
\curveto(760.42210785,32.07205334)(760.35710792,32.12705328)(760.31710449,32.21705373)
\curveto(760.29710798,32.24705316)(760.29210798,32.27705313)(760.30210449,32.30705373)
\curveto(760.30210797,32.33705307)(760.29710798,32.37705303)(760.28710449,32.42705373)
\curveto(760.277108,32.46705294)(760.272108,32.5070529)(760.27210449,32.54705373)
\lineto(760.27210449,32.69705373)
\curveto(760.272108,32.81705259)(760.277108,32.93705247)(760.28710449,33.05705373)
\curveto(760.28710799,33.18705222)(760.32210795,33.27705213)(760.39210449,33.32705373)
\curveto(760.45210782,33.36705204)(760.51210776,33.38705202)(760.57210449,33.38705373)
\curveto(760.63210764,33.38705202)(760.70210757,33.39705201)(760.78210449,33.41705373)
\curveto(760.81210746,33.42705198)(760.84710743,33.42705198)(760.88710449,33.41705373)
\curveto(760.91710736,33.41705199)(760.94210733,33.42205199)(760.96210449,33.43205373)
\lineto(761.17210449,33.43205373)
\curveto(761.22210705,33.45205196)(761.272107,33.45705195)(761.32210449,33.44705373)
\curveto(761.36210691,33.44705196)(761.40710687,33.45705195)(761.45710449,33.47705373)
\curveto(761.58710669,33.5070519)(761.71210656,33.53705187)(761.83210449,33.56705373)
\curveto(761.94210633,33.59705181)(762.04710623,33.64205177)(762.14710449,33.70205373)
\curveto(762.43710584,33.87205154)(762.64210563,34.14205127)(762.76210449,34.51205373)
\curveto(762.78210549,34.56205085)(762.79710548,34.6120508)(762.80710449,34.66205373)
\curveto(762.80710547,34.72205069)(762.81710546,34.77705063)(762.83710449,34.82705373)
\lineto(762.83710449,34.90205373)
\curveto(762.84710543,34.97205044)(762.85710542,35.06705034)(762.86710449,35.18705373)
\curveto(762.86710541,35.31705009)(762.85710542,35.41704999)(762.83710449,35.48705373)
\curveto(762.81710546,35.55704985)(762.80210547,35.62704978)(762.79210449,35.69705373)
\curveto(762.7721055,35.77704963)(762.75210552,35.84704956)(762.73210449,35.90705373)
\curveto(762.5721057,36.28704912)(762.29710598,36.56204885)(761.90710449,36.73205373)
\curveto(761.7771065,36.78204863)(761.62210665,36.81704859)(761.44210449,36.83705373)
\curveto(761.26210701,36.86704854)(761.0771072,36.88204853)(760.88710449,36.88205373)
\curveto(760.68710759,36.89204852)(760.48710779,36.89204852)(760.28710449,36.88205373)
\lineto(759.71710449,36.88205373)
\lineto(755.47210449,36.88205373)
\lineto(753.92710449,36.88205373)
\curveto(753.81711446,36.88204853)(753.69711458,36.87704853)(753.56710449,36.86705373)
\curveto(753.43711484,36.85704855)(753.33211494,36.87704853)(753.25210449,36.92705373)
\curveto(753.18211509,36.98704842)(753.13211514,37.06704834)(753.10210449,37.16705373)
\curveto(753.10211517,37.18704822)(753.10211517,37.2070482)(753.10210449,37.22705373)
\curveto(753.10211517,37.24704816)(753.09711518,37.26704814)(753.08710449,37.28705373)
}
}
{
\newrgbcolor{curcolor}{0 0 0}
\pscustom[linestyle=none,fillstyle=solid,fillcolor=curcolor]
{
\newpath
\moveto(756.04210449,40.82072561)
\lineto(756.04210449,41.25572561)
\curveto(756.04211223,41.40572364)(756.08211219,41.51072354)(756.16210449,41.57072561)
\curveto(756.24211203,41.62072343)(756.34211193,41.6457234)(756.46210449,41.64572561)
\curveto(756.58211169,41.65572339)(756.70211157,41.66072339)(756.82210449,41.66072561)
\lineto(758.24710449,41.66072561)
\lineto(760.51210449,41.66072561)
\lineto(761.20210449,41.66072561)
\curveto(761.43210684,41.66072339)(761.63210664,41.68572336)(761.80210449,41.73572561)
\curveto(762.25210602,41.89572315)(762.56710571,42.19572285)(762.74710449,42.63572561)
\curveto(762.83710544,42.85572219)(762.8721054,43.12072193)(762.85210449,43.43072561)
\curveto(762.82210545,43.74072131)(762.76710551,43.99072106)(762.68710449,44.18072561)
\curveto(762.54710573,44.51072054)(762.3721059,44.77072028)(762.16210449,44.96072561)
\curveto(761.94210633,45.16071989)(761.65710662,45.31571973)(761.30710449,45.42572561)
\curveto(761.22710705,45.45571959)(761.14710713,45.47571957)(761.06710449,45.48572561)
\curveto(760.98710729,45.49571955)(760.90210737,45.51071954)(760.81210449,45.53072561)
\curveto(760.76210751,45.54071951)(760.71710756,45.54071951)(760.67710449,45.53072561)
\curveto(760.63710764,45.53071952)(760.59210768,45.54071951)(760.54210449,45.56072561)
\lineto(760.22710449,45.56072561)
\curveto(760.14710813,45.58071947)(760.05710822,45.58571946)(759.95710449,45.57572561)
\curveto(759.84710843,45.56571948)(759.74710853,45.56071949)(759.65710449,45.56072561)
\lineto(758.48710449,45.56072561)
\lineto(756.89710449,45.56072561)
\curveto(756.7771115,45.56071949)(756.65211162,45.55571949)(756.52210449,45.54572561)
\curveto(756.38211189,45.5457195)(756.272112,45.57071948)(756.19210449,45.62072561)
\curveto(756.14211213,45.66071939)(756.11211216,45.70571934)(756.10210449,45.75572561)
\curveto(756.08211219,45.81571923)(756.06211221,45.88571916)(756.04210449,45.96572561)
\lineto(756.04210449,46.19072561)
\curveto(756.04211223,46.31071874)(756.04711223,46.41571863)(756.05710449,46.50572561)
\curveto(756.06711221,46.60571844)(756.11211216,46.68071837)(756.19210449,46.73072561)
\curveto(756.24211203,46.78071827)(756.31711196,46.80571824)(756.41710449,46.80572561)
\lineto(756.70210449,46.80572561)
\lineto(757.72210449,46.80572561)
\lineto(761.75710449,46.80572561)
\lineto(763.10710449,46.80572561)
\curveto(763.22710505,46.80571824)(763.34210493,46.80071825)(763.45210449,46.79072561)
\curveto(763.55210472,46.79071826)(763.62710465,46.75571829)(763.67710449,46.68572561)
\curveto(763.70710457,46.6457184)(763.73210454,46.58571846)(763.75210449,46.50572561)
\curveto(763.76210451,46.42571862)(763.7721045,46.33571871)(763.78210449,46.23572561)
\curveto(763.78210449,46.1457189)(763.7771045,46.05571899)(763.76710449,45.96572561)
\curveto(763.75710452,45.88571916)(763.73710454,45.82571922)(763.70710449,45.78572561)
\curveto(763.66710461,45.73571931)(763.60210467,45.69071936)(763.51210449,45.65072561)
\curveto(763.4721048,45.64071941)(763.41710486,45.63071942)(763.34710449,45.62072561)
\curveto(763.277105,45.62071943)(763.21210506,45.61571943)(763.15210449,45.60572561)
\curveto(763.08210519,45.59571945)(763.02710525,45.57571947)(762.98710449,45.54572561)
\curveto(762.94710533,45.51571953)(762.93210534,45.47071958)(762.94210449,45.41072561)
\curveto(762.96210531,45.33071972)(763.02210525,45.2507198)(763.12210449,45.17072561)
\curveto(763.21210506,45.09071996)(763.28210499,45.01572003)(763.33210449,44.94572561)
\curveto(763.49210478,44.72572032)(763.63210464,44.47572057)(763.75210449,44.19572561)
\curveto(763.80210447,44.08572096)(763.83210444,43.97072108)(763.84210449,43.85072561)
\curveto(763.86210441,43.74072131)(763.88710439,43.62572142)(763.91710449,43.50572561)
\curveto(763.92710435,43.45572159)(763.92710435,43.40072165)(763.91710449,43.34072561)
\curveto(763.90710437,43.29072176)(763.91210436,43.24072181)(763.93210449,43.19072561)
\curveto(763.95210432,43.09072196)(763.95210432,43.00072205)(763.93210449,42.92072561)
\lineto(763.93210449,42.77072561)
\curveto(763.91210436,42.72072233)(763.90210437,42.66072239)(763.90210449,42.59072561)
\curveto(763.90210437,42.53072252)(763.89710438,42.47572257)(763.88710449,42.42572561)
\curveto(763.86710441,42.38572266)(763.85710442,42.3457227)(763.85710449,42.30572561)
\curveto(763.86710441,42.27572277)(763.86210441,42.23572281)(763.84210449,42.18572561)
\lineto(763.78210449,41.94572561)
\curveto(763.76210451,41.87572317)(763.73210454,41.80072325)(763.69210449,41.72072561)
\curveto(763.58210469,41.46072359)(763.43710484,41.24072381)(763.25710449,41.06072561)
\curveto(763.06710521,40.89072416)(762.84210543,40.7507243)(762.58210449,40.64072561)
\curveto(762.49210578,40.60072445)(762.40210587,40.57072448)(762.31210449,40.55072561)
\lineto(762.01210449,40.49072561)
\curveto(761.95210632,40.47072458)(761.89710638,40.46072459)(761.84710449,40.46072561)
\curveto(761.78710649,40.47072458)(761.72210655,40.46572458)(761.65210449,40.44572561)
\curveto(761.63210664,40.43572461)(761.60710667,40.43072462)(761.57710449,40.43072561)
\curveto(761.53710674,40.43072462)(761.50210677,40.42572462)(761.47210449,40.41572561)
\lineto(761.32210449,40.41572561)
\curveto(761.28210699,40.40572464)(761.23710704,40.40072465)(761.18710449,40.40072561)
\curveto(761.12710715,40.41072464)(761.0721072,40.41572463)(761.02210449,40.41572561)
\lineto(760.42210449,40.41572561)
\lineto(757.66210449,40.41572561)
\lineto(756.70210449,40.41572561)
\lineto(756.43210449,40.41572561)
\curveto(756.34211193,40.41572463)(756.26711201,40.43572461)(756.20710449,40.47572561)
\curveto(756.13711214,40.51572453)(756.08711219,40.59072446)(756.05710449,40.70072561)
\curveto(756.04711223,40.72072433)(756.04711223,40.74072431)(756.05710449,40.76072561)
\curveto(756.05711222,40.78072427)(756.05211222,40.80072425)(756.04210449,40.82072561)
}
}
{
\newrgbcolor{curcolor}{0 0 0}
\pscustom[linestyle=none,fillstyle=solid,fillcolor=curcolor]
{
\newpath
\moveto(755.89210449,52.39533498)
\curveto(755.8721124,53.02532975)(755.95711232,53.53032924)(756.14710449,53.91033498)
\curveto(756.33711194,54.29032848)(756.62211165,54.59532818)(757.00210449,54.82533498)
\curveto(757.10211117,54.88532789)(757.21211106,54.93032784)(757.33210449,54.96033498)
\curveto(757.44211083,55.00032777)(757.55711072,55.03532774)(757.67710449,55.06533498)
\curveto(757.86711041,55.11532766)(758.0721102,55.14532763)(758.29210449,55.15533498)
\curveto(758.51210976,55.16532761)(758.73710954,55.1703276)(758.96710449,55.17033498)
\lineto(760.57210449,55.17033498)
\lineto(762.91210449,55.17033498)
\curveto(763.08210519,55.1703276)(763.25210502,55.16532761)(763.42210449,55.15533498)
\curveto(763.59210468,55.15532762)(763.70210457,55.09032768)(763.75210449,54.96033498)
\curveto(763.7721045,54.91032786)(763.78210449,54.85532792)(763.78210449,54.79533498)
\curveto(763.79210448,54.74532803)(763.79710448,54.69032808)(763.79710449,54.63033498)
\curveto(763.79710448,54.50032827)(763.79210448,54.3753284)(763.78210449,54.25533498)
\curveto(763.78210449,54.13532864)(763.74210453,54.05032872)(763.66210449,54.00033498)
\curveto(763.59210468,53.95032882)(763.50210477,53.92532885)(763.39210449,53.92533498)
\lineto(763.06210449,53.92533498)
\lineto(761.77210449,53.92533498)
\lineto(759.32710449,53.92533498)
\curveto(759.05710922,53.92532885)(758.79210948,53.92032885)(758.53210449,53.91033498)
\curveto(758.26211001,53.90032887)(758.03211024,53.85532892)(757.84210449,53.77533498)
\curveto(757.64211063,53.69532908)(757.48211079,53.5753292)(757.36210449,53.41533498)
\curveto(757.23211104,53.25532952)(757.13211114,53.0703297)(757.06210449,52.86033498)
\curveto(757.04211123,52.80032997)(757.03211124,52.73533004)(757.03210449,52.66533498)
\curveto(757.02211125,52.60533017)(757.00711127,52.54533023)(756.98710449,52.48533498)
\curveto(756.9771113,52.43533034)(756.9771113,52.35533042)(756.98710449,52.24533498)
\curveto(756.98711129,52.14533063)(756.99211128,52.0753307)(757.00210449,52.03533498)
\curveto(757.02211125,51.99533078)(757.03211124,51.96033081)(757.03210449,51.93033498)
\curveto(757.02211125,51.90033087)(757.02211125,51.86533091)(757.03210449,51.82533498)
\curveto(757.06211121,51.69533108)(757.09711118,51.5703312)(757.13710449,51.45033498)
\curveto(757.16711111,51.34033143)(757.21211106,51.23533154)(757.27210449,51.13533498)
\curveto(757.29211098,51.09533168)(757.31211096,51.06033171)(757.33210449,51.03033498)
\curveto(757.35211092,51.00033177)(757.3721109,50.96533181)(757.39210449,50.92533498)
\curveto(757.64211063,50.5753322)(758.01711026,50.32033245)(758.51710449,50.16033498)
\curveto(758.59710968,50.13033264)(758.68210959,50.11033266)(758.77210449,50.10033498)
\curveto(758.85210942,50.09033268)(758.93210934,50.0753327)(759.01210449,50.05533498)
\curveto(759.06210921,50.03533274)(759.11210916,50.03033274)(759.16210449,50.04033498)
\curveto(759.20210907,50.05033272)(759.24210903,50.04533273)(759.28210449,50.02533498)
\lineto(759.59710449,50.02533498)
\curveto(759.62710865,50.01533276)(759.66210861,50.01033276)(759.70210449,50.01033498)
\curveto(759.74210853,50.02033275)(759.78710849,50.02533275)(759.83710449,50.02533498)
\lineto(760.28710449,50.02533498)
\lineto(761.72710449,50.02533498)
\lineto(763.04710449,50.02533498)
\lineto(763.39210449,50.02533498)
\curveto(763.50210477,50.02533275)(763.59210468,50.00033277)(763.66210449,49.95033498)
\curveto(763.74210453,49.90033287)(763.78210449,49.81033296)(763.78210449,49.68033498)
\curveto(763.79210448,49.56033321)(763.79710448,49.43533334)(763.79710449,49.30533498)
\curveto(763.79710448,49.22533355)(763.79210448,49.15033362)(763.78210449,49.08033498)
\curveto(763.7721045,49.01033376)(763.74710453,48.95033382)(763.70710449,48.90033498)
\curveto(763.65710462,48.82033395)(763.56210471,48.78033399)(763.42210449,48.78033498)
\lineto(763.01710449,48.78033498)
\lineto(761.24710449,48.78033498)
\lineto(757.61710449,48.78033498)
\lineto(756.70210449,48.78033498)
\lineto(756.43210449,48.78033498)
\curveto(756.34211193,48.78033399)(756.272112,48.80033397)(756.22210449,48.84033498)
\curveto(756.16211211,48.8703339)(756.12211215,48.92033385)(756.10210449,48.99033498)
\curveto(756.09211218,49.03033374)(756.08211219,49.08533369)(756.07210449,49.15533498)
\curveto(756.06211221,49.23533354)(756.05711222,49.31533346)(756.05710449,49.39533498)
\curveto(756.05711222,49.4753333)(756.06211221,49.55033322)(756.07210449,49.62033498)
\curveto(756.08211219,49.70033307)(756.09711218,49.75533302)(756.11710449,49.78533498)
\curveto(756.18711209,49.89533288)(756.277112,49.94533283)(756.38710449,49.93533498)
\curveto(756.48711179,49.92533285)(756.60211167,49.94033283)(756.73210449,49.98033498)
\curveto(756.79211148,50.00033277)(756.84211143,50.04033273)(756.88210449,50.10033498)
\curveto(756.89211138,50.22033255)(756.84711143,50.31533246)(756.74710449,50.38533498)
\curveto(756.64711163,50.46533231)(756.56711171,50.54533223)(756.50710449,50.62533498)
\curveto(756.40711187,50.76533201)(756.31711196,50.90533187)(756.23710449,51.04533498)
\curveto(756.14711213,51.19533158)(756.0721122,51.36533141)(756.01210449,51.55533498)
\curveto(755.98211229,51.63533114)(755.96211231,51.72033105)(755.95210449,51.81033498)
\curveto(755.94211233,51.91033086)(755.92711235,52.00533077)(755.90710449,52.09533498)
\curveto(755.89711238,52.14533063)(755.89211238,52.19533058)(755.89210449,52.24533498)
\lineto(755.89210449,52.39533498)
}
}
{
\newrgbcolor{curcolor}{0 0 0}
\pscustom[linestyle=none,fillstyle=solid,fillcolor=curcolor]
{
}
}
{
\newrgbcolor{curcolor}{0 0 0}
\pscustom[linestyle=none,fillstyle=solid,fillcolor=curcolor]
{
\newpath
\moveto(753.16210449,64.24510061)
\curveto(753.15211512,64.93509597)(753.272115,65.53509537)(753.52210449,66.04510061)
\curveto(753.7721145,66.56509434)(754.10711417,66.96009395)(754.52710449,67.23010061)
\curveto(754.60711367,67.28009363)(754.69711358,67.32509358)(754.79710449,67.36510061)
\curveto(754.88711339,67.4050935)(754.98211329,67.45009346)(755.08210449,67.50010061)
\curveto(755.18211309,67.54009337)(755.28211299,67.57009334)(755.38210449,67.59010061)
\curveto(755.48211279,67.6100933)(755.58711269,67.63009328)(755.69710449,67.65010061)
\curveto(755.74711253,67.67009324)(755.79211248,67.67509323)(755.83210449,67.66510061)
\curveto(755.8721124,67.65509325)(755.91711236,67.66009325)(755.96710449,67.68010061)
\curveto(756.01711226,67.69009322)(756.10211217,67.69509321)(756.22210449,67.69510061)
\curveto(756.33211194,67.69509321)(756.41711186,67.69009322)(756.47710449,67.68010061)
\curveto(756.53711174,67.66009325)(756.59711168,67.65009326)(756.65710449,67.65010061)
\curveto(756.71711156,67.66009325)(756.7771115,67.65509325)(756.83710449,67.63510061)
\curveto(756.9771113,67.59509331)(757.11211116,67.56009335)(757.24210449,67.53010061)
\curveto(757.3721109,67.50009341)(757.49711078,67.46009345)(757.61710449,67.41010061)
\curveto(757.75711052,67.35009356)(757.88211039,67.28009363)(757.99210449,67.20010061)
\curveto(758.10211017,67.13009378)(758.21211006,67.05509385)(758.32210449,66.97510061)
\lineto(758.38210449,66.91510061)
\curveto(758.40210987,66.905094)(758.42210985,66.89009402)(758.44210449,66.87010061)
\curveto(758.60210967,66.75009416)(758.74710953,66.61509429)(758.87710449,66.46510061)
\curveto(759.00710927,66.31509459)(759.13210914,66.15509475)(759.25210449,65.98510061)
\curveto(759.4721088,65.67509523)(759.6771086,65.38009553)(759.86710449,65.10010061)
\curveto(760.00710827,64.87009604)(760.14210813,64.64009627)(760.27210449,64.41010061)
\curveto(760.40210787,64.19009672)(760.53710774,63.97009694)(760.67710449,63.75010061)
\curveto(760.84710743,63.50009741)(761.02710725,63.26009765)(761.21710449,63.03010061)
\curveto(761.40710687,62.8100981)(761.63210664,62.62009829)(761.89210449,62.46010061)
\curveto(761.95210632,62.42009849)(762.01210626,62.38509852)(762.07210449,62.35510061)
\curveto(762.12210615,62.32509858)(762.18710609,62.29509861)(762.26710449,62.26510061)
\curveto(762.33710594,62.24509866)(762.39710588,62.24009867)(762.44710449,62.25010061)
\curveto(762.51710576,62.27009864)(762.5721057,62.3050986)(762.61210449,62.35510061)
\curveto(762.64210563,62.4050985)(762.66210561,62.46509844)(762.67210449,62.53510061)
\lineto(762.67210449,62.77510061)
\lineto(762.67210449,63.52510061)
\lineto(762.67210449,66.33010061)
\lineto(762.67210449,66.99010061)
\curveto(762.6721056,67.08009383)(762.6771056,67.16509374)(762.68710449,67.24510061)
\curveto(762.68710559,67.32509358)(762.70710557,67.39009352)(762.74710449,67.44010061)
\curveto(762.78710549,67.49009342)(762.86210541,67.53009338)(762.97210449,67.56010061)
\curveto(763.0721052,67.60009331)(763.1721051,67.6100933)(763.27210449,67.59010061)
\lineto(763.40710449,67.59010061)
\curveto(763.4771048,67.57009334)(763.53710474,67.55009336)(763.58710449,67.53010061)
\curveto(763.63710464,67.5100934)(763.6771046,67.47509343)(763.70710449,67.42510061)
\curveto(763.74710453,67.37509353)(763.76710451,67.3050936)(763.76710449,67.21510061)
\lineto(763.76710449,66.94510061)
\lineto(763.76710449,66.04510061)
\lineto(763.76710449,62.53510061)
\lineto(763.76710449,61.47010061)
\curveto(763.76710451,61.39009952)(763.7721045,61.30009961)(763.78210449,61.20010061)
\curveto(763.78210449,61.10009981)(763.7721045,61.01509989)(763.75210449,60.94510061)
\curveto(763.68210459,60.73510017)(763.50210477,60.67010024)(763.21210449,60.75010061)
\curveto(763.1721051,60.76010015)(763.13710514,60.76010015)(763.10710449,60.75010061)
\curveto(763.06710521,60.75010016)(763.02210525,60.76010015)(762.97210449,60.78010061)
\curveto(762.89210538,60.80010011)(762.80710547,60.82010009)(762.71710449,60.84010061)
\curveto(762.62710565,60.86010005)(762.54210573,60.88510002)(762.46210449,60.91510061)
\curveto(761.9721063,61.07509983)(761.55710672,61.27509963)(761.21710449,61.51510061)
\curveto(760.96710731,61.69509921)(760.74210753,61.90009901)(760.54210449,62.13010061)
\curveto(760.33210794,62.36009855)(760.13710814,62.60009831)(759.95710449,62.85010061)
\curveto(759.7771085,63.1100978)(759.60710867,63.37509753)(759.44710449,63.64510061)
\curveto(759.277109,63.92509698)(759.10210917,64.19509671)(758.92210449,64.45510061)
\curveto(758.84210943,64.56509634)(758.76710951,64.67009624)(758.69710449,64.77010061)
\curveto(758.62710965,64.88009603)(758.55210972,64.99009592)(758.47210449,65.10010061)
\curveto(758.44210983,65.14009577)(758.41210986,65.17509573)(758.38210449,65.20510061)
\curveto(758.34210993,65.24509566)(758.31210996,65.28509562)(758.29210449,65.32510061)
\curveto(758.18211009,65.46509544)(758.05711022,65.59009532)(757.91710449,65.70010061)
\curveto(757.88711039,65.72009519)(757.86211041,65.74509516)(757.84210449,65.77510061)
\curveto(757.81211046,65.8050951)(757.78211049,65.83009508)(757.75210449,65.85010061)
\curveto(757.65211062,65.93009498)(757.55211072,65.99509491)(757.45210449,66.04510061)
\curveto(757.35211092,66.1050948)(757.24211103,66.16009475)(757.12210449,66.21010061)
\curveto(757.05211122,66.24009467)(756.9771113,66.26009465)(756.89710449,66.27010061)
\lineto(756.65710449,66.33010061)
\lineto(756.56710449,66.33010061)
\curveto(756.53711174,66.34009457)(756.50711177,66.34509456)(756.47710449,66.34510061)
\curveto(756.40711187,66.36509454)(756.31211196,66.37009454)(756.19210449,66.36010061)
\curveto(756.06211221,66.36009455)(755.96211231,66.35009456)(755.89210449,66.33010061)
\curveto(755.81211246,66.3100946)(755.73711254,66.29009462)(755.66710449,66.27010061)
\curveto(755.58711269,66.26009465)(755.50711277,66.24009467)(755.42710449,66.21010061)
\curveto(755.18711309,66.10009481)(754.98711329,65.95009496)(754.82710449,65.76010061)
\curveto(754.65711362,65.58009533)(754.51711376,65.36009555)(754.40710449,65.10010061)
\curveto(754.38711389,65.03009588)(754.3721139,64.96009595)(754.36210449,64.89010061)
\curveto(754.34211393,64.82009609)(754.32211395,64.74509616)(754.30210449,64.66510061)
\curveto(754.28211399,64.58509632)(754.272114,64.47509643)(754.27210449,64.33510061)
\curveto(754.272114,64.2050967)(754.28211399,64.10009681)(754.30210449,64.02010061)
\curveto(754.31211396,63.96009695)(754.31711396,63.905097)(754.31710449,63.85510061)
\curveto(754.31711396,63.8050971)(754.32711395,63.75509715)(754.34710449,63.70510061)
\curveto(754.38711389,63.6050973)(754.42711385,63.5100974)(754.46710449,63.42010061)
\curveto(754.50711377,63.34009757)(754.55211372,63.26009765)(754.60210449,63.18010061)
\curveto(754.62211365,63.15009776)(754.64711363,63.12009779)(754.67710449,63.09010061)
\curveto(754.70711357,63.07009784)(754.73211354,63.04509786)(754.75210449,63.01510061)
\lineto(754.82710449,62.94010061)
\curveto(754.84711343,62.910098)(754.86711341,62.88509802)(754.88710449,62.86510061)
\lineto(755.09710449,62.71510061)
\curveto(755.15711312,62.67509823)(755.22211305,62.63009828)(755.29210449,62.58010061)
\curveto(755.38211289,62.52009839)(755.48711279,62.47009844)(755.60710449,62.43010061)
\curveto(755.71711256,62.40009851)(755.82711245,62.36509854)(755.93710449,62.32510061)
\curveto(756.04711223,62.28509862)(756.19211208,62.26009865)(756.37210449,62.25010061)
\curveto(756.54211173,62.24009867)(756.66711161,62.2100987)(756.74710449,62.16010061)
\curveto(756.82711145,62.1100988)(756.8721114,62.03509887)(756.88210449,61.93510061)
\curveto(756.89211138,61.83509907)(756.89711138,61.72509918)(756.89710449,61.60510061)
\curveto(756.89711138,61.56509934)(756.90211137,61.52509938)(756.91210449,61.48510061)
\curveto(756.91211136,61.44509946)(756.90711137,61.4100995)(756.89710449,61.38010061)
\curveto(756.8771114,61.33009958)(756.86711141,61.28009963)(756.86710449,61.23010061)
\curveto(756.86711141,61.19009972)(756.85711142,61.15009976)(756.83710449,61.11010061)
\curveto(756.7771115,61.02009989)(756.64211163,60.97509993)(756.43210449,60.97510061)
\lineto(756.31210449,60.97510061)
\curveto(756.25211202,60.98509992)(756.19211208,60.99009992)(756.13210449,60.99010061)
\curveto(756.06211221,61.00009991)(755.99711228,61.0100999)(755.93710449,61.02010061)
\curveto(755.82711245,61.04009987)(755.72711255,61.06009985)(755.63710449,61.08010061)
\curveto(755.53711274,61.10009981)(755.44211283,61.13009978)(755.35210449,61.17010061)
\curveto(755.28211299,61.19009972)(755.22211305,61.2100997)(755.17210449,61.23010061)
\lineto(754.99210449,61.29010061)
\curveto(754.73211354,61.4100995)(754.48711379,61.56509934)(754.25710449,61.75510061)
\curveto(754.02711425,61.95509895)(753.84211443,62.17009874)(753.70210449,62.40010061)
\curveto(753.62211465,62.5100984)(753.55711472,62.62509828)(753.50710449,62.74510061)
\lineto(753.35710449,63.13510061)
\curveto(753.30711497,63.24509766)(753.277115,63.36009755)(753.26710449,63.48010061)
\curveto(753.24711503,63.60009731)(753.22211505,63.72509718)(753.19210449,63.85510061)
\curveto(753.19211508,63.92509698)(753.19211508,63.99009692)(753.19210449,64.05010061)
\curveto(753.18211509,64.1100968)(753.1721151,64.17509673)(753.16210449,64.24510061)
}
}
{
\newrgbcolor{curcolor}{0 0 0}
\pscustom[linestyle=none,fillstyle=solid,fillcolor=curcolor]
{
\newpath
\moveto(753.16210449,72.59470998)
\curveto(753.15211512,73.28470535)(753.272115,73.88470475)(753.52210449,74.39470998)
\curveto(753.7721145,74.91470372)(754.10711417,75.30970332)(754.52710449,75.57970998)
\curveto(754.60711367,75.629703)(754.69711358,75.67470296)(754.79710449,75.71470998)
\curveto(754.88711339,75.75470288)(754.98211329,75.79970283)(755.08210449,75.84970998)
\curveto(755.18211309,75.88970274)(755.28211299,75.91970271)(755.38210449,75.93970998)
\curveto(755.48211279,75.95970267)(755.58711269,75.97970265)(755.69710449,75.99970998)
\curveto(755.74711253,76.01970261)(755.79211248,76.02470261)(755.83210449,76.01470998)
\curveto(755.8721124,76.00470263)(755.91711236,76.00970262)(755.96710449,76.02970998)
\curveto(756.01711226,76.03970259)(756.10211217,76.04470259)(756.22210449,76.04470998)
\curveto(756.33211194,76.04470259)(756.41711186,76.03970259)(756.47710449,76.02970998)
\curveto(756.53711174,76.00970262)(756.59711168,75.99970263)(756.65710449,75.99970998)
\curveto(756.71711156,76.00970262)(756.7771115,76.00470263)(756.83710449,75.98470998)
\curveto(756.9771113,75.94470269)(757.11211116,75.90970272)(757.24210449,75.87970998)
\curveto(757.3721109,75.84970278)(757.49711078,75.80970282)(757.61710449,75.75970998)
\curveto(757.75711052,75.69970293)(757.88211039,75.629703)(757.99210449,75.54970998)
\curveto(758.10211017,75.47970315)(758.21211006,75.40470323)(758.32210449,75.32470998)
\lineto(758.38210449,75.26470998)
\curveto(758.40210987,75.25470338)(758.42210985,75.23970339)(758.44210449,75.21970998)
\curveto(758.60210967,75.09970353)(758.74710953,74.96470367)(758.87710449,74.81470998)
\curveto(759.00710927,74.66470397)(759.13210914,74.50470413)(759.25210449,74.33470998)
\curveto(759.4721088,74.02470461)(759.6771086,73.7297049)(759.86710449,73.44970998)
\curveto(760.00710827,73.21970541)(760.14210813,72.98970564)(760.27210449,72.75970998)
\curveto(760.40210787,72.53970609)(760.53710774,72.31970631)(760.67710449,72.09970998)
\curveto(760.84710743,71.84970678)(761.02710725,71.60970702)(761.21710449,71.37970998)
\curveto(761.40710687,71.15970747)(761.63210664,70.96970766)(761.89210449,70.80970998)
\curveto(761.95210632,70.76970786)(762.01210626,70.7347079)(762.07210449,70.70470998)
\curveto(762.12210615,70.67470796)(762.18710609,70.64470799)(762.26710449,70.61470998)
\curveto(762.33710594,70.59470804)(762.39710588,70.58970804)(762.44710449,70.59970998)
\curveto(762.51710576,70.61970801)(762.5721057,70.65470798)(762.61210449,70.70470998)
\curveto(762.64210563,70.75470788)(762.66210561,70.81470782)(762.67210449,70.88470998)
\lineto(762.67210449,71.12470998)
\lineto(762.67210449,71.87470998)
\lineto(762.67210449,74.67970998)
\lineto(762.67210449,75.33970998)
\curveto(762.6721056,75.4297032)(762.6771056,75.51470312)(762.68710449,75.59470998)
\curveto(762.68710559,75.67470296)(762.70710557,75.73970289)(762.74710449,75.78970998)
\curveto(762.78710549,75.83970279)(762.86210541,75.87970275)(762.97210449,75.90970998)
\curveto(763.0721052,75.94970268)(763.1721051,75.95970267)(763.27210449,75.93970998)
\lineto(763.40710449,75.93970998)
\curveto(763.4771048,75.91970271)(763.53710474,75.89970273)(763.58710449,75.87970998)
\curveto(763.63710464,75.85970277)(763.6771046,75.82470281)(763.70710449,75.77470998)
\curveto(763.74710453,75.72470291)(763.76710451,75.65470298)(763.76710449,75.56470998)
\lineto(763.76710449,75.29470998)
\lineto(763.76710449,74.39470998)
\lineto(763.76710449,70.88470998)
\lineto(763.76710449,69.81970998)
\curveto(763.76710451,69.73970889)(763.7721045,69.64970898)(763.78210449,69.54970998)
\curveto(763.78210449,69.44970918)(763.7721045,69.36470927)(763.75210449,69.29470998)
\curveto(763.68210459,69.08470955)(763.50210477,69.01970961)(763.21210449,69.09970998)
\curveto(763.1721051,69.10970952)(763.13710514,69.10970952)(763.10710449,69.09970998)
\curveto(763.06710521,69.09970953)(763.02210525,69.10970952)(762.97210449,69.12970998)
\curveto(762.89210538,69.14970948)(762.80710547,69.16970946)(762.71710449,69.18970998)
\curveto(762.62710565,69.20970942)(762.54210573,69.2347094)(762.46210449,69.26470998)
\curveto(761.9721063,69.42470921)(761.55710672,69.62470901)(761.21710449,69.86470998)
\curveto(760.96710731,70.04470859)(760.74210753,70.24970838)(760.54210449,70.47970998)
\curveto(760.33210794,70.70970792)(760.13710814,70.94970768)(759.95710449,71.19970998)
\curveto(759.7771085,71.45970717)(759.60710867,71.72470691)(759.44710449,71.99470998)
\curveto(759.277109,72.27470636)(759.10210917,72.54470609)(758.92210449,72.80470998)
\curveto(758.84210943,72.91470572)(758.76710951,73.01970561)(758.69710449,73.11970998)
\curveto(758.62710965,73.2297054)(758.55210972,73.33970529)(758.47210449,73.44970998)
\curveto(758.44210983,73.48970514)(758.41210986,73.52470511)(758.38210449,73.55470998)
\curveto(758.34210993,73.59470504)(758.31210996,73.634705)(758.29210449,73.67470998)
\curveto(758.18211009,73.81470482)(758.05711022,73.93970469)(757.91710449,74.04970998)
\curveto(757.88711039,74.06970456)(757.86211041,74.09470454)(757.84210449,74.12470998)
\curveto(757.81211046,74.15470448)(757.78211049,74.17970445)(757.75210449,74.19970998)
\curveto(757.65211062,74.27970435)(757.55211072,74.34470429)(757.45210449,74.39470998)
\curveto(757.35211092,74.45470418)(757.24211103,74.50970412)(757.12210449,74.55970998)
\curveto(757.05211122,74.58970404)(756.9771113,74.60970402)(756.89710449,74.61970998)
\lineto(756.65710449,74.67970998)
\lineto(756.56710449,74.67970998)
\curveto(756.53711174,74.68970394)(756.50711177,74.69470394)(756.47710449,74.69470998)
\curveto(756.40711187,74.71470392)(756.31211196,74.71970391)(756.19210449,74.70970998)
\curveto(756.06211221,74.70970392)(755.96211231,74.69970393)(755.89210449,74.67970998)
\curveto(755.81211246,74.65970397)(755.73711254,74.63970399)(755.66710449,74.61970998)
\curveto(755.58711269,74.60970402)(755.50711277,74.58970404)(755.42710449,74.55970998)
\curveto(755.18711309,74.44970418)(754.98711329,74.29970433)(754.82710449,74.10970998)
\curveto(754.65711362,73.9297047)(754.51711376,73.70970492)(754.40710449,73.44970998)
\curveto(754.38711389,73.37970525)(754.3721139,73.30970532)(754.36210449,73.23970998)
\curveto(754.34211393,73.16970546)(754.32211395,73.09470554)(754.30210449,73.01470998)
\curveto(754.28211399,72.9347057)(754.272114,72.82470581)(754.27210449,72.68470998)
\curveto(754.272114,72.55470608)(754.28211399,72.44970618)(754.30210449,72.36970998)
\curveto(754.31211396,72.30970632)(754.31711396,72.25470638)(754.31710449,72.20470998)
\curveto(754.31711396,72.15470648)(754.32711395,72.10470653)(754.34710449,72.05470998)
\curveto(754.38711389,71.95470668)(754.42711385,71.85970677)(754.46710449,71.76970998)
\curveto(754.50711377,71.68970694)(754.55211372,71.60970702)(754.60210449,71.52970998)
\curveto(754.62211365,71.49970713)(754.64711363,71.46970716)(754.67710449,71.43970998)
\curveto(754.70711357,71.41970721)(754.73211354,71.39470724)(754.75210449,71.36470998)
\lineto(754.82710449,71.28970998)
\curveto(754.84711343,71.25970737)(754.86711341,71.2347074)(754.88710449,71.21470998)
\lineto(755.09710449,71.06470998)
\curveto(755.15711312,71.02470761)(755.22211305,70.97970765)(755.29210449,70.92970998)
\curveto(755.38211289,70.86970776)(755.48711279,70.81970781)(755.60710449,70.77970998)
\curveto(755.71711256,70.74970788)(755.82711245,70.71470792)(755.93710449,70.67470998)
\curveto(756.04711223,70.634708)(756.19211208,70.60970802)(756.37210449,70.59970998)
\curveto(756.54211173,70.58970804)(756.66711161,70.55970807)(756.74710449,70.50970998)
\curveto(756.82711145,70.45970817)(756.8721114,70.38470825)(756.88210449,70.28470998)
\curveto(756.89211138,70.18470845)(756.89711138,70.07470856)(756.89710449,69.95470998)
\curveto(756.89711138,69.91470872)(756.90211137,69.87470876)(756.91210449,69.83470998)
\curveto(756.91211136,69.79470884)(756.90711137,69.75970887)(756.89710449,69.72970998)
\curveto(756.8771114,69.67970895)(756.86711141,69.629709)(756.86710449,69.57970998)
\curveto(756.86711141,69.53970909)(756.85711142,69.49970913)(756.83710449,69.45970998)
\curveto(756.7771115,69.36970926)(756.64211163,69.32470931)(756.43210449,69.32470998)
\lineto(756.31210449,69.32470998)
\curveto(756.25211202,69.3347093)(756.19211208,69.33970929)(756.13210449,69.33970998)
\curveto(756.06211221,69.34970928)(755.99711228,69.35970927)(755.93710449,69.36970998)
\curveto(755.82711245,69.38970924)(755.72711255,69.40970922)(755.63710449,69.42970998)
\curveto(755.53711274,69.44970918)(755.44211283,69.47970915)(755.35210449,69.51970998)
\curveto(755.28211299,69.53970909)(755.22211305,69.55970907)(755.17210449,69.57970998)
\lineto(754.99210449,69.63970998)
\curveto(754.73211354,69.75970887)(754.48711379,69.91470872)(754.25710449,70.10470998)
\curveto(754.02711425,70.30470833)(753.84211443,70.51970811)(753.70210449,70.74970998)
\curveto(753.62211465,70.85970777)(753.55711472,70.97470766)(753.50710449,71.09470998)
\lineto(753.35710449,71.48470998)
\curveto(753.30711497,71.59470704)(753.277115,71.70970692)(753.26710449,71.82970998)
\curveto(753.24711503,71.94970668)(753.22211505,72.07470656)(753.19210449,72.20470998)
\curveto(753.19211508,72.27470636)(753.19211508,72.33970629)(753.19210449,72.39970998)
\curveto(753.18211509,72.45970617)(753.1721151,72.52470611)(753.16210449,72.59470998)
}
}
{
\newrgbcolor{curcolor}{0 0 0}
\pscustom[linestyle=none,fillstyle=solid,fillcolor=curcolor]
{
\newpath
\moveto(762.13210449,78.63431936)
\lineto(762.13210449,79.26431936)
\lineto(762.13210449,79.45931936)
\curveto(762.13210614,79.52931683)(762.14210613,79.58931677)(762.16210449,79.63931936)
\curveto(762.20210607,79.70931665)(762.24210603,79.7593166)(762.28210449,79.78931936)
\curveto(762.33210594,79.82931653)(762.39710588,79.84931651)(762.47710449,79.84931936)
\curveto(762.55710572,79.8593165)(762.64210563,79.86431649)(762.73210449,79.86431936)
\lineto(763.45210449,79.86431936)
\curveto(763.93210434,79.86431649)(764.34210393,79.80431655)(764.68210449,79.68431936)
\curveto(765.02210325,79.56431679)(765.29710298,79.36931699)(765.50710449,79.09931936)
\curveto(765.55710272,79.02931733)(765.60210267,78.9593174)(765.64210449,78.88931936)
\curveto(765.69210258,78.82931753)(765.73710254,78.7543176)(765.77710449,78.66431936)
\curveto(765.78710249,78.64431771)(765.79710248,78.61431774)(765.80710449,78.57431936)
\curveto(765.82710245,78.53431782)(765.83210244,78.48931787)(765.82210449,78.43931936)
\curveto(765.79210248,78.34931801)(765.71710256,78.29431806)(765.59710449,78.27431936)
\curveto(765.48710279,78.2543181)(765.39210288,78.26931809)(765.31210449,78.31931936)
\curveto(765.24210303,78.34931801)(765.1771031,78.39431796)(765.11710449,78.45431936)
\curveto(765.06710321,78.52431783)(765.01710326,78.58931777)(764.96710449,78.64931936)
\curveto(764.91710336,78.71931764)(764.84210343,78.77931758)(764.74210449,78.82931936)
\curveto(764.65210362,78.88931747)(764.56210371,78.93931742)(764.47210449,78.97931936)
\curveto(764.44210383,78.99931736)(764.38210389,79.02431733)(764.29210449,79.05431936)
\curveto(764.21210406,79.08431727)(764.14210413,79.08931727)(764.08210449,79.06931936)
\curveto(763.94210433,79.03931732)(763.85210442,78.97931738)(763.81210449,78.88931936)
\curveto(763.78210449,78.80931755)(763.76710451,78.71931764)(763.76710449,78.61931936)
\curveto(763.76710451,78.51931784)(763.74210453,78.43431792)(763.69210449,78.36431936)
\curveto(763.62210465,78.27431808)(763.48210479,78.22931813)(763.27210449,78.22931936)
\lineto(762.71710449,78.22931936)
\lineto(762.49210449,78.22931936)
\curveto(762.41210586,78.23931812)(762.34710593,78.2593181)(762.29710449,78.28931936)
\curveto(762.21710606,78.34931801)(762.1721061,78.41931794)(762.16210449,78.49931936)
\curveto(762.15210612,78.51931784)(762.14710613,78.53931782)(762.14710449,78.55931936)
\curveto(762.14710613,78.58931777)(762.14210613,78.61431774)(762.13210449,78.63431936)
}
}
{
\newrgbcolor{curcolor}{0 0 0}
\pscustom[linestyle=none,fillstyle=solid,fillcolor=curcolor]
{
}
}
{
\newrgbcolor{curcolor}{0 0 0}
\pscustom[linestyle=none,fillstyle=solid,fillcolor=curcolor]
{
\newpath
\moveto(753.16210449,89.26463186)
\curveto(753.15211512,89.95462722)(753.272115,90.55462662)(753.52210449,91.06463186)
\curveto(753.7721145,91.58462559)(754.10711417,91.9796252)(754.52710449,92.24963186)
\curveto(754.60711367,92.29962488)(754.69711358,92.34462483)(754.79710449,92.38463186)
\curveto(754.88711339,92.42462475)(754.98211329,92.46962471)(755.08210449,92.51963186)
\curveto(755.18211309,92.55962462)(755.28211299,92.58962459)(755.38210449,92.60963186)
\curveto(755.48211279,92.62962455)(755.58711269,92.64962453)(755.69710449,92.66963186)
\curveto(755.74711253,92.68962449)(755.79211248,92.69462448)(755.83210449,92.68463186)
\curveto(755.8721124,92.6746245)(755.91711236,92.6796245)(755.96710449,92.69963186)
\curveto(756.01711226,92.70962447)(756.10211217,92.71462446)(756.22210449,92.71463186)
\curveto(756.33211194,92.71462446)(756.41711186,92.70962447)(756.47710449,92.69963186)
\curveto(756.53711174,92.6796245)(756.59711168,92.66962451)(756.65710449,92.66963186)
\curveto(756.71711156,92.6796245)(756.7771115,92.6746245)(756.83710449,92.65463186)
\curveto(756.9771113,92.61462456)(757.11211116,92.5796246)(757.24210449,92.54963186)
\curveto(757.3721109,92.51962466)(757.49711078,92.4796247)(757.61710449,92.42963186)
\curveto(757.75711052,92.36962481)(757.88211039,92.29962488)(757.99210449,92.21963186)
\curveto(758.10211017,92.14962503)(758.21211006,92.0746251)(758.32210449,91.99463186)
\lineto(758.38210449,91.93463186)
\curveto(758.40210987,91.92462525)(758.42210985,91.90962527)(758.44210449,91.88963186)
\curveto(758.60210967,91.76962541)(758.74710953,91.63462554)(758.87710449,91.48463186)
\curveto(759.00710927,91.33462584)(759.13210914,91.174626)(759.25210449,91.00463186)
\curveto(759.4721088,90.69462648)(759.6771086,90.39962678)(759.86710449,90.11963186)
\curveto(760.00710827,89.88962729)(760.14210813,89.65962752)(760.27210449,89.42963186)
\curveto(760.40210787,89.20962797)(760.53710774,88.98962819)(760.67710449,88.76963186)
\curveto(760.84710743,88.51962866)(761.02710725,88.2796289)(761.21710449,88.04963186)
\curveto(761.40710687,87.82962935)(761.63210664,87.63962954)(761.89210449,87.47963186)
\curveto(761.95210632,87.43962974)(762.01210626,87.40462977)(762.07210449,87.37463186)
\curveto(762.12210615,87.34462983)(762.18710609,87.31462986)(762.26710449,87.28463186)
\curveto(762.33710594,87.26462991)(762.39710588,87.25962992)(762.44710449,87.26963186)
\curveto(762.51710576,87.28962989)(762.5721057,87.32462985)(762.61210449,87.37463186)
\curveto(762.64210563,87.42462975)(762.66210561,87.48462969)(762.67210449,87.55463186)
\lineto(762.67210449,87.79463186)
\lineto(762.67210449,88.54463186)
\lineto(762.67210449,91.34963186)
\lineto(762.67210449,92.00963186)
\curveto(762.6721056,92.09962508)(762.6771056,92.18462499)(762.68710449,92.26463186)
\curveto(762.68710559,92.34462483)(762.70710557,92.40962477)(762.74710449,92.45963186)
\curveto(762.78710549,92.50962467)(762.86210541,92.54962463)(762.97210449,92.57963186)
\curveto(763.0721052,92.61962456)(763.1721051,92.62962455)(763.27210449,92.60963186)
\lineto(763.40710449,92.60963186)
\curveto(763.4771048,92.58962459)(763.53710474,92.56962461)(763.58710449,92.54963186)
\curveto(763.63710464,92.52962465)(763.6771046,92.49462468)(763.70710449,92.44463186)
\curveto(763.74710453,92.39462478)(763.76710451,92.32462485)(763.76710449,92.23463186)
\lineto(763.76710449,91.96463186)
\lineto(763.76710449,91.06463186)
\lineto(763.76710449,87.55463186)
\lineto(763.76710449,86.48963186)
\curveto(763.76710451,86.40963077)(763.7721045,86.31963086)(763.78210449,86.21963186)
\curveto(763.78210449,86.11963106)(763.7721045,86.03463114)(763.75210449,85.96463186)
\curveto(763.68210459,85.75463142)(763.50210477,85.68963149)(763.21210449,85.76963186)
\curveto(763.1721051,85.7796314)(763.13710514,85.7796314)(763.10710449,85.76963186)
\curveto(763.06710521,85.76963141)(763.02210525,85.7796314)(762.97210449,85.79963186)
\curveto(762.89210538,85.81963136)(762.80710547,85.83963134)(762.71710449,85.85963186)
\curveto(762.62710565,85.8796313)(762.54210573,85.90463127)(762.46210449,85.93463186)
\curveto(761.9721063,86.09463108)(761.55710672,86.29463088)(761.21710449,86.53463186)
\curveto(760.96710731,86.71463046)(760.74210753,86.91963026)(760.54210449,87.14963186)
\curveto(760.33210794,87.3796298)(760.13710814,87.61962956)(759.95710449,87.86963186)
\curveto(759.7771085,88.12962905)(759.60710867,88.39462878)(759.44710449,88.66463186)
\curveto(759.277109,88.94462823)(759.10210917,89.21462796)(758.92210449,89.47463186)
\curveto(758.84210943,89.58462759)(758.76710951,89.68962749)(758.69710449,89.78963186)
\curveto(758.62710965,89.89962728)(758.55210972,90.00962717)(758.47210449,90.11963186)
\curveto(758.44210983,90.15962702)(758.41210986,90.19462698)(758.38210449,90.22463186)
\curveto(758.34210993,90.26462691)(758.31210996,90.30462687)(758.29210449,90.34463186)
\curveto(758.18211009,90.48462669)(758.05711022,90.60962657)(757.91710449,90.71963186)
\curveto(757.88711039,90.73962644)(757.86211041,90.76462641)(757.84210449,90.79463186)
\curveto(757.81211046,90.82462635)(757.78211049,90.84962633)(757.75210449,90.86963186)
\curveto(757.65211062,90.94962623)(757.55211072,91.01462616)(757.45210449,91.06463186)
\curveto(757.35211092,91.12462605)(757.24211103,91.179626)(757.12210449,91.22963186)
\curveto(757.05211122,91.25962592)(756.9771113,91.2796259)(756.89710449,91.28963186)
\lineto(756.65710449,91.34963186)
\lineto(756.56710449,91.34963186)
\curveto(756.53711174,91.35962582)(756.50711177,91.36462581)(756.47710449,91.36463186)
\curveto(756.40711187,91.38462579)(756.31211196,91.38962579)(756.19210449,91.37963186)
\curveto(756.06211221,91.3796258)(755.96211231,91.36962581)(755.89210449,91.34963186)
\curveto(755.81211246,91.32962585)(755.73711254,91.30962587)(755.66710449,91.28963186)
\curveto(755.58711269,91.2796259)(755.50711277,91.25962592)(755.42710449,91.22963186)
\curveto(755.18711309,91.11962606)(754.98711329,90.96962621)(754.82710449,90.77963186)
\curveto(754.65711362,90.59962658)(754.51711376,90.3796268)(754.40710449,90.11963186)
\curveto(754.38711389,90.04962713)(754.3721139,89.9796272)(754.36210449,89.90963186)
\curveto(754.34211393,89.83962734)(754.32211395,89.76462741)(754.30210449,89.68463186)
\curveto(754.28211399,89.60462757)(754.272114,89.49462768)(754.27210449,89.35463186)
\curveto(754.272114,89.22462795)(754.28211399,89.11962806)(754.30210449,89.03963186)
\curveto(754.31211396,88.9796282)(754.31711396,88.92462825)(754.31710449,88.87463186)
\curveto(754.31711396,88.82462835)(754.32711395,88.7746284)(754.34710449,88.72463186)
\curveto(754.38711389,88.62462855)(754.42711385,88.52962865)(754.46710449,88.43963186)
\curveto(754.50711377,88.35962882)(754.55211372,88.2796289)(754.60210449,88.19963186)
\curveto(754.62211365,88.16962901)(754.64711363,88.13962904)(754.67710449,88.10963186)
\curveto(754.70711357,88.08962909)(754.73211354,88.06462911)(754.75210449,88.03463186)
\lineto(754.82710449,87.95963186)
\curveto(754.84711343,87.92962925)(754.86711341,87.90462927)(754.88710449,87.88463186)
\lineto(755.09710449,87.73463186)
\curveto(755.15711312,87.69462948)(755.22211305,87.64962953)(755.29210449,87.59963186)
\curveto(755.38211289,87.53962964)(755.48711279,87.48962969)(755.60710449,87.44963186)
\curveto(755.71711256,87.41962976)(755.82711245,87.38462979)(755.93710449,87.34463186)
\curveto(756.04711223,87.30462987)(756.19211208,87.2796299)(756.37210449,87.26963186)
\curveto(756.54211173,87.25962992)(756.66711161,87.22962995)(756.74710449,87.17963186)
\curveto(756.82711145,87.12963005)(756.8721114,87.05463012)(756.88210449,86.95463186)
\curveto(756.89211138,86.85463032)(756.89711138,86.74463043)(756.89710449,86.62463186)
\curveto(756.89711138,86.58463059)(756.90211137,86.54463063)(756.91210449,86.50463186)
\curveto(756.91211136,86.46463071)(756.90711137,86.42963075)(756.89710449,86.39963186)
\curveto(756.8771114,86.34963083)(756.86711141,86.29963088)(756.86710449,86.24963186)
\curveto(756.86711141,86.20963097)(756.85711142,86.16963101)(756.83710449,86.12963186)
\curveto(756.7771115,86.03963114)(756.64211163,85.99463118)(756.43210449,85.99463186)
\lineto(756.31210449,85.99463186)
\curveto(756.25211202,86.00463117)(756.19211208,86.00963117)(756.13210449,86.00963186)
\curveto(756.06211221,86.01963116)(755.99711228,86.02963115)(755.93710449,86.03963186)
\curveto(755.82711245,86.05963112)(755.72711255,86.0796311)(755.63710449,86.09963186)
\curveto(755.53711274,86.11963106)(755.44211283,86.14963103)(755.35210449,86.18963186)
\curveto(755.28211299,86.20963097)(755.22211305,86.22963095)(755.17210449,86.24963186)
\lineto(754.99210449,86.30963186)
\curveto(754.73211354,86.42963075)(754.48711379,86.58463059)(754.25710449,86.77463186)
\curveto(754.02711425,86.9746302)(753.84211443,87.18962999)(753.70210449,87.41963186)
\curveto(753.62211465,87.52962965)(753.55711472,87.64462953)(753.50710449,87.76463186)
\lineto(753.35710449,88.15463186)
\curveto(753.30711497,88.26462891)(753.277115,88.3796288)(753.26710449,88.49963186)
\curveto(753.24711503,88.61962856)(753.22211505,88.74462843)(753.19210449,88.87463186)
\curveto(753.19211508,88.94462823)(753.19211508,89.00962817)(753.19210449,89.06963186)
\curveto(753.18211509,89.12962805)(753.1721151,89.19462798)(753.16210449,89.26463186)
}
}
{
\newrgbcolor{curcolor}{0 0 0}
\pscustom[linestyle=none,fillstyle=solid,fillcolor=curcolor]
{
\newpath
\moveto(758.68210449,101.36424123)
\lineto(758.93710449,101.36424123)
\curveto(759.01710926,101.37423353)(759.09210918,101.36923353)(759.16210449,101.34924123)
\lineto(759.40210449,101.34924123)
\lineto(759.56710449,101.34924123)
\curveto(759.66710861,101.32923357)(759.7721085,101.31923358)(759.88210449,101.31924123)
\curveto(759.98210829,101.31923358)(760.08210819,101.30923359)(760.18210449,101.28924123)
\lineto(760.33210449,101.28924123)
\curveto(760.4721078,101.25923364)(760.61210766,101.23923366)(760.75210449,101.22924123)
\curveto(760.88210739,101.21923368)(761.01210726,101.19423371)(761.14210449,101.15424123)
\curveto(761.22210705,101.13423377)(761.30710697,101.11423379)(761.39710449,101.09424123)
\lineto(761.63710449,101.03424123)
\lineto(761.93710449,100.91424123)
\curveto(762.02710625,100.88423402)(762.11710616,100.84923405)(762.20710449,100.80924123)
\curveto(762.42710585,100.70923419)(762.64210563,100.57423433)(762.85210449,100.40424123)
\curveto(763.06210521,100.24423466)(763.23210504,100.06923483)(763.36210449,99.87924123)
\curveto(763.40210487,99.82923507)(763.44210483,99.76923513)(763.48210449,99.69924123)
\curveto(763.51210476,99.63923526)(763.54710473,99.57923532)(763.58710449,99.51924123)
\curveto(763.63710464,99.43923546)(763.6771046,99.34423556)(763.70710449,99.23424123)
\curveto(763.73710454,99.12423578)(763.76710451,99.01923588)(763.79710449,98.91924123)
\curveto(763.83710444,98.80923609)(763.86210441,98.6992362)(763.87210449,98.58924123)
\curveto(763.88210439,98.47923642)(763.89710438,98.36423654)(763.91710449,98.24424123)
\curveto(763.92710435,98.2042367)(763.92710435,98.15923674)(763.91710449,98.10924123)
\curveto(763.91710436,98.06923683)(763.92210435,98.02923687)(763.93210449,97.98924123)
\curveto(763.94210433,97.94923695)(763.94710433,97.89423701)(763.94710449,97.82424123)
\curveto(763.94710433,97.75423715)(763.94210433,97.7042372)(763.93210449,97.67424123)
\curveto(763.91210436,97.62423728)(763.90710437,97.57923732)(763.91710449,97.53924123)
\curveto(763.92710435,97.4992374)(763.92710435,97.46423744)(763.91710449,97.43424123)
\lineto(763.91710449,97.34424123)
\curveto(763.89710438,97.28423762)(763.88210439,97.21923768)(763.87210449,97.14924123)
\curveto(763.8721044,97.08923781)(763.86710441,97.02423788)(763.85710449,96.95424123)
\curveto(763.80710447,96.78423812)(763.75710452,96.62423828)(763.70710449,96.47424123)
\curveto(763.65710462,96.32423858)(763.59210468,96.17923872)(763.51210449,96.03924123)
\curveto(763.4721048,95.98923891)(763.44210483,95.93423897)(763.42210449,95.87424123)
\curveto(763.39210488,95.82423908)(763.35710492,95.77423913)(763.31710449,95.72424123)
\curveto(763.13710514,95.48423942)(762.91710536,95.28423962)(762.65710449,95.12424123)
\curveto(762.39710588,94.96423994)(762.11210616,94.82424008)(761.80210449,94.70424123)
\curveto(761.66210661,94.64424026)(761.52210675,94.5992403)(761.38210449,94.56924123)
\curveto(761.23210704,94.53924036)(761.0771072,94.5042404)(760.91710449,94.46424123)
\curveto(760.80710747,94.44424046)(760.69710758,94.42924047)(760.58710449,94.41924123)
\curveto(760.4771078,94.40924049)(760.36710791,94.39424051)(760.25710449,94.37424123)
\curveto(760.21710806,94.36424054)(760.1771081,94.35924054)(760.13710449,94.35924123)
\curveto(760.09710818,94.36924053)(760.05710822,94.36924053)(760.01710449,94.35924123)
\curveto(759.96710831,94.34924055)(759.91710836,94.34424056)(759.86710449,94.34424123)
\lineto(759.70210449,94.34424123)
\curveto(759.65210862,94.32424058)(759.60210867,94.31924058)(759.55210449,94.32924123)
\curveto(759.49210878,94.33924056)(759.43710884,94.33924056)(759.38710449,94.32924123)
\curveto(759.34710893,94.31924058)(759.30210897,94.31924058)(759.25210449,94.32924123)
\curveto(759.20210907,94.33924056)(759.15210912,94.33424057)(759.10210449,94.31424123)
\curveto(759.03210924,94.29424061)(758.95710932,94.28924061)(758.87710449,94.29924123)
\curveto(758.78710949,94.30924059)(758.70210957,94.31424059)(758.62210449,94.31424123)
\curveto(758.53210974,94.31424059)(758.43210984,94.30924059)(758.32210449,94.29924123)
\curveto(758.20211007,94.28924061)(758.10211017,94.29424061)(758.02210449,94.31424123)
\lineto(757.73710449,94.31424123)
\lineto(757.10710449,94.35924123)
\curveto(757.00711127,94.36924053)(756.91211136,94.37924052)(756.82210449,94.38924123)
\lineto(756.52210449,94.41924123)
\curveto(756.4721118,94.43924046)(756.42211185,94.44424046)(756.37210449,94.43424123)
\curveto(756.31211196,94.43424047)(756.25711202,94.44424046)(756.20710449,94.46424123)
\curveto(756.03711224,94.51424039)(755.8721124,94.55424035)(755.71210449,94.58424123)
\curveto(755.54211273,94.61424029)(755.38211289,94.66424024)(755.23210449,94.73424123)
\curveto(754.7721135,94.92423998)(754.39711388,95.14423976)(754.10710449,95.39424123)
\curveto(753.81711446,95.65423925)(753.5721147,96.01423889)(753.37210449,96.47424123)
\curveto(753.32211495,96.6042383)(753.28711499,96.73423817)(753.26710449,96.86424123)
\curveto(753.24711503,97.0042379)(753.22211505,97.14423776)(753.19210449,97.28424123)
\curveto(753.18211509,97.35423755)(753.1771151,97.41923748)(753.17710449,97.47924123)
\curveto(753.1771151,97.53923736)(753.1721151,97.6042373)(753.16210449,97.67424123)
\curveto(753.14211513,98.5042364)(753.29211498,99.17423573)(753.61210449,99.68424123)
\curveto(753.92211435,100.19423471)(754.36211391,100.57423433)(754.93210449,100.82424123)
\curveto(755.05211322,100.87423403)(755.1771131,100.91923398)(755.30710449,100.95924123)
\curveto(755.43711284,100.9992339)(755.5721127,101.04423386)(755.71210449,101.09424123)
\curveto(755.79211248,101.11423379)(755.8771124,101.12923377)(755.96710449,101.13924123)
\lineto(756.20710449,101.19924123)
\curveto(756.31711196,101.22923367)(756.42711185,101.24423366)(756.53710449,101.24424123)
\curveto(756.64711163,101.25423365)(756.75711152,101.26923363)(756.86710449,101.28924123)
\curveto(756.91711136,101.30923359)(756.96211131,101.31423359)(757.00210449,101.30424123)
\curveto(757.04211123,101.3042336)(757.08211119,101.30923359)(757.12210449,101.31924123)
\curveto(757.1721111,101.32923357)(757.22711105,101.32923357)(757.28710449,101.31924123)
\curveto(757.33711094,101.31923358)(757.38711089,101.32423358)(757.43710449,101.33424123)
\lineto(757.57210449,101.33424123)
\curveto(757.63211064,101.35423355)(757.70211057,101.35423355)(757.78210449,101.33424123)
\curveto(757.85211042,101.32423358)(757.91711036,101.32923357)(757.97710449,101.34924123)
\curveto(758.00711027,101.35923354)(758.04711023,101.36423354)(758.09710449,101.36424123)
\lineto(758.21710449,101.36424123)
\lineto(758.68210449,101.36424123)
\moveto(761.00710449,99.81924123)
\curveto(760.68710759,99.91923498)(760.32210795,99.97923492)(759.91210449,99.99924123)
\curveto(759.50210877,100.01923488)(759.09210918,100.02923487)(758.68210449,100.02924123)
\curveto(758.25211002,100.02923487)(757.83211044,100.01923488)(757.42210449,99.99924123)
\curveto(757.01211126,99.97923492)(756.62711165,99.93423497)(756.26710449,99.86424123)
\curveto(755.90711237,99.79423511)(755.58711269,99.68423522)(755.30710449,99.53424123)
\curveto(755.01711326,99.39423551)(754.78211349,99.1992357)(754.60210449,98.94924123)
\curveto(754.49211378,98.78923611)(754.41211386,98.60923629)(754.36210449,98.40924123)
\curveto(754.30211397,98.20923669)(754.272114,97.96423694)(754.27210449,97.67424123)
\curveto(754.29211398,97.65423725)(754.30211397,97.61923728)(754.30210449,97.56924123)
\curveto(754.29211398,97.51923738)(754.29211398,97.47923742)(754.30210449,97.44924123)
\curveto(754.32211395,97.36923753)(754.34211393,97.29423761)(754.36210449,97.22424123)
\curveto(754.3721139,97.16423774)(754.39211388,97.0992378)(754.42210449,97.02924123)
\curveto(754.54211373,96.75923814)(754.71211356,96.53923836)(754.93210449,96.36924123)
\curveto(755.14211313,96.20923869)(755.38711289,96.07423883)(755.66710449,95.96424123)
\curveto(755.7771125,95.91423899)(755.89711238,95.87423903)(756.02710449,95.84424123)
\curveto(756.14711213,95.82423908)(756.272112,95.7992391)(756.40210449,95.76924123)
\curveto(756.45211182,95.74923915)(756.50711177,95.73923916)(756.56710449,95.73924123)
\curveto(756.61711166,95.73923916)(756.66711161,95.73423917)(756.71710449,95.72424123)
\curveto(756.80711147,95.71423919)(756.90211137,95.7042392)(757.00210449,95.69424123)
\curveto(757.09211118,95.68423922)(757.18711109,95.67423923)(757.28710449,95.66424123)
\curveto(757.36711091,95.66423924)(757.45211082,95.65923924)(757.54210449,95.64924123)
\lineto(757.78210449,95.64924123)
\lineto(757.96210449,95.64924123)
\curveto(757.99211028,95.63923926)(758.02711025,95.63423927)(758.06710449,95.63424123)
\lineto(758.20210449,95.63424123)
\lineto(758.65210449,95.63424123)
\curveto(758.73210954,95.63423927)(758.81710946,95.62923927)(758.90710449,95.61924123)
\curveto(758.98710929,95.61923928)(759.06210921,95.62923927)(759.13210449,95.64924123)
\lineto(759.40210449,95.64924123)
\curveto(759.42210885,95.64923925)(759.45210882,95.64423926)(759.49210449,95.63424123)
\curveto(759.52210875,95.63423927)(759.54710873,95.63923926)(759.56710449,95.64924123)
\curveto(759.66710861,95.65923924)(759.76710851,95.66423924)(759.86710449,95.66424123)
\curveto(759.95710832,95.67423923)(760.05710822,95.68423922)(760.16710449,95.69424123)
\curveto(760.28710799,95.72423918)(760.41210786,95.73923916)(760.54210449,95.73924123)
\curveto(760.66210761,95.74923915)(760.7771075,95.77423913)(760.88710449,95.81424123)
\curveto(761.18710709,95.89423901)(761.45210682,95.97923892)(761.68210449,96.06924123)
\curveto(761.91210636,96.16923873)(762.12710615,96.31423859)(762.32710449,96.50424123)
\curveto(762.52710575,96.71423819)(762.6771056,96.97923792)(762.77710449,97.29924123)
\curveto(762.79710548,97.33923756)(762.80710547,97.37423753)(762.80710449,97.40424123)
\curveto(762.79710548,97.44423746)(762.80210547,97.48923741)(762.82210449,97.53924123)
\curveto(762.83210544,97.57923732)(762.84210543,97.64923725)(762.85210449,97.74924123)
\curveto(762.86210541,97.85923704)(762.85710542,97.94423696)(762.83710449,98.00424123)
\curveto(762.81710546,98.07423683)(762.80710547,98.14423676)(762.80710449,98.21424123)
\curveto(762.79710548,98.28423662)(762.78210549,98.34923655)(762.76210449,98.40924123)
\curveto(762.70210557,98.60923629)(762.61710566,98.78923611)(762.50710449,98.94924123)
\curveto(762.48710579,98.97923592)(762.46710581,99.0042359)(762.44710449,99.02424123)
\lineto(762.38710449,99.08424123)
\curveto(762.36710591,99.12423578)(762.32710595,99.17423573)(762.26710449,99.23424123)
\curveto(762.12710615,99.33423557)(761.99710628,99.41923548)(761.87710449,99.48924123)
\curveto(761.75710652,99.55923534)(761.61210666,99.62923527)(761.44210449,99.69924123)
\curveto(761.3721069,99.72923517)(761.30210697,99.74923515)(761.23210449,99.75924123)
\curveto(761.16210711,99.77923512)(761.08710719,99.7992351)(761.00710449,99.81924123)
}
}
{
\newrgbcolor{curcolor}{0 0 0}
\pscustom[linestyle=none,fillstyle=solid,fillcolor=curcolor]
{
\newpath
\moveto(753.16210449,106.77385061)
\curveto(753.16211511,106.87384575)(753.1721151,106.96884566)(753.19210449,107.05885061)
\curveto(753.20211507,107.14884548)(753.23211504,107.21384541)(753.28210449,107.25385061)
\curveto(753.36211491,107.31384531)(753.46711481,107.34384528)(753.59710449,107.34385061)
\lineto(753.98710449,107.34385061)
\lineto(755.48710449,107.34385061)
\lineto(761.87710449,107.34385061)
\lineto(763.04710449,107.34385061)
\lineto(763.36210449,107.34385061)
\curveto(763.46210481,107.35384527)(763.54210473,107.33884529)(763.60210449,107.29885061)
\curveto(763.68210459,107.24884538)(763.73210454,107.17384545)(763.75210449,107.07385061)
\curveto(763.76210451,106.98384564)(763.76710451,106.87384575)(763.76710449,106.74385061)
\lineto(763.76710449,106.51885061)
\curveto(763.74710453,106.43884619)(763.73210454,106.36884626)(763.72210449,106.30885061)
\curveto(763.70210457,106.24884638)(763.66210461,106.19884643)(763.60210449,106.15885061)
\curveto(763.54210473,106.11884651)(763.46710481,106.09884653)(763.37710449,106.09885061)
\lineto(763.07710449,106.09885061)
\lineto(761.98210449,106.09885061)
\lineto(756.64210449,106.09885061)
\curveto(756.55211172,106.07884655)(756.4771118,106.06384656)(756.41710449,106.05385061)
\curveto(756.34711193,106.05384657)(756.28711199,106.0238466)(756.23710449,105.96385061)
\curveto(756.18711209,105.89384673)(756.16211211,105.80384682)(756.16210449,105.69385061)
\curveto(756.15211212,105.59384703)(756.14711213,105.48384714)(756.14710449,105.36385061)
\lineto(756.14710449,104.22385061)
\lineto(756.14710449,103.72885061)
\curveto(756.13711214,103.56884906)(756.0771122,103.45884917)(755.96710449,103.39885061)
\curveto(755.93711234,103.37884925)(755.90711237,103.36884926)(755.87710449,103.36885061)
\curveto(755.83711244,103.36884926)(755.79211248,103.36384926)(755.74210449,103.35385061)
\curveto(755.62211265,103.33384929)(755.51211276,103.33884929)(755.41210449,103.36885061)
\curveto(755.31211296,103.40884922)(755.24211303,103.46384916)(755.20210449,103.53385061)
\curveto(755.15211312,103.61384901)(755.12711315,103.73384889)(755.12710449,103.89385061)
\curveto(755.12711315,104.05384857)(755.11211316,104.18884844)(755.08210449,104.29885061)
\curveto(755.0721132,104.34884828)(755.06711321,104.40384822)(755.06710449,104.46385061)
\curveto(755.05711322,104.5238481)(755.04211323,104.58384804)(755.02210449,104.64385061)
\curveto(754.9721133,104.79384783)(754.92211335,104.93884769)(754.87210449,105.07885061)
\curveto(754.81211346,105.21884741)(754.74211353,105.35384727)(754.66210449,105.48385061)
\curveto(754.5721137,105.623847)(754.46711381,105.74384688)(754.34710449,105.84385061)
\curveto(754.22711405,105.94384668)(754.09711418,106.03884659)(753.95710449,106.12885061)
\curveto(753.85711442,106.18884644)(753.74711453,106.23384639)(753.62710449,106.26385061)
\curveto(753.50711477,106.30384632)(753.40211487,106.35384627)(753.31210449,106.41385061)
\curveto(753.25211502,106.46384616)(753.21211506,106.53384609)(753.19210449,106.62385061)
\curveto(753.18211509,106.64384598)(753.1771151,106.66884596)(753.17710449,106.69885061)
\curveto(753.1771151,106.7288459)(753.1721151,106.75384587)(753.16210449,106.77385061)
}
}
{
\newrgbcolor{curcolor}{0 0 0}
\pscustom[linestyle=none,fillstyle=solid,fillcolor=curcolor]
{
\newpath
\moveto(753.16210449,115.12345998)
\curveto(753.16211511,115.22345513)(753.1721151,115.31845503)(753.19210449,115.40845998)
\curveto(753.20211507,115.49845485)(753.23211504,115.56345479)(753.28210449,115.60345998)
\curveto(753.36211491,115.66345469)(753.46711481,115.69345466)(753.59710449,115.69345998)
\lineto(753.98710449,115.69345998)
\lineto(755.48710449,115.69345998)
\lineto(761.87710449,115.69345998)
\lineto(763.04710449,115.69345998)
\lineto(763.36210449,115.69345998)
\curveto(763.46210481,115.70345465)(763.54210473,115.68845466)(763.60210449,115.64845998)
\curveto(763.68210459,115.59845475)(763.73210454,115.52345483)(763.75210449,115.42345998)
\curveto(763.76210451,115.33345502)(763.76710451,115.22345513)(763.76710449,115.09345998)
\lineto(763.76710449,114.86845998)
\curveto(763.74710453,114.78845556)(763.73210454,114.71845563)(763.72210449,114.65845998)
\curveto(763.70210457,114.59845575)(763.66210461,114.5484558)(763.60210449,114.50845998)
\curveto(763.54210473,114.46845588)(763.46710481,114.4484559)(763.37710449,114.44845998)
\lineto(763.07710449,114.44845998)
\lineto(761.98210449,114.44845998)
\lineto(756.64210449,114.44845998)
\curveto(756.55211172,114.42845592)(756.4771118,114.41345594)(756.41710449,114.40345998)
\curveto(756.34711193,114.40345595)(756.28711199,114.37345598)(756.23710449,114.31345998)
\curveto(756.18711209,114.24345611)(756.16211211,114.1534562)(756.16210449,114.04345998)
\curveto(756.15211212,113.94345641)(756.14711213,113.83345652)(756.14710449,113.71345998)
\lineto(756.14710449,112.57345998)
\lineto(756.14710449,112.07845998)
\curveto(756.13711214,111.91845843)(756.0771122,111.80845854)(755.96710449,111.74845998)
\curveto(755.93711234,111.72845862)(755.90711237,111.71845863)(755.87710449,111.71845998)
\curveto(755.83711244,111.71845863)(755.79211248,111.71345864)(755.74210449,111.70345998)
\curveto(755.62211265,111.68345867)(755.51211276,111.68845866)(755.41210449,111.71845998)
\curveto(755.31211296,111.75845859)(755.24211303,111.81345854)(755.20210449,111.88345998)
\curveto(755.15211312,111.96345839)(755.12711315,112.08345827)(755.12710449,112.24345998)
\curveto(755.12711315,112.40345795)(755.11211316,112.53845781)(755.08210449,112.64845998)
\curveto(755.0721132,112.69845765)(755.06711321,112.7534576)(755.06710449,112.81345998)
\curveto(755.05711322,112.87345748)(755.04211323,112.93345742)(755.02210449,112.99345998)
\curveto(754.9721133,113.14345721)(754.92211335,113.28845706)(754.87210449,113.42845998)
\curveto(754.81211346,113.56845678)(754.74211353,113.70345665)(754.66210449,113.83345998)
\curveto(754.5721137,113.97345638)(754.46711381,114.09345626)(754.34710449,114.19345998)
\curveto(754.22711405,114.29345606)(754.09711418,114.38845596)(753.95710449,114.47845998)
\curveto(753.85711442,114.53845581)(753.74711453,114.58345577)(753.62710449,114.61345998)
\curveto(753.50711477,114.6534557)(753.40211487,114.70345565)(753.31210449,114.76345998)
\curveto(753.25211502,114.81345554)(753.21211506,114.88345547)(753.19210449,114.97345998)
\curveto(753.18211509,114.99345536)(753.1771151,115.01845533)(753.17710449,115.04845998)
\curveto(753.1771151,115.07845527)(753.1721151,115.10345525)(753.16210449,115.12345998)
}
}
{
\newrgbcolor{curcolor}{0 0 0}
\pscustom[linestyle=none,fillstyle=solid,fillcolor=curcolor]
{
\newpath
\moveto(773.99842041,37.28705373)
\curveto(773.99843111,37.35704805)(773.99843111,37.43704797)(773.99842041,37.52705373)
\curveto(773.98843112,37.61704779)(773.98843112,37.70204771)(773.99842041,37.78205373)
\curveto(773.99843111,37.87204754)(774.0084311,37.95204746)(774.02842041,38.02205373)
\curveto(774.04843106,38.10204731)(774.07843103,38.15704725)(774.11842041,38.18705373)
\curveto(774.16843094,38.21704719)(774.24343086,38.23704717)(774.34342041,38.24705373)
\curveto(774.43343067,38.26704714)(774.53843057,38.27704713)(774.65842041,38.27705373)
\curveto(774.76843034,38.28704712)(774.88343022,38.28704712)(775.00342041,38.27705373)
\lineto(775.30342041,38.27705373)
\lineto(778.31842041,38.27705373)
\lineto(781.21342041,38.27705373)
\curveto(781.54342356,38.27704713)(781.86842324,38.27204714)(782.18842041,38.26205373)
\curveto(782.49842261,38.26204715)(782.77842233,38.22204719)(783.02842041,38.14205373)
\curveto(783.37842173,38.02204739)(783.67342143,37.86704754)(783.91342041,37.67705373)
\curveto(784.14342096,37.48704792)(784.34342076,37.24704816)(784.51342041,36.95705373)
\curveto(784.56342054,36.89704851)(784.59842051,36.83204858)(784.61842041,36.76205373)
\curveto(784.63842047,36.70204871)(784.66342044,36.63204878)(784.69342041,36.55205373)
\curveto(784.74342036,36.43204898)(784.77842033,36.30204911)(784.79842041,36.16205373)
\curveto(784.82842028,36.03204938)(784.85842025,35.89704951)(784.88842041,35.75705373)
\curveto(784.9084202,35.7070497)(784.91342019,35.65704975)(784.90342041,35.60705373)
\curveto(784.89342021,35.55704985)(784.89342021,35.50204991)(784.90342041,35.44205373)
\curveto(784.91342019,35.42204999)(784.91342019,35.39705001)(784.90342041,35.36705373)
\curveto(784.9034202,35.33705007)(784.9084202,35.3120501)(784.91842041,35.29205373)
\curveto(784.92842018,35.25205016)(784.93342017,35.19705021)(784.93342041,35.12705373)
\curveto(784.93342017,35.05705035)(784.92842018,35.00205041)(784.91842041,34.96205373)
\curveto(784.9084202,34.9120505)(784.9084202,34.85705055)(784.91842041,34.79705373)
\curveto(784.92842018,34.73705067)(784.92342018,34.68205073)(784.90342041,34.63205373)
\curveto(784.87342023,34.50205091)(784.85342025,34.37705103)(784.84342041,34.25705373)
\curveto(784.83342027,34.13705127)(784.8084203,34.02205139)(784.76842041,33.91205373)
\curveto(784.64842046,33.54205187)(784.47842063,33.22205219)(784.25842041,32.95205373)
\curveto(784.03842107,32.68205273)(783.75842135,32.47205294)(783.41842041,32.32205373)
\curveto(783.29842181,32.27205314)(783.17342193,32.22705318)(783.04342041,32.18705373)
\curveto(782.91342219,32.15705325)(782.77842233,32.12205329)(782.63842041,32.08205373)
\curveto(782.58842252,32.07205334)(782.54842256,32.06705334)(782.51842041,32.06705373)
\curveto(782.47842263,32.06705334)(782.43342267,32.06205335)(782.38342041,32.05205373)
\curveto(782.35342275,32.04205337)(782.31842279,32.03705337)(782.27842041,32.03705373)
\curveto(782.22842288,32.03705337)(782.18842292,32.03205338)(782.15842041,32.02205373)
\lineto(781.99342041,32.02205373)
\curveto(781.91342319,32.00205341)(781.81342329,31.99705341)(781.69342041,32.00705373)
\curveto(781.56342354,32.01705339)(781.47342363,32.03205338)(781.42342041,32.05205373)
\curveto(781.33342377,32.07205334)(781.26842384,32.12705328)(781.22842041,32.21705373)
\curveto(781.2084239,32.24705316)(781.2034239,32.27705313)(781.21342041,32.30705373)
\curveto(781.21342389,32.33705307)(781.2084239,32.37705303)(781.19842041,32.42705373)
\curveto(781.18842392,32.46705294)(781.18342392,32.5070529)(781.18342041,32.54705373)
\lineto(781.18342041,32.69705373)
\curveto(781.18342392,32.81705259)(781.18842392,32.93705247)(781.19842041,33.05705373)
\curveto(781.19842391,33.18705222)(781.23342387,33.27705213)(781.30342041,33.32705373)
\curveto(781.36342374,33.36705204)(781.42342368,33.38705202)(781.48342041,33.38705373)
\curveto(781.54342356,33.38705202)(781.61342349,33.39705201)(781.69342041,33.41705373)
\curveto(781.72342338,33.42705198)(781.75842335,33.42705198)(781.79842041,33.41705373)
\curveto(781.82842328,33.41705199)(781.85342325,33.42205199)(781.87342041,33.43205373)
\lineto(782.08342041,33.43205373)
\curveto(782.13342297,33.45205196)(782.18342292,33.45705195)(782.23342041,33.44705373)
\curveto(782.27342283,33.44705196)(782.31842279,33.45705195)(782.36842041,33.47705373)
\curveto(782.49842261,33.5070519)(782.62342248,33.53705187)(782.74342041,33.56705373)
\curveto(782.85342225,33.59705181)(782.95842215,33.64205177)(783.05842041,33.70205373)
\curveto(783.34842176,33.87205154)(783.55342155,34.14205127)(783.67342041,34.51205373)
\curveto(783.69342141,34.56205085)(783.7084214,34.6120508)(783.71842041,34.66205373)
\curveto(783.71842139,34.72205069)(783.72842138,34.77705063)(783.74842041,34.82705373)
\lineto(783.74842041,34.90205373)
\curveto(783.75842135,34.97205044)(783.76842134,35.06705034)(783.77842041,35.18705373)
\curveto(783.77842133,35.31705009)(783.76842134,35.41704999)(783.74842041,35.48705373)
\curveto(783.72842138,35.55704985)(783.71342139,35.62704978)(783.70342041,35.69705373)
\curveto(783.68342142,35.77704963)(783.66342144,35.84704956)(783.64342041,35.90705373)
\curveto(783.48342162,36.28704912)(783.2084219,36.56204885)(782.81842041,36.73205373)
\curveto(782.68842242,36.78204863)(782.53342257,36.81704859)(782.35342041,36.83705373)
\curveto(782.17342293,36.86704854)(781.98842312,36.88204853)(781.79842041,36.88205373)
\curveto(781.59842351,36.89204852)(781.39842371,36.89204852)(781.19842041,36.88205373)
\lineto(780.62842041,36.88205373)
\lineto(776.38342041,36.88205373)
\lineto(774.83842041,36.88205373)
\curveto(774.72843038,36.88204853)(774.6084305,36.87704853)(774.47842041,36.86705373)
\curveto(774.34843076,36.85704855)(774.24343086,36.87704853)(774.16342041,36.92705373)
\curveto(774.09343101,36.98704842)(774.04343106,37.06704834)(774.01342041,37.16705373)
\curveto(774.01343109,37.18704822)(774.01343109,37.2070482)(774.01342041,37.22705373)
\curveto(774.01343109,37.24704816)(774.0084311,37.26704814)(773.99842041,37.28705373)
}
}
{
\newrgbcolor{curcolor}{0 0 0}
\pscustom[linestyle=none,fillstyle=solid,fillcolor=curcolor]
{
\newpath
\moveto(776.95342041,40.82072561)
\lineto(776.95342041,41.25572561)
\curveto(776.95342815,41.40572364)(776.99342811,41.51072354)(777.07342041,41.57072561)
\curveto(777.15342795,41.62072343)(777.25342785,41.6457234)(777.37342041,41.64572561)
\curveto(777.49342761,41.65572339)(777.61342749,41.66072339)(777.73342041,41.66072561)
\lineto(779.15842041,41.66072561)
\lineto(781.42342041,41.66072561)
\lineto(782.11342041,41.66072561)
\curveto(782.34342276,41.66072339)(782.54342256,41.68572336)(782.71342041,41.73572561)
\curveto(783.16342194,41.89572315)(783.47842163,42.19572285)(783.65842041,42.63572561)
\curveto(783.74842136,42.85572219)(783.78342132,43.12072193)(783.76342041,43.43072561)
\curveto(783.73342137,43.74072131)(783.67842143,43.99072106)(783.59842041,44.18072561)
\curveto(783.45842165,44.51072054)(783.28342182,44.77072028)(783.07342041,44.96072561)
\curveto(782.85342225,45.16071989)(782.56842254,45.31571973)(782.21842041,45.42572561)
\curveto(782.13842297,45.45571959)(782.05842305,45.47571957)(781.97842041,45.48572561)
\curveto(781.89842321,45.49571955)(781.81342329,45.51071954)(781.72342041,45.53072561)
\curveto(781.67342343,45.54071951)(781.62842348,45.54071951)(781.58842041,45.53072561)
\curveto(781.54842356,45.53071952)(781.5034236,45.54071951)(781.45342041,45.56072561)
\lineto(781.13842041,45.56072561)
\curveto(781.05842405,45.58071947)(780.96842414,45.58571946)(780.86842041,45.57572561)
\curveto(780.75842435,45.56571948)(780.65842445,45.56071949)(780.56842041,45.56072561)
\lineto(779.39842041,45.56072561)
\lineto(777.80842041,45.56072561)
\curveto(777.68842742,45.56071949)(777.56342754,45.55571949)(777.43342041,45.54572561)
\curveto(777.29342781,45.5457195)(777.18342792,45.57071948)(777.10342041,45.62072561)
\curveto(777.05342805,45.66071939)(777.02342808,45.70571934)(777.01342041,45.75572561)
\curveto(776.99342811,45.81571923)(776.97342813,45.88571916)(776.95342041,45.96572561)
\lineto(776.95342041,46.19072561)
\curveto(776.95342815,46.31071874)(776.95842815,46.41571863)(776.96842041,46.50572561)
\curveto(776.97842813,46.60571844)(777.02342808,46.68071837)(777.10342041,46.73072561)
\curveto(777.15342795,46.78071827)(777.22842788,46.80571824)(777.32842041,46.80572561)
\lineto(777.61342041,46.80572561)
\lineto(778.63342041,46.80572561)
\lineto(782.66842041,46.80572561)
\lineto(784.01842041,46.80572561)
\curveto(784.13842097,46.80571824)(784.25342085,46.80071825)(784.36342041,46.79072561)
\curveto(784.46342064,46.79071826)(784.53842057,46.75571829)(784.58842041,46.68572561)
\curveto(784.61842049,46.6457184)(784.64342046,46.58571846)(784.66342041,46.50572561)
\curveto(784.67342043,46.42571862)(784.68342042,46.33571871)(784.69342041,46.23572561)
\curveto(784.69342041,46.1457189)(784.68842042,46.05571899)(784.67842041,45.96572561)
\curveto(784.66842044,45.88571916)(784.64842046,45.82571922)(784.61842041,45.78572561)
\curveto(784.57842053,45.73571931)(784.51342059,45.69071936)(784.42342041,45.65072561)
\curveto(784.38342072,45.64071941)(784.32842078,45.63071942)(784.25842041,45.62072561)
\curveto(784.18842092,45.62071943)(784.12342098,45.61571943)(784.06342041,45.60572561)
\curveto(783.99342111,45.59571945)(783.93842117,45.57571947)(783.89842041,45.54572561)
\curveto(783.85842125,45.51571953)(783.84342126,45.47071958)(783.85342041,45.41072561)
\curveto(783.87342123,45.33071972)(783.93342117,45.2507198)(784.03342041,45.17072561)
\curveto(784.12342098,45.09071996)(784.19342091,45.01572003)(784.24342041,44.94572561)
\curveto(784.4034207,44.72572032)(784.54342056,44.47572057)(784.66342041,44.19572561)
\curveto(784.71342039,44.08572096)(784.74342036,43.97072108)(784.75342041,43.85072561)
\curveto(784.77342033,43.74072131)(784.79842031,43.62572142)(784.82842041,43.50572561)
\curveto(784.83842027,43.45572159)(784.83842027,43.40072165)(784.82842041,43.34072561)
\curveto(784.81842029,43.29072176)(784.82342028,43.24072181)(784.84342041,43.19072561)
\curveto(784.86342024,43.09072196)(784.86342024,43.00072205)(784.84342041,42.92072561)
\lineto(784.84342041,42.77072561)
\curveto(784.82342028,42.72072233)(784.81342029,42.66072239)(784.81342041,42.59072561)
\curveto(784.81342029,42.53072252)(784.8084203,42.47572257)(784.79842041,42.42572561)
\curveto(784.77842033,42.38572266)(784.76842034,42.3457227)(784.76842041,42.30572561)
\curveto(784.77842033,42.27572277)(784.77342033,42.23572281)(784.75342041,42.18572561)
\lineto(784.69342041,41.94572561)
\curveto(784.67342043,41.87572317)(784.64342046,41.80072325)(784.60342041,41.72072561)
\curveto(784.49342061,41.46072359)(784.34842076,41.24072381)(784.16842041,41.06072561)
\curveto(783.97842113,40.89072416)(783.75342135,40.7507243)(783.49342041,40.64072561)
\curveto(783.4034217,40.60072445)(783.31342179,40.57072448)(783.22342041,40.55072561)
\lineto(782.92342041,40.49072561)
\curveto(782.86342224,40.47072458)(782.8084223,40.46072459)(782.75842041,40.46072561)
\curveto(782.69842241,40.47072458)(782.63342247,40.46572458)(782.56342041,40.44572561)
\curveto(782.54342256,40.43572461)(782.51842259,40.43072462)(782.48842041,40.43072561)
\curveto(782.44842266,40.43072462)(782.41342269,40.42572462)(782.38342041,40.41572561)
\lineto(782.23342041,40.41572561)
\curveto(782.19342291,40.40572464)(782.14842296,40.40072465)(782.09842041,40.40072561)
\curveto(782.03842307,40.41072464)(781.98342312,40.41572463)(781.93342041,40.41572561)
\lineto(781.33342041,40.41572561)
\lineto(778.57342041,40.41572561)
\lineto(777.61342041,40.41572561)
\lineto(777.34342041,40.41572561)
\curveto(777.25342785,40.41572463)(777.17842793,40.43572461)(777.11842041,40.47572561)
\curveto(777.04842806,40.51572453)(776.99842811,40.59072446)(776.96842041,40.70072561)
\curveto(776.95842815,40.72072433)(776.95842815,40.74072431)(776.96842041,40.76072561)
\curveto(776.96842814,40.78072427)(776.96342814,40.80072425)(776.95342041,40.82072561)
}
}
{
\newrgbcolor{curcolor}{0 0 0}
\pscustom[linestyle=none,fillstyle=solid,fillcolor=curcolor]
{
\newpath
\moveto(776.80342041,52.39533498)
\curveto(776.78342832,53.02532975)(776.86842824,53.53032924)(777.05842041,53.91033498)
\curveto(777.24842786,54.29032848)(777.53342757,54.59532818)(777.91342041,54.82533498)
\curveto(778.01342709,54.88532789)(778.12342698,54.93032784)(778.24342041,54.96033498)
\curveto(778.35342675,55.00032777)(778.46842664,55.03532774)(778.58842041,55.06533498)
\curveto(778.77842633,55.11532766)(778.98342612,55.14532763)(779.20342041,55.15533498)
\curveto(779.42342568,55.16532761)(779.64842546,55.1703276)(779.87842041,55.17033498)
\lineto(781.48342041,55.17033498)
\lineto(783.82342041,55.17033498)
\curveto(783.99342111,55.1703276)(784.16342094,55.16532761)(784.33342041,55.15533498)
\curveto(784.5034206,55.15532762)(784.61342049,55.09032768)(784.66342041,54.96033498)
\curveto(784.68342042,54.91032786)(784.69342041,54.85532792)(784.69342041,54.79533498)
\curveto(784.7034204,54.74532803)(784.7084204,54.69032808)(784.70842041,54.63033498)
\curveto(784.7084204,54.50032827)(784.7034204,54.3753284)(784.69342041,54.25533498)
\curveto(784.69342041,54.13532864)(784.65342045,54.05032872)(784.57342041,54.00033498)
\curveto(784.5034206,53.95032882)(784.41342069,53.92532885)(784.30342041,53.92533498)
\lineto(783.97342041,53.92533498)
\lineto(782.68342041,53.92533498)
\lineto(780.23842041,53.92533498)
\curveto(779.96842514,53.92532885)(779.7034254,53.92032885)(779.44342041,53.91033498)
\curveto(779.17342593,53.90032887)(778.94342616,53.85532892)(778.75342041,53.77533498)
\curveto(778.55342655,53.69532908)(778.39342671,53.5753292)(778.27342041,53.41533498)
\curveto(778.14342696,53.25532952)(778.04342706,53.0703297)(777.97342041,52.86033498)
\curveto(777.95342715,52.80032997)(777.94342716,52.73533004)(777.94342041,52.66533498)
\curveto(777.93342717,52.60533017)(777.91842719,52.54533023)(777.89842041,52.48533498)
\curveto(777.88842722,52.43533034)(777.88842722,52.35533042)(777.89842041,52.24533498)
\curveto(777.89842721,52.14533063)(777.9034272,52.0753307)(777.91342041,52.03533498)
\curveto(777.93342717,51.99533078)(777.94342716,51.96033081)(777.94342041,51.93033498)
\curveto(777.93342717,51.90033087)(777.93342717,51.86533091)(777.94342041,51.82533498)
\curveto(777.97342713,51.69533108)(778.0084271,51.5703312)(778.04842041,51.45033498)
\curveto(778.07842703,51.34033143)(778.12342698,51.23533154)(778.18342041,51.13533498)
\curveto(778.2034269,51.09533168)(778.22342688,51.06033171)(778.24342041,51.03033498)
\curveto(778.26342684,51.00033177)(778.28342682,50.96533181)(778.30342041,50.92533498)
\curveto(778.55342655,50.5753322)(778.92842618,50.32033245)(779.42842041,50.16033498)
\curveto(779.5084256,50.13033264)(779.59342551,50.11033266)(779.68342041,50.10033498)
\curveto(779.76342534,50.09033268)(779.84342526,50.0753327)(779.92342041,50.05533498)
\curveto(779.97342513,50.03533274)(780.02342508,50.03033274)(780.07342041,50.04033498)
\curveto(780.11342499,50.05033272)(780.15342495,50.04533273)(780.19342041,50.02533498)
\lineto(780.50842041,50.02533498)
\curveto(780.53842457,50.01533276)(780.57342453,50.01033276)(780.61342041,50.01033498)
\curveto(780.65342445,50.02033275)(780.69842441,50.02533275)(780.74842041,50.02533498)
\lineto(781.19842041,50.02533498)
\lineto(782.63842041,50.02533498)
\lineto(783.95842041,50.02533498)
\lineto(784.30342041,50.02533498)
\curveto(784.41342069,50.02533275)(784.5034206,50.00033277)(784.57342041,49.95033498)
\curveto(784.65342045,49.90033287)(784.69342041,49.81033296)(784.69342041,49.68033498)
\curveto(784.7034204,49.56033321)(784.7084204,49.43533334)(784.70842041,49.30533498)
\curveto(784.7084204,49.22533355)(784.7034204,49.15033362)(784.69342041,49.08033498)
\curveto(784.68342042,49.01033376)(784.65842045,48.95033382)(784.61842041,48.90033498)
\curveto(784.56842054,48.82033395)(784.47342063,48.78033399)(784.33342041,48.78033498)
\lineto(783.92842041,48.78033498)
\lineto(782.15842041,48.78033498)
\lineto(778.52842041,48.78033498)
\lineto(777.61342041,48.78033498)
\lineto(777.34342041,48.78033498)
\curveto(777.25342785,48.78033399)(777.18342792,48.80033397)(777.13342041,48.84033498)
\curveto(777.07342803,48.8703339)(777.03342807,48.92033385)(777.01342041,48.99033498)
\curveto(777.0034281,49.03033374)(776.99342811,49.08533369)(776.98342041,49.15533498)
\curveto(776.97342813,49.23533354)(776.96842814,49.31533346)(776.96842041,49.39533498)
\curveto(776.96842814,49.4753333)(776.97342813,49.55033322)(776.98342041,49.62033498)
\curveto(776.99342811,49.70033307)(777.0084281,49.75533302)(777.02842041,49.78533498)
\curveto(777.09842801,49.89533288)(777.18842792,49.94533283)(777.29842041,49.93533498)
\curveto(777.39842771,49.92533285)(777.51342759,49.94033283)(777.64342041,49.98033498)
\curveto(777.7034274,50.00033277)(777.75342735,50.04033273)(777.79342041,50.10033498)
\curveto(777.8034273,50.22033255)(777.75842735,50.31533246)(777.65842041,50.38533498)
\curveto(777.55842755,50.46533231)(777.47842763,50.54533223)(777.41842041,50.62533498)
\curveto(777.31842779,50.76533201)(777.22842788,50.90533187)(777.14842041,51.04533498)
\curveto(777.05842805,51.19533158)(776.98342812,51.36533141)(776.92342041,51.55533498)
\curveto(776.89342821,51.63533114)(776.87342823,51.72033105)(776.86342041,51.81033498)
\curveto(776.85342825,51.91033086)(776.83842827,52.00533077)(776.81842041,52.09533498)
\curveto(776.8084283,52.14533063)(776.8034283,52.19533058)(776.80342041,52.24533498)
\lineto(776.80342041,52.39533498)
}
}
{
\newrgbcolor{curcolor}{0 0 0}
\pscustom[linestyle=none,fillstyle=solid,fillcolor=curcolor]
{
}
}
{
\newrgbcolor{curcolor}{0 0 0}
\pscustom[linestyle=none,fillstyle=solid,fillcolor=curcolor]
{
\newpath
\moveto(774.07342041,64.24510061)
\curveto(774.06343104,64.93509597)(774.18343092,65.53509537)(774.43342041,66.04510061)
\curveto(774.68343042,66.56509434)(775.01843009,66.96009395)(775.43842041,67.23010061)
\curveto(775.51842959,67.28009363)(775.6084295,67.32509358)(775.70842041,67.36510061)
\curveto(775.79842931,67.4050935)(775.89342921,67.45009346)(775.99342041,67.50010061)
\curveto(776.09342901,67.54009337)(776.19342891,67.57009334)(776.29342041,67.59010061)
\curveto(776.39342871,67.6100933)(776.49842861,67.63009328)(776.60842041,67.65010061)
\curveto(776.65842845,67.67009324)(776.7034284,67.67509323)(776.74342041,67.66510061)
\curveto(776.78342832,67.65509325)(776.82842828,67.66009325)(776.87842041,67.68010061)
\curveto(776.92842818,67.69009322)(777.01342809,67.69509321)(777.13342041,67.69510061)
\curveto(777.24342786,67.69509321)(777.32842778,67.69009322)(777.38842041,67.68010061)
\curveto(777.44842766,67.66009325)(777.5084276,67.65009326)(777.56842041,67.65010061)
\curveto(777.62842748,67.66009325)(777.68842742,67.65509325)(777.74842041,67.63510061)
\curveto(777.88842722,67.59509331)(778.02342708,67.56009335)(778.15342041,67.53010061)
\curveto(778.28342682,67.50009341)(778.4084267,67.46009345)(778.52842041,67.41010061)
\curveto(778.66842644,67.35009356)(778.79342631,67.28009363)(778.90342041,67.20010061)
\curveto(779.01342609,67.13009378)(779.12342598,67.05509385)(779.23342041,66.97510061)
\lineto(779.29342041,66.91510061)
\curveto(779.31342579,66.905094)(779.33342577,66.89009402)(779.35342041,66.87010061)
\curveto(779.51342559,66.75009416)(779.65842545,66.61509429)(779.78842041,66.46510061)
\curveto(779.91842519,66.31509459)(780.04342506,66.15509475)(780.16342041,65.98510061)
\curveto(780.38342472,65.67509523)(780.58842452,65.38009553)(780.77842041,65.10010061)
\curveto(780.91842419,64.87009604)(781.05342405,64.64009627)(781.18342041,64.41010061)
\curveto(781.31342379,64.19009672)(781.44842366,63.97009694)(781.58842041,63.75010061)
\curveto(781.75842335,63.50009741)(781.93842317,63.26009765)(782.12842041,63.03010061)
\curveto(782.31842279,62.8100981)(782.54342256,62.62009829)(782.80342041,62.46010061)
\curveto(782.86342224,62.42009849)(782.92342218,62.38509852)(782.98342041,62.35510061)
\curveto(783.03342207,62.32509858)(783.09842201,62.29509861)(783.17842041,62.26510061)
\curveto(783.24842186,62.24509866)(783.3084218,62.24009867)(783.35842041,62.25010061)
\curveto(783.42842168,62.27009864)(783.48342162,62.3050986)(783.52342041,62.35510061)
\curveto(783.55342155,62.4050985)(783.57342153,62.46509844)(783.58342041,62.53510061)
\lineto(783.58342041,62.77510061)
\lineto(783.58342041,63.52510061)
\lineto(783.58342041,66.33010061)
\lineto(783.58342041,66.99010061)
\curveto(783.58342152,67.08009383)(783.58842152,67.16509374)(783.59842041,67.24510061)
\curveto(783.59842151,67.32509358)(783.61842149,67.39009352)(783.65842041,67.44010061)
\curveto(783.69842141,67.49009342)(783.77342133,67.53009338)(783.88342041,67.56010061)
\curveto(783.98342112,67.60009331)(784.08342102,67.6100933)(784.18342041,67.59010061)
\lineto(784.31842041,67.59010061)
\curveto(784.38842072,67.57009334)(784.44842066,67.55009336)(784.49842041,67.53010061)
\curveto(784.54842056,67.5100934)(784.58842052,67.47509343)(784.61842041,67.42510061)
\curveto(784.65842045,67.37509353)(784.67842043,67.3050936)(784.67842041,67.21510061)
\lineto(784.67842041,66.94510061)
\lineto(784.67842041,66.04510061)
\lineto(784.67842041,62.53510061)
\lineto(784.67842041,61.47010061)
\curveto(784.67842043,61.39009952)(784.68342042,61.30009961)(784.69342041,61.20010061)
\curveto(784.69342041,61.10009981)(784.68342042,61.01509989)(784.66342041,60.94510061)
\curveto(784.59342051,60.73510017)(784.41342069,60.67010024)(784.12342041,60.75010061)
\curveto(784.08342102,60.76010015)(784.04842106,60.76010015)(784.01842041,60.75010061)
\curveto(783.97842113,60.75010016)(783.93342117,60.76010015)(783.88342041,60.78010061)
\curveto(783.8034213,60.80010011)(783.71842139,60.82010009)(783.62842041,60.84010061)
\curveto(783.53842157,60.86010005)(783.45342165,60.88510002)(783.37342041,60.91510061)
\curveto(782.88342222,61.07509983)(782.46842264,61.27509963)(782.12842041,61.51510061)
\curveto(781.87842323,61.69509921)(781.65342345,61.90009901)(781.45342041,62.13010061)
\curveto(781.24342386,62.36009855)(781.04842406,62.60009831)(780.86842041,62.85010061)
\curveto(780.68842442,63.1100978)(780.51842459,63.37509753)(780.35842041,63.64510061)
\curveto(780.18842492,63.92509698)(780.01342509,64.19509671)(779.83342041,64.45510061)
\curveto(779.75342535,64.56509634)(779.67842543,64.67009624)(779.60842041,64.77010061)
\curveto(779.53842557,64.88009603)(779.46342564,64.99009592)(779.38342041,65.10010061)
\curveto(779.35342575,65.14009577)(779.32342578,65.17509573)(779.29342041,65.20510061)
\curveto(779.25342585,65.24509566)(779.22342588,65.28509562)(779.20342041,65.32510061)
\curveto(779.09342601,65.46509544)(778.96842614,65.59009532)(778.82842041,65.70010061)
\curveto(778.79842631,65.72009519)(778.77342633,65.74509516)(778.75342041,65.77510061)
\curveto(778.72342638,65.8050951)(778.69342641,65.83009508)(778.66342041,65.85010061)
\curveto(778.56342654,65.93009498)(778.46342664,65.99509491)(778.36342041,66.04510061)
\curveto(778.26342684,66.1050948)(778.15342695,66.16009475)(778.03342041,66.21010061)
\curveto(777.96342714,66.24009467)(777.88842722,66.26009465)(777.80842041,66.27010061)
\lineto(777.56842041,66.33010061)
\lineto(777.47842041,66.33010061)
\curveto(777.44842766,66.34009457)(777.41842769,66.34509456)(777.38842041,66.34510061)
\curveto(777.31842779,66.36509454)(777.22342788,66.37009454)(777.10342041,66.36010061)
\curveto(776.97342813,66.36009455)(776.87342823,66.35009456)(776.80342041,66.33010061)
\curveto(776.72342838,66.3100946)(776.64842846,66.29009462)(776.57842041,66.27010061)
\curveto(776.49842861,66.26009465)(776.41842869,66.24009467)(776.33842041,66.21010061)
\curveto(776.09842901,66.10009481)(775.89842921,65.95009496)(775.73842041,65.76010061)
\curveto(775.56842954,65.58009533)(775.42842968,65.36009555)(775.31842041,65.10010061)
\curveto(775.29842981,65.03009588)(775.28342982,64.96009595)(775.27342041,64.89010061)
\curveto(775.25342985,64.82009609)(775.23342987,64.74509616)(775.21342041,64.66510061)
\curveto(775.19342991,64.58509632)(775.18342992,64.47509643)(775.18342041,64.33510061)
\curveto(775.18342992,64.2050967)(775.19342991,64.10009681)(775.21342041,64.02010061)
\curveto(775.22342988,63.96009695)(775.22842988,63.905097)(775.22842041,63.85510061)
\curveto(775.22842988,63.8050971)(775.23842987,63.75509715)(775.25842041,63.70510061)
\curveto(775.29842981,63.6050973)(775.33842977,63.5100974)(775.37842041,63.42010061)
\curveto(775.41842969,63.34009757)(775.46342964,63.26009765)(775.51342041,63.18010061)
\curveto(775.53342957,63.15009776)(775.55842955,63.12009779)(775.58842041,63.09010061)
\curveto(775.61842949,63.07009784)(775.64342946,63.04509786)(775.66342041,63.01510061)
\lineto(775.73842041,62.94010061)
\curveto(775.75842935,62.910098)(775.77842933,62.88509802)(775.79842041,62.86510061)
\lineto(776.00842041,62.71510061)
\curveto(776.06842904,62.67509823)(776.13342897,62.63009828)(776.20342041,62.58010061)
\curveto(776.29342881,62.52009839)(776.39842871,62.47009844)(776.51842041,62.43010061)
\curveto(776.62842848,62.40009851)(776.73842837,62.36509854)(776.84842041,62.32510061)
\curveto(776.95842815,62.28509862)(777.103428,62.26009865)(777.28342041,62.25010061)
\curveto(777.45342765,62.24009867)(777.57842753,62.2100987)(777.65842041,62.16010061)
\curveto(777.73842737,62.1100988)(777.78342732,62.03509887)(777.79342041,61.93510061)
\curveto(777.8034273,61.83509907)(777.8084273,61.72509918)(777.80842041,61.60510061)
\curveto(777.8084273,61.56509934)(777.81342729,61.52509938)(777.82342041,61.48510061)
\curveto(777.82342728,61.44509946)(777.81842729,61.4100995)(777.80842041,61.38010061)
\curveto(777.78842732,61.33009958)(777.77842733,61.28009963)(777.77842041,61.23010061)
\curveto(777.77842733,61.19009972)(777.76842734,61.15009976)(777.74842041,61.11010061)
\curveto(777.68842742,61.02009989)(777.55342755,60.97509993)(777.34342041,60.97510061)
\lineto(777.22342041,60.97510061)
\curveto(777.16342794,60.98509992)(777.103428,60.99009992)(777.04342041,60.99010061)
\curveto(776.97342813,61.00009991)(776.9084282,61.0100999)(776.84842041,61.02010061)
\curveto(776.73842837,61.04009987)(776.63842847,61.06009985)(776.54842041,61.08010061)
\curveto(776.44842866,61.10009981)(776.35342875,61.13009978)(776.26342041,61.17010061)
\curveto(776.19342891,61.19009972)(776.13342897,61.2100997)(776.08342041,61.23010061)
\lineto(775.90342041,61.29010061)
\curveto(775.64342946,61.4100995)(775.39842971,61.56509934)(775.16842041,61.75510061)
\curveto(774.93843017,61.95509895)(774.75343035,62.17009874)(774.61342041,62.40010061)
\curveto(774.53343057,62.5100984)(774.46843064,62.62509828)(774.41842041,62.74510061)
\lineto(774.26842041,63.13510061)
\curveto(774.21843089,63.24509766)(774.18843092,63.36009755)(774.17842041,63.48010061)
\curveto(774.15843095,63.60009731)(774.13343097,63.72509718)(774.10342041,63.85510061)
\curveto(774.103431,63.92509698)(774.103431,63.99009692)(774.10342041,64.05010061)
\curveto(774.09343101,64.1100968)(774.08343102,64.17509673)(774.07342041,64.24510061)
}
}
{
\newrgbcolor{curcolor}{0 0 0}
\pscustom[linestyle=none,fillstyle=solid,fillcolor=curcolor]
{
\newpath
\moveto(780.35842041,76.34470998)
\curveto(780.47842463,76.37470226)(780.61842449,76.39970223)(780.77842041,76.41970998)
\curveto(780.93842417,76.43970219)(781.103424,76.44970218)(781.27342041,76.44970998)
\curveto(781.44342366,76.44970218)(781.6084235,76.43970219)(781.76842041,76.41970998)
\curveto(781.92842318,76.39970223)(782.06842304,76.37470226)(782.18842041,76.34470998)
\curveto(782.32842278,76.30470233)(782.45342265,76.26970236)(782.56342041,76.23970998)
\curveto(782.67342243,76.20970242)(782.78342232,76.16970246)(782.89342041,76.11970998)
\curveto(783.53342157,75.84970278)(784.01842109,75.4347032)(784.34842041,74.87470998)
\curveto(784.4084207,74.79470384)(784.45842065,74.70970392)(784.49842041,74.61970998)
\curveto(784.52842058,74.5297041)(784.56342054,74.4297042)(784.60342041,74.31970998)
\curveto(784.65342045,74.20970442)(784.68842042,74.08970454)(784.70842041,73.95970998)
\curveto(784.73842037,73.83970479)(784.76842034,73.70970492)(784.79842041,73.56970998)
\curveto(784.81842029,73.50970512)(784.82342028,73.44970518)(784.81342041,73.38970998)
\curveto(784.8034203,73.33970529)(784.8084203,73.27970535)(784.82842041,73.20970998)
\curveto(784.83842027,73.18970544)(784.83842027,73.16470547)(784.82842041,73.13470998)
\curveto(784.82842028,73.10470553)(784.83342027,73.07970555)(784.84342041,73.05970998)
\lineto(784.84342041,72.90970998)
\curveto(784.85342025,72.83970579)(784.85342025,72.78970584)(784.84342041,72.75970998)
\curveto(784.83342027,72.71970591)(784.82842028,72.67470596)(784.82842041,72.62470998)
\curveto(784.83842027,72.58470605)(784.83842027,72.54470609)(784.82842041,72.50470998)
\curveto(784.8084203,72.41470622)(784.79342031,72.32470631)(784.78342041,72.23470998)
\curveto(784.78342032,72.14470649)(784.77342033,72.05470658)(784.75342041,71.96470998)
\curveto(784.72342038,71.87470676)(784.69842041,71.78470685)(784.67842041,71.69470998)
\curveto(784.65842045,71.60470703)(784.62842048,71.51970711)(784.58842041,71.43970998)
\curveto(784.47842063,71.19970743)(784.34842076,70.97470766)(784.19842041,70.76470998)
\curveto(784.03842107,70.55470808)(783.85842125,70.37470826)(783.65842041,70.22470998)
\curveto(783.48842162,70.10470853)(783.31342179,69.99970863)(783.13342041,69.90970998)
\curveto(782.95342215,69.81970881)(782.76342234,69.7297089)(782.56342041,69.63970998)
\curveto(782.46342264,69.59970903)(782.36342274,69.56470907)(782.26342041,69.53470998)
\curveto(782.15342295,69.51470912)(782.04342306,69.48970914)(781.93342041,69.45970998)
\curveto(781.79342331,69.41970921)(781.65342345,69.39470924)(781.51342041,69.38470998)
\curveto(781.37342373,69.37470926)(781.23342387,69.35470928)(781.09342041,69.32470998)
\curveto(780.98342412,69.31470932)(780.88342422,69.30470933)(780.79342041,69.29470998)
\curveto(780.69342441,69.29470934)(780.59342451,69.28470935)(780.49342041,69.26470998)
\lineto(780.40342041,69.26470998)
\curveto(780.37342473,69.27470936)(780.34842476,69.27470936)(780.32842041,69.26470998)
\lineto(780.11842041,69.26470998)
\curveto(780.05842505,69.24470939)(779.99342511,69.2347094)(779.92342041,69.23470998)
\curveto(779.84342526,69.24470939)(779.76842534,69.24970938)(779.69842041,69.24970998)
\lineto(779.54842041,69.24970998)
\curveto(779.49842561,69.24970938)(779.44842566,69.25470938)(779.39842041,69.26470998)
\lineto(779.02342041,69.26470998)
\curveto(778.99342611,69.27470936)(778.95842615,69.27470936)(778.91842041,69.26470998)
\curveto(778.87842623,69.26470937)(778.83842627,69.26970936)(778.79842041,69.27970998)
\curveto(778.68842642,69.29970933)(778.57842653,69.31470932)(778.46842041,69.32470998)
\curveto(778.34842676,69.3347093)(778.23342687,69.34470929)(778.12342041,69.35470998)
\curveto(777.97342713,69.39470924)(777.82842728,69.41970921)(777.68842041,69.42970998)
\curveto(777.53842757,69.44970918)(777.39342771,69.47970915)(777.25342041,69.51970998)
\curveto(776.95342815,69.60970902)(776.66842844,69.70470893)(776.39842041,69.80470998)
\curveto(776.12842898,69.90470873)(775.87842923,70.0297086)(775.64842041,70.17970998)
\curveto(775.32842978,70.37970825)(775.04843006,70.62470801)(774.80842041,70.91470998)
\curveto(774.56843054,71.20470743)(774.38343072,71.54470709)(774.25342041,71.93470998)
\curveto(774.21343089,72.04470659)(774.18843092,72.15470648)(774.17842041,72.26470998)
\curveto(774.15843095,72.38470625)(774.13343097,72.50470613)(774.10342041,72.62470998)
\curveto(774.09343101,72.69470594)(774.08843102,72.75970587)(774.08842041,72.81970998)
\curveto(774.08843102,72.87970575)(774.08343102,72.94470569)(774.07342041,73.01470998)
\curveto(774.05343105,73.71470492)(774.16843094,74.28970434)(774.41842041,74.73970998)
\curveto(774.66843044,75.18970344)(775.01843009,75.5347031)(775.46842041,75.77470998)
\curveto(775.69842941,75.88470275)(775.97342913,75.98470265)(776.29342041,76.07470998)
\curveto(776.36342874,76.09470254)(776.43842867,76.09470254)(776.51842041,76.07470998)
\curveto(776.58842852,76.06470257)(776.63842847,76.03970259)(776.66842041,75.99970998)
\curveto(776.69842841,75.96970266)(776.72342838,75.90970272)(776.74342041,75.81970998)
\curveto(776.75342835,75.7297029)(776.76342834,75.629703)(776.77342041,75.51970998)
\curveto(776.77342833,75.41970321)(776.76842834,75.31970331)(776.75842041,75.21970998)
\curveto(776.74842836,75.1297035)(776.72842838,75.06470357)(776.69842041,75.02470998)
\curveto(776.62842848,74.91470372)(776.51842859,74.8347038)(776.36842041,74.78470998)
\curveto(776.21842889,74.74470389)(776.08842902,74.68970394)(775.97842041,74.61970998)
\curveto(775.66842944,74.4297042)(775.43842967,74.14970448)(775.28842041,73.77970998)
\curveto(775.25842985,73.70970492)(775.23842987,73.634705)(775.22842041,73.55470998)
\curveto(775.21842989,73.48470515)(775.2034299,73.40970522)(775.18342041,73.32970998)
\curveto(775.17342993,73.27970535)(775.16842994,73.20970542)(775.16842041,73.11970998)
\curveto(775.16842994,73.03970559)(775.17342993,72.97470566)(775.18342041,72.92470998)
\curveto(775.2034299,72.88470575)(775.2084299,72.84970578)(775.19842041,72.81970998)
\curveto(775.18842992,72.78970584)(775.18842992,72.75470588)(775.19842041,72.71470998)
\lineto(775.25842041,72.47470998)
\curveto(775.27842983,72.40470623)(775.3034298,72.3347063)(775.33342041,72.26470998)
\curveto(775.49342961,71.88470675)(775.7034294,71.59470704)(775.96342041,71.39470998)
\curveto(776.22342888,71.20470743)(776.53842857,71.0297076)(776.90842041,70.86970998)
\curveto(776.98842812,70.83970779)(777.06842804,70.81470782)(777.14842041,70.79470998)
\curveto(777.22842788,70.78470785)(777.3084278,70.76470787)(777.38842041,70.73470998)
\curveto(777.49842761,70.70470793)(777.61342749,70.67970795)(777.73342041,70.65970998)
\curveto(777.85342725,70.64970798)(777.97342713,70.629708)(778.09342041,70.59970998)
\curveto(778.14342696,70.57970805)(778.19342691,70.56970806)(778.24342041,70.56970998)
\curveto(778.29342681,70.57970805)(778.34342676,70.57470806)(778.39342041,70.55470998)
\curveto(778.45342665,70.54470809)(778.53342657,70.54470809)(778.63342041,70.55470998)
\curveto(778.72342638,70.56470807)(778.77842633,70.57970805)(778.79842041,70.59970998)
\curveto(778.83842627,70.61970801)(778.85842625,70.64970798)(778.85842041,70.68970998)
\curveto(778.85842625,70.73970789)(778.84842626,70.78470785)(778.82842041,70.82470998)
\curveto(778.78842632,70.89470774)(778.74342636,70.95470768)(778.69342041,71.00470998)
\curveto(778.64342646,71.05470758)(778.59342651,71.11470752)(778.54342041,71.18470998)
\lineto(778.48342041,71.24470998)
\curveto(778.45342665,71.27470736)(778.42842668,71.30470733)(778.40842041,71.33470998)
\curveto(778.24842686,71.56470707)(778.11342699,71.83970679)(778.00342041,72.15970998)
\curveto(777.98342712,72.2297064)(777.96842714,72.29970633)(777.95842041,72.36970998)
\curveto(777.94842716,72.43970619)(777.93342717,72.51470612)(777.91342041,72.59470998)
\curveto(777.91342719,72.634706)(777.9084272,72.66970596)(777.89842041,72.69970998)
\curveto(777.88842722,72.7297059)(777.88842722,72.76470587)(777.89842041,72.80470998)
\curveto(777.89842721,72.85470578)(777.88842722,72.89470574)(777.86842041,72.92470998)
\lineto(777.86842041,73.08970998)
\lineto(777.86842041,73.17970998)
\curveto(777.85842725,73.2297054)(777.85842725,73.26970536)(777.86842041,73.29970998)
\curveto(777.87842723,73.34970528)(777.88342722,73.39970523)(777.88342041,73.44970998)
\curveto(777.87342723,73.50970512)(777.87342723,73.56470507)(777.88342041,73.61470998)
\curveto(777.91342719,73.72470491)(777.93342717,73.8297048)(777.94342041,73.92970998)
\curveto(777.95342715,74.03970459)(777.97842713,74.14470449)(778.01842041,74.24470998)
\curveto(778.15842695,74.66470397)(778.34342676,75.00970362)(778.57342041,75.27970998)
\curveto(778.79342631,75.54970308)(779.07842603,75.78970284)(779.42842041,75.99970998)
\curveto(779.56842554,76.07970255)(779.71842539,76.14470249)(779.87842041,76.19470998)
\curveto(780.02842508,76.24470239)(780.18842492,76.29470234)(780.35842041,76.34470998)
\moveto(781.66342041,75.09970998)
\curveto(781.61342349,75.10970352)(781.56842354,75.11470352)(781.52842041,75.11470998)
\lineto(781.37842041,75.11470998)
\curveto(781.06842404,75.11470352)(780.78342432,75.07470356)(780.52342041,74.99470998)
\curveto(780.46342464,74.97470366)(780.4084247,74.95470368)(780.35842041,74.93470998)
\curveto(780.29842481,74.92470371)(780.24342486,74.90970372)(780.19342041,74.88970998)
\curveto(779.7034254,74.66970396)(779.35342575,74.32470431)(779.14342041,73.85470998)
\curveto(779.11342599,73.77470486)(779.08842602,73.69470494)(779.06842041,73.61470998)
\lineto(779.00842041,73.37470998)
\curveto(778.98842612,73.29470534)(778.97842613,73.20470543)(778.97842041,73.10470998)
\lineto(778.97842041,72.78970998)
\curveto(778.99842611,72.76970586)(779.0084261,72.7297059)(779.00842041,72.66970998)
\curveto(778.99842611,72.61970601)(778.99842611,72.57470606)(779.00842041,72.53470998)
\lineto(779.06842041,72.29470998)
\curveto(779.07842603,72.22470641)(779.09842601,72.15470648)(779.12842041,72.08470998)
\curveto(779.38842572,71.48470715)(779.85342525,71.07970755)(780.52342041,70.86970998)
\curveto(780.6034245,70.83970779)(780.68342442,70.81970781)(780.76342041,70.80970998)
\curveto(780.84342426,70.79970783)(780.92842418,70.78470785)(781.01842041,70.76470998)
\lineto(781.16842041,70.76470998)
\curveto(781.2084239,70.75470788)(781.27842383,70.74970788)(781.37842041,70.74970998)
\curveto(781.6084235,70.74970788)(781.8034233,70.76970786)(781.96342041,70.80970998)
\curveto(782.03342307,70.8297078)(782.09842301,70.84470779)(782.15842041,70.85470998)
\curveto(782.21842289,70.86470777)(782.28342282,70.88470775)(782.35342041,70.91470998)
\curveto(782.63342247,71.02470761)(782.87842223,71.16970746)(783.08842041,71.34970998)
\curveto(783.28842182,71.5297071)(783.44842166,71.76470687)(783.56842041,72.05470998)
\lineto(783.65842041,72.29470998)
\lineto(783.71842041,72.53470998)
\curveto(783.73842137,72.58470605)(783.74342136,72.62470601)(783.73342041,72.65470998)
\curveto(783.72342138,72.69470594)(783.72842138,72.73970589)(783.74842041,72.78970998)
\curveto(783.75842135,72.81970581)(783.76342134,72.87470576)(783.76342041,72.95470998)
\curveto(783.76342134,73.0347056)(783.75842135,73.09470554)(783.74842041,73.13470998)
\curveto(783.72842138,73.24470539)(783.71342139,73.34970528)(783.70342041,73.44970998)
\curveto(783.69342141,73.54970508)(783.66342144,73.64470499)(783.61342041,73.73470998)
\curveto(783.41342169,74.26470437)(783.03842207,74.65470398)(782.48842041,74.90470998)
\curveto(782.38842272,74.94470369)(782.28342282,74.97470366)(782.17342041,74.99470998)
\lineto(781.84342041,75.08470998)
\curveto(781.76342334,75.08470355)(781.7034234,75.08970354)(781.66342041,75.09970998)
}
}
{
\newrgbcolor{curcolor}{0 0 0}
\pscustom[linestyle=none,fillstyle=solid,fillcolor=curcolor]
{
\newpath
\moveto(783.04342041,78.63431936)
\lineto(783.04342041,79.26431936)
\lineto(783.04342041,79.45931936)
\curveto(783.04342206,79.52931683)(783.05342205,79.58931677)(783.07342041,79.63931936)
\curveto(783.11342199,79.70931665)(783.15342195,79.7593166)(783.19342041,79.78931936)
\curveto(783.24342186,79.82931653)(783.3084218,79.84931651)(783.38842041,79.84931936)
\curveto(783.46842164,79.8593165)(783.55342155,79.86431649)(783.64342041,79.86431936)
\lineto(784.36342041,79.86431936)
\curveto(784.84342026,79.86431649)(785.25341985,79.80431655)(785.59342041,79.68431936)
\curveto(785.93341917,79.56431679)(786.2084189,79.36931699)(786.41842041,79.09931936)
\curveto(786.46841864,79.02931733)(786.51341859,78.9593174)(786.55342041,78.88931936)
\curveto(786.6034185,78.82931753)(786.64841846,78.7543176)(786.68842041,78.66431936)
\curveto(786.69841841,78.64431771)(786.7084184,78.61431774)(786.71842041,78.57431936)
\curveto(786.73841837,78.53431782)(786.74341836,78.48931787)(786.73342041,78.43931936)
\curveto(786.7034184,78.34931801)(786.62841848,78.29431806)(786.50842041,78.27431936)
\curveto(786.39841871,78.2543181)(786.3034188,78.26931809)(786.22342041,78.31931936)
\curveto(786.15341895,78.34931801)(786.08841902,78.39431796)(786.02842041,78.45431936)
\curveto(785.97841913,78.52431783)(785.92841918,78.58931777)(785.87842041,78.64931936)
\curveto(785.82841928,78.71931764)(785.75341935,78.77931758)(785.65342041,78.82931936)
\curveto(785.56341954,78.88931747)(785.47341963,78.93931742)(785.38342041,78.97931936)
\curveto(785.35341975,78.99931736)(785.29341981,79.02431733)(785.20342041,79.05431936)
\curveto(785.12341998,79.08431727)(785.05342005,79.08931727)(784.99342041,79.06931936)
\curveto(784.85342025,79.03931732)(784.76342034,78.97931738)(784.72342041,78.88931936)
\curveto(784.69342041,78.80931755)(784.67842043,78.71931764)(784.67842041,78.61931936)
\curveto(784.67842043,78.51931784)(784.65342045,78.43431792)(784.60342041,78.36431936)
\curveto(784.53342057,78.27431808)(784.39342071,78.22931813)(784.18342041,78.22931936)
\lineto(783.62842041,78.22931936)
\lineto(783.40342041,78.22931936)
\curveto(783.32342178,78.23931812)(783.25842185,78.2593181)(783.20842041,78.28931936)
\curveto(783.12842198,78.34931801)(783.08342202,78.41931794)(783.07342041,78.49931936)
\curveto(783.06342204,78.51931784)(783.05842205,78.53931782)(783.05842041,78.55931936)
\curveto(783.05842205,78.58931777)(783.05342205,78.61431774)(783.04342041,78.63431936)
}
}
{
\newrgbcolor{curcolor}{0 0 0}
\pscustom[linestyle=none,fillstyle=solid,fillcolor=curcolor]
{
}
}
{
\newrgbcolor{curcolor}{0 0 0}
\pscustom[linestyle=none,fillstyle=solid,fillcolor=curcolor]
{
\newpath
\moveto(774.07342041,89.26463186)
\curveto(774.06343104,89.95462722)(774.18343092,90.55462662)(774.43342041,91.06463186)
\curveto(774.68343042,91.58462559)(775.01843009,91.9796252)(775.43842041,92.24963186)
\curveto(775.51842959,92.29962488)(775.6084295,92.34462483)(775.70842041,92.38463186)
\curveto(775.79842931,92.42462475)(775.89342921,92.46962471)(775.99342041,92.51963186)
\curveto(776.09342901,92.55962462)(776.19342891,92.58962459)(776.29342041,92.60963186)
\curveto(776.39342871,92.62962455)(776.49842861,92.64962453)(776.60842041,92.66963186)
\curveto(776.65842845,92.68962449)(776.7034284,92.69462448)(776.74342041,92.68463186)
\curveto(776.78342832,92.6746245)(776.82842828,92.6796245)(776.87842041,92.69963186)
\curveto(776.92842818,92.70962447)(777.01342809,92.71462446)(777.13342041,92.71463186)
\curveto(777.24342786,92.71462446)(777.32842778,92.70962447)(777.38842041,92.69963186)
\curveto(777.44842766,92.6796245)(777.5084276,92.66962451)(777.56842041,92.66963186)
\curveto(777.62842748,92.6796245)(777.68842742,92.6746245)(777.74842041,92.65463186)
\curveto(777.88842722,92.61462456)(778.02342708,92.5796246)(778.15342041,92.54963186)
\curveto(778.28342682,92.51962466)(778.4084267,92.4796247)(778.52842041,92.42963186)
\curveto(778.66842644,92.36962481)(778.79342631,92.29962488)(778.90342041,92.21963186)
\curveto(779.01342609,92.14962503)(779.12342598,92.0746251)(779.23342041,91.99463186)
\lineto(779.29342041,91.93463186)
\curveto(779.31342579,91.92462525)(779.33342577,91.90962527)(779.35342041,91.88963186)
\curveto(779.51342559,91.76962541)(779.65842545,91.63462554)(779.78842041,91.48463186)
\curveto(779.91842519,91.33462584)(780.04342506,91.174626)(780.16342041,91.00463186)
\curveto(780.38342472,90.69462648)(780.58842452,90.39962678)(780.77842041,90.11963186)
\curveto(780.91842419,89.88962729)(781.05342405,89.65962752)(781.18342041,89.42963186)
\curveto(781.31342379,89.20962797)(781.44842366,88.98962819)(781.58842041,88.76963186)
\curveto(781.75842335,88.51962866)(781.93842317,88.2796289)(782.12842041,88.04963186)
\curveto(782.31842279,87.82962935)(782.54342256,87.63962954)(782.80342041,87.47963186)
\curveto(782.86342224,87.43962974)(782.92342218,87.40462977)(782.98342041,87.37463186)
\curveto(783.03342207,87.34462983)(783.09842201,87.31462986)(783.17842041,87.28463186)
\curveto(783.24842186,87.26462991)(783.3084218,87.25962992)(783.35842041,87.26963186)
\curveto(783.42842168,87.28962989)(783.48342162,87.32462985)(783.52342041,87.37463186)
\curveto(783.55342155,87.42462975)(783.57342153,87.48462969)(783.58342041,87.55463186)
\lineto(783.58342041,87.79463186)
\lineto(783.58342041,88.54463186)
\lineto(783.58342041,91.34963186)
\lineto(783.58342041,92.00963186)
\curveto(783.58342152,92.09962508)(783.58842152,92.18462499)(783.59842041,92.26463186)
\curveto(783.59842151,92.34462483)(783.61842149,92.40962477)(783.65842041,92.45963186)
\curveto(783.69842141,92.50962467)(783.77342133,92.54962463)(783.88342041,92.57963186)
\curveto(783.98342112,92.61962456)(784.08342102,92.62962455)(784.18342041,92.60963186)
\lineto(784.31842041,92.60963186)
\curveto(784.38842072,92.58962459)(784.44842066,92.56962461)(784.49842041,92.54963186)
\curveto(784.54842056,92.52962465)(784.58842052,92.49462468)(784.61842041,92.44463186)
\curveto(784.65842045,92.39462478)(784.67842043,92.32462485)(784.67842041,92.23463186)
\lineto(784.67842041,91.96463186)
\lineto(784.67842041,91.06463186)
\lineto(784.67842041,87.55463186)
\lineto(784.67842041,86.48963186)
\curveto(784.67842043,86.40963077)(784.68342042,86.31963086)(784.69342041,86.21963186)
\curveto(784.69342041,86.11963106)(784.68342042,86.03463114)(784.66342041,85.96463186)
\curveto(784.59342051,85.75463142)(784.41342069,85.68963149)(784.12342041,85.76963186)
\curveto(784.08342102,85.7796314)(784.04842106,85.7796314)(784.01842041,85.76963186)
\curveto(783.97842113,85.76963141)(783.93342117,85.7796314)(783.88342041,85.79963186)
\curveto(783.8034213,85.81963136)(783.71842139,85.83963134)(783.62842041,85.85963186)
\curveto(783.53842157,85.8796313)(783.45342165,85.90463127)(783.37342041,85.93463186)
\curveto(782.88342222,86.09463108)(782.46842264,86.29463088)(782.12842041,86.53463186)
\curveto(781.87842323,86.71463046)(781.65342345,86.91963026)(781.45342041,87.14963186)
\curveto(781.24342386,87.3796298)(781.04842406,87.61962956)(780.86842041,87.86963186)
\curveto(780.68842442,88.12962905)(780.51842459,88.39462878)(780.35842041,88.66463186)
\curveto(780.18842492,88.94462823)(780.01342509,89.21462796)(779.83342041,89.47463186)
\curveto(779.75342535,89.58462759)(779.67842543,89.68962749)(779.60842041,89.78963186)
\curveto(779.53842557,89.89962728)(779.46342564,90.00962717)(779.38342041,90.11963186)
\curveto(779.35342575,90.15962702)(779.32342578,90.19462698)(779.29342041,90.22463186)
\curveto(779.25342585,90.26462691)(779.22342588,90.30462687)(779.20342041,90.34463186)
\curveto(779.09342601,90.48462669)(778.96842614,90.60962657)(778.82842041,90.71963186)
\curveto(778.79842631,90.73962644)(778.77342633,90.76462641)(778.75342041,90.79463186)
\curveto(778.72342638,90.82462635)(778.69342641,90.84962633)(778.66342041,90.86963186)
\curveto(778.56342654,90.94962623)(778.46342664,91.01462616)(778.36342041,91.06463186)
\curveto(778.26342684,91.12462605)(778.15342695,91.179626)(778.03342041,91.22963186)
\curveto(777.96342714,91.25962592)(777.88842722,91.2796259)(777.80842041,91.28963186)
\lineto(777.56842041,91.34963186)
\lineto(777.47842041,91.34963186)
\curveto(777.44842766,91.35962582)(777.41842769,91.36462581)(777.38842041,91.36463186)
\curveto(777.31842779,91.38462579)(777.22342788,91.38962579)(777.10342041,91.37963186)
\curveto(776.97342813,91.3796258)(776.87342823,91.36962581)(776.80342041,91.34963186)
\curveto(776.72342838,91.32962585)(776.64842846,91.30962587)(776.57842041,91.28963186)
\curveto(776.49842861,91.2796259)(776.41842869,91.25962592)(776.33842041,91.22963186)
\curveto(776.09842901,91.11962606)(775.89842921,90.96962621)(775.73842041,90.77963186)
\curveto(775.56842954,90.59962658)(775.42842968,90.3796268)(775.31842041,90.11963186)
\curveto(775.29842981,90.04962713)(775.28342982,89.9796272)(775.27342041,89.90963186)
\curveto(775.25342985,89.83962734)(775.23342987,89.76462741)(775.21342041,89.68463186)
\curveto(775.19342991,89.60462757)(775.18342992,89.49462768)(775.18342041,89.35463186)
\curveto(775.18342992,89.22462795)(775.19342991,89.11962806)(775.21342041,89.03963186)
\curveto(775.22342988,88.9796282)(775.22842988,88.92462825)(775.22842041,88.87463186)
\curveto(775.22842988,88.82462835)(775.23842987,88.7746284)(775.25842041,88.72463186)
\curveto(775.29842981,88.62462855)(775.33842977,88.52962865)(775.37842041,88.43963186)
\curveto(775.41842969,88.35962882)(775.46342964,88.2796289)(775.51342041,88.19963186)
\curveto(775.53342957,88.16962901)(775.55842955,88.13962904)(775.58842041,88.10963186)
\curveto(775.61842949,88.08962909)(775.64342946,88.06462911)(775.66342041,88.03463186)
\lineto(775.73842041,87.95963186)
\curveto(775.75842935,87.92962925)(775.77842933,87.90462927)(775.79842041,87.88463186)
\lineto(776.00842041,87.73463186)
\curveto(776.06842904,87.69462948)(776.13342897,87.64962953)(776.20342041,87.59963186)
\curveto(776.29342881,87.53962964)(776.39842871,87.48962969)(776.51842041,87.44963186)
\curveto(776.62842848,87.41962976)(776.73842837,87.38462979)(776.84842041,87.34463186)
\curveto(776.95842815,87.30462987)(777.103428,87.2796299)(777.28342041,87.26963186)
\curveto(777.45342765,87.25962992)(777.57842753,87.22962995)(777.65842041,87.17963186)
\curveto(777.73842737,87.12963005)(777.78342732,87.05463012)(777.79342041,86.95463186)
\curveto(777.8034273,86.85463032)(777.8084273,86.74463043)(777.80842041,86.62463186)
\curveto(777.8084273,86.58463059)(777.81342729,86.54463063)(777.82342041,86.50463186)
\curveto(777.82342728,86.46463071)(777.81842729,86.42963075)(777.80842041,86.39963186)
\curveto(777.78842732,86.34963083)(777.77842733,86.29963088)(777.77842041,86.24963186)
\curveto(777.77842733,86.20963097)(777.76842734,86.16963101)(777.74842041,86.12963186)
\curveto(777.68842742,86.03963114)(777.55342755,85.99463118)(777.34342041,85.99463186)
\lineto(777.22342041,85.99463186)
\curveto(777.16342794,86.00463117)(777.103428,86.00963117)(777.04342041,86.00963186)
\curveto(776.97342813,86.01963116)(776.9084282,86.02963115)(776.84842041,86.03963186)
\curveto(776.73842837,86.05963112)(776.63842847,86.0796311)(776.54842041,86.09963186)
\curveto(776.44842866,86.11963106)(776.35342875,86.14963103)(776.26342041,86.18963186)
\curveto(776.19342891,86.20963097)(776.13342897,86.22963095)(776.08342041,86.24963186)
\lineto(775.90342041,86.30963186)
\curveto(775.64342946,86.42963075)(775.39842971,86.58463059)(775.16842041,86.77463186)
\curveto(774.93843017,86.9746302)(774.75343035,87.18962999)(774.61342041,87.41963186)
\curveto(774.53343057,87.52962965)(774.46843064,87.64462953)(774.41842041,87.76463186)
\lineto(774.26842041,88.15463186)
\curveto(774.21843089,88.26462891)(774.18843092,88.3796288)(774.17842041,88.49963186)
\curveto(774.15843095,88.61962856)(774.13343097,88.74462843)(774.10342041,88.87463186)
\curveto(774.103431,88.94462823)(774.103431,89.00962817)(774.10342041,89.06963186)
\curveto(774.09343101,89.12962805)(774.08343102,89.19462798)(774.07342041,89.26463186)
}
}
{
\newrgbcolor{curcolor}{0 0 0}
\pscustom[linestyle=none,fillstyle=solid,fillcolor=curcolor]
{
\newpath
\moveto(779.59342041,101.36424123)
\lineto(779.84842041,101.36424123)
\curveto(779.92842518,101.37423353)(780.0034251,101.36923353)(780.07342041,101.34924123)
\lineto(780.31342041,101.34924123)
\lineto(780.47842041,101.34924123)
\curveto(780.57842453,101.32923357)(780.68342442,101.31923358)(780.79342041,101.31924123)
\curveto(780.89342421,101.31923358)(780.99342411,101.30923359)(781.09342041,101.28924123)
\lineto(781.24342041,101.28924123)
\curveto(781.38342372,101.25923364)(781.52342358,101.23923366)(781.66342041,101.22924123)
\curveto(781.79342331,101.21923368)(781.92342318,101.19423371)(782.05342041,101.15424123)
\curveto(782.13342297,101.13423377)(782.21842289,101.11423379)(782.30842041,101.09424123)
\lineto(782.54842041,101.03424123)
\lineto(782.84842041,100.91424123)
\curveto(782.93842217,100.88423402)(783.02842208,100.84923405)(783.11842041,100.80924123)
\curveto(783.33842177,100.70923419)(783.55342155,100.57423433)(783.76342041,100.40424123)
\curveto(783.97342113,100.24423466)(784.14342096,100.06923483)(784.27342041,99.87924123)
\curveto(784.31342079,99.82923507)(784.35342075,99.76923513)(784.39342041,99.69924123)
\curveto(784.42342068,99.63923526)(784.45842065,99.57923532)(784.49842041,99.51924123)
\curveto(784.54842056,99.43923546)(784.58842052,99.34423556)(784.61842041,99.23424123)
\curveto(784.64842046,99.12423578)(784.67842043,99.01923588)(784.70842041,98.91924123)
\curveto(784.74842036,98.80923609)(784.77342033,98.6992362)(784.78342041,98.58924123)
\curveto(784.79342031,98.47923642)(784.8084203,98.36423654)(784.82842041,98.24424123)
\curveto(784.83842027,98.2042367)(784.83842027,98.15923674)(784.82842041,98.10924123)
\curveto(784.82842028,98.06923683)(784.83342027,98.02923687)(784.84342041,97.98924123)
\curveto(784.85342025,97.94923695)(784.85842025,97.89423701)(784.85842041,97.82424123)
\curveto(784.85842025,97.75423715)(784.85342025,97.7042372)(784.84342041,97.67424123)
\curveto(784.82342028,97.62423728)(784.81842029,97.57923732)(784.82842041,97.53924123)
\curveto(784.83842027,97.4992374)(784.83842027,97.46423744)(784.82842041,97.43424123)
\lineto(784.82842041,97.34424123)
\curveto(784.8084203,97.28423762)(784.79342031,97.21923768)(784.78342041,97.14924123)
\curveto(784.78342032,97.08923781)(784.77842033,97.02423788)(784.76842041,96.95424123)
\curveto(784.71842039,96.78423812)(784.66842044,96.62423828)(784.61842041,96.47424123)
\curveto(784.56842054,96.32423858)(784.5034206,96.17923872)(784.42342041,96.03924123)
\curveto(784.38342072,95.98923891)(784.35342075,95.93423897)(784.33342041,95.87424123)
\curveto(784.3034208,95.82423908)(784.26842084,95.77423913)(784.22842041,95.72424123)
\curveto(784.04842106,95.48423942)(783.82842128,95.28423962)(783.56842041,95.12424123)
\curveto(783.3084218,94.96423994)(783.02342208,94.82424008)(782.71342041,94.70424123)
\curveto(782.57342253,94.64424026)(782.43342267,94.5992403)(782.29342041,94.56924123)
\curveto(782.14342296,94.53924036)(781.98842312,94.5042404)(781.82842041,94.46424123)
\curveto(781.71842339,94.44424046)(781.6084235,94.42924047)(781.49842041,94.41924123)
\curveto(781.38842372,94.40924049)(781.27842383,94.39424051)(781.16842041,94.37424123)
\curveto(781.12842398,94.36424054)(781.08842402,94.35924054)(781.04842041,94.35924123)
\curveto(781.0084241,94.36924053)(780.96842414,94.36924053)(780.92842041,94.35924123)
\curveto(780.87842423,94.34924055)(780.82842428,94.34424056)(780.77842041,94.34424123)
\lineto(780.61342041,94.34424123)
\curveto(780.56342454,94.32424058)(780.51342459,94.31924058)(780.46342041,94.32924123)
\curveto(780.4034247,94.33924056)(780.34842476,94.33924056)(780.29842041,94.32924123)
\curveto(780.25842485,94.31924058)(780.21342489,94.31924058)(780.16342041,94.32924123)
\curveto(780.11342499,94.33924056)(780.06342504,94.33424057)(780.01342041,94.31424123)
\curveto(779.94342516,94.29424061)(779.86842524,94.28924061)(779.78842041,94.29924123)
\curveto(779.69842541,94.30924059)(779.61342549,94.31424059)(779.53342041,94.31424123)
\curveto(779.44342566,94.31424059)(779.34342576,94.30924059)(779.23342041,94.29924123)
\curveto(779.11342599,94.28924061)(779.01342609,94.29424061)(778.93342041,94.31424123)
\lineto(778.64842041,94.31424123)
\lineto(778.01842041,94.35924123)
\curveto(777.91842719,94.36924053)(777.82342728,94.37924052)(777.73342041,94.38924123)
\lineto(777.43342041,94.41924123)
\curveto(777.38342772,94.43924046)(777.33342777,94.44424046)(777.28342041,94.43424123)
\curveto(777.22342788,94.43424047)(777.16842794,94.44424046)(777.11842041,94.46424123)
\curveto(776.94842816,94.51424039)(776.78342832,94.55424035)(776.62342041,94.58424123)
\curveto(776.45342865,94.61424029)(776.29342881,94.66424024)(776.14342041,94.73424123)
\curveto(775.68342942,94.92423998)(775.3084298,95.14423976)(775.01842041,95.39424123)
\curveto(774.72843038,95.65423925)(774.48343062,96.01423889)(774.28342041,96.47424123)
\curveto(774.23343087,96.6042383)(774.19843091,96.73423817)(774.17842041,96.86424123)
\curveto(774.15843095,97.0042379)(774.13343097,97.14423776)(774.10342041,97.28424123)
\curveto(774.09343101,97.35423755)(774.08843102,97.41923748)(774.08842041,97.47924123)
\curveto(774.08843102,97.53923736)(774.08343102,97.6042373)(774.07342041,97.67424123)
\curveto(774.05343105,98.5042364)(774.2034309,99.17423573)(774.52342041,99.68424123)
\curveto(774.83343027,100.19423471)(775.27342983,100.57423433)(775.84342041,100.82424123)
\curveto(775.96342914,100.87423403)(776.08842902,100.91923398)(776.21842041,100.95924123)
\curveto(776.34842876,100.9992339)(776.48342862,101.04423386)(776.62342041,101.09424123)
\curveto(776.7034284,101.11423379)(776.78842832,101.12923377)(776.87842041,101.13924123)
\lineto(777.11842041,101.19924123)
\curveto(777.22842788,101.22923367)(777.33842777,101.24423366)(777.44842041,101.24424123)
\curveto(777.55842755,101.25423365)(777.66842744,101.26923363)(777.77842041,101.28924123)
\curveto(777.82842728,101.30923359)(777.87342723,101.31423359)(777.91342041,101.30424123)
\curveto(777.95342715,101.3042336)(777.99342711,101.30923359)(778.03342041,101.31924123)
\curveto(778.08342702,101.32923357)(778.13842697,101.32923357)(778.19842041,101.31924123)
\curveto(778.24842686,101.31923358)(778.29842681,101.32423358)(778.34842041,101.33424123)
\lineto(778.48342041,101.33424123)
\curveto(778.54342656,101.35423355)(778.61342649,101.35423355)(778.69342041,101.33424123)
\curveto(778.76342634,101.32423358)(778.82842628,101.32923357)(778.88842041,101.34924123)
\curveto(778.91842619,101.35923354)(778.95842615,101.36423354)(779.00842041,101.36424123)
\lineto(779.12842041,101.36424123)
\lineto(779.59342041,101.36424123)
\moveto(781.91842041,99.81924123)
\curveto(781.59842351,99.91923498)(781.23342387,99.97923492)(780.82342041,99.99924123)
\curveto(780.41342469,100.01923488)(780.0034251,100.02923487)(779.59342041,100.02924123)
\curveto(779.16342594,100.02923487)(778.74342636,100.01923488)(778.33342041,99.99924123)
\curveto(777.92342718,99.97923492)(777.53842757,99.93423497)(777.17842041,99.86424123)
\curveto(776.81842829,99.79423511)(776.49842861,99.68423522)(776.21842041,99.53424123)
\curveto(775.92842918,99.39423551)(775.69342941,99.1992357)(775.51342041,98.94924123)
\curveto(775.4034297,98.78923611)(775.32342978,98.60923629)(775.27342041,98.40924123)
\curveto(775.21342989,98.20923669)(775.18342992,97.96423694)(775.18342041,97.67424123)
\curveto(775.2034299,97.65423725)(775.21342989,97.61923728)(775.21342041,97.56924123)
\curveto(775.2034299,97.51923738)(775.2034299,97.47923742)(775.21342041,97.44924123)
\curveto(775.23342987,97.36923753)(775.25342985,97.29423761)(775.27342041,97.22424123)
\curveto(775.28342982,97.16423774)(775.3034298,97.0992378)(775.33342041,97.02924123)
\curveto(775.45342965,96.75923814)(775.62342948,96.53923836)(775.84342041,96.36924123)
\curveto(776.05342905,96.20923869)(776.29842881,96.07423883)(776.57842041,95.96424123)
\curveto(776.68842842,95.91423899)(776.8084283,95.87423903)(776.93842041,95.84424123)
\curveto(777.05842805,95.82423908)(777.18342792,95.7992391)(777.31342041,95.76924123)
\curveto(777.36342774,95.74923915)(777.41842769,95.73923916)(777.47842041,95.73924123)
\curveto(777.52842758,95.73923916)(777.57842753,95.73423917)(777.62842041,95.72424123)
\curveto(777.71842739,95.71423919)(777.81342729,95.7042392)(777.91342041,95.69424123)
\curveto(778.0034271,95.68423922)(778.09842701,95.67423923)(778.19842041,95.66424123)
\curveto(778.27842683,95.66423924)(778.36342674,95.65923924)(778.45342041,95.64924123)
\lineto(778.69342041,95.64924123)
\lineto(778.87342041,95.64924123)
\curveto(778.9034262,95.63923926)(778.93842617,95.63423927)(778.97842041,95.63424123)
\lineto(779.11342041,95.63424123)
\lineto(779.56342041,95.63424123)
\curveto(779.64342546,95.63423927)(779.72842538,95.62923927)(779.81842041,95.61924123)
\curveto(779.89842521,95.61923928)(779.97342513,95.62923927)(780.04342041,95.64924123)
\lineto(780.31342041,95.64924123)
\curveto(780.33342477,95.64923925)(780.36342474,95.64423926)(780.40342041,95.63424123)
\curveto(780.43342467,95.63423927)(780.45842465,95.63923926)(780.47842041,95.64924123)
\curveto(780.57842453,95.65923924)(780.67842443,95.66423924)(780.77842041,95.66424123)
\curveto(780.86842424,95.67423923)(780.96842414,95.68423922)(781.07842041,95.69424123)
\curveto(781.19842391,95.72423918)(781.32342378,95.73923916)(781.45342041,95.73924123)
\curveto(781.57342353,95.74923915)(781.68842342,95.77423913)(781.79842041,95.81424123)
\curveto(782.09842301,95.89423901)(782.36342274,95.97923892)(782.59342041,96.06924123)
\curveto(782.82342228,96.16923873)(783.03842207,96.31423859)(783.23842041,96.50424123)
\curveto(783.43842167,96.71423819)(783.58842152,96.97923792)(783.68842041,97.29924123)
\curveto(783.7084214,97.33923756)(783.71842139,97.37423753)(783.71842041,97.40424123)
\curveto(783.7084214,97.44423746)(783.71342139,97.48923741)(783.73342041,97.53924123)
\curveto(783.74342136,97.57923732)(783.75342135,97.64923725)(783.76342041,97.74924123)
\curveto(783.77342133,97.85923704)(783.76842134,97.94423696)(783.74842041,98.00424123)
\curveto(783.72842138,98.07423683)(783.71842139,98.14423676)(783.71842041,98.21424123)
\curveto(783.7084214,98.28423662)(783.69342141,98.34923655)(783.67342041,98.40924123)
\curveto(783.61342149,98.60923629)(783.52842158,98.78923611)(783.41842041,98.94924123)
\curveto(783.39842171,98.97923592)(783.37842173,99.0042359)(783.35842041,99.02424123)
\lineto(783.29842041,99.08424123)
\curveto(783.27842183,99.12423578)(783.23842187,99.17423573)(783.17842041,99.23424123)
\curveto(783.03842207,99.33423557)(782.9084222,99.41923548)(782.78842041,99.48924123)
\curveto(782.66842244,99.55923534)(782.52342258,99.62923527)(782.35342041,99.69924123)
\curveto(782.28342282,99.72923517)(782.21342289,99.74923515)(782.14342041,99.75924123)
\curveto(782.07342303,99.77923512)(781.99842311,99.7992351)(781.91842041,99.81924123)
}
}
{
\newrgbcolor{curcolor}{0 0 0}
\pscustom[linestyle=none,fillstyle=solid,fillcolor=curcolor]
{
\newpath
\moveto(774.07342041,106.77385061)
\curveto(774.07343103,106.87384575)(774.08343102,106.96884566)(774.10342041,107.05885061)
\curveto(774.11343099,107.14884548)(774.14343096,107.21384541)(774.19342041,107.25385061)
\curveto(774.27343083,107.31384531)(774.37843073,107.34384528)(774.50842041,107.34385061)
\lineto(774.89842041,107.34385061)
\lineto(776.39842041,107.34385061)
\lineto(782.78842041,107.34385061)
\lineto(783.95842041,107.34385061)
\lineto(784.27342041,107.34385061)
\curveto(784.37342073,107.35384527)(784.45342065,107.33884529)(784.51342041,107.29885061)
\curveto(784.59342051,107.24884538)(784.64342046,107.17384545)(784.66342041,107.07385061)
\curveto(784.67342043,106.98384564)(784.67842043,106.87384575)(784.67842041,106.74385061)
\lineto(784.67842041,106.51885061)
\curveto(784.65842045,106.43884619)(784.64342046,106.36884626)(784.63342041,106.30885061)
\curveto(784.61342049,106.24884638)(784.57342053,106.19884643)(784.51342041,106.15885061)
\curveto(784.45342065,106.11884651)(784.37842073,106.09884653)(784.28842041,106.09885061)
\lineto(783.98842041,106.09885061)
\lineto(782.89342041,106.09885061)
\lineto(777.55342041,106.09885061)
\curveto(777.46342764,106.07884655)(777.38842772,106.06384656)(777.32842041,106.05385061)
\curveto(777.25842785,106.05384657)(777.19842791,106.0238466)(777.14842041,105.96385061)
\curveto(777.09842801,105.89384673)(777.07342803,105.80384682)(777.07342041,105.69385061)
\curveto(777.06342804,105.59384703)(777.05842805,105.48384714)(777.05842041,105.36385061)
\lineto(777.05842041,104.22385061)
\lineto(777.05842041,103.72885061)
\curveto(777.04842806,103.56884906)(776.98842812,103.45884917)(776.87842041,103.39885061)
\curveto(776.84842826,103.37884925)(776.81842829,103.36884926)(776.78842041,103.36885061)
\curveto(776.74842836,103.36884926)(776.7034284,103.36384926)(776.65342041,103.35385061)
\curveto(776.53342857,103.33384929)(776.42342868,103.33884929)(776.32342041,103.36885061)
\curveto(776.22342888,103.40884922)(776.15342895,103.46384916)(776.11342041,103.53385061)
\curveto(776.06342904,103.61384901)(776.03842907,103.73384889)(776.03842041,103.89385061)
\curveto(776.03842907,104.05384857)(776.02342908,104.18884844)(775.99342041,104.29885061)
\curveto(775.98342912,104.34884828)(775.97842913,104.40384822)(775.97842041,104.46385061)
\curveto(775.96842914,104.5238481)(775.95342915,104.58384804)(775.93342041,104.64385061)
\curveto(775.88342922,104.79384783)(775.83342927,104.93884769)(775.78342041,105.07885061)
\curveto(775.72342938,105.21884741)(775.65342945,105.35384727)(775.57342041,105.48385061)
\curveto(775.48342962,105.623847)(775.37842973,105.74384688)(775.25842041,105.84385061)
\curveto(775.13842997,105.94384668)(775.0084301,106.03884659)(774.86842041,106.12885061)
\curveto(774.76843034,106.18884644)(774.65843045,106.23384639)(774.53842041,106.26385061)
\curveto(774.41843069,106.30384632)(774.31343079,106.35384627)(774.22342041,106.41385061)
\curveto(774.16343094,106.46384616)(774.12343098,106.53384609)(774.10342041,106.62385061)
\curveto(774.09343101,106.64384598)(774.08843102,106.66884596)(774.08842041,106.69885061)
\curveto(774.08843102,106.7288459)(774.08343102,106.75384587)(774.07342041,106.77385061)
}
}
{
\newrgbcolor{curcolor}{0 0 0}
\pscustom[linestyle=none,fillstyle=solid,fillcolor=curcolor]
{
\newpath
\moveto(774.07342041,115.12345998)
\curveto(774.07343103,115.22345513)(774.08343102,115.31845503)(774.10342041,115.40845998)
\curveto(774.11343099,115.49845485)(774.14343096,115.56345479)(774.19342041,115.60345998)
\curveto(774.27343083,115.66345469)(774.37843073,115.69345466)(774.50842041,115.69345998)
\lineto(774.89842041,115.69345998)
\lineto(776.39842041,115.69345998)
\lineto(782.78842041,115.69345998)
\lineto(783.95842041,115.69345998)
\lineto(784.27342041,115.69345998)
\curveto(784.37342073,115.70345465)(784.45342065,115.68845466)(784.51342041,115.64845998)
\curveto(784.59342051,115.59845475)(784.64342046,115.52345483)(784.66342041,115.42345998)
\curveto(784.67342043,115.33345502)(784.67842043,115.22345513)(784.67842041,115.09345998)
\lineto(784.67842041,114.86845998)
\curveto(784.65842045,114.78845556)(784.64342046,114.71845563)(784.63342041,114.65845998)
\curveto(784.61342049,114.59845575)(784.57342053,114.5484558)(784.51342041,114.50845998)
\curveto(784.45342065,114.46845588)(784.37842073,114.4484559)(784.28842041,114.44845998)
\lineto(783.98842041,114.44845998)
\lineto(782.89342041,114.44845998)
\lineto(777.55342041,114.44845998)
\curveto(777.46342764,114.42845592)(777.38842772,114.41345594)(777.32842041,114.40345998)
\curveto(777.25842785,114.40345595)(777.19842791,114.37345598)(777.14842041,114.31345998)
\curveto(777.09842801,114.24345611)(777.07342803,114.1534562)(777.07342041,114.04345998)
\curveto(777.06342804,113.94345641)(777.05842805,113.83345652)(777.05842041,113.71345998)
\lineto(777.05842041,112.57345998)
\lineto(777.05842041,112.07845998)
\curveto(777.04842806,111.91845843)(776.98842812,111.80845854)(776.87842041,111.74845998)
\curveto(776.84842826,111.72845862)(776.81842829,111.71845863)(776.78842041,111.71845998)
\curveto(776.74842836,111.71845863)(776.7034284,111.71345864)(776.65342041,111.70345998)
\curveto(776.53342857,111.68345867)(776.42342868,111.68845866)(776.32342041,111.71845998)
\curveto(776.22342888,111.75845859)(776.15342895,111.81345854)(776.11342041,111.88345998)
\curveto(776.06342904,111.96345839)(776.03842907,112.08345827)(776.03842041,112.24345998)
\curveto(776.03842907,112.40345795)(776.02342908,112.53845781)(775.99342041,112.64845998)
\curveto(775.98342912,112.69845765)(775.97842913,112.7534576)(775.97842041,112.81345998)
\curveto(775.96842914,112.87345748)(775.95342915,112.93345742)(775.93342041,112.99345998)
\curveto(775.88342922,113.14345721)(775.83342927,113.28845706)(775.78342041,113.42845998)
\curveto(775.72342938,113.56845678)(775.65342945,113.70345665)(775.57342041,113.83345998)
\curveto(775.48342962,113.97345638)(775.37842973,114.09345626)(775.25842041,114.19345998)
\curveto(775.13842997,114.29345606)(775.0084301,114.38845596)(774.86842041,114.47845998)
\curveto(774.76843034,114.53845581)(774.65843045,114.58345577)(774.53842041,114.61345998)
\curveto(774.41843069,114.6534557)(774.31343079,114.70345565)(774.22342041,114.76345998)
\curveto(774.16343094,114.81345554)(774.12343098,114.88345547)(774.10342041,114.97345998)
\curveto(774.09343101,114.99345536)(774.08843102,115.01845533)(774.08842041,115.04845998)
\curveto(774.08843102,115.07845527)(774.08343102,115.10345525)(774.07342041,115.12345998)
}
}
{
\newrgbcolor{curcolor}{0 0 0}
\pscustom[linestyle=none,fillstyle=solid,fillcolor=curcolor]
{
\newpath
\moveto(794.90973633,37.28705373)
\curveto(794.90974702,37.35704805)(794.90974702,37.43704797)(794.90973633,37.52705373)
\curveto(794.89974703,37.61704779)(794.89974703,37.70204771)(794.90973633,37.78205373)
\curveto(794.90974702,37.87204754)(794.91974701,37.95204746)(794.93973633,38.02205373)
\curveto(794.95974697,38.10204731)(794.98974694,38.15704725)(795.02973633,38.18705373)
\curveto(795.07974685,38.21704719)(795.15474678,38.23704717)(795.25473633,38.24705373)
\curveto(795.34474659,38.26704714)(795.44974648,38.27704713)(795.56973633,38.27705373)
\curveto(795.67974625,38.28704712)(795.79474614,38.28704712)(795.91473633,38.27705373)
\lineto(796.21473633,38.27705373)
\lineto(799.22973633,38.27705373)
\lineto(802.12473633,38.27705373)
\curveto(802.45473948,38.27704713)(802.77973915,38.27204714)(803.09973633,38.26205373)
\curveto(803.40973852,38.26204715)(803.68973824,38.22204719)(803.93973633,38.14205373)
\curveto(804.28973764,38.02204739)(804.58473735,37.86704754)(804.82473633,37.67705373)
\curveto(805.05473688,37.48704792)(805.25473668,37.24704816)(805.42473633,36.95705373)
\curveto(805.47473646,36.89704851)(805.50973642,36.83204858)(805.52973633,36.76205373)
\curveto(805.54973638,36.70204871)(805.57473636,36.63204878)(805.60473633,36.55205373)
\curveto(805.65473628,36.43204898)(805.68973624,36.30204911)(805.70973633,36.16205373)
\curveto(805.73973619,36.03204938)(805.76973616,35.89704951)(805.79973633,35.75705373)
\curveto(805.81973611,35.7070497)(805.82473611,35.65704975)(805.81473633,35.60705373)
\curveto(805.80473613,35.55704985)(805.80473613,35.50204991)(805.81473633,35.44205373)
\curveto(805.82473611,35.42204999)(805.82473611,35.39705001)(805.81473633,35.36705373)
\curveto(805.81473612,35.33705007)(805.81973611,35.3120501)(805.82973633,35.29205373)
\curveto(805.83973609,35.25205016)(805.84473609,35.19705021)(805.84473633,35.12705373)
\curveto(805.84473609,35.05705035)(805.83973609,35.00205041)(805.82973633,34.96205373)
\curveto(805.81973611,34.9120505)(805.81973611,34.85705055)(805.82973633,34.79705373)
\curveto(805.83973609,34.73705067)(805.8347361,34.68205073)(805.81473633,34.63205373)
\curveto(805.78473615,34.50205091)(805.76473617,34.37705103)(805.75473633,34.25705373)
\curveto(805.74473619,34.13705127)(805.71973621,34.02205139)(805.67973633,33.91205373)
\curveto(805.55973637,33.54205187)(805.38973654,33.22205219)(805.16973633,32.95205373)
\curveto(804.94973698,32.68205273)(804.66973726,32.47205294)(804.32973633,32.32205373)
\curveto(804.20973772,32.27205314)(804.08473785,32.22705318)(803.95473633,32.18705373)
\curveto(803.82473811,32.15705325)(803.68973824,32.12205329)(803.54973633,32.08205373)
\curveto(803.49973843,32.07205334)(803.45973847,32.06705334)(803.42973633,32.06705373)
\curveto(803.38973854,32.06705334)(803.34473859,32.06205335)(803.29473633,32.05205373)
\curveto(803.26473867,32.04205337)(803.2297387,32.03705337)(803.18973633,32.03705373)
\curveto(803.13973879,32.03705337)(803.09973883,32.03205338)(803.06973633,32.02205373)
\lineto(802.90473633,32.02205373)
\curveto(802.82473911,32.00205341)(802.72473921,31.99705341)(802.60473633,32.00705373)
\curveto(802.47473946,32.01705339)(802.38473955,32.03205338)(802.33473633,32.05205373)
\curveto(802.24473969,32.07205334)(802.17973975,32.12705328)(802.13973633,32.21705373)
\curveto(802.11973981,32.24705316)(802.11473982,32.27705313)(802.12473633,32.30705373)
\curveto(802.12473981,32.33705307)(802.11973981,32.37705303)(802.10973633,32.42705373)
\curveto(802.09973983,32.46705294)(802.09473984,32.5070529)(802.09473633,32.54705373)
\lineto(802.09473633,32.69705373)
\curveto(802.09473984,32.81705259)(802.09973983,32.93705247)(802.10973633,33.05705373)
\curveto(802.10973982,33.18705222)(802.14473979,33.27705213)(802.21473633,33.32705373)
\curveto(802.27473966,33.36705204)(802.3347396,33.38705202)(802.39473633,33.38705373)
\curveto(802.45473948,33.38705202)(802.52473941,33.39705201)(802.60473633,33.41705373)
\curveto(802.6347393,33.42705198)(802.66973926,33.42705198)(802.70973633,33.41705373)
\curveto(802.73973919,33.41705199)(802.76473917,33.42205199)(802.78473633,33.43205373)
\lineto(802.99473633,33.43205373)
\curveto(803.04473889,33.45205196)(803.09473884,33.45705195)(803.14473633,33.44705373)
\curveto(803.18473875,33.44705196)(803.2297387,33.45705195)(803.27973633,33.47705373)
\curveto(803.40973852,33.5070519)(803.5347384,33.53705187)(803.65473633,33.56705373)
\curveto(803.76473817,33.59705181)(803.86973806,33.64205177)(803.96973633,33.70205373)
\curveto(804.25973767,33.87205154)(804.46473747,34.14205127)(804.58473633,34.51205373)
\curveto(804.60473733,34.56205085)(804.61973731,34.6120508)(804.62973633,34.66205373)
\curveto(804.6297373,34.72205069)(804.63973729,34.77705063)(804.65973633,34.82705373)
\lineto(804.65973633,34.90205373)
\curveto(804.66973726,34.97205044)(804.67973725,35.06705034)(804.68973633,35.18705373)
\curveto(804.68973724,35.31705009)(804.67973725,35.41704999)(804.65973633,35.48705373)
\curveto(804.63973729,35.55704985)(804.62473731,35.62704978)(804.61473633,35.69705373)
\curveto(804.59473734,35.77704963)(804.57473736,35.84704956)(804.55473633,35.90705373)
\curveto(804.39473754,36.28704912)(804.11973781,36.56204885)(803.72973633,36.73205373)
\curveto(803.59973833,36.78204863)(803.44473849,36.81704859)(803.26473633,36.83705373)
\curveto(803.08473885,36.86704854)(802.89973903,36.88204853)(802.70973633,36.88205373)
\curveto(802.50973942,36.89204852)(802.30973962,36.89204852)(802.10973633,36.88205373)
\lineto(801.53973633,36.88205373)
\lineto(797.29473633,36.88205373)
\lineto(795.74973633,36.88205373)
\curveto(795.63974629,36.88204853)(795.51974641,36.87704853)(795.38973633,36.86705373)
\curveto(795.25974667,36.85704855)(795.15474678,36.87704853)(795.07473633,36.92705373)
\curveto(795.00474693,36.98704842)(794.95474698,37.06704834)(794.92473633,37.16705373)
\curveto(794.92474701,37.18704822)(794.92474701,37.2070482)(794.92473633,37.22705373)
\curveto(794.92474701,37.24704816)(794.91974701,37.26704814)(794.90973633,37.28705373)
}
}
{
\newrgbcolor{curcolor}{0 0 0}
\pscustom[linestyle=none,fillstyle=solid,fillcolor=curcolor]
{
\newpath
\moveto(797.86473633,40.82072561)
\lineto(797.86473633,41.25572561)
\curveto(797.86474407,41.40572364)(797.90474403,41.51072354)(797.98473633,41.57072561)
\curveto(798.06474387,41.62072343)(798.16474377,41.6457234)(798.28473633,41.64572561)
\curveto(798.40474353,41.65572339)(798.52474341,41.66072339)(798.64473633,41.66072561)
\lineto(800.06973633,41.66072561)
\lineto(802.33473633,41.66072561)
\lineto(803.02473633,41.66072561)
\curveto(803.25473868,41.66072339)(803.45473848,41.68572336)(803.62473633,41.73572561)
\curveto(804.07473786,41.89572315)(804.38973754,42.19572285)(804.56973633,42.63572561)
\curveto(804.65973727,42.85572219)(804.69473724,43.12072193)(804.67473633,43.43072561)
\curveto(804.64473729,43.74072131)(804.58973734,43.99072106)(804.50973633,44.18072561)
\curveto(804.36973756,44.51072054)(804.19473774,44.77072028)(803.98473633,44.96072561)
\curveto(803.76473817,45.16071989)(803.47973845,45.31571973)(803.12973633,45.42572561)
\curveto(803.04973888,45.45571959)(802.96973896,45.47571957)(802.88973633,45.48572561)
\curveto(802.80973912,45.49571955)(802.72473921,45.51071954)(802.63473633,45.53072561)
\curveto(802.58473935,45.54071951)(802.53973939,45.54071951)(802.49973633,45.53072561)
\curveto(802.45973947,45.53071952)(802.41473952,45.54071951)(802.36473633,45.56072561)
\lineto(802.04973633,45.56072561)
\curveto(801.96973996,45.58071947)(801.87974005,45.58571946)(801.77973633,45.57572561)
\curveto(801.66974026,45.56571948)(801.56974036,45.56071949)(801.47973633,45.56072561)
\lineto(800.30973633,45.56072561)
\lineto(798.71973633,45.56072561)
\curveto(798.59974333,45.56071949)(798.47474346,45.55571949)(798.34473633,45.54572561)
\curveto(798.20474373,45.5457195)(798.09474384,45.57071948)(798.01473633,45.62072561)
\curveto(797.96474397,45.66071939)(797.934744,45.70571934)(797.92473633,45.75572561)
\curveto(797.90474403,45.81571923)(797.88474405,45.88571916)(797.86473633,45.96572561)
\lineto(797.86473633,46.19072561)
\curveto(797.86474407,46.31071874)(797.86974406,46.41571863)(797.87973633,46.50572561)
\curveto(797.88974404,46.60571844)(797.934744,46.68071837)(798.01473633,46.73072561)
\curveto(798.06474387,46.78071827)(798.13974379,46.80571824)(798.23973633,46.80572561)
\lineto(798.52473633,46.80572561)
\lineto(799.54473633,46.80572561)
\lineto(803.57973633,46.80572561)
\lineto(804.92973633,46.80572561)
\curveto(805.04973688,46.80571824)(805.16473677,46.80071825)(805.27473633,46.79072561)
\curveto(805.37473656,46.79071826)(805.44973648,46.75571829)(805.49973633,46.68572561)
\curveto(805.5297364,46.6457184)(805.55473638,46.58571846)(805.57473633,46.50572561)
\curveto(805.58473635,46.42571862)(805.59473634,46.33571871)(805.60473633,46.23572561)
\curveto(805.60473633,46.1457189)(805.59973633,46.05571899)(805.58973633,45.96572561)
\curveto(805.57973635,45.88571916)(805.55973637,45.82571922)(805.52973633,45.78572561)
\curveto(805.48973644,45.73571931)(805.42473651,45.69071936)(805.33473633,45.65072561)
\curveto(805.29473664,45.64071941)(805.23973669,45.63071942)(805.16973633,45.62072561)
\curveto(805.09973683,45.62071943)(805.0347369,45.61571943)(804.97473633,45.60572561)
\curveto(804.90473703,45.59571945)(804.84973708,45.57571947)(804.80973633,45.54572561)
\curveto(804.76973716,45.51571953)(804.75473718,45.47071958)(804.76473633,45.41072561)
\curveto(804.78473715,45.33071972)(804.84473709,45.2507198)(804.94473633,45.17072561)
\curveto(805.0347369,45.09071996)(805.10473683,45.01572003)(805.15473633,44.94572561)
\curveto(805.31473662,44.72572032)(805.45473648,44.47572057)(805.57473633,44.19572561)
\curveto(805.62473631,44.08572096)(805.65473628,43.97072108)(805.66473633,43.85072561)
\curveto(805.68473625,43.74072131)(805.70973622,43.62572142)(805.73973633,43.50572561)
\curveto(805.74973618,43.45572159)(805.74973618,43.40072165)(805.73973633,43.34072561)
\curveto(805.7297362,43.29072176)(805.7347362,43.24072181)(805.75473633,43.19072561)
\curveto(805.77473616,43.09072196)(805.77473616,43.00072205)(805.75473633,42.92072561)
\lineto(805.75473633,42.77072561)
\curveto(805.7347362,42.72072233)(805.72473621,42.66072239)(805.72473633,42.59072561)
\curveto(805.72473621,42.53072252)(805.71973621,42.47572257)(805.70973633,42.42572561)
\curveto(805.68973624,42.38572266)(805.67973625,42.3457227)(805.67973633,42.30572561)
\curveto(805.68973624,42.27572277)(805.68473625,42.23572281)(805.66473633,42.18572561)
\lineto(805.60473633,41.94572561)
\curveto(805.58473635,41.87572317)(805.55473638,41.80072325)(805.51473633,41.72072561)
\curveto(805.40473653,41.46072359)(805.25973667,41.24072381)(805.07973633,41.06072561)
\curveto(804.88973704,40.89072416)(804.66473727,40.7507243)(804.40473633,40.64072561)
\curveto(804.31473762,40.60072445)(804.22473771,40.57072448)(804.13473633,40.55072561)
\lineto(803.83473633,40.49072561)
\curveto(803.77473816,40.47072458)(803.71973821,40.46072459)(803.66973633,40.46072561)
\curveto(803.60973832,40.47072458)(803.54473839,40.46572458)(803.47473633,40.44572561)
\curveto(803.45473848,40.43572461)(803.4297385,40.43072462)(803.39973633,40.43072561)
\curveto(803.35973857,40.43072462)(803.32473861,40.42572462)(803.29473633,40.41572561)
\lineto(803.14473633,40.41572561)
\curveto(803.10473883,40.40572464)(803.05973887,40.40072465)(803.00973633,40.40072561)
\curveto(802.94973898,40.41072464)(802.89473904,40.41572463)(802.84473633,40.41572561)
\lineto(802.24473633,40.41572561)
\lineto(799.48473633,40.41572561)
\lineto(798.52473633,40.41572561)
\lineto(798.25473633,40.41572561)
\curveto(798.16474377,40.41572463)(798.08974384,40.43572461)(798.02973633,40.47572561)
\curveto(797.95974397,40.51572453)(797.90974402,40.59072446)(797.87973633,40.70072561)
\curveto(797.86974406,40.72072433)(797.86974406,40.74072431)(797.87973633,40.76072561)
\curveto(797.87974405,40.78072427)(797.87474406,40.80072425)(797.86473633,40.82072561)
}
}
{
\newrgbcolor{curcolor}{0 0 0}
\pscustom[linestyle=none,fillstyle=solid,fillcolor=curcolor]
{
\newpath
\moveto(797.71473633,52.39533498)
\curveto(797.69474424,53.02532975)(797.77974415,53.53032924)(797.96973633,53.91033498)
\curveto(798.15974377,54.29032848)(798.44474349,54.59532818)(798.82473633,54.82533498)
\curveto(798.92474301,54.88532789)(799.0347429,54.93032784)(799.15473633,54.96033498)
\curveto(799.26474267,55.00032777)(799.37974255,55.03532774)(799.49973633,55.06533498)
\curveto(799.68974224,55.11532766)(799.89474204,55.14532763)(800.11473633,55.15533498)
\curveto(800.3347416,55.16532761)(800.55974137,55.1703276)(800.78973633,55.17033498)
\lineto(802.39473633,55.17033498)
\lineto(804.73473633,55.17033498)
\curveto(804.90473703,55.1703276)(805.07473686,55.16532761)(805.24473633,55.15533498)
\curveto(805.41473652,55.15532762)(805.52473641,55.09032768)(805.57473633,54.96033498)
\curveto(805.59473634,54.91032786)(805.60473633,54.85532792)(805.60473633,54.79533498)
\curveto(805.61473632,54.74532803)(805.61973631,54.69032808)(805.61973633,54.63033498)
\curveto(805.61973631,54.50032827)(805.61473632,54.3753284)(805.60473633,54.25533498)
\curveto(805.60473633,54.13532864)(805.56473637,54.05032872)(805.48473633,54.00033498)
\curveto(805.41473652,53.95032882)(805.32473661,53.92532885)(805.21473633,53.92533498)
\lineto(804.88473633,53.92533498)
\lineto(803.59473633,53.92533498)
\lineto(801.14973633,53.92533498)
\curveto(800.87974105,53.92532885)(800.61474132,53.92032885)(800.35473633,53.91033498)
\curveto(800.08474185,53.90032887)(799.85474208,53.85532892)(799.66473633,53.77533498)
\curveto(799.46474247,53.69532908)(799.30474263,53.5753292)(799.18473633,53.41533498)
\curveto(799.05474288,53.25532952)(798.95474298,53.0703297)(798.88473633,52.86033498)
\curveto(798.86474307,52.80032997)(798.85474308,52.73533004)(798.85473633,52.66533498)
\curveto(798.84474309,52.60533017)(798.8297431,52.54533023)(798.80973633,52.48533498)
\curveto(798.79974313,52.43533034)(798.79974313,52.35533042)(798.80973633,52.24533498)
\curveto(798.80974312,52.14533063)(798.81474312,52.0753307)(798.82473633,52.03533498)
\curveto(798.84474309,51.99533078)(798.85474308,51.96033081)(798.85473633,51.93033498)
\curveto(798.84474309,51.90033087)(798.84474309,51.86533091)(798.85473633,51.82533498)
\curveto(798.88474305,51.69533108)(798.91974301,51.5703312)(798.95973633,51.45033498)
\curveto(798.98974294,51.34033143)(799.0347429,51.23533154)(799.09473633,51.13533498)
\curveto(799.11474282,51.09533168)(799.1347428,51.06033171)(799.15473633,51.03033498)
\curveto(799.17474276,51.00033177)(799.19474274,50.96533181)(799.21473633,50.92533498)
\curveto(799.46474247,50.5753322)(799.83974209,50.32033245)(800.33973633,50.16033498)
\curveto(800.41974151,50.13033264)(800.50474143,50.11033266)(800.59473633,50.10033498)
\curveto(800.67474126,50.09033268)(800.75474118,50.0753327)(800.83473633,50.05533498)
\curveto(800.88474105,50.03533274)(800.934741,50.03033274)(800.98473633,50.04033498)
\curveto(801.02474091,50.05033272)(801.06474087,50.04533273)(801.10473633,50.02533498)
\lineto(801.41973633,50.02533498)
\curveto(801.44974048,50.01533276)(801.48474045,50.01033276)(801.52473633,50.01033498)
\curveto(801.56474037,50.02033275)(801.60974032,50.02533275)(801.65973633,50.02533498)
\lineto(802.10973633,50.02533498)
\lineto(803.54973633,50.02533498)
\lineto(804.86973633,50.02533498)
\lineto(805.21473633,50.02533498)
\curveto(805.32473661,50.02533275)(805.41473652,50.00033277)(805.48473633,49.95033498)
\curveto(805.56473637,49.90033287)(805.60473633,49.81033296)(805.60473633,49.68033498)
\curveto(805.61473632,49.56033321)(805.61973631,49.43533334)(805.61973633,49.30533498)
\curveto(805.61973631,49.22533355)(805.61473632,49.15033362)(805.60473633,49.08033498)
\curveto(805.59473634,49.01033376)(805.56973636,48.95033382)(805.52973633,48.90033498)
\curveto(805.47973645,48.82033395)(805.38473655,48.78033399)(805.24473633,48.78033498)
\lineto(804.83973633,48.78033498)
\lineto(803.06973633,48.78033498)
\lineto(799.43973633,48.78033498)
\lineto(798.52473633,48.78033498)
\lineto(798.25473633,48.78033498)
\curveto(798.16474377,48.78033399)(798.09474384,48.80033397)(798.04473633,48.84033498)
\curveto(797.98474395,48.8703339)(797.94474399,48.92033385)(797.92473633,48.99033498)
\curveto(797.91474402,49.03033374)(797.90474403,49.08533369)(797.89473633,49.15533498)
\curveto(797.88474405,49.23533354)(797.87974405,49.31533346)(797.87973633,49.39533498)
\curveto(797.87974405,49.4753333)(797.88474405,49.55033322)(797.89473633,49.62033498)
\curveto(797.90474403,49.70033307)(797.91974401,49.75533302)(797.93973633,49.78533498)
\curveto(798.00974392,49.89533288)(798.09974383,49.94533283)(798.20973633,49.93533498)
\curveto(798.30974362,49.92533285)(798.42474351,49.94033283)(798.55473633,49.98033498)
\curveto(798.61474332,50.00033277)(798.66474327,50.04033273)(798.70473633,50.10033498)
\curveto(798.71474322,50.22033255)(798.66974326,50.31533246)(798.56973633,50.38533498)
\curveto(798.46974346,50.46533231)(798.38974354,50.54533223)(798.32973633,50.62533498)
\curveto(798.2297437,50.76533201)(798.13974379,50.90533187)(798.05973633,51.04533498)
\curveto(797.96974396,51.19533158)(797.89474404,51.36533141)(797.83473633,51.55533498)
\curveto(797.80474413,51.63533114)(797.78474415,51.72033105)(797.77473633,51.81033498)
\curveto(797.76474417,51.91033086)(797.74974418,52.00533077)(797.72973633,52.09533498)
\curveto(797.71974421,52.14533063)(797.71474422,52.19533058)(797.71473633,52.24533498)
\lineto(797.71473633,52.39533498)
}
}
{
\newrgbcolor{curcolor}{0 0 0}
\pscustom[linestyle=none,fillstyle=solid,fillcolor=curcolor]
{
}
}
{
\newrgbcolor{curcolor}{0 0 0}
\pscustom[linestyle=none,fillstyle=solid,fillcolor=curcolor]
{
\newpath
\moveto(794.98473633,64.11010061)
\curveto(794.95474698,65.74009517)(795.50974642,66.79009412)(796.64973633,67.26010061)
\curveto(796.87974505,67.36009355)(797.16974476,67.42509348)(797.51973633,67.45510061)
\curveto(797.85974407,67.49509341)(798.16974376,67.47009344)(798.44973633,67.38010061)
\curveto(798.70974322,67.29009362)(798.934743,67.17009374)(799.12473633,67.02010061)
\curveto(799.16474277,67.00009391)(799.19974273,66.97509393)(799.22973633,66.94510061)
\curveto(799.24974268,66.91509399)(799.27474266,66.89009402)(799.30473633,66.87010061)
\lineto(799.42473633,66.78010061)
\curveto(799.45474248,66.75009416)(799.47974245,66.71509419)(799.49973633,66.67510061)
\curveto(799.54974238,66.62509428)(799.59474234,66.57009434)(799.63473633,66.51010061)
\curveto(799.67474226,66.46009445)(799.72474221,66.41509449)(799.78473633,66.37510061)
\curveto(799.82474211,66.33509457)(799.87474206,66.32009459)(799.93473633,66.33010061)
\curveto(799.98474195,66.34009457)(800.0297419,66.37009454)(800.06973633,66.42010061)
\curveto(800.10974182,66.47009444)(800.14974178,66.52509438)(800.18973633,66.58510061)
\curveto(800.21974171,66.65509425)(800.24974168,66.72009419)(800.27973633,66.78010061)
\curveto(800.30974162,66.84009407)(800.33974159,66.89009402)(800.36973633,66.93010061)
\curveto(800.58974134,67.25009366)(800.89974103,67.5050934)(801.29973633,67.69510061)
\curveto(801.38974054,67.73509317)(801.48474045,67.76509314)(801.58473633,67.78510061)
\curveto(801.67474026,67.81509309)(801.76474017,67.84009307)(801.85473633,67.86010061)
\curveto(801.90474003,67.87009304)(801.95473998,67.87509303)(802.00473633,67.87510061)
\curveto(802.04473989,67.88509302)(802.08973984,67.89509301)(802.13973633,67.90510061)
\curveto(802.18973974,67.91509299)(802.23973969,67.91509299)(802.28973633,67.90510061)
\curveto(802.33973959,67.89509301)(802.38973954,67.90009301)(802.43973633,67.92010061)
\curveto(802.48973944,67.93009298)(802.54973938,67.93509297)(802.61973633,67.93510061)
\curveto(802.68973924,67.93509297)(802.74973918,67.92509298)(802.79973633,67.90510061)
\lineto(803.02473633,67.90510061)
\lineto(803.26473633,67.84510061)
\curveto(803.3347386,67.83509307)(803.40473853,67.82009309)(803.47473633,67.80010061)
\curveto(803.56473837,67.77009314)(803.64973828,67.74009317)(803.72973633,67.71010061)
\curveto(803.80973812,67.69009322)(803.88973804,67.66009325)(803.96973633,67.62010061)
\curveto(804.0297379,67.60009331)(804.08973784,67.57009334)(804.14973633,67.53010061)
\curveto(804.19973773,67.50009341)(804.24973768,67.46509344)(804.29973633,67.42510061)
\curveto(804.60973732,67.22509368)(804.86973706,66.97509393)(805.07973633,66.67510061)
\curveto(805.27973665,66.37509453)(805.44473649,66.03009488)(805.57473633,65.64010061)
\curveto(805.61473632,65.52009539)(805.63973629,65.39009552)(805.64973633,65.25010061)
\curveto(805.66973626,65.12009579)(805.69473624,64.98509592)(805.72473633,64.84510061)
\curveto(805.7347362,64.77509613)(805.73973619,64.7050962)(805.73973633,64.63510061)
\curveto(805.73973619,64.57509633)(805.74473619,64.5100964)(805.75473633,64.44010061)
\curveto(805.76473617,64.40009651)(805.76973616,64.34009657)(805.76973633,64.26010061)
\curveto(805.76973616,64.19009672)(805.76473617,64.14009677)(805.75473633,64.11010061)
\curveto(805.74473619,64.06009685)(805.73973619,64.01509689)(805.73973633,63.97510061)
\lineto(805.73973633,63.85510061)
\curveto(805.71973621,63.75509715)(805.70473623,63.65509725)(805.69473633,63.55510061)
\curveto(805.68473625,63.45509745)(805.66973626,63.36009755)(805.64973633,63.27010061)
\curveto(805.61973631,63.16009775)(805.59473634,63.05009786)(805.57473633,62.94010061)
\curveto(805.54473639,62.84009807)(805.50473643,62.73509817)(805.45473633,62.62510061)
\curveto(805.29473664,62.25509865)(805.09473684,61.94009897)(804.85473633,61.68010061)
\curveto(804.60473733,61.42009949)(804.29473764,61.2100997)(803.92473633,61.05010061)
\curveto(803.8347381,61.0100999)(803.73973819,60.97509993)(803.63973633,60.94510061)
\curveto(803.53973839,60.91509999)(803.4347385,60.88510002)(803.32473633,60.85510061)
\curveto(803.27473866,60.83510007)(803.22473871,60.82510008)(803.17473633,60.82510061)
\curveto(803.11473882,60.82510008)(803.05473888,60.81510009)(802.99473633,60.79510061)
\curveto(802.934739,60.77510013)(802.85473908,60.76510014)(802.75473633,60.76510061)
\curveto(802.65473928,60.76510014)(802.57973935,60.78010013)(802.52973633,60.81010061)
\curveto(802.49973943,60.82010009)(802.47473946,60.83510007)(802.45473633,60.85510061)
\lineto(802.39473633,60.91510061)
\curveto(802.37473956,60.95509995)(802.35973957,61.01509989)(802.34973633,61.09510061)
\curveto(802.33973959,61.18509972)(802.3347396,61.27509963)(802.33473633,61.36510061)
\curveto(802.3347396,61.45509945)(802.33973959,61.54009937)(802.34973633,61.62010061)
\curveto(802.35973957,61.7100992)(802.36973956,61.77509913)(802.37973633,61.81510061)
\curveto(802.39973953,61.83509907)(802.41473952,61.85509905)(802.42473633,61.87510061)
\curveto(802.42473951,61.89509901)(802.4347395,61.91509899)(802.45473633,61.93510061)
\curveto(802.54473939,62.0050989)(802.65973927,62.04509886)(802.79973633,62.05510061)
\curveto(802.93973899,62.07509883)(803.06473887,62.1050988)(803.17473633,62.14510061)
\lineto(803.53473633,62.29510061)
\curveto(803.64473829,62.34509856)(803.74973818,62.4100985)(803.84973633,62.49010061)
\curveto(803.87973805,62.5100984)(803.90473803,62.53009838)(803.92473633,62.55010061)
\curveto(803.94473799,62.58009833)(803.96973796,62.6050983)(803.99973633,62.62510061)
\curveto(804.05973787,62.66509824)(804.10473783,62.70009821)(804.13473633,62.73010061)
\curveto(804.16473777,62.77009814)(804.19473774,62.8050981)(804.22473633,62.83510061)
\curveto(804.25473768,62.87509803)(804.28473765,62.92009799)(804.31473633,62.97010061)
\curveto(804.37473756,63.06009785)(804.42473751,63.15509775)(804.46473633,63.25510061)
\lineto(804.58473633,63.58510061)
\curveto(804.6347373,63.73509717)(804.66473727,63.93509697)(804.67473633,64.18510061)
\curveto(804.68473725,64.43509647)(804.66473727,64.64509626)(804.61473633,64.81510061)
\curveto(804.59473734,64.89509601)(804.57973735,64.96509594)(804.56973633,65.02510061)
\lineto(804.50973633,65.23510061)
\curveto(804.38973754,65.51509539)(804.23973769,65.75509515)(804.05973633,65.95510061)
\curveto(803.87973805,66.16509474)(803.64973828,66.33009458)(803.36973633,66.45010061)
\curveto(803.29973863,66.48009443)(803.2297387,66.50009441)(803.15973633,66.51010061)
\lineto(802.91973633,66.57010061)
\curveto(802.77973915,66.6100943)(802.61973931,66.62009429)(802.43973633,66.60010061)
\curveto(802.24973968,66.58009433)(802.09973983,66.55009436)(801.98973633,66.51010061)
\curveto(801.60974032,66.38009453)(801.31974061,66.19509471)(801.11973633,65.95510061)
\curveto(800.91974101,65.72509518)(800.75974117,65.41509549)(800.63973633,65.02510061)
\curveto(800.60974132,64.91509599)(800.58974134,64.79509611)(800.57973633,64.66510061)
\curveto(800.56974136,64.54509636)(800.56474137,64.42009649)(800.56473633,64.29010061)
\curveto(800.56474137,64.13009678)(800.55974137,63.99009692)(800.54973633,63.87010061)
\curveto(800.53974139,63.75009716)(800.47974145,63.66509724)(800.36973633,63.61510061)
\curveto(800.33974159,63.59509731)(800.30474163,63.58509732)(800.26473633,63.58510061)
\lineto(800.12973633,63.58510061)
\curveto(800.0297419,63.57509733)(799.934742,63.57509733)(799.84473633,63.58510061)
\curveto(799.75474218,63.6050973)(799.68974224,63.64509726)(799.64973633,63.70510061)
\curveto(799.61974231,63.74509716)(799.59974233,63.78509712)(799.58973633,63.82510061)
\curveto(799.57974235,63.87509703)(799.56974236,63.93009698)(799.55973633,63.99010061)
\curveto(799.54974238,64.0100969)(799.54974238,64.03509687)(799.55973633,64.06510061)
\curveto(799.55974237,64.09509681)(799.55474238,64.12009679)(799.54473633,64.14010061)
\lineto(799.54473633,64.27510061)
\curveto(799.52474241,64.38509652)(799.51474242,64.48509642)(799.51473633,64.57510061)
\curveto(799.50474243,64.67509623)(799.48474245,64.77009614)(799.45473633,64.86010061)
\curveto(799.34474259,65.18009573)(799.19974273,65.43509547)(799.01973633,65.62510061)
\curveto(798.83974309,65.81509509)(798.58974334,65.96509494)(798.26973633,66.07510061)
\curveto(798.16974376,66.1050948)(798.04474389,66.12509478)(797.89473633,66.13510061)
\curveto(797.7347442,66.15509475)(797.58974434,66.15009476)(797.45973633,66.12010061)
\curveto(797.38974454,66.10009481)(797.32474461,66.08009483)(797.26473633,66.06010061)
\curveto(797.19474474,66.05009486)(797.1297448,66.03009488)(797.06973633,66.00010061)
\curveto(796.8297451,65.90009501)(796.63974529,65.75509515)(796.49973633,65.56510061)
\curveto(796.35974557,65.37509553)(796.24974568,65.15009576)(796.16973633,64.89010061)
\curveto(796.14974578,64.83009608)(796.13974579,64.77009614)(796.13973633,64.71010061)
\curveto(796.13974579,64.65009626)(796.1297458,64.58509632)(796.10973633,64.51510061)
\curveto(796.08974584,64.43509647)(796.07974585,64.34009657)(796.07973633,64.23010061)
\curveto(796.07974585,64.12009679)(796.08974584,64.02509688)(796.10973633,63.94510061)
\curveto(796.1297458,63.89509701)(796.13974579,63.84509706)(796.13973633,63.79510061)
\curveto(796.13974579,63.75509715)(796.14974578,63.7100972)(796.16973633,63.66010061)
\curveto(796.21974571,63.48009743)(796.29474564,63.3100976)(796.39473633,63.15010061)
\curveto(796.48474545,63.00009791)(796.59974533,62.87009804)(796.73973633,62.76010061)
\curveto(796.85974507,62.67009824)(796.98974494,62.59009832)(797.12973633,62.52010061)
\curveto(797.26974466,62.45009846)(797.42474451,62.38509852)(797.59473633,62.32510061)
\curveto(797.70474423,62.29509861)(797.82474411,62.27509863)(797.95473633,62.26510061)
\curveto(798.07474386,62.25509865)(798.17474376,62.22009869)(798.25473633,62.16010061)
\curveto(798.29474364,62.14009877)(798.3347436,62.08009883)(798.37473633,61.98010061)
\curveto(798.38474355,61.94009897)(798.39474354,61.88009903)(798.40473633,61.80010061)
\lineto(798.40473633,61.54510061)
\curveto(798.39474354,61.45509945)(798.38474355,61.37009954)(798.37473633,61.29010061)
\curveto(798.36474357,61.22009969)(798.34974358,61.17009974)(798.32973633,61.14010061)
\curveto(798.29974363,61.10009981)(798.24474369,61.06509984)(798.16473633,61.03510061)
\curveto(798.08474385,61.0050999)(797.99974393,61.00009991)(797.90973633,61.02010061)
\curveto(797.85974407,61.03009988)(797.80974412,61.03509987)(797.75973633,61.03510061)
\lineto(797.57973633,61.06510061)
\curveto(797.47974445,61.09509981)(797.37974455,61.12009979)(797.27973633,61.14010061)
\curveto(797.17974475,61.17009974)(797.08974484,61.2050997)(797.00973633,61.24510061)
\curveto(796.89974503,61.29509961)(796.79474514,61.34009957)(796.69473633,61.38010061)
\curveto(796.58474535,61.42009949)(796.47974545,61.47009944)(796.37973633,61.53010061)
\curveto(795.83974609,61.86009905)(795.44474649,62.33009858)(795.19473633,62.94010061)
\curveto(795.14474679,63.06009785)(795.10974682,63.18509772)(795.08973633,63.31510061)
\curveto(795.06974686,63.45509745)(795.04474689,63.59509731)(795.01473633,63.73510061)
\curveto(795.00474693,63.79509711)(794.99974693,63.85509705)(794.99973633,63.91510061)
\curveto(794.99974693,63.98509692)(794.99474694,64.05009686)(794.98473633,64.11010061)
}
}
{
\newrgbcolor{curcolor}{0 0 0}
\pscustom[linestyle=none,fillstyle=solid,fillcolor=curcolor]
{
\newpath
\moveto(800.50473633,76.34470998)
\lineto(800.75973633,76.34470998)
\curveto(800.83974109,76.35470228)(800.91474102,76.34970228)(800.98473633,76.32970998)
\lineto(801.22473633,76.32970998)
\lineto(801.38973633,76.32970998)
\curveto(801.48974044,76.30970232)(801.59474034,76.29970233)(801.70473633,76.29970998)
\curveto(801.80474013,76.29970233)(801.90474003,76.28970234)(802.00473633,76.26970998)
\lineto(802.15473633,76.26970998)
\curveto(802.29473964,76.23970239)(802.4347395,76.21970241)(802.57473633,76.20970998)
\curveto(802.70473923,76.19970243)(802.8347391,76.17470246)(802.96473633,76.13470998)
\curveto(803.04473889,76.11470252)(803.1297388,76.09470254)(803.21973633,76.07470998)
\lineto(803.45973633,76.01470998)
\lineto(803.75973633,75.89470998)
\curveto(803.84973808,75.86470277)(803.93973799,75.8297028)(804.02973633,75.78970998)
\curveto(804.24973768,75.68970294)(804.46473747,75.55470308)(804.67473633,75.38470998)
\curveto(804.88473705,75.22470341)(805.05473688,75.04970358)(805.18473633,74.85970998)
\curveto(805.22473671,74.80970382)(805.26473667,74.74970388)(805.30473633,74.67970998)
\curveto(805.3347366,74.61970401)(805.36973656,74.55970407)(805.40973633,74.49970998)
\curveto(805.45973647,74.41970421)(805.49973643,74.32470431)(805.52973633,74.21470998)
\curveto(805.55973637,74.10470453)(805.58973634,73.99970463)(805.61973633,73.89970998)
\curveto(805.65973627,73.78970484)(805.68473625,73.67970495)(805.69473633,73.56970998)
\curveto(805.70473623,73.45970517)(805.71973621,73.34470529)(805.73973633,73.22470998)
\curveto(805.74973618,73.18470545)(805.74973618,73.13970549)(805.73973633,73.08970998)
\curveto(805.73973619,73.04970558)(805.74473619,73.00970562)(805.75473633,72.96970998)
\curveto(805.76473617,72.9297057)(805.76973616,72.87470576)(805.76973633,72.80470998)
\curveto(805.76973616,72.7347059)(805.76473617,72.68470595)(805.75473633,72.65470998)
\curveto(805.7347362,72.60470603)(805.7297362,72.55970607)(805.73973633,72.51970998)
\curveto(805.74973618,72.47970615)(805.74973618,72.44470619)(805.73973633,72.41470998)
\lineto(805.73973633,72.32470998)
\curveto(805.71973621,72.26470637)(805.70473623,72.19970643)(805.69473633,72.12970998)
\curveto(805.69473624,72.06970656)(805.68973624,72.00470663)(805.67973633,71.93470998)
\curveto(805.6297363,71.76470687)(805.57973635,71.60470703)(805.52973633,71.45470998)
\curveto(805.47973645,71.30470733)(805.41473652,71.15970747)(805.33473633,71.01970998)
\curveto(805.29473664,70.96970766)(805.26473667,70.91470772)(805.24473633,70.85470998)
\curveto(805.21473672,70.80470783)(805.17973675,70.75470788)(805.13973633,70.70470998)
\curveto(804.95973697,70.46470817)(804.73973719,70.26470837)(804.47973633,70.10470998)
\curveto(804.21973771,69.94470869)(803.934738,69.80470883)(803.62473633,69.68470998)
\curveto(803.48473845,69.62470901)(803.34473859,69.57970905)(803.20473633,69.54970998)
\curveto(803.05473888,69.51970911)(802.89973903,69.48470915)(802.73973633,69.44470998)
\curveto(802.6297393,69.42470921)(802.51973941,69.40970922)(802.40973633,69.39970998)
\curveto(802.29973963,69.38970924)(802.18973974,69.37470926)(802.07973633,69.35470998)
\curveto(802.03973989,69.34470929)(801.99973993,69.33970929)(801.95973633,69.33970998)
\curveto(801.91974001,69.34970928)(801.87974005,69.34970928)(801.83973633,69.33970998)
\curveto(801.78974014,69.3297093)(801.73974019,69.32470931)(801.68973633,69.32470998)
\lineto(801.52473633,69.32470998)
\curveto(801.47474046,69.30470933)(801.42474051,69.29970933)(801.37473633,69.30970998)
\curveto(801.31474062,69.31970931)(801.25974067,69.31970931)(801.20973633,69.30970998)
\curveto(801.16974076,69.29970933)(801.12474081,69.29970933)(801.07473633,69.30970998)
\curveto(801.02474091,69.31970931)(800.97474096,69.31470932)(800.92473633,69.29470998)
\curveto(800.85474108,69.27470936)(800.77974115,69.26970936)(800.69973633,69.27970998)
\curveto(800.60974132,69.28970934)(800.52474141,69.29470934)(800.44473633,69.29470998)
\curveto(800.35474158,69.29470934)(800.25474168,69.28970934)(800.14473633,69.27970998)
\curveto(800.02474191,69.26970936)(799.92474201,69.27470936)(799.84473633,69.29470998)
\lineto(799.55973633,69.29470998)
\lineto(798.92973633,69.33970998)
\curveto(798.8297431,69.34970928)(798.7347432,69.35970927)(798.64473633,69.36970998)
\lineto(798.34473633,69.39970998)
\curveto(798.29474364,69.41970921)(798.24474369,69.42470921)(798.19473633,69.41470998)
\curveto(798.1347438,69.41470922)(798.07974385,69.42470921)(798.02973633,69.44470998)
\curveto(797.85974407,69.49470914)(797.69474424,69.5347091)(797.53473633,69.56470998)
\curveto(797.36474457,69.59470904)(797.20474473,69.64470899)(797.05473633,69.71470998)
\curveto(796.59474534,69.90470873)(796.21974571,70.12470851)(795.92973633,70.37470998)
\curveto(795.63974629,70.634708)(795.39474654,70.99470764)(795.19473633,71.45470998)
\curveto(795.14474679,71.58470705)(795.10974682,71.71470692)(795.08973633,71.84470998)
\curveto(795.06974686,71.98470665)(795.04474689,72.12470651)(795.01473633,72.26470998)
\curveto(795.00474693,72.3347063)(794.99974693,72.39970623)(794.99973633,72.45970998)
\curveto(794.99974693,72.51970611)(794.99474694,72.58470605)(794.98473633,72.65470998)
\curveto(794.96474697,73.48470515)(795.11474682,74.15470448)(795.43473633,74.66470998)
\curveto(795.74474619,75.17470346)(796.18474575,75.55470308)(796.75473633,75.80470998)
\curveto(796.87474506,75.85470278)(796.99974493,75.89970273)(797.12973633,75.93970998)
\curveto(797.25974467,75.97970265)(797.39474454,76.02470261)(797.53473633,76.07470998)
\curveto(797.61474432,76.09470254)(797.69974423,76.10970252)(797.78973633,76.11970998)
\lineto(798.02973633,76.17970998)
\curveto(798.13974379,76.20970242)(798.24974368,76.22470241)(798.35973633,76.22470998)
\curveto(798.46974346,76.2347024)(798.57974335,76.24970238)(798.68973633,76.26970998)
\curveto(798.73974319,76.28970234)(798.78474315,76.29470234)(798.82473633,76.28470998)
\curveto(798.86474307,76.28470235)(798.90474303,76.28970234)(798.94473633,76.29970998)
\curveto(798.99474294,76.30970232)(799.04974288,76.30970232)(799.10973633,76.29970998)
\curveto(799.15974277,76.29970233)(799.20974272,76.30470233)(799.25973633,76.31470998)
\lineto(799.39473633,76.31470998)
\curveto(799.45474248,76.3347023)(799.52474241,76.3347023)(799.60473633,76.31470998)
\curveto(799.67474226,76.30470233)(799.73974219,76.30970232)(799.79973633,76.32970998)
\curveto(799.8297421,76.33970229)(799.86974206,76.34470229)(799.91973633,76.34470998)
\lineto(800.03973633,76.34470998)
\lineto(800.50473633,76.34470998)
\moveto(802.82973633,74.79970998)
\curveto(802.50973942,74.89970373)(802.14473979,74.95970367)(801.73473633,74.97970998)
\curveto(801.32474061,74.99970363)(800.91474102,75.00970362)(800.50473633,75.00970998)
\curveto(800.07474186,75.00970362)(799.65474228,74.99970363)(799.24473633,74.97970998)
\curveto(798.8347431,74.95970367)(798.44974348,74.91470372)(798.08973633,74.84470998)
\curveto(797.7297442,74.77470386)(797.40974452,74.66470397)(797.12973633,74.51470998)
\curveto(796.83974509,74.37470426)(796.60474533,74.17970445)(796.42473633,73.92970998)
\curveto(796.31474562,73.76970486)(796.2347457,73.58970504)(796.18473633,73.38970998)
\curveto(796.12474581,73.18970544)(796.09474584,72.94470569)(796.09473633,72.65470998)
\curveto(796.11474582,72.634706)(796.12474581,72.59970603)(796.12473633,72.54970998)
\curveto(796.11474582,72.49970613)(796.11474582,72.45970617)(796.12473633,72.42970998)
\curveto(796.14474579,72.34970628)(796.16474577,72.27470636)(796.18473633,72.20470998)
\curveto(796.19474574,72.14470649)(796.21474572,72.07970655)(796.24473633,72.00970998)
\curveto(796.36474557,71.73970689)(796.5347454,71.51970711)(796.75473633,71.34970998)
\curveto(796.96474497,71.18970744)(797.20974472,71.05470758)(797.48973633,70.94470998)
\curveto(797.59974433,70.89470774)(797.71974421,70.85470778)(797.84973633,70.82470998)
\curveto(797.96974396,70.80470783)(798.09474384,70.77970785)(798.22473633,70.74970998)
\curveto(798.27474366,70.7297079)(798.3297436,70.71970791)(798.38973633,70.71970998)
\curveto(798.43974349,70.71970791)(798.48974344,70.71470792)(798.53973633,70.70470998)
\curveto(798.6297433,70.69470794)(798.72474321,70.68470795)(798.82473633,70.67470998)
\curveto(798.91474302,70.66470797)(799.00974292,70.65470798)(799.10973633,70.64470998)
\curveto(799.18974274,70.64470799)(799.27474266,70.63970799)(799.36473633,70.62970998)
\lineto(799.60473633,70.62970998)
\lineto(799.78473633,70.62970998)
\curveto(799.81474212,70.61970801)(799.84974208,70.61470802)(799.88973633,70.61470998)
\lineto(800.02473633,70.61470998)
\lineto(800.47473633,70.61470998)
\curveto(800.55474138,70.61470802)(800.63974129,70.60970802)(800.72973633,70.59970998)
\curveto(800.80974112,70.59970803)(800.88474105,70.60970802)(800.95473633,70.62970998)
\lineto(801.22473633,70.62970998)
\curveto(801.24474069,70.629708)(801.27474066,70.62470801)(801.31473633,70.61470998)
\curveto(801.34474059,70.61470802)(801.36974056,70.61970801)(801.38973633,70.62970998)
\curveto(801.48974044,70.63970799)(801.58974034,70.64470799)(801.68973633,70.64470998)
\curveto(801.77974015,70.65470798)(801.87974005,70.66470797)(801.98973633,70.67470998)
\curveto(802.10973982,70.70470793)(802.2347397,70.71970791)(802.36473633,70.71970998)
\curveto(802.48473945,70.7297079)(802.59973933,70.75470788)(802.70973633,70.79470998)
\curveto(803.00973892,70.87470776)(803.27473866,70.95970767)(803.50473633,71.04970998)
\curveto(803.7347382,71.14970748)(803.94973798,71.29470734)(804.14973633,71.48470998)
\curveto(804.34973758,71.69470694)(804.49973743,71.95970667)(804.59973633,72.27970998)
\curveto(804.61973731,72.31970631)(804.6297373,72.35470628)(804.62973633,72.38470998)
\curveto(804.61973731,72.42470621)(804.62473731,72.46970616)(804.64473633,72.51970998)
\curveto(804.65473728,72.55970607)(804.66473727,72.629706)(804.67473633,72.72970998)
\curveto(804.68473725,72.83970579)(804.67973725,72.92470571)(804.65973633,72.98470998)
\curveto(804.63973729,73.05470558)(804.6297373,73.12470551)(804.62973633,73.19470998)
\curveto(804.61973731,73.26470537)(804.60473733,73.3297053)(804.58473633,73.38970998)
\curveto(804.52473741,73.58970504)(804.43973749,73.76970486)(804.32973633,73.92970998)
\curveto(804.30973762,73.95970467)(804.28973764,73.98470465)(804.26973633,74.00470998)
\lineto(804.20973633,74.06470998)
\curveto(804.18973774,74.10470453)(804.14973778,74.15470448)(804.08973633,74.21470998)
\curveto(803.94973798,74.31470432)(803.81973811,74.39970423)(803.69973633,74.46970998)
\curveto(803.57973835,74.53970409)(803.4347385,74.60970402)(803.26473633,74.67970998)
\curveto(803.19473874,74.70970392)(803.12473881,74.7297039)(803.05473633,74.73970998)
\curveto(802.98473895,74.75970387)(802.90973902,74.77970385)(802.82973633,74.79970998)
}
}
{
\newrgbcolor{curcolor}{0 0 0}
\pscustom[linestyle=none,fillstyle=solid,fillcolor=curcolor]
{
\newpath
\moveto(803.95473633,78.63431936)
\lineto(803.95473633,79.26431936)
\lineto(803.95473633,79.45931936)
\curveto(803.95473798,79.52931683)(803.96473797,79.58931677)(803.98473633,79.63931936)
\curveto(804.02473791,79.70931665)(804.06473787,79.7593166)(804.10473633,79.78931936)
\curveto(804.15473778,79.82931653)(804.21973771,79.84931651)(804.29973633,79.84931936)
\curveto(804.37973755,79.8593165)(804.46473747,79.86431649)(804.55473633,79.86431936)
\lineto(805.27473633,79.86431936)
\curveto(805.75473618,79.86431649)(806.16473577,79.80431655)(806.50473633,79.68431936)
\curveto(806.84473509,79.56431679)(807.11973481,79.36931699)(807.32973633,79.09931936)
\curveto(807.37973455,79.02931733)(807.42473451,78.9593174)(807.46473633,78.88931936)
\curveto(807.51473442,78.82931753)(807.55973437,78.7543176)(807.59973633,78.66431936)
\curveto(807.60973432,78.64431771)(807.61973431,78.61431774)(807.62973633,78.57431936)
\curveto(807.64973428,78.53431782)(807.65473428,78.48931787)(807.64473633,78.43931936)
\curveto(807.61473432,78.34931801)(807.53973439,78.29431806)(807.41973633,78.27431936)
\curveto(807.30973462,78.2543181)(807.21473472,78.26931809)(807.13473633,78.31931936)
\curveto(807.06473487,78.34931801)(806.99973493,78.39431796)(806.93973633,78.45431936)
\curveto(806.88973504,78.52431783)(806.83973509,78.58931777)(806.78973633,78.64931936)
\curveto(806.73973519,78.71931764)(806.66473527,78.77931758)(806.56473633,78.82931936)
\curveto(806.47473546,78.88931747)(806.38473555,78.93931742)(806.29473633,78.97931936)
\curveto(806.26473567,78.99931736)(806.20473573,79.02431733)(806.11473633,79.05431936)
\curveto(806.0347359,79.08431727)(805.96473597,79.08931727)(805.90473633,79.06931936)
\curveto(805.76473617,79.03931732)(805.67473626,78.97931738)(805.63473633,78.88931936)
\curveto(805.60473633,78.80931755)(805.58973634,78.71931764)(805.58973633,78.61931936)
\curveto(805.58973634,78.51931784)(805.56473637,78.43431792)(805.51473633,78.36431936)
\curveto(805.44473649,78.27431808)(805.30473663,78.22931813)(805.09473633,78.22931936)
\lineto(804.53973633,78.22931936)
\lineto(804.31473633,78.22931936)
\curveto(804.2347377,78.23931812)(804.16973776,78.2593181)(804.11973633,78.28931936)
\curveto(804.03973789,78.34931801)(803.99473794,78.41931794)(803.98473633,78.49931936)
\curveto(803.97473796,78.51931784)(803.96973796,78.53931782)(803.96973633,78.55931936)
\curveto(803.96973796,78.58931777)(803.96473797,78.61431774)(803.95473633,78.63431936)
}
}
{
\newrgbcolor{curcolor}{0 0 0}
\pscustom[linestyle=none,fillstyle=solid,fillcolor=curcolor]
{
}
}
{
\newrgbcolor{curcolor}{0 0 0}
\pscustom[linestyle=none,fillstyle=solid,fillcolor=curcolor]
{
\newpath
\moveto(794.98473633,89.26463186)
\curveto(794.97474696,89.95462722)(795.09474684,90.55462662)(795.34473633,91.06463186)
\curveto(795.59474634,91.58462559)(795.929746,91.9796252)(796.34973633,92.24963186)
\curveto(796.4297455,92.29962488)(796.51974541,92.34462483)(796.61973633,92.38463186)
\curveto(796.70974522,92.42462475)(796.80474513,92.46962471)(796.90473633,92.51963186)
\curveto(797.00474493,92.55962462)(797.10474483,92.58962459)(797.20473633,92.60963186)
\curveto(797.30474463,92.62962455)(797.40974452,92.64962453)(797.51973633,92.66963186)
\curveto(797.56974436,92.68962449)(797.61474432,92.69462448)(797.65473633,92.68463186)
\curveto(797.69474424,92.6746245)(797.73974419,92.6796245)(797.78973633,92.69963186)
\curveto(797.83974409,92.70962447)(797.92474401,92.71462446)(798.04473633,92.71463186)
\curveto(798.15474378,92.71462446)(798.23974369,92.70962447)(798.29973633,92.69963186)
\curveto(798.35974357,92.6796245)(798.41974351,92.66962451)(798.47973633,92.66963186)
\curveto(798.53974339,92.6796245)(798.59974333,92.6746245)(798.65973633,92.65463186)
\curveto(798.79974313,92.61462456)(798.934743,92.5796246)(799.06473633,92.54963186)
\curveto(799.19474274,92.51962466)(799.31974261,92.4796247)(799.43973633,92.42963186)
\curveto(799.57974235,92.36962481)(799.70474223,92.29962488)(799.81473633,92.21963186)
\curveto(799.92474201,92.14962503)(800.0347419,92.0746251)(800.14473633,91.99463186)
\lineto(800.20473633,91.93463186)
\curveto(800.22474171,91.92462525)(800.24474169,91.90962527)(800.26473633,91.88963186)
\curveto(800.42474151,91.76962541)(800.56974136,91.63462554)(800.69973633,91.48463186)
\curveto(800.8297411,91.33462584)(800.95474098,91.174626)(801.07473633,91.00463186)
\curveto(801.29474064,90.69462648)(801.49974043,90.39962678)(801.68973633,90.11963186)
\curveto(801.8297401,89.88962729)(801.96473997,89.65962752)(802.09473633,89.42963186)
\curveto(802.22473971,89.20962797)(802.35973957,88.98962819)(802.49973633,88.76963186)
\curveto(802.66973926,88.51962866)(802.84973908,88.2796289)(803.03973633,88.04963186)
\curveto(803.2297387,87.82962935)(803.45473848,87.63962954)(803.71473633,87.47963186)
\curveto(803.77473816,87.43962974)(803.8347381,87.40462977)(803.89473633,87.37463186)
\curveto(803.94473799,87.34462983)(804.00973792,87.31462986)(804.08973633,87.28463186)
\curveto(804.15973777,87.26462991)(804.21973771,87.25962992)(804.26973633,87.26963186)
\curveto(804.33973759,87.28962989)(804.39473754,87.32462985)(804.43473633,87.37463186)
\curveto(804.46473747,87.42462975)(804.48473745,87.48462969)(804.49473633,87.55463186)
\lineto(804.49473633,87.79463186)
\lineto(804.49473633,88.54463186)
\lineto(804.49473633,91.34963186)
\lineto(804.49473633,92.00963186)
\curveto(804.49473744,92.09962508)(804.49973743,92.18462499)(804.50973633,92.26463186)
\curveto(804.50973742,92.34462483)(804.5297374,92.40962477)(804.56973633,92.45963186)
\curveto(804.60973732,92.50962467)(804.68473725,92.54962463)(804.79473633,92.57963186)
\curveto(804.89473704,92.61962456)(804.99473694,92.62962455)(805.09473633,92.60963186)
\lineto(805.22973633,92.60963186)
\curveto(805.29973663,92.58962459)(805.35973657,92.56962461)(805.40973633,92.54963186)
\curveto(805.45973647,92.52962465)(805.49973643,92.49462468)(805.52973633,92.44463186)
\curveto(805.56973636,92.39462478)(805.58973634,92.32462485)(805.58973633,92.23463186)
\lineto(805.58973633,91.96463186)
\lineto(805.58973633,91.06463186)
\lineto(805.58973633,87.55463186)
\lineto(805.58973633,86.48963186)
\curveto(805.58973634,86.40963077)(805.59473634,86.31963086)(805.60473633,86.21963186)
\curveto(805.60473633,86.11963106)(805.59473634,86.03463114)(805.57473633,85.96463186)
\curveto(805.50473643,85.75463142)(805.32473661,85.68963149)(805.03473633,85.76963186)
\curveto(804.99473694,85.7796314)(804.95973697,85.7796314)(804.92973633,85.76963186)
\curveto(804.88973704,85.76963141)(804.84473709,85.7796314)(804.79473633,85.79963186)
\curveto(804.71473722,85.81963136)(804.6297373,85.83963134)(804.53973633,85.85963186)
\curveto(804.44973748,85.8796313)(804.36473757,85.90463127)(804.28473633,85.93463186)
\curveto(803.79473814,86.09463108)(803.37973855,86.29463088)(803.03973633,86.53463186)
\curveto(802.78973914,86.71463046)(802.56473937,86.91963026)(802.36473633,87.14963186)
\curveto(802.15473978,87.3796298)(801.95973997,87.61962956)(801.77973633,87.86963186)
\curveto(801.59974033,88.12962905)(801.4297405,88.39462878)(801.26973633,88.66463186)
\curveto(801.09974083,88.94462823)(800.92474101,89.21462796)(800.74473633,89.47463186)
\curveto(800.66474127,89.58462759)(800.58974134,89.68962749)(800.51973633,89.78963186)
\curveto(800.44974148,89.89962728)(800.37474156,90.00962717)(800.29473633,90.11963186)
\curveto(800.26474167,90.15962702)(800.2347417,90.19462698)(800.20473633,90.22463186)
\curveto(800.16474177,90.26462691)(800.1347418,90.30462687)(800.11473633,90.34463186)
\curveto(800.00474193,90.48462669)(799.87974205,90.60962657)(799.73973633,90.71963186)
\curveto(799.70974222,90.73962644)(799.68474225,90.76462641)(799.66473633,90.79463186)
\curveto(799.6347423,90.82462635)(799.60474233,90.84962633)(799.57473633,90.86963186)
\curveto(799.47474246,90.94962623)(799.37474256,91.01462616)(799.27473633,91.06463186)
\curveto(799.17474276,91.12462605)(799.06474287,91.179626)(798.94473633,91.22963186)
\curveto(798.87474306,91.25962592)(798.79974313,91.2796259)(798.71973633,91.28963186)
\lineto(798.47973633,91.34963186)
\lineto(798.38973633,91.34963186)
\curveto(798.35974357,91.35962582)(798.3297436,91.36462581)(798.29973633,91.36463186)
\curveto(798.2297437,91.38462579)(798.1347438,91.38962579)(798.01473633,91.37963186)
\curveto(797.88474405,91.3796258)(797.78474415,91.36962581)(797.71473633,91.34963186)
\curveto(797.6347443,91.32962585)(797.55974437,91.30962587)(797.48973633,91.28963186)
\curveto(797.40974452,91.2796259)(797.3297446,91.25962592)(797.24973633,91.22963186)
\curveto(797.00974492,91.11962606)(796.80974512,90.96962621)(796.64973633,90.77963186)
\curveto(796.47974545,90.59962658)(796.33974559,90.3796268)(796.22973633,90.11963186)
\curveto(796.20974572,90.04962713)(796.19474574,89.9796272)(796.18473633,89.90963186)
\curveto(796.16474577,89.83962734)(796.14474579,89.76462741)(796.12473633,89.68463186)
\curveto(796.10474583,89.60462757)(796.09474584,89.49462768)(796.09473633,89.35463186)
\curveto(796.09474584,89.22462795)(796.10474583,89.11962806)(796.12473633,89.03963186)
\curveto(796.1347458,88.9796282)(796.13974579,88.92462825)(796.13973633,88.87463186)
\curveto(796.13974579,88.82462835)(796.14974578,88.7746284)(796.16973633,88.72463186)
\curveto(796.20974572,88.62462855)(796.24974568,88.52962865)(796.28973633,88.43963186)
\curveto(796.3297456,88.35962882)(796.37474556,88.2796289)(796.42473633,88.19963186)
\curveto(796.44474549,88.16962901)(796.46974546,88.13962904)(796.49973633,88.10963186)
\curveto(796.5297454,88.08962909)(796.55474538,88.06462911)(796.57473633,88.03463186)
\lineto(796.64973633,87.95963186)
\curveto(796.66974526,87.92962925)(796.68974524,87.90462927)(796.70973633,87.88463186)
\lineto(796.91973633,87.73463186)
\curveto(796.97974495,87.69462948)(797.04474489,87.64962953)(797.11473633,87.59963186)
\curveto(797.20474473,87.53962964)(797.30974462,87.48962969)(797.42973633,87.44963186)
\curveto(797.53974439,87.41962976)(797.64974428,87.38462979)(797.75973633,87.34463186)
\curveto(797.86974406,87.30462987)(798.01474392,87.2796299)(798.19473633,87.26963186)
\curveto(798.36474357,87.25962992)(798.48974344,87.22962995)(798.56973633,87.17963186)
\curveto(798.64974328,87.12963005)(798.69474324,87.05463012)(798.70473633,86.95463186)
\curveto(798.71474322,86.85463032)(798.71974321,86.74463043)(798.71973633,86.62463186)
\curveto(798.71974321,86.58463059)(798.72474321,86.54463063)(798.73473633,86.50463186)
\curveto(798.7347432,86.46463071)(798.7297432,86.42963075)(798.71973633,86.39963186)
\curveto(798.69974323,86.34963083)(798.68974324,86.29963088)(798.68973633,86.24963186)
\curveto(798.68974324,86.20963097)(798.67974325,86.16963101)(798.65973633,86.12963186)
\curveto(798.59974333,86.03963114)(798.46474347,85.99463118)(798.25473633,85.99463186)
\lineto(798.13473633,85.99463186)
\curveto(798.07474386,86.00463117)(798.01474392,86.00963117)(797.95473633,86.00963186)
\curveto(797.88474405,86.01963116)(797.81974411,86.02963115)(797.75973633,86.03963186)
\curveto(797.64974428,86.05963112)(797.54974438,86.0796311)(797.45973633,86.09963186)
\curveto(797.35974457,86.11963106)(797.26474467,86.14963103)(797.17473633,86.18963186)
\curveto(797.10474483,86.20963097)(797.04474489,86.22963095)(796.99473633,86.24963186)
\lineto(796.81473633,86.30963186)
\curveto(796.55474538,86.42963075)(796.30974562,86.58463059)(796.07973633,86.77463186)
\curveto(795.84974608,86.9746302)(795.66474627,87.18962999)(795.52473633,87.41963186)
\curveto(795.44474649,87.52962965)(795.37974655,87.64462953)(795.32973633,87.76463186)
\lineto(795.17973633,88.15463186)
\curveto(795.1297468,88.26462891)(795.09974683,88.3796288)(795.08973633,88.49963186)
\curveto(795.06974686,88.61962856)(795.04474689,88.74462843)(795.01473633,88.87463186)
\curveto(795.01474692,88.94462823)(795.01474692,89.00962817)(795.01473633,89.06963186)
\curveto(795.00474693,89.12962805)(794.99474694,89.19462798)(794.98473633,89.26463186)
}
}
{
\newrgbcolor{curcolor}{0 0 0}
\pscustom[linestyle=none,fillstyle=solid,fillcolor=curcolor]
{
\newpath
\moveto(800.50473633,101.36424123)
\lineto(800.75973633,101.36424123)
\curveto(800.83974109,101.37423353)(800.91474102,101.36923353)(800.98473633,101.34924123)
\lineto(801.22473633,101.34924123)
\lineto(801.38973633,101.34924123)
\curveto(801.48974044,101.32923357)(801.59474034,101.31923358)(801.70473633,101.31924123)
\curveto(801.80474013,101.31923358)(801.90474003,101.30923359)(802.00473633,101.28924123)
\lineto(802.15473633,101.28924123)
\curveto(802.29473964,101.25923364)(802.4347395,101.23923366)(802.57473633,101.22924123)
\curveto(802.70473923,101.21923368)(802.8347391,101.19423371)(802.96473633,101.15424123)
\curveto(803.04473889,101.13423377)(803.1297388,101.11423379)(803.21973633,101.09424123)
\lineto(803.45973633,101.03424123)
\lineto(803.75973633,100.91424123)
\curveto(803.84973808,100.88423402)(803.93973799,100.84923405)(804.02973633,100.80924123)
\curveto(804.24973768,100.70923419)(804.46473747,100.57423433)(804.67473633,100.40424123)
\curveto(804.88473705,100.24423466)(805.05473688,100.06923483)(805.18473633,99.87924123)
\curveto(805.22473671,99.82923507)(805.26473667,99.76923513)(805.30473633,99.69924123)
\curveto(805.3347366,99.63923526)(805.36973656,99.57923532)(805.40973633,99.51924123)
\curveto(805.45973647,99.43923546)(805.49973643,99.34423556)(805.52973633,99.23424123)
\curveto(805.55973637,99.12423578)(805.58973634,99.01923588)(805.61973633,98.91924123)
\curveto(805.65973627,98.80923609)(805.68473625,98.6992362)(805.69473633,98.58924123)
\curveto(805.70473623,98.47923642)(805.71973621,98.36423654)(805.73973633,98.24424123)
\curveto(805.74973618,98.2042367)(805.74973618,98.15923674)(805.73973633,98.10924123)
\curveto(805.73973619,98.06923683)(805.74473619,98.02923687)(805.75473633,97.98924123)
\curveto(805.76473617,97.94923695)(805.76973616,97.89423701)(805.76973633,97.82424123)
\curveto(805.76973616,97.75423715)(805.76473617,97.7042372)(805.75473633,97.67424123)
\curveto(805.7347362,97.62423728)(805.7297362,97.57923732)(805.73973633,97.53924123)
\curveto(805.74973618,97.4992374)(805.74973618,97.46423744)(805.73973633,97.43424123)
\lineto(805.73973633,97.34424123)
\curveto(805.71973621,97.28423762)(805.70473623,97.21923768)(805.69473633,97.14924123)
\curveto(805.69473624,97.08923781)(805.68973624,97.02423788)(805.67973633,96.95424123)
\curveto(805.6297363,96.78423812)(805.57973635,96.62423828)(805.52973633,96.47424123)
\curveto(805.47973645,96.32423858)(805.41473652,96.17923872)(805.33473633,96.03924123)
\curveto(805.29473664,95.98923891)(805.26473667,95.93423897)(805.24473633,95.87424123)
\curveto(805.21473672,95.82423908)(805.17973675,95.77423913)(805.13973633,95.72424123)
\curveto(804.95973697,95.48423942)(804.73973719,95.28423962)(804.47973633,95.12424123)
\curveto(804.21973771,94.96423994)(803.934738,94.82424008)(803.62473633,94.70424123)
\curveto(803.48473845,94.64424026)(803.34473859,94.5992403)(803.20473633,94.56924123)
\curveto(803.05473888,94.53924036)(802.89973903,94.5042404)(802.73973633,94.46424123)
\curveto(802.6297393,94.44424046)(802.51973941,94.42924047)(802.40973633,94.41924123)
\curveto(802.29973963,94.40924049)(802.18973974,94.39424051)(802.07973633,94.37424123)
\curveto(802.03973989,94.36424054)(801.99973993,94.35924054)(801.95973633,94.35924123)
\curveto(801.91974001,94.36924053)(801.87974005,94.36924053)(801.83973633,94.35924123)
\curveto(801.78974014,94.34924055)(801.73974019,94.34424056)(801.68973633,94.34424123)
\lineto(801.52473633,94.34424123)
\curveto(801.47474046,94.32424058)(801.42474051,94.31924058)(801.37473633,94.32924123)
\curveto(801.31474062,94.33924056)(801.25974067,94.33924056)(801.20973633,94.32924123)
\curveto(801.16974076,94.31924058)(801.12474081,94.31924058)(801.07473633,94.32924123)
\curveto(801.02474091,94.33924056)(800.97474096,94.33424057)(800.92473633,94.31424123)
\curveto(800.85474108,94.29424061)(800.77974115,94.28924061)(800.69973633,94.29924123)
\curveto(800.60974132,94.30924059)(800.52474141,94.31424059)(800.44473633,94.31424123)
\curveto(800.35474158,94.31424059)(800.25474168,94.30924059)(800.14473633,94.29924123)
\curveto(800.02474191,94.28924061)(799.92474201,94.29424061)(799.84473633,94.31424123)
\lineto(799.55973633,94.31424123)
\lineto(798.92973633,94.35924123)
\curveto(798.8297431,94.36924053)(798.7347432,94.37924052)(798.64473633,94.38924123)
\lineto(798.34473633,94.41924123)
\curveto(798.29474364,94.43924046)(798.24474369,94.44424046)(798.19473633,94.43424123)
\curveto(798.1347438,94.43424047)(798.07974385,94.44424046)(798.02973633,94.46424123)
\curveto(797.85974407,94.51424039)(797.69474424,94.55424035)(797.53473633,94.58424123)
\curveto(797.36474457,94.61424029)(797.20474473,94.66424024)(797.05473633,94.73424123)
\curveto(796.59474534,94.92423998)(796.21974571,95.14423976)(795.92973633,95.39424123)
\curveto(795.63974629,95.65423925)(795.39474654,96.01423889)(795.19473633,96.47424123)
\curveto(795.14474679,96.6042383)(795.10974682,96.73423817)(795.08973633,96.86424123)
\curveto(795.06974686,97.0042379)(795.04474689,97.14423776)(795.01473633,97.28424123)
\curveto(795.00474693,97.35423755)(794.99974693,97.41923748)(794.99973633,97.47924123)
\curveto(794.99974693,97.53923736)(794.99474694,97.6042373)(794.98473633,97.67424123)
\curveto(794.96474697,98.5042364)(795.11474682,99.17423573)(795.43473633,99.68424123)
\curveto(795.74474619,100.19423471)(796.18474575,100.57423433)(796.75473633,100.82424123)
\curveto(796.87474506,100.87423403)(796.99974493,100.91923398)(797.12973633,100.95924123)
\curveto(797.25974467,100.9992339)(797.39474454,101.04423386)(797.53473633,101.09424123)
\curveto(797.61474432,101.11423379)(797.69974423,101.12923377)(797.78973633,101.13924123)
\lineto(798.02973633,101.19924123)
\curveto(798.13974379,101.22923367)(798.24974368,101.24423366)(798.35973633,101.24424123)
\curveto(798.46974346,101.25423365)(798.57974335,101.26923363)(798.68973633,101.28924123)
\curveto(798.73974319,101.30923359)(798.78474315,101.31423359)(798.82473633,101.30424123)
\curveto(798.86474307,101.3042336)(798.90474303,101.30923359)(798.94473633,101.31924123)
\curveto(798.99474294,101.32923357)(799.04974288,101.32923357)(799.10973633,101.31924123)
\curveto(799.15974277,101.31923358)(799.20974272,101.32423358)(799.25973633,101.33424123)
\lineto(799.39473633,101.33424123)
\curveto(799.45474248,101.35423355)(799.52474241,101.35423355)(799.60473633,101.33424123)
\curveto(799.67474226,101.32423358)(799.73974219,101.32923357)(799.79973633,101.34924123)
\curveto(799.8297421,101.35923354)(799.86974206,101.36423354)(799.91973633,101.36424123)
\lineto(800.03973633,101.36424123)
\lineto(800.50473633,101.36424123)
\moveto(802.82973633,99.81924123)
\curveto(802.50973942,99.91923498)(802.14473979,99.97923492)(801.73473633,99.99924123)
\curveto(801.32474061,100.01923488)(800.91474102,100.02923487)(800.50473633,100.02924123)
\curveto(800.07474186,100.02923487)(799.65474228,100.01923488)(799.24473633,99.99924123)
\curveto(798.8347431,99.97923492)(798.44974348,99.93423497)(798.08973633,99.86424123)
\curveto(797.7297442,99.79423511)(797.40974452,99.68423522)(797.12973633,99.53424123)
\curveto(796.83974509,99.39423551)(796.60474533,99.1992357)(796.42473633,98.94924123)
\curveto(796.31474562,98.78923611)(796.2347457,98.60923629)(796.18473633,98.40924123)
\curveto(796.12474581,98.20923669)(796.09474584,97.96423694)(796.09473633,97.67424123)
\curveto(796.11474582,97.65423725)(796.12474581,97.61923728)(796.12473633,97.56924123)
\curveto(796.11474582,97.51923738)(796.11474582,97.47923742)(796.12473633,97.44924123)
\curveto(796.14474579,97.36923753)(796.16474577,97.29423761)(796.18473633,97.22424123)
\curveto(796.19474574,97.16423774)(796.21474572,97.0992378)(796.24473633,97.02924123)
\curveto(796.36474557,96.75923814)(796.5347454,96.53923836)(796.75473633,96.36924123)
\curveto(796.96474497,96.20923869)(797.20974472,96.07423883)(797.48973633,95.96424123)
\curveto(797.59974433,95.91423899)(797.71974421,95.87423903)(797.84973633,95.84424123)
\curveto(797.96974396,95.82423908)(798.09474384,95.7992391)(798.22473633,95.76924123)
\curveto(798.27474366,95.74923915)(798.3297436,95.73923916)(798.38973633,95.73924123)
\curveto(798.43974349,95.73923916)(798.48974344,95.73423917)(798.53973633,95.72424123)
\curveto(798.6297433,95.71423919)(798.72474321,95.7042392)(798.82473633,95.69424123)
\curveto(798.91474302,95.68423922)(799.00974292,95.67423923)(799.10973633,95.66424123)
\curveto(799.18974274,95.66423924)(799.27474266,95.65923924)(799.36473633,95.64924123)
\lineto(799.60473633,95.64924123)
\lineto(799.78473633,95.64924123)
\curveto(799.81474212,95.63923926)(799.84974208,95.63423927)(799.88973633,95.63424123)
\lineto(800.02473633,95.63424123)
\lineto(800.47473633,95.63424123)
\curveto(800.55474138,95.63423927)(800.63974129,95.62923927)(800.72973633,95.61924123)
\curveto(800.80974112,95.61923928)(800.88474105,95.62923927)(800.95473633,95.64924123)
\lineto(801.22473633,95.64924123)
\curveto(801.24474069,95.64923925)(801.27474066,95.64423926)(801.31473633,95.63424123)
\curveto(801.34474059,95.63423927)(801.36974056,95.63923926)(801.38973633,95.64924123)
\curveto(801.48974044,95.65923924)(801.58974034,95.66423924)(801.68973633,95.66424123)
\curveto(801.77974015,95.67423923)(801.87974005,95.68423922)(801.98973633,95.69424123)
\curveto(802.10973982,95.72423918)(802.2347397,95.73923916)(802.36473633,95.73924123)
\curveto(802.48473945,95.74923915)(802.59973933,95.77423913)(802.70973633,95.81424123)
\curveto(803.00973892,95.89423901)(803.27473866,95.97923892)(803.50473633,96.06924123)
\curveto(803.7347382,96.16923873)(803.94973798,96.31423859)(804.14973633,96.50424123)
\curveto(804.34973758,96.71423819)(804.49973743,96.97923792)(804.59973633,97.29924123)
\curveto(804.61973731,97.33923756)(804.6297373,97.37423753)(804.62973633,97.40424123)
\curveto(804.61973731,97.44423746)(804.62473731,97.48923741)(804.64473633,97.53924123)
\curveto(804.65473728,97.57923732)(804.66473727,97.64923725)(804.67473633,97.74924123)
\curveto(804.68473725,97.85923704)(804.67973725,97.94423696)(804.65973633,98.00424123)
\curveto(804.63973729,98.07423683)(804.6297373,98.14423676)(804.62973633,98.21424123)
\curveto(804.61973731,98.28423662)(804.60473733,98.34923655)(804.58473633,98.40924123)
\curveto(804.52473741,98.60923629)(804.43973749,98.78923611)(804.32973633,98.94924123)
\curveto(804.30973762,98.97923592)(804.28973764,99.0042359)(804.26973633,99.02424123)
\lineto(804.20973633,99.08424123)
\curveto(804.18973774,99.12423578)(804.14973778,99.17423573)(804.08973633,99.23424123)
\curveto(803.94973798,99.33423557)(803.81973811,99.41923548)(803.69973633,99.48924123)
\curveto(803.57973835,99.55923534)(803.4347385,99.62923527)(803.26473633,99.69924123)
\curveto(803.19473874,99.72923517)(803.12473881,99.74923515)(803.05473633,99.75924123)
\curveto(802.98473895,99.77923512)(802.90973902,99.7992351)(802.82973633,99.81924123)
}
}
{
\newrgbcolor{curcolor}{0 0 0}
\pscustom[linestyle=none,fillstyle=solid,fillcolor=curcolor]
{
\newpath
\moveto(794.98473633,106.77385061)
\curveto(794.98474695,106.87384575)(794.99474694,106.96884566)(795.01473633,107.05885061)
\curveto(795.02474691,107.14884548)(795.05474688,107.21384541)(795.10473633,107.25385061)
\curveto(795.18474675,107.31384531)(795.28974664,107.34384528)(795.41973633,107.34385061)
\lineto(795.80973633,107.34385061)
\lineto(797.30973633,107.34385061)
\lineto(803.69973633,107.34385061)
\lineto(804.86973633,107.34385061)
\lineto(805.18473633,107.34385061)
\curveto(805.28473665,107.35384527)(805.36473657,107.33884529)(805.42473633,107.29885061)
\curveto(805.50473643,107.24884538)(805.55473638,107.17384545)(805.57473633,107.07385061)
\curveto(805.58473635,106.98384564)(805.58973634,106.87384575)(805.58973633,106.74385061)
\lineto(805.58973633,106.51885061)
\curveto(805.56973636,106.43884619)(805.55473638,106.36884626)(805.54473633,106.30885061)
\curveto(805.52473641,106.24884638)(805.48473645,106.19884643)(805.42473633,106.15885061)
\curveto(805.36473657,106.11884651)(805.28973664,106.09884653)(805.19973633,106.09885061)
\lineto(804.89973633,106.09885061)
\lineto(803.80473633,106.09885061)
\lineto(798.46473633,106.09885061)
\curveto(798.37474356,106.07884655)(798.29974363,106.06384656)(798.23973633,106.05385061)
\curveto(798.16974376,106.05384657)(798.10974382,106.0238466)(798.05973633,105.96385061)
\curveto(798.00974392,105.89384673)(797.98474395,105.80384682)(797.98473633,105.69385061)
\curveto(797.97474396,105.59384703)(797.96974396,105.48384714)(797.96973633,105.36385061)
\lineto(797.96973633,104.22385061)
\lineto(797.96973633,103.72885061)
\curveto(797.95974397,103.56884906)(797.89974403,103.45884917)(797.78973633,103.39885061)
\curveto(797.75974417,103.37884925)(797.7297442,103.36884926)(797.69973633,103.36885061)
\curveto(797.65974427,103.36884926)(797.61474432,103.36384926)(797.56473633,103.35385061)
\curveto(797.44474449,103.33384929)(797.3347446,103.33884929)(797.23473633,103.36885061)
\curveto(797.1347448,103.40884922)(797.06474487,103.46384916)(797.02473633,103.53385061)
\curveto(796.97474496,103.61384901)(796.94974498,103.73384889)(796.94973633,103.89385061)
\curveto(796.94974498,104.05384857)(796.934745,104.18884844)(796.90473633,104.29885061)
\curveto(796.89474504,104.34884828)(796.88974504,104.40384822)(796.88973633,104.46385061)
\curveto(796.87974505,104.5238481)(796.86474507,104.58384804)(796.84473633,104.64385061)
\curveto(796.79474514,104.79384783)(796.74474519,104.93884769)(796.69473633,105.07885061)
\curveto(796.6347453,105.21884741)(796.56474537,105.35384727)(796.48473633,105.48385061)
\curveto(796.39474554,105.623847)(796.28974564,105.74384688)(796.16973633,105.84385061)
\curveto(796.04974588,105.94384668)(795.91974601,106.03884659)(795.77973633,106.12885061)
\curveto(795.67974625,106.18884644)(795.56974636,106.23384639)(795.44973633,106.26385061)
\curveto(795.3297466,106.30384632)(795.22474671,106.35384627)(795.13473633,106.41385061)
\curveto(795.07474686,106.46384616)(795.0347469,106.53384609)(795.01473633,106.62385061)
\curveto(795.00474693,106.64384598)(794.99974693,106.66884596)(794.99973633,106.69885061)
\curveto(794.99974693,106.7288459)(794.99474694,106.75384587)(794.98473633,106.77385061)
}
}
{
\newrgbcolor{curcolor}{0 0 0}
\pscustom[linestyle=none,fillstyle=solid,fillcolor=curcolor]
{
\newpath
\moveto(794.98473633,115.12345998)
\curveto(794.98474695,115.22345513)(794.99474694,115.31845503)(795.01473633,115.40845998)
\curveto(795.02474691,115.49845485)(795.05474688,115.56345479)(795.10473633,115.60345998)
\curveto(795.18474675,115.66345469)(795.28974664,115.69345466)(795.41973633,115.69345998)
\lineto(795.80973633,115.69345998)
\lineto(797.30973633,115.69345998)
\lineto(803.69973633,115.69345998)
\lineto(804.86973633,115.69345998)
\lineto(805.18473633,115.69345998)
\curveto(805.28473665,115.70345465)(805.36473657,115.68845466)(805.42473633,115.64845998)
\curveto(805.50473643,115.59845475)(805.55473638,115.52345483)(805.57473633,115.42345998)
\curveto(805.58473635,115.33345502)(805.58973634,115.22345513)(805.58973633,115.09345998)
\lineto(805.58973633,114.86845998)
\curveto(805.56973636,114.78845556)(805.55473638,114.71845563)(805.54473633,114.65845998)
\curveto(805.52473641,114.59845575)(805.48473645,114.5484558)(805.42473633,114.50845998)
\curveto(805.36473657,114.46845588)(805.28973664,114.4484559)(805.19973633,114.44845998)
\lineto(804.89973633,114.44845998)
\lineto(803.80473633,114.44845998)
\lineto(798.46473633,114.44845998)
\curveto(798.37474356,114.42845592)(798.29974363,114.41345594)(798.23973633,114.40345998)
\curveto(798.16974376,114.40345595)(798.10974382,114.37345598)(798.05973633,114.31345998)
\curveto(798.00974392,114.24345611)(797.98474395,114.1534562)(797.98473633,114.04345998)
\curveto(797.97474396,113.94345641)(797.96974396,113.83345652)(797.96973633,113.71345998)
\lineto(797.96973633,112.57345998)
\lineto(797.96973633,112.07845998)
\curveto(797.95974397,111.91845843)(797.89974403,111.80845854)(797.78973633,111.74845998)
\curveto(797.75974417,111.72845862)(797.7297442,111.71845863)(797.69973633,111.71845998)
\curveto(797.65974427,111.71845863)(797.61474432,111.71345864)(797.56473633,111.70345998)
\curveto(797.44474449,111.68345867)(797.3347446,111.68845866)(797.23473633,111.71845998)
\curveto(797.1347448,111.75845859)(797.06474487,111.81345854)(797.02473633,111.88345998)
\curveto(796.97474496,111.96345839)(796.94974498,112.08345827)(796.94973633,112.24345998)
\curveto(796.94974498,112.40345795)(796.934745,112.53845781)(796.90473633,112.64845998)
\curveto(796.89474504,112.69845765)(796.88974504,112.7534576)(796.88973633,112.81345998)
\curveto(796.87974505,112.87345748)(796.86474507,112.93345742)(796.84473633,112.99345998)
\curveto(796.79474514,113.14345721)(796.74474519,113.28845706)(796.69473633,113.42845998)
\curveto(796.6347453,113.56845678)(796.56474537,113.70345665)(796.48473633,113.83345998)
\curveto(796.39474554,113.97345638)(796.28974564,114.09345626)(796.16973633,114.19345998)
\curveto(796.04974588,114.29345606)(795.91974601,114.38845596)(795.77973633,114.47845998)
\curveto(795.67974625,114.53845581)(795.56974636,114.58345577)(795.44973633,114.61345998)
\curveto(795.3297466,114.6534557)(795.22474671,114.70345565)(795.13473633,114.76345998)
\curveto(795.07474686,114.81345554)(795.0347469,114.88345547)(795.01473633,114.97345998)
\curveto(795.00474693,114.99345536)(794.99974693,115.01845533)(794.99973633,115.04845998)
\curveto(794.99974693,115.07845527)(794.99474694,115.10345525)(794.98473633,115.12345998)
}
}
{
\newrgbcolor{curcolor}{0 0 0}
\pscustom[linestyle=none,fillstyle=solid,fillcolor=curcolor]
{
\newpath
\moveto(815.82105225,42.29681936)
\curveto(815.82106294,42.36681368)(815.82106294,42.4468136)(815.82105225,42.53681936)
\curveto(815.81106295,42.62681342)(815.81106295,42.71181333)(815.82105225,42.79181936)
\curveto(815.82106294,42.88181316)(815.83106293,42.96181308)(815.85105225,43.03181936)
\curveto(815.87106289,43.11181293)(815.90106286,43.16681288)(815.94105225,43.19681936)
\curveto(815.99106277,43.22681282)(816.0660627,43.2468128)(816.16605225,43.25681936)
\curveto(816.25606251,43.27681277)(816.3610624,43.28681276)(816.48105225,43.28681936)
\curveto(816.59106217,43.29681275)(816.70606206,43.29681275)(816.82605225,43.28681936)
\lineto(817.12605225,43.28681936)
\lineto(820.14105225,43.28681936)
\lineto(823.03605225,43.28681936)
\curveto(823.3660554,43.28681276)(823.69105507,43.28181276)(824.01105225,43.27181936)
\curveto(824.32105444,43.27181277)(824.60105416,43.23181281)(824.85105225,43.15181936)
\curveto(825.20105356,43.03181301)(825.49605327,42.87681317)(825.73605225,42.68681936)
\curveto(825.9660528,42.49681355)(826.1660526,42.25681379)(826.33605225,41.96681936)
\curveto(826.38605238,41.90681414)(826.42105234,41.8418142)(826.44105225,41.77181936)
\curveto(826.4610523,41.71181433)(826.48605228,41.6418144)(826.51605225,41.56181936)
\curveto(826.5660522,41.4418146)(826.60105216,41.31181473)(826.62105225,41.17181936)
\curveto(826.65105211,41.041815)(826.68105208,40.90681514)(826.71105225,40.76681936)
\curveto(826.73105203,40.71681533)(826.73605203,40.66681538)(826.72605225,40.61681936)
\curveto(826.71605205,40.56681548)(826.71605205,40.51181553)(826.72605225,40.45181936)
\curveto(826.73605203,40.43181561)(826.73605203,40.40681564)(826.72605225,40.37681936)
\curveto(826.72605204,40.3468157)(826.73105203,40.32181572)(826.74105225,40.30181936)
\curveto(826.75105201,40.26181578)(826.75605201,40.20681584)(826.75605225,40.13681936)
\curveto(826.75605201,40.06681598)(826.75105201,40.01181603)(826.74105225,39.97181936)
\curveto(826.73105203,39.92181612)(826.73105203,39.86681618)(826.74105225,39.80681936)
\curveto(826.75105201,39.7468163)(826.74605202,39.69181635)(826.72605225,39.64181936)
\curveto(826.69605207,39.51181653)(826.67605209,39.38681666)(826.66605225,39.26681936)
\curveto(826.65605211,39.1468169)(826.63105213,39.03181701)(826.59105225,38.92181936)
\curveto(826.47105229,38.55181749)(826.30105246,38.23181781)(826.08105225,37.96181936)
\curveto(825.8610529,37.69181835)(825.58105318,37.48181856)(825.24105225,37.33181936)
\curveto(825.12105364,37.28181876)(824.99605377,37.23681881)(824.86605225,37.19681936)
\curveto(824.73605403,37.16681888)(824.60105416,37.13181891)(824.46105225,37.09181936)
\curveto(824.41105435,37.08181896)(824.37105439,37.07681897)(824.34105225,37.07681936)
\curveto(824.30105446,37.07681897)(824.25605451,37.07181897)(824.20605225,37.06181936)
\curveto(824.17605459,37.05181899)(824.14105462,37.046819)(824.10105225,37.04681936)
\curveto(824.05105471,37.046819)(824.01105475,37.041819)(823.98105225,37.03181936)
\lineto(823.81605225,37.03181936)
\curveto(823.73605503,37.01181903)(823.63605513,37.00681904)(823.51605225,37.01681936)
\curveto(823.38605538,37.02681902)(823.29605547,37.041819)(823.24605225,37.06181936)
\curveto(823.15605561,37.08181896)(823.09105567,37.13681891)(823.05105225,37.22681936)
\curveto(823.03105573,37.25681879)(823.02605574,37.28681876)(823.03605225,37.31681936)
\curveto(823.03605573,37.3468187)(823.03105573,37.38681866)(823.02105225,37.43681936)
\curveto(823.01105575,37.47681857)(823.00605576,37.51681853)(823.00605225,37.55681936)
\lineto(823.00605225,37.70681936)
\curveto(823.00605576,37.82681822)(823.01105575,37.9468181)(823.02105225,38.06681936)
\curveto(823.02105574,38.19681785)(823.05605571,38.28681776)(823.12605225,38.33681936)
\curveto(823.18605558,38.37681767)(823.24605552,38.39681765)(823.30605225,38.39681936)
\curveto(823.3660554,38.39681765)(823.43605533,38.40681764)(823.51605225,38.42681936)
\curveto(823.54605522,38.43681761)(823.58105518,38.43681761)(823.62105225,38.42681936)
\curveto(823.65105511,38.42681762)(823.67605509,38.43181761)(823.69605225,38.44181936)
\lineto(823.90605225,38.44181936)
\curveto(823.95605481,38.46181758)(824.00605476,38.46681758)(824.05605225,38.45681936)
\curveto(824.09605467,38.45681759)(824.14105462,38.46681758)(824.19105225,38.48681936)
\curveto(824.32105444,38.51681753)(824.44605432,38.5468175)(824.56605225,38.57681936)
\curveto(824.67605409,38.60681744)(824.78105398,38.65181739)(824.88105225,38.71181936)
\curveto(825.17105359,38.88181716)(825.37605339,39.15181689)(825.49605225,39.52181936)
\curveto(825.51605325,39.57181647)(825.53105323,39.62181642)(825.54105225,39.67181936)
\curveto(825.54105322,39.73181631)(825.55105321,39.78681626)(825.57105225,39.83681936)
\lineto(825.57105225,39.91181936)
\curveto(825.58105318,39.98181606)(825.59105317,40.07681597)(825.60105225,40.19681936)
\curveto(825.60105316,40.32681572)(825.59105317,40.42681562)(825.57105225,40.49681936)
\curveto(825.55105321,40.56681548)(825.53605323,40.63681541)(825.52605225,40.70681936)
\curveto(825.50605326,40.78681526)(825.48605328,40.85681519)(825.46605225,40.91681936)
\curveto(825.30605346,41.29681475)(825.03105373,41.57181447)(824.64105225,41.74181936)
\curveto(824.51105425,41.79181425)(824.35605441,41.82681422)(824.17605225,41.84681936)
\curveto(823.99605477,41.87681417)(823.81105495,41.89181415)(823.62105225,41.89181936)
\curveto(823.42105534,41.90181414)(823.22105554,41.90181414)(823.02105225,41.89181936)
\lineto(822.45105225,41.89181936)
\lineto(818.20605225,41.89181936)
\lineto(816.66105225,41.89181936)
\curveto(816.55106221,41.89181415)(816.43106233,41.88681416)(816.30105225,41.87681936)
\curveto(816.17106259,41.86681418)(816.0660627,41.88681416)(815.98605225,41.93681936)
\curveto(815.91606285,41.99681405)(815.8660629,42.07681397)(815.83605225,42.17681936)
\curveto(815.83606293,42.19681385)(815.83606293,42.21681383)(815.83605225,42.23681936)
\curveto(815.83606293,42.25681379)(815.83106293,42.27681377)(815.82105225,42.29681936)
}
}
{
\newrgbcolor{curcolor}{0 0 0}
\pscustom[linestyle=none,fillstyle=solid,fillcolor=curcolor]
{
\newpath
\moveto(818.77605225,45.83049123)
\lineto(818.77605225,46.26549123)
\curveto(818.77605999,46.41548927)(818.81605995,46.52048916)(818.89605225,46.58049123)
\curveto(818.97605979,46.63048905)(819.07605969,46.65548903)(819.19605225,46.65549123)
\curveto(819.31605945,46.66548902)(819.43605933,46.67048901)(819.55605225,46.67049123)
\lineto(820.98105225,46.67049123)
\lineto(823.24605225,46.67049123)
\lineto(823.93605225,46.67049123)
\curveto(824.1660546,46.67048901)(824.3660544,46.69548899)(824.53605225,46.74549123)
\curveto(824.98605378,46.90548878)(825.30105346,47.20548848)(825.48105225,47.64549123)
\curveto(825.57105319,47.86548782)(825.60605316,48.13048755)(825.58605225,48.44049123)
\curveto(825.55605321,48.75048693)(825.50105326,49.00048668)(825.42105225,49.19049123)
\curveto(825.28105348,49.52048616)(825.10605366,49.7804859)(824.89605225,49.97049123)
\curveto(824.67605409,50.17048551)(824.39105437,50.32548536)(824.04105225,50.43549123)
\curveto(823.9610548,50.46548522)(823.88105488,50.4854852)(823.80105225,50.49549123)
\curveto(823.72105504,50.50548518)(823.63605513,50.52048516)(823.54605225,50.54049123)
\curveto(823.49605527,50.55048513)(823.45105531,50.55048513)(823.41105225,50.54049123)
\curveto(823.37105539,50.54048514)(823.32605544,50.55048513)(823.27605225,50.57049123)
\lineto(822.96105225,50.57049123)
\curveto(822.88105588,50.59048509)(822.79105597,50.59548509)(822.69105225,50.58549123)
\curveto(822.58105618,50.57548511)(822.48105628,50.57048511)(822.39105225,50.57049123)
\lineto(821.22105225,50.57049123)
\lineto(819.63105225,50.57049123)
\curveto(819.51105925,50.57048511)(819.38605938,50.56548512)(819.25605225,50.55549123)
\curveto(819.11605965,50.55548513)(819.00605976,50.5804851)(818.92605225,50.63049123)
\curveto(818.87605989,50.67048501)(818.84605992,50.71548497)(818.83605225,50.76549123)
\curveto(818.81605995,50.82548486)(818.79605997,50.89548479)(818.77605225,50.97549123)
\lineto(818.77605225,51.20049123)
\curveto(818.77605999,51.32048436)(818.78105998,51.42548426)(818.79105225,51.51549123)
\curveto(818.80105996,51.61548407)(818.84605992,51.69048399)(818.92605225,51.74049123)
\curveto(818.97605979,51.79048389)(819.05105971,51.81548387)(819.15105225,51.81549123)
\lineto(819.43605225,51.81549123)
\lineto(820.45605225,51.81549123)
\lineto(824.49105225,51.81549123)
\lineto(825.84105225,51.81549123)
\curveto(825.9610528,51.81548387)(826.07605269,51.81048387)(826.18605225,51.80049123)
\curveto(826.28605248,51.80048388)(826.3610524,51.76548392)(826.41105225,51.69549123)
\curveto(826.44105232,51.65548403)(826.4660523,51.59548409)(826.48605225,51.51549123)
\curveto(826.49605227,51.43548425)(826.50605226,51.34548434)(826.51605225,51.24549123)
\curveto(826.51605225,51.15548453)(826.51105225,51.06548462)(826.50105225,50.97549123)
\curveto(826.49105227,50.89548479)(826.47105229,50.83548485)(826.44105225,50.79549123)
\curveto(826.40105236,50.74548494)(826.33605243,50.70048498)(826.24605225,50.66049123)
\curveto(826.20605256,50.65048503)(826.15105261,50.64048504)(826.08105225,50.63049123)
\curveto(826.01105275,50.63048505)(825.94605282,50.62548506)(825.88605225,50.61549123)
\curveto(825.81605295,50.60548508)(825.761053,50.5854851)(825.72105225,50.55549123)
\curveto(825.68105308,50.52548516)(825.6660531,50.4804852)(825.67605225,50.42049123)
\curveto(825.69605307,50.34048534)(825.75605301,50.26048542)(825.85605225,50.18049123)
\curveto(825.94605282,50.10048558)(826.01605275,50.02548566)(826.06605225,49.95549123)
\curveto(826.22605254,49.73548595)(826.3660524,49.4854862)(826.48605225,49.20549123)
\curveto(826.53605223,49.09548659)(826.5660522,48.9804867)(826.57605225,48.86049123)
\curveto(826.59605217,48.75048693)(826.62105214,48.63548705)(826.65105225,48.51549123)
\curveto(826.6610521,48.46548722)(826.6610521,48.41048727)(826.65105225,48.35049123)
\curveto(826.64105212,48.30048738)(826.64605212,48.25048743)(826.66605225,48.20049123)
\curveto(826.68605208,48.10048758)(826.68605208,48.01048767)(826.66605225,47.93049123)
\lineto(826.66605225,47.78049123)
\curveto(826.64605212,47.73048795)(826.63605213,47.67048801)(826.63605225,47.60049123)
\curveto(826.63605213,47.54048814)(826.63105213,47.4854882)(826.62105225,47.43549123)
\curveto(826.60105216,47.39548829)(826.59105217,47.35548833)(826.59105225,47.31549123)
\curveto(826.60105216,47.2854884)(826.59605217,47.24548844)(826.57605225,47.19549123)
\lineto(826.51605225,46.95549123)
\curveto(826.49605227,46.8854888)(826.4660523,46.81048887)(826.42605225,46.73049123)
\curveto(826.31605245,46.47048921)(826.17105259,46.25048943)(825.99105225,46.07049123)
\curveto(825.80105296,45.90048978)(825.57605319,45.76048992)(825.31605225,45.65049123)
\curveto(825.22605354,45.61049007)(825.13605363,45.5804901)(825.04605225,45.56049123)
\lineto(824.74605225,45.50049123)
\curveto(824.68605408,45.4804902)(824.63105413,45.47049021)(824.58105225,45.47049123)
\curveto(824.52105424,45.4804902)(824.45605431,45.47549021)(824.38605225,45.45549123)
\curveto(824.3660544,45.44549024)(824.34105442,45.44049024)(824.31105225,45.44049123)
\curveto(824.27105449,45.44049024)(824.23605453,45.43549025)(824.20605225,45.42549123)
\lineto(824.05605225,45.42549123)
\curveto(824.01605475,45.41549027)(823.97105479,45.41049027)(823.92105225,45.41049123)
\curveto(823.8610549,45.42049026)(823.80605496,45.42549026)(823.75605225,45.42549123)
\lineto(823.15605225,45.42549123)
\lineto(820.39605225,45.42549123)
\lineto(819.43605225,45.42549123)
\lineto(819.16605225,45.42549123)
\curveto(819.07605969,45.42549026)(819.00105976,45.44549024)(818.94105225,45.48549123)
\curveto(818.87105989,45.52549016)(818.82105994,45.60049008)(818.79105225,45.71049123)
\curveto(818.78105998,45.73048995)(818.78105998,45.75048993)(818.79105225,45.77049123)
\curveto(818.79105997,45.79048989)(818.78605998,45.81048987)(818.77605225,45.83049123)
}
}
{
\newrgbcolor{curcolor}{0 0 0}
\pscustom[linestyle=none,fillstyle=solid,fillcolor=curcolor]
{
\newpath
\moveto(815.82105225,54.28510061)
\curveto(815.82106294,54.41509899)(815.82106294,54.55009886)(815.82105225,54.69010061)
\curveto(815.82106294,54.84009857)(815.85606291,54.95009846)(815.92605225,55.02010061)
\curveto(815.99606277,55.07009834)(816.09106267,55.09509831)(816.21105225,55.09510061)
\curveto(816.32106244,55.1050983)(816.43606233,55.1100983)(816.55605225,55.11010061)
\lineto(817.89105225,55.11010061)
\lineto(823.96605225,55.11010061)
\lineto(825.64605225,55.11010061)
\lineto(826.03605225,55.11010061)
\curveto(826.17605259,55.1100983)(826.28605248,55.08509832)(826.36605225,55.03510061)
\curveto(826.41605235,55.0050984)(826.44605232,54.96009845)(826.45605225,54.90010061)
\curveto(826.4660523,54.85009856)(826.48105228,54.78509862)(826.50105225,54.70510061)
\lineto(826.50105225,54.49510061)
\lineto(826.50105225,54.18010061)
\curveto(826.49105227,54.08009933)(826.45605231,54.0050994)(826.39605225,53.95510061)
\curveto(826.31605245,53.9050995)(826.21605255,53.87509953)(826.09605225,53.86510061)
\lineto(825.72105225,53.86510061)
\lineto(824.34105225,53.86510061)
\lineto(818.10105225,53.86510061)
\lineto(816.63105225,53.86510061)
\curveto(816.52106224,53.86509954)(816.40606236,53.86009955)(816.28605225,53.85010061)
\curveto(816.15606261,53.85009956)(816.05606271,53.87509953)(815.98605225,53.92510061)
\curveto(815.92606284,53.96509944)(815.87606289,54.04009937)(815.83605225,54.15010061)
\curveto(815.82606294,54.17009924)(815.82606294,54.19009922)(815.83605225,54.21010061)
\curveto(815.83606293,54.24009917)(815.83106293,54.26509914)(815.82105225,54.28510061)
}
}
{
\newrgbcolor{curcolor}{0 0 0}
\pscustom[linestyle=none,fillstyle=solid,fillcolor=curcolor]
{
}
}
{
\newrgbcolor{curcolor}{0 0 0}
\pscustom[linestyle=none,fillstyle=solid,fillcolor=curcolor]
{
\newpath
\moveto(821.41605225,67.99510061)
\lineto(821.67105225,67.99510061)
\curveto(821.75105701,68.0050929)(821.82605694,68.00009291)(821.89605225,67.98010061)
\lineto(822.13605225,67.98010061)
\lineto(822.30105225,67.98010061)
\curveto(822.40105636,67.96009295)(822.50605626,67.95009296)(822.61605225,67.95010061)
\curveto(822.71605605,67.95009296)(822.81605595,67.94009297)(822.91605225,67.92010061)
\lineto(823.06605225,67.92010061)
\curveto(823.20605556,67.89009302)(823.34605542,67.87009304)(823.48605225,67.86010061)
\curveto(823.61605515,67.85009306)(823.74605502,67.82509308)(823.87605225,67.78510061)
\curveto(823.95605481,67.76509314)(824.04105472,67.74509316)(824.13105225,67.72510061)
\lineto(824.37105225,67.66510061)
\lineto(824.67105225,67.54510061)
\curveto(824.761054,67.51509339)(824.85105391,67.48009343)(824.94105225,67.44010061)
\curveto(825.1610536,67.34009357)(825.37605339,67.2050937)(825.58605225,67.03510061)
\curveto(825.79605297,66.87509403)(825.9660528,66.70009421)(826.09605225,66.51010061)
\curveto(826.13605263,66.46009445)(826.17605259,66.40009451)(826.21605225,66.33010061)
\curveto(826.24605252,66.27009464)(826.28105248,66.2100947)(826.32105225,66.15010061)
\curveto(826.37105239,66.07009484)(826.41105235,65.97509493)(826.44105225,65.86510061)
\curveto(826.47105229,65.75509515)(826.50105226,65.65009526)(826.53105225,65.55010061)
\curveto(826.57105219,65.44009547)(826.59605217,65.33009558)(826.60605225,65.22010061)
\curveto(826.61605215,65.1100958)(826.63105213,64.99509591)(826.65105225,64.87510061)
\curveto(826.6610521,64.83509607)(826.6610521,64.79009612)(826.65105225,64.74010061)
\curveto(826.65105211,64.70009621)(826.65605211,64.66009625)(826.66605225,64.62010061)
\curveto(826.67605209,64.58009633)(826.68105208,64.52509638)(826.68105225,64.45510061)
\curveto(826.68105208,64.38509652)(826.67605209,64.33509657)(826.66605225,64.30510061)
\curveto(826.64605212,64.25509665)(826.64105212,64.2100967)(826.65105225,64.17010061)
\curveto(826.6610521,64.13009678)(826.6610521,64.09509681)(826.65105225,64.06510061)
\lineto(826.65105225,63.97510061)
\curveto(826.63105213,63.91509699)(826.61605215,63.85009706)(826.60605225,63.78010061)
\curveto(826.60605216,63.72009719)(826.60105216,63.65509725)(826.59105225,63.58510061)
\curveto(826.54105222,63.41509749)(826.49105227,63.25509765)(826.44105225,63.10510061)
\curveto(826.39105237,62.95509795)(826.32605244,62.8100981)(826.24605225,62.67010061)
\curveto(826.20605256,62.62009829)(826.17605259,62.56509834)(826.15605225,62.50510061)
\curveto(826.12605264,62.45509845)(826.09105267,62.4050985)(826.05105225,62.35510061)
\curveto(825.87105289,62.11509879)(825.65105311,61.91509899)(825.39105225,61.75510061)
\curveto(825.13105363,61.59509931)(824.84605392,61.45509945)(824.53605225,61.33510061)
\curveto(824.39605437,61.27509963)(824.25605451,61.23009968)(824.11605225,61.20010061)
\curveto(823.9660548,61.17009974)(823.81105495,61.13509977)(823.65105225,61.09510061)
\curveto(823.54105522,61.07509983)(823.43105533,61.06009985)(823.32105225,61.05010061)
\curveto(823.21105555,61.04009987)(823.10105566,61.02509988)(822.99105225,61.00510061)
\curveto(822.95105581,60.99509991)(822.91105585,60.99009992)(822.87105225,60.99010061)
\curveto(822.83105593,61.00009991)(822.79105597,61.00009991)(822.75105225,60.99010061)
\curveto(822.70105606,60.98009993)(822.65105611,60.97509993)(822.60105225,60.97510061)
\lineto(822.43605225,60.97510061)
\curveto(822.38605638,60.95509995)(822.33605643,60.95009996)(822.28605225,60.96010061)
\curveto(822.22605654,60.97009994)(822.17105659,60.97009994)(822.12105225,60.96010061)
\curveto(822.08105668,60.95009996)(822.03605673,60.95009996)(821.98605225,60.96010061)
\curveto(821.93605683,60.97009994)(821.88605688,60.96509994)(821.83605225,60.94510061)
\curveto(821.766057,60.92509998)(821.69105707,60.92009999)(821.61105225,60.93010061)
\curveto(821.52105724,60.94009997)(821.43605733,60.94509996)(821.35605225,60.94510061)
\curveto(821.2660575,60.94509996)(821.1660576,60.94009997)(821.05605225,60.93010061)
\curveto(820.93605783,60.92009999)(820.83605793,60.92509998)(820.75605225,60.94510061)
\lineto(820.47105225,60.94510061)
\lineto(819.84105225,60.99010061)
\curveto(819.74105902,61.00009991)(819.64605912,61.0100999)(819.55605225,61.02010061)
\lineto(819.25605225,61.05010061)
\curveto(819.20605956,61.07009984)(819.15605961,61.07509983)(819.10605225,61.06510061)
\curveto(819.04605972,61.06509984)(818.99105977,61.07509983)(818.94105225,61.09510061)
\curveto(818.77105999,61.14509976)(818.60606016,61.18509972)(818.44605225,61.21510061)
\curveto(818.27606049,61.24509966)(818.11606065,61.29509961)(817.96605225,61.36510061)
\curveto(817.50606126,61.55509935)(817.13106163,61.77509913)(816.84105225,62.02510061)
\curveto(816.55106221,62.28509862)(816.30606246,62.64509826)(816.10605225,63.10510061)
\curveto(816.05606271,63.23509767)(816.02106274,63.36509754)(816.00105225,63.49510061)
\curveto(815.98106278,63.63509727)(815.95606281,63.77509713)(815.92605225,63.91510061)
\curveto(815.91606285,63.98509692)(815.91106285,64.05009686)(815.91105225,64.11010061)
\curveto(815.91106285,64.17009674)(815.90606286,64.23509667)(815.89605225,64.30510061)
\curveto(815.87606289,65.13509577)(816.02606274,65.8050951)(816.34605225,66.31510061)
\curveto(816.65606211,66.82509408)(817.09606167,67.2050937)(817.66605225,67.45510061)
\curveto(817.78606098,67.5050934)(817.91106085,67.55009336)(818.04105225,67.59010061)
\curveto(818.17106059,67.63009328)(818.30606046,67.67509323)(818.44605225,67.72510061)
\curveto(818.52606024,67.74509316)(818.61106015,67.76009315)(818.70105225,67.77010061)
\lineto(818.94105225,67.83010061)
\curveto(819.05105971,67.86009305)(819.1610596,67.87509303)(819.27105225,67.87510061)
\curveto(819.38105938,67.88509302)(819.49105927,67.90009301)(819.60105225,67.92010061)
\curveto(819.65105911,67.94009297)(819.69605907,67.94509296)(819.73605225,67.93510061)
\curveto(819.77605899,67.93509297)(819.81605895,67.94009297)(819.85605225,67.95010061)
\curveto(819.90605886,67.96009295)(819.9610588,67.96009295)(820.02105225,67.95010061)
\curveto(820.07105869,67.95009296)(820.12105864,67.95509295)(820.17105225,67.96510061)
\lineto(820.30605225,67.96510061)
\curveto(820.3660584,67.98509292)(820.43605833,67.98509292)(820.51605225,67.96510061)
\curveto(820.58605818,67.95509295)(820.65105811,67.96009295)(820.71105225,67.98010061)
\curveto(820.74105802,67.99009292)(820.78105798,67.99509291)(820.83105225,67.99510061)
\lineto(820.95105225,67.99510061)
\lineto(821.41605225,67.99510061)
\moveto(823.74105225,66.45010061)
\curveto(823.42105534,66.55009436)(823.05605571,66.6100943)(822.64605225,66.63010061)
\curveto(822.23605653,66.65009426)(821.82605694,66.66009425)(821.41605225,66.66010061)
\curveto(820.98605778,66.66009425)(820.5660582,66.65009426)(820.15605225,66.63010061)
\curveto(819.74605902,66.6100943)(819.3610594,66.56509434)(819.00105225,66.49510061)
\curveto(818.64106012,66.42509448)(818.32106044,66.31509459)(818.04105225,66.16510061)
\curveto(817.75106101,66.02509488)(817.51606125,65.83009508)(817.33605225,65.58010061)
\curveto(817.22606154,65.42009549)(817.14606162,65.24009567)(817.09605225,65.04010061)
\curveto(817.03606173,64.84009607)(817.00606176,64.59509631)(817.00605225,64.30510061)
\curveto(817.02606174,64.28509662)(817.03606173,64.25009666)(817.03605225,64.20010061)
\curveto(817.02606174,64.15009676)(817.02606174,64.1100968)(817.03605225,64.08010061)
\curveto(817.05606171,64.00009691)(817.07606169,63.92509698)(817.09605225,63.85510061)
\curveto(817.10606166,63.79509711)(817.12606164,63.73009718)(817.15605225,63.66010061)
\curveto(817.27606149,63.39009752)(817.44606132,63.17009774)(817.66605225,63.00010061)
\curveto(817.87606089,62.84009807)(818.12106064,62.7050982)(818.40105225,62.59510061)
\curveto(818.51106025,62.54509836)(818.63106013,62.5050984)(818.76105225,62.47510061)
\curveto(818.88105988,62.45509845)(819.00605976,62.43009848)(819.13605225,62.40010061)
\curveto(819.18605958,62.38009853)(819.24105952,62.37009854)(819.30105225,62.37010061)
\curveto(819.35105941,62.37009854)(819.40105936,62.36509854)(819.45105225,62.35510061)
\curveto(819.54105922,62.34509856)(819.63605913,62.33509857)(819.73605225,62.32510061)
\curveto(819.82605894,62.31509859)(819.92105884,62.3050986)(820.02105225,62.29510061)
\curveto(820.10105866,62.29509861)(820.18605858,62.29009862)(820.27605225,62.28010061)
\lineto(820.51605225,62.28010061)
\lineto(820.69605225,62.28010061)
\curveto(820.72605804,62.27009864)(820.761058,62.26509864)(820.80105225,62.26510061)
\lineto(820.93605225,62.26510061)
\lineto(821.38605225,62.26510061)
\curveto(821.4660573,62.26509864)(821.55105721,62.26009865)(821.64105225,62.25010061)
\curveto(821.72105704,62.25009866)(821.79605697,62.26009865)(821.86605225,62.28010061)
\lineto(822.13605225,62.28010061)
\curveto(822.15605661,62.28009863)(822.18605658,62.27509863)(822.22605225,62.26510061)
\curveto(822.25605651,62.26509864)(822.28105648,62.27009864)(822.30105225,62.28010061)
\curveto(822.40105636,62.29009862)(822.50105626,62.29509861)(822.60105225,62.29510061)
\curveto(822.69105607,62.3050986)(822.79105597,62.31509859)(822.90105225,62.32510061)
\curveto(823.02105574,62.35509855)(823.14605562,62.37009854)(823.27605225,62.37010061)
\curveto(823.39605537,62.38009853)(823.51105525,62.4050985)(823.62105225,62.44510061)
\curveto(823.92105484,62.52509838)(824.18605458,62.6100983)(824.41605225,62.70010061)
\curveto(824.64605412,62.80009811)(824.8610539,62.94509796)(825.06105225,63.13510061)
\curveto(825.2610535,63.34509756)(825.41105335,63.6100973)(825.51105225,63.93010061)
\curveto(825.53105323,63.97009694)(825.54105322,64.0050969)(825.54105225,64.03510061)
\curveto(825.53105323,64.07509683)(825.53605323,64.12009679)(825.55605225,64.17010061)
\curveto(825.5660532,64.2100967)(825.57605319,64.28009663)(825.58605225,64.38010061)
\curveto(825.59605317,64.49009642)(825.59105317,64.57509633)(825.57105225,64.63510061)
\curveto(825.55105321,64.7050962)(825.54105322,64.77509613)(825.54105225,64.84510061)
\curveto(825.53105323,64.91509599)(825.51605325,64.98009593)(825.49605225,65.04010061)
\curveto(825.43605333,65.24009567)(825.35105341,65.42009549)(825.24105225,65.58010061)
\curveto(825.22105354,65.6100953)(825.20105356,65.63509527)(825.18105225,65.65510061)
\lineto(825.12105225,65.71510061)
\curveto(825.10105366,65.75509515)(825.0610537,65.8050951)(825.00105225,65.86510061)
\curveto(824.8610539,65.96509494)(824.73105403,66.05009486)(824.61105225,66.12010061)
\curveto(824.49105427,66.19009472)(824.34605442,66.26009465)(824.17605225,66.33010061)
\curveto(824.10605466,66.36009455)(824.03605473,66.38009453)(823.96605225,66.39010061)
\curveto(823.89605487,66.4100945)(823.82105494,66.43009448)(823.74105225,66.45010061)
}
}
{
\newrgbcolor{curcolor}{0 0 0}
\pscustom[linestyle=none,fillstyle=solid,fillcolor=curcolor]
{
\newpath
\moveto(823.00605225,76.22470998)
\curveto(823.05605571,76.29470234)(823.12605564,76.3347023)(823.21605225,76.34470998)
\curveto(823.30605546,76.36470227)(823.41105535,76.37470226)(823.53105225,76.37470998)
\curveto(823.58105518,76.37470226)(823.63105513,76.36970226)(823.68105225,76.35970998)
\curveto(823.73105503,76.35970227)(823.77605499,76.34970228)(823.81605225,76.32970998)
\curveto(823.90605486,76.29970233)(823.9660548,76.23970239)(823.99605225,76.14970998)
\curveto(824.01605475,76.06970256)(824.02605474,75.97470266)(824.02605225,75.86470998)
\lineto(824.02605225,75.54970998)
\curveto(824.01605475,75.43970319)(824.02605474,75.3347033)(824.05605225,75.23470998)
\curveto(824.08605468,75.09470354)(824.1660546,75.00470363)(824.29605225,74.96470998)
\curveto(824.3660544,74.94470369)(824.45105431,74.9347037)(824.55105225,74.93470998)
\lineto(824.82105225,74.93470998)
\lineto(825.76605225,74.93470998)
\lineto(826.09605225,74.93470998)
\curveto(826.20605256,74.9347037)(826.29105247,74.91470372)(826.35105225,74.87470998)
\curveto(826.41105235,74.8347038)(826.45105231,74.78470385)(826.47105225,74.72470998)
\curveto(826.48105228,74.67470396)(826.49605227,74.60970402)(826.51605225,74.52970998)
\lineto(826.51605225,74.33470998)
\curveto(826.51605225,74.21470442)(826.51105225,74.10970452)(826.50105225,74.01970998)
\curveto(826.48105228,73.9297047)(826.43105233,73.85970477)(826.35105225,73.80970998)
\curveto(826.30105246,73.77970485)(826.23105253,73.76470487)(826.14105225,73.76470998)
\lineto(825.84105225,73.76470998)
\lineto(824.80605225,73.76470998)
\curveto(824.64605412,73.76470487)(824.50105426,73.75470488)(824.37105225,73.73470998)
\curveto(824.23105453,73.72470491)(824.13605463,73.66970496)(824.08605225,73.56970998)
\curveto(824.0660547,73.51970511)(824.05105471,73.44970518)(824.04105225,73.35970998)
\curveto(824.03105473,73.27970535)(824.02605474,73.18970544)(824.02605225,73.08970998)
\lineto(824.02605225,72.80470998)
\lineto(824.02605225,72.56470998)
\lineto(824.02605225,70.29970998)
\curveto(824.02605474,70.20970842)(824.03105473,70.10470853)(824.04105225,69.98470998)
\lineto(824.04105225,69.65470998)
\curveto(824.04105472,69.54470909)(824.03105473,69.44470919)(824.01105225,69.35470998)
\curveto(823.99105477,69.26470937)(823.95605481,69.20470943)(823.90605225,69.17470998)
\curveto(823.83605493,69.12470951)(823.74105502,69.09970953)(823.62105225,69.09970998)
\lineto(823.27605225,69.09970998)
\lineto(823.00605225,69.09970998)
\curveto(822.83605593,69.13970949)(822.69605607,69.19470944)(822.58605225,69.26470998)
\curveto(822.47605629,69.3347093)(822.3610564,69.41470922)(822.24105225,69.50470998)
\lineto(821.70105225,69.86470998)
\curveto(821.07105769,70.30470833)(820.45105831,70.73970789)(819.84105225,71.16970998)
\lineto(817.98105225,72.48970998)
\curveto(817.75106101,72.64970598)(817.53106123,72.80470583)(817.32105225,72.95470998)
\curveto(817.10106166,73.10470553)(816.87606189,73.25970537)(816.64605225,73.41970998)
\curveto(816.57606219,73.46970516)(816.51106225,73.51970511)(816.45105225,73.56970998)
\curveto(816.38106238,73.61970501)(816.30606246,73.66970496)(816.22605225,73.71970998)
\lineto(816.13605225,73.77970998)
\curveto(816.09606267,73.80970482)(816.0660627,73.83970479)(816.04605225,73.86970998)
\curveto(816.01606275,73.90970472)(815.99606277,73.94970468)(815.98605225,73.98970998)
\curveto(815.9660628,74.0297046)(815.94606282,74.07470456)(815.92605225,74.12470998)
\curveto(815.92606284,74.14470449)(815.93106283,74.16470447)(815.94105225,74.18470998)
\curveto(815.94106282,74.21470442)(815.93106283,74.23970439)(815.91105225,74.25970998)
\curveto(815.91106285,74.38970424)(815.91606285,74.50970412)(815.92605225,74.61970998)
\curveto(815.93606283,74.7297039)(815.98106278,74.80970382)(816.06105225,74.85970998)
\curveto(816.11106265,74.89970373)(816.18106258,74.91970371)(816.27105225,74.91970998)
\curveto(816.3610624,74.9297037)(816.45606231,74.9347037)(816.55605225,74.93470998)
\lineto(822.01605225,74.93470998)
\curveto(822.08605668,74.9347037)(822.1610566,74.9297037)(822.24105225,74.91970998)
\curveto(822.31105645,74.91970371)(822.38105638,74.92470371)(822.45105225,74.93470998)
\lineto(822.55605225,74.93470998)
\curveto(822.60605616,74.95470368)(822.6610561,74.96970366)(822.72105225,74.97970998)
\curveto(822.77105599,74.98970364)(822.81105595,75.01470362)(822.84105225,75.05470998)
\curveto(822.89105587,75.12470351)(822.92105584,75.20970342)(822.93105225,75.30970998)
\lineto(822.93105225,75.63970998)
\curveto(822.93105583,75.74970288)(822.93605583,75.85470278)(822.94605225,75.95470998)
\curveto(822.94605582,76.06470257)(822.9660558,76.15470248)(823.00605225,76.22470998)
\moveto(822.81105225,73.65970998)
\curveto(822.70105606,73.73970489)(822.53105623,73.77470486)(822.30105225,73.76470998)
\lineto(821.68605225,73.76470998)
\lineto(819.21105225,73.76470998)
\lineto(818.89605225,73.76470998)
\curveto(818.77605999,73.77470486)(818.67606009,73.76970486)(818.59605225,73.74970998)
\lineto(818.44605225,73.74970998)
\curveto(818.35606041,73.74970488)(818.27106049,73.7347049)(818.19105225,73.70470998)
\curveto(818.17106059,73.69470494)(818.1610606,73.68470495)(818.16105225,73.67470998)
\lineto(818.11605225,73.62970998)
\curveto(818.10606066,73.60970502)(818.10106066,73.57970505)(818.10105225,73.53970998)
\curveto(818.12106064,73.51970511)(818.13606063,73.49970513)(818.14605225,73.47970998)
\curveto(818.14606062,73.46970516)(818.15106061,73.45470518)(818.16105225,73.43470998)
\curveto(818.21106055,73.37470526)(818.28106048,73.31470532)(818.37105225,73.25470998)
\curveto(818.4610603,73.19470544)(818.54106022,73.13970549)(818.61105225,73.08970998)
\curveto(818.75106001,72.98970564)(818.89605987,72.89470574)(819.04605225,72.80470998)
\curveto(819.18605958,72.71470592)(819.32605944,72.61970601)(819.46605225,72.51970998)
\lineto(820.24605225,71.97970998)
\curveto(820.50605826,71.80970682)(820.766058,71.634707)(821.02605225,71.45470998)
\curveto(821.13605763,71.37470726)(821.24105752,71.29970733)(821.34105225,71.22970998)
\lineto(821.64105225,71.01970998)
\curveto(821.72105704,70.96970766)(821.79605697,70.91970771)(821.86605225,70.86970998)
\curveto(821.93605683,70.8297078)(822.01105675,70.78470785)(822.09105225,70.73470998)
\curveto(822.15105661,70.68470795)(822.21605655,70.634708)(822.28605225,70.58470998)
\curveto(822.34605642,70.54470809)(822.41605635,70.50470813)(822.49605225,70.46470998)
\curveto(822.55605621,70.42470821)(822.62605614,70.39970823)(822.70605225,70.38970998)
\curveto(822.77605599,70.37970825)(822.83105593,70.41470822)(822.87105225,70.49470998)
\curveto(822.92105584,70.56470807)(822.94605582,70.67470796)(822.94605225,70.82470998)
\curveto(822.93605583,70.98470765)(822.93105583,71.11970751)(822.93105225,71.22970998)
\lineto(822.93105225,72.90970998)
\lineto(822.93105225,73.34470998)
\curveto(822.93105583,73.49470514)(822.89105587,73.59970503)(822.81105225,73.65970998)
}
}
{
\newrgbcolor{curcolor}{0 0 0}
\pscustom[linestyle=none,fillstyle=solid,fillcolor=curcolor]
{
\newpath
\moveto(824.86605225,78.63431936)
\lineto(824.86605225,79.26431936)
\lineto(824.86605225,79.45931936)
\curveto(824.8660539,79.52931683)(824.87605389,79.58931677)(824.89605225,79.63931936)
\curveto(824.93605383,79.70931665)(824.97605379,79.7593166)(825.01605225,79.78931936)
\curveto(825.0660537,79.82931653)(825.13105363,79.84931651)(825.21105225,79.84931936)
\curveto(825.29105347,79.8593165)(825.37605339,79.86431649)(825.46605225,79.86431936)
\lineto(826.18605225,79.86431936)
\curveto(826.6660521,79.86431649)(827.07605169,79.80431655)(827.41605225,79.68431936)
\curveto(827.75605101,79.56431679)(828.03105073,79.36931699)(828.24105225,79.09931936)
\curveto(828.29105047,79.02931733)(828.33605043,78.9593174)(828.37605225,78.88931936)
\curveto(828.42605034,78.82931753)(828.47105029,78.7543176)(828.51105225,78.66431936)
\curveto(828.52105024,78.64431771)(828.53105023,78.61431774)(828.54105225,78.57431936)
\curveto(828.5610502,78.53431782)(828.5660502,78.48931787)(828.55605225,78.43931936)
\curveto(828.52605024,78.34931801)(828.45105031,78.29431806)(828.33105225,78.27431936)
\curveto(828.22105054,78.2543181)(828.12605064,78.26931809)(828.04605225,78.31931936)
\curveto(827.97605079,78.34931801)(827.91105085,78.39431796)(827.85105225,78.45431936)
\curveto(827.80105096,78.52431783)(827.75105101,78.58931777)(827.70105225,78.64931936)
\curveto(827.65105111,78.71931764)(827.57605119,78.77931758)(827.47605225,78.82931936)
\curveto(827.38605138,78.88931747)(827.29605147,78.93931742)(827.20605225,78.97931936)
\curveto(827.17605159,78.99931736)(827.11605165,79.02431733)(827.02605225,79.05431936)
\curveto(826.94605182,79.08431727)(826.87605189,79.08931727)(826.81605225,79.06931936)
\curveto(826.67605209,79.03931732)(826.58605218,78.97931738)(826.54605225,78.88931936)
\curveto(826.51605225,78.80931755)(826.50105226,78.71931764)(826.50105225,78.61931936)
\curveto(826.50105226,78.51931784)(826.47605229,78.43431792)(826.42605225,78.36431936)
\curveto(826.35605241,78.27431808)(826.21605255,78.22931813)(826.00605225,78.22931936)
\lineto(825.45105225,78.22931936)
\lineto(825.22605225,78.22931936)
\curveto(825.14605362,78.23931812)(825.08105368,78.2593181)(825.03105225,78.28931936)
\curveto(824.95105381,78.34931801)(824.90605386,78.41931794)(824.89605225,78.49931936)
\curveto(824.88605388,78.51931784)(824.88105388,78.53931782)(824.88105225,78.55931936)
\curveto(824.88105388,78.58931777)(824.87605389,78.61431774)(824.86605225,78.63431936)
}
}
{
\newrgbcolor{curcolor}{0 0 0}
\pscustom[linestyle=none,fillstyle=solid,fillcolor=curcolor]
{
}
}
{
\newrgbcolor{curcolor}{0 0 0}
\pscustom[linestyle=none,fillstyle=solid,fillcolor=curcolor]
{
\newpath
\moveto(815.89605225,89.26463186)
\curveto(815.88606288,89.95462722)(816.00606276,90.55462662)(816.25605225,91.06463186)
\curveto(816.50606226,91.58462559)(816.84106192,91.9796252)(817.26105225,92.24963186)
\curveto(817.34106142,92.29962488)(817.43106133,92.34462483)(817.53105225,92.38463186)
\curveto(817.62106114,92.42462475)(817.71606105,92.46962471)(817.81605225,92.51963186)
\curveto(817.91606085,92.55962462)(818.01606075,92.58962459)(818.11605225,92.60963186)
\curveto(818.21606055,92.62962455)(818.32106044,92.64962453)(818.43105225,92.66963186)
\curveto(818.48106028,92.68962449)(818.52606024,92.69462448)(818.56605225,92.68463186)
\curveto(818.60606016,92.6746245)(818.65106011,92.6796245)(818.70105225,92.69963186)
\curveto(818.75106001,92.70962447)(818.83605993,92.71462446)(818.95605225,92.71463186)
\curveto(819.0660597,92.71462446)(819.15105961,92.70962447)(819.21105225,92.69963186)
\curveto(819.27105949,92.6796245)(819.33105943,92.66962451)(819.39105225,92.66963186)
\curveto(819.45105931,92.6796245)(819.51105925,92.6746245)(819.57105225,92.65463186)
\curveto(819.71105905,92.61462456)(819.84605892,92.5796246)(819.97605225,92.54963186)
\curveto(820.10605866,92.51962466)(820.23105853,92.4796247)(820.35105225,92.42963186)
\curveto(820.49105827,92.36962481)(820.61605815,92.29962488)(820.72605225,92.21963186)
\curveto(820.83605793,92.14962503)(820.94605782,92.0746251)(821.05605225,91.99463186)
\lineto(821.11605225,91.93463186)
\curveto(821.13605763,91.92462525)(821.15605761,91.90962527)(821.17605225,91.88963186)
\curveto(821.33605743,91.76962541)(821.48105728,91.63462554)(821.61105225,91.48463186)
\curveto(821.74105702,91.33462584)(821.8660569,91.174626)(821.98605225,91.00463186)
\curveto(822.20605656,90.69462648)(822.41105635,90.39962678)(822.60105225,90.11963186)
\curveto(822.74105602,89.88962729)(822.87605589,89.65962752)(823.00605225,89.42963186)
\curveto(823.13605563,89.20962797)(823.27105549,88.98962819)(823.41105225,88.76963186)
\curveto(823.58105518,88.51962866)(823.761055,88.2796289)(823.95105225,88.04963186)
\curveto(824.14105462,87.82962935)(824.3660544,87.63962954)(824.62605225,87.47963186)
\curveto(824.68605408,87.43962974)(824.74605402,87.40462977)(824.80605225,87.37463186)
\curveto(824.85605391,87.34462983)(824.92105384,87.31462986)(825.00105225,87.28463186)
\curveto(825.07105369,87.26462991)(825.13105363,87.25962992)(825.18105225,87.26963186)
\curveto(825.25105351,87.28962989)(825.30605346,87.32462985)(825.34605225,87.37463186)
\curveto(825.37605339,87.42462975)(825.39605337,87.48462969)(825.40605225,87.55463186)
\lineto(825.40605225,87.79463186)
\lineto(825.40605225,88.54463186)
\lineto(825.40605225,91.34963186)
\lineto(825.40605225,92.00963186)
\curveto(825.40605336,92.09962508)(825.41105335,92.18462499)(825.42105225,92.26463186)
\curveto(825.42105334,92.34462483)(825.44105332,92.40962477)(825.48105225,92.45963186)
\curveto(825.52105324,92.50962467)(825.59605317,92.54962463)(825.70605225,92.57963186)
\curveto(825.80605296,92.61962456)(825.90605286,92.62962455)(826.00605225,92.60963186)
\lineto(826.14105225,92.60963186)
\curveto(826.21105255,92.58962459)(826.27105249,92.56962461)(826.32105225,92.54963186)
\curveto(826.37105239,92.52962465)(826.41105235,92.49462468)(826.44105225,92.44463186)
\curveto(826.48105228,92.39462478)(826.50105226,92.32462485)(826.50105225,92.23463186)
\lineto(826.50105225,91.96463186)
\lineto(826.50105225,91.06463186)
\lineto(826.50105225,87.55463186)
\lineto(826.50105225,86.48963186)
\curveto(826.50105226,86.40963077)(826.50605226,86.31963086)(826.51605225,86.21963186)
\curveto(826.51605225,86.11963106)(826.50605226,86.03463114)(826.48605225,85.96463186)
\curveto(826.41605235,85.75463142)(826.23605253,85.68963149)(825.94605225,85.76963186)
\curveto(825.90605286,85.7796314)(825.87105289,85.7796314)(825.84105225,85.76963186)
\curveto(825.80105296,85.76963141)(825.75605301,85.7796314)(825.70605225,85.79963186)
\curveto(825.62605314,85.81963136)(825.54105322,85.83963134)(825.45105225,85.85963186)
\curveto(825.3610534,85.8796313)(825.27605349,85.90463127)(825.19605225,85.93463186)
\curveto(824.70605406,86.09463108)(824.29105447,86.29463088)(823.95105225,86.53463186)
\curveto(823.70105506,86.71463046)(823.47605529,86.91963026)(823.27605225,87.14963186)
\curveto(823.0660557,87.3796298)(822.87105589,87.61962956)(822.69105225,87.86963186)
\curveto(822.51105625,88.12962905)(822.34105642,88.39462878)(822.18105225,88.66463186)
\curveto(822.01105675,88.94462823)(821.83605693,89.21462796)(821.65605225,89.47463186)
\curveto(821.57605719,89.58462759)(821.50105726,89.68962749)(821.43105225,89.78963186)
\curveto(821.3610574,89.89962728)(821.28605748,90.00962717)(821.20605225,90.11963186)
\curveto(821.17605759,90.15962702)(821.14605762,90.19462698)(821.11605225,90.22463186)
\curveto(821.07605769,90.26462691)(821.04605772,90.30462687)(821.02605225,90.34463186)
\curveto(820.91605785,90.48462669)(820.79105797,90.60962657)(820.65105225,90.71963186)
\curveto(820.62105814,90.73962644)(820.59605817,90.76462641)(820.57605225,90.79463186)
\curveto(820.54605822,90.82462635)(820.51605825,90.84962633)(820.48605225,90.86963186)
\curveto(820.38605838,90.94962623)(820.28605848,91.01462616)(820.18605225,91.06463186)
\curveto(820.08605868,91.12462605)(819.97605879,91.179626)(819.85605225,91.22963186)
\curveto(819.78605898,91.25962592)(819.71105905,91.2796259)(819.63105225,91.28963186)
\lineto(819.39105225,91.34963186)
\lineto(819.30105225,91.34963186)
\curveto(819.27105949,91.35962582)(819.24105952,91.36462581)(819.21105225,91.36463186)
\curveto(819.14105962,91.38462579)(819.04605972,91.38962579)(818.92605225,91.37963186)
\curveto(818.79605997,91.3796258)(818.69606007,91.36962581)(818.62605225,91.34963186)
\curveto(818.54606022,91.32962585)(818.47106029,91.30962587)(818.40105225,91.28963186)
\curveto(818.32106044,91.2796259)(818.24106052,91.25962592)(818.16105225,91.22963186)
\curveto(817.92106084,91.11962606)(817.72106104,90.96962621)(817.56105225,90.77963186)
\curveto(817.39106137,90.59962658)(817.25106151,90.3796268)(817.14105225,90.11963186)
\curveto(817.12106164,90.04962713)(817.10606166,89.9796272)(817.09605225,89.90963186)
\curveto(817.07606169,89.83962734)(817.05606171,89.76462741)(817.03605225,89.68463186)
\curveto(817.01606175,89.60462757)(817.00606176,89.49462768)(817.00605225,89.35463186)
\curveto(817.00606176,89.22462795)(817.01606175,89.11962806)(817.03605225,89.03963186)
\curveto(817.04606172,88.9796282)(817.05106171,88.92462825)(817.05105225,88.87463186)
\curveto(817.05106171,88.82462835)(817.0610617,88.7746284)(817.08105225,88.72463186)
\curveto(817.12106164,88.62462855)(817.1610616,88.52962865)(817.20105225,88.43963186)
\curveto(817.24106152,88.35962882)(817.28606148,88.2796289)(817.33605225,88.19963186)
\curveto(817.35606141,88.16962901)(817.38106138,88.13962904)(817.41105225,88.10963186)
\curveto(817.44106132,88.08962909)(817.4660613,88.06462911)(817.48605225,88.03463186)
\lineto(817.56105225,87.95963186)
\curveto(817.58106118,87.92962925)(817.60106116,87.90462927)(817.62105225,87.88463186)
\lineto(817.83105225,87.73463186)
\curveto(817.89106087,87.69462948)(817.95606081,87.64962953)(818.02605225,87.59963186)
\curveto(818.11606065,87.53962964)(818.22106054,87.48962969)(818.34105225,87.44963186)
\curveto(818.45106031,87.41962976)(818.5610602,87.38462979)(818.67105225,87.34463186)
\curveto(818.78105998,87.30462987)(818.92605984,87.2796299)(819.10605225,87.26963186)
\curveto(819.27605949,87.25962992)(819.40105936,87.22962995)(819.48105225,87.17963186)
\curveto(819.5610592,87.12963005)(819.60605916,87.05463012)(819.61605225,86.95463186)
\curveto(819.62605914,86.85463032)(819.63105913,86.74463043)(819.63105225,86.62463186)
\curveto(819.63105913,86.58463059)(819.63605913,86.54463063)(819.64605225,86.50463186)
\curveto(819.64605912,86.46463071)(819.64105912,86.42963075)(819.63105225,86.39963186)
\curveto(819.61105915,86.34963083)(819.60105916,86.29963088)(819.60105225,86.24963186)
\curveto(819.60105916,86.20963097)(819.59105917,86.16963101)(819.57105225,86.12963186)
\curveto(819.51105925,86.03963114)(819.37605939,85.99463118)(819.16605225,85.99463186)
\lineto(819.04605225,85.99463186)
\curveto(818.98605978,86.00463117)(818.92605984,86.00963117)(818.86605225,86.00963186)
\curveto(818.79605997,86.01963116)(818.73106003,86.02963115)(818.67105225,86.03963186)
\curveto(818.5610602,86.05963112)(818.4610603,86.0796311)(818.37105225,86.09963186)
\curveto(818.27106049,86.11963106)(818.17606059,86.14963103)(818.08605225,86.18963186)
\curveto(818.01606075,86.20963097)(817.95606081,86.22963095)(817.90605225,86.24963186)
\lineto(817.72605225,86.30963186)
\curveto(817.4660613,86.42963075)(817.22106154,86.58463059)(816.99105225,86.77463186)
\curveto(816.761062,86.9746302)(816.57606219,87.18962999)(816.43605225,87.41963186)
\curveto(816.35606241,87.52962965)(816.29106247,87.64462953)(816.24105225,87.76463186)
\lineto(816.09105225,88.15463186)
\curveto(816.04106272,88.26462891)(816.01106275,88.3796288)(816.00105225,88.49963186)
\curveto(815.98106278,88.61962856)(815.95606281,88.74462843)(815.92605225,88.87463186)
\curveto(815.92606284,88.94462823)(815.92606284,89.00962817)(815.92605225,89.06963186)
\curveto(815.91606285,89.12962805)(815.90606286,89.19462798)(815.89605225,89.26463186)
}
}
{
\newrgbcolor{curcolor}{0 0 0}
\pscustom[linestyle=none,fillstyle=solid,fillcolor=curcolor]
{
\newpath
\moveto(821.41605225,101.36424123)
\lineto(821.67105225,101.36424123)
\curveto(821.75105701,101.37423353)(821.82605694,101.36923353)(821.89605225,101.34924123)
\lineto(822.13605225,101.34924123)
\lineto(822.30105225,101.34924123)
\curveto(822.40105636,101.32923357)(822.50605626,101.31923358)(822.61605225,101.31924123)
\curveto(822.71605605,101.31923358)(822.81605595,101.30923359)(822.91605225,101.28924123)
\lineto(823.06605225,101.28924123)
\curveto(823.20605556,101.25923364)(823.34605542,101.23923366)(823.48605225,101.22924123)
\curveto(823.61605515,101.21923368)(823.74605502,101.19423371)(823.87605225,101.15424123)
\curveto(823.95605481,101.13423377)(824.04105472,101.11423379)(824.13105225,101.09424123)
\lineto(824.37105225,101.03424123)
\lineto(824.67105225,100.91424123)
\curveto(824.761054,100.88423402)(824.85105391,100.84923405)(824.94105225,100.80924123)
\curveto(825.1610536,100.70923419)(825.37605339,100.57423433)(825.58605225,100.40424123)
\curveto(825.79605297,100.24423466)(825.9660528,100.06923483)(826.09605225,99.87924123)
\curveto(826.13605263,99.82923507)(826.17605259,99.76923513)(826.21605225,99.69924123)
\curveto(826.24605252,99.63923526)(826.28105248,99.57923532)(826.32105225,99.51924123)
\curveto(826.37105239,99.43923546)(826.41105235,99.34423556)(826.44105225,99.23424123)
\curveto(826.47105229,99.12423578)(826.50105226,99.01923588)(826.53105225,98.91924123)
\curveto(826.57105219,98.80923609)(826.59605217,98.6992362)(826.60605225,98.58924123)
\curveto(826.61605215,98.47923642)(826.63105213,98.36423654)(826.65105225,98.24424123)
\curveto(826.6610521,98.2042367)(826.6610521,98.15923674)(826.65105225,98.10924123)
\curveto(826.65105211,98.06923683)(826.65605211,98.02923687)(826.66605225,97.98924123)
\curveto(826.67605209,97.94923695)(826.68105208,97.89423701)(826.68105225,97.82424123)
\curveto(826.68105208,97.75423715)(826.67605209,97.7042372)(826.66605225,97.67424123)
\curveto(826.64605212,97.62423728)(826.64105212,97.57923732)(826.65105225,97.53924123)
\curveto(826.6610521,97.4992374)(826.6610521,97.46423744)(826.65105225,97.43424123)
\lineto(826.65105225,97.34424123)
\curveto(826.63105213,97.28423762)(826.61605215,97.21923768)(826.60605225,97.14924123)
\curveto(826.60605216,97.08923781)(826.60105216,97.02423788)(826.59105225,96.95424123)
\curveto(826.54105222,96.78423812)(826.49105227,96.62423828)(826.44105225,96.47424123)
\curveto(826.39105237,96.32423858)(826.32605244,96.17923872)(826.24605225,96.03924123)
\curveto(826.20605256,95.98923891)(826.17605259,95.93423897)(826.15605225,95.87424123)
\curveto(826.12605264,95.82423908)(826.09105267,95.77423913)(826.05105225,95.72424123)
\curveto(825.87105289,95.48423942)(825.65105311,95.28423962)(825.39105225,95.12424123)
\curveto(825.13105363,94.96423994)(824.84605392,94.82424008)(824.53605225,94.70424123)
\curveto(824.39605437,94.64424026)(824.25605451,94.5992403)(824.11605225,94.56924123)
\curveto(823.9660548,94.53924036)(823.81105495,94.5042404)(823.65105225,94.46424123)
\curveto(823.54105522,94.44424046)(823.43105533,94.42924047)(823.32105225,94.41924123)
\curveto(823.21105555,94.40924049)(823.10105566,94.39424051)(822.99105225,94.37424123)
\curveto(822.95105581,94.36424054)(822.91105585,94.35924054)(822.87105225,94.35924123)
\curveto(822.83105593,94.36924053)(822.79105597,94.36924053)(822.75105225,94.35924123)
\curveto(822.70105606,94.34924055)(822.65105611,94.34424056)(822.60105225,94.34424123)
\lineto(822.43605225,94.34424123)
\curveto(822.38605638,94.32424058)(822.33605643,94.31924058)(822.28605225,94.32924123)
\curveto(822.22605654,94.33924056)(822.17105659,94.33924056)(822.12105225,94.32924123)
\curveto(822.08105668,94.31924058)(822.03605673,94.31924058)(821.98605225,94.32924123)
\curveto(821.93605683,94.33924056)(821.88605688,94.33424057)(821.83605225,94.31424123)
\curveto(821.766057,94.29424061)(821.69105707,94.28924061)(821.61105225,94.29924123)
\curveto(821.52105724,94.30924059)(821.43605733,94.31424059)(821.35605225,94.31424123)
\curveto(821.2660575,94.31424059)(821.1660576,94.30924059)(821.05605225,94.29924123)
\curveto(820.93605783,94.28924061)(820.83605793,94.29424061)(820.75605225,94.31424123)
\lineto(820.47105225,94.31424123)
\lineto(819.84105225,94.35924123)
\curveto(819.74105902,94.36924053)(819.64605912,94.37924052)(819.55605225,94.38924123)
\lineto(819.25605225,94.41924123)
\curveto(819.20605956,94.43924046)(819.15605961,94.44424046)(819.10605225,94.43424123)
\curveto(819.04605972,94.43424047)(818.99105977,94.44424046)(818.94105225,94.46424123)
\curveto(818.77105999,94.51424039)(818.60606016,94.55424035)(818.44605225,94.58424123)
\curveto(818.27606049,94.61424029)(818.11606065,94.66424024)(817.96605225,94.73424123)
\curveto(817.50606126,94.92423998)(817.13106163,95.14423976)(816.84105225,95.39424123)
\curveto(816.55106221,95.65423925)(816.30606246,96.01423889)(816.10605225,96.47424123)
\curveto(816.05606271,96.6042383)(816.02106274,96.73423817)(816.00105225,96.86424123)
\curveto(815.98106278,97.0042379)(815.95606281,97.14423776)(815.92605225,97.28424123)
\curveto(815.91606285,97.35423755)(815.91106285,97.41923748)(815.91105225,97.47924123)
\curveto(815.91106285,97.53923736)(815.90606286,97.6042373)(815.89605225,97.67424123)
\curveto(815.87606289,98.5042364)(816.02606274,99.17423573)(816.34605225,99.68424123)
\curveto(816.65606211,100.19423471)(817.09606167,100.57423433)(817.66605225,100.82424123)
\curveto(817.78606098,100.87423403)(817.91106085,100.91923398)(818.04105225,100.95924123)
\curveto(818.17106059,100.9992339)(818.30606046,101.04423386)(818.44605225,101.09424123)
\curveto(818.52606024,101.11423379)(818.61106015,101.12923377)(818.70105225,101.13924123)
\lineto(818.94105225,101.19924123)
\curveto(819.05105971,101.22923367)(819.1610596,101.24423366)(819.27105225,101.24424123)
\curveto(819.38105938,101.25423365)(819.49105927,101.26923363)(819.60105225,101.28924123)
\curveto(819.65105911,101.30923359)(819.69605907,101.31423359)(819.73605225,101.30424123)
\curveto(819.77605899,101.3042336)(819.81605895,101.30923359)(819.85605225,101.31924123)
\curveto(819.90605886,101.32923357)(819.9610588,101.32923357)(820.02105225,101.31924123)
\curveto(820.07105869,101.31923358)(820.12105864,101.32423358)(820.17105225,101.33424123)
\lineto(820.30605225,101.33424123)
\curveto(820.3660584,101.35423355)(820.43605833,101.35423355)(820.51605225,101.33424123)
\curveto(820.58605818,101.32423358)(820.65105811,101.32923357)(820.71105225,101.34924123)
\curveto(820.74105802,101.35923354)(820.78105798,101.36423354)(820.83105225,101.36424123)
\lineto(820.95105225,101.36424123)
\lineto(821.41605225,101.36424123)
\moveto(823.74105225,99.81924123)
\curveto(823.42105534,99.91923498)(823.05605571,99.97923492)(822.64605225,99.99924123)
\curveto(822.23605653,100.01923488)(821.82605694,100.02923487)(821.41605225,100.02924123)
\curveto(820.98605778,100.02923487)(820.5660582,100.01923488)(820.15605225,99.99924123)
\curveto(819.74605902,99.97923492)(819.3610594,99.93423497)(819.00105225,99.86424123)
\curveto(818.64106012,99.79423511)(818.32106044,99.68423522)(818.04105225,99.53424123)
\curveto(817.75106101,99.39423551)(817.51606125,99.1992357)(817.33605225,98.94924123)
\curveto(817.22606154,98.78923611)(817.14606162,98.60923629)(817.09605225,98.40924123)
\curveto(817.03606173,98.20923669)(817.00606176,97.96423694)(817.00605225,97.67424123)
\curveto(817.02606174,97.65423725)(817.03606173,97.61923728)(817.03605225,97.56924123)
\curveto(817.02606174,97.51923738)(817.02606174,97.47923742)(817.03605225,97.44924123)
\curveto(817.05606171,97.36923753)(817.07606169,97.29423761)(817.09605225,97.22424123)
\curveto(817.10606166,97.16423774)(817.12606164,97.0992378)(817.15605225,97.02924123)
\curveto(817.27606149,96.75923814)(817.44606132,96.53923836)(817.66605225,96.36924123)
\curveto(817.87606089,96.20923869)(818.12106064,96.07423883)(818.40105225,95.96424123)
\curveto(818.51106025,95.91423899)(818.63106013,95.87423903)(818.76105225,95.84424123)
\curveto(818.88105988,95.82423908)(819.00605976,95.7992391)(819.13605225,95.76924123)
\curveto(819.18605958,95.74923915)(819.24105952,95.73923916)(819.30105225,95.73924123)
\curveto(819.35105941,95.73923916)(819.40105936,95.73423917)(819.45105225,95.72424123)
\curveto(819.54105922,95.71423919)(819.63605913,95.7042392)(819.73605225,95.69424123)
\curveto(819.82605894,95.68423922)(819.92105884,95.67423923)(820.02105225,95.66424123)
\curveto(820.10105866,95.66423924)(820.18605858,95.65923924)(820.27605225,95.64924123)
\lineto(820.51605225,95.64924123)
\lineto(820.69605225,95.64924123)
\curveto(820.72605804,95.63923926)(820.761058,95.63423927)(820.80105225,95.63424123)
\lineto(820.93605225,95.63424123)
\lineto(821.38605225,95.63424123)
\curveto(821.4660573,95.63423927)(821.55105721,95.62923927)(821.64105225,95.61924123)
\curveto(821.72105704,95.61923928)(821.79605697,95.62923927)(821.86605225,95.64924123)
\lineto(822.13605225,95.64924123)
\curveto(822.15605661,95.64923925)(822.18605658,95.64423926)(822.22605225,95.63424123)
\curveto(822.25605651,95.63423927)(822.28105648,95.63923926)(822.30105225,95.64924123)
\curveto(822.40105636,95.65923924)(822.50105626,95.66423924)(822.60105225,95.66424123)
\curveto(822.69105607,95.67423923)(822.79105597,95.68423922)(822.90105225,95.69424123)
\curveto(823.02105574,95.72423918)(823.14605562,95.73923916)(823.27605225,95.73924123)
\curveto(823.39605537,95.74923915)(823.51105525,95.77423913)(823.62105225,95.81424123)
\curveto(823.92105484,95.89423901)(824.18605458,95.97923892)(824.41605225,96.06924123)
\curveto(824.64605412,96.16923873)(824.8610539,96.31423859)(825.06105225,96.50424123)
\curveto(825.2610535,96.71423819)(825.41105335,96.97923792)(825.51105225,97.29924123)
\curveto(825.53105323,97.33923756)(825.54105322,97.37423753)(825.54105225,97.40424123)
\curveto(825.53105323,97.44423746)(825.53605323,97.48923741)(825.55605225,97.53924123)
\curveto(825.5660532,97.57923732)(825.57605319,97.64923725)(825.58605225,97.74924123)
\curveto(825.59605317,97.85923704)(825.59105317,97.94423696)(825.57105225,98.00424123)
\curveto(825.55105321,98.07423683)(825.54105322,98.14423676)(825.54105225,98.21424123)
\curveto(825.53105323,98.28423662)(825.51605325,98.34923655)(825.49605225,98.40924123)
\curveto(825.43605333,98.60923629)(825.35105341,98.78923611)(825.24105225,98.94924123)
\curveto(825.22105354,98.97923592)(825.20105356,99.0042359)(825.18105225,99.02424123)
\lineto(825.12105225,99.08424123)
\curveto(825.10105366,99.12423578)(825.0610537,99.17423573)(825.00105225,99.23424123)
\curveto(824.8610539,99.33423557)(824.73105403,99.41923548)(824.61105225,99.48924123)
\curveto(824.49105427,99.55923534)(824.34605442,99.62923527)(824.17605225,99.69924123)
\curveto(824.10605466,99.72923517)(824.03605473,99.74923515)(823.96605225,99.75924123)
\curveto(823.89605487,99.77923512)(823.82105494,99.7992351)(823.74105225,99.81924123)
}
}
{
\newrgbcolor{curcolor}{0 0 0}
\pscustom[linestyle=none,fillstyle=solid,fillcolor=curcolor]
{
\newpath
\moveto(815.89605225,106.77385061)
\curveto(815.89606287,106.87384575)(815.90606286,106.96884566)(815.92605225,107.05885061)
\curveto(815.93606283,107.14884548)(815.9660628,107.21384541)(816.01605225,107.25385061)
\curveto(816.09606267,107.31384531)(816.20106256,107.34384528)(816.33105225,107.34385061)
\lineto(816.72105225,107.34385061)
\lineto(818.22105225,107.34385061)
\lineto(824.61105225,107.34385061)
\lineto(825.78105225,107.34385061)
\lineto(826.09605225,107.34385061)
\curveto(826.19605257,107.35384527)(826.27605249,107.33884529)(826.33605225,107.29885061)
\curveto(826.41605235,107.24884538)(826.4660523,107.17384545)(826.48605225,107.07385061)
\curveto(826.49605227,106.98384564)(826.50105226,106.87384575)(826.50105225,106.74385061)
\lineto(826.50105225,106.51885061)
\curveto(826.48105228,106.43884619)(826.4660523,106.36884626)(826.45605225,106.30885061)
\curveto(826.43605233,106.24884638)(826.39605237,106.19884643)(826.33605225,106.15885061)
\curveto(826.27605249,106.11884651)(826.20105256,106.09884653)(826.11105225,106.09885061)
\lineto(825.81105225,106.09885061)
\lineto(824.71605225,106.09885061)
\lineto(819.37605225,106.09885061)
\curveto(819.28605948,106.07884655)(819.21105955,106.06384656)(819.15105225,106.05385061)
\curveto(819.08105968,106.05384657)(819.02105974,106.0238466)(818.97105225,105.96385061)
\curveto(818.92105984,105.89384673)(818.89605987,105.80384682)(818.89605225,105.69385061)
\curveto(818.88605988,105.59384703)(818.88105988,105.48384714)(818.88105225,105.36385061)
\lineto(818.88105225,104.22385061)
\lineto(818.88105225,103.72885061)
\curveto(818.87105989,103.56884906)(818.81105995,103.45884917)(818.70105225,103.39885061)
\curveto(818.67106009,103.37884925)(818.64106012,103.36884926)(818.61105225,103.36885061)
\curveto(818.57106019,103.36884926)(818.52606024,103.36384926)(818.47605225,103.35385061)
\curveto(818.35606041,103.33384929)(818.24606052,103.33884929)(818.14605225,103.36885061)
\curveto(818.04606072,103.40884922)(817.97606079,103.46384916)(817.93605225,103.53385061)
\curveto(817.88606088,103.61384901)(817.8610609,103.73384889)(817.86105225,103.89385061)
\curveto(817.8610609,104.05384857)(817.84606092,104.18884844)(817.81605225,104.29885061)
\curveto(817.80606096,104.34884828)(817.80106096,104.40384822)(817.80105225,104.46385061)
\curveto(817.79106097,104.5238481)(817.77606099,104.58384804)(817.75605225,104.64385061)
\curveto(817.70606106,104.79384783)(817.65606111,104.93884769)(817.60605225,105.07885061)
\curveto(817.54606122,105.21884741)(817.47606129,105.35384727)(817.39605225,105.48385061)
\curveto(817.30606146,105.623847)(817.20106156,105.74384688)(817.08105225,105.84385061)
\curveto(816.9610618,105.94384668)(816.83106193,106.03884659)(816.69105225,106.12885061)
\curveto(816.59106217,106.18884644)(816.48106228,106.23384639)(816.36105225,106.26385061)
\curveto(816.24106252,106.30384632)(816.13606263,106.35384627)(816.04605225,106.41385061)
\curveto(815.98606278,106.46384616)(815.94606282,106.53384609)(815.92605225,106.62385061)
\curveto(815.91606285,106.64384598)(815.91106285,106.66884596)(815.91105225,106.69885061)
\curveto(815.91106285,106.7288459)(815.90606286,106.75384587)(815.89605225,106.77385061)
}
}
{
\newrgbcolor{curcolor}{0 0 0}
\pscustom[linestyle=none,fillstyle=solid,fillcolor=curcolor]
{
\newpath
\moveto(815.89605225,115.12345998)
\curveto(815.89606287,115.22345513)(815.90606286,115.31845503)(815.92605225,115.40845998)
\curveto(815.93606283,115.49845485)(815.9660628,115.56345479)(816.01605225,115.60345998)
\curveto(816.09606267,115.66345469)(816.20106256,115.69345466)(816.33105225,115.69345998)
\lineto(816.72105225,115.69345998)
\lineto(818.22105225,115.69345998)
\lineto(824.61105225,115.69345998)
\lineto(825.78105225,115.69345998)
\lineto(826.09605225,115.69345998)
\curveto(826.19605257,115.70345465)(826.27605249,115.68845466)(826.33605225,115.64845998)
\curveto(826.41605235,115.59845475)(826.4660523,115.52345483)(826.48605225,115.42345998)
\curveto(826.49605227,115.33345502)(826.50105226,115.22345513)(826.50105225,115.09345998)
\lineto(826.50105225,114.86845998)
\curveto(826.48105228,114.78845556)(826.4660523,114.71845563)(826.45605225,114.65845998)
\curveto(826.43605233,114.59845575)(826.39605237,114.5484558)(826.33605225,114.50845998)
\curveto(826.27605249,114.46845588)(826.20105256,114.4484559)(826.11105225,114.44845998)
\lineto(825.81105225,114.44845998)
\lineto(824.71605225,114.44845998)
\lineto(819.37605225,114.44845998)
\curveto(819.28605948,114.42845592)(819.21105955,114.41345594)(819.15105225,114.40345998)
\curveto(819.08105968,114.40345595)(819.02105974,114.37345598)(818.97105225,114.31345998)
\curveto(818.92105984,114.24345611)(818.89605987,114.1534562)(818.89605225,114.04345998)
\curveto(818.88605988,113.94345641)(818.88105988,113.83345652)(818.88105225,113.71345998)
\lineto(818.88105225,112.57345998)
\lineto(818.88105225,112.07845998)
\curveto(818.87105989,111.91845843)(818.81105995,111.80845854)(818.70105225,111.74845998)
\curveto(818.67106009,111.72845862)(818.64106012,111.71845863)(818.61105225,111.71845998)
\curveto(818.57106019,111.71845863)(818.52606024,111.71345864)(818.47605225,111.70345998)
\curveto(818.35606041,111.68345867)(818.24606052,111.68845866)(818.14605225,111.71845998)
\curveto(818.04606072,111.75845859)(817.97606079,111.81345854)(817.93605225,111.88345998)
\curveto(817.88606088,111.96345839)(817.8610609,112.08345827)(817.86105225,112.24345998)
\curveto(817.8610609,112.40345795)(817.84606092,112.53845781)(817.81605225,112.64845998)
\curveto(817.80606096,112.69845765)(817.80106096,112.7534576)(817.80105225,112.81345998)
\curveto(817.79106097,112.87345748)(817.77606099,112.93345742)(817.75605225,112.99345998)
\curveto(817.70606106,113.14345721)(817.65606111,113.28845706)(817.60605225,113.42845998)
\curveto(817.54606122,113.56845678)(817.47606129,113.70345665)(817.39605225,113.83345998)
\curveto(817.30606146,113.97345638)(817.20106156,114.09345626)(817.08105225,114.19345998)
\curveto(816.9610618,114.29345606)(816.83106193,114.38845596)(816.69105225,114.47845998)
\curveto(816.59106217,114.53845581)(816.48106228,114.58345577)(816.36105225,114.61345998)
\curveto(816.24106252,114.6534557)(816.13606263,114.70345565)(816.04605225,114.76345998)
\curveto(815.98606278,114.81345554)(815.94606282,114.88345547)(815.92605225,114.97345998)
\curveto(815.91606285,114.99345536)(815.91106285,115.01845533)(815.91105225,115.04845998)
\curveto(815.91106285,115.07845527)(815.90606286,115.10345525)(815.89605225,115.12345998)
}
}
{
\newrgbcolor{curcolor}{0 0 0}
\pscustom[linestyle=none,fillstyle=solid,fillcolor=curcolor]
{
\newpath
\moveto(836.73236816,42.29681936)
\curveto(836.73237886,42.36681368)(836.73237886,42.4468136)(836.73236816,42.53681936)
\curveto(836.72237887,42.62681342)(836.72237887,42.71181333)(836.73236816,42.79181936)
\curveto(836.73237886,42.88181316)(836.74237885,42.96181308)(836.76236816,43.03181936)
\curveto(836.78237881,43.11181293)(836.81237878,43.16681288)(836.85236816,43.19681936)
\curveto(836.90237869,43.22681282)(836.97737861,43.2468128)(837.07736816,43.25681936)
\curveto(837.16737842,43.27681277)(837.27237832,43.28681276)(837.39236816,43.28681936)
\curveto(837.50237809,43.29681275)(837.61737797,43.29681275)(837.73736816,43.28681936)
\lineto(838.03736816,43.28681936)
\lineto(841.05236816,43.28681936)
\lineto(843.94736816,43.28681936)
\curveto(844.27737131,43.28681276)(844.60237099,43.28181276)(844.92236816,43.27181936)
\curveto(845.23237036,43.27181277)(845.51237008,43.23181281)(845.76236816,43.15181936)
\curveto(846.11236948,43.03181301)(846.40736918,42.87681317)(846.64736816,42.68681936)
\curveto(846.87736871,42.49681355)(847.07736851,42.25681379)(847.24736816,41.96681936)
\curveto(847.29736829,41.90681414)(847.33236826,41.8418142)(847.35236816,41.77181936)
\curveto(847.37236822,41.71181433)(847.39736819,41.6418144)(847.42736816,41.56181936)
\curveto(847.47736811,41.4418146)(847.51236808,41.31181473)(847.53236816,41.17181936)
\curveto(847.56236803,41.041815)(847.592368,40.90681514)(847.62236816,40.76681936)
\curveto(847.64236795,40.71681533)(847.64736794,40.66681538)(847.63736816,40.61681936)
\curveto(847.62736796,40.56681548)(847.62736796,40.51181553)(847.63736816,40.45181936)
\curveto(847.64736794,40.43181561)(847.64736794,40.40681564)(847.63736816,40.37681936)
\curveto(847.63736795,40.3468157)(847.64236795,40.32181572)(847.65236816,40.30181936)
\curveto(847.66236793,40.26181578)(847.66736792,40.20681584)(847.66736816,40.13681936)
\curveto(847.66736792,40.06681598)(847.66236793,40.01181603)(847.65236816,39.97181936)
\curveto(847.64236795,39.92181612)(847.64236795,39.86681618)(847.65236816,39.80681936)
\curveto(847.66236793,39.7468163)(847.65736793,39.69181635)(847.63736816,39.64181936)
\curveto(847.60736798,39.51181653)(847.587368,39.38681666)(847.57736816,39.26681936)
\curveto(847.56736802,39.1468169)(847.54236805,39.03181701)(847.50236816,38.92181936)
\curveto(847.38236821,38.55181749)(847.21236838,38.23181781)(846.99236816,37.96181936)
\curveto(846.77236882,37.69181835)(846.4923691,37.48181856)(846.15236816,37.33181936)
\curveto(846.03236956,37.28181876)(845.90736968,37.23681881)(845.77736816,37.19681936)
\curveto(845.64736994,37.16681888)(845.51237008,37.13181891)(845.37236816,37.09181936)
\curveto(845.32237027,37.08181896)(845.28237031,37.07681897)(845.25236816,37.07681936)
\curveto(845.21237038,37.07681897)(845.16737042,37.07181897)(845.11736816,37.06181936)
\curveto(845.0873705,37.05181899)(845.05237054,37.046819)(845.01236816,37.04681936)
\curveto(844.96237063,37.046819)(844.92237067,37.041819)(844.89236816,37.03181936)
\lineto(844.72736816,37.03181936)
\curveto(844.64737094,37.01181903)(844.54737104,37.00681904)(844.42736816,37.01681936)
\curveto(844.29737129,37.02681902)(844.20737138,37.041819)(844.15736816,37.06181936)
\curveto(844.06737152,37.08181896)(844.00237159,37.13681891)(843.96236816,37.22681936)
\curveto(843.94237165,37.25681879)(843.93737165,37.28681876)(843.94736816,37.31681936)
\curveto(843.94737164,37.3468187)(843.94237165,37.38681866)(843.93236816,37.43681936)
\curveto(843.92237167,37.47681857)(843.91737167,37.51681853)(843.91736816,37.55681936)
\lineto(843.91736816,37.70681936)
\curveto(843.91737167,37.82681822)(843.92237167,37.9468181)(843.93236816,38.06681936)
\curveto(843.93237166,38.19681785)(843.96737162,38.28681776)(844.03736816,38.33681936)
\curveto(844.09737149,38.37681767)(844.15737143,38.39681765)(844.21736816,38.39681936)
\curveto(844.27737131,38.39681765)(844.34737124,38.40681764)(844.42736816,38.42681936)
\curveto(844.45737113,38.43681761)(844.4923711,38.43681761)(844.53236816,38.42681936)
\curveto(844.56237103,38.42681762)(844.587371,38.43181761)(844.60736816,38.44181936)
\lineto(844.81736816,38.44181936)
\curveto(844.86737072,38.46181758)(844.91737067,38.46681758)(844.96736816,38.45681936)
\curveto(845.00737058,38.45681759)(845.05237054,38.46681758)(845.10236816,38.48681936)
\curveto(845.23237036,38.51681753)(845.35737023,38.5468175)(845.47736816,38.57681936)
\curveto(845.58737,38.60681744)(845.6923699,38.65181739)(845.79236816,38.71181936)
\curveto(846.08236951,38.88181716)(846.2873693,39.15181689)(846.40736816,39.52181936)
\curveto(846.42736916,39.57181647)(846.44236915,39.62181642)(846.45236816,39.67181936)
\curveto(846.45236914,39.73181631)(846.46236913,39.78681626)(846.48236816,39.83681936)
\lineto(846.48236816,39.91181936)
\curveto(846.4923691,39.98181606)(846.50236909,40.07681597)(846.51236816,40.19681936)
\curveto(846.51236908,40.32681572)(846.50236909,40.42681562)(846.48236816,40.49681936)
\curveto(846.46236913,40.56681548)(846.44736914,40.63681541)(846.43736816,40.70681936)
\curveto(846.41736917,40.78681526)(846.39736919,40.85681519)(846.37736816,40.91681936)
\curveto(846.21736937,41.29681475)(845.94236965,41.57181447)(845.55236816,41.74181936)
\curveto(845.42237017,41.79181425)(845.26737032,41.82681422)(845.08736816,41.84681936)
\curveto(844.90737068,41.87681417)(844.72237087,41.89181415)(844.53236816,41.89181936)
\curveto(844.33237126,41.90181414)(844.13237146,41.90181414)(843.93236816,41.89181936)
\lineto(843.36236816,41.89181936)
\lineto(839.11736816,41.89181936)
\lineto(837.57236816,41.89181936)
\curveto(837.46237813,41.89181415)(837.34237825,41.88681416)(837.21236816,41.87681936)
\curveto(837.08237851,41.86681418)(836.97737861,41.88681416)(836.89736816,41.93681936)
\curveto(836.82737876,41.99681405)(836.77737881,42.07681397)(836.74736816,42.17681936)
\curveto(836.74737884,42.19681385)(836.74737884,42.21681383)(836.74736816,42.23681936)
\curveto(836.74737884,42.25681379)(836.74237885,42.27681377)(836.73236816,42.29681936)
}
}
{
\newrgbcolor{curcolor}{0 0 0}
\pscustom[linestyle=none,fillstyle=solid,fillcolor=curcolor]
{
\newpath
\moveto(839.68736816,45.83049123)
\lineto(839.68736816,46.26549123)
\curveto(839.6873759,46.41548927)(839.72737586,46.52048916)(839.80736816,46.58049123)
\curveto(839.8873757,46.63048905)(839.9873756,46.65548903)(840.10736816,46.65549123)
\curveto(840.22737536,46.66548902)(840.34737524,46.67048901)(840.46736816,46.67049123)
\lineto(841.89236816,46.67049123)
\lineto(844.15736816,46.67049123)
\lineto(844.84736816,46.67049123)
\curveto(845.07737051,46.67048901)(845.27737031,46.69548899)(845.44736816,46.74549123)
\curveto(845.89736969,46.90548878)(846.21236938,47.20548848)(846.39236816,47.64549123)
\curveto(846.48236911,47.86548782)(846.51736907,48.13048755)(846.49736816,48.44049123)
\curveto(846.46736912,48.75048693)(846.41236918,49.00048668)(846.33236816,49.19049123)
\curveto(846.1923694,49.52048616)(846.01736957,49.7804859)(845.80736816,49.97049123)
\curveto(845.58737,50.17048551)(845.30237029,50.32548536)(844.95236816,50.43549123)
\curveto(844.87237072,50.46548522)(844.7923708,50.4854852)(844.71236816,50.49549123)
\curveto(844.63237096,50.50548518)(844.54737104,50.52048516)(844.45736816,50.54049123)
\curveto(844.40737118,50.55048513)(844.36237123,50.55048513)(844.32236816,50.54049123)
\curveto(844.28237131,50.54048514)(844.23737135,50.55048513)(844.18736816,50.57049123)
\lineto(843.87236816,50.57049123)
\curveto(843.7923718,50.59048509)(843.70237189,50.59548509)(843.60236816,50.58549123)
\curveto(843.4923721,50.57548511)(843.3923722,50.57048511)(843.30236816,50.57049123)
\lineto(842.13236816,50.57049123)
\lineto(840.54236816,50.57049123)
\curveto(840.42237517,50.57048511)(840.29737529,50.56548512)(840.16736816,50.55549123)
\curveto(840.02737556,50.55548513)(839.91737567,50.5804851)(839.83736816,50.63049123)
\curveto(839.7873758,50.67048501)(839.75737583,50.71548497)(839.74736816,50.76549123)
\curveto(839.72737586,50.82548486)(839.70737588,50.89548479)(839.68736816,50.97549123)
\lineto(839.68736816,51.20049123)
\curveto(839.6873759,51.32048436)(839.6923759,51.42548426)(839.70236816,51.51549123)
\curveto(839.71237588,51.61548407)(839.75737583,51.69048399)(839.83736816,51.74049123)
\curveto(839.8873757,51.79048389)(839.96237563,51.81548387)(840.06236816,51.81549123)
\lineto(840.34736816,51.81549123)
\lineto(841.36736816,51.81549123)
\lineto(845.40236816,51.81549123)
\lineto(846.75236816,51.81549123)
\curveto(846.87236872,51.81548387)(846.9873686,51.81048387)(847.09736816,51.80049123)
\curveto(847.19736839,51.80048388)(847.27236832,51.76548392)(847.32236816,51.69549123)
\curveto(847.35236824,51.65548403)(847.37736821,51.59548409)(847.39736816,51.51549123)
\curveto(847.40736818,51.43548425)(847.41736817,51.34548434)(847.42736816,51.24549123)
\curveto(847.42736816,51.15548453)(847.42236817,51.06548462)(847.41236816,50.97549123)
\curveto(847.40236819,50.89548479)(847.38236821,50.83548485)(847.35236816,50.79549123)
\curveto(847.31236828,50.74548494)(847.24736834,50.70048498)(847.15736816,50.66049123)
\curveto(847.11736847,50.65048503)(847.06236853,50.64048504)(846.99236816,50.63049123)
\curveto(846.92236867,50.63048505)(846.85736873,50.62548506)(846.79736816,50.61549123)
\curveto(846.72736886,50.60548508)(846.67236892,50.5854851)(846.63236816,50.55549123)
\curveto(846.592369,50.52548516)(846.57736901,50.4804852)(846.58736816,50.42049123)
\curveto(846.60736898,50.34048534)(846.66736892,50.26048542)(846.76736816,50.18049123)
\curveto(846.85736873,50.10048558)(846.92736866,50.02548566)(846.97736816,49.95549123)
\curveto(847.13736845,49.73548595)(847.27736831,49.4854862)(847.39736816,49.20549123)
\curveto(847.44736814,49.09548659)(847.47736811,48.9804867)(847.48736816,48.86049123)
\curveto(847.50736808,48.75048693)(847.53236806,48.63548705)(847.56236816,48.51549123)
\curveto(847.57236802,48.46548722)(847.57236802,48.41048727)(847.56236816,48.35049123)
\curveto(847.55236804,48.30048738)(847.55736803,48.25048743)(847.57736816,48.20049123)
\curveto(847.59736799,48.10048758)(847.59736799,48.01048767)(847.57736816,47.93049123)
\lineto(847.57736816,47.78049123)
\curveto(847.55736803,47.73048795)(847.54736804,47.67048801)(847.54736816,47.60049123)
\curveto(847.54736804,47.54048814)(847.54236805,47.4854882)(847.53236816,47.43549123)
\curveto(847.51236808,47.39548829)(847.50236809,47.35548833)(847.50236816,47.31549123)
\curveto(847.51236808,47.2854884)(847.50736808,47.24548844)(847.48736816,47.19549123)
\lineto(847.42736816,46.95549123)
\curveto(847.40736818,46.8854888)(847.37736821,46.81048887)(847.33736816,46.73049123)
\curveto(847.22736836,46.47048921)(847.08236851,46.25048943)(846.90236816,46.07049123)
\curveto(846.71236888,45.90048978)(846.4873691,45.76048992)(846.22736816,45.65049123)
\curveto(846.13736945,45.61049007)(846.04736954,45.5804901)(845.95736816,45.56049123)
\lineto(845.65736816,45.50049123)
\curveto(845.59736999,45.4804902)(845.54237005,45.47049021)(845.49236816,45.47049123)
\curveto(845.43237016,45.4804902)(845.36737022,45.47549021)(845.29736816,45.45549123)
\curveto(845.27737031,45.44549024)(845.25237034,45.44049024)(845.22236816,45.44049123)
\curveto(845.18237041,45.44049024)(845.14737044,45.43549025)(845.11736816,45.42549123)
\lineto(844.96736816,45.42549123)
\curveto(844.92737066,45.41549027)(844.88237071,45.41049027)(844.83236816,45.41049123)
\curveto(844.77237082,45.42049026)(844.71737087,45.42549026)(844.66736816,45.42549123)
\lineto(844.06736816,45.42549123)
\lineto(841.30736816,45.42549123)
\lineto(840.34736816,45.42549123)
\lineto(840.07736816,45.42549123)
\curveto(839.9873756,45.42549026)(839.91237568,45.44549024)(839.85236816,45.48549123)
\curveto(839.78237581,45.52549016)(839.73237586,45.60049008)(839.70236816,45.71049123)
\curveto(839.6923759,45.73048995)(839.6923759,45.75048993)(839.70236816,45.77049123)
\curveto(839.70237589,45.79048989)(839.69737589,45.81048987)(839.68736816,45.83049123)
}
}
{
\newrgbcolor{curcolor}{0 0 0}
\pscustom[linestyle=none,fillstyle=solid,fillcolor=curcolor]
{
\newpath
\moveto(836.73236816,54.28510061)
\curveto(836.73237886,54.41509899)(836.73237886,54.55009886)(836.73236816,54.69010061)
\curveto(836.73237886,54.84009857)(836.76737882,54.95009846)(836.83736816,55.02010061)
\curveto(836.90737868,55.07009834)(837.00237859,55.09509831)(837.12236816,55.09510061)
\curveto(837.23237836,55.1050983)(837.34737824,55.1100983)(837.46736816,55.11010061)
\lineto(838.80236816,55.11010061)
\lineto(844.87736816,55.11010061)
\lineto(846.55736816,55.11010061)
\lineto(846.94736816,55.11010061)
\curveto(847.0873685,55.1100983)(847.19736839,55.08509832)(847.27736816,55.03510061)
\curveto(847.32736826,55.0050984)(847.35736823,54.96009845)(847.36736816,54.90010061)
\curveto(847.37736821,54.85009856)(847.3923682,54.78509862)(847.41236816,54.70510061)
\lineto(847.41236816,54.49510061)
\lineto(847.41236816,54.18010061)
\curveto(847.40236819,54.08009933)(847.36736822,54.0050994)(847.30736816,53.95510061)
\curveto(847.22736836,53.9050995)(847.12736846,53.87509953)(847.00736816,53.86510061)
\lineto(846.63236816,53.86510061)
\lineto(845.25236816,53.86510061)
\lineto(839.01236816,53.86510061)
\lineto(837.54236816,53.86510061)
\curveto(837.43237816,53.86509954)(837.31737827,53.86009955)(837.19736816,53.85010061)
\curveto(837.06737852,53.85009956)(836.96737862,53.87509953)(836.89736816,53.92510061)
\curveto(836.83737875,53.96509944)(836.7873788,54.04009937)(836.74736816,54.15010061)
\curveto(836.73737885,54.17009924)(836.73737885,54.19009922)(836.74736816,54.21010061)
\curveto(836.74737884,54.24009917)(836.74237885,54.26509914)(836.73236816,54.28510061)
}
}
{
\newrgbcolor{curcolor}{0 0 0}
\pscustom[linestyle=none,fillstyle=solid,fillcolor=curcolor]
{
}
}
{
\newrgbcolor{curcolor}{0 0 0}
\pscustom[linestyle=none,fillstyle=solid,fillcolor=curcolor]
{
\newpath
\moveto(842.32736816,67.99510061)
\lineto(842.58236816,67.99510061)
\curveto(842.66237293,68.0050929)(842.73737285,68.00009291)(842.80736816,67.98010061)
\lineto(843.04736816,67.98010061)
\lineto(843.21236816,67.98010061)
\curveto(843.31237228,67.96009295)(843.41737217,67.95009296)(843.52736816,67.95010061)
\curveto(843.62737196,67.95009296)(843.72737186,67.94009297)(843.82736816,67.92010061)
\lineto(843.97736816,67.92010061)
\curveto(844.11737147,67.89009302)(844.25737133,67.87009304)(844.39736816,67.86010061)
\curveto(844.52737106,67.85009306)(844.65737093,67.82509308)(844.78736816,67.78510061)
\curveto(844.86737072,67.76509314)(844.95237064,67.74509316)(845.04236816,67.72510061)
\lineto(845.28236816,67.66510061)
\lineto(845.58236816,67.54510061)
\curveto(845.67236992,67.51509339)(845.76236983,67.48009343)(845.85236816,67.44010061)
\curveto(846.07236952,67.34009357)(846.2873693,67.2050937)(846.49736816,67.03510061)
\curveto(846.70736888,66.87509403)(846.87736871,66.70009421)(847.00736816,66.51010061)
\curveto(847.04736854,66.46009445)(847.0873685,66.40009451)(847.12736816,66.33010061)
\curveto(847.15736843,66.27009464)(847.1923684,66.2100947)(847.23236816,66.15010061)
\curveto(847.28236831,66.07009484)(847.32236827,65.97509493)(847.35236816,65.86510061)
\curveto(847.38236821,65.75509515)(847.41236818,65.65009526)(847.44236816,65.55010061)
\curveto(847.48236811,65.44009547)(847.50736808,65.33009558)(847.51736816,65.22010061)
\curveto(847.52736806,65.1100958)(847.54236805,64.99509591)(847.56236816,64.87510061)
\curveto(847.57236802,64.83509607)(847.57236802,64.79009612)(847.56236816,64.74010061)
\curveto(847.56236803,64.70009621)(847.56736802,64.66009625)(847.57736816,64.62010061)
\curveto(847.587368,64.58009633)(847.592368,64.52509638)(847.59236816,64.45510061)
\curveto(847.592368,64.38509652)(847.587368,64.33509657)(847.57736816,64.30510061)
\curveto(847.55736803,64.25509665)(847.55236804,64.2100967)(847.56236816,64.17010061)
\curveto(847.57236802,64.13009678)(847.57236802,64.09509681)(847.56236816,64.06510061)
\lineto(847.56236816,63.97510061)
\curveto(847.54236805,63.91509699)(847.52736806,63.85009706)(847.51736816,63.78010061)
\curveto(847.51736807,63.72009719)(847.51236808,63.65509725)(847.50236816,63.58510061)
\curveto(847.45236814,63.41509749)(847.40236819,63.25509765)(847.35236816,63.10510061)
\curveto(847.30236829,62.95509795)(847.23736835,62.8100981)(847.15736816,62.67010061)
\curveto(847.11736847,62.62009829)(847.0873685,62.56509834)(847.06736816,62.50510061)
\curveto(847.03736855,62.45509845)(847.00236859,62.4050985)(846.96236816,62.35510061)
\curveto(846.78236881,62.11509879)(846.56236903,61.91509899)(846.30236816,61.75510061)
\curveto(846.04236955,61.59509931)(845.75736983,61.45509945)(845.44736816,61.33510061)
\curveto(845.30737028,61.27509963)(845.16737042,61.23009968)(845.02736816,61.20010061)
\curveto(844.87737071,61.17009974)(844.72237087,61.13509977)(844.56236816,61.09510061)
\curveto(844.45237114,61.07509983)(844.34237125,61.06009985)(844.23236816,61.05010061)
\curveto(844.12237147,61.04009987)(844.01237158,61.02509988)(843.90236816,61.00510061)
\curveto(843.86237173,60.99509991)(843.82237177,60.99009992)(843.78236816,60.99010061)
\curveto(843.74237185,61.00009991)(843.70237189,61.00009991)(843.66236816,60.99010061)
\curveto(843.61237198,60.98009993)(843.56237203,60.97509993)(843.51236816,60.97510061)
\lineto(843.34736816,60.97510061)
\curveto(843.29737229,60.95509995)(843.24737234,60.95009996)(843.19736816,60.96010061)
\curveto(843.13737245,60.97009994)(843.08237251,60.97009994)(843.03236816,60.96010061)
\curveto(842.9923726,60.95009996)(842.94737264,60.95009996)(842.89736816,60.96010061)
\curveto(842.84737274,60.97009994)(842.79737279,60.96509994)(842.74736816,60.94510061)
\curveto(842.67737291,60.92509998)(842.60237299,60.92009999)(842.52236816,60.93010061)
\curveto(842.43237316,60.94009997)(842.34737324,60.94509996)(842.26736816,60.94510061)
\curveto(842.17737341,60.94509996)(842.07737351,60.94009997)(841.96736816,60.93010061)
\curveto(841.84737374,60.92009999)(841.74737384,60.92509998)(841.66736816,60.94510061)
\lineto(841.38236816,60.94510061)
\lineto(840.75236816,60.99010061)
\curveto(840.65237494,61.00009991)(840.55737503,61.0100999)(840.46736816,61.02010061)
\lineto(840.16736816,61.05010061)
\curveto(840.11737547,61.07009984)(840.06737552,61.07509983)(840.01736816,61.06510061)
\curveto(839.95737563,61.06509984)(839.90237569,61.07509983)(839.85236816,61.09510061)
\curveto(839.68237591,61.14509976)(839.51737607,61.18509972)(839.35736816,61.21510061)
\curveto(839.1873764,61.24509966)(839.02737656,61.29509961)(838.87736816,61.36510061)
\curveto(838.41737717,61.55509935)(838.04237755,61.77509913)(837.75236816,62.02510061)
\curveto(837.46237813,62.28509862)(837.21737837,62.64509826)(837.01736816,63.10510061)
\curveto(836.96737862,63.23509767)(836.93237866,63.36509754)(836.91236816,63.49510061)
\curveto(836.8923787,63.63509727)(836.86737872,63.77509713)(836.83736816,63.91510061)
\curveto(836.82737876,63.98509692)(836.82237877,64.05009686)(836.82236816,64.11010061)
\curveto(836.82237877,64.17009674)(836.81737877,64.23509667)(836.80736816,64.30510061)
\curveto(836.7873788,65.13509577)(836.93737865,65.8050951)(837.25736816,66.31510061)
\curveto(837.56737802,66.82509408)(838.00737758,67.2050937)(838.57736816,67.45510061)
\curveto(838.69737689,67.5050934)(838.82237677,67.55009336)(838.95236816,67.59010061)
\curveto(839.08237651,67.63009328)(839.21737637,67.67509323)(839.35736816,67.72510061)
\curveto(839.43737615,67.74509316)(839.52237607,67.76009315)(839.61236816,67.77010061)
\lineto(839.85236816,67.83010061)
\curveto(839.96237563,67.86009305)(840.07237552,67.87509303)(840.18236816,67.87510061)
\curveto(840.2923753,67.88509302)(840.40237519,67.90009301)(840.51236816,67.92010061)
\curveto(840.56237503,67.94009297)(840.60737498,67.94509296)(840.64736816,67.93510061)
\curveto(840.6873749,67.93509297)(840.72737486,67.94009297)(840.76736816,67.95010061)
\curveto(840.81737477,67.96009295)(840.87237472,67.96009295)(840.93236816,67.95010061)
\curveto(840.98237461,67.95009296)(841.03237456,67.95509295)(841.08236816,67.96510061)
\lineto(841.21736816,67.96510061)
\curveto(841.27737431,67.98509292)(841.34737424,67.98509292)(841.42736816,67.96510061)
\curveto(841.49737409,67.95509295)(841.56237403,67.96009295)(841.62236816,67.98010061)
\curveto(841.65237394,67.99009292)(841.6923739,67.99509291)(841.74236816,67.99510061)
\lineto(841.86236816,67.99510061)
\lineto(842.32736816,67.99510061)
\moveto(844.65236816,66.45010061)
\curveto(844.33237126,66.55009436)(843.96737162,66.6100943)(843.55736816,66.63010061)
\curveto(843.14737244,66.65009426)(842.73737285,66.66009425)(842.32736816,66.66010061)
\curveto(841.89737369,66.66009425)(841.47737411,66.65009426)(841.06736816,66.63010061)
\curveto(840.65737493,66.6100943)(840.27237532,66.56509434)(839.91236816,66.49510061)
\curveto(839.55237604,66.42509448)(839.23237636,66.31509459)(838.95236816,66.16510061)
\curveto(838.66237693,66.02509488)(838.42737716,65.83009508)(838.24736816,65.58010061)
\curveto(838.13737745,65.42009549)(838.05737753,65.24009567)(838.00736816,65.04010061)
\curveto(837.94737764,64.84009607)(837.91737767,64.59509631)(837.91736816,64.30510061)
\curveto(837.93737765,64.28509662)(837.94737764,64.25009666)(837.94736816,64.20010061)
\curveto(837.93737765,64.15009676)(837.93737765,64.1100968)(837.94736816,64.08010061)
\curveto(837.96737762,64.00009691)(837.9873776,63.92509698)(838.00736816,63.85510061)
\curveto(838.01737757,63.79509711)(838.03737755,63.73009718)(838.06736816,63.66010061)
\curveto(838.1873774,63.39009752)(838.35737723,63.17009774)(838.57736816,63.00010061)
\curveto(838.7873768,62.84009807)(839.03237656,62.7050982)(839.31236816,62.59510061)
\curveto(839.42237617,62.54509836)(839.54237605,62.5050984)(839.67236816,62.47510061)
\curveto(839.7923758,62.45509845)(839.91737567,62.43009848)(840.04736816,62.40010061)
\curveto(840.09737549,62.38009853)(840.15237544,62.37009854)(840.21236816,62.37010061)
\curveto(840.26237533,62.37009854)(840.31237528,62.36509854)(840.36236816,62.35510061)
\curveto(840.45237514,62.34509856)(840.54737504,62.33509857)(840.64736816,62.32510061)
\curveto(840.73737485,62.31509859)(840.83237476,62.3050986)(840.93236816,62.29510061)
\curveto(841.01237458,62.29509861)(841.09737449,62.29009862)(841.18736816,62.28010061)
\lineto(841.42736816,62.28010061)
\lineto(841.60736816,62.28010061)
\curveto(841.63737395,62.27009864)(841.67237392,62.26509864)(841.71236816,62.26510061)
\lineto(841.84736816,62.26510061)
\lineto(842.29736816,62.26510061)
\curveto(842.37737321,62.26509864)(842.46237313,62.26009865)(842.55236816,62.25010061)
\curveto(842.63237296,62.25009866)(842.70737288,62.26009865)(842.77736816,62.28010061)
\lineto(843.04736816,62.28010061)
\curveto(843.06737252,62.28009863)(843.09737249,62.27509863)(843.13736816,62.26510061)
\curveto(843.16737242,62.26509864)(843.1923724,62.27009864)(843.21236816,62.28010061)
\curveto(843.31237228,62.29009862)(843.41237218,62.29509861)(843.51236816,62.29510061)
\curveto(843.60237199,62.3050986)(843.70237189,62.31509859)(843.81236816,62.32510061)
\curveto(843.93237166,62.35509855)(844.05737153,62.37009854)(844.18736816,62.37010061)
\curveto(844.30737128,62.38009853)(844.42237117,62.4050985)(844.53236816,62.44510061)
\curveto(844.83237076,62.52509838)(845.09737049,62.6100983)(845.32736816,62.70010061)
\curveto(845.55737003,62.80009811)(845.77236982,62.94509796)(845.97236816,63.13510061)
\curveto(846.17236942,63.34509756)(846.32236927,63.6100973)(846.42236816,63.93010061)
\curveto(846.44236915,63.97009694)(846.45236914,64.0050969)(846.45236816,64.03510061)
\curveto(846.44236915,64.07509683)(846.44736914,64.12009679)(846.46736816,64.17010061)
\curveto(846.47736911,64.2100967)(846.4873691,64.28009663)(846.49736816,64.38010061)
\curveto(846.50736908,64.49009642)(846.50236909,64.57509633)(846.48236816,64.63510061)
\curveto(846.46236913,64.7050962)(846.45236914,64.77509613)(846.45236816,64.84510061)
\curveto(846.44236915,64.91509599)(846.42736916,64.98009593)(846.40736816,65.04010061)
\curveto(846.34736924,65.24009567)(846.26236933,65.42009549)(846.15236816,65.58010061)
\curveto(846.13236946,65.6100953)(846.11236948,65.63509527)(846.09236816,65.65510061)
\lineto(846.03236816,65.71510061)
\curveto(846.01236958,65.75509515)(845.97236962,65.8050951)(845.91236816,65.86510061)
\curveto(845.77236982,65.96509494)(845.64236995,66.05009486)(845.52236816,66.12010061)
\curveto(845.40237019,66.19009472)(845.25737033,66.26009465)(845.08736816,66.33010061)
\curveto(845.01737057,66.36009455)(844.94737064,66.38009453)(844.87736816,66.39010061)
\curveto(844.80737078,66.4100945)(844.73237086,66.43009448)(844.65236816,66.45010061)
}
}
{
\newrgbcolor{curcolor}{0 0 0}
\pscustom[linestyle=none,fillstyle=solid,fillcolor=curcolor]
{
\newpath
\moveto(844.33736816,76.35970998)
\curveto(844.37737121,76.36970226)(844.42737116,76.36970226)(844.48736816,76.35970998)
\curveto(844.54737104,76.35970227)(844.59737099,76.35470228)(844.63736816,76.34470998)
\curveto(844.67737091,76.34470229)(844.71737087,76.33970229)(844.75736816,76.32970998)
\lineto(844.86236816,76.32970998)
\curveto(844.94237065,76.30970232)(845.02237057,76.29470234)(845.10236816,76.28470998)
\curveto(845.18237041,76.27470236)(845.25737033,76.25470238)(845.32736816,76.22470998)
\curveto(845.40737018,76.20470243)(845.48237011,76.18470245)(845.55236816,76.16470998)
\curveto(845.62236997,76.14470249)(845.69736989,76.11470252)(845.77736816,76.07470998)
\curveto(846.19736939,75.89470274)(846.53736905,75.63970299)(846.79736816,75.30970998)
\curveto(847.05736853,74.97970365)(847.26236833,74.58970404)(847.41236816,74.13970998)
\curveto(847.45236814,74.01970461)(847.47736811,73.89470474)(847.48736816,73.76470998)
\curveto(847.50736808,73.64470499)(847.53236806,73.51970511)(847.56236816,73.38970998)
\curveto(847.57236802,73.3297053)(847.57736801,73.26470537)(847.57736816,73.19470998)
\curveto(847.57736801,73.1347055)(847.58236801,73.06970556)(847.59236816,72.99970998)
\lineto(847.59236816,72.87970998)
\lineto(847.59236816,72.68470998)
\curveto(847.60236799,72.62470601)(847.59736799,72.56970606)(847.57736816,72.51970998)
\curveto(847.55736803,72.44970618)(847.55236804,72.38470625)(847.56236816,72.32470998)
\curveto(847.57236802,72.26470637)(847.56736802,72.20470643)(847.54736816,72.14470998)
\curveto(847.53736805,72.09470654)(847.53236806,72.04970658)(847.53236816,72.00970998)
\curveto(847.53236806,71.96970666)(847.52236807,71.92470671)(847.50236816,71.87470998)
\curveto(847.48236811,71.79470684)(847.46236813,71.71970691)(847.44236816,71.64970998)
\curveto(847.43236816,71.57970705)(847.41736817,71.50970712)(847.39736816,71.43970998)
\curveto(847.22736836,70.95970767)(847.01736857,70.55970807)(846.76736816,70.23970998)
\curveto(846.50736908,69.9297087)(846.15236944,69.67970895)(845.70236816,69.48970998)
\curveto(845.64236995,69.45970917)(845.58237001,69.4347092)(845.52236816,69.41470998)
\curveto(845.45237014,69.40470923)(845.37737021,69.38970924)(845.29736816,69.36970998)
\curveto(845.23737035,69.34970928)(845.17237042,69.3347093)(845.10236816,69.32470998)
\curveto(845.03237056,69.31470932)(844.96237063,69.29970933)(844.89236816,69.27970998)
\curveto(844.84237075,69.26970936)(844.80237079,69.26470937)(844.77236816,69.26470998)
\lineto(844.65236816,69.26470998)
\curveto(844.61237098,69.25470938)(844.56237103,69.24470939)(844.50236816,69.23470998)
\curveto(844.44237115,69.2347094)(844.3923712,69.23970939)(844.35236816,69.24970998)
\lineto(844.21736816,69.24970998)
\curveto(844.16737142,69.25970937)(844.11737147,69.26470937)(844.06736816,69.26470998)
\curveto(843.96737162,69.28470935)(843.87237172,69.29970933)(843.78236816,69.30970998)
\curveto(843.68237191,69.31970931)(843.587372,69.33970929)(843.49736816,69.36970998)
\curveto(843.34737224,69.41970921)(843.20737238,69.47470916)(843.07736816,69.53470998)
\curveto(842.94737264,69.59470904)(842.82737276,69.66470897)(842.71736816,69.74470998)
\curveto(842.66737292,69.77470886)(842.62737296,69.80470883)(842.59736816,69.83470998)
\curveto(842.56737302,69.87470876)(842.53237306,69.90970872)(842.49236816,69.93970998)
\curveto(842.41237318,69.99970863)(842.34237325,70.06970856)(842.28236816,70.14970998)
\curveto(842.23237336,70.20970842)(842.1873734,70.26970836)(842.14736816,70.32970998)
\lineto(841.99736816,70.53970998)
\curveto(841.95737363,70.58970804)(841.92237367,70.63970799)(841.89236816,70.68970998)
\curveto(841.85237374,70.73970789)(841.79737379,70.77470786)(841.72736816,70.79470998)
\curveto(841.69737389,70.79470784)(841.67237392,70.78470785)(841.65236816,70.76470998)
\curveto(841.62237397,70.75470788)(841.59737399,70.74470789)(841.57736816,70.73470998)
\curveto(841.52737406,70.69470794)(841.48237411,70.64470799)(841.44236816,70.58470998)
\curveto(841.3923742,70.5347081)(841.34737424,70.48470815)(841.30736816,70.43470998)
\curveto(841.27737431,70.39470824)(841.22237437,70.34470829)(841.14236816,70.28470998)
\curveto(841.11237448,70.26470837)(841.0873745,70.2347084)(841.06736816,70.19470998)
\curveto(841.03737455,70.16470847)(841.00237459,70.13970849)(840.96236816,70.11970998)
\curveto(840.75237484,69.94970868)(840.50737508,69.81970881)(840.22736816,69.72970998)
\curveto(840.14737544,69.70970892)(840.06737552,69.69470894)(839.98736816,69.68470998)
\curveto(839.90737568,69.67470896)(839.82737576,69.65970897)(839.74736816,69.63970998)
\curveto(839.69737589,69.61970901)(839.63237596,69.60970902)(839.55236816,69.60970998)
\curveto(839.46237613,69.60970902)(839.3923762,69.61970901)(839.34236816,69.63970998)
\curveto(839.24237635,69.63970899)(839.17237642,69.64470899)(839.13236816,69.65470998)
\curveto(839.05237654,69.67470896)(838.98237661,69.68970894)(838.92236816,69.69970998)
\curveto(838.85237674,69.70970892)(838.78237681,69.72470891)(838.71236816,69.74470998)
\curveto(838.28237731,69.89470874)(837.93737765,70.10970852)(837.67736816,70.38970998)
\curveto(837.41737817,70.67970795)(837.20237839,71.0297076)(837.03236816,71.43970998)
\curveto(836.98237861,71.54970708)(836.95237864,71.66470697)(836.94236816,71.78470998)
\curveto(836.92237867,71.91470672)(836.8923787,72.04470659)(836.85236816,72.17470998)
\curveto(836.85237874,72.25470638)(836.85237874,72.32470631)(836.85236816,72.38470998)
\curveto(836.84237875,72.45470618)(836.83237876,72.5297061)(836.82236816,72.60970998)
\curveto(836.80237879,73.39970523)(836.93237866,74.05470458)(837.21236816,74.57470998)
\curveto(837.4923781,75.10470353)(837.90237769,75.48470315)(838.44236816,75.71470998)
\curveto(838.67237692,75.82470281)(838.95737663,75.89470274)(839.29736816,75.92470998)
\curveto(839.62737596,75.96470267)(839.93237566,75.9347027)(840.21236816,75.83470998)
\curveto(840.34237525,75.79470284)(840.46237513,75.74470289)(840.57236816,75.68470998)
\curveto(840.68237491,75.634703)(840.7873748,75.57470306)(840.88736816,75.50470998)
\curveto(840.92737466,75.48470315)(840.96237463,75.45470318)(840.99236816,75.41470998)
\lineto(841.08236816,75.32470998)
\curveto(841.17237442,75.27470336)(841.23737435,75.21470342)(841.27736816,75.14470998)
\curveto(841.32737426,75.09470354)(841.37737421,75.03970359)(841.42736816,74.97970998)
\curveto(841.46737412,74.9297037)(841.51237408,74.88470375)(841.56236816,74.84470998)
\curveto(841.58237401,74.82470381)(841.60737398,74.80470383)(841.63736816,74.78470998)
\curveto(841.65737393,74.77470386)(841.68237391,74.77470386)(841.71236816,74.78470998)
\curveto(841.76237383,74.79470384)(841.81237378,74.82470381)(841.86236816,74.87470998)
\curveto(841.90237369,74.92470371)(841.94237365,74.97970365)(841.98236816,75.03970998)
\lineto(842.10236816,75.21970998)
\curveto(842.13237346,75.27970335)(842.16237343,75.3297033)(842.19236816,75.36970998)
\curveto(842.43237316,75.69970293)(842.74237285,75.94970268)(843.12236816,76.11970998)
\curveto(843.20237239,76.15970247)(843.2873723,76.18970244)(843.37736816,76.20970998)
\curveto(843.46737212,76.23970239)(843.55737203,76.26470237)(843.64736816,76.28470998)
\curveto(843.69737189,76.29470234)(843.75237184,76.30470233)(843.81236816,76.31470998)
\lineto(843.96236816,76.34470998)
\curveto(844.02237157,76.35470228)(844.0873715,76.35470228)(844.15736816,76.34470998)
\curveto(844.21737137,76.3347023)(844.27737131,76.33970229)(844.33736816,76.35970998)
\moveto(839.29736816,70.97470998)
\curveto(839.40737618,70.94470769)(839.54737604,70.93970769)(839.71736816,70.95970998)
\curveto(839.87737571,70.97970765)(840.00237559,71.00470763)(840.09236816,71.03470998)
\curveto(840.41237518,71.14470749)(840.65737493,71.29470734)(840.82736816,71.48470998)
\curveto(840.9873746,71.67470696)(841.11737447,71.93970669)(841.21736816,72.27970998)
\curveto(841.24737434,72.40970622)(841.27237432,72.57470606)(841.29236816,72.77470998)
\curveto(841.30237429,72.97470566)(841.2873743,73.14470549)(841.24736816,73.28470998)
\curveto(841.16737442,73.57470506)(841.05737453,73.81470482)(840.91736816,74.00470998)
\curveto(840.76737482,74.20470443)(840.56737502,74.35970427)(840.31736816,74.46970998)
\curveto(840.26737532,74.48970414)(840.22237537,74.49970413)(840.18236816,74.49970998)
\curveto(840.14237545,74.50970412)(840.09737549,74.52470411)(840.04736816,74.54470998)
\curveto(839.93737565,74.57470406)(839.79737579,74.59470404)(839.62736816,74.60470998)
\curveto(839.45737613,74.61470402)(839.31237628,74.60470403)(839.19236816,74.57470998)
\curveto(839.10237649,74.55470408)(839.01737657,74.5297041)(838.93736816,74.49970998)
\curveto(838.85737673,74.47970415)(838.77737681,74.44470419)(838.69736816,74.39470998)
\curveto(838.42737716,74.22470441)(838.23237736,73.99970463)(838.11236816,73.71970998)
\curveto(837.9923776,73.44970518)(837.93237766,73.08970554)(837.93236816,72.63970998)
\curveto(837.95237764,72.61970601)(837.95737763,72.58970604)(837.94736816,72.54970998)
\curveto(837.93737765,72.50970612)(837.93737765,72.47470616)(837.94736816,72.44470998)
\curveto(837.96737762,72.39470624)(837.98237761,72.33970629)(837.99236816,72.27970998)
\curveto(837.9923776,72.2297064)(838.00237759,72.17970645)(838.02236816,72.12970998)
\curveto(838.11237748,71.88970674)(838.22737736,71.67970695)(838.36736816,71.49970998)
\curveto(838.49737709,71.31970731)(838.67737691,71.17970745)(838.90736816,71.07970998)
\curveto(838.96737662,71.05970757)(839.03237656,71.03970759)(839.10236816,71.01970998)
\curveto(839.16237643,71.00970762)(839.22737636,70.99470764)(839.29736816,70.97470998)
\moveto(844.83236816,74.99470998)
\curveto(844.64237095,75.04470359)(844.43737115,75.04970358)(844.21736816,75.00970998)
\curveto(843.99737159,74.97970365)(843.81737177,74.9347037)(843.67736816,74.87470998)
\curveto(843.30737228,74.70470393)(843.00237259,74.44470419)(842.76236816,74.09470998)
\curveto(842.52237307,73.75470488)(842.40237319,73.31970531)(842.40236816,72.78970998)
\curveto(842.42237317,72.75970587)(842.42737316,72.71970591)(842.41736816,72.66970998)
\curveto(842.39737319,72.61970601)(842.3923732,72.57970605)(842.40236816,72.54970998)
\lineto(842.46236816,72.27970998)
\curveto(842.47237312,72.19970643)(842.4873731,72.11970651)(842.50736816,72.03970998)
\curveto(842.61737297,71.73970689)(842.76237283,71.47470716)(842.94236816,71.24470998)
\curveto(843.12237247,71.02470761)(843.35237224,70.85470778)(843.63236816,70.73470998)
\curveto(843.71237188,70.70470793)(843.7923718,70.67970795)(843.87236816,70.65970998)
\curveto(843.95237164,70.63970799)(844.03737155,70.61970801)(844.12736816,70.59970998)
\curveto(844.24737134,70.56970806)(844.39737119,70.55970807)(844.57736816,70.56970998)
\curveto(844.75737083,70.58970804)(844.89737069,70.61470802)(844.99736816,70.64470998)
\curveto(845.04737054,70.66470797)(845.0923705,70.67470796)(845.13236816,70.67470998)
\curveto(845.16237043,70.68470795)(845.20237039,70.69970793)(845.25236816,70.71970998)
\curveto(845.47237012,70.81970781)(845.67236992,70.94970768)(845.85236816,71.10970998)
\curveto(846.03236956,71.27970735)(846.16736942,71.47470716)(846.25736816,71.69470998)
\curveto(846.29736929,71.76470687)(846.33236926,71.85970677)(846.36236816,71.97970998)
\curveto(846.45236914,72.19970643)(846.49736909,72.45470618)(846.49736816,72.74470998)
\lineto(846.49736816,73.02970998)
\curveto(846.47736911,73.1297055)(846.46236913,73.22470541)(846.45236816,73.31470998)
\curveto(846.44236915,73.40470523)(846.42236917,73.49470514)(846.39236816,73.58470998)
\curveto(846.31236928,73.84470479)(846.18236941,74.08470455)(846.00236816,74.30470998)
\curveto(845.81236978,74.5347041)(845.59736999,74.70470393)(845.35736816,74.81470998)
\curveto(845.27737031,74.85470378)(845.19737039,74.88470375)(845.11736816,74.90470998)
\curveto(845.02737056,74.9347037)(844.93237066,74.96470367)(844.83236816,74.99470998)
}
}
{
\newrgbcolor{curcolor}{0 0 0}
\pscustom[linestyle=none,fillstyle=solid,fillcolor=curcolor]
{
\newpath
\moveto(845.77736816,78.63431936)
\lineto(845.77736816,79.26431936)
\lineto(845.77736816,79.45931936)
\curveto(845.77736981,79.52931683)(845.7873698,79.58931677)(845.80736816,79.63931936)
\curveto(845.84736974,79.70931665)(845.8873697,79.7593166)(845.92736816,79.78931936)
\curveto(845.97736961,79.82931653)(846.04236955,79.84931651)(846.12236816,79.84931936)
\curveto(846.20236939,79.8593165)(846.2873693,79.86431649)(846.37736816,79.86431936)
\lineto(847.09736816,79.86431936)
\curveto(847.57736801,79.86431649)(847.9873676,79.80431655)(848.32736816,79.68431936)
\curveto(848.66736692,79.56431679)(848.94236665,79.36931699)(849.15236816,79.09931936)
\curveto(849.20236639,79.02931733)(849.24736634,78.9593174)(849.28736816,78.88931936)
\curveto(849.33736625,78.82931753)(849.38236621,78.7543176)(849.42236816,78.66431936)
\curveto(849.43236616,78.64431771)(849.44236615,78.61431774)(849.45236816,78.57431936)
\curveto(849.47236612,78.53431782)(849.47736611,78.48931787)(849.46736816,78.43931936)
\curveto(849.43736615,78.34931801)(849.36236623,78.29431806)(849.24236816,78.27431936)
\curveto(849.13236646,78.2543181)(849.03736655,78.26931809)(848.95736816,78.31931936)
\curveto(848.8873667,78.34931801)(848.82236677,78.39431796)(848.76236816,78.45431936)
\curveto(848.71236688,78.52431783)(848.66236693,78.58931777)(848.61236816,78.64931936)
\curveto(848.56236703,78.71931764)(848.4873671,78.77931758)(848.38736816,78.82931936)
\curveto(848.29736729,78.88931747)(848.20736738,78.93931742)(848.11736816,78.97931936)
\curveto(848.0873675,78.99931736)(848.02736756,79.02431733)(847.93736816,79.05431936)
\curveto(847.85736773,79.08431727)(847.7873678,79.08931727)(847.72736816,79.06931936)
\curveto(847.587368,79.03931732)(847.49736809,78.97931738)(847.45736816,78.88931936)
\curveto(847.42736816,78.80931755)(847.41236818,78.71931764)(847.41236816,78.61931936)
\curveto(847.41236818,78.51931784)(847.3873682,78.43431792)(847.33736816,78.36431936)
\curveto(847.26736832,78.27431808)(847.12736846,78.22931813)(846.91736816,78.22931936)
\lineto(846.36236816,78.22931936)
\lineto(846.13736816,78.22931936)
\curveto(846.05736953,78.23931812)(845.9923696,78.2593181)(845.94236816,78.28931936)
\curveto(845.86236973,78.34931801)(845.81736977,78.41931794)(845.80736816,78.49931936)
\curveto(845.79736979,78.51931784)(845.7923698,78.53931782)(845.79236816,78.55931936)
\curveto(845.7923698,78.58931777)(845.7873698,78.61431774)(845.77736816,78.63431936)
}
}
{
\newrgbcolor{curcolor}{0 0 0}
\pscustom[linestyle=none,fillstyle=solid,fillcolor=curcolor]
{
}
}
{
\newrgbcolor{curcolor}{0 0 0}
\pscustom[linestyle=none,fillstyle=solid,fillcolor=curcolor]
{
\newpath
\moveto(836.80736816,89.26463186)
\curveto(836.79737879,89.95462722)(836.91737867,90.55462662)(837.16736816,91.06463186)
\curveto(837.41737817,91.58462559)(837.75237784,91.9796252)(838.17236816,92.24963186)
\curveto(838.25237734,92.29962488)(838.34237725,92.34462483)(838.44236816,92.38463186)
\curveto(838.53237706,92.42462475)(838.62737696,92.46962471)(838.72736816,92.51963186)
\curveto(838.82737676,92.55962462)(838.92737666,92.58962459)(839.02736816,92.60963186)
\curveto(839.12737646,92.62962455)(839.23237636,92.64962453)(839.34236816,92.66963186)
\curveto(839.3923762,92.68962449)(839.43737615,92.69462448)(839.47736816,92.68463186)
\curveto(839.51737607,92.6746245)(839.56237603,92.6796245)(839.61236816,92.69963186)
\curveto(839.66237593,92.70962447)(839.74737584,92.71462446)(839.86736816,92.71463186)
\curveto(839.97737561,92.71462446)(840.06237553,92.70962447)(840.12236816,92.69963186)
\curveto(840.18237541,92.6796245)(840.24237535,92.66962451)(840.30236816,92.66963186)
\curveto(840.36237523,92.6796245)(840.42237517,92.6746245)(840.48236816,92.65463186)
\curveto(840.62237497,92.61462456)(840.75737483,92.5796246)(840.88736816,92.54963186)
\curveto(841.01737457,92.51962466)(841.14237445,92.4796247)(841.26236816,92.42963186)
\curveto(841.40237419,92.36962481)(841.52737406,92.29962488)(841.63736816,92.21963186)
\curveto(841.74737384,92.14962503)(841.85737373,92.0746251)(841.96736816,91.99463186)
\lineto(842.02736816,91.93463186)
\curveto(842.04737354,91.92462525)(842.06737352,91.90962527)(842.08736816,91.88963186)
\curveto(842.24737334,91.76962541)(842.3923732,91.63462554)(842.52236816,91.48463186)
\curveto(842.65237294,91.33462584)(842.77737281,91.174626)(842.89736816,91.00463186)
\curveto(843.11737247,90.69462648)(843.32237227,90.39962678)(843.51236816,90.11963186)
\curveto(843.65237194,89.88962729)(843.7873718,89.65962752)(843.91736816,89.42963186)
\curveto(844.04737154,89.20962797)(844.18237141,88.98962819)(844.32236816,88.76963186)
\curveto(844.4923711,88.51962866)(844.67237092,88.2796289)(844.86236816,88.04963186)
\curveto(845.05237054,87.82962935)(845.27737031,87.63962954)(845.53736816,87.47963186)
\curveto(845.59736999,87.43962974)(845.65736993,87.40462977)(845.71736816,87.37463186)
\curveto(845.76736982,87.34462983)(845.83236976,87.31462986)(845.91236816,87.28463186)
\curveto(845.98236961,87.26462991)(846.04236955,87.25962992)(846.09236816,87.26963186)
\curveto(846.16236943,87.28962989)(846.21736937,87.32462985)(846.25736816,87.37463186)
\curveto(846.2873693,87.42462975)(846.30736928,87.48462969)(846.31736816,87.55463186)
\lineto(846.31736816,87.79463186)
\lineto(846.31736816,88.54463186)
\lineto(846.31736816,91.34963186)
\lineto(846.31736816,92.00963186)
\curveto(846.31736927,92.09962508)(846.32236927,92.18462499)(846.33236816,92.26463186)
\curveto(846.33236926,92.34462483)(846.35236924,92.40962477)(846.39236816,92.45963186)
\curveto(846.43236916,92.50962467)(846.50736908,92.54962463)(846.61736816,92.57963186)
\curveto(846.71736887,92.61962456)(846.81736877,92.62962455)(846.91736816,92.60963186)
\lineto(847.05236816,92.60963186)
\curveto(847.12236847,92.58962459)(847.18236841,92.56962461)(847.23236816,92.54963186)
\curveto(847.28236831,92.52962465)(847.32236827,92.49462468)(847.35236816,92.44463186)
\curveto(847.3923682,92.39462478)(847.41236818,92.32462485)(847.41236816,92.23463186)
\lineto(847.41236816,91.96463186)
\lineto(847.41236816,91.06463186)
\lineto(847.41236816,87.55463186)
\lineto(847.41236816,86.48963186)
\curveto(847.41236818,86.40963077)(847.41736817,86.31963086)(847.42736816,86.21963186)
\curveto(847.42736816,86.11963106)(847.41736817,86.03463114)(847.39736816,85.96463186)
\curveto(847.32736826,85.75463142)(847.14736844,85.68963149)(846.85736816,85.76963186)
\curveto(846.81736877,85.7796314)(846.78236881,85.7796314)(846.75236816,85.76963186)
\curveto(846.71236888,85.76963141)(846.66736892,85.7796314)(846.61736816,85.79963186)
\curveto(846.53736905,85.81963136)(846.45236914,85.83963134)(846.36236816,85.85963186)
\curveto(846.27236932,85.8796313)(846.1873694,85.90463127)(846.10736816,85.93463186)
\curveto(845.61736997,86.09463108)(845.20237039,86.29463088)(844.86236816,86.53463186)
\curveto(844.61237098,86.71463046)(844.3873712,86.91963026)(844.18736816,87.14963186)
\curveto(843.97737161,87.3796298)(843.78237181,87.61962956)(843.60236816,87.86963186)
\curveto(843.42237217,88.12962905)(843.25237234,88.39462878)(843.09236816,88.66463186)
\curveto(842.92237267,88.94462823)(842.74737284,89.21462796)(842.56736816,89.47463186)
\curveto(842.4873731,89.58462759)(842.41237318,89.68962749)(842.34236816,89.78963186)
\curveto(842.27237332,89.89962728)(842.19737339,90.00962717)(842.11736816,90.11963186)
\curveto(842.0873735,90.15962702)(842.05737353,90.19462698)(842.02736816,90.22463186)
\curveto(841.9873736,90.26462691)(841.95737363,90.30462687)(841.93736816,90.34463186)
\curveto(841.82737376,90.48462669)(841.70237389,90.60962657)(841.56236816,90.71963186)
\curveto(841.53237406,90.73962644)(841.50737408,90.76462641)(841.48736816,90.79463186)
\curveto(841.45737413,90.82462635)(841.42737416,90.84962633)(841.39736816,90.86963186)
\curveto(841.29737429,90.94962623)(841.19737439,91.01462616)(841.09736816,91.06463186)
\curveto(840.99737459,91.12462605)(840.8873747,91.179626)(840.76736816,91.22963186)
\curveto(840.69737489,91.25962592)(840.62237497,91.2796259)(840.54236816,91.28963186)
\lineto(840.30236816,91.34963186)
\lineto(840.21236816,91.34963186)
\curveto(840.18237541,91.35962582)(840.15237544,91.36462581)(840.12236816,91.36463186)
\curveto(840.05237554,91.38462579)(839.95737563,91.38962579)(839.83736816,91.37963186)
\curveto(839.70737588,91.3796258)(839.60737598,91.36962581)(839.53736816,91.34963186)
\curveto(839.45737613,91.32962585)(839.38237621,91.30962587)(839.31236816,91.28963186)
\curveto(839.23237636,91.2796259)(839.15237644,91.25962592)(839.07236816,91.22963186)
\curveto(838.83237676,91.11962606)(838.63237696,90.96962621)(838.47236816,90.77963186)
\curveto(838.30237729,90.59962658)(838.16237743,90.3796268)(838.05236816,90.11963186)
\curveto(838.03237756,90.04962713)(838.01737757,89.9796272)(838.00736816,89.90963186)
\curveto(837.9873776,89.83962734)(837.96737762,89.76462741)(837.94736816,89.68463186)
\curveto(837.92737766,89.60462757)(837.91737767,89.49462768)(837.91736816,89.35463186)
\curveto(837.91737767,89.22462795)(837.92737766,89.11962806)(837.94736816,89.03963186)
\curveto(837.95737763,88.9796282)(837.96237763,88.92462825)(837.96236816,88.87463186)
\curveto(837.96237763,88.82462835)(837.97237762,88.7746284)(837.99236816,88.72463186)
\curveto(838.03237756,88.62462855)(838.07237752,88.52962865)(838.11236816,88.43963186)
\curveto(838.15237744,88.35962882)(838.19737739,88.2796289)(838.24736816,88.19963186)
\curveto(838.26737732,88.16962901)(838.2923773,88.13962904)(838.32236816,88.10963186)
\curveto(838.35237724,88.08962909)(838.37737721,88.06462911)(838.39736816,88.03463186)
\lineto(838.47236816,87.95963186)
\curveto(838.4923771,87.92962925)(838.51237708,87.90462927)(838.53236816,87.88463186)
\lineto(838.74236816,87.73463186)
\curveto(838.80237679,87.69462948)(838.86737672,87.64962953)(838.93736816,87.59963186)
\curveto(839.02737656,87.53962964)(839.13237646,87.48962969)(839.25236816,87.44963186)
\curveto(839.36237623,87.41962976)(839.47237612,87.38462979)(839.58236816,87.34463186)
\curveto(839.6923759,87.30462987)(839.83737575,87.2796299)(840.01736816,87.26963186)
\curveto(840.1873754,87.25962992)(840.31237528,87.22962995)(840.39236816,87.17963186)
\curveto(840.47237512,87.12963005)(840.51737507,87.05463012)(840.52736816,86.95463186)
\curveto(840.53737505,86.85463032)(840.54237505,86.74463043)(840.54236816,86.62463186)
\curveto(840.54237505,86.58463059)(840.54737504,86.54463063)(840.55736816,86.50463186)
\curveto(840.55737503,86.46463071)(840.55237504,86.42963075)(840.54236816,86.39963186)
\curveto(840.52237507,86.34963083)(840.51237508,86.29963088)(840.51236816,86.24963186)
\curveto(840.51237508,86.20963097)(840.50237509,86.16963101)(840.48236816,86.12963186)
\curveto(840.42237517,86.03963114)(840.2873753,85.99463118)(840.07736816,85.99463186)
\lineto(839.95736816,85.99463186)
\curveto(839.89737569,86.00463117)(839.83737575,86.00963117)(839.77736816,86.00963186)
\curveto(839.70737588,86.01963116)(839.64237595,86.02963115)(839.58236816,86.03963186)
\curveto(839.47237612,86.05963112)(839.37237622,86.0796311)(839.28236816,86.09963186)
\curveto(839.18237641,86.11963106)(839.0873765,86.14963103)(838.99736816,86.18963186)
\curveto(838.92737666,86.20963097)(838.86737672,86.22963095)(838.81736816,86.24963186)
\lineto(838.63736816,86.30963186)
\curveto(838.37737721,86.42963075)(838.13237746,86.58463059)(837.90236816,86.77463186)
\curveto(837.67237792,86.9746302)(837.4873781,87.18962999)(837.34736816,87.41963186)
\curveto(837.26737832,87.52962965)(837.20237839,87.64462953)(837.15236816,87.76463186)
\lineto(837.00236816,88.15463186)
\curveto(836.95237864,88.26462891)(836.92237867,88.3796288)(836.91236816,88.49963186)
\curveto(836.8923787,88.61962856)(836.86737872,88.74462843)(836.83736816,88.87463186)
\curveto(836.83737875,88.94462823)(836.83737875,89.00962817)(836.83736816,89.06963186)
\curveto(836.82737876,89.12962805)(836.81737877,89.19462798)(836.80736816,89.26463186)
}
}
{
\newrgbcolor{curcolor}{0 0 0}
\pscustom[linestyle=none,fillstyle=solid,fillcolor=curcolor]
{
\newpath
\moveto(842.32736816,101.36424123)
\lineto(842.58236816,101.36424123)
\curveto(842.66237293,101.37423353)(842.73737285,101.36923353)(842.80736816,101.34924123)
\lineto(843.04736816,101.34924123)
\lineto(843.21236816,101.34924123)
\curveto(843.31237228,101.32923357)(843.41737217,101.31923358)(843.52736816,101.31924123)
\curveto(843.62737196,101.31923358)(843.72737186,101.30923359)(843.82736816,101.28924123)
\lineto(843.97736816,101.28924123)
\curveto(844.11737147,101.25923364)(844.25737133,101.23923366)(844.39736816,101.22924123)
\curveto(844.52737106,101.21923368)(844.65737093,101.19423371)(844.78736816,101.15424123)
\curveto(844.86737072,101.13423377)(844.95237064,101.11423379)(845.04236816,101.09424123)
\lineto(845.28236816,101.03424123)
\lineto(845.58236816,100.91424123)
\curveto(845.67236992,100.88423402)(845.76236983,100.84923405)(845.85236816,100.80924123)
\curveto(846.07236952,100.70923419)(846.2873693,100.57423433)(846.49736816,100.40424123)
\curveto(846.70736888,100.24423466)(846.87736871,100.06923483)(847.00736816,99.87924123)
\curveto(847.04736854,99.82923507)(847.0873685,99.76923513)(847.12736816,99.69924123)
\curveto(847.15736843,99.63923526)(847.1923684,99.57923532)(847.23236816,99.51924123)
\curveto(847.28236831,99.43923546)(847.32236827,99.34423556)(847.35236816,99.23424123)
\curveto(847.38236821,99.12423578)(847.41236818,99.01923588)(847.44236816,98.91924123)
\curveto(847.48236811,98.80923609)(847.50736808,98.6992362)(847.51736816,98.58924123)
\curveto(847.52736806,98.47923642)(847.54236805,98.36423654)(847.56236816,98.24424123)
\curveto(847.57236802,98.2042367)(847.57236802,98.15923674)(847.56236816,98.10924123)
\curveto(847.56236803,98.06923683)(847.56736802,98.02923687)(847.57736816,97.98924123)
\curveto(847.587368,97.94923695)(847.592368,97.89423701)(847.59236816,97.82424123)
\curveto(847.592368,97.75423715)(847.587368,97.7042372)(847.57736816,97.67424123)
\curveto(847.55736803,97.62423728)(847.55236804,97.57923732)(847.56236816,97.53924123)
\curveto(847.57236802,97.4992374)(847.57236802,97.46423744)(847.56236816,97.43424123)
\lineto(847.56236816,97.34424123)
\curveto(847.54236805,97.28423762)(847.52736806,97.21923768)(847.51736816,97.14924123)
\curveto(847.51736807,97.08923781)(847.51236808,97.02423788)(847.50236816,96.95424123)
\curveto(847.45236814,96.78423812)(847.40236819,96.62423828)(847.35236816,96.47424123)
\curveto(847.30236829,96.32423858)(847.23736835,96.17923872)(847.15736816,96.03924123)
\curveto(847.11736847,95.98923891)(847.0873685,95.93423897)(847.06736816,95.87424123)
\curveto(847.03736855,95.82423908)(847.00236859,95.77423913)(846.96236816,95.72424123)
\curveto(846.78236881,95.48423942)(846.56236903,95.28423962)(846.30236816,95.12424123)
\curveto(846.04236955,94.96423994)(845.75736983,94.82424008)(845.44736816,94.70424123)
\curveto(845.30737028,94.64424026)(845.16737042,94.5992403)(845.02736816,94.56924123)
\curveto(844.87737071,94.53924036)(844.72237087,94.5042404)(844.56236816,94.46424123)
\curveto(844.45237114,94.44424046)(844.34237125,94.42924047)(844.23236816,94.41924123)
\curveto(844.12237147,94.40924049)(844.01237158,94.39424051)(843.90236816,94.37424123)
\curveto(843.86237173,94.36424054)(843.82237177,94.35924054)(843.78236816,94.35924123)
\curveto(843.74237185,94.36924053)(843.70237189,94.36924053)(843.66236816,94.35924123)
\curveto(843.61237198,94.34924055)(843.56237203,94.34424056)(843.51236816,94.34424123)
\lineto(843.34736816,94.34424123)
\curveto(843.29737229,94.32424058)(843.24737234,94.31924058)(843.19736816,94.32924123)
\curveto(843.13737245,94.33924056)(843.08237251,94.33924056)(843.03236816,94.32924123)
\curveto(842.9923726,94.31924058)(842.94737264,94.31924058)(842.89736816,94.32924123)
\curveto(842.84737274,94.33924056)(842.79737279,94.33424057)(842.74736816,94.31424123)
\curveto(842.67737291,94.29424061)(842.60237299,94.28924061)(842.52236816,94.29924123)
\curveto(842.43237316,94.30924059)(842.34737324,94.31424059)(842.26736816,94.31424123)
\curveto(842.17737341,94.31424059)(842.07737351,94.30924059)(841.96736816,94.29924123)
\curveto(841.84737374,94.28924061)(841.74737384,94.29424061)(841.66736816,94.31424123)
\lineto(841.38236816,94.31424123)
\lineto(840.75236816,94.35924123)
\curveto(840.65237494,94.36924053)(840.55737503,94.37924052)(840.46736816,94.38924123)
\lineto(840.16736816,94.41924123)
\curveto(840.11737547,94.43924046)(840.06737552,94.44424046)(840.01736816,94.43424123)
\curveto(839.95737563,94.43424047)(839.90237569,94.44424046)(839.85236816,94.46424123)
\curveto(839.68237591,94.51424039)(839.51737607,94.55424035)(839.35736816,94.58424123)
\curveto(839.1873764,94.61424029)(839.02737656,94.66424024)(838.87736816,94.73424123)
\curveto(838.41737717,94.92423998)(838.04237755,95.14423976)(837.75236816,95.39424123)
\curveto(837.46237813,95.65423925)(837.21737837,96.01423889)(837.01736816,96.47424123)
\curveto(836.96737862,96.6042383)(836.93237866,96.73423817)(836.91236816,96.86424123)
\curveto(836.8923787,97.0042379)(836.86737872,97.14423776)(836.83736816,97.28424123)
\curveto(836.82737876,97.35423755)(836.82237877,97.41923748)(836.82236816,97.47924123)
\curveto(836.82237877,97.53923736)(836.81737877,97.6042373)(836.80736816,97.67424123)
\curveto(836.7873788,98.5042364)(836.93737865,99.17423573)(837.25736816,99.68424123)
\curveto(837.56737802,100.19423471)(838.00737758,100.57423433)(838.57736816,100.82424123)
\curveto(838.69737689,100.87423403)(838.82237677,100.91923398)(838.95236816,100.95924123)
\curveto(839.08237651,100.9992339)(839.21737637,101.04423386)(839.35736816,101.09424123)
\curveto(839.43737615,101.11423379)(839.52237607,101.12923377)(839.61236816,101.13924123)
\lineto(839.85236816,101.19924123)
\curveto(839.96237563,101.22923367)(840.07237552,101.24423366)(840.18236816,101.24424123)
\curveto(840.2923753,101.25423365)(840.40237519,101.26923363)(840.51236816,101.28924123)
\curveto(840.56237503,101.30923359)(840.60737498,101.31423359)(840.64736816,101.30424123)
\curveto(840.6873749,101.3042336)(840.72737486,101.30923359)(840.76736816,101.31924123)
\curveto(840.81737477,101.32923357)(840.87237472,101.32923357)(840.93236816,101.31924123)
\curveto(840.98237461,101.31923358)(841.03237456,101.32423358)(841.08236816,101.33424123)
\lineto(841.21736816,101.33424123)
\curveto(841.27737431,101.35423355)(841.34737424,101.35423355)(841.42736816,101.33424123)
\curveto(841.49737409,101.32423358)(841.56237403,101.32923357)(841.62236816,101.34924123)
\curveto(841.65237394,101.35923354)(841.6923739,101.36423354)(841.74236816,101.36424123)
\lineto(841.86236816,101.36424123)
\lineto(842.32736816,101.36424123)
\moveto(844.65236816,99.81924123)
\curveto(844.33237126,99.91923498)(843.96737162,99.97923492)(843.55736816,99.99924123)
\curveto(843.14737244,100.01923488)(842.73737285,100.02923487)(842.32736816,100.02924123)
\curveto(841.89737369,100.02923487)(841.47737411,100.01923488)(841.06736816,99.99924123)
\curveto(840.65737493,99.97923492)(840.27237532,99.93423497)(839.91236816,99.86424123)
\curveto(839.55237604,99.79423511)(839.23237636,99.68423522)(838.95236816,99.53424123)
\curveto(838.66237693,99.39423551)(838.42737716,99.1992357)(838.24736816,98.94924123)
\curveto(838.13737745,98.78923611)(838.05737753,98.60923629)(838.00736816,98.40924123)
\curveto(837.94737764,98.20923669)(837.91737767,97.96423694)(837.91736816,97.67424123)
\curveto(837.93737765,97.65423725)(837.94737764,97.61923728)(837.94736816,97.56924123)
\curveto(837.93737765,97.51923738)(837.93737765,97.47923742)(837.94736816,97.44924123)
\curveto(837.96737762,97.36923753)(837.9873776,97.29423761)(838.00736816,97.22424123)
\curveto(838.01737757,97.16423774)(838.03737755,97.0992378)(838.06736816,97.02924123)
\curveto(838.1873774,96.75923814)(838.35737723,96.53923836)(838.57736816,96.36924123)
\curveto(838.7873768,96.20923869)(839.03237656,96.07423883)(839.31236816,95.96424123)
\curveto(839.42237617,95.91423899)(839.54237605,95.87423903)(839.67236816,95.84424123)
\curveto(839.7923758,95.82423908)(839.91737567,95.7992391)(840.04736816,95.76924123)
\curveto(840.09737549,95.74923915)(840.15237544,95.73923916)(840.21236816,95.73924123)
\curveto(840.26237533,95.73923916)(840.31237528,95.73423917)(840.36236816,95.72424123)
\curveto(840.45237514,95.71423919)(840.54737504,95.7042392)(840.64736816,95.69424123)
\curveto(840.73737485,95.68423922)(840.83237476,95.67423923)(840.93236816,95.66424123)
\curveto(841.01237458,95.66423924)(841.09737449,95.65923924)(841.18736816,95.64924123)
\lineto(841.42736816,95.64924123)
\lineto(841.60736816,95.64924123)
\curveto(841.63737395,95.63923926)(841.67237392,95.63423927)(841.71236816,95.63424123)
\lineto(841.84736816,95.63424123)
\lineto(842.29736816,95.63424123)
\curveto(842.37737321,95.63423927)(842.46237313,95.62923927)(842.55236816,95.61924123)
\curveto(842.63237296,95.61923928)(842.70737288,95.62923927)(842.77736816,95.64924123)
\lineto(843.04736816,95.64924123)
\curveto(843.06737252,95.64923925)(843.09737249,95.64423926)(843.13736816,95.63424123)
\curveto(843.16737242,95.63423927)(843.1923724,95.63923926)(843.21236816,95.64924123)
\curveto(843.31237228,95.65923924)(843.41237218,95.66423924)(843.51236816,95.66424123)
\curveto(843.60237199,95.67423923)(843.70237189,95.68423922)(843.81236816,95.69424123)
\curveto(843.93237166,95.72423918)(844.05737153,95.73923916)(844.18736816,95.73924123)
\curveto(844.30737128,95.74923915)(844.42237117,95.77423913)(844.53236816,95.81424123)
\curveto(844.83237076,95.89423901)(845.09737049,95.97923892)(845.32736816,96.06924123)
\curveto(845.55737003,96.16923873)(845.77236982,96.31423859)(845.97236816,96.50424123)
\curveto(846.17236942,96.71423819)(846.32236927,96.97923792)(846.42236816,97.29924123)
\curveto(846.44236915,97.33923756)(846.45236914,97.37423753)(846.45236816,97.40424123)
\curveto(846.44236915,97.44423746)(846.44736914,97.48923741)(846.46736816,97.53924123)
\curveto(846.47736911,97.57923732)(846.4873691,97.64923725)(846.49736816,97.74924123)
\curveto(846.50736908,97.85923704)(846.50236909,97.94423696)(846.48236816,98.00424123)
\curveto(846.46236913,98.07423683)(846.45236914,98.14423676)(846.45236816,98.21424123)
\curveto(846.44236915,98.28423662)(846.42736916,98.34923655)(846.40736816,98.40924123)
\curveto(846.34736924,98.60923629)(846.26236933,98.78923611)(846.15236816,98.94924123)
\curveto(846.13236946,98.97923592)(846.11236948,99.0042359)(846.09236816,99.02424123)
\lineto(846.03236816,99.08424123)
\curveto(846.01236958,99.12423578)(845.97236962,99.17423573)(845.91236816,99.23424123)
\curveto(845.77236982,99.33423557)(845.64236995,99.41923548)(845.52236816,99.48924123)
\curveto(845.40237019,99.55923534)(845.25737033,99.62923527)(845.08736816,99.69924123)
\curveto(845.01737057,99.72923517)(844.94737064,99.74923515)(844.87736816,99.75924123)
\curveto(844.80737078,99.77923512)(844.73237086,99.7992351)(844.65236816,99.81924123)
}
}
{
\newrgbcolor{curcolor}{0 0 0}
\pscustom[linestyle=none,fillstyle=solid,fillcolor=curcolor]
{
\newpath
\moveto(836.80736816,106.77385061)
\curveto(836.80737878,106.87384575)(836.81737877,106.96884566)(836.83736816,107.05885061)
\curveto(836.84737874,107.14884548)(836.87737871,107.21384541)(836.92736816,107.25385061)
\curveto(837.00737858,107.31384531)(837.11237848,107.34384528)(837.24236816,107.34385061)
\lineto(837.63236816,107.34385061)
\lineto(839.13236816,107.34385061)
\lineto(845.52236816,107.34385061)
\lineto(846.69236816,107.34385061)
\lineto(847.00736816,107.34385061)
\curveto(847.10736848,107.35384527)(847.1873684,107.33884529)(847.24736816,107.29885061)
\curveto(847.32736826,107.24884538)(847.37736821,107.17384545)(847.39736816,107.07385061)
\curveto(847.40736818,106.98384564)(847.41236818,106.87384575)(847.41236816,106.74385061)
\lineto(847.41236816,106.51885061)
\curveto(847.3923682,106.43884619)(847.37736821,106.36884626)(847.36736816,106.30885061)
\curveto(847.34736824,106.24884638)(847.30736828,106.19884643)(847.24736816,106.15885061)
\curveto(847.1873684,106.11884651)(847.11236848,106.09884653)(847.02236816,106.09885061)
\lineto(846.72236816,106.09885061)
\lineto(845.62736816,106.09885061)
\lineto(840.28736816,106.09885061)
\curveto(840.19737539,106.07884655)(840.12237547,106.06384656)(840.06236816,106.05385061)
\curveto(839.9923756,106.05384657)(839.93237566,106.0238466)(839.88236816,105.96385061)
\curveto(839.83237576,105.89384673)(839.80737578,105.80384682)(839.80736816,105.69385061)
\curveto(839.79737579,105.59384703)(839.7923758,105.48384714)(839.79236816,105.36385061)
\lineto(839.79236816,104.22385061)
\lineto(839.79236816,103.72885061)
\curveto(839.78237581,103.56884906)(839.72237587,103.45884917)(839.61236816,103.39885061)
\curveto(839.58237601,103.37884925)(839.55237604,103.36884926)(839.52236816,103.36885061)
\curveto(839.48237611,103.36884926)(839.43737615,103.36384926)(839.38736816,103.35385061)
\curveto(839.26737632,103.33384929)(839.15737643,103.33884929)(839.05736816,103.36885061)
\curveto(838.95737663,103.40884922)(838.8873767,103.46384916)(838.84736816,103.53385061)
\curveto(838.79737679,103.61384901)(838.77237682,103.73384889)(838.77236816,103.89385061)
\curveto(838.77237682,104.05384857)(838.75737683,104.18884844)(838.72736816,104.29885061)
\curveto(838.71737687,104.34884828)(838.71237688,104.40384822)(838.71236816,104.46385061)
\curveto(838.70237689,104.5238481)(838.6873769,104.58384804)(838.66736816,104.64385061)
\curveto(838.61737697,104.79384783)(838.56737702,104.93884769)(838.51736816,105.07885061)
\curveto(838.45737713,105.21884741)(838.3873772,105.35384727)(838.30736816,105.48385061)
\curveto(838.21737737,105.623847)(838.11237748,105.74384688)(837.99236816,105.84385061)
\curveto(837.87237772,105.94384668)(837.74237785,106.03884659)(837.60236816,106.12885061)
\curveto(837.50237809,106.18884644)(837.3923782,106.23384639)(837.27236816,106.26385061)
\curveto(837.15237844,106.30384632)(837.04737854,106.35384627)(836.95736816,106.41385061)
\curveto(836.89737869,106.46384616)(836.85737873,106.53384609)(836.83736816,106.62385061)
\curveto(836.82737876,106.64384598)(836.82237877,106.66884596)(836.82236816,106.69885061)
\curveto(836.82237877,106.7288459)(836.81737877,106.75384587)(836.80736816,106.77385061)
}
}
{
\newrgbcolor{curcolor}{0 0 0}
\pscustom[linestyle=none,fillstyle=solid,fillcolor=curcolor]
{
\newpath
\moveto(836.80736816,115.12345998)
\curveto(836.80737878,115.22345513)(836.81737877,115.31845503)(836.83736816,115.40845998)
\curveto(836.84737874,115.49845485)(836.87737871,115.56345479)(836.92736816,115.60345998)
\curveto(837.00737858,115.66345469)(837.11237848,115.69345466)(837.24236816,115.69345998)
\lineto(837.63236816,115.69345998)
\lineto(839.13236816,115.69345998)
\lineto(845.52236816,115.69345998)
\lineto(846.69236816,115.69345998)
\lineto(847.00736816,115.69345998)
\curveto(847.10736848,115.70345465)(847.1873684,115.68845466)(847.24736816,115.64845998)
\curveto(847.32736826,115.59845475)(847.37736821,115.52345483)(847.39736816,115.42345998)
\curveto(847.40736818,115.33345502)(847.41236818,115.22345513)(847.41236816,115.09345998)
\lineto(847.41236816,114.86845998)
\curveto(847.3923682,114.78845556)(847.37736821,114.71845563)(847.36736816,114.65845998)
\curveto(847.34736824,114.59845575)(847.30736828,114.5484558)(847.24736816,114.50845998)
\curveto(847.1873684,114.46845588)(847.11236848,114.4484559)(847.02236816,114.44845998)
\lineto(846.72236816,114.44845998)
\lineto(845.62736816,114.44845998)
\lineto(840.28736816,114.44845998)
\curveto(840.19737539,114.42845592)(840.12237547,114.41345594)(840.06236816,114.40345998)
\curveto(839.9923756,114.40345595)(839.93237566,114.37345598)(839.88236816,114.31345998)
\curveto(839.83237576,114.24345611)(839.80737578,114.1534562)(839.80736816,114.04345998)
\curveto(839.79737579,113.94345641)(839.7923758,113.83345652)(839.79236816,113.71345998)
\lineto(839.79236816,112.57345998)
\lineto(839.79236816,112.07845998)
\curveto(839.78237581,111.91845843)(839.72237587,111.80845854)(839.61236816,111.74845998)
\curveto(839.58237601,111.72845862)(839.55237604,111.71845863)(839.52236816,111.71845998)
\curveto(839.48237611,111.71845863)(839.43737615,111.71345864)(839.38736816,111.70345998)
\curveto(839.26737632,111.68345867)(839.15737643,111.68845866)(839.05736816,111.71845998)
\curveto(838.95737663,111.75845859)(838.8873767,111.81345854)(838.84736816,111.88345998)
\curveto(838.79737679,111.96345839)(838.77237682,112.08345827)(838.77236816,112.24345998)
\curveto(838.77237682,112.40345795)(838.75737683,112.53845781)(838.72736816,112.64845998)
\curveto(838.71737687,112.69845765)(838.71237688,112.7534576)(838.71236816,112.81345998)
\curveto(838.70237689,112.87345748)(838.6873769,112.93345742)(838.66736816,112.99345998)
\curveto(838.61737697,113.14345721)(838.56737702,113.28845706)(838.51736816,113.42845998)
\curveto(838.45737713,113.56845678)(838.3873772,113.70345665)(838.30736816,113.83345998)
\curveto(838.21737737,113.97345638)(838.11237748,114.09345626)(837.99236816,114.19345998)
\curveto(837.87237772,114.29345606)(837.74237785,114.38845596)(837.60236816,114.47845998)
\curveto(837.50237809,114.53845581)(837.3923782,114.58345577)(837.27236816,114.61345998)
\curveto(837.15237844,114.6534557)(837.04737854,114.70345565)(836.95736816,114.76345998)
\curveto(836.89737869,114.81345554)(836.85737873,114.88345547)(836.83736816,114.97345998)
\curveto(836.82737876,114.99345536)(836.82237877,115.01845533)(836.82236816,115.04845998)
\curveto(836.82237877,115.07845527)(836.81737877,115.10345525)(836.80736816,115.12345998)
}
}
{
\newrgbcolor{curcolor}{0 0 0}
\pscustom[linestyle=none,fillstyle=solid,fillcolor=curcolor]
{
\newpath
\moveto(857.64368408,42.29681936)
\curveto(857.64369478,42.36681368)(857.64369478,42.4468136)(857.64368408,42.53681936)
\curveto(857.63369479,42.62681342)(857.63369479,42.71181333)(857.64368408,42.79181936)
\curveto(857.64369478,42.88181316)(857.65369477,42.96181308)(857.67368408,43.03181936)
\curveto(857.69369473,43.11181293)(857.7236947,43.16681288)(857.76368408,43.19681936)
\curveto(857.81369461,43.22681282)(857.88869453,43.2468128)(857.98868408,43.25681936)
\curveto(858.07869434,43.27681277)(858.18369424,43.28681276)(858.30368408,43.28681936)
\curveto(858.41369401,43.29681275)(858.52869389,43.29681275)(858.64868408,43.28681936)
\lineto(858.94868408,43.28681936)
\lineto(861.96368408,43.28681936)
\lineto(864.85868408,43.28681936)
\curveto(865.18868723,43.28681276)(865.51368691,43.28181276)(865.83368408,43.27181936)
\curveto(866.14368628,43.27181277)(866.423686,43.23181281)(866.67368408,43.15181936)
\curveto(867.0236854,43.03181301)(867.3186851,42.87681317)(867.55868408,42.68681936)
\curveto(867.78868463,42.49681355)(867.98868443,42.25681379)(868.15868408,41.96681936)
\curveto(868.20868421,41.90681414)(868.24368418,41.8418142)(868.26368408,41.77181936)
\curveto(868.28368414,41.71181433)(868.30868411,41.6418144)(868.33868408,41.56181936)
\curveto(868.38868403,41.4418146)(868.423684,41.31181473)(868.44368408,41.17181936)
\curveto(868.47368395,41.041815)(868.50368392,40.90681514)(868.53368408,40.76681936)
\curveto(868.55368387,40.71681533)(868.55868386,40.66681538)(868.54868408,40.61681936)
\curveto(868.53868388,40.56681548)(868.53868388,40.51181553)(868.54868408,40.45181936)
\curveto(868.55868386,40.43181561)(868.55868386,40.40681564)(868.54868408,40.37681936)
\curveto(868.54868387,40.3468157)(868.55368387,40.32181572)(868.56368408,40.30181936)
\curveto(868.57368385,40.26181578)(868.57868384,40.20681584)(868.57868408,40.13681936)
\curveto(868.57868384,40.06681598)(868.57368385,40.01181603)(868.56368408,39.97181936)
\curveto(868.55368387,39.92181612)(868.55368387,39.86681618)(868.56368408,39.80681936)
\curveto(868.57368385,39.7468163)(868.56868385,39.69181635)(868.54868408,39.64181936)
\curveto(868.5186839,39.51181653)(868.49868392,39.38681666)(868.48868408,39.26681936)
\curveto(868.47868394,39.1468169)(868.45368397,39.03181701)(868.41368408,38.92181936)
\curveto(868.29368413,38.55181749)(868.1236843,38.23181781)(867.90368408,37.96181936)
\curveto(867.68368474,37.69181835)(867.40368502,37.48181856)(867.06368408,37.33181936)
\curveto(866.94368548,37.28181876)(866.8186856,37.23681881)(866.68868408,37.19681936)
\curveto(866.55868586,37.16681888)(866.423686,37.13181891)(866.28368408,37.09181936)
\curveto(866.23368619,37.08181896)(866.19368623,37.07681897)(866.16368408,37.07681936)
\curveto(866.1236863,37.07681897)(866.07868634,37.07181897)(866.02868408,37.06181936)
\curveto(865.99868642,37.05181899)(865.96368646,37.046819)(865.92368408,37.04681936)
\curveto(865.87368655,37.046819)(865.83368659,37.041819)(865.80368408,37.03181936)
\lineto(865.63868408,37.03181936)
\curveto(865.55868686,37.01181903)(865.45868696,37.00681904)(865.33868408,37.01681936)
\curveto(865.20868721,37.02681902)(865.1186873,37.041819)(865.06868408,37.06181936)
\curveto(864.97868744,37.08181896)(864.91368751,37.13681891)(864.87368408,37.22681936)
\curveto(864.85368757,37.25681879)(864.84868757,37.28681876)(864.85868408,37.31681936)
\curveto(864.85868756,37.3468187)(864.85368757,37.38681866)(864.84368408,37.43681936)
\curveto(864.83368759,37.47681857)(864.82868759,37.51681853)(864.82868408,37.55681936)
\lineto(864.82868408,37.70681936)
\curveto(864.82868759,37.82681822)(864.83368759,37.9468181)(864.84368408,38.06681936)
\curveto(864.84368758,38.19681785)(864.87868754,38.28681776)(864.94868408,38.33681936)
\curveto(865.00868741,38.37681767)(865.06868735,38.39681765)(865.12868408,38.39681936)
\curveto(865.18868723,38.39681765)(865.25868716,38.40681764)(865.33868408,38.42681936)
\curveto(865.36868705,38.43681761)(865.40368702,38.43681761)(865.44368408,38.42681936)
\curveto(865.47368695,38.42681762)(865.49868692,38.43181761)(865.51868408,38.44181936)
\lineto(865.72868408,38.44181936)
\curveto(865.77868664,38.46181758)(865.82868659,38.46681758)(865.87868408,38.45681936)
\curveto(865.9186865,38.45681759)(865.96368646,38.46681758)(866.01368408,38.48681936)
\curveto(866.14368628,38.51681753)(866.26868615,38.5468175)(866.38868408,38.57681936)
\curveto(866.49868592,38.60681744)(866.60368582,38.65181739)(866.70368408,38.71181936)
\curveto(866.99368543,38.88181716)(867.19868522,39.15181689)(867.31868408,39.52181936)
\curveto(867.33868508,39.57181647)(867.35368507,39.62181642)(867.36368408,39.67181936)
\curveto(867.36368506,39.73181631)(867.37368505,39.78681626)(867.39368408,39.83681936)
\lineto(867.39368408,39.91181936)
\curveto(867.40368502,39.98181606)(867.41368501,40.07681597)(867.42368408,40.19681936)
\curveto(867.423685,40.32681572)(867.41368501,40.42681562)(867.39368408,40.49681936)
\curveto(867.37368505,40.56681548)(867.35868506,40.63681541)(867.34868408,40.70681936)
\curveto(867.32868509,40.78681526)(867.30868511,40.85681519)(867.28868408,40.91681936)
\curveto(867.12868529,41.29681475)(866.85368557,41.57181447)(866.46368408,41.74181936)
\curveto(866.33368609,41.79181425)(866.17868624,41.82681422)(865.99868408,41.84681936)
\curveto(865.8186866,41.87681417)(865.63368679,41.89181415)(865.44368408,41.89181936)
\curveto(865.24368718,41.90181414)(865.04368738,41.90181414)(864.84368408,41.89181936)
\lineto(864.27368408,41.89181936)
\lineto(860.02868408,41.89181936)
\lineto(858.48368408,41.89181936)
\curveto(858.37369405,41.89181415)(858.25369417,41.88681416)(858.12368408,41.87681936)
\curveto(857.99369443,41.86681418)(857.88869453,41.88681416)(857.80868408,41.93681936)
\curveto(857.73869468,41.99681405)(857.68869473,42.07681397)(857.65868408,42.17681936)
\curveto(857.65869476,42.19681385)(857.65869476,42.21681383)(857.65868408,42.23681936)
\curveto(857.65869476,42.25681379)(857.65369477,42.27681377)(857.64368408,42.29681936)
}
}
{
\newrgbcolor{curcolor}{0 0 0}
\pscustom[linestyle=none,fillstyle=solid,fillcolor=curcolor]
{
\newpath
\moveto(860.59868408,45.83049123)
\lineto(860.59868408,46.26549123)
\curveto(860.59869182,46.41548927)(860.63869178,46.52048916)(860.71868408,46.58049123)
\curveto(860.79869162,46.63048905)(860.89869152,46.65548903)(861.01868408,46.65549123)
\curveto(861.13869128,46.66548902)(861.25869116,46.67048901)(861.37868408,46.67049123)
\lineto(862.80368408,46.67049123)
\lineto(865.06868408,46.67049123)
\lineto(865.75868408,46.67049123)
\curveto(865.98868643,46.67048901)(866.18868623,46.69548899)(866.35868408,46.74549123)
\curveto(866.80868561,46.90548878)(867.1236853,47.20548848)(867.30368408,47.64549123)
\curveto(867.39368503,47.86548782)(867.42868499,48.13048755)(867.40868408,48.44049123)
\curveto(867.37868504,48.75048693)(867.3236851,49.00048668)(867.24368408,49.19049123)
\curveto(867.10368532,49.52048616)(866.92868549,49.7804859)(866.71868408,49.97049123)
\curveto(866.49868592,50.17048551)(866.21368621,50.32548536)(865.86368408,50.43549123)
\curveto(865.78368664,50.46548522)(865.70368672,50.4854852)(865.62368408,50.49549123)
\curveto(865.54368688,50.50548518)(865.45868696,50.52048516)(865.36868408,50.54049123)
\curveto(865.3186871,50.55048513)(865.27368715,50.55048513)(865.23368408,50.54049123)
\curveto(865.19368723,50.54048514)(865.14868727,50.55048513)(865.09868408,50.57049123)
\lineto(864.78368408,50.57049123)
\curveto(864.70368772,50.59048509)(864.61368781,50.59548509)(864.51368408,50.58549123)
\curveto(864.40368802,50.57548511)(864.30368812,50.57048511)(864.21368408,50.57049123)
\lineto(863.04368408,50.57049123)
\lineto(861.45368408,50.57049123)
\curveto(861.33369109,50.57048511)(861.20869121,50.56548512)(861.07868408,50.55549123)
\curveto(860.93869148,50.55548513)(860.82869159,50.5804851)(860.74868408,50.63049123)
\curveto(860.69869172,50.67048501)(860.66869175,50.71548497)(860.65868408,50.76549123)
\curveto(860.63869178,50.82548486)(860.6186918,50.89548479)(860.59868408,50.97549123)
\lineto(860.59868408,51.20049123)
\curveto(860.59869182,51.32048436)(860.60369182,51.42548426)(860.61368408,51.51549123)
\curveto(860.6236918,51.61548407)(860.66869175,51.69048399)(860.74868408,51.74049123)
\curveto(860.79869162,51.79048389)(860.87369155,51.81548387)(860.97368408,51.81549123)
\lineto(861.25868408,51.81549123)
\lineto(862.27868408,51.81549123)
\lineto(866.31368408,51.81549123)
\lineto(867.66368408,51.81549123)
\curveto(867.78368464,51.81548387)(867.89868452,51.81048387)(868.00868408,51.80049123)
\curveto(868.10868431,51.80048388)(868.18368424,51.76548392)(868.23368408,51.69549123)
\curveto(868.26368416,51.65548403)(868.28868413,51.59548409)(868.30868408,51.51549123)
\curveto(868.3186841,51.43548425)(868.32868409,51.34548434)(868.33868408,51.24549123)
\curveto(868.33868408,51.15548453)(868.33368409,51.06548462)(868.32368408,50.97549123)
\curveto(868.31368411,50.89548479)(868.29368413,50.83548485)(868.26368408,50.79549123)
\curveto(868.2236842,50.74548494)(868.15868426,50.70048498)(868.06868408,50.66049123)
\curveto(868.02868439,50.65048503)(867.97368445,50.64048504)(867.90368408,50.63049123)
\curveto(867.83368459,50.63048505)(867.76868465,50.62548506)(867.70868408,50.61549123)
\curveto(867.63868478,50.60548508)(867.58368484,50.5854851)(867.54368408,50.55549123)
\curveto(867.50368492,50.52548516)(867.48868493,50.4804852)(867.49868408,50.42049123)
\curveto(867.5186849,50.34048534)(867.57868484,50.26048542)(867.67868408,50.18049123)
\curveto(867.76868465,50.10048558)(867.83868458,50.02548566)(867.88868408,49.95549123)
\curveto(868.04868437,49.73548595)(868.18868423,49.4854862)(868.30868408,49.20549123)
\curveto(868.35868406,49.09548659)(868.38868403,48.9804867)(868.39868408,48.86049123)
\curveto(868.418684,48.75048693)(868.44368398,48.63548705)(868.47368408,48.51549123)
\curveto(868.48368394,48.46548722)(868.48368394,48.41048727)(868.47368408,48.35049123)
\curveto(868.46368396,48.30048738)(868.46868395,48.25048743)(868.48868408,48.20049123)
\curveto(868.50868391,48.10048758)(868.50868391,48.01048767)(868.48868408,47.93049123)
\lineto(868.48868408,47.78049123)
\curveto(868.46868395,47.73048795)(868.45868396,47.67048801)(868.45868408,47.60049123)
\curveto(868.45868396,47.54048814)(868.45368397,47.4854882)(868.44368408,47.43549123)
\curveto(868.423684,47.39548829)(868.41368401,47.35548833)(868.41368408,47.31549123)
\curveto(868.423684,47.2854884)(868.418684,47.24548844)(868.39868408,47.19549123)
\lineto(868.33868408,46.95549123)
\curveto(868.3186841,46.8854888)(868.28868413,46.81048887)(868.24868408,46.73049123)
\curveto(868.13868428,46.47048921)(867.99368443,46.25048943)(867.81368408,46.07049123)
\curveto(867.6236848,45.90048978)(867.39868502,45.76048992)(867.13868408,45.65049123)
\curveto(867.04868537,45.61049007)(866.95868546,45.5804901)(866.86868408,45.56049123)
\lineto(866.56868408,45.50049123)
\curveto(866.50868591,45.4804902)(866.45368597,45.47049021)(866.40368408,45.47049123)
\curveto(866.34368608,45.4804902)(866.27868614,45.47549021)(866.20868408,45.45549123)
\curveto(866.18868623,45.44549024)(866.16368626,45.44049024)(866.13368408,45.44049123)
\curveto(866.09368633,45.44049024)(866.05868636,45.43549025)(866.02868408,45.42549123)
\lineto(865.87868408,45.42549123)
\curveto(865.83868658,45.41549027)(865.79368663,45.41049027)(865.74368408,45.41049123)
\curveto(865.68368674,45.42049026)(865.62868679,45.42549026)(865.57868408,45.42549123)
\lineto(864.97868408,45.42549123)
\lineto(862.21868408,45.42549123)
\lineto(861.25868408,45.42549123)
\lineto(860.98868408,45.42549123)
\curveto(860.89869152,45.42549026)(860.8236916,45.44549024)(860.76368408,45.48549123)
\curveto(860.69369173,45.52549016)(860.64369178,45.60049008)(860.61368408,45.71049123)
\curveto(860.60369182,45.73048995)(860.60369182,45.75048993)(860.61368408,45.77049123)
\curveto(860.61369181,45.79048989)(860.60869181,45.81048987)(860.59868408,45.83049123)
}
}
{
\newrgbcolor{curcolor}{0 0 0}
\pscustom[linestyle=none,fillstyle=solid,fillcolor=curcolor]
{
\newpath
\moveto(857.64368408,54.28510061)
\curveto(857.64369478,54.41509899)(857.64369478,54.55009886)(857.64368408,54.69010061)
\curveto(857.64369478,54.84009857)(857.67869474,54.95009846)(857.74868408,55.02010061)
\curveto(857.8186946,55.07009834)(857.91369451,55.09509831)(858.03368408,55.09510061)
\curveto(858.14369428,55.1050983)(858.25869416,55.1100983)(858.37868408,55.11010061)
\lineto(859.71368408,55.11010061)
\lineto(865.78868408,55.11010061)
\lineto(867.46868408,55.11010061)
\lineto(867.85868408,55.11010061)
\curveto(867.99868442,55.1100983)(868.10868431,55.08509832)(868.18868408,55.03510061)
\curveto(868.23868418,55.0050984)(868.26868415,54.96009845)(868.27868408,54.90010061)
\curveto(868.28868413,54.85009856)(868.30368412,54.78509862)(868.32368408,54.70510061)
\lineto(868.32368408,54.49510061)
\lineto(868.32368408,54.18010061)
\curveto(868.31368411,54.08009933)(868.27868414,54.0050994)(868.21868408,53.95510061)
\curveto(868.13868428,53.9050995)(868.03868438,53.87509953)(867.91868408,53.86510061)
\lineto(867.54368408,53.86510061)
\lineto(866.16368408,53.86510061)
\lineto(859.92368408,53.86510061)
\lineto(858.45368408,53.86510061)
\curveto(858.34369408,53.86509954)(858.22869419,53.86009955)(858.10868408,53.85010061)
\curveto(857.97869444,53.85009956)(857.87869454,53.87509953)(857.80868408,53.92510061)
\curveto(857.74869467,53.96509944)(857.69869472,54.04009937)(857.65868408,54.15010061)
\curveto(857.64869477,54.17009924)(857.64869477,54.19009922)(857.65868408,54.21010061)
\curveto(857.65869476,54.24009917)(857.65369477,54.26509914)(857.64368408,54.28510061)
}
}
{
\newrgbcolor{curcolor}{0 0 0}
\pscustom[linestyle=none,fillstyle=solid,fillcolor=curcolor]
{
}
}
{
\newrgbcolor{curcolor}{0 0 0}
\pscustom[linestyle=none,fillstyle=solid,fillcolor=curcolor]
{
\newpath
\moveto(857.71868408,65.05510061)
\curveto(857.7186947,65.15509575)(857.72869469,65.25009566)(857.74868408,65.34010061)
\curveto(857.75869466,65.43009548)(857.78869463,65.49509541)(857.83868408,65.53510061)
\curveto(857.9186945,65.59509531)(858.0236944,65.62509528)(858.15368408,65.62510061)
\lineto(858.54368408,65.62510061)
\lineto(860.04368408,65.62510061)
\lineto(866.43368408,65.62510061)
\lineto(867.60368408,65.62510061)
\lineto(867.91868408,65.62510061)
\curveto(868.0186844,65.63509527)(868.09868432,65.62009529)(868.15868408,65.58010061)
\curveto(868.23868418,65.53009538)(868.28868413,65.45509545)(868.30868408,65.35510061)
\curveto(868.3186841,65.26509564)(868.3236841,65.15509575)(868.32368408,65.02510061)
\lineto(868.32368408,64.80010061)
\curveto(868.30368412,64.72009619)(868.28868413,64.65009626)(868.27868408,64.59010061)
\curveto(868.25868416,64.53009638)(868.2186842,64.48009643)(868.15868408,64.44010061)
\curveto(868.09868432,64.40009651)(868.0236844,64.38009653)(867.93368408,64.38010061)
\lineto(867.63368408,64.38010061)
\lineto(866.53868408,64.38010061)
\lineto(861.19868408,64.38010061)
\curveto(861.10869131,64.36009655)(861.03369139,64.34509656)(860.97368408,64.33510061)
\curveto(860.90369152,64.33509657)(860.84369158,64.3050966)(860.79368408,64.24510061)
\curveto(860.74369168,64.17509673)(860.7186917,64.08509682)(860.71868408,63.97510061)
\curveto(860.70869171,63.87509703)(860.70369172,63.76509714)(860.70368408,63.64510061)
\lineto(860.70368408,62.50510061)
\lineto(860.70368408,62.01010061)
\curveto(860.69369173,61.85009906)(860.63369179,61.74009917)(860.52368408,61.68010061)
\curveto(860.49369193,61.66009925)(860.46369196,61.65009926)(860.43368408,61.65010061)
\curveto(860.39369203,61.65009926)(860.34869207,61.64509926)(860.29868408,61.63510061)
\curveto(860.17869224,61.61509929)(860.06869235,61.62009929)(859.96868408,61.65010061)
\curveto(859.86869255,61.69009922)(859.79869262,61.74509916)(859.75868408,61.81510061)
\curveto(859.70869271,61.89509901)(859.68369274,62.01509889)(859.68368408,62.17510061)
\curveto(859.68369274,62.33509857)(859.66869275,62.47009844)(859.63868408,62.58010061)
\curveto(859.62869279,62.63009828)(859.6236928,62.68509822)(859.62368408,62.74510061)
\curveto(859.61369281,62.8050981)(859.59869282,62.86509804)(859.57868408,62.92510061)
\curveto(859.52869289,63.07509783)(859.47869294,63.22009769)(859.42868408,63.36010061)
\curveto(859.36869305,63.50009741)(859.29869312,63.63509727)(859.21868408,63.76510061)
\curveto(859.12869329,63.905097)(859.0236934,64.02509688)(858.90368408,64.12510061)
\curveto(858.78369364,64.22509668)(858.65369377,64.32009659)(858.51368408,64.41010061)
\curveto(858.41369401,64.47009644)(858.30369412,64.51509639)(858.18368408,64.54510061)
\curveto(858.06369436,64.58509632)(857.95869446,64.63509627)(857.86868408,64.69510061)
\curveto(857.80869461,64.74509616)(857.76869465,64.81509609)(857.74868408,64.90510061)
\curveto(857.73869468,64.92509598)(857.73369469,64.95009596)(857.73368408,64.98010061)
\curveto(857.73369469,65.0100959)(857.72869469,65.03509587)(857.71868408,65.05510061)
}
}
{
\newrgbcolor{curcolor}{0 0 0}
\pscustom[linestyle=none,fillstyle=solid,fillcolor=curcolor]
{
\newpath
\moveto(857.71868408,72.59470998)
\curveto(857.70869471,73.28470535)(857.82869459,73.88470475)(858.07868408,74.39470998)
\curveto(858.32869409,74.91470372)(858.66369376,75.30970332)(859.08368408,75.57970998)
\curveto(859.16369326,75.629703)(859.25369317,75.67470296)(859.35368408,75.71470998)
\curveto(859.44369298,75.75470288)(859.53869288,75.79970283)(859.63868408,75.84970998)
\curveto(859.73869268,75.88970274)(859.83869258,75.91970271)(859.93868408,75.93970998)
\curveto(860.03869238,75.95970267)(860.14369228,75.97970265)(860.25368408,75.99970998)
\curveto(860.30369212,76.01970261)(860.34869207,76.02470261)(860.38868408,76.01470998)
\curveto(860.42869199,76.00470263)(860.47369195,76.00970262)(860.52368408,76.02970998)
\curveto(860.57369185,76.03970259)(860.65869176,76.04470259)(860.77868408,76.04470998)
\curveto(860.88869153,76.04470259)(860.97369145,76.03970259)(861.03368408,76.02970998)
\curveto(861.09369133,76.00970262)(861.15369127,75.99970263)(861.21368408,75.99970998)
\curveto(861.27369115,76.00970262)(861.33369109,76.00470263)(861.39368408,75.98470998)
\curveto(861.53369089,75.94470269)(861.66869075,75.90970272)(861.79868408,75.87970998)
\curveto(861.92869049,75.84970278)(862.05369037,75.80970282)(862.17368408,75.75970998)
\curveto(862.31369011,75.69970293)(862.43868998,75.629703)(862.54868408,75.54970998)
\curveto(862.65868976,75.47970315)(862.76868965,75.40470323)(862.87868408,75.32470998)
\lineto(862.93868408,75.26470998)
\curveto(862.95868946,75.25470338)(862.97868944,75.23970339)(862.99868408,75.21970998)
\curveto(863.15868926,75.09970353)(863.30368912,74.96470367)(863.43368408,74.81470998)
\curveto(863.56368886,74.66470397)(863.68868873,74.50470413)(863.80868408,74.33470998)
\curveto(864.02868839,74.02470461)(864.23368819,73.7297049)(864.42368408,73.44970998)
\curveto(864.56368786,73.21970541)(864.69868772,72.98970564)(864.82868408,72.75970998)
\curveto(864.95868746,72.53970609)(865.09368733,72.31970631)(865.23368408,72.09970998)
\curveto(865.40368702,71.84970678)(865.58368684,71.60970702)(865.77368408,71.37970998)
\curveto(865.96368646,71.15970747)(866.18868623,70.96970766)(866.44868408,70.80970998)
\curveto(866.50868591,70.76970786)(866.56868585,70.7347079)(866.62868408,70.70470998)
\curveto(866.67868574,70.67470796)(866.74368568,70.64470799)(866.82368408,70.61470998)
\curveto(866.89368553,70.59470804)(866.95368547,70.58970804)(867.00368408,70.59970998)
\curveto(867.07368535,70.61970801)(867.12868529,70.65470798)(867.16868408,70.70470998)
\curveto(867.19868522,70.75470788)(867.2186852,70.81470782)(867.22868408,70.88470998)
\lineto(867.22868408,71.12470998)
\lineto(867.22868408,71.87470998)
\lineto(867.22868408,74.67970998)
\lineto(867.22868408,75.33970998)
\curveto(867.22868519,75.4297032)(867.23368519,75.51470312)(867.24368408,75.59470998)
\curveto(867.24368518,75.67470296)(867.26368516,75.73970289)(867.30368408,75.78970998)
\curveto(867.34368508,75.83970279)(867.418685,75.87970275)(867.52868408,75.90970998)
\curveto(867.62868479,75.94970268)(867.72868469,75.95970267)(867.82868408,75.93970998)
\lineto(867.96368408,75.93970998)
\curveto(868.03368439,75.91970271)(868.09368433,75.89970273)(868.14368408,75.87970998)
\curveto(868.19368423,75.85970277)(868.23368419,75.82470281)(868.26368408,75.77470998)
\curveto(868.30368412,75.72470291)(868.3236841,75.65470298)(868.32368408,75.56470998)
\lineto(868.32368408,75.29470998)
\lineto(868.32368408,74.39470998)
\lineto(868.32368408,70.88470998)
\lineto(868.32368408,69.81970998)
\curveto(868.3236841,69.73970889)(868.32868409,69.64970898)(868.33868408,69.54970998)
\curveto(868.33868408,69.44970918)(868.32868409,69.36470927)(868.30868408,69.29470998)
\curveto(868.23868418,69.08470955)(868.05868436,69.01970961)(867.76868408,69.09970998)
\curveto(867.72868469,69.10970952)(867.69368473,69.10970952)(867.66368408,69.09970998)
\curveto(867.6236848,69.09970953)(867.57868484,69.10970952)(867.52868408,69.12970998)
\curveto(867.44868497,69.14970948)(867.36368506,69.16970946)(867.27368408,69.18970998)
\curveto(867.18368524,69.20970942)(867.09868532,69.2347094)(867.01868408,69.26470998)
\curveto(866.52868589,69.42470921)(866.11368631,69.62470901)(865.77368408,69.86470998)
\curveto(865.5236869,70.04470859)(865.29868712,70.24970838)(865.09868408,70.47970998)
\curveto(864.88868753,70.70970792)(864.69368773,70.94970768)(864.51368408,71.19970998)
\curveto(864.33368809,71.45970717)(864.16368826,71.72470691)(864.00368408,71.99470998)
\curveto(863.83368859,72.27470636)(863.65868876,72.54470609)(863.47868408,72.80470998)
\curveto(863.39868902,72.91470572)(863.3236891,73.01970561)(863.25368408,73.11970998)
\curveto(863.18368924,73.2297054)(863.10868931,73.33970529)(863.02868408,73.44970998)
\curveto(862.99868942,73.48970514)(862.96868945,73.52470511)(862.93868408,73.55470998)
\curveto(862.89868952,73.59470504)(862.86868955,73.634705)(862.84868408,73.67470998)
\curveto(862.73868968,73.81470482)(862.61368981,73.93970469)(862.47368408,74.04970998)
\curveto(862.44368998,74.06970456)(862.41869,74.09470454)(862.39868408,74.12470998)
\curveto(862.36869005,74.15470448)(862.33869008,74.17970445)(862.30868408,74.19970998)
\curveto(862.20869021,74.27970435)(862.10869031,74.34470429)(862.00868408,74.39470998)
\curveto(861.90869051,74.45470418)(861.79869062,74.50970412)(861.67868408,74.55970998)
\curveto(861.60869081,74.58970404)(861.53369089,74.60970402)(861.45368408,74.61970998)
\lineto(861.21368408,74.67970998)
\lineto(861.12368408,74.67970998)
\curveto(861.09369133,74.68970394)(861.06369136,74.69470394)(861.03368408,74.69470998)
\curveto(860.96369146,74.71470392)(860.86869155,74.71970391)(860.74868408,74.70970998)
\curveto(860.6186918,74.70970392)(860.5186919,74.69970393)(860.44868408,74.67970998)
\curveto(860.36869205,74.65970397)(860.29369213,74.63970399)(860.22368408,74.61970998)
\curveto(860.14369228,74.60970402)(860.06369236,74.58970404)(859.98368408,74.55970998)
\curveto(859.74369268,74.44970418)(859.54369288,74.29970433)(859.38368408,74.10970998)
\curveto(859.21369321,73.9297047)(859.07369335,73.70970492)(858.96368408,73.44970998)
\curveto(858.94369348,73.37970525)(858.92869349,73.30970532)(858.91868408,73.23970998)
\curveto(858.89869352,73.16970546)(858.87869354,73.09470554)(858.85868408,73.01470998)
\curveto(858.83869358,72.9347057)(858.82869359,72.82470581)(858.82868408,72.68470998)
\curveto(858.82869359,72.55470608)(858.83869358,72.44970618)(858.85868408,72.36970998)
\curveto(858.86869355,72.30970632)(858.87369355,72.25470638)(858.87368408,72.20470998)
\curveto(858.87369355,72.15470648)(858.88369354,72.10470653)(858.90368408,72.05470998)
\curveto(858.94369348,71.95470668)(858.98369344,71.85970677)(859.02368408,71.76970998)
\curveto(859.06369336,71.68970694)(859.10869331,71.60970702)(859.15868408,71.52970998)
\curveto(859.17869324,71.49970713)(859.20369322,71.46970716)(859.23368408,71.43970998)
\curveto(859.26369316,71.41970721)(859.28869313,71.39470724)(859.30868408,71.36470998)
\lineto(859.38368408,71.28970998)
\curveto(859.40369302,71.25970737)(859.423693,71.2347074)(859.44368408,71.21470998)
\lineto(859.65368408,71.06470998)
\curveto(859.71369271,71.02470761)(859.77869264,70.97970765)(859.84868408,70.92970998)
\curveto(859.93869248,70.86970776)(860.04369238,70.81970781)(860.16368408,70.77970998)
\curveto(860.27369215,70.74970788)(860.38369204,70.71470792)(860.49368408,70.67470998)
\curveto(860.60369182,70.634708)(860.74869167,70.60970802)(860.92868408,70.59970998)
\curveto(861.09869132,70.58970804)(861.2236912,70.55970807)(861.30368408,70.50970998)
\curveto(861.38369104,70.45970817)(861.42869099,70.38470825)(861.43868408,70.28470998)
\curveto(861.44869097,70.18470845)(861.45369097,70.07470856)(861.45368408,69.95470998)
\curveto(861.45369097,69.91470872)(861.45869096,69.87470876)(861.46868408,69.83470998)
\curveto(861.46869095,69.79470884)(861.46369096,69.75970887)(861.45368408,69.72970998)
\curveto(861.43369099,69.67970895)(861.423691,69.629709)(861.42368408,69.57970998)
\curveto(861.423691,69.53970909)(861.41369101,69.49970913)(861.39368408,69.45970998)
\curveto(861.33369109,69.36970926)(861.19869122,69.32470931)(860.98868408,69.32470998)
\lineto(860.86868408,69.32470998)
\curveto(860.80869161,69.3347093)(860.74869167,69.33970929)(860.68868408,69.33970998)
\curveto(860.6186918,69.34970928)(860.55369187,69.35970927)(860.49368408,69.36970998)
\curveto(860.38369204,69.38970924)(860.28369214,69.40970922)(860.19368408,69.42970998)
\curveto(860.09369233,69.44970918)(859.99869242,69.47970915)(859.90868408,69.51970998)
\curveto(859.83869258,69.53970909)(859.77869264,69.55970907)(859.72868408,69.57970998)
\lineto(859.54868408,69.63970998)
\curveto(859.28869313,69.75970887)(859.04369338,69.91470872)(858.81368408,70.10470998)
\curveto(858.58369384,70.30470833)(858.39869402,70.51970811)(858.25868408,70.74970998)
\curveto(858.17869424,70.85970777)(858.11369431,70.97470766)(858.06368408,71.09470998)
\lineto(857.91368408,71.48470998)
\curveto(857.86369456,71.59470704)(857.83369459,71.70970692)(857.82368408,71.82970998)
\curveto(857.80369462,71.94970668)(857.77869464,72.07470656)(857.74868408,72.20470998)
\curveto(857.74869467,72.27470636)(857.74869467,72.33970629)(857.74868408,72.39970998)
\curveto(857.73869468,72.45970617)(857.72869469,72.52470611)(857.71868408,72.59470998)
}
}
{
\newrgbcolor{curcolor}{0 0 0}
\pscustom[linestyle=none,fillstyle=solid,fillcolor=curcolor]
{
\newpath
\moveto(866.68868408,78.63431936)
\lineto(866.68868408,79.26431936)
\lineto(866.68868408,79.45931936)
\curveto(866.68868573,79.52931683)(866.69868572,79.58931677)(866.71868408,79.63931936)
\curveto(866.75868566,79.70931665)(866.79868562,79.7593166)(866.83868408,79.78931936)
\curveto(866.88868553,79.82931653)(866.95368547,79.84931651)(867.03368408,79.84931936)
\curveto(867.11368531,79.8593165)(867.19868522,79.86431649)(867.28868408,79.86431936)
\lineto(868.00868408,79.86431936)
\curveto(868.48868393,79.86431649)(868.89868352,79.80431655)(869.23868408,79.68431936)
\curveto(869.57868284,79.56431679)(869.85368257,79.36931699)(870.06368408,79.09931936)
\curveto(870.11368231,79.02931733)(870.15868226,78.9593174)(870.19868408,78.88931936)
\curveto(870.24868217,78.82931753)(870.29368213,78.7543176)(870.33368408,78.66431936)
\curveto(870.34368208,78.64431771)(870.35368207,78.61431774)(870.36368408,78.57431936)
\curveto(870.38368204,78.53431782)(870.38868203,78.48931787)(870.37868408,78.43931936)
\curveto(870.34868207,78.34931801)(870.27368215,78.29431806)(870.15368408,78.27431936)
\curveto(870.04368238,78.2543181)(869.94868247,78.26931809)(869.86868408,78.31931936)
\curveto(869.79868262,78.34931801)(869.73368269,78.39431796)(869.67368408,78.45431936)
\curveto(869.6236828,78.52431783)(869.57368285,78.58931777)(869.52368408,78.64931936)
\curveto(869.47368295,78.71931764)(869.39868302,78.77931758)(869.29868408,78.82931936)
\curveto(869.20868321,78.88931747)(869.1186833,78.93931742)(869.02868408,78.97931936)
\curveto(868.99868342,78.99931736)(868.93868348,79.02431733)(868.84868408,79.05431936)
\curveto(868.76868365,79.08431727)(868.69868372,79.08931727)(868.63868408,79.06931936)
\curveto(868.49868392,79.03931732)(868.40868401,78.97931738)(868.36868408,78.88931936)
\curveto(868.33868408,78.80931755)(868.3236841,78.71931764)(868.32368408,78.61931936)
\curveto(868.3236841,78.51931784)(868.29868412,78.43431792)(868.24868408,78.36431936)
\curveto(868.17868424,78.27431808)(868.03868438,78.22931813)(867.82868408,78.22931936)
\lineto(867.27368408,78.22931936)
\lineto(867.04868408,78.22931936)
\curveto(866.96868545,78.23931812)(866.90368552,78.2593181)(866.85368408,78.28931936)
\curveto(866.77368565,78.34931801)(866.72868569,78.41931794)(866.71868408,78.49931936)
\curveto(866.70868571,78.51931784)(866.70368572,78.53931782)(866.70368408,78.55931936)
\curveto(866.70368572,78.58931777)(866.69868572,78.61431774)(866.68868408,78.63431936)
}
}
{
\newrgbcolor{curcolor}{0 0 0}
\pscustom[linestyle=none,fillstyle=solid,fillcolor=curcolor]
{
}
}
{
\newrgbcolor{curcolor}{0 0 0}
\pscustom[linestyle=none,fillstyle=solid,fillcolor=curcolor]
{
\newpath
\moveto(857.71868408,89.26463186)
\curveto(857.70869471,89.95462722)(857.82869459,90.55462662)(858.07868408,91.06463186)
\curveto(858.32869409,91.58462559)(858.66369376,91.9796252)(859.08368408,92.24963186)
\curveto(859.16369326,92.29962488)(859.25369317,92.34462483)(859.35368408,92.38463186)
\curveto(859.44369298,92.42462475)(859.53869288,92.46962471)(859.63868408,92.51963186)
\curveto(859.73869268,92.55962462)(859.83869258,92.58962459)(859.93868408,92.60963186)
\curveto(860.03869238,92.62962455)(860.14369228,92.64962453)(860.25368408,92.66963186)
\curveto(860.30369212,92.68962449)(860.34869207,92.69462448)(860.38868408,92.68463186)
\curveto(860.42869199,92.6746245)(860.47369195,92.6796245)(860.52368408,92.69963186)
\curveto(860.57369185,92.70962447)(860.65869176,92.71462446)(860.77868408,92.71463186)
\curveto(860.88869153,92.71462446)(860.97369145,92.70962447)(861.03368408,92.69963186)
\curveto(861.09369133,92.6796245)(861.15369127,92.66962451)(861.21368408,92.66963186)
\curveto(861.27369115,92.6796245)(861.33369109,92.6746245)(861.39368408,92.65463186)
\curveto(861.53369089,92.61462456)(861.66869075,92.5796246)(861.79868408,92.54963186)
\curveto(861.92869049,92.51962466)(862.05369037,92.4796247)(862.17368408,92.42963186)
\curveto(862.31369011,92.36962481)(862.43868998,92.29962488)(862.54868408,92.21963186)
\curveto(862.65868976,92.14962503)(862.76868965,92.0746251)(862.87868408,91.99463186)
\lineto(862.93868408,91.93463186)
\curveto(862.95868946,91.92462525)(862.97868944,91.90962527)(862.99868408,91.88963186)
\curveto(863.15868926,91.76962541)(863.30368912,91.63462554)(863.43368408,91.48463186)
\curveto(863.56368886,91.33462584)(863.68868873,91.174626)(863.80868408,91.00463186)
\curveto(864.02868839,90.69462648)(864.23368819,90.39962678)(864.42368408,90.11963186)
\curveto(864.56368786,89.88962729)(864.69868772,89.65962752)(864.82868408,89.42963186)
\curveto(864.95868746,89.20962797)(865.09368733,88.98962819)(865.23368408,88.76963186)
\curveto(865.40368702,88.51962866)(865.58368684,88.2796289)(865.77368408,88.04963186)
\curveto(865.96368646,87.82962935)(866.18868623,87.63962954)(866.44868408,87.47963186)
\curveto(866.50868591,87.43962974)(866.56868585,87.40462977)(866.62868408,87.37463186)
\curveto(866.67868574,87.34462983)(866.74368568,87.31462986)(866.82368408,87.28463186)
\curveto(866.89368553,87.26462991)(866.95368547,87.25962992)(867.00368408,87.26963186)
\curveto(867.07368535,87.28962989)(867.12868529,87.32462985)(867.16868408,87.37463186)
\curveto(867.19868522,87.42462975)(867.2186852,87.48462969)(867.22868408,87.55463186)
\lineto(867.22868408,87.79463186)
\lineto(867.22868408,88.54463186)
\lineto(867.22868408,91.34963186)
\lineto(867.22868408,92.00963186)
\curveto(867.22868519,92.09962508)(867.23368519,92.18462499)(867.24368408,92.26463186)
\curveto(867.24368518,92.34462483)(867.26368516,92.40962477)(867.30368408,92.45963186)
\curveto(867.34368508,92.50962467)(867.418685,92.54962463)(867.52868408,92.57963186)
\curveto(867.62868479,92.61962456)(867.72868469,92.62962455)(867.82868408,92.60963186)
\lineto(867.96368408,92.60963186)
\curveto(868.03368439,92.58962459)(868.09368433,92.56962461)(868.14368408,92.54963186)
\curveto(868.19368423,92.52962465)(868.23368419,92.49462468)(868.26368408,92.44463186)
\curveto(868.30368412,92.39462478)(868.3236841,92.32462485)(868.32368408,92.23463186)
\lineto(868.32368408,91.96463186)
\lineto(868.32368408,91.06463186)
\lineto(868.32368408,87.55463186)
\lineto(868.32368408,86.48963186)
\curveto(868.3236841,86.40963077)(868.32868409,86.31963086)(868.33868408,86.21963186)
\curveto(868.33868408,86.11963106)(868.32868409,86.03463114)(868.30868408,85.96463186)
\curveto(868.23868418,85.75463142)(868.05868436,85.68963149)(867.76868408,85.76963186)
\curveto(867.72868469,85.7796314)(867.69368473,85.7796314)(867.66368408,85.76963186)
\curveto(867.6236848,85.76963141)(867.57868484,85.7796314)(867.52868408,85.79963186)
\curveto(867.44868497,85.81963136)(867.36368506,85.83963134)(867.27368408,85.85963186)
\curveto(867.18368524,85.8796313)(867.09868532,85.90463127)(867.01868408,85.93463186)
\curveto(866.52868589,86.09463108)(866.11368631,86.29463088)(865.77368408,86.53463186)
\curveto(865.5236869,86.71463046)(865.29868712,86.91963026)(865.09868408,87.14963186)
\curveto(864.88868753,87.3796298)(864.69368773,87.61962956)(864.51368408,87.86963186)
\curveto(864.33368809,88.12962905)(864.16368826,88.39462878)(864.00368408,88.66463186)
\curveto(863.83368859,88.94462823)(863.65868876,89.21462796)(863.47868408,89.47463186)
\curveto(863.39868902,89.58462759)(863.3236891,89.68962749)(863.25368408,89.78963186)
\curveto(863.18368924,89.89962728)(863.10868931,90.00962717)(863.02868408,90.11963186)
\curveto(862.99868942,90.15962702)(862.96868945,90.19462698)(862.93868408,90.22463186)
\curveto(862.89868952,90.26462691)(862.86868955,90.30462687)(862.84868408,90.34463186)
\curveto(862.73868968,90.48462669)(862.61368981,90.60962657)(862.47368408,90.71963186)
\curveto(862.44368998,90.73962644)(862.41869,90.76462641)(862.39868408,90.79463186)
\curveto(862.36869005,90.82462635)(862.33869008,90.84962633)(862.30868408,90.86963186)
\curveto(862.20869021,90.94962623)(862.10869031,91.01462616)(862.00868408,91.06463186)
\curveto(861.90869051,91.12462605)(861.79869062,91.179626)(861.67868408,91.22963186)
\curveto(861.60869081,91.25962592)(861.53369089,91.2796259)(861.45368408,91.28963186)
\lineto(861.21368408,91.34963186)
\lineto(861.12368408,91.34963186)
\curveto(861.09369133,91.35962582)(861.06369136,91.36462581)(861.03368408,91.36463186)
\curveto(860.96369146,91.38462579)(860.86869155,91.38962579)(860.74868408,91.37963186)
\curveto(860.6186918,91.3796258)(860.5186919,91.36962581)(860.44868408,91.34963186)
\curveto(860.36869205,91.32962585)(860.29369213,91.30962587)(860.22368408,91.28963186)
\curveto(860.14369228,91.2796259)(860.06369236,91.25962592)(859.98368408,91.22963186)
\curveto(859.74369268,91.11962606)(859.54369288,90.96962621)(859.38368408,90.77963186)
\curveto(859.21369321,90.59962658)(859.07369335,90.3796268)(858.96368408,90.11963186)
\curveto(858.94369348,90.04962713)(858.92869349,89.9796272)(858.91868408,89.90963186)
\curveto(858.89869352,89.83962734)(858.87869354,89.76462741)(858.85868408,89.68463186)
\curveto(858.83869358,89.60462757)(858.82869359,89.49462768)(858.82868408,89.35463186)
\curveto(858.82869359,89.22462795)(858.83869358,89.11962806)(858.85868408,89.03963186)
\curveto(858.86869355,88.9796282)(858.87369355,88.92462825)(858.87368408,88.87463186)
\curveto(858.87369355,88.82462835)(858.88369354,88.7746284)(858.90368408,88.72463186)
\curveto(858.94369348,88.62462855)(858.98369344,88.52962865)(859.02368408,88.43963186)
\curveto(859.06369336,88.35962882)(859.10869331,88.2796289)(859.15868408,88.19963186)
\curveto(859.17869324,88.16962901)(859.20369322,88.13962904)(859.23368408,88.10963186)
\curveto(859.26369316,88.08962909)(859.28869313,88.06462911)(859.30868408,88.03463186)
\lineto(859.38368408,87.95963186)
\curveto(859.40369302,87.92962925)(859.423693,87.90462927)(859.44368408,87.88463186)
\lineto(859.65368408,87.73463186)
\curveto(859.71369271,87.69462948)(859.77869264,87.64962953)(859.84868408,87.59963186)
\curveto(859.93869248,87.53962964)(860.04369238,87.48962969)(860.16368408,87.44963186)
\curveto(860.27369215,87.41962976)(860.38369204,87.38462979)(860.49368408,87.34463186)
\curveto(860.60369182,87.30462987)(860.74869167,87.2796299)(860.92868408,87.26963186)
\curveto(861.09869132,87.25962992)(861.2236912,87.22962995)(861.30368408,87.17963186)
\curveto(861.38369104,87.12963005)(861.42869099,87.05463012)(861.43868408,86.95463186)
\curveto(861.44869097,86.85463032)(861.45369097,86.74463043)(861.45368408,86.62463186)
\curveto(861.45369097,86.58463059)(861.45869096,86.54463063)(861.46868408,86.50463186)
\curveto(861.46869095,86.46463071)(861.46369096,86.42963075)(861.45368408,86.39963186)
\curveto(861.43369099,86.34963083)(861.423691,86.29963088)(861.42368408,86.24963186)
\curveto(861.423691,86.20963097)(861.41369101,86.16963101)(861.39368408,86.12963186)
\curveto(861.33369109,86.03963114)(861.19869122,85.99463118)(860.98868408,85.99463186)
\lineto(860.86868408,85.99463186)
\curveto(860.80869161,86.00463117)(860.74869167,86.00963117)(860.68868408,86.00963186)
\curveto(860.6186918,86.01963116)(860.55369187,86.02963115)(860.49368408,86.03963186)
\curveto(860.38369204,86.05963112)(860.28369214,86.0796311)(860.19368408,86.09963186)
\curveto(860.09369233,86.11963106)(859.99869242,86.14963103)(859.90868408,86.18963186)
\curveto(859.83869258,86.20963097)(859.77869264,86.22963095)(859.72868408,86.24963186)
\lineto(859.54868408,86.30963186)
\curveto(859.28869313,86.42963075)(859.04369338,86.58463059)(858.81368408,86.77463186)
\curveto(858.58369384,86.9746302)(858.39869402,87.18962999)(858.25868408,87.41963186)
\curveto(858.17869424,87.52962965)(858.11369431,87.64462953)(858.06368408,87.76463186)
\lineto(857.91368408,88.15463186)
\curveto(857.86369456,88.26462891)(857.83369459,88.3796288)(857.82368408,88.49963186)
\curveto(857.80369462,88.61962856)(857.77869464,88.74462843)(857.74868408,88.87463186)
\curveto(857.74869467,88.94462823)(857.74869467,89.00962817)(857.74868408,89.06963186)
\curveto(857.73869468,89.12962805)(857.72869469,89.19462798)(857.71868408,89.26463186)
}
}
{
\newrgbcolor{curcolor}{0 0 0}
\pscustom[linestyle=none,fillstyle=solid,fillcolor=curcolor]
{
\newpath
\moveto(863.23868408,101.36424123)
\lineto(863.49368408,101.36424123)
\curveto(863.57368885,101.37423353)(863.64868877,101.36923353)(863.71868408,101.34924123)
\lineto(863.95868408,101.34924123)
\lineto(864.12368408,101.34924123)
\curveto(864.2236882,101.32923357)(864.32868809,101.31923358)(864.43868408,101.31924123)
\curveto(864.53868788,101.31923358)(864.63868778,101.30923359)(864.73868408,101.28924123)
\lineto(864.88868408,101.28924123)
\curveto(865.02868739,101.25923364)(865.16868725,101.23923366)(865.30868408,101.22924123)
\curveto(865.43868698,101.21923368)(865.56868685,101.19423371)(865.69868408,101.15424123)
\curveto(865.77868664,101.13423377)(865.86368656,101.11423379)(865.95368408,101.09424123)
\lineto(866.19368408,101.03424123)
\lineto(866.49368408,100.91424123)
\curveto(866.58368584,100.88423402)(866.67368575,100.84923405)(866.76368408,100.80924123)
\curveto(866.98368544,100.70923419)(867.19868522,100.57423433)(867.40868408,100.40424123)
\curveto(867.6186848,100.24423466)(867.78868463,100.06923483)(867.91868408,99.87924123)
\curveto(867.95868446,99.82923507)(867.99868442,99.76923513)(868.03868408,99.69924123)
\curveto(868.06868435,99.63923526)(868.10368432,99.57923532)(868.14368408,99.51924123)
\curveto(868.19368423,99.43923546)(868.23368419,99.34423556)(868.26368408,99.23424123)
\curveto(868.29368413,99.12423578)(868.3236841,99.01923588)(868.35368408,98.91924123)
\curveto(868.39368403,98.80923609)(868.418684,98.6992362)(868.42868408,98.58924123)
\curveto(868.43868398,98.47923642)(868.45368397,98.36423654)(868.47368408,98.24424123)
\curveto(868.48368394,98.2042367)(868.48368394,98.15923674)(868.47368408,98.10924123)
\curveto(868.47368395,98.06923683)(868.47868394,98.02923687)(868.48868408,97.98924123)
\curveto(868.49868392,97.94923695)(868.50368392,97.89423701)(868.50368408,97.82424123)
\curveto(868.50368392,97.75423715)(868.49868392,97.7042372)(868.48868408,97.67424123)
\curveto(868.46868395,97.62423728)(868.46368396,97.57923732)(868.47368408,97.53924123)
\curveto(868.48368394,97.4992374)(868.48368394,97.46423744)(868.47368408,97.43424123)
\lineto(868.47368408,97.34424123)
\curveto(868.45368397,97.28423762)(868.43868398,97.21923768)(868.42868408,97.14924123)
\curveto(868.42868399,97.08923781)(868.423684,97.02423788)(868.41368408,96.95424123)
\curveto(868.36368406,96.78423812)(868.31368411,96.62423828)(868.26368408,96.47424123)
\curveto(868.21368421,96.32423858)(868.14868427,96.17923872)(868.06868408,96.03924123)
\curveto(868.02868439,95.98923891)(867.99868442,95.93423897)(867.97868408,95.87424123)
\curveto(867.94868447,95.82423908)(867.91368451,95.77423913)(867.87368408,95.72424123)
\curveto(867.69368473,95.48423942)(867.47368495,95.28423962)(867.21368408,95.12424123)
\curveto(866.95368547,94.96423994)(866.66868575,94.82424008)(866.35868408,94.70424123)
\curveto(866.2186862,94.64424026)(866.07868634,94.5992403)(865.93868408,94.56924123)
\curveto(865.78868663,94.53924036)(865.63368679,94.5042404)(865.47368408,94.46424123)
\curveto(865.36368706,94.44424046)(865.25368717,94.42924047)(865.14368408,94.41924123)
\curveto(865.03368739,94.40924049)(864.9236875,94.39424051)(864.81368408,94.37424123)
\curveto(864.77368765,94.36424054)(864.73368769,94.35924054)(864.69368408,94.35924123)
\curveto(864.65368777,94.36924053)(864.61368781,94.36924053)(864.57368408,94.35924123)
\curveto(864.5236879,94.34924055)(864.47368795,94.34424056)(864.42368408,94.34424123)
\lineto(864.25868408,94.34424123)
\curveto(864.20868821,94.32424058)(864.15868826,94.31924058)(864.10868408,94.32924123)
\curveto(864.04868837,94.33924056)(863.99368843,94.33924056)(863.94368408,94.32924123)
\curveto(863.90368852,94.31924058)(863.85868856,94.31924058)(863.80868408,94.32924123)
\curveto(863.75868866,94.33924056)(863.70868871,94.33424057)(863.65868408,94.31424123)
\curveto(863.58868883,94.29424061)(863.51368891,94.28924061)(863.43368408,94.29924123)
\curveto(863.34368908,94.30924059)(863.25868916,94.31424059)(863.17868408,94.31424123)
\curveto(863.08868933,94.31424059)(862.98868943,94.30924059)(862.87868408,94.29924123)
\curveto(862.75868966,94.28924061)(862.65868976,94.29424061)(862.57868408,94.31424123)
\lineto(862.29368408,94.31424123)
\lineto(861.66368408,94.35924123)
\curveto(861.56369086,94.36924053)(861.46869095,94.37924052)(861.37868408,94.38924123)
\lineto(861.07868408,94.41924123)
\curveto(861.02869139,94.43924046)(860.97869144,94.44424046)(860.92868408,94.43424123)
\curveto(860.86869155,94.43424047)(860.81369161,94.44424046)(860.76368408,94.46424123)
\curveto(860.59369183,94.51424039)(860.42869199,94.55424035)(860.26868408,94.58424123)
\curveto(860.09869232,94.61424029)(859.93869248,94.66424024)(859.78868408,94.73424123)
\curveto(859.32869309,94.92423998)(858.95369347,95.14423976)(858.66368408,95.39424123)
\curveto(858.37369405,95.65423925)(858.12869429,96.01423889)(857.92868408,96.47424123)
\curveto(857.87869454,96.6042383)(857.84369458,96.73423817)(857.82368408,96.86424123)
\curveto(857.80369462,97.0042379)(857.77869464,97.14423776)(857.74868408,97.28424123)
\curveto(857.73869468,97.35423755)(857.73369469,97.41923748)(857.73368408,97.47924123)
\curveto(857.73369469,97.53923736)(857.72869469,97.6042373)(857.71868408,97.67424123)
\curveto(857.69869472,98.5042364)(857.84869457,99.17423573)(858.16868408,99.68424123)
\curveto(858.47869394,100.19423471)(858.9186935,100.57423433)(859.48868408,100.82424123)
\curveto(859.60869281,100.87423403)(859.73369269,100.91923398)(859.86368408,100.95924123)
\curveto(859.99369243,100.9992339)(860.12869229,101.04423386)(860.26868408,101.09424123)
\curveto(860.34869207,101.11423379)(860.43369199,101.12923377)(860.52368408,101.13924123)
\lineto(860.76368408,101.19924123)
\curveto(860.87369155,101.22923367)(860.98369144,101.24423366)(861.09368408,101.24424123)
\curveto(861.20369122,101.25423365)(861.31369111,101.26923363)(861.42368408,101.28924123)
\curveto(861.47369095,101.30923359)(861.5186909,101.31423359)(861.55868408,101.30424123)
\curveto(861.59869082,101.3042336)(861.63869078,101.30923359)(861.67868408,101.31924123)
\curveto(861.72869069,101.32923357)(861.78369064,101.32923357)(861.84368408,101.31924123)
\curveto(861.89369053,101.31923358)(861.94369048,101.32423358)(861.99368408,101.33424123)
\lineto(862.12868408,101.33424123)
\curveto(862.18869023,101.35423355)(862.25869016,101.35423355)(862.33868408,101.33424123)
\curveto(862.40869001,101.32423358)(862.47368995,101.32923357)(862.53368408,101.34924123)
\curveto(862.56368986,101.35923354)(862.60368982,101.36423354)(862.65368408,101.36424123)
\lineto(862.77368408,101.36424123)
\lineto(863.23868408,101.36424123)
\moveto(865.56368408,99.81924123)
\curveto(865.24368718,99.91923498)(864.87868754,99.97923492)(864.46868408,99.99924123)
\curveto(864.05868836,100.01923488)(863.64868877,100.02923487)(863.23868408,100.02924123)
\curveto(862.80868961,100.02923487)(862.38869003,100.01923488)(861.97868408,99.99924123)
\curveto(861.56869085,99.97923492)(861.18369124,99.93423497)(860.82368408,99.86424123)
\curveto(860.46369196,99.79423511)(860.14369228,99.68423522)(859.86368408,99.53424123)
\curveto(859.57369285,99.39423551)(859.33869308,99.1992357)(859.15868408,98.94924123)
\curveto(859.04869337,98.78923611)(858.96869345,98.60923629)(858.91868408,98.40924123)
\curveto(858.85869356,98.20923669)(858.82869359,97.96423694)(858.82868408,97.67424123)
\curveto(858.84869357,97.65423725)(858.85869356,97.61923728)(858.85868408,97.56924123)
\curveto(858.84869357,97.51923738)(858.84869357,97.47923742)(858.85868408,97.44924123)
\curveto(858.87869354,97.36923753)(858.89869352,97.29423761)(858.91868408,97.22424123)
\curveto(858.92869349,97.16423774)(858.94869347,97.0992378)(858.97868408,97.02924123)
\curveto(859.09869332,96.75923814)(859.26869315,96.53923836)(859.48868408,96.36924123)
\curveto(859.69869272,96.20923869)(859.94369248,96.07423883)(860.22368408,95.96424123)
\curveto(860.33369209,95.91423899)(860.45369197,95.87423903)(860.58368408,95.84424123)
\curveto(860.70369172,95.82423908)(860.82869159,95.7992391)(860.95868408,95.76924123)
\curveto(861.00869141,95.74923915)(861.06369136,95.73923916)(861.12368408,95.73924123)
\curveto(861.17369125,95.73923916)(861.2236912,95.73423917)(861.27368408,95.72424123)
\curveto(861.36369106,95.71423919)(861.45869096,95.7042392)(861.55868408,95.69424123)
\curveto(861.64869077,95.68423922)(861.74369068,95.67423923)(861.84368408,95.66424123)
\curveto(861.9236905,95.66423924)(862.00869041,95.65923924)(862.09868408,95.64924123)
\lineto(862.33868408,95.64924123)
\lineto(862.51868408,95.64924123)
\curveto(862.54868987,95.63923926)(862.58368984,95.63423927)(862.62368408,95.63424123)
\lineto(862.75868408,95.63424123)
\lineto(863.20868408,95.63424123)
\curveto(863.28868913,95.63423927)(863.37368905,95.62923927)(863.46368408,95.61924123)
\curveto(863.54368888,95.61923928)(863.6186888,95.62923927)(863.68868408,95.64924123)
\lineto(863.95868408,95.64924123)
\curveto(863.97868844,95.64923925)(864.00868841,95.64423926)(864.04868408,95.63424123)
\curveto(864.07868834,95.63423927)(864.10368832,95.63923926)(864.12368408,95.64924123)
\curveto(864.2236882,95.65923924)(864.3236881,95.66423924)(864.42368408,95.66424123)
\curveto(864.51368791,95.67423923)(864.61368781,95.68423922)(864.72368408,95.69424123)
\curveto(864.84368758,95.72423918)(864.96868745,95.73923916)(865.09868408,95.73924123)
\curveto(865.2186872,95.74923915)(865.33368709,95.77423913)(865.44368408,95.81424123)
\curveto(865.74368668,95.89423901)(866.00868641,95.97923892)(866.23868408,96.06924123)
\curveto(866.46868595,96.16923873)(866.68368574,96.31423859)(866.88368408,96.50424123)
\curveto(867.08368534,96.71423819)(867.23368519,96.97923792)(867.33368408,97.29924123)
\curveto(867.35368507,97.33923756)(867.36368506,97.37423753)(867.36368408,97.40424123)
\curveto(867.35368507,97.44423746)(867.35868506,97.48923741)(867.37868408,97.53924123)
\curveto(867.38868503,97.57923732)(867.39868502,97.64923725)(867.40868408,97.74924123)
\curveto(867.418685,97.85923704)(867.41368501,97.94423696)(867.39368408,98.00424123)
\curveto(867.37368505,98.07423683)(867.36368506,98.14423676)(867.36368408,98.21424123)
\curveto(867.35368507,98.28423662)(867.33868508,98.34923655)(867.31868408,98.40924123)
\curveto(867.25868516,98.60923629)(867.17368525,98.78923611)(867.06368408,98.94924123)
\curveto(867.04368538,98.97923592)(867.0236854,99.0042359)(867.00368408,99.02424123)
\lineto(866.94368408,99.08424123)
\curveto(866.9236855,99.12423578)(866.88368554,99.17423573)(866.82368408,99.23424123)
\curveto(866.68368574,99.33423557)(866.55368587,99.41923548)(866.43368408,99.48924123)
\curveto(866.31368611,99.55923534)(866.16868625,99.62923527)(865.99868408,99.69924123)
\curveto(865.92868649,99.72923517)(865.85868656,99.74923515)(865.78868408,99.75924123)
\curveto(865.7186867,99.77923512)(865.64368678,99.7992351)(865.56368408,99.81924123)
}
}
{
\newrgbcolor{curcolor}{0 0 0}
\pscustom[linestyle=none,fillstyle=solid,fillcolor=curcolor]
{
\newpath
\moveto(857.71868408,106.77385061)
\curveto(857.7186947,106.87384575)(857.72869469,106.96884566)(857.74868408,107.05885061)
\curveto(857.75869466,107.14884548)(857.78869463,107.21384541)(857.83868408,107.25385061)
\curveto(857.9186945,107.31384531)(858.0236944,107.34384528)(858.15368408,107.34385061)
\lineto(858.54368408,107.34385061)
\lineto(860.04368408,107.34385061)
\lineto(866.43368408,107.34385061)
\lineto(867.60368408,107.34385061)
\lineto(867.91868408,107.34385061)
\curveto(868.0186844,107.35384527)(868.09868432,107.33884529)(868.15868408,107.29885061)
\curveto(868.23868418,107.24884538)(868.28868413,107.17384545)(868.30868408,107.07385061)
\curveto(868.3186841,106.98384564)(868.3236841,106.87384575)(868.32368408,106.74385061)
\lineto(868.32368408,106.51885061)
\curveto(868.30368412,106.43884619)(868.28868413,106.36884626)(868.27868408,106.30885061)
\curveto(868.25868416,106.24884638)(868.2186842,106.19884643)(868.15868408,106.15885061)
\curveto(868.09868432,106.11884651)(868.0236844,106.09884653)(867.93368408,106.09885061)
\lineto(867.63368408,106.09885061)
\lineto(866.53868408,106.09885061)
\lineto(861.19868408,106.09885061)
\curveto(861.10869131,106.07884655)(861.03369139,106.06384656)(860.97368408,106.05385061)
\curveto(860.90369152,106.05384657)(860.84369158,106.0238466)(860.79368408,105.96385061)
\curveto(860.74369168,105.89384673)(860.7186917,105.80384682)(860.71868408,105.69385061)
\curveto(860.70869171,105.59384703)(860.70369172,105.48384714)(860.70368408,105.36385061)
\lineto(860.70368408,104.22385061)
\lineto(860.70368408,103.72885061)
\curveto(860.69369173,103.56884906)(860.63369179,103.45884917)(860.52368408,103.39885061)
\curveto(860.49369193,103.37884925)(860.46369196,103.36884926)(860.43368408,103.36885061)
\curveto(860.39369203,103.36884926)(860.34869207,103.36384926)(860.29868408,103.35385061)
\curveto(860.17869224,103.33384929)(860.06869235,103.33884929)(859.96868408,103.36885061)
\curveto(859.86869255,103.40884922)(859.79869262,103.46384916)(859.75868408,103.53385061)
\curveto(859.70869271,103.61384901)(859.68369274,103.73384889)(859.68368408,103.89385061)
\curveto(859.68369274,104.05384857)(859.66869275,104.18884844)(859.63868408,104.29885061)
\curveto(859.62869279,104.34884828)(859.6236928,104.40384822)(859.62368408,104.46385061)
\curveto(859.61369281,104.5238481)(859.59869282,104.58384804)(859.57868408,104.64385061)
\curveto(859.52869289,104.79384783)(859.47869294,104.93884769)(859.42868408,105.07885061)
\curveto(859.36869305,105.21884741)(859.29869312,105.35384727)(859.21868408,105.48385061)
\curveto(859.12869329,105.623847)(859.0236934,105.74384688)(858.90368408,105.84385061)
\curveto(858.78369364,105.94384668)(858.65369377,106.03884659)(858.51368408,106.12885061)
\curveto(858.41369401,106.18884644)(858.30369412,106.23384639)(858.18368408,106.26385061)
\curveto(858.06369436,106.30384632)(857.95869446,106.35384627)(857.86868408,106.41385061)
\curveto(857.80869461,106.46384616)(857.76869465,106.53384609)(857.74868408,106.62385061)
\curveto(857.73869468,106.64384598)(857.73369469,106.66884596)(857.73368408,106.69885061)
\curveto(857.73369469,106.7288459)(857.72869469,106.75384587)(857.71868408,106.77385061)
}
}
{
\newrgbcolor{curcolor}{0 0 0}
\pscustom[linestyle=none,fillstyle=solid,fillcolor=curcolor]
{
\newpath
\moveto(857.71868408,115.12345998)
\curveto(857.7186947,115.22345513)(857.72869469,115.31845503)(857.74868408,115.40845998)
\curveto(857.75869466,115.49845485)(857.78869463,115.56345479)(857.83868408,115.60345998)
\curveto(857.9186945,115.66345469)(858.0236944,115.69345466)(858.15368408,115.69345998)
\lineto(858.54368408,115.69345998)
\lineto(860.04368408,115.69345998)
\lineto(866.43368408,115.69345998)
\lineto(867.60368408,115.69345998)
\lineto(867.91868408,115.69345998)
\curveto(868.0186844,115.70345465)(868.09868432,115.68845466)(868.15868408,115.64845998)
\curveto(868.23868418,115.59845475)(868.28868413,115.52345483)(868.30868408,115.42345998)
\curveto(868.3186841,115.33345502)(868.3236841,115.22345513)(868.32368408,115.09345998)
\lineto(868.32368408,114.86845998)
\curveto(868.30368412,114.78845556)(868.28868413,114.71845563)(868.27868408,114.65845998)
\curveto(868.25868416,114.59845575)(868.2186842,114.5484558)(868.15868408,114.50845998)
\curveto(868.09868432,114.46845588)(868.0236844,114.4484559)(867.93368408,114.44845998)
\lineto(867.63368408,114.44845998)
\lineto(866.53868408,114.44845998)
\lineto(861.19868408,114.44845998)
\curveto(861.10869131,114.42845592)(861.03369139,114.41345594)(860.97368408,114.40345998)
\curveto(860.90369152,114.40345595)(860.84369158,114.37345598)(860.79368408,114.31345998)
\curveto(860.74369168,114.24345611)(860.7186917,114.1534562)(860.71868408,114.04345998)
\curveto(860.70869171,113.94345641)(860.70369172,113.83345652)(860.70368408,113.71345998)
\lineto(860.70368408,112.57345998)
\lineto(860.70368408,112.07845998)
\curveto(860.69369173,111.91845843)(860.63369179,111.80845854)(860.52368408,111.74845998)
\curveto(860.49369193,111.72845862)(860.46369196,111.71845863)(860.43368408,111.71845998)
\curveto(860.39369203,111.71845863)(860.34869207,111.71345864)(860.29868408,111.70345998)
\curveto(860.17869224,111.68345867)(860.06869235,111.68845866)(859.96868408,111.71845998)
\curveto(859.86869255,111.75845859)(859.79869262,111.81345854)(859.75868408,111.88345998)
\curveto(859.70869271,111.96345839)(859.68369274,112.08345827)(859.68368408,112.24345998)
\curveto(859.68369274,112.40345795)(859.66869275,112.53845781)(859.63868408,112.64845998)
\curveto(859.62869279,112.69845765)(859.6236928,112.7534576)(859.62368408,112.81345998)
\curveto(859.61369281,112.87345748)(859.59869282,112.93345742)(859.57868408,112.99345998)
\curveto(859.52869289,113.14345721)(859.47869294,113.28845706)(859.42868408,113.42845998)
\curveto(859.36869305,113.56845678)(859.29869312,113.70345665)(859.21868408,113.83345998)
\curveto(859.12869329,113.97345638)(859.0236934,114.09345626)(858.90368408,114.19345998)
\curveto(858.78369364,114.29345606)(858.65369377,114.38845596)(858.51368408,114.47845998)
\curveto(858.41369401,114.53845581)(858.30369412,114.58345577)(858.18368408,114.61345998)
\curveto(858.06369436,114.6534557)(857.95869446,114.70345565)(857.86868408,114.76345998)
\curveto(857.80869461,114.81345554)(857.76869465,114.88345547)(857.74868408,114.97345998)
\curveto(857.73869468,114.99345536)(857.73369469,115.01845533)(857.73368408,115.04845998)
\curveto(857.73369469,115.07845527)(857.72869469,115.10345525)(857.71868408,115.12345998)
}
}
{
\newrgbcolor{curcolor}{0 0 0}
\pscustom[linestyle=none,fillstyle=solid,fillcolor=curcolor]
{
\newpath
\moveto(878.555,42.29681936)
\curveto(878.55501069,42.36681368)(878.55501069,42.4468136)(878.555,42.53681936)
\curveto(878.5450107,42.62681342)(878.5450107,42.71181333)(878.555,42.79181936)
\curveto(878.55501069,42.88181316)(878.56501068,42.96181308)(878.585,43.03181936)
\curveto(878.60501064,43.11181293)(878.63501062,43.16681288)(878.675,43.19681936)
\curveto(878.72501053,43.22681282)(878.80001045,43.2468128)(878.9,43.25681936)
\curveto(878.99001026,43.27681277)(879.09501015,43.28681276)(879.215,43.28681936)
\curveto(879.32500993,43.29681275)(879.44000981,43.29681275)(879.56,43.28681936)
\lineto(879.86,43.28681936)
\lineto(882.875,43.28681936)
\lineto(885.77,43.28681936)
\curveto(886.10000315,43.28681276)(886.42500282,43.28181276)(886.745,43.27181936)
\curveto(887.0550022,43.27181277)(887.33500192,43.23181281)(887.585,43.15181936)
\curveto(887.93500132,43.03181301)(888.23000102,42.87681317)(888.47,42.68681936)
\curveto(888.70000055,42.49681355)(888.90000035,42.25681379)(889.07,41.96681936)
\curveto(889.12000013,41.90681414)(889.15500009,41.8418142)(889.175,41.77181936)
\curveto(889.19500006,41.71181433)(889.22000003,41.6418144)(889.25,41.56181936)
\curveto(889.29999995,41.4418146)(889.33499992,41.31181473)(889.355,41.17181936)
\curveto(889.38499987,41.041815)(889.41499984,40.90681514)(889.445,40.76681936)
\curveto(889.46499979,40.71681533)(889.46999978,40.66681538)(889.46,40.61681936)
\curveto(889.4499998,40.56681548)(889.4499998,40.51181553)(889.46,40.45181936)
\curveto(889.46999978,40.43181561)(889.46999978,40.40681564)(889.46,40.37681936)
\curveto(889.45999979,40.3468157)(889.46499979,40.32181572)(889.475,40.30181936)
\curveto(889.48499976,40.26181578)(889.48999976,40.20681584)(889.49,40.13681936)
\curveto(889.48999976,40.06681598)(889.48499976,40.01181603)(889.475,39.97181936)
\curveto(889.46499979,39.92181612)(889.46499979,39.86681618)(889.475,39.80681936)
\curveto(889.48499976,39.7468163)(889.47999977,39.69181635)(889.46,39.64181936)
\curveto(889.42999982,39.51181653)(889.40999984,39.38681666)(889.4,39.26681936)
\curveto(889.38999986,39.1468169)(889.36499988,39.03181701)(889.325,38.92181936)
\curveto(889.20500005,38.55181749)(889.03500021,38.23181781)(888.815,37.96181936)
\curveto(888.59500066,37.69181835)(888.31500093,37.48181856)(887.975,37.33181936)
\curveto(887.8550014,37.28181876)(887.73000152,37.23681881)(887.6,37.19681936)
\curveto(887.47000178,37.16681888)(887.33500192,37.13181891)(887.195,37.09181936)
\curveto(887.14500211,37.08181896)(887.10500214,37.07681897)(887.075,37.07681936)
\curveto(887.03500221,37.07681897)(886.99000226,37.07181897)(886.94,37.06181936)
\curveto(886.91000234,37.05181899)(886.87500238,37.046819)(886.835,37.04681936)
\curveto(886.78500247,37.046819)(886.74500251,37.041819)(886.715,37.03181936)
\lineto(886.55,37.03181936)
\curveto(886.47000278,37.01181903)(886.37000288,37.00681904)(886.25,37.01681936)
\curveto(886.12000313,37.02681902)(886.03000322,37.041819)(885.98,37.06181936)
\curveto(885.89000336,37.08181896)(885.82500342,37.13681891)(885.785,37.22681936)
\curveto(885.76500348,37.25681879)(885.76000349,37.28681876)(885.77,37.31681936)
\curveto(885.77000348,37.3468187)(885.76500348,37.38681866)(885.755,37.43681936)
\curveto(885.7450035,37.47681857)(885.74000351,37.51681853)(885.74,37.55681936)
\lineto(885.74,37.70681936)
\curveto(885.74000351,37.82681822)(885.7450035,37.9468181)(885.755,38.06681936)
\curveto(885.75500349,38.19681785)(885.79000346,38.28681776)(885.86,38.33681936)
\curveto(885.92000333,38.37681767)(885.98000327,38.39681765)(886.04,38.39681936)
\curveto(886.10000315,38.39681765)(886.17000308,38.40681764)(886.25,38.42681936)
\curveto(886.28000297,38.43681761)(886.31500294,38.43681761)(886.355,38.42681936)
\curveto(886.38500287,38.42681762)(886.41000284,38.43181761)(886.43,38.44181936)
\lineto(886.64,38.44181936)
\curveto(886.69000256,38.46181758)(886.74000251,38.46681758)(886.79,38.45681936)
\curveto(886.83000242,38.45681759)(886.87500238,38.46681758)(886.925,38.48681936)
\curveto(887.0550022,38.51681753)(887.18000207,38.5468175)(887.3,38.57681936)
\curveto(887.41000184,38.60681744)(887.51500174,38.65181739)(887.615,38.71181936)
\curveto(887.90500134,38.88181716)(888.11000114,39.15181689)(888.23,39.52181936)
\curveto(888.250001,39.57181647)(888.26500099,39.62181642)(888.275,39.67181936)
\curveto(888.27500098,39.73181631)(888.28500096,39.78681626)(888.305,39.83681936)
\lineto(888.305,39.91181936)
\curveto(888.31500093,39.98181606)(888.32500093,40.07681597)(888.335,40.19681936)
\curveto(888.33500092,40.32681572)(888.32500093,40.42681562)(888.305,40.49681936)
\curveto(888.28500096,40.56681548)(888.27000098,40.63681541)(888.26,40.70681936)
\curveto(888.24000101,40.78681526)(888.22000103,40.85681519)(888.2,40.91681936)
\curveto(888.04000121,41.29681475)(887.76500148,41.57181447)(887.375,41.74181936)
\curveto(887.245002,41.79181425)(887.09000216,41.82681422)(886.91,41.84681936)
\curveto(886.73000252,41.87681417)(886.54500271,41.89181415)(886.355,41.89181936)
\curveto(886.15500309,41.90181414)(885.95500329,41.90181414)(885.755,41.89181936)
\lineto(885.185,41.89181936)
\lineto(880.94,41.89181936)
\lineto(879.395,41.89181936)
\curveto(879.28500996,41.89181415)(879.16501008,41.88681416)(879.035,41.87681936)
\curveto(878.90501035,41.86681418)(878.80001045,41.88681416)(878.72,41.93681936)
\curveto(878.6500106,41.99681405)(878.60001065,42.07681397)(878.57,42.17681936)
\curveto(878.57001068,42.19681385)(878.57001068,42.21681383)(878.57,42.23681936)
\curveto(878.57001068,42.25681379)(878.56501068,42.27681377)(878.555,42.29681936)
}
}
{
\newrgbcolor{curcolor}{0 0 0}
\pscustom[linestyle=none,fillstyle=solid,fillcolor=curcolor]
{
\newpath
\moveto(881.51,45.83049123)
\lineto(881.51,46.26549123)
\curveto(881.51000774,46.41548927)(881.5500077,46.52048916)(881.63,46.58049123)
\curveto(881.71000754,46.63048905)(881.81000744,46.65548903)(881.93,46.65549123)
\curveto(882.0500072,46.66548902)(882.17000708,46.67048901)(882.29,46.67049123)
\lineto(883.715,46.67049123)
\lineto(885.98,46.67049123)
\lineto(886.67,46.67049123)
\curveto(886.90000235,46.67048901)(887.10000215,46.69548899)(887.27,46.74549123)
\curveto(887.72000153,46.90548878)(888.03500121,47.20548848)(888.215,47.64549123)
\curveto(888.30500094,47.86548782)(888.34000091,48.13048755)(888.32,48.44049123)
\curveto(888.29000096,48.75048693)(888.23500101,49.00048668)(888.155,49.19049123)
\curveto(888.01500123,49.52048616)(887.84000141,49.7804859)(887.63,49.97049123)
\curveto(887.41000184,50.17048551)(887.12500213,50.32548536)(886.775,50.43549123)
\curveto(886.69500255,50.46548522)(886.61500263,50.4854852)(886.535,50.49549123)
\curveto(886.4550028,50.50548518)(886.37000288,50.52048516)(886.28,50.54049123)
\curveto(886.23000302,50.55048513)(886.18500307,50.55048513)(886.145,50.54049123)
\curveto(886.10500314,50.54048514)(886.06000319,50.55048513)(886.01,50.57049123)
\lineto(885.695,50.57049123)
\curveto(885.61500363,50.59048509)(885.52500373,50.59548509)(885.425,50.58549123)
\curveto(885.31500394,50.57548511)(885.21500403,50.57048511)(885.125,50.57049123)
\lineto(883.955,50.57049123)
\lineto(882.365,50.57049123)
\curveto(882.24500701,50.57048511)(882.12000713,50.56548512)(881.99,50.55549123)
\curveto(881.8500074,50.55548513)(881.74000751,50.5804851)(881.66,50.63049123)
\curveto(881.61000764,50.67048501)(881.58000767,50.71548497)(881.57,50.76549123)
\curveto(881.5500077,50.82548486)(881.53000772,50.89548479)(881.51,50.97549123)
\lineto(881.51,51.20049123)
\curveto(881.51000774,51.32048436)(881.51500774,51.42548426)(881.525,51.51549123)
\curveto(881.53500772,51.61548407)(881.58000767,51.69048399)(881.66,51.74049123)
\curveto(881.71000754,51.79048389)(881.78500747,51.81548387)(881.885,51.81549123)
\lineto(882.17,51.81549123)
\lineto(883.19,51.81549123)
\lineto(887.225,51.81549123)
\lineto(888.575,51.81549123)
\curveto(888.69500055,51.81548387)(888.81000044,51.81048387)(888.92,51.80049123)
\curveto(889.02000023,51.80048388)(889.09500015,51.76548392)(889.145,51.69549123)
\curveto(889.17500007,51.65548403)(889.20000005,51.59548409)(889.22,51.51549123)
\curveto(889.23000002,51.43548425)(889.24000001,51.34548434)(889.25,51.24549123)
\curveto(889.25,51.15548453)(889.245,51.06548462)(889.235,50.97549123)
\curveto(889.22500002,50.89548479)(889.20500005,50.83548485)(889.175,50.79549123)
\curveto(889.13500012,50.74548494)(889.07000018,50.70048498)(888.98,50.66049123)
\curveto(888.94000031,50.65048503)(888.88500036,50.64048504)(888.815,50.63049123)
\curveto(888.74500051,50.63048505)(888.68000057,50.62548506)(888.62,50.61549123)
\curveto(888.5500007,50.60548508)(888.49500075,50.5854851)(888.455,50.55549123)
\curveto(888.41500084,50.52548516)(888.40000085,50.4804852)(888.41,50.42049123)
\curveto(888.43000082,50.34048534)(888.49000076,50.26048542)(888.59,50.18049123)
\curveto(888.68000057,50.10048558)(888.7500005,50.02548566)(888.8,49.95549123)
\curveto(888.96000029,49.73548595)(889.10000015,49.4854862)(889.22,49.20549123)
\curveto(889.26999998,49.09548659)(889.29999995,48.9804867)(889.31,48.86049123)
\curveto(889.32999992,48.75048693)(889.35499989,48.63548705)(889.385,48.51549123)
\curveto(889.39499986,48.46548722)(889.39499986,48.41048727)(889.385,48.35049123)
\curveto(889.37499987,48.30048738)(889.37999987,48.25048743)(889.4,48.20049123)
\curveto(889.41999983,48.10048758)(889.41999983,48.01048767)(889.4,47.93049123)
\lineto(889.4,47.78049123)
\curveto(889.37999987,47.73048795)(889.36999988,47.67048801)(889.37,47.60049123)
\curveto(889.36999988,47.54048814)(889.36499988,47.4854882)(889.355,47.43549123)
\curveto(889.33499992,47.39548829)(889.32499993,47.35548833)(889.325,47.31549123)
\curveto(889.33499992,47.2854884)(889.32999992,47.24548844)(889.31,47.19549123)
\lineto(889.25,46.95549123)
\curveto(889.23000002,46.8854888)(889.20000005,46.81048887)(889.16,46.73049123)
\curveto(889.0500002,46.47048921)(888.90500034,46.25048943)(888.725,46.07049123)
\curveto(888.53500072,45.90048978)(888.31000094,45.76048992)(888.05,45.65049123)
\curveto(887.96000129,45.61049007)(887.87000138,45.5804901)(887.78,45.56049123)
\lineto(887.48,45.50049123)
\curveto(887.42000183,45.4804902)(887.36500188,45.47049021)(887.315,45.47049123)
\curveto(887.255002,45.4804902)(887.19000206,45.47549021)(887.12,45.45549123)
\curveto(887.10000215,45.44549024)(887.07500218,45.44049024)(887.045,45.44049123)
\curveto(887.00500225,45.44049024)(886.97000228,45.43549025)(886.94,45.42549123)
\lineto(886.79,45.42549123)
\curveto(886.7500025,45.41549027)(886.70500254,45.41049027)(886.655,45.41049123)
\curveto(886.59500266,45.42049026)(886.54000271,45.42549026)(886.49,45.42549123)
\lineto(885.89,45.42549123)
\lineto(883.13,45.42549123)
\lineto(882.17,45.42549123)
\lineto(881.9,45.42549123)
\curveto(881.81000744,45.42549026)(881.73500752,45.44549024)(881.675,45.48549123)
\curveto(881.60500765,45.52549016)(881.55500769,45.60049008)(881.525,45.71049123)
\curveto(881.51500774,45.73048995)(881.51500774,45.75048993)(881.525,45.77049123)
\curveto(881.52500773,45.79048989)(881.52000773,45.81048987)(881.51,45.83049123)
}
}
{
\newrgbcolor{curcolor}{0 0 0}
\pscustom[linestyle=none,fillstyle=solid,fillcolor=curcolor]
{
\newpath
\moveto(878.555,54.28510061)
\curveto(878.55501069,54.41509899)(878.55501069,54.55009886)(878.555,54.69010061)
\curveto(878.55501069,54.84009857)(878.59001066,54.95009846)(878.66,55.02010061)
\curveto(878.73001052,55.07009834)(878.82501042,55.09509831)(878.945,55.09510061)
\curveto(879.0550102,55.1050983)(879.17001008,55.1100983)(879.29,55.11010061)
\lineto(880.625,55.11010061)
\lineto(886.7,55.11010061)
\lineto(888.38,55.11010061)
\lineto(888.77,55.11010061)
\curveto(888.91000034,55.1100983)(889.02000023,55.08509832)(889.1,55.03510061)
\curveto(889.1500001,55.0050984)(889.18000007,54.96009845)(889.19,54.90010061)
\curveto(889.20000005,54.85009856)(889.21500004,54.78509862)(889.235,54.70510061)
\lineto(889.235,54.49510061)
\lineto(889.235,54.18010061)
\curveto(889.22500002,54.08009933)(889.19000006,54.0050994)(889.13,53.95510061)
\curveto(889.0500002,53.9050995)(888.9500003,53.87509953)(888.83,53.86510061)
\lineto(888.455,53.86510061)
\lineto(887.075,53.86510061)
\lineto(880.835,53.86510061)
\lineto(879.365,53.86510061)
\curveto(879.25501,53.86509954)(879.14001011,53.86009955)(879.02,53.85010061)
\curveto(878.89001036,53.85009956)(878.79001046,53.87509953)(878.72,53.92510061)
\curveto(878.66001059,53.96509944)(878.61001064,54.04009937)(878.57,54.15010061)
\curveto(878.56001069,54.17009924)(878.56001069,54.19009922)(878.57,54.21010061)
\curveto(878.57001068,54.24009917)(878.56501068,54.26509914)(878.555,54.28510061)
}
}
{
\newrgbcolor{curcolor}{0 0 0}
\pscustom[linestyle=none,fillstyle=solid,fillcolor=curcolor]
{
}
}
{
\newrgbcolor{curcolor}{0 0 0}
\pscustom[linestyle=none,fillstyle=solid,fillcolor=curcolor]
{
\newpath
\moveto(878.63,65.05510061)
\curveto(878.63001062,65.15509575)(878.64001061,65.25009566)(878.66,65.34010061)
\curveto(878.67001058,65.43009548)(878.70001055,65.49509541)(878.75,65.53510061)
\curveto(878.83001042,65.59509531)(878.93501031,65.62509528)(879.065,65.62510061)
\lineto(879.455,65.62510061)
\lineto(880.955,65.62510061)
\lineto(887.345,65.62510061)
\lineto(888.515,65.62510061)
\lineto(888.83,65.62510061)
\curveto(888.93000032,65.63509527)(889.01000024,65.62009529)(889.07,65.58010061)
\curveto(889.1500001,65.53009538)(889.20000005,65.45509545)(889.22,65.35510061)
\curveto(889.23000002,65.26509564)(889.23500001,65.15509575)(889.235,65.02510061)
\lineto(889.235,64.80010061)
\curveto(889.21500004,64.72009619)(889.20000005,64.65009626)(889.19,64.59010061)
\curveto(889.17000008,64.53009638)(889.13000012,64.48009643)(889.07,64.44010061)
\curveto(889.01000024,64.40009651)(888.93500032,64.38009653)(888.845,64.38010061)
\lineto(888.545,64.38010061)
\lineto(887.45,64.38010061)
\lineto(882.11,64.38010061)
\curveto(882.02000723,64.36009655)(881.9450073,64.34509656)(881.885,64.33510061)
\curveto(881.81500743,64.33509657)(881.75500749,64.3050966)(881.705,64.24510061)
\curveto(881.6550076,64.17509673)(881.63000762,64.08509682)(881.63,63.97510061)
\curveto(881.62000763,63.87509703)(881.61500763,63.76509714)(881.615,63.64510061)
\lineto(881.615,62.50510061)
\lineto(881.615,62.01010061)
\curveto(881.60500765,61.85009906)(881.5450077,61.74009917)(881.435,61.68010061)
\curveto(881.40500785,61.66009925)(881.37500788,61.65009926)(881.345,61.65010061)
\curveto(881.30500794,61.65009926)(881.26000799,61.64509926)(881.21,61.63510061)
\curveto(881.09000816,61.61509929)(880.98000827,61.62009929)(880.88,61.65010061)
\curveto(880.78000847,61.69009922)(880.71000854,61.74509916)(880.67,61.81510061)
\curveto(880.62000863,61.89509901)(880.59500866,62.01509889)(880.595,62.17510061)
\curveto(880.59500866,62.33509857)(880.58000867,62.47009844)(880.55,62.58010061)
\curveto(880.54000871,62.63009828)(880.53500872,62.68509822)(880.535,62.74510061)
\curveto(880.52500873,62.8050981)(880.51000874,62.86509804)(880.49,62.92510061)
\curveto(880.44000881,63.07509783)(880.39000886,63.22009769)(880.34,63.36010061)
\curveto(880.28000897,63.50009741)(880.21000904,63.63509727)(880.13,63.76510061)
\curveto(880.04000921,63.905097)(879.93500932,64.02509688)(879.815,64.12510061)
\curveto(879.69500955,64.22509668)(879.56500968,64.32009659)(879.425,64.41010061)
\curveto(879.32500993,64.47009644)(879.21501003,64.51509639)(879.095,64.54510061)
\curveto(878.97501028,64.58509632)(878.87001038,64.63509627)(878.78,64.69510061)
\curveto(878.72001053,64.74509616)(878.68001057,64.81509609)(878.66,64.90510061)
\curveto(878.6500106,64.92509598)(878.64501061,64.95009596)(878.645,64.98010061)
\curveto(878.64501061,65.0100959)(878.64001061,65.03509587)(878.63,65.05510061)
}
}
{
\newrgbcolor{curcolor}{0 0 0}
\pscustom[linestyle=none,fillstyle=solid,fillcolor=curcolor]
{
\newpath
\moveto(884.915,76.34470998)
\curveto(885.03500421,76.37470226)(885.17500407,76.39970223)(885.335,76.41970998)
\curveto(885.49500375,76.43970219)(885.66000359,76.44970218)(885.83,76.44970998)
\curveto(886.00000325,76.44970218)(886.16500308,76.43970219)(886.325,76.41970998)
\curveto(886.48500276,76.39970223)(886.62500262,76.37470226)(886.745,76.34470998)
\curveto(886.88500236,76.30470233)(887.01000224,76.26970236)(887.12,76.23970998)
\curveto(887.23000202,76.20970242)(887.34000191,76.16970246)(887.45,76.11970998)
\curveto(888.09000116,75.84970278)(888.57500067,75.4347032)(888.905,74.87470998)
\curveto(888.96500028,74.79470384)(889.01500024,74.70970392)(889.055,74.61970998)
\curveto(889.08500016,74.5297041)(889.12000013,74.4297042)(889.16,74.31970998)
\curveto(889.21000004,74.20970442)(889.245,74.08970454)(889.265,73.95970998)
\curveto(889.29499995,73.83970479)(889.32499993,73.70970492)(889.355,73.56970998)
\curveto(889.37499987,73.50970512)(889.37999987,73.44970518)(889.37,73.38970998)
\curveto(889.35999989,73.33970529)(889.36499988,73.27970535)(889.385,73.20970998)
\curveto(889.39499986,73.18970544)(889.39499986,73.16470547)(889.385,73.13470998)
\curveto(889.38499987,73.10470553)(889.38999986,73.07970555)(889.4,73.05970998)
\lineto(889.4,72.90970998)
\curveto(889.40999984,72.83970579)(889.40999984,72.78970584)(889.4,72.75970998)
\curveto(889.38999986,72.71970591)(889.38499987,72.67470596)(889.385,72.62470998)
\curveto(889.39499986,72.58470605)(889.39499986,72.54470609)(889.385,72.50470998)
\curveto(889.36499988,72.41470622)(889.3499999,72.32470631)(889.34,72.23470998)
\curveto(889.33999991,72.14470649)(889.32999992,72.05470658)(889.31,71.96470998)
\curveto(889.27999997,71.87470676)(889.255,71.78470685)(889.235,71.69470998)
\curveto(889.21500004,71.60470703)(889.18500007,71.51970711)(889.145,71.43970998)
\curveto(889.03500021,71.19970743)(888.90500034,70.97470766)(888.755,70.76470998)
\curveto(888.59500066,70.55470808)(888.41500084,70.37470826)(888.215,70.22470998)
\curveto(888.0450012,70.10470853)(887.87000138,69.99970863)(887.69,69.90970998)
\curveto(887.51000174,69.81970881)(887.32000193,69.7297089)(887.12,69.63970998)
\curveto(887.02000223,69.59970903)(886.92000233,69.56470907)(886.82,69.53470998)
\curveto(886.71000254,69.51470912)(886.60000265,69.48970914)(886.49,69.45970998)
\curveto(886.3500029,69.41970921)(886.21000304,69.39470924)(886.07,69.38470998)
\curveto(885.93000332,69.37470926)(885.79000346,69.35470928)(885.65,69.32470998)
\curveto(885.54000371,69.31470932)(885.44000381,69.30470933)(885.35,69.29470998)
\curveto(885.250004,69.29470934)(885.1500041,69.28470935)(885.05,69.26470998)
\lineto(884.96,69.26470998)
\curveto(884.93000432,69.27470936)(884.90500434,69.27470936)(884.885,69.26470998)
\lineto(884.675,69.26470998)
\curveto(884.61500463,69.24470939)(884.5500047,69.2347094)(884.48,69.23470998)
\curveto(884.40000485,69.24470939)(884.32500493,69.24970938)(884.255,69.24970998)
\lineto(884.105,69.24970998)
\curveto(884.0550052,69.24970938)(884.00500525,69.25470938)(883.955,69.26470998)
\lineto(883.58,69.26470998)
\curveto(883.5500057,69.27470936)(883.51500574,69.27470936)(883.475,69.26470998)
\curveto(883.43500581,69.26470937)(883.39500586,69.26970936)(883.355,69.27970998)
\curveto(883.24500601,69.29970933)(883.13500612,69.31470932)(883.025,69.32470998)
\curveto(882.90500634,69.3347093)(882.79000646,69.34470929)(882.68,69.35470998)
\curveto(882.53000672,69.39470924)(882.38500687,69.41970921)(882.245,69.42970998)
\curveto(882.09500715,69.44970918)(881.9500073,69.47970915)(881.81,69.51970998)
\curveto(881.51000774,69.60970902)(881.22500802,69.70470893)(880.955,69.80470998)
\curveto(880.68500856,69.90470873)(880.43500881,70.0297086)(880.205,70.17970998)
\curveto(879.88500936,70.37970825)(879.60500964,70.62470801)(879.365,70.91470998)
\curveto(879.12501013,71.20470743)(878.94001031,71.54470709)(878.81,71.93470998)
\curveto(878.77001048,72.04470659)(878.7450105,72.15470648)(878.735,72.26470998)
\curveto(878.71501054,72.38470625)(878.69001056,72.50470613)(878.66,72.62470998)
\curveto(878.6500106,72.69470594)(878.64501061,72.75970587)(878.645,72.81970998)
\curveto(878.64501061,72.87970575)(878.64001061,72.94470569)(878.63,73.01470998)
\curveto(878.61001064,73.71470492)(878.72501053,74.28970434)(878.975,74.73970998)
\curveto(879.22501002,75.18970344)(879.57500968,75.5347031)(880.025,75.77470998)
\curveto(880.255009,75.88470275)(880.53000872,75.98470265)(880.85,76.07470998)
\curveto(880.92000833,76.09470254)(880.99500826,76.09470254)(881.075,76.07470998)
\curveto(881.1450081,76.06470257)(881.19500806,76.03970259)(881.225,75.99970998)
\curveto(881.255008,75.96970266)(881.28000797,75.90970272)(881.3,75.81970998)
\curveto(881.31000794,75.7297029)(881.32000793,75.629703)(881.33,75.51970998)
\curveto(881.33000792,75.41970321)(881.32500793,75.31970331)(881.315,75.21970998)
\curveto(881.30500794,75.1297035)(881.28500796,75.06470357)(881.255,75.02470998)
\curveto(881.18500807,74.91470372)(881.07500817,74.8347038)(880.925,74.78470998)
\curveto(880.77500848,74.74470389)(880.64500861,74.68970394)(880.535,74.61970998)
\curveto(880.22500902,74.4297042)(879.99500926,74.14970448)(879.845,73.77970998)
\curveto(879.81500943,73.70970492)(879.79500946,73.634705)(879.785,73.55470998)
\curveto(879.77500948,73.48470515)(879.76000949,73.40970522)(879.74,73.32970998)
\curveto(879.73000952,73.27970535)(879.72500953,73.20970542)(879.725,73.11970998)
\curveto(879.72500953,73.03970559)(879.73000952,72.97470566)(879.74,72.92470998)
\curveto(879.76000949,72.88470575)(879.76500948,72.84970578)(879.755,72.81970998)
\curveto(879.7450095,72.78970584)(879.7450095,72.75470588)(879.755,72.71470998)
\lineto(879.815,72.47470998)
\curveto(879.83500941,72.40470623)(879.86000939,72.3347063)(879.89,72.26470998)
\curveto(880.0500092,71.88470675)(880.26000899,71.59470704)(880.52,71.39470998)
\curveto(880.78000847,71.20470743)(881.09500815,71.0297076)(881.465,70.86970998)
\curveto(881.5450077,70.83970779)(881.62500762,70.81470782)(881.705,70.79470998)
\curveto(881.78500747,70.78470785)(881.86500739,70.76470787)(881.945,70.73470998)
\curveto(882.0550072,70.70470793)(882.17000708,70.67970795)(882.29,70.65970998)
\curveto(882.41000684,70.64970798)(882.53000672,70.629708)(882.65,70.59970998)
\curveto(882.70000655,70.57970805)(882.7500065,70.56970806)(882.8,70.56970998)
\curveto(882.8500064,70.57970805)(882.90000635,70.57470806)(882.95,70.55470998)
\curveto(883.01000624,70.54470809)(883.09000616,70.54470809)(883.19,70.55470998)
\curveto(883.28000597,70.56470807)(883.33500592,70.57970805)(883.355,70.59970998)
\curveto(883.39500586,70.61970801)(883.41500583,70.64970798)(883.415,70.68970998)
\curveto(883.41500583,70.73970789)(883.40500585,70.78470785)(883.385,70.82470998)
\curveto(883.3450059,70.89470774)(883.30000595,70.95470768)(883.25,71.00470998)
\curveto(883.20000605,71.05470758)(883.1500061,71.11470752)(883.1,71.18470998)
\lineto(883.04,71.24470998)
\curveto(883.01000624,71.27470736)(882.98500627,71.30470733)(882.965,71.33470998)
\curveto(882.80500645,71.56470707)(882.67000658,71.83970679)(882.56,72.15970998)
\curveto(882.54000671,72.2297064)(882.52500673,72.29970633)(882.515,72.36970998)
\curveto(882.50500674,72.43970619)(882.49000676,72.51470612)(882.47,72.59470998)
\curveto(882.47000678,72.634706)(882.46500679,72.66970596)(882.455,72.69970998)
\curveto(882.44500681,72.7297059)(882.44500681,72.76470587)(882.455,72.80470998)
\curveto(882.4550068,72.85470578)(882.44500681,72.89470574)(882.425,72.92470998)
\lineto(882.425,73.08970998)
\lineto(882.425,73.17970998)
\curveto(882.41500683,73.2297054)(882.41500683,73.26970536)(882.425,73.29970998)
\curveto(882.43500681,73.34970528)(882.44000681,73.39970523)(882.44,73.44970998)
\curveto(882.43000682,73.50970512)(882.43000682,73.56470507)(882.44,73.61470998)
\curveto(882.47000678,73.72470491)(882.49000676,73.8297048)(882.5,73.92970998)
\curveto(882.51000674,74.03970459)(882.53500672,74.14470449)(882.575,74.24470998)
\curveto(882.71500654,74.66470397)(882.90000635,75.00970362)(883.13,75.27970998)
\curveto(883.3500059,75.54970308)(883.63500561,75.78970284)(883.985,75.99970998)
\curveto(884.12500513,76.07970255)(884.27500498,76.14470249)(884.435,76.19470998)
\curveto(884.58500467,76.24470239)(884.7450045,76.29470234)(884.915,76.34470998)
\moveto(886.22,75.09970998)
\curveto(886.17000308,75.10970352)(886.12500313,75.11470352)(886.085,75.11470998)
\lineto(885.935,75.11470998)
\curveto(885.62500362,75.11470352)(885.34000391,75.07470356)(885.08,74.99470998)
\curveto(885.02000423,74.97470366)(884.96500428,74.95470368)(884.915,74.93470998)
\curveto(884.8550044,74.92470371)(884.80000445,74.90970372)(884.75,74.88970998)
\curveto(884.26000499,74.66970396)(883.91000534,74.32470431)(883.7,73.85470998)
\curveto(883.67000558,73.77470486)(883.64500561,73.69470494)(883.625,73.61470998)
\lineto(883.565,73.37470998)
\curveto(883.5450057,73.29470534)(883.53500572,73.20470543)(883.535,73.10470998)
\lineto(883.535,72.78970998)
\curveto(883.55500569,72.76970586)(883.56500568,72.7297059)(883.565,72.66970998)
\curveto(883.55500569,72.61970601)(883.55500569,72.57470606)(883.565,72.53470998)
\lineto(883.625,72.29470998)
\curveto(883.63500561,72.22470641)(883.6550056,72.15470648)(883.685,72.08470998)
\curveto(883.9450053,71.48470715)(884.41000484,71.07970755)(885.08,70.86970998)
\curveto(885.16000409,70.83970779)(885.24000401,70.81970781)(885.32,70.80970998)
\curveto(885.40000385,70.79970783)(885.48500376,70.78470785)(885.575,70.76470998)
\lineto(885.725,70.76470998)
\curveto(885.76500348,70.75470788)(885.83500341,70.74970788)(885.935,70.74970998)
\curveto(886.16500308,70.74970788)(886.36000289,70.76970786)(886.52,70.80970998)
\curveto(886.59000266,70.8297078)(886.6550026,70.84470779)(886.715,70.85470998)
\curveto(886.77500247,70.86470777)(886.84000241,70.88470775)(886.91,70.91470998)
\curveto(887.19000206,71.02470761)(887.43500181,71.16970746)(887.645,71.34970998)
\curveto(887.8450014,71.5297071)(888.00500125,71.76470687)(888.125,72.05470998)
\lineto(888.215,72.29470998)
\lineto(888.275,72.53470998)
\curveto(888.29500095,72.58470605)(888.30000095,72.62470601)(888.29,72.65470998)
\curveto(888.28000097,72.69470594)(888.28500096,72.73970589)(888.305,72.78970998)
\curveto(888.31500093,72.81970581)(888.32000093,72.87470576)(888.32,72.95470998)
\curveto(888.32000093,73.0347056)(888.31500093,73.09470554)(888.305,73.13470998)
\curveto(888.28500096,73.24470539)(888.27000098,73.34970528)(888.26,73.44970998)
\curveto(888.250001,73.54970508)(888.22000103,73.64470499)(888.17,73.73470998)
\curveto(887.97000128,74.26470437)(887.59500166,74.65470398)(887.045,74.90470998)
\curveto(886.94500231,74.94470369)(886.84000241,74.97470366)(886.73,74.99470998)
\lineto(886.4,75.08470998)
\curveto(886.32000293,75.08470355)(886.26000299,75.08970354)(886.22,75.09970998)
}
}
{
\newrgbcolor{curcolor}{0 0 0}
\pscustom[linestyle=none,fillstyle=solid,fillcolor=curcolor]
{
\newpath
\moveto(887.6,78.63431936)
\lineto(887.6,79.26431936)
\lineto(887.6,79.45931936)
\curveto(887.60000165,79.52931683)(887.61000164,79.58931677)(887.63,79.63931936)
\curveto(887.67000158,79.70931665)(887.71000154,79.7593166)(887.75,79.78931936)
\curveto(887.80000145,79.82931653)(887.86500139,79.84931651)(887.945,79.84931936)
\curveto(888.02500122,79.8593165)(888.11000114,79.86431649)(888.2,79.86431936)
\lineto(888.92,79.86431936)
\curveto(889.39999985,79.86431649)(889.80999944,79.80431655)(890.15,79.68431936)
\curveto(890.48999876,79.56431679)(890.76499848,79.36931699)(890.975,79.09931936)
\curveto(891.02499822,79.02931733)(891.06999818,78.9593174)(891.11,78.88931936)
\curveto(891.15999809,78.82931753)(891.20499805,78.7543176)(891.245,78.66431936)
\curveto(891.254998,78.64431771)(891.26499799,78.61431774)(891.275,78.57431936)
\curveto(891.29499795,78.53431782)(891.29999795,78.48931787)(891.29,78.43931936)
\curveto(891.25999799,78.34931801)(891.18499806,78.29431806)(891.065,78.27431936)
\curveto(890.95499829,78.2543181)(890.85999839,78.26931809)(890.78,78.31931936)
\curveto(890.70999854,78.34931801)(890.6449986,78.39431796)(890.585,78.45431936)
\curveto(890.53499872,78.52431783)(890.48499877,78.58931777)(890.435,78.64931936)
\curveto(890.38499886,78.71931764)(890.30999894,78.77931758)(890.21,78.82931936)
\curveto(890.11999913,78.88931747)(890.02999922,78.93931742)(889.94,78.97931936)
\curveto(889.90999934,78.99931736)(889.8499994,79.02431733)(889.76,79.05431936)
\curveto(889.67999957,79.08431727)(889.60999964,79.08931727)(889.55,79.06931936)
\curveto(889.40999984,79.03931732)(889.31999993,78.97931738)(889.28,78.88931936)
\curveto(889.25,78.80931755)(889.23500001,78.71931764)(889.235,78.61931936)
\curveto(889.23500001,78.51931784)(889.21000004,78.43431792)(889.16,78.36431936)
\curveto(889.09000016,78.27431808)(888.9500003,78.22931813)(888.74,78.22931936)
\lineto(888.185,78.22931936)
\lineto(887.96,78.22931936)
\curveto(887.88000137,78.23931812)(887.81500143,78.2593181)(887.765,78.28931936)
\curveto(887.68500156,78.34931801)(887.64000161,78.41931794)(887.63,78.49931936)
\curveto(887.62000163,78.51931784)(887.61500163,78.53931782)(887.615,78.55931936)
\curveto(887.61500163,78.58931777)(887.61000164,78.61431774)(887.6,78.63431936)
}
}
{
\newrgbcolor{curcolor}{0 0 0}
\pscustom[linestyle=none,fillstyle=solid,fillcolor=curcolor]
{
}
}
{
\newrgbcolor{curcolor}{0 0 0}
\pscustom[linestyle=none,fillstyle=solid,fillcolor=curcolor]
{
\newpath
\moveto(878.63,89.26463186)
\curveto(878.62001063,89.95462722)(878.74001051,90.55462662)(878.99,91.06463186)
\curveto(879.24001001,91.58462559)(879.57500968,91.9796252)(879.995,92.24963186)
\curveto(880.07500917,92.29962488)(880.16500908,92.34462483)(880.265,92.38463186)
\curveto(880.35500889,92.42462475)(880.4500088,92.46962471)(880.55,92.51963186)
\curveto(880.6500086,92.55962462)(880.7500085,92.58962459)(880.85,92.60963186)
\curveto(880.9500083,92.62962455)(881.0550082,92.64962453)(881.165,92.66963186)
\curveto(881.21500803,92.68962449)(881.26000799,92.69462448)(881.3,92.68463186)
\curveto(881.34000791,92.6746245)(881.38500787,92.6796245)(881.435,92.69963186)
\curveto(881.48500776,92.70962447)(881.57000768,92.71462446)(881.69,92.71463186)
\curveto(881.80000745,92.71462446)(881.88500736,92.70962447)(881.945,92.69963186)
\curveto(882.00500725,92.6796245)(882.06500719,92.66962451)(882.125,92.66963186)
\curveto(882.18500707,92.6796245)(882.24500701,92.6746245)(882.305,92.65463186)
\curveto(882.44500681,92.61462456)(882.58000667,92.5796246)(882.71,92.54963186)
\curveto(882.84000641,92.51962466)(882.96500628,92.4796247)(883.085,92.42963186)
\curveto(883.22500602,92.36962481)(883.3500059,92.29962488)(883.46,92.21963186)
\curveto(883.57000568,92.14962503)(883.68000557,92.0746251)(883.79,91.99463186)
\lineto(883.85,91.93463186)
\curveto(883.87000538,91.92462525)(883.89000536,91.90962527)(883.91,91.88963186)
\curveto(884.07000518,91.76962541)(884.21500503,91.63462554)(884.345,91.48463186)
\curveto(884.47500478,91.33462584)(884.60000465,91.174626)(884.72,91.00463186)
\curveto(884.94000431,90.69462648)(885.1450041,90.39962678)(885.335,90.11963186)
\curveto(885.47500378,89.88962729)(885.61000364,89.65962752)(885.74,89.42963186)
\curveto(885.87000338,89.20962797)(886.00500325,88.98962819)(886.145,88.76963186)
\curveto(886.31500294,88.51962866)(886.49500275,88.2796289)(886.685,88.04963186)
\curveto(886.87500238,87.82962935)(887.10000215,87.63962954)(887.36,87.47963186)
\curveto(887.42000183,87.43962974)(887.48000177,87.40462977)(887.54,87.37463186)
\curveto(887.59000166,87.34462983)(887.6550016,87.31462986)(887.735,87.28463186)
\curveto(887.80500145,87.26462991)(887.86500139,87.25962992)(887.915,87.26963186)
\curveto(887.98500127,87.28962989)(888.04000121,87.32462985)(888.08,87.37463186)
\curveto(888.11000114,87.42462975)(888.13000112,87.48462969)(888.14,87.55463186)
\lineto(888.14,87.79463186)
\lineto(888.14,88.54463186)
\lineto(888.14,91.34963186)
\lineto(888.14,92.00963186)
\curveto(888.14000111,92.09962508)(888.14500111,92.18462499)(888.155,92.26463186)
\curveto(888.15500109,92.34462483)(888.17500107,92.40962477)(888.215,92.45963186)
\curveto(888.255001,92.50962467)(888.33000092,92.54962463)(888.44,92.57963186)
\curveto(888.54000071,92.61962456)(888.64000061,92.62962455)(888.74,92.60963186)
\lineto(888.875,92.60963186)
\curveto(888.94500031,92.58962459)(889.00500025,92.56962461)(889.055,92.54963186)
\curveto(889.10500014,92.52962465)(889.14500011,92.49462468)(889.175,92.44463186)
\curveto(889.21500004,92.39462478)(889.23500001,92.32462485)(889.235,92.23463186)
\lineto(889.235,91.96463186)
\lineto(889.235,91.06463186)
\lineto(889.235,87.55463186)
\lineto(889.235,86.48963186)
\curveto(889.23500001,86.40963077)(889.24000001,86.31963086)(889.25,86.21963186)
\curveto(889.25,86.11963106)(889.24000001,86.03463114)(889.22,85.96463186)
\curveto(889.1500001,85.75463142)(888.97000028,85.68963149)(888.68,85.76963186)
\curveto(888.64000061,85.7796314)(888.60500065,85.7796314)(888.575,85.76963186)
\curveto(888.53500072,85.76963141)(888.49000076,85.7796314)(888.44,85.79963186)
\curveto(888.36000089,85.81963136)(888.27500098,85.83963134)(888.185,85.85963186)
\curveto(888.09500115,85.8796313)(888.01000124,85.90463127)(887.93,85.93463186)
\curveto(887.44000181,86.09463108)(887.02500222,86.29463088)(886.685,86.53463186)
\curveto(886.43500281,86.71463046)(886.21000304,86.91963026)(886.01,87.14963186)
\curveto(885.80000345,87.3796298)(885.60500365,87.61962956)(885.425,87.86963186)
\curveto(885.24500401,88.12962905)(885.07500418,88.39462878)(884.915,88.66463186)
\curveto(884.7450045,88.94462823)(884.57000468,89.21462796)(884.39,89.47463186)
\curveto(884.31000494,89.58462759)(884.23500501,89.68962749)(884.165,89.78963186)
\curveto(884.09500515,89.89962728)(884.02000523,90.00962717)(883.94,90.11963186)
\curveto(883.91000534,90.15962702)(883.88000537,90.19462698)(883.85,90.22463186)
\curveto(883.81000544,90.26462691)(883.78000547,90.30462687)(883.76,90.34463186)
\curveto(883.6500056,90.48462669)(883.52500573,90.60962657)(883.385,90.71963186)
\curveto(883.35500589,90.73962644)(883.33000592,90.76462641)(883.31,90.79463186)
\curveto(883.28000597,90.82462635)(883.250006,90.84962633)(883.22,90.86963186)
\curveto(883.12000613,90.94962623)(883.02000623,91.01462616)(882.92,91.06463186)
\curveto(882.82000643,91.12462605)(882.71000654,91.179626)(882.59,91.22963186)
\curveto(882.52000673,91.25962592)(882.44500681,91.2796259)(882.365,91.28963186)
\lineto(882.125,91.34963186)
\lineto(882.035,91.34963186)
\curveto(882.00500725,91.35962582)(881.97500728,91.36462581)(881.945,91.36463186)
\curveto(881.87500737,91.38462579)(881.78000747,91.38962579)(881.66,91.37963186)
\curveto(881.53000772,91.3796258)(881.43000782,91.36962581)(881.36,91.34963186)
\curveto(881.28000797,91.32962585)(881.20500804,91.30962587)(881.135,91.28963186)
\curveto(881.0550082,91.2796259)(880.97500828,91.25962592)(880.895,91.22963186)
\curveto(880.6550086,91.11962606)(880.4550088,90.96962621)(880.295,90.77963186)
\curveto(880.12500913,90.59962658)(879.98500927,90.3796268)(879.875,90.11963186)
\curveto(879.8550094,90.04962713)(879.84000941,89.9796272)(879.83,89.90963186)
\curveto(879.81000944,89.83962734)(879.79000946,89.76462741)(879.77,89.68463186)
\curveto(879.7500095,89.60462757)(879.74000951,89.49462768)(879.74,89.35463186)
\curveto(879.74000951,89.22462795)(879.7500095,89.11962806)(879.77,89.03963186)
\curveto(879.78000947,88.9796282)(879.78500947,88.92462825)(879.785,88.87463186)
\curveto(879.78500947,88.82462835)(879.79500946,88.7746284)(879.815,88.72463186)
\curveto(879.8550094,88.62462855)(879.89500935,88.52962865)(879.935,88.43963186)
\curveto(879.97500928,88.35962882)(880.02000923,88.2796289)(880.07,88.19963186)
\curveto(880.09000916,88.16962901)(880.11500914,88.13962904)(880.145,88.10963186)
\curveto(880.17500908,88.08962909)(880.20000905,88.06462911)(880.22,88.03463186)
\lineto(880.295,87.95963186)
\curveto(880.31500894,87.92962925)(880.33500892,87.90462927)(880.355,87.88463186)
\lineto(880.565,87.73463186)
\curveto(880.62500862,87.69462948)(880.69000856,87.64962953)(880.76,87.59963186)
\curveto(880.8500084,87.53962964)(880.95500829,87.48962969)(881.075,87.44963186)
\curveto(881.18500807,87.41962976)(881.29500795,87.38462979)(881.405,87.34463186)
\curveto(881.51500774,87.30462987)(881.66000759,87.2796299)(881.84,87.26963186)
\curveto(882.01000724,87.25962992)(882.13500712,87.22962995)(882.215,87.17963186)
\curveto(882.29500695,87.12963005)(882.34000691,87.05463012)(882.35,86.95463186)
\curveto(882.36000689,86.85463032)(882.36500688,86.74463043)(882.365,86.62463186)
\curveto(882.36500688,86.58463059)(882.37000688,86.54463063)(882.38,86.50463186)
\curveto(882.38000687,86.46463071)(882.37500688,86.42963075)(882.365,86.39963186)
\curveto(882.3450069,86.34963083)(882.33500692,86.29963088)(882.335,86.24963186)
\curveto(882.33500692,86.20963097)(882.32500693,86.16963101)(882.305,86.12963186)
\curveto(882.24500701,86.03963114)(882.11000714,85.99463118)(881.9,85.99463186)
\lineto(881.78,85.99463186)
\curveto(881.72000753,86.00463117)(881.66000759,86.00963117)(881.6,86.00963186)
\curveto(881.53000772,86.01963116)(881.46500779,86.02963115)(881.405,86.03963186)
\curveto(881.29500795,86.05963112)(881.19500806,86.0796311)(881.105,86.09963186)
\curveto(881.00500824,86.11963106)(880.91000834,86.14963103)(880.82,86.18963186)
\curveto(880.7500085,86.20963097)(880.69000856,86.22963095)(880.64,86.24963186)
\lineto(880.46,86.30963186)
\curveto(880.20000905,86.42963075)(879.95500929,86.58463059)(879.725,86.77463186)
\curveto(879.49500975,86.9746302)(879.31000994,87.18962999)(879.17,87.41963186)
\curveto(879.09001016,87.52962965)(879.02501022,87.64462953)(878.975,87.76463186)
\lineto(878.825,88.15463186)
\curveto(878.77501048,88.26462891)(878.7450105,88.3796288)(878.735,88.49963186)
\curveto(878.71501054,88.61962856)(878.69001056,88.74462843)(878.66,88.87463186)
\curveto(878.66001059,88.94462823)(878.66001059,89.00962817)(878.66,89.06963186)
\curveto(878.6500106,89.12962805)(878.64001061,89.19462798)(878.63,89.26463186)
}
}
{
\newrgbcolor{curcolor}{0 0 0}
\pscustom[linestyle=none,fillstyle=solid,fillcolor=curcolor]
{
\newpath
\moveto(884.15,101.36424123)
\lineto(884.405,101.36424123)
\curveto(884.48500476,101.37423353)(884.56000469,101.36923353)(884.63,101.34924123)
\lineto(884.87,101.34924123)
\lineto(885.035,101.34924123)
\curveto(885.13500412,101.32923357)(885.24000401,101.31923358)(885.35,101.31924123)
\curveto(885.4500038,101.31923358)(885.5500037,101.30923359)(885.65,101.28924123)
\lineto(885.8,101.28924123)
\curveto(885.94000331,101.25923364)(886.08000317,101.23923366)(886.22,101.22924123)
\curveto(886.3500029,101.21923368)(886.48000277,101.19423371)(886.61,101.15424123)
\curveto(886.69000256,101.13423377)(886.77500247,101.11423379)(886.865,101.09424123)
\lineto(887.105,101.03424123)
\lineto(887.405,100.91424123)
\curveto(887.49500175,100.88423402)(887.58500167,100.84923405)(887.675,100.80924123)
\curveto(887.89500135,100.70923419)(888.11000114,100.57423433)(888.32,100.40424123)
\curveto(888.53000072,100.24423466)(888.70000055,100.06923483)(888.83,99.87924123)
\curveto(888.87000038,99.82923507)(888.91000034,99.76923513)(888.95,99.69924123)
\curveto(888.98000027,99.63923526)(889.01500024,99.57923532)(889.055,99.51924123)
\curveto(889.10500014,99.43923546)(889.14500011,99.34423556)(889.175,99.23424123)
\curveto(889.20500005,99.12423578)(889.23500001,99.01923588)(889.265,98.91924123)
\curveto(889.30499994,98.80923609)(889.32999992,98.6992362)(889.34,98.58924123)
\curveto(889.3499999,98.47923642)(889.36499988,98.36423654)(889.385,98.24424123)
\curveto(889.39499986,98.2042367)(889.39499986,98.15923674)(889.385,98.10924123)
\curveto(889.38499987,98.06923683)(889.38999986,98.02923687)(889.4,97.98924123)
\curveto(889.40999984,97.94923695)(889.41499984,97.89423701)(889.415,97.82424123)
\curveto(889.41499984,97.75423715)(889.40999984,97.7042372)(889.4,97.67424123)
\curveto(889.37999987,97.62423728)(889.37499987,97.57923732)(889.385,97.53924123)
\curveto(889.39499986,97.4992374)(889.39499986,97.46423744)(889.385,97.43424123)
\lineto(889.385,97.34424123)
\curveto(889.36499988,97.28423762)(889.3499999,97.21923768)(889.34,97.14924123)
\curveto(889.33999991,97.08923781)(889.33499992,97.02423788)(889.325,96.95424123)
\curveto(889.27499998,96.78423812)(889.22500002,96.62423828)(889.175,96.47424123)
\curveto(889.12500013,96.32423858)(889.06000019,96.17923872)(888.98,96.03924123)
\curveto(888.94000031,95.98923891)(888.91000034,95.93423897)(888.89,95.87424123)
\curveto(888.86000039,95.82423908)(888.82500042,95.77423913)(888.785,95.72424123)
\curveto(888.60500065,95.48423942)(888.38500087,95.28423962)(888.125,95.12424123)
\curveto(887.86500139,94.96423994)(887.58000167,94.82424008)(887.27,94.70424123)
\curveto(887.13000212,94.64424026)(886.99000226,94.5992403)(886.85,94.56924123)
\curveto(886.70000255,94.53924036)(886.54500271,94.5042404)(886.385,94.46424123)
\curveto(886.27500298,94.44424046)(886.16500308,94.42924047)(886.055,94.41924123)
\curveto(885.94500331,94.40924049)(885.83500341,94.39424051)(885.725,94.37424123)
\curveto(885.68500356,94.36424054)(885.64500361,94.35924054)(885.605,94.35924123)
\curveto(885.56500368,94.36924053)(885.52500373,94.36924053)(885.485,94.35924123)
\curveto(885.43500381,94.34924055)(885.38500387,94.34424056)(885.335,94.34424123)
\lineto(885.17,94.34424123)
\curveto(885.12000413,94.32424058)(885.07000418,94.31924058)(885.02,94.32924123)
\curveto(884.96000429,94.33924056)(884.90500434,94.33924056)(884.855,94.32924123)
\curveto(884.81500443,94.31924058)(884.77000448,94.31924058)(884.72,94.32924123)
\curveto(884.67000458,94.33924056)(884.62000463,94.33424057)(884.57,94.31424123)
\curveto(884.50000475,94.29424061)(884.42500482,94.28924061)(884.345,94.29924123)
\curveto(884.255005,94.30924059)(884.17000508,94.31424059)(884.09,94.31424123)
\curveto(884.00000525,94.31424059)(883.90000535,94.30924059)(883.79,94.29924123)
\curveto(883.67000558,94.28924061)(883.57000568,94.29424061)(883.49,94.31424123)
\lineto(883.205,94.31424123)
\lineto(882.575,94.35924123)
\curveto(882.47500677,94.36924053)(882.38000687,94.37924052)(882.29,94.38924123)
\lineto(881.99,94.41924123)
\curveto(881.94000731,94.43924046)(881.89000736,94.44424046)(881.84,94.43424123)
\curveto(881.78000747,94.43424047)(881.72500753,94.44424046)(881.675,94.46424123)
\curveto(881.50500775,94.51424039)(881.34000791,94.55424035)(881.18,94.58424123)
\curveto(881.01000824,94.61424029)(880.8500084,94.66424024)(880.7,94.73424123)
\curveto(880.24000901,94.92423998)(879.86500939,95.14423976)(879.575,95.39424123)
\curveto(879.28500996,95.65423925)(879.04001021,96.01423889)(878.84,96.47424123)
\curveto(878.79001046,96.6042383)(878.75501049,96.73423817)(878.735,96.86424123)
\curveto(878.71501054,97.0042379)(878.69001056,97.14423776)(878.66,97.28424123)
\curveto(878.6500106,97.35423755)(878.64501061,97.41923748)(878.645,97.47924123)
\curveto(878.64501061,97.53923736)(878.64001061,97.6042373)(878.63,97.67424123)
\curveto(878.61001064,98.5042364)(878.76001049,99.17423573)(879.08,99.68424123)
\curveto(879.39000986,100.19423471)(879.83000942,100.57423433)(880.4,100.82424123)
\curveto(880.52000873,100.87423403)(880.64500861,100.91923398)(880.775,100.95924123)
\curveto(880.90500835,100.9992339)(881.04000821,101.04423386)(881.18,101.09424123)
\curveto(881.26000799,101.11423379)(881.3450079,101.12923377)(881.435,101.13924123)
\lineto(881.675,101.19924123)
\curveto(881.78500747,101.22923367)(881.89500735,101.24423366)(882.005,101.24424123)
\curveto(882.11500714,101.25423365)(882.22500702,101.26923363)(882.335,101.28924123)
\curveto(882.38500687,101.30923359)(882.43000682,101.31423359)(882.47,101.30424123)
\curveto(882.51000674,101.3042336)(882.5500067,101.30923359)(882.59,101.31924123)
\curveto(882.64000661,101.32923357)(882.69500655,101.32923357)(882.755,101.31924123)
\curveto(882.80500645,101.31923358)(882.8550064,101.32423358)(882.905,101.33424123)
\lineto(883.04,101.33424123)
\curveto(883.10000615,101.35423355)(883.17000608,101.35423355)(883.25,101.33424123)
\curveto(883.32000593,101.32423358)(883.38500587,101.32923357)(883.445,101.34924123)
\curveto(883.47500577,101.35923354)(883.51500574,101.36423354)(883.565,101.36424123)
\lineto(883.685,101.36424123)
\lineto(884.15,101.36424123)
\moveto(886.475,99.81924123)
\curveto(886.15500309,99.91923498)(885.79000346,99.97923492)(885.38,99.99924123)
\curveto(884.97000428,100.01923488)(884.56000469,100.02923487)(884.15,100.02924123)
\curveto(883.72000553,100.02923487)(883.30000595,100.01923488)(882.89,99.99924123)
\curveto(882.48000677,99.97923492)(882.09500715,99.93423497)(881.735,99.86424123)
\curveto(881.37500788,99.79423511)(881.0550082,99.68423522)(880.775,99.53424123)
\curveto(880.48500876,99.39423551)(880.250009,99.1992357)(880.07,98.94924123)
\curveto(879.96000929,98.78923611)(879.88000937,98.60923629)(879.83,98.40924123)
\curveto(879.77000948,98.20923669)(879.74000951,97.96423694)(879.74,97.67424123)
\curveto(879.76000949,97.65423725)(879.77000948,97.61923728)(879.77,97.56924123)
\curveto(879.76000949,97.51923738)(879.76000949,97.47923742)(879.77,97.44924123)
\curveto(879.79000946,97.36923753)(879.81000944,97.29423761)(879.83,97.22424123)
\curveto(879.84000941,97.16423774)(879.86000939,97.0992378)(879.89,97.02924123)
\curveto(880.01000924,96.75923814)(880.18000907,96.53923836)(880.4,96.36924123)
\curveto(880.61000864,96.20923869)(880.8550084,96.07423883)(881.135,95.96424123)
\curveto(881.24500801,95.91423899)(881.36500788,95.87423903)(881.495,95.84424123)
\curveto(881.61500763,95.82423908)(881.74000751,95.7992391)(881.87,95.76924123)
\curveto(881.92000733,95.74923915)(881.97500728,95.73923916)(882.035,95.73924123)
\curveto(882.08500716,95.73923916)(882.13500712,95.73423917)(882.185,95.72424123)
\curveto(882.27500697,95.71423919)(882.37000688,95.7042392)(882.47,95.69424123)
\curveto(882.56000669,95.68423922)(882.6550066,95.67423923)(882.755,95.66424123)
\curveto(882.83500641,95.66423924)(882.92000633,95.65923924)(883.01,95.64924123)
\lineto(883.25,95.64924123)
\lineto(883.43,95.64924123)
\curveto(883.46000579,95.63923926)(883.49500575,95.63423927)(883.535,95.63424123)
\lineto(883.67,95.63424123)
\lineto(884.12,95.63424123)
\curveto(884.20000505,95.63423927)(884.28500496,95.62923927)(884.375,95.61924123)
\curveto(884.4550048,95.61923928)(884.53000472,95.62923927)(884.6,95.64924123)
\lineto(884.87,95.64924123)
\curveto(884.89000436,95.64923925)(884.92000433,95.64423926)(884.96,95.63424123)
\curveto(884.99000426,95.63423927)(885.01500423,95.63923926)(885.035,95.64924123)
\curveto(885.13500412,95.65923924)(885.23500401,95.66423924)(885.335,95.66424123)
\curveto(885.42500382,95.67423923)(885.52500373,95.68423922)(885.635,95.69424123)
\curveto(885.75500349,95.72423918)(885.88000337,95.73923916)(886.01,95.73924123)
\curveto(886.13000312,95.74923915)(886.245003,95.77423913)(886.355,95.81424123)
\curveto(886.6550026,95.89423901)(886.92000233,95.97923892)(887.15,96.06924123)
\curveto(887.38000187,96.16923873)(887.59500166,96.31423859)(887.795,96.50424123)
\curveto(887.99500126,96.71423819)(888.14500111,96.97923792)(888.245,97.29924123)
\curveto(888.26500099,97.33923756)(888.27500098,97.37423753)(888.275,97.40424123)
\curveto(888.26500099,97.44423746)(888.27000098,97.48923741)(888.29,97.53924123)
\curveto(888.30000095,97.57923732)(888.31000094,97.64923725)(888.32,97.74924123)
\curveto(888.33000092,97.85923704)(888.32500093,97.94423696)(888.305,98.00424123)
\curveto(888.28500096,98.07423683)(888.27500098,98.14423676)(888.275,98.21424123)
\curveto(888.26500099,98.28423662)(888.250001,98.34923655)(888.23,98.40924123)
\curveto(888.17000108,98.60923629)(888.08500116,98.78923611)(887.975,98.94924123)
\curveto(887.95500129,98.97923592)(887.93500132,99.0042359)(887.915,99.02424123)
\lineto(887.855,99.08424123)
\curveto(887.83500141,99.12423578)(887.79500146,99.17423573)(887.735,99.23424123)
\curveto(887.59500166,99.33423557)(887.46500179,99.41923548)(887.345,99.48924123)
\curveto(887.22500202,99.55923534)(887.08000217,99.62923527)(886.91,99.69924123)
\curveto(886.84000241,99.72923517)(886.77000248,99.74923515)(886.7,99.75924123)
\curveto(886.63000262,99.77923512)(886.55500269,99.7992351)(886.475,99.81924123)
}
}
{
\newrgbcolor{curcolor}{0 0 0}
\pscustom[linestyle=none,fillstyle=solid,fillcolor=curcolor]
{
\newpath
\moveto(878.63,106.77385061)
\curveto(878.63001062,106.87384575)(878.64001061,106.96884566)(878.66,107.05885061)
\curveto(878.67001058,107.14884548)(878.70001055,107.21384541)(878.75,107.25385061)
\curveto(878.83001042,107.31384531)(878.93501031,107.34384528)(879.065,107.34385061)
\lineto(879.455,107.34385061)
\lineto(880.955,107.34385061)
\lineto(887.345,107.34385061)
\lineto(888.515,107.34385061)
\lineto(888.83,107.34385061)
\curveto(888.93000032,107.35384527)(889.01000024,107.33884529)(889.07,107.29885061)
\curveto(889.1500001,107.24884538)(889.20000005,107.17384545)(889.22,107.07385061)
\curveto(889.23000002,106.98384564)(889.23500001,106.87384575)(889.235,106.74385061)
\lineto(889.235,106.51885061)
\curveto(889.21500004,106.43884619)(889.20000005,106.36884626)(889.19,106.30885061)
\curveto(889.17000008,106.24884638)(889.13000012,106.19884643)(889.07,106.15885061)
\curveto(889.01000024,106.11884651)(888.93500032,106.09884653)(888.845,106.09885061)
\lineto(888.545,106.09885061)
\lineto(887.45,106.09885061)
\lineto(882.11,106.09885061)
\curveto(882.02000723,106.07884655)(881.9450073,106.06384656)(881.885,106.05385061)
\curveto(881.81500743,106.05384657)(881.75500749,106.0238466)(881.705,105.96385061)
\curveto(881.6550076,105.89384673)(881.63000762,105.80384682)(881.63,105.69385061)
\curveto(881.62000763,105.59384703)(881.61500763,105.48384714)(881.615,105.36385061)
\lineto(881.615,104.22385061)
\lineto(881.615,103.72885061)
\curveto(881.60500765,103.56884906)(881.5450077,103.45884917)(881.435,103.39885061)
\curveto(881.40500785,103.37884925)(881.37500788,103.36884926)(881.345,103.36885061)
\curveto(881.30500794,103.36884926)(881.26000799,103.36384926)(881.21,103.35385061)
\curveto(881.09000816,103.33384929)(880.98000827,103.33884929)(880.88,103.36885061)
\curveto(880.78000847,103.40884922)(880.71000854,103.46384916)(880.67,103.53385061)
\curveto(880.62000863,103.61384901)(880.59500866,103.73384889)(880.595,103.89385061)
\curveto(880.59500866,104.05384857)(880.58000867,104.18884844)(880.55,104.29885061)
\curveto(880.54000871,104.34884828)(880.53500872,104.40384822)(880.535,104.46385061)
\curveto(880.52500873,104.5238481)(880.51000874,104.58384804)(880.49,104.64385061)
\curveto(880.44000881,104.79384783)(880.39000886,104.93884769)(880.34,105.07885061)
\curveto(880.28000897,105.21884741)(880.21000904,105.35384727)(880.13,105.48385061)
\curveto(880.04000921,105.623847)(879.93500932,105.74384688)(879.815,105.84385061)
\curveto(879.69500955,105.94384668)(879.56500968,106.03884659)(879.425,106.12885061)
\curveto(879.32500993,106.18884644)(879.21501003,106.23384639)(879.095,106.26385061)
\curveto(878.97501028,106.30384632)(878.87001038,106.35384627)(878.78,106.41385061)
\curveto(878.72001053,106.46384616)(878.68001057,106.53384609)(878.66,106.62385061)
\curveto(878.6500106,106.64384598)(878.64501061,106.66884596)(878.645,106.69885061)
\curveto(878.64501061,106.7288459)(878.64001061,106.75384587)(878.63,106.77385061)
}
}
{
\newrgbcolor{curcolor}{0 0 0}
\pscustom[linestyle=none,fillstyle=solid,fillcolor=curcolor]
{
\newpath
\moveto(878.63,115.12345998)
\curveto(878.63001062,115.22345513)(878.64001061,115.31845503)(878.66,115.40845998)
\curveto(878.67001058,115.49845485)(878.70001055,115.56345479)(878.75,115.60345998)
\curveto(878.83001042,115.66345469)(878.93501031,115.69345466)(879.065,115.69345998)
\lineto(879.455,115.69345998)
\lineto(880.955,115.69345998)
\lineto(887.345,115.69345998)
\lineto(888.515,115.69345998)
\lineto(888.83,115.69345998)
\curveto(888.93000032,115.70345465)(889.01000024,115.68845466)(889.07,115.64845998)
\curveto(889.1500001,115.59845475)(889.20000005,115.52345483)(889.22,115.42345998)
\curveto(889.23000002,115.33345502)(889.23500001,115.22345513)(889.235,115.09345998)
\lineto(889.235,114.86845998)
\curveto(889.21500004,114.78845556)(889.20000005,114.71845563)(889.19,114.65845998)
\curveto(889.17000008,114.59845575)(889.13000012,114.5484558)(889.07,114.50845998)
\curveto(889.01000024,114.46845588)(888.93500032,114.4484559)(888.845,114.44845998)
\lineto(888.545,114.44845998)
\lineto(887.45,114.44845998)
\lineto(882.11,114.44845998)
\curveto(882.02000723,114.42845592)(881.9450073,114.41345594)(881.885,114.40345998)
\curveto(881.81500743,114.40345595)(881.75500749,114.37345598)(881.705,114.31345998)
\curveto(881.6550076,114.24345611)(881.63000762,114.1534562)(881.63,114.04345998)
\curveto(881.62000763,113.94345641)(881.61500763,113.83345652)(881.615,113.71345998)
\lineto(881.615,112.57345998)
\lineto(881.615,112.07845998)
\curveto(881.60500765,111.91845843)(881.5450077,111.80845854)(881.435,111.74845998)
\curveto(881.40500785,111.72845862)(881.37500788,111.71845863)(881.345,111.71845998)
\curveto(881.30500794,111.71845863)(881.26000799,111.71345864)(881.21,111.70345998)
\curveto(881.09000816,111.68345867)(880.98000827,111.68845866)(880.88,111.71845998)
\curveto(880.78000847,111.75845859)(880.71000854,111.81345854)(880.67,111.88345998)
\curveto(880.62000863,111.96345839)(880.59500866,112.08345827)(880.595,112.24345998)
\curveto(880.59500866,112.40345795)(880.58000867,112.53845781)(880.55,112.64845998)
\curveto(880.54000871,112.69845765)(880.53500872,112.7534576)(880.535,112.81345998)
\curveto(880.52500873,112.87345748)(880.51000874,112.93345742)(880.49,112.99345998)
\curveto(880.44000881,113.14345721)(880.39000886,113.28845706)(880.34,113.42845998)
\curveto(880.28000897,113.56845678)(880.21000904,113.70345665)(880.13,113.83345998)
\curveto(880.04000921,113.97345638)(879.93500932,114.09345626)(879.815,114.19345998)
\curveto(879.69500955,114.29345606)(879.56500968,114.38845596)(879.425,114.47845998)
\curveto(879.32500993,114.53845581)(879.21501003,114.58345577)(879.095,114.61345998)
\curveto(878.97501028,114.6534557)(878.87001038,114.70345565)(878.78,114.76345998)
\curveto(878.72001053,114.81345554)(878.68001057,114.88345547)(878.66,114.97345998)
\curveto(878.6500106,114.99345536)(878.64501061,115.01845533)(878.645,115.04845998)
\curveto(878.64501061,115.07845527)(878.64001061,115.10345525)(878.63,115.12345998)
}
}
{
\newrgbcolor{curcolor}{0 0 0}
\pscustom[linestyle=none,fillstyle=solid,fillcolor=curcolor]
{
\newpath
\moveto(899.46631592,42.29681936)
\curveto(899.46632661,42.36681368)(899.46632661,42.4468136)(899.46631592,42.53681936)
\curveto(899.45632662,42.62681342)(899.45632662,42.71181333)(899.46631592,42.79181936)
\curveto(899.46632661,42.88181316)(899.4763266,42.96181308)(899.49631592,43.03181936)
\curveto(899.51632656,43.11181293)(899.54632653,43.16681288)(899.58631592,43.19681936)
\curveto(899.63632644,43.22681282)(899.71132637,43.2468128)(899.81131592,43.25681936)
\curveto(899.90132618,43.27681277)(900.00632607,43.28681276)(900.12631592,43.28681936)
\curveto(900.23632584,43.29681275)(900.35132573,43.29681275)(900.47131592,43.28681936)
\lineto(900.77131592,43.28681936)
\lineto(903.78631592,43.28681936)
\lineto(906.68131592,43.28681936)
\curveto(907.01131907,43.28681276)(907.33631874,43.28181276)(907.65631592,43.27181936)
\curveto(907.96631811,43.27181277)(908.24631783,43.23181281)(908.49631592,43.15181936)
\curveto(908.84631723,43.03181301)(909.14131694,42.87681317)(909.38131592,42.68681936)
\curveto(909.61131647,42.49681355)(909.81131627,42.25681379)(909.98131592,41.96681936)
\curveto(910.03131605,41.90681414)(910.06631601,41.8418142)(910.08631592,41.77181936)
\curveto(910.10631597,41.71181433)(910.13131595,41.6418144)(910.16131592,41.56181936)
\curveto(910.21131587,41.4418146)(910.24631583,41.31181473)(910.26631592,41.17181936)
\curveto(910.29631578,41.041815)(910.32631575,40.90681514)(910.35631592,40.76681936)
\curveto(910.3763157,40.71681533)(910.3813157,40.66681538)(910.37131592,40.61681936)
\curveto(910.36131572,40.56681548)(910.36131572,40.51181553)(910.37131592,40.45181936)
\curveto(910.3813157,40.43181561)(910.3813157,40.40681564)(910.37131592,40.37681936)
\curveto(910.37131571,40.3468157)(910.3763157,40.32181572)(910.38631592,40.30181936)
\curveto(910.39631568,40.26181578)(910.40131568,40.20681584)(910.40131592,40.13681936)
\curveto(910.40131568,40.06681598)(910.39631568,40.01181603)(910.38631592,39.97181936)
\curveto(910.3763157,39.92181612)(910.3763157,39.86681618)(910.38631592,39.80681936)
\curveto(910.39631568,39.7468163)(910.39131569,39.69181635)(910.37131592,39.64181936)
\curveto(910.34131574,39.51181653)(910.32131576,39.38681666)(910.31131592,39.26681936)
\curveto(910.30131578,39.1468169)(910.2763158,39.03181701)(910.23631592,38.92181936)
\curveto(910.11631596,38.55181749)(909.94631613,38.23181781)(909.72631592,37.96181936)
\curveto(909.50631657,37.69181835)(909.22631685,37.48181856)(908.88631592,37.33181936)
\curveto(908.76631731,37.28181876)(908.64131744,37.23681881)(908.51131592,37.19681936)
\curveto(908.3813177,37.16681888)(908.24631783,37.13181891)(908.10631592,37.09181936)
\curveto(908.05631802,37.08181896)(908.01631806,37.07681897)(907.98631592,37.07681936)
\curveto(907.94631813,37.07681897)(907.90131818,37.07181897)(907.85131592,37.06181936)
\curveto(907.82131826,37.05181899)(907.78631829,37.046819)(907.74631592,37.04681936)
\curveto(907.69631838,37.046819)(907.65631842,37.041819)(907.62631592,37.03181936)
\lineto(907.46131592,37.03181936)
\curveto(907.3813187,37.01181903)(907.2813188,37.00681904)(907.16131592,37.01681936)
\curveto(907.03131905,37.02681902)(906.94131914,37.041819)(906.89131592,37.06181936)
\curveto(906.80131928,37.08181896)(906.73631934,37.13681891)(906.69631592,37.22681936)
\curveto(906.6763194,37.25681879)(906.67131941,37.28681876)(906.68131592,37.31681936)
\curveto(906.6813194,37.3468187)(906.6763194,37.38681866)(906.66631592,37.43681936)
\curveto(906.65631942,37.47681857)(906.65131943,37.51681853)(906.65131592,37.55681936)
\lineto(906.65131592,37.70681936)
\curveto(906.65131943,37.82681822)(906.65631942,37.9468181)(906.66631592,38.06681936)
\curveto(906.66631941,38.19681785)(906.70131938,38.28681776)(906.77131592,38.33681936)
\curveto(906.83131925,38.37681767)(906.89131919,38.39681765)(906.95131592,38.39681936)
\curveto(907.01131907,38.39681765)(907.081319,38.40681764)(907.16131592,38.42681936)
\curveto(907.19131889,38.43681761)(907.22631885,38.43681761)(907.26631592,38.42681936)
\curveto(907.29631878,38.42681762)(907.32131876,38.43181761)(907.34131592,38.44181936)
\lineto(907.55131592,38.44181936)
\curveto(907.60131848,38.46181758)(907.65131843,38.46681758)(907.70131592,38.45681936)
\curveto(907.74131834,38.45681759)(907.78631829,38.46681758)(907.83631592,38.48681936)
\curveto(907.96631811,38.51681753)(908.09131799,38.5468175)(908.21131592,38.57681936)
\curveto(908.32131776,38.60681744)(908.42631765,38.65181739)(908.52631592,38.71181936)
\curveto(908.81631726,38.88181716)(909.02131706,39.15181689)(909.14131592,39.52181936)
\curveto(909.16131692,39.57181647)(909.1763169,39.62181642)(909.18631592,39.67181936)
\curveto(909.18631689,39.73181631)(909.19631688,39.78681626)(909.21631592,39.83681936)
\lineto(909.21631592,39.91181936)
\curveto(909.22631685,39.98181606)(909.23631684,40.07681597)(909.24631592,40.19681936)
\curveto(909.24631683,40.32681572)(909.23631684,40.42681562)(909.21631592,40.49681936)
\curveto(909.19631688,40.56681548)(909.1813169,40.63681541)(909.17131592,40.70681936)
\curveto(909.15131693,40.78681526)(909.13131695,40.85681519)(909.11131592,40.91681936)
\curveto(908.95131713,41.29681475)(908.6763174,41.57181447)(908.28631592,41.74181936)
\curveto(908.15631792,41.79181425)(908.00131808,41.82681422)(907.82131592,41.84681936)
\curveto(907.64131844,41.87681417)(907.45631862,41.89181415)(907.26631592,41.89181936)
\curveto(907.06631901,41.90181414)(906.86631921,41.90181414)(906.66631592,41.89181936)
\lineto(906.09631592,41.89181936)
\lineto(901.85131592,41.89181936)
\lineto(900.30631592,41.89181936)
\curveto(900.19632588,41.89181415)(900.076326,41.88681416)(899.94631592,41.87681936)
\curveto(899.81632626,41.86681418)(899.71132637,41.88681416)(899.63131592,41.93681936)
\curveto(899.56132652,41.99681405)(899.51132657,42.07681397)(899.48131592,42.17681936)
\curveto(899.4813266,42.19681385)(899.4813266,42.21681383)(899.48131592,42.23681936)
\curveto(899.4813266,42.25681379)(899.4763266,42.27681377)(899.46631592,42.29681936)
}
}
{
\newrgbcolor{curcolor}{0 0 0}
\pscustom[linestyle=none,fillstyle=solid,fillcolor=curcolor]
{
\newpath
\moveto(902.42131592,45.83049123)
\lineto(902.42131592,46.26549123)
\curveto(902.42132366,46.41548927)(902.46132362,46.52048916)(902.54131592,46.58049123)
\curveto(902.62132346,46.63048905)(902.72132336,46.65548903)(902.84131592,46.65549123)
\curveto(902.96132312,46.66548902)(903.081323,46.67048901)(903.20131592,46.67049123)
\lineto(904.62631592,46.67049123)
\lineto(906.89131592,46.67049123)
\lineto(907.58131592,46.67049123)
\curveto(907.81131827,46.67048901)(908.01131807,46.69548899)(908.18131592,46.74549123)
\curveto(908.63131745,46.90548878)(908.94631713,47.20548848)(909.12631592,47.64549123)
\curveto(909.21631686,47.86548782)(909.25131683,48.13048755)(909.23131592,48.44049123)
\curveto(909.20131688,48.75048693)(909.14631693,49.00048668)(909.06631592,49.19049123)
\curveto(908.92631715,49.52048616)(908.75131733,49.7804859)(908.54131592,49.97049123)
\curveto(908.32131776,50.17048551)(908.03631804,50.32548536)(907.68631592,50.43549123)
\curveto(907.60631847,50.46548522)(907.52631855,50.4854852)(907.44631592,50.49549123)
\curveto(907.36631871,50.50548518)(907.2813188,50.52048516)(907.19131592,50.54049123)
\curveto(907.14131894,50.55048513)(907.09631898,50.55048513)(907.05631592,50.54049123)
\curveto(907.01631906,50.54048514)(906.97131911,50.55048513)(906.92131592,50.57049123)
\lineto(906.60631592,50.57049123)
\curveto(906.52631955,50.59048509)(906.43631964,50.59548509)(906.33631592,50.58549123)
\curveto(906.22631985,50.57548511)(906.12631995,50.57048511)(906.03631592,50.57049123)
\lineto(904.86631592,50.57049123)
\lineto(903.27631592,50.57049123)
\curveto(903.15632292,50.57048511)(903.03132305,50.56548512)(902.90131592,50.55549123)
\curveto(902.76132332,50.55548513)(902.65132343,50.5804851)(902.57131592,50.63049123)
\curveto(902.52132356,50.67048501)(902.49132359,50.71548497)(902.48131592,50.76549123)
\curveto(902.46132362,50.82548486)(902.44132364,50.89548479)(902.42131592,50.97549123)
\lineto(902.42131592,51.20049123)
\curveto(902.42132366,51.32048436)(902.42632365,51.42548426)(902.43631592,51.51549123)
\curveto(902.44632363,51.61548407)(902.49132359,51.69048399)(902.57131592,51.74049123)
\curveto(902.62132346,51.79048389)(902.69632338,51.81548387)(902.79631592,51.81549123)
\lineto(903.08131592,51.81549123)
\lineto(904.10131592,51.81549123)
\lineto(908.13631592,51.81549123)
\lineto(909.48631592,51.81549123)
\curveto(909.60631647,51.81548387)(909.72131636,51.81048387)(909.83131592,51.80049123)
\curveto(909.93131615,51.80048388)(910.00631607,51.76548392)(910.05631592,51.69549123)
\curveto(910.08631599,51.65548403)(910.11131597,51.59548409)(910.13131592,51.51549123)
\curveto(910.14131594,51.43548425)(910.15131593,51.34548434)(910.16131592,51.24549123)
\curveto(910.16131592,51.15548453)(910.15631592,51.06548462)(910.14631592,50.97549123)
\curveto(910.13631594,50.89548479)(910.11631596,50.83548485)(910.08631592,50.79549123)
\curveto(910.04631603,50.74548494)(909.9813161,50.70048498)(909.89131592,50.66049123)
\curveto(909.85131623,50.65048503)(909.79631628,50.64048504)(909.72631592,50.63049123)
\curveto(909.65631642,50.63048505)(909.59131649,50.62548506)(909.53131592,50.61549123)
\curveto(909.46131662,50.60548508)(909.40631667,50.5854851)(909.36631592,50.55549123)
\curveto(909.32631675,50.52548516)(909.31131677,50.4804852)(909.32131592,50.42049123)
\curveto(909.34131674,50.34048534)(909.40131668,50.26048542)(909.50131592,50.18049123)
\curveto(909.59131649,50.10048558)(909.66131642,50.02548566)(909.71131592,49.95549123)
\curveto(909.87131621,49.73548595)(910.01131607,49.4854862)(910.13131592,49.20549123)
\curveto(910.1813159,49.09548659)(910.21131587,48.9804867)(910.22131592,48.86049123)
\curveto(910.24131584,48.75048693)(910.26631581,48.63548705)(910.29631592,48.51549123)
\curveto(910.30631577,48.46548722)(910.30631577,48.41048727)(910.29631592,48.35049123)
\curveto(910.28631579,48.30048738)(910.29131579,48.25048743)(910.31131592,48.20049123)
\curveto(910.33131575,48.10048758)(910.33131575,48.01048767)(910.31131592,47.93049123)
\lineto(910.31131592,47.78049123)
\curveto(910.29131579,47.73048795)(910.2813158,47.67048801)(910.28131592,47.60049123)
\curveto(910.2813158,47.54048814)(910.2763158,47.4854882)(910.26631592,47.43549123)
\curveto(910.24631583,47.39548829)(910.23631584,47.35548833)(910.23631592,47.31549123)
\curveto(910.24631583,47.2854884)(910.24131584,47.24548844)(910.22131592,47.19549123)
\lineto(910.16131592,46.95549123)
\curveto(910.14131594,46.8854888)(910.11131597,46.81048887)(910.07131592,46.73049123)
\curveto(909.96131612,46.47048921)(909.81631626,46.25048943)(909.63631592,46.07049123)
\curveto(909.44631663,45.90048978)(909.22131686,45.76048992)(908.96131592,45.65049123)
\curveto(908.87131721,45.61049007)(908.7813173,45.5804901)(908.69131592,45.56049123)
\lineto(908.39131592,45.50049123)
\curveto(908.33131775,45.4804902)(908.2763178,45.47049021)(908.22631592,45.47049123)
\curveto(908.16631791,45.4804902)(908.10131798,45.47549021)(908.03131592,45.45549123)
\curveto(908.01131807,45.44549024)(907.98631809,45.44049024)(907.95631592,45.44049123)
\curveto(907.91631816,45.44049024)(907.8813182,45.43549025)(907.85131592,45.42549123)
\lineto(907.70131592,45.42549123)
\curveto(907.66131842,45.41549027)(907.61631846,45.41049027)(907.56631592,45.41049123)
\curveto(907.50631857,45.42049026)(907.45131863,45.42549026)(907.40131592,45.42549123)
\lineto(906.80131592,45.42549123)
\lineto(904.04131592,45.42549123)
\lineto(903.08131592,45.42549123)
\lineto(902.81131592,45.42549123)
\curveto(902.72132336,45.42549026)(902.64632343,45.44549024)(902.58631592,45.48549123)
\curveto(902.51632356,45.52549016)(902.46632361,45.60049008)(902.43631592,45.71049123)
\curveto(902.42632365,45.73048995)(902.42632365,45.75048993)(902.43631592,45.77049123)
\curveto(902.43632364,45.79048989)(902.43132365,45.81048987)(902.42131592,45.83049123)
}
}
{
\newrgbcolor{curcolor}{0 0 0}
\pscustom[linestyle=none,fillstyle=solid,fillcolor=curcolor]
{
\newpath
\moveto(899.46631592,54.28510061)
\curveto(899.46632661,54.41509899)(899.46632661,54.55009886)(899.46631592,54.69010061)
\curveto(899.46632661,54.84009857)(899.50132658,54.95009846)(899.57131592,55.02010061)
\curveto(899.64132644,55.07009834)(899.73632634,55.09509831)(899.85631592,55.09510061)
\curveto(899.96632611,55.1050983)(900.081326,55.1100983)(900.20131592,55.11010061)
\lineto(901.53631592,55.11010061)
\lineto(907.61131592,55.11010061)
\lineto(909.29131592,55.11010061)
\lineto(909.68131592,55.11010061)
\curveto(909.82131626,55.1100983)(909.93131615,55.08509832)(910.01131592,55.03510061)
\curveto(910.06131602,55.0050984)(910.09131599,54.96009845)(910.10131592,54.90010061)
\curveto(910.11131597,54.85009856)(910.12631595,54.78509862)(910.14631592,54.70510061)
\lineto(910.14631592,54.49510061)
\lineto(910.14631592,54.18010061)
\curveto(910.13631594,54.08009933)(910.10131598,54.0050994)(910.04131592,53.95510061)
\curveto(909.96131612,53.9050995)(909.86131622,53.87509953)(909.74131592,53.86510061)
\lineto(909.36631592,53.86510061)
\lineto(907.98631592,53.86510061)
\lineto(901.74631592,53.86510061)
\lineto(900.27631592,53.86510061)
\curveto(900.16632591,53.86509954)(900.05132603,53.86009955)(899.93131592,53.85010061)
\curveto(899.80132628,53.85009956)(899.70132638,53.87509953)(899.63131592,53.92510061)
\curveto(899.57132651,53.96509944)(899.52132656,54.04009937)(899.48131592,54.15010061)
\curveto(899.47132661,54.17009924)(899.47132661,54.19009922)(899.48131592,54.21010061)
\curveto(899.4813266,54.24009917)(899.4763266,54.26509914)(899.46631592,54.28510061)
}
}
{
\newrgbcolor{curcolor}{0 0 0}
\pscustom[linestyle=none,fillstyle=solid,fillcolor=curcolor]
{
}
}
{
\newrgbcolor{curcolor}{0 0 0}
\pscustom[linestyle=none,fillstyle=solid,fillcolor=curcolor]
{
\newpath
\moveto(899.54131592,64.24510061)
\curveto(899.53132655,64.93509597)(899.65132643,65.53509537)(899.90131592,66.04510061)
\curveto(900.15132593,66.56509434)(900.48632559,66.96009395)(900.90631592,67.23010061)
\curveto(900.98632509,67.28009363)(901.076325,67.32509358)(901.17631592,67.36510061)
\curveto(901.26632481,67.4050935)(901.36132472,67.45009346)(901.46131592,67.50010061)
\curveto(901.56132452,67.54009337)(901.66132442,67.57009334)(901.76131592,67.59010061)
\curveto(901.86132422,67.6100933)(901.96632411,67.63009328)(902.07631592,67.65010061)
\curveto(902.12632395,67.67009324)(902.17132391,67.67509323)(902.21131592,67.66510061)
\curveto(902.25132383,67.65509325)(902.29632378,67.66009325)(902.34631592,67.68010061)
\curveto(902.39632368,67.69009322)(902.4813236,67.69509321)(902.60131592,67.69510061)
\curveto(902.71132337,67.69509321)(902.79632328,67.69009322)(902.85631592,67.68010061)
\curveto(902.91632316,67.66009325)(902.9763231,67.65009326)(903.03631592,67.65010061)
\curveto(903.09632298,67.66009325)(903.15632292,67.65509325)(903.21631592,67.63510061)
\curveto(903.35632272,67.59509331)(903.49132259,67.56009335)(903.62131592,67.53010061)
\curveto(903.75132233,67.50009341)(903.8763222,67.46009345)(903.99631592,67.41010061)
\curveto(904.13632194,67.35009356)(904.26132182,67.28009363)(904.37131592,67.20010061)
\curveto(904.4813216,67.13009378)(904.59132149,67.05509385)(904.70131592,66.97510061)
\lineto(904.76131592,66.91510061)
\curveto(904.7813213,66.905094)(904.80132128,66.89009402)(904.82131592,66.87010061)
\curveto(904.9813211,66.75009416)(905.12632095,66.61509429)(905.25631592,66.46510061)
\curveto(905.38632069,66.31509459)(905.51132057,66.15509475)(905.63131592,65.98510061)
\curveto(905.85132023,65.67509523)(906.05632002,65.38009553)(906.24631592,65.10010061)
\curveto(906.38631969,64.87009604)(906.52131956,64.64009627)(906.65131592,64.41010061)
\curveto(906.7813193,64.19009672)(906.91631916,63.97009694)(907.05631592,63.75010061)
\curveto(907.22631885,63.50009741)(907.40631867,63.26009765)(907.59631592,63.03010061)
\curveto(907.78631829,62.8100981)(908.01131807,62.62009829)(908.27131592,62.46010061)
\curveto(908.33131775,62.42009849)(908.39131769,62.38509852)(908.45131592,62.35510061)
\curveto(908.50131758,62.32509858)(908.56631751,62.29509861)(908.64631592,62.26510061)
\curveto(908.71631736,62.24509866)(908.7763173,62.24009867)(908.82631592,62.25010061)
\curveto(908.89631718,62.27009864)(908.95131713,62.3050986)(908.99131592,62.35510061)
\curveto(909.02131706,62.4050985)(909.04131704,62.46509844)(909.05131592,62.53510061)
\lineto(909.05131592,62.77510061)
\lineto(909.05131592,63.52510061)
\lineto(909.05131592,66.33010061)
\lineto(909.05131592,66.99010061)
\curveto(909.05131703,67.08009383)(909.05631702,67.16509374)(909.06631592,67.24510061)
\curveto(909.06631701,67.32509358)(909.08631699,67.39009352)(909.12631592,67.44010061)
\curveto(909.16631691,67.49009342)(909.24131684,67.53009338)(909.35131592,67.56010061)
\curveto(909.45131663,67.60009331)(909.55131653,67.6100933)(909.65131592,67.59010061)
\lineto(909.78631592,67.59010061)
\curveto(909.85631622,67.57009334)(909.91631616,67.55009336)(909.96631592,67.53010061)
\curveto(910.01631606,67.5100934)(910.05631602,67.47509343)(910.08631592,67.42510061)
\curveto(910.12631595,67.37509353)(910.14631593,67.3050936)(910.14631592,67.21510061)
\lineto(910.14631592,66.94510061)
\lineto(910.14631592,66.04510061)
\lineto(910.14631592,62.53510061)
\lineto(910.14631592,61.47010061)
\curveto(910.14631593,61.39009952)(910.15131593,61.30009961)(910.16131592,61.20010061)
\curveto(910.16131592,61.10009981)(910.15131593,61.01509989)(910.13131592,60.94510061)
\curveto(910.06131602,60.73510017)(909.8813162,60.67010024)(909.59131592,60.75010061)
\curveto(909.55131653,60.76010015)(909.51631656,60.76010015)(909.48631592,60.75010061)
\curveto(909.44631663,60.75010016)(909.40131668,60.76010015)(909.35131592,60.78010061)
\curveto(909.27131681,60.80010011)(909.18631689,60.82010009)(909.09631592,60.84010061)
\curveto(909.00631707,60.86010005)(908.92131716,60.88510002)(908.84131592,60.91510061)
\curveto(908.35131773,61.07509983)(907.93631814,61.27509963)(907.59631592,61.51510061)
\curveto(907.34631873,61.69509921)(907.12131896,61.90009901)(906.92131592,62.13010061)
\curveto(906.71131937,62.36009855)(906.51631956,62.60009831)(906.33631592,62.85010061)
\curveto(906.15631992,63.1100978)(905.98632009,63.37509753)(905.82631592,63.64510061)
\curveto(905.65632042,63.92509698)(905.4813206,64.19509671)(905.30131592,64.45510061)
\curveto(905.22132086,64.56509634)(905.14632093,64.67009624)(905.07631592,64.77010061)
\curveto(905.00632107,64.88009603)(904.93132115,64.99009592)(904.85131592,65.10010061)
\curveto(904.82132126,65.14009577)(904.79132129,65.17509573)(904.76131592,65.20510061)
\curveto(904.72132136,65.24509566)(904.69132139,65.28509562)(904.67131592,65.32510061)
\curveto(904.56132152,65.46509544)(904.43632164,65.59009532)(904.29631592,65.70010061)
\curveto(904.26632181,65.72009519)(904.24132184,65.74509516)(904.22131592,65.77510061)
\curveto(904.19132189,65.8050951)(904.16132192,65.83009508)(904.13131592,65.85010061)
\curveto(904.03132205,65.93009498)(903.93132215,65.99509491)(903.83131592,66.04510061)
\curveto(903.73132235,66.1050948)(903.62132246,66.16009475)(903.50131592,66.21010061)
\curveto(903.43132265,66.24009467)(903.35632272,66.26009465)(903.27631592,66.27010061)
\lineto(903.03631592,66.33010061)
\lineto(902.94631592,66.33010061)
\curveto(902.91632316,66.34009457)(902.88632319,66.34509456)(902.85631592,66.34510061)
\curveto(902.78632329,66.36509454)(902.69132339,66.37009454)(902.57131592,66.36010061)
\curveto(902.44132364,66.36009455)(902.34132374,66.35009456)(902.27131592,66.33010061)
\curveto(902.19132389,66.3100946)(902.11632396,66.29009462)(902.04631592,66.27010061)
\curveto(901.96632411,66.26009465)(901.88632419,66.24009467)(901.80631592,66.21010061)
\curveto(901.56632451,66.10009481)(901.36632471,65.95009496)(901.20631592,65.76010061)
\curveto(901.03632504,65.58009533)(900.89632518,65.36009555)(900.78631592,65.10010061)
\curveto(900.76632531,65.03009588)(900.75132533,64.96009595)(900.74131592,64.89010061)
\curveto(900.72132536,64.82009609)(900.70132538,64.74509616)(900.68131592,64.66510061)
\curveto(900.66132542,64.58509632)(900.65132543,64.47509643)(900.65131592,64.33510061)
\curveto(900.65132543,64.2050967)(900.66132542,64.10009681)(900.68131592,64.02010061)
\curveto(900.69132539,63.96009695)(900.69632538,63.905097)(900.69631592,63.85510061)
\curveto(900.69632538,63.8050971)(900.70632537,63.75509715)(900.72631592,63.70510061)
\curveto(900.76632531,63.6050973)(900.80632527,63.5100974)(900.84631592,63.42010061)
\curveto(900.88632519,63.34009757)(900.93132515,63.26009765)(900.98131592,63.18010061)
\curveto(901.00132508,63.15009776)(901.02632505,63.12009779)(901.05631592,63.09010061)
\curveto(901.08632499,63.07009784)(901.11132497,63.04509786)(901.13131592,63.01510061)
\lineto(901.20631592,62.94010061)
\curveto(901.22632485,62.910098)(901.24632483,62.88509802)(901.26631592,62.86510061)
\lineto(901.47631592,62.71510061)
\curveto(901.53632454,62.67509823)(901.60132448,62.63009828)(901.67131592,62.58010061)
\curveto(901.76132432,62.52009839)(901.86632421,62.47009844)(901.98631592,62.43010061)
\curveto(902.09632398,62.40009851)(902.20632387,62.36509854)(902.31631592,62.32510061)
\curveto(902.42632365,62.28509862)(902.57132351,62.26009865)(902.75131592,62.25010061)
\curveto(902.92132316,62.24009867)(903.04632303,62.2100987)(903.12631592,62.16010061)
\curveto(903.20632287,62.1100988)(903.25132283,62.03509887)(903.26131592,61.93510061)
\curveto(903.27132281,61.83509907)(903.2763228,61.72509918)(903.27631592,61.60510061)
\curveto(903.2763228,61.56509934)(903.2813228,61.52509938)(903.29131592,61.48510061)
\curveto(903.29132279,61.44509946)(903.28632279,61.4100995)(903.27631592,61.38010061)
\curveto(903.25632282,61.33009958)(903.24632283,61.28009963)(903.24631592,61.23010061)
\curveto(903.24632283,61.19009972)(903.23632284,61.15009976)(903.21631592,61.11010061)
\curveto(903.15632292,61.02009989)(903.02132306,60.97509993)(902.81131592,60.97510061)
\lineto(902.69131592,60.97510061)
\curveto(902.63132345,60.98509992)(902.57132351,60.99009992)(902.51131592,60.99010061)
\curveto(902.44132364,61.00009991)(902.3763237,61.0100999)(902.31631592,61.02010061)
\curveto(902.20632387,61.04009987)(902.10632397,61.06009985)(902.01631592,61.08010061)
\curveto(901.91632416,61.10009981)(901.82132426,61.13009978)(901.73131592,61.17010061)
\curveto(901.66132442,61.19009972)(901.60132448,61.2100997)(901.55131592,61.23010061)
\lineto(901.37131592,61.29010061)
\curveto(901.11132497,61.4100995)(900.86632521,61.56509934)(900.63631592,61.75510061)
\curveto(900.40632567,61.95509895)(900.22132586,62.17009874)(900.08131592,62.40010061)
\curveto(900.00132608,62.5100984)(899.93632614,62.62509828)(899.88631592,62.74510061)
\lineto(899.73631592,63.13510061)
\curveto(899.68632639,63.24509766)(899.65632642,63.36009755)(899.64631592,63.48010061)
\curveto(899.62632645,63.60009731)(899.60132648,63.72509718)(899.57131592,63.85510061)
\curveto(899.57132651,63.92509698)(899.57132651,63.99009692)(899.57131592,64.05010061)
\curveto(899.56132652,64.1100968)(899.55132653,64.17509673)(899.54131592,64.24510061)
}
}
{
\newrgbcolor{curcolor}{0 0 0}
\pscustom[linestyle=none,fillstyle=solid,fillcolor=curcolor]
{
\newpath
\moveto(905.06131592,76.34470998)
\lineto(905.31631592,76.34470998)
\curveto(905.39632068,76.35470228)(905.47132061,76.34970228)(905.54131592,76.32970998)
\lineto(905.78131592,76.32970998)
\lineto(905.94631592,76.32970998)
\curveto(906.04632003,76.30970232)(906.15131993,76.29970233)(906.26131592,76.29970998)
\curveto(906.36131972,76.29970233)(906.46131962,76.28970234)(906.56131592,76.26970998)
\lineto(906.71131592,76.26970998)
\curveto(906.85131923,76.23970239)(906.99131909,76.21970241)(907.13131592,76.20970998)
\curveto(907.26131882,76.19970243)(907.39131869,76.17470246)(907.52131592,76.13470998)
\curveto(907.60131848,76.11470252)(907.68631839,76.09470254)(907.77631592,76.07470998)
\lineto(908.01631592,76.01470998)
\lineto(908.31631592,75.89470998)
\curveto(908.40631767,75.86470277)(908.49631758,75.8297028)(908.58631592,75.78970998)
\curveto(908.80631727,75.68970294)(909.02131706,75.55470308)(909.23131592,75.38470998)
\curveto(909.44131664,75.22470341)(909.61131647,75.04970358)(909.74131592,74.85970998)
\curveto(909.7813163,74.80970382)(909.82131626,74.74970388)(909.86131592,74.67970998)
\curveto(909.89131619,74.61970401)(909.92631615,74.55970407)(909.96631592,74.49970998)
\curveto(910.01631606,74.41970421)(910.05631602,74.32470431)(910.08631592,74.21470998)
\curveto(910.11631596,74.10470453)(910.14631593,73.99970463)(910.17631592,73.89970998)
\curveto(910.21631586,73.78970484)(910.24131584,73.67970495)(910.25131592,73.56970998)
\curveto(910.26131582,73.45970517)(910.2763158,73.34470529)(910.29631592,73.22470998)
\curveto(910.30631577,73.18470545)(910.30631577,73.13970549)(910.29631592,73.08970998)
\curveto(910.29631578,73.04970558)(910.30131578,73.00970562)(910.31131592,72.96970998)
\curveto(910.32131576,72.9297057)(910.32631575,72.87470576)(910.32631592,72.80470998)
\curveto(910.32631575,72.7347059)(910.32131576,72.68470595)(910.31131592,72.65470998)
\curveto(910.29131579,72.60470603)(910.28631579,72.55970607)(910.29631592,72.51970998)
\curveto(910.30631577,72.47970615)(910.30631577,72.44470619)(910.29631592,72.41470998)
\lineto(910.29631592,72.32470998)
\curveto(910.2763158,72.26470637)(910.26131582,72.19970643)(910.25131592,72.12970998)
\curveto(910.25131583,72.06970656)(910.24631583,72.00470663)(910.23631592,71.93470998)
\curveto(910.18631589,71.76470687)(910.13631594,71.60470703)(910.08631592,71.45470998)
\curveto(910.03631604,71.30470733)(909.97131611,71.15970747)(909.89131592,71.01970998)
\curveto(909.85131623,70.96970766)(909.82131626,70.91470772)(909.80131592,70.85470998)
\curveto(909.77131631,70.80470783)(909.73631634,70.75470788)(909.69631592,70.70470998)
\curveto(909.51631656,70.46470817)(909.29631678,70.26470837)(909.03631592,70.10470998)
\curveto(908.7763173,69.94470869)(908.49131759,69.80470883)(908.18131592,69.68470998)
\curveto(908.04131804,69.62470901)(907.90131818,69.57970905)(907.76131592,69.54970998)
\curveto(907.61131847,69.51970911)(907.45631862,69.48470915)(907.29631592,69.44470998)
\curveto(907.18631889,69.42470921)(907.076319,69.40970922)(906.96631592,69.39970998)
\curveto(906.85631922,69.38970924)(906.74631933,69.37470926)(906.63631592,69.35470998)
\curveto(906.59631948,69.34470929)(906.55631952,69.33970929)(906.51631592,69.33970998)
\curveto(906.4763196,69.34970928)(906.43631964,69.34970928)(906.39631592,69.33970998)
\curveto(906.34631973,69.3297093)(906.29631978,69.32470931)(906.24631592,69.32470998)
\lineto(906.08131592,69.32470998)
\curveto(906.03132005,69.30470933)(905.9813201,69.29970933)(905.93131592,69.30970998)
\curveto(905.87132021,69.31970931)(905.81632026,69.31970931)(905.76631592,69.30970998)
\curveto(905.72632035,69.29970933)(905.6813204,69.29970933)(905.63131592,69.30970998)
\curveto(905.5813205,69.31970931)(905.53132055,69.31470932)(905.48131592,69.29470998)
\curveto(905.41132067,69.27470936)(905.33632074,69.26970936)(905.25631592,69.27970998)
\curveto(905.16632091,69.28970934)(905.081321,69.29470934)(905.00131592,69.29470998)
\curveto(904.91132117,69.29470934)(904.81132127,69.28970934)(904.70131592,69.27970998)
\curveto(904.5813215,69.26970936)(904.4813216,69.27470936)(904.40131592,69.29470998)
\lineto(904.11631592,69.29470998)
\lineto(903.48631592,69.33970998)
\curveto(903.38632269,69.34970928)(903.29132279,69.35970927)(903.20131592,69.36970998)
\lineto(902.90131592,69.39970998)
\curveto(902.85132323,69.41970921)(902.80132328,69.42470921)(902.75131592,69.41470998)
\curveto(902.69132339,69.41470922)(902.63632344,69.42470921)(902.58631592,69.44470998)
\curveto(902.41632366,69.49470914)(902.25132383,69.5347091)(902.09131592,69.56470998)
\curveto(901.92132416,69.59470904)(901.76132432,69.64470899)(901.61131592,69.71470998)
\curveto(901.15132493,69.90470873)(900.7763253,70.12470851)(900.48631592,70.37470998)
\curveto(900.19632588,70.634708)(899.95132613,70.99470764)(899.75131592,71.45470998)
\curveto(899.70132638,71.58470705)(899.66632641,71.71470692)(899.64631592,71.84470998)
\curveto(899.62632645,71.98470665)(899.60132648,72.12470651)(899.57131592,72.26470998)
\curveto(899.56132652,72.3347063)(899.55632652,72.39970623)(899.55631592,72.45970998)
\curveto(899.55632652,72.51970611)(899.55132653,72.58470605)(899.54131592,72.65470998)
\curveto(899.52132656,73.48470515)(899.67132641,74.15470448)(899.99131592,74.66470998)
\curveto(900.30132578,75.17470346)(900.74132534,75.55470308)(901.31131592,75.80470998)
\curveto(901.43132465,75.85470278)(901.55632452,75.89970273)(901.68631592,75.93970998)
\curveto(901.81632426,75.97970265)(901.95132413,76.02470261)(902.09131592,76.07470998)
\curveto(902.17132391,76.09470254)(902.25632382,76.10970252)(902.34631592,76.11970998)
\lineto(902.58631592,76.17970998)
\curveto(902.69632338,76.20970242)(902.80632327,76.22470241)(902.91631592,76.22470998)
\curveto(903.02632305,76.2347024)(903.13632294,76.24970238)(903.24631592,76.26970998)
\curveto(903.29632278,76.28970234)(903.34132274,76.29470234)(903.38131592,76.28470998)
\curveto(903.42132266,76.28470235)(903.46132262,76.28970234)(903.50131592,76.29970998)
\curveto(903.55132253,76.30970232)(903.60632247,76.30970232)(903.66631592,76.29970998)
\curveto(903.71632236,76.29970233)(903.76632231,76.30470233)(903.81631592,76.31470998)
\lineto(903.95131592,76.31470998)
\curveto(904.01132207,76.3347023)(904.081322,76.3347023)(904.16131592,76.31470998)
\curveto(904.23132185,76.30470233)(904.29632178,76.30970232)(904.35631592,76.32970998)
\curveto(904.38632169,76.33970229)(904.42632165,76.34470229)(904.47631592,76.34470998)
\lineto(904.59631592,76.34470998)
\lineto(905.06131592,76.34470998)
\moveto(907.38631592,74.79970998)
\curveto(907.06631901,74.89970373)(906.70131938,74.95970367)(906.29131592,74.97970998)
\curveto(905.8813202,74.99970363)(905.47132061,75.00970362)(905.06131592,75.00970998)
\curveto(904.63132145,75.00970362)(904.21132187,74.99970363)(903.80131592,74.97970998)
\curveto(903.39132269,74.95970367)(903.00632307,74.91470372)(902.64631592,74.84470998)
\curveto(902.28632379,74.77470386)(901.96632411,74.66470397)(901.68631592,74.51470998)
\curveto(901.39632468,74.37470426)(901.16132492,74.17970445)(900.98131592,73.92970998)
\curveto(900.87132521,73.76970486)(900.79132529,73.58970504)(900.74131592,73.38970998)
\curveto(900.6813254,73.18970544)(900.65132543,72.94470569)(900.65131592,72.65470998)
\curveto(900.67132541,72.634706)(900.6813254,72.59970603)(900.68131592,72.54970998)
\curveto(900.67132541,72.49970613)(900.67132541,72.45970617)(900.68131592,72.42970998)
\curveto(900.70132538,72.34970628)(900.72132536,72.27470636)(900.74131592,72.20470998)
\curveto(900.75132533,72.14470649)(900.77132531,72.07970655)(900.80131592,72.00970998)
\curveto(900.92132516,71.73970689)(901.09132499,71.51970711)(901.31131592,71.34970998)
\curveto(901.52132456,71.18970744)(901.76632431,71.05470758)(902.04631592,70.94470998)
\curveto(902.15632392,70.89470774)(902.2763238,70.85470778)(902.40631592,70.82470998)
\curveto(902.52632355,70.80470783)(902.65132343,70.77970785)(902.78131592,70.74970998)
\curveto(902.83132325,70.7297079)(902.88632319,70.71970791)(902.94631592,70.71970998)
\curveto(902.99632308,70.71970791)(903.04632303,70.71470792)(903.09631592,70.70470998)
\curveto(903.18632289,70.69470794)(903.2813228,70.68470795)(903.38131592,70.67470998)
\curveto(903.47132261,70.66470797)(903.56632251,70.65470798)(903.66631592,70.64470998)
\curveto(903.74632233,70.64470799)(903.83132225,70.63970799)(903.92131592,70.62970998)
\lineto(904.16131592,70.62970998)
\lineto(904.34131592,70.62970998)
\curveto(904.37132171,70.61970801)(904.40632167,70.61470802)(904.44631592,70.61470998)
\lineto(904.58131592,70.61470998)
\lineto(905.03131592,70.61470998)
\curveto(905.11132097,70.61470802)(905.19632088,70.60970802)(905.28631592,70.59970998)
\curveto(905.36632071,70.59970803)(905.44132064,70.60970802)(905.51131592,70.62970998)
\lineto(905.78131592,70.62970998)
\curveto(905.80132028,70.629708)(905.83132025,70.62470801)(905.87131592,70.61470998)
\curveto(905.90132018,70.61470802)(905.92632015,70.61970801)(905.94631592,70.62970998)
\curveto(906.04632003,70.63970799)(906.14631993,70.64470799)(906.24631592,70.64470998)
\curveto(906.33631974,70.65470798)(906.43631964,70.66470797)(906.54631592,70.67470998)
\curveto(906.66631941,70.70470793)(906.79131929,70.71970791)(906.92131592,70.71970998)
\curveto(907.04131904,70.7297079)(907.15631892,70.75470788)(907.26631592,70.79470998)
\curveto(907.56631851,70.87470776)(907.83131825,70.95970767)(908.06131592,71.04970998)
\curveto(908.29131779,71.14970748)(908.50631757,71.29470734)(908.70631592,71.48470998)
\curveto(908.90631717,71.69470694)(909.05631702,71.95970667)(909.15631592,72.27970998)
\curveto(909.1763169,72.31970631)(909.18631689,72.35470628)(909.18631592,72.38470998)
\curveto(909.1763169,72.42470621)(909.1813169,72.46970616)(909.20131592,72.51970998)
\curveto(909.21131687,72.55970607)(909.22131686,72.629706)(909.23131592,72.72970998)
\curveto(909.24131684,72.83970579)(909.23631684,72.92470571)(909.21631592,72.98470998)
\curveto(909.19631688,73.05470558)(909.18631689,73.12470551)(909.18631592,73.19470998)
\curveto(909.1763169,73.26470537)(909.16131692,73.3297053)(909.14131592,73.38970998)
\curveto(909.081317,73.58970504)(908.99631708,73.76970486)(908.88631592,73.92970998)
\curveto(908.86631721,73.95970467)(908.84631723,73.98470465)(908.82631592,74.00470998)
\lineto(908.76631592,74.06470998)
\curveto(908.74631733,74.10470453)(908.70631737,74.15470448)(908.64631592,74.21470998)
\curveto(908.50631757,74.31470432)(908.3763177,74.39970423)(908.25631592,74.46970998)
\curveto(908.13631794,74.53970409)(907.99131809,74.60970402)(907.82131592,74.67970998)
\curveto(907.75131833,74.70970392)(907.6813184,74.7297039)(907.61131592,74.73970998)
\curveto(907.54131854,74.75970387)(907.46631861,74.77970385)(907.38631592,74.79970998)
}
}
{
\newrgbcolor{curcolor}{0 0 0}
\pscustom[linestyle=none,fillstyle=solid,fillcolor=curcolor]
{
\newpath
\moveto(908.51131592,78.63431936)
\lineto(908.51131592,79.26431936)
\lineto(908.51131592,79.45931936)
\curveto(908.51131757,79.52931683)(908.52131756,79.58931677)(908.54131592,79.63931936)
\curveto(908.5813175,79.70931665)(908.62131746,79.7593166)(908.66131592,79.78931936)
\curveto(908.71131737,79.82931653)(908.7763173,79.84931651)(908.85631592,79.84931936)
\curveto(908.93631714,79.8593165)(909.02131706,79.86431649)(909.11131592,79.86431936)
\lineto(909.83131592,79.86431936)
\curveto(910.31131577,79.86431649)(910.72131536,79.80431655)(911.06131592,79.68431936)
\curveto(911.40131468,79.56431679)(911.6763144,79.36931699)(911.88631592,79.09931936)
\curveto(911.93631414,79.02931733)(911.9813141,78.9593174)(912.02131592,78.88931936)
\curveto(912.07131401,78.82931753)(912.11631396,78.7543176)(912.15631592,78.66431936)
\curveto(912.16631391,78.64431771)(912.1763139,78.61431774)(912.18631592,78.57431936)
\curveto(912.20631387,78.53431782)(912.21131387,78.48931787)(912.20131592,78.43931936)
\curveto(912.17131391,78.34931801)(912.09631398,78.29431806)(911.97631592,78.27431936)
\curveto(911.86631421,78.2543181)(911.77131431,78.26931809)(911.69131592,78.31931936)
\curveto(911.62131446,78.34931801)(911.55631452,78.39431796)(911.49631592,78.45431936)
\curveto(911.44631463,78.52431783)(911.39631468,78.58931777)(911.34631592,78.64931936)
\curveto(911.29631478,78.71931764)(911.22131486,78.77931758)(911.12131592,78.82931936)
\curveto(911.03131505,78.88931747)(910.94131514,78.93931742)(910.85131592,78.97931936)
\curveto(910.82131526,78.99931736)(910.76131532,79.02431733)(910.67131592,79.05431936)
\curveto(910.59131549,79.08431727)(910.52131556,79.08931727)(910.46131592,79.06931936)
\curveto(910.32131576,79.03931732)(910.23131585,78.97931738)(910.19131592,78.88931936)
\curveto(910.16131592,78.80931755)(910.14631593,78.71931764)(910.14631592,78.61931936)
\curveto(910.14631593,78.51931784)(910.12131596,78.43431792)(910.07131592,78.36431936)
\curveto(910.00131608,78.27431808)(909.86131622,78.22931813)(909.65131592,78.22931936)
\lineto(909.09631592,78.22931936)
\lineto(908.87131592,78.22931936)
\curveto(908.79131729,78.23931812)(908.72631735,78.2593181)(908.67631592,78.28931936)
\curveto(908.59631748,78.34931801)(908.55131753,78.41931794)(908.54131592,78.49931936)
\curveto(908.53131755,78.51931784)(908.52631755,78.53931782)(908.52631592,78.55931936)
\curveto(908.52631755,78.58931777)(908.52131756,78.61431774)(908.51131592,78.63431936)
}
}
{
\newrgbcolor{curcolor}{0 0 0}
\pscustom[linestyle=none,fillstyle=solid,fillcolor=curcolor]
{
}
}
{
\newrgbcolor{curcolor}{0 0 0}
\pscustom[linestyle=none,fillstyle=solid,fillcolor=curcolor]
{
\newpath
\moveto(899.54131592,89.26463186)
\curveto(899.53132655,89.95462722)(899.65132643,90.55462662)(899.90131592,91.06463186)
\curveto(900.15132593,91.58462559)(900.48632559,91.9796252)(900.90631592,92.24963186)
\curveto(900.98632509,92.29962488)(901.076325,92.34462483)(901.17631592,92.38463186)
\curveto(901.26632481,92.42462475)(901.36132472,92.46962471)(901.46131592,92.51963186)
\curveto(901.56132452,92.55962462)(901.66132442,92.58962459)(901.76131592,92.60963186)
\curveto(901.86132422,92.62962455)(901.96632411,92.64962453)(902.07631592,92.66963186)
\curveto(902.12632395,92.68962449)(902.17132391,92.69462448)(902.21131592,92.68463186)
\curveto(902.25132383,92.6746245)(902.29632378,92.6796245)(902.34631592,92.69963186)
\curveto(902.39632368,92.70962447)(902.4813236,92.71462446)(902.60131592,92.71463186)
\curveto(902.71132337,92.71462446)(902.79632328,92.70962447)(902.85631592,92.69963186)
\curveto(902.91632316,92.6796245)(902.9763231,92.66962451)(903.03631592,92.66963186)
\curveto(903.09632298,92.6796245)(903.15632292,92.6746245)(903.21631592,92.65463186)
\curveto(903.35632272,92.61462456)(903.49132259,92.5796246)(903.62131592,92.54963186)
\curveto(903.75132233,92.51962466)(903.8763222,92.4796247)(903.99631592,92.42963186)
\curveto(904.13632194,92.36962481)(904.26132182,92.29962488)(904.37131592,92.21963186)
\curveto(904.4813216,92.14962503)(904.59132149,92.0746251)(904.70131592,91.99463186)
\lineto(904.76131592,91.93463186)
\curveto(904.7813213,91.92462525)(904.80132128,91.90962527)(904.82131592,91.88963186)
\curveto(904.9813211,91.76962541)(905.12632095,91.63462554)(905.25631592,91.48463186)
\curveto(905.38632069,91.33462584)(905.51132057,91.174626)(905.63131592,91.00463186)
\curveto(905.85132023,90.69462648)(906.05632002,90.39962678)(906.24631592,90.11963186)
\curveto(906.38631969,89.88962729)(906.52131956,89.65962752)(906.65131592,89.42963186)
\curveto(906.7813193,89.20962797)(906.91631916,88.98962819)(907.05631592,88.76963186)
\curveto(907.22631885,88.51962866)(907.40631867,88.2796289)(907.59631592,88.04963186)
\curveto(907.78631829,87.82962935)(908.01131807,87.63962954)(908.27131592,87.47963186)
\curveto(908.33131775,87.43962974)(908.39131769,87.40462977)(908.45131592,87.37463186)
\curveto(908.50131758,87.34462983)(908.56631751,87.31462986)(908.64631592,87.28463186)
\curveto(908.71631736,87.26462991)(908.7763173,87.25962992)(908.82631592,87.26963186)
\curveto(908.89631718,87.28962989)(908.95131713,87.32462985)(908.99131592,87.37463186)
\curveto(909.02131706,87.42462975)(909.04131704,87.48462969)(909.05131592,87.55463186)
\lineto(909.05131592,87.79463186)
\lineto(909.05131592,88.54463186)
\lineto(909.05131592,91.34963186)
\lineto(909.05131592,92.00963186)
\curveto(909.05131703,92.09962508)(909.05631702,92.18462499)(909.06631592,92.26463186)
\curveto(909.06631701,92.34462483)(909.08631699,92.40962477)(909.12631592,92.45963186)
\curveto(909.16631691,92.50962467)(909.24131684,92.54962463)(909.35131592,92.57963186)
\curveto(909.45131663,92.61962456)(909.55131653,92.62962455)(909.65131592,92.60963186)
\lineto(909.78631592,92.60963186)
\curveto(909.85631622,92.58962459)(909.91631616,92.56962461)(909.96631592,92.54963186)
\curveto(910.01631606,92.52962465)(910.05631602,92.49462468)(910.08631592,92.44463186)
\curveto(910.12631595,92.39462478)(910.14631593,92.32462485)(910.14631592,92.23463186)
\lineto(910.14631592,91.96463186)
\lineto(910.14631592,91.06463186)
\lineto(910.14631592,87.55463186)
\lineto(910.14631592,86.48963186)
\curveto(910.14631593,86.40963077)(910.15131593,86.31963086)(910.16131592,86.21963186)
\curveto(910.16131592,86.11963106)(910.15131593,86.03463114)(910.13131592,85.96463186)
\curveto(910.06131602,85.75463142)(909.8813162,85.68963149)(909.59131592,85.76963186)
\curveto(909.55131653,85.7796314)(909.51631656,85.7796314)(909.48631592,85.76963186)
\curveto(909.44631663,85.76963141)(909.40131668,85.7796314)(909.35131592,85.79963186)
\curveto(909.27131681,85.81963136)(909.18631689,85.83963134)(909.09631592,85.85963186)
\curveto(909.00631707,85.8796313)(908.92131716,85.90463127)(908.84131592,85.93463186)
\curveto(908.35131773,86.09463108)(907.93631814,86.29463088)(907.59631592,86.53463186)
\curveto(907.34631873,86.71463046)(907.12131896,86.91963026)(906.92131592,87.14963186)
\curveto(906.71131937,87.3796298)(906.51631956,87.61962956)(906.33631592,87.86963186)
\curveto(906.15631992,88.12962905)(905.98632009,88.39462878)(905.82631592,88.66463186)
\curveto(905.65632042,88.94462823)(905.4813206,89.21462796)(905.30131592,89.47463186)
\curveto(905.22132086,89.58462759)(905.14632093,89.68962749)(905.07631592,89.78963186)
\curveto(905.00632107,89.89962728)(904.93132115,90.00962717)(904.85131592,90.11963186)
\curveto(904.82132126,90.15962702)(904.79132129,90.19462698)(904.76131592,90.22463186)
\curveto(904.72132136,90.26462691)(904.69132139,90.30462687)(904.67131592,90.34463186)
\curveto(904.56132152,90.48462669)(904.43632164,90.60962657)(904.29631592,90.71963186)
\curveto(904.26632181,90.73962644)(904.24132184,90.76462641)(904.22131592,90.79463186)
\curveto(904.19132189,90.82462635)(904.16132192,90.84962633)(904.13131592,90.86963186)
\curveto(904.03132205,90.94962623)(903.93132215,91.01462616)(903.83131592,91.06463186)
\curveto(903.73132235,91.12462605)(903.62132246,91.179626)(903.50131592,91.22963186)
\curveto(903.43132265,91.25962592)(903.35632272,91.2796259)(903.27631592,91.28963186)
\lineto(903.03631592,91.34963186)
\lineto(902.94631592,91.34963186)
\curveto(902.91632316,91.35962582)(902.88632319,91.36462581)(902.85631592,91.36463186)
\curveto(902.78632329,91.38462579)(902.69132339,91.38962579)(902.57131592,91.37963186)
\curveto(902.44132364,91.3796258)(902.34132374,91.36962581)(902.27131592,91.34963186)
\curveto(902.19132389,91.32962585)(902.11632396,91.30962587)(902.04631592,91.28963186)
\curveto(901.96632411,91.2796259)(901.88632419,91.25962592)(901.80631592,91.22963186)
\curveto(901.56632451,91.11962606)(901.36632471,90.96962621)(901.20631592,90.77963186)
\curveto(901.03632504,90.59962658)(900.89632518,90.3796268)(900.78631592,90.11963186)
\curveto(900.76632531,90.04962713)(900.75132533,89.9796272)(900.74131592,89.90963186)
\curveto(900.72132536,89.83962734)(900.70132538,89.76462741)(900.68131592,89.68463186)
\curveto(900.66132542,89.60462757)(900.65132543,89.49462768)(900.65131592,89.35463186)
\curveto(900.65132543,89.22462795)(900.66132542,89.11962806)(900.68131592,89.03963186)
\curveto(900.69132539,88.9796282)(900.69632538,88.92462825)(900.69631592,88.87463186)
\curveto(900.69632538,88.82462835)(900.70632537,88.7746284)(900.72631592,88.72463186)
\curveto(900.76632531,88.62462855)(900.80632527,88.52962865)(900.84631592,88.43963186)
\curveto(900.88632519,88.35962882)(900.93132515,88.2796289)(900.98131592,88.19963186)
\curveto(901.00132508,88.16962901)(901.02632505,88.13962904)(901.05631592,88.10963186)
\curveto(901.08632499,88.08962909)(901.11132497,88.06462911)(901.13131592,88.03463186)
\lineto(901.20631592,87.95963186)
\curveto(901.22632485,87.92962925)(901.24632483,87.90462927)(901.26631592,87.88463186)
\lineto(901.47631592,87.73463186)
\curveto(901.53632454,87.69462948)(901.60132448,87.64962953)(901.67131592,87.59963186)
\curveto(901.76132432,87.53962964)(901.86632421,87.48962969)(901.98631592,87.44963186)
\curveto(902.09632398,87.41962976)(902.20632387,87.38462979)(902.31631592,87.34463186)
\curveto(902.42632365,87.30462987)(902.57132351,87.2796299)(902.75131592,87.26963186)
\curveto(902.92132316,87.25962992)(903.04632303,87.22962995)(903.12631592,87.17963186)
\curveto(903.20632287,87.12963005)(903.25132283,87.05463012)(903.26131592,86.95463186)
\curveto(903.27132281,86.85463032)(903.2763228,86.74463043)(903.27631592,86.62463186)
\curveto(903.2763228,86.58463059)(903.2813228,86.54463063)(903.29131592,86.50463186)
\curveto(903.29132279,86.46463071)(903.28632279,86.42963075)(903.27631592,86.39963186)
\curveto(903.25632282,86.34963083)(903.24632283,86.29963088)(903.24631592,86.24963186)
\curveto(903.24632283,86.20963097)(903.23632284,86.16963101)(903.21631592,86.12963186)
\curveto(903.15632292,86.03963114)(903.02132306,85.99463118)(902.81131592,85.99463186)
\lineto(902.69131592,85.99463186)
\curveto(902.63132345,86.00463117)(902.57132351,86.00963117)(902.51131592,86.00963186)
\curveto(902.44132364,86.01963116)(902.3763237,86.02963115)(902.31631592,86.03963186)
\curveto(902.20632387,86.05963112)(902.10632397,86.0796311)(902.01631592,86.09963186)
\curveto(901.91632416,86.11963106)(901.82132426,86.14963103)(901.73131592,86.18963186)
\curveto(901.66132442,86.20963097)(901.60132448,86.22963095)(901.55131592,86.24963186)
\lineto(901.37131592,86.30963186)
\curveto(901.11132497,86.42963075)(900.86632521,86.58463059)(900.63631592,86.77463186)
\curveto(900.40632567,86.9746302)(900.22132586,87.18962999)(900.08131592,87.41963186)
\curveto(900.00132608,87.52962965)(899.93632614,87.64462953)(899.88631592,87.76463186)
\lineto(899.73631592,88.15463186)
\curveto(899.68632639,88.26462891)(899.65632642,88.3796288)(899.64631592,88.49963186)
\curveto(899.62632645,88.61962856)(899.60132648,88.74462843)(899.57131592,88.87463186)
\curveto(899.57132651,88.94462823)(899.57132651,89.00962817)(899.57131592,89.06963186)
\curveto(899.56132652,89.12962805)(899.55132653,89.19462798)(899.54131592,89.26463186)
}
}
{
\newrgbcolor{curcolor}{0 0 0}
\pscustom[linestyle=none,fillstyle=solid,fillcolor=curcolor]
{
\newpath
\moveto(905.06131592,101.36424123)
\lineto(905.31631592,101.36424123)
\curveto(905.39632068,101.37423353)(905.47132061,101.36923353)(905.54131592,101.34924123)
\lineto(905.78131592,101.34924123)
\lineto(905.94631592,101.34924123)
\curveto(906.04632003,101.32923357)(906.15131993,101.31923358)(906.26131592,101.31924123)
\curveto(906.36131972,101.31923358)(906.46131962,101.30923359)(906.56131592,101.28924123)
\lineto(906.71131592,101.28924123)
\curveto(906.85131923,101.25923364)(906.99131909,101.23923366)(907.13131592,101.22924123)
\curveto(907.26131882,101.21923368)(907.39131869,101.19423371)(907.52131592,101.15424123)
\curveto(907.60131848,101.13423377)(907.68631839,101.11423379)(907.77631592,101.09424123)
\lineto(908.01631592,101.03424123)
\lineto(908.31631592,100.91424123)
\curveto(908.40631767,100.88423402)(908.49631758,100.84923405)(908.58631592,100.80924123)
\curveto(908.80631727,100.70923419)(909.02131706,100.57423433)(909.23131592,100.40424123)
\curveto(909.44131664,100.24423466)(909.61131647,100.06923483)(909.74131592,99.87924123)
\curveto(909.7813163,99.82923507)(909.82131626,99.76923513)(909.86131592,99.69924123)
\curveto(909.89131619,99.63923526)(909.92631615,99.57923532)(909.96631592,99.51924123)
\curveto(910.01631606,99.43923546)(910.05631602,99.34423556)(910.08631592,99.23424123)
\curveto(910.11631596,99.12423578)(910.14631593,99.01923588)(910.17631592,98.91924123)
\curveto(910.21631586,98.80923609)(910.24131584,98.6992362)(910.25131592,98.58924123)
\curveto(910.26131582,98.47923642)(910.2763158,98.36423654)(910.29631592,98.24424123)
\curveto(910.30631577,98.2042367)(910.30631577,98.15923674)(910.29631592,98.10924123)
\curveto(910.29631578,98.06923683)(910.30131578,98.02923687)(910.31131592,97.98924123)
\curveto(910.32131576,97.94923695)(910.32631575,97.89423701)(910.32631592,97.82424123)
\curveto(910.32631575,97.75423715)(910.32131576,97.7042372)(910.31131592,97.67424123)
\curveto(910.29131579,97.62423728)(910.28631579,97.57923732)(910.29631592,97.53924123)
\curveto(910.30631577,97.4992374)(910.30631577,97.46423744)(910.29631592,97.43424123)
\lineto(910.29631592,97.34424123)
\curveto(910.2763158,97.28423762)(910.26131582,97.21923768)(910.25131592,97.14924123)
\curveto(910.25131583,97.08923781)(910.24631583,97.02423788)(910.23631592,96.95424123)
\curveto(910.18631589,96.78423812)(910.13631594,96.62423828)(910.08631592,96.47424123)
\curveto(910.03631604,96.32423858)(909.97131611,96.17923872)(909.89131592,96.03924123)
\curveto(909.85131623,95.98923891)(909.82131626,95.93423897)(909.80131592,95.87424123)
\curveto(909.77131631,95.82423908)(909.73631634,95.77423913)(909.69631592,95.72424123)
\curveto(909.51631656,95.48423942)(909.29631678,95.28423962)(909.03631592,95.12424123)
\curveto(908.7763173,94.96423994)(908.49131759,94.82424008)(908.18131592,94.70424123)
\curveto(908.04131804,94.64424026)(907.90131818,94.5992403)(907.76131592,94.56924123)
\curveto(907.61131847,94.53924036)(907.45631862,94.5042404)(907.29631592,94.46424123)
\curveto(907.18631889,94.44424046)(907.076319,94.42924047)(906.96631592,94.41924123)
\curveto(906.85631922,94.40924049)(906.74631933,94.39424051)(906.63631592,94.37424123)
\curveto(906.59631948,94.36424054)(906.55631952,94.35924054)(906.51631592,94.35924123)
\curveto(906.4763196,94.36924053)(906.43631964,94.36924053)(906.39631592,94.35924123)
\curveto(906.34631973,94.34924055)(906.29631978,94.34424056)(906.24631592,94.34424123)
\lineto(906.08131592,94.34424123)
\curveto(906.03132005,94.32424058)(905.9813201,94.31924058)(905.93131592,94.32924123)
\curveto(905.87132021,94.33924056)(905.81632026,94.33924056)(905.76631592,94.32924123)
\curveto(905.72632035,94.31924058)(905.6813204,94.31924058)(905.63131592,94.32924123)
\curveto(905.5813205,94.33924056)(905.53132055,94.33424057)(905.48131592,94.31424123)
\curveto(905.41132067,94.29424061)(905.33632074,94.28924061)(905.25631592,94.29924123)
\curveto(905.16632091,94.30924059)(905.081321,94.31424059)(905.00131592,94.31424123)
\curveto(904.91132117,94.31424059)(904.81132127,94.30924059)(904.70131592,94.29924123)
\curveto(904.5813215,94.28924061)(904.4813216,94.29424061)(904.40131592,94.31424123)
\lineto(904.11631592,94.31424123)
\lineto(903.48631592,94.35924123)
\curveto(903.38632269,94.36924053)(903.29132279,94.37924052)(903.20131592,94.38924123)
\lineto(902.90131592,94.41924123)
\curveto(902.85132323,94.43924046)(902.80132328,94.44424046)(902.75131592,94.43424123)
\curveto(902.69132339,94.43424047)(902.63632344,94.44424046)(902.58631592,94.46424123)
\curveto(902.41632366,94.51424039)(902.25132383,94.55424035)(902.09131592,94.58424123)
\curveto(901.92132416,94.61424029)(901.76132432,94.66424024)(901.61131592,94.73424123)
\curveto(901.15132493,94.92423998)(900.7763253,95.14423976)(900.48631592,95.39424123)
\curveto(900.19632588,95.65423925)(899.95132613,96.01423889)(899.75131592,96.47424123)
\curveto(899.70132638,96.6042383)(899.66632641,96.73423817)(899.64631592,96.86424123)
\curveto(899.62632645,97.0042379)(899.60132648,97.14423776)(899.57131592,97.28424123)
\curveto(899.56132652,97.35423755)(899.55632652,97.41923748)(899.55631592,97.47924123)
\curveto(899.55632652,97.53923736)(899.55132653,97.6042373)(899.54131592,97.67424123)
\curveto(899.52132656,98.5042364)(899.67132641,99.17423573)(899.99131592,99.68424123)
\curveto(900.30132578,100.19423471)(900.74132534,100.57423433)(901.31131592,100.82424123)
\curveto(901.43132465,100.87423403)(901.55632452,100.91923398)(901.68631592,100.95924123)
\curveto(901.81632426,100.9992339)(901.95132413,101.04423386)(902.09131592,101.09424123)
\curveto(902.17132391,101.11423379)(902.25632382,101.12923377)(902.34631592,101.13924123)
\lineto(902.58631592,101.19924123)
\curveto(902.69632338,101.22923367)(902.80632327,101.24423366)(902.91631592,101.24424123)
\curveto(903.02632305,101.25423365)(903.13632294,101.26923363)(903.24631592,101.28924123)
\curveto(903.29632278,101.30923359)(903.34132274,101.31423359)(903.38131592,101.30424123)
\curveto(903.42132266,101.3042336)(903.46132262,101.30923359)(903.50131592,101.31924123)
\curveto(903.55132253,101.32923357)(903.60632247,101.32923357)(903.66631592,101.31924123)
\curveto(903.71632236,101.31923358)(903.76632231,101.32423358)(903.81631592,101.33424123)
\lineto(903.95131592,101.33424123)
\curveto(904.01132207,101.35423355)(904.081322,101.35423355)(904.16131592,101.33424123)
\curveto(904.23132185,101.32423358)(904.29632178,101.32923357)(904.35631592,101.34924123)
\curveto(904.38632169,101.35923354)(904.42632165,101.36423354)(904.47631592,101.36424123)
\lineto(904.59631592,101.36424123)
\lineto(905.06131592,101.36424123)
\moveto(907.38631592,99.81924123)
\curveto(907.06631901,99.91923498)(906.70131938,99.97923492)(906.29131592,99.99924123)
\curveto(905.8813202,100.01923488)(905.47132061,100.02923487)(905.06131592,100.02924123)
\curveto(904.63132145,100.02923487)(904.21132187,100.01923488)(903.80131592,99.99924123)
\curveto(903.39132269,99.97923492)(903.00632307,99.93423497)(902.64631592,99.86424123)
\curveto(902.28632379,99.79423511)(901.96632411,99.68423522)(901.68631592,99.53424123)
\curveto(901.39632468,99.39423551)(901.16132492,99.1992357)(900.98131592,98.94924123)
\curveto(900.87132521,98.78923611)(900.79132529,98.60923629)(900.74131592,98.40924123)
\curveto(900.6813254,98.20923669)(900.65132543,97.96423694)(900.65131592,97.67424123)
\curveto(900.67132541,97.65423725)(900.6813254,97.61923728)(900.68131592,97.56924123)
\curveto(900.67132541,97.51923738)(900.67132541,97.47923742)(900.68131592,97.44924123)
\curveto(900.70132538,97.36923753)(900.72132536,97.29423761)(900.74131592,97.22424123)
\curveto(900.75132533,97.16423774)(900.77132531,97.0992378)(900.80131592,97.02924123)
\curveto(900.92132516,96.75923814)(901.09132499,96.53923836)(901.31131592,96.36924123)
\curveto(901.52132456,96.20923869)(901.76632431,96.07423883)(902.04631592,95.96424123)
\curveto(902.15632392,95.91423899)(902.2763238,95.87423903)(902.40631592,95.84424123)
\curveto(902.52632355,95.82423908)(902.65132343,95.7992391)(902.78131592,95.76924123)
\curveto(902.83132325,95.74923915)(902.88632319,95.73923916)(902.94631592,95.73924123)
\curveto(902.99632308,95.73923916)(903.04632303,95.73423917)(903.09631592,95.72424123)
\curveto(903.18632289,95.71423919)(903.2813228,95.7042392)(903.38131592,95.69424123)
\curveto(903.47132261,95.68423922)(903.56632251,95.67423923)(903.66631592,95.66424123)
\curveto(903.74632233,95.66423924)(903.83132225,95.65923924)(903.92131592,95.64924123)
\lineto(904.16131592,95.64924123)
\lineto(904.34131592,95.64924123)
\curveto(904.37132171,95.63923926)(904.40632167,95.63423927)(904.44631592,95.63424123)
\lineto(904.58131592,95.63424123)
\lineto(905.03131592,95.63424123)
\curveto(905.11132097,95.63423927)(905.19632088,95.62923927)(905.28631592,95.61924123)
\curveto(905.36632071,95.61923928)(905.44132064,95.62923927)(905.51131592,95.64924123)
\lineto(905.78131592,95.64924123)
\curveto(905.80132028,95.64923925)(905.83132025,95.64423926)(905.87131592,95.63424123)
\curveto(905.90132018,95.63423927)(905.92632015,95.63923926)(905.94631592,95.64924123)
\curveto(906.04632003,95.65923924)(906.14631993,95.66423924)(906.24631592,95.66424123)
\curveto(906.33631974,95.67423923)(906.43631964,95.68423922)(906.54631592,95.69424123)
\curveto(906.66631941,95.72423918)(906.79131929,95.73923916)(906.92131592,95.73924123)
\curveto(907.04131904,95.74923915)(907.15631892,95.77423913)(907.26631592,95.81424123)
\curveto(907.56631851,95.89423901)(907.83131825,95.97923892)(908.06131592,96.06924123)
\curveto(908.29131779,96.16923873)(908.50631757,96.31423859)(908.70631592,96.50424123)
\curveto(908.90631717,96.71423819)(909.05631702,96.97923792)(909.15631592,97.29924123)
\curveto(909.1763169,97.33923756)(909.18631689,97.37423753)(909.18631592,97.40424123)
\curveto(909.1763169,97.44423746)(909.1813169,97.48923741)(909.20131592,97.53924123)
\curveto(909.21131687,97.57923732)(909.22131686,97.64923725)(909.23131592,97.74924123)
\curveto(909.24131684,97.85923704)(909.23631684,97.94423696)(909.21631592,98.00424123)
\curveto(909.19631688,98.07423683)(909.18631689,98.14423676)(909.18631592,98.21424123)
\curveto(909.1763169,98.28423662)(909.16131692,98.34923655)(909.14131592,98.40924123)
\curveto(909.081317,98.60923629)(908.99631708,98.78923611)(908.88631592,98.94924123)
\curveto(908.86631721,98.97923592)(908.84631723,99.0042359)(908.82631592,99.02424123)
\lineto(908.76631592,99.08424123)
\curveto(908.74631733,99.12423578)(908.70631737,99.17423573)(908.64631592,99.23424123)
\curveto(908.50631757,99.33423557)(908.3763177,99.41923548)(908.25631592,99.48924123)
\curveto(908.13631794,99.55923534)(907.99131809,99.62923527)(907.82131592,99.69924123)
\curveto(907.75131833,99.72923517)(907.6813184,99.74923515)(907.61131592,99.75924123)
\curveto(907.54131854,99.77923512)(907.46631861,99.7992351)(907.38631592,99.81924123)
}
}
{
\newrgbcolor{curcolor}{0 0 0}
\pscustom[linestyle=none,fillstyle=solid,fillcolor=curcolor]
{
\newpath
\moveto(899.54131592,106.77385061)
\curveto(899.54132654,106.87384575)(899.55132653,106.96884566)(899.57131592,107.05885061)
\curveto(899.5813265,107.14884548)(899.61132647,107.21384541)(899.66131592,107.25385061)
\curveto(899.74132634,107.31384531)(899.84632623,107.34384528)(899.97631592,107.34385061)
\lineto(900.36631592,107.34385061)
\lineto(901.86631592,107.34385061)
\lineto(908.25631592,107.34385061)
\lineto(909.42631592,107.34385061)
\lineto(909.74131592,107.34385061)
\curveto(909.84131624,107.35384527)(909.92131616,107.33884529)(909.98131592,107.29885061)
\curveto(910.06131602,107.24884538)(910.11131597,107.17384545)(910.13131592,107.07385061)
\curveto(910.14131594,106.98384564)(910.14631593,106.87384575)(910.14631592,106.74385061)
\lineto(910.14631592,106.51885061)
\curveto(910.12631595,106.43884619)(910.11131597,106.36884626)(910.10131592,106.30885061)
\curveto(910.081316,106.24884638)(910.04131604,106.19884643)(909.98131592,106.15885061)
\curveto(909.92131616,106.11884651)(909.84631623,106.09884653)(909.75631592,106.09885061)
\lineto(909.45631592,106.09885061)
\lineto(908.36131592,106.09885061)
\lineto(903.02131592,106.09885061)
\curveto(902.93132315,106.07884655)(902.85632322,106.06384656)(902.79631592,106.05385061)
\curveto(902.72632335,106.05384657)(902.66632341,106.0238466)(902.61631592,105.96385061)
\curveto(902.56632351,105.89384673)(902.54132354,105.80384682)(902.54131592,105.69385061)
\curveto(902.53132355,105.59384703)(902.52632355,105.48384714)(902.52631592,105.36385061)
\lineto(902.52631592,104.22385061)
\lineto(902.52631592,103.72885061)
\curveto(902.51632356,103.56884906)(902.45632362,103.45884917)(902.34631592,103.39885061)
\curveto(902.31632376,103.37884925)(902.28632379,103.36884926)(902.25631592,103.36885061)
\curveto(902.21632386,103.36884926)(902.17132391,103.36384926)(902.12131592,103.35385061)
\curveto(902.00132408,103.33384929)(901.89132419,103.33884929)(901.79131592,103.36885061)
\curveto(901.69132439,103.40884922)(901.62132446,103.46384916)(901.58131592,103.53385061)
\curveto(901.53132455,103.61384901)(901.50632457,103.73384889)(901.50631592,103.89385061)
\curveto(901.50632457,104.05384857)(901.49132459,104.18884844)(901.46131592,104.29885061)
\curveto(901.45132463,104.34884828)(901.44632463,104.40384822)(901.44631592,104.46385061)
\curveto(901.43632464,104.5238481)(901.42132466,104.58384804)(901.40131592,104.64385061)
\curveto(901.35132473,104.79384783)(901.30132478,104.93884769)(901.25131592,105.07885061)
\curveto(901.19132489,105.21884741)(901.12132496,105.35384727)(901.04131592,105.48385061)
\curveto(900.95132513,105.623847)(900.84632523,105.74384688)(900.72631592,105.84385061)
\curveto(900.60632547,105.94384668)(900.4763256,106.03884659)(900.33631592,106.12885061)
\curveto(900.23632584,106.18884644)(900.12632595,106.23384639)(900.00631592,106.26385061)
\curveto(899.88632619,106.30384632)(899.7813263,106.35384627)(899.69131592,106.41385061)
\curveto(899.63132645,106.46384616)(899.59132649,106.53384609)(899.57131592,106.62385061)
\curveto(899.56132652,106.64384598)(899.55632652,106.66884596)(899.55631592,106.69885061)
\curveto(899.55632652,106.7288459)(899.55132653,106.75384587)(899.54131592,106.77385061)
}
}
{
\newrgbcolor{curcolor}{0 0 0}
\pscustom[linestyle=none,fillstyle=solid,fillcolor=curcolor]
{
\newpath
\moveto(899.54131592,115.12345998)
\curveto(899.54132654,115.22345513)(899.55132653,115.31845503)(899.57131592,115.40845998)
\curveto(899.5813265,115.49845485)(899.61132647,115.56345479)(899.66131592,115.60345998)
\curveto(899.74132634,115.66345469)(899.84632623,115.69345466)(899.97631592,115.69345998)
\lineto(900.36631592,115.69345998)
\lineto(901.86631592,115.69345998)
\lineto(908.25631592,115.69345998)
\lineto(909.42631592,115.69345998)
\lineto(909.74131592,115.69345998)
\curveto(909.84131624,115.70345465)(909.92131616,115.68845466)(909.98131592,115.64845998)
\curveto(910.06131602,115.59845475)(910.11131597,115.52345483)(910.13131592,115.42345998)
\curveto(910.14131594,115.33345502)(910.14631593,115.22345513)(910.14631592,115.09345998)
\lineto(910.14631592,114.86845998)
\curveto(910.12631595,114.78845556)(910.11131597,114.71845563)(910.10131592,114.65845998)
\curveto(910.081316,114.59845575)(910.04131604,114.5484558)(909.98131592,114.50845998)
\curveto(909.92131616,114.46845588)(909.84631623,114.4484559)(909.75631592,114.44845998)
\lineto(909.45631592,114.44845998)
\lineto(908.36131592,114.44845998)
\lineto(903.02131592,114.44845998)
\curveto(902.93132315,114.42845592)(902.85632322,114.41345594)(902.79631592,114.40345998)
\curveto(902.72632335,114.40345595)(902.66632341,114.37345598)(902.61631592,114.31345998)
\curveto(902.56632351,114.24345611)(902.54132354,114.1534562)(902.54131592,114.04345998)
\curveto(902.53132355,113.94345641)(902.52632355,113.83345652)(902.52631592,113.71345998)
\lineto(902.52631592,112.57345998)
\lineto(902.52631592,112.07845998)
\curveto(902.51632356,111.91845843)(902.45632362,111.80845854)(902.34631592,111.74845998)
\curveto(902.31632376,111.72845862)(902.28632379,111.71845863)(902.25631592,111.71845998)
\curveto(902.21632386,111.71845863)(902.17132391,111.71345864)(902.12131592,111.70345998)
\curveto(902.00132408,111.68345867)(901.89132419,111.68845866)(901.79131592,111.71845998)
\curveto(901.69132439,111.75845859)(901.62132446,111.81345854)(901.58131592,111.88345998)
\curveto(901.53132455,111.96345839)(901.50632457,112.08345827)(901.50631592,112.24345998)
\curveto(901.50632457,112.40345795)(901.49132459,112.53845781)(901.46131592,112.64845998)
\curveto(901.45132463,112.69845765)(901.44632463,112.7534576)(901.44631592,112.81345998)
\curveto(901.43632464,112.87345748)(901.42132466,112.93345742)(901.40131592,112.99345998)
\curveto(901.35132473,113.14345721)(901.30132478,113.28845706)(901.25131592,113.42845998)
\curveto(901.19132489,113.56845678)(901.12132496,113.70345665)(901.04131592,113.83345998)
\curveto(900.95132513,113.97345638)(900.84632523,114.09345626)(900.72631592,114.19345998)
\curveto(900.60632547,114.29345606)(900.4763256,114.38845596)(900.33631592,114.47845998)
\curveto(900.23632584,114.53845581)(900.12632595,114.58345577)(900.00631592,114.61345998)
\curveto(899.88632619,114.6534557)(899.7813263,114.70345565)(899.69131592,114.76345998)
\curveto(899.63132645,114.81345554)(899.59132649,114.88345547)(899.57131592,114.97345998)
\curveto(899.56132652,114.99345536)(899.55632652,115.01845533)(899.55631592,115.04845998)
\curveto(899.55632652,115.07845527)(899.55132653,115.10345525)(899.54131592,115.12345998)
}
}
{
\newrgbcolor{curcolor}{0 0 0}
\pscustom[linestyle=none,fillstyle=solid,fillcolor=curcolor]
{
\newpath
\moveto(920.37763184,42.29681936)
\curveto(920.37764253,42.36681368)(920.37764253,42.4468136)(920.37763184,42.53681936)
\curveto(920.36764254,42.62681342)(920.36764254,42.71181333)(920.37763184,42.79181936)
\curveto(920.37764253,42.88181316)(920.38764252,42.96181308)(920.40763184,43.03181936)
\curveto(920.42764248,43.11181293)(920.45764245,43.16681288)(920.49763184,43.19681936)
\curveto(920.54764236,43.22681282)(920.62264229,43.2468128)(920.72263184,43.25681936)
\curveto(920.8126421,43.27681277)(920.91764199,43.28681276)(921.03763184,43.28681936)
\curveto(921.14764176,43.29681275)(921.26264165,43.29681275)(921.38263184,43.28681936)
\lineto(921.68263184,43.28681936)
\lineto(924.69763184,43.28681936)
\lineto(927.59263184,43.28681936)
\curveto(927.92263499,43.28681276)(928.24763466,43.28181276)(928.56763184,43.27181936)
\curveto(928.87763403,43.27181277)(929.15763375,43.23181281)(929.40763184,43.15181936)
\curveto(929.75763315,43.03181301)(930.05263286,42.87681317)(930.29263184,42.68681936)
\curveto(930.52263239,42.49681355)(930.72263219,42.25681379)(930.89263184,41.96681936)
\curveto(930.94263197,41.90681414)(930.97763193,41.8418142)(930.99763184,41.77181936)
\curveto(931.01763189,41.71181433)(931.04263187,41.6418144)(931.07263184,41.56181936)
\curveto(931.12263179,41.4418146)(931.15763175,41.31181473)(931.17763184,41.17181936)
\curveto(931.2076317,41.041815)(931.23763167,40.90681514)(931.26763184,40.76681936)
\curveto(931.28763162,40.71681533)(931.29263162,40.66681538)(931.28263184,40.61681936)
\curveto(931.27263164,40.56681548)(931.27263164,40.51181553)(931.28263184,40.45181936)
\curveto(931.29263162,40.43181561)(931.29263162,40.40681564)(931.28263184,40.37681936)
\curveto(931.28263163,40.3468157)(931.28763162,40.32181572)(931.29763184,40.30181936)
\curveto(931.3076316,40.26181578)(931.3126316,40.20681584)(931.31263184,40.13681936)
\curveto(931.3126316,40.06681598)(931.3076316,40.01181603)(931.29763184,39.97181936)
\curveto(931.28763162,39.92181612)(931.28763162,39.86681618)(931.29763184,39.80681936)
\curveto(931.3076316,39.7468163)(931.30263161,39.69181635)(931.28263184,39.64181936)
\curveto(931.25263166,39.51181653)(931.23263168,39.38681666)(931.22263184,39.26681936)
\curveto(931.2126317,39.1468169)(931.18763172,39.03181701)(931.14763184,38.92181936)
\curveto(931.02763188,38.55181749)(930.85763205,38.23181781)(930.63763184,37.96181936)
\curveto(930.41763249,37.69181835)(930.13763277,37.48181856)(929.79763184,37.33181936)
\curveto(929.67763323,37.28181876)(929.55263336,37.23681881)(929.42263184,37.19681936)
\curveto(929.29263362,37.16681888)(929.15763375,37.13181891)(929.01763184,37.09181936)
\curveto(928.96763394,37.08181896)(928.92763398,37.07681897)(928.89763184,37.07681936)
\curveto(928.85763405,37.07681897)(928.8126341,37.07181897)(928.76263184,37.06181936)
\curveto(928.73263418,37.05181899)(928.69763421,37.046819)(928.65763184,37.04681936)
\curveto(928.6076343,37.046819)(928.56763434,37.041819)(928.53763184,37.03181936)
\lineto(928.37263184,37.03181936)
\curveto(928.29263462,37.01181903)(928.19263472,37.00681904)(928.07263184,37.01681936)
\curveto(927.94263497,37.02681902)(927.85263506,37.041819)(927.80263184,37.06181936)
\curveto(927.7126352,37.08181896)(927.64763526,37.13681891)(927.60763184,37.22681936)
\curveto(927.58763532,37.25681879)(927.58263533,37.28681876)(927.59263184,37.31681936)
\curveto(927.59263532,37.3468187)(927.58763532,37.38681866)(927.57763184,37.43681936)
\curveto(927.56763534,37.47681857)(927.56263535,37.51681853)(927.56263184,37.55681936)
\lineto(927.56263184,37.70681936)
\curveto(927.56263535,37.82681822)(927.56763534,37.9468181)(927.57763184,38.06681936)
\curveto(927.57763533,38.19681785)(927.6126353,38.28681776)(927.68263184,38.33681936)
\curveto(927.74263517,38.37681767)(927.80263511,38.39681765)(927.86263184,38.39681936)
\curveto(927.92263499,38.39681765)(927.99263492,38.40681764)(928.07263184,38.42681936)
\curveto(928.10263481,38.43681761)(928.13763477,38.43681761)(928.17763184,38.42681936)
\curveto(928.2076347,38.42681762)(928.23263468,38.43181761)(928.25263184,38.44181936)
\lineto(928.46263184,38.44181936)
\curveto(928.5126344,38.46181758)(928.56263435,38.46681758)(928.61263184,38.45681936)
\curveto(928.65263426,38.45681759)(928.69763421,38.46681758)(928.74763184,38.48681936)
\curveto(928.87763403,38.51681753)(929.00263391,38.5468175)(929.12263184,38.57681936)
\curveto(929.23263368,38.60681744)(929.33763357,38.65181739)(929.43763184,38.71181936)
\curveto(929.72763318,38.88181716)(929.93263298,39.15181689)(930.05263184,39.52181936)
\curveto(930.07263284,39.57181647)(930.08763282,39.62181642)(930.09763184,39.67181936)
\curveto(930.09763281,39.73181631)(930.1076328,39.78681626)(930.12763184,39.83681936)
\lineto(930.12763184,39.91181936)
\curveto(930.13763277,39.98181606)(930.14763276,40.07681597)(930.15763184,40.19681936)
\curveto(930.15763275,40.32681572)(930.14763276,40.42681562)(930.12763184,40.49681936)
\curveto(930.1076328,40.56681548)(930.09263282,40.63681541)(930.08263184,40.70681936)
\curveto(930.06263285,40.78681526)(930.04263287,40.85681519)(930.02263184,40.91681936)
\curveto(929.86263305,41.29681475)(929.58763332,41.57181447)(929.19763184,41.74181936)
\curveto(929.06763384,41.79181425)(928.912634,41.82681422)(928.73263184,41.84681936)
\curveto(928.55263436,41.87681417)(928.36763454,41.89181415)(928.17763184,41.89181936)
\curveto(927.97763493,41.90181414)(927.77763513,41.90181414)(927.57763184,41.89181936)
\lineto(927.00763184,41.89181936)
\lineto(922.76263184,41.89181936)
\lineto(921.21763184,41.89181936)
\curveto(921.1076418,41.89181415)(920.98764192,41.88681416)(920.85763184,41.87681936)
\curveto(920.72764218,41.86681418)(920.62264229,41.88681416)(920.54263184,41.93681936)
\curveto(920.47264244,41.99681405)(920.42264249,42.07681397)(920.39263184,42.17681936)
\curveto(920.39264252,42.19681385)(920.39264252,42.21681383)(920.39263184,42.23681936)
\curveto(920.39264252,42.25681379)(920.38764252,42.27681377)(920.37763184,42.29681936)
}
}
{
\newrgbcolor{curcolor}{0 0 0}
\pscustom[linestyle=none,fillstyle=solid,fillcolor=curcolor]
{
\newpath
\moveto(923.33263184,45.83049123)
\lineto(923.33263184,46.26549123)
\curveto(923.33263958,46.41548927)(923.37263954,46.52048916)(923.45263184,46.58049123)
\curveto(923.53263938,46.63048905)(923.63263928,46.65548903)(923.75263184,46.65549123)
\curveto(923.87263904,46.66548902)(923.99263892,46.67048901)(924.11263184,46.67049123)
\lineto(925.53763184,46.67049123)
\lineto(927.80263184,46.67049123)
\lineto(928.49263184,46.67049123)
\curveto(928.72263419,46.67048901)(928.92263399,46.69548899)(929.09263184,46.74549123)
\curveto(929.54263337,46.90548878)(929.85763305,47.20548848)(930.03763184,47.64549123)
\curveto(930.12763278,47.86548782)(930.16263275,48.13048755)(930.14263184,48.44049123)
\curveto(930.1126328,48.75048693)(930.05763285,49.00048668)(929.97763184,49.19049123)
\curveto(929.83763307,49.52048616)(929.66263325,49.7804859)(929.45263184,49.97049123)
\curveto(929.23263368,50.17048551)(928.94763396,50.32548536)(928.59763184,50.43549123)
\curveto(928.51763439,50.46548522)(928.43763447,50.4854852)(928.35763184,50.49549123)
\curveto(928.27763463,50.50548518)(928.19263472,50.52048516)(928.10263184,50.54049123)
\curveto(928.05263486,50.55048513)(928.0076349,50.55048513)(927.96763184,50.54049123)
\curveto(927.92763498,50.54048514)(927.88263503,50.55048513)(927.83263184,50.57049123)
\lineto(927.51763184,50.57049123)
\curveto(927.43763547,50.59048509)(927.34763556,50.59548509)(927.24763184,50.58549123)
\curveto(927.13763577,50.57548511)(927.03763587,50.57048511)(926.94763184,50.57049123)
\lineto(925.77763184,50.57049123)
\lineto(924.18763184,50.57049123)
\curveto(924.06763884,50.57048511)(923.94263897,50.56548512)(923.81263184,50.55549123)
\curveto(923.67263924,50.55548513)(923.56263935,50.5804851)(923.48263184,50.63049123)
\curveto(923.43263948,50.67048501)(923.40263951,50.71548497)(923.39263184,50.76549123)
\curveto(923.37263954,50.82548486)(923.35263956,50.89548479)(923.33263184,50.97549123)
\lineto(923.33263184,51.20049123)
\curveto(923.33263958,51.32048436)(923.33763957,51.42548426)(923.34763184,51.51549123)
\curveto(923.35763955,51.61548407)(923.40263951,51.69048399)(923.48263184,51.74049123)
\curveto(923.53263938,51.79048389)(923.6076393,51.81548387)(923.70763184,51.81549123)
\lineto(923.99263184,51.81549123)
\lineto(925.01263184,51.81549123)
\lineto(929.04763184,51.81549123)
\lineto(930.39763184,51.81549123)
\curveto(930.51763239,51.81548387)(930.63263228,51.81048387)(930.74263184,51.80049123)
\curveto(930.84263207,51.80048388)(930.91763199,51.76548392)(930.96763184,51.69549123)
\curveto(930.99763191,51.65548403)(931.02263189,51.59548409)(931.04263184,51.51549123)
\curveto(931.05263186,51.43548425)(931.06263185,51.34548434)(931.07263184,51.24549123)
\curveto(931.07263184,51.15548453)(931.06763184,51.06548462)(931.05763184,50.97549123)
\curveto(931.04763186,50.89548479)(931.02763188,50.83548485)(930.99763184,50.79549123)
\curveto(930.95763195,50.74548494)(930.89263202,50.70048498)(930.80263184,50.66049123)
\curveto(930.76263215,50.65048503)(930.7076322,50.64048504)(930.63763184,50.63049123)
\curveto(930.56763234,50.63048505)(930.50263241,50.62548506)(930.44263184,50.61549123)
\curveto(930.37263254,50.60548508)(930.31763259,50.5854851)(930.27763184,50.55549123)
\curveto(930.23763267,50.52548516)(930.22263269,50.4804852)(930.23263184,50.42049123)
\curveto(930.25263266,50.34048534)(930.3126326,50.26048542)(930.41263184,50.18049123)
\curveto(930.50263241,50.10048558)(930.57263234,50.02548566)(930.62263184,49.95549123)
\curveto(930.78263213,49.73548595)(930.92263199,49.4854862)(931.04263184,49.20549123)
\curveto(931.09263182,49.09548659)(931.12263179,48.9804867)(931.13263184,48.86049123)
\curveto(931.15263176,48.75048693)(931.17763173,48.63548705)(931.20763184,48.51549123)
\curveto(931.21763169,48.46548722)(931.21763169,48.41048727)(931.20763184,48.35049123)
\curveto(931.19763171,48.30048738)(931.20263171,48.25048743)(931.22263184,48.20049123)
\curveto(931.24263167,48.10048758)(931.24263167,48.01048767)(931.22263184,47.93049123)
\lineto(931.22263184,47.78049123)
\curveto(931.20263171,47.73048795)(931.19263172,47.67048801)(931.19263184,47.60049123)
\curveto(931.19263172,47.54048814)(931.18763172,47.4854882)(931.17763184,47.43549123)
\curveto(931.15763175,47.39548829)(931.14763176,47.35548833)(931.14763184,47.31549123)
\curveto(931.15763175,47.2854884)(931.15263176,47.24548844)(931.13263184,47.19549123)
\lineto(931.07263184,46.95549123)
\curveto(931.05263186,46.8854888)(931.02263189,46.81048887)(930.98263184,46.73049123)
\curveto(930.87263204,46.47048921)(930.72763218,46.25048943)(930.54763184,46.07049123)
\curveto(930.35763255,45.90048978)(930.13263278,45.76048992)(929.87263184,45.65049123)
\curveto(929.78263313,45.61049007)(929.69263322,45.5804901)(929.60263184,45.56049123)
\lineto(929.30263184,45.50049123)
\curveto(929.24263367,45.4804902)(929.18763372,45.47049021)(929.13763184,45.47049123)
\curveto(929.07763383,45.4804902)(929.0126339,45.47549021)(928.94263184,45.45549123)
\curveto(928.92263399,45.44549024)(928.89763401,45.44049024)(928.86763184,45.44049123)
\curveto(928.82763408,45.44049024)(928.79263412,45.43549025)(928.76263184,45.42549123)
\lineto(928.61263184,45.42549123)
\curveto(928.57263434,45.41549027)(928.52763438,45.41049027)(928.47763184,45.41049123)
\curveto(928.41763449,45.42049026)(928.36263455,45.42549026)(928.31263184,45.42549123)
\lineto(927.71263184,45.42549123)
\lineto(924.95263184,45.42549123)
\lineto(923.99263184,45.42549123)
\lineto(923.72263184,45.42549123)
\curveto(923.63263928,45.42549026)(923.55763935,45.44549024)(923.49763184,45.48549123)
\curveto(923.42763948,45.52549016)(923.37763953,45.60049008)(923.34763184,45.71049123)
\curveto(923.33763957,45.73048995)(923.33763957,45.75048993)(923.34763184,45.77049123)
\curveto(923.34763956,45.79048989)(923.34263957,45.81048987)(923.33263184,45.83049123)
}
}
{
\newrgbcolor{curcolor}{0 0 0}
\pscustom[linestyle=none,fillstyle=solid,fillcolor=curcolor]
{
\newpath
\moveto(920.37763184,54.28510061)
\curveto(920.37764253,54.41509899)(920.37764253,54.55009886)(920.37763184,54.69010061)
\curveto(920.37764253,54.84009857)(920.4126425,54.95009846)(920.48263184,55.02010061)
\curveto(920.55264236,55.07009834)(920.64764226,55.09509831)(920.76763184,55.09510061)
\curveto(920.87764203,55.1050983)(920.99264192,55.1100983)(921.11263184,55.11010061)
\lineto(922.44763184,55.11010061)
\lineto(928.52263184,55.11010061)
\lineto(930.20263184,55.11010061)
\lineto(930.59263184,55.11010061)
\curveto(930.73263218,55.1100983)(930.84263207,55.08509832)(930.92263184,55.03510061)
\curveto(930.97263194,55.0050984)(931.00263191,54.96009845)(931.01263184,54.90010061)
\curveto(931.02263189,54.85009856)(931.03763187,54.78509862)(931.05763184,54.70510061)
\lineto(931.05763184,54.49510061)
\lineto(931.05763184,54.18010061)
\curveto(931.04763186,54.08009933)(931.0126319,54.0050994)(930.95263184,53.95510061)
\curveto(930.87263204,53.9050995)(930.77263214,53.87509953)(930.65263184,53.86510061)
\lineto(930.27763184,53.86510061)
\lineto(928.89763184,53.86510061)
\lineto(922.65763184,53.86510061)
\lineto(921.18763184,53.86510061)
\curveto(921.07764183,53.86509954)(920.96264195,53.86009955)(920.84263184,53.85010061)
\curveto(920.7126422,53.85009956)(920.6126423,53.87509953)(920.54263184,53.92510061)
\curveto(920.48264243,53.96509944)(920.43264248,54.04009937)(920.39263184,54.15010061)
\curveto(920.38264253,54.17009924)(920.38264253,54.19009922)(920.39263184,54.21010061)
\curveto(920.39264252,54.24009917)(920.38764252,54.26509914)(920.37763184,54.28510061)
}
}
{
\newrgbcolor{curcolor}{0 0 0}
\pscustom[linestyle=none,fillstyle=solid,fillcolor=curcolor]
{
}
}
{
\newrgbcolor{curcolor}{0 0 0}
\pscustom[linestyle=none,fillstyle=solid,fillcolor=curcolor]
{
\newpath
\moveto(920.45263184,64.24510061)
\curveto(920.44264247,64.93509597)(920.56264235,65.53509537)(920.81263184,66.04510061)
\curveto(921.06264185,66.56509434)(921.39764151,66.96009395)(921.81763184,67.23010061)
\curveto(921.89764101,67.28009363)(921.98764092,67.32509358)(922.08763184,67.36510061)
\curveto(922.17764073,67.4050935)(922.27264064,67.45009346)(922.37263184,67.50010061)
\curveto(922.47264044,67.54009337)(922.57264034,67.57009334)(922.67263184,67.59010061)
\curveto(922.77264014,67.6100933)(922.87764003,67.63009328)(922.98763184,67.65010061)
\curveto(923.03763987,67.67009324)(923.08263983,67.67509323)(923.12263184,67.66510061)
\curveto(923.16263975,67.65509325)(923.2076397,67.66009325)(923.25763184,67.68010061)
\curveto(923.3076396,67.69009322)(923.39263952,67.69509321)(923.51263184,67.69510061)
\curveto(923.62263929,67.69509321)(923.7076392,67.69009322)(923.76763184,67.68010061)
\curveto(923.82763908,67.66009325)(923.88763902,67.65009326)(923.94763184,67.65010061)
\curveto(924.0076389,67.66009325)(924.06763884,67.65509325)(924.12763184,67.63510061)
\curveto(924.26763864,67.59509331)(924.40263851,67.56009335)(924.53263184,67.53010061)
\curveto(924.66263825,67.50009341)(924.78763812,67.46009345)(924.90763184,67.41010061)
\curveto(925.04763786,67.35009356)(925.17263774,67.28009363)(925.28263184,67.20010061)
\curveto(925.39263752,67.13009378)(925.50263741,67.05509385)(925.61263184,66.97510061)
\lineto(925.67263184,66.91510061)
\curveto(925.69263722,66.905094)(925.7126372,66.89009402)(925.73263184,66.87010061)
\curveto(925.89263702,66.75009416)(926.03763687,66.61509429)(926.16763184,66.46510061)
\curveto(926.29763661,66.31509459)(926.42263649,66.15509475)(926.54263184,65.98510061)
\curveto(926.76263615,65.67509523)(926.96763594,65.38009553)(927.15763184,65.10010061)
\curveto(927.29763561,64.87009604)(927.43263548,64.64009627)(927.56263184,64.41010061)
\curveto(927.69263522,64.19009672)(927.82763508,63.97009694)(927.96763184,63.75010061)
\curveto(928.13763477,63.50009741)(928.31763459,63.26009765)(928.50763184,63.03010061)
\curveto(928.69763421,62.8100981)(928.92263399,62.62009829)(929.18263184,62.46010061)
\curveto(929.24263367,62.42009849)(929.30263361,62.38509852)(929.36263184,62.35510061)
\curveto(929.4126335,62.32509858)(929.47763343,62.29509861)(929.55763184,62.26510061)
\curveto(929.62763328,62.24509866)(929.68763322,62.24009867)(929.73763184,62.25010061)
\curveto(929.8076331,62.27009864)(929.86263305,62.3050986)(929.90263184,62.35510061)
\curveto(929.93263298,62.4050985)(929.95263296,62.46509844)(929.96263184,62.53510061)
\lineto(929.96263184,62.77510061)
\lineto(929.96263184,63.52510061)
\lineto(929.96263184,66.33010061)
\lineto(929.96263184,66.99010061)
\curveto(929.96263295,67.08009383)(929.96763294,67.16509374)(929.97763184,67.24510061)
\curveto(929.97763293,67.32509358)(929.99763291,67.39009352)(930.03763184,67.44010061)
\curveto(930.07763283,67.49009342)(930.15263276,67.53009338)(930.26263184,67.56010061)
\curveto(930.36263255,67.60009331)(930.46263245,67.6100933)(930.56263184,67.59010061)
\lineto(930.69763184,67.59010061)
\curveto(930.76763214,67.57009334)(930.82763208,67.55009336)(930.87763184,67.53010061)
\curveto(930.92763198,67.5100934)(930.96763194,67.47509343)(930.99763184,67.42510061)
\curveto(931.03763187,67.37509353)(931.05763185,67.3050936)(931.05763184,67.21510061)
\lineto(931.05763184,66.94510061)
\lineto(931.05763184,66.04510061)
\lineto(931.05763184,62.53510061)
\lineto(931.05763184,61.47010061)
\curveto(931.05763185,61.39009952)(931.06263185,61.30009961)(931.07263184,61.20010061)
\curveto(931.07263184,61.10009981)(931.06263185,61.01509989)(931.04263184,60.94510061)
\curveto(930.97263194,60.73510017)(930.79263212,60.67010024)(930.50263184,60.75010061)
\curveto(930.46263245,60.76010015)(930.42763248,60.76010015)(930.39763184,60.75010061)
\curveto(930.35763255,60.75010016)(930.3126326,60.76010015)(930.26263184,60.78010061)
\curveto(930.18263273,60.80010011)(930.09763281,60.82010009)(930.00763184,60.84010061)
\curveto(929.91763299,60.86010005)(929.83263308,60.88510002)(929.75263184,60.91510061)
\curveto(929.26263365,61.07509983)(928.84763406,61.27509963)(928.50763184,61.51510061)
\curveto(928.25763465,61.69509921)(928.03263488,61.90009901)(927.83263184,62.13010061)
\curveto(927.62263529,62.36009855)(927.42763548,62.60009831)(927.24763184,62.85010061)
\curveto(927.06763584,63.1100978)(926.89763601,63.37509753)(926.73763184,63.64510061)
\curveto(926.56763634,63.92509698)(926.39263652,64.19509671)(926.21263184,64.45510061)
\curveto(926.13263678,64.56509634)(926.05763685,64.67009624)(925.98763184,64.77010061)
\curveto(925.91763699,64.88009603)(925.84263707,64.99009592)(925.76263184,65.10010061)
\curveto(925.73263718,65.14009577)(925.70263721,65.17509573)(925.67263184,65.20510061)
\curveto(925.63263728,65.24509566)(925.60263731,65.28509562)(925.58263184,65.32510061)
\curveto(925.47263744,65.46509544)(925.34763756,65.59009532)(925.20763184,65.70010061)
\curveto(925.17763773,65.72009519)(925.15263776,65.74509516)(925.13263184,65.77510061)
\curveto(925.10263781,65.8050951)(925.07263784,65.83009508)(925.04263184,65.85010061)
\curveto(924.94263797,65.93009498)(924.84263807,65.99509491)(924.74263184,66.04510061)
\curveto(924.64263827,66.1050948)(924.53263838,66.16009475)(924.41263184,66.21010061)
\curveto(924.34263857,66.24009467)(924.26763864,66.26009465)(924.18763184,66.27010061)
\lineto(923.94763184,66.33010061)
\lineto(923.85763184,66.33010061)
\curveto(923.82763908,66.34009457)(923.79763911,66.34509456)(923.76763184,66.34510061)
\curveto(923.69763921,66.36509454)(923.60263931,66.37009454)(923.48263184,66.36010061)
\curveto(923.35263956,66.36009455)(923.25263966,66.35009456)(923.18263184,66.33010061)
\curveto(923.10263981,66.3100946)(923.02763988,66.29009462)(922.95763184,66.27010061)
\curveto(922.87764003,66.26009465)(922.79764011,66.24009467)(922.71763184,66.21010061)
\curveto(922.47764043,66.10009481)(922.27764063,65.95009496)(922.11763184,65.76010061)
\curveto(921.94764096,65.58009533)(921.8076411,65.36009555)(921.69763184,65.10010061)
\curveto(921.67764123,65.03009588)(921.66264125,64.96009595)(921.65263184,64.89010061)
\curveto(921.63264128,64.82009609)(921.6126413,64.74509616)(921.59263184,64.66510061)
\curveto(921.57264134,64.58509632)(921.56264135,64.47509643)(921.56263184,64.33510061)
\curveto(921.56264135,64.2050967)(921.57264134,64.10009681)(921.59263184,64.02010061)
\curveto(921.60264131,63.96009695)(921.6076413,63.905097)(921.60763184,63.85510061)
\curveto(921.6076413,63.8050971)(921.61764129,63.75509715)(921.63763184,63.70510061)
\curveto(921.67764123,63.6050973)(921.71764119,63.5100974)(921.75763184,63.42010061)
\curveto(921.79764111,63.34009757)(921.84264107,63.26009765)(921.89263184,63.18010061)
\curveto(921.912641,63.15009776)(921.93764097,63.12009779)(921.96763184,63.09010061)
\curveto(921.99764091,63.07009784)(922.02264089,63.04509786)(922.04263184,63.01510061)
\lineto(922.11763184,62.94010061)
\curveto(922.13764077,62.910098)(922.15764075,62.88509802)(922.17763184,62.86510061)
\lineto(922.38763184,62.71510061)
\curveto(922.44764046,62.67509823)(922.5126404,62.63009828)(922.58263184,62.58010061)
\curveto(922.67264024,62.52009839)(922.77764013,62.47009844)(922.89763184,62.43010061)
\curveto(923.0076399,62.40009851)(923.11763979,62.36509854)(923.22763184,62.32510061)
\curveto(923.33763957,62.28509862)(923.48263943,62.26009865)(923.66263184,62.25010061)
\curveto(923.83263908,62.24009867)(923.95763895,62.2100987)(924.03763184,62.16010061)
\curveto(924.11763879,62.1100988)(924.16263875,62.03509887)(924.17263184,61.93510061)
\curveto(924.18263873,61.83509907)(924.18763872,61.72509918)(924.18763184,61.60510061)
\curveto(924.18763872,61.56509934)(924.19263872,61.52509938)(924.20263184,61.48510061)
\curveto(924.20263871,61.44509946)(924.19763871,61.4100995)(924.18763184,61.38010061)
\curveto(924.16763874,61.33009958)(924.15763875,61.28009963)(924.15763184,61.23010061)
\curveto(924.15763875,61.19009972)(924.14763876,61.15009976)(924.12763184,61.11010061)
\curveto(924.06763884,61.02009989)(923.93263898,60.97509993)(923.72263184,60.97510061)
\lineto(923.60263184,60.97510061)
\curveto(923.54263937,60.98509992)(923.48263943,60.99009992)(923.42263184,60.99010061)
\curveto(923.35263956,61.00009991)(923.28763962,61.0100999)(923.22763184,61.02010061)
\curveto(923.11763979,61.04009987)(923.01763989,61.06009985)(922.92763184,61.08010061)
\curveto(922.82764008,61.10009981)(922.73264018,61.13009978)(922.64263184,61.17010061)
\curveto(922.57264034,61.19009972)(922.5126404,61.2100997)(922.46263184,61.23010061)
\lineto(922.28263184,61.29010061)
\curveto(922.02264089,61.4100995)(921.77764113,61.56509934)(921.54763184,61.75510061)
\curveto(921.31764159,61.95509895)(921.13264178,62.17009874)(920.99263184,62.40010061)
\curveto(920.912642,62.5100984)(920.84764206,62.62509828)(920.79763184,62.74510061)
\lineto(920.64763184,63.13510061)
\curveto(920.59764231,63.24509766)(920.56764234,63.36009755)(920.55763184,63.48010061)
\curveto(920.53764237,63.60009731)(920.5126424,63.72509718)(920.48263184,63.85510061)
\curveto(920.48264243,63.92509698)(920.48264243,63.99009692)(920.48263184,64.05010061)
\curveto(920.47264244,64.1100968)(920.46264245,64.17509673)(920.45263184,64.24510061)
}
}
{
\newrgbcolor{curcolor}{0 0 0}
\pscustom[linestyle=none,fillstyle=solid,fillcolor=curcolor]
{
\newpath
\moveto(927.56263184,76.22470998)
\curveto(927.6126353,76.29470234)(927.68263523,76.3347023)(927.77263184,76.34470998)
\curveto(927.86263505,76.36470227)(927.96763494,76.37470226)(928.08763184,76.37470998)
\curveto(928.13763477,76.37470226)(928.18763472,76.36970226)(928.23763184,76.35970998)
\curveto(928.28763462,76.35970227)(928.33263458,76.34970228)(928.37263184,76.32970998)
\curveto(928.46263445,76.29970233)(928.52263439,76.23970239)(928.55263184,76.14970998)
\curveto(928.57263434,76.06970256)(928.58263433,75.97470266)(928.58263184,75.86470998)
\lineto(928.58263184,75.54970998)
\curveto(928.57263434,75.43970319)(928.58263433,75.3347033)(928.61263184,75.23470998)
\curveto(928.64263427,75.09470354)(928.72263419,75.00470363)(928.85263184,74.96470998)
\curveto(928.92263399,74.94470369)(929.0076339,74.9347037)(929.10763184,74.93470998)
\lineto(929.37763184,74.93470998)
\lineto(930.32263184,74.93470998)
\lineto(930.65263184,74.93470998)
\curveto(930.76263215,74.9347037)(930.84763206,74.91470372)(930.90763184,74.87470998)
\curveto(930.96763194,74.8347038)(931.0076319,74.78470385)(931.02763184,74.72470998)
\curveto(931.03763187,74.67470396)(931.05263186,74.60970402)(931.07263184,74.52970998)
\lineto(931.07263184,74.33470998)
\curveto(931.07263184,74.21470442)(931.06763184,74.10970452)(931.05763184,74.01970998)
\curveto(931.03763187,73.9297047)(930.98763192,73.85970477)(930.90763184,73.80970998)
\curveto(930.85763205,73.77970485)(930.78763212,73.76470487)(930.69763184,73.76470998)
\lineto(930.39763184,73.76470998)
\lineto(929.36263184,73.76470998)
\curveto(929.20263371,73.76470487)(929.05763385,73.75470488)(928.92763184,73.73470998)
\curveto(928.78763412,73.72470491)(928.69263422,73.66970496)(928.64263184,73.56970998)
\curveto(928.62263429,73.51970511)(928.6076343,73.44970518)(928.59763184,73.35970998)
\curveto(928.58763432,73.27970535)(928.58263433,73.18970544)(928.58263184,73.08970998)
\lineto(928.58263184,72.80470998)
\lineto(928.58263184,72.56470998)
\lineto(928.58263184,70.29970998)
\curveto(928.58263433,70.20970842)(928.58763432,70.10470853)(928.59763184,69.98470998)
\lineto(928.59763184,69.65470998)
\curveto(928.59763431,69.54470909)(928.58763432,69.44470919)(928.56763184,69.35470998)
\curveto(928.54763436,69.26470937)(928.5126344,69.20470943)(928.46263184,69.17470998)
\curveto(928.39263452,69.12470951)(928.29763461,69.09970953)(928.17763184,69.09970998)
\lineto(927.83263184,69.09970998)
\lineto(927.56263184,69.09970998)
\curveto(927.39263552,69.13970949)(927.25263566,69.19470944)(927.14263184,69.26470998)
\curveto(927.03263588,69.3347093)(926.91763599,69.41470922)(926.79763184,69.50470998)
\lineto(926.25763184,69.86470998)
\curveto(925.62763728,70.30470833)(925.0076379,70.73970789)(924.39763184,71.16970998)
\lineto(922.53763184,72.48970998)
\curveto(922.3076406,72.64970598)(922.08764082,72.80470583)(921.87763184,72.95470998)
\curveto(921.65764125,73.10470553)(921.43264148,73.25970537)(921.20263184,73.41970998)
\curveto(921.13264178,73.46970516)(921.06764184,73.51970511)(921.00763184,73.56970998)
\curveto(920.93764197,73.61970501)(920.86264205,73.66970496)(920.78263184,73.71970998)
\lineto(920.69263184,73.77970998)
\curveto(920.65264226,73.80970482)(920.62264229,73.83970479)(920.60263184,73.86970998)
\curveto(920.57264234,73.90970472)(920.55264236,73.94970468)(920.54263184,73.98970998)
\curveto(920.52264239,74.0297046)(920.50264241,74.07470456)(920.48263184,74.12470998)
\curveto(920.48264243,74.14470449)(920.48764242,74.16470447)(920.49763184,74.18470998)
\curveto(920.49764241,74.21470442)(920.48764242,74.23970439)(920.46763184,74.25970998)
\curveto(920.46764244,74.38970424)(920.47264244,74.50970412)(920.48263184,74.61970998)
\curveto(920.49264242,74.7297039)(920.53764237,74.80970382)(920.61763184,74.85970998)
\curveto(920.66764224,74.89970373)(920.73764217,74.91970371)(920.82763184,74.91970998)
\curveto(920.91764199,74.9297037)(921.0126419,74.9347037)(921.11263184,74.93470998)
\lineto(926.57263184,74.93470998)
\curveto(926.64263627,74.9347037)(926.71763619,74.9297037)(926.79763184,74.91970998)
\curveto(926.86763604,74.91970371)(926.93763597,74.92470371)(927.00763184,74.93470998)
\lineto(927.11263184,74.93470998)
\curveto(927.16263575,74.95470368)(927.21763569,74.96970366)(927.27763184,74.97970998)
\curveto(927.32763558,74.98970364)(927.36763554,75.01470362)(927.39763184,75.05470998)
\curveto(927.44763546,75.12470351)(927.47763543,75.20970342)(927.48763184,75.30970998)
\lineto(927.48763184,75.63970998)
\curveto(927.48763542,75.74970288)(927.49263542,75.85470278)(927.50263184,75.95470998)
\curveto(927.50263541,76.06470257)(927.52263539,76.15470248)(927.56263184,76.22470998)
\moveto(927.36763184,73.65970998)
\curveto(927.25763565,73.73970489)(927.08763582,73.77470486)(926.85763184,73.76470998)
\lineto(926.24263184,73.76470998)
\lineto(923.76763184,73.76470998)
\lineto(923.45263184,73.76470998)
\curveto(923.33263958,73.77470486)(923.23263968,73.76970486)(923.15263184,73.74970998)
\lineto(923.00263184,73.74970998)
\curveto(922.91264,73.74970488)(922.82764008,73.7347049)(922.74763184,73.70470998)
\curveto(922.72764018,73.69470494)(922.71764019,73.68470495)(922.71763184,73.67470998)
\lineto(922.67263184,73.62970998)
\curveto(922.66264025,73.60970502)(922.65764025,73.57970505)(922.65763184,73.53970998)
\curveto(922.67764023,73.51970511)(922.69264022,73.49970513)(922.70263184,73.47970998)
\curveto(922.70264021,73.46970516)(922.7076402,73.45470518)(922.71763184,73.43470998)
\curveto(922.76764014,73.37470526)(922.83764007,73.31470532)(922.92763184,73.25470998)
\curveto(923.01763989,73.19470544)(923.09763981,73.13970549)(923.16763184,73.08970998)
\curveto(923.3076396,72.98970564)(923.45263946,72.89470574)(923.60263184,72.80470998)
\curveto(923.74263917,72.71470592)(923.88263903,72.61970601)(924.02263184,72.51970998)
\lineto(924.80263184,71.97970998)
\curveto(925.06263785,71.80970682)(925.32263759,71.634707)(925.58263184,71.45470998)
\curveto(925.69263722,71.37470726)(925.79763711,71.29970733)(925.89763184,71.22970998)
\lineto(926.19763184,71.01970998)
\curveto(926.27763663,70.96970766)(926.35263656,70.91970771)(926.42263184,70.86970998)
\curveto(926.49263642,70.8297078)(926.56763634,70.78470785)(926.64763184,70.73470998)
\curveto(926.7076362,70.68470795)(926.77263614,70.634708)(926.84263184,70.58470998)
\curveto(926.90263601,70.54470809)(926.97263594,70.50470813)(927.05263184,70.46470998)
\curveto(927.1126358,70.42470821)(927.18263573,70.39970823)(927.26263184,70.38970998)
\curveto(927.33263558,70.37970825)(927.38763552,70.41470822)(927.42763184,70.49470998)
\curveto(927.47763543,70.56470807)(927.50263541,70.67470796)(927.50263184,70.82470998)
\curveto(927.49263542,70.98470765)(927.48763542,71.11970751)(927.48763184,71.22970998)
\lineto(927.48763184,72.90970998)
\lineto(927.48763184,73.34470998)
\curveto(927.48763542,73.49470514)(927.44763546,73.59970503)(927.36763184,73.65970998)
}
}
{
\newrgbcolor{curcolor}{0 0 0}
\pscustom[linestyle=none,fillstyle=solid,fillcolor=curcolor]
{
\newpath
\moveto(929.42263184,78.63431936)
\lineto(929.42263184,79.26431936)
\lineto(929.42263184,79.45931936)
\curveto(929.42263349,79.52931683)(929.43263348,79.58931677)(929.45263184,79.63931936)
\curveto(929.49263342,79.70931665)(929.53263338,79.7593166)(929.57263184,79.78931936)
\curveto(929.62263329,79.82931653)(929.68763322,79.84931651)(929.76763184,79.84931936)
\curveto(929.84763306,79.8593165)(929.93263298,79.86431649)(930.02263184,79.86431936)
\lineto(930.74263184,79.86431936)
\curveto(931.22263169,79.86431649)(931.63263128,79.80431655)(931.97263184,79.68431936)
\curveto(932.3126306,79.56431679)(932.58763032,79.36931699)(932.79763184,79.09931936)
\curveto(932.84763006,79.02931733)(932.89263002,78.9593174)(932.93263184,78.88931936)
\curveto(932.98262993,78.82931753)(933.02762988,78.7543176)(933.06763184,78.66431936)
\curveto(933.07762983,78.64431771)(933.08762982,78.61431774)(933.09763184,78.57431936)
\curveto(933.11762979,78.53431782)(933.12262979,78.48931787)(933.11263184,78.43931936)
\curveto(933.08262983,78.34931801)(933.0076299,78.29431806)(932.88763184,78.27431936)
\curveto(932.77763013,78.2543181)(932.68263023,78.26931809)(932.60263184,78.31931936)
\curveto(932.53263038,78.34931801)(932.46763044,78.39431796)(932.40763184,78.45431936)
\curveto(932.35763055,78.52431783)(932.3076306,78.58931777)(932.25763184,78.64931936)
\curveto(932.2076307,78.71931764)(932.13263078,78.77931758)(932.03263184,78.82931936)
\curveto(931.94263097,78.88931747)(931.85263106,78.93931742)(931.76263184,78.97931936)
\curveto(931.73263118,78.99931736)(931.67263124,79.02431733)(931.58263184,79.05431936)
\curveto(931.50263141,79.08431727)(931.43263148,79.08931727)(931.37263184,79.06931936)
\curveto(931.23263168,79.03931732)(931.14263177,78.97931738)(931.10263184,78.88931936)
\curveto(931.07263184,78.80931755)(931.05763185,78.71931764)(931.05763184,78.61931936)
\curveto(931.05763185,78.51931784)(931.03263188,78.43431792)(930.98263184,78.36431936)
\curveto(930.912632,78.27431808)(930.77263214,78.22931813)(930.56263184,78.22931936)
\lineto(930.00763184,78.22931936)
\lineto(929.78263184,78.22931936)
\curveto(929.70263321,78.23931812)(929.63763327,78.2593181)(929.58763184,78.28931936)
\curveto(929.5076334,78.34931801)(929.46263345,78.41931794)(929.45263184,78.49931936)
\curveto(929.44263347,78.51931784)(929.43763347,78.53931782)(929.43763184,78.55931936)
\curveto(929.43763347,78.58931777)(929.43263348,78.61431774)(929.42263184,78.63431936)
}
}
{
\newrgbcolor{curcolor}{0 0 0}
\pscustom[linestyle=none,fillstyle=solid,fillcolor=curcolor]
{
}
}
{
\newrgbcolor{curcolor}{0 0 0}
\pscustom[linestyle=none,fillstyle=solid,fillcolor=curcolor]
{
\newpath
\moveto(920.45263184,89.26463186)
\curveto(920.44264247,89.95462722)(920.56264235,90.55462662)(920.81263184,91.06463186)
\curveto(921.06264185,91.58462559)(921.39764151,91.9796252)(921.81763184,92.24963186)
\curveto(921.89764101,92.29962488)(921.98764092,92.34462483)(922.08763184,92.38463186)
\curveto(922.17764073,92.42462475)(922.27264064,92.46962471)(922.37263184,92.51963186)
\curveto(922.47264044,92.55962462)(922.57264034,92.58962459)(922.67263184,92.60963186)
\curveto(922.77264014,92.62962455)(922.87764003,92.64962453)(922.98763184,92.66963186)
\curveto(923.03763987,92.68962449)(923.08263983,92.69462448)(923.12263184,92.68463186)
\curveto(923.16263975,92.6746245)(923.2076397,92.6796245)(923.25763184,92.69963186)
\curveto(923.3076396,92.70962447)(923.39263952,92.71462446)(923.51263184,92.71463186)
\curveto(923.62263929,92.71462446)(923.7076392,92.70962447)(923.76763184,92.69963186)
\curveto(923.82763908,92.6796245)(923.88763902,92.66962451)(923.94763184,92.66963186)
\curveto(924.0076389,92.6796245)(924.06763884,92.6746245)(924.12763184,92.65463186)
\curveto(924.26763864,92.61462456)(924.40263851,92.5796246)(924.53263184,92.54963186)
\curveto(924.66263825,92.51962466)(924.78763812,92.4796247)(924.90763184,92.42963186)
\curveto(925.04763786,92.36962481)(925.17263774,92.29962488)(925.28263184,92.21963186)
\curveto(925.39263752,92.14962503)(925.50263741,92.0746251)(925.61263184,91.99463186)
\lineto(925.67263184,91.93463186)
\curveto(925.69263722,91.92462525)(925.7126372,91.90962527)(925.73263184,91.88963186)
\curveto(925.89263702,91.76962541)(926.03763687,91.63462554)(926.16763184,91.48463186)
\curveto(926.29763661,91.33462584)(926.42263649,91.174626)(926.54263184,91.00463186)
\curveto(926.76263615,90.69462648)(926.96763594,90.39962678)(927.15763184,90.11963186)
\curveto(927.29763561,89.88962729)(927.43263548,89.65962752)(927.56263184,89.42963186)
\curveto(927.69263522,89.20962797)(927.82763508,88.98962819)(927.96763184,88.76963186)
\curveto(928.13763477,88.51962866)(928.31763459,88.2796289)(928.50763184,88.04963186)
\curveto(928.69763421,87.82962935)(928.92263399,87.63962954)(929.18263184,87.47963186)
\curveto(929.24263367,87.43962974)(929.30263361,87.40462977)(929.36263184,87.37463186)
\curveto(929.4126335,87.34462983)(929.47763343,87.31462986)(929.55763184,87.28463186)
\curveto(929.62763328,87.26462991)(929.68763322,87.25962992)(929.73763184,87.26963186)
\curveto(929.8076331,87.28962989)(929.86263305,87.32462985)(929.90263184,87.37463186)
\curveto(929.93263298,87.42462975)(929.95263296,87.48462969)(929.96263184,87.55463186)
\lineto(929.96263184,87.79463186)
\lineto(929.96263184,88.54463186)
\lineto(929.96263184,91.34963186)
\lineto(929.96263184,92.00963186)
\curveto(929.96263295,92.09962508)(929.96763294,92.18462499)(929.97763184,92.26463186)
\curveto(929.97763293,92.34462483)(929.99763291,92.40962477)(930.03763184,92.45963186)
\curveto(930.07763283,92.50962467)(930.15263276,92.54962463)(930.26263184,92.57963186)
\curveto(930.36263255,92.61962456)(930.46263245,92.62962455)(930.56263184,92.60963186)
\lineto(930.69763184,92.60963186)
\curveto(930.76763214,92.58962459)(930.82763208,92.56962461)(930.87763184,92.54963186)
\curveto(930.92763198,92.52962465)(930.96763194,92.49462468)(930.99763184,92.44463186)
\curveto(931.03763187,92.39462478)(931.05763185,92.32462485)(931.05763184,92.23463186)
\lineto(931.05763184,91.96463186)
\lineto(931.05763184,91.06463186)
\lineto(931.05763184,87.55463186)
\lineto(931.05763184,86.48963186)
\curveto(931.05763185,86.40963077)(931.06263185,86.31963086)(931.07263184,86.21963186)
\curveto(931.07263184,86.11963106)(931.06263185,86.03463114)(931.04263184,85.96463186)
\curveto(930.97263194,85.75463142)(930.79263212,85.68963149)(930.50263184,85.76963186)
\curveto(930.46263245,85.7796314)(930.42763248,85.7796314)(930.39763184,85.76963186)
\curveto(930.35763255,85.76963141)(930.3126326,85.7796314)(930.26263184,85.79963186)
\curveto(930.18263273,85.81963136)(930.09763281,85.83963134)(930.00763184,85.85963186)
\curveto(929.91763299,85.8796313)(929.83263308,85.90463127)(929.75263184,85.93463186)
\curveto(929.26263365,86.09463108)(928.84763406,86.29463088)(928.50763184,86.53463186)
\curveto(928.25763465,86.71463046)(928.03263488,86.91963026)(927.83263184,87.14963186)
\curveto(927.62263529,87.3796298)(927.42763548,87.61962956)(927.24763184,87.86963186)
\curveto(927.06763584,88.12962905)(926.89763601,88.39462878)(926.73763184,88.66463186)
\curveto(926.56763634,88.94462823)(926.39263652,89.21462796)(926.21263184,89.47463186)
\curveto(926.13263678,89.58462759)(926.05763685,89.68962749)(925.98763184,89.78963186)
\curveto(925.91763699,89.89962728)(925.84263707,90.00962717)(925.76263184,90.11963186)
\curveto(925.73263718,90.15962702)(925.70263721,90.19462698)(925.67263184,90.22463186)
\curveto(925.63263728,90.26462691)(925.60263731,90.30462687)(925.58263184,90.34463186)
\curveto(925.47263744,90.48462669)(925.34763756,90.60962657)(925.20763184,90.71963186)
\curveto(925.17763773,90.73962644)(925.15263776,90.76462641)(925.13263184,90.79463186)
\curveto(925.10263781,90.82462635)(925.07263784,90.84962633)(925.04263184,90.86963186)
\curveto(924.94263797,90.94962623)(924.84263807,91.01462616)(924.74263184,91.06463186)
\curveto(924.64263827,91.12462605)(924.53263838,91.179626)(924.41263184,91.22963186)
\curveto(924.34263857,91.25962592)(924.26763864,91.2796259)(924.18763184,91.28963186)
\lineto(923.94763184,91.34963186)
\lineto(923.85763184,91.34963186)
\curveto(923.82763908,91.35962582)(923.79763911,91.36462581)(923.76763184,91.36463186)
\curveto(923.69763921,91.38462579)(923.60263931,91.38962579)(923.48263184,91.37963186)
\curveto(923.35263956,91.3796258)(923.25263966,91.36962581)(923.18263184,91.34963186)
\curveto(923.10263981,91.32962585)(923.02763988,91.30962587)(922.95763184,91.28963186)
\curveto(922.87764003,91.2796259)(922.79764011,91.25962592)(922.71763184,91.22963186)
\curveto(922.47764043,91.11962606)(922.27764063,90.96962621)(922.11763184,90.77963186)
\curveto(921.94764096,90.59962658)(921.8076411,90.3796268)(921.69763184,90.11963186)
\curveto(921.67764123,90.04962713)(921.66264125,89.9796272)(921.65263184,89.90963186)
\curveto(921.63264128,89.83962734)(921.6126413,89.76462741)(921.59263184,89.68463186)
\curveto(921.57264134,89.60462757)(921.56264135,89.49462768)(921.56263184,89.35463186)
\curveto(921.56264135,89.22462795)(921.57264134,89.11962806)(921.59263184,89.03963186)
\curveto(921.60264131,88.9796282)(921.6076413,88.92462825)(921.60763184,88.87463186)
\curveto(921.6076413,88.82462835)(921.61764129,88.7746284)(921.63763184,88.72463186)
\curveto(921.67764123,88.62462855)(921.71764119,88.52962865)(921.75763184,88.43963186)
\curveto(921.79764111,88.35962882)(921.84264107,88.2796289)(921.89263184,88.19963186)
\curveto(921.912641,88.16962901)(921.93764097,88.13962904)(921.96763184,88.10963186)
\curveto(921.99764091,88.08962909)(922.02264089,88.06462911)(922.04263184,88.03463186)
\lineto(922.11763184,87.95963186)
\curveto(922.13764077,87.92962925)(922.15764075,87.90462927)(922.17763184,87.88463186)
\lineto(922.38763184,87.73463186)
\curveto(922.44764046,87.69462948)(922.5126404,87.64962953)(922.58263184,87.59963186)
\curveto(922.67264024,87.53962964)(922.77764013,87.48962969)(922.89763184,87.44963186)
\curveto(923.0076399,87.41962976)(923.11763979,87.38462979)(923.22763184,87.34463186)
\curveto(923.33763957,87.30462987)(923.48263943,87.2796299)(923.66263184,87.26963186)
\curveto(923.83263908,87.25962992)(923.95763895,87.22962995)(924.03763184,87.17963186)
\curveto(924.11763879,87.12963005)(924.16263875,87.05463012)(924.17263184,86.95463186)
\curveto(924.18263873,86.85463032)(924.18763872,86.74463043)(924.18763184,86.62463186)
\curveto(924.18763872,86.58463059)(924.19263872,86.54463063)(924.20263184,86.50463186)
\curveto(924.20263871,86.46463071)(924.19763871,86.42963075)(924.18763184,86.39963186)
\curveto(924.16763874,86.34963083)(924.15763875,86.29963088)(924.15763184,86.24963186)
\curveto(924.15763875,86.20963097)(924.14763876,86.16963101)(924.12763184,86.12963186)
\curveto(924.06763884,86.03963114)(923.93263898,85.99463118)(923.72263184,85.99463186)
\lineto(923.60263184,85.99463186)
\curveto(923.54263937,86.00463117)(923.48263943,86.00963117)(923.42263184,86.00963186)
\curveto(923.35263956,86.01963116)(923.28763962,86.02963115)(923.22763184,86.03963186)
\curveto(923.11763979,86.05963112)(923.01763989,86.0796311)(922.92763184,86.09963186)
\curveto(922.82764008,86.11963106)(922.73264018,86.14963103)(922.64263184,86.18963186)
\curveto(922.57264034,86.20963097)(922.5126404,86.22963095)(922.46263184,86.24963186)
\lineto(922.28263184,86.30963186)
\curveto(922.02264089,86.42963075)(921.77764113,86.58463059)(921.54763184,86.77463186)
\curveto(921.31764159,86.9746302)(921.13264178,87.18962999)(920.99263184,87.41963186)
\curveto(920.912642,87.52962965)(920.84764206,87.64462953)(920.79763184,87.76463186)
\lineto(920.64763184,88.15463186)
\curveto(920.59764231,88.26462891)(920.56764234,88.3796288)(920.55763184,88.49963186)
\curveto(920.53764237,88.61962856)(920.5126424,88.74462843)(920.48263184,88.87463186)
\curveto(920.48264243,88.94462823)(920.48264243,89.00962817)(920.48263184,89.06963186)
\curveto(920.47264244,89.12962805)(920.46264245,89.19462798)(920.45263184,89.26463186)
}
}
{
\newrgbcolor{curcolor}{0 0 0}
\pscustom[linestyle=none,fillstyle=solid,fillcolor=curcolor]
{
\newpath
\moveto(925.97263184,101.36424123)
\lineto(926.22763184,101.36424123)
\curveto(926.3076366,101.37423353)(926.38263653,101.36923353)(926.45263184,101.34924123)
\lineto(926.69263184,101.34924123)
\lineto(926.85763184,101.34924123)
\curveto(926.95763595,101.32923357)(927.06263585,101.31923358)(927.17263184,101.31924123)
\curveto(927.27263564,101.31923358)(927.37263554,101.30923359)(927.47263184,101.28924123)
\lineto(927.62263184,101.28924123)
\curveto(927.76263515,101.25923364)(927.90263501,101.23923366)(928.04263184,101.22924123)
\curveto(928.17263474,101.21923368)(928.30263461,101.19423371)(928.43263184,101.15424123)
\curveto(928.5126344,101.13423377)(928.59763431,101.11423379)(928.68763184,101.09424123)
\lineto(928.92763184,101.03424123)
\lineto(929.22763184,100.91424123)
\curveto(929.31763359,100.88423402)(929.4076335,100.84923405)(929.49763184,100.80924123)
\curveto(929.71763319,100.70923419)(929.93263298,100.57423433)(930.14263184,100.40424123)
\curveto(930.35263256,100.24423466)(930.52263239,100.06923483)(930.65263184,99.87924123)
\curveto(930.69263222,99.82923507)(930.73263218,99.76923513)(930.77263184,99.69924123)
\curveto(930.80263211,99.63923526)(930.83763207,99.57923532)(930.87763184,99.51924123)
\curveto(930.92763198,99.43923546)(930.96763194,99.34423556)(930.99763184,99.23424123)
\curveto(931.02763188,99.12423578)(931.05763185,99.01923588)(931.08763184,98.91924123)
\curveto(931.12763178,98.80923609)(931.15263176,98.6992362)(931.16263184,98.58924123)
\curveto(931.17263174,98.47923642)(931.18763172,98.36423654)(931.20763184,98.24424123)
\curveto(931.21763169,98.2042367)(931.21763169,98.15923674)(931.20763184,98.10924123)
\curveto(931.2076317,98.06923683)(931.2126317,98.02923687)(931.22263184,97.98924123)
\curveto(931.23263168,97.94923695)(931.23763167,97.89423701)(931.23763184,97.82424123)
\curveto(931.23763167,97.75423715)(931.23263168,97.7042372)(931.22263184,97.67424123)
\curveto(931.20263171,97.62423728)(931.19763171,97.57923732)(931.20763184,97.53924123)
\curveto(931.21763169,97.4992374)(931.21763169,97.46423744)(931.20763184,97.43424123)
\lineto(931.20763184,97.34424123)
\curveto(931.18763172,97.28423762)(931.17263174,97.21923768)(931.16263184,97.14924123)
\curveto(931.16263175,97.08923781)(931.15763175,97.02423788)(931.14763184,96.95424123)
\curveto(931.09763181,96.78423812)(931.04763186,96.62423828)(930.99763184,96.47424123)
\curveto(930.94763196,96.32423858)(930.88263203,96.17923872)(930.80263184,96.03924123)
\curveto(930.76263215,95.98923891)(930.73263218,95.93423897)(930.71263184,95.87424123)
\curveto(930.68263223,95.82423908)(930.64763226,95.77423913)(930.60763184,95.72424123)
\curveto(930.42763248,95.48423942)(930.2076327,95.28423962)(929.94763184,95.12424123)
\curveto(929.68763322,94.96423994)(929.40263351,94.82424008)(929.09263184,94.70424123)
\curveto(928.95263396,94.64424026)(928.8126341,94.5992403)(928.67263184,94.56924123)
\curveto(928.52263439,94.53924036)(928.36763454,94.5042404)(928.20763184,94.46424123)
\curveto(928.09763481,94.44424046)(927.98763492,94.42924047)(927.87763184,94.41924123)
\curveto(927.76763514,94.40924049)(927.65763525,94.39424051)(927.54763184,94.37424123)
\curveto(927.5076354,94.36424054)(927.46763544,94.35924054)(927.42763184,94.35924123)
\curveto(927.38763552,94.36924053)(927.34763556,94.36924053)(927.30763184,94.35924123)
\curveto(927.25763565,94.34924055)(927.2076357,94.34424056)(927.15763184,94.34424123)
\lineto(926.99263184,94.34424123)
\curveto(926.94263597,94.32424058)(926.89263602,94.31924058)(926.84263184,94.32924123)
\curveto(926.78263613,94.33924056)(926.72763618,94.33924056)(926.67763184,94.32924123)
\curveto(926.63763627,94.31924058)(926.59263632,94.31924058)(926.54263184,94.32924123)
\curveto(926.49263642,94.33924056)(926.44263647,94.33424057)(926.39263184,94.31424123)
\curveto(926.32263659,94.29424061)(926.24763666,94.28924061)(926.16763184,94.29924123)
\curveto(926.07763683,94.30924059)(925.99263692,94.31424059)(925.91263184,94.31424123)
\curveto(925.82263709,94.31424059)(925.72263719,94.30924059)(925.61263184,94.29924123)
\curveto(925.49263742,94.28924061)(925.39263752,94.29424061)(925.31263184,94.31424123)
\lineto(925.02763184,94.31424123)
\lineto(924.39763184,94.35924123)
\curveto(924.29763861,94.36924053)(924.20263871,94.37924052)(924.11263184,94.38924123)
\lineto(923.81263184,94.41924123)
\curveto(923.76263915,94.43924046)(923.7126392,94.44424046)(923.66263184,94.43424123)
\curveto(923.60263931,94.43424047)(923.54763936,94.44424046)(923.49763184,94.46424123)
\curveto(923.32763958,94.51424039)(923.16263975,94.55424035)(923.00263184,94.58424123)
\curveto(922.83264008,94.61424029)(922.67264024,94.66424024)(922.52263184,94.73424123)
\curveto(922.06264085,94.92423998)(921.68764122,95.14423976)(921.39763184,95.39424123)
\curveto(921.1076418,95.65423925)(920.86264205,96.01423889)(920.66263184,96.47424123)
\curveto(920.6126423,96.6042383)(920.57764233,96.73423817)(920.55763184,96.86424123)
\curveto(920.53764237,97.0042379)(920.5126424,97.14423776)(920.48263184,97.28424123)
\curveto(920.47264244,97.35423755)(920.46764244,97.41923748)(920.46763184,97.47924123)
\curveto(920.46764244,97.53923736)(920.46264245,97.6042373)(920.45263184,97.67424123)
\curveto(920.43264248,98.5042364)(920.58264233,99.17423573)(920.90263184,99.68424123)
\curveto(921.2126417,100.19423471)(921.65264126,100.57423433)(922.22263184,100.82424123)
\curveto(922.34264057,100.87423403)(922.46764044,100.91923398)(922.59763184,100.95924123)
\curveto(922.72764018,100.9992339)(922.86264005,101.04423386)(923.00263184,101.09424123)
\curveto(923.08263983,101.11423379)(923.16763974,101.12923377)(923.25763184,101.13924123)
\lineto(923.49763184,101.19924123)
\curveto(923.6076393,101.22923367)(923.71763919,101.24423366)(923.82763184,101.24424123)
\curveto(923.93763897,101.25423365)(924.04763886,101.26923363)(924.15763184,101.28924123)
\curveto(924.2076387,101.30923359)(924.25263866,101.31423359)(924.29263184,101.30424123)
\curveto(924.33263858,101.3042336)(924.37263854,101.30923359)(924.41263184,101.31924123)
\curveto(924.46263845,101.32923357)(924.51763839,101.32923357)(924.57763184,101.31924123)
\curveto(924.62763828,101.31923358)(924.67763823,101.32423358)(924.72763184,101.33424123)
\lineto(924.86263184,101.33424123)
\curveto(924.92263799,101.35423355)(924.99263792,101.35423355)(925.07263184,101.33424123)
\curveto(925.14263777,101.32423358)(925.2076377,101.32923357)(925.26763184,101.34924123)
\curveto(925.29763761,101.35923354)(925.33763757,101.36423354)(925.38763184,101.36424123)
\lineto(925.50763184,101.36424123)
\lineto(925.97263184,101.36424123)
\moveto(928.29763184,99.81924123)
\curveto(927.97763493,99.91923498)(927.6126353,99.97923492)(927.20263184,99.99924123)
\curveto(926.79263612,100.01923488)(926.38263653,100.02923487)(925.97263184,100.02924123)
\curveto(925.54263737,100.02923487)(925.12263779,100.01923488)(924.71263184,99.99924123)
\curveto(924.30263861,99.97923492)(923.91763899,99.93423497)(923.55763184,99.86424123)
\curveto(923.19763971,99.79423511)(922.87764003,99.68423522)(922.59763184,99.53424123)
\curveto(922.3076406,99.39423551)(922.07264084,99.1992357)(921.89263184,98.94924123)
\curveto(921.78264113,98.78923611)(921.70264121,98.60923629)(921.65263184,98.40924123)
\curveto(921.59264132,98.20923669)(921.56264135,97.96423694)(921.56263184,97.67424123)
\curveto(921.58264133,97.65423725)(921.59264132,97.61923728)(921.59263184,97.56924123)
\curveto(921.58264133,97.51923738)(921.58264133,97.47923742)(921.59263184,97.44924123)
\curveto(921.6126413,97.36923753)(921.63264128,97.29423761)(921.65263184,97.22424123)
\curveto(921.66264125,97.16423774)(921.68264123,97.0992378)(921.71263184,97.02924123)
\curveto(921.83264108,96.75923814)(922.00264091,96.53923836)(922.22263184,96.36924123)
\curveto(922.43264048,96.20923869)(922.67764023,96.07423883)(922.95763184,95.96424123)
\curveto(923.06763984,95.91423899)(923.18763972,95.87423903)(923.31763184,95.84424123)
\curveto(923.43763947,95.82423908)(923.56263935,95.7992391)(923.69263184,95.76924123)
\curveto(923.74263917,95.74923915)(923.79763911,95.73923916)(923.85763184,95.73924123)
\curveto(923.907639,95.73923916)(923.95763895,95.73423917)(924.00763184,95.72424123)
\curveto(924.09763881,95.71423919)(924.19263872,95.7042392)(924.29263184,95.69424123)
\curveto(924.38263853,95.68423922)(924.47763843,95.67423923)(924.57763184,95.66424123)
\curveto(924.65763825,95.66423924)(924.74263817,95.65923924)(924.83263184,95.64924123)
\lineto(925.07263184,95.64924123)
\lineto(925.25263184,95.64924123)
\curveto(925.28263763,95.63923926)(925.31763759,95.63423927)(925.35763184,95.63424123)
\lineto(925.49263184,95.63424123)
\lineto(925.94263184,95.63424123)
\curveto(926.02263689,95.63423927)(926.1076368,95.62923927)(926.19763184,95.61924123)
\curveto(926.27763663,95.61923928)(926.35263656,95.62923927)(926.42263184,95.64924123)
\lineto(926.69263184,95.64924123)
\curveto(926.7126362,95.64923925)(926.74263617,95.64423926)(926.78263184,95.63424123)
\curveto(926.8126361,95.63423927)(926.83763607,95.63923926)(926.85763184,95.64924123)
\curveto(926.95763595,95.65923924)(927.05763585,95.66423924)(927.15763184,95.66424123)
\curveto(927.24763566,95.67423923)(927.34763556,95.68423922)(927.45763184,95.69424123)
\curveto(927.57763533,95.72423918)(927.70263521,95.73923916)(927.83263184,95.73924123)
\curveto(927.95263496,95.74923915)(928.06763484,95.77423913)(928.17763184,95.81424123)
\curveto(928.47763443,95.89423901)(928.74263417,95.97923892)(928.97263184,96.06924123)
\curveto(929.20263371,96.16923873)(929.41763349,96.31423859)(929.61763184,96.50424123)
\curveto(929.81763309,96.71423819)(929.96763294,96.97923792)(930.06763184,97.29924123)
\curveto(930.08763282,97.33923756)(930.09763281,97.37423753)(930.09763184,97.40424123)
\curveto(930.08763282,97.44423746)(930.09263282,97.48923741)(930.11263184,97.53924123)
\curveto(930.12263279,97.57923732)(930.13263278,97.64923725)(930.14263184,97.74924123)
\curveto(930.15263276,97.85923704)(930.14763276,97.94423696)(930.12763184,98.00424123)
\curveto(930.1076328,98.07423683)(930.09763281,98.14423676)(930.09763184,98.21424123)
\curveto(930.08763282,98.28423662)(930.07263284,98.34923655)(930.05263184,98.40924123)
\curveto(929.99263292,98.60923629)(929.907633,98.78923611)(929.79763184,98.94924123)
\curveto(929.77763313,98.97923592)(929.75763315,99.0042359)(929.73763184,99.02424123)
\lineto(929.67763184,99.08424123)
\curveto(929.65763325,99.12423578)(929.61763329,99.17423573)(929.55763184,99.23424123)
\curveto(929.41763349,99.33423557)(929.28763362,99.41923548)(929.16763184,99.48924123)
\curveto(929.04763386,99.55923534)(928.90263401,99.62923527)(928.73263184,99.69924123)
\curveto(928.66263425,99.72923517)(928.59263432,99.74923515)(928.52263184,99.75924123)
\curveto(928.45263446,99.77923512)(928.37763453,99.7992351)(928.29763184,99.81924123)
}
}
{
\newrgbcolor{curcolor}{0 0 0}
\pscustom[linestyle=none,fillstyle=solid,fillcolor=curcolor]
{
\newpath
\moveto(920.45263184,106.77385061)
\curveto(920.45264246,106.87384575)(920.46264245,106.96884566)(920.48263184,107.05885061)
\curveto(920.49264242,107.14884548)(920.52264239,107.21384541)(920.57263184,107.25385061)
\curveto(920.65264226,107.31384531)(920.75764215,107.34384528)(920.88763184,107.34385061)
\lineto(921.27763184,107.34385061)
\lineto(922.77763184,107.34385061)
\lineto(929.16763184,107.34385061)
\lineto(930.33763184,107.34385061)
\lineto(930.65263184,107.34385061)
\curveto(930.75263216,107.35384527)(930.83263208,107.33884529)(930.89263184,107.29885061)
\curveto(930.97263194,107.24884538)(931.02263189,107.17384545)(931.04263184,107.07385061)
\curveto(931.05263186,106.98384564)(931.05763185,106.87384575)(931.05763184,106.74385061)
\lineto(931.05763184,106.51885061)
\curveto(931.03763187,106.43884619)(931.02263189,106.36884626)(931.01263184,106.30885061)
\curveto(930.99263192,106.24884638)(930.95263196,106.19884643)(930.89263184,106.15885061)
\curveto(930.83263208,106.11884651)(930.75763215,106.09884653)(930.66763184,106.09885061)
\lineto(930.36763184,106.09885061)
\lineto(929.27263184,106.09885061)
\lineto(923.93263184,106.09885061)
\curveto(923.84263907,106.07884655)(923.76763914,106.06384656)(923.70763184,106.05385061)
\curveto(923.63763927,106.05384657)(923.57763933,106.0238466)(923.52763184,105.96385061)
\curveto(923.47763943,105.89384673)(923.45263946,105.80384682)(923.45263184,105.69385061)
\curveto(923.44263947,105.59384703)(923.43763947,105.48384714)(923.43763184,105.36385061)
\lineto(923.43763184,104.22385061)
\lineto(923.43763184,103.72885061)
\curveto(923.42763948,103.56884906)(923.36763954,103.45884917)(923.25763184,103.39885061)
\curveto(923.22763968,103.37884925)(923.19763971,103.36884926)(923.16763184,103.36885061)
\curveto(923.12763978,103.36884926)(923.08263983,103.36384926)(923.03263184,103.35385061)
\curveto(922.91264,103.33384929)(922.80264011,103.33884929)(922.70263184,103.36885061)
\curveto(922.60264031,103.40884922)(922.53264038,103.46384916)(922.49263184,103.53385061)
\curveto(922.44264047,103.61384901)(922.41764049,103.73384889)(922.41763184,103.89385061)
\curveto(922.41764049,104.05384857)(922.40264051,104.18884844)(922.37263184,104.29885061)
\curveto(922.36264055,104.34884828)(922.35764055,104.40384822)(922.35763184,104.46385061)
\curveto(922.34764056,104.5238481)(922.33264058,104.58384804)(922.31263184,104.64385061)
\curveto(922.26264065,104.79384783)(922.2126407,104.93884769)(922.16263184,105.07885061)
\curveto(922.10264081,105.21884741)(922.03264088,105.35384727)(921.95263184,105.48385061)
\curveto(921.86264105,105.623847)(921.75764115,105.74384688)(921.63763184,105.84385061)
\curveto(921.51764139,105.94384668)(921.38764152,106.03884659)(921.24763184,106.12885061)
\curveto(921.14764176,106.18884644)(921.03764187,106.23384639)(920.91763184,106.26385061)
\curveto(920.79764211,106.30384632)(920.69264222,106.35384627)(920.60263184,106.41385061)
\curveto(920.54264237,106.46384616)(920.50264241,106.53384609)(920.48263184,106.62385061)
\curveto(920.47264244,106.64384598)(920.46764244,106.66884596)(920.46763184,106.69885061)
\curveto(920.46764244,106.7288459)(920.46264245,106.75384587)(920.45263184,106.77385061)
}
}
{
\newrgbcolor{curcolor}{0 0 0}
\pscustom[linestyle=none,fillstyle=solid,fillcolor=curcolor]
{
\newpath
\moveto(920.45263184,115.12345998)
\curveto(920.45264246,115.22345513)(920.46264245,115.31845503)(920.48263184,115.40845998)
\curveto(920.49264242,115.49845485)(920.52264239,115.56345479)(920.57263184,115.60345998)
\curveto(920.65264226,115.66345469)(920.75764215,115.69345466)(920.88763184,115.69345998)
\lineto(921.27763184,115.69345998)
\lineto(922.77763184,115.69345998)
\lineto(929.16763184,115.69345998)
\lineto(930.33763184,115.69345998)
\lineto(930.65263184,115.69345998)
\curveto(930.75263216,115.70345465)(930.83263208,115.68845466)(930.89263184,115.64845998)
\curveto(930.97263194,115.59845475)(931.02263189,115.52345483)(931.04263184,115.42345998)
\curveto(931.05263186,115.33345502)(931.05763185,115.22345513)(931.05763184,115.09345998)
\lineto(931.05763184,114.86845998)
\curveto(931.03763187,114.78845556)(931.02263189,114.71845563)(931.01263184,114.65845998)
\curveto(930.99263192,114.59845575)(930.95263196,114.5484558)(930.89263184,114.50845998)
\curveto(930.83263208,114.46845588)(930.75763215,114.4484559)(930.66763184,114.44845998)
\lineto(930.36763184,114.44845998)
\lineto(929.27263184,114.44845998)
\lineto(923.93263184,114.44845998)
\curveto(923.84263907,114.42845592)(923.76763914,114.41345594)(923.70763184,114.40345998)
\curveto(923.63763927,114.40345595)(923.57763933,114.37345598)(923.52763184,114.31345998)
\curveto(923.47763943,114.24345611)(923.45263946,114.1534562)(923.45263184,114.04345998)
\curveto(923.44263947,113.94345641)(923.43763947,113.83345652)(923.43763184,113.71345998)
\lineto(923.43763184,112.57345998)
\lineto(923.43763184,112.07845998)
\curveto(923.42763948,111.91845843)(923.36763954,111.80845854)(923.25763184,111.74845998)
\curveto(923.22763968,111.72845862)(923.19763971,111.71845863)(923.16763184,111.71845998)
\curveto(923.12763978,111.71845863)(923.08263983,111.71345864)(923.03263184,111.70345998)
\curveto(922.91264,111.68345867)(922.80264011,111.68845866)(922.70263184,111.71845998)
\curveto(922.60264031,111.75845859)(922.53264038,111.81345854)(922.49263184,111.88345998)
\curveto(922.44264047,111.96345839)(922.41764049,112.08345827)(922.41763184,112.24345998)
\curveto(922.41764049,112.40345795)(922.40264051,112.53845781)(922.37263184,112.64845998)
\curveto(922.36264055,112.69845765)(922.35764055,112.7534576)(922.35763184,112.81345998)
\curveto(922.34764056,112.87345748)(922.33264058,112.93345742)(922.31263184,112.99345998)
\curveto(922.26264065,113.14345721)(922.2126407,113.28845706)(922.16263184,113.42845998)
\curveto(922.10264081,113.56845678)(922.03264088,113.70345665)(921.95263184,113.83345998)
\curveto(921.86264105,113.97345638)(921.75764115,114.09345626)(921.63763184,114.19345998)
\curveto(921.51764139,114.29345606)(921.38764152,114.38845596)(921.24763184,114.47845998)
\curveto(921.14764176,114.53845581)(921.03764187,114.58345577)(920.91763184,114.61345998)
\curveto(920.79764211,114.6534557)(920.69264222,114.70345565)(920.60263184,114.76345998)
\curveto(920.54264237,114.81345554)(920.50264241,114.88345547)(920.48263184,114.97345998)
\curveto(920.47264244,114.99345536)(920.46764244,115.01845533)(920.46763184,115.04845998)
\curveto(920.46764244,115.07845527)(920.46264245,115.10345525)(920.45263184,115.12345998)
}
}
{
\newrgbcolor{curcolor}{0 0 0}
\pscustom[linestyle=none,fillstyle=solid,fillcolor=curcolor]
{
\newpath
\moveto(941.28894775,42.29681936)
\curveto(941.28895845,42.36681368)(941.28895845,42.4468136)(941.28894775,42.53681936)
\curveto(941.27895846,42.62681342)(941.27895846,42.71181333)(941.28894775,42.79181936)
\curveto(941.28895845,42.88181316)(941.29895844,42.96181308)(941.31894775,43.03181936)
\curveto(941.3389584,43.11181293)(941.36895837,43.16681288)(941.40894775,43.19681936)
\curveto(941.45895828,43.22681282)(941.5339582,43.2468128)(941.63394775,43.25681936)
\curveto(941.72395801,43.27681277)(941.82895791,43.28681276)(941.94894775,43.28681936)
\curveto(942.05895768,43.29681275)(942.17395756,43.29681275)(942.29394775,43.28681936)
\lineto(942.59394775,43.28681936)
\lineto(945.60894775,43.28681936)
\lineto(948.50394775,43.28681936)
\curveto(948.8339509,43.28681276)(949.15895058,43.28181276)(949.47894775,43.27181936)
\curveto(949.78894995,43.27181277)(950.06894967,43.23181281)(950.31894775,43.15181936)
\curveto(950.66894907,43.03181301)(950.96394877,42.87681317)(951.20394775,42.68681936)
\curveto(951.4339483,42.49681355)(951.6339481,42.25681379)(951.80394775,41.96681936)
\curveto(951.85394788,41.90681414)(951.88894785,41.8418142)(951.90894775,41.77181936)
\curveto(951.92894781,41.71181433)(951.95394778,41.6418144)(951.98394775,41.56181936)
\curveto(952.0339477,41.4418146)(952.06894767,41.31181473)(952.08894775,41.17181936)
\curveto(952.11894762,41.041815)(952.14894759,40.90681514)(952.17894775,40.76681936)
\curveto(952.19894754,40.71681533)(952.20394753,40.66681538)(952.19394775,40.61681936)
\curveto(952.18394755,40.56681548)(952.18394755,40.51181553)(952.19394775,40.45181936)
\curveto(952.20394753,40.43181561)(952.20394753,40.40681564)(952.19394775,40.37681936)
\curveto(952.19394754,40.3468157)(952.19894754,40.32181572)(952.20894775,40.30181936)
\curveto(952.21894752,40.26181578)(952.22394751,40.20681584)(952.22394775,40.13681936)
\curveto(952.22394751,40.06681598)(952.21894752,40.01181603)(952.20894775,39.97181936)
\curveto(952.19894754,39.92181612)(952.19894754,39.86681618)(952.20894775,39.80681936)
\curveto(952.21894752,39.7468163)(952.21394752,39.69181635)(952.19394775,39.64181936)
\curveto(952.16394757,39.51181653)(952.14394759,39.38681666)(952.13394775,39.26681936)
\curveto(952.12394761,39.1468169)(952.09894764,39.03181701)(952.05894775,38.92181936)
\curveto(951.9389478,38.55181749)(951.76894797,38.23181781)(951.54894775,37.96181936)
\curveto(951.32894841,37.69181835)(951.04894869,37.48181856)(950.70894775,37.33181936)
\curveto(950.58894915,37.28181876)(950.46394927,37.23681881)(950.33394775,37.19681936)
\curveto(950.20394953,37.16681888)(950.06894967,37.13181891)(949.92894775,37.09181936)
\curveto(949.87894986,37.08181896)(949.8389499,37.07681897)(949.80894775,37.07681936)
\curveto(949.76894997,37.07681897)(949.72395001,37.07181897)(949.67394775,37.06181936)
\curveto(949.64395009,37.05181899)(949.60895013,37.046819)(949.56894775,37.04681936)
\curveto(949.51895022,37.046819)(949.47895026,37.041819)(949.44894775,37.03181936)
\lineto(949.28394775,37.03181936)
\curveto(949.20395053,37.01181903)(949.10395063,37.00681904)(948.98394775,37.01681936)
\curveto(948.85395088,37.02681902)(948.76395097,37.041819)(948.71394775,37.06181936)
\curveto(948.62395111,37.08181896)(948.55895118,37.13681891)(948.51894775,37.22681936)
\curveto(948.49895124,37.25681879)(948.49395124,37.28681876)(948.50394775,37.31681936)
\curveto(948.50395123,37.3468187)(948.49895124,37.38681866)(948.48894775,37.43681936)
\curveto(948.47895126,37.47681857)(948.47395126,37.51681853)(948.47394775,37.55681936)
\lineto(948.47394775,37.70681936)
\curveto(948.47395126,37.82681822)(948.47895126,37.9468181)(948.48894775,38.06681936)
\curveto(948.48895125,38.19681785)(948.52395121,38.28681776)(948.59394775,38.33681936)
\curveto(948.65395108,38.37681767)(948.71395102,38.39681765)(948.77394775,38.39681936)
\curveto(948.8339509,38.39681765)(948.90395083,38.40681764)(948.98394775,38.42681936)
\curveto(949.01395072,38.43681761)(949.04895069,38.43681761)(949.08894775,38.42681936)
\curveto(949.11895062,38.42681762)(949.14395059,38.43181761)(949.16394775,38.44181936)
\lineto(949.37394775,38.44181936)
\curveto(949.42395031,38.46181758)(949.47395026,38.46681758)(949.52394775,38.45681936)
\curveto(949.56395017,38.45681759)(949.60895013,38.46681758)(949.65894775,38.48681936)
\curveto(949.78894995,38.51681753)(949.91394982,38.5468175)(950.03394775,38.57681936)
\curveto(950.14394959,38.60681744)(950.24894949,38.65181739)(950.34894775,38.71181936)
\curveto(950.6389491,38.88181716)(950.84394889,39.15181689)(950.96394775,39.52181936)
\curveto(950.98394875,39.57181647)(950.99894874,39.62181642)(951.00894775,39.67181936)
\curveto(951.00894873,39.73181631)(951.01894872,39.78681626)(951.03894775,39.83681936)
\lineto(951.03894775,39.91181936)
\curveto(951.04894869,39.98181606)(951.05894868,40.07681597)(951.06894775,40.19681936)
\curveto(951.06894867,40.32681572)(951.05894868,40.42681562)(951.03894775,40.49681936)
\curveto(951.01894872,40.56681548)(951.00394873,40.63681541)(950.99394775,40.70681936)
\curveto(950.97394876,40.78681526)(950.95394878,40.85681519)(950.93394775,40.91681936)
\curveto(950.77394896,41.29681475)(950.49894924,41.57181447)(950.10894775,41.74181936)
\curveto(949.97894976,41.79181425)(949.82394991,41.82681422)(949.64394775,41.84681936)
\curveto(949.46395027,41.87681417)(949.27895046,41.89181415)(949.08894775,41.89181936)
\curveto(948.88895085,41.90181414)(948.68895105,41.90181414)(948.48894775,41.89181936)
\lineto(947.91894775,41.89181936)
\lineto(943.67394775,41.89181936)
\lineto(942.12894775,41.89181936)
\curveto(942.01895772,41.89181415)(941.89895784,41.88681416)(941.76894775,41.87681936)
\curveto(941.6389581,41.86681418)(941.5339582,41.88681416)(941.45394775,41.93681936)
\curveto(941.38395835,41.99681405)(941.3339584,42.07681397)(941.30394775,42.17681936)
\curveto(941.30395843,42.19681385)(941.30395843,42.21681383)(941.30394775,42.23681936)
\curveto(941.30395843,42.25681379)(941.29895844,42.27681377)(941.28894775,42.29681936)
}
}
{
\newrgbcolor{curcolor}{0 0 0}
\pscustom[linestyle=none,fillstyle=solid,fillcolor=curcolor]
{
\newpath
\moveto(944.24394775,45.83049123)
\lineto(944.24394775,46.26549123)
\curveto(944.24395549,46.41548927)(944.28395545,46.52048916)(944.36394775,46.58049123)
\curveto(944.44395529,46.63048905)(944.54395519,46.65548903)(944.66394775,46.65549123)
\curveto(944.78395495,46.66548902)(944.90395483,46.67048901)(945.02394775,46.67049123)
\lineto(946.44894775,46.67049123)
\lineto(948.71394775,46.67049123)
\lineto(949.40394775,46.67049123)
\curveto(949.6339501,46.67048901)(949.8339499,46.69548899)(950.00394775,46.74549123)
\curveto(950.45394928,46.90548878)(950.76894897,47.20548848)(950.94894775,47.64549123)
\curveto(951.0389487,47.86548782)(951.07394866,48.13048755)(951.05394775,48.44049123)
\curveto(951.02394871,48.75048693)(950.96894877,49.00048668)(950.88894775,49.19049123)
\curveto(950.74894899,49.52048616)(950.57394916,49.7804859)(950.36394775,49.97049123)
\curveto(950.14394959,50.17048551)(949.85894988,50.32548536)(949.50894775,50.43549123)
\curveto(949.42895031,50.46548522)(949.34895039,50.4854852)(949.26894775,50.49549123)
\curveto(949.18895055,50.50548518)(949.10395063,50.52048516)(949.01394775,50.54049123)
\curveto(948.96395077,50.55048513)(948.91895082,50.55048513)(948.87894775,50.54049123)
\curveto(948.8389509,50.54048514)(948.79395094,50.55048513)(948.74394775,50.57049123)
\lineto(948.42894775,50.57049123)
\curveto(948.34895139,50.59048509)(948.25895148,50.59548509)(948.15894775,50.58549123)
\curveto(948.04895169,50.57548511)(947.94895179,50.57048511)(947.85894775,50.57049123)
\lineto(946.68894775,50.57049123)
\lineto(945.09894775,50.57049123)
\curveto(944.97895476,50.57048511)(944.85395488,50.56548512)(944.72394775,50.55549123)
\curveto(944.58395515,50.55548513)(944.47395526,50.5804851)(944.39394775,50.63049123)
\curveto(944.34395539,50.67048501)(944.31395542,50.71548497)(944.30394775,50.76549123)
\curveto(944.28395545,50.82548486)(944.26395547,50.89548479)(944.24394775,50.97549123)
\lineto(944.24394775,51.20049123)
\curveto(944.24395549,51.32048436)(944.24895549,51.42548426)(944.25894775,51.51549123)
\curveto(944.26895547,51.61548407)(944.31395542,51.69048399)(944.39394775,51.74049123)
\curveto(944.44395529,51.79048389)(944.51895522,51.81548387)(944.61894775,51.81549123)
\lineto(944.90394775,51.81549123)
\lineto(945.92394775,51.81549123)
\lineto(949.95894775,51.81549123)
\lineto(951.30894775,51.81549123)
\curveto(951.42894831,51.81548387)(951.54394819,51.81048387)(951.65394775,51.80049123)
\curveto(951.75394798,51.80048388)(951.82894791,51.76548392)(951.87894775,51.69549123)
\curveto(951.90894783,51.65548403)(951.9339478,51.59548409)(951.95394775,51.51549123)
\curveto(951.96394777,51.43548425)(951.97394776,51.34548434)(951.98394775,51.24549123)
\curveto(951.98394775,51.15548453)(951.97894776,51.06548462)(951.96894775,50.97549123)
\curveto(951.95894778,50.89548479)(951.9389478,50.83548485)(951.90894775,50.79549123)
\curveto(951.86894787,50.74548494)(951.80394793,50.70048498)(951.71394775,50.66049123)
\curveto(951.67394806,50.65048503)(951.61894812,50.64048504)(951.54894775,50.63049123)
\curveto(951.47894826,50.63048505)(951.41394832,50.62548506)(951.35394775,50.61549123)
\curveto(951.28394845,50.60548508)(951.22894851,50.5854851)(951.18894775,50.55549123)
\curveto(951.14894859,50.52548516)(951.1339486,50.4804852)(951.14394775,50.42049123)
\curveto(951.16394857,50.34048534)(951.22394851,50.26048542)(951.32394775,50.18049123)
\curveto(951.41394832,50.10048558)(951.48394825,50.02548566)(951.53394775,49.95549123)
\curveto(951.69394804,49.73548595)(951.8339479,49.4854862)(951.95394775,49.20549123)
\curveto(952.00394773,49.09548659)(952.0339477,48.9804867)(952.04394775,48.86049123)
\curveto(952.06394767,48.75048693)(952.08894765,48.63548705)(952.11894775,48.51549123)
\curveto(952.12894761,48.46548722)(952.12894761,48.41048727)(952.11894775,48.35049123)
\curveto(952.10894763,48.30048738)(952.11394762,48.25048743)(952.13394775,48.20049123)
\curveto(952.15394758,48.10048758)(952.15394758,48.01048767)(952.13394775,47.93049123)
\lineto(952.13394775,47.78049123)
\curveto(952.11394762,47.73048795)(952.10394763,47.67048801)(952.10394775,47.60049123)
\curveto(952.10394763,47.54048814)(952.09894764,47.4854882)(952.08894775,47.43549123)
\curveto(952.06894767,47.39548829)(952.05894768,47.35548833)(952.05894775,47.31549123)
\curveto(952.06894767,47.2854884)(952.06394767,47.24548844)(952.04394775,47.19549123)
\lineto(951.98394775,46.95549123)
\curveto(951.96394777,46.8854888)(951.9339478,46.81048887)(951.89394775,46.73049123)
\curveto(951.78394795,46.47048921)(951.6389481,46.25048943)(951.45894775,46.07049123)
\curveto(951.26894847,45.90048978)(951.04394869,45.76048992)(950.78394775,45.65049123)
\curveto(950.69394904,45.61049007)(950.60394913,45.5804901)(950.51394775,45.56049123)
\lineto(950.21394775,45.50049123)
\curveto(950.15394958,45.4804902)(950.09894964,45.47049021)(950.04894775,45.47049123)
\curveto(949.98894975,45.4804902)(949.92394981,45.47549021)(949.85394775,45.45549123)
\curveto(949.8339499,45.44549024)(949.80894993,45.44049024)(949.77894775,45.44049123)
\curveto(949.73895,45.44049024)(949.70395003,45.43549025)(949.67394775,45.42549123)
\lineto(949.52394775,45.42549123)
\curveto(949.48395025,45.41549027)(949.4389503,45.41049027)(949.38894775,45.41049123)
\curveto(949.32895041,45.42049026)(949.27395046,45.42549026)(949.22394775,45.42549123)
\lineto(948.62394775,45.42549123)
\lineto(945.86394775,45.42549123)
\lineto(944.90394775,45.42549123)
\lineto(944.63394775,45.42549123)
\curveto(944.54395519,45.42549026)(944.46895527,45.44549024)(944.40894775,45.48549123)
\curveto(944.3389554,45.52549016)(944.28895545,45.60049008)(944.25894775,45.71049123)
\curveto(944.24895549,45.73048995)(944.24895549,45.75048993)(944.25894775,45.77049123)
\curveto(944.25895548,45.79048989)(944.25395548,45.81048987)(944.24394775,45.83049123)
}
}
{
\newrgbcolor{curcolor}{0 0 0}
\pscustom[linestyle=none,fillstyle=solid,fillcolor=curcolor]
{
\newpath
\moveto(941.28894775,54.28510061)
\curveto(941.28895845,54.41509899)(941.28895845,54.55009886)(941.28894775,54.69010061)
\curveto(941.28895845,54.84009857)(941.32395841,54.95009846)(941.39394775,55.02010061)
\curveto(941.46395827,55.07009834)(941.55895818,55.09509831)(941.67894775,55.09510061)
\curveto(941.78895795,55.1050983)(941.90395783,55.1100983)(942.02394775,55.11010061)
\lineto(943.35894775,55.11010061)
\lineto(949.43394775,55.11010061)
\lineto(951.11394775,55.11010061)
\lineto(951.50394775,55.11010061)
\curveto(951.64394809,55.1100983)(951.75394798,55.08509832)(951.83394775,55.03510061)
\curveto(951.88394785,55.0050984)(951.91394782,54.96009845)(951.92394775,54.90010061)
\curveto(951.9339478,54.85009856)(951.94894779,54.78509862)(951.96894775,54.70510061)
\lineto(951.96894775,54.49510061)
\lineto(951.96894775,54.18010061)
\curveto(951.95894778,54.08009933)(951.92394781,54.0050994)(951.86394775,53.95510061)
\curveto(951.78394795,53.9050995)(951.68394805,53.87509953)(951.56394775,53.86510061)
\lineto(951.18894775,53.86510061)
\lineto(949.80894775,53.86510061)
\lineto(943.56894775,53.86510061)
\lineto(942.09894775,53.86510061)
\curveto(941.98895775,53.86509954)(941.87395786,53.86009955)(941.75394775,53.85010061)
\curveto(941.62395811,53.85009956)(941.52395821,53.87509953)(941.45394775,53.92510061)
\curveto(941.39395834,53.96509944)(941.34395839,54.04009937)(941.30394775,54.15010061)
\curveto(941.29395844,54.17009924)(941.29395844,54.19009922)(941.30394775,54.21010061)
\curveto(941.30395843,54.24009917)(941.29895844,54.26509914)(941.28894775,54.28510061)
}
}
{
\newrgbcolor{curcolor}{0 0 0}
\pscustom[linestyle=none,fillstyle=solid,fillcolor=curcolor]
{
}
}
{
\newrgbcolor{curcolor}{0 0 0}
\pscustom[linestyle=none,fillstyle=solid,fillcolor=curcolor]
{
\newpath
\moveto(941.36394775,64.24510061)
\curveto(941.35395838,64.93509597)(941.47395826,65.53509537)(941.72394775,66.04510061)
\curveto(941.97395776,66.56509434)(942.30895743,66.96009395)(942.72894775,67.23010061)
\curveto(942.80895693,67.28009363)(942.89895684,67.32509358)(942.99894775,67.36510061)
\curveto(943.08895665,67.4050935)(943.18395655,67.45009346)(943.28394775,67.50010061)
\curveto(943.38395635,67.54009337)(943.48395625,67.57009334)(943.58394775,67.59010061)
\curveto(943.68395605,67.6100933)(943.78895595,67.63009328)(943.89894775,67.65010061)
\curveto(943.94895579,67.67009324)(943.99395574,67.67509323)(944.03394775,67.66510061)
\curveto(944.07395566,67.65509325)(944.11895562,67.66009325)(944.16894775,67.68010061)
\curveto(944.21895552,67.69009322)(944.30395543,67.69509321)(944.42394775,67.69510061)
\curveto(944.5339552,67.69509321)(944.61895512,67.69009322)(944.67894775,67.68010061)
\curveto(944.738955,67.66009325)(944.79895494,67.65009326)(944.85894775,67.65010061)
\curveto(944.91895482,67.66009325)(944.97895476,67.65509325)(945.03894775,67.63510061)
\curveto(945.17895456,67.59509331)(945.31395442,67.56009335)(945.44394775,67.53010061)
\curveto(945.57395416,67.50009341)(945.69895404,67.46009345)(945.81894775,67.41010061)
\curveto(945.95895378,67.35009356)(946.08395365,67.28009363)(946.19394775,67.20010061)
\curveto(946.30395343,67.13009378)(946.41395332,67.05509385)(946.52394775,66.97510061)
\lineto(946.58394775,66.91510061)
\curveto(946.60395313,66.905094)(946.62395311,66.89009402)(946.64394775,66.87010061)
\curveto(946.80395293,66.75009416)(946.94895279,66.61509429)(947.07894775,66.46510061)
\curveto(947.20895253,66.31509459)(947.3339524,66.15509475)(947.45394775,65.98510061)
\curveto(947.67395206,65.67509523)(947.87895186,65.38009553)(948.06894775,65.10010061)
\curveto(948.20895153,64.87009604)(948.34395139,64.64009627)(948.47394775,64.41010061)
\curveto(948.60395113,64.19009672)(948.738951,63.97009694)(948.87894775,63.75010061)
\curveto(949.04895069,63.50009741)(949.22895051,63.26009765)(949.41894775,63.03010061)
\curveto(949.60895013,62.8100981)(949.8339499,62.62009829)(950.09394775,62.46010061)
\curveto(950.15394958,62.42009849)(950.21394952,62.38509852)(950.27394775,62.35510061)
\curveto(950.32394941,62.32509858)(950.38894935,62.29509861)(950.46894775,62.26510061)
\curveto(950.5389492,62.24509866)(950.59894914,62.24009867)(950.64894775,62.25010061)
\curveto(950.71894902,62.27009864)(950.77394896,62.3050986)(950.81394775,62.35510061)
\curveto(950.84394889,62.4050985)(950.86394887,62.46509844)(950.87394775,62.53510061)
\lineto(950.87394775,62.77510061)
\lineto(950.87394775,63.52510061)
\lineto(950.87394775,66.33010061)
\lineto(950.87394775,66.99010061)
\curveto(950.87394886,67.08009383)(950.87894886,67.16509374)(950.88894775,67.24510061)
\curveto(950.88894885,67.32509358)(950.90894883,67.39009352)(950.94894775,67.44010061)
\curveto(950.98894875,67.49009342)(951.06394867,67.53009338)(951.17394775,67.56010061)
\curveto(951.27394846,67.60009331)(951.37394836,67.6100933)(951.47394775,67.59010061)
\lineto(951.60894775,67.59010061)
\curveto(951.67894806,67.57009334)(951.738948,67.55009336)(951.78894775,67.53010061)
\curveto(951.8389479,67.5100934)(951.87894786,67.47509343)(951.90894775,67.42510061)
\curveto(951.94894779,67.37509353)(951.96894777,67.3050936)(951.96894775,67.21510061)
\lineto(951.96894775,66.94510061)
\lineto(951.96894775,66.04510061)
\lineto(951.96894775,62.53510061)
\lineto(951.96894775,61.47010061)
\curveto(951.96894777,61.39009952)(951.97394776,61.30009961)(951.98394775,61.20010061)
\curveto(951.98394775,61.10009981)(951.97394776,61.01509989)(951.95394775,60.94510061)
\curveto(951.88394785,60.73510017)(951.70394803,60.67010024)(951.41394775,60.75010061)
\curveto(951.37394836,60.76010015)(951.3389484,60.76010015)(951.30894775,60.75010061)
\curveto(951.26894847,60.75010016)(951.22394851,60.76010015)(951.17394775,60.78010061)
\curveto(951.09394864,60.80010011)(951.00894873,60.82010009)(950.91894775,60.84010061)
\curveto(950.82894891,60.86010005)(950.74394899,60.88510002)(950.66394775,60.91510061)
\curveto(950.17394956,61.07509983)(949.75894998,61.27509963)(949.41894775,61.51510061)
\curveto(949.16895057,61.69509921)(948.94395079,61.90009901)(948.74394775,62.13010061)
\curveto(948.5339512,62.36009855)(948.3389514,62.60009831)(948.15894775,62.85010061)
\curveto(947.97895176,63.1100978)(947.80895193,63.37509753)(947.64894775,63.64510061)
\curveto(947.47895226,63.92509698)(947.30395243,64.19509671)(947.12394775,64.45510061)
\curveto(947.04395269,64.56509634)(946.96895277,64.67009624)(946.89894775,64.77010061)
\curveto(946.82895291,64.88009603)(946.75395298,64.99009592)(946.67394775,65.10010061)
\curveto(946.64395309,65.14009577)(946.61395312,65.17509573)(946.58394775,65.20510061)
\curveto(946.54395319,65.24509566)(946.51395322,65.28509562)(946.49394775,65.32510061)
\curveto(946.38395335,65.46509544)(946.25895348,65.59009532)(946.11894775,65.70010061)
\curveto(946.08895365,65.72009519)(946.06395367,65.74509516)(946.04394775,65.77510061)
\curveto(946.01395372,65.8050951)(945.98395375,65.83009508)(945.95394775,65.85010061)
\curveto(945.85395388,65.93009498)(945.75395398,65.99509491)(945.65394775,66.04510061)
\curveto(945.55395418,66.1050948)(945.44395429,66.16009475)(945.32394775,66.21010061)
\curveto(945.25395448,66.24009467)(945.17895456,66.26009465)(945.09894775,66.27010061)
\lineto(944.85894775,66.33010061)
\lineto(944.76894775,66.33010061)
\curveto(944.738955,66.34009457)(944.70895503,66.34509456)(944.67894775,66.34510061)
\curveto(944.60895513,66.36509454)(944.51395522,66.37009454)(944.39394775,66.36010061)
\curveto(944.26395547,66.36009455)(944.16395557,66.35009456)(944.09394775,66.33010061)
\curveto(944.01395572,66.3100946)(943.9389558,66.29009462)(943.86894775,66.27010061)
\curveto(943.78895595,66.26009465)(943.70895603,66.24009467)(943.62894775,66.21010061)
\curveto(943.38895635,66.10009481)(943.18895655,65.95009496)(943.02894775,65.76010061)
\curveto(942.85895688,65.58009533)(942.71895702,65.36009555)(942.60894775,65.10010061)
\curveto(942.58895715,65.03009588)(942.57395716,64.96009595)(942.56394775,64.89010061)
\curveto(942.54395719,64.82009609)(942.52395721,64.74509616)(942.50394775,64.66510061)
\curveto(942.48395725,64.58509632)(942.47395726,64.47509643)(942.47394775,64.33510061)
\curveto(942.47395726,64.2050967)(942.48395725,64.10009681)(942.50394775,64.02010061)
\curveto(942.51395722,63.96009695)(942.51895722,63.905097)(942.51894775,63.85510061)
\curveto(942.51895722,63.8050971)(942.52895721,63.75509715)(942.54894775,63.70510061)
\curveto(942.58895715,63.6050973)(942.62895711,63.5100974)(942.66894775,63.42010061)
\curveto(942.70895703,63.34009757)(942.75395698,63.26009765)(942.80394775,63.18010061)
\curveto(942.82395691,63.15009776)(942.84895689,63.12009779)(942.87894775,63.09010061)
\curveto(942.90895683,63.07009784)(942.9339568,63.04509786)(942.95394775,63.01510061)
\lineto(943.02894775,62.94010061)
\curveto(943.04895669,62.910098)(943.06895667,62.88509802)(943.08894775,62.86510061)
\lineto(943.29894775,62.71510061)
\curveto(943.35895638,62.67509823)(943.42395631,62.63009828)(943.49394775,62.58010061)
\curveto(943.58395615,62.52009839)(943.68895605,62.47009844)(943.80894775,62.43010061)
\curveto(943.91895582,62.40009851)(944.02895571,62.36509854)(944.13894775,62.32510061)
\curveto(944.24895549,62.28509862)(944.39395534,62.26009865)(944.57394775,62.25010061)
\curveto(944.74395499,62.24009867)(944.86895487,62.2100987)(944.94894775,62.16010061)
\curveto(945.02895471,62.1100988)(945.07395466,62.03509887)(945.08394775,61.93510061)
\curveto(945.09395464,61.83509907)(945.09895464,61.72509918)(945.09894775,61.60510061)
\curveto(945.09895464,61.56509934)(945.10395463,61.52509938)(945.11394775,61.48510061)
\curveto(945.11395462,61.44509946)(945.10895463,61.4100995)(945.09894775,61.38010061)
\curveto(945.07895466,61.33009958)(945.06895467,61.28009963)(945.06894775,61.23010061)
\curveto(945.06895467,61.19009972)(945.05895468,61.15009976)(945.03894775,61.11010061)
\curveto(944.97895476,61.02009989)(944.84395489,60.97509993)(944.63394775,60.97510061)
\lineto(944.51394775,60.97510061)
\curveto(944.45395528,60.98509992)(944.39395534,60.99009992)(944.33394775,60.99010061)
\curveto(944.26395547,61.00009991)(944.19895554,61.0100999)(944.13894775,61.02010061)
\curveto(944.02895571,61.04009987)(943.92895581,61.06009985)(943.83894775,61.08010061)
\curveto(943.738956,61.10009981)(943.64395609,61.13009978)(943.55394775,61.17010061)
\curveto(943.48395625,61.19009972)(943.42395631,61.2100997)(943.37394775,61.23010061)
\lineto(943.19394775,61.29010061)
\curveto(942.9339568,61.4100995)(942.68895705,61.56509934)(942.45894775,61.75510061)
\curveto(942.22895751,61.95509895)(942.04395769,62.17009874)(941.90394775,62.40010061)
\curveto(941.82395791,62.5100984)(941.75895798,62.62509828)(941.70894775,62.74510061)
\lineto(941.55894775,63.13510061)
\curveto(941.50895823,63.24509766)(941.47895826,63.36009755)(941.46894775,63.48010061)
\curveto(941.44895829,63.60009731)(941.42395831,63.72509718)(941.39394775,63.85510061)
\curveto(941.39395834,63.92509698)(941.39395834,63.99009692)(941.39394775,64.05010061)
\curveto(941.38395835,64.1100968)(941.37395836,64.17509673)(941.36394775,64.24510061)
}
}
{
\newrgbcolor{curcolor}{0 0 0}
\pscustom[linestyle=none,fillstyle=solid,fillcolor=curcolor]
{
\newpath
\moveto(948.89394775,76.35970998)
\curveto(948.9339508,76.36970226)(948.98395075,76.36970226)(949.04394775,76.35970998)
\curveto(949.10395063,76.35970227)(949.15395058,76.35470228)(949.19394775,76.34470998)
\curveto(949.2339505,76.34470229)(949.27395046,76.33970229)(949.31394775,76.32970998)
\lineto(949.41894775,76.32970998)
\curveto(949.49895024,76.30970232)(949.57895016,76.29470234)(949.65894775,76.28470998)
\curveto(949.73895,76.27470236)(949.81394992,76.25470238)(949.88394775,76.22470998)
\curveto(949.96394977,76.20470243)(950.0389497,76.18470245)(950.10894775,76.16470998)
\curveto(950.17894956,76.14470249)(950.25394948,76.11470252)(950.33394775,76.07470998)
\curveto(950.75394898,75.89470274)(951.09394864,75.63970299)(951.35394775,75.30970998)
\curveto(951.61394812,74.97970365)(951.81894792,74.58970404)(951.96894775,74.13970998)
\curveto(952.00894773,74.01970461)(952.0339477,73.89470474)(952.04394775,73.76470998)
\curveto(952.06394767,73.64470499)(952.08894765,73.51970511)(952.11894775,73.38970998)
\curveto(952.12894761,73.3297053)(952.1339476,73.26470537)(952.13394775,73.19470998)
\curveto(952.1339476,73.1347055)(952.1389476,73.06970556)(952.14894775,72.99970998)
\lineto(952.14894775,72.87970998)
\lineto(952.14894775,72.68470998)
\curveto(952.15894758,72.62470601)(952.15394758,72.56970606)(952.13394775,72.51970998)
\curveto(952.11394762,72.44970618)(952.10894763,72.38470625)(952.11894775,72.32470998)
\curveto(952.12894761,72.26470637)(952.12394761,72.20470643)(952.10394775,72.14470998)
\curveto(952.09394764,72.09470654)(952.08894765,72.04970658)(952.08894775,72.00970998)
\curveto(952.08894765,71.96970666)(952.07894766,71.92470671)(952.05894775,71.87470998)
\curveto(952.0389477,71.79470684)(952.01894772,71.71970691)(951.99894775,71.64970998)
\curveto(951.98894775,71.57970705)(951.97394776,71.50970712)(951.95394775,71.43970998)
\curveto(951.78394795,70.95970767)(951.57394816,70.55970807)(951.32394775,70.23970998)
\curveto(951.06394867,69.9297087)(950.70894903,69.67970895)(950.25894775,69.48970998)
\curveto(950.19894954,69.45970917)(950.1389496,69.4347092)(950.07894775,69.41470998)
\curveto(950.00894973,69.40470923)(949.9339498,69.38970924)(949.85394775,69.36970998)
\curveto(949.79394994,69.34970928)(949.72895001,69.3347093)(949.65894775,69.32470998)
\curveto(949.58895015,69.31470932)(949.51895022,69.29970933)(949.44894775,69.27970998)
\curveto(949.39895034,69.26970936)(949.35895038,69.26470937)(949.32894775,69.26470998)
\lineto(949.20894775,69.26470998)
\curveto(949.16895057,69.25470938)(949.11895062,69.24470939)(949.05894775,69.23470998)
\curveto(948.99895074,69.2347094)(948.94895079,69.23970939)(948.90894775,69.24970998)
\lineto(948.77394775,69.24970998)
\curveto(948.72395101,69.25970937)(948.67395106,69.26470937)(948.62394775,69.26470998)
\curveto(948.52395121,69.28470935)(948.42895131,69.29970933)(948.33894775,69.30970998)
\curveto(948.2389515,69.31970931)(948.14395159,69.33970929)(948.05394775,69.36970998)
\curveto(947.90395183,69.41970921)(947.76395197,69.47470916)(947.63394775,69.53470998)
\curveto(947.50395223,69.59470904)(947.38395235,69.66470897)(947.27394775,69.74470998)
\curveto(947.22395251,69.77470886)(947.18395255,69.80470883)(947.15394775,69.83470998)
\curveto(947.12395261,69.87470876)(947.08895265,69.90970872)(947.04894775,69.93970998)
\curveto(946.96895277,69.99970863)(946.89895284,70.06970856)(946.83894775,70.14970998)
\curveto(946.78895295,70.20970842)(946.74395299,70.26970836)(946.70394775,70.32970998)
\lineto(946.55394775,70.53970998)
\curveto(946.51395322,70.58970804)(946.47895326,70.63970799)(946.44894775,70.68970998)
\curveto(946.40895333,70.73970789)(946.35395338,70.77470786)(946.28394775,70.79470998)
\curveto(946.25395348,70.79470784)(946.22895351,70.78470785)(946.20894775,70.76470998)
\curveto(946.17895356,70.75470788)(946.15395358,70.74470789)(946.13394775,70.73470998)
\curveto(946.08395365,70.69470794)(946.0389537,70.64470799)(945.99894775,70.58470998)
\curveto(945.94895379,70.5347081)(945.90395383,70.48470815)(945.86394775,70.43470998)
\curveto(945.8339539,70.39470824)(945.77895396,70.34470829)(945.69894775,70.28470998)
\curveto(945.66895407,70.26470837)(945.64395409,70.2347084)(945.62394775,70.19470998)
\curveto(945.59395414,70.16470847)(945.55895418,70.13970849)(945.51894775,70.11970998)
\curveto(945.30895443,69.94970868)(945.06395467,69.81970881)(944.78394775,69.72970998)
\curveto(944.70395503,69.70970892)(944.62395511,69.69470894)(944.54394775,69.68470998)
\curveto(944.46395527,69.67470896)(944.38395535,69.65970897)(944.30394775,69.63970998)
\curveto(944.25395548,69.61970901)(944.18895555,69.60970902)(944.10894775,69.60970998)
\curveto(944.01895572,69.60970902)(943.94895579,69.61970901)(943.89894775,69.63970998)
\curveto(943.79895594,69.63970899)(943.72895601,69.64470899)(943.68894775,69.65470998)
\curveto(943.60895613,69.67470896)(943.5389562,69.68970894)(943.47894775,69.69970998)
\curveto(943.40895633,69.70970892)(943.3389564,69.72470891)(943.26894775,69.74470998)
\curveto(942.8389569,69.89470874)(942.49395724,70.10970852)(942.23394775,70.38970998)
\curveto(941.97395776,70.67970795)(941.75895798,71.0297076)(941.58894775,71.43970998)
\curveto(941.5389582,71.54970708)(941.50895823,71.66470697)(941.49894775,71.78470998)
\curveto(941.47895826,71.91470672)(941.44895829,72.04470659)(941.40894775,72.17470998)
\curveto(941.40895833,72.25470638)(941.40895833,72.32470631)(941.40894775,72.38470998)
\curveto(941.39895834,72.45470618)(941.38895835,72.5297061)(941.37894775,72.60970998)
\curveto(941.35895838,73.39970523)(941.48895825,74.05470458)(941.76894775,74.57470998)
\curveto(942.04895769,75.10470353)(942.45895728,75.48470315)(942.99894775,75.71470998)
\curveto(943.22895651,75.82470281)(943.51395622,75.89470274)(943.85394775,75.92470998)
\curveto(944.18395555,75.96470267)(944.48895525,75.9347027)(944.76894775,75.83470998)
\curveto(944.89895484,75.79470284)(945.01895472,75.74470289)(945.12894775,75.68470998)
\curveto(945.2389545,75.634703)(945.34395439,75.57470306)(945.44394775,75.50470998)
\curveto(945.48395425,75.48470315)(945.51895422,75.45470318)(945.54894775,75.41470998)
\lineto(945.63894775,75.32470998)
\curveto(945.72895401,75.27470336)(945.79395394,75.21470342)(945.83394775,75.14470998)
\curveto(945.88395385,75.09470354)(945.9339538,75.03970359)(945.98394775,74.97970998)
\curveto(946.02395371,74.9297037)(946.06895367,74.88470375)(946.11894775,74.84470998)
\curveto(946.1389536,74.82470381)(946.16395357,74.80470383)(946.19394775,74.78470998)
\curveto(946.21395352,74.77470386)(946.2389535,74.77470386)(946.26894775,74.78470998)
\curveto(946.31895342,74.79470384)(946.36895337,74.82470381)(946.41894775,74.87470998)
\curveto(946.45895328,74.92470371)(946.49895324,74.97970365)(946.53894775,75.03970998)
\lineto(946.65894775,75.21970998)
\curveto(946.68895305,75.27970335)(946.71895302,75.3297033)(946.74894775,75.36970998)
\curveto(946.98895275,75.69970293)(947.29895244,75.94970268)(947.67894775,76.11970998)
\curveto(947.75895198,76.15970247)(947.84395189,76.18970244)(947.93394775,76.20970998)
\curveto(948.02395171,76.23970239)(948.11395162,76.26470237)(948.20394775,76.28470998)
\curveto(948.25395148,76.29470234)(948.30895143,76.30470233)(948.36894775,76.31470998)
\lineto(948.51894775,76.34470998)
\curveto(948.57895116,76.35470228)(948.64395109,76.35470228)(948.71394775,76.34470998)
\curveto(948.77395096,76.3347023)(948.8339509,76.33970229)(948.89394775,76.35970998)
\moveto(943.85394775,70.97470998)
\curveto(943.96395577,70.94470769)(944.10395563,70.93970769)(944.27394775,70.95970998)
\curveto(944.4339553,70.97970765)(944.55895518,71.00470763)(944.64894775,71.03470998)
\curveto(944.96895477,71.14470749)(945.21395452,71.29470734)(945.38394775,71.48470998)
\curveto(945.54395419,71.67470696)(945.67395406,71.93970669)(945.77394775,72.27970998)
\curveto(945.80395393,72.40970622)(945.82895391,72.57470606)(945.84894775,72.77470998)
\curveto(945.85895388,72.97470566)(945.84395389,73.14470549)(945.80394775,73.28470998)
\curveto(945.72395401,73.57470506)(945.61395412,73.81470482)(945.47394775,74.00470998)
\curveto(945.32395441,74.20470443)(945.12395461,74.35970427)(944.87394775,74.46970998)
\curveto(944.82395491,74.48970414)(944.77895496,74.49970413)(944.73894775,74.49970998)
\curveto(944.69895504,74.50970412)(944.65395508,74.52470411)(944.60394775,74.54470998)
\curveto(944.49395524,74.57470406)(944.35395538,74.59470404)(944.18394775,74.60470998)
\curveto(944.01395572,74.61470402)(943.86895587,74.60470403)(943.74894775,74.57470998)
\curveto(943.65895608,74.55470408)(943.57395616,74.5297041)(943.49394775,74.49970998)
\curveto(943.41395632,74.47970415)(943.3339564,74.44470419)(943.25394775,74.39470998)
\curveto(942.98395675,74.22470441)(942.78895695,73.99970463)(942.66894775,73.71970998)
\curveto(942.54895719,73.44970518)(942.48895725,73.08970554)(942.48894775,72.63970998)
\curveto(942.50895723,72.61970601)(942.51395722,72.58970604)(942.50394775,72.54970998)
\curveto(942.49395724,72.50970612)(942.49395724,72.47470616)(942.50394775,72.44470998)
\curveto(942.52395721,72.39470624)(942.5389572,72.33970629)(942.54894775,72.27970998)
\curveto(942.54895719,72.2297064)(942.55895718,72.17970645)(942.57894775,72.12970998)
\curveto(942.66895707,71.88970674)(942.78395695,71.67970695)(942.92394775,71.49970998)
\curveto(943.05395668,71.31970731)(943.2339565,71.17970745)(943.46394775,71.07970998)
\curveto(943.52395621,71.05970757)(943.58895615,71.03970759)(943.65894775,71.01970998)
\curveto(943.71895602,71.00970762)(943.78395595,70.99470764)(943.85394775,70.97470998)
\moveto(949.38894775,74.99470998)
\curveto(949.19895054,75.04470359)(948.99395074,75.04970358)(948.77394775,75.00970998)
\curveto(948.55395118,74.97970365)(948.37395136,74.9347037)(948.23394775,74.87470998)
\curveto(947.86395187,74.70470393)(947.55895218,74.44470419)(947.31894775,74.09470998)
\curveto(947.07895266,73.75470488)(946.95895278,73.31970531)(946.95894775,72.78970998)
\curveto(946.97895276,72.75970587)(946.98395275,72.71970591)(946.97394775,72.66970998)
\curveto(946.95395278,72.61970601)(946.94895279,72.57970605)(946.95894775,72.54970998)
\lineto(947.01894775,72.27970998)
\curveto(947.02895271,72.19970643)(947.04395269,72.11970651)(947.06394775,72.03970998)
\curveto(947.17395256,71.73970689)(947.31895242,71.47470716)(947.49894775,71.24470998)
\curveto(947.67895206,71.02470761)(947.90895183,70.85470778)(948.18894775,70.73470998)
\curveto(948.26895147,70.70470793)(948.34895139,70.67970795)(948.42894775,70.65970998)
\curveto(948.50895123,70.63970799)(948.59395114,70.61970801)(948.68394775,70.59970998)
\curveto(948.80395093,70.56970806)(948.95395078,70.55970807)(949.13394775,70.56970998)
\curveto(949.31395042,70.58970804)(949.45395028,70.61470802)(949.55394775,70.64470998)
\curveto(949.60395013,70.66470797)(949.64895009,70.67470796)(949.68894775,70.67470998)
\curveto(949.71895002,70.68470795)(949.75894998,70.69970793)(949.80894775,70.71970998)
\curveto(950.02894971,70.81970781)(950.22894951,70.94970768)(950.40894775,71.10970998)
\curveto(950.58894915,71.27970735)(950.72394901,71.47470716)(950.81394775,71.69470998)
\curveto(950.85394888,71.76470687)(950.88894885,71.85970677)(950.91894775,71.97970998)
\curveto(951.00894873,72.19970643)(951.05394868,72.45470618)(951.05394775,72.74470998)
\lineto(951.05394775,73.02970998)
\curveto(951.0339487,73.1297055)(951.01894872,73.22470541)(951.00894775,73.31470998)
\curveto(950.99894874,73.40470523)(950.97894876,73.49470514)(950.94894775,73.58470998)
\curveto(950.86894887,73.84470479)(950.738949,74.08470455)(950.55894775,74.30470998)
\curveto(950.36894937,74.5347041)(950.15394958,74.70470393)(949.91394775,74.81470998)
\curveto(949.8339499,74.85470378)(949.75394998,74.88470375)(949.67394775,74.90470998)
\curveto(949.58395015,74.9347037)(949.48895025,74.96470367)(949.38894775,74.99470998)
}
}
{
\newrgbcolor{curcolor}{0 0 0}
\pscustom[linestyle=none,fillstyle=solid,fillcolor=curcolor]
{
\newpath
\moveto(950.33394775,78.63431936)
\lineto(950.33394775,79.26431936)
\lineto(950.33394775,79.45931936)
\curveto(950.3339494,79.52931683)(950.34394939,79.58931677)(950.36394775,79.63931936)
\curveto(950.40394933,79.70931665)(950.44394929,79.7593166)(950.48394775,79.78931936)
\curveto(950.5339492,79.82931653)(950.59894914,79.84931651)(950.67894775,79.84931936)
\curveto(950.75894898,79.8593165)(950.84394889,79.86431649)(950.93394775,79.86431936)
\lineto(951.65394775,79.86431936)
\curveto(952.1339476,79.86431649)(952.54394719,79.80431655)(952.88394775,79.68431936)
\curveto(953.22394651,79.56431679)(953.49894624,79.36931699)(953.70894775,79.09931936)
\curveto(953.75894598,79.02931733)(953.80394593,78.9593174)(953.84394775,78.88931936)
\curveto(953.89394584,78.82931753)(953.9389458,78.7543176)(953.97894775,78.66431936)
\curveto(953.98894575,78.64431771)(953.99894574,78.61431774)(954.00894775,78.57431936)
\curveto(954.02894571,78.53431782)(954.0339457,78.48931787)(954.02394775,78.43931936)
\curveto(953.99394574,78.34931801)(953.91894582,78.29431806)(953.79894775,78.27431936)
\curveto(953.68894605,78.2543181)(953.59394614,78.26931809)(953.51394775,78.31931936)
\curveto(953.44394629,78.34931801)(953.37894636,78.39431796)(953.31894775,78.45431936)
\curveto(953.26894647,78.52431783)(953.21894652,78.58931777)(953.16894775,78.64931936)
\curveto(953.11894662,78.71931764)(953.04394669,78.77931758)(952.94394775,78.82931936)
\curveto(952.85394688,78.88931747)(952.76394697,78.93931742)(952.67394775,78.97931936)
\curveto(952.64394709,78.99931736)(952.58394715,79.02431733)(952.49394775,79.05431936)
\curveto(952.41394732,79.08431727)(952.34394739,79.08931727)(952.28394775,79.06931936)
\curveto(952.14394759,79.03931732)(952.05394768,78.97931738)(952.01394775,78.88931936)
\curveto(951.98394775,78.80931755)(951.96894777,78.71931764)(951.96894775,78.61931936)
\curveto(951.96894777,78.51931784)(951.94394779,78.43431792)(951.89394775,78.36431936)
\curveto(951.82394791,78.27431808)(951.68394805,78.22931813)(951.47394775,78.22931936)
\lineto(950.91894775,78.22931936)
\lineto(950.69394775,78.22931936)
\curveto(950.61394912,78.23931812)(950.54894919,78.2593181)(950.49894775,78.28931936)
\curveto(950.41894932,78.34931801)(950.37394936,78.41931794)(950.36394775,78.49931936)
\curveto(950.35394938,78.51931784)(950.34894939,78.53931782)(950.34894775,78.55931936)
\curveto(950.34894939,78.58931777)(950.34394939,78.61431774)(950.33394775,78.63431936)
}
}
{
\newrgbcolor{curcolor}{0 0 0}
\pscustom[linestyle=none,fillstyle=solid,fillcolor=curcolor]
{
}
}
{
\newrgbcolor{curcolor}{0 0 0}
\pscustom[linestyle=none,fillstyle=solid,fillcolor=curcolor]
{
\newpath
\moveto(941.36394775,89.26463186)
\curveto(941.35395838,89.95462722)(941.47395826,90.55462662)(941.72394775,91.06463186)
\curveto(941.97395776,91.58462559)(942.30895743,91.9796252)(942.72894775,92.24963186)
\curveto(942.80895693,92.29962488)(942.89895684,92.34462483)(942.99894775,92.38463186)
\curveto(943.08895665,92.42462475)(943.18395655,92.46962471)(943.28394775,92.51963186)
\curveto(943.38395635,92.55962462)(943.48395625,92.58962459)(943.58394775,92.60963186)
\curveto(943.68395605,92.62962455)(943.78895595,92.64962453)(943.89894775,92.66963186)
\curveto(943.94895579,92.68962449)(943.99395574,92.69462448)(944.03394775,92.68463186)
\curveto(944.07395566,92.6746245)(944.11895562,92.6796245)(944.16894775,92.69963186)
\curveto(944.21895552,92.70962447)(944.30395543,92.71462446)(944.42394775,92.71463186)
\curveto(944.5339552,92.71462446)(944.61895512,92.70962447)(944.67894775,92.69963186)
\curveto(944.738955,92.6796245)(944.79895494,92.66962451)(944.85894775,92.66963186)
\curveto(944.91895482,92.6796245)(944.97895476,92.6746245)(945.03894775,92.65463186)
\curveto(945.17895456,92.61462456)(945.31395442,92.5796246)(945.44394775,92.54963186)
\curveto(945.57395416,92.51962466)(945.69895404,92.4796247)(945.81894775,92.42963186)
\curveto(945.95895378,92.36962481)(946.08395365,92.29962488)(946.19394775,92.21963186)
\curveto(946.30395343,92.14962503)(946.41395332,92.0746251)(946.52394775,91.99463186)
\lineto(946.58394775,91.93463186)
\curveto(946.60395313,91.92462525)(946.62395311,91.90962527)(946.64394775,91.88963186)
\curveto(946.80395293,91.76962541)(946.94895279,91.63462554)(947.07894775,91.48463186)
\curveto(947.20895253,91.33462584)(947.3339524,91.174626)(947.45394775,91.00463186)
\curveto(947.67395206,90.69462648)(947.87895186,90.39962678)(948.06894775,90.11963186)
\curveto(948.20895153,89.88962729)(948.34395139,89.65962752)(948.47394775,89.42963186)
\curveto(948.60395113,89.20962797)(948.738951,88.98962819)(948.87894775,88.76963186)
\curveto(949.04895069,88.51962866)(949.22895051,88.2796289)(949.41894775,88.04963186)
\curveto(949.60895013,87.82962935)(949.8339499,87.63962954)(950.09394775,87.47963186)
\curveto(950.15394958,87.43962974)(950.21394952,87.40462977)(950.27394775,87.37463186)
\curveto(950.32394941,87.34462983)(950.38894935,87.31462986)(950.46894775,87.28463186)
\curveto(950.5389492,87.26462991)(950.59894914,87.25962992)(950.64894775,87.26963186)
\curveto(950.71894902,87.28962989)(950.77394896,87.32462985)(950.81394775,87.37463186)
\curveto(950.84394889,87.42462975)(950.86394887,87.48462969)(950.87394775,87.55463186)
\lineto(950.87394775,87.79463186)
\lineto(950.87394775,88.54463186)
\lineto(950.87394775,91.34963186)
\lineto(950.87394775,92.00963186)
\curveto(950.87394886,92.09962508)(950.87894886,92.18462499)(950.88894775,92.26463186)
\curveto(950.88894885,92.34462483)(950.90894883,92.40962477)(950.94894775,92.45963186)
\curveto(950.98894875,92.50962467)(951.06394867,92.54962463)(951.17394775,92.57963186)
\curveto(951.27394846,92.61962456)(951.37394836,92.62962455)(951.47394775,92.60963186)
\lineto(951.60894775,92.60963186)
\curveto(951.67894806,92.58962459)(951.738948,92.56962461)(951.78894775,92.54963186)
\curveto(951.8389479,92.52962465)(951.87894786,92.49462468)(951.90894775,92.44463186)
\curveto(951.94894779,92.39462478)(951.96894777,92.32462485)(951.96894775,92.23463186)
\lineto(951.96894775,91.96463186)
\lineto(951.96894775,91.06463186)
\lineto(951.96894775,87.55463186)
\lineto(951.96894775,86.48963186)
\curveto(951.96894777,86.40963077)(951.97394776,86.31963086)(951.98394775,86.21963186)
\curveto(951.98394775,86.11963106)(951.97394776,86.03463114)(951.95394775,85.96463186)
\curveto(951.88394785,85.75463142)(951.70394803,85.68963149)(951.41394775,85.76963186)
\curveto(951.37394836,85.7796314)(951.3389484,85.7796314)(951.30894775,85.76963186)
\curveto(951.26894847,85.76963141)(951.22394851,85.7796314)(951.17394775,85.79963186)
\curveto(951.09394864,85.81963136)(951.00894873,85.83963134)(950.91894775,85.85963186)
\curveto(950.82894891,85.8796313)(950.74394899,85.90463127)(950.66394775,85.93463186)
\curveto(950.17394956,86.09463108)(949.75894998,86.29463088)(949.41894775,86.53463186)
\curveto(949.16895057,86.71463046)(948.94395079,86.91963026)(948.74394775,87.14963186)
\curveto(948.5339512,87.3796298)(948.3389514,87.61962956)(948.15894775,87.86963186)
\curveto(947.97895176,88.12962905)(947.80895193,88.39462878)(947.64894775,88.66463186)
\curveto(947.47895226,88.94462823)(947.30395243,89.21462796)(947.12394775,89.47463186)
\curveto(947.04395269,89.58462759)(946.96895277,89.68962749)(946.89894775,89.78963186)
\curveto(946.82895291,89.89962728)(946.75395298,90.00962717)(946.67394775,90.11963186)
\curveto(946.64395309,90.15962702)(946.61395312,90.19462698)(946.58394775,90.22463186)
\curveto(946.54395319,90.26462691)(946.51395322,90.30462687)(946.49394775,90.34463186)
\curveto(946.38395335,90.48462669)(946.25895348,90.60962657)(946.11894775,90.71963186)
\curveto(946.08895365,90.73962644)(946.06395367,90.76462641)(946.04394775,90.79463186)
\curveto(946.01395372,90.82462635)(945.98395375,90.84962633)(945.95394775,90.86963186)
\curveto(945.85395388,90.94962623)(945.75395398,91.01462616)(945.65394775,91.06463186)
\curveto(945.55395418,91.12462605)(945.44395429,91.179626)(945.32394775,91.22963186)
\curveto(945.25395448,91.25962592)(945.17895456,91.2796259)(945.09894775,91.28963186)
\lineto(944.85894775,91.34963186)
\lineto(944.76894775,91.34963186)
\curveto(944.738955,91.35962582)(944.70895503,91.36462581)(944.67894775,91.36463186)
\curveto(944.60895513,91.38462579)(944.51395522,91.38962579)(944.39394775,91.37963186)
\curveto(944.26395547,91.3796258)(944.16395557,91.36962581)(944.09394775,91.34963186)
\curveto(944.01395572,91.32962585)(943.9389558,91.30962587)(943.86894775,91.28963186)
\curveto(943.78895595,91.2796259)(943.70895603,91.25962592)(943.62894775,91.22963186)
\curveto(943.38895635,91.11962606)(943.18895655,90.96962621)(943.02894775,90.77963186)
\curveto(942.85895688,90.59962658)(942.71895702,90.3796268)(942.60894775,90.11963186)
\curveto(942.58895715,90.04962713)(942.57395716,89.9796272)(942.56394775,89.90963186)
\curveto(942.54395719,89.83962734)(942.52395721,89.76462741)(942.50394775,89.68463186)
\curveto(942.48395725,89.60462757)(942.47395726,89.49462768)(942.47394775,89.35463186)
\curveto(942.47395726,89.22462795)(942.48395725,89.11962806)(942.50394775,89.03963186)
\curveto(942.51395722,88.9796282)(942.51895722,88.92462825)(942.51894775,88.87463186)
\curveto(942.51895722,88.82462835)(942.52895721,88.7746284)(942.54894775,88.72463186)
\curveto(942.58895715,88.62462855)(942.62895711,88.52962865)(942.66894775,88.43963186)
\curveto(942.70895703,88.35962882)(942.75395698,88.2796289)(942.80394775,88.19963186)
\curveto(942.82395691,88.16962901)(942.84895689,88.13962904)(942.87894775,88.10963186)
\curveto(942.90895683,88.08962909)(942.9339568,88.06462911)(942.95394775,88.03463186)
\lineto(943.02894775,87.95963186)
\curveto(943.04895669,87.92962925)(943.06895667,87.90462927)(943.08894775,87.88463186)
\lineto(943.29894775,87.73463186)
\curveto(943.35895638,87.69462948)(943.42395631,87.64962953)(943.49394775,87.59963186)
\curveto(943.58395615,87.53962964)(943.68895605,87.48962969)(943.80894775,87.44963186)
\curveto(943.91895582,87.41962976)(944.02895571,87.38462979)(944.13894775,87.34463186)
\curveto(944.24895549,87.30462987)(944.39395534,87.2796299)(944.57394775,87.26963186)
\curveto(944.74395499,87.25962992)(944.86895487,87.22962995)(944.94894775,87.17963186)
\curveto(945.02895471,87.12963005)(945.07395466,87.05463012)(945.08394775,86.95463186)
\curveto(945.09395464,86.85463032)(945.09895464,86.74463043)(945.09894775,86.62463186)
\curveto(945.09895464,86.58463059)(945.10395463,86.54463063)(945.11394775,86.50463186)
\curveto(945.11395462,86.46463071)(945.10895463,86.42963075)(945.09894775,86.39963186)
\curveto(945.07895466,86.34963083)(945.06895467,86.29963088)(945.06894775,86.24963186)
\curveto(945.06895467,86.20963097)(945.05895468,86.16963101)(945.03894775,86.12963186)
\curveto(944.97895476,86.03963114)(944.84395489,85.99463118)(944.63394775,85.99463186)
\lineto(944.51394775,85.99463186)
\curveto(944.45395528,86.00463117)(944.39395534,86.00963117)(944.33394775,86.00963186)
\curveto(944.26395547,86.01963116)(944.19895554,86.02963115)(944.13894775,86.03963186)
\curveto(944.02895571,86.05963112)(943.92895581,86.0796311)(943.83894775,86.09963186)
\curveto(943.738956,86.11963106)(943.64395609,86.14963103)(943.55394775,86.18963186)
\curveto(943.48395625,86.20963097)(943.42395631,86.22963095)(943.37394775,86.24963186)
\lineto(943.19394775,86.30963186)
\curveto(942.9339568,86.42963075)(942.68895705,86.58463059)(942.45894775,86.77463186)
\curveto(942.22895751,86.9746302)(942.04395769,87.18962999)(941.90394775,87.41963186)
\curveto(941.82395791,87.52962965)(941.75895798,87.64462953)(941.70894775,87.76463186)
\lineto(941.55894775,88.15463186)
\curveto(941.50895823,88.26462891)(941.47895826,88.3796288)(941.46894775,88.49963186)
\curveto(941.44895829,88.61962856)(941.42395831,88.74462843)(941.39394775,88.87463186)
\curveto(941.39395834,88.94462823)(941.39395834,89.00962817)(941.39394775,89.06963186)
\curveto(941.38395835,89.12962805)(941.37395836,89.19462798)(941.36394775,89.26463186)
}
}
{
\newrgbcolor{curcolor}{0 0 0}
\pscustom[linestyle=none,fillstyle=solid,fillcolor=curcolor]
{
\newpath
\moveto(946.88394775,101.36424123)
\lineto(947.13894775,101.36424123)
\curveto(947.21895252,101.37423353)(947.29395244,101.36923353)(947.36394775,101.34924123)
\lineto(947.60394775,101.34924123)
\lineto(947.76894775,101.34924123)
\curveto(947.86895187,101.32923357)(947.97395176,101.31923358)(948.08394775,101.31924123)
\curveto(948.18395155,101.31923358)(948.28395145,101.30923359)(948.38394775,101.28924123)
\lineto(948.53394775,101.28924123)
\curveto(948.67395106,101.25923364)(948.81395092,101.23923366)(948.95394775,101.22924123)
\curveto(949.08395065,101.21923368)(949.21395052,101.19423371)(949.34394775,101.15424123)
\curveto(949.42395031,101.13423377)(949.50895023,101.11423379)(949.59894775,101.09424123)
\lineto(949.83894775,101.03424123)
\lineto(950.13894775,100.91424123)
\curveto(950.22894951,100.88423402)(950.31894942,100.84923405)(950.40894775,100.80924123)
\curveto(950.62894911,100.70923419)(950.84394889,100.57423433)(951.05394775,100.40424123)
\curveto(951.26394847,100.24423466)(951.4339483,100.06923483)(951.56394775,99.87924123)
\curveto(951.60394813,99.82923507)(951.64394809,99.76923513)(951.68394775,99.69924123)
\curveto(951.71394802,99.63923526)(951.74894799,99.57923532)(951.78894775,99.51924123)
\curveto(951.8389479,99.43923546)(951.87894786,99.34423556)(951.90894775,99.23424123)
\curveto(951.9389478,99.12423578)(951.96894777,99.01923588)(951.99894775,98.91924123)
\curveto(952.0389477,98.80923609)(952.06394767,98.6992362)(952.07394775,98.58924123)
\curveto(952.08394765,98.47923642)(952.09894764,98.36423654)(952.11894775,98.24424123)
\curveto(952.12894761,98.2042367)(952.12894761,98.15923674)(952.11894775,98.10924123)
\curveto(952.11894762,98.06923683)(952.12394761,98.02923687)(952.13394775,97.98924123)
\curveto(952.14394759,97.94923695)(952.14894759,97.89423701)(952.14894775,97.82424123)
\curveto(952.14894759,97.75423715)(952.14394759,97.7042372)(952.13394775,97.67424123)
\curveto(952.11394762,97.62423728)(952.10894763,97.57923732)(952.11894775,97.53924123)
\curveto(952.12894761,97.4992374)(952.12894761,97.46423744)(952.11894775,97.43424123)
\lineto(952.11894775,97.34424123)
\curveto(952.09894764,97.28423762)(952.08394765,97.21923768)(952.07394775,97.14924123)
\curveto(952.07394766,97.08923781)(952.06894767,97.02423788)(952.05894775,96.95424123)
\curveto(952.00894773,96.78423812)(951.95894778,96.62423828)(951.90894775,96.47424123)
\curveto(951.85894788,96.32423858)(951.79394794,96.17923872)(951.71394775,96.03924123)
\curveto(951.67394806,95.98923891)(951.64394809,95.93423897)(951.62394775,95.87424123)
\curveto(951.59394814,95.82423908)(951.55894818,95.77423913)(951.51894775,95.72424123)
\curveto(951.3389484,95.48423942)(951.11894862,95.28423962)(950.85894775,95.12424123)
\curveto(950.59894914,94.96423994)(950.31394942,94.82424008)(950.00394775,94.70424123)
\curveto(949.86394987,94.64424026)(949.72395001,94.5992403)(949.58394775,94.56924123)
\curveto(949.4339503,94.53924036)(949.27895046,94.5042404)(949.11894775,94.46424123)
\curveto(949.00895073,94.44424046)(948.89895084,94.42924047)(948.78894775,94.41924123)
\curveto(948.67895106,94.40924049)(948.56895117,94.39424051)(948.45894775,94.37424123)
\curveto(948.41895132,94.36424054)(948.37895136,94.35924054)(948.33894775,94.35924123)
\curveto(948.29895144,94.36924053)(948.25895148,94.36924053)(948.21894775,94.35924123)
\curveto(948.16895157,94.34924055)(948.11895162,94.34424056)(948.06894775,94.34424123)
\lineto(947.90394775,94.34424123)
\curveto(947.85395188,94.32424058)(947.80395193,94.31924058)(947.75394775,94.32924123)
\curveto(947.69395204,94.33924056)(947.6389521,94.33924056)(947.58894775,94.32924123)
\curveto(947.54895219,94.31924058)(947.50395223,94.31924058)(947.45394775,94.32924123)
\curveto(947.40395233,94.33924056)(947.35395238,94.33424057)(947.30394775,94.31424123)
\curveto(947.2339525,94.29424061)(947.15895258,94.28924061)(947.07894775,94.29924123)
\curveto(946.98895275,94.30924059)(946.90395283,94.31424059)(946.82394775,94.31424123)
\curveto(946.733953,94.31424059)(946.6339531,94.30924059)(946.52394775,94.29924123)
\curveto(946.40395333,94.28924061)(946.30395343,94.29424061)(946.22394775,94.31424123)
\lineto(945.93894775,94.31424123)
\lineto(945.30894775,94.35924123)
\curveto(945.20895453,94.36924053)(945.11395462,94.37924052)(945.02394775,94.38924123)
\lineto(944.72394775,94.41924123)
\curveto(944.67395506,94.43924046)(944.62395511,94.44424046)(944.57394775,94.43424123)
\curveto(944.51395522,94.43424047)(944.45895528,94.44424046)(944.40894775,94.46424123)
\curveto(944.2389555,94.51424039)(944.07395566,94.55424035)(943.91394775,94.58424123)
\curveto(943.74395599,94.61424029)(943.58395615,94.66424024)(943.43394775,94.73424123)
\curveto(942.97395676,94.92423998)(942.59895714,95.14423976)(942.30894775,95.39424123)
\curveto(942.01895772,95.65423925)(941.77395796,96.01423889)(941.57394775,96.47424123)
\curveto(941.52395821,96.6042383)(941.48895825,96.73423817)(941.46894775,96.86424123)
\curveto(941.44895829,97.0042379)(941.42395831,97.14423776)(941.39394775,97.28424123)
\curveto(941.38395835,97.35423755)(941.37895836,97.41923748)(941.37894775,97.47924123)
\curveto(941.37895836,97.53923736)(941.37395836,97.6042373)(941.36394775,97.67424123)
\curveto(941.34395839,98.5042364)(941.49395824,99.17423573)(941.81394775,99.68424123)
\curveto(942.12395761,100.19423471)(942.56395717,100.57423433)(943.13394775,100.82424123)
\curveto(943.25395648,100.87423403)(943.37895636,100.91923398)(943.50894775,100.95924123)
\curveto(943.6389561,100.9992339)(943.77395596,101.04423386)(943.91394775,101.09424123)
\curveto(943.99395574,101.11423379)(944.07895566,101.12923377)(944.16894775,101.13924123)
\lineto(944.40894775,101.19924123)
\curveto(944.51895522,101.22923367)(944.62895511,101.24423366)(944.73894775,101.24424123)
\curveto(944.84895489,101.25423365)(944.95895478,101.26923363)(945.06894775,101.28924123)
\curveto(945.11895462,101.30923359)(945.16395457,101.31423359)(945.20394775,101.30424123)
\curveto(945.24395449,101.3042336)(945.28395445,101.30923359)(945.32394775,101.31924123)
\curveto(945.37395436,101.32923357)(945.42895431,101.32923357)(945.48894775,101.31924123)
\curveto(945.5389542,101.31923358)(945.58895415,101.32423358)(945.63894775,101.33424123)
\lineto(945.77394775,101.33424123)
\curveto(945.8339539,101.35423355)(945.90395383,101.35423355)(945.98394775,101.33424123)
\curveto(946.05395368,101.32423358)(946.11895362,101.32923357)(946.17894775,101.34924123)
\curveto(946.20895353,101.35923354)(946.24895349,101.36423354)(946.29894775,101.36424123)
\lineto(946.41894775,101.36424123)
\lineto(946.88394775,101.36424123)
\moveto(949.20894775,99.81924123)
\curveto(948.88895085,99.91923498)(948.52395121,99.97923492)(948.11394775,99.99924123)
\curveto(947.70395203,100.01923488)(947.29395244,100.02923487)(946.88394775,100.02924123)
\curveto(946.45395328,100.02923487)(946.0339537,100.01923488)(945.62394775,99.99924123)
\curveto(945.21395452,99.97923492)(944.82895491,99.93423497)(944.46894775,99.86424123)
\curveto(944.10895563,99.79423511)(943.78895595,99.68423522)(943.50894775,99.53424123)
\curveto(943.21895652,99.39423551)(942.98395675,99.1992357)(942.80394775,98.94924123)
\curveto(942.69395704,98.78923611)(942.61395712,98.60923629)(942.56394775,98.40924123)
\curveto(942.50395723,98.20923669)(942.47395726,97.96423694)(942.47394775,97.67424123)
\curveto(942.49395724,97.65423725)(942.50395723,97.61923728)(942.50394775,97.56924123)
\curveto(942.49395724,97.51923738)(942.49395724,97.47923742)(942.50394775,97.44924123)
\curveto(942.52395721,97.36923753)(942.54395719,97.29423761)(942.56394775,97.22424123)
\curveto(942.57395716,97.16423774)(942.59395714,97.0992378)(942.62394775,97.02924123)
\curveto(942.74395699,96.75923814)(942.91395682,96.53923836)(943.13394775,96.36924123)
\curveto(943.34395639,96.20923869)(943.58895615,96.07423883)(943.86894775,95.96424123)
\curveto(943.97895576,95.91423899)(944.09895564,95.87423903)(944.22894775,95.84424123)
\curveto(944.34895539,95.82423908)(944.47395526,95.7992391)(944.60394775,95.76924123)
\curveto(944.65395508,95.74923915)(944.70895503,95.73923916)(944.76894775,95.73924123)
\curveto(944.81895492,95.73923916)(944.86895487,95.73423917)(944.91894775,95.72424123)
\curveto(945.00895473,95.71423919)(945.10395463,95.7042392)(945.20394775,95.69424123)
\curveto(945.29395444,95.68423922)(945.38895435,95.67423923)(945.48894775,95.66424123)
\curveto(945.56895417,95.66423924)(945.65395408,95.65923924)(945.74394775,95.64924123)
\lineto(945.98394775,95.64924123)
\lineto(946.16394775,95.64924123)
\curveto(946.19395354,95.63923926)(946.22895351,95.63423927)(946.26894775,95.63424123)
\lineto(946.40394775,95.63424123)
\lineto(946.85394775,95.63424123)
\curveto(946.9339528,95.63423927)(947.01895272,95.62923927)(947.10894775,95.61924123)
\curveto(947.18895255,95.61923928)(947.26395247,95.62923927)(947.33394775,95.64924123)
\lineto(947.60394775,95.64924123)
\curveto(947.62395211,95.64923925)(947.65395208,95.64423926)(947.69394775,95.63424123)
\curveto(947.72395201,95.63423927)(947.74895199,95.63923926)(947.76894775,95.64924123)
\curveto(947.86895187,95.65923924)(947.96895177,95.66423924)(948.06894775,95.66424123)
\curveto(948.15895158,95.67423923)(948.25895148,95.68423922)(948.36894775,95.69424123)
\curveto(948.48895125,95.72423918)(948.61395112,95.73923916)(948.74394775,95.73924123)
\curveto(948.86395087,95.74923915)(948.97895076,95.77423913)(949.08894775,95.81424123)
\curveto(949.38895035,95.89423901)(949.65395008,95.97923892)(949.88394775,96.06924123)
\curveto(950.11394962,96.16923873)(950.32894941,96.31423859)(950.52894775,96.50424123)
\curveto(950.72894901,96.71423819)(950.87894886,96.97923792)(950.97894775,97.29924123)
\curveto(950.99894874,97.33923756)(951.00894873,97.37423753)(951.00894775,97.40424123)
\curveto(950.99894874,97.44423746)(951.00394873,97.48923741)(951.02394775,97.53924123)
\curveto(951.0339487,97.57923732)(951.04394869,97.64923725)(951.05394775,97.74924123)
\curveto(951.06394867,97.85923704)(951.05894868,97.94423696)(951.03894775,98.00424123)
\curveto(951.01894872,98.07423683)(951.00894873,98.14423676)(951.00894775,98.21424123)
\curveto(950.99894874,98.28423662)(950.98394875,98.34923655)(950.96394775,98.40924123)
\curveto(950.90394883,98.60923629)(950.81894892,98.78923611)(950.70894775,98.94924123)
\curveto(950.68894905,98.97923592)(950.66894907,99.0042359)(950.64894775,99.02424123)
\lineto(950.58894775,99.08424123)
\curveto(950.56894917,99.12423578)(950.52894921,99.17423573)(950.46894775,99.23424123)
\curveto(950.32894941,99.33423557)(950.19894954,99.41923548)(950.07894775,99.48924123)
\curveto(949.95894978,99.55923534)(949.81394992,99.62923527)(949.64394775,99.69924123)
\curveto(949.57395016,99.72923517)(949.50395023,99.74923515)(949.43394775,99.75924123)
\curveto(949.36395037,99.77923512)(949.28895045,99.7992351)(949.20894775,99.81924123)
}
}
{
\newrgbcolor{curcolor}{0 0 0}
\pscustom[linestyle=none,fillstyle=solid,fillcolor=curcolor]
{
\newpath
\moveto(941.36394775,106.77385061)
\curveto(941.36395837,106.87384575)(941.37395836,106.96884566)(941.39394775,107.05885061)
\curveto(941.40395833,107.14884548)(941.4339583,107.21384541)(941.48394775,107.25385061)
\curveto(941.56395817,107.31384531)(941.66895807,107.34384528)(941.79894775,107.34385061)
\lineto(942.18894775,107.34385061)
\lineto(943.68894775,107.34385061)
\lineto(950.07894775,107.34385061)
\lineto(951.24894775,107.34385061)
\lineto(951.56394775,107.34385061)
\curveto(951.66394807,107.35384527)(951.74394799,107.33884529)(951.80394775,107.29885061)
\curveto(951.88394785,107.24884538)(951.9339478,107.17384545)(951.95394775,107.07385061)
\curveto(951.96394777,106.98384564)(951.96894777,106.87384575)(951.96894775,106.74385061)
\lineto(951.96894775,106.51885061)
\curveto(951.94894779,106.43884619)(951.9339478,106.36884626)(951.92394775,106.30885061)
\curveto(951.90394783,106.24884638)(951.86394787,106.19884643)(951.80394775,106.15885061)
\curveto(951.74394799,106.11884651)(951.66894807,106.09884653)(951.57894775,106.09885061)
\lineto(951.27894775,106.09885061)
\lineto(950.18394775,106.09885061)
\lineto(944.84394775,106.09885061)
\curveto(944.75395498,106.07884655)(944.67895506,106.06384656)(944.61894775,106.05385061)
\curveto(944.54895519,106.05384657)(944.48895525,106.0238466)(944.43894775,105.96385061)
\curveto(944.38895535,105.89384673)(944.36395537,105.80384682)(944.36394775,105.69385061)
\curveto(944.35395538,105.59384703)(944.34895539,105.48384714)(944.34894775,105.36385061)
\lineto(944.34894775,104.22385061)
\lineto(944.34894775,103.72885061)
\curveto(944.3389554,103.56884906)(944.27895546,103.45884917)(944.16894775,103.39885061)
\curveto(944.1389556,103.37884925)(944.10895563,103.36884926)(944.07894775,103.36885061)
\curveto(944.0389557,103.36884926)(943.99395574,103.36384926)(943.94394775,103.35385061)
\curveto(943.82395591,103.33384929)(943.71395602,103.33884929)(943.61394775,103.36885061)
\curveto(943.51395622,103.40884922)(943.44395629,103.46384916)(943.40394775,103.53385061)
\curveto(943.35395638,103.61384901)(943.32895641,103.73384889)(943.32894775,103.89385061)
\curveto(943.32895641,104.05384857)(943.31395642,104.18884844)(943.28394775,104.29885061)
\curveto(943.27395646,104.34884828)(943.26895647,104.40384822)(943.26894775,104.46385061)
\curveto(943.25895648,104.5238481)(943.24395649,104.58384804)(943.22394775,104.64385061)
\curveto(943.17395656,104.79384783)(943.12395661,104.93884769)(943.07394775,105.07885061)
\curveto(943.01395672,105.21884741)(942.94395679,105.35384727)(942.86394775,105.48385061)
\curveto(942.77395696,105.623847)(942.66895707,105.74384688)(942.54894775,105.84385061)
\curveto(942.42895731,105.94384668)(942.29895744,106.03884659)(942.15894775,106.12885061)
\curveto(942.05895768,106.18884644)(941.94895779,106.23384639)(941.82894775,106.26385061)
\curveto(941.70895803,106.30384632)(941.60395813,106.35384627)(941.51394775,106.41385061)
\curveto(941.45395828,106.46384616)(941.41395832,106.53384609)(941.39394775,106.62385061)
\curveto(941.38395835,106.64384598)(941.37895836,106.66884596)(941.37894775,106.69885061)
\curveto(941.37895836,106.7288459)(941.37395836,106.75384587)(941.36394775,106.77385061)
}
}
{
\newrgbcolor{curcolor}{0 0 0}
\pscustom[linestyle=none,fillstyle=solid,fillcolor=curcolor]
{
\newpath
\moveto(941.36394775,115.12345998)
\curveto(941.36395837,115.22345513)(941.37395836,115.31845503)(941.39394775,115.40845998)
\curveto(941.40395833,115.49845485)(941.4339583,115.56345479)(941.48394775,115.60345998)
\curveto(941.56395817,115.66345469)(941.66895807,115.69345466)(941.79894775,115.69345998)
\lineto(942.18894775,115.69345998)
\lineto(943.68894775,115.69345998)
\lineto(950.07894775,115.69345998)
\lineto(951.24894775,115.69345998)
\lineto(951.56394775,115.69345998)
\curveto(951.66394807,115.70345465)(951.74394799,115.68845466)(951.80394775,115.64845998)
\curveto(951.88394785,115.59845475)(951.9339478,115.52345483)(951.95394775,115.42345998)
\curveto(951.96394777,115.33345502)(951.96894777,115.22345513)(951.96894775,115.09345998)
\lineto(951.96894775,114.86845998)
\curveto(951.94894779,114.78845556)(951.9339478,114.71845563)(951.92394775,114.65845998)
\curveto(951.90394783,114.59845575)(951.86394787,114.5484558)(951.80394775,114.50845998)
\curveto(951.74394799,114.46845588)(951.66894807,114.4484559)(951.57894775,114.44845998)
\lineto(951.27894775,114.44845998)
\lineto(950.18394775,114.44845998)
\lineto(944.84394775,114.44845998)
\curveto(944.75395498,114.42845592)(944.67895506,114.41345594)(944.61894775,114.40345998)
\curveto(944.54895519,114.40345595)(944.48895525,114.37345598)(944.43894775,114.31345998)
\curveto(944.38895535,114.24345611)(944.36395537,114.1534562)(944.36394775,114.04345998)
\curveto(944.35395538,113.94345641)(944.34895539,113.83345652)(944.34894775,113.71345998)
\lineto(944.34894775,112.57345998)
\lineto(944.34894775,112.07845998)
\curveto(944.3389554,111.91845843)(944.27895546,111.80845854)(944.16894775,111.74845998)
\curveto(944.1389556,111.72845862)(944.10895563,111.71845863)(944.07894775,111.71845998)
\curveto(944.0389557,111.71845863)(943.99395574,111.71345864)(943.94394775,111.70345998)
\curveto(943.82395591,111.68345867)(943.71395602,111.68845866)(943.61394775,111.71845998)
\curveto(943.51395622,111.75845859)(943.44395629,111.81345854)(943.40394775,111.88345998)
\curveto(943.35395638,111.96345839)(943.32895641,112.08345827)(943.32894775,112.24345998)
\curveto(943.32895641,112.40345795)(943.31395642,112.53845781)(943.28394775,112.64845998)
\curveto(943.27395646,112.69845765)(943.26895647,112.7534576)(943.26894775,112.81345998)
\curveto(943.25895648,112.87345748)(943.24395649,112.93345742)(943.22394775,112.99345998)
\curveto(943.17395656,113.14345721)(943.12395661,113.28845706)(943.07394775,113.42845998)
\curveto(943.01395672,113.56845678)(942.94395679,113.70345665)(942.86394775,113.83345998)
\curveto(942.77395696,113.97345638)(942.66895707,114.09345626)(942.54894775,114.19345998)
\curveto(942.42895731,114.29345606)(942.29895744,114.38845596)(942.15894775,114.47845998)
\curveto(942.05895768,114.53845581)(941.94895779,114.58345577)(941.82894775,114.61345998)
\curveto(941.70895803,114.6534557)(941.60395813,114.70345565)(941.51394775,114.76345998)
\curveto(941.45395828,114.81345554)(941.41395832,114.88345547)(941.39394775,114.97345998)
\curveto(941.38395835,114.99345536)(941.37895836,115.01845533)(941.37894775,115.04845998)
\curveto(941.37895836,115.07845527)(941.37395836,115.10345525)(941.36394775,115.12345998)
}
}
{
\newrgbcolor{curcolor}{0 0 0}
\pscustom[linestyle=none,fillstyle=solid,fillcolor=curcolor]
{
\newpath
\moveto(972.10026367,38.71181936)
\curveto(972.15026442,38.73180981)(972.21026436,38.75680979)(972.28026367,38.78681936)
\curveto(972.35026422,38.81680973)(972.42526414,38.83680971)(972.50526367,38.84681936)
\curveto(972.57526399,38.86680968)(972.64526392,38.86680968)(972.71526367,38.84681936)
\curveto(972.77526379,38.83680971)(972.82026375,38.79680975)(972.85026367,38.72681936)
\curveto(972.8702637,38.67680987)(972.88026369,38.61680993)(972.88026367,38.54681936)
\lineto(972.88026367,38.33681936)
\lineto(972.88026367,37.88681936)
\curveto(972.88026369,37.73681081)(972.85526371,37.61681093)(972.80526367,37.52681936)
\curveto(972.74526382,37.42681112)(972.64026393,37.35181119)(972.49026367,37.30181936)
\curveto(972.34026423,37.26181128)(972.20526436,37.21681133)(972.08526367,37.16681936)
\curveto(971.82526474,37.05681149)(971.55526501,36.95681159)(971.27526367,36.86681936)
\curveto(970.99526557,36.77681177)(970.72026585,36.67681187)(970.45026367,36.56681936)
\curveto(970.36026621,36.53681201)(970.27526629,36.50681204)(970.19526367,36.47681936)
\curveto(970.11526645,36.45681209)(970.04026653,36.42681212)(969.97026367,36.38681936)
\curveto(969.90026667,36.35681219)(969.84026673,36.31181223)(969.79026367,36.25181936)
\curveto(969.74026683,36.19181235)(969.70026687,36.11181243)(969.67026367,36.01181936)
\curveto(969.65026692,35.96181258)(969.64526692,35.90181264)(969.65526367,35.83181936)
\lineto(969.65526367,35.63681936)
\lineto(969.65526367,32.80181936)
\lineto(969.65526367,32.50181936)
\curveto(969.64526692,32.39181615)(969.64526692,32.28681626)(969.65526367,32.18681936)
\curveto(969.6652669,32.08681646)(969.68026689,31.99181655)(969.70026367,31.90181936)
\curveto(969.72026685,31.82181672)(969.76026681,31.76181678)(969.82026367,31.72181936)
\curveto(969.92026665,31.6418169)(970.03526653,31.58181696)(970.16526367,31.54181936)
\curveto(970.28526628,31.51181703)(970.41026616,31.47181707)(970.54026367,31.42181936)
\curveto(970.7702658,31.32181722)(971.01026556,31.22681732)(971.26026367,31.13681936)
\curveto(971.51026506,31.05681749)(971.75026482,30.96681758)(971.98026367,30.86681936)
\curveto(972.04026453,30.8468177)(972.11026446,30.82181772)(972.19026367,30.79181936)
\curveto(972.26026431,30.77181777)(972.33526423,30.7468178)(972.41526367,30.71681936)
\curveto(972.49526407,30.68681786)(972.570264,30.65181789)(972.64026367,30.61181936)
\curveto(972.70026387,30.58181796)(972.74526382,30.546818)(972.77526367,30.50681936)
\curveto(972.83526373,30.42681812)(972.8702637,30.31681823)(972.88026367,30.17681936)
\lineto(972.88026367,29.75681936)
\lineto(972.88026367,29.51681936)
\curveto(972.8702637,29.4468191)(972.84526372,29.38681916)(972.80526367,29.33681936)
\curveto(972.77526379,29.28681926)(972.73026384,29.25681929)(972.67026367,29.24681936)
\curveto(972.61026396,29.2468193)(972.55026402,29.25181929)(972.49026367,29.26181936)
\curveto(972.42026415,29.28181926)(972.35526421,29.30181924)(972.29526367,29.32181936)
\curveto(972.22526434,29.35181919)(972.17526439,29.37681917)(972.14526367,29.39681936)
\curveto(971.82526474,29.53681901)(971.51026506,29.66181888)(971.20026367,29.77181936)
\curveto(970.88026569,29.88181866)(970.56026601,30.00181854)(970.24026367,30.13181936)
\curveto(970.02026655,30.22181832)(969.80526676,30.30681824)(969.59526367,30.38681936)
\curveto(969.37526719,30.46681808)(969.15526741,30.55181799)(968.93526367,30.64181936)
\curveto(968.21526835,30.9418176)(967.49026908,31.22681732)(966.76026367,31.49681936)
\curveto(966.02027055,31.76681678)(965.28527128,32.05181649)(964.55526367,32.35181936)
\curveto(964.29527227,32.46181608)(964.03027254,32.56181598)(963.76026367,32.65181936)
\curveto(963.49027308,32.75181579)(963.22527334,32.85681569)(962.96526367,32.96681936)
\curveto(962.85527371,33.01681553)(962.73527383,33.06181548)(962.60526367,33.10181936)
\curveto(962.4652741,33.15181539)(962.3652742,33.22181532)(962.30526367,33.31181936)
\curveto(962.2652743,33.35181519)(962.23527433,33.41681513)(962.21526367,33.50681936)
\curveto(962.20527436,33.52681502)(962.20527436,33.546815)(962.21526367,33.56681936)
\curveto(962.21527435,33.59681495)(962.21027436,33.62181492)(962.20026367,33.64181936)
\curveto(962.20027437,33.82181472)(962.20027437,34.03181451)(962.20026367,34.27181936)
\curveto(962.19027438,34.51181403)(962.22527434,34.68681386)(962.30526367,34.79681936)
\curveto(962.3652742,34.87681367)(962.4652741,34.93681361)(962.60526367,34.97681936)
\curveto(962.73527383,35.02681352)(962.85527371,35.07681347)(962.96526367,35.12681936)
\curveto(963.19527337,35.22681332)(963.42527314,35.31681323)(963.65526367,35.39681936)
\curveto(963.88527268,35.47681307)(964.11527245,35.56681298)(964.34526367,35.66681936)
\curveto(964.54527202,35.7468128)(964.75027182,35.82181272)(964.96026367,35.89181936)
\curveto(965.1702714,35.97181257)(965.37527119,36.05681249)(965.57526367,36.14681936)
\curveto(966.30527026,36.4468121)(967.04526952,36.73181181)(967.79526367,37.00181936)
\curveto(968.53526803,37.28181126)(969.2702673,37.57681097)(970.00026367,37.88681936)
\curveto(970.09026648,37.92681062)(970.17526639,37.95681059)(970.25526367,37.97681936)
\curveto(970.33526623,38.00681054)(970.42026615,38.03681051)(970.51026367,38.06681936)
\curveto(970.7702658,38.17681037)(971.03526553,38.28181026)(971.30526367,38.38181936)
\curveto(971.57526499,38.49181005)(971.84026473,38.60180994)(972.10026367,38.71181936)
\moveto(968.45526367,35.50181936)
\curveto(968.42526814,35.59181295)(968.37526819,35.6468129)(968.30526367,35.66681936)
\curveto(968.23526833,35.69681285)(968.16026841,35.70181284)(968.08026367,35.68181936)
\curveto(967.99026858,35.67181287)(967.90526866,35.6468129)(967.82526367,35.60681936)
\curveto(967.73526883,35.57681297)(967.66026891,35.546813)(967.60026367,35.51681936)
\curveto(967.56026901,35.49681305)(967.52526904,35.48681306)(967.49526367,35.48681936)
\curveto(967.4652691,35.48681306)(967.43026914,35.47681307)(967.39026367,35.45681936)
\lineto(967.15026367,35.36681936)
\curveto(967.06026951,35.3468132)(966.9702696,35.31681323)(966.88026367,35.27681936)
\curveto(966.52027005,35.12681342)(966.15527041,34.99181355)(965.78526367,34.87181936)
\curveto(965.40527116,34.76181378)(965.03527153,34.63181391)(964.67526367,34.48181936)
\curveto(964.565272,34.43181411)(964.45527211,34.38681416)(964.34526367,34.34681936)
\curveto(964.23527233,34.31681423)(964.13027244,34.27681427)(964.03026367,34.22681936)
\curveto(963.98027259,34.20681434)(963.93527263,34.18181436)(963.89526367,34.15181936)
\curveto(963.84527272,34.13181441)(963.82027275,34.08181446)(963.82026367,34.00181936)
\curveto(963.84027273,33.98181456)(963.85527271,33.96181458)(963.86526367,33.94181936)
\curveto(963.87527269,33.92181462)(963.89027268,33.90181464)(963.91026367,33.88181936)
\curveto(963.96027261,33.8418147)(964.01527255,33.81181473)(964.07526367,33.79181936)
\curveto(964.12527244,33.77181477)(964.18027239,33.75181479)(964.24026367,33.73181936)
\curveto(964.35027222,33.68181486)(964.46027211,33.6418149)(964.57026367,33.61181936)
\curveto(964.68027189,33.58181496)(964.79027178,33.541815)(964.90026367,33.49181936)
\curveto(965.29027128,33.32181522)(965.68527088,33.17181537)(966.08526367,33.04181936)
\curveto(966.48527008,32.92181562)(966.87526969,32.78181576)(967.25526367,32.62181936)
\lineto(967.40526367,32.56181936)
\curveto(967.45526911,32.55181599)(967.50526906,32.53681601)(967.55526367,32.51681936)
\lineto(967.79526367,32.42681936)
\curveto(967.87526869,32.39681615)(967.95526861,32.37181617)(968.03526367,32.35181936)
\curveto(968.08526848,32.33181621)(968.14026843,32.32181622)(968.20026367,32.32181936)
\curveto(968.26026831,32.33181621)(968.31026826,32.3468162)(968.35026367,32.36681936)
\curveto(968.43026814,32.41681613)(968.47526809,32.52181602)(968.48526367,32.68181936)
\lineto(968.48526367,33.13181936)
\lineto(968.48526367,34.73681936)
\curveto(968.48526808,34.8468137)(968.49026808,34.98181356)(968.50026367,35.14181936)
\curveto(968.50026807,35.30181324)(968.48526808,35.42181312)(968.45526367,35.50181936)
}
}
{
\newrgbcolor{curcolor}{0 0 0}
\pscustom[linestyle=none,fillstyle=solid,fillcolor=curcolor]
{
\newpath
\moveto(965.26026367,46.44338186)
\curveto(965.31027126,46.51337426)(965.38527118,46.54837422)(965.48526367,46.54838186)
\curveto(965.58527098,46.55837421)(965.69027088,46.56337421)(965.80026367,46.56338186)
\lineto(972.07026367,46.56338186)
\lineto(972.67026367,46.56338186)
\curveto(972.72026385,46.54337423)(972.7702638,46.53837423)(972.82026367,46.54838186)
\curveto(972.86026371,46.55837421)(972.90526366,46.55337422)(972.95526367,46.53338186)
\curveto(973.05526351,46.51337426)(973.15526341,46.49837427)(973.25526367,46.48838186)
\curveto(973.3652632,46.48837428)(973.4702631,46.4733743)(973.57026367,46.44338186)
\curveto(973.68026289,46.41337436)(973.78526278,46.38337439)(973.88526367,46.35338186)
\curveto(973.98526258,46.33337444)(974.08526248,46.29837447)(974.18526367,46.24838186)
\curveto(974.44526212,46.14837462)(974.68026189,46.01837475)(974.89026367,45.85838186)
\curveto(975.10026147,45.70837506)(975.27526129,45.52837524)(975.41526367,45.31838186)
\curveto(975.53526103,45.14837562)(975.63026094,44.9683758)(975.70026367,44.77838186)
\curveto(975.78026079,44.58837618)(975.85526071,44.38337639)(975.92526367,44.16338186)
\curveto(975.94526062,44.0733767)(975.95526061,43.98337679)(975.95526367,43.89338186)
\curveto(975.9652606,43.80337697)(975.98026059,43.71337706)(976.00026367,43.62338186)
\lineto(976.00026367,43.53338186)
\curveto(976.01026056,43.51337726)(976.01526055,43.49337728)(976.01526367,43.47338186)
\curveto(976.02526054,43.42337735)(976.02526054,43.3733774)(976.01526367,43.32338186)
\curveto(976.00526056,43.28337749)(976.01026056,43.23837753)(976.03026367,43.18838186)
\curveto(976.05026052,43.11837765)(976.05526051,43.00837776)(976.04526367,42.85838186)
\curveto(976.04526052,42.71837805)(976.03526053,42.61837815)(976.01526367,42.55838186)
\curveto(976.01526055,42.52837824)(976.01026056,42.49837827)(976.00026367,42.46838186)
\lineto(976.00026367,42.40838186)
\curveto(975.98026059,42.31837845)(975.9652606,42.22837854)(975.95526367,42.13838186)
\curveto(975.95526061,42.04837872)(975.94526062,41.96337881)(975.92526367,41.88338186)
\curveto(975.90526066,41.80337897)(975.88026069,41.72337905)(975.85026367,41.64338186)
\curveto(975.83026074,41.56337921)(975.80526076,41.48337929)(975.77526367,41.40338186)
\curveto(975.64526092,41.08337969)(975.50026107,40.81337996)(975.34026367,40.59338186)
\curveto(975.18026139,40.38338039)(974.95526161,40.19338058)(974.66526367,40.02338186)
\curveto(974.64526192,40.00338077)(974.62026195,39.98838078)(974.59026367,39.97838186)
\curveto(974.570262,39.97838079)(974.54526202,39.9683808)(974.51526367,39.94838186)
\curveto(974.43526213,39.91838085)(974.32026225,39.88338089)(974.17026367,39.84338186)
\curveto(974.03026254,39.81338096)(973.92526264,39.84338093)(973.85526367,39.93338186)
\curveto(973.80526276,39.99338078)(973.78026279,40.0733807)(973.78026367,40.17338186)
\lineto(973.78026367,40.50338186)
\lineto(973.78026367,40.66838186)
\curveto(973.78026279,40.72838004)(973.79026278,40.78337999)(973.81026367,40.83338186)
\curveto(973.84026273,40.92337985)(973.89026268,40.98837978)(973.96026367,41.02838186)
\curveto(974.03026254,41.0683797)(974.10526246,41.11337966)(974.18526367,41.16338186)
\lineto(974.36526367,41.28338186)
\curveto(974.43526213,41.33337944)(974.49026208,41.38337939)(974.53026367,41.43338186)
\curveto(974.72026185,41.68337909)(974.86026171,41.98337879)(974.95026367,42.33338186)
\curveto(974.9702616,42.39337838)(974.98026159,42.45337832)(974.98026367,42.51338186)
\curveto(974.99026158,42.58337819)(975.00526156,42.64837812)(975.02526367,42.70838186)
\lineto(975.02526367,42.79838186)
\curveto(975.04526152,42.8683779)(975.05526151,42.95337782)(975.05526367,43.05338186)
\curveto(975.05526151,43.15337762)(975.04526152,43.24337753)(975.02526367,43.32338186)
\curveto(975.01526155,43.35337742)(975.01026156,43.39337738)(975.01026367,43.44338186)
\curveto(974.99026158,43.54337723)(974.9702616,43.63837713)(974.95026367,43.72838186)
\curveto(974.94026163,43.81837695)(974.91526165,43.90337687)(974.87526367,43.98338186)
\curveto(974.75526181,44.2733765)(974.59026198,44.50837626)(974.38026367,44.68838186)
\curveto(974.18026239,44.87837589)(973.93526263,45.03337574)(973.64526367,45.15338186)
\curveto(973.55526301,45.19337558)(973.46026311,45.21837555)(973.36026367,45.22838186)
\curveto(973.26026331,45.24837552)(973.15526341,45.2733755)(973.04526367,45.30338186)
\curveto(972.99526357,45.32337545)(972.94526362,45.33337544)(972.89526367,45.33338186)
\curveto(972.84526372,45.33337544)(972.79526377,45.33837543)(972.74526367,45.34838186)
\curveto(972.71526385,45.35837541)(972.6652639,45.36337541)(972.59526367,45.36338186)
\curveto(972.51526405,45.38337539)(972.43026414,45.38337539)(972.34026367,45.36338186)
\curveto(972.29026428,45.35337542)(972.24526432,45.34837542)(972.20526367,45.34838186)
\curveto(972.1652644,45.35837541)(972.13026444,45.35337542)(972.10026367,45.33338186)
\curveto(972.08026449,45.31337546)(972.0702645,45.29837547)(972.07026367,45.28838186)
\lineto(972.02526367,45.24338186)
\curveto(972.02526454,45.14337563)(972.05526451,45.0683757)(972.11526367,45.01838186)
\curveto(972.1652644,44.97837579)(972.21026436,44.92837584)(972.25026367,44.86838186)
\lineto(972.46026367,44.62838186)
\curveto(972.52026405,44.54837622)(972.57526399,44.45837631)(972.62526367,44.35838186)
\curveto(972.71526385,44.21837655)(972.79026378,44.04337673)(972.85026367,43.83338186)
\curveto(972.90026367,43.62337715)(972.93526363,43.40337737)(972.95526367,43.17338186)
\curveto(972.97526359,42.94337783)(972.9702636,42.71337806)(972.94026367,42.48338186)
\curveto(972.92026365,42.25337852)(972.88026369,42.04337873)(972.82026367,41.85338186)
\curveto(972.51026406,40.91337986)(971.91526465,40.25338052)(971.03526367,39.87338186)
\curveto(970.93526563,39.82338095)(970.84026573,39.78338099)(970.75026367,39.75338186)
\curveto(970.65026592,39.72338105)(970.54526602,39.68838108)(970.43526367,39.64838186)
\curveto(970.38526618,39.62838114)(970.34026623,39.61838115)(970.30026367,39.61838186)
\curveto(970.26026631,39.61838115)(970.21526635,39.60838116)(970.16526367,39.58838186)
\curveto(970.09526647,39.5683812)(970.02526654,39.55338122)(969.95526367,39.54338186)
\curveto(969.87526669,39.54338123)(969.80026677,39.53338124)(969.73026367,39.51338186)
\curveto(969.69026688,39.50338127)(969.65526691,39.49838127)(969.62526367,39.49838186)
\curveto(969.58526698,39.50838126)(969.54526702,39.50838126)(969.50526367,39.49838186)
\curveto(969.4652671,39.49838127)(969.42526714,39.49338128)(969.38526367,39.48338186)
\lineto(969.26526367,39.48338186)
\curveto(969.14526742,39.46338131)(969.02026755,39.46338131)(968.89026367,39.48338186)
\curveto(968.83026774,39.49338128)(968.7702678,39.49838127)(968.71026367,39.49838186)
\lineto(968.54526367,39.49838186)
\curveto(968.49526807,39.50838126)(968.45526811,39.51338126)(968.42526367,39.51338186)
\curveto(968.38526818,39.51338126)(968.34026823,39.51838125)(968.29026367,39.52838186)
\curveto(968.18026839,39.55838121)(968.07526849,39.57838119)(967.97526367,39.58838186)
\curveto(967.8652687,39.59838117)(967.75526881,39.62338115)(967.64526367,39.66338186)
\curveto(967.52526904,39.70338107)(967.41026916,39.73838103)(967.30026367,39.76838186)
\curveto(967.18026939,39.80838096)(967.0652695,39.85338092)(966.95526367,39.90338186)
\curveto(966.79526977,39.9733808)(966.65026992,40.05338072)(966.52026367,40.14338186)
\curveto(966.38027019,40.23338054)(966.24527032,40.32838044)(966.11526367,40.42838186)
\curveto(966.00527056,40.49838027)(965.91527065,40.58838018)(965.84526367,40.69838186)
\lineto(965.78526367,40.75838186)
\lineto(965.72526367,40.81838186)
\lineto(965.60526367,40.96838186)
\lineto(965.48526367,41.14838186)
\curveto(965.40527116,41.27837949)(965.33527123,41.41337936)(965.27526367,41.55338186)
\curveto(965.21527135,41.70337907)(965.16027141,41.86337891)(965.11026367,42.03338186)
\curveto(965.08027149,42.13337864)(965.06027151,42.23337854)(965.05026367,42.33338186)
\curveto(965.04027153,42.44337833)(965.02527154,42.55337822)(965.00526367,42.66338186)
\curveto(964.99527157,42.70337807)(964.99527157,42.75337802)(965.00526367,42.81338186)
\curveto(965.01527155,42.88337789)(965.01027156,42.93337784)(964.99026367,42.96338186)
\curveto(964.98027159,43.28337749)(965.01027156,43.5683772)(965.08026367,43.81838186)
\curveto(965.15027142,44.07837669)(965.25027132,44.30837646)(965.38026367,44.50838186)
\curveto(965.42027115,44.57837619)(965.4652711,44.64337613)(965.51526367,44.70338186)
\lineto(965.66526367,44.88338186)
\curveto(965.70527086,44.93337584)(965.75027082,44.97837579)(965.80026367,45.01838186)
\curveto(965.84027073,45.0683757)(965.86027071,45.14337563)(965.86026367,45.24338186)
\lineto(965.81526367,45.28838186)
\curveto(965.79527077,45.30837546)(965.7702708,45.32837544)(965.74026367,45.34838186)
\curveto(965.66027091,45.37837539)(965.58027099,45.39337538)(965.50026367,45.39338186)
\curveto(965.42027115,45.40337537)(965.35027122,45.43337534)(965.29026367,45.48338186)
\curveto(965.25027132,45.51337526)(965.22027135,45.5733752)(965.20026367,45.66338186)
\curveto(965.1702714,45.75337502)(965.15527141,45.84837492)(965.15526367,45.94838186)
\curveto(965.15527141,46.04837472)(965.1652714,46.14337463)(965.18526367,46.23338186)
\curveto(965.20527136,46.33337444)(965.23027134,46.40337437)(965.26026367,46.44338186)
\moveto(969.04026367,45.31838186)
\curveto(969.00026757,45.32837544)(968.95026762,45.33337544)(968.89026367,45.33338186)
\curveto(968.82026775,45.33337544)(968.7652678,45.32837544)(968.72526367,45.31838186)
\lineto(968.48526367,45.31838186)
\curveto(968.39526817,45.29837547)(968.31026826,45.28337549)(968.23026367,45.27338186)
\curveto(968.14026843,45.26337551)(968.05526851,45.24837552)(967.97526367,45.22838186)
\curveto(967.89526867,45.20837556)(967.82026875,45.18837558)(967.75026367,45.16838186)
\curveto(967.6702689,45.15837561)(967.59526897,45.13837563)(967.52526367,45.10838186)
\curveto(967.24526932,44.99837577)(966.99526957,44.85337592)(966.77526367,44.67338186)
\curveto(966.55527001,44.50337627)(966.39027018,44.28337649)(966.28026367,44.01338186)
\curveto(966.24027033,43.93337684)(966.21027036,43.84837692)(966.19026367,43.75838186)
\curveto(966.16027041,43.6683771)(966.13527043,43.5733772)(966.11526367,43.47338186)
\curveto(966.09527047,43.39337738)(966.09027048,43.30337747)(966.10026367,43.20338186)
\lineto(966.10026367,42.93338186)
\curveto(966.11027046,42.88337789)(966.11527045,42.83337794)(966.11526367,42.78338186)
\curveto(966.11527045,42.74337803)(966.12027045,42.69837807)(966.13026367,42.64838186)
\curveto(966.18027039,42.45837831)(966.23027034,42.29837847)(966.28026367,42.16838186)
\curveto(966.42027015,41.82837894)(966.63026994,41.56337921)(966.91026367,41.37338186)
\curveto(967.19026938,41.18337959)(967.51526905,41.03337974)(967.88526367,40.92338186)
\curveto(967.9652686,40.90337987)(968.04526852,40.88837988)(968.12526367,40.87838186)
\curveto(968.19526837,40.87837989)(968.2702683,40.8683799)(968.35026367,40.84838186)
\curveto(968.38026819,40.82837994)(968.41526815,40.81837995)(968.45526367,40.81838186)
\curveto(968.49526807,40.82837994)(968.53026804,40.82837994)(968.56026367,40.81838186)
\lineto(968.89026367,40.81838186)
\lineto(969.23526367,40.81838186)
\curveto(969.34526722,40.81837995)(969.45026712,40.82837994)(969.55026367,40.84838186)
\lineto(969.62526367,40.84838186)
\curveto(969.65526691,40.85837991)(969.68026689,40.86337991)(969.70026367,40.86338186)
\curveto(969.79026678,40.88337989)(969.88026669,40.89837987)(969.97026367,40.90838186)
\curveto(970.06026651,40.92837984)(970.14526642,40.95337982)(970.22526367,40.98338186)
\curveto(970.48526608,41.06337971)(970.72526584,41.16337961)(970.94526367,41.28338186)
\curveto(971.1652654,41.40337937)(971.34526522,41.56337921)(971.48526367,41.76338186)
\lineto(971.57526367,41.88338186)
\curveto(971.59526497,41.92337885)(971.61526495,41.9683788)(971.63526367,42.01838186)
\curveto(971.68526488,42.09837867)(971.72526484,42.18337859)(971.75526367,42.27338186)
\curveto(971.78526478,42.36337841)(971.81526475,42.46337831)(971.84526367,42.57338186)
\curveto(971.85526471,42.62337815)(971.86026471,42.6683781)(971.86026367,42.70838186)
\curveto(971.85026472,42.75837801)(971.85526471,42.80837796)(971.87526367,42.85838186)
\curveto(971.88526468,42.88837788)(971.89026468,42.93837783)(971.89026367,43.00838186)
\curveto(971.89026468,43.07837769)(971.88526468,43.12837764)(971.87526367,43.15838186)
\curveto(971.8652647,43.18837758)(971.8652647,43.21837755)(971.87526367,43.24838186)
\curveto(971.87526469,43.28837748)(971.8702647,43.32837744)(971.86026367,43.36838186)
\curveto(971.84026473,43.45837731)(971.82026475,43.54337723)(971.80026367,43.62338186)
\curveto(971.78026479,43.70337707)(971.75526481,43.78337699)(971.72526367,43.86338186)
\curveto(971.57526499,44.20337657)(971.3652652,44.4733763)(971.09526367,44.67338186)
\curveto(970.82526574,44.8733759)(970.51026606,45.03337574)(970.15026367,45.15338186)
\curveto(970.06026651,45.18337559)(969.9702666,45.20337557)(969.88026367,45.21338186)
\curveto(969.78026679,45.23337554)(969.68526688,45.25337552)(969.59526367,45.27338186)
\curveto(969.55526701,45.28337549)(969.52026705,45.28837548)(969.49026367,45.28838186)
\curveto(969.45026712,45.28837548)(969.41026716,45.29337548)(969.37026367,45.30338186)
\curveto(969.32026725,45.32337545)(969.2702673,45.32337545)(969.22026367,45.30338186)
\curveto(969.16026741,45.29337548)(969.10026747,45.29837547)(969.04026367,45.31838186)
}
}
{
\newrgbcolor{curcolor}{0 0 0}
\pscustom[linestyle=none,fillstyle=solid,fillcolor=curcolor]
{
\newpath
\moveto(968.68026367,55.56666311)
\curveto(968.74026783,55.58665505)(968.83526773,55.59665504)(968.96526367,55.59666311)
\curveto(969.08526748,55.59665504)(969.1702674,55.59165504)(969.22026367,55.58166311)
\lineto(969.37026367,55.58166311)
\curveto(969.45026712,55.57165506)(969.52526704,55.56165507)(969.59526367,55.55166311)
\curveto(969.65526691,55.55165508)(969.72526684,55.54665509)(969.80526367,55.53666311)
\curveto(969.8652667,55.51665512)(969.92526664,55.50165513)(969.98526367,55.49166311)
\curveto(970.04526652,55.49165514)(970.10526646,55.48165515)(970.16526367,55.46166311)
\curveto(970.29526627,55.42165521)(970.42526614,55.38665525)(970.55526367,55.35666311)
\curveto(970.68526588,55.32665531)(970.80526576,55.28665535)(970.91526367,55.23666311)
\curveto(971.39526517,55.02665561)(971.80026477,54.74665589)(972.13026367,54.39666311)
\curveto(972.45026412,54.04665659)(972.69526387,53.61665702)(972.86526367,53.10666311)
\curveto(972.90526366,52.99665764)(972.93526363,52.87665776)(972.95526367,52.74666311)
\curveto(972.97526359,52.62665801)(972.99526357,52.50165813)(973.01526367,52.37166311)
\curveto(973.02526354,52.31165832)(973.03026354,52.24665839)(973.03026367,52.17666311)
\curveto(973.04026353,52.11665852)(973.04526352,52.05665858)(973.04526367,51.99666311)
\curveto(973.05526351,51.95665868)(973.06026351,51.89665874)(973.06026367,51.81666311)
\curveto(973.06026351,51.74665889)(973.05526351,51.69665894)(973.04526367,51.66666311)
\curveto(973.03526353,51.62665901)(973.03026354,51.58665905)(973.03026367,51.54666311)
\curveto(973.04026353,51.50665913)(973.04026353,51.47165916)(973.03026367,51.44166311)
\lineto(973.03026367,51.35166311)
\lineto(972.98526367,50.99166311)
\curveto(972.94526362,50.85165978)(972.90526366,50.71665992)(972.86526367,50.58666311)
\curveto(972.82526374,50.45666018)(972.78026379,50.3316603)(972.73026367,50.21166311)
\curveto(972.53026404,49.76166087)(972.2702643,49.39166124)(971.95026367,49.10166311)
\curveto(971.63026494,48.81166182)(971.24026533,48.57166206)(970.78026367,48.38166311)
\curveto(970.68026589,48.3316623)(970.58026599,48.29166234)(970.48026367,48.26166311)
\curveto(970.38026619,48.24166239)(970.27526629,48.22166241)(970.16526367,48.20166311)
\curveto(970.12526644,48.18166245)(970.09526647,48.17166246)(970.07526367,48.17166311)
\curveto(970.04526652,48.18166245)(970.01026656,48.18166245)(969.97026367,48.17166311)
\curveto(969.89026668,48.15166248)(969.81026676,48.1366625)(969.73026367,48.12666311)
\curveto(969.64026693,48.12666251)(969.55526701,48.11666252)(969.47526367,48.09666311)
\lineto(969.35526367,48.09666311)
\curveto(969.31526725,48.09666254)(969.2702673,48.09166254)(969.22026367,48.08166311)
\curveto(969.1702674,48.07166256)(969.08526748,48.06666257)(968.96526367,48.06666311)
\curveto(968.83526773,48.06666257)(968.74026783,48.07666256)(968.68026367,48.09666311)
\curveto(968.61026796,48.11666252)(968.54026803,48.12166251)(968.47026367,48.11166311)
\curveto(968.40026817,48.10166253)(968.33026824,48.10666253)(968.26026367,48.12666311)
\curveto(968.21026836,48.1366625)(968.1702684,48.14166249)(968.14026367,48.14166311)
\curveto(968.10026847,48.15166248)(968.05526851,48.16166247)(968.00526367,48.17166311)
\curveto(967.88526868,48.20166243)(967.7652688,48.22666241)(967.64526367,48.24666311)
\curveto(967.52526904,48.27666236)(967.41026916,48.31666232)(967.30026367,48.36666311)
\curveto(966.93026964,48.51666212)(966.60026997,48.69666194)(966.31026367,48.90666311)
\curveto(966.01027056,49.12666151)(965.76027081,49.39166124)(965.56026367,49.70166311)
\curveto(965.48027109,49.82166081)(965.41527115,49.94666069)(965.36526367,50.07666311)
\curveto(965.30527126,50.20666043)(965.24527132,50.34166029)(965.18526367,50.48166311)
\curveto(965.13527143,50.60166003)(965.10527146,50.7316599)(965.09526367,50.87166311)
\curveto(965.07527149,51.01165962)(965.04527152,51.15165948)(965.00526367,51.29166311)
\lineto(965.00526367,51.48666311)
\curveto(964.99527157,51.55665908)(964.98527158,51.62165901)(964.97526367,51.68166311)
\curveto(964.9652716,52.57165806)(965.15027142,53.31165732)(965.53026367,53.90166311)
\curveto(965.91027066,54.49165614)(966.40527016,54.91665572)(967.01526367,55.17666311)
\curveto(967.11526945,55.22665541)(967.21526935,55.26665537)(967.31526367,55.29666311)
\curveto(967.41526915,55.32665531)(967.52026905,55.36165527)(967.63026367,55.40166311)
\curveto(967.74026883,55.4316552)(967.86026871,55.45665518)(967.99026367,55.47666311)
\curveto(968.11026846,55.49665514)(968.23526833,55.52165511)(968.36526367,55.55166311)
\curveto(968.41526815,55.56165507)(968.4702681,55.56165507)(968.53026367,55.55166311)
\curveto(968.58026799,55.55165508)(968.63026794,55.55665508)(968.68026367,55.56666311)
\moveto(969.53526367,54.23166311)
\curveto(969.4652671,54.25165638)(969.38526718,54.25665638)(969.29526367,54.24666311)
\lineto(969.04026367,54.24666311)
\curveto(968.65026792,54.24665639)(968.32026825,54.21165642)(968.05026367,54.14166311)
\curveto(967.9702686,54.11165652)(967.89026868,54.08665655)(967.81026367,54.06666311)
\curveto(967.73026884,54.04665659)(967.65526891,54.02165661)(967.58526367,53.99166311)
\curveto(966.93526963,53.71165692)(966.48527008,53.26665737)(966.23526367,52.65666311)
\curveto(966.20527036,52.58665805)(966.18527038,52.51165812)(966.17526367,52.43166311)
\lineto(966.11526367,52.19166311)
\curveto(966.09527047,52.11165852)(966.08527048,52.02665861)(966.08526367,51.93666311)
\lineto(966.08526367,51.66666311)
\lineto(966.13026367,51.39666311)
\curveto(966.15027042,51.29665934)(966.17527039,51.20165943)(966.20526367,51.11166311)
\curveto(966.22527034,51.0316596)(966.25527031,50.95165968)(966.29526367,50.87166311)
\curveto(966.31527025,50.80165983)(966.34527022,50.7366599)(966.38526367,50.67666311)
\curveto(966.42527014,50.61666002)(966.4652701,50.56166007)(966.50526367,50.51166311)
\curveto(966.67526989,50.27166036)(966.88026969,50.07666056)(967.12026367,49.92666311)
\curveto(967.36026921,49.77666086)(967.64026893,49.64666099)(967.96026367,49.53666311)
\curveto(968.06026851,49.50666113)(968.1652684,49.48666115)(968.27526367,49.47666311)
\curveto(968.37526819,49.46666117)(968.48026809,49.45166118)(968.59026367,49.43166311)
\curveto(968.63026794,49.42166121)(968.69526787,49.41666122)(968.78526367,49.41666311)
\curveto(968.81526775,49.40666123)(968.85026772,49.40166123)(968.89026367,49.40166311)
\curveto(968.93026764,49.41166122)(968.97526759,49.41666122)(969.02526367,49.41666311)
\lineto(969.32526367,49.41666311)
\curveto(969.42526714,49.41666122)(969.51526705,49.42666121)(969.59526367,49.44666311)
\lineto(969.77526367,49.47666311)
\curveto(969.87526669,49.49666114)(969.97526659,49.51166112)(970.07526367,49.52166311)
\curveto(970.1652664,49.54166109)(970.25026632,49.57166106)(970.33026367,49.61166311)
\curveto(970.570266,49.71166092)(970.79526577,49.82666081)(971.00526367,49.95666311)
\curveto(971.21526535,50.09666054)(971.39026518,50.26666037)(971.53026367,50.46666311)
\curveto(971.56026501,50.51666012)(971.58526498,50.56166007)(971.60526367,50.60166311)
\curveto(971.62526494,50.64165999)(971.65026492,50.68665995)(971.68026367,50.73666311)
\curveto(971.73026484,50.81665982)(971.77526479,50.90165973)(971.81526367,50.99166311)
\curveto(971.84526472,51.09165954)(971.87526469,51.19665944)(971.90526367,51.30666311)
\curveto(971.92526464,51.35665928)(971.93526463,51.40165923)(971.93526367,51.44166311)
\curveto(971.92526464,51.49165914)(971.92526464,51.54165909)(971.93526367,51.59166311)
\curveto(971.94526462,51.62165901)(971.95526461,51.68165895)(971.96526367,51.77166311)
\curveto(971.97526459,51.87165876)(971.9702646,51.94665869)(971.95026367,51.99666311)
\curveto(971.94026463,52.0366586)(971.94026463,52.07665856)(971.95026367,52.11666311)
\curveto(971.95026462,52.15665848)(971.94026463,52.19665844)(971.92026367,52.23666311)
\curveto(971.90026467,52.31665832)(971.88526468,52.39665824)(971.87526367,52.47666311)
\curveto(971.85526471,52.55665808)(971.83026474,52.631658)(971.80026367,52.70166311)
\curveto(971.66026491,53.04165759)(971.4652651,53.31665732)(971.21526367,53.52666311)
\curveto(970.9652656,53.7366569)(970.6702659,53.91165672)(970.33026367,54.05166311)
\curveto(970.21026636,54.10165653)(970.08526648,54.1316565)(969.95526367,54.14166311)
\curveto(969.81526675,54.16165647)(969.67526689,54.19165644)(969.53526367,54.23166311)
}
}
{
\newrgbcolor{curcolor}{0 0 0}
\pscustom[linestyle=none,fillstyle=solid,fillcolor=curcolor]
{
}
}
{
\newrgbcolor{curcolor}{0 0 0}
\pscustom[linestyle=none,fillstyle=solid,fillcolor=curcolor]
{
\newpath
\moveto(967.79526367,67.99510061)
\lineto(968.05026367,67.99510061)
\curveto(968.13026844,68.0050929)(968.20526836,68.00009291)(968.27526367,67.98010061)
\lineto(968.51526367,67.98010061)
\lineto(968.68026367,67.98010061)
\curveto(968.78026779,67.96009295)(968.88526768,67.95009296)(968.99526367,67.95010061)
\curveto(969.09526747,67.95009296)(969.19526737,67.94009297)(969.29526367,67.92010061)
\lineto(969.44526367,67.92010061)
\curveto(969.58526698,67.89009302)(969.72526684,67.87009304)(969.86526367,67.86010061)
\curveto(969.99526657,67.85009306)(970.12526644,67.82509308)(970.25526367,67.78510061)
\curveto(970.33526623,67.76509314)(970.42026615,67.74509316)(970.51026367,67.72510061)
\lineto(970.75026367,67.66510061)
\lineto(971.05026367,67.54510061)
\curveto(971.14026543,67.51509339)(971.23026534,67.48009343)(971.32026367,67.44010061)
\curveto(971.54026503,67.34009357)(971.75526481,67.2050937)(971.96526367,67.03510061)
\curveto(972.17526439,66.87509403)(972.34526422,66.70009421)(972.47526367,66.51010061)
\curveto(972.51526405,66.46009445)(972.55526401,66.40009451)(972.59526367,66.33010061)
\curveto(972.62526394,66.27009464)(972.66026391,66.2100947)(972.70026367,66.15010061)
\curveto(972.75026382,66.07009484)(972.79026378,65.97509493)(972.82026367,65.86510061)
\curveto(972.85026372,65.75509515)(972.88026369,65.65009526)(972.91026367,65.55010061)
\curveto(972.95026362,65.44009547)(972.97526359,65.33009558)(972.98526367,65.22010061)
\curveto(972.99526357,65.1100958)(973.01026356,64.99509591)(973.03026367,64.87510061)
\curveto(973.04026353,64.83509607)(973.04026353,64.79009612)(973.03026367,64.74010061)
\curveto(973.03026354,64.70009621)(973.03526353,64.66009625)(973.04526367,64.62010061)
\curveto(973.05526351,64.58009633)(973.06026351,64.52509638)(973.06026367,64.45510061)
\curveto(973.06026351,64.38509652)(973.05526351,64.33509657)(973.04526367,64.30510061)
\curveto(973.02526354,64.25509665)(973.02026355,64.2100967)(973.03026367,64.17010061)
\curveto(973.04026353,64.13009678)(973.04026353,64.09509681)(973.03026367,64.06510061)
\lineto(973.03026367,63.97510061)
\curveto(973.01026356,63.91509699)(972.99526357,63.85009706)(972.98526367,63.78010061)
\curveto(972.98526358,63.72009719)(972.98026359,63.65509725)(972.97026367,63.58510061)
\curveto(972.92026365,63.41509749)(972.8702637,63.25509765)(972.82026367,63.10510061)
\curveto(972.7702638,62.95509795)(972.70526386,62.8100981)(972.62526367,62.67010061)
\curveto(972.58526398,62.62009829)(972.55526401,62.56509834)(972.53526367,62.50510061)
\curveto(972.50526406,62.45509845)(972.4702641,62.4050985)(972.43026367,62.35510061)
\curveto(972.25026432,62.11509879)(972.03026454,61.91509899)(971.77026367,61.75510061)
\curveto(971.51026506,61.59509931)(971.22526534,61.45509945)(970.91526367,61.33510061)
\curveto(970.77526579,61.27509963)(970.63526593,61.23009968)(970.49526367,61.20010061)
\curveto(970.34526622,61.17009974)(970.19026638,61.13509977)(970.03026367,61.09510061)
\curveto(969.92026665,61.07509983)(969.81026676,61.06009985)(969.70026367,61.05010061)
\curveto(969.59026698,61.04009987)(969.48026709,61.02509988)(969.37026367,61.00510061)
\curveto(969.33026724,60.99509991)(969.29026728,60.99009992)(969.25026367,60.99010061)
\curveto(969.21026736,61.00009991)(969.1702674,61.00009991)(969.13026367,60.99010061)
\curveto(969.08026749,60.98009993)(969.03026754,60.97509993)(968.98026367,60.97510061)
\lineto(968.81526367,60.97510061)
\curveto(968.7652678,60.95509995)(968.71526785,60.95009996)(968.66526367,60.96010061)
\curveto(968.60526796,60.97009994)(968.55026802,60.97009994)(968.50026367,60.96010061)
\curveto(968.46026811,60.95009996)(968.41526815,60.95009996)(968.36526367,60.96010061)
\curveto(968.31526825,60.97009994)(968.2652683,60.96509994)(968.21526367,60.94510061)
\curveto(968.14526842,60.92509998)(968.0702685,60.92009999)(967.99026367,60.93010061)
\curveto(967.90026867,60.94009997)(967.81526875,60.94509996)(967.73526367,60.94510061)
\curveto(967.64526892,60.94509996)(967.54526902,60.94009997)(967.43526367,60.93010061)
\curveto(967.31526925,60.92009999)(967.21526935,60.92509998)(967.13526367,60.94510061)
\lineto(966.85026367,60.94510061)
\lineto(966.22026367,60.99010061)
\curveto(966.12027045,61.00009991)(966.02527054,61.0100999)(965.93526367,61.02010061)
\lineto(965.63526367,61.05010061)
\curveto(965.58527098,61.07009984)(965.53527103,61.07509983)(965.48526367,61.06510061)
\curveto(965.42527114,61.06509984)(965.3702712,61.07509983)(965.32026367,61.09510061)
\curveto(965.15027142,61.14509976)(964.98527158,61.18509972)(964.82526367,61.21510061)
\curveto(964.65527191,61.24509966)(964.49527207,61.29509961)(964.34526367,61.36510061)
\curveto(963.88527268,61.55509935)(963.51027306,61.77509913)(963.22026367,62.02510061)
\curveto(962.93027364,62.28509862)(962.68527388,62.64509826)(962.48526367,63.10510061)
\curveto(962.43527413,63.23509767)(962.40027417,63.36509754)(962.38026367,63.49510061)
\curveto(962.36027421,63.63509727)(962.33527423,63.77509713)(962.30526367,63.91510061)
\curveto(962.29527427,63.98509692)(962.29027428,64.05009686)(962.29026367,64.11010061)
\curveto(962.29027428,64.17009674)(962.28527428,64.23509667)(962.27526367,64.30510061)
\curveto(962.25527431,65.13509577)(962.40527416,65.8050951)(962.72526367,66.31510061)
\curveto(963.03527353,66.82509408)(963.47527309,67.2050937)(964.04526367,67.45510061)
\curveto(964.1652724,67.5050934)(964.29027228,67.55009336)(964.42026367,67.59010061)
\curveto(964.55027202,67.63009328)(964.68527188,67.67509323)(964.82526367,67.72510061)
\curveto(964.90527166,67.74509316)(964.99027158,67.76009315)(965.08026367,67.77010061)
\lineto(965.32026367,67.83010061)
\curveto(965.43027114,67.86009305)(965.54027103,67.87509303)(965.65026367,67.87510061)
\curveto(965.76027081,67.88509302)(965.8702707,67.90009301)(965.98026367,67.92010061)
\curveto(966.03027054,67.94009297)(966.07527049,67.94509296)(966.11526367,67.93510061)
\curveto(966.15527041,67.93509297)(966.19527037,67.94009297)(966.23526367,67.95010061)
\curveto(966.28527028,67.96009295)(966.34027023,67.96009295)(966.40026367,67.95010061)
\curveto(966.45027012,67.95009296)(966.50027007,67.95509295)(966.55026367,67.96510061)
\lineto(966.68526367,67.96510061)
\curveto(966.74526982,67.98509292)(966.81526975,67.98509292)(966.89526367,67.96510061)
\curveto(966.9652696,67.95509295)(967.03026954,67.96009295)(967.09026367,67.98010061)
\curveto(967.12026945,67.99009292)(967.16026941,67.99509291)(967.21026367,67.99510061)
\lineto(967.33026367,67.99510061)
\lineto(967.79526367,67.99510061)
\moveto(970.12026367,66.45010061)
\curveto(969.80026677,66.55009436)(969.43526713,66.6100943)(969.02526367,66.63010061)
\curveto(968.61526795,66.65009426)(968.20526836,66.66009425)(967.79526367,66.66010061)
\curveto(967.3652692,66.66009425)(966.94526962,66.65009426)(966.53526367,66.63010061)
\curveto(966.12527044,66.6100943)(965.74027083,66.56509434)(965.38026367,66.49510061)
\curveto(965.02027155,66.42509448)(964.70027187,66.31509459)(964.42026367,66.16510061)
\curveto(964.13027244,66.02509488)(963.89527267,65.83009508)(963.71526367,65.58010061)
\curveto(963.60527296,65.42009549)(963.52527304,65.24009567)(963.47526367,65.04010061)
\curveto(963.41527315,64.84009607)(963.38527318,64.59509631)(963.38526367,64.30510061)
\curveto(963.40527316,64.28509662)(963.41527315,64.25009666)(963.41526367,64.20010061)
\curveto(963.40527316,64.15009676)(963.40527316,64.1100968)(963.41526367,64.08010061)
\curveto(963.43527313,64.00009691)(963.45527311,63.92509698)(963.47526367,63.85510061)
\curveto(963.48527308,63.79509711)(963.50527306,63.73009718)(963.53526367,63.66010061)
\curveto(963.65527291,63.39009752)(963.82527274,63.17009774)(964.04526367,63.00010061)
\curveto(964.25527231,62.84009807)(964.50027207,62.7050982)(964.78026367,62.59510061)
\curveto(964.89027168,62.54509836)(965.01027156,62.5050984)(965.14026367,62.47510061)
\curveto(965.26027131,62.45509845)(965.38527118,62.43009848)(965.51526367,62.40010061)
\curveto(965.565271,62.38009853)(965.62027095,62.37009854)(965.68026367,62.37010061)
\curveto(965.73027084,62.37009854)(965.78027079,62.36509854)(965.83026367,62.35510061)
\curveto(965.92027065,62.34509856)(966.01527055,62.33509857)(966.11526367,62.32510061)
\curveto(966.20527036,62.31509859)(966.30027027,62.3050986)(966.40026367,62.29510061)
\curveto(966.48027009,62.29509861)(966.56527,62.29009862)(966.65526367,62.28010061)
\lineto(966.89526367,62.28010061)
\lineto(967.07526367,62.28010061)
\curveto(967.10526946,62.27009864)(967.14026943,62.26509864)(967.18026367,62.26510061)
\lineto(967.31526367,62.26510061)
\lineto(967.76526367,62.26510061)
\curveto(967.84526872,62.26509864)(967.93026864,62.26009865)(968.02026367,62.25010061)
\curveto(968.10026847,62.25009866)(968.17526839,62.26009865)(968.24526367,62.28010061)
\lineto(968.51526367,62.28010061)
\curveto(968.53526803,62.28009863)(968.565268,62.27509863)(968.60526367,62.26510061)
\curveto(968.63526793,62.26509864)(968.66026791,62.27009864)(968.68026367,62.28010061)
\curveto(968.78026779,62.29009862)(968.88026769,62.29509861)(968.98026367,62.29510061)
\curveto(969.0702675,62.3050986)(969.1702674,62.31509859)(969.28026367,62.32510061)
\curveto(969.40026717,62.35509855)(969.52526704,62.37009854)(969.65526367,62.37010061)
\curveto(969.77526679,62.38009853)(969.89026668,62.4050985)(970.00026367,62.44510061)
\curveto(970.30026627,62.52509838)(970.565266,62.6100983)(970.79526367,62.70010061)
\curveto(971.02526554,62.80009811)(971.24026533,62.94509796)(971.44026367,63.13510061)
\curveto(971.64026493,63.34509756)(971.79026478,63.6100973)(971.89026367,63.93010061)
\curveto(971.91026466,63.97009694)(971.92026465,64.0050969)(971.92026367,64.03510061)
\curveto(971.91026466,64.07509683)(971.91526465,64.12009679)(971.93526367,64.17010061)
\curveto(971.94526462,64.2100967)(971.95526461,64.28009663)(971.96526367,64.38010061)
\curveto(971.97526459,64.49009642)(971.9702646,64.57509633)(971.95026367,64.63510061)
\curveto(971.93026464,64.7050962)(971.92026465,64.77509613)(971.92026367,64.84510061)
\curveto(971.91026466,64.91509599)(971.89526467,64.98009593)(971.87526367,65.04010061)
\curveto(971.81526475,65.24009567)(971.73026484,65.42009549)(971.62026367,65.58010061)
\curveto(971.60026497,65.6100953)(971.58026499,65.63509527)(971.56026367,65.65510061)
\lineto(971.50026367,65.71510061)
\curveto(971.48026509,65.75509515)(971.44026513,65.8050951)(971.38026367,65.86510061)
\curveto(971.24026533,65.96509494)(971.11026546,66.05009486)(970.99026367,66.12010061)
\curveto(970.8702657,66.19009472)(970.72526584,66.26009465)(970.55526367,66.33010061)
\curveto(970.48526608,66.36009455)(970.41526615,66.38009453)(970.34526367,66.39010061)
\curveto(970.27526629,66.4100945)(970.20026637,66.43009448)(970.12026367,66.45010061)
}
}
{
\newrgbcolor{curcolor}{0 0 0}
\pscustom[linestyle=none,fillstyle=solid,fillcolor=curcolor]
{
\newpath
\moveto(962.27526367,73.40470998)
\curveto(962.27527429,73.50470513)(962.28527428,73.59970503)(962.30526367,73.68970998)
\curveto(962.31527425,73.77970485)(962.34527422,73.84470479)(962.39526367,73.88470998)
\curveto(962.47527409,73.94470469)(962.58027399,73.97470466)(962.71026367,73.97470998)
\lineto(963.10026367,73.97470998)
\lineto(964.60026367,73.97470998)
\lineto(970.99026367,73.97470998)
\lineto(972.16026367,73.97470998)
\lineto(972.47526367,73.97470998)
\curveto(972.57526399,73.98470465)(972.65526391,73.96970466)(972.71526367,73.92970998)
\curveto(972.79526377,73.87970475)(972.84526372,73.80470483)(972.86526367,73.70470998)
\curveto(972.87526369,73.61470502)(972.88026369,73.50470513)(972.88026367,73.37470998)
\lineto(972.88026367,73.14970998)
\curveto(972.86026371,73.06970556)(972.84526372,72.99970563)(972.83526367,72.93970998)
\curveto(972.81526375,72.87970575)(972.77526379,72.8297058)(972.71526367,72.78970998)
\curveto(972.65526391,72.74970588)(972.58026399,72.7297059)(972.49026367,72.72970998)
\lineto(972.19026367,72.72970998)
\lineto(971.09526367,72.72970998)
\lineto(965.75526367,72.72970998)
\curveto(965.6652709,72.70970592)(965.59027098,72.69470594)(965.53026367,72.68470998)
\curveto(965.46027111,72.68470595)(965.40027117,72.65470598)(965.35026367,72.59470998)
\curveto(965.30027127,72.52470611)(965.27527129,72.4347062)(965.27526367,72.32470998)
\curveto(965.2652713,72.22470641)(965.26027131,72.11470652)(965.26026367,71.99470998)
\lineto(965.26026367,70.85470998)
\lineto(965.26026367,70.35970998)
\curveto(965.25027132,70.19970843)(965.19027138,70.08970854)(965.08026367,70.02970998)
\curveto(965.05027152,70.00970862)(965.02027155,69.99970863)(964.99026367,69.99970998)
\curveto(964.95027162,69.99970863)(964.90527166,69.99470864)(964.85526367,69.98470998)
\curveto(964.73527183,69.96470867)(964.62527194,69.96970866)(964.52526367,69.99970998)
\curveto(964.42527214,70.03970859)(964.35527221,70.09470854)(964.31526367,70.16470998)
\curveto(964.2652723,70.24470839)(964.24027233,70.36470827)(964.24026367,70.52470998)
\curveto(964.24027233,70.68470795)(964.22527234,70.81970781)(964.19526367,70.92970998)
\curveto(964.18527238,70.97970765)(964.18027239,71.0347076)(964.18026367,71.09470998)
\curveto(964.1702724,71.15470748)(964.15527241,71.21470742)(964.13526367,71.27470998)
\curveto(964.08527248,71.42470721)(964.03527253,71.56970706)(963.98526367,71.70970998)
\curveto(963.92527264,71.84970678)(963.85527271,71.98470665)(963.77526367,72.11470998)
\curveto(963.68527288,72.25470638)(963.58027299,72.37470626)(963.46026367,72.47470998)
\curveto(963.34027323,72.57470606)(963.21027336,72.66970596)(963.07026367,72.75970998)
\curveto(962.9702736,72.81970581)(962.86027371,72.86470577)(962.74026367,72.89470998)
\curveto(962.62027395,72.9347057)(962.51527405,72.98470565)(962.42526367,73.04470998)
\curveto(962.3652742,73.09470554)(962.32527424,73.16470547)(962.30526367,73.25470998)
\curveto(962.29527427,73.27470536)(962.29027428,73.29970533)(962.29026367,73.32970998)
\curveto(962.29027428,73.35970527)(962.28527428,73.38470525)(962.27526367,73.40470998)
}
}
{
\newrgbcolor{curcolor}{0 0 0}
\pscustom[linestyle=none,fillstyle=solid,fillcolor=curcolor]
{
\newpath
\moveto(971.24526367,78.63431936)
\lineto(971.24526367,79.26431936)
\lineto(971.24526367,79.45931936)
\curveto(971.24526532,79.52931683)(971.25526531,79.58931677)(971.27526367,79.63931936)
\curveto(971.31526525,79.70931665)(971.35526521,79.7593166)(971.39526367,79.78931936)
\curveto(971.44526512,79.82931653)(971.51026506,79.84931651)(971.59026367,79.84931936)
\curveto(971.6702649,79.8593165)(971.75526481,79.86431649)(971.84526367,79.86431936)
\lineto(972.56526367,79.86431936)
\curveto(973.04526352,79.86431649)(973.45526311,79.80431655)(973.79526367,79.68431936)
\curveto(974.13526243,79.56431679)(974.41026216,79.36931699)(974.62026367,79.09931936)
\curveto(974.6702619,79.02931733)(974.71526185,78.9593174)(974.75526367,78.88931936)
\curveto(974.80526176,78.82931753)(974.85026172,78.7543176)(974.89026367,78.66431936)
\curveto(974.90026167,78.64431771)(974.91026166,78.61431774)(974.92026367,78.57431936)
\curveto(974.94026163,78.53431782)(974.94526162,78.48931787)(974.93526367,78.43931936)
\curveto(974.90526166,78.34931801)(974.83026174,78.29431806)(974.71026367,78.27431936)
\curveto(974.60026197,78.2543181)(974.50526206,78.26931809)(974.42526367,78.31931936)
\curveto(974.35526221,78.34931801)(974.29026228,78.39431796)(974.23026367,78.45431936)
\curveto(974.18026239,78.52431783)(974.13026244,78.58931777)(974.08026367,78.64931936)
\curveto(974.03026254,78.71931764)(973.95526261,78.77931758)(973.85526367,78.82931936)
\curveto(973.7652628,78.88931747)(973.67526289,78.93931742)(973.58526367,78.97931936)
\curveto(973.55526301,78.99931736)(973.49526307,79.02431733)(973.40526367,79.05431936)
\curveto(973.32526324,79.08431727)(973.25526331,79.08931727)(973.19526367,79.06931936)
\curveto(973.05526351,79.03931732)(972.9652636,78.97931738)(972.92526367,78.88931936)
\curveto(972.89526367,78.80931755)(972.88026369,78.71931764)(972.88026367,78.61931936)
\curveto(972.88026369,78.51931784)(972.85526371,78.43431792)(972.80526367,78.36431936)
\curveto(972.73526383,78.27431808)(972.59526397,78.22931813)(972.38526367,78.22931936)
\lineto(971.83026367,78.22931936)
\lineto(971.60526367,78.22931936)
\curveto(971.52526504,78.23931812)(971.46026511,78.2593181)(971.41026367,78.28931936)
\curveto(971.33026524,78.34931801)(971.28526528,78.41931794)(971.27526367,78.49931936)
\curveto(971.2652653,78.51931784)(971.26026531,78.53931782)(971.26026367,78.55931936)
\curveto(971.26026531,78.58931777)(971.25526531,78.61431774)(971.24526367,78.63431936)
}
}
{
\newrgbcolor{curcolor}{0 0 0}
\pscustom[linestyle=none,fillstyle=solid,fillcolor=curcolor]
{
}
}
{
\newrgbcolor{curcolor}{0 0 0}
\pscustom[linestyle=none,fillstyle=solid,fillcolor=curcolor]
{
\newpath
\moveto(962.27526367,89.26463186)
\curveto(962.2652743,89.95462722)(962.38527418,90.55462662)(962.63526367,91.06463186)
\curveto(962.88527368,91.58462559)(963.22027335,91.9796252)(963.64026367,92.24963186)
\curveto(963.72027285,92.29962488)(963.81027276,92.34462483)(963.91026367,92.38463186)
\curveto(964.00027257,92.42462475)(964.09527247,92.46962471)(964.19526367,92.51963186)
\curveto(964.29527227,92.55962462)(964.39527217,92.58962459)(964.49526367,92.60963186)
\curveto(964.59527197,92.62962455)(964.70027187,92.64962453)(964.81026367,92.66963186)
\curveto(964.86027171,92.68962449)(964.90527166,92.69462448)(964.94526367,92.68463186)
\curveto(964.98527158,92.6746245)(965.03027154,92.6796245)(965.08026367,92.69963186)
\curveto(965.13027144,92.70962447)(965.21527135,92.71462446)(965.33526367,92.71463186)
\curveto(965.44527112,92.71462446)(965.53027104,92.70962447)(965.59026367,92.69963186)
\curveto(965.65027092,92.6796245)(965.71027086,92.66962451)(965.77026367,92.66963186)
\curveto(965.83027074,92.6796245)(965.89027068,92.6746245)(965.95026367,92.65463186)
\curveto(966.09027048,92.61462456)(966.22527034,92.5796246)(966.35526367,92.54963186)
\curveto(966.48527008,92.51962466)(966.61026996,92.4796247)(966.73026367,92.42963186)
\curveto(966.8702697,92.36962481)(966.99526957,92.29962488)(967.10526367,92.21963186)
\curveto(967.21526935,92.14962503)(967.32526924,92.0746251)(967.43526367,91.99463186)
\lineto(967.49526367,91.93463186)
\curveto(967.51526905,91.92462525)(967.53526903,91.90962527)(967.55526367,91.88963186)
\curveto(967.71526885,91.76962541)(967.86026871,91.63462554)(967.99026367,91.48463186)
\curveto(968.12026845,91.33462584)(968.24526832,91.174626)(968.36526367,91.00463186)
\curveto(968.58526798,90.69462648)(968.79026778,90.39962678)(968.98026367,90.11963186)
\curveto(969.12026745,89.88962729)(969.25526731,89.65962752)(969.38526367,89.42963186)
\curveto(969.51526705,89.20962797)(969.65026692,88.98962819)(969.79026367,88.76963186)
\curveto(969.96026661,88.51962866)(970.14026643,88.2796289)(970.33026367,88.04963186)
\curveto(970.52026605,87.82962935)(970.74526582,87.63962954)(971.00526367,87.47963186)
\curveto(971.0652655,87.43962974)(971.12526544,87.40462977)(971.18526367,87.37463186)
\curveto(971.23526533,87.34462983)(971.30026527,87.31462986)(971.38026367,87.28463186)
\curveto(971.45026512,87.26462991)(971.51026506,87.25962992)(971.56026367,87.26963186)
\curveto(971.63026494,87.28962989)(971.68526488,87.32462985)(971.72526367,87.37463186)
\curveto(971.75526481,87.42462975)(971.77526479,87.48462969)(971.78526367,87.55463186)
\lineto(971.78526367,87.79463186)
\lineto(971.78526367,88.54463186)
\lineto(971.78526367,91.34963186)
\lineto(971.78526367,92.00963186)
\curveto(971.78526478,92.09962508)(971.79026478,92.18462499)(971.80026367,92.26463186)
\curveto(971.80026477,92.34462483)(971.82026475,92.40962477)(971.86026367,92.45963186)
\curveto(971.90026467,92.50962467)(971.97526459,92.54962463)(972.08526367,92.57963186)
\curveto(972.18526438,92.61962456)(972.28526428,92.62962455)(972.38526367,92.60963186)
\lineto(972.52026367,92.60963186)
\curveto(972.59026398,92.58962459)(972.65026392,92.56962461)(972.70026367,92.54963186)
\curveto(972.75026382,92.52962465)(972.79026378,92.49462468)(972.82026367,92.44463186)
\curveto(972.86026371,92.39462478)(972.88026369,92.32462485)(972.88026367,92.23463186)
\lineto(972.88026367,91.96463186)
\lineto(972.88026367,91.06463186)
\lineto(972.88026367,87.55463186)
\lineto(972.88026367,86.48963186)
\curveto(972.88026369,86.40963077)(972.88526368,86.31963086)(972.89526367,86.21963186)
\curveto(972.89526367,86.11963106)(972.88526368,86.03463114)(972.86526367,85.96463186)
\curveto(972.79526377,85.75463142)(972.61526395,85.68963149)(972.32526367,85.76963186)
\curveto(972.28526428,85.7796314)(972.25026432,85.7796314)(972.22026367,85.76963186)
\curveto(972.18026439,85.76963141)(972.13526443,85.7796314)(972.08526367,85.79963186)
\curveto(972.00526456,85.81963136)(971.92026465,85.83963134)(971.83026367,85.85963186)
\curveto(971.74026483,85.8796313)(971.65526491,85.90463127)(971.57526367,85.93463186)
\curveto(971.08526548,86.09463108)(970.6702659,86.29463088)(970.33026367,86.53463186)
\curveto(970.08026649,86.71463046)(969.85526671,86.91963026)(969.65526367,87.14963186)
\curveto(969.44526712,87.3796298)(969.25026732,87.61962956)(969.07026367,87.86963186)
\curveto(968.89026768,88.12962905)(968.72026785,88.39462878)(968.56026367,88.66463186)
\curveto(968.39026818,88.94462823)(968.21526835,89.21462796)(968.03526367,89.47463186)
\curveto(967.95526861,89.58462759)(967.88026869,89.68962749)(967.81026367,89.78963186)
\curveto(967.74026883,89.89962728)(967.6652689,90.00962717)(967.58526367,90.11963186)
\curveto(967.55526901,90.15962702)(967.52526904,90.19462698)(967.49526367,90.22463186)
\curveto(967.45526911,90.26462691)(967.42526914,90.30462687)(967.40526367,90.34463186)
\curveto(967.29526927,90.48462669)(967.1702694,90.60962657)(967.03026367,90.71963186)
\curveto(967.00026957,90.73962644)(966.97526959,90.76462641)(966.95526367,90.79463186)
\curveto(966.92526964,90.82462635)(966.89526967,90.84962633)(966.86526367,90.86963186)
\curveto(966.7652698,90.94962623)(966.6652699,91.01462616)(966.56526367,91.06463186)
\curveto(966.4652701,91.12462605)(966.35527021,91.179626)(966.23526367,91.22963186)
\curveto(966.1652704,91.25962592)(966.09027048,91.2796259)(966.01026367,91.28963186)
\lineto(965.77026367,91.34963186)
\lineto(965.68026367,91.34963186)
\curveto(965.65027092,91.35962582)(965.62027095,91.36462581)(965.59026367,91.36463186)
\curveto(965.52027105,91.38462579)(965.42527114,91.38962579)(965.30526367,91.37963186)
\curveto(965.17527139,91.3796258)(965.07527149,91.36962581)(965.00526367,91.34963186)
\curveto(964.92527164,91.32962585)(964.85027172,91.30962587)(964.78026367,91.28963186)
\curveto(964.70027187,91.2796259)(964.62027195,91.25962592)(964.54026367,91.22963186)
\curveto(964.30027227,91.11962606)(964.10027247,90.96962621)(963.94026367,90.77963186)
\curveto(963.7702728,90.59962658)(963.63027294,90.3796268)(963.52026367,90.11963186)
\curveto(963.50027307,90.04962713)(963.48527308,89.9796272)(963.47526367,89.90963186)
\curveto(963.45527311,89.83962734)(963.43527313,89.76462741)(963.41526367,89.68463186)
\curveto(963.39527317,89.60462757)(963.38527318,89.49462768)(963.38526367,89.35463186)
\curveto(963.38527318,89.22462795)(963.39527317,89.11962806)(963.41526367,89.03963186)
\curveto(963.42527314,88.9796282)(963.43027314,88.92462825)(963.43026367,88.87463186)
\curveto(963.43027314,88.82462835)(963.44027313,88.7746284)(963.46026367,88.72463186)
\curveto(963.50027307,88.62462855)(963.54027303,88.52962865)(963.58026367,88.43963186)
\curveto(963.62027295,88.35962882)(963.6652729,88.2796289)(963.71526367,88.19963186)
\curveto(963.73527283,88.16962901)(963.76027281,88.13962904)(963.79026367,88.10963186)
\curveto(963.82027275,88.08962909)(963.84527272,88.06462911)(963.86526367,88.03463186)
\lineto(963.94026367,87.95963186)
\curveto(963.96027261,87.92962925)(963.98027259,87.90462927)(964.00026367,87.88463186)
\lineto(964.21026367,87.73463186)
\curveto(964.2702723,87.69462948)(964.33527223,87.64962953)(964.40526367,87.59963186)
\curveto(964.49527207,87.53962964)(964.60027197,87.48962969)(964.72026367,87.44963186)
\curveto(964.83027174,87.41962976)(964.94027163,87.38462979)(965.05026367,87.34463186)
\curveto(965.16027141,87.30462987)(965.30527126,87.2796299)(965.48526367,87.26963186)
\curveto(965.65527091,87.25962992)(965.78027079,87.22962995)(965.86026367,87.17963186)
\curveto(965.94027063,87.12963005)(965.98527058,87.05463012)(965.99526367,86.95463186)
\curveto(966.00527056,86.85463032)(966.01027056,86.74463043)(966.01026367,86.62463186)
\curveto(966.01027056,86.58463059)(966.01527055,86.54463063)(966.02526367,86.50463186)
\curveto(966.02527054,86.46463071)(966.02027055,86.42963075)(966.01026367,86.39963186)
\curveto(965.99027058,86.34963083)(965.98027059,86.29963088)(965.98026367,86.24963186)
\curveto(965.98027059,86.20963097)(965.9702706,86.16963101)(965.95026367,86.12963186)
\curveto(965.89027068,86.03963114)(965.75527081,85.99463118)(965.54526367,85.99463186)
\lineto(965.42526367,85.99463186)
\curveto(965.3652712,86.00463117)(965.30527126,86.00963117)(965.24526367,86.00963186)
\curveto(965.17527139,86.01963116)(965.11027146,86.02963115)(965.05026367,86.03963186)
\curveto(964.94027163,86.05963112)(964.84027173,86.0796311)(964.75026367,86.09963186)
\curveto(964.65027192,86.11963106)(964.55527201,86.14963103)(964.46526367,86.18963186)
\curveto(964.39527217,86.20963097)(964.33527223,86.22963095)(964.28526367,86.24963186)
\lineto(964.10526367,86.30963186)
\curveto(963.84527272,86.42963075)(963.60027297,86.58463059)(963.37026367,86.77463186)
\curveto(963.14027343,86.9746302)(962.95527361,87.18962999)(962.81526367,87.41963186)
\curveto(962.73527383,87.52962965)(962.6702739,87.64462953)(962.62026367,87.76463186)
\lineto(962.47026367,88.15463186)
\curveto(962.42027415,88.26462891)(962.39027418,88.3796288)(962.38026367,88.49963186)
\curveto(962.36027421,88.61962856)(962.33527423,88.74462843)(962.30526367,88.87463186)
\curveto(962.30527426,88.94462823)(962.30527426,89.00962817)(962.30526367,89.06963186)
\curveto(962.29527427,89.12962805)(962.28527428,89.19462798)(962.27526367,89.26463186)
}
}
{
\newrgbcolor{curcolor}{0 0 0}
\pscustom[linestyle=none,fillstyle=solid,fillcolor=curcolor]
{
\newpath
\moveto(967.79526367,101.36424123)
\lineto(968.05026367,101.36424123)
\curveto(968.13026844,101.37423353)(968.20526836,101.36923353)(968.27526367,101.34924123)
\lineto(968.51526367,101.34924123)
\lineto(968.68026367,101.34924123)
\curveto(968.78026779,101.32923357)(968.88526768,101.31923358)(968.99526367,101.31924123)
\curveto(969.09526747,101.31923358)(969.19526737,101.30923359)(969.29526367,101.28924123)
\lineto(969.44526367,101.28924123)
\curveto(969.58526698,101.25923364)(969.72526684,101.23923366)(969.86526367,101.22924123)
\curveto(969.99526657,101.21923368)(970.12526644,101.19423371)(970.25526367,101.15424123)
\curveto(970.33526623,101.13423377)(970.42026615,101.11423379)(970.51026367,101.09424123)
\lineto(970.75026367,101.03424123)
\lineto(971.05026367,100.91424123)
\curveto(971.14026543,100.88423402)(971.23026534,100.84923405)(971.32026367,100.80924123)
\curveto(971.54026503,100.70923419)(971.75526481,100.57423433)(971.96526367,100.40424123)
\curveto(972.17526439,100.24423466)(972.34526422,100.06923483)(972.47526367,99.87924123)
\curveto(972.51526405,99.82923507)(972.55526401,99.76923513)(972.59526367,99.69924123)
\curveto(972.62526394,99.63923526)(972.66026391,99.57923532)(972.70026367,99.51924123)
\curveto(972.75026382,99.43923546)(972.79026378,99.34423556)(972.82026367,99.23424123)
\curveto(972.85026372,99.12423578)(972.88026369,99.01923588)(972.91026367,98.91924123)
\curveto(972.95026362,98.80923609)(972.97526359,98.6992362)(972.98526367,98.58924123)
\curveto(972.99526357,98.47923642)(973.01026356,98.36423654)(973.03026367,98.24424123)
\curveto(973.04026353,98.2042367)(973.04026353,98.15923674)(973.03026367,98.10924123)
\curveto(973.03026354,98.06923683)(973.03526353,98.02923687)(973.04526367,97.98924123)
\curveto(973.05526351,97.94923695)(973.06026351,97.89423701)(973.06026367,97.82424123)
\curveto(973.06026351,97.75423715)(973.05526351,97.7042372)(973.04526367,97.67424123)
\curveto(973.02526354,97.62423728)(973.02026355,97.57923732)(973.03026367,97.53924123)
\curveto(973.04026353,97.4992374)(973.04026353,97.46423744)(973.03026367,97.43424123)
\lineto(973.03026367,97.34424123)
\curveto(973.01026356,97.28423762)(972.99526357,97.21923768)(972.98526367,97.14924123)
\curveto(972.98526358,97.08923781)(972.98026359,97.02423788)(972.97026367,96.95424123)
\curveto(972.92026365,96.78423812)(972.8702637,96.62423828)(972.82026367,96.47424123)
\curveto(972.7702638,96.32423858)(972.70526386,96.17923872)(972.62526367,96.03924123)
\curveto(972.58526398,95.98923891)(972.55526401,95.93423897)(972.53526367,95.87424123)
\curveto(972.50526406,95.82423908)(972.4702641,95.77423913)(972.43026367,95.72424123)
\curveto(972.25026432,95.48423942)(972.03026454,95.28423962)(971.77026367,95.12424123)
\curveto(971.51026506,94.96423994)(971.22526534,94.82424008)(970.91526367,94.70424123)
\curveto(970.77526579,94.64424026)(970.63526593,94.5992403)(970.49526367,94.56924123)
\curveto(970.34526622,94.53924036)(970.19026638,94.5042404)(970.03026367,94.46424123)
\curveto(969.92026665,94.44424046)(969.81026676,94.42924047)(969.70026367,94.41924123)
\curveto(969.59026698,94.40924049)(969.48026709,94.39424051)(969.37026367,94.37424123)
\curveto(969.33026724,94.36424054)(969.29026728,94.35924054)(969.25026367,94.35924123)
\curveto(969.21026736,94.36924053)(969.1702674,94.36924053)(969.13026367,94.35924123)
\curveto(969.08026749,94.34924055)(969.03026754,94.34424056)(968.98026367,94.34424123)
\lineto(968.81526367,94.34424123)
\curveto(968.7652678,94.32424058)(968.71526785,94.31924058)(968.66526367,94.32924123)
\curveto(968.60526796,94.33924056)(968.55026802,94.33924056)(968.50026367,94.32924123)
\curveto(968.46026811,94.31924058)(968.41526815,94.31924058)(968.36526367,94.32924123)
\curveto(968.31526825,94.33924056)(968.2652683,94.33424057)(968.21526367,94.31424123)
\curveto(968.14526842,94.29424061)(968.0702685,94.28924061)(967.99026367,94.29924123)
\curveto(967.90026867,94.30924059)(967.81526875,94.31424059)(967.73526367,94.31424123)
\curveto(967.64526892,94.31424059)(967.54526902,94.30924059)(967.43526367,94.29924123)
\curveto(967.31526925,94.28924061)(967.21526935,94.29424061)(967.13526367,94.31424123)
\lineto(966.85026367,94.31424123)
\lineto(966.22026367,94.35924123)
\curveto(966.12027045,94.36924053)(966.02527054,94.37924052)(965.93526367,94.38924123)
\lineto(965.63526367,94.41924123)
\curveto(965.58527098,94.43924046)(965.53527103,94.44424046)(965.48526367,94.43424123)
\curveto(965.42527114,94.43424047)(965.3702712,94.44424046)(965.32026367,94.46424123)
\curveto(965.15027142,94.51424039)(964.98527158,94.55424035)(964.82526367,94.58424123)
\curveto(964.65527191,94.61424029)(964.49527207,94.66424024)(964.34526367,94.73424123)
\curveto(963.88527268,94.92423998)(963.51027306,95.14423976)(963.22026367,95.39424123)
\curveto(962.93027364,95.65423925)(962.68527388,96.01423889)(962.48526367,96.47424123)
\curveto(962.43527413,96.6042383)(962.40027417,96.73423817)(962.38026367,96.86424123)
\curveto(962.36027421,97.0042379)(962.33527423,97.14423776)(962.30526367,97.28424123)
\curveto(962.29527427,97.35423755)(962.29027428,97.41923748)(962.29026367,97.47924123)
\curveto(962.29027428,97.53923736)(962.28527428,97.6042373)(962.27526367,97.67424123)
\curveto(962.25527431,98.5042364)(962.40527416,99.17423573)(962.72526367,99.68424123)
\curveto(963.03527353,100.19423471)(963.47527309,100.57423433)(964.04526367,100.82424123)
\curveto(964.1652724,100.87423403)(964.29027228,100.91923398)(964.42026367,100.95924123)
\curveto(964.55027202,100.9992339)(964.68527188,101.04423386)(964.82526367,101.09424123)
\curveto(964.90527166,101.11423379)(964.99027158,101.12923377)(965.08026367,101.13924123)
\lineto(965.32026367,101.19924123)
\curveto(965.43027114,101.22923367)(965.54027103,101.24423366)(965.65026367,101.24424123)
\curveto(965.76027081,101.25423365)(965.8702707,101.26923363)(965.98026367,101.28924123)
\curveto(966.03027054,101.30923359)(966.07527049,101.31423359)(966.11526367,101.30424123)
\curveto(966.15527041,101.3042336)(966.19527037,101.30923359)(966.23526367,101.31924123)
\curveto(966.28527028,101.32923357)(966.34027023,101.32923357)(966.40026367,101.31924123)
\curveto(966.45027012,101.31923358)(966.50027007,101.32423358)(966.55026367,101.33424123)
\lineto(966.68526367,101.33424123)
\curveto(966.74526982,101.35423355)(966.81526975,101.35423355)(966.89526367,101.33424123)
\curveto(966.9652696,101.32423358)(967.03026954,101.32923357)(967.09026367,101.34924123)
\curveto(967.12026945,101.35923354)(967.16026941,101.36423354)(967.21026367,101.36424123)
\lineto(967.33026367,101.36424123)
\lineto(967.79526367,101.36424123)
\moveto(970.12026367,99.81924123)
\curveto(969.80026677,99.91923498)(969.43526713,99.97923492)(969.02526367,99.99924123)
\curveto(968.61526795,100.01923488)(968.20526836,100.02923487)(967.79526367,100.02924123)
\curveto(967.3652692,100.02923487)(966.94526962,100.01923488)(966.53526367,99.99924123)
\curveto(966.12527044,99.97923492)(965.74027083,99.93423497)(965.38026367,99.86424123)
\curveto(965.02027155,99.79423511)(964.70027187,99.68423522)(964.42026367,99.53424123)
\curveto(964.13027244,99.39423551)(963.89527267,99.1992357)(963.71526367,98.94924123)
\curveto(963.60527296,98.78923611)(963.52527304,98.60923629)(963.47526367,98.40924123)
\curveto(963.41527315,98.20923669)(963.38527318,97.96423694)(963.38526367,97.67424123)
\curveto(963.40527316,97.65423725)(963.41527315,97.61923728)(963.41526367,97.56924123)
\curveto(963.40527316,97.51923738)(963.40527316,97.47923742)(963.41526367,97.44924123)
\curveto(963.43527313,97.36923753)(963.45527311,97.29423761)(963.47526367,97.22424123)
\curveto(963.48527308,97.16423774)(963.50527306,97.0992378)(963.53526367,97.02924123)
\curveto(963.65527291,96.75923814)(963.82527274,96.53923836)(964.04526367,96.36924123)
\curveto(964.25527231,96.20923869)(964.50027207,96.07423883)(964.78026367,95.96424123)
\curveto(964.89027168,95.91423899)(965.01027156,95.87423903)(965.14026367,95.84424123)
\curveto(965.26027131,95.82423908)(965.38527118,95.7992391)(965.51526367,95.76924123)
\curveto(965.565271,95.74923915)(965.62027095,95.73923916)(965.68026367,95.73924123)
\curveto(965.73027084,95.73923916)(965.78027079,95.73423917)(965.83026367,95.72424123)
\curveto(965.92027065,95.71423919)(966.01527055,95.7042392)(966.11526367,95.69424123)
\curveto(966.20527036,95.68423922)(966.30027027,95.67423923)(966.40026367,95.66424123)
\curveto(966.48027009,95.66423924)(966.56527,95.65923924)(966.65526367,95.64924123)
\lineto(966.89526367,95.64924123)
\lineto(967.07526367,95.64924123)
\curveto(967.10526946,95.63923926)(967.14026943,95.63423927)(967.18026367,95.63424123)
\lineto(967.31526367,95.63424123)
\lineto(967.76526367,95.63424123)
\curveto(967.84526872,95.63423927)(967.93026864,95.62923927)(968.02026367,95.61924123)
\curveto(968.10026847,95.61923928)(968.17526839,95.62923927)(968.24526367,95.64924123)
\lineto(968.51526367,95.64924123)
\curveto(968.53526803,95.64923925)(968.565268,95.64423926)(968.60526367,95.63424123)
\curveto(968.63526793,95.63423927)(968.66026791,95.63923926)(968.68026367,95.64924123)
\curveto(968.78026779,95.65923924)(968.88026769,95.66423924)(968.98026367,95.66424123)
\curveto(969.0702675,95.67423923)(969.1702674,95.68423922)(969.28026367,95.69424123)
\curveto(969.40026717,95.72423918)(969.52526704,95.73923916)(969.65526367,95.73924123)
\curveto(969.77526679,95.74923915)(969.89026668,95.77423913)(970.00026367,95.81424123)
\curveto(970.30026627,95.89423901)(970.565266,95.97923892)(970.79526367,96.06924123)
\curveto(971.02526554,96.16923873)(971.24026533,96.31423859)(971.44026367,96.50424123)
\curveto(971.64026493,96.71423819)(971.79026478,96.97923792)(971.89026367,97.29924123)
\curveto(971.91026466,97.33923756)(971.92026465,97.37423753)(971.92026367,97.40424123)
\curveto(971.91026466,97.44423746)(971.91526465,97.48923741)(971.93526367,97.53924123)
\curveto(971.94526462,97.57923732)(971.95526461,97.64923725)(971.96526367,97.74924123)
\curveto(971.97526459,97.85923704)(971.9702646,97.94423696)(971.95026367,98.00424123)
\curveto(971.93026464,98.07423683)(971.92026465,98.14423676)(971.92026367,98.21424123)
\curveto(971.91026466,98.28423662)(971.89526467,98.34923655)(971.87526367,98.40924123)
\curveto(971.81526475,98.60923629)(971.73026484,98.78923611)(971.62026367,98.94924123)
\curveto(971.60026497,98.97923592)(971.58026499,99.0042359)(971.56026367,99.02424123)
\lineto(971.50026367,99.08424123)
\curveto(971.48026509,99.12423578)(971.44026513,99.17423573)(971.38026367,99.23424123)
\curveto(971.24026533,99.33423557)(971.11026546,99.41923548)(970.99026367,99.48924123)
\curveto(970.8702657,99.55923534)(970.72526584,99.62923527)(970.55526367,99.69924123)
\curveto(970.48526608,99.72923517)(970.41526615,99.74923515)(970.34526367,99.75924123)
\curveto(970.27526629,99.77923512)(970.20026637,99.7992351)(970.12026367,99.81924123)
}
}
{
\newrgbcolor{curcolor}{0 0 0}
\pscustom[linestyle=none,fillstyle=solid,fillcolor=curcolor]
{
\newpath
\moveto(962.27526367,106.77385061)
\curveto(962.27527429,106.87384575)(962.28527428,106.96884566)(962.30526367,107.05885061)
\curveto(962.31527425,107.14884548)(962.34527422,107.21384541)(962.39526367,107.25385061)
\curveto(962.47527409,107.31384531)(962.58027399,107.34384528)(962.71026367,107.34385061)
\lineto(963.10026367,107.34385061)
\lineto(964.60026367,107.34385061)
\lineto(970.99026367,107.34385061)
\lineto(972.16026367,107.34385061)
\lineto(972.47526367,107.34385061)
\curveto(972.57526399,107.35384527)(972.65526391,107.33884529)(972.71526367,107.29885061)
\curveto(972.79526377,107.24884538)(972.84526372,107.17384545)(972.86526367,107.07385061)
\curveto(972.87526369,106.98384564)(972.88026369,106.87384575)(972.88026367,106.74385061)
\lineto(972.88026367,106.51885061)
\curveto(972.86026371,106.43884619)(972.84526372,106.36884626)(972.83526367,106.30885061)
\curveto(972.81526375,106.24884638)(972.77526379,106.19884643)(972.71526367,106.15885061)
\curveto(972.65526391,106.11884651)(972.58026399,106.09884653)(972.49026367,106.09885061)
\lineto(972.19026367,106.09885061)
\lineto(971.09526367,106.09885061)
\lineto(965.75526367,106.09885061)
\curveto(965.6652709,106.07884655)(965.59027098,106.06384656)(965.53026367,106.05385061)
\curveto(965.46027111,106.05384657)(965.40027117,106.0238466)(965.35026367,105.96385061)
\curveto(965.30027127,105.89384673)(965.27527129,105.80384682)(965.27526367,105.69385061)
\curveto(965.2652713,105.59384703)(965.26027131,105.48384714)(965.26026367,105.36385061)
\lineto(965.26026367,104.22385061)
\lineto(965.26026367,103.72885061)
\curveto(965.25027132,103.56884906)(965.19027138,103.45884917)(965.08026367,103.39885061)
\curveto(965.05027152,103.37884925)(965.02027155,103.36884926)(964.99026367,103.36885061)
\curveto(964.95027162,103.36884926)(964.90527166,103.36384926)(964.85526367,103.35385061)
\curveto(964.73527183,103.33384929)(964.62527194,103.33884929)(964.52526367,103.36885061)
\curveto(964.42527214,103.40884922)(964.35527221,103.46384916)(964.31526367,103.53385061)
\curveto(964.2652723,103.61384901)(964.24027233,103.73384889)(964.24026367,103.89385061)
\curveto(964.24027233,104.05384857)(964.22527234,104.18884844)(964.19526367,104.29885061)
\curveto(964.18527238,104.34884828)(964.18027239,104.40384822)(964.18026367,104.46385061)
\curveto(964.1702724,104.5238481)(964.15527241,104.58384804)(964.13526367,104.64385061)
\curveto(964.08527248,104.79384783)(964.03527253,104.93884769)(963.98526367,105.07885061)
\curveto(963.92527264,105.21884741)(963.85527271,105.35384727)(963.77526367,105.48385061)
\curveto(963.68527288,105.623847)(963.58027299,105.74384688)(963.46026367,105.84385061)
\curveto(963.34027323,105.94384668)(963.21027336,106.03884659)(963.07026367,106.12885061)
\curveto(962.9702736,106.18884644)(962.86027371,106.23384639)(962.74026367,106.26385061)
\curveto(962.62027395,106.30384632)(962.51527405,106.35384627)(962.42526367,106.41385061)
\curveto(962.3652742,106.46384616)(962.32527424,106.53384609)(962.30526367,106.62385061)
\curveto(962.29527427,106.64384598)(962.29027428,106.66884596)(962.29026367,106.69885061)
\curveto(962.29027428,106.7288459)(962.28527428,106.75384587)(962.27526367,106.77385061)
}
}
{
\newrgbcolor{curcolor}{0 0 0}
\pscustom[linestyle=none,fillstyle=solid,fillcolor=curcolor]
{
\newpath
\moveto(962.27526367,115.12345998)
\curveto(962.27527429,115.22345513)(962.28527428,115.31845503)(962.30526367,115.40845998)
\curveto(962.31527425,115.49845485)(962.34527422,115.56345479)(962.39526367,115.60345998)
\curveto(962.47527409,115.66345469)(962.58027399,115.69345466)(962.71026367,115.69345998)
\lineto(963.10026367,115.69345998)
\lineto(964.60026367,115.69345998)
\lineto(970.99026367,115.69345998)
\lineto(972.16026367,115.69345998)
\lineto(972.47526367,115.69345998)
\curveto(972.57526399,115.70345465)(972.65526391,115.68845466)(972.71526367,115.64845998)
\curveto(972.79526377,115.59845475)(972.84526372,115.52345483)(972.86526367,115.42345998)
\curveto(972.87526369,115.33345502)(972.88026369,115.22345513)(972.88026367,115.09345998)
\lineto(972.88026367,114.86845998)
\curveto(972.86026371,114.78845556)(972.84526372,114.71845563)(972.83526367,114.65845998)
\curveto(972.81526375,114.59845575)(972.77526379,114.5484558)(972.71526367,114.50845998)
\curveto(972.65526391,114.46845588)(972.58026399,114.4484559)(972.49026367,114.44845998)
\lineto(972.19026367,114.44845998)
\lineto(971.09526367,114.44845998)
\lineto(965.75526367,114.44845998)
\curveto(965.6652709,114.42845592)(965.59027098,114.41345594)(965.53026367,114.40345998)
\curveto(965.46027111,114.40345595)(965.40027117,114.37345598)(965.35026367,114.31345998)
\curveto(965.30027127,114.24345611)(965.27527129,114.1534562)(965.27526367,114.04345998)
\curveto(965.2652713,113.94345641)(965.26027131,113.83345652)(965.26026367,113.71345998)
\lineto(965.26026367,112.57345998)
\lineto(965.26026367,112.07845998)
\curveto(965.25027132,111.91845843)(965.19027138,111.80845854)(965.08026367,111.74845998)
\curveto(965.05027152,111.72845862)(965.02027155,111.71845863)(964.99026367,111.71845998)
\curveto(964.95027162,111.71845863)(964.90527166,111.71345864)(964.85526367,111.70345998)
\curveto(964.73527183,111.68345867)(964.62527194,111.68845866)(964.52526367,111.71845998)
\curveto(964.42527214,111.75845859)(964.35527221,111.81345854)(964.31526367,111.88345998)
\curveto(964.2652723,111.96345839)(964.24027233,112.08345827)(964.24026367,112.24345998)
\curveto(964.24027233,112.40345795)(964.22527234,112.53845781)(964.19526367,112.64845998)
\curveto(964.18527238,112.69845765)(964.18027239,112.7534576)(964.18026367,112.81345998)
\curveto(964.1702724,112.87345748)(964.15527241,112.93345742)(964.13526367,112.99345998)
\curveto(964.08527248,113.14345721)(964.03527253,113.28845706)(963.98526367,113.42845998)
\curveto(963.92527264,113.56845678)(963.85527271,113.70345665)(963.77526367,113.83345998)
\curveto(963.68527288,113.97345638)(963.58027299,114.09345626)(963.46026367,114.19345998)
\curveto(963.34027323,114.29345606)(963.21027336,114.38845596)(963.07026367,114.47845998)
\curveto(962.9702736,114.53845581)(962.86027371,114.58345577)(962.74026367,114.61345998)
\curveto(962.62027395,114.6534557)(962.51527405,114.70345565)(962.42526367,114.76345998)
\curveto(962.3652742,114.81345554)(962.32527424,114.88345547)(962.30526367,114.97345998)
\curveto(962.29527427,114.99345536)(962.29027428,115.01845533)(962.29026367,115.04845998)
\curveto(962.29027428,115.07845527)(962.28527428,115.10345525)(962.27526367,115.12345998)
}
}
{
\newrgbcolor{curcolor}{0 0 0}
\pscustom[linestyle=none,fillstyle=solid,fillcolor=curcolor]
{
\newpath
\moveto(994.12156494,38.71181936)
\curveto(994.17156569,38.73180981)(994.23156563,38.75680979)(994.30156494,38.78681936)
\curveto(994.37156549,38.81680973)(994.44656541,38.83680971)(994.52656494,38.84681936)
\curveto(994.59656526,38.86680968)(994.66656519,38.86680968)(994.73656494,38.84681936)
\curveto(994.79656506,38.83680971)(994.84156502,38.79680975)(994.87156494,38.72681936)
\curveto(994.89156497,38.67680987)(994.90156496,38.61680993)(994.90156494,38.54681936)
\lineto(994.90156494,38.33681936)
\lineto(994.90156494,37.88681936)
\curveto(994.90156496,37.73681081)(994.87656498,37.61681093)(994.82656494,37.52681936)
\curveto(994.76656509,37.42681112)(994.6615652,37.35181119)(994.51156494,37.30181936)
\curveto(994.3615655,37.26181128)(994.22656563,37.21681133)(994.10656494,37.16681936)
\curveto(993.84656601,37.05681149)(993.57656628,36.95681159)(993.29656494,36.86681936)
\curveto(993.01656684,36.77681177)(992.74156712,36.67681187)(992.47156494,36.56681936)
\curveto(992.38156748,36.53681201)(992.29656756,36.50681204)(992.21656494,36.47681936)
\curveto(992.13656772,36.45681209)(992.0615678,36.42681212)(991.99156494,36.38681936)
\curveto(991.92156794,36.35681219)(991.861568,36.31181223)(991.81156494,36.25181936)
\curveto(991.7615681,36.19181235)(991.72156814,36.11181243)(991.69156494,36.01181936)
\curveto(991.67156819,35.96181258)(991.66656819,35.90181264)(991.67656494,35.83181936)
\lineto(991.67656494,35.63681936)
\lineto(991.67656494,32.80181936)
\lineto(991.67656494,32.50181936)
\curveto(991.66656819,32.39181615)(991.66656819,32.28681626)(991.67656494,32.18681936)
\curveto(991.68656817,32.08681646)(991.70156816,31.99181655)(991.72156494,31.90181936)
\curveto(991.74156812,31.82181672)(991.78156808,31.76181678)(991.84156494,31.72181936)
\curveto(991.94156792,31.6418169)(992.0565678,31.58181696)(992.18656494,31.54181936)
\curveto(992.30656755,31.51181703)(992.43156743,31.47181707)(992.56156494,31.42181936)
\curveto(992.79156707,31.32181722)(993.03156683,31.22681732)(993.28156494,31.13681936)
\curveto(993.53156633,31.05681749)(993.77156609,30.96681758)(994.00156494,30.86681936)
\curveto(994.0615658,30.8468177)(994.13156573,30.82181772)(994.21156494,30.79181936)
\curveto(994.28156558,30.77181777)(994.3565655,30.7468178)(994.43656494,30.71681936)
\curveto(994.51656534,30.68681786)(994.59156527,30.65181789)(994.66156494,30.61181936)
\curveto(994.72156514,30.58181796)(994.76656509,30.546818)(994.79656494,30.50681936)
\curveto(994.856565,30.42681812)(994.89156497,30.31681823)(994.90156494,30.17681936)
\lineto(994.90156494,29.75681936)
\lineto(994.90156494,29.51681936)
\curveto(994.89156497,29.4468191)(994.86656499,29.38681916)(994.82656494,29.33681936)
\curveto(994.79656506,29.28681926)(994.75156511,29.25681929)(994.69156494,29.24681936)
\curveto(994.63156523,29.2468193)(994.57156529,29.25181929)(994.51156494,29.26181936)
\curveto(994.44156542,29.28181926)(994.37656548,29.30181924)(994.31656494,29.32181936)
\curveto(994.24656561,29.35181919)(994.19656566,29.37681917)(994.16656494,29.39681936)
\curveto(993.84656601,29.53681901)(993.53156633,29.66181888)(993.22156494,29.77181936)
\curveto(992.90156696,29.88181866)(992.58156728,30.00181854)(992.26156494,30.13181936)
\curveto(992.04156782,30.22181832)(991.82656803,30.30681824)(991.61656494,30.38681936)
\curveto(991.39656846,30.46681808)(991.17656868,30.55181799)(990.95656494,30.64181936)
\curveto(990.23656962,30.9418176)(989.51157035,31.22681732)(988.78156494,31.49681936)
\curveto(988.04157182,31.76681678)(987.30657255,32.05181649)(986.57656494,32.35181936)
\curveto(986.31657354,32.46181608)(986.05157381,32.56181598)(985.78156494,32.65181936)
\curveto(985.51157435,32.75181579)(985.24657461,32.85681569)(984.98656494,32.96681936)
\curveto(984.87657498,33.01681553)(984.7565751,33.06181548)(984.62656494,33.10181936)
\curveto(984.48657537,33.15181539)(984.38657547,33.22181532)(984.32656494,33.31181936)
\curveto(984.28657557,33.35181519)(984.2565756,33.41681513)(984.23656494,33.50681936)
\curveto(984.22657563,33.52681502)(984.22657563,33.546815)(984.23656494,33.56681936)
\curveto(984.23657562,33.59681495)(984.23157563,33.62181492)(984.22156494,33.64181936)
\curveto(984.22157564,33.82181472)(984.22157564,34.03181451)(984.22156494,34.27181936)
\curveto(984.21157565,34.51181403)(984.24657561,34.68681386)(984.32656494,34.79681936)
\curveto(984.38657547,34.87681367)(984.48657537,34.93681361)(984.62656494,34.97681936)
\curveto(984.7565751,35.02681352)(984.87657498,35.07681347)(984.98656494,35.12681936)
\curveto(985.21657464,35.22681332)(985.44657441,35.31681323)(985.67656494,35.39681936)
\curveto(985.90657395,35.47681307)(986.13657372,35.56681298)(986.36656494,35.66681936)
\curveto(986.56657329,35.7468128)(986.77157309,35.82181272)(986.98156494,35.89181936)
\curveto(987.19157267,35.97181257)(987.39657246,36.05681249)(987.59656494,36.14681936)
\curveto(988.32657153,36.4468121)(989.06657079,36.73181181)(989.81656494,37.00181936)
\curveto(990.5565693,37.28181126)(991.29156857,37.57681097)(992.02156494,37.88681936)
\curveto(992.11156775,37.92681062)(992.19656766,37.95681059)(992.27656494,37.97681936)
\curveto(992.3565675,38.00681054)(992.44156742,38.03681051)(992.53156494,38.06681936)
\curveto(992.79156707,38.17681037)(993.0565668,38.28181026)(993.32656494,38.38181936)
\curveto(993.59656626,38.49181005)(993.861566,38.60180994)(994.12156494,38.71181936)
\moveto(990.47656494,35.50181936)
\curveto(990.44656941,35.59181295)(990.39656946,35.6468129)(990.32656494,35.66681936)
\curveto(990.2565696,35.69681285)(990.18156968,35.70181284)(990.10156494,35.68181936)
\curveto(990.01156985,35.67181287)(989.92656993,35.6468129)(989.84656494,35.60681936)
\curveto(989.7565701,35.57681297)(989.68157018,35.546813)(989.62156494,35.51681936)
\curveto(989.58157028,35.49681305)(989.54657031,35.48681306)(989.51656494,35.48681936)
\curveto(989.48657037,35.48681306)(989.45157041,35.47681307)(989.41156494,35.45681936)
\lineto(989.17156494,35.36681936)
\curveto(989.08157078,35.3468132)(988.99157087,35.31681323)(988.90156494,35.27681936)
\curveto(988.54157132,35.12681342)(988.17657168,34.99181355)(987.80656494,34.87181936)
\curveto(987.42657243,34.76181378)(987.0565728,34.63181391)(986.69656494,34.48181936)
\curveto(986.58657327,34.43181411)(986.47657338,34.38681416)(986.36656494,34.34681936)
\curveto(986.2565736,34.31681423)(986.15157371,34.27681427)(986.05156494,34.22681936)
\curveto(986.00157386,34.20681434)(985.9565739,34.18181436)(985.91656494,34.15181936)
\curveto(985.86657399,34.13181441)(985.84157402,34.08181446)(985.84156494,34.00181936)
\curveto(985.861574,33.98181456)(985.87657398,33.96181458)(985.88656494,33.94181936)
\curveto(985.89657396,33.92181462)(985.91157395,33.90181464)(985.93156494,33.88181936)
\curveto(985.98157388,33.8418147)(986.03657382,33.81181473)(986.09656494,33.79181936)
\curveto(986.14657371,33.77181477)(986.20157366,33.75181479)(986.26156494,33.73181936)
\curveto(986.37157349,33.68181486)(986.48157338,33.6418149)(986.59156494,33.61181936)
\curveto(986.70157316,33.58181496)(986.81157305,33.541815)(986.92156494,33.49181936)
\curveto(987.31157255,33.32181522)(987.70657215,33.17181537)(988.10656494,33.04181936)
\curveto(988.50657135,32.92181562)(988.89657096,32.78181576)(989.27656494,32.62181936)
\lineto(989.42656494,32.56181936)
\curveto(989.47657038,32.55181599)(989.52657033,32.53681601)(989.57656494,32.51681936)
\lineto(989.81656494,32.42681936)
\curveto(989.89656996,32.39681615)(989.97656988,32.37181617)(990.05656494,32.35181936)
\curveto(990.10656975,32.33181621)(990.1615697,32.32181622)(990.22156494,32.32181936)
\curveto(990.28156958,32.33181621)(990.33156953,32.3468162)(990.37156494,32.36681936)
\curveto(990.45156941,32.41681613)(990.49656936,32.52181602)(990.50656494,32.68181936)
\lineto(990.50656494,33.13181936)
\lineto(990.50656494,34.73681936)
\curveto(990.50656935,34.8468137)(990.51156935,34.98181356)(990.52156494,35.14181936)
\curveto(990.52156934,35.30181324)(990.50656935,35.42181312)(990.47656494,35.50181936)
}
}
{
\newrgbcolor{curcolor}{0 0 0}
\pscustom[linestyle=none,fillstyle=solid,fillcolor=curcolor]
{
\newpath
\moveto(987.28156494,46.44338186)
\curveto(987.33157253,46.51337426)(987.40657245,46.54837422)(987.50656494,46.54838186)
\curveto(987.60657225,46.55837421)(987.71157215,46.56337421)(987.82156494,46.56338186)
\lineto(994.09156494,46.56338186)
\lineto(994.69156494,46.56338186)
\curveto(994.74156512,46.54337423)(994.79156507,46.53837423)(994.84156494,46.54838186)
\curveto(994.88156498,46.55837421)(994.92656493,46.55337422)(994.97656494,46.53338186)
\curveto(995.07656478,46.51337426)(995.17656468,46.49837427)(995.27656494,46.48838186)
\curveto(995.38656447,46.48837428)(995.49156437,46.4733743)(995.59156494,46.44338186)
\curveto(995.70156416,46.41337436)(995.80656405,46.38337439)(995.90656494,46.35338186)
\curveto(996.00656385,46.33337444)(996.10656375,46.29837447)(996.20656494,46.24838186)
\curveto(996.46656339,46.14837462)(996.70156316,46.01837475)(996.91156494,45.85838186)
\curveto(997.12156274,45.70837506)(997.29656256,45.52837524)(997.43656494,45.31838186)
\curveto(997.5565623,45.14837562)(997.65156221,44.9683758)(997.72156494,44.77838186)
\curveto(997.80156206,44.58837618)(997.87656198,44.38337639)(997.94656494,44.16338186)
\curveto(997.96656189,44.0733767)(997.97656188,43.98337679)(997.97656494,43.89338186)
\curveto(997.98656187,43.80337697)(998.00156186,43.71337706)(998.02156494,43.62338186)
\lineto(998.02156494,43.53338186)
\curveto(998.03156183,43.51337726)(998.03656182,43.49337728)(998.03656494,43.47338186)
\curveto(998.04656181,43.42337735)(998.04656181,43.3733774)(998.03656494,43.32338186)
\curveto(998.02656183,43.28337749)(998.03156183,43.23837753)(998.05156494,43.18838186)
\curveto(998.07156179,43.11837765)(998.07656178,43.00837776)(998.06656494,42.85838186)
\curveto(998.06656179,42.71837805)(998.0565618,42.61837815)(998.03656494,42.55838186)
\curveto(998.03656182,42.52837824)(998.03156183,42.49837827)(998.02156494,42.46838186)
\lineto(998.02156494,42.40838186)
\curveto(998.00156186,42.31837845)(997.98656187,42.22837854)(997.97656494,42.13838186)
\curveto(997.97656188,42.04837872)(997.96656189,41.96337881)(997.94656494,41.88338186)
\curveto(997.92656193,41.80337897)(997.90156196,41.72337905)(997.87156494,41.64338186)
\curveto(997.85156201,41.56337921)(997.82656203,41.48337929)(997.79656494,41.40338186)
\curveto(997.66656219,41.08337969)(997.52156234,40.81337996)(997.36156494,40.59338186)
\curveto(997.20156266,40.38338039)(996.97656288,40.19338058)(996.68656494,40.02338186)
\curveto(996.66656319,40.00338077)(996.64156322,39.98838078)(996.61156494,39.97838186)
\curveto(996.59156327,39.97838079)(996.56656329,39.9683808)(996.53656494,39.94838186)
\curveto(996.4565634,39.91838085)(996.34156352,39.88338089)(996.19156494,39.84338186)
\curveto(996.05156381,39.81338096)(995.94656391,39.84338093)(995.87656494,39.93338186)
\curveto(995.82656403,39.99338078)(995.80156406,40.0733807)(995.80156494,40.17338186)
\lineto(995.80156494,40.50338186)
\lineto(995.80156494,40.66838186)
\curveto(995.80156406,40.72838004)(995.81156405,40.78337999)(995.83156494,40.83338186)
\curveto(995.861564,40.92337985)(995.91156395,40.98837978)(995.98156494,41.02838186)
\curveto(996.05156381,41.0683797)(996.12656373,41.11337966)(996.20656494,41.16338186)
\lineto(996.38656494,41.28338186)
\curveto(996.4565634,41.33337944)(996.51156335,41.38337939)(996.55156494,41.43338186)
\curveto(996.74156312,41.68337909)(996.88156298,41.98337879)(996.97156494,42.33338186)
\curveto(996.99156287,42.39337838)(997.00156286,42.45337832)(997.00156494,42.51338186)
\curveto(997.01156285,42.58337819)(997.02656283,42.64837812)(997.04656494,42.70838186)
\lineto(997.04656494,42.79838186)
\curveto(997.06656279,42.8683779)(997.07656278,42.95337782)(997.07656494,43.05338186)
\curveto(997.07656278,43.15337762)(997.06656279,43.24337753)(997.04656494,43.32338186)
\curveto(997.03656282,43.35337742)(997.03156283,43.39337738)(997.03156494,43.44338186)
\curveto(997.01156285,43.54337723)(996.99156287,43.63837713)(996.97156494,43.72838186)
\curveto(996.9615629,43.81837695)(996.93656292,43.90337687)(996.89656494,43.98338186)
\curveto(996.77656308,44.2733765)(996.61156325,44.50837626)(996.40156494,44.68838186)
\curveto(996.20156366,44.87837589)(995.9565639,45.03337574)(995.66656494,45.15338186)
\curveto(995.57656428,45.19337558)(995.48156438,45.21837555)(995.38156494,45.22838186)
\curveto(995.28156458,45.24837552)(995.17656468,45.2733755)(995.06656494,45.30338186)
\curveto(995.01656484,45.32337545)(994.96656489,45.33337544)(994.91656494,45.33338186)
\curveto(994.86656499,45.33337544)(994.81656504,45.33837543)(994.76656494,45.34838186)
\curveto(994.73656512,45.35837541)(994.68656517,45.36337541)(994.61656494,45.36338186)
\curveto(994.53656532,45.38337539)(994.45156541,45.38337539)(994.36156494,45.36338186)
\curveto(994.31156555,45.35337542)(994.26656559,45.34837542)(994.22656494,45.34838186)
\curveto(994.18656567,45.35837541)(994.15156571,45.35337542)(994.12156494,45.33338186)
\curveto(994.10156576,45.31337546)(994.09156577,45.29837547)(994.09156494,45.28838186)
\lineto(994.04656494,45.24338186)
\curveto(994.04656581,45.14337563)(994.07656578,45.0683757)(994.13656494,45.01838186)
\curveto(994.18656567,44.97837579)(994.23156563,44.92837584)(994.27156494,44.86838186)
\lineto(994.48156494,44.62838186)
\curveto(994.54156532,44.54837622)(994.59656526,44.45837631)(994.64656494,44.35838186)
\curveto(994.73656512,44.21837655)(994.81156505,44.04337673)(994.87156494,43.83338186)
\curveto(994.92156494,43.62337715)(994.9565649,43.40337737)(994.97656494,43.17338186)
\curveto(994.99656486,42.94337783)(994.99156487,42.71337806)(994.96156494,42.48338186)
\curveto(994.94156492,42.25337852)(994.90156496,42.04337873)(994.84156494,41.85338186)
\curveto(994.53156533,40.91337986)(993.93656592,40.25338052)(993.05656494,39.87338186)
\curveto(992.9565669,39.82338095)(992.861567,39.78338099)(992.77156494,39.75338186)
\curveto(992.67156719,39.72338105)(992.56656729,39.68838108)(992.45656494,39.64838186)
\curveto(992.40656745,39.62838114)(992.3615675,39.61838115)(992.32156494,39.61838186)
\curveto(992.28156758,39.61838115)(992.23656762,39.60838116)(992.18656494,39.58838186)
\curveto(992.11656774,39.5683812)(992.04656781,39.55338122)(991.97656494,39.54338186)
\curveto(991.89656796,39.54338123)(991.82156804,39.53338124)(991.75156494,39.51338186)
\curveto(991.71156815,39.50338127)(991.67656818,39.49838127)(991.64656494,39.49838186)
\curveto(991.60656825,39.50838126)(991.56656829,39.50838126)(991.52656494,39.49838186)
\curveto(991.48656837,39.49838127)(991.44656841,39.49338128)(991.40656494,39.48338186)
\lineto(991.28656494,39.48338186)
\curveto(991.16656869,39.46338131)(991.04156882,39.46338131)(990.91156494,39.48338186)
\curveto(990.85156901,39.49338128)(990.79156907,39.49838127)(990.73156494,39.49838186)
\lineto(990.56656494,39.49838186)
\curveto(990.51656934,39.50838126)(990.47656938,39.51338126)(990.44656494,39.51338186)
\curveto(990.40656945,39.51338126)(990.3615695,39.51838125)(990.31156494,39.52838186)
\curveto(990.20156966,39.55838121)(990.09656976,39.57838119)(989.99656494,39.58838186)
\curveto(989.88656997,39.59838117)(989.77657008,39.62338115)(989.66656494,39.66338186)
\curveto(989.54657031,39.70338107)(989.43157043,39.73838103)(989.32156494,39.76838186)
\curveto(989.20157066,39.80838096)(989.08657077,39.85338092)(988.97656494,39.90338186)
\curveto(988.81657104,39.9733808)(988.67157119,40.05338072)(988.54156494,40.14338186)
\curveto(988.40157146,40.23338054)(988.26657159,40.32838044)(988.13656494,40.42838186)
\curveto(988.02657183,40.49838027)(987.93657192,40.58838018)(987.86656494,40.69838186)
\lineto(987.80656494,40.75838186)
\lineto(987.74656494,40.81838186)
\lineto(987.62656494,40.96838186)
\lineto(987.50656494,41.14838186)
\curveto(987.42657243,41.27837949)(987.3565725,41.41337936)(987.29656494,41.55338186)
\curveto(987.23657262,41.70337907)(987.18157268,41.86337891)(987.13156494,42.03338186)
\curveto(987.10157276,42.13337864)(987.08157278,42.23337854)(987.07156494,42.33338186)
\curveto(987.0615728,42.44337833)(987.04657281,42.55337822)(987.02656494,42.66338186)
\curveto(987.01657284,42.70337807)(987.01657284,42.75337802)(987.02656494,42.81338186)
\curveto(987.03657282,42.88337789)(987.03157283,42.93337784)(987.01156494,42.96338186)
\curveto(987.00157286,43.28337749)(987.03157283,43.5683772)(987.10156494,43.81838186)
\curveto(987.17157269,44.07837669)(987.27157259,44.30837646)(987.40156494,44.50838186)
\curveto(987.44157242,44.57837619)(987.48657237,44.64337613)(987.53656494,44.70338186)
\lineto(987.68656494,44.88338186)
\curveto(987.72657213,44.93337584)(987.77157209,44.97837579)(987.82156494,45.01838186)
\curveto(987.861572,45.0683757)(987.88157198,45.14337563)(987.88156494,45.24338186)
\lineto(987.83656494,45.28838186)
\curveto(987.81657204,45.30837546)(987.79157207,45.32837544)(987.76156494,45.34838186)
\curveto(987.68157218,45.37837539)(987.60157226,45.39337538)(987.52156494,45.39338186)
\curveto(987.44157242,45.40337537)(987.37157249,45.43337534)(987.31156494,45.48338186)
\curveto(987.27157259,45.51337526)(987.24157262,45.5733752)(987.22156494,45.66338186)
\curveto(987.19157267,45.75337502)(987.17657268,45.84837492)(987.17656494,45.94838186)
\curveto(987.17657268,46.04837472)(987.18657267,46.14337463)(987.20656494,46.23338186)
\curveto(987.22657263,46.33337444)(987.25157261,46.40337437)(987.28156494,46.44338186)
\moveto(991.06156494,45.31838186)
\curveto(991.02156884,45.32837544)(990.97156889,45.33337544)(990.91156494,45.33338186)
\curveto(990.84156902,45.33337544)(990.78656907,45.32837544)(990.74656494,45.31838186)
\lineto(990.50656494,45.31838186)
\curveto(990.41656944,45.29837547)(990.33156953,45.28337549)(990.25156494,45.27338186)
\curveto(990.1615697,45.26337551)(990.07656978,45.24837552)(989.99656494,45.22838186)
\curveto(989.91656994,45.20837556)(989.84157002,45.18837558)(989.77156494,45.16838186)
\curveto(989.69157017,45.15837561)(989.61657024,45.13837563)(989.54656494,45.10838186)
\curveto(989.26657059,44.99837577)(989.01657084,44.85337592)(988.79656494,44.67338186)
\curveto(988.57657128,44.50337627)(988.41157145,44.28337649)(988.30156494,44.01338186)
\curveto(988.2615716,43.93337684)(988.23157163,43.84837692)(988.21156494,43.75838186)
\curveto(988.18157168,43.6683771)(988.1565717,43.5733772)(988.13656494,43.47338186)
\curveto(988.11657174,43.39337738)(988.11157175,43.30337747)(988.12156494,43.20338186)
\lineto(988.12156494,42.93338186)
\curveto(988.13157173,42.88337789)(988.13657172,42.83337794)(988.13656494,42.78338186)
\curveto(988.13657172,42.74337803)(988.14157172,42.69837807)(988.15156494,42.64838186)
\curveto(988.20157166,42.45837831)(988.25157161,42.29837847)(988.30156494,42.16838186)
\curveto(988.44157142,41.82837894)(988.65157121,41.56337921)(988.93156494,41.37338186)
\curveto(989.21157065,41.18337959)(989.53657032,41.03337974)(989.90656494,40.92338186)
\curveto(989.98656987,40.90337987)(990.06656979,40.88837988)(990.14656494,40.87838186)
\curveto(990.21656964,40.87837989)(990.29156957,40.8683799)(990.37156494,40.84838186)
\curveto(990.40156946,40.82837994)(990.43656942,40.81837995)(990.47656494,40.81838186)
\curveto(990.51656934,40.82837994)(990.55156931,40.82837994)(990.58156494,40.81838186)
\lineto(990.91156494,40.81838186)
\lineto(991.25656494,40.81838186)
\curveto(991.36656849,40.81837995)(991.47156839,40.82837994)(991.57156494,40.84838186)
\lineto(991.64656494,40.84838186)
\curveto(991.67656818,40.85837991)(991.70156816,40.86337991)(991.72156494,40.86338186)
\curveto(991.81156805,40.88337989)(991.90156796,40.89837987)(991.99156494,40.90838186)
\curveto(992.08156778,40.92837984)(992.16656769,40.95337982)(992.24656494,40.98338186)
\curveto(992.50656735,41.06337971)(992.74656711,41.16337961)(992.96656494,41.28338186)
\curveto(993.18656667,41.40337937)(993.36656649,41.56337921)(993.50656494,41.76338186)
\lineto(993.59656494,41.88338186)
\curveto(993.61656624,41.92337885)(993.63656622,41.9683788)(993.65656494,42.01838186)
\curveto(993.70656615,42.09837867)(993.74656611,42.18337859)(993.77656494,42.27338186)
\curveto(993.80656605,42.36337841)(993.83656602,42.46337831)(993.86656494,42.57338186)
\curveto(993.87656598,42.62337815)(993.88156598,42.6683781)(993.88156494,42.70838186)
\curveto(993.87156599,42.75837801)(993.87656598,42.80837796)(993.89656494,42.85838186)
\curveto(993.90656595,42.88837788)(993.91156595,42.93837783)(993.91156494,43.00838186)
\curveto(993.91156595,43.07837769)(993.90656595,43.12837764)(993.89656494,43.15838186)
\curveto(993.88656597,43.18837758)(993.88656597,43.21837755)(993.89656494,43.24838186)
\curveto(993.89656596,43.28837748)(993.89156597,43.32837744)(993.88156494,43.36838186)
\curveto(993.861566,43.45837731)(993.84156602,43.54337723)(993.82156494,43.62338186)
\curveto(993.80156606,43.70337707)(993.77656608,43.78337699)(993.74656494,43.86338186)
\curveto(993.59656626,44.20337657)(993.38656647,44.4733763)(993.11656494,44.67338186)
\curveto(992.84656701,44.8733759)(992.53156733,45.03337574)(992.17156494,45.15338186)
\curveto(992.08156778,45.18337559)(991.99156787,45.20337557)(991.90156494,45.21338186)
\curveto(991.80156806,45.23337554)(991.70656815,45.25337552)(991.61656494,45.27338186)
\curveto(991.57656828,45.28337549)(991.54156832,45.28837548)(991.51156494,45.28838186)
\curveto(991.47156839,45.28837548)(991.43156843,45.29337548)(991.39156494,45.30338186)
\curveto(991.34156852,45.32337545)(991.29156857,45.32337545)(991.24156494,45.30338186)
\curveto(991.18156868,45.29337548)(991.12156874,45.29837547)(991.06156494,45.31838186)
}
}
{
\newrgbcolor{curcolor}{0 0 0}
\pscustom[linestyle=none,fillstyle=solid,fillcolor=curcolor]
{
\newpath
\moveto(990.70156494,55.56666311)
\curveto(990.7615691,55.58665505)(990.856569,55.59665504)(990.98656494,55.59666311)
\curveto(991.10656875,55.59665504)(991.19156867,55.59165504)(991.24156494,55.58166311)
\lineto(991.39156494,55.58166311)
\curveto(991.47156839,55.57165506)(991.54656831,55.56165507)(991.61656494,55.55166311)
\curveto(991.67656818,55.55165508)(991.74656811,55.54665509)(991.82656494,55.53666311)
\curveto(991.88656797,55.51665512)(991.94656791,55.50165513)(992.00656494,55.49166311)
\curveto(992.06656779,55.49165514)(992.12656773,55.48165515)(992.18656494,55.46166311)
\curveto(992.31656754,55.42165521)(992.44656741,55.38665525)(992.57656494,55.35666311)
\curveto(992.70656715,55.32665531)(992.82656703,55.28665535)(992.93656494,55.23666311)
\curveto(993.41656644,55.02665561)(993.82156604,54.74665589)(994.15156494,54.39666311)
\curveto(994.47156539,54.04665659)(994.71656514,53.61665702)(994.88656494,53.10666311)
\curveto(994.92656493,52.99665764)(994.9565649,52.87665776)(994.97656494,52.74666311)
\curveto(994.99656486,52.62665801)(995.01656484,52.50165813)(995.03656494,52.37166311)
\curveto(995.04656481,52.31165832)(995.05156481,52.24665839)(995.05156494,52.17666311)
\curveto(995.0615648,52.11665852)(995.06656479,52.05665858)(995.06656494,51.99666311)
\curveto(995.07656478,51.95665868)(995.08156478,51.89665874)(995.08156494,51.81666311)
\curveto(995.08156478,51.74665889)(995.07656478,51.69665894)(995.06656494,51.66666311)
\curveto(995.0565648,51.62665901)(995.05156481,51.58665905)(995.05156494,51.54666311)
\curveto(995.0615648,51.50665913)(995.0615648,51.47165916)(995.05156494,51.44166311)
\lineto(995.05156494,51.35166311)
\lineto(995.00656494,50.99166311)
\curveto(994.96656489,50.85165978)(994.92656493,50.71665992)(994.88656494,50.58666311)
\curveto(994.84656501,50.45666018)(994.80156506,50.3316603)(994.75156494,50.21166311)
\curveto(994.55156531,49.76166087)(994.29156557,49.39166124)(993.97156494,49.10166311)
\curveto(993.65156621,48.81166182)(993.2615666,48.57166206)(992.80156494,48.38166311)
\curveto(992.70156716,48.3316623)(992.60156726,48.29166234)(992.50156494,48.26166311)
\curveto(992.40156746,48.24166239)(992.29656756,48.22166241)(992.18656494,48.20166311)
\curveto(992.14656771,48.18166245)(992.11656774,48.17166246)(992.09656494,48.17166311)
\curveto(992.06656779,48.18166245)(992.03156783,48.18166245)(991.99156494,48.17166311)
\curveto(991.91156795,48.15166248)(991.83156803,48.1366625)(991.75156494,48.12666311)
\curveto(991.6615682,48.12666251)(991.57656828,48.11666252)(991.49656494,48.09666311)
\lineto(991.37656494,48.09666311)
\curveto(991.33656852,48.09666254)(991.29156857,48.09166254)(991.24156494,48.08166311)
\curveto(991.19156867,48.07166256)(991.10656875,48.06666257)(990.98656494,48.06666311)
\curveto(990.856569,48.06666257)(990.7615691,48.07666256)(990.70156494,48.09666311)
\curveto(990.63156923,48.11666252)(990.5615693,48.12166251)(990.49156494,48.11166311)
\curveto(990.42156944,48.10166253)(990.35156951,48.10666253)(990.28156494,48.12666311)
\curveto(990.23156963,48.1366625)(990.19156967,48.14166249)(990.16156494,48.14166311)
\curveto(990.12156974,48.15166248)(990.07656978,48.16166247)(990.02656494,48.17166311)
\curveto(989.90656995,48.20166243)(989.78657007,48.22666241)(989.66656494,48.24666311)
\curveto(989.54657031,48.27666236)(989.43157043,48.31666232)(989.32156494,48.36666311)
\curveto(988.95157091,48.51666212)(988.62157124,48.69666194)(988.33156494,48.90666311)
\curveto(988.03157183,49.12666151)(987.78157208,49.39166124)(987.58156494,49.70166311)
\curveto(987.50157236,49.82166081)(987.43657242,49.94666069)(987.38656494,50.07666311)
\curveto(987.32657253,50.20666043)(987.26657259,50.34166029)(987.20656494,50.48166311)
\curveto(987.1565727,50.60166003)(987.12657273,50.7316599)(987.11656494,50.87166311)
\curveto(987.09657276,51.01165962)(987.06657279,51.15165948)(987.02656494,51.29166311)
\lineto(987.02656494,51.48666311)
\curveto(987.01657284,51.55665908)(987.00657285,51.62165901)(986.99656494,51.68166311)
\curveto(986.98657287,52.57165806)(987.17157269,53.31165732)(987.55156494,53.90166311)
\curveto(987.93157193,54.49165614)(988.42657143,54.91665572)(989.03656494,55.17666311)
\curveto(989.13657072,55.22665541)(989.23657062,55.26665537)(989.33656494,55.29666311)
\curveto(989.43657042,55.32665531)(989.54157032,55.36165527)(989.65156494,55.40166311)
\curveto(989.7615701,55.4316552)(989.88156998,55.45665518)(990.01156494,55.47666311)
\curveto(990.13156973,55.49665514)(990.2565696,55.52165511)(990.38656494,55.55166311)
\curveto(990.43656942,55.56165507)(990.49156937,55.56165507)(990.55156494,55.55166311)
\curveto(990.60156926,55.55165508)(990.65156921,55.55665508)(990.70156494,55.56666311)
\moveto(991.55656494,54.23166311)
\curveto(991.48656837,54.25165638)(991.40656845,54.25665638)(991.31656494,54.24666311)
\lineto(991.06156494,54.24666311)
\curveto(990.67156919,54.24665639)(990.34156952,54.21165642)(990.07156494,54.14166311)
\curveto(989.99156987,54.11165652)(989.91156995,54.08665655)(989.83156494,54.06666311)
\curveto(989.75157011,54.04665659)(989.67657018,54.02165661)(989.60656494,53.99166311)
\curveto(988.9565709,53.71165692)(988.50657135,53.26665737)(988.25656494,52.65666311)
\curveto(988.22657163,52.58665805)(988.20657165,52.51165812)(988.19656494,52.43166311)
\lineto(988.13656494,52.19166311)
\curveto(988.11657174,52.11165852)(988.10657175,52.02665861)(988.10656494,51.93666311)
\lineto(988.10656494,51.66666311)
\lineto(988.15156494,51.39666311)
\curveto(988.17157169,51.29665934)(988.19657166,51.20165943)(988.22656494,51.11166311)
\curveto(988.24657161,51.0316596)(988.27657158,50.95165968)(988.31656494,50.87166311)
\curveto(988.33657152,50.80165983)(988.36657149,50.7366599)(988.40656494,50.67666311)
\curveto(988.44657141,50.61666002)(988.48657137,50.56166007)(988.52656494,50.51166311)
\curveto(988.69657116,50.27166036)(988.90157096,50.07666056)(989.14156494,49.92666311)
\curveto(989.38157048,49.77666086)(989.6615702,49.64666099)(989.98156494,49.53666311)
\curveto(990.08156978,49.50666113)(990.18656967,49.48666115)(990.29656494,49.47666311)
\curveto(990.39656946,49.46666117)(990.50156936,49.45166118)(990.61156494,49.43166311)
\curveto(990.65156921,49.42166121)(990.71656914,49.41666122)(990.80656494,49.41666311)
\curveto(990.83656902,49.40666123)(990.87156899,49.40166123)(990.91156494,49.40166311)
\curveto(990.95156891,49.41166122)(990.99656886,49.41666122)(991.04656494,49.41666311)
\lineto(991.34656494,49.41666311)
\curveto(991.44656841,49.41666122)(991.53656832,49.42666121)(991.61656494,49.44666311)
\lineto(991.79656494,49.47666311)
\curveto(991.89656796,49.49666114)(991.99656786,49.51166112)(992.09656494,49.52166311)
\curveto(992.18656767,49.54166109)(992.27156759,49.57166106)(992.35156494,49.61166311)
\curveto(992.59156727,49.71166092)(992.81656704,49.82666081)(993.02656494,49.95666311)
\curveto(993.23656662,50.09666054)(993.41156645,50.26666037)(993.55156494,50.46666311)
\curveto(993.58156628,50.51666012)(993.60656625,50.56166007)(993.62656494,50.60166311)
\curveto(993.64656621,50.64165999)(993.67156619,50.68665995)(993.70156494,50.73666311)
\curveto(993.75156611,50.81665982)(993.79656606,50.90165973)(993.83656494,50.99166311)
\curveto(993.86656599,51.09165954)(993.89656596,51.19665944)(993.92656494,51.30666311)
\curveto(993.94656591,51.35665928)(993.9565659,51.40165923)(993.95656494,51.44166311)
\curveto(993.94656591,51.49165914)(993.94656591,51.54165909)(993.95656494,51.59166311)
\curveto(993.96656589,51.62165901)(993.97656588,51.68165895)(993.98656494,51.77166311)
\curveto(993.99656586,51.87165876)(993.99156587,51.94665869)(993.97156494,51.99666311)
\curveto(993.9615659,52.0366586)(993.9615659,52.07665856)(993.97156494,52.11666311)
\curveto(993.97156589,52.15665848)(993.9615659,52.19665844)(993.94156494,52.23666311)
\curveto(993.92156594,52.31665832)(993.90656595,52.39665824)(993.89656494,52.47666311)
\curveto(993.87656598,52.55665808)(993.85156601,52.631658)(993.82156494,52.70166311)
\curveto(993.68156618,53.04165759)(993.48656637,53.31665732)(993.23656494,53.52666311)
\curveto(992.98656687,53.7366569)(992.69156717,53.91165672)(992.35156494,54.05166311)
\curveto(992.23156763,54.10165653)(992.10656775,54.1316565)(991.97656494,54.14166311)
\curveto(991.83656802,54.16165647)(991.69656816,54.19165644)(991.55656494,54.23166311)
}
}
{
\newrgbcolor{curcolor}{0 0 0}
\pscustom[linestyle=none,fillstyle=solid,fillcolor=curcolor]
{
}
}
{
\newrgbcolor{curcolor}{0 0 0}
\pscustom[linestyle=none,fillstyle=solid,fillcolor=curcolor]
{
\newpath
\moveto(989.81656494,67.99510061)
\lineto(990.07156494,67.99510061)
\curveto(990.15156971,68.0050929)(990.22656963,68.00009291)(990.29656494,67.98010061)
\lineto(990.53656494,67.98010061)
\lineto(990.70156494,67.98010061)
\curveto(990.80156906,67.96009295)(990.90656895,67.95009296)(991.01656494,67.95010061)
\curveto(991.11656874,67.95009296)(991.21656864,67.94009297)(991.31656494,67.92010061)
\lineto(991.46656494,67.92010061)
\curveto(991.60656825,67.89009302)(991.74656811,67.87009304)(991.88656494,67.86010061)
\curveto(992.01656784,67.85009306)(992.14656771,67.82509308)(992.27656494,67.78510061)
\curveto(992.3565675,67.76509314)(992.44156742,67.74509316)(992.53156494,67.72510061)
\lineto(992.77156494,67.66510061)
\lineto(993.07156494,67.54510061)
\curveto(993.1615667,67.51509339)(993.25156661,67.48009343)(993.34156494,67.44010061)
\curveto(993.5615663,67.34009357)(993.77656608,67.2050937)(993.98656494,67.03510061)
\curveto(994.19656566,66.87509403)(994.36656549,66.70009421)(994.49656494,66.51010061)
\curveto(994.53656532,66.46009445)(994.57656528,66.40009451)(994.61656494,66.33010061)
\curveto(994.64656521,66.27009464)(994.68156518,66.2100947)(994.72156494,66.15010061)
\curveto(994.77156509,66.07009484)(994.81156505,65.97509493)(994.84156494,65.86510061)
\curveto(994.87156499,65.75509515)(994.90156496,65.65009526)(994.93156494,65.55010061)
\curveto(994.97156489,65.44009547)(994.99656486,65.33009558)(995.00656494,65.22010061)
\curveto(995.01656484,65.1100958)(995.03156483,64.99509591)(995.05156494,64.87510061)
\curveto(995.0615648,64.83509607)(995.0615648,64.79009612)(995.05156494,64.74010061)
\curveto(995.05156481,64.70009621)(995.0565648,64.66009625)(995.06656494,64.62010061)
\curveto(995.07656478,64.58009633)(995.08156478,64.52509638)(995.08156494,64.45510061)
\curveto(995.08156478,64.38509652)(995.07656478,64.33509657)(995.06656494,64.30510061)
\curveto(995.04656481,64.25509665)(995.04156482,64.2100967)(995.05156494,64.17010061)
\curveto(995.0615648,64.13009678)(995.0615648,64.09509681)(995.05156494,64.06510061)
\lineto(995.05156494,63.97510061)
\curveto(995.03156483,63.91509699)(995.01656484,63.85009706)(995.00656494,63.78010061)
\curveto(995.00656485,63.72009719)(995.00156486,63.65509725)(994.99156494,63.58510061)
\curveto(994.94156492,63.41509749)(994.89156497,63.25509765)(994.84156494,63.10510061)
\curveto(994.79156507,62.95509795)(994.72656513,62.8100981)(994.64656494,62.67010061)
\curveto(994.60656525,62.62009829)(994.57656528,62.56509834)(994.55656494,62.50510061)
\curveto(994.52656533,62.45509845)(994.49156537,62.4050985)(994.45156494,62.35510061)
\curveto(994.27156559,62.11509879)(994.05156581,61.91509899)(993.79156494,61.75510061)
\curveto(993.53156633,61.59509931)(993.24656661,61.45509945)(992.93656494,61.33510061)
\curveto(992.79656706,61.27509963)(992.6565672,61.23009968)(992.51656494,61.20010061)
\curveto(992.36656749,61.17009974)(992.21156765,61.13509977)(992.05156494,61.09510061)
\curveto(991.94156792,61.07509983)(991.83156803,61.06009985)(991.72156494,61.05010061)
\curveto(991.61156825,61.04009987)(991.50156836,61.02509988)(991.39156494,61.00510061)
\curveto(991.35156851,60.99509991)(991.31156855,60.99009992)(991.27156494,60.99010061)
\curveto(991.23156863,61.00009991)(991.19156867,61.00009991)(991.15156494,60.99010061)
\curveto(991.10156876,60.98009993)(991.05156881,60.97509993)(991.00156494,60.97510061)
\lineto(990.83656494,60.97510061)
\curveto(990.78656907,60.95509995)(990.73656912,60.95009996)(990.68656494,60.96010061)
\curveto(990.62656923,60.97009994)(990.57156929,60.97009994)(990.52156494,60.96010061)
\curveto(990.48156938,60.95009996)(990.43656942,60.95009996)(990.38656494,60.96010061)
\curveto(990.33656952,60.97009994)(990.28656957,60.96509994)(990.23656494,60.94510061)
\curveto(990.16656969,60.92509998)(990.09156977,60.92009999)(990.01156494,60.93010061)
\curveto(989.92156994,60.94009997)(989.83657002,60.94509996)(989.75656494,60.94510061)
\curveto(989.66657019,60.94509996)(989.56657029,60.94009997)(989.45656494,60.93010061)
\curveto(989.33657052,60.92009999)(989.23657062,60.92509998)(989.15656494,60.94510061)
\lineto(988.87156494,60.94510061)
\lineto(988.24156494,60.99010061)
\curveto(988.14157172,61.00009991)(988.04657181,61.0100999)(987.95656494,61.02010061)
\lineto(987.65656494,61.05010061)
\curveto(987.60657225,61.07009984)(987.5565723,61.07509983)(987.50656494,61.06510061)
\curveto(987.44657241,61.06509984)(987.39157247,61.07509983)(987.34156494,61.09510061)
\curveto(987.17157269,61.14509976)(987.00657285,61.18509972)(986.84656494,61.21510061)
\curveto(986.67657318,61.24509966)(986.51657334,61.29509961)(986.36656494,61.36510061)
\curveto(985.90657395,61.55509935)(985.53157433,61.77509913)(985.24156494,62.02510061)
\curveto(984.95157491,62.28509862)(984.70657515,62.64509826)(984.50656494,63.10510061)
\curveto(984.4565754,63.23509767)(984.42157544,63.36509754)(984.40156494,63.49510061)
\curveto(984.38157548,63.63509727)(984.3565755,63.77509713)(984.32656494,63.91510061)
\curveto(984.31657554,63.98509692)(984.31157555,64.05009686)(984.31156494,64.11010061)
\curveto(984.31157555,64.17009674)(984.30657555,64.23509667)(984.29656494,64.30510061)
\curveto(984.27657558,65.13509577)(984.42657543,65.8050951)(984.74656494,66.31510061)
\curveto(985.0565748,66.82509408)(985.49657436,67.2050937)(986.06656494,67.45510061)
\curveto(986.18657367,67.5050934)(986.31157355,67.55009336)(986.44156494,67.59010061)
\curveto(986.57157329,67.63009328)(986.70657315,67.67509323)(986.84656494,67.72510061)
\curveto(986.92657293,67.74509316)(987.01157285,67.76009315)(987.10156494,67.77010061)
\lineto(987.34156494,67.83010061)
\curveto(987.45157241,67.86009305)(987.5615723,67.87509303)(987.67156494,67.87510061)
\curveto(987.78157208,67.88509302)(987.89157197,67.90009301)(988.00156494,67.92010061)
\curveto(988.05157181,67.94009297)(988.09657176,67.94509296)(988.13656494,67.93510061)
\curveto(988.17657168,67.93509297)(988.21657164,67.94009297)(988.25656494,67.95010061)
\curveto(988.30657155,67.96009295)(988.3615715,67.96009295)(988.42156494,67.95010061)
\curveto(988.47157139,67.95009296)(988.52157134,67.95509295)(988.57156494,67.96510061)
\lineto(988.70656494,67.96510061)
\curveto(988.76657109,67.98509292)(988.83657102,67.98509292)(988.91656494,67.96510061)
\curveto(988.98657087,67.95509295)(989.05157081,67.96009295)(989.11156494,67.98010061)
\curveto(989.14157072,67.99009292)(989.18157068,67.99509291)(989.23156494,67.99510061)
\lineto(989.35156494,67.99510061)
\lineto(989.81656494,67.99510061)
\moveto(992.14156494,66.45010061)
\curveto(991.82156804,66.55009436)(991.4565684,66.6100943)(991.04656494,66.63010061)
\curveto(990.63656922,66.65009426)(990.22656963,66.66009425)(989.81656494,66.66010061)
\curveto(989.38657047,66.66009425)(988.96657089,66.65009426)(988.55656494,66.63010061)
\curveto(988.14657171,66.6100943)(987.7615721,66.56509434)(987.40156494,66.49510061)
\curveto(987.04157282,66.42509448)(986.72157314,66.31509459)(986.44156494,66.16510061)
\curveto(986.15157371,66.02509488)(985.91657394,65.83009508)(985.73656494,65.58010061)
\curveto(985.62657423,65.42009549)(985.54657431,65.24009567)(985.49656494,65.04010061)
\curveto(985.43657442,64.84009607)(985.40657445,64.59509631)(985.40656494,64.30510061)
\curveto(985.42657443,64.28509662)(985.43657442,64.25009666)(985.43656494,64.20010061)
\curveto(985.42657443,64.15009676)(985.42657443,64.1100968)(985.43656494,64.08010061)
\curveto(985.4565744,64.00009691)(985.47657438,63.92509698)(985.49656494,63.85510061)
\curveto(985.50657435,63.79509711)(985.52657433,63.73009718)(985.55656494,63.66010061)
\curveto(985.67657418,63.39009752)(985.84657401,63.17009774)(986.06656494,63.00010061)
\curveto(986.27657358,62.84009807)(986.52157334,62.7050982)(986.80156494,62.59510061)
\curveto(986.91157295,62.54509836)(987.03157283,62.5050984)(987.16156494,62.47510061)
\curveto(987.28157258,62.45509845)(987.40657245,62.43009848)(987.53656494,62.40010061)
\curveto(987.58657227,62.38009853)(987.64157222,62.37009854)(987.70156494,62.37010061)
\curveto(987.75157211,62.37009854)(987.80157206,62.36509854)(987.85156494,62.35510061)
\curveto(987.94157192,62.34509856)(988.03657182,62.33509857)(988.13656494,62.32510061)
\curveto(988.22657163,62.31509859)(988.32157154,62.3050986)(988.42156494,62.29510061)
\curveto(988.50157136,62.29509861)(988.58657127,62.29009862)(988.67656494,62.28010061)
\lineto(988.91656494,62.28010061)
\lineto(989.09656494,62.28010061)
\curveto(989.12657073,62.27009864)(989.1615707,62.26509864)(989.20156494,62.26510061)
\lineto(989.33656494,62.26510061)
\lineto(989.78656494,62.26510061)
\curveto(989.86656999,62.26509864)(989.95156991,62.26009865)(990.04156494,62.25010061)
\curveto(990.12156974,62.25009866)(990.19656966,62.26009865)(990.26656494,62.28010061)
\lineto(990.53656494,62.28010061)
\curveto(990.5565693,62.28009863)(990.58656927,62.27509863)(990.62656494,62.26510061)
\curveto(990.6565692,62.26509864)(990.68156918,62.27009864)(990.70156494,62.28010061)
\curveto(990.80156906,62.29009862)(990.90156896,62.29509861)(991.00156494,62.29510061)
\curveto(991.09156877,62.3050986)(991.19156867,62.31509859)(991.30156494,62.32510061)
\curveto(991.42156844,62.35509855)(991.54656831,62.37009854)(991.67656494,62.37010061)
\curveto(991.79656806,62.38009853)(991.91156795,62.4050985)(992.02156494,62.44510061)
\curveto(992.32156754,62.52509838)(992.58656727,62.6100983)(992.81656494,62.70010061)
\curveto(993.04656681,62.80009811)(993.2615666,62.94509796)(993.46156494,63.13510061)
\curveto(993.6615662,63.34509756)(993.81156605,63.6100973)(993.91156494,63.93010061)
\curveto(993.93156593,63.97009694)(993.94156592,64.0050969)(993.94156494,64.03510061)
\curveto(993.93156593,64.07509683)(993.93656592,64.12009679)(993.95656494,64.17010061)
\curveto(993.96656589,64.2100967)(993.97656588,64.28009663)(993.98656494,64.38010061)
\curveto(993.99656586,64.49009642)(993.99156587,64.57509633)(993.97156494,64.63510061)
\curveto(993.95156591,64.7050962)(993.94156592,64.77509613)(993.94156494,64.84510061)
\curveto(993.93156593,64.91509599)(993.91656594,64.98009593)(993.89656494,65.04010061)
\curveto(993.83656602,65.24009567)(993.75156611,65.42009549)(993.64156494,65.58010061)
\curveto(993.62156624,65.6100953)(993.60156626,65.63509527)(993.58156494,65.65510061)
\lineto(993.52156494,65.71510061)
\curveto(993.50156636,65.75509515)(993.4615664,65.8050951)(993.40156494,65.86510061)
\curveto(993.2615666,65.96509494)(993.13156673,66.05009486)(993.01156494,66.12010061)
\curveto(992.89156697,66.19009472)(992.74656711,66.26009465)(992.57656494,66.33010061)
\curveto(992.50656735,66.36009455)(992.43656742,66.38009453)(992.36656494,66.39010061)
\curveto(992.29656756,66.4100945)(992.22156764,66.43009448)(992.14156494,66.45010061)
}
}
{
\newrgbcolor{curcolor}{0 0 0}
\pscustom[linestyle=none,fillstyle=solid,fillcolor=curcolor]
{
\newpath
\moveto(984.49156494,70.85470998)
\lineto(984.49156494,74.45470998)
\lineto(984.49156494,75.09970998)
\curveto(984.49157537,75.17970345)(984.49657536,75.25470338)(984.50656494,75.32470998)
\curveto(984.50657535,75.39470324)(984.51657534,75.45470318)(984.53656494,75.50470998)
\curveto(984.56657529,75.57470306)(984.62657523,75.629703)(984.71656494,75.66970998)
\curveto(984.74657511,75.68970294)(984.78657507,75.69970293)(984.83656494,75.69970998)
\lineto(984.97156494,75.69970998)
\curveto(985.08157478,75.70970292)(985.18657467,75.70470293)(985.28656494,75.68470998)
\curveto(985.38657447,75.67470296)(985.4565744,75.63970299)(985.49656494,75.57970998)
\curveto(985.56657429,75.48970314)(985.60157426,75.35470328)(985.60156494,75.17470998)
\curveto(985.59157427,74.99470364)(985.58657427,74.8297038)(985.58656494,74.67970998)
\lineto(985.58656494,72.68470998)
\lineto(985.58656494,72.18970998)
\lineto(985.58656494,72.05470998)
\curveto(985.58657427,72.01470662)(985.59157427,71.97470666)(985.60156494,71.93470998)
\lineto(985.60156494,71.72470998)
\curveto(985.63157423,71.61470702)(985.67157419,71.5347071)(985.72156494,71.48470998)
\curveto(985.7615741,71.4347072)(985.81657404,71.39970723)(985.88656494,71.37970998)
\curveto(985.94657391,71.35970727)(986.01657384,71.34470729)(986.09656494,71.33470998)
\curveto(986.17657368,71.32470731)(986.26657359,71.30470733)(986.36656494,71.27470998)
\curveto(986.56657329,71.22470741)(986.77157309,71.18470745)(986.98156494,71.15470998)
\curveto(987.19157267,71.12470751)(987.39657246,71.08470755)(987.59656494,71.03470998)
\curveto(987.66657219,71.01470762)(987.73657212,71.00470763)(987.80656494,71.00470998)
\curveto(987.86657199,71.00470763)(987.93157193,70.99470764)(988.00156494,70.97470998)
\curveto(988.03157183,70.96470767)(988.07157179,70.95470768)(988.12156494,70.94470998)
\curveto(988.1615717,70.94470769)(988.20157166,70.94970768)(988.24156494,70.95970998)
\curveto(988.29157157,70.97970765)(988.33657152,71.00470763)(988.37656494,71.03470998)
\curveto(988.40657145,71.07470756)(988.41157145,71.1347075)(988.39156494,71.21470998)
\curveto(988.37157149,71.27470736)(988.34657151,71.3347073)(988.31656494,71.39470998)
\curveto(988.27657158,71.45470718)(988.24157162,71.51470712)(988.21156494,71.57470998)
\curveto(988.19157167,71.634707)(988.17657168,71.68470695)(988.16656494,71.72470998)
\curveto(988.08657177,71.91470672)(988.03157183,72.11970651)(988.00156494,72.33970998)
\curveto(987.97157189,72.56970606)(987.9615719,72.79970583)(987.97156494,73.02970998)
\curveto(987.97157189,73.26970536)(987.99657186,73.49970513)(988.04656494,73.71970998)
\curveto(988.08657177,73.93970469)(988.14657171,74.13970449)(988.22656494,74.31970998)
\curveto(988.24657161,74.36970426)(988.26657159,74.41470422)(988.28656494,74.45470998)
\curveto(988.30657155,74.50470413)(988.33157153,74.55470408)(988.36156494,74.60470998)
\curveto(988.57157129,74.95470368)(988.80157106,75.2347034)(989.05156494,75.44470998)
\curveto(989.30157056,75.66470297)(989.62657023,75.85970277)(990.02656494,76.02970998)
\curveto(990.13656972,76.07970255)(990.24656961,76.11470252)(990.35656494,76.13470998)
\curveto(990.46656939,76.15470248)(990.58156928,76.17970245)(990.70156494,76.20970998)
\curveto(990.73156913,76.21970241)(990.77656908,76.22470241)(990.83656494,76.22470998)
\curveto(990.89656896,76.24470239)(990.96656889,76.25470238)(991.04656494,76.25470998)
\curveto(991.11656874,76.25470238)(991.18156868,76.26470237)(991.24156494,76.28470998)
\lineto(991.40656494,76.28470998)
\curveto(991.4565684,76.29470234)(991.52656833,76.29970233)(991.61656494,76.29970998)
\curveto(991.70656815,76.29970233)(991.77656808,76.28970234)(991.82656494,76.26970998)
\curveto(991.88656797,76.24970238)(991.94656791,76.24470239)(992.00656494,76.25470998)
\curveto(992.0565678,76.26470237)(992.10656775,76.25970237)(992.15656494,76.23970998)
\curveto(992.31656754,76.19970243)(992.46656739,76.16470247)(992.60656494,76.13470998)
\curveto(992.74656711,76.10470253)(992.88156698,76.05970257)(993.01156494,75.99970998)
\curveto(993.38156648,75.83970279)(993.71656614,75.61970301)(994.01656494,75.33970998)
\curveto(994.31656554,75.05970357)(994.54656531,74.73970389)(994.70656494,74.37970998)
\curveto(994.78656507,74.20970442)(994.861565,74.00970462)(994.93156494,73.77970998)
\curveto(994.97156489,73.66970496)(994.99656486,73.55470508)(995.00656494,73.43470998)
\curveto(995.01656484,73.31470532)(995.03656482,73.19470544)(995.06656494,73.07470998)
\curveto(995.08656477,73.02470561)(995.08656477,72.96970566)(995.06656494,72.90970998)
\curveto(995.0565648,72.84970578)(995.0615648,72.78970584)(995.08156494,72.72970998)
\curveto(995.10156476,72.629706)(995.10156476,72.5297061)(995.08156494,72.42970998)
\lineto(995.08156494,72.29470998)
\curveto(995.0615648,72.24470639)(995.05156481,72.18470645)(995.05156494,72.11470998)
\curveto(995.0615648,72.05470658)(995.0565648,71.99970663)(995.03656494,71.94970998)
\curveto(995.02656483,71.90970672)(995.02156484,71.87470676)(995.02156494,71.84470998)
\curveto(995.02156484,71.81470682)(995.01656484,71.77970685)(995.00656494,71.73970998)
\lineto(994.94656494,71.46970998)
\curveto(994.92656493,71.37970725)(994.89656496,71.29470734)(994.85656494,71.21470998)
\curveto(994.71656514,70.87470776)(994.5615653,70.58470805)(994.39156494,70.34470998)
\curveto(994.21156565,70.10470853)(993.98156588,69.88470875)(993.70156494,69.68470998)
\curveto(993.47156639,69.5347091)(993.23156663,69.41970921)(992.98156494,69.33970998)
\curveto(992.93156693,69.31970931)(992.88656697,69.30970932)(992.84656494,69.30970998)
\curveto(992.79656706,69.30970932)(992.74656711,69.29970933)(992.69656494,69.27970998)
\curveto(992.63656722,69.25970937)(992.5565673,69.24470939)(992.45656494,69.23470998)
\curveto(992.3565675,69.2347094)(992.28156758,69.25470938)(992.23156494,69.29470998)
\curveto(992.15156771,69.34470929)(992.10656775,69.42470921)(992.09656494,69.53470998)
\curveto(992.08656777,69.64470899)(992.08156778,69.75970887)(992.08156494,69.87970998)
\lineto(992.08156494,70.04470998)
\curveto(992.08156778,70.10470853)(992.09156777,70.15970847)(992.11156494,70.20970998)
\curveto(992.13156773,70.29970833)(992.17156769,70.36970826)(992.23156494,70.41970998)
\curveto(992.32156754,70.48970814)(992.43156743,70.5347081)(992.56156494,70.55470998)
\curveto(992.68156718,70.58470805)(992.78656707,70.629708)(992.87656494,70.68970998)
\curveto(993.21656664,70.87970775)(993.48656637,71.13970749)(993.68656494,71.46970998)
\curveto(993.74656611,71.56970706)(993.79656606,71.67470696)(993.83656494,71.78470998)
\curveto(993.86656599,71.90470673)(993.90156596,72.02470661)(993.94156494,72.14470998)
\curveto(993.99156587,72.31470632)(994.01156585,72.51970611)(994.00156494,72.75970998)
\curveto(993.98156588,73.00970562)(993.94656591,73.20970542)(993.89656494,73.35970998)
\curveto(993.77656608,73.7297049)(993.61656624,74.01970461)(993.41656494,74.22970998)
\curveto(993.20656665,74.44970418)(992.92656693,74.629704)(992.57656494,74.76970998)
\curveto(992.47656738,74.81970381)(992.37156749,74.84970378)(992.26156494,74.85970998)
\curveto(992.15156771,74.87970375)(992.03656782,74.90470373)(991.91656494,74.93470998)
\lineto(991.81156494,74.93470998)
\curveto(991.77156809,74.94470369)(991.73156813,74.94970368)(991.69156494,74.94970998)
\curveto(991.6615682,74.95970367)(991.62656823,74.95970367)(991.58656494,74.94970998)
\lineto(991.46656494,74.94970998)
\curveto(991.20656865,74.94970368)(990.9615689,74.91970371)(990.73156494,74.85970998)
\curveto(990.38156948,74.74970388)(990.08656977,74.59470404)(989.84656494,74.39470998)
\curveto(989.59657026,74.19470444)(989.40157046,73.9347047)(989.26156494,73.61470998)
\lineto(989.20156494,73.43470998)
\curveto(989.18157068,73.38470525)(989.1615707,73.32470531)(989.14156494,73.25470998)
\curveto(989.12157074,73.20470543)(989.11157075,73.14470549)(989.11156494,73.07470998)
\curveto(989.10157076,73.01470562)(989.08657077,72.94970568)(989.06656494,72.87970998)
\lineto(989.06656494,72.72970998)
\curveto(989.04657081,72.68970594)(989.03657082,72.634706)(989.03656494,72.56470998)
\curveto(989.03657082,72.50470613)(989.04657081,72.44970618)(989.06656494,72.39970998)
\lineto(989.06656494,72.29470998)
\curveto(989.06657079,72.26470637)(989.07157079,72.2297064)(989.08156494,72.18970998)
\lineto(989.14156494,71.94970998)
\curveto(989.15157071,71.86970676)(989.17157069,71.78970684)(989.20156494,71.70970998)
\curveto(989.30157056,71.46970716)(989.43657042,71.23970739)(989.60656494,71.01970998)
\curveto(989.67657018,70.9297077)(989.75157011,70.84470779)(989.83156494,70.76470998)
\curveto(989.90156996,70.68470795)(989.9565699,70.58470805)(989.99656494,70.46470998)
\curveto(990.02656983,70.37470826)(990.03656982,70.2347084)(990.02656494,70.04470998)
\curveto(990.01656984,69.86470877)(989.99156987,69.74470889)(989.95156494,69.68470998)
\curveto(989.91156995,69.634709)(989.85157001,69.59470904)(989.77156494,69.56470998)
\curveto(989.69157017,69.54470909)(989.60657025,69.54470909)(989.51656494,69.56470998)
\curveto(989.39657046,69.59470904)(989.27657058,69.61470902)(989.15656494,69.62470998)
\curveto(989.02657083,69.64470899)(988.90157096,69.66970896)(988.78156494,69.69970998)
\curveto(988.74157112,69.71970891)(988.70657115,69.72470891)(988.67656494,69.71470998)
\curveto(988.63657122,69.71470892)(988.59157127,69.72470891)(988.54156494,69.74470998)
\curveto(988.45157141,69.76470887)(988.3615715,69.77970885)(988.27156494,69.78970998)
\curveto(988.17157169,69.79970883)(988.07657178,69.81970881)(987.98656494,69.84970998)
\curveto(987.92657193,69.85970877)(987.86657199,69.86470877)(987.80656494,69.86470998)
\curveto(987.74657211,69.87470876)(987.68657217,69.88970874)(987.62656494,69.90970998)
\curveto(987.42657243,69.95970867)(987.22157264,69.99470864)(987.01156494,70.01470998)
\curveto(986.79157307,70.04470859)(986.58157328,70.08470855)(986.38156494,70.13470998)
\curveto(986.28157358,70.16470847)(986.18157368,70.18470845)(986.08156494,70.19470998)
\curveto(985.98157388,70.20470843)(985.88157398,70.21970841)(985.78156494,70.23970998)
\curveto(985.75157411,70.24970838)(985.71157415,70.25470838)(985.66156494,70.25470998)
\curveto(985.55157431,70.28470835)(985.44657441,70.30470833)(985.34656494,70.31470998)
\curveto(985.23657462,70.3347083)(985.12657473,70.35970827)(985.01656494,70.38970998)
\curveto(984.93657492,70.40970822)(984.86657499,70.42470821)(984.80656494,70.43470998)
\curveto(984.73657512,70.44470819)(984.67657518,70.46970816)(984.62656494,70.50970998)
\curveto(984.59657526,70.5297081)(984.57657528,70.55970807)(984.56656494,70.59970998)
\curveto(984.54657531,70.63970799)(984.52657533,70.68470795)(984.50656494,70.73470998)
\curveto(984.50657535,70.79470784)(984.50157536,70.8347078)(984.49156494,70.85470998)
}
}
{
\newrgbcolor{curcolor}{0 0 0}
\pscustom[linestyle=none,fillstyle=solid,fillcolor=curcolor]
{
\newpath
\moveto(993.26656494,78.63431936)
\lineto(993.26656494,79.26431936)
\lineto(993.26656494,79.45931936)
\curveto(993.26656659,79.52931683)(993.27656658,79.58931677)(993.29656494,79.63931936)
\curveto(993.33656652,79.70931665)(993.37656648,79.7593166)(993.41656494,79.78931936)
\curveto(993.46656639,79.82931653)(993.53156633,79.84931651)(993.61156494,79.84931936)
\curveto(993.69156617,79.8593165)(993.77656608,79.86431649)(993.86656494,79.86431936)
\lineto(994.58656494,79.86431936)
\curveto(995.06656479,79.86431649)(995.47656438,79.80431655)(995.81656494,79.68431936)
\curveto(996.1565637,79.56431679)(996.43156343,79.36931699)(996.64156494,79.09931936)
\curveto(996.69156317,79.02931733)(996.73656312,78.9593174)(996.77656494,78.88931936)
\curveto(996.82656303,78.82931753)(996.87156299,78.7543176)(996.91156494,78.66431936)
\curveto(996.92156294,78.64431771)(996.93156293,78.61431774)(996.94156494,78.57431936)
\curveto(996.9615629,78.53431782)(996.96656289,78.48931787)(996.95656494,78.43931936)
\curveto(996.92656293,78.34931801)(996.85156301,78.29431806)(996.73156494,78.27431936)
\curveto(996.62156324,78.2543181)(996.52656333,78.26931809)(996.44656494,78.31931936)
\curveto(996.37656348,78.34931801)(996.31156355,78.39431796)(996.25156494,78.45431936)
\curveto(996.20156366,78.52431783)(996.15156371,78.58931777)(996.10156494,78.64931936)
\curveto(996.05156381,78.71931764)(995.97656388,78.77931758)(995.87656494,78.82931936)
\curveto(995.78656407,78.88931747)(995.69656416,78.93931742)(995.60656494,78.97931936)
\curveto(995.57656428,78.99931736)(995.51656434,79.02431733)(995.42656494,79.05431936)
\curveto(995.34656451,79.08431727)(995.27656458,79.08931727)(995.21656494,79.06931936)
\curveto(995.07656478,79.03931732)(994.98656487,78.97931738)(994.94656494,78.88931936)
\curveto(994.91656494,78.80931755)(994.90156496,78.71931764)(994.90156494,78.61931936)
\curveto(994.90156496,78.51931784)(994.87656498,78.43431792)(994.82656494,78.36431936)
\curveto(994.7565651,78.27431808)(994.61656524,78.22931813)(994.40656494,78.22931936)
\lineto(993.85156494,78.22931936)
\lineto(993.62656494,78.22931936)
\curveto(993.54656631,78.23931812)(993.48156638,78.2593181)(993.43156494,78.28931936)
\curveto(993.35156651,78.34931801)(993.30656655,78.41931794)(993.29656494,78.49931936)
\curveto(993.28656657,78.51931784)(993.28156658,78.53931782)(993.28156494,78.55931936)
\curveto(993.28156658,78.58931777)(993.27656658,78.61431774)(993.26656494,78.63431936)
}
}
{
\newrgbcolor{curcolor}{0 0 0}
\pscustom[linestyle=none,fillstyle=solid,fillcolor=curcolor]
{
}
}
{
\newrgbcolor{curcolor}{0 0 0}
\pscustom[linestyle=none,fillstyle=solid,fillcolor=curcolor]
{
\newpath
\moveto(984.29656494,89.26463186)
\curveto(984.28657557,89.95462722)(984.40657545,90.55462662)(984.65656494,91.06463186)
\curveto(984.90657495,91.58462559)(985.24157462,91.9796252)(985.66156494,92.24963186)
\curveto(985.74157412,92.29962488)(985.83157403,92.34462483)(985.93156494,92.38463186)
\curveto(986.02157384,92.42462475)(986.11657374,92.46962471)(986.21656494,92.51963186)
\curveto(986.31657354,92.55962462)(986.41657344,92.58962459)(986.51656494,92.60963186)
\curveto(986.61657324,92.62962455)(986.72157314,92.64962453)(986.83156494,92.66963186)
\curveto(986.88157298,92.68962449)(986.92657293,92.69462448)(986.96656494,92.68463186)
\curveto(987.00657285,92.6746245)(987.05157281,92.6796245)(987.10156494,92.69963186)
\curveto(987.15157271,92.70962447)(987.23657262,92.71462446)(987.35656494,92.71463186)
\curveto(987.46657239,92.71462446)(987.55157231,92.70962447)(987.61156494,92.69963186)
\curveto(987.67157219,92.6796245)(987.73157213,92.66962451)(987.79156494,92.66963186)
\curveto(987.85157201,92.6796245)(987.91157195,92.6746245)(987.97156494,92.65463186)
\curveto(988.11157175,92.61462456)(988.24657161,92.5796246)(988.37656494,92.54963186)
\curveto(988.50657135,92.51962466)(988.63157123,92.4796247)(988.75156494,92.42963186)
\curveto(988.89157097,92.36962481)(989.01657084,92.29962488)(989.12656494,92.21963186)
\curveto(989.23657062,92.14962503)(989.34657051,92.0746251)(989.45656494,91.99463186)
\lineto(989.51656494,91.93463186)
\curveto(989.53657032,91.92462525)(989.5565703,91.90962527)(989.57656494,91.88963186)
\curveto(989.73657012,91.76962541)(989.88156998,91.63462554)(990.01156494,91.48463186)
\curveto(990.14156972,91.33462584)(990.26656959,91.174626)(990.38656494,91.00463186)
\curveto(990.60656925,90.69462648)(990.81156905,90.39962678)(991.00156494,90.11963186)
\curveto(991.14156872,89.88962729)(991.27656858,89.65962752)(991.40656494,89.42963186)
\curveto(991.53656832,89.20962797)(991.67156819,88.98962819)(991.81156494,88.76963186)
\curveto(991.98156788,88.51962866)(992.1615677,88.2796289)(992.35156494,88.04963186)
\curveto(992.54156732,87.82962935)(992.76656709,87.63962954)(993.02656494,87.47963186)
\curveto(993.08656677,87.43962974)(993.14656671,87.40462977)(993.20656494,87.37463186)
\curveto(993.2565666,87.34462983)(993.32156654,87.31462986)(993.40156494,87.28463186)
\curveto(993.47156639,87.26462991)(993.53156633,87.25962992)(993.58156494,87.26963186)
\curveto(993.65156621,87.28962989)(993.70656615,87.32462985)(993.74656494,87.37463186)
\curveto(993.77656608,87.42462975)(993.79656606,87.48462969)(993.80656494,87.55463186)
\lineto(993.80656494,87.79463186)
\lineto(993.80656494,88.54463186)
\lineto(993.80656494,91.34963186)
\lineto(993.80656494,92.00963186)
\curveto(993.80656605,92.09962508)(993.81156605,92.18462499)(993.82156494,92.26463186)
\curveto(993.82156604,92.34462483)(993.84156602,92.40962477)(993.88156494,92.45963186)
\curveto(993.92156594,92.50962467)(993.99656586,92.54962463)(994.10656494,92.57963186)
\curveto(994.20656565,92.61962456)(994.30656555,92.62962455)(994.40656494,92.60963186)
\lineto(994.54156494,92.60963186)
\curveto(994.61156525,92.58962459)(994.67156519,92.56962461)(994.72156494,92.54963186)
\curveto(994.77156509,92.52962465)(994.81156505,92.49462468)(994.84156494,92.44463186)
\curveto(994.88156498,92.39462478)(994.90156496,92.32462485)(994.90156494,92.23463186)
\lineto(994.90156494,91.96463186)
\lineto(994.90156494,91.06463186)
\lineto(994.90156494,87.55463186)
\lineto(994.90156494,86.48963186)
\curveto(994.90156496,86.40963077)(994.90656495,86.31963086)(994.91656494,86.21963186)
\curveto(994.91656494,86.11963106)(994.90656495,86.03463114)(994.88656494,85.96463186)
\curveto(994.81656504,85.75463142)(994.63656522,85.68963149)(994.34656494,85.76963186)
\curveto(994.30656555,85.7796314)(994.27156559,85.7796314)(994.24156494,85.76963186)
\curveto(994.20156566,85.76963141)(994.1565657,85.7796314)(994.10656494,85.79963186)
\curveto(994.02656583,85.81963136)(993.94156592,85.83963134)(993.85156494,85.85963186)
\curveto(993.7615661,85.8796313)(993.67656618,85.90463127)(993.59656494,85.93463186)
\curveto(993.10656675,86.09463108)(992.69156717,86.29463088)(992.35156494,86.53463186)
\curveto(992.10156776,86.71463046)(991.87656798,86.91963026)(991.67656494,87.14963186)
\curveto(991.46656839,87.3796298)(991.27156859,87.61962956)(991.09156494,87.86963186)
\curveto(990.91156895,88.12962905)(990.74156912,88.39462878)(990.58156494,88.66463186)
\curveto(990.41156945,88.94462823)(990.23656962,89.21462796)(990.05656494,89.47463186)
\curveto(989.97656988,89.58462759)(989.90156996,89.68962749)(989.83156494,89.78963186)
\curveto(989.7615701,89.89962728)(989.68657017,90.00962717)(989.60656494,90.11963186)
\curveto(989.57657028,90.15962702)(989.54657031,90.19462698)(989.51656494,90.22463186)
\curveto(989.47657038,90.26462691)(989.44657041,90.30462687)(989.42656494,90.34463186)
\curveto(989.31657054,90.48462669)(989.19157067,90.60962657)(989.05156494,90.71963186)
\curveto(989.02157084,90.73962644)(988.99657086,90.76462641)(988.97656494,90.79463186)
\curveto(988.94657091,90.82462635)(988.91657094,90.84962633)(988.88656494,90.86963186)
\curveto(988.78657107,90.94962623)(988.68657117,91.01462616)(988.58656494,91.06463186)
\curveto(988.48657137,91.12462605)(988.37657148,91.179626)(988.25656494,91.22963186)
\curveto(988.18657167,91.25962592)(988.11157175,91.2796259)(988.03156494,91.28963186)
\lineto(987.79156494,91.34963186)
\lineto(987.70156494,91.34963186)
\curveto(987.67157219,91.35962582)(987.64157222,91.36462581)(987.61156494,91.36463186)
\curveto(987.54157232,91.38462579)(987.44657241,91.38962579)(987.32656494,91.37963186)
\curveto(987.19657266,91.3796258)(987.09657276,91.36962581)(987.02656494,91.34963186)
\curveto(986.94657291,91.32962585)(986.87157299,91.30962587)(986.80156494,91.28963186)
\curveto(986.72157314,91.2796259)(986.64157322,91.25962592)(986.56156494,91.22963186)
\curveto(986.32157354,91.11962606)(986.12157374,90.96962621)(985.96156494,90.77963186)
\curveto(985.79157407,90.59962658)(985.65157421,90.3796268)(985.54156494,90.11963186)
\curveto(985.52157434,90.04962713)(985.50657435,89.9796272)(985.49656494,89.90963186)
\curveto(985.47657438,89.83962734)(985.4565744,89.76462741)(985.43656494,89.68463186)
\curveto(985.41657444,89.60462757)(985.40657445,89.49462768)(985.40656494,89.35463186)
\curveto(985.40657445,89.22462795)(985.41657444,89.11962806)(985.43656494,89.03963186)
\curveto(985.44657441,88.9796282)(985.45157441,88.92462825)(985.45156494,88.87463186)
\curveto(985.45157441,88.82462835)(985.4615744,88.7746284)(985.48156494,88.72463186)
\curveto(985.52157434,88.62462855)(985.5615743,88.52962865)(985.60156494,88.43963186)
\curveto(985.64157422,88.35962882)(985.68657417,88.2796289)(985.73656494,88.19963186)
\curveto(985.7565741,88.16962901)(985.78157408,88.13962904)(985.81156494,88.10963186)
\curveto(985.84157402,88.08962909)(985.86657399,88.06462911)(985.88656494,88.03463186)
\lineto(985.96156494,87.95963186)
\curveto(985.98157388,87.92962925)(986.00157386,87.90462927)(986.02156494,87.88463186)
\lineto(986.23156494,87.73463186)
\curveto(986.29157357,87.69462948)(986.3565735,87.64962953)(986.42656494,87.59963186)
\curveto(986.51657334,87.53962964)(986.62157324,87.48962969)(986.74156494,87.44963186)
\curveto(986.85157301,87.41962976)(986.9615729,87.38462979)(987.07156494,87.34463186)
\curveto(987.18157268,87.30462987)(987.32657253,87.2796299)(987.50656494,87.26963186)
\curveto(987.67657218,87.25962992)(987.80157206,87.22962995)(987.88156494,87.17963186)
\curveto(987.9615719,87.12963005)(988.00657185,87.05463012)(988.01656494,86.95463186)
\curveto(988.02657183,86.85463032)(988.03157183,86.74463043)(988.03156494,86.62463186)
\curveto(988.03157183,86.58463059)(988.03657182,86.54463063)(988.04656494,86.50463186)
\curveto(988.04657181,86.46463071)(988.04157182,86.42963075)(988.03156494,86.39963186)
\curveto(988.01157185,86.34963083)(988.00157186,86.29963088)(988.00156494,86.24963186)
\curveto(988.00157186,86.20963097)(987.99157187,86.16963101)(987.97156494,86.12963186)
\curveto(987.91157195,86.03963114)(987.77657208,85.99463118)(987.56656494,85.99463186)
\lineto(987.44656494,85.99463186)
\curveto(987.38657247,86.00463117)(987.32657253,86.00963117)(987.26656494,86.00963186)
\curveto(987.19657266,86.01963116)(987.13157273,86.02963115)(987.07156494,86.03963186)
\curveto(986.9615729,86.05963112)(986.861573,86.0796311)(986.77156494,86.09963186)
\curveto(986.67157319,86.11963106)(986.57657328,86.14963103)(986.48656494,86.18963186)
\curveto(986.41657344,86.20963097)(986.3565735,86.22963095)(986.30656494,86.24963186)
\lineto(986.12656494,86.30963186)
\curveto(985.86657399,86.42963075)(985.62157424,86.58463059)(985.39156494,86.77463186)
\curveto(985.1615747,86.9746302)(984.97657488,87.18962999)(984.83656494,87.41963186)
\curveto(984.7565751,87.52962965)(984.69157517,87.64462953)(984.64156494,87.76463186)
\lineto(984.49156494,88.15463186)
\curveto(984.44157542,88.26462891)(984.41157545,88.3796288)(984.40156494,88.49963186)
\curveto(984.38157548,88.61962856)(984.3565755,88.74462843)(984.32656494,88.87463186)
\curveto(984.32657553,88.94462823)(984.32657553,89.00962817)(984.32656494,89.06963186)
\curveto(984.31657554,89.12962805)(984.30657555,89.19462798)(984.29656494,89.26463186)
}
}
{
\newrgbcolor{curcolor}{0 0 0}
\pscustom[linestyle=none,fillstyle=solid,fillcolor=curcolor]
{
\newpath
\moveto(989.81656494,101.36424123)
\lineto(990.07156494,101.36424123)
\curveto(990.15156971,101.37423353)(990.22656963,101.36923353)(990.29656494,101.34924123)
\lineto(990.53656494,101.34924123)
\lineto(990.70156494,101.34924123)
\curveto(990.80156906,101.32923357)(990.90656895,101.31923358)(991.01656494,101.31924123)
\curveto(991.11656874,101.31923358)(991.21656864,101.30923359)(991.31656494,101.28924123)
\lineto(991.46656494,101.28924123)
\curveto(991.60656825,101.25923364)(991.74656811,101.23923366)(991.88656494,101.22924123)
\curveto(992.01656784,101.21923368)(992.14656771,101.19423371)(992.27656494,101.15424123)
\curveto(992.3565675,101.13423377)(992.44156742,101.11423379)(992.53156494,101.09424123)
\lineto(992.77156494,101.03424123)
\lineto(993.07156494,100.91424123)
\curveto(993.1615667,100.88423402)(993.25156661,100.84923405)(993.34156494,100.80924123)
\curveto(993.5615663,100.70923419)(993.77656608,100.57423433)(993.98656494,100.40424123)
\curveto(994.19656566,100.24423466)(994.36656549,100.06923483)(994.49656494,99.87924123)
\curveto(994.53656532,99.82923507)(994.57656528,99.76923513)(994.61656494,99.69924123)
\curveto(994.64656521,99.63923526)(994.68156518,99.57923532)(994.72156494,99.51924123)
\curveto(994.77156509,99.43923546)(994.81156505,99.34423556)(994.84156494,99.23424123)
\curveto(994.87156499,99.12423578)(994.90156496,99.01923588)(994.93156494,98.91924123)
\curveto(994.97156489,98.80923609)(994.99656486,98.6992362)(995.00656494,98.58924123)
\curveto(995.01656484,98.47923642)(995.03156483,98.36423654)(995.05156494,98.24424123)
\curveto(995.0615648,98.2042367)(995.0615648,98.15923674)(995.05156494,98.10924123)
\curveto(995.05156481,98.06923683)(995.0565648,98.02923687)(995.06656494,97.98924123)
\curveto(995.07656478,97.94923695)(995.08156478,97.89423701)(995.08156494,97.82424123)
\curveto(995.08156478,97.75423715)(995.07656478,97.7042372)(995.06656494,97.67424123)
\curveto(995.04656481,97.62423728)(995.04156482,97.57923732)(995.05156494,97.53924123)
\curveto(995.0615648,97.4992374)(995.0615648,97.46423744)(995.05156494,97.43424123)
\lineto(995.05156494,97.34424123)
\curveto(995.03156483,97.28423762)(995.01656484,97.21923768)(995.00656494,97.14924123)
\curveto(995.00656485,97.08923781)(995.00156486,97.02423788)(994.99156494,96.95424123)
\curveto(994.94156492,96.78423812)(994.89156497,96.62423828)(994.84156494,96.47424123)
\curveto(994.79156507,96.32423858)(994.72656513,96.17923872)(994.64656494,96.03924123)
\curveto(994.60656525,95.98923891)(994.57656528,95.93423897)(994.55656494,95.87424123)
\curveto(994.52656533,95.82423908)(994.49156537,95.77423913)(994.45156494,95.72424123)
\curveto(994.27156559,95.48423942)(994.05156581,95.28423962)(993.79156494,95.12424123)
\curveto(993.53156633,94.96423994)(993.24656661,94.82424008)(992.93656494,94.70424123)
\curveto(992.79656706,94.64424026)(992.6565672,94.5992403)(992.51656494,94.56924123)
\curveto(992.36656749,94.53924036)(992.21156765,94.5042404)(992.05156494,94.46424123)
\curveto(991.94156792,94.44424046)(991.83156803,94.42924047)(991.72156494,94.41924123)
\curveto(991.61156825,94.40924049)(991.50156836,94.39424051)(991.39156494,94.37424123)
\curveto(991.35156851,94.36424054)(991.31156855,94.35924054)(991.27156494,94.35924123)
\curveto(991.23156863,94.36924053)(991.19156867,94.36924053)(991.15156494,94.35924123)
\curveto(991.10156876,94.34924055)(991.05156881,94.34424056)(991.00156494,94.34424123)
\lineto(990.83656494,94.34424123)
\curveto(990.78656907,94.32424058)(990.73656912,94.31924058)(990.68656494,94.32924123)
\curveto(990.62656923,94.33924056)(990.57156929,94.33924056)(990.52156494,94.32924123)
\curveto(990.48156938,94.31924058)(990.43656942,94.31924058)(990.38656494,94.32924123)
\curveto(990.33656952,94.33924056)(990.28656957,94.33424057)(990.23656494,94.31424123)
\curveto(990.16656969,94.29424061)(990.09156977,94.28924061)(990.01156494,94.29924123)
\curveto(989.92156994,94.30924059)(989.83657002,94.31424059)(989.75656494,94.31424123)
\curveto(989.66657019,94.31424059)(989.56657029,94.30924059)(989.45656494,94.29924123)
\curveto(989.33657052,94.28924061)(989.23657062,94.29424061)(989.15656494,94.31424123)
\lineto(988.87156494,94.31424123)
\lineto(988.24156494,94.35924123)
\curveto(988.14157172,94.36924053)(988.04657181,94.37924052)(987.95656494,94.38924123)
\lineto(987.65656494,94.41924123)
\curveto(987.60657225,94.43924046)(987.5565723,94.44424046)(987.50656494,94.43424123)
\curveto(987.44657241,94.43424047)(987.39157247,94.44424046)(987.34156494,94.46424123)
\curveto(987.17157269,94.51424039)(987.00657285,94.55424035)(986.84656494,94.58424123)
\curveto(986.67657318,94.61424029)(986.51657334,94.66424024)(986.36656494,94.73424123)
\curveto(985.90657395,94.92423998)(985.53157433,95.14423976)(985.24156494,95.39424123)
\curveto(984.95157491,95.65423925)(984.70657515,96.01423889)(984.50656494,96.47424123)
\curveto(984.4565754,96.6042383)(984.42157544,96.73423817)(984.40156494,96.86424123)
\curveto(984.38157548,97.0042379)(984.3565755,97.14423776)(984.32656494,97.28424123)
\curveto(984.31657554,97.35423755)(984.31157555,97.41923748)(984.31156494,97.47924123)
\curveto(984.31157555,97.53923736)(984.30657555,97.6042373)(984.29656494,97.67424123)
\curveto(984.27657558,98.5042364)(984.42657543,99.17423573)(984.74656494,99.68424123)
\curveto(985.0565748,100.19423471)(985.49657436,100.57423433)(986.06656494,100.82424123)
\curveto(986.18657367,100.87423403)(986.31157355,100.91923398)(986.44156494,100.95924123)
\curveto(986.57157329,100.9992339)(986.70657315,101.04423386)(986.84656494,101.09424123)
\curveto(986.92657293,101.11423379)(987.01157285,101.12923377)(987.10156494,101.13924123)
\lineto(987.34156494,101.19924123)
\curveto(987.45157241,101.22923367)(987.5615723,101.24423366)(987.67156494,101.24424123)
\curveto(987.78157208,101.25423365)(987.89157197,101.26923363)(988.00156494,101.28924123)
\curveto(988.05157181,101.30923359)(988.09657176,101.31423359)(988.13656494,101.30424123)
\curveto(988.17657168,101.3042336)(988.21657164,101.30923359)(988.25656494,101.31924123)
\curveto(988.30657155,101.32923357)(988.3615715,101.32923357)(988.42156494,101.31924123)
\curveto(988.47157139,101.31923358)(988.52157134,101.32423358)(988.57156494,101.33424123)
\lineto(988.70656494,101.33424123)
\curveto(988.76657109,101.35423355)(988.83657102,101.35423355)(988.91656494,101.33424123)
\curveto(988.98657087,101.32423358)(989.05157081,101.32923357)(989.11156494,101.34924123)
\curveto(989.14157072,101.35923354)(989.18157068,101.36423354)(989.23156494,101.36424123)
\lineto(989.35156494,101.36424123)
\lineto(989.81656494,101.36424123)
\moveto(992.14156494,99.81924123)
\curveto(991.82156804,99.91923498)(991.4565684,99.97923492)(991.04656494,99.99924123)
\curveto(990.63656922,100.01923488)(990.22656963,100.02923487)(989.81656494,100.02924123)
\curveto(989.38657047,100.02923487)(988.96657089,100.01923488)(988.55656494,99.99924123)
\curveto(988.14657171,99.97923492)(987.7615721,99.93423497)(987.40156494,99.86424123)
\curveto(987.04157282,99.79423511)(986.72157314,99.68423522)(986.44156494,99.53424123)
\curveto(986.15157371,99.39423551)(985.91657394,99.1992357)(985.73656494,98.94924123)
\curveto(985.62657423,98.78923611)(985.54657431,98.60923629)(985.49656494,98.40924123)
\curveto(985.43657442,98.20923669)(985.40657445,97.96423694)(985.40656494,97.67424123)
\curveto(985.42657443,97.65423725)(985.43657442,97.61923728)(985.43656494,97.56924123)
\curveto(985.42657443,97.51923738)(985.42657443,97.47923742)(985.43656494,97.44924123)
\curveto(985.4565744,97.36923753)(985.47657438,97.29423761)(985.49656494,97.22424123)
\curveto(985.50657435,97.16423774)(985.52657433,97.0992378)(985.55656494,97.02924123)
\curveto(985.67657418,96.75923814)(985.84657401,96.53923836)(986.06656494,96.36924123)
\curveto(986.27657358,96.20923869)(986.52157334,96.07423883)(986.80156494,95.96424123)
\curveto(986.91157295,95.91423899)(987.03157283,95.87423903)(987.16156494,95.84424123)
\curveto(987.28157258,95.82423908)(987.40657245,95.7992391)(987.53656494,95.76924123)
\curveto(987.58657227,95.74923915)(987.64157222,95.73923916)(987.70156494,95.73924123)
\curveto(987.75157211,95.73923916)(987.80157206,95.73423917)(987.85156494,95.72424123)
\curveto(987.94157192,95.71423919)(988.03657182,95.7042392)(988.13656494,95.69424123)
\curveto(988.22657163,95.68423922)(988.32157154,95.67423923)(988.42156494,95.66424123)
\curveto(988.50157136,95.66423924)(988.58657127,95.65923924)(988.67656494,95.64924123)
\lineto(988.91656494,95.64924123)
\lineto(989.09656494,95.64924123)
\curveto(989.12657073,95.63923926)(989.1615707,95.63423927)(989.20156494,95.63424123)
\lineto(989.33656494,95.63424123)
\lineto(989.78656494,95.63424123)
\curveto(989.86656999,95.63423927)(989.95156991,95.62923927)(990.04156494,95.61924123)
\curveto(990.12156974,95.61923928)(990.19656966,95.62923927)(990.26656494,95.64924123)
\lineto(990.53656494,95.64924123)
\curveto(990.5565693,95.64923925)(990.58656927,95.64423926)(990.62656494,95.63424123)
\curveto(990.6565692,95.63423927)(990.68156918,95.63923926)(990.70156494,95.64924123)
\curveto(990.80156906,95.65923924)(990.90156896,95.66423924)(991.00156494,95.66424123)
\curveto(991.09156877,95.67423923)(991.19156867,95.68423922)(991.30156494,95.69424123)
\curveto(991.42156844,95.72423918)(991.54656831,95.73923916)(991.67656494,95.73924123)
\curveto(991.79656806,95.74923915)(991.91156795,95.77423913)(992.02156494,95.81424123)
\curveto(992.32156754,95.89423901)(992.58656727,95.97923892)(992.81656494,96.06924123)
\curveto(993.04656681,96.16923873)(993.2615666,96.31423859)(993.46156494,96.50424123)
\curveto(993.6615662,96.71423819)(993.81156605,96.97923792)(993.91156494,97.29924123)
\curveto(993.93156593,97.33923756)(993.94156592,97.37423753)(993.94156494,97.40424123)
\curveto(993.93156593,97.44423746)(993.93656592,97.48923741)(993.95656494,97.53924123)
\curveto(993.96656589,97.57923732)(993.97656588,97.64923725)(993.98656494,97.74924123)
\curveto(993.99656586,97.85923704)(993.99156587,97.94423696)(993.97156494,98.00424123)
\curveto(993.95156591,98.07423683)(993.94156592,98.14423676)(993.94156494,98.21424123)
\curveto(993.93156593,98.28423662)(993.91656594,98.34923655)(993.89656494,98.40924123)
\curveto(993.83656602,98.60923629)(993.75156611,98.78923611)(993.64156494,98.94924123)
\curveto(993.62156624,98.97923592)(993.60156626,99.0042359)(993.58156494,99.02424123)
\lineto(993.52156494,99.08424123)
\curveto(993.50156636,99.12423578)(993.4615664,99.17423573)(993.40156494,99.23424123)
\curveto(993.2615666,99.33423557)(993.13156673,99.41923548)(993.01156494,99.48924123)
\curveto(992.89156697,99.55923534)(992.74656711,99.62923527)(992.57656494,99.69924123)
\curveto(992.50656735,99.72923517)(992.43656742,99.74923515)(992.36656494,99.75924123)
\curveto(992.29656756,99.77923512)(992.22156764,99.7992351)(992.14156494,99.81924123)
}
}
{
\newrgbcolor{curcolor}{0 0 0}
\pscustom[linestyle=none,fillstyle=solid,fillcolor=curcolor]
{
\newpath
\moveto(984.29656494,106.77385061)
\curveto(984.29657556,106.87384575)(984.30657555,106.96884566)(984.32656494,107.05885061)
\curveto(984.33657552,107.14884548)(984.36657549,107.21384541)(984.41656494,107.25385061)
\curveto(984.49657536,107.31384531)(984.60157526,107.34384528)(984.73156494,107.34385061)
\lineto(985.12156494,107.34385061)
\lineto(986.62156494,107.34385061)
\lineto(993.01156494,107.34385061)
\lineto(994.18156494,107.34385061)
\lineto(994.49656494,107.34385061)
\curveto(994.59656526,107.35384527)(994.67656518,107.33884529)(994.73656494,107.29885061)
\curveto(994.81656504,107.24884538)(994.86656499,107.17384545)(994.88656494,107.07385061)
\curveto(994.89656496,106.98384564)(994.90156496,106.87384575)(994.90156494,106.74385061)
\lineto(994.90156494,106.51885061)
\curveto(994.88156498,106.43884619)(994.86656499,106.36884626)(994.85656494,106.30885061)
\curveto(994.83656502,106.24884638)(994.79656506,106.19884643)(994.73656494,106.15885061)
\curveto(994.67656518,106.11884651)(994.60156526,106.09884653)(994.51156494,106.09885061)
\lineto(994.21156494,106.09885061)
\lineto(993.11656494,106.09885061)
\lineto(987.77656494,106.09885061)
\curveto(987.68657217,106.07884655)(987.61157225,106.06384656)(987.55156494,106.05385061)
\curveto(987.48157238,106.05384657)(987.42157244,106.0238466)(987.37156494,105.96385061)
\curveto(987.32157254,105.89384673)(987.29657256,105.80384682)(987.29656494,105.69385061)
\curveto(987.28657257,105.59384703)(987.28157258,105.48384714)(987.28156494,105.36385061)
\lineto(987.28156494,104.22385061)
\lineto(987.28156494,103.72885061)
\curveto(987.27157259,103.56884906)(987.21157265,103.45884917)(987.10156494,103.39885061)
\curveto(987.07157279,103.37884925)(987.04157282,103.36884926)(987.01156494,103.36885061)
\curveto(986.97157289,103.36884926)(986.92657293,103.36384926)(986.87656494,103.35385061)
\curveto(986.7565731,103.33384929)(986.64657321,103.33884929)(986.54656494,103.36885061)
\curveto(986.44657341,103.40884922)(986.37657348,103.46384916)(986.33656494,103.53385061)
\curveto(986.28657357,103.61384901)(986.2615736,103.73384889)(986.26156494,103.89385061)
\curveto(986.2615736,104.05384857)(986.24657361,104.18884844)(986.21656494,104.29885061)
\curveto(986.20657365,104.34884828)(986.20157366,104.40384822)(986.20156494,104.46385061)
\curveto(986.19157367,104.5238481)(986.17657368,104.58384804)(986.15656494,104.64385061)
\curveto(986.10657375,104.79384783)(986.0565738,104.93884769)(986.00656494,105.07885061)
\curveto(985.94657391,105.21884741)(985.87657398,105.35384727)(985.79656494,105.48385061)
\curveto(985.70657415,105.623847)(985.60157426,105.74384688)(985.48156494,105.84385061)
\curveto(985.3615745,105.94384668)(985.23157463,106.03884659)(985.09156494,106.12885061)
\curveto(984.99157487,106.18884644)(984.88157498,106.23384639)(984.76156494,106.26385061)
\curveto(984.64157522,106.30384632)(984.53657532,106.35384627)(984.44656494,106.41385061)
\curveto(984.38657547,106.46384616)(984.34657551,106.53384609)(984.32656494,106.62385061)
\curveto(984.31657554,106.64384598)(984.31157555,106.66884596)(984.31156494,106.69885061)
\curveto(984.31157555,106.7288459)(984.30657555,106.75384587)(984.29656494,106.77385061)
}
}
{
\newrgbcolor{curcolor}{0 0 0}
\pscustom[linestyle=none,fillstyle=solid,fillcolor=curcolor]
{
\newpath
\moveto(984.29656494,115.12345998)
\curveto(984.29657556,115.22345513)(984.30657555,115.31845503)(984.32656494,115.40845998)
\curveto(984.33657552,115.49845485)(984.36657549,115.56345479)(984.41656494,115.60345998)
\curveto(984.49657536,115.66345469)(984.60157526,115.69345466)(984.73156494,115.69345998)
\lineto(985.12156494,115.69345998)
\lineto(986.62156494,115.69345998)
\lineto(993.01156494,115.69345998)
\lineto(994.18156494,115.69345998)
\lineto(994.49656494,115.69345998)
\curveto(994.59656526,115.70345465)(994.67656518,115.68845466)(994.73656494,115.64845998)
\curveto(994.81656504,115.59845475)(994.86656499,115.52345483)(994.88656494,115.42345998)
\curveto(994.89656496,115.33345502)(994.90156496,115.22345513)(994.90156494,115.09345998)
\lineto(994.90156494,114.86845998)
\curveto(994.88156498,114.78845556)(994.86656499,114.71845563)(994.85656494,114.65845998)
\curveto(994.83656502,114.59845575)(994.79656506,114.5484558)(994.73656494,114.50845998)
\curveto(994.67656518,114.46845588)(994.60156526,114.4484559)(994.51156494,114.44845998)
\lineto(994.21156494,114.44845998)
\lineto(993.11656494,114.44845998)
\lineto(987.77656494,114.44845998)
\curveto(987.68657217,114.42845592)(987.61157225,114.41345594)(987.55156494,114.40345998)
\curveto(987.48157238,114.40345595)(987.42157244,114.37345598)(987.37156494,114.31345998)
\curveto(987.32157254,114.24345611)(987.29657256,114.1534562)(987.29656494,114.04345998)
\curveto(987.28657257,113.94345641)(987.28157258,113.83345652)(987.28156494,113.71345998)
\lineto(987.28156494,112.57345998)
\lineto(987.28156494,112.07845998)
\curveto(987.27157259,111.91845843)(987.21157265,111.80845854)(987.10156494,111.74845998)
\curveto(987.07157279,111.72845862)(987.04157282,111.71845863)(987.01156494,111.71845998)
\curveto(986.97157289,111.71845863)(986.92657293,111.71345864)(986.87656494,111.70345998)
\curveto(986.7565731,111.68345867)(986.64657321,111.68845866)(986.54656494,111.71845998)
\curveto(986.44657341,111.75845859)(986.37657348,111.81345854)(986.33656494,111.88345998)
\curveto(986.28657357,111.96345839)(986.2615736,112.08345827)(986.26156494,112.24345998)
\curveto(986.2615736,112.40345795)(986.24657361,112.53845781)(986.21656494,112.64845998)
\curveto(986.20657365,112.69845765)(986.20157366,112.7534576)(986.20156494,112.81345998)
\curveto(986.19157367,112.87345748)(986.17657368,112.93345742)(986.15656494,112.99345998)
\curveto(986.10657375,113.14345721)(986.0565738,113.28845706)(986.00656494,113.42845998)
\curveto(985.94657391,113.56845678)(985.87657398,113.70345665)(985.79656494,113.83345998)
\curveto(985.70657415,113.97345638)(985.60157426,114.09345626)(985.48156494,114.19345998)
\curveto(985.3615745,114.29345606)(985.23157463,114.38845596)(985.09156494,114.47845998)
\curveto(984.99157487,114.53845581)(984.88157498,114.58345577)(984.76156494,114.61345998)
\curveto(984.64157522,114.6534557)(984.53657532,114.70345565)(984.44656494,114.76345998)
\curveto(984.38657547,114.81345554)(984.34657551,114.88345547)(984.32656494,114.97345998)
\curveto(984.31657554,114.99345536)(984.31157555,115.01845533)(984.31156494,115.04845998)
\curveto(984.31157555,115.07845527)(984.30657555,115.10345525)(984.29656494,115.12345998)
}
}
{
\newrgbcolor{curcolor}{0 0 0}
\pscustom[linestyle=none,fillstyle=solid,fillcolor=curcolor]
{
\newpath
\moveto(1016.14292725,38.71181936)
\curveto(1016.19292799,38.73180981)(1016.25292793,38.75680979)(1016.32292725,38.78681936)
\curveto(1016.39292779,38.81680973)(1016.46792772,38.83680971)(1016.54792725,38.84681936)
\curveto(1016.61792757,38.86680968)(1016.6879275,38.86680968)(1016.75792725,38.84681936)
\curveto(1016.81792737,38.83680971)(1016.86292732,38.79680975)(1016.89292725,38.72681936)
\curveto(1016.91292727,38.67680987)(1016.92292726,38.61680993)(1016.92292725,38.54681936)
\lineto(1016.92292725,38.33681936)
\lineto(1016.92292725,37.88681936)
\curveto(1016.92292726,37.73681081)(1016.89792729,37.61681093)(1016.84792725,37.52681936)
\curveto(1016.7879274,37.42681112)(1016.6829275,37.35181119)(1016.53292725,37.30181936)
\curveto(1016.3829278,37.26181128)(1016.24792794,37.21681133)(1016.12792725,37.16681936)
\curveto(1015.86792832,37.05681149)(1015.59792859,36.95681159)(1015.31792725,36.86681936)
\curveto(1015.03792915,36.77681177)(1014.76292942,36.67681187)(1014.49292725,36.56681936)
\curveto(1014.40292978,36.53681201)(1014.31792987,36.50681204)(1014.23792725,36.47681936)
\curveto(1014.15793003,36.45681209)(1014.0829301,36.42681212)(1014.01292725,36.38681936)
\curveto(1013.94293024,36.35681219)(1013.8829303,36.31181223)(1013.83292725,36.25181936)
\curveto(1013.7829304,36.19181235)(1013.74293044,36.11181243)(1013.71292725,36.01181936)
\curveto(1013.69293049,35.96181258)(1013.6879305,35.90181264)(1013.69792725,35.83181936)
\lineto(1013.69792725,35.63681936)
\lineto(1013.69792725,32.80181936)
\lineto(1013.69792725,32.50181936)
\curveto(1013.6879305,32.39181615)(1013.6879305,32.28681626)(1013.69792725,32.18681936)
\curveto(1013.70793048,32.08681646)(1013.72293046,31.99181655)(1013.74292725,31.90181936)
\curveto(1013.76293042,31.82181672)(1013.80293038,31.76181678)(1013.86292725,31.72181936)
\curveto(1013.96293022,31.6418169)(1014.07793011,31.58181696)(1014.20792725,31.54181936)
\curveto(1014.32792986,31.51181703)(1014.45292973,31.47181707)(1014.58292725,31.42181936)
\curveto(1014.81292937,31.32181722)(1015.05292913,31.22681732)(1015.30292725,31.13681936)
\curveto(1015.55292863,31.05681749)(1015.79292839,30.96681758)(1016.02292725,30.86681936)
\curveto(1016.0829281,30.8468177)(1016.15292803,30.82181772)(1016.23292725,30.79181936)
\curveto(1016.30292788,30.77181777)(1016.37792781,30.7468178)(1016.45792725,30.71681936)
\curveto(1016.53792765,30.68681786)(1016.61292757,30.65181789)(1016.68292725,30.61181936)
\curveto(1016.74292744,30.58181796)(1016.7879274,30.546818)(1016.81792725,30.50681936)
\curveto(1016.87792731,30.42681812)(1016.91292727,30.31681823)(1016.92292725,30.17681936)
\lineto(1016.92292725,29.75681936)
\lineto(1016.92292725,29.51681936)
\curveto(1016.91292727,29.4468191)(1016.8879273,29.38681916)(1016.84792725,29.33681936)
\curveto(1016.81792737,29.28681926)(1016.77292741,29.25681929)(1016.71292725,29.24681936)
\curveto(1016.65292753,29.2468193)(1016.59292759,29.25181929)(1016.53292725,29.26181936)
\curveto(1016.46292772,29.28181926)(1016.39792779,29.30181924)(1016.33792725,29.32181936)
\curveto(1016.26792792,29.35181919)(1016.21792797,29.37681917)(1016.18792725,29.39681936)
\curveto(1015.86792832,29.53681901)(1015.55292863,29.66181888)(1015.24292725,29.77181936)
\curveto(1014.92292926,29.88181866)(1014.60292958,30.00181854)(1014.28292725,30.13181936)
\curveto(1014.06293012,30.22181832)(1013.84793034,30.30681824)(1013.63792725,30.38681936)
\curveto(1013.41793077,30.46681808)(1013.19793099,30.55181799)(1012.97792725,30.64181936)
\curveto(1012.25793193,30.9418176)(1011.53293265,31.22681732)(1010.80292725,31.49681936)
\curveto(1010.06293412,31.76681678)(1009.32793486,32.05181649)(1008.59792725,32.35181936)
\curveto(1008.33793585,32.46181608)(1008.07293611,32.56181598)(1007.80292725,32.65181936)
\curveto(1007.53293665,32.75181579)(1007.26793692,32.85681569)(1007.00792725,32.96681936)
\curveto(1006.89793729,33.01681553)(1006.77793741,33.06181548)(1006.64792725,33.10181936)
\curveto(1006.50793768,33.15181539)(1006.40793778,33.22181532)(1006.34792725,33.31181936)
\curveto(1006.30793788,33.35181519)(1006.27793791,33.41681513)(1006.25792725,33.50681936)
\curveto(1006.24793794,33.52681502)(1006.24793794,33.546815)(1006.25792725,33.56681936)
\curveto(1006.25793793,33.59681495)(1006.25293793,33.62181492)(1006.24292725,33.64181936)
\curveto(1006.24293794,33.82181472)(1006.24293794,34.03181451)(1006.24292725,34.27181936)
\curveto(1006.23293795,34.51181403)(1006.26793792,34.68681386)(1006.34792725,34.79681936)
\curveto(1006.40793778,34.87681367)(1006.50793768,34.93681361)(1006.64792725,34.97681936)
\curveto(1006.77793741,35.02681352)(1006.89793729,35.07681347)(1007.00792725,35.12681936)
\curveto(1007.23793695,35.22681332)(1007.46793672,35.31681323)(1007.69792725,35.39681936)
\curveto(1007.92793626,35.47681307)(1008.15793603,35.56681298)(1008.38792725,35.66681936)
\curveto(1008.5879356,35.7468128)(1008.79293539,35.82181272)(1009.00292725,35.89181936)
\curveto(1009.21293497,35.97181257)(1009.41793477,36.05681249)(1009.61792725,36.14681936)
\curveto(1010.34793384,36.4468121)(1011.0879331,36.73181181)(1011.83792725,37.00181936)
\curveto(1012.57793161,37.28181126)(1013.31293087,37.57681097)(1014.04292725,37.88681936)
\curveto(1014.13293005,37.92681062)(1014.21792997,37.95681059)(1014.29792725,37.97681936)
\curveto(1014.37792981,38.00681054)(1014.46292972,38.03681051)(1014.55292725,38.06681936)
\curveto(1014.81292937,38.17681037)(1015.07792911,38.28181026)(1015.34792725,38.38181936)
\curveto(1015.61792857,38.49181005)(1015.8829283,38.60180994)(1016.14292725,38.71181936)
\moveto(1012.49792725,35.50181936)
\curveto(1012.46793172,35.59181295)(1012.41793177,35.6468129)(1012.34792725,35.66681936)
\curveto(1012.27793191,35.69681285)(1012.20293198,35.70181284)(1012.12292725,35.68181936)
\curveto(1012.03293215,35.67181287)(1011.94793224,35.6468129)(1011.86792725,35.60681936)
\curveto(1011.77793241,35.57681297)(1011.70293248,35.546813)(1011.64292725,35.51681936)
\curveto(1011.60293258,35.49681305)(1011.56793262,35.48681306)(1011.53792725,35.48681936)
\curveto(1011.50793268,35.48681306)(1011.47293271,35.47681307)(1011.43292725,35.45681936)
\lineto(1011.19292725,35.36681936)
\curveto(1011.10293308,35.3468132)(1011.01293317,35.31681323)(1010.92292725,35.27681936)
\curveto(1010.56293362,35.12681342)(1010.19793399,34.99181355)(1009.82792725,34.87181936)
\curveto(1009.44793474,34.76181378)(1009.07793511,34.63181391)(1008.71792725,34.48181936)
\curveto(1008.60793558,34.43181411)(1008.49793569,34.38681416)(1008.38792725,34.34681936)
\curveto(1008.27793591,34.31681423)(1008.17293601,34.27681427)(1008.07292725,34.22681936)
\curveto(1008.02293616,34.20681434)(1007.97793621,34.18181436)(1007.93792725,34.15181936)
\curveto(1007.8879363,34.13181441)(1007.86293632,34.08181446)(1007.86292725,34.00181936)
\curveto(1007.8829363,33.98181456)(1007.89793629,33.96181458)(1007.90792725,33.94181936)
\curveto(1007.91793627,33.92181462)(1007.93293625,33.90181464)(1007.95292725,33.88181936)
\curveto(1008.00293618,33.8418147)(1008.05793613,33.81181473)(1008.11792725,33.79181936)
\curveto(1008.16793602,33.77181477)(1008.22293596,33.75181479)(1008.28292725,33.73181936)
\curveto(1008.39293579,33.68181486)(1008.50293568,33.6418149)(1008.61292725,33.61181936)
\curveto(1008.72293546,33.58181496)(1008.83293535,33.541815)(1008.94292725,33.49181936)
\curveto(1009.33293485,33.32181522)(1009.72793446,33.17181537)(1010.12792725,33.04181936)
\curveto(1010.52793366,32.92181562)(1010.91793327,32.78181576)(1011.29792725,32.62181936)
\lineto(1011.44792725,32.56181936)
\curveto(1011.49793269,32.55181599)(1011.54793264,32.53681601)(1011.59792725,32.51681936)
\lineto(1011.83792725,32.42681936)
\curveto(1011.91793227,32.39681615)(1011.99793219,32.37181617)(1012.07792725,32.35181936)
\curveto(1012.12793206,32.33181621)(1012.182932,32.32181622)(1012.24292725,32.32181936)
\curveto(1012.30293188,32.33181621)(1012.35293183,32.3468162)(1012.39292725,32.36681936)
\curveto(1012.47293171,32.41681613)(1012.51793167,32.52181602)(1012.52792725,32.68181936)
\lineto(1012.52792725,33.13181936)
\lineto(1012.52792725,34.73681936)
\curveto(1012.52793166,34.8468137)(1012.53293165,34.98181356)(1012.54292725,35.14181936)
\curveto(1012.54293164,35.30181324)(1012.52793166,35.42181312)(1012.49792725,35.50181936)
}
}
{
\newrgbcolor{curcolor}{0 0 0}
\pscustom[linestyle=none,fillstyle=solid,fillcolor=curcolor]
{
\newpath
\moveto(1009.30292725,46.44338186)
\curveto(1009.35293483,46.51337426)(1009.42793476,46.54837422)(1009.52792725,46.54838186)
\curveto(1009.62793456,46.55837421)(1009.73293445,46.56337421)(1009.84292725,46.56338186)
\lineto(1016.11292725,46.56338186)
\lineto(1016.71292725,46.56338186)
\curveto(1016.76292742,46.54337423)(1016.81292737,46.53837423)(1016.86292725,46.54838186)
\curveto(1016.90292728,46.55837421)(1016.94792724,46.55337422)(1016.99792725,46.53338186)
\curveto(1017.09792709,46.51337426)(1017.19792699,46.49837427)(1017.29792725,46.48838186)
\curveto(1017.40792678,46.48837428)(1017.51292667,46.4733743)(1017.61292725,46.44338186)
\curveto(1017.72292646,46.41337436)(1017.82792636,46.38337439)(1017.92792725,46.35338186)
\curveto(1018.02792616,46.33337444)(1018.12792606,46.29837447)(1018.22792725,46.24838186)
\curveto(1018.4879257,46.14837462)(1018.72292546,46.01837475)(1018.93292725,45.85838186)
\curveto(1019.14292504,45.70837506)(1019.31792487,45.52837524)(1019.45792725,45.31838186)
\curveto(1019.57792461,45.14837562)(1019.67292451,44.9683758)(1019.74292725,44.77838186)
\curveto(1019.82292436,44.58837618)(1019.89792429,44.38337639)(1019.96792725,44.16338186)
\curveto(1019.9879242,44.0733767)(1019.99792419,43.98337679)(1019.99792725,43.89338186)
\curveto(1020.00792418,43.80337697)(1020.02292416,43.71337706)(1020.04292725,43.62338186)
\lineto(1020.04292725,43.53338186)
\curveto(1020.05292413,43.51337726)(1020.05792413,43.49337728)(1020.05792725,43.47338186)
\curveto(1020.06792412,43.42337735)(1020.06792412,43.3733774)(1020.05792725,43.32338186)
\curveto(1020.04792414,43.28337749)(1020.05292413,43.23837753)(1020.07292725,43.18838186)
\curveto(1020.09292409,43.11837765)(1020.09792409,43.00837776)(1020.08792725,42.85838186)
\curveto(1020.0879241,42.71837805)(1020.07792411,42.61837815)(1020.05792725,42.55838186)
\curveto(1020.05792413,42.52837824)(1020.05292413,42.49837827)(1020.04292725,42.46838186)
\lineto(1020.04292725,42.40838186)
\curveto(1020.02292416,42.31837845)(1020.00792418,42.22837854)(1019.99792725,42.13838186)
\curveto(1019.99792419,42.04837872)(1019.9879242,41.96337881)(1019.96792725,41.88338186)
\curveto(1019.94792424,41.80337897)(1019.92292426,41.72337905)(1019.89292725,41.64338186)
\curveto(1019.87292431,41.56337921)(1019.84792434,41.48337929)(1019.81792725,41.40338186)
\curveto(1019.6879245,41.08337969)(1019.54292464,40.81337996)(1019.38292725,40.59338186)
\curveto(1019.22292496,40.38338039)(1018.99792519,40.19338058)(1018.70792725,40.02338186)
\curveto(1018.6879255,40.00338077)(1018.66292552,39.98838078)(1018.63292725,39.97838186)
\curveto(1018.61292557,39.97838079)(1018.5879256,39.9683808)(1018.55792725,39.94838186)
\curveto(1018.47792571,39.91838085)(1018.36292582,39.88338089)(1018.21292725,39.84338186)
\curveto(1018.07292611,39.81338096)(1017.96792622,39.84338093)(1017.89792725,39.93338186)
\curveto(1017.84792634,39.99338078)(1017.82292636,40.0733807)(1017.82292725,40.17338186)
\lineto(1017.82292725,40.50338186)
\lineto(1017.82292725,40.66838186)
\curveto(1017.82292636,40.72838004)(1017.83292635,40.78337999)(1017.85292725,40.83338186)
\curveto(1017.8829263,40.92337985)(1017.93292625,40.98837978)(1018.00292725,41.02838186)
\curveto(1018.07292611,41.0683797)(1018.14792604,41.11337966)(1018.22792725,41.16338186)
\lineto(1018.40792725,41.28338186)
\curveto(1018.47792571,41.33337944)(1018.53292565,41.38337939)(1018.57292725,41.43338186)
\curveto(1018.76292542,41.68337909)(1018.90292528,41.98337879)(1018.99292725,42.33338186)
\curveto(1019.01292517,42.39337838)(1019.02292516,42.45337832)(1019.02292725,42.51338186)
\curveto(1019.03292515,42.58337819)(1019.04792514,42.64837812)(1019.06792725,42.70838186)
\lineto(1019.06792725,42.79838186)
\curveto(1019.0879251,42.8683779)(1019.09792509,42.95337782)(1019.09792725,43.05338186)
\curveto(1019.09792509,43.15337762)(1019.0879251,43.24337753)(1019.06792725,43.32338186)
\curveto(1019.05792513,43.35337742)(1019.05292513,43.39337738)(1019.05292725,43.44338186)
\curveto(1019.03292515,43.54337723)(1019.01292517,43.63837713)(1018.99292725,43.72838186)
\curveto(1018.9829252,43.81837695)(1018.95792523,43.90337687)(1018.91792725,43.98338186)
\curveto(1018.79792539,44.2733765)(1018.63292555,44.50837626)(1018.42292725,44.68838186)
\curveto(1018.22292596,44.87837589)(1017.97792621,45.03337574)(1017.68792725,45.15338186)
\curveto(1017.59792659,45.19337558)(1017.50292668,45.21837555)(1017.40292725,45.22838186)
\curveto(1017.30292688,45.24837552)(1017.19792699,45.2733755)(1017.08792725,45.30338186)
\curveto(1017.03792715,45.32337545)(1016.9879272,45.33337544)(1016.93792725,45.33338186)
\curveto(1016.8879273,45.33337544)(1016.83792735,45.33837543)(1016.78792725,45.34838186)
\curveto(1016.75792743,45.35837541)(1016.70792748,45.36337541)(1016.63792725,45.36338186)
\curveto(1016.55792763,45.38337539)(1016.47292771,45.38337539)(1016.38292725,45.36338186)
\curveto(1016.33292785,45.35337542)(1016.2879279,45.34837542)(1016.24792725,45.34838186)
\curveto(1016.20792798,45.35837541)(1016.17292801,45.35337542)(1016.14292725,45.33338186)
\curveto(1016.12292806,45.31337546)(1016.11292807,45.29837547)(1016.11292725,45.28838186)
\lineto(1016.06792725,45.24338186)
\curveto(1016.06792812,45.14337563)(1016.09792809,45.0683757)(1016.15792725,45.01838186)
\curveto(1016.20792798,44.97837579)(1016.25292793,44.92837584)(1016.29292725,44.86838186)
\lineto(1016.50292725,44.62838186)
\curveto(1016.56292762,44.54837622)(1016.61792757,44.45837631)(1016.66792725,44.35838186)
\curveto(1016.75792743,44.21837655)(1016.83292735,44.04337673)(1016.89292725,43.83338186)
\curveto(1016.94292724,43.62337715)(1016.97792721,43.40337737)(1016.99792725,43.17338186)
\curveto(1017.01792717,42.94337783)(1017.01292717,42.71337806)(1016.98292725,42.48338186)
\curveto(1016.96292722,42.25337852)(1016.92292726,42.04337873)(1016.86292725,41.85338186)
\curveto(1016.55292763,40.91337986)(1015.95792823,40.25338052)(1015.07792725,39.87338186)
\curveto(1014.97792921,39.82338095)(1014.8829293,39.78338099)(1014.79292725,39.75338186)
\curveto(1014.69292949,39.72338105)(1014.5879296,39.68838108)(1014.47792725,39.64838186)
\curveto(1014.42792976,39.62838114)(1014.3829298,39.61838115)(1014.34292725,39.61838186)
\curveto(1014.30292988,39.61838115)(1014.25792993,39.60838116)(1014.20792725,39.58838186)
\curveto(1014.13793005,39.5683812)(1014.06793012,39.55338122)(1013.99792725,39.54338186)
\curveto(1013.91793027,39.54338123)(1013.84293034,39.53338124)(1013.77292725,39.51338186)
\curveto(1013.73293045,39.50338127)(1013.69793049,39.49838127)(1013.66792725,39.49838186)
\curveto(1013.62793056,39.50838126)(1013.5879306,39.50838126)(1013.54792725,39.49838186)
\curveto(1013.50793068,39.49838127)(1013.46793072,39.49338128)(1013.42792725,39.48338186)
\lineto(1013.30792725,39.48338186)
\curveto(1013.187931,39.46338131)(1013.06293112,39.46338131)(1012.93292725,39.48338186)
\curveto(1012.87293131,39.49338128)(1012.81293137,39.49838127)(1012.75292725,39.49838186)
\lineto(1012.58792725,39.49838186)
\curveto(1012.53793165,39.50838126)(1012.49793169,39.51338126)(1012.46792725,39.51338186)
\curveto(1012.42793176,39.51338126)(1012.3829318,39.51838125)(1012.33292725,39.52838186)
\curveto(1012.22293196,39.55838121)(1012.11793207,39.57838119)(1012.01792725,39.58838186)
\curveto(1011.90793228,39.59838117)(1011.79793239,39.62338115)(1011.68792725,39.66338186)
\curveto(1011.56793262,39.70338107)(1011.45293273,39.73838103)(1011.34292725,39.76838186)
\curveto(1011.22293296,39.80838096)(1011.10793308,39.85338092)(1010.99792725,39.90338186)
\curveto(1010.83793335,39.9733808)(1010.69293349,40.05338072)(1010.56292725,40.14338186)
\curveto(1010.42293376,40.23338054)(1010.2879339,40.32838044)(1010.15792725,40.42838186)
\curveto(1010.04793414,40.49838027)(1009.95793423,40.58838018)(1009.88792725,40.69838186)
\lineto(1009.82792725,40.75838186)
\lineto(1009.76792725,40.81838186)
\lineto(1009.64792725,40.96838186)
\lineto(1009.52792725,41.14838186)
\curveto(1009.44793474,41.27837949)(1009.37793481,41.41337936)(1009.31792725,41.55338186)
\curveto(1009.25793493,41.70337907)(1009.20293498,41.86337891)(1009.15292725,42.03338186)
\curveto(1009.12293506,42.13337864)(1009.10293508,42.23337854)(1009.09292725,42.33338186)
\curveto(1009.0829351,42.44337833)(1009.06793512,42.55337822)(1009.04792725,42.66338186)
\curveto(1009.03793515,42.70337807)(1009.03793515,42.75337802)(1009.04792725,42.81338186)
\curveto(1009.05793513,42.88337789)(1009.05293513,42.93337784)(1009.03292725,42.96338186)
\curveto(1009.02293516,43.28337749)(1009.05293513,43.5683772)(1009.12292725,43.81838186)
\curveto(1009.19293499,44.07837669)(1009.29293489,44.30837646)(1009.42292725,44.50838186)
\curveto(1009.46293472,44.57837619)(1009.50793468,44.64337613)(1009.55792725,44.70338186)
\lineto(1009.70792725,44.88338186)
\curveto(1009.74793444,44.93337584)(1009.79293439,44.97837579)(1009.84292725,45.01838186)
\curveto(1009.8829343,45.0683757)(1009.90293428,45.14337563)(1009.90292725,45.24338186)
\lineto(1009.85792725,45.28838186)
\curveto(1009.83793435,45.30837546)(1009.81293437,45.32837544)(1009.78292725,45.34838186)
\curveto(1009.70293448,45.37837539)(1009.62293456,45.39337538)(1009.54292725,45.39338186)
\curveto(1009.46293472,45.40337537)(1009.39293479,45.43337534)(1009.33292725,45.48338186)
\curveto(1009.29293489,45.51337526)(1009.26293492,45.5733752)(1009.24292725,45.66338186)
\curveto(1009.21293497,45.75337502)(1009.19793499,45.84837492)(1009.19792725,45.94838186)
\curveto(1009.19793499,46.04837472)(1009.20793498,46.14337463)(1009.22792725,46.23338186)
\curveto(1009.24793494,46.33337444)(1009.27293491,46.40337437)(1009.30292725,46.44338186)
\moveto(1013.08292725,45.31838186)
\curveto(1013.04293114,45.32837544)(1012.99293119,45.33337544)(1012.93292725,45.33338186)
\curveto(1012.86293132,45.33337544)(1012.80793138,45.32837544)(1012.76792725,45.31838186)
\lineto(1012.52792725,45.31838186)
\curveto(1012.43793175,45.29837547)(1012.35293183,45.28337549)(1012.27292725,45.27338186)
\curveto(1012.182932,45.26337551)(1012.09793209,45.24837552)(1012.01792725,45.22838186)
\curveto(1011.93793225,45.20837556)(1011.86293232,45.18837558)(1011.79292725,45.16838186)
\curveto(1011.71293247,45.15837561)(1011.63793255,45.13837563)(1011.56792725,45.10838186)
\curveto(1011.2879329,44.99837577)(1011.03793315,44.85337592)(1010.81792725,44.67338186)
\curveto(1010.59793359,44.50337627)(1010.43293375,44.28337649)(1010.32292725,44.01338186)
\curveto(1010.2829339,43.93337684)(1010.25293393,43.84837692)(1010.23292725,43.75838186)
\curveto(1010.20293398,43.6683771)(1010.17793401,43.5733772)(1010.15792725,43.47338186)
\curveto(1010.13793405,43.39337738)(1010.13293405,43.30337747)(1010.14292725,43.20338186)
\lineto(1010.14292725,42.93338186)
\curveto(1010.15293403,42.88337789)(1010.15793403,42.83337794)(1010.15792725,42.78338186)
\curveto(1010.15793403,42.74337803)(1010.16293402,42.69837807)(1010.17292725,42.64838186)
\curveto(1010.22293396,42.45837831)(1010.27293391,42.29837847)(1010.32292725,42.16838186)
\curveto(1010.46293372,41.82837894)(1010.67293351,41.56337921)(1010.95292725,41.37338186)
\curveto(1011.23293295,41.18337959)(1011.55793263,41.03337974)(1011.92792725,40.92338186)
\curveto(1012.00793218,40.90337987)(1012.0879321,40.88837988)(1012.16792725,40.87838186)
\curveto(1012.23793195,40.87837989)(1012.31293187,40.8683799)(1012.39292725,40.84838186)
\curveto(1012.42293176,40.82837994)(1012.45793173,40.81837995)(1012.49792725,40.81838186)
\curveto(1012.53793165,40.82837994)(1012.57293161,40.82837994)(1012.60292725,40.81838186)
\lineto(1012.93292725,40.81838186)
\lineto(1013.27792725,40.81838186)
\curveto(1013.3879308,40.81837995)(1013.49293069,40.82837994)(1013.59292725,40.84838186)
\lineto(1013.66792725,40.84838186)
\curveto(1013.69793049,40.85837991)(1013.72293046,40.86337991)(1013.74292725,40.86338186)
\curveto(1013.83293035,40.88337989)(1013.92293026,40.89837987)(1014.01292725,40.90838186)
\curveto(1014.10293008,40.92837984)(1014.18793,40.95337982)(1014.26792725,40.98338186)
\curveto(1014.52792966,41.06337971)(1014.76792942,41.16337961)(1014.98792725,41.28338186)
\curveto(1015.20792898,41.40337937)(1015.3879288,41.56337921)(1015.52792725,41.76338186)
\lineto(1015.61792725,41.88338186)
\curveto(1015.63792855,41.92337885)(1015.65792853,41.9683788)(1015.67792725,42.01838186)
\curveto(1015.72792846,42.09837867)(1015.76792842,42.18337859)(1015.79792725,42.27338186)
\curveto(1015.82792836,42.36337841)(1015.85792833,42.46337831)(1015.88792725,42.57338186)
\curveto(1015.89792829,42.62337815)(1015.90292828,42.6683781)(1015.90292725,42.70838186)
\curveto(1015.89292829,42.75837801)(1015.89792829,42.80837796)(1015.91792725,42.85838186)
\curveto(1015.92792826,42.88837788)(1015.93292825,42.93837783)(1015.93292725,43.00838186)
\curveto(1015.93292825,43.07837769)(1015.92792826,43.12837764)(1015.91792725,43.15838186)
\curveto(1015.90792828,43.18837758)(1015.90792828,43.21837755)(1015.91792725,43.24838186)
\curveto(1015.91792827,43.28837748)(1015.91292827,43.32837744)(1015.90292725,43.36838186)
\curveto(1015.8829283,43.45837731)(1015.86292832,43.54337723)(1015.84292725,43.62338186)
\curveto(1015.82292836,43.70337707)(1015.79792839,43.78337699)(1015.76792725,43.86338186)
\curveto(1015.61792857,44.20337657)(1015.40792878,44.4733763)(1015.13792725,44.67338186)
\curveto(1014.86792932,44.8733759)(1014.55292963,45.03337574)(1014.19292725,45.15338186)
\curveto(1014.10293008,45.18337559)(1014.01293017,45.20337557)(1013.92292725,45.21338186)
\curveto(1013.82293036,45.23337554)(1013.72793046,45.25337552)(1013.63792725,45.27338186)
\curveto(1013.59793059,45.28337549)(1013.56293062,45.28837548)(1013.53292725,45.28838186)
\curveto(1013.49293069,45.28837548)(1013.45293073,45.29337548)(1013.41292725,45.30338186)
\curveto(1013.36293082,45.32337545)(1013.31293087,45.32337545)(1013.26292725,45.30338186)
\curveto(1013.20293098,45.29337548)(1013.14293104,45.29837547)(1013.08292725,45.31838186)
}
}
{
\newrgbcolor{curcolor}{0 0 0}
\pscustom[linestyle=none,fillstyle=solid,fillcolor=curcolor]
{
\newpath
\moveto(1012.72292725,55.56666311)
\curveto(1012.7829314,55.58665505)(1012.87793131,55.59665504)(1013.00792725,55.59666311)
\curveto(1013.12793106,55.59665504)(1013.21293097,55.59165504)(1013.26292725,55.58166311)
\lineto(1013.41292725,55.58166311)
\curveto(1013.49293069,55.57165506)(1013.56793062,55.56165507)(1013.63792725,55.55166311)
\curveto(1013.69793049,55.55165508)(1013.76793042,55.54665509)(1013.84792725,55.53666311)
\curveto(1013.90793028,55.51665512)(1013.96793022,55.50165513)(1014.02792725,55.49166311)
\curveto(1014.0879301,55.49165514)(1014.14793004,55.48165515)(1014.20792725,55.46166311)
\curveto(1014.33792985,55.42165521)(1014.46792972,55.38665525)(1014.59792725,55.35666311)
\curveto(1014.72792946,55.32665531)(1014.84792934,55.28665535)(1014.95792725,55.23666311)
\curveto(1015.43792875,55.02665561)(1015.84292834,54.74665589)(1016.17292725,54.39666311)
\curveto(1016.49292769,54.04665659)(1016.73792745,53.61665702)(1016.90792725,53.10666311)
\curveto(1016.94792724,52.99665764)(1016.97792721,52.87665776)(1016.99792725,52.74666311)
\curveto(1017.01792717,52.62665801)(1017.03792715,52.50165813)(1017.05792725,52.37166311)
\curveto(1017.06792712,52.31165832)(1017.07292711,52.24665839)(1017.07292725,52.17666311)
\curveto(1017.0829271,52.11665852)(1017.0879271,52.05665858)(1017.08792725,51.99666311)
\curveto(1017.09792709,51.95665868)(1017.10292708,51.89665874)(1017.10292725,51.81666311)
\curveto(1017.10292708,51.74665889)(1017.09792709,51.69665894)(1017.08792725,51.66666311)
\curveto(1017.07792711,51.62665901)(1017.07292711,51.58665905)(1017.07292725,51.54666311)
\curveto(1017.0829271,51.50665913)(1017.0829271,51.47165916)(1017.07292725,51.44166311)
\lineto(1017.07292725,51.35166311)
\lineto(1017.02792725,50.99166311)
\curveto(1016.9879272,50.85165978)(1016.94792724,50.71665992)(1016.90792725,50.58666311)
\curveto(1016.86792732,50.45666018)(1016.82292736,50.3316603)(1016.77292725,50.21166311)
\curveto(1016.57292761,49.76166087)(1016.31292787,49.39166124)(1015.99292725,49.10166311)
\curveto(1015.67292851,48.81166182)(1015.2829289,48.57166206)(1014.82292725,48.38166311)
\curveto(1014.72292946,48.3316623)(1014.62292956,48.29166234)(1014.52292725,48.26166311)
\curveto(1014.42292976,48.24166239)(1014.31792987,48.22166241)(1014.20792725,48.20166311)
\curveto(1014.16793002,48.18166245)(1014.13793005,48.17166246)(1014.11792725,48.17166311)
\curveto(1014.0879301,48.18166245)(1014.05293013,48.18166245)(1014.01292725,48.17166311)
\curveto(1013.93293025,48.15166248)(1013.85293033,48.1366625)(1013.77292725,48.12666311)
\curveto(1013.6829305,48.12666251)(1013.59793059,48.11666252)(1013.51792725,48.09666311)
\lineto(1013.39792725,48.09666311)
\curveto(1013.35793083,48.09666254)(1013.31293087,48.09166254)(1013.26292725,48.08166311)
\curveto(1013.21293097,48.07166256)(1013.12793106,48.06666257)(1013.00792725,48.06666311)
\curveto(1012.87793131,48.06666257)(1012.7829314,48.07666256)(1012.72292725,48.09666311)
\curveto(1012.65293153,48.11666252)(1012.5829316,48.12166251)(1012.51292725,48.11166311)
\curveto(1012.44293174,48.10166253)(1012.37293181,48.10666253)(1012.30292725,48.12666311)
\curveto(1012.25293193,48.1366625)(1012.21293197,48.14166249)(1012.18292725,48.14166311)
\curveto(1012.14293204,48.15166248)(1012.09793209,48.16166247)(1012.04792725,48.17166311)
\curveto(1011.92793226,48.20166243)(1011.80793238,48.22666241)(1011.68792725,48.24666311)
\curveto(1011.56793262,48.27666236)(1011.45293273,48.31666232)(1011.34292725,48.36666311)
\curveto(1010.97293321,48.51666212)(1010.64293354,48.69666194)(1010.35292725,48.90666311)
\curveto(1010.05293413,49.12666151)(1009.80293438,49.39166124)(1009.60292725,49.70166311)
\curveto(1009.52293466,49.82166081)(1009.45793473,49.94666069)(1009.40792725,50.07666311)
\curveto(1009.34793484,50.20666043)(1009.2879349,50.34166029)(1009.22792725,50.48166311)
\curveto(1009.17793501,50.60166003)(1009.14793504,50.7316599)(1009.13792725,50.87166311)
\curveto(1009.11793507,51.01165962)(1009.0879351,51.15165948)(1009.04792725,51.29166311)
\lineto(1009.04792725,51.48666311)
\curveto(1009.03793515,51.55665908)(1009.02793516,51.62165901)(1009.01792725,51.68166311)
\curveto(1009.00793518,52.57165806)(1009.19293499,53.31165732)(1009.57292725,53.90166311)
\curveto(1009.95293423,54.49165614)(1010.44793374,54.91665572)(1011.05792725,55.17666311)
\curveto(1011.15793303,55.22665541)(1011.25793293,55.26665537)(1011.35792725,55.29666311)
\curveto(1011.45793273,55.32665531)(1011.56293262,55.36165527)(1011.67292725,55.40166311)
\curveto(1011.7829324,55.4316552)(1011.90293228,55.45665518)(1012.03292725,55.47666311)
\curveto(1012.15293203,55.49665514)(1012.27793191,55.52165511)(1012.40792725,55.55166311)
\curveto(1012.45793173,55.56165507)(1012.51293167,55.56165507)(1012.57292725,55.55166311)
\curveto(1012.62293156,55.55165508)(1012.67293151,55.55665508)(1012.72292725,55.56666311)
\moveto(1013.57792725,54.23166311)
\curveto(1013.50793068,54.25165638)(1013.42793076,54.25665638)(1013.33792725,54.24666311)
\lineto(1013.08292725,54.24666311)
\curveto(1012.69293149,54.24665639)(1012.36293182,54.21165642)(1012.09292725,54.14166311)
\curveto(1012.01293217,54.11165652)(1011.93293225,54.08665655)(1011.85292725,54.06666311)
\curveto(1011.77293241,54.04665659)(1011.69793249,54.02165661)(1011.62792725,53.99166311)
\curveto(1010.97793321,53.71165692)(1010.52793366,53.26665737)(1010.27792725,52.65666311)
\curveto(1010.24793394,52.58665805)(1010.22793396,52.51165812)(1010.21792725,52.43166311)
\lineto(1010.15792725,52.19166311)
\curveto(1010.13793405,52.11165852)(1010.12793406,52.02665861)(1010.12792725,51.93666311)
\lineto(1010.12792725,51.66666311)
\lineto(1010.17292725,51.39666311)
\curveto(1010.19293399,51.29665934)(1010.21793397,51.20165943)(1010.24792725,51.11166311)
\curveto(1010.26793392,51.0316596)(1010.29793389,50.95165968)(1010.33792725,50.87166311)
\curveto(1010.35793383,50.80165983)(1010.3879338,50.7366599)(1010.42792725,50.67666311)
\curveto(1010.46793372,50.61666002)(1010.50793368,50.56166007)(1010.54792725,50.51166311)
\curveto(1010.71793347,50.27166036)(1010.92293326,50.07666056)(1011.16292725,49.92666311)
\curveto(1011.40293278,49.77666086)(1011.6829325,49.64666099)(1012.00292725,49.53666311)
\curveto(1012.10293208,49.50666113)(1012.20793198,49.48666115)(1012.31792725,49.47666311)
\curveto(1012.41793177,49.46666117)(1012.52293166,49.45166118)(1012.63292725,49.43166311)
\curveto(1012.67293151,49.42166121)(1012.73793145,49.41666122)(1012.82792725,49.41666311)
\curveto(1012.85793133,49.40666123)(1012.89293129,49.40166123)(1012.93292725,49.40166311)
\curveto(1012.97293121,49.41166122)(1013.01793117,49.41666122)(1013.06792725,49.41666311)
\lineto(1013.36792725,49.41666311)
\curveto(1013.46793072,49.41666122)(1013.55793063,49.42666121)(1013.63792725,49.44666311)
\lineto(1013.81792725,49.47666311)
\curveto(1013.91793027,49.49666114)(1014.01793017,49.51166112)(1014.11792725,49.52166311)
\curveto(1014.20792998,49.54166109)(1014.29292989,49.57166106)(1014.37292725,49.61166311)
\curveto(1014.61292957,49.71166092)(1014.83792935,49.82666081)(1015.04792725,49.95666311)
\curveto(1015.25792893,50.09666054)(1015.43292875,50.26666037)(1015.57292725,50.46666311)
\curveto(1015.60292858,50.51666012)(1015.62792856,50.56166007)(1015.64792725,50.60166311)
\curveto(1015.66792852,50.64165999)(1015.69292849,50.68665995)(1015.72292725,50.73666311)
\curveto(1015.77292841,50.81665982)(1015.81792837,50.90165973)(1015.85792725,50.99166311)
\curveto(1015.8879283,51.09165954)(1015.91792827,51.19665944)(1015.94792725,51.30666311)
\curveto(1015.96792822,51.35665928)(1015.97792821,51.40165923)(1015.97792725,51.44166311)
\curveto(1015.96792822,51.49165914)(1015.96792822,51.54165909)(1015.97792725,51.59166311)
\curveto(1015.9879282,51.62165901)(1015.99792819,51.68165895)(1016.00792725,51.77166311)
\curveto(1016.01792817,51.87165876)(1016.01292817,51.94665869)(1015.99292725,51.99666311)
\curveto(1015.9829282,52.0366586)(1015.9829282,52.07665856)(1015.99292725,52.11666311)
\curveto(1015.99292819,52.15665848)(1015.9829282,52.19665844)(1015.96292725,52.23666311)
\curveto(1015.94292824,52.31665832)(1015.92792826,52.39665824)(1015.91792725,52.47666311)
\curveto(1015.89792829,52.55665808)(1015.87292831,52.631658)(1015.84292725,52.70166311)
\curveto(1015.70292848,53.04165759)(1015.50792868,53.31665732)(1015.25792725,53.52666311)
\curveto(1015.00792918,53.7366569)(1014.71292947,53.91165672)(1014.37292725,54.05166311)
\curveto(1014.25292993,54.10165653)(1014.12793006,54.1316565)(1013.99792725,54.14166311)
\curveto(1013.85793033,54.16165647)(1013.71793047,54.19165644)(1013.57792725,54.23166311)
}
}
{
\newrgbcolor{curcolor}{0 0 0}
\pscustom[linestyle=none,fillstyle=solid,fillcolor=curcolor]
{
}
}
{
\newrgbcolor{curcolor}{0 0 0}
\pscustom[linestyle=none,fillstyle=solid,fillcolor=curcolor]
{
\newpath
\moveto(1011.83792725,67.99510061)
\lineto(1012.09292725,67.99510061)
\curveto(1012.17293201,68.0050929)(1012.24793194,68.00009291)(1012.31792725,67.98010061)
\lineto(1012.55792725,67.98010061)
\lineto(1012.72292725,67.98010061)
\curveto(1012.82293136,67.96009295)(1012.92793126,67.95009296)(1013.03792725,67.95010061)
\curveto(1013.13793105,67.95009296)(1013.23793095,67.94009297)(1013.33792725,67.92010061)
\lineto(1013.48792725,67.92010061)
\curveto(1013.62793056,67.89009302)(1013.76793042,67.87009304)(1013.90792725,67.86010061)
\curveto(1014.03793015,67.85009306)(1014.16793002,67.82509308)(1014.29792725,67.78510061)
\curveto(1014.37792981,67.76509314)(1014.46292972,67.74509316)(1014.55292725,67.72510061)
\lineto(1014.79292725,67.66510061)
\lineto(1015.09292725,67.54510061)
\curveto(1015.182929,67.51509339)(1015.27292891,67.48009343)(1015.36292725,67.44010061)
\curveto(1015.5829286,67.34009357)(1015.79792839,67.2050937)(1016.00792725,67.03510061)
\curveto(1016.21792797,66.87509403)(1016.3879278,66.70009421)(1016.51792725,66.51010061)
\curveto(1016.55792763,66.46009445)(1016.59792759,66.40009451)(1016.63792725,66.33010061)
\curveto(1016.66792752,66.27009464)(1016.70292748,66.2100947)(1016.74292725,66.15010061)
\curveto(1016.79292739,66.07009484)(1016.83292735,65.97509493)(1016.86292725,65.86510061)
\curveto(1016.89292729,65.75509515)(1016.92292726,65.65009526)(1016.95292725,65.55010061)
\curveto(1016.99292719,65.44009547)(1017.01792717,65.33009558)(1017.02792725,65.22010061)
\curveto(1017.03792715,65.1100958)(1017.05292713,64.99509591)(1017.07292725,64.87510061)
\curveto(1017.0829271,64.83509607)(1017.0829271,64.79009612)(1017.07292725,64.74010061)
\curveto(1017.07292711,64.70009621)(1017.07792711,64.66009625)(1017.08792725,64.62010061)
\curveto(1017.09792709,64.58009633)(1017.10292708,64.52509638)(1017.10292725,64.45510061)
\curveto(1017.10292708,64.38509652)(1017.09792709,64.33509657)(1017.08792725,64.30510061)
\curveto(1017.06792712,64.25509665)(1017.06292712,64.2100967)(1017.07292725,64.17010061)
\curveto(1017.0829271,64.13009678)(1017.0829271,64.09509681)(1017.07292725,64.06510061)
\lineto(1017.07292725,63.97510061)
\curveto(1017.05292713,63.91509699)(1017.03792715,63.85009706)(1017.02792725,63.78010061)
\curveto(1017.02792716,63.72009719)(1017.02292716,63.65509725)(1017.01292725,63.58510061)
\curveto(1016.96292722,63.41509749)(1016.91292727,63.25509765)(1016.86292725,63.10510061)
\curveto(1016.81292737,62.95509795)(1016.74792744,62.8100981)(1016.66792725,62.67010061)
\curveto(1016.62792756,62.62009829)(1016.59792759,62.56509834)(1016.57792725,62.50510061)
\curveto(1016.54792764,62.45509845)(1016.51292767,62.4050985)(1016.47292725,62.35510061)
\curveto(1016.29292789,62.11509879)(1016.07292811,61.91509899)(1015.81292725,61.75510061)
\curveto(1015.55292863,61.59509931)(1015.26792892,61.45509945)(1014.95792725,61.33510061)
\curveto(1014.81792937,61.27509963)(1014.67792951,61.23009968)(1014.53792725,61.20010061)
\curveto(1014.3879298,61.17009974)(1014.23292995,61.13509977)(1014.07292725,61.09510061)
\curveto(1013.96293022,61.07509983)(1013.85293033,61.06009985)(1013.74292725,61.05010061)
\curveto(1013.63293055,61.04009987)(1013.52293066,61.02509988)(1013.41292725,61.00510061)
\curveto(1013.37293081,60.99509991)(1013.33293085,60.99009992)(1013.29292725,60.99010061)
\curveto(1013.25293093,61.00009991)(1013.21293097,61.00009991)(1013.17292725,60.99010061)
\curveto(1013.12293106,60.98009993)(1013.07293111,60.97509993)(1013.02292725,60.97510061)
\lineto(1012.85792725,60.97510061)
\curveto(1012.80793138,60.95509995)(1012.75793143,60.95009996)(1012.70792725,60.96010061)
\curveto(1012.64793154,60.97009994)(1012.59293159,60.97009994)(1012.54292725,60.96010061)
\curveto(1012.50293168,60.95009996)(1012.45793173,60.95009996)(1012.40792725,60.96010061)
\curveto(1012.35793183,60.97009994)(1012.30793188,60.96509994)(1012.25792725,60.94510061)
\curveto(1012.187932,60.92509998)(1012.11293207,60.92009999)(1012.03292725,60.93010061)
\curveto(1011.94293224,60.94009997)(1011.85793233,60.94509996)(1011.77792725,60.94510061)
\curveto(1011.6879325,60.94509996)(1011.5879326,60.94009997)(1011.47792725,60.93010061)
\curveto(1011.35793283,60.92009999)(1011.25793293,60.92509998)(1011.17792725,60.94510061)
\lineto(1010.89292725,60.94510061)
\lineto(1010.26292725,60.99010061)
\curveto(1010.16293402,61.00009991)(1010.06793412,61.0100999)(1009.97792725,61.02010061)
\lineto(1009.67792725,61.05010061)
\curveto(1009.62793456,61.07009984)(1009.57793461,61.07509983)(1009.52792725,61.06510061)
\curveto(1009.46793472,61.06509984)(1009.41293477,61.07509983)(1009.36292725,61.09510061)
\curveto(1009.19293499,61.14509976)(1009.02793516,61.18509972)(1008.86792725,61.21510061)
\curveto(1008.69793549,61.24509966)(1008.53793565,61.29509961)(1008.38792725,61.36510061)
\curveto(1007.92793626,61.55509935)(1007.55293663,61.77509913)(1007.26292725,62.02510061)
\curveto(1006.97293721,62.28509862)(1006.72793746,62.64509826)(1006.52792725,63.10510061)
\curveto(1006.47793771,63.23509767)(1006.44293774,63.36509754)(1006.42292725,63.49510061)
\curveto(1006.40293778,63.63509727)(1006.37793781,63.77509713)(1006.34792725,63.91510061)
\curveto(1006.33793785,63.98509692)(1006.33293785,64.05009686)(1006.33292725,64.11010061)
\curveto(1006.33293785,64.17009674)(1006.32793786,64.23509667)(1006.31792725,64.30510061)
\curveto(1006.29793789,65.13509577)(1006.44793774,65.8050951)(1006.76792725,66.31510061)
\curveto(1007.07793711,66.82509408)(1007.51793667,67.2050937)(1008.08792725,67.45510061)
\curveto(1008.20793598,67.5050934)(1008.33293585,67.55009336)(1008.46292725,67.59010061)
\curveto(1008.59293559,67.63009328)(1008.72793546,67.67509323)(1008.86792725,67.72510061)
\curveto(1008.94793524,67.74509316)(1009.03293515,67.76009315)(1009.12292725,67.77010061)
\lineto(1009.36292725,67.83010061)
\curveto(1009.47293471,67.86009305)(1009.5829346,67.87509303)(1009.69292725,67.87510061)
\curveto(1009.80293438,67.88509302)(1009.91293427,67.90009301)(1010.02292725,67.92010061)
\curveto(1010.07293411,67.94009297)(1010.11793407,67.94509296)(1010.15792725,67.93510061)
\curveto(1010.19793399,67.93509297)(1010.23793395,67.94009297)(1010.27792725,67.95010061)
\curveto(1010.32793386,67.96009295)(1010.3829338,67.96009295)(1010.44292725,67.95010061)
\curveto(1010.49293369,67.95009296)(1010.54293364,67.95509295)(1010.59292725,67.96510061)
\lineto(1010.72792725,67.96510061)
\curveto(1010.7879334,67.98509292)(1010.85793333,67.98509292)(1010.93792725,67.96510061)
\curveto(1011.00793318,67.95509295)(1011.07293311,67.96009295)(1011.13292725,67.98010061)
\curveto(1011.16293302,67.99009292)(1011.20293298,67.99509291)(1011.25292725,67.99510061)
\lineto(1011.37292725,67.99510061)
\lineto(1011.83792725,67.99510061)
\moveto(1014.16292725,66.45010061)
\curveto(1013.84293034,66.55009436)(1013.47793071,66.6100943)(1013.06792725,66.63010061)
\curveto(1012.65793153,66.65009426)(1012.24793194,66.66009425)(1011.83792725,66.66010061)
\curveto(1011.40793278,66.66009425)(1010.9879332,66.65009426)(1010.57792725,66.63010061)
\curveto(1010.16793402,66.6100943)(1009.7829344,66.56509434)(1009.42292725,66.49510061)
\curveto(1009.06293512,66.42509448)(1008.74293544,66.31509459)(1008.46292725,66.16510061)
\curveto(1008.17293601,66.02509488)(1007.93793625,65.83009508)(1007.75792725,65.58010061)
\curveto(1007.64793654,65.42009549)(1007.56793662,65.24009567)(1007.51792725,65.04010061)
\curveto(1007.45793673,64.84009607)(1007.42793676,64.59509631)(1007.42792725,64.30510061)
\curveto(1007.44793674,64.28509662)(1007.45793673,64.25009666)(1007.45792725,64.20010061)
\curveto(1007.44793674,64.15009676)(1007.44793674,64.1100968)(1007.45792725,64.08010061)
\curveto(1007.47793671,64.00009691)(1007.49793669,63.92509698)(1007.51792725,63.85510061)
\curveto(1007.52793666,63.79509711)(1007.54793664,63.73009718)(1007.57792725,63.66010061)
\curveto(1007.69793649,63.39009752)(1007.86793632,63.17009774)(1008.08792725,63.00010061)
\curveto(1008.29793589,62.84009807)(1008.54293564,62.7050982)(1008.82292725,62.59510061)
\curveto(1008.93293525,62.54509836)(1009.05293513,62.5050984)(1009.18292725,62.47510061)
\curveto(1009.30293488,62.45509845)(1009.42793476,62.43009848)(1009.55792725,62.40010061)
\curveto(1009.60793458,62.38009853)(1009.66293452,62.37009854)(1009.72292725,62.37010061)
\curveto(1009.77293441,62.37009854)(1009.82293436,62.36509854)(1009.87292725,62.35510061)
\curveto(1009.96293422,62.34509856)(1010.05793413,62.33509857)(1010.15792725,62.32510061)
\curveto(1010.24793394,62.31509859)(1010.34293384,62.3050986)(1010.44292725,62.29510061)
\curveto(1010.52293366,62.29509861)(1010.60793358,62.29009862)(1010.69792725,62.28010061)
\lineto(1010.93792725,62.28010061)
\lineto(1011.11792725,62.28010061)
\curveto(1011.14793304,62.27009864)(1011.182933,62.26509864)(1011.22292725,62.26510061)
\lineto(1011.35792725,62.26510061)
\lineto(1011.80792725,62.26510061)
\curveto(1011.8879323,62.26509864)(1011.97293221,62.26009865)(1012.06292725,62.25010061)
\curveto(1012.14293204,62.25009866)(1012.21793197,62.26009865)(1012.28792725,62.28010061)
\lineto(1012.55792725,62.28010061)
\curveto(1012.57793161,62.28009863)(1012.60793158,62.27509863)(1012.64792725,62.26510061)
\curveto(1012.67793151,62.26509864)(1012.70293148,62.27009864)(1012.72292725,62.28010061)
\curveto(1012.82293136,62.29009862)(1012.92293126,62.29509861)(1013.02292725,62.29510061)
\curveto(1013.11293107,62.3050986)(1013.21293097,62.31509859)(1013.32292725,62.32510061)
\curveto(1013.44293074,62.35509855)(1013.56793062,62.37009854)(1013.69792725,62.37010061)
\curveto(1013.81793037,62.38009853)(1013.93293025,62.4050985)(1014.04292725,62.44510061)
\curveto(1014.34292984,62.52509838)(1014.60792958,62.6100983)(1014.83792725,62.70010061)
\curveto(1015.06792912,62.80009811)(1015.2829289,62.94509796)(1015.48292725,63.13510061)
\curveto(1015.6829285,63.34509756)(1015.83292835,63.6100973)(1015.93292725,63.93010061)
\curveto(1015.95292823,63.97009694)(1015.96292822,64.0050969)(1015.96292725,64.03510061)
\curveto(1015.95292823,64.07509683)(1015.95792823,64.12009679)(1015.97792725,64.17010061)
\curveto(1015.9879282,64.2100967)(1015.99792819,64.28009663)(1016.00792725,64.38010061)
\curveto(1016.01792817,64.49009642)(1016.01292817,64.57509633)(1015.99292725,64.63510061)
\curveto(1015.97292821,64.7050962)(1015.96292822,64.77509613)(1015.96292725,64.84510061)
\curveto(1015.95292823,64.91509599)(1015.93792825,64.98009593)(1015.91792725,65.04010061)
\curveto(1015.85792833,65.24009567)(1015.77292841,65.42009549)(1015.66292725,65.58010061)
\curveto(1015.64292854,65.6100953)(1015.62292856,65.63509527)(1015.60292725,65.65510061)
\lineto(1015.54292725,65.71510061)
\curveto(1015.52292866,65.75509515)(1015.4829287,65.8050951)(1015.42292725,65.86510061)
\curveto(1015.2829289,65.96509494)(1015.15292903,66.05009486)(1015.03292725,66.12010061)
\curveto(1014.91292927,66.19009472)(1014.76792942,66.26009465)(1014.59792725,66.33010061)
\curveto(1014.52792966,66.36009455)(1014.45792973,66.38009453)(1014.38792725,66.39010061)
\curveto(1014.31792987,66.4100945)(1014.24292994,66.43009448)(1014.16292725,66.45010061)
}
}
{
\newrgbcolor{curcolor}{0 0 0}
\pscustom[linestyle=none,fillstyle=solid,fillcolor=curcolor]
{
\newpath
\moveto(1011.32792725,76.28470998)
\curveto(1011.40793278,76.28470235)(1011.4879327,76.28970234)(1011.56792725,76.29970998)
\curveto(1011.64793254,76.30970232)(1011.72293246,76.30470233)(1011.79292725,76.28470998)
\curveto(1011.83293235,76.26470237)(1011.87793231,76.25970237)(1011.92792725,76.26970998)
\curveto(1011.96793222,76.27970235)(1012.00793218,76.27970235)(1012.04792725,76.26970998)
\lineto(1012.19792725,76.26970998)
\curveto(1012.2879319,76.25970237)(1012.37793181,76.25470238)(1012.46792725,76.25470998)
\curveto(1012.54793164,76.25470238)(1012.62793156,76.24970238)(1012.70792725,76.23970998)
\lineto(1012.94792725,76.20970998)
\curveto(1013.01793117,76.19970243)(1013.09293109,76.18970244)(1013.17292725,76.17970998)
\curveto(1013.21293097,76.16970246)(1013.25293093,76.16470247)(1013.29292725,76.16470998)
\curveto(1013.33293085,76.16470247)(1013.37793081,76.15970247)(1013.42792725,76.14970998)
\curveto(1013.56793062,76.10970252)(1013.70793048,76.07970255)(1013.84792725,76.05970998)
\curveto(1013.9879302,76.04970258)(1014.12293006,76.01970261)(1014.25292725,75.96970998)
\curveto(1014.42292976,75.91970271)(1014.5879296,75.86470277)(1014.74792725,75.80470998)
\curveto(1014.90792928,75.75470288)(1015.06292912,75.69470294)(1015.21292725,75.62470998)
\curveto(1015.27292891,75.60470303)(1015.33292885,75.57470306)(1015.39292725,75.53470998)
\lineto(1015.54292725,75.44470998)
\curveto(1015.86292832,75.24470339)(1016.12792806,75.0297036)(1016.33792725,74.79970998)
\curveto(1016.54792764,74.56970406)(1016.72792746,74.27470436)(1016.87792725,73.91470998)
\curveto(1016.92792726,73.79470484)(1016.96292722,73.66470497)(1016.98292725,73.52470998)
\curveto(1017.00292718,73.39470524)(1017.02792716,73.25970537)(1017.05792725,73.11970998)
\curveto(1017.06792712,73.05970557)(1017.07292711,72.99970563)(1017.07292725,72.93970998)
\curveto(1017.07292711,72.87970575)(1017.07792711,72.81470582)(1017.08792725,72.74470998)
\curveto(1017.09792709,72.71470592)(1017.09792709,72.66470597)(1017.08792725,72.59470998)
\lineto(1017.08792725,72.44470998)
\lineto(1017.08792725,72.29470998)
\curveto(1017.06792712,72.21470642)(1017.05292713,72.1297065)(1017.04292725,72.03970998)
\curveto(1017.04292714,71.95970667)(1017.03292715,71.88470675)(1017.01292725,71.81470998)
\curveto(1017.00292718,71.77470686)(1016.99792719,71.73970689)(1016.99792725,71.70970998)
\curveto(1017.00792718,71.68970694)(1017.00292718,71.66470697)(1016.98292725,71.63470998)
\lineto(1016.92292725,71.36470998)
\curveto(1016.89292729,71.27470736)(1016.86292732,71.18970744)(1016.83292725,71.10970998)
\curveto(1016.59292759,70.5297081)(1016.22292796,70.09470854)(1015.72292725,69.80470998)
\curveto(1015.59292859,69.72470891)(1015.45792873,69.65970897)(1015.31792725,69.60970998)
\curveto(1015.17792901,69.56970906)(1015.02792916,69.52470911)(1014.86792725,69.47470998)
\curveto(1014.7879294,69.45470918)(1014.70792948,69.44970918)(1014.62792725,69.45970998)
\curveto(1014.54792964,69.47970915)(1014.49292969,69.51470912)(1014.46292725,69.56470998)
\curveto(1014.44292974,69.59470904)(1014.42792976,69.64970898)(1014.41792725,69.72970998)
\curveto(1014.39792979,69.80970882)(1014.3879298,69.89470874)(1014.38792725,69.98470998)
\curveto(1014.37792981,70.07470856)(1014.37792981,70.15970847)(1014.38792725,70.23970998)
\curveto(1014.39792979,70.3297083)(1014.40792978,70.39970823)(1014.41792725,70.44970998)
\curveto(1014.42792976,70.46970816)(1014.44292974,70.49470814)(1014.46292725,70.52470998)
\curveto(1014.4829297,70.56470807)(1014.50292968,70.59470804)(1014.52292725,70.61470998)
\curveto(1014.60292958,70.67470796)(1014.69792949,70.71970791)(1014.80792725,70.74970998)
\curveto(1014.91792927,70.78970784)(1015.01792917,70.8347078)(1015.10792725,70.88470998)
\curveto(1015.49792869,71.1347075)(1015.76792842,71.50470713)(1015.91792725,71.99470998)
\curveto(1015.93792825,72.06470657)(1015.95292823,72.1347065)(1015.96292725,72.20470998)
\curveto(1015.96292822,72.28470635)(1015.97292821,72.36470627)(1015.99292725,72.44470998)
\curveto(1016.00292818,72.48470615)(1016.00792818,72.53970609)(1016.00792725,72.60970998)
\curveto(1016.00792818,72.68970594)(1016.00292818,72.74470589)(1015.99292725,72.77470998)
\curveto(1015.9829282,72.80470583)(1015.97792821,72.8347058)(1015.97792725,72.86470998)
\lineto(1015.97792725,72.96970998)
\curveto(1015.95792823,73.04970558)(1015.93792825,73.12470551)(1015.91792725,73.19470998)
\curveto(1015.89792829,73.27470536)(1015.87292831,73.34970528)(1015.84292725,73.41970998)
\curveto(1015.69292849,73.76970486)(1015.47792871,74.03970459)(1015.19792725,74.22970998)
\curveto(1014.91792927,74.41970421)(1014.59292959,74.57470406)(1014.22292725,74.69470998)
\curveto(1014.14293004,74.72470391)(1014.06793012,74.74470389)(1013.99792725,74.75470998)
\curveto(1013.92793026,74.77470386)(1013.85293033,74.79470384)(1013.77292725,74.81470998)
\curveto(1013.6829305,74.8347038)(1013.5879306,74.84970378)(1013.48792725,74.85970998)
\curveto(1013.37793081,74.87970375)(1013.27293091,74.89970373)(1013.17292725,74.91970998)
\curveto(1013.12293106,74.9297037)(1013.07293111,74.9347037)(1013.02292725,74.93470998)
\curveto(1012.96293122,74.94470369)(1012.90793128,74.94970368)(1012.85792725,74.94970998)
\curveto(1012.79793139,74.96970366)(1012.72293146,74.97970365)(1012.63292725,74.97970998)
\curveto(1012.53293165,74.97970365)(1012.45293173,74.96970366)(1012.39292725,74.94970998)
\curveto(1012.30293188,74.91970371)(1012.26293192,74.86970376)(1012.27292725,74.79970998)
\curveto(1012.2829319,74.73970389)(1012.31293187,74.68470395)(1012.36292725,74.63470998)
\curveto(1012.41293177,74.55470408)(1012.47293171,74.48470415)(1012.54292725,74.42470998)
\curveto(1012.61293157,74.37470426)(1012.67293151,74.30970432)(1012.72292725,74.22970998)
\curveto(1012.83293135,74.06970456)(1012.93293125,73.90470473)(1013.02292725,73.73470998)
\curveto(1013.10293108,73.56470507)(1013.17293101,73.36970526)(1013.23292725,73.14970998)
\curveto(1013.26293092,73.04970558)(1013.27793091,72.94970568)(1013.27792725,72.84970998)
\curveto(1013.27793091,72.75970587)(1013.2879309,72.65970597)(1013.30792725,72.54970998)
\lineto(1013.30792725,72.39970998)
\curveto(1013.2879309,72.34970628)(1013.2829309,72.29970633)(1013.29292725,72.24970998)
\curveto(1013.30293088,72.20970642)(1013.30293088,72.16970646)(1013.29292725,72.12970998)
\curveto(1013.2829309,72.09970653)(1013.27793091,72.05470658)(1013.27792725,71.99470998)
\curveto(1013.26793092,71.9347067)(1013.25793093,71.86970676)(1013.24792725,71.79970998)
\lineto(1013.21792725,71.61970998)
\curveto(1013.09793109,71.16970746)(1012.93293125,70.78970784)(1012.72292725,70.47970998)
\curveto(1012.53293165,70.20970842)(1012.30293188,69.97970865)(1012.03292725,69.78970998)
\curveto(1011.75293243,69.60970902)(1011.43793275,69.46470917)(1011.08792725,69.35470998)
\lineto(1010.87792725,69.29470998)
\curveto(1010.79793339,69.28470935)(1010.71793347,69.26970936)(1010.63792725,69.24970998)
\curveto(1010.60793358,69.23970939)(1010.57793361,69.2347094)(1010.54792725,69.23470998)
\curveto(1010.51793367,69.2347094)(1010.4879337,69.2297094)(1010.45792725,69.21970998)
\curveto(1010.39793379,69.20970942)(1010.33793385,69.20470943)(1010.27792725,69.20470998)
\curveto(1010.20793398,69.20470943)(1010.14793404,69.19470944)(1010.09792725,69.17470998)
\lineto(1009.91792725,69.17470998)
\curveto(1009.86793432,69.16470947)(1009.79793439,69.15970947)(1009.70792725,69.15970998)
\curveto(1009.61793457,69.15970947)(1009.54793464,69.16970946)(1009.49792725,69.18970998)
\lineto(1009.33292725,69.18970998)
\curveto(1009.25293493,69.20970942)(1009.17793501,69.21970941)(1009.10792725,69.21970998)
\curveto(1009.03793515,69.2297094)(1008.96793522,69.24470939)(1008.89792725,69.26470998)
\curveto(1008.69793549,69.32470931)(1008.50793568,69.38470925)(1008.32792725,69.44470998)
\curveto(1008.14793604,69.51470912)(1007.97793621,69.60470903)(1007.81792725,69.71470998)
\curveto(1007.74793644,69.75470888)(1007.6829365,69.79470884)(1007.62292725,69.83470998)
\lineto(1007.44292725,69.98470998)
\curveto(1007.43293675,70.00470863)(1007.41793677,70.02470861)(1007.39792725,70.04470998)
\curveto(1007.26793692,70.1347085)(1007.15793703,70.24470839)(1007.06792725,70.37470998)
\curveto(1006.86793732,70.634708)(1006.71293747,70.89970773)(1006.60292725,71.16970998)
\curveto(1006.56293762,71.24970738)(1006.53293765,71.3297073)(1006.51292725,71.40970998)
\curveto(1006.4829377,71.49970713)(1006.45793773,71.58970704)(1006.43792725,71.67970998)
\curveto(1006.40793778,71.77970685)(1006.3879378,71.87970675)(1006.37792725,71.97970998)
\curveto(1006.36793782,72.07970655)(1006.35293783,72.18470645)(1006.33292725,72.29470998)
\curveto(1006.32293786,72.32470631)(1006.32293786,72.36470627)(1006.33292725,72.41470998)
\curveto(1006.34293784,72.47470616)(1006.33793785,72.51470612)(1006.31792725,72.53470998)
\curveto(1006.29793789,73.25470538)(1006.41293777,73.85470478)(1006.66292725,74.33470998)
\curveto(1006.91293727,74.81470382)(1007.25293693,75.18970344)(1007.68292725,75.45970998)
\curveto(1007.82293636,75.54970308)(1007.96793622,75.629703)(1008.11792725,75.69970998)
\curveto(1008.26793592,75.76970286)(1008.42793576,75.83970279)(1008.59792725,75.90970998)
\curveto(1008.73793545,75.95970267)(1008.8879353,75.99970263)(1009.04792725,76.02970998)
\curveto(1009.20793498,76.05970257)(1009.36793482,76.09470254)(1009.52792725,76.13470998)
\curveto(1009.57793461,76.15470248)(1009.63293455,76.16470247)(1009.69292725,76.16470998)
\curveto(1009.74293444,76.16470247)(1009.79293439,76.16970246)(1009.84292725,76.17970998)
\curveto(1009.90293428,76.19970243)(1009.96793422,76.20970242)(1010.03792725,76.20970998)
\curveto(1010.09793409,76.20970242)(1010.15293403,76.21970241)(1010.20292725,76.23970998)
\lineto(1010.36792725,76.23970998)
\curveto(1010.41793377,76.25970237)(1010.46793372,76.26470237)(1010.51792725,76.25470998)
\curveto(1010.56793362,76.24470239)(1010.61793357,76.24970238)(1010.66792725,76.26970998)
\curveto(1010.6879335,76.26970236)(1010.71293347,76.26470237)(1010.74292725,76.25470998)
\curveto(1010.77293341,76.25470238)(1010.79793339,76.25970237)(1010.81792725,76.26970998)
\curveto(1010.84793334,76.27970235)(1010.8829333,76.27970235)(1010.92292725,76.26970998)
\curveto(1010.96293322,76.26970236)(1011.00293318,76.27470236)(1011.04292725,76.28470998)
\curveto(1011.0829331,76.29470234)(1011.12793306,76.29470234)(1011.17792725,76.28470998)
\lineto(1011.32792725,76.28470998)
\moveto(1010.02292725,74.78470998)
\curveto(1009.97293421,74.79470384)(1009.91293427,74.79970383)(1009.84292725,74.79970998)
\curveto(1009.77293441,74.79970383)(1009.71293447,74.79470384)(1009.66292725,74.78470998)
\curveto(1009.61293457,74.77470386)(1009.53793465,74.76970386)(1009.43792725,74.76970998)
\curveto(1009.35793483,74.74970388)(1009.2829349,74.7297039)(1009.21292725,74.70970998)
\curveto(1009.14293504,74.69970393)(1009.07293511,74.68470395)(1009.00292725,74.66470998)
\curveto(1008.57293561,74.52470411)(1008.23793595,74.3297043)(1007.99792725,74.07970998)
\curveto(1007.75793643,73.83970479)(1007.57793661,73.49470514)(1007.45792725,73.04470998)
\curveto(1007.43793675,72.95470568)(1007.42793676,72.85470578)(1007.42792725,72.74470998)
\lineto(1007.42792725,72.41470998)
\curveto(1007.44793674,72.39470624)(1007.45793673,72.35970627)(1007.45792725,72.30970998)
\curveto(1007.44793674,72.25970637)(1007.44793674,72.21470642)(1007.45792725,72.17470998)
\curveto(1007.47793671,72.09470654)(1007.49793669,72.01970661)(1007.51792725,71.94970998)
\lineto(1007.57792725,71.73970998)
\curveto(1007.70793648,71.44970718)(1007.8879363,71.21970741)(1008.11792725,71.04970998)
\curveto(1008.33793585,70.87970775)(1008.59793559,70.74470789)(1008.89792725,70.64470998)
\curveto(1008.9879352,70.61470802)(1009.0829351,70.58970804)(1009.18292725,70.56970998)
\curveto(1009.27293491,70.55970807)(1009.36793482,70.54470809)(1009.46792725,70.52470998)
\lineto(1009.60292725,70.52470998)
\curveto(1009.71293447,70.49470814)(1009.85293433,70.48470815)(1010.02292725,70.49470998)
\curveto(1010.182934,70.51470812)(1010.31293387,70.5347081)(1010.41292725,70.55470998)
\curveto(1010.47293371,70.57470806)(1010.53293365,70.58970804)(1010.59292725,70.59970998)
\curveto(1010.64293354,70.60970802)(1010.69293349,70.62470801)(1010.74292725,70.64470998)
\curveto(1010.94293324,70.72470791)(1011.13293305,70.81970781)(1011.31292725,70.92970998)
\curveto(1011.49293269,71.04970758)(1011.63793255,71.18970744)(1011.74792725,71.34970998)
\curveto(1011.79793239,71.39970723)(1011.83793235,71.45470718)(1011.86792725,71.51470998)
\curveto(1011.89793229,71.57470706)(1011.93293225,71.634707)(1011.97292725,71.69470998)
\curveto(1012.05293213,71.84470679)(1012.11793207,72.0297066)(1012.16792725,72.24970998)
\curveto(1012.187932,72.29970633)(1012.19293199,72.33970629)(1012.18292725,72.36970998)
\curveto(1012.17293201,72.40970622)(1012.17793201,72.45470618)(1012.19792725,72.50470998)
\curveto(1012.20793198,72.54470609)(1012.21293197,72.59970603)(1012.21292725,72.66970998)
\curveto(1012.21293197,72.73970589)(1012.20793198,72.79970583)(1012.19792725,72.84970998)
\curveto(1012.17793201,72.94970568)(1012.16293202,73.04470559)(1012.15292725,73.13470998)
\curveto(1012.13293205,73.22470541)(1012.10293208,73.31470532)(1012.06292725,73.40470998)
\curveto(1011.84293234,73.94470469)(1011.44793274,74.33970429)(1010.87792725,74.58970998)
\curveto(1010.77793341,74.63970399)(1010.67793351,74.67470396)(1010.57792725,74.69470998)
\curveto(1010.46793372,74.71470392)(1010.35793383,74.73970389)(1010.24792725,74.76970998)
\curveto(1010.14793404,74.76970386)(1010.07293411,74.77470386)(1010.02292725,74.78470998)
}
}
{
\newrgbcolor{curcolor}{0 0 0}
\pscustom[linestyle=none,fillstyle=solid,fillcolor=curcolor]
{
\newpath
\moveto(1015.28792725,78.63431936)
\lineto(1015.28792725,79.26431936)
\lineto(1015.28792725,79.45931936)
\curveto(1015.2879289,79.52931683)(1015.29792889,79.58931677)(1015.31792725,79.63931936)
\curveto(1015.35792883,79.70931665)(1015.39792879,79.7593166)(1015.43792725,79.78931936)
\curveto(1015.4879287,79.82931653)(1015.55292863,79.84931651)(1015.63292725,79.84931936)
\curveto(1015.71292847,79.8593165)(1015.79792839,79.86431649)(1015.88792725,79.86431936)
\lineto(1016.60792725,79.86431936)
\curveto(1017.0879271,79.86431649)(1017.49792669,79.80431655)(1017.83792725,79.68431936)
\curveto(1018.17792601,79.56431679)(1018.45292573,79.36931699)(1018.66292725,79.09931936)
\curveto(1018.71292547,79.02931733)(1018.75792543,78.9593174)(1018.79792725,78.88931936)
\curveto(1018.84792534,78.82931753)(1018.89292529,78.7543176)(1018.93292725,78.66431936)
\curveto(1018.94292524,78.64431771)(1018.95292523,78.61431774)(1018.96292725,78.57431936)
\curveto(1018.9829252,78.53431782)(1018.9879252,78.48931787)(1018.97792725,78.43931936)
\curveto(1018.94792524,78.34931801)(1018.87292531,78.29431806)(1018.75292725,78.27431936)
\curveto(1018.64292554,78.2543181)(1018.54792564,78.26931809)(1018.46792725,78.31931936)
\curveto(1018.39792579,78.34931801)(1018.33292585,78.39431796)(1018.27292725,78.45431936)
\curveto(1018.22292596,78.52431783)(1018.17292601,78.58931777)(1018.12292725,78.64931936)
\curveto(1018.07292611,78.71931764)(1017.99792619,78.77931758)(1017.89792725,78.82931936)
\curveto(1017.80792638,78.88931747)(1017.71792647,78.93931742)(1017.62792725,78.97931936)
\curveto(1017.59792659,78.99931736)(1017.53792665,79.02431733)(1017.44792725,79.05431936)
\curveto(1017.36792682,79.08431727)(1017.29792689,79.08931727)(1017.23792725,79.06931936)
\curveto(1017.09792709,79.03931732)(1017.00792718,78.97931738)(1016.96792725,78.88931936)
\curveto(1016.93792725,78.80931755)(1016.92292726,78.71931764)(1016.92292725,78.61931936)
\curveto(1016.92292726,78.51931784)(1016.89792729,78.43431792)(1016.84792725,78.36431936)
\curveto(1016.77792741,78.27431808)(1016.63792755,78.22931813)(1016.42792725,78.22931936)
\lineto(1015.87292725,78.22931936)
\lineto(1015.64792725,78.22931936)
\curveto(1015.56792862,78.23931812)(1015.50292868,78.2593181)(1015.45292725,78.28931936)
\curveto(1015.37292881,78.34931801)(1015.32792886,78.41931794)(1015.31792725,78.49931936)
\curveto(1015.30792888,78.51931784)(1015.30292888,78.53931782)(1015.30292725,78.55931936)
\curveto(1015.30292888,78.58931777)(1015.29792889,78.61431774)(1015.28792725,78.63431936)
}
}
{
\newrgbcolor{curcolor}{0 0 0}
\pscustom[linestyle=none,fillstyle=solid,fillcolor=curcolor]
{
}
}
{
\newrgbcolor{curcolor}{0 0 0}
\pscustom[linestyle=none,fillstyle=solid,fillcolor=curcolor]
{
\newpath
\moveto(1006.31792725,89.26463186)
\curveto(1006.30793788,89.95462722)(1006.42793776,90.55462662)(1006.67792725,91.06463186)
\curveto(1006.92793726,91.58462559)(1007.26293692,91.9796252)(1007.68292725,92.24963186)
\curveto(1007.76293642,92.29962488)(1007.85293633,92.34462483)(1007.95292725,92.38463186)
\curveto(1008.04293614,92.42462475)(1008.13793605,92.46962471)(1008.23792725,92.51963186)
\curveto(1008.33793585,92.55962462)(1008.43793575,92.58962459)(1008.53792725,92.60963186)
\curveto(1008.63793555,92.62962455)(1008.74293544,92.64962453)(1008.85292725,92.66963186)
\curveto(1008.90293528,92.68962449)(1008.94793524,92.69462448)(1008.98792725,92.68463186)
\curveto(1009.02793516,92.6746245)(1009.07293511,92.6796245)(1009.12292725,92.69963186)
\curveto(1009.17293501,92.70962447)(1009.25793493,92.71462446)(1009.37792725,92.71463186)
\curveto(1009.4879347,92.71462446)(1009.57293461,92.70962447)(1009.63292725,92.69963186)
\curveto(1009.69293449,92.6796245)(1009.75293443,92.66962451)(1009.81292725,92.66963186)
\curveto(1009.87293431,92.6796245)(1009.93293425,92.6746245)(1009.99292725,92.65463186)
\curveto(1010.13293405,92.61462456)(1010.26793392,92.5796246)(1010.39792725,92.54963186)
\curveto(1010.52793366,92.51962466)(1010.65293353,92.4796247)(1010.77292725,92.42963186)
\curveto(1010.91293327,92.36962481)(1011.03793315,92.29962488)(1011.14792725,92.21963186)
\curveto(1011.25793293,92.14962503)(1011.36793282,92.0746251)(1011.47792725,91.99463186)
\lineto(1011.53792725,91.93463186)
\curveto(1011.55793263,91.92462525)(1011.57793261,91.90962527)(1011.59792725,91.88963186)
\curveto(1011.75793243,91.76962541)(1011.90293228,91.63462554)(1012.03292725,91.48463186)
\curveto(1012.16293202,91.33462584)(1012.2879319,91.174626)(1012.40792725,91.00463186)
\curveto(1012.62793156,90.69462648)(1012.83293135,90.39962678)(1013.02292725,90.11963186)
\curveto(1013.16293102,89.88962729)(1013.29793089,89.65962752)(1013.42792725,89.42963186)
\curveto(1013.55793063,89.20962797)(1013.69293049,88.98962819)(1013.83292725,88.76963186)
\curveto(1014.00293018,88.51962866)(1014.18293,88.2796289)(1014.37292725,88.04963186)
\curveto(1014.56292962,87.82962935)(1014.7879294,87.63962954)(1015.04792725,87.47963186)
\curveto(1015.10792908,87.43962974)(1015.16792902,87.40462977)(1015.22792725,87.37463186)
\curveto(1015.27792891,87.34462983)(1015.34292884,87.31462986)(1015.42292725,87.28463186)
\curveto(1015.49292869,87.26462991)(1015.55292863,87.25962992)(1015.60292725,87.26963186)
\curveto(1015.67292851,87.28962989)(1015.72792846,87.32462985)(1015.76792725,87.37463186)
\curveto(1015.79792839,87.42462975)(1015.81792837,87.48462969)(1015.82792725,87.55463186)
\lineto(1015.82792725,87.79463186)
\lineto(1015.82792725,88.54463186)
\lineto(1015.82792725,91.34963186)
\lineto(1015.82792725,92.00963186)
\curveto(1015.82792836,92.09962508)(1015.83292835,92.18462499)(1015.84292725,92.26463186)
\curveto(1015.84292834,92.34462483)(1015.86292832,92.40962477)(1015.90292725,92.45963186)
\curveto(1015.94292824,92.50962467)(1016.01792817,92.54962463)(1016.12792725,92.57963186)
\curveto(1016.22792796,92.61962456)(1016.32792786,92.62962455)(1016.42792725,92.60963186)
\lineto(1016.56292725,92.60963186)
\curveto(1016.63292755,92.58962459)(1016.69292749,92.56962461)(1016.74292725,92.54963186)
\curveto(1016.79292739,92.52962465)(1016.83292735,92.49462468)(1016.86292725,92.44463186)
\curveto(1016.90292728,92.39462478)(1016.92292726,92.32462485)(1016.92292725,92.23463186)
\lineto(1016.92292725,91.96463186)
\lineto(1016.92292725,91.06463186)
\lineto(1016.92292725,87.55463186)
\lineto(1016.92292725,86.48963186)
\curveto(1016.92292726,86.40963077)(1016.92792726,86.31963086)(1016.93792725,86.21963186)
\curveto(1016.93792725,86.11963106)(1016.92792726,86.03463114)(1016.90792725,85.96463186)
\curveto(1016.83792735,85.75463142)(1016.65792753,85.68963149)(1016.36792725,85.76963186)
\curveto(1016.32792786,85.7796314)(1016.29292789,85.7796314)(1016.26292725,85.76963186)
\curveto(1016.22292796,85.76963141)(1016.17792801,85.7796314)(1016.12792725,85.79963186)
\curveto(1016.04792814,85.81963136)(1015.96292822,85.83963134)(1015.87292725,85.85963186)
\curveto(1015.7829284,85.8796313)(1015.69792849,85.90463127)(1015.61792725,85.93463186)
\curveto(1015.12792906,86.09463108)(1014.71292947,86.29463088)(1014.37292725,86.53463186)
\curveto(1014.12293006,86.71463046)(1013.89793029,86.91963026)(1013.69792725,87.14963186)
\curveto(1013.4879307,87.3796298)(1013.29293089,87.61962956)(1013.11292725,87.86963186)
\curveto(1012.93293125,88.12962905)(1012.76293142,88.39462878)(1012.60292725,88.66463186)
\curveto(1012.43293175,88.94462823)(1012.25793193,89.21462796)(1012.07792725,89.47463186)
\curveto(1011.99793219,89.58462759)(1011.92293226,89.68962749)(1011.85292725,89.78963186)
\curveto(1011.7829324,89.89962728)(1011.70793248,90.00962717)(1011.62792725,90.11963186)
\curveto(1011.59793259,90.15962702)(1011.56793262,90.19462698)(1011.53792725,90.22463186)
\curveto(1011.49793269,90.26462691)(1011.46793272,90.30462687)(1011.44792725,90.34463186)
\curveto(1011.33793285,90.48462669)(1011.21293297,90.60962657)(1011.07292725,90.71963186)
\curveto(1011.04293314,90.73962644)(1011.01793317,90.76462641)(1010.99792725,90.79463186)
\curveto(1010.96793322,90.82462635)(1010.93793325,90.84962633)(1010.90792725,90.86963186)
\curveto(1010.80793338,90.94962623)(1010.70793348,91.01462616)(1010.60792725,91.06463186)
\curveto(1010.50793368,91.12462605)(1010.39793379,91.179626)(1010.27792725,91.22963186)
\curveto(1010.20793398,91.25962592)(1010.13293405,91.2796259)(1010.05292725,91.28963186)
\lineto(1009.81292725,91.34963186)
\lineto(1009.72292725,91.34963186)
\curveto(1009.69293449,91.35962582)(1009.66293452,91.36462581)(1009.63292725,91.36463186)
\curveto(1009.56293462,91.38462579)(1009.46793472,91.38962579)(1009.34792725,91.37963186)
\curveto(1009.21793497,91.3796258)(1009.11793507,91.36962581)(1009.04792725,91.34963186)
\curveto(1008.96793522,91.32962585)(1008.89293529,91.30962587)(1008.82292725,91.28963186)
\curveto(1008.74293544,91.2796259)(1008.66293552,91.25962592)(1008.58292725,91.22963186)
\curveto(1008.34293584,91.11962606)(1008.14293604,90.96962621)(1007.98292725,90.77963186)
\curveto(1007.81293637,90.59962658)(1007.67293651,90.3796268)(1007.56292725,90.11963186)
\curveto(1007.54293664,90.04962713)(1007.52793666,89.9796272)(1007.51792725,89.90963186)
\curveto(1007.49793669,89.83962734)(1007.47793671,89.76462741)(1007.45792725,89.68463186)
\curveto(1007.43793675,89.60462757)(1007.42793676,89.49462768)(1007.42792725,89.35463186)
\curveto(1007.42793676,89.22462795)(1007.43793675,89.11962806)(1007.45792725,89.03963186)
\curveto(1007.46793672,88.9796282)(1007.47293671,88.92462825)(1007.47292725,88.87463186)
\curveto(1007.47293671,88.82462835)(1007.4829367,88.7746284)(1007.50292725,88.72463186)
\curveto(1007.54293664,88.62462855)(1007.5829366,88.52962865)(1007.62292725,88.43963186)
\curveto(1007.66293652,88.35962882)(1007.70793648,88.2796289)(1007.75792725,88.19963186)
\curveto(1007.77793641,88.16962901)(1007.80293638,88.13962904)(1007.83292725,88.10963186)
\curveto(1007.86293632,88.08962909)(1007.8879363,88.06462911)(1007.90792725,88.03463186)
\lineto(1007.98292725,87.95963186)
\curveto(1008.00293618,87.92962925)(1008.02293616,87.90462927)(1008.04292725,87.88463186)
\lineto(1008.25292725,87.73463186)
\curveto(1008.31293587,87.69462948)(1008.37793581,87.64962953)(1008.44792725,87.59963186)
\curveto(1008.53793565,87.53962964)(1008.64293554,87.48962969)(1008.76292725,87.44963186)
\curveto(1008.87293531,87.41962976)(1008.9829352,87.38462979)(1009.09292725,87.34463186)
\curveto(1009.20293498,87.30462987)(1009.34793484,87.2796299)(1009.52792725,87.26963186)
\curveto(1009.69793449,87.25962992)(1009.82293436,87.22962995)(1009.90292725,87.17963186)
\curveto(1009.9829342,87.12963005)(1010.02793416,87.05463012)(1010.03792725,86.95463186)
\curveto(1010.04793414,86.85463032)(1010.05293413,86.74463043)(1010.05292725,86.62463186)
\curveto(1010.05293413,86.58463059)(1010.05793413,86.54463063)(1010.06792725,86.50463186)
\curveto(1010.06793412,86.46463071)(1010.06293412,86.42963075)(1010.05292725,86.39963186)
\curveto(1010.03293415,86.34963083)(1010.02293416,86.29963088)(1010.02292725,86.24963186)
\curveto(1010.02293416,86.20963097)(1010.01293417,86.16963101)(1009.99292725,86.12963186)
\curveto(1009.93293425,86.03963114)(1009.79793439,85.99463118)(1009.58792725,85.99463186)
\lineto(1009.46792725,85.99463186)
\curveto(1009.40793478,86.00463117)(1009.34793484,86.00963117)(1009.28792725,86.00963186)
\curveto(1009.21793497,86.01963116)(1009.15293503,86.02963115)(1009.09292725,86.03963186)
\curveto(1008.9829352,86.05963112)(1008.8829353,86.0796311)(1008.79292725,86.09963186)
\curveto(1008.69293549,86.11963106)(1008.59793559,86.14963103)(1008.50792725,86.18963186)
\curveto(1008.43793575,86.20963097)(1008.37793581,86.22963095)(1008.32792725,86.24963186)
\lineto(1008.14792725,86.30963186)
\curveto(1007.8879363,86.42963075)(1007.64293654,86.58463059)(1007.41292725,86.77463186)
\curveto(1007.182937,86.9746302)(1006.99793719,87.18962999)(1006.85792725,87.41963186)
\curveto(1006.77793741,87.52962965)(1006.71293747,87.64462953)(1006.66292725,87.76463186)
\lineto(1006.51292725,88.15463186)
\curveto(1006.46293772,88.26462891)(1006.43293775,88.3796288)(1006.42292725,88.49963186)
\curveto(1006.40293778,88.61962856)(1006.37793781,88.74462843)(1006.34792725,88.87463186)
\curveto(1006.34793784,88.94462823)(1006.34793784,89.00962817)(1006.34792725,89.06963186)
\curveto(1006.33793785,89.12962805)(1006.32793786,89.19462798)(1006.31792725,89.26463186)
}
}
{
\newrgbcolor{curcolor}{0 0 0}
\pscustom[linestyle=none,fillstyle=solid,fillcolor=curcolor]
{
\newpath
\moveto(1011.83792725,101.36424123)
\lineto(1012.09292725,101.36424123)
\curveto(1012.17293201,101.37423353)(1012.24793194,101.36923353)(1012.31792725,101.34924123)
\lineto(1012.55792725,101.34924123)
\lineto(1012.72292725,101.34924123)
\curveto(1012.82293136,101.32923357)(1012.92793126,101.31923358)(1013.03792725,101.31924123)
\curveto(1013.13793105,101.31923358)(1013.23793095,101.30923359)(1013.33792725,101.28924123)
\lineto(1013.48792725,101.28924123)
\curveto(1013.62793056,101.25923364)(1013.76793042,101.23923366)(1013.90792725,101.22924123)
\curveto(1014.03793015,101.21923368)(1014.16793002,101.19423371)(1014.29792725,101.15424123)
\curveto(1014.37792981,101.13423377)(1014.46292972,101.11423379)(1014.55292725,101.09424123)
\lineto(1014.79292725,101.03424123)
\lineto(1015.09292725,100.91424123)
\curveto(1015.182929,100.88423402)(1015.27292891,100.84923405)(1015.36292725,100.80924123)
\curveto(1015.5829286,100.70923419)(1015.79792839,100.57423433)(1016.00792725,100.40424123)
\curveto(1016.21792797,100.24423466)(1016.3879278,100.06923483)(1016.51792725,99.87924123)
\curveto(1016.55792763,99.82923507)(1016.59792759,99.76923513)(1016.63792725,99.69924123)
\curveto(1016.66792752,99.63923526)(1016.70292748,99.57923532)(1016.74292725,99.51924123)
\curveto(1016.79292739,99.43923546)(1016.83292735,99.34423556)(1016.86292725,99.23424123)
\curveto(1016.89292729,99.12423578)(1016.92292726,99.01923588)(1016.95292725,98.91924123)
\curveto(1016.99292719,98.80923609)(1017.01792717,98.6992362)(1017.02792725,98.58924123)
\curveto(1017.03792715,98.47923642)(1017.05292713,98.36423654)(1017.07292725,98.24424123)
\curveto(1017.0829271,98.2042367)(1017.0829271,98.15923674)(1017.07292725,98.10924123)
\curveto(1017.07292711,98.06923683)(1017.07792711,98.02923687)(1017.08792725,97.98924123)
\curveto(1017.09792709,97.94923695)(1017.10292708,97.89423701)(1017.10292725,97.82424123)
\curveto(1017.10292708,97.75423715)(1017.09792709,97.7042372)(1017.08792725,97.67424123)
\curveto(1017.06792712,97.62423728)(1017.06292712,97.57923732)(1017.07292725,97.53924123)
\curveto(1017.0829271,97.4992374)(1017.0829271,97.46423744)(1017.07292725,97.43424123)
\lineto(1017.07292725,97.34424123)
\curveto(1017.05292713,97.28423762)(1017.03792715,97.21923768)(1017.02792725,97.14924123)
\curveto(1017.02792716,97.08923781)(1017.02292716,97.02423788)(1017.01292725,96.95424123)
\curveto(1016.96292722,96.78423812)(1016.91292727,96.62423828)(1016.86292725,96.47424123)
\curveto(1016.81292737,96.32423858)(1016.74792744,96.17923872)(1016.66792725,96.03924123)
\curveto(1016.62792756,95.98923891)(1016.59792759,95.93423897)(1016.57792725,95.87424123)
\curveto(1016.54792764,95.82423908)(1016.51292767,95.77423913)(1016.47292725,95.72424123)
\curveto(1016.29292789,95.48423942)(1016.07292811,95.28423962)(1015.81292725,95.12424123)
\curveto(1015.55292863,94.96423994)(1015.26792892,94.82424008)(1014.95792725,94.70424123)
\curveto(1014.81792937,94.64424026)(1014.67792951,94.5992403)(1014.53792725,94.56924123)
\curveto(1014.3879298,94.53924036)(1014.23292995,94.5042404)(1014.07292725,94.46424123)
\curveto(1013.96293022,94.44424046)(1013.85293033,94.42924047)(1013.74292725,94.41924123)
\curveto(1013.63293055,94.40924049)(1013.52293066,94.39424051)(1013.41292725,94.37424123)
\curveto(1013.37293081,94.36424054)(1013.33293085,94.35924054)(1013.29292725,94.35924123)
\curveto(1013.25293093,94.36924053)(1013.21293097,94.36924053)(1013.17292725,94.35924123)
\curveto(1013.12293106,94.34924055)(1013.07293111,94.34424056)(1013.02292725,94.34424123)
\lineto(1012.85792725,94.34424123)
\curveto(1012.80793138,94.32424058)(1012.75793143,94.31924058)(1012.70792725,94.32924123)
\curveto(1012.64793154,94.33924056)(1012.59293159,94.33924056)(1012.54292725,94.32924123)
\curveto(1012.50293168,94.31924058)(1012.45793173,94.31924058)(1012.40792725,94.32924123)
\curveto(1012.35793183,94.33924056)(1012.30793188,94.33424057)(1012.25792725,94.31424123)
\curveto(1012.187932,94.29424061)(1012.11293207,94.28924061)(1012.03292725,94.29924123)
\curveto(1011.94293224,94.30924059)(1011.85793233,94.31424059)(1011.77792725,94.31424123)
\curveto(1011.6879325,94.31424059)(1011.5879326,94.30924059)(1011.47792725,94.29924123)
\curveto(1011.35793283,94.28924061)(1011.25793293,94.29424061)(1011.17792725,94.31424123)
\lineto(1010.89292725,94.31424123)
\lineto(1010.26292725,94.35924123)
\curveto(1010.16293402,94.36924053)(1010.06793412,94.37924052)(1009.97792725,94.38924123)
\lineto(1009.67792725,94.41924123)
\curveto(1009.62793456,94.43924046)(1009.57793461,94.44424046)(1009.52792725,94.43424123)
\curveto(1009.46793472,94.43424047)(1009.41293477,94.44424046)(1009.36292725,94.46424123)
\curveto(1009.19293499,94.51424039)(1009.02793516,94.55424035)(1008.86792725,94.58424123)
\curveto(1008.69793549,94.61424029)(1008.53793565,94.66424024)(1008.38792725,94.73424123)
\curveto(1007.92793626,94.92423998)(1007.55293663,95.14423976)(1007.26292725,95.39424123)
\curveto(1006.97293721,95.65423925)(1006.72793746,96.01423889)(1006.52792725,96.47424123)
\curveto(1006.47793771,96.6042383)(1006.44293774,96.73423817)(1006.42292725,96.86424123)
\curveto(1006.40293778,97.0042379)(1006.37793781,97.14423776)(1006.34792725,97.28424123)
\curveto(1006.33793785,97.35423755)(1006.33293785,97.41923748)(1006.33292725,97.47924123)
\curveto(1006.33293785,97.53923736)(1006.32793786,97.6042373)(1006.31792725,97.67424123)
\curveto(1006.29793789,98.5042364)(1006.44793774,99.17423573)(1006.76792725,99.68424123)
\curveto(1007.07793711,100.19423471)(1007.51793667,100.57423433)(1008.08792725,100.82424123)
\curveto(1008.20793598,100.87423403)(1008.33293585,100.91923398)(1008.46292725,100.95924123)
\curveto(1008.59293559,100.9992339)(1008.72793546,101.04423386)(1008.86792725,101.09424123)
\curveto(1008.94793524,101.11423379)(1009.03293515,101.12923377)(1009.12292725,101.13924123)
\lineto(1009.36292725,101.19924123)
\curveto(1009.47293471,101.22923367)(1009.5829346,101.24423366)(1009.69292725,101.24424123)
\curveto(1009.80293438,101.25423365)(1009.91293427,101.26923363)(1010.02292725,101.28924123)
\curveto(1010.07293411,101.30923359)(1010.11793407,101.31423359)(1010.15792725,101.30424123)
\curveto(1010.19793399,101.3042336)(1010.23793395,101.30923359)(1010.27792725,101.31924123)
\curveto(1010.32793386,101.32923357)(1010.3829338,101.32923357)(1010.44292725,101.31924123)
\curveto(1010.49293369,101.31923358)(1010.54293364,101.32423358)(1010.59292725,101.33424123)
\lineto(1010.72792725,101.33424123)
\curveto(1010.7879334,101.35423355)(1010.85793333,101.35423355)(1010.93792725,101.33424123)
\curveto(1011.00793318,101.32423358)(1011.07293311,101.32923357)(1011.13292725,101.34924123)
\curveto(1011.16293302,101.35923354)(1011.20293298,101.36423354)(1011.25292725,101.36424123)
\lineto(1011.37292725,101.36424123)
\lineto(1011.83792725,101.36424123)
\moveto(1014.16292725,99.81924123)
\curveto(1013.84293034,99.91923498)(1013.47793071,99.97923492)(1013.06792725,99.99924123)
\curveto(1012.65793153,100.01923488)(1012.24793194,100.02923487)(1011.83792725,100.02924123)
\curveto(1011.40793278,100.02923487)(1010.9879332,100.01923488)(1010.57792725,99.99924123)
\curveto(1010.16793402,99.97923492)(1009.7829344,99.93423497)(1009.42292725,99.86424123)
\curveto(1009.06293512,99.79423511)(1008.74293544,99.68423522)(1008.46292725,99.53424123)
\curveto(1008.17293601,99.39423551)(1007.93793625,99.1992357)(1007.75792725,98.94924123)
\curveto(1007.64793654,98.78923611)(1007.56793662,98.60923629)(1007.51792725,98.40924123)
\curveto(1007.45793673,98.20923669)(1007.42793676,97.96423694)(1007.42792725,97.67424123)
\curveto(1007.44793674,97.65423725)(1007.45793673,97.61923728)(1007.45792725,97.56924123)
\curveto(1007.44793674,97.51923738)(1007.44793674,97.47923742)(1007.45792725,97.44924123)
\curveto(1007.47793671,97.36923753)(1007.49793669,97.29423761)(1007.51792725,97.22424123)
\curveto(1007.52793666,97.16423774)(1007.54793664,97.0992378)(1007.57792725,97.02924123)
\curveto(1007.69793649,96.75923814)(1007.86793632,96.53923836)(1008.08792725,96.36924123)
\curveto(1008.29793589,96.20923869)(1008.54293564,96.07423883)(1008.82292725,95.96424123)
\curveto(1008.93293525,95.91423899)(1009.05293513,95.87423903)(1009.18292725,95.84424123)
\curveto(1009.30293488,95.82423908)(1009.42793476,95.7992391)(1009.55792725,95.76924123)
\curveto(1009.60793458,95.74923915)(1009.66293452,95.73923916)(1009.72292725,95.73924123)
\curveto(1009.77293441,95.73923916)(1009.82293436,95.73423917)(1009.87292725,95.72424123)
\curveto(1009.96293422,95.71423919)(1010.05793413,95.7042392)(1010.15792725,95.69424123)
\curveto(1010.24793394,95.68423922)(1010.34293384,95.67423923)(1010.44292725,95.66424123)
\curveto(1010.52293366,95.66423924)(1010.60793358,95.65923924)(1010.69792725,95.64924123)
\lineto(1010.93792725,95.64924123)
\lineto(1011.11792725,95.64924123)
\curveto(1011.14793304,95.63923926)(1011.182933,95.63423927)(1011.22292725,95.63424123)
\lineto(1011.35792725,95.63424123)
\lineto(1011.80792725,95.63424123)
\curveto(1011.8879323,95.63423927)(1011.97293221,95.62923927)(1012.06292725,95.61924123)
\curveto(1012.14293204,95.61923928)(1012.21793197,95.62923927)(1012.28792725,95.64924123)
\lineto(1012.55792725,95.64924123)
\curveto(1012.57793161,95.64923925)(1012.60793158,95.64423926)(1012.64792725,95.63424123)
\curveto(1012.67793151,95.63423927)(1012.70293148,95.63923926)(1012.72292725,95.64924123)
\curveto(1012.82293136,95.65923924)(1012.92293126,95.66423924)(1013.02292725,95.66424123)
\curveto(1013.11293107,95.67423923)(1013.21293097,95.68423922)(1013.32292725,95.69424123)
\curveto(1013.44293074,95.72423918)(1013.56793062,95.73923916)(1013.69792725,95.73924123)
\curveto(1013.81793037,95.74923915)(1013.93293025,95.77423913)(1014.04292725,95.81424123)
\curveto(1014.34292984,95.89423901)(1014.60792958,95.97923892)(1014.83792725,96.06924123)
\curveto(1015.06792912,96.16923873)(1015.2829289,96.31423859)(1015.48292725,96.50424123)
\curveto(1015.6829285,96.71423819)(1015.83292835,96.97923792)(1015.93292725,97.29924123)
\curveto(1015.95292823,97.33923756)(1015.96292822,97.37423753)(1015.96292725,97.40424123)
\curveto(1015.95292823,97.44423746)(1015.95792823,97.48923741)(1015.97792725,97.53924123)
\curveto(1015.9879282,97.57923732)(1015.99792819,97.64923725)(1016.00792725,97.74924123)
\curveto(1016.01792817,97.85923704)(1016.01292817,97.94423696)(1015.99292725,98.00424123)
\curveto(1015.97292821,98.07423683)(1015.96292822,98.14423676)(1015.96292725,98.21424123)
\curveto(1015.95292823,98.28423662)(1015.93792825,98.34923655)(1015.91792725,98.40924123)
\curveto(1015.85792833,98.60923629)(1015.77292841,98.78923611)(1015.66292725,98.94924123)
\curveto(1015.64292854,98.97923592)(1015.62292856,99.0042359)(1015.60292725,99.02424123)
\lineto(1015.54292725,99.08424123)
\curveto(1015.52292866,99.12423578)(1015.4829287,99.17423573)(1015.42292725,99.23424123)
\curveto(1015.2829289,99.33423557)(1015.15292903,99.41923548)(1015.03292725,99.48924123)
\curveto(1014.91292927,99.55923534)(1014.76792942,99.62923527)(1014.59792725,99.69924123)
\curveto(1014.52792966,99.72923517)(1014.45792973,99.74923515)(1014.38792725,99.75924123)
\curveto(1014.31792987,99.77923512)(1014.24292994,99.7992351)(1014.16292725,99.81924123)
}
}
{
\newrgbcolor{curcolor}{0 0 0}
\pscustom[linestyle=none,fillstyle=solid,fillcolor=curcolor]
{
\newpath
\moveto(1006.31792725,106.77385061)
\curveto(1006.31793787,106.87384575)(1006.32793786,106.96884566)(1006.34792725,107.05885061)
\curveto(1006.35793783,107.14884548)(1006.3879378,107.21384541)(1006.43792725,107.25385061)
\curveto(1006.51793767,107.31384531)(1006.62293756,107.34384528)(1006.75292725,107.34385061)
\lineto(1007.14292725,107.34385061)
\lineto(1008.64292725,107.34385061)
\lineto(1015.03292725,107.34385061)
\lineto(1016.20292725,107.34385061)
\lineto(1016.51792725,107.34385061)
\curveto(1016.61792757,107.35384527)(1016.69792749,107.33884529)(1016.75792725,107.29885061)
\curveto(1016.83792735,107.24884538)(1016.8879273,107.17384545)(1016.90792725,107.07385061)
\curveto(1016.91792727,106.98384564)(1016.92292726,106.87384575)(1016.92292725,106.74385061)
\lineto(1016.92292725,106.51885061)
\curveto(1016.90292728,106.43884619)(1016.8879273,106.36884626)(1016.87792725,106.30885061)
\curveto(1016.85792733,106.24884638)(1016.81792737,106.19884643)(1016.75792725,106.15885061)
\curveto(1016.69792749,106.11884651)(1016.62292756,106.09884653)(1016.53292725,106.09885061)
\lineto(1016.23292725,106.09885061)
\lineto(1015.13792725,106.09885061)
\lineto(1009.79792725,106.09885061)
\curveto(1009.70793448,106.07884655)(1009.63293455,106.06384656)(1009.57292725,106.05385061)
\curveto(1009.50293468,106.05384657)(1009.44293474,106.0238466)(1009.39292725,105.96385061)
\curveto(1009.34293484,105.89384673)(1009.31793487,105.80384682)(1009.31792725,105.69385061)
\curveto(1009.30793488,105.59384703)(1009.30293488,105.48384714)(1009.30292725,105.36385061)
\lineto(1009.30292725,104.22385061)
\lineto(1009.30292725,103.72885061)
\curveto(1009.29293489,103.56884906)(1009.23293495,103.45884917)(1009.12292725,103.39885061)
\curveto(1009.09293509,103.37884925)(1009.06293512,103.36884926)(1009.03292725,103.36885061)
\curveto(1008.99293519,103.36884926)(1008.94793524,103.36384926)(1008.89792725,103.35385061)
\curveto(1008.77793541,103.33384929)(1008.66793552,103.33884929)(1008.56792725,103.36885061)
\curveto(1008.46793572,103.40884922)(1008.39793579,103.46384916)(1008.35792725,103.53385061)
\curveto(1008.30793588,103.61384901)(1008.2829359,103.73384889)(1008.28292725,103.89385061)
\curveto(1008.2829359,104.05384857)(1008.26793592,104.18884844)(1008.23792725,104.29885061)
\curveto(1008.22793596,104.34884828)(1008.22293596,104.40384822)(1008.22292725,104.46385061)
\curveto(1008.21293597,104.5238481)(1008.19793599,104.58384804)(1008.17792725,104.64385061)
\curveto(1008.12793606,104.79384783)(1008.07793611,104.93884769)(1008.02792725,105.07885061)
\curveto(1007.96793622,105.21884741)(1007.89793629,105.35384727)(1007.81792725,105.48385061)
\curveto(1007.72793646,105.623847)(1007.62293656,105.74384688)(1007.50292725,105.84385061)
\curveto(1007.3829368,105.94384668)(1007.25293693,106.03884659)(1007.11292725,106.12885061)
\curveto(1007.01293717,106.18884644)(1006.90293728,106.23384639)(1006.78292725,106.26385061)
\curveto(1006.66293752,106.30384632)(1006.55793763,106.35384627)(1006.46792725,106.41385061)
\curveto(1006.40793778,106.46384616)(1006.36793782,106.53384609)(1006.34792725,106.62385061)
\curveto(1006.33793785,106.64384598)(1006.33293785,106.66884596)(1006.33292725,106.69885061)
\curveto(1006.33293785,106.7288459)(1006.32793786,106.75384587)(1006.31792725,106.77385061)
}
}
{
\newrgbcolor{curcolor}{0 0 0}
\pscustom[linestyle=none,fillstyle=solid,fillcolor=curcolor]
{
\newpath
\moveto(1006.31792725,115.12345998)
\curveto(1006.31793787,115.22345513)(1006.32793786,115.31845503)(1006.34792725,115.40845998)
\curveto(1006.35793783,115.49845485)(1006.3879378,115.56345479)(1006.43792725,115.60345998)
\curveto(1006.51793767,115.66345469)(1006.62293756,115.69345466)(1006.75292725,115.69345998)
\lineto(1007.14292725,115.69345998)
\lineto(1008.64292725,115.69345998)
\lineto(1015.03292725,115.69345998)
\lineto(1016.20292725,115.69345998)
\lineto(1016.51792725,115.69345998)
\curveto(1016.61792757,115.70345465)(1016.69792749,115.68845466)(1016.75792725,115.64845998)
\curveto(1016.83792735,115.59845475)(1016.8879273,115.52345483)(1016.90792725,115.42345998)
\curveto(1016.91792727,115.33345502)(1016.92292726,115.22345513)(1016.92292725,115.09345998)
\lineto(1016.92292725,114.86845998)
\curveto(1016.90292728,114.78845556)(1016.8879273,114.71845563)(1016.87792725,114.65845998)
\curveto(1016.85792733,114.59845575)(1016.81792737,114.5484558)(1016.75792725,114.50845998)
\curveto(1016.69792749,114.46845588)(1016.62292756,114.4484559)(1016.53292725,114.44845998)
\lineto(1016.23292725,114.44845998)
\lineto(1015.13792725,114.44845998)
\lineto(1009.79792725,114.44845998)
\curveto(1009.70793448,114.42845592)(1009.63293455,114.41345594)(1009.57292725,114.40345998)
\curveto(1009.50293468,114.40345595)(1009.44293474,114.37345598)(1009.39292725,114.31345998)
\curveto(1009.34293484,114.24345611)(1009.31793487,114.1534562)(1009.31792725,114.04345998)
\curveto(1009.30793488,113.94345641)(1009.30293488,113.83345652)(1009.30292725,113.71345998)
\lineto(1009.30292725,112.57345998)
\lineto(1009.30292725,112.07845998)
\curveto(1009.29293489,111.91845843)(1009.23293495,111.80845854)(1009.12292725,111.74845998)
\curveto(1009.09293509,111.72845862)(1009.06293512,111.71845863)(1009.03292725,111.71845998)
\curveto(1008.99293519,111.71845863)(1008.94793524,111.71345864)(1008.89792725,111.70345998)
\curveto(1008.77793541,111.68345867)(1008.66793552,111.68845866)(1008.56792725,111.71845998)
\curveto(1008.46793572,111.75845859)(1008.39793579,111.81345854)(1008.35792725,111.88345998)
\curveto(1008.30793588,111.96345839)(1008.2829359,112.08345827)(1008.28292725,112.24345998)
\curveto(1008.2829359,112.40345795)(1008.26793592,112.53845781)(1008.23792725,112.64845998)
\curveto(1008.22793596,112.69845765)(1008.22293596,112.7534576)(1008.22292725,112.81345998)
\curveto(1008.21293597,112.87345748)(1008.19793599,112.93345742)(1008.17792725,112.99345998)
\curveto(1008.12793606,113.14345721)(1008.07793611,113.28845706)(1008.02792725,113.42845998)
\curveto(1007.96793622,113.56845678)(1007.89793629,113.70345665)(1007.81792725,113.83345998)
\curveto(1007.72793646,113.97345638)(1007.62293656,114.09345626)(1007.50292725,114.19345998)
\curveto(1007.3829368,114.29345606)(1007.25293693,114.38845596)(1007.11292725,114.47845998)
\curveto(1007.01293717,114.53845581)(1006.90293728,114.58345577)(1006.78292725,114.61345998)
\curveto(1006.66293752,114.6534557)(1006.55793763,114.70345565)(1006.46792725,114.76345998)
\curveto(1006.40793778,114.81345554)(1006.36793782,114.88345547)(1006.34792725,114.97345998)
\curveto(1006.33793785,114.99345536)(1006.33293785,115.01845533)(1006.33292725,115.04845998)
\curveto(1006.33293785,115.07845527)(1006.32793786,115.10345525)(1006.31792725,115.12345998)
}
}
{
\newrgbcolor{curcolor}{0 0 0}
\pscustom[linestyle=none,fillstyle=solid,fillcolor=curcolor]
{
\newpath
\moveto(222.02405518,31.67142873)
\lineto(222.02405518,32.58642873)
\curveto(222.02406587,32.68642608)(222.02406587,32.78142599)(222.02405518,32.87142873)
\curveto(222.02406587,32.96142581)(222.04406585,33.03642573)(222.08405518,33.09642873)
\curveto(222.14406575,33.18642558)(222.22406567,33.24642552)(222.32405518,33.27642873)
\curveto(222.42406547,33.31642545)(222.52906537,33.36142541)(222.63905518,33.41142873)
\curveto(222.82906507,33.49142528)(223.01906488,33.56142521)(223.20905518,33.62142873)
\curveto(223.3990645,33.69142508)(223.58906431,33.766425)(223.77905518,33.84642873)
\curveto(223.95906394,33.91642485)(224.14406375,33.98142479)(224.33405518,34.04142873)
\curveto(224.51406338,34.10142467)(224.6940632,34.1714246)(224.87405518,34.25142873)
\curveto(225.01406288,34.31142446)(225.15906274,34.3664244)(225.30905518,34.41642873)
\curveto(225.45906244,34.4664243)(225.60406229,34.52142425)(225.74405518,34.58142873)
\curveto(226.1940617,34.76142401)(226.64906125,34.93142384)(227.10905518,35.09142873)
\curveto(227.55906034,35.25142352)(228.00905989,35.42142335)(228.45905518,35.60142873)
\curveto(228.50905939,35.62142315)(228.55905934,35.63642313)(228.60905518,35.64642873)
\lineto(228.75905518,35.70642873)
\curveto(228.97905892,35.79642297)(229.20405869,35.88142289)(229.43405518,35.96142873)
\curveto(229.65405824,36.04142273)(229.87405802,36.12642264)(230.09405518,36.21642873)
\curveto(230.18405771,36.25642251)(230.2940576,36.29642247)(230.42405518,36.33642873)
\curveto(230.54405735,36.37642239)(230.61405728,36.44142233)(230.63405518,36.53142873)
\curveto(230.64405725,36.5714222)(230.64405725,36.60142217)(230.63405518,36.62142873)
\lineto(230.57405518,36.68142873)
\curveto(230.52405737,36.73142204)(230.46905743,36.766422)(230.40905518,36.78642873)
\curveto(230.34905755,36.81642195)(230.28405761,36.84642192)(230.21405518,36.87642873)
\lineto(229.58405518,37.11642873)
\curveto(229.36405853,37.19642157)(229.14905875,37.27642149)(228.93905518,37.35642873)
\lineto(228.78905518,37.41642873)
\lineto(228.60905518,37.47642873)
\curveto(228.41905948,37.55642121)(228.22905967,37.62642114)(228.03905518,37.68642873)
\curveto(227.83906006,37.75642101)(227.63906026,37.83142094)(227.43905518,37.91142873)
\curveto(226.85906104,38.15142062)(226.27406162,38.3714204)(225.68405518,38.57142873)
\curveto(225.0940628,38.78141999)(224.50906339,39.00641976)(223.92905518,39.24642873)
\curveto(223.72906417,39.32641944)(223.52406437,39.40141937)(223.31405518,39.47142873)
\curveto(223.10406479,39.55141922)(222.899065,39.63141914)(222.69905518,39.71142873)
\curveto(222.61906528,39.75141902)(222.51906538,39.78641898)(222.39905518,39.81642873)
\curveto(222.27906562,39.85641891)(222.1940657,39.91141886)(222.14405518,39.98142873)
\curveto(222.10406579,40.04141873)(222.07406582,40.11641865)(222.05405518,40.20642873)
\curveto(222.03406586,40.30641846)(222.02406587,40.41641835)(222.02405518,40.53642873)
\curveto(222.01406588,40.65641811)(222.01406588,40.77641799)(222.02405518,40.89642873)
\curveto(222.02406587,41.01641775)(222.02406587,41.12641764)(222.02405518,41.22642873)
\curveto(222.02406587,41.31641745)(222.02406587,41.40641736)(222.02405518,41.49642873)
\curveto(222.02406587,41.59641717)(222.04406585,41.6714171)(222.08405518,41.72142873)
\curveto(222.13406576,41.81141696)(222.22406567,41.86141691)(222.35405518,41.87142873)
\curveto(222.48406541,41.88141689)(222.62406527,41.88641688)(222.77405518,41.88642873)
\lineto(224.42405518,41.88642873)
\lineto(230.69405518,41.88642873)
\lineto(231.95405518,41.88642873)
\curveto(232.06405583,41.88641688)(232.17405572,41.88641688)(232.28405518,41.88642873)
\curveto(232.3940555,41.89641687)(232.47905542,41.87641689)(232.53905518,41.82642873)
\curveto(232.5990553,41.79641697)(232.63905526,41.75141702)(232.65905518,41.69142873)
\curveto(232.66905523,41.63141714)(232.68405521,41.56141721)(232.70405518,41.48142873)
\lineto(232.70405518,41.24142873)
\lineto(232.70405518,40.88142873)
\curveto(232.6940552,40.771418)(232.64905525,40.69141808)(232.56905518,40.64142873)
\curveto(232.53905536,40.62141815)(232.50905539,40.60641816)(232.47905518,40.59642873)
\curveto(232.43905546,40.59641817)(232.3940555,40.58641818)(232.34405518,40.56642873)
\lineto(232.17905518,40.56642873)
\curveto(232.11905578,40.55641821)(232.04905585,40.55141822)(231.96905518,40.55142873)
\curveto(231.88905601,40.56141821)(231.81405608,40.5664182)(231.74405518,40.56642873)
\lineto(230.90405518,40.56642873)
\lineto(226.47905518,40.56642873)
\curveto(226.22906167,40.5664182)(225.97906192,40.5664182)(225.72905518,40.56642873)
\curveto(225.46906243,40.5664182)(225.21906268,40.56141821)(224.97905518,40.55142873)
\curveto(224.87906302,40.55141822)(224.76906313,40.54641822)(224.64905518,40.53642873)
\curveto(224.52906337,40.52641824)(224.46906343,40.4714183)(224.46905518,40.37142873)
\lineto(224.48405518,40.37142873)
\curveto(224.50406339,40.30141847)(224.56906333,40.24141853)(224.67905518,40.19142873)
\curveto(224.78906311,40.15141862)(224.88406301,40.11641865)(224.96405518,40.08642873)
\curveto(225.13406276,40.01641875)(225.30906259,39.95141882)(225.48905518,39.89142873)
\curveto(225.65906224,39.83141894)(225.82906207,39.76141901)(225.99905518,39.68142873)
\curveto(226.04906185,39.66141911)(226.0940618,39.64641912)(226.13405518,39.63642873)
\curveto(226.17406172,39.62641914)(226.21906168,39.61141916)(226.26905518,39.59142873)
\curveto(226.44906145,39.51141926)(226.63406126,39.44141933)(226.82405518,39.38142873)
\curveto(227.00406089,39.33141944)(227.18406071,39.2664195)(227.36405518,39.18642873)
\curveto(227.51406038,39.11641965)(227.66906023,39.05641971)(227.82905518,39.00642873)
\curveto(227.97905992,38.95641981)(228.12905977,38.90141987)(228.27905518,38.84142873)
\curveto(228.74905915,38.64142013)(229.22405867,38.46142031)(229.70405518,38.30142873)
\curveto(230.17405772,38.14142063)(230.63905726,37.9664208)(231.09905518,37.77642873)
\curveto(231.27905662,37.69642107)(231.45905644,37.62642114)(231.63905518,37.56642873)
\curveto(231.81905608,37.50642126)(231.9990559,37.44142133)(232.17905518,37.37142873)
\curveto(232.28905561,37.32142145)(232.3940555,37.2714215)(232.49405518,37.22142873)
\curveto(232.58405531,37.18142159)(232.64905525,37.09642167)(232.68905518,36.96642873)
\curveto(232.6990552,36.94642182)(232.70405519,36.92142185)(232.70405518,36.89142873)
\curveto(232.6940552,36.8714219)(232.6940552,36.84642192)(232.70405518,36.81642873)
\curveto(232.71405518,36.78642198)(232.71905518,36.75142202)(232.71905518,36.71142873)
\curveto(232.70905519,36.6714221)(232.70405519,36.63142214)(232.70405518,36.59142873)
\lineto(232.70405518,36.29142873)
\curveto(232.70405519,36.19142258)(232.67905522,36.11142266)(232.62905518,36.05142873)
\curveto(232.57905532,35.9714228)(232.50905539,35.91142286)(232.41905518,35.87142873)
\curveto(232.31905558,35.84142293)(232.21905568,35.80142297)(232.11905518,35.75142873)
\curveto(231.91905598,35.6714231)(231.71405618,35.59142318)(231.50405518,35.51142873)
\curveto(231.28405661,35.44142333)(231.07405682,35.3664234)(230.87405518,35.28642873)
\curveto(230.6940572,35.20642356)(230.51405738,35.13642363)(230.33405518,35.07642873)
\curveto(230.14405775,35.02642374)(229.95905794,34.96142381)(229.77905518,34.88142873)
\curveto(229.21905868,34.65142412)(228.65405924,34.43642433)(228.08405518,34.23642873)
\curveto(227.51406038,34.03642473)(226.94906095,33.82142495)(226.38905518,33.59142873)
\lineto(225.75905518,33.35142873)
\curveto(225.53906236,33.28142549)(225.32906257,33.20642556)(225.12905518,33.12642873)
\curveto(225.01906288,33.07642569)(224.91406298,33.03142574)(224.81405518,32.99142873)
\curveto(224.70406319,32.96142581)(224.60906329,32.91142586)(224.52905518,32.84142873)
\curveto(224.50906339,32.83142594)(224.4990634,32.82142595)(224.49905518,32.81142873)
\lineto(224.46905518,32.78142873)
\lineto(224.46905518,32.70642873)
\lineto(224.49905518,32.67642873)
\curveto(224.4990634,32.6664261)(224.50406339,32.65642611)(224.51405518,32.64642873)
\curveto(224.56406333,32.62642614)(224.61906328,32.61642615)(224.67905518,32.61642873)
\curveto(224.73906316,32.61642615)(224.7990631,32.60642616)(224.85905518,32.58642873)
\lineto(225.02405518,32.58642873)
\curveto(225.08406281,32.5664262)(225.14906275,32.56142621)(225.21905518,32.57142873)
\curveto(225.28906261,32.58142619)(225.35906254,32.58642618)(225.42905518,32.58642873)
\lineto(226.23905518,32.58642873)
\lineto(230.79905518,32.58642873)
\lineto(231.98405518,32.58642873)
\curveto(232.0940558,32.58642618)(232.20405569,32.58142619)(232.31405518,32.57142873)
\curveto(232.42405547,32.5714262)(232.50905539,32.54642622)(232.56905518,32.49642873)
\curveto(232.64905525,32.44642632)(232.6940552,32.35642641)(232.70405518,32.22642873)
\lineto(232.70405518,31.83642873)
\lineto(232.70405518,31.64142873)
\curveto(232.70405519,31.59142718)(232.6940552,31.54142723)(232.67405518,31.49142873)
\curveto(232.63405526,31.36142741)(232.54905535,31.28642748)(232.41905518,31.26642873)
\curveto(232.28905561,31.25642751)(232.13905576,31.25142752)(231.96905518,31.25142873)
\lineto(230.22905518,31.25142873)
\lineto(224.22905518,31.25142873)
\lineto(222.81905518,31.25142873)
\curveto(222.70906519,31.25142752)(222.5940653,31.24642752)(222.47405518,31.23642873)
\curveto(222.35406554,31.23642753)(222.25906564,31.26142751)(222.18905518,31.31142873)
\curveto(222.12906577,31.35142742)(222.07906582,31.42642734)(222.03905518,31.53642873)
\curveto(222.02906587,31.55642721)(222.02906587,31.57642719)(222.03905518,31.59642873)
\curveto(222.03906586,31.62642714)(222.03406586,31.65142712)(222.02405518,31.67142873)
}
}
{
\newrgbcolor{curcolor}{0 0 0}
\pscustom[linestyle=none,fillstyle=solid,fillcolor=curcolor]
{
\newpath
\moveto(232.14905518,50.87353811)
\curveto(232.30905559,50.90353028)(232.44405545,50.88853029)(232.55405518,50.82853811)
\curveto(232.65405524,50.76853041)(232.72905517,50.68853049)(232.77905518,50.58853811)
\curveto(232.7990551,50.53853064)(232.80905509,50.4835307)(232.80905518,50.42353811)
\curveto(232.80905509,50.37353081)(232.81905508,50.31853086)(232.83905518,50.25853811)
\curveto(232.88905501,50.03853114)(232.87405502,49.81853136)(232.79405518,49.59853811)
\curveto(232.72405517,49.38853179)(232.63405526,49.24353194)(232.52405518,49.16353811)
\curveto(232.45405544,49.11353207)(232.37405552,49.06853211)(232.28405518,49.02853811)
\curveto(232.18405571,48.98853219)(232.10405579,48.93853224)(232.04405518,48.87853811)
\curveto(232.02405587,48.85853232)(232.00405589,48.83353235)(231.98405518,48.80353811)
\curveto(231.96405593,48.7835324)(231.95905594,48.75353243)(231.96905518,48.71353811)
\curveto(231.9990559,48.60353258)(232.05405584,48.49853268)(232.13405518,48.39853811)
\curveto(232.21405568,48.30853287)(232.28405561,48.21853296)(232.34405518,48.12853811)
\curveto(232.42405547,47.99853318)(232.4990554,47.85853332)(232.56905518,47.70853811)
\curveto(232.62905527,47.55853362)(232.68405521,47.39853378)(232.73405518,47.22853811)
\curveto(232.76405513,47.12853405)(232.78405511,47.01853416)(232.79405518,46.89853811)
\curveto(232.80405509,46.78853439)(232.81905508,46.6785345)(232.83905518,46.56853811)
\curveto(232.84905505,46.51853466)(232.85405504,46.47353471)(232.85405518,46.43353811)
\lineto(232.85405518,46.32853811)
\curveto(232.87405502,46.21853496)(232.87405502,46.11353507)(232.85405518,46.01353811)
\lineto(232.85405518,45.87853811)
\curveto(232.84405505,45.82853535)(232.83905506,45.7785354)(232.83905518,45.72853811)
\curveto(232.83905506,45.6785355)(232.82905507,45.63353555)(232.80905518,45.59353811)
\curveto(232.7990551,45.55353563)(232.7940551,45.51853566)(232.79405518,45.48853811)
\curveto(232.80405509,45.46853571)(232.80405509,45.44353574)(232.79405518,45.41353811)
\lineto(232.73405518,45.17353811)
\curveto(232.72405517,45.09353609)(232.70405519,45.01853616)(232.67405518,44.94853811)
\curveto(232.54405535,44.64853653)(232.3990555,44.40353678)(232.23905518,44.21353811)
\curveto(232.06905583,44.03353715)(231.83405606,43.8835373)(231.53405518,43.76353811)
\curveto(231.31405658,43.67353751)(231.04905685,43.62853755)(230.73905518,43.62853811)
\lineto(230.42405518,43.62853811)
\curveto(230.37405752,43.63853754)(230.32405757,43.64353754)(230.27405518,43.64353811)
\lineto(230.09405518,43.67353811)
\lineto(229.76405518,43.79353811)
\curveto(229.65405824,43.83353735)(229.55405834,43.8835373)(229.46405518,43.94353811)
\curveto(229.17405872,44.12353706)(228.95905894,44.36853681)(228.81905518,44.67853811)
\curveto(228.67905922,44.98853619)(228.55405934,45.32853585)(228.44405518,45.69853811)
\curveto(228.40405949,45.83853534)(228.37405952,45.9835352)(228.35405518,46.13353811)
\curveto(228.33405956,46.2835349)(228.30905959,46.43353475)(228.27905518,46.58353811)
\curveto(228.25905964,46.65353453)(228.24905965,46.71853446)(228.24905518,46.77853811)
\curveto(228.24905965,46.84853433)(228.23905966,46.92353426)(228.21905518,47.00353811)
\curveto(228.1990597,47.07353411)(228.18905971,47.14353404)(228.18905518,47.21353811)
\curveto(228.17905972,47.2835339)(228.16405973,47.35853382)(228.14405518,47.43853811)
\curveto(228.08405981,47.68853349)(228.03405986,47.92353326)(227.99405518,48.14353811)
\curveto(227.94405995,48.36353282)(227.82906007,48.53853264)(227.64905518,48.66853811)
\curveto(227.56906033,48.72853245)(227.46906043,48.7785324)(227.34905518,48.81853811)
\curveto(227.21906068,48.85853232)(227.07906082,48.85853232)(226.92905518,48.81853811)
\curveto(226.68906121,48.75853242)(226.4990614,48.66853251)(226.35905518,48.54853811)
\curveto(226.21906168,48.43853274)(226.10906179,48.2785329)(226.02905518,48.06853811)
\curveto(225.97906192,47.94853323)(225.94406195,47.80353338)(225.92405518,47.63353811)
\curveto(225.90406199,47.47353371)(225.894062,47.30353388)(225.89405518,47.12353811)
\curveto(225.894062,46.94353424)(225.90406199,46.76853441)(225.92405518,46.59853811)
\curveto(225.94406195,46.42853475)(225.97406192,46.2835349)(226.01405518,46.16353811)
\curveto(226.07406182,45.99353519)(226.15906174,45.82853535)(226.26905518,45.66853811)
\curveto(226.32906157,45.58853559)(226.40906149,45.51353567)(226.50905518,45.44353811)
\curveto(226.5990613,45.3835358)(226.6990612,45.32853585)(226.80905518,45.27853811)
\curveto(226.88906101,45.24853593)(226.97406092,45.21853596)(227.06405518,45.18853811)
\curveto(227.15406074,45.16853601)(227.22406067,45.12353606)(227.27405518,45.05353811)
\curveto(227.30406059,45.01353617)(227.32906057,44.94353624)(227.34905518,44.84353811)
\curveto(227.35906054,44.75353643)(227.36406053,44.65853652)(227.36405518,44.55853811)
\curveto(227.36406053,44.45853672)(227.35906054,44.35853682)(227.34905518,44.25853811)
\curveto(227.32906057,44.16853701)(227.30406059,44.10353708)(227.27405518,44.06353811)
\curveto(227.24406065,44.02353716)(227.1940607,43.99353719)(227.12405518,43.97353811)
\curveto(227.05406084,43.95353723)(226.97906092,43.95353723)(226.89905518,43.97353811)
\curveto(226.76906113,44.00353718)(226.64906125,44.03353715)(226.53905518,44.06353811)
\curveto(226.41906148,44.10353708)(226.30406159,44.14853703)(226.19405518,44.19853811)
\curveto(225.84406205,44.38853679)(225.57406232,44.62853655)(225.38405518,44.91853811)
\curveto(225.18406271,45.20853597)(225.02406287,45.56853561)(224.90405518,45.99853811)
\curveto(224.88406301,46.09853508)(224.86906303,46.19853498)(224.85905518,46.29853811)
\curveto(224.84906305,46.40853477)(224.83406306,46.51853466)(224.81405518,46.62853811)
\curveto(224.80406309,46.66853451)(224.80406309,46.73353445)(224.81405518,46.82353811)
\curveto(224.81406308,46.91353427)(224.80406309,46.96853421)(224.78405518,46.98853811)
\curveto(224.77406312,47.68853349)(224.85406304,48.29853288)(225.02405518,48.81853811)
\curveto(225.1940627,49.33853184)(225.51906238,49.70353148)(225.99905518,49.91353811)
\curveto(226.1990617,50.00353118)(226.43406146,50.05353113)(226.70405518,50.06353811)
\curveto(226.96406093,50.0835311)(227.23906066,50.09353109)(227.52905518,50.09353811)
\lineto(230.84405518,50.09353811)
\curveto(230.98405691,50.09353109)(231.11905678,50.09853108)(231.24905518,50.10853811)
\curveto(231.37905652,50.11853106)(231.48405641,50.14853103)(231.56405518,50.19853811)
\curveto(231.63405626,50.24853093)(231.68405621,50.31353087)(231.71405518,50.39353811)
\curveto(231.75405614,50.4835307)(231.78405611,50.56853061)(231.80405518,50.64853811)
\curveto(231.81405608,50.72853045)(231.85905604,50.78853039)(231.93905518,50.82853811)
\curveto(231.96905593,50.84853033)(231.9990559,50.85853032)(232.02905518,50.85853811)
\curveto(232.05905584,50.85853032)(232.0990558,50.86353032)(232.14905518,50.87353811)
\moveto(230.48405518,48.72853811)
\curveto(230.34405755,48.78853239)(230.18405771,48.81853236)(230.00405518,48.81853811)
\curveto(229.81405808,48.82853235)(229.61905828,48.83353235)(229.41905518,48.83353811)
\curveto(229.30905859,48.83353235)(229.20905869,48.82853235)(229.11905518,48.81853811)
\curveto(229.02905887,48.80853237)(228.95905894,48.76853241)(228.90905518,48.69853811)
\curveto(228.88905901,48.66853251)(228.87905902,48.59853258)(228.87905518,48.48853811)
\curveto(228.899059,48.46853271)(228.90905899,48.43353275)(228.90905518,48.38353811)
\curveto(228.90905899,48.33353285)(228.91905898,48.28853289)(228.93905518,48.24853811)
\curveto(228.95905894,48.16853301)(228.97905892,48.0785331)(228.99905518,47.97853811)
\lineto(229.05905518,47.67853811)
\curveto(229.05905884,47.64853353)(229.06405883,47.61353357)(229.07405518,47.57353811)
\lineto(229.07405518,47.46853811)
\curveto(229.11405878,47.31853386)(229.13905876,47.15353403)(229.14905518,46.97353811)
\curveto(229.14905875,46.80353438)(229.16905873,46.64353454)(229.20905518,46.49353811)
\curveto(229.22905867,46.41353477)(229.24905865,46.33853484)(229.26905518,46.26853811)
\curveto(229.27905862,46.20853497)(229.2940586,46.13853504)(229.31405518,46.05853811)
\curveto(229.36405853,45.89853528)(229.42905847,45.74853543)(229.50905518,45.60853811)
\curveto(229.57905832,45.46853571)(229.66905823,45.34853583)(229.77905518,45.24853811)
\curveto(229.88905801,45.14853603)(230.02405787,45.07353611)(230.18405518,45.02353811)
\curveto(230.33405756,44.97353621)(230.51905738,44.95353623)(230.73905518,44.96353811)
\curveto(230.83905706,44.96353622)(230.93405696,44.9785362)(231.02405518,45.00853811)
\curveto(231.10405679,45.04853613)(231.17905672,45.09353609)(231.24905518,45.14353811)
\curveto(231.35905654,45.22353596)(231.45405644,45.32853585)(231.53405518,45.45853811)
\curveto(231.60405629,45.58853559)(231.66405623,45.72853545)(231.71405518,45.87853811)
\curveto(231.72405617,45.92853525)(231.72905617,45.9785352)(231.72905518,46.02853811)
\curveto(231.72905617,46.0785351)(231.73405616,46.12853505)(231.74405518,46.17853811)
\curveto(231.76405613,46.24853493)(231.77905612,46.33353485)(231.78905518,46.43353811)
\curveto(231.78905611,46.54353464)(231.77905612,46.63353455)(231.75905518,46.70353811)
\curveto(231.73905616,46.76353442)(231.73405616,46.82353436)(231.74405518,46.88353811)
\curveto(231.74405615,46.94353424)(231.73405616,47.00353418)(231.71405518,47.06353811)
\curveto(231.6940562,47.14353404)(231.67905622,47.21853396)(231.66905518,47.28853811)
\curveto(231.65905624,47.36853381)(231.63905626,47.44353374)(231.60905518,47.51353811)
\curveto(231.48905641,47.80353338)(231.34405655,48.04853313)(231.17405518,48.24853811)
\curveto(231.00405689,48.45853272)(230.77405712,48.61853256)(230.48405518,48.72853811)
}
}
{
\newrgbcolor{curcolor}{0 0 0}
\pscustom[linestyle=none,fillstyle=solid,fillcolor=curcolor]
{
\newpath
\moveto(224.79905518,55.69017873)
\curveto(224.7990631,55.92017394)(224.85906304,56.05017381)(224.97905518,56.08017873)
\curveto(225.08906281,56.11017375)(225.25406264,56.12517374)(225.47405518,56.12517873)
\lineto(225.75905518,56.12517873)
\curveto(225.84906205,56.12517374)(225.92406197,56.10017376)(225.98405518,56.05017873)
\curveto(226.06406183,55.99017387)(226.10906179,55.90517396)(226.11905518,55.79517873)
\curveto(226.11906178,55.68517418)(226.13406176,55.57517429)(226.16405518,55.46517873)
\curveto(226.1940617,55.32517454)(226.22406167,55.19017467)(226.25405518,55.06017873)
\curveto(226.28406161,54.94017492)(226.32406157,54.82517504)(226.37405518,54.71517873)
\curveto(226.50406139,54.42517544)(226.68406121,54.19017567)(226.91405518,54.01017873)
\curveto(227.13406076,53.83017603)(227.38906051,53.67517619)(227.67905518,53.54517873)
\curveto(227.78906011,53.50517636)(227.90405999,53.47517639)(228.02405518,53.45517873)
\curveto(228.13405976,53.43517643)(228.24905965,53.41017645)(228.36905518,53.38017873)
\curveto(228.41905948,53.37017649)(228.46905943,53.3651765)(228.51905518,53.36517873)
\curveto(228.56905933,53.37517649)(228.61905928,53.37517649)(228.66905518,53.36517873)
\curveto(228.78905911,53.33517653)(228.92905897,53.32017654)(229.08905518,53.32017873)
\curveto(229.23905866,53.33017653)(229.38405851,53.33517653)(229.52405518,53.33517873)
\lineto(231.36905518,53.33517873)
\lineto(231.71405518,53.33517873)
\curveto(231.83405606,53.33517653)(231.94905595,53.33017653)(232.05905518,53.32017873)
\curveto(232.16905573,53.31017655)(232.26405563,53.30517656)(232.34405518,53.30517873)
\curveto(232.42405547,53.31517655)(232.4940554,53.29517657)(232.55405518,53.24517873)
\curveto(232.62405527,53.19517667)(232.66405523,53.11517675)(232.67405518,53.00517873)
\curveto(232.68405521,52.90517696)(232.68905521,52.79517707)(232.68905518,52.67517873)
\lineto(232.68905518,52.40517873)
\curveto(232.66905523,52.35517751)(232.65405524,52.30517756)(232.64405518,52.25517873)
\curveto(232.62405527,52.21517765)(232.5990553,52.18517768)(232.56905518,52.16517873)
\curveto(232.4990554,52.11517775)(232.41405548,52.08517778)(232.31405518,52.07517873)
\lineto(231.98405518,52.07517873)
\lineto(230.82905518,52.07517873)
\lineto(226.67405518,52.07517873)
\lineto(225.63905518,52.07517873)
\lineto(225.33905518,52.07517873)
\curveto(225.23906266,52.08517778)(225.15406274,52.11517775)(225.08405518,52.16517873)
\curveto(225.04406285,52.19517767)(225.01406288,52.24517762)(224.99405518,52.31517873)
\curveto(224.97406292,52.39517747)(224.96406293,52.48017738)(224.96405518,52.57017873)
\curveto(224.95406294,52.6601772)(224.95406294,52.75017711)(224.96405518,52.84017873)
\curveto(224.97406292,52.93017693)(224.98906291,53.00017686)(225.00905518,53.05017873)
\curveto(225.03906286,53.13017673)(225.0990628,53.18017668)(225.18905518,53.20017873)
\curveto(225.26906263,53.23017663)(225.35906254,53.24517662)(225.45905518,53.24517873)
\lineto(225.75905518,53.24517873)
\curveto(225.85906204,53.24517662)(225.94906195,53.2651766)(226.02905518,53.30517873)
\curveto(226.04906185,53.31517655)(226.06406183,53.32517654)(226.07405518,53.33517873)
\lineto(226.11905518,53.38017873)
\curveto(226.11906178,53.49017637)(226.07406182,53.58017628)(225.98405518,53.65017873)
\curveto(225.88406201,53.72017614)(225.80406209,53.78017608)(225.74405518,53.83017873)
\lineto(225.65405518,53.92017873)
\curveto(225.54406235,54.01017585)(225.42906247,54.13517573)(225.30905518,54.29517873)
\curveto(225.18906271,54.45517541)(225.0990628,54.60517526)(225.03905518,54.74517873)
\curveto(224.98906291,54.83517503)(224.95406294,54.93017493)(224.93405518,55.03017873)
\curveto(224.90406299,55.13017473)(224.87406302,55.23517463)(224.84405518,55.34517873)
\curveto(224.83406306,55.40517446)(224.82906307,55.4651744)(224.82905518,55.52517873)
\curveto(224.81906308,55.58517428)(224.80906309,55.64017422)(224.79905518,55.69017873)
}
}
{
\newrgbcolor{curcolor}{0 0 0}
\pscustom[linestyle=none,fillstyle=solid,fillcolor=curcolor]
{
}
}
{
\newrgbcolor{curcolor}{0 0 0}
\pscustom[linestyle=none,fillstyle=solid,fillcolor=curcolor]
{
\newpath
\moveto(222.09905518,64.24510061)
\curveto(222.08906581,64.93509597)(222.20906569,65.53509537)(222.45905518,66.04510061)
\curveto(222.70906519,66.56509434)(223.04406485,66.96009395)(223.46405518,67.23010061)
\curveto(223.54406435,67.28009363)(223.63406426,67.32509358)(223.73405518,67.36510061)
\curveto(223.82406407,67.4050935)(223.91906398,67.45009346)(224.01905518,67.50010061)
\curveto(224.11906378,67.54009337)(224.21906368,67.57009334)(224.31905518,67.59010061)
\curveto(224.41906348,67.6100933)(224.52406337,67.63009328)(224.63405518,67.65010061)
\curveto(224.68406321,67.67009324)(224.72906317,67.67509323)(224.76905518,67.66510061)
\curveto(224.80906309,67.65509325)(224.85406304,67.66009325)(224.90405518,67.68010061)
\curveto(224.95406294,67.69009322)(225.03906286,67.69509321)(225.15905518,67.69510061)
\curveto(225.26906263,67.69509321)(225.35406254,67.69009322)(225.41405518,67.68010061)
\curveto(225.47406242,67.66009325)(225.53406236,67.65009326)(225.59405518,67.65010061)
\curveto(225.65406224,67.66009325)(225.71406218,67.65509325)(225.77405518,67.63510061)
\curveto(225.91406198,67.59509331)(226.04906185,67.56009335)(226.17905518,67.53010061)
\curveto(226.30906159,67.50009341)(226.43406146,67.46009345)(226.55405518,67.41010061)
\curveto(226.6940612,67.35009356)(226.81906108,67.28009363)(226.92905518,67.20010061)
\curveto(227.03906086,67.13009378)(227.14906075,67.05509385)(227.25905518,66.97510061)
\lineto(227.31905518,66.91510061)
\curveto(227.33906056,66.905094)(227.35906054,66.89009402)(227.37905518,66.87010061)
\curveto(227.53906036,66.75009416)(227.68406021,66.61509429)(227.81405518,66.46510061)
\curveto(227.94405995,66.31509459)(228.06905983,66.15509475)(228.18905518,65.98510061)
\curveto(228.40905949,65.67509523)(228.61405928,65.38009553)(228.80405518,65.10010061)
\curveto(228.94405895,64.87009604)(229.07905882,64.64009627)(229.20905518,64.41010061)
\curveto(229.33905856,64.19009672)(229.47405842,63.97009694)(229.61405518,63.75010061)
\curveto(229.78405811,63.50009741)(229.96405793,63.26009765)(230.15405518,63.03010061)
\curveto(230.34405755,62.8100981)(230.56905733,62.62009829)(230.82905518,62.46010061)
\curveto(230.88905701,62.42009849)(230.94905695,62.38509852)(231.00905518,62.35510061)
\curveto(231.05905684,62.32509858)(231.12405677,62.29509861)(231.20405518,62.26510061)
\curveto(231.27405662,62.24509866)(231.33405656,62.24009867)(231.38405518,62.25010061)
\curveto(231.45405644,62.27009864)(231.50905639,62.3050986)(231.54905518,62.35510061)
\curveto(231.57905632,62.4050985)(231.5990563,62.46509844)(231.60905518,62.53510061)
\lineto(231.60905518,62.77510061)
\lineto(231.60905518,63.52510061)
\lineto(231.60905518,66.33010061)
\lineto(231.60905518,66.99010061)
\curveto(231.60905629,67.08009383)(231.61405628,67.16509374)(231.62405518,67.24510061)
\curveto(231.62405627,67.32509358)(231.64405625,67.39009352)(231.68405518,67.44010061)
\curveto(231.72405617,67.49009342)(231.7990561,67.53009338)(231.90905518,67.56010061)
\curveto(232.00905589,67.60009331)(232.10905579,67.6100933)(232.20905518,67.59010061)
\lineto(232.34405518,67.59010061)
\curveto(232.41405548,67.57009334)(232.47405542,67.55009336)(232.52405518,67.53010061)
\curveto(232.57405532,67.5100934)(232.61405528,67.47509343)(232.64405518,67.42510061)
\curveto(232.68405521,67.37509353)(232.70405519,67.3050936)(232.70405518,67.21510061)
\lineto(232.70405518,66.94510061)
\lineto(232.70405518,66.04510061)
\lineto(232.70405518,62.53510061)
\lineto(232.70405518,61.47010061)
\curveto(232.70405519,61.39009952)(232.70905519,61.30009961)(232.71905518,61.20010061)
\curveto(232.71905518,61.10009981)(232.70905519,61.01509989)(232.68905518,60.94510061)
\curveto(232.61905528,60.73510017)(232.43905546,60.67010024)(232.14905518,60.75010061)
\curveto(232.10905579,60.76010015)(232.07405582,60.76010015)(232.04405518,60.75010061)
\curveto(232.00405589,60.75010016)(231.95905594,60.76010015)(231.90905518,60.78010061)
\curveto(231.82905607,60.80010011)(231.74405615,60.82010009)(231.65405518,60.84010061)
\curveto(231.56405633,60.86010005)(231.47905642,60.88510002)(231.39905518,60.91510061)
\curveto(230.90905699,61.07509983)(230.4940574,61.27509963)(230.15405518,61.51510061)
\curveto(229.90405799,61.69509921)(229.67905822,61.90009901)(229.47905518,62.13010061)
\curveto(229.26905863,62.36009855)(229.07405882,62.60009831)(228.89405518,62.85010061)
\curveto(228.71405918,63.1100978)(228.54405935,63.37509753)(228.38405518,63.64510061)
\curveto(228.21405968,63.92509698)(228.03905986,64.19509671)(227.85905518,64.45510061)
\curveto(227.77906012,64.56509634)(227.70406019,64.67009624)(227.63405518,64.77010061)
\curveto(227.56406033,64.88009603)(227.48906041,64.99009592)(227.40905518,65.10010061)
\curveto(227.37906052,65.14009577)(227.34906055,65.17509573)(227.31905518,65.20510061)
\curveto(227.27906062,65.24509566)(227.24906065,65.28509562)(227.22905518,65.32510061)
\curveto(227.11906078,65.46509544)(226.9940609,65.59009532)(226.85405518,65.70010061)
\curveto(226.82406107,65.72009519)(226.7990611,65.74509516)(226.77905518,65.77510061)
\curveto(226.74906115,65.8050951)(226.71906118,65.83009508)(226.68905518,65.85010061)
\curveto(226.58906131,65.93009498)(226.48906141,65.99509491)(226.38905518,66.04510061)
\curveto(226.28906161,66.1050948)(226.17906172,66.16009475)(226.05905518,66.21010061)
\curveto(225.98906191,66.24009467)(225.91406198,66.26009465)(225.83405518,66.27010061)
\lineto(225.59405518,66.33010061)
\lineto(225.50405518,66.33010061)
\curveto(225.47406242,66.34009457)(225.44406245,66.34509456)(225.41405518,66.34510061)
\curveto(225.34406255,66.36509454)(225.24906265,66.37009454)(225.12905518,66.36010061)
\curveto(224.9990629,66.36009455)(224.899063,66.35009456)(224.82905518,66.33010061)
\curveto(224.74906315,66.3100946)(224.67406322,66.29009462)(224.60405518,66.27010061)
\curveto(224.52406337,66.26009465)(224.44406345,66.24009467)(224.36405518,66.21010061)
\curveto(224.12406377,66.10009481)(223.92406397,65.95009496)(223.76405518,65.76010061)
\curveto(223.5940643,65.58009533)(223.45406444,65.36009555)(223.34405518,65.10010061)
\curveto(223.32406457,65.03009588)(223.30906459,64.96009595)(223.29905518,64.89010061)
\curveto(223.27906462,64.82009609)(223.25906464,64.74509616)(223.23905518,64.66510061)
\curveto(223.21906468,64.58509632)(223.20906469,64.47509643)(223.20905518,64.33510061)
\curveto(223.20906469,64.2050967)(223.21906468,64.10009681)(223.23905518,64.02010061)
\curveto(223.24906465,63.96009695)(223.25406464,63.905097)(223.25405518,63.85510061)
\curveto(223.25406464,63.8050971)(223.26406463,63.75509715)(223.28405518,63.70510061)
\curveto(223.32406457,63.6050973)(223.36406453,63.5100974)(223.40405518,63.42010061)
\curveto(223.44406445,63.34009757)(223.48906441,63.26009765)(223.53905518,63.18010061)
\curveto(223.55906434,63.15009776)(223.58406431,63.12009779)(223.61405518,63.09010061)
\curveto(223.64406425,63.07009784)(223.66906423,63.04509786)(223.68905518,63.01510061)
\lineto(223.76405518,62.94010061)
\curveto(223.78406411,62.910098)(223.80406409,62.88509802)(223.82405518,62.86510061)
\lineto(224.03405518,62.71510061)
\curveto(224.0940638,62.67509823)(224.15906374,62.63009828)(224.22905518,62.58010061)
\curveto(224.31906358,62.52009839)(224.42406347,62.47009844)(224.54405518,62.43010061)
\curveto(224.65406324,62.40009851)(224.76406313,62.36509854)(224.87405518,62.32510061)
\curveto(224.98406291,62.28509862)(225.12906277,62.26009865)(225.30905518,62.25010061)
\curveto(225.47906242,62.24009867)(225.60406229,62.2100987)(225.68405518,62.16010061)
\curveto(225.76406213,62.1100988)(225.80906209,62.03509887)(225.81905518,61.93510061)
\curveto(225.82906207,61.83509907)(225.83406206,61.72509918)(225.83405518,61.60510061)
\curveto(225.83406206,61.56509934)(225.83906206,61.52509938)(225.84905518,61.48510061)
\curveto(225.84906205,61.44509946)(225.84406205,61.4100995)(225.83405518,61.38010061)
\curveto(225.81406208,61.33009958)(225.80406209,61.28009963)(225.80405518,61.23010061)
\curveto(225.80406209,61.19009972)(225.7940621,61.15009976)(225.77405518,61.11010061)
\curveto(225.71406218,61.02009989)(225.57906232,60.97509993)(225.36905518,60.97510061)
\lineto(225.24905518,60.97510061)
\curveto(225.18906271,60.98509992)(225.12906277,60.99009992)(225.06905518,60.99010061)
\curveto(224.9990629,61.00009991)(224.93406296,61.0100999)(224.87405518,61.02010061)
\curveto(224.76406313,61.04009987)(224.66406323,61.06009985)(224.57405518,61.08010061)
\curveto(224.47406342,61.10009981)(224.37906352,61.13009978)(224.28905518,61.17010061)
\curveto(224.21906368,61.19009972)(224.15906374,61.2100997)(224.10905518,61.23010061)
\lineto(223.92905518,61.29010061)
\curveto(223.66906423,61.4100995)(223.42406447,61.56509934)(223.19405518,61.75510061)
\curveto(222.96406493,61.95509895)(222.77906512,62.17009874)(222.63905518,62.40010061)
\curveto(222.55906534,62.5100984)(222.4940654,62.62509828)(222.44405518,62.74510061)
\lineto(222.29405518,63.13510061)
\curveto(222.24406565,63.24509766)(222.21406568,63.36009755)(222.20405518,63.48010061)
\curveto(222.18406571,63.60009731)(222.15906574,63.72509718)(222.12905518,63.85510061)
\curveto(222.12906577,63.92509698)(222.12906577,63.99009692)(222.12905518,64.05010061)
\curveto(222.11906578,64.1100968)(222.10906579,64.17509673)(222.09905518,64.24510061)
}
}
{
\newrgbcolor{curcolor}{0 0 0}
\pscustom[linestyle=none,fillstyle=solid,fillcolor=curcolor]
{
\newpath
\moveto(222.09905518,72.59470998)
\curveto(222.08906581,73.28470535)(222.20906569,73.88470475)(222.45905518,74.39470998)
\curveto(222.70906519,74.91470372)(223.04406485,75.30970332)(223.46405518,75.57970998)
\curveto(223.54406435,75.629703)(223.63406426,75.67470296)(223.73405518,75.71470998)
\curveto(223.82406407,75.75470288)(223.91906398,75.79970283)(224.01905518,75.84970998)
\curveto(224.11906378,75.88970274)(224.21906368,75.91970271)(224.31905518,75.93970998)
\curveto(224.41906348,75.95970267)(224.52406337,75.97970265)(224.63405518,75.99970998)
\curveto(224.68406321,76.01970261)(224.72906317,76.02470261)(224.76905518,76.01470998)
\curveto(224.80906309,76.00470263)(224.85406304,76.00970262)(224.90405518,76.02970998)
\curveto(224.95406294,76.03970259)(225.03906286,76.04470259)(225.15905518,76.04470998)
\curveto(225.26906263,76.04470259)(225.35406254,76.03970259)(225.41405518,76.02970998)
\curveto(225.47406242,76.00970262)(225.53406236,75.99970263)(225.59405518,75.99970998)
\curveto(225.65406224,76.00970262)(225.71406218,76.00470263)(225.77405518,75.98470998)
\curveto(225.91406198,75.94470269)(226.04906185,75.90970272)(226.17905518,75.87970998)
\curveto(226.30906159,75.84970278)(226.43406146,75.80970282)(226.55405518,75.75970998)
\curveto(226.6940612,75.69970293)(226.81906108,75.629703)(226.92905518,75.54970998)
\curveto(227.03906086,75.47970315)(227.14906075,75.40470323)(227.25905518,75.32470998)
\lineto(227.31905518,75.26470998)
\curveto(227.33906056,75.25470338)(227.35906054,75.23970339)(227.37905518,75.21970998)
\curveto(227.53906036,75.09970353)(227.68406021,74.96470367)(227.81405518,74.81470998)
\curveto(227.94405995,74.66470397)(228.06905983,74.50470413)(228.18905518,74.33470998)
\curveto(228.40905949,74.02470461)(228.61405928,73.7297049)(228.80405518,73.44970998)
\curveto(228.94405895,73.21970541)(229.07905882,72.98970564)(229.20905518,72.75970998)
\curveto(229.33905856,72.53970609)(229.47405842,72.31970631)(229.61405518,72.09970998)
\curveto(229.78405811,71.84970678)(229.96405793,71.60970702)(230.15405518,71.37970998)
\curveto(230.34405755,71.15970747)(230.56905733,70.96970766)(230.82905518,70.80970998)
\curveto(230.88905701,70.76970786)(230.94905695,70.7347079)(231.00905518,70.70470998)
\curveto(231.05905684,70.67470796)(231.12405677,70.64470799)(231.20405518,70.61470998)
\curveto(231.27405662,70.59470804)(231.33405656,70.58970804)(231.38405518,70.59970998)
\curveto(231.45405644,70.61970801)(231.50905639,70.65470798)(231.54905518,70.70470998)
\curveto(231.57905632,70.75470788)(231.5990563,70.81470782)(231.60905518,70.88470998)
\lineto(231.60905518,71.12470998)
\lineto(231.60905518,71.87470998)
\lineto(231.60905518,74.67970998)
\lineto(231.60905518,75.33970998)
\curveto(231.60905629,75.4297032)(231.61405628,75.51470312)(231.62405518,75.59470998)
\curveto(231.62405627,75.67470296)(231.64405625,75.73970289)(231.68405518,75.78970998)
\curveto(231.72405617,75.83970279)(231.7990561,75.87970275)(231.90905518,75.90970998)
\curveto(232.00905589,75.94970268)(232.10905579,75.95970267)(232.20905518,75.93970998)
\lineto(232.34405518,75.93970998)
\curveto(232.41405548,75.91970271)(232.47405542,75.89970273)(232.52405518,75.87970998)
\curveto(232.57405532,75.85970277)(232.61405528,75.82470281)(232.64405518,75.77470998)
\curveto(232.68405521,75.72470291)(232.70405519,75.65470298)(232.70405518,75.56470998)
\lineto(232.70405518,75.29470998)
\lineto(232.70405518,74.39470998)
\lineto(232.70405518,70.88470998)
\lineto(232.70405518,69.81970998)
\curveto(232.70405519,69.73970889)(232.70905519,69.64970898)(232.71905518,69.54970998)
\curveto(232.71905518,69.44970918)(232.70905519,69.36470927)(232.68905518,69.29470998)
\curveto(232.61905528,69.08470955)(232.43905546,69.01970961)(232.14905518,69.09970998)
\curveto(232.10905579,69.10970952)(232.07405582,69.10970952)(232.04405518,69.09970998)
\curveto(232.00405589,69.09970953)(231.95905594,69.10970952)(231.90905518,69.12970998)
\curveto(231.82905607,69.14970948)(231.74405615,69.16970946)(231.65405518,69.18970998)
\curveto(231.56405633,69.20970942)(231.47905642,69.2347094)(231.39905518,69.26470998)
\curveto(230.90905699,69.42470921)(230.4940574,69.62470901)(230.15405518,69.86470998)
\curveto(229.90405799,70.04470859)(229.67905822,70.24970838)(229.47905518,70.47970998)
\curveto(229.26905863,70.70970792)(229.07405882,70.94970768)(228.89405518,71.19970998)
\curveto(228.71405918,71.45970717)(228.54405935,71.72470691)(228.38405518,71.99470998)
\curveto(228.21405968,72.27470636)(228.03905986,72.54470609)(227.85905518,72.80470998)
\curveto(227.77906012,72.91470572)(227.70406019,73.01970561)(227.63405518,73.11970998)
\curveto(227.56406033,73.2297054)(227.48906041,73.33970529)(227.40905518,73.44970998)
\curveto(227.37906052,73.48970514)(227.34906055,73.52470511)(227.31905518,73.55470998)
\curveto(227.27906062,73.59470504)(227.24906065,73.634705)(227.22905518,73.67470998)
\curveto(227.11906078,73.81470482)(226.9940609,73.93970469)(226.85405518,74.04970998)
\curveto(226.82406107,74.06970456)(226.7990611,74.09470454)(226.77905518,74.12470998)
\curveto(226.74906115,74.15470448)(226.71906118,74.17970445)(226.68905518,74.19970998)
\curveto(226.58906131,74.27970435)(226.48906141,74.34470429)(226.38905518,74.39470998)
\curveto(226.28906161,74.45470418)(226.17906172,74.50970412)(226.05905518,74.55970998)
\curveto(225.98906191,74.58970404)(225.91406198,74.60970402)(225.83405518,74.61970998)
\lineto(225.59405518,74.67970998)
\lineto(225.50405518,74.67970998)
\curveto(225.47406242,74.68970394)(225.44406245,74.69470394)(225.41405518,74.69470998)
\curveto(225.34406255,74.71470392)(225.24906265,74.71970391)(225.12905518,74.70970998)
\curveto(224.9990629,74.70970392)(224.899063,74.69970393)(224.82905518,74.67970998)
\curveto(224.74906315,74.65970397)(224.67406322,74.63970399)(224.60405518,74.61970998)
\curveto(224.52406337,74.60970402)(224.44406345,74.58970404)(224.36405518,74.55970998)
\curveto(224.12406377,74.44970418)(223.92406397,74.29970433)(223.76405518,74.10970998)
\curveto(223.5940643,73.9297047)(223.45406444,73.70970492)(223.34405518,73.44970998)
\curveto(223.32406457,73.37970525)(223.30906459,73.30970532)(223.29905518,73.23970998)
\curveto(223.27906462,73.16970546)(223.25906464,73.09470554)(223.23905518,73.01470998)
\curveto(223.21906468,72.9347057)(223.20906469,72.82470581)(223.20905518,72.68470998)
\curveto(223.20906469,72.55470608)(223.21906468,72.44970618)(223.23905518,72.36970998)
\curveto(223.24906465,72.30970632)(223.25406464,72.25470638)(223.25405518,72.20470998)
\curveto(223.25406464,72.15470648)(223.26406463,72.10470653)(223.28405518,72.05470998)
\curveto(223.32406457,71.95470668)(223.36406453,71.85970677)(223.40405518,71.76970998)
\curveto(223.44406445,71.68970694)(223.48906441,71.60970702)(223.53905518,71.52970998)
\curveto(223.55906434,71.49970713)(223.58406431,71.46970716)(223.61405518,71.43970998)
\curveto(223.64406425,71.41970721)(223.66906423,71.39470724)(223.68905518,71.36470998)
\lineto(223.76405518,71.28970998)
\curveto(223.78406411,71.25970737)(223.80406409,71.2347074)(223.82405518,71.21470998)
\lineto(224.03405518,71.06470998)
\curveto(224.0940638,71.02470761)(224.15906374,70.97970765)(224.22905518,70.92970998)
\curveto(224.31906358,70.86970776)(224.42406347,70.81970781)(224.54405518,70.77970998)
\curveto(224.65406324,70.74970788)(224.76406313,70.71470792)(224.87405518,70.67470998)
\curveto(224.98406291,70.634708)(225.12906277,70.60970802)(225.30905518,70.59970998)
\curveto(225.47906242,70.58970804)(225.60406229,70.55970807)(225.68405518,70.50970998)
\curveto(225.76406213,70.45970817)(225.80906209,70.38470825)(225.81905518,70.28470998)
\curveto(225.82906207,70.18470845)(225.83406206,70.07470856)(225.83405518,69.95470998)
\curveto(225.83406206,69.91470872)(225.83906206,69.87470876)(225.84905518,69.83470998)
\curveto(225.84906205,69.79470884)(225.84406205,69.75970887)(225.83405518,69.72970998)
\curveto(225.81406208,69.67970895)(225.80406209,69.629709)(225.80405518,69.57970998)
\curveto(225.80406209,69.53970909)(225.7940621,69.49970913)(225.77405518,69.45970998)
\curveto(225.71406218,69.36970926)(225.57906232,69.32470931)(225.36905518,69.32470998)
\lineto(225.24905518,69.32470998)
\curveto(225.18906271,69.3347093)(225.12906277,69.33970929)(225.06905518,69.33970998)
\curveto(224.9990629,69.34970928)(224.93406296,69.35970927)(224.87405518,69.36970998)
\curveto(224.76406313,69.38970924)(224.66406323,69.40970922)(224.57405518,69.42970998)
\curveto(224.47406342,69.44970918)(224.37906352,69.47970915)(224.28905518,69.51970998)
\curveto(224.21906368,69.53970909)(224.15906374,69.55970907)(224.10905518,69.57970998)
\lineto(223.92905518,69.63970998)
\curveto(223.66906423,69.75970887)(223.42406447,69.91470872)(223.19405518,70.10470998)
\curveto(222.96406493,70.30470833)(222.77906512,70.51970811)(222.63905518,70.74970998)
\curveto(222.55906534,70.85970777)(222.4940654,70.97470766)(222.44405518,71.09470998)
\lineto(222.29405518,71.48470998)
\curveto(222.24406565,71.59470704)(222.21406568,71.70970692)(222.20405518,71.82970998)
\curveto(222.18406571,71.94970668)(222.15906574,72.07470656)(222.12905518,72.20470998)
\curveto(222.12906577,72.27470636)(222.12906577,72.33970629)(222.12905518,72.39970998)
\curveto(222.11906578,72.45970617)(222.10906579,72.52470611)(222.09905518,72.59470998)
}
}
{
\newrgbcolor{curcolor}{0 0 0}
\pscustom[linestyle=none,fillstyle=solid,fillcolor=curcolor]
{
\newpath
\moveto(231.06905518,78.63431936)
\lineto(231.06905518,79.26431936)
\lineto(231.06905518,79.45931936)
\curveto(231.06905683,79.52931683)(231.07905682,79.58931677)(231.09905518,79.63931936)
\curveto(231.13905676,79.70931665)(231.17905672,79.7593166)(231.21905518,79.78931936)
\curveto(231.26905663,79.82931653)(231.33405656,79.84931651)(231.41405518,79.84931936)
\curveto(231.4940564,79.8593165)(231.57905632,79.86431649)(231.66905518,79.86431936)
\lineto(232.38905518,79.86431936)
\curveto(232.86905503,79.86431649)(233.27905462,79.80431655)(233.61905518,79.68431936)
\curveto(233.95905394,79.56431679)(234.23405366,79.36931699)(234.44405518,79.09931936)
\curveto(234.4940534,79.02931733)(234.53905336,78.9593174)(234.57905518,78.88931936)
\curveto(234.62905327,78.82931753)(234.67405322,78.7543176)(234.71405518,78.66431936)
\curveto(234.72405317,78.64431771)(234.73405316,78.61431774)(234.74405518,78.57431936)
\curveto(234.76405313,78.53431782)(234.76905313,78.48931787)(234.75905518,78.43931936)
\curveto(234.72905317,78.34931801)(234.65405324,78.29431806)(234.53405518,78.27431936)
\curveto(234.42405347,78.2543181)(234.32905357,78.26931809)(234.24905518,78.31931936)
\curveto(234.17905372,78.34931801)(234.11405378,78.39431796)(234.05405518,78.45431936)
\curveto(234.00405389,78.52431783)(233.95405394,78.58931777)(233.90405518,78.64931936)
\curveto(233.85405404,78.71931764)(233.77905412,78.77931758)(233.67905518,78.82931936)
\curveto(233.58905431,78.88931747)(233.4990544,78.93931742)(233.40905518,78.97931936)
\curveto(233.37905452,78.99931736)(233.31905458,79.02431733)(233.22905518,79.05431936)
\curveto(233.14905475,79.08431727)(233.07905482,79.08931727)(233.01905518,79.06931936)
\curveto(232.87905502,79.03931732)(232.78905511,78.97931738)(232.74905518,78.88931936)
\curveto(232.71905518,78.80931755)(232.70405519,78.71931764)(232.70405518,78.61931936)
\curveto(232.70405519,78.51931784)(232.67905522,78.43431792)(232.62905518,78.36431936)
\curveto(232.55905534,78.27431808)(232.41905548,78.22931813)(232.20905518,78.22931936)
\lineto(231.65405518,78.22931936)
\lineto(231.42905518,78.22931936)
\curveto(231.34905655,78.23931812)(231.28405661,78.2593181)(231.23405518,78.28931936)
\curveto(231.15405674,78.34931801)(231.10905679,78.41931794)(231.09905518,78.49931936)
\curveto(231.08905681,78.51931784)(231.08405681,78.53931782)(231.08405518,78.55931936)
\curveto(231.08405681,78.58931777)(231.07905682,78.61431774)(231.06905518,78.63431936)
}
}
{
\newrgbcolor{curcolor}{0 0 0}
\pscustom[linestyle=none,fillstyle=solid,fillcolor=curcolor]
{
}
}
{
\newrgbcolor{curcolor}{0 0 0}
\pscustom[linestyle=none,fillstyle=solid,fillcolor=curcolor]
{
\newpath
\moveto(222.09905518,89.26463186)
\curveto(222.08906581,89.95462722)(222.20906569,90.55462662)(222.45905518,91.06463186)
\curveto(222.70906519,91.58462559)(223.04406485,91.9796252)(223.46405518,92.24963186)
\curveto(223.54406435,92.29962488)(223.63406426,92.34462483)(223.73405518,92.38463186)
\curveto(223.82406407,92.42462475)(223.91906398,92.46962471)(224.01905518,92.51963186)
\curveto(224.11906378,92.55962462)(224.21906368,92.58962459)(224.31905518,92.60963186)
\curveto(224.41906348,92.62962455)(224.52406337,92.64962453)(224.63405518,92.66963186)
\curveto(224.68406321,92.68962449)(224.72906317,92.69462448)(224.76905518,92.68463186)
\curveto(224.80906309,92.6746245)(224.85406304,92.6796245)(224.90405518,92.69963186)
\curveto(224.95406294,92.70962447)(225.03906286,92.71462446)(225.15905518,92.71463186)
\curveto(225.26906263,92.71462446)(225.35406254,92.70962447)(225.41405518,92.69963186)
\curveto(225.47406242,92.6796245)(225.53406236,92.66962451)(225.59405518,92.66963186)
\curveto(225.65406224,92.6796245)(225.71406218,92.6746245)(225.77405518,92.65463186)
\curveto(225.91406198,92.61462456)(226.04906185,92.5796246)(226.17905518,92.54963186)
\curveto(226.30906159,92.51962466)(226.43406146,92.4796247)(226.55405518,92.42963186)
\curveto(226.6940612,92.36962481)(226.81906108,92.29962488)(226.92905518,92.21963186)
\curveto(227.03906086,92.14962503)(227.14906075,92.0746251)(227.25905518,91.99463186)
\lineto(227.31905518,91.93463186)
\curveto(227.33906056,91.92462525)(227.35906054,91.90962527)(227.37905518,91.88963186)
\curveto(227.53906036,91.76962541)(227.68406021,91.63462554)(227.81405518,91.48463186)
\curveto(227.94405995,91.33462584)(228.06905983,91.174626)(228.18905518,91.00463186)
\curveto(228.40905949,90.69462648)(228.61405928,90.39962678)(228.80405518,90.11963186)
\curveto(228.94405895,89.88962729)(229.07905882,89.65962752)(229.20905518,89.42963186)
\curveto(229.33905856,89.20962797)(229.47405842,88.98962819)(229.61405518,88.76963186)
\curveto(229.78405811,88.51962866)(229.96405793,88.2796289)(230.15405518,88.04963186)
\curveto(230.34405755,87.82962935)(230.56905733,87.63962954)(230.82905518,87.47963186)
\curveto(230.88905701,87.43962974)(230.94905695,87.40462977)(231.00905518,87.37463186)
\curveto(231.05905684,87.34462983)(231.12405677,87.31462986)(231.20405518,87.28463186)
\curveto(231.27405662,87.26462991)(231.33405656,87.25962992)(231.38405518,87.26963186)
\curveto(231.45405644,87.28962989)(231.50905639,87.32462985)(231.54905518,87.37463186)
\curveto(231.57905632,87.42462975)(231.5990563,87.48462969)(231.60905518,87.55463186)
\lineto(231.60905518,87.79463186)
\lineto(231.60905518,88.54463186)
\lineto(231.60905518,91.34963186)
\lineto(231.60905518,92.00963186)
\curveto(231.60905629,92.09962508)(231.61405628,92.18462499)(231.62405518,92.26463186)
\curveto(231.62405627,92.34462483)(231.64405625,92.40962477)(231.68405518,92.45963186)
\curveto(231.72405617,92.50962467)(231.7990561,92.54962463)(231.90905518,92.57963186)
\curveto(232.00905589,92.61962456)(232.10905579,92.62962455)(232.20905518,92.60963186)
\lineto(232.34405518,92.60963186)
\curveto(232.41405548,92.58962459)(232.47405542,92.56962461)(232.52405518,92.54963186)
\curveto(232.57405532,92.52962465)(232.61405528,92.49462468)(232.64405518,92.44463186)
\curveto(232.68405521,92.39462478)(232.70405519,92.32462485)(232.70405518,92.23463186)
\lineto(232.70405518,91.96463186)
\lineto(232.70405518,91.06463186)
\lineto(232.70405518,87.55463186)
\lineto(232.70405518,86.48963186)
\curveto(232.70405519,86.40963077)(232.70905519,86.31963086)(232.71905518,86.21963186)
\curveto(232.71905518,86.11963106)(232.70905519,86.03463114)(232.68905518,85.96463186)
\curveto(232.61905528,85.75463142)(232.43905546,85.68963149)(232.14905518,85.76963186)
\curveto(232.10905579,85.7796314)(232.07405582,85.7796314)(232.04405518,85.76963186)
\curveto(232.00405589,85.76963141)(231.95905594,85.7796314)(231.90905518,85.79963186)
\curveto(231.82905607,85.81963136)(231.74405615,85.83963134)(231.65405518,85.85963186)
\curveto(231.56405633,85.8796313)(231.47905642,85.90463127)(231.39905518,85.93463186)
\curveto(230.90905699,86.09463108)(230.4940574,86.29463088)(230.15405518,86.53463186)
\curveto(229.90405799,86.71463046)(229.67905822,86.91963026)(229.47905518,87.14963186)
\curveto(229.26905863,87.3796298)(229.07405882,87.61962956)(228.89405518,87.86963186)
\curveto(228.71405918,88.12962905)(228.54405935,88.39462878)(228.38405518,88.66463186)
\curveto(228.21405968,88.94462823)(228.03905986,89.21462796)(227.85905518,89.47463186)
\curveto(227.77906012,89.58462759)(227.70406019,89.68962749)(227.63405518,89.78963186)
\curveto(227.56406033,89.89962728)(227.48906041,90.00962717)(227.40905518,90.11963186)
\curveto(227.37906052,90.15962702)(227.34906055,90.19462698)(227.31905518,90.22463186)
\curveto(227.27906062,90.26462691)(227.24906065,90.30462687)(227.22905518,90.34463186)
\curveto(227.11906078,90.48462669)(226.9940609,90.60962657)(226.85405518,90.71963186)
\curveto(226.82406107,90.73962644)(226.7990611,90.76462641)(226.77905518,90.79463186)
\curveto(226.74906115,90.82462635)(226.71906118,90.84962633)(226.68905518,90.86963186)
\curveto(226.58906131,90.94962623)(226.48906141,91.01462616)(226.38905518,91.06463186)
\curveto(226.28906161,91.12462605)(226.17906172,91.179626)(226.05905518,91.22963186)
\curveto(225.98906191,91.25962592)(225.91406198,91.2796259)(225.83405518,91.28963186)
\lineto(225.59405518,91.34963186)
\lineto(225.50405518,91.34963186)
\curveto(225.47406242,91.35962582)(225.44406245,91.36462581)(225.41405518,91.36463186)
\curveto(225.34406255,91.38462579)(225.24906265,91.38962579)(225.12905518,91.37963186)
\curveto(224.9990629,91.3796258)(224.899063,91.36962581)(224.82905518,91.34963186)
\curveto(224.74906315,91.32962585)(224.67406322,91.30962587)(224.60405518,91.28963186)
\curveto(224.52406337,91.2796259)(224.44406345,91.25962592)(224.36405518,91.22963186)
\curveto(224.12406377,91.11962606)(223.92406397,90.96962621)(223.76405518,90.77963186)
\curveto(223.5940643,90.59962658)(223.45406444,90.3796268)(223.34405518,90.11963186)
\curveto(223.32406457,90.04962713)(223.30906459,89.9796272)(223.29905518,89.90963186)
\curveto(223.27906462,89.83962734)(223.25906464,89.76462741)(223.23905518,89.68463186)
\curveto(223.21906468,89.60462757)(223.20906469,89.49462768)(223.20905518,89.35463186)
\curveto(223.20906469,89.22462795)(223.21906468,89.11962806)(223.23905518,89.03963186)
\curveto(223.24906465,88.9796282)(223.25406464,88.92462825)(223.25405518,88.87463186)
\curveto(223.25406464,88.82462835)(223.26406463,88.7746284)(223.28405518,88.72463186)
\curveto(223.32406457,88.62462855)(223.36406453,88.52962865)(223.40405518,88.43963186)
\curveto(223.44406445,88.35962882)(223.48906441,88.2796289)(223.53905518,88.19963186)
\curveto(223.55906434,88.16962901)(223.58406431,88.13962904)(223.61405518,88.10963186)
\curveto(223.64406425,88.08962909)(223.66906423,88.06462911)(223.68905518,88.03463186)
\lineto(223.76405518,87.95963186)
\curveto(223.78406411,87.92962925)(223.80406409,87.90462927)(223.82405518,87.88463186)
\lineto(224.03405518,87.73463186)
\curveto(224.0940638,87.69462948)(224.15906374,87.64962953)(224.22905518,87.59963186)
\curveto(224.31906358,87.53962964)(224.42406347,87.48962969)(224.54405518,87.44963186)
\curveto(224.65406324,87.41962976)(224.76406313,87.38462979)(224.87405518,87.34463186)
\curveto(224.98406291,87.30462987)(225.12906277,87.2796299)(225.30905518,87.26963186)
\curveto(225.47906242,87.25962992)(225.60406229,87.22962995)(225.68405518,87.17963186)
\curveto(225.76406213,87.12963005)(225.80906209,87.05463012)(225.81905518,86.95463186)
\curveto(225.82906207,86.85463032)(225.83406206,86.74463043)(225.83405518,86.62463186)
\curveto(225.83406206,86.58463059)(225.83906206,86.54463063)(225.84905518,86.50463186)
\curveto(225.84906205,86.46463071)(225.84406205,86.42963075)(225.83405518,86.39963186)
\curveto(225.81406208,86.34963083)(225.80406209,86.29963088)(225.80405518,86.24963186)
\curveto(225.80406209,86.20963097)(225.7940621,86.16963101)(225.77405518,86.12963186)
\curveto(225.71406218,86.03963114)(225.57906232,85.99463118)(225.36905518,85.99463186)
\lineto(225.24905518,85.99463186)
\curveto(225.18906271,86.00463117)(225.12906277,86.00963117)(225.06905518,86.00963186)
\curveto(224.9990629,86.01963116)(224.93406296,86.02963115)(224.87405518,86.03963186)
\curveto(224.76406313,86.05963112)(224.66406323,86.0796311)(224.57405518,86.09963186)
\curveto(224.47406342,86.11963106)(224.37906352,86.14963103)(224.28905518,86.18963186)
\curveto(224.21906368,86.20963097)(224.15906374,86.22963095)(224.10905518,86.24963186)
\lineto(223.92905518,86.30963186)
\curveto(223.66906423,86.42963075)(223.42406447,86.58463059)(223.19405518,86.77463186)
\curveto(222.96406493,86.9746302)(222.77906512,87.18962999)(222.63905518,87.41963186)
\curveto(222.55906534,87.52962965)(222.4940654,87.64462953)(222.44405518,87.76463186)
\lineto(222.29405518,88.15463186)
\curveto(222.24406565,88.26462891)(222.21406568,88.3796288)(222.20405518,88.49963186)
\curveto(222.18406571,88.61962856)(222.15906574,88.74462843)(222.12905518,88.87463186)
\curveto(222.12906577,88.94462823)(222.12906577,89.00962817)(222.12905518,89.06963186)
\curveto(222.11906578,89.12962805)(222.10906579,89.19462798)(222.09905518,89.26463186)
}
}
{
\newrgbcolor{curcolor}{0 0 0}
\pscustom[linestyle=none,fillstyle=solid,fillcolor=curcolor]
{
\newpath
\moveto(227.61905518,101.36424123)
\lineto(227.87405518,101.36424123)
\curveto(227.95405994,101.37423353)(228.02905987,101.36923353)(228.09905518,101.34924123)
\lineto(228.33905518,101.34924123)
\lineto(228.50405518,101.34924123)
\curveto(228.60405929,101.32923357)(228.70905919,101.31923358)(228.81905518,101.31924123)
\curveto(228.91905898,101.31923358)(229.01905888,101.30923359)(229.11905518,101.28924123)
\lineto(229.26905518,101.28924123)
\curveto(229.40905849,101.25923364)(229.54905835,101.23923366)(229.68905518,101.22924123)
\curveto(229.81905808,101.21923368)(229.94905795,101.19423371)(230.07905518,101.15424123)
\curveto(230.15905774,101.13423377)(230.24405765,101.11423379)(230.33405518,101.09424123)
\lineto(230.57405518,101.03424123)
\lineto(230.87405518,100.91424123)
\curveto(230.96405693,100.88423402)(231.05405684,100.84923405)(231.14405518,100.80924123)
\curveto(231.36405653,100.70923419)(231.57905632,100.57423433)(231.78905518,100.40424123)
\curveto(231.9990559,100.24423466)(232.16905573,100.06923483)(232.29905518,99.87924123)
\curveto(232.33905556,99.82923507)(232.37905552,99.76923513)(232.41905518,99.69924123)
\curveto(232.44905545,99.63923526)(232.48405541,99.57923532)(232.52405518,99.51924123)
\curveto(232.57405532,99.43923546)(232.61405528,99.34423556)(232.64405518,99.23424123)
\curveto(232.67405522,99.12423578)(232.70405519,99.01923588)(232.73405518,98.91924123)
\curveto(232.77405512,98.80923609)(232.7990551,98.6992362)(232.80905518,98.58924123)
\curveto(232.81905508,98.47923642)(232.83405506,98.36423654)(232.85405518,98.24424123)
\curveto(232.86405503,98.2042367)(232.86405503,98.15923674)(232.85405518,98.10924123)
\curveto(232.85405504,98.06923683)(232.85905504,98.02923687)(232.86905518,97.98924123)
\curveto(232.87905502,97.94923695)(232.88405501,97.89423701)(232.88405518,97.82424123)
\curveto(232.88405501,97.75423715)(232.87905502,97.7042372)(232.86905518,97.67424123)
\curveto(232.84905505,97.62423728)(232.84405505,97.57923732)(232.85405518,97.53924123)
\curveto(232.86405503,97.4992374)(232.86405503,97.46423744)(232.85405518,97.43424123)
\lineto(232.85405518,97.34424123)
\curveto(232.83405506,97.28423762)(232.81905508,97.21923768)(232.80905518,97.14924123)
\curveto(232.80905509,97.08923781)(232.80405509,97.02423788)(232.79405518,96.95424123)
\curveto(232.74405515,96.78423812)(232.6940552,96.62423828)(232.64405518,96.47424123)
\curveto(232.5940553,96.32423858)(232.52905537,96.17923872)(232.44905518,96.03924123)
\curveto(232.40905549,95.98923891)(232.37905552,95.93423897)(232.35905518,95.87424123)
\curveto(232.32905557,95.82423908)(232.2940556,95.77423913)(232.25405518,95.72424123)
\curveto(232.07405582,95.48423942)(231.85405604,95.28423962)(231.59405518,95.12424123)
\curveto(231.33405656,94.96423994)(231.04905685,94.82424008)(230.73905518,94.70424123)
\curveto(230.5990573,94.64424026)(230.45905744,94.5992403)(230.31905518,94.56924123)
\curveto(230.16905773,94.53924036)(230.01405788,94.5042404)(229.85405518,94.46424123)
\curveto(229.74405815,94.44424046)(229.63405826,94.42924047)(229.52405518,94.41924123)
\curveto(229.41405848,94.40924049)(229.30405859,94.39424051)(229.19405518,94.37424123)
\curveto(229.15405874,94.36424054)(229.11405878,94.35924054)(229.07405518,94.35924123)
\curveto(229.03405886,94.36924053)(228.9940589,94.36924053)(228.95405518,94.35924123)
\curveto(228.90405899,94.34924055)(228.85405904,94.34424056)(228.80405518,94.34424123)
\lineto(228.63905518,94.34424123)
\curveto(228.58905931,94.32424058)(228.53905936,94.31924058)(228.48905518,94.32924123)
\curveto(228.42905947,94.33924056)(228.37405952,94.33924056)(228.32405518,94.32924123)
\curveto(228.28405961,94.31924058)(228.23905966,94.31924058)(228.18905518,94.32924123)
\curveto(228.13905976,94.33924056)(228.08905981,94.33424057)(228.03905518,94.31424123)
\curveto(227.96905993,94.29424061)(227.89406,94.28924061)(227.81405518,94.29924123)
\curveto(227.72406017,94.30924059)(227.63906026,94.31424059)(227.55905518,94.31424123)
\curveto(227.46906043,94.31424059)(227.36906053,94.30924059)(227.25905518,94.29924123)
\curveto(227.13906076,94.28924061)(227.03906086,94.29424061)(226.95905518,94.31424123)
\lineto(226.67405518,94.31424123)
\lineto(226.04405518,94.35924123)
\curveto(225.94406195,94.36924053)(225.84906205,94.37924052)(225.75905518,94.38924123)
\lineto(225.45905518,94.41924123)
\curveto(225.40906249,94.43924046)(225.35906254,94.44424046)(225.30905518,94.43424123)
\curveto(225.24906265,94.43424047)(225.1940627,94.44424046)(225.14405518,94.46424123)
\curveto(224.97406292,94.51424039)(224.80906309,94.55424035)(224.64905518,94.58424123)
\curveto(224.47906342,94.61424029)(224.31906358,94.66424024)(224.16905518,94.73424123)
\curveto(223.70906419,94.92423998)(223.33406456,95.14423976)(223.04405518,95.39424123)
\curveto(222.75406514,95.65423925)(222.50906539,96.01423889)(222.30905518,96.47424123)
\curveto(222.25906564,96.6042383)(222.22406567,96.73423817)(222.20405518,96.86424123)
\curveto(222.18406571,97.0042379)(222.15906574,97.14423776)(222.12905518,97.28424123)
\curveto(222.11906578,97.35423755)(222.11406578,97.41923748)(222.11405518,97.47924123)
\curveto(222.11406578,97.53923736)(222.10906579,97.6042373)(222.09905518,97.67424123)
\curveto(222.07906582,98.5042364)(222.22906567,99.17423573)(222.54905518,99.68424123)
\curveto(222.85906504,100.19423471)(223.2990646,100.57423433)(223.86905518,100.82424123)
\curveto(223.98906391,100.87423403)(224.11406378,100.91923398)(224.24405518,100.95924123)
\curveto(224.37406352,100.9992339)(224.50906339,101.04423386)(224.64905518,101.09424123)
\curveto(224.72906317,101.11423379)(224.81406308,101.12923377)(224.90405518,101.13924123)
\lineto(225.14405518,101.19924123)
\curveto(225.25406264,101.22923367)(225.36406253,101.24423366)(225.47405518,101.24424123)
\curveto(225.58406231,101.25423365)(225.6940622,101.26923363)(225.80405518,101.28924123)
\curveto(225.85406204,101.30923359)(225.899062,101.31423359)(225.93905518,101.30424123)
\curveto(225.97906192,101.3042336)(226.01906188,101.30923359)(226.05905518,101.31924123)
\curveto(226.10906179,101.32923357)(226.16406173,101.32923357)(226.22405518,101.31924123)
\curveto(226.27406162,101.31923358)(226.32406157,101.32423358)(226.37405518,101.33424123)
\lineto(226.50905518,101.33424123)
\curveto(226.56906133,101.35423355)(226.63906126,101.35423355)(226.71905518,101.33424123)
\curveto(226.78906111,101.32423358)(226.85406104,101.32923357)(226.91405518,101.34924123)
\curveto(226.94406095,101.35923354)(226.98406091,101.36423354)(227.03405518,101.36424123)
\lineto(227.15405518,101.36424123)
\lineto(227.61905518,101.36424123)
\moveto(229.94405518,99.81924123)
\curveto(229.62405827,99.91923498)(229.25905864,99.97923492)(228.84905518,99.99924123)
\curveto(228.43905946,100.01923488)(228.02905987,100.02923487)(227.61905518,100.02924123)
\curveto(227.18906071,100.02923487)(226.76906113,100.01923488)(226.35905518,99.99924123)
\curveto(225.94906195,99.97923492)(225.56406233,99.93423497)(225.20405518,99.86424123)
\curveto(224.84406305,99.79423511)(224.52406337,99.68423522)(224.24405518,99.53424123)
\curveto(223.95406394,99.39423551)(223.71906418,99.1992357)(223.53905518,98.94924123)
\curveto(223.42906447,98.78923611)(223.34906455,98.60923629)(223.29905518,98.40924123)
\curveto(223.23906466,98.20923669)(223.20906469,97.96423694)(223.20905518,97.67424123)
\curveto(223.22906467,97.65423725)(223.23906466,97.61923728)(223.23905518,97.56924123)
\curveto(223.22906467,97.51923738)(223.22906467,97.47923742)(223.23905518,97.44924123)
\curveto(223.25906464,97.36923753)(223.27906462,97.29423761)(223.29905518,97.22424123)
\curveto(223.30906459,97.16423774)(223.32906457,97.0992378)(223.35905518,97.02924123)
\curveto(223.47906442,96.75923814)(223.64906425,96.53923836)(223.86905518,96.36924123)
\curveto(224.07906382,96.20923869)(224.32406357,96.07423883)(224.60405518,95.96424123)
\curveto(224.71406318,95.91423899)(224.83406306,95.87423903)(224.96405518,95.84424123)
\curveto(225.08406281,95.82423908)(225.20906269,95.7992391)(225.33905518,95.76924123)
\curveto(225.38906251,95.74923915)(225.44406245,95.73923916)(225.50405518,95.73924123)
\curveto(225.55406234,95.73923916)(225.60406229,95.73423917)(225.65405518,95.72424123)
\curveto(225.74406215,95.71423919)(225.83906206,95.7042392)(225.93905518,95.69424123)
\curveto(226.02906187,95.68423922)(226.12406177,95.67423923)(226.22405518,95.66424123)
\curveto(226.30406159,95.66423924)(226.38906151,95.65923924)(226.47905518,95.64924123)
\lineto(226.71905518,95.64924123)
\lineto(226.89905518,95.64924123)
\curveto(226.92906097,95.63923926)(226.96406093,95.63423927)(227.00405518,95.63424123)
\lineto(227.13905518,95.63424123)
\lineto(227.58905518,95.63424123)
\curveto(227.66906023,95.63423927)(227.75406014,95.62923927)(227.84405518,95.61924123)
\curveto(227.92405997,95.61923928)(227.9990599,95.62923927)(228.06905518,95.64924123)
\lineto(228.33905518,95.64924123)
\curveto(228.35905954,95.64923925)(228.38905951,95.64423926)(228.42905518,95.63424123)
\curveto(228.45905944,95.63423927)(228.48405941,95.63923926)(228.50405518,95.64924123)
\curveto(228.60405929,95.65923924)(228.70405919,95.66423924)(228.80405518,95.66424123)
\curveto(228.894059,95.67423923)(228.9940589,95.68423922)(229.10405518,95.69424123)
\curveto(229.22405867,95.72423918)(229.34905855,95.73923916)(229.47905518,95.73924123)
\curveto(229.5990583,95.74923915)(229.71405818,95.77423913)(229.82405518,95.81424123)
\curveto(230.12405777,95.89423901)(230.38905751,95.97923892)(230.61905518,96.06924123)
\curveto(230.84905705,96.16923873)(231.06405683,96.31423859)(231.26405518,96.50424123)
\curveto(231.46405643,96.71423819)(231.61405628,96.97923792)(231.71405518,97.29924123)
\curveto(231.73405616,97.33923756)(231.74405615,97.37423753)(231.74405518,97.40424123)
\curveto(231.73405616,97.44423746)(231.73905616,97.48923741)(231.75905518,97.53924123)
\curveto(231.76905613,97.57923732)(231.77905612,97.64923725)(231.78905518,97.74924123)
\curveto(231.7990561,97.85923704)(231.7940561,97.94423696)(231.77405518,98.00424123)
\curveto(231.75405614,98.07423683)(231.74405615,98.14423676)(231.74405518,98.21424123)
\curveto(231.73405616,98.28423662)(231.71905618,98.34923655)(231.69905518,98.40924123)
\curveto(231.63905626,98.60923629)(231.55405634,98.78923611)(231.44405518,98.94924123)
\curveto(231.42405647,98.97923592)(231.40405649,99.0042359)(231.38405518,99.02424123)
\lineto(231.32405518,99.08424123)
\curveto(231.30405659,99.12423578)(231.26405663,99.17423573)(231.20405518,99.23424123)
\curveto(231.06405683,99.33423557)(230.93405696,99.41923548)(230.81405518,99.48924123)
\curveto(230.6940572,99.55923534)(230.54905735,99.62923527)(230.37905518,99.69924123)
\curveto(230.30905759,99.72923517)(230.23905766,99.74923515)(230.16905518,99.75924123)
\curveto(230.0990578,99.77923512)(230.02405787,99.7992351)(229.94405518,99.81924123)
}
}
{
\newrgbcolor{curcolor}{0 0 0}
\pscustom[linestyle=none,fillstyle=solid,fillcolor=curcolor]
{
\newpath
\moveto(222.09905518,106.77385061)
\curveto(222.0990658,106.87384575)(222.10906579,106.96884566)(222.12905518,107.05885061)
\curveto(222.13906576,107.14884548)(222.16906573,107.21384541)(222.21905518,107.25385061)
\curveto(222.2990656,107.31384531)(222.40406549,107.34384528)(222.53405518,107.34385061)
\lineto(222.92405518,107.34385061)
\lineto(224.42405518,107.34385061)
\lineto(230.81405518,107.34385061)
\lineto(231.98405518,107.34385061)
\lineto(232.29905518,107.34385061)
\curveto(232.3990555,107.35384527)(232.47905542,107.33884529)(232.53905518,107.29885061)
\curveto(232.61905528,107.24884538)(232.66905523,107.17384545)(232.68905518,107.07385061)
\curveto(232.6990552,106.98384564)(232.70405519,106.87384575)(232.70405518,106.74385061)
\lineto(232.70405518,106.51885061)
\curveto(232.68405521,106.43884619)(232.66905523,106.36884626)(232.65905518,106.30885061)
\curveto(232.63905526,106.24884638)(232.5990553,106.19884643)(232.53905518,106.15885061)
\curveto(232.47905542,106.11884651)(232.40405549,106.09884653)(232.31405518,106.09885061)
\lineto(232.01405518,106.09885061)
\lineto(230.91905518,106.09885061)
\lineto(225.57905518,106.09885061)
\curveto(225.48906241,106.07884655)(225.41406248,106.06384656)(225.35405518,106.05385061)
\curveto(225.28406261,106.05384657)(225.22406267,106.0238466)(225.17405518,105.96385061)
\curveto(225.12406277,105.89384673)(225.0990628,105.80384682)(225.09905518,105.69385061)
\curveto(225.08906281,105.59384703)(225.08406281,105.48384714)(225.08405518,105.36385061)
\lineto(225.08405518,104.22385061)
\lineto(225.08405518,103.72885061)
\curveto(225.07406282,103.56884906)(225.01406288,103.45884917)(224.90405518,103.39885061)
\curveto(224.87406302,103.37884925)(224.84406305,103.36884926)(224.81405518,103.36885061)
\curveto(224.77406312,103.36884926)(224.72906317,103.36384926)(224.67905518,103.35385061)
\curveto(224.55906334,103.33384929)(224.44906345,103.33884929)(224.34905518,103.36885061)
\curveto(224.24906365,103.40884922)(224.17906372,103.46384916)(224.13905518,103.53385061)
\curveto(224.08906381,103.61384901)(224.06406383,103.73384889)(224.06405518,103.89385061)
\curveto(224.06406383,104.05384857)(224.04906385,104.18884844)(224.01905518,104.29885061)
\curveto(224.00906389,104.34884828)(224.00406389,104.40384822)(224.00405518,104.46385061)
\curveto(223.9940639,104.5238481)(223.97906392,104.58384804)(223.95905518,104.64385061)
\curveto(223.90906399,104.79384783)(223.85906404,104.93884769)(223.80905518,105.07885061)
\curveto(223.74906415,105.21884741)(223.67906422,105.35384727)(223.59905518,105.48385061)
\curveto(223.50906439,105.623847)(223.40406449,105.74384688)(223.28405518,105.84385061)
\curveto(223.16406473,105.94384668)(223.03406486,106.03884659)(222.89405518,106.12885061)
\curveto(222.7940651,106.18884644)(222.68406521,106.23384639)(222.56405518,106.26385061)
\curveto(222.44406545,106.30384632)(222.33906556,106.35384627)(222.24905518,106.41385061)
\curveto(222.18906571,106.46384616)(222.14906575,106.53384609)(222.12905518,106.62385061)
\curveto(222.11906578,106.64384598)(222.11406578,106.66884596)(222.11405518,106.69885061)
\curveto(222.11406578,106.7288459)(222.10906579,106.75384587)(222.09905518,106.77385061)
}
}
{
\newrgbcolor{curcolor}{0 0 0}
\pscustom[linestyle=none,fillstyle=solid,fillcolor=curcolor]
{
\newpath
\moveto(222.09905518,115.12345998)
\curveto(222.0990658,115.22345513)(222.10906579,115.31845503)(222.12905518,115.40845998)
\curveto(222.13906576,115.49845485)(222.16906573,115.56345479)(222.21905518,115.60345998)
\curveto(222.2990656,115.66345469)(222.40406549,115.69345466)(222.53405518,115.69345998)
\lineto(222.92405518,115.69345998)
\lineto(224.42405518,115.69345998)
\lineto(230.81405518,115.69345998)
\lineto(231.98405518,115.69345998)
\lineto(232.29905518,115.69345998)
\curveto(232.3990555,115.70345465)(232.47905542,115.68845466)(232.53905518,115.64845998)
\curveto(232.61905528,115.59845475)(232.66905523,115.52345483)(232.68905518,115.42345998)
\curveto(232.6990552,115.33345502)(232.70405519,115.22345513)(232.70405518,115.09345998)
\lineto(232.70405518,114.86845998)
\curveto(232.68405521,114.78845556)(232.66905523,114.71845563)(232.65905518,114.65845998)
\curveto(232.63905526,114.59845575)(232.5990553,114.5484558)(232.53905518,114.50845998)
\curveto(232.47905542,114.46845588)(232.40405549,114.4484559)(232.31405518,114.44845998)
\lineto(232.01405518,114.44845998)
\lineto(230.91905518,114.44845998)
\lineto(225.57905518,114.44845998)
\curveto(225.48906241,114.42845592)(225.41406248,114.41345594)(225.35405518,114.40345998)
\curveto(225.28406261,114.40345595)(225.22406267,114.37345598)(225.17405518,114.31345998)
\curveto(225.12406277,114.24345611)(225.0990628,114.1534562)(225.09905518,114.04345998)
\curveto(225.08906281,113.94345641)(225.08406281,113.83345652)(225.08405518,113.71345998)
\lineto(225.08405518,112.57345998)
\lineto(225.08405518,112.07845998)
\curveto(225.07406282,111.91845843)(225.01406288,111.80845854)(224.90405518,111.74845998)
\curveto(224.87406302,111.72845862)(224.84406305,111.71845863)(224.81405518,111.71845998)
\curveto(224.77406312,111.71845863)(224.72906317,111.71345864)(224.67905518,111.70345998)
\curveto(224.55906334,111.68345867)(224.44906345,111.68845866)(224.34905518,111.71845998)
\curveto(224.24906365,111.75845859)(224.17906372,111.81345854)(224.13905518,111.88345998)
\curveto(224.08906381,111.96345839)(224.06406383,112.08345827)(224.06405518,112.24345998)
\curveto(224.06406383,112.40345795)(224.04906385,112.53845781)(224.01905518,112.64845998)
\curveto(224.00906389,112.69845765)(224.00406389,112.7534576)(224.00405518,112.81345998)
\curveto(223.9940639,112.87345748)(223.97906392,112.93345742)(223.95905518,112.99345998)
\curveto(223.90906399,113.14345721)(223.85906404,113.28845706)(223.80905518,113.42845998)
\curveto(223.74906415,113.56845678)(223.67906422,113.70345665)(223.59905518,113.83345998)
\curveto(223.50906439,113.97345638)(223.40406449,114.09345626)(223.28405518,114.19345998)
\curveto(223.16406473,114.29345606)(223.03406486,114.38845596)(222.89405518,114.47845998)
\curveto(222.7940651,114.53845581)(222.68406521,114.58345577)(222.56405518,114.61345998)
\curveto(222.44406545,114.6534557)(222.33906556,114.70345565)(222.24905518,114.76345998)
\curveto(222.18906571,114.81345554)(222.14906575,114.88345547)(222.12905518,114.97345998)
\curveto(222.11906578,114.99345536)(222.11406578,115.01845533)(222.11405518,115.04845998)
\curveto(222.11406578,115.07845527)(222.10906579,115.10345525)(222.09905518,115.12345998)
}
}
{
\newrgbcolor{curcolor}{0 0 0}
\pscustom[linestyle=none,fillstyle=solid,fillcolor=curcolor]
{
\newpath
\moveto(242.93538635,31.67142873)
\lineto(242.93538635,32.58642873)
\curveto(242.93539705,32.68642608)(242.93539705,32.78142599)(242.93538635,32.87142873)
\curveto(242.93539705,32.96142581)(242.95539703,33.03642573)(242.99538635,33.09642873)
\curveto(243.05539693,33.18642558)(243.13539685,33.24642552)(243.23538635,33.27642873)
\curveto(243.33539665,33.31642545)(243.44039654,33.36142541)(243.55038635,33.41142873)
\curveto(243.74039624,33.49142528)(243.93039605,33.56142521)(244.12038635,33.62142873)
\curveto(244.31039567,33.69142508)(244.50039548,33.766425)(244.69038635,33.84642873)
\curveto(244.87039511,33.91642485)(245.05539493,33.98142479)(245.24538635,34.04142873)
\curveto(245.42539456,34.10142467)(245.60539438,34.1714246)(245.78538635,34.25142873)
\curveto(245.92539406,34.31142446)(246.07039391,34.3664244)(246.22038635,34.41642873)
\curveto(246.37039361,34.4664243)(246.51539347,34.52142425)(246.65538635,34.58142873)
\curveto(247.10539288,34.76142401)(247.56039242,34.93142384)(248.02038635,35.09142873)
\curveto(248.47039151,35.25142352)(248.92039106,35.42142335)(249.37038635,35.60142873)
\curveto(249.42039056,35.62142315)(249.47039051,35.63642313)(249.52038635,35.64642873)
\lineto(249.67038635,35.70642873)
\curveto(249.89039009,35.79642297)(250.11538987,35.88142289)(250.34538635,35.96142873)
\curveto(250.56538942,36.04142273)(250.7853892,36.12642264)(251.00538635,36.21642873)
\curveto(251.09538889,36.25642251)(251.20538878,36.29642247)(251.33538635,36.33642873)
\curveto(251.45538853,36.37642239)(251.52538846,36.44142233)(251.54538635,36.53142873)
\curveto(251.55538843,36.5714222)(251.55538843,36.60142217)(251.54538635,36.62142873)
\lineto(251.48538635,36.68142873)
\curveto(251.43538855,36.73142204)(251.3803886,36.766422)(251.32038635,36.78642873)
\curveto(251.26038872,36.81642195)(251.19538879,36.84642192)(251.12538635,36.87642873)
\lineto(250.49538635,37.11642873)
\curveto(250.27538971,37.19642157)(250.06038992,37.27642149)(249.85038635,37.35642873)
\lineto(249.70038635,37.41642873)
\lineto(249.52038635,37.47642873)
\curveto(249.33039065,37.55642121)(249.14039084,37.62642114)(248.95038635,37.68642873)
\curveto(248.75039123,37.75642101)(248.55039143,37.83142094)(248.35038635,37.91142873)
\curveto(247.77039221,38.15142062)(247.1853928,38.3714204)(246.59538635,38.57142873)
\curveto(246.00539398,38.78141999)(245.42039456,39.00641976)(244.84038635,39.24642873)
\curveto(244.64039534,39.32641944)(244.43539555,39.40141937)(244.22538635,39.47142873)
\curveto(244.01539597,39.55141922)(243.81039617,39.63141914)(243.61038635,39.71142873)
\curveto(243.53039645,39.75141902)(243.43039655,39.78641898)(243.31038635,39.81642873)
\curveto(243.19039679,39.85641891)(243.10539688,39.91141886)(243.05538635,39.98142873)
\curveto(243.01539697,40.04141873)(242.985397,40.11641865)(242.96538635,40.20642873)
\curveto(242.94539704,40.30641846)(242.93539705,40.41641835)(242.93538635,40.53642873)
\curveto(242.92539706,40.65641811)(242.92539706,40.77641799)(242.93538635,40.89642873)
\curveto(242.93539705,41.01641775)(242.93539705,41.12641764)(242.93538635,41.22642873)
\curveto(242.93539705,41.31641745)(242.93539705,41.40641736)(242.93538635,41.49642873)
\curveto(242.93539705,41.59641717)(242.95539703,41.6714171)(242.99538635,41.72142873)
\curveto(243.04539694,41.81141696)(243.13539685,41.86141691)(243.26538635,41.87142873)
\curveto(243.39539659,41.88141689)(243.53539645,41.88641688)(243.68538635,41.88642873)
\lineto(245.33538635,41.88642873)
\lineto(251.60538635,41.88642873)
\lineto(252.86538635,41.88642873)
\curveto(252.97538701,41.88641688)(253.0853869,41.88641688)(253.19538635,41.88642873)
\curveto(253.30538668,41.89641687)(253.39038659,41.87641689)(253.45038635,41.82642873)
\curveto(253.51038647,41.79641697)(253.55038643,41.75141702)(253.57038635,41.69142873)
\curveto(253.5803864,41.63141714)(253.59538639,41.56141721)(253.61538635,41.48142873)
\lineto(253.61538635,41.24142873)
\lineto(253.61538635,40.88142873)
\curveto(253.60538638,40.771418)(253.56038642,40.69141808)(253.48038635,40.64142873)
\curveto(253.45038653,40.62141815)(253.42038656,40.60641816)(253.39038635,40.59642873)
\curveto(253.35038663,40.59641817)(253.30538668,40.58641818)(253.25538635,40.56642873)
\lineto(253.09038635,40.56642873)
\curveto(253.03038695,40.55641821)(252.96038702,40.55141822)(252.88038635,40.55142873)
\curveto(252.80038718,40.56141821)(252.72538726,40.5664182)(252.65538635,40.56642873)
\lineto(251.81538635,40.56642873)
\lineto(247.39038635,40.56642873)
\curveto(247.14039284,40.5664182)(246.89039309,40.5664182)(246.64038635,40.56642873)
\curveto(246.3803936,40.5664182)(246.13039385,40.56141821)(245.89038635,40.55142873)
\curveto(245.79039419,40.55141822)(245.6803943,40.54641822)(245.56038635,40.53642873)
\curveto(245.44039454,40.52641824)(245.3803946,40.4714183)(245.38038635,40.37142873)
\lineto(245.39538635,40.37142873)
\curveto(245.41539457,40.30141847)(245.4803945,40.24141853)(245.59038635,40.19142873)
\curveto(245.70039428,40.15141862)(245.79539419,40.11641865)(245.87538635,40.08642873)
\curveto(246.04539394,40.01641875)(246.22039376,39.95141882)(246.40038635,39.89142873)
\curveto(246.57039341,39.83141894)(246.74039324,39.76141901)(246.91038635,39.68142873)
\curveto(246.96039302,39.66141911)(247.00539298,39.64641912)(247.04538635,39.63642873)
\curveto(247.0853929,39.62641914)(247.13039285,39.61141916)(247.18038635,39.59142873)
\curveto(247.36039262,39.51141926)(247.54539244,39.44141933)(247.73538635,39.38142873)
\curveto(247.91539207,39.33141944)(248.09539189,39.2664195)(248.27538635,39.18642873)
\curveto(248.42539156,39.11641965)(248.5803914,39.05641971)(248.74038635,39.00642873)
\curveto(248.89039109,38.95641981)(249.04039094,38.90141987)(249.19038635,38.84142873)
\curveto(249.66039032,38.64142013)(250.13538985,38.46142031)(250.61538635,38.30142873)
\curveto(251.0853889,38.14142063)(251.55038843,37.9664208)(252.01038635,37.77642873)
\curveto(252.19038779,37.69642107)(252.37038761,37.62642114)(252.55038635,37.56642873)
\curveto(252.73038725,37.50642126)(252.91038707,37.44142133)(253.09038635,37.37142873)
\curveto(253.20038678,37.32142145)(253.30538668,37.2714215)(253.40538635,37.22142873)
\curveto(253.49538649,37.18142159)(253.56038642,37.09642167)(253.60038635,36.96642873)
\curveto(253.61038637,36.94642182)(253.61538637,36.92142185)(253.61538635,36.89142873)
\curveto(253.60538638,36.8714219)(253.60538638,36.84642192)(253.61538635,36.81642873)
\curveto(253.62538636,36.78642198)(253.63038635,36.75142202)(253.63038635,36.71142873)
\curveto(253.62038636,36.6714221)(253.61538637,36.63142214)(253.61538635,36.59142873)
\lineto(253.61538635,36.29142873)
\curveto(253.61538637,36.19142258)(253.59038639,36.11142266)(253.54038635,36.05142873)
\curveto(253.49038649,35.9714228)(253.42038656,35.91142286)(253.33038635,35.87142873)
\curveto(253.23038675,35.84142293)(253.13038685,35.80142297)(253.03038635,35.75142873)
\curveto(252.83038715,35.6714231)(252.62538736,35.59142318)(252.41538635,35.51142873)
\curveto(252.19538779,35.44142333)(251.985388,35.3664234)(251.78538635,35.28642873)
\curveto(251.60538838,35.20642356)(251.42538856,35.13642363)(251.24538635,35.07642873)
\curveto(251.05538893,35.02642374)(250.87038911,34.96142381)(250.69038635,34.88142873)
\curveto(250.13038985,34.65142412)(249.56539042,34.43642433)(248.99538635,34.23642873)
\curveto(248.42539156,34.03642473)(247.86039212,33.82142495)(247.30038635,33.59142873)
\lineto(246.67038635,33.35142873)
\curveto(246.45039353,33.28142549)(246.24039374,33.20642556)(246.04038635,33.12642873)
\curveto(245.93039405,33.07642569)(245.82539416,33.03142574)(245.72538635,32.99142873)
\curveto(245.61539437,32.96142581)(245.52039446,32.91142586)(245.44038635,32.84142873)
\curveto(245.42039456,32.83142594)(245.41039457,32.82142595)(245.41038635,32.81142873)
\lineto(245.38038635,32.78142873)
\lineto(245.38038635,32.70642873)
\lineto(245.41038635,32.67642873)
\curveto(245.41039457,32.6664261)(245.41539457,32.65642611)(245.42538635,32.64642873)
\curveto(245.47539451,32.62642614)(245.53039445,32.61642615)(245.59038635,32.61642873)
\curveto(245.65039433,32.61642615)(245.71039427,32.60642616)(245.77038635,32.58642873)
\lineto(245.93538635,32.58642873)
\curveto(245.99539399,32.5664262)(246.06039392,32.56142621)(246.13038635,32.57142873)
\curveto(246.20039378,32.58142619)(246.27039371,32.58642618)(246.34038635,32.58642873)
\lineto(247.15038635,32.58642873)
\lineto(251.71038635,32.58642873)
\lineto(252.89538635,32.58642873)
\curveto(253.00538698,32.58642618)(253.11538687,32.58142619)(253.22538635,32.57142873)
\curveto(253.33538665,32.5714262)(253.42038656,32.54642622)(253.48038635,32.49642873)
\curveto(253.56038642,32.44642632)(253.60538638,32.35642641)(253.61538635,32.22642873)
\lineto(253.61538635,31.83642873)
\lineto(253.61538635,31.64142873)
\curveto(253.61538637,31.59142718)(253.60538638,31.54142723)(253.58538635,31.49142873)
\curveto(253.54538644,31.36142741)(253.46038652,31.28642748)(253.33038635,31.26642873)
\curveto(253.20038678,31.25642751)(253.05038693,31.25142752)(252.88038635,31.25142873)
\lineto(251.14038635,31.25142873)
\lineto(245.14038635,31.25142873)
\lineto(243.73038635,31.25142873)
\curveto(243.62039636,31.25142752)(243.50539648,31.24642752)(243.38538635,31.23642873)
\curveto(243.26539672,31.23642753)(243.17039681,31.26142751)(243.10038635,31.31142873)
\curveto(243.04039694,31.35142742)(242.99039699,31.42642734)(242.95038635,31.53642873)
\curveto(242.94039704,31.55642721)(242.94039704,31.57642719)(242.95038635,31.59642873)
\curveto(242.95039703,31.62642714)(242.94539704,31.65142712)(242.93538635,31.67142873)
}
}
{
\newrgbcolor{curcolor}{0 0 0}
\pscustom[linestyle=none,fillstyle=solid,fillcolor=curcolor]
{
\newpath
\moveto(253.06038635,50.87353811)
\curveto(253.22038676,50.90353028)(253.35538663,50.88853029)(253.46538635,50.82853811)
\curveto(253.56538642,50.76853041)(253.64038634,50.68853049)(253.69038635,50.58853811)
\curveto(253.71038627,50.53853064)(253.72038626,50.4835307)(253.72038635,50.42353811)
\curveto(253.72038626,50.37353081)(253.73038625,50.31853086)(253.75038635,50.25853811)
\curveto(253.80038618,50.03853114)(253.7853862,49.81853136)(253.70538635,49.59853811)
\curveto(253.63538635,49.38853179)(253.54538644,49.24353194)(253.43538635,49.16353811)
\curveto(253.36538662,49.11353207)(253.2853867,49.06853211)(253.19538635,49.02853811)
\curveto(253.09538689,48.98853219)(253.01538697,48.93853224)(252.95538635,48.87853811)
\curveto(252.93538705,48.85853232)(252.91538707,48.83353235)(252.89538635,48.80353811)
\curveto(252.87538711,48.7835324)(252.87038711,48.75353243)(252.88038635,48.71353811)
\curveto(252.91038707,48.60353258)(252.96538702,48.49853268)(253.04538635,48.39853811)
\curveto(253.12538686,48.30853287)(253.19538679,48.21853296)(253.25538635,48.12853811)
\curveto(253.33538665,47.99853318)(253.41038657,47.85853332)(253.48038635,47.70853811)
\curveto(253.54038644,47.55853362)(253.59538639,47.39853378)(253.64538635,47.22853811)
\curveto(253.67538631,47.12853405)(253.69538629,47.01853416)(253.70538635,46.89853811)
\curveto(253.71538627,46.78853439)(253.73038625,46.6785345)(253.75038635,46.56853811)
\curveto(253.76038622,46.51853466)(253.76538622,46.47353471)(253.76538635,46.43353811)
\lineto(253.76538635,46.32853811)
\curveto(253.7853862,46.21853496)(253.7853862,46.11353507)(253.76538635,46.01353811)
\lineto(253.76538635,45.87853811)
\curveto(253.75538623,45.82853535)(253.75038623,45.7785354)(253.75038635,45.72853811)
\curveto(253.75038623,45.6785355)(253.74038624,45.63353555)(253.72038635,45.59353811)
\curveto(253.71038627,45.55353563)(253.70538628,45.51853566)(253.70538635,45.48853811)
\curveto(253.71538627,45.46853571)(253.71538627,45.44353574)(253.70538635,45.41353811)
\lineto(253.64538635,45.17353811)
\curveto(253.63538635,45.09353609)(253.61538637,45.01853616)(253.58538635,44.94853811)
\curveto(253.45538653,44.64853653)(253.31038667,44.40353678)(253.15038635,44.21353811)
\curveto(252.980387,44.03353715)(252.74538724,43.8835373)(252.44538635,43.76353811)
\curveto(252.22538776,43.67353751)(251.96038802,43.62853755)(251.65038635,43.62853811)
\lineto(251.33538635,43.62853811)
\curveto(251.2853887,43.63853754)(251.23538875,43.64353754)(251.18538635,43.64353811)
\lineto(251.00538635,43.67353811)
\lineto(250.67538635,43.79353811)
\curveto(250.56538942,43.83353735)(250.46538952,43.8835373)(250.37538635,43.94353811)
\curveto(250.0853899,44.12353706)(249.87039011,44.36853681)(249.73038635,44.67853811)
\curveto(249.59039039,44.98853619)(249.46539052,45.32853585)(249.35538635,45.69853811)
\curveto(249.31539067,45.83853534)(249.2853907,45.9835352)(249.26538635,46.13353811)
\curveto(249.24539074,46.2835349)(249.22039076,46.43353475)(249.19038635,46.58353811)
\curveto(249.17039081,46.65353453)(249.16039082,46.71853446)(249.16038635,46.77853811)
\curveto(249.16039082,46.84853433)(249.15039083,46.92353426)(249.13038635,47.00353811)
\curveto(249.11039087,47.07353411)(249.10039088,47.14353404)(249.10038635,47.21353811)
\curveto(249.09039089,47.2835339)(249.07539091,47.35853382)(249.05538635,47.43853811)
\curveto(248.99539099,47.68853349)(248.94539104,47.92353326)(248.90538635,48.14353811)
\curveto(248.85539113,48.36353282)(248.74039124,48.53853264)(248.56038635,48.66853811)
\curveto(248.4803915,48.72853245)(248.3803916,48.7785324)(248.26038635,48.81853811)
\curveto(248.13039185,48.85853232)(247.99039199,48.85853232)(247.84038635,48.81853811)
\curveto(247.60039238,48.75853242)(247.41039257,48.66853251)(247.27038635,48.54853811)
\curveto(247.13039285,48.43853274)(247.02039296,48.2785329)(246.94038635,48.06853811)
\curveto(246.89039309,47.94853323)(246.85539313,47.80353338)(246.83538635,47.63353811)
\curveto(246.81539317,47.47353371)(246.80539318,47.30353388)(246.80538635,47.12353811)
\curveto(246.80539318,46.94353424)(246.81539317,46.76853441)(246.83538635,46.59853811)
\curveto(246.85539313,46.42853475)(246.8853931,46.2835349)(246.92538635,46.16353811)
\curveto(246.985393,45.99353519)(247.07039291,45.82853535)(247.18038635,45.66853811)
\curveto(247.24039274,45.58853559)(247.32039266,45.51353567)(247.42038635,45.44353811)
\curveto(247.51039247,45.3835358)(247.61039237,45.32853585)(247.72038635,45.27853811)
\curveto(247.80039218,45.24853593)(247.8853921,45.21853596)(247.97538635,45.18853811)
\curveto(248.06539192,45.16853601)(248.13539185,45.12353606)(248.18538635,45.05353811)
\curveto(248.21539177,45.01353617)(248.24039174,44.94353624)(248.26038635,44.84353811)
\curveto(248.27039171,44.75353643)(248.27539171,44.65853652)(248.27538635,44.55853811)
\curveto(248.27539171,44.45853672)(248.27039171,44.35853682)(248.26038635,44.25853811)
\curveto(248.24039174,44.16853701)(248.21539177,44.10353708)(248.18538635,44.06353811)
\curveto(248.15539183,44.02353716)(248.10539188,43.99353719)(248.03538635,43.97353811)
\curveto(247.96539202,43.95353723)(247.89039209,43.95353723)(247.81038635,43.97353811)
\curveto(247.6803923,44.00353718)(247.56039242,44.03353715)(247.45038635,44.06353811)
\curveto(247.33039265,44.10353708)(247.21539277,44.14853703)(247.10538635,44.19853811)
\curveto(246.75539323,44.38853679)(246.4853935,44.62853655)(246.29538635,44.91853811)
\curveto(246.09539389,45.20853597)(245.93539405,45.56853561)(245.81538635,45.99853811)
\curveto(245.79539419,46.09853508)(245.7803942,46.19853498)(245.77038635,46.29853811)
\curveto(245.76039422,46.40853477)(245.74539424,46.51853466)(245.72538635,46.62853811)
\curveto(245.71539427,46.66853451)(245.71539427,46.73353445)(245.72538635,46.82353811)
\curveto(245.72539426,46.91353427)(245.71539427,46.96853421)(245.69538635,46.98853811)
\curveto(245.6853943,47.68853349)(245.76539422,48.29853288)(245.93538635,48.81853811)
\curveto(246.10539388,49.33853184)(246.43039355,49.70353148)(246.91038635,49.91353811)
\curveto(247.11039287,50.00353118)(247.34539264,50.05353113)(247.61538635,50.06353811)
\curveto(247.87539211,50.0835311)(248.15039183,50.09353109)(248.44038635,50.09353811)
\lineto(251.75538635,50.09353811)
\curveto(251.89538809,50.09353109)(252.03038795,50.09853108)(252.16038635,50.10853811)
\curveto(252.29038769,50.11853106)(252.39538759,50.14853103)(252.47538635,50.19853811)
\curveto(252.54538744,50.24853093)(252.59538739,50.31353087)(252.62538635,50.39353811)
\curveto(252.66538732,50.4835307)(252.69538729,50.56853061)(252.71538635,50.64853811)
\curveto(252.72538726,50.72853045)(252.77038721,50.78853039)(252.85038635,50.82853811)
\curveto(252.8803871,50.84853033)(252.91038707,50.85853032)(252.94038635,50.85853811)
\curveto(252.97038701,50.85853032)(253.01038697,50.86353032)(253.06038635,50.87353811)
\moveto(251.39538635,48.72853811)
\curveto(251.25538873,48.78853239)(251.09538889,48.81853236)(250.91538635,48.81853811)
\curveto(250.72538926,48.82853235)(250.53038945,48.83353235)(250.33038635,48.83353811)
\curveto(250.22038976,48.83353235)(250.12038986,48.82853235)(250.03038635,48.81853811)
\curveto(249.94039004,48.80853237)(249.87039011,48.76853241)(249.82038635,48.69853811)
\curveto(249.80039018,48.66853251)(249.79039019,48.59853258)(249.79038635,48.48853811)
\curveto(249.81039017,48.46853271)(249.82039016,48.43353275)(249.82038635,48.38353811)
\curveto(249.82039016,48.33353285)(249.83039015,48.28853289)(249.85038635,48.24853811)
\curveto(249.87039011,48.16853301)(249.89039009,48.0785331)(249.91038635,47.97853811)
\lineto(249.97038635,47.67853811)
\curveto(249.97039001,47.64853353)(249.97539001,47.61353357)(249.98538635,47.57353811)
\lineto(249.98538635,47.46853811)
\curveto(250.02538996,47.31853386)(250.05038993,47.15353403)(250.06038635,46.97353811)
\curveto(250.06038992,46.80353438)(250.0803899,46.64353454)(250.12038635,46.49353811)
\curveto(250.14038984,46.41353477)(250.16038982,46.33853484)(250.18038635,46.26853811)
\curveto(250.19038979,46.20853497)(250.20538978,46.13853504)(250.22538635,46.05853811)
\curveto(250.27538971,45.89853528)(250.34038964,45.74853543)(250.42038635,45.60853811)
\curveto(250.49038949,45.46853571)(250.5803894,45.34853583)(250.69038635,45.24853811)
\curveto(250.80038918,45.14853603)(250.93538905,45.07353611)(251.09538635,45.02353811)
\curveto(251.24538874,44.97353621)(251.43038855,44.95353623)(251.65038635,44.96353811)
\curveto(251.75038823,44.96353622)(251.84538814,44.9785362)(251.93538635,45.00853811)
\curveto(252.01538797,45.04853613)(252.09038789,45.09353609)(252.16038635,45.14353811)
\curveto(252.27038771,45.22353596)(252.36538762,45.32853585)(252.44538635,45.45853811)
\curveto(252.51538747,45.58853559)(252.57538741,45.72853545)(252.62538635,45.87853811)
\curveto(252.63538735,45.92853525)(252.64038734,45.9785352)(252.64038635,46.02853811)
\curveto(252.64038734,46.0785351)(252.64538734,46.12853505)(252.65538635,46.17853811)
\curveto(252.67538731,46.24853493)(252.69038729,46.33353485)(252.70038635,46.43353811)
\curveto(252.70038728,46.54353464)(252.69038729,46.63353455)(252.67038635,46.70353811)
\curveto(252.65038733,46.76353442)(252.64538734,46.82353436)(252.65538635,46.88353811)
\curveto(252.65538733,46.94353424)(252.64538734,47.00353418)(252.62538635,47.06353811)
\curveto(252.60538738,47.14353404)(252.59038739,47.21853396)(252.58038635,47.28853811)
\curveto(252.57038741,47.36853381)(252.55038743,47.44353374)(252.52038635,47.51353811)
\curveto(252.40038758,47.80353338)(252.25538773,48.04853313)(252.08538635,48.24853811)
\curveto(251.91538807,48.45853272)(251.6853883,48.61853256)(251.39538635,48.72853811)
}
}
{
\newrgbcolor{curcolor}{0 0 0}
\pscustom[linestyle=none,fillstyle=solid,fillcolor=curcolor]
{
\newpath
\moveto(245.71038635,55.69017873)
\curveto(245.71039427,55.92017394)(245.77039421,56.05017381)(245.89038635,56.08017873)
\curveto(246.00039398,56.11017375)(246.16539382,56.12517374)(246.38538635,56.12517873)
\lineto(246.67038635,56.12517873)
\curveto(246.76039322,56.12517374)(246.83539315,56.10017376)(246.89538635,56.05017873)
\curveto(246.97539301,55.99017387)(247.02039296,55.90517396)(247.03038635,55.79517873)
\curveto(247.03039295,55.68517418)(247.04539294,55.57517429)(247.07538635,55.46517873)
\curveto(247.10539288,55.32517454)(247.13539285,55.19017467)(247.16538635,55.06017873)
\curveto(247.19539279,54.94017492)(247.23539275,54.82517504)(247.28538635,54.71517873)
\curveto(247.41539257,54.42517544)(247.59539239,54.19017567)(247.82538635,54.01017873)
\curveto(248.04539194,53.83017603)(248.30039168,53.67517619)(248.59038635,53.54517873)
\curveto(248.70039128,53.50517636)(248.81539117,53.47517639)(248.93538635,53.45517873)
\curveto(249.04539094,53.43517643)(249.16039082,53.41017645)(249.28038635,53.38017873)
\curveto(249.33039065,53.37017649)(249.3803906,53.3651765)(249.43038635,53.36517873)
\curveto(249.4803905,53.37517649)(249.53039045,53.37517649)(249.58038635,53.36517873)
\curveto(249.70039028,53.33517653)(249.84039014,53.32017654)(250.00038635,53.32017873)
\curveto(250.15038983,53.33017653)(250.29538969,53.33517653)(250.43538635,53.33517873)
\lineto(252.28038635,53.33517873)
\lineto(252.62538635,53.33517873)
\curveto(252.74538724,53.33517653)(252.86038712,53.33017653)(252.97038635,53.32017873)
\curveto(253.0803869,53.31017655)(253.17538681,53.30517656)(253.25538635,53.30517873)
\curveto(253.33538665,53.31517655)(253.40538658,53.29517657)(253.46538635,53.24517873)
\curveto(253.53538645,53.19517667)(253.57538641,53.11517675)(253.58538635,53.00517873)
\curveto(253.59538639,52.90517696)(253.60038638,52.79517707)(253.60038635,52.67517873)
\lineto(253.60038635,52.40517873)
\curveto(253.5803864,52.35517751)(253.56538642,52.30517756)(253.55538635,52.25517873)
\curveto(253.53538645,52.21517765)(253.51038647,52.18517768)(253.48038635,52.16517873)
\curveto(253.41038657,52.11517775)(253.32538666,52.08517778)(253.22538635,52.07517873)
\lineto(252.89538635,52.07517873)
\lineto(251.74038635,52.07517873)
\lineto(247.58538635,52.07517873)
\lineto(246.55038635,52.07517873)
\lineto(246.25038635,52.07517873)
\curveto(246.15039383,52.08517778)(246.06539392,52.11517775)(245.99538635,52.16517873)
\curveto(245.95539403,52.19517767)(245.92539406,52.24517762)(245.90538635,52.31517873)
\curveto(245.8853941,52.39517747)(245.87539411,52.48017738)(245.87538635,52.57017873)
\curveto(245.86539412,52.6601772)(245.86539412,52.75017711)(245.87538635,52.84017873)
\curveto(245.8853941,52.93017693)(245.90039408,53.00017686)(245.92038635,53.05017873)
\curveto(245.95039403,53.13017673)(246.01039397,53.18017668)(246.10038635,53.20017873)
\curveto(246.1803938,53.23017663)(246.27039371,53.24517662)(246.37038635,53.24517873)
\lineto(246.67038635,53.24517873)
\curveto(246.77039321,53.24517662)(246.86039312,53.2651766)(246.94038635,53.30517873)
\curveto(246.96039302,53.31517655)(246.97539301,53.32517654)(246.98538635,53.33517873)
\lineto(247.03038635,53.38017873)
\curveto(247.03039295,53.49017637)(246.985393,53.58017628)(246.89538635,53.65017873)
\curveto(246.79539319,53.72017614)(246.71539327,53.78017608)(246.65538635,53.83017873)
\lineto(246.56538635,53.92017873)
\curveto(246.45539353,54.01017585)(246.34039364,54.13517573)(246.22038635,54.29517873)
\curveto(246.10039388,54.45517541)(246.01039397,54.60517526)(245.95038635,54.74517873)
\curveto(245.90039408,54.83517503)(245.86539412,54.93017493)(245.84538635,55.03017873)
\curveto(245.81539417,55.13017473)(245.7853942,55.23517463)(245.75538635,55.34517873)
\curveto(245.74539424,55.40517446)(245.74039424,55.4651744)(245.74038635,55.52517873)
\curveto(245.73039425,55.58517428)(245.72039426,55.64017422)(245.71038635,55.69017873)
}
}
{
\newrgbcolor{curcolor}{0 0 0}
\pscustom[linestyle=none,fillstyle=solid,fillcolor=curcolor]
{
}
}
{
\newrgbcolor{curcolor}{0 0 0}
\pscustom[linestyle=none,fillstyle=solid,fillcolor=curcolor]
{
\newpath
\moveto(243.01038635,64.24510061)
\curveto(243.00039698,64.93509597)(243.12039686,65.53509537)(243.37038635,66.04510061)
\curveto(243.62039636,66.56509434)(243.95539603,66.96009395)(244.37538635,67.23010061)
\curveto(244.45539553,67.28009363)(244.54539544,67.32509358)(244.64538635,67.36510061)
\curveto(244.73539525,67.4050935)(244.83039515,67.45009346)(244.93038635,67.50010061)
\curveto(245.03039495,67.54009337)(245.13039485,67.57009334)(245.23038635,67.59010061)
\curveto(245.33039465,67.6100933)(245.43539455,67.63009328)(245.54538635,67.65010061)
\curveto(245.59539439,67.67009324)(245.64039434,67.67509323)(245.68038635,67.66510061)
\curveto(245.72039426,67.65509325)(245.76539422,67.66009325)(245.81538635,67.68010061)
\curveto(245.86539412,67.69009322)(245.95039403,67.69509321)(246.07038635,67.69510061)
\curveto(246.1803938,67.69509321)(246.26539372,67.69009322)(246.32538635,67.68010061)
\curveto(246.3853936,67.66009325)(246.44539354,67.65009326)(246.50538635,67.65010061)
\curveto(246.56539342,67.66009325)(246.62539336,67.65509325)(246.68538635,67.63510061)
\curveto(246.82539316,67.59509331)(246.96039302,67.56009335)(247.09038635,67.53010061)
\curveto(247.22039276,67.50009341)(247.34539264,67.46009345)(247.46538635,67.41010061)
\curveto(247.60539238,67.35009356)(247.73039225,67.28009363)(247.84038635,67.20010061)
\curveto(247.95039203,67.13009378)(248.06039192,67.05509385)(248.17038635,66.97510061)
\lineto(248.23038635,66.91510061)
\curveto(248.25039173,66.905094)(248.27039171,66.89009402)(248.29038635,66.87010061)
\curveto(248.45039153,66.75009416)(248.59539139,66.61509429)(248.72538635,66.46510061)
\curveto(248.85539113,66.31509459)(248.980391,66.15509475)(249.10038635,65.98510061)
\curveto(249.32039066,65.67509523)(249.52539046,65.38009553)(249.71538635,65.10010061)
\curveto(249.85539013,64.87009604)(249.99038999,64.64009627)(250.12038635,64.41010061)
\curveto(250.25038973,64.19009672)(250.3853896,63.97009694)(250.52538635,63.75010061)
\curveto(250.69538929,63.50009741)(250.87538911,63.26009765)(251.06538635,63.03010061)
\curveto(251.25538873,62.8100981)(251.4803885,62.62009829)(251.74038635,62.46010061)
\curveto(251.80038818,62.42009849)(251.86038812,62.38509852)(251.92038635,62.35510061)
\curveto(251.97038801,62.32509858)(252.03538795,62.29509861)(252.11538635,62.26510061)
\curveto(252.1853878,62.24509866)(252.24538774,62.24009867)(252.29538635,62.25010061)
\curveto(252.36538762,62.27009864)(252.42038756,62.3050986)(252.46038635,62.35510061)
\curveto(252.49038749,62.4050985)(252.51038747,62.46509844)(252.52038635,62.53510061)
\lineto(252.52038635,62.77510061)
\lineto(252.52038635,63.52510061)
\lineto(252.52038635,66.33010061)
\lineto(252.52038635,66.99010061)
\curveto(252.52038746,67.08009383)(252.52538746,67.16509374)(252.53538635,67.24510061)
\curveto(252.53538745,67.32509358)(252.55538743,67.39009352)(252.59538635,67.44010061)
\curveto(252.63538735,67.49009342)(252.71038727,67.53009338)(252.82038635,67.56010061)
\curveto(252.92038706,67.60009331)(253.02038696,67.6100933)(253.12038635,67.59010061)
\lineto(253.25538635,67.59010061)
\curveto(253.32538666,67.57009334)(253.3853866,67.55009336)(253.43538635,67.53010061)
\curveto(253.4853865,67.5100934)(253.52538646,67.47509343)(253.55538635,67.42510061)
\curveto(253.59538639,67.37509353)(253.61538637,67.3050936)(253.61538635,67.21510061)
\lineto(253.61538635,66.94510061)
\lineto(253.61538635,66.04510061)
\lineto(253.61538635,62.53510061)
\lineto(253.61538635,61.47010061)
\curveto(253.61538637,61.39009952)(253.62038636,61.30009961)(253.63038635,61.20010061)
\curveto(253.63038635,61.10009981)(253.62038636,61.01509989)(253.60038635,60.94510061)
\curveto(253.53038645,60.73510017)(253.35038663,60.67010024)(253.06038635,60.75010061)
\curveto(253.02038696,60.76010015)(252.985387,60.76010015)(252.95538635,60.75010061)
\curveto(252.91538707,60.75010016)(252.87038711,60.76010015)(252.82038635,60.78010061)
\curveto(252.74038724,60.80010011)(252.65538733,60.82010009)(252.56538635,60.84010061)
\curveto(252.47538751,60.86010005)(252.39038759,60.88510002)(252.31038635,60.91510061)
\curveto(251.82038816,61.07509983)(251.40538858,61.27509963)(251.06538635,61.51510061)
\curveto(250.81538917,61.69509921)(250.59038939,61.90009901)(250.39038635,62.13010061)
\curveto(250.1803898,62.36009855)(249.98539,62.60009831)(249.80538635,62.85010061)
\curveto(249.62539036,63.1100978)(249.45539053,63.37509753)(249.29538635,63.64510061)
\curveto(249.12539086,63.92509698)(248.95039103,64.19509671)(248.77038635,64.45510061)
\curveto(248.69039129,64.56509634)(248.61539137,64.67009624)(248.54538635,64.77010061)
\curveto(248.47539151,64.88009603)(248.40039158,64.99009592)(248.32038635,65.10010061)
\curveto(248.29039169,65.14009577)(248.26039172,65.17509573)(248.23038635,65.20510061)
\curveto(248.19039179,65.24509566)(248.16039182,65.28509562)(248.14038635,65.32510061)
\curveto(248.03039195,65.46509544)(247.90539208,65.59009532)(247.76538635,65.70010061)
\curveto(247.73539225,65.72009519)(247.71039227,65.74509516)(247.69038635,65.77510061)
\curveto(247.66039232,65.8050951)(247.63039235,65.83009508)(247.60038635,65.85010061)
\curveto(247.50039248,65.93009498)(247.40039258,65.99509491)(247.30038635,66.04510061)
\curveto(247.20039278,66.1050948)(247.09039289,66.16009475)(246.97038635,66.21010061)
\curveto(246.90039308,66.24009467)(246.82539316,66.26009465)(246.74538635,66.27010061)
\lineto(246.50538635,66.33010061)
\lineto(246.41538635,66.33010061)
\curveto(246.3853936,66.34009457)(246.35539363,66.34509456)(246.32538635,66.34510061)
\curveto(246.25539373,66.36509454)(246.16039382,66.37009454)(246.04038635,66.36010061)
\curveto(245.91039407,66.36009455)(245.81039417,66.35009456)(245.74038635,66.33010061)
\curveto(245.66039432,66.3100946)(245.5853944,66.29009462)(245.51538635,66.27010061)
\curveto(245.43539455,66.26009465)(245.35539463,66.24009467)(245.27538635,66.21010061)
\curveto(245.03539495,66.10009481)(244.83539515,65.95009496)(244.67538635,65.76010061)
\curveto(244.50539548,65.58009533)(244.36539562,65.36009555)(244.25538635,65.10010061)
\curveto(244.23539575,65.03009588)(244.22039576,64.96009595)(244.21038635,64.89010061)
\curveto(244.19039579,64.82009609)(244.17039581,64.74509616)(244.15038635,64.66510061)
\curveto(244.13039585,64.58509632)(244.12039586,64.47509643)(244.12038635,64.33510061)
\curveto(244.12039586,64.2050967)(244.13039585,64.10009681)(244.15038635,64.02010061)
\curveto(244.16039582,63.96009695)(244.16539582,63.905097)(244.16538635,63.85510061)
\curveto(244.16539582,63.8050971)(244.17539581,63.75509715)(244.19538635,63.70510061)
\curveto(244.23539575,63.6050973)(244.27539571,63.5100974)(244.31538635,63.42010061)
\curveto(244.35539563,63.34009757)(244.40039558,63.26009765)(244.45038635,63.18010061)
\curveto(244.47039551,63.15009776)(244.49539549,63.12009779)(244.52538635,63.09010061)
\curveto(244.55539543,63.07009784)(244.5803954,63.04509786)(244.60038635,63.01510061)
\lineto(244.67538635,62.94010061)
\curveto(244.69539529,62.910098)(244.71539527,62.88509802)(244.73538635,62.86510061)
\lineto(244.94538635,62.71510061)
\curveto(245.00539498,62.67509823)(245.07039491,62.63009828)(245.14038635,62.58010061)
\curveto(245.23039475,62.52009839)(245.33539465,62.47009844)(245.45538635,62.43010061)
\curveto(245.56539442,62.40009851)(245.67539431,62.36509854)(245.78538635,62.32510061)
\curveto(245.89539409,62.28509862)(246.04039394,62.26009865)(246.22038635,62.25010061)
\curveto(246.39039359,62.24009867)(246.51539347,62.2100987)(246.59538635,62.16010061)
\curveto(246.67539331,62.1100988)(246.72039326,62.03509887)(246.73038635,61.93510061)
\curveto(246.74039324,61.83509907)(246.74539324,61.72509918)(246.74538635,61.60510061)
\curveto(246.74539324,61.56509934)(246.75039323,61.52509938)(246.76038635,61.48510061)
\curveto(246.76039322,61.44509946)(246.75539323,61.4100995)(246.74538635,61.38010061)
\curveto(246.72539326,61.33009958)(246.71539327,61.28009963)(246.71538635,61.23010061)
\curveto(246.71539327,61.19009972)(246.70539328,61.15009976)(246.68538635,61.11010061)
\curveto(246.62539336,61.02009989)(246.49039349,60.97509993)(246.28038635,60.97510061)
\lineto(246.16038635,60.97510061)
\curveto(246.10039388,60.98509992)(246.04039394,60.99009992)(245.98038635,60.99010061)
\curveto(245.91039407,61.00009991)(245.84539414,61.0100999)(245.78538635,61.02010061)
\curveto(245.67539431,61.04009987)(245.57539441,61.06009985)(245.48538635,61.08010061)
\curveto(245.3853946,61.10009981)(245.29039469,61.13009978)(245.20038635,61.17010061)
\curveto(245.13039485,61.19009972)(245.07039491,61.2100997)(245.02038635,61.23010061)
\lineto(244.84038635,61.29010061)
\curveto(244.5803954,61.4100995)(244.33539565,61.56509934)(244.10538635,61.75510061)
\curveto(243.87539611,61.95509895)(243.69039629,62.17009874)(243.55038635,62.40010061)
\curveto(243.47039651,62.5100984)(243.40539658,62.62509828)(243.35538635,62.74510061)
\lineto(243.20538635,63.13510061)
\curveto(243.15539683,63.24509766)(243.12539686,63.36009755)(243.11538635,63.48010061)
\curveto(243.09539689,63.60009731)(243.07039691,63.72509718)(243.04038635,63.85510061)
\curveto(243.04039694,63.92509698)(243.04039694,63.99009692)(243.04038635,64.05010061)
\curveto(243.03039695,64.1100968)(243.02039696,64.17509673)(243.01038635,64.24510061)
}
}
{
\newrgbcolor{curcolor}{0 0 0}
\pscustom[linestyle=none,fillstyle=solid,fillcolor=curcolor]
{
\newpath
\moveto(249.29538635,76.34470998)
\curveto(249.41539057,76.37470226)(249.55539043,76.39970223)(249.71538635,76.41970998)
\curveto(249.87539011,76.43970219)(250.04038994,76.44970218)(250.21038635,76.44970998)
\curveto(250.3803896,76.44970218)(250.54538944,76.43970219)(250.70538635,76.41970998)
\curveto(250.86538912,76.39970223)(251.00538898,76.37470226)(251.12538635,76.34470998)
\curveto(251.26538872,76.30470233)(251.39038859,76.26970236)(251.50038635,76.23970998)
\curveto(251.61038837,76.20970242)(251.72038826,76.16970246)(251.83038635,76.11970998)
\curveto(252.47038751,75.84970278)(252.95538703,75.4347032)(253.28538635,74.87470998)
\curveto(253.34538664,74.79470384)(253.39538659,74.70970392)(253.43538635,74.61970998)
\curveto(253.46538652,74.5297041)(253.50038648,74.4297042)(253.54038635,74.31970998)
\curveto(253.59038639,74.20970442)(253.62538636,74.08970454)(253.64538635,73.95970998)
\curveto(253.67538631,73.83970479)(253.70538628,73.70970492)(253.73538635,73.56970998)
\curveto(253.75538623,73.50970512)(253.76038622,73.44970518)(253.75038635,73.38970998)
\curveto(253.74038624,73.33970529)(253.74538624,73.27970535)(253.76538635,73.20970998)
\curveto(253.77538621,73.18970544)(253.77538621,73.16470547)(253.76538635,73.13470998)
\curveto(253.76538622,73.10470553)(253.77038621,73.07970555)(253.78038635,73.05970998)
\lineto(253.78038635,72.90970998)
\curveto(253.79038619,72.83970579)(253.79038619,72.78970584)(253.78038635,72.75970998)
\curveto(253.77038621,72.71970591)(253.76538622,72.67470596)(253.76538635,72.62470998)
\curveto(253.77538621,72.58470605)(253.77538621,72.54470609)(253.76538635,72.50470998)
\curveto(253.74538624,72.41470622)(253.73038625,72.32470631)(253.72038635,72.23470998)
\curveto(253.72038626,72.14470649)(253.71038627,72.05470658)(253.69038635,71.96470998)
\curveto(253.66038632,71.87470676)(253.63538635,71.78470685)(253.61538635,71.69470998)
\curveto(253.59538639,71.60470703)(253.56538642,71.51970711)(253.52538635,71.43970998)
\curveto(253.41538657,71.19970743)(253.2853867,70.97470766)(253.13538635,70.76470998)
\curveto(252.97538701,70.55470808)(252.79538719,70.37470826)(252.59538635,70.22470998)
\curveto(252.42538756,70.10470853)(252.25038773,69.99970863)(252.07038635,69.90970998)
\curveto(251.89038809,69.81970881)(251.70038828,69.7297089)(251.50038635,69.63970998)
\curveto(251.40038858,69.59970903)(251.30038868,69.56470907)(251.20038635,69.53470998)
\curveto(251.09038889,69.51470912)(250.980389,69.48970914)(250.87038635,69.45970998)
\curveto(250.73038925,69.41970921)(250.59038939,69.39470924)(250.45038635,69.38470998)
\curveto(250.31038967,69.37470926)(250.17038981,69.35470928)(250.03038635,69.32470998)
\curveto(249.92039006,69.31470932)(249.82039016,69.30470933)(249.73038635,69.29470998)
\curveto(249.63039035,69.29470934)(249.53039045,69.28470935)(249.43038635,69.26470998)
\lineto(249.34038635,69.26470998)
\curveto(249.31039067,69.27470936)(249.2853907,69.27470936)(249.26538635,69.26470998)
\lineto(249.05538635,69.26470998)
\curveto(248.99539099,69.24470939)(248.93039105,69.2347094)(248.86038635,69.23470998)
\curveto(248.7803912,69.24470939)(248.70539128,69.24970938)(248.63538635,69.24970998)
\lineto(248.48538635,69.24970998)
\curveto(248.43539155,69.24970938)(248.3853916,69.25470938)(248.33538635,69.26470998)
\lineto(247.96038635,69.26470998)
\curveto(247.93039205,69.27470936)(247.89539209,69.27470936)(247.85538635,69.26470998)
\curveto(247.81539217,69.26470937)(247.77539221,69.26970936)(247.73538635,69.27970998)
\curveto(247.62539236,69.29970933)(247.51539247,69.31470932)(247.40538635,69.32470998)
\curveto(247.2853927,69.3347093)(247.17039281,69.34470929)(247.06038635,69.35470998)
\curveto(246.91039307,69.39470924)(246.76539322,69.41970921)(246.62538635,69.42970998)
\curveto(246.47539351,69.44970918)(246.33039365,69.47970915)(246.19038635,69.51970998)
\curveto(245.89039409,69.60970902)(245.60539438,69.70470893)(245.33538635,69.80470998)
\curveto(245.06539492,69.90470873)(244.81539517,70.0297086)(244.58538635,70.17970998)
\curveto(244.26539572,70.37970825)(243.985396,70.62470801)(243.74538635,70.91470998)
\curveto(243.50539648,71.20470743)(243.32039666,71.54470709)(243.19038635,71.93470998)
\curveto(243.15039683,72.04470659)(243.12539686,72.15470648)(243.11538635,72.26470998)
\curveto(243.09539689,72.38470625)(243.07039691,72.50470613)(243.04038635,72.62470998)
\curveto(243.03039695,72.69470594)(243.02539696,72.75970587)(243.02538635,72.81970998)
\curveto(243.02539696,72.87970575)(243.02039696,72.94470569)(243.01038635,73.01470998)
\curveto(242.99039699,73.71470492)(243.10539688,74.28970434)(243.35538635,74.73970998)
\curveto(243.60539638,75.18970344)(243.95539603,75.5347031)(244.40538635,75.77470998)
\curveto(244.63539535,75.88470275)(244.91039507,75.98470265)(245.23038635,76.07470998)
\curveto(245.30039468,76.09470254)(245.37539461,76.09470254)(245.45538635,76.07470998)
\curveto(245.52539446,76.06470257)(245.57539441,76.03970259)(245.60538635,75.99970998)
\curveto(245.63539435,75.96970266)(245.66039432,75.90970272)(245.68038635,75.81970998)
\curveto(245.69039429,75.7297029)(245.70039428,75.629703)(245.71038635,75.51970998)
\curveto(245.71039427,75.41970321)(245.70539428,75.31970331)(245.69538635,75.21970998)
\curveto(245.6853943,75.1297035)(245.66539432,75.06470357)(245.63538635,75.02470998)
\curveto(245.56539442,74.91470372)(245.45539453,74.8347038)(245.30538635,74.78470998)
\curveto(245.15539483,74.74470389)(245.02539496,74.68970394)(244.91538635,74.61970998)
\curveto(244.60539538,74.4297042)(244.37539561,74.14970448)(244.22538635,73.77970998)
\curveto(244.19539579,73.70970492)(244.17539581,73.634705)(244.16538635,73.55470998)
\curveto(244.15539583,73.48470515)(244.14039584,73.40970522)(244.12038635,73.32970998)
\curveto(244.11039587,73.27970535)(244.10539588,73.20970542)(244.10538635,73.11970998)
\curveto(244.10539588,73.03970559)(244.11039587,72.97470566)(244.12038635,72.92470998)
\curveto(244.14039584,72.88470575)(244.14539584,72.84970578)(244.13538635,72.81970998)
\curveto(244.12539586,72.78970584)(244.12539586,72.75470588)(244.13538635,72.71470998)
\lineto(244.19538635,72.47470998)
\curveto(244.21539577,72.40470623)(244.24039574,72.3347063)(244.27038635,72.26470998)
\curveto(244.43039555,71.88470675)(244.64039534,71.59470704)(244.90038635,71.39470998)
\curveto(245.16039482,71.20470743)(245.47539451,71.0297076)(245.84538635,70.86970998)
\curveto(245.92539406,70.83970779)(246.00539398,70.81470782)(246.08538635,70.79470998)
\curveto(246.16539382,70.78470785)(246.24539374,70.76470787)(246.32538635,70.73470998)
\curveto(246.43539355,70.70470793)(246.55039343,70.67970795)(246.67038635,70.65970998)
\curveto(246.79039319,70.64970798)(246.91039307,70.629708)(247.03038635,70.59970998)
\curveto(247.0803929,70.57970805)(247.13039285,70.56970806)(247.18038635,70.56970998)
\curveto(247.23039275,70.57970805)(247.2803927,70.57470806)(247.33038635,70.55470998)
\curveto(247.39039259,70.54470809)(247.47039251,70.54470809)(247.57038635,70.55470998)
\curveto(247.66039232,70.56470807)(247.71539227,70.57970805)(247.73538635,70.59970998)
\curveto(247.77539221,70.61970801)(247.79539219,70.64970798)(247.79538635,70.68970998)
\curveto(247.79539219,70.73970789)(247.7853922,70.78470785)(247.76538635,70.82470998)
\curveto(247.72539226,70.89470774)(247.6803923,70.95470768)(247.63038635,71.00470998)
\curveto(247.5803924,71.05470758)(247.53039245,71.11470752)(247.48038635,71.18470998)
\lineto(247.42038635,71.24470998)
\curveto(247.39039259,71.27470736)(247.36539262,71.30470733)(247.34538635,71.33470998)
\curveto(247.1853928,71.56470707)(247.05039293,71.83970679)(246.94038635,72.15970998)
\curveto(246.92039306,72.2297064)(246.90539308,72.29970633)(246.89538635,72.36970998)
\curveto(246.8853931,72.43970619)(246.87039311,72.51470612)(246.85038635,72.59470998)
\curveto(246.85039313,72.634706)(246.84539314,72.66970596)(246.83538635,72.69970998)
\curveto(246.82539316,72.7297059)(246.82539316,72.76470587)(246.83538635,72.80470998)
\curveto(246.83539315,72.85470578)(246.82539316,72.89470574)(246.80538635,72.92470998)
\lineto(246.80538635,73.08970998)
\lineto(246.80538635,73.17970998)
\curveto(246.79539319,73.2297054)(246.79539319,73.26970536)(246.80538635,73.29970998)
\curveto(246.81539317,73.34970528)(246.82039316,73.39970523)(246.82038635,73.44970998)
\curveto(246.81039317,73.50970512)(246.81039317,73.56470507)(246.82038635,73.61470998)
\curveto(246.85039313,73.72470491)(246.87039311,73.8297048)(246.88038635,73.92970998)
\curveto(246.89039309,74.03970459)(246.91539307,74.14470449)(246.95538635,74.24470998)
\curveto(247.09539289,74.66470397)(247.2803927,75.00970362)(247.51038635,75.27970998)
\curveto(247.73039225,75.54970308)(248.01539197,75.78970284)(248.36538635,75.99970998)
\curveto(248.50539148,76.07970255)(248.65539133,76.14470249)(248.81538635,76.19470998)
\curveto(248.96539102,76.24470239)(249.12539086,76.29470234)(249.29538635,76.34470998)
\moveto(250.60038635,75.09970998)
\curveto(250.55038943,75.10970352)(250.50538948,75.11470352)(250.46538635,75.11470998)
\lineto(250.31538635,75.11470998)
\curveto(250.00538998,75.11470352)(249.72039026,75.07470356)(249.46038635,74.99470998)
\curveto(249.40039058,74.97470366)(249.34539064,74.95470368)(249.29538635,74.93470998)
\curveto(249.23539075,74.92470371)(249.1803908,74.90970372)(249.13038635,74.88970998)
\curveto(248.64039134,74.66970396)(248.29039169,74.32470431)(248.08038635,73.85470998)
\curveto(248.05039193,73.77470486)(248.02539196,73.69470494)(248.00538635,73.61470998)
\lineto(247.94538635,73.37470998)
\curveto(247.92539206,73.29470534)(247.91539207,73.20470543)(247.91538635,73.10470998)
\lineto(247.91538635,72.78970998)
\curveto(247.93539205,72.76970586)(247.94539204,72.7297059)(247.94538635,72.66970998)
\curveto(247.93539205,72.61970601)(247.93539205,72.57470606)(247.94538635,72.53470998)
\lineto(248.00538635,72.29470998)
\curveto(248.01539197,72.22470641)(248.03539195,72.15470648)(248.06538635,72.08470998)
\curveto(248.32539166,71.48470715)(248.79039119,71.07970755)(249.46038635,70.86970998)
\curveto(249.54039044,70.83970779)(249.62039036,70.81970781)(249.70038635,70.80970998)
\curveto(249.7803902,70.79970783)(249.86539012,70.78470785)(249.95538635,70.76470998)
\lineto(250.10538635,70.76470998)
\curveto(250.14538984,70.75470788)(250.21538977,70.74970788)(250.31538635,70.74970998)
\curveto(250.54538944,70.74970788)(250.74038924,70.76970786)(250.90038635,70.80970998)
\curveto(250.97038901,70.8297078)(251.03538895,70.84470779)(251.09538635,70.85470998)
\curveto(251.15538883,70.86470777)(251.22038876,70.88470775)(251.29038635,70.91470998)
\curveto(251.57038841,71.02470761)(251.81538817,71.16970746)(252.02538635,71.34970998)
\curveto(252.22538776,71.5297071)(252.3853876,71.76470687)(252.50538635,72.05470998)
\lineto(252.59538635,72.29470998)
\lineto(252.65538635,72.53470998)
\curveto(252.67538731,72.58470605)(252.6803873,72.62470601)(252.67038635,72.65470998)
\curveto(252.66038732,72.69470594)(252.66538732,72.73970589)(252.68538635,72.78970998)
\curveto(252.69538729,72.81970581)(252.70038728,72.87470576)(252.70038635,72.95470998)
\curveto(252.70038728,73.0347056)(252.69538729,73.09470554)(252.68538635,73.13470998)
\curveto(252.66538732,73.24470539)(252.65038733,73.34970528)(252.64038635,73.44970998)
\curveto(252.63038735,73.54970508)(252.60038738,73.64470499)(252.55038635,73.73470998)
\curveto(252.35038763,74.26470437)(251.97538801,74.65470398)(251.42538635,74.90470998)
\curveto(251.32538866,74.94470369)(251.22038876,74.97470366)(251.11038635,74.99470998)
\lineto(250.78038635,75.08470998)
\curveto(250.70038928,75.08470355)(250.64038934,75.08970354)(250.60038635,75.09970998)
}
}
{
\newrgbcolor{curcolor}{0 0 0}
\pscustom[linestyle=none,fillstyle=solid,fillcolor=curcolor]
{
\newpath
\moveto(251.98038635,78.63431936)
\lineto(251.98038635,79.26431936)
\lineto(251.98038635,79.45931936)
\curveto(251.980388,79.52931683)(251.99038799,79.58931677)(252.01038635,79.63931936)
\curveto(252.05038793,79.70931665)(252.09038789,79.7593166)(252.13038635,79.78931936)
\curveto(252.1803878,79.82931653)(252.24538774,79.84931651)(252.32538635,79.84931936)
\curveto(252.40538758,79.8593165)(252.49038749,79.86431649)(252.58038635,79.86431936)
\lineto(253.30038635,79.86431936)
\curveto(253.7803862,79.86431649)(254.19038579,79.80431655)(254.53038635,79.68431936)
\curveto(254.87038511,79.56431679)(255.14538484,79.36931699)(255.35538635,79.09931936)
\curveto(255.40538458,79.02931733)(255.45038453,78.9593174)(255.49038635,78.88931936)
\curveto(255.54038444,78.82931753)(255.5853844,78.7543176)(255.62538635,78.66431936)
\curveto(255.63538435,78.64431771)(255.64538434,78.61431774)(255.65538635,78.57431936)
\curveto(255.67538431,78.53431782)(255.6803843,78.48931787)(255.67038635,78.43931936)
\curveto(255.64038434,78.34931801)(255.56538442,78.29431806)(255.44538635,78.27431936)
\curveto(255.33538465,78.2543181)(255.24038474,78.26931809)(255.16038635,78.31931936)
\curveto(255.09038489,78.34931801)(255.02538496,78.39431796)(254.96538635,78.45431936)
\curveto(254.91538507,78.52431783)(254.86538512,78.58931777)(254.81538635,78.64931936)
\curveto(254.76538522,78.71931764)(254.69038529,78.77931758)(254.59038635,78.82931936)
\curveto(254.50038548,78.88931747)(254.41038557,78.93931742)(254.32038635,78.97931936)
\curveto(254.29038569,78.99931736)(254.23038575,79.02431733)(254.14038635,79.05431936)
\curveto(254.06038592,79.08431727)(253.99038599,79.08931727)(253.93038635,79.06931936)
\curveto(253.79038619,79.03931732)(253.70038628,78.97931738)(253.66038635,78.88931936)
\curveto(253.63038635,78.80931755)(253.61538637,78.71931764)(253.61538635,78.61931936)
\curveto(253.61538637,78.51931784)(253.59038639,78.43431792)(253.54038635,78.36431936)
\curveto(253.47038651,78.27431808)(253.33038665,78.22931813)(253.12038635,78.22931936)
\lineto(252.56538635,78.22931936)
\lineto(252.34038635,78.22931936)
\curveto(252.26038772,78.23931812)(252.19538779,78.2593181)(252.14538635,78.28931936)
\curveto(252.06538792,78.34931801)(252.02038796,78.41931794)(252.01038635,78.49931936)
\curveto(252.00038798,78.51931784)(251.99538799,78.53931782)(251.99538635,78.55931936)
\curveto(251.99538799,78.58931777)(251.99038799,78.61431774)(251.98038635,78.63431936)
}
}
{
\newrgbcolor{curcolor}{0 0 0}
\pscustom[linestyle=none,fillstyle=solid,fillcolor=curcolor]
{
}
}
{
\newrgbcolor{curcolor}{0 0 0}
\pscustom[linestyle=none,fillstyle=solid,fillcolor=curcolor]
{
\newpath
\moveto(243.01038635,89.26463186)
\curveto(243.00039698,89.95462722)(243.12039686,90.55462662)(243.37038635,91.06463186)
\curveto(243.62039636,91.58462559)(243.95539603,91.9796252)(244.37538635,92.24963186)
\curveto(244.45539553,92.29962488)(244.54539544,92.34462483)(244.64538635,92.38463186)
\curveto(244.73539525,92.42462475)(244.83039515,92.46962471)(244.93038635,92.51963186)
\curveto(245.03039495,92.55962462)(245.13039485,92.58962459)(245.23038635,92.60963186)
\curveto(245.33039465,92.62962455)(245.43539455,92.64962453)(245.54538635,92.66963186)
\curveto(245.59539439,92.68962449)(245.64039434,92.69462448)(245.68038635,92.68463186)
\curveto(245.72039426,92.6746245)(245.76539422,92.6796245)(245.81538635,92.69963186)
\curveto(245.86539412,92.70962447)(245.95039403,92.71462446)(246.07038635,92.71463186)
\curveto(246.1803938,92.71462446)(246.26539372,92.70962447)(246.32538635,92.69963186)
\curveto(246.3853936,92.6796245)(246.44539354,92.66962451)(246.50538635,92.66963186)
\curveto(246.56539342,92.6796245)(246.62539336,92.6746245)(246.68538635,92.65463186)
\curveto(246.82539316,92.61462456)(246.96039302,92.5796246)(247.09038635,92.54963186)
\curveto(247.22039276,92.51962466)(247.34539264,92.4796247)(247.46538635,92.42963186)
\curveto(247.60539238,92.36962481)(247.73039225,92.29962488)(247.84038635,92.21963186)
\curveto(247.95039203,92.14962503)(248.06039192,92.0746251)(248.17038635,91.99463186)
\lineto(248.23038635,91.93463186)
\curveto(248.25039173,91.92462525)(248.27039171,91.90962527)(248.29038635,91.88963186)
\curveto(248.45039153,91.76962541)(248.59539139,91.63462554)(248.72538635,91.48463186)
\curveto(248.85539113,91.33462584)(248.980391,91.174626)(249.10038635,91.00463186)
\curveto(249.32039066,90.69462648)(249.52539046,90.39962678)(249.71538635,90.11963186)
\curveto(249.85539013,89.88962729)(249.99038999,89.65962752)(250.12038635,89.42963186)
\curveto(250.25038973,89.20962797)(250.3853896,88.98962819)(250.52538635,88.76963186)
\curveto(250.69538929,88.51962866)(250.87538911,88.2796289)(251.06538635,88.04963186)
\curveto(251.25538873,87.82962935)(251.4803885,87.63962954)(251.74038635,87.47963186)
\curveto(251.80038818,87.43962974)(251.86038812,87.40462977)(251.92038635,87.37463186)
\curveto(251.97038801,87.34462983)(252.03538795,87.31462986)(252.11538635,87.28463186)
\curveto(252.1853878,87.26462991)(252.24538774,87.25962992)(252.29538635,87.26963186)
\curveto(252.36538762,87.28962989)(252.42038756,87.32462985)(252.46038635,87.37463186)
\curveto(252.49038749,87.42462975)(252.51038747,87.48462969)(252.52038635,87.55463186)
\lineto(252.52038635,87.79463186)
\lineto(252.52038635,88.54463186)
\lineto(252.52038635,91.34963186)
\lineto(252.52038635,92.00963186)
\curveto(252.52038746,92.09962508)(252.52538746,92.18462499)(252.53538635,92.26463186)
\curveto(252.53538745,92.34462483)(252.55538743,92.40962477)(252.59538635,92.45963186)
\curveto(252.63538735,92.50962467)(252.71038727,92.54962463)(252.82038635,92.57963186)
\curveto(252.92038706,92.61962456)(253.02038696,92.62962455)(253.12038635,92.60963186)
\lineto(253.25538635,92.60963186)
\curveto(253.32538666,92.58962459)(253.3853866,92.56962461)(253.43538635,92.54963186)
\curveto(253.4853865,92.52962465)(253.52538646,92.49462468)(253.55538635,92.44463186)
\curveto(253.59538639,92.39462478)(253.61538637,92.32462485)(253.61538635,92.23463186)
\lineto(253.61538635,91.96463186)
\lineto(253.61538635,91.06463186)
\lineto(253.61538635,87.55463186)
\lineto(253.61538635,86.48963186)
\curveto(253.61538637,86.40963077)(253.62038636,86.31963086)(253.63038635,86.21963186)
\curveto(253.63038635,86.11963106)(253.62038636,86.03463114)(253.60038635,85.96463186)
\curveto(253.53038645,85.75463142)(253.35038663,85.68963149)(253.06038635,85.76963186)
\curveto(253.02038696,85.7796314)(252.985387,85.7796314)(252.95538635,85.76963186)
\curveto(252.91538707,85.76963141)(252.87038711,85.7796314)(252.82038635,85.79963186)
\curveto(252.74038724,85.81963136)(252.65538733,85.83963134)(252.56538635,85.85963186)
\curveto(252.47538751,85.8796313)(252.39038759,85.90463127)(252.31038635,85.93463186)
\curveto(251.82038816,86.09463108)(251.40538858,86.29463088)(251.06538635,86.53463186)
\curveto(250.81538917,86.71463046)(250.59038939,86.91963026)(250.39038635,87.14963186)
\curveto(250.1803898,87.3796298)(249.98539,87.61962956)(249.80538635,87.86963186)
\curveto(249.62539036,88.12962905)(249.45539053,88.39462878)(249.29538635,88.66463186)
\curveto(249.12539086,88.94462823)(248.95039103,89.21462796)(248.77038635,89.47463186)
\curveto(248.69039129,89.58462759)(248.61539137,89.68962749)(248.54538635,89.78963186)
\curveto(248.47539151,89.89962728)(248.40039158,90.00962717)(248.32038635,90.11963186)
\curveto(248.29039169,90.15962702)(248.26039172,90.19462698)(248.23038635,90.22463186)
\curveto(248.19039179,90.26462691)(248.16039182,90.30462687)(248.14038635,90.34463186)
\curveto(248.03039195,90.48462669)(247.90539208,90.60962657)(247.76538635,90.71963186)
\curveto(247.73539225,90.73962644)(247.71039227,90.76462641)(247.69038635,90.79463186)
\curveto(247.66039232,90.82462635)(247.63039235,90.84962633)(247.60038635,90.86963186)
\curveto(247.50039248,90.94962623)(247.40039258,91.01462616)(247.30038635,91.06463186)
\curveto(247.20039278,91.12462605)(247.09039289,91.179626)(246.97038635,91.22963186)
\curveto(246.90039308,91.25962592)(246.82539316,91.2796259)(246.74538635,91.28963186)
\lineto(246.50538635,91.34963186)
\lineto(246.41538635,91.34963186)
\curveto(246.3853936,91.35962582)(246.35539363,91.36462581)(246.32538635,91.36463186)
\curveto(246.25539373,91.38462579)(246.16039382,91.38962579)(246.04038635,91.37963186)
\curveto(245.91039407,91.3796258)(245.81039417,91.36962581)(245.74038635,91.34963186)
\curveto(245.66039432,91.32962585)(245.5853944,91.30962587)(245.51538635,91.28963186)
\curveto(245.43539455,91.2796259)(245.35539463,91.25962592)(245.27538635,91.22963186)
\curveto(245.03539495,91.11962606)(244.83539515,90.96962621)(244.67538635,90.77963186)
\curveto(244.50539548,90.59962658)(244.36539562,90.3796268)(244.25538635,90.11963186)
\curveto(244.23539575,90.04962713)(244.22039576,89.9796272)(244.21038635,89.90963186)
\curveto(244.19039579,89.83962734)(244.17039581,89.76462741)(244.15038635,89.68463186)
\curveto(244.13039585,89.60462757)(244.12039586,89.49462768)(244.12038635,89.35463186)
\curveto(244.12039586,89.22462795)(244.13039585,89.11962806)(244.15038635,89.03963186)
\curveto(244.16039582,88.9796282)(244.16539582,88.92462825)(244.16538635,88.87463186)
\curveto(244.16539582,88.82462835)(244.17539581,88.7746284)(244.19538635,88.72463186)
\curveto(244.23539575,88.62462855)(244.27539571,88.52962865)(244.31538635,88.43963186)
\curveto(244.35539563,88.35962882)(244.40039558,88.2796289)(244.45038635,88.19963186)
\curveto(244.47039551,88.16962901)(244.49539549,88.13962904)(244.52538635,88.10963186)
\curveto(244.55539543,88.08962909)(244.5803954,88.06462911)(244.60038635,88.03463186)
\lineto(244.67538635,87.95963186)
\curveto(244.69539529,87.92962925)(244.71539527,87.90462927)(244.73538635,87.88463186)
\lineto(244.94538635,87.73463186)
\curveto(245.00539498,87.69462948)(245.07039491,87.64962953)(245.14038635,87.59963186)
\curveto(245.23039475,87.53962964)(245.33539465,87.48962969)(245.45538635,87.44963186)
\curveto(245.56539442,87.41962976)(245.67539431,87.38462979)(245.78538635,87.34463186)
\curveto(245.89539409,87.30462987)(246.04039394,87.2796299)(246.22038635,87.26963186)
\curveto(246.39039359,87.25962992)(246.51539347,87.22962995)(246.59538635,87.17963186)
\curveto(246.67539331,87.12963005)(246.72039326,87.05463012)(246.73038635,86.95463186)
\curveto(246.74039324,86.85463032)(246.74539324,86.74463043)(246.74538635,86.62463186)
\curveto(246.74539324,86.58463059)(246.75039323,86.54463063)(246.76038635,86.50463186)
\curveto(246.76039322,86.46463071)(246.75539323,86.42963075)(246.74538635,86.39963186)
\curveto(246.72539326,86.34963083)(246.71539327,86.29963088)(246.71538635,86.24963186)
\curveto(246.71539327,86.20963097)(246.70539328,86.16963101)(246.68538635,86.12963186)
\curveto(246.62539336,86.03963114)(246.49039349,85.99463118)(246.28038635,85.99463186)
\lineto(246.16038635,85.99463186)
\curveto(246.10039388,86.00463117)(246.04039394,86.00963117)(245.98038635,86.00963186)
\curveto(245.91039407,86.01963116)(245.84539414,86.02963115)(245.78538635,86.03963186)
\curveto(245.67539431,86.05963112)(245.57539441,86.0796311)(245.48538635,86.09963186)
\curveto(245.3853946,86.11963106)(245.29039469,86.14963103)(245.20038635,86.18963186)
\curveto(245.13039485,86.20963097)(245.07039491,86.22963095)(245.02038635,86.24963186)
\lineto(244.84038635,86.30963186)
\curveto(244.5803954,86.42963075)(244.33539565,86.58463059)(244.10538635,86.77463186)
\curveto(243.87539611,86.9746302)(243.69039629,87.18962999)(243.55038635,87.41963186)
\curveto(243.47039651,87.52962965)(243.40539658,87.64462953)(243.35538635,87.76463186)
\lineto(243.20538635,88.15463186)
\curveto(243.15539683,88.26462891)(243.12539686,88.3796288)(243.11538635,88.49963186)
\curveto(243.09539689,88.61962856)(243.07039691,88.74462843)(243.04038635,88.87463186)
\curveto(243.04039694,88.94462823)(243.04039694,89.00962817)(243.04038635,89.06963186)
\curveto(243.03039695,89.12962805)(243.02039696,89.19462798)(243.01038635,89.26463186)
}
}
{
\newrgbcolor{curcolor}{0 0 0}
\pscustom[linestyle=none,fillstyle=solid,fillcolor=curcolor]
{
\newpath
\moveto(248.53038635,101.36424123)
\lineto(248.78538635,101.36424123)
\curveto(248.86539112,101.37423353)(248.94039104,101.36923353)(249.01038635,101.34924123)
\lineto(249.25038635,101.34924123)
\lineto(249.41538635,101.34924123)
\curveto(249.51539047,101.32923357)(249.62039036,101.31923358)(249.73038635,101.31924123)
\curveto(249.83039015,101.31923358)(249.93039005,101.30923359)(250.03038635,101.28924123)
\lineto(250.18038635,101.28924123)
\curveto(250.32038966,101.25923364)(250.46038952,101.23923366)(250.60038635,101.22924123)
\curveto(250.73038925,101.21923368)(250.86038912,101.19423371)(250.99038635,101.15424123)
\curveto(251.07038891,101.13423377)(251.15538883,101.11423379)(251.24538635,101.09424123)
\lineto(251.48538635,101.03424123)
\lineto(251.78538635,100.91424123)
\curveto(251.87538811,100.88423402)(251.96538802,100.84923405)(252.05538635,100.80924123)
\curveto(252.27538771,100.70923419)(252.49038749,100.57423433)(252.70038635,100.40424123)
\curveto(252.91038707,100.24423466)(253.0803869,100.06923483)(253.21038635,99.87924123)
\curveto(253.25038673,99.82923507)(253.29038669,99.76923513)(253.33038635,99.69924123)
\curveto(253.36038662,99.63923526)(253.39538659,99.57923532)(253.43538635,99.51924123)
\curveto(253.4853865,99.43923546)(253.52538646,99.34423556)(253.55538635,99.23424123)
\curveto(253.5853864,99.12423578)(253.61538637,99.01923588)(253.64538635,98.91924123)
\curveto(253.6853863,98.80923609)(253.71038627,98.6992362)(253.72038635,98.58924123)
\curveto(253.73038625,98.47923642)(253.74538624,98.36423654)(253.76538635,98.24424123)
\curveto(253.77538621,98.2042367)(253.77538621,98.15923674)(253.76538635,98.10924123)
\curveto(253.76538622,98.06923683)(253.77038621,98.02923687)(253.78038635,97.98924123)
\curveto(253.79038619,97.94923695)(253.79538619,97.89423701)(253.79538635,97.82424123)
\curveto(253.79538619,97.75423715)(253.79038619,97.7042372)(253.78038635,97.67424123)
\curveto(253.76038622,97.62423728)(253.75538623,97.57923732)(253.76538635,97.53924123)
\curveto(253.77538621,97.4992374)(253.77538621,97.46423744)(253.76538635,97.43424123)
\lineto(253.76538635,97.34424123)
\curveto(253.74538624,97.28423762)(253.73038625,97.21923768)(253.72038635,97.14924123)
\curveto(253.72038626,97.08923781)(253.71538627,97.02423788)(253.70538635,96.95424123)
\curveto(253.65538633,96.78423812)(253.60538638,96.62423828)(253.55538635,96.47424123)
\curveto(253.50538648,96.32423858)(253.44038654,96.17923872)(253.36038635,96.03924123)
\curveto(253.32038666,95.98923891)(253.29038669,95.93423897)(253.27038635,95.87424123)
\curveto(253.24038674,95.82423908)(253.20538678,95.77423913)(253.16538635,95.72424123)
\curveto(252.985387,95.48423942)(252.76538722,95.28423962)(252.50538635,95.12424123)
\curveto(252.24538774,94.96423994)(251.96038802,94.82424008)(251.65038635,94.70424123)
\curveto(251.51038847,94.64424026)(251.37038861,94.5992403)(251.23038635,94.56924123)
\curveto(251.0803889,94.53924036)(250.92538906,94.5042404)(250.76538635,94.46424123)
\curveto(250.65538933,94.44424046)(250.54538944,94.42924047)(250.43538635,94.41924123)
\curveto(250.32538966,94.40924049)(250.21538977,94.39424051)(250.10538635,94.37424123)
\curveto(250.06538992,94.36424054)(250.02538996,94.35924054)(249.98538635,94.35924123)
\curveto(249.94539004,94.36924053)(249.90539008,94.36924053)(249.86538635,94.35924123)
\curveto(249.81539017,94.34924055)(249.76539022,94.34424056)(249.71538635,94.34424123)
\lineto(249.55038635,94.34424123)
\curveto(249.50039048,94.32424058)(249.45039053,94.31924058)(249.40038635,94.32924123)
\curveto(249.34039064,94.33924056)(249.2853907,94.33924056)(249.23538635,94.32924123)
\curveto(249.19539079,94.31924058)(249.15039083,94.31924058)(249.10038635,94.32924123)
\curveto(249.05039093,94.33924056)(249.00039098,94.33424057)(248.95038635,94.31424123)
\curveto(248.8803911,94.29424061)(248.80539118,94.28924061)(248.72538635,94.29924123)
\curveto(248.63539135,94.30924059)(248.55039143,94.31424059)(248.47038635,94.31424123)
\curveto(248.3803916,94.31424059)(248.2803917,94.30924059)(248.17038635,94.29924123)
\curveto(248.05039193,94.28924061)(247.95039203,94.29424061)(247.87038635,94.31424123)
\lineto(247.58538635,94.31424123)
\lineto(246.95538635,94.35924123)
\curveto(246.85539313,94.36924053)(246.76039322,94.37924052)(246.67038635,94.38924123)
\lineto(246.37038635,94.41924123)
\curveto(246.32039366,94.43924046)(246.27039371,94.44424046)(246.22038635,94.43424123)
\curveto(246.16039382,94.43424047)(246.10539388,94.44424046)(246.05538635,94.46424123)
\curveto(245.8853941,94.51424039)(245.72039426,94.55424035)(245.56038635,94.58424123)
\curveto(245.39039459,94.61424029)(245.23039475,94.66424024)(245.08038635,94.73424123)
\curveto(244.62039536,94.92423998)(244.24539574,95.14423976)(243.95538635,95.39424123)
\curveto(243.66539632,95.65423925)(243.42039656,96.01423889)(243.22038635,96.47424123)
\curveto(243.17039681,96.6042383)(243.13539685,96.73423817)(243.11538635,96.86424123)
\curveto(243.09539689,97.0042379)(243.07039691,97.14423776)(243.04038635,97.28424123)
\curveto(243.03039695,97.35423755)(243.02539696,97.41923748)(243.02538635,97.47924123)
\curveto(243.02539696,97.53923736)(243.02039696,97.6042373)(243.01038635,97.67424123)
\curveto(242.99039699,98.5042364)(243.14039684,99.17423573)(243.46038635,99.68424123)
\curveto(243.77039621,100.19423471)(244.21039577,100.57423433)(244.78038635,100.82424123)
\curveto(244.90039508,100.87423403)(245.02539496,100.91923398)(245.15538635,100.95924123)
\curveto(245.2853947,100.9992339)(245.42039456,101.04423386)(245.56038635,101.09424123)
\curveto(245.64039434,101.11423379)(245.72539426,101.12923377)(245.81538635,101.13924123)
\lineto(246.05538635,101.19924123)
\curveto(246.16539382,101.22923367)(246.27539371,101.24423366)(246.38538635,101.24424123)
\curveto(246.49539349,101.25423365)(246.60539338,101.26923363)(246.71538635,101.28924123)
\curveto(246.76539322,101.30923359)(246.81039317,101.31423359)(246.85038635,101.30424123)
\curveto(246.89039309,101.3042336)(246.93039305,101.30923359)(246.97038635,101.31924123)
\curveto(247.02039296,101.32923357)(247.07539291,101.32923357)(247.13538635,101.31924123)
\curveto(247.1853928,101.31923358)(247.23539275,101.32423358)(247.28538635,101.33424123)
\lineto(247.42038635,101.33424123)
\curveto(247.4803925,101.35423355)(247.55039243,101.35423355)(247.63038635,101.33424123)
\curveto(247.70039228,101.32423358)(247.76539222,101.32923357)(247.82538635,101.34924123)
\curveto(247.85539213,101.35923354)(247.89539209,101.36423354)(247.94538635,101.36424123)
\lineto(248.06538635,101.36424123)
\lineto(248.53038635,101.36424123)
\moveto(250.85538635,99.81924123)
\curveto(250.53538945,99.91923498)(250.17038981,99.97923492)(249.76038635,99.99924123)
\curveto(249.35039063,100.01923488)(248.94039104,100.02923487)(248.53038635,100.02924123)
\curveto(248.10039188,100.02923487)(247.6803923,100.01923488)(247.27038635,99.99924123)
\curveto(246.86039312,99.97923492)(246.47539351,99.93423497)(246.11538635,99.86424123)
\curveto(245.75539423,99.79423511)(245.43539455,99.68423522)(245.15538635,99.53424123)
\curveto(244.86539512,99.39423551)(244.63039535,99.1992357)(244.45038635,98.94924123)
\curveto(244.34039564,98.78923611)(244.26039572,98.60923629)(244.21038635,98.40924123)
\curveto(244.15039583,98.20923669)(244.12039586,97.96423694)(244.12038635,97.67424123)
\curveto(244.14039584,97.65423725)(244.15039583,97.61923728)(244.15038635,97.56924123)
\curveto(244.14039584,97.51923738)(244.14039584,97.47923742)(244.15038635,97.44924123)
\curveto(244.17039581,97.36923753)(244.19039579,97.29423761)(244.21038635,97.22424123)
\curveto(244.22039576,97.16423774)(244.24039574,97.0992378)(244.27038635,97.02924123)
\curveto(244.39039559,96.75923814)(244.56039542,96.53923836)(244.78038635,96.36924123)
\curveto(244.99039499,96.20923869)(245.23539475,96.07423883)(245.51538635,95.96424123)
\curveto(245.62539436,95.91423899)(245.74539424,95.87423903)(245.87538635,95.84424123)
\curveto(245.99539399,95.82423908)(246.12039386,95.7992391)(246.25038635,95.76924123)
\curveto(246.30039368,95.74923915)(246.35539363,95.73923916)(246.41538635,95.73924123)
\curveto(246.46539352,95.73923916)(246.51539347,95.73423917)(246.56538635,95.72424123)
\curveto(246.65539333,95.71423919)(246.75039323,95.7042392)(246.85038635,95.69424123)
\curveto(246.94039304,95.68423922)(247.03539295,95.67423923)(247.13538635,95.66424123)
\curveto(247.21539277,95.66423924)(247.30039268,95.65923924)(247.39038635,95.64924123)
\lineto(247.63038635,95.64924123)
\lineto(247.81038635,95.64924123)
\curveto(247.84039214,95.63923926)(247.87539211,95.63423927)(247.91538635,95.63424123)
\lineto(248.05038635,95.63424123)
\lineto(248.50038635,95.63424123)
\curveto(248.5803914,95.63423927)(248.66539132,95.62923927)(248.75538635,95.61924123)
\curveto(248.83539115,95.61923928)(248.91039107,95.62923927)(248.98038635,95.64924123)
\lineto(249.25038635,95.64924123)
\curveto(249.27039071,95.64923925)(249.30039068,95.64423926)(249.34038635,95.63424123)
\curveto(249.37039061,95.63423927)(249.39539059,95.63923926)(249.41538635,95.64924123)
\curveto(249.51539047,95.65923924)(249.61539037,95.66423924)(249.71538635,95.66424123)
\curveto(249.80539018,95.67423923)(249.90539008,95.68423922)(250.01538635,95.69424123)
\curveto(250.13538985,95.72423918)(250.26038972,95.73923916)(250.39038635,95.73924123)
\curveto(250.51038947,95.74923915)(250.62538936,95.77423913)(250.73538635,95.81424123)
\curveto(251.03538895,95.89423901)(251.30038868,95.97923892)(251.53038635,96.06924123)
\curveto(251.76038822,96.16923873)(251.97538801,96.31423859)(252.17538635,96.50424123)
\curveto(252.37538761,96.71423819)(252.52538746,96.97923792)(252.62538635,97.29924123)
\curveto(252.64538734,97.33923756)(252.65538733,97.37423753)(252.65538635,97.40424123)
\curveto(252.64538734,97.44423746)(252.65038733,97.48923741)(252.67038635,97.53924123)
\curveto(252.6803873,97.57923732)(252.69038729,97.64923725)(252.70038635,97.74924123)
\curveto(252.71038727,97.85923704)(252.70538728,97.94423696)(252.68538635,98.00424123)
\curveto(252.66538732,98.07423683)(252.65538733,98.14423676)(252.65538635,98.21424123)
\curveto(252.64538734,98.28423662)(252.63038735,98.34923655)(252.61038635,98.40924123)
\curveto(252.55038743,98.60923629)(252.46538752,98.78923611)(252.35538635,98.94924123)
\curveto(252.33538765,98.97923592)(252.31538767,99.0042359)(252.29538635,99.02424123)
\lineto(252.23538635,99.08424123)
\curveto(252.21538777,99.12423578)(252.17538781,99.17423573)(252.11538635,99.23424123)
\curveto(251.97538801,99.33423557)(251.84538814,99.41923548)(251.72538635,99.48924123)
\curveto(251.60538838,99.55923534)(251.46038852,99.62923527)(251.29038635,99.69924123)
\curveto(251.22038876,99.72923517)(251.15038883,99.74923515)(251.08038635,99.75924123)
\curveto(251.01038897,99.77923512)(250.93538905,99.7992351)(250.85538635,99.81924123)
}
}
{
\newrgbcolor{curcolor}{0 0 0}
\pscustom[linestyle=none,fillstyle=solid,fillcolor=curcolor]
{
\newpath
\moveto(243.01038635,106.77385061)
\curveto(243.01039697,106.87384575)(243.02039696,106.96884566)(243.04038635,107.05885061)
\curveto(243.05039693,107.14884548)(243.0803969,107.21384541)(243.13038635,107.25385061)
\curveto(243.21039677,107.31384531)(243.31539667,107.34384528)(243.44538635,107.34385061)
\lineto(243.83538635,107.34385061)
\lineto(245.33538635,107.34385061)
\lineto(251.72538635,107.34385061)
\lineto(252.89538635,107.34385061)
\lineto(253.21038635,107.34385061)
\curveto(253.31038667,107.35384527)(253.39038659,107.33884529)(253.45038635,107.29885061)
\curveto(253.53038645,107.24884538)(253.5803864,107.17384545)(253.60038635,107.07385061)
\curveto(253.61038637,106.98384564)(253.61538637,106.87384575)(253.61538635,106.74385061)
\lineto(253.61538635,106.51885061)
\curveto(253.59538639,106.43884619)(253.5803864,106.36884626)(253.57038635,106.30885061)
\curveto(253.55038643,106.24884638)(253.51038647,106.19884643)(253.45038635,106.15885061)
\curveto(253.39038659,106.11884651)(253.31538667,106.09884653)(253.22538635,106.09885061)
\lineto(252.92538635,106.09885061)
\lineto(251.83038635,106.09885061)
\lineto(246.49038635,106.09885061)
\curveto(246.40039358,106.07884655)(246.32539366,106.06384656)(246.26538635,106.05385061)
\curveto(246.19539379,106.05384657)(246.13539385,106.0238466)(246.08538635,105.96385061)
\curveto(246.03539395,105.89384673)(246.01039397,105.80384682)(246.01038635,105.69385061)
\curveto(246.00039398,105.59384703)(245.99539399,105.48384714)(245.99538635,105.36385061)
\lineto(245.99538635,104.22385061)
\lineto(245.99538635,103.72885061)
\curveto(245.985394,103.56884906)(245.92539406,103.45884917)(245.81538635,103.39885061)
\curveto(245.7853942,103.37884925)(245.75539423,103.36884926)(245.72538635,103.36885061)
\curveto(245.6853943,103.36884926)(245.64039434,103.36384926)(245.59038635,103.35385061)
\curveto(245.47039451,103.33384929)(245.36039462,103.33884929)(245.26038635,103.36885061)
\curveto(245.16039482,103.40884922)(245.09039489,103.46384916)(245.05038635,103.53385061)
\curveto(245.00039498,103.61384901)(244.97539501,103.73384889)(244.97538635,103.89385061)
\curveto(244.97539501,104.05384857)(244.96039502,104.18884844)(244.93038635,104.29885061)
\curveto(244.92039506,104.34884828)(244.91539507,104.40384822)(244.91538635,104.46385061)
\curveto(244.90539508,104.5238481)(244.89039509,104.58384804)(244.87038635,104.64385061)
\curveto(244.82039516,104.79384783)(244.77039521,104.93884769)(244.72038635,105.07885061)
\curveto(244.66039532,105.21884741)(244.59039539,105.35384727)(244.51038635,105.48385061)
\curveto(244.42039556,105.623847)(244.31539567,105.74384688)(244.19538635,105.84385061)
\curveto(244.07539591,105.94384668)(243.94539604,106.03884659)(243.80538635,106.12885061)
\curveto(243.70539628,106.18884644)(243.59539639,106.23384639)(243.47538635,106.26385061)
\curveto(243.35539663,106.30384632)(243.25039673,106.35384627)(243.16038635,106.41385061)
\curveto(243.10039688,106.46384616)(243.06039692,106.53384609)(243.04038635,106.62385061)
\curveto(243.03039695,106.64384598)(243.02539696,106.66884596)(243.02538635,106.69885061)
\curveto(243.02539696,106.7288459)(243.02039696,106.75384587)(243.01038635,106.77385061)
}
}
{
\newrgbcolor{curcolor}{0 0 0}
\pscustom[linestyle=none,fillstyle=solid,fillcolor=curcolor]
{
\newpath
\moveto(243.01038635,115.12345998)
\curveto(243.01039697,115.22345513)(243.02039696,115.31845503)(243.04038635,115.40845998)
\curveto(243.05039693,115.49845485)(243.0803969,115.56345479)(243.13038635,115.60345998)
\curveto(243.21039677,115.66345469)(243.31539667,115.69345466)(243.44538635,115.69345998)
\lineto(243.83538635,115.69345998)
\lineto(245.33538635,115.69345998)
\lineto(251.72538635,115.69345998)
\lineto(252.89538635,115.69345998)
\lineto(253.21038635,115.69345998)
\curveto(253.31038667,115.70345465)(253.39038659,115.68845466)(253.45038635,115.64845998)
\curveto(253.53038645,115.59845475)(253.5803864,115.52345483)(253.60038635,115.42345998)
\curveto(253.61038637,115.33345502)(253.61538637,115.22345513)(253.61538635,115.09345998)
\lineto(253.61538635,114.86845998)
\curveto(253.59538639,114.78845556)(253.5803864,114.71845563)(253.57038635,114.65845998)
\curveto(253.55038643,114.59845575)(253.51038647,114.5484558)(253.45038635,114.50845998)
\curveto(253.39038659,114.46845588)(253.31538667,114.4484559)(253.22538635,114.44845998)
\lineto(252.92538635,114.44845998)
\lineto(251.83038635,114.44845998)
\lineto(246.49038635,114.44845998)
\curveto(246.40039358,114.42845592)(246.32539366,114.41345594)(246.26538635,114.40345998)
\curveto(246.19539379,114.40345595)(246.13539385,114.37345598)(246.08538635,114.31345998)
\curveto(246.03539395,114.24345611)(246.01039397,114.1534562)(246.01038635,114.04345998)
\curveto(246.00039398,113.94345641)(245.99539399,113.83345652)(245.99538635,113.71345998)
\lineto(245.99538635,112.57345998)
\lineto(245.99538635,112.07845998)
\curveto(245.985394,111.91845843)(245.92539406,111.80845854)(245.81538635,111.74845998)
\curveto(245.7853942,111.72845862)(245.75539423,111.71845863)(245.72538635,111.71845998)
\curveto(245.6853943,111.71845863)(245.64039434,111.71345864)(245.59038635,111.70345998)
\curveto(245.47039451,111.68345867)(245.36039462,111.68845866)(245.26038635,111.71845998)
\curveto(245.16039482,111.75845859)(245.09039489,111.81345854)(245.05038635,111.88345998)
\curveto(245.00039498,111.96345839)(244.97539501,112.08345827)(244.97538635,112.24345998)
\curveto(244.97539501,112.40345795)(244.96039502,112.53845781)(244.93038635,112.64845998)
\curveto(244.92039506,112.69845765)(244.91539507,112.7534576)(244.91538635,112.81345998)
\curveto(244.90539508,112.87345748)(244.89039509,112.93345742)(244.87038635,112.99345998)
\curveto(244.82039516,113.14345721)(244.77039521,113.28845706)(244.72038635,113.42845998)
\curveto(244.66039532,113.56845678)(244.59039539,113.70345665)(244.51038635,113.83345998)
\curveto(244.42039556,113.97345638)(244.31539567,114.09345626)(244.19538635,114.19345998)
\curveto(244.07539591,114.29345606)(243.94539604,114.38845596)(243.80538635,114.47845998)
\curveto(243.70539628,114.53845581)(243.59539639,114.58345577)(243.47538635,114.61345998)
\curveto(243.35539663,114.6534557)(243.25039673,114.70345565)(243.16038635,114.76345998)
\curveto(243.10039688,114.81345554)(243.06039692,114.88345547)(243.04038635,114.97345998)
\curveto(243.03039695,114.99345536)(243.02539696,115.01845533)(243.02538635,115.04845998)
\curveto(243.02539696,115.07845527)(243.02039696,115.10345525)(243.01038635,115.12345998)
}
}
{
\newrgbcolor{curcolor}{0 0 0}
\pscustom[linestyle=none,fillstyle=solid,fillcolor=curcolor]
{
\newpath
\moveto(180.20139282,31.67142873)
\lineto(180.20139282,32.58642873)
\curveto(180.20140352,32.68642608)(180.20140352,32.78142599)(180.20139282,32.87142873)
\curveto(180.20140352,32.96142581)(180.2214035,33.03642573)(180.26139282,33.09642873)
\curveto(180.3214034,33.18642558)(180.40140332,33.24642552)(180.50139282,33.27642873)
\curveto(180.60140312,33.31642545)(180.70640301,33.36142541)(180.81639282,33.41142873)
\curveto(181.00640271,33.49142528)(181.19640252,33.56142521)(181.38639282,33.62142873)
\curveto(181.57640214,33.69142508)(181.76640195,33.766425)(181.95639282,33.84642873)
\curveto(182.13640158,33.91642485)(182.3214014,33.98142479)(182.51139282,34.04142873)
\curveto(182.69140103,34.10142467)(182.87140085,34.1714246)(183.05139282,34.25142873)
\curveto(183.19140053,34.31142446)(183.33640038,34.3664244)(183.48639282,34.41642873)
\curveto(183.63640008,34.4664243)(183.78139994,34.52142425)(183.92139282,34.58142873)
\curveto(184.37139935,34.76142401)(184.82639889,34.93142384)(185.28639282,35.09142873)
\curveto(185.73639798,35.25142352)(186.18639753,35.42142335)(186.63639282,35.60142873)
\curveto(186.68639703,35.62142315)(186.73639698,35.63642313)(186.78639282,35.64642873)
\lineto(186.93639282,35.70642873)
\curveto(187.15639656,35.79642297)(187.38139634,35.88142289)(187.61139282,35.96142873)
\curveto(187.83139589,36.04142273)(188.05139567,36.12642264)(188.27139282,36.21642873)
\curveto(188.36139536,36.25642251)(188.47139525,36.29642247)(188.60139282,36.33642873)
\curveto(188.721395,36.37642239)(188.79139493,36.44142233)(188.81139282,36.53142873)
\curveto(188.8213949,36.5714222)(188.8213949,36.60142217)(188.81139282,36.62142873)
\lineto(188.75139282,36.68142873)
\curveto(188.70139502,36.73142204)(188.64639507,36.766422)(188.58639282,36.78642873)
\curveto(188.52639519,36.81642195)(188.46139526,36.84642192)(188.39139282,36.87642873)
\lineto(187.76139282,37.11642873)
\curveto(187.54139618,37.19642157)(187.32639639,37.27642149)(187.11639282,37.35642873)
\lineto(186.96639282,37.41642873)
\lineto(186.78639282,37.47642873)
\curveto(186.59639712,37.55642121)(186.40639731,37.62642114)(186.21639282,37.68642873)
\curveto(186.0163977,37.75642101)(185.8163979,37.83142094)(185.61639282,37.91142873)
\curveto(185.03639868,38.15142062)(184.45139927,38.3714204)(183.86139282,38.57142873)
\curveto(183.27140045,38.78141999)(182.68640103,39.00641976)(182.10639282,39.24642873)
\curveto(181.90640181,39.32641944)(181.70140202,39.40141937)(181.49139282,39.47142873)
\curveto(181.28140244,39.55141922)(181.07640264,39.63141914)(180.87639282,39.71142873)
\curveto(180.79640292,39.75141902)(180.69640302,39.78641898)(180.57639282,39.81642873)
\curveto(180.45640326,39.85641891)(180.37140335,39.91141886)(180.32139282,39.98142873)
\curveto(180.28140344,40.04141873)(180.25140347,40.11641865)(180.23139282,40.20642873)
\curveto(180.21140351,40.30641846)(180.20140352,40.41641835)(180.20139282,40.53642873)
\curveto(180.19140353,40.65641811)(180.19140353,40.77641799)(180.20139282,40.89642873)
\curveto(180.20140352,41.01641775)(180.20140352,41.12641764)(180.20139282,41.22642873)
\curveto(180.20140352,41.31641745)(180.20140352,41.40641736)(180.20139282,41.49642873)
\curveto(180.20140352,41.59641717)(180.2214035,41.6714171)(180.26139282,41.72142873)
\curveto(180.31140341,41.81141696)(180.40140332,41.86141691)(180.53139282,41.87142873)
\curveto(180.66140306,41.88141689)(180.80140292,41.88641688)(180.95139282,41.88642873)
\lineto(182.60139282,41.88642873)
\lineto(188.87139282,41.88642873)
\lineto(190.13139282,41.88642873)
\curveto(190.24139348,41.88641688)(190.35139337,41.88641688)(190.46139282,41.88642873)
\curveto(190.57139315,41.89641687)(190.65639306,41.87641689)(190.71639282,41.82642873)
\curveto(190.77639294,41.79641697)(190.8163929,41.75141702)(190.83639282,41.69142873)
\curveto(190.84639287,41.63141714)(190.86139286,41.56141721)(190.88139282,41.48142873)
\lineto(190.88139282,41.24142873)
\lineto(190.88139282,40.88142873)
\curveto(190.87139285,40.771418)(190.82639289,40.69141808)(190.74639282,40.64142873)
\curveto(190.716393,40.62141815)(190.68639303,40.60641816)(190.65639282,40.59642873)
\curveto(190.6163931,40.59641817)(190.57139315,40.58641818)(190.52139282,40.56642873)
\lineto(190.35639282,40.56642873)
\curveto(190.29639342,40.55641821)(190.22639349,40.55141822)(190.14639282,40.55142873)
\curveto(190.06639365,40.56141821)(189.99139373,40.5664182)(189.92139282,40.56642873)
\lineto(189.08139282,40.56642873)
\lineto(184.65639282,40.56642873)
\curveto(184.40639931,40.5664182)(184.15639956,40.5664182)(183.90639282,40.56642873)
\curveto(183.64640007,40.5664182)(183.39640032,40.56141821)(183.15639282,40.55142873)
\curveto(183.05640066,40.55141822)(182.94640077,40.54641822)(182.82639282,40.53642873)
\curveto(182.70640101,40.52641824)(182.64640107,40.4714183)(182.64639282,40.37142873)
\lineto(182.66139282,40.37142873)
\curveto(182.68140104,40.30141847)(182.74640097,40.24141853)(182.85639282,40.19142873)
\curveto(182.96640075,40.15141862)(183.06140066,40.11641865)(183.14139282,40.08642873)
\curveto(183.31140041,40.01641875)(183.48640023,39.95141882)(183.66639282,39.89142873)
\curveto(183.83639988,39.83141894)(184.00639971,39.76141901)(184.17639282,39.68142873)
\curveto(184.22639949,39.66141911)(184.27139945,39.64641912)(184.31139282,39.63642873)
\curveto(184.35139937,39.62641914)(184.39639932,39.61141916)(184.44639282,39.59142873)
\curveto(184.62639909,39.51141926)(184.81139891,39.44141933)(185.00139282,39.38142873)
\curveto(185.18139854,39.33141944)(185.36139836,39.2664195)(185.54139282,39.18642873)
\curveto(185.69139803,39.11641965)(185.84639787,39.05641971)(186.00639282,39.00642873)
\curveto(186.15639756,38.95641981)(186.30639741,38.90141987)(186.45639282,38.84142873)
\curveto(186.92639679,38.64142013)(187.40139632,38.46142031)(187.88139282,38.30142873)
\curveto(188.35139537,38.14142063)(188.8163949,37.9664208)(189.27639282,37.77642873)
\curveto(189.45639426,37.69642107)(189.63639408,37.62642114)(189.81639282,37.56642873)
\curveto(189.99639372,37.50642126)(190.17639354,37.44142133)(190.35639282,37.37142873)
\curveto(190.46639325,37.32142145)(190.57139315,37.2714215)(190.67139282,37.22142873)
\curveto(190.76139296,37.18142159)(190.82639289,37.09642167)(190.86639282,36.96642873)
\curveto(190.87639284,36.94642182)(190.88139284,36.92142185)(190.88139282,36.89142873)
\curveto(190.87139285,36.8714219)(190.87139285,36.84642192)(190.88139282,36.81642873)
\curveto(190.89139283,36.78642198)(190.89639282,36.75142202)(190.89639282,36.71142873)
\curveto(190.88639283,36.6714221)(190.88139284,36.63142214)(190.88139282,36.59142873)
\lineto(190.88139282,36.29142873)
\curveto(190.88139284,36.19142258)(190.85639286,36.11142266)(190.80639282,36.05142873)
\curveto(190.75639296,35.9714228)(190.68639303,35.91142286)(190.59639282,35.87142873)
\curveto(190.49639322,35.84142293)(190.39639332,35.80142297)(190.29639282,35.75142873)
\curveto(190.09639362,35.6714231)(189.89139383,35.59142318)(189.68139282,35.51142873)
\curveto(189.46139426,35.44142333)(189.25139447,35.3664234)(189.05139282,35.28642873)
\curveto(188.87139485,35.20642356)(188.69139503,35.13642363)(188.51139282,35.07642873)
\curveto(188.3213954,35.02642374)(188.13639558,34.96142381)(187.95639282,34.88142873)
\curveto(187.39639632,34.65142412)(186.83139689,34.43642433)(186.26139282,34.23642873)
\curveto(185.69139803,34.03642473)(185.12639859,33.82142495)(184.56639282,33.59142873)
\lineto(183.93639282,33.35142873)
\curveto(183.7164,33.28142549)(183.50640021,33.20642556)(183.30639282,33.12642873)
\curveto(183.19640052,33.07642569)(183.09140063,33.03142574)(182.99139282,32.99142873)
\curveto(182.88140084,32.96142581)(182.78640093,32.91142586)(182.70639282,32.84142873)
\curveto(182.68640103,32.83142594)(182.67640104,32.82142595)(182.67639282,32.81142873)
\lineto(182.64639282,32.78142873)
\lineto(182.64639282,32.70642873)
\lineto(182.67639282,32.67642873)
\curveto(182.67640104,32.6664261)(182.68140104,32.65642611)(182.69139282,32.64642873)
\curveto(182.74140098,32.62642614)(182.79640092,32.61642615)(182.85639282,32.61642873)
\curveto(182.9164008,32.61642615)(182.97640074,32.60642616)(183.03639282,32.58642873)
\lineto(183.20139282,32.58642873)
\curveto(183.26140046,32.5664262)(183.32640039,32.56142621)(183.39639282,32.57142873)
\curveto(183.46640025,32.58142619)(183.53640018,32.58642618)(183.60639282,32.58642873)
\lineto(184.41639282,32.58642873)
\lineto(188.97639282,32.58642873)
\lineto(190.16139282,32.58642873)
\curveto(190.27139345,32.58642618)(190.38139334,32.58142619)(190.49139282,32.57142873)
\curveto(190.60139312,32.5714262)(190.68639303,32.54642622)(190.74639282,32.49642873)
\curveto(190.82639289,32.44642632)(190.87139285,32.35642641)(190.88139282,32.22642873)
\lineto(190.88139282,31.83642873)
\lineto(190.88139282,31.64142873)
\curveto(190.88139284,31.59142718)(190.87139285,31.54142723)(190.85139282,31.49142873)
\curveto(190.81139291,31.36142741)(190.72639299,31.28642748)(190.59639282,31.26642873)
\curveto(190.46639325,31.25642751)(190.3163934,31.25142752)(190.14639282,31.25142873)
\lineto(188.40639282,31.25142873)
\lineto(182.40639282,31.25142873)
\lineto(180.99639282,31.25142873)
\curveto(180.88640283,31.25142752)(180.77140295,31.24642752)(180.65139282,31.23642873)
\curveto(180.53140319,31.23642753)(180.43640328,31.26142751)(180.36639282,31.31142873)
\curveto(180.30640341,31.35142742)(180.25640346,31.42642734)(180.21639282,31.53642873)
\curveto(180.20640351,31.55642721)(180.20640351,31.57642719)(180.21639282,31.59642873)
\curveto(180.2164035,31.62642714)(180.21140351,31.65142712)(180.20139282,31.67142873)
}
}
{
\newrgbcolor{curcolor}{0 0 0}
\pscustom[linestyle=none,fillstyle=solid,fillcolor=curcolor]
{
\newpath
\moveto(190.32639282,50.87353811)
\curveto(190.48639323,50.90353028)(190.6213931,50.88853029)(190.73139282,50.82853811)
\curveto(190.83139289,50.76853041)(190.90639281,50.68853049)(190.95639282,50.58853811)
\curveto(190.97639274,50.53853064)(190.98639273,50.4835307)(190.98639282,50.42353811)
\curveto(190.98639273,50.37353081)(190.99639272,50.31853086)(191.01639282,50.25853811)
\curveto(191.06639265,50.03853114)(191.05139267,49.81853136)(190.97139282,49.59853811)
\curveto(190.90139282,49.38853179)(190.81139291,49.24353194)(190.70139282,49.16353811)
\curveto(190.63139309,49.11353207)(190.55139317,49.06853211)(190.46139282,49.02853811)
\curveto(190.36139336,48.98853219)(190.28139344,48.93853224)(190.22139282,48.87853811)
\curveto(190.20139352,48.85853232)(190.18139354,48.83353235)(190.16139282,48.80353811)
\curveto(190.14139358,48.7835324)(190.13639358,48.75353243)(190.14639282,48.71353811)
\curveto(190.17639354,48.60353258)(190.23139349,48.49853268)(190.31139282,48.39853811)
\curveto(190.39139333,48.30853287)(190.46139326,48.21853296)(190.52139282,48.12853811)
\curveto(190.60139312,47.99853318)(190.67639304,47.85853332)(190.74639282,47.70853811)
\curveto(190.80639291,47.55853362)(190.86139286,47.39853378)(190.91139282,47.22853811)
\curveto(190.94139278,47.12853405)(190.96139276,47.01853416)(190.97139282,46.89853811)
\curveto(190.98139274,46.78853439)(190.99639272,46.6785345)(191.01639282,46.56853811)
\curveto(191.02639269,46.51853466)(191.03139269,46.47353471)(191.03139282,46.43353811)
\lineto(191.03139282,46.32853811)
\curveto(191.05139267,46.21853496)(191.05139267,46.11353507)(191.03139282,46.01353811)
\lineto(191.03139282,45.87853811)
\curveto(191.0213927,45.82853535)(191.0163927,45.7785354)(191.01639282,45.72853811)
\curveto(191.0163927,45.6785355)(191.00639271,45.63353555)(190.98639282,45.59353811)
\curveto(190.97639274,45.55353563)(190.97139275,45.51853566)(190.97139282,45.48853811)
\curveto(190.98139274,45.46853571)(190.98139274,45.44353574)(190.97139282,45.41353811)
\lineto(190.91139282,45.17353811)
\curveto(190.90139282,45.09353609)(190.88139284,45.01853616)(190.85139282,44.94853811)
\curveto(190.721393,44.64853653)(190.57639314,44.40353678)(190.41639282,44.21353811)
\curveto(190.24639347,44.03353715)(190.01139371,43.8835373)(189.71139282,43.76353811)
\curveto(189.49139423,43.67353751)(189.22639449,43.62853755)(188.91639282,43.62853811)
\lineto(188.60139282,43.62853811)
\curveto(188.55139517,43.63853754)(188.50139522,43.64353754)(188.45139282,43.64353811)
\lineto(188.27139282,43.67353811)
\lineto(187.94139282,43.79353811)
\curveto(187.83139589,43.83353735)(187.73139599,43.8835373)(187.64139282,43.94353811)
\curveto(187.35139637,44.12353706)(187.13639658,44.36853681)(186.99639282,44.67853811)
\curveto(186.85639686,44.98853619)(186.73139699,45.32853585)(186.62139282,45.69853811)
\curveto(186.58139714,45.83853534)(186.55139717,45.9835352)(186.53139282,46.13353811)
\curveto(186.51139721,46.2835349)(186.48639723,46.43353475)(186.45639282,46.58353811)
\curveto(186.43639728,46.65353453)(186.42639729,46.71853446)(186.42639282,46.77853811)
\curveto(186.42639729,46.84853433)(186.4163973,46.92353426)(186.39639282,47.00353811)
\curveto(186.37639734,47.07353411)(186.36639735,47.14353404)(186.36639282,47.21353811)
\curveto(186.35639736,47.2835339)(186.34139738,47.35853382)(186.32139282,47.43853811)
\curveto(186.26139746,47.68853349)(186.21139751,47.92353326)(186.17139282,48.14353811)
\curveto(186.1213976,48.36353282)(186.00639771,48.53853264)(185.82639282,48.66853811)
\curveto(185.74639797,48.72853245)(185.64639807,48.7785324)(185.52639282,48.81853811)
\curveto(185.39639832,48.85853232)(185.25639846,48.85853232)(185.10639282,48.81853811)
\curveto(184.86639885,48.75853242)(184.67639904,48.66853251)(184.53639282,48.54853811)
\curveto(184.39639932,48.43853274)(184.28639943,48.2785329)(184.20639282,48.06853811)
\curveto(184.15639956,47.94853323)(184.1213996,47.80353338)(184.10139282,47.63353811)
\curveto(184.08139964,47.47353371)(184.07139965,47.30353388)(184.07139282,47.12353811)
\curveto(184.07139965,46.94353424)(184.08139964,46.76853441)(184.10139282,46.59853811)
\curveto(184.1213996,46.42853475)(184.15139957,46.2835349)(184.19139282,46.16353811)
\curveto(184.25139947,45.99353519)(184.33639938,45.82853535)(184.44639282,45.66853811)
\curveto(184.50639921,45.58853559)(184.58639913,45.51353567)(184.68639282,45.44353811)
\curveto(184.77639894,45.3835358)(184.87639884,45.32853585)(184.98639282,45.27853811)
\curveto(185.06639865,45.24853593)(185.15139857,45.21853596)(185.24139282,45.18853811)
\curveto(185.33139839,45.16853601)(185.40139832,45.12353606)(185.45139282,45.05353811)
\curveto(185.48139824,45.01353617)(185.50639821,44.94353624)(185.52639282,44.84353811)
\curveto(185.53639818,44.75353643)(185.54139818,44.65853652)(185.54139282,44.55853811)
\curveto(185.54139818,44.45853672)(185.53639818,44.35853682)(185.52639282,44.25853811)
\curveto(185.50639821,44.16853701)(185.48139824,44.10353708)(185.45139282,44.06353811)
\curveto(185.4213983,44.02353716)(185.37139835,43.99353719)(185.30139282,43.97353811)
\curveto(185.23139849,43.95353723)(185.15639856,43.95353723)(185.07639282,43.97353811)
\curveto(184.94639877,44.00353718)(184.82639889,44.03353715)(184.71639282,44.06353811)
\curveto(184.59639912,44.10353708)(184.48139924,44.14853703)(184.37139282,44.19853811)
\curveto(184.0213997,44.38853679)(183.75139997,44.62853655)(183.56139282,44.91853811)
\curveto(183.36140036,45.20853597)(183.20140052,45.56853561)(183.08139282,45.99853811)
\curveto(183.06140066,46.09853508)(183.04640067,46.19853498)(183.03639282,46.29853811)
\curveto(183.02640069,46.40853477)(183.01140071,46.51853466)(182.99139282,46.62853811)
\curveto(182.98140074,46.66853451)(182.98140074,46.73353445)(182.99139282,46.82353811)
\curveto(182.99140073,46.91353427)(182.98140074,46.96853421)(182.96139282,46.98853811)
\curveto(182.95140077,47.68853349)(183.03140069,48.29853288)(183.20139282,48.81853811)
\curveto(183.37140035,49.33853184)(183.69640002,49.70353148)(184.17639282,49.91353811)
\curveto(184.37639934,50.00353118)(184.61139911,50.05353113)(184.88139282,50.06353811)
\curveto(185.14139858,50.0835311)(185.4163983,50.09353109)(185.70639282,50.09353811)
\lineto(189.02139282,50.09353811)
\curveto(189.16139456,50.09353109)(189.29639442,50.09853108)(189.42639282,50.10853811)
\curveto(189.55639416,50.11853106)(189.66139406,50.14853103)(189.74139282,50.19853811)
\curveto(189.81139391,50.24853093)(189.86139386,50.31353087)(189.89139282,50.39353811)
\curveto(189.93139379,50.4835307)(189.96139376,50.56853061)(189.98139282,50.64853811)
\curveto(189.99139373,50.72853045)(190.03639368,50.78853039)(190.11639282,50.82853811)
\curveto(190.14639357,50.84853033)(190.17639354,50.85853032)(190.20639282,50.85853811)
\curveto(190.23639348,50.85853032)(190.27639344,50.86353032)(190.32639282,50.87353811)
\moveto(188.66139282,48.72853811)
\curveto(188.5213952,48.78853239)(188.36139536,48.81853236)(188.18139282,48.81853811)
\curveto(187.99139573,48.82853235)(187.79639592,48.83353235)(187.59639282,48.83353811)
\curveto(187.48639623,48.83353235)(187.38639633,48.82853235)(187.29639282,48.81853811)
\curveto(187.20639651,48.80853237)(187.13639658,48.76853241)(187.08639282,48.69853811)
\curveto(187.06639665,48.66853251)(187.05639666,48.59853258)(187.05639282,48.48853811)
\curveto(187.07639664,48.46853271)(187.08639663,48.43353275)(187.08639282,48.38353811)
\curveto(187.08639663,48.33353285)(187.09639662,48.28853289)(187.11639282,48.24853811)
\curveto(187.13639658,48.16853301)(187.15639656,48.0785331)(187.17639282,47.97853811)
\lineto(187.23639282,47.67853811)
\curveto(187.23639648,47.64853353)(187.24139648,47.61353357)(187.25139282,47.57353811)
\lineto(187.25139282,47.46853811)
\curveto(187.29139643,47.31853386)(187.3163964,47.15353403)(187.32639282,46.97353811)
\curveto(187.32639639,46.80353438)(187.34639637,46.64353454)(187.38639282,46.49353811)
\curveto(187.40639631,46.41353477)(187.42639629,46.33853484)(187.44639282,46.26853811)
\curveto(187.45639626,46.20853497)(187.47139625,46.13853504)(187.49139282,46.05853811)
\curveto(187.54139618,45.89853528)(187.60639611,45.74853543)(187.68639282,45.60853811)
\curveto(187.75639596,45.46853571)(187.84639587,45.34853583)(187.95639282,45.24853811)
\curveto(188.06639565,45.14853603)(188.20139552,45.07353611)(188.36139282,45.02353811)
\curveto(188.51139521,44.97353621)(188.69639502,44.95353623)(188.91639282,44.96353811)
\curveto(189.0163947,44.96353622)(189.11139461,44.9785362)(189.20139282,45.00853811)
\curveto(189.28139444,45.04853613)(189.35639436,45.09353609)(189.42639282,45.14353811)
\curveto(189.53639418,45.22353596)(189.63139409,45.32853585)(189.71139282,45.45853811)
\curveto(189.78139394,45.58853559)(189.84139388,45.72853545)(189.89139282,45.87853811)
\curveto(189.90139382,45.92853525)(189.90639381,45.9785352)(189.90639282,46.02853811)
\curveto(189.90639381,46.0785351)(189.91139381,46.12853505)(189.92139282,46.17853811)
\curveto(189.94139378,46.24853493)(189.95639376,46.33353485)(189.96639282,46.43353811)
\curveto(189.96639375,46.54353464)(189.95639376,46.63353455)(189.93639282,46.70353811)
\curveto(189.9163938,46.76353442)(189.91139381,46.82353436)(189.92139282,46.88353811)
\curveto(189.9213938,46.94353424)(189.91139381,47.00353418)(189.89139282,47.06353811)
\curveto(189.87139385,47.14353404)(189.85639386,47.21853396)(189.84639282,47.28853811)
\curveto(189.83639388,47.36853381)(189.8163939,47.44353374)(189.78639282,47.51353811)
\curveto(189.66639405,47.80353338)(189.5213942,48.04853313)(189.35139282,48.24853811)
\curveto(189.18139454,48.45853272)(188.95139477,48.61853256)(188.66139282,48.72853811)
}
}
{
\newrgbcolor{curcolor}{0 0 0}
\pscustom[linestyle=none,fillstyle=solid,fillcolor=curcolor]
{
\newpath
\moveto(182.97639282,55.69017873)
\curveto(182.97640074,55.92017394)(183.03640068,56.05017381)(183.15639282,56.08017873)
\curveto(183.26640045,56.11017375)(183.43140029,56.12517374)(183.65139282,56.12517873)
\lineto(183.93639282,56.12517873)
\curveto(184.02639969,56.12517374)(184.10139962,56.10017376)(184.16139282,56.05017873)
\curveto(184.24139948,55.99017387)(184.28639943,55.90517396)(184.29639282,55.79517873)
\curveto(184.29639942,55.68517418)(184.31139941,55.57517429)(184.34139282,55.46517873)
\curveto(184.37139935,55.32517454)(184.40139932,55.19017467)(184.43139282,55.06017873)
\curveto(184.46139926,54.94017492)(184.50139922,54.82517504)(184.55139282,54.71517873)
\curveto(184.68139904,54.42517544)(184.86139886,54.19017567)(185.09139282,54.01017873)
\curveto(185.31139841,53.83017603)(185.56639815,53.67517619)(185.85639282,53.54517873)
\curveto(185.96639775,53.50517636)(186.08139764,53.47517639)(186.20139282,53.45517873)
\curveto(186.31139741,53.43517643)(186.42639729,53.41017645)(186.54639282,53.38017873)
\curveto(186.59639712,53.37017649)(186.64639707,53.3651765)(186.69639282,53.36517873)
\curveto(186.74639697,53.37517649)(186.79639692,53.37517649)(186.84639282,53.36517873)
\curveto(186.96639675,53.33517653)(187.10639661,53.32017654)(187.26639282,53.32017873)
\curveto(187.4163963,53.33017653)(187.56139616,53.33517653)(187.70139282,53.33517873)
\lineto(189.54639282,53.33517873)
\lineto(189.89139282,53.33517873)
\curveto(190.01139371,53.33517653)(190.12639359,53.33017653)(190.23639282,53.32017873)
\curveto(190.34639337,53.31017655)(190.44139328,53.30517656)(190.52139282,53.30517873)
\curveto(190.60139312,53.31517655)(190.67139305,53.29517657)(190.73139282,53.24517873)
\curveto(190.80139292,53.19517667)(190.84139288,53.11517675)(190.85139282,53.00517873)
\curveto(190.86139286,52.90517696)(190.86639285,52.79517707)(190.86639282,52.67517873)
\lineto(190.86639282,52.40517873)
\curveto(190.84639287,52.35517751)(190.83139289,52.30517756)(190.82139282,52.25517873)
\curveto(190.80139292,52.21517765)(190.77639294,52.18517768)(190.74639282,52.16517873)
\curveto(190.67639304,52.11517775)(190.59139313,52.08517778)(190.49139282,52.07517873)
\lineto(190.16139282,52.07517873)
\lineto(189.00639282,52.07517873)
\lineto(184.85139282,52.07517873)
\lineto(183.81639282,52.07517873)
\lineto(183.51639282,52.07517873)
\curveto(183.4164003,52.08517778)(183.33140039,52.11517775)(183.26139282,52.16517873)
\curveto(183.2214005,52.19517767)(183.19140053,52.24517762)(183.17139282,52.31517873)
\curveto(183.15140057,52.39517747)(183.14140058,52.48017738)(183.14139282,52.57017873)
\curveto(183.13140059,52.6601772)(183.13140059,52.75017711)(183.14139282,52.84017873)
\curveto(183.15140057,52.93017693)(183.16640055,53.00017686)(183.18639282,53.05017873)
\curveto(183.2164005,53.13017673)(183.27640044,53.18017668)(183.36639282,53.20017873)
\curveto(183.44640027,53.23017663)(183.53640018,53.24517662)(183.63639282,53.24517873)
\lineto(183.93639282,53.24517873)
\curveto(184.03639968,53.24517662)(184.12639959,53.2651766)(184.20639282,53.30517873)
\curveto(184.22639949,53.31517655)(184.24139948,53.32517654)(184.25139282,53.33517873)
\lineto(184.29639282,53.38017873)
\curveto(184.29639942,53.49017637)(184.25139947,53.58017628)(184.16139282,53.65017873)
\curveto(184.06139966,53.72017614)(183.98139974,53.78017608)(183.92139282,53.83017873)
\lineto(183.83139282,53.92017873)
\curveto(183.7214,54.01017585)(183.60640011,54.13517573)(183.48639282,54.29517873)
\curveto(183.36640035,54.45517541)(183.27640044,54.60517526)(183.21639282,54.74517873)
\curveto(183.16640055,54.83517503)(183.13140059,54.93017493)(183.11139282,55.03017873)
\curveto(183.08140064,55.13017473)(183.05140067,55.23517463)(183.02139282,55.34517873)
\curveto(183.01140071,55.40517446)(183.00640071,55.4651744)(183.00639282,55.52517873)
\curveto(182.99640072,55.58517428)(182.98640073,55.64017422)(182.97639282,55.69017873)
}
}
{
\newrgbcolor{curcolor}{0 0 0}
\pscustom[linestyle=none,fillstyle=solid,fillcolor=curcolor]
{
}
}
{
\newrgbcolor{curcolor}{0 0 0}
\pscustom[linestyle=none,fillstyle=solid,fillcolor=curcolor]
{
\newpath
\moveto(180.27639282,65.05510061)
\curveto(180.27640344,65.15509575)(180.28640343,65.25009566)(180.30639282,65.34010061)
\curveto(180.3164034,65.43009548)(180.34640337,65.49509541)(180.39639282,65.53510061)
\curveto(180.47640324,65.59509531)(180.58140314,65.62509528)(180.71139282,65.62510061)
\lineto(181.10139282,65.62510061)
\lineto(182.60139282,65.62510061)
\lineto(188.99139282,65.62510061)
\lineto(190.16139282,65.62510061)
\lineto(190.47639282,65.62510061)
\curveto(190.57639314,65.63509527)(190.65639306,65.62009529)(190.71639282,65.58010061)
\curveto(190.79639292,65.53009538)(190.84639287,65.45509545)(190.86639282,65.35510061)
\curveto(190.87639284,65.26509564)(190.88139284,65.15509575)(190.88139282,65.02510061)
\lineto(190.88139282,64.80010061)
\curveto(190.86139286,64.72009619)(190.84639287,64.65009626)(190.83639282,64.59010061)
\curveto(190.8163929,64.53009638)(190.77639294,64.48009643)(190.71639282,64.44010061)
\curveto(190.65639306,64.40009651)(190.58139314,64.38009653)(190.49139282,64.38010061)
\lineto(190.19139282,64.38010061)
\lineto(189.09639282,64.38010061)
\lineto(183.75639282,64.38010061)
\curveto(183.66640005,64.36009655)(183.59140013,64.34509656)(183.53139282,64.33510061)
\curveto(183.46140026,64.33509657)(183.40140032,64.3050966)(183.35139282,64.24510061)
\curveto(183.30140042,64.17509673)(183.27640044,64.08509682)(183.27639282,63.97510061)
\curveto(183.26640045,63.87509703)(183.26140046,63.76509714)(183.26139282,63.64510061)
\lineto(183.26139282,62.50510061)
\lineto(183.26139282,62.01010061)
\curveto(183.25140047,61.85009906)(183.19140053,61.74009917)(183.08139282,61.68010061)
\curveto(183.05140067,61.66009925)(183.0214007,61.65009926)(182.99139282,61.65010061)
\curveto(182.95140077,61.65009926)(182.90640081,61.64509926)(182.85639282,61.63510061)
\curveto(182.73640098,61.61509929)(182.62640109,61.62009929)(182.52639282,61.65010061)
\curveto(182.42640129,61.69009922)(182.35640136,61.74509916)(182.31639282,61.81510061)
\curveto(182.26640145,61.89509901)(182.24140148,62.01509889)(182.24139282,62.17510061)
\curveto(182.24140148,62.33509857)(182.22640149,62.47009844)(182.19639282,62.58010061)
\curveto(182.18640153,62.63009828)(182.18140154,62.68509822)(182.18139282,62.74510061)
\curveto(182.17140155,62.8050981)(182.15640156,62.86509804)(182.13639282,62.92510061)
\curveto(182.08640163,63.07509783)(182.03640168,63.22009769)(181.98639282,63.36010061)
\curveto(181.92640179,63.50009741)(181.85640186,63.63509727)(181.77639282,63.76510061)
\curveto(181.68640203,63.905097)(181.58140214,64.02509688)(181.46139282,64.12510061)
\curveto(181.34140238,64.22509668)(181.21140251,64.32009659)(181.07139282,64.41010061)
\curveto(180.97140275,64.47009644)(180.86140286,64.51509639)(180.74139282,64.54510061)
\curveto(180.6214031,64.58509632)(180.5164032,64.63509627)(180.42639282,64.69510061)
\curveto(180.36640335,64.74509616)(180.32640339,64.81509609)(180.30639282,64.90510061)
\curveto(180.29640342,64.92509598)(180.29140343,64.95009596)(180.29139282,64.98010061)
\curveto(180.29140343,65.0100959)(180.28640343,65.03509587)(180.27639282,65.05510061)
}
}
{
\newrgbcolor{curcolor}{0 0 0}
\pscustom[linestyle=none,fillstyle=solid,fillcolor=curcolor]
{
\newpath
\moveto(187.38639282,76.22470998)
\curveto(187.43639628,76.29470234)(187.50639621,76.3347023)(187.59639282,76.34470998)
\curveto(187.68639603,76.36470227)(187.79139593,76.37470226)(187.91139282,76.37470998)
\curveto(187.96139576,76.37470226)(188.01139571,76.36970226)(188.06139282,76.35970998)
\curveto(188.11139561,76.35970227)(188.15639556,76.34970228)(188.19639282,76.32970998)
\curveto(188.28639543,76.29970233)(188.34639537,76.23970239)(188.37639282,76.14970998)
\curveto(188.39639532,76.06970256)(188.40639531,75.97470266)(188.40639282,75.86470998)
\lineto(188.40639282,75.54970998)
\curveto(188.39639532,75.43970319)(188.40639531,75.3347033)(188.43639282,75.23470998)
\curveto(188.46639525,75.09470354)(188.54639517,75.00470363)(188.67639282,74.96470998)
\curveto(188.74639497,74.94470369)(188.83139489,74.9347037)(188.93139282,74.93470998)
\lineto(189.20139282,74.93470998)
\lineto(190.14639282,74.93470998)
\lineto(190.47639282,74.93470998)
\curveto(190.58639313,74.9347037)(190.67139305,74.91470372)(190.73139282,74.87470998)
\curveto(190.79139293,74.8347038)(190.83139289,74.78470385)(190.85139282,74.72470998)
\curveto(190.86139286,74.67470396)(190.87639284,74.60970402)(190.89639282,74.52970998)
\lineto(190.89639282,74.33470998)
\curveto(190.89639282,74.21470442)(190.89139283,74.10970452)(190.88139282,74.01970998)
\curveto(190.86139286,73.9297047)(190.81139291,73.85970477)(190.73139282,73.80970998)
\curveto(190.68139304,73.77970485)(190.61139311,73.76470487)(190.52139282,73.76470998)
\lineto(190.22139282,73.76470998)
\lineto(189.18639282,73.76470998)
\curveto(189.02639469,73.76470487)(188.88139484,73.75470488)(188.75139282,73.73470998)
\curveto(188.61139511,73.72470491)(188.5163952,73.66970496)(188.46639282,73.56970998)
\curveto(188.44639527,73.51970511)(188.43139529,73.44970518)(188.42139282,73.35970998)
\curveto(188.41139531,73.27970535)(188.40639531,73.18970544)(188.40639282,73.08970998)
\lineto(188.40639282,72.80470998)
\lineto(188.40639282,72.56470998)
\lineto(188.40639282,70.29970998)
\curveto(188.40639531,70.20970842)(188.41139531,70.10470853)(188.42139282,69.98470998)
\lineto(188.42139282,69.65470998)
\curveto(188.4213953,69.54470909)(188.41139531,69.44470919)(188.39139282,69.35470998)
\curveto(188.37139535,69.26470937)(188.33639538,69.20470943)(188.28639282,69.17470998)
\curveto(188.2163955,69.12470951)(188.1213956,69.09970953)(188.00139282,69.09970998)
\lineto(187.65639282,69.09970998)
\lineto(187.38639282,69.09970998)
\curveto(187.2163965,69.13970949)(187.07639664,69.19470944)(186.96639282,69.26470998)
\curveto(186.85639686,69.3347093)(186.74139698,69.41470922)(186.62139282,69.50470998)
\lineto(186.08139282,69.86470998)
\curveto(185.45139827,70.30470833)(184.83139889,70.73970789)(184.22139282,71.16970998)
\lineto(182.36139282,72.48970998)
\curveto(182.13140159,72.64970598)(181.91140181,72.80470583)(181.70139282,72.95470998)
\curveto(181.48140224,73.10470553)(181.25640246,73.25970537)(181.02639282,73.41970998)
\curveto(180.95640276,73.46970516)(180.89140283,73.51970511)(180.83139282,73.56970998)
\curveto(180.76140296,73.61970501)(180.68640303,73.66970496)(180.60639282,73.71970998)
\lineto(180.51639282,73.77970998)
\curveto(180.47640324,73.80970482)(180.44640327,73.83970479)(180.42639282,73.86970998)
\curveto(180.39640332,73.90970472)(180.37640334,73.94970468)(180.36639282,73.98970998)
\curveto(180.34640337,74.0297046)(180.32640339,74.07470456)(180.30639282,74.12470998)
\curveto(180.30640341,74.14470449)(180.31140341,74.16470447)(180.32139282,74.18470998)
\curveto(180.3214034,74.21470442)(180.31140341,74.23970439)(180.29139282,74.25970998)
\curveto(180.29140343,74.38970424)(180.29640342,74.50970412)(180.30639282,74.61970998)
\curveto(180.3164034,74.7297039)(180.36140336,74.80970382)(180.44139282,74.85970998)
\curveto(180.49140323,74.89970373)(180.56140316,74.91970371)(180.65139282,74.91970998)
\curveto(180.74140298,74.9297037)(180.83640288,74.9347037)(180.93639282,74.93470998)
\lineto(186.39639282,74.93470998)
\curveto(186.46639725,74.9347037)(186.54139718,74.9297037)(186.62139282,74.91970998)
\curveto(186.69139703,74.91970371)(186.76139696,74.92470371)(186.83139282,74.93470998)
\lineto(186.93639282,74.93470998)
\curveto(186.98639673,74.95470368)(187.04139668,74.96970366)(187.10139282,74.97970998)
\curveto(187.15139657,74.98970364)(187.19139653,75.01470362)(187.22139282,75.05470998)
\curveto(187.27139645,75.12470351)(187.30139642,75.20970342)(187.31139282,75.30970998)
\lineto(187.31139282,75.63970998)
\curveto(187.31139641,75.74970288)(187.3163964,75.85470278)(187.32639282,75.95470998)
\curveto(187.32639639,76.06470257)(187.34639637,76.15470248)(187.38639282,76.22470998)
\moveto(187.19139282,73.65970998)
\curveto(187.08139664,73.73970489)(186.91139681,73.77470486)(186.68139282,73.76470998)
\lineto(186.06639282,73.76470998)
\lineto(183.59139282,73.76470998)
\lineto(183.27639282,73.76470998)
\curveto(183.15640056,73.77470486)(183.05640066,73.76970486)(182.97639282,73.74970998)
\lineto(182.82639282,73.74970998)
\curveto(182.73640098,73.74970488)(182.65140107,73.7347049)(182.57139282,73.70470998)
\curveto(182.55140117,73.69470494)(182.54140118,73.68470495)(182.54139282,73.67470998)
\lineto(182.49639282,73.62970998)
\curveto(182.48640123,73.60970502)(182.48140124,73.57970505)(182.48139282,73.53970998)
\curveto(182.50140122,73.51970511)(182.5164012,73.49970513)(182.52639282,73.47970998)
\curveto(182.52640119,73.46970516)(182.53140119,73.45470518)(182.54139282,73.43470998)
\curveto(182.59140113,73.37470526)(182.66140106,73.31470532)(182.75139282,73.25470998)
\curveto(182.84140088,73.19470544)(182.9214008,73.13970549)(182.99139282,73.08970998)
\curveto(183.13140059,72.98970564)(183.27640044,72.89470574)(183.42639282,72.80470998)
\curveto(183.56640015,72.71470592)(183.70640001,72.61970601)(183.84639282,72.51970998)
\lineto(184.62639282,71.97970998)
\curveto(184.88639883,71.80970682)(185.14639857,71.634707)(185.40639282,71.45470998)
\curveto(185.5163982,71.37470726)(185.6213981,71.29970733)(185.72139282,71.22970998)
\lineto(186.02139282,71.01970998)
\curveto(186.10139762,70.96970766)(186.17639754,70.91970771)(186.24639282,70.86970998)
\curveto(186.3163974,70.8297078)(186.39139733,70.78470785)(186.47139282,70.73470998)
\curveto(186.53139719,70.68470795)(186.59639712,70.634708)(186.66639282,70.58470998)
\curveto(186.72639699,70.54470809)(186.79639692,70.50470813)(186.87639282,70.46470998)
\curveto(186.93639678,70.42470821)(187.00639671,70.39970823)(187.08639282,70.38970998)
\curveto(187.15639656,70.37970825)(187.21139651,70.41470822)(187.25139282,70.49470998)
\curveto(187.30139642,70.56470807)(187.32639639,70.67470796)(187.32639282,70.82470998)
\curveto(187.3163964,70.98470765)(187.31139641,71.11970751)(187.31139282,71.22970998)
\lineto(187.31139282,72.90970998)
\lineto(187.31139282,73.34470998)
\curveto(187.31139641,73.49470514)(187.27139645,73.59970503)(187.19139282,73.65970998)
}
}
{
\newrgbcolor{curcolor}{0 0 0}
\pscustom[linestyle=none,fillstyle=solid,fillcolor=curcolor]
{
\newpath
\moveto(189.24639282,78.63431936)
\lineto(189.24639282,79.26431936)
\lineto(189.24639282,79.45931936)
\curveto(189.24639447,79.52931683)(189.25639446,79.58931677)(189.27639282,79.63931936)
\curveto(189.3163944,79.70931665)(189.35639436,79.7593166)(189.39639282,79.78931936)
\curveto(189.44639427,79.82931653)(189.51139421,79.84931651)(189.59139282,79.84931936)
\curveto(189.67139405,79.8593165)(189.75639396,79.86431649)(189.84639282,79.86431936)
\lineto(190.56639282,79.86431936)
\curveto(191.04639267,79.86431649)(191.45639226,79.80431655)(191.79639282,79.68431936)
\curveto(192.13639158,79.56431679)(192.41139131,79.36931699)(192.62139282,79.09931936)
\curveto(192.67139105,79.02931733)(192.716391,78.9593174)(192.75639282,78.88931936)
\curveto(192.80639091,78.82931753)(192.85139087,78.7543176)(192.89139282,78.66431936)
\curveto(192.90139082,78.64431771)(192.91139081,78.61431774)(192.92139282,78.57431936)
\curveto(192.94139078,78.53431782)(192.94639077,78.48931787)(192.93639282,78.43931936)
\curveto(192.90639081,78.34931801)(192.83139089,78.29431806)(192.71139282,78.27431936)
\curveto(192.60139112,78.2543181)(192.50639121,78.26931809)(192.42639282,78.31931936)
\curveto(192.35639136,78.34931801)(192.29139143,78.39431796)(192.23139282,78.45431936)
\curveto(192.18139154,78.52431783)(192.13139159,78.58931777)(192.08139282,78.64931936)
\curveto(192.03139169,78.71931764)(191.95639176,78.77931758)(191.85639282,78.82931936)
\curveto(191.76639195,78.88931747)(191.67639204,78.93931742)(191.58639282,78.97931936)
\curveto(191.55639216,78.99931736)(191.49639222,79.02431733)(191.40639282,79.05431936)
\curveto(191.32639239,79.08431727)(191.25639246,79.08931727)(191.19639282,79.06931936)
\curveto(191.05639266,79.03931732)(190.96639275,78.97931738)(190.92639282,78.88931936)
\curveto(190.89639282,78.80931755)(190.88139284,78.71931764)(190.88139282,78.61931936)
\curveto(190.88139284,78.51931784)(190.85639286,78.43431792)(190.80639282,78.36431936)
\curveto(190.73639298,78.27431808)(190.59639312,78.22931813)(190.38639282,78.22931936)
\lineto(189.83139282,78.22931936)
\lineto(189.60639282,78.22931936)
\curveto(189.52639419,78.23931812)(189.46139426,78.2593181)(189.41139282,78.28931936)
\curveto(189.33139439,78.34931801)(189.28639443,78.41931794)(189.27639282,78.49931936)
\curveto(189.26639445,78.51931784)(189.26139446,78.53931782)(189.26139282,78.55931936)
\curveto(189.26139446,78.58931777)(189.25639446,78.61431774)(189.24639282,78.63431936)
}
}
{
\newrgbcolor{curcolor}{0 0 0}
\pscustom[linestyle=none,fillstyle=solid,fillcolor=curcolor]
{
}
}
{
\newrgbcolor{curcolor}{0 0 0}
\pscustom[linestyle=none,fillstyle=solid,fillcolor=curcolor]
{
\newpath
\moveto(180.27639282,89.26463186)
\curveto(180.26640345,89.95462722)(180.38640333,90.55462662)(180.63639282,91.06463186)
\curveto(180.88640283,91.58462559)(181.2214025,91.9796252)(181.64139282,92.24963186)
\curveto(181.721402,92.29962488)(181.81140191,92.34462483)(181.91139282,92.38463186)
\curveto(182.00140172,92.42462475)(182.09640162,92.46962471)(182.19639282,92.51963186)
\curveto(182.29640142,92.55962462)(182.39640132,92.58962459)(182.49639282,92.60963186)
\curveto(182.59640112,92.62962455)(182.70140102,92.64962453)(182.81139282,92.66963186)
\curveto(182.86140086,92.68962449)(182.90640081,92.69462448)(182.94639282,92.68463186)
\curveto(182.98640073,92.6746245)(183.03140069,92.6796245)(183.08139282,92.69963186)
\curveto(183.13140059,92.70962447)(183.2164005,92.71462446)(183.33639282,92.71463186)
\curveto(183.44640027,92.71462446)(183.53140019,92.70962447)(183.59139282,92.69963186)
\curveto(183.65140007,92.6796245)(183.71140001,92.66962451)(183.77139282,92.66963186)
\curveto(183.83139989,92.6796245)(183.89139983,92.6746245)(183.95139282,92.65463186)
\curveto(184.09139963,92.61462456)(184.22639949,92.5796246)(184.35639282,92.54963186)
\curveto(184.48639923,92.51962466)(184.61139911,92.4796247)(184.73139282,92.42963186)
\curveto(184.87139885,92.36962481)(184.99639872,92.29962488)(185.10639282,92.21963186)
\curveto(185.2163985,92.14962503)(185.32639839,92.0746251)(185.43639282,91.99463186)
\lineto(185.49639282,91.93463186)
\curveto(185.5163982,91.92462525)(185.53639818,91.90962527)(185.55639282,91.88963186)
\curveto(185.716398,91.76962541)(185.86139786,91.63462554)(185.99139282,91.48463186)
\curveto(186.1213976,91.33462584)(186.24639747,91.174626)(186.36639282,91.00463186)
\curveto(186.58639713,90.69462648)(186.79139693,90.39962678)(186.98139282,90.11963186)
\curveto(187.1213966,89.88962729)(187.25639646,89.65962752)(187.38639282,89.42963186)
\curveto(187.5163962,89.20962797)(187.65139607,88.98962819)(187.79139282,88.76963186)
\curveto(187.96139576,88.51962866)(188.14139558,88.2796289)(188.33139282,88.04963186)
\curveto(188.5213952,87.82962935)(188.74639497,87.63962954)(189.00639282,87.47963186)
\curveto(189.06639465,87.43962974)(189.12639459,87.40462977)(189.18639282,87.37463186)
\curveto(189.23639448,87.34462983)(189.30139442,87.31462986)(189.38139282,87.28463186)
\curveto(189.45139427,87.26462991)(189.51139421,87.25962992)(189.56139282,87.26963186)
\curveto(189.63139409,87.28962989)(189.68639403,87.32462985)(189.72639282,87.37463186)
\curveto(189.75639396,87.42462975)(189.77639394,87.48462969)(189.78639282,87.55463186)
\lineto(189.78639282,87.79463186)
\lineto(189.78639282,88.54463186)
\lineto(189.78639282,91.34963186)
\lineto(189.78639282,92.00963186)
\curveto(189.78639393,92.09962508)(189.79139393,92.18462499)(189.80139282,92.26463186)
\curveto(189.80139392,92.34462483)(189.8213939,92.40962477)(189.86139282,92.45963186)
\curveto(189.90139382,92.50962467)(189.97639374,92.54962463)(190.08639282,92.57963186)
\curveto(190.18639353,92.61962456)(190.28639343,92.62962455)(190.38639282,92.60963186)
\lineto(190.52139282,92.60963186)
\curveto(190.59139313,92.58962459)(190.65139307,92.56962461)(190.70139282,92.54963186)
\curveto(190.75139297,92.52962465)(190.79139293,92.49462468)(190.82139282,92.44463186)
\curveto(190.86139286,92.39462478)(190.88139284,92.32462485)(190.88139282,92.23463186)
\lineto(190.88139282,91.96463186)
\lineto(190.88139282,91.06463186)
\lineto(190.88139282,87.55463186)
\lineto(190.88139282,86.48963186)
\curveto(190.88139284,86.40963077)(190.88639283,86.31963086)(190.89639282,86.21963186)
\curveto(190.89639282,86.11963106)(190.88639283,86.03463114)(190.86639282,85.96463186)
\curveto(190.79639292,85.75463142)(190.6163931,85.68963149)(190.32639282,85.76963186)
\curveto(190.28639343,85.7796314)(190.25139347,85.7796314)(190.22139282,85.76963186)
\curveto(190.18139354,85.76963141)(190.13639358,85.7796314)(190.08639282,85.79963186)
\curveto(190.00639371,85.81963136)(189.9213938,85.83963134)(189.83139282,85.85963186)
\curveto(189.74139398,85.8796313)(189.65639406,85.90463127)(189.57639282,85.93463186)
\curveto(189.08639463,86.09463108)(188.67139505,86.29463088)(188.33139282,86.53463186)
\curveto(188.08139564,86.71463046)(187.85639586,86.91963026)(187.65639282,87.14963186)
\curveto(187.44639627,87.3796298)(187.25139647,87.61962956)(187.07139282,87.86963186)
\curveto(186.89139683,88.12962905)(186.721397,88.39462878)(186.56139282,88.66463186)
\curveto(186.39139733,88.94462823)(186.2163975,89.21462796)(186.03639282,89.47463186)
\curveto(185.95639776,89.58462759)(185.88139784,89.68962749)(185.81139282,89.78963186)
\curveto(185.74139798,89.89962728)(185.66639805,90.00962717)(185.58639282,90.11963186)
\curveto(185.55639816,90.15962702)(185.52639819,90.19462698)(185.49639282,90.22463186)
\curveto(185.45639826,90.26462691)(185.42639829,90.30462687)(185.40639282,90.34463186)
\curveto(185.29639842,90.48462669)(185.17139855,90.60962657)(185.03139282,90.71963186)
\curveto(185.00139872,90.73962644)(184.97639874,90.76462641)(184.95639282,90.79463186)
\curveto(184.92639879,90.82462635)(184.89639882,90.84962633)(184.86639282,90.86963186)
\curveto(184.76639895,90.94962623)(184.66639905,91.01462616)(184.56639282,91.06463186)
\curveto(184.46639925,91.12462605)(184.35639936,91.179626)(184.23639282,91.22963186)
\curveto(184.16639955,91.25962592)(184.09139963,91.2796259)(184.01139282,91.28963186)
\lineto(183.77139282,91.34963186)
\lineto(183.68139282,91.34963186)
\curveto(183.65140007,91.35962582)(183.6214001,91.36462581)(183.59139282,91.36463186)
\curveto(183.5214002,91.38462579)(183.42640029,91.38962579)(183.30639282,91.37963186)
\curveto(183.17640054,91.3796258)(183.07640064,91.36962581)(183.00639282,91.34963186)
\curveto(182.92640079,91.32962585)(182.85140087,91.30962587)(182.78139282,91.28963186)
\curveto(182.70140102,91.2796259)(182.6214011,91.25962592)(182.54139282,91.22963186)
\curveto(182.30140142,91.11962606)(182.10140162,90.96962621)(181.94139282,90.77963186)
\curveto(181.77140195,90.59962658)(181.63140209,90.3796268)(181.52139282,90.11963186)
\curveto(181.50140222,90.04962713)(181.48640223,89.9796272)(181.47639282,89.90963186)
\curveto(181.45640226,89.83962734)(181.43640228,89.76462741)(181.41639282,89.68463186)
\curveto(181.39640232,89.60462757)(181.38640233,89.49462768)(181.38639282,89.35463186)
\curveto(181.38640233,89.22462795)(181.39640232,89.11962806)(181.41639282,89.03963186)
\curveto(181.42640229,88.9796282)(181.43140229,88.92462825)(181.43139282,88.87463186)
\curveto(181.43140229,88.82462835)(181.44140228,88.7746284)(181.46139282,88.72463186)
\curveto(181.50140222,88.62462855)(181.54140218,88.52962865)(181.58139282,88.43963186)
\curveto(181.6214021,88.35962882)(181.66640205,88.2796289)(181.71639282,88.19963186)
\curveto(181.73640198,88.16962901)(181.76140196,88.13962904)(181.79139282,88.10963186)
\curveto(181.8214019,88.08962909)(181.84640187,88.06462911)(181.86639282,88.03463186)
\lineto(181.94139282,87.95963186)
\curveto(181.96140176,87.92962925)(181.98140174,87.90462927)(182.00139282,87.88463186)
\lineto(182.21139282,87.73463186)
\curveto(182.27140145,87.69462948)(182.33640138,87.64962953)(182.40639282,87.59963186)
\curveto(182.49640122,87.53962964)(182.60140112,87.48962969)(182.72139282,87.44963186)
\curveto(182.83140089,87.41962976)(182.94140078,87.38462979)(183.05139282,87.34463186)
\curveto(183.16140056,87.30462987)(183.30640041,87.2796299)(183.48639282,87.26963186)
\curveto(183.65640006,87.25962992)(183.78139994,87.22962995)(183.86139282,87.17963186)
\curveto(183.94139978,87.12963005)(183.98639973,87.05463012)(183.99639282,86.95463186)
\curveto(184.00639971,86.85463032)(184.01139971,86.74463043)(184.01139282,86.62463186)
\curveto(184.01139971,86.58463059)(184.0163997,86.54463063)(184.02639282,86.50463186)
\curveto(184.02639969,86.46463071)(184.0213997,86.42963075)(184.01139282,86.39963186)
\curveto(183.99139973,86.34963083)(183.98139974,86.29963088)(183.98139282,86.24963186)
\curveto(183.98139974,86.20963097)(183.97139975,86.16963101)(183.95139282,86.12963186)
\curveto(183.89139983,86.03963114)(183.75639996,85.99463118)(183.54639282,85.99463186)
\lineto(183.42639282,85.99463186)
\curveto(183.36640035,86.00463117)(183.30640041,86.00963117)(183.24639282,86.00963186)
\curveto(183.17640054,86.01963116)(183.11140061,86.02963115)(183.05139282,86.03963186)
\curveto(182.94140078,86.05963112)(182.84140088,86.0796311)(182.75139282,86.09963186)
\curveto(182.65140107,86.11963106)(182.55640116,86.14963103)(182.46639282,86.18963186)
\curveto(182.39640132,86.20963097)(182.33640138,86.22963095)(182.28639282,86.24963186)
\lineto(182.10639282,86.30963186)
\curveto(181.84640187,86.42963075)(181.60140212,86.58463059)(181.37139282,86.77463186)
\curveto(181.14140258,86.9746302)(180.95640276,87.18962999)(180.81639282,87.41963186)
\curveto(180.73640298,87.52962965)(180.67140305,87.64462953)(180.62139282,87.76463186)
\lineto(180.47139282,88.15463186)
\curveto(180.4214033,88.26462891)(180.39140333,88.3796288)(180.38139282,88.49963186)
\curveto(180.36140336,88.61962856)(180.33640338,88.74462843)(180.30639282,88.87463186)
\curveto(180.30640341,88.94462823)(180.30640341,89.00962817)(180.30639282,89.06963186)
\curveto(180.29640342,89.12962805)(180.28640343,89.19462798)(180.27639282,89.26463186)
}
}
{
\newrgbcolor{curcolor}{0 0 0}
\pscustom[linestyle=none,fillstyle=solid,fillcolor=curcolor]
{
\newpath
\moveto(185.79639282,101.36424123)
\lineto(186.05139282,101.36424123)
\curveto(186.13139759,101.37423353)(186.20639751,101.36923353)(186.27639282,101.34924123)
\lineto(186.51639282,101.34924123)
\lineto(186.68139282,101.34924123)
\curveto(186.78139694,101.32923357)(186.88639683,101.31923358)(186.99639282,101.31924123)
\curveto(187.09639662,101.31923358)(187.19639652,101.30923359)(187.29639282,101.28924123)
\lineto(187.44639282,101.28924123)
\curveto(187.58639613,101.25923364)(187.72639599,101.23923366)(187.86639282,101.22924123)
\curveto(187.99639572,101.21923368)(188.12639559,101.19423371)(188.25639282,101.15424123)
\curveto(188.33639538,101.13423377)(188.4213953,101.11423379)(188.51139282,101.09424123)
\lineto(188.75139282,101.03424123)
\lineto(189.05139282,100.91424123)
\curveto(189.14139458,100.88423402)(189.23139449,100.84923405)(189.32139282,100.80924123)
\curveto(189.54139418,100.70923419)(189.75639396,100.57423433)(189.96639282,100.40424123)
\curveto(190.17639354,100.24423466)(190.34639337,100.06923483)(190.47639282,99.87924123)
\curveto(190.5163932,99.82923507)(190.55639316,99.76923513)(190.59639282,99.69924123)
\curveto(190.62639309,99.63923526)(190.66139306,99.57923532)(190.70139282,99.51924123)
\curveto(190.75139297,99.43923546)(190.79139293,99.34423556)(190.82139282,99.23424123)
\curveto(190.85139287,99.12423578)(190.88139284,99.01923588)(190.91139282,98.91924123)
\curveto(190.95139277,98.80923609)(190.97639274,98.6992362)(190.98639282,98.58924123)
\curveto(190.99639272,98.47923642)(191.01139271,98.36423654)(191.03139282,98.24424123)
\curveto(191.04139268,98.2042367)(191.04139268,98.15923674)(191.03139282,98.10924123)
\curveto(191.03139269,98.06923683)(191.03639268,98.02923687)(191.04639282,97.98924123)
\curveto(191.05639266,97.94923695)(191.06139266,97.89423701)(191.06139282,97.82424123)
\curveto(191.06139266,97.75423715)(191.05639266,97.7042372)(191.04639282,97.67424123)
\curveto(191.02639269,97.62423728)(191.0213927,97.57923732)(191.03139282,97.53924123)
\curveto(191.04139268,97.4992374)(191.04139268,97.46423744)(191.03139282,97.43424123)
\lineto(191.03139282,97.34424123)
\curveto(191.01139271,97.28423762)(190.99639272,97.21923768)(190.98639282,97.14924123)
\curveto(190.98639273,97.08923781)(190.98139274,97.02423788)(190.97139282,96.95424123)
\curveto(190.9213928,96.78423812)(190.87139285,96.62423828)(190.82139282,96.47424123)
\curveto(190.77139295,96.32423858)(190.70639301,96.17923872)(190.62639282,96.03924123)
\curveto(190.58639313,95.98923891)(190.55639316,95.93423897)(190.53639282,95.87424123)
\curveto(190.50639321,95.82423908)(190.47139325,95.77423913)(190.43139282,95.72424123)
\curveto(190.25139347,95.48423942)(190.03139369,95.28423962)(189.77139282,95.12424123)
\curveto(189.51139421,94.96423994)(189.22639449,94.82424008)(188.91639282,94.70424123)
\curveto(188.77639494,94.64424026)(188.63639508,94.5992403)(188.49639282,94.56924123)
\curveto(188.34639537,94.53924036)(188.19139553,94.5042404)(188.03139282,94.46424123)
\curveto(187.9213958,94.44424046)(187.81139591,94.42924047)(187.70139282,94.41924123)
\curveto(187.59139613,94.40924049)(187.48139624,94.39424051)(187.37139282,94.37424123)
\curveto(187.33139639,94.36424054)(187.29139643,94.35924054)(187.25139282,94.35924123)
\curveto(187.21139651,94.36924053)(187.17139655,94.36924053)(187.13139282,94.35924123)
\curveto(187.08139664,94.34924055)(187.03139669,94.34424056)(186.98139282,94.34424123)
\lineto(186.81639282,94.34424123)
\curveto(186.76639695,94.32424058)(186.716397,94.31924058)(186.66639282,94.32924123)
\curveto(186.60639711,94.33924056)(186.55139717,94.33924056)(186.50139282,94.32924123)
\curveto(186.46139726,94.31924058)(186.4163973,94.31924058)(186.36639282,94.32924123)
\curveto(186.3163974,94.33924056)(186.26639745,94.33424057)(186.21639282,94.31424123)
\curveto(186.14639757,94.29424061)(186.07139765,94.28924061)(185.99139282,94.29924123)
\curveto(185.90139782,94.30924059)(185.8163979,94.31424059)(185.73639282,94.31424123)
\curveto(185.64639807,94.31424059)(185.54639817,94.30924059)(185.43639282,94.29924123)
\curveto(185.3163984,94.28924061)(185.2163985,94.29424061)(185.13639282,94.31424123)
\lineto(184.85139282,94.31424123)
\lineto(184.22139282,94.35924123)
\curveto(184.1213996,94.36924053)(184.02639969,94.37924052)(183.93639282,94.38924123)
\lineto(183.63639282,94.41924123)
\curveto(183.58640013,94.43924046)(183.53640018,94.44424046)(183.48639282,94.43424123)
\curveto(183.42640029,94.43424047)(183.37140035,94.44424046)(183.32139282,94.46424123)
\curveto(183.15140057,94.51424039)(182.98640073,94.55424035)(182.82639282,94.58424123)
\curveto(182.65640106,94.61424029)(182.49640122,94.66424024)(182.34639282,94.73424123)
\curveto(181.88640183,94.92423998)(181.51140221,95.14423976)(181.22139282,95.39424123)
\curveto(180.93140279,95.65423925)(180.68640303,96.01423889)(180.48639282,96.47424123)
\curveto(180.43640328,96.6042383)(180.40140332,96.73423817)(180.38139282,96.86424123)
\curveto(180.36140336,97.0042379)(180.33640338,97.14423776)(180.30639282,97.28424123)
\curveto(180.29640342,97.35423755)(180.29140343,97.41923748)(180.29139282,97.47924123)
\curveto(180.29140343,97.53923736)(180.28640343,97.6042373)(180.27639282,97.67424123)
\curveto(180.25640346,98.5042364)(180.40640331,99.17423573)(180.72639282,99.68424123)
\curveto(181.03640268,100.19423471)(181.47640224,100.57423433)(182.04639282,100.82424123)
\curveto(182.16640155,100.87423403)(182.29140143,100.91923398)(182.42139282,100.95924123)
\curveto(182.55140117,100.9992339)(182.68640103,101.04423386)(182.82639282,101.09424123)
\curveto(182.90640081,101.11423379)(182.99140073,101.12923377)(183.08139282,101.13924123)
\lineto(183.32139282,101.19924123)
\curveto(183.43140029,101.22923367)(183.54140018,101.24423366)(183.65139282,101.24424123)
\curveto(183.76139996,101.25423365)(183.87139985,101.26923363)(183.98139282,101.28924123)
\curveto(184.03139969,101.30923359)(184.07639964,101.31423359)(184.11639282,101.30424123)
\curveto(184.15639956,101.3042336)(184.19639952,101.30923359)(184.23639282,101.31924123)
\curveto(184.28639943,101.32923357)(184.34139938,101.32923357)(184.40139282,101.31924123)
\curveto(184.45139927,101.31923358)(184.50139922,101.32423358)(184.55139282,101.33424123)
\lineto(184.68639282,101.33424123)
\curveto(184.74639897,101.35423355)(184.8163989,101.35423355)(184.89639282,101.33424123)
\curveto(184.96639875,101.32423358)(185.03139869,101.32923357)(185.09139282,101.34924123)
\curveto(185.1213986,101.35923354)(185.16139856,101.36423354)(185.21139282,101.36424123)
\lineto(185.33139282,101.36424123)
\lineto(185.79639282,101.36424123)
\moveto(188.12139282,99.81924123)
\curveto(187.80139592,99.91923498)(187.43639628,99.97923492)(187.02639282,99.99924123)
\curveto(186.6163971,100.01923488)(186.20639751,100.02923487)(185.79639282,100.02924123)
\curveto(185.36639835,100.02923487)(184.94639877,100.01923488)(184.53639282,99.99924123)
\curveto(184.12639959,99.97923492)(183.74139998,99.93423497)(183.38139282,99.86424123)
\curveto(183.0214007,99.79423511)(182.70140102,99.68423522)(182.42139282,99.53424123)
\curveto(182.13140159,99.39423551)(181.89640182,99.1992357)(181.71639282,98.94924123)
\curveto(181.60640211,98.78923611)(181.52640219,98.60923629)(181.47639282,98.40924123)
\curveto(181.4164023,98.20923669)(181.38640233,97.96423694)(181.38639282,97.67424123)
\curveto(181.40640231,97.65423725)(181.4164023,97.61923728)(181.41639282,97.56924123)
\curveto(181.40640231,97.51923738)(181.40640231,97.47923742)(181.41639282,97.44924123)
\curveto(181.43640228,97.36923753)(181.45640226,97.29423761)(181.47639282,97.22424123)
\curveto(181.48640223,97.16423774)(181.50640221,97.0992378)(181.53639282,97.02924123)
\curveto(181.65640206,96.75923814)(181.82640189,96.53923836)(182.04639282,96.36924123)
\curveto(182.25640146,96.20923869)(182.50140122,96.07423883)(182.78139282,95.96424123)
\curveto(182.89140083,95.91423899)(183.01140071,95.87423903)(183.14139282,95.84424123)
\curveto(183.26140046,95.82423908)(183.38640033,95.7992391)(183.51639282,95.76924123)
\curveto(183.56640015,95.74923915)(183.6214001,95.73923916)(183.68139282,95.73924123)
\curveto(183.73139999,95.73923916)(183.78139994,95.73423917)(183.83139282,95.72424123)
\curveto(183.9213998,95.71423919)(184.0163997,95.7042392)(184.11639282,95.69424123)
\curveto(184.20639951,95.68423922)(184.30139942,95.67423923)(184.40139282,95.66424123)
\curveto(184.48139924,95.66423924)(184.56639915,95.65923924)(184.65639282,95.64924123)
\lineto(184.89639282,95.64924123)
\lineto(185.07639282,95.64924123)
\curveto(185.10639861,95.63923926)(185.14139858,95.63423927)(185.18139282,95.63424123)
\lineto(185.31639282,95.63424123)
\lineto(185.76639282,95.63424123)
\curveto(185.84639787,95.63423927)(185.93139779,95.62923927)(186.02139282,95.61924123)
\curveto(186.10139762,95.61923928)(186.17639754,95.62923927)(186.24639282,95.64924123)
\lineto(186.51639282,95.64924123)
\curveto(186.53639718,95.64923925)(186.56639715,95.64423926)(186.60639282,95.63424123)
\curveto(186.63639708,95.63423927)(186.66139706,95.63923926)(186.68139282,95.64924123)
\curveto(186.78139694,95.65923924)(186.88139684,95.66423924)(186.98139282,95.66424123)
\curveto(187.07139665,95.67423923)(187.17139655,95.68423922)(187.28139282,95.69424123)
\curveto(187.40139632,95.72423918)(187.52639619,95.73923916)(187.65639282,95.73924123)
\curveto(187.77639594,95.74923915)(187.89139583,95.77423913)(188.00139282,95.81424123)
\curveto(188.30139542,95.89423901)(188.56639515,95.97923892)(188.79639282,96.06924123)
\curveto(189.02639469,96.16923873)(189.24139448,96.31423859)(189.44139282,96.50424123)
\curveto(189.64139408,96.71423819)(189.79139393,96.97923792)(189.89139282,97.29924123)
\curveto(189.91139381,97.33923756)(189.9213938,97.37423753)(189.92139282,97.40424123)
\curveto(189.91139381,97.44423746)(189.9163938,97.48923741)(189.93639282,97.53924123)
\curveto(189.94639377,97.57923732)(189.95639376,97.64923725)(189.96639282,97.74924123)
\curveto(189.97639374,97.85923704)(189.97139375,97.94423696)(189.95139282,98.00424123)
\curveto(189.93139379,98.07423683)(189.9213938,98.14423676)(189.92139282,98.21424123)
\curveto(189.91139381,98.28423662)(189.89639382,98.34923655)(189.87639282,98.40924123)
\curveto(189.8163939,98.60923629)(189.73139399,98.78923611)(189.62139282,98.94924123)
\curveto(189.60139412,98.97923592)(189.58139414,99.0042359)(189.56139282,99.02424123)
\lineto(189.50139282,99.08424123)
\curveto(189.48139424,99.12423578)(189.44139428,99.17423573)(189.38139282,99.23424123)
\curveto(189.24139448,99.33423557)(189.11139461,99.41923548)(188.99139282,99.48924123)
\curveto(188.87139485,99.55923534)(188.72639499,99.62923527)(188.55639282,99.69924123)
\curveto(188.48639523,99.72923517)(188.4163953,99.74923515)(188.34639282,99.75924123)
\curveto(188.27639544,99.77923512)(188.20139552,99.7992351)(188.12139282,99.81924123)
}
}
{
\newrgbcolor{curcolor}{0 0 0}
\pscustom[linestyle=none,fillstyle=solid,fillcolor=curcolor]
{
\newpath
\moveto(180.27639282,106.77385061)
\curveto(180.27640344,106.87384575)(180.28640343,106.96884566)(180.30639282,107.05885061)
\curveto(180.3164034,107.14884548)(180.34640337,107.21384541)(180.39639282,107.25385061)
\curveto(180.47640324,107.31384531)(180.58140314,107.34384528)(180.71139282,107.34385061)
\lineto(181.10139282,107.34385061)
\lineto(182.60139282,107.34385061)
\lineto(188.99139282,107.34385061)
\lineto(190.16139282,107.34385061)
\lineto(190.47639282,107.34385061)
\curveto(190.57639314,107.35384527)(190.65639306,107.33884529)(190.71639282,107.29885061)
\curveto(190.79639292,107.24884538)(190.84639287,107.17384545)(190.86639282,107.07385061)
\curveto(190.87639284,106.98384564)(190.88139284,106.87384575)(190.88139282,106.74385061)
\lineto(190.88139282,106.51885061)
\curveto(190.86139286,106.43884619)(190.84639287,106.36884626)(190.83639282,106.30885061)
\curveto(190.8163929,106.24884638)(190.77639294,106.19884643)(190.71639282,106.15885061)
\curveto(190.65639306,106.11884651)(190.58139314,106.09884653)(190.49139282,106.09885061)
\lineto(190.19139282,106.09885061)
\lineto(189.09639282,106.09885061)
\lineto(183.75639282,106.09885061)
\curveto(183.66640005,106.07884655)(183.59140013,106.06384656)(183.53139282,106.05385061)
\curveto(183.46140026,106.05384657)(183.40140032,106.0238466)(183.35139282,105.96385061)
\curveto(183.30140042,105.89384673)(183.27640044,105.80384682)(183.27639282,105.69385061)
\curveto(183.26640045,105.59384703)(183.26140046,105.48384714)(183.26139282,105.36385061)
\lineto(183.26139282,104.22385061)
\lineto(183.26139282,103.72885061)
\curveto(183.25140047,103.56884906)(183.19140053,103.45884917)(183.08139282,103.39885061)
\curveto(183.05140067,103.37884925)(183.0214007,103.36884926)(182.99139282,103.36885061)
\curveto(182.95140077,103.36884926)(182.90640081,103.36384926)(182.85639282,103.35385061)
\curveto(182.73640098,103.33384929)(182.62640109,103.33884929)(182.52639282,103.36885061)
\curveto(182.42640129,103.40884922)(182.35640136,103.46384916)(182.31639282,103.53385061)
\curveto(182.26640145,103.61384901)(182.24140148,103.73384889)(182.24139282,103.89385061)
\curveto(182.24140148,104.05384857)(182.22640149,104.18884844)(182.19639282,104.29885061)
\curveto(182.18640153,104.34884828)(182.18140154,104.40384822)(182.18139282,104.46385061)
\curveto(182.17140155,104.5238481)(182.15640156,104.58384804)(182.13639282,104.64385061)
\curveto(182.08640163,104.79384783)(182.03640168,104.93884769)(181.98639282,105.07885061)
\curveto(181.92640179,105.21884741)(181.85640186,105.35384727)(181.77639282,105.48385061)
\curveto(181.68640203,105.623847)(181.58140214,105.74384688)(181.46139282,105.84385061)
\curveto(181.34140238,105.94384668)(181.21140251,106.03884659)(181.07139282,106.12885061)
\curveto(180.97140275,106.18884644)(180.86140286,106.23384639)(180.74139282,106.26385061)
\curveto(180.6214031,106.30384632)(180.5164032,106.35384627)(180.42639282,106.41385061)
\curveto(180.36640335,106.46384616)(180.32640339,106.53384609)(180.30639282,106.62385061)
\curveto(180.29640342,106.64384598)(180.29140343,106.66884596)(180.29139282,106.69885061)
\curveto(180.29140343,106.7288459)(180.28640343,106.75384587)(180.27639282,106.77385061)
}
}
{
\newrgbcolor{curcolor}{0 0 0}
\pscustom[linestyle=none,fillstyle=solid,fillcolor=curcolor]
{
\newpath
\moveto(180.27639282,115.12345998)
\curveto(180.27640344,115.22345513)(180.28640343,115.31845503)(180.30639282,115.40845998)
\curveto(180.3164034,115.49845485)(180.34640337,115.56345479)(180.39639282,115.60345998)
\curveto(180.47640324,115.66345469)(180.58140314,115.69345466)(180.71139282,115.69345998)
\lineto(181.10139282,115.69345998)
\lineto(182.60139282,115.69345998)
\lineto(188.99139282,115.69345998)
\lineto(190.16139282,115.69345998)
\lineto(190.47639282,115.69345998)
\curveto(190.57639314,115.70345465)(190.65639306,115.68845466)(190.71639282,115.64845998)
\curveto(190.79639292,115.59845475)(190.84639287,115.52345483)(190.86639282,115.42345998)
\curveto(190.87639284,115.33345502)(190.88139284,115.22345513)(190.88139282,115.09345998)
\lineto(190.88139282,114.86845998)
\curveto(190.86139286,114.78845556)(190.84639287,114.71845563)(190.83639282,114.65845998)
\curveto(190.8163929,114.59845575)(190.77639294,114.5484558)(190.71639282,114.50845998)
\curveto(190.65639306,114.46845588)(190.58139314,114.4484559)(190.49139282,114.44845998)
\lineto(190.19139282,114.44845998)
\lineto(189.09639282,114.44845998)
\lineto(183.75639282,114.44845998)
\curveto(183.66640005,114.42845592)(183.59140013,114.41345594)(183.53139282,114.40345998)
\curveto(183.46140026,114.40345595)(183.40140032,114.37345598)(183.35139282,114.31345998)
\curveto(183.30140042,114.24345611)(183.27640044,114.1534562)(183.27639282,114.04345998)
\curveto(183.26640045,113.94345641)(183.26140046,113.83345652)(183.26139282,113.71345998)
\lineto(183.26139282,112.57345998)
\lineto(183.26139282,112.07845998)
\curveto(183.25140047,111.91845843)(183.19140053,111.80845854)(183.08139282,111.74845998)
\curveto(183.05140067,111.72845862)(183.0214007,111.71845863)(182.99139282,111.71845998)
\curveto(182.95140077,111.71845863)(182.90640081,111.71345864)(182.85639282,111.70345998)
\curveto(182.73640098,111.68345867)(182.62640109,111.68845866)(182.52639282,111.71845998)
\curveto(182.42640129,111.75845859)(182.35640136,111.81345854)(182.31639282,111.88345998)
\curveto(182.26640145,111.96345839)(182.24140148,112.08345827)(182.24139282,112.24345998)
\curveto(182.24140148,112.40345795)(182.22640149,112.53845781)(182.19639282,112.64845998)
\curveto(182.18640153,112.69845765)(182.18140154,112.7534576)(182.18139282,112.81345998)
\curveto(182.17140155,112.87345748)(182.15640156,112.93345742)(182.13639282,112.99345998)
\curveto(182.08640163,113.14345721)(182.03640168,113.28845706)(181.98639282,113.42845998)
\curveto(181.92640179,113.56845678)(181.85640186,113.70345665)(181.77639282,113.83345998)
\curveto(181.68640203,113.97345638)(181.58140214,114.09345626)(181.46139282,114.19345998)
\curveto(181.34140238,114.29345606)(181.21140251,114.38845596)(181.07139282,114.47845998)
\curveto(180.97140275,114.53845581)(180.86140286,114.58345577)(180.74139282,114.61345998)
\curveto(180.6214031,114.6534557)(180.5164032,114.70345565)(180.42639282,114.76345998)
\curveto(180.36640335,114.81345554)(180.32640339,114.88345547)(180.30639282,114.97345998)
\curveto(180.29640342,114.99345536)(180.29140343,115.01845533)(180.29139282,115.04845998)
\curveto(180.29140343,115.07845527)(180.28640343,115.10345525)(180.27639282,115.12345998)
}
}
{
\newrgbcolor{curcolor}{0 0 0}
\pscustom[linestyle=none,fillstyle=solid,fillcolor=curcolor]
{
\newpath
\moveto(138.37873047,31.67142873)
\lineto(138.37873047,32.58642873)
\curveto(138.37874116,32.68642608)(138.37874116,32.78142599)(138.37873047,32.87142873)
\curveto(138.37874116,32.96142581)(138.39874114,33.03642573)(138.43873047,33.09642873)
\curveto(138.49874104,33.18642558)(138.57874096,33.24642552)(138.67873047,33.27642873)
\curveto(138.77874076,33.31642545)(138.88374066,33.36142541)(138.99373047,33.41142873)
\curveto(139.18374036,33.49142528)(139.37374017,33.56142521)(139.56373047,33.62142873)
\curveto(139.75373979,33.69142508)(139.9437396,33.766425)(140.13373047,33.84642873)
\curveto(140.31373923,33.91642485)(140.49873904,33.98142479)(140.68873047,34.04142873)
\curveto(140.86873867,34.10142467)(141.04873849,34.1714246)(141.22873047,34.25142873)
\curveto(141.36873817,34.31142446)(141.51373803,34.3664244)(141.66373047,34.41642873)
\curveto(141.81373773,34.4664243)(141.95873758,34.52142425)(142.09873047,34.58142873)
\curveto(142.54873699,34.76142401)(143.00373654,34.93142384)(143.46373047,35.09142873)
\curveto(143.91373563,35.25142352)(144.36373518,35.42142335)(144.81373047,35.60142873)
\curveto(144.86373468,35.62142315)(144.91373463,35.63642313)(144.96373047,35.64642873)
\lineto(145.11373047,35.70642873)
\curveto(145.33373421,35.79642297)(145.55873398,35.88142289)(145.78873047,35.96142873)
\curveto(146.00873353,36.04142273)(146.22873331,36.12642264)(146.44873047,36.21642873)
\curveto(146.538733,36.25642251)(146.64873289,36.29642247)(146.77873047,36.33642873)
\curveto(146.89873264,36.37642239)(146.96873257,36.44142233)(146.98873047,36.53142873)
\curveto(146.99873254,36.5714222)(146.99873254,36.60142217)(146.98873047,36.62142873)
\lineto(146.92873047,36.68142873)
\curveto(146.87873266,36.73142204)(146.82373272,36.766422)(146.76373047,36.78642873)
\curveto(146.70373284,36.81642195)(146.6387329,36.84642192)(146.56873047,36.87642873)
\lineto(145.93873047,37.11642873)
\curveto(145.71873382,37.19642157)(145.50373404,37.27642149)(145.29373047,37.35642873)
\lineto(145.14373047,37.41642873)
\lineto(144.96373047,37.47642873)
\curveto(144.77373477,37.55642121)(144.58373496,37.62642114)(144.39373047,37.68642873)
\curveto(144.19373535,37.75642101)(143.99373555,37.83142094)(143.79373047,37.91142873)
\curveto(143.21373633,38.15142062)(142.62873691,38.3714204)(142.03873047,38.57142873)
\curveto(141.44873809,38.78141999)(140.86373868,39.00641976)(140.28373047,39.24642873)
\curveto(140.08373946,39.32641944)(139.87873966,39.40141937)(139.66873047,39.47142873)
\curveto(139.45874008,39.55141922)(139.25374029,39.63141914)(139.05373047,39.71142873)
\curveto(138.97374057,39.75141902)(138.87374067,39.78641898)(138.75373047,39.81642873)
\curveto(138.63374091,39.85641891)(138.54874099,39.91141886)(138.49873047,39.98142873)
\curveto(138.45874108,40.04141873)(138.42874111,40.11641865)(138.40873047,40.20642873)
\curveto(138.38874115,40.30641846)(138.37874116,40.41641835)(138.37873047,40.53642873)
\curveto(138.36874117,40.65641811)(138.36874117,40.77641799)(138.37873047,40.89642873)
\curveto(138.37874116,41.01641775)(138.37874116,41.12641764)(138.37873047,41.22642873)
\curveto(138.37874116,41.31641745)(138.37874116,41.40641736)(138.37873047,41.49642873)
\curveto(138.37874116,41.59641717)(138.39874114,41.6714171)(138.43873047,41.72142873)
\curveto(138.48874105,41.81141696)(138.57874096,41.86141691)(138.70873047,41.87142873)
\curveto(138.8387407,41.88141689)(138.97874056,41.88641688)(139.12873047,41.88642873)
\lineto(140.77873047,41.88642873)
\lineto(147.04873047,41.88642873)
\lineto(148.30873047,41.88642873)
\curveto(148.41873112,41.88641688)(148.52873101,41.88641688)(148.63873047,41.88642873)
\curveto(148.74873079,41.89641687)(148.83373071,41.87641689)(148.89373047,41.82642873)
\curveto(148.95373059,41.79641697)(148.99373055,41.75141702)(149.01373047,41.69142873)
\curveto(149.02373052,41.63141714)(149.0387305,41.56141721)(149.05873047,41.48142873)
\lineto(149.05873047,41.24142873)
\lineto(149.05873047,40.88142873)
\curveto(149.04873049,40.771418)(149.00373054,40.69141808)(148.92373047,40.64142873)
\curveto(148.89373065,40.62141815)(148.86373068,40.60641816)(148.83373047,40.59642873)
\curveto(148.79373075,40.59641817)(148.74873079,40.58641818)(148.69873047,40.56642873)
\lineto(148.53373047,40.56642873)
\curveto(148.47373107,40.55641821)(148.40373114,40.55141822)(148.32373047,40.55142873)
\curveto(148.2437313,40.56141821)(148.16873137,40.5664182)(148.09873047,40.56642873)
\lineto(147.25873047,40.56642873)
\lineto(142.83373047,40.56642873)
\curveto(142.58373696,40.5664182)(142.33373721,40.5664182)(142.08373047,40.56642873)
\curveto(141.82373772,40.5664182)(141.57373797,40.56141821)(141.33373047,40.55142873)
\curveto(141.23373831,40.55141822)(141.12373842,40.54641822)(141.00373047,40.53642873)
\curveto(140.88373866,40.52641824)(140.82373872,40.4714183)(140.82373047,40.37142873)
\lineto(140.83873047,40.37142873)
\curveto(140.85873868,40.30141847)(140.92373862,40.24141853)(141.03373047,40.19142873)
\curveto(141.1437384,40.15141862)(141.2387383,40.11641865)(141.31873047,40.08642873)
\curveto(141.48873805,40.01641875)(141.66373788,39.95141882)(141.84373047,39.89142873)
\curveto(142.01373753,39.83141894)(142.18373736,39.76141901)(142.35373047,39.68142873)
\curveto(142.40373714,39.66141911)(142.44873709,39.64641912)(142.48873047,39.63642873)
\curveto(142.52873701,39.62641914)(142.57373697,39.61141916)(142.62373047,39.59142873)
\curveto(142.80373674,39.51141926)(142.98873655,39.44141933)(143.17873047,39.38142873)
\curveto(143.35873618,39.33141944)(143.538736,39.2664195)(143.71873047,39.18642873)
\curveto(143.86873567,39.11641965)(144.02373552,39.05641971)(144.18373047,39.00642873)
\curveto(144.33373521,38.95641981)(144.48373506,38.90141987)(144.63373047,38.84142873)
\curveto(145.10373444,38.64142013)(145.57873396,38.46142031)(146.05873047,38.30142873)
\curveto(146.52873301,38.14142063)(146.99373255,37.9664208)(147.45373047,37.77642873)
\curveto(147.63373191,37.69642107)(147.81373173,37.62642114)(147.99373047,37.56642873)
\curveto(148.17373137,37.50642126)(148.35373119,37.44142133)(148.53373047,37.37142873)
\curveto(148.6437309,37.32142145)(148.74873079,37.2714215)(148.84873047,37.22142873)
\curveto(148.9387306,37.18142159)(149.00373054,37.09642167)(149.04373047,36.96642873)
\curveto(149.05373049,36.94642182)(149.05873048,36.92142185)(149.05873047,36.89142873)
\curveto(149.04873049,36.8714219)(149.04873049,36.84642192)(149.05873047,36.81642873)
\curveto(149.06873047,36.78642198)(149.07373047,36.75142202)(149.07373047,36.71142873)
\curveto(149.06373048,36.6714221)(149.05873048,36.63142214)(149.05873047,36.59142873)
\lineto(149.05873047,36.29142873)
\curveto(149.05873048,36.19142258)(149.03373051,36.11142266)(148.98373047,36.05142873)
\curveto(148.93373061,35.9714228)(148.86373068,35.91142286)(148.77373047,35.87142873)
\curveto(148.67373087,35.84142293)(148.57373097,35.80142297)(148.47373047,35.75142873)
\curveto(148.27373127,35.6714231)(148.06873147,35.59142318)(147.85873047,35.51142873)
\curveto(147.6387319,35.44142333)(147.42873211,35.3664234)(147.22873047,35.28642873)
\curveto(147.04873249,35.20642356)(146.86873267,35.13642363)(146.68873047,35.07642873)
\curveto(146.49873304,35.02642374)(146.31373323,34.96142381)(146.13373047,34.88142873)
\curveto(145.57373397,34.65142412)(145.00873453,34.43642433)(144.43873047,34.23642873)
\curveto(143.86873567,34.03642473)(143.30373624,33.82142495)(142.74373047,33.59142873)
\lineto(142.11373047,33.35142873)
\curveto(141.89373765,33.28142549)(141.68373786,33.20642556)(141.48373047,33.12642873)
\curveto(141.37373817,33.07642569)(141.26873827,33.03142574)(141.16873047,32.99142873)
\curveto(141.05873848,32.96142581)(140.96373858,32.91142586)(140.88373047,32.84142873)
\curveto(140.86373868,32.83142594)(140.85373869,32.82142595)(140.85373047,32.81142873)
\lineto(140.82373047,32.78142873)
\lineto(140.82373047,32.70642873)
\lineto(140.85373047,32.67642873)
\curveto(140.85373869,32.6664261)(140.85873868,32.65642611)(140.86873047,32.64642873)
\curveto(140.91873862,32.62642614)(140.97373857,32.61642615)(141.03373047,32.61642873)
\curveto(141.09373845,32.61642615)(141.15373839,32.60642616)(141.21373047,32.58642873)
\lineto(141.37873047,32.58642873)
\curveto(141.4387381,32.5664262)(141.50373804,32.56142621)(141.57373047,32.57142873)
\curveto(141.6437379,32.58142619)(141.71373783,32.58642618)(141.78373047,32.58642873)
\lineto(142.59373047,32.58642873)
\lineto(147.15373047,32.58642873)
\lineto(148.33873047,32.58642873)
\curveto(148.44873109,32.58642618)(148.55873098,32.58142619)(148.66873047,32.57142873)
\curveto(148.77873076,32.5714262)(148.86373068,32.54642622)(148.92373047,32.49642873)
\curveto(149.00373054,32.44642632)(149.04873049,32.35642641)(149.05873047,32.22642873)
\lineto(149.05873047,31.83642873)
\lineto(149.05873047,31.64142873)
\curveto(149.05873048,31.59142718)(149.04873049,31.54142723)(149.02873047,31.49142873)
\curveto(148.98873055,31.36142741)(148.90373064,31.28642748)(148.77373047,31.26642873)
\curveto(148.6437309,31.25642751)(148.49373105,31.25142752)(148.32373047,31.25142873)
\lineto(146.58373047,31.25142873)
\lineto(140.58373047,31.25142873)
\lineto(139.17373047,31.25142873)
\curveto(139.06374048,31.25142752)(138.94874059,31.24642752)(138.82873047,31.23642873)
\curveto(138.70874083,31.23642753)(138.61374093,31.26142751)(138.54373047,31.31142873)
\curveto(138.48374106,31.35142742)(138.43374111,31.42642734)(138.39373047,31.53642873)
\curveto(138.38374116,31.55642721)(138.38374116,31.57642719)(138.39373047,31.59642873)
\curveto(138.39374115,31.62642714)(138.38874115,31.65142712)(138.37873047,31.67142873)
}
}
{
\newrgbcolor{curcolor}{0 0 0}
\pscustom[linestyle=none,fillstyle=solid,fillcolor=curcolor]
{
\newpath
\moveto(148.50373047,50.87353811)
\curveto(148.66373088,50.90353028)(148.79873074,50.88853029)(148.90873047,50.82853811)
\curveto(149.00873053,50.76853041)(149.08373046,50.68853049)(149.13373047,50.58853811)
\curveto(149.15373039,50.53853064)(149.16373038,50.4835307)(149.16373047,50.42353811)
\curveto(149.16373038,50.37353081)(149.17373037,50.31853086)(149.19373047,50.25853811)
\curveto(149.2437303,50.03853114)(149.22873031,49.81853136)(149.14873047,49.59853811)
\curveto(149.07873046,49.38853179)(148.98873055,49.24353194)(148.87873047,49.16353811)
\curveto(148.80873073,49.11353207)(148.72873081,49.06853211)(148.63873047,49.02853811)
\curveto(148.538731,48.98853219)(148.45873108,48.93853224)(148.39873047,48.87853811)
\curveto(148.37873116,48.85853232)(148.35873118,48.83353235)(148.33873047,48.80353811)
\curveto(148.31873122,48.7835324)(148.31373123,48.75353243)(148.32373047,48.71353811)
\curveto(148.35373119,48.60353258)(148.40873113,48.49853268)(148.48873047,48.39853811)
\curveto(148.56873097,48.30853287)(148.6387309,48.21853296)(148.69873047,48.12853811)
\curveto(148.77873076,47.99853318)(148.85373069,47.85853332)(148.92373047,47.70853811)
\curveto(148.98373056,47.55853362)(149.0387305,47.39853378)(149.08873047,47.22853811)
\curveto(149.11873042,47.12853405)(149.1387304,47.01853416)(149.14873047,46.89853811)
\curveto(149.15873038,46.78853439)(149.17373037,46.6785345)(149.19373047,46.56853811)
\curveto(149.20373034,46.51853466)(149.20873033,46.47353471)(149.20873047,46.43353811)
\lineto(149.20873047,46.32853811)
\curveto(149.22873031,46.21853496)(149.22873031,46.11353507)(149.20873047,46.01353811)
\lineto(149.20873047,45.87853811)
\curveto(149.19873034,45.82853535)(149.19373035,45.7785354)(149.19373047,45.72853811)
\curveto(149.19373035,45.6785355)(149.18373036,45.63353555)(149.16373047,45.59353811)
\curveto(149.15373039,45.55353563)(149.14873039,45.51853566)(149.14873047,45.48853811)
\curveto(149.15873038,45.46853571)(149.15873038,45.44353574)(149.14873047,45.41353811)
\lineto(149.08873047,45.17353811)
\curveto(149.07873046,45.09353609)(149.05873048,45.01853616)(149.02873047,44.94853811)
\curveto(148.89873064,44.64853653)(148.75373079,44.40353678)(148.59373047,44.21353811)
\curveto(148.42373112,44.03353715)(148.18873135,43.8835373)(147.88873047,43.76353811)
\curveto(147.66873187,43.67353751)(147.40373214,43.62853755)(147.09373047,43.62853811)
\lineto(146.77873047,43.62853811)
\curveto(146.72873281,43.63853754)(146.67873286,43.64353754)(146.62873047,43.64353811)
\lineto(146.44873047,43.67353811)
\lineto(146.11873047,43.79353811)
\curveto(146.00873353,43.83353735)(145.90873363,43.8835373)(145.81873047,43.94353811)
\curveto(145.52873401,44.12353706)(145.31373423,44.36853681)(145.17373047,44.67853811)
\curveto(145.03373451,44.98853619)(144.90873463,45.32853585)(144.79873047,45.69853811)
\curveto(144.75873478,45.83853534)(144.72873481,45.9835352)(144.70873047,46.13353811)
\curveto(144.68873485,46.2835349)(144.66373488,46.43353475)(144.63373047,46.58353811)
\curveto(144.61373493,46.65353453)(144.60373494,46.71853446)(144.60373047,46.77853811)
\curveto(144.60373494,46.84853433)(144.59373495,46.92353426)(144.57373047,47.00353811)
\curveto(144.55373499,47.07353411)(144.543735,47.14353404)(144.54373047,47.21353811)
\curveto(144.53373501,47.2835339)(144.51873502,47.35853382)(144.49873047,47.43853811)
\curveto(144.4387351,47.68853349)(144.38873515,47.92353326)(144.34873047,48.14353811)
\curveto(144.29873524,48.36353282)(144.18373536,48.53853264)(144.00373047,48.66853811)
\curveto(143.92373562,48.72853245)(143.82373572,48.7785324)(143.70373047,48.81853811)
\curveto(143.57373597,48.85853232)(143.43373611,48.85853232)(143.28373047,48.81853811)
\curveto(143.0437365,48.75853242)(142.85373669,48.66853251)(142.71373047,48.54853811)
\curveto(142.57373697,48.43853274)(142.46373708,48.2785329)(142.38373047,48.06853811)
\curveto(142.33373721,47.94853323)(142.29873724,47.80353338)(142.27873047,47.63353811)
\curveto(142.25873728,47.47353371)(142.24873729,47.30353388)(142.24873047,47.12353811)
\curveto(142.24873729,46.94353424)(142.25873728,46.76853441)(142.27873047,46.59853811)
\curveto(142.29873724,46.42853475)(142.32873721,46.2835349)(142.36873047,46.16353811)
\curveto(142.42873711,45.99353519)(142.51373703,45.82853535)(142.62373047,45.66853811)
\curveto(142.68373686,45.58853559)(142.76373678,45.51353567)(142.86373047,45.44353811)
\curveto(142.95373659,45.3835358)(143.05373649,45.32853585)(143.16373047,45.27853811)
\curveto(143.2437363,45.24853593)(143.32873621,45.21853596)(143.41873047,45.18853811)
\curveto(143.50873603,45.16853601)(143.57873596,45.12353606)(143.62873047,45.05353811)
\curveto(143.65873588,45.01353617)(143.68373586,44.94353624)(143.70373047,44.84353811)
\curveto(143.71373583,44.75353643)(143.71873582,44.65853652)(143.71873047,44.55853811)
\curveto(143.71873582,44.45853672)(143.71373583,44.35853682)(143.70373047,44.25853811)
\curveto(143.68373586,44.16853701)(143.65873588,44.10353708)(143.62873047,44.06353811)
\curveto(143.59873594,44.02353716)(143.54873599,43.99353719)(143.47873047,43.97353811)
\curveto(143.40873613,43.95353723)(143.33373621,43.95353723)(143.25373047,43.97353811)
\curveto(143.12373642,44.00353718)(143.00373654,44.03353715)(142.89373047,44.06353811)
\curveto(142.77373677,44.10353708)(142.65873688,44.14853703)(142.54873047,44.19853811)
\curveto(142.19873734,44.38853679)(141.92873761,44.62853655)(141.73873047,44.91853811)
\curveto(141.538738,45.20853597)(141.37873816,45.56853561)(141.25873047,45.99853811)
\curveto(141.2387383,46.09853508)(141.22373832,46.19853498)(141.21373047,46.29853811)
\curveto(141.20373834,46.40853477)(141.18873835,46.51853466)(141.16873047,46.62853811)
\curveto(141.15873838,46.66853451)(141.15873838,46.73353445)(141.16873047,46.82353811)
\curveto(141.16873837,46.91353427)(141.15873838,46.96853421)(141.13873047,46.98853811)
\curveto(141.12873841,47.68853349)(141.20873833,48.29853288)(141.37873047,48.81853811)
\curveto(141.54873799,49.33853184)(141.87373767,49.70353148)(142.35373047,49.91353811)
\curveto(142.55373699,50.00353118)(142.78873675,50.05353113)(143.05873047,50.06353811)
\curveto(143.31873622,50.0835311)(143.59373595,50.09353109)(143.88373047,50.09353811)
\lineto(147.19873047,50.09353811)
\curveto(147.3387322,50.09353109)(147.47373207,50.09853108)(147.60373047,50.10853811)
\curveto(147.73373181,50.11853106)(147.8387317,50.14853103)(147.91873047,50.19853811)
\curveto(147.98873155,50.24853093)(148.0387315,50.31353087)(148.06873047,50.39353811)
\curveto(148.10873143,50.4835307)(148.1387314,50.56853061)(148.15873047,50.64853811)
\curveto(148.16873137,50.72853045)(148.21373133,50.78853039)(148.29373047,50.82853811)
\curveto(148.32373122,50.84853033)(148.35373119,50.85853032)(148.38373047,50.85853811)
\curveto(148.41373113,50.85853032)(148.45373109,50.86353032)(148.50373047,50.87353811)
\moveto(146.83873047,48.72853811)
\curveto(146.69873284,48.78853239)(146.538733,48.81853236)(146.35873047,48.81853811)
\curveto(146.16873337,48.82853235)(145.97373357,48.83353235)(145.77373047,48.83353811)
\curveto(145.66373388,48.83353235)(145.56373398,48.82853235)(145.47373047,48.81853811)
\curveto(145.38373416,48.80853237)(145.31373423,48.76853241)(145.26373047,48.69853811)
\curveto(145.2437343,48.66853251)(145.23373431,48.59853258)(145.23373047,48.48853811)
\curveto(145.25373429,48.46853271)(145.26373428,48.43353275)(145.26373047,48.38353811)
\curveto(145.26373428,48.33353285)(145.27373427,48.28853289)(145.29373047,48.24853811)
\curveto(145.31373423,48.16853301)(145.33373421,48.0785331)(145.35373047,47.97853811)
\lineto(145.41373047,47.67853811)
\curveto(145.41373413,47.64853353)(145.41873412,47.61353357)(145.42873047,47.57353811)
\lineto(145.42873047,47.46853811)
\curveto(145.46873407,47.31853386)(145.49373405,47.15353403)(145.50373047,46.97353811)
\curveto(145.50373404,46.80353438)(145.52373402,46.64353454)(145.56373047,46.49353811)
\curveto(145.58373396,46.41353477)(145.60373394,46.33853484)(145.62373047,46.26853811)
\curveto(145.63373391,46.20853497)(145.64873389,46.13853504)(145.66873047,46.05853811)
\curveto(145.71873382,45.89853528)(145.78373376,45.74853543)(145.86373047,45.60853811)
\curveto(145.93373361,45.46853571)(146.02373352,45.34853583)(146.13373047,45.24853811)
\curveto(146.2437333,45.14853603)(146.37873316,45.07353611)(146.53873047,45.02353811)
\curveto(146.68873285,44.97353621)(146.87373267,44.95353623)(147.09373047,44.96353811)
\curveto(147.19373235,44.96353622)(147.28873225,44.9785362)(147.37873047,45.00853811)
\curveto(147.45873208,45.04853613)(147.53373201,45.09353609)(147.60373047,45.14353811)
\curveto(147.71373183,45.22353596)(147.80873173,45.32853585)(147.88873047,45.45853811)
\curveto(147.95873158,45.58853559)(148.01873152,45.72853545)(148.06873047,45.87853811)
\curveto(148.07873146,45.92853525)(148.08373146,45.9785352)(148.08373047,46.02853811)
\curveto(148.08373146,46.0785351)(148.08873145,46.12853505)(148.09873047,46.17853811)
\curveto(148.11873142,46.24853493)(148.13373141,46.33353485)(148.14373047,46.43353811)
\curveto(148.1437314,46.54353464)(148.13373141,46.63353455)(148.11373047,46.70353811)
\curveto(148.09373145,46.76353442)(148.08873145,46.82353436)(148.09873047,46.88353811)
\curveto(148.09873144,46.94353424)(148.08873145,47.00353418)(148.06873047,47.06353811)
\curveto(148.04873149,47.14353404)(148.03373151,47.21853396)(148.02373047,47.28853811)
\curveto(148.01373153,47.36853381)(147.99373155,47.44353374)(147.96373047,47.51353811)
\curveto(147.8437317,47.80353338)(147.69873184,48.04853313)(147.52873047,48.24853811)
\curveto(147.35873218,48.45853272)(147.12873241,48.61853256)(146.83873047,48.72853811)
}
}
{
\newrgbcolor{curcolor}{0 0 0}
\pscustom[linestyle=none,fillstyle=solid,fillcolor=curcolor]
{
\newpath
\moveto(141.15373047,55.69017873)
\curveto(141.15373839,55.92017394)(141.21373833,56.05017381)(141.33373047,56.08017873)
\curveto(141.4437381,56.11017375)(141.60873793,56.12517374)(141.82873047,56.12517873)
\lineto(142.11373047,56.12517873)
\curveto(142.20373734,56.12517374)(142.27873726,56.10017376)(142.33873047,56.05017873)
\curveto(142.41873712,55.99017387)(142.46373708,55.90517396)(142.47373047,55.79517873)
\curveto(142.47373707,55.68517418)(142.48873705,55.57517429)(142.51873047,55.46517873)
\curveto(142.54873699,55.32517454)(142.57873696,55.19017467)(142.60873047,55.06017873)
\curveto(142.6387369,54.94017492)(142.67873686,54.82517504)(142.72873047,54.71517873)
\curveto(142.85873668,54.42517544)(143.0387365,54.19017567)(143.26873047,54.01017873)
\curveto(143.48873605,53.83017603)(143.7437358,53.67517619)(144.03373047,53.54517873)
\curveto(144.1437354,53.50517636)(144.25873528,53.47517639)(144.37873047,53.45517873)
\curveto(144.48873505,53.43517643)(144.60373494,53.41017645)(144.72373047,53.38017873)
\curveto(144.77373477,53.37017649)(144.82373472,53.3651765)(144.87373047,53.36517873)
\curveto(144.92373462,53.37517649)(144.97373457,53.37517649)(145.02373047,53.36517873)
\curveto(145.1437344,53.33517653)(145.28373426,53.32017654)(145.44373047,53.32017873)
\curveto(145.59373395,53.33017653)(145.7387338,53.33517653)(145.87873047,53.33517873)
\lineto(147.72373047,53.33517873)
\lineto(148.06873047,53.33517873)
\curveto(148.18873135,53.33517653)(148.30373124,53.33017653)(148.41373047,53.32017873)
\curveto(148.52373102,53.31017655)(148.61873092,53.30517656)(148.69873047,53.30517873)
\curveto(148.77873076,53.31517655)(148.84873069,53.29517657)(148.90873047,53.24517873)
\curveto(148.97873056,53.19517667)(149.01873052,53.11517675)(149.02873047,53.00517873)
\curveto(149.0387305,52.90517696)(149.0437305,52.79517707)(149.04373047,52.67517873)
\lineto(149.04373047,52.40517873)
\curveto(149.02373052,52.35517751)(149.00873053,52.30517756)(148.99873047,52.25517873)
\curveto(148.97873056,52.21517765)(148.95373059,52.18517768)(148.92373047,52.16517873)
\curveto(148.85373069,52.11517775)(148.76873077,52.08517778)(148.66873047,52.07517873)
\lineto(148.33873047,52.07517873)
\lineto(147.18373047,52.07517873)
\lineto(143.02873047,52.07517873)
\lineto(141.99373047,52.07517873)
\lineto(141.69373047,52.07517873)
\curveto(141.59373795,52.08517778)(141.50873803,52.11517775)(141.43873047,52.16517873)
\curveto(141.39873814,52.19517767)(141.36873817,52.24517762)(141.34873047,52.31517873)
\curveto(141.32873821,52.39517747)(141.31873822,52.48017738)(141.31873047,52.57017873)
\curveto(141.30873823,52.6601772)(141.30873823,52.75017711)(141.31873047,52.84017873)
\curveto(141.32873821,52.93017693)(141.3437382,53.00017686)(141.36373047,53.05017873)
\curveto(141.39373815,53.13017673)(141.45373809,53.18017668)(141.54373047,53.20017873)
\curveto(141.62373792,53.23017663)(141.71373783,53.24517662)(141.81373047,53.24517873)
\lineto(142.11373047,53.24517873)
\curveto(142.21373733,53.24517662)(142.30373724,53.2651766)(142.38373047,53.30517873)
\curveto(142.40373714,53.31517655)(142.41873712,53.32517654)(142.42873047,53.33517873)
\lineto(142.47373047,53.38017873)
\curveto(142.47373707,53.49017637)(142.42873711,53.58017628)(142.33873047,53.65017873)
\curveto(142.2387373,53.72017614)(142.15873738,53.78017608)(142.09873047,53.83017873)
\lineto(142.00873047,53.92017873)
\curveto(141.89873764,54.01017585)(141.78373776,54.13517573)(141.66373047,54.29517873)
\curveto(141.543738,54.45517541)(141.45373809,54.60517526)(141.39373047,54.74517873)
\curveto(141.3437382,54.83517503)(141.30873823,54.93017493)(141.28873047,55.03017873)
\curveto(141.25873828,55.13017473)(141.22873831,55.23517463)(141.19873047,55.34517873)
\curveto(141.18873835,55.40517446)(141.18373836,55.4651744)(141.18373047,55.52517873)
\curveto(141.17373837,55.58517428)(141.16373838,55.64017422)(141.15373047,55.69017873)
}
}
{
\newrgbcolor{curcolor}{0 0 0}
\pscustom[linestyle=none,fillstyle=solid,fillcolor=curcolor]
{
}
}
{
\newrgbcolor{curcolor}{0 0 0}
\pscustom[linestyle=none,fillstyle=solid,fillcolor=curcolor]
{
\newpath
\moveto(143.97373047,67.99510061)
\lineto(144.22873047,67.99510061)
\curveto(144.30873523,68.0050929)(144.38373516,68.00009291)(144.45373047,67.98010061)
\lineto(144.69373047,67.98010061)
\lineto(144.85873047,67.98010061)
\curveto(144.95873458,67.96009295)(145.06373448,67.95009296)(145.17373047,67.95010061)
\curveto(145.27373427,67.95009296)(145.37373417,67.94009297)(145.47373047,67.92010061)
\lineto(145.62373047,67.92010061)
\curveto(145.76373378,67.89009302)(145.90373364,67.87009304)(146.04373047,67.86010061)
\curveto(146.17373337,67.85009306)(146.30373324,67.82509308)(146.43373047,67.78510061)
\curveto(146.51373303,67.76509314)(146.59873294,67.74509316)(146.68873047,67.72510061)
\lineto(146.92873047,67.66510061)
\lineto(147.22873047,67.54510061)
\curveto(147.31873222,67.51509339)(147.40873213,67.48009343)(147.49873047,67.44010061)
\curveto(147.71873182,67.34009357)(147.93373161,67.2050937)(148.14373047,67.03510061)
\curveto(148.35373119,66.87509403)(148.52373102,66.70009421)(148.65373047,66.51010061)
\curveto(148.69373085,66.46009445)(148.73373081,66.40009451)(148.77373047,66.33010061)
\curveto(148.80373074,66.27009464)(148.8387307,66.2100947)(148.87873047,66.15010061)
\curveto(148.92873061,66.07009484)(148.96873057,65.97509493)(148.99873047,65.86510061)
\curveto(149.02873051,65.75509515)(149.05873048,65.65009526)(149.08873047,65.55010061)
\curveto(149.12873041,65.44009547)(149.15373039,65.33009558)(149.16373047,65.22010061)
\curveto(149.17373037,65.1100958)(149.18873035,64.99509591)(149.20873047,64.87510061)
\curveto(149.21873032,64.83509607)(149.21873032,64.79009612)(149.20873047,64.74010061)
\curveto(149.20873033,64.70009621)(149.21373033,64.66009625)(149.22373047,64.62010061)
\curveto(149.23373031,64.58009633)(149.2387303,64.52509638)(149.23873047,64.45510061)
\curveto(149.2387303,64.38509652)(149.23373031,64.33509657)(149.22373047,64.30510061)
\curveto(149.20373034,64.25509665)(149.19873034,64.2100967)(149.20873047,64.17010061)
\curveto(149.21873032,64.13009678)(149.21873032,64.09509681)(149.20873047,64.06510061)
\lineto(149.20873047,63.97510061)
\curveto(149.18873035,63.91509699)(149.17373037,63.85009706)(149.16373047,63.78010061)
\curveto(149.16373038,63.72009719)(149.15873038,63.65509725)(149.14873047,63.58510061)
\curveto(149.09873044,63.41509749)(149.04873049,63.25509765)(148.99873047,63.10510061)
\curveto(148.94873059,62.95509795)(148.88373066,62.8100981)(148.80373047,62.67010061)
\curveto(148.76373078,62.62009829)(148.73373081,62.56509834)(148.71373047,62.50510061)
\curveto(148.68373086,62.45509845)(148.64873089,62.4050985)(148.60873047,62.35510061)
\curveto(148.42873111,62.11509879)(148.20873133,61.91509899)(147.94873047,61.75510061)
\curveto(147.68873185,61.59509931)(147.40373214,61.45509945)(147.09373047,61.33510061)
\curveto(146.95373259,61.27509963)(146.81373273,61.23009968)(146.67373047,61.20010061)
\curveto(146.52373302,61.17009974)(146.36873317,61.13509977)(146.20873047,61.09510061)
\curveto(146.09873344,61.07509983)(145.98873355,61.06009985)(145.87873047,61.05010061)
\curveto(145.76873377,61.04009987)(145.65873388,61.02509988)(145.54873047,61.00510061)
\curveto(145.50873403,60.99509991)(145.46873407,60.99009992)(145.42873047,60.99010061)
\curveto(145.38873415,61.00009991)(145.34873419,61.00009991)(145.30873047,60.99010061)
\curveto(145.25873428,60.98009993)(145.20873433,60.97509993)(145.15873047,60.97510061)
\lineto(144.99373047,60.97510061)
\curveto(144.9437346,60.95509995)(144.89373465,60.95009996)(144.84373047,60.96010061)
\curveto(144.78373476,60.97009994)(144.72873481,60.97009994)(144.67873047,60.96010061)
\curveto(144.6387349,60.95009996)(144.59373495,60.95009996)(144.54373047,60.96010061)
\curveto(144.49373505,60.97009994)(144.4437351,60.96509994)(144.39373047,60.94510061)
\curveto(144.32373522,60.92509998)(144.24873529,60.92009999)(144.16873047,60.93010061)
\curveto(144.07873546,60.94009997)(143.99373555,60.94509996)(143.91373047,60.94510061)
\curveto(143.82373572,60.94509996)(143.72373582,60.94009997)(143.61373047,60.93010061)
\curveto(143.49373605,60.92009999)(143.39373615,60.92509998)(143.31373047,60.94510061)
\lineto(143.02873047,60.94510061)
\lineto(142.39873047,60.99010061)
\curveto(142.29873724,61.00009991)(142.20373734,61.0100999)(142.11373047,61.02010061)
\lineto(141.81373047,61.05010061)
\curveto(141.76373778,61.07009984)(141.71373783,61.07509983)(141.66373047,61.06510061)
\curveto(141.60373794,61.06509984)(141.54873799,61.07509983)(141.49873047,61.09510061)
\curveto(141.32873821,61.14509976)(141.16373838,61.18509972)(141.00373047,61.21510061)
\curveto(140.83373871,61.24509966)(140.67373887,61.29509961)(140.52373047,61.36510061)
\curveto(140.06373948,61.55509935)(139.68873985,61.77509913)(139.39873047,62.02510061)
\curveto(139.10874043,62.28509862)(138.86374068,62.64509826)(138.66373047,63.10510061)
\curveto(138.61374093,63.23509767)(138.57874096,63.36509754)(138.55873047,63.49510061)
\curveto(138.538741,63.63509727)(138.51374103,63.77509713)(138.48373047,63.91510061)
\curveto(138.47374107,63.98509692)(138.46874107,64.05009686)(138.46873047,64.11010061)
\curveto(138.46874107,64.17009674)(138.46374108,64.23509667)(138.45373047,64.30510061)
\curveto(138.43374111,65.13509577)(138.58374096,65.8050951)(138.90373047,66.31510061)
\curveto(139.21374033,66.82509408)(139.65373989,67.2050937)(140.22373047,67.45510061)
\curveto(140.3437392,67.5050934)(140.46873907,67.55009336)(140.59873047,67.59010061)
\curveto(140.72873881,67.63009328)(140.86373868,67.67509323)(141.00373047,67.72510061)
\curveto(141.08373846,67.74509316)(141.16873837,67.76009315)(141.25873047,67.77010061)
\lineto(141.49873047,67.83010061)
\curveto(141.60873793,67.86009305)(141.71873782,67.87509303)(141.82873047,67.87510061)
\curveto(141.9387376,67.88509302)(142.04873749,67.90009301)(142.15873047,67.92010061)
\curveto(142.20873733,67.94009297)(142.25373729,67.94509296)(142.29373047,67.93510061)
\curveto(142.33373721,67.93509297)(142.37373717,67.94009297)(142.41373047,67.95010061)
\curveto(142.46373708,67.96009295)(142.51873702,67.96009295)(142.57873047,67.95010061)
\curveto(142.62873691,67.95009296)(142.67873686,67.95509295)(142.72873047,67.96510061)
\lineto(142.86373047,67.96510061)
\curveto(142.92373662,67.98509292)(142.99373655,67.98509292)(143.07373047,67.96510061)
\curveto(143.1437364,67.95509295)(143.20873633,67.96009295)(143.26873047,67.98010061)
\curveto(143.29873624,67.99009292)(143.3387362,67.99509291)(143.38873047,67.99510061)
\lineto(143.50873047,67.99510061)
\lineto(143.97373047,67.99510061)
\moveto(146.29873047,66.45010061)
\curveto(145.97873356,66.55009436)(145.61373393,66.6100943)(145.20373047,66.63010061)
\curveto(144.79373475,66.65009426)(144.38373516,66.66009425)(143.97373047,66.66010061)
\curveto(143.543736,66.66009425)(143.12373642,66.65009426)(142.71373047,66.63010061)
\curveto(142.30373724,66.6100943)(141.91873762,66.56509434)(141.55873047,66.49510061)
\curveto(141.19873834,66.42509448)(140.87873866,66.31509459)(140.59873047,66.16510061)
\curveto(140.30873923,66.02509488)(140.07373947,65.83009508)(139.89373047,65.58010061)
\curveto(139.78373976,65.42009549)(139.70373984,65.24009567)(139.65373047,65.04010061)
\curveto(139.59373995,64.84009607)(139.56373998,64.59509631)(139.56373047,64.30510061)
\curveto(139.58373996,64.28509662)(139.59373995,64.25009666)(139.59373047,64.20010061)
\curveto(139.58373996,64.15009676)(139.58373996,64.1100968)(139.59373047,64.08010061)
\curveto(139.61373993,64.00009691)(139.63373991,63.92509698)(139.65373047,63.85510061)
\curveto(139.66373988,63.79509711)(139.68373986,63.73009718)(139.71373047,63.66010061)
\curveto(139.83373971,63.39009752)(140.00373954,63.17009774)(140.22373047,63.00010061)
\curveto(140.43373911,62.84009807)(140.67873886,62.7050982)(140.95873047,62.59510061)
\curveto(141.06873847,62.54509836)(141.18873835,62.5050984)(141.31873047,62.47510061)
\curveto(141.4387381,62.45509845)(141.56373798,62.43009848)(141.69373047,62.40010061)
\curveto(141.7437378,62.38009853)(141.79873774,62.37009854)(141.85873047,62.37010061)
\curveto(141.90873763,62.37009854)(141.95873758,62.36509854)(142.00873047,62.35510061)
\curveto(142.09873744,62.34509856)(142.19373735,62.33509857)(142.29373047,62.32510061)
\curveto(142.38373716,62.31509859)(142.47873706,62.3050986)(142.57873047,62.29510061)
\curveto(142.65873688,62.29509861)(142.7437368,62.29009862)(142.83373047,62.28010061)
\lineto(143.07373047,62.28010061)
\lineto(143.25373047,62.28010061)
\curveto(143.28373626,62.27009864)(143.31873622,62.26509864)(143.35873047,62.26510061)
\lineto(143.49373047,62.26510061)
\lineto(143.94373047,62.26510061)
\curveto(144.02373552,62.26509864)(144.10873543,62.26009865)(144.19873047,62.25010061)
\curveto(144.27873526,62.25009866)(144.35373519,62.26009865)(144.42373047,62.28010061)
\lineto(144.69373047,62.28010061)
\curveto(144.71373483,62.28009863)(144.7437348,62.27509863)(144.78373047,62.26510061)
\curveto(144.81373473,62.26509864)(144.8387347,62.27009864)(144.85873047,62.28010061)
\curveto(144.95873458,62.29009862)(145.05873448,62.29509861)(145.15873047,62.29510061)
\curveto(145.24873429,62.3050986)(145.34873419,62.31509859)(145.45873047,62.32510061)
\curveto(145.57873396,62.35509855)(145.70373384,62.37009854)(145.83373047,62.37010061)
\curveto(145.95373359,62.38009853)(146.06873347,62.4050985)(146.17873047,62.44510061)
\curveto(146.47873306,62.52509838)(146.7437328,62.6100983)(146.97373047,62.70010061)
\curveto(147.20373234,62.80009811)(147.41873212,62.94509796)(147.61873047,63.13510061)
\curveto(147.81873172,63.34509756)(147.96873157,63.6100973)(148.06873047,63.93010061)
\curveto(148.08873145,63.97009694)(148.09873144,64.0050969)(148.09873047,64.03510061)
\curveto(148.08873145,64.07509683)(148.09373145,64.12009679)(148.11373047,64.17010061)
\curveto(148.12373142,64.2100967)(148.13373141,64.28009663)(148.14373047,64.38010061)
\curveto(148.15373139,64.49009642)(148.14873139,64.57509633)(148.12873047,64.63510061)
\curveto(148.10873143,64.7050962)(148.09873144,64.77509613)(148.09873047,64.84510061)
\curveto(148.08873145,64.91509599)(148.07373147,64.98009593)(148.05373047,65.04010061)
\curveto(147.99373155,65.24009567)(147.90873163,65.42009549)(147.79873047,65.58010061)
\curveto(147.77873176,65.6100953)(147.75873178,65.63509527)(147.73873047,65.65510061)
\lineto(147.67873047,65.71510061)
\curveto(147.65873188,65.75509515)(147.61873192,65.8050951)(147.55873047,65.86510061)
\curveto(147.41873212,65.96509494)(147.28873225,66.05009486)(147.16873047,66.12010061)
\curveto(147.04873249,66.19009472)(146.90373264,66.26009465)(146.73373047,66.33010061)
\curveto(146.66373288,66.36009455)(146.59373295,66.38009453)(146.52373047,66.39010061)
\curveto(146.45373309,66.4100945)(146.37873316,66.43009448)(146.29873047,66.45010061)
}
}
{
\newrgbcolor{curcolor}{0 0 0}
\pscustom[linestyle=none,fillstyle=solid,fillcolor=curcolor]
{
\newpath
\moveto(144.73873047,76.34470998)
\curveto(144.85873468,76.37470226)(144.99873454,76.39970223)(145.15873047,76.41970998)
\curveto(145.31873422,76.43970219)(145.48373406,76.44970218)(145.65373047,76.44970998)
\curveto(145.82373372,76.44970218)(145.98873355,76.43970219)(146.14873047,76.41970998)
\curveto(146.30873323,76.39970223)(146.44873309,76.37470226)(146.56873047,76.34470998)
\curveto(146.70873283,76.30470233)(146.83373271,76.26970236)(146.94373047,76.23970998)
\curveto(147.05373249,76.20970242)(147.16373238,76.16970246)(147.27373047,76.11970998)
\curveto(147.91373163,75.84970278)(148.39873114,75.4347032)(148.72873047,74.87470998)
\curveto(148.78873075,74.79470384)(148.8387307,74.70970392)(148.87873047,74.61970998)
\curveto(148.90873063,74.5297041)(148.9437306,74.4297042)(148.98373047,74.31970998)
\curveto(149.03373051,74.20970442)(149.06873047,74.08970454)(149.08873047,73.95970998)
\curveto(149.11873042,73.83970479)(149.14873039,73.70970492)(149.17873047,73.56970998)
\curveto(149.19873034,73.50970512)(149.20373034,73.44970518)(149.19373047,73.38970998)
\curveto(149.18373036,73.33970529)(149.18873035,73.27970535)(149.20873047,73.20970998)
\curveto(149.21873032,73.18970544)(149.21873032,73.16470547)(149.20873047,73.13470998)
\curveto(149.20873033,73.10470553)(149.21373033,73.07970555)(149.22373047,73.05970998)
\lineto(149.22373047,72.90970998)
\curveto(149.23373031,72.83970579)(149.23373031,72.78970584)(149.22373047,72.75970998)
\curveto(149.21373033,72.71970591)(149.20873033,72.67470596)(149.20873047,72.62470998)
\curveto(149.21873032,72.58470605)(149.21873032,72.54470609)(149.20873047,72.50470998)
\curveto(149.18873035,72.41470622)(149.17373037,72.32470631)(149.16373047,72.23470998)
\curveto(149.16373038,72.14470649)(149.15373039,72.05470658)(149.13373047,71.96470998)
\curveto(149.10373044,71.87470676)(149.07873046,71.78470685)(149.05873047,71.69470998)
\curveto(149.0387305,71.60470703)(149.00873053,71.51970711)(148.96873047,71.43970998)
\curveto(148.85873068,71.19970743)(148.72873081,70.97470766)(148.57873047,70.76470998)
\curveto(148.41873112,70.55470808)(148.2387313,70.37470826)(148.03873047,70.22470998)
\curveto(147.86873167,70.10470853)(147.69373185,69.99970863)(147.51373047,69.90970998)
\curveto(147.33373221,69.81970881)(147.1437324,69.7297089)(146.94373047,69.63970998)
\curveto(146.8437327,69.59970903)(146.7437328,69.56470907)(146.64373047,69.53470998)
\curveto(146.53373301,69.51470912)(146.42373312,69.48970914)(146.31373047,69.45970998)
\curveto(146.17373337,69.41970921)(146.03373351,69.39470924)(145.89373047,69.38470998)
\curveto(145.75373379,69.37470926)(145.61373393,69.35470928)(145.47373047,69.32470998)
\curveto(145.36373418,69.31470932)(145.26373428,69.30470933)(145.17373047,69.29470998)
\curveto(145.07373447,69.29470934)(144.97373457,69.28470935)(144.87373047,69.26470998)
\lineto(144.78373047,69.26470998)
\curveto(144.75373479,69.27470936)(144.72873481,69.27470936)(144.70873047,69.26470998)
\lineto(144.49873047,69.26470998)
\curveto(144.4387351,69.24470939)(144.37373517,69.2347094)(144.30373047,69.23470998)
\curveto(144.22373532,69.24470939)(144.14873539,69.24970938)(144.07873047,69.24970998)
\lineto(143.92873047,69.24970998)
\curveto(143.87873566,69.24970938)(143.82873571,69.25470938)(143.77873047,69.26470998)
\lineto(143.40373047,69.26470998)
\curveto(143.37373617,69.27470936)(143.3387362,69.27470936)(143.29873047,69.26470998)
\curveto(143.25873628,69.26470937)(143.21873632,69.26970936)(143.17873047,69.27970998)
\curveto(143.06873647,69.29970933)(142.95873658,69.31470932)(142.84873047,69.32470998)
\curveto(142.72873681,69.3347093)(142.61373693,69.34470929)(142.50373047,69.35470998)
\curveto(142.35373719,69.39470924)(142.20873733,69.41970921)(142.06873047,69.42970998)
\curveto(141.91873762,69.44970918)(141.77373777,69.47970915)(141.63373047,69.51970998)
\curveto(141.33373821,69.60970902)(141.04873849,69.70470893)(140.77873047,69.80470998)
\curveto(140.50873903,69.90470873)(140.25873928,70.0297086)(140.02873047,70.17970998)
\curveto(139.70873983,70.37970825)(139.42874011,70.62470801)(139.18873047,70.91470998)
\curveto(138.94874059,71.20470743)(138.76374078,71.54470709)(138.63373047,71.93470998)
\curveto(138.59374095,72.04470659)(138.56874097,72.15470648)(138.55873047,72.26470998)
\curveto(138.538741,72.38470625)(138.51374103,72.50470613)(138.48373047,72.62470998)
\curveto(138.47374107,72.69470594)(138.46874107,72.75970587)(138.46873047,72.81970998)
\curveto(138.46874107,72.87970575)(138.46374108,72.94470569)(138.45373047,73.01470998)
\curveto(138.43374111,73.71470492)(138.54874099,74.28970434)(138.79873047,74.73970998)
\curveto(139.04874049,75.18970344)(139.39874014,75.5347031)(139.84873047,75.77470998)
\curveto(140.07873946,75.88470275)(140.35373919,75.98470265)(140.67373047,76.07470998)
\curveto(140.7437388,76.09470254)(140.81873872,76.09470254)(140.89873047,76.07470998)
\curveto(140.96873857,76.06470257)(141.01873852,76.03970259)(141.04873047,75.99970998)
\curveto(141.07873846,75.96970266)(141.10373844,75.90970272)(141.12373047,75.81970998)
\curveto(141.13373841,75.7297029)(141.1437384,75.629703)(141.15373047,75.51970998)
\curveto(141.15373839,75.41970321)(141.14873839,75.31970331)(141.13873047,75.21970998)
\curveto(141.12873841,75.1297035)(141.10873843,75.06470357)(141.07873047,75.02470998)
\curveto(141.00873853,74.91470372)(140.89873864,74.8347038)(140.74873047,74.78470998)
\curveto(140.59873894,74.74470389)(140.46873907,74.68970394)(140.35873047,74.61970998)
\curveto(140.04873949,74.4297042)(139.81873972,74.14970448)(139.66873047,73.77970998)
\curveto(139.6387399,73.70970492)(139.61873992,73.634705)(139.60873047,73.55470998)
\curveto(139.59873994,73.48470515)(139.58373996,73.40970522)(139.56373047,73.32970998)
\curveto(139.55373999,73.27970535)(139.54873999,73.20970542)(139.54873047,73.11970998)
\curveto(139.54873999,73.03970559)(139.55373999,72.97470566)(139.56373047,72.92470998)
\curveto(139.58373996,72.88470575)(139.58873995,72.84970578)(139.57873047,72.81970998)
\curveto(139.56873997,72.78970584)(139.56873997,72.75470588)(139.57873047,72.71470998)
\lineto(139.63873047,72.47470998)
\curveto(139.65873988,72.40470623)(139.68373986,72.3347063)(139.71373047,72.26470998)
\curveto(139.87373967,71.88470675)(140.08373946,71.59470704)(140.34373047,71.39470998)
\curveto(140.60373894,71.20470743)(140.91873862,71.0297076)(141.28873047,70.86970998)
\curveto(141.36873817,70.83970779)(141.44873809,70.81470782)(141.52873047,70.79470998)
\curveto(141.60873793,70.78470785)(141.68873785,70.76470787)(141.76873047,70.73470998)
\curveto(141.87873766,70.70470793)(141.99373755,70.67970795)(142.11373047,70.65970998)
\curveto(142.23373731,70.64970798)(142.35373719,70.629708)(142.47373047,70.59970998)
\curveto(142.52373702,70.57970805)(142.57373697,70.56970806)(142.62373047,70.56970998)
\curveto(142.67373687,70.57970805)(142.72373682,70.57470806)(142.77373047,70.55470998)
\curveto(142.83373671,70.54470809)(142.91373663,70.54470809)(143.01373047,70.55470998)
\curveto(143.10373644,70.56470807)(143.15873638,70.57970805)(143.17873047,70.59970998)
\curveto(143.21873632,70.61970801)(143.2387363,70.64970798)(143.23873047,70.68970998)
\curveto(143.2387363,70.73970789)(143.22873631,70.78470785)(143.20873047,70.82470998)
\curveto(143.16873637,70.89470774)(143.12373642,70.95470768)(143.07373047,71.00470998)
\curveto(143.02373652,71.05470758)(142.97373657,71.11470752)(142.92373047,71.18470998)
\lineto(142.86373047,71.24470998)
\curveto(142.83373671,71.27470736)(142.80873673,71.30470733)(142.78873047,71.33470998)
\curveto(142.62873691,71.56470707)(142.49373705,71.83970679)(142.38373047,72.15970998)
\curveto(142.36373718,72.2297064)(142.34873719,72.29970633)(142.33873047,72.36970998)
\curveto(142.32873721,72.43970619)(142.31373723,72.51470612)(142.29373047,72.59470998)
\curveto(142.29373725,72.634706)(142.28873725,72.66970596)(142.27873047,72.69970998)
\curveto(142.26873727,72.7297059)(142.26873727,72.76470587)(142.27873047,72.80470998)
\curveto(142.27873726,72.85470578)(142.26873727,72.89470574)(142.24873047,72.92470998)
\lineto(142.24873047,73.08970998)
\lineto(142.24873047,73.17970998)
\curveto(142.2387373,73.2297054)(142.2387373,73.26970536)(142.24873047,73.29970998)
\curveto(142.25873728,73.34970528)(142.26373728,73.39970523)(142.26373047,73.44970998)
\curveto(142.25373729,73.50970512)(142.25373729,73.56470507)(142.26373047,73.61470998)
\curveto(142.29373725,73.72470491)(142.31373723,73.8297048)(142.32373047,73.92970998)
\curveto(142.33373721,74.03970459)(142.35873718,74.14470449)(142.39873047,74.24470998)
\curveto(142.538737,74.66470397)(142.72373682,75.00970362)(142.95373047,75.27970998)
\curveto(143.17373637,75.54970308)(143.45873608,75.78970284)(143.80873047,75.99970998)
\curveto(143.94873559,76.07970255)(144.09873544,76.14470249)(144.25873047,76.19470998)
\curveto(144.40873513,76.24470239)(144.56873497,76.29470234)(144.73873047,76.34470998)
\moveto(146.04373047,75.09970998)
\curveto(145.99373355,75.10970352)(145.94873359,75.11470352)(145.90873047,75.11470998)
\lineto(145.75873047,75.11470998)
\curveto(145.44873409,75.11470352)(145.16373438,75.07470356)(144.90373047,74.99470998)
\curveto(144.8437347,74.97470366)(144.78873475,74.95470368)(144.73873047,74.93470998)
\curveto(144.67873486,74.92470371)(144.62373492,74.90970372)(144.57373047,74.88970998)
\curveto(144.08373546,74.66970396)(143.73373581,74.32470431)(143.52373047,73.85470998)
\curveto(143.49373605,73.77470486)(143.46873607,73.69470494)(143.44873047,73.61470998)
\lineto(143.38873047,73.37470998)
\curveto(143.36873617,73.29470534)(143.35873618,73.20470543)(143.35873047,73.10470998)
\lineto(143.35873047,72.78970998)
\curveto(143.37873616,72.76970586)(143.38873615,72.7297059)(143.38873047,72.66970998)
\curveto(143.37873616,72.61970601)(143.37873616,72.57470606)(143.38873047,72.53470998)
\lineto(143.44873047,72.29470998)
\curveto(143.45873608,72.22470641)(143.47873606,72.15470648)(143.50873047,72.08470998)
\curveto(143.76873577,71.48470715)(144.23373531,71.07970755)(144.90373047,70.86970998)
\curveto(144.98373456,70.83970779)(145.06373448,70.81970781)(145.14373047,70.80970998)
\curveto(145.22373432,70.79970783)(145.30873423,70.78470785)(145.39873047,70.76470998)
\lineto(145.54873047,70.76470998)
\curveto(145.58873395,70.75470788)(145.65873388,70.74970788)(145.75873047,70.74970998)
\curveto(145.98873355,70.74970788)(146.18373336,70.76970786)(146.34373047,70.80970998)
\curveto(146.41373313,70.8297078)(146.47873306,70.84470779)(146.53873047,70.85470998)
\curveto(146.59873294,70.86470777)(146.66373288,70.88470775)(146.73373047,70.91470998)
\curveto(147.01373253,71.02470761)(147.25873228,71.16970746)(147.46873047,71.34970998)
\curveto(147.66873187,71.5297071)(147.82873171,71.76470687)(147.94873047,72.05470998)
\lineto(148.03873047,72.29470998)
\lineto(148.09873047,72.53470998)
\curveto(148.11873142,72.58470605)(148.12373142,72.62470601)(148.11373047,72.65470998)
\curveto(148.10373144,72.69470594)(148.10873143,72.73970589)(148.12873047,72.78970998)
\curveto(148.1387314,72.81970581)(148.1437314,72.87470576)(148.14373047,72.95470998)
\curveto(148.1437314,73.0347056)(148.1387314,73.09470554)(148.12873047,73.13470998)
\curveto(148.10873143,73.24470539)(148.09373145,73.34970528)(148.08373047,73.44970998)
\curveto(148.07373147,73.54970508)(148.0437315,73.64470499)(147.99373047,73.73470998)
\curveto(147.79373175,74.26470437)(147.41873212,74.65470398)(146.86873047,74.90470998)
\curveto(146.76873277,74.94470369)(146.66373288,74.97470366)(146.55373047,74.99470998)
\lineto(146.22373047,75.08470998)
\curveto(146.1437334,75.08470355)(146.08373346,75.08970354)(146.04373047,75.09970998)
}
}
{
\newrgbcolor{curcolor}{0 0 0}
\pscustom[linestyle=none,fillstyle=solid,fillcolor=curcolor]
{
\newpath
\moveto(147.42373047,78.63431936)
\lineto(147.42373047,79.26431936)
\lineto(147.42373047,79.45931936)
\curveto(147.42373212,79.52931683)(147.43373211,79.58931677)(147.45373047,79.63931936)
\curveto(147.49373205,79.70931665)(147.53373201,79.7593166)(147.57373047,79.78931936)
\curveto(147.62373192,79.82931653)(147.68873185,79.84931651)(147.76873047,79.84931936)
\curveto(147.84873169,79.8593165)(147.93373161,79.86431649)(148.02373047,79.86431936)
\lineto(148.74373047,79.86431936)
\curveto(149.22373032,79.86431649)(149.63372991,79.80431655)(149.97373047,79.68431936)
\curveto(150.31372923,79.56431679)(150.58872895,79.36931699)(150.79873047,79.09931936)
\curveto(150.84872869,79.02931733)(150.89372865,78.9593174)(150.93373047,78.88931936)
\curveto(150.98372856,78.82931753)(151.02872851,78.7543176)(151.06873047,78.66431936)
\curveto(151.07872846,78.64431771)(151.08872845,78.61431774)(151.09873047,78.57431936)
\curveto(151.11872842,78.53431782)(151.12372842,78.48931787)(151.11373047,78.43931936)
\curveto(151.08372846,78.34931801)(151.00872853,78.29431806)(150.88873047,78.27431936)
\curveto(150.77872876,78.2543181)(150.68372886,78.26931809)(150.60373047,78.31931936)
\curveto(150.53372901,78.34931801)(150.46872907,78.39431796)(150.40873047,78.45431936)
\curveto(150.35872918,78.52431783)(150.30872923,78.58931777)(150.25873047,78.64931936)
\curveto(150.20872933,78.71931764)(150.13372941,78.77931758)(150.03373047,78.82931936)
\curveto(149.9437296,78.88931747)(149.85372969,78.93931742)(149.76373047,78.97931936)
\curveto(149.73372981,78.99931736)(149.67372987,79.02431733)(149.58373047,79.05431936)
\curveto(149.50373004,79.08431727)(149.43373011,79.08931727)(149.37373047,79.06931936)
\curveto(149.23373031,79.03931732)(149.1437304,78.97931738)(149.10373047,78.88931936)
\curveto(149.07373047,78.80931755)(149.05873048,78.71931764)(149.05873047,78.61931936)
\curveto(149.05873048,78.51931784)(149.03373051,78.43431792)(148.98373047,78.36431936)
\curveto(148.91373063,78.27431808)(148.77373077,78.22931813)(148.56373047,78.22931936)
\lineto(148.00873047,78.22931936)
\lineto(147.78373047,78.22931936)
\curveto(147.70373184,78.23931812)(147.6387319,78.2593181)(147.58873047,78.28931936)
\curveto(147.50873203,78.34931801)(147.46373208,78.41931794)(147.45373047,78.49931936)
\curveto(147.4437321,78.51931784)(147.4387321,78.53931782)(147.43873047,78.55931936)
\curveto(147.4387321,78.58931777)(147.43373211,78.61431774)(147.42373047,78.63431936)
}
}
{
\newrgbcolor{curcolor}{0 0 0}
\pscustom[linestyle=none,fillstyle=solid,fillcolor=curcolor]
{
}
}
{
\newrgbcolor{curcolor}{0 0 0}
\pscustom[linestyle=none,fillstyle=solid,fillcolor=curcolor]
{
\newpath
\moveto(138.45373047,89.26463186)
\curveto(138.4437411,89.95462722)(138.56374098,90.55462662)(138.81373047,91.06463186)
\curveto(139.06374048,91.58462559)(139.39874014,91.9796252)(139.81873047,92.24963186)
\curveto(139.89873964,92.29962488)(139.98873955,92.34462483)(140.08873047,92.38463186)
\curveto(140.17873936,92.42462475)(140.27373927,92.46962471)(140.37373047,92.51963186)
\curveto(140.47373907,92.55962462)(140.57373897,92.58962459)(140.67373047,92.60963186)
\curveto(140.77373877,92.62962455)(140.87873866,92.64962453)(140.98873047,92.66963186)
\curveto(141.0387385,92.68962449)(141.08373846,92.69462448)(141.12373047,92.68463186)
\curveto(141.16373838,92.6746245)(141.20873833,92.6796245)(141.25873047,92.69963186)
\curveto(141.30873823,92.70962447)(141.39373815,92.71462446)(141.51373047,92.71463186)
\curveto(141.62373792,92.71462446)(141.70873783,92.70962447)(141.76873047,92.69963186)
\curveto(141.82873771,92.6796245)(141.88873765,92.66962451)(141.94873047,92.66963186)
\curveto(142.00873753,92.6796245)(142.06873747,92.6746245)(142.12873047,92.65463186)
\curveto(142.26873727,92.61462456)(142.40373714,92.5796246)(142.53373047,92.54963186)
\curveto(142.66373688,92.51962466)(142.78873675,92.4796247)(142.90873047,92.42963186)
\curveto(143.04873649,92.36962481)(143.17373637,92.29962488)(143.28373047,92.21963186)
\curveto(143.39373615,92.14962503)(143.50373604,92.0746251)(143.61373047,91.99463186)
\lineto(143.67373047,91.93463186)
\curveto(143.69373585,91.92462525)(143.71373583,91.90962527)(143.73373047,91.88963186)
\curveto(143.89373565,91.76962541)(144.0387355,91.63462554)(144.16873047,91.48463186)
\curveto(144.29873524,91.33462584)(144.42373512,91.174626)(144.54373047,91.00463186)
\curveto(144.76373478,90.69462648)(144.96873457,90.39962678)(145.15873047,90.11963186)
\curveto(145.29873424,89.88962729)(145.43373411,89.65962752)(145.56373047,89.42963186)
\curveto(145.69373385,89.20962797)(145.82873371,88.98962819)(145.96873047,88.76963186)
\curveto(146.1387334,88.51962866)(146.31873322,88.2796289)(146.50873047,88.04963186)
\curveto(146.69873284,87.82962935)(146.92373262,87.63962954)(147.18373047,87.47963186)
\curveto(147.2437323,87.43962974)(147.30373224,87.40462977)(147.36373047,87.37463186)
\curveto(147.41373213,87.34462983)(147.47873206,87.31462986)(147.55873047,87.28463186)
\curveto(147.62873191,87.26462991)(147.68873185,87.25962992)(147.73873047,87.26963186)
\curveto(147.80873173,87.28962989)(147.86373168,87.32462985)(147.90373047,87.37463186)
\curveto(147.93373161,87.42462975)(147.95373159,87.48462969)(147.96373047,87.55463186)
\lineto(147.96373047,87.79463186)
\lineto(147.96373047,88.54463186)
\lineto(147.96373047,91.34963186)
\lineto(147.96373047,92.00963186)
\curveto(147.96373158,92.09962508)(147.96873157,92.18462499)(147.97873047,92.26463186)
\curveto(147.97873156,92.34462483)(147.99873154,92.40962477)(148.03873047,92.45963186)
\curveto(148.07873146,92.50962467)(148.15373139,92.54962463)(148.26373047,92.57963186)
\curveto(148.36373118,92.61962456)(148.46373108,92.62962455)(148.56373047,92.60963186)
\lineto(148.69873047,92.60963186)
\curveto(148.76873077,92.58962459)(148.82873071,92.56962461)(148.87873047,92.54963186)
\curveto(148.92873061,92.52962465)(148.96873057,92.49462468)(148.99873047,92.44463186)
\curveto(149.0387305,92.39462478)(149.05873048,92.32462485)(149.05873047,92.23463186)
\lineto(149.05873047,91.96463186)
\lineto(149.05873047,91.06463186)
\lineto(149.05873047,87.55463186)
\lineto(149.05873047,86.48963186)
\curveto(149.05873048,86.40963077)(149.06373048,86.31963086)(149.07373047,86.21963186)
\curveto(149.07373047,86.11963106)(149.06373048,86.03463114)(149.04373047,85.96463186)
\curveto(148.97373057,85.75463142)(148.79373075,85.68963149)(148.50373047,85.76963186)
\curveto(148.46373108,85.7796314)(148.42873111,85.7796314)(148.39873047,85.76963186)
\curveto(148.35873118,85.76963141)(148.31373123,85.7796314)(148.26373047,85.79963186)
\curveto(148.18373136,85.81963136)(148.09873144,85.83963134)(148.00873047,85.85963186)
\curveto(147.91873162,85.8796313)(147.83373171,85.90463127)(147.75373047,85.93463186)
\curveto(147.26373228,86.09463108)(146.84873269,86.29463088)(146.50873047,86.53463186)
\curveto(146.25873328,86.71463046)(146.03373351,86.91963026)(145.83373047,87.14963186)
\curveto(145.62373392,87.3796298)(145.42873411,87.61962956)(145.24873047,87.86963186)
\curveto(145.06873447,88.12962905)(144.89873464,88.39462878)(144.73873047,88.66463186)
\curveto(144.56873497,88.94462823)(144.39373515,89.21462796)(144.21373047,89.47463186)
\curveto(144.13373541,89.58462759)(144.05873548,89.68962749)(143.98873047,89.78963186)
\curveto(143.91873562,89.89962728)(143.8437357,90.00962717)(143.76373047,90.11963186)
\curveto(143.73373581,90.15962702)(143.70373584,90.19462698)(143.67373047,90.22463186)
\curveto(143.63373591,90.26462691)(143.60373594,90.30462687)(143.58373047,90.34463186)
\curveto(143.47373607,90.48462669)(143.34873619,90.60962657)(143.20873047,90.71963186)
\curveto(143.17873636,90.73962644)(143.15373639,90.76462641)(143.13373047,90.79463186)
\curveto(143.10373644,90.82462635)(143.07373647,90.84962633)(143.04373047,90.86963186)
\curveto(142.9437366,90.94962623)(142.8437367,91.01462616)(142.74373047,91.06463186)
\curveto(142.6437369,91.12462605)(142.53373701,91.179626)(142.41373047,91.22963186)
\curveto(142.3437372,91.25962592)(142.26873727,91.2796259)(142.18873047,91.28963186)
\lineto(141.94873047,91.34963186)
\lineto(141.85873047,91.34963186)
\curveto(141.82873771,91.35962582)(141.79873774,91.36462581)(141.76873047,91.36463186)
\curveto(141.69873784,91.38462579)(141.60373794,91.38962579)(141.48373047,91.37963186)
\curveto(141.35373819,91.3796258)(141.25373829,91.36962581)(141.18373047,91.34963186)
\curveto(141.10373844,91.32962585)(141.02873851,91.30962587)(140.95873047,91.28963186)
\curveto(140.87873866,91.2796259)(140.79873874,91.25962592)(140.71873047,91.22963186)
\curveto(140.47873906,91.11962606)(140.27873926,90.96962621)(140.11873047,90.77963186)
\curveto(139.94873959,90.59962658)(139.80873973,90.3796268)(139.69873047,90.11963186)
\curveto(139.67873986,90.04962713)(139.66373988,89.9796272)(139.65373047,89.90963186)
\curveto(139.63373991,89.83962734)(139.61373993,89.76462741)(139.59373047,89.68463186)
\curveto(139.57373997,89.60462757)(139.56373998,89.49462768)(139.56373047,89.35463186)
\curveto(139.56373998,89.22462795)(139.57373997,89.11962806)(139.59373047,89.03963186)
\curveto(139.60373994,88.9796282)(139.60873993,88.92462825)(139.60873047,88.87463186)
\curveto(139.60873993,88.82462835)(139.61873992,88.7746284)(139.63873047,88.72463186)
\curveto(139.67873986,88.62462855)(139.71873982,88.52962865)(139.75873047,88.43963186)
\curveto(139.79873974,88.35962882)(139.8437397,88.2796289)(139.89373047,88.19963186)
\curveto(139.91373963,88.16962901)(139.9387396,88.13962904)(139.96873047,88.10963186)
\curveto(139.99873954,88.08962909)(140.02373952,88.06462911)(140.04373047,88.03463186)
\lineto(140.11873047,87.95963186)
\curveto(140.1387394,87.92962925)(140.15873938,87.90462927)(140.17873047,87.88463186)
\lineto(140.38873047,87.73463186)
\curveto(140.44873909,87.69462948)(140.51373903,87.64962953)(140.58373047,87.59963186)
\curveto(140.67373887,87.53962964)(140.77873876,87.48962969)(140.89873047,87.44963186)
\curveto(141.00873853,87.41962976)(141.11873842,87.38462979)(141.22873047,87.34463186)
\curveto(141.3387382,87.30462987)(141.48373806,87.2796299)(141.66373047,87.26963186)
\curveto(141.83373771,87.25962992)(141.95873758,87.22962995)(142.03873047,87.17963186)
\curveto(142.11873742,87.12963005)(142.16373738,87.05463012)(142.17373047,86.95463186)
\curveto(142.18373736,86.85463032)(142.18873735,86.74463043)(142.18873047,86.62463186)
\curveto(142.18873735,86.58463059)(142.19373735,86.54463063)(142.20373047,86.50463186)
\curveto(142.20373734,86.46463071)(142.19873734,86.42963075)(142.18873047,86.39963186)
\curveto(142.16873737,86.34963083)(142.15873738,86.29963088)(142.15873047,86.24963186)
\curveto(142.15873738,86.20963097)(142.14873739,86.16963101)(142.12873047,86.12963186)
\curveto(142.06873747,86.03963114)(141.93373761,85.99463118)(141.72373047,85.99463186)
\lineto(141.60373047,85.99463186)
\curveto(141.543738,86.00463117)(141.48373806,86.00963117)(141.42373047,86.00963186)
\curveto(141.35373819,86.01963116)(141.28873825,86.02963115)(141.22873047,86.03963186)
\curveto(141.11873842,86.05963112)(141.01873852,86.0796311)(140.92873047,86.09963186)
\curveto(140.82873871,86.11963106)(140.73373881,86.14963103)(140.64373047,86.18963186)
\curveto(140.57373897,86.20963097)(140.51373903,86.22963095)(140.46373047,86.24963186)
\lineto(140.28373047,86.30963186)
\curveto(140.02373952,86.42963075)(139.77873976,86.58463059)(139.54873047,86.77463186)
\curveto(139.31874022,86.9746302)(139.13374041,87.18962999)(138.99373047,87.41963186)
\curveto(138.91374063,87.52962965)(138.84874069,87.64462953)(138.79873047,87.76463186)
\lineto(138.64873047,88.15463186)
\curveto(138.59874094,88.26462891)(138.56874097,88.3796288)(138.55873047,88.49963186)
\curveto(138.538741,88.61962856)(138.51374103,88.74462843)(138.48373047,88.87463186)
\curveto(138.48374106,88.94462823)(138.48374106,89.00962817)(138.48373047,89.06963186)
\curveto(138.47374107,89.12962805)(138.46374108,89.19462798)(138.45373047,89.26463186)
}
}
{
\newrgbcolor{curcolor}{0 0 0}
\pscustom[linestyle=none,fillstyle=solid,fillcolor=curcolor]
{
\newpath
\moveto(143.97373047,101.36424123)
\lineto(144.22873047,101.36424123)
\curveto(144.30873523,101.37423353)(144.38373516,101.36923353)(144.45373047,101.34924123)
\lineto(144.69373047,101.34924123)
\lineto(144.85873047,101.34924123)
\curveto(144.95873458,101.32923357)(145.06373448,101.31923358)(145.17373047,101.31924123)
\curveto(145.27373427,101.31923358)(145.37373417,101.30923359)(145.47373047,101.28924123)
\lineto(145.62373047,101.28924123)
\curveto(145.76373378,101.25923364)(145.90373364,101.23923366)(146.04373047,101.22924123)
\curveto(146.17373337,101.21923368)(146.30373324,101.19423371)(146.43373047,101.15424123)
\curveto(146.51373303,101.13423377)(146.59873294,101.11423379)(146.68873047,101.09424123)
\lineto(146.92873047,101.03424123)
\lineto(147.22873047,100.91424123)
\curveto(147.31873222,100.88423402)(147.40873213,100.84923405)(147.49873047,100.80924123)
\curveto(147.71873182,100.70923419)(147.93373161,100.57423433)(148.14373047,100.40424123)
\curveto(148.35373119,100.24423466)(148.52373102,100.06923483)(148.65373047,99.87924123)
\curveto(148.69373085,99.82923507)(148.73373081,99.76923513)(148.77373047,99.69924123)
\curveto(148.80373074,99.63923526)(148.8387307,99.57923532)(148.87873047,99.51924123)
\curveto(148.92873061,99.43923546)(148.96873057,99.34423556)(148.99873047,99.23424123)
\curveto(149.02873051,99.12423578)(149.05873048,99.01923588)(149.08873047,98.91924123)
\curveto(149.12873041,98.80923609)(149.15373039,98.6992362)(149.16373047,98.58924123)
\curveto(149.17373037,98.47923642)(149.18873035,98.36423654)(149.20873047,98.24424123)
\curveto(149.21873032,98.2042367)(149.21873032,98.15923674)(149.20873047,98.10924123)
\curveto(149.20873033,98.06923683)(149.21373033,98.02923687)(149.22373047,97.98924123)
\curveto(149.23373031,97.94923695)(149.2387303,97.89423701)(149.23873047,97.82424123)
\curveto(149.2387303,97.75423715)(149.23373031,97.7042372)(149.22373047,97.67424123)
\curveto(149.20373034,97.62423728)(149.19873034,97.57923732)(149.20873047,97.53924123)
\curveto(149.21873032,97.4992374)(149.21873032,97.46423744)(149.20873047,97.43424123)
\lineto(149.20873047,97.34424123)
\curveto(149.18873035,97.28423762)(149.17373037,97.21923768)(149.16373047,97.14924123)
\curveto(149.16373038,97.08923781)(149.15873038,97.02423788)(149.14873047,96.95424123)
\curveto(149.09873044,96.78423812)(149.04873049,96.62423828)(148.99873047,96.47424123)
\curveto(148.94873059,96.32423858)(148.88373066,96.17923872)(148.80373047,96.03924123)
\curveto(148.76373078,95.98923891)(148.73373081,95.93423897)(148.71373047,95.87424123)
\curveto(148.68373086,95.82423908)(148.64873089,95.77423913)(148.60873047,95.72424123)
\curveto(148.42873111,95.48423942)(148.20873133,95.28423962)(147.94873047,95.12424123)
\curveto(147.68873185,94.96423994)(147.40373214,94.82424008)(147.09373047,94.70424123)
\curveto(146.95373259,94.64424026)(146.81373273,94.5992403)(146.67373047,94.56924123)
\curveto(146.52373302,94.53924036)(146.36873317,94.5042404)(146.20873047,94.46424123)
\curveto(146.09873344,94.44424046)(145.98873355,94.42924047)(145.87873047,94.41924123)
\curveto(145.76873377,94.40924049)(145.65873388,94.39424051)(145.54873047,94.37424123)
\curveto(145.50873403,94.36424054)(145.46873407,94.35924054)(145.42873047,94.35924123)
\curveto(145.38873415,94.36924053)(145.34873419,94.36924053)(145.30873047,94.35924123)
\curveto(145.25873428,94.34924055)(145.20873433,94.34424056)(145.15873047,94.34424123)
\lineto(144.99373047,94.34424123)
\curveto(144.9437346,94.32424058)(144.89373465,94.31924058)(144.84373047,94.32924123)
\curveto(144.78373476,94.33924056)(144.72873481,94.33924056)(144.67873047,94.32924123)
\curveto(144.6387349,94.31924058)(144.59373495,94.31924058)(144.54373047,94.32924123)
\curveto(144.49373505,94.33924056)(144.4437351,94.33424057)(144.39373047,94.31424123)
\curveto(144.32373522,94.29424061)(144.24873529,94.28924061)(144.16873047,94.29924123)
\curveto(144.07873546,94.30924059)(143.99373555,94.31424059)(143.91373047,94.31424123)
\curveto(143.82373572,94.31424059)(143.72373582,94.30924059)(143.61373047,94.29924123)
\curveto(143.49373605,94.28924061)(143.39373615,94.29424061)(143.31373047,94.31424123)
\lineto(143.02873047,94.31424123)
\lineto(142.39873047,94.35924123)
\curveto(142.29873724,94.36924053)(142.20373734,94.37924052)(142.11373047,94.38924123)
\lineto(141.81373047,94.41924123)
\curveto(141.76373778,94.43924046)(141.71373783,94.44424046)(141.66373047,94.43424123)
\curveto(141.60373794,94.43424047)(141.54873799,94.44424046)(141.49873047,94.46424123)
\curveto(141.32873821,94.51424039)(141.16373838,94.55424035)(141.00373047,94.58424123)
\curveto(140.83373871,94.61424029)(140.67373887,94.66424024)(140.52373047,94.73424123)
\curveto(140.06373948,94.92423998)(139.68873985,95.14423976)(139.39873047,95.39424123)
\curveto(139.10874043,95.65423925)(138.86374068,96.01423889)(138.66373047,96.47424123)
\curveto(138.61374093,96.6042383)(138.57874096,96.73423817)(138.55873047,96.86424123)
\curveto(138.538741,97.0042379)(138.51374103,97.14423776)(138.48373047,97.28424123)
\curveto(138.47374107,97.35423755)(138.46874107,97.41923748)(138.46873047,97.47924123)
\curveto(138.46874107,97.53923736)(138.46374108,97.6042373)(138.45373047,97.67424123)
\curveto(138.43374111,98.5042364)(138.58374096,99.17423573)(138.90373047,99.68424123)
\curveto(139.21374033,100.19423471)(139.65373989,100.57423433)(140.22373047,100.82424123)
\curveto(140.3437392,100.87423403)(140.46873907,100.91923398)(140.59873047,100.95924123)
\curveto(140.72873881,100.9992339)(140.86373868,101.04423386)(141.00373047,101.09424123)
\curveto(141.08373846,101.11423379)(141.16873837,101.12923377)(141.25873047,101.13924123)
\lineto(141.49873047,101.19924123)
\curveto(141.60873793,101.22923367)(141.71873782,101.24423366)(141.82873047,101.24424123)
\curveto(141.9387376,101.25423365)(142.04873749,101.26923363)(142.15873047,101.28924123)
\curveto(142.20873733,101.30923359)(142.25373729,101.31423359)(142.29373047,101.30424123)
\curveto(142.33373721,101.3042336)(142.37373717,101.30923359)(142.41373047,101.31924123)
\curveto(142.46373708,101.32923357)(142.51873702,101.32923357)(142.57873047,101.31924123)
\curveto(142.62873691,101.31923358)(142.67873686,101.32423358)(142.72873047,101.33424123)
\lineto(142.86373047,101.33424123)
\curveto(142.92373662,101.35423355)(142.99373655,101.35423355)(143.07373047,101.33424123)
\curveto(143.1437364,101.32423358)(143.20873633,101.32923357)(143.26873047,101.34924123)
\curveto(143.29873624,101.35923354)(143.3387362,101.36423354)(143.38873047,101.36424123)
\lineto(143.50873047,101.36424123)
\lineto(143.97373047,101.36424123)
\moveto(146.29873047,99.81924123)
\curveto(145.97873356,99.91923498)(145.61373393,99.97923492)(145.20373047,99.99924123)
\curveto(144.79373475,100.01923488)(144.38373516,100.02923487)(143.97373047,100.02924123)
\curveto(143.543736,100.02923487)(143.12373642,100.01923488)(142.71373047,99.99924123)
\curveto(142.30373724,99.97923492)(141.91873762,99.93423497)(141.55873047,99.86424123)
\curveto(141.19873834,99.79423511)(140.87873866,99.68423522)(140.59873047,99.53424123)
\curveto(140.30873923,99.39423551)(140.07373947,99.1992357)(139.89373047,98.94924123)
\curveto(139.78373976,98.78923611)(139.70373984,98.60923629)(139.65373047,98.40924123)
\curveto(139.59373995,98.20923669)(139.56373998,97.96423694)(139.56373047,97.67424123)
\curveto(139.58373996,97.65423725)(139.59373995,97.61923728)(139.59373047,97.56924123)
\curveto(139.58373996,97.51923738)(139.58373996,97.47923742)(139.59373047,97.44924123)
\curveto(139.61373993,97.36923753)(139.63373991,97.29423761)(139.65373047,97.22424123)
\curveto(139.66373988,97.16423774)(139.68373986,97.0992378)(139.71373047,97.02924123)
\curveto(139.83373971,96.75923814)(140.00373954,96.53923836)(140.22373047,96.36924123)
\curveto(140.43373911,96.20923869)(140.67873886,96.07423883)(140.95873047,95.96424123)
\curveto(141.06873847,95.91423899)(141.18873835,95.87423903)(141.31873047,95.84424123)
\curveto(141.4387381,95.82423908)(141.56373798,95.7992391)(141.69373047,95.76924123)
\curveto(141.7437378,95.74923915)(141.79873774,95.73923916)(141.85873047,95.73924123)
\curveto(141.90873763,95.73923916)(141.95873758,95.73423917)(142.00873047,95.72424123)
\curveto(142.09873744,95.71423919)(142.19373735,95.7042392)(142.29373047,95.69424123)
\curveto(142.38373716,95.68423922)(142.47873706,95.67423923)(142.57873047,95.66424123)
\curveto(142.65873688,95.66423924)(142.7437368,95.65923924)(142.83373047,95.64924123)
\lineto(143.07373047,95.64924123)
\lineto(143.25373047,95.64924123)
\curveto(143.28373626,95.63923926)(143.31873622,95.63423927)(143.35873047,95.63424123)
\lineto(143.49373047,95.63424123)
\lineto(143.94373047,95.63424123)
\curveto(144.02373552,95.63423927)(144.10873543,95.62923927)(144.19873047,95.61924123)
\curveto(144.27873526,95.61923928)(144.35373519,95.62923927)(144.42373047,95.64924123)
\lineto(144.69373047,95.64924123)
\curveto(144.71373483,95.64923925)(144.7437348,95.64423926)(144.78373047,95.63424123)
\curveto(144.81373473,95.63423927)(144.8387347,95.63923926)(144.85873047,95.64924123)
\curveto(144.95873458,95.65923924)(145.05873448,95.66423924)(145.15873047,95.66424123)
\curveto(145.24873429,95.67423923)(145.34873419,95.68423922)(145.45873047,95.69424123)
\curveto(145.57873396,95.72423918)(145.70373384,95.73923916)(145.83373047,95.73924123)
\curveto(145.95373359,95.74923915)(146.06873347,95.77423913)(146.17873047,95.81424123)
\curveto(146.47873306,95.89423901)(146.7437328,95.97923892)(146.97373047,96.06924123)
\curveto(147.20373234,96.16923873)(147.41873212,96.31423859)(147.61873047,96.50424123)
\curveto(147.81873172,96.71423819)(147.96873157,96.97923792)(148.06873047,97.29924123)
\curveto(148.08873145,97.33923756)(148.09873144,97.37423753)(148.09873047,97.40424123)
\curveto(148.08873145,97.44423746)(148.09373145,97.48923741)(148.11373047,97.53924123)
\curveto(148.12373142,97.57923732)(148.13373141,97.64923725)(148.14373047,97.74924123)
\curveto(148.15373139,97.85923704)(148.14873139,97.94423696)(148.12873047,98.00424123)
\curveto(148.10873143,98.07423683)(148.09873144,98.14423676)(148.09873047,98.21424123)
\curveto(148.08873145,98.28423662)(148.07373147,98.34923655)(148.05373047,98.40924123)
\curveto(147.99373155,98.60923629)(147.90873163,98.78923611)(147.79873047,98.94924123)
\curveto(147.77873176,98.97923592)(147.75873178,99.0042359)(147.73873047,99.02424123)
\lineto(147.67873047,99.08424123)
\curveto(147.65873188,99.12423578)(147.61873192,99.17423573)(147.55873047,99.23424123)
\curveto(147.41873212,99.33423557)(147.28873225,99.41923548)(147.16873047,99.48924123)
\curveto(147.04873249,99.55923534)(146.90373264,99.62923527)(146.73373047,99.69924123)
\curveto(146.66373288,99.72923517)(146.59373295,99.74923515)(146.52373047,99.75924123)
\curveto(146.45373309,99.77923512)(146.37873316,99.7992351)(146.29873047,99.81924123)
}
}
{
\newrgbcolor{curcolor}{0 0 0}
\pscustom[linestyle=none,fillstyle=solid,fillcolor=curcolor]
{
\newpath
\moveto(138.45373047,106.77385061)
\curveto(138.45374109,106.87384575)(138.46374108,106.96884566)(138.48373047,107.05885061)
\curveto(138.49374105,107.14884548)(138.52374102,107.21384541)(138.57373047,107.25385061)
\curveto(138.65374089,107.31384531)(138.75874078,107.34384528)(138.88873047,107.34385061)
\lineto(139.27873047,107.34385061)
\lineto(140.77873047,107.34385061)
\lineto(147.16873047,107.34385061)
\lineto(148.33873047,107.34385061)
\lineto(148.65373047,107.34385061)
\curveto(148.75373079,107.35384527)(148.83373071,107.33884529)(148.89373047,107.29885061)
\curveto(148.97373057,107.24884538)(149.02373052,107.17384545)(149.04373047,107.07385061)
\curveto(149.05373049,106.98384564)(149.05873048,106.87384575)(149.05873047,106.74385061)
\lineto(149.05873047,106.51885061)
\curveto(149.0387305,106.43884619)(149.02373052,106.36884626)(149.01373047,106.30885061)
\curveto(148.99373055,106.24884638)(148.95373059,106.19884643)(148.89373047,106.15885061)
\curveto(148.83373071,106.11884651)(148.75873078,106.09884653)(148.66873047,106.09885061)
\lineto(148.36873047,106.09885061)
\lineto(147.27373047,106.09885061)
\lineto(141.93373047,106.09885061)
\curveto(141.8437377,106.07884655)(141.76873777,106.06384656)(141.70873047,106.05385061)
\curveto(141.6387379,106.05384657)(141.57873796,106.0238466)(141.52873047,105.96385061)
\curveto(141.47873806,105.89384673)(141.45373809,105.80384682)(141.45373047,105.69385061)
\curveto(141.4437381,105.59384703)(141.4387381,105.48384714)(141.43873047,105.36385061)
\lineto(141.43873047,104.22385061)
\lineto(141.43873047,103.72885061)
\curveto(141.42873811,103.56884906)(141.36873817,103.45884917)(141.25873047,103.39885061)
\curveto(141.22873831,103.37884925)(141.19873834,103.36884926)(141.16873047,103.36885061)
\curveto(141.12873841,103.36884926)(141.08373846,103.36384926)(141.03373047,103.35385061)
\curveto(140.91373863,103.33384929)(140.80373874,103.33884929)(140.70373047,103.36885061)
\curveto(140.60373894,103.40884922)(140.53373901,103.46384916)(140.49373047,103.53385061)
\curveto(140.4437391,103.61384901)(140.41873912,103.73384889)(140.41873047,103.89385061)
\curveto(140.41873912,104.05384857)(140.40373914,104.18884844)(140.37373047,104.29885061)
\curveto(140.36373918,104.34884828)(140.35873918,104.40384822)(140.35873047,104.46385061)
\curveto(140.34873919,104.5238481)(140.33373921,104.58384804)(140.31373047,104.64385061)
\curveto(140.26373928,104.79384783)(140.21373933,104.93884769)(140.16373047,105.07885061)
\curveto(140.10373944,105.21884741)(140.03373951,105.35384727)(139.95373047,105.48385061)
\curveto(139.86373968,105.623847)(139.75873978,105.74384688)(139.63873047,105.84385061)
\curveto(139.51874002,105.94384668)(139.38874015,106.03884659)(139.24873047,106.12885061)
\curveto(139.14874039,106.18884644)(139.0387405,106.23384639)(138.91873047,106.26385061)
\curveto(138.79874074,106.30384632)(138.69374085,106.35384627)(138.60373047,106.41385061)
\curveto(138.543741,106.46384616)(138.50374104,106.53384609)(138.48373047,106.62385061)
\curveto(138.47374107,106.64384598)(138.46874107,106.66884596)(138.46873047,106.69885061)
\curveto(138.46874107,106.7288459)(138.46374108,106.75384587)(138.45373047,106.77385061)
}
}
{
\newrgbcolor{curcolor}{0 0 0}
\pscustom[linestyle=none,fillstyle=solid,fillcolor=curcolor]
{
\newpath
\moveto(138.45373047,115.12345998)
\curveto(138.45374109,115.22345513)(138.46374108,115.31845503)(138.48373047,115.40845998)
\curveto(138.49374105,115.49845485)(138.52374102,115.56345479)(138.57373047,115.60345998)
\curveto(138.65374089,115.66345469)(138.75874078,115.69345466)(138.88873047,115.69345998)
\lineto(139.27873047,115.69345998)
\lineto(140.77873047,115.69345998)
\lineto(147.16873047,115.69345998)
\lineto(148.33873047,115.69345998)
\lineto(148.65373047,115.69345998)
\curveto(148.75373079,115.70345465)(148.83373071,115.68845466)(148.89373047,115.64845998)
\curveto(148.97373057,115.59845475)(149.02373052,115.52345483)(149.04373047,115.42345998)
\curveto(149.05373049,115.33345502)(149.05873048,115.22345513)(149.05873047,115.09345998)
\lineto(149.05873047,114.86845998)
\curveto(149.0387305,114.78845556)(149.02373052,114.71845563)(149.01373047,114.65845998)
\curveto(148.99373055,114.59845575)(148.95373059,114.5484558)(148.89373047,114.50845998)
\curveto(148.83373071,114.46845588)(148.75873078,114.4484559)(148.66873047,114.44845998)
\lineto(148.36873047,114.44845998)
\lineto(147.27373047,114.44845998)
\lineto(141.93373047,114.44845998)
\curveto(141.8437377,114.42845592)(141.76873777,114.41345594)(141.70873047,114.40345998)
\curveto(141.6387379,114.40345595)(141.57873796,114.37345598)(141.52873047,114.31345998)
\curveto(141.47873806,114.24345611)(141.45373809,114.1534562)(141.45373047,114.04345998)
\curveto(141.4437381,113.94345641)(141.4387381,113.83345652)(141.43873047,113.71345998)
\lineto(141.43873047,112.57345998)
\lineto(141.43873047,112.07845998)
\curveto(141.42873811,111.91845843)(141.36873817,111.80845854)(141.25873047,111.74845998)
\curveto(141.22873831,111.72845862)(141.19873834,111.71845863)(141.16873047,111.71845998)
\curveto(141.12873841,111.71845863)(141.08373846,111.71345864)(141.03373047,111.70345998)
\curveto(140.91373863,111.68345867)(140.80373874,111.68845866)(140.70373047,111.71845998)
\curveto(140.60373894,111.75845859)(140.53373901,111.81345854)(140.49373047,111.88345998)
\curveto(140.4437391,111.96345839)(140.41873912,112.08345827)(140.41873047,112.24345998)
\curveto(140.41873912,112.40345795)(140.40373914,112.53845781)(140.37373047,112.64845998)
\curveto(140.36373918,112.69845765)(140.35873918,112.7534576)(140.35873047,112.81345998)
\curveto(140.34873919,112.87345748)(140.33373921,112.93345742)(140.31373047,112.99345998)
\curveto(140.26373928,113.14345721)(140.21373933,113.28845706)(140.16373047,113.42845998)
\curveto(140.10373944,113.56845678)(140.03373951,113.70345665)(139.95373047,113.83345998)
\curveto(139.86373968,113.97345638)(139.75873978,114.09345626)(139.63873047,114.19345998)
\curveto(139.51874002,114.29345606)(139.38874015,114.38845596)(139.24873047,114.47845998)
\curveto(139.14874039,114.53845581)(139.0387405,114.58345577)(138.91873047,114.61345998)
\curveto(138.79874074,114.6534557)(138.69374085,114.70345565)(138.60373047,114.76345998)
\curveto(138.543741,114.81345554)(138.50374104,114.88345547)(138.48373047,114.97345998)
\curveto(138.47374107,114.99345536)(138.46874107,115.01845533)(138.46873047,115.04845998)
\curveto(138.46874107,115.07845527)(138.46374108,115.10345525)(138.45373047,115.12345998)
}
}
{
\newrgbcolor{curcolor}{0 0 0}
\pscustom[linestyle=none,fillstyle=solid,fillcolor=curcolor]
{
\newpath
\moveto(159.29006165,31.67142873)
\lineto(159.29006165,32.58642873)
\curveto(159.29007234,32.68642608)(159.29007234,32.78142599)(159.29006165,32.87142873)
\curveto(159.29007234,32.96142581)(159.31007232,33.03642573)(159.35006165,33.09642873)
\curveto(159.41007222,33.18642558)(159.49007214,33.24642552)(159.59006165,33.27642873)
\curveto(159.69007194,33.31642545)(159.79507184,33.36142541)(159.90506165,33.41142873)
\curveto(160.09507154,33.49142528)(160.28507135,33.56142521)(160.47506165,33.62142873)
\curveto(160.66507097,33.69142508)(160.85507078,33.766425)(161.04506165,33.84642873)
\curveto(161.22507041,33.91642485)(161.41007022,33.98142479)(161.60006165,34.04142873)
\curveto(161.78006985,34.10142467)(161.96006967,34.1714246)(162.14006165,34.25142873)
\curveto(162.28006935,34.31142446)(162.42506921,34.3664244)(162.57506165,34.41642873)
\curveto(162.72506891,34.4664243)(162.87006876,34.52142425)(163.01006165,34.58142873)
\curveto(163.46006817,34.76142401)(163.91506772,34.93142384)(164.37506165,35.09142873)
\curveto(164.82506681,35.25142352)(165.27506636,35.42142335)(165.72506165,35.60142873)
\curveto(165.77506586,35.62142315)(165.82506581,35.63642313)(165.87506165,35.64642873)
\lineto(166.02506165,35.70642873)
\curveto(166.24506539,35.79642297)(166.47006516,35.88142289)(166.70006165,35.96142873)
\curveto(166.92006471,36.04142273)(167.14006449,36.12642264)(167.36006165,36.21642873)
\curveto(167.45006418,36.25642251)(167.56006407,36.29642247)(167.69006165,36.33642873)
\curveto(167.81006382,36.37642239)(167.88006375,36.44142233)(167.90006165,36.53142873)
\curveto(167.91006372,36.5714222)(167.91006372,36.60142217)(167.90006165,36.62142873)
\lineto(167.84006165,36.68142873)
\curveto(167.79006384,36.73142204)(167.7350639,36.766422)(167.67506165,36.78642873)
\curveto(167.61506402,36.81642195)(167.55006408,36.84642192)(167.48006165,36.87642873)
\lineto(166.85006165,37.11642873)
\curveto(166.630065,37.19642157)(166.41506522,37.27642149)(166.20506165,37.35642873)
\lineto(166.05506165,37.41642873)
\lineto(165.87506165,37.47642873)
\curveto(165.68506595,37.55642121)(165.49506614,37.62642114)(165.30506165,37.68642873)
\curveto(165.10506653,37.75642101)(164.90506673,37.83142094)(164.70506165,37.91142873)
\curveto(164.12506751,38.15142062)(163.54006809,38.3714204)(162.95006165,38.57142873)
\curveto(162.36006927,38.78141999)(161.77506986,39.00641976)(161.19506165,39.24642873)
\curveto(160.99507064,39.32641944)(160.79007084,39.40141937)(160.58006165,39.47142873)
\curveto(160.37007126,39.55141922)(160.16507147,39.63141914)(159.96506165,39.71142873)
\curveto(159.88507175,39.75141902)(159.78507185,39.78641898)(159.66506165,39.81642873)
\curveto(159.54507209,39.85641891)(159.46007217,39.91141886)(159.41006165,39.98142873)
\curveto(159.37007226,40.04141873)(159.34007229,40.11641865)(159.32006165,40.20642873)
\curveto(159.30007233,40.30641846)(159.29007234,40.41641835)(159.29006165,40.53642873)
\curveto(159.28007235,40.65641811)(159.28007235,40.77641799)(159.29006165,40.89642873)
\curveto(159.29007234,41.01641775)(159.29007234,41.12641764)(159.29006165,41.22642873)
\curveto(159.29007234,41.31641745)(159.29007234,41.40641736)(159.29006165,41.49642873)
\curveto(159.29007234,41.59641717)(159.31007232,41.6714171)(159.35006165,41.72142873)
\curveto(159.40007223,41.81141696)(159.49007214,41.86141691)(159.62006165,41.87142873)
\curveto(159.75007188,41.88141689)(159.89007174,41.88641688)(160.04006165,41.88642873)
\lineto(161.69006165,41.88642873)
\lineto(167.96006165,41.88642873)
\lineto(169.22006165,41.88642873)
\curveto(169.3300623,41.88641688)(169.44006219,41.88641688)(169.55006165,41.88642873)
\curveto(169.66006197,41.89641687)(169.74506189,41.87641689)(169.80506165,41.82642873)
\curveto(169.86506177,41.79641697)(169.90506173,41.75141702)(169.92506165,41.69142873)
\curveto(169.9350617,41.63141714)(169.95006168,41.56141721)(169.97006165,41.48142873)
\lineto(169.97006165,41.24142873)
\lineto(169.97006165,40.88142873)
\curveto(169.96006167,40.771418)(169.91506172,40.69141808)(169.83506165,40.64142873)
\curveto(169.80506183,40.62141815)(169.77506186,40.60641816)(169.74506165,40.59642873)
\curveto(169.70506193,40.59641817)(169.66006197,40.58641818)(169.61006165,40.56642873)
\lineto(169.44506165,40.56642873)
\curveto(169.38506225,40.55641821)(169.31506232,40.55141822)(169.23506165,40.55142873)
\curveto(169.15506248,40.56141821)(169.08006255,40.5664182)(169.01006165,40.56642873)
\lineto(168.17006165,40.56642873)
\lineto(163.74506165,40.56642873)
\curveto(163.49506814,40.5664182)(163.24506839,40.5664182)(162.99506165,40.56642873)
\curveto(162.7350689,40.5664182)(162.48506915,40.56141821)(162.24506165,40.55142873)
\curveto(162.14506949,40.55141822)(162.0350696,40.54641822)(161.91506165,40.53642873)
\curveto(161.79506984,40.52641824)(161.7350699,40.4714183)(161.73506165,40.37142873)
\lineto(161.75006165,40.37142873)
\curveto(161.77006986,40.30141847)(161.8350698,40.24141853)(161.94506165,40.19142873)
\curveto(162.05506958,40.15141862)(162.15006948,40.11641865)(162.23006165,40.08642873)
\curveto(162.40006923,40.01641875)(162.57506906,39.95141882)(162.75506165,39.89142873)
\curveto(162.92506871,39.83141894)(163.09506854,39.76141901)(163.26506165,39.68142873)
\curveto(163.31506832,39.66141911)(163.36006827,39.64641912)(163.40006165,39.63642873)
\curveto(163.44006819,39.62641914)(163.48506815,39.61141916)(163.53506165,39.59142873)
\curveto(163.71506792,39.51141926)(163.90006773,39.44141933)(164.09006165,39.38142873)
\curveto(164.27006736,39.33141944)(164.45006718,39.2664195)(164.63006165,39.18642873)
\curveto(164.78006685,39.11641965)(164.9350667,39.05641971)(165.09506165,39.00642873)
\curveto(165.24506639,38.95641981)(165.39506624,38.90141987)(165.54506165,38.84142873)
\curveto(166.01506562,38.64142013)(166.49006514,38.46142031)(166.97006165,38.30142873)
\curveto(167.44006419,38.14142063)(167.90506373,37.9664208)(168.36506165,37.77642873)
\curveto(168.54506309,37.69642107)(168.72506291,37.62642114)(168.90506165,37.56642873)
\curveto(169.08506255,37.50642126)(169.26506237,37.44142133)(169.44506165,37.37142873)
\curveto(169.55506208,37.32142145)(169.66006197,37.2714215)(169.76006165,37.22142873)
\curveto(169.85006178,37.18142159)(169.91506172,37.09642167)(169.95506165,36.96642873)
\curveto(169.96506167,36.94642182)(169.97006166,36.92142185)(169.97006165,36.89142873)
\curveto(169.96006167,36.8714219)(169.96006167,36.84642192)(169.97006165,36.81642873)
\curveto(169.98006165,36.78642198)(169.98506165,36.75142202)(169.98506165,36.71142873)
\curveto(169.97506166,36.6714221)(169.97006166,36.63142214)(169.97006165,36.59142873)
\lineto(169.97006165,36.29142873)
\curveto(169.97006166,36.19142258)(169.94506169,36.11142266)(169.89506165,36.05142873)
\curveto(169.84506179,35.9714228)(169.77506186,35.91142286)(169.68506165,35.87142873)
\curveto(169.58506205,35.84142293)(169.48506215,35.80142297)(169.38506165,35.75142873)
\curveto(169.18506245,35.6714231)(168.98006265,35.59142318)(168.77006165,35.51142873)
\curveto(168.55006308,35.44142333)(168.34006329,35.3664234)(168.14006165,35.28642873)
\curveto(167.96006367,35.20642356)(167.78006385,35.13642363)(167.60006165,35.07642873)
\curveto(167.41006422,35.02642374)(167.22506441,34.96142381)(167.04506165,34.88142873)
\curveto(166.48506515,34.65142412)(165.92006571,34.43642433)(165.35006165,34.23642873)
\curveto(164.78006685,34.03642473)(164.21506742,33.82142495)(163.65506165,33.59142873)
\lineto(163.02506165,33.35142873)
\curveto(162.80506883,33.28142549)(162.59506904,33.20642556)(162.39506165,33.12642873)
\curveto(162.28506935,33.07642569)(162.18006945,33.03142574)(162.08006165,32.99142873)
\curveto(161.97006966,32.96142581)(161.87506976,32.91142586)(161.79506165,32.84142873)
\curveto(161.77506986,32.83142594)(161.76506987,32.82142595)(161.76506165,32.81142873)
\lineto(161.73506165,32.78142873)
\lineto(161.73506165,32.70642873)
\lineto(161.76506165,32.67642873)
\curveto(161.76506987,32.6664261)(161.77006986,32.65642611)(161.78006165,32.64642873)
\curveto(161.8300698,32.62642614)(161.88506975,32.61642615)(161.94506165,32.61642873)
\curveto(162.00506963,32.61642615)(162.06506957,32.60642616)(162.12506165,32.58642873)
\lineto(162.29006165,32.58642873)
\curveto(162.35006928,32.5664262)(162.41506922,32.56142621)(162.48506165,32.57142873)
\curveto(162.55506908,32.58142619)(162.62506901,32.58642618)(162.69506165,32.58642873)
\lineto(163.50506165,32.58642873)
\lineto(168.06506165,32.58642873)
\lineto(169.25006165,32.58642873)
\curveto(169.36006227,32.58642618)(169.47006216,32.58142619)(169.58006165,32.57142873)
\curveto(169.69006194,32.5714262)(169.77506186,32.54642622)(169.83506165,32.49642873)
\curveto(169.91506172,32.44642632)(169.96006167,32.35642641)(169.97006165,32.22642873)
\lineto(169.97006165,31.83642873)
\lineto(169.97006165,31.64142873)
\curveto(169.97006166,31.59142718)(169.96006167,31.54142723)(169.94006165,31.49142873)
\curveto(169.90006173,31.36142741)(169.81506182,31.28642748)(169.68506165,31.26642873)
\curveto(169.55506208,31.25642751)(169.40506223,31.25142752)(169.23506165,31.25142873)
\lineto(167.49506165,31.25142873)
\lineto(161.49506165,31.25142873)
\lineto(160.08506165,31.25142873)
\curveto(159.97507166,31.25142752)(159.86007177,31.24642752)(159.74006165,31.23642873)
\curveto(159.62007201,31.23642753)(159.52507211,31.26142751)(159.45506165,31.31142873)
\curveto(159.39507224,31.35142742)(159.34507229,31.42642734)(159.30506165,31.53642873)
\curveto(159.29507234,31.55642721)(159.29507234,31.57642719)(159.30506165,31.59642873)
\curveto(159.30507233,31.62642714)(159.30007233,31.65142712)(159.29006165,31.67142873)
}
}
{
\newrgbcolor{curcolor}{0 0 0}
\pscustom[linestyle=none,fillstyle=solid,fillcolor=curcolor]
{
\newpath
\moveto(169.41506165,50.87353811)
\curveto(169.57506206,50.90353028)(169.71006192,50.88853029)(169.82006165,50.82853811)
\curveto(169.92006171,50.76853041)(169.99506164,50.68853049)(170.04506165,50.58853811)
\curveto(170.06506157,50.53853064)(170.07506156,50.4835307)(170.07506165,50.42353811)
\curveto(170.07506156,50.37353081)(170.08506155,50.31853086)(170.10506165,50.25853811)
\curveto(170.15506148,50.03853114)(170.14006149,49.81853136)(170.06006165,49.59853811)
\curveto(169.99006164,49.38853179)(169.90006173,49.24353194)(169.79006165,49.16353811)
\curveto(169.72006191,49.11353207)(169.64006199,49.06853211)(169.55006165,49.02853811)
\curveto(169.45006218,48.98853219)(169.37006226,48.93853224)(169.31006165,48.87853811)
\curveto(169.29006234,48.85853232)(169.27006236,48.83353235)(169.25006165,48.80353811)
\curveto(169.2300624,48.7835324)(169.22506241,48.75353243)(169.23506165,48.71353811)
\curveto(169.26506237,48.60353258)(169.32006231,48.49853268)(169.40006165,48.39853811)
\curveto(169.48006215,48.30853287)(169.55006208,48.21853296)(169.61006165,48.12853811)
\curveto(169.69006194,47.99853318)(169.76506187,47.85853332)(169.83506165,47.70853811)
\curveto(169.89506174,47.55853362)(169.95006168,47.39853378)(170.00006165,47.22853811)
\curveto(170.0300616,47.12853405)(170.05006158,47.01853416)(170.06006165,46.89853811)
\curveto(170.07006156,46.78853439)(170.08506155,46.6785345)(170.10506165,46.56853811)
\curveto(170.11506152,46.51853466)(170.12006151,46.47353471)(170.12006165,46.43353811)
\lineto(170.12006165,46.32853811)
\curveto(170.14006149,46.21853496)(170.14006149,46.11353507)(170.12006165,46.01353811)
\lineto(170.12006165,45.87853811)
\curveto(170.11006152,45.82853535)(170.10506153,45.7785354)(170.10506165,45.72853811)
\curveto(170.10506153,45.6785355)(170.09506154,45.63353555)(170.07506165,45.59353811)
\curveto(170.06506157,45.55353563)(170.06006157,45.51853566)(170.06006165,45.48853811)
\curveto(170.07006156,45.46853571)(170.07006156,45.44353574)(170.06006165,45.41353811)
\lineto(170.00006165,45.17353811)
\curveto(169.99006164,45.09353609)(169.97006166,45.01853616)(169.94006165,44.94853811)
\curveto(169.81006182,44.64853653)(169.66506197,44.40353678)(169.50506165,44.21353811)
\curveto(169.3350623,44.03353715)(169.10006253,43.8835373)(168.80006165,43.76353811)
\curveto(168.58006305,43.67353751)(168.31506332,43.62853755)(168.00506165,43.62853811)
\lineto(167.69006165,43.62853811)
\curveto(167.64006399,43.63853754)(167.59006404,43.64353754)(167.54006165,43.64353811)
\lineto(167.36006165,43.67353811)
\lineto(167.03006165,43.79353811)
\curveto(166.92006471,43.83353735)(166.82006481,43.8835373)(166.73006165,43.94353811)
\curveto(166.44006519,44.12353706)(166.22506541,44.36853681)(166.08506165,44.67853811)
\curveto(165.94506569,44.98853619)(165.82006581,45.32853585)(165.71006165,45.69853811)
\curveto(165.67006596,45.83853534)(165.64006599,45.9835352)(165.62006165,46.13353811)
\curveto(165.60006603,46.2835349)(165.57506606,46.43353475)(165.54506165,46.58353811)
\curveto(165.52506611,46.65353453)(165.51506612,46.71853446)(165.51506165,46.77853811)
\curveto(165.51506612,46.84853433)(165.50506613,46.92353426)(165.48506165,47.00353811)
\curveto(165.46506617,47.07353411)(165.45506618,47.14353404)(165.45506165,47.21353811)
\curveto(165.44506619,47.2835339)(165.4300662,47.35853382)(165.41006165,47.43853811)
\curveto(165.35006628,47.68853349)(165.30006633,47.92353326)(165.26006165,48.14353811)
\curveto(165.21006642,48.36353282)(165.09506654,48.53853264)(164.91506165,48.66853811)
\curveto(164.8350668,48.72853245)(164.7350669,48.7785324)(164.61506165,48.81853811)
\curveto(164.48506715,48.85853232)(164.34506729,48.85853232)(164.19506165,48.81853811)
\curveto(163.95506768,48.75853242)(163.76506787,48.66853251)(163.62506165,48.54853811)
\curveto(163.48506815,48.43853274)(163.37506826,48.2785329)(163.29506165,48.06853811)
\curveto(163.24506839,47.94853323)(163.21006842,47.80353338)(163.19006165,47.63353811)
\curveto(163.17006846,47.47353371)(163.16006847,47.30353388)(163.16006165,47.12353811)
\curveto(163.16006847,46.94353424)(163.17006846,46.76853441)(163.19006165,46.59853811)
\curveto(163.21006842,46.42853475)(163.24006839,46.2835349)(163.28006165,46.16353811)
\curveto(163.34006829,45.99353519)(163.42506821,45.82853535)(163.53506165,45.66853811)
\curveto(163.59506804,45.58853559)(163.67506796,45.51353567)(163.77506165,45.44353811)
\curveto(163.86506777,45.3835358)(163.96506767,45.32853585)(164.07506165,45.27853811)
\curveto(164.15506748,45.24853593)(164.24006739,45.21853596)(164.33006165,45.18853811)
\curveto(164.42006721,45.16853601)(164.49006714,45.12353606)(164.54006165,45.05353811)
\curveto(164.57006706,45.01353617)(164.59506704,44.94353624)(164.61506165,44.84353811)
\curveto(164.62506701,44.75353643)(164.630067,44.65853652)(164.63006165,44.55853811)
\curveto(164.630067,44.45853672)(164.62506701,44.35853682)(164.61506165,44.25853811)
\curveto(164.59506704,44.16853701)(164.57006706,44.10353708)(164.54006165,44.06353811)
\curveto(164.51006712,44.02353716)(164.46006717,43.99353719)(164.39006165,43.97353811)
\curveto(164.32006731,43.95353723)(164.24506739,43.95353723)(164.16506165,43.97353811)
\curveto(164.0350676,44.00353718)(163.91506772,44.03353715)(163.80506165,44.06353811)
\curveto(163.68506795,44.10353708)(163.57006806,44.14853703)(163.46006165,44.19853811)
\curveto(163.11006852,44.38853679)(162.84006879,44.62853655)(162.65006165,44.91853811)
\curveto(162.45006918,45.20853597)(162.29006934,45.56853561)(162.17006165,45.99853811)
\curveto(162.15006948,46.09853508)(162.1350695,46.19853498)(162.12506165,46.29853811)
\curveto(162.11506952,46.40853477)(162.10006953,46.51853466)(162.08006165,46.62853811)
\curveto(162.07006956,46.66853451)(162.07006956,46.73353445)(162.08006165,46.82353811)
\curveto(162.08006955,46.91353427)(162.07006956,46.96853421)(162.05006165,46.98853811)
\curveto(162.04006959,47.68853349)(162.12006951,48.29853288)(162.29006165,48.81853811)
\curveto(162.46006917,49.33853184)(162.78506885,49.70353148)(163.26506165,49.91353811)
\curveto(163.46506817,50.00353118)(163.70006793,50.05353113)(163.97006165,50.06353811)
\curveto(164.2300674,50.0835311)(164.50506713,50.09353109)(164.79506165,50.09353811)
\lineto(168.11006165,50.09353811)
\curveto(168.25006338,50.09353109)(168.38506325,50.09853108)(168.51506165,50.10853811)
\curveto(168.64506299,50.11853106)(168.75006288,50.14853103)(168.83006165,50.19853811)
\curveto(168.90006273,50.24853093)(168.95006268,50.31353087)(168.98006165,50.39353811)
\curveto(169.02006261,50.4835307)(169.05006258,50.56853061)(169.07006165,50.64853811)
\curveto(169.08006255,50.72853045)(169.12506251,50.78853039)(169.20506165,50.82853811)
\curveto(169.2350624,50.84853033)(169.26506237,50.85853032)(169.29506165,50.85853811)
\curveto(169.32506231,50.85853032)(169.36506227,50.86353032)(169.41506165,50.87353811)
\moveto(167.75006165,48.72853811)
\curveto(167.61006402,48.78853239)(167.45006418,48.81853236)(167.27006165,48.81853811)
\curveto(167.08006455,48.82853235)(166.88506475,48.83353235)(166.68506165,48.83353811)
\curveto(166.57506506,48.83353235)(166.47506516,48.82853235)(166.38506165,48.81853811)
\curveto(166.29506534,48.80853237)(166.22506541,48.76853241)(166.17506165,48.69853811)
\curveto(166.15506548,48.66853251)(166.14506549,48.59853258)(166.14506165,48.48853811)
\curveto(166.16506547,48.46853271)(166.17506546,48.43353275)(166.17506165,48.38353811)
\curveto(166.17506546,48.33353285)(166.18506545,48.28853289)(166.20506165,48.24853811)
\curveto(166.22506541,48.16853301)(166.24506539,48.0785331)(166.26506165,47.97853811)
\lineto(166.32506165,47.67853811)
\curveto(166.32506531,47.64853353)(166.3300653,47.61353357)(166.34006165,47.57353811)
\lineto(166.34006165,47.46853811)
\curveto(166.38006525,47.31853386)(166.40506523,47.15353403)(166.41506165,46.97353811)
\curveto(166.41506522,46.80353438)(166.4350652,46.64353454)(166.47506165,46.49353811)
\curveto(166.49506514,46.41353477)(166.51506512,46.33853484)(166.53506165,46.26853811)
\curveto(166.54506509,46.20853497)(166.56006507,46.13853504)(166.58006165,46.05853811)
\curveto(166.630065,45.89853528)(166.69506494,45.74853543)(166.77506165,45.60853811)
\curveto(166.84506479,45.46853571)(166.9350647,45.34853583)(167.04506165,45.24853811)
\curveto(167.15506448,45.14853603)(167.29006434,45.07353611)(167.45006165,45.02353811)
\curveto(167.60006403,44.97353621)(167.78506385,44.95353623)(168.00506165,44.96353811)
\curveto(168.10506353,44.96353622)(168.20006343,44.9785362)(168.29006165,45.00853811)
\curveto(168.37006326,45.04853613)(168.44506319,45.09353609)(168.51506165,45.14353811)
\curveto(168.62506301,45.22353596)(168.72006291,45.32853585)(168.80006165,45.45853811)
\curveto(168.87006276,45.58853559)(168.9300627,45.72853545)(168.98006165,45.87853811)
\curveto(168.99006264,45.92853525)(168.99506264,45.9785352)(168.99506165,46.02853811)
\curveto(168.99506264,46.0785351)(169.00006263,46.12853505)(169.01006165,46.17853811)
\curveto(169.0300626,46.24853493)(169.04506259,46.33353485)(169.05506165,46.43353811)
\curveto(169.05506258,46.54353464)(169.04506259,46.63353455)(169.02506165,46.70353811)
\curveto(169.00506263,46.76353442)(169.00006263,46.82353436)(169.01006165,46.88353811)
\curveto(169.01006262,46.94353424)(169.00006263,47.00353418)(168.98006165,47.06353811)
\curveto(168.96006267,47.14353404)(168.94506269,47.21853396)(168.93506165,47.28853811)
\curveto(168.92506271,47.36853381)(168.90506273,47.44353374)(168.87506165,47.51353811)
\curveto(168.75506288,47.80353338)(168.61006302,48.04853313)(168.44006165,48.24853811)
\curveto(168.27006336,48.45853272)(168.04006359,48.61853256)(167.75006165,48.72853811)
}
}
{
\newrgbcolor{curcolor}{0 0 0}
\pscustom[linestyle=none,fillstyle=solid,fillcolor=curcolor]
{
\newpath
\moveto(162.06506165,55.69017873)
\curveto(162.06506957,55.92017394)(162.12506951,56.05017381)(162.24506165,56.08017873)
\curveto(162.35506928,56.11017375)(162.52006911,56.12517374)(162.74006165,56.12517873)
\lineto(163.02506165,56.12517873)
\curveto(163.11506852,56.12517374)(163.19006844,56.10017376)(163.25006165,56.05017873)
\curveto(163.3300683,55.99017387)(163.37506826,55.90517396)(163.38506165,55.79517873)
\curveto(163.38506825,55.68517418)(163.40006823,55.57517429)(163.43006165,55.46517873)
\curveto(163.46006817,55.32517454)(163.49006814,55.19017467)(163.52006165,55.06017873)
\curveto(163.55006808,54.94017492)(163.59006804,54.82517504)(163.64006165,54.71517873)
\curveto(163.77006786,54.42517544)(163.95006768,54.19017567)(164.18006165,54.01017873)
\curveto(164.40006723,53.83017603)(164.65506698,53.67517619)(164.94506165,53.54517873)
\curveto(165.05506658,53.50517636)(165.17006646,53.47517639)(165.29006165,53.45517873)
\curveto(165.40006623,53.43517643)(165.51506612,53.41017645)(165.63506165,53.38017873)
\curveto(165.68506595,53.37017649)(165.7350659,53.3651765)(165.78506165,53.36517873)
\curveto(165.8350658,53.37517649)(165.88506575,53.37517649)(165.93506165,53.36517873)
\curveto(166.05506558,53.33517653)(166.19506544,53.32017654)(166.35506165,53.32017873)
\curveto(166.50506513,53.33017653)(166.65006498,53.33517653)(166.79006165,53.33517873)
\lineto(168.63506165,53.33517873)
\lineto(168.98006165,53.33517873)
\curveto(169.10006253,53.33517653)(169.21506242,53.33017653)(169.32506165,53.32017873)
\curveto(169.4350622,53.31017655)(169.5300621,53.30517656)(169.61006165,53.30517873)
\curveto(169.69006194,53.31517655)(169.76006187,53.29517657)(169.82006165,53.24517873)
\curveto(169.89006174,53.19517667)(169.9300617,53.11517675)(169.94006165,53.00517873)
\curveto(169.95006168,52.90517696)(169.95506168,52.79517707)(169.95506165,52.67517873)
\lineto(169.95506165,52.40517873)
\curveto(169.9350617,52.35517751)(169.92006171,52.30517756)(169.91006165,52.25517873)
\curveto(169.89006174,52.21517765)(169.86506177,52.18517768)(169.83506165,52.16517873)
\curveto(169.76506187,52.11517775)(169.68006195,52.08517778)(169.58006165,52.07517873)
\lineto(169.25006165,52.07517873)
\lineto(168.09506165,52.07517873)
\lineto(163.94006165,52.07517873)
\lineto(162.90506165,52.07517873)
\lineto(162.60506165,52.07517873)
\curveto(162.50506913,52.08517778)(162.42006921,52.11517775)(162.35006165,52.16517873)
\curveto(162.31006932,52.19517767)(162.28006935,52.24517762)(162.26006165,52.31517873)
\curveto(162.24006939,52.39517747)(162.2300694,52.48017738)(162.23006165,52.57017873)
\curveto(162.22006941,52.6601772)(162.22006941,52.75017711)(162.23006165,52.84017873)
\curveto(162.24006939,52.93017693)(162.25506938,53.00017686)(162.27506165,53.05017873)
\curveto(162.30506933,53.13017673)(162.36506927,53.18017668)(162.45506165,53.20017873)
\curveto(162.5350691,53.23017663)(162.62506901,53.24517662)(162.72506165,53.24517873)
\lineto(163.02506165,53.24517873)
\curveto(163.12506851,53.24517662)(163.21506842,53.2651766)(163.29506165,53.30517873)
\curveto(163.31506832,53.31517655)(163.3300683,53.32517654)(163.34006165,53.33517873)
\lineto(163.38506165,53.38017873)
\curveto(163.38506825,53.49017637)(163.34006829,53.58017628)(163.25006165,53.65017873)
\curveto(163.15006848,53.72017614)(163.07006856,53.78017608)(163.01006165,53.83017873)
\lineto(162.92006165,53.92017873)
\curveto(162.81006882,54.01017585)(162.69506894,54.13517573)(162.57506165,54.29517873)
\curveto(162.45506918,54.45517541)(162.36506927,54.60517526)(162.30506165,54.74517873)
\curveto(162.25506938,54.83517503)(162.22006941,54.93017493)(162.20006165,55.03017873)
\curveto(162.17006946,55.13017473)(162.14006949,55.23517463)(162.11006165,55.34517873)
\curveto(162.10006953,55.40517446)(162.09506954,55.4651744)(162.09506165,55.52517873)
\curveto(162.08506955,55.58517428)(162.07506956,55.64017422)(162.06506165,55.69017873)
}
}
{
\newrgbcolor{curcolor}{0 0 0}
\pscustom[linestyle=none,fillstyle=solid,fillcolor=curcolor]
{
}
}
{
\newrgbcolor{curcolor}{0 0 0}
\pscustom[linestyle=none,fillstyle=solid,fillcolor=curcolor]
{
\newpath
\moveto(159.36506165,65.05510061)
\curveto(159.36507227,65.15509575)(159.37507226,65.25009566)(159.39506165,65.34010061)
\curveto(159.40507223,65.43009548)(159.4350722,65.49509541)(159.48506165,65.53510061)
\curveto(159.56507207,65.59509531)(159.67007196,65.62509528)(159.80006165,65.62510061)
\lineto(160.19006165,65.62510061)
\lineto(161.69006165,65.62510061)
\lineto(168.08006165,65.62510061)
\lineto(169.25006165,65.62510061)
\lineto(169.56506165,65.62510061)
\curveto(169.66506197,65.63509527)(169.74506189,65.62009529)(169.80506165,65.58010061)
\curveto(169.88506175,65.53009538)(169.9350617,65.45509545)(169.95506165,65.35510061)
\curveto(169.96506167,65.26509564)(169.97006166,65.15509575)(169.97006165,65.02510061)
\lineto(169.97006165,64.80010061)
\curveto(169.95006168,64.72009619)(169.9350617,64.65009626)(169.92506165,64.59010061)
\curveto(169.90506173,64.53009638)(169.86506177,64.48009643)(169.80506165,64.44010061)
\curveto(169.74506189,64.40009651)(169.67006196,64.38009653)(169.58006165,64.38010061)
\lineto(169.28006165,64.38010061)
\lineto(168.18506165,64.38010061)
\lineto(162.84506165,64.38010061)
\curveto(162.75506888,64.36009655)(162.68006895,64.34509656)(162.62006165,64.33510061)
\curveto(162.55006908,64.33509657)(162.49006914,64.3050966)(162.44006165,64.24510061)
\curveto(162.39006924,64.17509673)(162.36506927,64.08509682)(162.36506165,63.97510061)
\curveto(162.35506928,63.87509703)(162.35006928,63.76509714)(162.35006165,63.64510061)
\lineto(162.35006165,62.50510061)
\lineto(162.35006165,62.01010061)
\curveto(162.34006929,61.85009906)(162.28006935,61.74009917)(162.17006165,61.68010061)
\curveto(162.14006949,61.66009925)(162.11006952,61.65009926)(162.08006165,61.65010061)
\curveto(162.04006959,61.65009926)(161.99506964,61.64509926)(161.94506165,61.63510061)
\curveto(161.82506981,61.61509929)(161.71506992,61.62009929)(161.61506165,61.65010061)
\curveto(161.51507012,61.69009922)(161.44507019,61.74509916)(161.40506165,61.81510061)
\curveto(161.35507028,61.89509901)(161.3300703,62.01509889)(161.33006165,62.17510061)
\curveto(161.3300703,62.33509857)(161.31507032,62.47009844)(161.28506165,62.58010061)
\curveto(161.27507036,62.63009828)(161.27007036,62.68509822)(161.27006165,62.74510061)
\curveto(161.26007037,62.8050981)(161.24507039,62.86509804)(161.22506165,62.92510061)
\curveto(161.17507046,63.07509783)(161.12507051,63.22009769)(161.07506165,63.36010061)
\curveto(161.01507062,63.50009741)(160.94507069,63.63509727)(160.86506165,63.76510061)
\curveto(160.77507086,63.905097)(160.67007096,64.02509688)(160.55006165,64.12510061)
\curveto(160.4300712,64.22509668)(160.30007133,64.32009659)(160.16006165,64.41010061)
\curveto(160.06007157,64.47009644)(159.95007168,64.51509639)(159.83006165,64.54510061)
\curveto(159.71007192,64.58509632)(159.60507203,64.63509627)(159.51506165,64.69510061)
\curveto(159.45507218,64.74509616)(159.41507222,64.81509609)(159.39506165,64.90510061)
\curveto(159.38507225,64.92509598)(159.38007225,64.95009596)(159.38006165,64.98010061)
\curveto(159.38007225,65.0100959)(159.37507226,65.03509587)(159.36506165,65.05510061)
}
}
{
\newrgbcolor{curcolor}{0 0 0}
\pscustom[linestyle=none,fillstyle=solid,fillcolor=curcolor]
{
\newpath
\moveto(164.88506165,76.34470998)
\lineto(165.14006165,76.34470998)
\curveto(165.22006641,76.35470228)(165.29506634,76.34970228)(165.36506165,76.32970998)
\lineto(165.60506165,76.32970998)
\lineto(165.77006165,76.32970998)
\curveto(165.87006576,76.30970232)(165.97506566,76.29970233)(166.08506165,76.29970998)
\curveto(166.18506545,76.29970233)(166.28506535,76.28970234)(166.38506165,76.26970998)
\lineto(166.53506165,76.26970998)
\curveto(166.67506496,76.23970239)(166.81506482,76.21970241)(166.95506165,76.20970998)
\curveto(167.08506455,76.19970243)(167.21506442,76.17470246)(167.34506165,76.13470998)
\curveto(167.42506421,76.11470252)(167.51006412,76.09470254)(167.60006165,76.07470998)
\lineto(167.84006165,76.01470998)
\lineto(168.14006165,75.89470998)
\curveto(168.2300634,75.86470277)(168.32006331,75.8297028)(168.41006165,75.78970998)
\curveto(168.630063,75.68970294)(168.84506279,75.55470308)(169.05506165,75.38470998)
\curveto(169.26506237,75.22470341)(169.4350622,75.04970358)(169.56506165,74.85970998)
\curveto(169.60506203,74.80970382)(169.64506199,74.74970388)(169.68506165,74.67970998)
\curveto(169.71506192,74.61970401)(169.75006188,74.55970407)(169.79006165,74.49970998)
\curveto(169.84006179,74.41970421)(169.88006175,74.32470431)(169.91006165,74.21470998)
\curveto(169.94006169,74.10470453)(169.97006166,73.99970463)(170.00006165,73.89970998)
\curveto(170.04006159,73.78970484)(170.06506157,73.67970495)(170.07506165,73.56970998)
\curveto(170.08506155,73.45970517)(170.10006153,73.34470529)(170.12006165,73.22470998)
\curveto(170.1300615,73.18470545)(170.1300615,73.13970549)(170.12006165,73.08970998)
\curveto(170.12006151,73.04970558)(170.12506151,73.00970562)(170.13506165,72.96970998)
\curveto(170.14506149,72.9297057)(170.15006148,72.87470576)(170.15006165,72.80470998)
\curveto(170.15006148,72.7347059)(170.14506149,72.68470595)(170.13506165,72.65470998)
\curveto(170.11506152,72.60470603)(170.11006152,72.55970607)(170.12006165,72.51970998)
\curveto(170.1300615,72.47970615)(170.1300615,72.44470619)(170.12006165,72.41470998)
\lineto(170.12006165,72.32470998)
\curveto(170.10006153,72.26470637)(170.08506155,72.19970643)(170.07506165,72.12970998)
\curveto(170.07506156,72.06970656)(170.07006156,72.00470663)(170.06006165,71.93470998)
\curveto(170.01006162,71.76470687)(169.96006167,71.60470703)(169.91006165,71.45470998)
\curveto(169.86006177,71.30470733)(169.79506184,71.15970747)(169.71506165,71.01970998)
\curveto(169.67506196,70.96970766)(169.64506199,70.91470772)(169.62506165,70.85470998)
\curveto(169.59506204,70.80470783)(169.56006207,70.75470788)(169.52006165,70.70470998)
\curveto(169.34006229,70.46470817)(169.12006251,70.26470837)(168.86006165,70.10470998)
\curveto(168.60006303,69.94470869)(168.31506332,69.80470883)(168.00506165,69.68470998)
\curveto(167.86506377,69.62470901)(167.72506391,69.57970905)(167.58506165,69.54970998)
\curveto(167.4350642,69.51970911)(167.28006435,69.48470915)(167.12006165,69.44470998)
\curveto(167.01006462,69.42470921)(166.90006473,69.40970922)(166.79006165,69.39970998)
\curveto(166.68006495,69.38970924)(166.57006506,69.37470926)(166.46006165,69.35470998)
\curveto(166.42006521,69.34470929)(166.38006525,69.33970929)(166.34006165,69.33970998)
\curveto(166.30006533,69.34970928)(166.26006537,69.34970928)(166.22006165,69.33970998)
\curveto(166.17006546,69.3297093)(166.12006551,69.32470931)(166.07006165,69.32470998)
\lineto(165.90506165,69.32470998)
\curveto(165.85506578,69.30470933)(165.80506583,69.29970933)(165.75506165,69.30970998)
\curveto(165.69506594,69.31970931)(165.64006599,69.31970931)(165.59006165,69.30970998)
\curveto(165.55006608,69.29970933)(165.50506613,69.29970933)(165.45506165,69.30970998)
\curveto(165.40506623,69.31970931)(165.35506628,69.31470932)(165.30506165,69.29470998)
\curveto(165.2350664,69.27470936)(165.16006647,69.26970936)(165.08006165,69.27970998)
\curveto(164.99006664,69.28970934)(164.90506673,69.29470934)(164.82506165,69.29470998)
\curveto(164.7350669,69.29470934)(164.635067,69.28970934)(164.52506165,69.27970998)
\curveto(164.40506723,69.26970936)(164.30506733,69.27470936)(164.22506165,69.29470998)
\lineto(163.94006165,69.29470998)
\lineto(163.31006165,69.33970998)
\curveto(163.21006842,69.34970928)(163.11506852,69.35970927)(163.02506165,69.36970998)
\lineto(162.72506165,69.39970998)
\curveto(162.67506896,69.41970921)(162.62506901,69.42470921)(162.57506165,69.41470998)
\curveto(162.51506912,69.41470922)(162.46006917,69.42470921)(162.41006165,69.44470998)
\curveto(162.24006939,69.49470914)(162.07506956,69.5347091)(161.91506165,69.56470998)
\curveto(161.74506989,69.59470904)(161.58507005,69.64470899)(161.43506165,69.71470998)
\curveto(160.97507066,69.90470873)(160.60007103,70.12470851)(160.31006165,70.37470998)
\curveto(160.02007161,70.634708)(159.77507186,70.99470764)(159.57506165,71.45470998)
\curveto(159.52507211,71.58470705)(159.49007214,71.71470692)(159.47006165,71.84470998)
\curveto(159.45007218,71.98470665)(159.42507221,72.12470651)(159.39506165,72.26470998)
\curveto(159.38507225,72.3347063)(159.38007225,72.39970623)(159.38006165,72.45970998)
\curveto(159.38007225,72.51970611)(159.37507226,72.58470605)(159.36506165,72.65470998)
\curveto(159.34507229,73.48470515)(159.49507214,74.15470448)(159.81506165,74.66470998)
\curveto(160.12507151,75.17470346)(160.56507107,75.55470308)(161.13506165,75.80470998)
\curveto(161.25507038,75.85470278)(161.38007025,75.89970273)(161.51006165,75.93970998)
\curveto(161.64006999,75.97970265)(161.77506986,76.02470261)(161.91506165,76.07470998)
\curveto(161.99506964,76.09470254)(162.08006955,76.10970252)(162.17006165,76.11970998)
\lineto(162.41006165,76.17970998)
\curveto(162.52006911,76.20970242)(162.630069,76.22470241)(162.74006165,76.22470998)
\curveto(162.85006878,76.2347024)(162.96006867,76.24970238)(163.07006165,76.26970998)
\curveto(163.12006851,76.28970234)(163.16506847,76.29470234)(163.20506165,76.28470998)
\curveto(163.24506839,76.28470235)(163.28506835,76.28970234)(163.32506165,76.29970998)
\curveto(163.37506826,76.30970232)(163.4300682,76.30970232)(163.49006165,76.29970998)
\curveto(163.54006809,76.29970233)(163.59006804,76.30470233)(163.64006165,76.31470998)
\lineto(163.77506165,76.31470998)
\curveto(163.8350678,76.3347023)(163.90506773,76.3347023)(163.98506165,76.31470998)
\curveto(164.05506758,76.30470233)(164.12006751,76.30970232)(164.18006165,76.32970998)
\curveto(164.21006742,76.33970229)(164.25006738,76.34470229)(164.30006165,76.34470998)
\lineto(164.42006165,76.34470998)
\lineto(164.88506165,76.34470998)
\moveto(167.21006165,74.79970998)
\curveto(166.89006474,74.89970373)(166.52506511,74.95970367)(166.11506165,74.97970998)
\curveto(165.70506593,74.99970363)(165.29506634,75.00970362)(164.88506165,75.00970998)
\curveto(164.45506718,75.00970362)(164.0350676,74.99970363)(163.62506165,74.97970998)
\curveto(163.21506842,74.95970367)(162.8300688,74.91470372)(162.47006165,74.84470998)
\curveto(162.11006952,74.77470386)(161.79006984,74.66470397)(161.51006165,74.51470998)
\curveto(161.22007041,74.37470426)(160.98507065,74.17970445)(160.80506165,73.92970998)
\curveto(160.69507094,73.76970486)(160.61507102,73.58970504)(160.56506165,73.38970998)
\curveto(160.50507113,73.18970544)(160.47507116,72.94470569)(160.47506165,72.65470998)
\curveto(160.49507114,72.634706)(160.50507113,72.59970603)(160.50506165,72.54970998)
\curveto(160.49507114,72.49970613)(160.49507114,72.45970617)(160.50506165,72.42970998)
\curveto(160.52507111,72.34970628)(160.54507109,72.27470636)(160.56506165,72.20470998)
\curveto(160.57507106,72.14470649)(160.59507104,72.07970655)(160.62506165,72.00970998)
\curveto(160.74507089,71.73970689)(160.91507072,71.51970711)(161.13506165,71.34970998)
\curveto(161.34507029,71.18970744)(161.59007004,71.05470758)(161.87006165,70.94470998)
\curveto(161.98006965,70.89470774)(162.10006953,70.85470778)(162.23006165,70.82470998)
\curveto(162.35006928,70.80470783)(162.47506916,70.77970785)(162.60506165,70.74970998)
\curveto(162.65506898,70.7297079)(162.71006892,70.71970791)(162.77006165,70.71970998)
\curveto(162.82006881,70.71970791)(162.87006876,70.71470792)(162.92006165,70.70470998)
\curveto(163.01006862,70.69470794)(163.10506853,70.68470795)(163.20506165,70.67470998)
\curveto(163.29506834,70.66470797)(163.39006824,70.65470798)(163.49006165,70.64470998)
\curveto(163.57006806,70.64470799)(163.65506798,70.63970799)(163.74506165,70.62970998)
\lineto(163.98506165,70.62970998)
\lineto(164.16506165,70.62970998)
\curveto(164.19506744,70.61970801)(164.2300674,70.61470802)(164.27006165,70.61470998)
\lineto(164.40506165,70.61470998)
\lineto(164.85506165,70.61470998)
\curveto(164.9350667,70.61470802)(165.02006661,70.60970802)(165.11006165,70.59970998)
\curveto(165.19006644,70.59970803)(165.26506637,70.60970802)(165.33506165,70.62970998)
\lineto(165.60506165,70.62970998)
\curveto(165.62506601,70.629708)(165.65506598,70.62470801)(165.69506165,70.61470998)
\curveto(165.72506591,70.61470802)(165.75006588,70.61970801)(165.77006165,70.62970998)
\curveto(165.87006576,70.63970799)(165.97006566,70.64470799)(166.07006165,70.64470998)
\curveto(166.16006547,70.65470798)(166.26006537,70.66470797)(166.37006165,70.67470998)
\curveto(166.49006514,70.70470793)(166.61506502,70.71970791)(166.74506165,70.71970998)
\curveto(166.86506477,70.7297079)(166.98006465,70.75470788)(167.09006165,70.79470998)
\curveto(167.39006424,70.87470776)(167.65506398,70.95970767)(167.88506165,71.04970998)
\curveto(168.11506352,71.14970748)(168.3300633,71.29470734)(168.53006165,71.48470998)
\curveto(168.7300629,71.69470694)(168.88006275,71.95970667)(168.98006165,72.27970998)
\curveto(169.00006263,72.31970631)(169.01006262,72.35470628)(169.01006165,72.38470998)
\curveto(169.00006263,72.42470621)(169.00506263,72.46970616)(169.02506165,72.51970998)
\curveto(169.0350626,72.55970607)(169.04506259,72.629706)(169.05506165,72.72970998)
\curveto(169.06506257,72.83970579)(169.06006257,72.92470571)(169.04006165,72.98470998)
\curveto(169.02006261,73.05470558)(169.01006262,73.12470551)(169.01006165,73.19470998)
\curveto(169.00006263,73.26470537)(168.98506265,73.3297053)(168.96506165,73.38970998)
\curveto(168.90506273,73.58970504)(168.82006281,73.76970486)(168.71006165,73.92970998)
\curveto(168.69006294,73.95970467)(168.67006296,73.98470465)(168.65006165,74.00470998)
\lineto(168.59006165,74.06470998)
\curveto(168.57006306,74.10470453)(168.5300631,74.15470448)(168.47006165,74.21470998)
\curveto(168.3300633,74.31470432)(168.20006343,74.39970423)(168.08006165,74.46970998)
\curveto(167.96006367,74.53970409)(167.81506382,74.60970402)(167.64506165,74.67970998)
\curveto(167.57506406,74.70970392)(167.50506413,74.7297039)(167.43506165,74.73970998)
\curveto(167.36506427,74.75970387)(167.29006434,74.77970385)(167.21006165,74.79970998)
}
}
{
\newrgbcolor{curcolor}{0 0 0}
\pscustom[linestyle=none,fillstyle=solid,fillcolor=curcolor]
{
\newpath
\moveto(168.33506165,78.63431936)
\lineto(168.33506165,79.26431936)
\lineto(168.33506165,79.45931936)
\curveto(168.3350633,79.52931683)(168.34506329,79.58931677)(168.36506165,79.63931936)
\curveto(168.40506323,79.70931665)(168.44506319,79.7593166)(168.48506165,79.78931936)
\curveto(168.5350631,79.82931653)(168.60006303,79.84931651)(168.68006165,79.84931936)
\curveto(168.76006287,79.8593165)(168.84506279,79.86431649)(168.93506165,79.86431936)
\lineto(169.65506165,79.86431936)
\curveto(170.1350615,79.86431649)(170.54506109,79.80431655)(170.88506165,79.68431936)
\curveto(171.22506041,79.56431679)(171.50006013,79.36931699)(171.71006165,79.09931936)
\curveto(171.76005987,79.02931733)(171.80505983,78.9593174)(171.84506165,78.88931936)
\curveto(171.89505974,78.82931753)(171.94005969,78.7543176)(171.98006165,78.66431936)
\curveto(171.99005964,78.64431771)(172.00005963,78.61431774)(172.01006165,78.57431936)
\curveto(172.0300596,78.53431782)(172.0350596,78.48931787)(172.02506165,78.43931936)
\curveto(171.99505964,78.34931801)(171.92005971,78.29431806)(171.80006165,78.27431936)
\curveto(171.69005994,78.2543181)(171.59506004,78.26931809)(171.51506165,78.31931936)
\curveto(171.44506019,78.34931801)(171.38006025,78.39431796)(171.32006165,78.45431936)
\curveto(171.27006036,78.52431783)(171.22006041,78.58931777)(171.17006165,78.64931936)
\curveto(171.12006051,78.71931764)(171.04506059,78.77931758)(170.94506165,78.82931936)
\curveto(170.85506078,78.88931747)(170.76506087,78.93931742)(170.67506165,78.97931936)
\curveto(170.64506099,78.99931736)(170.58506105,79.02431733)(170.49506165,79.05431936)
\curveto(170.41506122,79.08431727)(170.34506129,79.08931727)(170.28506165,79.06931936)
\curveto(170.14506149,79.03931732)(170.05506158,78.97931738)(170.01506165,78.88931936)
\curveto(169.98506165,78.80931755)(169.97006166,78.71931764)(169.97006165,78.61931936)
\curveto(169.97006166,78.51931784)(169.94506169,78.43431792)(169.89506165,78.36431936)
\curveto(169.82506181,78.27431808)(169.68506195,78.22931813)(169.47506165,78.22931936)
\lineto(168.92006165,78.22931936)
\lineto(168.69506165,78.22931936)
\curveto(168.61506302,78.23931812)(168.55006308,78.2593181)(168.50006165,78.28931936)
\curveto(168.42006321,78.34931801)(168.37506326,78.41931794)(168.36506165,78.49931936)
\curveto(168.35506328,78.51931784)(168.35006328,78.53931782)(168.35006165,78.55931936)
\curveto(168.35006328,78.58931777)(168.34506329,78.61431774)(168.33506165,78.63431936)
}
}
{
\newrgbcolor{curcolor}{0 0 0}
\pscustom[linestyle=none,fillstyle=solid,fillcolor=curcolor]
{
}
}
{
\newrgbcolor{curcolor}{0 0 0}
\pscustom[linestyle=none,fillstyle=solid,fillcolor=curcolor]
{
\newpath
\moveto(159.36506165,89.26463186)
\curveto(159.35507228,89.95462722)(159.47507216,90.55462662)(159.72506165,91.06463186)
\curveto(159.97507166,91.58462559)(160.31007132,91.9796252)(160.73006165,92.24963186)
\curveto(160.81007082,92.29962488)(160.90007073,92.34462483)(161.00006165,92.38463186)
\curveto(161.09007054,92.42462475)(161.18507045,92.46962471)(161.28506165,92.51963186)
\curveto(161.38507025,92.55962462)(161.48507015,92.58962459)(161.58506165,92.60963186)
\curveto(161.68506995,92.62962455)(161.79006984,92.64962453)(161.90006165,92.66963186)
\curveto(161.95006968,92.68962449)(161.99506964,92.69462448)(162.03506165,92.68463186)
\curveto(162.07506956,92.6746245)(162.12006951,92.6796245)(162.17006165,92.69963186)
\curveto(162.22006941,92.70962447)(162.30506933,92.71462446)(162.42506165,92.71463186)
\curveto(162.5350691,92.71462446)(162.62006901,92.70962447)(162.68006165,92.69963186)
\curveto(162.74006889,92.6796245)(162.80006883,92.66962451)(162.86006165,92.66963186)
\curveto(162.92006871,92.6796245)(162.98006865,92.6746245)(163.04006165,92.65463186)
\curveto(163.18006845,92.61462456)(163.31506832,92.5796246)(163.44506165,92.54963186)
\curveto(163.57506806,92.51962466)(163.70006793,92.4796247)(163.82006165,92.42963186)
\curveto(163.96006767,92.36962481)(164.08506755,92.29962488)(164.19506165,92.21963186)
\curveto(164.30506733,92.14962503)(164.41506722,92.0746251)(164.52506165,91.99463186)
\lineto(164.58506165,91.93463186)
\curveto(164.60506703,91.92462525)(164.62506701,91.90962527)(164.64506165,91.88963186)
\curveto(164.80506683,91.76962541)(164.95006668,91.63462554)(165.08006165,91.48463186)
\curveto(165.21006642,91.33462584)(165.3350663,91.174626)(165.45506165,91.00463186)
\curveto(165.67506596,90.69462648)(165.88006575,90.39962678)(166.07006165,90.11963186)
\curveto(166.21006542,89.88962729)(166.34506529,89.65962752)(166.47506165,89.42963186)
\curveto(166.60506503,89.20962797)(166.74006489,88.98962819)(166.88006165,88.76963186)
\curveto(167.05006458,88.51962866)(167.2300644,88.2796289)(167.42006165,88.04963186)
\curveto(167.61006402,87.82962935)(167.8350638,87.63962954)(168.09506165,87.47963186)
\curveto(168.15506348,87.43962974)(168.21506342,87.40462977)(168.27506165,87.37463186)
\curveto(168.32506331,87.34462983)(168.39006324,87.31462986)(168.47006165,87.28463186)
\curveto(168.54006309,87.26462991)(168.60006303,87.25962992)(168.65006165,87.26963186)
\curveto(168.72006291,87.28962989)(168.77506286,87.32462985)(168.81506165,87.37463186)
\curveto(168.84506279,87.42462975)(168.86506277,87.48462969)(168.87506165,87.55463186)
\lineto(168.87506165,87.79463186)
\lineto(168.87506165,88.54463186)
\lineto(168.87506165,91.34963186)
\lineto(168.87506165,92.00963186)
\curveto(168.87506276,92.09962508)(168.88006275,92.18462499)(168.89006165,92.26463186)
\curveto(168.89006274,92.34462483)(168.91006272,92.40962477)(168.95006165,92.45963186)
\curveto(168.99006264,92.50962467)(169.06506257,92.54962463)(169.17506165,92.57963186)
\curveto(169.27506236,92.61962456)(169.37506226,92.62962455)(169.47506165,92.60963186)
\lineto(169.61006165,92.60963186)
\curveto(169.68006195,92.58962459)(169.74006189,92.56962461)(169.79006165,92.54963186)
\curveto(169.84006179,92.52962465)(169.88006175,92.49462468)(169.91006165,92.44463186)
\curveto(169.95006168,92.39462478)(169.97006166,92.32462485)(169.97006165,92.23463186)
\lineto(169.97006165,91.96463186)
\lineto(169.97006165,91.06463186)
\lineto(169.97006165,87.55463186)
\lineto(169.97006165,86.48963186)
\curveto(169.97006166,86.40963077)(169.97506166,86.31963086)(169.98506165,86.21963186)
\curveto(169.98506165,86.11963106)(169.97506166,86.03463114)(169.95506165,85.96463186)
\curveto(169.88506175,85.75463142)(169.70506193,85.68963149)(169.41506165,85.76963186)
\curveto(169.37506226,85.7796314)(169.34006229,85.7796314)(169.31006165,85.76963186)
\curveto(169.27006236,85.76963141)(169.22506241,85.7796314)(169.17506165,85.79963186)
\curveto(169.09506254,85.81963136)(169.01006262,85.83963134)(168.92006165,85.85963186)
\curveto(168.8300628,85.8796313)(168.74506289,85.90463127)(168.66506165,85.93463186)
\curveto(168.17506346,86.09463108)(167.76006387,86.29463088)(167.42006165,86.53463186)
\curveto(167.17006446,86.71463046)(166.94506469,86.91963026)(166.74506165,87.14963186)
\curveto(166.5350651,87.3796298)(166.34006529,87.61962956)(166.16006165,87.86963186)
\curveto(165.98006565,88.12962905)(165.81006582,88.39462878)(165.65006165,88.66463186)
\curveto(165.48006615,88.94462823)(165.30506633,89.21462796)(165.12506165,89.47463186)
\curveto(165.04506659,89.58462759)(164.97006666,89.68962749)(164.90006165,89.78963186)
\curveto(164.8300668,89.89962728)(164.75506688,90.00962717)(164.67506165,90.11963186)
\curveto(164.64506699,90.15962702)(164.61506702,90.19462698)(164.58506165,90.22463186)
\curveto(164.54506709,90.26462691)(164.51506712,90.30462687)(164.49506165,90.34463186)
\curveto(164.38506725,90.48462669)(164.26006737,90.60962657)(164.12006165,90.71963186)
\curveto(164.09006754,90.73962644)(164.06506757,90.76462641)(164.04506165,90.79463186)
\curveto(164.01506762,90.82462635)(163.98506765,90.84962633)(163.95506165,90.86963186)
\curveto(163.85506778,90.94962623)(163.75506788,91.01462616)(163.65506165,91.06463186)
\curveto(163.55506808,91.12462605)(163.44506819,91.179626)(163.32506165,91.22963186)
\curveto(163.25506838,91.25962592)(163.18006845,91.2796259)(163.10006165,91.28963186)
\lineto(162.86006165,91.34963186)
\lineto(162.77006165,91.34963186)
\curveto(162.74006889,91.35962582)(162.71006892,91.36462581)(162.68006165,91.36463186)
\curveto(162.61006902,91.38462579)(162.51506912,91.38962579)(162.39506165,91.37963186)
\curveto(162.26506937,91.3796258)(162.16506947,91.36962581)(162.09506165,91.34963186)
\curveto(162.01506962,91.32962585)(161.94006969,91.30962587)(161.87006165,91.28963186)
\curveto(161.79006984,91.2796259)(161.71006992,91.25962592)(161.63006165,91.22963186)
\curveto(161.39007024,91.11962606)(161.19007044,90.96962621)(161.03006165,90.77963186)
\curveto(160.86007077,90.59962658)(160.72007091,90.3796268)(160.61006165,90.11963186)
\curveto(160.59007104,90.04962713)(160.57507106,89.9796272)(160.56506165,89.90963186)
\curveto(160.54507109,89.83962734)(160.52507111,89.76462741)(160.50506165,89.68463186)
\curveto(160.48507115,89.60462757)(160.47507116,89.49462768)(160.47506165,89.35463186)
\curveto(160.47507116,89.22462795)(160.48507115,89.11962806)(160.50506165,89.03963186)
\curveto(160.51507112,88.9796282)(160.52007111,88.92462825)(160.52006165,88.87463186)
\curveto(160.52007111,88.82462835)(160.5300711,88.7746284)(160.55006165,88.72463186)
\curveto(160.59007104,88.62462855)(160.630071,88.52962865)(160.67006165,88.43963186)
\curveto(160.71007092,88.35962882)(160.75507088,88.2796289)(160.80506165,88.19963186)
\curveto(160.82507081,88.16962901)(160.85007078,88.13962904)(160.88006165,88.10963186)
\curveto(160.91007072,88.08962909)(160.9350707,88.06462911)(160.95506165,88.03463186)
\lineto(161.03006165,87.95963186)
\curveto(161.05007058,87.92962925)(161.07007056,87.90462927)(161.09006165,87.88463186)
\lineto(161.30006165,87.73463186)
\curveto(161.36007027,87.69462948)(161.42507021,87.64962953)(161.49506165,87.59963186)
\curveto(161.58507005,87.53962964)(161.69006994,87.48962969)(161.81006165,87.44963186)
\curveto(161.92006971,87.41962976)(162.0300696,87.38462979)(162.14006165,87.34463186)
\curveto(162.25006938,87.30462987)(162.39506924,87.2796299)(162.57506165,87.26963186)
\curveto(162.74506889,87.25962992)(162.87006876,87.22962995)(162.95006165,87.17963186)
\curveto(163.0300686,87.12963005)(163.07506856,87.05463012)(163.08506165,86.95463186)
\curveto(163.09506854,86.85463032)(163.10006853,86.74463043)(163.10006165,86.62463186)
\curveto(163.10006853,86.58463059)(163.10506853,86.54463063)(163.11506165,86.50463186)
\curveto(163.11506852,86.46463071)(163.11006852,86.42963075)(163.10006165,86.39963186)
\curveto(163.08006855,86.34963083)(163.07006856,86.29963088)(163.07006165,86.24963186)
\curveto(163.07006856,86.20963097)(163.06006857,86.16963101)(163.04006165,86.12963186)
\curveto(162.98006865,86.03963114)(162.84506879,85.99463118)(162.63506165,85.99463186)
\lineto(162.51506165,85.99463186)
\curveto(162.45506918,86.00463117)(162.39506924,86.00963117)(162.33506165,86.00963186)
\curveto(162.26506937,86.01963116)(162.20006943,86.02963115)(162.14006165,86.03963186)
\curveto(162.0300696,86.05963112)(161.9300697,86.0796311)(161.84006165,86.09963186)
\curveto(161.74006989,86.11963106)(161.64506999,86.14963103)(161.55506165,86.18963186)
\curveto(161.48507015,86.20963097)(161.42507021,86.22963095)(161.37506165,86.24963186)
\lineto(161.19506165,86.30963186)
\curveto(160.9350707,86.42963075)(160.69007094,86.58463059)(160.46006165,86.77463186)
\curveto(160.2300714,86.9746302)(160.04507159,87.18962999)(159.90506165,87.41963186)
\curveto(159.82507181,87.52962965)(159.76007187,87.64462953)(159.71006165,87.76463186)
\lineto(159.56006165,88.15463186)
\curveto(159.51007212,88.26462891)(159.48007215,88.3796288)(159.47006165,88.49963186)
\curveto(159.45007218,88.61962856)(159.42507221,88.74462843)(159.39506165,88.87463186)
\curveto(159.39507224,88.94462823)(159.39507224,89.00962817)(159.39506165,89.06963186)
\curveto(159.38507225,89.12962805)(159.37507226,89.19462798)(159.36506165,89.26463186)
}
}
{
\newrgbcolor{curcolor}{0 0 0}
\pscustom[linestyle=none,fillstyle=solid,fillcolor=curcolor]
{
\newpath
\moveto(164.88506165,101.36424123)
\lineto(165.14006165,101.36424123)
\curveto(165.22006641,101.37423353)(165.29506634,101.36923353)(165.36506165,101.34924123)
\lineto(165.60506165,101.34924123)
\lineto(165.77006165,101.34924123)
\curveto(165.87006576,101.32923357)(165.97506566,101.31923358)(166.08506165,101.31924123)
\curveto(166.18506545,101.31923358)(166.28506535,101.30923359)(166.38506165,101.28924123)
\lineto(166.53506165,101.28924123)
\curveto(166.67506496,101.25923364)(166.81506482,101.23923366)(166.95506165,101.22924123)
\curveto(167.08506455,101.21923368)(167.21506442,101.19423371)(167.34506165,101.15424123)
\curveto(167.42506421,101.13423377)(167.51006412,101.11423379)(167.60006165,101.09424123)
\lineto(167.84006165,101.03424123)
\lineto(168.14006165,100.91424123)
\curveto(168.2300634,100.88423402)(168.32006331,100.84923405)(168.41006165,100.80924123)
\curveto(168.630063,100.70923419)(168.84506279,100.57423433)(169.05506165,100.40424123)
\curveto(169.26506237,100.24423466)(169.4350622,100.06923483)(169.56506165,99.87924123)
\curveto(169.60506203,99.82923507)(169.64506199,99.76923513)(169.68506165,99.69924123)
\curveto(169.71506192,99.63923526)(169.75006188,99.57923532)(169.79006165,99.51924123)
\curveto(169.84006179,99.43923546)(169.88006175,99.34423556)(169.91006165,99.23424123)
\curveto(169.94006169,99.12423578)(169.97006166,99.01923588)(170.00006165,98.91924123)
\curveto(170.04006159,98.80923609)(170.06506157,98.6992362)(170.07506165,98.58924123)
\curveto(170.08506155,98.47923642)(170.10006153,98.36423654)(170.12006165,98.24424123)
\curveto(170.1300615,98.2042367)(170.1300615,98.15923674)(170.12006165,98.10924123)
\curveto(170.12006151,98.06923683)(170.12506151,98.02923687)(170.13506165,97.98924123)
\curveto(170.14506149,97.94923695)(170.15006148,97.89423701)(170.15006165,97.82424123)
\curveto(170.15006148,97.75423715)(170.14506149,97.7042372)(170.13506165,97.67424123)
\curveto(170.11506152,97.62423728)(170.11006152,97.57923732)(170.12006165,97.53924123)
\curveto(170.1300615,97.4992374)(170.1300615,97.46423744)(170.12006165,97.43424123)
\lineto(170.12006165,97.34424123)
\curveto(170.10006153,97.28423762)(170.08506155,97.21923768)(170.07506165,97.14924123)
\curveto(170.07506156,97.08923781)(170.07006156,97.02423788)(170.06006165,96.95424123)
\curveto(170.01006162,96.78423812)(169.96006167,96.62423828)(169.91006165,96.47424123)
\curveto(169.86006177,96.32423858)(169.79506184,96.17923872)(169.71506165,96.03924123)
\curveto(169.67506196,95.98923891)(169.64506199,95.93423897)(169.62506165,95.87424123)
\curveto(169.59506204,95.82423908)(169.56006207,95.77423913)(169.52006165,95.72424123)
\curveto(169.34006229,95.48423942)(169.12006251,95.28423962)(168.86006165,95.12424123)
\curveto(168.60006303,94.96423994)(168.31506332,94.82424008)(168.00506165,94.70424123)
\curveto(167.86506377,94.64424026)(167.72506391,94.5992403)(167.58506165,94.56924123)
\curveto(167.4350642,94.53924036)(167.28006435,94.5042404)(167.12006165,94.46424123)
\curveto(167.01006462,94.44424046)(166.90006473,94.42924047)(166.79006165,94.41924123)
\curveto(166.68006495,94.40924049)(166.57006506,94.39424051)(166.46006165,94.37424123)
\curveto(166.42006521,94.36424054)(166.38006525,94.35924054)(166.34006165,94.35924123)
\curveto(166.30006533,94.36924053)(166.26006537,94.36924053)(166.22006165,94.35924123)
\curveto(166.17006546,94.34924055)(166.12006551,94.34424056)(166.07006165,94.34424123)
\lineto(165.90506165,94.34424123)
\curveto(165.85506578,94.32424058)(165.80506583,94.31924058)(165.75506165,94.32924123)
\curveto(165.69506594,94.33924056)(165.64006599,94.33924056)(165.59006165,94.32924123)
\curveto(165.55006608,94.31924058)(165.50506613,94.31924058)(165.45506165,94.32924123)
\curveto(165.40506623,94.33924056)(165.35506628,94.33424057)(165.30506165,94.31424123)
\curveto(165.2350664,94.29424061)(165.16006647,94.28924061)(165.08006165,94.29924123)
\curveto(164.99006664,94.30924059)(164.90506673,94.31424059)(164.82506165,94.31424123)
\curveto(164.7350669,94.31424059)(164.635067,94.30924059)(164.52506165,94.29924123)
\curveto(164.40506723,94.28924061)(164.30506733,94.29424061)(164.22506165,94.31424123)
\lineto(163.94006165,94.31424123)
\lineto(163.31006165,94.35924123)
\curveto(163.21006842,94.36924053)(163.11506852,94.37924052)(163.02506165,94.38924123)
\lineto(162.72506165,94.41924123)
\curveto(162.67506896,94.43924046)(162.62506901,94.44424046)(162.57506165,94.43424123)
\curveto(162.51506912,94.43424047)(162.46006917,94.44424046)(162.41006165,94.46424123)
\curveto(162.24006939,94.51424039)(162.07506956,94.55424035)(161.91506165,94.58424123)
\curveto(161.74506989,94.61424029)(161.58507005,94.66424024)(161.43506165,94.73424123)
\curveto(160.97507066,94.92423998)(160.60007103,95.14423976)(160.31006165,95.39424123)
\curveto(160.02007161,95.65423925)(159.77507186,96.01423889)(159.57506165,96.47424123)
\curveto(159.52507211,96.6042383)(159.49007214,96.73423817)(159.47006165,96.86424123)
\curveto(159.45007218,97.0042379)(159.42507221,97.14423776)(159.39506165,97.28424123)
\curveto(159.38507225,97.35423755)(159.38007225,97.41923748)(159.38006165,97.47924123)
\curveto(159.38007225,97.53923736)(159.37507226,97.6042373)(159.36506165,97.67424123)
\curveto(159.34507229,98.5042364)(159.49507214,99.17423573)(159.81506165,99.68424123)
\curveto(160.12507151,100.19423471)(160.56507107,100.57423433)(161.13506165,100.82424123)
\curveto(161.25507038,100.87423403)(161.38007025,100.91923398)(161.51006165,100.95924123)
\curveto(161.64006999,100.9992339)(161.77506986,101.04423386)(161.91506165,101.09424123)
\curveto(161.99506964,101.11423379)(162.08006955,101.12923377)(162.17006165,101.13924123)
\lineto(162.41006165,101.19924123)
\curveto(162.52006911,101.22923367)(162.630069,101.24423366)(162.74006165,101.24424123)
\curveto(162.85006878,101.25423365)(162.96006867,101.26923363)(163.07006165,101.28924123)
\curveto(163.12006851,101.30923359)(163.16506847,101.31423359)(163.20506165,101.30424123)
\curveto(163.24506839,101.3042336)(163.28506835,101.30923359)(163.32506165,101.31924123)
\curveto(163.37506826,101.32923357)(163.4300682,101.32923357)(163.49006165,101.31924123)
\curveto(163.54006809,101.31923358)(163.59006804,101.32423358)(163.64006165,101.33424123)
\lineto(163.77506165,101.33424123)
\curveto(163.8350678,101.35423355)(163.90506773,101.35423355)(163.98506165,101.33424123)
\curveto(164.05506758,101.32423358)(164.12006751,101.32923357)(164.18006165,101.34924123)
\curveto(164.21006742,101.35923354)(164.25006738,101.36423354)(164.30006165,101.36424123)
\lineto(164.42006165,101.36424123)
\lineto(164.88506165,101.36424123)
\moveto(167.21006165,99.81924123)
\curveto(166.89006474,99.91923498)(166.52506511,99.97923492)(166.11506165,99.99924123)
\curveto(165.70506593,100.01923488)(165.29506634,100.02923487)(164.88506165,100.02924123)
\curveto(164.45506718,100.02923487)(164.0350676,100.01923488)(163.62506165,99.99924123)
\curveto(163.21506842,99.97923492)(162.8300688,99.93423497)(162.47006165,99.86424123)
\curveto(162.11006952,99.79423511)(161.79006984,99.68423522)(161.51006165,99.53424123)
\curveto(161.22007041,99.39423551)(160.98507065,99.1992357)(160.80506165,98.94924123)
\curveto(160.69507094,98.78923611)(160.61507102,98.60923629)(160.56506165,98.40924123)
\curveto(160.50507113,98.20923669)(160.47507116,97.96423694)(160.47506165,97.67424123)
\curveto(160.49507114,97.65423725)(160.50507113,97.61923728)(160.50506165,97.56924123)
\curveto(160.49507114,97.51923738)(160.49507114,97.47923742)(160.50506165,97.44924123)
\curveto(160.52507111,97.36923753)(160.54507109,97.29423761)(160.56506165,97.22424123)
\curveto(160.57507106,97.16423774)(160.59507104,97.0992378)(160.62506165,97.02924123)
\curveto(160.74507089,96.75923814)(160.91507072,96.53923836)(161.13506165,96.36924123)
\curveto(161.34507029,96.20923869)(161.59007004,96.07423883)(161.87006165,95.96424123)
\curveto(161.98006965,95.91423899)(162.10006953,95.87423903)(162.23006165,95.84424123)
\curveto(162.35006928,95.82423908)(162.47506916,95.7992391)(162.60506165,95.76924123)
\curveto(162.65506898,95.74923915)(162.71006892,95.73923916)(162.77006165,95.73924123)
\curveto(162.82006881,95.73923916)(162.87006876,95.73423917)(162.92006165,95.72424123)
\curveto(163.01006862,95.71423919)(163.10506853,95.7042392)(163.20506165,95.69424123)
\curveto(163.29506834,95.68423922)(163.39006824,95.67423923)(163.49006165,95.66424123)
\curveto(163.57006806,95.66423924)(163.65506798,95.65923924)(163.74506165,95.64924123)
\lineto(163.98506165,95.64924123)
\lineto(164.16506165,95.64924123)
\curveto(164.19506744,95.63923926)(164.2300674,95.63423927)(164.27006165,95.63424123)
\lineto(164.40506165,95.63424123)
\lineto(164.85506165,95.63424123)
\curveto(164.9350667,95.63423927)(165.02006661,95.62923927)(165.11006165,95.61924123)
\curveto(165.19006644,95.61923928)(165.26506637,95.62923927)(165.33506165,95.64924123)
\lineto(165.60506165,95.64924123)
\curveto(165.62506601,95.64923925)(165.65506598,95.64423926)(165.69506165,95.63424123)
\curveto(165.72506591,95.63423927)(165.75006588,95.63923926)(165.77006165,95.64924123)
\curveto(165.87006576,95.65923924)(165.97006566,95.66423924)(166.07006165,95.66424123)
\curveto(166.16006547,95.67423923)(166.26006537,95.68423922)(166.37006165,95.69424123)
\curveto(166.49006514,95.72423918)(166.61506502,95.73923916)(166.74506165,95.73924123)
\curveto(166.86506477,95.74923915)(166.98006465,95.77423913)(167.09006165,95.81424123)
\curveto(167.39006424,95.89423901)(167.65506398,95.97923892)(167.88506165,96.06924123)
\curveto(168.11506352,96.16923873)(168.3300633,96.31423859)(168.53006165,96.50424123)
\curveto(168.7300629,96.71423819)(168.88006275,96.97923792)(168.98006165,97.29924123)
\curveto(169.00006263,97.33923756)(169.01006262,97.37423753)(169.01006165,97.40424123)
\curveto(169.00006263,97.44423746)(169.00506263,97.48923741)(169.02506165,97.53924123)
\curveto(169.0350626,97.57923732)(169.04506259,97.64923725)(169.05506165,97.74924123)
\curveto(169.06506257,97.85923704)(169.06006257,97.94423696)(169.04006165,98.00424123)
\curveto(169.02006261,98.07423683)(169.01006262,98.14423676)(169.01006165,98.21424123)
\curveto(169.00006263,98.28423662)(168.98506265,98.34923655)(168.96506165,98.40924123)
\curveto(168.90506273,98.60923629)(168.82006281,98.78923611)(168.71006165,98.94924123)
\curveto(168.69006294,98.97923592)(168.67006296,99.0042359)(168.65006165,99.02424123)
\lineto(168.59006165,99.08424123)
\curveto(168.57006306,99.12423578)(168.5300631,99.17423573)(168.47006165,99.23424123)
\curveto(168.3300633,99.33423557)(168.20006343,99.41923548)(168.08006165,99.48924123)
\curveto(167.96006367,99.55923534)(167.81506382,99.62923527)(167.64506165,99.69924123)
\curveto(167.57506406,99.72923517)(167.50506413,99.74923515)(167.43506165,99.75924123)
\curveto(167.36506427,99.77923512)(167.29006434,99.7992351)(167.21006165,99.81924123)
}
}
{
\newrgbcolor{curcolor}{0 0 0}
\pscustom[linestyle=none,fillstyle=solid,fillcolor=curcolor]
{
\newpath
\moveto(159.36506165,106.77385061)
\curveto(159.36507227,106.87384575)(159.37507226,106.96884566)(159.39506165,107.05885061)
\curveto(159.40507223,107.14884548)(159.4350722,107.21384541)(159.48506165,107.25385061)
\curveto(159.56507207,107.31384531)(159.67007196,107.34384528)(159.80006165,107.34385061)
\lineto(160.19006165,107.34385061)
\lineto(161.69006165,107.34385061)
\lineto(168.08006165,107.34385061)
\lineto(169.25006165,107.34385061)
\lineto(169.56506165,107.34385061)
\curveto(169.66506197,107.35384527)(169.74506189,107.33884529)(169.80506165,107.29885061)
\curveto(169.88506175,107.24884538)(169.9350617,107.17384545)(169.95506165,107.07385061)
\curveto(169.96506167,106.98384564)(169.97006166,106.87384575)(169.97006165,106.74385061)
\lineto(169.97006165,106.51885061)
\curveto(169.95006168,106.43884619)(169.9350617,106.36884626)(169.92506165,106.30885061)
\curveto(169.90506173,106.24884638)(169.86506177,106.19884643)(169.80506165,106.15885061)
\curveto(169.74506189,106.11884651)(169.67006196,106.09884653)(169.58006165,106.09885061)
\lineto(169.28006165,106.09885061)
\lineto(168.18506165,106.09885061)
\lineto(162.84506165,106.09885061)
\curveto(162.75506888,106.07884655)(162.68006895,106.06384656)(162.62006165,106.05385061)
\curveto(162.55006908,106.05384657)(162.49006914,106.0238466)(162.44006165,105.96385061)
\curveto(162.39006924,105.89384673)(162.36506927,105.80384682)(162.36506165,105.69385061)
\curveto(162.35506928,105.59384703)(162.35006928,105.48384714)(162.35006165,105.36385061)
\lineto(162.35006165,104.22385061)
\lineto(162.35006165,103.72885061)
\curveto(162.34006929,103.56884906)(162.28006935,103.45884917)(162.17006165,103.39885061)
\curveto(162.14006949,103.37884925)(162.11006952,103.36884926)(162.08006165,103.36885061)
\curveto(162.04006959,103.36884926)(161.99506964,103.36384926)(161.94506165,103.35385061)
\curveto(161.82506981,103.33384929)(161.71506992,103.33884929)(161.61506165,103.36885061)
\curveto(161.51507012,103.40884922)(161.44507019,103.46384916)(161.40506165,103.53385061)
\curveto(161.35507028,103.61384901)(161.3300703,103.73384889)(161.33006165,103.89385061)
\curveto(161.3300703,104.05384857)(161.31507032,104.18884844)(161.28506165,104.29885061)
\curveto(161.27507036,104.34884828)(161.27007036,104.40384822)(161.27006165,104.46385061)
\curveto(161.26007037,104.5238481)(161.24507039,104.58384804)(161.22506165,104.64385061)
\curveto(161.17507046,104.79384783)(161.12507051,104.93884769)(161.07506165,105.07885061)
\curveto(161.01507062,105.21884741)(160.94507069,105.35384727)(160.86506165,105.48385061)
\curveto(160.77507086,105.623847)(160.67007096,105.74384688)(160.55006165,105.84385061)
\curveto(160.4300712,105.94384668)(160.30007133,106.03884659)(160.16006165,106.12885061)
\curveto(160.06007157,106.18884644)(159.95007168,106.23384639)(159.83006165,106.26385061)
\curveto(159.71007192,106.30384632)(159.60507203,106.35384627)(159.51506165,106.41385061)
\curveto(159.45507218,106.46384616)(159.41507222,106.53384609)(159.39506165,106.62385061)
\curveto(159.38507225,106.64384598)(159.38007225,106.66884596)(159.38006165,106.69885061)
\curveto(159.38007225,106.7288459)(159.37507226,106.75384587)(159.36506165,106.77385061)
}
}
{
\newrgbcolor{curcolor}{0 0 0}
\pscustom[linestyle=none,fillstyle=solid,fillcolor=curcolor]
{
\newpath
\moveto(159.36506165,115.12345998)
\curveto(159.36507227,115.22345513)(159.37507226,115.31845503)(159.39506165,115.40845998)
\curveto(159.40507223,115.49845485)(159.4350722,115.56345479)(159.48506165,115.60345998)
\curveto(159.56507207,115.66345469)(159.67007196,115.69345466)(159.80006165,115.69345998)
\lineto(160.19006165,115.69345998)
\lineto(161.69006165,115.69345998)
\lineto(168.08006165,115.69345998)
\lineto(169.25006165,115.69345998)
\lineto(169.56506165,115.69345998)
\curveto(169.66506197,115.70345465)(169.74506189,115.68845466)(169.80506165,115.64845998)
\curveto(169.88506175,115.59845475)(169.9350617,115.52345483)(169.95506165,115.42345998)
\curveto(169.96506167,115.33345502)(169.97006166,115.22345513)(169.97006165,115.09345998)
\lineto(169.97006165,114.86845998)
\curveto(169.95006168,114.78845556)(169.9350617,114.71845563)(169.92506165,114.65845998)
\curveto(169.90506173,114.59845575)(169.86506177,114.5484558)(169.80506165,114.50845998)
\curveto(169.74506189,114.46845588)(169.67006196,114.4484559)(169.58006165,114.44845998)
\lineto(169.28006165,114.44845998)
\lineto(168.18506165,114.44845998)
\lineto(162.84506165,114.44845998)
\curveto(162.75506888,114.42845592)(162.68006895,114.41345594)(162.62006165,114.40345998)
\curveto(162.55006908,114.40345595)(162.49006914,114.37345598)(162.44006165,114.31345998)
\curveto(162.39006924,114.24345611)(162.36506927,114.1534562)(162.36506165,114.04345998)
\curveto(162.35506928,113.94345641)(162.35006928,113.83345652)(162.35006165,113.71345998)
\lineto(162.35006165,112.57345998)
\lineto(162.35006165,112.07845998)
\curveto(162.34006929,111.91845843)(162.28006935,111.80845854)(162.17006165,111.74845998)
\curveto(162.14006949,111.72845862)(162.11006952,111.71845863)(162.08006165,111.71845998)
\curveto(162.04006959,111.71845863)(161.99506964,111.71345864)(161.94506165,111.70345998)
\curveto(161.82506981,111.68345867)(161.71506992,111.68845866)(161.61506165,111.71845998)
\curveto(161.51507012,111.75845859)(161.44507019,111.81345854)(161.40506165,111.88345998)
\curveto(161.35507028,111.96345839)(161.3300703,112.08345827)(161.33006165,112.24345998)
\curveto(161.3300703,112.40345795)(161.31507032,112.53845781)(161.28506165,112.64845998)
\curveto(161.27507036,112.69845765)(161.27007036,112.7534576)(161.27006165,112.81345998)
\curveto(161.26007037,112.87345748)(161.24507039,112.93345742)(161.22506165,112.99345998)
\curveto(161.17507046,113.14345721)(161.12507051,113.28845706)(161.07506165,113.42845998)
\curveto(161.01507062,113.56845678)(160.94507069,113.70345665)(160.86506165,113.83345998)
\curveto(160.77507086,113.97345638)(160.67007096,114.09345626)(160.55006165,114.19345998)
\curveto(160.4300712,114.29345606)(160.30007133,114.38845596)(160.16006165,114.47845998)
\curveto(160.06007157,114.53845581)(159.95007168,114.58345577)(159.83006165,114.61345998)
\curveto(159.71007192,114.6534557)(159.60507203,114.70345565)(159.51506165,114.76345998)
\curveto(159.45507218,114.81345554)(159.41507222,114.88345547)(159.39506165,114.97345998)
\curveto(159.38507225,114.99345536)(159.38007225,115.01845533)(159.38006165,115.04845998)
\curveto(159.38007225,115.07845527)(159.37507226,115.10345525)(159.36506165,115.12345998)
}
}
{
\newrgbcolor{curcolor}{0 0 0}
\pscustom[linestyle=none,fillstyle=solid,fillcolor=curcolor]
{
\newpath
\moveto(201.112724,31.67142873)
\lineto(201.112724,32.58642873)
\curveto(201.11273469,32.68642608)(201.11273469,32.78142599)(201.112724,32.87142873)
\curveto(201.11273469,32.96142581)(201.13273467,33.03642573)(201.172724,33.09642873)
\curveto(201.23273457,33.18642558)(201.31273449,33.24642552)(201.412724,33.27642873)
\curveto(201.51273429,33.31642545)(201.61773419,33.36142541)(201.727724,33.41142873)
\curveto(201.91773389,33.49142528)(202.1077337,33.56142521)(202.297724,33.62142873)
\curveto(202.48773332,33.69142508)(202.67773313,33.766425)(202.867724,33.84642873)
\curveto(203.04773276,33.91642485)(203.23273257,33.98142479)(203.422724,34.04142873)
\curveto(203.6027322,34.10142467)(203.78273202,34.1714246)(203.962724,34.25142873)
\curveto(204.1027317,34.31142446)(204.24773156,34.3664244)(204.397724,34.41642873)
\curveto(204.54773126,34.4664243)(204.69273111,34.52142425)(204.832724,34.58142873)
\curveto(205.28273052,34.76142401)(205.73773007,34.93142384)(206.197724,35.09142873)
\curveto(206.64772916,35.25142352)(207.09772871,35.42142335)(207.547724,35.60142873)
\curveto(207.59772821,35.62142315)(207.64772816,35.63642313)(207.697724,35.64642873)
\lineto(207.847724,35.70642873)
\curveto(208.06772774,35.79642297)(208.29272751,35.88142289)(208.522724,35.96142873)
\curveto(208.74272706,36.04142273)(208.96272684,36.12642264)(209.182724,36.21642873)
\curveto(209.27272653,36.25642251)(209.38272642,36.29642247)(209.512724,36.33642873)
\curveto(209.63272617,36.37642239)(209.7027261,36.44142233)(209.722724,36.53142873)
\curveto(209.73272607,36.5714222)(209.73272607,36.60142217)(209.722724,36.62142873)
\lineto(209.662724,36.68142873)
\curveto(209.61272619,36.73142204)(209.55772625,36.766422)(209.497724,36.78642873)
\curveto(209.43772637,36.81642195)(209.37272643,36.84642192)(209.302724,36.87642873)
\lineto(208.672724,37.11642873)
\curveto(208.45272735,37.19642157)(208.23772757,37.27642149)(208.027724,37.35642873)
\lineto(207.877724,37.41642873)
\lineto(207.697724,37.47642873)
\curveto(207.5077283,37.55642121)(207.31772849,37.62642114)(207.127724,37.68642873)
\curveto(206.92772888,37.75642101)(206.72772908,37.83142094)(206.527724,37.91142873)
\curveto(205.94772986,38.15142062)(205.36273044,38.3714204)(204.772724,38.57142873)
\curveto(204.18273162,38.78141999)(203.59773221,39.00641976)(203.017724,39.24642873)
\curveto(202.81773299,39.32641944)(202.61273319,39.40141937)(202.402724,39.47142873)
\curveto(202.19273361,39.55141922)(201.98773382,39.63141914)(201.787724,39.71142873)
\curveto(201.7077341,39.75141902)(201.6077342,39.78641898)(201.487724,39.81642873)
\curveto(201.36773444,39.85641891)(201.28273452,39.91141886)(201.232724,39.98142873)
\curveto(201.19273461,40.04141873)(201.16273464,40.11641865)(201.142724,40.20642873)
\curveto(201.12273468,40.30641846)(201.11273469,40.41641835)(201.112724,40.53642873)
\curveto(201.1027347,40.65641811)(201.1027347,40.77641799)(201.112724,40.89642873)
\curveto(201.11273469,41.01641775)(201.11273469,41.12641764)(201.112724,41.22642873)
\curveto(201.11273469,41.31641745)(201.11273469,41.40641736)(201.112724,41.49642873)
\curveto(201.11273469,41.59641717)(201.13273467,41.6714171)(201.172724,41.72142873)
\curveto(201.22273458,41.81141696)(201.31273449,41.86141691)(201.442724,41.87142873)
\curveto(201.57273423,41.88141689)(201.71273409,41.88641688)(201.862724,41.88642873)
\lineto(203.512724,41.88642873)
\lineto(209.782724,41.88642873)
\lineto(211.042724,41.88642873)
\curveto(211.15272465,41.88641688)(211.26272454,41.88641688)(211.372724,41.88642873)
\curveto(211.48272432,41.89641687)(211.56772424,41.87641689)(211.627724,41.82642873)
\curveto(211.68772412,41.79641697)(211.72772408,41.75141702)(211.747724,41.69142873)
\curveto(211.75772405,41.63141714)(211.77272403,41.56141721)(211.792724,41.48142873)
\lineto(211.792724,41.24142873)
\lineto(211.792724,40.88142873)
\curveto(211.78272402,40.771418)(211.73772407,40.69141808)(211.657724,40.64142873)
\curveto(211.62772418,40.62141815)(211.59772421,40.60641816)(211.567724,40.59642873)
\curveto(211.52772428,40.59641817)(211.48272432,40.58641818)(211.432724,40.56642873)
\lineto(211.267724,40.56642873)
\curveto(211.2077246,40.55641821)(211.13772467,40.55141822)(211.057724,40.55142873)
\curveto(210.97772483,40.56141821)(210.9027249,40.5664182)(210.832724,40.56642873)
\lineto(209.992724,40.56642873)
\lineto(205.567724,40.56642873)
\curveto(205.31773049,40.5664182)(205.06773074,40.5664182)(204.817724,40.56642873)
\curveto(204.55773125,40.5664182)(204.3077315,40.56141821)(204.067724,40.55142873)
\curveto(203.96773184,40.55141822)(203.85773195,40.54641822)(203.737724,40.53642873)
\curveto(203.61773219,40.52641824)(203.55773225,40.4714183)(203.557724,40.37142873)
\lineto(203.572724,40.37142873)
\curveto(203.59273221,40.30141847)(203.65773215,40.24141853)(203.767724,40.19142873)
\curveto(203.87773193,40.15141862)(203.97273183,40.11641865)(204.052724,40.08642873)
\curveto(204.22273158,40.01641875)(204.39773141,39.95141882)(204.577724,39.89142873)
\curveto(204.74773106,39.83141894)(204.91773089,39.76141901)(205.087724,39.68142873)
\curveto(205.13773067,39.66141911)(205.18273062,39.64641912)(205.222724,39.63642873)
\curveto(205.26273054,39.62641914)(205.3077305,39.61141916)(205.357724,39.59142873)
\curveto(205.53773027,39.51141926)(205.72273008,39.44141933)(205.912724,39.38142873)
\curveto(206.09272971,39.33141944)(206.27272953,39.2664195)(206.452724,39.18642873)
\curveto(206.6027292,39.11641965)(206.75772905,39.05641971)(206.917724,39.00642873)
\curveto(207.06772874,38.95641981)(207.21772859,38.90141987)(207.367724,38.84142873)
\curveto(207.83772797,38.64142013)(208.31272749,38.46142031)(208.792724,38.30142873)
\curveto(209.26272654,38.14142063)(209.72772608,37.9664208)(210.187724,37.77642873)
\curveto(210.36772544,37.69642107)(210.54772526,37.62642114)(210.727724,37.56642873)
\curveto(210.9077249,37.50642126)(211.08772472,37.44142133)(211.267724,37.37142873)
\curveto(211.37772443,37.32142145)(211.48272432,37.2714215)(211.582724,37.22142873)
\curveto(211.67272413,37.18142159)(211.73772407,37.09642167)(211.777724,36.96642873)
\curveto(211.78772402,36.94642182)(211.79272401,36.92142185)(211.792724,36.89142873)
\curveto(211.78272402,36.8714219)(211.78272402,36.84642192)(211.792724,36.81642873)
\curveto(211.802724,36.78642198)(211.807724,36.75142202)(211.807724,36.71142873)
\curveto(211.79772401,36.6714221)(211.79272401,36.63142214)(211.792724,36.59142873)
\lineto(211.792724,36.29142873)
\curveto(211.79272401,36.19142258)(211.76772404,36.11142266)(211.717724,36.05142873)
\curveto(211.66772414,35.9714228)(211.59772421,35.91142286)(211.507724,35.87142873)
\curveto(211.4077244,35.84142293)(211.3077245,35.80142297)(211.207724,35.75142873)
\curveto(211.0077248,35.6714231)(210.802725,35.59142318)(210.592724,35.51142873)
\curveto(210.37272543,35.44142333)(210.16272564,35.3664234)(209.962724,35.28642873)
\curveto(209.78272602,35.20642356)(209.6027262,35.13642363)(209.422724,35.07642873)
\curveto(209.23272657,35.02642374)(209.04772676,34.96142381)(208.867724,34.88142873)
\curveto(208.3077275,34.65142412)(207.74272806,34.43642433)(207.172724,34.23642873)
\curveto(206.6027292,34.03642473)(206.03772977,33.82142495)(205.477724,33.59142873)
\lineto(204.847724,33.35142873)
\curveto(204.62773118,33.28142549)(204.41773139,33.20642556)(204.217724,33.12642873)
\curveto(204.1077317,33.07642569)(204.0027318,33.03142574)(203.902724,32.99142873)
\curveto(203.79273201,32.96142581)(203.69773211,32.91142586)(203.617724,32.84142873)
\curveto(203.59773221,32.83142594)(203.58773222,32.82142595)(203.587724,32.81142873)
\lineto(203.557724,32.78142873)
\lineto(203.557724,32.70642873)
\lineto(203.587724,32.67642873)
\curveto(203.58773222,32.6664261)(203.59273221,32.65642611)(203.602724,32.64642873)
\curveto(203.65273215,32.62642614)(203.7077321,32.61642615)(203.767724,32.61642873)
\curveto(203.82773198,32.61642615)(203.88773192,32.60642616)(203.947724,32.58642873)
\lineto(204.112724,32.58642873)
\curveto(204.17273163,32.5664262)(204.23773157,32.56142621)(204.307724,32.57142873)
\curveto(204.37773143,32.58142619)(204.44773136,32.58642618)(204.517724,32.58642873)
\lineto(205.327724,32.58642873)
\lineto(209.887724,32.58642873)
\lineto(211.072724,32.58642873)
\curveto(211.18272462,32.58642618)(211.29272451,32.58142619)(211.402724,32.57142873)
\curveto(211.51272429,32.5714262)(211.59772421,32.54642622)(211.657724,32.49642873)
\curveto(211.73772407,32.44642632)(211.78272402,32.35642641)(211.792724,32.22642873)
\lineto(211.792724,31.83642873)
\lineto(211.792724,31.64142873)
\curveto(211.79272401,31.59142718)(211.78272402,31.54142723)(211.762724,31.49142873)
\curveto(211.72272408,31.36142741)(211.63772417,31.28642748)(211.507724,31.26642873)
\curveto(211.37772443,31.25642751)(211.22772458,31.25142752)(211.057724,31.25142873)
\lineto(209.317724,31.25142873)
\lineto(203.317724,31.25142873)
\lineto(201.907724,31.25142873)
\curveto(201.79773401,31.25142752)(201.68273412,31.24642752)(201.562724,31.23642873)
\curveto(201.44273436,31.23642753)(201.34773446,31.26142751)(201.277724,31.31142873)
\curveto(201.21773459,31.35142742)(201.16773464,31.42642734)(201.127724,31.53642873)
\curveto(201.11773469,31.55642721)(201.11773469,31.57642719)(201.127724,31.59642873)
\curveto(201.12773468,31.62642714)(201.12273468,31.65142712)(201.112724,31.67142873)
}
}
{
\newrgbcolor{curcolor}{0 0 0}
\pscustom[linestyle=none,fillstyle=solid,fillcolor=curcolor]
{
\newpath
\moveto(211.237724,50.87353811)
\curveto(211.39772441,50.90353028)(211.53272427,50.88853029)(211.642724,50.82853811)
\curveto(211.74272406,50.76853041)(211.81772399,50.68853049)(211.867724,50.58853811)
\curveto(211.88772392,50.53853064)(211.89772391,50.4835307)(211.897724,50.42353811)
\curveto(211.89772391,50.37353081)(211.9077239,50.31853086)(211.927724,50.25853811)
\curveto(211.97772383,50.03853114)(211.96272384,49.81853136)(211.882724,49.59853811)
\curveto(211.81272399,49.38853179)(211.72272408,49.24353194)(211.612724,49.16353811)
\curveto(211.54272426,49.11353207)(211.46272434,49.06853211)(211.372724,49.02853811)
\curveto(211.27272453,48.98853219)(211.19272461,48.93853224)(211.132724,48.87853811)
\curveto(211.11272469,48.85853232)(211.09272471,48.83353235)(211.072724,48.80353811)
\curveto(211.05272475,48.7835324)(211.04772476,48.75353243)(211.057724,48.71353811)
\curveto(211.08772472,48.60353258)(211.14272466,48.49853268)(211.222724,48.39853811)
\curveto(211.3027245,48.30853287)(211.37272443,48.21853296)(211.432724,48.12853811)
\curveto(211.51272429,47.99853318)(211.58772422,47.85853332)(211.657724,47.70853811)
\curveto(211.71772409,47.55853362)(211.77272403,47.39853378)(211.822724,47.22853811)
\curveto(211.85272395,47.12853405)(211.87272393,47.01853416)(211.882724,46.89853811)
\curveto(211.89272391,46.78853439)(211.9077239,46.6785345)(211.927724,46.56853811)
\curveto(211.93772387,46.51853466)(211.94272386,46.47353471)(211.942724,46.43353811)
\lineto(211.942724,46.32853811)
\curveto(211.96272384,46.21853496)(211.96272384,46.11353507)(211.942724,46.01353811)
\lineto(211.942724,45.87853811)
\curveto(211.93272387,45.82853535)(211.92772388,45.7785354)(211.927724,45.72853811)
\curveto(211.92772388,45.6785355)(211.91772389,45.63353555)(211.897724,45.59353811)
\curveto(211.88772392,45.55353563)(211.88272392,45.51853566)(211.882724,45.48853811)
\curveto(211.89272391,45.46853571)(211.89272391,45.44353574)(211.882724,45.41353811)
\lineto(211.822724,45.17353811)
\curveto(211.81272399,45.09353609)(211.79272401,45.01853616)(211.762724,44.94853811)
\curveto(211.63272417,44.64853653)(211.48772432,44.40353678)(211.327724,44.21353811)
\curveto(211.15772465,44.03353715)(210.92272488,43.8835373)(210.622724,43.76353811)
\curveto(210.4027254,43.67353751)(210.13772567,43.62853755)(209.827724,43.62853811)
\lineto(209.512724,43.62853811)
\curveto(209.46272634,43.63853754)(209.41272639,43.64353754)(209.362724,43.64353811)
\lineto(209.182724,43.67353811)
\lineto(208.852724,43.79353811)
\curveto(208.74272706,43.83353735)(208.64272716,43.8835373)(208.552724,43.94353811)
\curveto(208.26272754,44.12353706)(208.04772776,44.36853681)(207.907724,44.67853811)
\curveto(207.76772804,44.98853619)(207.64272816,45.32853585)(207.532724,45.69853811)
\curveto(207.49272831,45.83853534)(207.46272834,45.9835352)(207.442724,46.13353811)
\curveto(207.42272838,46.2835349)(207.39772841,46.43353475)(207.367724,46.58353811)
\curveto(207.34772846,46.65353453)(207.33772847,46.71853446)(207.337724,46.77853811)
\curveto(207.33772847,46.84853433)(207.32772848,46.92353426)(207.307724,47.00353811)
\curveto(207.28772852,47.07353411)(207.27772853,47.14353404)(207.277724,47.21353811)
\curveto(207.26772854,47.2835339)(207.25272855,47.35853382)(207.232724,47.43853811)
\curveto(207.17272863,47.68853349)(207.12272868,47.92353326)(207.082724,48.14353811)
\curveto(207.03272877,48.36353282)(206.91772889,48.53853264)(206.737724,48.66853811)
\curveto(206.65772915,48.72853245)(206.55772925,48.7785324)(206.437724,48.81853811)
\curveto(206.3077295,48.85853232)(206.16772964,48.85853232)(206.017724,48.81853811)
\curveto(205.77773003,48.75853242)(205.58773022,48.66853251)(205.447724,48.54853811)
\curveto(205.3077305,48.43853274)(205.19773061,48.2785329)(205.117724,48.06853811)
\curveto(205.06773074,47.94853323)(205.03273077,47.80353338)(205.012724,47.63353811)
\curveto(204.99273081,47.47353371)(204.98273082,47.30353388)(204.982724,47.12353811)
\curveto(204.98273082,46.94353424)(204.99273081,46.76853441)(205.012724,46.59853811)
\curveto(205.03273077,46.42853475)(205.06273074,46.2835349)(205.102724,46.16353811)
\curveto(205.16273064,45.99353519)(205.24773056,45.82853535)(205.357724,45.66853811)
\curveto(205.41773039,45.58853559)(205.49773031,45.51353567)(205.597724,45.44353811)
\curveto(205.68773012,45.3835358)(205.78773002,45.32853585)(205.897724,45.27853811)
\curveto(205.97772983,45.24853593)(206.06272974,45.21853596)(206.152724,45.18853811)
\curveto(206.24272956,45.16853601)(206.31272949,45.12353606)(206.362724,45.05353811)
\curveto(206.39272941,45.01353617)(206.41772939,44.94353624)(206.437724,44.84353811)
\curveto(206.44772936,44.75353643)(206.45272935,44.65853652)(206.452724,44.55853811)
\curveto(206.45272935,44.45853672)(206.44772936,44.35853682)(206.437724,44.25853811)
\curveto(206.41772939,44.16853701)(206.39272941,44.10353708)(206.362724,44.06353811)
\curveto(206.33272947,44.02353716)(206.28272952,43.99353719)(206.212724,43.97353811)
\curveto(206.14272966,43.95353723)(206.06772974,43.95353723)(205.987724,43.97353811)
\curveto(205.85772995,44.00353718)(205.73773007,44.03353715)(205.627724,44.06353811)
\curveto(205.5077303,44.10353708)(205.39273041,44.14853703)(205.282724,44.19853811)
\curveto(204.93273087,44.38853679)(204.66273114,44.62853655)(204.472724,44.91853811)
\curveto(204.27273153,45.20853597)(204.11273169,45.56853561)(203.992724,45.99853811)
\curveto(203.97273183,46.09853508)(203.95773185,46.19853498)(203.947724,46.29853811)
\curveto(203.93773187,46.40853477)(203.92273188,46.51853466)(203.902724,46.62853811)
\curveto(203.89273191,46.66853451)(203.89273191,46.73353445)(203.902724,46.82353811)
\curveto(203.9027319,46.91353427)(203.89273191,46.96853421)(203.872724,46.98853811)
\curveto(203.86273194,47.68853349)(203.94273186,48.29853288)(204.112724,48.81853811)
\curveto(204.28273152,49.33853184)(204.6077312,49.70353148)(205.087724,49.91353811)
\curveto(205.28773052,50.00353118)(205.52273028,50.05353113)(205.792724,50.06353811)
\curveto(206.05272975,50.0835311)(206.32772948,50.09353109)(206.617724,50.09353811)
\lineto(209.932724,50.09353811)
\curveto(210.07272573,50.09353109)(210.2077256,50.09853108)(210.337724,50.10853811)
\curveto(210.46772534,50.11853106)(210.57272523,50.14853103)(210.652724,50.19853811)
\curveto(210.72272508,50.24853093)(210.77272503,50.31353087)(210.802724,50.39353811)
\curveto(210.84272496,50.4835307)(210.87272493,50.56853061)(210.892724,50.64853811)
\curveto(210.9027249,50.72853045)(210.94772486,50.78853039)(211.027724,50.82853811)
\curveto(211.05772475,50.84853033)(211.08772472,50.85853032)(211.117724,50.85853811)
\curveto(211.14772466,50.85853032)(211.18772462,50.86353032)(211.237724,50.87353811)
\moveto(209.572724,48.72853811)
\curveto(209.43272637,48.78853239)(209.27272653,48.81853236)(209.092724,48.81853811)
\curveto(208.9027269,48.82853235)(208.7077271,48.83353235)(208.507724,48.83353811)
\curveto(208.39772741,48.83353235)(208.29772751,48.82853235)(208.207724,48.81853811)
\curveto(208.11772769,48.80853237)(208.04772776,48.76853241)(207.997724,48.69853811)
\curveto(207.97772783,48.66853251)(207.96772784,48.59853258)(207.967724,48.48853811)
\curveto(207.98772782,48.46853271)(207.99772781,48.43353275)(207.997724,48.38353811)
\curveto(207.99772781,48.33353285)(208.0077278,48.28853289)(208.027724,48.24853811)
\curveto(208.04772776,48.16853301)(208.06772774,48.0785331)(208.087724,47.97853811)
\lineto(208.147724,47.67853811)
\curveto(208.14772766,47.64853353)(208.15272765,47.61353357)(208.162724,47.57353811)
\lineto(208.162724,47.46853811)
\curveto(208.2027276,47.31853386)(208.22772758,47.15353403)(208.237724,46.97353811)
\curveto(208.23772757,46.80353438)(208.25772755,46.64353454)(208.297724,46.49353811)
\curveto(208.31772749,46.41353477)(208.33772747,46.33853484)(208.357724,46.26853811)
\curveto(208.36772744,46.20853497)(208.38272742,46.13853504)(208.402724,46.05853811)
\curveto(208.45272735,45.89853528)(208.51772729,45.74853543)(208.597724,45.60853811)
\curveto(208.66772714,45.46853571)(208.75772705,45.34853583)(208.867724,45.24853811)
\curveto(208.97772683,45.14853603)(209.11272669,45.07353611)(209.272724,45.02353811)
\curveto(209.42272638,44.97353621)(209.6077262,44.95353623)(209.827724,44.96353811)
\curveto(209.92772588,44.96353622)(210.02272578,44.9785362)(210.112724,45.00853811)
\curveto(210.19272561,45.04853613)(210.26772554,45.09353609)(210.337724,45.14353811)
\curveto(210.44772536,45.22353596)(210.54272526,45.32853585)(210.622724,45.45853811)
\curveto(210.69272511,45.58853559)(210.75272505,45.72853545)(210.802724,45.87853811)
\curveto(210.81272499,45.92853525)(210.81772499,45.9785352)(210.817724,46.02853811)
\curveto(210.81772499,46.0785351)(210.82272498,46.12853505)(210.832724,46.17853811)
\curveto(210.85272495,46.24853493)(210.86772494,46.33353485)(210.877724,46.43353811)
\curveto(210.87772493,46.54353464)(210.86772494,46.63353455)(210.847724,46.70353811)
\curveto(210.82772498,46.76353442)(210.82272498,46.82353436)(210.832724,46.88353811)
\curveto(210.83272497,46.94353424)(210.82272498,47.00353418)(210.802724,47.06353811)
\curveto(210.78272502,47.14353404)(210.76772504,47.21853396)(210.757724,47.28853811)
\curveto(210.74772506,47.36853381)(210.72772508,47.44353374)(210.697724,47.51353811)
\curveto(210.57772523,47.80353338)(210.43272537,48.04853313)(210.262724,48.24853811)
\curveto(210.09272571,48.45853272)(209.86272594,48.61853256)(209.572724,48.72853811)
}
}
{
\newrgbcolor{curcolor}{0 0 0}
\pscustom[linestyle=none,fillstyle=solid,fillcolor=curcolor]
{
\newpath
\moveto(203.887724,55.69017873)
\curveto(203.88773192,55.92017394)(203.94773186,56.05017381)(204.067724,56.08017873)
\curveto(204.17773163,56.11017375)(204.34273146,56.12517374)(204.562724,56.12517873)
\lineto(204.847724,56.12517873)
\curveto(204.93773087,56.12517374)(205.01273079,56.10017376)(205.072724,56.05017873)
\curveto(205.15273065,55.99017387)(205.19773061,55.90517396)(205.207724,55.79517873)
\curveto(205.2077306,55.68517418)(205.22273058,55.57517429)(205.252724,55.46517873)
\curveto(205.28273052,55.32517454)(205.31273049,55.19017467)(205.342724,55.06017873)
\curveto(205.37273043,54.94017492)(205.41273039,54.82517504)(205.462724,54.71517873)
\curveto(205.59273021,54.42517544)(205.77273003,54.19017567)(206.002724,54.01017873)
\curveto(206.22272958,53.83017603)(206.47772933,53.67517619)(206.767724,53.54517873)
\curveto(206.87772893,53.50517636)(206.99272881,53.47517639)(207.112724,53.45517873)
\curveto(207.22272858,53.43517643)(207.33772847,53.41017645)(207.457724,53.38017873)
\curveto(207.5077283,53.37017649)(207.55772825,53.3651765)(207.607724,53.36517873)
\curveto(207.65772815,53.37517649)(207.7077281,53.37517649)(207.757724,53.36517873)
\curveto(207.87772793,53.33517653)(208.01772779,53.32017654)(208.177724,53.32017873)
\curveto(208.32772748,53.33017653)(208.47272733,53.33517653)(208.612724,53.33517873)
\lineto(210.457724,53.33517873)
\lineto(210.802724,53.33517873)
\curveto(210.92272488,53.33517653)(211.03772477,53.33017653)(211.147724,53.32017873)
\curveto(211.25772455,53.31017655)(211.35272445,53.30517656)(211.432724,53.30517873)
\curveto(211.51272429,53.31517655)(211.58272422,53.29517657)(211.642724,53.24517873)
\curveto(211.71272409,53.19517667)(211.75272405,53.11517675)(211.762724,53.00517873)
\curveto(211.77272403,52.90517696)(211.77772403,52.79517707)(211.777724,52.67517873)
\lineto(211.777724,52.40517873)
\curveto(211.75772405,52.35517751)(211.74272406,52.30517756)(211.732724,52.25517873)
\curveto(211.71272409,52.21517765)(211.68772412,52.18517768)(211.657724,52.16517873)
\curveto(211.58772422,52.11517775)(211.5027243,52.08517778)(211.402724,52.07517873)
\lineto(211.072724,52.07517873)
\lineto(209.917724,52.07517873)
\lineto(205.762724,52.07517873)
\lineto(204.727724,52.07517873)
\lineto(204.427724,52.07517873)
\curveto(204.32773148,52.08517778)(204.24273156,52.11517775)(204.172724,52.16517873)
\curveto(204.13273167,52.19517767)(204.1027317,52.24517762)(204.082724,52.31517873)
\curveto(204.06273174,52.39517747)(204.05273175,52.48017738)(204.052724,52.57017873)
\curveto(204.04273176,52.6601772)(204.04273176,52.75017711)(204.052724,52.84017873)
\curveto(204.06273174,52.93017693)(204.07773173,53.00017686)(204.097724,53.05017873)
\curveto(204.12773168,53.13017673)(204.18773162,53.18017668)(204.277724,53.20017873)
\curveto(204.35773145,53.23017663)(204.44773136,53.24517662)(204.547724,53.24517873)
\lineto(204.847724,53.24517873)
\curveto(204.94773086,53.24517662)(205.03773077,53.2651766)(205.117724,53.30517873)
\curveto(205.13773067,53.31517655)(205.15273065,53.32517654)(205.162724,53.33517873)
\lineto(205.207724,53.38017873)
\curveto(205.2077306,53.49017637)(205.16273064,53.58017628)(205.072724,53.65017873)
\curveto(204.97273083,53.72017614)(204.89273091,53.78017608)(204.832724,53.83017873)
\lineto(204.742724,53.92017873)
\curveto(204.63273117,54.01017585)(204.51773129,54.13517573)(204.397724,54.29517873)
\curveto(204.27773153,54.45517541)(204.18773162,54.60517526)(204.127724,54.74517873)
\curveto(204.07773173,54.83517503)(204.04273176,54.93017493)(204.022724,55.03017873)
\curveto(203.99273181,55.13017473)(203.96273184,55.23517463)(203.932724,55.34517873)
\curveto(203.92273188,55.40517446)(203.91773189,55.4651744)(203.917724,55.52517873)
\curveto(203.9077319,55.58517428)(203.89773191,55.64017422)(203.887724,55.69017873)
}
}
{
\newrgbcolor{curcolor}{0 0 0}
\pscustom[linestyle=none,fillstyle=solid,fillcolor=curcolor]
{
}
}
{
\newrgbcolor{curcolor}{0 0 0}
\pscustom[linestyle=none,fillstyle=solid,fillcolor=curcolor]
{
\newpath
\moveto(201.187724,65.05510061)
\curveto(201.18773462,65.15509575)(201.19773461,65.25009566)(201.217724,65.34010061)
\curveto(201.22773458,65.43009548)(201.25773455,65.49509541)(201.307724,65.53510061)
\curveto(201.38773442,65.59509531)(201.49273431,65.62509528)(201.622724,65.62510061)
\lineto(202.012724,65.62510061)
\lineto(203.512724,65.62510061)
\lineto(209.902724,65.62510061)
\lineto(211.072724,65.62510061)
\lineto(211.387724,65.62510061)
\curveto(211.48772432,65.63509527)(211.56772424,65.62009529)(211.627724,65.58010061)
\curveto(211.7077241,65.53009538)(211.75772405,65.45509545)(211.777724,65.35510061)
\curveto(211.78772402,65.26509564)(211.79272401,65.15509575)(211.792724,65.02510061)
\lineto(211.792724,64.80010061)
\curveto(211.77272403,64.72009619)(211.75772405,64.65009626)(211.747724,64.59010061)
\curveto(211.72772408,64.53009638)(211.68772412,64.48009643)(211.627724,64.44010061)
\curveto(211.56772424,64.40009651)(211.49272431,64.38009653)(211.402724,64.38010061)
\lineto(211.102724,64.38010061)
\lineto(210.007724,64.38010061)
\lineto(204.667724,64.38010061)
\curveto(204.57773123,64.36009655)(204.5027313,64.34509656)(204.442724,64.33510061)
\curveto(204.37273143,64.33509657)(204.31273149,64.3050966)(204.262724,64.24510061)
\curveto(204.21273159,64.17509673)(204.18773162,64.08509682)(204.187724,63.97510061)
\curveto(204.17773163,63.87509703)(204.17273163,63.76509714)(204.172724,63.64510061)
\lineto(204.172724,62.50510061)
\lineto(204.172724,62.01010061)
\curveto(204.16273164,61.85009906)(204.1027317,61.74009917)(203.992724,61.68010061)
\curveto(203.96273184,61.66009925)(203.93273187,61.65009926)(203.902724,61.65010061)
\curveto(203.86273194,61.65009926)(203.81773199,61.64509926)(203.767724,61.63510061)
\curveto(203.64773216,61.61509929)(203.53773227,61.62009929)(203.437724,61.65010061)
\curveto(203.33773247,61.69009922)(203.26773254,61.74509916)(203.227724,61.81510061)
\curveto(203.17773263,61.89509901)(203.15273265,62.01509889)(203.152724,62.17510061)
\curveto(203.15273265,62.33509857)(203.13773267,62.47009844)(203.107724,62.58010061)
\curveto(203.09773271,62.63009828)(203.09273271,62.68509822)(203.092724,62.74510061)
\curveto(203.08273272,62.8050981)(203.06773274,62.86509804)(203.047724,62.92510061)
\curveto(202.99773281,63.07509783)(202.94773286,63.22009769)(202.897724,63.36010061)
\curveto(202.83773297,63.50009741)(202.76773304,63.63509727)(202.687724,63.76510061)
\curveto(202.59773321,63.905097)(202.49273331,64.02509688)(202.372724,64.12510061)
\curveto(202.25273355,64.22509668)(202.12273368,64.32009659)(201.982724,64.41010061)
\curveto(201.88273392,64.47009644)(201.77273403,64.51509639)(201.652724,64.54510061)
\curveto(201.53273427,64.58509632)(201.42773438,64.63509627)(201.337724,64.69510061)
\curveto(201.27773453,64.74509616)(201.23773457,64.81509609)(201.217724,64.90510061)
\curveto(201.2077346,64.92509598)(201.2027346,64.95009596)(201.202724,64.98010061)
\curveto(201.2027346,65.0100959)(201.19773461,65.03509587)(201.187724,65.05510061)
}
}
{
\newrgbcolor{curcolor}{0 0 0}
\pscustom[linestyle=none,fillstyle=solid,fillcolor=curcolor]
{
\newpath
\moveto(208.717724,76.35970998)
\curveto(208.75772705,76.36970226)(208.807727,76.36970226)(208.867724,76.35970998)
\curveto(208.92772688,76.35970227)(208.97772683,76.35470228)(209.017724,76.34470998)
\curveto(209.05772675,76.34470229)(209.09772671,76.33970229)(209.137724,76.32970998)
\lineto(209.242724,76.32970998)
\curveto(209.32272648,76.30970232)(209.4027264,76.29470234)(209.482724,76.28470998)
\curveto(209.56272624,76.27470236)(209.63772617,76.25470238)(209.707724,76.22470998)
\curveto(209.78772602,76.20470243)(209.86272594,76.18470245)(209.932724,76.16470998)
\curveto(210.0027258,76.14470249)(210.07772573,76.11470252)(210.157724,76.07470998)
\curveto(210.57772523,75.89470274)(210.91772489,75.63970299)(211.177724,75.30970998)
\curveto(211.43772437,74.97970365)(211.64272416,74.58970404)(211.792724,74.13970998)
\curveto(211.83272397,74.01970461)(211.85772395,73.89470474)(211.867724,73.76470998)
\curveto(211.88772392,73.64470499)(211.91272389,73.51970511)(211.942724,73.38970998)
\curveto(211.95272385,73.3297053)(211.95772385,73.26470537)(211.957724,73.19470998)
\curveto(211.95772385,73.1347055)(211.96272384,73.06970556)(211.972724,72.99970998)
\lineto(211.972724,72.87970998)
\lineto(211.972724,72.68470998)
\curveto(211.98272382,72.62470601)(211.97772383,72.56970606)(211.957724,72.51970998)
\curveto(211.93772387,72.44970618)(211.93272387,72.38470625)(211.942724,72.32470998)
\curveto(211.95272385,72.26470637)(211.94772386,72.20470643)(211.927724,72.14470998)
\curveto(211.91772389,72.09470654)(211.91272389,72.04970658)(211.912724,72.00970998)
\curveto(211.91272389,71.96970666)(211.9027239,71.92470671)(211.882724,71.87470998)
\curveto(211.86272394,71.79470684)(211.84272396,71.71970691)(211.822724,71.64970998)
\curveto(211.81272399,71.57970705)(211.79772401,71.50970712)(211.777724,71.43970998)
\curveto(211.6077242,70.95970767)(211.39772441,70.55970807)(211.147724,70.23970998)
\curveto(210.88772492,69.9297087)(210.53272527,69.67970895)(210.082724,69.48970998)
\curveto(210.02272578,69.45970917)(209.96272584,69.4347092)(209.902724,69.41470998)
\curveto(209.83272597,69.40470923)(209.75772605,69.38970924)(209.677724,69.36970998)
\curveto(209.61772619,69.34970928)(209.55272625,69.3347093)(209.482724,69.32470998)
\curveto(209.41272639,69.31470932)(209.34272646,69.29970933)(209.272724,69.27970998)
\curveto(209.22272658,69.26970936)(209.18272662,69.26470937)(209.152724,69.26470998)
\lineto(209.032724,69.26470998)
\curveto(208.99272681,69.25470938)(208.94272686,69.24470939)(208.882724,69.23470998)
\curveto(208.82272698,69.2347094)(208.77272703,69.23970939)(208.732724,69.24970998)
\lineto(208.597724,69.24970998)
\curveto(208.54772726,69.25970937)(208.49772731,69.26470937)(208.447724,69.26470998)
\curveto(208.34772746,69.28470935)(208.25272755,69.29970933)(208.162724,69.30970998)
\curveto(208.06272774,69.31970931)(207.96772784,69.33970929)(207.877724,69.36970998)
\curveto(207.72772808,69.41970921)(207.58772822,69.47470916)(207.457724,69.53470998)
\curveto(207.32772848,69.59470904)(207.2077286,69.66470897)(207.097724,69.74470998)
\curveto(207.04772876,69.77470886)(207.0077288,69.80470883)(206.977724,69.83470998)
\curveto(206.94772886,69.87470876)(206.91272889,69.90970872)(206.872724,69.93970998)
\curveto(206.79272901,69.99970863)(206.72272908,70.06970856)(206.662724,70.14970998)
\curveto(206.61272919,70.20970842)(206.56772924,70.26970836)(206.527724,70.32970998)
\lineto(206.377724,70.53970998)
\curveto(206.33772947,70.58970804)(206.3027295,70.63970799)(206.272724,70.68970998)
\curveto(206.23272957,70.73970789)(206.17772963,70.77470786)(206.107724,70.79470998)
\curveto(206.07772973,70.79470784)(206.05272975,70.78470785)(206.032724,70.76470998)
\curveto(206.0027298,70.75470788)(205.97772983,70.74470789)(205.957724,70.73470998)
\curveto(205.9077299,70.69470794)(205.86272994,70.64470799)(205.822724,70.58470998)
\curveto(205.77273003,70.5347081)(205.72773008,70.48470815)(205.687724,70.43470998)
\curveto(205.65773015,70.39470824)(205.6027302,70.34470829)(205.522724,70.28470998)
\curveto(205.49273031,70.26470837)(205.46773034,70.2347084)(205.447724,70.19470998)
\curveto(205.41773039,70.16470847)(205.38273042,70.13970849)(205.342724,70.11970998)
\curveto(205.13273067,69.94970868)(204.88773092,69.81970881)(204.607724,69.72970998)
\curveto(204.52773128,69.70970892)(204.44773136,69.69470894)(204.367724,69.68470998)
\curveto(204.28773152,69.67470896)(204.2077316,69.65970897)(204.127724,69.63970998)
\curveto(204.07773173,69.61970901)(204.01273179,69.60970902)(203.932724,69.60970998)
\curveto(203.84273196,69.60970902)(203.77273203,69.61970901)(203.722724,69.63970998)
\curveto(203.62273218,69.63970899)(203.55273225,69.64470899)(203.512724,69.65470998)
\curveto(203.43273237,69.67470896)(203.36273244,69.68970894)(203.302724,69.69970998)
\curveto(203.23273257,69.70970892)(203.16273264,69.72470891)(203.092724,69.74470998)
\curveto(202.66273314,69.89470874)(202.31773349,70.10970852)(202.057724,70.38970998)
\curveto(201.79773401,70.67970795)(201.58273422,71.0297076)(201.412724,71.43970998)
\curveto(201.36273444,71.54970708)(201.33273447,71.66470697)(201.322724,71.78470998)
\curveto(201.3027345,71.91470672)(201.27273453,72.04470659)(201.232724,72.17470998)
\curveto(201.23273457,72.25470638)(201.23273457,72.32470631)(201.232724,72.38470998)
\curveto(201.22273458,72.45470618)(201.21273459,72.5297061)(201.202724,72.60970998)
\curveto(201.18273462,73.39970523)(201.31273449,74.05470458)(201.592724,74.57470998)
\curveto(201.87273393,75.10470353)(202.28273352,75.48470315)(202.822724,75.71470998)
\curveto(203.05273275,75.82470281)(203.33773247,75.89470274)(203.677724,75.92470998)
\curveto(204.0077318,75.96470267)(204.31273149,75.9347027)(204.592724,75.83470998)
\curveto(204.72273108,75.79470284)(204.84273096,75.74470289)(204.952724,75.68470998)
\curveto(205.06273074,75.634703)(205.16773064,75.57470306)(205.267724,75.50470998)
\curveto(205.3077305,75.48470315)(205.34273046,75.45470318)(205.372724,75.41470998)
\lineto(205.462724,75.32470998)
\curveto(205.55273025,75.27470336)(205.61773019,75.21470342)(205.657724,75.14470998)
\curveto(205.7077301,75.09470354)(205.75773005,75.03970359)(205.807724,74.97970998)
\curveto(205.84772996,74.9297037)(205.89272991,74.88470375)(205.942724,74.84470998)
\curveto(205.96272984,74.82470381)(205.98772982,74.80470383)(206.017724,74.78470998)
\curveto(206.03772977,74.77470386)(206.06272974,74.77470386)(206.092724,74.78470998)
\curveto(206.14272966,74.79470384)(206.19272961,74.82470381)(206.242724,74.87470998)
\curveto(206.28272952,74.92470371)(206.32272948,74.97970365)(206.362724,75.03970998)
\lineto(206.482724,75.21970998)
\curveto(206.51272929,75.27970335)(206.54272926,75.3297033)(206.572724,75.36970998)
\curveto(206.81272899,75.69970293)(207.12272868,75.94970268)(207.502724,76.11970998)
\curveto(207.58272822,76.15970247)(207.66772814,76.18970244)(207.757724,76.20970998)
\curveto(207.84772796,76.23970239)(207.93772787,76.26470237)(208.027724,76.28470998)
\curveto(208.07772773,76.29470234)(208.13272767,76.30470233)(208.192724,76.31470998)
\lineto(208.342724,76.34470998)
\curveto(208.4027274,76.35470228)(208.46772734,76.35470228)(208.537724,76.34470998)
\curveto(208.59772721,76.3347023)(208.65772715,76.33970229)(208.717724,76.35970998)
\moveto(203.677724,70.97470998)
\curveto(203.78773202,70.94470769)(203.92773188,70.93970769)(204.097724,70.95970998)
\curveto(204.25773155,70.97970765)(204.38273142,71.00470763)(204.472724,71.03470998)
\curveto(204.79273101,71.14470749)(205.03773077,71.29470734)(205.207724,71.48470998)
\curveto(205.36773044,71.67470696)(205.49773031,71.93970669)(205.597724,72.27970998)
\curveto(205.62773018,72.40970622)(205.65273015,72.57470606)(205.672724,72.77470998)
\curveto(205.68273012,72.97470566)(205.66773014,73.14470549)(205.627724,73.28470998)
\curveto(205.54773026,73.57470506)(205.43773037,73.81470482)(205.297724,74.00470998)
\curveto(205.14773066,74.20470443)(204.94773086,74.35970427)(204.697724,74.46970998)
\curveto(204.64773116,74.48970414)(204.6027312,74.49970413)(204.562724,74.49970998)
\curveto(204.52273128,74.50970412)(204.47773133,74.52470411)(204.427724,74.54470998)
\curveto(204.31773149,74.57470406)(204.17773163,74.59470404)(204.007724,74.60470998)
\curveto(203.83773197,74.61470402)(203.69273211,74.60470403)(203.572724,74.57470998)
\curveto(203.48273232,74.55470408)(203.39773241,74.5297041)(203.317724,74.49970998)
\curveto(203.23773257,74.47970415)(203.15773265,74.44470419)(203.077724,74.39470998)
\curveto(202.807733,74.22470441)(202.61273319,73.99970463)(202.492724,73.71970998)
\curveto(202.37273343,73.44970518)(202.31273349,73.08970554)(202.312724,72.63970998)
\curveto(202.33273347,72.61970601)(202.33773347,72.58970604)(202.327724,72.54970998)
\curveto(202.31773349,72.50970612)(202.31773349,72.47470616)(202.327724,72.44470998)
\curveto(202.34773346,72.39470624)(202.36273344,72.33970629)(202.372724,72.27970998)
\curveto(202.37273343,72.2297064)(202.38273342,72.17970645)(202.402724,72.12970998)
\curveto(202.49273331,71.88970674)(202.6077332,71.67970695)(202.747724,71.49970998)
\curveto(202.87773293,71.31970731)(203.05773275,71.17970745)(203.287724,71.07970998)
\curveto(203.34773246,71.05970757)(203.41273239,71.03970759)(203.482724,71.01970998)
\curveto(203.54273226,71.00970762)(203.6077322,70.99470764)(203.677724,70.97470998)
\moveto(209.212724,74.99470998)
\curveto(209.02272678,75.04470359)(208.81772699,75.04970358)(208.597724,75.00970998)
\curveto(208.37772743,74.97970365)(208.19772761,74.9347037)(208.057724,74.87470998)
\curveto(207.68772812,74.70470393)(207.38272842,74.44470419)(207.142724,74.09470998)
\curveto(206.9027289,73.75470488)(206.78272902,73.31970531)(206.782724,72.78970998)
\curveto(206.802729,72.75970587)(206.807729,72.71970591)(206.797724,72.66970998)
\curveto(206.77772903,72.61970601)(206.77272903,72.57970605)(206.782724,72.54970998)
\lineto(206.842724,72.27970998)
\curveto(206.85272895,72.19970643)(206.86772894,72.11970651)(206.887724,72.03970998)
\curveto(206.99772881,71.73970689)(207.14272866,71.47470716)(207.322724,71.24470998)
\curveto(207.5027283,71.02470761)(207.73272807,70.85470778)(208.012724,70.73470998)
\curveto(208.09272771,70.70470793)(208.17272763,70.67970795)(208.252724,70.65970998)
\curveto(208.33272747,70.63970799)(208.41772739,70.61970801)(208.507724,70.59970998)
\curveto(208.62772718,70.56970806)(208.77772703,70.55970807)(208.957724,70.56970998)
\curveto(209.13772667,70.58970804)(209.27772653,70.61470802)(209.377724,70.64470998)
\curveto(209.42772638,70.66470797)(209.47272633,70.67470796)(209.512724,70.67470998)
\curveto(209.54272626,70.68470795)(209.58272622,70.69970793)(209.632724,70.71970998)
\curveto(209.85272595,70.81970781)(210.05272575,70.94970768)(210.232724,71.10970998)
\curveto(210.41272539,71.27970735)(210.54772526,71.47470716)(210.637724,71.69470998)
\curveto(210.67772513,71.76470687)(210.71272509,71.85970677)(210.742724,71.97970998)
\curveto(210.83272497,72.19970643)(210.87772493,72.45470618)(210.877724,72.74470998)
\lineto(210.877724,73.02970998)
\curveto(210.85772495,73.1297055)(210.84272496,73.22470541)(210.832724,73.31470998)
\curveto(210.82272498,73.40470523)(210.802725,73.49470514)(210.772724,73.58470998)
\curveto(210.69272511,73.84470479)(210.56272524,74.08470455)(210.382724,74.30470998)
\curveto(210.19272561,74.5347041)(209.97772583,74.70470393)(209.737724,74.81470998)
\curveto(209.65772615,74.85470378)(209.57772623,74.88470375)(209.497724,74.90470998)
\curveto(209.4077264,74.9347037)(209.31272649,74.96470367)(209.212724,74.99470998)
}
}
{
\newrgbcolor{curcolor}{0 0 0}
\pscustom[linestyle=none,fillstyle=solid,fillcolor=curcolor]
{
\newpath
\moveto(210.157724,78.63431936)
\lineto(210.157724,79.26431936)
\lineto(210.157724,79.45931936)
\curveto(210.15772565,79.52931683)(210.16772564,79.58931677)(210.187724,79.63931936)
\curveto(210.22772558,79.70931665)(210.26772554,79.7593166)(210.307724,79.78931936)
\curveto(210.35772545,79.82931653)(210.42272538,79.84931651)(210.502724,79.84931936)
\curveto(210.58272522,79.8593165)(210.66772514,79.86431649)(210.757724,79.86431936)
\lineto(211.477724,79.86431936)
\curveto(211.95772385,79.86431649)(212.36772344,79.80431655)(212.707724,79.68431936)
\curveto(213.04772276,79.56431679)(213.32272248,79.36931699)(213.532724,79.09931936)
\curveto(213.58272222,79.02931733)(213.62772218,78.9593174)(213.667724,78.88931936)
\curveto(213.71772209,78.82931753)(213.76272204,78.7543176)(213.802724,78.66431936)
\curveto(213.81272199,78.64431771)(213.82272198,78.61431774)(213.832724,78.57431936)
\curveto(213.85272195,78.53431782)(213.85772195,78.48931787)(213.847724,78.43931936)
\curveto(213.81772199,78.34931801)(213.74272206,78.29431806)(213.622724,78.27431936)
\curveto(213.51272229,78.2543181)(213.41772239,78.26931809)(213.337724,78.31931936)
\curveto(213.26772254,78.34931801)(213.2027226,78.39431796)(213.142724,78.45431936)
\curveto(213.09272271,78.52431783)(213.04272276,78.58931777)(212.992724,78.64931936)
\curveto(212.94272286,78.71931764)(212.86772294,78.77931758)(212.767724,78.82931936)
\curveto(212.67772313,78.88931747)(212.58772322,78.93931742)(212.497724,78.97931936)
\curveto(212.46772334,78.99931736)(212.4077234,79.02431733)(212.317724,79.05431936)
\curveto(212.23772357,79.08431727)(212.16772364,79.08931727)(212.107724,79.06931936)
\curveto(211.96772384,79.03931732)(211.87772393,78.97931738)(211.837724,78.88931936)
\curveto(211.807724,78.80931755)(211.79272401,78.71931764)(211.792724,78.61931936)
\curveto(211.79272401,78.51931784)(211.76772404,78.43431792)(211.717724,78.36431936)
\curveto(211.64772416,78.27431808)(211.5077243,78.22931813)(211.297724,78.22931936)
\lineto(210.742724,78.22931936)
\lineto(210.517724,78.22931936)
\curveto(210.43772537,78.23931812)(210.37272543,78.2593181)(210.322724,78.28931936)
\curveto(210.24272556,78.34931801)(210.19772561,78.41931794)(210.187724,78.49931936)
\curveto(210.17772563,78.51931784)(210.17272563,78.53931782)(210.172724,78.55931936)
\curveto(210.17272563,78.58931777)(210.16772564,78.61431774)(210.157724,78.63431936)
}
}
{
\newrgbcolor{curcolor}{0 0 0}
\pscustom[linestyle=none,fillstyle=solid,fillcolor=curcolor]
{
}
}
{
\newrgbcolor{curcolor}{0 0 0}
\pscustom[linestyle=none,fillstyle=solid,fillcolor=curcolor]
{
\newpath
\moveto(201.187724,89.26463186)
\curveto(201.17773463,89.95462722)(201.29773451,90.55462662)(201.547724,91.06463186)
\curveto(201.79773401,91.58462559)(202.13273367,91.9796252)(202.552724,92.24963186)
\curveto(202.63273317,92.29962488)(202.72273308,92.34462483)(202.822724,92.38463186)
\curveto(202.91273289,92.42462475)(203.0077328,92.46962471)(203.107724,92.51963186)
\curveto(203.2077326,92.55962462)(203.3077325,92.58962459)(203.407724,92.60963186)
\curveto(203.5077323,92.62962455)(203.61273219,92.64962453)(203.722724,92.66963186)
\curveto(203.77273203,92.68962449)(203.81773199,92.69462448)(203.857724,92.68463186)
\curveto(203.89773191,92.6746245)(203.94273186,92.6796245)(203.992724,92.69963186)
\curveto(204.04273176,92.70962447)(204.12773168,92.71462446)(204.247724,92.71463186)
\curveto(204.35773145,92.71462446)(204.44273136,92.70962447)(204.502724,92.69963186)
\curveto(204.56273124,92.6796245)(204.62273118,92.66962451)(204.682724,92.66963186)
\curveto(204.74273106,92.6796245)(204.802731,92.6746245)(204.862724,92.65463186)
\curveto(205.0027308,92.61462456)(205.13773067,92.5796246)(205.267724,92.54963186)
\curveto(205.39773041,92.51962466)(205.52273028,92.4796247)(205.642724,92.42963186)
\curveto(205.78273002,92.36962481)(205.9077299,92.29962488)(206.017724,92.21963186)
\curveto(206.12772968,92.14962503)(206.23772957,92.0746251)(206.347724,91.99463186)
\lineto(206.407724,91.93463186)
\curveto(206.42772938,91.92462525)(206.44772936,91.90962527)(206.467724,91.88963186)
\curveto(206.62772918,91.76962541)(206.77272903,91.63462554)(206.902724,91.48463186)
\curveto(207.03272877,91.33462584)(207.15772865,91.174626)(207.277724,91.00463186)
\curveto(207.49772831,90.69462648)(207.7027281,90.39962678)(207.892724,90.11963186)
\curveto(208.03272777,89.88962729)(208.16772764,89.65962752)(208.297724,89.42963186)
\curveto(208.42772738,89.20962797)(208.56272724,88.98962819)(208.702724,88.76963186)
\curveto(208.87272693,88.51962866)(209.05272675,88.2796289)(209.242724,88.04963186)
\curveto(209.43272637,87.82962935)(209.65772615,87.63962954)(209.917724,87.47963186)
\curveto(209.97772583,87.43962974)(210.03772577,87.40462977)(210.097724,87.37463186)
\curveto(210.14772566,87.34462983)(210.21272559,87.31462986)(210.292724,87.28463186)
\curveto(210.36272544,87.26462991)(210.42272538,87.25962992)(210.472724,87.26963186)
\curveto(210.54272526,87.28962989)(210.59772521,87.32462985)(210.637724,87.37463186)
\curveto(210.66772514,87.42462975)(210.68772512,87.48462969)(210.697724,87.55463186)
\lineto(210.697724,87.79463186)
\lineto(210.697724,88.54463186)
\lineto(210.697724,91.34963186)
\lineto(210.697724,92.00963186)
\curveto(210.69772511,92.09962508)(210.7027251,92.18462499)(210.712724,92.26463186)
\curveto(210.71272509,92.34462483)(210.73272507,92.40962477)(210.772724,92.45963186)
\curveto(210.81272499,92.50962467)(210.88772492,92.54962463)(210.997724,92.57963186)
\curveto(211.09772471,92.61962456)(211.19772461,92.62962455)(211.297724,92.60963186)
\lineto(211.432724,92.60963186)
\curveto(211.5027243,92.58962459)(211.56272424,92.56962461)(211.612724,92.54963186)
\curveto(211.66272414,92.52962465)(211.7027241,92.49462468)(211.732724,92.44463186)
\curveto(211.77272403,92.39462478)(211.79272401,92.32462485)(211.792724,92.23463186)
\lineto(211.792724,91.96463186)
\lineto(211.792724,91.06463186)
\lineto(211.792724,87.55463186)
\lineto(211.792724,86.48963186)
\curveto(211.79272401,86.40963077)(211.79772401,86.31963086)(211.807724,86.21963186)
\curveto(211.807724,86.11963106)(211.79772401,86.03463114)(211.777724,85.96463186)
\curveto(211.7077241,85.75463142)(211.52772428,85.68963149)(211.237724,85.76963186)
\curveto(211.19772461,85.7796314)(211.16272464,85.7796314)(211.132724,85.76963186)
\curveto(211.09272471,85.76963141)(211.04772476,85.7796314)(210.997724,85.79963186)
\curveto(210.91772489,85.81963136)(210.83272497,85.83963134)(210.742724,85.85963186)
\curveto(210.65272515,85.8796313)(210.56772524,85.90463127)(210.487724,85.93463186)
\curveto(209.99772581,86.09463108)(209.58272622,86.29463088)(209.242724,86.53463186)
\curveto(208.99272681,86.71463046)(208.76772704,86.91963026)(208.567724,87.14963186)
\curveto(208.35772745,87.3796298)(208.16272764,87.61962956)(207.982724,87.86963186)
\curveto(207.802728,88.12962905)(207.63272817,88.39462878)(207.472724,88.66463186)
\curveto(207.3027285,88.94462823)(207.12772868,89.21462796)(206.947724,89.47463186)
\curveto(206.86772894,89.58462759)(206.79272901,89.68962749)(206.722724,89.78963186)
\curveto(206.65272915,89.89962728)(206.57772923,90.00962717)(206.497724,90.11963186)
\curveto(206.46772934,90.15962702)(206.43772937,90.19462698)(206.407724,90.22463186)
\curveto(206.36772944,90.26462691)(206.33772947,90.30462687)(206.317724,90.34463186)
\curveto(206.2077296,90.48462669)(206.08272972,90.60962657)(205.942724,90.71963186)
\curveto(205.91272989,90.73962644)(205.88772992,90.76462641)(205.867724,90.79463186)
\curveto(205.83772997,90.82462635)(205.80773,90.84962633)(205.777724,90.86963186)
\curveto(205.67773013,90.94962623)(205.57773023,91.01462616)(205.477724,91.06463186)
\curveto(205.37773043,91.12462605)(205.26773054,91.179626)(205.147724,91.22963186)
\curveto(205.07773073,91.25962592)(205.0027308,91.2796259)(204.922724,91.28963186)
\lineto(204.682724,91.34963186)
\lineto(204.592724,91.34963186)
\curveto(204.56273124,91.35962582)(204.53273127,91.36462581)(204.502724,91.36463186)
\curveto(204.43273137,91.38462579)(204.33773147,91.38962579)(204.217724,91.37963186)
\curveto(204.08773172,91.3796258)(203.98773182,91.36962581)(203.917724,91.34963186)
\curveto(203.83773197,91.32962585)(203.76273204,91.30962587)(203.692724,91.28963186)
\curveto(203.61273219,91.2796259)(203.53273227,91.25962592)(203.452724,91.22963186)
\curveto(203.21273259,91.11962606)(203.01273279,90.96962621)(202.852724,90.77963186)
\curveto(202.68273312,90.59962658)(202.54273326,90.3796268)(202.432724,90.11963186)
\curveto(202.41273339,90.04962713)(202.39773341,89.9796272)(202.387724,89.90963186)
\curveto(202.36773344,89.83962734)(202.34773346,89.76462741)(202.327724,89.68463186)
\curveto(202.3077335,89.60462757)(202.29773351,89.49462768)(202.297724,89.35463186)
\curveto(202.29773351,89.22462795)(202.3077335,89.11962806)(202.327724,89.03963186)
\curveto(202.33773347,88.9796282)(202.34273346,88.92462825)(202.342724,88.87463186)
\curveto(202.34273346,88.82462835)(202.35273345,88.7746284)(202.372724,88.72463186)
\curveto(202.41273339,88.62462855)(202.45273335,88.52962865)(202.492724,88.43963186)
\curveto(202.53273327,88.35962882)(202.57773323,88.2796289)(202.627724,88.19963186)
\curveto(202.64773316,88.16962901)(202.67273313,88.13962904)(202.702724,88.10963186)
\curveto(202.73273307,88.08962909)(202.75773305,88.06462911)(202.777724,88.03463186)
\lineto(202.852724,87.95963186)
\curveto(202.87273293,87.92962925)(202.89273291,87.90462927)(202.912724,87.88463186)
\lineto(203.122724,87.73463186)
\curveto(203.18273262,87.69462948)(203.24773256,87.64962953)(203.317724,87.59963186)
\curveto(203.4077324,87.53962964)(203.51273229,87.48962969)(203.632724,87.44963186)
\curveto(203.74273206,87.41962976)(203.85273195,87.38462979)(203.962724,87.34463186)
\curveto(204.07273173,87.30462987)(204.21773159,87.2796299)(204.397724,87.26963186)
\curveto(204.56773124,87.25962992)(204.69273111,87.22962995)(204.772724,87.17963186)
\curveto(204.85273095,87.12963005)(204.89773091,87.05463012)(204.907724,86.95463186)
\curveto(204.91773089,86.85463032)(204.92273088,86.74463043)(204.922724,86.62463186)
\curveto(204.92273088,86.58463059)(204.92773088,86.54463063)(204.937724,86.50463186)
\curveto(204.93773087,86.46463071)(204.93273087,86.42963075)(204.922724,86.39963186)
\curveto(204.9027309,86.34963083)(204.89273091,86.29963088)(204.892724,86.24963186)
\curveto(204.89273091,86.20963097)(204.88273092,86.16963101)(204.862724,86.12963186)
\curveto(204.802731,86.03963114)(204.66773114,85.99463118)(204.457724,85.99463186)
\lineto(204.337724,85.99463186)
\curveto(204.27773153,86.00463117)(204.21773159,86.00963117)(204.157724,86.00963186)
\curveto(204.08773172,86.01963116)(204.02273178,86.02963115)(203.962724,86.03963186)
\curveto(203.85273195,86.05963112)(203.75273205,86.0796311)(203.662724,86.09963186)
\curveto(203.56273224,86.11963106)(203.46773234,86.14963103)(203.377724,86.18963186)
\curveto(203.3077325,86.20963097)(203.24773256,86.22963095)(203.197724,86.24963186)
\lineto(203.017724,86.30963186)
\curveto(202.75773305,86.42963075)(202.51273329,86.58463059)(202.282724,86.77463186)
\curveto(202.05273375,86.9746302)(201.86773394,87.18962999)(201.727724,87.41963186)
\curveto(201.64773416,87.52962965)(201.58273422,87.64462953)(201.532724,87.76463186)
\lineto(201.382724,88.15463186)
\curveto(201.33273447,88.26462891)(201.3027345,88.3796288)(201.292724,88.49963186)
\curveto(201.27273453,88.61962856)(201.24773456,88.74462843)(201.217724,88.87463186)
\curveto(201.21773459,88.94462823)(201.21773459,89.00962817)(201.217724,89.06963186)
\curveto(201.2077346,89.12962805)(201.19773461,89.19462798)(201.187724,89.26463186)
}
}
{
\newrgbcolor{curcolor}{0 0 0}
\pscustom[linestyle=none,fillstyle=solid,fillcolor=curcolor]
{
\newpath
\moveto(206.707724,101.36424123)
\lineto(206.962724,101.36424123)
\curveto(207.04272876,101.37423353)(207.11772869,101.36923353)(207.187724,101.34924123)
\lineto(207.427724,101.34924123)
\lineto(207.592724,101.34924123)
\curveto(207.69272811,101.32923357)(207.79772801,101.31923358)(207.907724,101.31924123)
\curveto(208.0077278,101.31923358)(208.1077277,101.30923359)(208.207724,101.28924123)
\lineto(208.357724,101.28924123)
\curveto(208.49772731,101.25923364)(208.63772717,101.23923366)(208.777724,101.22924123)
\curveto(208.9077269,101.21923368)(209.03772677,101.19423371)(209.167724,101.15424123)
\curveto(209.24772656,101.13423377)(209.33272647,101.11423379)(209.422724,101.09424123)
\lineto(209.662724,101.03424123)
\lineto(209.962724,100.91424123)
\curveto(210.05272575,100.88423402)(210.14272566,100.84923405)(210.232724,100.80924123)
\curveto(210.45272535,100.70923419)(210.66772514,100.57423433)(210.877724,100.40424123)
\curveto(211.08772472,100.24423466)(211.25772455,100.06923483)(211.387724,99.87924123)
\curveto(211.42772438,99.82923507)(211.46772434,99.76923513)(211.507724,99.69924123)
\curveto(211.53772427,99.63923526)(211.57272423,99.57923532)(211.612724,99.51924123)
\curveto(211.66272414,99.43923546)(211.7027241,99.34423556)(211.732724,99.23424123)
\curveto(211.76272404,99.12423578)(211.79272401,99.01923588)(211.822724,98.91924123)
\curveto(211.86272394,98.80923609)(211.88772392,98.6992362)(211.897724,98.58924123)
\curveto(211.9077239,98.47923642)(211.92272388,98.36423654)(211.942724,98.24424123)
\curveto(211.95272385,98.2042367)(211.95272385,98.15923674)(211.942724,98.10924123)
\curveto(211.94272386,98.06923683)(211.94772386,98.02923687)(211.957724,97.98924123)
\curveto(211.96772384,97.94923695)(211.97272383,97.89423701)(211.972724,97.82424123)
\curveto(211.97272383,97.75423715)(211.96772384,97.7042372)(211.957724,97.67424123)
\curveto(211.93772387,97.62423728)(211.93272387,97.57923732)(211.942724,97.53924123)
\curveto(211.95272385,97.4992374)(211.95272385,97.46423744)(211.942724,97.43424123)
\lineto(211.942724,97.34424123)
\curveto(211.92272388,97.28423762)(211.9077239,97.21923768)(211.897724,97.14924123)
\curveto(211.89772391,97.08923781)(211.89272391,97.02423788)(211.882724,96.95424123)
\curveto(211.83272397,96.78423812)(211.78272402,96.62423828)(211.732724,96.47424123)
\curveto(211.68272412,96.32423858)(211.61772419,96.17923872)(211.537724,96.03924123)
\curveto(211.49772431,95.98923891)(211.46772434,95.93423897)(211.447724,95.87424123)
\curveto(211.41772439,95.82423908)(211.38272442,95.77423913)(211.342724,95.72424123)
\curveto(211.16272464,95.48423942)(210.94272486,95.28423962)(210.682724,95.12424123)
\curveto(210.42272538,94.96423994)(210.13772567,94.82424008)(209.827724,94.70424123)
\curveto(209.68772612,94.64424026)(209.54772626,94.5992403)(209.407724,94.56924123)
\curveto(209.25772655,94.53924036)(209.1027267,94.5042404)(208.942724,94.46424123)
\curveto(208.83272697,94.44424046)(208.72272708,94.42924047)(208.612724,94.41924123)
\curveto(208.5027273,94.40924049)(208.39272741,94.39424051)(208.282724,94.37424123)
\curveto(208.24272756,94.36424054)(208.2027276,94.35924054)(208.162724,94.35924123)
\curveto(208.12272768,94.36924053)(208.08272772,94.36924053)(208.042724,94.35924123)
\curveto(207.99272781,94.34924055)(207.94272786,94.34424056)(207.892724,94.34424123)
\lineto(207.727724,94.34424123)
\curveto(207.67772813,94.32424058)(207.62772818,94.31924058)(207.577724,94.32924123)
\curveto(207.51772829,94.33924056)(207.46272834,94.33924056)(207.412724,94.32924123)
\curveto(207.37272843,94.31924058)(207.32772848,94.31924058)(207.277724,94.32924123)
\curveto(207.22772858,94.33924056)(207.17772863,94.33424057)(207.127724,94.31424123)
\curveto(207.05772875,94.29424061)(206.98272882,94.28924061)(206.902724,94.29924123)
\curveto(206.81272899,94.30924059)(206.72772908,94.31424059)(206.647724,94.31424123)
\curveto(206.55772925,94.31424059)(206.45772935,94.30924059)(206.347724,94.29924123)
\curveto(206.22772958,94.28924061)(206.12772968,94.29424061)(206.047724,94.31424123)
\lineto(205.762724,94.31424123)
\lineto(205.132724,94.35924123)
\curveto(205.03273077,94.36924053)(204.93773087,94.37924052)(204.847724,94.38924123)
\lineto(204.547724,94.41924123)
\curveto(204.49773131,94.43924046)(204.44773136,94.44424046)(204.397724,94.43424123)
\curveto(204.33773147,94.43424047)(204.28273152,94.44424046)(204.232724,94.46424123)
\curveto(204.06273174,94.51424039)(203.89773191,94.55424035)(203.737724,94.58424123)
\curveto(203.56773224,94.61424029)(203.4077324,94.66424024)(203.257724,94.73424123)
\curveto(202.79773301,94.92423998)(202.42273338,95.14423976)(202.132724,95.39424123)
\curveto(201.84273396,95.65423925)(201.59773421,96.01423889)(201.397724,96.47424123)
\curveto(201.34773446,96.6042383)(201.31273449,96.73423817)(201.292724,96.86424123)
\curveto(201.27273453,97.0042379)(201.24773456,97.14423776)(201.217724,97.28424123)
\curveto(201.2077346,97.35423755)(201.2027346,97.41923748)(201.202724,97.47924123)
\curveto(201.2027346,97.53923736)(201.19773461,97.6042373)(201.187724,97.67424123)
\curveto(201.16773464,98.5042364)(201.31773449,99.17423573)(201.637724,99.68424123)
\curveto(201.94773386,100.19423471)(202.38773342,100.57423433)(202.957724,100.82424123)
\curveto(203.07773273,100.87423403)(203.2027326,100.91923398)(203.332724,100.95924123)
\curveto(203.46273234,100.9992339)(203.59773221,101.04423386)(203.737724,101.09424123)
\curveto(203.81773199,101.11423379)(203.9027319,101.12923377)(203.992724,101.13924123)
\lineto(204.232724,101.19924123)
\curveto(204.34273146,101.22923367)(204.45273135,101.24423366)(204.562724,101.24424123)
\curveto(204.67273113,101.25423365)(204.78273102,101.26923363)(204.892724,101.28924123)
\curveto(204.94273086,101.30923359)(204.98773082,101.31423359)(205.027724,101.30424123)
\curveto(205.06773074,101.3042336)(205.1077307,101.30923359)(205.147724,101.31924123)
\curveto(205.19773061,101.32923357)(205.25273055,101.32923357)(205.312724,101.31924123)
\curveto(205.36273044,101.31923358)(205.41273039,101.32423358)(205.462724,101.33424123)
\lineto(205.597724,101.33424123)
\curveto(205.65773015,101.35423355)(205.72773008,101.35423355)(205.807724,101.33424123)
\curveto(205.87772993,101.32423358)(205.94272986,101.32923357)(206.002724,101.34924123)
\curveto(206.03272977,101.35923354)(206.07272973,101.36423354)(206.122724,101.36424123)
\lineto(206.242724,101.36424123)
\lineto(206.707724,101.36424123)
\moveto(209.032724,99.81924123)
\curveto(208.71272709,99.91923498)(208.34772746,99.97923492)(207.937724,99.99924123)
\curveto(207.52772828,100.01923488)(207.11772869,100.02923487)(206.707724,100.02924123)
\curveto(206.27772953,100.02923487)(205.85772995,100.01923488)(205.447724,99.99924123)
\curveto(205.03773077,99.97923492)(204.65273115,99.93423497)(204.292724,99.86424123)
\curveto(203.93273187,99.79423511)(203.61273219,99.68423522)(203.332724,99.53424123)
\curveto(203.04273276,99.39423551)(202.807733,99.1992357)(202.627724,98.94924123)
\curveto(202.51773329,98.78923611)(202.43773337,98.60923629)(202.387724,98.40924123)
\curveto(202.32773348,98.20923669)(202.29773351,97.96423694)(202.297724,97.67424123)
\curveto(202.31773349,97.65423725)(202.32773348,97.61923728)(202.327724,97.56924123)
\curveto(202.31773349,97.51923738)(202.31773349,97.47923742)(202.327724,97.44924123)
\curveto(202.34773346,97.36923753)(202.36773344,97.29423761)(202.387724,97.22424123)
\curveto(202.39773341,97.16423774)(202.41773339,97.0992378)(202.447724,97.02924123)
\curveto(202.56773324,96.75923814)(202.73773307,96.53923836)(202.957724,96.36924123)
\curveto(203.16773264,96.20923869)(203.41273239,96.07423883)(203.692724,95.96424123)
\curveto(203.802732,95.91423899)(203.92273188,95.87423903)(204.052724,95.84424123)
\curveto(204.17273163,95.82423908)(204.29773151,95.7992391)(204.427724,95.76924123)
\curveto(204.47773133,95.74923915)(204.53273127,95.73923916)(204.592724,95.73924123)
\curveto(204.64273116,95.73923916)(204.69273111,95.73423917)(204.742724,95.72424123)
\curveto(204.83273097,95.71423919)(204.92773088,95.7042392)(205.027724,95.69424123)
\curveto(205.11773069,95.68423922)(205.21273059,95.67423923)(205.312724,95.66424123)
\curveto(205.39273041,95.66423924)(205.47773033,95.65923924)(205.567724,95.64924123)
\lineto(205.807724,95.64924123)
\lineto(205.987724,95.64924123)
\curveto(206.01772979,95.63923926)(206.05272975,95.63423927)(206.092724,95.63424123)
\lineto(206.227724,95.63424123)
\lineto(206.677724,95.63424123)
\curveto(206.75772905,95.63423927)(206.84272896,95.62923927)(206.932724,95.61924123)
\curveto(207.01272879,95.61923928)(207.08772872,95.62923927)(207.157724,95.64924123)
\lineto(207.427724,95.64924123)
\curveto(207.44772836,95.64923925)(207.47772833,95.64423926)(207.517724,95.63424123)
\curveto(207.54772826,95.63423927)(207.57272823,95.63923926)(207.592724,95.64924123)
\curveto(207.69272811,95.65923924)(207.79272801,95.66423924)(207.892724,95.66424123)
\curveto(207.98272782,95.67423923)(208.08272772,95.68423922)(208.192724,95.69424123)
\curveto(208.31272749,95.72423918)(208.43772737,95.73923916)(208.567724,95.73924123)
\curveto(208.68772712,95.74923915)(208.802727,95.77423913)(208.912724,95.81424123)
\curveto(209.21272659,95.89423901)(209.47772633,95.97923892)(209.707724,96.06924123)
\curveto(209.93772587,96.16923873)(210.15272565,96.31423859)(210.352724,96.50424123)
\curveto(210.55272525,96.71423819)(210.7027251,96.97923792)(210.802724,97.29924123)
\curveto(210.82272498,97.33923756)(210.83272497,97.37423753)(210.832724,97.40424123)
\curveto(210.82272498,97.44423746)(210.82772498,97.48923741)(210.847724,97.53924123)
\curveto(210.85772495,97.57923732)(210.86772494,97.64923725)(210.877724,97.74924123)
\curveto(210.88772492,97.85923704)(210.88272492,97.94423696)(210.862724,98.00424123)
\curveto(210.84272496,98.07423683)(210.83272497,98.14423676)(210.832724,98.21424123)
\curveto(210.82272498,98.28423662)(210.807725,98.34923655)(210.787724,98.40924123)
\curveto(210.72772508,98.60923629)(210.64272516,98.78923611)(210.532724,98.94924123)
\curveto(210.51272529,98.97923592)(210.49272531,99.0042359)(210.472724,99.02424123)
\lineto(210.412724,99.08424123)
\curveto(210.39272541,99.12423578)(210.35272545,99.17423573)(210.292724,99.23424123)
\curveto(210.15272565,99.33423557)(210.02272578,99.41923548)(209.902724,99.48924123)
\curveto(209.78272602,99.55923534)(209.63772617,99.62923527)(209.467724,99.69924123)
\curveto(209.39772641,99.72923517)(209.32772648,99.74923515)(209.257724,99.75924123)
\curveto(209.18772662,99.77923512)(209.11272669,99.7992351)(209.032724,99.81924123)
}
}
{
\newrgbcolor{curcolor}{0 0 0}
\pscustom[linestyle=none,fillstyle=solid,fillcolor=curcolor]
{
\newpath
\moveto(201.187724,106.77385061)
\curveto(201.18773462,106.87384575)(201.19773461,106.96884566)(201.217724,107.05885061)
\curveto(201.22773458,107.14884548)(201.25773455,107.21384541)(201.307724,107.25385061)
\curveto(201.38773442,107.31384531)(201.49273431,107.34384528)(201.622724,107.34385061)
\lineto(202.012724,107.34385061)
\lineto(203.512724,107.34385061)
\lineto(209.902724,107.34385061)
\lineto(211.072724,107.34385061)
\lineto(211.387724,107.34385061)
\curveto(211.48772432,107.35384527)(211.56772424,107.33884529)(211.627724,107.29885061)
\curveto(211.7077241,107.24884538)(211.75772405,107.17384545)(211.777724,107.07385061)
\curveto(211.78772402,106.98384564)(211.79272401,106.87384575)(211.792724,106.74385061)
\lineto(211.792724,106.51885061)
\curveto(211.77272403,106.43884619)(211.75772405,106.36884626)(211.747724,106.30885061)
\curveto(211.72772408,106.24884638)(211.68772412,106.19884643)(211.627724,106.15885061)
\curveto(211.56772424,106.11884651)(211.49272431,106.09884653)(211.402724,106.09885061)
\lineto(211.102724,106.09885061)
\lineto(210.007724,106.09885061)
\lineto(204.667724,106.09885061)
\curveto(204.57773123,106.07884655)(204.5027313,106.06384656)(204.442724,106.05385061)
\curveto(204.37273143,106.05384657)(204.31273149,106.0238466)(204.262724,105.96385061)
\curveto(204.21273159,105.89384673)(204.18773162,105.80384682)(204.187724,105.69385061)
\curveto(204.17773163,105.59384703)(204.17273163,105.48384714)(204.172724,105.36385061)
\lineto(204.172724,104.22385061)
\lineto(204.172724,103.72885061)
\curveto(204.16273164,103.56884906)(204.1027317,103.45884917)(203.992724,103.39885061)
\curveto(203.96273184,103.37884925)(203.93273187,103.36884926)(203.902724,103.36885061)
\curveto(203.86273194,103.36884926)(203.81773199,103.36384926)(203.767724,103.35385061)
\curveto(203.64773216,103.33384929)(203.53773227,103.33884929)(203.437724,103.36885061)
\curveto(203.33773247,103.40884922)(203.26773254,103.46384916)(203.227724,103.53385061)
\curveto(203.17773263,103.61384901)(203.15273265,103.73384889)(203.152724,103.89385061)
\curveto(203.15273265,104.05384857)(203.13773267,104.18884844)(203.107724,104.29885061)
\curveto(203.09773271,104.34884828)(203.09273271,104.40384822)(203.092724,104.46385061)
\curveto(203.08273272,104.5238481)(203.06773274,104.58384804)(203.047724,104.64385061)
\curveto(202.99773281,104.79384783)(202.94773286,104.93884769)(202.897724,105.07885061)
\curveto(202.83773297,105.21884741)(202.76773304,105.35384727)(202.687724,105.48385061)
\curveto(202.59773321,105.623847)(202.49273331,105.74384688)(202.372724,105.84385061)
\curveto(202.25273355,105.94384668)(202.12273368,106.03884659)(201.982724,106.12885061)
\curveto(201.88273392,106.18884644)(201.77273403,106.23384639)(201.652724,106.26385061)
\curveto(201.53273427,106.30384632)(201.42773438,106.35384627)(201.337724,106.41385061)
\curveto(201.27773453,106.46384616)(201.23773457,106.53384609)(201.217724,106.62385061)
\curveto(201.2077346,106.64384598)(201.2027346,106.66884596)(201.202724,106.69885061)
\curveto(201.2027346,106.7288459)(201.19773461,106.75384587)(201.187724,106.77385061)
}
}
{
\newrgbcolor{curcolor}{0 0 0}
\pscustom[linestyle=none,fillstyle=solid,fillcolor=curcolor]
{
\newpath
\moveto(201.187724,115.12345998)
\curveto(201.18773462,115.22345513)(201.19773461,115.31845503)(201.217724,115.40845998)
\curveto(201.22773458,115.49845485)(201.25773455,115.56345479)(201.307724,115.60345998)
\curveto(201.38773442,115.66345469)(201.49273431,115.69345466)(201.622724,115.69345998)
\lineto(202.012724,115.69345998)
\lineto(203.512724,115.69345998)
\lineto(209.902724,115.69345998)
\lineto(211.072724,115.69345998)
\lineto(211.387724,115.69345998)
\curveto(211.48772432,115.70345465)(211.56772424,115.68845466)(211.627724,115.64845998)
\curveto(211.7077241,115.59845475)(211.75772405,115.52345483)(211.777724,115.42345998)
\curveto(211.78772402,115.33345502)(211.79272401,115.22345513)(211.792724,115.09345998)
\lineto(211.792724,114.86845998)
\curveto(211.77272403,114.78845556)(211.75772405,114.71845563)(211.747724,114.65845998)
\curveto(211.72772408,114.59845575)(211.68772412,114.5484558)(211.627724,114.50845998)
\curveto(211.56772424,114.46845588)(211.49272431,114.4484559)(211.402724,114.44845998)
\lineto(211.102724,114.44845998)
\lineto(210.007724,114.44845998)
\lineto(204.667724,114.44845998)
\curveto(204.57773123,114.42845592)(204.5027313,114.41345594)(204.442724,114.40345998)
\curveto(204.37273143,114.40345595)(204.31273149,114.37345598)(204.262724,114.31345998)
\curveto(204.21273159,114.24345611)(204.18773162,114.1534562)(204.187724,114.04345998)
\curveto(204.17773163,113.94345641)(204.17273163,113.83345652)(204.172724,113.71345998)
\lineto(204.172724,112.57345998)
\lineto(204.172724,112.07845998)
\curveto(204.16273164,111.91845843)(204.1027317,111.80845854)(203.992724,111.74845998)
\curveto(203.96273184,111.72845862)(203.93273187,111.71845863)(203.902724,111.71845998)
\curveto(203.86273194,111.71845863)(203.81773199,111.71345864)(203.767724,111.70345998)
\curveto(203.64773216,111.68345867)(203.53773227,111.68845866)(203.437724,111.71845998)
\curveto(203.33773247,111.75845859)(203.26773254,111.81345854)(203.227724,111.88345998)
\curveto(203.17773263,111.96345839)(203.15273265,112.08345827)(203.152724,112.24345998)
\curveto(203.15273265,112.40345795)(203.13773267,112.53845781)(203.107724,112.64845998)
\curveto(203.09773271,112.69845765)(203.09273271,112.7534576)(203.092724,112.81345998)
\curveto(203.08273272,112.87345748)(203.06773274,112.93345742)(203.047724,112.99345998)
\curveto(202.99773281,113.14345721)(202.94773286,113.28845706)(202.897724,113.42845998)
\curveto(202.83773297,113.56845678)(202.76773304,113.70345665)(202.687724,113.83345998)
\curveto(202.59773321,113.97345638)(202.49273331,114.09345626)(202.372724,114.19345998)
\curveto(202.25273355,114.29345606)(202.12273368,114.38845596)(201.982724,114.47845998)
\curveto(201.88273392,114.53845581)(201.77273403,114.58345577)(201.652724,114.61345998)
\curveto(201.53273427,114.6534557)(201.42773438,114.70345565)(201.337724,114.76345998)
\curveto(201.27773453,114.81345554)(201.23773457,114.88345547)(201.217724,114.97345998)
\curveto(201.2077346,114.99345536)(201.2027346,115.01845533)(201.202724,115.04845998)
\curveto(201.2027346,115.07845527)(201.19773461,115.10345525)(201.187724,115.12345998)
}
}
{
\newrgbcolor{curcolor}{0 0 0}
\pscustom[linestyle=none,fillstyle=solid,fillcolor=curcolor]
{
\newpath
\moveto(117.46738403,31.67142873)
\lineto(117.46738403,32.58642873)
\curveto(117.46739473,32.68642608)(117.46739473,32.78142599)(117.46738403,32.87142873)
\curveto(117.46739473,32.96142581)(117.48739471,33.03642573)(117.52738403,33.09642873)
\curveto(117.58739461,33.18642558)(117.66739453,33.24642552)(117.76738403,33.27642873)
\curveto(117.86739433,33.31642545)(117.97239422,33.36142541)(118.08238403,33.41142873)
\curveto(118.27239392,33.49142528)(118.46239373,33.56142521)(118.65238403,33.62142873)
\curveto(118.84239335,33.69142508)(119.03239316,33.766425)(119.22238403,33.84642873)
\curveto(119.40239279,33.91642485)(119.58739261,33.98142479)(119.77738403,34.04142873)
\curveto(119.95739224,34.10142467)(120.13739206,34.1714246)(120.31738403,34.25142873)
\curveto(120.45739174,34.31142446)(120.60239159,34.3664244)(120.75238403,34.41642873)
\curveto(120.90239129,34.4664243)(121.04739115,34.52142425)(121.18738403,34.58142873)
\curveto(121.63739056,34.76142401)(122.0923901,34.93142384)(122.55238403,35.09142873)
\curveto(123.00238919,35.25142352)(123.45238874,35.42142335)(123.90238403,35.60142873)
\curveto(123.95238824,35.62142315)(124.00238819,35.63642313)(124.05238403,35.64642873)
\lineto(124.20238403,35.70642873)
\curveto(124.42238777,35.79642297)(124.64738755,35.88142289)(124.87738403,35.96142873)
\curveto(125.0973871,36.04142273)(125.31738688,36.12642264)(125.53738403,36.21642873)
\curveto(125.62738657,36.25642251)(125.73738646,36.29642247)(125.86738403,36.33642873)
\curveto(125.98738621,36.37642239)(126.05738614,36.44142233)(126.07738403,36.53142873)
\curveto(126.08738611,36.5714222)(126.08738611,36.60142217)(126.07738403,36.62142873)
\lineto(126.01738403,36.68142873)
\curveto(125.96738623,36.73142204)(125.91238628,36.766422)(125.85238403,36.78642873)
\curveto(125.7923864,36.81642195)(125.72738647,36.84642192)(125.65738403,36.87642873)
\lineto(125.02738403,37.11642873)
\curveto(124.80738739,37.19642157)(124.5923876,37.27642149)(124.38238403,37.35642873)
\lineto(124.23238403,37.41642873)
\lineto(124.05238403,37.47642873)
\curveto(123.86238833,37.55642121)(123.67238852,37.62642114)(123.48238403,37.68642873)
\curveto(123.28238891,37.75642101)(123.08238911,37.83142094)(122.88238403,37.91142873)
\curveto(122.30238989,38.15142062)(121.71739048,38.3714204)(121.12738403,38.57142873)
\curveto(120.53739166,38.78141999)(119.95239224,39.00641976)(119.37238403,39.24642873)
\curveto(119.17239302,39.32641944)(118.96739323,39.40141937)(118.75738403,39.47142873)
\curveto(118.54739365,39.55141922)(118.34239385,39.63141914)(118.14238403,39.71142873)
\curveto(118.06239413,39.75141902)(117.96239423,39.78641898)(117.84238403,39.81642873)
\curveto(117.72239447,39.85641891)(117.63739456,39.91141886)(117.58738403,39.98142873)
\curveto(117.54739465,40.04141873)(117.51739468,40.11641865)(117.49738403,40.20642873)
\curveto(117.47739472,40.30641846)(117.46739473,40.41641835)(117.46738403,40.53642873)
\curveto(117.45739474,40.65641811)(117.45739474,40.77641799)(117.46738403,40.89642873)
\curveto(117.46739473,41.01641775)(117.46739473,41.12641764)(117.46738403,41.22642873)
\curveto(117.46739473,41.31641745)(117.46739473,41.40641736)(117.46738403,41.49642873)
\curveto(117.46739473,41.59641717)(117.48739471,41.6714171)(117.52738403,41.72142873)
\curveto(117.57739462,41.81141696)(117.66739453,41.86141691)(117.79738403,41.87142873)
\curveto(117.92739427,41.88141689)(118.06739413,41.88641688)(118.21738403,41.88642873)
\lineto(119.86738403,41.88642873)
\lineto(126.13738403,41.88642873)
\lineto(127.39738403,41.88642873)
\curveto(127.50738469,41.88641688)(127.61738458,41.88641688)(127.72738403,41.88642873)
\curveto(127.83738436,41.89641687)(127.92238427,41.87641689)(127.98238403,41.82642873)
\curveto(128.04238415,41.79641697)(128.08238411,41.75141702)(128.10238403,41.69142873)
\curveto(128.11238408,41.63141714)(128.12738407,41.56141721)(128.14738403,41.48142873)
\lineto(128.14738403,41.24142873)
\lineto(128.14738403,40.88142873)
\curveto(128.13738406,40.771418)(128.0923841,40.69141808)(128.01238403,40.64142873)
\curveto(127.98238421,40.62141815)(127.95238424,40.60641816)(127.92238403,40.59642873)
\curveto(127.88238431,40.59641817)(127.83738436,40.58641818)(127.78738403,40.56642873)
\lineto(127.62238403,40.56642873)
\curveto(127.56238463,40.55641821)(127.4923847,40.55141822)(127.41238403,40.55142873)
\curveto(127.33238486,40.56141821)(127.25738494,40.5664182)(127.18738403,40.56642873)
\lineto(126.34738403,40.56642873)
\lineto(121.92238403,40.56642873)
\curveto(121.67239052,40.5664182)(121.42239077,40.5664182)(121.17238403,40.56642873)
\curveto(120.91239128,40.5664182)(120.66239153,40.56141821)(120.42238403,40.55142873)
\curveto(120.32239187,40.55141822)(120.21239198,40.54641822)(120.09238403,40.53642873)
\curveto(119.97239222,40.52641824)(119.91239228,40.4714183)(119.91238403,40.37142873)
\lineto(119.92738403,40.37142873)
\curveto(119.94739225,40.30141847)(120.01239218,40.24141853)(120.12238403,40.19142873)
\curveto(120.23239196,40.15141862)(120.32739187,40.11641865)(120.40738403,40.08642873)
\curveto(120.57739162,40.01641875)(120.75239144,39.95141882)(120.93238403,39.89142873)
\curveto(121.10239109,39.83141894)(121.27239092,39.76141901)(121.44238403,39.68142873)
\curveto(121.4923907,39.66141911)(121.53739066,39.64641912)(121.57738403,39.63642873)
\curveto(121.61739058,39.62641914)(121.66239053,39.61141916)(121.71238403,39.59142873)
\curveto(121.8923903,39.51141926)(122.07739012,39.44141933)(122.26738403,39.38142873)
\curveto(122.44738975,39.33141944)(122.62738957,39.2664195)(122.80738403,39.18642873)
\curveto(122.95738924,39.11641965)(123.11238908,39.05641971)(123.27238403,39.00642873)
\curveto(123.42238877,38.95641981)(123.57238862,38.90141987)(123.72238403,38.84142873)
\curveto(124.192388,38.64142013)(124.66738753,38.46142031)(125.14738403,38.30142873)
\curveto(125.61738658,38.14142063)(126.08238611,37.9664208)(126.54238403,37.77642873)
\curveto(126.72238547,37.69642107)(126.90238529,37.62642114)(127.08238403,37.56642873)
\curveto(127.26238493,37.50642126)(127.44238475,37.44142133)(127.62238403,37.37142873)
\curveto(127.73238446,37.32142145)(127.83738436,37.2714215)(127.93738403,37.22142873)
\curveto(128.02738417,37.18142159)(128.0923841,37.09642167)(128.13238403,36.96642873)
\curveto(128.14238405,36.94642182)(128.14738405,36.92142185)(128.14738403,36.89142873)
\curveto(128.13738406,36.8714219)(128.13738406,36.84642192)(128.14738403,36.81642873)
\curveto(128.15738404,36.78642198)(128.16238403,36.75142202)(128.16238403,36.71142873)
\curveto(128.15238404,36.6714221)(128.14738405,36.63142214)(128.14738403,36.59142873)
\lineto(128.14738403,36.29142873)
\curveto(128.14738405,36.19142258)(128.12238407,36.11142266)(128.07238403,36.05142873)
\curveto(128.02238417,35.9714228)(127.95238424,35.91142286)(127.86238403,35.87142873)
\curveto(127.76238443,35.84142293)(127.66238453,35.80142297)(127.56238403,35.75142873)
\curveto(127.36238483,35.6714231)(127.15738504,35.59142318)(126.94738403,35.51142873)
\curveto(126.72738547,35.44142333)(126.51738568,35.3664234)(126.31738403,35.28642873)
\curveto(126.13738606,35.20642356)(125.95738624,35.13642363)(125.77738403,35.07642873)
\curveto(125.58738661,35.02642374)(125.40238679,34.96142381)(125.22238403,34.88142873)
\curveto(124.66238753,34.65142412)(124.0973881,34.43642433)(123.52738403,34.23642873)
\curveto(122.95738924,34.03642473)(122.3923898,33.82142495)(121.83238403,33.59142873)
\lineto(121.20238403,33.35142873)
\curveto(120.98239121,33.28142549)(120.77239142,33.20642556)(120.57238403,33.12642873)
\curveto(120.46239173,33.07642569)(120.35739184,33.03142574)(120.25738403,32.99142873)
\curveto(120.14739205,32.96142581)(120.05239214,32.91142586)(119.97238403,32.84142873)
\curveto(119.95239224,32.83142594)(119.94239225,32.82142595)(119.94238403,32.81142873)
\lineto(119.91238403,32.78142873)
\lineto(119.91238403,32.70642873)
\lineto(119.94238403,32.67642873)
\curveto(119.94239225,32.6664261)(119.94739225,32.65642611)(119.95738403,32.64642873)
\curveto(120.00739219,32.62642614)(120.06239213,32.61642615)(120.12238403,32.61642873)
\curveto(120.18239201,32.61642615)(120.24239195,32.60642616)(120.30238403,32.58642873)
\lineto(120.46738403,32.58642873)
\curveto(120.52739167,32.5664262)(120.5923916,32.56142621)(120.66238403,32.57142873)
\curveto(120.73239146,32.58142619)(120.80239139,32.58642618)(120.87238403,32.58642873)
\lineto(121.68238403,32.58642873)
\lineto(126.24238403,32.58642873)
\lineto(127.42738403,32.58642873)
\curveto(127.53738466,32.58642618)(127.64738455,32.58142619)(127.75738403,32.57142873)
\curveto(127.86738433,32.5714262)(127.95238424,32.54642622)(128.01238403,32.49642873)
\curveto(128.0923841,32.44642632)(128.13738406,32.35642641)(128.14738403,32.22642873)
\lineto(128.14738403,31.83642873)
\lineto(128.14738403,31.64142873)
\curveto(128.14738405,31.59142718)(128.13738406,31.54142723)(128.11738403,31.49142873)
\curveto(128.07738412,31.36142741)(127.9923842,31.28642748)(127.86238403,31.26642873)
\curveto(127.73238446,31.25642751)(127.58238461,31.25142752)(127.41238403,31.25142873)
\lineto(125.67238403,31.25142873)
\lineto(119.67238403,31.25142873)
\lineto(118.26238403,31.25142873)
\curveto(118.15239404,31.25142752)(118.03739416,31.24642752)(117.91738403,31.23642873)
\curveto(117.7973944,31.23642753)(117.70239449,31.26142751)(117.63238403,31.31142873)
\curveto(117.57239462,31.35142742)(117.52239467,31.42642734)(117.48238403,31.53642873)
\curveto(117.47239472,31.55642721)(117.47239472,31.57642719)(117.48238403,31.59642873)
\curveto(117.48239471,31.62642714)(117.47739472,31.65142712)(117.46738403,31.67142873)
}
}
{
\newrgbcolor{curcolor}{0 0 0}
\pscustom[linestyle=none,fillstyle=solid,fillcolor=curcolor]
{
\newpath
\moveto(127.59238403,50.87353811)
\curveto(127.75238444,50.90353028)(127.88738431,50.88853029)(127.99738403,50.82853811)
\curveto(128.0973841,50.76853041)(128.17238402,50.68853049)(128.22238403,50.58853811)
\curveto(128.24238395,50.53853064)(128.25238394,50.4835307)(128.25238403,50.42353811)
\curveto(128.25238394,50.37353081)(128.26238393,50.31853086)(128.28238403,50.25853811)
\curveto(128.33238386,50.03853114)(128.31738388,49.81853136)(128.23738403,49.59853811)
\curveto(128.16738403,49.38853179)(128.07738412,49.24353194)(127.96738403,49.16353811)
\curveto(127.8973843,49.11353207)(127.81738438,49.06853211)(127.72738403,49.02853811)
\curveto(127.62738457,48.98853219)(127.54738465,48.93853224)(127.48738403,48.87853811)
\curveto(127.46738473,48.85853232)(127.44738475,48.83353235)(127.42738403,48.80353811)
\curveto(127.40738479,48.7835324)(127.40238479,48.75353243)(127.41238403,48.71353811)
\curveto(127.44238475,48.60353258)(127.4973847,48.49853268)(127.57738403,48.39853811)
\curveto(127.65738454,48.30853287)(127.72738447,48.21853296)(127.78738403,48.12853811)
\curveto(127.86738433,47.99853318)(127.94238425,47.85853332)(128.01238403,47.70853811)
\curveto(128.07238412,47.55853362)(128.12738407,47.39853378)(128.17738403,47.22853811)
\curveto(128.20738399,47.12853405)(128.22738397,47.01853416)(128.23738403,46.89853811)
\curveto(128.24738395,46.78853439)(128.26238393,46.6785345)(128.28238403,46.56853811)
\curveto(128.2923839,46.51853466)(128.2973839,46.47353471)(128.29738403,46.43353811)
\lineto(128.29738403,46.32853811)
\curveto(128.31738388,46.21853496)(128.31738388,46.11353507)(128.29738403,46.01353811)
\lineto(128.29738403,45.87853811)
\curveto(128.28738391,45.82853535)(128.28238391,45.7785354)(128.28238403,45.72853811)
\curveto(128.28238391,45.6785355)(128.27238392,45.63353555)(128.25238403,45.59353811)
\curveto(128.24238395,45.55353563)(128.23738396,45.51853566)(128.23738403,45.48853811)
\curveto(128.24738395,45.46853571)(128.24738395,45.44353574)(128.23738403,45.41353811)
\lineto(128.17738403,45.17353811)
\curveto(128.16738403,45.09353609)(128.14738405,45.01853616)(128.11738403,44.94853811)
\curveto(127.98738421,44.64853653)(127.84238435,44.40353678)(127.68238403,44.21353811)
\curveto(127.51238468,44.03353715)(127.27738492,43.8835373)(126.97738403,43.76353811)
\curveto(126.75738544,43.67353751)(126.4923857,43.62853755)(126.18238403,43.62853811)
\lineto(125.86738403,43.62853811)
\curveto(125.81738638,43.63853754)(125.76738643,43.64353754)(125.71738403,43.64353811)
\lineto(125.53738403,43.67353811)
\lineto(125.20738403,43.79353811)
\curveto(125.0973871,43.83353735)(124.9973872,43.8835373)(124.90738403,43.94353811)
\curveto(124.61738758,44.12353706)(124.40238779,44.36853681)(124.26238403,44.67853811)
\curveto(124.12238807,44.98853619)(123.9973882,45.32853585)(123.88738403,45.69853811)
\curveto(123.84738835,45.83853534)(123.81738838,45.9835352)(123.79738403,46.13353811)
\curveto(123.77738842,46.2835349)(123.75238844,46.43353475)(123.72238403,46.58353811)
\curveto(123.70238849,46.65353453)(123.6923885,46.71853446)(123.69238403,46.77853811)
\curveto(123.6923885,46.84853433)(123.68238851,46.92353426)(123.66238403,47.00353811)
\curveto(123.64238855,47.07353411)(123.63238856,47.14353404)(123.63238403,47.21353811)
\curveto(123.62238857,47.2835339)(123.60738859,47.35853382)(123.58738403,47.43853811)
\curveto(123.52738867,47.68853349)(123.47738872,47.92353326)(123.43738403,48.14353811)
\curveto(123.38738881,48.36353282)(123.27238892,48.53853264)(123.09238403,48.66853811)
\curveto(123.01238918,48.72853245)(122.91238928,48.7785324)(122.79238403,48.81853811)
\curveto(122.66238953,48.85853232)(122.52238967,48.85853232)(122.37238403,48.81853811)
\curveto(122.13239006,48.75853242)(121.94239025,48.66853251)(121.80238403,48.54853811)
\curveto(121.66239053,48.43853274)(121.55239064,48.2785329)(121.47238403,48.06853811)
\curveto(121.42239077,47.94853323)(121.38739081,47.80353338)(121.36738403,47.63353811)
\curveto(121.34739085,47.47353371)(121.33739086,47.30353388)(121.33738403,47.12353811)
\curveto(121.33739086,46.94353424)(121.34739085,46.76853441)(121.36738403,46.59853811)
\curveto(121.38739081,46.42853475)(121.41739078,46.2835349)(121.45738403,46.16353811)
\curveto(121.51739068,45.99353519)(121.60239059,45.82853535)(121.71238403,45.66853811)
\curveto(121.77239042,45.58853559)(121.85239034,45.51353567)(121.95238403,45.44353811)
\curveto(122.04239015,45.3835358)(122.14239005,45.32853585)(122.25238403,45.27853811)
\curveto(122.33238986,45.24853593)(122.41738978,45.21853596)(122.50738403,45.18853811)
\curveto(122.5973896,45.16853601)(122.66738953,45.12353606)(122.71738403,45.05353811)
\curveto(122.74738945,45.01353617)(122.77238942,44.94353624)(122.79238403,44.84353811)
\curveto(122.80238939,44.75353643)(122.80738939,44.65853652)(122.80738403,44.55853811)
\curveto(122.80738939,44.45853672)(122.80238939,44.35853682)(122.79238403,44.25853811)
\curveto(122.77238942,44.16853701)(122.74738945,44.10353708)(122.71738403,44.06353811)
\curveto(122.68738951,44.02353716)(122.63738956,43.99353719)(122.56738403,43.97353811)
\curveto(122.4973897,43.95353723)(122.42238977,43.95353723)(122.34238403,43.97353811)
\curveto(122.21238998,44.00353718)(122.0923901,44.03353715)(121.98238403,44.06353811)
\curveto(121.86239033,44.10353708)(121.74739045,44.14853703)(121.63738403,44.19853811)
\curveto(121.28739091,44.38853679)(121.01739118,44.62853655)(120.82738403,44.91853811)
\curveto(120.62739157,45.20853597)(120.46739173,45.56853561)(120.34738403,45.99853811)
\curveto(120.32739187,46.09853508)(120.31239188,46.19853498)(120.30238403,46.29853811)
\curveto(120.2923919,46.40853477)(120.27739192,46.51853466)(120.25738403,46.62853811)
\curveto(120.24739195,46.66853451)(120.24739195,46.73353445)(120.25738403,46.82353811)
\curveto(120.25739194,46.91353427)(120.24739195,46.96853421)(120.22738403,46.98853811)
\curveto(120.21739198,47.68853349)(120.2973919,48.29853288)(120.46738403,48.81853811)
\curveto(120.63739156,49.33853184)(120.96239123,49.70353148)(121.44238403,49.91353811)
\curveto(121.64239055,50.00353118)(121.87739032,50.05353113)(122.14738403,50.06353811)
\curveto(122.40738979,50.0835311)(122.68238951,50.09353109)(122.97238403,50.09353811)
\lineto(126.28738403,50.09353811)
\curveto(126.42738577,50.09353109)(126.56238563,50.09853108)(126.69238403,50.10853811)
\curveto(126.82238537,50.11853106)(126.92738527,50.14853103)(127.00738403,50.19853811)
\curveto(127.07738512,50.24853093)(127.12738507,50.31353087)(127.15738403,50.39353811)
\curveto(127.197385,50.4835307)(127.22738497,50.56853061)(127.24738403,50.64853811)
\curveto(127.25738494,50.72853045)(127.30238489,50.78853039)(127.38238403,50.82853811)
\curveto(127.41238478,50.84853033)(127.44238475,50.85853032)(127.47238403,50.85853811)
\curveto(127.50238469,50.85853032)(127.54238465,50.86353032)(127.59238403,50.87353811)
\moveto(125.92738403,48.72853811)
\curveto(125.78738641,48.78853239)(125.62738657,48.81853236)(125.44738403,48.81853811)
\curveto(125.25738694,48.82853235)(125.06238713,48.83353235)(124.86238403,48.83353811)
\curveto(124.75238744,48.83353235)(124.65238754,48.82853235)(124.56238403,48.81853811)
\curveto(124.47238772,48.80853237)(124.40238779,48.76853241)(124.35238403,48.69853811)
\curveto(124.33238786,48.66853251)(124.32238787,48.59853258)(124.32238403,48.48853811)
\curveto(124.34238785,48.46853271)(124.35238784,48.43353275)(124.35238403,48.38353811)
\curveto(124.35238784,48.33353285)(124.36238783,48.28853289)(124.38238403,48.24853811)
\curveto(124.40238779,48.16853301)(124.42238777,48.0785331)(124.44238403,47.97853811)
\lineto(124.50238403,47.67853811)
\curveto(124.50238769,47.64853353)(124.50738769,47.61353357)(124.51738403,47.57353811)
\lineto(124.51738403,47.46853811)
\curveto(124.55738764,47.31853386)(124.58238761,47.15353403)(124.59238403,46.97353811)
\curveto(124.5923876,46.80353438)(124.61238758,46.64353454)(124.65238403,46.49353811)
\curveto(124.67238752,46.41353477)(124.6923875,46.33853484)(124.71238403,46.26853811)
\curveto(124.72238747,46.20853497)(124.73738746,46.13853504)(124.75738403,46.05853811)
\curveto(124.80738739,45.89853528)(124.87238732,45.74853543)(124.95238403,45.60853811)
\curveto(125.02238717,45.46853571)(125.11238708,45.34853583)(125.22238403,45.24853811)
\curveto(125.33238686,45.14853603)(125.46738673,45.07353611)(125.62738403,45.02353811)
\curveto(125.77738642,44.97353621)(125.96238623,44.95353623)(126.18238403,44.96353811)
\curveto(126.28238591,44.96353622)(126.37738582,44.9785362)(126.46738403,45.00853811)
\curveto(126.54738565,45.04853613)(126.62238557,45.09353609)(126.69238403,45.14353811)
\curveto(126.80238539,45.22353596)(126.8973853,45.32853585)(126.97738403,45.45853811)
\curveto(127.04738515,45.58853559)(127.10738509,45.72853545)(127.15738403,45.87853811)
\curveto(127.16738503,45.92853525)(127.17238502,45.9785352)(127.17238403,46.02853811)
\curveto(127.17238502,46.0785351)(127.17738502,46.12853505)(127.18738403,46.17853811)
\curveto(127.20738499,46.24853493)(127.22238497,46.33353485)(127.23238403,46.43353811)
\curveto(127.23238496,46.54353464)(127.22238497,46.63353455)(127.20238403,46.70353811)
\curveto(127.18238501,46.76353442)(127.17738502,46.82353436)(127.18738403,46.88353811)
\curveto(127.18738501,46.94353424)(127.17738502,47.00353418)(127.15738403,47.06353811)
\curveto(127.13738506,47.14353404)(127.12238507,47.21853396)(127.11238403,47.28853811)
\curveto(127.10238509,47.36853381)(127.08238511,47.44353374)(127.05238403,47.51353811)
\curveto(126.93238526,47.80353338)(126.78738541,48.04853313)(126.61738403,48.24853811)
\curveto(126.44738575,48.45853272)(126.21738598,48.61853256)(125.92738403,48.72853811)
}
}
{
\newrgbcolor{curcolor}{0 0 0}
\pscustom[linestyle=none,fillstyle=solid,fillcolor=curcolor]
{
\newpath
\moveto(120.24238403,55.69017873)
\curveto(120.24239195,55.92017394)(120.30239189,56.05017381)(120.42238403,56.08017873)
\curveto(120.53239166,56.11017375)(120.6973915,56.12517374)(120.91738403,56.12517873)
\lineto(121.20238403,56.12517873)
\curveto(121.2923909,56.12517374)(121.36739083,56.10017376)(121.42738403,56.05017873)
\curveto(121.50739069,55.99017387)(121.55239064,55.90517396)(121.56238403,55.79517873)
\curveto(121.56239063,55.68517418)(121.57739062,55.57517429)(121.60738403,55.46517873)
\curveto(121.63739056,55.32517454)(121.66739053,55.19017467)(121.69738403,55.06017873)
\curveto(121.72739047,54.94017492)(121.76739043,54.82517504)(121.81738403,54.71517873)
\curveto(121.94739025,54.42517544)(122.12739007,54.19017567)(122.35738403,54.01017873)
\curveto(122.57738962,53.83017603)(122.83238936,53.67517619)(123.12238403,53.54517873)
\curveto(123.23238896,53.50517636)(123.34738885,53.47517639)(123.46738403,53.45517873)
\curveto(123.57738862,53.43517643)(123.6923885,53.41017645)(123.81238403,53.38017873)
\curveto(123.86238833,53.37017649)(123.91238828,53.3651765)(123.96238403,53.36517873)
\curveto(124.01238818,53.37517649)(124.06238813,53.37517649)(124.11238403,53.36517873)
\curveto(124.23238796,53.33517653)(124.37238782,53.32017654)(124.53238403,53.32017873)
\curveto(124.68238751,53.33017653)(124.82738737,53.33517653)(124.96738403,53.33517873)
\lineto(126.81238403,53.33517873)
\lineto(127.15738403,53.33517873)
\curveto(127.27738492,53.33517653)(127.3923848,53.33017653)(127.50238403,53.32017873)
\curveto(127.61238458,53.31017655)(127.70738449,53.30517656)(127.78738403,53.30517873)
\curveto(127.86738433,53.31517655)(127.93738426,53.29517657)(127.99738403,53.24517873)
\curveto(128.06738413,53.19517667)(128.10738409,53.11517675)(128.11738403,53.00517873)
\curveto(128.12738407,52.90517696)(128.13238406,52.79517707)(128.13238403,52.67517873)
\lineto(128.13238403,52.40517873)
\curveto(128.11238408,52.35517751)(128.0973841,52.30517756)(128.08738403,52.25517873)
\curveto(128.06738413,52.21517765)(128.04238415,52.18517768)(128.01238403,52.16517873)
\curveto(127.94238425,52.11517775)(127.85738434,52.08517778)(127.75738403,52.07517873)
\lineto(127.42738403,52.07517873)
\lineto(126.27238403,52.07517873)
\lineto(122.11738403,52.07517873)
\lineto(121.08238403,52.07517873)
\lineto(120.78238403,52.07517873)
\curveto(120.68239151,52.08517778)(120.5973916,52.11517775)(120.52738403,52.16517873)
\curveto(120.48739171,52.19517767)(120.45739174,52.24517762)(120.43738403,52.31517873)
\curveto(120.41739178,52.39517747)(120.40739179,52.48017738)(120.40738403,52.57017873)
\curveto(120.3973918,52.6601772)(120.3973918,52.75017711)(120.40738403,52.84017873)
\curveto(120.41739178,52.93017693)(120.43239176,53.00017686)(120.45238403,53.05017873)
\curveto(120.48239171,53.13017673)(120.54239165,53.18017668)(120.63238403,53.20017873)
\curveto(120.71239148,53.23017663)(120.80239139,53.24517662)(120.90238403,53.24517873)
\lineto(121.20238403,53.24517873)
\curveto(121.30239089,53.24517662)(121.3923908,53.2651766)(121.47238403,53.30517873)
\curveto(121.4923907,53.31517655)(121.50739069,53.32517654)(121.51738403,53.33517873)
\lineto(121.56238403,53.38017873)
\curveto(121.56239063,53.49017637)(121.51739068,53.58017628)(121.42738403,53.65017873)
\curveto(121.32739087,53.72017614)(121.24739095,53.78017608)(121.18738403,53.83017873)
\lineto(121.09738403,53.92017873)
\curveto(120.98739121,54.01017585)(120.87239132,54.13517573)(120.75238403,54.29517873)
\curveto(120.63239156,54.45517541)(120.54239165,54.60517526)(120.48238403,54.74517873)
\curveto(120.43239176,54.83517503)(120.3973918,54.93017493)(120.37738403,55.03017873)
\curveto(120.34739185,55.13017473)(120.31739188,55.23517463)(120.28738403,55.34517873)
\curveto(120.27739192,55.40517446)(120.27239192,55.4651744)(120.27238403,55.52517873)
\curveto(120.26239193,55.58517428)(120.25239194,55.64017422)(120.24238403,55.69017873)
}
}
{
\newrgbcolor{curcolor}{0 0 0}
\pscustom[linestyle=none,fillstyle=solid,fillcolor=curcolor]
{
}
}
{
\newrgbcolor{curcolor}{0 0 0}
\pscustom[linestyle=none,fillstyle=solid,fillcolor=curcolor]
{
\newpath
\moveto(123.06238403,67.99510061)
\lineto(123.31738403,67.99510061)
\curveto(123.3973888,68.0050929)(123.47238872,68.00009291)(123.54238403,67.98010061)
\lineto(123.78238403,67.98010061)
\lineto(123.94738403,67.98010061)
\curveto(124.04738815,67.96009295)(124.15238804,67.95009296)(124.26238403,67.95010061)
\curveto(124.36238783,67.95009296)(124.46238773,67.94009297)(124.56238403,67.92010061)
\lineto(124.71238403,67.92010061)
\curveto(124.85238734,67.89009302)(124.9923872,67.87009304)(125.13238403,67.86010061)
\curveto(125.26238693,67.85009306)(125.3923868,67.82509308)(125.52238403,67.78510061)
\curveto(125.60238659,67.76509314)(125.68738651,67.74509316)(125.77738403,67.72510061)
\lineto(126.01738403,67.66510061)
\lineto(126.31738403,67.54510061)
\curveto(126.40738579,67.51509339)(126.4973857,67.48009343)(126.58738403,67.44010061)
\curveto(126.80738539,67.34009357)(127.02238517,67.2050937)(127.23238403,67.03510061)
\curveto(127.44238475,66.87509403)(127.61238458,66.70009421)(127.74238403,66.51010061)
\curveto(127.78238441,66.46009445)(127.82238437,66.40009451)(127.86238403,66.33010061)
\curveto(127.8923843,66.27009464)(127.92738427,66.2100947)(127.96738403,66.15010061)
\curveto(128.01738418,66.07009484)(128.05738414,65.97509493)(128.08738403,65.86510061)
\curveto(128.11738408,65.75509515)(128.14738405,65.65009526)(128.17738403,65.55010061)
\curveto(128.21738398,65.44009547)(128.24238395,65.33009558)(128.25238403,65.22010061)
\curveto(128.26238393,65.1100958)(128.27738392,64.99509591)(128.29738403,64.87510061)
\curveto(128.30738389,64.83509607)(128.30738389,64.79009612)(128.29738403,64.74010061)
\curveto(128.2973839,64.70009621)(128.30238389,64.66009625)(128.31238403,64.62010061)
\curveto(128.32238387,64.58009633)(128.32738387,64.52509638)(128.32738403,64.45510061)
\curveto(128.32738387,64.38509652)(128.32238387,64.33509657)(128.31238403,64.30510061)
\curveto(128.2923839,64.25509665)(128.28738391,64.2100967)(128.29738403,64.17010061)
\curveto(128.30738389,64.13009678)(128.30738389,64.09509681)(128.29738403,64.06510061)
\lineto(128.29738403,63.97510061)
\curveto(128.27738392,63.91509699)(128.26238393,63.85009706)(128.25238403,63.78010061)
\curveto(128.25238394,63.72009719)(128.24738395,63.65509725)(128.23738403,63.58510061)
\curveto(128.18738401,63.41509749)(128.13738406,63.25509765)(128.08738403,63.10510061)
\curveto(128.03738416,62.95509795)(127.97238422,62.8100981)(127.89238403,62.67010061)
\curveto(127.85238434,62.62009829)(127.82238437,62.56509834)(127.80238403,62.50510061)
\curveto(127.77238442,62.45509845)(127.73738446,62.4050985)(127.69738403,62.35510061)
\curveto(127.51738468,62.11509879)(127.2973849,61.91509899)(127.03738403,61.75510061)
\curveto(126.77738542,61.59509931)(126.4923857,61.45509945)(126.18238403,61.33510061)
\curveto(126.04238615,61.27509963)(125.90238629,61.23009968)(125.76238403,61.20010061)
\curveto(125.61238658,61.17009974)(125.45738674,61.13509977)(125.29738403,61.09510061)
\curveto(125.18738701,61.07509983)(125.07738712,61.06009985)(124.96738403,61.05010061)
\curveto(124.85738734,61.04009987)(124.74738745,61.02509988)(124.63738403,61.00510061)
\curveto(124.5973876,60.99509991)(124.55738764,60.99009992)(124.51738403,60.99010061)
\curveto(124.47738772,61.00009991)(124.43738776,61.00009991)(124.39738403,60.99010061)
\curveto(124.34738785,60.98009993)(124.2973879,60.97509993)(124.24738403,60.97510061)
\lineto(124.08238403,60.97510061)
\curveto(124.03238816,60.95509995)(123.98238821,60.95009996)(123.93238403,60.96010061)
\curveto(123.87238832,60.97009994)(123.81738838,60.97009994)(123.76738403,60.96010061)
\curveto(123.72738847,60.95009996)(123.68238851,60.95009996)(123.63238403,60.96010061)
\curveto(123.58238861,60.97009994)(123.53238866,60.96509994)(123.48238403,60.94510061)
\curveto(123.41238878,60.92509998)(123.33738886,60.92009999)(123.25738403,60.93010061)
\curveto(123.16738903,60.94009997)(123.08238911,60.94509996)(123.00238403,60.94510061)
\curveto(122.91238928,60.94509996)(122.81238938,60.94009997)(122.70238403,60.93010061)
\curveto(122.58238961,60.92009999)(122.48238971,60.92509998)(122.40238403,60.94510061)
\lineto(122.11738403,60.94510061)
\lineto(121.48738403,60.99010061)
\curveto(121.38739081,61.00009991)(121.2923909,61.0100999)(121.20238403,61.02010061)
\lineto(120.90238403,61.05010061)
\curveto(120.85239134,61.07009984)(120.80239139,61.07509983)(120.75238403,61.06510061)
\curveto(120.6923915,61.06509984)(120.63739156,61.07509983)(120.58738403,61.09510061)
\curveto(120.41739178,61.14509976)(120.25239194,61.18509972)(120.09238403,61.21510061)
\curveto(119.92239227,61.24509966)(119.76239243,61.29509961)(119.61238403,61.36510061)
\curveto(119.15239304,61.55509935)(118.77739342,61.77509913)(118.48738403,62.02510061)
\curveto(118.197394,62.28509862)(117.95239424,62.64509826)(117.75238403,63.10510061)
\curveto(117.70239449,63.23509767)(117.66739453,63.36509754)(117.64738403,63.49510061)
\curveto(117.62739457,63.63509727)(117.60239459,63.77509713)(117.57238403,63.91510061)
\curveto(117.56239463,63.98509692)(117.55739464,64.05009686)(117.55738403,64.11010061)
\curveto(117.55739464,64.17009674)(117.55239464,64.23509667)(117.54238403,64.30510061)
\curveto(117.52239467,65.13509577)(117.67239452,65.8050951)(117.99238403,66.31510061)
\curveto(118.30239389,66.82509408)(118.74239345,67.2050937)(119.31238403,67.45510061)
\curveto(119.43239276,67.5050934)(119.55739264,67.55009336)(119.68738403,67.59010061)
\curveto(119.81739238,67.63009328)(119.95239224,67.67509323)(120.09238403,67.72510061)
\curveto(120.17239202,67.74509316)(120.25739194,67.76009315)(120.34738403,67.77010061)
\lineto(120.58738403,67.83010061)
\curveto(120.6973915,67.86009305)(120.80739139,67.87509303)(120.91738403,67.87510061)
\curveto(121.02739117,67.88509302)(121.13739106,67.90009301)(121.24738403,67.92010061)
\curveto(121.2973909,67.94009297)(121.34239085,67.94509296)(121.38238403,67.93510061)
\curveto(121.42239077,67.93509297)(121.46239073,67.94009297)(121.50238403,67.95010061)
\curveto(121.55239064,67.96009295)(121.60739059,67.96009295)(121.66738403,67.95010061)
\curveto(121.71739048,67.95009296)(121.76739043,67.95509295)(121.81738403,67.96510061)
\lineto(121.95238403,67.96510061)
\curveto(122.01239018,67.98509292)(122.08239011,67.98509292)(122.16238403,67.96510061)
\curveto(122.23238996,67.95509295)(122.2973899,67.96009295)(122.35738403,67.98010061)
\curveto(122.38738981,67.99009292)(122.42738977,67.99509291)(122.47738403,67.99510061)
\lineto(122.59738403,67.99510061)
\lineto(123.06238403,67.99510061)
\moveto(125.38738403,66.45010061)
\curveto(125.06738713,66.55009436)(124.70238749,66.6100943)(124.29238403,66.63010061)
\curveto(123.88238831,66.65009426)(123.47238872,66.66009425)(123.06238403,66.66010061)
\curveto(122.63238956,66.66009425)(122.21238998,66.65009426)(121.80238403,66.63010061)
\curveto(121.3923908,66.6100943)(121.00739119,66.56509434)(120.64738403,66.49510061)
\curveto(120.28739191,66.42509448)(119.96739223,66.31509459)(119.68738403,66.16510061)
\curveto(119.3973928,66.02509488)(119.16239303,65.83009508)(118.98238403,65.58010061)
\curveto(118.87239332,65.42009549)(118.7923934,65.24009567)(118.74238403,65.04010061)
\curveto(118.68239351,64.84009607)(118.65239354,64.59509631)(118.65238403,64.30510061)
\curveto(118.67239352,64.28509662)(118.68239351,64.25009666)(118.68238403,64.20010061)
\curveto(118.67239352,64.15009676)(118.67239352,64.1100968)(118.68238403,64.08010061)
\curveto(118.70239349,64.00009691)(118.72239347,63.92509698)(118.74238403,63.85510061)
\curveto(118.75239344,63.79509711)(118.77239342,63.73009718)(118.80238403,63.66010061)
\curveto(118.92239327,63.39009752)(119.0923931,63.17009774)(119.31238403,63.00010061)
\curveto(119.52239267,62.84009807)(119.76739243,62.7050982)(120.04738403,62.59510061)
\curveto(120.15739204,62.54509836)(120.27739192,62.5050984)(120.40738403,62.47510061)
\curveto(120.52739167,62.45509845)(120.65239154,62.43009848)(120.78238403,62.40010061)
\curveto(120.83239136,62.38009853)(120.88739131,62.37009854)(120.94738403,62.37010061)
\curveto(120.9973912,62.37009854)(121.04739115,62.36509854)(121.09738403,62.35510061)
\curveto(121.18739101,62.34509856)(121.28239091,62.33509857)(121.38238403,62.32510061)
\curveto(121.47239072,62.31509859)(121.56739063,62.3050986)(121.66738403,62.29510061)
\curveto(121.74739045,62.29509861)(121.83239036,62.29009862)(121.92238403,62.28010061)
\lineto(122.16238403,62.28010061)
\lineto(122.34238403,62.28010061)
\curveto(122.37238982,62.27009864)(122.40738979,62.26509864)(122.44738403,62.26510061)
\lineto(122.58238403,62.26510061)
\lineto(123.03238403,62.26510061)
\curveto(123.11238908,62.26509864)(123.197389,62.26009865)(123.28738403,62.25010061)
\curveto(123.36738883,62.25009866)(123.44238875,62.26009865)(123.51238403,62.28010061)
\lineto(123.78238403,62.28010061)
\curveto(123.80238839,62.28009863)(123.83238836,62.27509863)(123.87238403,62.26510061)
\curveto(123.90238829,62.26509864)(123.92738827,62.27009864)(123.94738403,62.28010061)
\curveto(124.04738815,62.29009862)(124.14738805,62.29509861)(124.24738403,62.29510061)
\curveto(124.33738786,62.3050986)(124.43738776,62.31509859)(124.54738403,62.32510061)
\curveto(124.66738753,62.35509855)(124.7923874,62.37009854)(124.92238403,62.37010061)
\curveto(125.04238715,62.38009853)(125.15738704,62.4050985)(125.26738403,62.44510061)
\curveto(125.56738663,62.52509838)(125.83238636,62.6100983)(126.06238403,62.70010061)
\curveto(126.2923859,62.80009811)(126.50738569,62.94509796)(126.70738403,63.13510061)
\curveto(126.90738529,63.34509756)(127.05738514,63.6100973)(127.15738403,63.93010061)
\curveto(127.17738502,63.97009694)(127.18738501,64.0050969)(127.18738403,64.03510061)
\curveto(127.17738502,64.07509683)(127.18238501,64.12009679)(127.20238403,64.17010061)
\curveto(127.21238498,64.2100967)(127.22238497,64.28009663)(127.23238403,64.38010061)
\curveto(127.24238495,64.49009642)(127.23738496,64.57509633)(127.21738403,64.63510061)
\curveto(127.197385,64.7050962)(127.18738501,64.77509613)(127.18738403,64.84510061)
\curveto(127.17738502,64.91509599)(127.16238503,64.98009593)(127.14238403,65.04010061)
\curveto(127.08238511,65.24009567)(126.9973852,65.42009549)(126.88738403,65.58010061)
\curveto(126.86738533,65.6100953)(126.84738535,65.63509527)(126.82738403,65.65510061)
\lineto(126.76738403,65.71510061)
\curveto(126.74738545,65.75509515)(126.70738549,65.8050951)(126.64738403,65.86510061)
\curveto(126.50738569,65.96509494)(126.37738582,66.05009486)(126.25738403,66.12010061)
\curveto(126.13738606,66.19009472)(125.9923862,66.26009465)(125.82238403,66.33010061)
\curveto(125.75238644,66.36009455)(125.68238651,66.38009453)(125.61238403,66.39010061)
\curveto(125.54238665,66.4100945)(125.46738673,66.43009448)(125.38738403,66.45010061)
}
}
{
\newrgbcolor{curcolor}{0 0 0}
\pscustom[linestyle=none,fillstyle=solid,fillcolor=curcolor]
{
\newpath
\moveto(117.54238403,72.59470998)
\curveto(117.53239466,73.28470535)(117.65239454,73.88470475)(117.90238403,74.39470998)
\curveto(118.15239404,74.91470372)(118.48739371,75.30970332)(118.90738403,75.57970998)
\curveto(118.98739321,75.629703)(119.07739312,75.67470296)(119.17738403,75.71470998)
\curveto(119.26739293,75.75470288)(119.36239283,75.79970283)(119.46238403,75.84970998)
\curveto(119.56239263,75.88970274)(119.66239253,75.91970271)(119.76238403,75.93970998)
\curveto(119.86239233,75.95970267)(119.96739223,75.97970265)(120.07738403,75.99970998)
\curveto(120.12739207,76.01970261)(120.17239202,76.02470261)(120.21238403,76.01470998)
\curveto(120.25239194,76.00470263)(120.2973919,76.00970262)(120.34738403,76.02970998)
\curveto(120.3973918,76.03970259)(120.48239171,76.04470259)(120.60238403,76.04470998)
\curveto(120.71239148,76.04470259)(120.7973914,76.03970259)(120.85738403,76.02970998)
\curveto(120.91739128,76.00970262)(120.97739122,75.99970263)(121.03738403,75.99970998)
\curveto(121.0973911,76.00970262)(121.15739104,76.00470263)(121.21738403,75.98470998)
\curveto(121.35739084,75.94470269)(121.4923907,75.90970272)(121.62238403,75.87970998)
\curveto(121.75239044,75.84970278)(121.87739032,75.80970282)(121.99738403,75.75970998)
\curveto(122.13739006,75.69970293)(122.26238993,75.629703)(122.37238403,75.54970998)
\curveto(122.48238971,75.47970315)(122.5923896,75.40470323)(122.70238403,75.32470998)
\lineto(122.76238403,75.26470998)
\curveto(122.78238941,75.25470338)(122.80238939,75.23970339)(122.82238403,75.21970998)
\curveto(122.98238921,75.09970353)(123.12738907,74.96470367)(123.25738403,74.81470998)
\curveto(123.38738881,74.66470397)(123.51238868,74.50470413)(123.63238403,74.33470998)
\curveto(123.85238834,74.02470461)(124.05738814,73.7297049)(124.24738403,73.44970998)
\curveto(124.38738781,73.21970541)(124.52238767,72.98970564)(124.65238403,72.75970998)
\curveto(124.78238741,72.53970609)(124.91738728,72.31970631)(125.05738403,72.09970998)
\curveto(125.22738697,71.84970678)(125.40738679,71.60970702)(125.59738403,71.37970998)
\curveto(125.78738641,71.15970747)(126.01238618,70.96970766)(126.27238403,70.80970998)
\curveto(126.33238586,70.76970786)(126.3923858,70.7347079)(126.45238403,70.70470998)
\curveto(126.50238569,70.67470796)(126.56738563,70.64470799)(126.64738403,70.61470998)
\curveto(126.71738548,70.59470804)(126.77738542,70.58970804)(126.82738403,70.59970998)
\curveto(126.8973853,70.61970801)(126.95238524,70.65470798)(126.99238403,70.70470998)
\curveto(127.02238517,70.75470788)(127.04238515,70.81470782)(127.05238403,70.88470998)
\lineto(127.05238403,71.12470998)
\lineto(127.05238403,71.87470998)
\lineto(127.05238403,74.67970998)
\lineto(127.05238403,75.33970998)
\curveto(127.05238514,75.4297032)(127.05738514,75.51470312)(127.06738403,75.59470998)
\curveto(127.06738513,75.67470296)(127.08738511,75.73970289)(127.12738403,75.78970998)
\curveto(127.16738503,75.83970279)(127.24238495,75.87970275)(127.35238403,75.90970998)
\curveto(127.45238474,75.94970268)(127.55238464,75.95970267)(127.65238403,75.93970998)
\lineto(127.78738403,75.93970998)
\curveto(127.85738434,75.91970271)(127.91738428,75.89970273)(127.96738403,75.87970998)
\curveto(128.01738418,75.85970277)(128.05738414,75.82470281)(128.08738403,75.77470998)
\curveto(128.12738407,75.72470291)(128.14738405,75.65470298)(128.14738403,75.56470998)
\lineto(128.14738403,75.29470998)
\lineto(128.14738403,74.39470998)
\lineto(128.14738403,70.88470998)
\lineto(128.14738403,69.81970998)
\curveto(128.14738405,69.73970889)(128.15238404,69.64970898)(128.16238403,69.54970998)
\curveto(128.16238403,69.44970918)(128.15238404,69.36470927)(128.13238403,69.29470998)
\curveto(128.06238413,69.08470955)(127.88238431,69.01970961)(127.59238403,69.09970998)
\curveto(127.55238464,69.10970952)(127.51738468,69.10970952)(127.48738403,69.09970998)
\curveto(127.44738475,69.09970953)(127.40238479,69.10970952)(127.35238403,69.12970998)
\curveto(127.27238492,69.14970948)(127.18738501,69.16970946)(127.09738403,69.18970998)
\curveto(127.00738519,69.20970942)(126.92238527,69.2347094)(126.84238403,69.26470998)
\curveto(126.35238584,69.42470921)(125.93738626,69.62470901)(125.59738403,69.86470998)
\curveto(125.34738685,70.04470859)(125.12238707,70.24970838)(124.92238403,70.47970998)
\curveto(124.71238748,70.70970792)(124.51738768,70.94970768)(124.33738403,71.19970998)
\curveto(124.15738804,71.45970717)(123.98738821,71.72470691)(123.82738403,71.99470998)
\curveto(123.65738854,72.27470636)(123.48238871,72.54470609)(123.30238403,72.80470998)
\curveto(123.22238897,72.91470572)(123.14738905,73.01970561)(123.07738403,73.11970998)
\curveto(123.00738919,73.2297054)(122.93238926,73.33970529)(122.85238403,73.44970998)
\curveto(122.82238937,73.48970514)(122.7923894,73.52470511)(122.76238403,73.55470998)
\curveto(122.72238947,73.59470504)(122.6923895,73.634705)(122.67238403,73.67470998)
\curveto(122.56238963,73.81470482)(122.43738976,73.93970469)(122.29738403,74.04970998)
\curveto(122.26738993,74.06970456)(122.24238995,74.09470454)(122.22238403,74.12470998)
\curveto(122.19239,74.15470448)(122.16239003,74.17970445)(122.13238403,74.19970998)
\curveto(122.03239016,74.27970435)(121.93239026,74.34470429)(121.83238403,74.39470998)
\curveto(121.73239046,74.45470418)(121.62239057,74.50970412)(121.50238403,74.55970998)
\curveto(121.43239076,74.58970404)(121.35739084,74.60970402)(121.27738403,74.61970998)
\lineto(121.03738403,74.67970998)
\lineto(120.94738403,74.67970998)
\curveto(120.91739128,74.68970394)(120.88739131,74.69470394)(120.85738403,74.69470998)
\curveto(120.78739141,74.71470392)(120.6923915,74.71970391)(120.57238403,74.70970998)
\curveto(120.44239175,74.70970392)(120.34239185,74.69970393)(120.27238403,74.67970998)
\curveto(120.192392,74.65970397)(120.11739208,74.63970399)(120.04738403,74.61970998)
\curveto(119.96739223,74.60970402)(119.88739231,74.58970404)(119.80738403,74.55970998)
\curveto(119.56739263,74.44970418)(119.36739283,74.29970433)(119.20738403,74.10970998)
\curveto(119.03739316,73.9297047)(118.8973933,73.70970492)(118.78738403,73.44970998)
\curveto(118.76739343,73.37970525)(118.75239344,73.30970532)(118.74238403,73.23970998)
\curveto(118.72239347,73.16970546)(118.70239349,73.09470554)(118.68238403,73.01470998)
\curveto(118.66239353,72.9347057)(118.65239354,72.82470581)(118.65238403,72.68470998)
\curveto(118.65239354,72.55470608)(118.66239353,72.44970618)(118.68238403,72.36970998)
\curveto(118.6923935,72.30970632)(118.6973935,72.25470638)(118.69738403,72.20470998)
\curveto(118.6973935,72.15470648)(118.70739349,72.10470653)(118.72738403,72.05470998)
\curveto(118.76739343,71.95470668)(118.80739339,71.85970677)(118.84738403,71.76970998)
\curveto(118.88739331,71.68970694)(118.93239326,71.60970702)(118.98238403,71.52970998)
\curveto(119.00239319,71.49970713)(119.02739317,71.46970716)(119.05738403,71.43970998)
\curveto(119.08739311,71.41970721)(119.11239308,71.39470724)(119.13238403,71.36470998)
\lineto(119.20738403,71.28970998)
\curveto(119.22739297,71.25970737)(119.24739295,71.2347074)(119.26738403,71.21470998)
\lineto(119.47738403,71.06470998)
\curveto(119.53739266,71.02470761)(119.60239259,70.97970765)(119.67238403,70.92970998)
\curveto(119.76239243,70.86970776)(119.86739233,70.81970781)(119.98738403,70.77970998)
\curveto(120.0973921,70.74970788)(120.20739199,70.71470792)(120.31738403,70.67470998)
\curveto(120.42739177,70.634708)(120.57239162,70.60970802)(120.75238403,70.59970998)
\curveto(120.92239127,70.58970804)(121.04739115,70.55970807)(121.12738403,70.50970998)
\curveto(121.20739099,70.45970817)(121.25239094,70.38470825)(121.26238403,70.28470998)
\curveto(121.27239092,70.18470845)(121.27739092,70.07470856)(121.27738403,69.95470998)
\curveto(121.27739092,69.91470872)(121.28239091,69.87470876)(121.29238403,69.83470998)
\curveto(121.2923909,69.79470884)(121.28739091,69.75970887)(121.27738403,69.72970998)
\curveto(121.25739094,69.67970895)(121.24739095,69.629709)(121.24738403,69.57970998)
\curveto(121.24739095,69.53970909)(121.23739096,69.49970913)(121.21738403,69.45970998)
\curveto(121.15739104,69.36970926)(121.02239117,69.32470931)(120.81238403,69.32470998)
\lineto(120.69238403,69.32470998)
\curveto(120.63239156,69.3347093)(120.57239162,69.33970929)(120.51238403,69.33970998)
\curveto(120.44239175,69.34970928)(120.37739182,69.35970927)(120.31738403,69.36970998)
\curveto(120.20739199,69.38970924)(120.10739209,69.40970922)(120.01738403,69.42970998)
\curveto(119.91739228,69.44970918)(119.82239237,69.47970915)(119.73238403,69.51970998)
\curveto(119.66239253,69.53970909)(119.60239259,69.55970907)(119.55238403,69.57970998)
\lineto(119.37238403,69.63970998)
\curveto(119.11239308,69.75970887)(118.86739333,69.91470872)(118.63738403,70.10470998)
\curveto(118.40739379,70.30470833)(118.22239397,70.51970811)(118.08238403,70.74970998)
\curveto(118.00239419,70.85970777)(117.93739426,70.97470766)(117.88738403,71.09470998)
\lineto(117.73738403,71.48470998)
\curveto(117.68739451,71.59470704)(117.65739454,71.70970692)(117.64738403,71.82970998)
\curveto(117.62739457,71.94970668)(117.60239459,72.07470656)(117.57238403,72.20470998)
\curveto(117.57239462,72.27470636)(117.57239462,72.33970629)(117.57238403,72.39970998)
\curveto(117.56239463,72.45970617)(117.55239464,72.52470611)(117.54238403,72.59470998)
}
}
{
\newrgbcolor{curcolor}{0 0 0}
\pscustom[linestyle=none,fillstyle=solid,fillcolor=curcolor]
{
\newpath
\moveto(126.51238403,78.63431936)
\lineto(126.51238403,79.26431936)
\lineto(126.51238403,79.45931936)
\curveto(126.51238568,79.52931683)(126.52238567,79.58931677)(126.54238403,79.63931936)
\curveto(126.58238561,79.70931665)(126.62238557,79.7593166)(126.66238403,79.78931936)
\curveto(126.71238548,79.82931653)(126.77738542,79.84931651)(126.85738403,79.84931936)
\curveto(126.93738526,79.8593165)(127.02238517,79.86431649)(127.11238403,79.86431936)
\lineto(127.83238403,79.86431936)
\curveto(128.31238388,79.86431649)(128.72238347,79.80431655)(129.06238403,79.68431936)
\curveto(129.40238279,79.56431679)(129.67738252,79.36931699)(129.88738403,79.09931936)
\curveto(129.93738226,79.02931733)(129.98238221,78.9593174)(130.02238403,78.88931936)
\curveto(130.07238212,78.82931753)(130.11738208,78.7543176)(130.15738403,78.66431936)
\curveto(130.16738203,78.64431771)(130.17738202,78.61431774)(130.18738403,78.57431936)
\curveto(130.20738199,78.53431782)(130.21238198,78.48931787)(130.20238403,78.43931936)
\curveto(130.17238202,78.34931801)(130.0973821,78.29431806)(129.97738403,78.27431936)
\curveto(129.86738233,78.2543181)(129.77238242,78.26931809)(129.69238403,78.31931936)
\curveto(129.62238257,78.34931801)(129.55738264,78.39431796)(129.49738403,78.45431936)
\curveto(129.44738275,78.52431783)(129.3973828,78.58931777)(129.34738403,78.64931936)
\curveto(129.2973829,78.71931764)(129.22238297,78.77931758)(129.12238403,78.82931936)
\curveto(129.03238316,78.88931747)(128.94238325,78.93931742)(128.85238403,78.97931936)
\curveto(128.82238337,78.99931736)(128.76238343,79.02431733)(128.67238403,79.05431936)
\curveto(128.5923836,79.08431727)(128.52238367,79.08931727)(128.46238403,79.06931936)
\curveto(128.32238387,79.03931732)(128.23238396,78.97931738)(128.19238403,78.88931936)
\curveto(128.16238403,78.80931755)(128.14738405,78.71931764)(128.14738403,78.61931936)
\curveto(128.14738405,78.51931784)(128.12238407,78.43431792)(128.07238403,78.36431936)
\curveto(128.00238419,78.27431808)(127.86238433,78.22931813)(127.65238403,78.22931936)
\lineto(127.09738403,78.22931936)
\lineto(126.87238403,78.22931936)
\curveto(126.7923854,78.23931812)(126.72738547,78.2593181)(126.67738403,78.28931936)
\curveto(126.5973856,78.34931801)(126.55238564,78.41931794)(126.54238403,78.49931936)
\curveto(126.53238566,78.51931784)(126.52738567,78.53931782)(126.52738403,78.55931936)
\curveto(126.52738567,78.58931777)(126.52238567,78.61431774)(126.51238403,78.63431936)
}
}
{
\newrgbcolor{curcolor}{0 0 0}
\pscustom[linestyle=none,fillstyle=solid,fillcolor=curcolor]
{
}
}
{
\newrgbcolor{curcolor}{0 0 0}
\pscustom[linestyle=none,fillstyle=solid,fillcolor=curcolor]
{
\newpath
\moveto(117.54238403,89.26463186)
\curveto(117.53239466,89.95462722)(117.65239454,90.55462662)(117.90238403,91.06463186)
\curveto(118.15239404,91.58462559)(118.48739371,91.9796252)(118.90738403,92.24963186)
\curveto(118.98739321,92.29962488)(119.07739312,92.34462483)(119.17738403,92.38463186)
\curveto(119.26739293,92.42462475)(119.36239283,92.46962471)(119.46238403,92.51963186)
\curveto(119.56239263,92.55962462)(119.66239253,92.58962459)(119.76238403,92.60963186)
\curveto(119.86239233,92.62962455)(119.96739223,92.64962453)(120.07738403,92.66963186)
\curveto(120.12739207,92.68962449)(120.17239202,92.69462448)(120.21238403,92.68463186)
\curveto(120.25239194,92.6746245)(120.2973919,92.6796245)(120.34738403,92.69963186)
\curveto(120.3973918,92.70962447)(120.48239171,92.71462446)(120.60238403,92.71463186)
\curveto(120.71239148,92.71462446)(120.7973914,92.70962447)(120.85738403,92.69963186)
\curveto(120.91739128,92.6796245)(120.97739122,92.66962451)(121.03738403,92.66963186)
\curveto(121.0973911,92.6796245)(121.15739104,92.6746245)(121.21738403,92.65463186)
\curveto(121.35739084,92.61462456)(121.4923907,92.5796246)(121.62238403,92.54963186)
\curveto(121.75239044,92.51962466)(121.87739032,92.4796247)(121.99738403,92.42963186)
\curveto(122.13739006,92.36962481)(122.26238993,92.29962488)(122.37238403,92.21963186)
\curveto(122.48238971,92.14962503)(122.5923896,92.0746251)(122.70238403,91.99463186)
\lineto(122.76238403,91.93463186)
\curveto(122.78238941,91.92462525)(122.80238939,91.90962527)(122.82238403,91.88963186)
\curveto(122.98238921,91.76962541)(123.12738907,91.63462554)(123.25738403,91.48463186)
\curveto(123.38738881,91.33462584)(123.51238868,91.174626)(123.63238403,91.00463186)
\curveto(123.85238834,90.69462648)(124.05738814,90.39962678)(124.24738403,90.11963186)
\curveto(124.38738781,89.88962729)(124.52238767,89.65962752)(124.65238403,89.42963186)
\curveto(124.78238741,89.20962797)(124.91738728,88.98962819)(125.05738403,88.76963186)
\curveto(125.22738697,88.51962866)(125.40738679,88.2796289)(125.59738403,88.04963186)
\curveto(125.78738641,87.82962935)(126.01238618,87.63962954)(126.27238403,87.47963186)
\curveto(126.33238586,87.43962974)(126.3923858,87.40462977)(126.45238403,87.37463186)
\curveto(126.50238569,87.34462983)(126.56738563,87.31462986)(126.64738403,87.28463186)
\curveto(126.71738548,87.26462991)(126.77738542,87.25962992)(126.82738403,87.26963186)
\curveto(126.8973853,87.28962989)(126.95238524,87.32462985)(126.99238403,87.37463186)
\curveto(127.02238517,87.42462975)(127.04238515,87.48462969)(127.05238403,87.55463186)
\lineto(127.05238403,87.79463186)
\lineto(127.05238403,88.54463186)
\lineto(127.05238403,91.34963186)
\lineto(127.05238403,92.00963186)
\curveto(127.05238514,92.09962508)(127.05738514,92.18462499)(127.06738403,92.26463186)
\curveto(127.06738513,92.34462483)(127.08738511,92.40962477)(127.12738403,92.45963186)
\curveto(127.16738503,92.50962467)(127.24238495,92.54962463)(127.35238403,92.57963186)
\curveto(127.45238474,92.61962456)(127.55238464,92.62962455)(127.65238403,92.60963186)
\lineto(127.78738403,92.60963186)
\curveto(127.85738434,92.58962459)(127.91738428,92.56962461)(127.96738403,92.54963186)
\curveto(128.01738418,92.52962465)(128.05738414,92.49462468)(128.08738403,92.44463186)
\curveto(128.12738407,92.39462478)(128.14738405,92.32462485)(128.14738403,92.23463186)
\lineto(128.14738403,91.96463186)
\lineto(128.14738403,91.06463186)
\lineto(128.14738403,87.55463186)
\lineto(128.14738403,86.48963186)
\curveto(128.14738405,86.40963077)(128.15238404,86.31963086)(128.16238403,86.21963186)
\curveto(128.16238403,86.11963106)(128.15238404,86.03463114)(128.13238403,85.96463186)
\curveto(128.06238413,85.75463142)(127.88238431,85.68963149)(127.59238403,85.76963186)
\curveto(127.55238464,85.7796314)(127.51738468,85.7796314)(127.48738403,85.76963186)
\curveto(127.44738475,85.76963141)(127.40238479,85.7796314)(127.35238403,85.79963186)
\curveto(127.27238492,85.81963136)(127.18738501,85.83963134)(127.09738403,85.85963186)
\curveto(127.00738519,85.8796313)(126.92238527,85.90463127)(126.84238403,85.93463186)
\curveto(126.35238584,86.09463108)(125.93738626,86.29463088)(125.59738403,86.53463186)
\curveto(125.34738685,86.71463046)(125.12238707,86.91963026)(124.92238403,87.14963186)
\curveto(124.71238748,87.3796298)(124.51738768,87.61962956)(124.33738403,87.86963186)
\curveto(124.15738804,88.12962905)(123.98738821,88.39462878)(123.82738403,88.66463186)
\curveto(123.65738854,88.94462823)(123.48238871,89.21462796)(123.30238403,89.47463186)
\curveto(123.22238897,89.58462759)(123.14738905,89.68962749)(123.07738403,89.78963186)
\curveto(123.00738919,89.89962728)(122.93238926,90.00962717)(122.85238403,90.11963186)
\curveto(122.82238937,90.15962702)(122.7923894,90.19462698)(122.76238403,90.22463186)
\curveto(122.72238947,90.26462691)(122.6923895,90.30462687)(122.67238403,90.34463186)
\curveto(122.56238963,90.48462669)(122.43738976,90.60962657)(122.29738403,90.71963186)
\curveto(122.26738993,90.73962644)(122.24238995,90.76462641)(122.22238403,90.79463186)
\curveto(122.19239,90.82462635)(122.16239003,90.84962633)(122.13238403,90.86963186)
\curveto(122.03239016,90.94962623)(121.93239026,91.01462616)(121.83238403,91.06463186)
\curveto(121.73239046,91.12462605)(121.62239057,91.179626)(121.50238403,91.22963186)
\curveto(121.43239076,91.25962592)(121.35739084,91.2796259)(121.27738403,91.28963186)
\lineto(121.03738403,91.34963186)
\lineto(120.94738403,91.34963186)
\curveto(120.91739128,91.35962582)(120.88739131,91.36462581)(120.85738403,91.36463186)
\curveto(120.78739141,91.38462579)(120.6923915,91.38962579)(120.57238403,91.37963186)
\curveto(120.44239175,91.3796258)(120.34239185,91.36962581)(120.27238403,91.34963186)
\curveto(120.192392,91.32962585)(120.11739208,91.30962587)(120.04738403,91.28963186)
\curveto(119.96739223,91.2796259)(119.88739231,91.25962592)(119.80738403,91.22963186)
\curveto(119.56739263,91.11962606)(119.36739283,90.96962621)(119.20738403,90.77963186)
\curveto(119.03739316,90.59962658)(118.8973933,90.3796268)(118.78738403,90.11963186)
\curveto(118.76739343,90.04962713)(118.75239344,89.9796272)(118.74238403,89.90963186)
\curveto(118.72239347,89.83962734)(118.70239349,89.76462741)(118.68238403,89.68463186)
\curveto(118.66239353,89.60462757)(118.65239354,89.49462768)(118.65238403,89.35463186)
\curveto(118.65239354,89.22462795)(118.66239353,89.11962806)(118.68238403,89.03963186)
\curveto(118.6923935,88.9796282)(118.6973935,88.92462825)(118.69738403,88.87463186)
\curveto(118.6973935,88.82462835)(118.70739349,88.7746284)(118.72738403,88.72463186)
\curveto(118.76739343,88.62462855)(118.80739339,88.52962865)(118.84738403,88.43963186)
\curveto(118.88739331,88.35962882)(118.93239326,88.2796289)(118.98238403,88.19963186)
\curveto(119.00239319,88.16962901)(119.02739317,88.13962904)(119.05738403,88.10963186)
\curveto(119.08739311,88.08962909)(119.11239308,88.06462911)(119.13238403,88.03463186)
\lineto(119.20738403,87.95963186)
\curveto(119.22739297,87.92962925)(119.24739295,87.90462927)(119.26738403,87.88463186)
\lineto(119.47738403,87.73463186)
\curveto(119.53739266,87.69462948)(119.60239259,87.64962953)(119.67238403,87.59963186)
\curveto(119.76239243,87.53962964)(119.86739233,87.48962969)(119.98738403,87.44963186)
\curveto(120.0973921,87.41962976)(120.20739199,87.38462979)(120.31738403,87.34463186)
\curveto(120.42739177,87.30462987)(120.57239162,87.2796299)(120.75238403,87.26963186)
\curveto(120.92239127,87.25962992)(121.04739115,87.22962995)(121.12738403,87.17963186)
\curveto(121.20739099,87.12963005)(121.25239094,87.05463012)(121.26238403,86.95463186)
\curveto(121.27239092,86.85463032)(121.27739092,86.74463043)(121.27738403,86.62463186)
\curveto(121.27739092,86.58463059)(121.28239091,86.54463063)(121.29238403,86.50463186)
\curveto(121.2923909,86.46463071)(121.28739091,86.42963075)(121.27738403,86.39963186)
\curveto(121.25739094,86.34963083)(121.24739095,86.29963088)(121.24738403,86.24963186)
\curveto(121.24739095,86.20963097)(121.23739096,86.16963101)(121.21738403,86.12963186)
\curveto(121.15739104,86.03963114)(121.02239117,85.99463118)(120.81238403,85.99463186)
\lineto(120.69238403,85.99463186)
\curveto(120.63239156,86.00463117)(120.57239162,86.00963117)(120.51238403,86.00963186)
\curveto(120.44239175,86.01963116)(120.37739182,86.02963115)(120.31738403,86.03963186)
\curveto(120.20739199,86.05963112)(120.10739209,86.0796311)(120.01738403,86.09963186)
\curveto(119.91739228,86.11963106)(119.82239237,86.14963103)(119.73238403,86.18963186)
\curveto(119.66239253,86.20963097)(119.60239259,86.22963095)(119.55238403,86.24963186)
\lineto(119.37238403,86.30963186)
\curveto(119.11239308,86.42963075)(118.86739333,86.58463059)(118.63738403,86.77463186)
\curveto(118.40739379,86.9746302)(118.22239397,87.18962999)(118.08238403,87.41963186)
\curveto(118.00239419,87.52962965)(117.93739426,87.64462953)(117.88738403,87.76463186)
\lineto(117.73738403,88.15463186)
\curveto(117.68739451,88.26462891)(117.65739454,88.3796288)(117.64738403,88.49963186)
\curveto(117.62739457,88.61962856)(117.60239459,88.74462843)(117.57238403,88.87463186)
\curveto(117.57239462,88.94462823)(117.57239462,89.00962817)(117.57238403,89.06963186)
\curveto(117.56239463,89.12962805)(117.55239464,89.19462798)(117.54238403,89.26463186)
}
}
{
\newrgbcolor{curcolor}{0 0 0}
\pscustom[linestyle=none,fillstyle=solid,fillcolor=curcolor]
{
\newpath
\moveto(123.06238403,101.36424123)
\lineto(123.31738403,101.36424123)
\curveto(123.3973888,101.37423353)(123.47238872,101.36923353)(123.54238403,101.34924123)
\lineto(123.78238403,101.34924123)
\lineto(123.94738403,101.34924123)
\curveto(124.04738815,101.32923357)(124.15238804,101.31923358)(124.26238403,101.31924123)
\curveto(124.36238783,101.31923358)(124.46238773,101.30923359)(124.56238403,101.28924123)
\lineto(124.71238403,101.28924123)
\curveto(124.85238734,101.25923364)(124.9923872,101.23923366)(125.13238403,101.22924123)
\curveto(125.26238693,101.21923368)(125.3923868,101.19423371)(125.52238403,101.15424123)
\curveto(125.60238659,101.13423377)(125.68738651,101.11423379)(125.77738403,101.09424123)
\lineto(126.01738403,101.03424123)
\lineto(126.31738403,100.91424123)
\curveto(126.40738579,100.88423402)(126.4973857,100.84923405)(126.58738403,100.80924123)
\curveto(126.80738539,100.70923419)(127.02238517,100.57423433)(127.23238403,100.40424123)
\curveto(127.44238475,100.24423466)(127.61238458,100.06923483)(127.74238403,99.87924123)
\curveto(127.78238441,99.82923507)(127.82238437,99.76923513)(127.86238403,99.69924123)
\curveto(127.8923843,99.63923526)(127.92738427,99.57923532)(127.96738403,99.51924123)
\curveto(128.01738418,99.43923546)(128.05738414,99.34423556)(128.08738403,99.23424123)
\curveto(128.11738408,99.12423578)(128.14738405,99.01923588)(128.17738403,98.91924123)
\curveto(128.21738398,98.80923609)(128.24238395,98.6992362)(128.25238403,98.58924123)
\curveto(128.26238393,98.47923642)(128.27738392,98.36423654)(128.29738403,98.24424123)
\curveto(128.30738389,98.2042367)(128.30738389,98.15923674)(128.29738403,98.10924123)
\curveto(128.2973839,98.06923683)(128.30238389,98.02923687)(128.31238403,97.98924123)
\curveto(128.32238387,97.94923695)(128.32738387,97.89423701)(128.32738403,97.82424123)
\curveto(128.32738387,97.75423715)(128.32238387,97.7042372)(128.31238403,97.67424123)
\curveto(128.2923839,97.62423728)(128.28738391,97.57923732)(128.29738403,97.53924123)
\curveto(128.30738389,97.4992374)(128.30738389,97.46423744)(128.29738403,97.43424123)
\lineto(128.29738403,97.34424123)
\curveto(128.27738392,97.28423762)(128.26238393,97.21923768)(128.25238403,97.14924123)
\curveto(128.25238394,97.08923781)(128.24738395,97.02423788)(128.23738403,96.95424123)
\curveto(128.18738401,96.78423812)(128.13738406,96.62423828)(128.08738403,96.47424123)
\curveto(128.03738416,96.32423858)(127.97238422,96.17923872)(127.89238403,96.03924123)
\curveto(127.85238434,95.98923891)(127.82238437,95.93423897)(127.80238403,95.87424123)
\curveto(127.77238442,95.82423908)(127.73738446,95.77423913)(127.69738403,95.72424123)
\curveto(127.51738468,95.48423942)(127.2973849,95.28423962)(127.03738403,95.12424123)
\curveto(126.77738542,94.96423994)(126.4923857,94.82424008)(126.18238403,94.70424123)
\curveto(126.04238615,94.64424026)(125.90238629,94.5992403)(125.76238403,94.56924123)
\curveto(125.61238658,94.53924036)(125.45738674,94.5042404)(125.29738403,94.46424123)
\curveto(125.18738701,94.44424046)(125.07738712,94.42924047)(124.96738403,94.41924123)
\curveto(124.85738734,94.40924049)(124.74738745,94.39424051)(124.63738403,94.37424123)
\curveto(124.5973876,94.36424054)(124.55738764,94.35924054)(124.51738403,94.35924123)
\curveto(124.47738772,94.36924053)(124.43738776,94.36924053)(124.39738403,94.35924123)
\curveto(124.34738785,94.34924055)(124.2973879,94.34424056)(124.24738403,94.34424123)
\lineto(124.08238403,94.34424123)
\curveto(124.03238816,94.32424058)(123.98238821,94.31924058)(123.93238403,94.32924123)
\curveto(123.87238832,94.33924056)(123.81738838,94.33924056)(123.76738403,94.32924123)
\curveto(123.72738847,94.31924058)(123.68238851,94.31924058)(123.63238403,94.32924123)
\curveto(123.58238861,94.33924056)(123.53238866,94.33424057)(123.48238403,94.31424123)
\curveto(123.41238878,94.29424061)(123.33738886,94.28924061)(123.25738403,94.29924123)
\curveto(123.16738903,94.30924059)(123.08238911,94.31424059)(123.00238403,94.31424123)
\curveto(122.91238928,94.31424059)(122.81238938,94.30924059)(122.70238403,94.29924123)
\curveto(122.58238961,94.28924061)(122.48238971,94.29424061)(122.40238403,94.31424123)
\lineto(122.11738403,94.31424123)
\lineto(121.48738403,94.35924123)
\curveto(121.38739081,94.36924053)(121.2923909,94.37924052)(121.20238403,94.38924123)
\lineto(120.90238403,94.41924123)
\curveto(120.85239134,94.43924046)(120.80239139,94.44424046)(120.75238403,94.43424123)
\curveto(120.6923915,94.43424047)(120.63739156,94.44424046)(120.58738403,94.46424123)
\curveto(120.41739178,94.51424039)(120.25239194,94.55424035)(120.09238403,94.58424123)
\curveto(119.92239227,94.61424029)(119.76239243,94.66424024)(119.61238403,94.73424123)
\curveto(119.15239304,94.92423998)(118.77739342,95.14423976)(118.48738403,95.39424123)
\curveto(118.197394,95.65423925)(117.95239424,96.01423889)(117.75238403,96.47424123)
\curveto(117.70239449,96.6042383)(117.66739453,96.73423817)(117.64738403,96.86424123)
\curveto(117.62739457,97.0042379)(117.60239459,97.14423776)(117.57238403,97.28424123)
\curveto(117.56239463,97.35423755)(117.55739464,97.41923748)(117.55738403,97.47924123)
\curveto(117.55739464,97.53923736)(117.55239464,97.6042373)(117.54238403,97.67424123)
\curveto(117.52239467,98.5042364)(117.67239452,99.17423573)(117.99238403,99.68424123)
\curveto(118.30239389,100.19423471)(118.74239345,100.57423433)(119.31238403,100.82424123)
\curveto(119.43239276,100.87423403)(119.55739264,100.91923398)(119.68738403,100.95924123)
\curveto(119.81739238,100.9992339)(119.95239224,101.04423386)(120.09238403,101.09424123)
\curveto(120.17239202,101.11423379)(120.25739194,101.12923377)(120.34738403,101.13924123)
\lineto(120.58738403,101.19924123)
\curveto(120.6973915,101.22923367)(120.80739139,101.24423366)(120.91738403,101.24424123)
\curveto(121.02739117,101.25423365)(121.13739106,101.26923363)(121.24738403,101.28924123)
\curveto(121.2973909,101.30923359)(121.34239085,101.31423359)(121.38238403,101.30424123)
\curveto(121.42239077,101.3042336)(121.46239073,101.30923359)(121.50238403,101.31924123)
\curveto(121.55239064,101.32923357)(121.60739059,101.32923357)(121.66738403,101.31924123)
\curveto(121.71739048,101.31923358)(121.76739043,101.32423358)(121.81738403,101.33424123)
\lineto(121.95238403,101.33424123)
\curveto(122.01239018,101.35423355)(122.08239011,101.35423355)(122.16238403,101.33424123)
\curveto(122.23238996,101.32423358)(122.2973899,101.32923357)(122.35738403,101.34924123)
\curveto(122.38738981,101.35923354)(122.42738977,101.36423354)(122.47738403,101.36424123)
\lineto(122.59738403,101.36424123)
\lineto(123.06238403,101.36424123)
\moveto(125.38738403,99.81924123)
\curveto(125.06738713,99.91923498)(124.70238749,99.97923492)(124.29238403,99.99924123)
\curveto(123.88238831,100.01923488)(123.47238872,100.02923487)(123.06238403,100.02924123)
\curveto(122.63238956,100.02923487)(122.21238998,100.01923488)(121.80238403,99.99924123)
\curveto(121.3923908,99.97923492)(121.00739119,99.93423497)(120.64738403,99.86424123)
\curveto(120.28739191,99.79423511)(119.96739223,99.68423522)(119.68738403,99.53424123)
\curveto(119.3973928,99.39423551)(119.16239303,99.1992357)(118.98238403,98.94924123)
\curveto(118.87239332,98.78923611)(118.7923934,98.60923629)(118.74238403,98.40924123)
\curveto(118.68239351,98.20923669)(118.65239354,97.96423694)(118.65238403,97.67424123)
\curveto(118.67239352,97.65423725)(118.68239351,97.61923728)(118.68238403,97.56924123)
\curveto(118.67239352,97.51923738)(118.67239352,97.47923742)(118.68238403,97.44924123)
\curveto(118.70239349,97.36923753)(118.72239347,97.29423761)(118.74238403,97.22424123)
\curveto(118.75239344,97.16423774)(118.77239342,97.0992378)(118.80238403,97.02924123)
\curveto(118.92239327,96.75923814)(119.0923931,96.53923836)(119.31238403,96.36924123)
\curveto(119.52239267,96.20923869)(119.76739243,96.07423883)(120.04738403,95.96424123)
\curveto(120.15739204,95.91423899)(120.27739192,95.87423903)(120.40738403,95.84424123)
\curveto(120.52739167,95.82423908)(120.65239154,95.7992391)(120.78238403,95.76924123)
\curveto(120.83239136,95.74923915)(120.88739131,95.73923916)(120.94738403,95.73924123)
\curveto(120.9973912,95.73923916)(121.04739115,95.73423917)(121.09738403,95.72424123)
\curveto(121.18739101,95.71423919)(121.28239091,95.7042392)(121.38238403,95.69424123)
\curveto(121.47239072,95.68423922)(121.56739063,95.67423923)(121.66738403,95.66424123)
\curveto(121.74739045,95.66423924)(121.83239036,95.65923924)(121.92238403,95.64924123)
\lineto(122.16238403,95.64924123)
\lineto(122.34238403,95.64924123)
\curveto(122.37238982,95.63923926)(122.40738979,95.63423927)(122.44738403,95.63424123)
\lineto(122.58238403,95.63424123)
\lineto(123.03238403,95.63424123)
\curveto(123.11238908,95.63423927)(123.197389,95.62923927)(123.28738403,95.61924123)
\curveto(123.36738883,95.61923928)(123.44238875,95.62923927)(123.51238403,95.64924123)
\lineto(123.78238403,95.64924123)
\curveto(123.80238839,95.64923925)(123.83238836,95.64423926)(123.87238403,95.63424123)
\curveto(123.90238829,95.63423927)(123.92738827,95.63923926)(123.94738403,95.64924123)
\curveto(124.04738815,95.65923924)(124.14738805,95.66423924)(124.24738403,95.66424123)
\curveto(124.33738786,95.67423923)(124.43738776,95.68423922)(124.54738403,95.69424123)
\curveto(124.66738753,95.72423918)(124.7923874,95.73923916)(124.92238403,95.73924123)
\curveto(125.04238715,95.74923915)(125.15738704,95.77423913)(125.26738403,95.81424123)
\curveto(125.56738663,95.89423901)(125.83238636,95.97923892)(126.06238403,96.06924123)
\curveto(126.2923859,96.16923873)(126.50738569,96.31423859)(126.70738403,96.50424123)
\curveto(126.90738529,96.71423819)(127.05738514,96.97923792)(127.15738403,97.29924123)
\curveto(127.17738502,97.33923756)(127.18738501,97.37423753)(127.18738403,97.40424123)
\curveto(127.17738502,97.44423746)(127.18238501,97.48923741)(127.20238403,97.53924123)
\curveto(127.21238498,97.57923732)(127.22238497,97.64923725)(127.23238403,97.74924123)
\curveto(127.24238495,97.85923704)(127.23738496,97.94423696)(127.21738403,98.00424123)
\curveto(127.197385,98.07423683)(127.18738501,98.14423676)(127.18738403,98.21424123)
\curveto(127.17738502,98.28423662)(127.16238503,98.34923655)(127.14238403,98.40924123)
\curveto(127.08238511,98.60923629)(126.9973852,98.78923611)(126.88738403,98.94924123)
\curveto(126.86738533,98.97923592)(126.84738535,99.0042359)(126.82738403,99.02424123)
\lineto(126.76738403,99.08424123)
\curveto(126.74738545,99.12423578)(126.70738549,99.17423573)(126.64738403,99.23424123)
\curveto(126.50738569,99.33423557)(126.37738582,99.41923548)(126.25738403,99.48924123)
\curveto(126.13738606,99.55923534)(125.9923862,99.62923527)(125.82238403,99.69924123)
\curveto(125.75238644,99.72923517)(125.68238651,99.74923515)(125.61238403,99.75924123)
\curveto(125.54238665,99.77923512)(125.46738673,99.7992351)(125.38738403,99.81924123)
}
}
{
\newrgbcolor{curcolor}{0 0 0}
\pscustom[linestyle=none,fillstyle=solid,fillcolor=curcolor]
{
\newpath
\moveto(117.54238403,106.77385061)
\curveto(117.54239465,106.87384575)(117.55239464,106.96884566)(117.57238403,107.05885061)
\curveto(117.58239461,107.14884548)(117.61239458,107.21384541)(117.66238403,107.25385061)
\curveto(117.74239445,107.31384531)(117.84739435,107.34384528)(117.97738403,107.34385061)
\lineto(118.36738403,107.34385061)
\lineto(119.86738403,107.34385061)
\lineto(126.25738403,107.34385061)
\lineto(127.42738403,107.34385061)
\lineto(127.74238403,107.34385061)
\curveto(127.84238435,107.35384527)(127.92238427,107.33884529)(127.98238403,107.29885061)
\curveto(128.06238413,107.24884538)(128.11238408,107.17384545)(128.13238403,107.07385061)
\curveto(128.14238405,106.98384564)(128.14738405,106.87384575)(128.14738403,106.74385061)
\lineto(128.14738403,106.51885061)
\curveto(128.12738407,106.43884619)(128.11238408,106.36884626)(128.10238403,106.30885061)
\curveto(128.08238411,106.24884638)(128.04238415,106.19884643)(127.98238403,106.15885061)
\curveto(127.92238427,106.11884651)(127.84738435,106.09884653)(127.75738403,106.09885061)
\lineto(127.45738403,106.09885061)
\lineto(126.36238403,106.09885061)
\lineto(121.02238403,106.09885061)
\curveto(120.93239126,106.07884655)(120.85739134,106.06384656)(120.79738403,106.05385061)
\curveto(120.72739147,106.05384657)(120.66739153,106.0238466)(120.61738403,105.96385061)
\curveto(120.56739163,105.89384673)(120.54239165,105.80384682)(120.54238403,105.69385061)
\curveto(120.53239166,105.59384703)(120.52739167,105.48384714)(120.52738403,105.36385061)
\lineto(120.52738403,104.22385061)
\lineto(120.52738403,103.72885061)
\curveto(120.51739168,103.56884906)(120.45739174,103.45884917)(120.34738403,103.39885061)
\curveto(120.31739188,103.37884925)(120.28739191,103.36884926)(120.25738403,103.36885061)
\curveto(120.21739198,103.36884926)(120.17239202,103.36384926)(120.12238403,103.35385061)
\curveto(120.00239219,103.33384929)(119.8923923,103.33884929)(119.79238403,103.36885061)
\curveto(119.6923925,103.40884922)(119.62239257,103.46384916)(119.58238403,103.53385061)
\curveto(119.53239266,103.61384901)(119.50739269,103.73384889)(119.50738403,103.89385061)
\curveto(119.50739269,104.05384857)(119.4923927,104.18884844)(119.46238403,104.29885061)
\curveto(119.45239274,104.34884828)(119.44739275,104.40384822)(119.44738403,104.46385061)
\curveto(119.43739276,104.5238481)(119.42239277,104.58384804)(119.40238403,104.64385061)
\curveto(119.35239284,104.79384783)(119.30239289,104.93884769)(119.25238403,105.07885061)
\curveto(119.192393,105.21884741)(119.12239307,105.35384727)(119.04238403,105.48385061)
\curveto(118.95239324,105.623847)(118.84739335,105.74384688)(118.72738403,105.84385061)
\curveto(118.60739359,105.94384668)(118.47739372,106.03884659)(118.33738403,106.12885061)
\curveto(118.23739396,106.18884644)(118.12739407,106.23384639)(118.00738403,106.26385061)
\curveto(117.88739431,106.30384632)(117.78239441,106.35384627)(117.69238403,106.41385061)
\curveto(117.63239456,106.46384616)(117.5923946,106.53384609)(117.57238403,106.62385061)
\curveto(117.56239463,106.64384598)(117.55739464,106.66884596)(117.55738403,106.69885061)
\curveto(117.55739464,106.7288459)(117.55239464,106.75384587)(117.54238403,106.77385061)
}
}
{
\newrgbcolor{curcolor}{0 0 0}
\pscustom[linestyle=none,fillstyle=solid,fillcolor=curcolor]
{
\newpath
\moveto(117.54238403,115.12345998)
\curveto(117.54239465,115.22345513)(117.55239464,115.31845503)(117.57238403,115.40845998)
\curveto(117.58239461,115.49845485)(117.61239458,115.56345479)(117.66238403,115.60345998)
\curveto(117.74239445,115.66345469)(117.84739435,115.69345466)(117.97738403,115.69345998)
\lineto(118.36738403,115.69345998)
\lineto(119.86738403,115.69345998)
\lineto(126.25738403,115.69345998)
\lineto(127.42738403,115.69345998)
\lineto(127.74238403,115.69345998)
\curveto(127.84238435,115.70345465)(127.92238427,115.68845466)(127.98238403,115.64845998)
\curveto(128.06238413,115.59845475)(128.11238408,115.52345483)(128.13238403,115.42345998)
\curveto(128.14238405,115.33345502)(128.14738405,115.22345513)(128.14738403,115.09345998)
\lineto(128.14738403,114.86845998)
\curveto(128.12738407,114.78845556)(128.11238408,114.71845563)(128.10238403,114.65845998)
\curveto(128.08238411,114.59845575)(128.04238415,114.5484558)(127.98238403,114.50845998)
\curveto(127.92238427,114.46845588)(127.84738435,114.4484559)(127.75738403,114.44845998)
\lineto(127.45738403,114.44845998)
\lineto(126.36238403,114.44845998)
\lineto(121.02238403,114.44845998)
\curveto(120.93239126,114.42845592)(120.85739134,114.41345594)(120.79738403,114.40345998)
\curveto(120.72739147,114.40345595)(120.66739153,114.37345598)(120.61738403,114.31345998)
\curveto(120.56739163,114.24345611)(120.54239165,114.1534562)(120.54238403,114.04345998)
\curveto(120.53239166,113.94345641)(120.52739167,113.83345652)(120.52738403,113.71345998)
\lineto(120.52738403,112.57345998)
\lineto(120.52738403,112.07845998)
\curveto(120.51739168,111.91845843)(120.45739174,111.80845854)(120.34738403,111.74845998)
\curveto(120.31739188,111.72845862)(120.28739191,111.71845863)(120.25738403,111.71845998)
\curveto(120.21739198,111.71845863)(120.17239202,111.71345864)(120.12238403,111.70345998)
\curveto(120.00239219,111.68345867)(119.8923923,111.68845866)(119.79238403,111.71845998)
\curveto(119.6923925,111.75845859)(119.62239257,111.81345854)(119.58238403,111.88345998)
\curveto(119.53239266,111.96345839)(119.50739269,112.08345827)(119.50738403,112.24345998)
\curveto(119.50739269,112.40345795)(119.4923927,112.53845781)(119.46238403,112.64845998)
\curveto(119.45239274,112.69845765)(119.44739275,112.7534576)(119.44738403,112.81345998)
\curveto(119.43739276,112.87345748)(119.42239277,112.93345742)(119.40238403,112.99345998)
\curveto(119.35239284,113.14345721)(119.30239289,113.28845706)(119.25238403,113.42845998)
\curveto(119.192393,113.56845678)(119.12239307,113.70345665)(119.04238403,113.83345998)
\curveto(118.95239324,113.97345638)(118.84739335,114.09345626)(118.72738403,114.19345998)
\curveto(118.60739359,114.29345606)(118.47739372,114.38845596)(118.33738403,114.47845998)
\curveto(118.23739396,114.53845581)(118.12739407,114.58345577)(118.00738403,114.61345998)
\curveto(117.88739431,114.6534557)(117.78239441,114.70345565)(117.69238403,114.76345998)
\curveto(117.63239456,114.81345554)(117.5923946,114.88345547)(117.57238403,114.97345998)
\curveto(117.56239463,114.99345536)(117.55739464,115.01845533)(117.55738403,115.04845998)
\curveto(117.55739464,115.07845527)(117.55239464,115.10345525)(117.54238403,115.12345998)
}
}
\end{pspicture}

%\caption{Linea de tiempo de la creación de elementos en el sistema}
%\label{tiempos_area_1}
%\end{figure}

En la figura \ref{tiempos_area_2} resalta la curiosa relación entre las lineas
de creación de recursos y la de creación de contactos, siendo esta la linea que
determina todo el objeto de investigación.

%\begin{figure}[H]
%\centering
%%LaTeX with PSTricks extensions
%%Creator: inkscape 0.48.5
%%Please note this file requires PSTricks extensions
\psset{xunit=.5pt,yunit=.5pt,runit=.5pt}
\begin{pspicture}(1052.35998535,480)
{
\newrgbcolor{curcolor}{0 0 0}
\pscustom[linestyle=none,fillstyle=solid,fillcolor=curcolor]
{
\newpath
\moveto(36.46913239,445.95727663)
\curveto(36.47912467,445.91727358)(36.47912467,445.86727363)(36.46913239,445.80727663)
\curveto(36.46912468,445.74727375)(36.46412468,445.6972738)(36.45413239,445.65727663)
\curveto(36.45412469,445.61727388)(36.4491247,445.57727392)(36.43913239,445.53727663)
\lineto(36.43913239,445.43227663)
\curveto(36.41912473,445.35227415)(36.40412474,445.27227423)(36.39413239,445.19227663)
\curveto(36.38412476,445.11227439)(36.36412478,445.03727446)(36.33413239,444.96727663)
\curveto(36.31412483,444.88727461)(36.29412485,444.81227469)(36.27413239,444.74227663)
\curveto(36.25412489,444.67227483)(36.22412492,444.5972749)(36.18413239,444.51727663)
\curveto(36.00412514,444.0972754)(35.7491254,443.75727574)(35.41913239,443.49727663)
\curveto(35.08912606,443.23727626)(34.69912645,443.03227647)(34.24913239,442.88227663)
\curveto(34.12912702,442.84227666)(34.00412714,442.81727668)(33.87413239,442.80727663)
\curveto(33.75412739,442.78727671)(33.62912752,442.76227674)(33.49913239,442.73227663)
\curveto(33.43912771,442.72227678)(33.37412777,442.71727678)(33.30413239,442.71727663)
\curveto(33.2441279,442.71727678)(33.17912797,442.71227679)(33.10913239,442.70227663)
\lineto(32.98913239,442.70227663)
\lineto(32.79413239,442.70227663)
\curveto(32.73412841,442.69227681)(32.67912847,442.6972768)(32.62913239,442.71727663)
\curveto(32.55912859,442.73727676)(32.49412865,442.74227676)(32.43413239,442.73227663)
\curveto(32.37412877,442.72227678)(32.31412883,442.72727677)(32.25413239,442.74727663)
\curveto(32.20412894,442.75727674)(32.15912899,442.76227674)(32.11913239,442.76227663)
\curveto(32.07912907,442.76227674)(32.03412911,442.77227673)(31.98413239,442.79227663)
\curveto(31.90412924,442.81227669)(31.82912932,442.83227667)(31.75913239,442.85227663)
\curveto(31.68912946,442.86227664)(31.61912953,442.87727662)(31.54913239,442.89727663)
\curveto(31.06913008,443.06727643)(30.66913048,443.27727622)(30.34913239,443.52727663)
\curveto(30.03913111,443.78727571)(29.78913136,444.14227536)(29.59913239,444.59227663)
\curveto(29.56913158,444.65227485)(29.5441316,444.71227479)(29.52413239,444.77227663)
\curveto(29.51413163,444.84227466)(29.49913165,444.91727458)(29.47913239,444.99727663)
\curveto(29.45913169,445.05727444)(29.4441317,445.12227438)(29.43413239,445.19227663)
\curveto(29.42413172,445.26227424)(29.40913174,445.33227417)(29.38913239,445.40227663)
\curveto(29.37913177,445.45227405)(29.37413177,445.49227401)(29.37413239,445.52227663)
\lineto(29.37413239,445.64227663)
\curveto(29.36413178,445.68227382)(29.35413179,445.73227377)(29.34413239,445.79227663)
\curveto(29.3441318,445.85227365)(29.3491318,445.9022736)(29.35913239,445.94227663)
\lineto(29.35913239,446.07727663)
\curveto(29.36913178,446.12727337)(29.37413177,446.17727332)(29.37413239,446.22727663)
\curveto(29.39413175,446.32727317)(29.40913174,446.42227308)(29.41913239,446.51227663)
\curveto(29.42913172,446.61227289)(29.4491317,446.70727279)(29.47913239,446.79727663)
\curveto(29.52913162,446.94727255)(29.58413156,447.08727241)(29.64413239,447.21727663)
\curveto(29.70413144,447.34727215)(29.77413137,447.46727203)(29.85413239,447.57727663)
\curveto(29.88413126,447.62727187)(29.91413123,447.66727183)(29.94413239,447.69727663)
\curveto(29.98413116,447.72727177)(30.01913113,447.76227174)(30.04913239,447.80227663)
\curveto(30.10913104,447.88227162)(30.17913097,447.95227155)(30.25913239,448.01227663)
\curveto(30.31913083,448.06227144)(30.37913077,448.10727139)(30.43913239,448.14727663)
\lineto(30.64913239,448.29727663)
\curveto(30.69913045,448.33727116)(30.7491304,448.37227113)(30.79913239,448.40227663)
\curveto(30.8491303,448.44227106)(30.88413026,448.497271)(30.90413239,448.56727663)
\curveto(30.90413024,448.5972709)(30.89413025,448.62227088)(30.87413239,448.64227663)
\curveto(30.86413028,448.67227083)(30.85413029,448.6972708)(30.84413239,448.71727663)
\curveto(30.80413034,448.76727073)(30.75413039,448.81227069)(30.69413239,448.85227663)
\curveto(30.6441305,448.9022706)(30.59413055,448.94727055)(30.54413239,448.98727663)
\curveto(30.50413064,449.01727048)(30.45413069,449.07227043)(30.39413239,449.15227663)
\curveto(30.37413077,449.18227032)(30.3441308,449.20727029)(30.30413239,449.22727663)
\curveto(30.27413087,449.25727024)(30.2491309,449.29227021)(30.22913239,449.33227663)
\curveto(30.05913109,449.54226996)(29.92913122,449.78726971)(29.83913239,450.06727663)
\curveto(29.81913133,450.14726935)(29.80413134,450.22726927)(29.79413239,450.30727663)
\curveto(29.78413136,450.38726911)(29.76913138,450.46726903)(29.74913239,450.54727663)
\curveto(29.72913142,450.5972689)(29.71913143,450.66226884)(29.71913239,450.74227663)
\curveto(29.71913143,450.83226867)(29.72913142,450.9022686)(29.74913239,450.95227663)
\curveto(29.7491314,451.05226845)(29.75413139,451.12226838)(29.76413239,451.16227663)
\curveto(29.78413136,451.24226826)(29.79913135,451.31226819)(29.80913239,451.37227663)
\curveto(29.81913133,451.44226806)(29.83413131,451.51226799)(29.85413239,451.58227663)
\curveto(30.00413114,452.01226749)(30.21913093,452.35726714)(30.49913239,452.61727663)
\curveto(30.78913036,452.87726662)(31.13913001,453.09226641)(31.54913239,453.26227663)
\curveto(31.65912949,453.31226619)(31.77412937,453.34226616)(31.89413239,453.35227663)
\curveto(32.02412912,453.37226613)(32.15412899,453.4022661)(32.28413239,453.44227663)
\curveto(32.36412878,453.44226606)(32.43412871,453.44226606)(32.49413239,453.44227663)
\curveto(32.56412858,453.45226605)(32.63912851,453.46226604)(32.71913239,453.47227663)
\curveto(33.50912764,453.49226601)(34.16412698,453.36226614)(34.68413239,453.08227663)
\curveto(35.21412593,452.8022667)(35.59412555,452.39226711)(35.82413239,451.85227663)
\curveto(35.93412521,451.62226788)(36.00412514,451.33726816)(36.03413239,450.99727663)
\curveto(36.07412507,450.66726883)(36.0441251,450.36226914)(35.94413239,450.08227663)
\curveto(35.90412524,449.95226955)(35.85412529,449.83226967)(35.79413239,449.72227663)
\curveto(35.7441254,449.61226989)(35.68412546,449.50726999)(35.61413239,449.40727663)
\curveto(35.59412555,449.36727013)(35.56412558,449.33227017)(35.52413239,449.30227663)
\lineto(35.43413239,449.21227663)
\curveto(35.38412576,449.12227038)(35.32412582,449.05727044)(35.25413239,449.01727663)
\curveto(35.20412594,448.96727053)(35.149126,448.91727058)(35.08913239,448.86727663)
\curveto(35.03912611,448.82727067)(34.99412615,448.78227072)(34.95413239,448.73227663)
\curveto(34.93412621,448.71227079)(34.91412623,448.68727081)(34.89413239,448.65727663)
\curveto(34.88412626,448.63727086)(34.88412626,448.61227089)(34.89413239,448.58227663)
\curveto(34.90412624,448.53227097)(34.93412621,448.48227102)(34.98413239,448.43227663)
\curveto(35.03412611,448.39227111)(35.08912606,448.35227115)(35.14913239,448.31227663)
\lineto(35.32913239,448.19227663)
\curveto(35.38912576,448.16227134)(35.43912571,448.13227137)(35.47913239,448.10227663)
\curveto(35.80912534,447.86227164)(36.05912509,447.55227195)(36.22913239,447.17227663)
\curveto(36.26912488,447.09227241)(36.29912485,447.00727249)(36.31913239,446.91727663)
\curveto(36.3491248,446.82727267)(36.37412477,446.73727276)(36.39413239,446.64727663)
\curveto(36.40412474,446.5972729)(36.41412473,446.54227296)(36.42413239,446.48227663)
\lineto(36.45413239,446.33227663)
\curveto(36.46412468,446.27227323)(36.46412468,446.20727329)(36.45413239,446.13727663)
\curveto(36.4441247,446.07727342)(36.4491247,446.01727348)(36.46913239,445.95727663)
\moveto(31.08413239,450.99727663)
\curveto(31.05413009,450.88726861)(31.0491301,450.74726875)(31.06913239,450.57727663)
\curveto(31.08913006,450.41726908)(31.11413003,450.29226921)(31.14413239,450.20227663)
\curveto(31.25412989,449.88226962)(31.40412974,449.63726986)(31.59413239,449.46727663)
\curveto(31.78412936,449.30727019)(32.0491291,449.17727032)(32.38913239,449.07727663)
\curveto(32.51912863,449.04727045)(32.68412846,449.02227048)(32.88413239,449.00227663)
\curveto(33.08412806,448.99227051)(33.25412789,449.00727049)(33.39413239,449.04727663)
\curveto(33.68412746,449.12727037)(33.92412722,449.23727026)(34.11413239,449.37727663)
\curveto(34.31412683,449.52726997)(34.46912668,449.72726977)(34.57913239,449.97727663)
\curveto(34.59912655,450.02726947)(34.60912654,450.07226943)(34.60913239,450.11227663)
\curveto(34.61912653,450.15226935)(34.63412651,450.1972693)(34.65413239,450.24727663)
\curveto(34.68412646,450.35726914)(34.70412644,450.497269)(34.71413239,450.66727663)
\curveto(34.72412642,450.83726866)(34.71412643,450.98226852)(34.68413239,451.10227663)
\curveto(34.66412648,451.19226831)(34.63912651,451.27726822)(34.60913239,451.35727663)
\curveto(34.58912656,451.43726806)(34.55412659,451.51726798)(34.50413239,451.59727663)
\curveto(34.33412681,451.86726763)(34.10912704,452.06226744)(33.82913239,452.18227663)
\curveto(33.55912759,452.3022672)(33.19912795,452.36226714)(32.74913239,452.36227663)
\curveto(32.72912842,452.34226716)(32.69912845,452.33726716)(32.65913239,452.34727663)
\curveto(32.61912853,452.35726714)(32.58412856,452.35726714)(32.55413239,452.34727663)
\curveto(32.50412864,452.32726717)(32.4491287,452.31226719)(32.38913239,452.30227663)
\curveto(32.33912881,452.3022672)(32.28912886,452.29226721)(32.23913239,452.27227663)
\curveto(31.99912915,452.18226732)(31.78912936,452.06726743)(31.60913239,451.92727663)
\curveto(31.42912972,451.7972677)(31.28912986,451.61726788)(31.18913239,451.38727663)
\curveto(31.16912998,451.32726817)(31.14913,451.26226824)(31.12913239,451.19227663)
\curveto(31.11913003,451.13226837)(31.10413004,451.06726843)(31.08413239,450.99727663)
\moveto(35.10413239,445.46227663)
\curveto(35.15412599,445.65227385)(35.15912599,445.85727364)(35.11913239,446.07727663)
\curveto(35.08912606,446.2972732)(35.0441261,446.47727302)(34.98413239,446.61727663)
\curveto(34.81412633,446.98727251)(34.55412659,447.29227221)(34.20413239,447.53227663)
\curveto(33.86412728,447.77227173)(33.42912772,447.89227161)(32.89913239,447.89227663)
\curveto(32.86912828,447.87227163)(32.82912832,447.86727163)(32.77913239,447.87727663)
\curveto(32.72912842,447.8972716)(32.68912846,447.9022716)(32.65913239,447.89227663)
\lineto(32.38913239,447.83227663)
\curveto(32.30912884,447.82227168)(32.22912892,447.80727169)(32.14913239,447.78727663)
\curveto(31.8491293,447.67727182)(31.58412956,447.53227197)(31.35413239,447.35227663)
\curveto(31.13413001,447.17227233)(30.96413018,446.94227256)(30.84413239,446.66227663)
\curveto(30.81413033,446.58227292)(30.78913036,446.502273)(30.76913239,446.42227663)
\curveto(30.7491304,446.34227316)(30.72913042,446.25727324)(30.70913239,446.16727663)
\curveto(30.67913047,446.04727345)(30.66913048,445.8972736)(30.67913239,445.71727663)
\curveto(30.69913045,445.53727396)(30.72413042,445.3972741)(30.75413239,445.29727663)
\curveto(30.77413037,445.24727425)(30.78413036,445.2022743)(30.78413239,445.16227663)
\curveto(30.79413035,445.13227437)(30.80913034,445.09227441)(30.82913239,445.04227663)
\curveto(30.92913022,444.82227468)(31.05913009,444.62227488)(31.21913239,444.44227663)
\curveto(31.38912976,444.26227524)(31.58412956,444.12727537)(31.80413239,444.03727663)
\curveto(31.87412927,443.9972755)(31.96912918,443.96227554)(32.08913239,443.93227663)
\curveto(32.30912884,443.84227566)(32.56412858,443.7972757)(32.85413239,443.79727663)
\lineto(33.13913239,443.79727663)
\curveto(33.23912791,443.81727568)(33.33412781,443.83227567)(33.42413239,443.84227663)
\curveto(33.51412763,443.85227565)(33.60412754,443.87227563)(33.69413239,443.90227663)
\curveto(33.95412719,443.98227552)(34.19412695,444.11227539)(34.41413239,444.29227663)
\curveto(34.6441265,444.48227502)(34.81412633,444.6972748)(34.92413239,444.93727663)
\curveto(34.96412618,445.01727448)(34.99412615,445.0972744)(35.01413239,445.17727663)
\curveto(35.0441261,445.26727423)(35.07412607,445.36227414)(35.10413239,445.46227663)
}
}
{
\newrgbcolor{curcolor}{0 0 0}
\pscustom[linestyle=none,fillstyle=solid,fillcolor=curcolor]
{
\newpath
\moveto(44.80374176,447.96727663)
\lineto(44.80374176,447.71227663)
\curveto(44.81373406,447.63227187)(44.80873406,447.55727194)(44.78874176,447.48727663)
\lineto(44.78874176,447.24727663)
\lineto(44.78874176,447.08227663)
\curveto(44.7687341,446.98227252)(44.75873411,446.87727262)(44.75874176,446.76727663)
\curveto(44.75873411,446.66727283)(44.74873412,446.56727293)(44.72874176,446.46727663)
\lineto(44.72874176,446.31727663)
\curveto(44.69873417,446.17727332)(44.67873419,446.03727346)(44.66874176,445.89727663)
\curveto(44.65873421,445.76727373)(44.63373424,445.63727386)(44.59374176,445.50727663)
\curveto(44.5737343,445.42727407)(44.55373432,445.34227416)(44.53374176,445.25227663)
\lineto(44.47374176,445.01227663)
\lineto(44.35374176,444.71227663)
\curveto(44.32373455,444.62227488)(44.28873458,444.53227497)(44.24874176,444.44227663)
\curveto(44.14873472,444.22227528)(44.01373486,444.00727549)(43.84374176,443.79727663)
\curveto(43.68373519,443.58727591)(43.50873536,443.41727608)(43.31874176,443.28727663)
\curveto(43.2687356,443.24727625)(43.20873566,443.20727629)(43.13874176,443.16727663)
\curveto(43.07873579,443.13727636)(43.01873585,443.1022764)(42.95874176,443.06227663)
\curveto(42.87873599,443.01227649)(42.78373609,442.97227653)(42.67374176,442.94227663)
\curveto(42.56373631,442.91227659)(42.45873641,442.88227662)(42.35874176,442.85227663)
\curveto(42.24873662,442.81227669)(42.13873673,442.78727671)(42.02874176,442.77727663)
\curveto(41.91873695,442.76727673)(41.80373707,442.75227675)(41.68374176,442.73227663)
\curveto(41.64373723,442.72227678)(41.59873727,442.72227678)(41.54874176,442.73227663)
\curveto(41.50873736,442.73227677)(41.4687374,442.72727677)(41.42874176,442.71727663)
\curveto(41.38873748,442.70727679)(41.33373754,442.7022768)(41.26374176,442.70227663)
\curveto(41.19373768,442.7022768)(41.14373773,442.70727679)(41.11374176,442.71727663)
\curveto(41.06373781,442.73727676)(41.01873785,442.74227676)(40.97874176,442.73227663)
\curveto(40.93873793,442.72227678)(40.90373797,442.72227678)(40.87374176,442.73227663)
\lineto(40.78374176,442.73227663)
\curveto(40.72373815,442.75227675)(40.65873821,442.76727673)(40.58874176,442.77727663)
\curveto(40.52873834,442.77727672)(40.46373841,442.78227672)(40.39374176,442.79227663)
\curveto(40.22373865,442.84227666)(40.06373881,442.89227661)(39.91374176,442.94227663)
\curveto(39.76373911,442.99227651)(39.61873925,443.05727644)(39.47874176,443.13727663)
\curveto(39.42873944,443.17727632)(39.3737395,443.20727629)(39.31374176,443.22727663)
\curveto(39.26373961,443.25727624)(39.21373966,443.29227621)(39.16374176,443.33227663)
\curveto(38.92373995,443.51227599)(38.72374015,443.73227577)(38.56374176,443.99227663)
\curveto(38.40374047,444.25227525)(38.26374061,444.53727496)(38.14374176,444.84727663)
\curveto(38.08374079,444.98727451)(38.03874083,445.12727437)(38.00874176,445.26727663)
\curveto(37.97874089,445.41727408)(37.94374093,445.57227393)(37.90374176,445.73227663)
\curveto(37.88374099,445.84227366)(37.868741,445.95227355)(37.85874176,446.06227663)
\curveto(37.84874102,446.17227333)(37.83374104,446.28227322)(37.81374176,446.39227663)
\curveto(37.80374107,446.43227307)(37.79874107,446.47227303)(37.79874176,446.51227663)
\curveto(37.80874106,446.55227295)(37.80874106,446.59227291)(37.79874176,446.63227663)
\curveto(37.78874108,446.68227282)(37.78374109,446.73227277)(37.78374176,446.78227663)
\lineto(37.78374176,446.94727663)
\curveto(37.76374111,446.9972725)(37.75874111,447.04727245)(37.76874176,447.09727663)
\curveto(37.77874109,447.15727234)(37.77874109,447.21227229)(37.76874176,447.26227663)
\curveto(37.75874111,447.3022722)(37.75874111,447.34727215)(37.76874176,447.39727663)
\curveto(37.77874109,447.44727205)(37.7737411,447.497272)(37.75374176,447.54727663)
\curveto(37.73374114,447.61727188)(37.72874114,447.69227181)(37.73874176,447.77227663)
\curveto(37.74874112,447.86227164)(37.75374112,447.94727155)(37.75374176,448.02727663)
\curveto(37.75374112,448.11727138)(37.74874112,448.21727128)(37.73874176,448.32727663)
\curveto(37.72874114,448.44727105)(37.73374114,448.54727095)(37.75374176,448.62727663)
\lineto(37.75374176,448.91227663)
\lineto(37.79874176,449.54227663)
\curveto(37.80874106,449.64226986)(37.81874105,449.73726976)(37.82874176,449.82727663)
\lineto(37.85874176,450.12727663)
\curveto(37.87874099,450.17726932)(37.88374099,450.22726927)(37.87374176,450.27727663)
\curveto(37.873741,450.33726916)(37.88374099,450.39226911)(37.90374176,450.44227663)
\curveto(37.95374092,450.61226889)(37.99374088,450.77726872)(38.02374176,450.93727663)
\curveto(38.05374082,451.10726839)(38.10374077,451.26726823)(38.17374176,451.41727663)
\curveto(38.36374051,451.87726762)(38.58374029,452.25226725)(38.83374176,452.54227663)
\curveto(39.09373978,452.83226667)(39.45373942,453.07726642)(39.91374176,453.27727663)
\curveto(40.04373883,453.32726617)(40.1737387,453.36226614)(40.30374176,453.38227663)
\curveto(40.44373843,453.4022661)(40.58373829,453.42726607)(40.72374176,453.45727663)
\curveto(40.79373808,453.46726603)(40.85873801,453.47226603)(40.91874176,453.47227663)
\curveto(40.97873789,453.47226603)(41.04373783,453.47726602)(41.11374176,453.48727663)
\curveto(41.94373693,453.50726599)(42.61373626,453.35726614)(43.12374176,453.03727663)
\curveto(43.63373524,452.72726677)(44.01373486,452.28726721)(44.26374176,451.71727663)
\curveto(44.31373456,451.5972679)(44.35873451,451.47226803)(44.39874176,451.34227663)
\curveto(44.43873443,451.21226829)(44.48373439,451.07726842)(44.53374176,450.93727663)
\curveto(44.55373432,450.85726864)(44.5687343,450.77226873)(44.57874176,450.68227663)
\lineto(44.63874176,450.44227663)
\curveto(44.6687342,450.33226917)(44.68373419,450.22226928)(44.68374176,450.11227663)
\curveto(44.69373418,450.0022695)(44.70873416,449.89226961)(44.72874176,449.78227663)
\curveto(44.74873412,449.73226977)(44.75373412,449.68726981)(44.74374176,449.64727663)
\curveto(44.74373413,449.60726989)(44.74873412,449.56726993)(44.75874176,449.52727663)
\curveto(44.7687341,449.47727002)(44.7687341,449.42227008)(44.75874176,449.36227663)
\curveto(44.75873411,449.31227019)(44.76373411,449.26227024)(44.77374176,449.21227663)
\lineto(44.77374176,449.07727663)
\curveto(44.79373408,449.01727048)(44.79373408,448.94727055)(44.77374176,448.86727663)
\curveto(44.76373411,448.7972707)(44.7687341,448.73227077)(44.78874176,448.67227663)
\curveto(44.79873407,448.64227086)(44.80373407,448.6022709)(44.80374176,448.55227663)
\lineto(44.80374176,448.43227663)
\lineto(44.80374176,447.96727663)
\moveto(43.25874176,445.64227663)
\curveto(43.35873551,445.96227354)(43.41873545,446.32727317)(43.43874176,446.73727663)
\curveto(43.45873541,447.14727235)(43.4687354,447.55727194)(43.46874176,447.96727663)
\curveto(43.4687354,448.3972711)(43.45873541,448.81727068)(43.43874176,449.22727663)
\curveto(43.41873545,449.63726986)(43.3737355,450.02226948)(43.30374176,450.38227663)
\curveto(43.23373564,450.74226876)(43.12373575,451.06226844)(42.97374176,451.34227663)
\curveto(42.83373604,451.63226787)(42.63873623,451.86726763)(42.38874176,452.04727663)
\curveto(42.22873664,452.15726734)(42.04873682,452.23726726)(41.84874176,452.28727663)
\curveto(41.64873722,452.34726715)(41.40373747,452.37726712)(41.11374176,452.37727663)
\curveto(41.09373778,452.35726714)(41.05873781,452.34726715)(41.00874176,452.34727663)
\curveto(40.95873791,452.35726714)(40.91873795,452.35726714)(40.88874176,452.34727663)
\curveto(40.80873806,452.32726717)(40.73373814,452.30726719)(40.66374176,452.28727663)
\curveto(40.60373827,452.27726722)(40.53873833,452.25726724)(40.46874176,452.22727663)
\curveto(40.19873867,452.10726739)(39.97873889,451.93726756)(39.80874176,451.71727663)
\curveto(39.64873922,451.50726799)(39.51373936,451.26226824)(39.40374176,450.98227663)
\curveto(39.35373952,450.87226863)(39.31373956,450.75226875)(39.28374176,450.62227663)
\curveto(39.26373961,450.502269)(39.23873963,450.37726912)(39.20874176,450.24727663)
\curveto(39.18873968,450.1972693)(39.17873969,450.14226936)(39.17874176,450.08227663)
\curveto(39.17873969,450.03226947)(39.1737397,449.98226952)(39.16374176,449.93227663)
\curveto(39.15373972,449.84226966)(39.14373973,449.74726975)(39.13374176,449.64727663)
\curveto(39.12373975,449.55726994)(39.11373976,449.46227004)(39.10374176,449.36227663)
\curveto(39.10373977,449.28227022)(39.09873977,449.1972703)(39.08874176,449.10727663)
\lineto(39.08874176,448.86727663)
\lineto(39.08874176,448.68727663)
\curveto(39.07873979,448.65727084)(39.0737398,448.62227088)(39.07374176,448.58227663)
\lineto(39.07374176,448.44727663)
\lineto(39.07374176,447.99727663)
\curveto(39.0737398,447.91727158)(39.0687398,447.83227167)(39.05874176,447.74227663)
\curveto(39.05873981,447.66227184)(39.0687398,447.58727191)(39.08874176,447.51727663)
\lineto(39.08874176,447.24727663)
\curveto(39.08873978,447.22727227)(39.08373979,447.1972723)(39.07374176,447.15727663)
\curveto(39.0737398,447.12727237)(39.07873979,447.1022724)(39.08874176,447.08227663)
\curveto(39.09873977,446.98227252)(39.10373977,446.88227262)(39.10374176,446.78227663)
\curveto(39.11373976,446.69227281)(39.12373975,446.59227291)(39.13374176,446.48227663)
\curveto(39.16373971,446.36227314)(39.17873969,446.23727326)(39.17874176,446.10727663)
\curveto(39.18873968,445.98727351)(39.21373966,445.87227363)(39.25374176,445.76227663)
\curveto(39.33373954,445.46227404)(39.41873945,445.1972743)(39.50874176,444.96727663)
\curveto(39.60873926,444.73727476)(39.75373912,444.52227498)(39.94374176,444.32227663)
\curveto(40.15373872,444.12227538)(40.41873845,443.97227553)(40.73874176,443.87227663)
\curveto(40.77873809,443.85227565)(40.81373806,443.84227566)(40.84374176,443.84227663)
\curveto(40.88373799,443.85227565)(40.92873794,443.84727565)(40.97874176,443.82727663)
\curveto(41.01873785,443.81727568)(41.08873778,443.80727569)(41.18874176,443.79727663)
\curveto(41.29873757,443.78727571)(41.38373749,443.79227571)(41.44374176,443.81227663)
\curveto(41.51373736,443.83227567)(41.58373729,443.84227566)(41.65374176,443.84227663)
\curveto(41.72373715,443.85227565)(41.78873708,443.86727563)(41.84874176,443.88727663)
\curveto(42.04873682,443.94727555)(42.22873664,444.03227547)(42.38874176,444.14227663)
\curveto(42.41873645,444.16227534)(42.44373643,444.18227532)(42.46374176,444.20227663)
\lineto(42.52374176,444.26227663)
\curveto(42.56373631,444.28227522)(42.61373626,444.32227518)(42.67374176,444.38227663)
\curveto(42.7737361,444.52227498)(42.85873601,444.65227485)(42.92874176,444.77227663)
\curveto(42.99873587,444.89227461)(43.0687358,445.03727446)(43.13874176,445.20727663)
\curveto(43.1687357,445.27727422)(43.18873568,445.34727415)(43.19874176,445.41727663)
\curveto(43.21873565,445.48727401)(43.23873563,445.56227394)(43.25874176,445.64227663)
}
}
{
\newrgbcolor{curcolor}{0 0 0}
\pscustom[linestyle=none,fillstyle=solid,fillcolor=curcolor]
{
\newpath
\moveto(36.45413239,367.04016726)
\curveto(36.48412466,366.92016304)(36.50912464,366.78016318)(36.52913239,366.62016726)
\curveto(36.5491246,366.4601635)(36.55912459,366.29516367)(36.55913239,366.12516726)
\curveto(36.55912459,365.95516401)(36.5491246,365.79016417)(36.52913239,365.63016726)
\curveto(36.50912464,365.47016449)(36.48412466,365.33016463)(36.45413239,365.21016726)
\curveto(36.41412473,365.07016489)(36.37912477,364.94516502)(36.34913239,364.83516726)
\curveto(36.31912483,364.72516524)(36.27912487,364.61516535)(36.22913239,364.50516726)
\curveto(35.95912519,363.8651661)(35.5441256,363.38016658)(34.98413239,363.05016726)
\curveto(34.90412624,362.99016697)(34.81912633,362.94016702)(34.72913239,362.90016726)
\curveto(34.63912651,362.87016709)(34.53912661,362.83516713)(34.42913239,362.79516726)
\curveto(34.31912683,362.74516722)(34.19912695,362.71016725)(34.06913239,362.69016726)
\curveto(33.9491272,362.6601673)(33.81912733,362.63016733)(33.67913239,362.60016726)
\curveto(33.61912753,362.58016738)(33.55912759,362.57516739)(33.49913239,362.58516726)
\curveto(33.4491277,362.59516737)(33.38912776,362.59016737)(33.31913239,362.57016726)
\curveto(33.29912785,362.5601674)(33.27412787,362.5601674)(33.24413239,362.57016726)
\curveto(33.21412793,362.57016739)(33.18912796,362.5651674)(33.16913239,362.55516726)
\lineto(33.01913239,362.55516726)
\curveto(32.9491282,362.54516742)(32.89912825,362.54516742)(32.86913239,362.55516726)
\curveto(32.82912832,362.5651674)(32.78412836,362.57016739)(32.73413239,362.57016726)
\curveto(32.69412845,362.5601674)(32.65412849,362.5601674)(32.61413239,362.57016726)
\curveto(32.52412862,362.59016737)(32.43412871,362.60516736)(32.34413239,362.61516726)
\curveto(32.25412889,362.61516735)(32.16412898,362.62516734)(32.07413239,362.64516726)
\curveto(31.98412916,362.67516729)(31.89412925,362.70016726)(31.80413239,362.72016726)
\curveto(31.71412943,362.74016722)(31.62912952,362.77016719)(31.54913239,362.81016726)
\curveto(31.30912984,362.92016704)(31.08413006,363.05016691)(30.87413239,363.20016726)
\curveto(30.66413048,363.3601666)(30.48413066,363.54016642)(30.33413239,363.74016726)
\curveto(30.21413093,363.91016605)(30.10913104,364.08516588)(30.01913239,364.26516726)
\curveto(29.92913122,364.44516552)(29.83913131,364.63516533)(29.74913239,364.83516726)
\curveto(29.70913144,364.93516503)(29.67413147,365.03516493)(29.64413239,365.13516726)
\curveto(29.62413152,365.24516472)(29.59913155,365.35516461)(29.56913239,365.46516726)
\curveto(29.52913162,365.60516436)(29.50413164,365.74516422)(29.49413239,365.88516726)
\curveto(29.48413166,366.02516394)(29.46413168,366.1651638)(29.43413239,366.30516726)
\curveto(29.42413172,366.41516355)(29.41413173,366.51516345)(29.40413239,366.60516726)
\curveto(29.40413174,366.70516326)(29.39413175,366.80516316)(29.37413239,366.90516726)
\lineto(29.37413239,366.99516726)
\curveto(29.38413176,367.02516294)(29.38413176,367.05016291)(29.37413239,367.07016726)
\lineto(29.37413239,367.28016726)
\curveto(29.35413179,367.34016262)(29.3441318,367.40516256)(29.34413239,367.47516726)
\curveto(29.35413179,367.55516241)(29.35913179,367.63016233)(29.35913239,367.70016726)
\lineto(29.35913239,367.85016726)
\curveto(29.35913179,367.90016206)(29.36413178,367.95016201)(29.37413239,368.00016726)
\lineto(29.37413239,368.37516726)
\curveto(29.38413176,368.40516156)(29.38413176,368.44016152)(29.37413239,368.48016726)
\curveto(29.37413177,368.52016144)(29.37913177,368.5601614)(29.38913239,368.60016726)
\curveto(29.40913174,368.71016125)(29.42413172,368.82016114)(29.43413239,368.93016726)
\curveto(29.4441317,369.05016091)(29.45413169,369.1651608)(29.46413239,369.27516726)
\curveto(29.50413164,369.42516054)(29.52913162,369.57016039)(29.53913239,369.71016726)
\curveto(29.55913159,369.8601601)(29.58913156,370.00515996)(29.62913239,370.14516726)
\curveto(29.71913143,370.44515952)(29.81413133,370.73015923)(29.91413239,371.00016726)
\curveto(30.01413113,371.27015869)(30.13913101,371.52015844)(30.28913239,371.75016726)
\curveto(30.48913066,372.07015789)(30.73413041,372.35015761)(31.02413239,372.59016726)
\curveto(31.31412983,372.83015713)(31.65412949,373.01515695)(32.04413239,373.14516726)
\curveto(32.15412899,373.18515678)(32.26412888,373.21015675)(32.37413239,373.22016726)
\curveto(32.49412865,373.24015672)(32.61412853,373.2651567)(32.73413239,373.29516726)
\curveto(32.80412834,373.30515666)(32.86912828,373.31015665)(32.92913239,373.31016726)
\curveto(32.98912816,373.31015665)(33.05412809,373.31515665)(33.12413239,373.32516726)
\curveto(33.82412732,373.34515662)(34.39912675,373.23015673)(34.84913239,372.98016726)
\curveto(35.29912585,372.73015723)(35.6441255,372.38015758)(35.88413239,371.93016726)
\curveto(35.99412515,371.70015826)(36.09412505,371.42515854)(36.18413239,371.10516726)
\curveto(36.20412494,371.03515893)(36.20412494,370.960159)(36.18413239,370.88016726)
\curveto(36.17412497,370.81015915)(36.149125,370.7601592)(36.10913239,370.73016726)
\curveto(36.07912507,370.70015926)(36.01912513,370.67515929)(35.92913239,370.65516726)
\curveto(35.83912531,370.64515932)(35.73912541,370.63515933)(35.62913239,370.62516726)
\curveto(35.52912562,370.62515934)(35.42912572,370.63015933)(35.32913239,370.64016726)
\curveto(35.23912591,370.65015931)(35.17412597,370.67015929)(35.13413239,370.70016726)
\curveto(35.02412612,370.77015919)(34.9441262,370.88015908)(34.89413239,371.03016726)
\curveto(34.85412629,371.18015878)(34.79912635,371.31015865)(34.72913239,371.42016726)
\curveto(34.53912661,371.73015823)(34.25912689,371.960158)(33.88913239,372.11016726)
\curveto(33.81912733,372.14015782)(33.7441274,372.1601578)(33.66413239,372.17016726)
\curveto(33.59412755,372.18015778)(33.51912763,372.19515777)(33.43913239,372.21516726)
\curveto(33.38912776,372.22515774)(33.31912783,372.23015773)(33.22913239,372.23016726)
\curveto(33.149128,372.23015773)(33.08412806,372.22515774)(33.03413239,372.21516726)
\curveto(32.99412815,372.19515777)(32.95912819,372.19015777)(32.92913239,372.20016726)
\curveto(32.89912825,372.21015775)(32.86412828,372.21015775)(32.82413239,372.20016726)
\lineto(32.58413239,372.14016726)
\curveto(32.51412863,372.12015784)(32.4441287,372.09515787)(32.37413239,372.06516726)
\curveto(31.99412915,371.90515806)(31.70412944,371.69515827)(31.50413239,371.43516726)
\curveto(31.31412983,371.17515879)(31.13913001,370.8601591)(30.97913239,370.49016726)
\curveto(30.9491302,370.41015955)(30.92413022,370.33015963)(30.90413239,370.25016726)
\curveto(30.89413025,370.17015979)(30.87413027,370.09015987)(30.84413239,370.01016726)
\curveto(30.81413033,369.90016006)(30.78913036,369.78516018)(30.76913239,369.66516726)
\curveto(30.75913039,369.54516042)(30.73913041,369.42516054)(30.70913239,369.30516726)
\curveto(30.68913046,369.25516071)(30.67913047,369.20516076)(30.67913239,369.15516726)
\curveto(30.68913046,369.10516086)(30.68413046,369.05516091)(30.66413239,369.00516726)
\curveto(30.65413049,368.94516102)(30.65413049,368.8651611)(30.66413239,368.76516726)
\curveto(30.67413047,368.67516129)(30.68913046,368.62016134)(30.70913239,368.60016726)
\curveto(30.72913042,368.5601614)(30.75913039,368.54016142)(30.79913239,368.54016726)
\curveto(30.8491303,368.54016142)(30.89413025,368.55016141)(30.93413239,368.57016726)
\curveto(31.00413014,368.61016135)(31.06413008,368.65516131)(31.11413239,368.70516726)
\curveto(31.16412998,368.75516121)(31.22412992,368.80516116)(31.29413239,368.85516726)
\lineto(31.35413239,368.91516726)
\curveto(31.38412976,368.94516102)(31.41412973,368.97016099)(31.44413239,368.99016726)
\curveto(31.67412947,369.15016081)(31.9491292,369.28516068)(32.26913239,369.39516726)
\curveto(32.33912881,369.41516055)(32.40912874,369.43016053)(32.47913239,369.44016726)
\curveto(32.5491286,369.45016051)(32.62412852,369.4651605)(32.70413239,369.48516726)
\curveto(32.7441284,369.48516048)(32.77912837,369.49016047)(32.80913239,369.50016726)
\curveto(32.83912831,369.51016045)(32.87412827,369.51016045)(32.91413239,369.50016726)
\curveto(32.96412818,369.50016046)(33.00412814,369.51016045)(33.03413239,369.53016726)
\lineto(33.19913239,369.53016726)
\lineto(33.28913239,369.53016726)
\curveto(33.33912781,369.54016042)(33.37912777,369.54016042)(33.40913239,369.53016726)
\curveto(33.45912769,369.52016044)(33.50912764,369.51516045)(33.55913239,369.51516726)
\curveto(33.61912753,369.52516044)(33.67412747,369.52516044)(33.72413239,369.51516726)
\curveto(33.83412731,369.48516048)(33.93912721,369.4651605)(34.03913239,369.45516726)
\curveto(34.149127,369.44516052)(34.25412689,369.42016054)(34.35413239,369.38016726)
\curveto(34.77412637,369.24016072)(35.11912603,369.05516091)(35.38913239,368.82516726)
\curveto(35.65912549,368.60516136)(35.89912525,368.32016164)(36.10913239,367.97016726)
\curveto(36.18912496,367.83016213)(36.25412489,367.68016228)(36.30413239,367.52016726)
\curveto(36.35412479,367.37016259)(36.40412474,367.21016275)(36.45413239,367.04016726)
\moveto(35.20913239,365.73516726)
\curveto(35.21912593,365.78516418)(35.22412592,365.83016413)(35.22413239,365.87016726)
\lineto(35.22413239,366.02016726)
\curveto(35.22412592,366.33016363)(35.18412596,366.61516335)(35.10413239,366.87516726)
\curveto(35.08412606,366.93516303)(35.06412608,366.99016297)(35.04413239,367.04016726)
\curveto(35.03412611,367.10016286)(35.01912613,367.15516281)(34.99913239,367.20516726)
\curveto(34.77912637,367.69516227)(34.43412671,368.04516192)(33.96413239,368.25516726)
\curveto(33.88412726,368.28516168)(33.80412734,368.31016165)(33.72413239,368.33016726)
\lineto(33.48413239,368.39016726)
\curveto(33.40412774,368.41016155)(33.31412783,368.42016154)(33.21413239,368.42016726)
\lineto(32.89913239,368.42016726)
\curveto(32.87912827,368.40016156)(32.83912831,368.39016157)(32.77913239,368.39016726)
\curveto(32.72912842,368.40016156)(32.68412846,368.40016156)(32.64413239,368.39016726)
\lineto(32.40413239,368.33016726)
\curveto(32.33412881,368.32016164)(32.26412888,368.30016166)(32.19413239,368.27016726)
\curveto(31.59412955,368.01016195)(31.18912996,367.54516242)(30.97913239,366.87516726)
\curveto(30.9491302,366.79516317)(30.92913022,366.71516325)(30.91913239,366.63516726)
\curveto(30.90913024,366.55516341)(30.89413025,366.47016349)(30.87413239,366.38016726)
\lineto(30.87413239,366.23016726)
\curveto(30.86413028,366.19016377)(30.85913029,366.12016384)(30.85913239,366.02016726)
\curveto(30.85913029,365.79016417)(30.87913027,365.59516437)(30.91913239,365.43516726)
\curveto(30.93913021,365.3651646)(30.95413019,365.30016466)(30.96413239,365.24016726)
\curveto(30.97413017,365.18016478)(30.99413015,365.11516485)(31.02413239,365.04516726)
\curveto(31.13413001,364.7651652)(31.27912987,364.52016544)(31.45913239,364.31016726)
\curveto(31.63912951,364.11016585)(31.87412927,363.95016601)(32.16413239,363.83016726)
\lineto(32.40413239,363.74016726)
\lineto(32.64413239,363.68016726)
\curveto(32.69412845,363.6601663)(32.73412841,363.65516631)(32.76413239,363.66516726)
\curveto(32.80412834,363.67516629)(32.8491283,363.67016629)(32.89913239,363.65016726)
\curveto(32.92912822,363.64016632)(32.98412816,363.63516633)(33.06413239,363.63516726)
\curveto(33.144128,363.63516633)(33.20412794,363.64016632)(33.24413239,363.65016726)
\curveto(33.35412779,363.67016629)(33.45912769,363.68516628)(33.55913239,363.69516726)
\curveto(33.65912749,363.70516626)(33.75412739,363.73516623)(33.84413239,363.78516726)
\curveto(34.37412677,363.98516598)(34.76412638,364.3601656)(35.01413239,364.91016726)
\curveto(35.05412609,365.01016495)(35.08412606,365.11516485)(35.10413239,365.22516726)
\lineto(35.19413239,365.55516726)
\curveto(35.19412595,365.63516433)(35.19912595,365.69516427)(35.20913239,365.73516726)
}
}
{
\newrgbcolor{curcolor}{0 0 0}
\pscustom[linestyle=none,fillstyle=solid,fillcolor=curcolor]
{
\newpath
\moveto(44.80374176,367.80516726)
\lineto(44.80374176,367.55016726)
\curveto(44.81373406,367.47016249)(44.80873406,367.39516257)(44.78874176,367.32516726)
\lineto(44.78874176,367.08516726)
\lineto(44.78874176,366.92016726)
\curveto(44.7687341,366.82016314)(44.75873411,366.71516325)(44.75874176,366.60516726)
\curveto(44.75873411,366.50516346)(44.74873412,366.40516356)(44.72874176,366.30516726)
\lineto(44.72874176,366.15516726)
\curveto(44.69873417,366.01516395)(44.67873419,365.87516409)(44.66874176,365.73516726)
\curveto(44.65873421,365.60516436)(44.63373424,365.47516449)(44.59374176,365.34516726)
\curveto(44.5737343,365.2651647)(44.55373432,365.18016478)(44.53374176,365.09016726)
\lineto(44.47374176,364.85016726)
\lineto(44.35374176,364.55016726)
\curveto(44.32373455,364.4601655)(44.28873458,364.37016559)(44.24874176,364.28016726)
\curveto(44.14873472,364.0601659)(44.01373486,363.84516612)(43.84374176,363.63516726)
\curveto(43.68373519,363.42516654)(43.50873536,363.25516671)(43.31874176,363.12516726)
\curveto(43.2687356,363.08516688)(43.20873566,363.04516692)(43.13874176,363.00516726)
\curveto(43.07873579,362.97516699)(43.01873585,362.94016702)(42.95874176,362.90016726)
\curveto(42.87873599,362.85016711)(42.78373609,362.81016715)(42.67374176,362.78016726)
\curveto(42.56373631,362.75016721)(42.45873641,362.72016724)(42.35874176,362.69016726)
\curveto(42.24873662,362.65016731)(42.13873673,362.62516734)(42.02874176,362.61516726)
\curveto(41.91873695,362.60516736)(41.80373707,362.59016737)(41.68374176,362.57016726)
\curveto(41.64373723,362.5601674)(41.59873727,362.5601674)(41.54874176,362.57016726)
\curveto(41.50873736,362.57016739)(41.4687374,362.5651674)(41.42874176,362.55516726)
\curveto(41.38873748,362.54516742)(41.33373754,362.54016742)(41.26374176,362.54016726)
\curveto(41.19373768,362.54016742)(41.14373773,362.54516742)(41.11374176,362.55516726)
\curveto(41.06373781,362.57516739)(41.01873785,362.58016738)(40.97874176,362.57016726)
\curveto(40.93873793,362.5601674)(40.90373797,362.5601674)(40.87374176,362.57016726)
\lineto(40.78374176,362.57016726)
\curveto(40.72373815,362.59016737)(40.65873821,362.60516736)(40.58874176,362.61516726)
\curveto(40.52873834,362.61516735)(40.46373841,362.62016734)(40.39374176,362.63016726)
\curveto(40.22373865,362.68016728)(40.06373881,362.73016723)(39.91374176,362.78016726)
\curveto(39.76373911,362.83016713)(39.61873925,362.89516707)(39.47874176,362.97516726)
\curveto(39.42873944,363.01516695)(39.3737395,363.04516692)(39.31374176,363.06516726)
\curveto(39.26373961,363.09516687)(39.21373966,363.13016683)(39.16374176,363.17016726)
\curveto(38.92373995,363.35016661)(38.72374015,363.57016639)(38.56374176,363.83016726)
\curveto(38.40374047,364.09016587)(38.26374061,364.37516559)(38.14374176,364.68516726)
\curveto(38.08374079,364.82516514)(38.03874083,364.965165)(38.00874176,365.10516726)
\curveto(37.97874089,365.25516471)(37.94374093,365.41016455)(37.90374176,365.57016726)
\curveto(37.88374099,365.68016428)(37.868741,365.79016417)(37.85874176,365.90016726)
\curveto(37.84874102,366.01016395)(37.83374104,366.12016384)(37.81374176,366.23016726)
\curveto(37.80374107,366.27016369)(37.79874107,366.31016365)(37.79874176,366.35016726)
\curveto(37.80874106,366.39016357)(37.80874106,366.43016353)(37.79874176,366.47016726)
\curveto(37.78874108,366.52016344)(37.78374109,366.57016339)(37.78374176,366.62016726)
\lineto(37.78374176,366.78516726)
\curveto(37.76374111,366.83516313)(37.75874111,366.88516308)(37.76874176,366.93516726)
\curveto(37.77874109,366.99516297)(37.77874109,367.05016291)(37.76874176,367.10016726)
\curveto(37.75874111,367.14016282)(37.75874111,367.18516278)(37.76874176,367.23516726)
\curveto(37.77874109,367.28516268)(37.7737411,367.33516263)(37.75374176,367.38516726)
\curveto(37.73374114,367.45516251)(37.72874114,367.53016243)(37.73874176,367.61016726)
\curveto(37.74874112,367.70016226)(37.75374112,367.78516218)(37.75374176,367.86516726)
\curveto(37.75374112,367.95516201)(37.74874112,368.05516191)(37.73874176,368.16516726)
\curveto(37.72874114,368.28516168)(37.73374114,368.38516158)(37.75374176,368.46516726)
\lineto(37.75374176,368.75016726)
\lineto(37.79874176,369.38016726)
\curveto(37.80874106,369.48016048)(37.81874105,369.57516039)(37.82874176,369.66516726)
\lineto(37.85874176,369.96516726)
\curveto(37.87874099,370.01515995)(37.88374099,370.0651599)(37.87374176,370.11516726)
\curveto(37.873741,370.17515979)(37.88374099,370.23015973)(37.90374176,370.28016726)
\curveto(37.95374092,370.45015951)(37.99374088,370.61515935)(38.02374176,370.77516726)
\curveto(38.05374082,370.94515902)(38.10374077,371.10515886)(38.17374176,371.25516726)
\curveto(38.36374051,371.71515825)(38.58374029,372.09015787)(38.83374176,372.38016726)
\curveto(39.09373978,372.67015729)(39.45373942,372.91515705)(39.91374176,373.11516726)
\curveto(40.04373883,373.1651568)(40.1737387,373.20015676)(40.30374176,373.22016726)
\curveto(40.44373843,373.24015672)(40.58373829,373.2651567)(40.72374176,373.29516726)
\curveto(40.79373808,373.30515666)(40.85873801,373.31015665)(40.91874176,373.31016726)
\curveto(40.97873789,373.31015665)(41.04373783,373.31515665)(41.11374176,373.32516726)
\curveto(41.94373693,373.34515662)(42.61373626,373.19515677)(43.12374176,372.87516726)
\curveto(43.63373524,372.5651574)(44.01373486,372.12515784)(44.26374176,371.55516726)
\curveto(44.31373456,371.43515853)(44.35873451,371.31015865)(44.39874176,371.18016726)
\curveto(44.43873443,371.05015891)(44.48373439,370.91515905)(44.53374176,370.77516726)
\curveto(44.55373432,370.69515927)(44.5687343,370.61015935)(44.57874176,370.52016726)
\lineto(44.63874176,370.28016726)
\curveto(44.6687342,370.17015979)(44.68373419,370.0601599)(44.68374176,369.95016726)
\curveto(44.69373418,369.84016012)(44.70873416,369.73016023)(44.72874176,369.62016726)
\curveto(44.74873412,369.57016039)(44.75373412,369.52516044)(44.74374176,369.48516726)
\curveto(44.74373413,369.44516052)(44.74873412,369.40516056)(44.75874176,369.36516726)
\curveto(44.7687341,369.31516065)(44.7687341,369.2601607)(44.75874176,369.20016726)
\curveto(44.75873411,369.15016081)(44.76373411,369.10016086)(44.77374176,369.05016726)
\lineto(44.77374176,368.91516726)
\curveto(44.79373408,368.85516111)(44.79373408,368.78516118)(44.77374176,368.70516726)
\curveto(44.76373411,368.63516133)(44.7687341,368.57016139)(44.78874176,368.51016726)
\curveto(44.79873407,368.48016148)(44.80373407,368.44016152)(44.80374176,368.39016726)
\lineto(44.80374176,368.27016726)
\lineto(44.80374176,367.80516726)
\moveto(43.25874176,365.48016726)
\curveto(43.35873551,365.80016416)(43.41873545,366.1651638)(43.43874176,366.57516726)
\curveto(43.45873541,366.98516298)(43.4687354,367.39516257)(43.46874176,367.80516726)
\curveto(43.4687354,368.23516173)(43.45873541,368.65516131)(43.43874176,369.06516726)
\curveto(43.41873545,369.47516049)(43.3737355,369.8601601)(43.30374176,370.22016726)
\curveto(43.23373564,370.58015938)(43.12373575,370.90015906)(42.97374176,371.18016726)
\curveto(42.83373604,371.47015849)(42.63873623,371.70515826)(42.38874176,371.88516726)
\curveto(42.22873664,371.99515797)(42.04873682,372.07515789)(41.84874176,372.12516726)
\curveto(41.64873722,372.18515778)(41.40373747,372.21515775)(41.11374176,372.21516726)
\curveto(41.09373778,372.19515777)(41.05873781,372.18515778)(41.00874176,372.18516726)
\curveto(40.95873791,372.19515777)(40.91873795,372.19515777)(40.88874176,372.18516726)
\curveto(40.80873806,372.1651578)(40.73373814,372.14515782)(40.66374176,372.12516726)
\curveto(40.60373827,372.11515785)(40.53873833,372.09515787)(40.46874176,372.06516726)
\curveto(40.19873867,371.94515802)(39.97873889,371.77515819)(39.80874176,371.55516726)
\curveto(39.64873922,371.34515862)(39.51373936,371.10015886)(39.40374176,370.82016726)
\curveto(39.35373952,370.71015925)(39.31373956,370.59015937)(39.28374176,370.46016726)
\curveto(39.26373961,370.34015962)(39.23873963,370.21515975)(39.20874176,370.08516726)
\curveto(39.18873968,370.03515993)(39.17873969,369.98015998)(39.17874176,369.92016726)
\curveto(39.17873969,369.87016009)(39.1737397,369.82016014)(39.16374176,369.77016726)
\curveto(39.15373972,369.68016028)(39.14373973,369.58516038)(39.13374176,369.48516726)
\curveto(39.12373975,369.39516057)(39.11373976,369.30016066)(39.10374176,369.20016726)
\curveto(39.10373977,369.12016084)(39.09873977,369.03516093)(39.08874176,368.94516726)
\lineto(39.08874176,368.70516726)
\lineto(39.08874176,368.52516726)
\curveto(39.07873979,368.49516147)(39.0737398,368.4601615)(39.07374176,368.42016726)
\lineto(39.07374176,368.28516726)
\lineto(39.07374176,367.83516726)
\curveto(39.0737398,367.75516221)(39.0687398,367.67016229)(39.05874176,367.58016726)
\curveto(39.05873981,367.50016246)(39.0687398,367.42516254)(39.08874176,367.35516726)
\lineto(39.08874176,367.08516726)
\curveto(39.08873978,367.0651629)(39.08373979,367.03516293)(39.07374176,366.99516726)
\curveto(39.0737398,366.965163)(39.07873979,366.94016302)(39.08874176,366.92016726)
\curveto(39.09873977,366.82016314)(39.10373977,366.72016324)(39.10374176,366.62016726)
\curveto(39.11373976,366.53016343)(39.12373975,366.43016353)(39.13374176,366.32016726)
\curveto(39.16373971,366.20016376)(39.17873969,366.07516389)(39.17874176,365.94516726)
\curveto(39.18873968,365.82516414)(39.21373966,365.71016425)(39.25374176,365.60016726)
\curveto(39.33373954,365.30016466)(39.41873945,365.03516493)(39.50874176,364.80516726)
\curveto(39.60873926,364.57516539)(39.75373912,364.3601656)(39.94374176,364.16016726)
\curveto(40.15373872,363.960166)(40.41873845,363.81016615)(40.73874176,363.71016726)
\curveto(40.77873809,363.69016627)(40.81373806,363.68016628)(40.84374176,363.68016726)
\curveto(40.88373799,363.69016627)(40.92873794,363.68516628)(40.97874176,363.66516726)
\curveto(41.01873785,363.65516631)(41.08873778,363.64516632)(41.18874176,363.63516726)
\curveto(41.29873757,363.62516634)(41.38373749,363.63016633)(41.44374176,363.65016726)
\curveto(41.51373736,363.67016629)(41.58373729,363.68016628)(41.65374176,363.68016726)
\curveto(41.72373715,363.69016627)(41.78873708,363.70516626)(41.84874176,363.72516726)
\curveto(42.04873682,363.78516618)(42.22873664,363.87016609)(42.38874176,363.98016726)
\curveto(42.41873645,364.00016596)(42.44373643,364.02016594)(42.46374176,364.04016726)
\lineto(42.52374176,364.10016726)
\curveto(42.56373631,364.12016584)(42.61373626,364.1601658)(42.67374176,364.22016726)
\curveto(42.7737361,364.3601656)(42.85873601,364.49016547)(42.92874176,364.61016726)
\curveto(42.99873587,364.73016523)(43.0687358,364.87516509)(43.13874176,365.04516726)
\curveto(43.1687357,365.11516485)(43.18873568,365.18516478)(43.19874176,365.25516726)
\curveto(43.21873565,365.32516464)(43.23873563,365.40016456)(43.25874176,365.48016726)
}
}
{
\newrgbcolor{curcolor}{0 0 0}
\pscustom[linestyle=none,fillstyle=solid,fillcolor=curcolor]
{
\newpath
\moveto(36.33413239,286.00807497)
\curveto(36.40412474,285.95807151)(36.4441247,285.88807158)(36.45413239,285.79807497)
\curveto(36.47412467,285.70807176)(36.48412466,285.60307187)(36.48413239,285.48307497)
\curveto(36.48412466,285.43307204)(36.47912467,285.38307209)(36.46913239,285.33307497)
\curveto(36.46912468,285.28307219)(36.45912469,285.23807223)(36.43913239,285.19807497)
\curveto(36.40912474,285.10807236)(36.3491248,285.04807242)(36.25913239,285.01807497)
\curveto(36.17912497,284.99807247)(36.08412506,284.98807248)(35.97413239,284.98807497)
\lineto(35.65913239,284.98807497)
\curveto(35.5491256,284.99807247)(35.4441257,284.98807248)(35.34413239,284.95807497)
\curveto(35.20412594,284.92807254)(35.11412603,284.84807262)(35.07413239,284.71807497)
\curveto(35.05412609,284.64807282)(35.0441261,284.56307291)(35.04413239,284.46307497)
\lineto(35.04413239,284.19307497)
\lineto(35.04413239,283.24807497)
\lineto(35.04413239,282.91807497)
\curveto(35.0441261,282.80807466)(35.02412612,282.72307475)(34.98413239,282.66307497)
\curveto(34.9441262,282.60307487)(34.89412625,282.56307491)(34.83413239,282.54307497)
\curveto(34.78412636,282.53307494)(34.71912643,282.51807495)(34.63913239,282.49807497)
\lineto(34.44413239,282.49807497)
\curveto(34.32412682,282.49807497)(34.21912693,282.50307497)(34.12913239,282.51307497)
\curveto(34.03912711,282.53307494)(33.96912718,282.58307489)(33.91913239,282.66307497)
\curveto(33.88912726,282.71307476)(33.87412727,282.78307469)(33.87413239,282.87307497)
\lineto(33.87413239,283.17307497)
\lineto(33.87413239,284.20807497)
\curveto(33.87412727,284.3680731)(33.86412728,284.51307296)(33.84413239,284.64307497)
\curveto(33.83412731,284.78307269)(33.77912737,284.87807259)(33.67913239,284.92807497)
\curveto(33.62912752,284.94807252)(33.55912759,284.96307251)(33.46913239,284.97307497)
\curveto(33.38912776,284.98307249)(33.29912785,284.98807248)(33.19913239,284.98807497)
\lineto(32.91413239,284.98807497)
\lineto(32.67413239,284.98807497)
\lineto(30.40913239,284.98807497)
\curveto(30.31913083,284.98807248)(30.21413093,284.98307249)(30.09413239,284.97307497)
\lineto(29.76413239,284.97307497)
\curveto(29.65413149,284.9730725)(29.55413159,284.98307249)(29.46413239,285.00307497)
\curveto(29.37413177,285.02307245)(29.31413183,285.05807241)(29.28413239,285.10807497)
\curveto(29.23413191,285.17807229)(29.20913194,285.2730722)(29.20913239,285.39307497)
\lineto(29.20913239,285.73807497)
\lineto(29.20913239,286.00807497)
\curveto(29.2491319,286.17807129)(29.30413184,286.31807115)(29.37413239,286.42807497)
\curveto(29.4441317,286.53807093)(29.52413162,286.65307082)(29.61413239,286.77307497)
\lineto(29.97413239,287.31307497)
\curveto(30.41413073,287.94306953)(30.8491303,288.56306891)(31.27913239,289.17307497)
\lineto(32.59913239,291.03307497)
\curveto(32.75912839,291.26306621)(32.91412823,291.48306599)(33.06413239,291.69307497)
\curveto(33.21412793,291.91306556)(33.36912778,292.13806533)(33.52913239,292.36807497)
\curveto(33.57912757,292.43806503)(33.62912752,292.50306497)(33.67913239,292.56307497)
\curveto(33.72912742,292.63306484)(33.77912737,292.70806476)(33.82913239,292.78807497)
\lineto(33.88913239,292.87807497)
\curveto(33.91912723,292.91806455)(33.9491272,292.94806452)(33.97913239,292.96807497)
\curveto(34.01912713,292.99806447)(34.05912709,293.01806445)(34.09913239,293.02807497)
\curveto(34.13912701,293.04806442)(34.18412696,293.0680644)(34.23413239,293.08807497)
\curveto(34.25412689,293.08806438)(34.27412687,293.08306439)(34.29413239,293.07307497)
\curveto(34.32412682,293.0730644)(34.3491268,293.08306439)(34.36913239,293.10307497)
\curveto(34.49912665,293.10306437)(34.61912653,293.09806437)(34.72913239,293.08807497)
\curveto(34.83912631,293.07806439)(34.91912623,293.03306444)(34.96913239,292.95307497)
\curveto(35.00912614,292.90306457)(35.02912612,292.83306464)(35.02913239,292.74307497)
\curveto(35.03912611,292.65306482)(35.0441261,292.55806491)(35.04413239,292.45807497)
\lineto(35.04413239,286.99807497)
\curveto(35.0441261,286.92807054)(35.03912611,286.85307062)(35.02913239,286.77307497)
\curveto(35.02912612,286.70307077)(35.03412611,286.63307084)(35.04413239,286.56307497)
\lineto(35.04413239,286.45807497)
\curveto(35.06412608,286.40807106)(35.07912607,286.35307112)(35.08913239,286.29307497)
\curveto(35.09912605,286.24307123)(35.12412602,286.20307127)(35.16413239,286.17307497)
\curveto(35.23412591,286.12307135)(35.31912583,286.09307138)(35.41913239,286.08307497)
\lineto(35.74913239,286.08307497)
\curveto(35.85912529,286.08307139)(35.96412518,286.07807139)(36.06413239,286.06807497)
\curveto(36.17412497,286.0680714)(36.26412488,286.04807142)(36.33413239,286.00807497)
\moveto(33.76913239,286.20307497)
\curveto(33.8491273,286.31307116)(33.88412726,286.48307099)(33.87413239,286.71307497)
\lineto(33.87413239,287.32807497)
\lineto(33.87413239,289.80307497)
\lineto(33.87413239,290.11807497)
\curveto(33.88412726,290.23806723)(33.87912727,290.33806713)(33.85913239,290.41807497)
\lineto(33.85913239,290.56807497)
\curveto(33.85912729,290.65806681)(33.8441273,290.74306673)(33.81413239,290.82307497)
\curveto(33.80412734,290.84306663)(33.79412735,290.85306662)(33.78413239,290.85307497)
\lineto(33.73913239,290.89807497)
\curveto(33.71912743,290.90806656)(33.68912746,290.91306656)(33.64913239,290.91307497)
\curveto(33.62912752,290.89306658)(33.60912754,290.87806659)(33.58913239,290.86807497)
\curveto(33.57912757,290.8680666)(33.56412758,290.86306661)(33.54413239,290.85307497)
\curveto(33.48412766,290.80306667)(33.42412772,290.73306674)(33.36413239,290.64307497)
\curveto(33.30412784,290.55306692)(33.2491279,290.473067)(33.19913239,290.40307497)
\curveto(33.09912805,290.26306721)(33.00412814,290.11806735)(32.91413239,289.96807497)
\curveto(32.82412832,289.82806764)(32.72912842,289.68806778)(32.62913239,289.54807497)
\lineto(32.08913239,288.76807497)
\curveto(31.91912923,288.50806896)(31.7441294,288.24806922)(31.56413239,287.98807497)
\curveto(31.48412966,287.87806959)(31.40912974,287.7730697)(31.33913239,287.67307497)
\lineto(31.12913239,287.37307497)
\curveto(31.07913007,287.29307018)(31.02913012,287.21807025)(30.97913239,287.14807497)
\curveto(30.93913021,287.07807039)(30.89413025,287.00307047)(30.84413239,286.92307497)
\curveto(30.79413035,286.86307061)(30.7441304,286.79807067)(30.69413239,286.72807497)
\curveto(30.65413049,286.6680708)(30.61413053,286.59807087)(30.57413239,286.51807497)
\curveto(30.53413061,286.45807101)(30.50913064,286.38807108)(30.49913239,286.30807497)
\curveto(30.48913066,286.23807123)(30.52413062,286.18307129)(30.60413239,286.14307497)
\curveto(30.67413047,286.09307138)(30.78413036,286.0680714)(30.93413239,286.06807497)
\curveto(31.09413005,286.07807139)(31.22912992,286.08307139)(31.33913239,286.08307497)
\lineto(33.01913239,286.08307497)
\lineto(33.45413239,286.08307497)
\curveto(33.60412754,286.08307139)(33.70912744,286.12307135)(33.76913239,286.20307497)
}
}
{
\newrgbcolor{curcolor}{0 0 0}
\pscustom[linestyle=none,fillstyle=solid,fillcolor=curcolor]
{
\newpath
\moveto(44.80374176,287.59807497)
\lineto(44.80374176,287.34307497)
\curveto(44.81373406,287.26307021)(44.80873406,287.18807028)(44.78874176,287.11807497)
\lineto(44.78874176,286.87807497)
\lineto(44.78874176,286.71307497)
\curveto(44.7687341,286.61307086)(44.75873411,286.50807096)(44.75874176,286.39807497)
\curveto(44.75873411,286.29807117)(44.74873412,286.19807127)(44.72874176,286.09807497)
\lineto(44.72874176,285.94807497)
\curveto(44.69873417,285.80807166)(44.67873419,285.6680718)(44.66874176,285.52807497)
\curveto(44.65873421,285.39807207)(44.63373424,285.2680722)(44.59374176,285.13807497)
\curveto(44.5737343,285.05807241)(44.55373432,284.9730725)(44.53374176,284.88307497)
\lineto(44.47374176,284.64307497)
\lineto(44.35374176,284.34307497)
\curveto(44.32373455,284.25307322)(44.28873458,284.16307331)(44.24874176,284.07307497)
\curveto(44.14873472,283.85307362)(44.01373486,283.63807383)(43.84374176,283.42807497)
\curveto(43.68373519,283.21807425)(43.50873536,283.04807442)(43.31874176,282.91807497)
\curveto(43.2687356,282.87807459)(43.20873566,282.83807463)(43.13874176,282.79807497)
\curveto(43.07873579,282.7680747)(43.01873585,282.73307474)(42.95874176,282.69307497)
\curveto(42.87873599,282.64307483)(42.78373609,282.60307487)(42.67374176,282.57307497)
\curveto(42.56373631,282.54307493)(42.45873641,282.51307496)(42.35874176,282.48307497)
\curveto(42.24873662,282.44307503)(42.13873673,282.41807505)(42.02874176,282.40807497)
\curveto(41.91873695,282.39807507)(41.80373707,282.38307509)(41.68374176,282.36307497)
\curveto(41.64373723,282.35307512)(41.59873727,282.35307512)(41.54874176,282.36307497)
\curveto(41.50873736,282.36307511)(41.4687374,282.35807511)(41.42874176,282.34807497)
\curveto(41.38873748,282.33807513)(41.33373754,282.33307514)(41.26374176,282.33307497)
\curveto(41.19373768,282.33307514)(41.14373773,282.33807513)(41.11374176,282.34807497)
\curveto(41.06373781,282.3680751)(41.01873785,282.3730751)(40.97874176,282.36307497)
\curveto(40.93873793,282.35307512)(40.90373797,282.35307512)(40.87374176,282.36307497)
\lineto(40.78374176,282.36307497)
\curveto(40.72373815,282.38307509)(40.65873821,282.39807507)(40.58874176,282.40807497)
\curveto(40.52873834,282.40807506)(40.46373841,282.41307506)(40.39374176,282.42307497)
\curveto(40.22373865,282.473075)(40.06373881,282.52307495)(39.91374176,282.57307497)
\curveto(39.76373911,282.62307485)(39.61873925,282.68807478)(39.47874176,282.76807497)
\curveto(39.42873944,282.80807466)(39.3737395,282.83807463)(39.31374176,282.85807497)
\curveto(39.26373961,282.88807458)(39.21373966,282.92307455)(39.16374176,282.96307497)
\curveto(38.92373995,283.14307433)(38.72374015,283.36307411)(38.56374176,283.62307497)
\curveto(38.40374047,283.88307359)(38.26374061,284.1680733)(38.14374176,284.47807497)
\curveto(38.08374079,284.61807285)(38.03874083,284.75807271)(38.00874176,284.89807497)
\curveto(37.97874089,285.04807242)(37.94374093,285.20307227)(37.90374176,285.36307497)
\curveto(37.88374099,285.473072)(37.868741,285.58307189)(37.85874176,285.69307497)
\curveto(37.84874102,285.80307167)(37.83374104,285.91307156)(37.81374176,286.02307497)
\curveto(37.80374107,286.06307141)(37.79874107,286.10307137)(37.79874176,286.14307497)
\curveto(37.80874106,286.18307129)(37.80874106,286.22307125)(37.79874176,286.26307497)
\curveto(37.78874108,286.31307116)(37.78374109,286.36307111)(37.78374176,286.41307497)
\lineto(37.78374176,286.57807497)
\curveto(37.76374111,286.62807084)(37.75874111,286.67807079)(37.76874176,286.72807497)
\curveto(37.77874109,286.78807068)(37.77874109,286.84307063)(37.76874176,286.89307497)
\curveto(37.75874111,286.93307054)(37.75874111,286.97807049)(37.76874176,287.02807497)
\curveto(37.77874109,287.07807039)(37.7737411,287.12807034)(37.75374176,287.17807497)
\curveto(37.73374114,287.24807022)(37.72874114,287.32307015)(37.73874176,287.40307497)
\curveto(37.74874112,287.49306998)(37.75374112,287.57806989)(37.75374176,287.65807497)
\curveto(37.75374112,287.74806972)(37.74874112,287.84806962)(37.73874176,287.95807497)
\curveto(37.72874114,288.07806939)(37.73374114,288.17806929)(37.75374176,288.25807497)
\lineto(37.75374176,288.54307497)
\lineto(37.79874176,289.17307497)
\curveto(37.80874106,289.2730682)(37.81874105,289.3680681)(37.82874176,289.45807497)
\lineto(37.85874176,289.75807497)
\curveto(37.87874099,289.80806766)(37.88374099,289.85806761)(37.87374176,289.90807497)
\curveto(37.873741,289.9680675)(37.88374099,290.02306745)(37.90374176,290.07307497)
\curveto(37.95374092,290.24306723)(37.99374088,290.40806706)(38.02374176,290.56807497)
\curveto(38.05374082,290.73806673)(38.10374077,290.89806657)(38.17374176,291.04807497)
\curveto(38.36374051,291.50806596)(38.58374029,291.88306559)(38.83374176,292.17307497)
\curveto(39.09373978,292.46306501)(39.45373942,292.70806476)(39.91374176,292.90807497)
\curveto(40.04373883,292.95806451)(40.1737387,292.99306448)(40.30374176,293.01307497)
\curveto(40.44373843,293.03306444)(40.58373829,293.05806441)(40.72374176,293.08807497)
\curveto(40.79373808,293.09806437)(40.85873801,293.10306437)(40.91874176,293.10307497)
\curveto(40.97873789,293.10306437)(41.04373783,293.10806436)(41.11374176,293.11807497)
\curveto(41.94373693,293.13806433)(42.61373626,292.98806448)(43.12374176,292.66807497)
\curveto(43.63373524,292.35806511)(44.01373486,291.91806555)(44.26374176,291.34807497)
\curveto(44.31373456,291.22806624)(44.35873451,291.10306637)(44.39874176,290.97307497)
\curveto(44.43873443,290.84306663)(44.48373439,290.70806676)(44.53374176,290.56807497)
\curveto(44.55373432,290.48806698)(44.5687343,290.40306707)(44.57874176,290.31307497)
\lineto(44.63874176,290.07307497)
\curveto(44.6687342,289.96306751)(44.68373419,289.85306762)(44.68374176,289.74307497)
\curveto(44.69373418,289.63306784)(44.70873416,289.52306795)(44.72874176,289.41307497)
\curveto(44.74873412,289.36306811)(44.75373412,289.31806815)(44.74374176,289.27807497)
\curveto(44.74373413,289.23806823)(44.74873412,289.19806827)(44.75874176,289.15807497)
\curveto(44.7687341,289.10806836)(44.7687341,289.05306842)(44.75874176,288.99307497)
\curveto(44.75873411,288.94306853)(44.76373411,288.89306858)(44.77374176,288.84307497)
\lineto(44.77374176,288.70807497)
\curveto(44.79373408,288.64806882)(44.79373408,288.57806889)(44.77374176,288.49807497)
\curveto(44.76373411,288.42806904)(44.7687341,288.36306911)(44.78874176,288.30307497)
\curveto(44.79873407,288.2730692)(44.80373407,288.23306924)(44.80374176,288.18307497)
\lineto(44.80374176,288.06307497)
\lineto(44.80374176,287.59807497)
\moveto(43.25874176,285.27307497)
\curveto(43.35873551,285.59307188)(43.41873545,285.95807151)(43.43874176,286.36807497)
\curveto(43.45873541,286.77807069)(43.4687354,287.18807028)(43.46874176,287.59807497)
\curveto(43.4687354,288.02806944)(43.45873541,288.44806902)(43.43874176,288.85807497)
\curveto(43.41873545,289.2680682)(43.3737355,289.65306782)(43.30374176,290.01307497)
\curveto(43.23373564,290.3730671)(43.12373575,290.69306678)(42.97374176,290.97307497)
\curveto(42.83373604,291.26306621)(42.63873623,291.49806597)(42.38874176,291.67807497)
\curveto(42.22873664,291.78806568)(42.04873682,291.8680656)(41.84874176,291.91807497)
\curveto(41.64873722,291.97806549)(41.40373747,292.00806546)(41.11374176,292.00807497)
\curveto(41.09373778,291.98806548)(41.05873781,291.97806549)(41.00874176,291.97807497)
\curveto(40.95873791,291.98806548)(40.91873795,291.98806548)(40.88874176,291.97807497)
\curveto(40.80873806,291.95806551)(40.73373814,291.93806553)(40.66374176,291.91807497)
\curveto(40.60373827,291.90806556)(40.53873833,291.88806558)(40.46874176,291.85807497)
\curveto(40.19873867,291.73806573)(39.97873889,291.5680659)(39.80874176,291.34807497)
\curveto(39.64873922,291.13806633)(39.51373936,290.89306658)(39.40374176,290.61307497)
\curveto(39.35373952,290.50306697)(39.31373956,290.38306709)(39.28374176,290.25307497)
\curveto(39.26373961,290.13306734)(39.23873963,290.00806746)(39.20874176,289.87807497)
\curveto(39.18873968,289.82806764)(39.17873969,289.7730677)(39.17874176,289.71307497)
\curveto(39.17873969,289.66306781)(39.1737397,289.61306786)(39.16374176,289.56307497)
\curveto(39.15373972,289.473068)(39.14373973,289.37806809)(39.13374176,289.27807497)
\curveto(39.12373975,289.18806828)(39.11373976,289.09306838)(39.10374176,288.99307497)
\curveto(39.10373977,288.91306856)(39.09873977,288.82806864)(39.08874176,288.73807497)
\lineto(39.08874176,288.49807497)
\lineto(39.08874176,288.31807497)
\curveto(39.07873979,288.28806918)(39.0737398,288.25306922)(39.07374176,288.21307497)
\lineto(39.07374176,288.07807497)
\lineto(39.07374176,287.62807497)
\curveto(39.0737398,287.54806992)(39.0687398,287.46307001)(39.05874176,287.37307497)
\curveto(39.05873981,287.29307018)(39.0687398,287.21807025)(39.08874176,287.14807497)
\lineto(39.08874176,286.87807497)
\curveto(39.08873978,286.85807061)(39.08373979,286.82807064)(39.07374176,286.78807497)
\curveto(39.0737398,286.75807071)(39.07873979,286.73307074)(39.08874176,286.71307497)
\curveto(39.09873977,286.61307086)(39.10373977,286.51307096)(39.10374176,286.41307497)
\curveto(39.11373976,286.32307115)(39.12373975,286.22307125)(39.13374176,286.11307497)
\curveto(39.16373971,285.99307148)(39.17873969,285.8680716)(39.17874176,285.73807497)
\curveto(39.18873968,285.61807185)(39.21373966,285.50307197)(39.25374176,285.39307497)
\curveto(39.33373954,285.09307238)(39.41873945,284.82807264)(39.50874176,284.59807497)
\curveto(39.60873926,284.3680731)(39.75373912,284.15307332)(39.94374176,283.95307497)
\curveto(40.15373872,283.75307372)(40.41873845,283.60307387)(40.73874176,283.50307497)
\curveto(40.77873809,283.48307399)(40.81373806,283.473074)(40.84374176,283.47307497)
\curveto(40.88373799,283.48307399)(40.92873794,283.47807399)(40.97874176,283.45807497)
\curveto(41.01873785,283.44807402)(41.08873778,283.43807403)(41.18874176,283.42807497)
\curveto(41.29873757,283.41807405)(41.38373749,283.42307405)(41.44374176,283.44307497)
\curveto(41.51373736,283.46307401)(41.58373729,283.473074)(41.65374176,283.47307497)
\curveto(41.72373715,283.48307399)(41.78873708,283.49807397)(41.84874176,283.51807497)
\curveto(42.04873682,283.57807389)(42.22873664,283.66307381)(42.38874176,283.77307497)
\curveto(42.41873645,283.79307368)(42.44373643,283.81307366)(42.46374176,283.83307497)
\lineto(42.52374176,283.89307497)
\curveto(42.56373631,283.91307356)(42.61373626,283.95307352)(42.67374176,284.01307497)
\curveto(42.7737361,284.15307332)(42.85873601,284.28307319)(42.92874176,284.40307497)
\curveto(42.99873587,284.52307295)(43.0687358,284.6680728)(43.13874176,284.83807497)
\curveto(43.1687357,284.90807256)(43.18873568,284.97807249)(43.19874176,285.04807497)
\curveto(43.21873565,285.11807235)(43.23873563,285.19307228)(43.25874176,285.27307497)
}
}
{
\newrgbcolor{curcolor}{0 0 0}
\pscustom[linestyle=none,fillstyle=solid,fillcolor=curcolor]
{
\newpath
\moveto(32.70413239,212.95593508)
\curveto(33.39412775,212.96592445)(33.99412715,212.84592457)(34.50413239,212.59593508)
\curveto(35.02412612,212.34592507)(35.41912573,212.0109254)(35.68913239,211.59093508)
\curveto(35.73912541,211.5109259)(35.78412536,211.42092599)(35.82413239,211.32093508)
\curveto(35.86412528,211.23092618)(35.90912524,211.13592628)(35.95913239,211.03593508)
\curveto(35.99912515,210.93592648)(36.02912512,210.83592658)(36.04913239,210.73593508)
\curveto(36.06912508,210.63592678)(36.08912506,210.53092688)(36.10913239,210.42093508)
\curveto(36.12912502,210.37092704)(36.13412501,210.32592709)(36.12413239,210.28593508)
\curveto(36.11412503,210.24592717)(36.11912503,210.20092721)(36.13913239,210.15093508)
\curveto(36.149125,210.10092731)(36.15412499,210.0159274)(36.15413239,209.89593508)
\curveto(36.15412499,209.78592763)(36.149125,209.70092771)(36.13913239,209.64093508)
\curveto(36.11912503,209.58092783)(36.10912504,209.52092789)(36.10913239,209.46093508)
\curveto(36.11912503,209.40092801)(36.11412503,209.34092807)(36.09413239,209.28093508)
\curveto(36.05412509,209.14092827)(36.01912513,209.00592841)(35.98913239,208.87593508)
\curveto(35.95912519,208.74592867)(35.91912523,208.62092879)(35.86913239,208.50093508)
\curveto(35.80912534,208.36092905)(35.73912541,208.23592918)(35.65913239,208.12593508)
\curveto(35.58912556,208.0159294)(35.51412563,207.90592951)(35.43413239,207.79593508)
\lineto(35.37413239,207.73593508)
\curveto(35.36412578,207.7159297)(35.3491258,207.69592972)(35.32913239,207.67593508)
\curveto(35.20912594,207.5159299)(35.07412607,207.37093004)(34.92413239,207.24093508)
\curveto(34.77412637,207.1109303)(34.61412653,206.98593043)(34.44413239,206.86593508)
\curveto(34.13412701,206.64593077)(33.83912731,206.44093097)(33.55913239,206.25093508)
\curveto(33.32912782,206.1109313)(33.09912805,205.97593144)(32.86913239,205.84593508)
\curveto(32.6491285,205.7159317)(32.42912872,205.58093183)(32.20913239,205.44093508)
\curveto(31.95912919,205.27093214)(31.71912943,205.09093232)(31.48913239,204.90093508)
\curveto(31.26912988,204.7109327)(31.07913007,204.48593293)(30.91913239,204.22593508)
\curveto(30.87913027,204.16593325)(30.8441303,204.10593331)(30.81413239,204.04593508)
\curveto(30.78413036,203.99593342)(30.75413039,203.93093348)(30.72413239,203.85093508)
\curveto(30.70413044,203.78093363)(30.69913045,203.72093369)(30.70913239,203.67093508)
\curveto(30.72913042,203.60093381)(30.76413038,203.54593387)(30.81413239,203.50593508)
\curveto(30.86413028,203.47593394)(30.92413022,203.45593396)(30.99413239,203.44593508)
\lineto(31.23413239,203.44593508)
\lineto(31.98413239,203.44593508)
\lineto(34.78913239,203.44593508)
\lineto(35.44913239,203.44593508)
\curveto(35.53912561,203.44593397)(35.62412552,203.44093397)(35.70413239,203.43093508)
\curveto(35.78412536,203.43093398)(35.8491253,203.410934)(35.89913239,203.37093508)
\curveto(35.9491252,203.33093408)(35.98912516,203.25593416)(36.01913239,203.14593508)
\curveto(36.05912509,203.04593437)(36.06912508,202.94593447)(36.04913239,202.84593508)
\lineto(36.04913239,202.71093508)
\curveto(36.02912512,202.64093477)(36.00912514,202.58093483)(35.98913239,202.53093508)
\curveto(35.96912518,202.48093493)(35.93412521,202.44093497)(35.88413239,202.41093508)
\curveto(35.83412531,202.37093504)(35.76412538,202.35093506)(35.67413239,202.35093508)
\lineto(35.40413239,202.35093508)
\lineto(34.50413239,202.35093508)
\lineto(30.99413239,202.35093508)
\lineto(29.92913239,202.35093508)
\curveto(29.8491313,202.35093506)(29.75913139,202.34593507)(29.65913239,202.33593508)
\curveto(29.55913159,202.33593508)(29.47413167,202.34593507)(29.40413239,202.36593508)
\curveto(29.19413195,202.43593498)(29.12913202,202.6159348)(29.20913239,202.90593508)
\curveto(29.21913193,202.94593447)(29.21913193,202.98093443)(29.20913239,203.01093508)
\curveto(29.20913194,203.05093436)(29.21913193,203.09593432)(29.23913239,203.14593508)
\curveto(29.25913189,203.22593419)(29.27913187,203.3109341)(29.29913239,203.40093508)
\curveto(29.31913183,203.49093392)(29.3441318,203.57593384)(29.37413239,203.65593508)
\curveto(29.53413161,204.14593327)(29.73413141,204.56093285)(29.97413239,204.90093508)
\curveto(30.15413099,205.15093226)(30.35913079,205.37593204)(30.58913239,205.57593508)
\curveto(30.81913033,205.78593163)(31.05913009,205.98093143)(31.30913239,206.16093508)
\curveto(31.56912958,206.34093107)(31.83412931,206.5109309)(32.10413239,206.67093508)
\curveto(32.38412876,206.84093057)(32.65412849,207.0159304)(32.91413239,207.19593508)
\curveto(33.02412812,207.27593014)(33.12912802,207.35093006)(33.22913239,207.42093508)
\curveto(33.33912781,207.49092992)(33.4491277,207.56592985)(33.55913239,207.64593508)
\curveto(33.59912755,207.67592974)(33.63412751,207.70592971)(33.66413239,207.73593508)
\curveto(33.70412744,207.77592964)(33.7441274,207.80592961)(33.78413239,207.82593508)
\curveto(33.92412722,207.93592948)(34.0491271,208.06092935)(34.15913239,208.20093508)
\curveto(34.17912697,208.23092918)(34.20412694,208.25592916)(34.23413239,208.27593508)
\curveto(34.26412688,208.30592911)(34.28912686,208.33592908)(34.30913239,208.36593508)
\curveto(34.38912676,208.46592895)(34.45412669,208.56592885)(34.50413239,208.66593508)
\curveto(34.56412658,208.76592865)(34.61912653,208.87592854)(34.66913239,208.99593508)
\curveto(34.69912645,209.06592835)(34.71912643,209.14092827)(34.72913239,209.22093508)
\lineto(34.78913239,209.46093508)
\lineto(34.78913239,209.55093508)
\curveto(34.79912635,209.58092783)(34.80412634,209.6109278)(34.80413239,209.64093508)
\curveto(34.82412632,209.7109277)(34.82912632,209.80592761)(34.81913239,209.92593508)
\curveto(34.81912633,210.05592736)(34.80912634,210.15592726)(34.78913239,210.22593508)
\curveto(34.76912638,210.30592711)(34.7491264,210.38092703)(34.72913239,210.45093508)
\curveto(34.71912643,210.53092688)(34.69912645,210.6109268)(34.66913239,210.69093508)
\curveto(34.55912659,210.93092648)(34.40912674,211.13092628)(34.21913239,211.29093508)
\curveto(34.03912711,211.46092595)(33.81912733,211.60092581)(33.55913239,211.71093508)
\curveto(33.48912766,211.73092568)(33.41912773,211.74592567)(33.34913239,211.75593508)
\curveto(33.27912787,211.77592564)(33.20412794,211.79592562)(33.12413239,211.81593508)
\curveto(33.0441281,211.83592558)(32.93412821,211.84592557)(32.79413239,211.84593508)
\curveto(32.66412848,211.84592557)(32.55912859,211.83592558)(32.47913239,211.81593508)
\curveto(32.41912873,211.80592561)(32.36412878,211.80092561)(32.31413239,211.80093508)
\curveto(32.26412888,211.80092561)(32.21412893,211.79092562)(32.16413239,211.77093508)
\curveto(32.06412908,211.73092568)(31.96912918,211.69092572)(31.87913239,211.65093508)
\curveto(31.79912935,211.6109258)(31.71912943,211.56592585)(31.63913239,211.51593508)
\curveto(31.60912954,211.49592592)(31.57912957,211.47092594)(31.54913239,211.44093508)
\curveto(31.52912962,211.410926)(31.50412964,211.38592603)(31.47413239,211.36593508)
\lineto(31.39913239,211.29093508)
\curveto(31.36912978,211.27092614)(31.3441298,211.25092616)(31.32413239,211.23093508)
\lineto(31.17413239,211.02093508)
\curveto(31.13413001,210.96092645)(31.08913006,210.89592652)(31.03913239,210.82593508)
\curveto(30.97913017,210.73592668)(30.92913022,210.63092678)(30.88913239,210.51093508)
\curveto(30.85913029,210.40092701)(30.82413032,210.29092712)(30.78413239,210.18093508)
\curveto(30.7441304,210.07092734)(30.71913043,209.92592749)(30.70913239,209.74593508)
\curveto(30.69913045,209.57592784)(30.66913048,209.45092796)(30.61913239,209.37093508)
\curveto(30.56913058,209.29092812)(30.49413065,209.24592817)(30.39413239,209.23593508)
\curveto(30.29413085,209.22592819)(30.18413096,209.22092819)(30.06413239,209.22093508)
\curveto(30.02413112,209.22092819)(29.98413116,209.2159282)(29.94413239,209.20593508)
\curveto(29.90413124,209.20592821)(29.86913128,209.2109282)(29.83913239,209.22093508)
\curveto(29.78913136,209.24092817)(29.73913141,209.25092816)(29.68913239,209.25093508)
\curveto(29.6491315,209.25092816)(29.60913154,209.26092815)(29.56913239,209.28093508)
\curveto(29.47913167,209.34092807)(29.43413171,209.47592794)(29.43413239,209.68593508)
\lineto(29.43413239,209.80593508)
\curveto(29.4441317,209.86592755)(29.4491317,209.92592749)(29.44913239,209.98593508)
\curveto(29.45913169,210.05592736)(29.46913168,210.12092729)(29.47913239,210.18093508)
\curveto(29.49913165,210.29092712)(29.51913163,210.39092702)(29.53913239,210.48093508)
\curveto(29.55913159,210.58092683)(29.58913156,210.67592674)(29.62913239,210.76593508)
\curveto(29.6491315,210.83592658)(29.66913148,210.89592652)(29.68913239,210.94593508)
\lineto(29.74913239,211.12593508)
\curveto(29.86913128,211.38592603)(30.02413112,211.63092578)(30.21413239,211.86093508)
\curveto(30.41413073,212.09092532)(30.62913052,212.27592514)(30.85913239,212.41593508)
\curveto(30.96913018,212.49592492)(31.08413006,212.56092485)(31.20413239,212.61093508)
\lineto(31.59413239,212.76093508)
\curveto(31.70412944,212.8109246)(31.81912933,212.84092457)(31.93913239,212.85093508)
\curveto(32.05912909,212.87092454)(32.18412896,212.89592452)(32.31413239,212.92593508)
\curveto(32.38412876,212.92592449)(32.4491287,212.92592449)(32.50913239,212.92593508)
\curveto(32.56912858,212.93592448)(32.63412851,212.94592447)(32.70413239,212.95593508)
}
}
{
\newrgbcolor{curcolor}{0 0 0}
\pscustom[linestyle=none,fillstyle=solid,fillcolor=curcolor]
{
\newpath
\moveto(44.80374176,207.43593508)
\lineto(44.80374176,207.18093508)
\curveto(44.81373406,207.10093031)(44.80873406,207.02593039)(44.78874176,206.95593508)
\lineto(44.78874176,206.71593508)
\lineto(44.78874176,206.55093508)
\curveto(44.7687341,206.45093096)(44.75873411,206.34593107)(44.75874176,206.23593508)
\curveto(44.75873411,206.13593128)(44.74873412,206.03593138)(44.72874176,205.93593508)
\lineto(44.72874176,205.78593508)
\curveto(44.69873417,205.64593177)(44.67873419,205.50593191)(44.66874176,205.36593508)
\curveto(44.65873421,205.23593218)(44.63373424,205.10593231)(44.59374176,204.97593508)
\curveto(44.5737343,204.89593252)(44.55373432,204.8109326)(44.53374176,204.72093508)
\lineto(44.47374176,204.48093508)
\lineto(44.35374176,204.18093508)
\curveto(44.32373455,204.09093332)(44.28873458,204.00093341)(44.24874176,203.91093508)
\curveto(44.14873472,203.69093372)(44.01373486,203.47593394)(43.84374176,203.26593508)
\curveto(43.68373519,203.05593436)(43.50873536,202.88593453)(43.31874176,202.75593508)
\curveto(43.2687356,202.7159347)(43.20873566,202.67593474)(43.13874176,202.63593508)
\curveto(43.07873579,202.60593481)(43.01873585,202.57093484)(42.95874176,202.53093508)
\curveto(42.87873599,202.48093493)(42.78373609,202.44093497)(42.67374176,202.41093508)
\curveto(42.56373631,202.38093503)(42.45873641,202.35093506)(42.35874176,202.32093508)
\curveto(42.24873662,202.28093513)(42.13873673,202.25593516)(42.02874176,202.24593508)
\curveto(41.91873695,202.23593518)(41.80373707,202.22093519)(41.68374176,202.20093508)
\curveto(41.64373723,202.19093522)(41.59873727,202.19093522)(41.54874176,202.20093508)
\curveto(41.50873736,202.20093521)(41.4687374,202.19593522)(41.42874176,202.18593508)
\curveto(41.38873748,202.17593524)(41.33373754,202.17093524)(41.26374176,202.17093508)
\curveto(41.19373768,202.17093524)(41.14373773,202.17593524)(41.11374176,202.18593508)
\curveto(41.06373781,202.20593521)(41.01873785,202.2109352)(40.97874176,202.20093508)
\curveto(40.93873793,202.19093522)(40.90373797,202.19093522)(40.87374176,202.20093508)
\lineto(40.78374176,202.20093508)
\curveto(40.72373815,202.22093519)(40.65873821,202.23593518)(40.58874176,202.24593508)
\curveto(40.52873834,202.24593517)(40.46373841,202.25093516)(40.39374176,202.26093508)
\curveto(40.22373865,202.3109351)(40.06373881,202.36093505)(39.91374176,202.41093508)
\curveto(39.76373911,202.46093495)(39.61873925,202.52593489)(39.47874176,202.60593508)
\curveto(39.42873944,202.64593477)(39.3737395,202.67593474)(39.31374176,202.69593508)
\curveto(39.26373961,202.72593469)(39.21373966,202.76093465)(39.16374176,202.80093508)
\curveto(38.92373995,202.98093443)(38.72374015,203.20093421)(38.56374176,203.46093508)
\curveto(38.40374047,203.72093369)(38.26374061,204.00593341)(38.14374176,204.31593508)
\curveto(38.08374079,204.45593296)(38.03874083,204.59593282)(38.00874176,204.73593508)
\curveto(37.97874089,204.88593253)(37.94374093,205.04093237)(37.90374176,205.20093508)
\curveto(37.88374099,205.3109321)(37.868741,205.42093199)(37.85874176,205.53093508)
\curveto(37.84874102,205.64093177)(37.83374104,205.75093166)(37.81374176,205.86093508)
\curveto(37.80374107,205.90093151)(37.79874107,205.94093147)(37.79874176,205.98093508)
\curveto(37.80874106,206.02093139)(37.80874106,206.06093135)(37.79874176,206.10093508)
\curveto(37.78874108,206.15093126)(37.78374109,206.20093121)(37.78374176,206.25093508)
\lineto(37.78374176,206.41593508)
\curveto(37.76374111,206.46593095)(37.75874111,206.5159309)(37.76874176,206.56593508)
\curveto(37.77874109,206.62593079)(37.77874109,206.68093073)(37.76874176,206.73093508)
\curveto(37.75874111,206.77093064)(37.75874111,206.8159306)(37.76874176,206.86593508)
\curveto(37.77874109,206.9159305)(37.7737411,206.96593045)(37.75374176,207.01593508)
\curveto(37.73374114,207.08593033)(37.72874114,207.16093025)(37.73874176,207.24093508)
\curveto(37.74874112,207.33093008)(37.75374112,207.41593)(37.75374176,207.49593508)
\curveto(37.75374112,207.58592983)(37.74874112,207.68592973)(37.73874176,207.79593508)
\curveto(37.72874114,207.9159295)(37.73374114,208.0159294)(37.75374176,208.09593508)
\lineto(37.75374176,208.38093508)
\lineto(37.79874176,209.01093508)
\curveto(37.80874106,209.1109283)(37.81874105,209.20592821)(37.82874176,209.29593508)
\lineto(37.85874176,209.59593508)
\curveto(37.87874099,209.64592777)(37.88374099,209.69592772)(37.87374176,209.74593508)
\curveto(37.873741,209.80592761)(37.88374099,209.86092755)(37.90374176,209.91093508)
\curveto(37.95374092,210.08092733)(37.99374088,210.24592717)(38.02374176,210.40593508)
\curveto(38.05374082,210.57592684)(38.10374077,210.73592668)(38.17374176,210.88593508)
\curveto(38.36374051,211.34592607)(38.58374029,211.72092569)(38.83374176,212.01093508)
\curveto(39.09373978,212.30092511)(39.45373942,212.54592487)(39.91374176,212.74593508)
\curveto(40.04373883,212.79592462)(40.1737387,212.83092458)(40.30374176,212.85093508)
\curveto(40.44373843,212.87092454)(40.58373829,212.89592452)(40.72374176,212.92593508)
\curveto(40.79373808,212.93592448)(40.85873801,212.94092447)(40.91874176,212.94093508)
\curveto(40.97873789,212.94092447)(41.04373783,212.94592447)(41.11374176,212.95593508)
\curveto(41.94373693,212.97592444)(42.61373626,212.82592459)(43.12374176,212.50593508)
\curveto(43.63373524,212.19592522)(44.01373486,211.75592566)(44.26374176,211.18593508)
\curveto(44.31373456,211.06592635)(44.35873451,210.94092647)(44.39874176,210.81093508)
\curveto(44.43873443,210.68092673)(44.48373439,210.54592687)(44.53374176,210.40593508)
\curveto(44.55373432,210.32592709)(44.5687343,210.24092717)(44.57874176,210.15093508)
\lineto(44.63874176,209.91093508)
\curveto(44.6687342,209.80092761)(44.68373419,209.69092772)(44.68374176,209.58093508)
\curveto(44.69373418,209.47092794)(44.70873416,209.36092805)(44.72874176,209.25093508)
\curveto(44.74873412,209.20092821)(44.75373412,209.15592826)(44.74374176,209.11593508)
\curveto(44.74373413,209.07592834)(44.74873412,209.03592838)(44.75874176,208.99593508)
\curveto(44.7687341,208.94592847)(44.7687341,208.89092852)(44.75874176,208.83093508)
\curveto(44.75873411,208.78092863)(44.76373411,208.73092868)(44.77374176,208.68093508)
\lineto(44.77374176,208.54593508)
\curveto(44.79373408,208.48592893)(44.79373408,208.415929)(44.77374176,208.33593508)
\curveto(44.76373411,208.26592915)(44.7687341,208.20092921)(44.78874176,208.14093508)
\curveto(44.79873407,208.1109293)(44.80373407,208.07092934)(44.80374176,208.02093508)
\lineto(44.80374176,207.90093508)
\lineto(44.80374176,207.43593508)
\moveto(43.25874176,205.11093508)
\curveto(43.35873551,205.43093198)(43.41873545,205.79593162)(43.43874176,206.20593508)
\curveto(43.45873541,206.6159308)(43.4687354,207.02593039)(43.46874176,207.43593508)
\curveto(43.4687354,207.86592955)(43.45873541,208.28592913)(43.43874176,208.69593508)
\curveto(43.41873545,209.10592831)(43.3737355,209.49092792)(43.30374176,209.85093508)
\curveto(43.23373564,210.2109272)(43.12373575,210.53092688)(42.97374176,210.81093508)
\curveto(42.83373604,211.10092631)(42.63873623,211.33592608)(42.38874176,211.51593508)
\curveto(42.22873664,211.62592579)(42.04873682,211.70592571)(41.84874176,211.75593508)
\curveto(41.64873722,211.8159256)(41.40373747,211.84592557)(41.11374176,211.84593508)
\curveto(41.09373778,211.82592559)(41.05873781,211.8159256)(41.00874176,211.81593508)
\curveto(40.95873791,211.82592559)(40.91873795,211.82592559)(40.88874176,211.81593508)
\curveto(40.80873806,211.79592562)(40.73373814,211.77592564)(40.66374176,211.75593508)
\curveto(40.60373827,211.74592567)(40.53873833,211.72592569)(40.46874176,211.69593508)
\curveto(40.19873867,211.57592584)(39.97873889,211.40592601)(39.80874176,211.18593508)
\curveto(39.64873922,210.97592644)(39.51373936,210.73092668)(39.40374176,210.45093508)
\curveto(39.35373952,210.34092707)(39.31373956,210.22092719)(39.28374176,210.09093508)
\curveto(39.26373961,209.97092744)(39.23873963,209.84592757)(39.20874176,209.71593508)
\curveto(39.18873968,209.66592775)(39.17873969,209.6109278)(39.17874176,209.55093508)
\curveto(39.17873969,209.50092791)(39.1737397,209.45092796)(39.16374176,209.40093508)
\curveto(39.15373972,209.3109281)(39.14373973,209.2159282)(39.13374176,209.11593508)
\curveto(39.12373975,209.02592839)(39.11373976,208.93092848)(39.10374176,208.83093508)
\curveto(39.10373977,208.75092866)(39.09873977,208.66592875)(39.08874176,208.57593508)
\lineto(39.08874176,208.33593508)
\lineto(39.08874176,208.15593508)
\curveto(39.07873979,208.12592929)(39.0737398,208.09092932)(39.07374176,208.05093508)
\lineto(39.07374176,207.91593508)
\lineto(39.07374176,207.46593508)
\curveto(39.0737398,207.38593003)(39.0687398,207.30093011)(39.05874176,207.21093508)
\curveto(39.05873981,207.13093028)(39.0687398,207.05593036)(39.08874176,206.98593508)
\lineto(39.08874176,206.71593508)
\curveto(39.08873978,206.69593072)(39.08373979,206.66593075)(39.07374176,206.62593508)
\curveto(39.0737398,206.59593082)(39.07873979,206.57093084)(39.08874176,206.55093508)
\curveto(39.09873977,206.45093096)(39.10373977,206.35093106)(39.10374176,206.25093508)
\curveto(39.11373976,206.16093125)(39.12373975,206.06093135)(39.13374176,205.95093508)
\curveto(39.16373971,205.83093158)(39.17873969,205.70593171)(39.17874176,205.57593508)
\curveto(39.18873968,205.45593196)(39.21373966,205.34093207)(39.25374176,205.23093508)
\curveto(39.33373954,204.93093248)(39.41873945,204.66593275)(39.50874176,204.43593508)
\curveto(39.60873926,204.20593321)(39.75373912,203.99093342)(39.94374176,203.79093508)
\curveto(40.15373872,203.59093382)(40.41873845,203.44093397)(40.73874176,203.34093508)
\curveto(40.77873809,203.32093409)(40.81373806,203.3109341)(40.84374176,203.31093508)
\curveto(40.88373799,203.32093409)(40.92873794,203.3159341)(40.97874176,203.29593508)
\curveto(41.01873785,203.28593413)(41.08873778,203.27593414)(41.18874176,203.26593508)
\curveto(41.29873757,203.25593416)(41.38373749,203.26093415)(41.44374176,203.28093508)
\curveto(41.51373736,203.30093411)(41.58373729,203.3109341)(41.65374176,203.31093508)
\curveto(41.72373715,203.32093409)(41.78873708,203.33593408)(41.84874176,203.35593508)
\curveto(42.04873682,203.415934)(42.22873664,203.50093391)(42.38874176,203.61093508)
\curveto(42.41873645,203.63093378)(42.44373643,203.65093376)(42.46374176,203.67093508)
\lineto(42.52374176,203.73093508)
\curveto(42.56373631,203.75093366)(42.61373626,203.79093362)(42.67374176,203.85093508)
\curveto(42.7737361,203.99093342)(42.85873601,204.12093329)(42.92874176,204.24093508)
\curveto(42.99873587,204.36093305)(43.0687358,204.50593291)(43.13874176,204.67593508)
\curveto(43.1687357,204.74593267)(43.18873568,204.8159326)(43.19874176,204.88593508)
\curveto(43.21873565,204.95593246)(43.23873563,205.03093238)(43.25874176,205.11093508)
}
}
{
\newrgbcolor{curcolor}{0 0 0}
\pscustom[linestyle=none,fillstyle=solid,fillcolor=curcolor]
{
\newpath
\moveto(44.80374176,127.24385744)
\lineto(44.80374176,126.98885744)
\curveto(44.81373406,126.90885268)(44.80873406,126.83385275)(44.78874176,126.76385744)
\lineto(44.78874176,126.52385744)
\lineto(44.78874176,126.35885744)
\curveto(44.7687341,126.25885333)(44.75873411,126.15385343)(44.75874176,126.04385744)
\curveto(44.75873411,125.94385364)(44.74873412,125.84385374)(44.72874176,125.74385744)
\lineto(44.72874176,125.59385744)
\curveto(44.69873417,125.45385413)(44.67873419,125.31385427)(44.66874176,125.17385744)
\curveto(44.65873421,125.04385454)(44.63373424,124.91385467)(44.59374176,124.78385744)
\curveto(44.5737343,124.70385488)(44.55373432,124.61885497)(44.53374176,124.52885744)
\lineto(44.47374176,124.28885744)
\lineto(44.35374176,123.98885744)
\curveto(44.32373455,123.89885569)(44.28873458,123.80885578)(44.24874176,123.71885744)
\curveto(44.14873472,123.49885609)(44.01373486,123.2838563)(43.84374176,123.07385744)
\curveto(43.68373519,122.86385672)(43.50873536,122.69385689)(43.31874176,122.56385744)
\curveto(43.2687356,122.52385706)(43.20873566,122.4838571)(43.13874176,122.44385744)
\curveto(43.07873579,122.41385717)(43.01873585,122.37885721)(42.95874176,122.33885744)
\curveto(42.87873599,122.2888573)(42.78373609,122.24885734)(42.67374176,122.21885744)
\curveto(42.56373631,122.1888574)(42.45873641,122.15885743)(42.35874176,122.12885744)
\curveto(42.24873662,122.0888575)(42.13873673,122.06385752)(42.02874176,122.05385744)
\curveto(41.91873695,122.04385754)(41.80373707,122.02885756)(41.68374176,122.00885744)
\curveto(41.64373723,121.99885759)(41.59873727,121.99885759)(41.54874176,122.00885744)
\curveto(41.50873736,122.00885758)(41.4687374,122.00385758)(41.42874176,121.99385744)
\curveto(41.38873748,121.9838576)(41.33373754,121.97885761)(41.26374176,121.97885744)
\curveto(41.19373768,121.97885761)(41.14373773,121.9838576)(41.11374176,121.99385744)
\curveto(41.06373781,122.01385757)(41.01873785,122.01885757)(40.97874176,122.00885744)
\curveto(40.93873793,121.99885759)(40.90373797,121.99885759)(40.87374176,122.00885744)
\lineto(40.78374176,122.00885744)
\curveto(40.72373815,122.02885756)(40.65873821,122.04385754)(40.58874176,122.05385744)
\curveto(40.52873834,122.05385753)(40.46373841,122.05885753)(40.39374176,122.06885744)
\curveto(40.22373865,122.11885747)(40.06373881,122.16885742)(39.91374176,122.21885744)
\curveto(39.76373911,122.26885732)(39.61873925,122.33385725)(39.47874176,122.41385744)
\curveto(39.42873944,122.45385713)(39.3737395,122.4838571)(39.31374176,122.50385744)
\curveto(39.26373961,122.53385705)(39.21373966,122.56885702)(39.16374176,122.60885744)
\curveto(38.92373995,122.7888568)(38.72374015,123.00885658)(38.56374176,123.26885744)
\curveto(38.40374047,123.52885606)(38.26374061,123.81385577)(38.14374176,124.12385744)
\curveto(38.08374079,124.26385532)(38.03874083,124.40385518)(38.00874176,124.54385744)
\curveto(37.97874089,124.69385489)(37.94374093,124.84885474)(37.90374176,125.00885744)
\curveto(37.88374099,125.11885447)(37.868741,125.22885436)(37.85874176,125.33885744)
\curveto(37.84874102,125.44885414)(37.83374104,125.55885403)(37.81374176,125.66885744)
\curveto(37.80374107,125.70885388)(37.79874107,125.74885384)(37.79874176,125.78885744)
\curveto(37.80874106,125.82885376)(37.80874106,125.86885372)(37.79874176,125.90885744)
\curveto(37.78874108,125.95885363)(37.78374109,126.00885358)(37.78374176,126.05885744)
\lineto(37.78374176,126.22385744)
\curveto(37.76374111,126.27385331)(37.75874111,126.32385326)(37.76874176,126.37385744)
\curveto(37.77874109,126.43385315)(37.77874109,126.4888531)(37.76874176,126.53885744)
\curveto(37.75874111,126.57885301)(37.75874111,126.62385296)(37.76874176,126.67385744)
\curveto(37.77874109,126.72385286)(37.7737411,126.77385281)(37.75374176,126.82385744)
\curveto(37.73374114,126.89385269)(37.72874114,126.96885262)(37.73874176,127.04885744)
\curveto(37.74874112,127.13885245)(37.75374112,127.22385236)(37.75374176,127.30385744)
\curveto(37.75374112,127.39385219)(37.74874112,127.49385209)(37.73874176,127.60385744)
\curveto(37.72874114,127.72385186)(37.73374114,127.82385176)(37.75374176,127.90385744)
\lineto(37.75374176,128.18885744)
\lineto(37.79874176,128.81885744)
\curveto(37.80874106,128.91885067)(37.81874105,129.01385057)(37.82874176,129.10385744)
\lineto(37.85874176,129.40385744)
\curveto(37.87874099,129.45385013)(37.88374099,129.50385008)(37.87374176,129.55385744)
\curveto(37.873741,129.61384997)(37.88374099,129.66884992)(37.90374176,129.71885744)
\curveto(37.95374092,129.8888497)(37.99374088,130.05384953)(38.02374176,130.21385744)
\curveto(38.05374082,130.3838492)(38.10374077,130.54384904)(38.17374176,130.69385744)
\curveto(38.36374051,131.15384843)(38.58374029,131.52884806)(38.83374176,131.81885744)
\curveto(39.09373978,132.10884748)(39.45373942,132.35384723)(39.91374176,132.55385744)
\curveto(40.04373883,132.60384698)(40.1737387,132.63884695)(40.30374176,132.65885744)
\curveto(40.44373843,132.67884691)(40.58373829,132.70384688)(40.72374176,132.73385744)
\curveto(40.79373808,132.74384684)(40.85873801,132.74884684)(40.91874176,132.74885744)
\curveto(40.97873789,132.74884684)(41.04373783,132.75384683)(41.11374176,132.76385744)
\curveto(41.94373693,132.7838468)(42.61373626,132.63384695)(43.12374176,132.31385744)
\curveto(43.63373524,132.00384758)(44.01373486,131.56384802)(44.26374176,130.99385744)
\curveto(44.31373456,130.87384871)(44.35873451,130.74884884)(44.39874176,130.61885744)
\curveto(44.43873443,130.4888491)(44.48373439,130.35384923)(44.53374176,130.21385744)
\curveto(44.55373432,130.13384945)(44.5687343,130.04884954)(44.57874176,129.95885744)
\lineto(44.63874176,129.71885744)
\curveto(44.6687342,129.60884998)(44.68373419,129.49885009)(44.68374176,129.38885744)
\curveto(44.69373418,129.27885031)(44.70873416,129.16885042)(44.72874176,129.05885744)
\curveto(44.74873412,129.00885058)(44.75373412,128.96385062)(44.74374176,128.92385744)
\curveto(44.74373413,128.8838507)(44.74873412,128.84385074)(44.75874176,128.80385744)
\curveto(44.7687341,128.75385083)(44.7687341,128.69885089)(44.75874176,128.63885744)
\curveto(44.75873411,128.588851)(44.76373411,128.53885105)(44.77374176,128.48885744)
\lineto(44.77374176,128.35385744)
\curveto(44.79373408,128.29385129)(44.79373408,128.22385136)(44.77374176,128.14385744)
\curveto(44.76373411,128.07385151)(44.7687341,128.00885158)(44.78874176,127.94885744)
\curveto(44.79873407,127.91885167)(44.80373407,127.87885171)(44.80374176,127.82885744)
\lineto(44.80374176,127.70885744)
\lineto(44.80374176,127.24385744)
\moveto(43.25874176,124.91885744)
\curveto(43.35873551,125.23885435)(43.41873545,125.60385398)(43.43874176,126.01385744)
\curveto(43.45873541,126.42385316)(43.4687354,126.83385275)(43.46874176,127.24385744)
\curveto(43.4687354,127.67385191)(43.45873541,128.09385149)(43.43874176,128.50385744)
\curveto(43.41873545,128.91385067)(43.3737355,129.29885029)(43.30374176,129.65885744)
\curveto(43.23373564,130.01884957)(43.12373575,130.33884925)(42.97374176,130.61885744)
\curveto(42.83373604,130.90884868)(42.63873623,131.14384844)(42.38874176,131.32385744)
\curveto(42.22873664,131.43384815)(42.04873682,131.51384807)(41.84874176,131.56385744)
\curveto(41.64873722,131.62384796)(41.40373747,131.65384793)(41.11374176,131.65385744)
\curveto(41.09373778,131.63384795)(41.05873781,131.62384796)(41.00874176,131.62385744)
\curveto(40.95873791,131.63384795)(40.91873795,131.63384795)(40.88874176,131.62385744)
\curveto(40.80873806,131.60384798)(40.73373814,131.583848)(40.66374176,131.56385744)
\curveto(40.60373827,131.55384803)(40.53873833,131.53384805)(40.46874176,131.50385744)
\curveto(40.19873867,131.3838482)(39.97873889,131.21384837)(39.80874176,130.99385744)
\curveto(39.64873922,130.7838488)(39.51373936,130.53884905)(39.40374176,130.25885744)
\curveto(39.35373952,130.14884944)(39.31373956,130.02884956)(39.28374176,129.89885744)
\curveto(39.26373961,129.77884981)(39.23873963,129.65384993)(39.20874176,129.52385744)
\curveto(39.18873968,129.47385011)(39.17873969,129.41885017)(39.17874176,129.35885744)
\curveto(39.17873969,129.30885028)(39.1737397,129.25885033)(39.16374176,129.20885744)
\curveto(39.15373972,129.11885047)(39.14373973,129.02385056)(39.13374176,128.92385744)
\curveto(39.12373975,128.83385075)(39.11373976,128.73885085)(39.10374176,128.63885744)
\curveto(39.10373977,128.55885103)(39.09873977,128.47385111)(39.08874176,128.38385744)
\lineto(39.08874176,128.14385744)
\lineto(39.08874176,127.96385744)
\curveto(39.07873979,127.93385165)(39.0737398,127.89885169)(39.07374176,127.85885744)
\lineto(39.07374176,127.72385744)
\lineto(39.07374176,127.27385744)
\curveto(39.0737398,127.19385239)(39.0687398,127.10885248)(39.05874176,127.01885744)
\curveto(39.05873981,126.93885265)(39.0687398,126.86385272)(39.08874176,126.79385744)
\lineto(39.08874176,126.52385744)
\curveto(39.08873978,126.50385308)(39.08373979,126.47385311)(39.07374176,126.43385744)
\curveto(39.0737398,126.40385318)(39.07873979,126.37885321)(39.08874176,126.35885744)
\curveto(39.09873977,126.25885333)(39.10373977,126.15885343)(39.10374176,126.05885744)
\curveto(39.11373976,125.96885362)(39.12373975,125.86885372)(39.13374176,125.75885744)
\curveto(39.16373971,125.63885395)(39.17873969,125.51385407)(39.17874176,125.38385744)
\curveto(39.18873968,125.26385432)(39.21373966,125.14885444)(39.25374176,125.03885744)
\curveto(39.33373954,124.73885485)(39.41873945,124.47385511)(39.50874176,124.24385744)
\curveto(39.60873926,124.01385557)(39.75373912,123.79885579)(39.94374176,123.59885744)
\curveto(40.15373872,123.39885619)(40.41873845,123.24885634)(40.73874176,123.14885744)
\curveto(40.77873809,123.12885646)(40.81373806,123.11885647)(40.84374176,123.11885744)
\curveto(40.88373799,123.12885646)(40.92873794,123.12385646)(40.97874176,123.10385744)
\curveto(41.01873785,123.09385649)(41.08873778,123.0838565)(41.18874176,123.07385744)
\curveto(41.29873757,123.06385652)(41.38373749,123.06885652)(41.44374176,123.08885744)
\curveto(41.51373736,123.10885648)(41.58373729,123.11885647)(41.65374176,123.11885744)
\curveto(41.72373715,123.12885646)(41.78873708,123.14385644)(41.84874176,123.16385744)
\curveto(42.04873682,123.22385636)(42.22873664,123.30885628)(42.38874176,123.41885744)
\curveto(42.41873645,123.43885615)(42.44373643,123.45885613)(42.46374176,123.47885744)
\lineto(42.52374176,123.53885744)
\curveto(42.56373631,123.55885603)(42.61373626,123.59885599)(42.67374176,123.65885744)
\curveto(42.7737361,123.79885579)(42.85873601,123.92885566)(42.92874176,124.04885744)
\curveto(42.99873587,124.16885542)(43.0687358,124.31385527)(43.13874176,124.48385744)
\curveto(43.1687357,124.55385503)(43.18873568,124.62385496)(43.19874176,124.69385744)
\curveto(43.21873565,124.76385482)(43.23873563,124.83885475)(43.25874176,124.91885744)
}
}
{
\newrgbcolor{curcolor}{0 0 0}
\pscustom[linewidth=1,linecolor=curcolor]
{
\newpath
\moveto(67.142857,287.5)
\lineto(1024.2857,287.5)
}
}
{
\newrgbcolor{curcolor}{0 0 0}
\pscustom[linewidth=1,linecolor=curcolor]
{
\newpath
\moveto(67.142857,207.54464)
\lineto(1024.2857,207.54464)
}
}
{
\newrgbcolor{curcolor}{0 0 0}
\pscustom[linewidth=1,linecolor=curcolor]
{
\newpath
\moveto(67.142857,127.5)
\lineto(1024.2857,127.5)
}
}
{
\newrgbcolor{curcolor}{0 0 0}
\pscustom[linewidth=1,linecolor=curcolor]
{
\newpath
\moveto(67.142857,367.5)
\lineto(1024.2857,367.5)
}
}
{
\newrgbcolor{curcolor}{1 1 1}
\pscustom[linestyle=none,fillstyle=solid,fillcolor=curcolor]
{
\newpath
\moveto(107.83379364,399.88729616)
\lineto(221.22342682,399.88729616)
\lineto(221.22342682,314.40996309)
\lineto(107.83379364,314.40996309)
\closepath
}
}
{
\newrgbcolor{curcolor}{0 0 0}
\pscustom[linewidth=0.89664584,linecolor=curcolor]
{
\newpath
\moveto(107.83379364,399.88729616)
\lineto(221.22342682,399.88729616)
\lineto(221.22342682,314.40996309)
\lineto(107.83379364,314.40996309)
\closepath
}
}
{
\newrgbcolor{curcolor}{0 0 0}
\pscustom[linewidth=1,linecolor=curcolor]
{
\newpath
\moveto(67.142857,447.45536)
\lineto(1024.2857,447.45536)
}
}
{
\newrgbcolor{curcolor}{0 0 0}
\pscustom[linestyle=none,fillstyle=solid,fillcolor=curcolor]
{
\newpath
\moveto(144.77876512,384.77531497)
\curveto(145.75875862,384.79530402)(146.5787578,384.63530418)(147.23876512,384.29531497)
\curveto(147.90875647,383.96530485)(148.42875595,383.50530531)(148.79876512,382.91531497)
\curveto(148.89875548,382.75530606)(148.9787554,382.60030621)(149.03876512,382.45031497)
\curveto(149.10875527,382.3103065)(149.1737552,382.14030667)(149.23376512,381.94031497)
\curveto(149.25375512,381.89030692)(149.2737551,381.82030699)(149.29376512,381.73031497)
\curveto(149.31375506,381.65030716)(149.30875507,381.57530724)(149.27876512,381.50531497)
\curveto(149.25875512,381.44530737)(149.21875516,381.40530741)(149.15876512,381.38531497)
\curveto(149.10875527,381.37530744)(149.05375532,381.36030745)(148.99376512,381.34031497)
\lineto(148.84376512,381.34031497)
\curveto(148.81375556,381.33030748)(148.7737556,381.32530749)(148.72376512,381.32531497)
\lineto(148.60376512,381.32531497)
\curveto(148.46375591,381.32530749)(148.33375604,381.33030748)(148.21376512,381.34031497)
\curveto(148.10375627,381.36030745)(148.02375635,381.4103074)(147.97376512,381.49031497)
\curveto(147.90375647,381.59030722)(147.84875653,381.70530711)(147.80876512,381.83531497)
\curveto(147.76875661,381.96530685)(147.71375666,382.08530673)(147.64376512,382.19531497)
\curveto(147.51375686,382.4153064)(147.36375701,382.60530621)(147.19376512,382.76531497)
\curveto(147.03375734,382.92530589)(146.84375753,383.07530574)(146.62376512,383.21531497)
\curveto(146.50375787,383.29530552)(146.36875801,383.35530546)(146.21876512,383.39531497)
\curveto(146.0787583,383.43530538)(145.93375844,383.47530534)(145.78376512,383.51531497)
\curveto(145.6737587,383.54530527)(145.54875883,383.56530525)(145.40876512,383.57531497)
\curveto(145.26875911,383.59530522)(145.11875926,383.60530521)(144.95876512,383.60531497)
\curveto(144.80875957,383.60530521)(144.65875972,383.59530522)(144.50876512,383.57531497)
\curveto(144.36876001,383.56530525)(144.24876013,383.54530527)(144.14876512,383.51531497)
\curveto(144.04876033,383.49530532)(143.95376042,383.47530534)(143.86376512,383.45531497)
\curveto(143.7737606,383.43530538)(143.68376069,383.40530541)(143.59376512,383.36531497)
\curveto(142.75376162,383.0153058)(142.14876223,382.4153064)(141.77876512,381.56531497)
\curveto(141.70876267,381.42530739)(141.64876273,381.27530754)(141.59876512,381.11531497)
\curveto(141.55876282,380.96530785)(141.51376286,380.810308)(141.46376512,380.65031497)
\curveto(141.44376293,380.59030822)(141.43376294,380.52530829)(141.43376512,380.45531497)
\curveto(141.43376294,380.39530842)(141.42376295,380.33530848)(141.40376512,380.27531497)
\curveto(141.39376298,380.23530858)(141.38876299,380.20030861)(141.38876512,380.17031497)
\curveto(141.38876299,380.14030867)(141.38376299,380.10530871)(141.37376512,380.06531497)
\curveto(141.35376302,379.95530886)(141.33876304,379.84030897)(141.32876512,379.72031497)
\lineto(141.32876512,379.37531497)
\curveto(141.32876305,379.30530951)(141.32376305,379.23030958)(141.31376512,379.15031497)
\curveto(141.31376306,379.08030973)(141.31876306,379.0153098)(141.32876512,378.95531497)
\lineto(141.32876512,378.80531497)
\curveto(141.34876303,378.73531008)(141.35376302,378.66531015)(141.34376512,378.59531497)
\curveto(141.34376303,378.52531029)(141.35376302,378.45531036)(141.37376512,378.38531497)
\curveto(141.39376298,378.32531049)(141.39876298,378.26531055)(141.38876512,378.20531497)
\curveto(141.38876299,378.14531067)(141.39876298,378.09031072)(141.41876512,378.04031497)
\curveto(141.44876293,377.9103109)(141.4737629,377.78031103)(141.49376512,377.65031497)
\curveto(141.52376285,377.53031128)(141.55876282,377.4103114)(141.59876512,377.29031497)
\curveto(141.76876261,376.79031202)(141.98876239,376.36031245)(142.25876512,376.00031497)
\curveto(142.52876185,375.65031316)(142.88376149,375.36031345)(143.32376512,375.13031497)
\curveto(143.46376091,375.06031375)(143.60376077,375.00531381)(143.74376512,374.96531497)
\curveto(143.89376048,374.92531389)(144.05376032,374.88031393)(144.22376512,374.83031497)
\curveto(144.29376008,374.810314)(144.35876002,374.80031401)(144.41876512,374.80031497)
\curveto(144.4787599,374.810314)(144.54875983,374.80531401)(144.62876512,374.78531497)
\curveto(144.6787597,374.77531404)(144.76875961,374.76531405)(144.89876512,374.75531497)
\curveto(145.02875935,374.75531406)(145.12375925,374.76531405)(145.18376512,374.78531497)
\lineto(145.28876512,374.78531497)
\curveto(145.32875905,374.79531402)(145.36875901,374.79531402)(145.40876512,374.78531497)
\curveto(145.44875893,374.78531403)(145.48875889,374.79531402)(145.52876512,374.81531497)
\curveto(145.62875875,374.83531398)(145.72375865,374.85031396)(145.81376512,374.86031497)
\curveto(145.91375846,374.88031393)(146.00875837,374.9103139)(146.09876512,374.95031497)
\curveto(146.8787575,375.27031354)(147.42875695,375.79531302)(147.74876512,376.52531497)
\curveto(147.82875655,376.70531211)(147.90375647,376.92031189)(147.97376512,377.17031497)
\curveto(147.99375638,377.26031155)(148.00875637,377.35031146)(148.01876512,377.44031497)
\curveto(148.03875634,377.54031127)(148.0737563,377.63031118)(148.12376512,377.71031497)
\curveto(148.1737562,377.79031102)(148.25375612,377.83531098)(148.36376512,377.84531497)
\curveto(148.4737559,377.85531096)(148.59375578,377.86031095)(148.72376512,377.86031497)
\lineto(148.87376512,377.86031497)
\curveto(148.92375545,377.86031095)(148.96875541,377.85531096)(149.00876512,377.84531497)
\lineto(149.11376512,377.84531497)
\lineto(149.20376512,377.81531497)
\curveto(149.24375513,377.815311)(149.2737551,377.80531101)(149.29376512,377.78531497)
\curveto(149.36375501,377.74531107)(149.40375497,377.67031114)(149.41376512,377.56031497)
\curveto(149.42375495,377.46031135)(149.41375496,377.36031145)(149.38376512,377.26031497)
\curveto(149.32375505,377.03031178)(149.26875511,376.810312)(149.21876512,376.60031497)
\curveto(149.16875521,376.39031242)(149.09375528,376.19031262)(148.99376512,376.00031497)
\curveto(148.91375546,375.87031294)(148.83875554,375.74531307)(148.76876512,375.62531497)
\curveto(148.70875567,375.50531331)(148.63875574,375.38531343)(148.55876512,375.26531497)
\curveto(148.378756,375.00531381)(148.15375622,374.76531405)(147.88376512,374.54531497)
\curveto(147.62375675,374.33531448)(147.33875704,374.16031465)(147.02876512,374.02031497)
\curveto(146.91875746,373.97031484)(146.80875757,373.93031488)(146.69876512,373.90031497)
\curveto(146.59875778,373.87031494)(146.49375788,373.83531498)(146.38376512,373.79531497)
\curveto(146.2737581,373.75531506)(146.15875822,373.73031508)(146.03876512,373.72031497)
\curveto(145.92875845,373.70031511)(145.81375856,373.68031513)(145.69376512,373.66031497)
\curveto(145.64375873,373.64031517)(145.59875878,373.63531518)(145.55876512,373.64531497)
\curveto(145.51875886,373.64531517)(145.4787589,373.64031517)(145.43876512,373.63031497)
\curveto(145.378759,373.62031519)(145.31875906,373.6153152)(145.25876512,373.61531497)
\curveto(145.19875918,373.6153152)(145.13375924,373.6103152)(145.06376512,373.60031497)
\curveto(145.03375934,373.59031522)(144.96375941,373.59031522)(144.85376512,373.60031497)
\curveto(144.75375962,373.60031521)(144.68875969,373.60531521)(144.65876512,373.61531497)
\curveto(144.60875977,373.62531519)(144.55875982,373.63031518)(144.50876512,373.63031497)
\curveto(144.46875991,373.62031519)(144.42375995,373.62031519)(144.37376512,373.63031497)
\lineto(144.22376512,373.63031497)
\curveto(144.14376023,373.65031516)(144.06876031,373.66531515)(143.99876512,373.67531497)
\curveto(143.92876045,373.67531514)(143.85376052,373.68531513)(143.77376512,373.70531497)
\lineto(143.50376512,373.76531497)
\curveto(143.41376096,373.77531504)(143.32876105,373.79531502)(143.24876512,373.82531497)
\curveto(143.03876134,373.88531493)(142.84876153,373.96031485)(142.67876512,374.05031497)
\curveto(142.04876233,374.32031449)(141.53876284,374.70531411)(141.14876512,375.20531497)
\curveto(140.75876362,375.70531311)(140.44876393,376.29531252)(140.21876512,376.97531497)
\curveto(140.1787642,377.09531172)(140.14376423,377.22031159)(140.11376512,377.35031497)
\curveto(140.09376428,377.48031133)(140.06876431,377.6153112)(140.03876512,377.75531497)
\curveto(140.01876436,377.80531101)(140.00876437,377.85031096)(140.00876512,377.89031497)
\curveto(140.01876436,377.93031088)(140.01876436,377.97531084)(140.00876512,378.02531497)
\curveto(139.98876439,378.1153107)(139.9737644,378.2103106)(139.96376512,378.31031497)
\curveto(139.96376441,378.4103104)(139.95376442,378.50531031)(139.93376512,378.59531497)
\lineto(139.93376512,378.88031497)
\curveto(139.91376446,378.93030988)(139.90376447,379.0153098)(139.90376512,379.13531497)
\curveto(139.90376447,379.25530956)(139.91376446,379.34030947)(139.93376512,379.39031497)
\curveto(139.94376443,379.42030939)(139.94376443,379.45030936)(139.93376512,379.48031497)
\curveto(139.92376445,379.52030929)(139.92376445,379.55030926)(139.93376512,379.57031497)
\lineto(139.93376512,379.70531497)
\curveto(139.94376443,379.78530903)(139.94876443,379.86530895)(139.94876512,379.94531497)
\curveto(139.95876442,380.03530878)(139.9737644,380.12030869)(139.99376512,380.20031497)
\curveto(140.01376436,380.26030855)(140.02376435,380.32030849)(140.02376512,380.38031497)
\curveto(140.02376435,380.45030836)(140.03376434,380.52030829)(140.05376512,380.59031497)
\curveto(140.10376427,380.76030805)(140.14376423,380.92530789)(140.17376512,381.08531497)
\curveto(140.20376417,381.24530757)(140.24876413,381.39530742)(140.30876512,381.53531497)
\lineto(140.45876512,381.92531497)
\curveto(140.51876386,382.06530675)(140.58376379,382.19030662)(140.65376512,382.30031497)
\curveto(140.80376357,382.56030625)(140.95376342,382.79530602)(141.10376512,383.00531497)
\curveto(141.13376324,383.05530576)(141.16876321,383.09530572)(141.20876512,383.12531497)
\curveto(141.25876312,383.16530565)(141.29876308,383.2103056)(141.32876512,383.26031497)
\curveto(141.38876299,383.34030547)(141.44876293,383.4103054)(141.50876512,383.47031497)
\lineto(141.71876512,383.65031497)
\curveto(141.7787626,383.70030511)(141.83376254,383.74530507)(141.88376512,383.78531497)
\curveto(141.94376243,383.83530498)(142.00876237,383.88530493)(142.07876512,383.93531497)
\curveto(142.22876215,384.04530477)(142.38376199,384.14030467)(142.54376512,384.22031497)
\curveto(142.71376166,384.30030451)(142.88876149,384.38030443)(143.06876512,384.46031497)
\curveto(143.1787612,384.5103043)(143.29376108,384.54530427)(143.41376512,384.56531497)
\curveto(143.54376083,384.59530422)(143.66876071,384.63030418)(143.78876512,384.67031497)
\curveto(143.85876052,384.68030413)(143.92376045,384.69030412)(143.98376512,384.70031497)
\lineto(144.16376512,384.73031497)
\curveto(144.24376013,384.74030407)(144.31876006,384.74530407)(144.38876512,384.74531497)
\curveto(144.46875991,384.75530406)(144.54875983,384.76530405)(144.62876512,384.77531497)
\curveto(144.64875973,384.78530403)(144.6737597,384.78530403)(144.70376512,384.77531497)
\curveto(144.73375964,384.76530405)(144.75875962,384.76530405)(144.77876512,384.77531497)
}
}
{
\newrgbcolor{curcolor}{0 0 0}
\pscustom[linestyle=none,fillstyle=solid,fillcolor=curcolor]
{
\newpath
\moveto(158.13860887,378.05531497)
\curveto(158.15860081,377.99531082)(158.1686008,377.90031091)(158.16860887,377.77031497)
\curveto(158.1686008,377.65031116)(158.1636008,377.56531125)(158.15360887,377.51531497)
\lineto(158.15360887,377.36531497)
\curveto(158.14360082,377.28531153)(158.13360083,377.2103116)(158.12360887,377.14031497)
\curveto(158.12360084,377.08031173)(158.11860085,377.0103118)(158.10860887,376.93031497)
\curveto(158.08860088,376.87031194)(158.07360089,376.810312)(158.06360887,376.75031497)
\curveto(158.0636009,376.69031212)(158.05360091,376.63031218)(158.03360887,376.57031497)
\curveto(157.99360097,376.44031237)(157.95860101,376.3103125)(157.92860887,376.18031497)
\curveto(157.89860107,376.05031276)(157.85860111,375.93031288)(157.80860887,375.82031497)
\curveto(157.59860137,375.34031347)(157.31860165,374.93531388)(156.96860887,374.60531497)
\curveto(156.61860235,374.28531453)(156.18860278,374.04031477)(155.67860887,373.87031497)
\curveto(155.5686034,373.83031498)(155.44860352,373.80031501)(155.31860887,373.78031497)
\curveto(155.19860377,373.76031505)(155.07360389,373.74031507)(154.94360887,373.72031497)
\curveto(154.88360408,373.7103151)(154.81860415,373.70531511)(154.74860887,373.70531497)
\curveto(154.68860428,373.69531512)(154.62860434,373.69031512)(154.56860887,373.69031497)
\curveto(154.52860444,373.68031513)(154.4686045,373.67531514)(154.38860887,373.67531497)
\curveto(154.31860465,373.67531514)(154.2686047,373.68031513)(154.23860887,373.69031497)
\curveto(154.19860477,373.70031511)(154.15860481,373.70531511)(154.11860887,373.70531497)
\curveto(154.07860489,373.69531512)(154.04360492,373.69531512)(154.01360887,373.70531497)
\lineto(153.92360887,373.70531497)
\lineto(153.56360887,373.75031497)
\curveto(153.42360554,373.79031502)(153.28860568,373.83031498)(153.15860887,373.87031497)
\curveto(153.02860594,373.9103149)(152.90360606,373.95531486)(152.78360887,374.00531497)
\curveto(152.33360663,374.20531461)(151.963607,374.46531435)(151.67360887,374.78531497)
\curveto(151.38360758,375.10531371)(151.14360782,375.49531332)(150.95360887,375.95531497)
\curveto(150.90360806,376.05531276)(150.8636081,376.15531266)(150.83360887,376.25531497)
\curveto(150.81360815,376.35531246)(150.79360817,376.46031235)(150.77360887,376.57031497)
\curveto(150.75360821,376.6103122)(150.74360822,376.64031217)(150.74360887,376.66031497)
\curveto(150.75360821,376.69031212)(150.75360821,376.72531209)(150.74360887,376.76531497)
\curveto(150.72360824,376.84531197)(150.70860826,376.92531189)(150.69860887,377.00531497)
\curveto(150.69860827,377.09531172)(150.68860828,377.18031163)(150.66860887,377.26031497)
\lineto(150.66860887,377.38031497)
\curveto(150.6686083,377.42031139)(150.6636083,377.46531135)(150.65360887,377.51531497)
\curveto(150.64360832,377.56531125)(150.63860833,377.65031116)(150.63860887,377.77031497)
\curveto(150.63860833,377.90031091)(150.64860832,377.99531082)(150.66860887,378.05531497)
\curveto(150.68860828,378.12531069)(150.69360827,378.19531062)(150.68360887,378.26531497)
\curveto(150.67360829,378.33531048)(150.67860829,378.40531041)(150.69860887,378.47531497)
\curveto(150.70860826,378.52531029)(150.71360825,378.56531025)(150.71360887,378.59531497)
\curveto(150.72360824,378.63531018)(150.73360823,378.68031013)(150.74360887,378.73031497)
\curveto(150.77360819,378.85030996)(150.79860817,378.97030984)(150.81860887,379.09031497)
\curveto(150.84860812,379.2103096)(150.88860808,379.32530949)(150.93860887,379.43531497)
\curveto(151.08860788,379.80530901)(151.2686077,380.13530868)(151.47860887,380.42531497)
\curveto(151.69860727,380.72530809)(151.963607,380.97530784)(152.27360887,381.17531497)
\curveto(152.39360657,381.25530756)(152.51860645,381.32030749)(152.64860887,381.37031497)
\curveto(152.77860619,381.43030738)(152.91360605,381.49030732)(153.05360887,381.55031497)
\curveto(153.17360579,381.60030721)(153.30360566,381.63030718)(153.44360887,381.64031497)
\curveto(153.58360538,381.66030715)(153.72360524,381.69030712)(153.86360887,381.73031497)
\lineto(154.05860887,381.73031497)
\curveto(154.12860484,381.74030707)(154.19360477,381.75030706)(154.25360887,381.76031497)
\curveto(155.14360382,381.77030704)(155.88360308,381.58530723)(156.47360887,381.20531497)
\curveto(157.0636019,380.82530799)(157.48860148,380.33030848)(157.74860887,379.72031497)
\curveto(157.79860117,379.62030919)(157.83860113,379.52030929)(157.86860887,379.42031497)
\curveto(157.89860107,379.32030949)(157.93360103,379.2153096)(157.97360887,379.10531497)
\curveto(158.00360096,378.99530982)(158.02860094,378.87530994)(158.04860887,378.74531497)
\curveto(158.0686009,378.62531019)(158.09360087,378.50031031)(158.12360887,378.37031497)
\curveto(158.13360083,378.32031049)(158.13360083,378.26531055)(158.12360887,378.20531497)
\curveto(158.12360084,378.15531066)(158.12860084,378.10531071)(158.13860887,378.05531497)
\moveto(156.80360887,377.20031497)
\curveto(156.82360214,377.27031154)(156.82860214,377.35031146)(156.81860887,377.44031497)
\lineto(156.81860887,377.69531497)
\curveto(156.81860215,378.08531073)(156.78360218,378.4153104)(156.71360887,378.68531497)
\curveto(156.68360228,378.76531005)(156.65860231,378.84530997)(156.63860887,378.92531497)
\curveto(156.61860235,379.00530981)(156.59360237,379.08030973)(156.56360887,379.15031497)
\curveto(156.28360268,379.80030901)(155.83860313,380.25030856)(155.22860887,380.50031497)
\curveto(155.15860381,380.53030828)(155.08360388,380.55030826)(155.00360887,380.56031497)
\lineto(154.76360887,380.62031497)
\curveto(154.68360428,380.64030817)(154.59860437,380.65030816)(154.50860887,380.65031497)
\lineto(154.23860887,380.65031497)
\lineto(153.96860887,380.60531497)
\curveto(153.8686051,380.58530823)(153.77360519,380.56030825)(153.68360887,380.53031497)
\curveto(153.60360536,380.5103083)(153.52360544,380.48030833)(153.44360887,380.44031497)
\curveto(153.37360559,380.42030839)(153.30860566,380.39030842)(153.24860887,380.35031497)
\curveto(153.18860578,380.3103085)(153.13360583,380.27030854)(153.08360887,380.23031497)
\curveto(152.84360612,380.06030875)(152.64860632,379.85530896)(152.49860887,379.61531497)
\curveto(152.34860662,379.37530944)(152.21860675,379.09530972)(152.10860887,378.77531497)
\curveto(152.07860689,378.67531014)(152.05860691,378.57031024)(152.04860887,378.46031497)
\curveto(152.03860693,378.36031045)(152.02360694,378.25531056)(152.00360887,378.14531497)
\curveto(151.99360697,378.10531071)(151.98860698,378.04031077)(151.98860887,377.95031497)
\curveto(151.97860699,377.92031089)(151.97360699,377.88531093)(151.97360887,377.84531497)
\curveto(151.98360698,377.80531101)(151.98860698,377.76031105)(151.98860887,377.71031497)
\lineto(151.98860887,377.41031497)
\curveto(151.98860698,377.3103115)(151.99860697,377.22031159)(152.01860887,377.14031497)
\lineto(152.04860887,376.96031497)
\curveto(152.0686069,376.86031195)(152.08360688,376.76031205)(152.09360887,376.66031497)
\curveto(152.11360685,376.57031224)(152.14360682,376.48531233)(152.18360887,376.40531497)
\curveto(152.28360668,376.16531265)(152.39860657,375.94031287)(152.52860887,375.73031497)
\curveto(152.6686063,375.52031329)(152.83860613,375.34531347)(153.03860887,375.20531497)
\curveto(153.08860588,375.17531364)(153.13360583,375.15031366)(153.17360887,375.13031497)
\curveto(153.21360575,375.1103137)(153.25860571,375.08531373)(153.30860887,375.05531497)
\curveto(153.38860558,375.00531381)(153.47360549,374.96031385)(153.56360887,374.92031497)
\curveto(153.6636053,374.89031392)(153.7686052,374.86031395)(153.87860887,374.83031497)
\curveto(153.92860504,374.810314)(153.97360499,374.80031401)(154.01360887,374.80031497)
\curveto(154.0636049,374.810314)(154.11360485,374.810314)(154.16360887,374.80031497)
\curveto(154.19360477,374.79031402)(154.25360471,374.78031403)(154.34360887,374.77031497)
\curveto(154.44360452,374.76031405)(154.51860445,374.76531405)(154.56860887,374.78531497)
\curveto(154.60860436,374.79531402)(154.64860432,374.79531402)(154.68860887,374.78531497)
\curveto(154.72860424,374.78531403)(154.7686042,374.79531402)(154.80860887,374.81531497)
\curveto(154.88860408,374.83531398)(154.968604,374.85031396)(155.04860887,374.86031497)
\curveto(155.12860384,374.88031393)(155.20360376,374.90531391)(155.27360887,374.93531497)
\curveto(155.61360335,375.07531374)(155.88860308,375.27031354)(156.09860887,375.52031497)
\curveto(156.30860266,375.77031304)(156.48360248,376.06531275)(156.62360887,376.40531497)
\curveto(156.67360229,376.52531229)(156.70360226,376.65031216)(156.71360887,376.78031497)
\curveto(156.73360223,376.92031189)(156.7636022,377.06031175)(156.80360887,377.20031497)
}
}
{
\newrgbcolor{curcolor}{0 0 0}
\pscustom[linestyle=none,fillstyle=solid,fillcolor=curcolor]
{
\newpath
\moveto(163.31689012,381.73031497)
\curveto(163.94688488,381.75030706)(164.45188438,381.66530715)(164.83189012,381.47531497)
\curveto(165.21188362,381.28530753)(165.51688331,381.00030781)(165.74689012,380.62031497)
\curveto(165.80688302,380.52030829)(165.85188298,380.4103084)(165.88189012,380.29031497)
\curveto(165.92188291,380.18030863)(165.95688287,380.06530875)(165.98689012,379.94531497)
\curveto(166.03688279,379.75530906)(166.06688276,379.55030926)(166.07689012,379.33031497)
\curveto(166.08688274,379.1103097)(166.09188274,378.88530993)(166.09189012,378.65531497)
\lineto(166.09189012,377.05031497)
\lineto(166.09189012,374.71031497)
\curveto(166.09188274,374.54031427)(166.08688274,374.37031444)(166.07689012,374.20031497)
\curveto(166.07688275,374.03031478)(166.01188282,373.92031489)(165.88189012,373.87031497)
\curveto(165.831883,373.85031496)(165.77688305,373.84031497)(165.71689012,373.84031497)
\curveto(165.66688316,373.83031498)(165.61188322,373.82531499)(165.55189012,373.82531497)
\curveto(165.42188341,373.82531499)(165.29688353,373.83031498)(165.17689012,373.84031497)
\curveto(165.05688377,373.84031497)(164.97188386,373.88031493)(164.92189012,373.96031497)
\curveto(164.87188396,374.03031478)(164.84688398,374.12031469)(164.84689012,374.23031497)
\lineto(164.84689012,374.56031497)
\lineto(164.84689012,375.85031497)
\lineto(164.84689012,378.29531497)
\curveto(164.84688398,378.56531025)(164.84188399,378.83030998)(164.83189012,379.09031497)
\curveto(164.82188401,379.36030945)(164.77688405,379.59030922)(164.69689012,379.78031497)
\curveto(164.61688421,379.98030883)(164.49688433,380.14030867)(164.33689012,380.26031497)
\curveto(164.17688465,380.39030842)(163.99188484,380.49030832)(163.78189012,380.56031497)
\curveto(163.72188511,380.58030823)(163.65688517,380.59030822)(163.58689012,380.59031497)
\curveto(163.5268853,380.60030821)(163.46688536,380.6153082)(163.40689012,380.63531497)
\curveto(163.35688547,380.64530817)(163.27688555,380.64530817)(163.16689012,380.63531497)
\curveto(163.06688576,380.63530818)(162.99688583,380.63030818)(162.95689012,380.62031497)
\curveto(162.91688591,380.60030821)(162.88188595,380.59030822)(162.85189012,380.59031497)
\curveto(162.82188601,380.60030821)(162.78688604,380.60030821)(162.74689012,380.59031497)
\curveto(162.61688621,380.56030825)(162.49188634,380.52530829)(162.37189012,380.48531497)
\curveto(162.26188657,380.45530836)(162.15688667,380.4103084)(162.05689012,380.35031497)
\curveto(162.01688681,380.33030848)(161.98188685,380.3103085)(161.95189012,380.29031497)
\curveto(161.92188691,380.27030854)(161.88688694,380.25030856)(161.84689012,380.23031497)
\curveto(161.49688733,379.98030883)(161.24188759,379.60530921)(161.08189012,379.10531497)
\curveto(161.05188778,379.02530979)(161.0318878,378.94030987)(161.02189012,378.85031497)
\curveto(161.01188782,378.77031004)(160.99688783,378.69031012)(160.97689012,378.61031497)
\curveto(160.95688787,378.56031025)(160.95188788,378.5103103)(160.96189012,378.46031497)
\curveto(160.97188786,378.42031039)(160.96688786,378.38031043)(160.94689012,378.34031497)
\lineto(160.94689012,378.02531497)
\curveto(160.93688789,377.99531082)(160.9318879,377.96031085)(160.93189012,377.92031497)
\curveto(160.94188789,377.88031093)(160.94688788,377.83531098)(160.94689012,377.78531497)
\lineto(160.94689012,377.33531497)
\lineto(160.94689012,375.89531497)
\lineto(160.94689012,374.57531497)
\lineto(160.94689012,374.23031497)
\curveto(160.94688788,374.12031469)(160.92188791,374.03031478)(160.87189012,373.96031497)
\curveto(160.82188801,373.88031493)(160.7318881,373.84031497)(160.60189012,373.84031497)
\curveto(160.48188835,373.83031498)(160.35688847,373.82531499)(160.22689012,373.82531497)
\curveto(160.14688868,373.82531499)(160.07188876,373.83031498)(160.00189012,373.84031497)
\curveto(159.9318889,373.85031496)(159.87188896,373.87531494)(159.82189012,373.91531497)
\curveto(159.74188909,373.96531485)(159.70188913,374.06031475)(159.70189012,374.20031497)
\lineto(159.70189012,374.60531497)
\lineto(159.70189012,376.37531497)
\lineto(159.70189012,380.00531497)
\lineto(159.70189012,380.92031497)
\lineto(159.70189012,381.19031497)
\curveto(159.70188913,381.28030753)(159.72188911,381.35030746)(159.76189012,381.40031497)
\curveto(159.79188904,381.46030735)(159.84188899,381.50030731)(159.91189012,381.52031497)
\curveto(159.95188888,381.53030728)(160.00688882,381.54030727)(160.07689012,381.55031497)
\curveto(160.15688867,381.56030725)(160.23688859,381.56530725)(160.31689012,381.56531497)
\curveto(160.39688843,381.56530725)(160.47188836,381.56030725)(160.54189012,381.55031497)
\curveto(160.62188821,381.54030727)(160.67688815,381.52530729)(160.70689012,381.50531497)
\curveto(160.81688801,381.43530738)(160.86688796,381.34530747)(160.85689012,381.23531497)
\curveto(160.84688798,381.13530768)(160.86188797,381.02030779)(160.90189012,380.89031497)
\curveto(160.92188791,380.83030798)(160.96188787,380.78030803)(161.02189012,380.74031497)
\curveto(161.14188769,380.73030808)(161.23688759,380.77530804)(161.30689012,380.87531497)
\curveto(161.38688744,380.97530784)(161.46688736,381.05530776)(161.54689012,381.11531497)
\curveto(161.68688714,381.2153076)(161.826887,381.30530751)(161.96689012,381.38531497)
\curveto(162.11688671,381.47530734)(162.28688654,381.55030726)(162.47689012,381.61031497)
\curveto(162.55688627,381.64030717)(162.64188619,381.66030715)(162.73189012,381.67031497)
\curveto(162.831886,381.68030713)(162.9268859,381.69530712)(163.01689012,381.71531497)
\curveto(163.06688576,381.72530709)(163.11688571,381.73030708)(163.16689012,381.73031497)
\lineto(163.31689012,381.73031497)
}
}
{
\newrgbcolor{curcolor}{0 0 0}
\pscustom[linestyle=none,fillstyle=solid,fillcolor=curcolor]
{
\newpath
\moveto(168.92149949,383.92031497)
\curveto(169.07149748,383.92030489)(169.22149733,383.9153049)(169.37149949,383.90531497)
\curveto(169.52149703,383.90530491)(169.62649693,383.86530495)(169.68649949,383.78531497)
\curveto(169.73649682,383.72530509)(169.76149679,383.64030517)(169.76149949,383.53031497)
\curveto(169.77149678,383.43030538)(169.77649678,383.32530549)(169.77649949,383.21531497)
\lineto(169.77649949,382.34531497)
\curveto(169.77649678,382.26530655)(169.77149678,382.18030663)(169.76149949,382.09031497)
\curveto(169.76149679,382.0103068)(169.77149678,381.94030687)(169.79149949,381.88031497)
\curveto(169.83149672,381.74030707)(169.92149663,381.65030716)(170.06149949,381.61031497)
\curveto(170.11149644,381.60030721)(170.1564964,381.59530722)(170.19649949,381.59531497)
\lineto(170.34649949,381.59531497)
\lineto(170.75149949,381.59531497)
\curveto(170.91149564,381.60530721)(171.02649553,381.59530722)(171.09649949,381.56531497)
\curveto(171.18649537,381.50530731)(171.24649531,381.44530737)(171.27649949,381.38531497)
\curveto(171.29649526,381.34530747)(171.30649525,381.30030751)(171.30649949,381.25031497)
\lineto(171.30649949,381.10031497)
\curveto(171.30649525,380.99030782)(171.30149525,380.88530793)(171.29149949,380.78531497)
\curveto(171.28149527,380.69530812)(171.24649531,380.62530819)(171.18649949,380.57531497)
\curveto(171.12649543,380.52530829)(171.04149551,380.49530832)(170.93149949,380.48531497)
\lineto(170.60149949,380.48531497)
\curveto(170.49149606,380.49530832)(170.38149617,380.50030831)(170.27149949,380.50031497)
\curveto(170.16149639,380.50030831)(170.06649649,380.48530833)(169.98649949,380.45531497)
\curveto(169.91649664,380.42530839)(169.86649669,380.37530844)(169.83649949,380.30531497)
\curveto(169.80649675,380.23530858)(169.78649677,380.15030866)(169.77649949,380.05031497)
\curveto(169.76649679,379.96030885)(169.76149679,379.86030895)(169.76149949,379.75031497)
\curveto(169.77149678,379.65030916)(169.77649678,379.55030926)(169.77649949,379.45031497)
\lineto(169.77649949,376.48031497)
\curveto(169.77649678,376.26031255)(169.77149678,376.02531279)(169.76149949,375.77531497)
\curveto(169.76149679,375.53531328)(169.80649675,375.35031346)(169.89649949,375.22031497)
\curveto(169.94649661,375.14031367)(170.01149654,375.08531373)(170.09149949,375.05531497)
\curveto(170.17149638,375.02531379)(170.26649629,375.00031381)(170.37649949,374.98031497)
\curveto(170.40649615,374.97031384)(170.43649612,374.96531385)(170.46649949,374.96531497)
\curveto(170.50649605,374.97531384)(170.54149601,374.97531384)(170.57149949,374.96531497)
\lineto(170.76649949,374.96531497)
\curveto(170.86649569,374.96531385)(170.9564956,374.95531386)(171.03649949,374.93531497)
\curveto(171.12649543,374.92531389)(171.19149536,374.89031392)(171.23149949,374.83031497)
\curveto(171.2514953,374.80031401)(171.26649529,374.74531407)(171.27649949,374.66531497)
\curveto(171.29649526,374.59531422)(171.30649525,374.52031429)(171.30649949,374.44031497)
\curveto(171.31649524,374.36031445)(171.31649524,374.28031453)(171.30649949,374.20031497)
\curveto(171.29649526,374.13031468)(171.27649528,374.07531474)(171.24649949,374.03531497)
\curveto(171.20649535,373.96531485)(171.13149542,373.9153149)(171.02149949,373.88531497)
\curveto(170.94149561,373.86531495)(170.8514957,373.85531496)(170.75149949,373.85531497)
\curveto(170.6514959,373.86531495)(170.56149599,373.87031494)(170.48149949,373.87031497)
\curveto(170.42149613,373.87031494)(170.36149619,373.86531495)(170.30149949,373.85531497)
\curveto(170.24149631,373.85531496)(170.18649637,373.86031495)(170.13649949,373.87031497)
\lineto(169.95649949,373.87031497)
\curveto(169.90649665,373.88031493)(169.8564967,373.88531493)(169.80649949,373.88531497)
\curveto(169.76649679,373.89531492)(169.72149683,373.90031491)(169.67149949,373.90031497)
\curveto(169.47149708,373.95031486)(169.29649726,374.00531481)(169.14649949,374.06531497)
\curveto(169.00649755,374.12531469)(168.88649767,374.23031458)(168.78649949,374.38031497)
\curveto(168.64649791,374.58031423)(168.56649799,374.83031398)(168.54649949,375.13031497)
\curveto(168.52649803,375.44031337)(168.51649804,375.77031304)(168.51649949,376.12031497)
\lineto(168.51649949,380.05031497)
\curveto(168.48649807,380.18030863)(168.4564981,380.27530854)(168.42649949,380.33531497)
\curveto(168.40649815,380.39530842)(168.33649822,380.44530837)(168.21649949,380.48531497)
\curveto(168.17649838,380.49530832)(168.13649842,380.49530832)(168.09649949,380.48531497)
\curveto(168.0564985,380.47530834)(168.01649854,380.48030833)(167.97649949,380.50031497)
\lineto(167.73649949,380.50031497)
\curveto(167.60649895,380.50030831)(167.49649906,380.5103083)(167.40649949,380.53031497)
\curveto(167.32649923,380.56030825)(167.27149928,380.62030819)(167.24149949,380.71031497)
\curveto(167.22149933,380.75030806)(167.20649935,380.79530802)(167.19649949,380.84531497)
\lineto(167.19649949,380.99531497)
\curveto(167.19649936,381.13530768)(167.20649935,381.25030756)(167.22649949,381.34031497)
\curveto(167.24649931,381.44030737)(167.30649925,381.5153073)(167.40649949,381.56531497)
\curveto(167.51649904,381.60530721)(167.6564989,381.6153072)(167.82649949,381.59531497)
\curveto(168.00649855,381.57530724)(168.1564984,381.58530723)(168.27649949,381.62531497)
\curveto(168.36649819,381.67530714)(168.43649812,381.74530707)(168.48649949,381.83531497)
\curveto(168.50649805,381.89530692)(168.51649804,381.97030684)(168.51649949,382.06031497)
\lineto(168.51649949,382.31531497)
\lineto(168.51649949,383.24531497)
\lineto(168.51649949,383.48531497)
\curveto(168.51649804,383.57530524)(168.52649803,383.65030516)(168.54649949,383.71031497)
\curveto(168.58649797,383.79030502)(168.66149789,383.85530496)(168.77149949,383.90531497)
\curveto(168.80149775,383.90530491)(168.82649773,383.90530491)(168.84649949,383.90531497)
\curveto(168.87649768,383.9153049)(168.90149765,383.92030489)(168.92149949,383.92031497)
}
}
{
\newrgbcolor{curcolor}{0 0 0}
\pscustom[linestyle=none,fillstyle=solid,fillcolor=curcolor]
{
\newpath
\moveto(179.57829637,374.41031497)
\curveto(179.60828854,374.25031456)(179.59328855,374.1153147)(179.53329637,374.00531497)
\curveto(179.47328867,373.90531491)(179.39328875,373.83031498)(179.29329637,373.78031497)
\curveto(179.2432889,373.76031505)(179.18828896,373.75031506)(179.12829637,373.75031497)
\curveto(179.07828907,373.75031506)(179.02328912,373.74031507)(178.96329637,373.72031497)
\curveto(178.7432894,373.67031514)(178.52328962,373.68531513)(178.30329637,373.76531497)
\curveto(178.09329005,373.83531498)(177.9482902,373.92531489)(177.86829637,374.03531497)
\curveto(177.81829033,374.10531471)(177.77329037,374.18531463)(177.73329637,374.27531497)
\curveto(177.69329045,374.37531444)(177.6432905,374.45531436)(177.58329637,374.51531497)
\curveto(177.56329058,374.53531428)(177.53829061,374.55531426)(177.50829637,374.57531497)
\curveto(177.48829066,374.59531422)(177.45829069,374.60031421)(177.41829637,374.59031497)
\curveto(177.30829084,374.56031425)(177.20329094,374.50531431)(177.10329637,374.42531497)
\curveto(177.01329113,374.34531447)(176.92329122,374.27531454)(176.83329637,374.21531497)
\curveto(176.70329144,374.13531468)(176.56329158,374.06031475)(176.41329637,373.99031497)
\curveto(176.26329188,373.93031488)(176.10329204,373.87531494)(175.93329637,373.82531497)
\curveto(175.83329231,373.79531502)(175.72329242,373.77531504)(175.60329637,373.76531497)
\curveto(175.49329265,373.75531506)(175.38329276,373.74031507)(175.27329637,373.72031497)
\curveto(175.22329292,373.7103151)(175.17829297,373.70531511)(175.13829637,373.70531497)
\lineto(175.03329637,373.70531497)
\curveto(174.92329322,373.68531513)(174.81829333,373.68531513)(174.71829637,373.70531497)
\lineto(174.58329637,373.70531497)
\curveto(174.53329361,373.7153151)(174.48329366,373.72031509)(174.43329637,373.72031497)
\curveto(174.38329376,373.72031509)(174.33829381,373.73031508)(174.29829637,373.75031497)
\curveto(174.25829389,373.76031505)(174.22329392,373.76531505)(174.19329637,373.76531497)
\curveto(174.17329397,373.75531506)(174.148294,373.75531506)(174.11829637,373.76531497)
\lineto(173.87829637,373.82531497)
\curveto(173.79829435,373.83531498)(173.72329442,373.85531496)(173.65329637,373.88531497)
\curveto(173.35329479,374.0153148)(173.10829504,374.16031465)(172.91829637,374.32031497)
\curveto(172.73829541,374.49031432)(172.58829556,374.72531409)(172.46829637,375.02531497)
\curveto(172.37829577,375.24531357)(172.33329581,375.5103133)(172.33329637,375.82031497)
\lineto(172.33329637,376.13531497)
\curveto(172.3432958,376.18531263)(172.3482958,376.23531258)(172.34829637,376.28531497)
\lineto(172.37829637,376.46531497)
\lineto(172.49829637,376.79531497)
\curveto(172.53829561,376.90531191)(172.58829556,377.00531181)(172.64829637,377.09531497)
\curveto(172.82829532,377.38531143)(173.07329507,377.60031121)(173.38329637,377.74031497)
\curveto(173.69329445,377.88031093)(174.03329411,378.00531081)(174.40329637,378.11531497)
\curveto(174.5432936,378.15531066)(174.68829346,378.18531063)(174.83829637,378.20531497)
\curveto(174.98829316,378.22531059)(175.13829301,378.25031056)(175.28829637,378.28031497)
\curveto(175.35829279,378.30031051)(175.42329272,378.3103105)(175.48329637,378.31031497)
\curveto(175.55329259,378.3103105)(175.62829252,378.32031049)(175.70829637,378.34031497)
\curveto(175.77829237,378.36031045)(175.8482923,378.37031044)(175.91829637,378.37031497)
\curveto(175.98829216,378.38031043)(176.06329208,378.39531042)(176.14329637,378.41531497)
\curveto(176.39329175,378.47531034)(176.62829152,378.52531029)(176.84829637,378.56531497)
\curveto(177.06829108,378.6153102)(177.2432909,378.73031008)(177.37329637,378.91031497)
\curveto(177.43329071,378.99030982)(177.48329066,379.09030972)(177.52329637,379.21031497)
\curveto(177.56329058,379.34030947)(177.56329058,379.48030933)(177.52329637,379.63031497)
\curveto(177.46329068,379.87030894)(177.37329077,380.06030875)(177.25329637,380.20031497)
\curveto(177.143291,380.34030847)(176.98329116,380.45030836)(176.77329637,380.53031497)
\curveto(176.65329149,380.58030823)(176.50829164,380.6153082)(176.33829637,380.63531497)
\curveto(176.17829197,380.65530816)(176.00829214,380.66530815)(175.82829637,380.66531497)
\curveto(175.6482925,380.66530815)(175.47329267,380.65530816)(175.30329637,380.63531497)
\curveto(175.13329301,380.6153082)(174.98829316,380.58530823)(174.86829637,380.54531497)
\curveto(174.69829345,380.48530833)(174.53329361,380.40030841)(174.37329637,380.29031497)
\curveto(174.29329385,380.23030858)(174.21829393,380.15030866)(174.14829637,380.05031497)
\curveto(174.08829406,379.96030885)(174.03329411,379.86030895)(173.98329637,379.75031497)
\curveto(173.95329419,379.67030914)(173.92329422,379.58530923)(173.89329637,379.49531497)
\curveto(173.87329427,379.40530941)(173.82829432,379.33530948)(173.75829637,379.28531497)
\curveto(173.71829443,379.25530956)(173.6482945,379.23030958)(173.54829637,379.21031497)
\curveto(173.45829469,379.20030961)(173.36329478,379.19530962)(173.26329637,379.19531497)
\curveto(173.16329498,379.19530962)(173.06329508,379.20030961)(172.96329637,379.21031497)
\curveto(172.87329527,379.23030958)(172.80829534,379.25530956)(172.76829637,379.28531497)
\curveto(172.72829542,379.3153095)(172.69829545,379.36530945)(172.67829637,379.43531497)
\curveto(172.65829549,379.50530931)(172.65829549,379.58030923)(172.67829637,379.66031497)
\curveto(172.70829544,379.79030902)(172.73829541,379.9103089)(172.76829637,380.02031497)
\curveto(172.80829534,380.14030867)(172.85329529,380.25530856)(172.90329637,380.36531497)
\curveto(173.09329505,380.7153081)(173.33329481,380.98530783)(173.62329637,381.17531497)
\curveto(173.91329423,381.37530744)(174.27329387,381.53530728)(174.70329637,381.65531497)
\curveto(174.80329334,381.67530714)(174.90329324,381.69030712)(175.00329637,381.70031497)
\curveto(175.11329303,381.7103071)(175.22329292,381.72530709)(175.33329637,381.74531497)
\curveto(175.37329277,381.75530706)(175.43829271,381.75530706)(175.52829637,381.74531497)
\curveto(175.61829253,381.74530707)(175.67329247,381.75530706)(175.69329637,381.77531497)
\curveto(176.39329175,381.78530703)(177.00329114,381.70530711)(177.52329637,381.53531497)
\curveto(178.0432901,381.36530745)(178.40828974,381.04030777)(178.61829637,380.56031497)
\curveto(178.70828944,380.36030845)(178.75828939,380.12530869)(178.76829637,379.85531497)
\curveto(178.78828936,379.59530922)(178.79828935,379.32030949)(178.79829637,379.03031497)
\lineto(178.79829637,375.71531497)
\curveto(178.79828935,375.57531324)(178.80328934,375.44031337)(178.81329637,375.31031497)
\curveto(178.82328932,375.18031363)(178.85328929,375.07531374)(178.90329637,374.99531497)
\curveto(178.95328919,374.92531389)(179.01828913,374.87531394)(179.09829637,374.84531497)
\curveto(179.18828896,374.80531401)(179.27328887,374.77531404)(179.35329637,374.75531497)
\curveto(179.43328871,374.74531407)(179.49328865,374.70031411)(179.53329637,374.62031497)
\curveto(179.55328859,374.59031422)(179.56328858,374.56031425)(179.56329637,374.53031497)
\curveto(179.56328858,374.50031431)(179.56828858,374.46031435)(179.57829637,374.41031497)
\moveto(177.43329637,376.07531497)
\curveto(177.49329065,376.2153126)(177.52329062,376.37531244)(177.52329637,376.55531497)
\curveto(177.53329061,376.74531207)(177.53829061,376.94031187)(177.53829637,377.14031497)
\curveto(177.53829061,377.25031156)(177.53329061,377.35031146)(177.52329637,377.44031497)
\curveto(177.51329063,377.53031128)(177.47329067,377.60031121)(177.40329637,377.65031497)
\curveto(177.37329077,377.67031114)(177.30329084,377.68031113)(177.19329637,377.68031497)
\curveto(177.17329097,377.66031115)(177.13829101,377.65031116)(177.08829637,377.65031497)
\curveto(177.03829111,377.65031116)(176.99329115,377.64031117)(176.95329637,377.62031497)
\curveto(176.87329127,377.60031121)(176.78329136,377.58031123)(176.68329637,377.56031497)
\lineto(176.38329637,377.50031497)
\curveto(176.35329179,377.50031131)(176.31829183,377.49531132)(176.27829637,377.48531497)
\lineto(176.17329637,377.48531497)
\curveto(176.02329212,377.44531137)(175.85829229,377.42031139)(175.67829637,377.41031497)
\curveto(175.50829264,377.4103114)(175.3482928,377.39031142)(175.19829637,377.35031497)
\curveto(175.11829303,377.33031148)(175.0432931,377.3103115)(174.97329637,377.29031497)
\curveto(174.91329323,377.28031153)(174.8432933,377.26531155)(174.76329637,377.24531497)
\curveto(174.60329354,377.19531162)(174.45329369,377.13031168)(174.31329637,377.05031497)
\curveto(174.17329397,376.98031183)(174.05329409,376.89031192)(173.95329637,376.78031497)
\curveto(173.85329429,376.67031214)(173.77829437,376.53531228)(173.72829637,376.37531497)
\curveto(173.67829447,376.22531259)(173.65829449,376.04031277)(173.66829637,375.82031497)
\curveto(173.66829448,375.72031309)(173.68329446,375.62531319)(173.71329637,375.53531497)
\curveto(173.75329439,375.45531336)(173.79829435,375.38031343)(173.84829637,375.31031497)
\curveto(173.92829422,375.20031361)(174.03329411,375.10531371)(174.16329637,375.02531497)
\curveto(174.29329385,374.95531386)(174.43329371,374.89531392)(174.58329637,374.84531497)
\curveto(174.63329351,374.83531398)(174.68329346,374.83031398)(174.73329637,374.83031497)
\curveto(174.78329336,374.83031398)(174.83329331,374.82531399)(174.88329637,374.81531497)
\curveto(174.95329319,374.79531402)(175.03829311,374.78031403)(175.13829637,374.77031497)
\curveto(175.2482929,374.77031404)(175.33829281,374.78031403)(175.40829637,374.80031497)
\curveto(175.46829268,374.82031399)(175.52829262,374.82531399)(175.58829637,374.81531497)
\curveto(175.6482925,374.815314)(175.70829244,374.82531399)(175.76829637,374.84531497)
\curveto(175.8482923,374.86531395)(175.92329222,374.88031393)(175.99329637,374.89031497)
\curveto(176.07329207,374.90031391)(176.148292,374.92031389)(176.21829637,374.95031497)
\curveto(176.50829164,375.07031374)(176.75329139,375.2153136)(176.95329637,375.38531497)
\curveto(177.16329098,375.55531326)(177.32329082,375.78531303)(177.43329637,376.07531497)
}
}
{
\newrgbcolor{curcolor}{0 0 0}
\pscustom[linestyle=none,fillstyle=solid,fillcolor=curcolor]
{
\newpath
\moveto(183.88493699,381.76031497)
\curveto(184.6249322,381.77030704)(185.23993159,381.66030715)(185.72993699,381.43031497)
\curveto(186.2299306,381.2103076)(186.6249302,380.87530794)(186.91493699,380.42531497)
\curveto(187.04492978,380.22530859)(187.15492967,379.98030883)(187.24493699,379.69031497)
\curveto(187.26492956,379.64030917)(187.27992955,379.57530924)(187.28993699,379.49531497)
\curveto(187.29992953,379.4153094)(187.29492953,379.34530947)(187.27493699,379.28531497)
\curveto(187.24492958,379.23530958)(187.19492963,379.19030962)(187.12493699,379.15031497)
\curveto(187.09492973,379.13030968)(187.06492976,379.12030969)(187.03493699,379.12031497)
\curveto(187.00492982,379.13030968)(186.96992986,379.13030968)(186.92993699,379.12031497)
\curveto(186.88992994,379.1103097)(186.84992998,379.10530971)(186.80993699,379.10531497)
\curveto(186.76993006,379.1153097)(186.7299301,379.12030969)(186.68993699,379.12031497)
\lineto(186.37493699,379.12031497)
\curveto(186.27493055,379.13030968)(186.18993064,379.16030965)(186.11993699,379.21031497)
\curveto(186.03993079,379.27030954)(185.98493084,379.35530946)(185.95493699,379.46531497)
\curveto(185.9249309,379.57530924)(185.88493094,379.67030914)(185.83493699,379.75031497)
\curveto(185.68493114,380.0103088)(185.48993134,380.2153086)(185.24993699,380.36531497)
\curveto(185.16993166,380.4153084)(185.08493174,380.45530836)(184.99493699,380.48531497)
\curveto(184.90493192,380.52530829)(184.80993202,380.56030825)(184.70993699,380.59031497)
\curveto(184.56993226,380.63030818)(184.38493244,380.65030816)(184.15493699,380.65031497)
\curveto(183.9249329,380.66030815)(183.73493309,380.64030817)(183.58493699,380.59031497)
\curveto(183.51493331,380.57030824)(183.44993338,380.55530826)(183.38993699,380.54531497)
\curveto(183.3299335,380.53530828)(183.26493356,380.52030829)(183.19493699,380.50031497)
\curveto(182.93493389,380.39030842)(182.70493412,380.24030857)(182.50493699,380.05031497)
\curveto(182.30493452,379.86030895)(182.14993468,379.63530918)(182.03993699,379.37531497)
\curveto(181.99993483,379.28530953)(181.96493486,379.19030962)(181.93493699,379.09031497)
\curveto(181.90493492,379.00030981)(181.87493495,378.90030991)(181.84493699,378.79031497)
\lineto(181.75493699,378.38531497)
\curveto(181.74493508,378.33531048)(181.73993509,378.28031053)(181.73993699,378.22031497)
\curveto(181.74993508,378.16031065)(181.74493508,378.10531071)(181.72493699,378.05531497)
\lineto(181.72493699,377.93531497)
\curveto(181.71493511,377.89531092)(181.70493512,377.83031098)(181.69493699,377.74031497)
\curveto(181.69493513,377.65031116)(181.70493512,377.58531123)(181.72493699,377.54531497)
\curveto(181.73493509,377.49531132)(181.73493509,377.44531137)(181.72493699,377.39531497)
\curveto(181.71493511,377.34531147)(181.71493511,377.29531152)(181.72493699,377.24531497)
\curveto(181.73493509,377.20531161)(181.73993509,377.13531168)(181.73993699,377.03531497)
\curveto(181.75993507,376.95531186)(181.77493505,376.87031194)(181.78493699,376.78031497)
\curveto(181.80493502,376.69031212)(181.824935,376.60531221)(181.84493699,376.52531497)
\curveto(181.95493487,376.20531261)(182.07993475,375.92531289)(182.21993699,375.68531497)
\curveto(182.36993446,375.45531336)(182.57493425,375.25531356)(182.83493699,375.08531497)
\curveto(182.9249339,375.03531378)(183.01493381,374.99031382)(183.10493699,374.95031497)
\curveto(183.20493362,374.9103139)(183.30993352,374.87031394)(183.41993699,374.83031497)
\curveto(183.46993336,374.82031399)(183.50993332,374.815314)(183.53993699,374.81531497)
\curveto(183.56993326,374.815314)(183.60993322,374.810314)(183.65993699,374.80031497)
\curveto(183.68993314,374.79031402)(183.73993309,374.78531403)(183.80993699,374.78531497)
\lineto(183.97493699,374.78531497)
\curveto(183.97493285,374.77531404)(183.99493283,374.77031404)(184.03493699,374.77031497)
\curveto(184.05493277,374.78031403)(184.07993275,374.78031403)(184.10993699,374.77031497)
\curveto(184.13993269,374.77031404)(184.16993266,374.77531404)(184.19993699,374.78531497)
\curveto(184.26993256,374.80531401)(184.33493249,374.810314)(184.39493699,374.80031497)
\curveto(184.46493236,374.80031401)(184.53493229,374.810314)(184.60493699,374.83031497)
\curveto(184.86493196,374.9103139)(185.08993174,375.0103138)(185.27993699,375.13031497)
\curveto(185.46993136,375.26031355)(185.6299312,375.42531339)(185.75993699,375.62531497)
\curveto(185.80993102,375.70531311)(185.85493097,375.79031302)(185.89493699,375.88031497)
\lineto(186.01493699,376.15031497)
\curveto(186.03493079,376.23031258)(186.05493077,376.30531251)(186.07493699,376.37531497)
\curveto(186.10493072,376.45531236)(186.15493067,376.52031229)(186.22493699,376.57031497)
\curveto(186.25493057,376.60031221)(186.31493051,376.62031219)(186.40493699,376.63031497)
\curveto(186.49493033,376.65031216)(186.58993024,376.66031215)(186.68993699,376.66031497)
\curveto(186.79993003,376.67031214)(186.89992993,376.67031214)(186.98993699,376.66031497)
\curveto(187.08992974,376.65031216)(187.15992967,376.63031218)(187.19993699,376.60031497)
\curveto(187.25992957,376.56031225)(187.29492953,376.50031231)(187.30493699,376.42031497)
\curveto(187.3249295,376.34031247)(187.3249295,376.25531256)(187.30493699,376.16531497)
\curveto(187.25492957,376.0153128)(187.20492962,375.87031294)(187.15493699,375.73031497)
\curveto(187.11492971,375.60031321)(187.05992977,375.47031334)(186.98993699,375.34031497)
\curveto(186.83992999,375.04031377)(186.64993018,374.77531404)(186.41993699,374.54531497)
\curveto(186.19993063,374.3153145)(185.9299309,374.13031468)(185.60993699,373.99031497)
\curveto(185.5299313,373.95031486)(185.44493138,373.9153149)(185.35493699,373.88531497)
\curveto(185.26493156,373.86531495)(185.16993166,373.84031497)(185.06993699,373.81031497)
\curveto(184.95993187,373.77031504)(184.84993198,373.75031506)(184.73993699,373.75031497)
\curveto(184.6299322,373.74031507)(184.51993231,373.72531509)(184.40993699,373.70531497)
\curveto(184.36993246,373.68531513)(184.3299325,373.68031513)(184.28993699,373.69031497)
\curveto(184.24993258,373.70031511)(184.20993262,373.70031511)(184.16993699,373.69031497)
\lineto(184.03493699,373.69031497)
\lineto(183.79493699,373.69031497)
\curveto(183.7249331,373.68031513)(183.65993317,373.68531513)(183.59993699,373.70531497)
\lineto(183.52493699,373.70531497)
\lineto(183.16493699,373.75031497)
\curveto(183.03493379,373.79031502)(182.90993392,373.82531499)(182.78993699,373.85531497)
\curveto(182.66993416,373.88531493)(182.55493427,373.92531489)(182.44493699,373.97531497)
\curveto(182.08493474,374.13531468)(181.78493504,374.32531449)(181.54493699,374.54531497)
\curveto(181.31493551,374.76531405)(181.09993573,375.03531378)(180.89993699,375.35531497)
\curveto(180.84993598,375.43531338)(180.80493602,375.52531329)(180.76493699,375.62531497)
\lineto(180.64493699,375.92531497)
\curveto(180.59493623,376.03531278)(180.55993627,376.15031266)(180.53993699,376.27031497)
\curveto(180.51993631,376.39031242)(180.49493633,376.5103123)(180.46493699,376.63031497)
\curveto(180.45493637,376.67031214)(180.44993638,376.7103121)(180.44993699,376.75031497)
\curveto(180.44993638,376.79031202)(180.44493638,376.83031198)(180.43493699,376.87031497)
\curveto(180.41493641,376.93031188)(180.40493642,376.99531182)(180.40493699,377.06531497)
\curveto(180.41493641,377.13531168)(180.40993642,377.20031161)(180.38993699,377.26031497)
\lineto(180.38993699,377.41031497)
\curveto(180.37993645,377.46031135)(180.37493645,377.53031128)(180.37493699,377.62031497)
\curveto(180.37493645,377.7103111)(180.37993645,377.78031103)(180.38993699,377.83031497)
\curveto(180.39993643,377.88031093)(180.39993643,377.92531089)(180.38993699,377.96531497)
\curveto(180.38993644,378.00531081)(180.39493643,378.04531077)(180.40493699,378.08531497)
\curveto(180.4249364,378.15531066)(180.4299364,378.22531059)(180.41993699,378.29531497)
\curveto(180.41993641,378.36531045)(180.4299364,378.43031038)(180.44993699,378.49031497)
\curveto(180.48993634,378.66031015)(180.5249363,378.83030998)(180.55493699,379.00031497)
\curveto(180.58493624,379.17030964)(180.6299362,379.33030948)(180.68993699,379.48031497)
\curveto(180.89993593,380.00030881)(181.15493567,380.42030839)(181.45493699,380.74031497)
\curveto(181.75493507,381.06030775)(182.16493466,381.32530749)(182.68493699,381.53531497)
\curveto(182.79493403,381.58530723)(182.91493391,381.62030719)(183.04493699,381.64031497)
\curveto(183.17493365,381.66030715)(183.30993352,381.68530713)(183.44993699,381.71531497)
\curveto(183.51993331,381.72530709)(183.58993324,381.73030708)(183.65993699,381.73031497)
\curveto(183.7299331,381.74030707)(183.80493302,381.75030706)(183.88493699,381.76031497)
}
}
{
\newrgbcolor{curcolor}{0 0 0}
\pscustom[linestyle=none,fillstyle=solid,fillcolor=curcolor]
{
\newpath
\moveto(189.75157762,383.92031497)
\curveto(189.90157561,383.92030489)(190.05157546,383.9153049)(190.20157762,383.90531497)
\curveto(190.35157516,383.90530491)(190.45657505,383.86530495)(190.51657762,383.78531497)
\curveto(190.56657494,383.72530509)(190.59157492,383.64030517)(190.59157762,383.53031497)
\curveto(190.60157491,383.43030538)(190.6065749,383.32530549)(190.60657762,383.21531497)
\lineto(190.60657762,382.34531497)
\curveto(190.6065749,382.26530655)(190.60157491,382.18030663)(190.59157762,382.09031497)
\curveto(190.59157492,382.0103068)(190.60157491,381.94030687)(190.62157762,381.88031497)
\curveto(190.66157485,381.74030707)(190.75157476,381.65030716)(190.89157762,381.61031497)
\curveto(190.94157457,381.60030721)(190.98657452,381.59530722)(191.02657762,381.59531497)
\lineto(191.17657762,381.59531497)
\lineto(191.58157762,381.59531497)
\curveto(191.74157377,381.60530721)(191.85657365,381.59530722)(191.92657762,381.56531497)
\curveto(192.01657349,381.50530731)(192.07657343,381.44530737)(192.10657762,381.38531497)
\curveto(192.12657338,381.34530747)(192.13657337,381.30030751)(192.13657762,381.25031497)
\lineto(192.13657762,381.10031497)
\curveto(192.13657337,380.99030782)(192.13157338,380.88530793)(192.12157762,380.78531497)
\curveto(192.1115734,380.69530812)(192.07657343,380.62530819)(192.01657762,380.57531497)
\curveto(191.95657355,380.52530829)(191.87157364,380.49530832)(191.76157762,380.48531497)
\lineto(191.43157762,380.48531497)
\curveto(191.32157419,380.49530832)(191.2115743,380.50030831)(191.10157762,380.50031497)
\curveto(190.99157452,380.50030831)(190.89657461,380.48530833)(190.81657762,380.45531497)
\curveto(190.74657476,380.42530839)(190.69657481,380.37530844)(190.66657762,380.30531497)
\curveto(190.63657487,380.23530858)(190.61657489,380.15030866)(190.60657762,380.05031497)
\curveto(190.59657491,379.96030885)(190.59157492,379.86030895)(190.59157762,379.75031497)
\curveto(190.60157491,379.65030916)(190.6065749,379.55030926)(190.60657762,379.45031497)
\lineto(190.60657762,376.48031497)
\curveto(190.6065749,376.26031255)(190.60157491,376.02531279)(190.59157762,375.77531497)
\curveto(190.59157492,375.53531328)(190.63657487,375.35031346)(190.72657762,375.22031497)
\curveto(190.77657473,375.14031367)(190.84157467,375.08531373)(190.92157762,375.05531497)
\curveto(191.00157451,375.02531379)(191.09657441,375.00031381)(191.20657762,374.98031497)
\curveto(191.23657427,374.97031384)(191.26657424,374.96531385)(191.29657762,374.96531497)
\curveto(191.33657417,374.97531384)(191.37157414,374.97531384)(191.40157762,374.96531497)
\lineto(191.59657762,374.96531497)
\curveto(191.69657381,374.96531385)(191.78657372,374.95531386)(191.86657762,374.93531497)
\curveto(191.95657355,374.92531389)(192.02157349,374.89031392)(192.06157762,374.83031497)
\curveto(192.08157343,374.80031401)(192.09657341,374.74531407)(192.10657762,374.66531497)
\curveto(192.12657338,374.59531422)(192.13657337,374.52031429)(192.13657762,374.44031497)
\curveto(192.14657336,374.36031445)(192.14657336,374.28031453)(192.13657762,374.20031497)
\curveto(192.12657338,374.13031468)(192.1065734,374.07531474)(192.07657762,374.03531497)
\curveto(192.03657347,373.96531485)(191.96157355,373.9153149)(191.85157762,373.88531497)
\curveto(191.77157374,373.86531495)(191.68157383,373.85531496)(191.58157762,373.85531497)
\curveto(191.48157403,373.86531495)(191.39157412,373.87031494)(191.31157762,373.87031497)
\curveto(191.25157426,373.87031494)(191.19157432,373.86531495)(191.13157762,373.85531497)
\curveto(191.07157444,373.85531496)(191.01657449,373.86031495)(190.96657762,373.87031497)
\lineto(190.78657762,373.87031497)
\curveto(190.73657477,373.88031493)(190.68657482,373.88531493)(190.63657762,373.88531497)
\curveto(190.59657491,373.89531492)(190.55157496,373.90031491)(190.50157762,373.90031497)
\curveto(190.30157521,373.95031486)(190.12657538,374.00531481)(189.97657762,374.06531497)
\curveto(189.83657567,374.12531469)(189.71657579,374.23031458)(189.61657762,374.38031497)
\curveto(189.47657603,374.58031423)(189.39657611,374.83031398)(189.37657762,375.13031497)
\curveto(189.35657615,375.44031337)(189.34657616,375.77031304)(189.34657762,376.12031497)
\lineto(189.34657762,380.05031497)
\curveto(189.31657619,380.18030863)(189.28657622,380.27530854)(189.25657762,380.33531497)
\curveto(189.23657627,380.39530842)(189.16657634,380.44530837)(189.04657762,380.48531497)
\curveto(189.0065765,380.49530832)(188.96657654,380.49530832)(188.92657762,380.48531497)
\curveto(188.88657662,380.47530834)(188.84657666,380.48030833)(188.80657762,380.50031497)
\lineto(188.56657762,380.50031497)
\curveto(188.43657707,380.50030831)(188.32657718,380.5103083)(188.23657762,380.53031497)
\curveto(188.15657735,380.56030825)(188.10157741,380.62030819)(188.07157762,380.71031497)
\curveto(188.05157746,380.75030806)(188.03657747,380.79530802)(188.02657762,380.84531497)
\lineto(188.02657762,380.99531497)
\curveto(188.02657748,381.13530768)(188.03657747,381.25030756)(188.05657762,381.34031497)
\curveto(188.07657743,381.44030737)(188.13657737,381.5153073)(188.23657762,381.56531497)
\curveto(188.34657716,381.60530721)(188.48657702,381.6153072)(188.65657762,381.59531497)
\curveto(188.83657667,381.57530724)(188.98657652,381.58530723)(189.10657762,381.62531497)
\curveto(189.19657631,381.67530714)(189.26657624,381.74530707)(189.31657762,381.83531497)
\curveto(189.33657617,381.89530692)(189.34657616,381.97030684)(189.34657762,382.06031497)
\lineto(189.34657762,382.31531497)
\lineto(189.34657762,383.24531497)
\lineto(189.34657762,383.48531497)
\curveto(189.34657616,383.57530524)(189.35657615,383.65030516)(189.37657762,383.71031497)
\curveto(189.41657609,383.79030502)(189.49157602,383.85530496)(189.60157762,383.90531497)
\curveto(189.63157588,383.90530491)(189.65657585,383.90530491)(189.67657762,383.90531497)
\curveto(189.7065758,383.9153049)(189.73157578,383.92030489)(189.75157762,383.92031497)
}
}
{
\newrgbcolor{curcolor}{0 0 0}
\pscustom[linestyle=none,fillstyle=solid,fillcolor=curcolor]
{
\newpath
\moveto(200.64837449,378.05531497)
\curveto(200.66836643,377.99531082)(200.67836642,377.90031091)(200.67837449,377.77031497)
\curveto(200.67836642,377.65031116)(200.67336643,377.56531125)(200.66337449,377.51531497)
\lineto(200.66337449,377.36531497)
\curveto(200.65336645,377.28531153)(200.64336646,377.2103116)(200.63337449,377.14031497)
\curveto(200.63336647,377.08031173)(200.62836647,377.0103118)(200.61837449,376.93031497)
\curveto(200.5983665,376.87031194)(200.58336652,376.810312)(200.57337449,376.75031497)
\curveto(200.57336653,376.69031212)(200.56336654,376.63031218)(200.54337449,376.57031497)
\curveto(200.5033666,376.44031237)(200.46836663,376.3103125)(200.43837449,376.18031497)
\curveto(200.40836669,376.05031276)(200.36836673,375.93031288)(200.31837449,375.82031497)
\curveto(200.10836699,375.34031347)(199.82836727,374.93531388)(199.47837449,374.60531497)
\curveto(199.12836797,374.28531453)(198.6983684,374.04031477)(198.18837449,373.87031497)
\curveto(198.07836902,373.83031498)(197.95836914,373.80031501)(197.82837449,373.78031497)
\curveto(197.70836939,373.76031505)(197.58336952,373.74031507)(197.45337449,373.72031497)
\curveto(197.39336971,373.7103151)(197.32836977,373.70531511)(197.25837449,373.70531497)
\curveto(197.1983699,373.69531512)(197.13836996,373.69031512)(197.07837449,373.69031497)
\curveto(197.03837006,373.68031513)(196.97837012,373.67531514)(196.89837449,373.67531497)
\curveto(196.82837027,373.67531514)(196.77837032,373.68031513)(196.74837449,373.69031497)
\curveto(196.70837039,373.70031511)(196.66837043,373.70531511)(196.62837449,373.70531497)
\curveto(196.58837051,373.69531512)(196.55337055,373.69531512)(196.52337449,373.70531497)
\lineto(196.43337449,373.70531497)
\lineto(196.07337449,373.75031497)
\curveto(195.93337117,373.79031502)(195.7983713,373.83031498)(195.66837449,373.87031497)
\curveto(195.53837156,373.9103149)(195.41337169,373.95531486)(195.29337449,374.00531497)
\curveto(194.84337226,374.20531461)(194.47337263,374.46531435)(194.18337449,374.78531497)
\curveto(193.89337321,375.10531371)(193.65337345,375.49531332)(193.46337449,375.95531497)
\curveto(193.41337369,376.05531276)(193.37337373,376.15531266)(193.34337449,376.25531497)
\curveto(193.32337378,376.35531246)(193.3033738,376.46031235)(193.28337449,376.57031497)
\curveto(193.26337384,376.6103122)(193.25337385,376.64031217)(193.25337449,376.66031497)
\curveto(193.26337384,376.69031212)(193.26337384,376.72531209)(193.25337449,376.76531497)
\curveto(193.23337387,376.84531197)(193.21837388,376.92531189)(193.20837449,377.00531497)
\curveto(193.20837389,377.09531172)(193.1983739,377.18031163)(193.17837449,377.26031497)
\lineto(193.17837449,377.38031497)
\curveto(193.17837392,377.42031139)(193.17337393,377.46531135)(193.16337449,377.51531497)
\curveto(193.15337395,377.56531125)(193.14837395,377.65031116)(193.14837449,377.77031497)
\curveto(193.14837395,377.90031091)(193.15837394,377.99531082)(193.17837449,378.05531497)
\curveto(193.1983739,378.12531069)(193.2033739,378.19531062)(193.19337449,378.26531497)
\curveto(193.18337392,378.33531048)(193.18837391,378.40531041)(193.20837449,378.47531497)
\curveto(193.21837388,378.52531029)(193.22337388,378.56531025)(193.22337449,378.59531497)
\curveto(193.23337387,378.63531018)(193.24337386,378.68031013)(193.25337449,378.73031497)
\curveto(193.28337382,378.85030996)(193.30837379,378.97030984)(193.32837449,379.09031497)
\curveto(193.35837374,379.2103096)(193.3983737,379.32530949)(193.44837449,379.43531497)
\curveto(193.5983735,379.80530901)(193.77837332,380.13530868)(193.98837449,380.42531497)
\curveto(194.20837289,380.72530809)(194.47337263,380.97530784)(194.78337449,381.17531497)
\curveto(194.9033722,381.25530756)(195.02837207,381.32030749)(195.15837449,381.37031497)
\curveto(195.28837181,381.43030738)(195.42337168,381.49030732)(195.56337449,381.55031497)
\curveto(195.68337142,381.60030721)(195.81337129,381.63030718)(195.95337449,381.64031497)
\curveto(196.09337101,381.66030715)(196.23337087,381.69030712)(196.37337449,381.73031497)
\lineto(196.56837449,381.73031497)
\curveto(196.63837046,381.74030707)(196.7033704,381.75030706)(196.76337449,381.76031497)
\curveto(197.65336945,381.77030704)(198.39336871,381.58530723)(198.98337449,381.20531497)
\curveto(199.57336753,380.82530799)(199.9983671,380.33030848)(200.25837449,379.72031497)
\curveto(200.30836679,379.62030919)(200.34836675,379.52030929)(200.37837449,379.42031497)
\curveto(200.40836669,379.32030949)(200.44336666,379.2153096)(200.48337449,379.10531497)
\curveto(200.51336659,378.99530982)(200.53836656,378.87530994)(200.55837449,378.74531497)
\curveto(200.57836652,378.62531019)(200.6033665,378.50031031)(200.63337449,378.37031497)
\curveto(200.64336646,378.32031049)(200.64336646,378.26531055)(200.63337449,378.20531497)
\curveto(200.63336647,378.15531066)(200.63836646,378.10531071)(200.64837449,378.05531497)
\moveto(199.31337449,377.20031497)
\curveto(199.33336777,377.27031154)(199.33836776,377.35031146)(199.32837449,377.44031497)
\lineto(199.32837449,377.69531497)
\curveto(199.32836777,378.08531073)(199.29336781,378.4153104)(199.22337449,378.68531497)
\curveto(199.19336791,378.76531005)(199.16836793,378.84530997)(199.14837449,378.92531497)
\curveto(199.12836797,379.00530981)(199.103368,379.08030973)(199.07337449,379.15031497)
\curveto(198.79336831,379.80030901)(198.34836875,380.25030856)(197.73837449,380.50031497)
\curveto(197.66836943,380.53030828)(197.59336951,380.55030826)(197.51337449,380.56031497)
\lineto(197.27337449,380.62031497)
\curveto(197.19336991,380.64030817)(197.10836999,380.65030816)(197.01837449,380.65031497)
\lineto(196.74837449,380.65031497)
\lineto(196.47837449,380.60531497)
\curveto(196.37837072,380.58530823)(196.28337082,380.56030825)(196.19337449,380.53031497)
\curveto(196.11337099,380.5103083)(196.03337107,380.48030833)(195.95337449,380.44031497)
\curveto(195.88337122,380.42030839)(195.81837128,380.39030842)(195.75837449,380.35031497)
\curveto(195.6983714,380.3103085)(195.64337146,380.27030854)(195.59337449,380.23031497)
\curveto(195.35337175,380.06030875)(195.15837194,379.85530896)(195.00837449,379.61531497)
\curveto(194.85837224,379.37530944)(194.72837237,379.09530972)(194.61837449,378.77531497)
\curveto(194.58837251,378.67531014)(194.56837253,378.57031024)(194.55837449,378.46031497)
\curveto(194.54837255,378.36031045)(194.53337257,378.25531056)(194.51337449,378.14531497)
\curveto(194.5033726,378.10531071)(194.4983726,378.04031077)(194.49837449,377.95031497)
\curveto(194.48837261,377.92031089)(194.48337262,377.88531093)(194.48337449,377.84531497)
\curveto(194.49337261,377.80531101)(194.4983726,377.76031105)(194.49837449,377.71031497)
\lineto(194.49837449,377.41031497)
\curveto(194.4983726,377.3103115)(194.50837259,377.22031159)(194.52837449,377.14031497)
\lineto(194.55837449,376.96031497)
\curveto(194.57837252,376.86031195)(194.59337251,376.76031205)(194.60337449,376.66031497)
\curveto(194.62337248,376.57031224)(194.65337245,376.48531233)(194.69337449,376.40531497)
\curveto(194.79337231,376.16531265)(194.90837219,375.94031287)(195.03837449,375.73031497)
\curveto(195.17837192,375.52031329)(195.34837175,375.34531347)(195.54837449,375.20531497)
\curveto(195.5983715,375.17531364)(195.64337146,375.15031366)(195.68337449,375.13031497)
\curveto(195.72337138,375.1103137)(195.76837133,375.08531373)(195.81837449,375.05531497)
\curveto(195.8983712,375.00531381)(195.98337112,374.96031385)(196.07337449,374.92031497)
\curveto(196.17337093,374.89031392)(196.27837082,374.86031395)(196.38837449,374.83031497)
\curveto(196.43837066,374.810314)(196.48337062,374.80031401)(196.52337449,374.80031497)
\curveto(196.57337053,374.810314)(196.62337048,374.810314)(196.67337449,374.80031497)
\curveto(196.7033704,374.79031402)(196.76337034,374.78031403)(196.85337449,374.77031497)
\curveto(196.95337015,374.76031405)(197.02837007,374.76531405)(197.07837449,374.78531497)
\curveto(197.11836998,374.79531402)(197.15836994,374.79531402)(197.19837449,374.78531497)
\curveto(197.23836986,374.78531403)(197.27836982,374.79531402)(197.31837449,374.81531497)
\curveto(197.3983697,374.83531398)(197.47836962,374.85031396)(197.55837449,374.86031497)
\curveto(197.63836946,374.88031393)(197.71336939,374.90531391)(197.78337449,374.93531497)
\curveto(198.12336898,375.07531374)(198.3983687,375.27031354)(198.60837449,375.52031497)
\curveto(198.81836828,375.77031304)(198.99336811,376.06531275)(199.13337449,376.40531497)
\curveto(199.18336792,376.52531229)(199.21336789,376.65031216)(199.22337449,376.78031497)
\curveto(199.24336786,376.92031189)(199.27336783,377.06031175)(199.31337449,377.20031497)
}
}
{
\newrgbcolor{curcolor}{0 0 0}
\pscustom[linestyle=none,fillstyle=solid,fillcolor=curcolor]
{
\newpath
\moveto(204.56665574,381.76031497)
\curveto(205.28665168,381.77030704)(205.89165107,381.68530713)(206.38165574,381.50531497)
\curveto(206.87165009,381.33530748)(207.25164971,381.03030778)(207.52165574,380.59031497)
\curveto(207.59164937,380.48030833)(207.64664932,380.36530845)(207.68665574,380.24531497)
\curveto(207.72664924,380.13530868)(207.7666492,380.0103088)(207.80665574,379.87031497)
\curveto(207.82664914,379.80030901)(207.83164913,379.72530909)(207.82165574,379.64531497)
\curveto(207.81164915,379.57530924)(207.79664917,379.52030929)(207.77665574,379.48031497)
\curveto(207.75664921,379.46030935)(207.73164923,379.44030937)(207.70165574,379.42031497)
\curveto(207.67164929,379.4103094)(207.64664932,379.39530942)(207.62665574,379.37531497)
\curveto(207.57664939,379.35530946)(207.52664944,379.35030946)(207.47665574,379.36031497)
\curveto(207.42664954,379.37030944)(207.37664959,379.37030944)(207.32665574,379.36031497)
\curveto(207.24664972,379.34030947)(207.14164982,379.33530948)(207.01165574,379.34531497)
\curveto(206.88165008,379.36530945)(206.79165017,379.39030942)(206.74165574,379.42031497)
\curveto(206.6616503,379.47030934)(206.60665036,379.53530928)(206.57665574,379.61531497)
\curveto(206.55665041,379.70530911)(206.52165044,379.79030902)(206.47165574,379.87031497)
\curveto(206.38165058,380.03030878)(206.25665071,380.17530864)(206.09665574,380.30531497)
\curveto(205.98665098,380.38530843)(205.8666511,380.44530837)(205.73665574,380.48531497)
\curveto(205.60665136,380.52530829)(205.4666515,380.56530825)(205.31665574,380.60531497)
\curveto(205.2666517,380.62530819)(205.21665175,380.63030818)(205.16665574,380.62031497)
\curveto(205.11665185,380.62030819)(205.0666519,380.62530819)(205.01665574,380.63531497)
\curveto(204.95665201,380.65530816)(204.88165208,380.66530815)(204.79165574,380.66531497)
\curveto(204.70165226,380.66530815)(204.62665234,380.65530816)(204.56665574,380.63531497)
\lineto(204.47665574,380.63531497)
\lineto(204.32665574,380.60531497)
\curveto(204.27665269,380.60530821)(204.22665274,380.60030821)(204.17665574,380.59031497)
\curveto(203.91665305,380.53030828)(203.70165326,380.44530837)(203.53165574,380.33531497)
\curveto(203.3616536,380.22530859)(203.24665372,380.04030877)(203.18665574,379.78031497)
\curveto(203.1666538,379.7103091)(203.1616538,379.64030917)(203.17165574,379.57031497)
\curveto(203.19165377,379.50030931)(203.21165375,379.44030937)(203.23165574,379.39031497)
\curveto(203.29165367,379.24030957)(203.3616536,379.13030968)(203.44165574,379.06031497)
\curveto(203.53165343,379.00030981)(203.64165332,378.93030988)(203.77165574,378.85031497)
\curveto(203.93165303,378.75031006)(204.11165285,378.67531014)(204.31165574,378.62531497)
\curveto(204.51165245,378.58531023)(204.71165225,378.53531028)(204.91165574,378.47531497)
\curveto(205.04165192,378.43531038)(205.17165179,378.40531041)(205.30165574,378.38531497)
\curveto(205.43165153,378.36531045)(205.5616514,378.33531048)(205.69165574,378.29531497)
\curveto(205.90165106,378.23531058)(206.10665086,378.17531064)(206.30665574,378.11531497)
\curveto(206.50665046,378.06531075)(206.70665026,378.00031081)(206.90665574,377.92031497)
\lineto(207.05665574,377.86031497)
\curveto(207.10664986,377.84031097)(207.15664981,377.815311)(207.20665574,377.78531497)
\curveto(207.40664956,377.66531115)(207.58164938,377.53031128)(207.73165574,377.38031497)
\curveto(207.88164908,377.23031158)(208.00664896,377.04031177)(208.10665574,376.81031497)
\curveto(208.12664884,376.74031207)(208.14664882,376.64531217)(208.16665574,376.52531497)
\curveto(208.18664878,376.45531236)(208.19664877,376.38031243)(208.19665574,376.30031497)
\curveto(208.20664876,376.23031258)(208.21164875,376.15031266)(208.21165574,376.06031497)
\lineto(208.21165574,375.91031497)
\curveto(208.19164877,375.84031297)(208.18164878,375.77031304)(208.18165574,375.70031497)
\curveto(208.18164878,375.63031318)(208.17164879,375.56031325)(208.15165574,375.49031497)
\curveto(208.12164884,375.38031343)(208.08664888,375.27531354)(208.04665574,375.17531497)
\curveto(208.00664896,375.07531374)(207.961649,374.98531383)(207.91165574,374.90531497)
\curveto(207.75164921,374.64531417)(207.54664942,374.43531438)(207.29665574,374.27531497)
\curveto(207.04664992,374.12531469)(206.7666502,373.99531482)(206.45665574,373.88531497)
\curveto(206.3666506,373.85531496)(206.27165069,373.83531498)(206.17165574,373.82531497)
\curveto(206.08165088,373.80531501)(205.99165097,373.78031503)(205.90165574,373.75031497)
\curveto(205.80165116,373.73031508)(205.70165126,373.72031509)(205.60165574,373.72031497)
\curveto(205.50165146,373.72031509)(205.40165156,373.7103151)(205.30165574,373.69031497)
\lineto(205.15165574,373.69031497)
\curveto(205.10165186,373.68031513)(205.03165193,373.67531514)(204.94165574,373.67531497)
\curveto(204.85165211,373.67531514)(204.78165218,373.68031513)(204.73165574,373.69031497)
\lineto(204.56665574,373.69031497)
\curveto(204.50665246,373.7103151)(204.44165252,373.72031509)(204.37165574,373.72031497)
\curveto(204.30165266,373.7103151)(204.24165272,373.7153151)(204.19165574,373.73531497)
\curveto(204.14165282,373.74531507)(204.07665289,373.75031506)(203.99665574,373.75031497)
\lineto(203.75665574,373.81031497)
\curveto(203.68665328,373.82031499)(203.61165335,373.84031497)(203.53165574,373.87031497)
\curveto(203.22165374,373.97031484)(202.95165401,374.09531472)(202.72165574,374.24531497)
\curveto(202.49165447,374.39531442)(202.29165467,374.59031422)(202.12165574,374.83031497)
\curveto(202.03165493,374.96031385)(201.95665501,375.09531372)(201.89665574,375.23531497)
\curveto(201.83665513,375.37531344)(201.78165518,375.53031328)(201.73165574,375.70031497)
\curveto(201.71165525,375.76031305)(201.70165526,375.83031298)(201.70165574,375.91031497)
\curveto(201.71165525,376.00031281)(201.72665524,376.07031274)(201.74665574,376.12031497)
\curveto(201.77665519,376.16031265)(201.82665514,376.20031261)(201.89665574,376.24031497)
\curveto(201.94665502,376.26031255)(202.01665495,376.27031254)(202.10665574,376.27031497)
\curveto(202.19665477,376.28031253)(202.28665468,376.28031253)(202.37665574,376.27031497)
\curveto(202.4666545,376.26031255)(202.55165441,376.24531257)(202.63165574,376.22531497)
\curveto(202.72165424,376.2153126)(202.78165418,376.20031261)(202.81165574,376.18031497)
\curveto(202.88165408,376.13031268)(202.92665404,376.05531276)(202.94665574,375.95531497)
\curveto(202.97665399,375.86531295)(203.01165395,375.78031303)(203.05165574,375.70031497)
\curveto(203.15165381,375.48031333)(203.28665368,375.3103135)(203.45665574,375.19031497)
\curveto(203.57665339,375.10031371)(203.71165325,375.03031378)(203.86165574,374.98031497)
\curveto(204.01165295,374.93031388)(204.17165279,374.88031393)(204.34165574,374.83031497)
\lineto(204.65665574,374.78531497)
\lineto(204.74665574,374.78531497)
\curveto(204.81665215,374.76531405)(204.90665206,374.75531406)(205.01665574,374.75531497)
\curveto(205.13665183,374.75531406)(205.23665173,374.76531405)(205.31665574,374.78531497)
\curveto(205.38665158,374.78531403)(205.44165152,374.79031402)(205.48165574,374.80031497)
\curveto(205.54165142,374.810314)(205.60165136,374.815314)(205.66165574,374.81531497)
\curveto(205.72165124,374.82531399)(205.77665119,374.83531398)(205.82665574,374.84531497)
\curveto(206.11665085,374.92531389)(206.34665062,375.03031378)(206.51665574,375.16031497)
\curveto(206.68665028,375.29031352)(206.80665016,375.5103133)(206.87665574,375.82031497)
\curveto(206.89665007,375.87031294)(206.90165006,375.92531289)(206.89165574,375.98531497)
\curveto(206.88165008,376.04531277)(206.87165009,376.09031272)(206.86165574,376.12031497)
\curveto(206.81165015,376.3103125)(206.74165022,376.45031236)(206.65165574,376.54031497)
\curveto(206.5616504,376.64031217)(206.44665052,376.73031208)(206.30665574,376.81031497)
\curveto(206.21665075,376.87031194)(206.11665085,376.92031189)(206.00665574,376.96031497)
\lineto(205.67665574,377.08031497)
\curveto(205.64665132,377.09031172)(205.61665135,377.09531172)(205.58665574,377.09531497)
\curveto(205.5666514,377.09531172)(205.54165142,377.10531171)(205.51165574,377.12531497)
\curveto(205.17165179,377.23531158)(204.81665215,377.3153115)(204.44665574,377.36531497)
\curveto(204.08665288,377.42531139)(203.74665322,377.52031129)(203.42665574,377.65031497)
\curveto(203.32665364,377.69031112)(203.23165373,377.72531109)(203.14165574,377.75531497)
\curveto(203.05165391,377.78531103)(202.966654,377.82531099)(202.88665574,377.87531497)
\curveto(202.69665427,377.98531083)(202.52165444,378.1103107)(202.36165574,378.25031497)
\curveto(202.20165476,378.39031042)(202.07665489,378.56531025)(201.98665574,378.77531497)
\curveto(201.95665501,378.84530997)(201.93165503,378.9153099)(201.91165574,378.98531497)
\curveto(201.90165506,379.05530976)(201.88665508,379.13030968)(201.86665574,379.21031497)
\curveto(201.83665513,379.33030948)(201.82665514,379.46530935)(201.83665574,379.61531497)
\curveto(201.84665512,379.77530904)(201.8616551,379.9103089)(201.88165574,380.02031497)
\curveto(201.90165506,380.07030874)(201.91165505,380.1103087)(201.91165574,380.14031497)
\curveto(201.92165504,380.18030863)(201.93665503,380.22030859)(201.95665574,380.26031497)
\curveto(202.04665492,380.49030832)(202.1666548,380.69030812)(202.31665574,380.86031497)
\curveto(202.47665449,381.03030778)(202.65665431,381.18030763)(202.85665574,381.31031497)
\curveto(203.00665396,381.40030741)(203.17165379,381.47030734)(203.35165574,381.52031497)
\curveto(203.53165343,381.58030723)(203.72165324,381.63530718)(203.92165574,381.68531497)
\curveto(203.99165297,381.69530712)(204.05665291,381.70530711)(204.11665574,381.71531497)
\curveto(204.18665278,381.72530709)(204.2616527,381.73530708)(204.34165574,381.74531497)
\curveto(204.37165259,381.75530706)(204.41165255,381.75530706)(204.46165574,381.74531497)
\curveto(204.51165245,381.73530708)(204.54665242,381.74030707)(204.56665574,381.76031497)
}
}
{
\newrgbcolor{curcolor}{0.7019608 0.7019608 0.7019608}
\pscustom[linestyle=none,fillstyle=solid,fillcolor=curcolor]
{
\newpath
\moveto(120.84556474,387.06538211)
\lineto(135.84556474,387.06538211)
\lineto(135.84556474,372.06538211)
\lineto(120.84556474,372.06538211)
\closepath
}
}
{
\newrgbcolor{curcolor}{0 0 0}
\pscustom[linestyle=none,fillstyle=solid,fillcolor=curcolor]
{
\newpath
\moveto(148.82876512,351.37313724)
\curveto(148.8787555,351.24313698)(148.85875552,351.14313708)(148.76876512,351.07313724)
\curveto(148.71875566,351.04313718)(148.65375572,351.0231372)(148.57376512,351.01313724)
\lineto(148.34876512,351.01313724)
\lineto(147.86876512,351.01313724)
\curveto(147.70875667,351.01313721)(147.58375679,351.04813717)(147.49376512,351.11813724)
\curveto(147.41375696,351.16813705)(147.35875702,351.24313698)(147.32876512,351.34313724)
\lineto(147.26876512,351.67313724)
\curveto(147.25875712,351.71313651)(147.25375712,351.74813647)(147.25376512,351.77813724)
\lineto(147.25376512,351.88313724)
\curveto(147.23375714,351.93313629)(147.22875715,351.97813624)(147.23876512,352.01813724)
\curveto(147.24875713,352.05813616)(147.24875713,352.09813612)(147.23876512,352.13813724)
\curveto(147.22875715,352.19813602)(147.22375715,352.25813596)(147.22376512,352.31813724)
\lineto(147.22376512,352.49813724)
\lineto(147.17876512,353.17313724)
\curveto(147.15875722,353.24313498)(147.14875723,353.31313491)(147.14876512,353.38313724)
\curveto(147.14875723,353.45313477)(147.13875724,353.52813469)(147.11876512,353.60813724)
\curveto(147.06875731,353.78813443)(147.02875735,353.96813425)(146.99876512,354.14813724)
\curveto(146.9787574,354.32813389)(146.93375744,354.49813372)(146.86376512,354.65813724)
\curveto(146.6737577,355.07813314)(146.35875802,355.35813286)(145.91876512,355.49813724)
\curveto(145.78875859,355.54813267)(145.64375873,355.57313265)(145.48376512,355.57313724)
\curveto(145.33375904,355.58313264)(145.1737592,355.58813263)(145.00376512,355.58813724)
\lineto(142.24376512,355.58813724)
\curveto(142.1737622,355.56813265)(142.10876227,355.54813267)(142.04876512,355.52813724)
\curveto(141.99876238,355.5181327)(141.95376242,355.48813273)(141.91376512,355.43813724)
\curveto(141.84376253,355.33813288)(141.80876257,355.17313305)(141.80876512,354.94313724)
\curveto(141.81876256,354.7231335)(141.82376255,354.52813369)(141.82376512,354.35813724)
\lineto(141.82376512,352.18313724)
\curveto(141.82376255,352.04313618)(141.82876255,351.86813635)(141.83876512,351.65813724)
\curveto(141.84876253,351.45813676)(141.82876255,351.30813691)(141.77876512,351.20813724)
\curveto(141.75876262,351.13813708)(141.71876266,351.09313713)(141.65876512,351.07313724)
\curveto(141.61876276,351.05313717)(141.5787628,351.04313718)(141.53876512,351.04313724)
\curveto(141.50876287,351.04313718)(141.46876291,351.03313719)(141.41876512,351.01313724)
\curveto(141.378763,351.00313722)(141.33376304,350.99813722)(141.28376512,350.99813724)
\curveto(141.23376314,351.00813721)(141.18376319,351.01313721)(141.13376512,351.01313724)
\lineto(140.80376512,351.01313724)
\curveto(140.70376367,351.0231372)(140.61876376,351.05313717)(140.54876512,351.10313724)
\curveto(140.46876391,351.15313707)(140.42876395,351.24313698)(140.42876512,351.37313724)
\lineto(140.42876512,351.77813724)
\lineto(140.42876512,360.89813724)
\curveto(140.42876395,361.00812721)(140.42376395,361.1231271)(140.41376512,361.24313724)
\curveto(140.41376396,361.36312686)(140.43876394,361.45812676)(140.48876512,361.52813724)
\curveto(140.52876385,361.58812663)(140.60376377,361.63812658)(140.71376512,361.67813724)
\curveto(140.73376364,361.68812653)(140.75376362,361.68812653)(140.77376512,361.67813724)
\curveto(140.79376358,361.67812654)(140.81376356,361.68312654)(140.83376512,361.69313724)
\lineto(145.18376512,361.69313724)
\curveto(145.25375912,361.69312653)(145.32875905,361.69312653)(145.40876512,361.69313724)
\curveto(145.48875889,361.70312652)(145.55875882,361.70312652)(145.61876512,361.69313724)
\lineto(145.78376512,361.69313724)
\curveto(145.84375853,361.68312654)(145.90375847,361.67312655)(145.96376512,361.66313724)
\curveto(146.02375835,361.66312656)(146.08875829,361.65812656)(146.15876512,361.64813724)
\curveto(146.23875814,361.62812659)(146.31875806,361.61312661)(146.39876512,361.60313724)
\curveto(146.48875789,361.59312663)(146.5737578,361.57812664)(146.65376512,361.55813724)
\curveto(146.84375753,361.49812672)(147.01875736,361.43312679)(147.17876512,361.36313724)
\curveto(147.33875704,361.29312693)(147.48875689,361.20812701)(147.62876512,361.10813724)
\curveto(147.8787565,360.93812728)(148.0787563,360.72812749)(148.22876512,360.47813724)
\curveto(148.38875599,360.23812798)(148.51875586,359.95312827)(148.61876512,359.62313724)
\curveto(148.63875574,359.54312868)(148.64875573,359.45812876)(148.64876512,359.36813724)
\curveto(148.65875572,359.28812893)(148.6737557,359.20812901)(148.69376512,359.12813724)
\lineto(148.69376512,358.97813724)
\curveto(148.70375567,358.92812929)(148.70375567,358.86812935)(148.69376512,358.79813724)
\curveto(148.69375568,358.73812948)(148.68875569,358.68312954)(148.67876512,358.63313724)
\lineto(148.67876512,358.46813724)
\curveto(148.65875572,358.38812983)(148.64375573,358.31312991)(148.63376512,358.24313724)
\curveto(148.63375574,358.17313005)(148.62375575,358.10313012)(148.60376512,358.03313724)
\curveto(148.55375582,357.88313034)(148.50375587,357.73813048)(148.45376512,357.59813724)
\curveto(148.41375596,357.46813075)(148.35375602,357.34313088)(148.27376512,357.22313724)
\curveto(148.24375613,357.17313105)(148.20875617,357.12813109)(148.16876512,357.08813724)
\curveto(148.13875624,357.04813117)(148.10875627,357.00313122)(148.07876512,356.95313724)
\lineto(148.04876512,356.92313724)
\curveto(148.03875634,356.9231313)(148.02875635,356.9181313)(148.01876512,356.90813724)
\lineto(147.94376512,356.83313724)
\curveto(147.92375645,356.80313142)(147.90375647,356.77813144)(147.88376512,356.75813724)
\curveto(147.80375657,356.69813152)(147.72875665,356.63813158)(147.65876512,356.57813724)
\curveto(147.58875679,356.52813169)(147.51375686,356.47813174)(147.43376512,356.42813724)
\curveto(147.38375699,356.39813182)(147.33875704,356.36313186)(147.29876512,356.32313724)
\curveto(147.25875712,356.29313193)(147.23375714,356.24813197)(147.22376512,356.18813724)
\curveto(147.21375716,356.12813209)(147.23375714,356.07813214)(147.28376512,356.03813724)
\curveto(147.34375703,355.99813222)(147.39375698,355.96813225)(147.43376512,355.94813724)
\curveto(147.54375683,355.87813234)(147.64375673,355.80313242)(147.73376512,355.72313724)
\curveto(147.83375654,355.64313258)(147.91875646,355.54813267)(147.98876512,355.43813724)
\curveto(148.09875628,355.29813292)(148.1787562,355.13813308)(148.22876512,354.95813724)
\curveto(148.2787561,354.78813343)(148.32875605,354.60313362)(148.37876512,354.40313724)
\lineto(148.40876512,354.16313724)
\curveto(148.41875596,354.09313413)(148.42875595,354.0181342)(148.43876512,353.93813724)
\curveto(148.45875592,353.86813435)(148.46375591,353.79813442)(148.45376512,353.72813724)
\curveto(148.44375593,353.65813456)(148.44875593,353.58813463)(148.46876512,353.51813724)
\lineto(148.46876512,353.38313724)
\curveto(148.48875589,353.31313491)(148.49375588,353.23813498)(148.48376512,353.15813724)
\curveto(148.4737559,353.07813514)(148.4787559,352.99813522)(148.49876512,352.91813724)
\curveto(148.50875587,352.87813534)(148.50875587,352.83813538)(148.49876512,352.79813724)
\curveto(148.49875588,352.76813545)(148.50375587,352.72813549)(148.51376512,352.67813724)
\curveto(148.53375584,352.57813564)(148.54875583,352.47313575)(148.55876512,352.36313724)
\curveto(148.56875581,352.26313596)(148.58875579,352.16813605)(148.61876512,352.07813724)
\curveto(148.63875574,352.0181362)(148.64875573,351.95813626)(148.64876512,351.89813724)
\curveto(148.65875572,351.84813637)(148.6737557,351.79313643)(148.69376512,351.73313724)
\lineto(148.82876512,351.37313724)
\moveto(147.01376512,357.59813724)
\curveto(147.08375729,357.70813051)(147.13375724,357.8231304)(147.16376512,357.94313724)
\curveto(147.20375717,358.06313016)(147.23875714,358.19313003)(147.26876512,358.33313724)
\lineto(147.26876512,358.46813724)
\curveto(147.29875708,358.60812961)(147.30375707,358.75812946)(147.28376512,358.91813724)
\curveto(147.26375711,359.08812913)(147.23375714,359.22812899)(147.19376512,359.33813724)
\curveto(147.03375734,359.83812838)(146.71875766,360.18312804)(146.24876512,360.37313724)
\curveto(146.04875833,360.45312777)(145.81375856,360.49812772)(145.54376512,360.50813724)
\curveto(145.28375909,360.5181277)(145.01375936,360.5231277)(144.73376512,360.52313724)
\lineto(142.25876512,360.52313724)
\curveto(142.23876214,360.51312771)(142.21376216,360.50812771)(142.18376512,360.50813724)
\curveto(142.16376221,360.50812771)(142.13876224,360.50312772)(142.10876512,360.49313724)
\curveto(141.98876239,360.46312776)(141.90876247,360.39812782)(141.86876512,360.29813724)
\curveto(141.82876255,360.20812801)(141.80876257,360.08312814)(141.80876512,359.92313724)
\curveto(141.81876256,359.76312846)(141.82376255,359.6181286)(141.82376512,359.48813724)
\lineto(141.82376512,357.76313724)
\curveto(141.82376255,357.61313061)(141.81876256,357.45313077)(141.80876512,357.28313724)
\curveto(141.80876257,357.1231311)(141.84376253,356.99813122)(141.91376512,356.90813724)
\curveto(141.96376241,356.83813138)(142.03876234,356.79313143)(142.13876512,356.77313724)
\curveto(142.23876214,356.76313146)(142.34876203,356.75813146)(142.46876512,356.75813724)
\lineto(143.39876512,356.75813724)
\curveto(143.78876059,356.75813146)(144.16876021,356.75313147)(144.53876512,356.74313724)
\curveto(144.90875947,356.74313148)(145.24875913,356.76313146)(145.55876512,356.80313724)
\curveto(145.8787585,356.85313137)(146.16375821,356.93813128)(146.41376512,357.05813724)
\curveto(146.66375771,357.17813104)(146.86375751,357.35813086)(147.01376512,357.59813724)
}
}
{
\newrgbcolor{curcolor}{0 0 0}
\pscustom[linestyle=none,fillstyle=solid,fillcolor=curcolor]
{
\newpath
\moveto(157.20696824,355.16813724)
\curveto(157.22696056,355.06813315)(157.22696056,354.95313327)(157.20696824,354.82313724)
\curveto(157.19696059,354.70313352)(157.16696062,354.6181336)(157.11696824,354.56813724)
\curveto(157.06696072,354.52813369)(156.99196079,354.49813372)(156.89196824,354.47813724)
\curveto(156.80196098,354.46813375)(156.69696109,354.46313376)(156.57696824,354.46313724)
\lineto(156.21696824,354.46313724)
\curveto(156.09696169,354.47313375)(155.99196179,354.47813374)(155.90196824,354.47813724)
\lineto(152.06196824,354.47813724)
\curveto(151.9819658,354.47813374)(151.90196588,354.47313375)(151.82196824,354.46313724)
\curveto(151.74196604,354.46313376)(151.67696611,354.44813377)(151.62696824,354.41813724)
\curveto(151.5869662,354.39813382)(151.54696624,354.35813386)(151.50696824,354.29813724)
\curveto(151.4869663,354.26813395)(151.46696632,354.223134)(151.44696824,354.16313724)
\curveto(151.42696636,354.11313411)(151.42696636,354.06313416)(151.44696824,354.01313724)
\curveto(151.45696633,353.96313426)(151.46196632,353.9181343)(151.46196824,353.87813724)
\curveto(151.46196632,353.83813438)(151.46696632,353.79813442)(151.47696824,353.75813724)
\curveto(151.49696629,353.67813454)(151.51696627,353.59313463)(151.53696824,353.50313724)
\curveto(151.55696623,353.4231348)(151.5869662,353.34313488)(151.62696824,353.26313724)
\curveto(151.85696593,352.7231355)(152.23696555,352.33813588)(152.76696824,352.10813724)
\curveto(152.82696496,352.07813614)(152.89196489,352.05313617)(152.96196824,352.03313724)
\lineto(153.17196824,351.97313724)
\curveto(153.20196458,351.96313626)(153.25196453,351.95813626)(153.32196824,351.95813724)
\curveto(153.46196432,351.9181363)(153.64696414,351.89813632)(153.87696824,351.89813724)
\curveto(154.10696368,351.89813632)(154.29196349,351.9181363)(154.43196824,351.95813724)
\curveto(154.57196321,351.99813622)(154.69696309,352.03813618)(154.80696824,352.07813724)
\curveto(154.92696286,352.12813609)(155.03696275,352.18813603)(155.13696824,352.25813724)
\curveto(155.24696254,352.32813589)(155.34196244,352.40813581)(155.42196824,352.49813724)
\curveto(155.50196228,352.59813562)(155.57196221,352.70313552)(155.63196824,352.81313724)
\curveto(155.69196209,352.91313531)(155.74196204,353.0181352)(155.78196824,353.12813724)
\curveto(155.83196195,353.23813498)(155.91196187,353.3181349)(156.02196824,353.36813724)
\curveto(156.06196172,353.38813483)(156.12696166,353.40313482)(156.21696824,353.41313724)
\curveto(156.30696148,353.4231348)(156.39696139,353.4231348)(156.48696824,353.41313724)
\curveto(156.57696121,353.41313481)(156.66196112,353.40813481)(156.74196824,353.39813724)
\curveto(156.82196096,353.38813483)(156.87696091,353.36813485)(156.90696824,353.33813724)
\curveto(157.00696078,353.26813495)(157.03196075,353.15313507)(156.98196824,352.99313724)
\curveto(156.90196088,352.7231355)(156.79696099,352.48313574)(156.66696824,352.27313724)
\curveto(156.46696132,351.95313627)(156.23696155,351.68813653)(155.97696824,351.47813724)
\curveto(155.72696206,351.27813694)(155.40696238,351.11313711)(155.01696824,350.98313724)
\curveto(154.91696287,350.94313728)(154.81696297,350.9181373)(154.71696824,350.90813724)
\curveto(154.61696317,350.88813733)(154.51196327,350.86813735)(154.40196824,350.84813724)
\curveto(154.35196343,350.83813738)(154.30196348,350.83313739)(154.25196824,350.83313724)
\curveto(154.21196357,350.83313739)(154.16696362,350.82813739)(154.11696824,350.81813724)
\lineto(153.96696824,350.81813724)
\curveto(153.91696387,350.80813741)(153.85696393,350.80313742)(153.78696824,350.80313724)
\curveto(153.72696406,350.80313742)(153.67696411,350.80813741)(153.63696824,350.81813724)
\lineto(153.50196824,350.81813724)
\curveto(153.45196433,350.82813739)(153.40696438,350.83313739)(153.36696824,350.83313724)
\curveto(153.32696446,350.83313739)(153.2869645,350.83813738)(153.24696824,350.84813724)
\curveto(153.19696459,350.85813736)(153.14196464,350.86813735)(153.08196824,350.87813724)
\curveto(153.02196476,350.87813734)(152.96696482,350.88313734)(152.91696824,350.89313724)
\curveto(152.82696496,350.91313731)(152.73696505,350.93813728)(152.64696824,350.96813724)
\curveto(152.55696523,350.98813723)(152.47196531,351.01313721)(152.39196824,351.04313724)
\curveto(152.35196543,351.06313716)(152.31696547,351.07313715)(152.28696824,351.07313724)
\curveto(152.25696553,351.08313714)(152.22196556,351.09813712)(152.18196824,351.11813724)
\curveto(152.03196575,351.18813703)(151.87196591,351.27313695)(151.70196824,351.37313724)
\curveto(151.41196637,351.56313666)(151.16196662,351.79313643)(150.95196824,352.06313724)
\curveto(150.75196703,352.34313588)(150.5819672,352.65313557)(150.44196824,352.99313724)
\curveto(150.39196739,353.10313512)(150.35196743,353.218135)(150.32196824,353.33813724)
\curveto(150.30196748,353.45813476)(150.27196751,353.57813464)(150.23196824,353.69813724)
\curveto(150.22196756,353.73813448)(150.21696757,353.77313445)(150.21696824,353.80313724)
\curveto(150.21696757,353.83313439)(150.21196757,353.87313435)(150.20196824,353.92313724)
\curveto(150.1819676,354.00313422)(150.16696762,354.08813413)(150.15696824,354.17813724)
\curveto(150.14696764,354.26813395)(150.13196765,354.35813386)(150.11196824,354.44813724)
\lineto(150.11196824,354.65813724)
\curveto(150.10196768,354.69813352)(150.09196769,354.75313347)(150.08196824,354.82313724)
\curveto(150.0819677,354.90313332)(150.0869677,354.96813325)(150.09696824,355.01813724)
\lineto(150.09696824,355.18313724)
\curveto(150.11696767,355.23313299)(150.12196766,355.28313294)(150.11196824,355.33313724)
\curveto(150.11196767,355.39313283)(150.11696767,355.44813277)(150.12696824,355.49813724)
\curveto(150.16696762,355.65813256)(150.19696759,355.8181324)(150.21696824,355.97813724)
\curveto(150.24696754,356.13813208)(150.29196749,356.28813193)(150.35196824,356.42813724)
\curveto(150.40196738,356.53813168)(150.44696734,356.64813157)(150.48696824,356.75813724)
\curveto(150.53696725,356.87813134)(150.59196719,356.99313123)(150.65196824,357.10313724)
\curveto(150.87196691,357.45313077)(151.12196666,357.75313047)(151.40196824,358.00313724)
\curveto(151.6819661,358.26312996)(152.02696576,358.47812974)(152.43696824,358.64813724)
\curveto(152.55696523,358.69812952)(152.67696511,358.73312949)(152.79696824,358.75313724)
\curveto(152.92696486,358.78312944)(153.06196472,358.81312941)(153.20196824,358.84313724)
\curveto(153.25196453,358.85312937)(153.29696449,358.85812936)(153.33696824,358.85813724)
\curveto(153.37696441,358.86812935)(153.42196436,358.87312935)(153.47196824,358.87313724)
\curveto(153.49196429,358.88312934)(153.51696427,358.88312934)(153.54696824,358.87313724)
\curveto(153.57696421,358.86312936)(153.60196418,358.86812935)(153.62196824,358.88813724)
\curveto(154.04196374,358.89812932)(154.40696338,358.85312937)(154.71696824,358.75313724)
\curveto(155.02696276,358.66312956)(155.30696248,358.53812968)(155.55696824,358.37813724)
\curveto(155.60696218,358.35812986)(155.64696214,358.32812989)(155.67696824,358.28813724)
\curveto(155.70696208,358.25812996)(155.74196204,358.23312999)(155.78196824,358.21313724)
\curveto(155.86196192,358.15313007)(155.94196184,358.08313014)(156.02196824,358.00313724)
\curveto(156.11196167,357.9231303)(156.1869616,357.84313038)(156.24696824,357.76313724)
\curveto(156.40696138,357.55313067)(156.54196124,357.35313087)(156.65196824,357.16313724)
\curveto(156.72196106,357.05313117)(156.77696101,356.93313129)(156.81696824,356.80313724)
\curveto(156.85696093,356.67313155)(156.90196088,356.54313168)(156.95196824,356.41313724)
\curveto(157.00196078,356.28313194)(157.03696075,356.14813207)(157.05696824,356.00813724)
\curveto(157.0869607,355.86813235)(157.12196066,355.72813249)(157.16196824,355.58813724)
\curveto(157.17196061,355.5181327)(157.17696061,355.44813277)(157.17696824,355.37813724)
\lineto(157.20696824,355.16813724)
\moveto(155.75196824,355.67813724)
\curveto(155.781962,355.7181325)(155.80696198,355.76813245)(155.82696824,355.82813724)
\curveto(155.84696194,355.89813232)(155.84696194,355.96813225)(155.82696824,356.03813724)
\curveto(155.76696202,356.25813196)(155.6819621,356.46313176)(155.57196824,356.65313724)
\curveto(155.43196235,356.88313134)(155.27696251,357.07813114)(155.10696824,357.23813724)
\curveto(154.93696285,357.39813082)(154.71696307,357.53313069)(154.44696824,357.64313724)
\curveto(154.37696341,357.66313056)(154.30696348,357.67813054)(154.23696824,357.68813724)
\curveto(154.16696362,357.70813051)(154.09196369,357.72813049)(154.01196824,357.74813724)
\curveto(153.93196385,357.76813045)(153.84696394,357.77813044)(153.75696824,357.77813724)
\lineto(153.50196824,357.77813724)
\curveto(153.47196431,357.75813046)(153.43696435,357.74813047)(153.39696824,357.74813724)
\curveto(153.35696443,357.75813046)(153.32196446,357.75813046)(153.29196824,357.74813724)
\lineto(153.05196824,357.68813724)
\curveto(152.9819648,357.67813054)(152.91196487,357.66313056)(152.84196824,357.64313724)
\curveto(152.55196523,357.5231307)(152.31696547,357.37313085)(152.13696824,357.19313724)
\curveto(151.96696582,357.01313121)(151.81196597,356.78813143)(151.67196824,356.51813724)
\curveto(151.64196614,356.46813175)(151.61196617,356.40313182)(151.58196824,356.32313724)
\curveto(151.55196623,356.25313197)(151.52696626,356.17313205)(151.50696824,356.08313724)
\curveto(151.4869663,355.99313223)(151.4819663,355.90813231)(151.49196824,355.82813724)
\curveto(151.50196628,355.74813247)(151.53696625,355.68813253)(151.59696824,355.64813724)
\curveto(151.67696611,355.58813263)(151.81196597,355.55813266)(152.00196824,355.55813724)
\curveto(152.20196558,355.56813265)(152.37196541,355.57313265)(152.51196824,355.57313724)
\lineto(154.79196824,355.57313724)
\curveto(154.94196284,355.57313265)(155.12196266,355.56813265)(155.33196824,355.55813724)
\curveto(155.54196224,355.55813266)(155.6819621,355.59813262)(155.75196824,355.67813724)
}
}
{
\newrgbcolor{curcolor}{0 0 0}
\pscustom[linestyle=none,fillstyle=solid,fillcolor=curcolor]
{
\newpath
\moveto(161.64860887,358.90313724)
\curveto(162.38860408,358.91312931)(163.00360346,358.80312942)(163.49360887,358.57313724)
\curveto(163.99360247,358.35312987)(164.38860208,358.0181302)(164.67860887,357.56813724)
\curveto(164.80860166,357.36813085)(164.91860155,357.1231311)(165.00860887,356.83313724)
\curveto(165.02860144,356.78313144)(165.04360142,356.7181315)(165.05360887,356.63813724)
\curveto(165.0636014,356.55813166)(165.05860141,356.48813173)(165.03860887,356.42813724)
\curveto(165.00860146,356.37813184)(164.95860151,356.33313189)(164.88860887,356.29313724)
\curveto(164.85860161,356.27313195)(164.82860164,356.26313196)(164.79860887,356.26313724)
\curveto(164.7686017,356.27313195)(164.73360173,356.27313195)(164.69360887,356.26313724)
\curveto(164.65360181,356.25313197)(164.61360185,356.24813197)(164.57360887,356.24813724)
\curveto(164.53360193,356.25813196)(164.49360197,356.26313196)(164.45360887,356.26313724)
\lineto(164.13860887,356.26313724)
\curveto(164.03860243,356.27313195)(163.95360251,356.30313192)(163.88360887,356.35313724)
\curveto(163.80360266,356.41313181)(163.74860272,356.49813172)(163.71860887,356.60813724)
\curveto(163.68860278,356.7181315)(163.64860282,356.81313141)(163.59860887,356.89313724)
\curveto(163.44860302,357.15313107)(163.25360321,357.35813086)(163.01360887,357.50813724)
\curveto(162.93360353,357.55813066)(162.84860362,357.59813062)(162.75860887,357.62813724)
\curveto(162.6686038,357.66813055)(162.57360389,357.70313052)(162.47360887,357.73313724)
\curveto(162.33360413,357.77313045)(162.14860432,357.79313043)(161.91860887,357.79313724)
\curveto(161.68860478,357.80313042)(161.49860497,357.78313044)(161.34860887,357.73313724)
\curveto(161.27860519,357.71313051)(161.21360525,357.69813052)(161.15360887,357.68813724)
\curveto(161.09360537,357.67813054)(161.02860544,357.66313056)(160.95860887,357.64313724)
\curveto(160.69860577,357.53313069)(160.468606,357.38313084)(160.26860887,357.19313724)
\curveto(160.0686064,357.00313122)(159.91360655,356.77813144)(159.80360887,356.51813724)
\curveto(159.7636067,356.42813179)(159.72860674,356.33313189)(159.69860887,356.23313724)
\curveto(159.6686068,356.14313208)(159.63860683,356.04313218)(159.60860887,355.93313724)
\lineto(159.51860887,355.52813724)
\curveto(159.50860696,355.47813274)(159.50360696,355.4231328)(159.50360887,355.36313724)
\curveto(159.51360695,355.30313292)(159.50860696,355.24813297)(159.48860887,355.19813724)
\lineto(159.48860887,355.07813724)
\curveto(159.47860699,355.03813318)(159.468607,354.97313325)(159.45860887,354.88313724)
\curveto(159.45860701,354.79313343)(159.468607,354.72813349)(159.48860887,354.68813724)
\curveto(159.49860697,354.63813358)(159.49860697,354.58813363)(159.48860887,354.53813724)
\curveto(159.47860699,354.48813373)(159.47860699,354.43813378)(159.48860887,354.38813724)
\curveto(159.49860697,354.34813387)(159.50360696,354.27813394)(159.50360887,354.17813724)
\curveto(159.52360694,354.09813412)(159.53860693,354.01313421)(159.54860887,353.92313724)
\curveto(159.5686069,353.83313439)(159.58860688,353.74813447)(159.60860887,353.66813724)
\curveto(159.71860675,353.34813487)(159.84360662,353.06813515)(159.98360887,352.82813724)
\curveto(160.13360633,352.59813562)(160.33860613,352.39813582)(160.59860887,352.22813724)
\curveto(160.68860578,352.17813604)(160.77860569,352.13313609)(160.86860887,352.09313724)
\curveto(160.9686055,352.05313617)(161.07360539,352.01313621)(161.18360887,351.97313724)
\curveto(161.23360523,351.96313626)(161.27360519,351.95813626)(161.30360887,351.95813724)
\curveto(161.33360513,351.95813626)(161.37360509,351.95313627)(161.42360887,351.94313724)
\curveto(161.45360501,351.93313629)(161.50360496,351.92813629)(161.57360887,351.92813724)
\lineto(161.73860887,351.92813724)
\curveto(161.73860473,351.9181363)(161.75860471,351.91313631)(161.79860887,351.91313724)
\curveto(161.81860465,351.9231363)(161.84360462,351.9231363)(161.87360887,351.91313724)
\curveto(161.90360456,351.91313631)(161.93360453,351.9181363)(161.96360887,351.92813724)
\curveto(162.03360443,351.94813627)(162.09860437,351.95313627)(162.15860887,351.94313724)
\curveto(162.22860424,351.94313628)(162.29860417,351.95313627)(162.36860887,351.97313724)
\curveto(162.62860384,352.05313617)(162.85360361,352.15313607)(163.04360887,352.27313724)
\curveto(163.23360323,352.40313582)(163.39360307,352.56813565)(163.52360887,352.76813724)
\curveto(163.57360289,352.84813537)(163.61860285,352.93313529)(163.65860887,353.02313724)
\lineto(163.77860887,353.29313724)
\curveto(163.79860267,353.37313485)(163.81860265,353.44813477)(163.83860887,353.51813724)
\curveto(163.8686026,353.59813462)(163.91860255,353.66313456)(163.98860887,353.71313724)
\curveto(164.01860245,353.74313448)(164.07860239,353.76313446)(164.16860887,353.77313724)
\curveto(164.25860221,353.79313443)(164.35360211,353.80313442)(164.45360887,353.80313724)
\curveto(164.5636019,353.81313441)(164.6636018,353.81313441)(164.75360887,353.80313724)
\curveto(164.85360161,353.79313443)(164.92360154,353.77313445)(164.96360887,353.74313724)
\curveto(165.02360144,353.70313452)(165.05860141,353.64313458)(165.06860887,353.56313724)
\curveto(165.08860138,353.48313474)(165.08860138,353.39813482)(165.06860887,353.30813724)
\curveto(165.01860145,353.15813506)(164.9686015,353.01313521)(164.91860887,352.87313724)
\curveto(164.87860159,352.74313548)(164.82360164,352.61313561)(164.75360887,352.48313724)
\curveto(164.60360186,352.18313604)(164.41360205,351.9181363)(164.18360887,351.68813724)
\curveto(163.9636025,351.45813676)(163.69360277,351.27313695)(163.37360887,351.13313724)
\curveto(163.29360317,351.09313713)(163.20860326,351.05813716)(163.11860887,351.02813724)
\curveto(163.02860344,351.00813721)(162.93360353,350.98313724)(162.83360887,350.95313724)
\curveto(162.72360374,350.91313731)(162.61360385,350.89313733)(162.50360887,350.89313724)
\curveto(162.39360407,350.88313734)(162.28360418,350.86813735)(162.17360887,350.84813724)
\curveto(162.13360433,350.82813739)(162.09360437,350.8231374)(162.05360887,350.83313724)
\curveto(162.01360445,350.84313738)(161.97360449,350.84313738)(161.93360887,350.83313724)
\lineto(161.79860887,350.83313724)
\lineto(161.55860887,350.83313724)
\curveto(161.48860498,350.8231374)(161.42360504,350.82813739)(161.36360887,350.84813724)
\lineto(161.28860887,350.84813724)
\lineto(160.92860887,350.89313724)
\curveto(160.79860567,350.93313729)(160.67360579,350.96813725)(160.55360887,350.99813724)
\curveto(160.43360603,351.02813719)(160.31860615,351.06813715)(160.20860887,351.11813724)
\curveto(159.84860662,351.27813694)(159.54860692,351.46813675)(159.30860887,351.68813724)
\curveto(159.07860739,351.90813631)(158.8636076,352.17813604)(158.66360887,352.49813724)
\curveto(158.61360785,352.57813564)(158.5686079,352.66813555)(158.52860887,352.76813724)
\lineto(158.40860887,353.06813724)
\curveto(158.35860811,353.17813504)(158.32360814,353.29313493)(158.30360887,353.41313724)
\curveto(158.28360818,353.53313469)(158.25860821,353.65313457)(158.22860887,353.77313724)
\curveto(158.21860825,353.81313441)(158.21360825,353.85313437)(158.21360887,353.89313724)
\curveto(158.21360825,353.93313429)(158.20860826,353.97313425)(158.19860887,354.01313724)
\curveto(158.17860829,354.07313415)(158.1686083,354.13813408)(158.16860887,354.20813724)
\curveto(158.17860829,354.27813394)(158.17360829,354.34313388)(158.15360887,354.40313724)
\lineto(158.15360887,354.55313724)
\curveto(158.14360832,354.60313362)(158.13860833,354.67313355)(158.13860887,354.76313724)
\curveto(158.13860833,354.85313337)(158.14360832,354.9231333)(158.15360887,354.97313724)
\curveto(158.1636083,355.0231332)(158.1636083,355.06813315)(158.15360887,355.10813724)
\curveto(158.15360831,355.14813307)(158.15860831,355.18813303)(158.16860887,355.22813724)
\curveto(158.18860828,355.29813292)(158.19360827,355.36813285)(158.18360887,355.43813724)
\curveto(158.18360828,355.50813271)(158.19360827,355.57313265)(158.21360887,355.63313724)
\curveto(158.25360821,355.80313242)(158.28860818,355.97313225)(158.31860887,356.14313724)
\curveto(158.34860812,356.31313191)(158.39360807,356.47313175)(158.45360887,356.62313724)
\curveto(158.6636078,357.14313108)(158.91860755,357.56313066)(159.21860887,357.88313724)
\curveto(159.51860695,358.20313002)(159.92860654,358.46812975)(160.44860887,358.67813724)
\curveto(160.55860591,358.72812949)(160.67860579,358.76312946)(160.80860887,358.78313724)
\curveto(160.93860553,358.80312942)(161.07360539,358.82812939)(161.21360887,358.85813724)
\curveto(161.28360518,358.86812935)(161.35360511,358.87312935)(161.42360887,358.87313724)
\curveto(161.49360497,358.88312934)(161.5686049,358.89312933)(161.64860887,358.90313724)
}
}
{
\newrgbcolor{curcolor}{0 0 0}
\pscustom[linestyle=none,fillstyle=solid,fillcolor=curcolor]
{
\newpath
\moveto(167.03524949,358.72313724)
\lineto(167.47024949,358.72313724)
\curveto(167.62024753,358.7231295)(167.72524742,358.68312954)(167.78524949,358.60313724)
\curveto(167.83524731,358.5231297)(167.86024729,358.4231298)(167.86024949,358.30313724)
\curveto(167.87024728,358.18313004)(167.87524727,358.06313016)(167.87524949,357.94313724)
\lineto(167.87524949,356.51813724)
\lineto(167.87524949,354.25313724)
\lineto(167.87524949,353.56313724)
\curveto(167.87524727,353.33313489)(167.90024725,353.13313509)(167.95024949,352.96313724)
\curveto(168.11024704,352.51313571)(168.41024674,352.19813602)(168.85024949,352.01813724)
\curveto(169.07024608,351.92813629)(169.33524581,351.89313633)(169.64524949,351.91313724)
\curveto(169.95524519,351.94313628)(170.20524494,351.99813622)(170.39524949,352.07813724)
\curveto(170.72524442,352.218136)(170.98524416,352.39313583)(171.17524949,352.60313724)
\curveto(171.37524377,352.8231354)(171.53024362,353.10813511)(171.64024949,353.45813724)
\curveto(171.67024348,353.53813468)(171.69024346,353.6181346)(171.70024949,353.69813724)
\curveto(171.71024344,353.77813444)(171.72524342,353.86313436)(171.74524949,353.95313724)
\curveto(171.75524339,354.00313422)(171.75524339,354.04813417)(171.74524949,354.08813724)
\curveto(171.7452434,354.12813409)(171.75524339,354.17313405)(171.77524949,354.22313724)
\lineto(171.77524949,354.53813724)
\curveto(171.79524335,354.6181336)(171.80024335,354.70813351)(171.79024949,354.80813724)
\curveto(171.78024337,354.9181333)(171.77524337,355.0181332)(171.77524949,355.10813724)
\lineto(171.77524949,356.27813724)
\lineto(171.77524949,357.86813724)
\curveto(171.77524337,357.98813023)(171.77024338,358.11313011)(171.76024949,358.24313724)
\curveto(171.76024339,358.38312984)(171.78524336,358.49312973)(171.83524949,358.57313724)
\curveto(171.87524327,358.6231296)(171.92024323,358.65312957)(171.97024949,358.66313724)
\curveto(172.03024312,358.68312954)(172.10024305,358.70312952)(172.18024949,358.72313724)
\lineto(172.40524949,358.72313724)
\curveto(172.52524262,358.7231295)(172.63024252,358.7181295)(172.72024949,358.70813724)
\curveto(172.82024233,358.69812952)(172.89524225,358.65312957)(172.94524949,358.57313724)
\curveto(172.99524215,358.5231297)(173.02024213,358.44812977)(173.02024949,358.34813724)
\lineto(173.02024949,358.06313724)
\lineto(173.02024949,357.04313724)
\lineto(173.02024949,353.00813724)
\lineto(173.02024949,351.65813724)
\curveto(173.02024213,351.53813668)(173.01524213,351.4231368)(173.00524949,351.31313724)
\curveto(173.00524214,351.21313701)(172.97024218,351.13813708)(172.90024949,351.08813724)
\curveto(172.86024229,351.05813716)(172.80024235,351.03313719)(172.72024949,351.01313724)
\curveto(172.64024251,351.00313722)(172.5502426,350.99313723)(172.45024949,350.98313724)
\curveto(172.36024279,350.98313724)(172.27024288,350.98813723)(172.18024949,350.99813724)
\curveto(172.10024305,351.00813721)(172.04024311,351.02813719)(172.00024949,351.05813724)
\curveto(171.9502432,351.09813712)(171.90524324,351.16313706)(171.86524949,351.25313724)
\curveto(171.85524329,351.29313693)(171.8452433,351.34813687)(171.83524949,351.41813724)
\curveto(171.83524331,351.48813673)(171.83024332,351.55313667)(171.82024949,351.61313724)
\curveto(171.81024334,351.68313654)(171.79024336,351.73813648)(171.76024949,351.77813724)
\curveto(171.73024342,351.8181364)(171.68524346,351.83313639)(171.62524949,351.82313724)
\curveto(171.5452436,351.80313642)(171.46524368,351.74313648)(171.38524949,351.64313724)
\curveto(171.30524384,351.55313667)(171.23024392,351.48313674)(171.16024949,351.43313724)
\curveto(170.94024421,351.27313695)(170.69024446,351.13313709)(170.41024949,351.01313724)
\curveto(170.30024485,350.96313726)(170.18524496,350.93313729)(170.06524949,350.92313724)
\curveto(169.95524519,350.90313732)(169.84024531,350.87813734)(169.72024949,350.84813724)
\curveto(169.67024548,350.83813738)(169.61524553,350.83813738)(169.55524949,350.84813724)
\curveto(169.50524564,350.85813736)(169.45524569,350.85313737)(169.40524949,350.83313724)
\curveto(169.30524584,350.81313741)(169.21524593,350.81313741)(169.13524949,350.83313724)
\lineto(168.98524949,350.83313724)
\curveto(168.93524621,350.85313737)(168.87524627,350.86313736)(168.80524949,350.86313724)
\curveto(168.7452464,350.86313736)(168.69024646,350.86813735)(168.64024949,350.87813724)
\curveto(168.60024655,350.89813732)(168.56024659,350.90813731)(168.52024949,350.90813724)
\curveto(168.49024666,350.89813732)(168.4502467,350.90313732)(168.40024949,350.92313724)
\lineto(168.16024949,350.98313724)
\curveto(168.09024706,351.00313722)(168.01524713,351.03313719)(167.93524949,351.07313724)
\curveto(167.67524747,351.18313704)(167.45524769,351.32813689)(167.27524949,351.50813724)
\curveto(167.10524804,351.69813652)(166.96524818,351.9231363)(166.85524949,352.18313724)
\curveto(166.81524833,352.27313595)(166.78524836,352.36313586)(166.76524949,352.45313724)
\lineto(166.70524949,352.75313724)
\curveto(166.68524846,352.81313541)(166.67524847,352.86813535)(166.67524949,352.91813724)
\curveto(166.68524846,352.97813524)(166.68024847,353.04313518)(166.66024949,353.11313724)
\curveto(166.6502485,353.13313509)(166.6452485,353.15813506)(166.64524949,353.18813724)
\curveto(166.6452485,353.22813499)(166.64024851,353.26313496)(166.63024949,353.29313724)
\lineto(166.63024949,353.44313724)
\curveto(166.62024853,353.48313474)(166.61524853,353.52813469)(166.61524949,353.57813724)
\curveto(166.62524852,353.63813458)(166.63024852,353.69313453)(166.63024949,353.74313724)
\lineto(166.63024949,354.34313724)
\lineto(166.63024949,357.10313724)
\lineto(166.63024949,358.06313724)
\lineto(166.63024949,358.33313724)
\curveto(166.63024852,358.4231298)(166.6502485,358.49812972)(166.69024949,358.55813724)
\curveto(166.73024842,358.62812959)(166.80524834,358.67812954)(166.91524949,358.70813724)
\curveto(166.93524821,358.7181295)(166.95524819,358.7181295)(166.97524949,358.70813724)
\curveto(166.99524815,358.70812951)(167.01524813,358.71312951)(167.03524949,358.72313724)
}
}
{
\newrgbcolor{curcolor}{0 0 0}
\pscustom[linestyle=none,fillstyle=solid,fillcolor=curcolor]
{
\newpath
\moveto(178.56485887,358.90313724)
\curveto(178.79485408,358.90312932)(178.92485395,358.84312938)(178.95485887,358.72313724)
\curveto(178.98485389,358.61312961)(178.99985387,358.44812977)(178.99985887,358.22813724)
\lineto(178.99985887,357.94313724)
\curveto(178.99985387,357.85313037)(178.9748539,357.77813044)(178.92485887,357.71813724)
\curveto(178.86485401,357.63813058)(178.77985409,357.59313063)(178.66985887,357.58313724)
\curveto(178.55985431,357.58313064)(178.44985442,357.56813065)(178.33985887,357.53813724)
\curveto(178.19985467,357.50813071)(178.06485481,357.47813074)(177.93485887,357.44813724)
\curveto(177.81485506,357.4181308)(177.69985517,357.37813084)(177.58985887,357.32813724)
\curveto(177.29985557,357.19813102)(177.06485581,357.0181312)(176.88485887,356.78813724)
\curveto(176.70485617,356.56813165)(176.54985632,356.31313191)(176.41985887,356.02313724)
\curveto(176.37985649,355.91313231)(176.34985652,355.79813242)(176.32985887,355.67813724)
\curveto(176.30985656,355.56813265)(176.28485659,355.45313277)(176.25485887,355.33313724)
\curveto(176.24485663,355.28313294)(176.23985663,355.23313299)(176.23985887,355.18313724)
\curveto(176.24985662,355.13313309)(176.24985662,355.08313314)(176.23985887,355.03313724)
\curveto(176.20985666,354.91313331)(176.19485668,354.77313345)(176.19485887,354.61313724)
\curveto(176.20485667,354.46313376)(176.20985666,354.3181339)(176.20985887,354.17813724)
\lineto(176.20985887,352.33313724)
\lineto(176.20985887,351.98813724)
\curveto(176.20985666,351.86813635)(176.20485667,351.75313647)(176.19485887,351.64313724)
\curveto(176.18485669,351.53313669)(176.17985669,351.43813678)(176.17985887,351.35813724)
\curveto(176.18985668,351.27813694)(176.1698567,351.20813701)(176.11985887,351.14813724)
\curveto(176.0698568,351.07813714)(175.98985688,351.03813718)(175.87985887,351.02813724)
\curveto(175.77985709,351.0181372)(175.6698572,351.01313721)(175.54985887,351.01313724)
\lineto(175.27985887,351.01313724)
\curveto(175.22985764,351.03313719)(175.17985769,351.04813717)(175.12985887,351.05813724)
\curveto(175.08985778,351.07813714)(175.05985781,351.10313712)(175.03985887,351.13313724)
\curveto(174.98985788,351.20313702)(174.95985791,351.28813693)(174.94985887,351.38813724)
\lineto(174.94985887,351.71813724)
\lineto(174.94985887,352.87313724)
\lineto(174.94985887,357.02813724)
\lineto(174.94985887,358.06313724)
\lineto(174.94985887,358.36313724)
\curveto(174.95985791,358.46312976)(174.98985788,358.54812967)(175.03985887,358.61813724)
\curveto(175.0698578,358.65812956)(175.11985775,358.68812953)(175.18985887,358.70813724)
\curveto(175.2698576,358.72812949)(175.35485752,358.73812948)(175.44485887,358.73813724)
\curveto(175.53485734,358.74812947)(175.62485725,358.74812947)(175.71485887,358.73813724)
\curveto(175.80485707,358.72812949)(175.874857,358.71312951)(175.92485887,358.69313724)
\curveto(176.00485687,358.66312956)(176.05485682,358.60312962)(176.07485887,358.51313724)
\curveto(176.10485677,358.43312979)(176.11985675,358.34312988)(176.11985887,358.24313724)
\lineto(176.11985887,357.94313724)
\curveto(176.11985675,357.84313038)(176.13985673,357.75313047)(176.17985887,357.67313724)
\curveto(176.18985668,357.65313057)(176.19985667,357.63813058)(176.20985887,357.62813724)
\lineto(176.25485887,357.58313724)
\curveto(176.36485651,357.58313064)(176.45485642,357.62813059)(176.52485887,357.71813724)
\curveto(176.59485628,357.8181304)(176.65485622,357.89813032)(176.70485887,357.95813724)
\lineto(176.79485887,358.04813724)
\curveto(176.88485599,358.15813006)(177.00985586,358.27312995)(177.16985887,358.39313724)
\curveto(177.32985554,358.51312971)(177.47985539,358.60312962)(177.61985887,358.66313724)
\curveto(177.70985516,358.71312951)(177.80485507,358.74812947)(177.90485887,358.76813724)
\curveto(178.00485487,358.79812942)(178.10985476,358.82812939)(178.21985887,358.85813724)
\curveto(178.27985459,358.86812935)(178.33985453,358.87312935)(178.39985887,358.87313724)
\curveto(178.45985441,358.88312934)(178.51485436,358.89312933)(178.56485887,358.90313724)
}
}
{
\newrgbcolor{curcolor}{0 0 0}
\pscustom[linestyle=none,fillstyle=solid,fillcolor=curcolor]
{
\newpath
\moveto(182.35962449,358.90313724)
\curveto(183.07962043,358.91312931)(183.68461982,358.82812939)(184.17462449,358.64813724)
\curveto(184.66461884,358.47812974)(185.04461846,358.17313005)(185.31462449,357.73313724)
\curveto(185.38461812,357.6231306)(185.43961807,357.50813071)(185.47962449,357.38813724)
\curveto(185.51961799,357.27813094)(185.55961795,357.15313107)(185.59962449,357.01313724)
\curveto(185.61961789,356.94313128)(185.62461788,356.86813135)(185.61462449,356.78813724)
\curveto(185.6046179,356.7181315)(185.58961792,356.66313156)(185.56962449,356.62313724)
\curveto(185.54961796,356.60313162)(185.52461798,356.58313164)(185.49462449,356.56313724)
\curveto(185.46461804,356.55313167)(185.43961807,356.53813168)(185.41962449,356.51813724)
\curveto(185.36961814,356.49813172)(185.31961819,356.49313173)(185.26962449,356.50313724)
\curveto(185.21961829,356.51313171)(185.16961834,356.51313171)(185.11962449,356.50313724)
\curveto(185.03961847,356.48313174)(184.93461857,356.47813174)(184.80462449,356.48813724)
\curveto(184.67461883,356.50813171)(184.58461892,356.53313169)(184.53462449,356.56313724)
\curveto(184.45461905,356.61313161)(184.39961911,356.67813154)(184.36962449,356.75813724)
\curveto(184.34961916,356.84813137)(184.31461919,356.93313129)(184.26462449,357.01313724)
\curveto(184.17461933,357.17313105)(184.04961946,357.3181309)(183.88962449,357.44813724)
\curveto(183.77961973,357.52813069)(183.65961985,357.58813063)(183.52962449,357.62813724)
\curveto(183.39962011,357.66813055)(183.25962025,357.70813051)(183.10962449,357.74813724)
\curveto(183.05962045,357.76813045)(183.0096205,357.77313045)(182.95962449,357.76313724)
\curveto(182.9096206,357.76313046)(182.85962065,357.76813045)(182.80962449,357.77813724)
\curveto(182.74962076,357.79813042)(182.67462083,357.80813041)(182.58462449,357.80813724)
\curveto(182.49462101,357.80813041)(182.41962109,357.79813042)(182.35962449,357.77813724)
\lineto(182.26962449,357.77813724)
\lineto(182.11962449,357.74813724)
\curveto(182.06962144,357.74813047)(182.01962149,357.74313048)(181.96962449,357.73313724)
\curveto(181.7096218,357.67313055)(181.49462201,357.58813063)(181.32462449,357.47813724)
\curveto(181.15462235,357.36813085)(181.03962247,357.18313104)(180.97962449,356.92313724)
\curveto(180.95962255,356.85313137)(180.95462255,356.78313144)(180.96462449,356.71313724)
\curveto(180.98462252,356.64313158)(181.0046225,356.58313164)(181.02462449,356.53313724)
\curveto(181.08462242,356.38313184)(181.15462235,356.27313195)(181.23462449,356.20313724)
\curveto(181.32462218,356.14313208)(181.43462207,356.07313215)(181.56462449,355.99313724)
\curveto(181.72462178,355.89313233)(181.9046216,355.8181324)(182.10462449,355.76813724)
\curveto(182.3046212,355.72813249)(182.504621,355.67813254)(182.70462449,355.61813724)
\curveto(182.83462067,355.57813264)(182.96462054,355.54813267)(183.09462449,355.52813724)
\curveto(183.22462028,355.50813271)(183.35462015,355.47813274)(183.48462449,355.43813724)
\curveto(183.69461981,355.37813284)(183.89961961,355.3181329)(184.09962449,355.25813724)
\curveto(184.29961921,355.20813301)(184.49961901,355.14313308)(184.69962449,355.06313724)
\lineto(184.84962449,355.00313724)
\curveto(184.89961861,354.98313324)(184.94961856,354.95813326)(184.99962449,354.92813724)
\curveto(185.19961831,354.80813341)(185.37461813,354.67313355)(185.52462449,354.52313724)
\curveto(185.67461783,354.37313385)(185.79961771,354.18313404)(185.89962449,353.95313724)
\curveto(185.91961759,353.88313434)(185.93961757,353.78813443)(185.95962449,353.66813724)
\curveto(185.97961753,353.59813462)(185.98961752,353.5231347)(185.98962449,353.44313724)
\curveto(185.99961751,353.37313485)(186.0046175,353.29313493)(186.00462449,353.20313724)
\lineto(186.00462449,353.05313724)
\curveto(185.98461752,352.98313524)(185.97461753,352.91313531)(185.97462449,352.84313724)
\curveto(185.97461753,352.77313545)(185.96461754,352.70313552)(185.94462449,352.63313724)
\curveto(185.91461759,352.5231357)(185.87961763,352.4181358)(185.83962449,352.31813724)
\curveto(185.79961771,352.218136)(185.75461775,352.12813609)(185.70462449,352.04813724)
\curveto(185.54461796,351.78813643)(185.33961817,351.57813664)(185.08962449,351.41813724)
\curveto(184.83961867,351.26813695)(184.55961895,351.13813708)(184.24962449,351.02813724)
\curveto(184.15961935,350.99813722)(184.06461944,350.97813724)(183.96462449,350.96813724)
\curveto(183.87461963,350.94813727)(183.78461972,350.9231373)(183.69462449,350.89313724)
\curveto(183.59461991,350.87313735)(183.49462001,350.86313736)(183.39462449,350.86313724)
\curveto(183.29462021,350.86313736)(183.19462031,350.85313737)(183.09462449,350.83313724)
\lineto(182.94462449,350.83313724)
\curveto(182.89462061,350.8231374)(182.82462068,350.8181374)(182.73462449,350.81813724)
\curveto(182.64462086,350.8181374)(182.57462093,350.8231374)(182.52462449,350.83313724)
\lineto(182.35962449,350.83313724)
\curveto(182.29962121,350.85313737)(182.23462127,350.86313736)(182.16462449,350.86313724)
\curveto(182.09462141,350.85313737)(182.03462147,350.85813736)(181.98462449,350.87813724)
\curveto(181.93462157,350.88813733)(181.86962164,350.89313733)(181.78962449,350.89313724)
\lineto(181.54962449,350.95313724)
\curveto(181.47962203,350.96313726)(181.4046221,350.98313724)(181.32462449,351.01313724)
\curveto(181.01462249,351.11313711)(180.74462276,351.23813698)(180.51462449,351.38813724)
\curveto(180.28462322,351.53813668)(180.08462342,351.73313649)(179.91462449,351.97313724)
\curveto(179.82462368,352.10313612)(179.74962376,352.23813598)(179.68962449,352.37813724)
\curveto(179.62962388,352.5181357)(179.57462393,352.67313555)(179.52462449,352.84313724)
\curveto(179.504624,352.90313532)(179.49462401,352.97313525)(179.49462449,353.05313724)
\curveto(179.504624,353.14313508)(179.51962399,353.21313501)(179.53962449,353.26313724)
\curveto(179.56962394,353.30313492)(179.61962389,353.34313488)(179.68962449,353.38313724)
\curveto(179.73962377,353.40313482)(179.8096237,353.41313481)(179.89962449,353.41313724)
\curveto(179.98962352,353.4231348)(180.07962343,353.4231348)(180.16962449,353.41313724)
\curveto(180.25962325,353.40313482)(180.34462316,353.38813483)(180.42462449,353.36813724)
\curveto(180.51462299,353.35813486)(180.57462293,353.34313488)(180.60462449,353.32313724)
\curveto(180.67462283,353.27313495)(180.71962279,353.19813502)(180.73962449,353.09813724)
\curveto(180.76962274,353.00813521)(180.8046227,352.9231353)(180.84462449,352.84313724)
\curveto(180.94462256,352.6231356)(181.07962243,352.45313577)(181.24962449,352.33313724)
\curveto(181.36962214,352.24313598)(181.504622,352.17313605)(181.65462449,352.12313724)
\curveto(181.8046217,352.07313615)(181.96462154,352.0231362)(182.13462449,351.97313724)
\lineto(182.44962449,351.92813724)
\lineto(182.53962449,351.92813724)
\curveto(182.6096209,351.90813631)(182.69962081,351.89813632)(182.80962449,351.89813724)
\curveto(182.92962058,351.89813632)(183.02962048,351.90813631)(183.10962449,351.92813724)
\curveto(183.17962033,351.92813629)(183.23462027,351.93313629)(183.27462449,351.94313724)
\curveto(183.33462017,351.95313627)(183.39462011,351.95813626)(183.45462449,351.95813724)
\curveto(183.51461999,351.96813625)(183.56961994,351.97813624)(183.61962449,351.98813724)
\curveto(183.9096196,352.06813615)(184.13961937,352.17313605)(184.30962449,352.30313724)
\curveto(184.47961903,352.43313579)(184.59961891,352.65313557)(184.66962449,352.96313724)
\curveto(184.68961882,353.01313521)(184.69461881,353.06813515)(184.68462449,353.12813724)
\curveto(184.67461883,353.18813503)(184.66461884,353.23313499)(184.65462449,353.26313724)
\curveto(184.6046189,353.45313477)(184.53461897,353.59313463)(184.44462449,353.68313724)
\curveto(184.35461915,353.78313444)(184.23961927,353.87313435)(184.09962449,353.95313724)
\curveto(184.0096195,354.01313421)(183.9096196,354.06313416)(183.79962449,354.10313724)
\lineto(183.46962449,354.22313724)
\curveto(183.43962007,354.23313399)(183.4096201,354.23813398)(183.37962449,354.23813724)
\curveto(183.35962015,354.23813398)(183.33462017,354.24813397)(183.30462449,354.26813724)
\curveto(182.96462054,354.37813384)(182.6096209,354.45813376)(182.23962449,354.50813724)
\curveto(181.87962163,354.56813365)(181.53962197,354.66313356)(181.21962449,354.79313724)
\curveto(181.11962239,354.83313339)(181.02462248,354.86813335)(180.93462449,354.89813724)
\curveto(180.84462266,354.92813329)(180.75962275,354.96813325)(180.67962449,355.01813724)
\curveto(180.48962302,355.12813309)(180.31462319,355.25313297)(180.15462449,355.39313724)
\curveto(179.99462351,355.53313269)(179.86962364,355.70813251)(179.77962449,355.91813724)
\curveto(179.74962376,355.98813223)(179.72462378,356.05813216)(179.70462449,356.12813724)
\curveto(179.69462381,356.19813202)(179.67962383,356.27313195)(179.65962449,356.35313724)
\curveto(179.62962388,356.47313175)(179.61962389,356.60813161)(179.62962449,356.75813724)
\curveto(179.63962387,356.9181313)(179.65462385,357.05313117)(179.67462449,357.16313724)
\curveto(179.69462381,357.21313101)(179.7046238,357.25313097)(179.70462449,357.28313724)
\curveto(179.71462379,357.3231309)(179.72962378,357.36313086)(179.74962449,357.40313724)
\curveto(179.83962367,357.63313059)(179.95962355,357.83313039)(180.10962449,358.00313724)
\curveto(180.26962324,358.17313005)(180.44962306,358.3231299)(180.64962449,358.45313724)
\curveto(180.79962271,358.54312968)(180.96462254,358.61312961)(181.14462449,358.66313724)
\curveto(181.32462218,358.7231295)(181.51462199,358.77812944)(181.71462449,358.82813724)
\curveto(181.78462172,358.83812938)(181.84962166,358.84812937)(181.90962449,358.85813724)
\curveto(181.97962153,358.86812935)(182.05462145,358.87812934)(182.13462449,358.88813724)
\curveto(182.16462134,358.89812932)(182.2046213,358.89812932)(182.25462449,358.88813724)
\curveto(182.3046212,358.87812934)(182.33962117,358.88312934)(182.35962449,358.90313724)
}
}
{
\newrgbcolor{curcolor}{0 0 0}
\pscustom[linestyle=none,fillstyle=solid,fillcolor=curcolor]
{
\newpath
\moveto(194.55462449,355.19813724)
\curveto(194.57461643,355.13813308)(194.58461642,355.04313318)(194.58462449,354.91313724)
\curveto(194.58461642,354.79313343)(194.57961643,354.70813351)(194.56962449,354.65813724)
\lineto(194.56962449,354.50813724)
\curveto(194.55961645,354.42813379)(194.54961646,354.35313387)(194.53962449,354.28313724)
\curveto(194.53961647,354.223134)(194.53461647,354.15313407)(194.52462449,354.07313724)
\curveto(194.5046165,354.01313421)(194.48961652,353.95313427)(194.47962449,353.89313724)
\curveto(194.47961653,353.83313439)(194.46961654,353.77313445)(194.44962449,353.71313724)
\curveto(194.4096166,353.58313464)(194.37461663,353.45313477)(194.34462449,353.32313724)
\curveto(194.31461669,353.19313503)(194.27461673,353.07313515)(194.22462449,352.96313724)
\curveto(194.01461699,352.48313574)(193.73461727,352.07813614)(193.38462449,351.74813724)
\curveto(193.03461797,351.42813679)(192.6046184,351.18313704)(192.09462449,351.01313724)
\curveto(191.98461902,350.97313725)(191.86461914,350.94313728)(191.73462449,350.92313724)
\curveto(191.61461939,350.90313732)(191.48961952,350.88313734)(191.35962449,350.86313724)
\curveto(191.29961971,350.85313737)(191.23461977,350.84813737)(191.16462449,350.84813724)
\curveto(191.1046199,350.83813738)(191.04461996,350.83313739)(190.98462449,350.83313724)
\curveto(190.94462006,350.8231374)(190.88462012,350.8181374)(190.80462449,350.81813724)
\curveto(190.73462027,350.8181374)(190.68462032,350.8231374)(190.65462449,350.83313724)
\curveto(190.61462039,350.84313738)(190.57462043,350.84813737)(190.53462449,350.84813724)
\curveto(190.49462051,350.83813738)(190.45962055,350.83813738)(190.42962449,350.84813724)
\lineto(190.33962449,350.84813724)
\lineto(189.97962449,350.89313724)
\curveto(189.83962117,350.93313729)(189.7046213,350.97313725)(189.57462449,351.01313724)
\curveto(189.44462156,351.05313717)(189.31962169,351.09813712)(189.19962449,351.14813724)
\curveto(188.74962226,351.34813687)(188.37962263,351.60813661)(188.08962449,351.92813724)
\curveto(187.79962321,352.24813597)(187.55962345,352.63813558)(187.36962449,353.09813724)
\curveto(187.31962369,353.19813502)(187.27962373,353.29813492)(187.24962449,353.39813724)
\curveto(187.22962378,353.49813472)(187.2096238,353.60313462)(187.18962449,353.71313724)
\curveto(187.16962384,353.75313447)(187.15962385,353.78313444)(187.15962449,353.80313724)
\curveto(187.16962384,353.83313439)(187.16962384,353.86813435)(187.15962449,353.90813724)
\curveto(187.13962387,353.98813423)(187.12462388,354.06813415)(187.11462449,354.14813724)
\curveto(187.11462389,354.23813398)(187.1046239,354.3231339)(187.08462449,354.40313724)
\lineto(187.08462449,354.52313724)
\curveto(187.08462392,354.56313366)(187.07962393,354.60813361)(187.06962449,354.65813724)
\curveto(187.05962395,354.70813351)(187.05462395,354.79313343)(187.05462449,354.91313724)
\curveto(187.05462395,355.04313318)(187.06462394,355.13813308)(187.08462449,355.19813724)
\curveto(187.1046239,355.26813295)(187.1096239,355.33813288)(187.09962449,355.40813724)
\curveto(187.08962392,355.47813274)(187.09462391,355.54813267)(187.11462449,355.61813724)
\curveto(187.12462388,355.66813255)(187.12962388,355.70813251)(187.12962449,355.73813724)
\curveto(187.13962387,355.77813244)(187.14962386,355.8231324)(187.15962449,355.87313724)
\curveto(187.18962382,355.99313223)(187.21462379,356.11313211)(187.23462449,356.23313724)
\curveto(187.26462374,356.35313187)(187.3046237,356.46813175)(187.35462449,356.57813724)
\curveto(187.5046235,356.94813127)(187.68462332,357.27813094)(187.89462449,357.56813724)
\curveto(188.11462289,357.86813035)(188.37962263,358.1181301)(188.68962449,358.31813724)
\curveto(188.8096222,358.39812982)(188.93462207,358.46312976)(189.06462449,358.51313724)
\curveto(189.19462181,358.57312965)(189.32962168,358.63312959)(189.46962449,358.69313724)
\curveto(189.58962142,358.74312948)(189.71962129,358.77312945)(189.85962449,358.78313724)
\curveto(189.99962101,358.80312942)(190.13962087,358.83312939)(190.27962449,358.87313724)
\lineto(190.47462449,358.87313724)
\curveto(190.54462046,358.88312934)(190.6096204,358.89312933)(190.66962449,358.90313724)
\curveto(191.55961945,358.91312931)(192.29961871,358.72812949)(192.88962449,358.34813724)
\curveto(193.47961753,357.96813025)(193.9046171,357.47313075)(194.16462449,356.86313724)
\curveto(194.21461679,356.76313146)(194.25461675,356.66313156)(194.28462449,356.56313724)
\curveto(194.31461669,356.46313176)(194.34961666,356.35813186)(194.38962449,356.24813724)
\curveto(194.41961659,356.13813208)(194.44461656,356.0181322)(194.46462449,355.88813724)
\curveto(194.48461652,355.76813245)(194.5096165,355.64313258)(194.53962449,355.51313724)
\curveto(194.54961646,355.46313276)(194.54961646,355.40813281)(194.53962449,355.34813724)
\curveto(194.53961647,355.29813292)(194.54461646,355.24813297)(194.55462449,355.19813724)
\moveto(193.21962449,354.34313724)
\curveto(193.23961777,354.41313381)(193.24461776,354.49313373)(193.23462449,354.58313724)
\lineto(193.23462449,354.83813724)
\curveto(193.23461777,355.22813299)(193.19961781,355.55813266)(193.12962449,355.82813724)
\curveto(193.09961791,355.90813231)(193.07461793,355.98813223)(193.05462449,356.06813724)
\curveto(193.03461797,356.14813207)(193.009618,356.223132)(192.97962449,356.29313724)
\curveto(192.69961831,356.94313128)(192.25461875,357.39313083)(191.64462449,357.64313724)
\curveto(191.57461943,357.67313055)(191.49961951,357.69313053)(191.41962449,357.70313724)
\lineto(191.17962449,357.76313724)
\curveto(191.09961991,357.78313044)(191.01461999,357.79313043)(190.92462449,357.79313724)
\lineto(190.65462449,357.79313724)
\lineto(190.38462449,357.74813724)
\curveto(190.28462072,357.72813049)(190.18962082,357.70313052)(190.09962449,357.67313724)
\curveto(190.01962099,357.65313057)(189.93962107,357.6231306)(189.85962449,357.58313724)
\curveto(189.78962122,357.56313066)(189.72462128,357.53313069)(189.66462449,357.49313724)
\curveto(189.6046214,357.45313077)(189.54962146,357.41313081)(189.49962449,357.37313724)
\curveto(189.25962175,357.20313102)(189.06462194,356.99813122)(188.91462449,356.75813724)
\curveto(188.76462224,356.5181317)(188.63462237,356.23813198)(188.52462449,355.91813724)
\curveto(188.49462251,355.8181324)(188.47462253,355.71313251)(188.46462449,355.60313724)
\curveto(188.45462255,355.50313272)(188.43962257,355.39813282)(188.41962449,355.28813724)
\curveto(188.4096226,355.24813297)(188.4046226,355.18313304)(188.40462449,355.09313724)
\curveto(188.39462261,355.06313316)(188.38962262,355.02813319)(188.38962449,354.98813724)
\curveto(188.39962261,354.94813327)(188.4046226,354.90313332)(188.40462449,354.85313724)
\lineto(188.40462449,354.55313724)
\curveto(188.4046226,354.45313377)(188.41462259,354.36313386)(188.43462449,354.28313724)
\lineto(188.46462449,354.10313724)
\curveto(188.48462252,354.00313422)(188.49962251,353.90313432)(188.50962449,353.80313724)
\curveto(188.52962248,353.71313451)(188.55962245,353.62813459)(188.59962449,353.54813724)
\curveto(188.69962231,353.30813491)(188.81462219,353.08313514)(188.94462449,352.87313724)
\curveto(189.08462192,352.66313556)(189.25462175,352.48813573)(189.45462449,352.34813724)
\curveto(189.5046215,352.3181359)(189.54962146,352.29313593)(189.58962449,352.27313724)
\curveto(189.62962138,352.25313597)(189.67462133,352.22813599)(189.72462449,352.19813724)
\curveto(189.8046212,352.14813607)(189.88962112,352.10313612)(189.97962449,352.06313724)
\curveto(190.07962093,352.03313619)(190.18462082,352.00313622)(190.29462449,351.97313724)
\curveto(190.34462066,351.95313627)(190.38962062,351.94313628)(190.42962449,351.94313724)
\curveto(190.47962053,351.95313627)(190.52962048,351.95313627)(190.57962449,351.94313724)
\curveto(190.6096204,351.93313629)(190.66962034,351.9231363)(190.75962449,351.91313724)
\curveto(190.85962015,351.90313632)(190.93462007,351.90813631)(190.98462449,351.92813724)
\curveto(191.02461998,351.93813628)(191.06461994,351.93813628)(191.10462449,351.92813724)
\curveto(191.14461986,351.92813629)(191.18461982,351.93813628)(191.22462449,351.95813724)
\curveto(191.3046197,351.97813624)(191.38461962,351.99313623)(191.46462449,352.00313724)
\curveto(191.54461946,352.0231362)(191.61961939,352.04813617)(191.68962449,352.07813724)
\curveto(192.02961898,352.218136)(192.3046187,352.41313581)(192.51462449,352.66313724)
\curveto(192.72461828,352.91313531)(192.89961811,353.20813501)(193.03962449,353.54813724)
\curveto(193.08961792,353.66813455)(193.11961789,353.79313443)(193.12962449,353.92313724)
\curveto(193.14961786,354.06313416)(193.17961783,354.20313402)(193.21962449,354.34313724)
}
}
{
\newrgbcolor{curcolor}{0 0 0}
\pscustom[linestyle=none,fillstyle=solid,fillcolor=curcolor]
{
\newpath
\moveto(198.47290574,358.90313724)
\curveto(199.19290168,358.91312931)(199.79790107,358.82812939)(200.28790574,358.64813724)
\curveto(200.77790009,358.47812974)(201.15789971,358.17313005)(201.42790574,357.73313724)
\curveto(201.49789937,357.6231306)(201.55289932,357.50813071)(201.59290574,357.38813724)
\curveto(201.63289924,357.27813094)(201.6728992,357.15313107)(201.71290574,357.01313724)
\curveto(201.73289914,356.94313128)(201.73789913,356.86813135)(201.72790574,356.78813724)
\curveto(201.71789915,356.7181315)(201.70289917,356.66313156)(201.68290574,356.62313724)
\curveto(201.66289921,356.60313162)(201.63789923,356.58313164)(201.60790574,356.56313724)
\curveto(201.57789929,356.55313167)(201.55289932,356.53813168)(201.53290574,356.51813724)
\curveto(201.48289939,356.49813172)(201.43289944,356.49313173)(201.38290574,356.50313724)
\curveto(201.33289954,356.51313171)(201.28289959,356.51313171)(201.23290574,356.50313724)
\curveto(201.15289972,356.48313174)(201.04789982,356.47813174)(200.91790574,356.48813724)
\curveto(200.78790008,356.50813171)(200.69790017,356.53313169)(200.64790574,356.56313724)
\curveto(200.5679003,356.61313161)(200.51290036,356.67813154)(200.48290574,356.75813724)
\curveto(200.46290041,356.84813137)(200.42790044,356.93313129)(200.37790574,357.01313724)
\curveto(200.28790058,357.17313105)(200.16290071,357.3181309)(200.00290574,357.44813724)
\curveto(199.89290098,357.52813069)(199.7729011,357.58813063)(199.64290574,357.62813724)
\curveto(199.51290136,357.66813055)(199.3729015,357.70813051)(199.22290574,357.74813724)
\curveto(199.1729017,357.76813045)(199.12290175,357.77313045)(199.07290574,357.76313724)
\curveto(199.02290185,357.76313046)(198.9729019,357.76813045)(198.92290574,357.77813724)
\curveto(198.86290201,357.79813042)(198.78790208,357.80813041)(198.69790574,357.80813724)
\curveto(198.60790226,357.80813041)(198.53290234,357.79813042)(198.47290574,357.77813724)
\lineto(198.38290574,357.77813724)
\lineto(198.23290574,357.74813724)
\curveto(198.18290269,357.74813047)(198.13290274,357.74313048)(198.08290574,357.73313724)
\curveto(197.82290305,357.67313055)(197.60790326,357.58813063)(197.43790574,357.47813724)
\curveto(197.2679036,357.36813085)(197.15290372,357.18313104)(197.09290574,356.92313724)
\curveto(197.0729038,356.85313137)(197.0679038,356.78313144)(197.07790574,356.71313724)
\curveto(197.09790377,356.64313158)(197.11790375,356.58313164)(197.13790574,356.53313724)
\curveto(197.19790367,356.38313184)(197.2679036,356.27313195)(197.34790574,356.20313724)
\curveto(197.43790343,356.14313208)(197.54790332,356.07313215)(197.67790574,355.99313724)
\curveto(197.83790303,355.89313233)(198.01790285,355.8181324)(198.21790574,355.76813724)
\curveto(198.41790245,355.72813249)(198.61790225,355.67813254)(198.81790574,355.61813724)
\curveto(198.94790192,355.57813264)(199.07790179,355.54813267)(199.20790574,355.52813724)
\curveto(199.33790153,355.50813271)(199.4679014,355.47813274)(199.59790574,355.43813724)
\curveto(199.80790106,355.37813284)(200.01290086,355.3181329)(200.21290574,355.25813724)
\curveto(200.41290046,355.20813301)(200.61290026,355.14313308)(200.81290574,355.06313724)
\lineto(200.96290574,355.00313724)
\curveto(201.01289986,354.98313324)(201.06289981,354.95813326)(201.11290574,354.92813724)
\curveto(201.31289956,354.80813341)(201.48789938,354.67313355)(201.63790574,354.52313724)
\curveto(201.78789908,354.37313385)(201.91289896,354.18313404)(202.01290574,353.95313724)
\curveto(202.03289884,353.88313434)(202.05289882,353.78813443)(202.07290574,353.66813724)
\curveto(202.09289878,353.59813462)(202.10289877,353.5231347)(202.10290574,353.44313724)
\curveto(202.11289876,353.37313485)(202.11789875,353.29313493)(202.11790574,353.20313724)
\lineto(202.11790574,353.05313724)
\curveto(202.09789877,352.98313524)(202.08789878,352.91313531)(202.08790574,352.84313724)
\curveto(202.08789878,352.77313545)(202.07789879,352.70313552)(202.05790574,352.63313724)
\curveto(202.02789884,352.5231357)(201.99289888,352.4181358)(201.95290574,352.31813724)
\curveto(201.91289896,352.218136)(201.867899,352.12813609)(201.81790574,352.04813724)
\curveto(201.65789921,351.78813643)(201.45289942,351.57813664)(201.20290574,351.41813724)
\curveto(200.95289992,351.26813695)(200.6729002,351.13813708)(200.36290574,351.02813724)
\curveto(200.2729006,350.99813722)(200.17790069,350.97813724)(200.07790574,350.96813724)
\curveto(199.98790088,350.94813727)(199.89790097,350.9231373)(199.80790574,350.89313724)
\curveto(199.70790116,350.87313735)(199.60790126,350.86313736)(199.50790574,350.86313724)
\curveto(199.40790146,350.86313736)(199.30790156,350.85313737)(199.20790574,350.83313724)
\lineto(199.05790574,350.83313724)
\curveto(199.00790186,350.8231374)(198.93790193,350.8181374)(198.84790574,350.81813724)
\curveto(198.75790211,350.8181374)(198.68790218,350.8231374)(198.63790574,350.83313724)
\lineto(198.47290574,350.83313724)
\curveto(198.41290246,350.85313737)(198.34790252,350.86313736)(198.27790574,350.86313724)
\curveto(198.20790266,350.85313737)(198.14790272,350.85813736)(198.09790574,350.87813724)
\curveto(198.04790282,350.88813733)(197.98290289,350.89313733)(197.90290574,350.89313724)
\lineto(197.66290574,350.95313724)
\curveto(197.59290328,350.96313726)(197.51790335,350.98313724)(197.43790574,351.01313724)
\curveto(197.12790374,351.11313711)(196.85790401,351.23813698)(196.62790574,351.38813724)
\curveto(196.39790447,351.53813668)(196.19790467,351.73313649)(196.02790574,351.97313724)
\curveto(195.93790493,352.10313612)(195.86290501,352.23813598)(195.80290574,352.37813724)
\curveto(195.74290513,352.5181357)(195.68790518,352.67313555)(195.63790574,352.84313724)
\curveto(195.61790525,352.90313532)(195.60790526,352.97313525)(195.60790574,353.05313724)
\curveto(195.61790525,353.14313508)(195.63290524,353.21313501)(195.65290574,353.26313724)
\curveto(195.68290519,353.30313492)(195.73290514,353.34313488)(195.80290574,353.38313724)
\curveto(195.85290502,353.40313482)(195.92290495,353.41313481)(196.01290574,353.41313724)
\curveto(196.10290477,353.4231348)(196.19290468,353.4231348)(196.28290574,353.41313724)
\curveto(196.3729045,353.40313482)(196.45790441,353.38813483)(196.53790574,353.36813724)
\curveto(196.62790424,353.35813486)(196.68790418,353.34313488)(196.71790574,353.32313724)
\curveto(196.78790408,353.27313495)(196.83290404,353.19813502)(196.85290574,353.09813724)
\curveto(196.88290399,353.00813521)(196.91790395,352.9231353)(196.95790574,352.84313724)
\curveto(197.05790381,352.6231356)(197.19290368,352.45313577)(197.36290574,352.33313724)
\curveto(197.48290339,352.24313598)(197.61790325,352.17313605)(197.76790574,352.12313724)
\curveto(197.91790295,352.07313615)(198.07790279,352.0231362)(198.24790574,351.97313724)
\lineto(198.56290574,351.92813724)
\lineto(198.65290574,351.92813724)
\curveto(198.72290215,351.90813631)(198.81290206,351.89813632)(198.92290574,351.89813724)
\curveto(199.04290183,351.89813632)(199.14290173,351.90813631)(199.22290574,351.92813724)
\curveto(199.29290158,351.92813629)(199.34790152,351.93313629)(199.38790574,351.94313724)
\curveto(199.44790142,351.95313627)(199.50790136,351.95813626)(199.56790574,351.95813724)
\curveto(199.62790124,351.96813625)(199.68290119,351.97813624)(199.73290574,351.98813724)
\curveto(200.02290085,352.06813615)(200.25290062,352.17313605)(200.42290574,352.30313724)
\curveto(200.59290028,352.43313579)(200.71290016,352.65313557)(200.78290574,352.96313724)
\curveto(200.80290007,353.01313521)(200.80790006,353.06813515)(200.79790574,353.12813724)
\curveto(200.78790008,353.18813503)(200.77790009,353.23313499)(200.76790574,353.26313724)
\curveto(200.71790015,353.45313477)(200.64790022,353.59313463)(200.55790574,353.68313724)
\curveto(200.4679004,353.78313444)(200.35290052,353.87313435)(200.21290574,353.95313724)
\curveto(200.12290075,354.01313421)(200.02290085,354.06313416)(199.91290574,354.10313724)
\lineto(199.58290574,354.22313724)
\curveto(199.55290132,354.23313399)(199.52290135,354.23813398)(199.49290574,354.23813724)
\curveto(199.4729014,354.23813398)(199.44790142,354.24813397)(199.41790574,354.26813724)
\curveto(199.07790179,354.37813384)(198.72290215,354.45813376)(198.35290574,354.50813724)
\curveto(197.99290288,354.56813365)(197.65290322,354.66313356)(197.33290574,354.79313724)
\curveto(197.23290364,354.83313339)(197.13790373,354.86813335)(197.04790574,354.89813724)
\curveto(196.95790391,354.92813329)(196.872904,354.96813325)(196.79290574,355.01813724)
\curveto(196.60290427,355.12813309)(196.42790444,355.25313297)(196.26790574,355.39313724)
\curveto(196.10790476,355.53313269)(195.98290489,355.70813251)(195.89290574,355.91813724)
\curveto(195.86290501,355.98813223)(195.83790503,356.05813216)(195.81790574,356.12813724)
\curveto(195.80790506,356.19813202)(195.79290508,356.27313195)(195.77290574,356.35313724)
\curveto(195.74290513,356.47313175)(195.73290514,356.60813161)(195.74290574,356.75813724)
\curveto(195.75290512,356.9181313)(195.7679051,357.05313117)(195.78790574,357.16313724)
\curveto(195.80790506,357.21313101)(195.81790505,357.25313097)(195.81790574,357.28313724)
\curveto(195.82790504,357.3231309)(195.84290503,357.36313086)(195.86290574,357.40313724)
\curveto(195.95290492,357.63313059)(196.0729048,357.83313039)(196.22290574,358.00313724)
\curveto(196.38290449,358.17313005)(196.56290431,358.3231299)(196.76290574,358.45313724)
\curveto(196.91290396,358.54312968)(197.07790379,358.61312961)(197.25790574,358.66313724)
\curveto(197.43790343,358.7231295)(197.62790324,358.77812944)(197.82790574,358.82813724)
\curveto(197.89790297,358.83812938)(197.96290291,358.84812937)(198.02290574,358.85813724)
\curveto(198.09290278,358.86812935)(198.1679027,358.87812934)(198.24790574,358.88813724)
\curveto(198.27790259,358.89812932)(198.31790255,358.89812932)(198.36790574,358.88813724)
\curveto(198.41790245,358.87812934)(198.45290242,358.88312934)(198.47290574,358.90313724)
}
}
{
\newrgbcolor{curcolor}{0.60000002 0.60000002 0.60000002}
\pscustom[linestyle=none,fillstyle=solid,fillcolor=curcolor]
{
\newpath
\moveto(120.84556474,364.20820438)
\lineto(135.84556474,364.20820438)
\lineto(135.84556474,349.20820438)
\lineto(120.84556474,349.20820438)
\closepath
}
}
{
\newrgbcolor{curcolor}{0 0 0}
\pscustom[linestyle=none,fillstyle=solid,fillcolor=curcolor]
{
\newpath
\moveto(140.83376512,338.64237064)
\lineto(145.73876512,338.64237064)
\lineto(147.02876512,338.64237064)
\curveto(147.13875724,338.64235994)(147.24875713,338.64235994)(147.35876512,338.64237064)
\curveto(147.46875691,338.65235993)(147.55875682,338.63235995)(147.62876512,338.58237064)
\curveto(147.65875672,338.56236002)(147.68375669,338.53736005)(147.70376512,338.50737064)
\curveto(147.72375665,338.47736011)(147.74375663,338.44736014)(147.76376512,338.41737064)
\curveto(147.78375659,338.34736024)(147.79375658,338.23236035)(147.79376512,338.07237064)
\curveto(147.79375658,337.92236066)(147.78375659,337.80736078)(147.76376512,337.72737064)
\curveto(147.72375665,337.587361)(147.63875674,337.50736108)(147.50876512,337.48737064)
\curveto(147.378757,337.47736111)(147.22375715,337.47236111)(147.04376512,337.47237064)
\lineto(145.54376512,337.47237064)
\lineto(143.02376512,337.47237064)
\lineto(142.45376512,337.47237064)
\curveto(142.24376213,337.4823611)(142.08876229,337.45736113)(141.98876512,337.39737064)
\curveto(141.88876249,337.33736125)(141.83376254,337.23236135)(141.82376512,337.08237064)
\lineto(141.82376512,336.61737064)
\lineto(141.82376512,335.08737064)
\curveto(141.82376255,334.97736361)(141.81876256,334.84736374)(141.80876512,334.69737064)
\curveto(141.80876257,334.54736404)(141.81876256,334.42736416)(141.83876512,334.33737064)
\curveto(141.86876251,334.21736437)(141.92876245,334.13736445)(142.01876512,334.09737064)
\curveto(142.05876232,334.07736451)(142.12876225,334.05736453)(142.22876512,334.03737064)
\lineto(142.37876512,334.03737064)
\curveto(142.41876196,334.02736456)(142.45876192,334.02236456)(142.49876512,334.02237064)
\curveto(142.54876183,334.03236455)(142.59876178,334.03736455)(142.64876512,334.03737064)
\lineto(143.15876512,334.03737064)
\lineto(146.09876512,334.03737064)
\lineto(146.39876512,334.03737064)
\curveto(146.50875787,334.04736454)(146.61875776,334.04736454)(146.72876512,334.03737064)
\curveto(146.84875753,334.03736455)(146.95375742,334.02736456)(147.04376512,334.00737064)
\curveto(147.14375723,333.99736459)(147.21875716,333.97736461)(147.26876512,333.94737064)
\curveto(147.29875708,333.92736466)(147.32375705,333.8823647)(147.34376512,333.81237064)
\curveto(147.36375701,333.74236484)(147.378757,333.66736492)(147.38876512,333.58737064)
\curveto(147.39875698,333.50736508)(147.39875698,333.42236516)(147.38876512,333.33237064)
\curveto(147.38875699,333.25236533)(147.378757,333.1823654)(147.35876512,333.12237064)
\curveto(147.33875704,333.03236555)(147.29375708,332.96736562)(147.22376512,332.92737064)
\curveto(147.20375717,332.90736568)(147.1737572,332.89236569)(147.13376512,332.88237064)
\curveto(147.10375727,332.8823657)(147.0737573,332.87736571)(147.04376512,332.86737064)
\lineto(146.95376512,332.86737064)
\curveto(146.90375747,332.85736573)(146.85375752,332.85236573)(146.80376512,332.85237064)
\curveto(146.75375762,332.86236572)(146.70375767,332.86736572)(146.65376512,332.86737064)
\lineto(146.09876512,332.86737064)
\lineto(142.93376512,332.86737064)
\lineto(142.57376512,332.86737064)
\curveto(142.46376191,332.87736571)(142.35876202,332.87236571)(142.25876512,332.85237064)
\curveto(142.15876222,332.84236574)(142.06876231,332.81736577)(141.98876512,332.77737064)
\curveto(141.91876246,332.73736585)(141.86876251,332.66736592)(141.83876512,332.56737064)
\curveto(141.81876256,332.50736608)(141.80876257,332.43736615)(141.80876512,332.35737064)
\curveto(141.81876256,332.27736631)(141.82376255,332.19736639)(141.82376512,332.11737064)
\lineto(141.82376512,331.27737064)
\lineto(141.82376512,329.85237064)
\curveto(141.82376255,329.71236887)(141.82876255,329.582369)(141.83876512,329.46237064)
\curveto(141.84876253,329.35236923)(141.88876249,329.27236931)(141.95876512,329.22237064)
\curveto(142.02876235,329.17236941)(142.10876227,329.14236944)(142.19876512,329.13237064)
\lineto(142.49876512,329.13237064)
\lineto(143.45876512,329.13237064)
\lineto(146.23376512,329.13237064)
\lineto(147.08876512,329.13237064)
\lineto(147.32876512,329.13237064)
\curveto(147.40875697,329.14236944)(147.4787569,329.13736945)(147.53876512,329.11737064)
\curveto(147.65875672,329.07736951)(147.73875664,329.02236956)(147.77876512,328.95237064)
\curveto(147.79875658,328.92236966)(147.81375656,328.87236971)(147.82376512,328.80237064)
\curveto(147.83375654,328.73236985)(147.83875654,328.65736993)(147.83876512,328.57737064)
\curveto(147.84875653,328.50737008)(147.84875653,328.43237015)(147.83876512,328.35237064)
\curveto(147.82875655,328.2823703)(147.81875656,328.22737036)(147.80876512,328.18737064)
\curveto(147.76875661,328.10737048)(147.72375665,328.05237053)(147.67376512,328.02237064)
\curveto(147.61375676,327.9823706)(147.53375684,327.96237062)(147.43376512,327.96237064)
\lineto(147.16376512,327.96237064)
\lineto(146.11376512,327.96237064)
\lineto(142.12376512,327.96237064)
\lineto(141.07376512,327.96237064)
\curveto(140.93376344,327.96237062)(140.81376356,327.96737062)(140.71376512,327.97737064)
\curveto(140.61376376,327.99737059)(140.53876384,328.04737054)(140.48876512,328.12737064)
\curveto(140.44876393,328.1873704)(140.42876395,328.26237032)(140.42876512,328.35237064)
\lineto(140.42876512,328.63737064)
\lineto(140.42876512,329.68737064)
\lineto(140.42876512,333.70737064)
\lineto(140.42876512,337.06737064)
\lineto(140.42876512,337.99737064)
\lineto(140.42876512,338.26737064)
\curveto(140.42876395,338.35736023)(140.44876393,338.42736016)(140.48876512,338.47737064)
\curveto(140.52876385,338.54736004)(140.60376377,338.59735999)(140.71376512,338.62737064)
\curveto(140.73376364,338.63735995)(140.75376362,338.63735995)(140.77376512,338.62737064)
\curveto(140.79376358,338.62735996)(140.81376356,338.63235995)(140.83376512,338.64237064)
}
}
{
\newrgbcolor{curcolor}{0 0 0}
\pscustom[linestyle=none,fillstyle=solid,fillcolor=curcolor]
{
\newpath
\moveto(151.77368699,335.86737064)
\curveto(152.49368293,335.87736271)(153.09868232,335.79236279)(153.58868699,335.61237064)
\curveto(154.07868134,335.44236314)(154.45868096,335.13736345)(154.72868699,334.69737064)
\curveto(154.79868062,334.587364)(154.85368057,334.47236411)(154.89368699,334.35237064)
\curveto(154.93368049,334.24236434)(154.97368045,334.11736447)(155.01368699,333.97737064)
\curveto(155.03368039,333.90736468)(155.03868038,333.83236475)(155.02868699,333.75237064)
\curveto(155.0186804,333.6823649)(155.00368042,333.62736496)(154.98368699,333.58737064)
\curveto(154.96368046,333.56736502)(154.93868048,333.54736504)(154.90868699,333.52737064)
\curveto(154.87868054,333.51736507)(154.85368057,333.50236508)(154.83368699,333.48237064)
\curveto(154.78368064,333.46236512)(154.73368069,333.45736513)(154.68368699,333.46737064)
\curveto(154.63368079,333.47736511)(154.58368084,333.47736511)(154.53368699,333.46737064)
\curveto(154.45368097,333.44736514)(154.34868107,333.44236514)(154.21868699,333.45237064)
\curveto(154.08868133,333.47236511)(153.99868142,333.49736509)(153.94868699,333.52737064)
\curveto(153.86868155,333.57736501)(153.81368161,333.64236494)(153.78368699,333.72237064)
\curveto(153.76368166,333.81236477)(153.72868169,333.89736469)(153.67868699,333.97737064)
\curveto(153.58868183,334.13736445)(153.46368196,334.2823643)(153.30368699,334.41237064)
\curveto(153.19368223,334.49236409)(153.07368235,334.55236403)(152.94368699,334.59237064)
\curveto(152.81368261,334.63236395)(152.67368275,334.67236391)(152.52368699,334.71237064)
\curveto(152.47368295,334.73236385)(152.423683,334.73736385)(152.37368699,334.72737064)
\curveto(152.3236831,334.72736386)(152.27368315,334.73236385)(152.22368699,334.74237064)
\curveto(152.16368326,334.76236382)(152.08868333,334.77236381)(151.99868699,334.77237064)
\curveto(151.90868351,334.77236381)(151.83368359,334.76236382)(151.77368699,334.74237064)
\lineto(151.68368699,334.74237064)
\lineto(151.53368699,334.71237064)
\curveto(151.48368394,334.71236387)(151.43368399,334.70736388)(151.38368699,334.69737064)
\curveto(151.1236843,334.63736395)(150.90868451,334.55236403)(150.73868699,334.44237064)
\curveto(150.56868485,334.33236425)(150.45368497,334.14736444)(150.39368699,333.88737064)
\curveto(150.37368505,333.81736477)(150.36868505,333.74736484)(150.37868699,333.67737064)
\curveto(150.39868502,333.60736498)(150.418685,333.54736504)(150.43868699,333.49737064)
\curveto(150.49868492,333.34736524)(150.56868485,333.23736535)(150.64868699,333.16737064)
\curveto(150.73868468,333.10736548)(150.84868457,333.03736555)(150.97868699,332.95737064)
\curveto(151.13868428,332.85736573)(151.3186841,332.7823658)(151.51868699,332.73237064)
\curveto(151.7186837,332.69236589)(151.9186835,332.64236594)(152.11868699,332.58237064)
\curveto(152.24868317,332.54236604)(152.37868304,332.51236607)(152.50868699,332.49237064)
\curveto(152.63868278,332.47236611)(152.76868265,332.44236614)(152.89868699,332.40237064)
\curveto(153.10868231,332.34236624)(153.31368211,332.2823663)(153.51368699,332.22237064)
\curveto(153.71368171,332.17236641)(153.91368151,332.10736648)(154.11368699,332.02737064)
\lineto(154.26368699,331.96737064)
\curveto(154.31368111,331.94736664)(154.36368106,331.92236666)(154.41368699,331.89237064)
\curveto(154.61368081,331.77236681)(154.78868063,331.63736695)(154.93868699,331.48737064)
\curveto(155.08868033,331.33736725)(155.21368021,331.14736744)(155.31368699,330.91737064)
\curveto(155.33368009,330.84736774)(155.35368007,330.75236783)(155.37368699,330.63237064)
\curveto(155.39368003,330.56236802)(155.40368002,330.4873681)(155.40368699,330.40737064)
\curveto(155.41368001,330.33736825)(155.41868,330.25736833)(155.41868699,330.16737064)
\lineto(155.41868699,330.01737064)
\curveto(155.39868002,329.94736864)(155.38868003,329.87736871)(155.38868699,329.80737064)
\curveto(155.38868003,329.73736885)(155.37868004,329.66736892)(155.35868699,329.59737064)
\curveto(155.32868009,329.4873691)(155.29368013,329.3823692)(155.25368699,329.28237064)
\curveto(155.21368021,329.1823694)(155.16868025,329.09236949)(155.11868699,329.01237064)
\curveto(154.95868046,328.75236983)(154.75368067,328.54237004)(154.50368699,328.38237064)
\curveto(154.25368117,328.23237035)(153.97368145,328.10237048)(153.66368699,327.99237064)
\curveto(153.57368185,327.96237062)(153.47868194,327.94237064)(153.37868699,327.93237064)
\curveto(153.28868213,327.91237067)(153.19868222,327.8873707)(153.10868699,327.85737064)
\curveto(153.00868241,327.83737075)(152.90868251,327.82737076)(152.80868699,327.82737064)
\curveto(152.70868271,327.82737076)(152.60868281,327.81737077)(152.50868699,327.79737064)
\lineto(152.35868699,327.79737064)
\curveto(152.30868311,327.7873708)(152.23868318,327.7823708)(152.14868699,327.78237064)
\curveto(152.05868336,327.7823708)(151.98868343,327.7873708)(151.93868699,327.79737064)
\lineto(151.77368699,327.79737064)
\curveto(151.71368371,327.81737077)(151.64868377,327.82737076)(151.57868699,327.82737064)
\curveto(151.50868391,327.81737077)(151.44868397,327.82237076)(151.39868699,327.84237064)
\curveto(151.34868407,327.85237073)(151.28368414,327.85737073)(151.20368699,327.85737064)
\lineto(150.96368699,327.91737064)
\curveto(150.89368453,327.92737066)(150.8186846,327.94737064)(150.73868699,327.97737064)
\curveto(150.42868499,328.07737051)(150.15868526,328.20237038)(149.92868699,328.35237064)
\curveto(149.69868572,328.50237008)(149.49868592,328.69736989)(149.32868699,328.93737064)
\curveto(149.23868618,329.06736952)(149.16368626,329.20236938)(149.10368699,329.34237064)
\curveto(149.04368638,329.4823691)(148.98868643,329.63736895)(148.93868699,329.80737064)
\curveto(148.9186865,329.86736872)(148.90868651,329.93736865)(148.90868699,330.01737064)
\curveto(148.9186865,330.10736848)(148.93368649,330.17736841)(148.95368699,330.22737064)
\curveto(148.98368644,330.26736832)(149.03368639,330.30736828)(149.10368699,330.34737064)
\curveto(149.15368627,330.36736822)(149.2236862,330.37736821)(149.31368699,330.37737064)
\curveto(149.40368602,330.3873682)(149.49368593,330.3873682)(149.58368699,330.37737064)
\curveto(149.67368575,330.36736822)(149.75868566,330.35236823)(149.83868699,330.33237064)
\curveto(149.92868549,330.32236826)(149.98868543,330.30736828)(150.01868699,330.28737064)
\curveto(150.08868533,330.23736835)(150.13368529,330.16236842)(150.15368699,330.06237064)
\curveto(150.18368524,329.97236861)(150.2186852,329.8873687)(150.25868699,329.80737064)
\curveto(150.35868506,329.587369)(150.49368493,329.41736917)(150.66368699,329.29737064)
\curveto(150.78368464,329.20736938)(150.9186845,329.13736945)(151.06868699,329.08737064)
\curveto(151.2186842,329.03736955)(151.37868404,328.9873696)(151.54868699,328.93737064)
\lineto(151.86368699,328.89237064)
\lineto(151.95368699,328.89237064)
\curveto(152.0236834,328.87236971)(152.11368331,328.86236972)(152.22368699,328.86237064)
\curveto(152.34368308,328.86236972)(152.44368298,328.87236971)(152.52368699,328.89237064)
\curveto(152.59368283,328.89236969)(152.64868277,328.89736969)(152.68868699,328.90737064)
\curveto(152.74868267,328.91736967)(152.80868261,328.92236966)(152.86868699,328.92237064)
\curveto(152.92868249,328.93236965)(152.98368244,328.94236964)(153.03368699,328.95237064)
\curveto(153.3236821,329.03236955)(153.55368187,329.13736945)(153.72368699,329.26737064)
\curveto(153.89368153,329.39736919)(154.01368141,329.61736897)(154.08368699,329.92737064)
\curveto(154.10368132,329.97736861)(154.10868131,330.03236855)(154.09868699,330.09237064)
\curveto(154.08868133,330.15236843)(154.07868134,330.19736839)(154.06868699,330.22737064)
\curveto(154.0186814,330.41736817)(153.94868147,330.55736803)(153.85868699,330.64737064)
\curveto(153.76868165,330.74736784)(153.65368177,330.83736775)(153.51368699,330.91737064)
\curveto(153.423682,330.97736761)(153.3236821,331.02736756)(153.21368699,331.06737064)
\lineto(152.88368699,331.18737064)
\curveto(152.85368257,331.19736739)(152.8236826,331.20236738)(152.79368699,331.20237064)
\curveto(152.77368265,331.20236738)(152.74868267,331.21236737)(152.71868699,331.23237064)
\curveto(152.37868304,331.34236724)(152.0236834,331.42236716)(151.65368699,331.47237064)
\curveto(151.29368413,331.53236705)(150.95368447,331.62736696)(150.63368699,331.75737064)
\curveto(150.53368489,331.79736679)(150.43868498,331.83236675)(150.34868699,331.86237064)
\curveto(150.25868516,331.89236669)(150.17368525,331.93236665)(150.09368699,331.98237064)
\curveto(149.90368552,332.09236649)(149.72868569,332.21736637)(149.56868699,332.35737064)
\curveto(149.40868601,332.49736609)(149.28368614,332.67236591)(149.19368699,332.88237064)
\curveto(149.16368626,332.95236563)(149.13868628,333.02236556)(149.11868699,333.09237064)
\curveto(149.10868631,333.16236542)(149.09368633,333.23736535)(149.07368699,333.31737064)
\curveto(149.04368638,333.43736515)(149.03368639,333.57236501)(149.04368699,333.72237064)
\curveto(149.05368637,333.8823647)(149.06868635,334.01736457)(149.08868699,334.12737064)
\curveto(149.10868631,334.17736441)(149.1186863,334.21736437)(149.11868699,334.24737064)
\curveto(149.12868629,334.2873643)(149.14368628,334.32736426)(149.16368699,334.36737064)
\curveto(149.25368617,334.59736399)(149.37368605,334.79736379)(149.52368699,334.96737064)
\curveto(149.68368574,335.13736345)(149.86368556,335.2873633)(150.06368699,335.41737064)
\curveto(150.21368521,335.50736308)(150.37868504,335.57736301)(150.55868699,335.62737064)
\curveto(150.73868468,335.6873629)(150.92868449,335.74236284)(151.12868699,335.79237064)
\curveto(151.19868422,335.80236278)(151.26368416,335.81236277)(151.32368699,335.82237064)
\curveto(151.39368403,335.83236275)(151.46868395,335.84236274)(151.54868699,335.85237064)
\curveto(151.57868384,335.86236272)(151.6186838,335.86236272)(151.66868699,335.85237064)
\curveto(151.7186837,335.84236274)(151.75368367,335.84736274)(151.77368699,335.86737064)
}
}
{
\newrgbcolor{curcolor}{0 0 0}
\pscustom[linestyle=none,fillstyle=solid,fillcolor=curcolor]
{
\newpath
\moveto(164.26868699,332.02737064)
\curveto(164.27867864,331.97736661)(164.28367864,331.91236667)(164.28368699,331.83237064)
\curveto(164.28367864,331.75236683)(164.27867864,331.6873669)(164.26868699,331.63737064)
\curveto(164.24867867,331.587367)(164.24367868,331.53736705)(164.25368699,331.48737064)
\curveto(164.26367866,331.44736714)(164.26367866,331.40736718)(164.25368699,331.36737064)
\curveto(164.25367867,331.29736729)(164.24867867,331.24236734)(164.23868699,331.20237064)
\curveto(164.2186787,331.11236747)(164.20367872,331.02236756)(164.19368699,330.93237064)
\curveto(164.19367873,330.84236774)(164.18367874,330.75236783)(164.16368699,330.66237064)
\lineto(164.10368699,330.42237064)
\curveto(164.08367884,330.35236823)(164.05867886,330.27736831)(164.02868699,330.19737064)
\curveto(163.90867901,329.82736876)(163.74367918,329.49236909)(163.53368699,329.19237064)
\curveto(163.47367945,329.10236948)(163.40867951,329.01236957)(163.33868699,328.92237064)
\curveto(163.26867965,328.84236974)(163.19367973,328.76736982)(163.11368699,328.69737064)
\lineto(163.03868699,328.62237064)
\curveto(162.96867995,328.57237001)(162.90368002,328.52237006)(162.84368699,328.47237064)
\curveto(162.78368014,328.42237016)(162.71368021,328.37237021)(162.63368699,328.32237064)
\curveto(162.5236804,328.24237034)(162.39868052,328.17237041)(162.25868699,328.11237064)
\curveto(162.12868079,328.06237052)(161.99368093,328.01237057)(161.85368699,327.96237064)
\curveto(161.77368115,327.94237064)(161.69368123,327.92737066)(161.61368699,327.91737064)
\curveto(161.54368138,327.90737068)(161.46868145,327.89237069)(161.38868699,327.87237064)
\lineto(161.32868699,327.87237064)
\curveto(161.3186816,327.86237072)(161.30368162,327.85737073)(161.28368699,327.85737064)
\curveto(161.19368173,327.83737075)(161.05868186,327.82737076)(160.87868699,327.82737064)
\curveto(160.70868221,327.81737077)(160.57368235,327.82237076)(160.47368699,327.84237064)
\lineto(160.39868699,327.84237064)
\curveto(160.32868259,327.85237073)(160.26368266,327.86237072)(160.20368699,327.87237064)
\curveto(160.14368278,327.87237071)(160.08368284,327.8823707)(160.02368699,327.90237064)
\curveto(159.85368307,327.95237063)(159.69368323,327.99737059)(159.54368699,328.03737064)
\curveto(159.39368353,328.07737051)(159.25368367,328.13737045)(159.12368699,328.21737064)
\curveto(158.96368396,328.30737028)(158.8236841,328.40237018)(158.70368699,328.50237064)
\curveto(158.66368426,328.53237005)(158.60368432,328.57237001)(158.52368699,328.62237064)
\curveto(158.44368448,328.6823699)(158.36868455,328.6873699)(158.29868699,328.63737064)
\curveto(158.25868466,328.60736998)(158.23868468,328.56737002)(158.23868699,328.51737064)
\curveto(158.23868468,328.46737012)(158.22868469,328.41237017)(158.20868699,328.35237064)
\curveto(158.19868472,328.32237026)(158.19868472,328.2873703)(158.20868699,328.24737064)
\curveto(158.2186847,328.21737037)(158.2186847,328.1823704)(158.20868699,328.14237064)
\curveto(158.18868473,328.0823705)(158.17868474,328.01737057)(158.17868699,327.94737064)
\curveto(158.18868473,327.86737072)(158.19368473,327.79737079)(158.19368699,327.73737064)
\lineto(158.19368699,325.93737064)
\lineto(158.19368699,325.50237064)
\curveto(158.19368473,325.35237323)(158.16368476,325.23737335)(158.10368699,325.15737064)
\curveto(158.05368487,325.0873735)(157.97368495,325.05237353)(157.86368699,325.05237064)
\curveto(157.75368517,325.04237354)(157.64368528,325.03737355)(157.53368699,325.03737064)
\lineto(157.29368699,325.03737064)
\curveto(157.2236857,325.05737353)(157.16368576,325.07737351)(157.11368699,325.09737064)
\curveto(157.07368585,325.11737347)(157.03868588,325.15237343)(157.00868699,325.20237064)
\curveto(156.95868596,325.27237331)(156.93368599,325.3823732)(156.93368699,325.53237064)
\curveto(156.94368598,325.6823729)(156.94868597,325.81237277)(156.94868699,325.92237064)
\lineto(156.94868699,334.92237064)
\lineto(156.94868699,335.28237064)
\curveto(156.95868596,335.41236317)(156.98868593,335.51736307)(157.03868699,335.59737064)
\curveto(157.06868585,335.63736295)(157.13368579,335.66736292)(157.23368699,335.68737064)
\curveto(157.34368558,335.71736287)(157.45868546,335.72736286)(157.57868699,335.71737064)
\curveto(157.69868522,335.71736287)(157.80868511,335.70236288)(157.90868699,335.67237064)
\curveto(158.0186849,335.65236293)(158.08868483,335.62236296)(158.11868699,335.58237064)
\curveto(158.15868476,335.53236305)(158.17868474,335.47236311)(158.17868699,335.40237064)
\curveto(158.18868473,335.33236325)(158.20868471,335.26236332)(158.23868699,335.19237064)
\curveto(158.25868466,335.16236342)(158.27368465,335.13736345)(158.28368699,335.11737064)
\curveto(158.30368462,335.10736348)(158.3236846,335.09236349)(158.34368699,335.07237064)
\curveto(158.45368447,335.06236352)(158.54368438,335.09736349)(158.61368699,335.17737064)
\curveto(158.69368423,335.25736333)(158.76868415,335.32236326)(158.83868699,335.37237064)
\curveto(159.09868382,335.55236303)(159.40868351,335.69236289)(159.76868699,335.79237064)
\curveto(159.85868306,335.81236277)(159.94868297,335.82736276)(160.03868699,335.83737064)
\curveto(160.13868278,335.84736274)(160.23868268,335.86236272)(160.33868699,335.88237064)
\curveto(160.37868254,335.89236269)(160.42868249,335.89236269)(160.48868699,335.88237064)
\curveto(160.54868237,335.87236271)(160.58868233,335.87736271)(160.60868699,335.89737064)
\curveto(161.03868188,335.90736268)(161.4186815,335.86236272)(161.74868699,335.76237064)
\curveto(162.07868084,335.67236291)(162.37368055,335.54236304)(162.63368699,335.37237064)
\lineto(162.78368699,335.25237064)
\curveto(162.83368009,335.22236336)(162.88368004,335.1873634)(162.93368699,335.14737064)
\curveto(162.95367997,335.12736346)(162.96867995,335.10736348)(162.97868699,335.08737064)
\curveto(162.99867992,335.07736351)(163.0186799,335.06236352)(163.03868699,335.04237064)
\curveto(163.08867983,334.99236359)(163.14367978,334.93736365)(163.20368699,334.87737064)
\curveto(163.26367966,334.81736377)(163.3186796,334.75736383)(163.36868699,334.69737064)
\curveto(163.48867943,334.52736406)(163.61367931,334.34236424)(163.74368699,334.14237064)
\curveto(163.8236791,334.01236457)(163.88867903,333.86736472)(163.93868699,333.70737064)
\curveto(163.99867892,333.54736504)(164.05367887,333.3873652)(164.10368699,333.22737064)
\curveto(164.1236788,333.14736544)(164.13867878,333.06236552)(164.14868699,332.97237064)
\curveto(164.16867875,332.8823657)(164.18867873,332.79736579)(164.20868699,332.71737064)
\lineto(164.20868699,332.59737064)
\curveto(164.2186787,332.56736602)(164.2236787,332.53736605)(164.22368699,332.50737064)
\curveto(164.24367868,332.45736613)(164.24867867,332.40236618)(164.23868699,332.34237064)
\curveto(164.23867868,332.2823663)(164.24867867,332.22736636)(164.26868699,332.17737064)
\lineto(164.26868699,332.02737064)
\moveto(162.93368699,331.62237064)
\curveto(162.95367997,331.67236691)(162.95867996,331.73236685)(162.94868699,331.80237064)
\curveto(162.93867998,331.8823667)(162.93367999,331.95236663)(162.93368699,332.01237064)
\curveto(162.93367999,332.1823664)(162.92368,332.34236624)(162.90368699,332.49237064)
\curveto(162.89368003,332.64236594)(162.86368006,332.7873658)(162.81368699,332.92737064)
\lineto(162.75368699,333.10737064)
\curveto(162.74368018,333.17736541)(162.7236802,333.24236534)(162.69368699,333.30237064)
\curveto(162.58368034,333.57236501)(162.40868051,333.83236475)(162.16868699,334.08237064)
\curveto(161.93868098,334.33236425)(161.7186812,334.50236408)(161.50868699,334.59237064)
\curveto(161.42868149,334.63236395)(161.34368158,334.66236392)(161.25368699,334.68237064)
\curveto(161.17368175,334.70236388)(161.08868183,334.72736386)(160.99868699,334.75737064)
\curveto(160.90868201,334.77736381)(160.80368212,334.7873638)(160.68368699,334.78737064)
\lineto(160.35368699,334.78737064)
\curveto(160.33368259,334.76736382)(160.29368263,334.75736383)(160.23368699,334.75737064)
\curveto(160.18368274,334.76736382)(160.13868278,334.76736382)(160.09868699,334.75737064)
\lineto(159.82868699,334.69737064)
\curveto(159.74868317,334.67736391)(159.66868325,334.64736394)(159.58868699,334.60737064)
\curveto(159.26868365,334.46736412)(159.00368392,334.26236432)(158.79368699,333.99237064)
\curveto(158.59368433,333.73236485)(158.43868448,333.42736516)(158.32868699,333.07737064)
\curveto(158.28868463,332.96736562)(158.25868466,332.85736573)(158.23868699,332.74737064)
\curveto(158.22868469,332.63736595)(158.21368471,332.52736606)(158.19368699,332.41737064)
\curveto(158.18368474,332.37736621)(158.17868474,332.33736625)(158.17868699,332.29737064)
\curveto(158.17868474,332.26736632)(158.17368475,332.23236635)(158.16368699,332.19237064)
\lineto(158.16368699,332.07237064)
\curveto(158.15368477,332.02236656)(158.14868477,331.94736664)(158.14868699,331.84737064)
\curveto(158.14868477,331.75736683)(158.15368477,331.6873669)(158.16368699,331.63737064)
\lineto(158.16368699,331.51737064)
\curveto(158.17368475,331.47736711)(158.17868474,331.43736715)(158.17868699,331.39737064)
\curveto(158.17868474,331.35736723)(158.18368474,331.32236726)(158.19368699,331.29237064)
\curveto(158.20368472,331.26236732)(158.20868471,331.23236735)(158.20868699,331.20237064)
\curveto(158.20868471,331.17236741)(158.21368471,331.13736745)(158.22368699,331.09737064)
\curveto(158.24368468,331.01736757)(158.25868466,330.93736765)(158.26868699,330.85737064)
\lineto(158.32868699,330.61737064)
\curveto(158.43868448,330.27736831)(158.58868433,329.97736861)(158.77868699,329.71737064)
\curveto(158.97868394,329.46736912)(159.23868368,329.27236931)(159.55868699,329.13237064)
\curveto(159.74868317,329.05236953)(159.94368298,328.99236959)(160.14368699,328.95237064)
\curveto(160.18368274,328.93236965)(160.2236827,328.92236966)(160.26368699,328.92237064)
\curveto(160.30368262,328.93236965)(160.34368258,328.93236965)(160.38368699,328.92237064)
\lineto(160.50368699,328.92237064)
\curveto(160.57368235,328.90236968)(160.64368228,328.90236968)(160.71368699,328.92237064)
\lineto(160.83368699,328.92237064)
\curveto(160.94368198,328.94236964)(161.04868187,328.95736963)(161.14868699,328.96737064)
\curveto(161.24868167,328.97736961)(161.34868157,329.00236958)(161.44868699,329.04237064)
\curveto(161.75868116,329.17236941)(162.00868091,329.34236924)(162.19868699,329.55237064)
\curveto(162.39868052,329.77236881)(162.56368036,330.03736855)(162.69368699,330.34737064)
\curveto(162.74368018,330.4873681)(162.77868014,330.62736796)(162.79868699,330.76737064)
\curveto(162.82868009,330.91736767)(162.86368006,331.07236751)(162.90368699,331.23237064)
\curveto(162.91368001,331.2823673)(162.91868,331.32736726)(162.91868699,331.36737064)
\curveto(162.91868,331.40736718)(162.92368,331.45236713)(162.93368699,331.50237064)
\lineto(162.93368699,331.62237064)
}
}
{
\newrgbcolor{curcolor}{0 0 0}
\pscustom[linestyle=none,fillstyle=solid,fillcolor=curcolor]
{
\newpath
\moveto(172.63493699,328.51737064)
\curveto(172.66492916,328.35737023)(172.64992918,328.22237036)(172.58993699,328.11237064)
\curveto(172.5299293,328.01237057)(172.44992938,327.93737065)(172.34993699,327.88737064)
\curveto(172.29992953,327.86737072)(172.24492958,327.85737073)(172.18493699,327.85737064)
\curveto(172.13492969,327.85737073)(172.07992975,327.84737074)(172.01993699,327.82737064)
\curveto(171.79993003,327.77737081)(171.57993025,327.79237079)(171.35993699,327.87237064)
\curveto(171.14993068,327.94237064)(171.00493082,328.03237055)(170.92493699,328.14237064)
\curveto(170.87493095,328.21237037)(170.829931,328.29237029)(170.78993699,328.38237064)
\curveto(170.74993108,328.4823701)(170.69993113,328.56237002)(170.63993699,328.62237064)
\curveto(170.61993121,328.64236994)(170.59493123,328.66236992)(170.56493699,328.68237064)
\curveto(170.54493128,328.70236988)(170.51493131,328.70736988)(170.47493699,328.69737064)
\curveto(170.36493146,328.66736992)(170.25993157,328.61236997)(170.15993699,328.53237064)
\curveto(170.06993176,328.45237013)(169.97993185,328.3823702)(169.88993699,328.32237064)
\curveto(169.75993207,328.24237034)(169.61993221,328.16737042)(169.46993699,328.09737064)
\curveto(169.31993251,328.03737055)(169.15993267,327.9823706)(168.98993699,327.93237064)
\curveto(168.88993294,327.90237068)(168.77993305,327.8823707)(168.65993699,327.87237064)
\curveto(168.54993328,327.86237072)(168.43993339,327.84737074)(168.32993699,327.82737064)
\curveto(168.27993355,327.81737077)(168.23493359,327.81237077)(168.19493699,327.81237064)
\lineto(168.08993699,327.81237064)
\curveto(167.97993385,327.79237079)(167.87493395,327.79237079)(167.77493699,327.81237064)
\lineto(167.63993699,327.81237064)
\curveto(167.58993424,327.82237076)(167.53993429,327.82737076)(167.48993699,327.82737064)
\curveto(167.43993439,327.82737076)(167.39493443,327.83737075)(167.35493699,327.85737064)
\curveto(167.31493451,327.86737072)(167.27993455,327.87237071)(167.24993699,327.87237064)
\curveto(167.2299346,327.86237072)(167.20493462,327.86237072)(167.17493699,327.87237064)
\lineto(166.93493699,327.93237064)
\curveto(166.85493497,327.94237064)(166.77993505,327.96237062)(166.70993699,327.99237064)
\curveto(166.40993542,328.12237046)(166.16493566,328.26737032)(165.97493699,328.42737064)
\curveto(165.79493603,328.59736999)(165.64493618,328.83236975)(165.52493699,329.13237064)
\curveto(165.43493639,329.35236923)(165.38993644,329.61736897)(165.38993699,329.92737064)
\lineto(165.38993699,330.24237064)
\curveto(165.39993643,330.29236829)(165.40493642,330.34236824)(165.40493699,330.39237064)
\lineto(165.43493699,330.57237064)
\lineto(165.55493699,330.90237064)
\curveto(165.59493623,331.01236757)(165.64493618,331.11236747)(165.70493699,331.20237064)
\curveto(165.88493594,331.49236709)(166.1299357,331.70736688)(166.43993699,331.84737064)
\curveto(166.74993508,331.9873666)(167.08993474,332.11236647)(167.45993699,332.22237064)
\curveto(167.59993423,332.26236632)(167.74493408,332.29236629)(167.89493699,332.31237064)
\curveto(168.04493378,332.33236625)(168.19493363,332.35736623)(168.34493699,332.38737064)
\curveto(168.41493341,332.40736618)(168.47993335,332.41736617)(168.53993699,332.41737064)
\curveto(168.60993322,332.41736617)(168.68493314,332.42736616)(168.76493699,332.44737064)
\curveto(168.83493299,332.46736612)(168.90493292,332.47736611)(168.97493699,332.47737064)
\curveto(169.04493278,332.4873661)(169.11993271,332.50236608)(169.19993699,332.52237064)
\curveto(169.44993238,332.582366)(169.68493214,332.63236595)(169.90493699,332.67237064)
\curveto(170.1249317,332.72236586)(170.29993153,332.83736575)(170.42993699,333.01737064)
\curveto(170.48993134,333.09736549)(170.53993129,333.19736539)(170.57993699,333.31737064)
\curveto(170.61993121,333.44736514)(170.61993121,333.587365)(170.57993699,333.73737064)
\curveto(170.51993131,333.97736461)(170.4299314,334.16736442)(170.30993699,334.30737064)
\curveto(170.19993163,334.44736414)(170.03993179,334.55736403)(169.82993699,334.63737064)
\curveto(169.70993212,334.6873639)(169.56493226,334.72236386)(169.39493699,334.74237064)
\curveto(169.23493259,334.76236382)(169.06493276,334.77236381)(168.88493699,334.77237064)
\curveto(168.70493312,334.77236381)(168.5299333,334.76236382)(168.35993699,334.74237064)
\curveto(168.18993364,334.72236386)(168.04493378,334.69236389)(167.92493699,334.65237064)
\curveto(167.75493407,334.59236399)(167.58993424,334.50736408)(167.42993699,334.39737064)
\curveto(167.34993448,334.33736425)(167.27493455,334.25736433)(167.20493699,334.15737064)
\curveto(167.14493468,334.06736452)(167.08993474,333.96736462)(167.03993699,333.85737064)
\curveto(167.00993482,333.77736481)(166.97993485,333.69236489)(166.94993699,333.60237064)
\curveto(166.9299349,333.51236507)(166.88493494,333.44236514)(166.81493699,333.39237064)
\curveto(166.77493505,333.36236522)(166.70493512,333.33736525)(166.60493699,333.31737064)
\curveto(166.51493531,333.30736528)(166.41993541,333.30236528)(166.31993699,333.30237064)
\curveto(166.21993561,333.30236528)(166.11993571,333.30736528)(166.01993699,333.31737064)
\curveto(165.9299359,333.33736525)(165.86493596,333.36236522)(165.82493699,333.39237064)
\curveto(165.78493604,333.42236516)(165.75493607,333.47236511)(165.73493699,333.54237064)
\curveto(165.71493611,333.61236497)(165.71493611,333.6873649)(165.73493699,333.76737064)
\curveto(165.76493606,333.89736469)(165.79493603,334.01736457)(165.82493699,334.12737064)
\curveto(165.86493596,334.24736434)(165.90993592,334.36236422)(165.95993699,334.47237064)
\curveto(166.14993568,334.82236376)(166.38993544,335.09236349)(166.67993699,335.28237064)
\curveto(166.96993486,335.4823631)(167.3299345,335.64236294)(167.75993699,335.76237064)
\curveto(167.85993397,335.7823628)(167.95993387,335.79736279)(168.05993699,335.80737064)
\curveto(168.16993366,335.81736277)(168.27993355,335.83236275)(168.38993699,335.85237064)
\curveto(168.4299334,335.86236272)(168.49493333,335.86236272)(168.58493699,335.85237064)
\curveto(168.67493315,335.85236273)(168.7299331,335.86236272)(168.74993699,335.88237064)
\curveto(169.44993238,335.89236269)(170.05993177,335.81236277)(170.57993699,335.64237064)
\curveto(171.09993073,335.47236311)(171.46493036,335.14736344)(171.67493699,334.66737064)
\curveto(171.76493006,334.46736412)(171.81493001,334.23236435)(171.82493699,333.96237064)
\curveto(171.84492998,333.70236488)(171.85492997,333.42736516)(171.85493699,333.13737064)
\lineto(171.85493699,329.82237064)
\curveto(171.85492997,329.6823689)(171.85992997,329.54736904)(171.86993699,329.41737064)
\curveto(171.87992995,329.2873693)(171.90992992,329.1823694)(171.95993699,329.10237064)
\curveto(172.00992982,329.03236955)(172.07492975,328.9823696)(172.15493699,328.95237064)
\curveto(172.24492958,328.91236967)(172.3299295,328.8823697)(172.40993699,328.86237064)
\curveto(172.48992934,328.85236973)(172.54992928,328.80736978)(172.58993699,328.72737064)
\curveto(172.60992922,328.69736989)(172.61992921,328.66736992)(172.61993699,328.63737064)
\curveto(172.61992921,328.60736998)(172.6249292,328.56737002)(172.63493699,328.51737064)
\moveto(170.48993699,330.18237064)
\curveto(170.54993128,330.32236826)(170.57993125,330.4823681)(170.57993699,330.66237064)
\curveto(170.58993124,330.85236773)(170.59493123,331.04736754)(170.59493699,331.24737064)
\curveto(170.59493123,331.35736723)(170.58993124,331.45736713)(170.57993699,331.54737064)
\curveto(170.56993126,331.63736695)(170.5299313,331.70736688)(170.45993699,331.75737064)
\curveto(170.4299314,331.77736681)(170.35993147,331.7873668)(170.24993699,331.78737064)
\curveto(170.2299316,331.76736682)(170.19493163,331.75736683)(170.14493699,331.75737064)
\curveto(170.09493173,331.75736683)(170.04993178,331.74736684)(170.00993699,331.72737064)
\curveto(169.9299319,331.70736688)(169.83993199,331.6873669)(169.73993699,331.66737064)
\lineto(169.43993699,331.60737064)
\curveto(169.40993242,331.60736698)(169.37493245,331.60236698)(169.33493699,331.59237064)
\lineto(169.22993699,331.59237064)
\curveto(169.07993275,331.55236703)(168.91493291,331.52736706)(168.73493699,331.51737064)
\curveto(168.56493326,331.51736707)(168.40493342,331.49736709)(168.25493699,331.45737064)
\curveto(168.17493365,331.43736715)(168.09993373,331.41736717)(168.02993699,331.39737064)
\curveto(167.96993386,331.3873672)(167.89993393,331.37236721)(167.81993699,331.35237064)
\curveto(167.65993417,331.30236728)(167.50993432,331.23736735)(167.36993699,331.15737064)
\curveto(167.2299346,331.0873675)(167.10993472,330.99736759)(167.00993699,330.88737064)
\curveto(166.90993492,330.77736781)(166.83493499,330.64236794)(166.78493699,330.48237064)
\curveto(166.73493509,330.33236825)(166.71493511,330.14736844)(166.72493699,329.92737064)
\curveto(166.7249351,329.82736876)(166.73993509,329.73236885)(166.76993699,329.64237064)
\curveto(166.80993502,329.56236902)(166.85493497,329.4873691)(166.90493699,329.41737064)
\curveto(166.98493484,329.30736928)(167.08993474,329.21236937)(167.21993699,329.13237064)
\curveto(167.34993448,329.06236952)(167.48993434,329.00236958)(167.63993699,328.95237064)
\curveto(167.68993414,328.94236964)(167.73993409,328.93736965)(167.78993699,328.93737064)
\curveto(167.83993399,328.93736965)(167.88993394,328.93236965)(167.93993699,328.92237064)
\curveto(168.00993382,328.90236968)(168.09493373,328.8873697)(168.19493699,328.87737064)
\curveto(168.30493352,328.87736971)(168.39493343,328.8873697)(168.46493699,328.90737064)
\curveto(168.5249333,328.92736966)(168.58493324,328.93236965)(168.64493699,328.92237064)
\curveto(168.70493312,328.92236966)(168.76493306,328.93236965)(168.82493699,328.95237064)
\curveto(168.90493292,328.97236961)(168.97993285,328.9873696)(169.04993699,328.99737064)
\curveto(169.1299327,329.00736958)(169.20493262,329.02736956)(169.27493699,329.05737064)
\curveto(169.56493226,329.17736941)(169.80993202,329.32236926)(170.00993699,329.49237064)
\curveto(170.21993161,329.66236892)(170.37993145,329.89236869)(170.48993699,330.18237064)
}
}
{
\newrgbcolor{curcolor}{0 0 0}
\pscustom[linestyle=none,fillstyle=solid,fillcolor=curcolor]
{
\newpath
\moveto(176.94157762,335.86737064)
\curveto(177.68157283,335.87736271)(178.29657221,335.76736282)(178.78657762,335.53737064)
\curveto(179.28657122,335.31736327)(179.68157083,334.9823636)(179.97157762,334.53237064)
\curveto(180.10157041,334.33236425)(180.2115703,334.0873645)(180.30157762,333.79737064)
\curveto(180.32157019,333.74736484)(180.33657017,333.6823649)(180.34657762,333.60237064)
\curveto(180.35657015,333.52236506)(180.35157016,333.45236513)(180.33157762,333.39237064)
\curveto(180.30157021,333.34236524)(180.25157026,333.29736529)(180.18157762,333.25737064)
\curveto(180.15157036,333.23736535)(180.12157039,333.22736536)(180.09157762,333.22737064)
\curveto(180.06157045,333.23736535)(180.02657048,333.23736535)(179.98657762,333.22737064)
\curveto(179.94657056,333.21736537)(179.9065706,333.21236537)(179.86657762,333.21237064)
\curveto(179.82657068,333.22236536)(179.78657072,333.22736536)(179.74657762,333.22737064)
\lineto(179.43157762,333.22737064)
\curveto(179.33157118,333.23736535)(179.24657126,333.26736532)(179.17657762,333.31737064)
\curveto(179.09657141,333.37736521)(179.04157147,333.46236512)(179.01157762,333.57237064)
\curveto(178.98157153,333.6823649)(178.94157157,333.77736481)(178.89157762,333.85737064)
\curveto(178.74157177,334.11736447)(178.54657196,334.32236426)(178.30657762,334.47237064)
\curveto(178.22657228,334.52236406)(178.14157237,334.56236402)(178.05157762,334.59237064)
\curveto(177.96157255,334.63236395)(177.86657264,334.66736392)(177.76657762,334.69737064)
\curveto(177.62657288,334.73736385)(177.44157307,334.75736383)(177.21157762,334.75737064)
\curveto(176.98157353,334.76736382)(176.79157372,334.74736384)(176.64157762,334.69737064)
\curveto(176.57157394,334.67736391)(176.506574,334.66236392)(176.44657762,334.65237064)
\curveto(176.38657412,334.64236394)(176.32157419,334.62736396)(176.25157762,334.60737064)
\curveto(175.99157452,334.49736409)(175.76157475,334.34736424)(175.56157762,334.15737064)
\curveto(175.36157515,333.96736462)(175.2065753,333.74236484)(175.09657762,333.48237064)
\curveto(175.05657545,333.39236519)(175.02157549,333.29736529)(174.99157762,333.19737064)
\curveto(174.96157555,333.10736548)(174.93157558,333.00736558)(174.90157762,332.89737064)
\lineto(174.81157762,332.49237064)
\curveto(174.80157571,332.44236614)(174.79657571,332.3873662)(174.79657762,332.32737064)
\curveto(174.8065757,332.26736632)(174.80157571,332.21236637)(174.78157762,332.16237064)
\lineto(174.78157762,332.04237064)
\curveto(174.77157574,332.00236658)(174.76157575,331.93736665)(174.75157762,331.84737064)
\curveto(174.75157576,331.75736683)(174.76157575,331.69236689)(174.78157762,331.65237064)
\curveto(174.79157572,331.60236698)(174.79157572,331.55236703)(174.78157762,331.50237064)
\curveto(174.77157574,331.45236713)(174.77157574,331.40236718)(174.78157762,331.35237064)
\curveto(174.79157572,331.31236727)(174.79657571,331.24236734)(174.79657762,331.14237064)
\curveto(174.81657569,331.06236752)(174.83157568,330.97736761)(174.84157762,330.88737064)
\curveto(174.86157565,330.79736779)(174.88157563,330.71236787)(174.90157762,330.63237064)
\curveto(175.0115755,330.31236827)(175.13657537,330.03236855)(175.27657762,329.79237064)
\curveto(175.42657508,329.56236902)(175.63157488,329.36236922)(175.89157762,329.19237064)
\curveto(175.98157453,329.14236944)(176.07157444,329.09736949)(176.16157762,329.05737064)
\curveto(176.26157425,329.01736957)(176.36657414,328.97736961)(176.47657762,328.93737064)
\curveto(176.52657398,328.92736966)(176.56657394,328.92236966)(176.59657762,328.92237064)
\curveto(176.62657388,328.92236966)(176.66657384,328.91736967)(176.71657762,328.90737064)
\curveto(176.74657376,328.89736969)(176.79657371,328.89236969)(176.86657762,328.89237064)
\lineto(177.03157762,328.89237064)
\curveto(177.03157348,328.8823697)(177.05157346,328.87736971)(177.09157762,328.87737064)
\curveto(177.1115734,328.8873697)(177.13657337,328.8873697)(177.16657762,328.87737064)
\curveto(177.19657331,328.87736971)(177.22657328,328.8823697)(177.25657762,328.89237064)
\curveto(177.32657318,328.91236967)(177.39157312,328.91736967)(177.45157762,328.90737064)
\curveto(177.52157299,328.90736968)(177.59157292,328.91736967)(177.66157762,328.93737064)
\curveto(177.92157259,329.01736957)(178.14657236,329.11736947)(178.33657762,329.23737064)
\curveto(178.52657198,329.36736922)(178.68657182,329.53236905)(178.81657762,329.73237064)
\curveto(178.86657164,329.81236877)(178.9115716,329.89736869)(178.95157762,329.98737064)
\lineto(179.07157762,330.25737064)
\curveto(179.09157142,330.33736825)(179.1115714,330.41236817)(179.13157762,330.48237064)
\curveto(179.16157135,330.56236802)(179.2115713,330.62736796)(179.28157762,330.67737064)
\curveto(179.3115712,330.70736788)(179.37157114,330.72736786)(179.46157762,330.73737064)
\curveto(179.55157096,330.75736783)(179.64657086,330.76736782)(179.74657762,330.76737064)
\curveto(179.85657065,330.77736781)(179.95657055,330.77736781)(180.04657762,330.76737064)
\curveto(180.14657036,330.75736783)(180.21657029,330.73736785)(180.25657762,330.70737064)
\curveto(180.31657019,330.66736792)(180.35157016,330.60736798)(180.36157762,330.52737064)
\curveto(180.38157013,330.44736814)(180.38157013,330.36236822)(180.36157762,330.27237064)
\curveto(180.3115702,330.12236846)(180.26157025,329.97736861)(180.21157762,329.83737064)
\curveto(180.17157034,329.70736888)(180.11657039,329.57736901)(180.04657762,329.44737064)
\curveto(179.89657061,329.14736944)(179.7065708,328.8823697)(179.47657762,328.65237064)
\curveto(179.25657125,328.42237016)(178.98657152,328.23737035)(178.66657762,328.09737064)
\curveto(178.58657192,328.05737053)(178.50157201,328.02237056)(178.41157762,327.99237064)
\curveto(178.32157219,327.97237061)(178.22657228,327.94737064)(178.12657762,327.91737064)
\curveto(178.01657249,327.87737071)(177.9065726,327.85737073)(177.79657762,327.85737064)
\curveto(177.68657282,327.84737074)(177.57657293,327.83237075)(177.46657762,327.81237064)
\curveto(177.42657308,327.79237079)(177.38657312,327.7873708)(177.34657762,327.79737064)
\curveto(177.3065732,327.80737078)(177.26657324,327.80737078)(177.22657762,327.79737064)
\lineto(177.09157762,327.79737064)
\lineto(176.85157762,327.79737064)
\curveto(176.78157373,327.7873708)(176.71657379,327.79237079)(176.65657762,327.81237064)
\lineto(176.58157762,327.81237064)
\lineto(176.22157762,327.85737064)
\curveto(176.09157442,327.89737069)(175.96657454,327.93237065)(175.84657762,327.96237064)
\curveto(175.72657478,327.99237059)(175.6115749,328.03237055)(175.50157762,328.08237064)
\curveto(175.14157537,328.24237034)(174.84157567,328.43237015)(174.60157762,328.65237064)
\curveto(174.37157614,328.87236971)(174.15657635,329.14236944)(173.95657762,329.46237064)
\curveto(173.9065766,329.54236904)(173.86157665,329.63236895)(173.82157762,329.73237064)
\lineto(173.70157762,330.03237064)
\curveto(173.65157686,330.14236844)(173.61657689,330.25736833)(173.59657762,330.37737064)
\curveto(173.57657693,330.49736809)(173.55157696,330.61736797)(173.52157762,330.73737064)
\curveto(173.511577,330.77736781)(173.506577,330.81736777)(173.50657762,330.85737064)
\curveto(173.506577,330.89736769)(173.50157701,330.93736765)(173.49157762,330.97737064)
\curveto(173.47157704,331.03736755)(173.46157705,331.10236748)(173.46157762,331.17237064)
\curveto(173.47157704,331.24236734)(173.46657704,331.30736728)(173.44657762,331.36737064)
\lineto(173.44657762,331.51737064)
\curveto(173.43657707,331.56736702)(173.43157708,331.63736695)(173.43157762,331.72737064)
\curveto(173.43157708,331.81736677)(173.43657707,331.8873667)(173.44657762,331.93737064)
\curveto(173.45657705,331.9873666)(173.45657705,332.03236655)(173.44657762,332.07237064)
\curveto(173.44657706,332.11236647)(173.45157706,332.15236643)(173.46157762,332.19237064)
\curveto(173.48157703,332.26236632)(173.48657702,332.33236625)(173.47657762,332.40237064)
\curveto(173.47657703,332.47236611)(173.48657702,332.53736605)(173.50657762,332.59737064)
\curveto(173.54657696,332.76736582)(173.58157693,332.93736565)(173.61157762,333.10737064)
\curveto(173.64157687,333.27736531)(173.68657682,333.43736515)(173.74657762,333.58737064)
\curveto(173.95657655,334.10736448)(174.2115763,334.52736406)(174.51157762,334.84737064)
\curveto(174.8115757,335.16736342)(175.22157529,335.43236315)(175.74157762,335.64237064)
\curveto(175.85157466,335.69236289)(175.97157454,335.72736286)(176.10157762,335.74737064)
\curveto(176.23157428,335.76736282)(176.36657414,335.79236279)(176.50657762,335.82237064)
\curveto(176.57657393,335.83236275)(176.64657386,335.83736275)(176.71657762,335.83737064)
\curveto(176.78657372,335.84736274)(176.86157365,335.85736273)(176.94157762,335.86737064)
}
}
{
\newrgbcolor{curcolor}{0 0 0}
\pscustom[linestyle=none,fillstyle=solid,fillcolor=curcolor]
{
\newpath
\moveto(182.14821824,337.18737064)
\curveto(182.06821712,337.24736134)(182.02321717,337.35236123)(182.01321824,337.50237064)
\lineto(182.01321824,337.96737064)
\lineto(182.01321824,338.22237064)
\curveto(182.01321718,338.31236027)(182.02821716,338.3873602)(182.05821824,338.44737064)
\curveto(182.09821709,338.52736006)(182.17821701,338.58736)(182.29821824,338.62737064)
\curveto(182.31821687,338.63735995)(182.33821685,338.63735995)(182.35821824,338.62737064)
\curveto(182.3882168,338.62735996)(182.41321678,338.63235995)(182.43321824,338.64237064)
\curveto(182.60321659,338.64235994)(182.76321643,338.63735995)(182.91321824,338.62737064)
\curveto(183.06321613,338.61735997)(183.16321603,338.55736003)(183.21321824,338.44737064)
\curveto(183.24321595,338.3873602)(183.25821593,338.31236027)(183.25821824,338.22237064)
\lineto(183.25821824,337.96737064)
\curveto(183.25821593,337.7873608)(183.25321594,337.61736097)(183.24321824,337.45737064)
\curveto(183.24321595,337.29736129)(183.17821601,337.19236139)(183.04821824,337.14237064)
\curveto(182.99821619,337.12236146)(182.94321625,337.11236147)(182.88321824,337.11237064)
\lineto(182.71821824,337.11237064)
\lineto(182.40321824,337.11237064)
\curveto(182.30321689,337.11236147)(182.21821697,337.13736145)(182.14821824,337.18737064)
\moveto(183.25821824,328.68237064)
\lineto(183.25821824,328.36737064)
\curveto(183.26821592,328.26737032)(183.24821594,328.1873704)(183.19821824,328.12737064)
\curveto(183.16821602,328.06737052)(183.12321607,328.02737056)(183.06321824,328.00737064)
\curveto(183.00321619,327.99737059)(182.93321626,327.9823706)(182.85321824,327.96237064)
\lineto(182.62821824,327.96237064)
\curveto(182.49821669,327.96237062)(182.38321681,327.96737062)(182.28321824,327.97737064)
\curveto(182.193217,327.99737059)(182.12321707,328.04737054)(182.07321824,328.12737064)
\curveto(182.03321716,328.1873704)(182.01321718,328.26237032)(182.01321824,328.35237064)
\lineto(182.01321824,328.63737064)
\lineto(182.01321824,334.98237064)
\lineto(182.01321824,335.29737064)
\curveto(182.01321718,335.40736318)(182.03821715,335.49236309)(182.08821824,335.55237064)
\curveto(182.11821707,335.60236298)(182.15821703,335.63236295)(182.20821824,335.64237064)
\curveto(182.25821693,335.65236293)(182.31321688,335.66736292)(182.37321824,335.68737064)
\curveto(182.3932168,335.6873629)(182.41321678,335.6823629)(182.43321824,335.67237064)
\curveto(182.46321673,335.67236291)(182.4882167,335.67736291)(182.50821824,335.68737064)
\curveto(182.63821655,335.6873629)(182.76821642,335.6823629)(182.89821824,335.67237064)
\curveto(183.03821615,335.67236291)(183.13321606,335.63236295)(183.18321824,335.55237064)
\curveto(183.23321596,335.49236309)(183.25821593,335.41236317)(183.25821824,335.31237064)
\lineto(183.25821824,335.02737064)
\lineto(183.25821824,328.68237064)
}
}
{
\newrgbcolor{curcolor}{0 0 0}
\pscustom[linestyle=none,fillstyle=solid,fillcolor=curcolor]
{
\newpath
\moveto(192.32806199,332.16237064)
\curveto(192.34805393,332.10236648)(192.35805392,332.00736658)(192.35806199,331.87737064)
\curveto(192.35805392,331.75736683)(192.35305393,331.67236691)(192.34306199,331.62237064)
\lineto(192.34306199,331.47237064)
\curveto(192.33305395,331.39236719)(192.32305396,331.31736727)(192.31306199,331.24737064)
\curveto(192.31305397,331.1873674)(192.30805397,331.11736747)(192.29806199,331.03737064)
\curveto(192.278054,330.97736761)(192.26305402,330.91736767)(192.25306199,330.85737064)
\curveto(192.25305403,330.79736779)(192.24305404,330.73736785)(192.22306199,330.67737064)
\curveto(192.1830541,330.54736804)(192.14805413,330.41736817)(192.11806199,330.28737064)
\curveto(192.08805419,330.15736843)(192.04805423,330.03736855)(191.99806199,329.92737064)
\curveto(191.78805449,329.44736914)(191.50805477,329.04236954)(191.15806199,328.71237064)
\curveto(190.80805547,328.39237019)(190.3780559,328.14737044)(189.86806199,327.97737064)
\curveto(189.75805652,327.93737065)(189.63805664,327.90737068)(189.50806199,327.88737064)
\curveto(189.38805689,327.86737072)(189.26305702,327.84737074)(189.13306199,327.82737064)
\curveto(189.07305721,327.81737077)(189.00805727,327.81237077)(188.93806199,327.81237064)
\curveto(188.8780574,327.80237078)(188.81805746,327.79737079)(188.75806199,327.79737064)
\curveto(188.71805756,327.7873708)(188.65805762,327.7823708)(188.57806199,327.78237064)
\curveto(188.50805777,327.7823708)(188.45805782,327.7873708)(188.42806199,327.79737064)
\curveto(188.38805789,327.80737078)(188.34805793,327.81237077)(188.30806199,327.81237064)
\curveto(188.26805801,327.80237078)(188.23305805,327.80237078)(188.20306199,327.81237064)
\lineto(188.11306199,327.81237064)
\lineto(187.75306199,327.85737064)
\curveto(187.61305867,327.89737069)(187.4780588,327.93737065)(187.34806199,327.97737064)
\curveto(187.21805906,328.01737057)(187.09305919,328.06237052)(186.97306199,328.11237064)
\curveto(186.52305976,328.31237027)(186.15306013,328.57237001)(185.86306199,328.89237064)
\curveto(185.57306071,329.21236937)(185.33306095,329.60236898)(185.14306199,330.06237064)
\curveto(185.09306119,330.16236842)(185.05306123,330.26236832)(185.02306199,330.36237064)
\curveto(185.00306128,330.46236812)(184.9830613,330.56736802)(184.96306199,330.67737064)
\curveto(184.94306134,330.71736787)(184.93306135,330.74736784)(184.93306199,330.76737064)
\curveto(184.94306134,330.79736779)(184.94306134,330.83236775)(184.93306199,330.87237064)
\curveto(184.91306137,330.95236763)(184.89806138,331.03236755)(184.88806199,331.11237064)
\curveto(184.88806139,331.20236738)(184.8780614,331.2873673)(184.85806199,331.36737064)
\lineto(184.85806199,331.48737064)
\curveto(184.85806142,331.52736706)(184.85306143,331.57236701)(184.84306199,331.62237064)
\curveto(184.83306145,331.67236691)(184.82806145,331.75736683)(184.82806199,331.87737064)
\curveto(184.82806145,332.00736658)(184.83806144,332.10236648)(184.85806199,332.16237064)
\curveto(184.8780614,332.23236635)(184.8830614,332.30236628)(184.87306199,332.37237064)
\curveto(184.86306142,332.44236614)(184.86806141,332.51236607)(184.88806199,332.58237064)
\curveto(184.89806138,332.63236595)(184.90306138,332.67236591)(184.90306199,332.70237064)
\curveto(184.91306137,332.74236584)(184.92306136,332.7873658)(184.93306199,332.83737064)
\curveto(184.96306132,332.95736563)(184.98806129,333.07736551)(185.00806199,333.19737064)
\curveto(185.03806124,333.31736527)(185.0780612,333.43236515)(185.12806199,333.54237064)
\curveto(185.278061,333.91236467)(185.45806082,334.24236434)(185.66806199,334.53237064)
\curveto(185.88806039,334.83236375)(186.15306013,335.0823635)(186.46306199,335.28237064)
\curveto(186.5830597,335.36236322)(186.70805957,335.42736316)(186.83806199,335.47737064)
\curveto(186.96805931,335.53736305)(187.10305918,335.59736299)(187.24306199,335.65737064)
\curveto(187.36305892,335.70736288)(187.49305879,335.73736285)(187.63306199,335.74737064)
\curveto(187.77305851,335.76736282)(187.91305837,335.79736279)(188.05306199,335.83737064)
\lineto(188.24806199,335.83737064)
\curveto(188.31805796,335.84736274)(188.3830579,335.85736273)(188.44306199,335.86737064)
\curveto(189.33305695,335.87736271)(190.07305621,335.69236289)(190.66306199,335.31237064)
\curveto(191.25305503,334.93236365)(191.6780546,334.43736415)(191.93806199,333.82737064)
\curveto(191.98805429,333.72736486)(192.02805425,333.62736496)(192.05806199,333.52737064)
\curveto(192.08805419,333.42736516)(192.12305416,333.32236526)(192.16306199,333.21237064)
\curveto(192.19305409,333.10236548)(192.21805406,332.9823656)(192.23806199,332.85237064)
\curveto(192.25805402,332.73236585)(192.283054,332.60736598)(192.31306199,332.47737064)
\curveto(192.32305396,332.42736616)(192.32305396,332.37236621)(192.31306199,332.31237064)
\curveto(192.31305397,332.26236632)(192.31805396,332.21236637)(192.32806199,332.16237064)
\moveto(190.99306199,331.30737064)
\curveto(191.01305527,331.37736721)(191.01805526,331.45736713)(191.00806199,331.54737064)
\lineto(191.00806199,331.80237064)
\curveto(191.00805527,332.19236639)(190.97305531,332.52236606)(190.90306199,332.79237064)
\curveto(190.87305541,332.87236571)(190.84805543,332.95236563)(190.82806199,333.03237064)
\curveto(190.80805547,333.11236547)(190.7830555,333.1873654)(190.75306199,333.25737064)
\curveto(190.47305581,333.90736468)(190.02805625,334.35736423)(189.41806199,334.60737064)
\curveto(189.34805693,334.63736395)(189.27305701,334.65736393)(189.19306199,334.66737064)
\lineto(188.95306199,334.72737064)
\curveto(188.87305741,334.74736384)(188.78805749,334.75736383)(188.69806199,334.75737064)
\lineto(188.42806199,334.75737064)
\lineto(188.15806199,334.71237064)
\curveto(188.05805822,334.69236389)(187.96305832,334.66736392)(187.87306199,334.63737064)
\curveto(187.79305849,334.61736397)(187.71305857,334.587364)(187.63306199,334.54737064)
\curveto(187.56305872,334.52736406)(187.49805878,334.49736409)(187.43806199,334.45737064)
\curveto(187.3780589,334.41736417)(187.32305896,334.37736421)(187.27306199,334.33737064)
\curveto(187.03305925,334.16736442)(186.83805944,333.96236462)(186.68806199,333.72237064)
\curveto(186.53805974,333.4823651)(186.40805987,333.20236538)(186.29806199,332.88237064)
\curveto(186.26806001,332.7823658)(186.24806003,332.67736591)(186.23806199,332.56737064)
\curveto(186.22806005,332.46736612)(186.21306007,332.36236622)(186.19306199,332.25237064)
\curveto(186.1830601,332.21236637)(186.1780601,332.14736644)(186.17806199,332.05737064)
\curveto(186.16806011,332.02736656)(186.16306012,331.99236659)(186.16306199,331.95237064)
\curveto(186.17306011,331.91236667)(186.1780601,331.86736672)(186.17806199,331.81737064)
\lineto(186.17806199,331.51737064)
\curveto(186.1780601,331.41736717)(186.18806009,331.32736726)(186.20806199,331.24737064)
\lineto(186.23806199,331.06737064)
\curveto(186.25806002,330.96736762)(186.27306001,330.86736772)(186.28306199,330.76737064)
\curveto(186.30305998,330.67736791)(186.33305995,330.59236799)(186.37306199,330.51237064)
\curveto(186.47305981,330.27236831)(186.58805969,330.04736854)(186.71806199,329.83737064)
\curveto(186.85805942,329.62736896)(187.02805925,329.45236913)(187.22806199,329.31237064)
\curveto(187.278059,329.2823693)(187.32305896,329.25736933)(187.36306199,329.23737064)
\curveto(187.40305888,329.21736937)(187.44805883,329.19236939)(187.49806199,329.16237064)
\curveto(187.5780587,329.11236947)(187.66305862,329.06736952)(187.75306199,329.02737064)
\curveto(187.85305843,328.99736959)(187.95805832,328.96736962)(188.06806199,328.93737064)
\curveto(188.11805816,328.91736967)(188.16305812,328.90736968)(188.20306199,328.90737064)
\curveto(188.25305803,328.91736967)(188.30305798,328.91736967)(188.35306199,328.90737064)
\curveto(188.3830579,328.89736969)(188.44305784,328.8873697)(188.53306199,328.87737064)
\curveto(188.63305765,328.86736972)(188.70805757,328.87236971)(188.75806199,328.89237064)
\curveto(188.79805748,328.90236968)(188.83805744,328.90236968)(188.87806199,328.89237064)
\curveto(188.91805736,328.89236969)(188.95805732,328.90236968)(188.99806199,328.92237064)
\curveto(189.0780572,328.94236964)(189.15805712,328.95736963)(189.23806199,328.96737064)
\curveto(189.31805696,328.9873696)(189.39305689,329.01236957)(189.46306199,329.04237064)
\curveto(189.80305648,329.1823694)(190.0780562,329.37736921)(190.28806199,329.62737064)
\curveto(190.49805578,329.87736871)(190.67305561,330.17236841)(190.81306199,330.51237064)
\curveto(190.86305542,330.63236795)(190.89305539,330.75736783)(190.90306199,330.88737064)
\curveto(190.92305536,331.02736756)(190.95305533,331.16736742)(190.99306199,331.30737064)
}
}
{
\newrgbcolor{curcolor}{0 0 0}
\pscustom[linestyle=none,fillstyle=solid,fillcolor=curcolor]
{
\newpath
\moveto(196.24634324,335.86737064)
\curveto(196.96633918,335.87736271)(197.57133857,335.79236279)(198.06134324,335.61237064)
\curveto(198.55133759,335.44236314)(198.93133721,335.13736345)(199.20134324,334.69737064)
\curveto(199.27133687,334.587364)(199.32633682,334.47236411)(199.36634324,334.35237064)
\curveto(199.40633674,334.24236434)(199.4463367,334.11736447)(199.48634324,333.97737064)
\curveto(199.50633664,333.90736468)(199.51133663,333.83236475)(199.50134324,333.75237064)
\curveto(199.49133665,333.6823649)(199.47633667,333.62736496)(199.45634324,333.58737064)
\curveto(199.43633671,333.56736502)(199.41133673,333.54736504)(199.38134324,333.52737064)
\curveto(199.35133679,333.51736507)(199.32633682,333.50236508)(199.30634324,333.48237064)
\curveto(199.25633689,333.46236512)(199.20633694,333.45736513)(199.15634324,333.46737064)
\curveto(199.10633704,333.47736511)(199.05633709,333.47736511)(199.00634324,333.46737064)
\curveto(198.92633722,333.44736514)(198.82133732,333.44236514)(198.69134324,333.45237064)
\curveto(198.56133758,333.47236511)(198.47133767,333.49736509)(198.42134324,333.52737064)
\curveto(198.3413378,333.57736501)(198.28633786,333.64236494)(198.25634324,333.72237064)
\curveto(198.23633791,333.81236477)(198.20133794,333.89736469)(198.15134324,333.97737064)
\curveto(198.06133808,334.13736445)(197.93633821,334.2823643)(197.77634324,334.41237064)
\curveto(197.66633848,334.49236409)(197.5463386,334.55236403)(197.41634324,334.59237064)
\curveto(197.28633886,334.63236395)(197.146339,334.67236391)(196.99634324,334.71237064)
\curveto(196.9463392,334.73236385)(196.89633925,334.73736385)(196.84634324,334.72737064)
\curveto(196.79633935,334.72736386)(196.7463394,334.73236385)(196.69634324,334.74237064)
\curveto(196.63633951,334.76236382)(196.56133958,334.77236381)(196.47134324,334.77237064)
\curveto(196.38133976,334.77236381)(196.30633984,334.76236382)(196.24634324,334.74237064)
\lineto(196.15634324,334.74237064)
\lineto(196.00634324,334.71237064)
\curveto(195.95634019,334.71236387)(195.90634024,334.70736388)(195.85634324,334.69737064)
\curveto(195.59634055,334.63736395)(195.38134076,334.55236403)(195.21134324,334.44237064)
\curveto(195.0413411,334.33236425)(194.92634122,334.14736444)(194.86634324,333.88737064)
\curveto(194.8463413,333.81736477)(194.8413413,333.74736484)(194.85134324,333.67737064)
\curveto(194.87134127,333.60736498)(194.89134125,333.54736504)(194.91134324,333.49737064)
\curveto(194.97134117,333.34736524)(195.0413411,333.23736535)(195.12134324,333.16737064)
\curveto(195.21134093,333.10736548)(195.32134082,333.03736555)(195.45134324,332.95737064)
\curveto(195.61134053,332.85736573)(195.79134035,332.7823658)(195.99134324,332.73237064)
\curveto(196.19133995,332.69236589)(196.39133975,332.64236594)(196.59134324,332.58237064)
\curveto(196.72133942,332.54236604)(196.85133929,332.51236607)(196.98134324,332.49237064)
\curveto(197.11133903,332.47236611)(197.2413389,332.44236614)(197.37134324,332.40237064)
\curveto(197.58133856,332.34236624)(197.78633836,332.2823663)(197.98634324,332.22237064)
\curveto(198.18633796,332.17236641)(198.38633776,332.10736648)(198.58634324,332.02737064)
\lineto(198.73634324,331.96737064)
\curveto(198.78633736,331.94736664)(198.83633731,331.92236666)(198.88634324,331.89237064)
\curveto(199.08633706,331.77236681)(199.26133688,331.63736695)(199.41134324,331.48737064)
\curveto(199.56133658,331.33736725)(199.68633646,331.14736744)(199.78634324,330.91737064)
\curveto(199.80633634,330.84736774)(199.82633632,330.75236783)(199.84634324,330.63237064)
\curveto(199.86633628,330.56236802)(199.87633627,330.4873681)(199.87634324,330.40737064)
\curveto(199.88633626,330.33736825)(199.89133625,330.25736833)(199.89134324,330.16737064)
\lineto(199.89134324,330.01737064)
\curveto(199.87133627,329.94736864)(199.86133628,329.87736871)(199.86134324,329.80737064)
\curveto(199.86133628,329.73736885)(199.85133629,329.66736892)(199.83134324,329.59737064)
\curveto(199.80133634,329.4873691)(199.76633638,329.3823692)(199.72634324,329.28237064)
\curveto(199.68633646,329.1823694)(199.6413365,329.09236949)(199.59134324,329.01237064)
\curveto(199.43133671,328.75236983)(199.22633692,328.54237004)(198.97634324,328.38237064)
\curveto(198.72633742,328.23237035)(198.4463377,328.10237048)(198.13634324,327.99237064)
\curveto(198.0463381,327.96237062)(197.95133819,327.94237064)(197.85134324,327.93237064)
\curveto(197.76133838,327.91237067)(197.67133847,327.8873707)(197.58134324,327.85737064)
\curveto(197.48133866,327.83737075)(197.38133876,327.82737076)(197.28134324,327.82737064)
\curveto(197.18133896,327.82737076)(197.08133906,327.81737077)(196.98134324,327.79737064)
\lineto(196.83134324,327.79737064)
\curveto(196.78133936,327.7873708)(196.71133943,327.7823708)(196.62134324,327.78237064)
\curveto(196.53133961,327.7823708)(196.46133968,327.7873708)(196.41134324,327.79737064)
\lineto(196.24634324,327.79737064)
\curveto(196.18633996,327.81737077)(196.12134002,327.82737076)(196.05134324,327.82737064)
\curveto(195.98134016,327.81737077)(195.92134022,327.82237076)(195.87134324,327.84237064)
\curveto(195.82134032,327.85237073)(195.75634039,327.85737073)(195.67634324,327.85737064)
\lineto(195.43634324,327.91737064)
\curveto(195.36634078,327.92737066)(195.29134085,327.94737064)(195.21134324,327.97737064)
\curveto(194.90134124,328.07737051)(194.63134151,328.20237038)(194.40134324,328.35237064)
\curveto(194.17134197,328.50237008)(193.97134217,328.69736989)(193.80134324,328.93737064)
\curveto(193.71134243,329.06736952)(193.63634251,329.20236938)(193.57634324,329.34237064)
\curveto(193.51634263,329.4823691)(193.46134268,329.63736895)(193.41134324,329.80737064)
\curveto(193.39134275,329.86736872)(193.38134276,329.93736865)(193.38134324,330.01737064)
\curveto(193.39134275,330.10736848)(193.40634274,330.17736841)(193.42634324,330.22737064)
\curveto(193.45634269,330.26736832)(193.50634264,330.30736828)(193.57634324,330.34737064)
\curveto(193.62634252,330.36736822)(193.69634245,330.37736821)(193.78634324,330.37737064)
\curveto(193.87634227,330.3873682)(193.96634218,330.3873682)(194.05634324,330.37737064)
\curveto(194.146342,330.36736822)(194.23134191,330.35236823)(194.31134324,330.33237064)
\curveto(194.40134174,330.32236826)(194.46134168,330.30736828)(194.49134324,330.28737064)
\curveto(194.56134158,330.23736835)(194.60634154,330.16236842)(194.62634324,330.06237064)
\curveto(194.65634149,329.97236861)(194.69134145,329.8873687)(194.73134324,329.80737064)
\curveto(194.83134131,329.587369)(194.96634118,329.41736917)(195.13634324,329.29737064)
\curveto(195.25634089,329.20736938)(195.39134075,329.13736945)(195.54134324,329.08737064)
\curveto(195.69134045,329.03736955)(195.85134029,328.9873696)(196.02134324,328.93737064)
\lineto(196.33634324,328.89237064)
\lineto(196.42634324,328.89237064)
\curveto(196.49633965,328.87236971)(196.58633956,328.86236972)(196.69634324,328.86237064)
\curveto(196.81633933,328.86236972)(196.91633923,328.87236971)(196.99634324,328.89237064)
\curveto(197.06633908,328.89236969)(197.12133902,328.89736969)(197.16134324,328.90737064)
\curveto(197.22133892,328.91736967)(197.28133886,328.92236966)(197.34134324,328.92237064)
\curveto(197.40133874,328.93236965)(197.45633869,328.94236964)(197.50634324,328.95237064)
\curveto(197.79633835,329.03236955)(198.02633812,329.13736945)(198.19634324,329.26737064)
\curveto(198.36633778,329.39736919)(198.48633766,329.61736897)(198.55634324,329.92737064)
\curveto(198.57633757,329.97736861)(198.58133756,330.03236855)(198.57134324,330.09237064)
\curveto(198.56133758,330.15236843)(198.55133759,330.19736839)(198.54134324,330.22737064)
\curveto(198.49133765,330.41736817)(198.42133772,330.55736803)(198.33134324,330.64737064)
\curveto(198.2413379,330.74736784)(198.12633802,330.83736775)(197.98634324,330.91737064)
\curveto(197.89633825,330.97736761)(197.79633835,331.02736756)(197.68634324,331.06737064)
\lineto(197.35634324,331.18737064)
\curveto(197.32633882,331.19736739)(197.29633885,331.20236738)(197.26634324,331.20237064)
\curveto(197.2463389,331.20236738)(197.22133892,331.21236737)(197.19134324,331.23237064)
\curveto(196.85133929,331.34236724)(196.49633965,331.42236716)(196.12634324,331.47237064)
\curveto(195.76634038,331.53236705)(195.42634072,331.62736696)(195.10634324,331.75737064)
\curveto(195.00634114,331.79736679)(194.91134123,331.83236675)(194.82134324,331.86237064)
\curveto(194.73134141,331.89236669)(194.6463415,331.93236665)(194.56634324,331.98237064)
\curveto(194.37634177,332.09236649)(194.20134194,332.21736637)(194.04134324,332.35737064)
\curveto(193.88134226,332.49736609)(193.75634239,332.67236591)(193.66634324,332.88237064)
\curveto(193.63634251,332.95236563)(193.61134253,333.02236556)(193.59134324,333.09237064)
\curveto(193.58134256,333.16236542)(193.56634258,333.23736535)(193.54634324,333.31737064)
\curveto(193.51634263,333.43736515)(193.50634264,333.57236501)(193.51634324,333.72237064)
\curveto(193.52634262,333.8823647)(193.5413426,334.01736457)(193.56134324,334.12737064)
\curveto(193.58134256,334.17736441)(193.59134255,334.21736437)(193.59134324,334.24737064)
\curveto(193.60134254,334.2873643)(193.61634253,334.32736426)(193.63634324,334.36737064)
\curveto(193.72634242,334.59736399)(193.8463423,334.79736379)(193.99634324,334.96737064)
\curveto(194.15634199,335.13736345)(194.33634181,335.2873633)(194.53634324,335.41737064)
\curveto(194.68634146,335.50736308)(194.85134129,335.57736301)(195.03134324,335.62737064)
\curveto(195.21134093,335.6873629)(195.40134074,335.74236284)(195.60134324,335.79237064)
\curveto(195.67134047,335.80236278)(195.73634041,335.81236277)(195.79634324,335.82237064)
\curveto(195.86634028,335.83236275)(195.9413402,335.84236274)(196.02134324,335.85237064)
\curveto(196.05134009,335.86236272)(196.09134005,335.86236272)(196.14134324,335.85237064)
\curveto(196.19133995,335.84236274)(196.22633992,335.84736274)(196.24634324,335.86737064)
}
}
{
\newrgbcolor{curcolor}{0.50196081 0.50196081 0.50196081}
\pscustom[linestyle=none,fillstyle=solid,fillcolor=curcolor]
{
\newpath
\moveto(120.84556474,341.17243777)
\lineto(135.84556474,341.17243777)
\lineto(135.84556474,326.17243777)
\lineto(120.84556474,326.17243777)
\closepath
}
}
{
\newrgbcolor{curcolor}{0 0 0}
\pscustom[linestyle=none,fillstyle=solid,fillcolor=curcolor]
{
\newpath
\moveto(263.84668701,31.67142873)
\lineto(263.84668701,32.58642873)
\curveto(263.84669771,32.68642608)(263.84669771,32.78142599)(263.84668701,32.87142873)
\curveto(263.84669771,32.96142581)(263.86669769,33.03642573)(263.90668701,33.09642873)
\curveto(263.96669759,33.18642558)(264.04669751,33.24642552)(264.14668701,33.27642873)
\curveto(264.24669731,33.31642545)(264.3516972,33.36142541)(264.46168701,33.41142873)
\curveto(264.6516969,33.49142528)(264.84169671,33.56142521)(265.03168701,33.62142873)
\curveto(265.22169633,33.69142508)(265.41169614,33.766425)(265.60168701,33.84642873)
\curveto(265.78169577,33.91642485)(265.96669559,33.98142479)(266.15668701,34.04142873)
\curveto(266.33669522,34.10142467)(266.51669504,34.1714246)(266.69668701,34.25142873)
\curveto(266.83669472,34.31142446)(266.98169457,34.3664244)(267.13168701,34.41642873)
\curveto(267.28169427,34.4664243)(267.42669413,34.52142425)(267.56668701,34.58142873)
\curveto(268.01669354,34.76142401)(268.47169308,34.93142384)(268.93168701,35.09142873)
\curveto(269.38169217,35.25142352)(269.83169172,35.42142335)(270.28168701,35.60142873)
\curveto(270.33169122,35.62142315)(270.38169117,35.63642313)(270.43168701,35.64642873)
\lineto(270.58168701,35.70642873)
\curveto(270.80169075,35.79642297)(271.02669053,35.88142289)(271.25668701,35.96142873)
\curveto(271.47669008,36.04142273)(271.69668986,36.12642264)(271.91668701,36.21642873)
\curveto(272.00668955,36.25642251)(272.11668944,36.29642247)(272.24668701,36.33642873)
\curveto(272.36668919,36.37642239)(272.43668912,36.44142233)(272.45668701,36.53142873)
\curveto(272.46668909,36.5714222)(272.46668909,36.60142217)(272.45668701,36.62142873)
\lineto(272.39668701,36.68142873)
\curveto(272.34668921,36.73142204)(272.29168926,36.766422)(272.23168701,36.78642873)
\curveto(272.17168938,36.81642195)(272.10668945,36.84642192)(272.03668701,36.87642873)
\lineto(271.40668701,37.11642873)
\curveto(271.18669037,37.19642157)(270.97169058,37.27642149)(270.76168701,37.35642873)
\lineto(270.61168701,37.41642873)
\lineto(270.43168701,37.47642873)
\curveto(270.24169131,37.55642121)(270.0516915,37.62642114)(269.86168701,37.68642873)
\curveto(269.66169189,37.75642101)(269.46169209,37.83142094)(269.26168701,37.91142873)
\curveto(268.68169287,38.15142062)(268.09669346,38.3714204)(267.50668701,38.57142873)
\curveto(266.91669464,38.78141999)(266.33169522,39.00641976)(265.75168701,39.24642873)
\curveto(265.551696,39.32641944)(265.34669621,39.40141937)(265.13668701,39.47142873)
\curveto(264.92669663,39.55141922)(264.72169683,39.63141914)(264.52168701,39.71142873)
\curveto(264.44169711,39.75141902)(264.34169721,39.78641898)(264.22168701,39.81642873)
\curveto(264.10169745,39.85641891)(264.01669754,39.91141886)(263.96668701,39.98142873)
\curveto(263.92669763,40.04141873)(263.89669766,40.11641865)(263.87668701,40.20642873)
\curveto(263.8566977,40.30641846)(263.84669771,40.41641835)(263.84668701,40.53642873)
\curveto(263.83669772,40.65641811)(263.83669772,40.77641799)(263.84668701,40.89642873)
\curveto(263.84669771,41.01641775)(263.84669771,41.12641764)(263.84668701,41.22642873)
\curveto(263.84669771,41.31641745)(263.84669771,41.40641736)(263.84668701,41.49642873)
\curveto(263.84669771,41.59641717)(263.86669769,41.6714171)(263.90668701,41.72142873)
\curveto(263.9566976,41.81141696)(264.04669751,41.86141691)(264.17668701,41.87142873)
\curveto(264.30669725,41.88141689)(264.44669711,41.88641688)(264.59668701,41.88642873)
\lineto(266.24668701,41.88642873)
\lineto(272.51668701,41.88642873)
\lineto(273.77668701,41.88642873)
\curveto(273.88668767,41.88641688)(273.99668756,41.88641688)(274.10668701,41.88642873)
\curveto(274.21668734,41.89641687)(274.30168725,41.87641689)(274.36168701,41.82642873)
\curveto(274.42168713,41.79641697)(274.46168709,41.75141702)(274.48168701,41.69142873)
\curveto(274.49168706,41.63141714)(274.50668705,41.56141721)(274.52668701,41.48142873)
\lineto(274.52668701,41.24142873)
\lineto(274.52668701,40.88142873)
\curveto(274.51668704,40.771418)(274.47168708,40.69141808)(274.39168701,40.64142873)
\curveto(274.36168719,40.62141815)(274.33168722,40.60641816)(274.30168701,40.59642873)
\curveto(274.26168729,40.59641817)(274.21668734,40.58641818)(274.16668701,40.56642873)
\lineto(274.00168701,40.56642873)
\curveto(273.94168761,40.55641821)(273.87168768,40.55141822)(273.79168701,40.55142873)
\curveto(273.71168784,40.56141821)(273.63668792,40.5664182)(273.56668701,40.56642873)
\lineto(272.72668701,40.56642873)
\lineto(268.30168701,40.56642873)
\curveto(268.0516935,40.5664182)(267.80169375,40.5664182)(267.55168701,40.56642873)
\curveto(267.29169426,40.5664182)(267.04169451,40.56141821)(266.80168701,40.55142873)
\curveto(266.70169485,40.55141822)(266.59169496,40.54641822)(266.47168701,40.53642873)
\curveto(266.3516952,40.52641824)(266.29169526,40.4714183)(266.29168701,40.37142873)
\lineto(266.30668701,40.37142873)
\curveto(266.32669523,40.30141847)(266.39169516,40.24141853)(266.50168701,40.19142873)
\curveto(266.61169494,40.15141862)(266.70669485,40.11641865)(266.78668701,40.08642873)
\curveto(266.9566946,40.01641875)(267.13169442,39.95141882)(267.31168701,39.89142873)
\curveto(267.48169407,39.83141894)(267.6516939,39.76141901)(267.82168701,39.68142873)
\curveto(267.87169368,39.66141911)(267.91669364,39.64641912)(267.95668701,39.63642873)
\curveto(267.99669356,39.62641914)(268.04169351,39.61141916)(268.09168701,39.59142873)
\curveto(268.27169328,39.51141926)(268.4566931,39.44141933)(268.64668701,39.38142873)
\curveto(268.82669273,39.33141944)(269.00669255,39.2664195)(269.18668701,39.18642873)
\curveto(269.33669222,39.11641965)(269.49169206,39.05641971)(269.65168701,39.00642873)
\curveto(269.80169175,38.95641981)(269.9516916,38.90141987)(270.10168701,38.84142873)
\curveto(270.57169098,38.64142013)(271.04669051,38.46142031)(271.52668701,38.30142873)
\curveto(271.99668956,38.14142063)(272.46168909,37.9664208)(272.92168701,37.77642873)
\curveto(273.10168845,37.69642107)(273.28168827,37.62642114)(273.46168701,37.56642873)
\curveto(273.64168791,37.50642126)(273.82168773,37.44142133)(274.00168701,37.37142873)
\curveto(274.11168744,37.32142145)(274.21668734,37.2714215)(274.31668701,37.22142873)
\curveto(274.40668715,37.18142159)(274.47168708,37.09642167)(274.51168701,36.96642873)
\curveto(274.52168703,36.94642182)(274.52668703,36.92142185)(274.52668701,36.89142873)
\curveto(274.51668704,36.8714219)(274.51668704,36.84642192)(274.52668701,36.81642873)
\curveto(274.53668702,36.78642198)(274.54168701,36.75142202)(274.54168701,36.71142873)
\curveto(274.53168702,36.6714221)(274.52668703,36.63142214)(274.52668701,36.59142873)
\lineto(274.52668701,36.29142873)
\curveto(274.52668703,36.19142258)(274.50168705,36.11142266)(274.45168701,36.05142873)
\curveto(274.40168715,35.9714228)(274.33168722,35.91142286)(274.24168701,35.87142873)
\curveto(274.14168741,35.84142293)(274.04168751,35.80142297)(273.94168701,35.75142873)
\curveto(273.74168781,35.6714231)(273.53668802,35.59142318)(273.32668701,35.51142873)
\curveto(273.10668845,35.44142333)(272.89668866,35.3664234)(272.69668701,35.28642873)
\curveto(272.51668904,35.20642356)(272.33668922,35.13642363)(272.15668701,35.07642873)
\curveto(271.96668959,35.02642374)(271.78168977,34.96142381)(271.60168701,34.88142873)
\curveto(271.04169051,34.65142412)(270.47669108,34.43642433)(269.90668701,34.23642873)
\curveto(269.33669222,34.03642473)(268.77169278,33.82142495)(268.21168701,33.59142873)
\lineto(267.58168701,33.35142873)
\curveto(267.36169419,33.28142549)(267.1516944,33.20642556)(266.95168701,33.12642873)
\curveto(266.84169471,33.07642569)(266.73669482,33.03142574)(266.63668701,32.99142873)
\curveto(266.52669503,32.96142581)(266.43169512,32.91142586)(266.35168701,32.84142873)
\curveto(266.33169522,32.83142594)(266.32169523,32.82142595)(266.32168701,32.81142873)
\lineto(266.29168701,32.78142873)
\lineto(266.29168701,32.70642873)
\lineto(266.32168701,32.67642873)
\curveto(266.32169523,32.6664261)(266.32669523,32.65642611)(266.33668701,32.64642873)
\curveto(266.38669517,32.62642614)(266.44169511,32.61642615)(266.50168701,32.61642873)
\curveto(266.56169499,32.61642615)(266.62169493,32.60642616)(266.68168701,32.58642873)
\lineto(266.84668701,32.58642873)
\curveto(266.90669465,32.5664262)(266.97169458,32.56142621)(267.04168701,32.57142873)
\curveto(267.11169444,32.58142619)(267.18169437,32.58642618)(267.25168701,32.58642873)
\lineto(268.06168701,32.58642873)
\lineto(272.62168701,32.58642873)
\lineto(273.80668701,32.58642873)
\curveto(273.91668764,32.58642618)(274.02668753,32.58142619)(274.13668701,32.57142873)
\curveto(274.24668731,32.5714262)(274.33168722,32.54642622)(274.39168701,32.49642873)
\curveto(274.47168708,32.44642632)(274.51668704,32.35642641)(274.52668701,32.22642873)
\lineto(274.52668701,31.83642873)
\lineto(274.52668701,31.64142873)
\curveto(274.52668703,31.59142718)(274.51668704,31.54142723)(274.49668701,31.49142873)
\curveto(274.4566871,31.36142741)(274.37168718,31.28642748)(274.24168701,31.26642873)
\curveto(274.11168744,31.25642751)(273.96168759,31.25142752)(273.79168701,31.25142873)
\lineto(272.05168701,31.25142873)
\lineto(266.05168701,31.25142873)
\lineto(264.64168701,31.25142873)
\curveto(264.53169702,31.25142752)(264.41669714,31.24642752)(264.29668701,31.23642873)
\curveto(264.17669738,31.23642753)(264.08169747,31.26142751)(264.01168701,31.31142873)
\curveto(263.9516976,31.35142742)(263.90169765,31.42642734)(263.86168701,31.53642873)
\curveto(263.8516977,31.55642721)(263.8516977,31.57642719)(263.86168701,31.59642873)
\curveto(263.86169769,31.62642714)(263.8566977,31.65142712)(263.84668701,31.67142873)
}
}
{
\newrgbcolor{curcolor}{0 0 0}
\pscustom[linestyle=none,fillstyle=solid,fillcolor=curcolor]
{
\newpath
\moveto(273.97168701,50.87353811)
\curveto(274.13168742,50.90353028)(274.26668729,50.88853029)(274.37668701,50.82853811)
\curveto(274.47668708,50.76853041)(274.551687,50.68853049)(274.60168701,50.58853811)
\curveto(274.62168693,50.53853064)(274.63168692,50.4835307)(274.63168701,50.42353811)
\curveto(274.63168692,50.37353081)(274.64168691,50.31853086)(274.66168701,50.25853811)
\curveto(274.71168684,50.03853114)(274.69668686,49.81853136)(274.61668701,49.59853811)
\curveto(274.54668701,49.38853179)(274.4566871,49.24353194)(274.34668701,49.16353811)
\curveto(274.27668728,49.11353207)(274.19668736,49.06853211)(274.10668701,49.02853811)
\curveto(274.00668755,48.98853219)(273.92668763,48.93853224)(273.86668701,48.87853811)
\curveto(273.84668771,48.85853232)(273.82668773,48.83353235)(273.80668701,48.80353811)
\curveto(273.78668777,48.7835324)(273.78168777,48.75353243)(273.79168701,48.71353811)
\curveto(273.82168773,48.60353258)(273.87668768,48.49853268)(273.95668701,48.39853811)
\curveto(274.03668752,48.30853287)(274.10668745,48.21853296)(274.16668701,48.12853811)
\curveto(274.24668731,47.99853318)(274.32168723,47.85853332)(274.39168701,47.70853811)
\curveto(274.4516871,47.55853362)(274.50668705,47.39853378)(274.55668701,47.22853811)
\curveto(274.58668697,47.12853405)(274.60668695,47.01853416)(274.61668701,46.89853811)
\curveto(274.62668693,46.78853439)(274.64168691,46.6785345)(274.66168701,46.56853811)
\curveto(274.67168688,46.51853466)(274.67668688,46.47353471)(274.67668701,46.43353811)
\lineto(274.67668701,46.32853811)
\curveto(274.69668686,46.21853496)(274.69668686,46.11353507)(274.67668701,46.01353811)
\lineto(274.67668701,45.87853811)
\curveto(274.66668689,45.82853535)(274.66168689,45.7785354)(274.66168701,45.72853811)
\curveto(274.66168689,45.6785355)(274.6516869,45.63353555)(274.63168701,45.59353811)
\curveto(274.62168693,45.55353563)(274.61668694,45.51853566)(274.61668701,45.48853811)
\curveto(274.62668693,45.46853571)(274.62668693,45.44353574)(274.61668701,45.41353811)
\lineto(274.55668701,45.17353811)
\curveto(274.54668701,45.09353609)(274.52668703,45.01853616)(274.49668701,44.94853811)
\curveto(274.36668719,44.64853653)(274.22168733,44.40353678)(274.06168701,44.21353811)
\curveto(273.89168766,44.03353715)(273.6566879,43.8835373)(273.35668701,43.76353811)
\curveto(273.13668842,43.67353751)(272.87168868,43.62853755)(272.56168701,43.62853811)
\lineto(272.24668701,43.62853811)
\curveto(272.19668936,43.63853754)(272.14668941,43.64353754)(272.09668701,43.64353811)
\lineto(271.91668701,43.67353811)
\lineto(271.58668701,43.79353811)
\curveto(271.47669008,43.83353735)(271.37669018,43.8835373)(271.28668701,43.94353811)
\curveto(270.99669056,44.12353706)(270.78169077,44.36853681)(270.64168701,44.67853811)
\curveto(270.50169105,44.98853619)(270.37669118,45.32853585)(270.26668701,45.69853811)
\curveto(270.22669133,45.83853534)(270.19669136,45.9835352)(270.17668701,46.13353811)
\curveto(270.1566914,46.2835349)(270.13169142,46.43353475)(270.10168701,46.58353811)
\curveto(270.08169147,46.65353453)(270.07169148,46.71853446)(270.07168701,46.77853811)
\curveto(270.07169148,46.84853433)(270.06169149,46.92353426)(270.04168701,47.00353811)
\curveto(270.02169153,47.07353411)(270.01169154,47.14353404)(270.01168701,47.21353811)
\curveto(270.00169155,47.2835339)(269.98669157,47.35853382)(269.96668701,47.43853811)
\curveto(269.90669165,47.68853349)(269.8566917,47.92353326)(269.81668701,48.14353811)
\curveto(269.76669179,48.36353282)(269.6516919,48.53853264)(269.47168701,48.66853811)
\curveto(269.39169216,48.72853245)(269.29169226,48.7785324)(269.17168701,48.81853811)
\curveto(269.04169251,48.85853232)(268.90169265,48.85853232)(268.75168701,48.81853811)
\curveto(268.51169304,48.75853242)(268.32169323,48.66853251)(268.18168701,48.54853811)
\curveto(268.04169351,48.43853274)(267.93169362,48.2785329)(267.85168701,48.06853811)
\curveto(267.80169375,47.94853323)(267.76669379,47.80353338)(267.74668701,47.63353811)
\curveto(267.72669383,47.47353371)(267.71669384,47.30353388)(267.71668701,47.12353811)
\curveto(267.71669384,46.94353424)(267.72669383,46.76853441)(267.74668701,46.59853811)
\curveto(267.76669379,46.42853475)(267.79669376,46.2835349)(267.83668701,46.16353811)
\curveto(267.89669366,45.99353519)(267.98169357,45.82853535)(268.09168701,45.66853811)
\curveto(268.1516934,45.58853559)(268.23169332,45.51353567)(268.33168701,45.44353811)
\curveto(268.42169313,45.3835358)(268.52169303,45.32853585)(268.63168701,45.27853811)
\curveto(268.71169284,45.24853593)(268.79669276,45.21853596)(268.88668701,45.18853811)
\curveto(268.97669258,45.16853601)(269.04669251,45.12353606)(269.09668701,45.05353811)
\curveto(269.12669243,45.01353617)(269.1516924,44.94353624)(269.17168701,44.84353811)
\curveto(269.18169237,44.75353643)(269.18669237,44.65853652)(269.18668701,44.55853811)
\curveto(269.18669237,44.45853672)(269.18169237,44.35853682)(269.17168701,44.25853811)
\curveto(269.1516924,44.16853701)(269.12669243,44.10353708)(269.09668701,44.06353811)
\curveto(269.06669249,44.02353716)(269.01669254,43.99353719)(268.94668701,43.97353811)
\curveto(268.87669268,43.95353723)(268.80169275,43.95353723)(268.72168701,43.97353811)
\curveto(268.59169296,44.00353718)(268.47169308,44.03353715)(268.36168701,44.06353811)
\curveto(268.24169331,44.10353708)(268.12669343,44.14853703)(268.01668701,44.19853811)
\curveto(267.66669389,44.38853679)(267.39669416,44.62853655)(267.20668701,44.91853811)
\curveto(267.00669455,45.20853597)(266.84669471,45.56853561)(266.72668701,45.99853811)
\curveto(266.70669485,46.09853508)(266.69169486,46.19853498)(266.68168701,46.29853811)
\curveto(266.67169488,46.40853477)(266.6566949,46.51853466)(266.63668701,46.62853811)
\curveto(266.62669493,46.66853451)(266.62669493,46.73353445)(266.63668701,46.82353811)
\curveto(266.63669492,46.91353427)(266.62669493,46.96853421)(266.60668701,46.98853811)
\curveto(266.59669496,47.68853349)(266.67669488,48.29853288)(266.84668701,48.81853811)
\curveto(267.01669454,49.33853184)(267.34169421,49.70353148)(267.82168701,49.91353811)
\curveto(268.02169353,50.00353118)(268.2566933,50.05353113)(268.52668701,50.06353811)
\curveto(268.78669277,50.0835311)(269.06169249,50.09353109)(269.35168701,50.09353811)
\lineto(272.66668701,50.09353811)
\curveto(272.80668875,50.09353109)(272.94168861,50.09853108)(273.07168701,50.10853811)
\curveto(273.20168835,50.11853106)(273.30668825,50.14853103)(273.38668701,50.19853811)
\curveto(273.4566881,50.24853093)(273.50668805,50.31353087)(273.53668701,50.39353811)
\curveto(273.57668798,50.4835307)(273.60668795,50.56853061)(273.62668701,50.64853811)
\curveto(273.63668792,50.72853045)(273.68168787,50.78853039)(273.76168701,50.82853811)
\curveto(273.79168776,50.84853033)(273.82168773,50.85853032)(273.85168701,50.85853811)
\curveto(273.88168767,50.85853032)(273.92168763,50.86353032)(273.97168701,50.87353811)
\moveto(272.30668701,48.72853811)
\curveto(272.16668939,48.78853239)(272.00668955,48.81853236)(271.82668701,48.81853811)
\curveto(271.63668992,48.82853235)(271.44169011,48.83353235)(271.24168701,48.83353811)
\curveto(271.13169042,48.83353235)(271.03169052,48.82853235)(270.94168701,48.81853811)
\curveto(270.8516907,48.80853237)(270.78169077,48.76853241)(270.73168701,48.69853811)
\curveto(270.71169084,48.66853251)(270.70169085,48.59853258)(270.70168701,48.48853811)
\curveto(270.72169083,48.46853271)(270.73169082,48.43353275)(270.73168701,48.38353811)
\curveto(270.73169082,48.33353285)(270.74169081,48.28853289)(270.76168701,48.24853811)
\curveto(270.78169077,48.16853301)(270.80169075,48.0785331)(270.82168701,47.97853811)
\lineto(270.88168701,47.67853811)
\curveto(270.88169067,47.64853353)(270.88669067,47.61353357)(270.89668701,47.57353811)
\lineto(270.89668701,47.46853811)
\curveto(270.93669062,47.31853386)(270.96169059,47.15353403)(270.97168701,46.97353811)
\curveto(270.97169058,46.80353438)(270.99169056,46.64353454)(271.03168701,46.49353811)
\curveto(271.0516905,46.41353477)(271.07169048,46.33853484)(271.09168701,46.26853811)
\curveto(271.10169045,46.20853497)(271.11669044,46.13853504)(271.13668701,46.05853811)
\curveto(271.18669037,45.89853528)(271.2516903,45.74853543)(271.33168701,45.60853811)
\curveto(271.40169015,45.46853571)(271.49169006,45.34853583)(271.60168701,45.24853811)
\curveto(271.71168984,45.14853603)(271.84668971,45.07353611)(272.00668701,45.02353811)
\curveto(272.1566894,44.97353621)(272.34168921,44.95353623)(272.56168701,44.96353811)
\curveto(272.66168889,44.96353622)(272.7566888,44.9785362)(272.84668701,45.00853811)
\curveto(272.92668863,45.04853613)(273.00168855,45.09353609)(273.07168701,45.14353811)
\curveto(273.18168837,45.22353596)(273.27668828,45.32853585)(273.35668701,45.45853811)
\curveto(273.42668813,45.58853559)(273.48668807,45.72853545)(273.53668701,45.87853811)
\curveto(273.54668801,45.92853525)(273.551688,45.9785352)(273.55168701,46.02853811)
\curveto(273.551688,46.0785351)(273.556688,46.12853505)(273.56668701,46.17853811)
\curveto(273.58668797,46.24853493)(273.60168795,46.33353485)(273.61168701,46.43353811)
\curveto(273.61168794,46.54353464)(273.60168795,46.63353455)(273.58168701,46.70353811)
\curveto(273.56168799,46.76353442)(273.556688,46.82353436)(273.56668701,46.88353811)
\curveto(273.56668799,46.94353424)(273.556688,47.00353418)(273.53668701,47.06353811)
\curveto(273.51668804,47.14353404)(273.50168805,47.21853396)(273.49168701,47.28853811)
\curveto(273.48168807,47.36853381)(273.46168809,47.44353374)(273.43168701,47.51353811)
\curveto(273.31168824,47.80353338)(273.16668839,48.04853313)(272.99668701,48.24853811)
\curveto(272.82668873,48.45853272)(272.59668896,48.61853256)(272.30668701,48.72853811)
}
}
{
\newrgbcolor{curcolor}{0 0 0}
\pscustom[linestyle=none,fillstyle=solid,fillcolor=curcolor]
{
\newpath
\moveto(266.62168701,55.69017873)
\curveto(266.62169493,55.92017394)(266.68169487,56.05017381)(266.80168701,56.08017873)
\curveto(266.91169464,56.11017375)(267.07669448,56.12517374)(267.29668701,56.12517873)
\lineto(267.58168701,56.12517873)
\curveto(267.67169388,56.12517374)(267.74669381,56.10017376)(267.80668701,56.05017873)
\curveto(267.88669367,55.99017387)(267.93169362,55.90517396)(267.94168701,55.79517873)
\curveto(267.94169361,55.68517418)(267.9566936,55.57517429)(267.98668701,55.46517873)
\curveto(268.01669354,55.32517454)(268.04669351,55.19017467)(268.07668701,55.06017873)
\curveto(268.10669345,54.94017492)(268.14669341,54.82517504)(268.19668701,54.71517873)
\curveto(268.32669323,54.42517544)(268.50669305,54.19017567)(268.73668701,54.01017873)
\curveto(268.9566926,53.83017603)(269.21169234,53.67517619)(269.50168701,53.54517873)
\curveto(269.61169194,53.50517636)(269.72669183,53.47517639)(269.84668701,53.45517873)
\curveto(269.9566916,53.43517643)(270.07169148,53.41017645)(270.19168701,53.38017873)
\curveto(270.24169131,53.37017649)(270.29169126,53.3651765)(270.34168701,53.36517873)
\curveto(270.39169116,53.37517649)(270.44169111,53.37517649)(270.49168701,53.36517873)
\curveto(270.61169094,53.33517653)(270.7516908,53.32017654)(270.91168701,53.32017873)
\curveto(271.06169049,53.33017653)(271.20669035,53.33517653)(271.34668701,53.33517873)
\lineto(273.19168701,53.33517873)
\lineto(273.53668701,53.33517873)
\curveto(273.6566879,53.33517653)(273.77168778,53.33017653)(273.88168701,53.32017873)
\curveto(273.99168756,53.31017655)(274.08668747,53.30517656)(274.16668701,53.30517873)
\curveto(274.24668731,53.31517655)(274.31668724,53.29517657)(274.37668701,53.24517873)
\curveto(274.44668711,53.19517667)(274.48668707,53.11517675)(274.49668701,53.00517873)
\curveto(274.50668705,52.90517696)(274.51168704,52.79517707)(274.51168701,52.67517873)
\lineto(274.51168701,52.40517873)
\curveto(274.49168706,52.35517751)(274.47668708,52.30517756)(274.46668701,52.25517873)
\curveto(274.44668711,52.21517765)(274.42168713,52.18517768)(274.39168701,52.16517873)
\curveto(274.32168723,52.11517775)(274.23668732,52.08517778)(274.13668701,52.07517873)
\lineto(273.80668701,52.07517873)
\lineto(272.65168701,52.07517873)
\lineto(268.49668701,52.07517873)
\lineto(267.46168701,52.07517873)
\lineto(267.16168701,52.07517873)
\curveto(267.06169449,52.08517778)(266.97669458,52.11517775)(266.90668701,52.16517873)
\curveto(266.86669469,52.19517767)(266.83669472,52.24517762)(266.81668701,52.31517873)
\curveto(266.79669476,52.39517747)(266.78669477,52.48017738)(266.78668701,52.57017873)
\curveto(266.77669478,52.6601772)(266.77669478,52.75017711)(266.78668701,52.84017873)
\curveto(266.79669476,52.93017693)(266.81169474,53.00017686)(266.83168701,53.05017873)
\curveto(266.86169469,53.13017673)(266.92169463,53.18017668)(267.01168701,53.20017873)
\curveto(267.09169446,53.23017663)(267.18169437,53.24517662)(267.28168701,53.24517873)
\lineto(267.58168701,53.24517873)
\curveto(267.68169387,53.24517662)(267.77169378,53.2651766)(267.85168701,53.30517873)
\curveto(267.87169368,53.31517655)(267.88669367,53.32517654)(267.89668701,53.33517873)
\lineto(267.94168701,53.38017873)
\curveto(267.94169361,53.49017637)(267.89669366,53.58017628)(267.80668701,53.65017873)
\curveto(267.70669385,53.72017614)(267.62669393,53.78017608)(267.56668701,53.83017873)
\lineto(267.47668701,53.92017873)
\curveto(267.36669419,54.01017585)(267.2516943,54.13517573)(267.13168701,54.29517873)
\curveto(267.01169454,54.45517541)(266.92169463,54.60517526)(266.86168701,54.74517873)
\curveto(266.81169474,54.83517503)(266.77669478,54.93017493)(266.75668701,55.03017873)
\curveto(266.72669483,55.13017473)(266.69669486,55.23517463)(266.66668701,55.34517873)
\curveto(266.6566949,55.40517446)(266.6516949,55.4651744)(266.65168701,55.52517873)
\curveto(266.64169491,55.58517428)(266.63169492,55.64017422)(266.62168701,55.69017873)
}
}
{
\newrgbcolor{curcolor}{0 0 0}
\pscustom[linestyle=none,fillstyle=solid,fillcolor=curcolor]
{
}
}
{
\newrgbcolor{curcolor}{0 0 0}
\pscustom[linestyle=none,fillstyle=solid,fillcolor=curcolor]
{
\newpath
\moveto(263.92168701,64.24510061)
\curveto(263.91169764,64.93509597)(264.03169752,65.53509537)(264.28168701,66.04510061)
\curveto(264.53169702,66.56509434)(264.86669669,66.96009395)(265.28668701,67.23010061)
\curveto(265.36669619,67.28009363)(265.4566961,67.32509358)(265.55668701,67.36510061)
\curveto(265.64669591,67.4050935)(265.74169581,67.45009346)(265.84168701,67.50010061)
\curveto(265.94169561,67.54009337)(266.04169551,67.57009334)(266.14168701,67.59010061)
\curveto(266.24169531,67.6100933)(266.34669521,67.63009328)(266.45668701,67.65010061)
\curveto(266.50669505,67.67009324)(266.551695,67.67509323)(266.59168701,67.66510061)
\curveto(266.63169492,67.65509325)(266.67669488,67.66009325)(266.72668701,67.68010061)
\curveto(266.77669478,67.69009322)(266.86169469,67.69509321)(266.98168701,67.69510061)
\curveto(267.09169446,67.69509321)(267.17669438,67.69009322)(267.23668701,67.68010061)
\curveto(267.29669426,67.66009325)(267.3566942,67.65009326)(267.41668701,67.65010061)
\curveto(267.47669408,67.66009325)(267.53669402,67.65509325)(267.59668701,67.63510061)
\curveto(267.73669382,67.59509331)(267.87169368,67.56009335)(268.00168701,67.53010061)
\curveto(268.13169342,67.50009341)(268.2566933,67.46009345)(268.37668701,67.41010061)
\curveto(268.51669304,67.35009356)(268.64169291,67.28009363)(268.75168701,67.20010061)
\curveto(268.86169269,67.13009378)(268.97169258,67.05509385)(269.08168701,66.97510061)
\lineto(269.14168701,66.91510061)
\curveto(269.16169239,66.905094)(269.18169237,66.89009402)(269.20168701,66.87010061)
\curveto(269.36169219,66.75009416)(269.50669205,66.61509429)(269.63668701,66.46510061)
\curveto(269.76669179,66.31509459)(269.89169166,66.15509475)(270.01168701,65.98510061)
\curveto(270.23169132,65.67509523)(270.43669112,65.38009553)(270.62668701,65.10010061)
\curveto(270.76669079,64.87009604)(270.90169065,64.64009627)(271.03168701,64.41010061)
\curveto(271.16169039,64.19009672)(271.29669026,63.97009694)(271.43668701,63.75010061)
\curveto(271.60668995,63.50009741)(271.78668977,63.26009765)(271.97668701,63.03010061)
\curveto(272.16668939,62.8100981)(272.39168916,62.62009829)(272.65168701,62.46010061)
\curveto(272.71168884,62.42009849)(272.77168878,62.38509852)(272.83168701,62.35510061)
\curveto(272.88168867,62.32509858)(272.94668861,62.29509861)(273.02668701,62.26510061)
\curveto(273.09668846,62.24509866)(273.1566884,62.24009867)(273.20668701,62.25010061)
\curveto(273.27668828,62.27009864)(273.33168822,62.3050986)(273.37168701,62.35510061)
\curveto(273.40168815,62.4050985)(273.42168813,62.46509844)(273.43168701,62.53510061)
\lineto(273.43168701,62.77510061)
\lineto(273.43168701,63.52510061)
\lineto(273.43168701,66.33010061)
\lineto(273.43168701,66.99010061)
\curveto(273.43168812,67.08009383)(273.43668812,67.16509374)(273.44668701,67.24510061)
\curveto(273.44668811,67.32509358)(273.46668809,67.39009352)(273.50668701,67.44010061)
\curveto(273.54668801,67.49009342)(273.62168793,67.53009338)(273.73168701,67.56010061)
\curveto(273.83168772,67.60009331)(273.93168762,67.6100933)(274.03168701,67.59010061)
\lineto(274.16668701,67.59010061)
\curveto(274.23668732,67.57009334)(274.29668726,67.55009336)(274.34668701,67.53010061)
\curveto(274.39668716,67.5100934)(274.43668712,67.47509343)(274.46668701,67.42510061)
\curveto(274.50668705,67.37509353)(274.52668703,67.3050936)(274.52668701,67.21510061)
\lineto(274.52668701,66.94510061)
\lineto(274.52668701,66.04510061)
\lineto(274.52668701,62.53510061)
\lineto(274.52668701,61.47010061)
\curveto(274.52668703,61.39009952)(274.53168702,61.30009961)(274.54168701,61.20010061)
\curveto(274.54168701,61.10009981)(274.53168702,61.01509989)(274.51168701,60.94510061)
\curveto(274.44168711,60.73510017)(274.26168729,60.67010024)(273.97168701,60.75010061)
\curveto(273.93168762,60.76010015)(273.89668766,60.76010015)(273.86668701,60.75010061)
\curveto(273.82668773,60.75010016)(273.78168777,60.76010015)(273.73168701,60.78010061)
\curveto(273.6516879,60.80010011)(273.56668799,60.82010009)(273.47668701,60.84010061)
\curveto(273.38668817,60.86010005)(273.30168825,60.88510002)(273.22168701,60.91510061)
\curveto(272.73168882,61.07509983)(272.31668924,61.27509963)(271.97668701,61.51510061)
\curveto(271.72668983,61.69509921)(271.50169005,61.90009901)(271.30168701,62.13010061)
\curveto(271.09169046,62.36009855)(270.89669066,62.60009831)(270.71668701,62.85010061)
\curveto(270.53669102,63.1100978)(270.36669119,63.37509753)(270.20668701,63.64510061)
\curveto(270.03669152,63.92509698)(269.86169169,64.19509671)(269.68168701,64.45510061)
\curveto(269.60169195,64.56509634)(269.52669203,64.67009624)(269.45668701,64.77010061)
\curveto(269.38669217,64.88009603)(269.31169224,64.99009592)(269.23168701,65.10010061)
\curveto(269.20169235,65.14009577)(269.17169238,65.17509573)(269.14168701,65.20510061)
\curveto(269.10169245,65.24509566)(269.07169248,65.28509562)(269.05168701,65.32510061)
\curveto(268.94169261,65.46509544)(268.81669274,65.59009532)(268.67668701,65.70010061)
\curveto(268.64669291,65.72009519)(268.62169293,65.74509516)(268.60168701,65.77510061)
\curveto(268.57169298,65.8050951)(268.54169301,65.83009508)(268.51168701,65.85010061)
\curveto(268.41169314,65.93009498)(268.31169324,65.99509491)(268.21168701,66.04510061)
\curveto(268.11169344,66.1050948)(268.00169355,66.16009475)(267.88168701,66.21010061)
\curveto(267.81169374,66.24009467)(267.73669382,66.26009465)(267.65668701,66.27010061)
\lineto(267.41668701,66.33010061)
\lineto(267.32668701,66.33010061)
\curveto(267.29669426,66.34009457)(267.26669429,66.34509456)(267.23668701,66.34510061)
\curveto(267.16669439,66.36509454)(267.07169448,66.37009454)(266.95168701,66.36010061)
\curveto(266.82169473,66.36009455)(266.72169483,66.35009456)(266.65168701,66.33010061)
\curveto(266.57169498,66.3100946)(266.49669506,66.29009462)(266.42668701,66.27010061)
\curveto(266.34669521,66.26009465)(266.26669529,66.24009467)(266.18668701,66.21010061)
\curveto(265.94669561,66.10009481)(265.74669581,65.95009496)(265.58668701,65.76010061)
\curveto(265.41669614,65.58009533)(265.27669628,65.36009555)(265.16668701,65.10010061)
\curveto(265.14669641,65.03009588)(265.13169642,64.96009595)(265.12168701,64.89010061)
\curveto(265.10169645,64.82009609)(265.08169647,64.74509616)(265.06168701,64.66510061)
\curveto(265.04169651,64.58509632)(265.03169652,64.47509643)(265.03168701,64.33510061)
\curveto(265.03169652,64.2050967)(265.04169651,64.10009681)(265.06168701,64.02010061)
\curveto(265.07169648,63.96009695)(265.07669648,63.905097)(265.07668701,63.85510061)
\curveto(265.07669648,63.8050971)(265.08669647,63.75509715)(265.10668701,63.70510061)
\curveto(265.14669641,63.6050973)(265.18669637,63.5100974)(265.22668701,63.42010061)
\curveto(265.26669629,63.34009757)(265.31169624,63.26009765)(265.36168701,63.18010061)
\curveto(265.38169617,63.15009776)(265.40669615,63.12009779)(265.43668701,63.09010061)
\curveto(265.46669609,63.07009784)(265.49169606,63.04509786)(265.51168701,63.01510061)
\lineto(265.58668701,62.94010061)
\curveto(265.60669595,62.910098)(265.62669593,62.88509802)(265.64668701,62.86510061)
\lineto(265.85668701,62.71510061)
\curveto(265.91669564,62.67509823)(265.98169557,62.63009828)(266.05168701,62.58010061)
\curveto(266.14169541,62.52009839)(266.24669531,62.47009844)(266.36668701,62.43010061)
\curveto(266.47669508,62.40009851)(266.58669497,62.36509854)(266.69668701,62.32510061)
\curveto(266.80669475,62.28509862)(266.9516946,62.26009865)(267.13168701,62.25010061)
\curveto(267.30169425,62.24009867)(267.42669413,62.2100987)(267.50668701,62.16010061)
\curveto(267.58669397,62.1100988)(267.63169392,62.03509887)(267.64168701,61.93510061)
\curveto(267.6516939,61.83509907)(267.6566939,61.72509918)(267.65668701,61.60510061)
\curveto(267.6566939,61.56509934)(267.66169389,61.52509938)(267.67168701,61.48510061)
\curveto(267.67169388,61.44509946)(267.66669389,61.4100995)(267.65668701,61.38010061)
\curveto(267.63669392,61.33009958)(267.62669393,61.28009963)(267.62668701,61.23010061)
\curveto(267.62669393,61.19009972)(267.61669394,61.15009976)(267.59668701,61.11010061)
\curveto(267.53669402,61.02009989)(267.40169415,60.97509993)(267.19168701,60.97510061)
\lineto(267.07168701,60.97510061)
\curveto(267.01169454,60.98509992)(266.9516946,60.99009992)(266.89168701,60.99010061)
\curveto(266.82169473,61.00009991)(266.7566948,61.0100999)(266.69668701,61.02010061)
\curveto(266.58669497,61.04009987)(266.48669507,61.06009985)(266.39668701,61.08010061)
\curveto(266.29669526,61.10009981)(266.20169535,61.13009978)(266.11168701,61.17010061)
\curveto(266.04169551,61.19009972)(265.98169557,61.2100997)(265.93168701,61.23010061)
\lineto(265.75168701,61.29010061)
\curveto(265.49169606,61.4100995)(265.24669631,61.56509934)(265.01668701,61.75510061)
\curveto(264.78669677,61.95509895)(264.60169695,62.17009874)(264.46168701,62.40010061)
\curveto(264.38169717,62.5100984)(264.31669724,62.62509828)(264.26668701,62.74510061)
\lineto(264.11668701,63.13510061)
\curveto(264.06669749,63.24509766)(264.03669752,63.36009755)(264.02668701,63.48010061)
\curveto(264.00669755,63.60009731)(263.98169757,63.72509718)(263.95168701,63.85510061)
\curveto(263.9516976,63.92509698)(263.9516976,63.99009692)(263.95168701,64.05010061)
\curveto(263.94169761,64.1100968)(263.93169762,64.17509673)(263.92168701,64.24510061)
}
}
{
\newrgbcolor{curcolor}{0 0 0}
\pscustom[linestyle=none,fillstyle=solid,fillcolor=curcolor]
{
\newpath
\moveto(263.92168701,72.59470998)
\curveto(263.91169764,73.28470535)(264.03169752,73.88470475)(264.28168701,74.39470998)
\curveto(264.53169702,74.91470372)(264.86669669,75.30970332)(265.28668701,75.57970998)
\curveto(265.36669619,75.629703)(265.4566961,75.67470296)(265.55668701,75.71470998)
\curveto(265.64669591,75.75470288)(265.74169581,75.79970283)(265.84168701,75.84970998)
\curveto(265.94169561,75.88970274)(266.04169551,75.91970271)(266.14168701,75.93970998)
\curveto(266.24169531,75.95970267)(266.34669521,75.97970265)(266.45668701,75.99970998)
\curveto(266.50669505,76.01970261)(266.551695,76.02470261)(266.59168701,76.01470998)
\curveto(266.63169492,76.00470263)(266.67669488,76.00970262)(266.72668701,76.02970998)
\curveto(266.77669478,76.03970259)(266.86169469,76.04470259)(266.98168701,76.04470998)
\curveto(267.09169446,76.04470259)(267.17669438,76.03970259)(267.23668701,76.02970998)
\curveto(267.29669426,76.00970262)(267.3566942,75.99970263)(267.41668701,75.99970998)
\curveto(267.47669408,76.00970262)(267.53669402,76.00470263)(267.59668701,75.98470998)
\curveto(267.73669382,75.94470269)(267.87169368,75.90970272)(268.00168701,75.87970998)
\curveto(268.13169342,75.84970278)(268.2566933,75.80970282)(268.37668701,75.75970998)
\curveto(268.51669304,75.69970293)(268.64169291,75.629703)(268.75168701,75.54970998)
\curveto(268.86169269,75.47970315)(268.97169258,75.40470323)(269.08168701,75.32470998)
\lineto(269.14168701,75.26470998)
\curveto(269.16169239,75.25470338)(269.18169237,75.23970339)(269.20168701,75.21970998)
\curveto(269.36169219,75.09970353)(269.50669205,74.96470367)(269.63668701,74.81470998)
\curveto(269.76669179,74.66470397)(269.89169166,74.50470413)(270.01168701,74.33470998)
\curveto(270.23169132,74.02470461)(270.43669112,73.7297049)(270.62668701,73.44970998)
\curveto(270.76669079,73.21970541)(270.90169065,72.98970564)(271.03168701,72.75970998)
\curveto(271.16169039,72.53970609)(271.29669026,72.31970631)(271.43668701,72.09970998)
\curveto(271.60668995,71.84970678)(271.78668977,71.60970702)(271.97668701,71.37970998)
\curveto(272.16668939,71.15970747)(272.39168916,70.96970766)(272.65168701,70.80970998)
\curveto(272.71168884,70.76970786)(272.77168878,70.7347079)(272.83168701,70.70470998)
\curveto(272.88168867,70.67470796)(272.94668861,70.64470799)(273.02668701,70.61470998)
\curveto(273.09668846,70.59470804)(273.1566884,70.58970804)(273.20668701,70.59970998)
\curveto(273.27668828,70.61970801)(273.33168822,70.65470798)(273.37168701,70.70470998)
\curveto(273.40168815,70.75470788)(273.42168813,70.81470782)(273.43168701,70.88470998)
\lineto(273.43168701,71.12470998)
\lineto(273.43168701,71.87470998)
\lineto(273.43168701,74.67970998)
\lineto(273.43168701,75.33970998)
\curveto(273.43168812,75.4297032)(273.43668812,75.51470312)(273.44668701,75.59470998)
\curveto(273.44668811,75.67470296)(273.46668809,75.73970289)(273.50668701,75.78970998)
\curveto(273.54668801,75.83970279)(273.62168793,75.87970275)(273.73168701,75.90970998)
\curveto(273.83168772,75.94970268)(273.93168762,75.95970267)(274.03168701,75.93970998)
\lineto(274.16668701,75.93970998)
\curveto(274.23668732,75.91970271)(274.29668726,75.89970273)(274.34668701,75.87970998)
\curveto(274.39668716,75.85970277)(274.43668712,75.82470281)(274.46668701,75.77470998)
\curveto(274.50668705,75.72470291)(274.52668703,75.65470298)(274.52668701,75.56470998)
\lineto(274.52668701,75.29470998)
\lineto(274.52668701,74.39470998)
\lineto(274.52668701,70.88470998)
\lineto(274.52668701,69.81970998)
\curveto(274.52668703,69.73970889)(274.53168702,69.64970898)(274.54168701,69.54970998)
\curveto(274.54168701,69.44970918)(274.53168702,69.36470927)(274.51168701,69.29470998)
\curveto(274.44168711,69.08470955)(274.26168729,69.01970961)(273.97168701,69.09970998)
\curveto(273.93168762,69.10970952)(273.89668766,69.10970952)(273.86668701,69.09970998)
\curveto(273.82668773,69.09970953)(273.78168777,69.10970952)(273.73168701,69.12970998)
\curveto(273.6516879,69.14970948)(273.56668799,69.16970946)(273.47668701,69.18970998)
\curveto(273.38668817,69.20970942)(273.30168825,69.2347094)(273.22168701,69.26470998)
\curveto(272.73168882,69.42470921)(272.31668924,69.62470901)(271.97668701,69.86470998)
\curveto(271.72668983,70.04470859)(271.50169005,70.24970838)(271.30168701,70.47970998)
\curveto(271.09169046,70.70970792)(270.89669066,70.94970768)(270.71668701,71.19970998)
\curveto(270.53669102,71.45970717)(270.36669119,71.72470691)(270.20668701,71.99470998)
\curveto(270.03669152,72.27470636)(269.86169169,72.54470609)(269.68168701,72.80470998)
\curveto(269.60169195,72.91470572)(269.52669203,73.01970561)(269.45668701,73.11970998)
\curveto(269.38669217,73.2297054)(269.31169224,73.33970529)(269.23168701,73.44970998)
\curveto(269.20169235,73.48970514)(269.17169238,73.52470511)(269.14168701,73.55470998)
\curveto(269.10169245,73.59470504)(269.07169248,73.634705)(269.05168701,73.67470998)
\curveto(268.94169261,73.81470482)(268.81669274,73.93970469)(268.67668701,74.04970998)
\curveto(268.64669291,74.06970456)(268.62169293,74.09470454)(268.60168701,74.12470998)
\curveto(268.57169298,74.15470448)(268.54169301,74.17970445)(268.51168701,74.19970998)
\curveto(268.41169314,74.27970435)(268.31169324,74.34470429)(268.21168701,74.39470998)
\curveto(268.11169344,74.45470418)(268.00169355,74.50970412)(267.88168701,74.55970998)
\curveto(267.81169374,74.58970404)(267.73669382,74.60970402)(267.65668701,74.61970998)
\lineto(267.41668701,74.67970998)
\lineto(267.32668701,74.67970998)
\curveto(267.29669426,74.68970394)(267.26669429,74.69470394)(267.23668701,74.69470998)
\curveto(267.16669439,74.71470392)(267.07169448,74.71970391)(266.95168701,74.70970998)
\curveto(266.82169473,74.70970392)(266.72169483,74.69970393)(266.65168701,74.67970998)
\curveto(266.57169498,74.65970397)(266.49669506,74.63970399)(266.42668701,74.61970998)
\curveto(266.34669521,74.60970402)(266.26669529,74.58970404)(266.18668701,74.55970998)
\curveto(265.94669561,74.44970418)(265.74669581,74.29970433)(265.58668701,74.10970998)
\curveto(265.41669614,73.9297047)(265.27669628,73.70970492)(265.16668701,73.44970998)
\curveto(265.14669641,73.37970525)(265.13169642,73.30970532)(265.12168701,73.23970998)
\curveto(265.10169645,73.16970546)(265.08169647,73.09470554)(265.06168701,73.01470998)
\curveto(265.04169651,72.9347057)(265.03169652,72.82470581)(265.03168701,72.68470998)
\curveto(265.03169652,72.55470608)(265.04169651,72.44970618)(265.06168701,72.36970998)
\curveto(265.07169648,72.30970632)(265.07669648,72.25470638)(265.07668701,72.20470998)
\curveto(265.07669648,72.15470648)(265.08669647,72.10470653)(265.10668701,72.05470998)
\curveto(265.14669641,71.95470668)(265.18669637,71.85970677)(265.22668701,71.76970998)
\curveto(265.26669629,71.68970694)(265.31169624,71.60970702)(265.36168701,71.52970998)
\curveto(265.38169617,71.49970713)(265.40669615,71.46970716)(265.43668701,71.43970998)
\curveto(265.46669609,71.41970721)(265.49169606,71.39470724)(265.51168701,71.36470998)
\lineto(265.58668701,71.28970998)
\curveto(265.60669595,71.25970737)(265.62669593,71.2347074)(265.64668701,71.21470998)
\lineto(265.85668701,71.06470998)
\curveto(265.91669564,71.02470761)(265.98169557,70.97970765)(266.05168701,70.92970998)
\curveto(266.14169541,70.86970776)(266.24669531,70.81970781)(266.36668701,70.77970998)
\curveto(266.47669508,70.74970788)(266.58669497,70.71470792)(266.69668701,70.67470998)
\curveto(266.80669475,70.634708)(266.9516946,70.60970802)(267.13168701,70.59970998)
\curveto(267.30169425,70.58970804)(267.42669413,70.55970807)(267.50668701,70.50970998)
\curveto(267.58669397,70.45970817)(267.63169392,70.38470825)(267.64168701,70.28470998)
\curveto(267.6516939,70.18470845)(267.6566939,70.07470856)(267.65668701,69.95470998)
\curveto(267.6566939,69.91470872)(267.66169389,69.87470876)(267.67168701,69.83470998)
\curveto(267.67169388,69.79470884)(267.66669389,69.75970887)(267.65668701,69.72970998)
\curveto(267.63669392,69.67970895)(267.62669393,69.629709)(267.62668701,69.57970998)
\curveto(267.62669393,69.53970909)(267.61669394,69.49970913)(267.59668701,69.45970998)
\curveto(267.53669402,69.36970926)(267.40169415,69.32470931)(267.19168701,69.32470998)
\lineto(267.07168701,69.32470998)
\curveto(267.01169454,69.3347093)(266.9516946,69.33970929)(266.89168701,69.33970998)
\curveto(266.82169473,69.34970928)(266.7566948,69.35970927)(266.69668701,69.36970998)
\curveto(266.58669497,69.38970924)(266.48669507,69.40970922)(266.39668701,69.42970998)
\curveto(266.29669526,69.44970918)(266.20169535,69.47970915)(266.11168701,69.51970998)
\curveto(266.04169551,69.53970909)(265.98169557,69.55970907)(265.93168701,69.57970998)
\lineto(265.75168701,69.63970998)
\curveto(265.49169606,69.75970887)(265.24669631,69.91470872)(265.01668701,70.10470998)
\curveto(264.78669677,70.30470833)(264.60169695,70.51970811)(264.46168701,70.74970998)
\curveto(264.38169717,70.85970777)(264.31669724,70.97470766)(264.26668701,71.09470998)
\lineto(264.11668701,71.48470998)
\curveto(264.06669749,71.59470704)(264.03669752,71.70970692)(264.02668701,71.82970998)
\curveto(264.00669755,71.94970668)(263.98169757,72.07470656)(263.95168701,72.20470998)
\curveto(263.9516976,72.27470636)(263.9516976,72.33970629)(263.95168701,72.39970998)
\curveto(263.94169761,72.45970617)(263.93169762,72.52470611)(263.92168701,72.59470998)
}
}
{
\newrgbcolor{curcolor}{0 0 0}
\pscustom[linestyle=none,fillstyle=solid,fillcolor=curcolor]
{
\newpath
\moveto(272.89168701,78.63431936)
\lineto(272.89168701,79.26431936)
\lineto(272.89168701,79.45931936)
\curveto(272.89168866,79.52931683)(272.90168865,79.58931677)(272.92168701,79.63931936)
\curveto(272.96168859,79.70931665)(273.00168855,79.7593166)(273.04168701,79.78931936)
\curveto(273.09168846,79.82931653)(273.1566884,79.84931651)(273.23668701,79.84931936)
\curveto(273.31668824,79.8593165)(273.40168815,79.86431649)(273.49168701,79.86431936)
\lineto(274.21168701,79.86431936)
\curveto(274.69168686,79.86431649)(275.10168645,79.80431655)(275.44168701,79.68431936)
\curveto(275.78168577,79.56431679)(276.0566855,79.36931699)(276.26668701,79.09931936)
\curveto(276.31668524,79.02931733)(276.36168519,78.9593174)(276.40168701,78.88931936)
\curveto(276.4516851,78.82931753)(276.49668506,78.7543176)(276.53668701,78.66431936)
\curveto(276.54668501,78.64431771)(276.556685,78.61431774)(276.56668701,78.57431936)
\curveto(276.58668497,78.53431782)(276.59168496,78.48931787)(276.58168701,78.43931936)
\curveto(276.551685,78.34931801)(276.47668508,78.29431806)(276.35668701,78.27431936)
\curveto(276.24668531,78.2543181)(276.1516854,78.26931809)(276.07168701,78.31931936)
\curveto(276.00168555,78.34931801)(275.93668562,78.39431796)(275.87668701,78.45431936)
\curveto(275.82668573,78.52431783)(275.77668578,78.58931777)(275.72668701,78.64931936)
\curveto(275.67668588,78.71931764)(275.60168595,78.77931758)(275.50168701,78.82931936)
\curveto(275.41168614,78.88931747)(275.32168623,78.93931742)(275.23168701,78.97931936)
\curveto(275.20168635,78.99931736)(275.14168641,79.02431733)(275.05168701,79.05431936)
\curveto(274.97168658,79.08431727)(274.90168665,79.08931727)(274.84168701,79.06931936)
\curveto(274.70168685,79.03931732)(274.61168694,78.97931738)(274.57168701,78.88931936)
\curveto(274.54168701,78.80931755)(274.52668703,78.71931764)(274.52668701,78.61931936)
\curveto(274.52668703,78.51931784)(274.50168705,78.43431792)(274.45168701,78.36431936)
\curveto(274.38168717,78.27431808)(274.24168731,78.22931813)(274.03168701,78.22931936)
\lineto(273.47668701,78.22931936)
\lineto(273.25168701,78.22931936)
\curveto(273.17168838,78.23931812)(273.10668845,78.2593181)(273.05668701,78.28931936)
\curveto(272.97668858,78.34931801)(272.93168862,78.41931794)(272.92168701,78.49931936)
\curveto(272.91168864,78.51931784)(272.90668865,78.53931782)(272.90668701,78.55931936)
\curveto(272.90668865,78.58931777)(272.90168865,78.61431774)(272.89168701,78.63431936)
}
}
{
\newrgbcolor{curcolor}{0 0 0}
\pscustom[linestyle=none,fillstyle=solid,fillcolor=curcolor]
{
}
}
{
\newrgbcolor{curcolor}{0 0 0}
\pscustom[linestyle=none,fillstyle=solid,fillcolor=curcolor]
{
\newpath
\moveto(263.92168701,89.26463186)
\curveto(263.91169764,89.95462722)(264.03169752,90.55462662)(264.28168701,91.06463186)
\curveto(264.53169702,91.58462559)(264.86669669,91.9796252)(265.28668701,92.24963186)
\curveto(265.36669619,92.29962488)(265.4566961,92.34462483)(265.55668701,92.38463186)
\curveto(265.64669591,92.42462475)(265.74169581,92.46962471)(265.84168701,92.51963186)
\curveto(265.94169561,92.55962462)(266.04169551,92.58962459)(266.14168701,92.60963186)
\curveto(266.24169531,92.62962455)(266.34669521,92.64962453)(266.45668701,92.66963186)
\curveto(266.50669505,92.68962449)(266.551695,92.69462448)(266.59168701,92.68463186)
\curveto(266.63169492,92.6746245)(266.67669488,92.6796245)(266.72668701,92.69963186)
\curveto(266.77669478,92.70962447)(266.86169469,92.71462446)(266.98168701,92.71463186)
\curveto(267.09169446,92.71462446)(267.17669438,92.70962447)(267.23668701,92.69963186)
\curveto(267.29669426,92.6796245)(267.3566942,92.66962451)(267.41668701,92.66963186)
\curveto(267.47669408,92.6796245)(267.53669402,92.6746245)(267.59668701,92.65463186)
\curveto(267.73669382,92.61462456)(267.87169368,92.5796246)(268.00168701,92.54963186)
\curveto(268.13169342,92.51962466)(268.2566933,92.4796247)(268.37668701,92.42963186)
\curveto(268.51669304,92.36962481)(268.64169291,92.29962488)(268.75168701,92.21963186)
\curveto(268.86169269,92.14962503)(268.97169258,92.0746251)(269.08168701,91.99463186)
\lineto(269.14168701,91.93463186)
\curveto(269.16169239,91.92462525)(269.18169237,91.90962527)(269.20168701,91.88963186)
\curveto(269.36169219,91.76962541)(269.50669205,91.63462554)(269.63668701,91.48463186)
\curveto(269.76669179,91.33462584)(269.89169166,91.174626)(270.01168701,91.00463186)
\curveto(270.23169132,90.69462648)(270.43669112,90.39962678)(270.62668701,90.11963186)
\curveto(270.76669079,89.88962729)(270.90169065,89.65962752)(271.03168701,89.42963186)
\curveto(271.16169039,89.20962797)(271.29669026,88.98962819)(271.43668701,88.76963186)
\curveto(271.60668995,88.51962866)(271.78668977,88.2796289)(271.97668701,88.04963186)
\curveto(272.16668939,87.82962935)(272.39168916,87.63962954)(272.65168701,87.47963186)
\curveto(272.71168884,87.43962974)(272.77168878,87.40462977)(272.83168701,87.37463186)
\curveto(272.88168867,87.34462983)(272.94668861,87.31462986)(273.02668701,87.28463186)
\curveto(273.09668846,87.26462991)(273.1566884,87.25962992)(273.20668701,87.26963186)
\curveto(273.27668828,87.28962989)(273.33168822,87.32462985)(273.37168701,87.37463186)
\curveto(273.40168815,87.42462975)(273.42168813,87.48462969)(273.43168701,87.55463186)
\lineto(273.43168701,87.79463186)
\lineto(273.43168701,88.54463186)
\lineto(273.43168701,91.34963186)
\lineto(273.43168701,92.00963186)
\curveto(273.43168812,92.09962508)(273.43668812,92.18462499)(273.44668701,92.26463186)
\curveto(273.44668811,92.34462483)(273.46668809,92.40962477)(273.50668701,92.45963186)
\curveto(273.54668801,92.50962467)(273.62168793,92.54962463)(273.73168701,92.57963186)
\curveto(273.83168772,92.61962456)(273.93168762,92.62962455)(274.03168701,92.60963186)
\lineto(274.16668701,92.60963186)
\curveto(274.23668732,92.58962459)(274.29668726,92.56962461)(274.34668701,92.54963186)
\curveto(274.39668716,92.52962465)(274.43668712,92.49462468)(274.46668701,92.44463186)
\curveto(274.50668705,92.39462478)(274.52668703,92.32462485)(274.52668701,92.23463186)
\lineto(274.52668701,91.96463186)
\lineto(274.52668701,91.06463186)
\lineto(274.52668701,87.55463186)
\lineto(274.52668701,86.48963186)
\curveto(274.52668703,86.40963077)(274.53168702,86.31963086)(274.54168701,86.21963186)
\curveto(274.54168701,86.11963106)(274.53168702,86.03463114)(274.51168701,85.96463186)
\curveto(274.44168711,85.75463142)(274.26168729,85.68963149)(273.97168701,85.76963186)
\curveto(273.93168762,85.7796314)(273.89668766,85.7796314)(273.86668701,85.76963186)
\curveto(273.82668773,85.76963141)(273.78168777,85.7796314)(273.73168701,85.79963186)
\curveto(273.6516879,85.81963136)(273.56668799,85.83963134)(273.47668701,85.85963186)
\curveto(273.38668817,85.8796313)(273.30168825,85.90463127)(273.22168701,85.93463186)
\curveto(272.73168882,86.09463108)(272.31668924,86.29463088)(271.97668701,86.53463186)
\curveto(271.72668983,86.71463046)(271.50169005,86.91963026)(271.30168701,87.14963186)
\curveto(271.09169046,87.3796298)(270.89669066,87.61962956)(270.71668701,87.86963186)
\curveto(270.53669102,88.12962905)(270.36669119,88.39462878)(270.20668701,88.66463186)
\curveto(270.03669152,88.94462823)(269.86169169,89.21462796)(269.68168701,89.47463186)
\curveto(269.60169195,89.58462759)(269.52669203,89.68962749)(269.45668701,89.78963186)
\curveto(269.38669217,89.89962728)(269.31169224,90.00962717)(269.23168701,90.11963186)
\curveto(269.20169235,90.15962702)(269.17169238,90.19462698)(269.14168701,90.22463186)
\curveto(269.10169245,90.26462691)(269.07169248,90.30462687)(269.05168701,90.34463186)
\curveto(268.94169261,90.48462669)(268.81669274,90.60962657)(268.67668701,90.71963186)
\curveto(268.64669291,90.73962644)(268.62169293,90.76462641)(268.60168701,90.79463186)
\curveto(268.57169298,90.82462635)(268.54169301,90.84962633)(268.51168701,90.86963186)
\curveto(268.41169314,90.94962623)(268.31169324,91.01462616)(268.21168701,91.06463186)
\curveto(268.11169344,91.12462605)(268.00169355,91.179626)(267.88168701,91.22963186)
\curveto(267.81169374,91.25962592)(267.73669382,91.2796259)(267.65668701,91.28963186)
\lineto(267.41668701,91.34963186)
\lineto(267.32668701,91.34963186)
\curveto(267.29669426,91.35962582)(267.26669429,91.36462581)(267.23668701,91.36463186)
\curveto(267.16669439,91.38462579)(267.07169448,91.38962579)(266.95168701,91.37963186)
\curveto(266.82169473,91.3796258)(266.72169483,91.36962581)(266.65168701,91.34963186)
\curveto(266.57169498,91.32962585)(266.49669506,91.30962587)(266.42668701,91.28963186)
\curveto(266.34669521,91.2796259)(266.26669529,91.25962592)(266.18668701,91.22963186)
\curveto(265.94669561,91.11962606)(265.74669581,90.96962621)(265.58668701,90.77963186)
\curveto(265.41669614,90.59962658)(265.27669628,90.3796268)(265.16668701,90.11963186)
\curveto(265.14669641,90.04962713)(265.13169642,89.9796272)(265.12168701,89.90963186)
\curveto(265.10169645,89.83962734)(265.08169647,89.76462741)(265.06168701,89.68463186)
\curveto(265.04169651,89.60462757)(265.03169652,89.49462768)(265.03168701,89.35463186)
\curveto(265.03169652,89.22462795)(265.04169651,89.11962806)(265.06168701,89.03963186)
\curveto(265.07169648,88.9796282)(265.07669648,88.92462825)(265.07668701,88.87463186)
\curveto(265.07669648,88.82462835)(265.08669647,88.7746284)(265.10668701,88.72463186)
\curveto(265.14669641,88.62462855)(265.18669637,88.52962865)(265.22668701,88.43963186)
\curveto(265.26669629,88.35962882)(265.31169624,88.2796289)(265.36168701,88.19963186)
\curveto(265.38169617,88.16962901)(265.40669615,88.13962904)(265.43668701,88.10963186)
\curveto(265.46669609,88.08962909)(265.49169606,88.06462911)(265.51168701,88.03463186)
\lineto(265.58668701,87.95963186)
\curveto(265.60669595,87.92962925)(265.62669593,87.90462927)(265.64668701,87.88463186)
\lineto(265.85668701,87.73463186)
\curveto(265.91669564,87.69462948)(265.98169557,87.64962953)(266.05168701,87.59963186)
\curveto(266.14169541,87.53962964)(266.24669531,87.48962969)(266.36668701,87.44963186)
\curveto(266.47669508,87.41962976)(266.58669497,87.38462979)(266.69668701,87.34463186)
\curveto(266.80669475,87.30462987)(266.9516946,87.2796299)(267.13168701,87.26963186)
\curveto(267.30169425,87.25962992)(267.42669413,87.22962995)(267.50668701,87.17963186)
\curveto(267.58669397,87.12963005)(267.63169392,87.05463012)(267.64168701,86.95463186)
\curveto(267.6516939,86.85463032)(267.6566939,86.74463043)(267.65668701,86.62463186)
\curveto(267.6566939,86.58463059)(267.66169389,86.54463063)(267.67168701,86.50463186)
\curveto(267.67169388,86.46463071)(267.66669389,86.42963075)(267.65668701,86.39963186)
\curveto(267.63669392,86.34963083)(267.62669393,86.29963088)(267.62668701,86.24963186)
\curveto(267.62669393,86.20963097)(267.61669394,86.16963101)(267.59668701,86.12963186)
\curveto(267.53669402,86.03963114)(267.40169415,85.99463118)(267.19168701,85.99463186)
\lineto(267.07168701,85.99463186)
\curveto(267.01169454,86.00463117)(266.9516946,86.00963117)(266.89168701,86.00963186)
\curveto(266.82169473,86.01963116)(266.7566948,86.02963115)(266.69668701,86.03963186)
\curveto(266.58669497,86.05963112)(266.48669507,86.0796311)(266.39668701,86.09963186)
\curveto(266.29669526,86.11963106)(266.20169535,86.14963103)(266.11168701,86.18963186)
\curveto(266.04169551,86.20963097)(265.98169557,86.22963095)(265.93168701,86.24963186)
\lineto(265.75168701,86.30963186)
\curveto(265.49169606,86.42963075)(265.24669631,86.58463059)(265.01668701,86.77463186)
\curveto(264.78669677,86.9746302)(264.60169695,87.18962999)(264.46168701,87.41963186)
\curveto(264.38169717,87.52962965)(264.31669724,87.64462953)(264.26668701,87.76463186)
\lineto(264.11668701,88.15463186)
\curveto(264.06669749,88.26462891)(264.03669752,88.3796288)(264.02668701,88.49963186)
\curveto(264.00669755,88.61962856)(263.98169757,88.74462843)(263.95168701,88.87463186)
\curveto(263.9516976,88.94462823)(263.9516976,89.00962817)(263.95168701,89.06963186)
\curveto(263.94169761,89.12962805)(263.93169762,89.19462798)(263.92168701,89.26463186)
}
}
{
\newrgbcolor{curcolor}{0 0 0}
\pscustom[linestyle=none,fillstyle=solid,fillcolor=curcolor]
{
\newpath
\moveto(269.44168701,101.36424123)
\lineto(269.69668701,101.36424123)
\curveto(269.77669178,101.37423353)(269.8516917,101.36923353)(269.92168701,101.34924123)
\lineto(270.16168701,101.34924123)
\lineto(270.32668701,101.34924123)
\curveto(270.42669113,101.32923357)(270.53169102,101.31923358)(270.64168701,101.31924123)
\curveto(270.74169081,101.31923358)(270.84169071,101.30923359)(270.94168701,101.28924123)
\lineto(271.09168701,101.28924123)
\curveto(271.23169032,101.25923364)(271.37169018,101.23923366)(271.51168701,101.22924123)
\curveto(271.64168991,101.21923368)(271.77168978,101.19423371)(271.90168701,101.15424123)
\curveto(271.98168957,101.13423377)(272.06668949,101.11423379)(272.15668701,101.09424123)
\lineto(272.39668701,101.03424123)
\lineto(272.69668701,100.91424123)
\curveto(272.78668877,100.88423402)(272.87668868,100.84923405)(272.96668701,100.80924123)
\curveto(273.18668837,100.70923419)(273.40168815,100.57423433)(273.61168701,100.40424123)
\curveto(273.82168773,100.24423466)(273.99168756,100.06923483)(274.12168701,99.87924123)
\curveto(274.16168739,99.82923507)(274.20168735,99.76923513)(274.24168701,99.69924123)
\curveto(274.27168728,99.63923526)(274.30668725,99.57923532)(274.34668701,99.51924123)
\curveto(274.39668716,99.43923546)(274.43668712,99.34423556)(274.46668701,99.23424123)
\curveto(274.49668706,99.12423578)(274.52668703,99.01923588)(274.55668701,98.91924123)
\curveto(274.59668696,98.80923609)(274.62168693,98.6992362)(274.63168701,98.58924123)
\curveto(274.64168691,98.47923642)(274.6566869,98.36423654)(274.67668701,98.24424123)
\curveto(274.68668687,98.2042367)(274.68668687,98.15923674)(274.67668701,98.10924123)
\curveto(274.67668688,98.06923683)(274.68168687,98.02923687)(274.69168701,97.98924123)
\curveto(274.70168685,97.94923695)(274.70668685,97.89423701)(274.70668701,97.82424123)
\curveto(274.70668685,97.75423715)(274.70168685,97.7042372)(274.69168701,97.67424123)
\curveto(274.67168688,97.62423728)(274.66668689,97.57923732)(274.67668701,97.53924123)
\curveto(274.68668687,97.4992374)(274.68668687,97.46423744)(274.67668701,97.43424123)
\lineto(274.67668701,97.34424123)
\curveto(274.6566869,97.28423762)(274.64168691,97.21923768)(274.63168701,97.14924123)
\curveto(274.63168692,97.08923781)(274.62668693,97.02423788)(274.61668701,96.95424123)
\curveto(274.56668699,96.78423812)(274.51668704,96.62423828)(274.46668701,96.47424123)
\curveto(274.41668714,96.32423858)(274.3516872,96.17923872)(274.27168701,96.03924123)
\curveto(274.23168732,95.98923891)(274.20168735,95.93423897)(274.18168701,95.87424123)
\curveto(274.1516874,95.82423908)(274.11668744,95.77423913)(274.07668701,95.72424123)
\curveto(273.89668766,95.48423942)(273.67668788,95.28423962)(273.41668701,95.12424123)
\curveto(273.1566884,94.96423994)(272.87168868,94.82424008)(272.56168701,94.70424123)
\curveto(272.42168913,94.64424026)(272.28168927,94.5992403)(272.14168701,94.56924123)
\curveto(271.99168956,94.53924036)(271.83668972,94.5042404)(271.67668701,94.46424123)
\curveto(271.56668999,94.44424046)(271.4566901,94.42924047)(271.34668701,94.41924123)
\curveto(271.23669032,94.40924049)(271.12669043,94.39424051)(271.01668701,94.37424123)
\curveto(270.97669058,94.36424054)(270.93669062,94.35924054)(270.89668701,94.35924123)
\curveto(270.8566907,94.36924053)(270.81669074,94.36924053)(270.77668701,94.35924123)
\curveto(270.72669083,94.34924055)(270.67669088,94.34424056)(270.62668701,94.34424123)
\lineto(270.46168701,94.34424123)
\curveto(270.41169114,94.32424058)(270.36169119,94.31924058)(270.31168701,94.32924123)
\curveto(270.2516913,94.33924056)(270.19669136,94.33924056)(270.14668701,94.32924123)
\curveto(270.10669145,94.31924058)(270.06169149,94.31924058)(270.01168701,94.32924123)
\curveto(269.96169159,94.33924056)(269.91169164,94.33424057)(269.86168701,94.31424123)
\curveto(269.79169176,94.29424061)(269.71669184,94.28924061)(269.63668701,94.29924123)
\curveto(269.54669201,94.30924059)(269.46169209,94.31424059)(269.38168701,94.31424123)
\curveto(269.29169226,94.31424059)(269.19169236,94.30924059)(269.08168701,94.29924123)
\curveto(268.96169259,94.28924061)(268.86169269,94.29424061)(268.78168701,94.31424123)
\lineto(268.49668701,94.31424123)
\lineto(267.86668701,94.35924123)
\curveto(267.76669379,94.36924053)(267.67169388,94.37924052)(267.58168701,94.38924123)
\lineto(267.28168701,94.41924123)
\curveto(267.23169432,94.43924046)(267.18169437,94.44424046)(267.13168701,94.43424123)
\curveto(267.07169448,94.43424047)(267.01669454,94.44424046)(266.96668701,94.46424123)
\curveto(266.79669476,94.51424039)(266.63169492,94.55424035)(266.47168701,94.58424123)
\curveto(266.30169525,94.61424029)(266.14169541,94.66424024)(265.99168701,94.73424123)
\curveto(265.53169602,94.92423998)(265.1566964,95.14423976)(264.86668701,95.39424123)
\curveto(264.57669698,95.65423925)(264.33169722,96.01423889)(264.13168701,96.47424123)
\curveto(264.08169747,96.6042383)(264.04669751,96.73423817)(264.02668701,96.86424123)
\curveto(264.00669755,97.0042379)(263.98169757,97.14423776)(263.95168701,97.28424123)
\curveto(263.94169761,97.35423755)(263.93669762,97.41923748)(263.93668701,97.47924123)
\curveto(263.93669762,97.53923736)(263.93169762,97.6042373)(263.92168701,97.67424123)
\curveto(263.90169765,98.5042364)(264.0516975,99.17423573)(264.37168701,99.68424123)
\curveto(264.68169687,100.19423471)(265.12169643,100.57423433)(265.69168701,100.82424123)
\curveto(265.81169574,100.87423403)(265.93669562,100.91923398)(266.06668701,100.95924123)
\curveto(266.19669536,100.9992339)(266.33169522,101.04423386)(266.47168701,101.09424123)
\curveto(266.551695,101.11423379)(266.63669492,101.12923377)(266.72668701,101.13924123)
\lineto(266.96668701,101.19924123)
\curveto(267.07669448,101.22923367)(267.18669437,101.24423366)(267.29668701,101.24424123)
\curveto(267.40669415,101.25423365)(267.51669404,101.26923363)(267.62668701,101.28924123)
\curveto(267.67669388,101.30923359)(267.72169383,101.31423359)(267.76168701,101.30424123)
\curveto(267.80169375,101.3042336)(267.84169371,101.30923359)(267.88168701,101.31924123)
\curveto(267.93169362,101.32923357)(267.98669357,101.32923357)(268.04668701,101.31924123)
\curveto(268.09669346,101.31923358)(268.14669341,101.32423358)(268.19668701,101.33424123)
\lineto(268.33168701,101.33424123)
\curveto(268.39169316,101.35423355)(268.46169309,101.35423355)(268.54168701,101.33424123)
\curveto(268.61169294,101.32423358)(268.67669288,101.32923357)(268.73668701,101.34924123)
\curveto(268.76669279,101.35923354)(268.80669275,101.36423354)(268.85668701,101.36424123)
\lineto(268.97668701,101.36424123)
\lineto(269.44168701,101.36424123)
\moveto(271.76668701,99.81924123)
\curveto(271.44669011,99.91923498)(271.08169047,99.97923492)(270.67168701,99.99924123)
\curveto(270.26169129,100.01923488)(269.8516917,100.02923487)(269.44168701,100.02924123)
\curveto(269.01169254,100.02923487)(268.59169296,100.01923488)(268.18168701,99.99924123)
\curveto(267.77169378,99.97923492)(267.38669417,99.93423497)(267.02668701,99.86424123)
\curveto(266.66669489,99.79423511)(266.34669521,99.68423522)(266.06668701,99.53424123)
\curveto(265.77669578,99.39423551)(265.54169601,99.1992357)(265.36168701,98.94924123)
\curveto(265.2516963,98.78923611)(265.17169638,98.60923629)(265.12168701,98.40924123)
\curveto(265.06169649,98.20923669)(265.03169652,97.96423694)(265.03168701,97.67424123)
\curveto(265.0516965,97.65423725)(265.06169649,97.61923728)(265.06168701,97.56924123)
\curveto(265.0516965,97.51923738)(265.0516965,97.47923742)(265.06168701,97.44924123)
\curveto(265.08169647,97.36923753)(265.10169645,97.29423761)(265.12168701,97.22424123)
\curveto(265.13169642,97.16423774)(265.1516964,97.0992378)(265.18168701,97.02924123)
\curveto(265.30169625,96.75923814)(265.47169608,96.53923836)(265.69168701,96.36924123)
\curveto(265.90169565,96.20923869)(266.14669541,96.07423883)(266.42668701,95.96424123)
\curveto(266.53669502,95.91423899)(266.6566949,95.87423903)(266.78668701,95.84424123)
\curveto(266.90669465,95.82423908)(267.03169452,95.7992391)(267.16168701,95.76924123)
\curveto(267.21169434,95.74923915)(267.26669429,95.73923916)(267.32668701,95.73924123)
\curveto(267.37669418,95.73923916)(267.42669413,95.73423917)(267.47668701,95.72424123)
\curveto(267.56669399,95.71423919)(267.66169389,95.7042392)(267.76168701,95.69424123)
\curveto(267.8516937,95.68423922)(267.94669361,95.67423923)(268.04668701,95.66424123)
\curveto(268.12669343,95.66423924)(268.21169334,95.65923924)(268.30168701,95.64924123)
\lineto(268.54168701,95.64924123)
\lineto(268.72168701,95.64924123)
\curveto(268.7516928,95.63923926)(268.78669277,95.63423927)(268.82668701,95.63424123)
\lineto(268.96168701,95.63424123)
\lineto(269.41168701,95.63424123)
\curveto(269.49169206,95.63423927)(269.57669198,95.62923927)(269.66668701,95.61924123)
\curveto(269.74669181,95.61923928)(269.82169173,95.62923927)(269.89168701,95.64924123)
\lineto(270.16168701,95.64924123)
\curveto(270.18169137,95.64923925)(270.21169134,95.64423926)(270.25168701,95.63424123)
\curveto(270.28169127,95.63423927)(270.30669125,95.63923926)(270.32668701,95.64924123)
\curveto(270.42669113,95.65923924)(270.52669103,95.66423924)(270.62668701,95.66424123)
\curveto(270.71669084,95.67423923)(270.81669074,95.68423922)(270.92668701,95.69424123)
\curveto(271.04669051,95.72423918)(271.17169038,95.73923916)(271.30168701,95.73924123)
\curveto(271.42169013,95.74923915)(271.53669002,95.77423913)(271.64668701,95.81424123)
\curveto(271.94668961,95.89423901)(272.21168934,95.97923892)(272.44168701,96.06924123)
\curveto(272.67168888,96.16923873)(272.88668867,96.31423859)(273.08668701,96.50424123)
\curveto(273.28668827,96.71423819)(273.43668812,96.97923792)(273.53668701,97.29924123)
\curveto(273.556688,97.33923756)(273.56668799,97.37423753)(273.56668701,97.40424123)
\curveto(273.556688,97.44423746)(273.56168799,97.48923741)(273.58168701,97.53924123)
\curveto(273.59168796,97.57923732)(273.60168795,97.64923725)(273.61168701,97.74924123)
\curveto(273.62168793,97.85923704)(273.61668794,97.94423696)(273.59668701,98.00424123)
\curveto(273.57668798,98.07423683)(273.56668799,98.14423676)(273.56668701,98.21424123)
\curveto(273.556688,98.28423662)(273.54168801,98.34923655)(273.52168701,98.40924123)
\curveto(273.46168809,98.60923629)(273.37668818,98.78923611)(273.26668701,98.94924123)
\curveto(273.24668831,98.97923592)(273.22668833,99.0042359)(273.20668701,99.02424123)
\lineto(273.14668701,99.08424123)
\curveto(273.12668843,99.12423578)(273.08668847,99.17423573)(273.02668701,99.23424123)
\curveto(272.88668867,99.33423557)(272.7566888,99.41923548)(272.63668701,99.48924123)
\curveto(272.51668904,99.55923534)(272.37168918,99.62923527)(272.20168701,99.69924123)
\curveto(272.13168942,99.72923517)(272.06168949,99.74923515)(271.99168701,99.75924123)
\curveto(271.92168963,99.77923512)(271.84668971,99.7992351)(271.76668701,99.81924123)
}
}
{
\newrgbcolor{curcolor}{0 0 0}
\pscustom[linestyle=none,fillstyle=solid,fillcolor=curcolor]
{
\newpath
\moveto(263.92168701,106.77385061)
\curveto(263.92169763,106.87384575)(263.93169762,106.96884566)(263.95168701,107.05885061)
\curveto(263.96169759,107.14884548)(263.99169756,107.21384541)(264.04168701,107.25385061)
\curveto(264.12169743,107.31384531)(264.22669733,107.34384528)(264.35668701,107.34385061)
\lineto(264.74668701,107.34385061)
\lineto(266.24668701,107.34385061)
\lineto(272.63668701,107.34385061)
\lineto(273.80668701,107.34385061)
\lineto(274.12168701,107.34385061)
\curveto(274.22168733,107.35384527)(274.30168725,107.33884529)(274.36168701,107.29885061)
\curveto(274.44168711,107.24884538)(274.49168706,107.17384545)(274.51168701,107.07385061)
\curveto(274.52168703,106.98384564)(274.52668703,106.87384575)(274.52668701,106.74385061)
\lineto(274.52668701,106.51885061)
\curveto(274.50668705,106.43884619)(274.49168706,106.36884626)(274.48168701,106.30885061)
\curveto(274.46168709,106.24884638)(274.42168713,106.19884643)(274.36168701,106.15885061)
\curveto(274.30168725,106.11884651)(274.22668733,106.09884653)(274.13668701,106.09885061)
\lineto(273.83668701,106.09885061)
\lineto(272.74168701,106.09885061)
\lineto(267.40168701,106.09885061)
\curveto(267.31169424,106.07884655)(267.23669432,106.06384656)(267.17668701,106.05385061)
\curveto(267.10669445,106.05384657)(267.04669451,106.0238466)(266.99668701,105.96385061)
\curveto(266.94669461,105.89384673)(266.92169463,105.80384682)(266.92168701,105.69385061)
\curveto(266.91169464,105.59384703)(266.90669465,105.48384714)(266.90668701,105.36385061)
\lineto(266.90668701,104.22385061)
\lineto(266.90668701,103.72885061)
\curveto(266.89669466,103.56884906)(266.83669472,103.45884917)(266.72668701,103.39885061)
\curveto(266.69669486,103.37884925)(266.66669489,103.36884926)(266.63668701,103.36885061)
\curveto(266.59669496,103.36884926)(266.551695,103.36384926)(266.50168701,103.35385061)
\curveto(266.38169517,103.33384929)(266.27169528,103.33884929)(266.17168701,103.36885061)
\curveto(266.07169548,103.40884922)(266.00169555,103.46384916)(265.96168701,103.53385061)
\curveto(265.91169564,103.61384901)(265.88669567,103.73384889)(265.88668701,103.89385061)
\curveto(265.88669567,104.05384857)(265.87169568,104.18884844)(265.84168701,104.29885061)
\curveto(265.83169572,104.34884828)(265.82669573,104.40384822)(265.82668701,104.46385061)
\curveto(265.81669574,104.5238481)(265.80169575,104.58384804)(265.78168701,104.64385061)
\curveto(265.73169582,104.79384783)(265.68169587,104.93884769)(265.63168701,105.07885061)
\curveto(265.57169598,105.21884741)(265.50169605,105.35384727)(265.42168701,105.48385061)
\curveto(265.33169622,105.623847)(265.22669633,105.74384688)(265.10668701,105.84385061)
\curveto(264.98669657,105.94384668)(264.8566967,106.03884659)(264.71668701,106.12885061)
\curveto(264.61669694,106.18884644)(264.50669705,106.23384639)(264.38668701,106.26385061)
\curveto(264.26669729,106.30384632)(264.16169739,106.35384627)(264.07168701,106.41385061)
\curveto(264.01169754,106.46384616)(263.97169758,106.53384609)(263.95168701,106.62385061)
\curveto(263.94169761,106.64384598)(263.93669762,106.66884596)(263.93668701,106.69885061)
\curveto(263.93669762,106.7288459)(263.93169762,106.75384587)(263.92168701,106.77385061)
}
}
{
\newrgbcolor{curcolor}{0 0 0}
\pscustom[linestyle=none,fillstyle=solid,fillcolor=curcolor]
{
\newpath
\moveto(263.92168701,115.12345998)
\curveto(263.92169763,115.22345513)(263.93169762,115.31845503)(263.95168701,115.40845998)
\curveto(263.96169759,115.49845485)(263.99169756,115.56345479)(264.04168701,115.60345998)
\curveto(264.12169743,115.66345469)(264.22669733,115.69345466)(264.35668701,115.69345998)
\lineto(264.74668701,115.69345998)
\lineto(266.24668701,115.69345998)
\lineto(272.63668701,115.69345998)
\lineto(273.80668701,115.69345998)
\lineto(274.12168701,115.69345998)
\curveto(274.22168733,115.70345465)(274.30168725,115.68845466)(274.36168701,115.64845998)
\curveto(274.44168711,115.59845475)(274.49168706,115.52345483)(274.51168701,115.42345998)
\curveto(274.52168703,115.33345502)(274.52668703,115.22345513)(274.52668701,115.09345998)
\lineto(274.52668701,114.86845998)
\curveto(274.50668705,114.78845556)(274.49168706,114.71845563)(274.48168701,114.65845998)
\curveto(274.46168709,114.59845575)(274.42168713,114.5484558)(274.36168701,114.50845998)
\curveto(274.30168725,114.46845588)(274.22668733,114.4484559)(274.13668701,114.44845998)
\lineto(273.83668701,114.44845998)
\lineto(272.74168701,114.44845998)
\lineto(267.40168701,114.44845998)
\curveto(267.31169424,114.42845592)(267.23669432,114.41345594)(267.17668701,114.40345998)
\curveto(267.10669445,114.40345595)(267.04669451,114.37345598)(266.99668701,114.31345998)
\curveto(266.94669461,114.24345611)(266.92169463,114.1534562)(266.92168701,114.04345998)
\curveto(266.91169464,113.94345641)(266.90669465,113.83345652)(266.90668701,113.71345998)
\lineto(266.90668701,112.57345998)
\lineto(266.90668701,112.07845998)
\curveto(266.89669466,111.91845843)(266.83669472,111.80845854)(266.72668701,111.74845998)
\curveto(266.69669486,111.72845862)(266.66669489,111.71845863)(266.63668701,111.71845998)
\curveto(266.59669496,111.71845863)(266.551695,111.71345864)(266.50168701,111.70345998)
\curveto(266.38169517,111.68345867)(266.27169528,111.68845866)(266.17168701,111.71845998)
\curveto(266.07169548,111.75845859)(266.00169555,111.81345854)(265.96168701,111.88345998)
\curveto(265.91169564,111.96345839)(265.88669567,112.08345827)(265.88668701,112.24345998)
\curveto(265.88669567,112.40345795)(265.87169568,112.53845781)(265.84168701,112.64845998)
\curveto(265.83169572,112.69845765)(265.82669573,112.7534576)(265.82668701,112.81345998)
\curveto(265.81669574,112.87345748)(265.80169575,112.93345742)(265.78168701,112.99345998)
\curveto(265.73169582,113.14345721)(265.68169587,113.28845706)(265.63168701,113.42845998)
\curveto(265.57169598,113.56845678)(265.50169605,113.70345665)(265.42168701,113.83345998)
\curveto(265.33169622,113.97345638)(265.22669633,114.09345626)(265.10668701,114.19345998)
\curveto(264.98669657,114.29345606)(264.8566967,114.38845596)(264.71668701,114.47845998)
\curveto(264.61669694,114.53845581)(264.50669705,114.58345577)(264.38668701,114.61345998)
\curveto(264.26669729,114.6534557)(264.16169739,114.70345565)(264.07168701,114.76345998)
\curveto(264.01169754,114.81345554)(263.97169758,114.88345547)(263.95168701,114.97345998)
\curveto(263.94169761,114.99345536)(263.93669762,115.01845533)(263.93668701,115.04845998)
\curveto(263.93669762,115.07845527)(263.93169762,115.10345525)(263.92168701,115.12345998)
}
}
{
\newrgbcolor{curcolor}{0 0 0}
\pscustom[linestyle=none,fillstyle=solid,fillcolor=curcolor]
{
\newpath
\moveto(284.75803345,31.67142873)
\lineto(284.75803345,32.58642873)
\curveto(284.75804414,32.68642608)(284.75804414,32.78142599)(284.75803345,32.87142873)
\curveto(284.75804414,32.96142581)(284.77804412,33.03642573)(284.81803345,33.09642873)
\curveto(284.87804402,33.18642558)(284.95804394,33.24642552)(285.05803345,33.27642873)
\curveto(285.15804374,33.31642545)(285.26304364,33.36142541)(285.37303345,33.41142873)
\curveto(285.56304334,33.49142528)(285.75304315,33.56142521)(285.94303345,33.62142873)
\curveto(286.13304277,33.69142508)(286.32304258,33.766425)(286.51303345,33.84642873)
\curveto(286.69304221,33.91642485)(286.87804202,33.98142479)(287.06803345,34.04142873)
\curveto(287.24804165,34.10142467)(287.42804147,34.1714246)(287.60803345,34.25142873)
\curveto(287.74804115,34.31142446)(287.89304101,34.3664244)(288.04303345,34.41642873)
\curveto(288.19304071,34.4664243)(288.33804056,34.52142425)(288.47803345,34.58142873)
\curveto(288.92803997,34.76142401)(289.38303952,34.93142384)(289.84303345,35.09142873)
\curveto(290.29303861,35.25142352)(290.74303816,35.42142335)(291.19303345,35.60142873)
\curveto(291.24303766,35.62142315)(291.29303761,35.63642313)(291.34303345,35.64642873)
\lineto(291.49303345,35.70642873)
\curveto(291.71303719,35.79642297)(291.93803696,35.88142289)(292.16803345,35.96142873)
\curveto(292.38803651,36.04142273)(292.60803629,36.12642264)(292.82803345,36.21642873)
\curveto(292.91803598,36.25642251)(293.02803587,36.29642247)(293.15803345,36.33642873)
\curveto(293.27803562,36.37642239)(293.34803555,36.44142233)(293.36803345,36.53142873)
\curveto(293.37803552,36.5714222)(293.37803552,36.60142217)(293.36803345,36.62142873)
\lineto(293.30803345,36.68142873)
\curveto(293.25803564,36.73142204)(293.2030357,36.766422)(293.14303345,36.78642873)
\curveto(293.08303582,36.81642195)(293.01803588,36.84642192)(292.94803345,36.87642873)
\lineto(292.31803345,37.11642873)
\curveto(292.0980368,37.19642157)(291.88303702,37.27642149)(291.67303345,37.35642873)
\lineto(291.52303345,37.41642873)
\lineto(291.34303345,37.47642873)
\curveto(291.15303775,37.55642121)(290.96303794,37.62642114)(290.77303345,37.68642873)
\curveto(290.57303833,37.75642101)(290.37303853,37.83142094)(290.17303345,37.91142873)
\curveto(289.59303931,38.15142062)(289.00803989,38.3714204)(288.41803345,38.57142873)
\curveto(287.82804107,38.78141999)(287.24304166,39.00641976)(286.66303345,39.24642873)
\curveto(286.46304244,39.32641944)(286.25804264,39.40141937)(286.04803345,39.47142873)
\curveto(285.83804306,39.55141922)(285.63304327,39.63141914)(285.43303345,39.71142873)
\curveto(285.35304355,39.75141902)(285.25304365,39.78641898)(285.13303345,39.81642873)
\curveto(285.01304389,39.85641891)(284.92804397,39.91141886)(284.87803345,39.98142873)
\curveto(284.83804406,40.04141873)(284.80804409,40.11641865)(284.78803345,40.20642873)
\curveto(284.76804413,40.30641846)(284.75804414,40.41641835)(284.75803345,40.53642873)
\curveto(284.74804415,40.65641811)(284.74804415,40.77641799)(284.75803345,40.89642873)
\curveto(284.75804414,41.01641775)(284.75804414,41.12641764)(284.75803345,41.22642873)
\curveto(284.75804414,41.31641745)(284.75804414,41.40641736)(284.75803345,41.49642873)
\curveto(284.75804414,41.59641717)(284.77804412,41.6714171)(284.81803345,41.72142873)
\curveto(284.86804403,41.81141696)(284.95804394,41.86141691)(285.08803345,41.87142873)
\curveto(285.21804368,41.88141689)(285.35804354,41.88641688)(285.50803345,41.88642873)
\lineto(287.15803345,41.88642873)
\lineto(293.42803345,41.88642873)
\lineto(294.68803345,41.88642873)
\curveto(294.7980341,41.88641688)(294.90803399,41.88641688)(295.01803345,41.88642873)
\curveto(295.12803377,41.89641687)(295.21303369,41.87641689)(295.27303345,41.82642873)
\curveto(295.33303357,41.79641697)(295.37303353,41.75141702)(295.39303345,41.69142873)
\curveto(295.4030335,41.63141714)(295.41803348,41.56141721)(295.43803345,41.48142873)
\lineto(295.43803345,41.24142873)
\lineto(295.43803345,40.88142873)
\curveto(295.42803347,40.771418)(295.38303352,40.69141808)(295.30303345,40.64142873)
\curveto(295.27303363,40.62141815)(295.24303366,40.60641816)(295.21303345,40.59642873)
\curveto(295.17303373,40.59641817)(295.12803377,40.58641818)(295.07803345,40.56642873)
\lineto(294.91303345,40.56642873)
\curveto(294.85303405,40.55641821)(294.78303412,40.55141822)(294.70303345,40.55142873)
\curveto(294.62303428,40.56141821)(294.54803435,40.5664182)(294.47803345,40.56642873)
\lineto(293.63803345,40.56642873)
\lineto(289.21303345,40.56642873)
\curveto(288.96303994,40.5664182)(288.71304019,40.5664182)(288.46303345,40.56642873)
\curveto(288.2030407,40.5664182)(287.95304095,40.56141821)(287.71303345,40.55142873)
\curveto(287.61304129,40.55141822)(287.5030414,40.54641822)(287.38303345,40.53642873)
\curveto(287.26304164,40.52641824)(287.2030417,40.4714183)(287.20303345,40.37142873)
\lineto(287.21803345,40.37142873)
\curveto(287.23804166,40.30141847)(287.3030416,40.24141853)(287.41303345,40.19142873)
\curveto(287.52304138,40.15141862)(287.61804128,40.11641865)(287.69803345,40.08642873)
\curveto(287.86804103,40.01641875)(288.04304086,39.95141882)(288.22303345,39.89142873)
\curveto(288.39304051,39.83141894)(288.56304034,39.76141901)(288.73303345,39.68142873)
\curveto(288.78304012,39.66141911)(288.82804007,39.64641912)(288.86803345,39.63642873)
\curveto(288.90803999,39.62641914)(288.95303995,39.61141916)(289.00303345,39.59142873)
\curveto(289.18303972,39.51141926)(289.36803953,39.44141933)(289.55803345,39.38142873)
\curveto(289.73803916,39.33141944)(289.91803898,39.2664195)(290.09803345,39.18642873)
\curveto(290.24803865,39.11641965)(290.4030385,39.05641971)(290.56303345,39.00642873)
\curveto(290.71303819,38.95641981)(290.86303804,38.90141987)(291.01303345,38.84142873)
\curveto(291.48303742,38.64142013)(291.95803694,38.46142031)(292.43803345,38.30142873)
\curveto(292.90803599,38.14142063)(293.37303553,37.9664208)(293.83303345,37.77642873)
\curveto(294.01303489,37.69642107)(294.19303471,37.62642114)(294.37303345,37.56642873)
\curveto(294.55303435,37.50642126)(294.73303417,37.44142133)(294.91303345,37.37142873)
\curveto(295.02303388,37.32142145)(295.12803377,37.2714215)(295.22803345,37.22142873)
\curveto(295.31803358,37.18142159)(295.38303352,37.09642167)(295.42303345,36.96642873)
\curveto(295.43303347,36.94642182)(295.43803346,36.92142185)(295.43803345,36.89142873)
\curveto(295.42803347,36.8714219)(295.42803347,36.84642192)(295.43803345,36.81642873)
\curveto(295.44803345,36.78642198)(295.45303345,36.75142202)(295.45303345,36.71142873)
\curveto(295.44303346,36.6714221)(295.43803346,36.63142214)(295.43803345,36.59142873)
\lineto(295.43803345,36.29142873)
\curveto(295.43803346,36.19142258)(295.41303349,36.11142266)(295.36303345,36.05142873)
\curveto(295.31303359,35.9714228)(295.24303366,35.91142286)(295.15303345,35.87142873)
\curveto(295.05303385,35.84142293)(294.95303395,35.80142297)(294.85303345,35.75142873)
\curveto(294.65303425,35.6714231)(294.44803445,35.59142318)(294.23803345,35.51142873)
\curveto(294.01803488,35.44142333)(293.80803509,35.3664234)(293.60803345,35.28642873)
\curveto(293.42803547,35.20642356)(293.24803565,35.13642363)(293.06803345,35.07642873)
\curveto(292.87803602,35.02642374)(292.69303621,34.96142381)(292.51303345,34.88142873)
\curveto(291.95303695,34.65142412)(291.38803751,34.43642433)(290.81803345,34.23642873)
\curveto(290.24803865,34.03642473)(289.68303922,33.82142495)(289.12303345,33.59142873)
\lineto(288.49303345,33.35142873)
\curveto(288.27304063,33.28142549)(288.06304084,33.20642556)(287.86303345,33.12642873)
\curveto(287.75304115,33.07642569)(287.64804125,33.03142574)(287.54803345,32.99142873)
\curveto(287.43804146,32.96142581)(287.34304156,32.91142586)(287.26303345,32.84142873)
\curveto(287.24304166,32.83142594)(287.23304167,32.82142595)(287.23303345,32.81142873)
\lineto(287.20303345,32.78142873)
\lineto(287.20303345,32.70642873)
\lineto(287.23303345,32.67642873)
\curveto(287.23304167,32.6664261)(287.23804166,32.65642611)(287.24803345,32.64642873)
\curveto(287.2980416,32.62642614)(287.35304155,32.61642615)(287.41303345,32.61642873)
\curveto(287.47304143,32.61642615)(287.53304137,32.60642616)(287.59303345,32.58642873)
\lineto(287.75803345,32.58642873)
\curveto(287.81804108,32.5664262)(287.88304102,32.56142621)(287.95303345,32.57142873)
\curveto(288.02304088,32.58142619)(288.09304081,32.58642618)(288.16303345,32.58642873)
\lineto(288.97303345,32.58642873)
\lineto(293.53303345,32.58642873)
\lineto(294.71803345,32.58642873)
\curveto(294.82803407,32.58642618)(294.93803396,32.58142619)(295.04803345,32.57142873)
\curveto(295.15803374,32.5714262)(295.24303366,32.54642622)(295.30303345,32.49642873)
\curveto(295.38303352,32.44642632)(295.42803347,32.35642641)(295.43803345,32.22642873)
\lineto(295.43803345,31.83642873)
\lineto(295.43803345,31.64142873)
\curveto(295.43803346,31.59142718)(295.42803347,31.54142723)(295.40803345,31.49142873)
\curveto(295.36803353,31.36142741)(295.28303362,31.28642748)(295.15303345,31.26642873)
\curveto(295.02303388,31.25642751)(294.87303403,31.25142752)(294.70303345,31.25142873)
\lineto(292.96303345,31.25142873)
\lineto(286.96303345,31.25142873)
\lineto(285.55303345,31.25142873)
\curveto(285.44304346,31.25142752)(285.32804357,31.24642752)(285.20803345,31.23642873)
\curveto(285.08804381,31.23642753)(284.99304391,31.26142751)(284.92303345,31.31142873)
\curveto(284.86304404,31.35142742)(284.81304409,31.42642734)(284.77303345,31.53642873)
\curveto(284.76304414,31.55642721)(284.76304414,31.57642719)(284.77303345,31.59642873)
\curveto(284.77304413,31.62642714)(284.76804413,31.65142712)(284.75803345,31.67142873)
}
}
{
\newrgbcolor{curcolor}{0 0 0}
\pscustom[linestyle=none,fillstyle=solid,fillcolor=curcolor]
{
\newpath
\moveto(294.88303345,50.87353811)
\curveto(295.04303386,50.90353028)(295.17803372,50.88853029)(295.28803345,50.82853811)
\curveto(295.38803351,50.76853041)(295.46303344,50.68853049)(295.51303345,50.58853811)
\curveto(295.53303337,50.53853064)(295.54303336,50.4835307)(295.54303345,50.42353811)
\curveto(295.54303336,50.37353081)(295.55303335,50.31853086)(295.57303345,50.25853811)
\curveto(295.62303328,50.03853114)(295.60803329,49.81853136)(295.52803345,49.59853811)
\curveto(295.45803344,49.38853179)(295.36803353,49.24353194)(295.25803345,49.16353811)
\curveto(295.18803371,49.11353207)(295.10803379,49.06853211)(295.01803345,49.02853811)
\curveto(294.91803398,48.98853219)(294.83803406,48.93853224)(294.77803345,48.87853811)
\curveto(294.75803414,48.85853232)(294.73803416,48.83353235)(294.71803345,48.80353811)
\curveto(294.6980342,48.7835324)(294.69303421,48.75353243)(294.70303345,48.71353811)
\curveto(294.73303417,48.60353258)(294.78803411,48.49853268)(294.86803345,48.39853811)
\curveto(294.94803395,48.30853287)(295.01803388,48.21853296)(295.07803345,48.12853811)
\curveto(295.15803374,47.99853318)(295.23303367,47.85853332)(295.30303345,47.70853811)
\curveto(295.36303354,47.55853362)(295.41803348,47.39853378)(295.46803345,47.22853811)
\curveto(295.4980334,47.12853405)(295.51803338,47.01853416)(295.52803345,46.89853811)
\curveto(295.53803336,46.78853439)(295.55303335,46.6785345)(295.57303345,46.56853811)
\curveto(295.58303332,46.51853466)(295.58803331,46.47353471)(295.58803345,46.43353811)
\lineto(295.58803345,46.32853811)
\curveto(295.60803329,46.21853496)(295.60803329,46.11353507)(295.58803345,46.01353811)
\lineto(295.58803345,45.87853811)
\curveto(295.57803332,45.82853535)(295.57303333,45.7785354)(295.57303345,45.72853811)
\curveto(295.57303333,45.6785355)(295.56303334,45.63353555)(295.54303345,45.59353811)
\curveto(295.53303337,45.55353563)(295.52803337,45.51853566)(295.52803345,45.48853811)
\curveto(295.53803336,45.46853571)(295.53803336,45.44353574)(295.52803345,45.41353811)
\lineto(295.46803345,45.17353811)
\curveto(295.45803344,45.09353609)(295.43803346,45.01853616)(295.40803345,44.94853811)
\curveto(295.27803362,44.64853653)(295.13303377,44.40353678)(294.97303345,44.21353811)
\curveto(294.8030341,44.03353715)(294.56803433,43.8835373)(294.26803345,43.76353811)
\curveto(294.04803485,43.67353751)(293.78303512,43.62853755)(293.47303345,43.62853811)
\lineto(293.15803345,43.62853811)
\curveto(293.10803579,43.63853754)(293.05803584,43.64353754)(293.00803345,43.64353811)
\lineto(292.82803345,43.67353811)
\lineto(292.49803345,43.79353811)
\curveto(292.38803651,43.83353735)(292.28803661,43.8835373)(292.19803345,43.94353811)
\curveto(291.90803699,44.12353706)(291.69303721,44.36853681)(291.55303345,44.67853811)
\curveto(291.41303749,44.98853619)(291.28803761,45.32853585)(291.17803345,45.69853811)
\curveto(291.13803776,45.83853534)(291.10803779,45.9835352)(291.08803345,46.13353811)
\curveto(291.06803783,46.2835349)(291.04303786,46.43353475)(291.01303345,46.58353811)
\curveto(290.99303791,46.65353453)(290.98303792,46.71853446)(290.98303345,46.77853811)
\curveto(290.98303792,46.84853433)(290.97303793,46.92353426)(290.95303345,47.00353811)
\curveto(290.93303797,47.07353411)(290.92303798,47.14353404)(290.92303345,47.21353811)
\curveto(290.91303799,47.2835339)(290.898038,47.35853382)(290.87803345,47.43853811)
\curveto(290.81803808,47.68853349)(290.76803813,47.92353326)(290.72803345,48.14353811)
\curveto(290.67803822,48.36353282)(290.56303834,48.53853264)(290.38303345,48.66853811)
\curveto(290.3030386,48.72853245)(290.2030387,48.7785324)(290.08303345,48.81853811)
\curveto(289.95303895,48.85853232)(289.81303909,48.85853232)(289.66303345,48.81853811)
\curveto(289.42303948,48.75853242)(289.23303967,48.66853251)(289.09303345,48.54853811)
\curveto(288.95303995,48.43853274)(288.84304006,48.2785329)(288.76303345,48.06853811)
\curveto(288.71304019,47.94853323)(288.67804022,47.80353338)(288.65803345,47.63353811)
\curveto(288.63804026,47.47353371)(288.62804027,47.30353388)(288.62803345,47.12353811)
\curveto(288.62804027,46.94353424)(288.63804026,46.76853441)(288.65803345,46.59853811)
\curveto(288.67804022,46.42853475)(288.70804019,46.2835349)(288.74803345,46.16353811)
\curveto(288.80804009,45.99353519)(288.89304001,45.82853535)(289.00303345,45.66853811)
\curveto(289.06303984,45.58853559)(289.14303976,45.51353567)(289.24303345,45.44353811)
\curveto(289.33303957,45.3835358)(289.43303947,45.32853585)(289.54303345,45.27853811)
\curveto(289.62303928,45.24853593)(289.70803919,45.21853596)(289.79803345,45.18853811)
\curveto(289.88803901,45.16853601)(289.95803894,45.12353606)(290.00803345,45.05353811)
\curveto(290.03803886,45.01353617)(290.06303884,44.94353624)(290.08303345,44.84353811)
\curveto(290.09303881,44.75353643)(290.0980388,44.65853652)(290.09803345,44.55853811)
\curveto(290.0980388,44.45853672)(290.09303881,44.35853682)(290.08303345,44.25853811)
\curveto(290.06303884,44.16853701)(290.03803886,44.10353708)(290.00803345,44.06353811)
\curveto(289.97803892,44.02353716)(289.92803897,43.99353719)(289.85803345,43.97353811)
\curveto(289.78803911,43.95353723)(289.71303919,43.95353723)(289.63303345,43.97353811)
\curveto(289.5030394,44.00353718)(289.38303952,44.03353715)(289.27303345,44.06353811)
\curveto(289.15303975,44.10353708)(289.03803986,44.14853703)(288.92803345,44.19853811)
\curveto(288.57804032,44.38853679)(288.30804059,44.62853655)(288.11803345,44.91853811)
\curveto(287.91804098,45.20853597)(287.75804114,45.56853561)(287.63803345,45.99853811)
\curveto(287.61804128,46.09853508)(287.6030413,46.19853498)(287.59303345,46.29853811)
\curveto(287.58304132,46.40853477)(287.56804133,46.51853466)(287.54803345,46.62853811)
\curveto(287.53804136,46.66853451)(287.53804136,46.73353445)(287.54803345,46.82353811)
\curveto(287.54804135,46.91353427)(287.53804136,46.96853421)(287.51803345,46.98853811)
\curveto(287.50804139,47.68853349)(287.58804131,48.29853288)(287.75803345,48.81853811)
\curveto(287.92804097,49.33853184)(288.25304065,49.70353148)(288.73303345,49.91353811)
\curveto(288.93303997,50.00353118)(289.16803973,50.05353113)(289.43803345,50.06353811)
\curveto(289.6980392,50.0835311)(289.97303893,50.09353109)(290.26303345,50.09353811)
\lineto(293.57803345,50.09353811)
\curveto(293.71803518,50.09353109)(293.85303505,50.09853108)(293.98303345,50.10853811)
\curveto(294.11303479,50.11853106)(294.21803468,50.14853103)(294.29803345,50.19853811)
\curveto(294.36803453,50.24853093)(294.41803448,50.31353087)(294.44803345,50.39353811)
\curveto(294.48803441,50.4835307)(294.51803438,50.56853061)(294.53803345,50.64853811)
\curveto(294.54803435,50.72853045)(294.59303431,50.78853039)(294.67303345,50.82853811)
\curveto(294.7030342,50.84853033)(294.73303417,50.85853032)(294.76303345,50.85853811)
\curveto(294.79303411,50.85853032)(294.83303407,50.86353032)(294.88303345,50.87353811)
\moveto(293.21803345,48.72853811)
\curveto(293.07803582,48.78853239)(292.91803598,48.81853236)(292.73803345,48.81853811)
\curveto(292.54803635,48.82853235)(292.35303655,48.83353235)(292.15303345,48.83353811)
\curveto(292.04303686,48.83353235)(291.94303696,48.82853235)(291.85303345,48.81853811)
\curveto(291.76303714,48.80853237)(291.69303721,48.76853241)(291.64303345,48.69853811)
\curveto(291.62303728,48.66853251)(291.61303729,48.59853258)(291.61303345,48.48853811)
\curveto(291.63303727,48.46853271)(291.64303726,48.43353275)(291.64303345,48.38353811)
\curveto(291.64303726,48.33353285)(291.65303725,48.28853289)(291.67303345,48.24853811)
\curveto(291.69303721,48.16853301)(291.71303719,48.0785331)(291.73303345,47.97853811)
\lineto(291.79303345,47.67853811)
\curveto(291.79303711,47.64853353)(291.7980371,47.61353357)(291.80803345,47.57353811)
\lineto(291.80803345,47.46853811)
\curveto(291.84803705,47.31853386)(291.87303703,47.15353403)(291.88303345,46.97353811)
\curveto(291.88303702,46.80353438)(291.903037,46.64353454)(291.94303345,46.49353811)
\curveto(291.96303694,46.41353477)(291.98303692,46.33853484)(292.00303345,46.26853811)
\curveto(292.01303689,46.20853497)(292.02803687,46.13853504)(292.04803345,46.05853811)
\curveto(292.0980368,45.89853528)(292.16303674,45.74853543)(292.24303345,45.60853811)
\curveto(292.31303659,45.46853571)(292.4030365,45.34853583)(292.51303345,45.24853811)
\curveto(292.62303628,45.14853603)(292.75803614,45.07353611)(292.91803345,45.02353811)
\curveto(293.06803583,44.97353621)(293.25303565,44.95353623)(293.47303345,44.96353811)
\curveto(293.57303533,44.96353622)(293.66803523,44.9785362)(293.75803345,45.00853811)
\curveto(293.83803506,45.04853613)(293.91303499,45.09353609)(293.98303345,45.14353811)
\curveto(294.09303481,45.22353596)(294.18803471,45.32853585)(294.26803345,45.45853811)
\curveto(294.33803456,45.58853559)(294.3980345,45.72853545)(294.44803345,45.87853811)
\curveto(294.45803444,45.92853525)(294.46303444,45.9785352)(294.46303345,46.02853811)
\curveto(294.46303444,46.0785351)(294.46803443,46.12853505)(294.47803345,46.17853811)
\curveto(294.4980344,46.24853493)(294.51303439,46.33353485)(294.52303345,46.43353811)
\curveto(294.52303438,46.54353464)(294.51303439,46.63353455)(294.49303345,46.70353811)
\curveto(294.47303443,46.76353442)(294.46803443,46.82353436)(294.47803345,46.88353811)
\curveto(294.47803442,46.94353424)(294.46803443,47.00353418)(294.44803345,47.06353811)
\curveto(294.42803447,47.14353404)(294.41303449,47.21853396)(294.40303345,47.28853811)
\curveto(294.39303451,47.36853381)(294.37303453,47.44353374)(294.34303345,47.51353811)
\curveto(294.22303468,47.80353338)(294.07803482,48.04853313)(293.90803345,48.24853811)
\curveto(293.73803516,48.45853272)(293.50803539,48.61853256)(293.21803345,48.72853811)
}
}
{
\newrgbcolor{curcolor}{0 0 0}
\pscustom[linestyle=none,fillstyle=solid,fillcolor=curcolor]
{
\newpath
\moveto(287.53303345,55.69017873)
\curveto(287.53304137,55.92017394)(287.59304131,56.05017381)(287.71303345,56.08017873)
\curveto(287.82304108,56.11017375)(287.98804091,56.12517374)(288.20803345,56.12517873)
\lineto(288.49303345,56.12517873)
\curveto(288.58304032,56.12517374)(288.65804024,56.10017376)(288.71803345,56.05017873)
\curveto(288.7980401,55.99017387)(288.84304006,55.90517396)(288.85303345,55.79517873)
\curveto(288.85304005,55.68517418)(288.86804003,55.57517429)(288.89803345,55.46517873)
\curveto(288.92803997,55.32517454)(288.95803994,55.19017467)(288.98803345,55.06017873)
\curveto(289.01803988,54.94017492)(289.05803984,54.82517504)(289.10803345,54.71517873)
\curveto(289.23803966,54.42517544)(289.41803948,54.19017567)(289.64803345,54.01017873)
\curveto(289.86803903,53.83017603)(290.12303878,53.67517619)(290.41303345,53.54517873)
\curveto(290.52303838,53.50517636)(290.63803826,53.47517639)(290.75803345,53.45517873)
\curveto(290.86803803,53.43517643)(290.98303792,53.41017645)(291.10303345,53.38017873)
\curveto(291.15303775,53.37017649)(291.2030377,53.3651765)(291.25303345,53.36517873)
\curveto(291.3030376,53.37517649)(291.35303755,53.37517649)(291.40303345,53.36517873)
\curveto(291.52303738,53.33517653)(291.66303724,53.32017654)(291.82303345,53.32017873)
\curveto(291.97303693,53.33017653)(292.11803678,53.33517653)(292.25803345,53.33517873)
\lineto(294.10303345,53.33517873)
\lineto(294.44803345,53.33517873)
\curveto(294.56803433,53.33517653)(294.68303422,53.33017653)(294.79303345,53.32017873)
\curveto(294.903034,53.31017655)(294.9980339,53.30517656)(295.07803345,53.30517873)
\curveto(295.15803374,53.31517655)(295.22803367,53.29517657)(295.28803345,53.24517873)
\curveto(295.35803354,53.19517667)(295.3980335,53.11517675)(295.40803345,53.00517873)
\curveto(295.41803348,52.90517696)(295.42303348,52.79517707)(295.42303345,52.67517873)
\lineto(295.42303345,52.40517873)
\curveto(295.4030335,52.35517751)(295.38803351,52.30517756)(295.37803345,52.25517873)
\curveto(295.35803354,52.21517765)(295.33303357,52.18517768)(295.30303345,52.16517873)
\curveto(295.23303367,52.11517775)(295.14803375,52.08517778)(295.04803345,52.07517873)
\lineto(294.71803345,52.07517873)
\lineto(293.56303345,52.07517873)
\lineto(289.40803345,52.07517873)
\lineto(288.37303345,52.07517873)
\lineto(288.07303345,52.07517873)
\curveto(287.97304093,52.08517778)(287.88804101,52.11517775)(287.81803345,52.16517873)
\curveto(287.77804112,52.19517767)(287.74804115,52.24517762)(287.72803345,52.31517873)
\curveto(287.70804119,52.39517747)(287.6980412,52.48017738)(287.69803345,52.57017873)
\curveto(287.68804121,52.6601772)(287.68804121,52.75017711)(287.69803345,52.84017873)
\curveto(287.70804119,52.93017693)(287.72304118,53.00017686)(287.74303345,53.05017873)
\curveto(287.77304113,53.13017673)(287.83304107,53.18017668)(287.92303345,53.20017873)
\curveto(288.0030409,53.23017663)(288.09304081,53.24517662)(288.19303345,53.24517873)
\lineto(288.49303345,53.24517873)
\curveto(288.59304031,53.24517662)(288.68304022,53.2651766)(288.76303345,53.30517873)
\curveto(288.78304012,53.31517655)(288.7980401,53.32517654)(288.80803345,53.33517873)
\lineto(288.85303345,53.38017873)
\curveto(288.85304005,53.49017637)(288.80804009,53.58017628)(288.71803345,53.65017873)
\curveto(288.61804028,53.72017614)(288.53804036,53.78017608)(288.47803345,53.83017873)
\lineto(288.38803345,53.92017873)
\curveto(288.27804062,54.01017585)(288.16304074,54.13517573)(288.04303345,54.29517873)
\curveto(287.92304098,54.45517541)(287.83304107,54.60517526)(287.77303345,54.74517873)
\curveto(287.72304118,54.83517503)(287.68804121,54.93017493)(287.66803345,55.03017873)
\curveto(287.63804126,55.13017473)(287.60804129,55.23517463)(287.57803345,55.34517873)
\curveto(287.56804133,55.40517446)(287.56304134,55.4651744)(287.56303345,55.52517873)
\curveto(287.55304135,55.58517428)(287.54304136,55.64017422)(287.53303345,55.69017873)
}
}
{
\newrgbcolor{curcolor}{0 0 0}
\pscustom[linestyle=none,fillstyle=solid,fillcolor=curcolor]
{
}
}
{
\newrgbcolor{curcolor}{0 0 0}
\pscustom[linestyle=none,fillstyle=solid,fillcolor=curcolor]
{
\newpath
\moveto(284.83303345,64.24510061)
\curveto(284.82304408,64.93509597)(284.94304396,65.53509537)(285.19303345,66.04510061)
\curveto(285.44304346,66.56509434)(285.77804312,66.96009395)(286.19803345,67.23010061)
\curveto(286.27804262,67.28009363)(286.36804253,67.32509358)(286.46803345,67.36510061)
\curveto(286.55804234,67.4050935)(286.65304225,67.45009346)(286.75303345,67.50010061)
\curveto(286.85304205,67.54009337)(286.95304195,67.57009334)(287.05303345,67.59010061)
\curveto(287.15304175,67.6100933)(287.25804164,67.63009328)(287.36803345,67.65010061)
\curveto(287.41804148,67.67009324)(287.46304144,67.67509323)(287.50303345,67.66510061)
\curveto(287.54304136,67.65509325)(287.58804131,67.66009325)(287.63803345,67.68010061)
\curveto(287.68804121,67.69009322)(287.77304113,67.69509321)(287.89303345,67.69510061)
\curveto(288.0030409,67.69509321)(288.08804081,67.69009322)(288.14803345,67.68010061)
\curveto(288.20804069,67.66009325)(288.26804063,67.65009326)(288.32803345,67.65010061)
\curveto(288.38804051,67.66009325)(288.44804045,67.65509325)(288.50803345,67.63510061)
\curveto(288.64804025,67.59509331)(288.78304012,67.56009335)(288.91303345,67.53010061)
\curveto(289.04303986,67.50009341)(289.16803973,67.46009345)(289.28803345,67.41010061)
\curveto(289.42803947,67.35009356)(289.55303935,67.28009363)(289.66303345,67.20010061)
\curveto(289.77303913,67.13009378)(289.88303902,67.05509385)(289.99303345,66.97510061)
\lineto(290.05303345,66.91510061)
\curveto(290.07303883,66.905094)(290.09303881,66.89009402)(290.11303345,66.87010061)
\curveto(290.27303863,66.75009416)(290.41803848,66.61509429)(290.54803345,66.46510061)
\curveto(290.67803822,66.31509459)(290.8030381,66.15509475)(290.92303345,65.98510061)
\curveto(291.14303776,65.67509523)(291.34803755,65.38009553)(291.53803345,65.10010061)
\curveto(291.67803722,64.87009604)(291.81303709,64.64009627)(291.94303345,64.41010061)
\curveto(292.07303683,64.19009672)(292.20803669,63.97009694)(292.34803345,63.75010061)
\curveto(292.51803638,63.50009741)(292.6980362,63.26009765)(292.88803345,63.03010061)
\curveto(293.07803582,62.8100981)(293.3030356,62.62009829)(293.56303345,62.46010061)
\curveto(293.62303528,62.42009849)(293.68303522,62.38509852)(293.74303345,62.35510061)
\curveto(293.79303511,62.32509858)(293.85803504,62.29509861)(293.93803345,62.26510061)
\curveto(294.00803489,62.24509866)(294.06803483,62.24009867)(294.11803345,62.25010061)
\curveto(294.18803471,62.27009864)(294.24303466,62.3050986)(294.28303345,62.35510061)
\curveto(294.31303459,62.4050985)(294.33303457,62.46509844)(294.34303345,62.53510061)
\lineto(294.34303345,62.77510061)
\lineto(294.34303345,63.52510061)
\lineto(294.34303345,66.33010061)
\lineto(294.34303345,66.99010061)
\curveto(294.34303456,67.08009383)(294.34803455,67.16509374)(294.35803345,67.24510061)
\curveto(294.35803454,67.32509358)(294.37803452,67.39009352)(294.41803345,67.44010061)
\curveto(294.45803444,67.49009342)(294.53303437,67.53009338)(294.64303345,67.56010061)
\curveto(294.74303416,67.60009331)(294.84303406,67.6100933)(294.94303345,67.59010061)
\lineto(295.07803345,67.59010061)
\curveto(295.14803375,67.57009334)(295.20803369,67.55009336)(295.25803345,67.53010061)
\curveto(295.30803359,67.5100934)(295.34803355,67.47509343)(295.37803345,67.42510061)
\curveto(295.41803348,67.37509353)(295.43803346,67.3050936)(295.43803345,67.21510061)
\lineto(295.43803345,66.94510061)
\lineto(295.43803345,66.04510061)
\lineto(295.43803345,62.53510061)
\lineto(295.43803345,61.47010061)
\curveto(295.43803346,61.39009952)(295.44303346,61.30009961)(295.45303345,61.20010061)
\curveto(295.45303345,61.10009981)(295.44303346,61.01509989)(295.42303345,60.94510061)
\curveto(295.35303355,60.73510017)(295.17303373,60.67010024)(294.88303345,60.75010061)
\curveto(294.84303406,60.76010015)(294.80803409,60.76010015)(294.77803345,60.75010061)
\curveto(294.73803416,60.75010016)(294.69303421,60.76010015)(294.64303345,60.78010061)
\curveto(294.56303434,60.80010011)(294.47803442,60.82010009)(294.38803345,60.84010061)
\curveto(294.2980346,60.86010005)(294.21303469,60.88510002)(294.13303345,60.91510061)
\curveto(293.64303526,61.07509983)(293.22803567,61.27509963)(292.88803345,61.51510061)
\curveto(292.63803626,61.69509921)(292.41303649,61.90009901)(292.21303345,62.13010061)
\curveto(292.0030369,62.36009855)(291.80803709,62.60009831)(291.62803345,62.85010061)
\curveto(291.44803745,63.1100978)(291.27803762,63.37509753)(291.11803345,63.64510061)
\curveto(290.94803795,63.92509698)(290.77303813,64.19509671)(290.59303345,64.45510061)
\curveto(290.51303839,64.56509634)(290.43803846,64.67009624)(290.36803345,64.77010061)
\curveto(290.2980386,64.88009603)(290.22303868,64.99009592)(290.14303345,65.10010061)
\curveto(290.11303879,65.14009577)(290.08303882,65.17509573)(290.05303345,65.20510061)
\curveto(290.01303889,65.24509566)(289.98303892,65.28509562)(289.96303345,65.32510061)
\curveto(289.85303905,65.46509544)(289.72803917,65.59009532)(289.58803345,65.70010061)
\curveto(289.55803934,65.72009519)(289.53303937,65.74509516)(289.51303345,65.77510061)
\curveto(289.48303942,65.8050951)(289.45303945,65.83009508)(289.42303345,65.85010061)
\curveto(289.32303958,65.93009498)(289.22303968,65.99509491)(289.12303345,66.04510061)
\curveto(289.02303988,66.1050948)(288.91303999,66.16009475)(288.79303345,66.21010061)
\curveto(288.72304018,66.24009467)(288.64804025,66.26009465)(288.56803345,66.27010061)
\lineto(288.32803345,66.33010061)
\lineto(288.23803345,66.33010061)
\curveto(288.20804069,66.34009457)(288.17804072,66.34509456)(288.14803345,66.34510061)
\curveto(288.07804082,66.36509454)(287.98304092,66.37009454)(287.86303345,66.36010061)
\curveto(287.73304117,66.36009455)(287.63304127,66.35009456)(287.56303345,66.33010061)
\curveto(287.48304142,66.3100946)(287.40804149,66.29009462)(287.33803345,66.27010061)
\curveto(287.25804164,66.26009465)(287.17804172,66.24009467)(287.09803345,66.21010061)
\curveto(286.85804204,66.10009481)(286.65804224,65.95009496)(286.49803345,65.76010061)
\curveto(286.32804257,65.58009533)(286.18804271,65.36009555)(286.07803345,65.10010061)
\curveto(286.05804284,65.03009588)(286.04304286,64.96009595)(286.03303345,64.89010061)
\curveto(286.01304289,64.82009609)(285.99304291,64.74509616)(285.97303345,64.66510061)
\curveto(285.95304295,64.58509632)(285.94304296,64.47509643)(285.94303345,64.33510061)
\curveto(285.94304296,64.2050967)(285.95304295,64.10009681)(285.97303345,64.02010061)
\curveto(285.98304292,63.96009695)(285.98804291,63.905097)(285.98803345,63.85510061)
\curveto(285.98804291,63.8050971)(285.9980429,63.75509715)(286.01803345,63.70510061)
\curveto(286.05804284,63.6050973)(286.0980428,63.5100974)(286.13803345,63.42010061)
\curveto(286.17804272,63.34009757)(286.22304268,63.26009765)(286.27303345,63.18010061)
\curveto(286.29304261,63.15009776)(286.31804258,63.12009779)(286.34803345,63.09010061)
\curveto(286.37804252,63.07009784)(286.4030425,63.04509786)(286.42303345,63.01510061)
\lineto(286.49803345,62.94010061)
\curveto(286.51804238,62.910098)(286.53804236,62.88509802)(286.55803345,62.86510061)
\lineto(286.76803345,62.71510061)
\curveto(286.82804207,62.67509823)(286.89304201,62.63009828)(286.96303345,62.58010061)
\curveto(287.05304185,62.52009839)(287.15804174,62.47009844)(287.27803345,62.43010061)
\curveto(287.38804151,62.40009851)(287.4980414,62.36509854)(287.60803345,62.32510061)
\curveto(287.71804118,62.28509862)(287.86304104,62.26009865)(288.04303345,62.25010061)
\curveto(288.21304069,62.24009867)(288.33804056,62.2100987)(288.41803345,62.16010061)
\curveto(288.4980404,62.1100988)(288.54304036,62.03509887)(288.55303345,61.93510061)
\curveto(288.56304034,61.83509907)(288.56804033,61.72509918)(288.56803345,61.60510061)
\curveto(288.56804033,61.56509934)(288.57304033,61.52509938)(288.58303345,61.48510061)
\curveto(288.58304032,61.44509946)(288.57804032,61.4100995)(288.56803345,61.38010061)
\curveto(288.54804035,61.33009958)(288.53804036,61.28009963)(288.53803345,61.23010061)
\curveto(288.53804036,61.19009972)(288.52804037,61.15009976)(288.50803345,61.11010061)
\curveto(288.44804045,61.02009989)(288.31304059,60.97509993)(288.10303345,60.97510061)
\lineto(287.98303345,60.97510061)
\curveto(287.92304098,60.98509992)(287.86304104,60.99009992)(287.80303345,60.99010061)
\curveto(287.73304117,61.00009991)(287.66804123,61.0100999)(287.60803345,61.02010061)
\curveto(287.4980414,61.04009987)(287.3980415,61.06009985)(287.30803345,61.08010061)
\curveto(287.20804169,61.10009981)(287.11304179,61.13009978)(287.02303345,61.17010061)
\curveto(286.95304195,61.19009972)(286.89304201,61.2100997)(286.84303345,61.23010061)
\lineto(286.66303345,61.29010061)
\curveto(286.4030425,61.4100995)(286.15804274,61.56509934)(285.92803345,61.75510061)
\curveto(285.6980432,61.95509895)(285.51304339,62.17009874)(285.37303345,62.40010061)
\curveto(285.29304361,62.5100984)(285.22804367,62.62509828)(285.17803345,62.74510061)
\lineto(285.02803345,63.13510061)
\curveto(284.97804392,63.24509766)(284.94804395,63.36009755)(284.93803345,63.48010061)
\curveto(284.91804398,63.60009731)(284.89304401,63.72509718)(284.86303345,63.85510061)
\curveto(284.86304404,63.92509698)(284.86304404,63.99009692)(284.86303345,64.05010061)
\curveto(284.85304405,64.1100968)(284.84304406,64.17509673)(284.83303345,64.24510061)
}
}
{
\newrgbcolor{curcolor}{0 0 0}
\pscustom[linestyle=none,fillstyle=solid,fillcolor=curcolor]
{
\newpath
\moveto(291.11803345,76.34470998)
\curveto(291.23803766,76.37470226)(291.37803752,76.39970223)(291.53803345,76.41970998)
\curveto(291.6980372,76.43970219)(291.86303704,76.44970218)(292.03303345,76.44970998)
\curveto(292.2030367,76.44970218)(292.36803653,76.43970219)(292.52803345,76.41970998)
\curveto(292.68803621,76.39970223)(292.82803607,76.37470226)(292.94803345,76.34470998)
\curveto(293.08803581,76.30470233)(293.21303569,76.26970236)(293.32303345,76.23970998)
\curveto(293.43303547,76.20970242)(293.54303536,76.16970246)(293.65303345,76.11970998)
\curveto(294.29303461,75.84970278)(294.77803412,75.4347032)(295.10803345,74.87470998)
\curveto(295.16803373,74.79470384)(295.21803368,74.70970392)(295.25803345,74.61970998)
\curveto(295.28803361,74.5297041)(295.32303358,74.4297042)(295.36303345,74.31970998)
\curveto(295.41303349,74.20970442)(295.44803345,74.08970454)(295.46803345,73.95970998)
\curveto(295.4980334,73.83970479)(295.52803337,73.70970492)(295.55803345,73.56970998)
\curveto(295.57803332,73.50970512)(295.58303332,73.44970518)(295.57303345,73.38970998)
\curveto(295.56303334,73.33970529)(295.56803333,73.27970535)(295.58803345,73.20970998)
\curveto(295.5980333,73.18970544)(295.5980333,73.16470547)(295.58803345,73.13470998)
\curveto(295.58803331,73.10470553)(295.59303331,73.07970555)(295.60303345,73.05970998)
\lineto(295.60303345,72.90970998)
\curveto(295.61303329,72.83970579)(295.61303329,72.78970584)(295.60303345,72.75970998)
\curveto(295.59303331,72.71970591)(295.58803331,72.67470596)(295.58803345,72.62470998)
\curveto(295.5980333,72.58470605)(295.5980333,72.54470609)(295.58803345,72.50470998)
\curveto(295.56803333,72.41470622)(295.55303335,72.32470631)(295.54303345,72.23470998)
\curveto(295.54303336,72.14470649)(295.53303337,72.05470658)(295.51303345,71.96470998)
\curveto(295.48303342,71.87470676)(295.45803344,71.78470685)(295.43803345,71.69470998)
\curveto(295.41803348,71.60470703)(295.38803351,71.51970711)(295.34803345,71.43970998)
\curveto(295.23803366,71.19970743)(295.10803379,70.97470766)(294.95803345,70.76470998)
\curveto(294.7980341,70.55470808)(294.61803428,70.37470826)(294.41803345,70.22470998)
\curveto(294.24803465,70.10470853)(294.07303483,69.99970863)(293.89303345,69.90970998)
\curveto(293.71303519,69.81970881)(293.52303538,69.7297089)(293.32303345,69.63970998)
\curveto(293.22303568,69.59970903)(293.12303578,69.56470907)(293.02303345,69.53470998)
\curveto(292.91303599,69.51470912)(292.8030361,69.48970914)(292.69303345,69.45970998)
\curveto(292.55303635,69.41970921)(292.41303649,69.39470924)(292.27303345,69.38470998)
\curveto(292.13303677,69.37470926)(291.99303691,69.35470928)(291.85303345,69.32470998)
\curveto(291.74303716,69.31470932)(291.64303726,69.30470933)(291.55303345,69.29470998)
\curveto(291.45303745,69.29470934)(291.35303755,69.28470935)(291.25303345,69.26470998)
\lineto(291.16303345,69.26470998)
\curveto(291.13303777,69.27470936)(291.10803779,69.27470936)(291.08803345,69.26470998)
\lineto(290.87803345,69.26470998)
\curveto(290.81803808,69.24470939)(290.75303815,69.2347094)(290.68303345,69.23470998)
\curveto(290.6030383,69.24470939)(290.52803837,69.24970938)(290.45803345,69.24970998)
\lineto(290.30803345,69.24970998)
\curveto(290.25803864,69.24970938)(290.20803869,69.25470938)(290.15803345,69.26470998)
\lineto(289.78303345,69.26470998)
\curveto(289.75303915,69.27470936)(289.71803918,69.27470936)(289.67803345,69.26470998)
\curveto(289.63803926,69.26470937)(289.5980393,69.26970936)(289.55803345,69.27970998)
\curveto(289.44803945,69.29970933)(289.33803956,69.31470932)(289.22803345,69.32470998)
\curveto(289.10803979,69.3347093)(288.99303991,69.34470929)(288.88303345,69.35470998)
\curveto(288.73304017,69.39470924)(288.58804031,69.41970921)(288.44803345,69.42970998)
\curveto(288.2980406,69.44970918)(288.15304075,69.47970915)(288.01303345,69.51970998)
\curveto(287.71304119,69.60970902)(287.42804147,69.70470893)(287.15803345,69.80470998)
\curveto(286.88804201,69.90470873)(286.63804226,70.0297086)(286.40803345,70.17970998)
\curveto(286.08804281,70.37970825)(285.80804309,70.62470801)(285.56803345,70.91470998)
\curveto(285.32804357,71.20470743)(285.14304376,71.54470709)(285.01303345,71.93470998)
\curveto(284.97304393,72.04470659)(284.94804395,72.15470648)(284.93803345,72.26470998)
\curveto(284.91804398,72.38470625)(284.89304401,72.50470613)(284.86303345,72.62470998)
\curveto(284.85304405,72.69470594)(284.84804405,72.75970587)(284.84803345,72.81970998)
\curveto(284.84804405,72.87970575)(284.84304406,72.94470569)(284.83303345,73.01470998)
\curveto(284.81304409,73.71470492)(284.92804397,74.28970434)(285.17803345,74.73970998)
\curveto(285.42804347,75.18970344)(285.77804312,75.5347031)(286.22803345,75.77470998)
\curveto(286.45804244,75.88470275)(286.73304217,75.98470265)(287.05303345,76.07470998)
\curveto(287.12304178,76.09470254)(287.1980417,76.09470254)(287.27803345,76.07470998)
\curveto(287.34804155,76.06470257)(287.3980415,76.03970259)(287.42803345,75.99970998)
\curveto(287.45804144,75.96970266)(287.48304142,75.90970272)(287.50303345,75.81970998)
\curveto(287.51304139,75.7297029)(287.52304138,75.629703)(287.53303345,75.51970998)
\curveto(287.53304137,75.41970321)(287.52804137,75.31970331)(287.51803345,75.21970998)
\curveto(287.50804139,75.1297035)(287.48804141,75.06470357)(287.45803345,75.02470998)
\curveto(287.38804151,74.91470372)(287.27804162,74.8347038)(287.12803345,74.78470998)
\curveto(286.97804192,74.74470389)(286.84804205,74.68970394)(286.73803345,74.61970998)
\curveto(286.42804247,74.4297042)(286.1980427,74.14970448)(286.04803345,73.77970998)
\curveto(286.01804288,73.70970492)(285.9980429,73.634705)(285.98803345,73.55470998)
\curveto(285.97804292,73.48470515)(285.96304294,73.40970522)(285.94303345,73.32970998)
\curveto(285.93304297,73.27970535)(285.92804297,73.20970542)(285.92803345,73.11970998)
\curveto(285.92804297,73.03970559)(285.93304297,72.97470566)(285.94303345,72.92470998)
\curveto(285.96304294,72.88470575)(285.96804293,72.84970578)(285.95803345,72.81970998)
\curveto(285.94804295,72.78970584)(285.94804295,72.75470588)(285.95803345,72.71470998)
\lineto(286.01803345,72.47470998)
\curveto(286.03804286,72.40470623)(286.06304284,72.3347063)(286.09303345,72.26470998)
\curveto(286.25304265,71.88470675)(286.46304244,71.59470704)(286.72303345,71.39470998)
\curveto(286.98304192,71.20470743)(287.2980416,71.0297076)(287.66803345,70.86970998)
\curveto(287.74804115,70.83970779)(287.82804107,70.81470782)(287.90803345,70.79470998)
\curveto(287.98804091,70.78470785)(288.06804083,70.76470787)(288.14803345,70.73470998)
\curveto(288.25804064,70.70470793)(288.37304053,70.67970795)(288.49303345,70.65970998)
\curveto(288.61304029,70.64970798)(288.73304017,70.629708)(288.85303345,70.59970998)
\curveto(288.90304,70.57970805)(288.95303995,70.56970806)(289.00303345,70.56970998)
\curveto(289.05303985,70.57970805)(289.1030398,70.57470806)(289.15303345,70.55470998)
\curveto(289.21303969,70.54470809)(289.29303961,70.54470809)(289.39303345,70.55470998)
\curveto(289.48303942,70.56470807)(289.53803936,70.57970805)(289.55803345,70.59970998)
\curveto(289.5980393,70.61970801)(289.61803928,70.64970798)(289.61803345,70.68970998)
\curveto(289.61803928,70.73970789)(289.60803929,70.78470785)(289.58803345,70.82470998)
\curveto(289.54803935,70.89470774)(289.5030394,70.95470768)(289.45303345,71.00470998)
\curveto(289.4030395,71.05470758)(289.35303955,71.11470752)(289.30303345,71.18470998)
\lineto(289.24303345,71.24470998)
\curveto(289.21303969,71.27470736)(289.18803971,71.30470733)(289.16803345,71.33470998)
\curveto(289.00803989,71.56470707)(288.87304003,71.83970679)(288.76303345,72.15970998)
\curveto(288.74304016,72.2297064)(288.72804017,72.29970633)(288.71803345,72.36970998)
\curveto(288.70804019,72.43970619)(288.69304021,72.51470612)(288.67303345,72.59470998)
\curveto(288.67304023,72.634706)(288.66804023,72.66970596)(288.65803345,72.69970998)
\curveto(288.64804025,72.7297059)(288.64804025,72.76470587)(288.65803345,72.80470998)
\curveto(288.65804024,72.85470578)(288.64804025,72.89470574)(288.62803345,72.92470998)
\lineto(288.62803345,73.08970998)
\lineto(288.62803345,73.17970998)
\curveto(288.61804028,73.2297054)(288.61804028,73.26970536)(288.62803345,73.29970998)
\curveto(288.63804026,73.34970528)(288.64304026,73.39970523)(288.64303345,73.44970998)
\curveto(288.63304027,73.50970512)(288.63304027,73.56470507)(288.64303345,73.61470998)
\curveto(288.67304023,73.72470491)(288.69304021,73.8297048)(288.70303345,73.92970998)
\curveto(288.71304019,74.03970459)(288.73804016,74.14470449)(288.77803345,74.24470998)
\curveto(288.91803998,74.66470397)(289.1030398,75.00970362)(289.33303345,75.27970998)
\curveto(289.55303935,75.54970308)(289.83803906,75.78970284)(290.18803345,75.99970998)
\curveto(290.32803857,76.07970255)(290.47803842,76.14470249)(290.63803345,76.19470998)
\curveto(290.78803811,76.24470239)(290.94803795,76.29470234)(291.11803345,76.34470998)
\moveto(292.42303345,75.09970998)
\curveto(292.37303653,75.10970352)(292.32803657,75.11470352)(292.28803345,75.11470998)
\lineto(292.13803345,75.11470998)
\curveto(291.82803707,75.11470352)(291.54303736,75.07470356)(291.28303345,74.99470998)
\curveto(291.22303768,74.97470366)(291.16803773,74.95470368)(291.11803345,74.93470998)
\curveto(291.05803784,74.92470371)(291.0030379,74.90970372)(290.95303345,74.88970998)
\curveto(290.46303844,74.66970396)(290.11303879,74.32470431)(289.90303345,73.85470998)
\curveto(289.87303903,73.77470486)(289.84803905,73.69470494)(289.82803345,73.61470998)
\lineto(289.76803345,73.37470998)
\curveto(289.74803915,73.29470534)(289.73803916,73.20470543)(289.73803345,73.10470998)
\lineto(289.73803345,72.78970998)
\curveto(289.75803914,72.76970586)(289.76803913,72.7297059)(289.76803345,72.66970998)
\curveto(289.75803914,72.61970601)(289.75803914,72.57470606)(289.76803345,72.53470998)
\lineto(289.82803345,72.29470998)
\curveto(289.83803906,72.22470641)(289.85803904,72.15470648)(289.88803345,72.08470998)
\curveto(290.14803875,71.48470715)(290.61303829,71.07970755)(291.28303345,70.86970998)
\curveto(291.36303754,70.83970779)(291.44303746,70.81970781)(291.52303345,70.80970998)
\curveto(291.6030373,70.79970783)(291.68803721,70.78470785)(291.77803345,70.76470998)
\lineto(291.92803345,70.76470998)
\curveto(291.96803693,70.75470788)(292.03803686,70.74970788)(292.13803345,70.74970998)
\curveto(292.36803653,70.74970788)(292.56303634,70.76970786)(292.72303345,70.80970998)
\curveto(292.79303611,70.8297078)(292.85803604,70.84470779)(292.91803345,70.85470998)
\curveto(292.97803592,70.86470777)(293.04303586,70.88470775)(293.11303345,70.91470998)
\curveto(293.39303551,71.02470761)(293.63803526,71.16970746)(293.84803345,71.34970998)
\curveto(294.04803485,71.5297071)(294.20803469,71.76470687)(294.32803345,72.05470998)
\lineto(294.41803345,72.29470998)
\lineto(294.47803345,72.53470998)
\curveto(294.4980344,72.58470605)(294.5030344,72.62470601)(294.49303345,72.65470998)
\curveto(294.48303442,72.69470594)(294.48803441,72.73970589)(294.50803345,72.78970998)
\curveto(294.51803438,72.81970581)(294.52303438,72.87470576)(294.52303345,72.95470998)
\curveto(294.52303438,73.0347056)(294.51803438,73.09470554)(294.50803345,73.13470998)
\curveto(294.48803441,73.24470539)(294.47303443,73.34970528)(294.46303345,73.44970998)
\curveto(294.45303445,73.54970508)(294.42303448,73.64470499)(294.37303345,73.73470998)
\curveto(294.17303473,74.26470437)(293.7980351,74.65470398)(293.24803345,74.90470998)
\curveto(293.14803575,74.94470369)(293.04303586,74.97470366)(292.93303345,74.99470998)
\lineto(292.60303345,75.08470998)
\curveto(292.52303638,75.08470355)(292.46303644,75.08970354)(292.42303345,75.09970998)
}
}
{
\newrgbcolor{curcolor}{0 0 0}
\pscustom[linestyle=none,fillstyle=solid,fillcolor=curcolor]
{
\newpath
\moveto(293.80303345,78.63431936)
\lineto(293.80303345,79.26431936)
\lineto(293.80303345,79.45931936)
\curveto(293.8030351,79.52931683)(293.81303509,79.58931677)(293.83303345,79.63931936)
\curveto(293.87303503,79.70931665)(293.91303499,79.7593166)(293.95303345,79.78931936)
\curveto(294.0030349,79.82931653)(294.06803483,79.84931651)(294.14803345,79.84931936)
\curveto(294.22803467,79.8593165)(294.31303459,79.86431649)(294.40303345,79.86431936)
\lineto(295.12303345,79.86431936)
\curveto(295.6030333,79.86431649)(296.01303289,79.80431655)(296.35303345,79.68431936)
\curveto(296.69303221,79.56431679)(296.96803193,79.36931699)(297.17803345,79.09931936)
\curveto(297.22803167,79.02931733)(297.27303163,78.9593174)(297.31303345,78.88931936)
\curveto(297.36303154,78.82931753)(297.40803149,78.7543176)(297.44803345,78.66431936)
\curveto(297.45803144,78.64431771)(297.46803143,78.61431774)(297.47803345,78.57431936)
\curveto(297.4980314,78.53431782)(297.5030314,78.48931787)(297.49303345,78.43931936)
\curveto(297.46303144,78.34931801)(297.38803151,78.29431806)(297.26803345,78.27431936)
\curveto(297.15803174,78.2543181)(297.06303184,78.26931809)(296.98303345,78.31931936)
\curveto(296.91303199,78.34931801)(296.84803205,78.39431796)(296.78803345,78.45431936)
\curveto(296.73803216,78.52431783)(296.68803221,78.58931777)(296.63803345,78.64931936)
\curveto(296.58803231,78.71931764)(296.51303239,78.77931758)(296.41303345,78.82931936)
\curveto(296.32303258,78.88931747)(296.23303267,78.93931742)(296.14303345,78.97931936)
\curveto(296.11303279,78.99931736)(296.05303285,79.02431733)(295.96303345,79.05431936)
\curveto(295.88303302,79.08431727)(295.81303309,79.08931727)(295.75303345,79.06931936)
\curveto(295.61303329,79.03931732)(295.52303338,78.97931738)(295.48303345,78.88931936)
\curveto(295.45303345,78.80931755)(295.43803346,78.71931764)(295.43803345,78.61931936)
\curveto(295.43803346,78.51931784)(295.41303349,78.43431792)(295.36303345,78.36431936)
\curveto(295.29303361,78.27431808)(295.15303375,78.22931813)(294.94303345,78.22931936)
\lineto(294.38803345,78.22931936)
\lineto(294.16303345,78.22931936)
\curveto(294.08303482,78.23931812)(294.01803488,78.2593181)(293.96803345,78.28931936)
\curveto(293.88803501,78.34931801)(293.84303506,78.41931794)(293.83303345,78.49931936)
\curveto(293.82303508,78.51931784)(293.81803508,78.53931782)(293.81803345,78.55931936)
\curveto(293.81803508,78.58931777)(293.81303509,78.61431774)(293.80303345,78.63431936)
}
}
{
\newrgbcolor{curcolor}{0 0 0}
\pscustom[linestyle=none,fillstyle=solid,fillcolor=curcolor]
{
}
}
{
\newrgbcolor{curcolor}{0 0 0}
\pscustom[linestyle=none,fillstyle=solid,fillcolor=curcolor]
{
\newpath
\moveto(284.83303345,89.26463186)
\curveto(284.82304408,89.95462722)(284.94304396,90.55462662)(285.19303345,91.06463186)
\curveto(285.44304346,91.58462559)(285.77804312,91.9796252)(286.19803345,92.24963186)
\curveto(286.27804262,92.29962488)(286.36804253,92.34462483)(286.46803345,92.38463186)
\curveto(286.55804234,92.42462475)(286.65304225,92.46962471)(286.75303345,92.51963186)
\curveto(286.85304205,92.55962462)(286.95304195,92.58962459)(287.05303345,92.60963186)
\curveto(287.15304175,92.62962455)(287.25804164,92.64962453)(287.36803345,92.66963186)
\curveto(287.41804148,92.68962449)(287.46304144,92.69462448)(287.50303345,92.68463186)
\curveto(287.54304136,92.6746245)(287.58804131,92.6796245)(287.63803345,92.69963186)
\curveto(287.68804121,92.70962447)(287.77304113,92.71462446)(287.89303345,92.71463186)
\curveto(288.0030409,92.71462446)(288.08804081,92.70962447)(288.14803345,92.69963186)
\curveto(288.20804069,92.6796245)(288.26804063,92.66962451)(288.32803345,92.66963186)
\curveto(288.38804051,92.6796245)(288.44804045,92.6746245)(288.50803345,92.65463186)
\curveto(288.64804025,92.61462456)(288.78304012,92.5796246)(288.91303345,92.54963186)
\curveto(289.04303986,92.51962466)(289.16803973,92.4796247)(289.28803345,92.42963186)
\curveto(289.42803947,92.36962481)(289.55303935,92.29962488)(289.66303345,92.21963186)
\curveto(289.77303913,92.14962503)(289.88303902,92.0746251)(289.99303345,91.99463186)
\lineto(290.05303345,91.93463186)
\curveto(290.07303883,91.92462525)(290.09303881,91.90962527)(290.11303345,91.88963186)
\curveto(290.27303863,91.76962541)(290.41803848,91.63462554)(290.54803345,91.48463186)
\curveto(290.67803822,91.33462584)(290.8030381,91.174626)(290.92303345,91.00463186)
\curveto(291.14303776,90.69462648)(291.34803755,90.39962678)(291.53803345,90.11963186)
\curveto(291.67803722,89.88962729)(291.81303709,89.65962752)(291.94303345,89.42963186)
\curveto(292.07303683,89.20962797)(292.20803669,88.98962819)(292.34803345,88.76963186)
\curveto(292.51803638,88.51962866)(292.6980362,88.2796289)(292.88803345,88.04963186)
\curveto(293.07803582,87.82962935)(293.3030356,87.63962954)(293.56303345,87.47963186)
\curveto(293.62303528,87.43962974)(293.68303522,87.40462977)(293.74303345,87.37463186)
\curveto(293.79303511,87.34462983)(293.85803504,87.31462986)(293.93803345,87.28463186)
\curveto(294.00803489,87.26462991)(294.06803483,87.25962992)(294.11803345,87.26963186)
\curveto(294.18803471,87.28962989)(294.24303466,87.32462985)(294.28303345,87.37463186)
\curveto(294.31303459,87.42462975)(294.33303457,87.48462969)(294.34303345,87.55463186)
\lineto(294.34303345,87.79463186)
\lineto(294.34303345,88.54463186)
\lineto(294.34303345,91.34963186)
\lineto(294.34303345,92.00963186)
\curveto(294.34303456,92.09962508)(294.34803455,92.18462499)(294.35803345,92.26463186)
\curveto(294.35803454,92.34462483)(294.37803452,92.40962477)(294.41803345,92.45963186)
\curveto(294.45803444,92.50962467)(294.53303437,92.54962463)(294.64303345,92.57963186)
\curveto(294.74303416,92.61962456)(294.84303406,92.62962455)(294.94303345,92.60963186)
\lineto(295.07803345,92.60963186)
\curveto(295.14803375,92.58962459)(295.20803369,92.56962461)(295.25803345,92.54963186)
\curveto(295.30803359,92.52962465)(295.34803355,92.49462468)(295.37803345,92.44463186)
\curveto(295.41803348,92.39462478)(295.43803346,92.32462485)(295.43803345,92.23463186)
\lineto(295.43803345,91.96463186)
\lineto(295.43803345,91.06463186)
\lineto(295.43803345,87.55463186)
\lineto(295.43803345,86.48963186)
\curveto(295.43803346,86.40963077)(295.44303346,86.31963086)(295.45303345,86.21963186)
\curveto(295.45303345,86.11963106)(295.44303346,86.03463114)(295.42303345,85.96463186)
\curveto(295.35303355,85.75463142)(295.17303373,85.68963149)(294.88303345,85.76963186)
\curveto(294.84303406,85.7796314)(294.80803409,85.7796314)(294.77803345,85.76963186)
\curveto(294.73803416,85.76963141)(294.69303421,85.7796314)(294.64303345,85.79963186)
\curveto(294.56303434,85.81963136)(294.47803442,85.83963134)(294.38803345,85.85963186)
\curveto(294.2980346,85.8796313)(294.21303469,85.90463127)(294.13303345,85.93463186)
\curveto(293.64303526,86.09463108)(293.22803567,86.29463088)(292.88803345,86.53463186)
\curveto(292.63803626,86.71463046)(292.41303649,86.91963026)(292.21303345,87.14963186)
\curveto(292.0030369,87.3796298)(291.80803709,87.61962956)(291.62803345,87.86963186)
\curveto(291.44803745,88.12962905)(291.27803762,88.39462878)(291.11803345,88.66463186)
\curveto(290.94803795,88.94462823)(290.77303813,89.21462796)(290.59303345,89.47463186)
\curveto(290.51303839,89.58462759)(290.43803846,89.68962749)(290.36803345,89.78963186)
\curveto(290.2980386,89.89962728)(290.22303868,90.00962717)(290.14303345,90.11963186)
\curveto(290.11303879,90.15962702)(290.08303882,90.19462698)(290.05303345,90.22463186)
\curveto(290.01303889,90.26462691)(289.98303892,90.30462687)(289.96303345,90.34463186)
\curveto(289.85303905,90.48462669)(289.72803917,90.60962657)(289.58803345,90.71963186)
\curveto(289.55803934,90.73962644)(289.53303937,90.76462641)(289.51303345,90.79463186)
\curveto(289.48303942,90.82462635)(289.45303945,90.84962633)(289.42303345,90.86963186)
\curveto(289.32303958,90.94962623)(289.22303968,91.01462616)(289.12303345,91.06463186)
\curveto(289.02303988,91.12462605)(288.91303999,91.179626)(288.79303345,91.22963186)
\curveto(288.72304018,91.25962592)(288.64804025,91.2796259)(288.56803345,91.28963186)
\lineto(288.32803345,91.34963186)
\lineto(288.23803345,91.34963186)
\curveto(288.20804069,91.35962582)(288.17804072,91.36462581)(288.14803345,91.36463186)
\curveto(288.07804082,91.38462579)(287.98304092,91.38962579)(287.86303345,91.37963186)
\curveto(287.73304117,91.3796258)(287.63304127,91.36962581)(287.56303345,91.34963186)
\curveto(287.48304142,91.32962585)(287.40804149,91.30962587)(287.33803345,91.28963186)
\curveto(287.25804164,91.2796259)(287.17804172,91.25962592)(287.09803345,91.22963186)
\curveto(286.85804204,91.11962606)(286.65804224,90.96962621)(286.49803345,90.77963186)
\curveto(286.32804257,90.59962658)(286.18804271,90.3796268)(286.07803345,90.11963186)
\curveto(286.05804284,90.04962713)(286.04304286,89.9796272)(286.03303345,89.90963186)
\curveto(286.01304289,89.83962734)(285.99304291,89.76462741)(285.97303345,89.68463186)
\curveto(285.95304295,89.60462757)(285.94304296,89.49462768)(285.94303345,89.35463186)
\curveto(285.94304296,89.22462795)(285.95304295,89.11962806)(285.97303345,89.03963186)
\curveto(285.98304292,88.9796282)(285.98804291,88.92462825)(285.98803345,88.87463186)
\curveto(285.98804291,88.82462835)(285.9980429,88.7746284)(286.01803345,88.72463186)
\curveto(286.05804284,88.62462855)(286.0980428,88.52962865)(286.13803345,88.43963186)
\curveto(286.17804272,88.35962882)(286.22304268,88.2796289)(286.27303345,88.19963186)
\curveto(286.29304261,88.16962901)(286.31804258,88.13962904)(286.34803345,88.10963186)
\curveto(286.37804252,88.08962909)(286.4030425,88.06462911)(286.42303345,88.03463186)
\lineto(286.49803345,87.95963186)
\curveto(286.51804238,87.92962925)(286.53804236,87.90462927)(286.55803345,87.88463186)
\lineto(286.76803345,87.73463186)
\curveto(286.82804207,87.69462948)(286.89304201,87.64962953)(286.96303345,87.59963186)
\curveto(287.05304185,87.53962964)(287.15804174,87.48962969)(287.27803345,87.44963186)
\curveto(287.38804151,87.41962976)(287.4980414,87.38462979)(287.60803345,87.34463186)
\curveto(287.71804118,87.30462987)(287.86304104,87.2796299)(288.04303345,87.26963186)
\curveto(288.21304069,87.25962992)(288.33804056,87.22962995)(288.41803345,87.17963186)
\curveto(288.4980404,87.12963005)(288.54304036,87.05463012)(288.55303345,86.95463186)
\curveto(288.56304034,86.85463032)(288.56804033,86.74463043)(288.56803345,86.62463186)
\curveto(288.56804033,86.58463059)(288.57304033,86.54463063)(288.58303345,86.50463186)
\curveto(288.58304032,86.46463071)(288.57804032,86.42963075)(288.56803345,86.39963186)
\curveto(288.54804035,86.34963083)(288.53804036,86.29963088)(288.53803345,86.24963186)
\curveto(288.53804036,86.20963097)(288.52804037,86.16963101)(288.50803345,86.12963186)
\curveto(288.44804045,86.03963114)(288.31304059,85.99463118)(288.10303345,85.99463186)
\lineto(287.98303345,85.99463186)
\curveto(287.92304098,86.00463117)(287.86304104,86.00963117)(287.80303345,86.00963186)
\curveto(287.73304117,86.01963116)(287.66804123,86.02963115)(287.60803345,86.03963186)
\curveto(287.4980414,86.05963112)(287.3980415,86.0796311)(287.30803345,86.09963186)
\curveto(287.20804169,86.11963106)(287.11304179,86.14963103)(287.02303345,86.18963186)
\curveto(286.95304195,86.20963097)(286.89304201,86.22963095)(286.84303345,86.24963186)
\lineto(286.66303345,86.30963186)
\curveto(286.4030425,86.42963075)(286.15804274,86.58463059)(285.92803345,86.77463186)
\curveto(285.6980432,86.9746302)(285.51304339,87.18962999)(285.37303345,87.41963186)
\curveto(285.29304361,87.52962965)(285.22804367,87.64462953)(285.17803345,87.76463186)
\lineto(285.02803345,88.15463186)
\curveto(284.97804392,88.26462891)(284.94804395,88.3796288)(284.93803345,88.49963186)
\curveto(284.91804398,88.61962856)(284.89304401,88.74462843)(284.86303345,88.87463186)
\curveto(284.86304404,88.94462823)(284.86304404,89.00962817)(284.86303345,89.06963186)
\curveto(284.85304405,89.12962805)(284.84304406,89.19462798)(284.83303345,89.26463186)
}
}
{
\newrgbcolor{curcolor}{0 0 0}
\pscustom[linestyle=none,fillstyle=solid,fillcolor=curcolor]
{
\newpath
\moveto(290.35303345,101.36424123)
\lineto(290.60803345,101.36424123)
\curveto(290.68803821,101.37423353)(290.76303814,101.36923353)(290.83303345,101.34924123)
\lineto(291.07303345,101.34924123)
\lineto(291.23803345,101.34924123)
\curveto(291.33803756,101.32923357)(291.44303746,101.31923358)(291.55303345,101.31924123)
\curveto(291.65303725,101.31923358)(291.75303715,101.30923359)(291.85303345,101.28924123)
\lineto(292.00303345,101.28924123)
\curveto(292.14303676,101.25923364)(292.28303662,101.23923366)(292.42303345,101.22924123)
\curveto(292.55303635,101.21923368)(292.68303622,101.19423371)(292.81303345,101.15424123)
\curveto(292.89303601,101.13423377)(292.97803592,101.11423379)(293.06803345,101.09424123)
\lineto(293.30803345,101.03424123)
\lineto(293.60803345,100.91424123)
\curveto(293.6980352,100.88423402)(293.78803511,100.84923405)(293.87803345,100.80924123)
\curveto(294.0980348,100.70923419)(294.31303459,100.57423433)(294.52303345,100.40424123)
\curveto(294.73303417,100.24423466)(294.903034,100.06923483)(295.03303345,99.87924123)
\curveto(295.07303383,99.82923507)(295.11303379,99.76923513)(295.15303345,99.69924123)
\curveto(295.18303372,99.63923526)(295.21803368,99.57923532)(295.25803345,99.51924123)
\curveto(295.30803359,99.43923546)(295.34803355,99.34423556)(295.37803345,99.23424123)
\curveto(295.40803349,99.12423578)(295.43803346,99.01923588)(295.46803345,98.91924123)
\curveto(295.50803339,98.80923609)(295.53303337,98.6992362)(295.54303345,98.58924123)
\curveto(295.55303335,98.47923642)(295.56803333,98.36423654)(295.58803345,98.24424123)
\curveto(295.5980333,98.2042367)(295.5980333,98.15923674)(295.58803345,98.10924123)
\curveto(295.58803331,98.06923683)(295.59303331,98.02923687)(295.60303345,97.98924123)
\curveto(295.61303329,97.94923695)(295.61803328,97.89423701)(295.61803345,97.82424123)
\curveto(295.61803328,97.75423715)(295.61303329,97.7042372)(295.60303345,97.67424123)
\curveto(295.58303332,97.62423728)(295.57803332,97.57923732)(295.58803345,97.53924123)
\curveto(295.5980333,97.4992374)(295.5980333,97.46423744)(295.58803345,97.43424123)
\lineto(295.58803345,97.34424123)
\curveto(295.56803333,97.28423762)(295.55303335,97.21923768)(295.54303345,97.14924123)
\curveto(295.54303336,97.08923781)(295.53803336,97.02423788)(295.52803345,96.95424123)
\curveto(295.47803342,96.78423812)(295.42803347,96.62423828)(295.37803345,96.47424123)
\curveto(295.32803357,96.32423858)(295.26303364,96.17923872)(295.18303345,96.03924123)
\curveto(295.14303376,95.98923891)(295.11303379,95.93423897)(295.09303345,95.87424123)
\curveto(295.06303384,95.82423908)(295.02803387,95.77423913)(294.98803345,95.72424123)
\curveto(294.80803409,95.48423942)(294.58803431,95.28423962)(294.32803345,95.12424123)
\curveto(294.06803483,94.96423994)(293.78303512,94.82424008)(293.47303345,94.70424123)
\curveto(293.33303557,94.64424026)(293.19303571,94.5992403)(293.05303345,94.56924123)
\curveto(292.903036,94.53924036)(292.74803615,94.5042404)(292.58803345,94.46424123)
\curveto(292.47803642,94.44424046)(292.36803653,94.42924047)(292.25803345,94.41924123)
\curveto(292.14803675,94.40924049)(292.03803686,94.39424051)(291.92803345,94.37424123)
\curveto(291.88803701,94.36424054)(291.84803705,94.35924054)(291.80803345,94.35924123)
\curveto(291.76803713,94.36924053)(291.72803717,94.36924053)(291.68803345,94.35924123)
\curveto(291.63803726,94.34924055)(291.58803731,94.34424056)(291.53803345,94.34424123)
\lineto(291.37303345,94.34424123)
\curveto(291.32303758,94.32424058)(291.27303763,94.31924058)(291.22303345,94.32924123)
\curveto(291.16303774,94.33924056)(291.10803779,94.33924056)(291.05803345,94.32924123)
\curveto(291.01803788,94.31924058)(290.97303793,94.31924058)(290.92303345,94.32924123)
\curveto(290.87303803,94.33924056)(290.82303808,94.33424057)(290.77303345,94.31424123)
\curveto(290.7030382,94.29424061)(290.62803827,94.28924061)(290.54803345,94.29924123)
\curveto(290.45803844,94.30924059)(290.37303853,94.31424059)(290.29303345,94.31424123)
\curveto(290.2030387,94.31424059)(290.1030388,94.30924059)(289.99303345,94.29924123)
\curveto(289.87303903,94.28924061)(289.77303913,94.29424061)(289.69303345,94.31424123)
\lineto(289.40803345,94.31424123)
\lineto(288.77803345,94.35924123)
\curveto(288.67804022,94.36924053)(288.58304032,94.37924052)(288.49303345,94.38924123)
\lineto(288.19303345,94.41924123)
\curveto(288.14304076,94.43924046)(288.09304081,94.44424046)(288.04303345,94.43424123)
\curveto(287.98304092,94.43424047)(287.92804097,94.44424046)(287.87803345,94.46424123)
\curveto(287.70804119,94.51424039)(287.54304136,94.55424035)(287.38303345,94.58424123)
\curveto(287.21304169,94.61424029)(287.05304185,94.66424024)(286.90303345,94.73424123)
\curveto(286.44304246,94.92423998)(286.06804283,95.14423976)(285.77803345,95.39424123)
\curveto(285.48804341,95.65423925)(285.24304366,96.01423889)(285.04303345,96.47424123)
\curveto(284.99304391,96.6042383)(284.95804394,96.73423817)(284.93803345,96.86424123)
\curveto(284.91804398,97.0042379)(284.89304401,97.14423776)(284.86303345,97.28424123)
\curveto(284.85304405,97.35423755)(284.84804405,97.41923748)(284.84803345,97.47924123)
\curveto(284.84804405,97.53923736)(284.84304406,97.6042373)(284.83303345,97.67424123)
\curveto(284.81304409,98.5042364)(284.96304394,99.17423573)(285.28303345,99.68424123)
\curveto(285.59304331,100.19423471)(286.03304287,100.57423433)(286.60303345,100.82424123)
\curveto(286.72304218,100.87423403)(286.84804205,100.91923398)(286.97803345,100.95924123)
\curveto(287.10804179,100.9992339)(287.24304166,101.04423386)(287.38303345,101.09424123)
\curveto(287.46304144,101.11423379)(287.54804135,101.12923377)(287.63803345,101.13924123)
\lineto(287.87803345,101.19924123)
\curveto(287.98804091,101.22923367)(288.0980408,101.24423366)(288.20803345,101.24424123)
\curveto(288.31804058,101.25423365)(288.42804047,101.26923363)(288.53803345,101.28924123)
\curveto(288.58804031,101.30923359)(288.63304027,101.31423359)(288.67303345,101.30424123)
\curveto(288.71304019,101.3042336)(288.75304015,101.30923359)(288.79303345,101.31924123)
\curveto(288.84304006,101.32923357)(288.89804,101.32923357)(288.95803345,101.31924123)
\curveto(289.00803989,101.31923358)(289.05803984,101.32423358)(289.10803345,101.33424123)
\lineto(289.24303345,101.33424123)
\curveto(289.3030396,101.35423355)(289.37303953,101.35423355)(289.45303345,101.33424123)
\curveto(289.52303938,101.32423358)(289.58803931,101.32923357)(289.64803345,101.34924123)
\curveto(289.67803922,101.35923354)(289.71803918,101.36423354)(289.76803345,101.36424123)
\lineto(289.88803345,101.36424123)
\lineto(290.35303345,101.36424123)
\moveto(292.67803345,99.81924123)
\curveto(292.35803654,99.91923498)(291.99303691,99.97923492)(291.58303345,99.99924123)
\curveto(291.17303773,100.01923488)(290.76303814,100.02923487)(290.35303345,100.02924123)
\curveto(289.92303898,100.02923487)(289.5030394,100.01923488)(289.09303345,99.99924123)
\curveto(288.68304022,99.97923492)(288.2980406,99.93423497)(287.93803345,99.86424123)
\curveto(287.57804132,99.79423511)(287.25804164,99.68423522)(286.97803345,99.53424123)
\curveto(286.68804221,99.39423551)(286.45304245,99.1992357)(286.27303345,98.94924123)
\curveto(286.16304274,98.78923611)(286.08304282,98.60923629)(286.03303345,98.40924123)
\curveto(285.97304293,98.20923669)(285.94304296,97.96423694)(285.94303345,97.67424123)
\curveto(285.96304294,97.65423725)(285.97304293,97.61923728)(285.97303345,97.56924123)
\curveto(285.96304294,97.51923738)(285.96304294,97.47923742)(285.97303345,97.44924123)
\curveto(285.99304291,97.36923753)(286.01304289,97.29423761)(286.03303345,97.22424123)
\curveto(286.04304286,97.16423774)(286.06304284,97.0992378)(286.09303345,97.02924123)
\curveto(286.21304269,96.75923814)(286.38304252,96.53923836)(286.60303345,96.36924123)
\curveto(286.81304209,96.20923869)(287.05804184,96.07423883)(287.33803345,95.96424123)
\curveto(287.44804145,95.91423899)(287.56804133,95.87423903)(287.69803345,95.84424123)
\curveto(287.81804108,95.82423908)(287.94304096,95.7992391)(288.07303345,95.76924123)
\curveto(288.12304078,95.74923915)(288.17804072,95.73923916)(288.23803345,95.73924123)
\curveto(288.28804061,95.73923916)(288.33804056,95.73423917)(288.38803345,95.72424123)
\curveto(288.47804042,95.71423919)(288.57304033,95.7042392)(288.67303345,95.69424123)
\curveto(288.76304014,95.68423922)(288.85804004,95.67423923)(288.95803345,95.66424123)
\curveto(289.03803986,95.66423924)(289.12303978,95.65923924)(289.21303345,95.64924123)
\lineto(289.45303345,95.64924123)
\lineto(289.63303345,95.64924123)
\curveto(289.66303924,95.63923926)(289.6980392,95.63423927)(289.73803345,95.63424123)
\lineto(289.87303345,95.63424123)
\lineto(290.32303345,95.63424123)
\curveto(290.4030385,95.63423927)(290.48803841,95.62923927)(290.57803345,95.61924123)
\curveto(290.65803824,95.61923928)(290.73303817,95.62923927)(290.80303345,95.64924123)
\lineto(291.07303345,95.64924123)
\curveto(291.09303781,95.64923925)(291.12303778,95.64423926)(291.16303345,95.63424123)
\curveto(291.19303771,95.63423927)(291.21803768,95.63923926)(291.23803345,95.64924123)
\curveto(291.33803756,95.65923924)(291.43803746,95.66423924)(291.53803345,95.66424123)
\curveto(291.62803727,95.67423923)(291.72803717,95.68423922)(291.83803345,95.69424123)
\curveto(291.95803694,95.72423918)(292.08303682,95.73923916)(292.21303345,95.73924123)
\curveto(292.33303657,95.74923915)(292.44803645,95.77423913)(292.55803345,95.81424123)
\curveto(292.85803604,95.89423901)(293.12303578,95.97923892)(293.35303345,96.06924123)
\curveto(293.58303532,96.16923873)(293.7980351,96.31423859)(293.99803345,96.50424123)
\curveto(294.1980347,96.71423819)(294.34803455,96.97923792)(294.44803345,97.29924123)
\curveto(294.46803443,97.33923756)(294.47803442,97.37423753)(294.47803345,97.40424123)
\curveto(294.46803443,97.44423746)(294.47303443,97.48923741)(294.49303345,97.53924123)
\curveto(294.5030344,97.57923732)(294.51303439,97.64923725)(294.52303345,97.74924123)
\curveto(294.53303437,97.85923704)(294.52803437,97.94423696)(294.50803345,98.00424123)
\curveto(294.48803441,98.07423683)(294.47803442,98.14423676)(294.47803345,98.21424123)
\curveto(294.46803443,98.28423662)(294.45303445,98.34923655)(294.43303345,98.40924123)
\curveto(294.37303453,98.60923629)(294.28803461,98.78923611)(294.17803345,98.94924123)
\curveto(294.15803474,98.97923592)(294.13803476,99.0042359)(294.11803345,99.02424123)
\lineto(294.05803345,99.08424123)
\curveto(294.03803486,99.12423578)(293.9980349,99.17423573)(293.93803345,99.23424123)
\curveto(293.7980351,99.33423557)(293.66803523,99.41923548)(293.54803345,99.48924123)
\curveto(293.42803547,99.55923534)(293.28303562,99.62923527)(293.11303345,99.69924123)
\curveto(293.04303586,99.72923517)(292.97303593,99.74923515)(292.90303345,99.75924123)
\curveto(292.83303607,99.77923512)(292.75803614,99.7992351)(292.67803345,99.81924123)
}
}
{
\newrgbcolor{curcolor}{0 0 0}
\pscustom[linestyle=none,fillstyle=solid,fillcolor=curcolor]
{
\newpath
\moveto(284.83303345,106.77385061)
\curveto(284.83304407,106.87384575)(284.84304406,106.96884566)(284.86303345,107.05885061)
\curveto(284.87304403,107.14884548)(284.903044,107.21384541)(284.95303345,107.25385061)
\curveto(285.03304387,107.31384531)(285.13804376,107.34384528)(285.26803345,107.34385061)
\lineto(285.65803345,107.34385061)
\lineto(287.15803345,107.34385061)
\lineto(293.54803345,107.34385061)
\lineto(294.71803345,107.34385061)
\lineto(295.03303345,107.34385061)
\curveto(295.13303377,107.35384527)(295.21303369,107.33884529)(295.27303345,107.29885061)
\curveto(295.35303355,107.24884538)(295.4030335,107.17384545)(295.42303345,107.07385061)
\curveto(295.43303347,106.98384564)(295.43803346,106.87384575)(295.43803345,106.74385061)
\lineto(295.43803345,106.51885061)
\curveto(295.41803348,106.43884619)(295.4030335,106.36884626)(295.39303345,106.30885061)
\curveto(295.37303353,106.24884638)(295.33303357,106.19884643)(295.27303345,106.15885061)
\curveto(295.21303369,106.11884651)(295.13803376,106.09884653)(295.04803345,106.09885061)
\lineto(294.74803345,106.09885061)
\lineto(293.65303345,106.09885061)
\lineto(288.31303345,106.09885061)
\curveto(288.22304068,106.07884655)(288.14804075,106.06384656)(288.08803345,106.05385061)
\curveto(288.01804088,106.05384657)(287.95804094,106.0238466)(287.90803345,105.96385061)
\curveto(287.85804104,105.89384673)(287.83304107,105.80384682)(287.83303345,105.69385061)
\curveto(287.82304108,105.59384703)(287.81804108,105.48384714)(287.81803345,105.36385061)
\lineto(287.81803345,104.22385061)
\lineto(287.81803345,103.72885061)
\curveto(287.80804109,103.56884906)(287.74804115,103.45884917)(287.63803345,103.39885061)
\curveto(287.60804129,103.37884925)(287.57804132,103.36884926)(287.54803345,103.36885061)
\curveto(287.50804139,103.36884926)(287.46304144,103.36384926)(287.41303345,103.35385061)
\curveto(287.29304161,103.33384929)(287.18304172,103.33884929)(287.08303345,103.36885061)
\curveto(286.98304192,103.40884922)(286.91304199,103.46384916)(286.87303345,103.53385061)
\curveto(286.82304208,103.61384901)(286.7980421,103.73384889)(286.79803345,103.89385061)
\curveto(286.7980421,104.05384857)(286.78304212,104.18884844)(286.75303345,104.29885061)
\curveto(286.74304216,104.34884828)(286.73804216,104.40384822)(286.73803345,104.46385061)
\curveto(286.72804217,104.5238481)(286.71304219,104.58384804)(286.69303345,104.64385061)
\curveto(286.64304226,104.79384783)(286.59304231,104.93884769)(286.54303345,105.07885061)
\curveto(286.48304242,105.21884741)(286.41304249,105.35384727)(286.33303345,105.48385061)
\curveto(286.24304266,105.623847)(286.13804276,105.74384688)(286.01803345,105.84385061)
\curveto(285.898043,105.94384668)(285.76804313,106.03884659)(285.62803345,106.12885061)
\curveto(285.52804337,106.18884644)(285.41804348,106.23384639)(285.29803345,106.26385061)
\curveto(285.17804372,106.30384632)(285.07304383,106.35384627)(284.98303345,106.41385061)
\curveto(284.92304398,106.46384616)(284.88304402,106.53384609)(284.86303345,106.62385061)
\curveto(284.85304405,106.64384598)(284.84804405,106.66884596)(284.84803345,106.69885061)
\curveto(284.84804405,106.7288459)(284.84304406,106.75384587)(284.83303345,106.77385061)
}
}
{
\newrgbcolor{curcolor}{0 0 0}
\pscustom[linestyle=none,fillstyle=solid,fillcolor=curcolor]
{
\newpath
\moveto(284.83303345,115.12345998)
\curveto(284.83304407,115.22345513)(284.84304406,115.31845503)(284.86303345,115.40845998)
\curveto(284.87304403,115.49845485)(284.903044,115.56345479)(284.95303345,115.60345998)
\curveto(285.03304387,115.66345469)(285.13804376,115.69345466)(285.26803345,115.69345998)
\lineto(285.65803345,115.69345998)
\lineto(287.15803345,115.69345998)
\lineto(293.54803345,115.69345998)
\lineto(294.71803345,115.69345998)
\lineto(295.03303345,115.69345998)
\curveto(295.13303377,115.70345465)(295.21303369,115.68845466)(295.27303345,115.64845998)
\curveto(295.35303355,115.59845475)(295.4030335,115.52345483)(295.42303345,115.42345998)
\curveto(295.43303347,115.33345502)(295.43803346,115.22345513)(295.43803345,115.09345998)
\lineto(295.43803345,114.86845998)
\curveto(295.41803348,114.78845556)(295.4030335,114.71845563)(295.39303345,114.65845998)
\curveto(295.37303353,114.59845575)(295.33303357,114.5484558)(295.27303345,114.50845998)
\curveto(295.21303369,114.46845588)(295.13803376,114.4484559)(295.04803345,114.44845998)
\lineto(294.74803345,114.44845998)
\lineto(293.65303345,114.44845998)
\lineto(288.31303345,114.44845998)
\curveto(288.22304068,114.42845592)(288.14804075,114.41345594)(288.08803345,114.40345998)
\curveto(288.01804088,114.40345595)(287.95804094,114.37345598)(287.90803345,114.31345998)
\curveto(287.85804104,114.24345611)(287.83304107,114.1534562)(287.83303345,114.04345998)
\curveto(287.82304108,113.94345641)(287.81804108,113.83345652)(287.81803345,113.71345998)
\lineto(287.81803345,112.57345998)
\lineto(287.81803345,112.07845998)
\curveto(287.80804109,111.91845843)(287.74804115,111.80845854)(287.63803345,111.74845998)
\curveto(287.60804129,111.72845862)(287.57804132,111.71845863)(287.54803345,111.71845998)
\curveto(287.50804139,111.71845863)(287.46304144,111.71345864)(287.41303345,111.70345998)
\curveto(287.29304161,111.68345867)(287.18304172,111.68845866)(287.08303345,111.71845998)
\curveto(286.98304192,111.75845859)(286.91304199,111.81345854)(286.87303345,111.88345998)
\curveto(286.82304208,111.96345839)(286.7980421,112.08345827)(286.79803345,112.24345998)
\curveto(286.7980421,112.40345795)(286.78304212,112.53845781)(286.75303345,112.64845998)
\curveto(286.74304216,112.69845765)(286.73804216,112.7534576)(286.73803345,112.81345998)
\curveto(286.72804217,112.87345748)(286.71304219,112.93345742)(286.69303345,112.99345998)
\curveto(286.64304226,113.14345721)(286.59304231,113.28845706)(286.54303345,113.42845998)
\curveto(286.48304242,113.56845678)(286.41304249,113.70345665)(286.33303345,113.83345998)
\curveto(286.24304266,113.97345638)(286.13804276,114.09345626)(286.01803345,114.19345998)
\curveto(285.898043,114.29345606)(285.76804313,114.38845596)(285.62803345,114.47845998)
\curveto(285.52804337,114.53845581)(285.41804348,114.58345577)(285.29803345,114.61345998)
\curveto(285.17804372,114.6534557)(285.07304383,114.70345565)(284.98303345,114.76345998)
\curveto(284.92304398,114.81345554)(284.88304402,114.88345547)(284.86303345,114.97345998)
\curveto(284.85304405,114.99345536)(284.84804405,115.01845533)(284.84803345,115.04845998)
\curveto(284.84804405,115.07845527)(284.84304406,115.10345525)(284.83303345,115.12345998)
}
}
{
\newrgbcolor{curcolor}{0 0 0}
\pscustom[linestyle=none,fillstyle=solid,fillcolor=curcolor]
{
\newpath
\moveto(305.66937988,31.67142873)
\lineto(305.66937988,32.58642873)
\curveto(305.66939058,32.68642608)(305.66939058,32.78142599)(305.66937988,32.87142873)
\curveto(305.66939058,32.96142581)(305.68939056,33.03642573)(305.72937988,33.09642873)
\curveto(305.78939046,33.18642558)(305.86939038,33.24642552)(305.96937988,33.27642873)
\curveto(306.06939018,33.31642545)(306.17439007,33.36142541)(306.28437988,33.41142873)
\curveto(306.47438977,33.49142528)(306.66438958,33.56142521)(306.85437988,33.62142873)
\curveto(307.0443892,33.69142508)(307.23438901,33.766425)(307.42437988,33.84642873)
\curveto(307.60438864,33.91642485)(307.78938846,33.98142479)(307.97937988,34.04142873)
\curveto(308.15938809,34.10142467)(308.33938791,34.1714246)(308.51937988,34.25142873)
\curveto(308.65938759,34.31142446)(308.80438744,34.3664244)(308.95437988,34.41642873)
\curveto(309.10438714,34.4664243)(309.249387,34.52142425)(309.38937988,34.58142873)
\curveto(309.83938641,34.76142401)(310.29438595,34.93142384)(310.75437988,35.09142873)
\curveto(311.20438504,35.25142352)(311.65438459,35.42142335)(312.10437988,35.60142873)
\curveto(312.15438409,35.62142315)(312.20438404,35.63642313)(312.25437988,35.64642873)
\lineto(312.40437988,35.70642873)
\curveto(312.62438362,35.79642297)(312.8493834,35.88142289)(313.07937988,35.96142873)
\curveto(313.29938295,36.04142273)(313.51938273,36.12642264)(313.73937988,36.21642873)
\curveto(313.82938242,36.25642251)(313.93938231,36.29642247)(314.06937988,36.33642873)
\curveto(314.18938206,36.37642239)(314.25938199,36.44142233)(314.27937988,36.53142873)
\curveto(314.28938196,36.5714222)(314.28938196,36.60142217)(314.27937988,36.62142873)
\lineto(314.21937988,36.68142873)
\curveto(314.16938208,36.73142204)(314.11438213,36.766422)(314.05437988,36.78642873)
\curveto(313.99438225,36.81642195)(313.92938232,36.84642192)(313.85937988,36.87642873)
\lineto(313.22937988,37.11642873)
\curveto(313.00938324,37.19642157)(312.79438345,37.27642149)(312.58437988,37.35642873)
\lineto(312.43437988,37.41642873)
\lineto(312.25437988,37.47642873)
\curveto(312.06438418,37.55642121)(311.87438437,37.62642114)(311.68437988,37.68642873)
\curveto(311.48438476,37.75642101)(311.28438496,37.83142094)(311.08437988,37.91142873)
\curveto(310.50438574,38.15142062)(309.91938633,38.3714204)(309.32937988,38.57142873)
\curveto(308.73938751,38.78141999)(308.15438809,39.00641976)(307.57437988,39.24642873)
\curveto(307.37438887,39.32641944)(307.16938908,39.40141937)(306.95937988,39.47142873)
\curveto(306.7493895,39.55141922)(306.5443897,39.63141914)(306.34437988,39.71142873)
\curveto(306.26438998,39.75141902)(306.16439008,39.78641898)(306.04437988,39.81642873)
\curveto(305.92439032,39.85641891)(305.83939041,39.91141886)(305.78937988,39.98142873)
\curveto(305.7493905,40.04141873)(305.71939053,40.11641865)(305.69937988,40.20642873)
\curveto(305.67939057,40.30641846)(305.66939058,40.41641835)(305.66937988,40.53642873)
\curveto(305.65939059,40.65641811)(305.65939059,40.77641799)(305.66937988,40.89642873)
\curveto(305.66939058,41.01641775)(305.66939058,41.12641764)(305.66937988,41.22642873)
\curveto(305.66939058,41.31641745)(305.66939058,41.40641736)(305.66937988,41.49642873)
\curveto(305.66939058,41.59641717)(305.68939056,41.6714171)(305.72937988,41.72142873)
\curveto(305.77939047,41.81141696)(305.86939038,41.86141691)(305.99937988,41.87142873)
\curveto(306.12939012,41.88141689)(306.26938998,41.88641688)(306.41937988,41.88642873)
\lineto(308.06937988,41.88642873)
\lineto(314.33937988,41.88642873)
\lineto(315.59937988,41.88642873)
\curveto(315.70938054,41.88641688)(315.81938043,41.88641688)(315.92937988,41.88642873)
\curveto(316.03938021,41.89641687)(316.12438012,41.87641689)(316.18437988,41.82642873)
\curveto(316.24438,41.79641697)(316.28437996,41.75141702)(316.30437988,41.69142873)
\curveto(316.31437993,41.63141714)(316.32937992,41.56141721)(316.34937988,41.48142873)
\lineto(316.34937988,41.24142873)
\lineto(316.34937988,40.88142873)
\curveto(316.33937991,40.771418)(316.29437995,40.69141808)(316.21437988,40.64142873)
\curveto(316.18438006,40.62141815)(316.15438009,40.60641816)(316.12437988,40.59642873)
\curveto(316.08438016,40.59641817)(316.03938021,40.58641818)(315.98937988,40.56642873)
\lineto(315.82437988,40.56642873)
\curveto(315.76438048,40.55641821)(315.69438055,40.55141822)(315.61437988,40.55142873)
\curveto(315.53438071,40.56141821)(315.45938079,40.5664182)(315.38937988,40.56642873)
\lineto(314.54937988,40.56642873)
\lineto(310.12437988,40.56642873)
\curveto(309.87438637,40.5664182)(309.62438662,40.5664182)(309.37437988,40.56642873)
\curveto(309.11438713,40.5664182)(308.86438738,40.56141821)(308.62437988,40.55142873)
\curveto(308.52438772,40.55141822)(308.41438783,40.54641822)(308.29437988,40.53642873)
\curveto(308.17438807,40.52641824)(308.11438813,40.4714183)(308.11437988,40.37142873)
\lineto(308.12937988,40.37142873)
\curveto(308.1493881,40.30141847)(308.21438803,40.24141853)(308.32437988,40.19142873)
\curveto(308.43438781,40.15141862)(308.52938772,40.11641865)(308.60937988,40.08642873)
\curveto(308.77938747,40.01641875)(308.95438729,39.95141882)(309.13437988,39.89142873)
\curveto(309.30438694,39.83141894)(309.47438677,39.76141901)(309.64437988,39.68142873)
\curveto(309.69438655,39.66141911)(309.73938651,39.64641912)(309.77937988,39.63642873)
\curveto(309.81938643,39.62641914)(309.86438638,39.61141916)(309.91437988,39.59142873)
\curveto(310.09438615,39.51141926)(310.27938597,39.44141933)(310.46937988,39.38142873)
\curveto(310.6493856,39.33141944)(310.82938542,39.2664195)(311.00937988,39.18642873)
\curveto(311.15938509,39.11641965)(311.31438493,39.05641971)(311.47437988,39.00642873)
\curveto(311.62438462,38.95641981)(311.77438447,38.90141987)(311.92437988,38.84142873)
\curveto(312.39438385,38.64142013)(312.86938338,38.46142031)(313.34937988,38.30142873)
\curveto(313.81938243,38.14142063)(314.28438196,37.9664208)(314.74437988,37.77642873)
\curveto(314.92438132,37.69642107)(315.10438114,37.62642114)(315.28437988,37.56642873)
\curveto(315.46438078,37.50642126)(315.6443806,37.44142133)(315.82437988,37.37142873)
\curveto(315.93438031,37.32142145)(316.03938021,37.2714215)(316.13937988,37.22142873)
\curveto(316.22938002,37.18142159)(316.29437995,37.09642167)(316.33437988,36.96642873)
\curveto(316.3443799,36.94642182)(316.3493799,36.92142185)(316.34937988,36.89142873)
\curveto(316.33937991,36.8714219)(316.33937991,36.84642192)(316.34937988,36.81642873)
\curveto(316.35937989,36.78642198)(316.36437988,36.75142202)(316.36437988,36.71142873)
\curveto(316.35437989,36.6714221)(316.3493799,36.63142214)(316.34937988,36.59142873)
\lineto(316.34937988,36.29142873)
\curveto(316.3493799,36.19142258)(316.32437992,36.11142266)(316.27437988,36.05142873)
\curveto(316.22438002,35.9714228)(316.15438009,35.91142286)(316.06437988,35.87142873)
\curveto(315.96438028,35.84142293)(315.86438038,35.80142297)(315.76437988,35.75142873)
\curveto(315.56438068,35.6714231)(315.35938089,35.59142318)(315.14937988,35.51142873)
\curveto(314.92938132,35.44142333)(314.71938153,35.3664234)(314.51937988,35.28642873)
\curveto(314.33938191,35.20642356)(314.15938209,35.13642363)(313.97937988,35.07642873)
\curveto(313.78938246,35.02642374)(313.60438264,34.96142381)(313.42437988,34.88142873)
\curveto(312.86438338,34.65142412)(312.29938395,34.43642433)(311.72937988,34.23642873)
\curveto(311.15938509,34.03642473)(310.59438565,33.82142495)(310.03437988,33.59142873)
\lineto(309.40437988,33.35142873)
\curveto(309.18438706,33.28142549)(308.97438727,33.20642556)(308.77437988,33.12642873)
\curveto(308.66438758,33.07642569)(308.55938769,33.03142574)(308.45937988,32.99142873)
\curveto(308.3493879,32.96142581)(308.25438799,32.91142586)(308.17437988,32.84142873)
\curveto(308.15438809,32.83142594)(308.1443881,32.82142595)(308.14437988,32.81142873)
\lineto(308.11437988,32.78142873)
\lineto(308.11437988,32.70642873)
\lineto(308.14437988,32.67642873)
\curveto(308.1443881,32.6664261)(308.1493881,32.65642611)(308.15937988,32.64642873)
\curveto(308.20938804,32.62642614)(308.26438798,32.61642615)(308.32437988,32.61642873)
\curveto(308.38438786,32.61642615)(308.4443878,32.60642616)(308.50437988,32.58642873)
\lineto(308.66937988,32.58642873)
\curveto(308.72938752,32.5664262)(308.79438745,32.56142621)(308.86437988,32.57142873)
\curveto(308.93438731,32.58142619)(309.00438724,32.58642618)(309.07437988,32.58642873)
\lineto(309.88437988,32.58642873)
\lineto(314.44437988,32.58642873)
\lineto(315.62937988,32.58642873)
\curveto(315.73938051,32.58642618)(315.8493804,32.58142619)(315.95937988,32.57142873)
\curveto(316.06938018,32.5714262)(316.15438009,32.54642622)(316.21437988,32.49642873)
\curveto(316.29437995,32.44642632)(316.33937991,32.35642641)(316.34937988,32.22642873)
\lineto(316.34937988,31.83642873)
\lineto(316.34937988,31.64142873)
\curveto(316.3493799,31.59142718)(316.33937991,31.54142723)(316.31937988,31.49142873)
\curveto(316.27937997,31.36142741)(316.19438005,31.28642748)(316.06437988,31.26642873)
\curveto(315.93438031,31.25642751)(315.78438046,31.25142752)(315.61437988,31.25142873)
\lineto(313.87437988,31.25142873)
\lineto(307.87437988,31.25142873)
\lineto(306.46437988,31.25142873)
\curveto(306.35438989,31.25142752)(306.23939001,31.24642752)(306.11937988,31.23642873)
\curveto(305.99939025,31.23642753)(305.90439034,31.26142751)(305.83437988,31.31142873)
\curveto(305.77439047,31.35142742)(305.72439052,31.42642734)(305.68437988,31.53642873)
\curveto(305.67439057,31.55642721)(305.67439057,31.57642719)(305.68437988,31.59642873)
\curveto(305.68439056,31.62642714)(305.67939057,31.65142712)(305.66937988,31.67142873)
}
}
{
\newrgbcolor{curcolor}{0 0 0}
\pscustom[linestyle=none,fillstyle=solid,fillcolor=curcolor]
{
\newpath
\moveto(315.79437988,50.87353811)
\curveto(315.95438029,50.90353028)(316.08938016,50.88853029)(316.19937988,50.82853811)
\curveto(316.29937995,50.76853041)(316.37437987,50.68853049)(316.42437988,50.58853811)
\curveto(316.4443798,50.53853064)(316.45437979,50.4835307)(316.45437988,50.42353811)
\curveto(316.45437979,50.37353081)(316.46437978,50.31853086)(316.48437988,50.25853811)
\curveto(316.53437971,50.03853114)(316.51937973,49.81853136)(316.43937988,49.59853811)
\curveto(316.36937988,49.38853179)(316.27937997,49.24353194)(316.16937988,49.16353811)
\curveto(316.09938015,49.11353207)(316.01938023,49.06853211)(315.92937988,49.02853811)
\curveto(315.82938042,48.98853219)(315.7493805,48.93853224)(315.68937988,48.87853811)
\curveto(315.66938058,48.85853232)(315.6493806,48.83353235)(315.62937988,48.80353811)
\curveto(315.60938064,48.7835324)(315.60438064,48.75353243)(315.61437988,48.71353811)
\curveto(315.6443806,48.60353258)(315.69938055,48.49853268)(315.77937988,48.39853811)
\curveto(315.85938039,48.30853287)(315.92938032,48.21853296)(315.98937988,48.12853811)
\curveto(316.06938018,47.99853318)(316.1443801,47.85853332)(316.21437988,47.70853811)
\curveto(316.27437997,47.55853362)(316.32937992,47.39853378)(316.37937988,47.22853811)
\curveto(316.40937984,47.12853405)(316.42937982,47.01853416)(316.43937988,46.89853811)
\curveto(316.4493798,46.78853439)(316.46437978,46.6785345)(316.48437988,46.56853811)
\curveto(316.49437975,46.51853466)(316.49937975,46.47353471)(316.49937988,46.43353811)
\lineto(316.49937988,46.32853811)
\curveto(316.51937973,46.21853496)(316.51937973,46.11353507)(316.49937988,46.01353811)
\lineto(316.49937988,45.87853811)
\curveto(316.48937976,45.82853535)(316.48437976,45.7785354)(316.48437988,45.72853811)
\curveto(316.48437976,45.6785355)(316.47437977,45.63353555)(316.45437988,45.59353811)
\curveto(316.4443798,45.55353563)(316.43937981,45.51853566)(316.43937988,45.48853811)
\curveto(316.4493798,45.46853571)(316.4493798,45.44353574)(316.43937988,45.41353811)
\lineto(316.37937988,45.17353811)
\curveto(316.36937988,45.09353609)(316.3493799,45.01853616)(316.31937988,44.94853811)
\curveto(316.18938006,44.64853653)(316.0443802,44.40353678)(315.88437988,44.21353811)
\curveto(315.71438053,44.03353715)(315.47938077,43.8835373)(315.17937988,43.76353811)
\curveto(314.95938129,43.67353751)(314.69438155,43.62853755)(314.38437988,43.62853811)
\lineto(314.06937988,43.62853811)
\curveto(314.01938223,43.63853754)(313.96938228,43.64353754)(313.91937988,43.64353811)
\lineto(313.73937988,43.67353811)
\lineto(313.40937988,43.79353811)
\curveto(313.29938295,43.83353735)(313.19938305,43.8835373)(313.10937988,43.94353811)
\curveto(312.81938343,44.12353706)(312.60438364,44.36853681)(312.46437988,44.67853811)
\curveto(312.32438392,44.98853619)(312.19938405,45.32853585)(312.08937988,45.69853811)
\curveto(312.0493842,45.83853534)(312.01938423,45.9835352)(311.99937988,46.13353811)
\curveto(311.97938427,46.2835349)(311.95438429,46.43353475)(311.92437988,46.58353811)
\curveto(311.90438434,46.65353453)(311.89438435,46.71853446)(311.89437988,46.77853811)
\curveto(311.89438435,46.84853433)(311.88438436,46.92353426)(311.86437988,47.00353811)
\curveto(311.8443844,47.07353411)(311.83438441,47.14353404)(311.83437988,47.21353811)
\curveto(311.82438442,47.2835339)(311.80938444,47.35853382)(311.78937988,47.43853811)
\curveto(311.72938452,47.68853349)(311.67938457,47.92353326)(311.63937988,48.14353811)
\curveto(311.58938466,48.36353282)(311.47438477,48.53853264)(311.29437988,48.66853811)
\curveto(311.21438503,48.72853245)(311.11438513,48.7785324)(310.99437988,48.81853811)
\curveto(310.86438538,48.85853232)(310.72438552,48.85853232)(310.57437988,48.81853811)
\curveto(310.33438591,48.75853242)(310.1443861,48.66853251)(310.00437988,48.54853811)
\curveto(309.86438638,48.43853274)(309.75438649,48.2785329)(309.67437988,48.06853811)
\curveto(309.62438662,47.94853323)(309.58938666,47.80353338)(309.56937988,47.63353811)
\curveto(309.5493867,47.47353371)(309.53938671,47.30353388)(309.53937988,47.12353811)
\curveto(309.53938671,46.94353424)(309.5493867,46.76853441)(309.56937988,46.59853811)
\curveto(309.58938666,46.42853475)(309.61938663,46.2835349)(309.65937988,46.16353811)
\curveto(309.71938653,45.99353519)(309.80438644,45.82853535)(309.91437988,45.66853811)
\curveto(309.97438627,45.58853559)(310.05438619,45.51353567)(310.15437988,45.44353811)
\curveto(310.244386,45.3835358)(310.3443859,45.32853585)(310.45437988,45.27853811)
\curveto(310.53438571,45.24853593)(310.61938563,45.21853596)(310.70937988,45.18853811)
\curveto(310.79938545,45.16853601)(310.86938538,45.12353606)(310.91937988,45.05353811)
\curveto(310.9493853,45.01353617)(310.97438527,44.94353624)(310.99437988,44.84353811)
\curveto(311.00438524,44.75353643)(311.00938524,44.65853652)(311.00937988,44.55853811)
\curveto(311.00938524,44.45853672)(311.00438524,44.35853682)(310.99437988,44.25853811)
\curveto(310.97438527,44.16853701)(310.9493853,44.10353708)(310.91937988,44.06353811)
\curveto(310.88938536,44.02353716)(310.83938541,43.99353719)(310.76937988,43.97353811)
\curveto(310.69938555,43.95353723)(310.62438562,43.95353723)(310.54437988,43.97353811)
\curveto(310.41438583,44.00353718)(310.29438595,44.03353715)(310.18437988,44.06353811)
\curveto(310.06438618,44.10353708)(309.9493863,44.14853703)(309.83937988,44.19853811)
\curveto(309.48938676,44.38853679)(309.21938703,44.62853655)(309.02937988,44.91853811)
\curveto(308.82938742,45.20853597)(308.66938758,45.56853561)(308.54937988,45.99853811)
\curveto(308.52938772,46.09853508)(308.51438773,46.19853498)(308.50437988,46.29853811)
\curveto(308.49438775,46.40853477)(308.47938777,46.51853466)(308.45937988,46.62853811)
\curveto(308.4493878,46.66853451)(308.4493878,46.73353445)(308.45937988,46.82353811)
\curveto(308.45938779,46.91353427)(308.4493878,46.96853421)(308.42937988,46.98853811)
\curveto(308.41938783,47.68853349)(308.49938775,48.29853288)(308.66937988,48.81853811)
\curveto(308.83938741,49.33853184)(309.16438708,49.70353148)(309.64437988,49.91353811)
\curveto(309.8443864,50.00353118)(310.07938617,50.05353113)(310.34937988,50.06353811)
\curveto(310.60938564,50.0835311)(310.88438536,50.09353109)(311.17437988,50.09353811)
\lineto(314.48937988,50.09353811)
\curveto(314.62938162,50.09353109)(314.76438148,50.09853108)(314.89437988,50.10853811)
\curveto(315.02438122,50.11853106)(315.12938112,50.14853103)(315.20937988,50.19853811)
\curveto(315.27938097,50.24853093)(315.32938092,50.31353087)(315.35937988,50.39353811)
\curveto(315.39938085,50.4835307)(315.42938082,50.56853061)(315.44937988,50.64853811)
\curveto(315.45938079,50.72853045)(315.50438074,50.78853039)(315.58437988,50.82853811)
\curveto(315.61438063,50.84853033)(315.6443806,50.85853032)(315.67437988,50.85853811)
\curveto(315.70438054,50.85853032)(315.7443805,50.86353032)(315.79437988,50.87353811)
\moveto(314.12937988,48.72853811)
\curveto(313.98938226,48.78853239)(313.82938242,48.81853236)(313.64937988,48.81853811)
\curveto(313.45938279,48.82853235)(313.26438298,48.83353235)(313.06437988,48.83353811)
\curveto(312.95438329,48.83353235)(312.85438339,48.82853235)(312.76437988,48.81853811)
\curveto(312.67438357,48.80853237)(312.60438364,48.76853241)(312.55437988,48.69853811)
\curveto(312.53438371,48.66853251)(312.52438372,48.59853258)(312.52437988,48.48853811)
\curveto(312.5443837,48.46853271)(312.55438369,48.43353275)(312.55437988,48.38353811)
\curveto(312.55438369,48.33353285)(312.56438368,48.28853289)(312.58437988,48.24853811)
\curveto(312.60438364,48.16853301)(312.62438362,48.0785331)(312.64437988,47.97853811)
\lineto(312.70437988,47.67853811)
\curveto(312.70438354,47.64853353)(312.70938354,47.61353357)(312.71937988,47.57353811)
\lineto(312.71937988,47.46853811)
\curveto(312.75938349,47.31853386)(312.78438346,47.15353403)(312.79437988,46.97353811)
\curveto(312.79438345,46.80353438)(312.81438343,46.64353454)(312.85437988,46.49353811)
\curveto(312.87438337,46.41353477)(312.89438335,46.33853484)(312.91437988,46.26853811)
\curveto(312.92438332,46.20853497)(312.93938331,46.13853504)(312.95937988,46.05853811)
\curveto(313.00938324,45.89853528)(313.07438317,45.74853543)(313.15437988,45.60853811)
\curveto(313.22438302,45.46853571)(313.31438293,45.34853583)(313.42437988,45.24853811)
\curveto(313.53438271,45.14853603)(313.66938258,45.07353611)(313.82937988,45.02353811)
\curveto(313.97938227,44.97353621)(314.16438208,44.95353623)(314.38437988,44.96353811)
\curveto(314.48438176,44.96353622)(314.57938167,44.9785362)(314.66937988,45.00853811)
\curveto(314.7493815,45.04853613)(314.82438142,45.09353609)(314.89437988,45.14353811)
\curveto(315.00438124,45.22353596)(315.09938115,45.32853585)(315.17937988,45.45853811)
\curveto(315.249381,45.58853559)(315.30938094,45.72853545)(315.35937988,45.87853811)
\curveto(315.36938088,45.92853525)(315.37438087,45.9785352)(315.37437988,46.02853811)
\curveto(315.37438087,46.0785351)(315.37938087,46.12853505)(315.38937988,46.17853811)
\curveto(315.40938084,46.24853493)(315.42438082,46.33353485)(315.43437988,46.43353811)
\curveto(315.43438081,46.54353464)(315.42438082,46.63353455)(315.40437988,46.70353811)
\curveto(315.38438086,46.76353442)(315.37938087,46.82353436)(315.38937988,46.88353811)
\curveto(315.38938086,46.94353424)(315.37938087,47.00353418)(315.35937988,47.06353811)
\curveto(315.33938091,47.14353404)(315.32438092,47.21853396)(315.31437988,47.28853811)
\curveto(315.30438094,47.36853381)(315.28438096,47.44353374)(315.25437988,47.51353811)
\curveto(315.13438111,47.80353338)(314.98938126,48.04853313)(314.81937988,48.24853811)
\curveto(314.6493816,48.45853272)(314.41938183,48.61853256)(314.12937988,48.72853811)
}
}
{
\newrgbcolor{curcolor}{0 0 0}
\pscustom[linestyle=none,fillstyle=solid,fillcolor=curcolor]
{
\newpath
\moveto(308.44437988,55.69017873)
\curveto(308.4443878,55.92017394)(308.50438774,56.05017381)(308.62437988,56.08017873)
\curveto(308.73438751,56.11017375)(308.89938735,56.12517374)(309.11937988,56.12517873)
\lineto(309.40437988,56.12517873)
\curveto(309.49438675,56.12517374)(309.56938668,56.10017376)(309.62937988,56.05017873)
\curveto(309.70938654,55.99017387)(309.75438649,55.90517396)(309.76437988,55.79517873)
\curveto(309.76438648,55.68517418)(309.77938647,55.57517429)(309.80937988,55.46517873)
\curveto(309.83938641,55.32517454)(309.86938638,55.19017467)(309.89937988,55.06017873)
\curveto(309.92938632,54.94017492)(309.96938628,54.82517504)(310.01937988,54.71517873)
\curveto(310.1493861,54.42517544)(310.32938592,54.19017567)(310.55937988,54.01017873)
\curveto(310.77938547,53.83017603)(311.03438521,53.67517619)(311.32437988,53.54517873)
\curveto(311.43438481,53.50517636)(311.5493847,53.47517639)(311.66937988,53.45517873)
\curveto(311.77938447,53.43517643)(311.89438435,53.41017645)(312.01437988,53.38017873)
\curveto(312.06438418,53.37017649)(312.11438413,53.3651765)(312.16437988,53.36517873)
\curveto(312.21438403,53.37517649)(312.26438398,53.37517649)(312.31437988,53.36517873)
\curveto(312.43438381,53.33517653)(312.57438367,53.32017654)(312.73437988,53.32017873)
\curveto(312.88438336,53.33017653)(313.02938322,53.33517653)(313.16937988,53.33517873)
\lineto(315.01437988,53.33517873)
\lineto(315.35937988,53.33517873)
\curveto(315.47938077,53.33517653)(315.59438065,53.33017653)(315.70437988,53.32017873)
\curveto(315.81438043,53.31017655)(315.90938034,53.30517656)(315.98937988,53.30517873)
\curveto(316.06938018,53.31517655)(316.13938011,53.29517657)(316.19937988,53.24517873)
\curveto(316.26937998,53.19517667)(316.30937994,53.11517675)(316.31937988,53.00517873)
\curveto(316.32937992,52.90517696)(316.33437991,52.79517707)(316.33437988,52.67517873)
\lineto(316.33437988,52.40517873)
\curveto(316.31437993,52.35517751)(316.29937995,52.30517756)(316.28937988,52.25517873)
\curveto(316.26937998,52.21517765)(316.24438,52.18517768)(316.21437988,52.16517873)
\curveto(316.1443801,52.11517775)(316.05938019,52.08517778)(315.95937988,52.07517873)
\lineto(315.62937988,52.07517873)
\lineto(314.47437988,52.07517873)
\lineto(310.31937988,52.07517873)
\lineto(309.28437988,52.07517873)
\lineto(308.98437988,52.07517873)
\curveto(308.88438736,52.08517778)(308.79938745,52.11517775)(308.72937988,52.16517873)
\curveto(308.68938756,52.19517767)(308.65938759,52.24517762)(308.63937988,52.31517873)
\curveto(308.61938763,52.39517747)(308.60938764,52.48017738)(308.60937988,52.57017873)
\curveto(308.59938765,52.6601772)(308.59938765,52.75017711)(308.60937988,52.84017873)
\curveto(308.61938763,52.93017693)(308.63438761,53.00017686)(308.65437988,53.05017873)
\curveto(308.68438756,53.13017673)(308.7443875,53.18017668)(308.83437988,53.20017873)
\curveto(308.91438733,53.23017663)(309.00438724,53.24517662)(309.10437988,53.24517873)
\lineto(309.40437988,53.24517873)
\curveto(309.50438674,53.24517662)(309.59438665,53.2651766)(309.67437988,53.30517873)
\curveto(309.69438655,53.31517655)(309.70938654,53.32517654)(309.71937988,53.33517873)
\lineto(309.76437988,53.38017873)
\curveto(309.76438648,53.49017637)(309.71938653,53.58017628)(309.62937988,53.65017873)
\curveto(309.52938672,53.72017614)(309.4493868,53.78017608)(309.38937988,53.83017873)
\lineto(309.29937988,53.92017873)
\curveto(309.18938706,54.01017585)(309.07438717,54.13517573)(308.95437988,54.29517873)
\curveto(308.83438741,54.45517541)(308.7443875,54.60517526)(308.68437988,54.74517873)
\curveto(308.63438761,54.83517503)(308.59938765,54.93017493)(308.57937988,55.03017873)
\curveto(308.5493877,55.13017473)(308.51938773,55.23517463)(308.48937988,55.34517873)
\curveto(308.47938777,55.40517446)(308.47438777,55.4651744)(308.47437988,55.52517873)
\curveto(308.46438778,55.58517428)(308.45438779,55.64017422)(308.44437988,55.69017873)
}
}
{
\newrgbcolor{curcolor}{0 0 0}
\pscustom[linestyle=none,fillstyle=solid,fillcolor=curcolor]
{
}
}
{
\newrgbcolor{curcolor}{0 0 0}
\pscustom[linestyle=none,fillstyle=solid,fillcolor=curcolor]
{
\newpath
\moveto(305.74437988,64.11010061)
\curveto(305.71439053,65.74009517)(306.26938998,66.79009412)(307.40937988,67.26010061)
\curveto(307.63938861,67.36009355)(307.92938832,67.42509348)(308.27937988,67.45510061)
\curveto(308.61938763,67.49509341)(308.92938732,67.47009344)(309.20937988,67.38010061)
\curveto(309.46938678,67.29009362)(309.69438655,67.17009374)(309.88437988,67.02010061)
\curveto(309.92438632,67.00009391)(309.95938629,66.97509393)(309.98937988,66.94510061)
\curveto(310.00938624,66.91509399)(310.03438621,66.89009402)(310.06437988,66.87010061)
\lineto(310.18437988,66.78010061)
\curveto(310.21438603,66.75009416)(310.23938601,66.71509419)(310.25937988,66.67510061)
\curveto(310.30938594,66.62509428)(310.35438589,66.57009434)(310.39437988,66.51010061)
\curveto(310.43438581,66.46009445)(310.48438576,66.41509449)(310.54437988,66.37510061)
\curveto(310.58438566,66.33509457)(310.63438561,66.32009459)(310.69437988,66.33010061)
\curveto(310.7443855,66.34009457)(310.78938546,66.37009454)(310.82937988,66.42010061)
\curveto(310.86938538,66.47009444)(310.90938534,66.52509438)(310.94937988,66.58510061)
\curveto(310.97938527,66.65509425)(311.00938524,66.72009419)(311.03937988,66.78010061)
\curveto(311.06938518,66.84009407)(311.09938515,66.89009402)(311.12937988,66.93010061)
\curveto(311.3493849,67.25009366)(311.65938459,67.5050934)(312.05937988,67.69510061)
\curveto(312.1493841,67.73509317)(312.244384,67.76509314)(312.34437988,67.78510061)
\curveto(312.43438381,67.81509309)(312.52438372,67.84009307)(312.61437988,67.86010061)
\curveto(312.66438358,67.87009304)(312.71438353,67.87509303)(312.76437988,67.87510061)
\curveto(312.80438344,67.88509302)(312.8493834,67.89509301)(312.89937988,67.90510061)
\curveto(312.9493833,67.91509299)(312.99938325,67.91509299)(313.04937988,67.90510061)
\curveto(313.09938315,67.89509301)(313.1493831,67.90009301)(313.19937988,67.92010061)
\curveto(313.249383,67.93009298)(313.30938294,67.93509297)(313.37937988,67.93510061)
\curveto(313.4493828,67.93509297)(313.50938274,67.92509298)(313.55937988,67.90510061)
\lineto(313.78437988,67.90510061)
\lineto(314.02437988,67.84510061)
\curveto(314.09438215,67.83509307)(314.16438208,67.82009309)(314.23437988,67.80010061)
\curveto(314.32438192,67.77009314)(314.40938184,67.74009317)(314.48937988,67.71010061)
\curveto(314.56938168,67.69009322)(314.6493816,67.66009325)(314.72937988,67.62010061)
\curveto(314.78938146,67.60009331)(314.8493814,67.57009334)(314.90937988,67.53010061)
\curveto(314.95938129,67.50009341)(315.00938124,67.46509344)(315.05937988,67.42510061)
\curveto(315.36938088,67.22509368)(315.62938062,66.97509393)(315.83937988,66.67510061)
\curveto(316.03938021,66.37509453)(316.20438004,66.03009488)(316.33437988,65.64010061)
\curveto(316.37437987,65.52009539)(316.39937985,65.39009552)(316.40937988,65.25010061)
\curveto(316.42937982,65.12009579)(316.45437979,64.98509592)(316.48437988,64.84510061)
\curveto(316.49437975,64.77509613)(316.49937975,64.7050962)(316.49937988,64.63510061)
\curveto(316.49937975,64.57509633)(316.50437974,64.5100964)(316.51437988,64.44010061)
\curveto(316.52437972,64.40009651)(316.52937972,64.34009657)(316.52937988,64.26010061)
\curveto(316.52937972,64.19009672)(316.52437972,64.14009677)(316.51437988,64.11010061)
\curveto(316.50437974,64.06009685)(316.49937975,64.01509689)(316.49937988,63.97510061)
\lineto(316.49937988,63.85510061)
\curveto(316.47937977,63.75509715)(316.46437978,63.65509725)(316.45437988,63.55510061)
\curveto(316.4443798,63.45509745)(316.42937982,63.36009755)(316.40937988,63.27010061)
\curveto(316.37937987,63.16009775)(316.35437989,63.05009786)(316.33437988,62.94010061)
\curveto(316.30437994,62.84009807)(316.26437998,62.73509817)(316.21437988,62.62510061)
\curveto(316.05438019,62.25509865)(315.85438039,61.94009897)(315.61437988,61.68010061)
\curveto(315.36438088,61.42009949)(315.05438119,61.2100997)(314.68437988,61.05010061)
\curveto(314.59438165,61.0100999)(314.49938175,60.97509993)(314.39937988,60.94510061)
\curveto(314.29938195,60.91509999)(314.19438205,60.88510002)(314.08437988,60.85510061)
\curveto(314.03438221,60.83510007)(313.98438226,60.82510008)(313.93437988,60.82510061)
\curveto(313.87438237,60.82510008)(313.81438243,60.81510009)(313.75437988,60.79510061)
\curveto(313.69438255,60.77510013)(313.61438263,60.76510014)(313.51437988,60.76510061)
\curveto(313.41438283,60.76510014)(313.33938291,60.78010013)(313.28937988,60.81010061)
\curveto(313.25938299,60.82010009)(313.23438301,60.83510007)(313.21437988,60.85510061)
\lineto(313.15437988,60.91510061)
\curveto(313.13438311,60.95509995)(313.11938313,61.01509989)(313.10937988,61.09510061)
\curveto(313.09938315,61.18509972)(313.09438315,61.27509963)(313.09437988,61.36510061)
\curveto(313.09438315,61.45509945)(313.09938315,61.54009937)(313.10937988,61.62010061)
\curveto(313.11938313,61.7100992)(313.12938312,61.77509913)(313.13937988,61.81510061)
\curveto(313.15938309,61.83509907)(313.17438307,61.85509905)(313.18437988,61.87510061)
\curveto(313.18438306,61.89509901)(313.19438305,61.91509899)(313.21437988,61.93510061)
\curveto(313.30438294,62.0050989)(313.41938283,62.04509886)(313.55937988,62.05510061)
\curveto(313.69938255,62.07509883)(313.82438242,62.1050988)(313.93437988,62.14510061)
\lineto(314.29437988,62.29510061)
\curveto(314.40438184,62.34509856)(314.50938174,62.4100985)(314.60937988,62.49010061)
\curveto(314.63938161,62.5100984)(314.66438158,62.53009838)(314.68437988,62.55010061)
\curveto(314.70438154,62.58009833)(314.72938152,62.6050983)(314.75937988,62.62510061)
\curveto(314.81938143,62.66509824)(314.86438138,62.70009821)(314.89437988,62.73010061)
\curveto(314.92438132,62.77009814)(314.95438129,62.8050981)(314.98437988,62.83510061)
\curveto(315.01438123,62.87509803)(315.0443812,62.92009799)(315.07437988,62.97010061)
\curveto(315.13438111,63.06009785)(315.18438106,63.15509775)(315.22437988,63.25510061)
\lineto(315.34437988,63.58510061)
\curveto(315.39438085,63.73509717)(315.42438082,63.93509697)(315.43437988,64.18510061)
\curveto(315.4443808,64.43509647)(315.42438082,64.64509626)(315.37437988,64.81510061)
\curveto(315.35438089,64.89509601)(315.33938091,64.96509594)(315.32937988,65.02510061)
\lineto(315.26937988,65.23510061)
\curveto(315.1493811,65.51509539)(314.99938125,65.75509515)(314.81937988,65.95510061)
\curveto(314.63938161,66.16509474)(314.40938184,66.33009458)(314.12937988,66.45010061)
\curveto(314.05938219,66.48009443)(313.98938226,66.50009441)(313.91937988,66.51010061)
\lineto(313.67937988,66.57010061)
\curveto(313.53938271,66.6100943)(313.37938287,66.62009429)(313.19937988,66.60010061)
\curveto(313.00938324,66.58009433)(312.85938339,66.55009436)(312.74937988,66.51010061)
\curveto(312.36938388,66.38009453)(312.07938417,66.19509471)(311.87937988,65.95510061)
\curveto(311.67938457,65.72509518)(311.51938473,65.41509549)(311.39937988,65.02510061)
\curveto(311.36938488,64.91509599)(311.3493849,64.79509611)(311.33937988,64.66510061)
\curveto(311.32938492,64.54509636)(311.32438492,64.42009649)(311.32437988,64.29010061)
\curveto(311.32438492,64.13009678)(311.31938493,63.99009692)(311.30937988,63.87010061)
\curveto(311.29938495,63.75009716)(311.23938501,63.66509724)(311.12937988,63.61510061)
\curveto(311.09938515,63.59509731)(311.06438518,63.58509732)(311.02437988,63.58510061)
\lineto(310.88937988,63.58510061)
\curveto(310.78938546,63.57509733)(310.69438555,63.57509733)(310.60437988,63.58510061)
\curveto(310.51438573,63.6050973)(310.4493858,63.64509726)(310.40937988,63.70510061)
\curveto(310.37938587,63.74509716)(310.35938589,63.78509712)(310.34937988,63.82510061)
\curveto(310.33938591,63.87509703)(310.32938592,63.93009698)(310.31937988,63.99010061)
\curveto(310.30938594,64.0100969)(310.30938594,64.03509687)(310.31937988,64.06510061)
\curveto(310.31938593,64.09509681)(310.31438593,64.12009679)(310.30437988,64.14010061)
\lineto(310.30437988,64.27510061)
\curveto(310.28438596,64.38509652)(310.27438597,64.48509642)(310.27437988,64.57510061)
\curveto(310.26438598,64.67509623)(310.244386,64.77009614)(310.21437988,64.86010061)
\curveto(310.10438614,65.18009573)(309.95938629,65.43509547)(309.77937988,65.62510061)
\curveto(309.59938665,65.81509509)(309.3493869,65.96509494)(309.02937988,66.07510061)
\curveto(308.92938732,66.1050948)(308.80438744,66.12509478)(308.65437988,66.13510061)
\curveto(308.49438775,66.15509475)(308.3493879,66.15009476)(308.21937988,66.12010061)
\curveto(308.1493881,66.10009481)(308.08438816,66.08009483)(308.02437988,66.06010061)
\curveto(307.95438829,66.05009486)(307.88938836,66.03009488)(307.82937988,66.00010061)
\curveto(307.58938866,65.90009501)(307.39938885,65.75509515)(307.25937988,65.56510061)
\curveto(307.11938913,65.37509553)(307.00938924,65.15009576)(306.92937988,64.89010061)
\curveto(306.90938934,64.83009608)(306.89938935,64.77009614)(306.89937988,64.71010061)
\curveto(306.89938935,64.65009626)(306.88938936,64.58509632)(306.86937988,64.51510061)
\curveto(306.8493894,64.43509647)(306.83938941,64.34009657)(306.83937988,64.23010061)
\curveto(306.83938941,64.12009679)(306.8493894,64.02509688)(306.86937988,63.94510061)
\curveto(306.88938936,63.89509701)(306.89938935,63.84509706)(306.89937988,63.79510061)
\curveto(306.89938935,63.75509715)(306.90938934,63.7100972)(306.92937988,63.66010061)
\curveto(306.97938927,63.48009743)(307.05438919,63.3100976)(307.15437988,63.15010061)
\curveto(307.244389,63.00009791)(307.35938889,62.87009804)(307.49937988,62.76010061)
\curveto(307.61938863,62.67009824)(307.7493885,62.59009832)(307.88937988,62.52010061)
\curveto(308.02938822,62.45009846)(308.18438806,62.38509852)(308.35437988,62.32510061)
\curveto(308.46438778,62.29509861)(308.58438766,62.27509863)(308.71437988,62.26510061)
\curveto(308.83438741,62.25509865)(308.93438731,62.22009869)(309.01437988,62.16010061)
\curveto(309.05438719,62.14009877)(309.09438715,62.08009883)(309.13437988,61.98010061)
\curveto(309.1443871,61.94009897)(309.15438709,61.88009903)(309.16437988,61.80010061)
\lineto(309.16437988,61.54510061)
\curveto(309.15438709,61.45509945)(309.1443871,61.37009954)(309.13437988,61.29010061)
\curveto(309.12438712,61.22009969)(309.10938714,61.17009974)(309.08937988,61.14010061)
\curveto(309.05938719,61.10009981)(309.00438724,61.06509984)(308.92437988,61.03510061)
\curveto(308.8443874,61.0050999)(308.75938749,61.00009991)(308.66937988,61.02010061)
\curveto(308.61938763,61.03009988)(308.56938768,61.03509987)(308.51937988,61.03510061)
\lineto(308.33937988,61.06510061)
\curveto(308.23938801,61.09509981)(308.13938811,61.12009979)(308.03937988,61.14010061)
\curveto(307.93938831,61.17009974)(307.8493884,61.2050997)(307.76937988,61.24510061)
\curveto(307.65938859,61.29509961)(307.55438869,61.34009957)(307.45437988,61.38010061)
\curveto(307.3443889,61.42009949)(307.23938901,61.47009944)(307.13937988,61.53010061)
\curveto(306.59938965,61.86009905)(306.20439004,62.33009858)(305.95437988,62.94010061)
\curveto(305.90439034,63.06009785)(305.86939038,63.18509772)(305.84937988,63.31510061)
\curveto(305.82939042,63.45509745)(305.80439044,63.59509731)(305.77437988,63.73510061)
\curveto(305.76439048,63.79509711)(305.75939049,63.85509705)(305.75937988,63.91510061)
\curveto(305.75939049,63.98509692)(305.75439049,64.05009686)(305.74437988,64.11010061)
}
}
{
\newrgbcolor{curcolor}{0 0 0}
\pscustom[linestyle=none,fillstyle=solid,fillcolor=curcolor]
{
\newpath
\moveto(311.26437988,76.34470998)
\lineto(311.51937988,76.34470998)
\curveto(311.59938465,76.35470228)(311.67438457,76.34970228)(311.74437988,76.32970998)
\lineto(311.98437988,76.32970998)
\lineto(312.14937988,76.32970998)
\curveto(312.249384,76.30970232)(312.35438389,76.29970233)(312.46437988,76.29970998)
\curveto(312.56438368,76.29970233)(312.66438358,76.28970234)(312.76437988,76.26970998)
\lineto(312.91437988,76.26970998)
\curveto(313.05438319,76.23970239)(313.19438305,76.21970241)(313.33437988,76.20970998)
\curveto(313.46438278,76.19970243)(313.59438265,76.17470246)(313.72437988,76.13470998)
\curveto(313.80438244,76.11470252)(313.88938236,76.09470254)(313.97937988,76.07470998)
\lineto(314.21937988,76.01470998)
\lineto(314.51937988,75.89470998)
\curveto(314.60938164,75.86470277)(314.69938155,75.8297028)(314.78937988,75.78970998)
\curveto(315.00938124,75.68970294)(315.22438102,75.55470308)(315.43437988,75.38470998)
\curveto(315.6443806,75.22470341)(315.81438043,75.04970358)(315.94437988,74.85970998)
\curveto(315.98438026,74.80970382)(316.02438022,74.74970388)(316.06437988,74.67970998)
\curveto(316.09438015,74.61970401)(316.12938012,74.55970407)(316.16937988,74.49970998)
\curveto(316.21938003,74.41970421)(316.25937999,74.32470431)(316.28937988,74.21470998)
\curveto(316.31937993,74.10470453)(316.3493799,73.99970463)(316.37937988,73.89970998)
\curveto(316.41937983,73.78970484)(316.4443798,73.67970495)(316.45437988,73.56970998)
\curveto(316.46437978,73.45970517)(316.47937977,73.34470529)(316.49937988,73.22470998)
\curveto(316.50937974,73.18470545)(316.50937974,73.13970549)(316.49937988,73.08970998)
\curveto(316.49937975,73.04970558)(316.50437974,73.00970562)(316.51437988,72.96970998)
\curveto(316.52437972,72.9297057)(316.52937972,72.87470576)(316.52937988,72.80470998)
\curveto(316.52937972,72.7347059)(316.52437972,72.68470595)(316.51437988,72.65470998)
\curveto(316.49437975,72.60470603)(316.48937976,72.55970607)(316.49937988,72.51970998)
\curveto(316.50937974,72.47970615)(316.50937974,72.44470619)(316.49937988,72.41470998)
\lineto(316.49937988,72.32470998)
\curveto(316.47937977,72.26470637)(316.46437978,72.19970643)(316.45437988,72.12970998)
\curveto(316.45437979,72.06970656)(316.4493798,72.00470663)(316.43937988,71.93470998)
\curveto(316.38937986,71.76470687)(316.33937991,71.60470703)(316.28937988,71.45470998)
\curveto(316.23938001,71.30470733)(316.17438007,71.15970747)(316.09437988,71.01970998)
\curveto(316.05438019,70.96970766)(316.02438022,70.91470772)(316.00437988,70.85470998)
\curveto(315.97438027,70.80470783)(315.93938031,70.75470788)(315.89937988,70.70470998)
\curveto(315.71938053,70.46470817)(315.49938075,70.26470837)(315.23937988,70.10470998)
\curveto(314.97938127,69.94470869)(314.69438155,69.80470883)(314.38437988,69.68470998)
\curveto(314.244382,69.62470901)(314.10438214,69.57970905)(313.96437988,69.54970998)
\curveto(313.81438243,69.51970911)(313.65938259,69.48470915)(313.49937988,69.44470998)
\curveto(313.38938286,69.42470921)(313.27938297,69.40970922)(313.16937988,69.39970998)
\curveto(313.05938319,69.38970924)(312.9493833,69.37470926)(312.83937988,69.35470998)
\curveto(312.79938345,69.34470929)(312.75938349,69.33970929)(312.71937988,69.33970998)
\curveto(312.67938357,69.34970928)(312.63938361,69.34970928)(312.59937988,69.33970998)
\curveto(312.5493837,69.3297093)(312.49938375,69.32470931)(312.44937988,69.32470998)
\lineto(312.28437988,69.32470998)
\curveto(312.23438401,69.30470933)(312.18438406,69.29970933)(312.13437988,69.30970998)
\curveto(312.07438417,69.31970931)(312.01938423,69.31970931)(311.96937988,69.30970998)
\curveto(311.92938432,69.29970933)(311.88438436,69.29970933)(311.83437988,69.30970998)
\curveto(311.78438446,69.31970931)(311.73438451,69.31470932)(311.68437988,69.29470998)
\curveto(311.61438463,69.27470936)(311.53938471,69.26970936)(311.45937988,69.27970998)
\curveto(311.36938488,69.28970934)(311.28438496,69.29470934)(311.20437988,69.29470998)
\curveto(311.11438513,69.29470934)(311.01438523,69.28970934)(310.90437988,69.27970998)
\curveto(310.78438546,69.26970936)(310.68438556,69.27470936)(310.60437988,69.29470998)
\lineto(310.31937988,69.29470998)
\lineto(309.68937988,69.33970998)
\curveto(309.58938666,69.34970928)(309.49438675,69.35970927)(309.40437988,69.36970998)
\lineto(309.10437988,69.39970998)
\curveto(309.05438719,69.41970921)(309.00438724,69.42470921)(308.95437988,69.41470998)
\curveto(308.89438735,69.41470922)(308.83938741,69.42470921)(308.78937988,69.44470998)
\curveto(308.61938763,69.49470914)(308.45438779,69.5347091)(308.29437988,69.56470998)
\curveto(308.12438812,69.59470904)(307.96438828,69.64470899)(307.81437988,69.71470998)
\curveto(307.35438889,69.90470873)(306.97938927,70.12470851)(306.68937988,70.37470998)
\curveto(306.39938985,70.634708)(306.15439009,70.99470764)(305.95437988,71.45470998)
\curveto(305.90439034,71.58470705)(305.86939038,71.71470692)(305.84937988,71.84470998)
\curveto(305.82939042,71.98470665)(305.80439044,72.12470651)(305.77437988,72.26470998)
\curveto(305.76439048,72.3347063)(305.75939049,72.39970623)(305.75937988,72.45970998)
\curveto(305.75939049,72.51970611)(305.75439049,72.58470605)(305.74437988,72.65470998)
\curveto(305.72439052,73.48470515)(305.87439037,74.15470448)(306.19437988,74.66470998)
\curveto(306.50438974,75.17470346)(306.9443893,75.55470308)(307.51437988,75.80470998)
\curveto(307.63438861,75.85470278)(307.75938849,75.89970273)(307.88937988,75.93970998)
\curveto(308.01938823,75.97970265)(308.15438809,76.02470261)(308.29437988,76.07470998)
\curveto(308.37438787,76.09470254)(308.45938779,76.10970252)(308.54937988,76.11970998)
\lineto(308.78937988,76.17970998)
\curveto(308.89938735,76.20970242)(309.00938724,76.22470241)(309.11937988,76.22470998)
\curveto(309.22938702,76.2347024)(309.33938691,76.24970238)(309.44937988,76.26970998)
\curveto(309.49938675,76.28970234)(309.5443867,76.29470234)(309.58437988,76.28470998)
\curveto(309.62438662,76.28470235)(309.66438658,76.28970234)(309.70437988,76.29970998)
\curveto(309.75438649,76.30970232)(309.80938644,76.30970232)(309.86937988,76.29970998)
\curveto(309.91938633,76.29970233)(309.96938628,76.30470233)(310.01937988,76.31470998)
\lineto(310.15437988,76.31470998)
\curveto(310.21438603,76.3347023)(310.28438596,76.3347023)(310.36437988,76.31470998)
\curveto(310.43438581,76.30470233)(310.49938575,76.30970232)(310.55937988,76.32970998)
\curveto(310.58938566,76.33970229)(310.62938562,76.34470229)(310.67937988,76.34470998)
\lineto(310.79937988,76.34470998)
\lineto(311.26437988,76.34470998)
\moveto(313.58937988,74.79970998)
\curveto(313.26938298,74.89970373)(312.90438334,74.95970367)(312.49437988,74.97970998)
\curveto(312.08438416,74.99970363)(311.67438457,75.00970362)(311.26437988,75.00970998)
\curveto(310.83438541,75.00970362)(310.41438583,74.99970363)(310.00437988,74.97970998)
\curveto(309.59438665,74.95970367)(309.20938704,74.91470372)(308.84937988,74.84470998)
\curveto(308.48938776,74.77470386)(308.16938808,74.66470397)(307.88937988,74.51470998)
\curveto(307.59938865,74.37470426)(307.36438888,74.17970445)(307.18437988,73.92970998)
\curveto(307.07438917,73.76970486)(306.99438925,73.58970504)(306.94437988,73.38970998)
\curveto(306.88438936,73.18970544)(306.85438939,72.94470569)(306.85437988,72.65470998)
\curveto(306.87438937,72.634706)(306.88438936,72.59970603)(306.88437988,72.54970998)
\curveto(306.87438937,72.49970613)(306.87438937,72.45970617)(306.88437988,72.42970998)
\curveto(306.90438934,72.34970628)(306.92438932,72.27470636)(306.94437988,72.20470998)
\curveto(306.95438929,72.14470649)(306.97438927,72.07970655)(307.00437988,72.00970998)
\curveto(307.12438912,71.73970689)(307.29438895,71.51970711)(307.51437988,71.34970998)
\curveto(307.72438852,71.18970744)(307.96938828,71.05470758)(308.24937988,70.94470998)
\curveto(308.35938789,70.89470774)(308.47938777,70.85470778)(308.60937988,70.82470998)
\curveto(308.72938752,70.80470783)(308.85438739,70.77970785)(308.98437988,70.74970998)
\curveto(309.03438721,70.7297079)(309.08938716,70.71970791)(309.14937988,70.71970998)
\curveto(309.19938705,70.71970791)(309.249387,70.71470792)(309.29937988,70.70470998)
\curveto(309.38938686,70.69470794)(309.48438676,70.68470795)(309.58437988,70.67470998)
\curveto(309.67438657,70.66470797)(309.76938648,70.65470798)(309.86937988,70.64470998)
\curveto(309.9493863,70.64470799)(310.03438621,70.63970799)(310.12437988,70.62970998)
\lineto(310.36437988,70.62970998)
\lineto(310.54437988,70.62970998)
\curveto(310.57438567,70.61970801)(310.60938564,70.61470802)(310.64937988,70.61470998)
\lineto(310.78437988,70.61470998)
\lineto(311.23437988,70.61470998)
\curveto(311.31438493,70.61470802)(311.39938485,70.60970802)(311.48937988,70.59970998)
\curveto(311.56938468,70.59970803)(311.6443846,70.60970802)(311.71437988,70.62970998)
\lineto(311.98437988,70.62970998)
\curveto(312.00438424,70.629708)(312.03438421,70.62470801)(312.07437988,70.61470998)
\curveto(312.10438414,70.61470802)(312.12938412,70.61970801)(312.14937988,70.62970998)
\curveto(312.249384,70.63970799)(312.3493839,70.64470799)(312.44937988,70.64470998)
\curveto(312.53938371,70.65470798)(312.63938361,70.66470797)(312.74937988,70.67470998)
\curveto(312.86938338,70.70470793)(312.99438325,70.71970791)(313.12437988,70.71970998)
\curveto(313.244383,70.7297079)(313.35938289,70.75470788)(313.46937988,70.79470998)
\curveto(313.76938248,70.87470776)(314.03438221,70.95970767)(314.26437988,71.04970998)
\curveto(314.49438175,71.14970748)(314.70938154,71.29470734)(314.90937988,71.48470998)
\curveto(315.10938114,71.69470694)(315.25938099,71.95970667)(315.35937988,72.27970998)
\curveto(315.37938087,72.31970631)(315.38938086,72.35470628)(315.38937988,72.38470998)
\curveto(315.37938087,72.42470621)(315.38438086,72.46970616)(315.40437988,72.51970998)
\curveto(315.41438083,72.55970607)(315.42438082,72.629706)(315.43437988,72.72970998)
\curveto(315.4443808,72.83970579)(315.43938081,72.92470571)(315.41937988,72.98470998)
\curveto(315.39938085,73.05470558)(315.38938086,73.12470551)(315.38937988,73.19470998)
\curveto(315.37938087,73.26470537)(315.36438088,73.3297053)(315.34437988,73.38970998)
\curveto(315.28438096,73.58970504)(315.19938105,73.76970486)(315.08937988,73.92970998)
\curveto(315.06938118,73.95970467)(315.0493812,73.98470465)(315.02937988,74.00470998)
\lineto(314.96937988,74.06470998)
\curveto(314.9493813,74.10470453)(314.90938134,74.15470448)(314.84937988,74.21470998)
\curveto(314.70938154,74.31470432)(314.57938167,74.39970423)(314.45937988,74.46970998)
\curveto(314.33938191,74.53970409)(314.19438205,74.60970402)(314.02437988,74.67970998)
\curveto(313.95438229,74.70970392)(313.88438236,74.7297039)(313.81437988,74.73970998)
\curveto(313.7443825,74.75970387)(313.66938258,74.77970385)(313.58937988,74.79970998)
}
}
{
\newrgbcolor{curcolor}{0 0 0}
\pscustom[linestyle=none,fillstyle=solid,fillcolor=curcolor]
{
\newpath
\moveto(314.71437988,78.63431936)
\lineto(314.71437988,79.26431936)
\lineto(314.71437988,79.45931936)
\curveto(314.71438153,79.52931683)(314.72438152,79.58931677)(314.74437988,79.63931936)
\curveto(314.78438146,79.70931665)(314.82438142,79.7593166)(314.86437988,79.78931936)
\curveto(314.91438133,79.82931653)(314.97938127,79.84931651)(315.05937988,79.84931936)
\curveto(315.13938111,79.8593165)(315.22438102,79.86431649)(315.31437988,79.86431936)
\lineto(316.03437988,79.86431936)
\curveto(316.51437973,79.86431649)(316.92437932,79.80431655)(317.26437988,79.68431936)
\curveto(317.60437864,79.56431679)(317.87937837,79.36931699)(318.08937988,79.09931936)
\curveto(318.13937811,79.02931733)(318.18437806,78.9593174)(318.22437988,78.88931936)
\curveto(318.27437797,78.82931753)(318.31937793,78.7543176)(318.35937988,78.66431936)
\curveto(318.36937788,78.64431771)(318.37937787,78.61431774)(318.38937988,78.57431936)
\curveto(318.40937784,78.53431782)(318.41437783,78.48931787)(318.40437988,78.43931936)
\curveto(318.37437787,78.34931801)(318.29937795,78.29431806)(318.17937988,78.27431936)
\curveto(318.06937818,78.2543181)(317.97437827,78.26931809)(317.89437988,78.31931936)
\curveto(317.82437842,78.34931801)(317.75937849,78.39431796)(317.69937988,78.45431936)
\curveto(317.6493786,78.52431783)(317.59937865,78.58931777)(317.54937988,78.64931936)
\curveto(317.49937875,78.71931764)(317.42437882,78.77931758)(317.32437988,78.82931936)
\curveto(317.23437901,78.88931747)(317.1443791,78.93931742)(317.05437988,78.97931936)
\curveto(317.02437922,78.99931736)(316.96437928,79.02431733)(316.87437988,79.05431936)
\curveto(316.79437945,79.08431727)(316.72437952,79.08931727)(316.66437988,79.06931936)
\curveto(316.52437972,79.03931732)(316.43437981,78.97931738)(316.39437988,78.88931936)
\curveto(316.36437988,78.80931755)(316.3493799,78.71931764)(316.34937988,78.61931936)
\curveto(316.3493799,78.51931784)(316.32437992,78.43431792)(316.27437988,78.36431936)
\curveto(316.20438004,78.27431808)(316.06438018,78.22931813)(315.85437988,78.22931936)
\lineto(315.29937988,78.22931936)
\lineto(315.07437988,78.22931936)
\curveto(314.99438125,78.23931812)(314.92938132,78.2593181)(314.87937988,78.28931936)
\curveto(314.79938145,78.34931801)(314.75438149,78.41931794)(314.74437988,78.49931936)
\curveto(314.73438151,78.51931784)(314.72938152,78.53931782)(314.72937988,78.55931936)
\curveto(314.72938152,78.58931777)(314.72438152,78.61431774)(314.71437988,78.63431936)
}
}
{
\newrgbcolor{curcolor}{0 0 0}
\pscustom[linestyle=none,fillstyle=solid,fillcolor=curcolor]
{
}
}
{
\newrgbcolor{curcolor}{0 0 0}
\pscustom[linestyle=none,fillstyle=solid,fillcolor=curcolor]
{
\newpath
\moveto(305.74437988,89.26463186)
\curveto(305.73439051,89.95462722)(305.85439039,90.55462662)(306.10437988,91.06463186)
\curveto(306.35438989,91.58462559)(306.68938956,91.9796252)(307.10937988,92.24963186)
\curveto(307.18938906,92.29962488)(307.27938897,92.34462483)(307.37937988,92.38463186)
\curveto(307.46938878,92.42462475)(307.56438868,92.46962471)(307.66437988,92.51963186)
\curveto(307.76438848,92.55962462)(307.86438838,92.58962459)(307.96437988,92.60963186)
\curveto(308.06438818,92.62962455)(308.16938808,92.64962453)(308.27937988,92.66963186)
\curveto(308.32938792,92.68962449)(308.37438787,92.69462448)(308.41437988,92.68463186)
\curveto(308.45438779,92.6746245)(308.49938775,92.6796245)(308.54937988,92.69963186)
\curveto(308.59938765,92.70962447)(308.68438756,92.71462446)(308.80437988,92.71463186)
\curveto(308.91438733,92.71462446)(308.99938725,92.70962447)(309.05937988,92.69963186)
\curveto(309.11938713,92.6796245)(309.17938707,92.66962451)(309.23937988,92.66963186)
\curveto(309.29938695,92.6796245)(309.35938689,92.6746245)(309.41937988,92.65463186)
\curveto(309.55938669,92.61462456)(309.69438655,92.5796246)(309.82437988,92.54963186)
\curveto(309.95438629,92.51962466)(310.07938617,92.4796247)(310.19937988,92.42963186)
\curveto(310.33938591,92.36962481)(310.46438578,92.29962488)(310.57437988,92.21963186)
\curveto(310.68438556,92.14962503)(310.79438545,92.0746251)(310.90437988,91.99463186)
\lineto(310.96437988,91.93463186)
\curveto(310.98438526,91.92462525)(311.00438524,91.90962527)(311.02437988,91.88963186)
\curveto(311.18438506,91.76962541)(311.32938492,91.63462554)(311.45937988,91.48463186)
\curveto(311.58938466,91.33462584)(311.71438453,91.174626)(311.83437988,91.00463186)
\curveto(312.05438419,90.69462648)(312.25938399,90.39962678)(312.44937988,90.11963186)
\curveto(312.58938366,89.88962729)(312.72438352,89.65962752)(312.85437988,89.42963186)
\curveto(312.98438326,89.20962797)(313.11938313,88.98962819)(313.25937988,88.76963186)
\curveto(313.42938282,88.51962866)(313.60938264,88.2796289)(313.79937988,88.04963186)
\curveto(313.98938226,87.82962935)(314.21438203,87.63962954)(314.47437988,87.47963186)
\curveto(314.53438171,87.43962974)(314.59438165,87.40462977)(314.65437988,87.37463186)
\curveto(314.70438154,87.34462983)(314.76938148,87.31462986)(314.84937988,87.28463186)
\curveto(314.91938133,87.26462991)(314.97938127,87.25962992)(315.02937988,87.26963186)
\curveto(315.09938115,87.28962989)(315.15438109,87.32462985)(315.19437988,87.37463186)
\curveto(315.22438102,87.42462975)(315.244381,87.48462969)(315.25437988,87.55463186)
\lineto(315.25437988,87.79463186)
\lineto(315.25437988,88.54463186)
\lineto(315.25437988,91.34963186)
\lineto(315.25437988,92.00963186)
\curveto(315.25438099,92.09962508)(315.25938099,92.18462499)(315.26937988,92.26463186)
\curveto(315.26938098,92.34462483)(315.28938096,92.40962477)(315.32937988,92.45963186)
\curveto(315.36938088,92.50962467)(315.4443808,92.54962463)(315.55437988,92.57963186)
\curveto(315.65438059,92.61962456)(315.75438049,92.62962455)(315.85437988,92.60963186)
\lineto(315.98937988,92.60963186)
\curveto(316.05938019,92.58962459)(316.11938013,92.56962461)(316.16937988,92.54963186)
\curveto(316.21938003,92.52962465)(316.25937999,92.49462468)(316.28937988,92.44463186)
\curveto(316.32937992,92.39462478)(316.3493799,92.32462485)(316.34937988,92.23463186)
\lineto(316.34937988,91.96463186)
\lineto(316.34937988,91.06463186)
\lineto(316.34937988,87.55463186)
\lineto(316.34937988,86.48963186)
\curveto(316.3493799,86.40963077)(316.35437989,86.31963086)(316.36437988,86.21963186)
\curveto(316.36437988,86.11963106)(316.35437989,86.03463114)(316.33437988,85.96463186)
\curveto(316.26437998,85.75463142)(316.08438016,85.68963149)(315.79437988,85.76963186)
\curveto(315.75438049,85.7796314)(315.71938053,85.7796314)(315.68937988,85.76963186)
\curveto(315.6493806,85.76963141)(315.60438064,85.7796314)(315.55437988,85.79963186)
\curveto(315.47438077,85.81963136)(315.38938086,85.83963134)(315.29937988,85.85963186)
\curveto(315.20938104,85.8796313)(315.12438112,85.90463127)(315.04437988,85.93463186)
\curveto(314.55438169,86.09463108)(314.13938211,86.29463088)(313.79937988,86.53463186)
\curveto(313.5493827,86.71463046)(313.32438292,86.91963026)(313.12437988,87.14963186)
\curveto(312.91438333,87.3796298)(312.71938353,87.61962956)(312.53937988,87.86963186)
\curveto(312.35938389,88.12962905)(312.18938406,88.39462878)(312.02937988,88.66463186)
\curveto(311.85938439,88.94462823)(311.68438456,89.21462796)(311.50437988,89.47463186)
\curveto(311.42438482,89.58462759)(311.3493849,89.68962749)(311.27937988,89.78963186)
\curveto(311.20938504,89.89962728)(311.13438511,90.00962717)(311.05437988,90.11963186)
\curveto(311.02438522,90.15962702)(310.99438525,90.19462698)(310.96437988,90.22463186)
\curveto(310.92438532,90.26462691)(310.89438535,90.30462687)(310.87437988,90.34463186)
\curveto(310.76438548,90.48462669)(310.63938561,90.60962657)(310.49937988,90.71963186)
\curveto(310.46938578,90.73962644)(310.4443858,90.76462641)(310.42437988,90.79463186)
\curveto(310.39438585,90.82462635)(310.36438588,90.84962633)(310.33437988,90.86963186)
\curveto(310.23438601,90.94962623)(310.13438611,91.01462616)(310.03437988,91.06463186)
\curveto(309.93438631,91.12462605)(309.82438642,91.179626)(309.70437988,91.22963186)
\curveto(309.63438661,91.25962592)(309.55938669,91.2796259)(309.47937988,91.28963186)
\lineto(309.23937988,91.34963186)
\lineto(309.14937988,91.34963186)
\curveto(309.11938713,91.35962582)(309.08938716,91.36462581)(309.05937988,91.36463186)
\curveto(308.98938726,91.38462579)(308.89438735,91.38962579)(308.77437988,91.37963186)
\curveto(308.6443876,91.3796258)(308.5443877,91.36962581)(308.47437988,91.34963186)
\curveto(308.39438785,91.32962585)(308.31938793,91.30962587)(308.24937988,91.28963186)
\curveto(308.16938808,91.2796259)(308.08938816,91.25962592)(308.00937988,91.22963186)
\curveto(307.76938848,91.11962606)(307.56938868,90.96962621)(307.40937988,90.77963186)
\curveto(307.23938901,90.59962658)(307.09938915,90.3796268)(306.98937988,90.11963186)
\curveto(306.96938928,90.04962713)(306.95438929,89.9796272)(306.94437988,89.90963186)
\curveto(306.92438932,89.83962734)(306.90438934,89.76462741)(306.88437988,89.68463186)
\curveto(306.86438938,89.60462757)(306.85438939,89.49462768)(306.85437988,89.35463186)
\curveto(306.85438939,89.22462795)(306.86438938,89.11962806)(306.88437988,89.03963186)
\curveto(306.89438935,88.9796282)(306.89938935,88.92462825)(306.89937988,88.87463186)
\curveto(306.89938935,88.82462835)(306.90938934,88.7746284)(306.92937988,88.72463186)
\curveto(306.96938928,88.62462855)(307.00938924,88.52962865)(307.04937988,88.43963186)
\curveto(307.08938916,88.35962882)(307.13438911,88.2796289)(307.18437988,88.19963186)
\curveto(307.20438904,88.16962901)(307.22938902,88.13962904)(307.25937988,88.10963186)
\curveto(307.28938896,88.08962909)(307.31438893,88.06462911)(307.33437988,88.03463186)
\lineto(307.40937988,87.95963186)
\curveto(307.42938882,87.92962925)(307.4493888,87.90462927)(307.46937988,87.88463186)
\lineto(307.67937988,87.73463186)
\curveto(307.73938851,87.69462948)(307.80438844,87.64962953)(307.87437988,87.59963186)
\curveto(307.96438828,87.53962964)(308.06938818,87.48962969)(308.18937988,87.44963186)
\curveto(308.29938795,87.41962976)(308.40938784,87.38462979)(308.51937988,87.34463186)
\curveto(308.62938762,87.30462987)(308.77438747,87.2796299)(308.95437988,87.26963186)
\curveto(309.12438712,87.25962992)(309.249387,87.22962995)(309.32937988,87.17963186)
\curveto(309.40938684,87.12963005)(309.45438679,87.05463012)(309.46437988,86.95463186)
\curveto(309.47438677,86.85463032)(309.47938677,86.74463043)(309.47937988,86.62463186)
\curveto(309.47938677,86.58463059)(309.48438676,86.54463063)(309.49437988,86.50463186)
\curveto(309.49438675,86.46463071)(309.48938676,86.42963075)(309.47937988,86.39963186)
\curveto(309.45938679,86.34963083)(309.4493868,86.29963088)(309.44937988,86.24963186)
\curveto(309.4493868,86.20963097)(309.43938681,86.16963101)(309.41937988,86.12963186)
\curveto(309.35938689,86.03963114)(309.22438702,85.99463118)(309.01437988,85.99463186)
\lineto(308.89437988,85.99463186)
\curveto(308.83438741,86.00463117)(308.77438747,86.00963117)(308.71437988,86.00963186)
\curveto(308.6443876,86.01963116)(308.57938767,86.02963115)(308.51937988,86.03963186)
\curveto(308.40938784,86.05963112)(308.30938794,86.0796311)(308.21937988,86.09963186)
\curveto(308.11938813,86.11963106)(308.02438822,86.14963103)(307.93437988,86.18963186)
\curveto(307.86438838,86.20963097)(307.80438844,86.22963095)(307.75437988,86.24963186)
\lineto(307.57437988,86.30963186)
\curveto(307.31438893,86.42963075)(307.06938918,86.58463059)(306.83937988,86.77463186)
\curveto(306.60938964,86.9746302)(306.42438982,87.18962999)(306.28437988,87.41963186)
\curveto(306.20439004,87.52962965)(306.13939011,87.64462953)(306.08937988,87.76463186)
\lineto(305.93937988,88.15463186)
\curveto(305.88939036,88.26462891)(305.85939039,88.3796288)(305.84937988,88.49963186)
\curveto(305.82939042,88.61962856)(305.80439044,88.74462843)(305.77437988,88.87463186)
\curveto(305.77439047,88.94462823)(305.77439047,89.00962817)(305.77437988,89.06963186)
\curveto(305.76439048,89.12962805)(305.75439049,89.19462798)(305.74437988,89.26463186)
}
}
{
\newrgbcolor{curcolor}{0 0 0}
\pscustom[linestyle=none,fillstyle=solid,fillcolor=curcolor]
{
\newpath
\moveto(311.26437988,101.36424123)
\lineto(311.51937988,101.36424123)
\curveto(311.59938465,101.37423353)(311.67438457,101.36923353)(311.74437988,101.34924123)
\lineto(311.98437988,101.34924123)
\lineto(312.14937988,101.34924123)
\curveto(312.249384,101.32923357)(312.35438389,101.31923358)(312.46437988,101.31924123)
\curveto(312.56438368,101.31923358)(312.66438358,101.30923359)(312.76437988,101.28924123)
\lineto(312.91437988,101.28924123)
\curveto(313.05438319,101.25923364)(313.19438305,101.23923366)(313.33437988,101.22924123)
\curveto(313.46438278,101.21923368)(313.59438265,101.19423371)(313.72437988,101.15424123)
\curveto(313.80438244,101.13423377)(313.88938236,101.11423379)(313.97937988,101.09424123)
\lineto(314.21937988,101.03424123)
\lineto(314.51937988,100.91424123)
\curveto(314.60938164,100.88423402)(314.69938155,100.84923405)(314.78937988,100.80924123)
\curveto(315.00938124,100.70923419)(315.22438102,100.57423433)(315.43437988,100.40424123)
\curveto(315.6443806,100.24423466)(315.81438043,100.06923483)(315.94437988,99.87924123)
\curveto(315.98438026,99.82923507)(316.02438022,99.76923513)(316.06437988,99.69924123)
\curveto(316.09438015,99.63923526)(316.12938012,99.57923532)(316.16937988,99.51924123)
\curveto(316.21938003,99.43923546)(316.25937999,99.34423556)(316.28937988,99.23424123)
\curveto(316.31937993,99.12423578)(316.3493799,99.01923588)(316.37937988,98.91924123)
\curveto(316.41937983,98.80923609)(316.4443798,98.6992362)(316.45437988,98.58924123)
\curveto(316.46437978,98.47923642)(316.47937977,98.36423654)(316.49937988,98.24424123)
\curveto(316.50937974,98.2042367)(316.50937974,98.15923674)(316.49937988,98.10924123)
\curveto(316.49937975,98.06923683)(316.50437974,98.02923687)(316.51437988,97.98924123)
\curveto(316.52437972,97.94923695)(316.52937972,97.89423701)(316.52937988,97.82424123)
\curveto(316.52937972,97.75423715)(316.52437972,97.7042372)(316.51437988,97.67424123)
\curveto(316.49437975,97.62423728)(316.48937976,97.57923732)(316.49937988,97.53924123)
\curveto(316.50937974,97.4992374)(316.50937974,97.46423744)(316.49937988,97.43424123)
\lineto(316.49937988,97.34424123)
\curveto(316.47937977,97.28423762)(316.46437978,97.21923768)(316.45437988,97.14924123)
\curveto(316.45437979,97.08923781)(316.4493798,97.02423788)(316.43937988,96.95424123)
\curveto(316.38937986,96.78423812)(316.33937991,96.62423828)(316.28937988,96.47424123)
\curveto(316.23938001,96.32423858)(316.17438007,96.17923872)(316.09437988,96.03924123)
\curveto(316.05438019,95.98923891)(316.02438022,95.93423897)(316.00437988,95.87424123)
\curveto(315.97438027,95.82423908)(315.93938031,95.77423913)(315.89937988,95.72424123)
\curveto(315.71938053,95.48423942)(315.49938075,95.28423962)(315.23937988,95.12424123)
\curveto(314.97938127,94.96423994)(314.69438155,94.82424008)(314.38437988,94.70424123)
\curveto(314.244382,94.64424026)(314.10438214,94.5992403)(313.96437988,94.56924123)
\curveto(313.81438243,94.53924036)(313.65938259,94.5042404)(313.49937988,94.46424123)
\curveto(313.38938286,94.44424046)(313.27938297,94.42924047)(313.16937988,94.41924123)
\curveto(313.05938319,94.40924049)(312.9493833,94.39424051)(312.83937988,94.37424123)
\curveto(312.79938345,94.36424054)(312.75938349,94.35924054)(312.71937988,94.35924123)
\curveto(312.67938357,94.36924053)(312.63938361,94.36924053)(312.59937988,94.35924123)
\curveto(312.5493837,94.34924055)(312.49938375,94.34424056)(312.44937988,94.34424123)
\lineto(312.28437988,94.34424123)
\curveto(312.23438401,94.32424058)(312.18438406,94.31924058)(312.13437988,94.32924123)
\curveto(312.07438417,94.33924056)(312.01938423,94.33924056)(311.96937988,94.32924123)
\curveto(311.92938432,94.31924058)(311.88438436,94.31924058)(311.83437988,94.32924123)
\curveto(311.78438446,94.33924056)(311.73438451,94.33424057)(311.68437988,94.31424123)
\curveto(311.61438463,94.29424061)(311.53938471,94.28924061)(311.45937988,94.29924123)
\curveto(311.36938488,94.30924059)(311.28438496,94.31424059)(311.20437988,94.31424123)
\curveto(311.11438513,94.31424059)(311.01438523,94.30924059)(310.90437988,94.29924123)
\curveto(310.78438546,94.28924061)(310.68438556,94.29424061)(310.60437988,94.31424123)
\lineto(310.31937988,94.31424123)
\lineto(309.68937988,94.35924123)
\curveto(309.58938666,94.36924053)(309.49438675,94.37924052)(309.40437988,94.38924123)
\lineto(309.10437988,94.41924123)
\curveto(309.05438719,94.43924046)(309.00438724,94.44424046)(308.95437988,94.43424123)
\curveto(308.89438735,94.43424047)(308.83938741,94.44424046)(308.78937988,94.46424123)
\curveto(308.61938763,94.51424039)(308.45438779,94.55424035)(308.29437988,94.58424123)
\curveto(308.12438812,94.61424029)(307.96438828,94.66424024)(307.81437988,94.73424123)
\curveto(307.35438889,94.92423998)(306.97938927,95.14423976)(306.68937988,95.39424123)
\curveto(306.39938985,95.65423925)(306.15439009,96.01423889)(305.95437988,96.47424123)
\curveto(305.90439034,96.6042383)(305.86939038,96.73423817)(305.84937988,96.86424123)
\curveto(305.82939042,97.0042379)(305.80439044,97.14423776)(305.77437988,97.28424123)
\curveto(305.76439048,97.35423755)(305.75939049,97.41923748)(305.75937988,97.47924123)
\curveto(305.75939049,97.53923736)(305.75439049,97.6042373)(305.74437988,97.67424123)
\curveto(305.72439052,98.5042364)(305.87439037,99.17423573)(306.19437988,99.68424123)
\curveto(306.50438974,100.19423471)(306.9443893,100.57423433)(307.51437988,100.82424123)
\curveto(307.63438861,100.87423403)(307.75938849,100.91923398)(307.88937988,100.95924123)
\curveto(308.01938823,100.9992339)(308.15438809,101.04423386)(308.29437988,101.09424123)
\curveto(308.37438787,101.11423379)(308.45938779,101.12923377)(308.54937988,101.13924123)
\lineto(308.78937988,101.19924123)
\curveto(308.89938735,101.22923367)(309.00938724,101.24423366)(309.11937988,101.24424123)
\curveto(309.22938702,101.25423365)(309.33938691,101.26923363)(309.44937988,101.28924123)
\curveto(309.49938675,101.30923359)(309.5443867,101.31423359)(309.58437988,101.30424123)
\curveto(309.62438662,101.3042336)(309.66438658,101.30923359)(309.70437988,101.31924123)
\curveto(309.75438649,101.32923357)(309.80938644,101.32923357)(309.86937988,101.31924123)
\curveto(309.91938633,101.31923358)(309.96938628,101.32423358)(310.01937988,101.33424123)
\lineto(310.15437988,101.33424123)
\curveto(310.21438603,101.35423355)(310.28438596,101.35423355)(310.36437988,101.33424123)
\curveto(310.43438581,101.32423358)(310.49938575,101.32923357)(310.55937988,101.34924123)
\curveto(310.58938566,101.35923354)(310.62938562,101.36423354)(310.67937988,101.36424123)
\lineto(310.79937988,101.36424123)
\lineto(311.26437988,101.36424123)
\moveto(313.58937988,99.81924123)
\curveto(313.26938298,99.91923498)(312.90438334,99.97923492)(312.49437988,99.99924123)
\curveto(312.08438416,100.01923488)(311.67438457,100.02923487)(311.26437988,100.02924123)
\curveto(310.83438541,100.02923487)(310.41438583,100.01923488)(310.00437988,99.99924123)
\curveto(309.59438665,99.97923492)(309.20938704,99.93423497)(308.84937988,99.86424123)
\curveto(308.48938776,99.79423511)(308.16938808,99.68423522)(307.88937988,99.53424123)
\curveto(307.59938865,99.39423551)(307.36438888,99.1992357)(307.18437988,98.94924123)
\curveto(307.07438917,98.78923611)(306.99438925,98.60923629)(306.94437988,98.40924123)
\curveto(306.88438936,98.20923669)(306.85438939,97.96423694)(306.85437988,97.67424123)
\curveto(306.87438937,97.65423725)(306.88438936,97.61923728)(306.88437988,97.56924123)
\curveto(306.87438937,97.51923738)(306.87438937,97.47923742)(306.88437988,97.44924123)
\curveto(306.90438934,97.36923753)(306.92438932,97.29423761)(306.94437988,97.22424123)
\curveto(306.95438929,97.16423774)(306.97438927,97.0992378)(307.00437988,97.02924123)
\curveto(307.12438912,96.75923814)(307.29438895,96.53923836)(307.51437988,96.36924123)
\curveto(307.72438852,96.20923869)(307.96938828,96.07423883)(308.24937988,95.96424123)
\curveto(308.35938789,95.91423899)(308.47938777,95.87423903)(308.60937988,95.84424123)
\curveto(308.72938752,95.82423908)(308.85438739,95.7992391)(308.98437988,95.76924123)
\curveto(309.03438721,95.74923915)(309.08938716,95.73923916)(309.14937988,95.73924123)
\curveto(309.19938705,95.73923916)(309.249387,95.73423917)(309.29937988,95.72424123)
\curveto(309.38938686,95.71423919)(309.48438676,95.7042392)(309.58437988,95.69424123)
\curveto(309.67438657,95.68423922)(309.76938648,95.67423923)(309.86937988,95.66424123)
\curveto(309.9493863,95.66423924)(310.03438621,95.65923924)(310.12437988,95.64924123)
\lineto(310.36437988,95.64924123)
\lineto(310.54437988,95.64924123)
\curveto(310.57438567,95.63923926)(310.60938564,95.63423927)(310.64937988,95.63424123)
\lineto(310.78437988,95.63424123)
\lineto(311.23437988,95.63424123)
\curveto(311.31438493,95.63423927)(311.39938485,95.62923927)(311.48937988,95.61924123)
\curveto(311.56938468,95.61923928)(311.6443846,95.62923927)(311.71437988,95.64924123)
\lineto(311.98437988,95.64924123)
\curveto(312.00438424,95.64923925)(312.03438421,95.64423926)(312.07437988,95.63424123)
\curveto(312.10438414,95.63423927)(312.12938412,95.63923926)(312.14937988,95.64924123)
\curveto(312.249384,95.65923924)(312.3493839,95.66423924)(312.44937988,95.66424123)
\curveto(312.53938371,95.67423923)(312.63938361,95.68423922)(312.74937988,95.69424123)
\curveto(312.86938338,95.72423918)(312.99438325,95.73923916)(313.12437988,95.73924123)
\curveto(313.244383,95.74923915)(313.35938289,95.77423913)(313.46937988,95.81424123)
\curveto(313.76938248,95.89423901)(314.03438221,95.97923892)(314.26437988,96.06924123)
\curveto(314.49438175,96.16923873)(314.70938154,96.31423859)(314.90937988,96.50424123)
\curveto(315.10938114,96.71423819)(315.25938099,96.97923792)(315.35937988,97.29924123)
\curveto(315.37938087,97.33923756)(315.38938086,97.37423753)(315.38937988,97.40424123)
\curveto(315.37938087,97.44423746)(315.38438086,97.48923741)(315.40437988,97.53924123)
\curveto(315.41438083,97.57923732)(315.42438082,97.64923725)(315.43437988,97.74924123)
\curveto(315.4443808,97.85923704)(315.43938081,97.94423696)(315.41937988,98.00424123)
\curveto(315.39938085,98.07423683)(315.38938086,98.14423676)(315.38937988,98.21424123)
\curveto(315.37938087,98.28423662)(315.36438088,98.34923655)(315.34437988,98.40924123)
\curveto(315.28438096,98.60923629)(315.19938105,98.78923611)(315.08937988,98.94924123)
\curveto(315.06938118,98.97923592)(315.0493812,99.0042359)(315.02937988,99.02424123)
\lineto(314.96937988,99.08424123)
\curveto(314.9493813,99.12423578)(314.90938134,99.17423573)(314.84937988,99.23424123)
\curveto(314.70938154,99.33423557)(314.57938167,99.41923548)(314.45937988,99.48924123)
\curveto(314.33938191,99.55923534)(314.19438205,99.62923527)(314.02437988,99.69924123)
\curveto(313.95438229,99.72923517)(313.88438236,99.74923515)(313.81437988,99.75924123)
\curveto(313.7443825,99.77923512)(313.66938258,99.7992351)(313.58937988,99.81924123)
}
}
{
\newrgbcolor{curcolor}{0 0 0}
\pscustom[linestyle=none,fillstyle=solid,fillcolor=curcolor]
{
\newpath
\moveto(305.74437988,106.77385061)
\curveto(305.7443905,106.87384575)(305.75439049,106.96884566)(305.77437988,107.05885061)
\curveto(305.78439046,107.14884548)(305.81439043,107.21384541)(305.86437988,107.25385061)
\curveto(305.9443903,107.31384531)(306.0493902,107.34384528)(306.17937988,107.34385061)
\lineto(306.56937988,107.34385061)
\lineto(308.06937988,107.34385061)
\lineto(314.45937988,107.34385061)
\lineto(315.62937988,107.34385061)
\lineto(315.94437988,107.34385061)
\curveto(316.0443802,107.35384527)(316.12438012,107.33884529)(316.18437988,107.29885061)
\curveto(316.26437998,107.24884538)(316.31437993,107.17384545)(316.33437988,107.07385061)
\curveto(316.3443799,106.98384564)(316.3493799,106.87384575)(316.34937988,106.74385061)
\lineto(316.34937988,106.51885061)
\curveto(316.32937992,106.43884619)(316.31437993,106.36884626)(316.30437988,106.30885061)
\curveto(316.28437996,106.24884638)(316.24438,106.19884643)(316.18437988,106.15885061)
\curveto(316.12438012,106.11884651)(316.0493802,106.09884653)(315.95937988,106.09885061)
\lineto(315.65937988,106.09885061)
\lineto(314.56437988,106.09885061)
\lineto(309.22437988,106.09885061)
\curveto(309.13438711,106.07884655)(309.05938719,106.06384656)(308.99937988,106.05385061)
\curveto(308.92938732,106.05384657)(308.86938738,106.0238466)(308.81937988,105.96385061)
\curveto(308.76938748,105.89384673)(308.7443875,105.80384682)(308.74437988,105.69385061)
\curveto(308.73438751,105.59384703)(308.72938752,105.48384714)(308.72937988,105.36385061)
\lineto(308.72937988,104.22385061)
\lineto(308.72937988,103.72885061)
\curveto(308.71938753,103.56884906)(308.65938759,103.45884917)(308.54937988,103.39885061)
\curveto(308.51938773,103.37884925)(308.48938776,103.36884926)(308.45937988,103.36885061)
\curveto(308.41938783,103.36884926)(308.37438787,103.36384926)(308.32437988,103.35385061)
\curveto(308.20438804,103.33384929)(308.09438815,103.33884929)(307.99437988,103.36885061)
\curveto(307.89438835,103.40884922)(307.82438842,103.46384916)(307.78437988,103.53385061)
\curveto(307.73438851,103.61384901)(307.70938854,103.73384889)(307.70937988,103.89385061)
\curveto(307.70938854,104.05384857)(307.69438855,104.18884844)(307.66437988,104.29885061)
\curveto(307.65438859,104.34884828)(307.6493886,104.40384822)(307.64937988,104.46385061)
\curveto(307.63938861,104.5238481)(307.62438862,104.58384804)(307.60437988,104.64385061)
\curveto(307.55438869,104.79384783)(307.50438874,104.93884769)(307.45437988,105.07885061)
\curveto(307.39438885,105.21884741)(307.32438892,105.35384727)(307.24437988,105.48385061)
\curveto(307.15438909,105.623847)(307.0493892,105.74384688)(306.92937988,105.84385061)
\curveto(306.80938944,105.94384668)(306.67938957,106.03884659)(306.53937988,106.12885061)
\curveto(306.43938981,106.18884644)(306.32938992,106.23384639)(306.20937988,106.26385061)
\curveto(306.08939016,106.30384632)(305.98439026,106.35384627)(305.89437988,106.41385061)
\curveto(305.83439041,106.46384616)(305.79439045,106.53384609)(305.77437988,106.62385061)
\curveto(305.76439048,106.64384598)(305.75939049,106.66884596)(305.75937988,106.69885061)
\curveto(305.75939049,106.7288459)(305.75439049,106.75384587)(305.74437988,106.77385061)
}
}
{
\newrgbcolor{curcolor}{0 0 0}
\pscustom[linestyle=none,fillstyle=solid,fillcolor=curcolor]
{
\newpath
\moveto(305.74437988,115.12345998)
\curveto(305.7443905,115.22345513)(305.75439049,115.31845503)(305.77437988,115.40845998)
\curveto(305.78439046,115.49845485)(305.81439043,115.56345479)(305.86437988,115.60345998)
\curveto(305.9443903,115.66345469)(306.0493902,115.69345466)(306.17937988,115.69345998)
\lineto(306.56937988,115.69345998)
\lineto(308.06937988,115.69345998)
\lineto(314.45937988,115.69345998)
\lineto(315.62937988,115.69345998)
\lineto(315.94437988,115.69345998)
\curveto(316.0443802,115.70345465)(316.12438012,115.68845466)(316.18437988,115.64845998)
\curveto(316.26437998,115.59845475)(316.31437993,115.52345483)(316.33437988,115.42345998)
\curveto(316.3443799,115.33345502)(316.3493799,115.22345513)(316.34937988,115.09345998)
\lineto(316.34937988,114.86845998)
\curveto(316.32937992,114.78845556)(316.31437993,114.71845563)(316.30437988,114.65845998)
\curveto(316.28437996,114.59845575)(316.24438,114.5484558)(316.18437988,114.50845998)
\curveto(316.12438012,114.46845588)(316.0493802,114.4484559)(315.95937988,114.44845998)
\lineto(315.65937988,114.44845998)
\lineto(314.56437988,114.44845998)
\lineto(309.22437988,114.44845998)
\curveto(309.13438711,114.42845592)(309.05938719,114.41345594)(308.99937988,114.40345998)
\curveto(308.92938732,114.40345595)(308.86938738,114.37345598)(308.81937988,114.31345998)
\curveto(308.76938748,114.24345611)(308.7443875,114.1534562)(308.74437988,114.04345998)
\curveto(308.73438751,113.94345641)(308.72938752,113.83345652)(308.72937988,113.71345998)
\lineto(308.72937988,112.57345998)
\lineto(308.72937988,112.07845998)
\curveto(308.71938753,111.91845843)(308.65938759,111.80845854)(308.54937988,111.74845998)
\curveto(308.51938773,111.72845862)(308.48938776,111.71845863)(308.45937988,111.71845998)
\curveto(308.41938783,111.71845863)(308.37438787,111.71345864)(308.32437988,111.70345998)
\curveto(308.20438804,111.68345867)(308.09438815,111.68845866)(307.99437988,111.71845998)
\curveto(307.89438835,111.75845859)(307.82438842,111.81345854)(307.78437988,111.88345998)
\curveto(307.73438851,111.96345839)(307.70938854,112.08345827)(307.70937988,112.24345998)
\curveto(307.70938854,112.40345795)(307.69438855,112.53845781)(307.66437988,112.64845998)
\curveto(307.65438859,112.69845765)(307.6493886,112.7534576)(307.64937988,112.81345998)
\curveto(307.63938861,112.87345748)(307.62438862,112.93345742)(307.60437988,112.99345998)
\curveto(307.55438869,113.14345721)(307.50438874,113.28845706)(307.45437988,113.42845998)
\curveto(307.39438885,113.56845678)(307.32438892,113.70345665)(307.24437988,113.83345998)
\curveto(307.15438909,113.97345638)(307.0493892,114.09345626)(306.92937988,114.19345998)
\curveto(306.80938944,114.29345606)(306.67938957,114.38845596)(306.53937988,114.47845998)
\curveto(306.43938981,114.53845581)(306.32938992,114.58345577)(306.20937988,114.61345998)
\curveto(306.08939016,114.6534557)(305.98439026,114.70345565)(305.89437988,114.76345998)
\curveto(305.83439041,114.81345554)(305.79439045,114.88345547)(305.77437988,114.97345998)
\curveto(305.76439048,114.99345536)(305.75939049,115.01845533)(305.75937988,115.04845998)
\curveto(305.75939049,115.07845527)(305.75439049,115.10345525)(305.74437988,115.12345998)
}
}
{
\newrgbcolor{curcolor}{0 0 0}
\pscustom[linestyle=none,fillstyle=solid,fillcolor=curcolor]
{
\newpath
\moveto(336.48072632,42.02236623)
\curveto(336.53072706,42.04235669)(336.590727,42.06735666)(336.66072632,42.09736623)
\curveto(336.73072686,42.1273566)(336.80572679,42.14735658)(336.88572632,42.15736623)
\curveto(336.95572664,42.17735655)(337.02572657,42.17735655)(337.09572632,42.15736623)
\curveto(337.15572644,42.14735658)(337.20072639,42.10735662)(337.23072632,42.03736623)
\curveto(337.25072634,41.98735674)(337.26072633,41.9273568)(337.26072632,41.85736623)
\lineto(337.26072632,41.64736623)
\lineto(337.26072632,41.19736623)
\curveto(337.26072633,41.04735768)(337.23572636,40.9273578)(337.18572632,40.83736623)
\curveto(337.12572647,40.73735799)(337.02072657,40.66235807)(336.87072632,40.61236623)
\curveto(336.72072687,40.57235816)(336.58572701,40.5273582)(336.46572632,40.47736623)
\curveto(336.20572739,40.36735836)(335.93572766,40.26735846)(335.65572632,40.17736623)
\curveto(335.37572822,40.08735864)(335.10072849,39.98735874)(334.83072632,39.87736623)
\curveto(334.74072885,39.84735888)(334.65572894,39.81735891)(334.57572632,39.78736623)
\curveto(334.4957291,39.76735896)(334.42072917,39.73735899)(334.35072632,39.69736623)
\curveto(334.28072931,39.66735906)(334.22072937,39.62235911)(334.17072632,39.56236623)
\curveto(334.12072947,39.50235923)(334.08072951,39.42235931)(334.05072632,39.32236623)
\curveto(334.03072956,39.27235946)(334.02572957,39.21235952)(334.03572632,39.14236623)
\lineto(334.03572632,38.94736623)
\lineto(334.03572632,36.11236623)
\lineto(334.03572632,35.81236623)
\curveto(334.02572957,35.70236303)(334.02572957,35.59736313)(334.03572632,35.49736623)
\curveto(334.04572955,35.39736333)(334.06072953,35.30236343)(334.08072632,35.21236623)
\curveto(334.10072949,35.1323636)(334.14072945,35.07236366)(334.20072632,35.03236623)
\curveto(334.30072929,34.95236378)(334.41572918,34.89236384)(334.54572632,34.85236623)
\curveto(334.66572893,34.82236391)(334.7907288,34.78236395)(334.92072632,34.73236623)
\curveto(335.15072844,34.6323641)(335.3907282,34.53736419)(335.64072632,34.44736623)
\curveto(335.8907277,34.36736436)(336.13072746,34.27736445)(336.36072632,34.17736623)
\curveto(336.42072717,34.15736457)(336.4907271,34.1323646)(336.57072632,34.10236623)
\curveto(336.64072695,34.08236465)(336.71572688,34.05736467)(336.79572632,34.02736623)
\curveto(336.87572672,33.99736473)(336.95072664,33.96236477)(337.02072632,33.92236623)
\curveto(337.08072651,33.89236484)(337.12572647,33.85736487)(337.15572632,33.81736623)
\curveto(337.21572638,33.73736499)(337.25072634,33.6273651)(337.26072632,33.48736623)
\lineto(337.26072632,33.06736623)
\lineto(337.26072632,32.82736623)
\curveto(337.25072634,32.75736597)(337.22572637,32.69736603)(337.18572632,32.64736623)
\curveto(337.15572644,32.59736613)(337.11072648,32.56736616)(337.05072632,32.55736623)
\curveto(336.9907266,32.55736617)(336.93072666,32.56236617)(336.87072632,32.57236623)
\curveto(336.80072679,32.59236614)(336.73572686,32.61236612)(336.67572632,32.63236623)
\curveto(336.60572699,32.66236607)(336.55572704,32.68736604)(336.52572632,32.70736623)
\curveto(336.20572739,32.84736588)(335.8907277,32.97236576)(335.58072632,33.08236623)
\curveto(335.26072833,33.19236554)(334.94072865,33.31236542)(334.62072632,33.44236623)
\curveto(334.40072919,33.5323652)(334.18572941,33.61736511)(333.97572632,33.69736623)
\curveto(333.75572984,33.77736495)(333.53573006,33.86236487)(333.31572632,33.95236623)
\curveto(332.595731,34.25236448)(331.87073172,34.53736419)(331.14072632,34.80736623)
\curveto(330.40073319,35.07736365)(329.66573393,35.36236337)(328.93572632,35.66236623)
\curveto(328.67573492,35.77236296)(328.41073518,35.87236286)(328.14072632,35.96236623)
\curveto(327.87073572,36.06236267)(327.60573599,36.16736256)(327.34572632,36.27736623)
\curveto(327.23573636,36.3273624)(327.11573648,36.37236236)(326.98572632,36.41236623)
\curveto(326.84573675,36.46236227)(326.74573685,36.5323622)(326.68572632,36.62236623)
\curveto(326.64573695,36.66236207)(326.61573698,36.727362)(326.59572632,36.81736623)
\curveto(326.58573701,36.83736189)(326.58573701,36.85736187)(326.59572632,36.87736623)
\curveto(326.595737,36.90736182)(326.590737,36.9323618)(326.58072632,36.95236623)
\curveto(326.58073701,37.1323616)(326.58073701,37.34236139)(326.58072632,37.58236623)
\curveto(326.57073702,37.82236091)(326.60573699,37.99736073)(326.68572632,38.10736623)
\curveto(326.74573685,38.18736054)(326.84573675,38.24736048)(326.98572632,38.28736623)
\curveto(327.11573648,38.33736039)(327.23573636,38.38736034)(327.34572632,38.43736623)
\curveto(327.57573602,38.53736019)(327.80573579,38.6273601)(328.03572632,38.70736623)
\curveto(328.26573533,38.78735994)(328.4957351,38.87735985)(328.72572632,38.97736623)
\curveto(328.92573467,39.05735967)(329.13073446,39.1323596)(329.34072632,39.20236623)
\curveto(329.55073404,39.28235945)(329.75573384,39.36735936)(329.95572632,39.45736623)
\curveto(330.68573291,39.75735897)(331.42573217,40.04235869)(332.17572632,40.31236623)
\curveto(332.91573068,40.59235814)(333.65072994,40.88735784)(334.38072632,41.19736623)
\curveto(334.47072912,41.23735749)(334.55572904,41.26735746)(334.63572632,41.28736623)
\curveto(334.71572888,41.31735741)(334.80072879,41.34735738)(334.89072632,41.37736623)
\curveto(335.15072844,41.48735724)(335.41572818,41.59235714)(335.68572632,41.69236623)
\curveto(335.95572764,41.80235693)(336.22072737,41.91235682)(336.48072632,42.02236623)
\moveto(332.83572632,38.81236623)
\curveto(332.80573079,38.90235983)(332.75573084,38.95735977)(332.68572632,38.97736623)
\curveto(332.61573098,39.00735972)(332.54073105,39.01235972)(332.46072632,38.99236623)
\curveto(332.37073122,38.98235975)(332.28573131,38.95735977)(332.20572632,38.91736623)
\curveto(332.11573148,38.88735984)(332.04073155,38.85735987)(331.98072632,38.82736623)
\curveto(331.94073165,38.80735992)(331.90573169,38.79735993)(331.87572632,38.79736623)
\curveto(331.84573175,38.79735993)(331.81073178,38.78735994)(331.77072632,38.76736623)
\lineto(331.53072632,38.67736623)
\curveto(331.44073215,38.65736007)(331.35073224,38.6273601)(331.26072632,38.58736623)
\curveto(330.90073269,38.43736029)(330.53573306,38.30236043)(330.16572632,38.18236623)
\curveto(329.78573381,38.07236066)(329.41573418,37.94236079)(329.05572632,37.79236623)
\curveto(328.94573465,37.74236099)(328.83573476,37.69736103)(328.72572632,37.65736623)
\curveto(328.61573498,37.6273611)(328.51073508,37.58736114)(328.41072632,37.53736623)
\curveto(328.36073523,37.51736121)(328.31573528,37.49236124)(328.27572632,37.46236623)
\curveto(328.22573537,37.44236129)(328.20073539,37.39236134)(328.20072632,37.31236623)
\curveto(328.22073537,37.29236144)(328.23573536,37.27236146)(328.24572632,37.25236623)
\curveto(328.25573534,37.2323615)(328.27073532,37.21236152)(328.29072632,37.19236623)
\curveto(328.34073525,37.15236158)(328.3957352,37.12236161)(328.45572632,37.10236623)
\curveto(328.50573509,37.08236165)(328.56073503,37.06236167)(328.62072632,37.04236623)
\curveto(328.73073486,36.99236174)(328.84073475,36.95236178)(328.95072632,36.92236623)
\curveto(329.06073453,36.89236184)(329.17073442,36.85236188)(329.28072632,36.80236623)
\curveto(329.67073392,36.6323621)(330.06573353,36.48236225)(330.46572632,36.35236623)
\curveto(330.86573273,36.2323625)(331.25573234,36.09236264)(331.63572632,35.93236623)
\lineto(331.78572632,35.87236623)
\curveto(331.83573176,35.86236287)(331.88573171,35.84736288)(331.93572632,35.82736623)
\lineto(332.17572632,35.73736623)
\curveto(332.25573134,35.70736302)(332.33573126,35.68236305)(332.41572632,35.66236623)
\curveto(332.46573113,35.64236309)(332.52073107,35.6323631)(332.58072632,35.63236623)
\curveto(332.64073095,35.64236309)(332.6907309,35.65736307)(332.73072632,35.67736623)
\curveto(332.81073078,35.727363)(332.85573074,35.8323629)(332.86572632,35.99236623)
\lineto(332.86572632,36.44236623)
\lineto(332.86572632,38.04736623)
\curveto(332.86573073,38.15736057)(332.87073072,38.29236044)(332.88072632,38.45236623)
\curveto(332.88073071,38.61236012)(332.86573073,38.73236)(332.83572632,38.81236623)
}
}
{
\newrgbcolor{curcolor}{0 0 0}
\pscustom[linestyle=none,fillstyle=solid,fillcolor=curcolor]
{
\newpath
\moveto(333.22572632,50.56392873)
\curveto(333.27573032,50.57392038)(333.34573025,50.57892038)(333.43572632,50.57892873)
\curveto(333.51573008,50.57892038)(333.58073001,50.57392038)(333.63072632,50.56392873)
\curveto(333.67072992,50.56392039)(333.71072988,50.5589204)(333.75072632,50.54892873)
\lineto(333.87072632,50.54892873)
\curveto(333.95072964,50.52892043)(334.03072956,50.51892044)(334.11072632,50.51892873)
\curveto(334.1907294,50.51892044)(334.27072932,50.50892045)(334.35072632,50.48892873)
\curveto(334.3907292,50.47892048)(334.43072916,50.47392048)(334.47072632,50.47392873)
\curveto(334.50072909,50.47392048)(334.53572906,50.46892049)(334.57572632,50.45892873)
\curveto(334.68572891,50.42892053)(334.7907288,50.39892056)(334.89072632,50.36892873)
\curveto(334.9907286,50.34892061)(335.0907285,50.31892064)(335.19072632,50.27892873)
\curveto(335.54072805,50.13892082)(335.85572774,49.96892099)(336.13572632,49.76892873)
\curveto(336.41572718,49.56892139)(336.65572694,49.31892164)(336.85572632,49.01892873)
\curveto(336.95572664,48.86892209)(337.04072655,48.72392223)(337.11072632,48.58392873)
\curveto(337.16072643,48.47392248)(337.20072639,48.36392259)(337.23072632,48.25392873)
\curveto(337.26072633,48.1539228)(337.2907263,48.04892291)(337.32072632,47.93892873)
\curveto(337.34072625,47.86892309)(337.35072624,47.80392315)(337.35072632,47.74392873)
\curveto(337.36072623,47.68392327)(337.37572622,47.62392333)(337.39572632,47.56392873)
\lineto(337.39572632,47.41392873)
\curveto(337.41572618,47.36392359)(337.42572617,47.28892367)(337.42572632,47.18892873)
\curveto(337.43572616,47.08892387)(337.43072616,47.00892395)(337.41072632,46.94892873)
\lineto(337.41072632,46.79892873)
\curveto(337.40072619,46.7589242)(337.3957262,46.71392424)(337.39572632,46.66392873)
\curveto(337.3957262,46.62392433)(337.3907262,46.57892438)(337.38072632,46.52892873)
\curveto(337.34072625,46.37892458)(337.30572629,46.22892473)(337.27572632,46.07892873)
\curveto(337.24572635,45.93892502)(337.20072639,45.79892516)(337.14072632,45.65892873)
\curveto(337.06072653,45.4589255)(336.96072663,45.27892568)(336.84072632,45.11892873)
\lineto(336.69072632,44.93892873)
\curveto(336.63072696,44.87892608)(336.590727,44.80892615)(336.57072632,44.72892873)
\curveto(336.56072703,44.66892629)(336.57572702,44.61892634)(336.61572632,44.57892873)
\curveto(336.64572695,44.54892641)(336.6907269,44.52392643)(336.75072632,44.50392873)
\curveto(336.81072678,44.49392646)(336.87572672,44.48392647)(336.94572632,44.47392873)
\curveto(337.00572659,44.47392648)(337.05072654,44.46392649)(337.08072632,44.44392873)
\curveto(337.13072646,44.40392655)(337.17572642,44.3589266)(337.21572632,44.30892873)
\curveto(337.23572636,44.2589267)(337.25072634,44.18892677)(337.26072632,44.09892873)
\lineto(337.26072632,43.82892873)
\curveto(337.26072633,43.73892722)(337.25572634,43.6539273)(337.24572632,43.57392873)
\curveto(337.22572637,43.49392746)(337.20572639,43.43392752)(337.18572632,43.39392873)
\curveto(337.16572643,43.37392758)(337.14072645,43.3539276)(337.11072632,43.33392873)
\lineto(337.02072632,43.27392873)
\curveto(336.94072665,43.24392771)(336.82072677,43.22892773)(336.66072632,43.22892873)
\curveto(336.50072709,43.23892772)(336.36572723,43.24392771)(336.25572632,43.24392873)
\lineto(327.45072632,43.24392873)
\curveto(327.33073626,43.24392771)(327.20573639,43.23892772)(327.07572632,43.22892873)
\curveto(326.93573666,43.22892773)(326.82573677,43.2539277)(326.74572632,43.30392873)
\curveto(326.68573691,43.34392761)(326.63573696,43.40892755)(326.59572632,43.49892873)
\curveto(326.595737,43.51892744)(326.595737,43.54392741)(326.59572632,43.57392873)
\curveto(326.58573701,43.60392735)(326.58073701,43.62892733)(326.58072632,43.64892873)
\curveto(326.57073702,43.78892717)(326.57073702,43.93392702)(326.58072632,44.08392873)
\curveto(326.58073701,44.24392671)(326.62073697,44.3539266)(326.70072632,44.41392873)
\curveto(326.78073681,44.46392649)(326.8957367,44.48892647)(327.04572632,44.48892873)
\lineto(327.45072632,44.48892873)
\lineto(329.20572632,44.48892873)
\lineto(329.46072632,44.48892873)
\lineto(329.74572632,44.48892873)
\curveto(329.83573376,44.49892646)(329.92073367,44.50892645)(330.00072632,44.51892873)
\curveto(330.07073352,44.53892642)(330.12073347,44.56892639)(330.15072632,44.60892873)
\curveto(330.18073341,44.64892631)(330.18573341,44.69392626)(330.16572632,44.74392873)
\curveto(330.14573345,44.79392616)(330.12573347,44.83392612)(330.10572632,44.86392873)
\curveto(330.06573353,44.91392604)(330.02573357,44.958926)(329.98572632,44.99892873)
\lineto(329.86572632,45.14892873)
\curveto(329.81573378,45.21892574)(329.77073382,45.28892567)(329.73072632,45.35892873)
\lineto(329.61072632,45.59892873)
\curveto(329.52073407,45.77892518)(329.45573414,45.99392496)(329.41572632,46.24392873)
\curveto(329.37573422,46.49392446)(329.35573424,46.74892421)(329.35572632,47.00892873)
\curveto(329.35573424,47.26892369)(329.38073421,47.52392343)(329.43072632,47.77392873)
\curveto(329.47073412,48.02392293)(329.53073406,48.24392271)(329.61072632,48.43392873)
\curveto(329.78073381,48.83392212)(330.01573358,49.17892178)(330.31572632,49.46892873)
\curveto(330.61573298,49.7589212)(330.96573263,49.98892097)(331.36572632,50.15892873)
\curveto(331.47573212,50.20892075)(331.58573201,50.24892071)(331.69572632,50.27892873)
\curveto(331.7957318,50.31892064)(331.90073169,50.3589206)(332.01072632,50.39892873)
\curveto(332.12073147,50.42892053)(332.23573136,50.44892051)(332.35572632,50.45892873)
\lineto(332.68572632,50.51892873)
\curveto(332.71573088,50.52892043)(332.75073084,50.53392042)(332.79072632,50.53392873)
\curveto(332.82073077,50.53392042)(332.85073074,50.53892042)(332.88072632,50.54892873)
\curveto(332.94073065,50.56892039)(333.00073059,50.56892039)(333.06072632,50.54892873)
\curveto(333.11073048,50.53892042)(333.16573043,50.54392041)(333.22572632,50.56392873)
\moveto(333.61572632,49.22892873)
\curveto(333.56573003,49.24892171)(333.50573009,49.2539217)(333.43572632,49.24392873)
\curveto(333.36573023,49.23392172)(333.30073029,49.22892173)(333.24072632,49.22892873)
\curveto(333.07073052,49.22892173)(332.91073068,49.21892174)(332.76072632,49.19892873)
\curveto(332.61073098,49.18892177)(332.47573112,49.1589218)(332.35572632,49.10892873)
\curveto(332.25573134,49.07892188)(332.16573143,49.0539219)(332.08572632,49.03392873)
\curveto(332.00573159,49.01392194)(331.92573167,48.98392197)(331.84572632,48.94392873)
\curveto(331.595732,48.83392212)(331.36573223,48.68392227)(331.15572632,48.49392873)
\curveto(330.93573266,48.30392265)(330.77073282,48.08392287)(330.66072632,47.83392873)
\curveto(330.63073296,47.7539232)(330.60573299,47.67392328)(330.58572632,47.59392873)
\curveto(330.55573304,47.52392343)(330.53073306,47.44892351)(330.51072632,47.36892873)
\curveto(330.48073311,47.2589237)(330.46573313,47.14892381)(330.46572632,47.03892873)
\curveto(330.45573314,46.92892403)(330.45073314,46.80892415)(330.45072632,46.67892873)
\curveto(330.46073313,46.62892433)(330.47073312,46.58392437)(330.48072632,46.54392873)
\lineto(330.48072632,46.40892873)
\lineto(330.54072632,46.13892873)
\curveto(330.56073303,46.0589249)(330.590733,45.97892498)(330.63072632,45.89892873)
\curveto(330.77073282,45.5589254)(330.98073261,45.28892567)(331.26072632,45.08892873)
\curveto(331.53073206,44.88892607)(331.85073174,44.72892623)(332.22072632,44.60892873)
\curveto(332.33073126,44.56892639)(332.44073115,44.54392641)(332.55072632,44.53392873)
\curveto(332.66073093,44.52392643)(332.77573082,44.50392645)(332.89572632,44.47392873)
\curveto(332.94573065,44.46392649)(332.9907306,44.46392649)(333.03072632,44.47392873)
\curveto(333.07073052,44.48392647)(333.11573048,44.47892648)(333.16572632,44.45892873)
\curveto(333.21573038,44.44892651)(333.2907303,44.44392651)(333.39072632,44.44392873)
\curveto(333.48073011,44.44392651)(333.55073004,44.44892651)(333.60072632,44.45892873)
\lineto(333.72072632,44.45892873)
\curveto(333.76072983,44.46892649)(333.80072979,44.47392648)(333.84072632,44.47392873)
\curveto(333.88072971,44.47392648)(333.91572968,44.47892648)(333.94572632,44.48892873)
\curveto(333.97572962,44.49892646)(334.01072958,44.50392645)(334.05072632,44.50392873)
\curveto(334.08072951,44.50392645)(334.11072948,44.50892645)(334.14072632,44.51892873)
\curveto(334.22072937,44.53892642)(334.30072929,44.5539264)(334.38072632,44.56392873)
\lineto(334.62072632,44.62392873)
\curveto(334.96072863,44.73392622)(335.25072834,44.88392607)(335.49072632,45.07392873)
\curveto(335.73072786,45.27392568)(335.93072766,45.51892544)(336.09072632,45.80892873)
\curveto(336.14072745,45.89892506)(336.18072741,45.99392496)(336.21072632,46.09392873)
\curveto(336.23072736,46.19392476)(336.25572734,46.29892466)(336.28572632,46.40892873)
\curveto(336.30572729,46.4589245)(336.31572728,46.50392445)(336.31572632,46.54392873)
\curveto(336.30572729,46.59392436)(336.30572729,46.64392431)(336.31572632,46.69392873)
\curveto(336.32572727,46.73392422)(336.33072726,46.77892418)(336.33072632,46.82892873)
\lineto(336.33072632,46.96392873)
\lineto(336.33072632,47.09892873)
\curveto(336.32072727,47.13892382)(336.31572728,47.17392378)(336.31572632,47.20392873)
\curveto(336.31572728,47.23392372)(336.31072728,47.26892369)(336.30072632,47.30892873)
\curveto(336.28072731,47.38892357)(336.26572733,47.46392349)(336.25572632,47.53392873)
\curveto(336.23572736,47.60392335)(336.21072738,47.67892328)(336.18072632,47.75892873)
\curveto(336.05072754,48.06892289)(335.88072771,48.31892264)(335.67072632,48.50892873)
\curveto(335.45072814,48.69892226)(335.18572841,48.8589221)(334.87572632,48.98892873)
\curveto(334.73572886,49.03892192)(334.595729,49.07392188)(334.45572632,49.09392873)
\curveto(334.30572929,49.12392183)(334.15572944,49.1589218)(334.00572632,49.19892873)
\curveto(333.95572964,49.21892174)(333.91072968,49.22392173)(333.87072632,49.21392873)
\curveto(333.82072977,49.21392174)(333.77072982,49.21892174)(333.72072632,49.22892873)
\lineto(333.61572632,49.22892873)
}
}
{
\newrgbcolor{curcolor}{0 0 0}
\pscustom[linestyle=none,fillstyle=solid,fillcolor=curcolor]
{
\newpath
\moveto(329.35572632,55.69017873)
\curveto(329.35573424,55.92017394)(329.41573418,56.05017381)(329.53572632,56.08017873)
\curveto(329.64573395,56.11017375)(329.81073378,56.12517374)(330.03072632,56.12517873)
\lineto(330.31572632,56.12517873)
\curveto(330.40573319,56.12517374)(330.48073311,56.10017376)(330.54072632,56.05017873)
\curveto(330.62073297,55.99017387)(330.66573293,55.90517396)(330.67572632,55.79517873)
\curveto(330.67573292,55.68517418)(330.6907329,55.57517429)(330.72072632,55.46517873)
\curveto(330.75073284,55.32517454)(330.78073281,55.19017467)(330.81072632,55.06017873)
\curveto(330.84073275,54.94017492)(330.88073271,54.82517504)(330.93072632,54.71517873)
\curveto(331.06073253,54.42517544)(331.24073235,54.19017567)(331.47072632,54.01017873)
\curveto(331.6907319,53.83017603)(331.94573165,53.67517619)(332.23572632,53.54517873)
\curveto(332.34573125,53.50517636)(332.46073113,53.47517639)(332.58072632,53.45517873)
\curveto(332.6907309,53.43517643)(332.80573079,53.41017645)(332.92572632,53.38017873)
\curveto(332.97573062,53.37017649)(333.02573057,53.3651765)(333.07572632,53.36517873)
\curveto(333.12573047,53.37517649)(333.17573042,53.37517649)(333.22572632,53.36517873)
\curveto(333.34573025,53.33517653)(333.48573011,53.32017654)(333.64572632,53.32017873)
\curveto(333.7957298,53.33017653)(333.94072965,53.33517653)(334.08072632,53.33517873)
\lineto(335.92572632,53.33517873)
\lineto(336.27072632,53.33517873)
\curveto(336.3907272,53.33517653)(336.50572709,53.33017653)(336.61572632,53.32017873)
\curveto(336.72572687,53.31017655)(336.82072677,53.30517656)(336.90072632,53.30517873)
\curveto(336.98072661,53.31517655)(337.05072654,53.29517657)(337.11072632,53.24517873)
\curveto(337.18072641,53.19517667)(337.22072637,53.11517675)(337.23072632,53.00517873)
\curveto(337.24072635,52.90517696)(337.24572635,52.79517707)(337.24572632,52.67517873)
\lineto(337.24572632,52.40517873)
\curveto(337.22572637,52.35517751)(337.21072638,52.30517756)(337.20072632,52.25517873)
\curveto(337.18072641,52.21517765)(337.15572644,52.18517768)(337.12572632,52.16517873)
\curveto(337.05572654,52.11517775)(336.97072662,52.08517778)(336.87072632,52.07517873)
\lineto(336.54072632,52.07517873)
\lineto(335.38572632,52.07517873)
\lineto(331.23072632,52.07517873)
\lineto(330.19572632,52.07517873)
\lineto(329.89572632,52.07517873)
\curveto(329.7957338,52.08517778)(329.71073388,52.11517775)(329.64072632,52.16517873)
\curveto(329.60073399,52.19517767)(329.57073402,52.24517762)(329.55072632,52.31517873)
\curveto(329.53073406,52.39517747)(329.52073407,52.48017738)(329.52072632,52.57017873)
\curveto(329.51073408,52.6601772)(329.51073408,52.75017711)(329.52072632,52.84017873)
\curveto(329.53073406,52.93017693)(329.54573405,53.00017686)(329.56572632,53.05017873)
\curveto(329.595734,53.13017673)(329.65573394,53.18017668)(329.74572632,53.20017873)
\curveto(329.82573377,53.23017663)(329.91573368,53.24517662)(330.01572632,53.24517873)
\lineto(330.31572632,53.24517873)
\curveto(330.41573318,53.24517662)(330.50573309,53.2651766)(330.58572632,53.30517873)
\curveto(330.60573299,53.31517655)(330.62073297,53.32517654)(330.63072632,53.33517873)
\lineto(330.67572632,53.38017873)
\curveto(330.67573292,53.49017637)(330.63073296,53.58017628)(330.54072632,53.65017873)
\curveto(330.44073315,53.72017614)(330.36073323,53.78017608)(330.30072632,53.83017873)
\lineto(330.21072632,53.92017873)
\curveto(330.10073349,54.01017585)(329.98573361,54.13517573)(329.86572632,54.29517873)
\curveto(329.74573385,54.45517541)(329.65573394,54.60517526)(329.59572632,54.74517873)
\curveto(329.54573405,54.83517503)(329.51073408,54.93017493)(329.49072632,55.03017873)
\curveto(329.46073413,55.13017473)(329.43073416,55.23517463)(329.40072632,55.34517873)
\curveto(329.3907342,55.40517446)(329.38573421,55.4651744)(329.38572632,55.52517873)
\curveto(329.37573422,55.58517428)(329.36573423,55.64017422)(329.35572632,55.69017873)
}
}
{
\newrgbcolor{curcolor}{0 0 0}
\pscustom[linestyle=none,fillstyle=solid,fillcolor=curcolor]
{
}
}
{
\newrgbcolor{curcolor}{0 0 0}
\pscustom[linestyle=none,fillstyle=solid,fillcolor=curcolor]
{
\newpath
\moveto(332.17572632,67.99510061)
\lineto(332.43072632,67.99510061)
\curveto(332.51073108,68.0050929)(332.58573101,68.00009291)(332.65572632,67.98010061)
\lineto(332.89572632,67.98010061)
\lineto(333.06072632,67.98010061)
\curveto(333.16073043,67.96009295)(333.26573033,67.95009296)(333.37572632,67.95010061)
\curveto(333.47573012,67.95009296)(333.57573002,67.94009297)(333.67572632,67.92010061)
\lineto(333.82572632,67.92010061)
\curveto(333.96572963,67.89009302)(334.10572949,67.87009304)(334.24572632,67.86010061)
\curveto(334.37572922,67.85009306)(334.50572909,67.82509308)(334.63572632,67.78510061)
\curveto(334.71572888,67.76509314)(334.80072879,67.74509316)(334.89072632,67.72510061)
\lineto(335.13072632,67.66510061)
\lineto(335.43072632,67.54510061)
\curveto(335.52072807,67.51509339)(335.61072798,67.48009343)(335.70072632,67.44010061)
\curveto(335.92072767,67.34009357)(336.13572746,67.2050937)(336.34572632,67.03510061)
\curveto(336.55572704,66.87509403)(336.72572687,66.70009421)(336.85572632,66.51010061)
\curveto(336.8957267,66.46009445)(336.93572666,66.40009451)(336.97572632,66.33010061)
\curveto(337.00572659,66.27009464)(337.04072655,66.2100947)(337.08072632,66.15010061)
\curveto(337.13072646,66.07009484)(337.17072642,65.97509493)(337.20072632,65.86510061)
\curveto(337.23072636,65.75509515)(337.26072633,65.65009526)(337.29072632,65.55010061)
\curveto(337.33072626,65.44009547)(337.35572624,65.33009558)(337.36572632,65.22010061)
\curveto(337.37572622,65.1100958)(337.3907262,64.99509591)(337.41072632,64.87510061)
\curveto(337.42072617,64.83509607)(337.42072617,64.79009612)(337.41072632,64.74010061)
\curveto(337.41072618,64.70009621)(337.41572618,64.66009625)(337.42572632,64.62010061)
\curveto(337.43572616,64.58009633)(337.44072615,64.52509638)(337.44072632,64.45510061)
\curveto(337.44072615,64.38509652)(337.43572616,64.33509657)(337.42572632,64.30510061)
\curveto(337.40572619,64.25509665)(337.40072619,64.2100967)(337.41072632,64.17010061)
\curveto(337.42072617,64.13009678)(337.42072617,64.09509681)(337.41072632,64.06510061)
\lineto(337.41072632,63.97510061)
\curveto(337.3907262,63.91509699)(337.37572622,63.85009706)(337.36572632,63.78010061)
\curveto(337.36572623,63.72009719)(337.36072623,63.65509725)(337.35072632,63.58510061)
\curveto(337.30072629,63.41509749)(337.25072634,63.25509765)(337.20072632,63.10510061)
\curveto(337.15072644,62.95509795)(337.08572651,62.8100981)(337.00572632,62.67010061)
\curveto(336.96572663,62.62009829)(336.93572666,62.56509834)(336.91572632,62.50510061)
\curveto(336.88572671,62.45509845)(336.85072674,62.4050985)(336.81072632,62.35510061)
\curveto(336.63072696,62.11509879)(336.41072718,61.91509899)(336.15072632,61.75510061)
\curveto(335.8907277,61.59509931)(335.60572799,61.45509945)(335.29572632,61.33510061)
\curveto(335.15572844,61.27509963)(335.01572858,61.23009968)(334.87572632,61.20010061)
\curveto(334.72572887,61.17009974)(334.57072902,61.13509977)(334.41072632,61.09510061)
\curveto(334.30072929,61.07509983)(334.1907294,61.06009985)(334.08072632,61.05010061)
\curveto(333.97072962,61.04009987)(333.86072973,61.02509988)(333.75072632,61.00510061)
\curveto(333.71072988,60.99509991)(333.67072992,60.99009992)(333.63072632,60.99010061)
\curveto(333.59073,61.00009991)(333.55073004,61.00009991)(333.51072632,60.99010061)
\curveto(333.46073013,60.98009993)(333.41073018,60.97509993)(333.36072632,60.97510061)
\lineto(333.19572632,60.97510061)
\curveto(333.14573045,60.95509995)(333.0957305,60.95009996)(333.04572632,60.96010061)
\curveto(332.98573061,60.97009994)(332.93073066,60.97009994)(332.88072632,60.96010061)
\curveto(332.84073075,60.95009996)(332.7957308,60.95009996)(332.74572632,60.96010061)
\curveto(332.6957309,60.97009994)(332.64573095,60.96509994)(332.59572632,60.94510061)
\curveto(332.52573107,60.92509998)(332.45073114,60.92009999)(332.37072632,60.93010061)
\curveto(332.28073131,60.94009997)(332.1957314,60.94509996)(332.11572632,60.94510061)
\curveto(332.02573157,60.94509996)(331.92573167,60.94009997)(331.81572632,60.93010061)
\curveto(331.6957319,60.92009999)(331.595732,60.92509998)(331.51572632,60.94510061)
\lineto(331.23072632,60.94510061)
\lineto(330.60072632,60.99010061)
\curveto(330.50073309,61.00009991)(330.40573319,61.0100999)(330.31572632,61.02010061)
\lineto(330.01572632,61.05010061)
\curveto(329.96573363,61.07009984)(329.91573368,61.07509983)(329.86572632,61.06510061)
\curveto(329.80573379,61.06509984)(329.75073384,61.07509983)(329.70072632,61.09510061)
\curveto(329.53073406,61.14509976)(329.36573423,61.18509972)(329.20572632,61.21510061)
\curveto(329.03573456,61.24509966)(328.87573472,61.29509961)(328.72572632,61.36510061)
\curveto(328.26573533,61.55509935)(327.8907357,61.77509913)(327.60072632,62.02510061)
\curveto(327.31073628,62.28509862)(327.06573653,62.64509826)(326.86572632,63.10510061)
\curveto(326.81573678,63.23509767)(326.78073681,63.36509754)(326.76072632,63.49510061)
\curveto(326.74073685,63.63509727)(326.71573688,63.77509713)(326.68572632,63.91510061)
\curveto(326.67573692,63.98509692)(326.67073692,64.05009686)(326.67072632,64.11010061)
\curveto(326.67073692,64.17009674)(326.66573693,64.23509667)(326.65572632,64.30510061)
\curveto(326.63573696,65.13509577)(326.78573681,65.8050951)(327.10572632,66.31510061)
\curveto(327.41573618,66.82509408)(327.85573574,67.2050937)(328.42572632,67.45510061)
\curveto(328.54573505,67.5050934)(328.67073492,67.55009336)(328.80072632,67.59010061)
\curveto(328.93073466,67.63009328)(329.06573453,67.67509323)(329.20572632,67.72510061)
\curveto(329.28573431,67.74509316)(329.37073422,67.76009315)(329.46072632,67.77010061)
\lineto(329.70072632,67.83010061)
\curveto(329.81073378,67.86009305)(329.92073367,67.87509303)(330.03072632,67.87510061)
\curveto(330.14073345,67.88509302)(330.25073334,67.90009301)(330.36072632,67.92010061)
\curveto(330.41073318,67.94009297)(330.45573314,67.94509296)(330.49572632,67.93510061)
\curveto(330.53573306,67.93509297)(330.57573302,67.94009297)(330.61572632,67.95010061)
\curveto(330.66573293,67.96009295)(330.72073287,67.96009295)(330.78072632,67.95010061)
\curveto(330.83073276,67.95009296)(330.88073271,67.95509295)(330.93072632,67.96510061)
\lineto(331.06572632,67.96510061)
\curveto(331.12573247,67.98509292)(331.1957324,67.98509292)(331.27572632,67.96510061)
\curveto(331.34573225,67.95509295)(331.41073218,67.96009295)(331.47072632,67.98010061)
\curveto(331.50073209,67.99009292)(331.54073205,67.99509291)(331.59072632,67.99510061)
\lineto(331.71072632,67.99510061)
\lineto(332.17572632,67.99510061)
\moveto(334.50072632,66.45010061)
\curveto(334.18072941,66.55009436)(333.81572978,66.6100943)(333.40572632,66.63010061)
\curveto(332.9957306,66.65009426)(332.58573101,66.66009425)(332.17572632,66.66010061)
\curveto(331.74573185,66.66009425)(331.32573227,66.65009426)(330.91572632,66.63010061)
\curveto(330.50573309,66.6100943)(330.12073347,66.56509434)(329.76072632,66.49510061)
\curveto(329.40073419,66.42509448)(329.08073451,66.31509459)(328.80072632,66.16510061)
\curveto(328.51073508,66.02509488)(328.27573532,65.83009508)(328.09572632,65.58010061)
\curveto(327.98573561,65.42009549)(327.90573569,65.24009567)(327.85572632,65.04010061)
\curveto(327.7957358,64.84009607)(327.76573583,64.59509631)(327.76572632,64.30510061)
\curveto(327.78573581,64.28509662)(327.7957358,64.25009666)(327.79572632,64.20010061)
\curveto(327.78573581,64.15009676)(327.78573581,64.1100968)(327.79572632,64.08010061)
\curveto(327.81573578,64.00009691)(327.83573576,63.92509698)(327.85572632,63.85510061)
\curveto(327.86573573,63.79509711)(327.88573571,63.73009718)(327.91572632,63.66010061)
\curveto(328.03573556,63.39009752)(328.20573539,63.17009774)(328.42572632,63.00010061)
\curveto(328.63573496,62.84009807)(328.88073471,62.7050982)(329.16072632,62.59510061)
\curveto(329.27073432,62.54509836)(329.3907342,62.5050984)(329.52072632,62.47510061)
\curveto(329.64073395,62.45509845)(329.76573383,62.43009848)(329.89572632,62.40010061)
\curveto(329.94573365,62.38009853)(330.00073359,62.37009854)(330.06072632,62.37010061)
\curveto(330.11073348,62.37009854)(330.16073343,62.36509854)(330.21072632,62.35510061)
\curveto(330.30073329,62.34509856)(330.3957332,62.33509857)(330.49572632,62.32510061)
\curveto(330.58573301,62.31509859)(330.68073291,62.3050986)(330.78072632,62.29510061)
\curveto(330.86073273,62.29509861)(330.94573265,62.29009862)(331.03572632,62.28010061)
\lineto(331.27572632,62.28010061)
\lineto(331.45572632,62.28010061)
\curveto(331.48573211,62.27009864)(331.52073207,62.26509864)(331.56072632,62.26510061)
\lineto(331.69572632,62.26510061)
\lineto(332.14572632,62.26510061)
\curveto(332.22573137,62.26509864)(332.31073128,62.26009865)(332.40072632,62.25010061)
\curveto(332.48073111,62.25009866)(332.55573104,62.26009865)(332.62572632,62.28010061)
\lineto(332.89572632,62.28010061)
\curveto(332.91573068,62.28009863)(332.94573065,62.27509863)(332.98572632,62.26510061)
\curveto(333.01573058,62.26509864)(333.04073055,62.27009864)(333.06072632,62.28010061)
\curveto(333.16073043,62.29009862)(333.26073033,62.29509861)(333.36072632,62.29510061)
\curveto(333.45073014,62.3050986)(333.55073004,62.31509859)(333.66072632,62.32510061)
\curveto(333.78072981,62.35509855)(333.90572969,62.37009854)(334.03572632,62.37010061)
\curveto(334.15572944,62.38009853)(334.27072932,62.4050985)(334.38072632,62.44510061)
\curveto(334.68072891,62.52509838)(334.94572865,62.6100983)(335.17572632,62.70010061)
\curveto(335.40572819,62.80009811)(335.62072797,62.94509796)(335.82072632,63.13510061)
\curveto(336.02072757,63.34509756)(336.17072742,63.6100973)(336.27072632,63.93010061)
\curveto(336.2907273,63.97009694)(336.30072729,64.0050969)(336.30072632,64.03510061)
\curveto(336.2907273,64.07509683)(336.2957273,64.12009679)(336.31572632,64.17010061)
\curveto(336.32572727,64.2100967)(336.33572726,64.28009663)(336.34572632,64.38010061)
\curveto(336.35572724,64.49009642)(336.35072724,64.57509633)(336.33072632,64.63510061)
\curveto(336.31072728,64.7050962)(336.30072729,64.77509613)(336.30072632,64.84510061)
\curveto(336.2907273,64.91509599)(336.27572732,64.98009593)(336.25572632,65.04010061)
\curveto(336.1957274,65.24009567)(336.11072748,65.42009549)(336.00072632,65.58010061)
\curveto(335.98072761,65.6100953)(335.96072763,65.63509527)(335.94072632,65.65510061)
\lineto(335.88072632,65.71510061)
\curveto(335.86072773,65.75509515)(335.82072777,65.8050951)(335.76072632,65.86510061)
\curveto(335.62072797,65.96509494)(335.4907281,66.05009486)(335.37072632,66.12010061)
\curveto(335.25072834,66.19009472)(335.10572849,66.26009465)(334.93572632,66.33010061)
\curveto(334.86572873,66.36009455)(334.7957288,66.38009453)(334.72572632,66.39010061)
\curveto(334.65572894,66.4100945)(334.58072901,66.43009448)(334.50072632,66.45010061)
}
}
{
\newrgbcolor{curcolor}{0 0 0}
\pscustom[linestyle=none,fillstyle=solid,fillcolor=curcolor]
{
\newpath
\moveto(326.65572632,72.45970998)
\curveto(326.62573697,74.08970454)(327.18073641,75.13970349)(328.32072632,75.60970998)
\curveto(328.55073504,75.70970292)(328.84073475,75.77470286)(329.19072632,75.80470998)
\curveto(329.53073406,75.84470279)(329.84073375,75.81970281)(330.12072632,75.72970998)
\curveto(330.38073321,75.63970299)(330.60573299,75.51970311)(330.79572632,75.36970998)
\curveto(330.83573276,75.34970328)(330.87073272,75.32470331)(330.90072632,75.29470998)
\curveto(330.92073267,75.26470337)(330.94573265,75.23970339)(330.97572632,75.21970998)
\lineto(331.09572632,75.12970998)
\curveto(331.12573247,75.09970353)(331.15073244,75.06470357)(331.17072632,75.02470998)
\curveto(331.22073237,74.97470366)(331.26573233,74.91970371)(331.30572632,74.85970998)
\curveto(331.34573225,74.80970382)(331.3957322,74.76470387)(331.45572632,74.72470998)
\curveto(331.4957321,74.68470395)(331.54573205,74.66970396)(331.60572632,74.67970998)
\curveto(331.65573194,74.68970394)(331.70073189,74.71970391)(331.74072632,74.76970998)
\curveto(331.78073181,74.81970381)(331.82073177,74.87470376)(331.86072632,74.93470998)
\curveto(331.8907317,75.00470363)(331.92073167,75.06970356)(331.95072632,75.12970998)
\curveto(331.98073161,75.18970344)(332.01073158,75.23970339)(332.04072632,75.27970998)
\curveto(332.26073133,75.59970303)(332.57073102,75.85470278)(332.97072632,76.04470998)
\curveto(333.06073053,76.08470255)(333.15573044,76.11470252)(333.25572632,76.13470998)
\curveto(333.34573025,76.16470247)(333.43573016,76.18970244)(333.52572632,76.20970998)
\curveto(333.57573002,76.21970241)(333.62572997,76.22470241)(333.67572632,76.22470998)
\curveto(333.71572988,76.2347024)(333.76072983,76.24470239)(333.81072632,76.25470998)
\curveto(333.86072973,76.26470237)(333.91072968,76.26470237)(333.96072632,76.25470998)
\curveto(334.01072958,76.24470239)(334.06072953,76.24970238)(334.11072632,76.26970998)
\curveto(334.16072943,76.27970235)(334.22072937,76.28470235)(334.29072632,76.28470998)
\curveto(334.36072923,76.28470235)(334.42072917,76.27470236)(334.47072632,76.25470998)
\lineto(334.69572632,76.25470998)
\lineto(334.93572632,76.19470998)
\curveto(335.00572859,76.18470245)(335.07572852,76.16970246)(335.14572632,76.14970998)
\curveto(335.23572836,76.11970251)(335.32072827,76.08970254)(335.40072632,76.05970998)
\curveto(335.48072811,76.03970259)(335.56072803,76.00970262)(335.64072632,75.96970998)
\curveto(335.70072789,75.94970268)(335.76072783,75.91970271)(335.82072632,75.87970998)
\curveto(335.87072772,75.84970278)(335.92072767,75.81470282)(335.97072632,75.77470998)
\curveto(336.28072731,75.57470306)(336.54072705,75.32470331)(336.75072632,75.02470998)
\curveto(336.95072664,74.72470391)(337.11572648,74.37970425)(337.24572632,73.98970998)
\curveto(337.28572631,73.86970476)(337.31072628,73.73970489)(337.32072632,73.59970998)
\curveto(337.34072625,73.46970516)(337.36572623,73.3347053)(337.39572632,73.19470998)
\curveto(337.40572619,73.12470551)(337.41072618,73.05470558)(337.41072632,72.98470998)
\curveto(337.41072618,72.92470571)(337.41572618,72.85970577)(337.42572632,72.78970998)
\curveto(337.43572616,72.74970588)(337.44072615,72.68970594)(337.44072632,72.60970998)
\curveto(337.44072615,72.53970609)(337.43572616,72.48970614)(337.42572632,72.45970998)
\curveto(337.41572618,72.40970622)(337.41072618,72.36470627)(337.41072632,72.32470998)
\lineto(337.41072632,72.20470998)
\curveto(337.3907262,72.10470653)(337.37572622,72.00470663)(337.36572632,71.90470998)
\curveto(337.35572624,71.80470683)(337.34072625,71.70970692)(337.32072632,71.61970998)
\curveto(337.2907263,71.50970712)(337.26572633,71.39970723)(337.24572632,71.28970998)
\curveto(337.21572638,71.18970744)(337.17572642,71.08470755)(337.12572632,70.97470998)
\curveto(336.96572663,70.60470803)(336.76572683,70.28970834)(336.52572632,70.02970998)
\curveto(336.27572732,69.76970886)(335.96572763,69.55970907)(335.59572632,69.39970998)
\curveto(335.50572809,69.35970927)(335.41072818,69.32470931)(335.31072632,69.29470998)
\curveto(335.21072838,69.26470937)(335.10572849,69.2347094)(334.99572632,69.20470998)
\curveto(334.94572865,69.18470945)(334.8957287,69.17470946)(334.84572632,69.17470998)
\curveto(334.78572881,69.17470946)(334.72572887,69.16470947)(334.66572632,69.14470998)
\curveto(334.60572899,69.12470951)(334.52572907,69.11470952)(334.42572632,69.11470998)
\curveto(334.32572927,69.11470952)(334.25072934,69.1297095)(334.20072632,69.15970998)
\curveto(334.17072942,69.16970946)(334.14572945,69.18470945)(334.12572632,69.20470998)
\lineto(334.06572632,69.26470998)
\curveto(334.04572955,69.30470933)(334.03072956,69.36470927)(334.02072632,69.44470998)
\curveto(334.01072958,69.5347091)(334.00572959,69.62470901)(334.00572632,69.71470998)
\curveto(334.00572959,69.80470883)(334.01072958,69.88970874)(334.02072632,69.96970998)
\curveto(334.03072956,70.05970857)(334.04072955,70.12470851)(334.05072632,70.16470998)
\curveto(334.07072952,70.18470845)(334.08572951,70.20470843)(334.09572632,70.22470998)
\curveto(334.0957295,70.24470839)(334.10572949,70.26470837)(334.12572632,70.28470998)
\curveto(334.21572938,70.35470828)(334.33072926,70.39470824)(334.47072632,70.40470998)
\curveto(334.61072898,70.42470821)(334.73572886,70.45470818)(334.84572632,70.49470998)
\lineto(335.20572632,70.64470998)
\curveto(335.31572828,70.69470794)(335.42072817,70.75970787)(335.52072632,70.83970998)
\curveto(335.55072804,70.85970777)(335.57572802,70.87970775)(335.59572632,70.89970998)
\curveto(335.61572798,70.9297077)(335.64072795,70.95470768)(335.67072632,70.97470998)
\curveto(335.73072786,71.01470762)(335.77572782,71.04970758)(335.80572632,71.07970998)
\curveto(335.83572776,71.11970751)(335.86572773,71.15470748)(335.89572632,71.18470998)
\curveto(335.92572767,71.22470741)(335.95572764,71.26970736)(335.98572632,71.31970998)
\curveto(336.04572755,71.40970722)(336.0957275,71.50470713)(336.13572632,71.60470998)
\lineto(336.25572632,71.93470998)
\curveto(336.30572729,72.08470655)(336.33572726,72.28470635)(336.34572632,72.53470998)
\curveto(336.35572724,72.78470585)(336.33572726,72.99470564)(336.28572632,73.16470998)
\curveto(336.26572733,73.24470539)(336.25072734,73.31470532)(336.24072632,73.37470998)
\lineto(336.18072632,73.58470998)
\curveto(336.06072753,73.86470477)(335.91072768,74.10470453)(335.73072632,74.30470998)
\curveto(335.55072804,74.51470412)(335.32072827,74.67970395)(335.04072632,74.79970998)
\curveto(334.97072862,74.8297038)(334.90072869,74.84970378)(334.83072632,74.85970998)
\lineto(334.59072632,74.91970998)
\curveto(334.45072914,74.95970367)(334.2907293,74.96970366)(334.11072632,74.94970998)
\curveto(333.92072967,74.9297037)(333.77072982,74.89970373)(333.66072632,74.85970998)
\curveto(333.28073031,74.7297039)(332.9907306,74.54470409)(332.79072632,74.30470998)
\curveto(332.590731,74.07470456)(332.43073116,73.76470487)(332.31072632,73.37470998)
\curveto(332.28073131,73.26470537)(332.26073133,73.14470549)(332.25072632,73.01470998)
\curveto(332.24073135,72.89470574)(332.23573136,72.76970586)(332.23572632,72.63970998)
\curveto(332.23573136,72.47970615)(332.23073136,72.33970629)(332.22072632,72.21970998)
\curveto(332.21073138,72.09970653)(332.15073144,72.01470662)(332.04072632,71.96470998)
\curveto(332.01073158,71.94470669)(331.97573162,71.9347067)(331.93572632,71.93470998)
\lineto(331.80072632,71.93470998)
\curveto(331.70073189,71.92470671)(331.60573199,71.92470671)(331.51572632,71.93470998)
\curveto(331.42573217,71.95470668)(331.36073223,71.99470664)(331.32072632,72.05470998)
\curveto(331.2907323,72.09470654)(331.27073232,72.1347065)(331.26072632,72.17470998)
\curveto(331.25073234,72.22470641)(331.24073235,72.27970635)(331.23072632,72.33970998)
\curveto(331.22073237,72.35970627)(331.22073237,72.38470625)(331.23072632,72.41470998)
\curveto(331.23073236,72.44470619)(331.22573237,72.46970616)(331.21572632,72.48970998)
\lineto(331.21572632,72.62470998)
\curveto(331.1957324,72.7347059)(331.18573241,72.8347058)(331.18572632,72.92470998)
\curveto(331.17573242,73.02470561)(331.15573244,73.11970551)(331.12572632,73.20970998)
\curveto(331.01573258,73.5297051)(330.87073272,73.78470485)(330.69072632,73.97470998)
\curveto(330.51073308,74.16470447)(330.26073333,74.31470432)(329.94072632,74.42470998)
\curveto(329.84073375,74.45470418)(329.71573388,74.47470416)(329.56572632,74.48470998)
\curveto(329.40573419,74.50470413)(329.26073433,74.49970413)(329.13072632,74.46970998)
\curveto(329.06073453,74.44970418)(328.9957346,74.4297042)(328.93572632,74.40970998)
\curveto(328.86573473,74.39970423)(328.80073479,74.37970425)(328.74072632,74.34970998)
\curveto(328.50073509,74.24970438)(328.31073528,74.10470453)(328.17072632,73.91470998)
\curveto(328.03073556,73.72470491)(327.92073567,73.49970513)(327.84072632,73.23970998)
\curveto(327.82073577,73.17970545)(327.81073578,73.11970551)(327.81072632,73.05970998)
\curveto(327.81073578,72.99970563)(327.80073579,72.9347057)(327.78072632,72.86470998)
\curveto(327.76073583,72.78470585)(327.75073584,72.68970594)(327.75072632,72.57970998)
\curveto(327.75073584,72.46970616)(327.76073583,72.37470626)(327.78072632,72.29470998)
\curveto(327.80073579,72.24470639)(327.81073578,72.19470644)(327.81072632,72.14470998)
\curveto(327.81073578,72.10470653)(327.82073577,72.05970657)(327.84072632,72.00970998)
\curveto(327.8907357,71.8297068)(327.96573563,71.65970697)(328.06572632,71.49970998)
\curveto(328.15573544,71.34970728)(328.27073532,71.21970741)(328.41072632,71.10970998)
\curveto(328.53073506,71.01970761)(328.66073493,70.93970769)(328.80072632,70.86970998)
\curveto(328.94073465,70.79970783)(329.0957345,70.7347079)(329.26572632,70.67470998)
\curveto(329.37573422,70.64470799)(329.4957341,70.62470801)(329.62572632,70.61470998)
\curveto(329.74573385,70.60470803)(329.84573375,70.56970806)(329.92572632,70.50970998)
\curveto(329.96573363,70.48970814)(330.00573359,70.4297082)(330.04572632,70.32970998)
\curveto(330.05573354,70.28970834)(330.06573353,70.2297084)(330.07572632,70.14970998)
\lineto(330.07572632,69.89470998)
\curveto(330.06573353,69.80470883)(330.05573354,69.71970891)(330.04572632,69.63970998)
\curveto(330.03573356,69.56970906)(330.02073357,69.51970911)(330.00072632,69.48970998)
\curveto(329.97073362,69.44970918)(329.91573368,69.41470922)(329.83572632,69.38470998)
\curveto(329.75573384,69.35470928)(329.67073392,69.34970928)(329.58072632,69.36970998)
\curveto(329.53073406,69.37970925)(329.48073411,69.38470925)(329.43072632,69.38470998)
\lineto(329.25072632,69.41470998)
\curveto(329.15073444,69.44470919)(329.05073454,69.46970916)(328.95072632,69.48970998)
\curveto(328.85073474,69.51970911)(328.76073483,69.55470908)(328.68072632,69.59470998)
\curveto(328.57073502,69.64470899)(328.46573513,69.68970894)(328.36572632,69.72970998)
\curveto(328.25573534,69.76970886)(328.15073544,69.81970881)(328.05072632,69.87970998)
\curveto(327.51073608,70.20970842)(327.11573648,70.67970795)(326.86572632,71.28970998)
\curveto(326.81573678,71.40970722)(326.78073681,71.5347071)(326.76072632,71.66470998)
\curveto(326.74073685,71.80470683)(326.71573688,71.94470669)(326.68572632,72.08470998)
\curveto(326.67573692,72.14470649)(326.67073692,72.20470643)(326.67072632,72.26470998)
\curveto(326.67073692,72.3347063)(326.66573693,72.39970623)(326.65572632,72.45970998)
}
}
{
\newrgbcolor{curcolor}{0 0 0}
\pscustom[linestyle=none,fillstyle=solid,fillcolor=curcolor]
{
\newpath
\moveto(335.62572632,78.63431936)
\lineto(335.62572632,79.26431936)
\lineto(335.62572632,79.45931936)
\curveto(335.62572797,79.52931683)(335.63572796,79.58931677)(335.65572632,79.63931936)
\curveto(335.6957279,79.70931665)(335.73572786,79.7593166)(335.77572632,79.78931936)
\curveto(335.82572777,79.82931653)(335.8907277,79.84931651)(335.97072632,79.84931936)
\curveto(336.05072754,79.8593165)(336.13572746,79.86431649)(336.22572632,79.86431936)
\lineto(336.94572632,79.86431936)
\curveto(337.42572617,79.86431649)(337.83572576,79.80431655)(338.17572632,79.68431936)
\curveto(338.51572508,79.56431679)(338.7907248,79.36931699)(339.00072632,79.09931936)
\curveto(339.05072454,79.02931733)(339.0957245,78.9593174)(339.13572632,78.88931936)
\curveto(339.18572441,78.82931753)(339.23072436,78.7543176)(339.27072632,78.66431936)
\curveto(339.28072431,78.64431771)(339.2907243,78.61431774)(339.30072632,78.57431936)
\curveto(339.32072427,78.53431782)(339.32572427,78.48931787)(339.31572632,78.43931936)
\curveto(339.28572431,78.34931801)(339.21072438,78.29431806)(339.09072632,78.27431936)
\curveto(338.98072461,78.2543181)(338.88572471,78.26931809)(338.80572632,78.31931936)
\curveto(338.73572486,78.34931801)(338.67072492,78.39431796)(338.61072632,78.45431936)
\curveto(338.56072503,78.52431783)(338.51072508,78.58931777)(338.46072632,78.64931936)
\curveto(338.41072518,78.71931764)(338.33572526,78.77931758)(338.23572632,78.82931936)
\curveto(338.14572545,78.88931747)(338.05572554,78.93931742)(337.96572632,78.97931936)
\curveto(337.93572566,78.99931736)(337.87572572,79.02431733)(337.78572632,79.05431936)
\curveto(337.70572589,79.08431727)(337.63572596,79.08931727)(337.57572632,79.06931936)
\curveto(337.43572616,79.03931732)(337.34572625,78.97931738)(337.30572632,78.88931936)
\curveto(337.27572632,78.80931755)(337.26072633,78.71931764)(337.26072632,78.61931936)
\curveto(337.26072633,78.51931784)(337.23572636,78.43431792)(337.18572632,78.36431936)
\curveto(337.11572648,78.27431808)(336.97572662,78.22931813)(336.76572632,78.22931936)
\lineto(336.21072632,78.22931936)
\lineto(335.98572632,78.22931936)
\curveto(335.90572769,78.23931812)(335.84072775,78.2593181)(335.79072632,78.28931936)
\curveto(335.71072788,78.34931801)(335.66572793,78.41931794)(335.65572632,78.49931936)
\curveto(335.64572795,78.51931784)(335.64072795,78.53931782)(335.64072632,78.55931936)
\curveto(335.64072795,78.58931777)(335.63572796,78.61431774)(335.62572632,78.63431936)
}
}
{
\newrgbcolor{curcolor}{0 0 0}
\pscustom[linestyle=none,fillstyle=solid,fillcolor=curcolor]
{
}
}
{
\newrgbcolor{curcolor}{0 0 0}
\pscustom[linestyle=none,fillstyle=solid,fillcolor=curcolor]
{
\newpath
\moveto(326.65572632,89.26463186)
\curveto(326.64573695,89.95462722)(326.76573683,90.55462662)(327.01572632,91.06463186)
\curveto(327.26573633,91.58462559)(327.60073599,91.9796252)(328.02072632,92.24963186)
\curveto(328.10073549,92.29962488)(328.1907354,92.34462483)(328.29072632,92.38463186)
\curveto(328.38073521,92.42462475)(328.47573512,92.46962471)(328.57572632,92.51963186)
\curveto(328.67573492,92.55962462)(328.77573482,92.58962459)(328.87572632,92.60963186)
\curveto(328.97573462,92.62962455)(329.08073451,92.64962453)(329.19072632,92.66963186)
\curveto(329.24073435,92.68962449)(329.28573431,92.69462448)(329.32572632,92.68463186)
\curveto(329.36573423,92.6746245)(329.41073418,92.6796245)(329.46072632,92.69963186)
\curveto(329.51073408,92.70962447)(329.595734,92.71462446)(329.71572632,92.71463186)
\curveto(329.82573377,92.71462446)(329.91073368,92.70962447)(329.97072632,92.69963186)
\curveto(330.03073356,92.6796245)(330.0907335,92.66962451)(330.15072632,92.66963186)
\curveto(330.21073338,92.6796245)(330.27073332,92.6746245)(330.33072632,92.65463186)
\curveto(330.47073312,92.61462456)(330.60573299,92.5796246)(330.73572632,92.54963186)
\curveto(330.86573273,92.51962466)(330.9907326,92.4796247)(331.11072632,92.42963186)
\curveto(331.25073234,92.36962481)(331.37573222,92.29962488)(331.48572632,92.21963186)
\curveto(331.595732,92.14962503)(331.70573189,92.0746251)(331.81572632,91.99463186)
\lineto(331.87572632,91.93463186)
\curveto(331.8957317,91.92462525)(331.91573168,91.90962527)(331.93572632,91.88963186)
\curveto(332.0957315,91.76962541)(332.24073135,91.63462554)(332.37072632,91.48463186)
\curveto(332.50073109,91.33462584)(332.62573097,91.174626)(332.74572632,91.00463186)
\curveto(332.96573063,90.69462648)(333.17073042,90.39962678)(333.36072632,90.11963186)
\curveto(333.50073009,89.88962729)(333.63572996,89.65962752)(333.76572632,89.42963186)
\curveto(333.8957297,89.20962797)(334.03072956,88.98962819)(334.17072632,88.76963186)
\curveto(334.34072925,88.51962866)(334.52072907,88.2796289)(334.71072632,88.04963186)
\curveto(334.90072869,87.82962935)(335.12572847,87.63962954)(335.38572632,87.47963186)
\curveto(335.44572815,87.43962974)(335.50572809,87.40462977)(335.56572632,87.37463186)
\curveto(335.61572798,87.34462983)(335.68072791,87.31462986)(335.76072632,87.28463186)
\curveto(335.83072776,87.26462991)(335.8907277,87.25962992)(335.94072632,87.26963186)
\curveto(336.01072758,87.28962989)(336.06572753,87.32462985)(336.10572632,87.37463186)
\curveto(336.13572746,87.42462975)(336.15572744,87.48462969)(336.16572632,87.55463186)
\lineto(336.16572632,87.79463186)
\lineto(336.16572632,88.54463186)
\lineto(336.16572632,91.34963186)
\lineto(336.16572632,92.00963186)
\curveto(336.16572743,92.09962508)(336.17072742,92.18462499)(336.18072632,92.26463186)
\curveto(336.18072741,92.34462483)(336.20072739,92.40962477)(336.24072632,92.45963186)
\curveto(336.28072731,92.50962467)(336.35572724,92.54962463)(336.46572632,92.57963186)
\curveto(336.56572703,92.61962456)(336.66572693,92.62962455)(336.76572632,92.60963186)
\lineto(336.90072632,92.60963186)
\curveto(336.97072662,92.58962459)(337.03072656,92.56962461)(337.08072632,92.54963186)
\curveto(337.13072646,92.52962465)(337.17072642,92.49462468)(337.20072632,92.44463186)
\curveto(337.24072635,92.39462478)(337.26072633,92.32462485)(337.26072632,92.23463186)
\lineto(337.26072632,91.96463186)
\lineto(337.26072632,91.06463186)
\lineto(337.26072632,87.55463186)
\lineto(337.26072632,86.48963186)
\curveto(337.26072633,86.40963077)(337.26572633,86.31963086)(337.27572632,86.21963186)
\curveto(337.27572632,86.11963106)(337.26572633,86.03463114)(337.24572632,85.96463186)
\curveto(337.17572642,85.75463142)(336.9957266,85.68963149)(336.70572632,85.76963186)
\curveto(336.66572693,85.7796314)(336.63072696,85.7796314)(336.60072632,85.76963186)
\curveto(336.56072703,85.76963141)(336.51572708,85.7796314)(336.46572632,85.79963186)
\curveto(336.38572721,85.81963136)(336.30072729,85.83963134)(336.21072632,85.85963186)
\curveto(336.12072747,85.8796313)(336.03572756,85.90463127)(335.95572632,85.93463186)
\curveto(335.46572813,86.09463108)(335.05072854,86.29463088)(334.71072632,86.53463186)
\curveto(334.46072913,86.71463046)(334.23572936,86.91963026)(334.03572632,87.14963186)
\curveto(333.82572977,87.3796298)(333.63072996,87.61962956)(333.45072632,87.86963186)
\curveto(333.27073032,88.12962905)(333.10073049,88.39462878)(332.94072632,88.66463186)
\curveto(332.77073082,88.94462823)(332.595731,89.21462796)(332.41572632,89.47463186)
\curveto(332.33573126,89.58462759)(332.26073133,89.68962749)(332.19072632,89.78963186)
\curveto(332.12073147,89.89962728)(332.04573155,90.00962717)(331.96572632,90.11963186)
\curveto(331.93573166,90.15962702)(331.90573169,90.19462698)(331.87572632,90.22463186)
\curveto(331.83573176,90.26462691)(331.80573179,90.30462687)(331.78572632,90.34463186)
\curveto(331.67573192,90.48462669)(331.55073204,90.60962657)(331.41072632,90.71963186)
\curveto(331.38073221,90.73962644)(331.35573224,90.76462641)(331.33572632,90.79463186)
\curveto(331.30573229,90.82462635)(331.27573232,90.84962633)(331.24572632,90.86963186)
\curveto(331.14573245,90.94962623)(331.04573255,91.01462616)(330.94572632,91.06463186)
\curveto(330.84573275,91.12462605)(330.73573286,91.179626)(330.61572632,91.22963186)
\curveto(330.54573305,91.25962592)(330.47073312,91.2796259)(330.39072632,91.28963186)
\lineto(330.15072632,91.34963186)
\lineto(330.06072632,91.34963186)
\curveto(330.03073356,91.35962582)(330.00073359,91.36462581)(329.97072632,91.36463186)
\curveto(329.90073369,91.38462579)(329.80573379,91.38962579)(329.68572632,91.37963186)
\curveto(329.55573404,91.3796258)(329.45573414,91.36962581)(329.38572632,91.34963186)
\curveto(329.30573429,91.32962585)(329.23073436,91.30962587)(329.16072632,91.28963186)
\curveto(329.08073451,91.2796259)(329.00073459,91.25962592)(328.92072632,91.22963186)
\curveto(328.68073491,91.11962606)(328.48073511,90.96962621)(328.32072632,90.77963186)
\curveto(328.15073544,90.59962658)(328.01073558,90.3796268)(327.90072632,90.11963186)
\curveto(327.88073571,90.04962713)(327.86573573,89.9796272)(327.85572632,89.90963186)
\curveto(327.83573576,89.83962734)(327.81573578,89.76462741)(327.79572632,89.68463186)
\curveto(327.77573582,89.60462757)(327.76573583,89.49462768)(327.76572632,89.35463186)
\curveto(327.76573583,89.22462795)(327.77573582,89.11962806)(327.79572632,89.03963186)
\curveto(327.80573579,88.9796282)(327.81073578,88.92462825)(327.81072632,88.87463186)
\curveto(327.81073578,88.82462835)(327.82073577,88.7746284)(327.84072632,88.72463186)
\curveto(327.88073571,88.62462855)(327.92073567,88.52962865)(327.96072632,88.43963186)
\curveto(328.00073559,88.35962882)(328.04573555,88.2796289)(328.09572632,88.19963186)
\curveto(328.11573548,88.16962901)(328.14073545,88.13962904)(328.17072632,88.10963186)
\curveto(328.20073539,88.08962909)(328.22573537,88.06462911)(328.24572632,88.03463186)
\lineto(328.32072632,87.95963186)
\curveto(328.34073525,87.92962925)(328.36073523,87.90462927)(328.38072632,87.88463186)
\lineto(328.59072632,87.73463186)
\curveto(328.65073494,87.69462948)(328.71573488,87.64962953)(328.78572632,87.59963186)
\curveto(328.87573472,87.53962964)(328.98073461,87.48962969)(329.10072632,87.44963186)
\curveto(329.21073438,87.41962976)(329.32073427,87.38462979)(329.43072632,87.34463186)
\curveto(329.54073405,87.30462987)(329.68573391,87.2796299)(329.86572632,87.26963186)
\curveto(330.03573356,87.25962992)(330.16073343,87.22962995)(330.24072632,87.17963186)
\curveto(330.32073327,87.12963005)(330.36573323,87.05463012)(330.37572632,86.95463186)
\curveto(330.38573321,86.85463032)(330.3907332,86.74463043)(330.39072632,86.62463186)
\curveto(330.3907332,86.58463059)(330.3957332,86.54463063)(330.40572632,86.50463186)
\curveto(330.40573319,86.46463071)(330.40073319,86.42963075)(330.39072632,86.39963186)
\curveto(330.37073322,86.34963083)(330.36073323,86.29963088)(330.36072632,86.24963186)
\curveto(330.36073323,86.20963097)(330.35073324,86.16963101)(330.33072632,86.12963186)
\curveto(330.27073332,86.03963114)(330.13573346,85.99463118)(329.92572632,85.99463186)
\lineto(329.80572632,85.99463186)
\curveto(329.74573385,86.00463117)(329.68573391,86.00963117)(329.62572632,86.00963186)
\curveto(329.55573404,86.01963116)(329.4907341,86.02963115)(329.43072632,86.03963186)
\curveto(329.32073427,86.05963112)(329.22073437,86.0796311)(329.13072632,86.09963186)
\curveto(329.03073456,86.11963106)(328.93573466,86.14963103)(328.84572632,86.18963186)
\curveto(328.77573482,86.20963097)(328.71573488,86.22963095)(328.66572632,86.24963186)
\lineto(328.48572632,86.30963186)
\curveto(328.22573537,86.42963075)(327.98073561,86.58463059)(327.75072632,86.77463186)
\curveto(327.52073607,86.9746302)(327.33573626,87.18962999)(327.19572632,87.41963186)
\curveto(327.11573648,87.52962965)(327.05073654,87.64462953)(327.00072632,87.76463186)
\lineto(326.85072632,88.15463186)
\curveto(326.80073679,88.26462891)(326.77073682,88.3796288)(326.76072632,88.49963186)
\curveto(326.74073685,88.61962856)(326.71573688,88.74462843)(326.68572632,88.87463186)
\curveto(326.68573691,88.94462823)(326.68573691,89.00962817)(326.68572632,89.06963186)
\curveto(326.67573692,89.12962805)(326.66573693,89.19462798)(326.65572632,89.26463186)
}
}
{
\newrgbcolor{curcolor}{0 0 0}
\pscustom[linestyle=none,fillstyle=solid,fillcolor=curcolor]
{
\newpath
\moveto(332.17572632,101.36424123)
\lineto(332.43072632,101.36424123)
\curveto(332.51073108,101.37423353)(332.58573101,101.36923353)(332.65572632,101.34924123)
\lineto(332.89572632,101.34924123)
\lineto(333.06072632,101.34924123)
\curveto(333.16073043,101.32923357)(333.26573033,101.31923358)(333.37572632,101.31924123)
\curveto(333.47573012,101.31923358)(333.57573002,101.30923359)(333.67572632,101.28924123)
\lineto(333.82572632,101.28924123)
\curveto(333.96572963,101.25923364)(334.10572949,101.23923366)(334.24572632,101.22924123)
\curveto(334.37572922,101.21923368)(334.50572909,101.19423371)(334.63572632,101.15424123)
\curveto(334.71572888,101.13423377)(334.80072879,101.11423379)(334.89072632,101.09424123)
\lineto(335.13072632,101.03424123)
\lineto(335.43072632,100.91424123)
\curveto(335.52072807,100.88423402)(335.61072798,100.84923405)(335.70072632,100.80924123)
\curveto(335.92072767,100.70923419)(336.13572746,100.57423433)(336.34572632,100.40424123)
\curveto(336.55572704,100.24423466)(336.72572687,100.06923483)(336.85572632,99.87924123)
\curveto(336.8957267,99.82923507)(336.93572666,99.76923513)(336.97572632,99.69924123)
\curveto(337.00572659,99.63923526)(337.04072655,99.57923532)(337.08072632,99.51924123)
\curveto(337.13072646,99.43923546)(337.17072642,99.34423556)(337.20072632,99.23424123)
\curveto(337.23072636,99.12423578)(337.26072633,99.01923588)(337.29072632,98.91924123)
\curveto(337.33072626,98.80923609)(337.35572624,98.6992362)(337.36572632,98.58924123)
\curveto(337.37572622,98.47923642)(337.3907262,98.36423654)(337.41072632,98.24424123)
\curveto(337.42072617,98.2042367)(337.42072617,98.15923674)(337.41072632,98.10924123)
\curveto(337.41072618,98.06923683)(337.41572618,98.02923687)(337.42572632,97.98924123)
\curveto(337.43572616,97.94923695)(337.44072615,97.89423701)(337.44072632,97.82424123)
\curveto(337.44072615,97.75423715)(337.43572616,97.7042372)(337.42572632,97.67424123)
\curveto(337.40572619,97.62423728)(337.40072619,97.57923732)(337.41072632,97.53924123)
\curveto(337.42072617,97.4992374)(337.42072617,97.46423744)(337.41072632,97.43424123)
\lineto(337.41072632,97.34424123)
\curveto(337.3907262,97.28423762)(337.37572622,97.21923768)(337.36572632,97.14924123)
\curveto(337.36572623,97.08923781)(337.36072623,97.02423788)(337.35072632,96.95424123)
\curveto(337.30072629,96.78423812)(337.25072634,96.62423828)(337.20072632,96.47424123)
\curveto(337.15072644,96.32423858)(337.08572651,96.17923872)(337.00572632,96.03924123)
\curveto(336.96572663,95.98923891)(336.93572666,95.93423897)(336.91572632,95.87424123)
\curveto(336.88572671,95.82423908)(336.85072674,95.77423913)(336.81072632,95.72424123)
\curveto(336.63072696,95.48423942)(336.41072718,95.28423962)(336.15072632,95.12424123)
\curveto(335.8907277,94.96423994)(335.60572799,94.82424008)(335.29572632,94.70424123)
\curveto(335.15572844,94.64424026)(335.01572858,94.5992403)(334.87572632,94.56924123)
\curveto(334.72572887,94.53924036)(334.57072902,94.5042404)(334.41072632,94.46424123)
\curveto(334.30072929,94.44424046)(334.1907294,94.42924047)(334.08072632,94.41924123)
\curveto(333.97072962,94.40924049)(333.86072973,94.39424051)(333.75072632,94.37424123)
\curveto(333.71072988,94.36424054)(333.67072992,94.35924054)(333.63072632,94.35924123)
\curveto(333.59073,94.36924053)(333.55073004,94.36924053)(333.51072632,94.35924123)
\curveto(333.46073013,94.34924055)(333.41073018,94.34424056)(333.36072632,94.34424123)
\lineto(333.19572632,94.34424123)
\curveto(333.14573045,94.32424058)(333.0957305,94.31924058)(333.04572632,94.32924123)
\curveto(332.98573061,94.33924056)(332.93073066,94.33924056)(332.88072632,94.32924123)
\curveto(332.84073075,94.31924058)(332.7957308,94.31924058)(332.74572632,94.32924123)
\curveto(332.6957309,94.33924056)(332.64573095,94.33424057)(332.59572632,94.31424123)
\curveto(332.52573107,94.29424061)(332.45073114,94.28924061)(332.37072632,94.29924123)
\curveto(332.28073131,94.30924059)(332.1957314,94.31424059)(332.11572632,94.31424123)
\curveto(332.02573157,94.31424059)(331.92573167,94.30924059)(331.81572632,94.29924123)
\curveto(331.6957319,94.28924061)(331.595732,94.29424061)(331.51572632,94.31424123)
\lineto(331.23072632,94.31424123)
\lineto(330.60072632,94.35924123)
\curveto(330.50073309,94.36924053)(330.40573319,94.37924052)(330.31572632,94.38924123)
\lineto(330.01572632,94.41924123)
\curveto(329.96573363,94.43924046)(329.91573368,94.44424046)(329.86572632,94.43424123)
\curveto(329.80573379,94.43424047)(329.75073384,94.44424046)(329.70072632,94.46424123)
\curveto(329.53073406,94.51424039)(329.36573423,94.55424035)(329.20572632,94.58424123)
\curveto(329.03573456,94.61424029)(328.87573472,94.66424024)(328.72572632,94.73424123)
\curveto(328.26573533,94.92423998)(327.8907357,95.14423976)(327.60072632,95.39424123)
\curveto(327.31073628,95.65423925)(327.06573653,96.01423889)(326.86572632,96.47424123)
\curveto(326.81573678,96.6042383)(326.78073681,96.73423817)(326.76072632,96.86424123)
\curveto(326.74073685,97.0042379)(326.71573688,97.14423776)(326.68572632,97.28424123)
\curveto(326.67573692,97.35423755)(326.67073692,97.41923748)(326.67072632,97.47924123)
\curveto(326.67073692,97.53923736)(326.66573693,97.6042373)(326.65572632,97.67424123)
\curveto(326.63573696,98.5042364)(326.78573681,99.17423573)(327.10572632,99.68424123)
\curveto(327.41573618,100.19423471)(327.85573574,100.57423433)(328.42572632,100.82424123)
\curveto(328.54573505,100.87423403)(328.67073492,100.91923398)(328.80072632,100.95924123)
\curveto(328.93073466,100.9992339)(329.06573453,101.04423386)(329.20572632,101.09424123)
\curveto(329.28573431,101.11423379)(329.37073422,101.12923377)(329.46072632,101.13924123)
\lineto(329.70072632,101.19924123)
\curveto(329.81073378,101.22923367)(329.92073367,101.24423366)(330.03072632,101.24424123)
\curveto(330.14073345,101.25423365)(330.25073334,101.26923363)(330.36072632,101.28924123)
\curveto(330.41073318,101.30923359)(330.45573314,101.31423359)(330.49572632,101.30424123)
\curveto(330.53573306,101.3042336)(330.57573302,101.30923359)(330.61572632,101.31924123)
\curveto(330.66573293,101.32923357)(330.72073287,101.32923357)(330.78072632,101.31924123)
\curveto(330.83073276,101.31923358)(330.88073271,101.32423358)(330.93072632,101.33424123)
\lineto(331.06572632,101.33424123)
\curveto(331.12573247,101.35423355)(331.1957324,101.35423355)(331.27572632,101.33424123)
\curveto(331.34573225,101.32423358)(331.41073218,101.32923357)(331.47072632,101.34924123)
\curveto(331.50073209,101.35923354)(331.54073205,101.36423354)(331.59072632,101.36424123)
\lineto(331.71072632,101.36424123)
\lineto(332.17572632,101.36424123)
\moveto(334.50072632,99.81924123)
\curveto(334.18072941,99.91923498)(333.81572978,99.97923492)(333.40572632,99.99924123)
\curveto(332.9957306,100.01923488)(332.58573101,100.02923487)(332.17572632,100.02924123)
\curveto(331.74573185,100.02923487)(331.32573227,100.01923488)(330.91572632,99.99924123)
\curveto(330.50573309,99.97923492)(330.12073347,99.93423497)(329.76072632,99.86424123)
\curveto(329.40073419,99.79423511)(329.08073451,99.68423522)(328.80072632,99.53424123)
\curveto(328.51073508,99.39423551)(328.27573532,99.1992357)(328.09572632,98.94924123)
\curveto(327.98573561,98.78923611)(327.90573569,98.60923629)(327.85572632,98.40924123)
\curveto(327.7957358,98.20923669)(327.76573583,97.96423694)(327.76572632,97.67424123)
\curveto(327.78573581,97.65423725)(327.7957358,97.61923728)(327.79572632,97.56924123)
\curveto(327.78573581,97.51923738)(327.78573581,97.47923742)(327.79572632,97.44924123)
\curveto(327.81573578,97.36923753)(327.83573576,97.29423761)(327.85572632,97.22424123)
\curveto(327.86573573,97.16423774)(327.88573571,97.0992378)(327.91572632,97.02924123)
\curveto(328.03573556,96.75923814)(328.20573539,96.53923836)(328.42572632,96.36924123)
\curveto(328.63573496,96.20923869)(328.88073471,96.07423883)(329.16072632,95.96424123)
\curveto(329.27073432,95.91423899)(329.3907342,95.87423903)(329.52072632,95.84424123)
\curveto(329.64073395,95.82423908)(329.76573383,95.7992391)(329.89572632,95.76924123)
\curveto(329.94573365,95.74923915)(330.00073359,95.73923916)(330.06072632,95.73924123)
\curveto(330.11073348,95.73923916)(330.16073343,95.73423917)(330.21072632,95.72424123)
\curveto(330.30073329,95.71423919)(330.3957332,95.7042392)(330.49572632,95.69424123)
\curveto(330.58573301,95.68423922)(330.68073291,95.67423923)(330.78072632,95.66424123)
\curveto(330.86073273,95.66423924)(330.94573265,95.65923924)(331.03572632,95.64924123)
\lineto(331.27572632,95.64924123)
\lineto(331.45572632,95.64924123)
\curveto(331.48573211,95.63923926)(331.52073207,95.63423927)(331.56072632,95.63424123)
\lineto(331.69572632,95.63424123)
\lineto(332.14572632,95.63424123)
\curveto(332.22573137,95.63423927)(332.31073128,95.62923927)(332.40072632,95.61924123)
\curveto(332.48073111,95.61923928)(332.55573104,95.62923927)(332.62572632,95.64924123)
\lineto(332.89572632,95.64924123)
\curveto(332.91573068,95.64923925)(332.94573065,95.64423926)(332.98572632,95.63424123)
\curveto(333.01573058,95.63423927)(333.04073055,95.63923926)(333.06072632,95.64924123)
\curveto(333.16073043,95.65923924)(333.26073033,95.66423924)(333.36072632,95.66424123)
\curveto(333.45073014,95.67423923)(333.55073004,95.68423922)(333.66072632,95.69424123)
\curveto(333.78072981,95.72423918)(333.90572969,95.73923916)(334.03572632,95.73924123)
\curveto(334.15572944,95.74923915)(334.27072932,95.77423913)(334.38072632,95.81424123)
\curveto(334.68072891,95.89423901)(334.94572865,95.97923892)(335.17572632,96.06924123)
\curveto(335.40572819,96.16923873)(335.62072797,96.31423859)(335.82072632,96.50424123)
\curveto(336.02072757,96.71423819)(336.17072742,96.97923792)(336.27072632,97.29924123)
\curveto(336.2907273,97.33923756)(336.30072729,97.37423753)(336.30072632,97.40424123)
\curveto(336.2907273,97.44423746)(336.2957273,97.48923741)(336.31572632,97.53924123)
\curveto(336.32572727,97.57923732)(336.33572726,97.64923725)(336.34572632,97.74924123)
\curveto(336.35572724,97.85923704)(336.35072724,97.94423696)(336.33072632,98.00424123)
\curveto(336.31072728,98.07423683)(336.30072729,98.14423676)(336.30072632,98.21424123)
\curveto(336.2907273,98.28423662)(336.27572732,98.34923655)(336.25572632,98.40924123)
\curveto(336.1957274,98.60923629)(336.11072748,98.78923611)(336.00072632,98.94924123)
\curveto(335.98072761,98.97923592)(335.96072763,99.0042359)(335.94072632,99.02424123)
\lineto(335.88072632,99.08424123)
\curveto(335.86072773,99.12423578)(335.82072777,99.17423573)(335.76072632,99.23424123)
\curveto(335.62072797,99.33423557)(335.4907281,99.41923548)(335.37072632,99.48924123)
\curveto(335.25072834,99.55923534)(335.10572849,99.62923527)(334.93572632,99.69924123)
\curveto(334.86572873,99.72923517)(334.7957288,99.74923515)(334.72572632,99.75924123)
\curveto(334.65572894,99.77923512)(334.58072901,99.7992351)(334.50072632,99.81924123)
}
}
{
\newrgbcolor{curcolor}{0 0 0}
\pscustom[linestyle=none,fillstyle=solid,fillcolor=curcolor]
{
\newpath
\moveto(326.65572632,106.77385061)
\curveto(326.65573694,106.87384575)(326.66573693,106.96884566)(326.68572632,107.05885061)
\curveto(326.6957369,107.14884548)(326.72573687,107.21384541)(326.77572632,107.25385061)
\curveto(326.85573674,107.31384531)(326.96073663,107.34384528)(327.09072632,107.34385061)
\lineto(327.48072632,107.34385061)
\lineto(328.98072632,107.34385061)
\lineto(335.37072632,107.34385061)
\lineto(336.54072632,107.34385061)
\lineto(336.85572632,107.34385061)
\curveto(336.95572664,107.35384527)(337.03572656,107.33884529)(337.09572632,107.29885061)
\curveto(337.17572642,107.24884538)(337.22572637,107.17384545)(337.24572632,107.07385061)
\curveto(337.25572634,106.98384564)(337.26072633,106.87384575)(337.26072632,106.74385061)
\lineto(337.26072632,106.51885061)
\curveto(337.24072635,106.43884619)(337.22572637,106.36884626)(337.21572632,106.30885061)
\curveto(337.1957264,106.24884638)(337.15572644,106.19884643)(337.09572632,106.15885061)
\curveto(337.03572656,106.11884651)(336.96072663,106.09884653)(336.87072632,106.09885061)
\lineto(336.57072632,106.09885061)
\lineto(335.47572632,106.09885061)
\lineto(330.13572632,106.09885061)
\curveto(330.04573355,106.07884655)(329.97073362,106.06384656)(329.91072632,106.05385061)
\curveto(329.84073375,106.05384657)(329.78073381,106.0238466)(329.73072632,105.96385061)
\curveto(329.68073391,105.89384673)(329.65573394,105.80384682)(329.65572632,105.69385061)
\curveto(329.64573395,105.59384703)(329.64073395,105.48384714)(329.64072632,105.36385061)
\lineto(329.64072632,104.22385061)
\lineto(329.64072632,103.72885061)
\curveto(329.63073396,103.56884906)(329.57073402,103.45884917)(329.46072632,103.39885061)
\curveto(329.43073416,103.37884925)(329.40073419,103.36884926)(329.37072632,103.36885061)
\curveto(329.33073426,103.36884926)(329.28573431,103.36384926)(329.23572632,103.35385061)
\curveto(329.11573448,103.33384929)(329.00573459,103.33884929)(328.90572632,103.36885061)
\curveto(328.80573479,103.40884922)(328.73573486,103.46384916)(328.69572632,103.53385061)
\curveto(328.64573495,103.61384901)(328.62073497,103.73384889)(328.62072632,103.89385061)
\curveto(328.62073497,104.05384857)(328.60573499,104.18884844)(328.57572632,104.29885061)
\curveto(328.56573503,104.34884828)(328.56073503,104.40384822)(328.56072632,104.46385061)
\curveto(328.55073504,104.5238481)(328.53573506,104.58384804)(328.51572632,104.64385061)
\curveto(328.46573513,104.79384783)(328.41573518,104.93884769)(328.36572632,105.07885061)
\curveto(328.30573529,105.21884741)(328.23573536,105.35384727)(328.15572632,105.48385061)
\curveto(328.06573553,105.623847)(327.96073563,105.74384688)(327.84072632,105.84385061)
\curveto(327.72073587,105.94384668)(327.590736,106.03884659)(327.45072632,106.12885061)
\curveto(327.35073624,106.18884644)(327.24073635,106.23384639)(327.12072632,106.26385061)
\curveto(327.00073659,106.30384632)(326.8957367,106.35384627)(326.80572632,106.41385061)
\curveto(326.74573685,106.46384616)(326.70573689,106.53384609)(326.68572632,106.62385061)
\curveto(326.67573692,106.64384598)(326.67073692,106.66884596)(326.67072632,106.69885061)
\curveto(326.67073692,106.7288459)(326.66573693,106.75384587)(326.65572632,106.77385061)
}
}
{
\newrgbcolor{curcolor}{0 0 0}
\pscustom[linestyle=none,fillstyle=solid,fillcolor=curcolor]
{
\newpath
\moveto(326.65572632,115.12345998)
\curveto(326.65573694,115.22345513)(326.66573693,115.31845503)(326.68572632,115.40845998)
\curveto(326.6957369,115.49845485)(326.72573687,115.56345479)(326.77572632,115.60345998)
\curveto(326.85573674,115.66345469)(326.96073663,115.69345466)(327.09072632,115.69345998)
\lineto(327.48072632,115.69345998)
\lineto(328.98072632,115.69345998)
\lineto(335.37072632,115.69345998)
\lineto(336.54072632,115.69345998)
\lineto(336.85572632,115.69345998)
\curveto(336.95572664,115.70345465)(337.03572656,115.68845466)(337.09572632,115.64845998)
\curveto(337.17572642,115.59845475)(337.22572637,115.52345483)(337.24572632,115.42345998)
\curveto(337.25572634,115.33345502)(337.26072633,115.22345513)(337.26072632,115.09345998)
\lineto(337.26072632,114.86845998)
\curveto(337.24072635,114.78845556)(337.22572637,114.71845563)(337.21572632,114.65845998)
\curveto(337.1957264,114.59845575)(337.15572644,114.5484558)(337.09572632,114.50845998)
\curveto(337.03572656,114.46845588)(336.96072663,114.4484559)(336.87072632,114.44845998)
\lineto(336.57072632,114.44845998)
\lineto(335.47572632,114.44845998)
\lineto(330.13572632,114.44845998)
\curveto(330.04573355,114.42845592)(329.97073362,114.41345594)(329.91072632,114.40345998)
\curveto(329.84073375,114.40345595)(329.78073381,114.37345598)(329.73072632,114.31345998)
\curveto(329.68073391,114.24345611)(329.65573394,114.1534562)(329.65572632,114.04345998)
\curveto(329.64573395,113.94345641)(329.64073395,113.83345652)(329.64072632,113.71345998)
\lineto(329.64072632,112.57345998)
\lineto(329.64072632,112.07845998)
\curveto(329.63073396,111.91845843)(329.57073402,111.80845854)(329.46072632,111.74845998)
\curveto(329.43073416,111.72845862)(329.40073419,111.71845863)(329.37072632,111.71845998)
\curveto(329.33073426,111.71845863)(329.28573431,111.71345864)(329.23572632,111.70345998)
\curveto(329.11573448,111.68345867)(329.00573459,111.68845866)(328.90572632,111.71845998)
\curveto(328.80573479,111.75845859)(328.73573486,111.81345854)(328.69572632,111.88345998)
\curveto(328.64573495,111.96345839)(328.62073497,112.08345827)(328.62072632,112.24345998)
\curveto(328.62073497,112.40345795)(328.60573499,112.53845781)(328.57572632,112.64845998)
\curveto(328.56573503,112.69845765)(328.56073503,112.7534576)(328.56072632,112.81345998)
\curveto(328.55073504,112.87345748)(328.53573506,112.93345742)(328.51572632,112.99345998)
\curveto(328.46573513,113.14345721)(328.41573518,113.28845706)(328.36572632,113.42845998)
\curveto(328.30573529,113.56845678)(328.23573536,113.70345665)(328.15572632,113.83345998)
\curveto(328.06573553,113.97345638)(327.96073563,114.09345626)(327.84072632,114.19345998)
\curveto(327.72073587,114.29345606)(327.590736,114.38845596)(327.45072632,114.47845998)
\curveto(327.35073624,114.53845581)(327.24073635,114.58345577)(327.12072632,114.61345998)
\curveto(327.00073659,114.6534557)(326.8957367,114.70345565)(326.80572632,114.76345998)
\curveto(326.74573685,114.81345554)(326.70573689,114.88345547)(326.68572632,114.97345998)
\curveto(326.67573692,114.99345536)(326.67073692,115.01845533)(326.67072632,115.04845998)
\curveto(326.67073692,115.07845527)(326.66573693,115.10345525)(326.65572632,115.12345998)
}
}
{
\newrgbcolor{curcolor}{0 0 0}
\pscustom[linestyle=none,fillstyle=solid,fillcolor=curcolor]
{
\newpath
\moveto(357.39207275,42.02236623)
\curveto(357.4420735,42.04235669)(357.50207344,42.06735666)(357.57207275,42.09736623)
\curveto(357.6420733,42.1273566)(357.71707322,42.14735658)(357.79707275,42.15736623)
\curveto(357.86707307,42.17735655)(357.937073,42.17735655)(358.00707275,42.15736623)
\curveto(358.06707287,42.14735658)(358.11207283,42.10735662)(358.14207275,42.03736623)
\curveto(358.16207278,41.98735674)(358.17207277,41.9273568)(358.17207275,41.85736623)
\lineto(358.17207275,41.64736623)
\lineto(358.17207275,41.19736623)
\curveto(358.17207277,41.04735768)(358.14707279,40.9273578)(358.09707275,40.83736623)
\curveto(358.0370729,40.73735799)(357.93207301,40.66235807)(357.78207275,40.61236623)
\curveto(357.63207331,40.57235816)(357.49707344,40.5273582)(357.37707275,40.47736623)
\curveto(357.11707382,40.36735836)(356.84707409,40.26735846)(356.56707275,40.17736623)
\curveto(356.28707465,40.08735864)(356.01207493,39.98735874)(355.74207275,39.87736623)
\curveto(355.65207529,39.84735888)(355.56707537,39.81735891)(355.48707275,39.78736623)
\curveto(355.40707553,39.76735896)(355.33207561,39.73735899)(355.26207275,39.69736623)
\curveto(355.19207575,39.66735906)(355.13207581,39.62235911)(355.08207275,39.56236623)
\curveto(355.03207591,39.50235923)(354.99207595,39.42235931)(354.96207275,39.32236623)
\curveto(354.942076,39.27235946)(354.937076,39.21235952)(354.94707275,39.14236623)
\lineto(354.94707275,38.94736623)
\lineto(354.94707275,36.11236623)
\lineto(354.94707275,35.81236623)
\curveto(354.937076,35.70236303)(354.937076,35.59736313)(354.94707275,35.49736623)
\curveto(354.95707598,35.39736333)(354.97207597,35.30236343)(354.99207275,35.21236623)
\curveto(355.01207593,35.1323636)(355.05207589,35.07236366)(355.11207275,35.03236623)
\curveto(355.21207573,34.95236378)(355.32707561,34.89236384)(355.45707275,34.85236623)
\curveto(355.57707536,34.82236391)(355.70207524,34.78236395)(355.83207275,34.73236623)
\curveto(356.06207488,34.6323641)(356.30207464,34.53736419)(356.55207275,34.44736623)
\curveto(356.80207414,34.36736436)(357.0420739,34.27736445)(357.27207275,34.17736623)
\curveto(357.33207361,34.15736457)(357.40207354,34.1323646)(357.48207275,34.10236623)
\curveto(357.55207339,34.08236465)(357.62707331,34.05736467)(357.70707275,34.02736623)
\curveto(357.78707315,33.99736473)(357.86207308,33.96236477)(357.93207275,33.92236623)
\curveto(357.99207295,33.89236484)(358.0370729,33.85736487)(358.06707275,33.81736623)
\curveto(358.12707281,33.73736499)(358.16207278,33.6273651)(358.17207275,33.48736623)
\lineto(358.17207275,33.06736623)
\lineto(358.17207275,32.82736623)
\curveto(358.16207278,32.75736597)(358.1370728,32.69736603)(358.09707275,32.64736623)
\curveto(358.06707287,32.59736613)(358.02207292,32.56736616)(357.96207275,32.55736623)
\curveto(357.90207304,32.55736617)(357.8420731,32.56236617)(357.78207275,32.57236623)
\curveto(357.71207323,32.59236614)(357.64707329,32.61236612)(357.58707275,32.63236623)
\curveto(357.51707342,32.66236607)(357.46707347,32.68736604)(357.43707275,32.70736623)
\curveto(357.11707382,32.84736588)(356.80207414,32.97236576)(356.49207275,33.08236623)
\curveto(356.17207477,33.19236554)(355.85207509,33.31236542)(355.53207275,33.44236623)
\curveto(355.31207563,33.5323652)(355.09707584,33.61736511)(354.88707275,33.69736623)
\curveto(354.66707627,33.77736495)(354.44707649,33.86236487)(354.22707275,33.95236623)
\curveto(353.50707743,34.25236448)(352.78207816,34.53736419)(352.05207275,34.80736623)
\curveto(351.31207963,35.07736365)(350.57708036,35.36236337)(349.84707275,35.66236623)
\curveto(349.58708135,35.77236296)(349.32208162,35.87236286)(349.05207275,35.96236623)
\curveto(348.78208216,36.06236267)(348.51708242,36.16736256)(348.25707275,36.27736623)
\curveto(348.14708279,36.3273624)(348.02708291,36.37236236)(347.89707275,36.41236623)
\curveto(347.75708318,36.46236227)(347.65708328,36.5323622)(347.59707275,36.62236623)
\curveto(347.55708338,36.66236207)(347.52708341,36.727362)(347.50707275,36.81736623)
\curveto(347.49708344,36.83736189)(347.49708344,36.85736187)(347.50707275,36.87736623)
\curveto(347.50708343,36.90736182)(347.50208344,36.9323618)(347.49207275,36.95236623)
\curveto(347.49208345,37.1323616)(347.49208345,37.34236139)(347.49207275,37.58236623)
\curveto(347.48208346,37.82236091)(347.51708342,37.99736073)(347.59707275,38.10736623)
\curveto(347.65708328,38.18736054)(347.75708318,38.24736048)(347.89707275,38.28736623)
\curveto(348.02708291,38.33736039)(348.14708279,38.38736034)(348.25707275,38.43736623)
\curveto(348.48708245,38.53736019)(348.71708222,38.6273601)(348.94707275,38.70736623)
\curveto(349.17708176,38.78735994)(349.40708153,38.87735985)(349.63707275,38.97736623)
\curveto(349.8370811,39.05735967)(350.0420809,39.1323596)(350.25207275,39.20236623)
\curveto(350.46208048,39.28235945)(350.66708027,39.36735936)(350.86707275,39.45736623)
\curveto(351.59707934,39.75735897)(352.3370786,40.04235869)(353.08707275,40.31236623)
\curveto(353.82707711,40.59235814)(354.56207638,40.88735784)(355.29207275,41.19736623)
\curveto(355.38207556,41.23735749)(355.46707547,41.26735746)(355.54707275,41.28736623)
\curveto(355.62707531,41.31735741)(355.71207523,41.34735738)(355.80207275,41.37736623)
\curveto(356.06207488,41.48735724)(356.32707461,41.59235714)(356.59707275,41.69236623)
\curveto(356.86707407,41.80235693)(357.13207381,41.91235682)(357.39207275,42.02236623)
\moveto(353.74707275,38.81236623)
\curveto(353.71707722,38.90235983)(353.66707727,38.95735977)(353.59707275,38.97736623)
\curveto(353.52707741,39.00735972)(353.45207749,39.01235972)(353.37207275,38.99236623)
\curveto(353.28207766,38.98235975)(353.19707774,38.95735977)(353.11707275,38.91736623)
\curveto(353.02707791,38.88735984)(352.95207799,38.85735987)(352.89207275,38.82736623)
\curveto(352.85207809,38.80735992)(352.81707812,38.79735993)(352.78707275,38.79736623)
\curveto(352.75707818,38.79735993)(352.72207822,38.78735994)(352.68207275,38.76736623)
\lineto(352.44207275,38.67736623)
\curveto(352.35207859,38.65736007)(352.26207868,38.6273601)(352.17207275,38.58736623)
\curveto(351.81207913,38.43736029)(351.44707949,38.30236043)(351.07707275,38.18236623)
\curveto(350.69708024,38.07236066)(350.32708061,37.94236079)(349.96707275,37.79236623)
\curveto(349.85708108,37.74236099)(349.74708119,37.69736103)(349.63707275,37.65736623)
\curveto(349.52708141,37.6273611)(349.42208152,37.58736114)(349.32207275,37.53736623)
\curveto(349.27208167,37.51736121)(349.22708171,37.49236124)(349.18707275,37.46236623)
\curveto(349.1370818,37.44236129)(349.11208183,37.39236134)(349.11207275,37.31236623)
\curveto(349.13208181,37.29236144)(349.14708179,37.27236146)(349.15707275,37.25236623)
\curveto(349.16708177,37.2323615)(349.18208176,37.21236152)(349.20207275,37.19236623)
\curveto(349.25208169,37.15236158)(349.30708163,37.12236161)(349.36707275,37.10236623)
\curveto(349.41708152,37.08236165)(349.47208147,37.06236167)(349.53207275,37.04236623)
\curveto(349.6420813,36.99236174)(349.75208119,36.95236178)(349.86207275,36.92236623)
\curveto(349.97208097,36.89236184)(350.08208086,36.85236188)(350.19207275,36.80236623)
\curveto(350.58208036,36.6323621)(350.97707996,36.48236225)(351.37707275,36.35236623)
\curveto(351.77707916,36.2323625)(352.16707877,36.09236264)(352.54707275,35.93236623)
\lineto(352.69707275,35.87236623)
\curveto(352.74707819,35.86236287)(352.79707814,35.84736288)(352.84707275,35.82736623)
\lineto(353.08707275,35.73736623)
\curveto(353.16707777,35.70736302)(353.24707769,35.68236305)(353.32707275,35.66236623)
\curveto(353.37707756,35.64236309)(353.43207751,35.6323631)(353.49207275,35.63236623)
\curveto(353.55207739,35.64236309)(353.60207734,35.65736307)(353.64207275,35.67736623)
\curveto(353.72207722,35.727363)(353.76707717,35.8323629)(353.77707275,35.99236623)
\lineto(353.77707275,36.44236623)
\lineto(353.77707275,38.04736623)
\curveto(353.77707716,38.15736057)(353.78207716,38.29236044)(353.79207275,38.45236623)
\curveto(353.79207715,38.61236012)(353.77707716,38.73236)(353.74707275,38.81236623)
}
}
{
\newrgbcolor{curcolor}{0 0 0}
\pscustom[linestyle=none,fillstyle=solid,fillcolor=curcolor]
{
\newpath
\moveto(354.13707275,50.56392873)
\curveto(354.18707675,50.57392038)(354.25707668,50.57892038)(354.34707275,50.57892873)
\curveto(354.42707651,50.57892038)(354.49207645,50.57392038)(354.54207275,50.56392873)
\curveto(354.58207636,50.56392039)(354.62207632,50.5589204)(354.66207275,50.54892873)
\lineto(354.78207275,50.54892873)
\curveto(354.86207608,50.52892043)(354.942076,50.51892044)(355.02207275,50.51892873)
\curveto(355.10207584,50.51892044)(355.18207576,50.50892045)(355.26207275,50.48892873)
\curveto(355.30207564,50.47892048)(355.3420756,50.47392048)(355.38207275,50.47392873)
\curveto(355.41207553,50.47392048)(355.44707549,50.46892049)(355.48707275,50.45892873)
\curveto(355.59707534,50.42892053)(355.70207524,50.39892056)(355.80207275,50.36892873)
\curveto(355.90207504,50.34892061)(356.00207494,50.31892064)(356.10207275,50.27892873)
\curveto(356.45207449,50.13892082)(356.76707417,49.96892099)(357.04707275,49.76892873)
\curveto(357.32707361,49.56892139)(357.56707337,49.31892164)(357.76707275,49.01892873)
\curveto(357.86707307,48.86892209)(357.95207299,48.72392223)(358.02207275,48.58392873)
\curveto(358.07207287,48.47392248)(358.11207283,48.36392259)(358.14207275,48.25392873)
\curveto(358.17207277,48.1539228)(358.20207274,48.04892291)(358.23207275,47.93892873)
\curveto(358.25207269,47.86892309)(358.26207268,47.80392315)(358.26207275,47.74392873)
\curveto(358.27207267,47.68392327)(358.28707265,47.62392333)(358.30707275,47.56392873)
\lineto(358.30707275,47.41392873)
\curveto(358.32707261,47.36392359)(358.3370726,47.28892367)(358.33707275,47.18892873)
\curveto(358.34707259,47.08892387)(358.3420726,47.00892395)(358.32207275,46.94892873)
\lineto(358.32207275,46.79892873)
\curveto(358.31207263,46.7589242)(358.30707263,46.71392424)(358.30707275,46.66392873)
\curveto(358.30707263,46.62392433)(358.30207264,46.57892438)(358.29207275,46.52892873)
\curveto(358.25207269,46.37892458)(358.21707272,46.22892473)(358.18707275,46.07892873)
\curveto(358.15707278,45.93892502)(358.11207283,45.79892516)(358.05207275,45.65892873)
\curveto(357.97207297,45.4589255)(357.87207307,45.27892568)(357.75207275,45.11892873)
\lineto(357.60207275,44.93892873)
\curveto(357.5420734,44.87892608)(357.50207344,44.80892615)(357.48207275,44.72892873)
\curveto(357.47207347,44.66892629)(357.48707345,44.61892634)(357.52707275,44.57892873)
\curveto(357.55707338,44.54892641)(357.60207334,44.52392643)(357.66207275,44.50392873)
\curveto(357.72207322,44.49392646)(357.78707315,44.48392647)(357.85707275,44.47392873)
\curveto(357.91707302,44.47392648)(357.96207298,44.46392649)(357.99207275,44.44392873)
\curveto(358.0420729,44.40392655)(358.08707285,44.3589266)(358.12707275,44.30892873)
\curveto(358.14707279,44.2589267)(358.16207278,44.18892677)(358.17207275,44.09892873)
\lineto(358.17207275,43.82892873)
\curveto(358.17207277,43.73892722)(358.16707277,43.6539273)(358.15707275,43.57392873)
\curveto(358.1370728,43.49392746)(358.11707282,43.43392752)(358.09707275,43.39392873)
\curveto(358.07707286,43.37392758)(358.05207289,43.3539276)(358.02207275,43.33392873)
\lineto(357.93207275,43.27392873)
\curveto(357.85207309,43.24392771)(357.73207321,43.22892773)(357.57207275,43.22892873)
\curveto(357.41207353,43.23892772)(357.27707366,43.24392771)(357.16707275,43.24392873)
\lineto(348.36207275,43.24392873)
\curveto(348.2420827,43.24392771)(348.11708282,43.23892772)(347.98707275,43.22892873)
\curveto(347.84708309,43.22892773)(347.7370832,43.2539277)(347.65707275,43.30392873)
\curveto(347.59708334,43.34392761)(347.54708339,43.40892755)(347.50707275,43.49892873)
\curveto(347.50708343,43.51892744)(347.50708343,43.54392741)(347.50707275,43.57392873)
\curveto(347.49708344,43.60392735)(347.49208345,43.62892733)(347.49207275,43.64892873)
\curveto(347.48208346,43.78892717)(347.48208346,43.93392702)(347.49207275,44.08392873)
\curveto(347.49208345,44.24392671)(347.53208341,44.3539266)(347.61207275,44.41392873)
\curveto(347.69208325,44.46392649)(347.80708313,44.48892647)(347.95707275,44.48892873)
\lineto(348.36207275,44.48892873)
\lineto(350.11707275,44.48892873)
\lineto(350.37207275,44.48892873)
\lineto(350.65707275,44.48892873)
\curveto(350.74708019,44.49892646)(350.83208011,44.50892645)(350.91207275,44.51892873)
\curveto(350.98207996,44.53892642)(351.03207991,44.56892639)(351.06207275,44.60892873)
\curveto(351.09207985,44.64892631)(351.09707984,44.69392626)(351.07707275,44.74392873)
\curveto(351.05707988,44.79392616)(351.0370799,44.83392612)(351.01707275,44.86392873)
\curveto(350.97707996,44.91392604)(350.93708,44.958926)(350.89707275,44.99892873)
\lineto(350.77707275,45.14892873)
\curveto(350.72708021,45.21892574)(350.68208026,45.28892567)(350.64207275,45.35892873)
\lineto(350.52207275,45.59892873)
\curveto(350.43208051,45.77892518)(350.36708057,45.99392496)(350.32707275,46.24392873)
\curveto(350.28708065,46.49392446)(350.26708067,46.74892421)(350.26707275,47.00892873)
\curveto(350.26708067,47.26892369)(350.29208065,47.52392343)(350.34207275,47.77392873)
\curveto(350.38208056,48.02392293)(350.4420805,48.24392271)(350.52207275,48.43392873)
\curveto(350.69208025,48.83392212)(350.92708001,49.17892178)(351.22707275,49.46892873)
\curveto(351.52707941,49.7589212)(351.87707906,49.98892097)(352.27707275,50.15892873)
\curveto(352.38707855,50.20892075)(352.49707844,50.24892071)(352.60707275,50.27892873)
\curveto(352.70707823,50.31892064)(352.81207813,50.3589206)(352.92207275,50.39892873)
\curveto(353.03207791,50.42892053)(353.14707779,50.44892051)(353.26707275,50.45892873)
\lineto(353.59707275,50.51892873)
\curveto(353.62707731,50.52892043)(353.66207728,50.53392042)(353.70207275,50.53392873)
\curveto(353.73207721,50.53392042)(353.76207718,50.53892042)(353.79207275,50.54892873)
\curveto(353.85207709,50.56892039)(353.91207703,50.56892039)(353.97207275,50.54892873)
\curveto(354.02207692,50.53892042)(354.07707686,50.54392041)(354.13707275,50.56392873)
\moveto(354.52707275,49.22892873)
\curveto(354.47707646,49.24892171)(354.41707652,49.2539217)(354.34707275,49.24392873)
\curveto(354.27707666,49.23392172)(354.21207673,49.22892173)(354.15207275,49.22892873)
\curveto(353.98207696,49.22892173)(353.82207712,49.21892174)(353.67207275,49.19892873)
\curveto(353.52207742,49.18892177)(353.38707755,49.1589218)(353.26707275,49.10892873)
\curveto(353.16707777,49.07892188)(353.07707786,49.0539219)(352.99707275,49.03392873)
\curveto(352.91707802,49.01392194)(352.8370781,48.98392197)(352.75707275,48.94392873)
\curveto(352.50707843,48.83392212)(352.27707866,48.68392227)(352.06707275,48.49392873)
\curveto(351.84707909,48.30392265)(351.68207926,48.08392287)(351.57207275,47.83392873)
\curveto(351.5420794,47.7539232)(351.51707942,47.67392328)(351.49707275,47.59392873)
\curveto(351.46707947,47.52392343)(351.4420795,47.44892351)(351.42207275,47.36892873)
\curveto(351.39207955,47.2589237)(351.37707956,47.14892381)(351.37707275,47.03892873)
\curveto(351.36707957,46.92892403)(351.36207958,46.80892415)(351.36207275,46.67892873)
\curveto(351.37207957,46.62892433)(351.38207956,46.58392437)(351.39207275,46.54392873)
\lineto(351.39207275,46.40892873)
\lineto(351.45207275,46.13892873)
\curveto(351.47207947,46.0589249)(351.50207944,45.97892498)(351.54207275,45.89892873)
\curveto(351.68207926,45.5589254)(351.89207905,45.28892567)(352.17207275,45.08892873)
\curveto(352.4420785,44.88892607)(352.76207818,44.72892623)(353.13207275,44.60892873)
\curveto(353.2420777,44.56892639)(353.35207759,44.54392641)(353.46207275,44.53392873)
\curveto(353.57207737,44.52392643)(353.68707725,44.50392645)(353.80707275,44.47392873)
\curveto(353.85707708,44.46392649)(353.90207704,44.46392649)(353.94207275,44.47392873)
\curveto(353.98207696,44.48392647)(354.02707691,44.47892648)(354.07707275,44.45892873)
\curveto(354.12707681,44.44892651)(354.20207674,44.44392651)(354.30207275,44.44392873)
\curveto(354.39207655,44.44392651)(354.46207648,44.44892651)(354.51207275,44.45892873)
\lineto(354.63207275,44.45892873)
\curveto(354.67207627,44.46892649)(354.71207623,44.47392648)(354.75207275,44.47392873)
\curveto(354.79207615,44.47392648)(354.82707611,44.47892648)(354.85707275,44.48892873)
\curveto(354.88707605,44.49892646)(354.92207602,44.50392645)(354.96207275,44.50392873)
\curveto(354.99207595,44.50392645)(355.02207592,44.50892645)(355.05207275,44.51892873)
\curveto(355.13207581,44.53892642)(355.21207573,44.5539264)(355.29207275,44.56392873)
\lineto(355.53207275,44.62392873)
\curveto(355.87207507,44.73392622)(356.16207478,44.88392607)(356.40207275,45.07392873)
\curveto(356.6420743,45.27392568)(356.8420741,45.51892544)(357.00207275,45.80892873)
\curveto(357.05207389,45.89892506)(357.09207385,45.99392496)(357.12207275,46.09392873)
\curveto(357.1420738,46.19392476)(357.16707377,46.29892466)(357.19707275,46.40892873)
\curveto(357.21707372,46.4589245)(357.22707371,46.50392445)(357.22707275,46.54392873)
\curveto(357.21707372,46.59392436)(357.21707372,46.64392431)(357.22707275,46.69392873)
\curveto(357.2370737,46.73392422)(357.2420737,46.77892418)(357.24207275,46.82892873)
\lineto(357.24207275,46.96392873)
\lineto(357.24207275,47.09892873)
\curveto(357.23207371,47.13892382)(357.22707371,47.17392378)(357.22707275,47.20392873)
\curveto(357.22707371,47.23392372)(357.22207372,47.26892369)(357.21207275,47.30892873)
\curveto(357.19207375,47.38892357)(357.17707376,47.46392349)(357.16707275,47.53392873)
\curveto(357.14707379,47.60392335)(357.12207382,47.67892328)(357.09207275,47.75892873)
\curveto(356.96207398,48.06892289)(356.79207415,48.31892264)(356.58207275,48.50892873)
\curveto(356.36207458,48.69892226)(356.09707484,48.8589221)(355.78707275,48.98892873)
\curveto(355.64707529,49.03892192)(355.50707543,49.07392188)(355.36707275,49.09392873)
\curveto(355.21707572,49.12392183)(355.06707587,49.1589218)(354.91707275,49.19892873)
\curveto(354.86707607,49.21892174)(354.82207612,49.22392173)(354.78207275,49.21392873)
\curveto(354.73207621,49.21392174)(354.68207626,49.21892174)(354.63207275,49.22892873)
\lineto(354.52707275,49.22892873)
}
}
{
\newrgbcolor{curcolor}{0 0 0}
\pscustom[linestyle=none,fillstyle=solid,fillcolor=curcolor]
{
\newpath
\moveto(350.26707275,55.69017873)
\curveto(350.26708067,55.92017394)(350.32708061,56.05017381)(350.44707275,56.08017873)
\curveto(350.55708038,56.11017375)(350.72208022,56.12517374)(350.94207275,56.12517873)
\lineto(351.22707275,56.12517873)
\curveto(351.31707962,56.12517374)(351.39207955,56.10017376)(351.45207275,56.05017873)
\curveto(351.53207941,55.99017387)(351.57707936,55.90517396)(351.58707275,55.79517873)
\curveto(351.58707935,55.68517418)(351.60207934,55.57517429)(351.63207275,55.46517873)
\curveto(351.66207928,55.32517454)(351.69207925,55.19017467)(351.72207275,55.06017873)
\curveto(351.75207919,54.94017492)(351.79207915,54.82517504)(351.84207275,54.71517873)
\curveto(351.97207897,54.42517544)(352.15207879,54.19017567)(352.38207275,54.01017873)
\curveto(352.60207834,53.83017603)(352.85707808,53.67517619)(353.14707275,53.54517873)
\curveto(353.25707768,53.50517636)(353.37207757,53.47517639)(353.49207275,53.45517873)
\curveto(353.60207734,53.43517643)(353.71707722,53.41017645)(353.83707275,53.38017873)
\curveto(353.88707705,53.37017649)(353.937077,53.3651765)(353.98707275,53.36517873)
\curveto(354.0370769,53.37517649)(354.08707685,53.37517649)(354.13707275,53.36517873)
\curveto(354.25707668,53.33517653)(354.39707654,53.32017654)(354.55707275,53.32017873)
\curveto(354.70707623,53.33017653)(354.85207609,53.33517653)(354.99207275,53.33517873)
\lineto(356.83707275,53.33517873)
\lineto(357.18207275,53.33517873)
\curveto(357.30207364,53.33517653)(357.41707352,53.33017653)(357.52707275,53.32017873)
\curveto(357.6370733,53.31017655)(357.73207321,53.30517656)(357.81207275,53.30517873)
\curveto(357.89207305,53.31517655)(357.96207298,53.29517657)(358.02207275,53.24517873)
\curveto(358.09207285,53.19517667)(358.13207281,53.11517675)(358.14207275,53.00517873)
\curveto(358.15207279,52.90517696)(358.15707278,52.79517707)(358.15707275,52.67517873)
\lineto(358.15707275,52.40517873)
\curveto(358.1370728,52.35517751)(358.12207282,52.30517756)(358.11207275,52.25517873)
\curveto(358.09207285,52.21517765)(358.06707287,52.18517768)(358.03707275,52.16517873)
\curveto(357.96707297,52.11517775)(357.88207306,52.08517778)(357.78207275,52.07517873)
\lineto(357.45207275,52.07517873)
\lineto(356.29707275,52.07517873)
\lineto(352.14207275,52.07517873)
\lineto(351.10707275,52.07517873)
\lineto(350.80707275,52.07517873)
\curveto(350.70708023,52.08517778)(350.62208032,52.11517775)(350.55207275,52.16517873)
\curveto(350.51208043,52.19517767)(350.48208046,52.24517762)(350.46207275,52.31517873)
\curveto(350.4420805,52.39517747)(350.43208051,52.48017738)(350.43207275,52.57017873)
\curveto(350.42208052,52.6601772)(350.42208052,52.75017711)(350.43207275,52.84017873)
\curveto(350.4420805,52.93017693)(350.45708048,53.00017686)(350.47707275,53.05017873)
\curveto(350.50708043,53.13017673)(350.56708037,53.18017668)(350.65707275,53.20017873)
\curveto(350.7370802,53.23017663)(350.82708011,53.24517662)(350.92707275,53.24517873)
\lineto(351.22707275,53.24517873)
\curveto(351.32707961,53.24517662)(351.41707952,53.2651766)(351.49707275,53.30517873)
\curveto(351.51707942,53.31517655)(351.53207941,53.32517654)(351.54207275,53.33517873)
\lineto(351.58707275,53.38017873)
\curveto(351.58707935,53.49017637)(351.5420794,53.58017628)(351.45207275,53.65017873)
\curveto(351.35207959,53.72017614)(351.27207967,53.78017608)(351.21207275,53.83017873)
\lineto(351.12207275,53.92017873)
\curveto(351.01207993,54.01017585)(350.89708004,54.13517573)(350.77707275,54.29517873)
\curveto(350.65708028,54.45517541)(350.56708037,54.60517526)(350.50707275,54.74517873)
\curveto(350.45708048,54.83517503)(350.42208052,54.93017493)(350.40207275,55.03017873)
\curveto(350.37208057,55.13017473)(350.3420806,55.23517463)(350.31207275,55.34517873)
\curveto(350.30208064,55.40517446)(350.29708064,55.4651744)(350.29707275,55.52517873)
\curveto(350.28708065,55.58517428)(350.27708066,55.64017422)(350.26707275,55.69017873)
}
}
{
\newrgbcolor{curcolor}{0 0 0}
\pscustom[linestyle=none,fillstyle=solid,fillcolor=curcolor]
{
}
}
{
\newrgbcolor{curcolor}{0 0 0}
\pscustom[linestyle=none,fillstyle=solid,fillcolor=curcolor]
{
\newpath
\moveto(353.08707275,67.99510061)
\lineto(353.34207275,67.99510061)
\curveto(353.42207752,68.0050929)(353.49707744,68.00009291)(353.56707275,67.98010061)
\lineto(353.80707275,67.98010061)
\lineto(353.97207275,67.98010061)
\curveto(354.07207687,67.96009295)(354.17707676,67.95009296)(354.28707275,67.95010061)
\curveto(354.38707655,67.95009296)(354.48707645,67.94009297)(354.58707275,67.92010061)
\lineto(354.73707275,67.92010061)
\curveto(354.87707606,67.89009302)(355.01707592,67.87009304)(355.15707275,67.86010061)
\curveto(355.28707565,67.85009306)(355.41707552,67.82509308)(355.54707275,67.78510061)
\curveto(355.62707531,67.76509314)(355.71207523,67.74509316)(355.80207275,67.72510061)
\lineto(356.04207275,67.66510061)
\lineto(356.34207275,67.54510061)
\curveto(356.43207451,67.51509339)(356.52207442,67.48009343)(356.61207275,67.44010061)
\curveto(356.83207411,67.34009357)(357.04707389,67.2050937)(357.25707275,67.03510061)
\curveto(357.46707347,66.87509403)(357.6370733,66.70009421)(357.76707275,66.51010061)
\curveto(357.80707313,66.46009445)(357.84707309,66.40009451)(357.88707275,66.33010061)
\curveto(357.91707302,66.27009464)(357.95207299,66.2100947)(357.99207275,66.15010061)
\curveto(358.0420729,66.07009484)(358.08207286,65.97509493)(358.11207275,65.86510061)
\curveto(358.1420728,65.75509515)(358.17207277,65.65009526)(358.20207275,65.55010061)
\curveto(358.2420727,65.44009547)(358.26707267,65.33009558)(358.27707275,65.22010061)
\curveto(358.28707265,65.1100958)(358.30207264,64.99509591)(358.32207275,64.87510061)
\curveto(358.33207261,64.83509607)(358.33207261,64.79009612)(358.32207275,64.74010061)
\curveto(358.32207262,64.70009621)(358.32707261,64.66009625)(358.33707275,64.62010061)
\curveto(358.34707259,64.58009633)(358.35207259,64.52509638)(358.35207275,64.45510061)
\curveto(358.35207259,64.38509652)(358.34707259,64.33509657)(358.33707275,64.30510061)
\curveto(358.31707262,64.25509665)(358.31207263,64.2100967)(358.32207275,64.17010061)
\curveto(358.33207261,64.13009678)(358.33207261,64.09509681)(358.32207275,64.06510061)
\lineto(358.32207275,63.97510061)
\curveto(358.30207264,63.91509699)(358.28707265,63.85009706)(358.27707275,63.78010061)
\curveto(358.27707266,63.72009719)(358.27207267,63.65509725)(358.26207275,63.58510061)
\curveto(358.21207273,63.41509749)(358.16207278,63.25509765)(358.11207275,63.10510061)
\curveto(358.06207288,62.95509795)(357.99707294,62.8100981)(357.91707275,62.67010061)
\curveto(357.87707306,62.62009829)(357.84707309,62.56509834)(357.82707275,62.50510061)
\curveto(357.79707314,62.45509845)(357.76207318,62.4050985)(357.72207275,62.35510061)
\curveto(357.5420734,62.11509879)(357.32207362,61.91509899)(357.06207275,61.75510061)
\curveto(356.80207414,61.59509931)(356.51707442,61.45509945)(356.20707275,61.33510061)
\curveto(356.06707487,61.27509963)(355.92707501,61.23009968)(355.78707275,61.20010061)
\curveto(355.6370753,61.17009974)(355.48207546,61.13509977)(355.32207275,61.09510061)
\curveto(355.21207573,61.07509983)(355.10207584,61.06009985)(354.99207275,61.05010061)
\curveto(354.88207606,61.04009987)(354.77207617,61.02509988)(354.66207275,61.00510061)
\curveto(354.62207632,60.99509991)(354.58207636,60.99009992)(354.54207275,60.99010061)
\curveto(354.50207644,61.00009991)(354.46207648,61.00009991)(354.42207275,60.99010061)
\curveto(354.37207657,60.98009993)(354.32207662,60.97509993)(354.27207275,60.97510061)
\lineto(354.10707275,60.97510061)
\curveto(354.05707688,60.95509995)(354.00707693,60.95009996)(353.95707275,60.96010061)
\curveto(353.89707704,60.97009994)(353.8420771,60.97009994)(353.79207275,60.96010061)
\curveto(353.75207719,60.95009996)(353.70707723,60.95009996)(353.65707275,60.96010061)
\curveto(353.60707733,60.97009994)(353.55707738,60.96509994)(353.50707275,60.94510061)
\curveto(353.4370775,60.92509998)(353.36207758,60.92009999)(353.28207275,60.93010061)
\curveto(353.19207775,60.94009997)(353.10707783,60.94509996)(353.02707275,60.94510061)
\curveto(352.937078,60.94509996)(352.8370781,60.94009997)(352.72707275,60.93010061)
\curveto(352.60707833,60.92009999)(352.50707843,60.92509998)(352.42707275,60.94510061)
\lineto(352.14207275,60.94510061)
\lineto(351.51207275,60.99010061)
\curveto(351.41207953,61.00009991)(351.31707962,61.0100999)(351.22707275,61.02010061)
\lineto(350.92707275,61.05010061)
\curveto(350.87708006,61.07009984)(350.82708011,61.07509983)(350.77707275,61.06510061)
\curveto(350.71708022,61.06509984)(350.66208028,61.07509983)(350.61207275,61.09510061)
\curveto(350.4420805,61.14509976)(350.27708066,61.18509972)(350.11707275,61.21510061)
\curveto(349.94708099,61.24509966)(349.78708115,61.29509961)(349.63707275,61.36510061)
\curveto(349.17708176,61.55509935)(348.80208214,61.77509913)(348.51207275,62.02510061)
\curveto(348.22208272,62.28509862)(347.97708296,62.64509826)(347.77707275,63.10510061)
\curveto(347.72708321,63.23509767)(347.69208325,63.36509754)(347.67207275,63.49510061)
\curveto(347.65208329,63.63509727)(347.62708331,63.77509713)(347.59707275,63.91510061)
\curveto(347.58708335,63.98509692)(347.58208336,64.05009686)(347.58207275,64.11010061)
\curveto(347.58208336,64.17009674)(347.57708336,64.23509667)(347.56707275,64.30510061)
\curveto(347.54708339,65.13509577)(347.69708324,65.8050951)(348.01707275,66.31510061)
\curveto(348.32708261,66.82509408)(348.76708217,67.2050937)(349.33707275,67.45510061)
\curveto(349.45708148,67.5050934)(349.58208136,67.55009336)(349.71207275,67.59010061)
\curveto(349.8420811,67.63009328)(349.97708096,67.67509323)(350.11707275,67.72510061)
\curveto(350.19708074,67.74509316)(350.28208066,67.76009315)(350.37207275,67.77010061)
\lineto(350.61207275,67.83010061)
\curveto(350.72208022,67.86009305)(350.83208011,67.87509303)(350.94207275,67.87510061)
\curveto(351.05207989,67.88509302)(351.16207978,67.90009301)(351.27207275,67.92010061)
\curveto(351.32207962,67.94009297)(351.36707957,67.94509296)(351.40707275,67.93510061)
\curveto(351.44707949,67.93509297)(351.48707945,67.94009297)(351.52707275,67.95010061)
\curveto(351.57707936,67.96009295)(351.63207931,67.96009295)(351.69207275,67.95010061)
\curveto(351.7420792,67.95009296)(351.79207915,67.95509295)(351.84207275,67.96510061)
\lineto(351.97707275,67.96510061)
\curveto(352.0370789,67.98509292)(352.10707883,67.98509292)(352.18707275,67.96510061)
\curveto(352.25707868,67.95509295)(352.32207862,67.96009295)(352.38207275,67.98010061)
\curveto(352.41207853,67.99009292)(352.45207849,67.99509291)(352.50207275,67.99510061)
\lineto(352.62207275,67.99510061)
\lineto(353.08707275,67.99510061)
\moveto(355.41207275,66.45010061)
\curveto(355.09207585,66.55009436)(354.72707621,66.6100943)(354.31707275,66.63010061)
\curveto(353.90707703,66.65009426)(353.49707744,66.66009425)(353.08707275,66.66010061)
\curveto(352.65707828,66.66009425)(352.2370787,66.65009426)(351.82707275,66.63010061)
\curveto(351.41707952,66.6100943)(351.03207991,66.56509434)(350.67207275,66.49510061)
\curveto(350.31208063,66.42509448)(349.99208095,66.31509459)(349.71207275,66.16510061)
\curveto(349.42208152,66.02509488)(349.18708175,65.83009508)(349.00707275,65.58010061)
\curveto(348.89708204,65.42009549)(348.81708212,65.24009567)(348.76707275,65.04010061)
\curveto(348.70708223,64.84009607)(348.67708226,64.59509631)(348.67707275,64.30510061)
\curveto(348.69708224,64.28509662)(348.70708223,64.25009666)(348.70707275,64.20010061)
\curveto(348.69708224,64.15009676)(348.69708224,64.1100968)(348.70707275,64.08010061)
\curveto(348.72708221,64.00009691)(348.74708219,63.92509698)(348.76707275,63.85510061)
\curveto(348.77708216,63.79509711)(348.79708214,63.73009718)(348.82707275,63.66010061)
\curveto(348.94708199,63.39009752)(349.11708182,63.17009774)(349.33707275,63.00010061)
\curveto(349.54708139,62.84009807)(349.79208115,62.7050982)(350.07207275,62.59510061)
\curveto(350.18208076,62.54509836)(350.30208064,62.5050984)(350.43207275,62.47510061)
\curveto(350.55208039,62.45509845)(350.67708026,62.43009848)(350.80707275,62.40010061)
\curveto(350.85708008,62.38009853)(350.91208003,62.37009854)(350.97207275,62.37010061)
\curveto(351.02207992,62.37009854)(351.07207987,62.36509854)(351.12207275,62.35510061)
\curveto(351.21207973,62.34509856)(351.30707963,62.33509857)(351.40707275,62.32510061)
\curveto(351.49707944,62.31509859)(351.59207935,62.3050986)(351.69207275,62.29510061)
\curveto(351.77207917,62.29509861)(351.85707908,62.29009862)(351.94707275,62.28010061)
\lineto(352.18707275,62.28010061)
\lineto(352.36707275,62.28010061)
\curveto(352.39707854,62.27009864)(352.43207851,62.26509864)(352.47207275,62.26510061)
\lineto(352.60707275,62.26510061)
\lineto(353.05707275,62.26510061)
\curveto(353.1370778,62.26509864)(353.22207772,62.26009865)(353.31207275,62.25010061)
\curveto(353.39207755,62.25009866)(353.46707747,62.26009865)(353.53707275,62.28010061)
\lineto(353.80707275,62.28010061)
\curveto(353.82707711,62.28009863)(353.85707708,62.27509863)(353.89707275,62.26510061)
\curveto(353.92707701,62.26509864)(353.95207699,62.27009864)(353.97207275,62.28010061)
\curveto(354.07207687,62.29009862)(354.17207677,62.29509861)(354.27207275,62.29510061)
\curveto(354.36207658,62.3050986)(354.46207648,62.31509859)(354.57207275,62.32510061)
\curveto(354.69207625,62.35509855)(354.81707612,62.37009854)(354.94707275,62.37010061)
\curveto(355.06707587,62.38009853)(355.18207576,62.4050985)(355.29207275,62.44510061)
\curveto(355.59207535,62.52509838)(355.85707508,62.6100983)(356.08707275,62.70010061)
\curveto(356.31707462,62.80009811)(356.53207441,62.94509796)(356.73207275,63.13510061)
\curveto(356.93207401,63.34509756)(357.08207386,63.6100973)(357.18207275,63.93010061)
\curveto(357.20207374,63.97009694)(357.21207373,64.0050969)(357.21207275,64.03510061)
\curveto(357.20207374,64.07509683)(357.20707373,64.12009679)(357.22707275,64.17010061)
\curveto(357.2370737,64.2100967)(357.24707369,64.28009663)(357.25707275,64.38010061)
\curveto(357.26707367,64.49009642)(357.26207368,64.57509633)(357.24207275,64.63510061)
\curveto(357.22207372,64.7050962)(357.21207373,64.77509613)(357.21207275,64.84510061)
\curveto(357.20207374,64.91509599)(357.18707375,64.98009593)(357.16707275,65.04010061)
\curveto(357.10707383,65.24009567)(357.02207392,65.42009549)(356.91207275,65.58010061)
\curveto(356.89207405,65.6100953)(356.87207407,65.63509527)(356.85207275,65.65510061)
\lineto(356.79207275,65.71510061)
\curveto(356.77207417,65.75509515)(356.73207421,65.8050951)(356.67207275,65.86510061)
\curveto(356.53207441,65.96509494)(356.40207454,66.05009486)(356.28207275,66.12010061)
\curveto(356.16207478,66.19009472)(356.01707492,66.26009465)(355.84707275,66.33010061)
\curveto(355.77707516,66.36009455)(355.70707523,66.38009453)(355.63707275,66.39010061)
\curveto(355.56707537,66.4100945)(355.49207545,66.43009448)(355.41207275,66.45010061)
}
}
{
\newrgbcolor{curcolor}{0 0 0}
\pscustom[linestyle=none,fillstyle=solid,fillcolor=curcolor]
{
\newpath
\moveto(347.76207275,69.80470998)
\lineto(347.76207275,74.60470998)
\lineto(347.76207275,75.60970998)
\curveto(347.76208318,75.74970288)(347.77208317,75.86970276)(347.79207275,75.96970998)
\curveto(347.80208314,76.07970255)(347.84708309,76.15970247)(347.92707275,76.20970998)
\curveto(347.96708297,76.2297024)(348.01708292,76.23970239)(348.07707275,76.23970998)
\curveto(348.1370828,76.24970238)(348.20208274,76.25470238)(348.27207275,76.25470998)
\lineto(348.54207275,76.25470998)
\curveto(348.63208231,76.25470238)(348.71208223,76.24470239)(348.78207275,76.22470998)
\curveto(348.86208208,76.18470245)(348.93208201,76.13970249)(348.99207275,76.08970998)
\lineto(349.17207275,75.93970998)
\curveto(349.22208172,75.90970272)(349.26208168,75.87470276)(349.29207275,75.83470998)
\curveto(349.32208162,75.79470284)(349.36208158,75.75470288)(349.41207275,75.71470998)
\curveto(349.52208142,75.634703)(349.63208131,75.54970308)(349.74207275,75.45970998)
\curveto(349.8420811,75.36970326)(349.94708099,75.28470335)(350.05707275,75.20470998)
\curveto(350.25708068,75.06470357)(350.46708047,74.92470371)(350.68707275,74.78470998)
\curveto(350.89708004,74.64470399)(351.11207983,74.50470413)(351.33207275,74.36470998)
\curveto(351.42207952,74.31470432)(351.51707942,74.26470437)(351.61707275,74.21470998)
\curveto(351.71707922,74.16470447)(351.81207913,74.10970452)(351.90207275,74.04970998)
\curveto(351.92207902,74.0297046)(351.94707899,74.01970461)(351.97707275,74.01970998)
\curveto(352.00707893,74.01970461)(352.03207891,74.00970462)(352.05207275,73.98970998)
\curveto(352.15207879,73.91970471)(352.26707867,73.85470478)(352.39707275,73.79470998)
\curveto(352.51707842,73.7347049)(352.63207831,73.67970495)(352.74207275,73.62970998)
\curveto(352.97207797,73.5297051)(353.20707773,73.4347052)(353.44707275,73.34470998)
\curveto(353.68707725,73.25470538)(353.92707701,73.15470548)(354.16707275,73.04470998)
\curveto(354.21707672,73.02470561)(354.26207668,73.00970562)(354.30207275,72.99970998)
\curveto(354.3420766,72.99970563)(354.38707655,72.98970564)(354.43707275,72.96970998)
\curveto(354.55707638,72.91970571)(354.68207626,72.87470576)(354.81207275,72.83470998)
\curveto(354.93207601,72.80470583)(355.05207589,72.76970586)(355.17207275,72.72970998)
\curveto(355.40207554,72.64970598)(355.6420753,72.58470605)(355.89207275,72.53470998)
\curveto(356.13207481,72.49470614)(356.37207457,72.44470619)(356.61207275,72.38470998)
\curveto(356.76207418,72.34470629)(356.91207403,72.31970631)(357.06207275,72.30970998)
\curveto(357.21207373,72.29970633)(357.36207358,72.27970635)(357.51207275,72.24970998)
\curveto(357.55207339,72.23970639)(357.61207333,72.2347064)(357.69207275,72.23470998)
\curveto(357.81207313,72.20470643)(357.91207303,72.17470646)(357.99207275,72.14470998)
\curveto(358.07207287,72.11470652)(358.12707281,72.04470659)(358.15707275,71.93470998)
\curveto(358.17707276,71.88470675)(358.18707275,71.8297068)(358.18707275,71.76970998)
\lineto(358.18707275,71.57470998)
\curveto(358.18707275,71.4347072)(358.18207276,71.29470734)(358.17207275,71.15470998)
\curveto(358.16207278,71.02470761)(358.11707282,70.9297077)(358.03707275,70.86970998)
\curveto(357.97707296,70.8297078)(357.89207305,70.80970782)(357.78207275,70.80970998)
\curveto(357.67207327,70.81970781)(357.57707336,70.8347078)(357.49707275,70.85470998)
\lineto(357.42207275,70.85470998)
\curveto(357.39207355,70.86470777)(357.36207358,70.86970776)(357.33207275,70.86970998)
\curveto(357.25207369,70.88970774)(357.17707376,70.89970773)(357.10707275,70.89970998)
\curveto(357.0370739,70.89970773)(356.96707397,70.90970772)(356.89707275,70.92970998)
\curveto(356.70707423,70.97970765)(356.52207442,71.01970761)(356.34207275,71.04970998)
\curveto(356.15207479,71.07970755)(355.97207497,71.11970751)(355.80207275,71.16970998)
\curveto(355.75207519,71.18970744)(355.71207523,71.19970743)(355.68207275,71.19970998)
\curveto(355.65207529,71.19970743)(355.61707532,71.20470743)(355.57707275,71.21470998)
\curveto(355.27707566,71.31470732)(354.98207596,71.40470723)(354.69207275,71.48470998)
\curveto(354.40207654,71.57470706)(354.12207682,71.67970695)(353.85207275,71.79970998)
\curveto(353.27207767,72.05970657)(352.72207822,72.3297063)(352.20207275,72.60970998)
\curveto(351.67207927,72.88970574)(351.16707977,73.19970543)(350.68707275,73.53970998)
\curveto(350.48708045,73.67970495)(350.29708064,73.8297048)(350.11707275,73.98970998)
\curveto(349.92708101,74.14970448)(349.7370812,74.29970433)(349.54707275,74.43970998)
\curveto(349.49708144,74.47970415)(349.45208149,74.51470412)(349.41207275,74.54470998)
\curveto(349.36208158,74.58470405)(349.31208163,74.61970401)(349.26207275,74.64970998)
\curveto(349.2420817,74.65970397)(349.21708172,74.66970396)(349.18707275,74.67970998)
\curveto(349.15708178,74.69970393)(349.12708181,74.69970393)(349.09707275,74.67970998)
\curveto(349.0370819,74.65970397)(349.00208194,74.62470401)(348.99207275,74.57470998)
\curveto(348.97208197,74.52470411)(348.95208199,74.47470416)(348.93207275,74.42470998)
\lineto(348.93207275,74.31970998)
\curveto(348.92208202,74.27970435)(348.92208202,74.2297044)(348.93207275,74.16970998)
\lineto(348.93207275,74.01970998)
\lineto(348.93207275,73.41970998)
\lineto(348.93207275,70.77970998)
\lineto(348.93207275,70.04470998)
\lineto(348.93207275,69.80470998)
\curveto(348.92208202,69.7347089)(348.90708203,69.67470896)(348.88707275,69.62470998)
\curveto(348.84708209,69.5347091)(348.78708215,69.47470916)(348.70707275,69.44470998)
\curveto(348.60708233,69.39470924)(348.46208248,69.37970925)(348.27207275,69.39970998)
\curveto(348.07208287,69.41970921)(347.937083,69.45470918)(347.86707275,69.50470998)
\curveto(347.84708309,69.52470911)(347.83208311,69.54970908)(347.82207275,69.57970998)
\lineto(347.76207275,69.69970998)
\curveto(347.76208318,69.71970891)(347.76708317,69.7347089)(347.77707275,69.74470998)
\curveto(347.77708316,69.76470887)(347.77208317,69.78470885)(347.76207275,69.80470998)
}
}
{
\newrgbcolor{curcolor}{0 0 0}
\pscustom[linestyle=none,fillstyle=solid,fillcolor=curcolor]
{
\newpath
\moveto(356.53707275,78.63431936)
\lineto(356.53707275,79.26431936)
\lineto(356.53707275,79.45931936)
\curveto(356.5370744,79.52931683)(356.54707439,79.58931677)(356.56707275,79.63931936)
\curveto(356.60707433,79.70931665)(356.64707429,79.7593166)(356.68707275,79.78931936)
\curveto(356.7370742,79.82931653)(356.80207414,79.84931651)(356.88207275,79.84931936)
\curveto(356.96207398,79.8593165)(357.04707389,79.86431649)(357.13707275,79.86431936)
\lineto(357.85707275,79.86431936)
\curveto(358.3370726,79.86431649)(358.74707219,79.80431655)(359.08707275,79.68431936)
\curveto(359.42707151,79.56431679)(359.70207124,79.36931699)(359.91207275,79.09931936)
\curveto(359.96207098,79.02931733)(360.00707093,78.9593174)(360.04707275,78.88931936)
\curveto(360.09707084,78.82931753)(360.1420708,78.7543176)(360.18207275,78.66431936)
\curveto(360.19207075,78.64431771)(360.20207074,78.61431774)(360.21207275,78.57431936)
\curveto(360.23207071,78.53431782)(360.2370707,78.48931787)(360.22707275,78.43931936)
\curveto(360.19707074,78.34931801)(360.12207082,78.29431806)(360.00207275,78.27431936)
\curveto(359.89207105,78.2543181)(359.79707114,78.26931809)(359.71707275,78.31931936)
\curveto(359.64707129,78.34931801)(359.58207136,78.39431796)(359.52207275,78.45431936)
\curveto(359.47207147,78.52431783)(359.42207152,78.58931777)(359.37207275,78.64931936)
\curveto(359.32207162,78.71931764)(359.24707169,78.77931758)(359.14707275,78.82931936)
\curveto(359.05707188,78.88931747)(358.96707197,78.93931742)(358.87707275,78.97931936)
\curveto(358.84707209,78.99931736)(358.78707215,79.02431733)(358.69707275,79.05431936)
\curveto(358.61707232,79.08431727)(358.54707239,79.08931727)(358.48707275,79.06931936)
\curveto(358.34707259,79.03931732)(358.25707268,78.97931738)(358.21707275,78.88931936)
\curveto(358.18707275,78.80931755)(358.17207277,78.71931764)(358.17207275,78.61931936)
\curveto(358.17207277,78.51931784)(358.14707279,78.43431792)(358.09707275,78.36431936)
\curveto(358.02707291,78.27431808)(357.88707305,78.22931813)(357.67707275,78.22931936)
\lineto(357.12207275,78.22931936)
\lineto(356.89707275,78.22931936)
\curveto(356.81707412,78.23931812)(356.75207419,78.2593181)(356.70207275,78.28931936)
\curveto(356.62207432,78.34931801)(356.57707436,78.41931794)(356.56707275,78.49931936)
\curveto(356.55707438,78.51931784)(356.55207439,78.53931782)(356.55207275,78.55931936)
\curveto(356.55207439,78.58931777)(356.54707439,78.61431774)(356.53707275,78.63431936)
}
}
{
\newrgbcolor{curcolor}{0 0 0}
\pscustom[linestyle=none,fillstyle=solid,fillcolor=curcolor]
{
}
}
{
\newrgbcolor{curcolor}{0 0 0}
\pscustom[linestyle=none,fillstyle=solid,fillcolor=curcolor]
{
\newpath
\moveto(347.56707275,89.26463186)
\curveto(347.55708338,89.95462722)(347.67708326,90.55462662)(347.92707275,91.06463186)
\curveto(348.17708276,91.58462559)(348.51208243,91.9796252)(348.93207275,92.24963186)
\curveto(349.01208193,92.29962488)(349.10208184,92.34462483)(349.20207275,92.38463186)
\curveto(349.29208165,92.42462475)(349.38708155,92.46962471)(349.48707275,92.51963186)
\curveto(349.58708135,92.55962462)(349.68708125,92.58962459)(349.78707275,92.60963186)
\curveto(349.88708105,92.62962455)(349.99208095,92.64962453)(350.10207275,92.66963186)
\curveto(350.15208079,92.68962449)(350.19708074,92.69462448)(350.23707275,92.68463186)
\curveto(350.27708066,92.6746245)(350.32208062,92.6796245)(350.37207275,92.69963186)
\curveto(350.42208052,92.70962447)(350.50708043,92.71462446)(350.62707275,92.71463186)
\curveto(350.7370802,92.71462446)(350.82208012,92.70962447)(350.88207275,92.69963186)
\curveto(350.94208,92.6796245)(351.00207994,92.66962451)(351.06207275,92.66963186)
\curveto(351.12207982,92.6796245)(351.18207976,92.6746245)(351.24207275,92.65463186)
\curveto(351.38207956,92.61462456)(351.51707942,92.5796246)(351.64707275,92.54963186)
\curveto(351.77707916,92.51962466)(351.90207904,92.4796247)(352.02207275,92.42963186)
\curveto(352.16207878,92.36962481)(352.28707865,92.29962488)(352.39707275,92.21963186)
\curveto(352.50707843,92.14962503)(352.61707832,92.0746251)(352.72707275,91.99463186)
\lineto(352.78707275,91.93463186)
\curveto(352.80707813,91.92462525)(352.82707811,91.90962527)(352.84707275,91.88963186)
\curveto(353.00707793,91.76962541)(353.15207779,91.63462554)(353.28207275,91.48463186)
\curveto(353.41207753,91.33462584)(353.5370774,91.174626)(353.65707275,91.00463186)
\curveto(353.87707706,90.69462648)(354.08207686,90.39962678)(354.27207275,90.11963186)
\curveto(354.41207653,89.88962729)(354.54707639,89.65962752)(354.67707275,89.42963186)
\curveto(354.80707613,89.20962797)(354.942076,88.98962819)(355.08207275,88.76963186)
\curveto(355.25207569,88.51962866)(355.43207551,88.2796289)(355.62207275,88.04963186)
\curveto(355.81207513,87.82962935)(356.0370749,87.63962954)(356.29707275,87.47963186)
\curveto(356.35707458,87.43962974)(356.41707452,87.40462977)(356.47707275,87.37463186)
\curveto(356.52707441,87.34462983)(356.59207435,87.31462986)(356.67207275,87.28463186)
\curveto(356.7420742,87.26462991)(356.80207414,87.25962992)(356.85207275,87.26963186)
\curveto(356.92207402,87.28962989)(356.97707396,87.32462985)(357.01707275,87.37463186)
\curveto(357.04707389,87.42462975)(357.06707387,87.48462969)(357.07707275,87.55463186)
\lineto(357.07707275,87.79463186)
\lineto(357.07707275,88.54463186)
\lineto(357.07707275,91.34963186)
\lineto(357.07707275,92.00963186)
\curveto(357.07707386,92.09962508)(357.08207386,92.18462499)(357.09207275,92.26463186)
\curveto(357.09207385,92.34462483)(357.11207383,92.40962477)(357.15207275,92.45963186)
\curveto(357.19207375,92.50962467)(357.26707367,92.54962463)(357.37707275,92.57963186)
\curveto(357.47707346,92.61962456)(357.57707336,92.62962455)(357.67707275,92.60963186)
\lineto(357.81207275,92.60963186)
\curveto(357.88207306,92.58962459)(357.942073,92.56962461)(357.99207275,92.54963186)
\curveto(358.0420729,92.52962465)(358.08207286,92.49462468)(358.11207275,92.44463186)
\curveto(358.15207279,92.39462478)(358.17207277,92.32462485)(358.17207275,92.23463186)
\lineto(358.17207275,91.96463186)
\lineto(358.17207275,91.06463186)
\lineto(358.17207275,87.55463186)
\lineto(358.17207275,86.48963186)
\curveto(358.17207277,86.40963077)(358.17707276,86.31963086)(358.18707275,86.21963186)
\curveto(358.18707275,86.11963106)(358.17707276,86.03463114)(358.15707275,85.96463186)
\curveto(358.08707285,85.75463142)(357.90707303,85.68963149)(357.61707275,85.76963186)
\curveto(357.57707336,85.7796314)(357.5420734,85.7796314)(357.51207275,85.76963186)
\curveto(357.47207347,85.76963141)(357.42707351,85.7796314)(357.37707275,85.79963186)
\curveto(357.29707364,85.81963136)(357.21207373,85.83963134)(357.12207275,85.85963186)
\curveto(357.03207391,85.8796313)(356.94707399,85.90463127)(356.86707275,85.93463186)
\curveto(356.37707456,86.09463108)(355.96207498,86.29463088)(355.62207275,86.53463186)
\curveto(355.37207557,86.71463046)(355.14707579,86.91963026)(354.94707275,87.14963186)
\curveto(354.7370762,87.3796298)(354.5420764,87.61962956)(354.36207275,87.86963186)
\curveto(354.18207676,88.12962905)(354.01207693,88.39462878)(353.85207275,88.66463186)
\curveto(353.68207726,88.94462823)(353.50707743,89.21462796)(353.32707275,89.47463186)
\curveto(353.24707769,89.58462759)(353.17207777,89.68962749)(353.10207275,89.78963186)
\curveto(353.03207791,89.89962728)(352.95707798,90.00962717)(352.87707275,90.11963186)
\curveto(352.84707809,90.15962702)(352.81707812,90.19462698)(352.78707275,90.22463186)
\curveto(352.74707819,90.26462691)(352.71707822,90.30462687)(352.69707275,90.34463186)
\curveto(352.58707835,90.48462669)(352.46207848,90.60962657)(352.32207275,90.71963186)
\curveto(352.29207865,90.73962644)(352.26707867,90.76462641)(352.24707275,90.79463186)
\curveto(352.21707872,90.82462635)(352.18707875,90.84962633)(352.15707275,90.86963186)
\curveto(352.05707888,90.94962623)(351.95707898,91.01462616)(351.85707275,91.06463186)
\curveto(351.75707918,91.12462605)(351.64707929,91.179626)(351.52707275,91.22963186)
\curveto(351.45707948,91.25962592)(351.38207956,91.2796259)(351.30207275,91.28963186)
\lineto(351.06207275,91.34963186)
\lineto(350.97207275,91.34963186)
\curveto(350.94208,91.35962582)(350.91208003,91.36462581)(350.88207275,91.36463186)
\curveto(350.81208013,91.38462579)(350.71708022,91.38962579)(350.59707275,91.37963186)
\curveto(350.46708047,91.3796258)(350.36708057,91.36962581)(350.29707275,91.34963186)
\curveto(350.21708072,91.32962585)(350.1420808,91.30962587)(350.07207275,91.28963186)
\curveto(349.99208095,91.2796259)(349.91208103,91.25962592)(349.83207275,91.22963186)
\curveto(349.59208135,91.11962606)(349.39208155,90.96962621)(349.23207275,90.77963186)
\curveto(349.06208188,90.59962658)(348.92208202,90.3796268)(348.81207275,90.11963186)
\curveto(348.79208215,90.04962713)(348.77708216,89.9796272)(348.76707275,89.90963186)
\curveto(348.74708219,89.83962734)(348.72708221,89.76462741)(348.70707275,89.68463186)
\curveto(348.68708225,89.60462757)(348.67708226,89.49462768)(348.67707275,89.35463186)
\curveto(348.67708226,89.22462795)(348.68708225,89.11962806)(348.70707275,89.03963186)
\curveto(348.71708222,88.9796282)(348.72208222,88.92462825)(348.72207275,88.87463186)
\curveto(348.72208222,88.82462835)(348.73208221,88.7746284)(348.75207275,88.72463186)
\curveto(348.79208215,88.62462855)(348.83208211,88.52962865)(348.87207275,88.43963186)
\curveto(348.91208203,88.35962882)(348.95708198,88.2796289)(349.00707275,88.19963186)
\curveto(349.02708191,88.16962901)(349.05208189,88.13962904)(349.08207275,88.10963186)
\curveto(349.11208183,88.08962909)(349.1370818,88.06462911)(349.15707275,88.03463186)
\lineto(349.23207275,87.95963186)
\curveto(349.25208169,87.92962925)(349.27208167,87.90462927)(349.29207275,87.88463186)
\lineto(349.50207275,87.73463186)
\curveto(349.56208138,87.69462948)(349.62708131,87.64962953)(349.69707275,87.59963186)
\curveto(349.78708115,87.53962964)(349.89208105,87.48962969)(350.01207275,87.44963186)
\curveto(350.12208082,87.41962976)(350.23208071,87.38462979)(350.34207275,87.34463186)
\curveto(350.45208049,87.30462987)(350.59708034,87.2796299)(350.77707275,87.26963186)
\curveto(350.94707999,87.25962992)(351.07207987,87.22962995)(351.15207275,87.17963186)
\curveto(351.23207971,87.12963005)(351.27707966,87.05463012)(351.28707275,86.95463186)
\curveto(351.29707964,86.85463032)(351.30207964,86.74463043)(351.30207275,86.62463186)
\curveto(351.30207964,86.58463059)(351.30707963,86.54463063)(351.31707275,86.50463186)
\curveto(351.31707962,86.46463071)(351.31207963,86.42963075)(351.30207275,86.39963186)
\curveto(351.28207966,86.34963083)(351.27207967,86.29963088)(351.27207275,86.24963186)
\curveto(351.27207967,86.20963097)(351.26207968,86.16963101)(351.24207275,86.12963186)
\curveto(351.18207976,86.03963114)(351.04707989,85.99463118)(350.83707275,85.99463186)
\lineto(350.71707275,85.99463186)
\curveto(350.65708028,86.00463117)(350.59708034,86.00963117)(350.53707275,86.00963186)
\curveto(350.46708047,86.01963116)(350.40208054,86.02963115)(350.34207275,86.03963186)
\curveto(350.23208071,86.05963112)(350.13208081,86.0796311)(350.04207275,86.09963186)
\curveto(349.942081,86.11963106)(349.84708109,86.14963103)(349.75707275,86.18963186)
\curveto(349.68708125,86.20963097)(349.62708131,86.22963095)(349.57707275,86.24963186)
\lineto(349.39707275,86.30963186)
\curveto(349.1370818,86.42963075)(348.89208205,86.58463059)(348.66207275,86.77463186)
\curveto(348.43208251,86.9746302)(348.24708269,87.18962999)(348.10707275,87.41963186)
\curveto(348.02708291,87.52962965)(347.96208298,87.64462953)(347.91207275,87.76463186)
\lineto(347.76207275,88.15463186)
\curveto(347.71208323,88.26462891)(347.68208326,88.3796288)(347.67207275,88.49963186)
\curveto(347.65208329,88.61962856)(347.62708331,88.74462843)(347.59707275,88.87463186)
\curveto(347.59708334,88.94462823)(347.59708334,89.00962817)(347.59707275,89.06963186)
\curveto(347.58708335,89.12962805)(347.57708336,89.19462798)(347.56707275,89.26463186)
}
}
{
\newrgbcolor{curcolor}{0 0 0}
\pscustom[linestyle=none,fillstyle=solid,fillcolor=curcolor]
{
\newpath
\moveto(353.08707275,101.36424123)
\lineto(353.34207275,101.36424123)
\curveto(353.42207752,101.37423353)(353.49707744,101.36923353)(353.56707275,101.34924123)
\lineto(353.80707275,101.34924123)
\lineto(353.97207275,101.34924123)
\curveto(354.07207687,101.32923357)(354.17707676,101.31923358)(354.28707275,101.31924123)
\curveto(354.38707655,101.31923358)(354.48707645,101.30923359)(354.58707275,101.28924123)
\lineto(354.73707275,101.28924123)
\curveto(354.87707606,101.25923364)(355.01707592,101.23923366)(355.15707275,101.22924123)
\curveto(355.28707565,101.21923368)(355.41707552,101.19423371)(355.54707275,101.15424123)
\curveto(355.62707531,101.13423377)(355.71207523,101.11423379)(355.80207275,101.09424123)
\lineto(356.04207275,101.03424123)
\lineto(356.34207275,100.91424123)
\curveto(356.43207451,100.88423402)(356.52207442,100.84923405)(356.61207275,100.80924123)
\curveto(356.83207411,100.70923419)(357.04707389,100.57423433)(357.25707275,100.40424123)
\curveto(357.46707347,100.24423466)(357.6370733,100.06923483)(357.76707275,99.87924123)
\curveto(357.80707313,99.82923507)(357.84707309,99.76923513)(357.88707275,99.69924123)
\curveto(357.91707302,99.63923526)(357.95207299,99.57923532)(357.99207275,99.51924123)
\curveto(358.0420729,99.43923546)(358.08207286,99.34423556)(358.11207275,99.23424123)
\curveto(358.1420728,99.12423578)(358.17207277,99.01923588)(358.20207275,98.91924123)
\curveto(358.2420727,98.80923609)(358.26707267,98.6992362)(358.27707275,98.58924123)
\curveto(358.28707265,98.47923642)(358.30207264,98.36423654)(358.32207275,98.24424123)
\curveto(358.33207261,98.2042367)(358.33207261,98.15923674)(358.32207275,98.10924123)
\curveto(358.32207262,98.06923683)(358.32707261,98.02923687)(358.33707275,97.98924123)
\curveto(358.34707259,97.94923695)(358.35207259,97.89423701)(358.35207275,97.82424123)
\curveto(358.35207259,97.75423715)(358.34707259,97.7042372)(358.33707275,97.67424123)
\curveto(358.31707262,97.62423728)(358.31207263,97.57923732)(358.32207275,97.53924123)
\curveto(358.33207261,97.4992374)(358.33207261,97.46423744)(358.32207275,97.43424123)
\lineto(358.32207275,97.34424123)
\curveto(358.30207264,97.28423762)(358.28707265,97.21923768)(358.27707275,97.14924123)
\curveto(358.27707266,97.08923781)(358.27207267,97.02423788)(358.26207275,96.95424123)
\curveto(358.21207273,96.78423812)(358.16207278,96.62423828)(358.11207275,96.47424123)
\curveto(358.06207288,96.32423858)(357.99707294,96.17923872)(357.91707275,96.03924123)
\curveto(357.87707306,95.98923891)(357.84707309,95.93423897)(357.82707275,95.87424123)
\curveto(357.79707314,95.82423908)(357.76207318,95.77423913)(357.72207275,95.72424123)
\curveto(357.5420734,95.48423942)(357.32207362,95.28423962)(357.06207275,95.12424123)
\curveto(356.80207414,94.96423994)(356.51707442,94.82424008)(356.20707275,94.70424123)
\curveto(356.06707487,94.64424026)(355.92707501,94.5992403)(355.78707275,94.56924123)
\curveto(355.6370753,94.53924036)(355.48207546,94.5042404)(355.32207275,94.46424123)
\curveto(355.21207573,94.44424046)(355.10207584,94.42924047)(354.99207275,94.41924123)
\curveto(354.88207606,94.40924049)(354.77207617,94.39424051)(354.66207275,94.37424123)
\curveto(354.62207632,94.36424054)(354.58207636,94.35924054)(354.54207275,94.35924123)
\curveto(354.50207644,94.36924053)(354.46207648,94.36924053)(354.42207275,94.35924123)
\curveto(354.37207657,94.34924055)(354.32207662,94.34424056)(354.27207275,94.34424123)
\lineto(354.10707275,94.34424123)
\curveto(354.05707688,94.32424058)(354.00707693,94.31924058)(353.95707275,94.32924123)
\curveto(353.89707704,94.33924056)(353.8420771,94.33924056)(353.79207275,94.32924123)
\curveto(353.75207719,94.31924058)(353.70707723,94.31924058)(353.65707275,94.32924123)
\curveto(353.60707733,94.33924056)(353.55707738,94.33424057)(353.50707275,94.31424123)
\curveto(353.4370775,94.29424061)(353.36207758,94.28924061)(353.28207275,94.29924123)
\curveto(353.19207775,94.30924059)(353.10707783,94.31424059)(353.02707275,94.31424123)
\curveto(352.937078,94.31424059)(352.8370781,94.30924059)(352.72707275,94.29924123)
\curveto(352.60707833,94.28924061)(352.50707843,94.29424061)(352.42707275,94.31424123)
\lineto(352.14207275,94.31424123)
\lineto(351.51207275,94.35924123)
\curveto(351.41207953,94.36924053)(351.31707962,94.37924052)(351.22707275,94.38924123)
\lineto(350.92707275,94.41924123)
\curveto(350.87708006,94.43924046)(350.82708011,94.44424046)(350.77707275,94.43424123)
\curveto(350.71708022,94.43424047)(350.66208028,94.44424046)(350.61207275,94.46424123)
\curveto(350.4420805,94.51424039)(350.27708066,94.55424035)(350.11707275,94.58424123)
\curveto(349.94708099,94.61424029)(349.78708115,94.66424024)(349.63707275,94.73424123)
\curveto(349.17708176,94.92423998)(348.80208214,95.14423976)(348.51207275,95.39424123)
\curveto(348.22208272,95.65423925)(347.97708296,96.01423889)(347.77707275,96.47424123)
\curveto(347.72708321,96.6042383)(347.69208325,96.73423817)(347.67207275,96.86424123)
\curveto(347.65208329,97.0042379)(347.62708331,97.14423776)(347.59707275,97.28424123)
\curveto(347.58708335,97.35423755)(347.58208336,97.41923748)(347.58207275,97.47924123)
\curveto(347.58208336,97.53923736)(347.57708336,97.6042373)(347.56707275,97.67424123)
\curveto(347.54708339,98.5042364)(347.69708324,99.17423573)(348.01707275,99.68424123)
\curveto(348.32708261,100.19423471)(348.76708217,100.57423433)(349.33707275,100.82424123)
\curveto(349.45708148,100.87423403)(349.58208136,100.91923398)(349.71207275,100.95924123)
\curveto(349.8420811,100.9992339)(349.97708096,101.04423386)(350.11707275,101.09424123)
\curveto(350.19708074,101.11423379)(350.28208066,101.12923377)(350.37207275,101.13924123)
\lineto(350.61207275,101.19924123)
\curveto(350.72208022,101.22923367)(350.83208011,101.24423366)(350.94207275,101.24424123)
\curveto(351.05207989,101.25423365)(351.16207978,101.26923363)(351.27207275,101.28924123)
\curveto(351.32207962,101.30923359)(351.36707957,101.31423359)(351.40707275,101.30424123)
\curveto(351.44707949,101.3042336)(351.48707945,101.30923359)(351.52707275,101.31924123)
\curveto(351.57707936,101.32923357)(351.63207931,101.32923357)(351.69207275,101.31924123)
\curveto(351.7420792,101.31923358)(351.79207915,101.32423358)(351.84207275,101.33424123)
\lineto(351.97707275,101.33424123)
\curveto(352.0370789,101.35423355)(352.10707883,101.35423355)(352.18707275,101.33424123)
\curveto(352.25707868,101.32423358)(352.32207862,101.32923357)(352.38207275,101.34924123)
\curveto(352.41207853,101.35923354)(352.45207849,101.36423354)(352.50207275,101.36424123)
\lineto(352.62207275,101.36424123)
\lineto(353.08707275,101.36424123)
\moveto(355.41207275,99.81924123)
\curveto(355.09207585,99.91923498)(354.72707621,99.97923492)(354.31707275,99.99924123)
\curveto(353.90707703,100.01923488)(353.49707744,100.02923487)(353.08707275,100.02924123)
\curveto(352.65707828,100.02923487)(352.2370787,100.01923488)(351.82707275,99.99924123)
\curveto(351.41707952,99.97923492)(351.03207991,99.93423497)(350.67207275,99.86424123)
\curveto(350.31208063,99.79423511)(349.99208095,99.68423522)(349.71207275,99.53424123)
\curveto(349.42208152,99.39423551)(349.18708175,99.1992357)(349.00707275,98.94924123)
\curveto(348.89708204,98.78923611)(348.81708212,98.60923629)(348.76707275,98.40924123)
\curveto(348.70708223,98.20923669)(348.67708226,97.96423694)(348.67707275,97.67424123)
\curveto(348.69708224,97.65423725)(348.70708223,97.61923728)(348.70707275,97.56924123)
\curveto(348.69708224,97.51923738)(348.69708224,97.47923742)(348.70707275,97.44924123)
\curveto(348.72708221,97.36923753)(348.74708219,97.29423761)(348.76707275,97.22424123)
\curveto(348.77708216,97.16423774)(348.79708214,97.0992378)(348.82707275,97.02924123)
\curveto(348.94708199,96.75923814)(349.11708182,96.53923836)(349.33707275,96.36924123)
\curveto(349.54708139,96.20923869)(349.79208115,96.07423883)(350.07207275,95.96424123)
\curveto(350.18208076,95.91423899)(350.30208064,95.87423903)(350.43207275,95.84424123)
\curveto(350.55208039,95.82423908)(350.67708026,95.7992391)(350.80707275,95.76924123)
\curveto(350.85708008,95.74923915)(350.91208003,95.73923916)(350.97207275,95.73924123)
\curveto(351.02207992,95.73923916)(351.07207987,95.73423917)(351.12207275,95.72424123)
\curveto(351.21207973,95.71423919)(351.30707963,95.7042392)(351.40707275,95.69424123)
\curveto(351.49707944,95.68423922)(351.59207935,95.67423923)(351.69207275,95.66424123)
\curveto(351.77207917,95.66423924)(351.85707908,95.65923924)(351.94707275,95.64924123)
\lineto(352.18707275,95.64924123)
\lineto(352.36707275,95.64924123)
\curveto(352.39707854,95.63923926)(352.43207851,95.63423927)(352.47207275,95.63424123)
\lineto(352.60707275,95.63424123)
\lineto(353.05707275,95.63424123)
\curveto(353.1370778,95.63423927)(353.22207772,95.62923927)(353.31207275,95.61924123)
\curveto(353.39207755,95.61923928)(353.46707747,95.62923927)(353.53707275,95.64924123)
\lineto(353.80707275,95.64924123)
\curveto(353.82707711,95.64923925)(353.85707708,95.64423926)(353.89707275,95.63424123)
\curveto(353.92707701,95.63423927)(353.95207699,95.63923926)(353.97207275,95.64924123)
\curveto(354.07207687,95.65923924)(354.17207677,95.66423924)(354.27207275,95.66424123)
\curveto(354.36207658,95.67423923)(354.46207648,95.68423922)(354.57207275,95.69424123)
\curveto(354.69207625,95.72423918)(354.81707612,95.73923916)(354.94707275,95.73924123)
\curveto(355.06707587,95.74923915)(355.18207576,95.77423913)(355.29207275,95.81424123)
\curveto(355.59207535,95.89423901)(355.85707508,95.97923892)(356.08707275,96.06924123)
\curveto(356.31707462,96.16923873)(356.53207441,96.31423859)(356.73207275,96.50424123)
\curveto(356.93207401,96.71423819)(357.08207386,96.97923792)(357.18207275,97.29924123)
\curveto(357.20207374,97.33923756)(357.21207373,97.37423753)(357.21207275,97.40424123)
\curveto(357.20207374,97.44423746)(357.20707373,97.48923741)(357.22707275,97.53924123)
\curveto(357.2370737,97.57923732)(357.24707369,97.64923725)(357.25707275,97.74924123)
\curveto(357.26707367,97.85923704)(357.26207368,97.94423696)(357.24207275,98.00424123)
\curveto(357.22207372,98.07423683)(357.21207373,98.14423676)(357.21207275,98.21424123)
\curveto(357.20207374,98.28423662)(357.18707375,98.34923655)(357.16707275,98.40924123)
\curveto(357.10707383,98.60923629)(357.02207392,98.78923611)(356.91207275,98.94924123)
\curveto(356.89207405,98.97923592)(356.87207407,99.0042359)(356.85207275,99.02424123)
\lineto(356.79207275,99.08424123)
\curveto(356.77207417,99.12423578)(356.73207421,99.17423573)(356.67207275,99.23424123)
\curveto(356.53207441,99.33423557)(356.40207454,99.41923548)(356.28207275,99.48924123)
\curveto(356.16207478,99.55923534)(356.01707492,99.62923527)(355.84707275,99.69924123)
\curveto(355.77707516,99.72923517)(355.70707523,99.74923515)(355.63707275,99.75924123)
\curveto(355.56707537,99.77923512)(355.49207545,99.7992351)(355.41207275,99.81924123)
}
}
{
\newrgbcolor{curcolor}{0 0 0}
\pscustom[linestyle=none,fillstyle=solid,fillcolor=curcolor]
{
\newpath
\moveto(347.56707275,106.77385061)
\curveto(347.56708337,106.87384575)(347.57708336,106.96884566)(347.59707275,107.05885061)
\curveto(347.60708333,107.14884548)(347.6370833,107.21384541)(347.68707275,107.25385061)
\curveto(347.76708317,107.31384531)(347.87208307,107.34384528)(348.00207275,107.34385061)
\lineto(348.39207275,107.34385061)
\lineto(349.89207275,107.34385061)
\lineto(356.28207275,107.34385061)
\lineto(357.45207275,107.34385061)
\lineto(357.76707275,107.34385061)
\curveto(357.86707307,107.35384527)(357.94707299,107.33884529)(358.00707275,107.29885061)
\curveto(358.08707285,107.24884538)(358.1370728,107.17384545)(358.15707275,107.07385061)
\curveto(358.16707277,106.98384564)(358.17207277,106.87384575)(358.17207275,106.74385061)
\lineto(358.17207275,106.51885061)
\curveto(358.15207279,106.43884619)(358.1370728,106.36884626)(358.12707275,106.30885061)
\curveto(358.10707283,106.24884638)(358.06707287,106.19884643)(358.00707275,106.15885061)
\curveto(357.94707299,106.11884651)(357.87207307,106.09884653)(357.78207275,106.09885061)
\lineto(357.48207275,106.09885061)
\lineto(356.38707275,106.09885061)
\lineto(351.04707275,106.09885061)
\curveto(350.95707998,106.07884655)(350.88208006,106.06384656)(350.82207275,106.05385061)
\curveto(350.75208019,106.05384657)(350.69208025,106.0238466)(350.64207275,105.96385061)
\curveto(350.59208035,105.89384673)(350.56708037,105.80384682)(350.56707275,105.69385061)
\curveto(350.55708038,105.59384703)(350.55208039,105.48384714)(350.55207275,105.36385061)
\lineto(350.55207275,104.22385061)
\lineto(350.55207275,103.72885061)
\curveto(350.5420804,103.56884906)(350.48208046,103.45884917)(350.37207275,103.39885061)
\curveto(350.3420806,103.37884925)(350.31208063,103.36884926)(350.28207275,103.36885061)
\curveto(350.2420807,103.36884926)(350.19708074,103.36384926)(350.14707275,103.35385061)
\curveto(350.02708091,103.33384929)(349.91708102,103.33884929)(349.81707275,103.36885061)
\curveto(349.71708122,103.40884922)(349.64708129,103.46384916)(349.60707275,103.53385061)
\curveto(349.55708138,103.61384901)(349.53208141,103.73384889)(349.53207275,103.89385061)
\curveto(349.53208141,104.05384857)(349.51708142,104.18884844)(349.48707275,104.29885061)
\curveto(349.47708146,104.34884828)(349.47208147,104.40384822)(349.47207275,104.46385061)
\curveto(349.46208148,104.5238481)(349.44708149,104.58384804)(349.42707275,104.64385061)
\curveto(349.37708156,104.79384783)(349.32708161,104.93884769)(349.27707275,105.07885061)
\curveto(349.21708172,105.21884741)(349.14708179,105.35384727)(349.06707275,105.48385061)
\curveto(348.97708196,105.623847)(348.87208207,105.74384688)(348.75207275,105.84385061)
\curveto(348.63208231,105.94384668)(348.50208244,106.03884659)(348.36207275,106.12885061)
\curveto(348.26208268,106.18884644)(348.15208279,106.23384639)(348.03207275,106.26385061)
\curveto(347.91208303,106.30384632)(347.80708313,106.35384627)(347.71707275,106.41385061)
\curveto(347.65708328,106.46384616)(347.61708332,106.53384609)(347.59707275,106.62385061)
\curveto(347.58708335,106.64384598)(347.58208336,106.66884596)(347.58207275,106.69885061)
\curveto(347.58208336,106.7288459)(347.57708336,106.75384587)(347.56707275,106.77385061)
}
}
{
\newrgbcolor{curcolor}{0 0 0}
\pscustom[linestyle=none,fillstyle=solid,fillcolor=curcolor]
{
\newpath
\moveto(347.56707275,115.12345998)
\curveto(347.56708337,115.22345513)(347.57708336,115.31845503)(347.59707275,115.40845998)
\curveto(347.60708333,115.49845485)(347.6370833,115.56345479)(347.68707275,115.60345998)
\curveto(347.76708317,115.66345469)(347.87208307,115.69345466)(348.00207275,115.69345998)
\lineto(348.39207275,115.69345998)
\lineto(349.89207275,115.69345998)
\lineto(356.28207275,115.69345998)
\lineto(357.45207275,115.69345998)
\lineto(357.76707275,115.69345998)
\curveto(357.86707307,115.70345465)(357.94707299,115.68845466)(358.00707275,115.64845998)
\curveto(358.08707285,115.59845475)(358.1370728,115.52345483)(358.15707275,115.42345998)
\curveto(358.16707277,115.33345502)(358.17207277,115.22345513)(358.17207275,115.09345998)
\lineto(358.17207275,114.86845998)
\curveto(358.15207279,114.78845556)(358.1370728,114.71845563)(358.12707275,114.65845998)
\curveto(358.10707283,114.59845575)(358.06707287,114.5484558)(358.00707275,114.50845998)
\curveto(357.94707299,114.46845588)(357.87207307,114.4484559)(357.78207275,114.44845998)
\lineto(357.48207275,114.44845998)
\lineto(356.38707275,114.44845998)
\lineto(351.04707275,114.44845998)
\curveto(350.95707998,114.42845592)(350.88208006,114.41345594)(350.82207275,114.40345998)
\curveto(350.75208019,114.40345595)(350.69208025,114.37345598)(350.64207275,114.31345998)
\curveto(350.59208035,114.24345611)(350.56708037,114.1534562)(350.56707275,114.04345998)
\curveto(350.55708038,113.94345641)(350.55208039,113.83345652)(350.55207275,113.71345998)
\lineto(350.55207275,112.57345998)
\lineto(350.55207275,112.07845998)
\curveto(350.5420804,111.91845843)(350.48208046,111.80845854)(350.37207275,111.74845998)
\curveto(350.3420806,111.72845862)(350.31208063,111.71845863)(350.28207275,111.71845998)
\curveto(350.2420807,111.71845863)(350.19708074,111.71345864)(350.14707275,111.70345998)
\curveto(350.02708091,111.68345867)(349.91708102,111.68845866)(349.81707275,111.71845998)
\curveto(349.71708122,111.75845859)(349.64708129,111.81345854)(349.60707275,111.88345998)
\curveto(349.55708138,111.96345839)(349.53208141,112.08345827)(349.53207275,112.24345998)
\curveto(349.53208141,112.40345795)(349.51708142,112.53845781)(349.48707275,112.64845998)
\curveto(349.47708146,112.69845765)(349.47208147,112.7534576)(349.47207275,112.81345998)
\curveto(349.46208148,112.87345748)(349.44708149,112.93345742)(349.42707275,112.99345998)
\curveto(349.37708156,113.14345721)(349.32708161,113.28845706)(349.27707275,113.42845998)
\curveto(349.21708172,113.56845678)(349.14708179,113.70345665)(349.06707275,113.83345998)
\curveto(348.97708196,113.97345638)(348.87208207,114.09345626)(348.75207275,114.19345998)
\curveto(348.63208231,114.29345606)(348.50208244,114.38845596)(348.36207275,114.47845998)
\curveto(348.26208268,114.53845581)(348.15208279,114.58345577)(348.03207275,114.61345998)
\curveto(347.91208303,114.6534557)(347.80708313,114.70345565)(347.71707275,114.76345998)
\curveto(347.65708328,114.81345554)(347.61708332,114.88345547)(347.59707275,114.97345998)
\curveto(347.58708335,114.99345536)(347.58208336,115.01845533)(347.58207275,115.04845998)
\curveto(347.58208336,115.07845527)(347.57708336,115.10345525)(347.56707275,115.12345998)
}
}
{
\newrgbcolor{curcolor}{0 0 0}
\pscustom[linestyle=none,fillstyle=solid,fillcolor=curcolor]
{
\newpath
\moveto(378.30341919,42.02236623)
\curveto(378.35341993,42.04235669)(378.41341987,42.06735666)(378.48341919,42.09736623)
\curveto(378.55341973,42.1273566)(378.62841966,42.14735658)(378.70841919,42.15736623)
\curveto(378.77841951,42.17735655)(378.84841944,42.17735655)(378.91841919,42.15736623)
\curveto(378.97841931,42.14735658)(379.02341926,42.10735662)(379.05341919,42.03736623)
\curveto(379.07341921,41.98735674)(379.0834192,41.9273568)(379.08341919,41.85736623)
\lineto(379.08341919,41.64736623)
\lineto(379.08341919,41.19736623)
\curveto(379.0834192,41.04735768)(379.05841923,40.9273578)(379.00841919,40.83736623)
\curveto(378.94841934,40.73735799)(378.84341944,40.66235807)(378.69341919,40.61236623)
\curveto(378.54341974,40.57235816)(378.40841988,40.5273582)(378.28841919,40.47736623)
\curveto(378.02842026,40.36735836)(377.75842053,40.26735846)(377.47841919,40.17736623)
\curveto(377.19842109,40.08735864)(376.92342136,39.98735874)(376.65341919,39.87736623)
\curveto(376.56342172,39.84735888)(376.47842181,39.81735891)(376.39841919,39.78736623)
\curveto(376.31842197,39.76735896)(376.24342204,39.73735899)(376.17341919,39.69736623)
\curveto(376.10342218,39.66735906)(376.04342224,39.62235911)(375.99341919,39.56236623)
\curveto(375.94342234,39.50235923)(375.90342238,39.42235931)(375.87341919,39.32236623)
\curveto(375.85342243,39.27235946)(375.84842244,39.21235952)(375.85841919,39.14236623)
\lineto(375.85841919,38.94736623)
\lineto(375.85841919,36.11236623)
\lineto(375.85841919,35.81236623)
\curveto(375.84842244,35.70236303)(375.84842244,35.59736313)(375.85841919,35.49736623)
\curveto(375.86842242,35.39736333)(375.8834224,35.30236343)(375.90341919,35.21236623)
\curveto(375.92342236,35.1323636)(375.96342232,35.07236366)(376.02341919,35.03236623)
\curveto(376.12342216,34.95236378)(376.23842205,34.89236384)(376.36841919,34.85236623)
\curveto(376.4884218,34.82236391)(376.61342167,34.78236395)(376.74341919,34.73236623)
\curveto(376.97342131,34.6323641)(377.21342107,34.53736419)(377.46341919,34.44736623)
\curveto(377.71342057,34.36736436)(377.95342033,34.27736445)(378.18341919,34.17736623)
\curveto(378.24342004,34.15736457)(378.31341997,34.1323646)(378.39341919,34.10236623)
\curveto(378.46341982,34.08236465)(378.53841975,34.05736467)(378.61841919,34.02736623)
\curveto(378.69841959,33.99736473)(378.77341951,33.96236477)(378.84341919,33.92236623)
\curveto(378.90341938,33.89236484)(378.94841934,33.85736487)(378.97841919,33.81736623)
\curveto(379.03841925,33.73736499)(379.07341921,33.6273651)(379.08341919,33.48736623)
\lineto(379.08341919,33.06736623)
\lineto(379.08341919,32.82736623)
\curveto(379.07341921,32.75736597)(379.04841924,32.69736603)(379.00841919,32.64736623)
\curveto(378.97841931,32.59736613)(378.93341935,32.56736616)(378.87341919,32.55736623)
\curveto(378.81341947,32.55736617)(378.75341953,32.56236617)(378.69341919,32.57236623)
\curveto(378.62341966,32.59236614)(378.55841973,32.61236612)(378.49841919,32.63236623)
\curveto(378.42841986,32.66236607)(378.37841991,32.68736604)(378.34841919,32.70736623)
\curveto(378.02842026,32.84736588)(377.71342057,32.97236576)(377.40341919,33.08236623)
\curveto(377.0834212,33.19236554)(376.76342152,33.31236542)(376.44341919,33.44236623)
\curveto(376.22342206,33.5323652)(376.00842228,33.61736511)(375.79841919,33.69736623)
\curveto(375.57842271,33.77736495)(375.35842293,33.86236487)(375.13841919,33.95236623)
\curveto(374.41842387,34.25236448)(373.69342459,34.53736419)(372.96341919,34.80736623)
\curveto(372.22342606,35.07736365)(371.4884268,35.36236337)(370.75841919,35.66236623)
\curveto(370.49842779,35.77236296)(370.23342805,35.87236286)(369.96341919,35.96236623)
\curveto(369.69342859,36.06236267)(369.42842886,36.16736256)(369.16841919,36.27736623)
\curveto(369.05842923,36.3273624)(368.93842935,36.37236236)(368.80841919,36.41236623)
\curveto(368.66842962,36.46236227)(368.56842972,36.5323622)(368.50841919,36.62236623)
\curveto(368.46842982,36.66236207)(368.43842985,36.727362)(368.41841919,36.81736623)
\curveto(368.40842988,36.83736189)(368.40842988,36.85736187)(368.41841919,36.87736623)
\curveto(368.41842987,36.90736182)(368.41342987,36.9323618)(368.40341919,36.95236623)
\curveto(368.40342988,37.1323616)(368.40342988,37.34236139)(368.40341919,37.58236623)
\curveto(368.39342989,37.82236091)(368.42842986,37.99736073)(368.50841919,38.10736623)
\curveto(368.56842972,38.18736054)(368.66842962,38.24736048)(368.80841919,38.28736623)
\curveto(368.93842935,38.33736039)(369.05842923,38.38736034)(369.16841919,38.43736623)
\curveto(369.39842889,38.53736019)(369.62842866,38.6273601)(369.85841919,38.70736623)
\curveto(370.0884282,38.78735994)(370.31842797,38.87735985)(370.54841919,38.97736623)
\curveto(370.74842754,39.05735967)(370.95342733,39.1323596)(371.16341919,39.20236623)
\curveto(371.37342691,39.28235945)(371.57842671,39.36735936)(371.77841919,39.45736623)
\curveto(372.50842578,39.75735897)(373.24842504,40.04235869)(373.99841919,40.31236623)
\curveto(374.73842355,40.59235814)(375.47342281,40.88735784)(376.20341919,41.19736623)
\curveto(376.29342199,41.23735749)(376.37842191,41.26735746)(376.45841919,41.28736623)
\curveto(376.53842175,41.31735741)(376.62342166,41.34735738)(376.71341919,41.37736623)
\curveto(376.97342131,41.48735724)(377.23842105,41.59235714)(377.50841919,41.69236623)
\curveto(377.77842051,41.80235693)(378.04342024,41.91235682)(378.30341919,42.02236623)
\moveto(374.65841919,38.81236623)
\curveto(374.62842366,38.90235983)(374.57842371,38.95735977)(374.50841919,38.97736623)
\curveto(374.43842385,39.00735972)(374.36342392,39.01235972)(374.28341919,38.99236623)
\curveto(374.19342409,38.98235975)(374.10842418,38.95735977)(374.02841919,38.91736623)
\curveto(373.93842435,38.88735984)(373.86342442,38.85735987)(373.80341919,38.82736623)
\curveto(373.76342452,38.80735992)(373.72842456,38.79735993)(373.69841919,38.79736623)
\curveto(373.66842462,38.79735993)(373.63342465,38.78735994)(373.59341919,38.76736623)
\lineto(373.35341919,38.67736623)
\curveto(373.26342502,38.65736007)(373.17342511,38.6273601)(373.08341919,38.58736623)
\curveto(372.72342556,38.43736029)(372.35842593,38.30236043)(371.98841919,38.18236623)
\curveto(371.60842668,38.07236066)(371.23842705,37.94236079)(370.87841919,37.79236623)
\curveto(370.76842752,37.74236099)(370.65842763,37.69736103)(370.54841919,37.65736623)
\curveto(370.43842785,37.6273611)(370.33342795,37.58736114)(370.23341919,37.53736623)
\curveto(370.1834281,37.51736121)(370.13842815,37.49236124)(370.09841919,37.46236623)
\curveto(370.04842824,37.44236129)(370.02342826,37.39236134)(370.02341919,37.31236623)
\curveto(370.04342824,37.29236144)(370.05842823,37.27236146)(370.06841919,37.25236623)
\curveto(370.07842821,37.2323615)(370.09342819,37.21236152)(370.11341919,37.19236623)
\curveto(370.16342812,37.15236158)(370.21842807,37.12236161)(370.27841919,37.10236623)
\curveto(370.32842796,37.08236165)(370.3834279,37.06236167)(370.44341919,37.04236623)
\curveto(370.55342773,36.99236174)(370.66342762,36.95236178)(370.77341919,36.92236623)
\curveto(370.8834274,36.89236184)(370.99342729,36.85236188)(371.10341919,36.80236623)
\curveto(371.49342679,36.6323621)(371.8884264,36.48236225)(372.28841919,36.35236623)
\curveto(372.6884256,36.2323625)(373.07842521,36.09236264)(373.45841919,35.93236623)
\lineto(373.60841919,35.87236623)
\curveto(373.65842463,35.86236287)(373.70842458,35.84736288)(373.75841919,35.82736623)
\lineto(373.99841919,35.73736623)
\curveto(374.07842421,35.70736302)(374.15842413,35.68236305)(374.23841919,35.66236623)
\curveto(374.288424,35.64236309)(374.34342394,35.6323631)(374.40341919,35.63236623)
\curveto(374.46342382,35.64236309)(374.51342377,35.65736307)(374.55341919,35.67736623)
\curveto(374.63342365,35.727363)(374.67842361,35.8323629)(374.68841919,35.99236623)
\lineto(374.68841919,36.44236623)
\lineto(374.68841919,38.04736623)
\curveto(374.6884236,38.15736057)(374.69342359,38.29236044)(374.70341919,38.45236623)
\curveto(374.70342358,38.61236012)(374.6884236,38.73236)(374.65841919,38.81236623)
}
}
{
\newrgbcolor{curcolor}{0 0 0}
\pscustom[linestyle=none,fillstyle=solid,fillcolor=curcolor]
{
\newpath
\moveto(375.04841919,50.56392873)
\curveto(375.09842319,50.57392038)(375.16842312,50.57892038)(375.25841919,50.57892873)
\curveto(375.33842295,50.57892038)(375.40342288,50.57392038)(375.45341919,50.56392873)
\curveto(375.49342279,50.56392039)(375.53342275,50.5589204)(375.57341919,50.54892873)
\lineto(375.69341919,50.54892873)
\curveto(375.77342251,50.52892043)(375.85342243,50.51892044)(375.93341919,50.51892873)
\curveto(376.01342227,50.51892044)(376.09342219,50.50892045)(376.17341919,50.48892873)
\curveto(376.21342207,50.47892048)(376.25342203,50.47392048)(376.29341919,50.47392873)
\curveto(376.32342196,50.47392048)(376.35842193,50.46892049)(376.39841919,50.45892873)
\curveto(376.50842178,50.42892053)(376.61342167,50.39892056)(376.71341919,50.36892873)
\curveto(376.81342147,50.34892061)(376.91342137,50.31892064)(377.01341919,50.27892873)
\curveto(377.36342092,50.13892082)(377.67842061,49.96892099)(377.95841919,49.76892873)
\curveto(378.23842005,49.56892139)(378.47841981,49.31892164)(378.67841919,49.01892873)
\curveto(378.77841951,48.86892209)(378.86341942,48.72392223)(378.93341919,48.58392873)
\curveto(378.9834193,48.47392248)(379.02341926,48.36392259)(379.05341919,48.25392873)
\curveto(379.0834192,48.1539228)(379.11341917,48.04892291)(379.14341919,47.93892873)
\curveto(379.16341912,47.86892309)(379.17341911,47.80392315)(379.17341919,47.74392873)
\curveto(379.1834191,47.68392327)(379.19841909,47.62392333)(379.21841919,47.56392873)
\lineto(379.21841919,47.41392873)
\curveto(379.23841905,47.36392359)(379.24841904,47.28892367)(379.24841919,47.18892873)
\curveto(379.25841903,47.08892387)(379.25341903,47.00892395)(379.23341919,46.94892873)
\lineto(379.23341919,46.79892873)
\curveto(379.22341906,46.7589242)(379.21841907,46.71392424)(379.21841919,46.66392873)
\curveto(379.21841907,46.62392433)(379.21341907,46.57892438)(379.20341919,46.52892873)
\curveto(379.16341912,46.37892458)(379.12841916,46.22892473)(379.09841919,46.07892873)
\curveto(379.06841922,45.93892502)(379.02341926,45.79892516)(378.96341919,45.65892873)
\curveto(378.8834194,45.4589255)(378.7834195,45.27892568)(378.66341919,45.11892873)
\lineto(378.51341919,44.93892873)
\curveto(378.45341983,44.87892608)(378.41341987,44.80892615)(378.39341919,44.72892873)
\curveto(378.3834199,44.66892629)(378.39841989,44.61892634)(378.43841919,44.57892873)
\curveto(378.46841982,44.54892641)(378.51341977,44.52392643)(378.57341919,44.50392873)
\curveto(378.63341965,44.49392646)(378.69841959,44.48392647)(378.76841919,44.47392873)
\curveto(378.82841946,44.47392648)(378.87341941,44.46392649)(378.90341919,44.44392873)
\curveto(378.95341933,44.40392655)(378.99841929,44.3589266)(379.03841919,44.30892873)
\curveto(379.05841923,44.2589267)(379.07341921,44.18892677)(379.08341919,44.09892873)
\lineto(379.08341919,43.82892873)
\curveto(379.0834192,43.73892722)(379.07841921,43.6539273)(379.06841919,43.57392873)
\curveto(379.04841924,43.49392746)(379.02841926,43.43392752)(379.00841919,43.39392873)
\curveto(378.9884193,43.37392758)(378.96341932,43.3539276)(378.93341919,43.33392873)
\lineto(378.84341919,43.27392873)
\curveto(378.76341952,43.24392771)(378.64341964,43.22892773)(378.48341919,43.22892873)
\curveto(378.32341996,43.23892772)(378.1884201,43.24392771)(378.07841919,43.24392873)
\lineto(369.27341919,43.24392873)
\curveto(369.15342913,43.24392771)(369.02842926,43.23892772)(368.89841919,43.22892873)
\curveto(368.75842953,43.22892773)(368.64842964,43.2539277)(368.56841919,43.30392873)
\curveto(368.50842978,43.34392761)(368.45842983,43.40892755)(368.41841919,43.49892873)
\curveto(368.41842987,43.51892744)(368.41842987,43.54392741)(368.41841919,43.57392873)
\curveto(368.40842988,43.60392735)(368.40342988,43.62892733)(368.40341919,43.64892873)
\curveto(368.39342989,43.78892717)(368.39342989,43.93392702)(368.40341919,44.08392873)
\curveto(368.40342988,44.24392671)(368.44342984,44.3539266)(368.52341919,44.41392873)
\curveto(368.60342968,44.46392649)(368.71842957,44.48892647)(368.86841919,44.48892873)
\lineto(369.27341919,44.48892873)
\lineto(371.02841919,44.48892873)
\lineto(371.28341919,44.48892873)
\lineto(371.56841919,44.48892873)
\curveto(371.65842663,44.49892646)(371.74342654,44.50892645)(371.82341919,44.51892873)
\curveto(371.89342639,44.53892642)(371.94342634,44.56892639)(371.97341919,44.60892873)
\curveto(372.00342628,44.64892631)(372.00842628,44.69392626)(371.98841919,44.74392873)
\curveto(371.96842632,44.79392616)(371.94842634,44.83392612)(371.92841919,44.86392873)
\curveto(371.8884264,44.91392604)(371.84842644,44.958926)(371.80841919,44.99892873)
\lineto(371.68841919,45.14892873)
\curveto(371.63842665,45.21892574)(371.59342669,45.28892567)(371.55341919,45.35892873)
\lineto(371.43341919,45.59892873)
\curveto(371.34342694,45.77892518)(371.27842701,45.99392496)(371.23841919,46.24392873)
\curveto(371.19842709,46.49392446)(371.17842711,46.74892421)(371.17841919,47.00892873)
\curveto(371.17842711,47.26892369)(371.20342708,47.52392343)(371.25341919,47.77392873)
\curveto(371.29342699,48.02392293)(371.35342693,48.24392271)(371.43341919,48.43392873)
\curveto(371.60342668,48.83392212)(371.83842645,49.17892178)(372.13841919,49.46892873)
\curveto(372.43842585,49.7589212)(372.7884255,49.98892097)(373.18841919,50.15892873)
\curveto(373.29842499,50.20892075)(373.40842488,50.24892071)(373.51841919,50.27892873)
\curveto(373.61842467,50.31892064)(373.72342456,50.3589206)(373.83341919,50.39892873)
\curveto(373.94342434,50.42892053)(374.05842423,50.44892051)(374.17841919,50.45892873)
\lineto(374.50841919,50.51892873)
\curveto(374.53842375,50.52892043)(374.57342371,50.53392042)(374.61341919,50.53392873)
\curveto(374.64342364,50.53392042)(374.67342361,50.53892042)(374.70341919,50.54892873)
\curveto(374.76342352,50.56892039)(374.82342346,50.56892039)(374.88341919,50.54892873)
\curveto(374.93342335,50.53892042)(374.9884233,50.54392041)(375.04841919,50.56392873)
\moveto(375.43841919,49.22892873)
\curveto(375.3884229,49.24892171)(375.32842296,49.2539217)(375.25841919,49.24392873)
\curveto(375.1884231,49.23392172)(375.12342316,49.22892173)(375.06341919,49.22892873)
\curveto(374.89342339,49.22892173)(374.73342355,49.21892174)(374.58341919,49.19892873)
\curveto(374.43342385,49.18892177)(374.29842399,49.1589218)(374.17841919,49.10892873)
\curveto(374.07842421,49.07892188)(373.9884243,49.0539219)(373.90841919,49.03392873)
\curveto(373.82842446,49.01392194)(373.74842454,48.98392197)(373.66841919,48.94392873)
\curveto(373.41842487,48.83392212)(373.1884251,48.68392227)(372.97841919,48.49392873)
\curveto(372.75842553,48.30392265)(372.59342569,48.08392287)(372.48341919,47.83392873)
\curveto(372.45342583,47.7539232)(372.42842586,47.67392328)(372.40841919,47.59392873)
\curveto(372.37842591,47.52392343)(372.35342593,47.44892351)(372.33341919,47.36892873)
\curveto(372.30342598,47.2589237)(372.288426,47.14892381)(372.28841919,47.03892873)
\curveto(372.27842601,46.92892403)(372.27342601,46.80892415)(372.27341919,46.67892873)
\curveto(372.283426,46.62892433)(372.29342599,46.58392437)(372.30341919,46.54392873)
\lineto(372.30341919,46.40892873)
\lineto(372.36341919,46.13892873)
\curveto(372.3834259,46.0589249)(372.41342587,45.97892498)(372.45341919,45.89892873)
\curveto(372.59342569,45.5589254)(372.80342548,45.28892567)(373.08341919,45.08892873)
\curveto(373.35342493,44.88892607)(373.67342461,44.72892623)(374.04341919,44.60892873)
\curveto(374.15342413,44.56892639)(374.26342402,44.54392641)(374.37341919,44.53392873)
\curveto(374.4834238,44.52392643)(374.59842369,44.50392645)(374.71841919,44.47392873)
\curveto(374.76842352,44.46392649)(374.81342347,44.46392649)(374.85341919,44.47392873)
\curveto(374.89342339,44.48392647)(374.93842335,44.47892648)(374.98841919,44.45892873)
\curveto(375.03842325,44.44892651)(375.11342317,44.44392651)(375.21341919,44.44392873)
\curveto(375.30342298,44.44392651)(375.37342291,44.44892651)(375.42341919,44.45892873)
\lineto(375.54341919,44.45892873)
\curveto(375.5834227,44.46892649)(375.62342266,44.47392648)(375.66341919,44.47392873)
\curveto(375.70342258,44.47392648)(375.73842255,44.47892648)(375.76841919,44.48892873)
\curveto(375.79842249,44.49892646)(375.83342245,44.50392645)(375.87341919,44.50392873)
\curveto(375.90342238,44.50392645)(375.93342235,44.50892645)(375.96341919,44.51892873)
\curveto(376.04342224,44.53892642)(376.12342216,44.5539264)(376.20341919,44.56392873)
\lineto(376.44341919,44.62392873)
\curveto(376.7834215,44.73392622)(377.07342121,44.88392607)(377.31341919,45.07392873)
\curveto(377.55342073,45.27392568)(377.75342053,45.51892544)(377.91341919,45.80892873)
\curveto(377.96342032,45.89892506)(378.00342028,45.99392496)(378.03341919,46.09392873)
\curveto(378.05342023,46.19392476)(378.07842021,46.29892466)(378.10841919,46.40892873)
\curveto(378.12842016,46.4589245)(378.13842015,46.50392445)(378.13841919,46.54392873)
\curveto(378.12842016,46.59392436)(378.12842016,46.64392431)(378.13841919,46.69392873)
\curveto(378.14842014,46.73392422)(378.15342013,46.77892418)(378.15341919,46.82892873)
\lineto(378.15341919,46.96392873)
\lineto(378.15341919,47.09892873)
\curveto(378.14342014,47.13892382)(378.13842015,47.17392378)(378.13841919,47.20392873)
\curveto(378.13842015,47.23392372)(378.13342015,47.26892369)(378.12341919,47.30892873)
\curveto(378.10342018,47.38892357)(378.0884202,47.46392349)(378.07841919,47.53392873)
\curveto(378.05842023,47.60392335)(378.03342025,47.67892328)(378.00341919,47.75892873)
\curveto(377.87342041,48.06892289)(377.70342058,48.31892264)(377.49341919,48.50892873)
\curveto(377.27342101,48.69892226)(377.00842128,48.8589221)(376.69841919,48.98892873)
\curveto(376.55842173,49.03892192)(376.41842187,49.07392188)(376.27841919,49.09392873)
\curveto(376.12842216,49.12392183)(375.97842231,49.1589218)(375.82841919,49.19892873)
\curveto(375.77842251,49.21892174)(375.73342255,49.22392173)(375.69341919,49.21392873)
\curveto(375.64342264,49.21392174)(375.59342269,49.21892174)(375.54341919,49.22892873)
\lineto(375.43841919,49.22892873)
}
}
{
\newrgbcolor{curcolor}{0 0 0}
\pscustom[linestyle=none,fillstyle=solid,fillcolor=curcolor]
{
\newpath
\moveto(371.17841919,55.69017873)
\curveto(371.17842711,55.92017394)(371.23842705,56.05017381)(371.35841919,56.08017873)
\curveto(371.46842682,56.11017375)(371.63342665,56.12517374)(371.85341919,56.12517873)
\lineto(372.13841919,56.12517873)
\curveto(372.22842606,56.12517374)(372.30342598,56.10017376)(372.36341919,56.05017873)
\curveto(372.44342584,55.99017387)(372.4884258,55.90517396)(372.49841919,55.79517873)
\curveto(372.49842579,55.68517418)(372.51342577,55.57517429)(372.54341919,55.46517873)
\curveto(372.57342571,55.32517454)(372.60342568,55.19017467)(372.63341919,55.06017873)
\curveto(372.66342562,54.94017492)(372.70342558,54.82517504)(372.75341919,54.71517873)
\curveto(372.8834254,54.42517544)(373.06342522,54.19017567)(373.29341919,54.01017873)
\curveto(373.51342477,53.83017603)(373.76842452,53.67517619)(374.05841919,53.54517873)
\curveto(374.16842412,53.50517636)(374.283424,53.47517639)(374.40341919,53.45517873)
\curveto(374.51342377,53.43517643)(374.62842366,53.41017645)(374.74841919,53.38017873)
\curveto(374.79842349,53.37017649)(374.84842344,53.3651765)(374.89841919,53.36517873)
\curveto(374.94842334,53.37517649)(374.99842329,53.37517649)(375.04841919,53.36517873)
\curveto(375.16842312,53.33517653)(375.30842298,53.32017654)(375.46841919,53.32017873)
\curveto(375.61842267,53.33017653)(375.76342252,53.33517653)(375.90341919,53.33517873)
\lineto(377.74841919,53.33517873)
\lineto(378.09341919,53.33517873)
\curveto(378.21342007,53.33517653)(378.32841996,53.33017653)(378.43841919,53.32017873)
\curveto(378.54841974,53.31017655)(378.64341964,53.30517656)(378.72341919,53.30517873)
\curveto(378.80341948,53.31517655)(378.87341941,53.29517657)(378.93341919,53.24517873)
\curveto(379.00341928,53.19517667)(379.04341924,53.11517675)(379.05341919,53.00517873)
\curveto(379.06341922,52.90517696)(379.06841922,52.79517707)(379.06841919,52.67517873)
\lineto(379.06841919,52.40517873)
\curveto(379.04841924,52.35517751)(379.03341925,52.30517756)(379.02341919,52.25517873)
\curveto(379.00341928,52.21517765)(378.97841931,52.18517768)(378.94841919,52.16517873)
\curveto(378.87841941,52.11517775)(378.79341949,52.08517778)(378.69341919,52.07517873)
\lineto(378.36341919,52.07517873)
\lineto(377.20841919,52.07517873)
\lineto(373.05341919,52.07517873)
\lineto(372.01841919,52.07517873)
\lineto(371.71841919,52.07517873)
\curveto(371.61842667,52.08517778)(371.53342675,52.11517775)(371.46341919,52.16517873)
\curveto(371.42342686,52.19517767)(371.39342689,52.24517762)(371.37341919,52.31517873)
\curveto(371.35342693,52.39517747)(371.34342694,52.48017738)(371.34341919,52.57017873)
\curveto(371.33342695,52.6601772)(371.33342695,52.75017711)(371.34341919,52.84017873)
\curveto(371.35342693,52.93017693)(371.36842692,53.00017686)(371.38841919,53.05017873)
\curveto(371.41842687,53.13017673)(371.47842681,53.18017668)(371.56841919,53.20017873)
\curveto(371.64842664,53.23017663)(371.73842655,53.24517662)(371.83841919,53.24517873)
\lineto(372.13841919,53.24517873)
\curveto(372.23842605,53.24517662)(372.32842596,53.2651766)(372.40841919,53.30517873)
\curveto(372.42842586,53.31517655)(372.44342584,53.32517654)(372.45341919,53.33517873)
\lineto(372.49841919,53.38017873)
\curveto(372.49842579,53.49017637)(372.45342583,53.58017628)(372.36341919,53.65017873)
\curveto(372.26342602,53.72017614)(372.1834261,53.78017608)(372.12341919,53.83017873)
\lineto(372.03341919,53.92017873)
\curveto(371.92342636,54.01017585)(371.80842648,54.13517573)(371.68841919,54.29517873)
\curveto(371.56842672,54.45517541)(371.47842681,54.60517526)(371.41841919,54.74517873)
\curveto(371.36842692,54.83517503)(371.33342695,54.93017493)(371.31341919,55.03017873)
\curveto(371.283427,55.13017473)(371.25342703,55.23517463)(371.22341919,55.34517873)
\curveto(371.21342707,55.40517446)(371.20842708,55.4651744)(371.20841919,55.52517873)
\curveto(371.19842709,55.58517428)(371.1884271,55.64017422)(371.17841919,55.69017873)
}
}
{
\newrgbcolor{curcolor}{0 0 0}
\pscustom[linestyle=none,fillstyle=solid,fillcolor=curcolor]
{
}
}
{
\newrgbcolor{curcolor}{0 0 0}
\pscustom[linestyle=none,fillstyle=solid,fillcolor=curcolor]
{
\newpath
\moveto(368.47841919,65.05510061)
\curveto(368.47842981,65.15509575)(368.4884298,65.25009566)(368.50841919,65.34010061)
\curveto(368.51842977,65.43009548)(368.54842974,65.49509541)(368.59841919,65.53510061)
\curveto(368.67842961,65.59509531)(368.7834295,65.62509528)(368.91341919,65.62510061)
\lineto(369.30341919,65.62510061)
\lineto(370.80341919,65.62510061)
\lineto(377.19341919,65.62510061)
\lineto(378.36341919,65.62510061)
\lineto(378.67841919,65.62510061)
\curveto(378.77841951,65.63509527)(378.85841943,65.62009529)(378.91841919,65.58010061)
\curveto(378.99841929,65.53009538)(379.04841924,65.45509545)(379.06841919,65.35510061)
\curveto(379.07841921,65.26509564)(379.0834192,65.15509575)(379.08341919,65.02510061)
\lineto(379.08341919,64.80010061)
\curveto(379.06341922,64.72009619)(379.04841924,64.65009626)(379.03841919,64.59010061)
\curveto(379.01841927,64.53009638)(378.97841931,64.48009643)(378.91841919,64.44010061)
\curveto(378.85841943,64.40009651)(378.7834195,64.38009653)(378.69341919,64.38010061)
\lineto(378.39341919,64.38010061)
\lineto(377.29841919,64.38010061)
\lineto(371.95841919,64.38010061)
\curveto(371.86842642,64.36009655)(371.79342649,64.34509656)(371.73341919,64.33510061)
\curveto(371.66342662,64.33509657)(371.60342668,64.3050966)(371.55341919,64.24510061)
\curveto(371.50342678,64.17509673)(371.47842681,64.08509682)(371.47841919,63.97510061)
\curveto(371.46842682,63.87509703)(371.46342682,63.76509714)(371.46341919,63.64510061)
\lineto(371.46341919,62.50510061)
\lineto(371.46341919,62.01010061)
\curveto(371.45342683,61.85009906)(371.39342689,61.74009917)(371.28341919,61.68010061)
\curveto(371.25342703,61.66009925)(371.22342706,61.65009926)(371.19341919,61.65010061)
\curveto(371.15342713,61.65009926)(371.10842718,61.64509926)(371.05841919,61.63510061)
\curveto(370.93842735,61.61509929)(370.82842746,61.62009929)(370.72841919,61.65010061)
\curveto(370.62842766,61.69009922)(370.55842773,61.74509916)(370.51841919,61.81510061)
\curveto(370.46842782,61.89509901)(370.44342784,62.01509889)(370.44341919,62.17510061)
\curveto(370.44342784,62.33509857)(370.42842786,62.47009844)(370.39841919,62.58010061)
\curveto(370.3884279,62.63009828)(370.3834279,62.68509822)(370.38341919,62.74510061)
\curveto(370.37342791,62.8050981)(370.35842793,62.86509804)(370.33841919,62.92510061)
\curveto(370.288428,63.07509783)(370.23842805,63.22009769)(370.18841919,63.36010061)
\curveto(370.12842816,63.50009741)(370.05842823,63.63509727)(369.97841919,63.76510061)
\curveto(369.8884284,63.905097)(369.7834285,64.02509688)(369.66341919,64.12510061)
\curveto(369.54342874,64.22509668)(369.41342887,64.32009659)(369.27341919,64.41010061)
\curveto(369.17342911,64.47009644)(369.06342922,64.51509639)(368.94341919,64.54510061)
\curveto(368.82342946,64.58509632)(368.71842957,64.63509627)(368.62841919,64.69510061)
\curveto(368.56842972,64.74509616)(368.52842976,64.81509609)(368.50841919,64.90510061)
\curveto(368.49842979,64.92509598)(368.49342979,64.95009596)(368.49341919,64.98010061)
\curveto(368.49342979,65.0100959)(368.4884298,65.03509587)(368.47841919,65.05510061)
}
}
{
\newrgbcolor{curcolor}{0 0 0}
\pscustom[linestyle=none,fillstyle=solid,fillcolor=curcolor]
{
\newpath
\moveto(368.47841919,73.40470998)
\curveto(368.47842981,73.50470513)(368.4884298,73.59970503)(368.50841919,73.68970998)
\curveto(368.51842977,73.77970485)(368.54842974,73.84470479)(368.59841919,73.88470998)
\curveto(368.67842961,73.94470469)(368.7834295,73.97470466)(368.91341919,73.97470998)
\lineto(369.30341919,73.97470998)
\lineto(370.80341919,73.97470998)
\lineto(377.19341919,73.97470998)
\lineto(378.36341919,73.97470998)
\lineto(378.67841919,73.97470998)
\curveto(378.77841951,73.98470465)(378.85841943,73.96970466)(378.91841919,73.92970998)
\curveto(378.99841929,73.87970475)(379.04841924,73.80470483)(379.06841919,73.70470998)
\curveto(379.07841921,73.61470502)(379.0834192,73.50470513)(379.08341919,73.37470998)
\lineto(379.08341919,73.14970998)
\curveto(379.06341922,73.06970556)(379.04841924,72.99970563)(379.03841919,72.93970998)
\curveto(379.01841927,72.87970575)(378.97841931,72.8297058)(378.91841919,72.78970998)
\curveto(378.85841943,72.74970588)(378.7834195,72.7297059)(378.69341919,72.72970998)
\lineto(378.39341919,72.72970998)
\lineto(377.29841919,72.72970998)
\lineto(371.95841919,72.72970998)
\curveto(371.86842642,72.70970592)(371.79342649,72.69470594)(371.73341919,72.68470998)
\curveto(371.66342662,72.68470595)(371.60342668,72.65470598)(371.55341919,72.59470998)
\curveto(371.50342678,72.52470611)(371.47842681,72.4347062)(371.47841919,72.32470998)
\curveto(371.46842682,72.22470641)(371.46342682,72.11470652)(371.46341919,71.99470998)
\lineto(371.46341919,70.85470998)
\lineto(371.46341919,70.35970998)
\curveto(371.45342683,70.19970843)(371.39342689,70.08970854)(371.28341919,70.02970998)
\curveto(371.25342703,70.00970862)(371.22342706,69.99970863)(371.19341919,69.99970998)
\curveto(371.15342713,69.99970863)(371.10842718,69.99470864)(371.05841919,69.98470998)
\curveto(370.93842735,69.96470867)(370.82842746,69.96970866)(370.72841919,69.99970998)
\curveto(370.62842766,70.03970859)(370.55842773,70.09470854)(370.51841919,70.16470998)
\curveto(370.46842782,70.24470839)(370.44342784,70.36470827)(370.44341919,70.52470998)
\curveto(370.44342784,70.68470795)(370.42842786,70.81970781)(370.39841919,70.92970998)
\curveto(370.3884279,70.97970765)(370.3834279,71.0347076)(370.38341919,71.09470998)
\curveto(370.37342791,71.15470748)(370.35842793,71.21470742)(370.33841919,71.27470998)
\curveto(370.288428,71.42470721)(370.23842805,71.56970706)(370.18841919,71.70970998)
\curveto(370.12842816,71.84970678)(370.05842823,71.98470665)(369.97841919,72.11470998)
\curveto(369.8884284,72.25470638)(369.7834285,72.37470626)(369.66341919,72.47470998)
\curveto(369.54342874,72.57470606)(369.41342887,72.66970596)(369.27341919,72.75970998)
\curveto(369.17342911,72.81970581)(369.06342922,72.86470577)(368.94341919,72.89470998)
\curveto(368.82342946,72.9347057)(368.71842957,72.98470565)(368.62841919,73.04470998)
\curveto(368.56842972,73.09470554)(368.52842976,73.16470547)(368.50841919,73.25470998)
\curveto(368.49842979,73.27470536)(368.49342979,73.29970533)(368.49341919,73.32970998)
\curveto(368.49342979,73.35970527)(368.4884298,73.38470525)(368.47841919,73.40470998)
}
}
{
\newrgbcolor{curcolor}{0 0 0}
\pscustom[linestyle=none,fillstyle=solid,fillcolor=curcolor]
{
\newpath
\moveto(377.44841919,78.63431936)
\lineto(377.44841919,79.26431936)
\lineto(377.44841919,79.45931936)
\curveto(377.44842084,79.52931683)(377.45842083,79.58931677)(377.47841919,79.63931936)
\curveto(377.51842077,79.70931665)(377.55842073,79.7593166)(377.59841919,79.78931936)
\curveto(377.64842064,79.82931653)(377.71342057,79.84931651)(377.79341919,79.84931936)
\curveto(377.87342041,79.8593165)(377.95842033,79.86431649)(378.04841919,79.86431936)
\lineto(378.76841919,79.86431936)
\curveto(379.24841904,79.86431649)(379.65841863,79.80431655)(379.99841919,79.68431936)
\curveto(380.33841795,79.56431679)(380.61341767,79.36931699)(380.82341919,79.09931936)
\curveto(380.87341741,79.02931733)(380.91841737,78.9593174)(380.95841919,78.88931936)
\curveto(381.00841728,78.82931753)(381.05341723,78.7543176)(381.09341919,78.66431936)
\curveto(381.10341718,78.64431771)(381.11341717,78.61431774)(381.12341919,78.57431936)
\curveto(381.14341714,78.53431782)(381.14841714,78.48931787)(381.13841919,78.43931936)
\curveto(381.10841718,78.34931801)(381.03341725,78.29431806)(380.91341919,78.27431936)
\curveto(380.80341748,78.2543181)(380.70841758,78.26931809)(380.62841919,78.31931936)
\curveto(380.55841773,78.34931801)(380.49341779,78.39431796)(380.43341919,78.45431936)
\curveto(380.3834179,78.52431783)(380.33341795,78.58931777)(380.28341919,78.64931936)
\curveto(380.23341805,78.71931764)(380.15841813,78.77931758)(380.05841919,78.82931936)
\curveto(379.96841832,78.88931747)(379.87841841,78.93931742)(379.78841919,78.97931936)
\curveto(379.75841853,78.99931736)(379.69841859,79.02431733)(379.60841919,79.05431936)
\curveto(379.52841876,79.08431727)(379.45841883,79.08931727)(379.39841919,79.06931936)
\curveto(379.25841903,79.03931732)(379.16841912,78.97931738)(379.12841919,78.88931936)
\curveto(379.09841919,78.80931755)(379.0834192,78.71931764)(379.08341919,78.61931936)
\curveto(379.0834192,78.51931784)(379.05841923,78.43431792)(379.00841919,78.36431936)
\curveto(378.93841935,78.27431808)(378.79841949,78.22931813)(378.58841919,78.22931936)
\lineto(378.03341919,78.22931936)
\lineto(377.80841919,78.22931936)
\curveto(377.72842056,78.23931812)(377.66342062,78.2593181)(377.61341919,78.28931936)
\curveto(377.53342075,78.34931801)(377.4884208,78.41931794)(377.47841919,78.49931936)
\curveto(377.46842082,78.51931784)(377.46342082,78.53931782)(377.46341919,78.55931936)
\curveto(377.46342082,78.58931777)(377.45842083,78.61431774)(377.44841919,78.63431936)
}
}
{
\newrgbcolor{curcolor}{0 0 0}
\pscustom[linestyle=none,fillstyle=solid,fillcolor=curcolor]
{
}
}
{
\newrgbcolor{curcolor}{0 0 0}
\pscustom[linestyle=none,fillstyle=solid,fillcolor=curcolor]
{
\newpath
\moveto(368.47841919,89.26463186)
\curveto(368.46842982,89.95462722)(368.5884297,90.55462662)(368.83841919,91.06463186)
\curveto(369.0884292,91.58462559)(369.42342886,91.9796252)(369.84341919,92.24963186)
\curveto(369.92342836,92.29962488)(370.01342827,92.34462483)(370.11341919,92.38463186)
\curveto(370.20342808,92.42462475)(370.29842799,92.46962471)(370.39841919,92.51963186)
\curveto(370.49842779,92.55962462)(370.59842769,92.58962459)(370.69841919,92.60963186)
\curveto(370.79842749,92.62962455)(370.90342738,92.64962453)(371.01341919,92.66963186)
\curveto(371.06342722,92.68962449)(371.10842718,92.69462448)(371.14841919,92.68463186)
\curveto(371.1884271,92.6746245)(371.23342705,92.6796245)(371.28341919,92.69963186)
\curveto(371.33342695,92.70962447)(371.41842687,92.71462446)(371.53841919,92.71463186)
\curveto(371.64842664,92.71462446)(371.73342655,92.70962447)(371.79341919,92.69963186)
\curveto(371.85342643,92.6796245)(371.91342637,92.66962451)(371.97341919,92.66963186)
\curveto(372.03342625,92.6796245)(372.09342619,92.6746245)(372.15341919,92.65463186)
\curveto(372.29342599,92.61462456)(372.42842586,92.5796246)(372.55841919,92.54963186)
\curveto(372.6884256,92.51962466)(372.81342547,92.4796247)(372.93341919,92.42963186)
\curveto(373.07342521,92.36962481)(373.19842509,92.29962488)(373.30841919,92.21963186)
\curveto(373.41842487,92.14962503)(373.52842476,92.0746251)(373.63841919,91.99463186)
\lineto(373.69841919,91.93463186)
\curveto(373.71842457,91.92462525)(373.73842455,91.90962527)(373.75841919,91.88963186)
\curveto(373.91842437,91.76962541)(374.06342422,91.63462554)(374.19341919,91.48463186)
\curveto(374.32342396,91.33462584)(374.44842384,91.174626)(374.56841919,91.00463186)
\curveto(374.7884235,90.69462648)(374.99342329,90.39962678)(375.18341919,90.11963186)
\curveto(375.32342296,89.88962729)(375.45842283,89.65962752)(375.58841919,89.42963186)
\curveto(375.71842257,89.20962797)(375.85342243,88.98962819)(375.99341919,88.76963186)
\curveto(376.16342212,88.51962866)(376.34342194,88.2796289)(376.53341919,88.04963186)
\curveto(376.72342156,87.82962935)(376.94842134,87.63962954)(377.20841919,87.47963186)
\curveto(377.26842102,87.43962974)(377.32842096,87.40462977)(377.38841919,87.37463186)
\curveto(377.43842085,87.34462983)(377.50342078,87.31462986)(377.58341919,87.28463186)
\curveto(377.65342063,87.26462991)(377.71342057,87.25962992)(377.76341919,87.26963186)
\curveto(377.83342045,87.28962989)(377.8884204,87.32462985)(377.92841919,87.37463186)
\curveto(377.95842033,87.42462975)(377.97842031,87.48462969)(377.98841919,87.55463186)
\lineto(377.98841919,87.79463186)
\lineto(377.98841919,88.54463186)
\lineto(377.98841919,91.34963186)
\lineto(377.98841919,92.00963186)
\curveto(377.9884203,92.09962508)(377.99342029,92.18462499)(378.00341919,92.26463186)
\curveto(378.00342028,92.34462483)(378.02342026,92.40962477)(378.06341919,92.45963186)
\curveto(378.10342018,92.50962467)(378.17842011,92.54962463)(378.28841919,92.57963186)
\curveto(378.3884199,92.61962456)(378.4884198,92.62962455)(378.58841919,92.60963186)
\lineto(378.72341919,92.60963186)
\curveto(378.79341949,92.58962459)(378.85341943,92.56962461)(378.90341919,92.54963186)
\curveto(378.95341933,92.52962465)(378.99341929,92.49462468)(379.02341919,92.44463186)
\curveto(379.06341922,92.39462478)(379.0834192,92.32462485)(379.08341919,92.23463186)
\lineto(379.08341919,91.96463186)
\lineto(379.08341919,91.06463186)
\lineto(379.08341919,87.55463186)
\lineto(379.08341919,86.48963186)
\curveto(379.0834192,86.40963077)(379.0884192,86.31963086)(379.09841919,86.21963186)
\curveto(379.09841919,86.11963106)(379.0884192,86.03463114)(379.06841919,85.96463186)
\curveto(378.99841929,85.75463142)(378.81841947,85.68963149)(378.52841919,85.76963186)
\curveto(378.4884198,85.7796314)(378.45341983,85.7796314)(378.42341919,85.76963186)
\curveto(378.3834199,85.76963141)(378.33841995,85.7796314)(378.28841919,85.79963186)
\curveto(378.20842008,85.81963136)(378.12342016,85.83963134)(378.03341919,85.85963186)
\curveto(377.94342034,85.8796313)(377.85842043,85.90463127)(377.77841919,85.93463186)
\curveto(377.288421,86.09463108)(376.87342141,86.29463088)(376.53341919,86.53463186)
\curveto(376.283422,86.71463046)(376.05842223,86.91963026)(375.85841919,87.14963186)
\curveto(375.64842264,87.3796298)(375.45342283,87.61962956)(375.27341919,87.86963186)
\curveto(375.09342319,88.12962905)(374.92342336,88.39462878)(374.76341919,88.66463186)
\curveto(374.59342369,88.94462823)(374.41842387,89.21462796)(374.23841919,89.47463186)
\curveto(374.15842413,89.58462759)(374.0834242,89.68962749)(374.01341919,89.78963186)
\curveto(373.94342434,89.89962728)(373.86842442,90.00962717)(373.78841919,90.11963186)
\curveto(373.75842453,90.15962702)(373.72842456,90.19462698)(373.69841919,90.22463186)
\curveto(373.65842463,90.26462691)(373.62842466,90.30462687)(373.60841919,90.34463186)
\curveto(373.49842479,90.48462669)(373.37342491,90.60962657)(373.23341919,90.71963186)
\curveto(373.20342508,90.73962644)(373.17842511,90.76462641)(373.15841919,90.79463186)
\curveto(373.12842516,90.82462635)(373.09842519,90.84962633)(373.06841919,90.86963186)
\curveto(372.96842532,90.94962623)(372.86842542,91.01462616)(372.76841919,91.06463186)
\curveto(372.66842562,91.12462605)(372.55842573,91.179626)(372.43841919,91.22963186)
\curveto(372.36842592,91.25962592)(372.29342599,91.2796259)(372.21341919,91.28963186)
\lineto(371.97341919,91.34963186)
\lineto(371.88341919,91.34963186)
\curveto(371.85342643,91.35962582)(371.82342646,91.36462581)(371.79341919,91.36463186)
\curveto(371.72342656,91.38462579)(371.62842666,91.38962579)(371.50841919,91.37963186)
\curveto(371.37842691,91.3796258)(371.27842701,91.36962581)(371.20841919,91.34963186)
\curveto(371.12842716,91.32962585)(371.05342723,91.30962587)(370.98341919,91.28963186)
\curveto(370.90342738,91.2796259)(370.82342746,91.25962592)(370.74341919,91.22963186)
\curveto(370.50342778,91.11962606)(370.30342798,90.96962621)(370.14341919,90.77963186)
\curveto(369.97342831,90.59962658)(369.83342845,90.3796268)(369.72341919,90.11963186)
\curveto(369.70342858,90.04962713)(369.6884286,89.9796272)(369.67841919,89.90963186)
\curveto(369.65842863,89.83962734)(369.63842865,89.76462741)(369.61841919,89.68463186)
\curveto(369.59842869,89.60462757)(369.5884287,89.49462768)(369.58841919,89.35463186)
\curveto(369.5884287,89.22462795)(369.59842869,89.11962806)(369.61841919,89.03963186)
\curveto(369.62842866,88.9796282)(369.63342865,88.92462825)(369.63341919,88.87463186)
\curveto(369.63342865,88.82462835)(369.64342864,88.7746284)(369.66341919,88.72463186)
\curveto(369.70342858,88.62462855)(369.74342854,88.52962865)(369.78341919,88.43963186)
\curveto(369.82342846,88.35962882)(369.86842842,88.2796289)(369.91841919,88.19963186)
\curveto(369.93842835,88.16962901)(369.96342832,88.13962904)(369.99341919,88.10963186)
\curveto(370.02342826,88.08962909)(370.04842824,88.06462911)(370.06841919,88.03463186)
\lineto(370.14341919,87.95963186)
\curveto(370.16342812,87.92962925)(370.1834281,87.90462927)(370.20341919,87.88463186)
\lineto(370.41341919,87.73463186)
\curveto(370.47342781,87.69462948)(370.53842775,87.64962953)(370.60841919,87.59963186)
\curveto(370.69842759,87.53962964)(370.80342748,87.48962969)(370.92341919,87.44963186)
\curveto(371.03342725,87.41962976)(371.14342714,87.38462979)(371.25341919,87.34463186)
\curveto(371.36342692,87.30462987)(371.50842678,87.2796299)(371.68841919,87.26963186)
\curveto(371.85842643,87.25962992)(371.9834263,87.22962995)(372.06341919,87.17963186)
\curveto(372.14342614,87.12963005)(372.1884261,87.05463012)(372.19841919,86.95463186)
\curveto(372.20842608,86.85463032)(372.21342607,86.74463043)(372.21341919,86.62463186)
\curveto(372.21342607,86.58463059)(372.21842607,86.54463063)(372.22841919,86.50463186)
\curveto(372.22842606,86.46463071)(372.22342606,86.42963075)(372.21341919,86.39963186)
\curveto(372.19342609,86.34963083)(372.1834261,86.29963088)(372.18341919,86.24963186)
\curveto(372.1834261,86.20963097)(372.17342611,86.16963101)(372.15341919,86.12963186)
\curveto(372.09342619,86.03963114)(371.95842633,85.99463118)(371.74841919,85.99463186)
\lineto(371.62841919,85.99463186)
\curveto(371.56842672,86.00463117)(371.50842678,86.00963117)(371.44841919,86.00963186)
\curveto(371.37842691,86.01963116)(371.31342697,86.02963115)(371.25341919,86.03963186)
\curveto(371.14342714,86.05963112)(371.04342724,86.0796311)(370.95341919,86.09963186)
\curveto(370.85342743,86.11963106)(370.75842753,86.14963103)(370.66841919,86.18963186)
\curveto(370.59842769,86.20963097)(370.53842775,86.22963095)(370.48841919,86.24963186)
\lineto(370.30841919,86.30963186)
\curveto(370.04842824,86.42963075)(369.80342848,86.58463059)(369.57341919,86.77463186)
\curveto(369.34342894,86.9746302)(369.15842913,87.18962999)(369.01841919,87.41963186)
\curveto(368.93842935,87.52962965)(368.87342941,87.64462953)(368.82341919,87.76463186)
\lineto(368.67341919,88.15463186)
\curveto(368.62342966,88.26462891)(368.59342969,88.3796288)(368.58341919,88.49963186)
\curveto(368.56342972,88.61962856)(368.53842975,88.74462843)(368.50841919,88.87463186)
\curveto(368.50842978,88.94462823)(368.50842978,89.00962817)(368.50841919,89.06963186)
\curveto(368.49842979,89.12962805)(368.4884298,89.19462798)(368.47841919,89.26463186)
}
}
{
\newrgbcolor{curcolor}{0 0 0}
\pscustom[linestyle=none,fillstyle=solid,fillcolor=curcolor]
{
\newpath
\moveto(373.99841919,101.36424123)
\lineto(374.25341919,101.36424123)
\curveto(374.33342395,101.37423353)(374.40842388,101.36923353)(374.47841919,101.34924123)
\lineto(374.71841919,101.34924123)
\lineto(374.88341919,101.34924123)
\curveto(374.9834233,101.32923357)(375.0884232,101.31923358)(375.19841919,101.31924123)
\curveto(375.29842299,101.31923358)(375.39842289,101.30923359)(375.49841919,101.28924123)
\lineto(375.64841919,101.28924123)
\curveto(375.7884225,101.25923364)(375.92842236,101.23923366)(376.06841919,101.22924123)
\curveto(376.19842209,101.21923368)(376.32842196,101.19423371)(376.45841919,101.15424123)
\curveto(376.53842175,101.13423377)(376.62342166,101.11423379)(376.71341919,101.09424123)
\lineto(376.95341919,101.03424123)
\lineto(377.25341919,100.91424123)
\curveto(377.34342094,100.88423402)(377.43342085,100.84923405)(377.52341919,100.80924123)
\curveto(377.74342054,100.70923419)(377.95842033,100.57423433)(378.16841919,100.40424123)
\curveto(378.37841991,100.24423466)(378.54841974,100.06923483)(378.67841919,99.87924123)
\curveto(378.71841957,99.82923507)(378.75841953,99.76923513)(378.79841919,99.69924123)
\curveto(378.82841946,99.63923526)(378.86341942,99.57923532)(378.90341919,99.51924123)
\curveto(378.95341933,99.43923546)(378.99341929,99.34423556)(379.02341919,99.23424123)
\curveto(379.05341923,99.12423578)(379.0834192,99.01923588)(379.11341919,98.91924123)
\curveto(379.15341913,98.80923609)(379.17841911,98.6992362)(379.18841919,98.58924123)
\curveto(379.19841909,98.47923642)(379.21341907,98.36423654)(379.23341919,98.24424123)
\curveto(379.24341904,98.2042367)(379.24341904,98.15923674)(379.23341919,98.10924123)
\curveto(379.23341905,98.06923683)(379.23841905,98.02923687)(379.24841919,97.98924123)
\curveto(379.25841903,97.94923695)(379.26341902,97.89423701)(379.26341919,97.82424123)
\curveto(379.26341902,97.75423715)(379.25841903,97.7042372)(379.24841919,97.67424123)
\curveto(379.22841906,97.62423728)(379.22341906,97.57923732)(379.23341919,97.53924123)
\curveto(379.24341904,97.4992374)(379.24341904,97.46423744)(379.23341919,97.43424123)
\lineto(379.23341919,97.34424123)
\curveto(379.21341907,97.28423762)(379.19841909,97.21923768)(379.18841919,97.14924123)
\curveto(379.1884191,97.08923781)(379.1834191,97.02423788)(379.17341919,96.95424123)
\curveto(379.12341916,96.78423812)(379.07341921,96.62423828)(379.02341919,96.47424123)
\curveto(378.97341931,96.32423858)(378.90841938,96.17923872)(378.82841919,96.03924123)
\curveto(378.7884195,95.98923891)(378.75841953,95.93423897)(378.73841919,95.87424123)
\curveto(378.70841958,95.82423908)(378.67341961,95.77423913)(378.63341919,95.72424123)
\curveto(378.45341983,95.48423942)(378.23342005,95.28423962)(377.97341919,95.12424123)
\curveto(377.71342057,94.96423994)(377.42842086,94.82424008)(377.11841919,94.70424123)
\curveto(376.97842131,94.64424026)(376.83842145,94.5992403)(376.69841919,94.56924123)
\curveto(376.54842174,94.53924036)(376.39342189,94.5042404)(376.23341919,94.46424123)
\curveto(376.12342216,94.44424046)(376.01342227,94.42924047)(375.90341919,94.41924123)
\curveto(375.79342249,94.40924049)(375.6834226,94.39424051)(375.57341919,94.37424123)
\curveto(375.53342275,94.36424054)(375.49342279,94.35924054)(375.45341919,94.35924123)
\curveto(375.41342287,94.36924053)(375.37342291,94.36924053)(375.33341919,94.35924123)
\curveto(375.283423,94.34924055)(375.23342305,94.34424056)(375.18341919,94.34424123)
\lineto(375.01841919,94.34424123)
\curveto(374.96842332,94.32424058)(374.91842337,94.31924058)(374.86841919,94.32924123)
\curveto(374.80842348,94.33924056)(374.75342353,94.33924056)(374.70341919,94.32924123)
\curveto(374.66342362,94.31924058)(374.61842367,94.31924058)(374.56841919,94.32924123)
\curveto(374.51842377,94.33924056)(374.46842382,94.33424057)(374.41841919,94.31424123)
\curveto(374.34842394,94.29424061)(374.27342401,94.28924061)(374.19341919,94.29924123)
\curveto(374.10342418,94.30924059)(374.01842427,94.31424059)(373.93841919,94.31424123)
\curveto(373.84842444,94.31424059)(373.74842454,94.30924059)(373.63841919,94.29924123)
\curveto(373.51842477,94.28924061)(373.41842487,94.29424061)(373.33841919,94.31424123)
\lineto(373.05341919,94.31424123)
\lineto(372.42341919,94.35924123)
\curveto(372.32342596,94.36924053)(372.22842606,94.37924052)(372.13841919,94.38924123)
\lineto(371.83841919,94.41924123)
\curveto(371.7884265,94.43924046)(371.73842655,94.44424046)(371.68841919,94.43424123)
\curveto(371.62842666,94.43424047)(371.57342671,94.44424046)(371.52341919,94.46424123)
\curveto(371.35342693,94.51424039)(371.1884271,94.55424035)(371.02841919,94.58424123)
\curveto(370.85842743,94.61424029)(370.69842759,94.66424024)(370.54841919,94.73424123)
\curveto(370.0884282,94.92423998)(369.71342857,95.14423976)(369.42341919,95.39424123)
\curveto(369.13342915,95.65423925)(368.8884294,96.01423889)(368.68841919,96.47424123)
\curveto(368.63842965,96.6042383)(368.60342968,96.73423817)(368.58341919,96.86424123)
\curveto(368.56342972,97.0042379)(368.53842975,97.14423776)(368.50841919,97.28424123)
\curveto(368.49842979,97.35423755)(368.49342979,97.41923748)(368.49341919,97.47924123)
\curveto(368.49342979,97.53923736)(368.4884298,97.6042373)(368.47841919,97.67424123)
\curveto(368.45842983,98.5042364)(368.60842968,99.17423573)(368.92841919,99.68424123)
\curveto(369.23842905,100.19423471)(369.67842861,100.57423433)(370.24841919,100.82424123)
\curveto(370.36842792,100.87423403)(370.49342779,100.91923398)(370.62341919,100.95924123)
\curveto(370.75342753,100.9992339)(370.8884274,101.04423386)(371.02841919,101.09424123)
\curveto(371.10842718,101.11423379)(371.19342709,101.12923377)(371.28341919,101.13924123)
\lineto(371.52341919,101.19924123)
\curveto(371.63342665,101.22923367)(371.74342654,101.24423366)(371.85341919,101.24424123)
\curveto(371.96342632,101.25423365)(372.07342621,101.26923363)(372.18341919,101.28924123)
\curveto(372.23342605,101.30923359)(372.27842601,101.31423359)(372.31841919,101.30424123)
\curveto(372.35842593,101.3042336)(372.39842589,101.30923359)(372.43841919,101.31924123)
\curveto(372.4884258,101.32923357)(372.54342574,101.32923357)(372.60341919,101.31924123)
\curveto(372.65342563,101.31923358)(372.70342558,101.32423358)(372.75341919,101.33424123)
\lineto(372.88841919,101.33424123)
\curveto(372.94842534,101.35423355)(373.01842527,101.35423355)(373.09841919,101.33424123)
\curveto(373.16842512,101.32423358)(373.23342505,101.32923357)(373.29341919,101.34924123)
\curveto(373.32342496,101.35923354)(373.36342492,101.36423354)(373.41341919,101.36424123)
\lineto(373.53341919,101.36424123)
\lineto(373.99841919,101.36424123)
\moveto(376.32341919,99.81924123)
\curveto(376.00342228,99.91923498)(375.63842265,99.97923492)(375.22841919,99.99924123)
\curveto(374.81842347,100.01923488)(374.40842388,100.02923487)(373.99841919,100.02924123)
\curveto(373.56842472,100.02923487)(373.14842514,100.01923488)(372.73841919,99.99924123)
\curveto(372.32842596,99.97923492)(371.94342634,99.93423497)(371.58341919,99.86424123)
\curveto(371.22342706,99.79423511)(370.90342738,99.68423522)(370.62341919,99.53424123)
\curveto(370.33342795,99.39423551)(370.09842819,99.1992357)(369.91841919,98.94924123)
\curveto(369.80842848,98.78923611)(369.72842856,98.60923629)(369.67841919,98.40924123)
\curveto(369.61842867,98.20923669)(369.5884287,97.96423694)(369.58841919,97.67424123)
\curveto(369.60842868,97.65423725)(369.61842867,97.61923728)(369.61841919,97.56924123)
\curveto(369.60842868,97.51923738)(369.60842868,97.47923742)(369.61841919,97.44924123)
\curveto(369.63842865,97.36923753)(369.65842863,97.29423761)(369.67841919,97.22424123)
\curveto(369.6884286,97.16423774)(369.70842858,97.0992378)(369.73841919,97.02924123)
\curveto(369.85842843,96.75923814)(370.02842826,96.53923836)(370.24841919,96.36924123)
\curveto(370.45842783,96.20923869)(370.70342758,96.07423883)(370.98341919,95.96424123)
\curveto(371.09342719,95.91423899)(371.21342707,95.87423903)(371.34341919,95.84424123)
\curveto(371.46342682,95.82423908)(371.5884267,95.7992391)(371.71841919,95.76924123)
\curveto(371.76842652,95.74923915)(371.82342646,95.73923916)(371.88341919,95.73924123)
\curveto(371.93342635,95.73923916)(371.9834263,95.73423917)(372.03341919,95.72424123)
\curveto(372.12342616,95.71423919)(372.21842607,95.7042392)(372.31841919,95.69424123)
\curveto(372.40842588,95.68423922)(372.50342578,95.67423923)(372.60341919,95.66424123)
\curveto(372.6834256,95.66423924)(372.76842552,95.65923924)(372.85841919,95.64924123)
\lineto(373.09841919,95.64924123)
\lineto(373.27841919,95.64924123)
\curveto(373.30842498,95.63923926)(373.34342494,95.63423927)(373.38341919,95.63424123)
\lineto(373.51841919,95.63424123)
\lineto(373.96841919,95.63424123)
\curveto(374.04842424,95.63423927)(374.13342415,95.62923927)(374.22341919,95.61924123)
\curveto(374.30342398,95.61923928)(374.37842391,95.62923927)(374.44841919,95.64924123)
\lineto(374.71841919,95.64924123)
\curveto(374.73842355,95.64923925)(374.76842352,95.64423926)(374.80841919,95.63424123)
\curveto(374.83842345,95.63423927)(374.86342342,95.63923926)(374.88341919,95.64924123)
\curveto(374.9834233,95.65923924)(375.0834232,95.66423924)(375.18341919,95.66424123)
\curveto(375.27342301,95.67423923)(375.37342291,95.68423922)(375.48341919,95.69424123)
\curveto(375.60342268,95.72423918)(375.72842256,95.73923916)(375.85841919,95.73924123)
\curveto(375.97842231,95.74923915)(376.09342219,95.77423913)(376.20341919,95.81424123)
\curveto(376.50342178,95.89423901)(376.76842152,95.97923892)(376.99841919,96.06924123)
\curveto(377.22842106,96.16923873)(377.44342084,96.31423859)(377.64341919,96.50424123)
\curveto(377.84342044,96.71423819)(377.99342029,96.97923792)(378.09341919,97.29924123)
\curveto(378.11342017,97.33923756)(378.12342016,97.37423753)(378.12341919,97.40424123)
\curveto(378.11342017,97.44423746)(378.11842017,97.48923741)(378.13841919,97.53924123)
\curveto(378.14842014,97.57923732)(378.15842013,97.64923725)(378.16841919,97.74924123)
\curveto(378.17842011,97.85923704)(378.17342011,97.94423696)(378.15341919,98.00424123)
\curveto(378.13342015,98.07423683)(378.12342016,98.14423676)(378.12341919,98.21424123)
\curveto(378.11342017,98.28423662)(378.09842019,98.34923655)(378.07841919,98.40924123)
\curveto(378.01842027,98.60923629)(377.93342035,98.78923611)(377.82341919,98.94924123)
\curveto(377.80342048,98.97923592)(377.7834205,99.0042359)(377.76341919,99.02424123)
\lineto(377.70341919,99.08424123)
\curveto(377.6834206,99.12423578)(377.64342064,99.17423573)(377.58341919,99.23424123)
\curveto(377.44342084,99.33423557)(377.31342097,99.41923548)(377.19341919,99.48924123)
\curveto(377.07342121,99.55923534)(376.92842136,99.62923527)(376.75841919,99.69924123)
\curveto(376.6884216,99.72923517)(376.61842167,99.74923515)(376.54841919,99.75924123)
\curveto(376.47842181,99.77923512)(376.40342188,99.7992351)(376.32341919,99.81924123)
}
}
{
\newrgbcolor{curcolor}{0 0 0}
\pscustom[linestyle=none,fillstyle=solid,fillcolor=curcolor]
{
\newpath
\moveto(368.47841919,106.77385061)
\curveto(368.47842981,106.87384575)(368.4884298,106.96884566)(368.50841919,107.05885061)
\curveto(368.51842977,107.14884548)(368.54842974,107.21384541)(368.59841919,107.25385061)
\curveto(368.67842961,107.31384531)(368.7834295,107.34384528)(368.91341919,107.34385061)
\lineto(369.30341919,107.34385061)
\lineto(370.80341919,107.34385061)
\lineto(377.19341919,107.34385061)
\lineto(378.36341919,107.34385061)
\lineto(378.67841919,107.34385061)
\curveto(378.77841951,107.35384527)(378.85841943,107.33884529)(378.91841919,107.29885061)
\curveto(378.99841929,107.24884538)(379.04841924,107.17384545)(379.06841919,107.07385061)
\curveto(379.07841921,106.98384564)(379.0834192,106.87384575)(379.08341919,106.74385061)
\lineto(379.08341919,106.51885061)
\curveto(379.06341922,106.43884619)(379.04841924,106.36884626)(379.03841919,106.30885061)
\curveto(379.01841927,106.24884638)(378.97841931,106.19884643)(378.91841919,106.15885061)
\curveto(378.85841943,106.11884651)(378.7834195,106.09884653)(378.69341919,106.09885061)
\lineto(378.39341919,106.09885061)
\lineto(377.29841919,106.09885061)
\lineto(371.95841919,106.09885061)
\curveto(371.86842642,106.07884655)(371.79342649,106.06384656)(371.73341919,106.05385061)
\curveto(371.66342662,106.05384657)(371.60342668,106.0238466)(371.55341919,105.96385061)
\curveto(371.50342678,105.89384673)(371.47842681,105.80384682)(371.47841919,105.69385061)
\curveto(371.46842682,105.59384703)(371.46342682,105.48384714)(371.46341919,105.36385061)
\lineto(371.46341919,104.22385061)
\lineto(371.46341919,103.72885061)
\curveto(371.45342683,103.56884906)(371.39342689,103.45884917)(371.28341919,103.39885061)
\curveto(371.25342703,103.37884925)(371.22342706,103.36884926)(371.19341919,103.36885061)
\curveto(371.15342713,103.36884926)(371.10842718,103.36384926)(371.05841919,103.35385061)
\curveto(370.93842735,103.33384929)(370.82842746,103.33884929)(370.72841919,103.36885061)
\curveto(370.62842766,103.40884922)(370.55842773,103.46384916)(370.51841919,103.53385061)
\curveto(370.46842782,103.61384901)(370.44342784,103.73384889)(370.44341919,103.89385061)
\curveto(370.44342784,104.05384857)(370.42842786,104.18884844)(370.39841919,104.29885061)
\curveto(370.3884279,104.34884828)(370.3834279,104.40384822)(370.38341919,104.46385061)
\curveto(370.37342791,104.5238481)(370.35842793,104.58384804)(370.33841919,104.64385061)
\curveto(370.288428,104.79384783)(370.23842805,104.93884769)(370.18841919,105.07885061)
\curveto(370.12842816,105.21884741)(370.05842823,105.35384727)(369.97841919,105.48385061)
\curveto(369.8884284,105.623847)(369.7834285,105.74384688)(369.66341919,105.84385061)
\curveto(369.54342874,105.94384668)(369.41342887,106.03884659)(369.27341919,106.12885061)
\curveto(369.17342911,106.18884644)(369.06342922,106.23384639)(368.94341919,106.26385061)
\curveto(368.82342946,106.30384632)(368.71842957,106.35384627)(368.62841919,106.41385061)
\curveto(368.56842972,106.46384616)(368.52842976,106.53384609)(368.50841919,106.62385061)
\curveto(368.49842979,106.64384598)(368.49342979,106.66884596)(368.49341919,106.69885061)
\curveto(368.49342979,106.7288459)(368.4884298,106.75384587)(368.47841919,106.77385061)
}
}
{
\newrgbcolor{curcolor}{0 0 0}
\pscustom[linestyle=none,fillstyle=solid,fillcolor=curcolor]
{
\newpath
\moveto(368.47841919,115.12345998)
\curveto(368.47842981,115.22345513)(368.4884298,115.31845503)(368.50841919,115.40845998)
\curveto(368.51842977,115.49845485)(368.54842974,115.56345479)(368.59841919,115.60345998)
\curveto(368.67842961,115.66345469)(368.7834295,115.69345466)(368.91341919,115.69345998)
\lineto(369.30341919,115.69345998)
\lineto(370.80341919,115.69345998)
\lineto(377.19341919,115.69345998)
\lineto(378.36341919,115.69345998)
\lineto(378.67841919,115.69345998)
\curveto(378.77841951,115.70345465)(378.85841943,115.68845466)(378.91841919,115.64845998)
\curveto(378.99841929,115.59845475)(379.04841924,115.52345483)(379.06841919,115.42345998)
\curveto(379.07841921,115.33345502)(379.0834192,115.22345513)(379.08341919,115.09345998)
\lineto(379.08341919,114.86845998)
\curveto(379.06341922,114.78845556)(379.04841924,114.71845563)(379.03841919,114.65845998)
\curveto(379.01841927,114.59845575)(378.97841931,114.5484558)(378.91841919,114.50845998)
\curveto(378.85841943,114.46845588)(378.7834195,114.4484559)(378.69341919,114.44845998)
\lineto(378.39341919,114.44845998)
\lineto(377.29841919,114.44845998)
\lineto(371.95841919,114.44845998)
\curveto(371.86842642,114.42845592)(371.79342649,114.41345594)(371.73341919,114.40345998)
\curveto(371.66342662,114.40345595)(371.60342668,114.37345598)(371.55341919,114.31345998)
\curveto(371.50342678,114.24345611)(371.47842681,114.1534562)(371.47841919,114.04345998)
\curveto(371.46842682,113.94345641)(371.46342682,113.83345652)(371.46341919,113.71345998)
\lineto(371.46341919,112.57345998)
\lineto(371.46341919,112.07845998)
\curveto(371.45342683,111.91845843)(371.39342689,111.80845854)(371.28341919,111.74845998)
\curveto(371.25342703,111.72845862)(371.22342706,111.71845863)(371.19341919,111.71845998)
\curveto(371.15342713,111.71845863)(371.10842718,111.71345864)(371.05841919,111.70345998)
\curveto(370.93842735,111.68345867)(370.82842746,111.68845866)(370.72841919,111.71845998)
\curveto(370.62842766,111.75845859)(370.55842773,111.81345854)(370.51841919,111.88345998)
\curveto(370.46842782,111.96345839)(370.44342784,112.08345827)(370.44341919,112.24345998)
\curveto(370.44342784,112.40345795)(370.42842786,112.53845781)(370.39841919,112.64845998)
\curveto(370.3884279,112.69845765)(370.3834279,112.7534576)(370.38341919,112.81345998)
\curveto(370.37342791,112.87345748)(370.35842793,112.93345742)(370.33841919,112.99345998)
\curveto(370.288428,113.14345721)(370.23842805,113.28845706)(370.18841919,113.42845998)
\curveto(370.12842816,113.56845678)(370.05842823,113.70345665)(369.97841919,113.83345998)
\curveto(369.8884284,113.97345638)(369.7834285,114.09345626)(369.66341919,114.19345998)
\curveto(369.54342874,114.29345606)(369.41342887,114.38845596)(369.27341919,114.47845998)
\curveto(369.17342911,114.53845581)(369.06342922,114.58345577)(368.94341919,114.61345998)
\curveto(368.82342946,114.6534557)(368.71842957,114.70345565)(368.62841919,114.76345998)
\curveto(368.56842972,114.81345554)(368.52842976,114.88345547)(368.50841919,114.97345998)
\curveto(368.49842979,114.99345536)(368.49342979,115.01845533)(368.49341919,115.04845998)
\curveto(368.49342979,115.07845527)(368.4884298,115.10345525)(368.47841919,115.12345998)
}
}
{
\newrgbcolor{curcolor}{0 0 0}
\pscustom[linestyle=none,fillstyle=solid,fillcolor=curcolor]
{
\newpath
\moveto(399.21476562,42.02236623)
\curveto(399.26476637,42.04235669)(399.32476631,42.06735666)(399.39476562,42.09736623)
\curveto(399.46476617,42.1273566)(399.5397661,42.14735658)(399.61976563,42.15736623)
\curveto(399.68976594,42.17735655)(399.75976588,42.17735655)(399.82976562,42.15736623)
\curveto(399.88976574,42.14735658)(399.9347657,42.10735662)(399.96476562,42.03736623)
\curveto(399.98476565,41.98735674)(399.99476564,41.9273568)(399.99476563,41.85736623)
\lineto(399.99476563,41.64736623)
\lineto(399.99476563,41.19736623)
\curveto(399.99476564,41.04735768)(399.96976567,40.9273578)(399.91976563,40.83736623)
\curveto(399.85976578,40.73735799)(399.75476588,40.66235807)(399.60476563,40.61236623)
\curveto(399.45476618,40.57235816)(399.31976631,40.5273582)(399.19976562,40.47736623)
\curveto(398.9397667,40.36735836)(398.66976697,40.26735846)(398.38976562,40.17736623)
\curveto(398.10976752,40.08735864)(397.8347678,39.98735874)(397.56476563,39.87736623)
\curveto(397.47476816,39.84735888)(397.38976825,39.81735891)(397.30976563,39.78736623)
\curveto(397.22976841,39.76735896)(397.15476848,39.73735899)(397.08476562,39.69736623)
\curveto(397.01476862,39.66735906)(396.95476868,39.62235911)(396.90476562,39.56236623)
\curveto(396.85476878,39.50235923)(396.81476882,39.42235931)(396.78476562,39.32236623)
\curveto(396.76476887,39.27235946)(396.75976888,39.21235952)(396.76976562,39.14236623)
\lineto(396.76976562,38.94736623)
\lineto(396.76976562,36.11236623)
\lineto(396.76976562,35.81236623)
\curveto(396.75976888,35.70236303)(396.75976888,35.59736313)(396.76976562,35.49736623)
\curveto(396.77976885,35.39736333)(396.79476884,35.30236343)(396.81476563,35.21236623)
\curveto(396.8347688,35.1323636)(396.87476876,35.07236366)(396.93476563,35.03236623)
\curveto(397.0347686,34.95236378)(397.14976848,34.89236384)(397.27976562,34.85236623)
\curveto(397.39976824,34.82236391)(397.52476811,34.78236395)(397.65476562,34.73236623)
\curveto(397.88476775,34.6323641)(398.12476751,34.53736419)(398.37476563,34.44736623)
\curveto(398.62476701,34.36736436)(398.86476677,34.27736445)(399.09476562,34.17736623)
\curveto(399.15476648,34.15736457)(399.22476641,34.1323646)(399.30476563,34.10236623)
\curveto(399.37476626,34.08236465)(399.44976618,34.05736467)(399.52976562,34.02736623)
\curveto(399.60976602,33.99736473)(399.68476595,33.96236477)(399.75476563,33.92236623)
\curveto(399.81476582,33.89236484)(399.85976578,33.85736487)(399.88976562,33.81736623)
\curveto(399.94976568,33.73736499)(399.98476565,33.6273651)(399.99476563,33.48736623)
\lineto(399.99476563,33.06736623)
\lineto(399.99476563,32.82736623)
\curveto(399.98476565,32.75736597)(399.95976568,32.69736603)(399.91976563,32.64736623)
\curveto(399.88976574,32.59736613)(399.84476579,32.56736616)(399.78476562,32.55736623)
\curveto(399.72476591,32.55736617)(399.66476597,32.56236617)(399.60476563,32.57236623)
\curveto(399.5347661,32.59236614)(399.46976617,32.61236612)(399.40976562,32.63236623)
\curveto(399.3397663,32.66236607)(399.28976634,32.68736604)(399.25976562,32.70736623)
\curveto(398.9397667,32.84736588)(398.62476701,32.97236576)(398.31476563,33.08236623)
\curveto(397.99476764,33.19236554)(397.67476796,33.31236542)(397.35476563,33.44236623)
\curveto(397.1347685,33.5323652)(396.91976871,33.61736511)(396.70976562,33.69736623)
\curveto(396.48976915,33.77736495)(396.26976937,33.86236487)(396.04976563,33.95236623)
\curveto(395.32977031,34.25236448)(394.60477103,34.53736419)(393.87476563,34.80736623)
\curveto(393.1347725,35.07736365)(392.39977323,35.36236337)(391.66976563,35.66236623)
\curveto(391.40977422,35.77236296)(391.14477449,35.87236286)(390.87476563,35.96236623)
\curveto(390.60477503,36.06236267)(390.33977529,36.16736256)(390.07976562,36.27736623)
\curveto(389.96977566,36.3273624)(389.84977579,36.37236236)(389.71976562,36.41236623)
\curveto(389.57977606,36.46236227)(389.47977616,36.5323622)(389.41976563,36.62236623)
\curveto(389.37977626,36.66236207)(389.34977629,36.727362)(389.32976562,36.81736623)
\curveto(389.31977632,36.83736189)(389.31977632,36.85736187)(389.32976562,36.87736623)
\curveto(389.3297763,36.90736182)(389.32477631,36.9323618)(389.31476563,36.95236623)
\curveto(389.31477632,37.1323616)(389.31477632,37.34236139)(389.31476563,37.58236623)
\curveto(389.30477633,37.82236091)(389.33977629,37.99736073)(389.41976563,38.10736623)
\curveto(389.47977616,38.18736054)(389.57977606,38.24736048)(389.71976562,38.28736623)
\curveto(389.84977579,38.33736039)(389.96977566,38.38736034)(390.07976562,38.43736623)
\curveto(390.30977532,38.53736019)(390.53977509,38.6273601)(390.76976562,38.70736623)
\curveto(390.99977463,38.78735994)(391.22977441,38.87735985)(391.45976562,38.97736623)
\curveto(391.65977398,39.05735967)(391.86477377,39.1323596)(392.07476562,39.20236623)
\curveto(392.28477335,39.28235945)(392.48977315,39.36735936)(392.68976563,39.45736623)
\curveto(393.41977222,39.75735897)(394.15977148,40.04235869)(394.90976562,40.31236623)
\curveto(395.64976998,40.59235814)(396.38476925,40.88735784)(397.11476563,41.19736623)
\curveto(397.20476843,41.23735749)(397.28976835,41.26735746)(397.36976563,41.28736623)
\curveto(397.44976818,41.31735741)(397.5347681,41.34735738)(397.62476563,41.37736623)
\curveto(397.88476775,41.48735724)(398.14976748,41.59235714)(398.41976563,41.69236623)
\curveto(398.68976694,41.80235693)(398.95476668,41.91235682)(399.21476562,42.02236623)
\moveto(395.56976563,38.81236623)
\curveto(395.53977009,38.90235983)(395.48977015,38.95735977)(395.41976563,38.97736623)
\curveto(395.34977028,39.00735972)(395.27477036,39.01235972)(395.19476563,38.99236623)
\curveto(395.10477053,38.98235975)(395.01977062,38.95735977)(394.93976563,38.91736623)
\curveto(394.84977078,38.88735984)(394.77477086,38.85735987)(394.71476562,38.82736623)
\curveto(394.67477096,38.80735992)(394.63977099,38.79735993)(394.60976563,38.79736623)
\curveto(394.57977105,38.79735993)(394.54477109,38.78735994)(394.50476563,38.76736623)
\lineto(394.26476562,38.67736623)
\curveto(394.17477146,38.65736007)(394.08477155,38.6273601)(393.99476563,38.58736623)
\curveto(393.634772,38.43736029)(393.26977236,38.30236043)(392.89976562,38.18236623)
\curveto(392.51977312,38.07236066)(392.14977349,37.94236079)(391.78976562,37.79236623)
\curveto(391.67977395,37.74236099)(391.56977406,37.69736103)(391.45976562,37.65736623)
\curveto(391.34977429,37.6273611)(391.24477439,37.58736114)(391.14476562,37.53736623)
\curveto(391.09477454,37.51736121)(391.04977459,37.49236124)(391.00976562,37.46236623)
\curveto(390.95977468,37.44236129)(390.9347747,37.39236134)(390.93476563,37.31236623)
\curveto(390.95477468,37.29236144)(390.96977466,37.27236146)(390.97976563,37.25236623)
\curveto(390.98977465,37.2323615)(391.00477463,37.21236152)(391.02476562,37.19236623)
\curveto(391.07477456,37.15236158)(391.12977451,37.12236161)(391.18976563,37.10236623)
\curveto(391.23977439,37.08236165)(391.29477434,37.06236167)(391.35476563,37.04236623)
\curveto(391.46477417,36.99236174)(391.57477406,36.95236178)(391.68476563,36.92236623)
\curveto(391.79477384,36.89236184)(391.90477373,36.85236188)(392.01476562,36.80236623)
\curveto(392.40477323,36.6323621)(392.79977283,36.48236225)(393.19976562,36.35236623)
\curveto(393.59977204,36.2323625)(393.98977165,36.09236264)(394.36976563,35.93236623)
\lineto(394.51976562,35.87236623)
\curveto(394.56977106,35.86236287)(394.61977102,35.84736288)(394.66976563,35.82736623)
\lineto(394.90976562,35.73736623)
\curveto(394.98977065,35.70736302)(395.06977056,35.68236305)(395.14976562,35.66236623)
\curveto(395.19977044,35.64236309)(395.25477038,35.6323631)(395.31476563,35.63236623)
\curveto(395.37477026,35.64236309)(395.42477021,35.65736307)(395.46476562,35.67736623)
\curveto(395.54477009,35.727363)(395.58977005,35.8323629)(395.59976562,35.99236623)
\lineto(395.59976562,36.44236623)
\lineto(395.59976562,38.04736623)
\curveto(395.59977004,38.15736057)(395.60477003,38.29236044)(395.61476563,38.45236623)
\curveto(395.61477002,38.61236012)(395.59977004,38.73236)(395.56976563,38.81236623)
}
}
{
\newrgbcolor{curcolor}{0 0 0}
\pscustom[linestyle=none,fillstyle=solid,fillcolor=curcolor]
{
\newpath
\moveto(395.95976562,50.56392873)
\curveto(396.00976962,50.57392038)(396.07976955,50.57892038)(396.16976563,50.57892873)
\curveto(396.24976938,50.57892038)(396.31476932,50.57392038)(396.36476563,50.56392873)
\curveto(396.40476923,50.56392039)(396.44476919,50.5589204)(396.48476563,50.54892873)
\lineto(396.60476563,50.54892873)
\curveto(396.68476895,50.52892043)(396.76476887,50.51892044)(396.84476562,50.51892873)
\curveto(396.92476871,50.51892044)(397.00476863,50.50892045)(397.08476562,50.48892873)
\curveto(397.12476851,50.47892048)(397.16476847,50.47392048)(397.20476562,50.47392873)
\curveto(397.2347684,50.47392048)(397.26976837,50.46892049)(397.30976563,50.45892873)
\curveto(397.41976821,50.42892053)(397.52476811,50.39892056)(397.62476563,50.36892873)
\curveto(397.72476791,50.34892061)(397.82476781,50.31892064)(397.92476563,50.27892873)
\curveto(398.27476736,50.13892082)(398.58976704,49.96892099)(398.86976563,49.76892873)
\curveto(399.14976648,49.56892139)(399.38976624,49.31892164)(399.58976562,49.01892873)
\curveto(399.68976594,48.86892209)(399.77476586,48.72392223)(399.84476562,48.58392873)
\curveto(399.89476574,48.47392248)(399.9347657,48.36392259)(399.96476562,48.25392873)
\curveto(399.99476564,48.1539228)(400.02476561,48.04892291)(400.05476563,47.93892873)
\curveto(400.07476556,47.86892309)(400.08476555,47.80392315)(400.08476562,47.74392873)
\curveto(400.09476554,47.68392327)(400.10976553,47.62392333)(400.12976563,47.56392873)
\lineto(400.12976563,47.41392873)
\curveto(400.14976548,47.36392359)(400.15976547,47.28892367)(400.15976562,47.18892873)
\curveto(400.16976547,47.08892387)(400.16476547,47.00892395)(400.14476562,46.94892873)
\lineto(400.14476562,46.79892873)
\curveto(400.1347655,46.7589242)(400.12976551,46.71392424)(400.12976563,46.66392873)
\curveto(400.12976551,46.62392433)(400.12476551,46.57892438)(400.11476563,46.52892873)
\curveto(400.07476556,46.37892458)(400.0397656,46.22892473)(400.00976562,46.07892873)
\curveto(399.97976565,45.93892502)(399.9347657,45.79892516)(399.87476563,45.65892873)
\curveto(399.79476584,45.4589255)(399.69476594,45.27892568)(399.57476562,45.11892873)
\lineto(399.42476563,44.93892873)
\curveto(399.36476627,44.87892608)(399.32476631,44.80892615)(399.30476563,44.72892873)
\curveto(399.29476634,44.66892629)(399.30976632,44.61892634)(399.34976562,44.57892873)
\curveto(399.37976625,44.54892641)(399.42476621,44.52392643)(399.48476563,44.50392873)
\curveto(399.54476609,44.49392646)(399.60976602,44.48392647)(399.67976563,44.47392873)
\curveto(399.7397659,44.47392648)(399.78476585,44.46392649)(399.81476563,44.44392873)
\curveto(399.86476577,44.40392655)(399.90976572,44.3589266)(399.94976562,44.30892873)
\curveto(399.96976567,44.2589267)(399.98476565,44.18892677)(399.99476563,44.09892873)
\lineto(399.99476563,43.82892873)
\curveto(399.99476564,43.73892722)(399.98976564,43.6539273)(399.97976563,43.57392873)
\curveto(399.95976568,43.49392746)(399.9397657,43.43392752)(399.91976563,43.39392873)
\curveto(399.89976574,43.37392758)(399.87476576,43.3539276)(399.84476562,43.33392873)
\lineto(399.75476563,43.27392873)
\curveto(399.67476596,43.24392771)(399.55476608,43.22892773)(399.39476562,43.22892873)
\curveto(399.2347664,43.23892772)(399.09976654,43.24392771)(398.98976563,43.24392873)
\lineto(390.18476563,43.24392873)
\curveto(390.06477557,43.24392771)(389.93977569,43.23892772)(389.80976563,43.22892873)
\curveto(389.66977596,43.22892773)(389.55977608,43.2539277)(389.47976563,43.30392873)
\curveto(389.41977622,43.34392761)(389.36977626,43.40892755)(389.32976562,43.49892873)
\curveto(389.3297763,43.51892744)(389.3297763,43.54392741)(389.32976562,43.57392873)
\curveto(389.31977632,43.60392735)(389.31477632,43.62892733)(389.31476563,43.64892873)
\curveto(389.30477633,43.78892717)(389.30477633,43.93392702)(389.31476563,44.08392873)
\curveto(389.31477632,44.24392671)(389.35477628,44.3539266)(389.43476563,44.41392873)
\curveto(389.51477612,44.46392649)(389.629776,44.48892647)(389.77976562,44.48892873)
\lineto(390.18476563,44.48892873)
\lineto(391.93976563,44.48892873)
\lineto(392.19476563,44.48892873)
\lineto(392.47976563,44.48892873)
\curveto(392.56977306,44.49892646)(392.65477298,44.50892645)(392.73476563,44.51892873)
\curveto(392.80477283,44.53892642)(392.85477278,44.56892639)(392.88476562,44.60892873)
\curveto(392.91477272,44.64892631)(392.91977272,44.69392626)(392.89976562,44.74392873)
\curveto(392.87977275,44.79392616)(392.85977278,44.83392612)(392.83976562,44.86392873)
\curveto(392.79977283,44.91392604)(392.75977288,44.958926)(392.71976562,44.99892873)
\lineto(392.59976562,45.14892873)
\curveto(392.54977309,45.21892574)(392.50477313,45.28892567)(392.46476562,45.35892873)
\lineto(392.34476562,45.59892873)
\curveto(392.25477338,45.77892518)(392.18977345,45.99392496)(392.14976562,46.24392873)
\curveto(392.10977352,46.49392446)(392.08977355,46.74892421)(392.08976562,47.00892873)
\curveto(392.08977355,47.26892369)(392.11477352,47.52392343)(392.16476562,47.77392873)
\curveto(392.20477343,48.02392293)(392.26477337,48.24392271)(392.34476562,48.43392873)
\curveto(392.51477312,48.83392212)(392.74977289,49.17892178)(393.04976563,49.46892873)
\curveto(393.34977228,49.7589212)(393.69977194,49.98892097)(394.09976562,50.15892873)
\curveto(394.20977142,50.20892075)(394.31977132,50.24892071)(394.42976563,50.27892873)
\curveto(394.52977111,50.31892064)(394.634771,50.3589206)(394.74476563,50.39892873)
\curveto(394.85477078,50.42892053)(394.96977066,50.44892051)(395.08976562,50.45892873)
\lineto(395.41976563,50.51892873)
\curveto(395.44977018,50.52892043)(395.48477015,50.53392042)(395.52476562,50.53392873)
\curveto(395.55477008,50.53392042)(395.58477005,50.53892042)(395.61476563,50.54892873)
\curveto(395.67476996,50.56892039)(395.7347699,50.56892039)(395.79476563,50.54892873)
\curveto(395.84476979,50.53892042)(395.89976974,50.54392041)(395.95976562,50.56392873)
\moveto(396.34976562,49.22892873)
\curveto(396.29976934,49.24892171)(396.23976939,49.2539217)(396.16976563,49.24392873)
\curveto(396.09976954,49.23392172)(396.0347696,49.22892173)(395.97476562,49.22892873)
\curveto(395.80476983,49.22892173)(395.64476999,49.21892174)(395.49476563,49.19892873)
\curveto(395.34477029,49.18892177)(395.20977042,49.1589218)(395.08976562,49.10892873)
\curveto(394.98977065,49.07892188)(394.89977074,49.0539219)(394.81976563,49.03392873)
\curveto(394.73977089,49.01392194)(394.65977098,48.98392197)(394.57976562,48.94392873)
\curveto(394.32977131,48.83392212)(394.09977154,48.68392227)(393.88976562,48.49392873)
\curveto(393.66977196,48.30392265)(393.50477213,48.08392287)(393.39476562,47.83392873)
\curveto(393.36477227,47.7539232)(393.33977229,47.67392328)(393.31976563,47.59392873)
\curveto(393.28977235,47.52392343)(393.26477237,47.44892351)(393.24476563,47.36892873)
\curveto(393.21477242,47.2589237)(393.19977243,47.14892381)(393.19976562,47.03892873)
\curveto(393.18977245,46.92892403)(393.18477245,46.80892415)(393.18476563,46.67892873)
\curveto(393.19477244,46.62892433)(393.20477243,46.58392437)(393.21476562,46.54392873)
\lineto(393.21476562,46.40892873)
\lineto(393.27476562,46.13892873)
\curveto(393.29477234,46.0589249)(393.32477231,45.97892498)(393.36476563,45.89892873)
\curveto(393.50477213,45.5589254)(393.71477192,45.28892567)(393.99476563,45.08892873)
\curveto(394.26477137,44.88892607)(394.58477105,44.72892623)(394.95476562,44.60892873)
\curveto(395.06477057,44.56892639)(395.17477046,44.54392641)(395.28476562,44.53392873)
\curveto(395.39477024,44.52392643)(395.50977012,44.50392645)(395.62976563,44.47392873)
\curveto(395.67976995,44.46392649)(395.72476991,44.46392649)(395.76476562,44.47392873)
\curveto(395.80476983,44.48392647)(395.84976978,44.47892648)(395.89976562,44.45892873)
\curveto(395.94976968,44.44892651)(396.02476961,44.44392651)(396.12476563,44.44392873)
\curveto(396.21476942,44.44392651)(396.28476935,44.44892651)(396.33476562,44.45892873)
\lineto(396.45476562,44.45892873)
\curveto(396.49476914,44.46892649)(396.5347691,44.47392648)(396.57476562,44.47392873)
\curveto(396.61476902,44.47392648)(396.64976898,44.47892648)(396.67976563,44.48892873)
\curveto(396.70976892,44.49892646)(396.74476889,44.50392645)(396.78476562,44.50392873)
\curveto(396.81476882,44.50392645)(396.84476879,44.50892645)(396.87476563,44.51892873)
\curveto(396.95476868,44.53892642)(397.0347686,44.5539264)(397.11476563,44.56392873)
\lineto(397.35476563,44.62392873)
\curveto(397.69476794,44.73392622)(397.98476765,44.88392607)(398.22476562,45.07392873)
\curveto(398.46476717,45.27392568)(398.66476697,45.51892544)(398.82476562,45.80892873)
\curveto(398.87476676,45.89892506)(398.91476672,45.99392496)(398.94476563,46.09392873)
\curveto(398.96476667,46.19392476)(398.98976664,46.29892466)(399.01976562,46.40892873)
\curveto(399.0397666,46.4589245)(399.04976658,46.50392445)(399.04976563,46.54392873)
\curveto(399.0397666,46.59392436)(399.0397666,46.64392431)(399.04976563,46.69392873)
\curveto(399.05976658,46.73392422)(399.06476657,46.77892418)(399.06476563,46.82892873)
\lineto(399.06476563,46.96392873)
\lineto(399.06476563,47.09892873)
\curveto(399.05476658,47.13892382)(399.04976658,47.17392378)(399.04976563,47.20392873)
\curveto(399.04976658,47.23392372)(399.04476659,47.26892369)(399.03476562,47.30892873)
\curveto(399.01476662,47.38892357)(398.99976664,47.46392349)(398.98976563,47.53392873)
\curveto(398.96976667,47.60392335)(398.94476669,47.67892328)(398.91476562,47.75892873)
\curveto(398.78476685,48.06892289)(398.61476702,48.31892264)(398.40476562,48.50892873)
\curveto(398.18476745,48.69892226)(397.91976771,48.8589221)(397.60976563,48.98892873)
\curveto(397.46976817,49.03892192)(397.32976831,49.07392188)(397.18976563,49.09392873)
\curveto(397.03976859,49.12392183)(396.88976875,49.1589218)(396.73976563,49.19892873)
\curveto(396.68976895,49.21892174)(396.64476899,49.22392173)(396.60476563,49.21392873)
\curveto(396.55476908,49.21392174)(396.50476913,49.21892174)(396.45476562,49.22892873)
\lineto(396.34976562,49.22892873)
}
}
{
\newrgbcolor{curcolor}{0 0 0}
\pscustom[linestyle=none,fillstyle=solid,fillcolor=curcolor]
{
\newpath
\moveto(392.08976562,55.69017873)
\curveto(392.08977355,55.92017394)(392.14977349,56.05017381)(392.26976562,56.08017873)
\curveto(392.37977325,56.11017375)(392.54477309,56.12517374)(392.76476562,56.12517873)
\lineto(393.04976563,56.12517873)
\curveto(393.13977249,56.12517374)(393.21477242,56.10017376)(393.27476562,56.05017873)
\curveto(393.35477228,55.99017387)(393.39977224,55.90517396)(393.40976562,55.79517873)
\curveto(393.40977222,55.68517418)(393.42477221,55.57517429)(393.45476562,55.46517873)
\curveto(393.48477215,55.32517454)(393.51477212,55.19017467)(393.54476563,55.06017873)
\curveto(393.57477206,54.94017492)(393.61477202,54.82517504)(393.66476562,54.71517873)
\curveto(393.79477184,54.42517544)(393.97477166,54.19017567)(394.20476562,54.01017873)
\curveto(394.42477121,53.83017603)(394.67977095,53.67517619)(394.96976562,53.54517873)
\curveto(395.07977055,53.50517636)(395.19477044,53.47517639)(395.31476563,53.45517873)
\curveto(395.42477021,53.43517643)(395.53977009,53.41017645)(395.65976562,53.38017873)
\curveto(395.70976992,53.37017649)(395.75976988,53.3651765)(395.80976563,53.36517873)
\curveto(395.85976978,53.37517649)(395.90976972,53.37517649)(395.95976562,53.36517873)
\curveto(396.07976955,53.33517653)(396.21976941,53.32017654)(396.37976563,53.32017873)
\curveto(396.52976911,53.33017653)(396.67476896,53.33517653)(396.81476563,53.33517873)
\lineto(398.65976562,53.33517873)
\lineto(399.00476563,53.33517873)
\curveto(399.12476651,53.33517653)(399.2397664,53.33017653)(399.34976562,53.32017873)
\curveto(399.45976618,53.31017655)(399.55476608,53.30517656)(399.63476562,53.30517873)
\curveto(399.71476592,53.31517655)(399.78476585,53.29517657)(399.84476562,53.24517873)
\curveto(399.91476572,53.19517667)(399.95476568,53.11517675)(399.96476562,53.00517873)
\curveto(399.97476566,52.90517696)(399.97976565,52.79517707)(399.97976563,52.67517873)
\lineto(399.97976563,52.40517873)
\curveto(399.95976568,52.35517751)(399.94476569,52.30517756)(399.93476563,52.25517873)
\curveto(399.91476572,52.21517765)(399.88976574,52.18517768)(399.85976563,52.16517873)
\curveto(399.78976584,52.11517775)(399.70476593,52.08517778)(399.60476563,52.07517873)
\lineto(399.27476562,52.07517873)
\lineto(398.11976563,52.07517873)
\lineto(393.96476562,52.07517873)
\lineto(392.92976563,52.07517873)
\lineto(392.62976563,52.07517873)
\curveto(392.52977311,52.08517778)(392.44477319,52.11517775)(392.37476563,52.16517873)
\curveto(392.3347733,52.19517767)(392.30477333,52.24517762)(392.28476562,52.31517873)
\curveto(392.26477337,52.39517747)(392.25477338,52.48017738)(392.25476563,52.57017873)
\curveto(392.24477339,52.6601772)(392.24477339,52.75017711)(392.25476563,52.84017873)
\curveto(392.26477337,52.93017693)(392.27977335,53.00017686)(392.29976563,53.05017873)
\curveto(392.32977331,53.13017673)(392.38977325,53.18017668)(392.47976563,53.20017873)
\curveto(392.55977308,53.23017663)(392.64977299,53.24517662)(392.74976563,53.24517873)
\lineto(393.04976563,53.24517873)
\curveto(393.14977248,53.24517662)(393.23977239,53.2651766)(393.31976563,53.30517873)
\curveto(393.33977229,53.31517655)(393.35477228,53.32517654)(393.36476563,53.33517873)
\lineto(393.40976562,53.38017873)
\curveto(393.40977222,53.49017637)(393.36477227,53.58017628)(393.27476562,53.65017873)
\curveto(393.17477246,53.72017614)(393.09477254,53.78017608)(393.03476562,53.83017873)
\lineto(392.94476563,53.92017873)
\curveto(392.8347728,54.01017585)(392.71977292,54.13517573)(392.59976562,54.29517873)
\curveto(392.47977315,54.45517541)(392.38977325,54.60517526)(392.32976562,54.74517873)
\curveto(392.27977335,54.83517503)(392.24477339,54.93017493)(392.22476562,55.03017873)
\curveto(392.19477344,55.13017473)(392.16477347,55.23517463)(392.13476562,55.34517873)
\curveto(392.12477351,55.40517446)(392.11977352,55.4651744)(392.11976563,55.52517873)
\curveto(392.10977352,55.58517428)(392.09977353,55.64017422)(392.08976562,55.69017873)
}
}
{
\newrgbcolor{curcolor}{0 0 0}
\pscustom[linestyle=none,fillstyle=solid,fillcolor=curcolor]
{
}
}
{
\newrgbcolor{curcolor}{0 0 0}
\pscustom[linestyle=none,fillstyle=solid,fillcolor=curcolor]
{
\newpath
\moveto(389.38976562,65.05510061)
\curveto(389.38977625,65.15509575)(389.39977623,65.25009566)(389.41976563,65.34010061)
\curveto(389.4297762,65.43009548)(389.45977618,65.49509541)(389.50976562,65.53510061)
\curveto(389.58977605,65.59509531)(389.69477594,65.62509528)(389.82476562,65.62510061)
\lineto(390.21476562,65.62510061)
\lineto(391.71476562,65.62510061)
\lineto(398.10476563,65.62510061)
\lineto(399.27476562,65.62510061)
\lineto(399.58976562,65.62510061)
\curveto(399.68976594,65.63509527)(399.76976587,65.62009529)(399.82976562,65.58010061)
\curveto(399.90976572,65.53009538)(399.95976568,65.45509545)(399.97976563,65.35510061)
\curveto(399.98976564,65.26509564)(399.99476564,65.15509575)(399.99476563,65.02510061)
\lineto(399.99476563,64.80010061)
\curveto(399.97476566,64.72009619)(399.95976568,64.65009626)(399.94976562,64.59010061)
\curveto(399.92976571,64.53009638)(399.88976574,64.48009643)(399.82976562,64.44010061)
\curveto(399.76976587,64.40009651)(399.69476594,64.38009653)(399.60476563,64.38010061)
\lineto(399.30476563,64.38010061)
\lineto(398.20976562,64.38010061)
\lineto(392.86976563,64.38010061)
\curveto(392.77977285,64.36009655)(392.70477293,64.34509656)(392.64476562,64.33510061)
\curveto(392.57477306,64.33509657)(392.51477312,64.3050966)(392.46476562,64.24510061)
\curveto(392.41477322,64.17509673)(392.38977325,64.08509682)(392.38976562,63.97510061)
\curveto(392.37977325,63.87509703)(392.37477326,63.76509714)(392.37476563,63.64510061)
\lineto(392.37476563,62.50510061)
\lineto(392.37476563,62.01010061)
\curveto(392.36477327,61.85009906)(392.30477333,61.74009917)(392.19476563,61.68010061)
\curveto(392.16477347,61.66009925)(392.1347735,61.65009926)(392.10476563,61.65010061)
\curveto(392.06477357,61.65009926)(392.01977362,61.64509926)(391.96976562,61.63510061)
\curveto(391.84977379,61.61509929)(391.73977389,61.62009929)(391.63976562,61.65010061)
\curveto(391.53977409,61.69009922)(391.46977416,61.74509916)(391.42976563,61.81510061)
\curveto(391.37977426,61.89509901)(391.35477428,62.01509889)(391.35476563,62.17510061)
\curveto(391.35477428,62.33509857)(391.33977429,62.47009844)(391.30976563,62.58010061)
\curveto(391.29977433,62.63009828)(391.29477434,62.68509822)(391.29476563,62.74510061)
\curveto(391.28477435,62.8050981)(391.26977436,62.86509804)(391.24976563,62.92510061)
\curveto(391.19977443,63.07509783)(391.14977449,63.22009769)(391.09976562,63.36010061)
\curveto(391.03977459,63.50009741)(390.96977466,63.63509727)(390.88976562,63.76510061)
\curveto(390.79977483,63.905097)(390.69477494,64.02509688)(390.57476562,64.12510061)
\curveto(390.45477518,64.22509668)(390.32477531,64.32009659)(390.18476563,64.41010061)
\curveto(390.08477555,64.47009644)(389.97477566,64.51509639)(389.85476563,64.54510061)
\curveto(389.7347759,64.58509632)(389.629776,64.63509627)(389.53976562,64.69510061)
\curveto(389.47977616,64.74509616)(389.43977619,64.81509609)(389.41976563,64.90510061)
\curveto(389.40977622,64.92509598)(389.40477623,64.95009596)(389.40476562,64.98010061)
\curveto(389.40477623,65.0100959)(389.39977623,65.03509587)(389.38976562,65.05510061)
}
}
{
\newrgbcolor{curcolor}{0 0 0}
\pscustom[linestyle=none,fillstyle=solid,fillcolor=curcolor]
{
\newpath
\moveto(389.58476562,70.85470998)
\lineto(389.58476562,74.45470998)
\lineto(389.58476562,75.09970998)
\curveto(389.58477605,75.17970345)(389.58977605,75.25470338)(389.59976562,75.32470998)
\curveto(389.59977603,75.39470324)(389.60977602,75.45470318)(389.62976563,75.50470998)
\curveto(389.65977598,75.57470306)(389.71977592,75.629703)(389.80976563,75.66970998)
\curveto(389.83977579,75.68970294)(389.87977576,75.69970293)(389.92976563,75.69970998)
\lineto(390.06476563,75.69970998)
\curveto(390.17477546,75.70970292)(390.27977536,75.70470293)(390.37976563,75.68470998)
\curveto(390.47977515,75.67470296)(390.54977509,75.63970299)(390.58976562,75.57970998)
\curveto(390.65977498,75.48970314)(390.69477494,75.35470328)(390.69476563,75.17470998)
\curveto(390.68477495,74.99470364)(390.67977495,74.8297038)(390.67976563,74.67970998)
\lineto(390.67976563,72.68470998)
\lineto(390.67976563,72.18970998)
\lineto(390.67976563,72.05470998)
\curveto(390.67977495,72.01470662)(390.68477495,71.97470666)(390.69476563,71.93470998)
\lineto(390.69476563,71.72470998)
\curveto(390.72477491,71.61470702)(390.76477487,71.5347071)(390.81476563,71.48470998)
\curveto(390.85477478,71.4347072)(390.90977472,71.39970723)(390.97976563,71.37970998)
\curveto(391.03977459,71.35970727)(391.10977452,71.34470729)(391.18976563,71.33470998)
\curveto(391.26977436,71.32470731)(391.35977428,71.30470733)(391.45976562,71.27470998)
\curveto(391.65977398,71.22470741)(391.86477377,71.18470745)(392.07476562,71.15470998)
\curveto(392.28477335,71.12470751)(392.48977315,71.08470755)(392.68976563,71.03470998)
\curveto(392.75977288,71.01470762)(392.82977281,71.00470763)(392.89976562,71.00470998)
\curveto(392.95977268,71.00470763)(393.02477261,70.99470764)(393.09476562,70.97470998)
\curveto(393.12477251,70.96470767)(393.16477247,70.95470768)(393.21476562,70.94470998)
\curveto(393.25477238,70.94470769)(393.29477234,70.94970768)(393.33476562,70.95970998)
\curveto(393.38477225,70.97970765)(393.42977221,71.00470763)(393.46976562,71.03470998)
\curveto(393.49977214,71.07470756)(393.50477213,71.1347075)(393.48476563,71.21470998)
\curveto(393.46477217,71.27470736)(393.43977219,71.3347073)(393.40976562,71.39470998)
\curveto(393.36977226,71.45470718)(393.3347723,71.51470712)(393.30476563,71.57470998)
\curveto(393.28477235,71.634707)(393.26977236,71.68470695)(393.25976562,71.72470998)
\curveto(393.17977245,71.91470672)(393.12477251,72.11970651)(393.09476562,72.33970998)
\curveto(393.06477257,72.56970606)(393.05477258,72.79970583)(393.06476563,73.02970998)
\curveto(393.06477257,73.26970536)(393.08977255,73.49970513)(393.13976562,73.71970998)
\curveto(393.17977245,73.93970469)(393.23977239,74.13970449)(393.31976563,74.31970998)
\curveto(393.33977229,74.36970426)(393.35977228,74.41470422)(393.37976563,74.45470998)
\curveto(393.39977224,74.50470413)(393.42477221,74.55470408)(393.45476562,74.60470998)
\curveto(393.66477197,74.95470368)(393.89477174,75.2347034)(394.14476562,75.44470998)
\curveto(394.39477124,75.66470297)(394.71977092,75.85970277)(395.11976563,76.02970998)
\curveto(395.22977041,76.07970255)(395.33977029,76.11470252)(395.44976562,76.13470998)
\curveto(395.55977008,76.15470248)(395.67476996,76.17970245)(395.79476563,76.20970998)
\curveto(395.82476981,76.21970241)(395.86976977,76.22470241)(395.92976563,76.22470998)
\curveto(395.98976965,76.24470239)(396.05976958,76.25470238)(396.13976562,76.25470998)
\curveto(396.20976942,76.25470238)(396.27476936,76.26470237)(396.33476562,76.28470998)
\lineto(396.49976563,76.28470998)
\curveto(396.54976908,76.29470234)(396.61976901,76.29970233)(396.70976562,76.29970998)
\curveto(396.79976884,76.29970233)(396.86976877,76.28970234)(396.91976563,76.26970998)
\curveto(396.97976865,76.24970238)(397.03976859,76.24470239)(397.09976562,76.25470998)
\curveto(397.14976848,76.26470237)(397.19976844,76.25970237)(397.24976563,76.23970998)
\curveto(397.40976822,76.19970243)(397.55976808,76.16470247)(397.69976562,76.13470998)
\curveto(397.8397678,76.10470253)(397.97476766,76.05970257)(398.10476563,75.99970998)
\curveto(398.47476716,75.83970279)(398.80976682,75.61970301)(399.10976563,75.33970998)
\curveto(399.40976622,75.05970357)(399.639766,74.73970389)(399.79976563,74.37970998)
\curveto(399.87976575,74.20970442)(399.95476568,74.00970462)(400.02476562,73.77970998)
\curveto(400.06476557,73.66970496)(400.08976554,73.55470508)(400.09976562,73.43470998)
\curveto(400.10976553,73.31470532)(400.12976551,73.19470544)(400.15976562,73.07470998)
\curveto(400.17976545,73.02470561)(400.17976545,72.96970566)(400.15976562,72.90970998)
\curveto(400.14976548,72.84970578)(400.15476548,72.78970584)(400.17476563,72.72970998)
\curveto(400.19476544,72.629706)(400.19476544,72.5297061)(400.17476563,72.42970998)
\lineto(400.17476563,72.29470998)
\curveto(400.15476548,72.24470639)(400.14476549,72.18470645)(400.14476562,72.11470998)
\curveto(400.15476548,72.05470658)(400.14976548,71.99970663)(400.12976563,71.94970998)
\curveto(400.11976551,71.90970672)(400.11476552,71.87470676)(400.11476563,71.84470998)
\curveto(400.11476552,71.81470682)(400.10976553,71.77970685)(400.09976562,71.73970998)
\lineto(400.03976562,71.46970998)
\curveto(400.01976561,71.37970725)(399.98976564,71.29470734)(399.94976562,71.21470998)
\curveto(399.80976582,70.87470776)(399.65476598,70.58470805)(399.48476563,70.34470998)
\curveto(399.30476633,70.10470853)(399.07476656,69.88470875)(398.79476563,69.68470998)
\curveto(398.56476707,69.5347091)(398.32476731,69.41970921)(398.07476562,69.33970998)
\curveto(398.02476761,69.31970931)(397.97976765,69.30970932)(397.93976563,69.30970998)
\curveto(397.88976774,69.30970932)(397.8397678,69.29970933)(397.78976562,69.27970998)
\curveto(397.72976791,69.25970937)(397.64976798,69.24470939)(397.54976563,69.23470998)
\curveto(397.44976818,69.2347094)(397.37476826,69.25470938)(397.32476562,69.29470998)
\curveto(397.24476839,69.34470929)(397.19976844,69.42470921)(397.18976563,69.53470998)
\curveto(397.17976845,69.64470899)(397.17476846,69.75970887)(397.17476563,69.87970998)
\lineto(397.17476563,70.04470998)
\curveto(397.17476846,70.10470853)(397.18476845,70.15970847)(397.20476562,70.20970998)
\curveto(397.22476841,70.29970833)(397.26476837,70.36970826)(397.32476562,70.41970998)
\curveto(397.41476822,70.48970814)(397.52476811,70.5347081)(397.65476562,70.55470998)
\curveto(397.77476786,70.58470805)(397.87976775,70.629708)(397.96976562,70.68970998)
\curveto(398.30976732,70.87970775)(398.57976705,71.13970749)(398.77976562,71.46970998)
\curveto(398.8397668,71.56970706)(398.88976674,71.67470696)(398.92976563,71.78470998)
\curveto(398.95976668,71.90470673)(398.99476664,72.02470661)(399.03476562,72.14470998)
\curveto(399.08476655,72.31470632)(399.10476653,72.51970611)(399.09476562,72.75970998)
\curveto(399.07476656,73.00970562)(399.0397666,73.20970542)(398.98976563,73.35970998)
\curveto(398.86976677,73.7297049)(398.70976692,74.01970461)(398.50976562,74.22970998)
\curveto(398.29976734,74.44970418)(398.01976761,74.629704)(397.66976563,74.76970998)
\curveto(397.56976807,74.81970381)(397.46476817,74.84970378)(397.35476563,74.85970998)
\curveto(397.24476839,74.87970375)(397.12976851,74.90470373)(397.00976562,74.93470998)
\lineto(396.90476562,74.93470998)
\curveto(396.86476877,74.94470369)(396.82476881,74.94970368)(396.78476562,74.94970998)
\curveto(396.75476888,74.95970367)(396.71976891,74.95970367)(396.67976563,74.94970998)
\lineto(396.55976563,74.94970998)
\curveto(396.29976934,74.94970368)(396.05476958,74.91970371)(395.82476562,74.85970998)
\curveto(395.47477016,74.74970388)(395.17977045,74.59470404)(394.93976563,74.39470998)
\curveto(394.68977095,74.19470444)(394.49477114,73.9347047)(394.35476563,73.61470998)
\lineto(394.29476563,73.43470998)
\curveto(394.27477136,73.38470525)(394.25477138,73.32470531)(394.23476563,73.25470998)
\curveto(394.21477142,73.20470543)(394.20477143,73.14470549)(394.20476562,73.07470998)
\curveto(394.19477144,73.01470562)(394.17977145,72.94970568)(394.15976562,72.87970998)
\lineto(394.15976562,72.72970998)
\curveto(394.13977149,72.68970594)(394.12977151,72.634706)(394.12976563,72.56470998)
\curveto(394.12977151,72.50470613)(394.13977149,72.44970618)(394.15976562,72.39970998)
\lineto(394.15976562,72.29470998)
\curveto(394.15977148,72.26470637)(394.16477147,72.2297064)(394.17476563,72.18970998)
\lineto(394.23476563,71.94970998)
\curveto(394.24477139,71.86970676)(394.26477137,71.78970684)(394.29476563,71.70970998)
\curveto(394.39477124,71.46970716)(394.52977111,71.23970739)(394.69976562,71.01970998)
\curveto(394.76977086,70.9297077)(394.84477079,70.84470779)(394.92476563,70.76470998)
\curveto(394.99477064,70.68470795)(395.04977058,70.58470805)(395.08976562,70.46470998)
\curveto(395.11977052,70.37470826)(395.12977051,70.2347084)(395.11976563,70.04470998)
\curveto(395.10977052,69.86470877)(395.08477055,69.74470889)(395.04476563,69.68470998)
\curveto(395.00477063,69.634709)(394.94477069,69.59470904)(394.86476563,69.56470998)
\curveto(394.78477085,69.54470909)(394.69977094,69.54470909)(394.60976563,69.56470998)
\curveto(394.48977115,69.59470904)(394.36977126,69.61470902)(394.24976563,69.62470998)
\curveto(394.11977152,69.64470899)(393.99477164,69.66970896)(393.87476563,69.69970998)
\curveto(393.8347718,69.71970891)(393.79977184,69.72470891)(393.76976562,69.71470998)
\curveto(393.72977191,69.71470892)(393.68477195,69.72470891)(393.63476562,69.74470998)
\curveto(393.54477209,69.76470887)(393.45477218,69.77970885)(393.36476563,69.78970998)
\curveto(393.26477237,69.79970883)(393.16977246,69.81970881)(393.07976562,69.84970998)
\curveto(393.01977262,69.85970877)(392.95977268,69.86470877)(392.89976562,69.86470998)
\curveto(392.83977279,69.87470876)(392.77977285,69.88970874)(392.71976562,69.90970998)
\curveto(392.51977312,69.95970867)(392.31477332,69.99470864)(392.10476563,70.01470998)
\curveto(391.88477375,70.04470859)(391.67477396,70.08470855)(391.47476562,70.13470998)
\curveto(391.37477426,70.16470847)(391.27477436,70.18470845)(391.17476563,70.19470998)
\curveto(391.07477456,70.20470843)(390.97477466,70.21970841)(390.87476563,70.23970998)
\curveto(390.84477479,70.24970838)(390.80477483,70.25470838)(390.75476563,70.25470998)
\curveto(390.64477499,70.28470835)(390.53977509,70.30470833)(390.43976563,70.31470998)
\curveto(390.3297753,70.3347083)(390.21977542,70.35970827)(390.10976563,70.38970998)
\curveto(390.0297756,70.40970822)(389.95977568,70.42470821)(389.89976562,70.43470998)
\curveto(389.8297758,70.44470819)(389.76977586,70.46970816)(389.71976562,70.50970998)
\curveto(389.68977595,70.5297081)(389.66977596,70.55970807)(389.65976562,70.59970998)
\curveto(389.63977599,70.63970799)(389.61977602,70.68470795)(389.59976562,70.73470998)
\curveto(389.59977603,70.79470784)(389.59477604,70.8347078)(389.58476562,70.85470998)
}
}
{
\newrgbcolor{curcolor}{0 0 0}
\pscustom[linestyle=none,fillstyle=solid,fillcolor=curcolor]
{
\newpath
\moveto(398.35976563,78.63431936)
\lineto(398.35976563,79.26431936)
\lineto(398.35976563,79.45931936)
\curveto(398.35976728,79.52931683)(398.36976727,79.58931677)(398.38976562,79.63931936)
\curveto(398.42976721,79.70931665)(398.46976717,79.7593166)(398.50976562,79.78931936)
\curveto(398.55976708,79.82931653)(398.62476701,79.84931651)(398.70476562,79.84931936)
\curveto(398.78476685,79.8593165)(398.86976677,79.86431649)(398.95976562,79.86431936)
\lineto(399.67976563,79.86431936)
\curveto(400.15976547,79.86431649)(400.56976507,79.80431655)(400.90976562,79.68431936)
\curveto(401.24976438,79.56431679)(401.52476411,79.36931699)(401.73476563,79.09931936)
\curveto(401.78476385,79.02931733)(401.82976381,78.9593174)(401.86976563,78.88931936)
\curveto(401.91976371,78.82931753)(401.96476367,78.7543176)(402.00476563,78.66431936)
\curveto(402.01476362,78.64431771)(402.02476361,78.61431774)(402.03476562,78.57431936)
\curveto(402.05476358,78.53431782)(402.05976357,78.48931787)(402.04976563,78.43931936)
\curveto(402.01976361,78.34931801)(401.94476369,78.29431806)(401.82476562,78.27431936)
\curveto(401.71476392,78.2543181)(401.61976401,78.26931809)(401.53976562,78.31931936)
\curveto(401.46976417,78.34931801)(401.40476423,78.39431796)(401.34476562,78.45431936)
\curveto(401.29476434,78.52431783)(401.24476439,78.58931777)(401.19476563,78.64931936)
\curveto(401.14476449,78.71931764)(401.06976457,78.77931758)(400.96976562,78.82931936)
\curveto(400.87976475,78.88931747)(400.78976484,78.93931742)(400.69976562,78.97931936)
\curveto(400.66976497,78.99931736)(400.60976503,79.02431733)(400.51976562,79.05431936)
\curveto(400.4397652,79.08431727)(400.36976527,79.08931727)(400.30976563,79.06931936)
\curveto(400.16976547,79.03931732)(400.07976555,78.97931738)(400.03976562,78.88931936)
\curveto(400.00976562,78.80931755)(399.99476564,78.71931764)(399.99476563,78.61931936)
\curveto(399.99476564,78.51931784)(399.96976567,78.43431792)(399.91976563,78.36431936)
\curveto(399.84976578,78.27431808)(399.70976592,78.22931813)(399.49976563,78.22931936)
\lineto(398.94476563,78.22931936)
\lineto(398.71976562,78.22931936)
\curveto(398.639767,78.23931812)(398.57476706,78.2593181)(398.52476562,78.28931936)
\curveto(398.44476719,78.34931801)(398.39976724,78.41931794)(398.38976562,78.49931936)
\curveto(398.37976725,78.51931784)(398.37476726,78.53931782)(398.37476563,78.55931936)
\curveto(398.37476726,78.58931777)(398.36976727,78.61431774)(398.35976563,78.63431936)
}
}
{
\newrgbcolor{curcolor}{0 0 0}
\pscustom[linestyle=none,fillstyle=solid,fillcolor=curcolor]
{
}
}
{
\newrgbcolor{curcolor}{0 0 0}
\pscustom[linestyle=none,fillstyle=solid,fillcolor=curcolor]
{
\newpath
\moveto(389.38976562,89.26463186)
\curveto(389.37977626,89.95462722)(389.49977613,90.55462662)(389.74976563,91.06463186)
\curveto(389.99977563,91.58462559)(390.3347753,91.9796252)(390.75476563,92.24963186)
\curveto(390.8347748,92.29962488)(390.92477471,92.34462483)(391.02476562,92.38463186)
\curveto(391.11477452,92.42462475)(391.20977442,92.46962471)(391.30976563,92.51963186)
\curveto(391.40977422,92.55962462)(391.50977412,92.58962459)(391.60976563,92.60963186)
\curveto(391.70977392,92.62962455)(391.81477382,92.64962453)(391.92476563,92.66963186)
\curveto(391.97477366,92.68962449)(392.01977362,92.69462448)(392.05976563,92.68463186)
\curveto(392.09977353,92.6746245)(392.14477349,92.6796245)(392.19476563,92.69963186)
\curveto(392.24477339,92.70962447)(392.32977331,92.71462446)(392.44976562,92.71463186)
\curveto(392.55977308,92.71462446)(392.64477299,92.70962447)(392.70476562,92.69963186)
\curveto(392.76477287,92.6796245)(392.82477281,92.66962451)(392.88476562,92.66963186)
\curveto(392.94477269,92.6796245)(393.00477263,92.6746245)(393.06476563,92.65463186)
\curveto(393.20477243,92.61462456)(393.33977229,92.5796246)(393.46976562,92.54963186)
\curveto(393.59977204,92.51962466)(393.72477191,92.4796247)(393.84476562,92.42963186)
\curveto(393.98477165,92.36962481)(394.10977152,92.29962488)(394.21976562,92.21963186)
\curveto(394.32977131,92.14962503)(394.43977119,92.0746251)(394.54976563,91.99463186)
\lineto(394.60976563,91.93463186)
\curveto(394.62977101,91.92462525)(394.64977098,91.90962527)(394.66976563,91.88963186)
\curveto(394.82977081,91.76962541)(394.97477066,91.63462554)(395.10476563,91.48463186)
\curveto(395.2347704,91.33462584)(395.35977028,91.174626)(395.47976563,91.00463186)
\curveto(395.69976994,90.69462648)(395.90476973,90.39962678)(396.09476562,90.11963186)
\curveto(396.2347694,89.88962729)(396.36976927,89.65962752)(396.49976563,89.42963186)
\curveto(396.62976901,89.20962797)(396.76476887,88.98962819)(396.90476562,88.76963186)
\curveto(397.07476856,88.51962866)(397.25476838,88.2796289)(397.44476563,88.04963186)
\curveto(397.634768,87.82962935)(397.85976778,87.63962954)(398.11976563,87.47963186)
\curveto(398.17976745,87.43962974)(398.2397674,87.40462977)(398.29976563,87.37463186)
\curveto(398.34976728,87.34462983)(398.41476722,87.31462986)(398.49476563,87.28463186)
\curveto(398.56476707,87.26462991)(398.62476701,87.25962992)(398.67476563,87.26963186)
\curveto(398.74476689,87.28962989)(398.79976684,87.32462985)(398.83976562,87.37463186)
\curveto(398.86976677,87.42462975)(398.88976674,87.48462969)(398.89976562,87.55463186)
\lineto(398.89976562,87.79463186)
\lineto(398.89976562,88.54463186)
\lineto(398.89976562,91.34963186)
\lineto(398.89976562,92.00963186)
\curveto(398.89976674,92.09962508)(398.90476673,92.18462499)(398.91476562,92.26463186)
\curveto(398.91476672,92.34462483)(398.9347667,92.40962477)(398.97476562,92.45963186)
\curveto(399.01476662,92.50962467)(399.08976654,92.54962463)(399.19976562,92.57963186)
\curveto(399.29976634,92.61962456)(399.39976624,92.62962455)(399.49976563,92.60963186)
\lineto(399.63476562,92.60963186)
\curveto(399.70476593,92.58962459)(399.76476587,92.56962461)(399.81476563,92.54963186)
\curveto(399.86476577,92.52962465)(399.90476573,92.49462468)(399.93476563,92.44463186)
\curveto(399.97476566,92.39462478)(399.99476564,92.32462485)(399.99476563,92.23463186)
\lineto(399.99476563,91.96463186)
\lineto(399.99476563,91.06463186)
\lineto(399.99476563,87.55463186)
\lineto(399.99476563,86.48963186)
\curveto(399.99476564,86.40963077)(399.99976564,86.31963086)(400.00976562,86.21963186)
\curveto(400.00976562,86.11963106)(399.99976564,86.03463114)(399.97976563,85.96463186)
\curveto(399.90976572,85.75463142)(399.72976591,85.68963149)(399.43976563,85.76963186)
\curveto(399.39976624,85.7796314)(399.36476627,85.7796314)(399.33476562,85.76963186)
\curveto(399.29476634,85.76963141)(399.24976638,85.7796314)(399.19976562,85.79963186)
\curveto(399.11976651,85.81963136)(399.0347666,85.83963134)(398.94476563,85.85963186)
\curveto(398.85476678,85.8796313)(398.76976687,85.90463127)(398.68976563,85.93463186)
\curveto(398.19976744,86.09463108)(397.78476785,86.29463088)(397.44476563,86.53463186)
\curveto(397.19476844,86.71463046)(396.96976867,86.91963026)(396.76976562,87.14963186)
\curveto(396.55976908,87.3796298)(396.36476927,87.61962956)(396.18476563,87.86963186)
\curveto(396.00476963,88.12962905)(395.8347698,88.39462878)(395.67476563,88.66463186)
\curveto(395.50477013,88.94462823)(395.32977031,89.21462796)(395.14976562,89.47463186)
\curveto(395.06977056,89.58462759)(394.99477064,89.68962749)(394.92476563,89.78963186)
\curveto(394.85477078,89.89962728)(394.77977085,90.00962717)(394.69976562,90.11963186)
\curveto(394.66977096,90.15962702)(394.63977099,90.19462698)(394.60976563,90.22463186)
\curveto(394.56977106,90.26462691)(394.53977109,90.30462687)(394.51976562,90.34463186)
\curveto(394.40977122,90.48462669)(394.28477135,90.60962657)(394.14476562,90.71963186)
\curveto(394.11477152,90.73962644)(394.08977155,90.76462641)(394.06976563,90.79463186)
\curveto(394.03977159,90.82462635)(394.00977162,90.84962633)(393.97976563,90.86963186)
\curveto(393.87977175,90.94962623)(393.77977185,91.01462616)(393.67976563,91.06463186)
\curveto(393.57977205,91.12462605)(393.46977216,91.179626)(393.34976562,91.22963186)
\curveto(393.27977235,91.25962592)(393.20477243,91.2796259)(393.12476563,91.28963186)
\lineto(392.88476562,91.34963186)
\lineto(392.79476563,91.34963186)
\curveto(392.76477287,91.35962582)(392.7347729,91.36462581)(392.70476562,91.36463186)
\curveto(392.634773,91.38462579)(392.53977309,91.38962579)(392.41976563,91.37963186)
\curveto(392.28977335,91.3796258)(392.18977345,91.36962581)(392.11976563,91.34963186)
\curveto(392.03977359,91.32962585)(391.96477367,91.30962587)(391.89476562,91.28963186)
\curveto(391.81477382,91.2796259)(391.7347739,91.25962592)(391.65476562,91.22963186)
\curveto(391.41477422,91.11962606)(391.21477442,90.96962621)(391.05476563,90.77963186)
\curveto(390.88477475,90.59962658)(390.74477489,90.3796268)(390.63476562,90.11963186)
\curveto(390.61477502,90.04962713)(390.59977503,89.9796272)(390.58976562,89.90963186)
\curveto(390.56977506,89.83962734)(390.54977509,89.76462741)(390.52976562,89.68463186)
\curveto(390.50977512,89.60462757)(390.49977513,89.49462768)(390.49976563,89.35463186)
\curveto(390.49977513,89.22462795)(390.50977512,89.11962806)(390.52976562,89.03963186)
\curveto(390.53977509,88.9796282)(390.54477509,88.92462825)(390.54476563,88.87463186)
\curveto(390.54477509,88.82462835)(390.55477508,88.7746284)(390.57476562,88.72463186)
\curveto(390.61477502,88.62462855)(390.65477498,88.52962865)(390.69476563,88.43963186)
\curveto(390.7347749,88.35962882)(390.77977486,88.2796289)(390.82976562,88.19963186)
\curveto(390.84977479,88.16962901)(390.87477476,88.13962904)(390.90476562,88.10963186)
\curveto(390.9347747,88.08962909)(390.95977468,88.06462911)(390.97976563,88.03463186)
\lineto(391.05476563,87.95963186)
\curveto(391.07477456,87.92962925)(391.09477454,87.90462927)(391.11476563,87.88463186)
\lineto(391.32476562,87.73463186)
\curveto(391.38477425,87.69462948)(391.44977419,87.64962953)(391.51976562,87.59963186)
\curveto(391.60977402,87.53962964)(391.71477392,87.48962969)(391.83476562,87.44963186)
\curveto(391.94477369,87.41962976)(392.05477358,87.38462979)(392.16476562,87.34463186)
\curveto(392.27477336,87.30462987)(392.41977322,87.2796299)(392.59976562,87.26963186)
\curveto(392.76977286,87.25962992)(392.89477274,87.22962995)(392.97476562,87.17963186)
\curveto(393.05477258,87.12963005)(393.09977253,87.05463012)(393.10976563,86.95463186)
\curveto(393.11977252,86.85463032)(393.12477251,86.74463043)(393.12476563,86.62463186)
\curveto(393.12477251,86.58463059)(393.12977251,86.54463063)(393.13976562,86.50463186)
\curveto(393.13977249,86.46463071)(393.1347725,86.42963075)(393.12476563,86.39963186)
\curveto(393.10477253,86.34963083)(393.09477254,86.29963088)(393.09476562,86.24963186)
\curveto(393.09477254,86.20963097)(393.08477255,86.16963101)(393.06476563,86.12963186)
\curveto(393.00477263,86.03963114)(392.86977276,85.99463118)(392.65976562,85.99463186)
\lineto(392.53976562,85.99463186)
\curveto(392.47977315,86.00463117)(392.41977322,86.00963117)(392.35976563,86.00963186)
\curveto(392.28977335,86.01963116)(392.22477341,86.02963115)(392.16476562,86.03963186)
\curveto(392.05477358,86.05963112)(391.95477368,86.0796311)(391.86476563,86.09963186)
\curveto(391.76477387,86.11963106)(391.66977396,86.14963103)(391.57976562,86.18963186)
\curveto(391.50977412,86.20963097)(391.44977419,86.22963095)(391.39976562,86.24963186)
\lineto(391.21976562,86.30963186)
\curveto(390.95977468,86.42963075)(390.71477492,86.58463059)(390.48476563,86.77463186)
\curveto(390.25477538,86.9746302)(390.06977556,87.18962999)(389.92976563,87.41963186)
\curveto(389.84977579,87.52962965)(389.78477585,87.64462953)(389.73476563,87.76463186)
\lineto(389.58476562,88.15463186)
\curveto(389.5347761,88.26462891)(389.50477613,88.3796288)(389.49476563,88.49963186)
\curveto(389.47477616,88.61962856)(389.44977619,88.74462843)(389.41976563,88.87463186)
\curveto(389.41977622,88.94462823)(389.41977622,89.00962817)(389.41976563,89.06963186)
\curveto(389.40977622,89.12962805)(389.39977623,89.19462798)(389.38976562,89.26463186)
}
}
{
\newrgbcolor{curcolor}{0 0 0}
\pscustom[linestyle=none,fillstyle=solid,fillcolor=curcolor]
{
\newpath
\moveto(394.90976562,101.36424123)
\lineto(395.16476562,101.36424123)
\curveto(395.24477039,101.37423353)(395.31977032,101.36923353)(395.38976562,101.34924123)
\lineto(395.62976563,101.34924123)
\lineto(395.79476563,101.34924123)
\curveto(395.89476974,101.32923357)(395.99976964,101.31923358)(396.10976563,101.31924123)
\curveto(396.20976942,101.31923358)(396.30976932,101.30923359)(396.40976562,101.28924123)
\lineto(396.55976563,101.28924123)
\curveto(396.69976894,101.25923364)(396.83976879,101.23923366)(396.97976563,101.22924123)
\curveto(397.10976852,101.21923368)(397.23976839,101.19423371)(397.36976563,101.15424123)
\curveto(397.44976818,101.13423377)(397.5347681,101.11423379)(397.62476563,101.09424123)
\lineto(397.86476563,101.03424123)
\lineto(398.16476562,100.91424123)
\curveto(398.25476738,100.88423402)(398.34476729,100.84923405)(398.43476563,100.80924123)
\curveto(398.65476698,100.70923419)(398.86976677,100.57423433)(399.07976562,100.40424123)
\curveto(399.28976634,100.24423466)(399.45976618,100.06923483)(399.58976562,99.87924123)
\curveto(399.62976601,99.82923507)(399.66976597,99.76923513)(399.70976562,99.69924123)
\curveto(399.7397659,99.63923526)(399.77476586,99.57923532)(399.81476563,99.51924123)
\curveto(399.86476577,99.43923546)(399.90476573,99.34423556)(399.93476563,99.23424123)
\curveto(399.96476567,99.12423578)(399.99476564,99.01923588)(400.02476562,98.91924123)
\curveto(400.06476557,98.80923609)(400.08976554,98.6992362)(400.09976562,98.58924123)
\curveto(400.10976553,98.47923642)(400.12476551,98.36423654)(400.14476562,98.24424123)
\curveto(400.15476548,98.2042367)(400.15476548,98.15923674)(400.14476562,98.10924123)
\curveto(400.14476549,98.06923683)(400.14976548,98.02923687)(400.15976562,97.98924123)
\curveto(400.16976547,97.94923695)(400.17476546,97.89423701)(400.17476563,97.82424123)
\curveto(400.17476546,97.75423715)(400.16976547,97.7042372)(400.15976562,97.67424123)
\curveto(400.1397655,97.62423728)(400.1347655,97.57923732)(400.14476562,97.53924123)
\curveto(400.15476548,97.4992374)(400.15476548,97.46423744)(400.14476562,97.43424123)
\lineto(400.14476562,97.34424123)
\curveto(400.12476551,97.28423762)(400.10976553,97.21923768)(400.09976562,97.14924123)
\curveto(400.09976554,97.08923781)(400.09476554,97.02423788)(400.08476562,96.95424123)
\curveto(400.0347656,96.78423812)(399.98476565,96.62423828)(399.93476563,96.47424123)
\curveto(399.88476575,96.32423858)(399.81976581,96.17923872)(399.73976563,96.03924123)
\curveto(399.69976594,95.98923891)(399.66976597,95.93423897)(399.64976562,95.87424123)
\curveto(399.61976601,95.82423908)(399.58476605,95.77423913)(399.54476563,95.72424123)
\curveto(399.36476627,95.48423942)(399.14476649,95.28423962)(398.88476562,95.12424123)
\curveto(398.62476701,94.96423994)(398.3397673,94.82424008)(398.02976562,94.70424123)
\curveto(397.88976774,94.64424026)(397.74976788,94.5992403)(397.60976563,94.56924123)
\curveto(397.45976818,94.53924036)(397.30476833,94.5042404)(397.14476562,94.46424123)
\curveto(397.0347686,94.44424046)(396.92476871,94.42924047)(396.81476563,94.41924123)
\curveto(396.70476893,94.40924049)(396.59476904,94.39424051)(396.48476563,94.37424123)
\curveto(396.44476919,94.36424054)(396.40476923,94.35924054)(396.36476563,94.35924123)
\curveto(396.32476931,94.36924053)(396.28476935,94.36924053)(396.24476563,94.35924123)
\curveto(396.19476944,94.34924055)(396.14476949,94.34424056)(396.09476562,94.34424123)
\lineto(395.92976563,94.34424123)
\curveto(395.87976975,94.32424058)(395.82976981,94.31924058)(395.77976562,94.32924123)
\curveto(395.71976991,94.33924056)(395.66476997,94.33924056)(395.61476563,94.32924123)
\curveto(395.57477006,94.31924058)(395.52977011,94.31924058)(395.47976563,94.32924123)
\curveto(395.42977021,94.33924056)(395.37977025,94.33424057)(395.32976562,94.31424123)
\curveto(395.25977038,94.29424061)(395.18477045,94.28924061)(395.10476563,94.29924123)
\curveto(395.01477062,94.30924059)(394.92977071,94.31424059)(394.84976562,94.31424123)
\curveto(394.75977088,94.31424059)(394.65977098,94.30924059)(394.54976563,94.29924123)
\curveto(394.42977121,94.28924061)(394.32977131,94.29424061)(394.24976563,94.31424123)
\lineto(393.96476562,94.31424123)
\lineto(393.33476562,94.35924123)
\curveto(393.2347724,94.36924053)(393.13977249,94.37924052)(393.04976563,94.38924123)
\lineto(392.74976563,94.41924123)
\curveto(392.69977293,94.43924046)(392.64977299,94.44424046)(392.59976562,94.43424123)
\curveto(392.53977309,94.43424047)(392.48477315,94.44424046)(392.43476563,94.46424123)
\curveto(392.26477337,94.51424039)(392.09977353,94.55424035)(391.93976563,94.58424123)
\curveto(391.76977386,94.61424029)(391.60977402,94.66424024)(391.45976562,94.73424123)
\curveto(390.99977463,94.92423998)(390.62477501,95.14423976)(390.33476562,95.39424123)
\curveto(390.04477559,95.65423925)(389.79977583,96.01423889)(389.59976562,96.47424123)
\curveto(389.54977609,96.6042383)(389.51477612,96.73423817)(389.49476563,96.86424123)
\curveto(389.47477616,97.0042379)(389.44977619,97.14423776)(389.41976563,97.28424123)
\curveto(389.40977622,97.35423755)(389.40477623,97.41923748)(389.40476562,97.47924123)
\curveto(389.40477623,97.53923736)(389.39977623,97.6042373)(389.38976562,97.67424123)
\curveto(389.36977626,98.5042364)(389.51977612,99.17423573)(389.83976562,99.68424123)
\curveto(390.14977549,100.19423471)(390.58977505,100.57423433)(391.15976562,100.82424123)
\curveto(391.27977435,100.87423403)(391.40477423,100.91923398)(391.53476562,100.95924123)
\curveto(391.66477397,100.9992339)(391.79977383,101.04423386)(391.93976563,101.09424123)
\curveto(392.01977362,101.11423379)(392.10477353,101.12923377)(392.19476563,101.13924123)
\lineto(392.43476563,101.19924123)
\curveto(392.54477309,101.22923367)(392.65477298,101.24423366)(392.76476562,101.24424123)
\curveto(392.87477276,101.25423365)(392.98477265,101.26923363)(393.09476562,101.28924123)
\curveto(393.14477249,101.30923359)(393.18977245,101.31423359)(393.22976563,101.30424123)
\curveto(393.26977236,101.3042336)(393.30977232,101.30923359)(393.34976562,101.31924123)
\curveto(393.39977224,101.32923357)(393.45477218,101.32923357)(393.51476562,101.31924123)
\curveto(393.56477207,101.31923358)(393.61477202,101.32423358)(393.66476562,101.33424123)
\lineto(393.79976563,101.33424123)
\curveto(393.85977178,101.35423355)(393.92977171,101.35423355)(394.00976562,101.33424123)
\curveto(394.07977155,101.32423358)(394.14477149,101.32923357)(394.20476562,101.34924123)
\curveto(394.2347714,101.35923354)(394.27477136,101.36423354)(394.32476562,101.36424123)
\lineto(394.44476563,101.36424123)
\lineto(394.90976562,101.36424123)
\moveto(397.23476563,99.81924123)
\curveto(396.91476872,99.91923498)(396.54976908,99.97923492)(396.13976562,99.99924123)
\curveto(395.72976991,100.01923488)(395.31977032,100.02923487)(394.90976562,100.02924123)
\curveto(394.47977115,100.02923487)(394.05977158,100.01923488)(393.64976562,99.99924123)
\curveto(393.23977239,99.97923492)(392.85477278,99.93423497)(392.49476563,99.86424123)
\curveto(392.1347735,99.79423511)(391.81477382,99.68423522)(391.53476562,99.53424123)
\curveto(391.24477439,99.39423551)(391.00977462,99.1992357)(390.82976562,98.94924123)
\curveto(390.71977492,98.78923611)(390.63977499,98.60923629)(390.58976562,98.40924123)
\curveto(390.5297751,98.20923669)(390.49977513,97.96423694)(390.49976563,97.67424123)
\curveto(390.51977512,97.65423725)(390.5297751,97.61923728)(390.52976562,97.56924123)
\curveto(390.51977512,97.51923738)(390.51977512,97.47923742)(390.52976562,97.44924123)
\curveto(390.54977509,97.36923753)(390.56977506,97.29423761)(390.58976562,97.22424123)
\curveto(390.59977503,97.16423774)(390.61977502,97.0992378)(390.64976562,97.02924123)
\curveto(390.76977486,96.75923814)(390.93977469,96.53923836)(391.15976562,96.36924123)
\curveto(391.36977426,96.20923869)(391.61477402,96.07423883)(391.89476562,95.96424123)
\curveto(392.00477363,95.91423899)(392.12477351,95.87423903)(392.25476563,95.84424123)
\curveto(392.37477326,95.82423908)(392.49977313,95.7992391)(392.62976563,95.76924123)
\curveto(392.67977295,95.74923915)(392.7347729,95.73923916)(392.79476563,95.73924123)
\curveto(392.84477279,95.73923916)(392.89477274,95.73423917)(392.94476563,95.72424123)
\curveto(393.0347726,95.71423919)(393.12977251,95.7042392)(393.22976563,95.69424123)
\curveto(393.31977232,95.68423922)(393.41477222,95.67423923)(393.51476562,95.66424123)
\curveto(393.59477204,95.66423924)(393.67977195,95.65923924)(393.76976562,95.64924123)
\lineto(394.00976562,95.64924123)
\lineto(394.18976563,95.64924123)
\curveto(394.21977142,95.63923926)(394.25477138,95.63423927)(394.29476563,95.63424123)
\lineto(394.42976563,95.63424123)
\lineto(394.87976563,95.63424123)
\curveto(394.95977068,95.63423927)(395.04477059,95.62923927)(395.13476562,95.61924123)
\curveto(395.21477042,95.61923928)(395.28977035,95.62923927)(395.35976563,95.64924123)
\lineto(395.62976563,95.64924123)
\curveto(395.64976998,95.64923925)(395.67976995,95.64423926)(395.71976562,95.63424123)
\curveto(395.74976988,95.63423927)(395.77476986,95.63923926)(395.79476563,95.64924123)
\curveto(395.89476974,95.65923924)(395.99476964,95.66423924)(396.09476562,95.66424123)
\curveto(396.18476945,95.67423923)(396.28476935,95.68423922)(396.39476562,95.69424123)
\curveto(396.51476912,95.72423918)(396.63976899,95.73923916)(396.76976562,95.73924123)
\curveto(396.88976875,95.74923915)(397.00476863,95.77423913)(397.11476563,95.81424123)
\curveto(397.41476822,95.89423901)(397.67976795,95.97923892)(397.90976562,96.06924123)
\curveto(398.1397675,96.16923873)(398.35476728,96.31423859)(398.55476563,96.50424123)
\curveto(398.75476688,96.71423819)(398.90476673,96.97923792)(399.00476563,97.29924123)
\curveto(399.02476661,97.33923756)(399.0347666,97.37423753)(399.03476562,97.40424123)
\curveto(399.02476661,97.44423746)(399.02976661,97.48923741)(399.04976563,97.53924123)
\curveto(399.05976658,97.57923732)(399.06976657,97.64923725)(399.07976562,97.74924123)
\curveto(399.08976654,97.85923704)(399.08476655,97.94423696)(399.06476563,98.00424123)
\curveto(399.04476659,98.07423683)(399.0347666,98.14423676)(399.03476562,98.21424123)
\curveto(399.02476661,98.28423662)(399.00976662,98.34923655)(398.98976563,98.40924123)
\curveto(398.92976671,98.60923629)(398.84476679,98.78923611)(398.73476563,98.94924123)
\curveto(398.71476692,98.97923592)(398.69476694,99.0042359)(398.67476563,99.02424123)
\lineto(398.61476563,99.08424123)
\curveto(398.59476704,99.12423578)(398.55476708,99.17423573)(398.49476563,99.23424123)
\curveto(398.35476728,99.33423557)(398.22476741,99.41923548)(398.10476563,99.48924123)
\curveto(397.98476765,99.55923534)(397.8397678,99.62923527)(397.66976563,99.69924123)
\curveto(397.59976804,99.72923517)(397.52976811,99.74923515)(397.45976562,99.75924123)
\curveto(397.38976825,99.77923512)(397.31476832,99.7992351)(397.23476563,99.81924123)
}
}
{
\newrgbcolor{curcolor}{0 0 0}
\pscustom[linestyle=none,fillstyle=solid,fillcolor=curcolor]
{
\newpath
\moveto(389.38976562,106.77385061)
\curveto(389.38977625,106.87384575)(389.39977623,106.96884566)(389.41976563,107.05885061)
\curveto(389.4297762,107.14884548)(389.45977618,107.21384541)(389.50976562,107.25385061)
\curveto(389.58977605,107.31384531)(389.69477594,107.34384528)(389.82476562,107.34385061)
\lineto(390.21476562,107.34385061)
\lineto(391.71476562,107.34385061)
\lineto(398.10476563,107.34385061)
\lineto(399.27476562,107.34385061)
\lineto(399.58976562,107.34385061)
\curveto(399.68976594,107.35384527)(399.76976587,107.33884529)(399.82976562,107.29885061)
\curveto(399.90976572,107.24884538)(399.95976568,107.17384545)(399.97976563,107.07385061)
\curveto(399.98976564,106.98384564)(399.99476564,106.87384575)(399.99476563,106.74385061)
\lineto(399.99476563,106.51885061)
\curveto(399.97476566,106.43884619)(399.95976568,106.36884626)(399.94976562,106.30885061)
\curveto(399.92976571,106.24884638)(399.88976574,106.19884643)(399.82976562,106.15885061)
\curveto(399.76976587,106.11884651)(399.69476594,106.09884653)(399.60476563,106.09885061)
\lineto(399.30476563,106.09885061)
\lineto(398.20976562,106.09885061)
\lineto(392.86976563,106.09885061)
\curveto(392.77977285,106.07884655)(392.70477293,106.06384656)(392.64476562,106.05385061)
\curveto(392.57477306,106.05384657)(392.51477312,106.0238466)(392.46476562,105.96385061)
\curveto(392.41477322,105.89384673)(392.38977325,105.80384682)(392.38976562,105.69385061)
\curveto(392.37977325,105.59384703)(392.37477326,105.48384714)(392.37476563,105.36385061)
\lineto(392.37476563,104.22385061)
\lineto(392.37476563,103.72885061)
\curveto(392.36477327,103.56884906)(392.30477333,103.45884917)(392.19476563,103.39885061)
\curveto(392.16477347,103.37884925)(392.1347735,103.36884926)(392.10476563,103.36885061)
\curveto(392.06477357,103.36884926)(392.01977362,103.36384926)(391.96976562,103.35385061)
\curveto(391.84977379,103.33384929)(391.73977389,103.33884929)(391.63976562,103.36885061)
\curveto(391.53977409,103.40884922)(391.46977416,103.46384916)(391.42976563,103.53385061)
\curveto(391.37977426,103.61384901)(391.35477428,103.73384889)(391.35476563,103.89385061)
\curveto(391.35477428,104.05384857)(391.33977429,104.18884844)(391.30976563,104.29885061)
\curveto(391.29977433,104.34884828)(391.29477434,104.40384822)(391.29476563,104.46385061)
\curveto(391.28477435,104.5238481)(391.26977436,104.58384804)(391.24976563,104.64385061)
\curveto(391.19977443,104.79384783)(391.14977449,104.93884769)(391.09976562,105.07885061)
\curveto(391.03977459,105.21884741)(390.96977466,105.35384727)(390.88976562,105.48385061)
\curveto(390.79977483,105.623847)(390.69477494,105.74384688)(390.57476562,105.84385061)
\curveto(390.45477518,105.94384668)(390.32477531,106.03884659)(390.18476563,106.12885061)
\curveto(390.08477555,106.18884644)(389.97477566,106.23384639)(389.85476563,106.26385061)
\curveto(389.7347759,106.30384632)(389.629776,106.35384627)(389.53976562,106.41385061)
\curveto(389.47977616,106.46384616)(389.43977619,106.53384609)(389.41976563,106.62385061)
\curveto(389.40977622,106.64384598)(389.40477623,106.66884596)(389.40476562,106.69885061)
\curveto(389.40477623,106.7288459)(389.39977623,106.75384587)(389.38976562,106.77385061)
}
}
{
\newrgbcolor{curcolor}{0 0 0}
\pscustom[linestyle=none,fillstyle=solid,fillcolor=curcolor]
{
\newpath
\moveto(389.38976562,115.12345998)
\curveto(389.38977625,115.22345513)(389.39977623,115.31845503)(389.41976563,115.40845998)
\curveto(389.4297762,115.49845485)(389.45977618,115.56345479)(389.50976562,115.60345998)
\curveto(389.58977605,115.66345469)(389.69477594,115.69345466)(389.82476562,115.69345998)
\lineto(390.21476562,115.69345998)
\lineto(391.71476562,115.69345998)
\lineto(398.10476563,115.69345998)
\lineto(399.27476562,115.69345998)
\lineto(399.58976562,115.69345998)
\curveto(399.68976594,115.70345465)(399.76976587,115.68845466)(399.82976562,115.64845998)
\curveto(399.90976572,115.59845475)(399.95976568,115.52345483)(399.97976563,115.42345998)
\curveto(399.98976564,115.33345502)(399.99476564,115.22345513)(399.99476563,115.09345998)
\lineto(399.99476563,114.86845998)
\curveto(399.97476566,114.78845556)(399.95976568,114.71845563)(399.94976562,114.65845998)
\curveto(399.92976571,114.59845575)(399.88976574,114.5484558)(399.82976562,114.50845998)
\curveto(399.76976587,114.46845588)(399.69476594,114.4484559)(399.60476563,114.44845998)
\lineto(399.30476563,114.44845998)
\lineto(398.20976562,114.44845998)
\lineto(392.86976563,114.44845998)
\curveto(392.77977285,114.42845592)(392.70477293,114.41345594)(392.64476562,114.40345998)
\curveto(392.57477306,114.40345595)(392.51477312,114.37345598)(392.46476562,114.31345998)
\curveto(392.41477322,114.24345611)(392.38977325,114.1534562)(392.38976562,114.04345998)
\curveto(392.37977325,113.94345641)(392.37477326,113.83345652)(392.37476563,113.71345998)
\lineto(392.37476563,112.57345998)
\lineto(392.37476563,112.07845998)
\curveto(392.36477327,111.91845843)(392.30477333,111.80845854)(392.19476563,111.74845998)
\curveto(392.16477347,111.72845862)(392.1347735,111.71845863)(392.10476563,111.71845998)
\curveto(392.06477357,111.71845863)(392.01977362,111.71345864)(391.96976562,111.70345998)
\curveto(391.84977379,111.68345867)(391.73977389,111.68845866)(391.63976562,111.71845998)
\curveto(391.53977409,111.75845859)(391.46977416,111.81345854)(391.42976563,111.88345998)
\curveto(391.37977426,111.96345839)(391.35477428,112.08345827)(391.35476563,112.24345998)
\curveto(391.35477428,112.40345795)(391.33977429,112.53845781)(391.30976563,112.64845998)
\curveto(391.29977433,112.69845765)(391.29477434,112.7534576)(391.29476563,112.81345998)
\curveto(391.28477435,112.87345748)(391.26977436,112.93345742)(391.24976563,112.99345998)
\curveto(391.19977443,113.14345721)(391.14977449,113.28845706)(391.09976562,113.42845998)
\curveto(391.03977459,113.56845678)(390.96977466,113.70345665)(390.88976562,113.83345998)
\curveto(390.79977483,113.97345638)(390.69477494,114.09345626)(390.57476562,114.19345998)
\curveto(390.45477518,114.29345606)(390.32477531,114.38845596)(390.18476563,114.47845998)
\curveto(390.08477555,114.53845581)(389.97477566,114.58345577)(389.85476563,114.61345998)
\curveto(389.7347759,114.6534557)(389.629776,114.70345565)(389.53976562,114.76345998)
\curveto(389.47977616,114.81345554)(389.43977619,114.88345547)(389.41976563,114.97345998)
\curveto(389.40977622,114.99345536)(389.40477623,115.01845533)(389.40476562,115.04845998)
\curveto(389.40477623,115.07845527)(389.39977623,115.10345525)(389.38976562,115.12345998)
}
}
{
\newrgbcolor{curcolor}{0 0 0}
\pscustom[linestyle=none,fillstyle=solid,fillcolor=curcolor]
{
\newpath
\moveto(420.12611206,42.02236623)
\curveto(420.17611281,42.04235669)(420.23611275,42.06735666)(420.30611206,42.09736623)
\curveto(420.37611261,42.1273566)(420.45111253,42.14735658)(420.53111206,42.15736623)
\curveto(420.60111238,42.17735655)(420.67111231,42.17735655)(420.74111206,42.15736623)
\curveto(420.80111218,42.14735658)(420.84611214,42.10735662)(420.87611206,42.03736623)
\curveto(420.89611209,41.98735674)(420.90611208,41.9273568)(420.90611206,41.85736623)
\lineto(420.90611206,41.64736623)
\lineto(420.90611206,41.19736623)
\curveto(420.90611208,41.04735768)(420.8811121,40.9273578)(420.83111206,40.83736623)
\curveto(420.77111221,40.73735799)(420.66611232,40.66235807)(420.51611206,40.61236623)
\curveto(420.36611262,40.57235816)(420.23111275,40.5273582)(420.11111206,40.47736623)
\curveto(419.85111313,40.36735836)(419.5811134,40.26735846)(419.30111206,40.17736623)
\curveto(419.02111396,40.08735864)(418.74611424,39.98735874)(418.47611206,39.87736623)
\curveto(418.3861146,39.84735888)(418.30111468,39.81735891)(418.22111206,39.78736623)
\curveto(418.14111484,39.76735896)(418.06611492,39.73735899)(417.99611206,39.69736623)
\curveto(417.92611506,39.66735906)(417.86611512,39.62235911)(417.81611206,39.56236623)
\curveto(417.76611522,39.50235923)(417.72611526,39.42235931)(417.69611206,39.32236623)
\curveto(417.67611531,39.27235946)(417.67111531,39.21235952)(417.68111206,39.14236623)
\lineto(417.68111206,38.94736623)
\lineto(417.68111206,36.11236623)
\lineto(417.68111206,35.81236623)
\curveto(417.67111531,35.70236303)(417.67111531,35.59736313)(417.68111206,35.49736623)
\curveto(417.69111529,35.39736333)(417.70611528,35.30236343)(417.72611206,35.21236623)
\curveto(417.74611524,35.1323636)(417.7861152,35.07236366)(417.84611206,35.03236623)
\curveto(417.94611504,34.95236378)(418.06111492,34.89236384)(418.19111206,34.85236623)
\curveto(418.31111467,34.82236391)(418.43611455,34.78236395)(418.56611206,34.73236623)
\curveto(418.79611419,34.6323641)(419.03611395,34.53736419)(419.28611206,34.44736623)
\curveto(419.53611345,34.36736436)(419.77611321,34.27736445)(420.00611206,34.17736623)
\curveto(420.06611292,34.15736457)(420.13611285,34.1323646)(420.21611206,34.10236623)
\curveto(420.2861127,34.08236465)(420.36111262,34.05736467)(420.44111206,34.02736623)
\curveto(420.52111246,33.99736473)(420.59611239,33.96236477)(420.66611206,33.92236623)
\curveto(420.72611226,33.89236484)(420.77111221,33.85736487)(420.80111206,33.81736623)
\curveto(420.86111212,33.73736499)(420.89611209,33.6273651)(420.90611206,33.48736623)
\lineto(420.90611206,33.06736623)
\lineto(420.90611206,32.82736623)
\curveto(420.89611209,32.75736597)(420.87111211,32.69736603)(420.83111206,32.64736623)
\curveto(420.80111218,32.59736613)(420.75611223,32.56736616)(420.69611206,32.55736623)
\curveto(420.63611235,32.55736617)(420.57611241,32.56236617)(420.51611206,32.57236623)
\curveto(420.44611254,32.59236614)(420.3811126,32.61236612)(420.32111206,32.63236623)
\curveto(420.25111273,32.66236607)(420.20111278,32.68736604)(420.17111206,32.70736623)
\curveto(419.85111313,32.84736588)(419.53611345,32.97236576)(419.22611206,33.08236623)
\curveto(418.90611408,33.19236554)(418.5861144,33.31236542)(418.26611206,33.44236623)
\curveto(418.04611494,33.5323652)(417.83111515,33.61736511)(417.62111206,33.69736623)
\curveto(417.40111558,33.77736495)(417.1811158,33.86236487)(416.96111206,33.95236623)
\curveto(416.24111674,34.25236448)(415.51611747,34.53736419)(414.78611206,34.80736623)
\curveto(414.04611894,35.07736365)(413.31111967,35.36236337)(412.58111206,35.66236623)
\curveto(412.32112066,35.77236296)(412.05612093,35.87236286)(411.78611206,35.96236623)
\curveto(411.51612147,36.06236267)(411.25112173,36.16736256)(410.99111206,36.27736623)
\curveto(410.8811221,36.3273624)(410.76112222,36.37236236)(410.63111206,36.41236623)
\curveto(410.49112249,36.46236227)(410.39112259,36.5323622)(410.33111206,36.62236623)
\curveto(410.29112269,36.66236207)(410.26112272,36.727362)(410.24111206,36.81736623)
\curveto(410.23112275,36.83736189)(410.23112275,36.85736187)(410.24111206,36.87736623)
\curveto(410.24112274,36.90736182)(410.23612275,36.9323618)(410.22611206,36.95236623)
\curveto(410.22612276,37.1323616)(410.22612276,37.34236139)(410.22611206,37.58236623)
\curveto(410.21612277,37.82236091)(410.25112273,37.99736073)(410.33111206,38.10736623)
\curveto(410.39112259,38.18736054)(410.49112249,38.24736048)(410.63111206,38.28736623)
\curveto(410.76112222,38.33736039)(410.8811221,38.38736034)(410.99111206,38.43736623)
\curveto(411.22112176,38.53736019)(411.45112153,38.6273601)(411.68111206,38.70736623)
\curveto(411.91112107,38.78735994)(412.14112084,38.87735985)(412.37111206,38.97736623)
\curveto(412.57112041,39.05735967)(412.77612021,39.1323596)(412.98611206,39.20236623)
\curveto(413.19611979,39.28235945)(413.40111958,39.36735936)(413.60111206,39.45736623)
\curveto(414.33111865,39.75735897)(415.07111791,40.04235869)(415.82111206,40.31236623)
\curveto(416.56111642,40.59235814)(417.29611569,40.88735784)(418.02611206,41.19736623)
\curveto(418.11611487,41.23735749)(418.20111478,41.26735746)(418.28111206,41.28736623)
\curveto(418.36111462,41.31735741)(418.44611454,41.34735738)(418.53611206,41.37736623)
\curveto(418.79611419,41.48735724)(419.06111392,41.59235714)(419.33111206,41.69236623)
\curveto(419.60111338,41.80235693)(419.86611312,41.91235682)(420.12611206,42.02236623)
\moveto(416.48111206,38.81236623)
\curveto(416.45111653,38.90235983)(416.40111658,38.95735977)(416.33111206,38.97736623)
\curveto(416.26111672,39.00735972)(416.1861168,39.01235972)(416.10611206,38.99236623)
\curveto(416.01611697,38.98235975)(415.93111705,38.95735977)(415.85111206,38.91736623)
\curveto(415.76111722,38.88735984)(415.6861173,38.85735987)(415.62611206,38.82736623)
\curveto(415.5861174,38.80735992)(415.55111743,38.79735993)(415.52111206,38.79736623)
\curveto(415.49111749,38.79735993)(415.45611753,38.78735994)(415.41611206,38.76736623)
\lineto(415.17611206,38.67736623)
\curveto(415.0861179,38.65736007)(414.99611799,38.6273601)(414.90611206,38.58736623)
\curveto(414.54611844,38.43736029)(414.1811188,38.30236043)(413.81111206,38.18236623)
\curveto(413.43111955,38.07236066)(413.06111992,37.94236079)(412.70111206,37.79236623)
\curveto(412.59112039,37.74236099)(412.4811205,37.69736103)(412.37111206,37.65736623)
\curveto(412.26112072,37.6273611)(412.15612083,37.58736114)(412.05611206,37.53736623)
\curveto(412.00612098,37.51736121)(411.96112102,37.49236124)(411.92111206,37.46236623)
\curveto(411.87112111,37.44236129)(411.84612114,37.39236134)(411.84611206,37.31236623)
\curveto(411.86612112,37.29236144)(411.8811211,37.27236146)(411.89111206,37.25236623)
\curveto(411.90112108,37.2323615)(411.91612107,37.21236152)(411.93611206,37.19236623)
\curveto(411.986121,37.15236158)(412.04112094,37.12236161)(412.10111206,37.10236623)
\curveto(412.15112083,37.08236165)(412.20612078,37.06236167)(412.26611206,37.04236623)
\curveto(412.37612061,36.99236174)(412.4861205,36.95236178)(412.59611206,36.92236623)
\curveto(412.70612028,36.89236184)(412.81612017,36.85236188)(412.92611206,36.80236623)
\curveto(413.31611967,36.6323621)(413.71111927,36.48236225)(414.11111206,36.35236623)
\curveto(414.51111847,36.2323625)(414.90111808,36.09236264)(415.28111206,35.93236623)
\lineto(415.43111206,35.87236623)
\curveto(415.4811175,35.86236287)(415.53111745,35.84736288)(415.58111206,35.82736623)
\lineto(415.82111206,35.73736623)
\curveto(415.90111708,35.70736302)(415.981117,35.68236305)(416.06111206,35.66236623)
\curveto(416.11111687,35.64236309)(416.16611682,35.6323631)(416.22611206,35.63236623)
\curveto(416.2861167,35.64236309)(416.33611665,35.65736307)(416.37611206,35.67736623)
\curveto(416.45611653,35.727363)(416.50111648,35.8323629)(416.51111206,35.99236623)
\lineto(416.51111206,36.44236623)
\lineto(416.51111206,38.04736623)
\curveto(416.51111647,38.15736057)(416.51611647,38.29236044)(416.52611206,38.45236623)
\curveto(416.52611646,38.61236012)(416.51111647,38.73236)(416.48111206,38.81236623)
}
}
{
\newrgbcolor{curcolor}{0 0 0}
\pscustom[linestyle=none,fillstyle=solid,fillcolor=curcolor]
{
\newpath
\moveto(416.87111206,50.56392873)
\curveto(416.92111606,50.57392038)(416.99111599,50.57892038)(417.08111206,50.57892873)
\curveto(417.16111582,50.57892038)(417.22611576,50.57392038)(417.27611206,50.56392873)
\curveto(417.31611567,50.56392039)(417.35611563,50.5589204)(417.39611206,50.54892873)
\lineto(417.51611206,50.54892873)
\curveto(417.59611539,50.52892043)(417.67611531,50.51892044)(417.75611206,50.51892873)
\curveto(417.83611515,50.51892044)(417.91611507,50.50892045)(417.99611206,50.48892873)
\curveto(418.03611495,50.47892048)(418.07611491,50.47392048)(418.11611206,50.47392873)
\curveto(418.14611484,50.47392048)(418.1811148,50.46892049)(418.22111206,50.45892873)
\curveto(418.33111465,50.42892053)(418.43611455,50.39892056)(418.53611206,50.36892873)
\curveto(418.63611435,50.34892061)(418.73611425,50.31892064)(418.83611206,50.27892873)
\curveto(419.1861138,50.13892082)(419.50111348,49.96892099)(419.78111206,49.76892873)
\curveto(420.06111292,49.56892139)(420.30111268,49.31892164)(420.50111206,49.01892873)
\curveto(420.60111238,48.86892209)(420.6861123,48.72392223)(420.75611206,48.58392873)
\curveto(420.80611218,48.47392248)(420.84611214,48.36392259)(420.87611206,48.25392873)
\curveto(420.90611208,48.1539228)(420.93611205,48.04892291)(420.96611206,47.93892873)
\curveto(420.986112,47.86892309)(420.99611199,47.80392315)(420.99611206,47.74392873)
\curveto(421.00611198,47.68392327)(421.02111196,47.62392333)(421.04111206,47.56392873)
\lineto(421.04111206,47.41392873)
\curveto(421.06111192,47.36392359)(421.07111191,47.28892367)(421.07111206,47.18892873)
\curveto(421.0811119,47.08892387)(421.07611191,47.00892395)(421.05611206,46.94892873)
\lineto(421.05611206,46.79892873)
\curveto(421.04611194,46.7589242)(421.04111194,46.71392424)(421.04111206,46.66392873)
\curveto(421.04111194,46.62392433)(421.03611195,46.57892438)(421.02611206,46.52892873)
\curveto(420.986112,46.37892458)(420.95111203,46.22892473)(420.92111206,46.07892873)
\curveto(420.89111209,45.93892502)(420.84611214,45.79892516)(420.78611206,45.65892873)
\curveto(420.70611228,45.4589255)(420.60611238,45.27892568)(420.48611206,45.11892873)
\lineto(420.33611206,44.93892873)
\curveto(420.27611271,44.87892608)(420.23611275,44.80892615)(420.21611206,44.72892873)
\curveto(420.20611278,44.66892629)(420.22111276,44.61892634)(420.26111206,44.57892873)
\curveto(420.29111269,44.54892641)(420.33611265,44.52392643)(420.39611206,44.50392873)
\curveto(420.45611253,44.49392646)(420.52111246,44.48392647)(420.59111206,44.47392873)
\curveto(420.65111233,44.47392648)(420.69611229,44.46392649)(420.72611206,44.44392873)
\curveto(420.77611221,44.40392655)(420.82111216,44.3589266)(420.86111206,44.30892873)
\curveto(420.8811121,44.2589267)(420.89611209,44.18892677)(420.90611206,44.09892873)
\lineto(420.90611206,43.82892873)
\curveto(420.90611208,43.73892722)(420.90111208,43.6539273)(420.89111206,43.57392873)
\curveto(420.87111211,43.49392746)(420.85111213,43.43392752)(420.83111206,43.39392873)
\curveto(420.81111217,43.37392758)(420.7861122,43.3539276)(420.75611206,43.33392873)
\lineto(420.66611206,43.27392873)
\curveto(420.5861124,43.24392771)(420.46611252,43.22892773)(420.30611206,43.22892873)
\curveto(420.14611284,43.23892772)(420.01111297,43.24392771)(419.90111206,43.24392873)
\lineto(411.09611206,43.24392873)
\curveto(410.97612201,43.24392771)(410.85112213,43.23892772)(410.72111206,43.22892873)
\curveto(410.5811224,43.22892773)(410.47112251,43.2539277)(410.39111206,43.30392873)
\curveto(410.33112265,43.34392761)(410.2811227,43.40892755)(410.24111206,43.49892873)
\curveto(410.24112274,43.51892744)(410.24112274,43.54392741)(410.24111206,43.57392873)
\curveto(410.23112275,43.60392735)(410.22612276,43.62892733)(410.22611206,43.64892873)
\curveto(410.21612277,43.78892717)(410.21612277,43.93392702)(410.22611206,44.08392873)
\curveto(410.22612276,44.24392671)(410.26612272,44.3539266)(410.34611206,44.41392873)
\curveto(410.42612256,44.46392649)(410.54112244,44.48892647)(410.69111206,44.48892873)
\lineto(411.09611206,44.48892873)
\lineto(412.85111206,44.48892873)
\lineto(413.10611206,44.48892873)
\lineto(413.39111206,44.48892873)
\curveto(413.4811195,44.49892646)(413.56611942,44.50892645)(413.64611206,44.51892873)
\curveto(413.71611927,44.53892642)(413.76611922,44.56892639)(413.79611206,44.60892873)
\curveto(413.82611916,44.64892631)(413.83111915,44.69392626)(413.81111206,44.74392873)
\curveto(413.79111919,44.79392616)(413.77111921,44.83392612)(413.75111206,44.86392873)
\curveto(413.71111927,44.91392604)(413.67111931,44.958926)(413.63111206,44.99892873)
\lineto(413.51111206,45.14892873)
\curveto(413.46111952,45.21892574)(413.41611957,45.28892567)(413.37611206,45.35892873)
\lineto(413.25611206,45.59892873)
\curveto(413.16611982,45.77892518)(413.10111988,45.99392496)(413.06111206,46.24392873)
\curveto(413.02111996,46.49392446)(413.00111998,46.74892421)(413.00111206,47.00892873)
\curveto(413.00111998,47.26892369)(413.02611996,47.52392343)(413.07611206,47.77392873)
\curveto(413.11611987,48.02392293)(413.17611981,48.24392271)(413.25611206,48.43392873)
\curveto(413.42611956,48.83392212)(413.66111932,49.17892178)(413.96111206,49.46892873)
\curveto(414.26111872,49.7589212)(414.61111837,49.98892097)(415.01111206,50.15892873)
\curveto(415.12111786,50.20892075)(415.23111775,50.24892071)(415.34111206,50.27892873)
\curveto(415.44111754,50.31892064)(415.54611744,50.3589206)(415.65611206,50.39892873)
\curveto(415.76611722,50.42892053)(415.8811171,50.44892051)(416.00111206,50.45892873)
\lineto(416.33111206,50.51892873)
\curveto(416.36111662,50.52892043)(416.39611659,50.53392042)(416.43611206,50.53392873)
\curveto(416.46611652,50.53392042)(416.49611649,50.53892042)(416.52611206,50.54892873)
\curveto(416.5861164,50.56892039)(416.64611634,50.56892039)(416.70611206,50.54892873)
\curveto(416.75611623,50.53892042)(416.81111617,50.54392041)(416.87111206,50.56392873)
\moveto(417.26111206,49.22892873)
\curveto(417.21111577,49.24892171)(417.15111583,49.2539217)(417.08111206,49.24392873)
\curveto(417.01111597,49.23392172)(416.94611604,49.22892173)(416.88611206,49.22892873)
\curveto(416.71611627,49.22892173)(416.55611643,49.21892174)(416.40611206,49.19892873)
\curveto(416.25611673,49.18892177)(416.12111686,49.1589218)(416.00111206,49.10892873)
\curveto(415.90111708,49.07892188)(415.81111717,49.0539219)(415.73111206,49.03392873)
\curveto(415.65111733,49.01392194)(415.57111741,48.98392197)(415.49111206,48.94392873)
\curveto(415.24111774,48.83392212)(415.01111797,48.68392227)(414.80111206,48.49392873)
\curveto(414.5811184,48.30392265)(414.41611857,48.08392287)(414.30611206,47.83392873)
\curveto(414.27611871,47.7539232)(414.25111873,47.67392328)(414.23111206,47.59392873)
\curveto(414.20111878,47.52392343)(414.17611881,47.44892351)(414.15611206,47.36892873)
\curveto(414.12611886,47.2589237)(414.11111887,47.14892381)(414.11111206,47.03892873)
\curveto(414.10111888,46.92892403)(414.09611889,46.80892415)(414.09611206,46.67892873)
\curveto(414.10611888,46.62892433)(414.11611887,46.58392437)(414.12611206,46.54392873)
\lineto(414.12611206,46.40892873)
\lineto(414.18611206,46.13892873)
\curveto(414.20611878,46.0589249)(414.23611875,45.97892498)(414.27611206,45.89892873)
\curveto(414.41611857,45.5589254)(414.62611836,45.28892567)(414.90611206,45.08892873)
\curveto(415.17611781,44.88892607)(415.49611749,44.72892623)(415.86611206,44.60892873)
\curveto(415.97611701,44.56892639)(416.0861169,44.54392641)(416.19611206,44.53392873)
\curveto(416.30611668,44.52392643)(416.42111656,44.50392645)(416.54111206,44.47392873)
\curveto(416.59111639,44.46392649)(416.63611635,44.46392649)(416.67611206,44.47392873)
\curveto(416.71611627,44.48392647)(416.76111622,44.47892648)(416.81111206,44.45892873)
\curveto(416.86111612,44.44892651)(416.93611605,44.44392651)(417.03611206,44.44392873)
\curveto(417.12611586,44.44392651)(417.19611579,44.44892651)(417.24611206,44.45892873)
\lineto(417.36611206,44.45892873)
\curveto(417.40611558,44.46892649)(417.44611554,44.47392648)(417.48611206,44.47392873)
\curveto(417.52611546,44.47392648)(417.56111542,44.47892648)(417.59111206,44.48892873)
\curveto(417.62111536,44.49892646)(417.65611533,44.50392645)(417.69611206,44.50392873)
\curveto(417.72611526,44.50392645)(417.75611523,44.50892645)(417.78611206,44.51892873)
\curveto(417.86611512,44.53892642)(417.94611504,44.5539264)(418.02611206,44.56392873)
\lineto(418.26611206,44.62392873)
\curveto(418.60611438,44.73392622)(418.89611409,44.88392607)(419.13611206,45.07392873)
\curveto(419.37611361,45.27392568)(419.57611341,45.51892544)(419.73611206,45.80892873)
\curveto(419.7861132,45.89892506)(419.82611316,45.99392496)(419.85611206,46.09392873)
\curveto(419.87611311,46.19392476)(419.90111308,46.29892466)(419.93111206,46.40892873)
\curveto(419.95111303,46.4589245)(419.96111302,46.50392445)(419.96111206,46.54392873)
\curveto(419.95111303,46.59392436)(419.95111303,46.64392431)(419.96111206,46.69392873)
\curveto(419.97111301,46.73392422)(419.97611301,46.77892418)(419.97611206,46.82892873)
\lineto(419.97611206,46.96392873)
\lineto(419.97611206,47.09892873)
\curveto(419.96611302,47.13892382)(419.96111302,47.17392378)(419.96111206,47.20392873)
\curveto(419.96111302,47.23392372)(419.95611303,47.26892369)(419.94611206,47.30892873)
\curveto(419.92611306,47.38892357)(419.91111307,47.46392349)(419.90111206,47.53392873)
\curveto(419.8811131,47.60392335)(419.85611313,47.67892328)(419.82611206,47.75892873)
\curveto(419.69611329,48.06892289)(419.52611346,48.31892264)(419.31611206,48.50892873)
\curveto(419.09611389,48.69892226)(418.83111415,48.8589221)(418.52111206,48.98892873)
\curveto(418.3811146,49.03892192)(418.24111474,49.07392188)(418.10111206,49.09392873)
\curveto(417.95111503,49.12392183)(417.80111518,49.1589218)(417.65111206,49.19892873)
\curveto(417.60111538,49.21892174)(417.55611543,49.22392173)(417.51611206,49.21392873)
\curveto(417.46611552,49.21392174)(417.41611557,49.21892174)(417.36611206,49.22892873)
\lineto(417.26111206,49.22892873)
}
}
{
\newrgbcolor{curcolor}{0 0 0}
\pscustom[linestyle=none,fillstyle=solid,fillcolor=curcolor]
{
\newpath
\moveto(413.00111206,55.69017873)
\curveto(413.00111998,55.92017394)(413.06111992,56.05017381)(413.18111206,56.08017873)
\curveto(413.29111969,56.11017375)(413.45611953,56.12517374)(413.67611206,56.12517873)
\lineto(413.96111206,56.12517873)
\curveto(414.05111893,56.12517374)(414.12611886,56.10017376)(414.18611206,56.05017873)
\curveto(414.26611872,55.99017387)(414.31111867,55.90517396)(414.32111206,55.79517873)
\curveto(414.32111866,55.68517418)(414.33611865,55.57517429)(414.36611206,55.46517873)
\curveto(414.39611859,55.32517454)(414.42611856,55.19017467)(414.45611206,55.06017873)
\curveto(414.4861185,54.94017492)(414.52611846,54.82517504)(414.57611206,54.71517873)
\curveto(414.70611828,54.42517544)(414.8861181,54.19017567)(415.11611206,54.01017873)
\curveto(415.33611765,53.83017603)(415.59111739,53.67517619)(415.88111206,53.54517873)
\curveto(415.99111699,53.50517636)(416.10611688,53.47517639)(416.22611206,53.45517873)
\curveto(416.33611665,53.43517643)(416.45111653,53.41017645)(416.57111206,53.38017873)
\curveto(416.62111636,53.37017649)(416.67111631,53.3651765)(416.72111206,53.36517873)
\curveto(416.77111621,53.37517649)(416.82111616,53.37517649)(416.87111206,53.36517873)
\curveto(416.99111599,53.33517653)(417.13111585,53.32017654)(417.29111206,53.32017873)
\curveto(417.44111554,53.33017653)(417.5861154,53.33517653)(417.72611206,53.33517873)
\lineto(419.57111206,53.33517873)
\lineto(419.91611206,53.33517873)
\curveto(420.03611295,53.33517653)(420.15111283,53.33017653)(420.26111206,53.32017873)
\curveto(420.37111261,53.31017655)(420.46611252,53.30517656)(420.54611206,53.30517873)
\curveto(420.62611236,53.31517655)(420.69611229,53.29517657)(420.75611206,53.24517873)
\curveto(420.82611216,53.19517667)(420.86611212,53.11517675)(420.87611206,53.00517873)
\curveto(420.8861121,52.90517696)(420.89111209,52.79517707)(420.89111206,52.67517873)
\lineto(420.89111206,52.40517873)
\curveto(420.87111211,52.35517751)(420.85611213,52.30517756)(420.84611206,52.25517873)
\curveto(420.82611216,52.21517765)(420.80111218,52.18517768)(420.77111206,52.16517873)
\curveto(420.70111228,52.11517775)(420.61611237,52.08517778)(420.51611206,52.07517873)
\lineto(420.18611206,52.07517873)
\lineto(419.03111206,52.07517873)
\lineto(414.87611206,52.07517873)
\lineto(413.84111206,52.07517873)
\lineto(413.54111206,52.07517873)
\curveto(413.44111954,52.08517778)(413.35611963,52.11517775)(413.28611206,52.16517873)
\curveto(413.24611974,52.19517767)(413.21611977,52.24517762)(413.19611206,52.31517873)
\curveto(413.17611981,52.39517747)(413.16611982,52.48017738)(413.16611206,52.57017873)
\curveto(413.15611983,52.6601772)(413.15611983,52.75017711)(413.16611206,52.84017873)
\curveto(413.17611981,52.93017693)(413.19111979,53.00017686)(413.21111206,53.05017873)
\curveto(413.24111974,53.13017673)(413.30111968,53.18017668)(413.39111206,53.20017873)
\curveto(413.47111951,53.23017663)(413.56111942,53.24517662)(413.66111206,53.24517873)
\lineto(413.96111206,53.24517873)
\curveto(414.06111892,53.24517662)(414.15111883,53.2651766)(414.23111206,53.30517873)
\curveto(414.25111873,53.31517655)(414.26611872,53.32517654)(414.27611206,53.33517873)
\lineto(414.32111206,53.38017873)
\curveto(414.32111866,53.49017637)(414.27611871,53.58017628)(414.18611206,53.65017873)
\curveto(414.0861189,53.72017614)(414.00611898,53.78017608)(413.94611206,53.83017873)
\lineto(413.85611206,53.92017873)
\curveto(413.74611924,54.01017585)(413.63111935,54.13517573)(413.51111206,54.29517873)
\curveto(413.39111959,54.45517541)(413.30111968,54.60517526)(413.24111206,54.74517873)
\curveto(413.19111979,54.83517503)(413.15611983,54.93017493)(413.13611206,55.03017873)
\curveto(413.10611988,55.13017473)(413.07611991,55.23517463)(413.04611206,55.34517873)
\curveto(413.03611995,55.40517446)(413.03111995,55.4651744)(413.03111206,55.52517873)
\curveto(413.02111996,55.58517428)(413.01111997,55.64017422)(413.00111206,55.69017873)
}
}
{
\newrgbcolor{curcolor}{0 0 0}
\pscustom[linestyle=none,fillstyle=solid,fillcolor=curcolor]
{
}
}
{
\newrgbcolor{curcolor}{0 0 0}
\pscustom[linestyle=none,fillstyle=solid,fillcolor=curcolor]
{
\newpath
\moveto(410.30111206,65.05510061)
\curveto(410.30112268,65.15509575)(410.31112267,65.25009566)(410.33111206,65.34010061)
\curveto(410.34112264,65.43009548)(410.37112261,65.49509541)(410.42111206,65.53510061)
\curveto(410.50112248,65.59509531)(410.60612238,65.62509528)(410.73611206,65.62510061)
\lineto(411.12611206,65.62510061)
\lineto(412.62611206,65.62510061)
\lineto(419.01611206,65.62510061)
\lineto(420.18611206,65.62510061)
\lineto(420.50111206,65.62510061)
\curveto(420.60111238,65.63509527)(420.6811123,65.62009529)(420.74111206,65.58010061)
\curveto(420.82111216,65.53009538)(420.87111211,65.45509545)(420.89111206,65.35510061)
\curveto(420.90111208,65.26509564)(420.90611208,65.15509575)(420.90611206,65.02510061)
\lineto(420.90611206,64.80010061)
\curveto(420.8861121,64.72009619)(420.87111211,64.65009626)(420.86111206,64.59010061)
\curveto(420.84111214,64.53009638)(420.80111218,64.48009643)(420.74111206,64.44010061)
\curveto(420.6811123,64.40009651)(420.60611238,64.38009653)(420.51611206,64.38010061)
\lineto(420.21611206,64.38010061)
\lineto(419.12111206,64.38010061)
\lineto(413.78111206,64.38010061)
\curveto(413.69111929,64.36009655)(413.61611937,64.34509656)(413.55611206,64.33510061)
\curveto(413.4861195,64.33509657)(413.42611956,64.3050966)(413.37611206,64.24510061)
\curveto(413.32611966,64.17509673)(413.30111968,64.08509682)(413.30111206,63.97510061)
\curveto(413.29111969,63.87509703)(413.2861197,63.76509714)(413.28611206,63.64510061)
\lineto(413.28611206,62.50510061)
\lineto(413.28611206,62.01010061)
\curveto(413.27611971,61.85009906)(413.21611977,61.74009917)(413.10611206,61.68010061)
\curveto(413.07611991,61.66009925)(413.04611994,61.65009926)(413.01611206,61.65010061)
\curveto(412.97612001,61.65009926)(412.93112005,61.64509926)(412.88111206,61.63510061)
\curveto(412.76112022,61.61509929)(412.65112033,61.62009929)(412.55111206,61.65010061)
\curveto(412.45112053,61.69009922)(412.3811206,61.74509916)(412.34111206,61.81510061)
\curveto(412.29112069,61.89509901)(412.26612072,62.01509889)(412.26611206,62.17510061)
\curveto(412.26612072,62.33509857)(412.25112073,62.47009844)(412.22111206,62.58010061)
\curveto(412.21112077,62.63009828)(412.20612078,62.68509822)(412.20611206,62.74510061)
\curveto(412.19612079,62.8050981)(412.1811208,62.86509804)(412.16111206,62.92510061)
\curveto(412.11112087,63.07509783)(412.06112092,63.22009769)(412.01111206,63.36010061)
\curveto(411.95112103,63.50009741)(411.8811211,63.63509727)(411.80111206,63.76510061)
\curveto(411.71112127,63.905097)(411.60612138,64.02509688)(411.48611206,64.12510061)
\curveto(411.36612162,64.22509668)(411.23612175,64.32009659)(411.09611206,64.41010061)
\curveto(410.99612199,64.47009644)(410.8861221,64.51509639)(410.76611206,64.54510061)
\curveto(410.64612234,64.58509632)(410.54112244,64.63509627)(410.45111206,64.69510061)
\curveto(410.39112259,64.74509616)(410.35112263,64.81509609)(410.33111206,64.90510061)
\curveto(410.32112266,64.92509598)(410.31612267,64.95009596)(410.31611206,64.98010061)
\curveto(410.31612267,65.0100959)(410.31112267,65.03509587)(410.30111206,65.05510061)
}
}
{
\newrgbcolor{curcolor}{0 0 0}
\pscustom[linestyle=none,fillstyle=solid,fillcolor=curcolor]
{
\newpath
\moveto(415.31111206,76.28470998)
\curveto(415.39111759,76.28470235)(415.47111751,76.28970234)(415.55111206,76.29970998)
\curveto(415.63111735,76.30970232)(415.70611728,76.30470233)(415.77611206,76.28470998)
\curveto(415.81611717,76.26470237)(415.86111712,76.25970237)(415.91111206,76.26970998)
\curveto(415.95111703,76.27970235)(415.99111699,76.27970235)(416.03111206,76.26970998)
\lineto(416.18111206,76.26970998)
\curveto(416.27111671,76.25970237)(416.36111662,76.25470238)(416.45111206,76.25470998)
\curveto(416.53111645,76.25470238)(416.61111637,76.24970238)(416.69111206,76.23970998)
\lineto(416.93111206,76.20970998)
\curveto(417.00111598,76.19970243)(417.07611591,76.18970244)(417.15611206,76.17970998)
\curveto(417.19611579,76.16970246)(417.23611575,76.16470247)(417.27611206,76.16470998)
\curveto(417.31611567,76.16470247)(417.36111562,76.15970247)(417.41111206,76.14970998)
\curveto(417.55111543,76.10970252)(417.69111529,76.07970255)(417.83111206,76.05970998)
\curveto(417.97111501,76.04970258)(418.10611488,76.01970261)(418.23611206,75.96970998)
\curveto(418.40611458,75.91970271)(418.57111441,75.86470277)(418.73111206,75.80470998)
\curveto(418.89111409,75.75470288)(419.04611394,75.69470294)(419.19611206,75.62470998)
\curveto(419.25611373,75.60470303)(419.31611367,75.57470306)(419.37611206,75.53470998)
\lineto(419.52611206,75.44470998)
\curveto(419.84611314,75.24470339)(420.11111287,75.0297036)(420.32111206,74.79970998)
\curveto(420.53111245,74.56970406)(420.71111227,74.27470436)(420.86111206,73.91470998)
\curveto(420.91111207,73.79470484)(420.94611204,73.66470497)(420.96611206,73.52470998)
\curveto(420.986112,73.39470524)(421.01111197,73.25970537)(421.04111206,73.11970998)
\curveto(421.05111193,73.05970557)(421.05611193,72.99970563)(421.05611206,72.93970998)
\curveto(421.05611193,72.87970575)(421.06111192,72.81470582)(421.07111206,72.74470998)
\curveto(421.0811119,72.71470592)(421.0811119,72.66470597)(421.07111206,72.59470998)
\lineto(421.07111206,72.44470998)
\lineto(421.07111206,72.29470998)
\curveto(421.05111193,72.21470642)(421.03611195,72.1297065)(421.02611206,72.03970998)
\curveto(421.02611196,71.95970667)(421.01611197,71.88470675)(420.99611206,71.81470998)
\curveto(420.986112,71.77470686)(420.981112,71.73970689)(420.98111206,71.70970998)
\curveto(420.99111199,71.68970694)(420.986112,71.66470697)(420.96611206,71.63470998)
\lineto(420.90611206,71.36470998)
\curveto(420.87611211,71.27470736)(420.84611214,71.18970744)(420.81611206,71.10970998)
\curveto(420.57611241,70.5297081)(420.20611278,70.09470854)(419.70611206,69.80470998)
\curveto(419.57611341,69.72470891)(419.44111354,69.65970897)(419.30111206,69.60970998)
\curveto(419.16111382,69.56970906)(419.01111397,69.52470911)(418.85111206,69.47470998)
\curveto(418.77111421,69.45470918)(418.69111429,69.44970918)(418.61111206,69.45970998)
\curveto(418.53111445,69.47970915)(418.47611451,69.51470912)(418.44611206,69.56470998)
\curveto(418.42611456,69.59470904)(418.41111457,69.64970898)(418.40111206,69.72970998)
\curveto(418.3811146,69.80970882)(418.37111461,69.89470874)(418.37111206,69.98470998)
\curveto(418.36111462,70.07470856)(418.36111462,70.15970847)(418.37111206,70.23970998)
\curveto(418.3811146,70.3297083)(418.39111459,70.39970823)(418.40111206,70.44970998)
\curveto(418.41111457,70.46970816)(418.42611456,70.49470814)(418.44611206,70.52470998)
\curveto(418.46611452,70.56470807)(418.4861145,70.59470804)(418.50611206,70.61470998)
\curveto(418.5861144,70.67470796)(418.6811143,70.71970791)(418.79111206,70.74970998)
\curveto(418.90111408,70.78970784)(419.00111398,70.8347078)(419.09111206,70.88470998)
\curveto(419.4811135,71.1347075)(419.75111323,71.50470713)(419.90111206,71.99470998)
\curveto(419.92111306,72.06470657)(419.93611305,72.1347065)(419.94611206,72.20470998)
\curveto(419.94611304,72.28470635)(419.95611303,72.36470627)(419.97611206,72.44470998)
\curveto(419.986113,72.48470615)(419.99111299,72.53970609)(419.99111206,72.60970998)
\curveto(419.99111299,72.68970594)(419.986113,72.74470589)(419.97611206,72.77470998)
\curveto(419.96611302,72.80470583)(419.96111302,72.8347058)(419.96111206,72.86470998)
\lineto(419.96111206,72.96970998)
\curveto(419.94111304,73.04970558)(419.92111306,73.12470551)(419.90111206,73.19470998)
\curveto(419.8811131,73.27470536)(419.85611313,73.34970528)(419.82611206,73.41970998)
\curveto(419.67611331,73.76970486)(419.46111352,74.03970459)(419.18111206,74.22970998)
\curveto(418.90111408,74.41970421)(418.57611441,74.57470406)(418.20611206,74.69470998)
\curveto(418.12611486,74.72470391)(418.05111493,74.74470389)(417.98111206,74.75470998)
\curveto(417.91111507,74.77470386)(417.83611515,74.79470384)(417.75611206,74.81470998)
\curveto(417.66611532,74.8347038)(417.57111541,74.84970378)(417.47111206,74.85970998)
\curveto(417.36111562,74.87970375)(417.25611573,74.89970373)(417.15611206,74.91970998)
\curveto(417.10611588,74.9297037)(417.05611593,74.9347037)(417.00611206,74.93470998)
\curveto(416.94611604,74.94470369)(416.89111609,74.94970368)(416.84111206,74.94970998)
\curveto(416.7811162,74.96970366)(416.70611628,74.97970365)(416.61611206,74.97970998)
\curveto(416.51611647,74.97970365)(416.43611655,74.96970366)(416.37611206,74.94970998)
\curveto(416.2861167,74.91970371)(416.24611674,74.86970376)(416.25611206,74.79970998)
\curveto(416.26611672,74.73970389)(416.29611669,74.68470395)(416.34611206,74.63470998)
\curveto(416.39611659,74.55470408)(416.45611653,74.48470415)(416.52611206,74.42470998)
\curveto(416.59611639,74.37470426)(416.65611633,74.30970432)(416.70611206,74.22970998)
\curveto(416.81611617,74.06970456)(416.91611607,73.90470473)(417.00611206,73.73470998)
\curveto(417.0861159,73.56470507)(417.15611583,73.36970526)(417.21611206,73.14970998)
\curveto(417.24611574,73.04970558)(417.26111572,72.94970568)(417.26111206,72.84970998)
\curveto(417.26111572,72.75970587)(417.27111571,72.65970597)(417.29111206,72.54970998)
\lineto(417.29111206,72.39970998)
\curveto(417.27111571,72.34970628)(417.26611572,72.29970633)(417.27611206,72.24970998)
\curveto(417.2861157,72.20970642)(417.2861157,72.16970646)(417.27611206,72.12970998)
\curveto(417.26611572,72.09970653)(417.26111572,72.05470658)(417.26111206,71.99470998)
\curveto(417.25111573,71.9347067)(417.24111574,71.86970676)(417.23111206,71.79970998)
\lineto(417.20111206,71.61970998)
\curveto(417.0811159,71.16970746)(416.91611607,70.78970784)(416.70611206,70.47970998)
\curveto(416.51611647,70.20970842)(416.2861167,69.97970865)(416.01611206,69.78970998)
\curveto(415.73611725,69.60970902)(415.42111756,69.46470917)(415.07111206,69.35470998)
\lineto(414.86111206,69.29470998)
\curveto(414.7811182,69.28470935)(414.70111828,69.26970936)(414.62111206,69.24970998)
\curveto(414.59111839,69.23970939)(414.56111842,69.2347094)(414.53111206,69.23470998)
\curveto(414.50111848,69.2347094)(414.47111851,69.2297094)(414.44111206,69.21970998)
\curveto(414.3811186,69.20970942)(414.32111866,69.20470943)(414.26111206,69.20470998)
\curveto(414.19111879,69.20470943)(414.13111885,69.19470944)(414.08111206,69.17470998)
\lineto(413.90111206,69.17470998)
\curveto(413.85111913,69.16470947)(413.7811192,69.15970947)(413.69111206,69.15970998)
\curveto(413.60111938,69.15970947)(413.53111945,69.16970946)(413.48111206,69.18970998)
\lineto(413.31611206,69.18970998)
\curveto(413.23611975,69.20970942)(413.16111982,69.21970941)(413.09111206,69.21970998)
\curveto(413.02111996,69.2297094)(412.95112003,69.24470939)(412.88111206,69.26470998)
\curveto(412.6811203,69.32470931)(412.49112049,69.38470925)(412.31111206,69.44470998)
\curveto(412.13112085,69.51470912)(411.96112102,69.60470903)(411.80111206,69.71470998)
\curveto(411.73112125,69.75470888)(411.66612132,69.79470884)(411.60611206,69.83470998)
\lineto(411.42611206,69.98470998)
\curveto(411.41612157,70.00470863)(411.40112158,70.02470861)(411.38111206,70.04470998)
\curveto(411.25112173,70.1347085)(411.14112184,70.24470839)(411.05111206,70.37470998)
\curveto(410.85112213,70.634708)(410.69612229,70.89970773)(410.58611206,71.16970998)
\curveto(410.54612244,71.24970738)(410.51612247,71.3297073)(410.49611206,71.40970998)
\curveto(410.46612252,71.49970713)(410.44112254,71.58970704)(410.42111206,71.67970998)
\curveto(410.39112259,71.77970685)(410.37112261,71.87970675)(410.36111206,71.97970998)
\curveto(410.35112263,72.07970655)(410.33612265,72.18470645)(410.31611206,72.29470998)
\curveto(410.30612268,72.32470631)(410.30612268,72.36470627)(410.31611206,72.41470998)
\curveto(410.32612266,72.47470616)(410.32112266,72.51470612)(410.30111206,72.53470998)
\curveto(410.2811227,73.25470538)(410.39612259,73.85470478)(410.64611206,74.33470998)
\curveto(410.89612209,74.81470382)(411.23612175,75.18970344)(411.66611206,75.45970998)
\curveto(411.80612118,75.54970308)(411.95112103,75.629703)(412.10111206,75.69970998)
\curveto(412.25112073,75.76970286)(412.41112057,75.83970279)(412.58111206,75.90970998)
\curveto(412.72112026,75.95970267)(412.87112011,75.99970263)(413.03111206,76.02970998)
\curveto(413.19111979,76.05970257)(413.35111963,76.09470254)(413.51111206,76.13470998)
\curveto(413.56111942,76.15470248)(413.61611937,76.16470247)(413.67611206,76.16470998)
\curveto(413.72611926,76.16470247)(413.77611921,76.16970246)(413.82611206,76.17970998)
\curveto(413.8861191,76.19970243)(413.95111903,76.20970242)(414.02111206,76.20970998)
\curveto(414.0811189,76.20970242)(414.13611885,76.21970241)(414.18611206,76.23970998)
\lineto(414.35111206,76.23970998)
\curveto(414.40111858,76.25970237)(414.45111853,76.26470237)(414.50111206,76.25470998)
\curveto(414.55111843,76.24470239)(414.60111838,76.24970238)(414.65111206,76.26970998)
\curveto(414.67111831,76.26970236)(414.69611829,76.26470237)(414.72611206,76.25470998)
\curveto(414.75611823,76.25470238)(414.7811182,76.25970237)(414.80111206,76.26970998)
\curveto(414.83111815,76.27970235)(414.86611812,76.27970235)(414.90611206,76.26970998)
\curveto(414.94611804,76.26970236)(414.986118,76.27470236)(415.02611206,76.28470998)
\curveto(415.06611792,76.29470234)(415.11111787,76.29470234)(415.16111206,76.28470998)
\lineto(415.31111206,76.28470998)
\moveto(414.00611206,74.78470998)
\curveto(413.95611903,74.79470384)(413.89611909,74.79970383)(413.82611206,74.79970998)
\curveto(413.75611923,74.79970383)(413.69611929,74.79470384)(413.64611206,74.78470998)
\curveto(413.59611939,74.77470386)(413.52111946,74.76970386)(413.42111206,74.76970998)
\curveto(413.34111964,74.74970388)(413.26611972,74.7297039)(413.19611206,74.70970998)
\curveto(413.12611986,74.69970393)(413.05611993,74.68470395)(412.98611206,74.66470998)
\curveto(412.55612043,74.52470411)(412.22112076,74.3297043)(411.98111206,74.07970998)
\curveto(411.74112124,73.83970479)(411.56112142,73.49470514)(411.44111206,73.04470998)
\curveto(411.42112156,72.95470568)(411.41112157,72.85470578)(411.41111206,72.74470998)
\lineto(411.41111206,72.41470998)
\curveto(411.43112155,72.39470624)(411.44112154,72.35970627)(411.44111206,72.30970998)
\curveto(411.43112155,72.25970637)(411.43112155,72.21470642)(411.44111206,72.17470998)
\curveto(411.46112152,72.09470654)(411.4811215,72.01970661)(411.50111206,71.94970998)
\lineto(411.56111206,71.73970998)
\curveto(411.69112129,71.44970718)(411.87112111,71.21970741)(412.10111206,71.04970998)
\curveto(412.32112066,70.87970775)(412.5811204,70.74470789)(412.88111206,70.64470998)
\curveto(412.97112001,70.61470802)(413.06611992,70.58970804)(413.16611206,70.56970998)
\curveto(413.25611973,70.55970807)(413.35111963,70.54470809)(413.45111206,70.52470998)
\lineto(413.58611206,70.52470998)
\curveto(413.69611929,70.49470814)(413.83611915,70.48470815)(414.00611206,70.49470998)
\curveto(414.16611882,70.51470812)(414.29611869,70.5347081)(414.39611206,70.55470998)
\curveto(414.45611853,70.57470806)(414.51611847,70.58970804)(414.57611206,70.59970998)
\curveto(414.62611836,70.60970802)(414.67611831,70.62470801)(414.72611206,70.64470998)
\curveto(414.92611806,70.72470791)(415.11611787,70.81970781)(415.29611206,70.92970998)
\curveto(415.47611751,71.04970758)(415.62111736,71.18970744)(415.73111206,71.34970998)
\curveto(415.7811172,71.39970723)(415.82111716,71.45470718)(415.85111206,71.51470998)
\curveto(415.8811171,71.57470706)(415.91611707,71.634707)(415.95611206,71.69470998)
\curveto(416.03611695,71.84470679)(416.10111688,72.0297066)(416.15111206,72.24970998)
\curveto(416.17111681,72.29970633)(416.17611681,72.33970629)(416.16611206,72.36970998)
\curveto(416.15611683,72.40970622)(416.16111682,72.45470618)(416.18111206,72.50470998)
\curveto(416.19111679,72.54470609)(416.19611679,72.59970603)(416.19611206,72.66970998)
\curveto(416.19611679,72.73970589)(416.19111679,72.79970583)(416.18111206,72.84970998)
\curveto(416.16111682,72.94970568)(416.14611684,73.04470559)(416.13611206,73.13470998)
\curveto(416.11611687,73.22470541)(416.0861169,73.31470532)(416.04611206,73.40470998)
\curveto(415.82611716,73.94470469)(415.43111755,74.33970429)(414.86111206,74.58970998)
\curveto(414.76111822,74.63970399)(414.66111832,74.67470396)(414.56111206,74.69470998)
\curveto(414.45111853,74.71470392)(414.34111864,74.73970389)(414.23111206,74.76970998)
\curveto(414.13111885,74.76970386)(414.05611893,74.77470386)(414.00611206,74.78470998)
}
}
{
\newrgbcolor{curcolor}{0 0 0}
\pscustom[linestyle=none,fillstyle=solid,fillcolor=curcolor]
{
\newpath
\moveto(419.27111206,78.63431936)
\lineto(419.27111206,79.26431936)
\lineto(419.27111206,79.45931936)
\curveto(419.27111371,79.52931683)(419.2811137,79.58931677)(419.30111206,79.63931936)
\curveto(419.34111364,79.70931665)(419.3811136,79.7593166)(419.42111206,79.78931936)
\curveto(419.47111351,79.82931653)(419.53611345,79.84931651)(419.61611206,79.84931936)
\curveto(419.69611329,79.8593165)(419.7811132,79.86431649)(419.87111206,79.86431936)
\lineto(420.59111206,79.86431936)
\curveto(421.07111191,79.86431649)(421.4811115,79.80431655)(421.82111206,79.68431936)
\curveto(422.16111082,79.56431679)(422.43611055,79.36931699)(422.64611206,79.09931936)
\curveto(422.69611029,79.02931733)(422.74111024,78.9593174)(422.78111206,78.88931936)
\curveto(422.83111015,78.82931753)(422.87611011,78.7543176)(422.91611206,78.66431936)
\curveto(422.92611006,78.64431771)(422.93611005,78.61431774)(422.94611206,78.57431936)
\curveto(422.96611002,78.53431782)(422.97111001,78.48931787)(422.96111206,78.43931936)
\curveto(422.93111005,78.34931801)(422.85611013,78.29431806)(422.73611206,78.27431936)
\curveto(422.62611036,78.2543181)(422.53111045,78.26931809)(422.45111206,78.31931936)
\curveto(422.3811106,78.34931801)(422.31611067,78.39431796)(422.25611206,78.45431936)
\curveto(422.20611078,78.52431783)(422.15611083,78.58931777)(422.10611206,78.64931936)
\curveto(422.05611093,78.71931764)(421.981111,78.77931758)(421.88111206,78.82931936)
\curveto(421.79111119,78.88931747)(421.70111128,78.93931742)(421.61111206,78.97931936)
\curveto(421.5811114,78.99931736)(421.52111146,79.02431733)(421.43111206,79.05431936)
\curveto(421.35111163,79.08431727)(421.2811117,79.08931727)(421.22111206,79.06931936)
\curveto(421.0811119,79.03931732)(420.99111199,78.97931738)(420.95111206,78.88931936)
\curveto(420.92111206,78.80931755)(420.90611208,78.71931764)(420.90611206,78.61931936)
\curveto(420.90611208,78.51931784)(420.8811121,78.43431792)(420.83111206,78.36431936)
\curveto(420.76111222,78.27431808)(420.62111236,78.22931813)(420.41111206,78.22931936)
\lineto(419.85611206,78.22931936)
\lineto(419.63111206,78.22931936)
\curveto(419.55111343,78.23931812)(419.4861135,78.2593181)(419.43611206,78.28931936)
\curveto(419.35611363,78.34931801)(419.31111367,78.41931794)(419.30111206,78.49931936)
\curveto(419.29111369,78.51931784)(419.2861137,78.53931782)(419.28611206,78.55931936)
\curveto(419.2861137,78.58931777)(419.2811137,78.61431774)(419.27111206,78.63431936)
}
}
{
\newrgbcolor{curcolor}{0 0 0}
\pscustom[linestyle=none,fillstyle=solid,fillcolor=curcolor]
{
}
}
{
\newrgbcolor{curcolor}{0 0 0}
\pscustom[linestyle=none,fillstyle=solid,fillcolor=curcolor]
{
\newpath
\moveto(410.30111206,89.26463186)
\curveto(410.29112269,89.95462722)(410.41112257,90.55462662)(410.66111206,91.06463186)
\curveto(410.91112207,91.58462559)(411.24612174,91.9796252)(411.66611206,92.24963186)
\curveto(411.74612124,92.29962488)(411.83612115,92.34462483)(411.93611206,92.38463186)
\curveto(412.02612096,92.42462475)(412.12112086,92.46962471)(412.22111206,92.51963186)
\curveto(412.32112066,92.55962462)(412.42112056,92.58962459)(412.52111206,92.60963186)
\curveto(412.62112036,92.62962455)(412.72612026,92.64962453)(412.83611206,92.66963186)
\curveto(412.8861201,92.68962449)(412.93112005,92.69462448)(412.97111206,92.68463186)
\curveto(413.01111997,92.6746245)(413.05611993,92.6796245)(413.10611206,92.69963186)
\curveto(413.15611983,92.70962447)(413.24111974,92.71462446)(413.36111206,92.71463186)
\curveto(413.47111951,92.71462446)(413.55611943,92.70962447)(413.61611206,92.69963186)
\curveto(413.67611931,92.6796245)(413.73611925,92.66962451)(413.79611206,92.66963186)
\curveto(413.85611913,92.6796245)(413.91611907,92.6746245)(413.97611206,92.65463186)
\curveto(414.11611887,92.61462456)(414.25111873,92.5796246)(414.38111206,92.54963186)
\curveto(414.51111847,92.51962466)(414.63611835,92.4796247)(414.75611206,92.42963186)
\curveto(414.89611809,92.36962481)(415.02111796,92.29962488)(415.13111206,92.21963186)
\curveto(415.24111774,92.14962503)(415.35111763,92.0746251)(415.46111206,91.99463186)
\lineto(415.52111206,91.93463186)
\curveto(415.54111744,91.92462525)(415.56111742,91.90962527)(415.58111206,91.88963186)
\curveto(415.74111724,91.76962541)(415.8861171,91.63462554)(416.01611206,91.48463186)
\curveto(416.14611684,91.33462584)(416.27111671,91.174626)(416.39111206,91.00463186)
\curveto(416.61111637,90.69462648)(416.81611617,90.39962678)(417.00611206,90.11963186)
\curveto(417.14611584,89.88962729)(417.2811157,89.65962752)(417.41111206,89.42963186)
\curveto(417.54111544,89.20962797)(417.67611531,88.98962819)(417.81611206,88.76963186)
\curveto(417.986115,88.51962866)(418.16611482,88.2796289)(418.35611206,88.04963186)
\curveto(418.54611444,87.82962935)(418.77111421,87.63962954)(419.03111206,87.47963186)
\curveto(419.09111389,87.43962974)(419.15111383,87.40462977)(419.21111206,87.37463186)
\curveto(419.26111372,87.34462983)(419.32611366,87.31462986)(419.40611206,87.28463186)
\curveto(419.47611351,87.26462991)(419.53611345,87.25962992)(419.58611206,87.26963186)
\curveto(419.65611333,87.28962989)(419.71111327,87.32462985)(419.75111206,87.37463186)
\curveto(419.7811132,87.42462975)(419.80111318,87.48462969)(419.81111206,87.55463186)
\lineto(419.81111206,87.79463186)
\lineto(419.81111206,88.54463186)
\lineto(419.81111206,91.34963186)
\lineto(419.81111206,92.00963186)
\curveto(419.81111317,92.09962508)(419.81611317,92.18462499)(419.82611206,92.26463186)
\curveto(419.82611316,92.34462483)(419.84611314,92.40962477)(419.88611206,92.45963186)
\curveto(419.92611306,92.50962467)(420.00111298,92.54962463)(420.11111206,92.57963186)
\curveto(420.21111277,92.61962456)(420.31111267,92.62962455)(420.41111206,92.60963186)
\lineto(420.54611206,92.60963186)
\curveto(420.61611237,92.58962459)(420.67611231,92.56962461)(420.72611206,92.54963186)
\curveto(420.77611221,92.52962465)(420.81611217,92.49462468)(420.84611206,92.44463186)
\curveto(420.8861121,92.39462478)(420.90611208,92.32462485)(420.90611206,92.23463186)
\lineto(420.90611206,91.96463186)
\lineto(420.90611206,91.06463186)
\lineto(420.90611206,87.55463186)
\lineto(420.90611206,86.48963186)
\curveto(420.90611208,86.40963077)(420.91111207,86.31963086)(420.92111206,86.21963186)
\curveto(420.92111206,86.11963106)(420.91111207,86.03463114)(420.89111206,85.96463186)
\curveto(420.82111216,85.75463142)(420.64111234,85.68963149)(420.35111206,85.76963186)
\curveto(420.31111267,85.7796314)(420.27611271,85.7796314)(420.24611206,85.76963186)
\curveto(420.20611278,85.76963141)(420.16111282,85.7796314)(420.11111206,85.79963186)
\curveto(420.03111295,85.81963136)(419.94611304,85.83963134)(419.85611206,85.85963186)
\curveto(419.76611322,85.8796313)(419.6811133,85.90463127)(419.60111206,85.93463186)
\curveto(419.11111387,86.09463108)(418.69611429,86.29463088)(418.35611206,86.53463186)
\curveto(418.10611488,86.71463046)(417.8811151,86.91963026)(417.68111206,87.14963186)
\curveto(417.47111551,87.3796298)(417.27611571,87.61962956)(417.09611206,87.86963186)
\curveto(416.91611607,88.12962905)(416.74611624,88.39462878)(416.58611206,88.66463186)
\curveto(416.41611657,88.94462823)(416.24111674,89.21462796)(416.06111206,89.47463186)
\curveto(415.981117,89.58462759)(415.90611708,89.68962749)(415.83611206,89.78963186)
\curveto(415.76611722,89.89962728)(415.69111729,90.00962717)(415.61111206,90.11963186)
\curveto(415.5811174,90.15962702)(415.55111743,90.19462698)(415.52111206,90.22463186)
\curveto(415.4811175,90.26462691)(415.45111753,90.30462687)(415.43111206,90.34463186)
\curveto(415.32111766,90.48462669)(415.19611779,90.60962657)(415.05611206,90.71963186)
\curveto(415.02611796,90.73962644)(415.00111798,90.76462641)(414.98111206,90.79463186)
\curveto(414.95111803,90.82462635)(414.92111806,90.84962633)(414.89111206,90.86963186)
\curveto(414.79111819,90.94962623)(414.69111829,91.01462616)(414.59111206,91.06463186)
\curveto(414.49111849,91.12462605)(414.3811186,91.179626)(414.26111206,91.22963186)
\curveto(414.19111879,91.25962592)(414.11611887,91.2796259)(414.03611206,91.28963186)
\lineto(413.79611206,91.34963186)
\lineto(413.70611206,91.34963186)
\curveto(413.67611931,91.35962582)(413.64611934,91.36462581)(413.61611206,91.36463186)
\curveto(413.54611944,91.38462579)(413.45111953,91.38962579)(413.33111206,91.37963186)
\curveto(413.20111978,91.3796258)(413.10111988,91.36962581)(413.03111206,91.34963186)
\curveto(412.95112003,91.32962585)(412.87612011,91.30962587)(412.80611206,91.28963186)
\curveto(412.72612026,91.2796259)(412.64612034,91.25962592)(412.56611206,91.22963186)
\curveto(412.32612066,91.11962606)(412.12612086,90.96962621)(411.96611206,90.77963186)
\curveto(411.79612119,90.59962658)(411.65612133,90.3796268)(411.54611206,90.11963186)
\curveto(411.52612146,90.04962713)(411.51112147,89.9796272)(411.50111206,89.90963186)
\curveto(411.4811215,89.83962734)(411.46112152,89.76462741)(411.44111206,89.68463186)
\curveto(411.42112156,89.60462757)(411.41112157,89.49462768)(411.41111206,89.35463186)
\curveto(411.41112157,89.22462795)(411.42112156,89.11962806)(411.44111206,89.03963186)
\curveto(411.45112153,88.9796282)(411.45612153,88.92462825)(411.45611206,88.87463186)
\curveto(411.45612153,88.82462835)(411.46612152,88.7746284)(411.48611206,88.72463186)
\curveto(411.52612146,88.62462855)(411.56612142,88.52962865)(411.60611206,88.43963186)
\curveto(411.64612134,88.35962882)(411.69112129,88.2796289)(411.74111206,88.19963186)
\curveto(411.76112122,88.16962901)(411.7861212,88.13962904)(411.81611206,88.10963186)
\curveto(411.84612114,88.08962909)(411.87112111,88.06462911)(411.89111206,88.03463186)
\lineto(411.96611206,87.95963186)
\curveto(411.986121,87.92962925)(412.00612098,87.90462927)(412.02611206,87.88463186)
\lineto(412.23611206,87.73463186)
\curveto(412.29612069,87.69462948)(412.36112062,87.64962953)(412.43111206,87.59963186)
\curveto(412.52112046,87.53962964)(412.62612036,87.48962969)(412.74611206,87.44963186)
\curveto(412.85612013,87.41962976)(412.96612002,87.38462979)(413.07611206,87.34463186)
\curveto(413.1861198,87.30462987)(413.33111965,87.2796299)(413.51111206,87.26963186)
\curveto(413.6811193,87.25962992)(413.80611918,87.22962995)(413.88611206,87.17963186)
\curveto(413.96611902,87.12963005)(414.01111897,87.05463012)(414.02111206,86.95463186)
\curveto(414.03111895,86.85463032)(414.03611895,86.74463043)(414.03611206,86.62463186)
\curveto(414.03611895,86.58463059)(414.04111894,86.54463063)(414.05111206,86.50463186)
\curveto(414.05111893,86.46463071)(414.04611894,86.42963075)(414.03611206,86.39963186)
\curveto(414.01611897,86.34963083)(414.00611898,86.29963088)(414.00611206,86.24963186)
\curveto(414.00611898,86.20963097)(413.99611899,86.16963101)(413.97611206,86.12963186)
\curveto(413.91611907,86.03963114)(413.7811192,85.99463118)(413.57111206,85.99463186)
\lineto(413.45111206,85.99463186)
\curveto(413.39111959,86.00463117)(413.33111965,86.00963117)(413.27111206,86.00963186)
\curveto(413.20111978,86.01963116)(413.13611985,86.02963115)(413.07611206,86.03963186)
\curveto(412.96612002,86.05963112)(412.86612012,86.0796311)(412.77611206,86.09963186)
\curveto(412.67612031,86.11963106)(412.5811204,86.14963103)(412.49111206,86.18963186)
\curveto(412.42112056,86.20963097)(412.36112062,86.22963095)(412.31111206,86.24963186)
\lineto(412.13111206,86.30963186)
\curveto(411.87112111,86.42963075)(411.62612136,86.58463059)(411.39611206,86.77463186)
\curveto(411.16612182,86.9746302)(410.981122,87.18962999)(410.84111206,87.41963186)
\curveto(410.76112222,87.52962965)(410.69612229,87.64462953)(410.64611206,87.76463186)
\lineto(410.49611206,88.15463186)
\curveto(410.44612254,88.26462891)(410.41612257,88.3796288)(410.40611206,88.49963186)
\curveto(410.3861226,88.61962856)(410.36112262,88.74462843)(410.33111206,88.87463186)
\curveto(410.33112265,88.94462823)(410.33112265,89.00962817)(410.33111206,89.06963186)
\curveto(410.32112266,89.12962805)(410.31112267,89.19462798)(410.30111206,89.26463186)
}
}
{
\newrgbcolor{curcolor}{0 0 0}
\pscustom[linestyle=none,fillstyle=solid,fillcolor=curcolor]
{
\newpath
\moveto(415.82111206,101.36424123)
\lineto(416.07611206,101.36424123)
\curveto(416.15611683,101.37423353)(416.23111675,101.36923353)(416.30111206,101.34924123)
\lineto(416.54111206,101.34924123)
\lineto(416.70611206,101.34924123)
\curveto(416.80611618,101.32923357)(416.91111607,101.31923358)(417.02111206,101.31924123)
\curveto(417.12111586,101.31923358)(417.22111576,101.30923359)(417.32111206,101.28924123)
\lineto(417.47111206,101.28924123)
\curveto(417.61111537,101.25923364)(417.75111523,101.23923366)(417.89111206,101.22924123)
\curveto(418.02111496,101.21923368)(418.15111483,101.19423371)(418.28111206,101.15424123)
\curveto(418.36111462,101.13423377)(418.44611454,101.11423379)(418.53611206,101.09424123)
\lineto(418.77611206,101.03424123)
\lineto(419.07611206,100.91424123)
\curveto(419.16611382,100.88423402)(419.25611373,100.84923405)(419.34611206,100.80924123)
\curveto(419.56611342,100.70923419)(419.7811132,100.57423433)(419.99111206,100.40424123)
\curveto(420.20111278,100.24423466)(420.37111261,100.06923483)(420.50111206,99.87924123)
\curveto(420.54111244,99.82923507)(420.5811124,99.76923513)(420.62111206,99.69924123)
\curveto(420.65111233,99.63923526)(420.6861123,99.57923532)(420.72611206,99.51924123)
\curveto(420.77611221,99.43923546)(420.81611217,99.34423556)(420.84611206,99.23424123)
\curveto(420.87611211,99.12423578)(420.90611208,99.01923588)(420.93611206,98.91924123)
\curveto(420.97611201,98.80923609)(421.00111198,98.6992362)(421.01111206,98.58924123)
\curveto(421.02111196,98.47923642)(421.03611195,98.36423654)(421.05611206,98.24424123)
\curveto(421.06611192,98.2042367)(421.06611192,98.15923674)(421.05611206,98.10924123)
\curveto(421.05611193,98.06923683)(421.06111192,98.02923687)(421.07111206,97.98924123)
\curveto(421.0811119,97.94923695)(421.0861119,97.89423701)(421.08611206,97.82424123)
\curveto(421.0861119,97.75423715)(421.0811119,97.7042372)(421.07111206,97.67424123)
\curveto(421.05111193,97.62423728)(421.04611194,97.57923732)(421.05611206,97.53924123)
\curveto(421.06611192,97.4992374)(421.06611192,97.46423744)(421.05611206,97.43424123)
\lineto(421.05611206,97.34424123)
\curveto(421.03611195,97.28423762)(421.02111196,97.21923768)(421.01111206,97.14924123)
\curveto(421.01111197,97.08923781)(421.00611198,97.02423788)(420.99611206,96.95424123)
\curveto(420.94611204,96.78423812)(420.89611209,96.62423828)(420.84611206,96.47424123)
\curveto(420.79611219,96.32423858)(420.73111225,96.17923872)(420.65111206,96.03924123)
\curveto(420.61111237,95.98923891)(420.5811124,95.93423897)(420.56111206,95.87424123)
\curveto(420.53111245,95.82423908)(420.49611249,95.77423913)(420.45611206,95.72424123)
\curveto(420.27611271,95.48423942)(420.05611293,95.28423962)(419.79611206,95.12424123)
\curveto(419.53611345,94.96423994)(419.25111373,94.82424008)(418.94111206,94.70424123)
\curveto(418.80111418,94.64424026)(418.66111432,94.5992403)(418.52111206,94.56924123)
\curveto(418.37111461,94.53924036)(418.21611477,94.5042404)(418.05611206,94.46424123)
\curveto(417.94611504,94.44424046)(417.83611515,94.42924047)(417.72611206,94.41924123)
\curveto(417.61611537,94.40924049)(417.50611548,94.39424051)(417.39611206,94.37424123)
\curveto(417.35611563,94.36424054)(417.31611567,94.35924054)(417.27611206,94.35924123)
\curveto(417.23611575,94.36924053)(417.19611579,94.36924053)(417.15611206,94.35924123)
\curveto(417.10611588,94.34924055)(417.05611593,94.34424056)(417.00611206,94.34424123)
\lineto(416.84111206,94.34424123)
\curveto(416.79111619,94.32424058)(416.74111624,94.31924058)(416.69111206,94.32924123)
\curveto(416.63111635,94.33924056)(416.57611641,94.33924056)(416.52611206,94.32924123)
\curveto(416.4861165,94.31924058)(416.44111654,94.31924058)(416.39111206,94.32924123)
\curveto(416.34111664,94.33924056)(416.29111669,94.33424057)(416.24111206,94.31424123)
\curveto(416.17111681,94.29424061)(416.09611689,94.28924061)(416.01611206,94.29924123)
\curveto(415.92611706,94.30924059)(415.84111714,94.31424059)(415.76111206,94.31424123)
\curveto(415.67111731,94.31424059)(415.57111741,94.30924059)(415.46111206,94.29924123)
\curveto(415.34111764,94.28924061)(415.24111774,94.29424061)(415.16111206,94.31424123)
\lineto(414.87611206,94.31424123)
\lineto(414.24611206,94.35924123)
\curveto(414.14611884,94.36924053)(414.05111893,94.37924052)(413.96111206,94.38924123)
\lineto(413.66111206,94.41924123)
\curveto(413.61111937,94.43924046)(413.56111942,94.44424046)(413.51111206,94.43424123)
\curveto(413.45111953,94.43424047)(413.39611959,94.44424046)(413.34611206,94.46424123)
\curveto(413.17611981,94.51424039)(413.01111997,94.55424035)(412.85111206,94.58424123)
\curveto(412.6811203,94.61424029)(412.52112046,94.66424024)(412.37111206,94.73424123)
\curveto(411.91112107,94.92423998)(411.53612145,95.14423976)(411.24611206,95.39424123)
\curveto(410.95612203,95.65423925)(410.71112227,96.01423889)(410.51111206,96.47424123)
\curveto(410.46112252,96.6042383)(410.42612256,96.73423817)(410.40611206,96.86424123)
\curveto(410.3861226,97.0042379)(410.36112262,97.14423776)(410.33111206,97.28424123)
\curveto(410.32112266,97.35423755)(410.31612267,97.41923748)(410.31611206,97.47924123)
\curveto(410.31612267,97.53923736)(410.31112267,97.6042373)(410.30111206,97.67424123)
\curveto(410.2811227,98.5042364)(410.43112255,99.17423573)(410.75111206,99.68424123)
\curveto(411.06112192,100.19423471)(411.50112148,100.57423433)(412.07111206,100.82424123)
\curveto(412.19112079,100.87423403)(412.31612067,100.91923398)(412.44611206,100.95924123)
\curveto(412.57612041,100.9992339)(412.71112027,101.04423386)(412.85111206,101.09424123)
\curveto(412.93112005,101.11423379)(413.01611997,101.12923377)(413.10611206,101.13924123)
\lineto(413.34611206,101.19924123)
\curveto(413.45611953,101.22923367)(413.56611942,101.24423366)(413.67611206,101.24424123)
\curveto(413.7861192,101.25423365)(413.89611909,101.26923363)(414.00611206,101.28924123)
\curveto(414.05611893,101.30923359)(414.10111888,101.31423359)(414.14111206,101.30424123)
\curveto(414.1811188,101.3042336)(414.22111876,101.30923359)(414.26111206,101.31924123)
\curveto(414.31111867,101.32923357)(414.36611862,101.32923357)(414.42611206,101.31924123)
\curveto(414.47611851,101.31923358)(414.52611846,101.32423358)(414.57611206,101.33424123)
\lineto(414.71111206,101.33424123)
\curveto(414.77111821,101.35423355)(414.84111814,101.35423355)(414.92111206,101.33424123)
\curveto(414.99111799,101.32423358)(415.05611793,101.32923357)(415.11611206,101.34924123)
\curveto(415.14611784,101.35923354)(415.1861178,101.36423354)(415.23611206,101.36424123)
\lineto(415.35611206,101.36424123)
\lineto(415.82111206,101.36424123)
\moveto(418.14611206,99.81924123)
\curveto(417.82611516,99.91923498)(417.46111552,99.97923492)(417.05111206,99.99924123)
\curveto(416.64111634,100.01923488)(416.23111675,100.02923487)(415.82111206,100.02924123)
\curveto(415.39111759,100.02923487)(414.97111801,100.01923488)(414.56111206,99.99924123)
\curveto(414.15111883,99.97923492)(413.76611922,99.93423497)(413.40611206,99.86424123)
\curveto(413.04611994,99.79423511)(412.72612026,99.68423522)(412.44611206,99.53424123)
\curveto(412.15612083,99.39423551)(411.92112106,99.1992357)(411.74111206,98.94924123)
\curveto(411.63112135,98.78923611)(411.55112143,98.60923629)(411.50111206,98.40924123)
\curveto(411.44112154,98.20923669)(411.41112157,97.96423694)(411.41111206,97.67424123)
\curveto(411.43112155,97.65423725)(411.44112154,97.61923728)(411.44111206,97.56924123)
\curveto(411.43112155,97.51923738)(411.43112155,97.47923742)(411.44111206,97.44924123)
\curveto(411.46112152,97.36923753)(411.4811215,97.29423761)(411.50111206,97.22424123)
\curveto(411.51112147,97.16423774)(411.53112145,97.0992378)(411.56111206,97.02924123)
\curveto(411.6811213,96.75923814)(411.85112113,96.53923836)(412.07111206,96.36924123)
\curveto(412.2811207,96.20923869)(412.52612046,96.07423883)(412.80611206,95.96424123)
\curveto(412.91612007,95.91423899)(413.03611995,95.87423903)(413.16611206,95.84424123)
\curveto(413.2861197,95.82423908)(413.41111957,95.7992391)(413.54111206,95.76924123)
\curveto(413.59111939,95.74923915)(413.64611934,95.73923916)(413.70611206,95.73924123)
\curveto(413.75611923,95.73923916)(413.80611918,95.73423917)(413.85611206,95.72424123)
\curveto(413.94611904,95.71423919)(414.04111894,95.7042392)(414.14111206,95.69424123)
\curveto(414.23111875,95.68423922)(414.32611866,95.67423923)(414.42611206,95.66424123)
\curveto(414.50611848,95.66423924)(414.59111839,95.65923924)(414.68111206,95.64924123)
\lineto(414.92111206,95.64924123)
\lineto(415.10111206,95.64924123)
\curveto(415.13111785,95.63923926)(415.16611782,95.63423927)(415.20611206,95.63424123)
\lineto(415.34111206,95.63424123)
\lineto(415.79111206,95.63424123)
\curveto(415.87111711,95.63423927)(415.95611703,95.62923927)(416.04611206,95.61924123)
\curveto(416.12611686,95.61923928)(416.20111678,95.62923927)(416.27111206,95.64924123)
\lineto(416.54111206,95.64924123)
\curveto(416.56111642,95.64923925)(416.59111639,95.64423926)(416.63111206,95.63424123)
\curveto(416.66111632,95.63423927)(416.6861163,95.63923926)(416.70611206,95.64924123)
\curveto(416.80611618,95.65923924)(416.90611608,95.66423924)(417.00611206,95.66424123)
\curveto(417.09611589,95.67423923)(417.19611579,95.68423922)(417.30611206,95.69424123)
\curveto(417.42611556,95.72423918)(417.55111543,95.73923916)(417.68111206,95.73924123)
\curveto(417.80111518,95.74923915)(417.91611507,95.77423913)(418.02611206,95.81424123)
\curveto(418.32611466,95.89423901)(418.59111439,95.97923892)(418.82111206,96.06924123)
\curveto(419.05111393,96.16923873)(419.26611372,96.31423859)(419.46611206,96.50424123)
\curveto(419.66611332,96.71423819)(419.81611317,96.97923792)(419.91611206,97.29924123)
\curveto(419.93611305,97.33923756)(419.94611304,97.37423753)(419.94611206,97.40424123)
\curveto(419.93611305,97.44423746)(419.94111304,97.48923741)(419.96111206,97.53924123)
\curveto(419.97111301,97.57923732)(419.981113,97.64923725)(419.99111206,97.74924123)
\curveto(420.00111298,97.85923704)(419.99611299,97.94423696)(419.97611206,98.00424123)
\curveto(419.95611303,98.07423683)(419.94611304,98.14423676)(419.94611206,98.21424123)
\curveto(419.93611305,98.28423662)(419.92111306,98.34923655)(419.90111206,98.40924123)
\curveto(419.84111314,98.60923629)(419.75611323,98.78923611)(419.64611206,98.94924123)
\curveto(419.62611336,98.97923592)(419.60611338,99.0042359)(419.58611206,99.02424123)
\lineto(419.52611206,99.08424123)
\curveto(419.50611348,99.12423578)(419.46611352,99.17423573)(419.40611206,99.23424123)
\curveto(419.26611372,99.33423557)(419.13611385,99.41923548)(419.01611206,99.48924123)
\curveto(418.89611409,99.55923534)(418.75111423,99.62923527)(418.58111206,99.69924123)
\curveto(418.51111447,99.72923517)(418.44111454,99.74923515)(418.37111206,99.75924123)
\curveto(418.30111468,99.77923512)(418.22611476,99.7992351)(418.14611206,99.81924123)
}
}
{
\newrgbcolor{curcolor}{0 0 0}
\pscustom[linestyle=none,fillstyle=solid,fillcolor=curcolor]
{
\newpath
\moveto(410.30111206,106.77385061)
\curveto(410.30112268,106.87384575)(410.31112267,106.96884566)(410.33111206,107.05885061)
\curveto(410.34112264,107.14884548)(410.37112261,107.21384541)(410.42111206,107.25385061)
\curveto(410.50112248,107.31384531)(410.60612238,107.34384528)(410.73611206,107.34385061)
\lineto(411.12611206,107.34385061)
\lineto(412.62611206,107.34385061)
\lineto(419.01611206,107.34385061)
\lineto(420.18611206,107.34385061)
\lineto(420.50111206,107.34385061)
\curveto(420.60111238,107.35384527)(420.6811123,107.33884529)(420.74111206,107.29885061)
\curveto(420.82111216,107.24884538)(420.87111211,107.17384545)(420.89111206,107.07385061)
\curveto(420.90111208,106.98384564)(420.90611208,106.87384575)(420.90611206,106.74385061)
\lineto(420.90611206,106.51885061)
\curveto(420.8861121,106.43884619)(420.87111211,106.36884626)(420.86111206,106.30885061)
\curveto(420.84111214,106.24884638)(420.80111218,106.19884643)(420.74111206,106.15885061)
\curveto(420.6811123,106.11884651)(420.60611238,106.09884653)(420.51611206,106.09885061)
\lineto(420.21611206,106.09885061)
\lineto(419.12111206,106.09885061)
\lineto(413.78111206,106.09885061)
\curveto(413.69111929,106.07884655)(413.61611937,106.06384656)(413.55611206,106.05385061)
\curveto(413.4861195,106.05384657)(413.42611956,106.0238466)(413.37611206,105.96385061)
\curveto(413.32611966,105.89384673)(413.30111968,105.80384682)(413.30111206,105.69385061)
\curveto(413.29111969,105.59384703)(413.2861197,105.48384714)(413.28611206,105.36385061)
\lineto(413.28611206,104.22385061)
\lineto(413.28611206,103.72885061)
\curveto(413.27611971,103.56884906)(413.21611977,103.45884917)(413.10611206,103.39885061)
\curveto(413.07611991,103.37884925)(413.04611994,103.36884926)(413.01611206,103.36885061)
\curveto(412.97612001,103.36884926)(412.93112005,103.36384926)(412.88111206,103.35385061)
\curveto(412.76112022,103.33384929)(412.65112033,103.33884929)(412.55111206,103.36885061)
\curveto(412.45112053,103.40884922)(412.3811206,103.46384916)(412.34111206,103.53385061)
\curveto(412.29112069,103.61384901)(412.26612072,103.73384889)(412.26611206,103.89385061)
\curveto(412.26612072,104.05384857)(412.25112073,104.18884844)(412.22111206,104.29885061)
\curveto(412.21112077,104.34884828)(412.20612078,104.40384822)(412.20611206,104.46385061)
\curveto(412.19612079,104.5238481)(412.1811208,104.58384804)(412.16111206,104.64385061)
\curveto(412.11112087,104.79384783)(412.06112092,104.93884769)(412.01111206,105.07885061)
\curveto(411.95112103,105.21884741)(411.8811211,105.35384727)(411.80111206,105.48385061)
\curveto(411.71112127,105.623847)(411.60612138,105.74384688)(411.48611206,105.84385061)
\curveto(411.36612162,105.94384668)(411.23612175,106.03884659)(411.09611206,106.12885061)
\curveto(410.99612199,106.18884644)(410.8861221,106.23384639)(410.76611206,106.26385061)
\curveto(410.64612234,106.30384632)(410.54112244,106.35384627)(410.45111206,106.41385061)
\curveto(410.39112259,106.46384616)(410.35112263,106.53384609)(410.33111206,106.62385061)
\curveto(410.32112266,106.64384598)(410.31612267,106.66884596)(410.31611206,106.69885061)
\curveto(410.31612267,106.7288459)(410.31112267,106.75384587)(410.30111206,106.77385061)
}
}
{
\newrgbcolor{curcolor}{0 0 0}
\pscustom[linestyle=none,fillstyle=solid,fillcolor=curcolor]
{
\newpath
\moveto(410.30111206,115.12345998)
\curveto(410.30112268,115.22345513)(410.31112267,115.31845503)(410.33111206,115.40845998)
\curveto(410.34112264,115.49845485)(410.37112261,115.56345479)(410.42111206,115.60345998)
\curveto(410.50112248,115.66345469)(410.60612238,115.69345466)(410.73611206,115.69345998)
\lineto(411.12611206,115.69345998)
\lineto(412.62611206,115.69345998)
\lineto(419.01611206,115.69345998)
\lineto(420.18611206,115.69345998)
\lineto(420.50111206,115.69345998)
\curveto(420.60111238,115.70345465)(420.6811123,115.68845466)(420.74111206,115.64845998)
\curveto(420.82111216,115.59845475)(420.87111211,115.52345483)(420.89111206,115.42345998)
\curveto(420.90111208,115.33345502)(420.90611208,115.22345513)(420.90611206,115.09345998)
\lineto(420.90611206,114.86845998)
\curveto(420.8861121,114.78845556)(420.87111211,114.71845563)(420.86111206,114.65845998)
\curveto(420.84111214,114.59845575)(420.80111218,114.5484558)(420.74111206,114.50845998)
\curveto(420.6811123,114.46845588)(420.60611238,114.4484559)(420.51611206,114.44845998)
\lineto(420.21611206,114.44845998)
\lineto(419.12111206,114.44845998)
\lineto(413.78111206,114.44845998)
\curveto(413.69111929,114.42845592)(413.61611937,114.41345594)(413.55611206,114.40345998)
\curveto(413.4861195,114.40345595)(413.42611956,114.37345598)(413.37611206,114.31345998)
\curveto(413.32611966,114.24345611)(413.30111968,114.1534562)(413.30111206,114.04345998)
\curveto(413.29111969,113.94345641)(413.2861197,113.83345652)(413.28611206,113.71345998)
\lineto(413.28611206,112.57345998)
\lineto(413.28611206,112.07845998)
\curveto(413.27611971,111.91845843)(413.21611977,111.80845854)(413.10611206,111.74845998)
\curveto(413.07611991,111.72845862)(413.04611994,111.71845863)(413.01611206,111.71845998)
\curveto(412.97612001,111.71845863)(412.93112005,111.71345864)(412.88111206,111.70345998)
\curveto(412.76112022,111.68345867)(412.65112033,111.68845866)(412.55111206,111.71845998)
\curveto(412.45112053,111.75845859)(412.3811206,111.81345854)(412.34111206,111.88345998)
\curveto(412.29112069,111.96345839)(412.26612072,112.08345827)(412.26611206,112.24345998)
\curveto(412.26612072,112.40345795)(412.25112073,112.53845781)(412.22111206,112.64845998)
\curveto(412.21112077,112.69845765)(412.20612078,112.7534576)(412.20611206,112.81345998)
\curveto(412.19612079,112.87345748)(412.1811208,112.93345742)(412.16111206,112.99345998)
\curveto(412.11112087,113.14345721)(412.06112092,113.28845706)(412.01111206,113.42845998)
\curveto(411.95112103,113.56845678)(411.8811211,113.70345665)(411.80111206,113.83345998)
\curveto(411.71112127,113.97345638)(411.60612138,114.09345626)(411.48611206,114.19345998)
\curveto(411.36612162,114.29345606)(411.23612175,114.38845596)(411.09611206,114.47845998)
\curveto(410.99612199,114.53845581)(410.8861221,114.58345577)(410.76611206,114.61345998)
\curveto(410.64612234,114.6534557)(410.54112244,114.70345565)(410.45111206,114.76345998)
\curveto(410.39112259,114.81345554)(410.35112263,114.88345547)(410.33111206,114.97345998)
\curveto(410.32112266,114.99345536)(410.31612267,115.01845533)(410.31611206,115.04845998)
\curveto(410.31612267,115.07845527)(410.31112267,115.10345525)(410.30111206,115.12345998)
}
}
{
\newrgbcolor{curcolor}{0 0 0}
\pscustom[linestyle=none,fillstyle=solid,fillcolor=curcolor]
{
\newpath
\moveto(441.0374585,42.02236623)
\curveto(441.08745924,42.04235669)(441.14745918,42.06735666)(441.2174585,42.09736623)
\curveto(441.28745904,42.1273566)(441.36245897,42.14735658)(441.4424585,42.15736623)
\curveto(441.51245882,42.17735655)(441.58245875,42.17735655)(441.6524585,42.15736623)
\curveto(441.71245862,42.14735658)(441.75745857,42.10735662)(441.7874585,42.03736623)
\curveto(441.80745852,41.98735674)(441.81745851,41.9273568)(441.8174585,41.85736623)
\lineto(441.8174585,41.64736623)
\lineto(441.8174585,41.19736623)
\curveto(441.81745851,41.04735768)(441.79245854,40.9273578)(441.7424585,40.83736623)
\curveto(441.68245865,40.73735799)(441.57745875,40.66235807)(441.4274585,40.61236623)
\curveto(441.27745905,40.57235816)(441.14245919,40.5273582)(441.0224585,40.47736623)
\curveto(440.76245957,40.36735836)(440.49245984,40.26735846)(440.2124585,40.17736623)
\curveto(439.9324604,40.08735864)(439.65746067,39.98735874)(439.3874585,39.87736623)
\curveto(439.29746103,39.84735888)(439.21246112,39.81735891)(439.1324585,39.78736623)
\curveto(439.05246128,39.76735896)(438.97746135,39.73735899)(438.9074585,39.69736623)
\curveto(438.83746149,39.66735906)(438.77746155,39.62235911)(438.7274585,39.56236623)
\curveto(438.67746165,39.50235923)(438.63746169,39.42235931)(438.6074585,39.32236623)
\curveto(438.58746174,39.27235946)(438.58246175,39.21235952)(438.5924585,39.14236623)
\lineto(438.5924585,38.94736623)
\lineto(438.5924585,36.11236623)
\lineto(438.5924585,35.81236623)
\curveto(438.58246175,35.70236303)(438.58246175,35.59736313)(438.5924585,35.49736623)
\curveto(438.60246173,35.39736333)(438.61746171,35.30236343)(438.6374585,35.21236623)
\curveto(438.65746167,35.1323636)(438.69746163,35.07236366)(438.7574585,35.03236623)
\curveto(438.85746147,34.95236378)(438.97246136,34.89236384)(439.1024585,34.85236623)
\curveto(439.22246111,34.82236391)(439.34746098,34.78236395)(439.4774585,34.73236623)
\curveto(439.70746062,34.6323641)(439.94746038,34.53736419)(440.1974585,34.44736623)
\curveto(440.44745988,34.36736436)(440.68745964,34.27736445)(440.9174585,34.17736623)
\curveto(440.97745935,34.15736457)(441.04745928,34.1323646)(441.1274585,34.10236623)
\curveto(441.19745913,34.08236465)(441.27245906,34.05736467)(441.3524585,34.02736623)
\curveto(441.4324589,33.99736473)(441.50745882,33.96236477)(441.5774585,33.92236623)
\curveto(441.63745869,33.89236484)(441.68245865,33.85736487)(441.7124585,33.81736623)
\curveto(441.77245856,33.73736499)(441.80745852,33.6273651)(441.8174585,33.48736623)
\lineto(441.8174585,33.06736623)
\lineto(441.8174585,32.82736623)
\curveto(441.80745852,32.75736597)(441.78245855,32.69736603)(441.7424585,32.64736623)
\curveto(441.71245862,32.59736613)(441.66745866,32.56736616)(441.6074585,32.55736623)
\curveto(441.54745878,32.55736617)(441.48745884,32.56236617)(441.4274585,32.57236623)
\curveto(441.35745897,32.59236614)(441.29245904,32.61236612)(441.2324585,32.63236623)
\curveto(441.16245917,32.66236607)(441.11245922,32.68736604)(441.0824585,32.70736623)
\curveto(440.76245957,32.84736588)(440.44745988,32.97236576)(440.1374585,33.08236623)
\curveto(439.81746051,33.19236554)(439.49746083,33.31236542)(439.1774585,33.44236623)
\curveto(438.95746137,33.5323652)(438.74246159,33.61736511)(438.5324585,33.69736623)
\curveto(438.31246202,33.77736495)(438.09246224,33.86236487)(437.8724585,33.95236623)
\curveto(437.15246318,34.25236448)(436.4274639,34.53736419)(435.6974585,34.80736623)
\curveto(434.95746537,35.07736365)(434.22246611,35.36236337)(433.4924585,35.66236623)
\curveto(433.2324671,35.77236296)(432.96746736,35.87236286)(432.6974585,35.96236623)
\curveto(432.4274679,36.06236267)(432.16246817,36.16736256)(431.9024585,36.27736623)
\curveto(431.79246854,36.3273624)(431.67246866,36.37236236)(431.5424585,36.41236623)
\curveto(431.40246893,36.46236227)(431.30246903,36.5323622)(431.2424585,36.62236623)
\curveto(431.20246913,36.66236207)(431.17246916,36.727362)(431.1524585,36.81736623)
\curveto(431.14246919,36.83736189)(431.14246919,36.85736187)(431.1524585,36.87736623)
\curveto(431.15246918,36.90736182)(431.14746918,36.9323618)(431.1374585,36.95236623)
\curveto(431.13746919,37.1323616)(431.13746919,37.34236139)(431.1374585,37.58236623)
\curveto(431.1274692,37.82236091)(431.16246917,37.99736073)(431.2424585,38.10736623)
\curveto(431.30246903,38.18736054)(431.40246893,38.24736048)(431.5424585,38.28736623)
\curveto(431.67246866,38.33736039)(431.79246854,38.38736034)(431.9024585,38.43736623)
\curveto(432.1324682,38.53736019)(432.36246797,38.6273601)(432.5924585,38.70736623)
\curveto(432.82246751,38.78735994)(433.05246728,38.87735985)(433.2824585,38.97736623)
\curveto(433.48246685,39.05735967)(433.68746664,39.1323596)(433.8974585,39.20236623)
\curveto(434.10746622,39.28235945)(434.31246602,39.36735936)(434.5124585,39.45736623)
\curveto(435.24246509,39.75735897)(435.98246435,40.04235869)(436.7324585,40.31236623)
\curveto(437.47246286,40.59235814)(438.20746212,40.88735784)(438.9374585,41.19736623)
\curveto(439.0274613,41.23735749)(439.11246122,41.26735746)(439.1924585,41.28736623)
\curveto(439.27246106,41.31735741)(439.35746097,41.34735738)(439.4474585,41.37736623)
\curveto(439.70746062,41.48735724)(439.97246036,41.59235714)(440.2424585,41.69236623)
\curveto(440.51245982,41.80235693)(440.77745955,41.91235682)(441.0374585,42.02236623)
\moveto(437.3924585,38.81236623)
\curveto(437.36246297,38.90235983)(437.31246302,38.95735977)(437.2424585,38.97736623)
\curveto(437.17246316,39.00735972)(437.09746323,39.01235972)(437.0174585,38.99236623)
\curveto(436.9274634,38.98235975)(436.84246349,38.95735977)(436.7624585,38.91736623)
\curveto(436.67246366,38.88735984)(436.59746373,38.85735987)(436.5374585,38.82736623)
\curveto(436.49746383,38.80735992)(436.46246387,38.79735993)(436.4324585,38.79736623)
\curveto(436.40246393,38.79735993)(436.36746396,38.78735994)(436.3274585,38.76736623)
\lineto(436.0874585,38.67736623)
\curveto(435.99746433,38.65736007)(435.90746442,38.6273601)(435.8174585,38.58736623)
\curveto(435.45746487,38.43736029)(435.09246524,38.30236043)(434.7224585,38.18236623)
\curveto(434.34246599,38.07236066)(433.97246636,37.94236079)(433.6124585,37.79236623)
\curveto(433.50246683,37.74236099)(433.39246694,37.69736103)(433.2824585,37.65736623)
\curveto(433.17246716,37.6273611)(433.06746726,37.58736114)(432.9674585,37.53736623)
\curveto(432.91746741,37.51736121)(432.87246746,37.49236124)(432.8324585,37.46236623)
\curveto(432.78246755,37.44236129)(432.75746757,37.39236134)(432.7574585,37.31236623)
\curveto(432.77746755,37.29236144)(432.79246754,37.27236146)(432.8024585,37.25236623)
\curveto(432.81246752,37.2323615)(432.8274675,37.21236152)(432.8474585,37.19236623)
\curveto(432.89746743,37.15236158)(432.95246738,37.12236161)(433.0124585,37.10236623)
\curveto(433.06246727,37.08236165)(433.11746721,37.06236167)(433.1774585,37.04236623)
\curveto(433.28746704,36.99236174)(433.39746693,36.95236178)(433.5074585,36.92236623)
\curveto(433.61746671,36.89236184)(433.7274666,36.85236188)(433.8374585,36.80236623)
\curveto(434.2274661,36.6323621)(434.62246571,36.48236225)(435.0224585,36.35236623)
\curveto(435.42246491,36.2323625)(435.81246452,36.09236264)(436.1924585,35.93236623)
\lineto(436.3424585,35.87236623)
\curveto(436.39246394,35.86236287)(436.44246389,35.84736288)(436.4924585,35.82736623)
\lineto(436.7324585,35.73736623)
\curveto(436.81246352,35.70736302)(436.89246344,35.68236305)(436.9724585,35.66236623)
\curveto(437.02246331,35.64236309)(437.07746325,35.6323631)(437.1374585,35.63236623)
\curveto(437.19746313,35.64236309)(437.24746308,35.65736307)(437.2874585,35.67736623)
\curveto(437.36746296,35.727363)(437.41246292,35.8323629)(437.4224585,35.99236623)
\lineto(437.4224585,36.44236623)
\lineto(437.4224585,38.04736623)
\curveto(437.42246291,38.15736057)(437.4274629,38.29236044)(437.4374585,38.45236623)
\curveto(437.43746289,38.61236012)(437.42246291,38.73236)(437.3924585,38.81236623)
}
}
{
\newrgbcolor{curcolor}{0 0 0}
\pscustom[linestyle=none,fillstyle=solid,fillcolor=curcolor]
{
\newpath
\moveto(437.7824585,50.56392873)
\curveto(437.8324625,50.57392038)(437.90246243,50.57892038)(437.9924585,50.57892873)
\curveto(438.07246226,50.57892038)(438.13746219,50.57392038)(438.1874585,50.56392873)
\curveto(438.2274621,50.56392039)(438.26746206,50.5589204)(438.3074585,50.54892873)
\lineto(438.4274585,50.54892873)
\curveto(438.50746182,50.52892043)(438.58746174,50.51892044)(438.6674585,50.51892873)
\curveto(438.74746158,50.51892044)(438.8274615,50.50892045)(438.9074585,50.48892873)
\curveto(438.94746138,50.47892048)(438.98746134,50.47392048)(439.0274585,50.47392873)
\curveto(439.05746127,50.47392048)(439.09246124,50.46892049)(439.1324585,50.45892873)
\curveto(439.24246109,50.42892053)(439.34746098,50.39892056)(439.4474585,50.36892873)
\curveto(439.54746078,50.34892061)(439.64746068,50.31892064)(439.7474585,50.27892873)
\curveto(440.09746023,50.13892082)(440.41245992,49.96892099)(440.6924585,49.76892873)
\curveto(440.97245936,49.56892139)(441.21245912,49.31892164)(441.4124585,49.01892873)
\curveto(441.51245882,48.86892209)(441.59745873,48.72392223)(441.6674585,48.58392873)
\curveto(441.71745861,48.47392248)(441.75745857,48.36392259)(441.7874585,48.25392873)
\curveto(441.81745851,48.1539228)(441.84745848,48.04892291)(441.8774585,47.93892873)
\curveto(441.89745843,47.86892309)(441.90745842,47.80392315)(441.9074585,47.74392873)
\curveto(441.91745841,47.68392327)(441.9324584,47.62392333)(441.9524585,47.56392873)
\lineto(441.9524585,47.41392873)
\curveto(441.97245836,47.36392359)(441.98245835,47.28892367)(441.9824585,47.18892873)
\curveto(441.99245834,47.08892387)(441.98745834,47.00892395)(441.9674585,46.94892873)
\lineto(441.9674585,46.79892873)
\curveto(441.95745837,46.7589242)(441.95245838,46.71392424)(441.9524585,46.66392873)
\curveto(441.95245838,46.62392433)(441.94745838,46.57892438)(441.9374585,46.52892873)
\curveto(441.89745843,46.37892458)(441.86245847,46.22892473)(441.8324585,46.07892873)
\curveto(441.80245853,45.93892502)(441.75745857,45.79892516)(441.6974585,45.65892873)
\curveto(441.61745871,45.4589255)(441.51745881,45.27892568)(441.3974585,45.11892873)
\lineto(441.2474585,44.93892873)
\curveto(441.18745914,44.87892608)(441.14745918,44.80892615)(441.1274585,44.72892873)
\curveto(441.11745921,44.66892629)(441.1324592,44.61892634)(441.1724585,44.57892873)
\curveto(441.20245913,44.54892641)(441.24745908,44.52392643)(441.3074585,44.50392873)
\curveto(441.36745896,44.49392646)(441.4324589,44.48392647)(441.5024585,44.47392873)
\curveto(441.56245877,44.47392648)(441.60745872,44.46392649)(441.6374585,44.44392873)
\curveto(441.68745864,44.40392655)(441.7324586,44.3589266)(441.7724585,44.30892873)
\curveto(441.79245854,44.2589267)(441.80745852,44.18892677)(441.8174585,44.09892873)
\lineto(441.8174585,43.82892873)
\curveto(441.81745851,43.73892722)(441.81245852,43.6539273)(441.8024585,43.57392873)
\curveto(441.78245855,43.49392746)(441.76245857,43.43392752)(441.7424585,43.39392873)
\curveto(441.72245861,43.37392758)(441.69745863,43.3539276)(441.6674585,43.33392873)
\lineto(441.5774585,43.27392873)
\curveto(441.49745883,43.24392771)(441.37745895,43.22892773)(441.2174585,43.22892873)
\curveto(441.05745927,43.23892772)(440.92245941,43.24392771)(440.8124585,43.24392873)
\lineto(432.0074585,43.24392873)
\curveto(431.88746844,43.24392771)(431.76246857,43.23892772)(431.6324585,43.22892873)
\curveto(431.49246884,43.22892773)(431.38246895,43.2539277)(431.3024585,43.30392873)
\curveto(431.24246909,43.34392761)(431.19246914,43.40892755)(431.1524585,43.49892873)
\curveto(431.15246918,43.51892744)(431.15246918,43.54392741)(431.1524585,43.57392873)
\curveto(431.14246919,43.60392735)(431.13746919,43.62892733)(431.1374585,43.64892873)
\curveto(431.1274692,43.78892717)(431.1274692,43.93392702)(431.1374585,44.08392873)
\curveto(431.13746919,44.24392671)(431.17746915,44.3539266)(431.2574585,44.41392873)
\curveto(431.33746899,44.46392649)(431.45246888,44.48892647)(431.6024585,44.48892873)
\lineto(432.0074585,44.48892873)
\lineto(433.7624585,44.48892873)
\lineto(434.0174585,44.48892873)
\lineto(434.3024585,44.48892873)
\curveto(434.39246594,44.49892646)(434.47746585,44.50892645)(434.5574585,44.51892873)
\curveto(434.6274657,44.53892642)(434.67746565,44.56892639)(434.7074585,44.60892873)
\curveto(434.73746559,44.64892631)(434.74246559,44.69392626)(434.7224585,44.74392873)
\curveto(434.70246563,44.79392616)(434.68246565,44.83392612)(434.6624585,44.86392873)
\curveto(434.62246571,44.91392604)(434.58246575,44.958926)(434.5424585,44.99892873)
\lineto(434.4224585,45.14892873)
\curveto(434.37246596,45.21892574)(434.327466,45.28892567)(434.2874585,45.35892873)
\lineto(434.1674585,45.59892873)
\curveto(434.07746625,45.77892518)(434.01246632,45.99392496)(433.9724585,46.24392873)
\curveto(433.9324664,46.49392446)(433.91246642,46.74892421)(433.9124585,47.00892873)
\curveto(433.91246642,47.26892369)(433.93746639,47.52392343)(433.9874585,47.77392873)
\curveto(434.0274663,48.02392293)(434.08746624,48.24392271)(434.1674585,48.43392873)
\curveto(434.33746599,48.83392212)(434.57246576,49.17892178)(434.8724585,49.46892873)
\curveto(435.17246516,49.7589212)(435.52246481,49.98892097)(435.9224585,50.15892873)
\curveto(436.0324643,50.20892075)(436.14246419,50.24892071)(436.2524585,50.27892873)
\curveto(436.35246398,50.31892064)(436.45746387,50.3589206)(436.5674585,50.39892873)
\curveto(436.67746365,50.42892053)(436.79246354,50.44892051)(436.9124585,50.45892873)
\lineto(437.2424585,50.51892873)
\curveto(437.27246306,50.52892043)(437.30746302,50.53392042)(437.3474585,50.53392873)
\curveto(437.37746295,50.53392042)(437.40746292,50.53892042)(437.4374585,50.54892873)
\curveto(437.49746283,50.56892039)(437.55746277,50.56892039)(437.6174585,50.54892873)
\curveto(437.66746266,50.53892042)(437.72246261,50.54392041)(437.7824585,50.56392873)
\moveto(438.1724585,49.22892873)
\curveto(438.12246221,49.24892171)(438.06246227,49.2539217)(437.9924585,49.24392873)
\curveto(437.92246241,49.23392172)(437.85746247,49.22892173)(437.7974585,49.22892873)
\curveto(437.6274627,49.22892173)(437.46746286,49.21892174)(437.3174585,49.19892873)
\curveto(437.16746316,49.18892177)(437.0324633,49.1589218)(436.9124585,49.10892873)
\curveto(436.81246352,49.07892188)(436.72246361,49.0539219)(436.6424585,49.03392873)
\curveto(436.56246377,49.01392194)(436.48246385,48.98392197)(436.4024585,48.94392873)
\curveto(436.15246418,48.83392212)(435.92246441,48.68392227)(435.7124585,48.49392873)
\curveto(435.49246484,48.30392265)(435.327465,48.08392287)(435.2174585,47.83392873)
\curveto(435.18746514,47.7539232)(435.16246517,47.67392328)(435.1424585,47.59392873)
\curveto(435.11246522,47.52392343)(435.08746524,47.44892351)(435.0674585,47.36892873)
\curveto(435.03746529,47.2589237)(435.02246531,47.14892381)(435.0224585,47.03892873)
\curveto(435.01246532,46.92892403)(435.00746532,46.80892415)(435.0074585,46.67892873)
\curveto(435.01746531,46.62892433)(435.0274653,46.58392437)(435.0374585,46.54392873)
\lineto(435.0374585,46.40892873)
\lineto(435.0974585,46.13892873)
\curveto(435.11746521,46.0589249)(435.14746518,45.97892498)(435.1874585,45.89892873)
\curveto(435.327465,45.5589254)(435.53746479,45.28892567)(435.8174585,45.08892873)
\curveto(436.08746424,44.88892607)(436.40746392,44.72892623)(436.7774585,44.60892873)
\curveto(436.88746344,44.56892639)(436.99746333,44.54392641)(437.1074585,44.53392873)
\curveto(437.21746311,44.52392643)(437.332463,44.50392645)(437.4524585,44.47392873)
\curveto(437.50246283,44.46392649)(437.54746278,44.46392649)(437.5874585,44.47392873)
\curveto(437.6274627,44.48392647)(437.67246266,44.47892648)(437.7224585,44.45892873)
\curveto(437.77246256,44.44892651)(437.84746248,44.44392651)(437.9474585,44.44392873)
\curveto(438.03746229,44.44392651)(438.10746222,44.44892651)(438.1574585,44.45892873)
\lineto(438.2774585,44.45892873)
\curveto(438.31746201,44.46892649)(438.35746197,44.47392648)(438.3974585,44.47392873)
\curveto(438.43746189,44.47392648)(438.47246186,44.47892648)(438.5024585,44.48892873)
\curveto(438.5324618,44.49892646)(438.56746176,44.50392645)(438.6074585,44.50392873)
\curveto(438.63746169,44.50392645)(438.66746166,44.50892645)(438.6974585,44.51892873)
\curveto(438.77746155,44.53892642)(438.85746147,44.5539264)(438.9374585,44.56392873)
\lineto(439.1774585,44.62392873)
\curveto(439.51746081,44.73392622)(439.80746052,44.88392607)(440.0474585,45.07392873)
\curveto(440.28746004,45.27392568)(440.48745984,45.51892544)(440.6474585,45.80892873)
\curveto(440.69745963,45.89892506)(440.73745959,45.99392496)(440.7674585,46.09392873)
\curveto(440.78745954,46.19392476)(440.81245952,46.29892466)(440.8424585,46.40892873)
\curveto(440.86245947,46.4589245)(440.87245946,46.50392445)(440.8724585,46.54392873)
\curveto(440.86245947,46.59392436)(440.86245947,46.64392431)(440.8724585,46.69392873)
\curveto(440.88245945,46.73392422)(440.88745944,46.77892418)(440.8874585,46.82892873)
\lineto(440.8874585,46.96392873)
\lineto(440.8874585,47.09892873)
\curveto(440.87745945,47.13892382)(440.87245946,47.17392378)(440.8724585,47.20392873)
\curveto(440.87245946,47.23392372)(440.86745946,47.26892369)(440.8574585,47.30892873)
\curveto(440.83745949,47.38892357)(440.82245951,47.46392349)(440.8124585,47.53392873)
\curveto(440.79245954,47.60392335)(440.76745956,47.67892328)(440.7374585,47.75892873)
\curveto(440.60745972,48.06892289)(440.43745989,48.31892264)(440.2274585,48.50892873)
\curveto(440.00746032,48.69892226)(439.74246059,48.8589221)(439.4324585,48.98892873)
\curveto(439.29246104,49.03892192)(439.15246118,49.07392188)(439.0124585,49.09392873)
\curveto(438.86246147,49.12392183)(438.71246162,49.1589218)(438.5624585,49.19892873)
\curveto(438.51246182,49.21892174)(438.46746186,49.22392173)(438.4274585,49.21392873)
\curveto(438.37746195,49.21392174)(438.327462,49.21892174)(438.2774585,49.22892873)
\lineto(438.1724585,49.22892873)
}
}
{
\newrgbcolor{curcolor}{0 0 0}
\pscustom[linestyle=none,fillstyle=solid,fillcolor=curcolor]
{
\newpath
\moveto(433.9124585,55.69017873)
\curveto(433.91246642,55.92017394)(433.97246636,56.05017381)(434.0924585,56.08017873)
\curveto(434.20246613,56.11017375)(434.36746596,56.12517374)(434.5874585,56.12517873)
\lineto(434.8724585,56.12517873)
\curveto(434.96246537,56.12517374)(435.03746529,56.10017376)(435.0974585,56.05017873)
\curveto(435.17746515,55.99017387)(435.22246511,55.90517396)(435.2324585,55.79517873)
\curveto(435.2324651,55.68517418)(435.24746508,55.57517429)(435.2774585,55.46517873)
\curveto(435.30746502,55.32517454)(435.33746499,55.19017467)(435.3674585,55.06017873)
\curveto(435.39746493,54.94017492)(435.43746489,54.82517504)(435.4874585,54.71517873)
\curveto(435.61746471,54.42517544)(435.79746453,54.19017567)(436.0274585,54.01017873)
\curveto(436.24746408,53.83017603)(436.50246383,53.67517619)(436.7924585,53.54517873)
\curveto(436.90246343,53.50517636)(437.01746331,53.47517639)(437.1374585,53.45517873)
\curveto(437.24746308,53.43517643)(437.36246297,53.41017645)(437.4824585,53.38017873)
\curveto(437.5324628,53.37017649)(437.58246275,53.3651765)(437.6324585,53.36517873)
\curveto(437.68246265,53.37517649)(437.7324626,53.37517649)(437.7824585,53.36517873)
\curveto(437.90246243,53.33517653)(438.04246229,53.32017654)(438.2024585,53.32017873)
\curveto(438.35246198,53.33017653)(438.49746183,53.33517653)(438.6374585,53.33517873)
\lineto(440.4824585,53.33517873)
\lineto(440.8274585,53.33517873)
\curveto(440.94745938,53.33517653)(441.06245927,53.33017653)(441.1724585,53.32017873)
\curveto(441.28245905,53.31017655)(441.37745895,53.30517656)(441.4574585,53.30517873)
\curveto(441.53745879,53.31517655)(441.60745872,53.29517657)(441.6674585,53.24517873)
\curveto(441.73745859,53.19517667)(441.77745855,53.11517675)(441.7874585,53.00517873)
\curveto(441.79745853,52.90517696)(441.80245853,52.79517707)(441.8024585,52.67517873)
\lineto(441.8024585,52.40517873)
\curveto(441.78245855,52.35517751)(441.76745856,52.30517756)(441.7574585,52.25517873)
\curveto(441.73745859,52.21517765)(441.71245862,52.18517768)(441.6824585,52.16517873)
\curveto(441.61245872,52.11517775)(441.5274588,52.08517778)(441.4274585,52.07517873)
\lineto(441.0974585,52.07517873)
\lineto(439.9424585,52.07517873)
\lineto(435.7874585,52.07517873)
\lineto(434.7524585,52.07517873)
\lineto(434.4524585,52.07517873)
\curveto(434.35246598,52.08517778)(434.26746606,52.11517775)(434.1974585,52.16517873)
\curveto(434.15746617,52.19517767)(434.1274662,52.24517762)(434.1074585,52.31517873)
\curveto(434.08746624,52.39517747)(434.07746625,52.48017738)(434.0774585,52.57017873)
\curveto(434.06746626,52.6601772)(434.06746626,52.75017711)(434.0774585,52.84017873)
\curveto(434.08746624,52.93017693)(434.10246623,53.00017686)(434.1224585,53.05017873)
\curveto(434.15246618,53.13017673)(434.21246612,53.18017668)(434.3024585,53.20017873)
\curveto(434.38246595,53.23017663)(434.47246586,53.24517662)(434.5724585,53.24517873)
\lineto(434.8724585,53.24517873)
\curveto(434.97246536,53.24517662)(435.06246527,53.2651766)(435.1424585,53.30517873)
\curveto(435.16246517,53.31517655)(435.17746515,53.32517654)(435.1874585,53.33517873)
\lineto(435.2324585,53.38017873)
\curveto(435.2324651,53.49017637)(435.18746514,53.58017628)(435.0974585,53.65017873)
\curveto(434.99746533,53.72017614)(434.91746541,53.78017608)(434.8574585,53.83017873)
\lineto(434.7674585,53.92017873)
\curveto(434.65746567,54.01017585)(434.54246579,54.13517573)(434.4224585,54.29517873)
\curveto(434.30246603,54.45517541)(434.21246612,54.60517526)(434.1524585,54.74517873)
\curveto(434.10246623,54.83517503)(434.06746626,54.93017493)(434.0474585,55.03017873)
\curveto(434.01746631,55.13017473)(433.98746634,55.23517463)(433.9574585,55.34517873)
\curveto(433.94746638,55.40517446)(433.94246639,55.4651744)(433.9424585,55.52517873)
\curveto(433.9324664,55.58517428)(433.92246641,55.64017422)(433.9124585,55.69017873)
}
}
{
\newrgbcolor{curcolor}{0 0 0}
\pscustom[linestyle=none,fillstyle=solid,fillcolor=curcolor]
{
}
}
{
\newrgbcolor{curcolor}{0 0 0}
\pscustom[linestyle=none,fillstyle=solid,fillcolor=curcolor]
{
\newpath
\moveto(431.2124585,64.24510061)
\curveto(431.20246913,64.93509597)(431.32246901,65.53509537)(431.5724585,66.04510061)
\curveto(431.82246851,66.56509434)(432.15746817,66.96009395)(432.5774585,67.23010061)
\curveto(432.65746767,67.28009363)(432.74746758,67.32509358)(432.8474585,67.36510061)
\curveto(432.93746739,67.4050935)(433.0324673,67.45009346)(433.1324585,67.50010061)
\curveto(433.2324671,67.54009337)(433.332467,67.57009334)(433.4324585,67.59010061)
\curveto(433.5324668,67.6100933)(433.63746669,67.63009328)(433.7474585,67.65010061)
\curveto(433.79746653,67.67009324)(433.84246649,67.67509323)(433.8824585,67.66510061)
\curveto(433.92246641,67.65509325)(433.96746636,67.66009325)(434.0174585,67.68010061)
\curveto(434.06746626,67.69009322)(434.15246618,67.69509321)(434.2724585,67.69510061)
\curveto(434.38246595,67.69509321)(434.46746586,67.69009322)(434.5274585,67.68010061)
\curveto(434.58746574,67.66009325)(434.64746568,67.65009326)(434.7074585,67.65010061)
\curveto(434.76746556,67.66009325)(434.8274655,67.65509325)(434.8874585,67.63510061)
\curveto(435.0274653,67.59509331)(435.16246517,67.56009335)(435.2924585,67.53010061)
\curveto(435.42246491,67.50009341)(435.54746478,67.46009345)(435.6674585,67.41010061)
\curveto(435.80746452,67.35009356)(435.9324644,67.28009363)(436.0424585,67.20010061)
\curveto(436.15246418,67.13009378)(436.26246407,67.05509385)(436.3724585,66.97510061)
\lineto(436.4324585,66.91510061)
\curveto(436.45246388,66.905094)(436.47246386,66.89009402)(436.4924585,66.87010061)
\curveto(436.65246368,66.75009416)(436.79746353,66.61509429)(436.9274585,66.46510061)
\curveto(437.05746327,66.31509459)(437.18246315,66.15509475)(437.3024585,65.98510061)
\curveto(437.52246281,65.67509523)(437.7274626,65.38009553)(437.9174585,65.10010061)
\curveto(438.05746227,64.87009604)(438.19246214,64.64009627)(438.3224585,64.41010061)
\curveto(438.45246188,64.19009672)(438.58746174,63.97009694)(438.7274585,63.75010061)
\curveto(438.89746143,63.50009741)(439.07746125,63.26009765)(439.2674585,63.03010061)
\curveto(439.45746087,62.8100981)(439.68246065,62.62009829)(439.9424585,62.46010061)
\curveto(440.00246033,62.42009849)(440.06246027,62.38509852)(440.1224585,62.35510061)
\curveto(440.17246016,62.32509858)(440.23746009,62.29509861)(440.3174585,62.26510061)
\curveto(440.38745994,62.24509866)(440.44745988,62.24009867)(440.4974585,62.25010061)
\curveto(440.56745976,62.27009864)(440.62245971,62.3050986)(440.6624585,62.35510061)
\curveto(440.69245964,62.4050985)(440.71245962,62.46509844)(440.7224585,62.53510061)
\lineto(440.7224585,62.77510061)
\lineto(440.7224585,63.52510061)
\lineto(440.7224585,66.33010061)
\lineto(440.7224585,66.99010061)
\curveto(440.72245961,67.08009383)(440.7274596,67.16509374)(440.7374585,67.24510061)
\curveto(440.73745959,67.32509358)(440.75745957,67.39009352)(440.7974585,67.44010061)
\curveto(440.83745949,67.49009342)(440.91245942,67.53009338)(441.0224585,67.56010061)
\curveto(441.12245921,67.60009331)(441.22245911,67.6100933)(441.3224585,67.59010061)
\lineto(441.4574585,67.59010061)
\curveto(441.5274588,67.57009334)(441.58745874,67.55009336)(441.6374585,67.53010061)
\curveto(441.68745864,67.5100934)(441.7274586,67.47509343)(441.7574585,67.42510061)
\curveto(441.79745853,67.37509353)(441.81745851,67.3050936)(441.8174585,67.21510061)
\lineto(441.8174585,66.94510061)
\lineto(441.8174585,66.04510061)
\lineto(441.8174585,62.53510061)
\lineto(441.8174585,61.47010061)
\curveto(441.81745851,61.39009952)(441.82245851,61.30009961)(441.8324585,61.20010061)
\curveto(441.8324585,61.10009981)(441.82245851,61.01509989)(441.8024585,60.94510061)
\curveto(441.7324586,60.73510017)(441.55245878,60.67010024)(441.2624585,60.75010061)
\curveto(441.22245911,60.76010015)(441.18745914,60.76010015)(441.1574585,60.75010061)
\curveto(441.11745921,60.75010016)(441.07245926,60.76010015)(441.0224585,60.78010061)
\curveto(440.94245939,60.80010011)(440.85745947,60.82010009)(440.7674585,60.84010061)
\curveto(440.67745965,60.86010005)(440.59245974,60.88510002)(440.5124585,60.91510061)
\curveto(440.02246031,61.07509983)(439.60746072,61.27509963)(439.2674585,61.51510061)
\curveto(439.01746131,61.69509921)(438.79246154,61.90009901)(438.5924585,62.13010061)
\curveto(438.38246195,62.36009855)(438.18746214,62.60009831)(438.0074585,62.85010061)
\curveto(437.8274625,63.1100978)(437.65746267,63.37509753)(437.4974585,63.64510061)
\curveto(437.327463,63.92509698)(437.15246318,64.19509671)(436.9724585,64.45510061)
\curveto(436.89246344,64.56509634)(436.81746351,64.67009624)(436.7474585,64.77010061)
\curveto(436.67746365,64.88009603)(436.60246373,64.99009592)(436.5224585,65.10010061)
\curveto(436.49246384,65.14009577)(436.46246387,65.17509573)(436.4324585,65.20510061)
\curveto(436.39246394,65.24509566)(436.36246397,65.28509562)(436.3424585,65.32510061)
\curveto(436.2324641,65.46509544)(436.10746422,65.59009532)(435.9674585,65.70010061)
\curveto(435.93746439,65.72009519)(435.91246442,65.74509516)(435.8924585,65.77510061)
\curveto(435.86246447,65.8050951)(435.8324645,65.83009508)(435.8024585,65.85010061)
\curveto(435.70246463,65.93009498)(435.60246473,65.99509491)(435.5024585,66.04510061)
\curveto(435.40246493,66.1050948)(435.29246504,66.16009475)(435.1724585,66.21010061)
\curveto(435.10246523,66.24009467)(435.0274653,66.26009465)(434.9474585,66.27010061)
\lineto(434.7074585,66.33010061)
\lineto(434.6174585,66.33010061)
\curveto(434.58746574,66.34009457)(434.55746577,66.34509456)(434.5274585,66.34510061)
\curveto(434.45746587,66.36509454)(434.36246597,66.37009454)(434.2424585,66.36010061)
\curveto(434.11246622,66.36009455)(434.01246632,66.35009456)(433.9424585,66.33010061)
\curveto(433.86246647,66.3100946)(433.78746654,66.29009462)(433.7174585,66.27010061)
\curveto(433.63746669,66.26009465)(433.55746677,66.24009467)(433.4774585,66.21010061)
\curveto(433.23746709,66.10009481)(433.03746729,65.95009496)(432.8774585,65.76010061)
\curveto(432.70746762,65.58009533)(432.56746776,65.36009555)(432.4574585,65.10010061)
\curveto(432.43746789,65.03009588)(432.42246791,64.96009595)(432.4124585,64.89010061)
\curveto(432.39246794,64.82009609)(432.37246796,64.74509616)(432.3524585,64.66510061)
\curveto(432.332468,64.58509632)(432.32246801,64.47509643)(432.3224585,64.33510061)
\curveto(432.32246801,64.2050967)(432.332468,64.10009681)(432.3524585,64.02010061)
\curveto(432.36246797,63.96009695)(432.36746796,63.905097)(432.3674585,63.85510061)
\curveto(432.36746796,63.8050971)(432.37746795,63.75509715)(432.3974585,63.70510061)
\curveto(432.43746789,63.6050973)(432.47746785,63.5100974)(432.5174585,63.42010061)
\curveto(432.55746777,63.34009757)(432.60246773,63.26009765)(432.6524585,63.18010061)
\curveto(432.67246766,63.15009776)(432.69746763,63.12009779)(432.7274585,63.09010061)
\curveto(432.75746757,63.07009784)(432.78246755,63.04509786)(432.8024585,63.01510061)
\lineto(432.8774585,62.94010061)
\curveto(432.89746743,62.910098)(432.91746741,62.88509802)(432.9374585,62.86510061)
\lineto(433.1474585,62.71510061)
\curveto(433.20746712,62.67509823)(433.27246706,62.63009828)(433.3424585,62.58010061)
\curveto(433.4324669,62.52009839)(433.53746679,62.47009844)(433.6574585,62.43010061)
\curveto(433.76746656,62.40009851)(433.87746645,62.36509854)(433.9874585,62.32510061)
\curveto(434.09746623,62.28509862)(434.24246609,62.26009865)(434.4224585,62.25010061)
\curveto(434.59246574,62.24009867)(434.71746561,62.2100987)(434.7974585,62.16010061)
\curveto(434.87746545,62.1100988)(434.92246541,62.03509887)(434.9324585,61.93510061)
\curveto(434.94246539,61.83509907)(434.94746538,61.72509918)(434.9474585,61.60510061)
\curveto(434.94746538,61.56509934)(434.95246538,61.52509938)(434.9624585,61.48510061)
\curveto(434.96246537,61.44509946)(434.95746537,61.4100995)(434.9474585,61.38010061)
\curveto(434.9274654,61.33009958)(434.91746541,61.28009963)(434.9174585,61.23010061)
\curveto(434.91746541,61.19009972)(434.90746542,61.15009976)(434.8874585,61.11010061)
\curveto(434.8274655,61.02009989)(434.69246564,60.97509993)(434.4824585,60.97510061)
\lineto(434.3624585,60.97510061)
\curveto(434.30246603,60.98509992)(434.24246609,60.99009992)(434.1824585,60.99010061)
\curveto(434.11246622,61.00009991)(434.04746628,61.0100999)(433.9874585,61.02010061)
\curveto(433.87746645,61.04009987)(433.77746655,61.06009985)(433.6874585,61.08010061)
\curveto(433.58746674,61.10009981)(433.49246684,61.13009978)(433.4024585,61.17010061)
\curveto(433.332467,61.19009972)(433.27246706,61.2100997)(433.2224585,61.23010061)
\lineto(433.0424585,61.29010061)
\curveto(432.78246755,61.4100995)(432.53746779,61.56509934)(432.3074585,61.75510061)
\curveto(432.07746825,61.95509895)(431.89246844,62.17009874)(431.7524585,62.40010061)
\curveto(431.67246866,62.5100984)(431.60746872,62.62509828)(431.5574585,62.74510061)
\lineto(431.4074585,63.13510061)
\curveto(431.35746897,63.24509766)(431.327469,63.36009755)(431.3174585,63.48010061)
\curveto(431.29746903,63.60009731)(431.27246906,63.72509718)(431.2424585,63.85510061)
\curveto(431.24246909,63.92509698)(431.24246909,63.99009692)(431.2424585,64.05010061)
\curveto(431.2324691,64.1100968)(431.22246911,64.17509673)(431.2124585,64.24510061)
}
}
{
\newrgbcolor{curcolor}{0 0 0}
\pscustom[linestyle=none,fillstyle=solid,fillcolor=curcolor]
{
\newpath
\moveto(431.2124585,72.45970998)
\curveto(431.18246915,74.08970454)(431.73746859,75.13970349)(432.8774585,75.60970998)
\curveto(433.10746722,75.70970292)(433.39746693,75.77470286)(433.7474585,75.80470998)
\curveto(434.08746624,75.84470279)(434.39746593,75.81970281)(434.6774585,75.72970998)
\curveto(434.93746539,75.63970299)(435.16246517,75.51970311)(435.3524585,75.36970998)
\curveto(435.39246494,75.34970328)(435.4274649,75.32470331)(435.4574585,75.29470998)
\curveto(435.47746485,75.26470337)(435.50246483,75.23970339)(435.5324585,75.21970998)
\lineto(435.6524585,75.12970998)
\curveto(435.68246465,75.09970353)(435.70746462,75.06470357)(435.7274585,75.02470998)
\curveto(435.77746455,74.97470366)(435.82246451,74.91970371)(435.8624585,74.85970998)
\curveto(435.90246443,74.80970382)(435.95246438,74.76470387)(436.0124585,74.72470998)
\curveto(436.05246428,74.68470395)(436.10246423,74.66970396)(436.1624585,74.67970998)
\curveto(436.21246412,74.68970394)(436.25746407,74.71970391)(436.2974585,74.76970998)
\curveto(436.33746399,74.81970381)(436.37746395,74.87470376)(436.4174585,74.93470998)
\curveto(436.44746388,75.00470363)(436.47746385,75.06970356)(436.5074585,75.12970998)
\curveto(436.53746379,75.18970344)(436.56746376,75.23970339)(436.5974585,75.27970998)
\curveto(436.81746351,75.59970303)(437.1274632,75.85470278)(437.5274585,76.04470998)
\curveto(437.61746271,76.08470255)(437.71246262,76.11470252)(437.8124585,76.13470998)
\curveto(437.90246243,76.16470247)(437.99246234,76.18970244)(438.0824585,76.20970998)
\curveto(438.1324622,76.21970241)(438.18246215,76.22470241)(438.2324585,76.22470998)
\curveto(438.27246206,76.2347024)(438.31746201,76.24470239)(438.3674585,76.25470998)
\curveto(438.41746191,76.26470237)(438.46746186,76.26470237)(438.5174585,76.25470998)
\curveto(438.56746176,76.24470239)(438.61746171,76.24970238)(438.6674585,76.26970998)
\curveto(438.71746161,76.27970235)(438.77746155,76.28470235)(438.8474585,76.28470998)
\curveto(438.91746141,76.28470235)(438.97746135,76.27470236)(439.0274585,76.25470998)
\lineto(439.2524585,76.25470998)
\lineto(439.4924585,76.19470998)
\curveto(439.56246077,76.18470245)(439.6324607,76.16970246)(439.7024585,76.14970998)
\curveto(439.79246054,76.11970251)(439.87746045,76.08970254)(439.9574585,76.05970998)
\curveto(440.03746029,76.03970259)(440.11746021,76.00970262)(440.1974585,75.96970998)
\curveto(440.25746007,75.94970268)(440.31746001,75.91970271)(440.3774585,75.87970998)
\curveto(440.4274599,75.84970278)(440.47745985,75.81470282)(440.5274585,75.77470998)
\curveto(440.83745949,75.57470306)(441.09745923,75.32470331)(441.3074585,75.02470998)
\curveto(441.50745882,74.72470391)(441.67245866,74.37970425)(441.8024585,73.98970998)
\curveto(441.84245849,73.86970476)(441.86745846,73.73970489)(441.8774585,73.59970998)
\curveto(441.89745843,73.46970516)(441.92245841,73.3347053)(441.9524585,73.19470998)
\curveto(441.96245837,73.12470551)(441.96745836,73.05470558)(441.9674585,72.98470998)
\curveto(441.96745836,72.92470571)(441.97245836,72.85970577)(441.9824585,72.78970998)
\curveto(441.99245834,72.74970588)(441.99745833,72.68970594)(441.9974585,72.60970998)
\curveto(441.99745833,72.53970609)(441.99245834,72.48970614)(441.9824585,72.45970998)
\curveto(441.97245836,72.40970622)(441.96745836,72.36470627)(441.9674585,72.32470998)
\lineto(441.9674585,72.20470998)
\curveto(441.94745838,72.10470653)(441.9324584,72.00470663)(441.9224585,71.90470998)
\curveto(441.91245842,71.80470683)(441.89745843,71.70970692)(441.8774585,71.61970998)
\curveto(441.84745848,71.50970712)(441.82245851,71.39970723)(441.8024585,71.28970998)
\curveto(441.77245856,71.18970744)(441.7324586,71.08470755)(441.6824585,70.97470998)
\curveto(441.52245881,70.60470803)(441.32245901,70.28970834)(441.0824585,70.02970998)
\curveto(440.8324595,69.76970886)(440.52245981,69.55970907)(440.1524585,69.39970998)
\curveto(440.06246027,69.35970927)(439.96746036,69.32470931)(439.8674585,69.29470998)
\curveto(439.76746056,69.26470937)(439.66246067,69.2347094)(439.5524585,69.20470998)
\curveto(439.50246083,69.18470945)(439.45246088,69.17470946)(439.4024585,69.17470998)
\curveto(439.34246099,69.17470946)(439.28246105,69.16470947)(439.2224585,69.14470998)
\curveto(439.16246117,69.12470951)(439.08246125,69.11470952)(438.9824585,69.11470998)
\curveto(438.88246145,69.11470952)(438.80746152,69.1297095)(438.7574585,69.15970998)
\curveto(438.7274616,69.16970946)(438.70246163,69.18470945)(438.6824585,69.20470998)
\lineto(438.6224585,69.26470998)
\curveto(438.60246173,69.30470933)(438.58746174,69.36470927)(438.5774585,69.44470998)
\curveto(438.56746176,69.5347091)(438.56246177,69.62470901)(438.5624585,69.71470998)
\curveto(438.56246177,69.80470883)(438.56746176,69.88970874)(438.5774585,69.96970998)
\curveto(438.58746174,70.05970857)(438.59746173,70.12470851)(438.6074585,70.16470998)
\curveto(438.6274617,70.18470845)(438.64246169,70.20470843)(438.6524585,70.22470998)
\curveto(438.65246168,70.24470839)(438.66246167,70.26470837)(438.6824585,70.28470998)
\curveto(438.77246156,70.35470828)(438.88746144,70.39470824)(439.0274585,70.40470998)
\curveto(439.16746116,70.42470821)(439.29246104,70.45470818)(439.4024585,70.49470998)
\lineto(439.7624585,70.64470998)
\curveto(439.87246046,70.69470794)(439.97746035,70.75970787)(440.0774585,70.83970998)
\curveto(440.10746022,70.85970777)(440.1324602,70.87970775)(440.1524585,70.89970998)
\curveto(440.17246016,70.9297077)(440.19746013,70.95470768)(440.2274585,70.97470998)
\curveto(440.28746004,71.01470762)(440.33246,71.04970758)(440.3624585,71.07970998)
\curveto(440.39245994,71.11970751)(440.42245991,71.15470748)(440.4524585,71.18470998)
\curveto(440.48245985,71.22470741)(440.51245982,71.26970736)(440.5424585,71.31970998)
\curveto(440.60245973,71.40970722)(440.65245968,71.50470713)(440.6924585,71.60470998)
\lineto(440.8124585,71.93470998)
\curveto(440.86245947,72.08470655)(440.89245944,72.28470635)(440.9024585,72.53470998)
\curveto(440.91245942,72.78470585)(440.89245944,72.99470564)(440.8424585,73.16470998)
\curveto(440.82245951,73.24470539)(440.80745952,73.31470532)(440.7974585,73.37470998)
\lineto(440.7374585,73.58470998)
\curveto(440.61745971,73.86470477)(440.46745986,74.10470453)(440.2874585,74.30470998)
\curveto(440.10746022,74.51470412)(439.87746045,74.67970395)(439.5974585,74.79970998)
\curveto(439.5274608,74.8297038)(439.45746087,74.84970378)(439.3874585,74.85970998)
\lineto(439.1474585,74.91970998)
\curveto(439.00746132,74.95970367)(438.84746148,74.96970366)(438.6674585,74.94970998)
\curveto(438.47746185,74.9297037)(438.327462,74.89970373)(438.2174585,74.85970998)
\curveto(437.83746249,74.7297039)(437.54746278,74.54470409)(437.3474585,74.30470998)
\curveto(437.14746318,74.07470456)(436.98746334,73.76470487)(436.8674585,73.37470998)
\curveto(436.83746349,73.26470537)(436.81746351,73.14470549)(436.8074585,73.01470998)
\curveto(436.79746353,72.89470574)(436.79246354,72.76970586)(436.7924585,72.63970998)
\curveto(436.79246354,72.47970615)(436.78746354,72.33970629)(436.7774585,72.21970998)
\curveto(436.76746356,72.09970653)(436.70746362,72.01470662)(436.5974585,71.96470998)
\curveto(436.56746376,71.94470669)(436.5324638,71.9347067)(436.4924585,71.93470998)
\lineto(436.3574585,71.93470998)
\curveto(436.25746407,71.92470671)(436.16246417,71.92470671)(436.0724585,71.93470998)
\curveto(435.98246435,71.95470668)(435.91746441,71.99470664)(435.8774585,72.05470998)
\curveto(435.84746448,72.09470654)(435.8274645,72.1347065)(435.8174585,72.17470998)
\curveto(435.80746452,72.22470641)(435.79746453,72.27970635)(435.7874585,72.33970998)
\curveto(435.77746455,72.35970627)(435.77746455,72.38470625)(435.7874585,72.41470998)
\curveto(435.78746454,72.44470619)(435.78246455,72.46970616)(435.7724585,72.48970998)
\lineto(435.7724585,72.62470998)
\curveto(435.75246458,72.7347059)(435.74246459,72.8347058)(435.7424585,72.92470998)
\curveto(435.7324646,73.02470561)(435.71246462,73.11970551)(435.6824585,73.20970998)
\curveto(435.57246476,73.5297051)(435.4274649,73.78470485)(435.2474585,73.97470998)
\curveto(435.06746526,74.16470447)(434.81746551,74.31470432)(434.4974585,74.42470998)
\curveto(434.39746593,74.45470418)(434.27246606,74.47470416)(434.1224585,74.48470998)
\curveto(433.96246637,74.50470413)(433.81746651,74.49970413)(433.6874585,74.46970998)
\curveto(433.61746671,74.44970418)(433.55246678,74.4297042)(433.4924585,74.40970998)
\curveto(433.42246691,74.39970423)(433.35746697,74.37970425)(433.2974585,74.34970998)
\curveto(433.05746727,74.24970438)(432.86746746,74.10470453)(432.7274585,73.91470998)
\curveto(432.58746774,73.72470491)(432.47746785,73.49970513)(432.3974585,73.23970998)
\curveto(432.37746795,73.17970545)(432.36746796,73.11970551)(432.3674585,73.05970998)
\curveto(432.36746796,72.99970563)(432.35746797,72.9347057)(432.3374585,72.86470998)
\curveto(432.31746801,72.78470585)(432.30746802,72.68970594)(432.3074585,72.57970998)
\curveto(432.30746802,72.46970616)(432.31746801,72.37470626)(432.3374585,72.29470998)
\curveto(432.35746797,72.24470639)(432.36746796,72.19470644)(432.3674585,72.14470998)
\curveto(432.36746796,72.10470653)(432.37746795,72.05970657)(432.3974585,72.00970998)
\curveto(432.44746788,71.8297068)(432.52246781,71.65970697)(432.6224585,71.49970998)
\curveto(432.71246762,71.34970728)(432.8274675,71.21970741)(432.9674585,71.10970998)
\curveto(433.08746724,71.01970761)(433.21746711,70.93970769)(433.3574585,70.86970998)
\curveto(433.49746683,70.79970783)(433.65246668,70.7347079)(433.8224585,70.67470998)
\curveto(433.9324664,70.64470799)(434.05246628,70.62470801)(434.1824585,70.61470998)
\curveto(434.30246603,70.60470803)(434.40246593,70.56970806)(434.4824585,70.50970998)
\curveto(434.52246581,70.48970814)(434.56246577,70.4297082)(434.6024585,70.32970998)
\curveto(434.61246572,70.28970834)(434.62246571,70.2297084)(434.6324585,70.14970998)
\lineto(434.6324585,69.89470998)
\curveto(434.62246571,69.80470883)(434.61246572,69.71970891)(434.6024585,69.63970998)
\curveto(434.59246574,69.56970906)(434.57746575,69.51970911)(434.5574585,69.48970998)
\curveto(434.5274658,69.44970918)(434.47246586,69.41470922)(434.3924585,69.38470998)
\curveto(434.31246602,69.35470928)(434.2274661,69.34970928)(434.1374585,69.36970998)
\curveto(434.08746624,69.37970925)(434.03746629,69.38470925)(433.9874585,69.38470998)
\lineto(433.8074585,69.41470998)
\curveto(433.70746662,69.44470919)(433.60746672,69.46970916)(433.5074585,69.48970998)
\curveto(433.40746692,69.51970911)(433.31746701,69.55470908)(433.2374585,69.59470998)
\curveto(433.1274672,69.64470899)(433.02246731,69.68970894)(432.9224585,69.72970998)
\curveto(432.81246752,69.76970886)(432.70746762,69.81970881)(432.6074585,69.87970998)
\curveto(432.06746826,70.20970842)(431.67246866,70.67970795)(431.4224585,71.28970998)
\curveto(431.37246896,71.40970722)(431.33746899,71.5347071)(431.3174585,71.66470998)
\curveto(431.29746903,71.80470683)(431.27246906,71.94470669)(431.2424585,72.08470998)
\curveto(431.2324691,72.14470649)(431.2274691,72.20470643)(431.2274585,72.26470998)
\curveto(431.2274691,72.3347063)(431.22246911,72.39970623)(431.2124585,72.45970998)
}
}
{
\newrgbcolor{curcolor}{0 0 0}
\pscustom[linestyle=none,fillstyle=solid,fillcolor=curcolor]
{
\newpath
\moveto(440.1824585,78.63431936)
\lineto(440.1824585,79.26431936)
\lineto(440.1824585,79.45931936)
\curveto(440.18246015,79.52931683)(440.19246014,79.58931677)(440.2124585,79.63931936)
\curveto(440.25246008,79.70931665)(440.29246004,79.7593166)(440.3324585,79.78931936)
\curveto(440.38245995,79.82931653)(440.44745988,79.84931651)(440.5274585,79.84931936)
\curveto(440.60745972,79.8593165)(440.69245964,79.86431649)(440.7824585,79.86431936)
\lineto(441.5024585,79.86431936)
\curveto(441.98245835,79.86431649)(442.39245794,79.80431655)(442.7324585,79.68431936)
\curveto(443.07245726,79.56431679)(443.34745698,79.36931699)(443.5574585,79.09931936)
\curveto(443.60745672,79.02931733)(443.65245668,78.9593174)(443.6924585,78.88931936)
\curveto(443.74245659,78.82931753)(443.78745654,78.7543176)(443.8274585,78.66431936)
\curveto(443.83745649,78.64431771)(443.84745648,78.61431774)(443.8574585,78.57431936)
\curveto(443.87745645,78.53431782)(443.88245645,78.48931787)(443.8724585,78.43931936)
\curveto(443.84245649,78.34931801)(443.76745656,78.29431806)(443.6474585,78.27431936)
\curveto(443.53745679,78.2543181)(443.44245689,78.26931809)(443.3624585,78.31931936)
\curveto(443.29245704,78.34931801)(443.2274571,78.39431796)(443.1674585,78.45431936)
\curveto(443.11745721,78.52431783)(443.06745726,78.58931777)(443.0174585,78.64931936)
\curveto(442.96745736,78.71931764)(442.89245744,78.77931758)(442.7924585,78.82931936)
\curveto(442.70245763,78.88931747)(442.61245772,78.93931742)(442.5224585,78.97931936)
\curveto(442.49245784,78.99931736)(442.4324579,79.02431733)(442.3424585,79.05431936)
\curveto(442.26245807,79.08431727)(442.19245814,79.08931727)(442.1324585,79.06931936)
\curveto(441.99245834,79.03931732)(441.90245843,78.97931738)(441.8624585,78.88931936)
\curveto(441.8324585,78.80931755)(441.81745851,78.71931764)(441.8174585,78.61931936)
\curveto(441.81745851,78.51931784)(441.79245854,78.43431792)(441.7424585,78.36431936)
\curveto(441.67245866,78.27431808)(441.5324588,78.22931813)(441.3224585,78.22931936)
\lineto(440.7674585,78.22931936)
\lineto(440.5424585,78.22931936)
\curveto(440.46245987,78.23931812)(440.39745993,78.2593181)(440.3474585,78.28931936)
\curveto(440.26746006,78.34931801)(440.22246011,78.41931794)(440.2124585,78.49931936)
\curveto(440.20246013,78.51931784)(440.19746013,78.53931782)(440.1974585,78.55931936)
\curveto(440.19746013,78.58931777)(440.19246014,78.61431774)(440.1824585,78.63431936)
}
}
{
\newrgbcolor{curcolor}{0 0 0}
\pscustom[linestyle=none,fillstyle=solid,fillcolor=curcolor]
{
}
}
{
\newrgbcolor{curcolor}{0 0 0}
\pscustom[linestyle=none,fillstyle=solid,fillcolor=curcolor]
{
\newpath
\moveto(431.2124585,89.26463186)
\curveto(431.20246913,89.95462722)(431.32246901,90.55462662)(431.5724585,91.06463186)
\curveto(431.82246851,91.58462559)(432.15746817,91.9796252)(432.5774585,92.24963186)
\curveto(432.65746767,92.29962488)(432.74746758,92.34462483)(432.8474585,92.38463186)
\curveto(432.93746739,92.42462475)(433.0324673,92.46962471)(433.1324585,92.51963186)
\curveto(433.2324671,92.55962462)(433.332467,92.58962459)(433.4324585,92.60963186)
\curveto(433.5324668,92.62962455)(433.63746669,92.64962453)(433.7474585,92.66963186)
\curveto(433.79746653,92.68962449)(433.84246649,92.69462448)(433.8824585,92.68463186)
\curveto(433.92246641,92.6746245)(433.96746636,92.6796245)(434.0174585,92.69963186)
\curveto(434.06746626,92.70962447)(434.15246618,92.71462446)(434.2724585,92.71463186)
\curveto(434.38246595,92.71462446)(434.46746586,92.70962447)(434.5274585,92.69963186)
\curveto(434.58746574,92.6796245)(434.64746568,92.66962451)(434.7074585,92.66963186)
\curveto(434.76746556,92.6796245)(434.8274655,92.6746245)(434.8874585,92.65463186)
\curveto(435.0274653,92.61462456)(435.16246517,92.5796246)(435.2924585,92.54963186)
\curveto(435.42246491,92.51962466)(435.54746478,92.4796247)(435.6674585,92.42963186)
\curveto(435.80746452,92.36962481)(435.9324644,92.29962488)(436.0424585,92.21963186)
\curveto(436.15246418,92.14962503)(436.26246407,92.0746251)(436.3724585,91.99463186)
\lineto(436.4324585,91.93463186)
\curveto(436.45246388,91.92462525)(436.47246386,91.90962527)(436.4924585,91.88963186)
\curveto(436.65246368,91.76962541)(436.79746353,91.63462554)(436.9274585,91.48463186)
\curveto(437.05746327,91.33462584)(437.18246315,91.174626)(437.3024585,91.00463186)
\curveto(437.52246281,90.69462648)(437.7274626,90.39962678)(437.9174585,90.11963186)
\curveto(438.05746227,89.88962729)(438.19246214,89.65962752)(438.3224585,89.42963186)
\curveto(438.45246188,89.20962797)(438.58746174,88.98962819)(438.7274585,88.76963186)
\curveto(438.89746143,88.51962866)(439.07746125,88.2796289)(439.2674585,88.04963186)
\curveto(439.45746087,87.82962935)(439.68246065,87.63962954)(439.9424585,87.47963186)
\curveto(440.00246033,87.43962974)(440.06246027,87.40462977)(440.1224585,87.37463186)
\curveto(440.17246016,87.34462983)(440.23746009,87.31462986)(440.3174585,87.28463186)
\curveto(440.38745994,87.26462991)(440.44745988,87.25962992)(440.4974585,87.26963186)
\curveto(440.56745976,87.28962989)(440.62245971,87.32462985)(440.6624585,87.37463186)
\curveto(440.69245964,87.42462975)(440.71245962,87.48462969)(440.7224585,87.55463186)
\lineto(440.7224585,87.79463186)
\lineto(440.7224585,88.54463186)
\lineto(440.7224585,91.34963186)
\lineto(440.7224585,92.00963186)
\curveto(440.72245961,92.09962508)(440.7274596,92.18462499)(440.7374585,92.26463186)
\curveto(440.73745959,92.34462483)(440.75745957,92.40962477)(440.7974585,92.45963186)
\curveto(440.83745949,92.50962467)(440.91245942,92.54962463)(441.0224585,92.57963186)
\curveto(441.12245921,92.61962456)(441.22245911,92.62962455)(441.3224585,92.60963186)
\lineto(441.4574585,92.60963186)
\curveto(441.5274588,92.58962459)(441.58745874,92.56962461)(441.6374585,92.54963186)
\curveto(441.68745864,92.52962465)(441.7274586,92.49462468)(441.7574585,92.44463186)
\curveto(441.79745853,92.39462478)(441.81745851,92.32462485)(441.8174585,92.23463186)
\lineto(441.8174585,91.96463186)
\lineto(441.8174585,91.06463186)
\lineto(441.8174585,87.55463186)
\lineto(441.8174585,86.48963186)
\curveto(441.81745851,86.40963077)(441.82245851,86.31963086)(441.8324585,86.21963186)
\curveto(441.8324585,86.11963106)(441.82245851,86.03463114)(441.8024585,85.96463186)
\curveto(441.7324586,85.75463142)(441.55245878,85.68963149)(441.2624585,85.76963186)
\curveto(441.22245911,85.7796314)(441.18745914,85.7796314)(441.1574585,85.76963186)
\curveto(441.11745921,85.76963141)(441.07245926,85.7796314)(441.0224585,85.79963186)
\curveto(440.94245939,85.81963136)(440.85745947,85.83963134)(440.7674585,85.85963186)
\curveto(440.67745965,85.8796313)(440.59245974,85.90463127)(440.5124585,85.93463186)
\curveto(440.02246031,86.09463108)(439.60746072,86.29463088)(439.2674585,86.53463186)
\curveto(439.01746131,86.71463046)(438.79246154,86.91963026)(438.5924585,87.14963186)
\curveto(438.38246195,87.3796298)(438.18746214,87.61962956)(438.0074585,87.86963186)
\curveto(437.8274625,88.12962905)(437.65746267,88.39462878)(437.4974585,88.66463186)
\curveto(437.327463,88.94462823)(437.15246318,89.21462796)(436.9724585,89.47463186)
\curveto(436.89246344,89.58462759)(436.81746351,89.68962749)(436.7474585,89.78963186)
\curveto(436.67746365,89.89962728)(436.60246373,90.00962717)(436.5224585,90.11963186)
\curveto(436.49246384,90.15962702)(436.46246387,90.19462698)(436.4324585,90.22463186)
\curveto(436.39246394,90.26462691)(436.36246397,90.30462687)(436.3424585,90.34463186)
\curveto(436.2324641,90.48462669)(436.10746422,90.60962657)(435.9674585,90.71963186)
\curveto(435.93746439,90.73962644)(435.91246442,90.76462641)(435.8924585,90.79463186)
\curveto(435.86246447,90.82462635)(435.8324645,90.84962633)(435.8024585,90.86963186)
\curveto(435.70246463,90.94962623)(435.60246473,91.01462616)(435.5024585,91.06463186)
\curveto(435.40246493,91.12462605)(435.29246504,91.179626)(435.1724585,91.22963186)
\curveto(435.10246523,91.25962592)(435.0274653,91.2796259)(434.9474585,91.28963186)
\lineto(434.7074585,91.34963186)
\lineto(434.6174585,91.34963186)
\curveto(434.58746574,91.35962582)(434.55746577,91.36462581)(434.5274585,91.36463186)
\curveto(434.45746587,91.38462579)(434.36246597,91.38962579)(434.2424585,91.37963186)
\curveto(434.11246622,91.3796258)(434.01246632,91.36962581)(433.9424585,91.34963186)
\curveto(433.86246647,91.32962585)(433.78746654,91.30962587)(433.7174585,91.28963186)
\curveto(433.63746669,91.2796259)(433.55746677,91.25962592)(433.4774585,91.22963186)
\curveto(433.23746709,91.11962606)(433.03746729,90.96962621)(432.8774585,90.77963186)
\curveto(432.70746762,90.59962658)(432.56746776,90.3796268)(432.4574585,90.11963186)
\curveto(432.43746789,90.04962713)(432.42246791,89.9796272)(432.4124585,89.90963186)
\curveto(432.39246794,89.83962734)(432.37246796,89.76462741)(432.3524585,89.68463186)
\curveto(432.332468,89.60462757)(432.32246801,89.49462768)(432.3224585,89.35463186)
\curveto(432.32246801,89.22462795)(432.332468,89.11962806)(432.3524585,89.03963186)
\curveto(432.36246797,88.9796282)(432.36746796,88.92462825)(432.3674585,88.87463186)
\curveto(432.36746796,88.82462835)(432.37746795,88.7746284)(432.3974585,88.72463186)
\curveto(432.43746789,88.62462855)(432.47746785,88.52962865)(432.5174585,88.43963186)
\curveto(432.55746777,88.35962882)(432.60246773,88.2796289)(432.6524585,88.19963186)
\curveto(432.67246766,88.16962901)(432.69746763,88.13962904)(432.7274585,88.10963186)
\curveto(432.75746757,88.08962909)(432.78246755,88.06462911)(432.8024585,88.03463186)
\lineto(432.8774585,87.95963186)
\curveto(432.89746743,87.92962925)(432.91746741,87.90462927)(432.9374585,87.88463186)
\lineto(433.1474585,87.73463186)
\curveto(433.20746712,87.69462948)(433.27246706,87.64962953)(433.3424585,87.59963186)
\curveto(433.4324669,87.53962964)(433.53746679,87.48962969)(433.6574585,87.44963186)
\curveto(433.76746656,87.41962976)(433.87746645,87.38462979)(433.9874585,87.34463186)
\curveto(434.09746623,87.30462987)(434.24246609,87.2796299)(434.4224585,87.26963186)
\curveto(434.59246574,87.25962992)(434.71746561,87.22962995)(434.7974585,87.17963186)
\curveto(434.87746545,87.12963005)(434.92246541,87.05463012)(434.9324585,86.95463186)
\curveto(434.94246539,86.85463032)(434.94746538,86.74463043)(434.9474585,86.62463186)
\curveto(434.94746538,86.58463059)(434.95246538,86.54463063)(434.9624585,86.50463186)
\curveto(434.96246537,86.46463071)(434.95746537,86.42963075)(434.9474585,86.39963186)
\curveto(434.9274654,86.34963083)(434.91746541,86.29963088)(434.9174585,86.24963186)
\curveto(434.91746541,86.20963097)(434.90746542,86.16963101)(434.8874585,86.12963186)
\curveto(434.8274655,86.03963114)(434.69246564,85.99463118)(434.4824585,85.99463186)
\lineto(434.3624585,85.99463186)
\curveto(434.30246603,86.00463117)(434.24246609,86.00963117)(434.1824585,86.00963186)
\curveto(434.11246622,86.01963116)(434.04746628,86.02963115)(433.9874585,86.03963186)
\curveto(433.87746645,86.05963112)(433.77746655,86.0796311)(433.6874585,86.09963186)
\curveto(433.58746674,86.11963106)(433.49246684,86.14963103)(433.4024585,86.18963186)
\curveto(433.332467,86.20963097)(433.27246706,86.22963095)(433.2224585,86.24963186)
\lineto(433.0424585,86.30963186)
\curveto(432.78246755,86.42963075)(432.53746779,86.58463059)(432.3074585,86.77463186)
\curveto(432.07746825,86.9746302)(431.89246844,87.18962999)(431.7524585,87.41963186)
\curveto(431.67246866,87.52962965)(431.60746872,87.64462953)(431.5574585,87.76463186)
\lineto(431.4074585,88.15463186)
\curveto(431.35746897,88.26462891)(431.327469,88.3796288)(431.3174585,88.49963186)
\curveto(431.29746903,88.61962856)(431.27246906,88.74462843)(431.2424585,88.87463186)
\curveto(431.24246909,88.94462823)(431.24246909,89.00962817)(431.2424585,89.06963186)
\curveto(431.2324691,89.12962805)(431.22246911,89.19462798)(431.2124585,89.26463186)
}
}
{
\newrgbcolor{curcolor}{0 0 0}
\pscustom[linestyle=none,fillstyle=solid,fillcolor=curcolor]
{
\newpath
\moveto(436.7324585,101.36424123)
\lineto(436.9874585,101.36424123)
\curveto(437.06746326,101.37423353)(437.14246319,101.36923353)(437.2124585,101.34924123)
\lineto(437.4524585,101.34924123)
\lineto(437.6174585,101.34924123)
\curveto(437.71746261,101.32923357)(437.82246251,101.31923358)(437.9324585,101.31924123)
\curveto(438.0324623,101.31923358)(438.1324622,101.30923359)(438.2324585,101.28924123)
\lineto(438.3824585,101.28924123)
\curveto(438.52246181,101.25923364)(438.66246167,101.23923366)(438.8024585,101.22924123)
\curveto(438.9324614,101.21923368)(439.06246127,101.19423371)(439.1924585,101.15424123)
\curveto(439.27246106,101.13423377)(439.35746097,101.11423379)(439.4474585,101.09424123)
\lineto(439.6874585,101.03424123)
\lineto(439.9874585,100.91424123)
\curveto(440.07746025,100.88423402)(440.16746016,100.84923405)(440.2574585,100.80924123)
\curveto(440.47745985,100.70923419)(440.69245964,100.57423433)(440.9024585,100.40424123)
\curveto(441.11245922,100.24423466)(441.28245905,100.06923483)(441.4124585,99.87924123)
\curveto(441.45245888,99.82923507)(441.49245884,99.76923513)(441.5324585,99.69924123)
\curveto(441.56245877,99.63923526)(441.59745873,99.57923532)(441.6374585,99.51924123)
\curveto(441.68745864,99.43923546)(441.7274586,99.34423556)(441.7574585,99.23424123)
\curveto(441.78745854,99.12423578)(441.81745851,99.01923588)(441.8474585,98.91924123)
\curveto(441.88745844,98.80923609)(441.91245842,98.6992362)(441.9224585,98.58924123)
\curveto(441.9324584,98.47923642)(441.94745838,98.36423654)(441.9674585,98.24424123)
\curveto(441.97745835,98.2042367)(441.97745835,98.15923674)(441.9674585,98.10924123)
\curveto(441.96745836,98.06923683)(441.97245836,98.02923687)(441.9824585,97.98924123)
\curveto(441.99245834,97.94923695)(441.99745833,97.89423701)(441.9974585,97.82424123)
\curveto(441.99745833,97.75423715)(441.99245834,97.7042372)(441.9824585,97.67424123)
\curveto(441.96245837,97.62423728)(441.95745837,97.57923732)(441.9674585,97.53924123)
\curveto(441.97745835,97.4992374)(441.97745835,97.46423744)(441.9674585,97.43424123)
\lineto(441.9674585,97.34424123)
\curveto(441.94745838,97.28423762)(441.9324584,97.21923768)(441.9224585,97.14924123)
\curveto(441.92245841,97.08923781)(441.91745841,97.02423788)(441.9074585,96.95424123)
\curveto(441.85745847,96.78423812)(441.80745852,96.62423828)(441.7574585,96.47424123)
\curveto(441.70745862,96.32423858)(441.64245869,96.17923872)(441.5624585,96.03924123)
\curveto(441.52245881,95.98923891)(441.49245884,95.93423897)(441.4724585,95.87424123)
\curveto(441.44245889,95.82423908)(441.40745892,95.77423913)(441.3674585,95.72424123)
\curveto(441.18745914,95.48423942)(440.96745936,95.28423962)(440.7074585,95.12424123)
\curveto(440.44745988,94.96423994)(440.16246017,94.82424008)(439.8524585,94.70424123)
\curveto(439.71246062,94.64424026)(439.57246076,94.5992403)(439.4324585,94.56924123)
\curveto(439.28246105,94.53924036)(439.1274612,94.5042404)(438.9674585,94.46424123)
\curveto(438.85746147,94.44424046)(438.74746158,94.42924047)(438.6374585,94.41924123)
\curveto(438.5274618,94.40924049)(438.41746191,94.39424051)(438.3074585,94.37424123)
\curveto(438.26746206,94.36424054)(438.2274621,94.35924054)(438.1874585,94.35924123)
\curveto(438.14746218,94.36924053)(438.10746222,94.36924053)(438.0674585,94.35924123)
\curveto(438.01746231,94.34924055)(437.96746236,94.34424056)(437.9174585,94.34424123)
\lineto(437.7524585,94.34424123)
\curveto(437.70246263,94.32424058)(437.65246268,94.31924058)(437.6024585,94.32924123)
\curveto(437.54246279,94.33924056)(437.48746284,94.33924056)(437.4374585,94.32924123)
\curveto(437.39746293,94.31924058)(437.35246298,94.31924058)(437.3024585,94.32924123)
\curveto(437.25246308,94.33924056)(437.20246313,94.33424057)(437.1524585,94.31424123)
\curveto(437.08246325,94.29424061)(437.00746332,94.28924061)(436.9274585,94.29924123)
\curveto(436.83746349,94.30924059)(436.75246358,94.31424059)(436.6724585,94.31424123)
\curveto(436.58246375,94.31424059)(436.48246385,94.30924059)(436.3724585,94.29924123)
\curveto(436.25246408,94.28924061)(436.15246418,94.29424061)(436.0724585,94.31424123)
\lineto(435.7874585,94.31424123)
\lineto(435.1574585,94.35924123)
\curveto(435.05746527,94.36924053)(434.96246537,94.37924052)(434.8724585,94.38924123)
\lineto(434.5724585,94.41924123)
\curveto(434.52246581,94.43924046)(434.47246586,94.44424046)(434.4224585,94.43424123)
\curveto(434.36246597,94.43424047)(434.30746602,94.44424046)(434.2574585,94.46424123)
\curveto(434.08746624,94.51424039)(433.92246641,94.55424035)(433.7624585,94.58424123)
\curveto(433.59246674,94.61424029)(433.4324669,94.66424024)(433.2824585,94.73424123)
\curveto(432.82246751,94.92423998)(432.44746788,95.14423976)(432.1574585,95.39424123)
\curveto(431.86746846,95.65423925)(431.62246871,96.01423889)(431.4224585,96.47424123)
\curveto(431.37246896,96.6042383)(431.33746899,96.73423817)(431.3174585,96.86424123)
\curveto(431.29746903,97.0042379)(431.27246906,97.14423776)(431.2424585,97.28424123)
\curveto(431.2324691,97.35423755)(431.2274691,97.41923748)(431.2274585,97.47924123)
\curveto(431.2274691,97.53923736)(431.22246911,97.6042373)(431.2124585,97.67424123)
\curveto(431.19246914,98.5042364)(431.34246899,99.17423573)(431.6624585,99.68424123)
\curveto(431.97246836,100.19423471)(432.41246792,100.57423433)(432.9824585,100.82424123)
\curveto(433.10246723,100.87423403)(433.2274671,100.91923398)(433.3574585,100.95924123)
\curveto(433.48746684,100.9992339)(433.62246671,101.04423386)(433.7624585,101.09424123)
\curveto(433.84246649,101.11423379)(433.9274664,101.12923377)(434.0174585,101.13924123)
\lineto(434.2574585,101.19924123)
\curveto(434.36746596,101.22923367)(434.47746585,101.24423366)(434.5874585,101.24424123)
\curveto(434.69746563,101.25423365)(434.80746552,101.26923363)(434.9174585,101.28924123)
\curveto(434.96746536,101.30923359)(435.01246532,101.31423359)(435.0524585,101.30424123)
\curveto(435.09246524,101.3042336)(435.1324652,101.30923359)(435.1724585,101.31924123)
\curveto(435.22246511,101.32923357)(435.27746505,101.32923357)(435.3374585,101.31924123)
\curveto(435.38746494,101.31923358)(435.43746489,101.32423358)(435.4874585,101.33424123)
\lineto(435.6224585,101.33424123)
\curveto(435.68246465,101.35423355)(435.75246458,101.35423355)(435.8324585,101.33424123)
\curveto(435.90246443,101.32423358)(435.96746436,101.32923357)(436.0274585,101.34924123)
\curveto(436.05746427,101.35923354)(436.09746423,101.36423354)(436.1474585,101.36424123)
\lineto(436.2674585,101.36424123)
\lineto(436.7324585,101.36424123)
\moveto(439.0574585,99.81924123)
\curveto(438.73746159,99.91923498)(438.37246196,99.97923492)(437.9624585,99.99924123)
\curveto(437.55246278,100.01923488)(437.14246319,100.02923487)(436.7324585,100.02924123)
\curveto(436.30246403,100.02923487)(435.88246445,100.01923488)(435.4724585,99.99924123)
\curveto(435.06246527,99.97923492)(434.67746565,99.93423497)(434.3174585,99.86424123)
\curveto(433.95746637,99.79423511)(433.63746669,99.68423522)(433.3574585,99.53424123)
\curveto(433.06746726,99.39423551)(432.8324675,99.1992357)(432.6524585,98.94924123)
\curveto(432.54246779,98.78923611)(432.46246787,98.60923629)(432.4124585,98.40924123)
\curveto(432.35246798,98.20923669)(432.32246801,97.96423694)(432.3224585,97.67424123)
\curveto(432.34246799,97.65423725)(432.35246798,97.61923728)(432.3524585,97.56924123)
\curveto(432.34246799,97.51923738)(432.34246799,97.47923742)(432.3524585,97.44924123)
\curveto(432.37246796,97.36923753)(432.39246794,97.29423761)(432.4124585,97.22424123)
\curveto(432.42246791,97.16423774)(432.44246789,97.0992378)(432.4724585,97.02924123)
\curveto(432.59246774,96.75923814)(432.76246757,96.53923836)(432.9824585,96.36924123)
\curveto(433.19246714,96.20923869)(433.43746689,96.07423883)(433.7174585,95.96424123)
\curveto(433.8274665,95.91423899)(433.94746638,95.87423903)(434.0774585,95.84424123)
\curveto(434.19746613,95.82423908)(434.32246601,95.7992391)(434.4524585,95.76924123)
\curveto(434.50246583,95.74923915)(434.55746577,95.73923916)(434.6174585,95.73924123)
\curveto(434.66746566,95.73923916)(434.71746561,95.73423917)(434.7674585,95.72424123)
\curveto(434.85746547,95.71423919)(434.95246538,95.7042392)(435.0524585,95.69424123)
\curveto(435.14246519,95.68423922)(435.23746509,95.67423923)(435.3374585,95.66424123)
\curveto(435.41746491,95.66423924)(435.50246483,95.65923924)(435.5924585,95.64924123)
\lineto(435.8324585,95.64924123)
\lineto(436.0124585,95.64924123)
\curveto(436.04246429,95.63923926)(436.07746425,95.63423927)(436.1174585,95.63424123)
\lineto(436.2524585,95.63424123)
\lineto(436.7024585,95.63424123)
\curveto(436.78246355,95.63423927)(436.86746346,95.62923927)(436.9574585,95.61924123)
\curveto(437.03746329,95.61923928)(437.11246322,95.62923927)(437.1824585,95.64924123)
\lineto(437.4524585,95.64924123)
\curveto(437.47246286,95.64923925)(437.50246283,95.64423926)(437.5424585,95.63424123)
\curveto(437.57246276,95.63423927)(437.59746273,95.63923926)(437.6174585,95.64924123)
\curveto(437.71746261,95.65923924)(437.81746251,95.66423924)(437.9174585,95.66424123)
\curveto(438.00746232,95.67423923)(438.10746222,95.68423922)(438.2174585,95.69424123)
\curveto(438.33746199,95.72423918)(438.46246187,95.73923916)(438.5924585,95.73924123)
\curveto(438.71246162,95.74923915)(438.8274615,95.77423913)(438.9374585,95.81424123)
\curveto(439.23746109,95.89423901)(439.50246083,95.97923892)(439.7324585,96.06924123)
\curveto(439.96246037,96.16923873)(440.17746015,96.31423859)(440.3774585,96.50424123)
\curveto(440.57745975,96.71423819)(440.7274596,96.97923792)(440.8274585,97.29924123)
\curveto(440.84745948,97.33923756)(440.85745947,97.37423753)(440.8574585,97.40424123)
\curveto(440.84745948,97.44423746)(440.85245948,97.48923741)(440.8724585,97.53924123)
\curveto(440.88245945,97.57923732)(440.89245944,97.64923725)(440.9024585,97.74924123)
\curveto(440.91245942,97.85923704)(440.90745942,97.94423696)(440.8874585,98.00424123)
\curveto(440.86745946,98.07423683)(440.85745947,98.14423676)(440.8574585,98.21424123)
\curveto(440.84745948,98.28423662)(440.8324595,98.34923655)(440.8124585,98.40924123)
\curveto(440.75245958,98.60923629)(440.66745966,98.78923611)(440.5574585,98.94924123)
\curveto(440.53745979,98.97923592)(440.51745981,99.0042359)(440.4974585,99.02424123)
\lineto(440.4374585,99.08424123)
\curveto(440.41745991,99.12423578)(440.37745995,99.17423573)(440.3174585,99.23424123)
\curveto(440.17746015,99.33423557)(440.04746028,99.41923548)(439.9274585,99.48924123)
\curveto(439.80746052,99.55923534)(439.66246067,99.62923527)(439.4924585,99.69924123)
\curveto(439.42246091,99.72923517)(439.35246098,99.74923515)(439.2824585,99.75924123)
\curveto(439.21246112,99.77923512)(439.13746119,99.7992351)(439.0574585,99.81924123)
}
}
{
\newrgbcolor{curcolor}{0 0 0}
\pscustom[linestyle=none,fillstyle=solid,fillcolor=curcolor]
{
\newpath
\moveto(431.2124585,106.77385061)
\curveto(431.21246912,106.87384575)(431.22246911,106.96884566)(431.2424585,107.05885061)
\curveto(431.25246908,107.14884548)(431.28246905,107.21384541)(431.3324585,107.25385061)
\curveto(431.41246892,107.31384531)(431.51746881,107.34384528)(431.6474585,107.34385061)
\lineto(432.0374585,107.34385061)
\lineto(433.5374585,107.34385061)
\lineto(439.9274585,107.34385061)
\lineto(441.0974585,107.34385061)
\lineto(441.4124585,107.34385061)
\curveto(441.51245882,107.35384527)(441.59245874,107.33884529)(441.6524585,107.29885061)
\curveto(441.7324586,107.24884538)(441.78245855,107.17384545)(441.8024585,107.07385061)
\curveto(441.81245852,106.98384564)(441.81745851,106.87384575)(441.8174585,106.74385061)
\lineto(441.8174585,106.51885061)
\curveto(441.79745853,106.43884619)(441.78245855,106.36884626)(441.7724585,106.30885061)
\curveto(441.75245858,106.24884638)(441.71245862,106.19884643)(441.6524585,106.15885061)
\curveto(441.59245874,106.11884651)(441.51745881,106.09884653)(441.4274585,106.09885061)
\lineto(441.1274585,106.09885061)
\lineto(440.0324585,106.09885061)
\lineto(434.6924585,106.09885061)
\curveto(434.60246573,106.07884655)(434.5274658,106.06384656)(434.4674585,106.05385061)
\curveto(434.39746593,106.05384657)(434.33746599,106.0238466)(434.2874585,105.96385061)
\curveto(434.23746609,105.89384673)(434.21246612,105.80384682)(434.2124585,105.69385061)
\curveto(434.20246613,105.59384703)(434.19746613,105.48384714)(434.1974585,105.36385061)
\lineto(434.1974585,104.22385061)
\lineto(434.1974585,103.72885061)
\curveto(434.18746614,103.56884906)(434.1274662,103.45884917)(434.0174585,103.39885061)
\curveto(433.98746634,103.37884925)(433.95746637,103.36884926)(433.9274585,103.36885061)
\curveto(433.88746644,103.36884926)(433.84246649,103.36384926)(433.7924585,103.35385061)
\curveto(433.67246666,103.33384929)(433.56246677,103.33884929)(433.4624585,103.36885061)
\curveto(433.36246697,103.40884922)(433.29246704,103.46384916)(433.2524585,103.53385061)
\curveto(433.20246713,103.61384901)(433.17746715,103.73384889)(433.1774585,103.89385061)
\curveto(433.17746715,104.05384857)(433.16246717,104.18884844)(433.1324585,104.29885061)
\curveto(433.12246721,104.34884828)(433.11746721,104.40384822)(433.1174585,104.46385061)
\curveto(433.10746722,104.5238481)(433.09246724,104.58384804)(433.0724585,104.64385061)
\curveto(433.02246731,104.79384783)(432.97246736,104.93884769)(432.9224585,105.07885061)
\curveto(432.86246747,105.21884741)(432.79246754,105.35384727)(432.7124585,105.48385061)
\curveto(432.62246771,105.623847)(432.51746781,105.74384688)(432.3974585,105.84385061)
\curveto(432.27746805,105.94384668)(432.14746818,106.03884659)(432.0074585,106.12885061)
\curveto(431.90746842,106.18884644)(431.79746853,106.23384639)(431.6774585,106.26385061)
\curveto(431.55746877,106.30384632)(431.45246888,106.35384627)(431.3624585,106.41385061)
\curveto(431.30246903,106.46384616)(431.26246907,106.53384609)(431.2424585,106.62385061)
\curveto(431.2324691,106.64384598)(431.2274691,106.66884596)(431.2274585,106.69885061)
\curveto(431.2274691,106.7288459)(431.22246911,106.75384587)(431.2124585,106.77385061)
}
}
{
\newrgbcolor{curcolor}{0 0 0}
\pscustom[linestyle=none,fillstyle=solid,fillcolor=curcolor]
{
\newpath
\moveto(431.2124585,115.12345998)
\curveto(431.21246912,115.22345513)(431.22246911,115.31845503)(431.2424585,115.40845998)
\curveto(431.25246908,115.49845485)(431.28246905,115.56345479)(431.3324585,115.60345998)
\curveto(431.41246892,115.66345469)(431.51746881,115.69345466)(431.6474585,115.69345998)
\lineto(432.0374585,115.69345998)
\lineto(433.5374585,115.69345998)
\lineto(439.9274585,115.69345998)
\lineto(441.0974585,115.69345998)
\lineto(441.4124585,115.69345998)
\curveto(441.51245882,115.70345465)(441.59245874,115.68845466)(441.6524585,115.64845998)
\curveto(441.7324586,115.59845475)(441.78245855,115.52345483)(441.8024585,115.42345998)
\curveto(441.81245852,115.33345502)(441.81745851,115.22345513)(441.8174585,115.09345998)
\lineto(441.8174585,114.86845998)
\curveto(441.79745853,114.78845556)(441.78245855,114.71845563)(441.7724585,114.65845998)
\curveto(441.75245858,114.59845575)(441.71245862,114.5484558)(441.6524585,114.50845998)
\curveto(441.59245874,114.46845588)(441.51745881,114.4484559)(441.4274585,114.44845998)
\lineto(441.1274585,114.44845998)
\lineto(440.0324585,114.44845998)
\lineto(434.6924585,114.44845998)
\curveto(434.60246573,114.42845592)(434.5274658,114.41345594)(434.4674585,114.40345998)
\curveto(434.39746593,114.40345595)(434.33746599,114.37345598)(434.2874585,114.31345998)
\curveto(434.23746609,114.24345611)(434.21246612,114.1534562)(434.2124585,114.04345998)
\curveto(434.20246613,113.94345641)(434.19746613,113.83345652)(434.1974585,113.71345998)
\lineto(434.1974585,112.57345998)
\lineto(434.1974585,112.07845998)
\curveto(434.18746614,111.91845843)(434.1274662,111.80845854)(434.0174585,111.74845998)
\curveto(433.98746634,111.72845862)(433.95746637,111.71845863)(433.9274585,111.71845998)
\curveto(433.88746644,111.71845863)(433.84246649,111.71345864)(433.7924585,111.70345998)
\curveto(433.67246666,111.68345867)(433.56246677,111.68845866)(433.4624585,111.71845998)
\curveto(433.36246697,111.75845859)(433.29246704,111.81345854)(433.2524585,111.88345998)
\curveto(433.20246713,111.96345839)(433.17746715,112.08345827)(433.1774585,112.24345998)
\curveto(433.17746715,112.40345795)(433.16246717,112.53845781)(433.1324585,112.64845998)
\curveto(433.12246721,112.69845765)(433.11746721,112.7534576)(433.1174585,112.81345998)
\curveto(433.10746722,112.87345748)(433.09246724,112.93345742)(433.0724585,112.99345998)
\curveto(433.02246731,113.14345721)(432.97246736,113.28845706)(432.9224585,113.42845998)
\curveto(432.86246747,113.56845678)(432.79246754,113.70345665)(432.7124585,113.83345998)
\curveto(432.62246771,113.97345638)(432.51746781,114.09345626)(432.3974585,114.19345998)
\curveto(432.27746805,114.29345606)(432.14746818,114.38845596)(432.0074585,114.47845998)
\curveto(431.90746842,114.53845581)(431.79746853,114.58345577)(431.6774585,114.61345998)
\curveto(431.55746877,114.6534557)(431.45246888,114.70345565)(431.3624585,114.76345998)
\curveto(431.30246903,114.81345554)(431.26246907,114.88345547)(431.2424585,114.97345998)
\curveto(431.2324691,114.99345536)(431.2274691,115.01845533)(431.2274585,115.04845998)
\curveto(431.2274691,115.07845527)(431.22246911,115.10345525)(431.2124585,115.12345998)
}
}
{
\newrgbcolor{curcolor}{0 0 0}
\pscustom[linestyle=none,fillstyle=solid,fillcolor=curcolor]
{
\newpath
\moveto(461.94880493,42.02236623)
\curveto(461.99880568,42.04235669)(462.05880562,42.06735666)(462.12880493,42.09736623)
\curveto(462.19880548,42.1273566)(462.2738054,42.14735658)(462.35380493,42.15736623)
\curveto(462.42380525,42.17735655)(462.49380518,42.17735655)(462.56380493,42.15736623)
\curveto(462.62380505,42.14735658)(462.66880501,42.10735662)(462.69880493,42.03736623)
\curveto(462.71880496,41.98735674)(462.72880495,41.9273568)(462.72880493,41.85736623)
\lineto(462.72880493,41.64736623)
\lineto(462.72880493,41.19736623)
\curveto(462.72880495,41.04735768)(462.70380497,40.9273578)(462.65380493,40.83736623)
\curveto(462.59380508,40.73735799)(462.48880519,40.66235807)(462.33880493,40.61236623)
\curveto(462.18880549,40.57235816)(462.05380562,40.5273582)(461.93380493,40.47736623)
\curveto(461.673806,40.36735836)(461.40380627,40.26735846)(461.12380493,40.17736623)
\curveto(460.84380683,40.08735864)(460.56880711,39.98735874)(460.29880493,39.87736623)
\curveto(460.20880747,39.84735888)(460.12380755,39.81735891)(460.04380493,39.78736623)
\curveto(459.96380771,39.76735896)(459.88880779,39.73735899)(459.81880493,39.69736623)
\curveto(459.74880793,39.66735906)(459.68880799,39.62235911)(459.63880493,39.56236623)
\curveto(459.58880809,39.50235923)(459.54880813,39.42235931)(459.51880493,39.32236623)
\curveto(459.49880818,39.27235946)(459.49380818,39.21235952)(459.50380493,39.14236623)
\lineto(459.50380493,38.94736623)
\lineto(459.50380493,36.11236623)
\lineto(459.50380493,35.81236623)
\curveto(459.49380818,35.70236303)(459.49380818,35.59736313)(459.50380493,35.49736623)
\curveto(459.51380816,35.39736333)(459.52880815,35.30236343)(459.54880493,35.21236623)
\curveto(459.56880811,35.1323636)(459.60880807,35.07236366)(459.66880493,35.03236623)
\curveto(459.76880791,34.95236378)(459.88380779,34.89236384)(460.01380493,34.85236623)
\curveto(460.13380754,34.82236391)(460.25880742,34.78236395)(460.38880493,34.73236623)
\curveto(460.61880706,34.6323641)(460.85880682,34.53736419)(461.10880493,34.44736623)
\curveto(461.35880632,34.36736436)(461.59880608,34.27736445)(461.82880493,34.17736623)
\curveto(461.88880579,34.15736457)(461.95880572,34.1323646)(462.03880493,34.10236623)
\curveto(462.10880557,34.08236465)(462.18380549,34.05736467)(462.26380493,34.02736623)
\curveto(462.34380533,33.99736473)(462.41880526,33.96236477)(462.48880493,33.92236623)
\curveto(462.54880513,33.89236484)(462.59380508,33.85736487)(462.62380493,33.81736623)
\curveto(462.68380499,33.73736499)(462.71880496,33.6273651)(462.72880493,33.48736623)
\lineto(462.72880493,33.06736623)
\lineto(462.72880493,32.82736623)
\curveto(462.71880496,32.75736597)(462.69380498,32.69736603)(462.65380493,32.64736623)
\curveto(462.62380505,32.59736613)(462.5788051,32.56736616)(462.51880493,32.55736623)
\curveto(462.45880522,32.55736617)(462.39880528,32.56236617)(462.33880493,32.57236623)
\curveto(462.26880541,32.59236614)(462.20380547,32.61236612)(462.14380493,32.63236623)
\curveto(462.0738056,32.66236607)(462.02380565,32.68736604)(461.99380493,32.70736623)
\curveto(461.673806,32.84736588)(461.35880632,32.97236576)(461.04880493,33.08236623)
\curveto(460.72880695,33.19236554)(460.40880727,33.31236542)(460.08880493,33.44236623)
\curveto(459.86880781,33.5323652)(459.65380802,33.61736511)(459.44380493,33.69736623)
\curveto(459.22380845,33.77736495)(459.00380867,33.86236487)(458.78380493,33.95236623)
\curveto(458.06380961,34.25236448)(457.33881034,34.53736419)(456.60880493,34.80736623)
\curveto(455.86881181,35.07736365)(455.13381254,35.36236337)(454.40380493,35.66236623)
\curveto(454.14381353,35.77236296)(453.8788138,35.87236286)(453.60880493,35.96236623)
\curveto(453.33881434,36.06236267)(453.0738146,36.16736256)(452.81380493,36.27736623)
\curveto(452.70381497,36.3273624)(452.58381509,36.37236236)(452.45380493,36.41236623)
\curveto(452.31381536,36.46236227)(452.21381546,36.5323622)(452.15380493,36.62236623)
\curveto(452.11381556,36.66236207)(452.08381559,36.727362)(452.06380493,36.81736623)
\curveto(452.05381562,36.83736189)(452.05381562,36.85736187)(452.06380493,36.87736623)
\curveto(452.06381561,36.90736182)(452.05881562,36.9323618)(452.04880493,36.95236623)
\curveto(452.04881563,37.1323616)(452.04881563,37.34236139)(452.04880493,37.58236623)
\curveto(452.03881564,37.82236091)(452.0738156,37.99736073)(452.15380493,38.10736623)
\curveto(452.21381546,38.18736054)(452.31381536,38.24736048)(452.45380493,38.28736623)
\curveto(452.58381509,38.33736039)(452.70381497,38.38736034)(452.81380493,38.43736623)
\curveto(453.04381463,38.53736019)(453.2738144,38.6273601)(453.50380493,38.70736623)
\curveto(453.73381394,38.78735994)(453.96381371,38.87735985)(454.19380493,38.97736623)
\curveto(454.39381328,39.05735967)(454.59881308,39.1323596)(454.80880493,39.20236623)
\curveto(455.01881266,39.28235945)(455.22381245,39.36735936)(455.42380493,39.45736623)
\curveto(456.15381152,39.75735897)(456.89381078,40.04235869)(457.64380493,40.31236623)
\curveto(458.38380929,40.59235814)(459.11880856,40.88735784)(459.84880493,41.19736623)
\curveto(459.93880774,41.23735749)(460.02380765,41.26735746)(460.10380493,41.28736623)
\curveto(460.18380749,41.31735741)(460.26880741,41.34735738)(460.35880493,41.37736623)
\curveto(460.61880706,41.48735724)(460.88380679,41.59235714)(461.15380493,41.69236623)
\curveto(461.42380625,41.80235693)(461.68880599,41.91235682)(461.94880493,42.02236623)
\moveto(458.30380493,38.81236623)
\curveto(458.2738094,38.90235983)(458.22380945,38.95735977)(458.15380493,38.97736623)
\curveto(458.08380959,39.00735972)(458.00880967,39.01235972)(457.92880493,38.99236623)
\curveto(457.83880984,38.98235975)(457.75380992,38.95735977)(457.67380493,38.91736623)
\curveto(457.58381009,38.88735984)(457.50881017,38.85735987)(457.44880493,38.82736623)
\curveto(457.40881027,38.80735992)(457.3738103,38.79735993)(457.34380493,38.79736623)
\curveto(457.31381036,38.79735993)(457.2788104,38.78735994)(457.23880493,38.76736623)
\lineto(456.99880493,38.67736623)
\curveto(456.90881077,38.65736007)(456.81881086,38.6273601)(456.72880493,38.58736623)
\curveto(456.36881131,38.43736029)(456.00381167,38.30236043)(455.63380493,38.18236623)
\curveto(455.25381242,38.07236066)(454.88381279,37.94236079)(454.52380493,37.79236623)
\curveto(454.41381326,37.74236099)(454.30381337,37.69736103)(454.19380493,37.65736623)
\curveto(454.08381359,37.6273611)(453.9788137,37.58736114)(453.87880493,37.53736623)
\curveto(453.82881385,37.51736121)(453.78381389,37.49236124)(453.74380493,37.46236623)
\curveto(453.69381398,37.44236129)(453.66881401,37.39236134)(453.66880493,37.31236623)
\curveto(453.68881399,37.29236144)(453.70381397,37.27236146)(453.71380493,37.25236623)
\curveto(453.72381395,37.2323615)(453.73881394,37.21236152)(453.75880493,37.19236623)
\curveto(453.80881387,37.15236158)(453.86381381,37.12236161)(453.92380493,37.10236623)
\curveto(453.9738137,37.08236165)(454.02881365,37.06236167)(454.08880493,37.04236623)
\curveto(454.19881348,36.99236174)(454.30881337,36.95236178)(454.41880493,36.92236623)
\curveto(454.52881315,36.89236184)(454.63881304,36.85236188)(454.74880493,36.80236623)
\curveto(455.13881254,36.6323621)(455.53381214,36.48236225)(455.93380493,36.35236623)
\curveto(456.33381134,36.2323625)(456.72381095,36.09236264)(457.10380493,35.93236623)
\lineto(457.25380493,35.87236623)
\curveto(457.30381037,35.86236287)(457.35381032,35.84736288)(457.40380493,35.82736623)
\lineto(457.64380493,35.73736623)
\curveto(457.72380995,35.70736302)(457.80380987,35.68236305)(457.88380493,35.66236623)
\curveto(457.93380974,35.64236309)(457.98880969,35.6323631)(458.04880493,35.63236623)
\curveto(458.10880957,35.64236309)(458.15880952,35.65736307)(458.19880493,35.67736623)
\curveto(458.2788094,35.727363)(458.32380935,35.8323629)(458.33380493,35.99236623)
\lineto(458.33380493,36.44236623)
\lineto(458.33380493,38.04736623)
\curveto(458.33380934,38.15736057)(458.33880934,38.29236044)(458.34880493,38.45236623)
\curveto(458.34880933,38.61236012)(458.33380934,38.73236)(458.30380493,38.81236623)
}
}
{
\newrgbcolor{curcolor}{0 0 0}
\pscustom[linestyle=none,fillstyle=solid,fillcolor=curcolor]
{
\newpath
\moveto(458.69380493,50.56392873)
\curveto(458.74380893,50.57392038)(458.81380886,50.57892038)(458.90380493,50.57892873)
\curveto(458.98380869,50.57892038)(459.04880863,50.57392038)(459.09880493,50.56392873)
\curveto(459.13880854,50.56392039)(459.1788085,50.5589204)(459.21880493,50.54892873)
\lineto(459.33880493,50.54892873)
\curveto(459.41880826,50.52892043)(459.49880818,50.51892044)(459.57880493,50.51892873)
\curveto(459.65880802,50.51892044)(459.73880794,50.50892045)(459.81880493,50.48892873)
\curveto(459.85880782,50.47892048)(459.89880778,50.47392048)(459.93880493,50.47392873)
\curveto(459.96880771,50.47392048)(460.00380767,50.46892049)(460.04380493,50.45892873)
\curveto(460.15380752,50.42892053)(460.25880742,50.39892056)(460.35880493,50.36892873)
\curveto(460.45880722,50.34892061)(460.55880712,50.31892064)(460.65880493,50.27892873)
\curveto(461.00880667,50.13892082)(461.32380635,49.96892099)(461.60380493,49.76892873)
\curveto(461.88380579,49.56892139)(462.12380555,49.31892164)(462.32380493,49.01892873)
\curveto(462.42380525,48.86892209)(462.50880517,48.72392223)(462.57880493,48.58392873)
\curveto(462.62880505,48.47392248)(462.66880501,48.36392259)(462.69880493,48.25392873)
\curveto(462.72880495,48.1539228)(462.75880492,48.04892291)(462.78880493,47.93892873)
\curveto(462.80880487,47.86892309)(462.81880486,47.80392315)(462.81880493,47.74392873)
\curveto(462.82880485,47.68392327)(462.84380483,47.62392333)(462.86380493,47.56392873)
\lineto(462.86380493,47.41392873)
\curveto(462.88380479,47.36392359)(462.89380478,47.28892367)(462.89380493,47.18892873)
\curveto(462.90380477,47.08892387)(462.89880478,47.00892395)(462.87880493,46.94892873)
\lineto(462.87880493,46.79892873)
\curveto(462.86880481,46.7589242)(462.86380481,46.71392424)(462.86380493,46.66392873)
\curveto(462.86380481,46.62392433)(462.85880482,46.57892438)(462.84880493,46.52892873)
\curveto(462.80880487,46.37892458)(462.7738049,46.22892473)(462.74380493,46.07892873)
\curveto(462.71380496,45.93892502)(462.66880501,45.79892516)(462.60880493,45.65892873)
\curveto(462.52880515,45.4589255)(462.42880525,45.27892568)(462.30880493,45.11892873)
\lineto(462.15880493,44.93892873)
\curveto(462.09880558,44.87892608)(462.05880562,44.80892615)(462.03880493,44.72892873)
\curveto(462.02880565,44.66892629)(462.04380563,44.61892634)(462.08380493,44.57892873)
\curveto(462.11380556,44.54892641)(462.15880552,44.52392643)(462.21880493,44.50392873)
\curveto(462.2788054,44.49392646)(462.34380533,44.48392647)(462.41380493,44.47392873)
\curveto(462.4738052,44.47392648)(462.51880516,44.46392649)(462.54880493,44.44392873)
\curveto(462.59880508,44.40392655)(462.64380503,44.3589266)(462.68380493,44.30892873)
\curveto(462.70380497,44.2589267)(462.71880496,44.18892677)(462.72880493,44.09892873)
\lineto(462.72880493,43.82892873)
\curveto(462.72880495,43.73892722)(462.72380495,43.6539273)(462.71380493,43.57392873)
\curveto(462.69380498,43.49392746)(462.673805,43.43392752)(462.65380493,43.39392873)
\curveto(462.63380504,43.37392758)(462.60880507,43.3539276)(462.57880493,43.33392873)
\lineto(462.48880493,43.27392873)
\curveto(462.40880527,43.24392771)(462.28880539,43.22892773)(462.12880493,43.22892873)
\curveto(461.96880571,43.23892772)(461.83380584,43.24392771)(461.72380493,43.24392873)
\lineto(452.91880493,43.24392873)
\curveto(452.79881488,43.24392771)(452.673815,43.23892772)(452.54380493,43.22892873)
\curveto(452.40381527,43.22892773)(452.29381538,43.2539277)(452.21380493,43.30392873)
\curveto(452.15381552,43.34392761)(452.10381557,43.40892755)(452.06380493,43.49892873)
\curveto(452.06381561,43.51892744)(452.06381561,43.54392741)(452.06380493,43.57392873)
\curveto(452.05381562,43.60392735)(452.04881563,43.62892733)(452.04880493,43.64892873)
\curveto(452.03881564,43.78892717)(452.03881564,43.93392702)(452.04880493,44.08392873)
\curveto(452.04881563,44.24392671)(452.08881559,44.3539266)(452.16880493,44.41392873)
\curveto(452.24881543,44.46392649)(452.36381531,44.48892647)(452.51380493,44.48892873)
\lineto(452.91880493,44.48892873)
\lineto(454.67380493,44.48892873)
\lineto(454.92880493,44.48892873)
\lineto(455.21380493,44.48892873)
\curveto(455.30381237,44.49892646)(455.38881229,44.50892645)(455.46880493,44.51892873)
\curveto(455.53881214,44.53892642)(455.58881209,44.56892639)(455.61880493,44.60892873)
\curveto(455.64881203,44.64892631)(455.65381202,44.69392626)(455.63380493,44.74392873)
\curveto(455.61381206,44.79392616)(455.59381208,44.83392612)(455.57380493,44.86392873)
\curveto(455.53381214,44.91392604)(455.49381218,44.958926)(455.45380493,44.99892873)
\lineto(455.33380493,45.14892873)
\curveto(455.28381239,45.21892574)(455.23881244,45.28892567)(455.19880493,45.35892873)
\lineto(455.07880493,45.59892873)
\curveto(454.98881269,45.77892518)(454.92381275,45.99392496)(454.88380493,46.24392873)
\curveto(454.84381283,46.49392446)(454.82381285,46.74892421)(454.82380493,47.00892873)
\curveto(454.82381285,47.26892369)(454.84881283,47.52392343)(454.89880493,47.77392873)
\curveto(454.93881274,48.02392293)(454.99881268,48.24392271)(455.07880493,48.43392873)
\curveto(455.24881243,48.83392212)(455.48381219,49.17892178)(455.78380493,49.46892873)
\curveto(456.08381159,49.7589212)(456.43381124,49.98892097)(456.83380493,50.15892873)
\curveto(456.94381073,50.20892075)(457.05381062,50.24892071)(457.16380493,50.27892873)
\curveto(457.26381041,50.31892064)(457.36881031,50.3589206)(457.47880493,50.39892873)
\curveto(457.58881009,50.42892053)(457.70380997,50.44892051)(457.82380493,50.45892873)
\lineto(458.15380493,50.51892873)
\curveto(458.18380949,50.52892043)(458.21880946,50.53392042)(458.25880493,50.53392873)
\curveto(458.28880939,50.53392042)(458.31880936,50.53892042)(458.34880493,50.54892873)
\curveto(458.40880927,50.56892039)(458.46880921,50.56892039)(458.52880493,50.54892873)
\curveto(458.5788091,50.53892042)(458.63380904,50.54392041)(458.69380493,50.56392873)
\moveto(459.08380493,49.22892873)
\curveto(459.03380864,49.24892171)(458.9738087,49.2539217)(458.90380493,49.24392873)
\curveto(458.83380884,49.23392172)(458.76880891,49.22892173)(458.70880493,49.22892873)
\curveto(458.53880914,49.22892173)(458.3788093,49.21892174)(458.22880493,49.19892873)
\curveto(458.0788096,49.18892177)(457.94380973,49.1589218)(457.82380493,49.10892873)
\curveto(457.72380995,49.07892188)(457.63381004,49.0539219)(457.55380493,49.03392873)
\curveto(457.4738102,49.01392194)(457.39381028,48.98392197)(457.31380493,48.94392873)
\curveto(457.06381061,48.83392212)(456.83381084,48.68392227)(456.62380493,48.49392873)
\curveto(456.40381127,48.30392265)(456.23881144,48.08392287)(456.12880493,47.83392873)
\curveto(456.09881158,47.7539232)(456.0738116,47.67392328)(456.05380493,47.59392873)
\curveto(456.02381165,47.52392343)(455.99881168,47.44892351)(455.97880493,47.36892873)
\curveto(455.94881173,47.2589237)(455.93381174,47.14892381)(455.93380493,47.03892873)
\curveto(455.92381175,46.92892403)(455.91881176,46.80892415)(455.91880493,46.67892873)
\curveto(455.92881175,46.62892433)(455.93881174,46.58392437)(455.94880493,46.54392873)
\lineto(455.94880493,46.40892873)
\lineto(456.00880493,46.13892873)
\curveto(456.02881165,46.0589249)(456.05881162,45.97892498)(456.09880493,45.89892873)
\curveto(456.23881144,45.5589254)(456.44881123,45.28892567)(456.72880493,45.08892873)
\curveto(456.99881068,44.88892607)(457.31881036,44.72892623)(457.68880493,44.60892873)
\curveto(457.79880988,44.56892639)(457.90880977,44.54392641)(458.01880493,44.53392873)
\curveto(458.12880955,44.52392643)(458.24380943,44.50392645)(458.36380493,44.47392873)
\curveto(458.41380926,44.46392649)(458.45880922,44.46392649)(458.49880493,44.47392873)
\curveto(458.53880914,44.48392647)(458.58380909,44.47892648)(458.63380493,44.45892873)
\curveto(458.68380899,44.44892651)(458.75880892,44.44392651)(458.85880493,44.44392873)
\curveto(458.94880873,44.44392651)(459.01880866,44.44892651)(459.06880493,44.45892873)
\lineto(459.18880493,44.45892873)
\curveto(459.22880845,44.46892649)(459.26880841,44.47392648)(459.30880493,44.47392873)
\curveto(459.34880833,44.47392648)(459.38380829,44.47892648)(459.41380493,44.48892873)
\curveto(459.44380823,44.49892646)(459.4788082,44.50392645)(459.51880493,44.50392873)
\curveto(459.54880813,44.50392645)(459.5788081,44.50892645)(459.60880493,44.51892873)
\curveto(459.68880799,44.53892642)(459.76880791,44.5539264)(459.84880493,44.56392873)
\lineto(460.08880493,44.62392873)
\curveto(460.42880725,44.73392622)(460.71880696,44.88392607)(460.95880493,45.07392873)
\curveto(461.19880648,45.27392568)(461.39880628,45.51892544)(461.55880493,45.80892873)
\curveto(461.60880607,45.89892506)(461.64880603,45.99392496)(461.67880493,46.09392873)
\curveto(461.69880598,46.19392476)(461.72380595,46.29892466)(461.75380493,46.40892873)
\curveto(461.7738059,46.4589245)(461.78380589,46.50392445)(461.78380493,46.54392873)
\curveto(461.7738059,46.59392436)(461.7738059,46.64392431)(461.78380493,46.69392873)
\curveto(461.79380588,46.73392422)(461.79880588,46.77892418)(461.79880493,46.82892873)
\lineto(461.79880493,46.96392873)
\lineto(461.79880493,47.09892873)
\curveto(461.78880589,47.13892382)(461.78380589,47.17392378)(461.78380493,47.20392873)
\curveto(461.78380589,47.23392372)(461.7788059,47.26892369)(461.76880493,47.30892873)
\curveto(461.74880593,47.38892357)(461.73380594,47.46392349)(461.72380493,47.53392873)
\curveto(461.70380597,47.60392335)(461.678806,47.67892328)(461.64880493,47.75892873)
\curveto(461.51880616,48.06892289)(461.34880633,48.31892264)(461.13880493,48.50892873)
\curveto(460.91880676,48.69892226)(460.65380702,48.8589221)(460.34380493,48.98892873)
\curveto(460.20380747,49.03892192)(460.06380761,49.07392188)(459.92380493,49.09392873)
\curveto(459.7738079,49.12392183)(459.62380805,49.1589218)(459.47380493,49.19892873)
\curveto(459.42380825,49.21892174)(459.3788083,49.22392173)(459.33880493,49.21392873)
\curveto(459.28880839,49.21392174)(459.23880844,49.21892174)(459.18880493,49.22892873)
\lineto(459.08380493,49.22892873)
}
}
{
\newrgbcolor{curcolor}{0 0 0}
\pscustom[linestyle=none,fillstyle=solid,fillcolor=curcolor]
{
\newpath
\moveto(454.82380493,55.69017873)
\curveto(454.82381285,55.92017394)(454.88381279,56.05017381)(455.00380493,56.08017873)
\curveto(455.11381256,56.11017375)(455.2788124,56.12517374)(455.49880493,56.12517873)
\lineto(455.78380493,56.12517873)
\curveto(455.8738118,56.12517374)(455.94881173,56.10017376)(456.00880493,56.05017873)
\curveto(456.08881159,55.99017387)(456.13381154,55.90517396)(456.14380493,55.79517873)
\curveto(456.14381153,55.68517418)(456.15881152,55.57517429)(456.18880493,55.46517873)
\curveto(456.21881146,55.32517454)(456.24881143,55.19017467)(456.27880493,55.06017873)
\curveto(456.30881137,54.94017492)(456.34881133,54.82517504)(456.39880493,54.71517873)
\curveto(456.52881115,54.42517544)(456.70881097,54.19017567)(456.93880493,54.01017873)
\curveto(457.15881052,53.83017603)(457.41381026,53.67517619)(457.70380493,53.54517873)
\curveto(457.81380986,53.50517636)(457.92880975,53.47517639)(458.04880493,53.45517873)
\curveto(458.15880952,53.43517643)(458.2738094,53.41017645)(458.39380493,53.38017873)
\curveto(458.44380923,53.37017649)(458.49380918,53.3651765)(458.54380493,53.36517873)
\curveto(458.59380908,53.37517649)(458.64380903,53.37517649)(458.69380493,53.36517873)
\curveto(458.81380886,53.33517653)(458.95380872,53.32017654)(459.11380493,53.32017873)
\curveto(459.26380841,53.33017653)(459.40880827,53.33517653)(459.54880493,53.33517873)
\lineto(461.39380493,53.33517873)
\lineto(461.73880493,53.33517873)
\curveto(461.85880582,53.33517653)(461.9738057,53.33017653)(462.08380493,53.32017873)
\curveto(462.19380548,53.31017655)(462.28880539,53.30517656)(462.36880493,53.30517873)
\curveto(462.44880523,53.31517655)(462.51880516,53.29517657)(462.57880493,53.24517873)
\curveto(462.64880503,53.19517667)(462.68880499,53.11517675)(462.69880493,53.00517873)
\curveto(462.70880497,52.90517696)(462.71380496,52.79517707)(462.71380493,52.67517873)
\lineto(462.71380493,52.40517873)
\curveto(462.69380498,52.35517751)(462.678805,52.30517756)(462.66880493,52.25517873)
\curveto(462.64880503,52.21517765)(462.62380505,52.18517768)(462.59380493,52.16517873)
\curveto(462.52380515,52.11517775)(462.43880524,52.08517778)(462.33880493,52.07517873)
\lineto(462.00880493,52.07517873)
\lineto(460.85380493,52.07517873)
\lineto(456.69880493,52.07517873)
\lineto(455.66380493,52.07517873)
\lineto(455.36380493,52.07517873)
\curveto(455.26381241,52.08517778)(455.1788125,52.11517775)(455.10880493,52.16517873)
\curveto(455.06881261,52.19517767)(455.03881264,52.24517762)(455.01880493,52.31517873)
\curveto(454.99881268,52.39517747)(454.98881269,52.48017738)(454.98880493,52.57017873)
\curveto(454.9788127,52.6601772)(454.9788127,52.75017711)(454.98880493,52.84017873)
\curveto(454.99881268,52.93017693)(455.01381266,53.00017686)(455.03380493,53.05017873)
\curveto(455.06381261,53.13017673)(455.12381255,53.18017668)(455.21380493,53.20017873)
\curveto(455.29381238,53.23017663)(455.38381229,53.24517662)(455.48380493,53.24517873)
\lineto(455.78380493,53.24517873)
\curveto(455.88381179,53.24517662)(455.9738117,53.2651766)(456.05380493,53.30517873)
\curveto(456.0738116,53.31517655)(456.08881159,53.32517654)(456.09880493,53.33517873)
\lineto(456.14380493,53.38017873)
\curveto(456.14381153,53.49017637)(456.09881158,53.58017628)(456.00880493,53.65017873)
\curveto(455.90881177,53.72017614)(455.82881185,53.78017608)(455.76880493,53.83017873)
\lineto(455.67880493,53.92017873)
\curveto(455.56881211,54.01017585)(455.45381222,54.13517573)(455.33380493,54.29517873)
\curveto(455.21381246,54.45517541)(455.12381255,54.60517526)(455.06380493,54.74517873)
\curveto(455.01381266,54.83517503)(454.9788127,54.93017493)(454.95880493,55.03017873)
\curveto(454.92881275,55.13017473)(454.89881278,55.23517463)(454.86880493,55.34517873)
\curveto(454.85881282,55.40517446)(454.85381282,55.4651744)(454.85380493,55.52517873)
\curveto(454.84381283,55.58517428)(454.83381284,55.64017422)(454.82380493,55.69017873)
}
}
{
\newrgbcolor{curcolor}{0 0 0}
\pscustom[linestyle=none,fillstyle=solid,fillcolor=curcolor]
{
}
}
{
\newrgbcolor{curcolor}{0 0 0}
\pscustom[linestyle=none,fillstyle=solid,fillcolor=curcolor]
{
\newpath
\moveto(452.12380493,64.24510061)
\curveto(452.11381556,64.93509597)(452.23381544,65.53509537)(452.48380493,66.04510061)
\curveto(452.73381494,66.56509434)(453.06881461,66.96009395)(453.48880493,67.23010061)
\curveto(453.56881411,67.28009363)(453.65881402,67.32509358)(453.75880493,67.36510061)
\curveto(453.84881383,67.4050935)(453.94381373,67.45009346)(454.04380493,67.50010061)
\curveto(454.14381353,67.54009337)(454.24381343,67.57009334)(454.34380493,67.59010061)
\curveto(454.44381323,67.6100933)(454.54881313,67.63009328)(454.65880493,67.65010061)
\curveto(454.70881297,67.67009324)(454.75381292,67.67509323)(454.79380493,67.66510061)
\curveto(454.83381284,67.65509325)(454.8788128,67.66009325)(454.92880493,67.68010061)
\curveto(454.9788127,67.69009322)(455.06381261,67.69509321)(455.18380493,67.69510061)
\curveto(455.29381238,67.69509321)(455.3788123,67.69009322)(455.43880493,67.68010061)
\curveto(455.49881218,67.66009325)(455.55881212,67.65009326)(455.61880493,67.65010061)
\curveto(455.678812,67.66009325)(455.73881194,67.65509325)(455.79880493,67.63510061)
\curveto(455.93881174,67.59509331)(456.0738116,67.56009335)(456.20380493,67.53010061)
\curveto(456.33381134,67.50009341)(456.45881122,67.46009345)(456.57880493,67.41010061)
\curveto(456.71881096,67.35009356)(456.84381083,67.28009363)(456.95380493,67.20010061)
\curveto(457.06381061,67.13009378)(457.1738105,67.05509385)(457.28380493,66.97510061)
\lineto(457.34380493,66.91510061)
\curveto(457.36381031,66.905094)(457.38381029,66.89009402)(457.40380493,66.87010061)
\curveto(457.56381011,66.75009416)(457.70880997,66.61509429)(457.83880493,66.46510061)
\curveto(457.96880971,66.31509459)(458.09380958,66.15509475)(458.21380493,65.98510061)
\curveto(458.43380924,65.67509523)(458.63880904,65.38009553)(458.82880493,65.10010061)
\curveto(458.96880871,64.87009604)(459.10380857,64.64009627)(459.23380493,64.41010061)
\curveto(459.36380831,64.19009672)(459.49880818,63.97009694)(459.63880493,63.75010061)
\curveto(459.80880787,63.50009741)(459.98880769,63.26009765)(460.17880493,63.03010061)
\curveto(460.36880731,62.8100981)(460.59380708,62.62009829)(460.85380493,62.46010061)
\curveto(460.91380676,62.42009849)(460.9738067,62.38509852)(461.03380493,62.35510061)
\curveto(461.08380659,62.32509858)(461.14880653,62.29509861)(461.22880493,62.26510061)
\curveto(461.29880638,62.24509866)(461.35880632,62.24009867)(461.40880493,62.25010061)
\curveto(461.4788062,62.27009864)(461.53380614,62.3050986)(461.57380493,62.35510061)
\curveto(461.60380607,62.4050985)(461.62380605,62.46509844)(461.63380493,62.53510061)
\lineto(461.63380493,62.77510061)
\lineto(461.63380493,63.52510061)
\lineto(461.63380493,66.33010061)
\lineto(461.63380493,66.99010061)
\curveto(461.63380604,67.08009383)(461.63880604,67.16509374)(461.64880493,67.24510061)
\curveto(461.64880603,67.32509358)(461.66880601,67.39009352)(461.70880493,67.44010061)
\curveto(461.74880593,67.49009342)(461.82380585,67.53009338)(461.93380493,67.56010061)
\curveto(462.03380564,67.60009331)(462.13380554,67.6100933)(462.23380493,67.59010061)
\lineto(462.36880493,67.59010061)
\curveto(462.43880524,67.57009334)(462.49880518,67.55009336)(462.54880493,67.53010061)
\curveto(462.59880508,67.5100934)(462.63880504,67.47509343)(462.66880493,67.42510061)
\curveto(462.70880497,67.37509353)(462.72880495,67.3050936)(462.72880493,67.21510061)
\lineto(462.72880493,66.94510061)
\lineto(462.72880493,66.04510061)
\lineto(462.72880493,62.53510061)
\lineto(462.72880493,61.47010061)
\curveto(462.72880495,61.39009952)(462.73380494,61.30009961)(462.74380493,61.20010061)
\curveto(462.74380493,61.10009981)(462.73380494,61.01509989)(462.71380493,60.94510061)
\curveto(462.64380503,60.73510017)(462.46380521,60.67010024)(462.17380493,60.75010061)
\curveto(462.13380554,60.76010015)(462.09880558,60.76010015)(462.06880493,60.75010061)
\curveto(462.02880565,60.75010016)(461.98380569,60.76010015)(461.93380493,60.78010061)
\curveto(461.85380582,60.80010011)(461.76880591,60.82010009)(461.67880493,60.84010061)
\curveto(461.58880609,60.86010005)(461.50380617,60.88510002)(461.42380493,60.91510061)
\curveto(460.93380674,61.07509983)(460.51880716,61.27509963)(460.17880493,61.51510061)
\curveto(459.92880775,61.69509921)(459.70380797,61.90009901)(459.50380493,62.13010061)
\curveto(459.29380838,62.36009855)(459.09880858,62.60009831)(458.91880493,62.85010061)
\curveto(458.73880894,63.1100978)(458.56880911,63.37509753)(458.40880493,63.64510061)
\curveto(458.23880944,63.92509698)(458.06380961,64.19509671)(457.88380493,64.45510061)
\curveto(457.80380987,64.56509634)(457.72880995,64.67009624)(457.65880493,64.77010061)
\curveto(457.58881009,64.88009603)(457.51381016,64.99009592)(457.43380493,65.10010061)
\curveto(457.40381027,65.14009577)(457.3738103,65.17509573)(457.34380493,65.20510061)
\curveto(457.30381037,65.24509566)(457.2738104,65.28509562)(457.25380493,65.32510061)
\curveto(457.14381053,65.46509544)(457.01881066,65.59009532)(456.87880493,65.70010061)
\curveto(456.84881083,65.72009519)(456.82381085,65.74509516)(456.80380493,65.77510061)
\curveto(456.7738109,65.8050951)(456.74381093,65.83009508)(456.71380493,65.85010061)
\curveto(456.61381106,65.93009498)(456.51381116,65.99509491)(456.41380493,66.04510061)
\curveto(456.31381136,66.1050948)(456.20381147,66.16009475)(456.08380493,66.21010061)
\curveto(456.01381166,66.24009467)(455.93881174,66.26009465)(455.85880493,66.27010061)
\lineto(455.61880493,66.33010061)
\lineto(455.52880493,66.33010061)
\curveto(455.49881218,66.34009457)(455.46881221,66.34509456)(455.43880493,66.34510061)
\curveto(455.36881231,66.36509454)(455.2738124,66.37009454)(455.15380493,66.36010061)
\curveto(455.02381265,66.36009455)(454.92381275,66.35009456)(454.85380493,66.33010061)
\curveto(454.7738129,66.3100946)(454.69881298,66.29009462)(454.62880493,66.27010061)
\curveto(454.54881313,66.26009465)(454.46881321,66.24009467)(454.38880493,66.21010061)
\curveto(454.14881353,66.10009481)(453.94881373,65.95009496)(453.78880493,65.76010061)
\curveto(453.61881406,65.58009533)(453.4788142,65.36009555)(453.36880493,65.10010061)
\curveto(453.34881433,65.03009588)(453.33381434,64.96009595)(453.32380493,64.89010061)
\curveto(453.30381437,64.82009609)(453.28381439,64.74509616)(453.26380493,64.66510061)
\curveto(453.24381443,64.58509632)(453.23381444,64.47509643)(453.23380493,64.33510061)
\curveto(453.23381444,64.2050967)(453.24381443,64.10009681)(453.26380493,64.02010061)
\curveto(453.2738144,63.96009695)(453.2788144,63.905097)(453.27880493,63.85510061)
\curveto(453.2788144,63.8050971)(453.28881439,63.75509715)(453.30880493,63.70510061)
\curveto(453.34881433,63.6050973)(453.38881429,63.5100974)(453.42880493,63.42010061)
\curveto(453.46881421,63.34009757)(453.51381416,63.26009765)(453.56380493,63.18010061)
\curveto(453.58381409,63.15009776)(453.60881407,63.12009779)(453.63880493,63.09010061)
\curveto(453.66881401,63.07009784)(453.69381398,63.04509786)(453.71380493,63.01510061)
\lineto(453.78880493,62.94010061)
\curveto(453.80881387,62.910098)(453.82881385,62.88509802)(453.84880493,62.86510061)
\lineto(454.05880493,62.71510061)
\curveto(454.11881356,62.67509823)(454.18381349,62.63009828)(454.25380493,62.58010061)
\curveto(454.34381333,62.52009839)(454.44881323,62.47009844)(454.56880493,62.43010061)
\curveto(454.678813,62.40009851)(454.78881289,62.36509854)(454.89880493,62.32510061)
\curveto(455.00881267,62.28509862)(455.15381252,62.26009865)(455.33380493,62.25010061)
\curveto(455.50381217,62.24009867)(455.62881205,62.2100987)(455.70880493,62.16010061)
\curveto(455.78881189,62.1100988)(455.83381184,62.03509887)(455.84380493,61.93510061)
\curveto(455.85381182,61.83509907)(455.85881182,61.72509918)(455.85880493,61.60510061)
\curveto(455.85881182,61.56509934)(455.86381181,61.52509938)(455.87380493,61.48510061)
\curveto(455.8738118,61.44509946)(455.86881181,61.4100995)(455.85880493,61.38010061)
\curveto(455.83881184,61.33009958)(455.82881185,61.28009963)(455.82880493,61.23010061)
\curveto(455.82881185,61.19009972)(455.81881186,61.15009976)(455.79880493,61.11010061)
\curveto(455.73881194,61.02009989)(455.60381207,60.97509993)(455.39380493,60.97510061)
\lineto(455.27380493,60.97510061)
\curveto(455.21381246,60.98509992)(455.15381252,60.99009992)(455.09380493,60.99010061)
\curveto(455.02381265,61.00009991)(454.95881272,61.0100999)(454.89880493,61.02010061)
\curveto(454.78881289,61.04009987)(454.68881299,61.06009985)(454.59880493,61.08010061)
\curveto(454.49881318,61.10009981)(454.40381327,61.13009978)(454.31380493,61.17010061)
\curveto(454.24381343,61.19009972)(454.18381349,61.2100997)(454.13380493,61.23010061)
\lineto(453.95380493,61.29010061)
\curveto(453.69381398,61.4100995)(453.44881423,61.56509934)(453.21880493,61.75510061)
\curveto(452.98881469,61.95509895)(452.80381487,62.17009874)(452.66380493,62.40010061)
\curveto(452.58381509,62.5100984)(452.51881516,62.62509828)(452.46880493,62.74510061)
\lineto(452.31880493,63.13510061)
\curveto(452.26881541,63.24509766)(452.23881544,63.36009755)(452.22880493,63.48010061)
\curveto(452.20881547,63.60009731)(452.18381549,63.72509718)(452.15380493,63.85510061)
\curveto(452.15381552,63.92509698)(452.15381552,63.99009692)(452.15380493,64.05010061)
\curveto(452.14381553,64.1100968)(452.13381554,64.17509673)(452.12380493,64.24510061)
}
}
{
\newrgbcolor{curcolor}{0 0 0}
\pscustom[linestyle=none,fillstyle=solid,fillcolor=curcolor]
{
\newpath
\moveto(452.31880493,69.80470998)
\lineto(452.31880493,74.60470998)
\lineto(452.31880493,75.60970998)
\curveto(452.31881536,75.74970288)(452.32881535,75.86970276)(452.34880493,75.96970998)
\curveto(452.35881532,76.07970255)(452.40381527,76.15970247)(452.48380493,76.20970998)
\curveto(452.52381515,76.2297024)(452.5738151,76.23970239)(452.63380493,76.23970998)
\curveto(452.69381498,76.24970238)(452.75881492,76.25470238)(452.82880493,76.25470998)
\lineto(453.09880493,76.25470998)
\curveto(453.18881449,76.25470238)(453.26881441,76.24470239)(453.33880493,76.22470998)
\curveto(453.41881426,76.18470245)(453.48881419,76.13970249)(453.54880493,76.08970998)
\lineto(453.72880493,75.93970998)
\curveto(453.7788139,75.90970272)(453.81881386,75.87470276)(453.84880493,75.83470998)
\curveto(453.8788138,75.79470284)(453.91881376,75.75470288)(453.96880493,75.71470998)
\curveto(454.0788136,75.634703)(454.18881349,75.54970308)(454.29880493,75.45970998)
\curveto(454.39881328,75.36970326)(454.50381317,75.28470335)(454.61380493,75.20470998)
\curveto(454.81381286,75.06470357)(455.02381265,74.92470371)(455.24380493,74.78470998)
\curveto(455.45381222,74.64470399)(455.66881201,74.50470413)(455.88880493,74.36470998)
\curveto(455.9788117,74.31470432)(456.0738116,74.26470437)(456.17380493,74.21470998)
\curveto(456.2738114,74.16470447)(456.36881131,74.10970452)(456.45880493,74.04970998)
\curveto(456.4788112,74.0297046)(456.50381117,74.01970461)(456.53380493,74.01970998)
\curveto(456.56381111,74.01970461)(456.58881109,74.00970462)(456.60880493,73.98970998)
\curveto(456.70881097,73.91970471)(456.82381085,73.85470478)(456.95380493,73.79470998)
\curveto(457.0738106,73.7347049)(457.18881049,73.67970495)(457.29880493,73.62970998)
\curveto(457.52881015,73.5297051)(457.76380991,73.4347052)(458.00380493,73.34470998)
\curveto(458.24380943,73.25470538)(458.48380919,73.15470548)(458.72380493,73.04470998)
\curveto(458.7738089,73.02470561)(458.81880886,73.00970562)(458.85880493,72.99970998)
\curveto(458.89880878,72.99970563)(458.94380873,72.98970564)(458.99380493,72.96970998)
\curveto(459.11380856,72.91970571)(459.23880844,72.87470576)(459.36880493,72.83470998)
\curveto(459.48880819,72.80470583)(459.60880807,72.76970586)(459.72880493,72.72970998)
\curveto(459.95880772,72.64970598)(460.19880748,72.58470605)(460.44880493,72.53470998)
\curveto(460.68880699,72.49470614)(460.92880675,72.44470619)(461.16880493,72.38470998)
\curveto(461.31880636,72.34470629)(461.46880621,72.31970631)(461.61880493,72.30970998)
\curveto(461.76880591,72.29970633)(461.91880576,72.27970635)(462.06880493,72.24970998)
\curveto(462.10880557,72.23970639)(462.16880551,72.2347064)(462.24880493,72.23470998)
\curveto(462.36880531,72.20470643)(462.46880521,72.17470646)(462.54880493,72.14470998)
\curveto(462.62880505,72.11470652)(462.68380499,72.04470659)(462.71380493,71.93470998)
\curveto(462.73380494,71.88470675)(462.74380493,71.8297068)(462.74380493,71.76970998)
\lineto(462.74380493,71.57470998)
\curveto(462.74380493,71.4347072)(462.73880494,71.29470734)(462.72880493,71.15470998)
\curveto(462.71880496,71.02470761)(462.673805,70.9297077)(462.59380493,70.86970998)
\curveto(462.53380514,70.8297078)(462.44880523,70.80970782)(462.33880493,70.80970998)
\curveto(462.22880545,70.81970781)(462.13380554,70.8347078)(462.05380493,70.85470998)
\lineto(461.97880493,70.85470998)
\curveto(461.94880573,70.86470777)(461.91880576,70.86970776)(461.88880493,70.86970998)
\curveto(461.80880587,70.88970774)(461.73380594,70.89970773)(461.66380493,70.89970998)
\curveto(461.59380608,70.89970773)(461.52380615,70.90970772)(461.45380493,70.92970998)
\curveto(461.26380641,70.97970765)(461.0788066,71.01970761)(460.89880493,71.04970998)
\curveto(460.70880697,71.07970755)(460.52880715,71.11970751)(460.35880493,71.16970998)
\curveto(460.30880737,71.18970744)(460.26880741,71.19970743)(460.23880493,71.19970998)
\curveto(460.20880747,71.19970743)(460.1738075,71.20470743)(460.13380493,71.21470998)
\curveto(459.83380784,71.31470732)(459.53880814,71.40470723)(459.24880493,71.48470998)
\curveto(458.95880872,71.57470706)(458.678809,71.67970695)(458.40880493,71.79970998)
\curveto(457.82880985,72.05970657)(457.2788104,72.3297063)(456.75880493,72.60970998)
\curveto(456.22881145,72.88970574)(455.72381195,73.19970543)(455.24380493,73.53970998)
\curveto(455.04381263,73.67970495)(454.85381282,73.8297048)(454.67380493,73.98970998)
\curveto(454.48381319,74.14970448)(454.29381338,74.29970433)(454.10380493,74.43970998)
\curveto(454.05381362,74.47970415)(454.00881367,74.51470412)(453.96880493,74.54470998)
\curveto(453.91881376,74.58470405)(453.86881381,74.61970401)(453.81880493,74.64970998)
\curveto(453.79881388,74.65970397)(453.7738139,74.66970396)(453.74380493,74.67970998)
\curveto(453.71381396,74.69970393)(453.68381399,74.69970393)(453.65380493,74.67970998)
\curveto(453.59381408,74.65970397)(453.55881412,74.62470401)(453.54880493,74.57470998)
\curveto(453.52881415,74.52470411)(453.50881417,74.47470416)(453.48880493,74.42470998)
\lineto(453.48880493,74.31970998)
\curveto(453.4788142,74.27970435)(453.4788142,74.2297044)(453.48880493,74.16970998)
\lineto(453.48880493,74.01970998)
\lineto(453.48880493,73.41970998)
\lineto(453.48880493,70.77970998)
\lineto(453.48880493,70.04470998)
\lineto(453.48880493,69.80470998)
\curveto(453.4788142,69.7347089)(453.46381421,69.67470896)(453.44380493,69.62470998)
\curveto(453.40381427,69.5347091)(453.34381433,69.47470916)(453.26380493,69.44470998)
\curveto(453.16381451,69.39470924)(453.01881466,69.37970925)(452.82880493,69.39970998)
\curveto(452.62881505,69.41970921)(452.49381518,69.45470918)(452.42380493,69.50470998)
\curveto(452.40381527,69.52470911)(452.38881529,69.54970908)(452.37880493,69.57970998)
\lineto(452.31880493,69.69970998)
\curveto(452.31881536,69.71970891)(452.32381535,69.7347089)(452.33380493,69.74470998)
\curveto(452.33381534,69.76470887)(452.32881535,69.78470885)(452.31880493,69.80470998)
}
}
{
\newrgbcolor{curcolor}{0 0 0}
\pscustom[linestyle=none,fillstyle=solid,fillcolor=curcolor]
{
\newpath
\moveto(461.09380493,78.63431936)
\lineto(461.09380493,79.26431936)
\lineto(461.09380493,79.45931936)
\curveto(461.09380658,79.52931683)(461.10380657,79.58931677)(461.12380493,79.63931936)
\curveto(461.16380651,79.70931665)(461.20380647,79.7593166)(461.24380493,79.78931936)
\curveto(461.29380638,79.82931653)(461.35880632,79.84931651)(461.43880493,79.84931936)
\curveto(461.51880616,79.8593165)(461.60380607,79.86431649)(461.69380493,79.86431936)
\lineto(462.41380493,79.86431936)
\curveto(462.89380478,79.86431649)(463.30380437,79.80431655)(463.64380493,79.68431936)
\curveto(463.98380369,79.56431679)(464.25880342,79.36931699)(464.46880493,79.09931936)
\curveto(464.51880316,79.02931733)(464.56380311,78.9593174)(464.60380493,78.88931936)
\curveto(464.65380302,78.82931753)(464.69880298,78.7543176)(464.73880493,78.66431936)
\curveto(464.74880293,78.64431771)(464.75880292,78.61431774)(464.76880493,78.57431936)
\curveto(464.78880289,78.53431782)(464.79380288,78.48931787)(464.78380493,78.43931936)
\curveto(464.75380292,78.34931801)(464.678803,78.29431806)(464.55880493,78.27431936)
\curveto(464.44880323,78.2543181)(464.35380332,78.26931809)(464.27380493,78.31931936)
\curveto(464.20380347,78.34931801)(464.13880354,78.39431796)(464.07880493,78.45431936)
\curveto(464.02880365,78.52431783)(463.9788037,78.58931777)(463.92880493,78.64931936)
\curveto(463.8788038,78.71931764)(463.80380387,78.77931758)(463.70380493,78.82931936)
\curveto(463.61380406,78.88931747)(463.52380415,78.93931742)(463.43380493,78.97931936)
\curveto(463.40380427,78.99931736)(463.34380433,79.02431733)(463.25380493,79.05431936)
\curveto(463.1738045,79.08431727)(463.10380457,79.08931727)(463.04380493,79.06931936)
\curveto(462.90380477,79.03931732)(462.81380486,78.97931738)(462.77380493,78.88931936)
\curveto(462.74380493,78.80931755)(462.72880495,78.71931764)(462.72880493,78.61931936)
\curveto(462.72880495,78.51931784)(462.70380497,78.43431792)(462.65380493,78.36431936)
\curveto(462.58380509,78.27431808)(462.44380523,78.22931813)(462.23380493,78.22931936)
\lineto(461.67880493,78.22931936)
\lineto(461.45380493,78.22931936)
\curveto(461.3738063,78.23931812)(461.30880637,78.2593181)(461.25880493,78.28931936)
\curveto(461.1788065,78.34931801)(461.13380654,78.41931794)(461.12380493,78.49931936)
\curveto(461.11380656,78.51931784)(461.10880657,78.53931782)(461.10880493,78.55931936)
\curveto(461.10880657,78.58931777)(461.10380657,78.61431774)(461.09380493,78.63431936)
}
}
{
\newrgbcolor{curcolor}{0 0 0}
\pscustom[linestyle=none,fillstyle=solid,fillcolor=curcolor]
{
}
}
{
\newrgbcolor{curcolor}{0 0 0}
\pscustom[linestyle=none,fillstyle=solid,fillcolor=curcolor]
{
\newpath
\moveto(452.12380493,89.26463186)
\curveto(452.11381556,89.95462722)(452.23381544,90.55462662)(452.48380493,91.06463186)
\curveto(452.73381494,91.58462559)(453.06881461,91.9796252)(453.48880493,92.24963186)
\curveto(453.56881411,92.29962488)(453.65881402,92.34462483)(453.75880493,92.38463186)
\curveto(453.84881383,92.42462475)(453.94381373,92.46962471)(454.04380493,92.51963186)
\curveto(454.14381353,92.55962462)(454.24381343,92.58962459)(454.34380493,92.60963186)
\curveto(454.44381323,92.62962455)(454.54881313,92.64962453)(454.65880493,92.66963186)
\curveto(454.70881297,92.68962449)(454.75381292,92.69462448)(454.79380493,92.68463186)
\curveto(454.83381284,92.6746245)(454.8788128,92.6796245)(454.92880493,92.69963186)
\curveto(454.9788127,92.70962447)(455.06381261,92.71462446)(455.18380493,92.71463186)
\curveto(455.29381238,92.71462446)(455.3788123,92.70962447)(455.43880493,92.69963186)
\curveto(455.49881218,92.6796245)(455.55881212,92.66962451)(455.61880493,92.66963186)
\curveto(455.678812,92.6796245)(455.73881194,92.6746245)(455.79880493,92.65463186)
\curveto(455.93881174,92.61462456)(456.0738116,92.5796246)(456.20380493,92.54963186)
\curveto(456.33381134,92.51962466)(456.45881122,92.4796247)(456.57880493,92.42963186)
\curveto(456.71881096,92.36962481)(456.84381083,92.29962488)(456.95380493,92.21963186)
\curveto(457.06381061,92.14962503)(457.1738105,92.0746251)(457.28380493,91.99463186)
\lineto(457.34380493,91.93463186)
\curveto(457.36381031,91.92462525)(457.38381029,91.90962527)(457.40380493,91.88963186)
\curveto(457.56381011,91.76962541)(457.70880997,91.63462554)(457.83880493,91.48463186)
\curveto(457.96880971,91.33462584)(458.09380958,91.174626)(458.21380493,91.00463186)
\curveto(458.43380924,90.69462648)(458.63880904,90.39962678)(458.82880493,90.11963186)
\curveto(458.96880871,89.88962729)(459.10380857,89.65962752)(459.23380493,89.42963186)
\curveto(459.36380831,89.20962797)(459.49880818,88.98962819)(459.63880493,88.76963186)
\curveto(459.80880787,88.51962866)(459.98880769,88.2796289)(460.17880493,88.04963186)
\curveto(460.36880731,87.82962935)(460.59380708,87.63962954)(460.85380493,87.47963186)
\curveto(460.91380676,87.43962974)(460.9738067,87.40462977)(461.03380493,87.37463186)
\curveto(461.08380659,87.34462983)(461.14880653,87.31462986)(461.22880493,87.28463186)
\curveto(461.29880638,87.26462991)(461.35880632,87.25962992)(461.40880493,87.26963186)
\curveto(461.4788062,87.28962989)(461.53380614,87.32462985)(461.57380493,87.37463186)
\curveto(461.60380607,87.42462975)(461.62380605,87.48462969)(461.63380493,87.55463186)
\lineto(461.63380493,87.79463186)
\lineto(461.63380493,88.54463186)
\lineto(461.63380493,91.34963186)
\lineto(461.63380493,92.00963186)
\curveto(461.63380604,92.09962508)(461.63880604,92.18462499)(461.64880493,92.26463186)
\curveto(461.64880603,92.34462483)(461.66880601,92.40962477)(461.70880493,92.45963186)
\curveto(461.74880593,92.50962467)(461.82380585,92.54962463)(461.93380493,92.57963186)
\curveto(462.03380564,92.61962456)(462.13380554,92.62962455)(462.23380493,92.60963186)
\lineto(462.36880493,92.60963186)
\curveto(462.43880524,92.58962459)(462.49880518,92.56962461)(462.54880493,92.54963186)
\curveto(462.59880508,92.52962465)(462.63880504,92.49462468)(462.66880493,92.44463186)
\curveto(462.70880497,92.39462478)(462.72880495,92.32462485)(462.72880493,92.23463186)
\lineto(462.72880493,91.96463186)
\lineto(462.72880493,91.06463186)
\lineto(462.72880493,87.55463186)
\lineto(462.72880493,86.48963186)
\curveto(462.72880495,86.40963077)(462.73380494,86.31963086)(462.74380493,86.21963186)
\curveto(462.74380493,86.11963106)(462.73380494,86.03463114)(462.71380493,85.96463186)
\curveto(462.64380503,85.75463142)(462.46380521,85.68963149)(462.17380493,85.76963186)
\curveto(462.13380554,85.7796314)(462.09880558,85.7796314)(462.06880493,85.76963186)
\curveto(462.02880565,85.76963141)(461.98380569,85.7796314)(461.93380493,85.79963186)
\curveto(461.85380582,85.81963136)(461.76880591,85.83963134)(461.67880493,85.85963186)
\curveto(461.58880609,85.8796313)(461.50380617,85.90463127)(461.42380493,85.93463186)
\curveto(460.93380674,86.09463108)(460.51880716,86.29463088)(460.17880493,86.53463186)
\curveto(459.92880775,86.71463046)(459.70380797,86.91963026)(459.50380493,87.14963186)
\curveto(459.29380838,87.3796298)(459.09880858,87.61962956)(458.91880493,87.86963186)
\curveto(458.73880894,88.12962905)(458.56880911,88.39462878)(458.40880493,88.66463186)
\curveto(458.23880944,88.94462823)(458.06380961,89.21462796)(457.88380493,89.47463186)
\curveto(457.80380987,89.58462759)(457.72880995,89.68962749)(457.65880493,89.78963186)
\curveto(457.58881009,89.89962728)(457.51381016,90.00962717)(457.43380493,90.11963186)
\curveto(457.40381027,90.15962702)(457.3738103,90.19462698)(457.34380493,90.22463186)
\curveto(457.30381037,90.26462691)(457.2738104,90.30462687)(457.25380493,90.34463186)
\curveto(457.14381053,90.48462669)(457.01881066,90.60962657)(456.87880493,90.71963186)
\curveto(456.84881083,90.73962644)(456.82381085,90.76462641)(456.80380493,90.79463186)
\curveto(456.7738109,90.82462635)(456.74381093,90.84962633)(456.71380493,90.86963186)
\curveto(456.61381106,90.94962623)(456.51381116,91.01462616)(456.41380493,91.06463186)
\curveto(456.31381136,91.12462605)(456.20381147,91.179626)(456.08380493,91.22963186)
\curveto(456.01381166,91.25962592)(455.93881174,91.2796259)(455.85880493,91.28963186)
\lineto(455.61880493,91.34963186)
\lineto(455.52880493,91.34963186)
\curveto(455.49881218,91.35962582)(455.46881221,91.36462581)(455.43880493,91.36463186)
\curveto(455.36881231,91.38462579)(455.2738124,91.38962579)(455.15380493,91.37963186)
\curveto(455.02381265,91.3796258)(454.92381275,91.36962581)(454.85380493,91.34963186)
\curveto(454.7738129,91.32962585)(454.69881298,91.30962587)(454.62880493,91.28963186)
\curveto(454.54881313,91.2796259)(454.46881321,91.25962592)(454.38880493,91.22963186)
\curveto(454.14881353,91.11962606)(453.94881373,90.96962621)(453.78880493,90.77963186)
\curveto(453.61881406,90.59962658)(453.4788142,90.3796268)(453.36880493,90.11963186)
\curveto(453.34881433,90.04962713)(453.33381434,89.9796272)(453.32380493,89.90963186)
\curveto(453.30381437,89.83962734)(453.28381439,89.76462741)(453.26380493,89.68463186)
\curveto(453.24381443,89.60462757)(453.23381444,89.49462768)(453.23380493,89.35463186)
\curveto(453.23381444,89.22462795)(453.24381443,89.11962806)(453.26380493,89.03963186)
\curveto(453.2738144,88.9796282)(453.2788144,88.92462825)(453.27880493,88.87463186)
\curveto(453.2788144,88.82462835)(453.28881439,88.7746284)(453.30880493,88.72463186)
\curveto(453.34881433,88.62462855)(453.38881429,88.52962865)(453.42880493,88.43963186)
\curveto(453.46881421,88.35962882)(453.51381416,88.2796289)(453.56380493,88.19963186)
\curveto(453.58381409,88.16962901)(453.60881407,88.13962904)(453.63880493,88.10963186)
\curveto(453.66881401,88.08962909)(453.69381398,88.06462911)(453.71380493,88.03463186)
\lineto(453.78880493,87.95963186)
\curveto(453.80881387,87.92962925)(453.82881385,87.90462927)(453.84880493,87.88463186)
\lineto(454.05880493,87.73463186)
\curveto(454.11881356,87.69462948)(454.18381349,87.64962953)(454.25380493,87.59963186)
\curveto(454.34381333,87.53962964)(454.44881323,87.48962969)(454.56880493,87.44963186)
\curveto(454.678813,87.41962976)(454.78881289,87.38462979)(454.89880493,87.34463186)
\curveto(455.00881267,87.30462987)(455.15381252,87.2796299)(455.33380493,87.26963186)
\curveto(455.50381217,87.25962992)(455.62881205,87.22962995)(455.70880493,87.17963186)
\curveto(455.78881189,87.12963005)(455.83381184,87.05463012)(455.84380493,86.95463186)
\curveto(455.85381182,86.85463032)(455.85881182,86.74463043)(455.85880493,86.62463186)
\curveto(455.85881182,86.58463059)(455.86381181,86.54463063)(455.87380493,86.50463186)
\curveto(455.8738118,86.46463071)(455.86881181,86.42963075)(455.85880493,86.39963186)
\curveto(455.83881184,86.34963083)(455.82881185,86.29963088)(455.82880493,86.24963186)
\curveto(455.82881185,86.20963097)(455.81881186,86.16963101)(455.79880493,86.12963186)
\curveto(455.73881194,86.03963114)(455.60381207,85.99463118)(455.39380493,85.99463186)
\lineto(455.27380493,85.99463186)
\curveto(455.21381246,86.00463117)(455.15381252,86.00963117)(455.09380493,86.00963186)
\curveto(455.02381265,86.01963116)(454.95881272,86.02963115)(454.89880493,86.03963186)
\curveto(454.78881289,86.05963112)(454.68881299,86.0796311)(454.59880493,86.09963186)
\curveto(454.49881318,86.11963106)(454.40381327,86.14963103)(454.31380493,86.18963186)
\curveto(454.24381343,86.20963097)(454.18381349,86.22963095)(454.13380493,86.24963186)
\lineto(453.95380493,86.30963186)
\curveto(453.69381398,86.42963075)(453.44881423,86.58463059)(453.21880493,86.77463186)
\curveto(452.98881469,86.9746302)(452.80381487,87.18962999)(452.66380493,87.41963186)
\curveto(452.58381509,87.52962965)(452.51881516,87.64462953)(452.46880493,87.76463186)
\lineto(452.31880493,88.15463186)
\curveto(452.26881541,88.26462891)(452.23881544,88.3796288)(452.22880493,88.49963186)
\curveto(452.20881547,88.61962856)(452.18381549,88.74462843)(452.15380493,88.87463186)
\curveto(452.15381552,88.94462823)(452.15381552,89.00962817)(452.15380493,89.06963186)
\curveto(452.14381553,89.12962805)(452.13381554,89.19462798)(452.12380493,89.26463186)
}
}
{
\newrgbcolor{curcolor}{0 0 0}
\pscustom[linestyle=none,fillstyle=solid,fillcolor=curcolor]
{
\newpath
\moveto(457.64380493,101.36424123)
\lineto(457.89880493,101.36424123)
\curveto(457.9788097,101.37423353)(458.05380962,101.36923353)(458.12380493,101.34924123)
\lineto(458.36380493,101.34924123)
\lineto(458.52880493,101.34924123)
\curveto(458.62880905,101.32923357)(458.73380894,101.31923358)(458.84380493,101.31924123)
\curveto(458.94380873,101.31923358)(459.04380863,101.30923359)(459.14380493,101.28924123)
\lineto(459.29380493,101.28924123)
\curveto(459.43380824,101.25923364)(459.5738081,101.23923366)(459.71380493,101.22924123)
\curveto(459.84380783,101.21923368)(459.9738077,101.19423371)(460.10380493,101.15424123)
\curveto(460.18380749,101.13423377)(460.26880741,101.11423379)(460.35880493,101.09424123)
\lineto(460.59880493,101.03424123)
\lineto(460.89880493,100.91424123)
\curveto(460.98880669,100.88423402)(461.0788066,100.84923405)(461.16880493,100.80924123)
\curveto(461.38880629,100.70923419)(461.60380607,100.57423433)(461.81380493,100.40424123)
\curveto(462.02380565,100.24423466)(462.19380548,100.06923483)(462.32380493,99.87924123)
\curveto(462.36380531,99.82923507)(462.40380527,99.76923513)(462.44380493,99.69924123)
\curveto(462.4738052,99.63923526)(462.50880517,99.57923532)(462.54880493,99.51924123)
\curveto(462.59880508,99.43923546)(462.63880504,99.34423556)(462.66880493,99.23424123)
\curveto(462.69880498,99.12423578)(462.72880495,99.01923588)(462.75880493,98.91924123)
\curveto(462.79880488,98.80923609)(462.82380485,98.6992362)(462.83380493,98.58924123)
\curveto(462.84380483,98.47923642)(462.85880482,98.36423654)(462.87880493,98.24424123)
\curveto(462.88880479,98.2042367)(462.88880479,98.15923674)(462.87880493,98.10924123)
\curveto(462.8788048,98.06923683)(462.88380479,98.02923687)(462.89380493,97.98924123)
\curveto(462.90380477,97.94923695)(462.90880477,97.89423701)(462.90880493,97.82424123)
\curveto(462.90880477,97.75423715)(462.90380477,97.7042372)(462.89380493,97.67424123)
\curveto(462.8738048,97.62423728)(462.86880481,97.57923732)(462.87880493,97.53924123)
\curveto(462.88880479,97.4992374)(462.88880479,97.46423744)(462.87880493,97.43424123)
\lineto(462.87880493,97.34424123)
\curveto(462.85880482,97.28423762)(462.84380483,97.21923768)(462.83380493,97.14924123)
\curveto(462.83380484,97.08923781)(462.82880485,97.02423788)(462.81880493,96.95424123)
\curveto(462.76880491,96.78423812)(462.71880496,96.62423828)(462.66880493,96.47424123)
\curveto(462.61880506,96.32423858)(462.55380512,96.17923872)(462.47380493,96.03924123)
\curveto(462.43380524,95.98923891)(462.40380527,95.93423897)(462.38380493,95.87424123)
\curveto(462.35380532,95.82423908)(462.31880536,95.77423913)(462.27880493,95.72424123)
\curveto(462.09880558,95.48423942)(461.8788058,95.28423962)(461.61880493,95.12424123)
\curveto(461.35880632,94.96423994)(461.0738066,94.82424008)(460.76380493,94.70424123)
\curveto(460.62380705,94.64424026)(460.48380719,94.5992403)(460.34380493,94.56924123)
\curveto(460.19380748,94.53924036)(460.03880764,94.5042404)(459.87880493,94.46424123)
\curveto(459.76880791,94.44424046)(459.65880802,94.42924047)(459.54880493,94.41924123)
\curveto(459.43880824,94.40924049)(459.32880835,94.39424051)(459.21880493,94.37424123)
\curveto(459.1788085,94.36424054)(459.13880854,94.35924054)(459.09880493,94.35924123)
\curveto(459.05880862,94.36924053)(459.01880866,94.36924053)(458.97880493,94.35924123)
\curveto(458.92880875,94.34924055)(458.8788088,94.34424056)(458.82880493,94.34424123)
\lineto(458.66380493,94.34424123)
\curveto(458.61380906,94.32424058)(458.56380911,94.31924058)(458.51380493,94.32924123)
\curveto(458.45380922,94.33924056)(458.39880928,94.33924056)(458.34880493,94.32924123)
\curveto(458.30880937,94.31924058)(458.26380941,94.31924058)(458.21380493,94.32924123)
\curveto(458.16380951,94.33924056)(458.11380956,94.33424057)(458.06380493,94.31424123)
\curveto(457.99380968,94.29424061)(457.91880976,94.28924061)(457.83880493,94.29924123)
\curveto(457.74880993,94.30924059)(457.66381001,94.31424059)(457.58380493,94.31424123)
\curveto(457.49381018,94.31424059)(457.39381028,94.30924059)(457.28380493,94.29924123)
\curveto(457.16381051,94.28924061)(457.06381061,94.29424061)(456.98380493,94.31424123)
\lineto(456.69880493,94.31424123)
\lineto(456.06880493,94.35924123)
\curveto(455.96881171,94.36924053)(455.8738118,94.37924052)(455.78380493,94.38924123)
\lineto(455.48380493,94.41924123)
\curveto(455.43381224,94.43924046)(455.38381229,94.44424046)(455.33380493,94.43424123)
\curveto(455.2738124,94.43424047)(455.21881246,94.44424046)(455.16880493,94.46424123)
\curveto(454.99881268,94.51424039)(454.83381284,94.55424035)(454.67380493,94.58424123)
\curveto(454.50381317,94.61424029)(454.34381333,94.66424024)(454.19380493,94.73424123)
\curveto(453.73381394,94.92423998)(453.35881432,95.14423976)(453.06880493,95.39424123)
\curveto(452.7788149,95.65423925)(452.53381514,96.01423889)(452.33380493,96.47424123)
\curveto(452.28381539,96.6042383)(452.24881543,96.73423817)(452.22880493,96.86424123)
\curveto(452.20881547,97.0042379)(452.18381549,97.14423776)(452.15380493,97.28424123)
\curveto(452.14381553,97.35423755)(452.13881554,97.41923748)(452.13880493,97.47924123)
\curveto(452.13881554,97.53923736)(452.13381554,97.6042373)(452.12380493,97.67424123)
\curveto(452.10381557,98.5042364)(452.25381542,99.17423573)(452.57380493,99.68424123)
\curveto(452.88381479,100.19423471)(453.32381435,100.57423433)(453.89380493,100.82424123)
\curveto(454.01381366,100.87423403)(454.13881354,100.91923398)(454.26880493,100.95924123)
\curveto(454.39881328,100.9992339)(454.53381314,101.04423386)(454.67380493,101.09424123)
\curveto(454.75381292,101.11423379)(454.83881284,101.12923377)(454.92880493,101.13924123)
\lineto(455.16880493,101.19924123)
\curveto(455.2788124,101.22923367)(455.38881229,101.24423366)(455.49880493,101.24424123)
\curveto(455.60881207,101.25423365)(455.71881196,101.26923363)(455.82880493,101.28924123)
\curveto(455.8788118,101.30923359)(455.92381175,101.31423359)(455.96380493,101.30424123)
\curveto(456.00381167,101.3042336)(456.04381163,101.30923359)(456.08380493,101.31924123)
\curveto(456.13381154,101.32923357)(456.18881149,101.32923357)(456.24880493,101.31924123)
\curveto(456.29881138,101.31923358)(456.34881133,101.32423358)(456.39880493,101.33424123)
\lineto(456.53380493,101.33424123)
\curveto(456.59381108,101.35423355)(456.66381101,101.35423355)(456.74380493,101.33424123)
\curveto(456.81381086,101.32423358)(456.8788108,101.32923357)(456.93880493,101.34924123)
\curveto(456.96881071,101.35923354)(457.00881067,101.36423354)(457.05880493,101.36424123)
\lineto(457.17880493,101.36424123)
\lineto(457.64380493,101.36424123)
\moveto(459.96880493,99.81924123)
\curveto(459.64880803,99.91923498)(459.28380839,99.97923492)(458.87380493,99.99924123)
\curveto(458.46380921,100.01923488)(458.05380962,100.02923487)(457.64380493,100.02924123)
\curveto(457.21381046,100.02923487)(456.79381088,100.01923488)(456.38380493,99.99924123)
\curveto(455.9738117,99.97923492)(455.58881209,99.93423497)(455.22880493,99.86424123)
\curveto(454.86881281,99.79423511)(454.54881313,99.68423522)(454.26880493,99.53424123)
\curveto(453.9788137,99.39423551)(453.74381393,99.1992357)(453.56380493,98.94924123)
\curveto(453.45381422,98.78923611)(453.3738143,98.60923629)(453.32380493,98.40924123)
\curveto(453.26381441,98.20923669)(453.23381444,97.96423694)(453.23380493,97.67424123)
\curveto(453.25381442,97.65423725)(453.26381441,97.61923728)(453.26380493,97.56924123)
\curveto(453.25381442,97.51923738)(453.25381442,97.47923742)(453.26380493,97.44924123)
\curveto(453.28381439,97.36923753)(453.30381437,97.29423761)(453.32380493,97.22424123)
\curveto(453.33381434,97.16423774)(453.35381432,97.0992378)(453.38380493,97.02924123)
\curveto(453.50381417,96.75923814)(453.673814,96.53923836)(453.89380493,96.36924123)
\curveto(454.10381357,96.20923869)(454.34881333,96.07423883)(454.62880493,95.96424123)
\curveto(454.73881294,95.91423899)(454.85881282,95.87423903)(454.98880493,95.84424123)
\curveto(455.10881257,95.82423908)(455.23381244,95.7992391)(455.36380493,95.76924123)
\curveto(455.41381226,95.74923915)(455.46881221,95.73923916)(455.52880493,95.73924123)
\curveto(455.5788121,95.73923916)(455.62881205,95.73423917)(455.67880493,95.72424123)
\curveto(455.76881191,95.71423919)(455.86381181,95.7042392)(455.96380493,95.69424123)
\curveto(456.05381162,95.68423922)(456.14881153,95.67423923)(456.24880493,95.66424123)
\curveto(456.32881135,95.66423924)(456.41381126,95.65923924)(456.50380493,95.64924123)
\lineto(456.74380493,95.64924123)
\lineto(456.92380493,95.64924123)
\curveto(456.95381072,95.63923926)(456.98881069,95.63423927)(457.02880493,95.63424123)
\lineto(457.16380493,95.63424123)
\lineto(457.61380493,95.63424123)
\curveto(457.69380998,95.63423927)(457.7788099,95.62923927)(457.86880493,95.61924123)
\curveto(457.94880973,95.61923928)(458.02380965,95.62923927)(458.09380493,95.64924123)
\lineto(458.36380493,95.64924123)
\curveto(458.38380929,95.64923925)(458.41380926,95.64423926)(458.45380493,95.63424123)
\curveto(458.48380919,95.63423927)(458.50880917,95.63923926)(458.52880493,95.64924123)
\curveto(458.62880905,95.65923924)(458.72880895,95.66423924)(458.82880493,95.66424123)
\curveto(458.91880876,95.67423923)(459.01880866,95.68423922)(459.12880493,95.69424123)
\curveto(459.24880843,95.72423918)(459.3738083,95.73923916)(459.50380493,95.73924123)
\curveto(459.62380805,95.74923915)(459.73880794,95.77423913)(459.84880493,95.81424123)
\curveto(460.14880753,95.89423901)(460.41380726,95.97923892)(460.64380493,96.06924123)
\curveto(460.8738068,96.16923873)(461.08880659,96.31423859)(461.28880493,96.50424123)
\curveto(461.48880619,96.71423819)(461.63880604,96.97923792)(461.73880493,97.29924123)
\curveto(461.75880592,97.33923756)(461.76880591,97.37423753)(461.76880493,97.40424123)
\curveto(461.75880592,97.44423746)(461.76380591,97.48923741)(461.78380493,97.53924123)
\curveto(461.79380588,97.57923732)(461.80380587,97.64923725)(461.81380493,97.74924123)
\curveto(461.82380585,97.85923704)(461.81880586,97.94423696)(461.79880493,98.00424123)
\curveto(461.7788059,98.07423683)(461.76880591,98.14423676)(461.76880493,98.21424123)
\curveto(461.75880592,98.28423662)(461.74380593,98.34923655)(461.72380493,98.40924123)
\curveto(461.66380601,98.60923629)(461.5788061,98.78923611)(461.46880493,98.94924123)
\curveto(461.44880623,98.97923592)(461.42880625,99.0042359)(461.40880493,99.02424123)
\lineto(461.34880493,99.08424123)
\curveto(461.32880635,99.12423578)(461.28880639,99.17423573)(461.22880493,99.23424123)
\curveto(461.08880659,99.33423557)(460.95880672,99.41923548)(460.83880493,99.48924123)
\curveto(460.71880696,99.55923534)(460.5738071,99.62923527)(460.40380493,99.69924123)
\curveto(460.33380734,99.72923517)(460.26380741,99.74923515)(460.19380493,99.75924123)
\curveto(460.12380755,99.77923512)(460.04880763,99.7992351)(459.96880493,99.81924123)
}
}
{
\newrgbcolor{curcolor}{0 0 0}
\pscustom[linestyle=none,fillstyle=solid,fillcolor=curcolor]
{
\newpath
\moveto(452.12380493,106.77385061)
\curveto(452.12381555,106.87384575)(452.13381554,106.96884566)(452.15380493,107.05885061)
\curveto(452.16381551,107.14884548)(452.19381548,107.21384541)(452.24380493,107.25385061)
\curveto(452.32381535,107.31384531)(452.42881525,107.34384528)(452.55880493,107.34385061)
\lineto(452.94880493,107.34385061)
\lineto(454.44880493,107.34385061)
\lineto(460.83880493,107.34385061)
\lineto(462.00880493,107.34385061)
\lineto(462.32380493,107.34385061)
\curveto(462.42380525,107.35384527)(462.50380517,107.33884529)(462.56380493,107.29885061)
\curveto(462.64380503,107.24884538)(462.69380498,107.17384545)(462.71380493,107.07385061)
\curveto(462.72380495,106.98384564)(462.72880495,106.87384575)(462.72880493,106.74385061)
\lineto(462.72880493,106.51885061)
\curveto(462.70880497,106.43884619)(462.69380498,106.36884626)(462.68380493,106.30885061)
\curveto(462.66380501,106.24884638)(462.62380505,106.19884643)(462.56380493,106.15885061)
\curveto(462.50380517,106.11884651)(462.42880525,106.09884653)(462.33880493,106.09885061)
\lineto(462.03880493,106.09885061)
\lineto(460.94380493,106.09885061)
\lineto(455.60380493,106.09885061)
\curveto(455.51381216,106.07884655)(455.43881224,106.06384656)(455.37880493,106.05385061)
\curveto(455.30881237,106.05384657)(455.24881243,106.0238466)(455.19880493,105.96385061)
\curveto(455.14881253,105.89384673)(455.12381255,105.80384682)(455.12380493,105.69385061)
\curveto(455.11381256,105.59384703)(455.10881257,105.48384714)(455.10880493,105.36385061)
\lineto(455.10880493,104.22385061)
\lineto(455.10880493,103.72885061)
\curveto(455.09881258,103.56884906)(455.03881264,103.45884917)(454.92880493,103.39885061)
\curveto(454.89881278,103.37884925)(454.86881281,103.36884926)(454.83880493,103.36885061)
\curveto(454.79881288,103.36884926)(454.75381292,103.36384926)(454.70380493,103.35385061)
\curveto(454.58381309,103.33384929)(454.4738132,103.33884929)(454.37380493,103.36885061)
\curveto(454.2738134,103.40884922)(454.20381347,103.46384916)(454.16380493,103.53385061)
\curveto(454.11381356,103.61384901)(454.08881359,103.73384889)(454.08880493,103.89385061)
\curveto(454.08881359,104.05384857)(454.0738136,104.18884844)(454.04380493,104.29885061)
\curveto(454.03381364,104.34884828)(454.02881365,104.40384822)(454.02880493,104.46385061)
\curveto(454.01881366,104.5238481)(454.00381367,104.58384804)(453.98380493,104.64385061)
\curveto(453.93381374,104.79384783)(453.88381379,104.93884769)(453.83380493,105.07885061)
\curveto(453.7738139,105.21884741)(453.70381397,105.35384727)(453.62380493,105.48385061)
\curveto(453.53381414,105.623847)(453.42881425,105.74384688)(453.30880493,105.84385061)
\curveto(453.18881449,105.94384668)(453.05881462,106.03884659)(452.91880493,106.12885061)
\curveto(452.81881486,106.18884644)(452.70881497,106.23384639)(452.58880493,106.26385061)
\curveto(452.46881521,106.30384632)(452.36381531,106.35384627)(452.27380493,106.41385061)
\curveto(452.21381546,106.46384616)(452.1738155,106.53384609)(452.15380493,106.62385061)
\curveto(452.14381553,106.64384598)(452.13881554,106.66884596)(452.13880493,106.69885061)
\curveto(452.13881554,106.7288459)(452.13381554,106.75384587)(452.12380493,106.77385061)
}
}
{
\newrgbcolor{curcolor}{0 0 0}
\pscustom[linestyle=none,fillstyle=solid,fillcolor=curcolor]
{
\newpath
\moveto(452.12380493,115.12345998)
\curveto(452.12381555,115.22345513)(452.13381554,115.31845503)(452.15380493,115.40845998)
\curveto(452.16381551,115.49845485)(452.19381548,115.56345479)(452.24380493,115.60345998)
\curveto(452.32381535,115.66345469)(452.42881525,115.69345466)(452.55880493,115.69345998)
\lineto(452.94880493,115.69345998)
\lineto(454.44880493,115.69345998)
\lineto(460.83880493,115.69345998)
\lineto(462.00880493,115.69345998)
\lineto(462.32380493,115.69345998)
\curveto(462.42380525,115.70345465)(462.50380517,115.68845466)(462.56380493,115.64845998)
\curveto(462.64380503,115.59845475)(462.69380498,115.52345483)(462.71380493,115.42345998)
\curveto(462.72380495,115.33345502)(462.72880495,115.22345513)(462.72880493,115.09345998)
\lineto(462.72880493,114.86845998)
\curveto(462.70880497,114.78845556)(462.69380498,114.71845563)(462.68380493,114.65845998)
\curveto(462.66380501,114.59845575)(462.62380505,114.5484558)(462.56380493,114.50845998)
\curveto(462.50380517,114.46845588)(462.42880525,114.4484559)(462.33880493,114.44845998)
\lineto(462.03880493,114.44845998)
\lineto(460.94380493,114.44845998)
\lineto(455.60380493,114.44845998)
\curveto(455.51381216,114.42845592)(455.43881224,114.41345594)(455.37880493,114.40345998)
\curveto(455.30881237,114.40345595)(455.24881243,114.37345598)(455.19880493,114.31345998)
\curveto(455.14881253,114.24345611)(455.12381255,114.1534562)(455.12380493,114.04345998)
\curveto(455.11381256,113.94345641)(455.10881257,113.83345652)(455.10880493,113.71345998)
\lineto(455.10880493,112.57345998)
\lineto(455.10880493,112.07845998)
\curveto(455.09881258,111.91845843)(455.03881264,111.80845854)(454.92880493,111.74845998)
\curveto(454.89881278,111.72845862)(454.86881281,111.71845863)(454.83880493,111.71845998)
\curveto(454.79881288,111.71845863)(454.75381292,111.71345864)(454.70380493,111.70345998)
\curveto(454.58381309,111.68345867)(454.4738132,111.68845866)(454.37380493,111.71845998)
\curveto(454.2738134,111.75845859)(454.20381347,111.81345854)(454.16380493,111.88345998)
\curveto(454.11381356,111.96345839)(454.08881359,112.08345827)(454.08880493,112.24345998)
\curveto(454.08881359,112.40345795)(454.0738136,112.53845781)(454.04380493,112.64845998)
\curveto(454.03381364,112.69845765)(454.02881365,112.7534576)(454.02880493,112.81345998)
\curveto(454.01881366,112.87345748)(454.00381367,112.93345742)(453.98380493,112.99345998)
\curveto(453.93381374,113.14345721)(453.88381379,113.28845706)(453.83380493,113.42845998)
\curveto(453.7738139,113.56845678)(453.70381397,113.70345665)(453.62380493,113.83345998)
\curveto(453.53381414,113.97345638)(453.42881425,114.09345626)(453.30880493,114.19345998)
\curveto(453.18881449,114.29345606)(453.05881462,114.38845596)(452.91880493,114.47845998)
\curveto(452.81881486,114.53845581)(452.70881497,114.58345577)(452.58880493,114.61345998)
\curveto(452.46881521,114.6534557)(452.36381531,114.70345565)(452.27380493,114.76345998)
\curveto(452.21381546,114.81345554)(452.1738155,114.88345547)(452.15380493,114.97345998)
\curveto(452.14381553,114.99345536)(452.13881554,115.01845533)(452.13880493,115.04845998)
\curveto(452.13881554,115.07845527)(452.13381554,115.10345525)(452.12380493,115.12345998)
}
}
{
\newrgbcolor{curcolor}{0 0 0}
\pscustom[linestyle=none,fillstyle=solid,fillcolor=curcolor]
{
\newpath
\moveto(472.96015137,29.18119436)
\lineto(472.96015137,30.09619436)
\curveto(472.96016206,30.19619171)(472.96016206,30.29119161)(472.96015137,30.38119436)
\curveto(472.96016206,30.47119143)(472.98016204,30.54619136)(473.02015137,30.60619436)
\curveto(473.08016194,30.69619121)(473.16016186,30.75619115)(473.26015137,30.78619436)
\curveto(473.36016166,30.82619108)(473.46516156,30.87119103)(473.57515137,30.92119436)
\curveto(473.76516126,31.0011909)(473.95516107,31.07119083)(474.14515137,31.13119436)
\curveto(474.33516069,31.2011907)(474.5251605,31.27619063)(474.71515137,31.35619436)
\curveto(474.89516013,31.42619048)(475.08015994,31.49119041)(475.27015137,31.55119436)
\curveto(475.45015957,31.61119029)(475.63015939,31.68119022)(475.81015137,31.76119436)
\curveto(475.95015907,31.82119008)(476.09515893,31.87619003)(476.24515137,31.92619436)
\curveto(476.39515863,31.97618993)(476.54015848,32.03118987)(476.68015137,32.09119436)
\curveto(477.13015789,32.27118963)(477.58515744,32.44118946)(478.04515137,32.60119436)
\curveto(478.49515653,32.76118914)(478.94515608,32.93118897)(479.39515137,33.11119436)
\curveto(479.44515558,33.13118877)(479.49515553,33.14618876)(479.54515137,33.15619436)
\lineto(479.69515137,33.21619436)
\curveto(479.91515511,33.3061886)(480.14015488,33.39118851)(480.37015137,33.47119436)
\curveto(480.59015443,33.55118835)(480.81015421,33.63618827)(481.03015137,33.72619436)
\curveto(481.1201539,33.76618814)(481.23015379,33.8061881)(481.36015137,33.84619436)
\curveto(481.48015354,33.88618802)(481.55015347,33.95118795)(481.57015137,34.04119436)
\curveto(481.58015344,34.08118782)(481.58015344,34.11118779)(481.57015137,34.13119436)
\lineto(481.51015137,34.19119436)
\curveto(481.46015356,34.24118766)(481.40515362,34.27618763)(481.34515137,34.29619436)
\curveto(481.28515374,34.32618758)(481.2201538,34.35618755)(481.15015137,34.38619436)
\lineto(480.52015137,34.62619436)
\curveto(480.30015472,34.7061872)(480.08515494,34.78618712)(479.87515137,34.86619436)
\lineto(479.72515137,34.92619436)
\lineto(479.54515137,34.98619436)
\curveto(479.35515567,35.06618684)(479.16515586,35.13618677)(478.97515137,35.19619436)
\curveto(478.77515625,35.26618664)(478.57515645,35.34118656)(478.37515137,35.42119436)
\curveto(477.79515723,35.66118624)(477.21015781,35.88118602)(476.62015137,36.08119436)
\curveto(476.03015899,36.29118561)(475.44515958,36.51618539)(474.86515137,36.75619436)
\curveto(474.66516036,36.83618507)(474.46016056,36.91118499)(474.25015137,36.98119436)
\curveto(474.04016098,37.06118484)(473.83516119,37.14118476)(473.63515137,37.22119436)
\curveto(473.55516147,37.26118464)(473.45516157,37.29618461)(473.33515137,37.32619436)
\curveto(473.21516181,37.36618454)(473.13016189,37.42118448)(473.08015137,37.49119436)
\curveto(473.04016198,37.55118435)(473.01016201,37.62618428)(472.99015137,37.71619436)
\curveto(472.97016205,37.81618409)(472.96016206,37.92618398)(472.96015137,38.04619436)
\curveto(472.95016207,38.16618374)(472.95016207,38.28618362)(472.96015137,38.40619436)
\curveto(472.96016206,38.52618338)(472.96016206,38.63618327)(472.96015137,38.73619436)
\curveto(472.96016206,38.82618308)(472.96016206,38.91618299)(472.96015137,39.00619436)
\curveto(472.96016206,39.1061828)(472.98016204,39.18118272)(473.02015137,39.23119436)
\curveto(473.07016195,39.32118258)(473.16016186,39.37118253)(473.29015137,39.38119436)
\curveto(473.4201616,39.39118251)(473.56016146,39.39618251)(473.71015137,39.39619436)
\lineto(475.36015137,39.39619436)
\lineto(481.63015137,39.39619436)
\lineto(482.89015137,39.39619436)
\curveto(483.00015202,39.39618251)(483.11015191,39.39618251)(483.22015137,39.39619436)
\curveto(483.33015169,39.4061825)(483.41515161,39.38618252)(483.47515137,39.33619436)
\curveto(483.53515149,39.3061826)(483.57515145,39.26118264)(483.59515137,39.20119436)
\curveto(483.60515142,39.14118276)(483.6201514,39.07118283)(483.64015137,38.99119436)
\lineto(483.64015137,38.75119436)
\lineto(483.64015137,38.39119436)
\curveto(483.63015139,38.28118362)(483.58515144,38.2011837)(483.50515137,38.15119436)
\curveto(483.47515155,38.13118377)(483.44515158,38.11618379)(483.41515137,38.10619436)
\curveto(483.37515165,38.1061838)(483.33015169,38.09618381)(483.28015137,38.07619436)
\lineto(483.11515137,38.07619436)
\curveto(483.05515197,38.06618384)(482.98515204,38.06118384)(482.90515137,38.06119436)
\curveto(482.8251522,38.07118383)(482.75015227,38.07618383)(482.68015137,38.07619436)
\lineto(481.84015137,38.07619436)
\lineto(477.41515137,38.07619436)
\curveto(477.16515786,38.07618383)(476.91515811,38.07618383)(476.66515137,38.07619436)
\curveto(476.40515862,38.07618383)(476.15515887,38.07118383)(475.91515137,38.06119436)
\curveto(475.81515921,38.06118384)(475.70515932,38.05618385)(475.58515137,38.04619436)
\curveto(475.46515956,38.03618387)(475.40515962,37.98118392)(475.40515137,37.88119436)
\lineto(475.42015137,37.88119436)
\curveto(475.44015958,37.81118409)(475.50515952,37.75118415)(475.61515137,37.70119436)
\curveto(475.7251593,37.66118424)(475.8201592,37.62618428)(475.90015137,37.59619436)
\curveto(476.07015895,37.52618438)(476.24515878,37.46118444)(476.42515137,37.40119436)
\curveto(476.59515843,37.34118456)(476.76515826,37.27118463)(476.93515137,37.19119436)
\curveto(476.98515804,37.17118473)(477.03015799,37.15618475)(477.07015137,37.14619436)
\curveto(477.11015791,37.13618477)(477.15515787,37.12118478)(477.20515137,37.10119436)
\curveto(477.38515764,37.02118488)(477.57015745,36.95118495)(477.76015137,36.89119436)
\curveto(477.94015708,36.84118506)(478.1201569,36.77618513)(478.30015137,36.69619436)
\curveto(478.45015657,36.62618528)(478.60515642,36.56618534)(478.76515137,36.51619436)
\curveto(478.91515611,36.46618544)(479.06515596,36.41118549)(479.21515137,36.35119436)
\curveto(479.68515534,36.15118575)(480.16015486,35.97118593)(480.64015137,35.81119436)
\curveto(481.11015391,35.65118625)(481.57515345,35.47618643)(482.03515137,35.28619436)
\curveto(482.21515281,35.2061867)(482.39515263,35.13618677)(482.57515137,35.07619436)
\curveto(482.75515227,35.01618689)(482.93515209,34.95118695)(483.11515137,34.88119436)
\curveto(483.2251518,34.83118707)(483.33015169,34.78118712)(483.43015137,34.73119436)
\curveto(483.5201515,34.69118721)(483.58515144,34.6061873)(483.62515137,34.47619436)
\curveto(483.63515139,34.45618745)(483.64015138,34.43118747)(483.64015137,34.40119436)
\curveto(483.63015139,34.38118752)(483.63015139,34.35618755)(483.64015137,34.32619436)
\curveto(483.65015137,34.29618761)(483.65515137,34.26118764)(483.65515137,34.22119436)
\curveto(483.64515138,34.18118772)(483.64015138,34.14118776)(483.64015137,34.10119436)
\lineto(483.64015137,33.80119436)
\curveto(483.64015138,33.7011882)(483.61515141,33.62118828)(483.56515137,33.56119436)
\curveto(483.51515151,33.48118842)(483.44515158,33.42118848)(483.35515137,33.38119436)
\curveto(483.25515177,33.35118855)(483.15515187,33.31118859)(483.05515137,33.26119436)
\curveto(482.85515217,33.18118872)(482.65015237,33.1011888)(482.44015137,33.02119436)
\curveto(482.2201528,32.95118895)(482.01015301,32.87618903)(481.81015137,32.79619436)
\curveto(481.63015339,32.71618919)(481.45015357,32.64618926)(481.27015137,32.58619436)
\curveto(481.08015394,32.53618937)(480.89515413,32.47118943)(480.71515137,32.39119436)
\curveto(480.15515487,32.16118974)(479.59015543,31.94618996)(479.02015137,31.74619436)
\curveto(478.45015657,31.54619036)(477.88515714,31.33119057)(477.32515137,31.10119436)
\lineto(476.69515137,30.86119436)
\curveto(476.47515855,30.79119111)(476.26515876,30.71619119)(476.06515137,30.63619436)
\curveto(475.95515907,30.58619132)(475.85015917,30.54119136)(475.75015137,30.50119436)
\curveto(475.64015938,30.47119143)(475.54515948,30.42119148)(475.46515137,30.35119436)
\curveto(475.44515958,30.34119156)(475.43515959,30.33119157)(475.43515137,30.32119436)
\lineto(475.40515137,30.29119436)
\lineto(475.40515137,30.21619436)
\lineto(475.43515137,30.18619436)
\curveto(475.43515959,30.17619173)(475.44015958,30.16619174)(475.45015137,30.15619436)
\curveto(475.50015952,30.13619177)(475.55515947,30.12619178)(475.61515137,30.12619436)
\curveto(475.67515935,30.12619178)(475.73515929,30.11619179)(475.79515137,30.09619436)
\lineto(475.96015137,30.09619436)
\curveto(476.020159,30.07619183)(476.08515894,30.07119183)(476.15515137,30.08119436)
\curveto(476.2251588,30.09119181)(476.29515873,30.09619181)(476.36515137,30.09619436)
\lineto(477.17515137,30.09619436)
\lineto(481.73515137,30.09619436)
\lineto(482.92015137,30.09619436)
\curveto(483.03015199,30.09619181)(483.14015188,30.09119181)(483.25015137,30.08119436)
\curveto(483.36015166,30.08119182)(483.44515158,30.05619185)(483.50515137,30.00619436)
\curveto(483.58515144,29.95619195)(483.63015139,29.86619204)(483.64015137,29.73619436)
\lineto(483.64015137,29.34619436)
\lineto(483.64015137,29.15119436)
\curveto(483.64015138,29.1011928)(483.63015139,29.05119285)(483.61015137,29.00119436)
\curveto(483.57015145,28.87119303)(483.48515154,28.79619311)(483.35515137,28.77619436)
\curveto(483.2251518,28.76619314)(483.07515195,28.76119314)(482.90515137,28.76119436)
\lineto(481.16515137,28.76119436)
\lineto(475.16515137,28.76119436)
\lineto(473.75515137,28.76119436)
\curveto(473.64516138,28.76119314)(473.53016149,28.75619315)(473.41015137,28.74619436)
\curveto(473.29016173,28.74619316)(473.19516183,28.77119313)(473.12515137,28.82119436)
\curveto(473.06516196,28.86119304)(473.01516201,28.93619297)(472.97515137,29.04619436)
\curveto(472.96516206,29.06619284)(472.96516206,29.08619282)(472.97515137,29.10619436)
\curveto(472.97516205,29.13619277)(472.97016205,29.16119274)(472.96015137,29.18119436)
}
}
{
\newrgbcolor{curcolor}{0 0 0}
\pscustom[linestyle=none,fillstyle=solid,fillcolor=curcolor]
{
\newpath
\moveto(483.08515137,48.38330373)
\curveto(483.24515178,48.4132959)(483.38015164,48.39829592)(483.49015137,48.33830373)
\curveto(483.59015143,48.27829604)(483.66515136,48.19829612)(483.71515137,48.09830373)
\curveto(483.73515129,48.04829627)(483.74515128,47.99329632)(483.74515137,47.93330373)
\curveto(483.74515128,47.88329643)(483.75515127,47.82829649)(483.77515137,47.76830373)
\curveto(483.8251512,47.54829677)(483.81015121,47.32829699)(483.73015137,47.10830373)
\curveto(483.66015136,46.89829742)(483.57015145,46.75329756)(483.46015137,46.67330373)
\curveto(483.39015163,46.62329769)(483.31015171,46.57829774)(483.22015137,46.53830373)
\curveto(483.1201519,46.49829782)(483.04015198,46.44829787)(482.98015137,46.38830373)
\curveto(482.96015206,46.36829795)(482.94015208,46.34329797)(482.92015137,46.31330373)
\curveto(482.90015212,46.29329802)(482.89515213,46.26329805)(482.90515137,46.22330373)
\curveto(482.93515209,46.1132982)(482.99015203,46.00829831)(483.07015137,45.90830373)
\curveto(483.15015187,45.8182985)(483.2201518,45.72829859)(483.28015137,45.63830373)
\curveto(483.36015166,45.50829881)(483.43515159,45.36829895)(483.50515137,45.21830373)
\curveto(483.56515146,45.06829925)(483.6201514,44.90829941)(483.67015137,44.73830373)
\curveto(483.70015132,44.63829968)(483.7201513,44.52829979)(483.73015137,44.40830373)
\curveto(483.74015128,44.29830002)(483.75515127,44.18830013)(483.77515137,44.07830373)
\curveto(483.78515124,44.02830029)(483.79015123,43.98330033)(483.79015137,43.94330373)
\lineto(483.79015137,43.83830373)
\curveto(483.81015121,43.72830059)(483.81015121,43.62330069)(483.79015137,43.52330373)
\lineto(483.79015137,43.38830373)
\curveto(483.78015124,43.33830098)(483.77515125,43.28830103)(483.77515137,43.23830373)
\curveto(483.77515125,43.18830113)(483.76515126,43.14330117)(483.74515137,43.10330373)
\curveto(483.73515129,43.06330125)(483.73015129,43.02830129)(483.73015137,42.99830373)
\curveto(483.74015128,42.97830134)(483.74015128,42.95330136)(483.73015137,42.92330373)
\lineto(483.67015137,42.68330373)
\curveto(483.66015136,42.60330171)(483.64015138,42.52830179)(483.61015137,42.45830373)
\curveto(483.48015154,42.15830216)(483.33515169,41.9133024)(483.17515137,41.72330373)
\curveto(483.00515202,41.54330277)(482.77015225,41.39330292)(482.47015137,41.27330373)
\curveto(482.25015277,41.18330313)(481.98515304,41.13830318)(481.67515137,41.13830373)
\lineto(481.36015137,41.13830373)
\curveto(481.31015371,41.14830317)(481.26015376,41.15330316)(481.21015137,41.15330373)
\lineto(481.03015137,41.18330373)
\lineto(480.70015137,41.30330373)
\curveto(480.59015443,41.34330297)(480.49015453,41.39330292)(480.40015137,41.45330373)
\curveto(480.11015491,41.63330268)(479.89515513,41.87830244)(479.75515137,42.18830373)
\curveto(479.61515541,42.49830182)(479.49015553,42.83830148)(479.38015137,43.20830373)
\curveto(479.34015568,43.34830097)(479.31015571,43.49330082)(479.29015137,43.64330373)
\curveto(479.27015575,43.79330052)(479.24515578,43.94330037)(479.21515137,44.09330373)
\curveto(479.19515583,44.16330015)(479.18515584,44.22830009)(479.18515137,44.28830373)
\curveto(479.18515584,44.35829996)(479.17515585,44.43329988)(479.15515137,44.51330373)
\curveto(479.13515589,44.58329973)(479.1251559,44.65329966)(479.12515137,44.72330373)
\curveto(479.11515591,44.79329952)(479.10015592,44.86829945)(479.08015137,44.94830373)
\curveto(479.020156,45.19829912)(478.97015605,45.43329888)(478.93015137,45.65330373)
\curveto(478.88015614,45.87329844)(478.76515626,46.04829827)(478.58515137,46.17830373)
\curveto(478.50515652,46.23829808)(478.40515662,46.28829803)(478.28515137,46.32830373)
\curveto(478.15515687,46.36829795)(478.01515701,46.36829795)(477.86515137,46.32830373)
\curveto(477.6251574,46.26829805)(477.43515759,46.17829814)(477.29515137,46.05830373)
\curveto(477.15515787,45.94829837)(477.04515798,45.78829853)(476.96515137,45.57830373)
\curveto(476.91515811,45.45829886)(476.88015814,45.313299)(476.86015137,45.14330373)
\curveto(476.84015818,44.98329933)(476.83015819,44.8132995)(476.83015137,44.63330373)
\curveto(476.83015819,44.45329986)(476.84015818,44.27830004)(476.86015137,44.10830373)
\curveto(476.88015814,43.93830038)(476.91015811,43.79330052)(476.95015137,43.67330373)
\curveto(477.01015801,43.50330081)(477.09515793,43.33830098)(477.20515137,43.17830373)
\curveto(477.26515776,43.09830122)(477.34515768,43.02330129)(477.44515137,42.95330373)
\curveto(477.53515749,42.89330142)(477.63515739,42.83830148)(477.74515137,42.78830373)
\curveto(477.8251572,42.75830156)(477.91015711,42.72830159)(478.00015137,42.69830373)
\curveto(478.09015693,42.67830164)(478.16015686,42.63330168)(478.21015137,42.56330373)
\curveto(478.24015678,42.52330179)(478.26515676,42.45330186)(478.28515137,42.35330373)
\curveto(478.29515673,42.26330205)(478.30015672,42.16830215)(478.30015137,42.06830373)
\curveto(478.30015672,41.96830235)(478.29515673,41.86830245)(478.28515137,41.76830373)
\curveto(478.26515676,41.67830264)(478.24015678,41.6133027)(478.21015137,41.57330373)
\curveto(478.18015684,41.53330278)(478.13015689,41.50330281)(478.06015137,41.48330373)
\curveto(477.99015703,41.46330285)(477.91515711,41.46330285)(477.83515137,41.48330373)
\curveto(477.70515732,41.5133028)(477.58515744,41.54330277)(477.47515137,41.57330373)
\curveto(477.35515767,41.6133027)(477.24015778,41.65830266)(477.13015137,41.70830373)
\curveto(476.78015824,41.89830242)(476.51015851,42.13830218)(476.32015137,42.42830373)
\curveto(476.1201589,42.7183016)(475.96015906,43.07830124)(475.84015137,43.50830373)
\curveto(475.8201592,43.60830071)(475.80515922,43.70830061)(475.79515137,43.80830373)
\curveto(475.78515924,43.9183004)(475.77015925,44.02830029)(475.75015137,44.13830373)
\curveto(475.74015928,44.17830014)(475.74015928,44.24330007)(475.75015137,44.33330373)
\curveto(475.75015927,44.42329989)(475.74015928,44.47829984)(475.72015137,44.49830373)
\curveto(475.71015931,45.19829912)(475.79015923,45.80829851)(475.96015137,46.32830373)
\curveto(476.13015889,46.84829747)(476.45515857,47.2132971)(476.93515137,47.42330373)
\curveto(477.13515789,47.5132968)(477.37015765,47.56329675)(477.64015137,47.57330373)
\curveto(477.90015712,47.59329672)(478.17515685,47.60329671)(478.46515137,47.60330373)
\lineto(481.78015137,47.60330373)
\curveto(481.9201531,47.60329671)(482.05515297,47.60829671)(482.18515137,47.61830373)
\curveto(482.31515271,47.62829669)(482.4201526,47.65829666)(482.50015137,47.70830373)
\curveto(482.57015245,47.75829656)(482.6201524,47.82329649)(482.65015137,47.90330373)
\curveto(482.69015233,47.99329632)(482.7201523,48.07829624)(482.74015137,48.15830373)
\curveto(482.75015227,48.23829608)(482.79515223,48.29829602)(482.87515137,48.33830373)
\curveto(482.90515212,48.35829596)(482.93515209,48.36829595)(482.96515137,48.36830373)
\curveto(482.99515203,48.36829595)(483.03515199,48.37329594)(483.08515137,48.38330373)
\moveto(481.42015137,46.23830373)
\curveto(481.28015374,46.29829802)(481.1201539,46.32829799)(480.94015137,46.32830373)
\curveto(480.75015427,46.33829798)(480.55515447,46.34329797)(480.35515137,46.34330373)
\curveto(480.24515478,46.34329797)(480.14515488,46.33829798)(480.05515137,46.32830373)
\curveto(479.96515506,46.318298)(479.89515513,46.27829804)(479.84515137,46.20830373)
\curveto(479.8251552,46.17829814)(479.81515521,46.10829821)(479.81515137,45.99830373)
\curveto(479.83515519,45.97829834)(479.84515518,45.94329837)(479.84515137,45.89330373)
\curveto(479.84515518,45.84329847)(479.85515517,45.79829852)(479.87515137,45.75830373)
\curveto(479.89515513,45.67829864)(479.91515511,45.58829873)(479.93515137,45.48830373)
\lineto(479.99515137,45.18830373)
\curveto(479.99515503,45.15829916)(480.00015502,45.12329919)(480.01015137,45.08330373)
\lineto(480.01015137,44.97830373)
\curveto(480.05015497,44.82829949)(480.07515495,44.66329965)(480.08515137,44.48330373)
\curveto(480.08515494,44.3133)(480.10515492,44.15330016)(480.14515137,44.00330373)
\curveto(480.16515486,43.92330039)(480.18515484,43.84830047)(480.20515137,43.77830373)
\curveto(480.21515481,43.7183006)(480.23015479,43.64830067)(480.25015137,43.56830373)
\curveto(480.30015472,43.40830091)(480.36515466,43.25830106)(480.44515137,43.11830373)
\curveto(480.51515451,42.97830134)(480.60515442,42.85830146)(480.71515137,42.75830373)
\curveto(480.8251542,42.65830166)(480.96015406,42.58330173)(481.12015137,42.53330373)
\curveto(481.27015375,42.48330183)(481.45515357,42.46330185)(481.67515137,42.47330373)
\curveto(481.77515325,42.47330184)(481.87015315,42.48830183)(481.96015137,42.51830373)
\curveto(482.04015298,42.55830176)(482.11515291,42.60330171)(482.18515137,42.65330373)
\curveto(482.29515273,42.73330158)(482.39015263,42.83830148)(482.47015137,42.96830373)
\curveto(482.54015248,43.09830122)(482.60015242,43.23830108)(482.65015137,43.38830373)
\curveto(482.66015236,43.43830088)(482.66515236,43.48830083)(482.66515137,43.53830373)
\curveto(482.66515236,43.58830073)(482.67015235,43.63830068)(482.68015137,43.68830373)
\curveto(482.70015232,43.75830056)(482.71515231,43.84330047)(482.72515137,43.94330373)
\curveto(482.7251523,44.05330026)(482.71515231,44.14330017)(482.69515137,44.21330373)
\curveto(482.67515235,44.27330004)(482.67015235,44.33329998)(482.68015137,44.39330373)
\curveto(482.68015234,44.45329986)(482.67015235,44.5132998)(482.65015137,44.57330373)
\curveto(482.63015239,44.65329966)(482.61515241,44.72829959)(482.60515137,44.79830373)
\curveto(482.59515243,44.87829944)(482.57515245,44.95329936)(482.54515137,45.02330373)
\curveto(482.4251526,45.313299)(482.28015274,45.55829876)(482.11015137,45.75830373)
\curveto(481.94015308,45.96829835)(481.71015331,46.12829819)(481.42015137,46.23830373)
}
}
{
\newrgbcolor{curcolor}{0 0 0}
\pscustom[linestyle=none,fillstyle=solid,fillcolor=curcolor]
{
\newpath
\moveto(475.91515137,49.26994436)
\lineto(475.91515137,49.71994436)
\curveto(475.90515912,49.88994311)(475.9251591,50.01494298)(475.97515137,50.09494436)
\curveto(476.025159,50.17494282)(476.09015893,50.22994277)(476.17015137,50.25994436)
\curveto(476.25015877,50.2999427)(476.33515869,50.33994266)(476.42515137,50.37994436)
\curveto(476.55515847,50.42994257)(476.68515834,50.47494252)(476.81515137,50.51494436)
\curveto(476.94515808,50.55494244)(477.07515795,50.5999424)(477.20515137,50.64994436)
\curveto(477.3251577,50.6999423)(477.45015757,50.74494225)(477.58015137,50.78494436)
\curveto(477.70015732,50.82494217)(477.8201572,50.86994213)(477.94015137,50.91994436)
\curveto(478.05015697,50.96994203)(478.16515686,51.00994199)(478.28515137,51.03994436)
\curveto(478.40515662,51.06994193)(478.5251565,51.10994189)(478.64515137,51.15994436)
\curveto(478.93515609,51.27994172)(479.23515579,51.38994161)(479.54515137,51.48994436)
\curveto(479.85515517,51.58994141)(480.15515487,51.6999413)(480.44515137,51.81994436)
\curveto(480.48515454,51.83994116)(480.5251545,51.84994115)(480.56515137,51.84994436)
\curveto(480.59515443,51.84994115)(480.6251544,51.85994114)(480.65515137,51.87994436)
\curveto(480.79515423,51.93994106)(480.94015408,51.994941)(481.09015137,52.04494436)
\lineto(481.51015137,52.19494436)
\curveto(481.58015344,52.22494077)(481.65515337,52.25494074)(481.73515137,52.28494436)
\curveto(481.80515322,52.31494068)(481.85015317,52.36494063)(481.87015137,52.43494436)
\curveto(481.90015312,52.51494048)(481.87515315,52.57494042)(481.79515137,52.61494436)
\curveto(481.70515332,52.66494033)(481.63515339,52.6999403)(481.58515137,52.71994436)
\curveto(481.41515361,52.7999402)(481.23515379,52.86494013)(481.04515137,52.91494436)
\curveto(480.85515417,52.96494003)(480.67015435,53.02493997)(480.49015137,53.09494436)
\curveto(480.26015476,53.18493981)(480.03015499,53.26493973)(479.80015137,53.33494436)
\curveto(479.56015546,53.40493959)(479.33015569,53.48993951)(479.11015137,53.58994436)
\curveto(479.06015596,53.5999394)(478.99515603,53.61493938)(478.91515137,53.63494436)
\curveto(478.8251562,53.67493932)(478.73515629,53.70993929)(478.64515137,53.73994436)
\curveto(478.54515648,53.76993923)(478.45515657,53.7999392)(478.37515137,53.82994436)
\curveto(478.3251567,53.84993915)(478.28015674,53.86493913)(478.24015137,53.87494436)
\curveto(478.20015682,53.88493911)(478.15515687,53.8999391)(478.10515137,53.91994436)
\curveto(477.98515704,53.96993903)(477.86515716,54.00993899)(477.74515137,54.03994436)
\curveto(477.61515741,54.07993892)(477.49015753,54.12493887)(477.37015137,54.17494436)
\curveto(477.3201577,54.1949388)(477.27515775,54.20993879)(477.23515137,54.21994436)
\curveto(477.19515783,54.22993877)(477.15015787,54.24493875)(477.10015137,54.26494436)
\curveto(477.01015801,54.30493869)(476.9201581,54.33993866)(476.83015137,54.36994436)
\curveto(476.73015829,54.3999386)(476.63515839,54.42993857)(476.54515137,54.45994436)
\curveto(476.46515856,54.48993851)(476.38515864,54.51493848)(476.30515137,54.53494436)
\curveto(476.21515881,54.56493843)(476.14015888,54.60493839)(476.08015137,54.65494436)
\curveto(475.99015903,54.72493827)(475.94015908,54.81993818)(475.93015137,54.93994436)
\curveto(475.9201591,55.06993793)(475.91515911,55.20993779)(475.91515137,55.35994436)
\curveto(475.91515911,55.43993756)(475.9201591,55.51493748)(475.93015137,55.58494436)
\curveto(475.93015909,55.66493733)(475.94515908,55.72993727)(475.97515137,55.77994436)
\curveto(476.03515899,55.86993713)(476.13015889,55.8949371)(476.26015137,55.85494436)
\curveto(476.39015863,55.81493718)(476.49015853,55.77993722)(476.56015137,55.74994436)
\lineto(476.62015137,55.71994436)
\curveto(476.64015838,55.71993728)(476.66015836,55.71493728)(476.68015137,55.70494436)
\curveto(476.96015806,55.5949374)(477.24515778,55.48493751)(477.53515137,55.37494436)
\lineto(478.37515137,55.04494436)
\curveto(478.45515657,55.01493798)(478.53015649,54.98993801)(478.60015137,54.96994436)
\curveto(478.66015636,54.94993805)(478.7251563,54.92493807)(478.79515137,54.89494436)
\curveto(478.99515603,54.81493818)(479.20015582,54.73493826)(479.41015137,54.65494436)
\curveto(479.61015541,54.58493841)(479.81015521,54.50993849)(480.01015137,54.42994436)
\curveto(480.70015432,54.13993886)(481.39515363,53.86993913)(482.09515137,53.61994436)
\curveto(482.79515223,53.36993963)(483.49015153,53.0999399)(484.18015137,52.80994436)
\lineto(484.33015137,52.74994436)
\curveto(484.39015063,52.73994026)(484.45015057,52.72494027)(484.51015137,52.70494436)
\curveto(484.88015014,52.54494045)(485.24514978,52.37494062)(485.60515137,52.19494436)
\curveto(485.97514905,52.01494098)(486.26014876,51.76494123)(486.46015137,51.44494436)
\curveto(486.5201485,51.33494166)(486.56514846,51.22494177)(486.59515137,51.11494436)
\curveto(486.63514839,51.00494199)(486.67014835,50.87994212)(486.70015137,50.73994436)
\curveto(486.7201483,50.68994231)(486.7251483,50.63494236)(486.71515137,50.57494436)
\curveto(486.70514832,50.52494247)(486.70514832,50.46994253)(486.71515137,50.40994436)
\curveto(486.73514829,50.32994267)(486.73514829,50.24994275)(486.71515137,50.16994436)
\curveto(486.70514832,50.12994287)(486.70014832,50.07994292)(486.70015137,50.01994436)
\lineto(486.64015137,49.77994436)
\curveto(486.6201484,49.70994329)(486.58014844,49.65494334)(486.52015137,49.61494436)
\curveto(486.46014856,49.56494343)(486.38514864,49.53494346)(486.29515137,49.52494436)
\lineto(486.02515137,49.52494436)
\lineto(485.81515137,49.52494436)
\curveto(485.75514927,49.53494346)(485.70514932,49.55494344)(485.66515137,49.58494436)
\curveto(485.55514947,49.65494334)(485.5251495,49.77494322)(485.57515137,49.94494436)
\curveto(485.59514943,50.05494294)(485.60514942,50.17494282)(485.60515137,50.30494436)
\curveto(485.60514942,50.43494256)(485.58514944,50.54994245)(485.54515137,50.64994436)
\curveto(485.49514953,50.7999422)(485.4201496,50.91994208)(485.32015137,51.00994436)
\curveto(485.2201498,51.10994189)(485.10514992,51.1949418)(484.97515137,51.26494436)
\curveto(484.85515017,51.33494166)(484.7251503,51.3949416)(484.58515137,51.44494436)
\lineto(484.16515137,51.62494436)
\curveto(484.07515095,51.66494133)(483.96515106,51.70494129)(483.83515137,51.74494436)
\curveto(483.70515132,51.7949412)(483.57015145,51.7999412)(483.43015137,51.75994436)
\curveto(483.27015175,51.70994129)(483.1201519,51.65494134)(482.98015137,51.59494436)
\curveto(482.84015218,51.54494145)(482.70015232,51.48994151)(482.56015137,51.42994436)
\curveto(482.35015267,51.33994166)(482.14015288,51.25494174)(481.93015137,51.17494436)
\curveto(481.7201533,51.0949419)(481.51515351,51.01494198)(481.31515137,50.93494436)
\curveto(481.17515385,50.87494212)(481.04015398,50.81994218)(480.91015137,50.76994436)
\curveto(480.78015424,50.71994228)(480.64515438,50.66994233)(480.50515137,50.61994436)
\lineto(479.18515137,50.07994436)
\curveto(478.74515628,49.90994309)(478.30515672,49.73494326)(477.86515137,49.55494436)
\curveto(477.63515739,49.45494354)(477.41515761,49.36494363)(477.20515137,49.28494436)
\curveto(476.98515804,49.20494379)(476.76515826,49.11994388)(476.54515137,49.02994436)
\curveto(476.48515854,49.00994399)(476.40515862,48.97994402)(476.30515137,48.93994436)
\curveto(476.19515883,48.8999441)(476.10515892,48.90494409)(476.03515137,48.95494436)
\curveto(475.98515904,48.98494401)(475.95015907,49.04494395)(475.93015137,49.13494436)
\curveto(475.9201591,49.15494384)(475.9201591,49.17494382)(475.93015137,49.19494436)
\curveto(475.93015909,49.22494377)(475.9251591,49.24994375)(475.91515137,49.26994436)
}
}
{
\newrgbcolor{curcolor}{0 0 0}
\pscustom[linestyle=none,fillstyle=solid,fillcolor=curcolor]
{
}
}
{
\newrgbcolor{curcolor}{0 0 0}
\pscustom[linestyle=none,fillstyle=solid,fillcolor=curcolor]
{
\newpath
\moveto(478.55515137,67.99510061)
\lineto(478.81015137,67.99510061)
\curveto(478.89015613,68.0050929)(478.96515606,68.00009291)(479.03515137,67.98010061)
\lineto(479.27515137,67.98010061)
\lineto(479.44015137,67.98010061)
\curveto(479.54015548,67.96009295)(479.64515538,67.95009296)(479.75515137,67.95010061)
\curveto(479.85515517,67.95009296)(479.95515507,67.94009297)(480.05515137,67.92010061)
\lineto(480.20515137,67.92010061)
\curveto(480.34515468,67.89009302)(480.48515454,67.87009304)(480.62515137,67.86010061)
\curveto(480.75515427,67.85009306)(480.88515414,67.82509308)(481.01515137,67.78510061)
\curveto(481.09515393,67.76509314)(481.18015384,67.74509316)(481.27015137,67.72510061)
\lineto(481.51015137,67.66510061)
\lineto(481.81015137,67.54510061)
\curveto(481.90015312,67.51509339)(481.99015303,67.48009343)(482.08015137,67.44010061)
\curveto(482.30015272,67.34009357)(482.51515251,67.2050937)(482.72515137,67.03510061)
\curveto(482.93515209,66.87509403)(483.10515192,66.70009421)(483.23515137,66.51010061)
\curveto(483.27515175,66.46009445)(483.31515171,66.40009451)(483.35515137,66.33010061)
\curveto(483.38515164,66.27009464)(483.4201516,66.2100947)(483.46015137,66.15010061)
\curveto(483.51015151,66.07009484)(483.55015147,65.97509493)(483.58015137,65.86510061)
\curveto(483.61015141,65.75509515)(483.64015138,65.65009526)(483.67015137,65.55010061)
\curveto(483.71015131,65.44009547)(483.73515129,65.33009558)(483.74515137,65.22010061)
\curveto(483.75515127,65.1100958)(483.77015125,64.99509591)(483.79015137,64.87510061)
\curveto(483.80015122,64.83509607)(483.80015122,64.79009612)(483.79015137,64.74010061)
\curveto(483.79015123,64.70009621)(483.79515123,64.66009625)(483.80515137,64.62010061)
\curveto(483.81515121,64.58009633)(483.8201512,64.52509638)(483.82015137,64.45510061)
\curveto(483.8201512,64.38509652)(483.81515121,64.33509657)(483.80515137,64.30510061)
\curveto(483.78515124,64.25509665)(483.78015124,64.2100967)(483.79015137,64.17010061)
\curveto(483.80015122,64.13009678)(483.80015122,64.09509681)(483.79015137,64.06510061)
\lineto(483.79015137,63.97510061)
\curveto(483.77015125,63.91509699)(483.75515127,63.85009706)(483.74515137,63.78010061)
\curveto(483.74515128,63.72009719)(483.74015128,63.65509725)(483.73015137,63.58510061)
\curveto(483.68015134,63.41509749)(483.63015139,63.25509765)(483.58015137,63.10510061)
\curveto(483.53015149,62.95509795)(483.46515156,62.8100981)(483.38515137,62.67010061)
\curveto(483.34515168,62.62009829)(483.31515171,62.56509834)(483.29515137,62.50510061)
\curveto(483.26515176,62.45509845)(483.23015179,62.4050985)(483.19015137,62.35510061)
\curveto(483.01015201,62.11509879)(482.79015223,61.91509899)(482.53015137,61.75510061)
\curveto(482.27015275,61.59509931)(481.98515304,61.45509945)(481.67515137,61.33510061)
\curveto(481.53515349,61.27509963)(481.39515363,61.23009968)(481.25515137,61.20010061)
\curveto(481.10515392,61.17009974)(480.95015407,61.13509977)(480.79015137,61.09510061)
\curveto(480.68015434,61.07509983)(480.57015445,61.06009985)(480.46015137,61.05010061)
\curveto(480.35015467,61.04009987)(480.24015478,61.02509988)(480.13015137,61.00510061)
\curveto(480.09015493,60.99509991)(480.05015497,60.99009992)(480.01015137,60.99010061)
\curveto(479.97015505,61.00009991)(479.93015509,61.00009991)(479.89015137,60.99010061)
\curveto(479.84015518,60.98009993)(479.79015523,60.97509993)(479.74015137,60.97510061)
\lineto(479.57515137,60.97510061)
\curveto(479.5251555,60.95509995)(479.47515555,60.95009996)(479.42515137,60.96010061)
\curveto(479.36515566,60.97009994)(479.31015571,60.97009994)(479.26015137,60.96010061)
\curveto(479.2201558,60.95009996)(479.17515585,60.95009996)(479.12515137,60.96010061)
\curveto(479.07515595,60.97009994)(479.025156,60.96509994)(478.97515137,60.94510061)
\curveto(478.90515612,60.92509998)(478.83015619,60.92009999)(478.75015137,60.93010061)
\curveto(478.66015636,60.94009997)(478.57515645,60.94509996)(478.49515137,60.94510061)
\curveto(478.40515662,60.94509996)(478.30515672,60.94009997)(478.19515137,60.93010061)
\curveto(478.07515695,60.92009999)(477.97515705,60.92509998)(477.89515137,60.94510061)
\lineto(477.61015137,60.94510061)
\lineto(476.98015137,60.99010061)
\curveto(476.88015814,61.00009991)(476.78515824,61.0100999)(476.69515137,61.02010061)
\lineto(476.39515137,61.05010061)
\curveto(476.34515868,61.07009984)(476.29515873,61.07509983)(476.24515137,61.06510061)
\curveto(476.18515884,61.06509984)(476.13015889,61.07509983)(476.08015137,61.09510061)
\curveto(475.91015911,61.14509976)(475.74515928,61.18509972)(475.58515137,61.21510061)
\curveto(475.41515961,61.24509966)(475.25515977,61.29509961)(475.10515137,61.36510061)
\curveto(474.64516038,61.55509935)(474.27016075,61.77509913)(473.98015137,62.02510061)
\curveto(473.69016133,62.28509862)(473.44516158,62.64509826)(473.24515137,63.10510061)
\curveto(473.19516183,63.23509767)(473.16016186,63.36509754)(473.14015137,63.49510061)
\curveto(473.1201619,63.63509727)(473.09516193,63.77509713)(473.06515137,63.91510061)
\curveto(473.05516197,63.98509692)(473.05016197,64.05009686)(473.05015137,64.11010061)
\curveto(473.05016197,64.17009674)(473.04516198,64.23509667)(473.03515137,64.30510061)
\curveto(473.01516201,65.13509577)(473.16516186,65.8050951)(473.48515137,66.31510061)
\curveto(473.79516123,66.82509408)(474.23516079,67.2050937)(474.80515137,67.45510061)
\curveto(474.9251601,67.5050934)(475.05015997,67.55009336)(475.18015137,67.59010061)
\curveto(475.31015971,67.63009328)(475.44515958,67.67509323)(475.58515137,67.72510061)
\curveto(475.66515936,67.74509316)(475.75015927,67.76009315)(475.84015137,67.77010061)
\lineto(476.08015137,67.83010061)
\curveto(476.19015883,67.86009305)(476.30015872,67.87509303)(476.41015137,67.87510061)
\curveto(476.5201585,67.88509302)(476.63015839,67.90009301)(476.74015137,67.92010061)
\curveto(476.79015823,67.94009297)(476.83515819,67.94509296)(476.87515137,67.93510061)
\curveto(476.91515811,67.93509297)(476.95515807,67.94009297)(476.99515137,67.95010061)
\curveto(477.04515798,67.96009295)(477.10015792,67.96009295)(477.16015137,67.95010061)
\curveto(477.21015781,67.95009296)(477.26015776,67.95509295)(477.31015137,67.96510061)
\lineto(477.44515137,67.96510061)
\curveto(477.50515752,67.98509292)(477.57515745,67.98509292)(477.65515137,67.96510061)
\curveto(477.7251573,67.95509295)(477.79015723,67.96009295)(477.85015137,67.98010061)
\curveto(477.88015714,67.99009292)(477.9201571,67.99509291)(477.97015137,67.99510061)
\lineto(478.09015137,67.99510061)
\lineto(478.55515137,67.99510061)
\moveto(480.88015137,66.45010061)
\curveto(480.56015446,66.55009436)(480.19515483,66.6100943)(479.78515137,66.63010061)
\curveto(479.37515565,66.65009426)(478.96515606,66.66009425)(478.55515137,66.66010061)
\curveto(478.1251569,66.66009425)(477.70515732,66.65009426)(477.29515137,66.63010061)
\curveto(476.88515814,66.6100943)(476.50015852,66.56509434)(476.14015137,66.49510061)
\curveto(475.78015924,66.42509448)(475.46015956,66.31509459)(475.18015137,66.16510061)
\curveto(474.89016013,66.02509488)(474.65516037,65.83009508)(474.47515137,65.58010061)
\curveto(474.36516066,65.42009549)(474.28516074,65.24009567)(474.23515137,65.04010061)
\curveto(474.17516085,64.84009607)(474.14516088,64.59509631)(474.14515137,64.30510061)
\curveto(474.16516086,64.28509662)(474.17516085,64.25009666)(474.17515137,64.20010061)
\curveto(474.16516086,64.15009676)(474.16516086,64.1100968)(474.17515137,64.08010061)
\curveto(474.19516083,64.00009691)(474.21516081,63.92509698)(474.23515137,63.85510061)
\curveto(474.24516078,63.79509711)(474.26516076,63.73009718)(474.29515137,63.66010061)
\curveto(474.41516061,63.39009752)(474.58516044,63.17009774)(474.80515137,63.00010061)
\curveto(475.01516001,62.84009807)(475.26015976,62.7050982)(475.54015137,62.59510061)
\curveto(475.65015937,62.54509836)(475.77015925,62.5050984)(475.90015137,62.47510061)
\curveto(476.020159,62.45509845)(476.14515888,62.43009848)(476.27515137,62.40010061)
\curveto(476.3251587,62.38009853)(476.38015864,62.37009854)(476.44015137,62.37010061)
\curveto(476.49015853,62.37009854)(476.54015848,62.36509854)(476.59015137,62.35510061)
\curveto(476.68015834,62.34509856)(476.77515825,62.33509857)(476.87515137,62.32510061)
\curveto(476.96515806,62.31509859)(477.06015796,62.3050986)(477.16015137,62.29510061)
\curveto(477.24015778,62.29509861)(477.3251577,62.29009862)(477.41515137,62.28010061)
\lineto(477.65515137,62.28010061)
\lineto(477.83515137,62.28010061)
\curveto(477.86515716,62.27009864)(477.90015712,62.26509864)(477.94015137,62.26510061)
\lineto(478.07515137,62.26510061)
\lineto(478.52515137,62.26510061)
\curveto(478.60515642,62.26509864)(478.69015633,62.26009865)(478.78015137,62.25010061)
\curveto(478.86015616,62.25009866)(478.93515609,62.26009865)(479.00515137,62.28010061)
\lineto(479.27515137,62.28010061)
\curveto(479.29515573,62.28009863)(479.3251557,62.27509863)(479.36515137,62.26510061)
\curveto(479.39515563,62.26509864)(479.4201556,62.27009864)(479.44015137,62.28010061)
\curveto(479.54015548,62.29009862)(479.64015538,62.29509861)(479.74015137,62.29510061)
\curveto(479.83015519,62.3050986)(479.93015509,62.31509859)(480.04015137,62.32510061)
\curveto(480.16015486,62.35509855)(480.28515474,62.37009854)(480.41515137,62.37010061)
\curveto(480.53515449,62.38009853)(480.65015437,62.4050985)(480.76015137,62.44510061)
\curveto(481.06015396,62.52509838)(481.3251537,62.6100983)(481.55515137,62.70010061)
\curveto(481.78515324,62.80009811)(482.00015302,62.94509796)(482.20015137,63.13510061)
\curveto(482.40015262,63.34509756)(482.55015247,63.6100973)(482.65015137,63.93010061)
\curveto(482.67015235,63.97009694)(482.68015234,64.0050969)(482.68015137,64.03510061)
\curveto(482.67015235,64.07509683)(482.67515235,64.12009679)(482.69515137,64.17010061)
\curveto(482.70515232,64.2100967)(482.71515231,64.28009663)(482.72515137,64.38010061)
\curveto(482.73515229,64.49009642)(482.73015229,64.57509633)(482.71015137,64.63510061)
\curveto(482.69015233,64.7050962)(482.68015234,64.77509613)(482.68015137,64.84510061)
\curveto(482.67015235,64.91509599)(482.65515237,64.98009593)(482.63515137,65.04010061)
\curveto(482.57515245,65.24009567)(482.49015253,65.42009549)(482.38015137,65.58010061)
\curveto(482.36015266,65.6100953)(482.34015268,65.63509527)(482.32015137,65.65510061)
\lineto(482.26015137,65.71510061)
\curveto(482.24015278,65.75509515)(482.20015282,65.8050951)(482.14015137,65.86510061)
\curveto(482.00015302,65.96509494)(481.87015315,66.05009486)(481.75015137,66.12010061)
\curveto(481.63015339,66.19009472)(481.48515354,66.26009465)(481.31515137,66.33010061)
\curveto(481.24515378,66.36009455)(481.17515385,66.38009453)(481.10515137,66.39010061)
\curveto(481.03515399,66.4100945)(480.96015406,66.43009448)(480.88015137,66.45010061)
}
}
{
\newrgbcolor{curcolor}{0 0 0}
\pscustom[linestyle=none,fillstyle=solid,fillcolor=curcolor]
{
\newpath
\moveto(473.03515137,73.40470998)
\curveto(473.03516199,73.50470513)(473.04516198,73.59970503)(473.06515137,73.68970998)
\curveto(473.07516195,73.77970485)(473.10516192,73.84470479)(473.15515137,73.88470998)
\curveto(473.23516179,73.94470469)(473.34016168,73.97470466)(473.47015137,73.97470998)
\lineto(473.86015137,73.97470998)
\lineto(475.36015137,73.97470998)
\lineto(481.75015137,73.97470998)
\lineto(482.92015137,73.97470998)
\lineto(483.23515137,73.97470998)
\curveto(483.33515169,73.98470465)(483.41515161,73.96970466)(483.47515137,73.92970998)
\curveto(483.55515147,73.87970475)(483.60515142,73.80470483)(483.62515137,73.70470998)
\curveto(483.63515139,73.61470502)(483.64015138,73.50470513)(483.64015137,73.37470998)
\lineto(483.64015137,73.14970998)
\curveto(483.6201514,73.06970556)(483.60515142,72.99970563)(483.59515137,72.93970998)
\curveto(483.57515145,72.87970575)(483.53515149,72.8297058)(483.47515137,72.78970998)
\curveto(483.41515161,72.74970588)(483.34015168,72.7297059)(483.25015137,72.72970998)
\lineto(482.95015137,72.72970998)
\lineto(481.85515137,72.72970998)
\lineto(476.51515137,72.72970998)
\curveto(476.4251586,72.70970592)(476.35015867,72.69470594)(476.29015137,72.68470998)
\curveto(476.2201588,72.68470595)(476.16015886,72.65470598)(476.11015137,72.59470998)
\curveto(476.06015896,72.52470611)(476.03515899,72.4347062)(476.03515137,72.32470998)
\curveto(476.025159,72.22470641)(476.020159,72.11470652)(476.02015137,71.99470998)
\lineto(476.02015137,70.85470998)
\lineto(476.02015137,70.35970998)
\curveto(476.01015901,70.19970843)(475.95015907,70.08970854)(475.84015137,70.02970998)
\curveto(475.81015921,70.00970862)(475.78015924,69.99970863)(475.75015137,69.99970998)
\curveto(475.71015931,69.99970863)(475.66515936,69.99470864)(475.61515137,69.98470998)
\curveto(475.49515953,69.96470867)(475.38515964,69.96970866)(475.28515137,69.99970998)
\curveto(475.18515984,70.03970859)(475.11515991,70.09470854)(475.07515137,70.16470998)
\curveto(475.02516,70.24470839)(475.00016002,70.36470827)(475.00015137,70.52470998)
\curveto(475.00016002,70.68470795)(474.98516004,70.81970781)(474.95515137,70.92970998)
\curveto(474.94516008,70.97970765)(474.94016008,71.0347076)(474.94015137,71.09470998)
\curveto(474.93016009,71.15470748)(474.91516011,71.21470742)(474.89515137,71.27470998)
\curveto(474.84516018,71.42470721)(474.79516023,71.56970706)(474.74515137,71.70970998)
\curveto(474.68516034,71.84970678)(474.61516041,71.98470665)(474.53515137,72.11470998)
\curveto(474.44516058,72.25470638)(474.34016068,72.37470626)(474.22015137,72.47470998)
\curveto(474.10016092,72.57470606)(473.97016105,72.66970596)(473.83015137,72.75970998)
\curveto(473.73016129,72.81970581)(473.6201614,72.86470577)(473.50015137,72.89470998)
\curveto(473.38016164,72.9347057)(473.27516175,72.98470565)(473.18515137,73.04470998)
\curveto(473.1251619,73.09470554)(473.08516194,73.16470547)(473.06515137,73.25470998)
\curveto(473.05516197,73.27470536)(473.05016197,73.29970533)(473.05015137,73.32970998)
\curveto(473.05016197,73.35970527)(473.04516198,73.38470525)(473.03515137,73.40470998)
}
}
{
\newrgbcolor{curcolor}{0 0 0}
\pscustom[linestyle=none,fillstyle=solid,fillcolor=curcolor]
{
\newpath
\moveto(482.00515137,78.63431936)
\lineto(482.00515137,79.26431936)
\lineto(482.00515137,79.45931936)
\curveto(482.00515302,79.52931683)(482.01515301,79.58931677)(482.03515137,79.63931936)
\curveto(482.07515295,79.70931665)(482.11515291,79.7593166)(482.15515137,79.78931936)
\curveto(482.20515282,79.82931653)(482.27015275,79.84931651)(482.35015137,79.84931936)
\curveto(482.43015259,79.8593165)(482.51515251,79.86431649)(482.60515137,79.86431936)
\lineto(483.32515137,79.86431936)
\curveto(483.80515122,79.86431649)(484.21515081,79.80431655)(484.55515137,79.68431936)
\curveto(484.89515013,79.56431679)(485.17014985,79.36931699)(485.38015137,79.09931936)
\curveto(485.43014959,79.02931733)(485.47514955,78.9593174)(485.51515137,78.88931936)
\curveto(485.56514946,78.82931753)(485.61014941,78.7543176)(485.65015137,78.66431936)
\curveto(485.66014936,78.64431771)(485.67014935,78.61431774)(485.68015137,78.57431936)
\curveto(485.70014932,78.53431782)(485.70514932,78.48931787)(485.69515137,78.43931936)
\curveto(485.66514936,78.34931801)(485.59014943,78.29431806)(485.47015137,78.27431936)
\curveto(485.36014966,78.2543181)(485.26514976,78.26931809)(485.18515137,78.31931936)
\curveto(485.11514991,78.34931801)(485.05014997,78.39431796)(484.99015137,78.45431936)
\curveto(484.94015008,78.52431783)(484.89015013,78.58931777)(484.84015137,78.64931936)
\curveto(484.79015023,78.71931764)(484.71515031,78.77931758)(484.61515137,78.82931936)
\curveto(484.5251505,78.88931747)(484.43515059,78.93931742)(484.34515137,78.97931936)
\curveto(484.31515071,78.99931736)(484.25515077,79.02431733)(484.16515137,79.05431936)
\curveto(484.08515094,79.08431727)(484.01515101,79.08931727)(483.95515137,79.06931936)
\curveto(483.81515121,79.03931732)(483.7251513,78.97931738)(483.68515137,78.88931936)
\curveto(483.65515137,78.80931755)(483.64015138,78.71931764)(483.64015137,78.61931936)
\curveto(483.64015138,78.51931784)(483.61515141,78.43431792)(483.56515137,78.36431936)
\curveto(483.49515153,78.27431808)(483.35515167,78.22931813)(483.14515137,78.22931936)
\lineto(482.59015137,78.22931936)
\lineto(482.36515137,78.22931936)
\curveto(482.28515274,78.23931812)(482.2201528,78.2593181)(482.17015137,78.28931936)
\curveto(482.09015293,78.34931801)(482.04515298,78.41931794)(482.03515137,78.49931936)
\curveto(482.025153,78.51931784)(482.020153,78.53931782)(482.02015137,78.55931936)
\curveto(482.020153,78.58931777)(482.01515301,78.61431774)(482.00515137,78.63431936)
}
}
{
\newrgbcolor{curcolor}{0 0 0}
\pscustom[linestyle=none,fillstyle=solid,fillcolor=curcolor]
{
}
}
{
\newrgbcolor{curcolor}{0 0 0}
\pscustom[linestyle=none,fillstyle=solid,fillcolor=curcolor]
{
\newpath
\moveto(473.03515137,89.26463186)
\curveto(473.025162,89.95462722)(473.14516188,90.55462662)(473.39515137,91.06463186)
\curveto(473.64516138,91.58462559)(473.98016104,91.9796252)(474.40015137,92.24963186)
\curveto(474.48016054,92.29962488)(474.57016045,92.34462483)(474.67015137,92.38463186)
\curveto(474.76016026,92.42462475)(474.85516017,92.46962471)(474.95515137,92.51963186)
\curveto(475.05515997,92.55962462)(475.15515987,92.58962459)(475.25515137,92.60963186)
\curveto(475.35515967,92.62962455)(475.46015956,92.64962453)(475.57015137,92.66963186)
\curveto(475.6201594,92.68962449)(475.66515936,92.69462448)(475.70515137,92.68463186)
\curveto(475.74515928,92.6746245)(475.79015923,92.6796245)(475.84015137,92.69963186)
\curveto(475.89015913,92.70962447)(475.97515905,92.71462446)(476.09515137,92.71463186)
\curveto(476.20515882,92.71462446)(476.29015873,92.70962447)(476.35015137,92.69963186)
\curveto(476.41015861,92.6796245)(476.47015855,92.66962451)(476.53015137,92.66963186)
\curveto(476.59015843,92.6796245)(476.65015837,92.6746245)(476.71015137,92.65463186)
\curveto(476.85015817,92.61462456)(476.98515804,92.5796246)(477.11515137,92.54963186)
\curveto(477.24515778,92.51962466)(477.37015765,92.4796247)(477.49015137,92.42963186)
\curveto(477.63015739,92.36962481)(477.75515727,92.29962488)(477.86515137,92.21963186)
\curveto(477.97515705,92.14962503)(478.08515694,92.0746251)(478.19515137,91.99463186)
\lineto(478.25515137,91.93463186)
\curveto(478.27515675,91.92462525)(478.29515673,91.90962527)(478.31515137,91.88963186)
\curveto(478.47515655,91.76962541)(478.6201564,91.63462554)(478.75015137,91.48463186)
\curveto(478.88015614,91.33462584)(479.00515602,91.174626)(479.12515137,91.00463186)
\curveto(479.34515568,90.69462648)(479.55015547,90.39962678)(479.74015137,90.11963186)
\curveto(479.88015514,89.88962729)(480.01515501,89.65962752)(480.14515137,89.42963186)
\curveto(480.27515475,89.20962797)(480.41015461,88.98962819)(480.55015137,88.76963186)
\curveto(480.7201543,88.51962866)(480.90015412,88.2796289)(481.09015137,88.04963186)
\curveto(481.28015374,87.82962935)(481.50515352,87.63962954)(481.76515137,87.47963186)
\curveto(481.8251532,87.43962974)(481.88515314,87.40462977)(481.94515137,87.37463186)
\curveto(481.99515303,87.34462983)(482.06015296,87.31462986)(482.14015137,87.28463186)
\curveto(482.21015281,87.26462991)(482.27015275,87.25962992)(482.32015137,87.26963186)
\curveto(482.39015263,87.28962989)(482.44515258,87.32462985)(482.48515137,87.37463186)
\curveto(482.51515251,87.42462975)(482.53515249,87.48462969)(482.54515137,87.55463186)
\lineto(482.54515137,87.79463186)
\lineto(482.54515137,88.54463186)
\lineto(482.54515137,91.34963186)
\lineto(482.54515137,92.00963186)
\curveto(482.54515248,92.09962508)(482.55015247,92.18462499)(482.56015137,92.26463186)
\curveto(482.56015246,92.34462483)(482.58015244,92.40962477)(482.62015137,92.45963186)
\curveto(482.66015236,92.50962467)(482.73515229,92.54962463)(482.84515137,92.57963186)
\curveto(482.94515208,92.61962456)(483.04515198,92.62962455)(483.14515137,92.60963186)
\lineto(483.28015137,92.60963186)
\curveto(483.35015167,92.58962459)(483.41015161,92.56962461)(483.46015137,92.54963186)
\curveto(483.51015151,92.52962465)(483.55015147,92.49462468)(483.58015137,92.44463186)
\curveto(483.6201514,92.39462478)(483.64015138,92.32462485)(483.64015137,92.23463186)
\lineto(483.64015137,91.96463186)
\lineto(483.64015137,91.06463186)
\lineto(483.64015137,87.55463186)
\lineto(483.64015137,86.48963186)
\curveto(483.64015138,86.40963077)(483.64515138,86.31963086)(483.65515137,86.21963186)
\curveto(483.65515137,86.11963106)(483.64515138,86.03463114)(483.62515137,85.96463186)
\curveto(483.55515147,85.75463142)(483.37515165,85.68963149)(483.08515137,85.76963186)
\curveto(483.04515198,85.7796314)(483.01015201,85.7796314)(482.98015137,85.76963186)
\curveto(482.94015208,85.76963141)(482.89515213,85.7796314)(482.84515137,85.79963186)
\curveto(482.76515226,85.81963136)(482.68015234,85.83963134)(482.59015137,85.85963186)
\curveto(482.50015252,85.8796313)(482.41515261,85.90463127)(482.33515137,85.93463186)
\curveto(481.84515318,86.09463108)(481.43015359,86.29463088)(481.09015137,86.53463186)
\curveto(480.84015418,86.71463046)(480.61515441,86.91963026)(480.41515137,87.14963186)
\curveto(480.20515482,87.3796298)(480.01015501,87.61962956)(479.83015137,87.86963186)
\curveto(479.65015537,88.12962905)(479.48015554,88.39462878)(479.32015137,88.66463186)
\curveto(479.15015587,88.94462823)(478.97515605,89.21462796)(478.79515137,89.47463186)
\curveto(478.71515631,89.58462759)(478.64015638,89.68962749)(478.57015137,89.78963186)
\curveto(478.50015652,89.89962728)(478.4251566,90.00962717)(478.34515137,90.11963186)
\curveto(478.31515671,90.15962702)(478.28515674,90.19462698)(478.25515137,90.22463186)
\curveto(478.21515681,90.26462691)(478.18515684,90.30462687)(478.16515137,90.34463186)
\curveto(478.05515697,90.48462669)(477.93015709,90.60962657)(477.79015137,90.71963186)
\curveto(477.76015726,90.73962644)(477.73515729,90.76462641)(477.71515137,90.79463186)
\curveto(477.68515734,90.82462635)(477.65515737,90.84962633)(477.62515137,90.86963186)
\curveto(477.5251575,90.94962623)(477.4251576,91.01462616)(477.32515137,91.06463186)
\curveto(477.2251578,91.12462605)(477.11515791,91.179626)(476.99515137,91.22963186)
\curveto(476.9251581,91.25962592)(476.85015817,91.2796259)(476.77015137,91.28963186)
\lineto(476.53015137,91.34963186)
\lineto(476.44015137,91.34963186)
\curveto(476.41015861,91.35962582)(476.38015864,91.36462581)(476.35015137,91.36463186)
\curveto(476.28015874,91.38462579)(476.18515884,91.38962579)(476.06515137,91.37963186)
\curveto(475.93515909,91.3796258)(475.83515919,91.36962581)(475.76515137,91.34963186)
\curveto(475.68515934,91.32962585)(475.61015941,91.30962587)(475.54015137,91.28963186)
\curveto(475.46015956,91.2796259)(475.38015964,91.25962592)(475.30015137,91.22963186)
\curveto(475.06015996,91.11962606)(474.86016016,90.96962621)(474.70015137,90.77963186)
\curveto(474.53016049,90.59962658)(474.39016063,90.3796268)(474.28015137,90.11963186)
\curveto(474.26016076,90.04962713)(474.24516078,89.9796272)(474.23515137,89.90963186)
\curveto(474.21516081,89.83962734)(474.19516083,89.76462741)(474.17515137,89.68463186)
\curveto(474.15516087,89.60462757)(474.14516088,89.49462768)(474.14515137,89.35463186)
\curveto(474.14516088,89.22462795)(474.15516087,89.11962806)(474.17515137,89.03963186)
\curveto(474.18516084,88.9796282)(474.19016083,88.92462825)(474.19015137,88.87463186)
\curveto(474.19016083,88.82462835)(474.20016082,88.7746284)(474.22015137,88.72463186)
\curveto(474.26016076,88.62462855)(474.30016072,88.52962865)(474.34015137,88.43963186)
\curveto(474.38016064,88.35962882)(474.4251606,88.2796289)(474.47515137,88.19963186)
\curveto(474.49516053,88.16962901)(474.5201605,88.13962904)(474.55015137,88.10963186)
\curveto(474.58016044,88.08962909)(474.60516042,88.06462911)(474.62515137,88.03463186)
\lineto(474.70015137,87.95963186)
\curveto(474.7201603,87.92962925)(474.74016028,87.90462927)(474.76015137,87.88463186)
\lineto(474.97015137,87.73463186)
\curveto(475.03015999,87.69462948)(475.09515993,87.64962953)(475.16515137,87.59963186)
\curveto(475.25515977,87.53962964)(475.36015966,87.48962969)(475.48015137,87.44963186)
\curveto(475.59015943,87.41962976)(475.70015932,87.38462979)(475.81015137,87.34463186)
\curveto(475.9201591,87.30462987)(476.06515896,87.2796299)(476.24515137,87.26963186)
\curveto(476.41515861,87.25962992)(476.54015848,87.22962995)(476.62015137,87.17963186)
\curveto(476.70015832,87.12963005)(476.74515828,87.05463012)(476.75515137,86.95463186)
\curveto(476.76515826,86.85463032)(476.77015825,86.74463043)(476.77015137,86.62463186)
\curveto(476.77015825,86.58463059)(476.77515825,86.54463063)(476.78515137,86.50463186)
\curveto(476.78515824,86.46463071)(476.78015824,86.42963075)(476.77015137,86.39963186)
\curveto(476.75015827,86.34963083)(476.74015828,86.29963088)(476.74015137,86.24963186)
\curveto(476.74015828,86.20963097)(476.73015829,86.16963101)(476.71015137,86.12963186)
\curveto(476.65015837,86.03963114)(476.51515851,85.99463118)(476.30515137,85.99463186)
\lineto(476.18515137,85.99463186)
\curveto(476.1251589,86.00463117)(476.06515896,86.00963117)(476.00515137,86.00963186)
\curveto(475.93515909,86.01963116)(475.87015915,86.02963115)(475.81015137,86.03963186)
\curveto(475.70015932,86.05963112)(475.60015942,86.0796311)(475.51015137,86.09963186)
\curveto(475.41015961,86.11963106)(475.31515971,86.14963103)(475.22515137,86.18963186)
\curveto(475.15515987,86.20963097)(475.09515993,86.22963095)(475.04515137,86.24963186)
\lineto(474.86515137,86.30963186)
\curveto(474.60516042,86.42963075)(474.36016066,86.58463059)(474.13015137,86.77463186)
\curveto(473.90016112,86.9746302)(473.71516131,87.18962999)(473.57515137,87.41963186)
\curveto(473.49516153,87.52962965)(473.43016159,87.64462953)(473.38015137,87.76463186)
\lineto(473.23015137,88.15463186)
\curveto(473.18016184,88.26462891)(473.15016187,88.3796288)(473.14015137,88.49963186)
\curveto(473.1201619,88.61962856)(473.09516193,88.74462843)(473.06515137,88.87463186)
\curveto(473.06516196,88.94462823)(473.06516196,89.00962817)(473.06515137,89.06963186)
\curveto(473.05516197,89.12962805)(473.04516198,89.19462798)(473.03515137,89.26463186)
}
}
{
\newrgbcolor{curcolor}{0 0 0}
\pscustom[linestyle=none,fillstyle=solid,fillcolor=curcolor]
{
\newpath
\moveto(478.55515137,101.36424123)
\lineto(478.81015137,101.36424123)
\curveto(478.89015613,101.37423353)(478.96515606,101.36923353)(479.03515137,101.34924123)
\lineto(479.27515137,101.34924123)
\lineto(479.44015137,101.34924123)
\curveto(479.54015548,101.32923357)(479.64515538,101.31923358)(479.75515137,101.31924123)
\curveto(479.85515517,101.31923358)(479.95515507,101.30923359)(480.05515137,101.28924123)
\lineto(480.20515137,101.28924123)
\curveto(480.34515468,101.25923364)(480.48515454,101.23923366)(480.62515137,101.22924123)
\curveto(480.75515427,101.21923368)(480.88515414,101.19423371)(481.01515137,101.15424123)
\curveto(481.09515393,101.13423377)(481.18015384,101.11423379)(481.27015137,101.09424123)
\lineto(481.51015137,101.03424123)
\lineto(481.81015137,100.91424123)
\curveto(481.90015312,100.88423402)(481.99015303,100.84923405)(482.08015137,100.80924123)
\curveto(482.30015272,100.70923419)(482.51515251,100.57423433)(482.72515137,100.40424123)
\curveto(482.93515209,100.24423466)(483.10515192,100.06923483)(483.23515137,99.87924123)
\curveto(483.27515175,99.82923507)(483.31515171,99.76923513)(483.35515137,99.69924123)
\curveto(483.38515164,99.63923526)(483.4201516,99.57923532)(483.46015137,99.51924123)
\curveto(483.51015151,99.43923546)(483.55015147,99.34423556)(483.58015137,99.23424123)
\curveto(483.61015141,99.12423578)(483.64015138,99.01923588)(483.67015137,98.91924123)
\curveto(483.71015131,98.80923609)(483.73515129,98.6992362)(483.74515137,98.58924123)
\curveto(483.75515127,98.47923642)(483.77015125,98.36423654)(483.79015137,98.24424123)
\curveto(483.80015122,98.2042367)(483.80015122,98.15923674)(483.79015137,98.10924123)
\curveto(483.79015123,98.06923683)(483.79515123,98.02923687)(483.80515137,97.98924123)
\curveto(483.81515121,97.94923695)(483.8201512,97.89423701)(483.82015137,97.82424123)
\curveto(483.8201512,97.75423715)(483.81515121,97.7042372)(483.80515137,97.67424123)
\curveto(483.78515124,97.62423728)(483.78015124,97.57923732)(483.79015137,97.53924123)
\curveto(483.80015122,97.4992374)(483.80015122,97.46423744)(483.79015137,97.43424123)
\lineto(483.79015137,97.34424123)
\curveto(483.77015125,97.28423762)(483.75515127,97.21923768)(483.74515137,97.14924123)
\curveto(483.74515128,97.08923781)(483.74015128,97.02423788)(483.73015137,96.95424123)
\curveto(483.68015134,96.78423812)(483.63015139,96.62423828)(483.58015137,96.47424123)
\curveto(483.53015149,96.32423858)(483.46515156,96.17923872)(483.38515137,96.03924123)
\curveto(483.34515168,95.98923891)(483.31515171,95.93423897)(483.29515137,95.87424123)
\curveto(483.26515176,95.82423908)(483.23015179,95.77423913)(483.19015137,95.72424123)
\curveto(483.01015201,95.48423942)(482.79015223,95.28423962)(482.53015137,95.12424123)
\curveto(482.27015275,94.96423994)(481.98515304,94.82424008)(481.67515137,94.70424123)
\curveto(481.53515349,94.64424026)(481.39515363,94.5992403)(481.25515137,94.56924123)
\curveto(481.10515392,94.53924036)(480.95015407,94.5042404)(480.79015137,94.46424123)
\curveto(480.68015434,94.44424046)(480.57015445,94.42924047)(480.46015137,94.41924123)
\curveto(480.35015467,94.40924049)(480.24015478,94.39424051)(480.13015137,94.37424123)
\curveto(480.09015493,94.36424054)(480.05015497,94.35924054)(480.01015137,94.35924123)
\curveto(479.97015505,94.36924053)(479.93015509,94.36924053)(479.89015137,94.35924123)
\curveto(479.84015518,94.34924055)(479.79015523,94.34424056)(479.74015137,94.34424123)
\lineto(479.57515137,94.34424123)
\curveto(479.5251555,94.32424058)(479.47515555,94.31924058)(479.42515137,94.32924123)
\curveto(479.36515566,94.33924056)(479.31015571,94.33924056)(479.26015137,94.32924123)
\curveto(479.2201558,94.31924058)(479.17515585,94.31924058)(479.12515137,94.32924123)
\curveto(479.07515595,94.33924056)(479.025156,94.33424057)(478.97515137,94.31424123)
\curveto(478.90515612,94.29424061)(478.83015619,94.28924061)(478.75015137,94.29924123)
\curveto(478.66015636,94.30924059)(478.57515645,94.31424059)(478.49515137,94.31424123)
\curveto(478.40515662,94.31424059)(478.30515672,94.30924059)(478.19515137,94.29924123)
\curveto(478.07515695,94.28924061)(477.97515705,94.29424061)(477.89515137,94.31424123)
\lineto(477.61015137,94.31424123)
\lineto(476.98015137,94.35924123)
\curveto(476.88015814,94.36924053)(476.78515824,94.37924052)(476.69515137,94.38924123)
\lineto(476.39515137,94.41924123)
\curveto(476.34515868,94.43924046)(476.29515873,94.44424046)(476.24515137,94.43424123)
\curveto(476.18515884,94.43424047)(476.13015889,94.44424046)(476.08015137,94.46424123)
\curveto(475.91015911,94.51424039)(475.74515928,94.55424035)(475.58515137,94.58424123)
\curveto(475.41515961,94.61424029)(475.25515977,94.66424024)(475.10515137,94.73424123)
\curveto(474.64516038,94.92423998)(474.27016075,95.14423976)(473.98015137,95.39424123)
\curveto(473.69016133,95.65423925)(473.44516158,96.01423889)(473.24515137,96.47424123)
\curveto(473.19516183,96.6042383)(473.16016186,96.73423817)(473.14015137,96.86424123)
\curveto(473.1201619,97.0042379)(473.09516193,97.14423776)(473.06515137,97.28424123)
\curveto(473.05516197,97.35423755)(473.05016197,97.41923748)(473.05015137,97.47924123)
\curveto(473.05016197,97.53923736)(473.04516198,97.6042373)(473.03515137,97.67424123)
\curveto(473.01516201,98.5042364)(473.16516186,99.17423573)(473.48515137,99.68424123)
\curveto(473.79516123,100.19423471)(474.23516079,100.57423433)(474.80515137,100.82424123)
\curveto(474.9251601,100.87423403)(475.05015997,100.91923398)(475.18015137,100.95924123)
\curveto(475.31015971,100.9992339)(475.44515958,101.04423386)(475.58515137,101.09424123)
\curveto(475.66515936,101.11423379)(475.75015927,101.12923377)(475.84015137,101.13924123)
\lineto(476.08015137,101.19924123)
\curveto(476.19015883,101.22923367)(476.30015872,101.24423366)(476.41015137,101.24424123)
\curveto(476.5201585,101.25423365)(476.63015839,101.26923363)(476.74015137,101.28924123)
\curveto(476.79015823,101.30923359)(476.83515819,101.31423359)(476.87515137,101.30424123)
\curveto(476.91515811,101.3042336)(476.95515807,101.30923359)(476.99515137,101.31924123)
\curveto(477.04515798,101.32923357)(477.10015792,101.32923357)(477.16015137,101.31924123)
\curveto(477.21015781,101.31923358)(477.26015776,101.32423358)(477.31015137,101.33424123)
\lineto(477.44515137,101.33424123)
\curveto(477.50515752,101.35423355)(477.57515745,101.35423355)(477.65515137,101.33424123)
\curveto(477.7251573,101.32423358)(477.79015723,101.32923357)(477.85015137,101.34924123)
\curveto(477.88015714,101.35923354)(477.9201571,101.36423354)(477.97015137,101.36424123)
\lineto(478.09015137,101.36424123)
\lineto(478.55515137,101.36424123)
\moveto(480.88015137,99.81924123)
\curveto(480.56015446,99.91923498)(480.19515483,99.97923492)(479.78515137,99.99924123)
\curveto(479.37515565,100.01923488)(478.96515606,100.02923487)(478.55515137,100.02924123)
\curveto(478.1251569,100.02923487)(477.70515732,100.01923488)(477.29515137,99.99924123)
\curveto(476.88515814,99.97923492)(476.50015852,99.93423497)(476.14015137,99.86424123)
\curveto(475.78015924,99.79423511)(475.46015956,99.68423522)(475.18015137,99.53424123)
\curveto(474.89016013,99.39423551)(474.65516037,99.1992357)(474.47515137,98.94924123)
\curveto(474.36516066,98.78923611)(474.28516074,98.60923629)(474.23515137,98.40924123)
\curveto(474.17516085,98.20923669)(474.14516088,97.96423694)(474.14515137,97.67424123)
\curveto(474.16516086,97.65423725)(474.17516085,97.61923728)(474.17515137,97.56924123)
\curveto(474.16516086,97.51923738)(474.16516086,97.47923742)(474.17515137,97.44924123)
\curveto(474.19516083,97.36923753)(474.21516081,97.29423761)(474.23515137,97.22424123)
\curveto(474.24516078,97.16423774)(474.26516076,97.0992378)(474.29515137,97.02924123)
\curveto(474.41516061,96.75923814)(474.58516044,96.53923836)(474.80515137,96.36924123)
\curveto(475.01516001,96.20923869)(475.26015976,96.07423883)(475.54015137,95.96424123)
\curveto(475.65015937,95.91423899)(475.77015925,95.87423903)(475.90015137,95.84424123)
\curveto(476.020159,95.82423908)(476.14515888,95.7992391)(476.27515137,95.76924123)
\curveto(476.3251587,95.74923915)(476.38015864,95.73923916)(476.44015137,95.73924123)
\curveto(476.49015853,95.73923916)(476.54015848,95.73423917)(476.59015137,95.72424123)
\curveto(476.68015834,95.71423919)(476.77515825,95.7042392)(476.87515137,95.69424123)
\curveto(476.96515806,95.68423922)(477.06015796,95.67423923)(477.16015137,95.66424123)
\curveto(477.24015778,95.66423924)(477.3251577,95.65923924)(477.41515137,95.64924123)
\lineto(477.65515137,95.64924123)
\lineto(477.83515137,95.64924123)
\curveto(477.86515716,95.63923926)(477.90015712,95.63423927)(477.94015137,95.63424123)
\lineto(478.07515137,95.63424123)
\lineto(478.52515137,95.63424123)
\curveto(478.60515642,95.63423927)(478.69015633,95.62923927)(478.78015137,95.61924123)
\curveto(478.86015616,95.61923928)(478.93515609,95.62923927)(479.00515137,95.64924123)
\lineto(479.27515137,95.64924123)
\curveto(479.29515573,95.64923925)(479.3251557,95.64423926)(479.36515137,95.63424123)
\curveto(479.39515563,95.63423927)(479.4201556,95.63923926)(479.44015137,95.64924123)
\curveto(479.54015548,95.65923924)(479.64015538,95.66423924)(479.74015137,95.66424123)
\curveto(479.83015519,95.67423923)(479.93015509,95.68423922)(480.04015137,95.69424123)
\curveto(480.16015486,95.72423918)(480.28515474,95.73923916)(480.41515137,95.73924123)
\curveto(480.53515449,95.74923915)(480.65015437,95.77423913)(480.76015137,95.81424123)
\curveto(481.06015396,95.89423901)(481.3251537,95.97923892)(481.55515137,96.06924123)
\curveto(481.78515324,96.16923873)(482.00015302,96.31423859)(482.20015137,96.50424123)
\curveto(482.40015262,96.71423819)(482.55015247,96.97923792)(482.65015137,97.29924123)
\curveto(482.67015235,97.33923756)(482.68015234,97.37423753)(482.68015137,97.40424123)
\curveto(482.67015235,97.44423746)(482.67515235,97.48923741)(482.69515137,97.53924123)
\curveto(482.70515232,97.57923732)(482.71515231,97.64923725)(482.72515137,97.74924123)
\curveto(482.73515229,97.85923704)(482.73015229,97.94423696)(482.71015137,98.00424123)
\curveto(482.69015233,98.07423683)(482.68015234,98.14423676)(482.68015137,98.21424123)
\curveto(482.67015235,98.28423662)(482.65515237,98.34923655)(482.63515137,98.40924123)
\curveto(482.57515245,98.60923629)(482.49015253,98.78923611)(482.38015137,98.94924123)
\curveto(482.36015266,98.97923592)(482.34015268,99.0042359)(482.32015137,99.02424123)
\lineto(482.26015137,99.08424123)
\curveto(482.24015278,99.12423578)(482.20015282,99.17423573)(482.14015137,99.23424123)
\curveto(482.00015302,99.33423557)(481.87015315,99.41923548)(481.75015137,99.48924123)
\curveto(481.63015339,99.55923534)(481.48515354,99.62923527)(481.31515137,99.69924123)
\curveto(481.24515378,99.72923517)(481.17515385,99.74923515)(481.10515137,99.75924123)
\curveto(481.03515399,99.77923512)(480.96015406,99.7992351)(480.88015137,99.81924123)
}
}
{
\newrgbcolor{curcolor}{0 0 0}
\pscustom[linestyle=none,fillstyle=solid,fillcolor=curcolor]
{
\newpath
\moveto(473.03515137,106.77385061)
\curveto(473.03516199,106.87384575)(473.04516198,106.96884566)(473.06515137,107.05885061)
\curveto(473.07516195,107.14884548)(473.10516192,107.21384541)(473.15515137,107.25385061)
\curveto(473.23516179,107.31384531)(473.34016168,107.34384528)(473.47015137,107.34385061)
\lineto(473.86015137,107.34385061)
\lineto(475.36015137,107.34385061)
\lineto(481.75015137,107.34385061)
\lineto(482.92015137,107.34385061)
\lineto(483.23515137,107.34385061)
\curveto(483.33515169,107.35384527)(483.41515161,107.33884529)(483.47515137,107.29885061)
\curveto(483.55515147,107.24884538)(483.60515142,107.17384545)(483.62515137,107.07385061)
\curveto(483.63515139,106.98384564)(483.64015138,106.87384575)(483.64015137,106.74385061)
\lineto(483.64015137,106.51885061)
\curveto(483.6201514,106.43884619)(483.60515142,106.36884626)(483.59515137,106.30885061)
\curveto(483.57515145,106.24884638)(483.53515149,106.19884643)(483.47515137,106.15885061)
\curveto(483.41515161,106.11884651)(483.34015168,106.09884653)(483.25015137,106.09885061)
\lineto(482.95015137,106.09885061)
\lineto(481.85515137,106.09885061)
\lineto(476.51515137,106.09885061)
\curveto(476.4251586,106.07884655)(476.35015867,106.06384656)(476.29015137,106.05385061)
\curveto(476.2201588,106.05384657)(476.16015886,106.0238466)(476.11015137,105.96385061)
\curveto(476.06015896,105.89384673)(476.03515899,105.80384682)(476.03515137,105.69385061)
\curveto(476.025159,105.59384703)(476.020159,105.48384714)(476.02015137,105.36385061)
\lineto(476.02015137,104.22385061)
\lineto(476.02015137,103.72885061)
\curveto(476.01015901,103.56884906)(475.95015907,103.45884917)(475.84015137,103.39885061)
\curveto(475.81015921,103.37884925)(475.78015924,103.36884926)(475.75015137,103.36885061)
\curveto(475.71015931,103.36884926)(475.66515936,103.36384926)(475.61515137,103.35385061)
\curveto(475.49515953,103.33384929)(475.38515964,103.33884929)(475.28515137,103.36885061)
\curveto(475.18515984,103.40884922)(475.11515991,103.46384916)(475.07515137,103.53385061)
\curveto(475.02516,103.61384901)(475.00016002,103.73384889)(475.00015137,103.89385061)
\curveto(475.00016002,104.05384857)(474.98516004,104.18884844)(474.95515137,104.29885061)
\curveto(474.94516008,104.34884828)(474.94016008,104.40384822)(474.94015137,104.46385061)
\curveto(474.93016009,104.5238481)(474.91516011,104.58384804)(474.89515137,104.64385061)
\curveto(474.84516018,104.79384783)(474.79516023,104.93884769)(474.74515137,105.07885061)
\curveto(474.68516034,105.21884741)(474.61516041,105.35384727)(474.53515137,105.48385061)
\curveto(474.44516058,105.623847)(474.34016068,105.74384688)(474.22015137,105.84385061)
\curveto(474.10016092,105.94384668)(473.97016105,106.03884659)(473.83015137,106.12885061)
\curveto(473.73016129,106.18884644)(473.6201614,106.23384639)(473.50015137,106.26385061)
\curveto(473.38016164,106.30384632)(473.27516175,106.35384627)(473.18515137,106.41385061)
\curveto(473.1251619,106.46384616)(473.08516194,106.53384609)(473.06515137,106.62385061)
\curveto(473.05516197,106.64384598)(473.05016197,106.66884596)(473.05015137,106.69885061)
\curveto(473.05016197,106.7288459)(473.04516198,106.75384587)(473.03515137,106.77385061)
}
}
{
\newrgbcolor{curcolor}{0 0 0}
\pscustom[linestyle=none,fillstyle=solid,fillcolor=curcolor]
{
\newpath
\moveto(473.03515137,115.12345998)
\curveto(473.03516199,115.22345513)(473.04516198,115.31845503)(473.06515137,115.40845998)
\curveto(473.07516195,115.49845485)(473.10516192,115.56345479)(473.15515137,115.60345998)
\curveto(473.23516179,115.66345469)(473.34016168,115.69345466)(473.47015137,115.69345998)
\lineto(473.86015137,115.69345998)
\lineto(475.36015137,115.69345998)
\lineto(481.75015137,115.69345998)
\lineto(482.92015137,115.69345998)
\lineto(483.23515137,115.69345998)
\curveto(483.33515169,115.70345465)(483.41515161,115.68845466)(483.47515137,115.64845998)
\curveto(483.55515147,115.59845475)(483.60515142,115.52345483)(483.62515137,115.42345998)
\curveto(483.63515139,115.33345502)(483.64015138,115.22345513)(483.64015137,115.09345998)
\lineto(483.64015137,114.86845998)
\curveto(483.6201514,114.78845556)(483.60515142,114.71845563)(483.59515137,114.65845998)
\curveto(483.57515145,114.59845575)(483.53515149,114.5484558)(483.47515137,114.50845998)
\curveto(483.41515161,114.46845588)(483.34015168,114.4484559)(483.25015137,114.44845998)
\lineto(482.95015137,114.44845998)
\lineto(481.85515137,114.44845998)
\lineto(476.51515137,114.44845998)
\curveto(476.4251586,114.42845592)(476.35015867,114.41345594)(476.29015137,114.40345998)
\curveto(476.2201588,114.40345595)(476.16015886,114.37345598)(476.11015137,114.31345998)
\curveto(476.06015896,114.24345611)(476.03515899,114.1534562)(476.03515137,114.04345998)
\curveto(476.025159,113.94345641)(476.020159,113.83345652)(476.02015137,113.71345998)
\lineto(476.02015137,112.57345998)
\lineto(476.02015137,112.07845998)
\curveto(476.01015901,111.91845843)(475.95015907,111.80845854)(475.84015137,111.74845998)
\curveto(475.81015921,111.72845862)(475.78015924,111.71845863)(475.75015137,111.71845998)
\curveto(475.71015931,111.71845863)(475.66515936,111.71345864)(475.61515137,111.70345998)
\curveto(475.49515953,111.68345867)(475.38515964,111.68845866)(475.28515137,111.71845998)
\curveto(475.18515984,111.75845859)(475.11515991,111.81345854)(475.07515137,111.88345998)
\curveto(475.02516,111.96345839)(475.00016002,112.08345827)(475.00015137,112.24345998)
\curveto(475.00016002,112.40345795)(474.98516004,112.53845781)(474.95515137,112.64845998)
\curveto(474.94516008,112.69845765)(474.94016008,112.7534576)(474.94015137,112.81345998)
\curveto(474.93016009,112.87345748)(474.91516011,112.93345742)(474.89515137,112.99345998)
\curveto(474.84516018,113.14345721)(474.79516023,113.28845706)(474.74515137,113.42845998)
\curveto(474.68516034,113.56845678)(474.61516041,113.70345665)(474.53515137,113.83345998)
\curveto(474.44516058,113.97345638)(474.34016068,114.09345626)(474.22015137,114.19345998)
\curveto(474.10016092,114.29345606)(473.97016105,114.38845596)(473.83015137,114.47845998)
\curveto(473.73016129,114.53845581)(473.6201614,114.58345577)(473.50015137,114.61345998)
\curveto(473.38016164,114.6534557)(473.27516175,114.70345565)(473.18515137,114.76345998)
\curveto(473.1251619,114.81345554)(473.08516194,114.88345547)(473.06515137,114.97345998)
\curveto(473.05516197,114.99345536)(473.05016197,115.01845533)(473.05015137,115.04845998)
\curveto(473.05016197,115.07845527)(473.04516198,115.10345525)(473.03515137,115.12345998)
}
}
{
\newrgbcolor{curcolor}{0 0 0}
\pscustom[linestyle=none,fillstyle=solid,fillcolor=curcolor]
{
\newpath
\moveto(494.90647095,29.18119436)
\lineto(494.90647095,30.09619436)
\curveto(494.90648164,30.19619171)(494.90648164,30.29119161)(494.90647095,30.38119436)
\curveto(494.90648164,30.47119143)(494.92648162,30.54619136)(494.96647095,30.60619436)
\curveto(495.02648152,30.69619121)(495.10648144,30.75619115)(495.20647095,30.78619436)
\curveto(495.30648124,30.82619108)(495.41148114,30.87119103)(495.52147095,30.92119436)
\curveto(495.71148084,31.0011909)(495.90148065,31.07119083)(496.09147095,31.13119436)
\curveto(496.28148027,31.2011907)(496.47148008,31.27619063)(496.66147095,31.35619436)
\curveto(496.84147971,31.42619048)(497.02647952,31.49119041)(497.21647095,31.55119436)
\curveto(497.39647915,31.61119029)(497.57647897,31.68119022)(497.75647095,31.76119436)
\curveto(497.89647865,31.82119008)(498.04147851,31.87619003)(498.19147095,31.92619436)
\curveto(498.34147821,31.97618993)(498.48647806,32.03118987)(498.62647095,32.09119436)
\curveto(499.07647747,32.27118963)(499.53147702,32.44118946)(499.99147095,32.60119436)
\curveto(500.44147611,32.76118914)(500.89147566,32.93118897)(501.34147095,33.11119436)
\curveto(501.39147516,33.13118877)(501.44147511,33.14618876)(501.49147095,33.15619436)
\lineto(501.64147095,33.21619436)
\curveto(501.86147469,33.3061886)(502.08647446,33.39118851)(502.31647095,33.47119436)
\curveto(502.53647401,33.55118835)(502.75647379,33.63618827)(502.97647095,33.72619436)
\curveto(503.06647348,33.76618814)(503.17647337,33.8061881)(503.30647095,33.84619436)
\curveto(503.42647312,33.88618802)(503.49647305,33.95118795)(503.51647095,34.04119436)
\curveto(503.52647302,34.08118782)(503.52647302,34.11118779)(503.51647095,34.13119436)
\lineto(503.45647095,34.19119436)
\curveto(503.40647314,34.24118766)(503.3514732,34.27618763)(503.29147095,34.29619436)
\curveto(503.23147332,34.32618758)(503.16647338,34.35618755)(503.09647095,34.38619436)
\lineto(502.46647095,34.62619436)
\curveto(502.2464743,34.7061872)(502.03147452,34.78618712)(501.82147095,34.86619436)
\lineto(501.67147095,34.92619436)
\lineto(501.49147095,34.98619436)
\curveto(501.30147525,35.06618684)(501.11147544,35.13618677)(500.92147095,35.19619436)
\curveto(500.72147583,35.26618664)(500.52147603,35.34118656)(500.32147095,35.42119436)
\curveto(499.74147681,35.66118624)(499.15647739,35.88118602)(498.56647095,36.08119436)
\curveto(497.97647857,36.29118561)(497.39147916,36.51618539)(496.81147095,36.75619436)
\curveto(496.61147994,36.83618507)(496.40648014,36.91118499)(496.19647095,36.98119436)
\curveto(495.98648056,37.06118484)(495.78148077,37.14118476)(495.58147095,37.22119436)
\curveto(495.50148105,37.26118464)(495.40148115,37.29618461)(495.28147095,37.32619436)
\curveto(495.16148139,37.36618454)(495.07648147,37.42118448)(495.02647095,37.49119436)
\curveto(494.98648156,37.55118435)(494.95648159,37.62618428)(494.93647095,37.71619436)
\curveto(494.91648163,37.81618409)(494.90648164,37.92618398)(494.90647095,38.04619436)
\curveto(494.89648165,38.16618374)(494.89648165,38.28618362)(494.90647095,38.40619436)
\curveto(494.90648164,38.52618338)(494.90648164,38.63618327)(494.90647095,38.73619436)
\curveto(494.90648164,38.82618308)(494.90648164,38.91618299)(494.90647095,39.00619436)
\curveto(494.90648164,39.1061828)(494.92648162,39.18118272)(494.96647095,39.23119436)
\curveto(495.01648153,39.32118258)(495.10648144,39.37118253)(495.23647095,39.38119436)
\curveto(495.36648118,39.39118251)(495.50648104,39.39618251)(495.65647095,39.39619436)
\lineto(497.30647095,39.39619436)
\lineto(503.57647095,39.39619436)
\lineto(504.83647095,39.39619436)
\curveto(504.9464716,39.39618251)(505.05647149,39.39618251)(505.16647095,39.39619436)
\curveto(505.27647127,39.4061825)(505.36147119,39.38618252)(505.42147095,39.33619436)
\curveto(505.48147107,39.3061826)(505.52147103,39.26118264)(505.54147095,39.20119436)
\curveto(505.551471,39.14118276)(505.56647098,39.07118283)(505.58647095,38.99119436)
\lineto(505.58647095,38.75119436)
\lineto(505.58647095,38.39119436)
\curveto(505.57647097,38.28118362)(505.53147102,38.2011837)(505.45147095,38.15119436)
\curveto(505.42147113,38.13118377)(505.39147116,38.11618379)(505.36147095,38.10619436)
\curveto(505.32147123,38.1061838)(505.27647127,38.09618381)(505.22647095,38.07619436)
\lineto(505.06147095,38.07619436)
\curveto(505.00147155,38.06618384)(504.93147162,38.06118384)(504.85147095,38.06119436)
\curveto(504.77147178,38.07118383)(504.69647185,38.07618383)(504.62647095,38.07619436)
\lineto(503.78647095,38.07619436)
\lineto(499.36147095,38.07619436)
\curveto(499.11147744,38.07618383)(498.86147769,38.07618383)(498.61147095,38.07619436)
\curveto(498.3514782,38.07618383)(498.10147845,38.07118383)(497.86147095,38.06119436)
\curveto(497.76147879,38.06118384)(497.6514789,38.05618385)(497.53147095,38.04619436)
\curveto(497.41147914,38.03618387)(497.3514792,37.98118392)(497.35147095,37.88119436)
\lineto(497.36647095,37.88119436)
\curveto(497.38647916,37.81118409)(497.4514791,37.75118415)(497.56147095,37.70119436)
\curveto(497.67147888,37.66118424)(497.76647878,37.62618428)(497.84647095,37.59619436)
\curveto(498.01647853,37.52618438)(498.19147836,37.46118444)(498.37147095,37.40119436)
\curveto(498.54147801,37.34118456)(498.71147784,37.27118463)(498.88147095,37.19119436)
\curveto(498.93147762,37.17118473)(498.97647757,37.15618475)(499.01647095,37.14619436)
\curveto(499.05647749,37.13618477)(499.10147745,37.12118478)(499.15147095,37.10119436)
\curveto(499.33147722,37.02118488)(499.51647703,36.95118495)(499.70647095,36.89119436)
\curveto(499.88647666,36.84118506)(500.06647648,36.77618513)(500.24647095,36.69619436)
\curveto(500.39647615,36.62618528)(500.551476,36.56618534)(500.71147095,36.51619436)
\curveto(500.86147569,36.46618544)(501.01147554,36.41118549)(501.16147095,36.35119436)
\curveto(501.63147492,36.15118575)(502.10647444,35.97118593)(502.58647095,35.81119436)
\curveto(503.05647349,35.65118625)(503.52147303,35.47618643)(503.98147095,35.28619436)
\curveto(504.16147239,35.2061867)(504.34147221,35.13618677)(504.52147095,35.07619436)
\curveto(504.70147185,35.01618689)(504.88147167,34.95118695)(505.06147095,34.88119436)
\curveto(505.17147138,34.83118707)(505.27647127,34.78118712)(505.37647095,34.73119436)
\curveto(505.46647108,34.69118721)(505.53147102,34.6061873)(505.57147095,34.47619436)
\curveto(505.58147097,34.45618745)(505.58647096,34.43118747)(505.58647095,34.40119436)
\curveto(505.57647097,34.38118752)(505.57647097,34.35618755)(505.58647095,34.32619436)
\curveto(505.59647095,34.29618761)(505.60147095,34.26118764)(505.60147095,34.22119436)
\curveto(505.59147096,34.18118772)(505.58647096,34.14118776)(505.58647095,34.10119436)
\lineto(505.58647095,33.80119436)
\curveto(505.58647096,33.7011882)(505.56147099,33.62118828)(505.51147095,33.56119436)
\curveto(505.46147109,33.48118842)(505.39147116,33.42118848)(505.30147095,33.38119436)
\curveto(505.20147135,33.35118855)(505.10147145,33.31118859)(505.00147095,33.26119436)
\curveto(504.80147175,33.18118872)(504.59647195,33.1011888)(504.38647095,33.02119436)
\curveto(504.16647238,32.95118895)(503.95647259,32.87618903)(503.75647095,32.79619436)
\curveto(503.57647297,32.71618919)(503.39647315,32.64618926)(503.21647095,32.58619436)
\curveto(503.02647352,32.53618937)(502.84147371,32.47118943)(502.66147095,32.39119436)
\curveto(502.10147445,32.16118974)(501.53647501,31.94618996)(500.96647095,31.74619436)
\curveto(500.39647615,31.54619036)(499.83147672,31.33119057)(499.27147095,31.10119436)
\lineto(498.64147095,30.86119436)
\curveto(498.42147813,30.79119111)(498.21147834,30.71619119)(498.01147095,30.63619436)
\curveto(497.90147865,30.58619132)(497.79647875,30.54119136)(497.69647095,30.50119436)
\curveto(497.58647896,30.47119143)(497.49147906,30.42119148)(497.41147095,30.35119436)
\curveto(497.39147916,30.34119156)(497.38147917,30.33119157)(497.38147095,30.32119436)
\lineto(497.35147095,30.29119436)
\lineto(497.35147095,30.21619436)
\lineto(497.38147095,30.18619436)
\curveto(497.38147917,30.17619173)(497.38647916,30.16619174)(497.39647095,30.15619436)
\curveto(497.4464791,30.13619177)(497.50147905,30.12619178)(497.56147095,30.12619436)
\curveto(497.62147893,30.12619178)(497.68147887,30.11619179)(497.74147095,30.09619436)
\lineto(497.90647095,30.09619436)
\curveto(497.96647858,30.07619183)(498.03147852,30.07119183)(498.10147095,30.08119436)
\curveto(498.17147838,30.09119181)(498.24147831,30.09619181)(498.31147095,30.09619436)
\lineto(499.12147095,30.09619436)
\lineto(503.68147095,30.09619436)
\lineto(504.86647095,30.09619436)
\curveto(504.97647157,30.09619181)(505.08647146,30.09119181)(505.19647095,30.08119436)
\curveto(505.30647124,30.08119182)(505.39147116,30.05619185)(505.45147095,30.00619436)
\curveto(505.53147102,29.95619195)(505.57647097,29.86619204)(505.58647095,29.73619436)
\lineto(505.58647095,29.34619436)
\lineto(505.58647095,29.15119436)
\curveto(505.58647096,29.1011928)(505.57647097,29.05119285)(505.55647095,29.00119436)
\curveto(505.51647103,28.87119303)(505.43147112,28.79619311)(505.30147095,28.77619436)
\curveto(505.17147138,28.76619314)(505.02147153,28.76119314)(504.85147095,28.76119436)
\lineto(503.11147095,28.76119436)
\lineto(497.11147095,28.76119436)
\lineto(495.70147095,28.76119436)
\curveto(495.59148096,28.76119314)(495.47648107,28.75619315)(495.35647095,28.74619436)
\curveto(495.23648131,28.74619316)(495.14148141,28.77119313)(495.07147095,28.82119436)
\curveto(495.01148154,28.86119304)(494.96148159,28.93619297)(494.92147095,29.04619436)
\curveto(494.91148164,29.06619284)(494.91148164,29.08619282)(494.92147095,29.10619436)
\curveto(494.92148163,29.13619277)(494.91648163,29.16119274)(494.90647095,29.18119436)
}
}
{
\newrgbcolor{curcolor}{0 0 0}
\pscustom[linestyle=none,fillstyle=solid,fillcolor=curcolor]
{
\newpath
\moveto(505.03147095,48.38330373)
\curveto(505.19147136,48.4132959)(505.32647122,48.39829592)(505.43647095,48.33830373)
\curveto(505.53647101,48.27829604)(505.61147094,48.19829612)(505.66147095,48.09830373)
\curveto(505.68147087,48.04829627)(505.69147086,47.99329632)(505.69147095,47.93330373)
\curveto(505.69147086,47.88329643)(505.70147085,47.82829649)(505.72147095,47.76830373)
\curveto(505.77147078,47.54829677)(505.75647079,47.32829699)(505.67647095,47.10830373)
\curveto(505.60647094,46.89829742)(505.51647103,46.75329756)(505.40647095,46.67330373)
\curveto(505.33647121,46.62329769)(505.25647129,46.57829774)(505.16647095,46.53830373)
\curveto(505.06647148,46.49829782)(504.98647156,46.44829787)(504.92647095,46.38830373)
\curveto(504.90647164,46.36829795)(504.88647166,46.34329797)(504.86647095,46.31330373)
\curveto(504.8464717,46.29329802)(504.84147171,46.26329805)(504.85147095,46.22330373)
\curveto(504.88147167,46.1132982)(504.93647161,46.00829831)(505.01647095,45.90830373)
\curveto(505.09647145,45.8182985)(505.16647138,45.72829859)(505.22647095,45.63830373)
\curveto(505.30647124,45.50829881)(505.38147117,45.36829895)(505.45147095,45.21830373)
\curveto(505.51147104,45.06829925)(505.56647098,44.90829941)(505.61647095,44.73830373)
\curveto(505.6464709,44.63829968)(505.66647088,44.52829979)(505.67647095,44.40830373)
\curveto(505.68647086,44.29830002)(505.70147085,44.18830013)(505.72147095,44.07830373)
\curveto(505.73147082,44.02830029)(505.73647081,43.98330033)(505.73647095,43.94330373)
\lineto(505.73647095,43.83830373)
\curveto(505.75647079,43.72830059)(505.75647079,43.62330069)(505.73647095,43.52330373)
\lineto(505.73647095,43.38830373)
\curveto(505.72647082,43.33830098)(505.72147083,43.28830103)(505.72147095,43.23830373)
\curveto(505.72147083,43.18830113)(505.71147084,43.14330117)(505.69147095,43.10330373)
\curveto(505.68147087,43.06330125)(505.67647087,43.02830129)(505.67647095,42.99830373)
\curveto(505.68647086,42.97830134)(505.68647086,42.95330136)(505.67647095,42.92330373)
\lineto(505.61647095,42.68330373)
\curveto(505.60647094,42.60330171)(505.58647096,42.52830179)(505.55647095,42.45830373)
\curveto(505.42647112,42.15830216)(505.28147127,41.9133024)(505.12147095,41.72330373)
\curveto(504.9514716,41.54330277)(504.71647183,41.39330292)(504.41647095,41.27330373)
\curveto(504.19647235,41.18330313)(503.93147262,41.13830318)(503.62147095,41.13830373)
\lineto(503.30647095,41.13830373)
\curveto(503.25647329,41.14830317)(503.20647334,41.15330316)(503.15647095,41.15330373)
\lineto(502.97647095,41.18330373)
\lineto(502.64647095,41.30330373)
\curveto(502.53647401,41.34330297)(502.43647411,41.39330292)(502.34647095,41.45330373)
\curveto(502.05647449,41.63330268)(501.84147471,41.87830244)(501.70147095,42.18830373)
\curveto(501.56147499,42.49830182)(501.43647511,42.83830148)(501.32647095,43.20830373)
\curveto(501.28647526,43.34830097)(501.25647529,43.49330082)(501.23647095,43.64330373)
\curveto(501.21647533,43.79330052)(501.19147536,43.94330037)(501.16147095,44.09330373)
\curveto(501.14147541,44.16330015)(501.13147542,44.22830009)(501.13147095,44.28830373)
\curveto(501.13147542,44.35829996)(501.12147543,44.43329988)(501.10147095,44.51330373)
\curveto(501.08147547,44.58329973)(501.07147548,44.65329966)(501.07147095,44.72330373)
\curveto(501.06147549,44.79329952)(501.0464755,44.86829945)(501.02647095,44.94830373)
\curveto(500.96647558,45.19829912)(500.91647563,45.43329888)(500.87647095,45.65330373)
\curveto(500.82647572,45.87329844)(500.71147584,46.04829827)(500.53147095,46.17830373)
\curveto(500.4514761,46.23829808)(500.3514762,46.28829803)(500.23147095,46.32830373)
\curveto(500.10147645,46.36829795)(499.96147659,46.36829795)(499.81147095,46.32830373)
\curveto(499.57147698,46.26829805)(499.38147717,46.17829814)(499.24147095,46.05830373)
\curveto(499.10147745,45.94829837)(498.99147756,45.78829853)(498.91147095,45.57830373)
\curveto(498.86147769,45.45829886)(498.82647772,45.313299)(498.80647095,45.14330373)
\curveto(498.78647776,44.98329933)(498.77647777,44.8132995)(498.77647095,44.63330373)
\curveto(498.77647777,44.45329986)(498.78647776,44.27830004)(498.80647095,44.10830373)
\curveto(498.82647772,43.93830038)(498.85647769,43.79330052)(498.89647095,43.67330373)
\curveto(498.95647759,43.50330081)(499.04147751,43.33830098)(499.15147095,43.17830373)
\curveto(499.21147734,43.09830122)(499.29147726,43.02330129)(499.39147095,42.95330373)
\curveto(499.48147707,42.89330142)(499.58147697,42.83830148)(499.69147095,42.78830373)
\curveto(499.77147678,42.75830156)(499.85647669,42.72830159)(499.94647095,42.69830373)
\curveto(500.03647651,42.67830164)(500.10647644,42.63330168)(500.15647095,42.56330373)
\curveto(500.18647636,42.52330179)(500.21147634,42.45330186)(500.23147095,42.35330373)
\curveto(500.24147631,42.26330205)(500.2464763,42.16830215)(500.24647095,42.06830373)
\curveto(500.2464763,41.96830235)(500.24147631,41.86830245)(500.23147095,41.76830373)
\curveto(500.21147634,41.67830264)(500.18647636,41.6133027)(500.15647095,41.57330373)
\curveto(500.12647642,41.53330278)(500.07647647,41.50330281)(500.00647095,41.48330373)
\curveto(499.93647661,41.46330285)(499.86147669,41.46330285)(499.78147095,41.48330373)
\curveto(499.6514769,41.5133028)(499.53147702,41.54330277)(499.42147095,41.57330373)
\curveto(499.30147725,41.6133027)(499.18647736,41.65830266)(499.07647095,41.70830373)
\curveto(498.72647782,41.89830242)(498.45647809,42.13830218)(498.26647095,42.42830373)
\curveto(498.06647848,42.7183016)(497.90647864,43.07830124)(497.78647095,43.50830373)
\curveto(497.76647878,43.60830071)(497.7514788,43.70830061)(497.74147095,43.80830373)
\curveto(497.73147882,43.9183004)(497.71647883,44.02830029)(497.69647095,44.13830373)
\curveto(497.68647886,44.17830014)(497.68647886,44.24330007)(497.69647095,44.33330373)
\curveto(497.69647885,44.42329989)(497.68647886,44.47829984)(497.66647095,44.49830373)
\curveto(497.65647889,45.19829912)(497.73647881,45.80829851)(497.90647095,46.32830373)
\curveto(498.07647847,46.84829747)(498.40147815,47.2132971)(498.88147095,47.42330373)
\curveto(499.08147747,47.5132968)(499.31647723,47.56329675)(499.58647095,47.57330373)
\curveto(499.8464767,47.59329672)(500.12147643,47.60329671)(500.41147095,47.60330373)
\lineto(503.72647095,47.60330373)
\curveto(503.86647268,47.60329671)(504.00147255,47.60829671)(504.13147095,47.61830373)
\curveto(504.26147229,47.62829669)(504.36647218,47.65829666)(504.44647095,47.70830373)
\curveto(504.51647203,47.75829656)(504.56647198,47.82329649)(504.59647095,47.90330373)
\curveto(504.63647191,47.99329632)(504.66647188,48.07829624)(504.68647095,48.15830373)
\curveto(504.69647185,48.23829608)(504.74147181,48.29829602)(504.82147095,48.33830373)
\curveto(504.8514717,48.35829596)(504.88147167,48.36829595)(504.91147095,48.36830373)
\curveto(504.94147161,48.36829595)(504.98147157,48.37329594)(505.03147095,48.38330373)
\moveto(503.36647095,46.23830373)
\curveto(503.22647332,46.29829802)(503.06647348,46.32829799)(502.88647095,46.32830373)
\curveto(502.69647385,46.33829798)(502.50147405,46.34329797)(502.30147095,46.34330373)
\curveto(502.19147436,46.34329797)(502.09147446,46.33829798)(502.00147095,46.32830373)
\curveto(501.91147464,46.318298)(501.84147471,46.27829804)(501.79147095,46.20830373)
\curveto(501.77147478,46.17829814)(501.76147479,46.10829821)(501.76147095,45.99830373)
\curveto(501.78147477,45.97829834)(501.79147476,45.94329837)(501.79147095,45.89330373)
\curveto(501.79147476,45.84329847)(501.80147475,45.79829852)(501.82147095,45.75830373)
\curveto(501.84147471,45.67829864)(501.86147469,45.58829873)(501.88147095,45.48830373)
\lineto(501.94147095,45.18830373)
\curveto(501.94147461,45.15829916)(501.9464746,45.12329919)(501.95647095,45.08330373)
\lineto(501.95647095,44.97830373)
\curveto(501.99647455,44.82829949)(502.02147453,44.66329965)(502.03147095,44.48330373)
\curveto(502.03147452,44.3133)(502.0514745,44.15330016)(502.09147095,44.00330373)
\curveto(502.11147444,43.92330039)(502.13147442,43.84830047)(502.15147095,43.77830373)
\curveto(502.16147439,43.7183006)(502.17647437,43.64830067)(502.19647095,43.56830373)
\curveto(502.2464743,43.40830091)(502.31147424,43.25830106)(502.39147095,43.11830373)
\curveto(502.46147409,42.97830134)(502.551474,42.85830146)(502.66147095,42.75830373)
\curveto(502.77147378,42.65830166)(502.90647364,42.58330173)(503.06647095,42.53330373)
\curveto(503.21647333,42.48330183)(503.40147315,42.46330185)(503.62147095,42.47330373)
\curveto(503.72147283,42.47330184)(503.81647273,42.48830183)(503.90647095,42.51830373)
\curveto(503.98647256,42.55830176)(504.06147249,42.60330171)(504.13147095,42.65330373)
\curveto(504.24147231,42.73330158)(504.33647221,42.83830148)(504.41647095,42.96830373)
\curveto(504.48647206,43.09830122)(504.546472,43.23830108)(504.59647095,43.38830373)
\curveto(504.60647194,43.43830088)(504.61147194,43.48830083)(504.61147095,43.53830373)
\curveto(504.61147194,43.58830073)(504.61647193,43.63830068)(504.62647095,43.68830373)
\curveto(504.6464719,43.75830056)(504.66147189,43.84330047)(504.67147095,43.94330373)
\curveto(504.67147188,44.05330026)(504.66147189,44.14330017)(504.64147095,44.21330373)
\curveto(504.62147193,44.27330004)(504.61647193,44.33329998)(504.62647095,44.39330373)
\curveto(504.62647192,44.45329986)(504.61647193,44.5132998)(504.59647095,44.57330373)
\curveto(504.57647197,44.65329966)(504.56147199,44.72829959)(504.55147095,44.79830373)
\curveto(504.54147201,44.87829944)(504.52147203,44.95329936)(504.49147095,45.02330373)
\curveto(504.37147218,45.313299)(504.22647232,45.55829876)(504.05647095,45.75830373)
\curveto(503.88647266,45.96829835)(503.65647289,46.12829819)(503.36647095,46.23830373)
}
}
{
\newrgbcolor{curcolor}{0 0 0}
\pscustom[linestyle=none,fillstyle=solid,fillcolor=curcolor]
{
\newpath
\moveto(497.86147095,49.26994436)
\lineto(497.86147095,49.71994436)
\curveto(497.8514787,49.88994311)(497.87147868,50.01494298)(497.92147095,50.09494436)
\curveto(497.97147858,50.17494282)(498.03647851,50.22994277)(498.11647095,50.25994436)
\curveto(498.19647835,50.2999427)(498.28147827,50.33994266)(498.37147095,50.37994436)
\curveto(498.50147805,50.42994257)(498.63147792,50.47494252)(498.76147095,50.51494436)
\curveto(498.89147766,50.55494244)(499.02147753,50.5999424)(499.15147095,50.64994436)
\curveto(499.27147728,50.6999423)(499.39647715,50.74494225)(499.52647095,50.78494436)
\curveto(499.6464769,50.82494217)(499.76647678,50.86994213)(499.88647095,50.91994436)
\curveto(499.99647655,50.96994203)(500.11147644,51.00994199)(500.23147095,51.03994436)
\curveto(500.3514762,51.06994193)(500.47147608,51.10994189)(500.59147095,51.15994436)
\curveto(500.88147567,51.27994172)(501.18147537,51.38994161)(501.49147095,51.48994436)
\curveto(501.80147475,51.58994141)(502.10147445,51.6999413)(502.39147095,51.81994436)
\curveto(502.43147412,51.83994116)(502.47147408,51.84994115)(502.51147095,51.84994436)
\curveto(502.54147401,51.84994115)(502.57147398,51.85994114)(502.60147095,51.87994436)
\curveto(502.74147381,51.93994106)(502.88647366,51.994941)(503.03647095,52.04494436)
\lineto(503.45647095,52.19494436)
\curveto(503.52647302,52.22494077)(503.60147295,52.25494074)(503.68147095,52.28494436)
\curveto(503.7514728,52.31494068)(503.79647275,52.36494063)(503.81647095,52.43494436)
\curveto(503.8464727,52.51494048)(503.82147273,52.57494042)(503.74147095,52.61494436)
\curveto(503.6514729,52.66494033)(503.58147297,52.6999403)(503.53147095,52.71994436)
\curveto(503.36147319,52.7999402)(503.18147337,52.86494013)(502.99147095,52.91494436)
\curveto(502.80147375,52.96494003)(502.61647393,53.02493997)(502.43647095,53.09494436)
\curveto(502.20647434,53.18493981)(501.97647457,53.26493973)(501.74647095,53.33494436)
\curveto(501.50647504,53.40493959)(501.27647527,53.48993951)(501.05647095,53.58994436)
\curveto(501.00647554,53.5999394)(500.94147561,53.61493938)(500.86147095,53.63494436)
\curveto(500.77147578,53.67493932)(500.68147587,53.70993929)(500.59147095,53.73994436)
\curveto(500.49147606,53.76993923)(500.40147615,53.7999392)(500.32147095,53.82994436)
\curveto(500.27147628,53.84993915)(500.22647632,53.86493913)(500.18647095,53.87494436)
\curveto(500.1464764,53.88493911)(500.10147645,53.8999391)(500.05147095,53.91994436)
\curveto(499.93147662,53.96993903)(499.81147674,54.00993899)(499.69147095,54.03994436)
\curveto(499.56147699,54.07993892)(499.43647711,54.12493887)(499.31647095,54.17494436)
\curveto(499.26647728,54.1949388)(499.22147733,54.20993879)(499.18147095,54.21994436)
\curveto(499.14147741,54.22993877)(499.09647745,54.24493875)(499.04647095,54.26494436)
\curveto(498.95647759,54.30493869)(498.86647768,54.33993866)(498.77647095,54.36994436)
\curveto(498.67647787,54.3999386)(498.58147797,54.42993857)(498.49147095,54.45994436)
\curveto(498.41147814,54.48993851)(498.33147822,54.51493848)(498.25147095,54.53494436)
\curveto(498.16147839,54.56493843)(498.08647846,54.60493839)(498.02647095,54.65494436)
\curveto(497.93647861,54.72493827)(497.88647866,54.81993818)(497.87647095,54.93994436)
\curveto(497.86647868,55.06993793)(497.86147869,55.20993779)(497.86147095,55.35994436)
\curveto(497.86147869,55.43993756)(497.86647868,55.51493748)(497.87647095,55.58494436)
\curveto(497.87647867,55.66493733)(497.89147866,55.72993727)(497.92147095,55.77994436)
\curveto(497.98147857,55.86993713)(498.07647847,55.8949371)(498.20647095,55.85494436)
\curveto(498.33647821,55.81493718)(498.43647811,55.77993722)(498.50647095,55.74994436)
\lineto(498.56647095,55.71994436)
\curveto(498.58647796,55.71993728)(498.60647794,55.71493728)(498.62647095,55.70494436)
\curveto(498.90647764,55.5949374)(499.19147736,55.48493751)(499.48147095,55.37494436)
\lineto(500.32147095,55.04494436)
\curveto(500.40147615,55.01493798)(500.47647607,54.98993801)(500.54647095,54.96994436)
\curveto(500.60647594,54.94993805)(500.67147588,54.92493807)(500.74147095,54.89494436)
\curveto(500.94147561,54.81493818)(501.1464754,54.73493826)(501.35647095,54.65494436)
\curveto(501.55647499,54.58493841)(501.75647479,54.50993849)(501.95647095,54.42994436)
\curveto(502.6464739,54.13993886)(503.34147321,53.86993913)(504.04147095,53.61994436)
\curveto(504.74147181,53.36993963)(505.43647111,53.0999399)(506.12647095,52.80994436)
\lineto(506.27647095,52.74994436)
\curveto(506.33647021,52.73994026)(506.39647015,52.72494027)(506.45647095,52.70494436)
\curveto(506.82646972,52.54494045)(507.19146936,52.37494062)(507.55147095,52.19494436)
\curveto(507.92146863,52.01494098)(508.20646834,51.76494123)(508.40647095,51.44494436)
\curveto(508.46646808,51.33494166)(508.51146804,51.22494177)(508.54147095,51.11494436)
\curveto(508.58146797,51.00494199)(508.61646793,50.87994212)(508.64647095,50.73994436)
\curveto(508.66646788,50.68994231)(508.67146788,50.63494236)(508.66147095,50.57494436)
\curveto(508.6514679,50.52494247)(508.6514679,50.46994253)(508.66147095,50.40994436)
\curveto(508.68146787,50.32994267)(508.68146787,50.24994275)(508.66147095,50.16994436)
\curveto(508.6514679,50.12994287)(508.6464679,50.07994292)(508.64647095,50.01994436)
\lineto(508.58647095,49.77994436)
\curveto(508.56646798,49.70994329)(508.52646802,49.65494334)(508.46647095,49.61494436)
\curveto(508.40646814,49.56494343)(508.33146822,49.53494346)(508.24147095,49.52494436)
\lineto(507.97147095,49.52494436)
\lineto(507.76147095,49.52494436)
\curveto(507.70146885,49.53494346)(507.6514689,49.55494344)(507.61147095,49.58494436)
\curveto(507.50146905,49.65494334)(507.47146908,49.77494322)(507.52147095,49.94494436)
\curveto(507.54146901,50.05494294)(507.551469,50.17494282)(507.55147095,50.30494436)
\curveto(507.551469,50.43494256)(507.53146902,50.54994245)(507.49147095,50.64994436)
\curveto(507.44146911,50.7999422)(507.36646918,50.91994208)(507.26647095,51.00994436)
\curveto(507.16646938,51.10994189)(507.0514695,51.1949418)(506.92147095,51.26494436)
\curveto(506.80146975,51.33494166)(506.67146988,51.3949416)(506.53147095,51.44494436)
\lineto(506.11147095,51.62494436)
\curveto(506.02147053,51.66494133)(505.91147064,51.70494129)(505.78147095,51.74494436)
\curveto(505.6514709,51.7949412)(505.51647103,51.7999412)(505.37647095,51.75994436)
\curveto(505.21647133,51.70994129)(505.06647148,51.65494134)(504.92647095,51.59494436)
\curveto(504.78647176,51.54494145)(504.6464719,51.48994151)(504.50647095,51.42994436)
\curveto(504.29647225,51.33994166)(504.08647246,51.25494174)(503.87647095,51.17494436)
\curveto(503.66647288,51.0949419)(503.46147309,51.01494198)(503.26147095,50.93494436)
\curveto(503.12147343,50.87494212)(502.98647356,50.81994218)(502.85647095,50.76994436)
\curveto(502.72647382,50.71994228)(502.59147396,50.66994233)(502.45147095,50.61994436)
\lineto(501.13147095,50.07994436)
\curveto(500.69147586,49.90994309)(500.2514763,49.73494326)(499.81147095,49.55494436)
\curveto(499.58147697,49.45494354)(499.36147719,49.36494363)(499.15147095,49.28494436)
\curveto(498.93147762,49.20494379)(498.71147784,49.11994388)(498.49147095,49.02994436)
\curveto(498.43147812,49.00994399)(498.3514782,48.97994402)(498.25147095,48.93994436)
\curveto(498.14147841,48.8999441)(498.0514785,48.90494409)(497.98147095,48.95494436)
\curveto(497.93147862,48.98494401)(497.89647865,49.04494395)(497.87647095,49.13494436)
\curveto(497.86647868,49.15494384)(497.86647868,49.17494382)(497.87647095,49.19494436)
\curveto(497.87647867,49.22494377)(497.87147868,49.24994375)(497.86147095,49.26994436)
}
}
{
\newrgbcolor{curcolor}{0 0 0}
\pscustom[linestyle=none,fillstyle=solid,fillcolor=curcolor]
{
}
}
{
\newrgbcolor{curcolor}{0 0 0}
\pscustom[linestyle=none,fillstyle=solid,fillcolor=curcolor]
{
\newpath
\moveto(500.50147095,67.99510061)
\lineto(500.75647095,67.99510061)
\curveto(500.83647571,68.0050929)(500.91147564,68.00009291)(500.98147095,67.98010061)
\lineto(501.22147095,67.98010061)
\lineto(501.38647095,67.98010061)
\curveto(501.48647506,67.96009295)(501.59147496,67.95009296)(501.70147095,67.95010061)
\curveto(501.80147475,67.95009296)(501.90147465,67.94009297)(502.00147095,67.92010061)
\lineto(502.15147095,67.92010061)
\curveto(502.29147426,67.89009302)(502.43147412,67.87009304)(502.57147095,67.86010061)
\curveto(502.70147385,67.85009306)(502.83147372,67.82509308)(502.96147095,67.78510061)
\curveto(503.04147351,67.76509314)(503.12647342,67.74509316)(503.21647095,67.72510061)
\lineto(503.45647095,67.66510061)
\lineto(503.75647095,67.54510061)
\curveto(503.8464727,67.51509339)(503.93647261,67.48009343)(504.02647095,67.44010061)
\curveto(504.2464723,67.34009357)(504.46147209,67.2050937)(504.67147095,67.03510061)
\curveto(504.88147167,66.87509403)(505.0514715,66.70009421)(505.18147095,66.51010061)
\curveto(505.22147133,66.46009445)(505.26147129,66.40009451)(505.30147095,66.33010061)
\curveto(505.33147122,66.27009464)(505.36647118,66.2100947)(505.40647095,66.15010061)
\curveto(505.45647109,66.07009484)(505.49647105,65.97509493)(505.52647095,65.86510061)
\curveto(505.55647099,65.75509515)(505.58647096,65.65009526)(505.61647095,65.55010061)
\curveto(505.65647089,65.44009547)(505.68147087,65.33009558)(505.69147095,65.22010061)
\curveto(505.70147085,65.1100958)(505.71647083,64.99509591)(505.73647095,64.87510061)
\curveto(505.7464708,64.83509607)(505.7464708,64.79009612)(505.73647095,64.74010061)
\curveto(505.73647081,64.70009621)(505.74147081,64.66009625)(505.75147095,64.62010061)
\curveto(505.76147079,64.58009633)(505.76647078,64.52509638)(505.76647095,64.45510061)
\curveto(505.76647078,64.38509652)(505.76147079,64.33509657)(505.75147095,64.30510061)
\curveto(505.73147082,64.25509665)(505.72647082,64.2100967)(505.73647095,64.17010061)
\curveto(505.7464708,64.13009678)(505.7464708,64.09509681)(505.73647095,64.06510061)
\lineto(505.73647095,63.97510061)
\curveto(505.71647083,63.91509699)(505.70147085,63.85009706)(505.69147095,63.78010061)
\curveto(505.69147086,63.72009719)(505.68647086,63.65509725)(505.67647095,63.58510061)
\curveto(505.62647092,63.41509749)(505.57647097,63.25509765)(505.52647095,63.10510061)
\curveto(505.47647107,62.95509795)(505.41147114,62.8100981)(505.33147095,62.67010061)
\curveto(505.29147126,62.62009829)(505.26147129,62.56509834)(505.24147095,62.50510061)
\curveto(505.21147134,62.45509845)(505.17647137,62.4050985)(505.13647095,62.35510061)
\curveto(504.95647159,62.11509879)(504.73647181,61.91509899)(504.47647095,61.75510061)
\curveto(504.21647233,61.59509931)(503.93147262,61.45509945)(503.62147095,61.33510061)
\curveto(503.48147307,61.27509963)(503.34147321,61.23009968)(503.20147095,61.20010061)
\curveto(503.0514735,61.17009974)(502.89647365,61.13509977)(502.73647095,61.09510061)
\curveto(502.62647392,61.07509983)(502.51647403,61.06009985)(502.40647095,61.05010061)
\curveto(502.29647425,61.04009987)(502.18647436,61.02509988)(502.07647095,61.00510061)
\curveto(502.03647451,60.99509991)(501.99647455,60.99009992)(501.95647095,60.99010061)
\curveto(501.91647463,61.00009991)(501.87647467,61.00009991)(501.83647095,60.99010061)
\curveto(501.78647476,60.98009993)(501.73647481,60.97509993)(501.68647095,60.97510061)
\lineto(501.52147095,60.97510061)
\curveto(501.47147508,60.95509995)(501.42147513,60.95009996)(501.37147095,60.96010061)
\curveto(501.31147524,60.97009994)(501.25647529,60.97009994)(501.20647095,60.96010061)
\curveto(501.16647538,60.95009996)(501.12147543,60.95009996)(501.07147095,60.96010061)
\curveto(501.02147553,60.97009994)(500.97147558,60.96509994)(500.92147095,60.94510061)
\curveto(500.8514757,60.92509998)(500.77647577,60.92009999)(500.69647095,60.93010061)
\curveto(500.60647594,60.94009997)(500.52147603,60.94509996)(500.44147095,60.94510061)
\curveto(500.3514762,60.94509996)(500.2514763,60.94009997)(500.14147095,60.93010061)
\curveto(500.02147653,60.92009999)(499.92147663,60.92509998)(499.84147095,60.94510061)
\lineto(499.55647095,60.94510061)
\lineto(498.92647095,60.99010061)
\curveto(498.82647772,61.00009991)(498.73147782,61.0100999)(498.64147095,61.02010061)
\lineto(498.34147095,61.05010061)
\curveto(498.29147826,61.07009984)(498.24147831,61.07509983)(498.19147095,61.06510061)
\curveto(498.13147842,61.06509984)(498.07647847,61.07509983)(498.02647095,61.09510061)
\curveto(497.85647869,61.14509976)(497.69147886,61.18509972)(497.53147095,61.21510061)
\curveto(497.36147919,61.24509966)(497.20147935,61.29509961)(497.05147095,61.36510061)
\curveto(496.59147996,61.55509935)(496.21648033,61.77509913)(495.92647095,62.02510061)
\curveto(495.63648091,62.28509862)(495.39148116,62.64509826)(495.19147095,63.10510061)
\curveto(495.14148141,63.23509767)(495.10648144,63.36509754)(495.08647095,63.49510061)
\curveto(495.06648148,63.63509727)(495.04148151,63.77509713)(495.01147095,63.91510061)
\curveto(495.00148155,63.98509692)(494.99648155,64.05009686)(494.99647095,64.11010061)
\curveto(494.99648155,64.17009674)(494.99148156,64.23509667)(494.98147095,64.30510061)
\curveto(494.96148159,65.13509577)(495.11148144,65.8050951)(495.43147095,66.31510061)
\curveto(495.74148081,66.82509408)(496.18148037,67.2050937)(496.75147095,67.45510061)
\curveto(496.87147968,67.5050934)(496.99647955,67.55009336)(497.12647095,67.59010061)
\curveto(497.25647929,67.63009328)(497.39147916,67.67509323)(497.53147095,67.72510061)
\curveto(497.61147894,67.74509316)(497.69647885,67.76009315)(497.78647095,67.77010061)
\lineto(498.02647095,67.83010061)
\curveto(498.13647841,67.86009305)(498.2464783,67.87509303)(498.35647095,67.87510061)
\curveto(498.46647808,67.88509302)(498.57647797,67.90009301)(498.68647095,67.92010061)
\curveto(498.73647781,67.94009297)(498.78147777,67.94509296)(498.82147095,67.93510061)
\curveto(498.86147769,67.93509297)(498.90147765,67.94009297)(498.94147095,67.95010061)
\curveto(498.99147756,67.96009295)(499.0464775,67.96009295)(499.10647095,67.95010061)
\curveto(499.15647739,67.95009296)(499.20647734,67.95509295)(499.25647095,67.96510061)
\lineto(499.39147095,67.96510061)
\curveto(499.4514771,67.98509292)(499.52147703,67.98509292)(499.60147095,67.96510061)
\curveto(499.67147688,67.95509295)(499.73647681,67.96009295)(499.79647095,67.98010061)
\curveto(499.82647672,67.99009292)(499.86647668,67.99509291)(499.91647095,67.99510061)
\lineto(500.03647095,67.99510061)
\lineto(500.50147095,67.99510061)
\moveto(502.82647095,66.45010061)
\curveto(502.50647404,66.55009436)(502.14147441,66.6100943)(501.73147095,66.63010061)
\curveto(501.32147523,66.65009426)(500.91147564,66.66009425)(500.50147095,66.66010061)
\curveto(500.07147648,66.66009425)(499.6514769,66.65009426)(499.24147095,66.63010061)
\curveto(498.83147772,66.6100943)(498.4464781,66.56509434)(498.08647095,66.49510061)
\curveto(497.72647882,66.42509448)(497.40647914,66.31509459)(497.12647095,66.16510061)
\curveto(496.83647971,66.02509488)(496.60147995,65.83009508)(496.42147095,65.58010061)
\curveto(496.31148024,65.42009549)(496.23148032,65.24009567)(496.18147095,65.04010061)
\curveto(496.12148043,64.84009607)(496.09148046,64.59509631)(496.09147095,64.30510061)
\curveto(496.11148044,64.28509662)(496.12148043,64.25009666)(496.12147095,64.20010061)
\curveto(496.11148044,64.15009676)(496.11148044,64.1100968)(496.12147095,64.08010061)
\curveto(496.14148041,64.00009691)(496.16148039,63.92509698)(496.18147095,63.85510061)
\curveto(496.19148036,63.79509711)(496.21148034,63.73009718)(496.24147095,63.66010061)
\curveto(496.36148019,63.39009752)(496.53148002,63.17009774)(496.75147095,63.00010061)
\curveto(496.96147959,62.84009807)(497.20647934,62.7050982)(497.48647095,62.59510061)
\curveto(497.59647895,62.54509836)(497.71647883,62.5050984)(497.84647095,62.47510061)
\curveto(497.96647858,62.45509845)(498.09147846,62.43009848)(498.22147095,62.40010061)
\curveto(498.27147828,62.38009853)(498.32647822,62.37009854)(498.38647095,62.37010061)
\curveto(498.43647811,62.37009854)(498.48647806,62.36509854)(498.53647095,62.35510061)
\curveto(498.62647792,62.34509856)(498.72147783,62.33509857)(498.82147095,62.32510061)
\curveto(498.91147764,62.31509859)(499.00647754,62.3050986)(499.10647095,62.29510061)
\curveto(499.18647736,62.29509861)(499.27147728,62.29009862)(499.36147095,62.28010061)
\lineto(499.60147095,62.28010061)
\lineto(499.78147095,62.28010061)
\curveto(499.81147674,62.27009864)(499.8464767,62.26509864)(499.88647095,62.26510061)
\lineto(500.02147095,62.26510061)
\lineto(500.47147095,62.26510061)
\curveto(500.551476,62.26509864)(500.63647591,62.26009865)(500.72647095,62.25010061)
\curveto(500.80647574,62.25009866)(500.88147567,62.26009865)(500.95147095,62.28010061)
\lineto(501.22147095,62.28010061)
\curveto(501.24147531,62.28009863)(501.27147528,62.27509863)(501.31147095,62.26510061)
\curveto(501.34147521,62.26509864)(501.36647518,62.27009864)(501.38647095,62.28010061)
\curveto(501.48647506,62.29009862)(501.58647496,62.29509861)(501.68647095,62.29510061)
\curveto(501.77647477,62.3050986)(501.87647467,62.31509859)(501.98647095,62.32510061)
\curveto(502.10647444,62.35509855)(502.23147432,62.37009854)(502.36147095,62.37010061)
\curveto(502.48147407,62.38009853)(502.59647395,62.4050985)(502.70647095,62.44510061)
\curveto(503.00647354,62.52509838)(503.27147328,62.6100983)(503.50147095,62.70010061)
\curveto(503.73147282,62.80009811)(503.9464726,62.94509796)(504.14647095,63.13510061)
\curveto(504.3464722,63.34509756)(504.49647205,63.6100973)(504.59647095,63.93010061)
\curveto(504.61647193,63.97009694)(504.62647192,64.0050969)(504.62647095,64.03510061)
\curveto(504.61647193,64.07509683)(504.62147193,64.12009679)(504.64147095,64.17010061)
\curveto(504.6514719,64.2100967)(504.66147189,64.28009663)(504.67147095,64.38010061)
\curveto(504.68147187,64.49009642)(504.67647187,64.57509633)(504.65647095,64.63510061)
\curveto(504.63647191,64.7050962)(504.62647192,64.77509613)(504.62647095,64.84510061)
\curveto(504.61647193,64.91509599)(504.60147195,64.98009593)(504.58147095,65.04010061)
\curveto(504.52147203,65.24009567)(504.43647211,65.42009549)(504.32647095,65.58010061)
\curveto(504.30647224,65.6100953)(504.28647226,65.63509527)(504.26647095,65.65510061)
\lineto(504.20647095,65.71510061)
\curveto(504.18647236,65.75509515)(504.1464724,65.8050951)(504.08647095,65.86510061)
\curveto(503.9464726,65.96509494)(503.81647273,66.05009486)(503.69647095,66.12010061)
\curveto(503.57647297,66.19009472)(503.43147312,66.26009465)(503.26147095,66.33010061)
\curveto(503.19147336,66.36009455)(503.12147343,66.38009453)(503.05147095,66.39010061)
\curveto(502.98147357,66.4100945)(502.90647364,66.43009448)(502.82647095,66.45010061)
}
}
{
\newrgbcolor{curcolor}{0 0 0}
\pscustom[linestyle=none,fillstyle=solid,fillcolor=curcolor]
{
\newpath
\moveto(495.17647095,70.85470998)
\lineto(495.17647095,74.45470998)
\lineto(495.17647095,75.09970998)
\curveto(495.17648137,75.17970345)(495.18148137,75.25470338)(495.19147095,75.32470998)
\curveto(495.19148136,75.39470324)(495.20148135,75.45470318)(495.22147095,75.50470998)
\curveto(495.2514813,75.57470306)(495.31148124,75.629703)(495.40147095,75.66970998)
\curveto(495.43148112,75.68970294)(495.47148108,75.69970293)(495.52147095,75.69970998)
\lineto(495.65647095,75.69970998)
\curveto(495.76648078,75.70970292)(495.87148068,75.70470293)(495.97147095,75.68470998)
\curveto(496.07148048,75.67470296)(496.14148041,75.63970299)(496.18147095,75.57970998)
\curveto(496.2514803,75.48970314)(496.28648026,75.35470328)(496.28647095,75.17470998)
\curveto(496.27648027,74.99470364)(496.27148028,74.8297038)(496.27147095,74.67970998)
\lineto(496.27147095,72.68470998)
\lineto(496.27147095,72.18970998)
\lineto(496.27147095,72.05470998)
\curveto(496.27148028,72.01470662)(496.27648027,71.97470666)(496.28647095,71.93470998)
\lineto(496.28647095,71.72470998)
\curveto(496.31648023,71.61470702)(496.35648019,71.5347071)(496.40647095,71.48470998)
\curveto(496.4464801,71.4347072)(496.50148005,71.39970723)(496.57147095,71.37970998)
\curveto(496.63147992,71.35970727)(496.70147985,71.34470729)(496.78147095,71.33470998)
\curveto(496.86147969,71.32470731)(496.9514796,71.30470733)(497.05147095,71.27470998)
\curveto(497.2514793,71.22470741)(497.45647909,71.18470745)(497.66647095,71.15470998)
\curveto(497.87647867,71.12470751)(498.08147847,71.08470755)(498.28147095,71.03470998)
\curveto(498.3514782,71.01470762)(498.42147813,71.00470763)(498.49147095,71.00470998)
\curveto(498.551478,71.00470763)(498.61647793,70.99470764)(498.68647095,70.97470998)
\curveto(498.71647783,70.96470767)(498.75647779,70.95470768)(498.80647095,70.94470998)
\curveto(498.8464777,70.94470769)(498.88647766,70.94970768)(498.92647095,70.95970998)
\curveto(498.97647757,70.97970765)(499.02147753,71.00470763)(499.06147095,71.03470998)
\curveto(499.09147746,71.07470756)(499.09647745,71.1347075)(499.07647095,71.21470998)
\curveto(499.05647749,71.27470736)(499.03147752,71.3347073)(499.00147095,71.39470998)
\curveto(498.96147759,71.45470718)(498.92647762,71.51470712)(498.89647095,71.57470998)
\curveto(498.87647767,71.634707)(498.86147769,71.68470695)(498.85147095,71.72470998)
\curveto(498.77147778,71.91470672)(498.71647783,72.11970651)(498.68647095,72.33970998)
\curveto(498.65647789,72.56970606)(498.6464779,72.79970583)(498.65647095,73.02970998)
\curveto(498.65647789,73.26970536)(498.68147787,73.49970513)(498.73147095,73.71970998)
\curveto(498.77147778,73.93970469)(498.83147772,74.13970449)(498.91147095,74.31970998)
\curveto(498.93147762,74.36970426)(498.9514776,74.41470422)(498.97147095,74.45470998)
\curveto(498.99147756,74.50470413)(499.01647753,74.55470408)(499.04647095,74.60470998)
\curveto(499.25647729,74.95470368)(499.48647706,75.2347034)(499.73647095,75.44470998)
\curveto(499.98647656,75.66470297)(500.31147624,75.85970277)(500.71147095,76.02970998)
\curveto(500.82147573,76.07970255)(500.93147562,76.11470252)(501.04147095,76.13470998)
\curveto(501.1514754,76.15470248)(501.26647528,76.17970245)(501.38647095,76.20970998)
\curveto(501.41647513,76.21970241)(501.46147509,76.22470241)(501.52147095,76.22470998)
\curveto(501.58147497,76.24470239)(501.6514749,76.25470238)(501.73147095,76.25470998)
\curveto(501.80147475,76.25470238)(501.86647468,76.26470237)(501.92647095,76.28470998)
\lineto(502.09147095,76.28470998)
\curveto(502.14147441,76.29470234)(502.21147434,76.29970233)(502.30147095,76.29970998)
\curveto(502.39147416,76.29970233)(502.46147409,76.28970234)(502.51147095,76.26970998)
\curveto(502.57147398,76.24970238)(502.63147392,76.24470239)(502.69147095,76.25470998)
\curveto(502.74147381,76.26470237)(502.79147376,76.25970237)(502.84147095,76.23970998)
\curveto(503.00147355,76.19970243)(503.1514734,76.16470247)(503.29147095,76.13470998)
\curveto(503.43147312,76.10470253)(503.56647298,76.05970257)(503.69647095,75.99970998)
\curveto(504.06647248,75.83970279)(504.40147215,75.61970301)(504.70147095,75.33970998)
\curveto(505.00147155,75.05970357)(505.23147132,74.73970389)(505.39147095,74.37970998)
\curveto(505.47147108,74.20970442)(505.546471,74.00970462)(505.61647095,73.77970998)
\curveto(505.65647089,73.66970496)(505.68147087,73.55470508)(505.69147095,73.43470998)
\curveto(505.70147085,73.31470532)(505.72147083,73.19470544)(505.75147095,73.07470998)
\curveto(505.77147078,73.02470561)(505.77147078,72.96970566)(505.75147095,72.90970998)
\curveto(505.74147081,72.84970578)(505.7464708,72.78970584)(505.76647095,72.72970998)
\curveto(505.78647076,72.629706)(505.78647076,72.5297061)(505.76647095,72.42970998)
\lineto(505.76647095,72.29470998)
\curveto(505.7464708,72.24470639)(505.73647081,72.18470645)(505.73647095,72.11470998)
\curveto(505.7464708,72.05470658)(505.74147081,71.99970663)(505.72147095,71.94970998)
\curveto(505.71147084,71.90970672)(505.70647084,71.87470676)(505.70647095,71.84470998)
\curveto(505.70647084,71.81470682)(505.70147085,71.77970685)(505.69147095,71.73970998)
\lineto(505.63147095,71.46970998)
\curveto(505.61147094,71.37970725)(505.58147097,71.29470734)(505.54147095,71.21470998)
\curveto(505.40147115,70.87470776)(505.2464713,70.58470805)(505.07647095,70.34470998)
\curveto(504.89647165,70.10470853)(504.66647188,69.88470875)(504.38647095,69.68470998)
\curveto(504.15647239,69.5347091)(503.91647263,69.41970921)(503.66647095,69.33970998)
\curveto(503.61647293,69.31970931)(503.57147298,69.30970932)(503.53147095,69.30970998)
\curveto(503.48147307,69.30970932)(503.43147312,69.29970933)(503.38147095,69.27970998)
\curveto(503.32147323,69.25970937)(503.24147331,69.24470939)(503.14147095,69.23470998)
\curveto(503.04147351,69.2347094)(502.96647358,69.25470938)(502.91647095,69.29470998)
\curveto(502.83647371,69.34470929)(502.79147376,69.42470921)(502.78147095,69.53470998)
\curveto(502.77147378,69.64470899)(502.76647378,69.75970887)(502.76647095,69.87970998)
\lineto(502.76647095,70.04470998)
\curveto(502.76647378,70.10470853)(502.77647377,70.15970847)(502.79647095,70.20970998)
\curveto(502.81647373,70.29970833)(502.85647369,70.36970826)(502.91647095,70.41970998)
\curveto(503.00647354,70.48970814)(503.11647343,70.5347081)(503.24647095,70.55470998)
\curveto(503.36647318,70.58470805)(503.47147308,70.629708)(503.56147095,70.68970998)
\curveto(503.90147265,70.87970775)(504.17147238,71.13970749)(504.37147095,71.46970998)
\curveto(504.43147212,71.56970706)(504.48147207,71.67470696)(504.52147095,71.78470998)
\curveto(504.551472,71.90470673)(504.58647196,72.02470661)(504.62647095,72.14470998)
\curveto(504.67647187,72.31470632)(504.69647185,72.51970611)(504.68647095,72.75970998)
\curveto(504.66647188,73.00970562)(504.63147192,73.20970542)(504.58147095,73.35970998)
\curveto(504.46147209,73.7297049)(504.30147225,74.01970461)(504.10147095,74.22970998)
\curveto(503.89147266,74.44970418)(503.61147294,74.629704)(503.26147095,74.76970998)
\curveto(503.16147339,74.81970381)(503.05647349,74.84970378)(502.94647095,74.85970998)
\curveto(502.83647371,74.87970375)(502.72147383,74.90470373)(502.60147095,74.93470998)
\lineto(502.49647095,74.93470998)
\curveto(502.45647409,74.94470369)(502.41647413,74.94970368)(502.37647095,74.94970998)
\curveto(502.3464742,74.95970367)(502.31147424,74.95970367)(502.27147095,74.94970998)
\lineto(502.15147095,74.94970998)
\curveto(501.89147466,74.94970368)(501.6464749,74.91970371)(501.41647095,74.85970998)
\curveto(501.06647548,74.74970388)(500.77147578,74.59470404)(500.53147095,74.39470998)
\curveto(500.28147627,74.19470444)(500.08647646,73.9347047)(499.94647095,73.61470998)
\lineto(499.88647095,73.43470998)
\curveto(499.86647668,73.38470525)(499.8464767,73.32470531)(499.82647095,73.25470998)
\curveto(499.80647674,73.20470543)(499.79647675,73.14470549)(499.79647095,73.07470998)
\curveto(499.78647676,73.01470562)(499.77147678,72.94970568)(499.75147095,72.87970998)
\lineto(499.75147095,72.72970998)
\curveto(499.73147682,72.68970594)(499.72147683,72.634706)(499.72147095,72.56470998)
\curveto(499.72147683,72.50470613)(499.73147682,72.44970618)(499.75147095,72.39970998)
\lineto(499.75147095,72.29470998)
\curveto(499.7514768,72.26470637)(499.75647679,72.2297064)(499.76647095,72.18970998)
\lineto(499.82647095,71.94970998)
\curveto(499.83647671,71.86970676)(499.85647669,71.78970684)(499.88647095,71.70970998)
\curveto(499.98647656,71.46970716)(500.12147643,71.23970739)(500.29147095,71.01970998)
\curveto(500.36147619,70.9297077)(500.43647611,70.84470779)(500.51647095,70.76470998)
\curveto(500.58647596,70.68470795)(500.64147591,70.58470805)(500.68147095,70.46470998)
\curveto(500.71147584,70.37470826)(500.72147583,70.2347084)(500.71147095,70.04470998)
\curveto(500.70147585,69.86470877)(500.67647587,69.74470889)(500.63647095,69.68470998)
\curveto(500.59647595,69.634709)(500.53647601,69.59470904)(500.45647095,69.56470998)
\curveto(500.37647617,69.54470909)(500.29147626,69.54470909)(500.20147095,69.56470998)
\curveto(500.08147647,69.59470904)(499.96147659,69.61470902)(499.84147095,69.62470998)
\curveto(499.71147684,69.64470899)(499.58647696,69.66970896)(499.46647095,69.69970998)
\curveto(499.42647712,69.71970891)(499.39147716,69.72470891)(499.36147095,69.71470998)
\curveto(499.32147723,69.71470892)(499.27647727,69.72470891)(499.22647095,69.74470998)
\curveto(499.13647741,69.76470887)(499.0464775,69.77970885)(498.95647095,69.78970998)
\curveto(498.85647769,69.79970883)(498.76147779,69.81970881)(498.67147095,69.84970998)
\curveto(498.61147794,69.85970877)(498.551478,69.86470877)(498.49147095,69.86470998)
\curveto(498.43147812,69.87470876)(498.37147818,69.88970874)(498.31147095,69.90970998)
\curveto(498.11147844,69.95970867)(497.90647864,69.99470864)(497.69647095,70.01470998)
\curveto(497.47647907,70.04470859)(497.26647928,70.08470855)(497.06647095,70.13470998)
\curveto(496.96647958,70.16470847)(496.86647968,70.18470845)(496.76647095,70.19470998)
\curveto(496.66647988,70.20470843)(496.56647998,70.21970841)(496.46647095,70.23970998)
\curveto(496.43648011,70.24970838)(496.39648015,70.25470838)(496.34647095,70.25470998)
\curveto(496.23648031,70.28470835)(496.13148042,70.30470833)(496.03147095,70.31470998)
\curveto(495.92148063,70.3347083)(495.81148074,70.35970827)(495.70147095,70.38970998)
\curveto(495.62148093,70.40970822)(495.551481,70.42470821)(495.49147095,70.43470998)
\curveto(495.42148113,70.44470819)(495.36148119,70.46970816)(495.31147095,70.50970998)
\curveto(495.28148127,70.5297081)(495.26148129,70.55970807)(495.25147095,70.59970998)
\curveto(495.23148132,70.63970799)(495.21148134,70.68470795)(495.19147095,70.73470998)
\curveto(495.19148136,70.79470784)(495.18648136,70.8347078)(495.17647095,70.85470998)
}
}
{
\newrgbcolor{curcolor}{0 0 0}
\pscustom[linestyle=none,fillstyle=solid,fillcolor=curcolor]
{
\newpath
\moveto(503.95147095,78.63431936)
\lineto(503.95147095,79.26431936)
\lineto(503.95147095,79.45931936)
\curveto(503.9514726,79.52931683)(503.96147259,79.58931677)(503.98147095,79.63931936)
\curveto(504.02147253,79.70931665)(504.06147249,79.7593166)(504.10147095,79.78931936)
\curveto(504.1514724,79.82931653)(504.21647233,79.84931651)(504.29647095,79.84931936)
\curveto(504.37647217,79.8593165)(504.46147209,79.86431649)(504.55147095,79.86431936)
\lineto(505.27147095,79.86431936)
\curveto(505.7514708,79.86431649)(506.16147039,79.80431655)(506.50147095,79.68431936)
\curveto(506.84146971,79.56431679)(507.11646943,79.36931699)(507.32647095,79.09931936)
\curveto(507.37646917,79.02931733)(507.42146913,78.9593174)(507.46147095,78.88931936)
\curveto(507.51146904,78.82931753)(507.55646899,78.7543176)(507.59647095,78.66431936)
\curveto(507.60646894,78.64431771)(507.61646893,78.61431774)(507.62647095,78.57431936)
\curveto(507.6464689,78.53431782)(507.6514689,78.48931787)(507.64147095,78.43931936)
\curveto(507.61146894,78.34931801)(507.53646901,78.29431806)(507.41647095,78.27431936)
\curveto(507.30646924,78.2543181)(507.21146934,78.26931809)(507.13147095,78.31931936)
\curveto(507.06146949,78.34931801)(506.99646955,78.39431796)(506.93647095,78.45431936)
\curveto(506.88646966,78.52431783)(506.83646971,78.58931777)(506.78647095,78.64931936)
\curveto(506.73646981,78.71931764)(506.66146989,78.77931758)(506.56147095,78.82931936)
\curveto(506.47147008,78.88931747)(506.38147017,78.93931742)(506.29147095,78.97931936)
\curveto(506.26147029,78.99931736)(506.20147035,79.02431733)(506.11147095,79.05431936)
\curveto(506.03147052,79.08431727)(505.96147059,79.08931727)(505.90147095,79.06931936)
\curveto(505.76147079,79.03931732)(505.67147088,78.97931738)(505.63147095,78.88931936)
\curveto(505.60147095,78.80931755)(505.58647096,78.71931764)(505.58647095,78.61931936)
\curveto(505.58647096,78.51931784)(505.56147099,78.43431792)(505.51147095,78.36431936)
\curveto(505.44147111,78.27431808)(505.30147125,78.22931813)(505.09147095,78.22931936)
\lineto(504.53647095,78.22931936)
\lineto(504.31147095,78.22931936)
\curveto(504.23147232,78.23931812)(504.16647238,78.2593181)(504.11647095,78.28931936)
\curveto(504.03647251,78.34931801)(503.99147256,78.41931794)(503.98147095,78.49931936)
\curveto(503.97147258,78.51931784)(503.96647258,78.53931782)(503.96647095,78.55931936)
\curveto(503.96647258,78.58931777)(503.96147259,78.61431774)(503.95147095,78.63431936)
}
}
{
\newrgbcolor{curcolor}{0 0 0}
\pscustom[linestyle=none,fillstyle=solid,fillcolor=curcolor]
{
}
}
{
\newrgbcolor{curcolor}{0 0 0}
\pscustom[linestyle=none,fillstyle=solid,fillcolor=curcolor]
{
\newpath
\moveto(494.98147095,89.26463186)
\curveto(494.97148158,89.95462722)(495.09148146,90.55462662)(495.34147095,91.06463186)
\curveto(495.59148096,91.58462559)(495.92648062,91.9796252)(496.34647095,92.24963186)
\curveto(496.42648012,92.29962488)(496.51648003,92.34462483)(496.61647095,92.38463186)
\curveto(496.70647984,92.42462475)(496.80147975,92.46962471)(496.90147095,92.51963186)
\curveto(497.00147955,92.55962462)(497.10147945,92.58962459)(497.20147095,92.60963186)
\curveto(497.30147925,92.62962455)(497.40647914,92.64962453)(497.51647095,92.66963186)
\curveto(497.56647898,92.68962449)(497.61147894,92.69462448)(497.65147095,92.68463186)
\curveto(497.69147886,92.6746245)(497.73647881,92.6796245)(497.78647095,92.69963186)
\curveto(497.83647871,92.70962447)(497.92147863,92.71462446)(498.04147095,92.71463186)
\curveto(498.1514784,92.71462446)(498.23647831,92.70962447)(498.29647095,92.69963186)
\curveto(498.35647819,92.6796245)(498.41647813,92.66962451)(498.47647095,92.66963186)
\curveto(498.53647801,92.6796245)(498.59647795,92.6746245)(498.65647095,92.65463186)
\curveto(498.79647775,92.61462456)(498.93147762,92.5796246)(499.06147095,92.54963186)
\curveto(499.19147736,92.51962466)(499.31647723,92.4796247)(499.43647095,92.42963186)
\curveto(499.57647697,92.36962481)(499.70147685,92.29962488)(499.81147095,92.21963186)
\curveto(499.92147663,92.14962503)(500.03147652,92.0746251)(500.14147095,91.99463186)
\lineto(500.20147095,91.93463186)
\curveto(500.22147633,91.92462525)(500.24147631,91.90962527)(500.26147095,91.88963186)
\curveto(500.42147613,91.76962541)(500.56647598,91.63462554)(500.69647095,91.48463186)
\curveto(500.82647572,91.33462584)(500.9514756,91.174626)(501.07147095,91.00463186)
\curveto(501.29147526,90.69462648)(501.49647505,90.39962678)(501.68647095,90.11963186)
\curveto(501.82647472,89.88962729)(501.96147459,89.65962752)(502.09147095,89.42963186)
\curveto(502.22147433,89.20962797)(502.35647419,88.98962819)(502.49647095,88.76963186)
\curveto(502.66647388,88.51962866)(502.8464737,88.2796289)(503.03647095,88.04963186)
\curveto(503.22647332,87.82962935)(503.4514731,87.63962954)(503.71147095,87.47963186)
\curveto(503.77147278,87.43962974)(503.83147272,87.40462977)(503.89147095,87.37463186)
\curveto(503.94147261,87.34462983)(504.00647254,87.31462986)(504.08647095,87.28463186)
\curveto(504.15647239,87.26462991)(504.21647233,87.25962992)(504.26647095,87.26963186)
\curveto(504.33647221,87.28962989)(504.39147216,87.32462985)(504.43147095,87.37463186)
\curveto(504.46147209,87.42462975)(504.48147207,87.48462969)(504.49147095,87.55463186)
\lineto(504.49147095,87.79463186)
\lineto(504.49147095,88.54463186)
\lineto(504.49147095,91.34963186)
\lineto(504.49147095,92.00963186)
\curveto(504.49147206,92.09962508)(504.49647205,92.18462499)(504.50647095,92.26463186)
\curveto(504.50647204,92.34462483)(504.52647202,92.40962477)(504.56647095,92.45963186)
\curveto(504.60647194,92.50962467)(504.68147187,92.54962463)(504.79147095,92.57963186)
\curveto(504.89147166,92.61962456)(504.99147156,92.62962455)(505.09147095,92.60963186)
\lineto(505.22647095,92.60963186)
\curveto(505.29647125,92.58962459)(505.35647119,92.56962461)(505.40647095,92.54963186)
\curveto(505.45647109,92.52962465)(505.49647105,92.49462468)(505.52647095,92.44463186)
\curveto(505.56647098,92.39462478)(505.58647096,92.32462485)(505.58647095,92.23463186)
\lineto(505.58647095,91.96463186)
\lineto(505.58647095,91.06463186)
\lineto(505.58647095,87.55463186)
\lineto(505.58647095,86.48963186)
\curveto(505.58647096,86.40963077)(505.59147096,86.31963086)(505.60147095,86.21963186)
\curveto(505.60147095,86.11963106)(505.59147096,86.03463114)(505.57147095,85.96463186)
\curveto(505.50147105,85.75463142)(505.32147123,85.68963149)(505.03147095,85.76963186)
\curveto(504.99147156,85.7796314)(504.95647159,85.7796314)(504.92647095,85.76963186)
\curveto(504.88647166,85.76963141)(504.84147171,85.7796314)(504.79147095,85.79963186)
\curveto(504.71147184,85.81963136)(504.62647192,85.83963134)(504.53647095,85.85963186)
\curveto(504.4464721,85.8796313)(504.36147219,85.90463127)(504.28147095,85.93463186)
\curveto(503.79147276,86.09463108)(503.37647317,86.29463088)(503.03647095,86.53463186)
\curveto(502.78647376,86.71463046)(502.56147399,86.91963026)(502.36147095,87.14963186)
\curveto(502.1514744,87.3796298)(501.95647459,87.61962956)(501.77647095,87.86963186)
\curveto(501.59647495,88.12962905)(501.42647512,88.39462878)(501.26647095,88.66463186)
\curveto(501.09647545,88.94462823)(500.92147563,89.21462796)(500.74147095,89.47463186)
\curveto(500.66147589,89.58462759)(500.58647596,89.68962749)(500.51647095,89.78963186)
\curveto(500.4464761,89.89962728)(500.37147618,90.00962717)(500.29147095,90.11963186)
\curveto(500.26147629,90.15962702)(500.23147632,90.19462698)(500.20147095,90.22463186)
\curveto(500.16147639,90.26462691)(500.13147642,90.30462687)(500.11147095,90.34463186)
\curveto(500.00147655,90.48462669)(499.87647667,90.60962657)(499.73647095,90.71963186)
\curveto(499.70647684,90.73962644)(499.68147687,90.76462641)(499.66147095,90.79463186)
\curveto(499.63147692,90.82462635)(499.60147695,90.84962633)(499.57147095,90.86963186)
\curveto(499.47147708,90.94962623)(499.37147718,91.01462616)(499.27147095,91.06463186)
\curveto(499.17147738,91.12462605)(499.06147749,91.179626)(498.94147095,91.22963186)
\curveto(498.87147768,91.25962592)(498.79647775,91.2796259)(498.71647095,91.28963186)
\lineto(498.47647095,91.34963186)
\lineto(498.38647095,91.34963186)
\curveto(498.35647819,91.35962582)(498.32647822,91.36462581)(498.29647095,91.36463186)
\curveto(498.22647832,91.38462579)(498.13147842,91.38962579)(498.01147095,91.37963186)
\curveto(497.88147867,91.3796258)(497.78147877,91.36962581)(497.71147095,91.34963186)
\curveto(497.63147892,91.32962585)(497.55647899,91.30962587)(497.48647095,91.28963186)
\curveto(497.40647914,91.2796259)(497.32647922,91.25962592)(497.24647095,91.22963186)
\curveto(497.00647954,91.11962606)(496.80647974,90.96962621)(496.64647095,90.77963186)
\curveto(496.47648007,90.59962658)(496.33648021,90.3796268)(496.22647095,90.11963186)
\curveto(496.20648034,90.04962713)(496.19148036,89.9796272)(496.18147095,89.90963186)
\curveto(496.16148039,89.83962734)(496.14148041,89.76462741)(496.12147095,89.68463186)
\curveto(496.10148045,89.60462757)(496.09148046,89.49462768)(496.09147095,89.35463186)
\curveto(496.09148046,89.22462795)(496.10148045,89.11962806)(496.12147095,89.03963186)
\curveto(496.13148042,88.9796282)(496.13648041,88.92462825)(496.13647095,88.87463186)
\curveto(496.13648041,88.82462835)(496.1464804,88.7746284)(496.16647095,88.72463186)
\curveto(496.20648034,88.62462855)(496.2464803,88.52962865)(496.28647095,88.43963186)
\curveto(496.32648022,88.35962882)(496.37148018,88.2796289)(496.42147095,88.19963186)
\curveto(496.44148011,88.16962901)(496.46648008,88.13962904)(496.49647095,88.10963186)
\curveto(496.52648002,88.08962909)(496.55148,88.06462911)(496.57147095,88.03463186)
\lineto(496.64647095,87.95963186)
\curveto(496.66647988,87.92962925)(496.68647986,87.90462927)(496.70647095,87.88463186)
\lineto(496.91647095,87.73463186)
\curveto(496.97647957,87.69462948)(497.04147951,87.64962953)(497.11147095,87.59963186)
\curveto(497.20147935,87.53962964)(497.30647924,87.48962969)(497.42647095,87.44963186)
\curveto(497.53647901,87.41962976)(497.6464789,87.38462979)(497.75647095,87.34463186)
\curveto(497.86647868,87.30462987)(498.01147854,87.2796299)(498.19147095,87.26963186)
\curveto(498.36147819,87.25962992)(498.48647806,87.22962995)(498.56647095,87.17963186)
\curveto(498.6464779,87.12963005)(498.69147786,87.05463012)(498.70147095,86.95463186)
\curveto(498.71147784,86.85463032)(498.71647783,86.74463043)(498.71647095,86.62463186)
\curveto(498.71647783,86.58463059)(498.72147783,86.54463063)(498.73147095,86.50463186)
\curveto(498.73147782,86.46463071)(498.72647782,86.42963075)(498.71647095,86.39963186)
\curveto(498.69647785,86.34963083)(498.68647786,86.29963088)(498.68647095,86.24963186)
\curveto(498.68647786,86.20963097)(498.67647787,86.16963101)(498.65647095,86.12963186)
\curveto(498.59647795,86.03963114)(498.46147809,85.99463118)(498.25147095,85.99463186)
\lineto(498.13147095,85.99463186)
\curveto(498.07147848,86.00463117)(498.01147854,86.00963117)(497.95147095,86.00963186)
\curveto(497.88147867,86.01963116)(497.81647873,86.02963115)(497.75647095,86.03963186)
\curveto(497.6464789,86.05963112)(497.546479,86.0796311)(497.45647095,86.09963186)
\curveto(497.35647919,86.11963106)(497.26147929,86.14963103)(497.17147095,86.18963186)
\curveto(497.10147945,86.20963097)(497.04147951,86.22963095)(496.99147095,86.24963186)
\lineto(496.81147095,86.30963186)
\curveto(496.55148,86.42963075)(496.30648024,86.58463059)(496.07647095,86.77463186)
\curveto(495.8464807,86.9746302)(495.66148089,87.18962999)(495.52147095,87.41963186)
\curveto(495.44148111,87.52962965)(495.37648117,87.64462953)(495.32647095,87.76463186)
\lineto(495.17647095,88.15463186)
\curveto(495.12648142,88.26462891)(495.09648145,88.3796288)(495.08647095,88.49963186)
\curveto(495.06648148,88.61962856)(495.04148151,88.74462843)(495.01147095,88.87463186)
\curveto(495.01148154,88.94462823)(495.01148154,89.00962817)(495.01147095,89.06963186)
\curveto(495.00148155,89.12962805)(494.99148156,89.19462798)(494.98147095,89.26463186)
}
}
{
\newrgbcolor{curcolor}{0 0 0}
\pscustom[linestyle=none,fillstyle=solid,fillcolor=curcolor]
{
\newpath
\moveto(500.50147095,101.36424123)
\lineto(500.75647095,101.36424123)
\curveto(500.83647571,101.37423353)(500.91147564,101.36923353)(500.98147095,101.34924123)
\lineto(501.22147095,101.34924123)
\lineto(501.38647095,101.34924123)
\curveto(501.48647506,101.32923357)(501.59147496,101.31923358)(501.70147095,101.31924123)
\curveto(501.80147475,101.31923358)(501.90147465,101.30923359)(502.00147095,101.28924123)
\lineto(502.15147095,101.28924123)
\curveto(502.29147426,101.25923364)(502.43147412,101.23923366)(502.57147095,101.22924123)
\curveto(502.70147385,101.21923368)(502.83147372,101.19423371)(502.96147095,101.15424123)
\curveto(503.04147351,101.13423377)(503.12647342,101.11423379)(503.21647095,101.09424123)
\lineto(503.45647095,101.03424123)
\lineto(503.75647095,100.91424123)
\curveto(503.8464727,100.88423402)(503.93647261,100.84923405)(504.02647095,100.80924123)
\curveto(504.2464723,100.70923419)(504.46147209,100.57423433)(504.67147095,100.40424123)
\curveto(504.88147167,100.24423466)(505.0514715,100.06923483)(505.18147095,99.87924123)
\curveto(505.22147133,99.82923507)(505.26147129,99.76923513)(505.30147095,99.69924123)
\curveto(505.33147122,99.63923526)(505.36647118,99.57923532)(505.40647095,99.51924123)
\curveto(505.45647109,99.43923546)(505.49647105,99.34423556)(505.52647095,99.23424123)
\curveto(505.55647099,99.12423578)(505.58647096,99.01923588)(505.61647095,98.91924123)
\curveto(505.65647089,98.80923609)(505.68147087,98.6992362)(505.69147095,98.58924123)
\curveto(505.70147085,98.47923642)(505.71647083,98.36423654)(505.73647095,98.24424123)
\curveto(505.7464708,98.2042367)(505.7464708,98.15923674)(505.73647095,98.10924123)
\curveto(505.73647081,98.06923683)(505.74147081,98.02923687)(505.75147095,97.98924123)
\curveto(505.76147079,97.94923695)(505.76647078,97.89423701)(505.76647095,97.82424123)
\curveto(505.76647078,97.75423715)(505.76147079,97.7042372)(505.75147095,97.67424123)
\curveto(505.73147082,97.62423728)(505.72647082,97.57923732)(505.73647095,97.53924123)
\curveto(505.7464708,97.4992374)(505.7464708,97.46423744)(505.73647095,97.43424123)
\lineto(505.73647095,97.34424123)
\curveto(505.71647083,97.28423762)(505.70147085,97.21923768)(505.69147095,97.14924123)
\curveto(505.69147086,97.08923781)(505.68647086,97.02423788)(505.67647095,96.95424123)
\curveto(505.62647092,96.78423812)(505.57647097,96.62423828)(505.52647095,96.47424123)
\curveto(505.47647107,96.32423858)(505.41147114,96.17923872)(505.33147095,96.03924123)
\curveto(505.29147126,95.98923891)(505.26147129,95.93423897)(505.24147095,95.87424123)
\curveto(505.21147134,95.82423908)(505.17647137,95.77423913)(505.13647095,95.72424123)
\curveto(504.95647159,95.48423942)(504.73647181,95.28423962)(504.47647095,95.12424123)
\curveto(504.21647233,94.96423994)(503.93147262,94.82424008)(503.62147095,94.70424123)
\curveto(503.48147307,94.64424026)(503.34147321,94.5992403)(503.20147095,94.56924123)
\curveto(503.0514735,94.53924036)(502.89647365,94.5042404)(502.73647095,94.46424123)
\curveto(502.62647392,94.44424046)(502.51647403,94.42924047)(502.40647095,94.41924123)
\curveto(502.29647425,94.40924049)(502.18647436,94.39424051)(502.07647095,94.37424123)
\curveto(502.03647451,94.36424054)(501.99647455,94.35924054)(501.95647095,94.35924123)
\curveto(501.91647463,94.36924053)(501.87647467,94.36924053)(501.83647095,94.35924123)
\curveto(501.78647476,94.34924055)(501.73647481,94.34424056)(501.68647095,94.34424123)
\lineto(501.52147095,94.34424123)
\curveto(501.47147508,94.32424058)(501.42147513,94.31924058)(501.37147095,94.32924123)
\curveto(501.31147524,94.33924056)(501.25647529,94.33924056)(501.20647095,94.32924123)
\curveto(501.16647538,94.31924058)(501.12147543,94.31924058)(501.07147095,94.32924123)
\curveto(501.02147553,94.33924056)(500.97147558,94.33424057)(500.92147095,94.31424123)
\curveto(500.8514757,94.29424061)(500.77647577,94.28924061)(500.69647095,94.29924123)
\curveto(500.60647594,94.30924059)(500.52147603,94.31424059)(500.44147095,94.31424123)
\curveto(500.3514762,94.31424059)(500.2514763,94.30924059)(500.14147095,94.29924123)
\curveto(500.02147653,94.28924061)(499.92147663,94.29424061)(499.84147095,94.31424123)
\lineto(499.55647095,94.31424123)
\lineto(498.92647095,94.35924123)
\curveto(498.82647772,94.36924053)(498.73147782,94.37924052)(498.64147095,94.38924123)
\lineto(498.34147095,94.41924123)
\curveto(498.29147826,94.43924046)(498.24147831,94.44424046)(498.19147095,94.43424123)
\curveto(498.13147842,94.43424047)(498.07647847,94.44424046)(498.02647095,94.46424123)
\curveto(497.85647869,94.51424039)(497.69147886,94.55424035)(497.53147095,94.58424123)
\curveto(497.36147919,94.61424029)(497.20147935,94.66424024)(497.05147095,94.73424123)
\curveto(496.59147996,94.92423998)(496.21648033,95.14423976)(495.92647095,95.39424123)
\curveto(495.63648091,95.65423925)(495.39148116,96.01423889)(495.19147095,96.47424123)
\curveto(495.14148141,96.6042383)(495.10648144,96.73423817)(495.08647095,96.86424123)
\curveto(495.06648148,97.0042379)(495.04148151,97.14423776)(495.01147095,97.28424123)
\curveto(495.00148155,97.35423755)(494.99648155,97.41923748)(494.99647095,97.47924123)
\curveto(494.99648155,97.53923736)(494.99148156,97.6042373)(494.98147095,97.67424123)
\curveto(494.96148159,98.5042364)(495.11148144,99.17423573)(495.43147095,99.68424123)
\curveto(495.74148081,100.19423471)(496.18148037,100.57423433)(496.75147095,100.82424123)
\curveto(496.87147968,100.87423403)(496.99647955,100.91923398)(497.12647095,100.95924123)
\curveto(497.25647929,100.9992339)(497.39147916,101.04423386)(497.53147095,101.09424123)
\curveto(497.61147894,101.11423379)(497.69647885,101.12923377)(497.78647095,101.13924123)
\lineto(498.02647095,101.19924123)
\curveto(498.13647841,101.22923367)(498.2464783,101.24423366)(498.35647095,101.24424123)
\curveto(498.46647808,101.25423365)(498.57647797,101.26923363)(498.68647095,101.28924123)
\curveto(498.73647781,101.30923359)(498.78147777,101.31423359)(498.82147095,101.30424123)
\curveto(498.86147769,101.3042336)(498.90147765,101.30923359)(498.94147095,101.31924123)
\curveto(498.99147756,101.32923357)(499.0464775,101.32923357)(499.10647095,101.31924123)
\curveto(499.15647739,101.31923358)(499.20647734,101.32423358)(499.25647095,101.33424123)
\lineto(499.39147095,101.33424123)
\curveto(499.4514771,101.35423355)(499.52147703,101.35423355)(499.60147095,101.33424123)
\curveto(499.67147688,101.32423358)(499.73647681,101.32923357)(499.79647095,101.34924123)
\curveto(499.82647672,101.35923354)(499.86647668,101.36423354)(499.91647095,101.36424123)
\lineto(500.03647095,101.36424123)
\lineto(500.50147095,101.36424123)
\moveto(502.82647095,99.81924123)
\curveto(502.50647404,99.91923498)(502.14147441,99.97923492)(501.73147095,99.99924123)
\curveto(501.32147523,100.01923488)(500.91147564,100.02923487)(500.50147095,100.02924123)
\curveto(500.07147648,100.02923487)(499.6514769,100.01923488)(499.24147095,99.99924123)
\curveto(498.83147772,99.97923492)(498.4464781,99.93423497)(498.08647095,99.86424123)
\curveto(497.72647882,99.79423511)(497.40647914,99.68423522)(497.12647095,99.53424123)
\curveto(496.83647971,99.39423551)(496.60147995,99.1992357)(496.42147095,98.94924123)
\curveto(496.31148024,98.78923611)(496.23148032,98.60923629)(496.18147095,98.40924123)
\curveto(496.12148043,98.20923669)(496.09148046,97.96423694)(496.09147095,97.67424123)
\curveto(496.11148044,97.65423725)(496.12148043,97.61923728)(496.12147095,97.56924123)
\curveto(496.11148044,97.51923738)(496.11148044,97.47923742)(496.12147095,97.44924123)
\curveto(496.14148041,97.36923753)(496.16148039,97.29423761)(496.18147095,97.22424123)
\curveto(496.19148036,97.16423774)(496.21148034,97.0992378)(496.24147095,97.02924123)
\curveto(496.36148019,96.75923814)(496.53148002,96.53923836)(496.75147095,96.36924123)
\curveto(496.96147959,96.20923869)(497.20647934,96.07423883)(497.48647095,95.96424123)
\curveto(497.59647895,95.91423899)(497.71647883,95.87423903)(497.84647095,95.84424123)
\curveto(497.96647858,95.82423908)(498.09147846,95.7992391)(498.22147095,95.76924123)
\curveto(498.27147828,95.74923915)(498.32647822,95.73923916)(498.38647095,95.73924123)
\curveto(498.43647811,95.73923916)(498.48647806,95.73423917)(498.53647095,95.72424123)
\curveto(498.62647792,95.71423919)(498.72147783,95.7042392)(498.82147095,95.69424123)
\curveto(498.91147764,95.68423922)(499.00647754,95.67423923)(499.10647095,95.66424123)
\curveto(499.18647736,95.66423924)(499.27147728,95.65923924)(499.36147095,95.64924123)
\lineto(499.60147095,95.64924123)
\lineto(499.78147095,95.64924123)
\curveto(499.81147674,95.63923926)(499.8464767,95.63423927)(499.88647095,95.63424123)
\lineto(500.02147095,95.63424123)
\lineto(500.47147095,95.63424123)
\curveto(500.551476,95.63423927)(500.63647591,95.62923927)(500.72647095,95.61924123)
\curveto(500.80647574,95.61923928)(500.88147567,95.62923927)(500.95147095,95.64924123)
\lineto(501.22147095,95.64924123)
\curveto(501.24147531,95.64923925)(501.27147528,95.64423926)(501.31147095,95.63424123)
\curveto(501.34147521,95.63423927)(501.36647518,95.63923926)(501.38647095,95.64924123)
\curveto(501.48647506,95.65923924)(501.58647496,95.66423924)(501.68647095,95.66424123)
\curveto(501.77647477,95.67423923)(501.87647467,95.68423922)(501.98647095,95.69424123)
\curveto(502.10647444,95.72423918)(502.23147432,95.73923916)(502.36147095,95.73924123)
\curveto(502.48147407,95.74923915)(502.59647395,95.77423913)(502.70647095,95.81424123)
\curveto(503.00647354,95.89423901)(503.27147328,95.97923892)(503.50147095,96.06924123)
\curveto(503.73147282,96.16923873)(503.9464726,96.31423859)(504.14647095,96.50424123)
\curveto(504.3464722,96.71423819)(504.49647205,96.97923792)(504.59647095,97.29924123)
\curveto(504.61647193,97.33923756)(504.62647192,97.37423753)(504.62647095,97.40424123)
\curveto(504.61647193,97.44423746)(504.62147193,97.48923741)(504.64147095,97.53924123)
\curveto(504.6514719,97.57923732)(504.66147189,97.64923725)(504.67147095,97.74924123)
\curveto(504.68147187,97.85923704)(504.67647187,97.94423696)(504.65647095,98.00424123)
\curveto(504.63647191,98.07423683)(504.62647192,98.14423676)(504.62647095,98.21424123)
\curveto(504.61647193,98.28423662)(504.60147195,98.34923655)(504.58147095,98.40924123)
\curveto(504.52147203,98.60923629)(504.43647211,98.78923611)(504.32647095,98.94924123)
\curveto(504.30647224,98.97923592)(504.28647226,99.0042359)(504.26647095,99.02424123)
\lineto(504.20647095,99.08424123)
\curveto(504.18647236,99.12423578)(504.1464724,99.17423573)(504.08647095,99.23424123)
\curveto(503.9464726,99.33423557)(503.81647273,99.41923548)(503.69647095,99.48924123)
\curveto(503.57647297,99.55923534)(503.43147312,99.62923527)(503.26147095,99.69924123)
\curveto(503.19147336,99.72923517)(503.12147343,99.74923515)(503.05147095,99.75924123)
\curveto(502.98147357,99.77923512)(502.90647364,99.7992351)(502.82647095,99.81924123)
}
}
{
\newrgbcolor{curcolor}{0 0 0}
\pscustom[linestyle=none,fillstyle=solid,fillcolor=curcolor]
{
\newpath
\moveto(494.98147095,106.77385061)
\curveto(494.98148157,106.87384575)(494.99148156,106.96884566)(495.01147095,107.05885061)
\curveto(495.02148153,107.14884548)(495.0514815,107.21384541)(495.10147095,107.25385061)
\curveto(495.18148137,107.31384531)(495.28648126,107.34384528)(495.41647095,107.34385061)
\lineto(495.80647095,107.34385061)
\lineto(497.30647095,107.34385061)
\lineto(503.69647095,107.34385061)
\lineto(504.86647095,107.34385061)
\lineto(505.18147095,107.34385061)
\curveto(505.28147127,107.35384527)(505.36147119,107.33884529)(505.42147095,107.29885061)
\curveto(505.50147105,107.24884538)(505.551471,107.17384545)(505.57147095,107.07385061)
\curveto(505.58147097,106.98384564)(505.58647096,106.87384575)(505.58647095,106.74385061)
\lineto(505.58647095,106.51885061)
\curveto(505.56647098,106.43884619)(505.551471,106.36884626)(505.54147095,106.30885061)
\curveto(505.52147103,106.24884638)(505.48147107,106.19884643)(505.42147095,106.15885061)
\curveto(505.36147119,106.11884651)(505.28647126,106.09884653)(505.19647095,106.09885061)
\lineto(504.89647095,106.09885061)
\lineto(503.80147095,106.09885061)
\lineto(498.46147095,106.09885061)
\curveto(498.37147818,106.07884655)(498.29647825,106.06384656)(498.23647095,106.05385061)
\curveto(498.16647838,106.05384657)(498.10647844,106.0238466)(498.05647095,105.96385061)
\curveto(498.00647854,105.89384673)(497.98147857,105.80384682)(497.98147095,105.69385061)
\curveto(497.97147858,105.59384703)(497.96647858,105.48384714)(497.96647095,105.36385061)
\lineto(497.96647095,104.22385061)
\lineto(497.96647095,103.72885061)
\curveto(497.95647859,103.56884906)(497.89647865,103.45884917)(497.78647095,103.39885061)
\curveto(497.75647879,103.37884925)(497.72647882,103.36884926)(497.69647095,103.36885061)
\curveto(497.65647889,103.36884926)(497.61147894,103.36384926)(497.56147095,103.35385061)
\curveto(497.44147911,103.33384929)(497.33147922,103.33884929)(497.23147095,103.36885061)
\curveto(497.13147942,103.40884922)(497.06147949,103.46384916)(497.02147095,103.53385061)
\curveto(496.97147958,103.61384901)(496.9464796,103.73384889)(496.94647095,103.89385061)
\curveto(496.9464796,104.05384857)(496.93147962,104.18884844)(496.90147095,104.29885061)
\curveto(496.89147966,104.34884828)(496.88647966,104.40384822)(496.88647095,104.46385061)
\curveto(496.87647967,104.5238481)(496.86147969,104.58384804)(496.84147095,104.64385061)
\curveto(496.79147976,104.79384783)(496.74147981,104.93884769)(496.69147095,105.07885061)
\curveto(496.63147992,105.21884741)(496.56147999,105.35384727)(496.48147095,105.48385061)
\curveto(496.39148016,105.623847)(496.28648026,105.74384688)(496.16647095,105.84385061)
\curveto(496.0464805,105.94384668)(495.91648063,106.03884659)(495.77647095,106.12885061)
\curveto(495.67648087,106.18884644)(495.56648098,106.23384639)(495.44647095,106.26385061)
\curveto(495.32648122,106.30384632)(495.22148133,106.35384627)(495.13147095,106.41385061)
\curveto(495.07148148,106.46384616)(495.03148152,106.53384609)(495.01147095,106.62385061)
\curveto(495.00148155,106.64384598)(494.99648155,106.66884596)(494.99647095,106.69885061)
\curveto(494.99648155,106.7288459)(494.99148156,106.75384587)(494.98147095,106.77385061)
}
}
{
\newrgbcolor{curcolor}{0 0 0}
\pscustom[linestyle=none,fillstyle=solid,fillcolor=curcolor]
{
\newpath
\moveto(494.98147095,115.12345998)
\curveto(494.98148157,115.22345513)(494.99148156,115.31845503)(495.01147095,115.40845998)
\curveto(495.02148153,115.49845485)(495.0514815,115.56345479)(495.10147095,115.60345998)
\curveto(495.18148137,115.66345469)(495.28648126,115.69345466)(495.41647095,115.69345998)
\lineto(495.80647095,115.69345998)
\lineto(497.30647095,115.69345998)
\lineto(503.69647095,115.69345998)
\lineto(504.86647095,115.69345998)
\lineto(505.18147095,115.69345998)
\curveto(505.28147127,115.70345465)(505.36147119,115.68845466)(505.42147095,115.64845998)
\curveto(505.50147105,115.59845475)(505.551471,115.52345483)(505.57147095,115.42345998)
\curveto(505.58147097,115.33345502)(505.58647096,115.22345513)(505.58647095,115.09345998)
\lineto(505.58647095,114.86845998)
\curveto(505.56647098,114.78845556)(505.551471,114.71845563)(505.54147095,114.65845998)
\curveto(505.52147103,114.59845575)(505.48147107,114.5484558)(505.42147095,114.50845998)
\curveto(505.36147119,114.46845588)(505.28647126,114.4484559)(505.19647095,114.44845998)
\lineto(504.89647095,114.44845998)
\lineto(503.80147095,114.44845998)
\lineto(498.46147095,114.44845998)
\curveto(498.37147818,114.42845592)(498.29647825,114.41345594)(498.23647095,114.40345998)
\curveto(498.16647838,114.40345595)(498.10647844,114.37345598)(498.05647095,114.31345998)
\curveto(498.00647854,114.24345611)(497.98147857,114.1534562)(497.98147095,114.04345998)
\curveto(497.97147858,113.94345641)(497.96647858,113.83345652)(497.96647095,113.71345998)
\lineto(497.96647095,112.57345998)
\lineto(497.96647095,112.07845998)
\curveto(497.95647859,111.91845843)(497.89647865,111.80845854)(497.78647095,111.74845998)
\curveto(497.75647879,111.72845862)(497.72647882,111.71845863)(497.69647095,111.71845998)
\curveto(497.65647889,111.71845863)(497.61147894,111.71345864)(497.56147095,111.70345998)
\curveto(497.44147911,111.68345867)(497.33147922,111.68845866)(497.23147095,111.71845998)
\curveto(497.13147942,111.75845859)(497.06147949,111.81345854)(497.02147095,111.88345998)
\curveto(496.97147958,111.96345839)(496.9464796,112.08345827)(496.94647095,112.24345998)
\curveto(496.9464796,112.40345795)(496.93147962,112.53845781)(496.90147095,112.64845998)
\curveto(496.89147966,112.69845765)(496.88647966,112.7534576)(496.88647095,112.81345998)
\curveto(496.87647967,112.87345748)(496.86147969,112.93345742)(496.84147095,112.99345998)
\curveto(496.79147976,113.14345721)(496.74147981,113.28845706)(496.69147095,113.42845998)
\curveto(496.63147992,113.56845678)(496.56147999,113.70345665)(496.48147095,113.83345998)
\curveto(496.39148016,113.97345638)(496.28648026,114.09345626)(496.16647095,114.19345998)
\curveto(496.0464805,114.29345606)(495.91648063,114.38845596)(495.77647095,114.47845998)
\curveto(495.67648087,114.53845581)(495.56648098,114.58345577)(495.44647095,114.61345998)
\curveto(495.32648122,114.6534557)(495.22148133,114.70345565)(495.13147095,114.76345998)
\curveto(495.07148148,114.81345554)(495.03148152,114.88345547)(495.01147095,114.97345998)
\curveto(495.00148155,114.99345536)(494.99648155,115.01845533)(494.99647095,115.04845998)
\curveto(494.99648155,115.07845527)(494.99148156,115.10345525)(494.98147095,115.12345998)
}
}
{
\newrgbcolor{curcolor}{0 0 0}
\pscustom[linestyle=none,fillstyle=solid,fillcolor=curcolor]
{
\newpath
\moveto(516.85279053,29.18119436)
\lineto(516.85279053,30.09619436)
\curveto(516.85280122,30.19619171)(516.85280122,30.29119161)(516.85279053,30.38119436)
\curveto(516.85280122,30.47119143)(516.8728012,30.54619136)(516.91279053,30.60619436)
\curveto(516.9728011,30.69619121)(517.05280102,30.75619115)(517.15279053,30.78619436)
\curveto(517.25280082,30.82619108)(517.35780072,30.87119103)(517.46779053,30.92119436)
\curveto(517.65780042,31.0011909)(517.84780023,31.07119083)(518.03779053,31.13119436)
\curveto(518.22779985,31.2011907)(518.41779966,31.27619063)(518.60779053,31.35619436)
\curveto(518.78779929,31.42619048)(518.9727991,31.49119041)(519.16279053,31.55119436)
\curveto(519.34279873,31.61119029)(519.52279855,31.68119022)(519.70279053,31.76119436)
\curveto(519.84279823,31.82119008)(519.98779809,31.87619003)(520.13779053,31.92619436)
\curveto(520.28779779,31.97618993)(520.43279764,32.03118987)(520.57279053,32.09119436)
\curveto(521.02279705,32.27118963)(521.4777966,32.44118946)(521.93779053,32.60119436)
\curveto(522.38779569,32.76118914)(522.83779524,32.93118897)(523.28779053,33.11119436)
\curveto(523.33779474,33.13118877)(523.38779469,33.14618876)(523.43779053,33.15619436)
\lineto(523.58779053,33.21619436)
\curveto(523.80779427,33.3061886)(524.03279404,33.39118851)(524.26279053,33.47119436)
\curveto(524.48279359,33.55118835)(524.70279337,33.63618827)(524.92279053,33.72619436)
\curveto(525.01279306,33.76618814)(525.12279295,33.8061881)(525.25279053,33.84619436)
\curveto(525.3727927,33.88618802)(525.44279263,33.95118795)(525.46279053,34.04119436)
\curveto(525.4727926,34.08118782)(525.4727926,34.11118779)(525.46279053,34.13119436)
\lineto(525.40279053,34.19119436)
\curveto(525.35279272,34.24118766)(525.29779278,34.27618763)(525.23779053,34.29619436)
\curveto(525.1777929,34.32618758)(525.11279296,34.35618755)(525.04279053,34.38619436)
\lineto(524.41279053,34.62619436)
\curveto(524.19279388,34.7061872)(523.9777941,34.78618712)(523.76779053,34.86619436)
\lineto(523.61779053,34.92619436)
\lineto(523.43779053,34.98619436)
\curveto(523.24779483,35.06618684)(523.05779502,35.13618677)(522.86779053,35.19619436)
\curveto(522.66779541,35.26618664)(522.46779561,35.34118656)(522.26779053,35.42119436)
\curveto(521.68779639,35.66118624)(521.10279697,35.88118602)(520.51279053,36.08119436)
\curveto(519.92279815,36.29118561)(519.33779874,36.51618539)(518.75779053,36.75619436)
\curveto(518.55779952,36.83618507)(518.35279972,36.91118499)(518.14279053,36.98119436)
\curveto(517.93280014,37.06118484)(517.72780035,37.14118476)(517.52779053,37.22119436)
\curveto(517.44780063,37.26118464)(517.34780073,37.29618461)(517.22779053,37.32619436)
\curveto(517.10780097,37.36618454)(517.02280105,37.42118448)(516.97279053,37.49119436)
\curveto(516.93280114,37.55118435)(516.90280117,37.62618428)(516.88279053,37.71619436)
\curveto(516.86280121,37.81618409)(516.85280122,37.92618398)(516.85279053,38.04619436)
\curveto(516.84280123,38.16618374)(516.84280123,38.28618362)(516.85279053,38.40619436)
\curveto(516.85280122,38.52618338)(516.85280122,38.63618327)(516.85279053,38.73619436)
\curveto(516.85280122,38.82618308)(516.85280122,38.91618299)(516.85279053,39.00619436)
\curveto(516.85280122,39.1061828)(516.8728012,39.18118272)(516.91279053,39.23119436)
\curveto(516.96280111,39.32118258)(517.05280102,39.37118253)(517.18279053,39.38119436)
\curveto(517.31280076,39.39118251)(517.45280062,39.39618251)(517.60279053,39.39619436)
\lineto(519.25279053,39.39619436)
\lineto(525.52279053,39.39619436)
\lineto(526.78279053,39.39619436)
\curveto(526.89279118,39.39618251)(527.00279107,39.39618251)(527.11279053,39.39619436)
\curveto(527.22279085,39.4061825)(527.30779077,39.38618252)(527.36779053,39.33619436)
\curveto(527.42779065,39.3061826)(527.46779061,39.26118264)(527.48779053,39.20119436)
\curveto(527.49779058,39.14118276)(527.51279056,39.07118283)(527.53279053,38.99119436)
\lineto(527.53279053,38.75119436)
\lineto(527.53279053,38.39119436)
\curveto(527.52279055,38.28118362)(527.4777906,38.2011837)(527.39779053,38.15119436)
\curveto(527.36779071,38.13118377)(527.33779074,38.11618379)(527.30779053,38.10619436)
\curveto(527.26779081,38.1061838)(527.22279085,38.09618381)(527.17279053,38.07619436)
\lineto(527.00779053,38.07619436)
\curveto(526.94779113,38.06618384)(526.8777912,38.06118384)(526.79779053,38.06119436)
\curveto(526.71779136,38.07118383)(526.64279143,38.07618383)(526.57279053,38.07619436)
\lineto(525.73279053,38.07619436)
\lineto(521.30779053,38.07619436)
\curveto(521.05779702,38.07618383)(520.80779727,38.07618383)(520.55779053,38.07619436)
\curveto(520.29779778,38.07618383)(520.04779803,38.07118383)(519.80779053,38.06119436)
\curveto(519.70779837,38.06118384)(519.59779848,38.05618385)(519.47779053,38.04619436)
\curveto(519.35779872,38.03618387)(519.29779878,37.98118392)(519.29779053,37.88119436)
\lineto(519.31279053,37.88119436)
\curveto(519.33279874,37.81118409)(519.39779868,37.75118415)(519.50779053,37.70119436)
\curveto(519.61779846,37.66118424)(519.71279836,37.62618428)(519.79279053,37.59619436)
\curveto(519.96279811,37.52618438)(520.13779794,37.46118444)(520.31779053,37.40119436)
\curveto(520.48779759,37.34118456)(520.65779742,37.27118463)(520.82779053,37.19119436)
\curveto(520.8777972,37.17118473)(520.92279715,37.15618475)(520.96279053,37.14619436)
\curveto(521.00279707,37.13618477)(521.04779703,37.12118478)(521.09779053,37.10119436)
\curveto(521.2777968,37.02118488)(521.46279661,36.95118495)(521.65279053,36.89119436)
\curveto(521.83279624,36.84118506)(522.01279606,36.77618513)(522.19279053,36.69619436)
\curveto(522.34279573,36.62618528)(522.49779558,36.56618534)(522.65779053,36.51619436)
\curveto(522.80779527,36.46618544)(522.95779512,36.41118549)(523.10779053,36.35119436)
\curveto(523.5777945,36.15118575)(524.05279402,35.97118593)(524.53279053,35.81119436)
\curveto(525.00279307,35.65118625)(525.46779261,35.47618643)(525.92779053,35.28619436)
\curveto(526.10779197,35.2061867)(526.28779179,35.13618677)(526.46779053,35.07619436)
\curveto(526.64779143,35.01618689)(526.82779125,34.95118695)(527.00779053,34.88119436)
\curveto(527.11779096,34.83118707)(527.22279085,34.78118712)(527.32279053,34.73119436)
\curveto(527.41279066,34.69118721)(527.4777906,34.6061873)(527.51779053,34.47619436)
\curveto(527.52779055,34.45618745)(527.53279054,34.43118747)(527.53279053,34.40119436)
\curveto(527.52279055,34.38118752)(527.52279055,34.35618755)(527.53279053,34.32619436)
\curveto(527.54279053,34.29618761)(527.54779053,34.26118764)(527.54779053,34.22119436)
\curveto(527.53779054,34.18118772)(527.53279054,34.14118776)(527.53279053,34.10119436)
\lineto(527.53279053,33.80119436)
\curveto(527.53279054,33.7011882)(527.50779057,33.62118828)(527.45779053,33.56119436)
\curveto(527.40779067,33.48118842)(527.33779074,33.42118848)(527.24779053,33.38119436)
\curveto(527.14779093,33.35118855)(527.04779103,33.31118859)(526.94779053,33.26119436)
\curveto(526.74779133,33.18118872)(526.54279153,33.1011888)(526.33279053,33.02119436)
\curveto(526.11279196,32.95118895)(525.90279217,32.87618903)(525.70279053,32.79619436)
\curveto(525.52279255,32.71618919)(525.34279273,32.64618926)(525.16279053,32.58619436)
\curveto(524.9727931,32.53618937)(524.78779329,32.47118943)(524.60779053,32.39119436)
\curveto(524.04779403,32.16118974)(523.48279459,31.94618996)(522.91279053,31.74619436)
\curveto(522.34279573,31.54619036)(521.7777963,31.33119057)(521.21779053,31.10119436)
\lineto(520.58779053,30.86119436)
\curveto(520.36779771,30.79119111)(520.15779792,30.71619119)(519.95779053,30.63619436)
\curveto(519.84779823,30.58619132)(519.74279833,30.54119136)(519.64279053,30.50119436)
\curveto(519.53279854,30.47119143)(519.43779864,30.42119148)(519.35779053,30.35119436)
\curveto(519.33779874,30.34119156)(519.32779875,30.33119157)(519.32779053,30.32119436)
\lineto(519.29779053,30.29119436)
\lineto(519.29779053,30.21619436)
\lineto(519.32779053,30.18619436)
\curveto(519.32779875,30.17619173)(519.33279874,30.16619174)(519.34279053,30.15619436)
\curveto(519.39279868,30.13619177)(519.44779863,30.12619178)(519.50779053,30.12619436)
\curveto(519.56779851,30.12619178)(519.62779845,30.11619179)(519.68779053,30.09619436)
\lineto(519.85279053,30.09619436)
\curveto(519.91279816,30.07619183)(519.9777981,30.07119183)(520.04779053,30.08119436)
\curveto(520.11779796,30.09119181)(520.18779789,30.09619181)(520.25779053,30.09619436)
\lineto(521.06779053,30.09619436)
\lineto(525.62779053,30.09619436)
\lineto(526.81279053,30.09619436)
\curveto(526.92279115,30.09619181)(527.03279104,30.09119181)(527.14279053,30.08119436)
\curveto(527.25279082,30.08119182)(527.33779074,30.05619185)(527.39779053,30.00619436)
\curveto(527.4777906,29.95619195)(527.52279055,29.86619204)(527.53279053,29.73619436)
\lineto(527.53279053,29.34619436)
\lineto(527.53279053,29.15119436)
\curveto(527.53279054,29.1011928)(527.52279055,29.05119285)(527.50279053,29.00119436)
\curveto(527.46279061,28.87119303)(527.3777907,28.79619311)(527.24779053,28.77619436)
\curveto(527.11779096,28.76619314)(526.96779111,28.76119314)(526.79779053,28.76119436)
\lineto(525.05779053,28.76119436)
\lineto(519.05779053,28.76119436)
\lineto(517.64779053,28.76119436)
\curveto(517.53780054,28.76119314)(517.42280065,28.75619315)(517.30279053,28.74619436)
\curveto(517.18280089,28.74619316)(517.08780099,28.77119313)(517.01779053,28.82119436)
\curveto(516.95780112,28.86119304)(516.90780117,28.93619297)(516.86779053,29.04619436)
\curveto(516.85780122,29.06619284)(516.85780122,29.08619282)(516.86779053,29.10619436)
\curveto(516.86780121,29.13619277)(516.86280121,29.16119274)(516.85279053,29.18119436)
}
}
{
\newrgbcolor{curcolor}{0 0 0}
\pscustom[linestyle=none,fillstyle=solid,fillcolor=curcolor]
{
\newpath
\moveto(526.97779053,48.38330373)
\curveto(527.13779094,48.4132959)(527.2727908,48.39829592)(527.38279053,48.33830373)
\curveto(527.48279059,48.27829604)(527.55779052,48.19829612)(527.60779053,48.09830373)
\curveto(527.62779045,48.04829627)(527.63779044,47.99329632)(527.63779053,47.93330373)
\curveto(527.63779044,47.88329643)(527.64779043,47.82829649)(527.66779053,47.76830373)
\curveto(527.71779036,47.54829677)(527.70279037,47.32829699)(527.62279053,47.10830373)
\curveto(527.55279052,46.89829742)(527.46279061,46.75329756)(527.35279053,46.67330373)
\curveto(527.28279079,46.62329769)(527.20279087,46.57829774)(527.11279053,46.53830373)
\curveto(527.01279106,46.49829782)(526.93279114,46.44829787)(526.87279053,46.38830373)
\curveto(526.85279122,46.36829795)(526.83279124,46.34329797)(526.81279053,46.31330373)
\curveto(526.79279128,46.29329802)(526.78779129,46.26329805)(526.79779053,46.22330373)
\curveto(526.82779125,46.1132982)(526.88279119,46.00829831)(526.96279053,45.90830373)
\curveto(527.04279103,45.8182985)(527.11279096,45.72829859)(527.17279053,45.63830373)
\curveto(527.25279082,45.50829881)(527.32779075,45.36829895)(527.39779053,45.21830373)
\curveto(527.45779062,45.06829925)(527.51279056,44.90829941)(527.56279053,44.73830373)
\curveto(527.59279048,44.63829968)(527.61279046,44.52829979)(527.62279053,44.40830373)
\curveto(527.63279044,44.29830002)(527.64779043,44.18830013)(527.66779053,44.07830373)
\curveto(527.6777904,44.02830029)(527.68279039,43.98330033)(527.68279053,43.94330373)
\lineto(527.68279053,43.83830373)
\curveto(527.70279037,43.72830059)(527.70279037,43.62330069)(527.68279053,43.52330373)
\lineto(527.68279053,43.38830373)
\curveto(527.6727904,43.33830098)(527.66779041,43.28830103)(527.66779053,43.23830373)
\curveto(527.66779041,43.18830113)(527.65779042,43.14330117)(527.63779053,43.10330373)
\curveto(527.62779045,43.06330125)(527.62279045,43.02830129)(527.62279053,42.99830373)
\curveto(527.63279044,42.97830134)(527.63279044,42.95330136)(527.62279053,42.92330373)
\lineto(527.56279053,42.68330373)
\curveto(527.55279052,42.60330171)(527.53279054,42.52830179)(527.50279053,42.45830373)
\curveto(527.3727907,42.15830216)(527.22779085,41.9133024)(527.06779053,41.72330373)
\curveto(526.89779118,41.54330277)(526.66279141,41.39330292)(526.36279053,41.27330373)
\curveto(526.14279193,41.18330313)(525.8777922,41.13830318)(525.56779053,41.13830373)
\lineto(525.25279053,41.13830373)
\curveto(525.20279287,41.14830317)(525.15279292,41.15330316)(525.10279053,41.15330373)
\lineto(524.92279053,41.18330373)
\lineto(524.59279053,41.30330373)
\curveto(524.48279359,41.34330297)(524.38279369,41.39330292)(524.29279053,41.45330373)
\curveto(524.00279407,41.63330268)(523.78779429,41.87830244)(523.64779053,42.18830373)
\curveto(523.50779457,42.49830182)(523.38279469,42.83830148)(523.27279053,43.20830373)
\curveto(523.23279484,43.34830097)(523.20279487,43.49330082)(523.18279053,43.64330373)
\curveto(523.16279491,43.79330052)(523.13779494,43.94330037)(523.10779053,44.09330373)
\curveto(523.08779499,44.16330015)(523.077795,44.22830009)(523.07779053,44.28830373)
\curveto(523.077795,44.35829996)(523.06779501,44.43329988)(523.04779053,44.51330373)
\curveto(523.02779505,44.58329973)(523.01779506,44.65329966)(523.01779053,44.72330373)
\curveto(523.00779507,44.79329952)(522.99279508,44.86829945)(522.97279053,44.94830373)
\curveto(522.91279516,45.19829912)(522.86279521,45.43329888)(522.82279053,45.65330373)
\curveto(522.7727953,45.87329844)(522.65779542,46.04829827)(522.47779053,46.17830373)
\curveto(522.39779568,46.23829808)(522.29779578,46.28829803)(522.17779053,46.32830373)
\curveto(522.04779603,46.36829795)(521.90779617,46.36829795)(521.75779053,46.32830373)
\curveto(521.51779656,46.26829805)(521.32779675,46.17829814)(521.18779053,46.05830373)
\curveto(521.04779703,45.94829837)(520.93779714,45.78829853)(520.85779053,45.57830373)
\curveto(520.80779727,45.45829886)(520.7727973,45.313299)(520.75279053,45.14330373)
\curveto(520.73279734,44.98329933)(520.72279735,44.8132995)(520.72279053,44.63330373)
\curveto(520.72279735,44.45329986)(520.73279734,44.27830004)(520.75279053,44.10830373)
\curveto(520.7727973,43.93830038)(520.80279727,43.79330052)(520.84279053,43.67330373)
\curveto(520.90279717,43.50330081)(520.98779709,43.33830098)(521.09779053,43.17830373)
\curveto(521.15779692,43.09830122)(521.23779684,43.02330129)(521.33779053,42.95330373)
\curveto(521.42779665,42.89330142)(521.52779655,42.83830148)(521.63779053,42.78830373)
\curveto(521.71779636,42.75830156)(521.80279627,42.72830159)(521.89279053,42.69830373)
\curveto(521.98279609,42.67830164)(522.05279602,42.63330168)(522.10279053,42.56330373)
\curveto(522.13279594,42.52330179)(522.15779592,42.45330186)(522.17779053,42.35330373)
\curveto(522.18779589,42.26330205)(522.19279588,42.16830215)(522.19279053,42.06830373)
\curveto(522.19279588,41.96830235)(522.18779589,41.86830245)(522.17779053,41.76830373)
\curveto(522.15779592,41.67830264)(522.13279594,41.6133027)(522.10279053,41.57330373)
\curveto(522.072796,41.53330278)(522.02279605,41.50330281)(521.95279053,41.48330373)
\curveto(521.88279619,41.46330285)(521.80779627,41.46330285)(521.72779053,41.48330373)
\curveto(521.59779648,41.5133028)(521.4777966,41.54330277)(521.36779053,41.57330373)
\curveto(521.24779683,41.6133027)(521.13279694,41.65830266)(521.02279053,41.70830373)
\curveto(520.6727974,41.89830242)(520.40279767,42.13830218)(520.21279053,42.42830373)
\curveto(520.01279806,42.7183016)(519.85279822,43.07830124)(519.73279053,43.50830373)
\curveto(519.71279836,43.60830071)(519.69779838,43.70830061)(519.68779053,43.80830373)
\curveto(519.6777984,43.9183004)(519.66279841,44.02830029)(519.64279053,44.13830373)
\curveto(519.63279844,44.17830014)(519.63279844,44.24330007)(519.64279053,44.33330373)
\curveto(519.64279843,44.42329989)(519.63279844,44.47829984)(519.61279053,44.49830373)
\curveto(519.60279847,45.19829912)(519.68279839,45.80829851)(519.85279053,46.32830373)
\curveto(520.02279805,46.84829747)(520.34779773,47.2132971)(520.82779053,47.42330373)
\curveto(521.02779705,47.5132968)(521.26279681,47.56329675)(521.53279053,47.57330373)
\curveto(521.79279628,47.59329672)(522.06779601,47.60329671)(522.35779053,47.60330373)
\lineto(525.67279053,47.60330373)
\curveto(525.81279226,47.60329671)(525.94779213,47.60829671)(526.07779053,47.61830373)
\curveto(526.20779187,47.62829669)(526.31279176,47.65829666)(526.39279053,47.70830373)
\curveto(526.46279161,47.75829656)(526.51279156,47.82329649)(526.54279053,47.90330373)
\curveto(526.58279149,47.99329632)(526.61279146,48.07829624)(526.63279053,48.15830373)
\curveto(526.64279143,48.23829608)(526.68779139,48.29829602)(526.76779053,48.33830373)
\curveto(526.79779128,48.35829596)(526.82779125,48.36829595)(526.85779053,48.36830373)
\curveto(526.88779119,48.36829595)(526.92779115,48.37329594)(526.97779053,48.38330373)
\moveto(525.31279053,46.23830373)
\curveto(525.1727929,46.29829802)(525.01279306,46.32829799)(524.83279053,46.32830373)
\curveto(524.64279343,46.33829798)(524.44779363,46.34329797)(524.24779053,46.34330373)
\curveto(524.13779394,46.34329797)(524.03779404,46.33829798)(523.94779053,46.32830373)
\curveto(523.85779422,46.318298)(523.78779429,46.27829804)(523.73779053,46.20830373)
\curveto(523.71779436,46.17829814)(523.70779437,46.10829821)(523.70779053,45.99830373)
\curveto(523.72779435,45.97829834)(523.73779434,45.94329837)(523.73779053,45.89330373)
\curveto(523.73779434,45.84329847)(523.74779433,45.79829852)(523.76779053,45.75830373)
\curveto(523.78779429,45.67829864)(523.80779427,45.58829873)(523.82779053,45.48830373)
\lineto(523.88779053,45.18830373)
\curveto(523.88779419,45.15829916)(523.89279418,45.12329919)(523.90279053,45.08330373)
\lineto(523.90279053,44.97830373)
\curveto(523.94279413,44.82829949)(523.96779411,44.66329965)(523.97779053,44.48330373)
\curveto(523.9777941,44.3133)(523.99779408,44.15330016)(524.03779053,44.00330373)
\curveto(524.05779402,43.92330039)(524.077794,43.84830047)(524.09779053,43.77830373)
\curveto(524.10779397,43.7183006)(524.12279395,43.64830067)(524.14279053,43.56830373)
\curveto(524.19279388,43.40830091)(524.25779382,43.25830106)(524.33779053,43.11830373)
\curveto(524.40779367,42.97830134)(524.49779358,42.85830146)(524.60779053,42.75830373)
\curveto(524.71779336,42.65830166)(524.85279322,42.58330173)(525.01279053,42.53330373)
\curveto(525.16279291,42.48330183)(525.34779273,42.46330185)(525.56779053,42.47330373)
\curveto(525.66779241,42.47330184)(525.76279231,42.48830183)(525.85279053,42.51830373)
\curveto(525.93279214,42.55830176)(526.00779207,42.60330171)(526.07779053,42.65330373)
\curveto(526.18779189,42.73330158)(526.28279179,42.83830148)(526.36279053,42.96830373)
\curveto(526.43279164,43.09830122)(526.49279158,43.23830108)(526.54279053,43.38830373)
\curveto(526.55279152,43.43830088)(526.55779152,43.48830083)(526.55779053,43.53830373)
\curveto(526.55779152,43.58830073)(526.56279151,43.63830068)(526.57279053,43.68830373)
\curveto(526.59279148,43.75830056)(526.60779147,43.84330047)(526.61779053,43.94330373)
\curveto(526.61779146,44.05330026)(526.60779147,44.14330017)(526.58779053,44.21330373)
\curveto(526.56779151,44.27330004)(526.56279151,44.33329998)(526.57279053,44.39330373)
\curveto(526.5727915,44.45329986)(526.56279151,44.5132998)(526.54279053,44.57330373)
\curveto(526.52279155,44.65329966)(526.50779157,44.72829959)(526.49779053,44.79830373)
\curveto(526.48779159,44.87829944)(526.46779161,44.95329936)(526.43779053,45.02330373)
\curveto(526.31779176,45.313299)(526.1727919,45.55829876)(526.00279053,45.75830373)
\curveto(525.83279224,45.96829835)(525.60279247,46.12829819)(525.31279053,46.23830373)
}
}
{
\newrgbcolor{curcolor}{0 0 0}
\pscustom[linestyle=none,fillstyle=solid,fillcolor=curcolor]
{
\newpath
\moveto(519.80779053,49.26994436)
\lineto(519.80779053,49.71994436)
\curveto(519.79779828,49.88994311)(519.81779826,50.01494298)(519.86779053,50.09494436)
\curveto(519.91779816,50.17494282)(519.98279809,50.22994277)(520.06279053,50.25994436)
\curveto(520.14279793,50.2999427)(520.22779785,50.33994266)(520.31779053,50.37994436)
\curveto(520.44779763,50.42994257)(520.5777975,50.47494252)(520.70779053,50.51494436)
\curveto(520.83779724,50.55494244)(520.96779711,50.5999424)(521.09779053,50.64994436)
\curveto(521.21779686,50.6999423)(521.34279673,50.74494225)(521.47279053,50.78494436)
\curveto(521.59279648,50.82494217)(521.71279636,50.86994213)(521.83279053,50.91994436)
\curveto(521.94279613,50.96994203)(522.05779602,51.00994199)(522.17779053,51.03994436)
\curveto(522.29779578,51.06994193)(522.41779566,51.10994189)(522.53779053,51.15994436)
\curveto(522.82779525,51.27994172)(523.12779495,51.38994161)(523.43779053,51.48994436)
\curveto(523.74779433,51.58994141)(524.04779403,51.6999413)(524.33779053,51.81994436)
\curveto(524.3777937,51.83994116)(524.41779366,51.84994115)(524.45779053,51.84994436)
\curveto(524.48779359,51.84994115)(524.51779356,51.85994114)(524.54779053,51.87994436)
\curveto(524.68779339,51.93994106)(524.83279324,51.994941)(524.98279053,52.04494436)
\lineto(525.40279053,52.19494436)
\curveto(525.4727926,52.22494077)(525.54779253,52.25494074)(525.62779053,52.28494436)
\curveto(525.69779238,52.31494068)(525.74279233,52.36494063)(525.76279053,52.43494436)
\curveto(525.79279228,52.51494048)(525.76779231,52.57494042)(525.68779053,52.61494436)
\curveto(525.59779248,52.66494033)(525.52779255,52.6999403)(525.47779053,52.71994436)
\curveto(525.30779277,52.7999402)(525.12779295,52.86494013)(524.93779053,52.91494436)
\curveto(524.74779333,52.96494003)(524.56279351,53.02493997)(524.38279053,53.09494436)
\curveto(524.15279392,53.18493981)(523.92279415,53.26493973)(523.69279053,53.33494436)
\curveto(523.45279462,53.40493959)(523.22279485,53.48993951)(523.00279053,53.58994436)
\curveto(522.95279512,53.5999394)(522.88779519,53.61493938)(522.80779053,53.63494436)
\curveto(522.71779536,53.67493932)(522.62779545,53.70993929)(522.53779053,53.73994436)
\curveto(522.43779564,53.76993923)(522.34779573,53.7999392)(522.26779053,53.82994436)
\curveto(522.21779586,53.84993915)(522.1727959,53.86493913)(522.13279053,53.87494436)
\curveto(522.09279598,53.88493911)(522.04779603,53.8999391)(521.99779053,53.91994436)
\curveto(521.8777962,53.96993903)(521.75779632,54.00993899)(521.63779053,54.03994436)
\curveto(521.50779657,54.07993892)(521.38279669,54.12493887)(521.26279053,54.17494436)
\curveto(521.21279686,54.1949388)(521.16779691,54.20993879)(521.12779053,54.21994436)
\curveto(521.08779699,54.22993877)(521.04279703,54.24493875)(520.99279053,54.26494436)
\curveto(520.90279717,54.30493869)(520.81279726,54.33993866)(520.72279053,54.36994436)
\curveto(520.62279745,54.3999386)(520.52779755,54.42993857)(520.43779053,54.45994436)
\curveto(520.35779772,54.48993851)(520.2777978,54.51493848)(520.19779053,54.53494436)
\curveto(520.10779797,54.56493843)(520.03279804,54.60493839)(519.97279053,54.65494436)
\curveto(519.88279819,54.72493827)(519.83279824,54.81993818)(519.82279053,54.93994436)
\curveto(519.81279826,55.06993793)(519.80779827,55.20993779)(519.80779053,55.35994436)
\curveto(519.80779827,55.43993756)(519.81279826,55.51493748)(519.82279053,55.58494436)
\curveto(519.82279825,55.66493733)(519.83779824,55.72993727)(519.86779053,55.77994436)
\curveto(519.92779815,55.86993713)(520.02279805,55.8949371)(520.15279053,55.85494436)
\curveto(520.28279779,55.81493718)(520.38279769,55.77993722)(520.45279053,55.74994436)
\lineto(520.51279053,55.71994436)
\curveto(520.53279754,55.71993728)(520.55279752,55.71493728)(520.57279053,55.70494436)
\curveto(520.85279722,55.5949374)(521.13779694,55.48493751)(521.42779053,55.37494436)
\lineto(522.26779053,55.04494436)
\curveto(522.34779573,55.01493798)(522.42279565,54.98993801)(522.49279053,54.96994436)
\curveto(522.55279552,54.94993805)(522.61779546,54.92493807)(522.68779053,54.89494436)
\curveto(522.88779519,54.81493818)(523.09279498,54.73493826)(523.30279053,54.65494436)
\curveto(523.50279457,54.58493841)(523.70279437,54.50993849)(523.90279053,54.42994436)
\curveto(524.59279348,54.13993886)(525.28779279,53.86993913)(525.98779053,53.61994436)
\curveto(526.68779139,53.36993963)(527.38279069,53.0999399)(528.07279053,52.80994436)
\lineto(528.22279053,52.74994436)
\curveto(528.28278979,52.73994026)(528.34278973,52.72494027)(528.40279053,52.70494436)
\curveto(528.7727893,52.54494045)(529.13778894,52.37494062)(529.49779053,52.19494436)
\curveto(529.86778821,52.01494098)(530.15278792,51.76494123)(530.35279053,51.44494436)
\curveto(530.41278766,51.33494166)(530.45778762,51.22494177)(530.48779053,51.11494436)
\curveto(530.52778755,51.00494199)(530.56278751,50.87994212)(530.59279053,50.73994436)
\curveto(530.61278746,50.68994231)(530.61778746,50.63494236)(530.60779053,50.57494436)
\curveto(530.59778748,50.52494247)(530.59778748,50.46994253)(530.60779053,50.40994436)
\curveto(530.62778745,50.32994267)(530.62778745,50.24994275)(530.60779053,50.16994436)
\curveto(530.59778748,50.12994287)(530.59278748,50.07994292)(530.59279053,50.01994436)
\lineto(530.53279053,49.77994436)
\curveto(530.51278756,49.70994329)(530.4727876,49.65494334)(530.41279053,49.61494436)
\curveto(530.35278772,49.56494343)(530.2777878,49.53494346)(530.18779053,49.52494436)
\lineto(529.91779053,49.52494436)
\lineto(529.70779053,49.52494436)
\curveto(529.64778843,49.53494346)(529.59778848,49.55494344)(529.55779053,49.58494436)
\curveto(529.44778863,49.65494334)(529.41778866,49.77494322)(529.46779053,49.94494436)
\curveto(529.48778859,50.05494294)(529.49778858,50.17494282)(529.49779053,50.30494436)
\curveto(529.49778858,50.43494256)(529.4777886,50.54994245)(529.43779053,50.64994436)
\curveto(529.38778869,50.7999422)(529.31278876,50.91994208)(529.21279053,51.00994436)
\curveto(529.11278896,51.10994189)(528.99778908,51.1949418)(528.86779053,51.26494436)
\curveto(528.74778933,51.33494166)(528.61778946,51.3949416)(528.47779053,51.44494436)
\lineto(528.05779053,51.62494436)
\curveto(527.96779011,51.66494133)(527.85779022,51.70494129)(527.72779053,51.74494436)
\curveto(527.59779048,51.7949412)(527.46279061,51.7999412)(527.32279053,51.75994436)
\curveto(527.16279091,51.70994129)(527.01279106,51.65494134)(526.87279053,51.59494436)
\curveto(526.73279134,51.54494145)(526.59279148,51.48994151)(526.45279053,51.42994436)
\curveto(526.24279183,51.33994166)(526.03279204,51.25494174)(525.82279053,51.17494436)
\curveto(525.61279246,51.0949419)(525.40779267,51.01494198)(525.20779053,50.93494436)
\curveto(525.06779301,50.87494212)(524.93279314,50.81994218)(524.80279053,50.76994436)
\curveto(524.6727934,50.71994228)(524.53779354,50.66994233)(524.39779053,50.61994436)
\lineto(523.07779053,50.07994436)
\curveto(522.63779544,49.90994309)(522.19779588,49.73494326)(521.75779053,49.55494436)
\curveto(521.52779655,49.45494354)(521.30779677,49.36494363)(521.09779053,49.28494436)
\curveto(520.8777972,49.20494379)(520.65779742,49.11994388)(520.43779053,49.02994436)
\curveto(520.3777977,49.00994399)(520.29779778,48.97994402)(520.19779053,48.93994436)
\curveto(520.08779799,48.8999441)(519.99779808,48.90494409)(519.92779053,48.95494436)
\curveto(519.8777982,48.98494401)(519.84279823,49.04494395)(519.82279053,49.13494436)
\curveto(519.81279826,49.15494384)(519.81279826,49.17494382)(519.82279053,49.19494436)
\curveto(519.82279825,49.22494377)(519.81779826,49.24994375)(519.80779053,49.26994436)
}
}
{
\newrgbcolor{curcolor}{0 0 0}
\pscustom[linestyle=none,fillstyle=solid,fillcolor=curcolor]
{
}
}
{
\newrgbcolor{curcolor}{0 0 0}
\pscustom[linestyle=none,fillstyle=solid,fillcolor=curcolor]
{
\newpath
\moveto(522.44779053,67.99510061)
\lineto(522.70279053,67.99510061)
\curveto(522.78279529,68.0050929)(522.85779522,68.00009291)(522.92779053,67.98010061)
\lineto(523.16779053,67.98010061)
\lineto(523.33279053,67.98010061)
\curveto(523.43279464,67.96009295)(523.53779454,67.95009296)(523.64779053,67.95010061)
\curveto(523.74779433,67.95009296)(523.84779423,67.94009297)(523.94779053,67.92010061)
\lineto(524.09779053,67.92010061)
\curveto(524.23779384,67.89009302)(524.3777937,67.87009304)(524.51779053,67.86010061)
\curveto(524.64779343,67.85009306)(524.7777933,67.82509308)(524.90779053,67.78510061)
\curveto(524.98779309,67.76509314)(525.072793,67.74509316)(525.16279053,67.72510061)
\lineto(525.40279053,67.66510061)
\lineto(525.70279053,67.54510061)
\curveto(525.79279228,67.51509339)(525.88279219,67.48009343)(525.97279053,67.44010061)
\curveto(526.19279188,67.34009357)(526.40779167,67.2050937)(526.61779053,67.03510061)
\curveto(526.82779125,66.87509403)(526.99779108,66.70009421)(527.12779053,66.51010061)
\curveto(527.16779091,66.46009445)(527.20779087,66.40009451)(527.24779053,66.33010061)
\curveto(527.2777908,66.27009464)(527.31279076,66.2100947)(527.35279053,66.15010061)
\curveto(527.40279067,66.07009484)(527.44279063,65.97509493)(527.47279053,65.86510061)
\curveto(527.50279057,65.75509515)(527.53279054,65.65009526)(527.56279053,65.55010061)
\curveto(527.60279047,65.44009547)(527.62779045,65.33009558)(527.63779053,65.22010061)
\curveto(527.64779043,65.1100958)(527.66279041,64.99509591)(527.68279053,64.87510061)
\curveto(527.69279038,64.83509607)(527.69279038,64.79009612)(527.68279053,64.74010061)
\curveto(527.68279039,64.70009621)(527.68779039,64.66009625)(527.69779053,64.62010061)
\curveto(527.70779037,64.58009633)(527.71279036,64.52509638)(527.71279053,64.45510061)
\curveto(527.71279036,64.38509652)(527.70779037,64.33509657)(527.69779053,64.30510061)
\curveto(527.6777904,64.25509665)(527.6727904,64.2100967)(527.68279053,64.17010061)
\curveto(527.69279038,64.13009678)(527.69279038,64.09509681)(527.68279053,64.06510061)
\lineto(527.68279053,63.97510061)
\curveto(527.66279041,63.91509699)(527.64779043,63.85009706)(527.63779053,63.78010061)
\curveto(527.63779044,63.72009719)(527.63279044,63.65509725)(527.62279053,63.58510061)
\curveto(527.5727905,63.41509749)(527.52279055,63.25509765)(527.47279053,63.10510061)
\curveto(527.42279065,62.95509795)(527.35779072,62.8100981)(527.27779053,62.67010061)
\curveto(527.23779084,62.62009829)(527.20779087,62.56509834)(527.18779053,62.50510061)
\curveto(527.15779092,62.45509845)(527.12279095,62.4050985)(527.08279053,62.35510061)
\curveto(526.90279117,62.11509879)(526.68279139,61.91509899)(526.42279053,61.75510061)
\curveto(526.16279191,61.59509931)(525.8777922,61.45509945)(525.56779053,61.33510061)
\curveto(525.42779265,61.27509963)(525.28779279,61.23009968)(525.14779053,61.20010061)
\curveto(524.99779308,61.17009974)(524.84279323,61.13509977)(524.68279053,61.09510061)
\curveto(524.5727935,61.07509983)(524.46279361,61.06009985)(524.35279053,61.05010061)
\curveto(524.24279383,61.04009987)(524.13279394,61.02509988)(524.02279053,61.00510061)
\curveto(523.98279409,60.99509991)(523.94279413,60.99009992)(523.90279053,60.99010061)
\curveto(523.86279421,61.00009991)(523.82279425,61.00009991)(523.78279053,60.99010061)
\curveto(523.73279434,60.98009993)(523.68279439,60.97509993)(523.63279053,60.97510061)
\lineto(523.46779053,60.97510061)
\curveto(523.41779466,60.95509995)(523.36779471,60.95009996)(523.31779053,60.96010061)
\curveto(523.25779482,60.97009994)(523.20279487,60.97009994)(523.15279053,60.96010061)
\curveto(523.11279496,60.95009996)(523.06779501,60.95009996)(523.01779053,60.96010061)
\curveto(522.96779511,60.97009994)(522.91779516,60.96509994)(522.86779053,60.94510061)
\curveto(522.79779528,60.92509998)(522.72279535,60.92009999)(522.64279053,60.93010061)
\curveto(522.55279552,60.94009997)(522.46779561,60.94509996)(522.38779053,60.94510061)
\curveto(522.29779578,60.94509996)(522.19779588,60.94009997)(522.08779053,60.93010061)
\curveto(521.96779611,60.92009999)(521.86779621,60.92509998)(521.78779053,60.94510061)
\lineto(521.50279053,60.94510061)
\lineto(520.87279053,60.99010061)
\curveto(520.7727973,61.00009991)(520.6777974,61.0100999)(520.58779053,61.02010061)
\lineto(520.28779053,61.05010061)
\curveto(520.23779784,61.07009984)(520.18779789,61.07509983)(520.13779053,61.06510061)
\curveto(520.077798,61.06509984)(520.02279805,61.07509983)(519.97279053,61.09510061)
\curveto(519.80279827,61.14509976)(519.63779844,61.18509972)(519.47779053,61.21510061)
\curveto(519.30779877,61.24509966)(519.14779893,61.29509961)(518.99779053,61.36510061)
\curveto(518.53779954,61.55509935)(518.16279991,61.77509913)(517.87279053,62.02510061)
\curveto(517.58280049,62.28509862)(517.33780074,62.64509826)(517.13779053,63.10510061)
\curveto(517.08780099,63.23509767)(517.05280102,63.36509754)(517.03279053,63.49510061)
\curveto(517.01280106,63.63509727)(516.98780109,63.77509713)(516.95779053,63.91510061)
\curveto(516.94780113,63.98509692)(516.94280113,64.05009686)(516.94279053,64.11010061)
\curveto(516.94280113,64.17009674)(516.93780114,64.23509667)(516.92779053,64.30510061)
\curveto(516.90780117,65.13509577)(517.05780102,65.8050951)(517.37779053,66.31510061)
\curveto(517.68780039,66.82509408)(518.12779995,67.2050937)(518.69779053,67.45510061)
\curveto(518.81779926,67.5050934)(518.94279913,67.55009336)(519.07279053,67.59010061)
\curveto(519.20279887,67.63009328)(519.33779874,67.67509323)(519.47779053,67.72510061)
\curveto(519.55779852,67.74509316)(519.64279843,67.76009315)(519.73279053,67.77010061)
\lineto(519.97279053,67.83010061)
\curveto(520.08279799,67.86009305)(520.19279788,67.87509303)(520.30279053,67.87510061)
\curveto(520.41279766,67.88509302)(520.52279755,67.90009301)(520.63279053,67.92010061)
\curveto(520.68279739,67.94009297)(520.72779735,67.94509296)(520.76779053,67.93510061)
\curveto(520.80779727,67.93509297)(520.84779723,67.94009297)(520.88779053,67.95010061)
\curveto(520.93779714,67.96009295)(520.99279708,67.96009295)(521.05279053,67.95010061)
\curveto(521.10279697,67.95009296)(521.15279692,67.95509295)(521.20279053,67.96510061)
\lineto(521.33779053,67.96510061)
\curveto(521.39779668,67.98509292)(521.46779661,67.98509292)(521.54779053,67.96510061)
\curveto(521.61779646,67.95509295)(521.68279639,67.96009295)(521.74279053,67.98010061)
\curveto(521.7727963,67.99009292)(521.81279626,67.99509291)(521.86279053,67.99510061)
\lineto(521.98279053,67.99510061)
\lineto(522.44779053,67.99510061)
\moveto(524.77279053,66.45010061)
\curveto(524.45279362,66.55009436)(524.08779399,66.6100943)(523.67779053,66.63010061)
\curveto(523.26779481,66.65009426)(522.85779522,66.66009425)(522.44779053,66.66010061)
\curveto(522.01779606,66.66009425)(521.59779648,66.65009426)(521.18779053,66.63010061)
\curveto(520.7777973,66.6100943)(520.39279768,66.56509434)(520.03279053,66.49510061)
\curveto(519.6727984,66.42509448)(519.35279872,66.31509459)(519.07279053,66.16510061)
\curveto(518.78279929,66.02509488)(518.54779953,65.83009508)(518.36779053,65.58010061)
\curveto(518.25779982,65.42009549)(518.1777999,65.24009567)(518.12779053,65.04010061)
\curveto(518.06780001,64.84009607)(518.03780004,64.59509631)(518.03779053,64.30510061)
\curveto(518.05780002,64.28509662)(518.06780001,64.25009666)(518.06779053,64.20010061)
\curveto(518.05780002,64.15009676)(518.05780002,64.1100968)(518.06779053,64.08010061)
\curveto(518.08779999,64.00009691)(518.10779997,63.92509698)(518.12779053,63.85510061)
\curveto(518.13779994,63.79509711)(518.15779992,63.73009718)(518.18779053,63.66010061)
\curveto(518.30779977,63.39009752)(518.4777996,63.17009774)(518.69779053,63.00010061)
\curveto(518.90779917,62.84009807)(519.15279892,62.7050982)(519.43279053,62.59510061)
\curveto(519.54279853,62.54509836)(519.66279841,62.5050984)(519.79279053,62.47510061)
\curveto(519.91279816,62.45509845)(520.03779804,62.43009848)(520.16779053,62.40010061)
\curveto(520.21779786,62.38009853)(520.2727978,62.37009854)(520.33279053,62.37010061)
\curveto(520.38279769,62.37009854)(520.43279764,62.36509854)(520.48279053,62.35510061)
\curveto(520.5727975,62.34509856)(520.66779741,62.33509857)(520.76779053,62.32510061)
\curveto(520.85779722,62.31509859)(520.95279712,62.3050986)(521.05279053,62.29510061)
\curveto(521.13279694,62.29509861)(521.21779686,62.29009862)(521.30779053,62.28010061)
\lineto(521.54779053,62.28010061)
\lineto(521.72779053,62.28010061)
\curveto(521.75779632,62.27009864)(521.79279628,62.26509864)(521.83279053,62.26510061)
\lineto(521.96779053,62.26510061)
\lineto(522.41779053,62.26510061)
\curveto(522.49779558,62.26509864)(522.58279549,62.26009865)(522.67279053,62.25010061)
\curveto(522.75279532,62.25009866)(522.82779525,62.26009865)(522.89779053,62.28010061)
\lineto(523.16779053,62.28010061)
\curveto(523.18779489,62.28009863)(523.21779486,62.27509863)(523.25779053,62.26510061)
\curveto(523.28779479,62.26509864)(523.31279476,62.27009864)(523.33279053,62.28010061)
\curveto(523.43279464,62.29009862)(523.53279454,62.29509861)(523.63279053,62.29510061)
\curveto(523.72279435,62.3050986)(523.82279425,62.31509859)(523.93279053,62.32510061)
\curveto(524.05279402,62.35509855)(524.1777939,62.37009854)(524.30779053,62.37010061)
\curveto(524.42779365,62.38009853)(524.54279353,62.4050985)(524.65279053,62.44510061)
\curveto(524.95279312,62.52509838)(525.21779286,62.6100983)(525.44779053,62.70010061)
\curveto(525.6777924,62.80009811)(525.89279218,62.94509796)(526.09279053,63.13510061)
\curveto(526.29279178,63.34509756)(526.44279163,63.6100973)(526.54279053,63.93010061)
\curveto(526.56279151,63.97009694)(526.5727915,64.0050969)(526.57279053,64.03510061)
\curveto(526.56279151,64.07509683)(526.56779151,64.12009679)(526.58779053,64.17010061)
\curveto(526.59779148,64.2100967)(526.60779147,64.28009663)(526.61779053,64.38010061)
\curveto(526.62779145,64.49009642)(526.62279145,64.57509633)(526.60279053,64.63510061)
\curveto(526.58279149,64.7050962)(526.5727915,64.77509613)(526.57279053,64.84510061)
\curveto(526.56279151,64.91509599)(526.54779153,64.98009593)(526.52779053,65.04010061)
\curveto(526.46779161,65.24009567)(526.38279169,65.42009549)(526.27279053,65.58010061)
\curveto(526.25279182,65.6100953)(526.23279184,65.63509527)(526.21279053,65.65510061)
\lineto(526.15279053,65.71510061)
\curveto(526.13279194,65.75509515)(526.09279198,65.8050951)(526.03279053,65.86510061)
\curveto(525.89279218,65.96509494)(525.76279231,66.05009486)(525.64279053,66.12010061)
\curveto(525.52279255,66.19009472)(525.3777927,66.26009465)(525.20779053,66.33010061)
\curveto(525.13779294,66.36009455)(525.06779301,66.38009453)(524.99779053,66.39010061)
\curveto(524.92779315,66.4100945)(524.85279322,66.43009448)(524.77279053,66.45010061)
}
}
{
\newrgbcolor{curcolor}{0 0 0}
\pscustom[linestyle=none,fillstyle=solid,fillcolor=curcolor]
{
\newpath
\moveto(521.93779053,76.28470998)
\curveto(522.01779606,76.28470235)(522.09779598,76.28970234)(522.17779053,76.29970998)
\curveto(522.25779582,76.30970232)(522.33279574,76.30470233)(522.40279053,76.28470998)
\curveto(522.44279563,76.26470237)(522.48779559,76.25970237)(522.53779053,76.26970998)
\curveto(522.5777955,76.27970235)(522.61779546,76.27970235)(522.65779053,76.26970998)
\lineto(522.80779053,76.26970998)
\curveto(522.89779518,76.25970237)(522.98779509,76.25470238)(523.07779053,76.25470998)
\curveto(523.15779492,76.25470238)(523.23779484,76.24970238)(523.31779053,76.23970998)
\lineto(523.55779053,76.20970998)
\curveto(523.62779445,76.19970243)(523.70279437,76.18970244)(523.78279053,76.17970998)
\curveto(523.82279425,76.16970246)(523.86279421,76.16470247)(523.90279053,76.16470998)
\curveto(523.94279413,76.16470247)(523.98779409,76.15970247)(524.03779053,76.14970998)
\curveto(524.1777939,76.10970252)(524.31779376,76.07970255)(524.45779053,76.05970998)
\curveto(524.59779348,76.04970258)(524.73279334,76.01970261)(524.86279053,75.96970998)
\curveto(525.03279304,75.91970271)(525.19779288,75.86470277)(525.35779053,75.80470998)
\curveto(525.51779256,75.75470288)(525.6727924,75.69470294)(525.82279053,75.62470998)
\curveto(525.88279219,75.60470303)(525.94279213,75.57470306)(526.00279053,75.53470998)
\lineto(526.15279053,75.44470998)
\curveto(526.4727916,75.24470339)(526.73779134,75.0297036)(526.94779053,74.79970998)
\curveto(527.15779092,74.56970406)(527.33779074,74.27470436)(527.48779053,73.91470998)
\curveto(527.53779054,73.79470484)(527.5727905,73.66470497)(527.59279053,73.52470998)
\curveto(527.61279046,73.39470524)(527.63779044,73.25970537)(527.66779053,73.11970998)
\curveto(527.6777904,73.05970557)(527.68279039,72.99970563)(527.68279053,72.93970998)
\curveto(527.68279039,72.87970575)(527.68779039,72.81470582)(527.69779053,72.74470998)
\curveto(527.70779037,72.71470592)(527.70779037,72.66470597)(527.69779053,72.59470998)
\lineto(527.69779053,72.44470998)
\lineto(527.69779053,72.29470998)
\curveto(527.6777904,72.21470642)(527.66279041,72.1297065)(527.65279053,72.03970998)
\curveto(527.65279042,71.95970667)(527.64279043,71.88470675)(527.62279053,71.81470998)
\curveto(527.61279046,71.77470686)(527.60779047,71.73970689)(527.60779053,71.70970998)
\curveto(527.61779046,71.68970694)(527.61279046,71.66470697)(527.59279053,71.63470998)
\lineto(527.53279053,71.36470998)
\curveto(527.50279057,71.27470736)(527.4727906,71.18970744)(527.44279053,71.10970998)
\curveto(527.20279087,70.5297081)(526.83279124,70.09470854)(526.33279053,69.80470998)
\curveto(526.20279187,69.72470891)(526.06779201,69.65970897)(525.92779053,69.60970998)
\curveto(525.78779229,69.56970906)(525.63779244,69.52470911)(525.47779053,69.47470998)
\curveto(525.39779268,69.45470918)(525.31779276,69.44970918)(525.23779053,69.45970998)
\curveto(525.15779292,69.47970915)(525.10279297,69.51470912)(525.07279053,69.56470998)
\curveto(525.05279302,69.59470904)(525.03779304,69.64970898)(525.02779053,69.72970998)
\curveto(525.00779307,69.80970882)(524.99779308,69.89470874)(524.99779053,69.98470998)
\curveto(524.98779309,70.07470856)(524.98779309,70.15970847)(524.99779053,70.23970998)
\curveto(525.00779307,70.3297083)(525.01779306,70.39970823)(525.02779053,70.44970998)
\curveto(525.03779304,70.46970816)(525.05279302,70.49470814)(525.07279053,70.52470998)
\curveto(525.09279298,70.56470807)(525.11279296,70.59470804)(525.13279053,70.61470998)
\curveto(525.21279286,70.67470796)(525.30779277,70.71970791)(525.41779053,70.74970998)
\curveto(525.52779255,70.78970784)(525.62779245,70.8347078)(525.71779053,70.88470998)
\curveto(526.10779197,71.1347075)(526.3777917,71.50470713)(526.52779053,71.99470998)
\curveto(526.54779153,72.06470657)(526.56279151,72.1347065)(526.57279053,72.20470998)
\curveto(526.5727915,72.28470635)(526.58279149,72.36470627)(526.60279053,72.44470998)
\curveto(526.61279146,72.48470615)(526.61779146,72.53970609)(526.61779053,72.60970998)
\curveto(526.61779146,72.68970594)(526.61279146,72.74470589)(526.60279053,72.77470998)
\curveto(526.59279148,72.80470583)(526.58779149,72.8347058)(526.58779053,72.86470998)
\lineto(526.58779053,72.96970998)
\curveto(526.56779151,73.04970558)(526.54779153,73.12470551)(526.52779053,73.19470998)
\curveto(526.50779157,73.27470536)(526.48279159,73.34970528)(526.45279053,73.41970998)
\curveto(526.30279177,73.76970486)(526.08779199,74.03970459)(525.80779053,74.22970998)
\curveto(525.52779255,74.41970421)(525.20279287,74.57470406)(524.83279053,74.69470998)
\curveto(524.75279332,74.72470391)(524.6777934,74.74470389)(524.60779053,74.75470998)
\curveto(524.53779354,74.77470386)(524.46279361,74.79470384)(524.38279053,74.81470998)
\curveto(524.29279378,74.8347038)(524.19779388,74.84970378)(524.09779053,74.85970998)
\curveto(523.98779409,74.87970375)(523.88279419,74.89970373)(523.78279053,74.91970998)
\curveto(523.73279434,74.9297037)(523.68279439,74.9347037)(523.63279053,74.93470998)
\curveto(523.5727945,74.94470369)(523.51779456,74.94970368)(523.46779053,74.94970998)
\curveto(523.40779467,74.96970366)(523.33279474,74.97970365)(523.24279053,74.97970998)
\curveto(523.14279493,74.97970365)(523.06279501,74.96970366)(523.00279053,74.94970998)
\curveto(522.91279516,74.91970371)(522.8727952,74.86970376)(522.88279053,74.79970998)
\curveto(522.89279518,74.73970389)(522.92279515,74.68470395)(522.97279053,74.63470998)
\curveto(523.02279505,74.55470408)(523.08279499,74.48470415)(523.15279053,74.42470998)
\curveto(523.22279485,74.37470426)(523.28279479,74.30970432)(523.33279053,74.22970998)
\curveto(523.44279463,74.06970456)(523.54279453,73.90470473)(523.63279053,73.73470998)
\curveto(523.71279436,73.56470507)(523.78279429,73.36970526)(523.84279053,73.14970998)
\curveto(523.8727942,73.04970558)(523.88779419,72.94970568)(523.88779053,72.84970998)
\curveto(523.88779419,72.75970587)(523.89779418,72.65970597)(523.91779053,72.54970998)
\lineto(523.91779053,72.39970998)
\curveto(523.89779418,72.34970628)(523.89279418,72.29970633)(523.90279053,72.24970998)
\curveto(523.91279416,72.20970642)(523.91279416,72.16970646)(523.90279053,72.12970998)
\curveto(523.89279418,72.09970653)(523.88779419,72.05470658)(523.88779053,71.99470998)
\curveto(523.8777942,71.9347067)(523.86779421,71.86970676)(523.85779053,71.79970998)
\lineto(523.82779053,71.61970998)
\curveto(523.70779437,71.16970746)(523.54279453,70.78970784)(523.33279053,70.47970998)
\curveto(523.14279493,70.20970842)(522.91279516,69.97970865)(522.64279053,69.78970998)
\curveto(522.36279571,69.60970902)(522.04779603,69.46470917)(521.69779053,69.35470998)
\lineto(521.48779053,69.29470998)
\curveto(521.40779667,69.28470935)(521.32779675,69.26970936)(521.24779053,69.24970998)
\curveto(521.21779686,69.23970939)(521.18779689,69.2347094)(521.15779053,69.23470998)
\curveto(521.12779695,69.2347094)(521.09779698,69.2297094)(521.06779053,69.21970998)
\curveto(521.00779707,69.20970942)(520.94779713,69.20470943)(520.88779053,69.20470998)
\curveto(520.81779726,69.20470943)(520.75779732,69.19470944)(520.70779053,69.17470998)
\lineto(520.52779053,69.17470998)
\curveto(520.4777976,69.16470947)(520.40779767,69.15970947)(520.31779053,69.15970998)
\curveto(520.22779785,69.15970947)(520.15779792,69.16970946)(520.10779053,69.18970998)
\lineto(519.94279053,69.18970998)
\curveto(519.86279821,69.20970942)(519.78779829,69.21970941)(519.71779053,69.21970998)
\curveto(519.64779843,69.2297094)(519.5777985,69.24470939)(519.50779053,69.26470998)
\curveto(519.30779877,69.32470931)(519.11779896,69.38470925)(518.93779053,69.44470998)
\curveto(518.75779932,69.51470912)(518.58779949,69.60470903)(518.42779053,69.71470998)
\curveto(518.35779972,69.75470888)(518.29279978,69.79470884)(518.23279053,69.83470998)
\lineto(518.05279053,69.98470998)
\curveto(518.04280003,70.00470863)(518.02780005,70.02470861)(518.00779053,70.04470998)
\curveto(517.8778002,70.1347085)(517.76780031,70.24470839)(517.67779053,70.37470998)
\curveto(517.4778006,70.634708)(517.32280075,70.89970773)(517.21279053,71.16970998)
\curveto(517.1728009,71.24970738)(517.14280093,71.3297073)(517.12279053,71.40970998)
\curveto(517.09280098,71.49970713)(517.06780101,71.58970704)(517.04779053,71.67970998)
\curveto(517.01780106,71.77970685)(516.99780108,71.87970675)(516.98779053,71.97970998)
\curveto(516.9778011,72.07970655)(516.96280111,72.18470645)(516.94279053,72.29470998)
\curveto(516.93280114,72.32470631)(516.93280114,72.36470627)(516.94279053,72.41470998)
\curveto(516.95280112,72.47470616)(516.94780113,72.51470612)(516.92779053,72.53470998)
\curveto(516.90780117,73.25470538)(517.02280105,73.85470478)(517.27279053,74.33470998)
\curveto(517.52280055,74.81470382)(517.86280021,75.18970344)(518.29279053,75.45970998)
\curveto(518.43279964,75.54970308)(518.5777995,75.629703)(518.72779053,75.69970998)
\curveto(518.8777992,75.76970286)(519.03779904,75.83970279)(519.20779053,75.90970998)
\curveto(519.34779873,75.95970267)(519.49779858,75.99970263)(519.65779053,76.02970998)
\curveto(519.81779826,76.05970257)(519.9777981,76.09470254)(520.13779053,76.13470998)
\curveto(520.18779789,76.15470248)(520.24279783,76.16470247)(520.30279053,76.16470998)
\curveto(520.35279772,76.16470247)(520.40279767,76.16970246)(520.45279053,76.17970998)
\curveto(520.51279756,76.19970243)(520.5777975,76.20970242)(520.64779053,76.20970998)
\curveto(520.70779737,76.20970242)(520.76279731,76.21970241)(520.81279053,76.23970998)
\lineto(520.97779053,76.23970998)
\curveto(521.02779705,76.25970237)(521.077797,76.26470237)(521.12779053,76.25470998)
\curveto(521.1777969,76.24470239)(521.22779685,76.24970238)(521.27779053,76.26970998)
\curveto(521.29779678,76.26970236)(521.32279675,76.26470237)(521.35279053,76.25470998)
\curveto(521.38279669,76.25470238)(521.40779667,76.25970237)(521.42779053,76.26970998)
\curveto(521.45779662,76.27970235)(521.49279658,76.27970235)(521.53279053,76.26970998)
\curveto(521.5727965,76.26970236)(521.61279646,76.27470236)(521.65279053,76.28470998)
\curveto(521.69279638,76.29470234)(521.73779634,76.29470234)(521.78779053,76.28470998)
\lineto(521.93779053,76.28470998)
\moveto(520.63279053,74.78470998)
\curveto(520.58279749,74.79470384)(520.52279755,74.79970383)(520.45279053,74.79970998)
\curveto(520.38279769,74.79970383)(520.32279775,74.79470384)(520.27279053,74.78470998)
\curveto(520.22279785,74.77470386)(520.14779793,74.76970386)(520.04779053,74.76970998)
\curveto(519.96779811,74.74970388)(519.89279818,74.7297039)(519.82279053,74.70970998)
\curveto(519.75279832,74.69970393)(519.68279839,74.68470395)(519.61279053,74.66470998)
\curveto(519.18279889,74.52470411)(518.84779923,74.3297043)(518.60779053,74.07970998)
\curveto(518.36779971,73.83970479)(518.18779989,73.49470514)(518.06779053,73.04470998)
\curveto(518.04780003,72.95470568)(518.03780004,72.85470578)(518.03779053,72.74470998)
\lineto(518.03779053,72.41470998)
\curveto(518.05780002,72.39470624)(518.06780001,72.35970627)(518.06779053,72.30970998)
\curveto(518.05780002,72.25970637)(518.05780002,72.21470642)(518.06779053,72.17470998)
\curveto(518.08779999,72.09470654)(518.10779997,72.01970661)(518.12779053,71.94970998)
\lineto(518.18779053,71.73970998)
\curveto(518.31779976,71.44970718)(518.49779958,71.21970741)(518.72779053,71.04970998)
\curveto(518.94779913,70.87970775)(519.20779887,70.74470789)(519.50779053,70.64470998)
\curveto(519.59779848,70.61470802)(519.69279838,70.58970804)(519.79279053,70.56970998)
\curveto(519.88279819,70.55970807)(519.9777981,70.54470809)(520.07779053,70.52470998)
\lineto(520.21279053,70.52470998)
\curveto(520.32279775,70.49470814)(520.46279761,70.48470815)(520.63279053,70.49470998)
\curveto(520.79279728,70.51470812)(520.92279715,70.5347081)(521.02279053,70.55470998)
\curveto(521.08279699,70.57470806)(521.14279693,70.58970804)(521.20279053,70.59970998)
\curveto(521.25279682,70.60970802)(521.30279677,70.62470801)(521.35279053,70.64470998)
\curveto(521.55279652,70.72470791)(521.74279633,70.81970781)(521.92279053,70.92970998)
\curveto(522.10279597,71.04970758)(522.24779583,71.18970744)(522.35779053,71.34970998)
\curveto(522.40779567,71.39970723)(522.44779563,71.45470718)(522.47779053,71.51470998)
\curveto(522.50779557,71.57470706)(522.54279553,71.634707)(522.58279053,71.69470998)
\curveto(522.66279541,71.84470679)(522.72779535,72.0297066)(522.77779053,72.24970998)
\curveto(522.79779528,72.29970633)(522.80279527,72.33970629)(522.79279053,72.36970998)
\curveto(522.78279529,72.40970622)(522.78779529,72.45470618)(522.80779053,72.50470998)
\curveto(522.81779526,72.54470609)(522.82279525,72.59970603)(522.82279053,72.66970998)
\curveto(522.82279525,72.73970589)(522.81779526,72.79970583)(522.80779053,72.84970998)
\curveto(522.78779529,72.94970568)(522.7727953,73.04470559)(522.76279053,73.13470998)
\curveto(522.74279533,73.22470541)(522.71279536,73.31470532)(522.67279053,73.40470998)
\curveto(522.45279562,73.94470469)(522.05779602,74.33970429)(521.48779053,74.58970998)
\curveto(521.38779669,74.63970399)(521.28779679,74.67470396)(521.18779053,74.69470998)
\curveto(521.077797,74.71470392)(520.96779711,74.73970389)(520.85779053,74.76970998)
\curveto(520.75779732,74.76970386)(520.68279739,74.77470386)(520.63279053,74.78470998)
}
}
{
\newrgbcolor{curcolor}{0 0 0}
\pscustom[linestyle=none,fillstyle=solid,fillcolor=curcolor]
{
\newpath
\moveto(525.89779053,78.63431936)
\lineto(525.89779053,79.26431936)
\lineto(525.89779053,79.45931936)
\curveto(525.89779218,79.52931683)(525.90779217,79.58931677)(525.92779053,79.63931936)
\curveto(525.96779211,79.70931665)(526.00779207,79.7593166)(526.04779053,79.78931936)
\curveto(526.09779198,79.82931653)(526.16279191,79.84931651)(526.24279053,79.84931936)
\curveto(526.32279175,79.8593165)(526.40779167,79.86431649)(526.49779053,79.86431936)
\lineto(527.21779053,79.86431936)
\curveto(527.69779038,79.86431649)(528.10778997,79.80431655)(528.44779053,79.68431936)
\curveto(528.78778929,79.56431679)(529.06278901,79.36931699)(529.27279053,79.09931936)
\curveto(529.32278875,79.02931733)(529.36778871,78.9593174)(529.40779053,78.88931936)
\curveto(529.45778862,78.82931753)(529.50278857,78.7543176)(529.54279053,78.66431936)
\curveto(529.55278852,78.64431771)(529.56278851,78.61431774)(529.57279053,78.57431936)
\curveto(529.59278848,78.53431782)(529.59778848,78.48931787)(529.58779053,78.43931936)
\curveto(529.55778852,78.34931801)(529.48278859,78.29431806)(529.36279053,78.27431936)
\curveto(529.25278882,78.2543181)(529.15778892,78.26931809)(529.07779053,78.31931936)
\curveto(529.00778907,78.34931801)(528.94278913,78.39431796)(528.88279053,78.45431936)
\curveto(528.83278924,78.52431783)(528.78278929,78.58931777)(528.73279053,78.64931936)
\curveto(528.68278939,78.71931764)(528.60778947,78.77931758)(528.50779053,78.82931936)
\curveto(528.41778966,78.88931747)(528.32778975,78.93931742)(528.23779053,78.97931936)
\curveto(528.20778987,78.99931736)(528.14778993,79.02431733)(528.05779053,79.05431936)
\curveto(527.9777901,79.08431727)(527.90779017,79.08931727)(527.84779053,79.06931936)
\curveto(527.70779037,79.03931732)(527.61779046,78.97931738)(527.57779053,78.88931936)
\curveto(527.54779053,78.80931755)(527.53279054,78.71931764)(527.53279053,78.61931936)
\curveto(527.53279054,78.51931784)(527.50779057,78.43431792)(527.45779053,78.36431936)
\curveto(527.38779069,78.27431808)(527.24779083,78.22931813)(527.03779053,78.22931936)
\lineto(526.48279053,78.22931936)
\lineto(526.25779053,78.22931936)
\curveto(526.1777919,78.23931812)(526.11279196,78.2593181)(526.06279053,78.28931936)
\curveto(525.98279209,78.34931801)(525.93779214,78.41931794)(525.92779053,78.49931936)
\curveto(525.91779216,78.51931784)(525.91279216,78.53931782)(525.91279053,78.55931936)
\curveto(525.91279216,78.58931777)(525.90779217,78.61431774)(525.89779053,78.63431936)
}
}
{
\newrgbcolor{curcolor}{0 0 0}
\pscustom[linestyle=none,fillstyle=solid,fillcolor=curcolor]
{
}
}
{
\newrgbcolor{curcolor}{0 0 0}
\pscustom[linestyle=none,fillstyle=solid,fillcolor=curcolor]
{
\newpath
\moveto(516.92779053,89.26463186)
\curveto(516.91780116,89.95462722)(517.03780104,90.55462662)(517.28779053,91.06463186)
\curveto(517.53780054,91.58462559)(517.8728002,91.9796252)(518.29279053,92.24963186)
\curveto(518.3727997,92.29962488)(518.46279961,92.34462483)(518.56279053,92.38463186)
\curveto(518.65279942,92.42462475)(518.74779933,92.46962471)(518.84779053,92.51963186)
\curveto(518.94779913,92.55962462)(519.04779903,92.58962459)(519.14779053,92.60963186)
\curveto(519.24779883,92.62962455)(519.35279872,92.64962453)(519.46279053,92.66963186)
\curveto(519.51279856,92.68962449)(519.55779852,92.69462448)(519.59779053,92.68463186)
\curveto(519.63779844,92.6746245)(519.68279839,92.6796245)(519.73279053,92.69963186)
\curveto(519.78279829,92.70962447)(519.86779821,92.71462446)(519.98779053,92.71463186)
\curveto(520.09779798,92.71462446)(520.18279789,92.70962447)(520.24279053,92.69963186)
\curveto(520.30279777,92.6796245)(520.36279771,92.66962451)(520.42279053,92.66963186)
\curveto(520.48279759,92.6796245)(520.54279753,92.6746245)(520.60279053,92.65463186)
\curveto(520.74279733,92.61462456)(520.8777972,92.5796246)(521.00779053,92.54963186)
\curveto(521.13779694,92.51962466)(521.26279681,92.4796247)(521.38279053,92.42963186)
\curveto(521.52279655,92.36962481)(521.64779643,92.29962488)(521.75779053,92.21963186)
\curveto(521.86779621,92.14962503)(521.9777961,92.0746251)(522.08779053,91.99463186)
\lineto(522.14779053,91.93463186)
\curveto(522.16779591,91.92462525)(522.18779589,91.90962527)(522.20779053,91.88963186)
\curveto(522.36779571,91.76962541)(522.51279556,91.63462554)(522.64279053,91.48463186)
\curveto(522.7727953,91.33462584)(522.89779518,91.174626)(523.01779053,91.00463186)
\curveto(523.23779484,90.69462648)(523.44279463,90.39962678)(523.63279053,90.11963186)
\curveto(523.7727943,89.88962729)(523.90779417,89.65962752)(524.03779053,89.42963186)
\curveto(524.16779391,89.20962797)(524.30279377,88.98962819)(524.44279053,88.76963186)
\curveto(524.61279346,88.51962866)(524.79279328,88.2796289)(524.98279053,88.04963186)
\curveto(525.1727929,87.82962935)(525.39779268,87.63962954)(525.65779053,87.47963186)
\curveto(525.71779236,87.43962974)(525.7777923,87.40462977)(525.83779053,87.37463186)
\curveto(525.88779219,87.34462983)(525.95279212,87.31462986)(526.03279053,87.28463186)
\curveto(526.10279197,87.26462991)(526.16279191,87.25962992)(526.21279053,87.26963186)
\curveto(526.28279179,87.28962989)(526.33779174,87.32462985)(526.37779053,87.37463186)
\curveto(526.40779167,87.42462975)(526.42779165,87.48462969)(526.43779053,87.55463186)
\lineto(526.43779053,87.79463186)
\lineto(526.43779053,88.54463186)
\lineto(526.43779053,91.34963186)
\lineto(526.43779053,92.00963186)
\curveto(526.43779164,92.09962508)(526.44279163,92.18462499)(526.45279053,92.26463186)
\curveto(526.45279162,92.34462483)(526.4727916,92.40962477)(526.51279053,92.45963186)
\curveto(526.55279152,92.50962467)(526.62779145,92.54962463)(526.73779053,92.57963186)
\curveto(526.83779124,92.61962456)(526.93779114,92.62962455)(527.03779053,92.60963186)
\lineto(527.17279053,92.60963186)
\curveto(527.24279083,92.58962459)(527.30279077,92.56962461)(527.35279053,92.54963186)
\curveto(527.40279067,92.52962465)(527.44279063,92.49462468)(527.47279053,92.44463186)
\curveto(527.51279056,92.39462478)(527.53279054,92.32462485)(527.53279053,92.23463186)
\lineto(527.53279053,91.96463186)
\lineto(527.53279053,91.06463186)
\lineto(527.53279053,87.55463186)
\lineto(527.53279053,86.48963186)
\curveto(527.53279054,86.40963077)(527.53779054,86.31963086)(527.54779053,86.21963186)
\curveto(527.54779053,86.11963106)(527.53779054,86.03463114)(527.51779053,85.96463186)
\curveto(527.44779063,85.75463142)(527.26779081,85.68963149)(526.97779053,85.76963186)
\curveto(526.93779114,85.7796314)(526.90279117,85.7796314)(526.87279053,85.76963186)
\curveto(526.83279124,85.76963141)(526.78779129,85.7796314)(526.73779053,85.79963186)
\curveto(526.65779142,85.81963136)(526.5727915,85.83963134)(526.48279053,85.85963186)
\curveto(526.39279168,85.8796313)(526.30779177,85.90463127)(526.22779053,85.93463186)
\curveto(525.73779234,86.09463108)(525.32279275,86.29463088)(524.98279053,86.53463186)
\curveto(524.73279334,86.71463046)(524.50779357,86.91963026)(524.30779053,87.14963186)
\curveto(524.09779398,87.3796298)(523.90279417,87.61962956)(523.72279053,87.86963186)
\curveto(523.54279453,88.12962905)(523.3727947,88.39462878)(523.21279053,88.66463186)
\curveto(523.04279503,88.94462823)(522.86779521,89.21462796)(522.68779053,89.47463186)
\curveto(522.60779547,89.58462759)(522.53279554,89.68962749)(522.46279053,89.78963186)
\curveto(522.39279568,89.89962728)(522.31779576,90.00962717)(522.23779053,90.11963186)
\curveto(522.20779587,90.15962702)(522.1777959,90.19462698)(522.14779053,90.22463186)
\curveto(522.10779597,90.26462691)(522.077796,90.30462687)(522.05779053,90.34463186)
\curveto(521.94779613,90.48462669)(521.82279625,90.60962657)(521.68279053,90.71963186)
\curveto(521.65279642,90.73962644)(521.62779645,90.76462641)(521.60779053,90.79463186)
\curveto(521.5777965,90.82462635)(521.54779653,90.84962633)(521.51779053,90.86963186)
\curveto(521.41779666,90.94962623)(521.31779676,91.01462616)(521.21779053,91.06463186)
\curveto(521.11779696,91.12462605)(521.00779707,91.179626)(520.88779053,91.22963186)
\curveto(520.81779726,91.25962592)(520.74279733,91.2796259)(520.66279053,91.28963186)
\lineto(520.42279053,91.34963186)
\lineto(520.33279053,91.34963186)
\curveto(520.30279777,91.35962582)(520.2727978,91.36462581)(520.24279053,91.36463186)
\curveto(520.1727979,91.38462579)(520.077798,91.38962579)(519.95779053,91.37963186)
\curveto(519.82779825,91.3796258)(519.72779835,91.36962581)(519.65779053,91.34963186)
\curveto(519.5777985,91.32962585)(519.50279857,91.30962587)(519.43279053,91.28963186)
\curveto(519.35279872,91.2796259)(519.2727988,91.25962592)(519.19279053,91.22963186)
\curveto(518.95279912,91.11962606)(518.75279932,90.96962621)(518.59279053,90.77963186)
\curveto(518.42279965,90.59962658)(518.28279979,90.3796268)(518.17279053,90.11963186)
\curveto(518.15279992,90.04962713)(518.13779994,89.9796272)(518.12779053,89.90963186)
\curveto(518.10779997,89.83962734)(518.08779999,89.76462741)(518.06779053,89.68463186)
\curveto(518.04780003,89.60462757)(518.03780004,89.49462768)(518.03779053,89.35463186)
\curveto(518.03780004,89.22462795)(518.04780003,89.11962806)(518.06779053,89.03963186)
\curveto(518.0778,88.9796282)(518.08279999,88.92462825)(518.08279053,88.87463186)
\curveto(518.08279999,88.82462835)(518.09279998,88.7746284)(518.11279053,88.72463186)
\curveto(518.15279992,88.62462855)(518.19279988,88.52962865)(518.23279053,88.43963186)
\curveto(518.2727998,88.35962882)(518.31779976,88.2796289)(518.36779053,88.19963186)
\curveto(518.38779969,88.16962901)(518.41279966,88.13962904)(518.44279053,88.10963186)
\curveto(518.4727996,88.08962909)(518.49779958,88.06462911)(518.51779053,88.03463186)
\lineto(518.59279053,87.95963186)
\curveto(518.61279946,87.92962925)(518.63279944,87.90462927)(518.65279053,87.88463186)
\lineto(518.86279053,87.73463186)
\curveto(518.92279915,87.69462948)(518.98779909,87.64962953)(519.05779053,87.59963186)
\curveto(519.14779893,87.53962964)(519.25279882,87.48962969)(519.37279053,87.44963186)
\curveto(519.48279859,87.41962976)(519.59279848,87.38462979)(519.70279053,87.34463186)
\curveto(519.81279826,87.30462987)(519.95779812,87.2796299)(520.13779053,87.26963186)
\curveto(520.30779777,87.25962992)(520.43279764,87.22962995)(520.51279053,87.17963186)
\curveto(520.59279748,87.12963005)(520.63779744,87.05463012)(520.64779053,86.95463186)
\curveto(520.65779742,86.85463032)(520.66279741,86.74463043)(520.66279053,86.62463186)
\curveto(520.66279741,86.58463059)(520.66779741,86.54463063)(520.67779053,86.50463186)
\curveto(520.6777974,86.46463071)(520.6727974,86.42963075)(520.66279053,86.39963186)
\curveto(520.64279743,86.34963083)(520.63279744,86.29963088)(520.63279053,86.24963186)
\curveto(520.63279744,86.20963097)(520.62279745,86.16963101)(520.60279053,86.12963186)
\curveto(520.54279753,86.03963114)(520.40779767,85.99463118)(520.19779053,85.99463186)
\lineto(520.07779053,85.99463186)
\curveto(520.01779806,86.00463117)(519.95779812,86.00963117)(519.89779053,86.00963186)
\curveto(519.82779825,86.01963116)(519.76279831,86.02963115)(519.70279053,86.03963186)
\curveto(519.59279848,86.05963112)(519.49279858,86.0796311)(519.40279053,86.09963186)
\curveto(519.30279877,86.11963106)(519.20779887,86.14963103)(519.11779053,86.18963186)
\curveto(519.04779903,86.20963097)(518.98779909,86.22963095)(518.93779053,86.24963186)
\lineto(518.75779053,86.30963186)
\curveto(518.49779958,86.42963075)(518.25279982,86.58463059)(518.02279053,86.77463186)
\curveto(517.79280028,86.9746302)(517.60780047,87.18962999)(517.46779053,87.41963186)
\curveto(517.38780069,87.52962965)(517.32280075,87.64462953)(517.27279053,87.76463186)
\lineto(517.12279053,88.15463186)
\curveto(517.072801,88.26462891)(517.04280103,88.3796288)(517.03279053,88.49963186)
\curveto(517.01280106,88.61962856)(516.98780109,88.74462843)(516.95779053,88.87463186)
\curveto(516.95780112,88.94462823)(516.95780112,89.00962817)(516.95779053,89.06963186)
\curveto(516.94780113,89.12962805)(516.93780114,89.19462798)(516.92779053,89.26463186)
}
}
{
\newrgbcolor{curcolor}{0 0 0}
\pscustom[linestyle=none,fillstyle=solid,fillcolor=curcolor]
{
\newpath
\moveto(522.44779053,101.36424123)
\lineto(522.70279053,101.36424123)
\curveto(522.78279529,101.37423353)(522.85779522,101.36923353)(522.92779053,101.34924123)
\lineto(523.16779053,101.34924123)
\lineto(523.33279053,101.34924123)
\curveto(523.43279464,101.32923357)(523.53779454,101.31923358)(523.64779053,101.31924123)
\curveto(523.74779433,101.31923358)(523.84779423,101.30923359)(523.94779053,101.28924123)
\lineto(524.09779053,101.28924123)
\curveto(524.23779384,101.25923364)(524.3777937,101.23923366)(524.51779053,101.22924123)
\curveto(524.64779343,101.21923368)(524.7777933,101.19423371)(524.90779053,101.15424123)
\curveto(524.98779309,101.13423377)(525.072793,101.11423379)(525.16279053,101.09424123)
\lineto(525.40279053,101.03424123)
\lineto(525.70279053,100.91424123)
\curveto(525.79279228,100.88423402)(525.88279219,100.84923405)(525.97279053,100.80924123)
\curveto(526.19279188,100.70923419)(526.40779167,100.57423433)(526.61779053,100.40424123)
\curveto(526.82779125,100.24423466)(526.99779108,100.06923483)(527.12779053,99.87924123)
\curveto(527.16779091,99.82923507)(527.20779087,99.76923513)(527.24779053,99.69924123)
\curveto(527.2777908,99.63923526)(527.31279076,99.57923532)(527.35279053,99.51924123)
\curveto(527.40279067,99.43923546)(527.44279063,99.34423556)(527.47279053,99.23424123)
\curveto(527.50279057,99.12423578)(527.53279054,99.01923588)(527.56279053,98.91924123)
\curveto(527.60279047,98.80923609)(527.62779045,98.6992362)(527.63779053,98.58924123)
\curveto(527.64779043,98.47923642)(527.66279041,98.36423654)(527.68279053,98.24424123)
\curveto(527.69279038,98.2042367)(527.69279038,98.15923674)(527.68279053,98.10924123)
\curveto(527.68279039,98.06923683)(527.68779039,98.02923687)(527.69779053,97.98924123)
\curveto(527.70779037,97.94923695)(527.71279036,97.89423701)(527.71279053,97.82424123)
\curveto(527.71279036,97.75423715)(527.70779037,97.7042372)(527.69779053,97.67424123)
\curveto(527.6777904,97.62423728)(527.6727904,97.57923732)(527.68279053,97.53924123)
\curveto(527.69279038,97.4992374)(527.69279038,97.46423744)(527.68279053,97.43424123)
\lineto(527.68279053,97.34424123)
\curveto(527.66279041,97.28423762)(527.64779043,97.21923768)(527.63779053,97.14924123)
\curveto(527.63779044,97.08923781)(527.63279044,97.02423788)(527.62279053,96.95424123)
\curveto(527.5727905,96.78423812)(527.52279055,96.62423828)(527.47279053,96.47424123)
\curveto(527.42279065,96.32423858)(527.35779072,96.17923872)(527.27779053,96.03924123)
\curveto(527.23779084,95.98923891)(527.20779087,95.93423897)(527.18779053,95.87424123)
\curveto(527.15779092,95.82423908)(527.12279095,95.77423913)(527.08279053,95.72424123)
\curveto(526.90279117,95.48423942)(526.68279139,95.28423962)(526.42279053,95.12424123)
\curveto(526.16279191,94.96423994)(525.8777922,94.82424008)(525.56779053,94.70424123)
\curveto(525.42779265,94.64424026)(525.28779279,94.5992403)(525.14779053,94.56924123)
\curveto(524.99779308,94.53924036)(524.84279323,94.5042404)(524.68279053,94.46424123)
\curveto(524.5727935,94.44424046)(524.46279361,94.42924047)(524.35279053,94.41924123)
\curveto(524.24279383,94.40924049)(524.13279394,94.39424051)(524.02279053,94.37424123)
\curveto(523.98279409,94.36424054)(523.94279413,94.35924054)(523.90279053,94.35924123)
\curveto(523.86279421,94.36924053)(523.82279425,94.36924053)(523.78279053,94.35924123)
\curveto(523.73279434,94.34924055)(523.68279439,94.34424056)(523.63279053,94.34424123)
\lineto(523.46779053,94.34424123)
\curveto(523.41779466,94.32424058)(523.36779471,94.31924058)(523.31779053,94.32924123)
\curveto(523.25779482,94.33924056)(523.20279487,94.33924056)(523.15279053,94.32924123)
\curveto(523.11279496,94.31924058)(523.06779501,94.31924058)(523.01779053,94.32924123)
\curveto(522.96779511,94.33924056)(522.91779516,94.33424057)(522.86779053,94.31424123)
\curveto(522.79779528,94.29424061)(522.72279535,94.28924061)(522.64279053,94.29924123)
\curveto(522.55279552,94.30924059)(522.46779561,94.31424059)(522.38779053,94.31424123)
\curveto(522.29779578,94.31424059)(522.19779588,94.30924059)(522.08779053,94.29924123)
\curveto(521.96779611,94.28924061)(521.86779621,94.29424061)(521.78779053,94.31424123)
\lineto(521.50279053,94.31424123)
\lineto(520.87279053,94.35924123)
\curveto(520.7727973,94.36924053)(520.6777974,94.37924052)(520.58779053,94.38924123)
\lineto(520.28779053,94.41924123)
\curveto(520.23779784,94.43924046)(520.18779789,94.44424046)(520.13779053,94.43424123)
\curveto(520.077798,94.43424047)(520.02279805,94.44424046)(519.97279053,94.46424123)
\curveto(519.80279827,94.51424039)(519.63779844,94.55424035)(519.47779053,94.58424123)
\curveto(519.30779877,94.61424029)(519.14779893,94.66424024)(518.99779053,94.73424123)
\curveto(518.53779954,94.92423998)(518.16279991,95.14423976)(517.87279053,95.39424123)
\curveto(517.58280049,95.65423925)(517.33780074,96.01423889)(517.13779053,96.47424123)
\curveto(517.08780099,96.6042383)(517.05280102,96.73423817)(517.03279053,96.86424123)
\curveto(517.01280106,97.0042379)(516.98780109,97.14423776)(516.95779053,97.28424123)
\curveto(516.94780113,97.35423755)(516.94280113,97.41923748)(516.94279053,97.47924123)
\curveto(516.94280113,97.53923736)(516.93780114,97.6042373)(516.92779053,97.67424123)
\curveto(516.90780117,98.5042364)(517.05780102,99.17423573)(517.37779053,99.68424123)
\curveto(517.68780039,100.19423471)(518.12779995,100.57423433)(518.69779053,100.82424123)
\curveto(518.81779926,100.87423403)(518.94279913,100.91923398)(519.07279053,100.95924123)
\curveto(519.20279887,100.9992339)(519.33779874,101.04423386)(519.47779053,101.09424123)
\curveto(519.55779852,101.11423379)(519.64279843,101.12923377)(519.73279053,101.13924123)
\lineto(519.97279053,101.19924123)
\curveto(520.08279799,101.22923367)(520.19279788,101.24423366)(520.30279053,101.24424123)
\curveto(520.41279766,101.25423365)(520.52279755,101.26923363)(520.63279053,101.28924123)
\curveto(520.68279739,101.30923359)(520.72779735,101.31423359)(520.76779053,101.30424123)
\curveto(520.80779727,101.3042336)(520.84779723,101.30923359)(520.88779053,101.31924123)
\curveto(520.93779714,101.32923357)(520.99279708,101.32923357)(521.05279053,101.31924123)
\curveto(521.10279697,101.31923358)(521.15279692,101.32423358)(521.20279053,101.33424123)
\lineto(521.33779053,101.33424123)
\curveto(521.39779668,101.35423355)(521.46779661,101.35423355)(521.54779053,101.33424123)
\curveto(521.61779646,101.32423358)(521.68279639,101.32923357)(521.74279053,101.34924123)
\curveto(521.7727963,101.35923354)(521.81279626,101.36423354)(521.86279053,101.36424123)
\lineto(521.98279053,101.36424123)
\lineto(522.44779053,101.36424123)
\moveto(524.77279053,99.81924123)
\curveto(524.45279362,99.91923498)(524.08779399,99.97923492)(523.67779053,99.99924123)
\curveto(523.26779481,100.01923488)(522.85779522,100.02923487)(522.44779053,100.02924123)
\curveto(522.01779606,100.02923487)(521.59779648,100.01923488)(521.18779053,99.99924123)
\curveto(520.7777973,99.97923492)(520.39279768,99.93423497)(520.03279053,99.86424123)
\curveto(519.6727984,99.79423511)(519.35279872,99.68423522)(519.07279053,99.53424123)
\curveto(518.78279929,99.39423551)(518.54779953,99.1992357)(518.36779053,98.94924123)
\curveto(518.25779982,98.78923611)(518.1777999,98.60923629)(518.12779053,98.40924123)
\curveto(518.06780001,98.20923669)(518.03780004,97.96423694)(518.03779053,97.67424123)
\curveto(518.05780002,97.65423725)(518.06780001,97.61923728)(518.06779053,97.56924123)
\curveto(518.05780002,97.51923738)(518.05780002,97.47923742)(518.06779053,97.44924123)
\curveto(518.08779999,97.36923753)(518.10779997,97.29423761)(518.12779053,97.22424123)
\curveto(518.13779994,97.16423774)(518.15779992,97.0992378)(518.18779053,97.02924123)
\curveto(518.30779977,96.75923814)(518.4777996,96.53923836)(518.69779053,96.36924123)
\curveto(518.90779917,96.20923869)(519.15279892,96.07423883)(519.43279053,95.96424123)
\curveto(519.54279853,95.91423899)(519.66279841,95.87423903)(519.79279053,95.84424123)
\curveto(519.91279816,95.82423908)(520.03779804,95.7992391)(520.16779053,95.76924123)
\curveto(520.21779786,95.74923915)(520.2727978,95.73923916)(520.33279053,95.73924123)
\curveto(520.38279769,95.73923916)(520.43279764,95.73423917)(520.48279053,95.72424123)
\curveto(520.5727975,95.71423919)(520.66779741,95.7042392)(520.76779053,95.69424123)
\curveto(520.85779722,95.68423922)(520.95279712,95.67423923)(521.05279053,95.66424123)
\curveto(521.13279694,95.66423924)(521.21779686,95.65923924)(521.30779053,95.64924123)
\lineto(521.54779053,95.64924123)
\lineto(521.72779053,95.64924123)
\curveto(521.75779632,95.63923926)(521.79279628,95.63423927)(521.83279053,95.63424123)
\lineto(521.96779053,95.63424123)
\lineto(522.41779053,95.63424123)
\curveto(522.49779558,95.63423927)(522.58279549,95.62923927)(522.67279053,95.61924123)
\curveto(522.75279532,95.61923928)(522.82779525,95.62923927)(522.89779053,95.64924123)
\lineto(523.16779053,95.64924123)
\curveto(523.18779489,95.64923925)(523.21779486,95.64423926)(523.25779053,95.63424123)
\curveto(523.28779479,95.63423927)(523.31279476,95.63923926)(523.33279053,95.64924123)
\curveto(523.43279464,95.65923924)(523.53279454,95.66423924)(523.63279053,95.66424123)
\curveto(523.72279435,95.67423923)(523.82279425,95.68423922)(523.93279053,95.69424123)
\curveto(524.05279402,95.72423918)(524.1777939,95.73923916)(524.30779053,95.73924123)
\curveto(524.42779365,95.74923915)(524.54279353,95.77423913)(524.65279053,95.81424123)
\curveto(524.95279312,95.89423901)(525.21779286,95.97923892)(525.44779053,96.06924123)
\curveto(525.6777924,96.16923873)(525.89279218,96.31423859)(526.09279053,96.50424123)
\curveto(526.29279178,96.71423819)(526.44279163,96.97923792)(526.54279053,97.29924123)
\curveto(526.56279151,97.33923756)(526.5727915,97.37423753)(526.57279053,97.40424123)
\curveto(526.56279151,97.44423746)(526.56779151,97.48923741)(526.58779053,97.53924123)
\curveto(526.59779148,97.57923732)(526.60779147,97.64923725)(526.61779053,97.74924123)
\curveto(526.62779145,97.85923704)(526.62279145,97.94423696)(526.60279053,98.00424123)
\curveto(526.58279149,98.07423683)(526.5727915,98.14423676)(526.57279053,98.21424123)
\curveto(526.56279151,98.28423662)(526.54779153,98.34923655)(526.52779053,98.40924123)
\curveto(526.46779161,98.60923629)(526.38279169,98.78923611)(526.27279053,98.94924123)
\curveto(526.25279182,98.97923592)(526.23279184,99.0042359)(526.21279053,99.02424123)
\lineto(526.15279053,99.08424123)
\curveto(526.13279194,99.12423578)(526.09279198,99.17423573)(526.03279053,99.23424123)
\curveto(525.89279218,99.33423557)(525.76279231,99.41923548)(525.64279053,99.48924123)
\curveto(525.52279255,99.55923534)(525.3777927,99.62923527)(525.20779053,99.69924123)
\curveto(525.13779294,99.72923517)(525.06779301,99.74923515)(524.99779053,99.75924123)
\curveto(524.92779315,99.77923512)(524.85279322,99.7992351)(524.77279053,99.81924123)
}
}
{
\newrgbcolor{curcolor}{0 0 0}
\pscustom[linestyle=none,fillstyle=solid,fillcolor=curcolor]
{
\newpath
\moveto(516.92779053,106.77385061)
\curveto(516.92780115,106.87384575)(516.93780114,106.96884566)(516.95779053,107.05885061)
\curveto(516.96780111,107.14884548)(516.99780108,107.21384541)(517.04779053,107.25385061)
\curveto(517.12780095,107.31384531)(517.23280084,107.34384528)(517.36279053,107.34385061)
\lineto(517.75279053,107.34385061)
\lineto(519.25279053,107.34385061)
\lineto(525.64279053,107.34385061)
\lineto(526.81279053,107.34385061)
\lineto(527.12779053,107.34385061)
\curveto(527.22779085,107.35384527)(527.30779077,107.33884529)(527.36779053,107.29885061)
\curveto(527.44779063,107.24884538)(527.49779058,107.17384545)(527.51779053,107.07385061)
\curveto(527.52779055,106.98384564)(527.53279054,106.87384575)(527.53279053,106.74385061)
\lineto(527.53279053,106.51885061)
\curveto(527.51279056,106.43884619)(527.49779058,106.36884626)(527.48779053,106.30885061)
\curveto(527.46779061,106.24884638)(527.42779065,106.19884643)(527.36779053,106.15885061)
\curveto(527.30779077,106.11884651)(527.23279084,106.09884653)(527.14279053,106.09885061)
\lineto(526.84279053,106.09885061)
\lineto(525.74779053,106.09885061)
\lineto(520.40779053,106.09885061)
\curveto(520.31779776,106.07884655)(520.24279783,106.06384656)(520.18279053,106.05385061)
\curveto(520.11279796,106.05384657)(520.05279802,106.0238466)(520.00279053,105.96385061)
\curveto(519.95279812,105.89384673)(519.92779815,105.80384682)(519.92779053,105.69385061)
\curveto(519.91779816,105.59384703)(519.91279816,105.48384714)(519.91279053,105.36385061)
\lineto(519.91279053,104.22385061)
\lineto(519.91279053,103.72885061)
\curveto(519.90279817,103.56884906)(519.84279823,103.45884917)(519.73279053,103.39885061)
\curveto(519.70279837,103.37884925)(519.6727984,103.36884926)(519.64279053,103.36885061)
\curveto(519.60279847,103.36884926)(519.55779852,103.36384926)(519.50779053,103.35385061)
\curveto(519.38779869,103.33384929)(519.2777988,103.33884929)(519.17779053,103.36885061)
\curveto(519.077799,103.40884922)(519.00779907,103.46384916)(518.96779053,103.53385061)
\curveto(518.91779916,103.61384901)(518.89279918,103.73384889)(518.89279053,103.89385061)
\curveto(518.89279918,104.05384857)(518.8777992,104.18884844)(518.84779053,104.29885061)
\curveto(518.83779924,104.34884828)(518.83279924,104.40384822)(518.83279053,104.46385061)
\curveto(518.82279925,104.5238481)(518.80779927,104.58384804)(518.78779053,104.64385061)
\curveto(518.73779934,104.79384783)(518.68779939,104.93884769)(518.63779053,105.07885061)
\curveto(518.5777995,105.21884741)(518.50779957,105.35384727)(518.42779053,105.48385061)
\curveto(518.33779974,105.623847)(518.23279984,105.74384688)(518.11279053,105.84385061)
\curveto(517.99280008,105.94384668)(517.86280021,106.03884659)(517.72279053,106.12885061)
\curveto(517.62280045,106.18884644)(517.51280056,106.23384639)(517.39279053,106.26385061)
\curveto(517.2728008,106.30384632)(517.16780091,106.35384627)(517.07779053,106.41385061)
\curveto(517.01780106,106.46384616)(516.9778011,106.53384609)(516.95779053,106.62385061)
\curveto(516.94780113,106.64384598)(516.94280113,106.66884596)(516.94279053,106.69885061)
\curveto(516.94280113,106.7288459)(516.93780114,106.75384587)(516.92779053,106.77385061)
}
}
{
\newrgbcolor{curcolor}{0 0 0}
\pscustom[linestyle=none,fillstyle=solid,fillcolor=curcolor]
{
\newpath
\moveto(516.92779053,115.12345998)
\curveto(516.92780115,115.22345513)(516.93780114,115.31845503)(516.95779053,115.40845998)
\curveto(516.96780111,115.49845485)(516.99780108,115.56345479)(517.04779053,115.60345998)
\curveto(517.12780095,115.66345469)(517.23280084,115.69345466)(517.36279053,115.69345998)
\lineto(517.75279053,115.69345998)
\lineto(519.25279053,115.69345998)
\lineto(525.64279053,115.69345998)
\lineto(526.81279053,115.69345998)
\lineto(527.12779053,115.69345998)
\curveto(527.22779085,115.70345465)(527.30779077,115.68845466)(527.36779053,115.64845998)
\curveto(527.44779063,115.59845475)(527.49779058,115.52345483)(527.51779053,115.42345998)
\curveto(527.52779055,115.33345502)(527.53279054,115.22345513)(527.53279053,115.09345998)
\lineto(527.53279053,114.86845998)
\curveto(527.51279056,114.78845556)(527.49779058,114.71845563)(527.48779053,114.65845998)
\curveto(527.46779061,114.59845575)(527.42779065,114.5484558)(527.36779053,114.50845998)
\curveto(527.30779077,114.46845588)(527.23279084,114.4484559)(527.14279053,114.44845998)
\lineto(526.84279053,114.44845998)
\lineto(525.74779053,114.44845998)
\lineto(520.40779053,114.44845998)
\curveto(520.31779776,114.42845592)(520.24279783,114.41345594)(520.18279053,114.40345998)
\curveto(520.11279796,114.40345595)(520.05279802,114.37345598)(520.00279053,114.31345998)
\curveto(519.95279812,114.24345611)(519.92779815,114.1534562)(519.92779053,114.04345998)
\curveto(519.91779816,113.94345641)(519.91279816,113.83345652)(519.91279053,113.71345998)
\lineto(519.91279053,112.57345998)
\lineto(519.91279053,112.07845998)
\curveto(519.90279817,111.91845843)(519.84279823,111.80845854)(519.73279053,111.74845998)
\curveto(519.70279837,111.72845862)(519.6727984,111.71845863)(519.64279053,111.71845998)
\curveto(519.60279847,111.71845863)(519.55779852,111.71345864)(519.50779053,111.70345998)
\curveto(519.38779869,111.68345867)(519.2777988,111.68845866)(519.17779053,111.71845998)
\curveto(519.077799,111.75845859)(519.00779907,111.81345854)(518.96779053,111.88345998)
\curveto(518.91779916,111.96345839)(518.89279918,112.08345827)(518.89279053,112.24345998)
\curveto(518.89279918,112.40345795)(518.8777992,112.53845781)(518.84779053,112.64845998)
\curveto(518.83779924,112.69845765)(518.83279924,112.7534576)(518.83279053,112.81345998)
\curveto(518.82279925,112.87345748)(518.80779927,112.93345742)(518.78779053,112.99345998)
\curveto(518.73779934,113.14345721)(518.68779939,113.28845706)(518.63779053,113.42845998)
\curveto(518.5777995,113.56845678)(518.50779957,113.70345665)(518.42779053,113.83345998)
\curveto(518.33779974,113.97345638)(518.23279984,114.09345626)(518.11279053,114.19345998)
\curveto(517.99280008,114.29345606)(517.86280021,114.38845596)(517.72279053,114.47845998)
\curveto(517.62280045,114.53845581)(517.51280056,114.58345577)(517.39279053,114.61345998)
\curveto(517.2728008,114.6534557)(517.16780091,114.70345565)(517.07779053,114.76345998)
\curveto(517.01780106,114.81345554)(516.9778011,114.88345547)(516.95779053,114.97345998)
\curveto(516.94780113,114.99345536)(516.94280113,115.01845533)(516.94279053,115.04845998)
\curveto(516.94280113,115.07845527)(516.93780114,115.10345525)(516.92779053,115.12345998)
}
}
{
\newrgbcolor{curcolor}{0 0 0}
\pscustom[linestyle=none,fillstyle=solid,fillcolor=curcolor]
{
\newpath
\moveto(538.79907959,29.18119436)
\lineto(538.79907959,30.09619436)
\curveto(538.79909028,30.19619171)(538.79909028,30.29119161)(538.79907959,30.38119436)
\curveto(538.79909028,30.47119143)(538.81909026,30.54619136)(538.85907959,30.60619436)
\curveto(538.91909016,30.69619121)(538.99909008,30.75619115)(539.09907959,30.78619436)
\curveto(539.19908988,30.82619108)(539.30408978,30.87119103)(539.41407959,30.92119436)
\curveto(539.60408948,31.0011909)(539.79408929,31.07119083)(539.98407959,31.13119436)
\curveto(540.17408891,31.2011907)(540.36408872,31.27619063)(540.55407959,31.35619436)
\curveto(540.73408835,31.42619048)(540.91908816,31.49119041)(541.10907959,31.55119436)
\curveto(541.28908779,31.61119029)(541.46908761,31.68119022)(541.64907959,31.76119436)
\curveto(541.78908729,31.82119008)(541.93408715,31.87619003)(542.08407959,31.92619436)
\curveto(542.23408685,31.97618993)(542.3790867,32.03118987)(542.51907959,32.09119436)
\curveto(542.96908611,32.27118963)(543.42408566,32.44118946)(543.88407959,32.60119436)
\curveto(544.33408475,32.76118914)(544.7840843,32.93118897)(545.23407959,33.11119436)
\curveto(545.2840838,33.13118877)(545.33408375,33.14618876)(545.38407959,33.15619436)
\lineto(545.53407959,33.21619436)
\curveto(545.75408333,33.3061886)(545.9790831,33.39118851)(546.20907959,33.47119436)
\curveto(546.42908265,33.55118835)(546.64908243,33.63618827)(546.86907959,33.72619436)
\curveto(546.95908212,33.76618814)(547.06908201,33.8061881)(547.19907959,33.84619436)
\curveto(547.31908176,33.88618802)(547.38908169,33.95118795)(547.40907959,34.04119436)
\curveto(547.41908166,34.08118782)(547.41908166,34.11118779)(547.40907959,34.13119436)
\lineto(547.34907959,34.19119436)
\curveto(547.29908178,34.24118766)(547.24408184,34.27618763)(547.18407959,34.29619436)
\curveto(547.12408196,34.32618758)(547.05908202,34.35618755)(546.98907959,34.38619436)
\lineto(546.35907959,34.62619436)
\curveto(546.13908294,34.7061872)(545.92408316,34.78618712)(545.71407959,34.86619436)
\lineto(545.56407959,34.92619436)
\lineto(545.38407959,34.98619436)
\curveto(545.19408389,35.06618684)(545.00408408,35.13618677)(544.81407959,35.19619436)
\curveto(544.61408447,35.26618664)(544.41408467,35.34118656)(544.21407959,35.42119436)
\curveto(543.63408545,35.66118624)(543.04908603,35.88118602)(542.45907959,36.08119436)
\curveto(541.86908721,36.29118561)(541.2840878,36.51618539)(540.70407959,36.75619436)
\curveto(540.50408858,36.83618507)(540.29908878,36.91118499)(540.08907959,36.98119436)
\curveto(539.8790892,37.06118484)(539.67408941,37.14118476)(539.47407959,37.22119436)
\curveto(539.39408969,37.26118464)(539.29408979,37.29618461)(539.17407959,37.32619436)
\curveto(539.05409003,37.36618454)(538.96909011,37.42118448)(538.91907959,37.49119436)
\curveto(538.8790902,37.55118435)(538.84909023,37.62618428)(538.82907959,37.71619436)
\curveto(538.80909027,37.81618409)(538.79909028,37.92618398)(538.79907959,38.04619436)
\curveto(538.78909029,38.16618374)(538.78909029,38.28618362)(538.79907959,38.40619436)
\curveto(538.79909028,38.52618338)(538.79909028,38.63618327)(538.79907959,38.73619436)
\curveto(538.79909028,38.82618308)(538.79909028,38.91618299)(538.79907959,39.00619436)
\curveto(538.79909028,39.1061828)(538.81909026,39.18118272)(538.85907959,39.23119436)
\curveto(538.90909017,39.32118258)(538.99909008,39.37118253)(539.12907959,39.38119436)
\curveto(539.25908982,39.39118251)(539.39908968,39.39618251)(539.54907959,39.39619436)
\lineto(541.19907959,39.39619436)
\lineto(547.46907959,39.39619436)
\lineto(548.72907959,39.39619436)
\curveto(548.83908024,39.39618251)(548.94908013,39.39618251)(549.05907959,39.39619436)
\curveto(549.16907991,39.4061825)(549.25407983,39.38618252)(549.31407959,39.33619436)
\curveto(549.37407971,39.3061826)(549.41407967,39.26118264)(549.43407959,39.20119436)
\curveto(549.44407964,39.14118276)(549.45907962,39.07118283)(549.47907959,38.99119436)
\lineto(549.47907959,38.75119436)
\lineto(549.47907959,38.39119436)
\curveto(549.46907961,38.28118362)(549.42407966,38.2011837)(549.34407959,38.15119436)
\curveto(549.31407977,38.13118377)(549.2840798,38.11618379)(549.25407959,38.10619436)
\curveto(549.21407987,38.1061838)(549.16907991,38.09618381)(549.11907959,38.07619436)
\lineto(548.95407959,38.07619436)
\curveto(548.89408019,38.06618384)(548.82408026,38.06118384)(548.74407959,38.06119436)
\curveto(548.66408042,38.07118383)(548.58908049,38.07618383)(548.51907959,38.07619436)
\lineto(547.67907959,38.07619436)
\lineto(543.25407959,38.07619436)
\curveto(543.00408608,38.07618383)(542.75408633,38.07618383)(542.50407959,38.07619436)
\curveto(542.24408684,38.07618383)(541.99408709,38.07118383)(541.75407959,38.06119436)
\curveto(541.65408743,38.06118384)(541.54408754,38.05618385)(541.42407959,38.04619436)
\curveto(541.30408778,38.03618387)(541.24408784,37.98118392)(541.24407959,37.88119436)
\lineto(541.25907959,37.88119436)
\curveto(541.2790878,37.81118409)(541.34408774,37.75118415)(541.45407959,37.70119436)
\curveto(541.56408752,37.66118424)(541.65908742,37.62618428)(541.73907959,37.59619436)
\curveto(541.90908717,37.52618438)(542.084087,37.46118444)(542.26407959,37.40119436)
\curveto(542.43408665,37.34118456)(542.60408648,37.27118463)(542.77407959,37.19119436)
\curveto(542.82408626,37.17118473)(542.86908621,37.15618475)(542.90907959,37.14619436)
\curveto(542.94908613,37.13618477)(542.99408609,37.12118478)(543.04407959,37.10119436)
\curveto(543.22408586,37.02118488)(543.40908567,36.95118495)(543.59907959,36.89119436)
\curveto(543.7790853,36.84118506)(543.95908512,36.77618513)(544.13907959,36.69619436)
\curveto(544.28908479,36.62618528)(544.44408464,36.56618534)(544.60407959,36.51619436)
\curveto(544.75408433,36.46618544)(544.90408418,36.41118549)(545.05407959,36.35119436)
\curveto(545.52408356,36.15118575)(545.99908308,35.97118593)(546.47907959,35.81119436)
\curveto(546.94908213,35.65118625)(547.41408167,35.47618643)(547.87407959,35.28619436)
\curveto(548.05408103,35.2061867)(548.23408085,35.13618677)(548.41407959,35.07619436)
\curveto(548.59408049,35.01618689)(548.77408031,34.95118695)(548.95407959,34.88119436)
\curveto(549.06408002,34.83118707)(549.16907991,34.78118712)(549.26907959,34.73119436)
\curveto(549.35907972,34.69118721)(549.42407966,34.6061873)(549.46407959,34.47619436)
\curveto(549.47407961,34.45618745)(549.4790796,34.43118747)(549.47907959,34.40119436)
\curveto(549.46907961,34.38118752)(549.46907961,34.35618755)(549.47907959,34.32619436)
\curveto(549.48907959,34.29618761)(549.49407959,34.26118764)(549.49407959,34.22119436)
\curveto(549.4840796,34.18118772)(549.4790796,34.14118776)(549.47907959,34.10119436)
\lineto(549.47907959,33.80119436)
\curveto(549.4790796,33.7011882)(549.45407963,33.62118828)(549.40407959,33.56119436)
\curveto(549.35407973,33.48118842)(549.2840798,33.42118848)(549.19407959,33.38119436)
\curveto(549.09407999,33.35118855)(548.99408009,33.31118859)(548.89407959,33.26119436)
\curveto(548.69408039,33.18118872)(548.48908059,33.1011888)(548.27907959,33.02119436)
\curveto(548.05908102,32.95118895)(547.84908123,32.87618903)(547.64907959,32.79619436)
\curveto(547.46908161,32.71618919)(547.28908179,32.64618926)(547.10907959,32.58619436)
\curveto(546.91908216,32.53618937)(546.73408235,32.47118943)(546.55407959,32.39119436)
\curveto(545.99408309,32.16118974)(545.42908365,31.94618996)(544.85907959,31.74619436)
\curveto(544.28908479,31.54619036)(543.72408536,31.33119057)(543.16407959,31.10119436)
\lineto(542.53407959,30.86119436)
\curveto(542.31408677,30.79119111)(542.10408698,30.71619119)(541.90407959,30.63619436)
\curveto(541.79408729,30.58619132)(541.68908739,30.54119136)(541.58907959,30.50119436)
\curveto(541.4790876,30.47119143)(541.3840877,30.42119148)(541.30407959,30.35119436)
\curveto(541.2840878,30.34119156)(541.27408781,30.33119157)(541.27407959,30.32119436)
\lineto(541.24407959,30.29119436)
\lineto(541.24407959,30.21619436)
\lineto(541.27407959,30.18619436)
\curveto(541.27408781,30.17619173)(541.2790878,30.16619174)(541.28907959,30.15619436)
\curveto(541.33908774,30.13619177)(541.39408769,30.12619178)(541.45407959,30.12619436)
\curveto(541.51408757,30.12619178)(541.57408751,30.11619179)(541.63407959,30.09619436)
\lineto(541.79907959,30.09619436)
\curveto(541.85908722,30.07619183)(541.92408716,30.07119183)(541.99407959,30.08119436)
\curveto(542.06408702,30.09119181)(542.13408695,30.09619181)(542.20407959,30.09619436)
\lineto(543.01407959,30.09619436)
\lineto(547.57407959,30.09619436)
\lineto(548.75907959,30.09619436)
\curveto(548.86908021,30.09619181)(548.9790801,30.09119181)(549.08907959,30.08119436)
\curveto(549.19907988,30.08119182)(549.2840798,30.05619185)(549.34407959,30.00619436)
\curveto(549.42407966,29.95619195)(549.46907961,29.86619204)(549.47907959,29.73619436)
\lineto(549.47907959,29.34619436)
\lineto(549.47907959,29.15119436)
\curveto(549.4790796,29.1011928)(549.46907961,29.05119285)(549.44907959,29.00119436)
\curveto(549.40907967,28.87119303)(549.32407976,28.79619311)(549.19407959,28.77619436)
\curveto(549.06408002,28.76619314)(548.91408017,28.76119314)(548.74407959,28.76119436)
\lineto(547.00407959,28.76119436)
\lineto(541.00407959,28.76119436)
\lineto(539.59407959,28.76119436)
\curveto(539.4840896,28.76119314)(539.36908971,28.75619315)(539.24907959,28.74619436)
\curveto(539.12908995,28.74619316)(539.03409005,28.77119313)(538.96407959,28.82119436)
\curveto(538.90409018,28.86119304)(538.85409023,28.93619297)(538.81407959,29.04619436)
\curveto(538.80409028,29.06619284)(538.80409028,29.08619282)(538.81407959,29.10619436)
\curveto(538.81409027,29.13619277)(538.80909027,29.16119274)(538.79907959,29.18119436)
}
}
{
\newrgbcolor{curcolor}{0 0 0}
\pscustom[linestyle=none,fillstyle=solid,fillcolor=curcolor]
{
\newpath
\moveto(548.92407959,48.38330373)
\curveto(549.08408,48.4132959)(549.21907986,48.39829592)(549.32907959,48.33830373)
\curveto(549.42907965,48.27829604)(549.50407958,48.19829612)(549.55407959,48.09830373)
\curveto(549.57407951,48.04829627)(549.5840795,47.99329632)(549.58407959,47.93330373)
\curveto(549.5840795,47.88329643)(549.59407949,47.82829649)(549.61407959,47.76830373)
\curveto(549.66407942,47.54829677)(549.64907943,47.32829699)(549.56907959,47.10830373)
\curveto(549.49907958,46.89829742)(549.40907967,46.75329756)(549.29907959,46.67330373)
\curveto(549.22907985,46.62329769)(549.14907993,46.57829774)(549.05907959,46.53830373)
\curveto(548.95908012,46.49829782)(548.8790802,46.44829787)(548.81907959,46.38830373)
\curveto(548.79908028,46.36829795)(548.7790803,46.34329797)(548.75907959,46.31330373)
\curveto(548.73908034,46.29329802)(548.73408035,46.26329805)(548.74407959,46.22330373)
\curveto(548.77408031,46.1132982)(548.82908025,46.00829831)(548.90907959,45.90830373)
\curveto(548.98908009,45.8182985)(549.05908002,45.72829859)(549.11907959,45.63830373)
\curveto(549.19907988,45.50829881)(549.27407981,45.36829895)(549.34407959,45.21830373)
\curveto(549.40407968,45.06829925)(549.45907962,44.90829941)(549.50907959,44.73830373)
\curveto(549.53907954,44.63829968)(549.55907952,44.52829979)(549.56907959,44.40830373)
\curveto(549.5790795,44.29830002)(549.59407949,44.18830013)(549.61407959,44.07830373)
\curveto(549.62407946,44.02830029)(549.62907945,43.98330033)(549.62907959,43.94330373)
\lineto(549.62907959,43.83830373)
\curveto(549.64907943,43.72830059)(549.64907943,43.62330069)(549.62907959,43.52330373)
\lineto(549.62907959,43.38830373)
\curveto(549.61907946,43.33830098)(549.61407947,43.28830103)(549.61407959,43.23830373)
\curveto(549.61407947,43.18830113)(549.60407948,43.14330117)(549.58407959,43.10330373)
\curveto(549.57407951,43.06330125)(549.56907951,43.02830129)(549.56907959,42.99830373)
\curveto(549.5790795,42.97830134)(549.5790795,42.95330136)(549.56907959,42.92330373)
\lineto(549.50907959,42.68330373)
\curveto(549.49907958,42.60330171)(549.4790796,42.52830179)(549.44907959,42.45830373)
\curveto(549.31907976,42.15830216)(549.17407991,41.9133024)(549.01407959,41.72330373)
\curveto(548.84408024,41.54330277)(548.60908047,41.39330292)(548.30907959,41.27330373)
\curveto(548.08908099,41.18330313)(547.82408126,41.13830318)(547.51407959,41.13830373)
\lineto(547.19907959,41.13830373)
\curveto(547.14908193,41.14830317)(547.09908198,41.15330316)(547.04907959,41.15330373)
\lineto(546.86907959,41.18330373)
\lineto(546.53907959,41.30330373)
\curveto(546.42908265,41.34330297)(546.32908275,41.39330292)(546.23907959,41.45330373)
\curveto(545.94908313,41.63330268)(545.73408335,41.87830244)(545.59407959,42.18830373)
\curveto(545.45408363,42.49830182)(545.32908375,42.83830148)(545.21907959,43.20830373)
\curveto(545.1790839,43.34830097)(545.14908393,43.49330082)(545.12907959,43.64330373)
\curveto(545.10908397,43.79330052)(545.084084,43.94330037)(545.05407959,44.09330373)
\curveto(545.03408405,44.16330015)(545.02408406,44.22830009)(545.02407959,44.28830373)
\curveto(545.02408406,44.35829996)(545.01408407,44.43329988)(544.99407959,44.51330373)
\curveto(544.97408411,44.58329973)(544.96408412,44.65329966)(544.96407959,44.72330373)
\curveto(544.95408413,44.79329952)(544.93908414,44.86829945)(544.91907959,44.94830373)
\curveto(544.85908422,45.19829912)(544.80908427,45.43329888)(544.76907959,45.65330373)
\curveto(544.71908436,45.87329844)(544.60408448,46.04829827)(544.42407959,46.17830373)
\curveto(544.34408474,46.23829808)(544.24408484,46.28829803)(544.12407959,46.32830373)
\curveto(543.99408509,46.36829795)(543.85408523,46.36829795)(543.70407959,46.32830373)
\curveto(543.46408562,46.26829805)(543.27408581,46.17829814)(543.13407959,46.05830373)
\curveto(542.99408609,45.94829837)(542.8840862,45.78829853)(542.80407959,45.57830373)
\curveto(542.75408633,45.45829886)(542.71908636,45.313299)(542.69907959,45.14330373)
\curveto(542.6790864,44.98329933)(542.66908641,44.8132995)(542.66907959,44.63330373)
\curveto(542.66908641,44.45329986)(542.6790864,44.27830004)(542.69907959,44.10830373)
\curveto(542.71908636,43.93830038)(542.74908633,43.79330052)(542.78907959,43.67330373)
\curveto(542.84908623,43.50330081)(542.93408615,43.33830098)(543.04407959,43.17830373)
\curveto(543.10408598,43.09830122)(543.1840859,43.02330129)(543.28407959,42.95330373)
\curveto(543.37408571,42.89330142)(543.47408561,42.83830148)(543.58407959,42.78830373)
\curveto(543.66408542,42.75830156)(543.74908533,42.72830159)(543.83907959,42.69830373)
\curveto(543.92908515,42.67830164)(543.99908508,42.63330168)(544.04907959,42.56330373)
\curveto(544.079085,42.52330179)(544.10408498,42.45330186)(544.12407959,42.35330373)
\curveto(544.13408495,42.26330205)(544.13908494,42.16830215)(544.13907959,42.06830373)
\curveto(544.13908494,41.96830235)(544.13408495,41.86830245)(544.12407959,41.76830373)
\curveto(544.10408498,41.67830264)(544.079085,41.6133027)(544.04907959,41.57330373)
\curveto(544.01908506,41.53330278)(543.96908511,41.50330281)(543.89907959,41.48330373)
\curveto(543.82908525,41.46330285)(543.75408533,41.46330285)(543.67407959,41.48330373)
\curveto(543.54408554,41.5133028)(543.42408566,41.54330277)(543.31407959,41.57330373)
\curveto(543.19408589,41.6133027)(543.079086,41.65830266)(542.96907959,41.70830373)
\curveto(542.61908646,41.89830242)(542.34908673,42.13830218)(542.15907959,42.42830373)
\curveto(541.95908712,42.7183016)(541.79908728,43.07830124)(541.67907959,43.50830373)
\curveto(541.65908742,43.60830071)(541.64408744,43.70830061)(541.63407959,43.80830373)
\curveto(541.62408746,43.9183004)(541.60908747,44.02830029)(541.58907959,44.13830373)
\curveto(541.5790875,44.17830014)(541.5790875,44.24330007)(541.58907959,44.33330373)
\curveto(541.58908749,44.42329989)(541.5790875,44.47829984)(541.55907959,44.49830373)
\curveto(541.54908753,45.19829912)(541.62908745,45.80829851)(541.79907959,46.32830373)
\curveto(541.96908711,46.84829747)(542.29408679,47.2132971)(542.77407959,47.42330373)
\curveto(542.97408611,47.5132968)(543.20908587,47.56329675)(543.47907959,47.57330373)
\curveto(543.73908534,47.59329672)(544.01408507,47.60329671)(544.30407959,47.60330373)
\lineto(547.61907959,47.60330373)
\curveto(547.75908132,47.60329671)(547.89408119,47.60829671)(548.02407959,47.61830373)
\curveto(548.15408093,47.62829669)(548.25908082,47.65829666)(548.33907959,47.70830373)
\curveto(548.40908067,47.75829656)(548.45908062,47.82329649)(548.48907959,47.90330373)
\curveto(548.52908055,47.99329632)(548.55908052,48.07829624)(548.57907959,48.15830373)
\curveto(548.58908049,48.23829608)(548.63408045,48.29829602)(548.71407959,48.33830373)
\curveto(548.74408034,48.35829596)(548.77408031,48.36829595)(548.80407959,48.36830373)
\curveto(548.83408025,48.36829595)(548.87408021,48.37329594)(548.92407959,48.38330373)
\moveto(547.25907959,46.23830373)
\curveto(547.11908196,46.29829802)(546.95908212,46.32829799)(546.77907959,46.32830373)
\curveto(546.58908249,46.33829798)(546.39408269,46.34329797)(546.19407959,46.34330373)
\curveto(546.084083,46.34329797)(545.9840831,46.33829798)(545.89407959,46.32830373)
\curveto(545.80408328,46.318298)(545.73408335,46.27829804)(545.68407959,46.20830373)
\curveto(545.66408342,46.17829814)(545.65408343,46.10829821)(545.65407959,45.99830373)
\curveto(545.67408341,45.97829834)(545.6840834,45.94329837)(545.68407959,45.89330373)
\curveto(545.6840834,45.84329847)(545.69408339,45.79829852)(545.71407959,45.75830373)
\curveto(545.73408335,45.67829864)(545.75408333,45.58829873)(545.77407959,45.48830373)
\lineto(545.83407959,45.18830373)
\curveto(545.83408325,45.15829916)(545.83908324,45.12329919)(545.84907959,45.08330373)
\lineto(545.84907959,44.97830373)
\curveto(545.88908319,44.82829949)(545.91408317,44.66329965)(545.92407959,44.48330373)
\curveto(545.92408316,44.3133)(545.94408314,44.15330016)(545.98407959,44.00330373)
\curveto(546.00408308,43.92330039)(546.02408306,43.84830047)(546.04407959,43.77830373)
\curveto(546.05408303,43.7183006)(546.06908301,43.64830067)(546.08907959,43.56830373)
\curveto(546.13908294,43.40830091)(546.20408288,43.25830106)(546.28407959,43.11830373)
\curveto(546.35408273,42.97830134)(546.44408264,42.85830146)(546.55407959,42.75830373)
\curveto(546.66408242,42.65830166)(546.79908228,42.58330173)(546.95907959,42.53330373)
\curveto(547.10908197,42.48330183)(547.29408179,42.46330185)(547.51407959,42.47330373)
\curveto(547.61408147,42.47330184)(547.70908137,42.48830183)(547.79907959,42.51830373)
\curveto(547.8790812,42.55830176)(547.95408113,42.60330171)(548.02407959,42.65330373)
\curveto(548.13408095,42.73330158)(548.22908085,42.83830148)(548.30907959,42.96830373)
\curveto(548.3790807,43.09830122)(548.43908064,43.23830108)(548.48907959,43.38830373)
\curveto(548.49908058,43.43830088)(548.50408058,43.48830083)(548.50407959,43.53830373)
\curveto(548.50408058,43.58830073)(548.50908057,43.63830068)(548.51907959,43.68830373)
\curveto(548.53908054,43.75830056)(548.55408053,43.84330047)(548.56407959,43.94330373)
\curveto(548.56408052,44.05330026)(548.55408053,44.14330017)(548.53407959,44.21330373)
\curveto(548.51408057,44.27330004)(548.50908057,44.33329998)(548.51907959,44.39330373)
\curveto(548.51908056,44.45329986)(548.50908057,44.5132998)(548.48907959,44.57330373)
\curveto(548.46908061,44.65329966)(548.45408063,44.72829959)(548.44407959,44.79830373)
\curveto(548.43408065,44.87829944)(548.41408067,44.95329936)(548.38407959,45.02330373)
\curveto(548.26408082,45.313299)(548.11908096,45.55829876)(547.94907959,45.75830373)
\curveto(547.7790813,45.96829835)(547.54908153,46.12829819)(547.25907959,46.23830373)
}
}
{
\newrgbcolor{curcolor}{0 0 0}
\pscustom[linestyle=none,fillstyle=solid,fillcolor=curcolor]
{
\newpath
\moveto(541.75407959,49.26994436)
\lineto(541.75407959,49.71994436)
\curveto(541.74408734,49.88994311)(541.76408732,50.01494298)(541.81407959,50.09494436)
\curveto(541.86408722,50.17494282)(541.92908715,50.22994277)(542.00907959,50.25994436)
\curveto(542.08908699,50.2999427)(542.17408691,50.33994266)(542.26407959,50.37994436)
\curveto(542.39408669,50.42994257)(542.52408656,50.47494252)(542.65407959,50.51494436)
\curveto(542.7840863,50.55494244)(542.91408617,50.5999424)(543.04407959,50.64994436)
\curveto(543.16408592,50.6999423)(543.28908579,50.74494225)(543.41907959,50.78494436)
\curveto(543.53908554,50.82494217)(543.65908542,50.86994213)(543.77907959,50.91994436)
\curveto(543.88908519,50.96994203)(544.00408508,51.00994199)(544.12407959,51.03994436)
\curveto(544.24408484,51.06994193)(544.36408472,51.10994189)(544.48407959,51.15994436)
\curveto(544.77408431,51.27994172)(545.07408401,51.38994161)(545.38407959,51.48994436)
\curveto(545.69408339,51.58994141)(545.99408309,51.6999413)(546.28407959,51.81994436)
\curveto(546.32408276,51.83994116)(546.36408272,51.84994115)(546.40407959,51.84994436)
\curveto(546.43408265,51.84994115)(546.46408262,51.85994114)(546.49407959,51.87994436)
\curveto(546.63408245,51.93994106)(546.7790823,51.994941)(546.92907959,52.04494436)
\lineto(547.34907959,52.19494436)
\curveto(547.41908166,52.22494077)(547.49408159,52.25494074)(547.57407959,52.28494436)
\curveto(547.64408144,52.31494068)(547.68908139,52.36494063)(547.70907959,52.43494436)
\curveto(547.73908134,52.51494048)(547.71408137,52.57494042)(547.63407959,52.61494436)
\curveto(547.54408154,52.66494033)(547.47408161,52.6999403)(547.42407959,52.71994436)
\curveto(547.25408183,52.7999402)(547.07408201,52.86494013)(546.88407959,52.91494436)
\curveto(546.69408239,52.96494003)(546.50908257,53.02493997)(546.32907959,53.09494436)
\curveto(546.09908298,53.18493981)(545.86908321,53.26493973)(545.63907959,53.33494436)
\curveto(545.39908368,53.40493959)(545.16908391,53.48993951)(544.94907959,53.58994436)
\curveto(544.89908418,53.5999394)(544.83408425,53.61493938)(544.75407959,53.63494436)
\curveto(544.66408442,53.67493932)(544.57408451,53.70993929)(544.48407959,53.73994436)
\curveto(544.3840847,53.76993923)(544.29408479,53.7999392)(544.21407959,53.82994436)
\curveto(544.16408492,53.84993915)(544.11908496,53.86493913)(544.07907959,53.87494436)
\curveto(544.03908504,53.88493911)(543.99408509,53.8999391)(543.94407959,53.91994436)
\curveto(543.82408526,53.96993903)(543.70408538,54.00993899)(543.58407959,54.03994436)
\curveto(543.45408563,54.07993892)(543.32908575,54.12493887)(543.20907959,54.17494436)
\curveto(543.15908592,54.1949388)(543.11408597,54.20993879)(543.07407959,54.21994436)
\curveto(543.03408605,54.22993877)(542.98908609,54.24493875)(542.93907959,54.26494436)
\curveto(542.84908623,54.30493869)(542.75908632,54.33993866)(542.66907959,54.36994436)
\curveto(542.56908651,54.3999386)(542.47408661,54.42993857)(542.38407959,54.45994436)
\curveto(542.30408678,54.48993851)(542.22408686,54.51493848)(542.14407959,54.53494436)
\curveto(542.05408703,54.56493843)(541.9790871,54.60493839)(541.91907959,54.65494436)
\curveto(541.82908725,54.72493827)(541.7790873,54.81993818)(541.76907959,54.93994436)
\curveto(541.75908732,55.06993793)(541.75408733,55.20993779)(541.75407959,55.35994436)
\curveto(541.75408733,55.43993756)(541.75908732,55.51493748)(541.76907959,55.58494436)
\curveto(541.76908731,55.66493733)(541.7840873,55.72993727)(541.81407959,55.77994436)
\curveto(541.87408721,55.86993713)(541.96908711,55.8949371)(542.09907959,55.85494436)
\curveto(542.22908685,55.81493718)(542.32908675,55.77993722)(542.39907959,55.74994436)
\lineto(542.45907959,55.71994436)
\curveto(542.4790866,55.71993728)(542.49908658,55.71493728)(542.51907959,55.70494436)
\curveto(542.79908628,55.5949374)(543.084086,55.48493751)(543.37407959,55.37494436)
\lineto(544.21407959,55.04494436)
\curveto(544.29408479,55.01493798)(544.36908471,54.98993801)(544.43907959,54.96994436)
\curveto(544.49908458,54.94993805)(544.56408452,54.92493807)(544.63407959,54.89494436)
\curveto(544.83408425,54.81493818)(545.03908404,54.73493826)(545.24907959,54.65494436)
\curveto(545.44908363,54.58493841)(545.64908343,54.50993849)(545.84907959,54.42994436)
\curveto(546.53908254,54.13993886)(547.23408185,53.86993913)(547.93407959,53.61994436)
\curveto(548.63408045,53.36993963)(549.32907975,53.0999399)(550.01907959,52.80994436)
\lineto(550.16907959,52.74994436)
\curveto(550.22907885,52.73994026)(550.28907879,52.72494027)(550.34907959,52.70494436)
\curveto(550.71907836,52.54494045)(551.084078,52.37494062)(551.44407959,52.19494436)
\curveto(551.81407727,52.01494098)(552.09907698,51.76494123)(552.29907959,51.44494436)
\curveto(552.35907672,51.33494166)(552.40407668,51.22494177)(552.43407959,51.11494436)
\curveto(552.47407661,51.00494199)(552.50907657,50.87994212)(552.53907959,50.73994436)
\curveto(552.55907652,50.68994231)(552.56407652,50.63494236)(552.55407959,50.57494436)
\curveto(552.54407654,50.52494247)(552.54407654,50.46994253)(552.55407959,50.40994436)
\curveto(552.57407651,50.32994267)(552.57407651,50.24994275)(552.55407959,50.16994436)
\curveto(552.54407654,50.12994287)(552.53907654,50.07994292)(552.53907959,50.01994436)
\lineto(552.47907959,49.77994436)
\curveto(552.45907662,49.70994329)(552.41907666,49.65494334)(552.35907959,49.61494436)
\curveto(552.29907678,49.56494343)(552.22407686,49.53494346)(552.13407959,49.52494436)
\lineto(551.86407959,49.52494436)
\lineto(551.65407959,49.52494436)
\curveto(551.59407749,49.53494346)(551.54407754,49.55494344)(551.50407959,49.58494436)
\curveto(551.39407769,49.65494334)(551.36407772,49.77494322)(551.41407959,49.94494436)
\curveto(551.43407765,50.05494294)(551.44407764,50.17494282)(551.44407959,50.30494436)
\curveto(551.44407764,50.43494256)(551.42407766,50.54994245)(551.38407959,50.64994436)
\curveto(551.33407775,50.7999422)(551.25907782,50.91994208)(551.15907959,51.00994436)
\curveto(551.05907802,51.10994189)(550.94407814,51.1949418)(550.81407959,51.26494436)
\curveto(550.69407839,51.33494166)(550.56407852,51.3949416)(550.42407959,51.44494436)
\lineto(550.00407959,51.62494436)
\curveto(549.91407917,51.66494133)(549.80407928,51.70494129)(549.67407959,51.74494436)
\curveto(549.54407954,51.7949412)(549.40907967,51.7999412)(549.26907959,51.75994436)
\curveto(549.10907997,51.70994129)(548.95908012,51.65494134)(548.81907959,51.59494436)
\curveto(548.6790804,51.54494145)(548.53908054,51.48994151)(548.39907959,51.42994436)
\curveto(548.18908089,51.33994166)(547.9790811,51.25494174)(547.76907959,51.17494436)
\curveto(547.55908152,51.0949419)(547.35408173,51.01494198)(547.15407959,50.93494436)
\curveto(547.01408207,50.87494212)(546.8790822,50.81994218)(546.74907959,50.76994436)
\curveto(546.61908246,50.71994228)(546.4840826,50.66994233)(546.34407959,50.61994436)
\lineto(545.02407959,50.07994436)
\curveto(544.5840845,49.90994309)(544.14408494,49.73494326)(543.70407959,49.55494436)
\curveto(543.47408561,49.45494354)(543.25408583,49.36494363)(543.04407959,49.28494436)
\curveto(542.82408626,49.20494379)(542.60408648,49.11994388)(542.38407959,49.02994436)
\curveto(542.32408676,49.00994399)(542.24408684,48.97994402)(542.14407959,48.93994436)
\curveto(542.03408705,48.8999441)(541.94408714,48.90494409)(541.87407959,48.95494436)
\curveto(541.82408726,48.98494401)(541.78908729,49.04494395)(541.76907959,49.13494436)
\curveto(541.75908732,49.15494384)(541.75908732,49.17494382)(541.76907959,49.19494436)
\curveto(541.76908731,49.22494377)(541.76408732,49.24994375)(541.75407959,49.26994436)
}
}
{
\newrgbcolor{curcolor}{0 0 0}
\pscustom[linestyle=none,fillstyle=solid,fillcolor=curcolor]
{
}
}
{
\newrgbcolor{curcolor}{0 0 0}
\pscustom[linestyle=none,fillstyle=solid,fillcolor=curcolor]
{
\newpath
\moveto(538.87407959,65.05510061)
\curveto(538.87409021,65.15509575)(538.8840902,65.25009566)(538.90407959,65.34010061)
\curveto(538.91409017,65.43009548)(538.94409014,65.49509541)(538.99407959,65.53510061)
\curveto(539.07409001,65.59509531)(539.1790899,65.62509528)(539.30907959,65.62510061)
\lineto(539.69907959,65.62510061)
\lineto(541.19907959,65.62510061)
\lineto(547.58907959,65.62510061)
\lineto(548.75907959,65.62510061)
\lineto(549.07407959,65.62510061)
\curveto(549.17407991,65.63509527)(549.25407983,65.62009529)(549.31407959,65.58010061)
\curveto(549.39407969,65.53009538)(549.44407964,65.45509545)(549.46407959,65.35510061)
\curveto(549.47407961,65.26509564)(549.4790796,65.15509575)(549.47907959,65.02510061)
\lineto(549.47907959,64.80010061)
\curveto(549.45907962,64.72009619)(549.44407964,64.65009626)(549.43407959,64.59010061)
\curveto(549.41407967,64.53009638)(549.37407971,64.48009643)(549.31407959,64.44010061)
\curveto(549.25407983,64.40009651)(549.1790799,64.38009653)(549.08907959,64.38010061)
\lineto(548.78907959,64.38010061)
\lineto(547.69407959,64.38010061)
\lineto(542.35407959,64.38010061)
\curveto(542.26408682,64.36009655)(542.18908689,64.34509656)(542.12907959,64.33510061)
\curveto(542.05908702,64.33509657)(541.99908708,64.3050966)(541.94907959,64.24510061)
\curveto(541.89908718,64.17509673)(541.87408721,64.08509682)(541.87407959,63.97510061)
\curveto(541.86408722,63.87509703)(541.85908722,63.76509714)(541.85907959,63.64510061)
\lineto(541.85907959,62.50510061)
\lineto(541.85907959,62.01010061)
\curveto(541.84908723,61.85009906)(541.78908729,61.74009917)(541.67907959,61.68010061)
\curveto(541.64908743,61.66009925)(541.61908746,61.65009926)(541.58907959,61.65010061)
\curveto(541.54908753,61.65009926)(541.50408758,61.64509926)(541.45407959,61.63510061)
\curveto(541.33408775,61.61509929)(541.22408786,61.62009929)(541.12407959,61.65010061)
\curveto(541.02408806,61.69009922)(540.95408813,61.74509916)(540.91407959,61.81510061)
\curveto(540.86408822,61.89509901)(540.83908824,62.01509889)(540.83907959,62.17510061)
\curveto(540.83908824,62.33509857)(540.82408826,62.47009844)(540.79407959,62.58010061)
\curveto(540.7840883,62.63009828)(540.7790883,62.68509822)(540.77907959,62.74510061)
\curveto(540.76908831,62.8050981)(540.75408833,62.86509804)(540.73407959,62.92510061)
\curveto(540.6840884,63.07509783)(540.63408845,63.22009769)(540.58407959,63.36010061)
\curveto(540.52408856,63.50009741)(540.45408863,63.63509727)(540.37407959,63.76510061)
\curveto(540.2840888,63.905097)(540.1790889,64.02509688)(540.05907959,64.12510061)
\curveto(539.93908914,64.22509668)(539.80908927,64.32009659)(539.66907959,64.41010061)
\curveto(539.56908951,64.47009644)(539.45908962,64.51509639)(539.33907959,64.54510061)
\curveto(539.21908986,64.58509632)(539.11408997,64.63509627)(539.02407959,64.69510061)
\curveto(538.96409012,64.74509616)(538.92409016,64.81509609)(538.90407959,64.90510061)
\curveto(538.89409019,64.92509598)(538.88909019,64.95009596)(538.88907959,64.98010061)
\curveto(538.88909019,65.0100959)(538.8840902,65.03509587)(538.87407959,65.05510061)
}
}
{
\newrgbcolor{curcolor}{0 0 0}
\pscustom[linestyle=none,fillstyle=solid,fillcolor=curcolor]
{
\newpath
\moveto(538.87407959,72.45970998)
\curveto(538.84409024,74.08970454)(539.39908968,75.13970349)(540.53907959,75.60970998)
\curveto(540.76908831,75.70970292)(541.05908802,75.77470286)(541.40907959,75.80470998)
\curveto(541.74908733,75.84470279)(542.05908702,75.81970281)(542.33907959,75.72970998)
\curveto(542.59908648,75.63970299)(542.82408626,75.51970311)(543.01407959,75.36970998)
\curveto(543.05408603,75.34970328)(543.08908599,75.32470331)(543.11907959,75.29470998)
\curveto(543.13908594,75.26470337)(543.16408592,75.23970339)(543.19407959,75.21970998)
\lineto(543.31407959,75.12970998)
\curveto(543.34408574,75.09970353)(543.36908571,75.06470357)(543.38907959,75.02470998)
\curveto(543.43908564,74.97470366)(543.4840856,74.91970371)(543.52407959,74.85970998)
\curveto(543.56408552,74.80970382)(543.61408547,74.76470387)(543.67407959,74.72470998)
\curveto(543.71408537,74.68470395)(543.76408532,74.66970396)(543.82407959,74.67970998)
\curveto(543.87408521,74.68970394)(543.91908516,74.71970391)(543.95907959,74.76970998)
\curveto(543.99908508,74.81970381)(544.03908504,74.87470376)(544.07907959,74.93470998)
\curveto(544.10908497,75.00470363)(544.13908494,75.06970356)(544.16907959,75.12970998)
\curveto(544.19908488,75.18970344)(544.22908485,75.23970339)(544.25907959,75.27970998)
\curveto(544.4790846,75.59970303)(544.78908429,75.85470278)(545.18907959,76.04470998)
\curveto(545.2790838,76.08470255)(545.37408371,76.11470252)(545.47407959,76.13470998)
\curveto(545.56408352,76.16470247)(545.65408343,76.18970244)(545.74407959,76.20970998)
\curveto(545.79408329,76.21970241)(545.84408324,76.22470241)(545.89407959,76.22470998)
\curveto(545.93408315,76.2347024)(545.9790831,76.24470239)(546.02907959,76.25470998)
\curveto(546.079083,76.26470237)(546.12908295,76.26470237)(546.17907959,76.25470998)
\curveto(546.22908285,76.24470239)(546.2790828,76.24970238)(546.32907959,76.26970998)
\curveto(546.3790827,76.27970235)(546.43908264,76.28470235)(546.50907959,76.28470998)
\curveto(546.5790825,76.28470235)(546.63908244,76.27470236)(546.68907959,76.25470998)
\lineto(546.91407959,76.25470998)
\lineto(547.15407959,76.19470998)
\curveto(547.22408186,76.18470245)(547.29408179,76.16970246)(547.36407959,76.14970998)
\curveto(547.45408163,76.11970251)(547.53908154,76.08970254)(547.61907959,76.05970998)
\curveto(547.69908138,76.03970259)(547.7790813,76.00970262)(547.85907959,75.96970998)
\curveto(547.91908116,75.94970268)(547.9790811,75.91970271)(548.03907959,75.87970998)
\curveto(548.08908099,75.84970278)(548.13908094,75.81470282)(548.18907959,75.77470998)
\curveto(548.49908058,75.57470306)(548.75908032,75.32470331)(548.96907959,75.02470998)
\curveto(549.16907991,74.72470391)(549.33407975,74.37970425)(549.46407959,73.98970998)
\curveto(549.50407958,73.86970476)(549.52907955,73.73970489)(549.53907959,73.59970998)
\curveto(549.55907952,73.46970516)(549.5840795,73.3347053)(549.61407959,73.19470998)
\curveto(549.62407946,73.12470551)(549.62907945,73.05470558)(549.62907959,72.98470998)
\curveto(549.62907945,72.92470571)(549.63407945,72.85970577)(549.64407959,72.78970998)
\curveto(549.65407943,72.74970588)(549.65907942,72.68970594)(549.65907959,72.60970998)
\curveto(549.65907942,72.53970609)(549.65407943,72.48970614)(549.64407959,72.45970998)
\curveto(549.63407945,72.40970622)(549.62907945,72.36470627)(549.62907959,72.32470998)
\lineto(549.62907959,72.20470998)
\curveto(549.60907947,72.10470653)(549.59407949,72.00470663)(549.58407959,71.90470998)
\curveto(549.57407951,71.80470683)(549.55907952,71.70970692)(549.53907959,71.61970998)
\curveto(549.50907957,71.50970712)(549.4840796,71.39970723)(549.46407959,71.28970998)
\curveto(549.43407965,71.18970744)(549.39407969,71.08470755)(549.34407959,70.97470998)
\curveto(549.1840799,70.60470803)(548.9840801,70.28970834)(548.74407959,70.02970998)
\curveto(548.49408059,69.76970886)(548.1840809,69.55970907)(547.81407959,69.39970998)
\curveto(547.72408136,69.35970927)(547.62908145,69.32470931)(547.52907959,69.29470998)
\curveto(547.42908165,69.26470937)(547.32408176,69.2347094)(547.21407959,69.20470998)
\curveto(547.16408192,69.18470945)(547.11408197,69.17470946)(547.06407959,69.17470998)
\curveto(547.00408208,69.17470946)(546.94408214,69.16470947)(546.88407959,69.14470998)
\curveto(546.82408226,69.12470951)(546.74408234,69.11470952)(546.64407959,69.11470998)
\curveto(546.54408254,69.11470952)(546.46908261,69.1297095)(546.41907959,69.15970998)
\curveto(546.38908269,69.16970946)(546.36408272,69.18470945)(546.34407959,69.20470998)
\lineto(546.28407959,69.26470998)
\curveto(546.26408282,69.30470933)(546.24908283,69.36470927)(546.23907959,69.44470998)
\curveto(546.22908285,69.5347091)(546.22408286,69.62470901)(546.22407959,69.71470998)
\curveto(546.22408286,69.80470883)(546.22908285,69.88970874)(546.23907959,69.96970998)
\curveto(546.24908283,70.05970857)(546.25908282,70.12470851)(546.26907959,70.16470998)
\curveto(546.28908279,70.18470845)(546.30408278,70.20470843)(546.31407959,70.22470998)
\curveto(546.31408277,70.24470839)(546.32408276,70.26470837)(546.34407959,70.28470998)
\curveto(546.43408265,70.35470828)(546.54908253,70.39470824)(546.68907959,70.40470998)
\curveto(546.82908225,70.42470821)(546.95408213,70.45470818)(547.06407959,70.49470998)
\lineto(547.42407959,70.64470998)
\curveto(547.53408155,70.69470794)(547.63908144,70.75970787)(547.73907959,70.83970998)
\curveto(547.76908131,70.85970777)(547.79408129,70.87970775)(547.81407959,70.89970998)
\curveto(547.83408125,70.9297077)(547.85908122,70.95470768)(547.88907959,70.97470998)
\curveto(547.94908113,71.01470762)(547.99408109,71.04970758)(548.02407959,71.07970998)
\curveto(548.05408103,71.11970751)(548.084081,71.15470748)(548.11407959,71.18470998)
\curveto(548.14408094,71.22470741)(548.17408091,71.26970736)(548.20407959,71.31970998)
\curveto(548.26408082,71.40970722)(548.31408077,71.50470713)(548.35407959,71.60470998)
\lineto(548.47407959,71.93470998)
\curveto(548.52408056,72.08470655)(548.55408053,72.28470635)(548.56407959,72.53470998)
\curveto(548.57408051,72.78470585)(548.55408053,72.99470564)(548.50407959,73.16470998)
\curveto(548.4840806,73.24470539)(548.46908061,73.31470532)(548.45907959,73.37470998)
\lineto(548.39907959,73.58470998)
\curveto(548.2790808,73.86470477)(548.12908095,74.10470453)(547.94907959,74.30470998)
\curveto(547.76908131,74.51470412)(547.53908154,74.67970395)(547.25907959,74.79970998)
\curveto(547.18908189,74.8297038)(547.11908196,74.84970378)(547.04907959,74.85970998)
\lineto(546.80907959,74.91970998)
\curveto(546.66908241,74.95970367)(546.50908257,74.96970366)(546.32907959,74.94970998)
\curveto(546.13908294,74.9297037)(545.98908309,74.89970373)(545.87907959,74.85970998)
\curveto(545.49908358,74.7297039)(545.20908387,74.54470409)(545.00907959,74.30470998)
\curveto(544.80908427,74.07470456)(544.64908443,73.76470487)(544.52907959,73.37470998)
\curveto(544.49908458,73.26470537)(544.4790846,73.14470549)(544.46907959,73.01470998)
\curveto(544.45908462,72.89470574)(544.45408463,72.76970586)(544.45407959,72.63970998)
\curveto(544.45408463,72.47970615)(544.44908463,72.33970629)(544.43907959,72.21970998)
\curveto(544.42908465,72.09970653)(544.36908471,72.01470662)(544.25907959,71.96470998)
\curveto(544.22908485,71.94470669)(544.19408489,71.9347067)(544.15407959,71.93470998)
\lineto(544.01907959,71.93470998)
\curveto(543.91908516,71.92470671)(543.82408526,71.92470671)(543.73407959,71.93470998)
\curveto(543.64408544,71.95470668)(543.5790855,71.99470664)(543.53907959,72.05470998)
\curveto(543.50908557,72.09470654)(543.48908559,72.1347065)(543.47907959,72.17470998)
\curveto(543.46908561,72.22470641)(543.45908562,72.27970635)(543.44907959,72.33970998)
\curveto(543.43908564,72.35970627)(543.43908564,72.38470625)(543.44907959,72.41470998)
\curveto(543.44908563,72.44470619)(543.44408564,72.46970616)(543.43407959,72.48970998)
\lineto(543.43407959,72.62470998)
\curveto(543.41408567,72.7347059)(543.40408568,72.8347058)(543.40407959,72.92470998)
\curveto(543.39408569,73.02470561)(543.37408571,73.11970551)(543.34407959,73.20970998)
\curveto(543.23408585,73.5297051)(543.08908599,73.78470485)(542.90907959,73.97470998)
\curveto(542.72908635,74.16470447)(542.4790866,74.31470432)(542.15907959,74.42470998)
\curveto(542.05908702,74.45470418)(541.93408715,74.47470416)(541.78407959,74.48470998)
\curveto(541.62408746,74.50470413)(541.4790876,74.49970413)(541.34907959,74.46970998)
\curveto(541.2790878,74.44970418)(541.21408787,74.4297042)(541.15407959,74.40970998)
\curveto(541.084088,74.39970423)(541.01908806,74.37970425)(540.95907959,74.34970998)
\curveto(540.71908836,74.24970438)(540.52908855,74.10470453)(540.38907959,73.91470998)
\curveto(540.24908883,73.72470491)(540.13908894,73.49970513)(540.05907959,73.23970998)
\curveto(540.03908904,73.17970545)(540.02908905,73.11970551)(540.02907959,73.05970998)
\curveto(540.02908905,72.99970563)(540.01908906,72.9347057)(539.99907959,72.86470998)
\curveto(539.9790891,72.78470585)(539.96908911,72.68970594)(539.96907959,72.57970998)
\curveto(539.96908911,72.46970616)(539.9790891,72.37470626)(539.99907959,72.29470998)
\curveto(540.01908906,72.24470639)(540.02908905,72.19470644)(540.02907959,72.14470998)
\curveto(540.02908905,72.10470653)(540.03908904,72.05970657)(540.05907959,72.00970998)
\curveto(540.10908897,71.8297068)(540.1840889,71.65970697)(540.28407959,71.49970998)
\curveto(540.37408871,71.34970728)(540.48908859,71.21970741)(540.62907959,71.10970998)
\curveto(540.74908833,71.01970761)(540.8790882,70.93970769)(541.01907959,70.86970998)
\curveto(541.15908792,70.79970783)(541.31408777,70.7347079)(541.48407959,70.67470998)
\curveto(541.59408749,70.64470799)(541.71408737,70.62470801)(541.84407959,70.61470998)
\curveto(541.96408712,70.60470803)(542.06408702,70.56970806)(542.14407959,70.50970998)
\curveto(542.1840869,70.48970814)(542.22408686,70.4297082)(542.26407959,70.32970998)
\curveto(542.27408681,70.28970834)(542.2840868,70.2297084)(542.29407959,70.14970998)
\lineto(542.29407959,69.89470998)
\curveto(542.2840868,69.80470883)(542.27408681,69.71970891)(542.26407959,69.63970998)
\curveto(542.25408683,69.56970906)(542.23908684,69.51970911)(542.21907959,69.48970998)
\curveto(542.18908689,69.44970918)(542.13408695,69.41470922)(542.05407959,69.38470998)
\curveto(541.97408711,69.35470928)(541.88908719,69.34970928)(541.79907959,69.36970998)
\curveto(541.74908733,69.37970925)(541.69908738,69.38470925)(541.64907959,69.38470998)
\lineto(541.46907959,69.41470998)
\curveto(541.36908771,69.44470919)(541.26908781,69.46970916)(541.16907959,69.48970998)
\curveto(541.06908801,69.51970911)(540.9790881,69.55470908)(540.89907959,69.59470998)
\curveto(540.78908829,69.64470899)(540.6840884,69.68970894)(540.58407959,69.72970998)
\curveto(540.47408861,69.76970886)(540.36908871,69.81970881)(540.26907959,69.87970998)
\curveto(539.72908935,70.20970842)(539.33408975,70.67970795)(539.08407959,71.28970998)
\curveto(539.03409005,71.40970722)(538.99909008,71.5347071)(538.97907959,71.66470998)
\curveto(538.95909012,71.80470683)(538.93409015,71.94470669)(538.90407959,72.08470998)
\curveto(538.89409019,72.14470649)(538.88909019,72.20470643)(538.88907959,72.26470998)
\curveto(538.88909019,72.3347063)(538.8840902,72.39970623)(538.87407959,72.45970998)
}
}
{
\newrgbcolor{curcolor}{0 0 0}
\pscustom[linestyle=none,fillstyle=solid,fillcolor=curcolor]
{
\newpath
\moveto(547.84407959,78.63431936)
\lineto(547.84407959,79.26431936)
\lineto(547.84407959,79.45931936)
\curveto(547.84408124,79.52931683)(547.85408123,79.58931677)(547.87407959,79.63931936)
\curveto(547.91408117,79.70931665)(547.95408113,79.7593166)(547.99407959,79.78931936)
\curveto(548.04408104,79.82931653)(548.10908097,79.84931651)(548.18907959,79.84931936)
\curveto(548.26908081,79.8593165)(548.35408073,79.86431649)(548.44407959,79.86431936)
\lineto(549.16407959,79.86431936)
\curveto(549.64407944,79.86431649)(550.05407903,79.80431655)(550.39407959,79.68431936)
\curveto(550.73407835,79.56431679)(551.00907807,79.36931699)(551.21907959,79.09931936)
\curveto(551.26907781,79.02931733)(551.31407777,78.9593174)(551.35407959,78.88931936)
\curveto(551.40407768,78.82931753)(551.44907763,78.7543176)(551.48907959,78.66431936)
\curveto(551.49907758,78.64431771)(551.50907757,78.61431774)(551.51907959,78.57431936)
\curveto(551.53907754,78.53431782)(551.54407754,78.48931787)(551.53407959,78.43931936)
\curveto(551.50407758,78.34931801)(551.42907765,78.29431806)(551.30907959,78.27431936)
\curveto(551.19907788,78.2543181)(551.10407798,78.26931809)(551.02407959,78.31931936)
\curveto(550.95407813,78.34931801)(550.88907819,78.39431796)(550.82907959,78.45431936)
\curveto(550.7790783,78.52431783)(550.72907835,78.58931777)(550.67907959,78.64931936)
\curveto(550.62907845,78.71931764)(550.55407853,78.77931758)(550.45407959,78.82931936)
\curveto(550.36407872,78.88931747)(550.27407881,78.93931742)(550.18407959,78.97931936)
\curveto(550.15407893,78.99931736)(550.09407899,79.02431733)(550.00407959,79.05431936)
\curveto(549.92407916,79.08431727)(549.85407923,79.08931727)(549.79407959,79.06931936)
\curveto(549.65407943,79.03931732)(549.56407952,78.97931738)(549.52407959,78.88931936)
\curveto(549.49407959,78.80931755)(549.4790796,78.71931764)(549.47907959,78.61931936)
\curveto(549.4790796,78.51931784)(549.45407963,78.43431792)(549.40407959,78.36431936)
\curveto(549.33407975,78.27431808)(549.19407989,78.22931813)(548.98407959,78.22931936)
\lineto(548.42907959,78.22931936)
\lineto(548.20407959,78.22931936)
\curveto(548.12408096,78.23931812)(548.05908102,78.2593181)(548.00907959,78.28931936)
\curveto(547.92908115,78.34931801)(547.8840812,78.41931794)(547.87407959,78.49931936)
\curveto(547.86408122,78.51931784)(547.85908122,78.53931782)(547.85907959,78.55931936)
\curveto(547.85908122,78.58931777)(547.85408123,78.61431774)(547.84407959,78.63431936)
}
}
{
\newrgbcolor{curcolor}{0 0 0}
\pscustom[linestyle=none,fillstyle=solid,fillcolor=curcolor]
{
}
}
{
\newrgbcolor{curcolor}{0 0 0}
\pscustom[linestyle=none,fillstyle=solid,fillcolor=curcolor]
{
\newpath
\moveto(538.87407959,89.26463186)
\curveto(538.86409022,89.95462722)(538.9840901,90.55462662)(539.23407959,91.06463186)
\curveto(539.4840896,91.58462559)(539.81908926,91.9796252)(540.23907959,92.24963186)
\curveto(540.31908876,92.29962488)(540.40908867,92.34462483)(540.50907959,92.38463186)
\curveto(540.59908848,92.42462475)(540.69408839,92.46962471)(540.79407959,92.51963186)
\curveto(540.89408819,92.55962462)(540.99408809,92.58962459)(541.09407959,92.60963186)
\curveto(541.19408789,92.62962455)(541.29908778,92.64962453)(541.40907959,92.66963186)
\curveto(541.45908762,92.68962449)(541.50408758,92.69462448)(541.54407959,92.68463186)
\curveto(541.5840875,92.6746245)(541.62908745,92.6796245)(541.67907959,92.69963186)
\curveto(541.72908735,92.70962447)(541.81408727,92.71462446)(541.93407959,92.71463186)
\curveto(542.04408704,92.71462446)(542.12908695,92.70962447)(542.18907959,92.69963186)
\curveto(542.24908683,92.6796245)(542.30908677,92.66962451)(542.36907959,92.66963186)
\curveto(542.42908665,92.6796245)(542.48908659,92.6746245)(542.54907959,92.65463186)
\curveto(542.68908639,92.61462456)(542.82408626,92.5796246)(542.95407959,92.54963186)
\curveto(543.084086,92.51962466)(543.20908587,92.4796247)(543.32907959,92.42963186)
\curveto(543.46908561,92.36962481)(543.59408549,92.29962488)(543.70407959,92.21963186)
\curveto(543.81408527,92.14962503)(543.92408516,92.0746251)(544.03407959,91.99463186)
\lineto(544.09407959,91.93463186)
\curveto(544.11408497,91.92462525)(544.13408495,91.90962527)(544.15407959,91.88963186)
\curveto(544.31408477,91.76962541)(544.45908462,91.63462554)(544.58907959,91.48463186)
\curveto(544.71908436,91.33462584)(544.84408424,91.174626)(544.96407959,91.00463186)
\curveto(545.1840839,90.69462648)(545.38908369,90.39962678)(545.57907959,90.11963186)
\curveto(545.71908336,89.88962729)(545.85408323,89.65962752)(545.98407959,89.42963186)
\curveto(546.11408297,89.20962797)(546.24908283,88.98962819)(546.38907959,88.76963186)
\curveto(546.55908252,88.51962866)(546.73908234,88.2796289)(546.92907959,88.04963186)
\curveto(547.11908196,87.82962935)(547.34408174,87.63962954)(547.60407959,87.47963186)
\curveto(547.66408142,87.43962974)(547.72408136,87.40462977)(547.78407959,87.37463186)
\curveto(547.83408125,87.34462983)(547.89908118,87.31462986)(547.97907959,87.28463186)
\curveto(548.04908103,87.26462991)(548.10908097,87.25962992)(548.15907959,87.26963186)
\curveto(548.22908085,87.28962989)(548.2840808,87.32462985)(548.32407959,87.37463186)
\curveto(548.35408073,87.42462975)(548.37408071,87.48462969)(548.38407959,87.55463186)
\lineto(548.38407959,87.79463186)
\lineto(548.38407959,88.54463186)
\lineto(548.38407959,91.34963186)
\lineto(548.38407959,92.00963186)
\curveto(548.3840807,92.09962508)(548.38908069,92.18462499)(548.39907959,92.26463186)
\curveto(548.39908068,92.34462483)(548.41908066,92.40962477)(548.45907959,92.45963186)
\curveto(548.49908058,92.50962467)(548.57408051,92.54962463)(548.68407959,92.57963186)
\curveto(548.7840803,92.61962456)(548.8840802,92.62962455)(548.98407959,92.60963186)
\lineto(549.11907959,92.60963186)
\curveto(549.18907989,92.58962459)(549.24907983,92.56962461)(549.29907959,92.54963186)
\curveto(549.34907973,92.52962465)(549.38907969,92.49462468)(549.41907959,92.44463186)
\curveto(549.45907962,92.39462478)(549.4790796,92.32462485)(549.47907959,92.23463186)
\lineto(549.47907959,91.96463186)
\lineto(549.47907959,91.06463186)
\lineto(549.47907959,87.55463186)
\lineto(549.47907959,86.48963186)
\curveto(549.4790796,86.40963077)(549.4840796,86.31963086)(549.49407959,86.21963186)
\curveto(549.49407959,86.11963106)(549.4840796,86.03463114)(549.46407959,85.96463186)
\curveto(549.39407969,85.75463142)(549.21407987,85.68963149)(548.92407959,85.76963186)
\curveto(548.8840802,85.7796314)(548.84908023,85.7796314)(548.81907959,85.76963186)
\curveto(548.7790803,85.76963141)(548.73408035,85.7796314)(548.68407959,85.79963186)
\curveto(548.60408048,85.81963136)(548.51908056,85.83963134)(548.42907959,85.85963186)
\curveto(548.33908074,85.8796313)(548.25408083,85.90463127)(548.17407959,85.93463186)
\curveto(547.6840814,86.09463108)(547.26908181,86.29463088)(546.92907959,86.53463186)
\curveto(546.6790824,86.71463046)(546.45408263,86.91963026)(546.25407959,87.14963186)
\curveto(546.04408304,87.3796298)(545.84908323,87.61962956)(545.66907959,87.86963186)
\curveto(545.48908359,88.12962905)(545.31908376,88.39462878)(545.15907959,88.66463186)
\curveto(544.98908409,88.94462823)(544.81408427,89.21462796)(544.63407959,89.47463186)
\curveto(544.55408453,89.58462759)(544.4790846,89.68962749)(544.40907959,89.78963186)
\curveto(544.33908474,89.89962728)(544.26408482,90.00962717)(544.18407959,90.11963186)
\curveto(544.15408493,90.15962702)(544.12408496,90.19462698)(544.09407959,90.22463186)
\curveto(544.05408503,90.26462691)(544.02408506,90.30462687)(544.00407959,90.34463186)
\curveto(543.89408519,90.48462669)(543.76908531,90.60962657)(543.62907959,90.71963186)
\curveto(543.59908548,90.73962644)(543.57408551,90.76462641)(543.55407959,90.79463186)
\curveto(543.52408556,90.82462635)(543.49408559,90.84962633)(543.46407959,90.86963186)
\curveto(543.36408572,90.94962623)(543.26408582,91.01462616)(543.16407959,91.06463186)
\curveto(543.06408602,91.12462605)(542.95408613,91.179626)(542.83407959,91.22963186)
\curveto(542.76408632,91.25962592)(542.68908639,91.2796259)(542.60907959,91.28963186)
\lineto(542.36907959,91.34963186)
\lineto(542.27907959,91.34963186)
\curveto(542.24908683,91.35962582)(542.21908686,91.36462581)(542.18907959,91.36463186)
\curveto(542.11908696,91.38462579)(542.02408706,91.38962579)(541.90407959,91.37963186)
\curveto(541.77408731,91.3796258)(541.67408741,91.36962581)(541.60407959,91.34963186)
\curveto(541.52408756,91.32962585)(541.44908763,91.30962587)(541.37907959,91.28963186)
\curveto(541.29908778,91.2796259)(541.21908786,91.25962592)(541.13907959,91.22963186)
\curveto(540.89908818,91.11962606)(540.69908838,90.96962621)(540.53907959,90.77963186)
\curveto(540.36908871,90.59962658)(540.22908885,90.3796268)(540.11907959,90.11963186)
\curveto(540.09908898,90.04962713)(540.084089,89.9796272)(540.07407959,89.90963186)
\curveto(540.05408903,89.83962734)(540.03408905,89.76462741)(540.01407959,89.68463186)
\curveto(539.99408909,89.60462757)(539.9840891,89.49462768)(539.98407959,89.35463186)
\curveto(539.9840891,89.22462795)(539.99408909,89.11962806)(540.01407959,89.03963186)
\curveto(540.02408906,88.9796282)(540.02908905,88.92462825)(540.02907959,88.87463186)
\curveto(540.02908905,88.82462835)(540.03908904,88.7746284)(540.05907959,88.72463186)
\curveto(540.09908898,88.62462855)(540.13908894,88.52962865)(540.17907959,88.43963186)
\curveto(540.21908886,88.35962882)(540.26408882,88.2796289)(540.31407959,88.19963186)
\curveto(540.33408875,88.16962901)(540.35908872,88.13962904)(540.38907959,88.10963186)
\curveto(540.41908866,88.08962909)(540.44408864,88.06462911)(540.46407959,88.03463186)
\lineto(540.53907959,87.95963186)
\curveto(540.55908852,87.92962925)(540.5790885,87.90462927)(540.59907959,87.88463186)
\lineto(540.80907959,87.73463186)
\curveto(540.86908821,87.69462948)(540.93408815,87.64962953)(541.00407959,87.59963186)
\curveto(541.09408799,87.53962964)(541.19908788,87.48962969)(541.31907959,87.44963186)
\curveto(541.42908765,87.41962976)(541.53908754,87.38462979)(541.64907959,87.34463186)
\curveto(541.75908732,87.30462987)(541.90408718,87.2796299)(542.08407959,87.26963186)
\curveto(542.25408683,87.25962992)(542.3790867,87.22962995)(542.45907959,87.17963186)
\curveto(542.53908654,87.12963005)(542.5840865,87.05463012)(542.59407959,86.95463186)
\curveto(542.60408648,86.85463032)(542.60908647,86.74463043)(542.60907959,86.62463186)
\curveto(542.60908647,86.58463059)(542.61408647,86.54463063)(542.62407959,86.50463186)
\curveto(542.62408646,86.46463071)(542.61908646,86.42963075)(542.60907959,86.39963186)
\curveto(542.58908649,86.34963083)(542.5790865,86.29963088)(542.57907959,86.24963186)
\curveto(542.5790865,86.20963097)(542.56908651,86.16963101)(542.54907959,86.12963186)
\curveto(542.48908659,86.03963114)(542.35408673,85.99463118)(542.14407959,85.99463186)
\lineto(542.02407959,85.99463186)
\curveto(541.96408712,86.00463117)(541.90408718,86.00963117)(541.84407959,86.00963186)
\curveto(541.77408731,86.01963116)(541.70908737,86.02963115)(541.64907959,86.03963186)
\curveto(541.53908754,86.05963112)(541.43908764,86.0796311)(541.34907959,86.09963186)
\curveto(541.24908783,86.11963106)(541.15408793,86.14963103)(541.06407959,86.18963186)
\curveto(540.99408809,86.20963097)(540.93408815,86.22963095)(540.88407959,86.24963186)
\lineto(540.70407959,86.30963186)
\curveto(540.44408864,86.42963075)(540.19908888,86.58463059)(539.96907959,86.77463186)
\curveto(539.73908934,86.9746302)(539.55408953,87.18962999)(539.41407959,87.41963186)
\curveto(539.33408975,87.52962965)(539.26908981,87.64462953)(539.21907959,87.76463186)
\lineto(539.06907959,88.15463186)
\curveto(539.01909006,88.26462891)(538.98909009,88.3796288)(538.97907959,88.49963186)
\curveto(538.95909012,88.61962856)(538.93409015,88.74462843)(538.90407959,88.87463186)
\curveto(538.90409018,88.94462823)(538.90409018,89.00962817)(538.90407959,89.06963186)
\curveto(538.89409019,89.12962805)(538.8840902,89.19462798)(538.87407959,89.26463186)
}
}
{
\newrgbcolor{curcolor}{0 0 0}
\pscustom[linestyle=none,fillstyle=solid,fillcolor=curcolor]
{
\newpath
\moveto(544.39407959,101.36424123)
\lineto(544.64907959,101.36424123)
\curveto(544.72908435,101.37423353)(544.80408428,101.36923353)(544.87407959,101.34924123)
\lineto(545.11407959,101.34924123)
\lineto(545.27907959,101.34924123)
\curveto(545.3790837,101.32923357)(545.4840836,101.31923358)(545.59407959,101.31924123)
\curveto(545.69408339,101.31923358)(545.79408329,101.30923359)(545.89407959,101.28924123)
\lineto(546.04407959,101.28924123)
\curveto(546.1840829,101.25923364)(546.32408276,101.23923366)(546.46407959,101.22924123)
\curveto(546.59408249,101.21923368)(546.72408236,101.19423371)(546.85407959,101.15424123)
\curveto(546.93408215,101.13423377)(547.01908206,101.11423379)(547.10907959,101.09424123)
\lineto(547.34907959,101.03424123)
\lineto(547.64907959,100.91424123)
\curveto(547.73908134,100.88423402)(547.82908125,100.84923405)(547.91907959,100.80924123)
\curveto(548.13908094,100.70923419)(548.35408073,100.57423433)(548.56407959,100.40424123)
\curveto(548.77408031,100.24423466)(548.94408014,100.06923483)(549.07407959,99.87924123)
\curveto(549.11407997,99.82923507)(549.15407993,99.76923513)(549.19407959,99.69924123)
\curveto(549.22407986,99.63923526)(549.25907982,99.57923532)(549.29907959,99.51924123)
\curveto(549.34907973,99.43923546)(549.38907969,99.34423556)(549.41907959,99.23424123)
\curveto(549.44907963,99.12423578)(549.4790796,99.01923588)(549.50907959,98.91924123)
\curveto(549.54907953,98.80923609)(549.57407951,98.6992362)(549.58407959,98.58924123)
\curveto(549.59407949,98.47923642)(549.60907947,98.36423654)(549.62907959,98.24424123)
\curveto(549.63907944,98.2042367)(549.63907944,98.15923674)(549.62907959,98.10924123)
\curveto(549.62907945,98.06923683)(549.63407945,98.02923687)(549.64407959,97.98924123)
\curveto(549.65407943,97.94923695)(549.65907942,97.89423701)(549.65907959,97.82424123)
\curveto(549.65907942,97.75423715)(549.65407943,97.7042372)(549.64407959,97.67424123)
\curveto(549.62407946,97.62423728)(549.61907946,97.57923732)(549.62907959,97.53924123)
\curveto(549.63907944,97.4992374)(549.63907944,97.46423744)(549.62907959,97.43424123)
\lineto(549.62907959,97.34424123)
\curveto(549.60907947,97.28423762)(549.59407949,97.21923768)(549.58407959,97.14924123)
\curveto(549.5840795,97.08923781)(549.5790795,97.02423788)(549.56907959,96.95424123)
\curveto(549.51907956,96.78423812)(549.46907961,96.62423828)(549.41907959,96.47424123)
\curveto(549.36907971,96.32423858)(549.30407978,96.17923872)(549.22407959,96.03924123)
\curveto(549.1840799,95.98923891)(549.15407993,95.93423897)(549.13407959,95.87424123)
\curveto(549.10407998,95.82423908)(549.06908001,95.77423913)(549.02907959,95.72424123)
\curveto(548.84908023,95.48423942)(548.62908045,95.28423962)(548.36907959,95.12424123)
\curveto(548.10908097,94.96423994)(547.82408126,94.82424008)(547.51407959,94.70424123)
\curveto(547.37408171,94.64424026)(547.23408185,94.5992403)(547.09407959,94.56924123)
\curveto(546.94408214,94.53924036)(546.78908229,94.5042404)(546.62907959,94.46424123)
\curveto(546.51908256,94.44424046)(546.40908267,94.42924047)(546.29907959,94.41924123)
\curveto(546.18908289,94.40924049)(546.079083,94.39424051)(545.96907959,94.37424123)
\curveto(545.92908315,94.36424054)(545.88908319,94.35924054)(545.84907959,94.35924123)
\curveto(545.80908327,94.36924053)(545.76908331,94.36924053)(545.72907959,94.35924123)
\curveto(545.6790834,94.34924055)(545.62908345,94.34424056)(545.57907959,94.34424123)
\lineto(545.41407959,94.34424123)
\curveto(545.36408372,94.32424058)(545.31408377,94.31924058)(545.26407959,94.32924123)
\curveto(545.20408388,94.33924056)(545.14908393,94.33924056)(545.09907959,94.32924123)
\curveto(545.05908402,94.31924058)(545.01408407,94.31924058)(544.96407959,94.32924123)
\curveto(544.91408417,94.33924056)(544.86408422,94.33424057)(544.81407959,94.31424123)
\curveto(544.74408434,94.29424061)(544.66908441,94.28924061)(544.58907959,94.29924123)
\curveto(544.49908458,94.30924059)(544.41408467,94.31424059)(544.33407959,94.31424123)
\curveto(544.24408484,94.31424059)(544.14408494,94.30924059)(544.03407959,94.29924123)
\curveto(543.91408517,94.28924061)(543.81408527,94.29424061)(543.73407959,94.31424123)
\lineto(543.44907959,94.31424123)
\lineto(542.81907959,94.35924123)
\curveto(542.71908636,94.36924053)(542.62408646,94.37924052)(542.53407959,94.38924123)
\lineto(542.23407959,94.41924123)
\curveto(542.1840869,94.43924046)(542.13408695,94.44424046)(542.08407959,94.43424123)
\curveto(542.02408706,94.43424047)(541.96908711,94.44424046)(541.91907959,94.46424123)
\curveto(541.74908733,94.51424039)(541.5840875,94.55424035)(541.42407959,94.58424123)
\curveto(541.25408783,94.61424029)(541.09408799,94.66424024)(540.94407959,94.73424123)
\curveto(540.4840886,94.92423998)(540.10908897,95.14423976)(539.81907959,95.39424123)
\curveto(539.52908955,95.65423925)(539.2840898,96.01423889)(539.08407959,96.47424123)
\curveto(539.03409005,96.6042383)(538.99909008,96.73423817)(538.97907959,96.86424123)
\curveto(538.95909012,97.0042379)(538.93409015,97.14423776)(538.90407959,97.28424123)
\curveto(538.89409019,97.35423755)(538.88909019,97.41923748)(538.88907959,97.47924123)
\curveto(538.88909019,97.53923736)(538.8840902,97.6042373)(538.87407959,97.67424123)
\curveto(538.85409023,98.5042364)(539.00409008,99.17423573)(539.32407959,99.68424123)
\curveto(539.63408945,100.19423471)(540.07408901,100.57423433)(540.64407959,100.82424123)
\curveto(540.76408832,100.87423403)(540.88908819,100.91923398)(541.01907959,100.95924123)
\curveto(541.14908793,100.9992339)(541.2840878,101.04423386)(541.42407959,101.09424123)
\curveto(541.50408758,101.11423379)(541.58908749,101.12923377)(541.67907959,101.13924123)
\lineto(541.91907959,101.19924123)
\curveto(542.02908705,101.22923367)(542.13908694,101.24423366)(542.24907959,101.24424123)
\curveto(542.35908672,101.25423365)(542.46908661,101.26923363)(542.57907959,101.28924123)
\curveto(542.62908645,101.30923359)(542.67408641,101.31423359)(542.71407959,101.30424123)
\curveto(542.75408633,101.3042336)(542.79408629,101.30923359)(542.83407959,101.31924123)
\curveto(542.8840862,101.32923357)(542.93908614,101.32923357)(542.99907959,101.31924123)
\curveto(543.04908603,101.31923358)(543.09908598,101.32423358)(543.14907959,101.33424123)
\lineto(543.28407959,101.33424123)
\curveto(543.34408574,101.35423355)(543.41408567,101.35423355)(543.49407959,101.33424123)
\curveto(543.56408552,101.32423358)(543.62908545,101.32923357)(543.68907959,101.34924123)
\curveto(543.71908536,101.35923354)(543.75908532,101.36423354)(543.80907959,101.36424123)
\lineto(543.92907959,101.36424123)
\lineto(544.39407959,101.36424123)
\moveto(546.71907959,99.81924123)
\curveto(546.39908268,99.91923498)(546.03408305,99.97923492)(545.62407959,99.99924123)
\curveto(545.21408387,100.01923488)(544.80408428,100.02923487)(544.39407959,100.02924123)
\curveto(543.96408512,100.02923487)(543.54408554,100.01923488)(543.13407959,99.99924123)
\curveto(542.72408636,99.97923492)(542.33908674,99.93423497)(541.97907959,99.86424123)
\curveto(541.61908746,99.79423511)(541.29908778,99.68423522)(541.01907959,99.53424123)
\curveto(540.72908835,99.39423551)(540.49408859,99.1992357)(540.31407959,98.94924123)
\curveto(540.20408888,98.78923611)(540.12408896,98.60923629)(540.07407959,98.40924123)
\curveto(540.01408907,98.20923669)(539.9840891,97.96423694)(539.98407959,97.67424123)
\curveto(540.00408908,97.65423725)(540.01408907,97.61923728)(540.01407959,97.56924123)
\curveto(540.00408908,97.51923738)(540.00408908,97.47923742)(540.01407959,97.44924123)
\curveto(540.03408905,97.36923753)(540.05408903,97.29423761)(540.07407959,97.22424123)
\curveto(540.084089,97.16423774)(540.10408898,97.0992378)(540.13407959,97.02924123)
\curveto(540.25408883,96.75923814)(540.42408866,96.53923836)(540.64407959,96.36924123)
\curveto(540.85408823,96.20923869)(541.09908798,96.07423883)(541.37907959,95.96424123)
\curveto(541.48908759,95.91423899)(541.60908747,95.87423903)(541.73907959,95.84424123)
\curveto(541.85908722,95.82423908)(541.9840871,95.7992391)(542.11407959,95.76924123)
\curveto(542.16408692,95.74923915)(542.21908686,95.73923916)(542.27907959,95.73924123)
\curveto(542.32908675,95.73923916)(542.3790867,95.73423917)(542.42907959,95.72424123)
\curveto(542.51908656,95.71423919)(542.61408647,95.7042392)(542.71407959,95.69424123)
\curveto(542.80408628,95.68423922)(542.89908618,95.67423923)(542.99907959,95.66424123)
\curveto(543.079086,95.66423924)(543.16408592,95.65923924)(543.25407959,95.64924123)
\lineto(543.49407959,95.64924123)
\lineto(543.67407959,95.64924123)
\curveto(543.70408538,95.63923926)(543.73908534,95.63423927)(543.77907959,95.63424123)
\lineto(543.91407959,95.63424123)
\lineto(544.36407959,95.63424123)
\curveto(544.44408464,95.63423927)(544.52908455,95.62923927)(544.61907959,95.61924123)
\curveto(544.69908438,95.61923928)(544.77408431,95.62923927)(544.84407959,95.64924123)
\lineto(545.11407959,95.64924123)
\curveto(545.13408395,95.64923925)(545.16408392,95.64423926)(545.20407959,95.63424123)
\curveto(545.23408385,95.63423927)(545.25908382,95.63923926)(545.27907959,95.64924123)
\curveto(545.3790837,95.65923924)(545.4790836,95.66423924)(545.57907959,95.66424123)
\curveto(545.66908341,95.67423923)(545.76908331,95.68423922)(545.87907959,95.69424123)
\curveto(545.99908308,95.72423918)(546.12408296,95.73923916)(546.25407959,95.73924123)
\curveto(546.37408271,95.74923915)(546.48908259,95.77423913)(546.59907959,95.81424123)
\curveto(546.89908218,95.89423901)(547.16408192,95.97923892)(547.39407959,96.06924123)
\curveto(547.62408146,96.16923873)(547.83908124,96.31423859)(548.03907959,96.50424123)
\curveto(548.23908084,96.71423819)(548.38908069,96.97923792)(548.48907959,97.29924123)
\curveto(548.50908057,97.33923756)(548.51908056,97.37423753)(548.51907959,97.40424123)
\curveto(548.50908057,97.44423746)(548.51408057,97.48923741)(548.53407959,97.53924123)
\curveto(548.54408054,97.57923732)(548.55408053,97.64923725)(548.56407959,97.74924123)
\curveto(548.57408051,97.85923704)(548.56908051,97.94423696)(548.54907959,98.00424123)
\curveto(548.52908055,98.07423683)(548.51908056,98.14423676)(548.51907959,98.21424123)
\curveto(548.50908057,98.28423662)(548.49408059,98.34923655)(548.47407959,98.40924123)
\curveto(548.41408067,98.60923629)(548.32908075,98.78923611)(548.21907959,98.94924123)
\curveto(548.19908088,98.97923592)(548.1790809,99.0042359)(548.15907959,99.02424123)
\lineto(548.09907959,99.08424123)
\curveto(548.079081,99.12423578)(548.03908104,99.17423573)(547.97907959,99.23424123)
\curveto(547.83908124,99.33423557)(547.70908137,99.41923548)(547.58907959,99.48924123)
\curveto(547.46908161,99.55923534)(547.32408176,99.62923527)(547.15407959,99.69924123)
\curveto(547.084082,99.72923517)(547.01408207,99.74923515)(546.94407959,99.75924123)
\curveto(546.87408221,99.77923512)(546.79908228,99.7992351)(546.71907959,99.81924123)
}
}
{
\newrgbcolor{curcolor}{0 0 0}
\pscustom[linestyle=none,fillstyle=solid,fillcolor=curcolor]
{
\newpath
\moveto(538.87407959,106.77385061)
\curveto(538.87409021,106.87384575)(538.8840902,106.96884566)(538.90407959,107.05885061)
\curveto(538.91409017,107.14884548)(538.94409014,107.21384541)(538.99407959,107.25385061)
\curveto(539.07409001,107.31384531)(539.1790899,107.34384528)(539.30907959,107.34385061)
\lineto(539.69907959,107.34385061)
\lineto(541.19907959,107.34385061)
\lineto(547.58907959,107.34385061)
\lineto(548.75907959,107.34385061)
\lineto(549.07407959,107.34385061)
\curveto(549.17407991,107.35384527)(549.25407983,107.33884529)(549.31407959,107.29885061)
\curveto(549.39407969,107.24884538)(549.44407964,107.17384545)(549.46407959,107.07385061)
\curveto(549.47407961,106.98384564)(549.4790796,106.87384575)(549.47907959,106.74385061)
\lineto(549.47907959,106.51885061)
\curveto(549.45907962,106.43884619)(549.44407964,106.36884626)(549.43407959,106.30885061)
\curveto(549.41407967,106.24884638)(549.37407971,106.19884643)(549.31407959,106.15885061)
\curveto(549.25407983,106.11884651)(549.1790799,106.09884653)(549.08907959,106.09885061)
\lineto(548.78907959,106.09885061)
\lineto(547.69407959,106.09885061)
\lineto(542.35407959,106.09885061)
\curveto(542.26408682,106.07884655)(542.18908689,106.06384656)(542.12907959,106.05385061)
\curveto(542.05908702,106.05384657)(541.99908708,106.0238466)(541.94907959,105.96385061)
\curveto(541.89908718,105.89384673)(541.87408721,105.80384682)(541.87407959,105.69385061)
\curveto(541.86408722,105.59384703)(541.85908722,105.48384714)(541.85907959,105.36385061)
\lineto(541.85907959,104.22385061)
\lineto(541.85907959,103.72885061)
\curveto(541.84908723,103.56884906)(541.78908729,103.45884917)(541.67907959,103.39885061)
\curveto(541.64908743,103.37884925)(541.61908746,103.36884926)(541.58907959,103.36885061)
\curveto(541.54908753,103.36884926)(541.50408758,103.36384926)(541.45407959,103.35385061)
\curveto(541.33408775,103.33384929)(541.22408786,103.33884929)(541.12407959,103.36885061)
\curveto(541.02408806,103.40884922)(540.95408813,103.46384916)(540.91407959,103.53385061)
\curveto(540.86408822,103.61384901)(540.83908824,103.73384889)(540.83907959,103.89385061)
\curveto(540.83908824,104.05384857)(540.82408826,104.18884844)(540.79407959,104.29885061)
\curveto(540.7840883,104.34884828)(540.7790883,104.40384822)(540.77907959,104.46385061)
\curveto(540.76908831,104.5238481)(540.75408833,104.58384804)(540.73407959,104.64385061)
\curveto(540.6840884,104.79384783)(540.63408845,104.93884769)(540.58407959,105.07885061)
\curveto(540.52408856,105.21884741)(540.45408863,105.35384727)(540.37407959,105.48385061)
\curveto(540.2840888,105.623847)(540.1790889,105.74384688)(540.05907959,105.84385061)
\curveto(539.93908914,105.94384668)(539.80908927,106.03884659)(539.66907959,106.12885061)
\curveto(539.56908951,106.18884644)(539.45908962,106.23384639)(539.33907959,106.26385061)
\curveto(539.21908986,106.30384632)(539.11408997,106.35384627)(539.02407959,106.41385061)
\curveto(538.96409012,106.46384616)(538.92409016,106.53384609)(538.90407959,106.62385061)
\curveto(538.89409019,106.64384598)(538.88909019,106.66884596)(538.88907959,106.69885061)
\curveto(538.88909019,106.7288459)(538.8840902,106.75384587)(538.87407959,106.77385061)
}
}
{
\newrgbcolor{curcolor}{0 0 0}
\pscustom[linestyle=none,fillstyle=solid,fillcolor=curcolor]
{
\newpath
\moveto(538.87407959,115.12345998)
\curveto(538.87409021,115.22345513)(538.8840902,115.31845503)(538.90407959,115.40845998)
\curveto(538.91409017,115.49845485)(538.94409014,115.56345479)(538.99407959,115.60345998)
\curveto(539.07409001,115.66345469)(539.1790899,115.69345466)(539.30907959,115.69345998)
\lineto(539.69907959,115.69345998)
\lineto(541.19907959,115.69345998)
\lineto(547.58907959,115.69345998)
\lineto(548.75907959,115.69345998)
\lineto(549.07407959,115.69345998)
\curveto(549.17407991,115.70345465)(549.25407983,115.68845466)(549.31407959,115.64845998)
\curveto(549.39407969,115.59845475)(549.44407964,115.52345483)(549.46407959,115.42345998)
\curveto(549.47407961,115.33345502)(549.4790796,115.22345513)(549.47907959,115.09345998)
\lineto(549.47907959,114.86845998)
\curveto(549.45907962,114.78845556)(549.44407964,114.71845563)(549.43407959,114.65845998)
\curveto(549.41407967,114.59845575)(549.37407971,114.5484558)(549.31407959,114.50845998)
\curveto(549.25407983,114.46845588)(549.1790799,114.4484559)(549.08907959,114.44845998)
\lineto(548.78907959,114.44845998)
\lineto(547.69407959,114.44845998)
\lineto(542.35407959,114.44845998)
\curveto(542.26408682,114.42845592)(542.18908689,114.41345594)(542.12907959,114.40345998)
\curveto(542.05908702,114.40345595)(541.99908708,114.37345598)(541.94907959,114.31345998)
\curveto(541.89908718,114.24345611)(541.87408721,114.1534562)(541.87407959,114.04345998)
\curveto(541.86408722,113.94345641)(541.85908722,113.83345652)(541.85907959,113.71345998)
\lineto(541.85907959,112.57345998)
\lineto(541.85907959,112.07845998)
\curveto(541.84908723,111.91845843)(541.78908729,111.80845854)(541.67907959,111.74845998)
\curveto(541.64908743,111.72845862)(541.61908746,111.71845863)(541.58907959,111.71845998)
\curveto(541.54908753,111.71845863)(541.50408758,111.71345864)(541.45407959,111.70345998)
\curveto(541.33408775,111.68345867)(541.22408786,111.68845866)(541.12407959,111.71845998)
\curveto(541.02408806,111.75845859)(540.95408813,111.81345854)(540.91407959,111.88345998)
\curveto(540.86408822,111.96345839)(540.83908824,112.08345827)(540.83907959,112.24345998)
\curveto(540.83908824,112.40345795)(540.82408826,112.53845781)(540.79407959,112.64845998)
\curveto(540.7840883,112.69845765)(540.7790883,112.7534576)(540.77907959,112.81345998)
\curveto(540.76908831,112.87345748)(540.75408833,112.93345742)(540.73407959,112.99345998)
\curveto(540.6840884,113.14345721)(540.63408845,113.28845706)(540.58407959,113.42845998)
\curveto(540.52408856,113.56845678)(540.45408863,113.70345665)(540.37407959,113.83345998)
\curveto(540.2840888,113.97345638)(540.1790889,114.09345626)(540.05907959,114.19345998)
\curveto(539.93908914,114.29345606)(539.80908927,114.38845596)(539.66907959,114.47845998)
\curveto(539.56908951,114.53845581)(539.45908962,114.58345577)(539.33907959,114.61345998)
\curveto(539.21908986,114.6534557)(539.11408997,114.70345565)(539.02407959,114.76345998)
\curveto(538.96409012,114.81345554)(538.92409016,114.88345547)(538.90407959,114.97345998)
\curveto(538.89409019,114.99345536)(538.88909019,115.01845533)(538.88907959,115.04845998)
\curveto(538.88909019,115.07845527)(538.8840902,115.10345525)(538.87407959,115.12345998)
}
}
{
\newrgbcolor{curcolor}{0 0 0}
\pscustom[linestyle=none,fillstyle=solid,fillcolor=curcolor]
{
\newpath
\moveto(560.74536865,29.18119436)
\lineto(560.74536865,30.09619436)
\curveto(560.74537935,30.19619171)(560.74537935,30.29119161)(560.74536865,30.38119436)
\curveto(560.74537935,30.47119143)(560.76537933,30.54619136)(560.80536865,30.60619436)
\curveto(560.86537923,30.69619121)(560.94537915,30.75619115)(561.04536865,30.78619436)
\curveto(561.14537895,30.82619108)(561.25037884,30.87119103)(561.36036865,30.92119436)
\curveto(561.55037854,31.0011909)(561.74037835,31.07119083)(561.93036865,31.13119436)
\curveto(562.12037797,31.2011907)(562.31037778,31.27619063)(562.50036865,31.35619436)
\curveto(562.68037741,31.42619048)(562.86537723,31.49119041)(563.05536865,31.55119436)
\curveto(563.23537686,31.61119029)(563.41537668,31.68119022)(563.59536865,31.76119436)
\curveto(563.73537636,31.82119008)(563.88037621,31.87619003)(564.03036865,31.92619436)
\curveto(564.18037591,31.97618993)(564.32537577,32.03118987)(564.46536865,32.09119436)
\curveto(564.91537518,32.27118963)(565.37037472,32.44118946)(565.83036865,32.60119436)
\curveto(566.28037381,32.76118914)(566.73037336,32.93118897)(567.18036865,33.11119436)
\curveto(567.23037286,33.13118877)(567.28037281,33.14618876)(567.33036865,33.15619436)
\lineto(567.48036865,33.21619436)
\curveto(567.70037239,33.3061886)(567.92537217,33.39118851)(568.15536865,33.47119436)
\curveto(568.37537172,33.55118835)(568.5953715,33.63618827)(568.81536865,33.72619436)
\curveto(568.90537119,33.76618814)(569.01537108,33.8061881)(569.14536865,33.84619436)
\curveto(569.26537083,33.88618802)(569.33537076,33.95118795)(569.35536865,34.04119436)
\curveto(569.36537073,34.08118782)(569.36537073,34.11118779)(569.35536865,34.13119436)
\lineto(569.29536865,34.19119436)
\curveto(569.24537085,34.24118766)(569.1903709,34.27618763)(569.13036865,34.29619436)
\curveto(569.07037102,34.32618758)(569.00537109,34.35618755)(568.93536865,34.38619436)
\lineto(568.30536865,34.62619436)
\curveto(568.08537201,34.7061872)(567.87037222,34.78618712)(567.66036865,34.86619436)
\lineto(567.51036865,34.92619436)
\lineto(567.33036865,34.98619436)
\curveto(567.14037295,35.06618684)(566.95037314,35.13618677)(566.76036865,35.19619436)
\curveto(566.56037353,35.26618664)(566.36037373,35.34118656)(566.16036865,35.42119436)
\curveto(565.58037451,35.66118624)(564.9953751,35.88118602)(564.40536865,36.08119436)
\curveto(563.81537628,36.29118561)(563.23037686,36.51618539)(562.65036865,36.75619436)
\curveto(562.45037764,36.83618507)(562.24537785,36.91118499)(562.03536865,36.98119436)
\curveto(561.82537827,37.06118484)(561.62037847,37.14118476)(561.42036865,37.22119436)
\curveto(561.34037875,37.26118464)(561.24037885,37.29618461)(561.12036865,37.32619436)
\curveto(561.00037909,37.36618454)(560.91537918,37.42118448)(560.86536865,37.49119436)
\curveto(560.82537927,37.55118435)(560.7953793,37.62618428)(560.77536865,37.71619436)
\curveto(560.75537934,37.81618409)(560.74537935,37.92618398)(560.74536865,38.04619436)
\curveto(560.73537936,38.16618374)(560.73537936,38.28618362)(560.74536865,38.40619436)
\curveto(560.74537935,38.52618338)(560.74537935,38.63618327)(560.74536865,38.73619436)
\curveto(560.74537935,38.82618308)(560.74537935,38.91618299)(560.74536865,39.00619436)
\curveto(560.74537935,39.1061828)(560.76537933,39.18118272)(560.80536865,39.23119436)
\curveto(560.85537924,39.32118258)(560.94537915,39.37118253)(561.07536865,39.38119436)
\curveto(561.20537889,39.39118251)(561.34537875,39.39618251)(561.49536865,39.39619436)
\lineto(563.14536865,39.39619436)
\lineto(569.41536865,39.39619436)
\lineto(570.67536865,39.39619436)
\curveto(570.78536931,39.39618251)(570.8953692,39.39618251)(571.00536865,39.39619436)
\curveto(571.11536898,39.4061825)(571.20036889,39.38618252)(571.26036865,39.33619436)
\curveto(571.32036877,39.3061826)(571.36036873,39.26118264)(571.38036865,39.20119436)
\curveto(571.3903687,39.14118276)(571.40536869,39.07118283)(571.42536865,38.99119436)
\lineto(571.42536865,38.75119436)
\lineto(571.42536865,38.39119436)
\curveto(571.41536868,38.28118362)(571.37036872,38.2011837)(571.29036865,38.15119436)
\curveto(571.26036883,38.13118377)(571.23036886,38.11618379)(571.20036865,38.10619436)
\curveto(571.16036893,38.1061838)(571.11536898,38.09618381)(571.06536865,38.07619436)
\lineto(570.90036865,38.07619436)
\curveto(570.84036925,38.06618384)(570.77036932,38.06118384)(570.69036865,38.06119436)
\curveto(570.61036948,38.07118383)(570.53536956,38.07618383)(570.46536865,38.07619436)
\lineto(569.62536865,38.07619436)
\lineto(565.20036865,38.07619436)
\curveto(564.95037514,38.07618383)(564.70037539,38.07618383)(564.45036865,38.07619436)
\curveto(564.1903759,38.07618383)(563.94037615,38.07118383)(563.70036865,38.06119436)
\curveto(563.60037649,38.06118384)(563.4903766,38.05618385)(563.37036865,38.04619436)
\curveto(563.25037684,38.03618387)(563.1903769,37.98118392)(563.19036865,37.88119436)
\lineto(563.20536865,37.88119436)
\curveto(563.22537687,37.81118409)(563.2903768,37.75118415)(563.40036865,37.70119436)
\curveto(563.51037658,37.66118424)(563.60537649,37.62618428)(563.68536865,37.59619436)
\curveto(563.85537624,37.52618438)(564.03037606,37.46118444)(564.21036865,37.40119436)
\curveto(564.38037571,37.34118456)(564.55037554,37.27118463)(564.72036865,37.19119436)
\curveto(564.77037532,37.17118473)(564.81537528,37.15618475)(564.85536865,37.14619436)
\curveto(564.8953752,37.13618477)(564.94037515,37.12118478)(564.99036865,37.10119436)
\curveto(565.17037492,37.02118488)(565.35537474,36.95118495)(565.54536865,36.89119436)
\curveto(565.72537437,36.84118506)(565.90537419,36.77618513)(566.08536865,36.69619436)
\curveto(566.23537386,36.62618528)(566.3903737,36.56618534)(566.55036865,36.51619436)
\curveto(566.70037339,36.46618544)(566.85037324,36.41118549)(567.00036865,36.35119436)
\curveto(567.47037262,36.15118575)(567.94537215,35.97118593)(568.42536865,35.81119436)
\curveto(568.8953712,35.65118625)(569.36037073,35.47618643)(569.82036865,35.28619436)
\curveto(570.00037009,35.2061867)(570.18036991,35.13618677)(570.36036865,35.07619436)
\curveto(570.54036955,35.01618689)(570.72036937,34.95118695)(570.90036865,34.88119436)
\curveto(571.01036908,34.83118707)(571.11536898,34.78118712)(571.21536865,34.73119436)
\curveto(571.30536879,34.69118721)(571.37036872,34.6061873)(571.41036865,34.47619436)
\curveto(571.42036867,34.45618745)(571.42536867,34.43118747)(571.42536865,34.40119436)
\curveto(571.41536868,34.38118752)(571.41536868,34.35618755)(571.42536865,34.32619436)
\curveto(571.43536866,34.29618761)(571.44036865,34.26118764)(571.44036865,34.22119436)
\curveto(571.43036866,34.18118772)(571.42536867,34.14118776)(571.42536865,34.10119436)
\lineto(571.42536865,33.80119436)
\curveto(571.42536867,33.7011882)(571.40036869,33.62118828)(571.35036865,33.56119436)
\curveto(571.30036879,33.48118842)(571.23036886,33.42118848)(571.14036865,33.38119436)
\curveto(571.04036905,33.35118855)(570.94036915,33.31118859)(570.84036865,33.26119436)
\curveto(570.64036945,33.18118872)(570.43536966,33.1011888)(570.22536865,33.02119436)
\curveto(570.00537009,32.95118895)(569.7953703,32.87618903)(569.59536865,32.79619436)
\curveto(569.41537068,32.71618919)(569.23537086,32.64618926)(569.05536865,32.58619436)
\curveto(568.86537123,32.53618937)(568.68037141,32.47118943)(568.50036865,32.39119436)
\curveto(567.94037215,32.16118974)(567.37537272,31.94618996)(566.80536865,31.74619436)
\curveto(566.23537386,31.54619036)(565.67037442,31.33119057)(565.11036865,31.10119436)
\lineto(564.48036865,30.86119436)
\curveto(564.26037583,30.79119111)(564.05037604,30.71619119)(563.85036865,30.63619436)
\curveto(563.74037635,30.58619132)(563.63537646,30.54119136)(563.53536865,30.50119436)
\curveto(563.42537667,30.47119143)(563.33037676,30.42119148)(563.25036865,30.35119436)
\curveto(563.23037686,30.34119156)(563.22037687,30.33119157)(563.22036865,30.32119436)
\lineto(563.19036865,30.29119436)
\lineto(563.19036865,30.21619436)
\lineto(563.22036865,30.18619436)
\curveto(563.22037687,30.17619173)(563.22537687,30.16619174)(563.23536865,30.15619436)
\curveto(563.28537681,30.13619177)(563.34037675,30.12619178)(563.40036865,30.12619436)
\curveto(563.46037663,30.12619178)(563.52037657,30.11619179)(563.58036865,30.09619436)
\lineto(563.74536865,30.09619436)
\curveto(563.80537629,30.07619183)(563.87037622,30.07119183)(563.94036865,30.08119436)
\curveto(564.01037608,30.09119181)(564.08037601,30.09619181)(564.15036865,30.09619436)
\lineto(564.96036865,30.09619436)
\lineto(569.52036865,30.09619436)
\lineto(570.70536865,30.09619436)
\curveto(570.81536928,30.09619181)(570.92536917,30.09119181)(571.03536865,30.08119436)
\curveto(571.14536895,30.08119182)(571.23036886,30.05619185)(571.29036865,30.00619436)
\curveto(571.37036872,29.95619195)(571.41536868,29.86619204)(571.42536865,29.73619436)
\lineto(571.42536865,29.34619436)
\lineto(571.42536865,29.15119436)
\curveto(571.42536867,29.1011928)(571.41536868,29.05119285)(571.39536865,29.00119436)
\curveto(571.35536874,28.87119303)(571.27036882,28.79619311)(571.14036865,28.77619436)
\curveto(571.01036908,28.76619314)(570.86036923,28.76119314)(570.69036865,28.76119436)
\lineto(568.95036865,28.76119436)
\lineto(562.95036865,28.76119436)
\lineto(561.54036865,28.76119436)
\curveto(561.43037866,28.76119314)(561.31537878,28.75619315)(561.19536865,28.74619436)
\curveto(561.07537902,28.74619316)(560.98037911,28.77119313)(560.91036865,28.82119436)
\curveto(560.85037924,28.86119304)(560.80037929,28.93619297)(560.76036865,29.04619436)
\curveto(560.75037934,29.06619284)(560.75037934,29.08619282)(560.76036865,29.10619436)
\curveto(560.76037933,29.13619277)(560.75537934,29.16119274)(560.74536865,29.18119436)
}
}
{
\newrgbcolor{curcolor}{0 0 0}
\pscustom[linestyle=none,fillstyle=solid,fillcolor=curcolor]
{
\newpath
\moveto(570.87036865,48.38330373)
\curveto(571.03036906,48.4132959)(571.16536893,48.39829592)(571.27536865,48.33830373)
\curveto(571.37536872,48.27829604)(571.45036864,48.19829612)(571.50036865,48.09830373)
\curveto(571.52036857,48.04829627)(571.53036856,47.99329632)(571.53036865,47.93330373)
\curveto(571.53036856,47.88329643)(571.54036855,47.82829649)(571.56036865,47.76830373)
\curveto(571.61036848,47.54829677)(571.5953685,47.32829699)(571.51536865,47.10830373)
\curveto(571.44536865,46.89829742)(571.35536874,46.75329756)(571.24536865,46.67330373)
\curveto(571.17536892,46.62329769)(571.095369,46.57829774)(571.00536865,46.53830373)
\curveto(570.90536919,46.49829782)(570.82536927,46.44829787)(570.76536865,46.38830373)
\curveto(570.74536935,46.36829795)(570.72536937,46.34329797)(570.70536865,46.31330373)
\curveto(570.68536941,46.29329802)(570.68036941,46.26329805)(570.69036865,46.22330373)
\curveto(570.72036937,46.1132982)(570.77536932,46.00829831)(570.85536865,45.90830373)
\curveto(570.93536916,45.8182985)(571.00536909,45.72829859)(571.06536865,45.63830373)
\curveto(571.14536895,45.50829881)(571.22036887,45.36829895)(571.29036865,45.21830373)
\curveto(571.35036874,45.06829925)(571.40536869,44.90829941)(571.45536865,44.73830373)
\curveto(571.48536861,44.63829968)(571.50536859,44.52829979)(571.51536865,44.40830373)
\curveto(571.52536857,44.29830002)(571.54036855,44.18830013)(571.56036865,44.07830373)
\curveto(571.57036852,44.02830029)(571.57536852,43.98330033)(571.57536865,43.94330373)
\lineto(571.57536865,43.83830373)
\curveto(571.5953685,43.72830059)(571.5953685,43.62330069)(571.57536865,43.52330373)
\lineto(571.57536865,43.38830373)
\curveto(571.56536853,43.33830098)(571.56036853,43.28830103)(571.56036865,43.23830373)
\curveto(571.56036853,43.18830113)(571.55036854,43.14330117)(571.53036865,43.10330373)
\curveto(571.52036857,43.06330125)(571.51536858,43.02830129)(571.51536865,42.99830373)
\curveto(571.52536857,42.97830134)(571.52536857,42.95330136)(571.51536865,42.92330373)
\lineto(571.45536865,42.68330373)
\curveto(571.44536865,42.60330171)(571.42536867,42.52830179)(571.39536865,42.45830373)
\curveto(571.26536883,42.15830216)(571.12036897,41.9133024)(570.96036865,41.72330373)
\curveto(570.7903693,41.54330277)(570.55536954,41.39330292)(570.25536865,41.27330373)
\curveto(570.03537006,41.18330313)(569.77037032,41.13830318)(569.46036865,41.13830373)
\lineto(569.14536865,41.13830373)
\curveto(569.095371,41.14830317)(569.04537105,41.15330316)(568.99536865,41.15330373)
\lineto(568.81536865,41.18330373)
\lineto(568.48536865,41.30330373)
\curveto(568.37537172,41.34330297)(568.27537182,41.39330292)(568.18536865,41.45330373)
\curveto(567.8953722,41.63330268)(567.68037241,41.87830244)(567.54036865,42.18830373)
\curveto(567.40037269,42.49830182)(567.27537282,42.83830148)(567.16536865,43.20830373)
\curveto(567.12537297,43.34830097)(567.095373,43.49330082)(567.07536865,43.64330373)
\curveto(567.05537304,43.79330052)(567.03037306,43.94330037)(567.00036865,44.09330373)
\curveto(566.98037311,44.16330015)(566.97037312,44.22830009)(566.97036865,44.28830373)
\curveto(566.97037312,44.35829996)(566.96037313,44.43329988)(566.94036865,44.51330373)
\curveto(566.92037317,44.58329973)(566.91037318,44.65329966)(566.91036865,44.72330373)
\curveto(566.90037319,44.79329952)(566.88537321,44.86829945)(566.86536865,44.94830373)
\curveto(566.80537329,45.19829912)(566.75537334,45.43329888)(566.71536865,45.65330373)
\curveto(566.66537343,45.87329844)(566.55037354,46.04829827)(566.37036865,46.17830373)
\curveto(566.2903738,46.23829808)(566.1903739,46.28829803)(566.07036865,46.32830373)
\curveto(565.94037415,46.36829795)(565.80037429,46.36829795)(565.65036865,46.32830373)
\curveto(565.41037468,46.26829805)(565.22037487,46.17829814)(565.08036865,46.05830373)
\curveto(564.94037515,45.94829837)(564.83037526,45.78829853)(564.75036865,45.57830373)
\curveto(564.70037539,45.45829886)(564.66537543,45.313299)(564.64536865,45.14330373)
\curveto(564.62537547,44.98329933)(564.61537548,44.8132995)(564.61536865,44.63330373)
\curveto(564.61537548,44.45329986)(564.62537547,44.27830004)(564.64536865,44.10830373)
\curveto(564.66537543,43.93830038)(564.6953754,43.79330052)(564.73536865,43.67330373)
\curveto(564.7953753,43.50330081)(564.88037521,43.33830098)(564.99036865,43.17830373)
\curveto(565.05037504,43.09830122)(565.13037496,43.02330129)(565.23036865,42.95330373)
\curveto(565.32037477,42.89330142)(565.42037467,42.83830148)(565.53036865,42.78830373)
\curveto(565.61037448,42.75830156)(565.6953744,42.72830159)(565.78536865,42.69830373)
\curveto(565.87537422,42.67830164)(565.94537415,42.63330168)(565.99536865,42.56330373)
\curveto(566.02537407,42.52330179)(566.05037404,42.45330186)(566.07036865,42.35330373)
\curveto(566.08037401,42.26330205)(566.08537401,42.16830215)(566.08536865,42.06830373)
\curveto(566.08537401,41.96830235)(566.08037401,41.86830245)(566.07036865,41.76830373)
\curveto(566.05037404,41.67830264)(566.02537407,41.6133027)(565.99536865,41.57330373)
\curveto(565.96537413,41.53330278)(565.91537418,41.50330281)(565.84536865,41.48330373)
\curveto(565.77537432,41.46330285)(565.70037439,41.46330285)(565.62036865,41.48330373)
\curveto(565.4903746,41.5133028)(565.37037472,41.54330277)(565.26036865,41.57330373)
\curveto(565.14037495,41.6133027)(565.02537507,41.65830266)(564.91536865,41.70830373)
\curveto(564.56537553,41.89830242)(564.2953758,42.13830218)(564.10536865,42.42830373)
\curveto(563.90537619,42.7183016)(563.74537635,43.07830124)(563.62536865,43.50830373)
\curveto(563.60537649,43.60830071)(563.5903765,43.70830061)(563.58036865,43.80830373)
\curveto(563.57037652,43.9183004)(563.55537654,44.02830029)(563.53536865,44.13830373)
\curveto(563.52537657,44.17830014)(563.52537657,44.24330007)(563.53536865,44.33330373)
\curveto(563.53537656,44.42329989)(563.52537657,44.47829984)(563.50536865,44.49830373)
\curveto(563.4953766,45.19829912)(563.57537652,45.80829851)(563.74536865,46.32830373)
\curveto(563.91537618,46.84829747)(564.24037585,47.2132971)(564.72036865,47.42330373)
\curveto(564.92037517,47.5132968)(565.15537494,47.56329675)(565.42536865,47.57330373)
\curveto(565.68537441,47.59329672)(565.96037413,47.60329671)(566.25036865,47.60330373)
\lineto(569.56536865,47.60330373)
\curveto(569.70537039,47.60329671)(569.84037025,47.60829671)(569.97036865,47.61830373)
\curveto(570.10036999,47.62829669)(570.20536989,47.65829666)(570.28536865,47.70830373)
\curveto(570.35536974,47.75829656)(570.40536969,47.82329649)(570.43536865,47.90330373)
\curveto(570.47536962,47.99329632)(570.50536959,48.07829624)(570.52536865,48.15830373)
\curveto(570.53536956,48.23829608)(570.58036951,48.29829602)(570.66036865,48.33830373)
\curveto(570.6903694,48.35829596)(570.72036937,48.36829595)(570.75036865,48.36830373)
\curveto(570.78036931,48.36829595)(570.82036927,48.37329594)(570.87036865,48.38330373)
\moveto(569.20536865,46.23830373)
\curveto(569.06537103,46.29829802)(568.90537119,46.32829799)(568.72536865,46.32830373)
\curveto(568.53537156,46.33829798)(568.34037175,46.34329797)(568.14036865,46.34330373)
\curveto(568.03037206,46.34329797)(567.93037216,46.33829798)(567.84036865,46.32830373)
\curveto(567.75037234,46.318298)(567.68037241,46.27829804)(567.63036865,46.20830373)
\curveto(567.61037248,46.17829814)(567.60037249,46.10829821)(567.60036865,45.99830373)
\curveto(567.62037247,45.97829834)(567.63037246,45.94329837)(567.63036865,45.89330373)
\curveto(567.63037246,45.84329847)(567.64037245,45.79829852)(567.66036865,45.75830373)
\curveto(567.68037241,45.67829864)(567.70037239,45.58829873)(567.72036865,45.48830373)
\lineto(567.78036865,45.18830373)
\curveto(567.78037231,45.15829916)(567.78537231,45.12329919)(567.79536865,45.08330373)
\lineto(567.79536865,44.97830373)
\curveto(567.83537226,44.82829949)(567.86037223,44.66329965)(567.87036865,44.48330373)
\curveto(567.87037222,44.3133)(567.8903722,44.15330016)(567.93036865,44.00330373)
\curveto(567.95037214,43.92330039)(567.97037212,43.84830047)(567.99036865,43.77830373)
\curveto(568.00037209,43.7183006)(568.01537208,43.64830067)(568.03536865,43.56830373)
\curveto(568.08537201,43.40830091)(568.15037194,43.25830106)(568.23036865,43.11830373)
\curveto(568.30037179,42.97830134)(568.3903717,42.85830146)(568.50036865,42.75830373)
\curveto(568.61037148,42.65830166)(568.74537135,42.58330173)(568.90536865,42.53330373)
\curveto(569.05537104,42.48330183)(569.24037085,42.46330185)(569.46036865,42.47330373)
\curveto(569.56037053,42.47330184)(569.65537044,42.48830183)(569.74536865,42.51830373)
\curveto(569.82537027,42.55830176)(569.90037019,42.60330171)(569.97036865,42.65330373)
\curveto(570.08037001,42.73330158)(570.17536992,42.83830148)(570.25536865,42.96830373)
\curveto(570.32536977,43.09830122)(570.38536971,43.23830108)(570.43536865,43.38830373)
\curveto(570.44536965,43.43830088)(570.45036964,43.48830083)(570.45036865,43.53830373)
\curveto(570.45036964,43.58830073)(570.45536964,43.63830068)(570.46536865,43.68830373)
\curveto(570.48536961,43.75830056)(570.50036959,43.84330047)(570.51036865,43.94330373)
\curveto(570.51036958,44.05330026)(570.50036959,44.14330017)(570.48036865,44.21330373)
\curveto(570.46036963,44.27330004)(570.45536964,44.33329998)(570.46536865,44.39330373)
\curveto(570.46536963,44.45329986)(570.45536964,44.5132998)(570.43536865,44.57330373)
\curveto(570.41536968,44.65329966)(570.40036969,44.72829959)(570.39036865,44.79830373)
\curveto(570.38036971,44.87829944)(570.36036973,44.95329936)(570.33036865,45.02330373)
\curveto(570.21036988,45.313299)(570.06537003,45.55829876)(569.89536865,45.75830373)
\curveto(569.72537037,45.96829835)(569.4953706,46.12829819)(569.20536865,46.23830373)
}
}
{
\newrgbcolor{curcolor}{0 0 0}
\pscustom[linestyle=none,fillstyle=solid,fillcolor=curcolor]
{
\newpath
\moveto(563.70036865,49.26994436)
\lineto(563.70036865,49.71994436)
\curveto(563.6903764,49.88994311)(563.71037638,50.01494298)(563.76036865,50.09494436)
\curveto(563.81037628,50.17494282)(563.87537622,50.22994277)(563.95536865,50.25994436)
\curveto(564.03537606,50.2999427)(564.12037597,50.33994266)(564.21036865,50.37994436)
\curveto(564.34037575,50.42994257)(564.47037562,50.47494252)(564.60036865,50.51494436)
\curveto(564.73037536,50.55494244)(564.86037523,50.5999424)(564.99036865,50.64994436)
\curveto(565.11037498,50.6999423)(565.23537486,50.74494225)(565.36536865,50.78494436)
\curveto(565.48537461,50.82494217)(565.60537449,50.86994213)(565.72536865,50.91994436)
\curveto(565.83537426,50.96994203)(565.95037414,51.00994199)(566.07036865,51.03994436)
\curveto(566.1903739,51.06994193)(566.31037378,51.10994189)(566.43036865,51.15994436)
\curveto(566.72037337,51.27994172)(567.02037307,51.38994161)(567.33036865,51.48994436)
\curveto(567.64037245,51.58994141)(567.94037215,51.6999413)(568.23036865,51.81994436)
\curveto(568.27037182,51.83994116)(568.31037178,51.84994115)(568.35036865,51.84994436)
\curveto(568.38037171,51.84994115)(568.41037168,51.85994114)(568.44036865,51.87994436)
\curveto(568.58037151,51.93994106)(568.72537137,51.994941)(568.87536865,52.04494436)
\lineto(569.29536865,52.19494436)
\curveto(569.36537073,52.22494077)(569.44037065,52.25494074)(569.52036865,52.28494436)
\curveto(569.5903705,52.31494068)(569.63537046,52.36494063)(569.65536865,52.43494436)
\curveto(569.68537041,52.51494048)(569.66037043,52.57494042)(569.58036865,52.61494436)
\curveto(569.4903706,52.66494033)(569.42037067,52.6999403)(569.37036865,52.71994436)
\curveto(569.20037089,52.7999402)(569.02037107,52.86494013)(568.83036865,52.91494436)
\curveto(568.64037145,52.96494003)(568.45537164,53.02493997)(568.27536865,53.09494436)
\curveto(568.04537205,53.18493981)(567.81537228,53.26493973)(567.58536865,53.33494436)
\curveto(567.34537275,53.40493959)(567.11537298,53.48993951)(566.89536865,53.58994436)
\curveto(566.84537325,53.5999394)(566.78037331,53.61493938)(566.70036865,53.63494436)
\curveto(566.61037348,53.67493932)(566.52037357,53.70993929)(566.43036865,53.73994436)
\curveto(566.33037376,53.76993923)(566.24037385,53.7999392)(566.16036865,53.82994436)
\curveto(566.11037398,53.84993915)(566.06537403,53.86493913)(566.02536865,53.87494436)
\curveto(565.98537411,53.88493911)(565.94037415,53.8999391)(565.89036865,53.91994436)
\curveto(565.77037432,53.96993903)(565.65037444,54.00993899)(565.53036865,54.03994436)
\curveto(565.40037469,54.07993892)(565.27537482,54.12493887)(565.15536865,54.17494436)
\curveto(565.10537499,54.1949388)(565.06037503,54.20993879)(565.02036865,54.21994436)
\curveto(564.98037511,54.22993877)(564.93537516,54.24493875)(564.88536865,54.26494436)
\curveto(564.7953753,54.30493869)(564.70537539,54.33993866)(564.61536865,54.36994436)
\curveto(564.51537558,54.3999386)(564.42037567,54.42993857)(564.33036865,54.45994436)
\curveto(564.25037584,54.48993851)(564.17037592,54.51493848)(564.09036865,54.53494436)
\curveto(564.00037609,54.56493843)(563.92537617,54.60493839)(563.86536865,54.65494436)
\curveto(563.77537632,54.72493827)(563.72537637,54.81993818)(563.71536865,54.93994436)
\curveto(563.70537639,55.06993793)(563.70037639,55.20993779)(563.70036865,55.35994436)
\curveto(563.70037639,55.43993756)(563.70537639,55.51493748)(563.71536865,55.58494436)
\curveto(563.71537638,55.66493733)(563.73037636,55.72993727)(563.76036865,55.77994436)
\curveto(563.82037627,55.86993713)(563.91537618,55.8949371)(564.04536865,55.85494436)
\curveto(564.17537592,55.81493718)(564.27537582,55.77993722)(564.34536865,55.74994436)
\lineto(564.40536865,55.71994436)
\curveto(564.42537567,55.71993728)(564.44537565,55.71493728)(564.46536865,55.70494436)
\curveto(564.74537535,55.5949374)(565.03037506,55.48493751)(565.32036865,55.37494436)
\lineto(566.16036865,55.04494436)
\curveto(566.24037385,55.01493798)(566.31537378,54.98993801)(566.38536865,54.96994436)
\curveto(566.44537365,54.94993805)(566.51037358,54.92493807)(566.58036865,54.89494436)
\curveto(566.78037331,54.81493818)(566.98537311,54.73493826)(567.19536865,54.65494436)
\curveto(567.3953727,54.58493841)(567.5953725,54.50993849)(567.79536865,54.42994436)
\curveto(568.48537161,54.13993886)(569.18037091,53.86993913)(569.88036865,53.61994436)
\curveto(570.58036951,53.36993963)(571.27536882,53.0999399)(571.96536865,52.80994436)
\lineto(572.11536865,52.74994436)
\curveto(572.17536792,52.73994026)(572.23536786,52.72494027)(572.29536865,52.70494436)
\curveto(572.66536743,52.54494045)(573.03036706,52.37494062)(573.39036865,52.19494436)
\curveto(573.76036633,52.01494098)(574.04536605,51.76494123)(574.24536865,51.44494436)
\curveto(574.30536579,51.33494166)(574.35036574,51.22494177)(574.38036865,51.11494436)
\curveto(574.42036567,51.00494199)(574.45536564,50.87994212)(574.48536865,50.73994436)
\curveto(574.50536559,50.68994231)(574.51036558,50.63494236)(574.50036865,50.57494436)
\curveto(574.4903656,50.52494247)(574.4903656,50.46994253)(574.50036865,50.40994436)
\curveto(574.52036557,50.32994267)(574.52036557,50.24994275)(574.50036865,50.16994436)
\curveto(574.4903656,50.12994287)(574.48536561,50.07994292)(574.48536865,50.01994436)
\lineto(574.42536865,49.77994436)
\curveto(574.40536569,49.70994329)(574.36536573,49.65494334)(574.30536865,49.61494436)
\curveto(574.24536585,49.56494343)(574.17036592,49.53494346)(574.08036865,49.52494436)
\lineto(573.81036865,49.52494436)
\lineto(573.60036865,49.52494436)
\curveto(573.54036655,49.53494346)(573.4903666,49.55494344)(573.45036865,49.58494436)
\curveto(573.34036675,49.65494334)(573.31036678,49.77494322)(573.36036865,49.94494436)
\curveto(573.38036671,50.05494294)(573.3903667,50.17494282)(573.39036865,50.30494436)
\curveto(573.3903667,50.43494256)(573.37036672,50.54994245)(573.33036865,50.64994436)
\curveto(573.28036681,50.7999422)(573.20536689,50.91994208)(573.10536865,51.00994436)
\curveto(573.00536709,51.10994189)(572.8903672,51.1949418)(572.76036865,51.26494436)
\curveto(572.64036745,51.33494166)(572.51036758,51.3949416)(572.37036865,51.44494436)
\lineto(571.95036865,51.62494436)
\curveto(571.86036823,51.66494133)(571.75036834,51.70494129)(571.62036865,51.74494436)
\curveto(571.4903686,51.7949412)(571.35536874,51.7999412)(571.21536865,51.75994436)
\curveto(571.05536904,51.70994129)(570.90536919,51.65494134)(570.76536865,51.59494436)
\curveto(570.62536947,51.54494145)(570.48536961,51.48994151)(570.34536865,51.42994436)
\curveto(570.13536996,51.33994166)(569.92537017,51.25494174)(569.71536865,51.17494436)
\curveto(569.50537059,51.0949419)(569.30037079,51.01494198)(569.10036865,50.93494436)
\curveto(568.96037113,50.87494212)(568.82537127,50.81994218)(568.69536865,50.76994436)
\curveto(568.56537153,50.71994228)(568.43037166,50.66994233)(568.29036865,50.61994436)
\lineto(566.97036865,50.07994436)
\curveto(566.53037356,49.90994309)(566.090374,49.73494326)(565.65036865,49.55494436)
\curveto(565.42037467,49.45494354)(565.20037489,49.36494363)(564.99036865,49.28494436)
\curveto(564.77037532,49.20494379)(564.55037554,49.11994388)(564.33036865,49.02994436)
\curveto(564.27037582,49.00994399)(564.1903759,48.97994402)(564.09036865,48.93994436)
\curveto(563.98037611,48.8999441)(563.8903762,48.90494409)(563.82036865,48.95494436)
\curveto(563.77037632,48.98494401)(563.73537636,49.04494395)(563.71536865,49.13494436)
\curveto(563.70537639,49.15494384)(563.70537639,49.17494382)(563.71536865,49.19494436)
\curveto(563.71537638,49.22494377)(563.71037638,49.24994375)(563.70036865,49.26994436)
}
}
{
\newrgbcolor{curcolor}{0 0 0}
\pscustom[linestyle=none,fillstyle=solid,fillcolor=curcolor]
{
}
}
{
\newrgbcolor{curcolor}{0 0 0}
\pscustom[linestyle=none,fillstyle=solid,fillcolor=curcolor]
{
\newpath
\moveto(560.82036865,65.05510061)
\curveto(560.82037927,65.15509575)(560.83037926,65.25009566)(560.85036865,65.34010061)
\curveto(560.86037923,65.43009548)(560.8903792,65.49509541)(560.94036865,65.53510061)
\curveto(561.02037907,65.59509531)(561.12537897,65.62509528)(561.25536865,65.62510061)
\lineto(561.64536865,65.62510061)
\lineto(563.14536865,65.62510061)
\lineto(569.53536865,65.62510061)
\lineto(570.70536865,65.62510061)
\lineto(571.02036865,65.62510061)
\curveto(571.12036897,65.63509527)(571.20036889,65.62009529)(571.26036865,65.58010061)
\curveto(571.34036875,65.53009538)(571.3903687,65.45509545)(571.41036865,65.35510061)
\curveto(571.42036867,65.26509564)(571.42536867,65.15509575)(571.42536865,65.02510061)
\lineto(571.42536865,64.80010061)
\curveto(571.40536869,64.72009619)(571.3903687,64.65009626)(571.38036865,64.59010061)
\curveto(571.36036873,64.53009638)(571.32036877,64.48009643)(571.26036865,64.44010061)
\curveto(571.20036889,64.40009651)(571.12536897,64.38009653)(571.03536865,64.38010061)
\lineto(570.73536865,64.38010061)
\lineto(569.64036865,64.38010061)
\lineto(564.30036865,64.38010061)
\curveto(564.21037588,64.36009655)(564.13537596,64.34509656)(564.07536865,64.33510061)
\curveto(564.00537609,64.33509657)(563.94537615,64.3050966)(563.89536865,64.24510061)
\curveto(563.84537625,64.17509673)(563.82037627,64.08509682)(563.82036865,63.97510061)
\curveto(563.81037628,63.87509703)(563.80537629,63.76509714)(563.80536865,63.64510061)
\lineto(563.80536865,62.50510061)
\lineto(563.80536865,62.01010061)
\curveto(563.7953763,61.85009906)(563.73537636,61.74009917)(563.62536865,61.68010061)
\curveto(563.5953765,61.66009925)(563.56537653,61.65009926)(563.53536865,61.65010061)
\curveto(563.4953766,61.65009926)(563.45037664,61.64509926)(563.40036865,61.63510061)
\curveto(563.28037681,61.61509929)(563.17037692,61.62009929)(563.07036865,61.65010061)
\curveto(562.97037712,61.69009922)(562.90037719,61.74509916)(562.86036865,61.81510061)
\curveto(562.81037728,61.89509901)(562.78537731,62.01509889)(562.78536865,62.17510061)
\curveto(562.78537731,62.33509857)(562.77037732,62.47009844)(562.74036865,62.58010061)
\curveto(562.73037736,62.63009828)(562.72537737,62.68509822)(562.72536865,62.74510061)
\curveto(562.71537738,62.8050981)(562.70037739,62.86509804)(562.68036865,62.92510061)
\curveto(562.63037746,63.07509783)(562.58037751,63.22009769)(562.53036865,63.36010061)
\curveto(562.47037762,63.50009741)(562.40037769,63.63509727)(562.32036865,63.76510061)
\curveto(562.23037786,63.905097)(562.12537797,64.02509688)(562.00536865,64.12510061)
\curveto(561.88537821,64.22509668)(561.75537834,64.32009659)(561.61536865,64.41010061)
\curveto(561.51537858,64.47009644)(561.40537869,64.51509639)(561.28536865,64.54510061)
\curveto(561.16537893,64.58509632)(561.06037903,64.63509627)(560.97036865,64.69510061)
\curveto(560.91037918,64.74509616)(560.87037922,64.81509609)(560.85036865,64.90510061)
\curveto(560.84037925,64.92509598)(560.83537926,64.95009596)(560.83536865,64.98010061)
\curveto(560.83537926,65.0100959)(560.83037926,65.03509587)(560.82036865,65.05510061)
}
}
{
\newrgbcolor{curcolor}{0 0 0}
\pscustom[linestyle=none,fillstyle=solid,fillcolor=curcolor]
{
\newpath
\moveto(561.01536865,69.80470998)
\lineto(561.01536865,74.60470998)
\lineto(561.01536865,75.60970998)
\curveto(561.01537908,75.74970288)(561.02537907,75.86970276)(561.04536865,75.96970998)
\curveto(561.05537904,76.07970255)(561.10037899,76.15970247)(561.18036865,76.20970998)
\curveto(561.22037887,76.2297024)(561.27037882,76.23970239)(561.33036865,76.23970998)
\curveto(561.3903787,76.24970238)(561.45537864,76.25470238)(561.52536865,76.25470998)
\lineto(561.79536865,76.25470998)
\curveto(561.88537821,76.25470238)(561.96537813,76.24470239)(562.03536865,76.22470998)
\curveto(562.11537798,76.18470245)(562.18537791,76.13970249)(562.24536865,76.08970998)
\lineto(562.42536865,75.93970998)
\curveto(562.47537762,75.90970272)(562.51537758,75.87470276)(562.54536865,75.83470998)
\curveto(562.57537752,75.79470284)(562.61537748,75.75470288)(562.66536865,75.71470998)
\curveto(562.77537732,75.634703)(562.88537721,75.54970308)(562.99536865,75.45970998)
\curveto(563.095377,75.36970326)(563.20037689,75.28470335)(563.31036865,75.20470998)
\curveto(563.51037658,75.06470357)(563.72037637,74.92470371)(563.94036865,74.78470998)
\curveto(564.15037594,74.64470399)(564.36537573,74.50470413)(564.58536865,74.36470998)
\curveto(564.67537542,74.31470432)(564.77037532,74.26470437)(564.87036865,74.21470998)
\curveto(564.97037512,74.16470447)(565.06537503,74.10970452)(565.15536865,74.04970998)
\curveto(565.17537492,74.0297046)(565.20037489,74.01970461)(565.23036865,74.01970998)
\curveto(565.26037483,74.01970461)(565.28537481,74.00970462)(565.30536865,73.98970998)
\curveto(565.40537469,73.91970471)(565.52037457,73.85470478)(565.65036865,73.79470998)
\curveto(565.77037432,73.7347049)(565.88537421,73.67970495)(565.99536865,73.62970998)
\curveto(566.22537387,73.5297051)(566.46037363,73.4347052)(566.70036865,73.34470998)
\curveto(566.94037315,73.25470538)(567.18037291,73.15470548)(567.42036865,73.04470998)
\curveto(567.47037262,73.02470561)(567.51537258,73.00970562)(567.55536865,72.99970998)
\curveto(567.5953725,72.99970563)(567.64037245,72.98970564)(567.69036865,72.96970998)
\curveto(567.81037228,72.91970571)(567.93537216,72.87470576)(568.06536865,72.83470998)
\curveto(568.18537191,72.80470583)(568.30537179,72.76970586)(568.42536865,72.72970998)
\curveto(568.65537144,72.64970598)(568.8953712,72.58470605)(569.14536865,72.53470998)
\curveto(569.38537071,72.49470614)(569.62537047,72.44470619)(569.86536865,72.38470998)
\curveto(570.01537008,72.34470629)(570.16536993,72.31970631)(570.31536865,72.30970998)
\curveto(570.46536963,72.29970633)(570.61536948,72.27970635)(570.76536865,72.24970998)
\curveto(570.80536929,72.23970639)(570.86536923,72.2347064)(570.94536865,72.23470998)
\curveto(571.06536903,72.20470643)(571.16536893,72.17470646)(571.24536865,72.14470998)
\curveto(571.32536877,72.11470652)(571.38036871,72.04470659)(571.41036865,71.93470998)
\curveto(571.43036866,71.88470675)(571.44036865,71.8297068)(571.44036865,71.76970998)
\lineto(571.44036865,71.57470998)
\curveto(571.44036865,71.4347072)(571.43536866,71.29470734)(571.42536865,71.15470998)
\curveto(571.41536868,71.02470761)(571.37036872,70.9297077)(571.29036865,70.86970998)
\curveto(571.23036886,70.8297078)(571.14536895,70.80970782)(571.03536865,70.80970998)
\curveto(570.92536917,70.81970781)(570.83036926,70.8347078)(570.75036865,70.85470998)
\lineto(570.67536865,70.85470998)
\curveto(570.64536945,70.86470777)(570.61536948,70.86970776)(570.58536865,70.86970998)
\curveto(570.50536959,70.88970774)(570.43036966,70.89970773)(570.36036865,70.89970998)
\curveto(570.2903698,70.89970773)(570.22036987,70.90970772)(570.15036865,70.92970998)
\curveto(569.96037013,70.97970765)(569.77537032,71.01970761)(569.59536865,71.04970998)
\curveto(569.40537069,71.07970755)(569.22537087,71.11970751)(569.05536865,71.16970998)
\curveto(569.00537109,71.18970744)(568.96537113,71.19970743)(568.93536865,71.19970998)
\curveto(568.90537119,71.19970743)(568.87037122,71.20470743)(568.83036865,71.21470998)
\curveto(568.53037156,71.31470732)(568.23537186,71.40470723)(567.94536865,71.48470998)
\curveto(567.65537244,71.57470706)(567.37537272,71.67970695)(567.10536865,71.79970998)
\curveto(566.52537357,72.05970657)(565.97537412,72.3297063)(565.45536865,72.60970998)
\curveto(564.92537517,72.88970574)(564.42037567,73.19970543)(563.94036865,73.53970998)
\curveto(563.74037635,73.67970495)(563.55037654,73.8297048)(563.37036865,73.98970998)
\curveto(563.18037691,74.14970448)(562.9903771,74.29970433)(562.80036865,74.43970998)
\curveto(562.75037734,74.47970415)(562.70537739,74.51470412)(562.66536865,74.54470998)
\curveto(562.61537748,74.58470405)(562.56537753,74.61970401)(562.51536865,74.64970998)
\curveto(562.4953776,74.65970397)(562.47037762,74.66970396)(562.44036865,74.67970998)
\curveto(562.41037768,74.69970393)(562.38037771,74.69970393)(562.35036865,74.67970998)
\curveto(562.2903778,74.65970397)(562.25537784,74.62470401)(562.24536865,74.57470998)
\curveto(562.22537787,74.52470411)(562.20537789,74.47470416)(562.18536865,74.42470998)
\lineto(562.18536865,74.31970998)
\curveto(562.17537792,74.27970435)(562.17537792,74.2297044)(562.18536865,74.16970998)
\lineto(562.18536865,74.01970998)
\lineto(562.18536865,73.41970998)
\lineto(562.18536865,70.77970998)
\lineto(562.18536865,70.04470998)
\lineto(562.18536865,69.80470998)
\curveto(562.17537792,69.7347089)(562.16037793,69.67470896)(562.14036865,69.62470998)
\curveto(562.10037799,69.5347091)(562.04037805,69.47470916)(561.96036865,69.44470998)
\curveto(561.86037823,69.39470924)(561.71537838,69.37970925)(561.52536865,69.39970998)
\curveto(561.32537877,69.41970921)(561.1903789,69.45470918)(561.12036865,69.50470998)
\curveto(561.10037899,69.52470911)(561.08537901,69.54970908)(561.07536865,69.57970998)
\lineto(561.01536865,69.69970998)
\curveto(561.01537908,69.71970891)(561.02037907,69.7347089)(561.03036865,69.74470998)
\curveto(561.03037906,69.76470887)(561.02537907,69.78470885)(561.01536865,69.80470998)
}
}
{
\newrgbcolor{curcolor}{0 0 0}
\pscustom[linestyle=none,fillstyle=solid,fillcolor=curcolor]
{
\newpath
\moveto(569.79036865,78.63431936)
\lineto(569.79036865,79.26431936)
\lineto(569.79036865,79.45931936)
\curveto(569.7903703,79.52931683)(569.80037029,79.58931677)(569.82036865,79.63931936)
\curveto(569.86037023,79.70931665)(569.90037019,79.7593166)(569.94036865,79.78931936)
\curveto(569.9903701,79.82931653)(570.05537004,79.84931651)(570.13536865,79.84931936)
\curveto(570.21536988,79.8593165)(570.30036979,79.86431649)(570.39036865,79.86431936)
\lineto(571.11036865,79.86431936)
\curveto(571.5903685,79.86431649)(572.00036809,79.80431655)(572.34036865,79.68431936)
\curveto(572.68036741,79.56431679)(572.95536714,79.36931699)(573.16536865,79.09931936)
\curveto(573.21536688,79.02931733)(573.26036683,78.9593174)(573.30036865,78.88931936)
\curveto(573.35036674,78.82931753)(573.3953667,78.7543176)(573.43536865,78.66431936)
\curveto(573.44536665,78.64431771)(573.45536664,78.61431774)(573.46536865,78.57431936)
\curveto(573.48536661,78.53431782)(573.4903666,78.48931787)(573.48036865,78.43931936)
\curveto(573.45036664,78.34931801)(573.37536672,78.29431806)(573.25536865,78.27431936)
\curveto(573.14536695,78.2543181)(573.05036704,78.26931809)(572.97036865,78.31931936)
\curveto(572.90036719,78.34931801)(572.83536726,78.39431796)(572.77536865,78.45431936)
\curveto(572.72536737,78.52431783)(572.67536742,78.58931777)(572.62536865,78.64931936)
\curveto(572.57536752,78.71931764)(572.50036759,78.77931758)(572.40036865,78.82931936)
\curveto(572.31036778,78.88931747)(572.22036787,78.93931742)(572.13036865,78.97931936)
\curveto(572.10036799,78.99931736)(572.04036805,79.02431733)(571.95036865,79.05431936)
\curveto(571.87036822,79.08431727)(571.80036829,79.08931727)(571.74036865,79.06931936)
\curveto(571.60036849,79.03931732)(571.51036858,78.97931738)(571.47036865,78.88931936)
\curveto(571.44036865,78.80931755)(571.42536867,78.71931764)(571.42536865,78.61931936)
\curveto(571.42536867,78.51931784)(571.40036869,78.43431792)(571.35036865,78.36431936)
\curveto(571.28036881,78.27431808)(571.14036895,78.22931813)(570.93036865,78.22931936)
\lineto(570.37536865,78.22931936)
\lineto(570.15036865,78.22931936)
\curveto(570.07037002,78.23931812)(570.00537009,78.2593181)(569.95536865,78.28931936)
\curveto(569.87537022,78.34931801)(569.83037026,78.41931794)(569.82036865,78.49931936)
\curveto(569.81037028,78.51931784)(569.80537029,78.53931782)(569.80536865,78.55931936)
\curveto(569.80537029,78.58931777)(569.80037029,78.61431774)(569.79036865,78.63431936)
}
}
{
\newrgbcolor{curcolor}{0 0 0}
\pscustom[linestyle=none,fillstyle=solid,fillcolor=curcolor]
{
}
}
{
\newrgbcolor{curcolor}{0 0 0}
\pscustom[linestyle=none,fillstyle=solid,fillcolor=curcolor]
{
\newpath
\moveto(560.82036865,89.26463186)
\curveto(560.81037928,89.95462722)(560.93037916,90.55462662)(561.18036865,91.06463186)
\curveto(561.43037866,91.58462559)(561.76537833,91.9796252)(562.18536865,92.24963186)
\curveto(562.26537783,92.29962488)(562.35537774,92.34462483)(562.45536865,92.38463186)
\curveto(562.54537755,92.42462475)(562.64037745,92.46962471)(562.74036865,92.51963186)
\curveto(562.84037725,92.55962462)(562.94037715,92.58962459)(563.04036865,92.60963186)
\curveto(563.14037695,92.62962455)(563.24537685,92.64962453)(563.35536865,92.66963186)
\curveto(563.40537669,92.68962449)(563.45037664,92.69462448)(563.49036865,92.68463186)
\curveto(563.53037656,92.6746245)(563.57537652,92.6796245)(563.62536865,92.69963186)
\curveto(563.67537642,92.70962447)(563.76037633,92.71462446)(563.88036865,92.71463186)
\curveto(563.9903761,92.71462446)(564.07537602,92.70962447)(564.13536865,92.69963186)
\curveto(564.1953759,92.6796245)(564.25537584,92.66962451)(564.31536865,92.66963186)
\curveto(564.37537572,92.6796245)(564.43537566,92.6746245)(564.49536865,92.65463186)
\curveto(564.63537546,92.61462456)(564.77037532,92.5796246)(564.90036865,92.54963186)
\curveto(565.03037506,92.51962466)(565.15537494,92.4796247)(565.27536865,92.42963186)
\curveto(565.41537468,92.36962481)(565.54037455,92.29962488)(565.65036865,92.21963186)
\curveto(565.76037433,92.14962503)(565.87037422,92.0746251)(565.98036865,91.99463186)
\lineto(566.04036865,91.93463186)
\curveto(566.06037403,91.92462525)(566.08037401,91.90962527)(566.10036865,91.88963186)
\curveto(566.26037383,91.76962541)(566.40537369,91.63462554)(566.53536865,91.48463186)
\curveto(566.66537343,91.33462584)(566.7903733,91.174626)(566.91036865,91.00463186)
\curveto(567.13037296,90.69462648)(567.33537276,90.39962678)(567.52536865,90.11963186)
\curveto(567.66537243,89.88962729)(567.80037229,89.65962752)(567.93036865,89.42963186)
\curveto(568.06037203,89.20962797)(568.1953719,88.98962819)(568.33536865,88.76963186)
\curveto(568.50537159,88.51962866)(568.68537141,88.2796289)(568.87536865,88.04963186)
\curveto(569.06537103,87.82962935)(569.2903708,87.63962954)(569.55036865,87.47963186)
\curveto(569.61037048,87.43962974)(569.67037042,87.40462977)(569.73036865,87.37463186)
\curveto(569.78037031,87.34462983)(569.84537025,87.31462986)(569.92536865,87.28463186)
\curveto(569.9953701,87.26462991)(570.05537004,87.25962992)(570.10536865,87.26963186)
\curveto(570.17536992,87.28962989)(570.23036986,87.32462985)(570.27036865,87.37463186)
\curveto(570.30036979,87.42462975)(570.32036977,87.48462969)(570.33036865,87.55463186)
\lineto(570.33036865,87.79463186)
\lineto(570.33036865,88.54463186)
\lineto(570.33036865,91.34963186)
\lineto(570.33036865,92.00963186)
\curveto(570.33036976,92.09962508)(570.33536976,92.18462499)(570.34536865,92.26463186)
\curveto(570.34536975,92.34462483)(570.36536973,92.40962477)(570.40536865,92.45963186)
\curveto(570.44536965,92.50962467)(570.52036957,92.54962463)(570.63036865,92.57963186)
\curveto(570.73036936,92.61962456)(570.83036926,92.62962455)(570.93036865,92.60963186)
\lineto(571.06536865,92.60963186)
\curveto(571.13536896,92.58962459)(571.1953689,92.56962461)(571.24536865,92.54963186)
\curveto(571.2953688,92.52962465)(571.33536876,92.49462468)(571.36536865,92.44463186)
\curveto(571.40536869,92.39462478)(571.42536867,92.32462485)(571.42536865,92.23463186)
\lineto(571.42536865,91.96463186)
\lineto(571.42536865,91.06463186)
\lineto(571.42536865,87.55463186)
\lineto(571.42536865,86.48963186)
\curveto(571.42536867,86.40963077)(571.43036866,86.31963086)(571.44036865,86.21963186)
\curveto(571.44036865,86.11963106)(571.43036866,86.03463114)(571.41036865,85.96463186)
\curveto(571.34036875,85.75463142)(571.16036893,85.68963149)(570.87036865,85.76963186)
\curveto(570.83036926,85.7796314)(570.7953693,85.7796314)(570.76536865,85.76963186)
\curveto(570.72536937,85.76963141)(570.68036941,85.7796314)(570.63036865,85.79963186)
\curveto(570.55036954,85.81963136)(570.46536963,85.83963134)(570.37536865,85.85963186)
\curveto(570.28536981,85.8796313)(570.20036989,85.90463127)(570.12036865,85.93463186)
\curveto(569.63037046,86.09463108)(569.21537088,86.29463088)(568.87536865,86.53463186)
\curveto(568.62537147,86.71463046)(568.40037169,86.91963026)(568.20036865,87.14963186)
\curveto(567.9903721,87.3796298)(567.7953723,87.61962956)(567.61536865,87.86963186)
\curveto(567.43537266,88.12962905)(567.26537283,88.39462878)(567.10536865,88.66463186)
\curveto(566.93537316,88.94462823)(566.76037333,89.21462796)(566.58036865,89.47463186)
\curveto(566.50037359,89.58462759)(566.42537367,89.68962749)(566.35536865,89.78963186)
\curveto(566.28537381,89.89962728)(566.21037388,90.00962717)(566.13036865,90.11963186)
\curveto(566.10037399,90.15962702)(566.07037402,90.19462698)(566.04036865,90.22463186)
\curveto(566.00037409,90.26462691)(565.97037412,90.30462687)(565.95036865,90.34463186)
\curveto(565.84037425,90.48462669)(565.71537438,90.60962657)(565.57536865,90.71963186)
\curveto(565.54537455,90.73962644)(565.52037457,90.76462641)(565.50036865,90.79463186)
\curveto(565.47037462,90.82462635)(565.44037465,90.84962633)(565.41036865,90.86963186)
\curveto(565.31037478,90.94962623)(565.21037488,91.01462616)(565.11036865,91.06463186)
\curveto(565.01037508,91.12462605)(564.90037519,91.179626)(564.78036865,91.22963186)
\curveto(564.71037538,91.25962592)(564.63537546,91.2796259)(564.55536865,91.28963186)
\lineto(564.31536865,91.34963186)
\lineto(564.22536865,91.34963186)
\curveto(564.1953759,91.35962582)(564.16537593,91.36462581)(564.13536865,91.36463186)
\curveto(564.06537603,91.38462579)(563.97037612,91.38962579)(563.85036865,91.37963186)
\curveto(563.72037637,91.3796258)(563.62037647,91.36962581)(563.55036865,91.34963186)
\curveto(563.47037662,91.32962585)(563.3953767,91.30962587)(563.32536865,91.28963186)
\curveto(563.24537685,91.2796259)(563.16537693,91.25962592)(563.08536865,91.22963186)
\curveto(562.84537725,91.11962606)(562.64537745,90.96962621)(562.48536865,90.77963186)
\curveto(562.31537778,90.59962658)(562.17537792,90.3796268)(562.06536865,90.11963186)
\curveto(562.04537805,90.04962713)(562.03037806,89.9796272)(562.02036865,89.90963186)
\curveto(562.00037809,89.83962734)(561.98037811,89.76462741)(561.96036865,89.68463186)
\curveto(561.94037815,89.60462757)(561.93037816,89.49462768)(561.93036865,89.35463186)
\curveto(561.93037816,89.22462795)(561.94037815,89.11962806)(561.96036865,89.03963186)
\curveto(561.97037812,88.9796282)(561.97537812,88.92462825)(561.97536865,88.87463186)
\curveto(561.97537812,88.82462835)(561.98537811,88.7746284)(562.00536865,88.72463186)
\curveto(562.04537805,88.62462855)(562.08537801,88.52962865)(562.12536865,88.43963186)
\curveto(562.16537793,88.35962882)(562.21037788,88.2796289)(562.26036865,88.19963186)
\curveto(562.28037781,88.16962901)(562.30537779,88.13962904)(562.33536865,88.10963186)
\curveto(562.36537773,88.08962909)(562.3903777,88.06462911)(562.41036865,88.03463186)
\lineto(562.48536865,87.95963186)
\curveto(562.50537759,87.92962925)(562.52537757,87.90462927)(562.54536865,87.88463186)
\lineto(562.75536865,87.73463186)
\curveto(562.81537728,87.69462948)(562.88037721,87.64962953)(562.95036865,87.59963186)
\curveto(563.04037705,87.53962964)(563.14537695,87.48962969)(563.26536865,87.44963186)
\curveto(563.37537672,87.41962976)(563.48537661,87.38462979)(563.59536865,87.34463186)
\curveto(563.70537639,87.30462987)(563.85037624,87.2796299)(564.03036865,87.26963186)
\curveto(564.20037589,87.25962992)(564.32537577,87.22962995)(564.40536865,87.17963186)
\curveto(564.48537561,87.12963005)(564.53037556,87.05463012)(564.54036865,86.95463186)
\curveto(564.55037554,86.85463032)(564.55537554,86.74463043)(564.55536865,86.62463186)
\curveto(564.55537554,86.58463059)(564.56037553,86.54463063)(564.57036865,86.50463186)
\curveto(564.57037552,86.46463071)(564.56537553,86.42963075)(564.55536865,86.39963186)
\curveto(564.53537556,86.34963083)(564.52537557,86.29963088)(564.52536865,86.24963186)
\curveto(564.52537557,86.20963097)(564.51537558,86.16963101)(564.49536865,86.12963186)
\curveto(564.43537566,86.03963114)(564.30037579,85.99463118)(564.09036865,85.99463186)
\lineto(563.97036865,85.99463186)
\curveto(563.91037618,86.00463117)(563.85037624,86.00963117)(563.79036865,86.00963186)
\curveto(563.72037637,86.01963116)(563.65537644,86.02963115)(563.59536865,86.03963186)
\curveto(563.48537661,86.05963112)(563.38537671,86.0796311)(563.29536865,86.09963186)
\curveto(563.1953769,86.11963106)(563.10037699,86.14963103)(563.01036865,86.18963186)
\curveto(562.94037715,86.20963097)(562.88037721,86.22963095)(562.83036865,86.24963186)
\lineto(562.65036865,86.30963186)
\curveto(562.3903777,86.42963075)(562.14537795,86.58463059)(561.91536865,86.77463186)
\curveto(561.68537841,86.9746302)(561.50037859,87.18962999)(561.36036865,87.41963186)
\curveto(561.28037881,87.52962965)(561.21537888,87.64462953)(561.16536865,87.76463186)
\lineto(561.01536865,88.15463186)
\curveto(560.96537913,88.26462891)(560.93537916,88.3796288)(560.92536865,88.49963186)
\curveto(560.90537919,88.61962856)(560.88037921,88.74462843)(560.85036865,88.87463186)
\curveto(560.85037924,88.94462823)(560.85037924,89.00962817)(560.85036865,89.06963186)
\curveto(560.84037925,89.12962805)(560.83037926,89.19462798)(560.82036865,89.26463186)
}
}
{
\newrgbcolor{curcolor}{0 0 0}
\pscustom[linestyle=none,fillstyle=solid,fillcolor=curcolor]
{
\newpath
\moveto(566.34036865,101.36424123)
\lineto(566.59536865,101.36424123)
\curveto(566.67537342,101.37423353)(566.75037334,101.36923353)(566.82036865,101.34924123)
\lineto(567.06036865,101.34924123)
\lineto(567.22536865,101.34924123)
\curveto(567.32537277,101.32923357)(567.43037266,101.31923358)(567.54036865,101.31924123)
\curveto(567.64037245,101.31923358)(567.74037235,101.30923359)(567.84036865,101.28924123)
\lineto(567.99036865,101.28924123)
\curveto(568.13037196,101.25923364)(568.27037182,101.23923366)(568.41036865,101.22924123)
\curveto(568.54037155,101.21923368)(568.67037142,101.19423371)(568.80036865,101.15424123)
\curveto(568.88037121,101.13423377)(568.96537113,101.11423379)(569.05536865,101.09424123)
\lineto(569.29536865,101.03424123)
\lineto(569.59536865,100.91424123)
\curveto(569.68537041,100.88423402)(569.77537032,100.84923405)(569.86536865,100.80924123)
\curveto(570.08537001,100.70923419)(570.30036979,100.57423433)(570.51036865,100.40424123)
\curveto(570.72036937,100.24423466)(570.8903692,100.06923483)(571.02036865,99.87924123)
\curveto(571.06036903,99.82923507)(571.10036899,99.76923513)(571.14036865,99.69924123)
\curveto(571.17036892,99.63923526)(571.20536889,99.57923532)(571.24536865,99.51924123)
\curveto(571.2953688,99.43923546)(571.33536876,99.34423556)(571.36536865,99.23424123)
\curveto(571.3953687,99.12423578)(571.42536867,99.01923588)(571.45536865,98.91924123)
\curveto(571.4953686,98.80923609)(571.52036857,98.6992362)(571.53036865,98.58924123)
\curveto(571.54036855,98.47923642)(571.55536854,98.36423654)(571.57536865,98.24424123)
\curveto(571.58536851,98.2042367)(571.58536851,98.15923674)(571.57536865,98.10924123)
\curveto(571.57536852,98.06923683)(571.58036851,98.02923687)(571.59036865,97.98924123)
\curveto(571.60036849,97.94923695)(571.60536849,97.89423701)(571.60536865,97.82424123)
\curveto(571.60536849,97.75423715)(571.60036849,97.7042372)(571.59036865,97.67424123)
\curveto(571.57036852,97.62423728)(571.56536853,97.57923732)(571.57536865,97.53924123)
\curveto(571.58536851,97.4992374)(571.58536851,97.46423744)(571.57536865,97.43424123)
\lineto(571.57536865,97.34424123)
\curveto(571.55536854,97.28423762)(571.54036855,97.21923768)(571.53036865,97.14924123)
\curveto(571.53036856,97.08923781)(571.52536857,97.02423788)(571.51536865,96.95424123)
\curveto(571.46536863,96.78423812)(571.41536868,96.62423828)(571.36536865,96.47424123)
\curveto(571.31536878,96.32423858)(571.25036884,96.17923872)(571.17036865,96.03924123)
\curveto(571.13036896,95.98923891)(571.10036899,95.93423897)(571.08036865,95.87424123)
\curveto(571.05036904,95.82423908)(571.01536908,95.77423913)(570.97536865,95.72424123)
\curveto(570.7953693,95.48423942)(570.57536952,95.28423962)(570.31536865,95.12424123)
\curveto(570.05537004,94.96423994)(569.77037032,94.82424008)(569.46036865,94.70424123)
\curveto(569.32037077,94.64424026)(569.18037091,94.5992403)(569.04036865,94.56924123)
\curveto(568.8903712,94.53924036)(568.73537136,94.5042404)(568.57536865,94.46424123)
\curveto(568.46537163,94.44424046)(568.35537174,94.42924047)(568.24536865,94.41924123)
\curveto(568.13537196,94.40924049)(568.02537207,94.39424051)(567.91536865,94.37424123)
\curveto(567.87537222,94.36424054)(567.83537226,94.35924054)(567.79536865,94.35924123)
\curveto(567.75537234,94.36924053)(567.71537238,94.36924053)(567.67536865,94.35924123)
\curveto(567.62537247,94.34924055)(567.57537252,94.34424056)(567.52536865,94.34424123)
\lineto(567.36036865,94.34424123)
\curveto(567.31037278,94.32424058)(567.26037283,94.31924058)(567.21036865,94.32924123)
\curveto(567.15037294,94.33924056)(567.095373,94.33924056)(567.04536865,94.32924123)
\curveto(567.00537309,94.31924058)(566.96037313,94.31924058)(566.91036865,94.32924123)
\curveto(566.86037323,94.33924056)(566.81037328,94.33424057)(566.76036865,94.31424123)
\curveto(566.6903734,94.29424061)(566.61537348,94.28924061)(566.53536865,94.29924123)
\curveto(566.44537365,94.30924059)(566.36037373,94.31424059)(566.28036865,94.31424123)
\curveto(566.1903739,94.31424059)(566.090374,94.30924059)(565.98036865,94.29924123)
\curveto(565.86037423,94.28924061)(565.76037433,94.29424061)(565.68036865,94.31424123)
\lineto(565.39536865,94.31424123)
\lineto(564.76536865,94.35924123)
\curveto(564.66537543,94.36924053)(564.57037552,94.37924052)(564.48036865,94.38924123)
\lineto(564.18036865,94.41924123)
\curveto(564.13037596,94.43924046)(564.08037601,94.44424046)(564.03036865,94.43424123)
\curveto(563.97037612,94.43424047)(563.91537618,94.44424046)(563.86536865,94.46424123)
\curveto(563.6953764,94.51424039)(563.53037656,94.55424035)(563.37036865,94.58424123)
\curveto(563.20037689,94.61424029)(563.04037705,94.66424024)(562.89036865,94.73424123)
\curveto(562.43037766,94.92423998)(562.05537804,95.14423976)(561.76536865,95.39424123)
\curveto(561.47537862,95.65423925)(561.23037886,96.01423889)(561.03036865,96.47424123)
\curveto(560.98037911,96.6042383)(560.94537915,96.73423817)(560.92536865,96.86424123)
\curveto(560.90537919,97.0042379)(560.88037921,97.14423776)(560.85036865,97.28424123)
\curveto(560.84037925,97.35423755)(560.83537926,97.41923748)(560.83536865,97.47924123)
\curveto(560.83537926,97.53923736)(560.83037926,97.6042373)(560.82036865,97.67424123)
\curveto(560.80037929,98.5042364)(560.95037914,99.17423573)(561.27036865,99.68424123)
\curveto(561.58037851,100.19423471)(562.02037807,100.57423433)(562.59036865,100.82424123)
\curveto(562.71037738,100.87423403)(562.83537726,100.91923398)(562.96536865,100.95924123)
\curveto(563.095377,100.9992339)(563.23037686,101.04423386)(563.37036865,101.09424123)
\curveto(563.45037664,101.11423379)(563.53537656,101.12923377)(563.62536865,101.13924123)
\lineto(563.86536865,101.19924123)
\curveto(563.97537612,101.22923367)(564.08537601,101.24423366)(564.19536865,101.24424123)
\curveto(564.30537579,101.25423365)(564.41537568,101.26923363)(564.52536865,101.28924123)
\curveto(564.57537552,101.30923359)(564.62037547,101.31423359)(564.66036865,101.30424123)
\curveto(564.70037539,101.3042336)(564.74037535,101.30923359)(564.78036865,101.31924123)
\curveto(564.83037526,101.32923357)(564.88537521,101.32923357)(564.94536865,101.31924123)
\curveto(564.9953751,101.31923358)(565.04537505,101.32423358)(565.09536865,101.33424123)
\lineto(565.23036865,101.33424123)
\curveto(565.2903748,101.35423355)(565.36037473,101.35423355)(565.44036865,101.33424123)
\curveto(565.51037458,101.32423358)(565.57537452,101.32923357)(565.63536865,101.34924123)
\curveto(565.66537443,101.35923354)(565.70537439,101.36423354)(565.75536865,101.36424123)
\lineto(565.87536865,101.36424123)
\lineto(566.34036865,101.36424123)
\moveto(568.66536865,99.81924123)
\curveto(568.34537175,99.91923498)(567.98037211,99.97923492)(567.57036865,99.99924123)
\curveto(567.16037293,100.01923488)(566.75037334,100.02923487)(566.34036865,100.02924123)
\curveto(565.91037418,100.02923487)(565.4903746,100.01923488)(565.08036865,99.99924123)
\curveto(564.67037542,99.97923492)(564.28537581,99.93423497)(563.92536865,99.86424123)
\curveto(563.56537653,99.79423511)(563.24537685,99.68423522)(562.96536865,99.53424123)
\curveto(562.67537742,99.39423551)(562.44037765,99.1992357)(562.26036865,98.94924123)
\curveto(562.15037794,98.78923611)(562.07037802,98.60923629)(562.02036865,98.40924123)
\curveto(561.96037813,98.20923669)(561.93037816,97.96423694)(561.93036865,97.67424123)
\curveto(561.95037814,97.65423725)(561.96037813,97.61923728)(561.96036865,97.56924123)
\curveto(561.95037814,97.51923738)(561.95037814,97.47923742)(561.96036865,97.44924123)
\curveto(561.98037811,97.36923753)(562.00037809,97.29423761)(562.02036865,97.22424123)
\curveto(562.03037806,97.16423774)(562.05037804,97.0992378)(562.08036865,97.02924123)
\curveto(562.20037789,96.75923814)(562.37037772,96.53923836)(562.59036865,96.36924123)
\curveto(562.80037729,96.20923869)(563.04537705,96.07423883)(563.32536865,95.96424123)
\curveto(563.43537666,95.91423899)(563.55537654,95.87423903)(563.68536865,95.84424123)
\curveto(563.80537629,95.82423908)(563.93037616,95.7992391)(564.06036865,95.76924123)
\curveto(564.11037598,95.74923915)(564.16537593,95.73923916)(564.22536865,95.73924123)
\curveto(564.27537582,95.73923916)(564.32537577,95.73423917)(564.37536865,95.72424123)
\curveto(564.46537563,95.71423919)(564.56037553,95.7042392)(564.66036865,95.69424123)
\curveto(564.75037534,95.68423922)(564.84537525,95.67423923)(564.94536865,95.66424123)
\curveto(565.02537507,95.66423924)(565.11037498,95.65923924)(565.20036865,95.64924123)
\lineto(565.44036865,95.64924123)
\lineto(565.62036865,95.64924123)
\curveto(565.65037444,95.63923926)(565.68537441,95.63423927)(565.72536865,95.63424123)
\lineto(565.86036865,95.63424123)
\lineto(566.31036865,95.63424123)
\curveto(566.3903737,95.63423927)(566.47537362,95.62923927)(566.56536865,95.61924123)
\curveto(566.64537345,95.61923928)(566.72037337,95.62923927)(566.79036865,95.64924123)
\lineto(567.06036865,95.64924123)
\curveto(567.08037301,95.64923925)(567.11037298,95.64423926)(567.15036865,95.63424123)
\curveto(567.18037291,95.63423927)(567.20537289,95.63923926)(567.22536865,95.64924123)
\curveto(567.32537277,95.65923924)(567.42537267,95.66423924)(567.52536865,95.66424123)
\curveto(567.61537248,95.67423923)(567.71537238,95.68423922)(567.82536865,95.69424123)
\curveto(567.94537215,95.72423918)(568.07037202,95.73923916)(568.20036865,95.73924123)
\curveto(568.32037177,95.74923915)(568.43537166,95.77423913)(568.54536865,95.81424123)
\curveto(568.84537125,95.89423901)(569.11037098,95.97923892)(569.34036865,96.06924123)
\curveto(569.57037052,96.16923873)(569.78537031,96.31423859)(569.98536865,96.50424123)
\curveto(570.18536991,96.71423819)(570.33536976,96.97923792)(570.43536865,97.29924123)
\curveto(570.45536964,97.33923756)(570.46536963,97.37423753)(570.46536865,97.40424123)
\curveto(570.45536964,97.44423746)(570.46036963,97.48923741)(570.48036865,97.53924123)
\curveto(570.4903696,97.57923732)(570.50036959,97.64923725)(570.51036865,97.74924123)
\curveto(570.52036957,97.85923704)(570.51536958,97.94423696)(570.49536865,98.00424123)
\curveto(570.47536962,98.07423683)(570.46536963,98.14423676)(570.46536865,98.21424123)
\curveto(570.45536964,98.28423662)(570.44036965,98.34923655)(570.42036865,98.40924123)
\curveto(570.36036973,98.60923629)(570.27536982,98.78923611)(570.16536865,98.94924123)
\curveto(570.14536995,98.97923592)(570.12536997,99.0042359)(570.10536865,99.02424123)
\lineto(570.04536865,99.08424123)
\curveto(570.02537007,99.12423578)(569.98537011,99.17423573)(569.92536865,99.23424123)
\curveto(569.78537031,99.33423557)(569.65537044,99.41923548)(569.53536865,99.48924123)
\curveto(569.41537068,99.55923534)(569.27037082,99.62923527)(569.10036865,99.69924123)
\curveto(569.03037106,99.72923517)(568.96037113,99.74923515)(568.89036865,99.75924123)
\curveto(568.82037127,99.77923512)(568.74537135,99.7992351)(568.66536865,99.81924123)
}
}
{
\newrgbcolor{curcolor}{0 0 0}
\pscustom[linestyle=none,fillstyle=solid,fillcolor=curcolor]
{
\newpath
\moveto(560.82036865,106.77385061)
\curveto(560.82037927,106.87384575)(560.83037926,106.96884566)(560.85036865,107.05885061)
\curveto(560.86037923,107.14884548)(560.8903792,107.21384541)(560.94036865,107.25385061)
\curveto(561.02037907,107.31384531)(561.12537897,107.34384528)(561.25536865,107.34385061)
\lineto(561.64536865,107.34385061)
\lineto(563.14536865,107.34385061)
\lineto(569.53536865,107.34385061)
\lineto(570.70536865,107.34385061)
\lineto(571.02036865,107.34385061)
\curveto(571.12036897,107.35384527)(571.20036889,107.33884529)(571.26036865,107.29885061)
\curveto(571.34036875,107.24884538)(571.3903687,107.17384545)(571.41036865,107.07385061)
\curveto(571.42036867,106.98384564)(571.42536867,106.87384575)(571.42536865,106.74385061)
\lineto(571.42536865,106.51885061)
\curveto(571.40536869,106.43884619)(571.3903687,106.36884626)(571.38036865,106.30885061)
\curveto(571.36036873,106.24884638)(571.32036877,106.19884643)(571.26036865,106.15885061)
\curveto(571.20036889,106.11884651)(571.12536897,106.09884653)(571.03536865,106.09885061)
\lineto(570.73536865,106.09885061)
\lineto(569.64036865,106.09885061)
\lineto(564.30036865,106.09885061)
\curveto(564.21037588,106.07884655)(564.13537596,106.06384656)(564.07536865,106.05385061)
\curveto(564.00537609,106.05384657)(563.94537615,106.0238466)(563.89536865,105.96385061)
\curveto(563.84537625,105.89384673)(563.82037627,105.80384682)(563.82036865,105.69385061)
\curveto(563.81037628,105.59384703)(563.80537629,105.48384714)(563.80536865,105.36385061)
\lineto(563.80536865,104.22385061)
\lineto(563.80536865,103.72885061)
\curveto(563.7953763,103.56884906)(563.73537636,103.45884917)(563.62536865,103.39885061)
\curveto(563.5953765,103.37884925)(563.56537653,103.36884926)(563.53536865,103.36885061)
\curveto(563.4953766,103.36884926)(563.45037664,103.36384926)(563.40036865,103.35385061)
\curveto(563.28037681,103.33384929)(563.17037692,103.33884929)(563.07036865,103.36885061)
\curveto(562.97037712,103.40884922)(562.90037719,103.46384916)(562.86036865,103.53385061)
\curveto(562.81037728,103.61384901)(562.78537731,103.73384889)(562.78536865,103.89385061)
\curveto(562.78537731,104.05384857)(562.77037732,104.18884844)(562.74036865,104.29885061)
\curveto(562.73037736,104.34884828)(562.72537737,104.40384822)(562.72536865,104.46385061)
\curveto(562.71537738,104.5238481)(562.70037739,104.58384804)(562.68036865,104.64385061)
\curveto(562.63037746,104.79384783)(562.58037751,104.93884769)(562.53036865,105.07885061)
\curveto(562.47037762,105.21884741)(562.40037769,105.35384727)(562.32036865,105.48385061)
\curveto(562.23037786,105.623847)(562.12537797,105.74384688)(562.00536865,105.84385061)
\curveto(561.88537821,105.94384668)(561.75537834,106.03884659)(561.61536865,106.12885061)
\curveto(561.51537858,106.18884644)(561.40537869,106.23384639)(561.28536865,106.26385061)
\curveto(561.16537893,106.30384632)(561.06037903,106.35384627)(560.97036865,106.41385061)
\curveto(560.91037918,106.46384616)(560.87037922,106.53384609)(560.85036865,106.62385061)
\curveto(560.84037925,106.64384598)(560.83537926,106.66884596)(560.83536865,106.69885061)
\curveto(560.83537926,106.7288459)(560.83037926,106.75384587)(560.82036865,106.77385061)
}
}
{
\newrgbcolor{curcolor}{0 0 0}
\pscustom[linestyle=none,fillstyle=solid,fillcolor=curcolor]
{
\newpath
\moveto(560.82036865,115.12345998)
\curveto(560.82037927,115.22345513)(560.83037926,115.31845503)(560.85036865,115.40845998)
\curveto(560.86037923,115.49845485)(560.8903792,115.56345479)(560.94036865,115.60345998)
\curveto(561.02037907,115.66345469)(561.12537897,115.69345466)(561.25536865,115.69345998)
\lineto(561.64536865,115.69345998)
\lineto(563.14536865,115.69345998)
\lineto(569.53536865,115.69345998)
\lineto(570.70536865,115.69345998)
\lineto(571.02036865,115.69345998)
\curveto(571.12036897,115.70345465)(571.20036889,115.68845466)(571.26036865,115.64845998)
\curveto(571.34036875,115.59845475)(571.3903687,115.52345483)(571.41036865,115.42345998)
\curveto(571.42036867,115.33345502)(571.42536867,115.22345513)(571.42536865,115.09345998)
\lineto(571.42536865,114.86845998)
\curveto(571.40536869,114.78845556)(571.3903687,114.71845563)(571.38036865,114.65845998)
\curveto(571.36036873,114.59845575)(571.32036877,114.5484558)(571.26036865,114.50845998)
\curveto(571.20036889,114.46845588)(571.12536897,114.4484559)(571.03536865,114.44845998)
\lineto(570.73536865,114.44845998)
\lineto(569.64036865,114.44845998)
\lineto(564.30036865,114.44845998)
\curveto(564.21037588,114.42845592)(564.13537596,114.41345594)(564.07536865,114.40345998)
\curveto(564.00537609,114.40345595)(563.94537615,114.37345598)(563.89536865,114.31345998)
\curveto(563.84537625,114.24345611)(563.82037627,114.1534562)(563.82036865,114.04345998)
\curveto(563.81037628,113.94345641)(563.80537629,113.83345652)(563.80536865,113.71345998)
\lineto(563.80536865,112.57345998)
\lineto(563.80536865,112.07845998)
\curveto(563.7953763,111.91845843)(563.73537636,111.80845854)(563.62536865,111.74845998)
\curveto(563.5953765,111.72845862)(563.56537653,111.71845863)(563.53536865,111.71845998)
\curveto(563.4953766,111.71845863)(563.45037664,111.71345864)(563.40036865,111.70345998)
\curveto(563.28037681,111.68345867)(563.17037692,111.68845866)(563.07036865,111.71845998)
\curveto(562.97037712,111.75845859)(562.90037719,111.81345854)(562.86036865,111.88345998)
\curveto(562.81037728,111.96345839)(562.78537731,112.08345827)(562.78536865,112.24345998)
\curveto(562.78537731,112.40345795)(562.77037732,112.53845781)(562.74036865,112.64845998)
\curveto(562.73037736,112.69845765)(562.72537737,112.7534576)(562.72536865,112.81345998)
\curveto(562.71537738,112.87345748)(562.70037739,112.93345742)(562.68036865,112.99345998)
\curveto(562.63037746,113.14345721)(562.58037751,113.28845706)(562.53036865,113.42845998)
\curveto(562.47037762,113.56845678)(562.40037769,113.70345665)(562.32036865,113.83345998)
\curveto(562.23037786,113.97345638)(562.12537797,114.09345626)(562.00536865,114.19345998)
\curveto(561.88537821,114.29345606)(561.75537834,114.38845596)(561.61536865,114.47845998)
\curveto(561.51537858,114.53845581)(561.40537869,114.58345577)(561.28536865,114.61345998)
\curveto(561.16537893,114.6534557)(561.06037903,114.70345565)(560.97036865,114.76345998)
\curveto(560.91037918,114.81345554)(560.87037922,114.88345547)(560.85036865,114.97345998)
\curveto(560.84037925,114.99345536)(560.83537926,115.01845533)(560.83536865,115.04845998)
\curveto(560.83537926,115.07845527)(560.83037926,115.10345525)(560.82036865,115.12345998)
}
}
{
\newrgbcolor{curcolor}{0 0 0}
\pscustom[linestyle=none,fillstyle=solid,fillcolor=curcolor]
{
\newpath
\moveto(582.69165771,29.18119436)
\lineto(582.69165771,30.09619436)
\curveto(582.69166841,30.19619171)(582.69166841,30.29119161)(582.69165771,30.38119436)
\curveto(582.69166841,30.47119143)(582.71166839,30.54619136)(582.75165771,30.60619436)
\curveto(582.81166829,30.69619121)(582.89166821,30.75619115)(582.99165771,30.78619436)
\curveto(583.09166801,30.82619108)(583.1966679,30.87119103)(583.30665771,30.92119436)
\curveto(583.4966676,31.0011909)(583.68666741,31.07119083)(583.87665771,31.13119436)
\curveto(584.06666703,31.2011907)(584.25666684,31.27619063)(584.44665771,31.35619436)
\curveto(584.62666647,31.42619048)(584.81166629,31.49119041)(585.00165771,31.55119436)
\curveto(585.18166592,31.61119029)(585.36166574,31.68119022)(585.54165771,31.76119436)
\curveto(585.68166542,31.82119008)(585.82666527,31.87619003)(585.97665771,31.92619436)
\curveto(586.12666497,31.97618993)(586.27166483,32.03118987)(586.41165771,32.09119436)
\curveto(586.86166424,32.27118963)(587.31666378,32.44118946)(587.77665771,32.60119436)
\curveto(588.22666287,32.76118914)(588.67666242,32.93118897)(589.12665771,33.11119436)
\curveto(589.17666192,33.13118877)(589.22666187,33.14618876)(589.27665771,33.15619436)
\lineto(589.42665771,33.21619436)
\curveto(589.64666145,33.3061886)(589.87166123,33.39118851)(590.10165771,33.47119436)
\curveto(590.32166078,33.55118835)(590.54166056,33.63618827)(590.76165771,33.72619436)
\curveto(590.85166025,33.76618814)(590.96166014,33.8061881)(591.09165771,33.84619436)
\curveto(591.21165989,33.88618802)(591.28165982,33.95118795)(591.30165771,34.04119436)
\curveto(591.31165979,34.08118782)(591.31165979,34.11118779)(591.30165771,34.13119436)
\lineto(591.24165771,34.19119436)
\curveto(591.19165991,34.24118766)(591.13665996,34.27618763)(591.07665771,34.29619436)
\curveto(591.01666008,34.32618758)(590.95166015,34.35618755)(590.88165771,34.38619436)
\lineto(590.25165771,34.62619436)
\curveto(590.03166107,34.7061872)(589.81666128,34.78618712)(589.60665771,34.86619436)
\lineto(589.45665771,34.92619436)
\lineto(589.27665771,34.98619436)
\curveto(589.08666201,35.06618684)(588.8966622,35.13618677)(588.70665771,35.19619436)
\curveto(588.50666259,35.26618664)(588.30666279,35.34118656)(588.10665771,35.42119436)
\curveto(587.52666357,35.66118624)(586.94166416,35.88118602)(586.35165771,36.08119436)
\curveto(585.76166534,36.29118561)(585.17666592,36.51618539)(584.59665771,36.75619436)
\curveto(584.3966667,36.83618507)(584.19166691,36.91118499)(583.98165771,36.98119436)
\curveto(583.77166733,37.06118484)(583.56666753,37.14118476)(583.36665771,37.22119436)
\curveto(583.28666781,37.26118464)(583.18666791,37.29618461)(583.06665771,37.32619436)
\curveto(582.94666815,37.36618454)(582.86166824,37.42118448)(582.81165771,37.49119436)
\curveto(582.77166833,37.55118435)(582.74166836,37.62618428)(582.72165771,37.71619436)
\curveto(582.7016684,37.81618409)(582.69166841,37.92618398)(582.69165771,38.04619436)
\curveto(582.68166842,38.16618374)(582.68166842,38.28618362)(582.69165771,38.40619436)
\curveto(582.69166841,38.52618338)(582.69166841,38.63618327)(582.69165771,38.73619436)
\curveto(582.69166841,38.82618308)(582.69166841,38.91618299)(582.69165771,39.00619436)
\curveto(582.69166841,39.1061828)(582.71166839,39.18118272)(582.75165771,39.23119436)
\curveto(582.8016683,39.32118258)(582.89166821,39.37118253)(583.02165771,39.38119436)
\curveto(583.15166795,39.39118251)(583.29166781,39.39618251)(583.44165771,39.39619436)
\lineto(585.09165771,39.39619436)
\lineto(591.36165771,39.39619436)
\lineto(592.62165771,39.39619436)
\curveto(592.73165837,39.39618251)(592.84165826,39.39618251)(592.95165771,39.39619436)
\curveto(593.06165804,39.4061825)(593.14665795,39.38618252)(593.20665771,39.33619436)
\curveto(593.26665783,39.3061826)(593.30665779,39.26118264)(593.32665771,39.20119436)
\curveto(593.33665776,39.14118276)(593.35165775,39.07118283)(593.37165771,38.99119436)
\lineto(593.37165771,38.75119436)
\lineto(593.37165771,38.39119436)
\curveto(593.36165774,38.28118362)(593.31665778,38.2011837)(593.23665771,38.15119436)
\curveto(593.20665789,38.13118377)(593.17665792,38.11618379)(593.14665771,38.10619436)
\curveto(593.10665799,38.1061838)(593.06165804,38.09618381)(593.01165771,38.07619436)
\lineto(592.84665771,38.07619436)
\curveto(592.78665831,38.06618384)(592.71665838,38.06118384)(592.63665771,38.06119436)
\curveto(592.55665854,38.07118383)(592.48165862,38.07618383)(592.41165771,38.07619436)
\lineto(591.57165771,38.07619436)
\lineto(587.14665771,38.07619436)
\curveto(586.8966642,38.07618383)(586.64666445,38.07618383)(586.39665771,38.07619436)
\curveto(586.13666496,38.07618383)(585.88666521,38.07118383)(585.64665771,38.06119436)
\curveto(585.54666555,38.06118384)(585.43666566,38.05618385)(585.31665771,38.04619436)
\curveto(585.1966659,38.03618387)(585.13666596,37.98118392)(585.13665771,37.88119436)
\lineto(585.15165771,37.88119436)
\curveto(585.17166593,37.81118409)(585.23666586,37.75118415)(585.34665771,37.70119436)
\curveto(585.45666564,37.66118424)(585.55166555,37.62618428)(585.63165771,37.59619436)
\curveto(585.8016653,37.52618438)(585.97666512,37.46118444)(586.15665771,37.40119436)
\curveto(586.32666477,37.34118456)(586.4966646,37.27118463)(586.66665771,37.19119436)
\curveto(586.71666438,37.17118473)(586.76166434,37.15618475)(586.80165771,37.14619436)
\curveto(586.84166426,37.13618477)(586.88666421,37.12118478)(586.93665771,37.10119436)
\curveto(587.11666398,37.02118488)(587.3016638,36.95118495)(587.49165771,36.89119436)
\curveto(587.67166343,36.84118506)(587.85166325,36.77618513)(588.03165771,36.69619436)
\curveto(588.18166292,36.62618528)(588.33666276,36.56618534)(588.49665771,36.51619436)
\curveto(588.64666245,36.46618544)(588.7966623,36.41118549)(588.94665771,36.35119436)
\curveto(589.41666168,36.15118575)(589.89166121,35.97118593)(590.37165771,35.81119436)
\curveto(590.84166026,35.65118625)(591.30665979,35.47618643)(591.76665771,35.28619436)
\curveto(591.94665915,35.2061867)(592.12665897,35.13618677)(592.30665771,35.07619436)
\curveto(592.48665861,35.01618689)(592.66665843,34.95118695)(592.84665771,34.88119436)
\curveto(592.95665814,34.83118707)(593.06165804,34.78118712)(593.16165771,34.73119436)
\curveto(593.25165785,34.69118721)(593.31665778,34.6061873)(593.35665771,34.47619436)
\curveto(593.36665773,34.45618745)(593.37165773,34.43118747)(593.37165771,34.40119436)
\curveto(593.36165774,34.38118752)(593.36165774,34.35618755)(593.37165771,34.32619436)
\curveto(593.38165772,34.29618761)(593.38665771,34.26118764)(593.38665771,34.22119436)
\curveto(593.37665772,34.18118772)(593.37165773,34.14118776)(593.37165771,34.10119436)
\lineto(593.37165771,33.80119436)
\curveto(593.37165773,33.7011882)(593.34665775,33.62118828)(593.29665771,33.56119436)
\curveto(593.24665785,33.48118842)(593.17665792,33.42118848)(593.08665771,33.38119436)
\curveto(592.98665811,33.35118855)(592.88665821,33.31118859)(592.78665771,33.26119436)
\curveto(592.58665851,33.18118872)(592.38165872,33.1011888)(592.17165771,33.02119436)
\curveto(591.95165915,32.95118895)(591.74165936,32.87618903)(591.54165771,32.79619436)
\curveto(591.36165974,32.71618919)(591.18165992,32.64618926)(591.00165771,32.58619436)
\curveto(590.81166029,32.53618937)(590.62666047,32.47118943)(590.44665771,32.39119436)
\curveto(589.88666121,32.16118974)(589.32166178,31.94618996)(588.75165771,31.74619436)
\curveto(588.18166292,31.54619036)(587.61666348,31.33119057)(587.05665771,31.10119436)
\lineto(586.42665771,30.86119436)
\curveto(586.20666489,30.79119111)(585.9966651,30.71619119)(585.79665771,30.63619436)
\curveto(585.68666541,30.58619132)(585.58166552,30.54119136)(585.48165771,30.50119436)
\curveto(585.37166573,30.47119143)(585.27666582,30.42119148)(585.19665771,30.35119436)
\curveto(585.17666592,30.34119156)(585.16666593,30.33119157)(585.16665771,30.32119436)
\lineto(585.13665771,30.29119436)
\lineto(585.13665771,30.21619436)
\lineto(585.16665771,30.18619436)
\curveto(585.16666593,30.17619173)(585.17166593,30.16619174)(585.18165771,30.15619436)
\curveto(585.23166587,30.13619177)(585.28666581,30.12619178)(585.34665771,30.12619436)
\curveto(585.40666569,30.12619178)(585.46666563,30.11619179)(585.52665771,30.09619436)
\lineto(585.69165771,30.09619436)
\curveto(585.75166535,30.07619183)(585.81666528,30.07119183)(585.88665771,30.08119436)
\curveto(585.95666514,30.09119181)(586.02666507,30.09619181)(586.09665771,30.09619436)
\lineto(586.90665771,30.09619436)
\lineto(591.46665771,30.09619436)
\lineto(592.65165771,30.09619436)
\curveto(592.76165834,30.09619181)(592.87165823,30.09119181)(592.98165771,30.08119436)
\curveto(593.09165801,30.08119182)(593.17665792,30.05619185)(593.23665771,30.00619436)
\curveto(593.31665778,29.95619195)(593.36165774,29.86619204)(593.37165771,29.73619436)
\lineto(593.37165771,29.34619436)
\lineto(593.37165771,29.15119436)
\curveto(593.37165773,29.1011928)(593.36165774,29.05119285)(593.34165771,29.00119436)
\curveto(593.3016578,28.87119303)(593.21665788,28.79619311)(593.08665771,28.77619436)
\curveto(592.95665814,28.76619314)(592.80665829,28.76119314)(592.63665771,28.76119436)
\lineto(590.89665771,28.76119436)
\lineto(584.89665771,28.76119436)
\lineto(583.48665771,28.76119436)
\curveto(583.37666772,28.76119314)(583.26166784,28.75619315)(583.14165771,28.74619436)
\curveto(583.02166808,28.74619316)(582.92666817,28.77119313)(582.85665771,28.82119436)
\curveto(582.7966683,28.86119304)(582.74666835,28.93619297)(582.70665771,29.04619436)
\curveto(582.6966684,29.06619284)(582.6966684,29.08619282)(582.70665771,29.10619436)
\curveto(582.70666839,29.13619277)(582.7016684,29.16119274)(582.69165771,29.18119436)
}
}
{
\newrgbcolor{curcolor}{0 0 0}
\pscustom[linestyle=none,fillstyle=solid,fillcolor=curcolor]
{
\newpath
\moveto(592.81665771,48.38330373)
\curveto(592.97665812,48.4132959)(593.11165799,48.39829592)(593.22165771,48.33830373)
\curveto(593.32165778,48.27829604)(593.3966577,48.19829612)(593.44665771,48.09830373)
\curveto(593.46665763,48.04829627)(593.47665762,47.99329632)(593.47665771,47.93330373)
\curveto(593.47665762,47.88329643)(593.48665761,47.82829649)(593.50665771,47.76830373)
\curveto(593.55665754,47.54829677)(593.54165756,47.32829699)(593.46165771,47.10830373)
\curveto(593.39165771,46.89829742)(593.3016578,46.75329756)(593.19165771,46.67330373)
\curveto(593.12165798,46.62329769)(593.04165806,46.57829774)(592.95165771,46.53830373)
\curveto(592.85165825,46.49829782)(592.77165833,46.44829787)(592.71165771,46.38830373)
\curveto(592.69165841,46.36829795)(592.67165843,46.34329797)(592.65165771,46.31330373)
\curveto(592.63165847,46.29329802)(592.62665847,46.26329805)(592.63665771,46.22330373)
\curveto(592.66665843,46.1132982)(592.72165838,46.00829831)(592.80165771,45.90830373)
\curveto(592.88165822,45.8182985)(592.95165815,45.72829859)(593.01165771,45.63830373)
\curveto(593.09165801,45.50829881)(593.16665793,45.36829895)(593.23665771,45.21830373)
\curveto(593.2966578,45.06829925)(593.35165775,44.90829941)(593.40165771,44.73830373)
\curveto(593.43165767,44.63829968)(593.45165765,44.52829979)(593.46165771,44.40830373)
\curveto(593.47165763,44.29830002)(593.48665761,44.18830013)(593.50665771,44.07830373)
\curveto(593.51665758,44.02830029)(593.52165758,43.98330033)(593.52165771,43.94330373)
\lineto(593.52165771,43.83830373)
\curveto(593.54165756,43.72830059)(593.54165756,43.62330069)(593.52165771,43.52330373)
\lineto(593.52165771,43.38830373)
\curveto(593.51165759,43.33830098)(593.50665759,43.28830103)(593.50665771,43.23830373)
\curveto(593.50665759,43.18830113)(593.4966576,43.14330117)(593.47665771,43.10330373)
\curveto(593.46665763,43.06330125)(593.46165764,43.02830129)(593.46165771,42.99830373)
\curveto(593.47165763,42.97830134)(593.47165763,42.95330136)(593.46165771,42.92330373)
\lineto(593.40165771,42.68330373)
\curveto(593.39165771,42.60330171)(593.37165773,42.52830179)(593.34165771,42.45830373)
\curveto(593.21165789,42.15830216)(593.06665803,41.9133024)(592.90665771,41.72330373)
\curveto(592.73665836,41.54330277)(592.5016586,41.39330292)(592.20165771,41.27330373)
\curveto(591.98165912,41.18330313)(591.71665938,41.13830318)(591.40665771,41.13830373)
\lineto(591.09165771,41.13830373)
\curveto(591.04166006,41.14830317)(590.99166011,41.15330316)(590.94165771,41.15330373)
\lineto(590.76165771,41.18330373)
\lineto(590.43165771,41.30330373)
\curveto(590.32166078,41.34330297)(590.22166088,41.39330292)(590.13165771,41.45330373)
\curveto(589.84166126,41.63330268)(589.62666147,41.87830244)(589.48665771,42.18830373)
\curveto(589.34666175,42.49830182)(589.22166188,42.83830148)(589.11165771,43.20830373)
\curveto(589.07166203,43.34830097)(589.04166206,43.49330082)(589.02165771,43.64330373)
\curveto(589.0016621,43.79330052)(588.97666212,43.94330037)(588.94665771,44.09330373)
\curveto(588.92666217,44.16330015)(588.91666218,44.22830009)(588.91665771,44.28830373)
\curveto(588.91666218,44.35829996)(588.90666219,44.43329988)(588.88665771,44.51330373)
\curveto(588.86666223,44.58329973)(588.85666224,44.65329966)(588.85665771,44.72330373)
\curveto(588.84666225,44.79329952)(588.83166227,44.86829945)(588.81165771,44.94830373)
\curveto(588.75166235,45.19829912)(588.7016624,45.43329888)(588.66165771,45.65330373)
\curveto(588.61166249,45.87329844)(588.4966626,46.04829827)(588.31665771,46.17830373)
\curveto(588.23666286,46.23829808)(588.13666296,46.28829803)(588.01665771,46.32830373)
\curveto(587.88666321,46.36829795)(587.74666335,46.36829795)(587.59665771,46.32830373)
\curveto(587.35666374,46.26829805)(587.16666393,46.17829814)(587.02665771,46.05830373)
\curveto(586.88666421,45.94829837)(586.77666432,45.78829853)(586.69665771,45.57830373)
\curveto(586.64666445,45.45829886)(586.61166449,45.313299)(586.59165771,45.14330373)
\curveto(586.57166453,44.98329933)(586.56166454,44.8132995)(586.56165771,44.63330373)
\curveto(586.56166454,44.45329986)(586.57166453,44.27830004)(586.59165771,44.10830373)
\curveto(586.61166449,43.93830038)(586.64166446,43.79330052)(586.68165771,43.67330373)
\curveto(586.74166436,43.50330081)(586.82666427,43.33830098)(586.93665771,43.17830373)
\curveto(586.9966641,43.09830122)(587.07666402,43.02330129)(587.17665771,42.95330373)
\curveto(587.26666383,42.89330142)(587.36666373,42.83830148)(587.47665771,42.78830373)
\curveto(587.55666354,42.75830156)(587.64166346,42.72830159)(587.73165771,42.69830373)
\curveto(587.82166328,42.67830164)(587.89166321,42.63330168)(587.94165771,42.56330373)
\curveto(587.97166313,42.52330179)(587.9966631,42.45330186)(588.01665771,42.35330373)
\curveto(588.02666307,42.26330205)(588.03166307,42.16830215)(588.03165771,42.06830373)
\curveto(588.03166307,41.96830235)(588.02666307,41.86830245)(588.01665771,41.76830373)
\curveto(587.9966631,41.67830264)(587.97166313,41.6133027)(587.94165771,41.57330373)
\curveto(587.91166319,41.53330278)(587.86166324,41.50330281)(587.79165771,41.48330373)
\curveto(587.72166338,41.46330285)(587.64666345,41.46330285)(587.56665771,41.48330373)
\curveto(587.43666366,41.5133028)(587.31666378,41.54330277)(587.20665771,41.57330373)
\curveto(587.08666401,41.6133027)(586.97166413,41.65830266)(586.86165771,41.70830373)
\curveto(586.51166459,41.89830242)(586.24166486,42.13830218)(586.05165771,42.42830373)
\curveto(585.85166525,42.7183016)(585.69166541,43.07830124)(585.57165771,43.50830373)
\curveto(585.55166555,43.60830071)(585.53666556,43.70830061)(585.52665771,43.80830373)
\curveto(585.51666558,43.9183004)(585.5016656,44.02830029)(585.48165771,44.13830373)
\curveto(585.47166563,44.17830014)(585.47166563,44.24330007)(585.48165771,44.33330373)
\curveto(585.48166562,44.42329989)(585.47166563,44.47829984)(585.45165771,44.49830373)
\curveto(585.44166566,45.19829912)(585.52166558,45.80829851)(585.69165771,46.32830373)
\curveto(585.86166524,46.84829747)(586.18666491,47.2132971)(586.66665771,47.42330373)
\curveto(586.86666423,47.5132968)(587.101664,47.56329675)(587.37165771,47.57330373)
\curveto(587.63166347,47.59329672)(587.90666319,47.60329671)(588.19665771,47.60330373)
\lineto(591.51165771,47.60330373)
\curveto(591.65165945,47.60329671)(591.78665931,47.60829671)(591.91665771,47.61830373)
\curveto(592.04665905,47.62829669)(592.15165895,47.65829666)(592.23165771,47.70830373)
\curveto(592.3016588,47.75829656)(592.35165875,47.82329649)(592.38165771,47.90330373)
\curveto(592.42165868,47.99329632)(592.45165865,48.07829624)(592.47165771,48.15830373)
\curveto(592.48165862,48.23829608)(592.52665857,48.29829602)(592.60665771,48.33830373)
\curveto(592.63665846,48.35829596)(592.66665843,48.36829595)(592.69665771,48.36830373)
\curveto(592.72665837,48.36829595)(592.76665833,48.37329594)(592.81665771,48.38330373)
\moveto(591.15165771,46.23830373)
\curveto(591.01166009,46.29829802)(590.85166025,46.32829799)(590.67165771,46.32830373)
\curveto(590.48166062,46.33829798)(590.28666081,46.34329797)(590.08665771,46.34330373)
\curveto(589.97666112,46.34329797)(589.87666122,46.33829798)(589.78665771,46.32830373)
\curveto(589.6966614,46.318298)(589.62666147,46.27829804)(589.57665771,46.20830373)
\curveto(589.55666154,46.17829814)(589.54666155,46.10829821)(589.54665771,45.99830373)
\curveto(589.56666153,45.97829834)(589.57666152,45.94329837)(589.57665771,45.89330373)
\curveto(589.57666152,45.84329847)(589.58666151,45.79829852)(589.60665771,45.75830373)
\curveto(589.62666147,45.67829864)(589.64666145,45.58829873)(589.66665771,45.48830373)
\lineto(589.72665771,45.18830373)
\curveto(589.72666137,45.15829916)(589.73166137,45.12329919)(589.74165771,45.08330373)
\lineto(589.74165771,44.97830373)
\curveto(589.78166132,44.82829949)(589.80666129,44.66329965)(589.81665771,44.48330373)
\curveto(589.81666128,44.3133)(589.83666126,44.15330016)(589.87665771,44.00330373)
\curveto(589.8966612,43.92330039)(589.91666118,43.84830047)(589.93665771,43.77830373)
\curveto(589.94666115,43.7183006)(589.96166114,43.64830067)(589.98165771,43.56830373)
\curveto(590.03166107,43.40830091)(590.096661,43.25830106)(590.17665771,43.11830373)
\curveto(590.24666085,42.97830134)(590.33666076,42.85830146)(590.44665771,42.75830373)
\curveto(590.55666054,42.65830166)(590.69166041,42.58330173)(590.85165771,42.53330373)
\curveto(591.0016601,42.48330183)(591.18665991,42.46330185)(591.40665771,42.47330373)
\curveto(591.50665959,42.47330184)(591.6016595,42.48830183)(591.69165771,42.51830373)
\curveto(591.77165933,42.55830176)(591.84665925,42.60330171)(591.91665771,42.65330373)
\curveto(592.02665907,42.73330158)(592.12165898,42.83830148)(592.20165771,42.96830373)
\curveto(592.27165883,43.09830122)(592.33165877,43.23830108)(592.38165771,43.38830373)
\curveto(592.39165871,43.43830088)(592.3966587,43.48830083)(592.39665771,43.53830373)
\curveto(592.3966587,43.58830073)(592.4016587,43.63830068)(592.41165771,43.68830373)
\curveto(592.43165867,43.75830056)(592.44665865,43.84330047)(592.45665771,43.94330373)
\curveto(592.45665864,44.05330026)(592.44665865,44.14330017)(592.42665771,44.21330373)
\curveto(592.40665869,44.27330004)(592.4016587,44.33329998)(592.41165771,44.39330373)
\curveto(592.41165869,44.45329986)(592.4016587,44.5132998)(592.38165771,44.57330373)
\curveto(592.36165874,44.65329966)(592.34665875,44.72829959)(592.33665771,44.79830373)
\curveto(592.32665877,44.87829944)(592.30665879,44.95329936)(592.27665771,45.02330373)
\curveto(592.15665894,45.313299)(592.01165909,45.55829876)(591.84165771,45.75830373)
\curveto(591.67165943,45.96829835)(591.44165966,46.12829819)(591.15165771,46.23830373)
}
}
{
\newrgbcolor{curcolor}{0 0 0}
\pscustom[linestyle=none,fillstyle=solid,fillcolor=curcolor]
{
\newpath
\moveto(585.64665771,49.26994436)
\lineto(585.64665771,49.71994436)
\curveto(585.63666546,49.88994311)(585.65666544,50.01494298)(585.70665771,50.09494436)
\curveto(585.75666534,50.17494282)(585.82166528,50.22994277)(585.90165771,50.25994436)
\curveto(585.98166512,50.2999427)(586.06666503,50.33994266)(586.15665771,50.37994436)
\curveto(586.28666481,50.42994257)(586.41666468,50.47494252)(586.54665771,50.51494436)
\curveto(586.67666442,50.55494244)(586.80666429,50.5999424)(586.93665771,50.64994436)
\curveto(587.05666404,50.6999423)(587.18166392,50.74494225)(587.31165771,50.78494436)
\curveto(587.43166367,50.82494217)(587.55166355,50.86994213)(587.67165771,50.91994436)
\curveto(587.78166332,50.96994203)(587.8966632,51.00994199)(588.01665771,51.03994436)
\curveto(588.13666296,51.06994193)(588.25666284,51.10994189)(588.37665771,51.15994436)
\curveto(588.66666243,51.27994172)(588.96666213,51.38994161)(589.27665771,51.48994436)
\curveto(589.58666151,51.58994141)(589.88666121,51.6999413)(590.17665771,51.81994436)
\curveto(590.21666088,51.83994116)(590.25666084,51.84994115)(590.29665771,51.84994436)
\curveto(590.32666077,51.84994115)(590.35666074,51.85994114)(590.38665771,51.87994436)
\curveto(590.52666057,51.93994106)(590.67166043,51.994941)(590.82165771,52.04494436)
\lineto(591.24165771,52.19494436)
\curveto(591.31165979,52.22494077)(591.38665971,52.25494074)(591.46665771,52.28494436)
\curveto(591.53665956,52.31494068)(591.58165952,52.36494063)(591.60165771,52.43494436)
\curveto(591.63165947,52.51494048)(591.60665949,52.57494042)(591.52665771,52.61494436)
\curveto(591.43665966,52.66494033)(591.36665973,52.6999403)(591.31665771,52.71994436)
\curveto(591.14665995,52.7999402)(590.96666013,52.86494013)(590.77665771,52.91494436)
\curveto(590.58666051,52.96494003)(590.4016607,53.02493997)(590.22165771,53.09494436)
\curveto(589.99166111,53.18493981)(589.76166134,53.26493973)(589.53165771,53.33494436)
\curveto(589.29166181,53.40493959)(589.06166204,53.48993951)(588.84165771,53.58994436)
\curveto(588.79166231,53.5999394)(588.72666237,53.61493938)(588.64665771,53.63494436)
\curveto(588.55666254,53.67493932)(588.46666263,53.70993929)(588.37665771,53.73994436)
\curveto(588.27666282,53.76993923)(588.18666291,53.7999392)(588.10665771,53.82994436)
\curveto(588.05666304,53.84993915)(588.01166309,53.86493913)(587.97165771,53.87494436)
\curveto(587.93166317,53.88493911)(587.88666321,53.8999391)(587.83665771,53.91994436)
\curveto(587.71666338,53.96993903)(587.5966635,54.00993899)(587.47665771,54.03994436)
\curveto(587.34666375,54.07993892)(587.22166388,54.12493887)(587.10165771,54.17494436)
\curveto(587.05166405,54.1949388)(587.00666409,54.20993879)(586.96665771,54.21994436)
\curveto(586.92666417,54.22993877)(586.88166422,54.24493875)(586.83165771,54.26494436)
\curveto(586.74166436,54.30493869)(586.65166445,54.33993866)(586.56165771,54.36994436)
\curveto(586.46166464,54.3999386)(586.36666473,54.42993857)(586.27665771,54.45994436)
\curveto(586.1966649,54.48993851)(586.11666498,54.51493848)(586.03665771,54.53494436)
\curveto(585.94666515,54.56493843)(585.87166523,54.60493839)(585.81165771,54.65494436)
\curveto(585.72166538,54.72493827)(585.67166543,54.81993818)(585.66165771,54.93994436)
\curveto(585.65166545,55.06993793)(585.64666545,55.20993779)(585.64665771,55.35994436)
\curveto(585.64666545,55.43993756)(585.65166545,55.51493748)(585.66165771,55.58494436)
\curveto(585.66166544,55.66493733)(585.67666542,55.72993727)(585.70665771,55.77994436)
\curveto(585.76666533,55.86993713)(585.86166524,55.8949371)(585.99165771,55.85494436)
\curveto(586.12166498,55.81493718)(586.22166488,55.77993722)(586.29165771,55.74994436)
\lineto(586.35165771,55.71994436)
\curveto(586.37166473,55.71993728)(586.39166471,55.71493728)(586.41165771,55.70494436)
\curveto(586.69166441,55.5949374)(586.97666412,55.48493751)(587.26665771,55.37494436)
\lineto(588.10665771,55.04494436)
\curveto(588.18666291,55.01493798)(588.26166284,54.98993801)(588.33165771,54.96994436)
\curveto(588.39166271,54.94993805)(588.45666264,54.92493807)(588.52665771,54.89494436)
\curveto(588.72666237,54.81493818)(588.93166217,54.73493826)(589.14165771,54.65494436)
\curveto(589.34166176,54.58493841)(589.54166156,54.50993849)(589.74165771,54.42994436)
\curveto(590.43166067,54.13993886)(591.12665997,53.86993913)(591.82665771,53.61994436)
\curveto(592.52665857,53.36993963)(593.22165788,53.0999399)(593.91165771,52.80994436)
\lineto(594.06165771,52.74994436)
\curveto(594.12165698,52.73994026)(594.18165692,52.72494027)(594.24165771,52.70494436)
\curveto(594.61165649,52.54494045)(594.97665612,52.37494062)(595.33665771,52.19494436)
\curveto(595.70665539,52.01494098)(595.99165511,51.76494123)(596.19165771,51.44494436)
\curveto(596.25165485,51.33494166)(596.2966548,51.22494177)(596.32665771,51.11494436)
\curveto(596.36665473,51.00494199)(596.4016547,50.87994212)(596.43165771,50.73994436)
\curveto(596.45165465,50.68994231)(596.45665464,50.63494236)(596.44665771,50.57494436)
\curveto(596.43665466,50.52494247)(596.43665466,50.46994253)(596.44665771,50.40994436)
\curveto(596.46665463,50.32994267)(596.46665463,50.24994275)(596.44665771,50.16994436)
\curveto(596.43665466,50.12994287)(596.43165467,50.07994292)(596.43165771,50.01994436)
\lineto(596.37165771,49.77994436)
\curveto(596.35165475,49.70994329)(596.31165479,49.65494334)(596.25165771,49.61494436)
\curveto(596.19165491,49.56494343)(596.11665498,49.53494346)(596.02665771,49.52494436)
\lineto(595.75665771,49.52494436)
\lineto(595.54665771,49.52494436)
\curveto(595.48665561,49.53494346)(595.43665566,49.55494344)(595.39665771,49.58494436)
\curveto(595.28665581,49.65494334)(595.25665584,49.77494322)(595.30665771,49.94494436)
\curveto(595.32665577,50.05494294)(595.33665576,50.17494282)(595.33665771,50.30494436)
\curveto(595.33665576,50.43494256)(595.31665578,50.54994245)(595.27665771,50.64994436)
\curveto(595.22665587,50.7999422)(595.15165595,50.91994208)(595.05165771,51.00994436)
\curveto(594.95165615,51.10994189)(594.83665626,51.1949418)(594.70665771,51.26494436)
\curveto(594.58665651,51.33494166)(594.45665664,51.3949416)(594.31665771,51.44494436)
\lineto(593.89665771,51.62494436)
\curveto(593.80665729,51.66494133)(593.6966574,51.70494129)(593.56665771,51.74494436)
\curveto(593.43665766,51.7949412)(593.3016578,51.7999412)(593.16165771,51.75994436)
\curveto(593.0016581,51.70994129)(592.85165825,51.65494134)(592.71165771,51.59494436)
\curveto(592.57165853,51.54494145)(592.43165867,51.48994151)(592.29165771,51.42994436)
\curveto(592.08165902,51.33994166)(591.87165923,51.25494174)(591.66165771,51.17494436)
\curveto(591.45165965,51.0949419)(591.24665985,51.01494198)(591.04665771,50.93494436)
\curveto(590.90666019,50.87494212)(590.77166033,50.81994218)(590.64165771,50.76994436)
\curveto(590.51166059,50.71994228)(590.37666072,50.66994233)(590.23665771,50.61994436)
\lineto(588.91665771,50.07994436)
\curveto(588.47666262,49.90994309)(588.03666306,49.73494326)(587.59665771,49.55494436)
\curveto(587.36666373,49.45494354)(587.14666395,49.36494363)(586.93665771,49.28494436)
\curveto(586.71666438,49.20494379)(586.4966646,49.11994388)(586.27665771,49.02994436)
\curveto(586.21666488,49.00994399)(586.13666496,48.97994402)(586.03665771,48.93994436)
\curveto(585.92666517,48.8999441)(585.83666526,48.90494409)(585.76665771,48.95494436)
\curveto(585.71666538,48.98494401)(585.68166542,49.04494395)(585.66165771,49.13494436)
\curveto(585.65166545,49.15494384)(585.65166545,49.17494382)(585.66165771,49.19494436)
\curveto(585.66166544,49.22494377)(585.65666544,49.24994375)(585.64665771,49.26994436)
}
}
{
\newrgbcolor{curcolor}{0 0 0}
\pscustom[linestyle=none,fillstyle=solid,fillcolor=curcolor]
{
}
}
{
\newrgbcolor{curcolor}{0 0 0}
\pscustom[linestyle=none,fillstyle=solid,fillcolor=curcolor]
{
\newpath
\moveto(582.76665771,64.24510061)
\curveto(582.75666834,64.93509597)(582.87666822,65.53509537)(583.12665771,66.04510061)
\curveto(583.37666772,66.56509434)(583.71166739,66.96009395)(584.13165771,67.23010061)
\curveto(584.21166689,67.28009363)(584.3016668,67.32509358)(584.40165771,67.36510061)
\curveto(584.49166661,67.4050935)(584.58666651,67.45009346)(584.68665771,67.50010061)
\curveto(584.78666631,67.54009337)(584.88666621,67.57009334)(584.98665771,67.59010061)
\curveto(585.08666601,67.6100933)(585.19166591,67.63009328)(585.30165771,67.65010061)
\curveto(585.35166575,67.67009324)(585.3966657,67.67509323)(585.43665771,67.66510061)
\curveto(585.47666562,67.65509325)(585.52166558,67.66009325)(585.57165771,67.68010061)
\curveto(585.62166548,67.69009322)(585.70666539,67.69509321)(585.82665771,67.69510061)
\curveto(585.93666516,67.69509321)(586.02166508,67.69009322)(586.08165771,67.68010061)
\curveto(586.14166496,67.66009325)(586.2016649,67.65009326)(586.26165771,67.65010061)
\curveto(586.32166478,67.66009325)(586.38166472,67.65509325)(586.44165771,67.63510061)
\curveto(586.58166452,67.59509331)(586.71666438,67.56009335)(586.84665771,67.53010061)
\curveto(586.97666412,67.50009341)(587.101664,67.46009345)(587.22165771,67.41010061)
\curveto(587.36166374,67.35009356)(587.48666361,67.28009363)(587.59665771,67.20010061)
\curveto(587.70666339,67.13009378)(587.81666328,67.05509385)(587.92665771,66.97510061)
\lineto(587.98665771,66.91510061)
\curveto(588.00666309,66.905094)(588.02666307,66.89009402)(588.04665771,66.87010061)
\curveto(588.20666289,66.75009416)(588.35166275,66.61509429)(588.48165771,66.46510061)
\curveto(588.61166249,66.31509459)(588.73666236,66.15509475)(588.85665771,65.98510061)
\curveto(589.07666202,65.67509523)(589.28166182,65.38009553)(589.47165771,65.10010061)
\curveto(589.61166149,64.87009604)(589.74666135,64.64009627)(589.87665771,64.41010061)
\curveto(590.00666109,64.19009672)(590.14166096,63.97009694)(590.28165771,63.75010061)
\curveto(590.45166065,63.50009741)(590.63166047,63.26009765)(590.82165771,63.03010061)
\curveto(591.01166009,62.8100981)(591.23665986,62.62009829)(591.49665771,62.46010061)
\curveto(591.55665954,62.42009849)(591.61665948,62.38509852)(591.67665771,62.35510061)
\curveto(591.72665937,62.32509858)(591.79165931,62.29509861)(591.87165771,62.26510061)
\curveto(591.94165916,62.24509866)(592.0016591,62.24009867)(592.05165771,62.25010061)
\curveto(592.12165898,62.27009864)(592.17665892,62.3050986)(592.21665771,62.35510061)
\curveto(592.24665885,62.4050985)(592.26665883,62.46509844)(592.27665771,62.53510061)
\lineto(592.27665771,62.77510061)
\lineto(592.27665771,63.52510061)
\lineto(592.27665771,66.33010061)
\lineto(592.27665771,66.99010061)
\curveto(592.27665882,67.08009383)(592.28165882,67.16509374)(592.29165771,67.24510061)
\curveto(592.29165881,67.32509358)(592.31165879,67.39009352)(592.35165771,67.44010061)
\curveto(592.39165871,67.49009342)(592.46665863,67.53009338)(592.57665771,67.56010061)
\curveto(592.67665842,67.60009331)(592.77665832,67.6100933)(592.87665771,67.59010061)
\lineto(593.01165771,67.59010061)
\curveto(593.08165802,67.57009334)(593.14165796,67.55009336)(593.19165771,67.53010061)
\curveto(593.24165786,67.5100934)(593.28165782,67.47509343)(593.31165771,67.42510061)
\curveto(593.35165775,67.37509353)(593.37165773,67.3050936)(593.37165771,67.21510061)
\lineto(593.37165771,66.94510061)
\lineto(593.37165771,66.04510061)
\lineto(593.37165771,62.53510061)
\lineto(593.37165771,61.47010061)
\curveto(593.37165773,61.39009952)(593.37665772,61.30009961)(593.38665771,61.20010061)
\curveto(593.38665771,61.10009981)(593.37665772,61.01509989)(593.35665771,60.94510061)
\curveto(593.28665781,60.73510017)(593.10665799,60.67010024)(592.81665771,60.75010061)
\curveto(592.77665832,60.76010015)(592.74165836,60.76010015)(592.71165771,60.75010061)
\curveto(592.67165843,60.75010016)(592.62665847,60.76010015)(592.57665771,60.78010061)
\curveto(592.4966586,60.80010011)(592.41165869,60.82010009)(592.32165771,60.84010061)
\curveto(592.23165887,60.86010005)(592.14665895,60.88510002)(592.06665771,60.91510061)
\curveto(591.57665952,61.07509983)(591.16165994,61.27509963)(590.82165771,61.51510061)
\curveto(590.57166053,61.69509921)(590.34666075,61.90009901)(590.14665771,62.13010061)
\curveto(589.93666116,62.36009855)(589.74166136,62.60009831)(589.56165771,62.85010061)
\curveto(589.38166172,63.1100978)(589.21166189,63.37509753)(589.05165771,63.64510061)
\curveto(588.88166222,63.92509698)(588.70666239,64.19509671)(588.52665771,64.45510061)
\curveto(588.44666265,64.56509634)(588.37166273,64.67009624)(588.30165771,64.77010061)
\curveto(588.23166287,64.88009603)(588.15666294,64.99009592)(588.07665771,65.10010061)
\curveto(588.04666305,65.14009577)(588.01666308,65.17509573)(587.98665771,65.20510061)
\curveto(587.94666315,65.24509566)(587.91666318,65.28509562)(587.89665771,65.32510061)
\curveto(587.78666331,65.46509544)(587.66166344,65.59009532)(587.52165771,65.70010061)
\curveto(587.49166361,65.72009519)(587.46666363,65.74509516)(587.44665771,65.77510061)
\curveto(587.41666368,65.8050951)(587.38666371,65.83009508)(587.35665771,65.85010061)
\curveto(587.25666384,65.93009498)(587.15666394,65.99509491)(587.05665771,66.04510061)
\curveto(586.95666414,66.1050948)(586.84666425,66.16009475)(586.72665771,66.21010061)
\curveto(586.65666444,66.24009467)(586.58166452,66.26009465)(586.50165771,66.27010061)
\lineto(586.26165771,66.33010061)
\lineto(586.17165771,66.33010061)
\curveto(586.14166496,66.34009457)(586.11166499,66.34509456)(586.08165771,66.34510061)
\curveto(586.01166509,66.36509454)(585.91666518,66.37009454)(585.79665771,66.36010061)
\curveto(585.66666543,66.36009455)(585.56666553,66.35009456)(585.49665771,66.33010061)
\curveto(585.41666568,66.3100946)(585.34166576,66.29009462)(585.27165771,66.27010061)
\curveto(585.19166591,66.26009465)(585.11166599,66.24009467)(585.03165771,66.21010061)
\curveto(584.79166631,66.10009481)(584.59166651,65.95009496)(584.43165771,65.76010061)
\curveto(584.26166684,65.58009533)(584.12166698,65.36009555)(584.01165771,65.10010061)
\curveto(583.99166711,65.03009588)(583.97666712,64.96009595)(583.96665771,64.89010061)
\curveto(583.94666715,64.82009609)(583.92666717,64.74509616)(583.90665771,64.66510061)
\curveto(583.88666721,64.58509632)(583.87666722,64.47509643)(583.87665771,64.33510061)
\curveto(583.87666722,64.2050967)(583.88666721,64.10009681)(583.90665771,64.02010061)
\curveto(583.91666718,63.96009695)(583.92166718,63.905097)(583.92165771,63.85510061)
\curveto(583.92166718,63.8050971)(583.93166717,63.75509715)(583.95165771,63.70510061)
\curveto(583.99166711,63.6050973)(584.03166707,63.5100974)(584.07165771,63.42010061)
\curveto(584.11166699,63.34009757)(584.15666694,63.26009765)(584.20665771,63.18010061)
\curveto(584.22666687,63.15009776)(584.25166685,63.12009779)(584.28165771,63.09010061)
\curveto(584.31166679,63.07009784)(584.33666676,63.04509786)(584.35665771,63.01510061)
\lineto(584.43165771,62.94010061)
\curveto(584.45166665,62.910098)(584.47166663,62.88509802)(584.49165771,62.86510061)
\lineto(584.70165771,62.71510061)
\curveto(584.76166634,62.67509823)(584.82666627,62.63009828)(584.89665771,62.58010061)
\curveto(584.98666611,62.52009839)(585.09166601,62.47009844)(585.21165771,62.43010061)
\curveto(585.32166578,62.40009851)(585.43166567,62.36509854)(585.54165771,62.32510061)
\curveto(585.65166545,62.28509862)(585.7966653,62.26009865)(585.97665771,62.25010061)
\curveto(586.14666495,62.24009867)(586.27166483,62.2100987)(586.35165771,62.16010061)
\curveto(586.43166467,62.1100988)(586.47666462,62.03509887)(586.48665771,61.93510061)
\curveto(586.4966646,61.83509907)(586.5016646,61.72509918)(586.50165771,61.60510061)
\curveto(586.5016646,61.56509934)(586.50666459,61.52509938)(586.51665771,61.48510061)
\curveto(586.51666458,61.44509946)(586.51166459,61.4100995)(586.50165771,61.38010061)
\curveto(586.48166462,61.33009958)(586.47166463,61.28009963)(586.47165771,61.23010061)
\curveto(586.47166463,61.19009972)(586.46166464,61.15009976)(586.44165771,61.11010061)
\curveto(586.38166472,61.02009989)(586.24666485,60.97509993)(586.03665771,60.97510061)
\lineto(585.91665771,60.97510061)
\curveto(585.85666524,60.98509992)(585.7966653,60.99009992)(585.73665771,60.99010061)
\curveto(585.66666543,61.00009991)(585.6016655,61.0100999)(585.54165771,61.02010061)
\curveto(585.43166567,61.04009987)(585.33166577,61.06009985)(585.24165771,61.08010061)
\curveto(585.14166596,61.10009981)(585.04666605,61.13009978)(584.95665771,61.17010061)
\curveto(584.88666621,61.19009972)(584.82666627,61.2100997)(584.77665771,61.23010061)
\lineto(584.59665771,61.29010061)
\curveto(584.33666676,61.4100995)(584.09166701,61.56509934)(583.86165771,61.75510061)
\curveto(583.63166747,61.95509895)(583.44666765,62.17009874)(583.30665771,62.40010061)
\curveto(583.22666787,62.5100984)(583.16166794,62.62509828)(583.11165771,62.74510061)
\lineto(582.96165771,63.13510061)
\curveto(582.91166819,63.24509766)(582.88166822,63.36009755)(582.87165771,63.48010061)
\curveto(582.85166825,63.60009731)(582.82666827,63.72509718)(582.79665771,63.85510061)
\curveto(582.7966683,63.92509698)(582.7966683,63.99009692)(582.79665771,64.05010061)
\curveto(582.78666831,64.1100968)(582.77666832,64.17509673)(582.76665771,64.24510061)
}
}
{
\newrgbcolor{curcolor}{0 0 0}
\pscustom[linestyle=none,fillstyle=solid,fillcolor=curcolor]
{
\newpath
\moveto(582.76665771,73.40470998)
\curveto(582.76666833,73.50470513)(582.77666832,73.59970503)(582.79665771,73.68970998)
\curveto(582.80666829,73.77970485)(582.83666826,73.84470479)(582.88665771,73.88470998)
\curveto(582.96666813,73.94470469)(583.07166803,73.97470466)(583.20165771,73.97470998)
\lineto(583.59165771,73.97470998)
\lineto(585.09165771,73.97470998)
\lineto(591.48165771,73.97470998)
\lineto(592.65165771,73.97470998)
\lineto(592.96665771,73.97470998)
\curveto(593.06665803,73.98470465)(593.14665795,73.96970466)(593.20665771,73.92970998)
\curveto(593.28665781,73.87970475)(593.33665776,73.80470483)(593.35665771,73.70470998)
\curveto(593.36665773,73.61470502)(593.37165773,73.50470513)(593.37165771,73.37470998)
\lineto(593.37165771,73.14970998)
\curveto(593.35165775,73.06970556)(593.33665776,72.99970563)(593.32665771,72.93970998)
\curveto(593.30665779,72.87970575)(593.26665783,72.8297058)(593.20665771,72.78970998)
\curveto(593.14665795,72.74970588)(593.07165803,72.7297059)(592.98165771,72.72970998)
\lineto(592.68165771,72.72970998)
\lineto(591.58665771,72.72970998)
\lineto(586.24665771,72.72970998)
\curveto(586.15666494,72.70970592)(586.08166502,72.69470594)(586.02165771,72.68470998)
\curveto(585.95166515,72.68470595)(585.89166521,72.65470598)(585.84165771,72.59470998)
\curveto(585.79166531,72.52470611)(585.76666533,72.4347062)(585.76665771,72.32470998)
\curveto(585.75666534,72.22470641)(585.75166535,72.11470652)(585.75165771,71.99470998)
\lineto(585.75165771,70.85470998)
\lineto(585.75165771,70.35970998)
\curveto(585.74166536,70.19970843)(585.68166542,70.08970854)(585.57165771,70.02970998)
\curveto(585.54166556,70.00970862)(585.51166559,69.99970863)(585.48165771,69.99970998)
\curveto(585.44166566,69.99970863)(585.3966657,69.99470864)(585.34665771,69.98470998)
\curveto(585.22666587,69.96470867)(585.11666598,69.96970866)(585.01665771,69.99970998)
\curveto(584.91666618,70.03970859)(584.84666625,70.09470854)(584.80665771,70.16470998)
\curveto(584.75666634,70.24470839)(584.73166637,70.36470827)(584.73165771,70.52470998)
\curveto(584.73166637,70.68470795)(584.71666638,70.81970781)(584.68665771,70.92970998)
\curveto(584.67666642,70.97970765)(584.67166643,71.0347076)(584.67165771,71.09470998)
\curveto(584.66166644,71.15470748)(584.64666645,71.21470742)(584.62665771,71.27470998)
\curveto(584.57666652,71.42470721)(584.52666657,71.56970706)(584.47665771,71.70970998)
\curveto(584.41666668,71.84970678)(584.34666675,71.98470665)(584.26665771,72.11470998)
\curveto(584.17666692,72.25470638)(584.07166703,72.37470626)(583.95165771,72.47470998)
\curveto(583.83166727,72.57470606)(583.7016674,72.66970596)(583.56165771,72.75970998)
\curveto(583.46166764,72.81970581)(583.35166775,72.86470577)(583.23165771,72.89470998)
\curveto(583.11166799,72.9347057)(583.00666809,72.98470565)(582.91665771,73.04470998)
\curveto(582.85666824,73.09470554)(582.81666828,73.16470547)(582.79665771,73.25470998)
\curveto(582.78666831,73.27470536)(582.78166832,73.29970533)(582.78165771,73.32970998)
\curveto(582.78166832,73.35970527)(582.77666832,73.38470525)(582.76665771,73.40470998)
}
}
{
\newrgbcolor{curcolor}{0 0 0}
\pscustom[linestyle=none,fillstyle=solid,fillcolor=curcolor]
{
\newpath
\moveto(591.73665771,78.63431936)
\lineto(591.73665771,79.26431936)
\lineto(591.73665771,79.45931936)
\curveto(591.73665936,79.52931683)(591.74665935,79.58931677)(591.76665771,79.63931936)
\curveto(591.80665929,79.70931665)(591.84665925,79.7593166)(591.88665771,79.78931936)
\curveto(591.93665916,79.82931653)(592.0016591,79.84931651)(592.08165771,79.84931936)
\curveto(592.16165894,79.8593165)(592.24665885,79.86431649)(592.33665771,79.86431936)
\lineto(593.05665771,79.86431936)
\curveto(593.53665756,79.86431649)(593.94665715,79.80431655)(594.28665771,79.68431936)
\curveto(594.62665647,79.56431679)(594.9016562,79.36931699)(595.11165771,79.09931936)
\curveto(595.16165594,79.02931733)(595.20665589,78.9593174)(595.24665771,78.88931936)
\curveto(595.2966558,78.82931753)(595.34165576,78.7543176)(595.38165771,78.66431936)
\curveto(595.39165571,78.64431771)(595.4016557,78.61431774)(595.41165771,78.57431936)
\curveto(595.43165567,78.53431782)(595.43665566,78.48931787)(595.42665771,78.43931936)
\curveto(595.3966557,78.34931801)(595.32165578,78.29431806)(595.20165771,78.27431936)
\curveto(595.09165601,78.2543181)(594.9966561,78.26931809)(594.91665771,78.31931936)
\curveto(594.84665625,78.34931801)(594.78165632,78.39431796)(594.72165771,78.45431936)
\curveto(594.67165643,78.52431783)(594.62165648,78.58931777)(594.57165771,78.64931936)
\curveto(594.52165658,78.71931764)(594.44665665,78.77931758)(594.34665771,78.82931936)
\curveto(594.25665684,78.88931747)(594.16665693,78.93931742)(594.07665771,78.97931936)
\curveto(594.04665705,78.99931736)(593.98665711,79.02431733)(593.89665771,79.05431936)
\curveto(593.81665728,79.08431727)(593.74665735,79.08931727)(593.68665771,79.06931936)
\curveto(593.54665755,79.03931732)(593.45665764,78.97931738)(593.41665771,78.88931936)
\curveto(593.38665771,78.80931755)(593.37165773,78.71931764)(593.37165771,78.61931936)
\curveto(593.37165773,78.51931784)(593.34665775,78.43431792)(593.29665771,78.36431936)
\curveto(593.22665787,78.27431808)(593.08665801,78.22931813)(592.87665771,78.22931936)
\lineto(592.32165771,78.22931936)
\lineto(592.09665771,78.22931936)
\curveto(592.01665908,78.23931812)(591.95165915,78.2593181)(591.90165771,78.28931936)
\curveto(591.82165928,78.34931801)(591.77665932,78.41931794)(591.76665771,78.49931936)
\curveto(591.75665934,78.51931784)(591.75165935,78.53931782)(591.75165771,78.55931936)
\curveto(591.75165935,78.58931777)(591.74665935,78.61431774)(591.73665771,78.63431936)
}
}
{
\newrgbcolor{curcolor}{0 0 0}
\pscustom[linestyle=none,fillstyle=solid,fillcolor=curcolor]
{
}
}
{
\newrgbcolor{curcolor}{0 0 0}
\pscustom[linestyle=none,fillstyle=solid,fillcolor=curcolor]
{
\newpath
\moveto(582.76665771,89.26463186)
\curveto(582.75666834,89.95462722)(582.87666822,90.55462662)(583.12665771,91.06463186)
\curveto(583.37666772,91.58462559)(583.71166739,91.9796252)(584.13165771,92.24963186)
\curveto(584.21166689,92.29962488)(584.3016668,92.34462483)(584.40165771,92.38463186)
\curveto(584.49166661,92.42462475)(584.58666651,92.46962471)(584.68665771,92.51963186)
\curveto(584.78666631,92.55962462)(584.88666621,92.58962459)(584.98665771,92.60963186)
\curveto(585.08666601,92.62962455)(585.19166591,92.64962453)(585.30165771,92.66963186)
\curveto(585.35166575,92.68962449)(585.3966657,92.69462448)(585.43665771,92.68463186)
\curveto(585.47666562,92.6746245)(585.52166558,92.6796245)(585.57165771,92.69963186)
\curveto(585.62166548,92.70962447)(585.70666539,92.71462446)(585.82665771,92.71463186)
\curveto(585.93666516,92.71462446)(586.02166508,92.70962447)(586.08165771,92.69963186)
\curveto(586.14166496,92.6796245)(586.2016649,92.66962451)(586.26165771,92.66963186)
\curveto(586.32166478,92.6796245)(586.38166472,92.6746245)(586.44165771,92.65463186)
\curveto(586.58166452,92.61462456)(586.71666438,92.5796246)(586.84665771,92.54963186)
\curveto(586.97666412,92.51962466)(587.101664,92.4796247)(587.22165771,92.42963186)
\curveto(587.36166374,92.36962481)(587.48666361,92.29962488)(587.59665771,92.21963186)
\curveto(587.70666339,92.14962503)(587.81666328,92.0746251)(587.92665771,91.99463186)
\lineto(587.98665771,91.93463186)
\curveto(588.00666309,91.92462525)(588.02666307,91.90962527)(588.04665771,91.88963186)
\curveto(588.20666289,91.76962541)(588.35166275,91.63462554)(588.48165771,91.48463186)
\curveto(588.61166249,91.33462584)(588.73666236,91.174626)(588.85665771,91.00463186)
\curveto(589.07666202,90.69462648)(589.28166182,90.39962678)(589.47165771,90.11963186)
\curveto(589.61166149,89.88962729)(589.74666135,89.65962752)(589.87665771,89.42963186)
\curveto(590.00666109,89.20962797)(590.14166096,88.98962819)(590.28165771,88.76963186)
\curveto(590.45166065,88.51962866)(590.63166047,88.2796289)(590.82165771,88.04963186)
\curveto(591.01166009,87.82962935)(591.23665986,87.63962954)(591.49665771,87.47963186)
\curveto(591.55665954,87.43962974)(591.61665948,87.40462977)(591.67665771,87.37463186)
\curveto(591.72665937,87.34462983)(591.79165931,87.31462986)(591.87165771,87.28463186)
\curveto(591.94165916,87.26462991)(592.0016591,87.25962992)(592.05165771,87.26963186)
\curveto(592.12165898,87.28962989)(592.17665892,87.32462985)(592.21665771,87.37463186)
\curveto(592.24665885,87.42462975)(592.26665883,87.48462969)(592.27665771,87.55463186)
\lineto(592.27665771,87.79463186)
\lineto(592.27665771,88.54463186)
\lineto(592.27665771,91.34963186)
\lineto(592.27665771,92.00963186)
\curveto(592.27665882,92.09962508)(592.28165882,92.18462499)(592.29165771,92.26463186)
\curveto(592.29165881,92.34462483)(592.31165879,92.40962477)(592.35165771,92.45963186)
\curveto(592.39165871,92.50962467)(592.46665863,92.54962463)(592.57665771,92.57963186)
\curveto(592.67665842,92.61962456)(592.77665832,92.62962455)(592.87665771,92.60963186)
\lineto(593.01165771,92.60963186)
\curveto(593.08165802,92.58962459)(593.14165796,92.56962461)(593.19165771,92.54963186)
\curveto(593.24165786,92.52962465)(593.28165782,92.49462468)(593.31165771,92.44463186)
\curveto(593.35165775,92.39462478)(593.37165773,92.32462485)(593.37165771,92.23463186)
\lineto(593.37165771,91.96463186)
\lineto(593.37165771,91.06463186)
\lineto(593.37165771,87.55463186)
\lineto(593.37165771,86.48963186)
\curveto(593.37165773,86.40963077)(593.37665772,86.31963086)(593.38665771,86.21963186)
\curveto(593.38665771,86.11963106)(593.37665772,86.03463114)(593.35665771,85.96463186)
\curveto(593.28665781,85.75463142)(593.10665799,85.68963149)(592.81665771,85.76963186)
\curveto(592.77665832,85.7796314)(592.74165836,85.7796314)(592.71165771,85.76963186)
\curveto(592.67165843,85.76963141)(592.62665847,85.7796314)(592.57665771,85.79963186)
\curveto(592.4966586,85.81963136)(592.41165869,85.83963134)(592.32165771,85.85963186)
\curveto(592.23165887,85.8796313)(592.14665895,85.90463127)(592.06665771,85.93463186)
\curveto(591.57665952,86.09463108)(591.16165994,86.29463088)(590.82165771,86.53463186)
\curveto(590.57166053,86.71463046)(590.34666075,86.91963026)(590.14665771,87.14963186)
\curveto(589.93666116,87.3796298)(589.74166136,87.61962956)(589.56165771,87.86963186)
\curveto(589.38166172,88.12962905)(589.21166189,88.39462878)(589.05165771,88.66463186)
\curveto(588.88166222,88.94462823)(588.70666239,89.21462796)(588.52665771,89.47463186)
\curveto(588.44666265,89.58462759)(588.37166273,89.68962749)(588.30165771,89.78963186)
\curveto(588.23166287,89.89962728)(588.15666294,90.00962717)(588.07665771,90.11963186)
\curveto(588.04666305,90.15962702)(588.01666308,90.19462698)(587.98665771,90.22463186)
\curveto(587.94666315,90.26462691)(587.91666318,90.30462687)(587.89665771,90.34463186)
\curveto(587.78666331,90.48462669)(587.66166344,90.60962657)(587.52165771,90.71963186)
\curveto(587.49166361,90.73962644)(587.46666363,90.76462641)(587.44665771,90.79463186)
\curveto(587.41666368,90.82462635)(587.38666371,90.84962633)(587.35665771,90.86963186)
\curveto(587.25666384,90.94962623)(587.15666394,91.01462616)(587.05665771,91.06463186)
\curveto(586.95666414,91.12462605)(586.84666425,91.179626)(586.72665771,91.22963186)
\curveto(586.65666444,91.25962592)(586.58166452,91.2796259)(586.50165771,91.28963186)
\lineto(586.26165771,91.34963186)
\lineto(586.17165771,91.34963186)
\curveto(586.14166496,91.35962582)(586.11166499,91.36462581)(586.08165771,91.36463186)
\curveto(586.01166509,91.38462579)(585.91666518,91.38962579)(585.79665771,91.37963186)
\curveto(585.66666543,91.3796258)(585.56666553,91.36962581)(585.49665771,91.34963186)
\curveto(585.41666568,91.32962585)(585.34166576,91.30962587)(585.27165771,91.28963186)
\curveto(585.19166591,91.2796259)(585.11166599,91.25962592)(585.03165771,91.22963186)
\curveto(584.79166631,91.11962606)(584.59166651,90.96962621)(584.43165771,90.77963186)
\curveto(584.26166684,90.59962658)(584.12166698,90.3796268)(584.01165771,90.11963186)
\curveto(583.99166711,90.04962713)(583.97666712,89.9796272)(583.96665771,89.90963186)
\curveto(583.94666715,89.83962734)(583.92666717,89.76462741)(583.90665771,89.68463186)
\curveto(583.88666721,89.60462757)(583.87666722,89.49462768)(583.87665771,89.35463186)
\curveto(583.87666722,89.22462795)(583.88666721,89.11962806)(583.90665771,89.03963186)
\curveto(583.91666718,88.9796282)(583.92166718,88.92462825)(583.92165771,88.87463186)
\curveto(583.92166718,88.82462835)(583.93166717,88.7746284)(583.95165771,88.72463186)
\curveto(583.99166711,88.62462855)(584.03166707,88.52962865)(584.07165771,88.43963186)
\curveto(584.11166699,88.35962882)(584.15666694,88.2796289)(584.20665771,88.19963186)
\curveto(584.22666687,88.16962901)(584.25166685,88.13962904)(584.28165771,88.10963186)
\curveto(584.31166679,88.08962909)(584.33666676,88.06462911)(584.35665771,88.03463186)
\lineto(584.43165771,87.95963186)
\curveto(584.45166665,87.92962925)(584.47166663,87.90462927)(584.49165771,87.88463186)
\lineto(584.70165771,87.73463186)
\curveto(584.76166634,87.69462948)(584.82666627,87.64962953)(584.89665771,87.59963186)
\curveto(584.98666611,87.53962964)(585.09166601,87.48962969)(585.21165771,87.44963186)
\curveto(585.32166578,87.41962976)(585.43166567,87.38462979)(585.54165771,87.34463186)
\curveto(585.65166545,87.30462987)(585.7966653,87.2796299)(585.97665771,87.26963186)
\curveto(586.14666495,87.25962992)(586.27166483,87.22962995)(586.35165771,87.17963186)
\curveto(586.43166467,87.12963005)(586.47666462,87.05463012)(586.48665771,86.95463186)
\curveto(586.4966646,86.85463032)(586.5016646,86.74463043)(586.50165771,86.62463186)
\curveto(586.5016646,86.58463059)(586.50666459,86.54463063)(586.51665771,86.50463186)
\curveto(586.51666458,86.46463071)(586.51166459,86.42963075)(586.50165771,86.39963186)
\curveto(586.48166462,86.34963083)(586.47166463,86.29963088)(586.47165771,86.24963186)
\curveto(586.47166463,86.20963097)(586.46166464,86.16963101)(586.44165771,86.12963186)
\curveto(586.38166472,86.03963114)(586.24666485,85.99463118)(586.03665771,85.99463186)
\lineto(585.91665771,85.99463186)
\curveto(585.85666524,86.00463117)(585.7966653,86.00963117)(585.73665771,86.00963186)
\curveto(585.66666543,86.01963116)(585.6016655,86.02963115)(585.54165771,86.03963186)
\curveto(585.43166567,86.05963112)(585.33166577,86.0796311)(585.24165771,86.09963186)
\curveto(585.14166596,86.11963106)(585.04666605,86.14963103)(584.95665771,86.18963186)
\curveto(584.88666621,86.20963097)(584.82666627,86.22963095)(584.77665771,86.24963186)
\lineto(584.59665771,86.30963186)
\curveto(584.33666676,86.42963075)(584.09166701,86.58463059)(583.86165771,86.77463186)
\curveto(583.63166747,86.9746302)(583.44666765,87.18962999)(583.30665771,87.41963186)
\curveto(583.22666787,87.52962965)(583.16166794,87.64462953)(583.11165771,87.76463186)
\lineto(582.96165771,88.15463186)
\curveto(582.91166819,88.26462891)(582.88166822,88.3796288)(582.87165771,88.49963186)
\curveto(582.85166825,88.61962856)(582.82666827,88.74462843)(582.79665771,88.87463186)
\curveto(582.7966683,88.94462823)(582.7966683,89.00962817)(582.79665771,89.06963186)
\curveto(582.78666831,89.12962805)(582.77666832,89.19462798)(582.76665771,89.26463186)
}
}
{
\newrgbcolor{curcolor}{0 0 0}
\pscustom[linestyle=none,fillstyle=solid,fillcolor=curcolor]
{
\newpath
\moveto(588.28665771,101.36424123)
\lineto(588.54165771,101.36424123)
\curveto(588.62166248,101.37423353)(588.6966624,101.36923353)(588.76665771,101.34924123)
\lineto(589.00665771,101.34924123)
\lineto(589.17165771,101.34924123)
\curveto(589.27166183,101.32923357)(589.37666172,101.31923358)(589.48665771,101.31924123)
\curveto(589.58666151,101.31923358)(589.68666141,101.30923359)(589.78665771,101.28924123)
\lineto(589.93665771,101.28924123)
\curveto(590.07666102,101.25923364)(590.21666088,101.23923366)(590.35665771,101.22924123)
\curveto(590.48666061,101.21923368)(590.61666048,101.19423371)(590.74665771,101.15424123)
\curveto(590.82666027,101.13423377)(590.91166019,101.11423379)(591.00165771,101.09424123)
\lineto(591.24165771,101.03424123)
\lineto(591.54165771,100.91424123)
\curveto(591.63165947,100.88423402)(591.72165938,100.84923405)(591.81165771,100.80924123)
\curveto(592.03165907,100.70923419)(592.24665885,100.57423433)(592.45665771,100.40424123)
\curveto(592.66665843,100.24423466)(592.83665826,100.06923483)(592.96665771,99.87924123)
\curveto(593.00665809,99.82923507)(593.04665805,99.76923513)(593.08665771,99.69924123)
\curveto(593.11665798,99.63923526)(593.15165795,99.57923532)(593.19165771,99.51924123)
\curveto(593.24165786,99.43923546)(593.28165782,99.34423556)(593.31165771,99.23424123)
\curveto(593.34165776,99.12423578)(593.37165773,99.01923588)(593.40165771,98.91924123)
\curveto(593.44165766,98.80923609)(593.46665763,98.6992362)(593.47665771,98.58924123)
\curveto(593.48665761,98.47923642)(593.5016576,98.36423654)(593.52165771,98.24424123)
\curveto(593.53165757,98.2042367)(593.53165757,98.15923674)(593.52165771,98.10924123)
\curveto(593.52165758,98.06923683)(593.52665757,98.02923687)(593.53665771,97.98924123)
\curveto(593.54665755,97.94923695)(593.55165755,97.89423701)(593.55165771,97.82424123)
\curveto(593.55165755,97.75423715)(593.54665755,97.7042372)(593.53665771,97.67424123)
\curveto(593.51665758,97.62423728)(593.51165759,97.57923732)(593.52165771,97.53924123)
\curveto(593.53165757,97.4992374)(593.53165757,97.46423744)(593.52165771,97.43424123)
\lineto(593.52165771,97.34424123)
\curveto(593.5016576,97.28423762)(593.48665761,97.21923768)(593.47665771,97.14924123)
\curveto(593.47665762,97.08923781)(593.47165763,97.02423788)(593.46165771,96.95424123)
\curveto(593.41165769,96.78423812)(593.36165774,96.62423828)(593.31165771,96.47424123)
\curveto(593.26165784,96.32423858)(593.1966579,96.17923872)(593.11665771,96.03924123)
\curveto(593.07665802,95.98923891)(593.04665805,95.93423897)(593.02665771,95.87424123)
\curveto(592.9966581,95.82423908)(592.96165814,95.77423913)(592.92165771,95.72424123)
\curveto(592.74165836,95.48423942)(592.52165858,95.28423962)(592.26165771,95.12424123)
\curveto(592.0016591,94.96423994)(591.71665938,94.82424008)(591.40665771,94.70424123)
\curveto(591.26665983,94.64424026)(591.12665997,94.5992403)(590.98665771,94.56924123)
\curveto(590.83666026,94.53924036)(590.68166042,94.5042404)(590.52165771,94.46424123)
\curveto(590.41166069,94.44424046)(590.3016608,94.42924047)(590.19165771,94.41924123)
\curveto(590.08166102,94.40924049)(589.97166113,94.39424051)(589.86165771,94.37424123)
\curveto(589.82166128,94.36424054)(589.78166132,94.35924054)(589.74165771,94.35924123)
\curveto(589.7016614,94.36924053)(589.66166144,94.36924053)(589.62165771,94.35924123)
\curveto(589.57166153,94.34924055)(589.52166158,94.34424056)(589.47165771,94.34424123)
\lineto(589.30665771,94.34424123)
\curveto(589.25666184,94.32424058)(589.20666189,94.31924058)(589.15665771,94.32924123)
\curveto(589.096662,94.33924056)(589.04166206,94.33924056)(588.99165771,94.32924123)
\curveto(588.95166215,94.31924058)(588.90666219,94.31924058)(588.85665771,94.32924123)
\curveto(588.80666229,94.33924056)(588.75666234,94.33424057)(588.70665771,94.31424123)
\curveto(588.63666246,94.29424061)(588.56166254,94.28924061)(588.48165771,94.29924123)
\curveto(588.39166271,94.30924059)(588.30666279,94.31424059)(588.22665771,94.31424123)
\curveto(588.13666296,94.31424059)(588.03666306,94.30924059)(587.92665771,94.29924123)
\curveto(587.80666329,94.28924061)(587.70666339,94.29424061)(587.62665771,94.31424123)
\lineto(587.34165771,94.31424123)
\lineto(586.71165771,94.35924123)
\curveto(586.61166449,94.36924053)(586.51666458,94.37924052)(586.42665771,94.38924123)
\lineto(586.12665771,94.41924123)
\curveto(586.07666502,94.43924046)(586.02666507,94.44424046)(585.97665771,94.43424123)
\curveto(585.91666518,94.43424047)(585.86166524,94.44424046)(585.81165771,94.46424123)
\curveto(585.64166546,94.51424039)(585.47666562,94.55424035)(585.31665771,94.58424123)
\curveto(585.14666595,94.61424029)(584.98666611,94.66424024)(584.83665771,94.73424123)
\curveto(584.37666672,94.92423998)(584.0016671,95.14423976)(583.71165771,95.39424123)
\curveto(583.42166768,95.65423925)(583.17666792,96.01423889)(582.97665771,96.47424123)
\curveto(582.92666817,96.6042383)(582.89166821,96.73423817)(582.87165771,96.86424123)
\curveto(582.85166825,97.0042379)(582.82666827,97.14423776)(582.79665771,97.28424123)
\curveto(582.78666831,97.35423755)(582.78166832,97.41923748)(582.78165771,97.47924123)
\curveto(582.78166832,97.53923736)(582.77666832,97.6042373)(582.76665771,97.67424123)
\curveto(582.74666835,98.5042364)(582.8966682,99.17423573)(583.21665771,99.68424123)
\curveto(583.52666757,100.19423471)(583.96666713,100.57423433)(584.53665771,100.82424123)
\curveto(584.65666644,100.87423403)(584.78166632,100.91923398)(584.91165771,100.95924123)
\curveto(585.04166606,100.9992339)(585.17666592,101.04423386)(585.31665771,101.09424123)
\curveto(585.3966657,101.11423379)(585.48166562,101.12923377)(585.57165771,101.13924123)
\lineto(585.81165771,101.19924123)
\curveto(585.92166518,101.22923367)(586.03166507,101.24423366)(586.14165771,101.24424123)
\curveto(586.25166485,101.25423365)(586.36166474,101.26923363)(586.47165771,101.28924123)
\curveto(586.52166458,101.30923359)(586.56666453,101.31423359)(586.60665771,101.30424123)
\curveto(586.64666445,101.3042336)(586.68666441,101.30923359)(586.72665771,101.31924123)
\curveto(586.77666432,101.32923357)(586.83166427,101.32923357)(586.89165771,101.31924123)
\curveto(586.94166416,101.31923358)(586.99166411,101.32423358)(587.04165771,101.33424123)
\lineto(587.17665771,101.33424123)
\curveto(587.23666386,101.35423355)(587.30666379,101.35423355)(587.38665771,101.33424123)
\curveto(587.45666364,101.32423358)(587.52166358,101.32923357)(587.58165771,101.34924123)
\curveto(587.61166349,101.35923354)(587.65166345,101.36423354)(587.70165771,101.36424123)
\lineto(587.82165771,101.36424123)
\lineto(588.28665771,101.36424123)
\moveto(590.61165771,99.81924123)
\curveto(590.29166081,99.91923498)(589.92666117,99.97923492)(589.51665771,99.99924123)
\curveto(589.10666199,100.01923488)(588.6966624,100.02923487)(588.28665771,100.02924123)
\curveto(587.85666324,100.02923487)(587.43666366,100.01923488)(587.02665771,99.99924123)
\curveto(586.61666448,99.97923492)(586.23166487,99.93423497)(585.87165771,99.86424123)
\curveto(585.51166559,99.79423511)(585.19166591,99.68423522)(584.91165771,99.53424123)
\curveto(584.62166648,99.39423551)(584.38666671,99.1992357)(584.20665771,98.94924123)
\curveto(584.096667,98.78923611)(584.01666708,98.60923629)(583.96665771,98.40924123)
\curveto(583.90666719,98.20923669)(583.87666722,97.96423694)(583.87665771,97.67424123)
\curveto(583.8966672,97.65423725)(583.90666719,97.61923728)(583.90665771,97.56924123)
\curveto(583.8966672,97.51923738)(583.8966672,97.47923742)(583.90665771,97.44924123)
\curveto(583.92666717,97.36923753)(583.94666715,97.29423761)(583.96665771,97.22424123)
\curveto(583.97666712,97.16423774)(583.9966671,97.0992378)(584.02665771,97.02924123)
\curveto(584.14666695,96.75923814)(584.31666678,96.53923836)(584.53665771,96.36924123)
\curveto(584.74666635,96.20923869)(584.99166611,96.07423883)(585.27165771,95.96424123)
\curveto(585.38166572,95.91423899)(585.5016656,95.87423903)(585.63165771,95.84424123)
\curveto(585.75166535,95.82423908)(585.87666522,95.7992391)(586.00665771,95.76924123)
\curveto(586.05666504,95.74923915)(586.11166499,95.73923916)(586.17165771,95.73924123)
\curveto(586.22166488,95.73923916)(586.27166483,95.73423917)(586.32165771,95.72424123)
\curveto(586.41166469,95.71423919)(586.50666459,95.7042392)(586.60665771,95.69424123)
\curveto(586.6966644,95.68423922)(586.79166431,95.67423923)(586.89165771,95.66424123)
\curveto(586.97166413,95.66423924)(587.05666404,95.65923924)(587.14665771,95.64924123)
\lineto(587.38665771,95.64924123)
\lineto(587.56665771,95.64924123)
\curveto(587.5966635,95.63923926)(587.63166347,95.63423927)(587.67165771,95.63424123)
\lineto(587.80665771,95.63424123)
\lineto(588.25665771,95.63424123)
\curveto(588.33666276,95.63423927)(588.42166268,95.62923927)(588.51165771,95.61924123)
\curveto(588.59166251,95.61923928)(588.66666243,95.62923927)(588.73665771,95.64924123)
\lineto(589.00665771,95.64924123)
\curveto(589.02666207,95.64923925)(589.05666204,95.64423926)(589.09665771,95.63424123)
\curveto(589.12666197,95.63423927)(589.15166195,95.63923926)(589.17165771,95.64924123)
\curveto(589.27166183,95.65923924)(589.37166173,95.66423924)(589.47165771,95.66424123)
\curveto(589.56166154,95.67423923)(589.66166144,95.68423922)(589.77165771,95.69424123)
\curveto(589.89166121,95.72423918)(590.01666108,95.73923916)(590.14665771,95.73924123)
\curveto(590.26666083,95.74923915)(590.38166072,95.77423913)(590.49165771,95.81424123)
\curveto(590.79166031,95.89423901)(591.05666004,95.97923892)(591.28665771,96.06924123)
\curveto(591.51665958,96.16923873)(591.73165937,96.31423859)(591.93165771,96.50424123)
\curveto(592.13165897,96.71423819)(592.28165882,96.97923792)(592.38165771,97.29924123)
\curveto(592.4016587,97.33923756)(592.41165869,97.37423753)(592.41165771,97.40424123)
\curveto(592.4016587,97.44423746)(592.40665869,97.48923741)(592.42665771,97.53924123)
\curveto(592.43665866,97.57923732)(592.44665865,97.64923725)(592.45665771,97.74924123)
\curveto(592.46665863,97.85923704)(592.46165864,97.94423696)(592.44165771,98.00424123)
\curveto(592.42165868,98.07423683)(592.41165869,98.14423676)(592.41165771,98.21424123)
\curveto(592.4016587,98.28423662)(592.38665871,98.34923655)(592.36665771,98.40924123)
\curveto(592.30665879,98.60923629)(592.22165888,98.78923611)(592.11165771,98.94924123)
\curveto(592.09165901,98.97923592)(592.07165903,99.0042359)(592.05165771,99.02424123)
\lineto(591.99165771,99.08424123)
\curveto(591.97165913,99.12423578)(591.93165917,99.17423573)(591.87165771,99.23424123)
\curveto(591.73165937,99.33423557)(591.6016595,99.41923548)(591.48165771,99.48924123)
\curveto(591.36165974,99.55923534)(591.21665988,99.62923527)(591.04665771,99.69924123)
\curveto(590.97666012,99.72923517)(590.90666019,99.74923515)(590.83665771,99.75924123)
\curveto(590.76666033,99.77923512)(590.69166041,99.7992351)(590.61165771,99.81924123)
}
}
{
\newrgbcolor{curcolor}{0 0 0}
\pscustom[linestyle=none,fillstyle=solid,fillcolor=curcolor]
{
\newpath
\moveto(582.76665771,106.77385061)
\curveto(582.76666833,106.87384575)(582.77666832,106.96884566)(582.79665771,107.05885061)
\curveto(582.80666829,107.14884548)(582.83666826,107.21384541)(582.88665771,107.25385061)
\curveto(582.96666813,107.31384531)(583.07166803,107.34384528)(583.20165771,107.34385061)
\lineto(583.59165771,107.34385061)
\lineto(585.09165771,107.34385061)
\lineto(591.48165771,107.34385061)
\lineto(592.65165771,107.34385061)
\lineto(592.96665771,107.34385061)
\curveto(593.06665803,107.35384527)(593.14665795,107.33884529)(593.20665771,107.29885061)
\curveto(593.28665781,107.24884538)(593.33665776,107.17384545)(593.35665771,107.07385061)
\curveto(593.36665773,106.98384564)(593.37165773,106.87384575)(593.37165771,106.74385061)
\lineto(593.37165771,106.51885061)
\curveto(593.35165775,106.43884619)(593.33665776,106.36884626)(593.32665771,106.30885061)
\curveto(593.30665779,106.24884638)(593.26665783,106.19884643)(593.20665771,106.15885061)
\curveto(593.14665795,106.11884651)(593.07165803,106.09884653)(592.98165771,106.09885061)
\lineto(592.68165771,106.09885061)
\lineto(591.58665771,106.09885061)
\lineto(586.24665771,106.09885061)
\curveto(586.15666494,106.07884655)(586.08166502,106.06384656)(586.02165771,106.05385061)
\curveto(585.95166515,106.05384657)(585.89166521,106.0238466)(585.84165771,105.96385061)
\curveto(585.79166531,105.89384673)(585.76666533,105.80384682)(585.76665771,105.69385061)
\curveto(585.75666534,105.59384703)(585.75166535,105.48384714)(585.75165771,105.36385061)
\lineto(585.75165771,104.22385061)
\lineto(585.75165771,103.72885061)
\curveto(585.74166536,103.56884906)(585.68166542,103.45884917)(585.57165771,103.39885061)
\curveto(585.54166556,103.37884925)(585.51166559,103.36884926)(585.48165771,103.36885061)
\curveto(585.44166566,103.36884926)(585.3966657,103.36384926)(585.34665771,103.35385061)
\curveto(585.22666587,103.33384929)(585.11666598,103.33884929)(585.01665771,103.36885061)
\curveto(584.91666618,103.40884922)(584.84666625,103.46384916)(584.80665771,103.53385061)
\curveto(584.75666634,103.61384901)(584.73166637,103.73384889)(584.73165771,103.89385061)
\curveto(584.73166637,104.05384857)(584.71666638,104.18884844)(584.68665771,104.29885061)
\curveto(584.67666642,104.34884828)(584.67166643,104.40384822)(584.67165771,104.46385061)
\curveto(584.66166644,104.5238481)(584.64666645,104.58384804)(584.62665771,104.64385061)
\curveto(584.57666652,104.79384783)(584.52666657,104.93884769)(584.47665771,105.07885061)
\curveto(584.41666668,105.21884741)(584.34666675,105.35384727)(584.26665771,105.48385061)
\curveto(584.17666692,105.623847)(584.07166703,105.74384688)(583.95165771,105.84385061)
\curveto(583.83166727,105.94384668)(583.7016674,106.03884659)(583.56165771,106.12885061)
\curveto(583.46166764,106.18884644)(583.35166775,106.23384639)(583.23165771,106.26385061)
\curveto(583.11166799,106.30384632)(583.00666809,106.35384627)(582.91665771,106.41385061)
\curveto(582.85666824,106.46384616)(582.81666828,106.53384609)(582.79665771,106.62385061)
\curveto(582.78666831,106.64384598)(582.78166832,106.66884596)(582.78165771,106.69885061)
\curveto(582.78166832,106.7288459)(582.77666832,106.75384587)(582.76665771,106.77385061)
}
}
{
\newrgbcolor{curcolor}{0 0 0}
\pscustom[linestyle=none,fillstyle=solid,fillcolor=curcolor]
{
\newpath
\moveto(582.76665771,115.12345998)
\curveto(582.76666833,115.22345513)(582.77666832,115.31845503)(582.79665771,115.40845998)
\curveto(582.80666829,115.49845485)(582.83666826,115.56345479)(582.88665771,115.60345998)
\curveto(582.96666813,115.66345469)(583.07166803,115.69345466)(583.20165771,115.69345998)
\lineto(583.59165771,115.69345998)
\lineto(585.09165771,115.69345998)
\lineto(591.48165771,115.69345998)
\lineto(592.65165771,115.69345998)
\lineto(592.96665771,115.69345998)
\curveto(593.06665803,115.70345465)(593.14665795,115.68845466)(593.20665771,115.64845998)
\curveto(593.28665781,115.59845475)(593.33665776,115.52345483)(593.35665771,115.42345998)
\curveto(593.36665773,115.33345502)(593.37165773,115.22345513)(593.37165771,115.09345998)
\lineto(593.37165771,114.86845998)
\curveto(593.35165775,114.78845556)(593.33665776,114.71845563)(593.32665771,114.65845998)
\curveto(593.30665779,114.59845575)(593.26665783,114.5484558)(593.20665771,114.50845998)
\curveto(593.14665795,114.46845588)(593.07165803,114.4484559)(592.98165771,114.44845998)
\lineto(592.68165771,114.44845998)
\lineto(591.58665771,114.44845998)
\lineto(586.24665771,114.44845998)
\curveto(586.15666494,114.42845592)(586.08166502,114.41345594)(586.02165771,114.40345998)
\curveto(585.95166515,114.40345595)(585.89166521,114.37345598)(585.84165771,114.31345998)
\curveto(585.79166531,114.24345611)(585.76666533,114.1534562)(585.76665771,114.04345998)
\curveto(585.75666534,113.94345641)(585.75166535,113.83345652)(585.75165771,113.71345998)
\lineto(585.75165771,112.57345998)
\lineto(585.75165771,112.07845998)
\curveto(585.74166536,111.91845843)(585.68166542,111.80845854)(585.57165771,111.74845998)
\curveto(585.54166556,111.72845862)(585.51166559,111.71845863)(585.48165771,111.71845998)
\curveto(585.44166566,111.71845863)(585.3966657,111.71345864)(585.34665771,111.70345998)
\curveto(585.22666587,111.68345867)(585.11666598,111.68845866)(585.01665771,111.71845998)
\curveto(584.91666618,111.75845859)(584.84666625,111.81345854)(584.80665771,111.88345998)
\curveto(584.75666634,111.96345839)(584.73166637,112.08345827)(584.73165771,112.24345998)
\curveto(584.73166637,112.40345795)(584.71666638,112.53845781)(584.68665771,112.64845998)
\curveto(584.67666642,112.69845765)(584.67166643,112.7534576)(584.67165771,112.81345998)
\curveto(584.66166644,112.87345748)(584.64666645,112.93345742)(584.62665771,112.99345998)
\curveto(584.57666652,113.14345721)(584.52666657,113.28845706)(584.47665771,113.42845998)
\curveto(584.41666668,113.56845678)(584.34666675,113.70345665)(584.26665771,113.83345998)
\curveto(584.17666692,113.97345638)(584.07166703,114.09345626)(583.95165771,114.19345998)
\curveto(583.83166727,114.29345606)(583.7016674,114.38845596)(583.56165771,114.47845998)
\curveto(583.46166764,114.53845581)(583.35166775,114.58345577)(583.23165771,114.61345998)
\curveto(583.11166799,114.6534557)(583.00666809,114.70345565)(582.91665771,114.76345998)
\curveto(582.85666824,114.81345554)(582.81666828,114.88345547)(582.79665771,114.97345998)
\curveto(582.78666831,114.99345536)(582.78166832,115.01845533)(582.78165771,115.04845998)
\curveto(582.78166832,115.07845527)(582.77666832,115.10345525)(582.76665771,115.12345998)
}
}
{
\newrgbcolor{curcolor}{0 0 0}
\pscustom[linestyle=none,fillstyle=solid,fillcolor=curcolor]
{
\newpath
\moveto(604.63794678,29.18119436)
\lineto(604.63794678,30.09619436)
\curveto(604.63795747,30.19619171)(604.63795747,30.29119161)(604.63794678,30.38119436)
\curveto(604.63795747,30.47119143)(604.65795745,30.54619136)(604.69794678,30.60619436)
\curveto(604.75795735,30.69619121)(604.83795727,30.75619115)(604.93794678,30.78619436)
\curveto(605.03795707,30.82619108)(605.14295697,30.87119103)(605.25294678,30.92119436)
\curveto(605.44295667,31.0011909)(605.63295648,31.07119083)(605.82294678,31.13119436)
\curveto(606.0129561,31.2011907)(606.20295591,31.27619063)(606.39294678,31.35619436)
\curveto(606.57295554,31.42619048)(606.75795535,31.49119041)(606.94794678,31.55119436)
\curveto(607.12795498,31.61119029)(607.3079548,31.68119022)(607.48794678,31.76119436)
\curveto(607.62795448,31.82119008)(607.77295434,31.87619003)(607.92294678,31.92619436)
\curveto(608.07295404,31.97618993)(608.21795389,32.03118987)(608.35794678,32.09119436)
\curveto(608.8079533,32.27118963)(609.26295285,32.44118946)(609.72294678,32.60119436)
\curveto(610.17295194,32.76118914)(610.62295149,32.93118897)(611.07294678,33.11119436)
\curveto(611.12295099,33.13118877)(611.17295094,33.14618876)(611.22294678,33.15619436)
\lineto(611.37294678,33.21619436)
\curveto(611.59295052,33.3061886)(611.81795029,33.39118851)(612.04794678,33.47119436)
\curveto(612.26794984,33.55118835)(612.48794962,33.63618827)(612.70794678,33.72619436)
\curveto(612.79794931,33.76618814)(612.9079492,33.8061881)(613.03794678,33.84619436)
\curveto(613.15794895,33.88618802)(613.22794888,33.95118795)(613.24794678,34.04119436)
\curveto(613.25794885,34.08118782)(613.25794885,34.11118779)(613.24794678,34.13119436)
\lineto(613.18794678,34.19119436)
\curveto(613.13794897,34.24118766)(613.08294903,34.27618763)(613.02294678,34.29619436)
\curveto(612.96294915,34.32618758)(612.89794921,34.35618755)(612.82794678,34.38619436)
\lineto(612.19794678,34.62619436)
\curveto(611.97795013,34.7061872)(611.76295035,34.78618712)(611.55294678,34.86619436)
\lineto(611.40294678,34.92619436)
\lineto(611.22294678,34.98619436)
\curveto(611.03295108,35.06618684)(610.84295127,35.13618677)(610.65294678,35.19619436)
\curveto(610.45295166,35.26618664)(610.25295186,35.34118656)(610.05294678,35.42119436)
\curveto(609.47295264,35.66118624)(608.88795322,35.88118602)(608.29794678,36.08119436)
\curveto(607.7079544,36.29118561)(607.12295499,36.51618539)(606.54294678,36.75619436)
\curveto(606.34295577,36.83618507)(606.13795597,36.91118499)(605.92794678,36.98119436)
\curveto(605.71795639,37.06118484)(605.5129566,37.14118476)(605.31294678,37.22119436)
\curveto(605.23295688,37.26118464)(605.13295698,37.29618461)(605.01294678,37.32619436)
\curveto(604.89295722,37.36618454)(604.8079573,37.42118448)(604.75794678,37.49119436)
\curveto(604.71795739,37.55118435)(604.68795742,37.62618428)(604.66794678,37.71619436)
\curveto(604.64795746,37.81618409)(604.63795747,37.92618398)(604.63794678,38.04619436)
\curveto(604.62795748,38.16618374)(604.62795748,38.28618362)(604.63794678,38.40619436)
\curveto(604.63795747,38.52618338)(604.63795747,38.63618327)(604.63794678,38.73619436)
\curveto(604.63795747,38.82618308)(604.63795747,38.91618299)(604.63794678,39.00619436)
\curveto(604.63795747,39.1061828)(604.65795745,39.18118272)(604.69794678,39.23119436)
\curveto(604.74795736,39.32118258)(604.83795727,39.37118253)(604.96794678,39.38119436)
\curveto(605.09795701,39.39118251)(605.23795687,39.39618251)(605.38794678,39.39619436)
\lineto(607.03794678,39.39619436)
\lineto(613.30794678,39.39619436)
\lineto(614.56794678,39.39619436)
\curveto(614.67794743,39.39618251)(614.78794732,39.39618251)(614.89794678,39.39619436)
\curveto(615.0079471,39.4061825)(615.09294702,39.38618252)(615.15294678,39.33619436)
\curveto(615.2129469,39.3061826)(615.25294686,39.26118264)(615.27294678,39.20119436)
\curveto(615.28294683,39.14118276)(615.29794681,39.07118283)(615.31794678,38.99119436)
\lineto(615.31794678,38.75119436)
\lineto(615.31794678,38.39119436)
\curveto(615.3079468,38.28118362)(615.26294685,38.2011837)(615.18294678,38.15119436)
\curveto(615.15294696,38.13118377)(615.12294699,38.11618379)(615.09294678,38.10619436)
\curveto(615.05294706,38.1061838)(615.0079471,38.09618381)(614.95794678,38.07619436)
\lineto(614.79294678,38.07619436)
\curveto(614.73294738,38.06618384)(614.66294745,38.06118384)(614.58294678,38.06119436)
\curveto(614.50294761,38.07118383)(614.42794768,38.07618383)(614.35794678,38.07619436)
\lineto(613.51794678,38.07619436)
\lineto(609.09294678,38.07619436)
\curveto(608.84295327,38.07618383)(608.59295352,38.07618383)(608.34294678,38.07619436)
\curveto(608.08295403,38.07618383)(607.83295428,38.07118383)(607.59294678,38.06119436)
\curveto(607.49295462,38.06118384)(607.38295473,38.05618385)(607.26294678,38.04619436)
\curveto(607.14295497,38.03618387)(607.08295503,37.98118392)(607.08294678,37.88119436)
\lineto(607.09794678,37.88119436)
\curveto(607.11795499,37.81118409)(607.18295493,37.75118415)(607.29294678,37.70119436)
\curveto(607.40295471,37.66118424)(607.49795461,37.62618428)(607.57794678,37.59619436)
\curveto(607.74795436,37.52618438)(607.92295419,37.46118444)(608.10294678,37.40119436)
\curveto(608.27295384,37.34118456)(608.44295367,37.27118463)(608.61294678,37.19119436)
\curveto(608.66295345,37.17118473)(608.7079534,37.15618475)(608.74794678,37.14619436)
\curveto(608.78795332,37.13618477)(608.83295328,37.12118478)(608.88294678,37.10119436)
\curveto(609.06295305,37.02118488)(609.24795286,36.95118495)(609.43794678,36.89119436)
\curveto(609.61795249,36.84118506)(609.79795231,36.77618513)(609.97794678,36.69619436)
\curveto(610.12795198,36.62618528)(610.28295183,36.56618534)(610.44294678,36.51619436)
\curveto(610.59295152,36.46618544)(610.74295137,36.41118549)(610.89294678,36.35119436)
\curveto(611.36295075,36.15118575)(611.83795027,35.97118593)(612.31794678,35.81119436)
\curveto(612.78794932,35.65118625)(613.25294886,35.47618643)(613.71294678,35.28619436)
\curveto(613.89294822,35.2061867)(614.07294804,35.13618677)(614.25294678,35.07619436)
\curveto(614.43294768,35.01618689)(614.6129475,34.95118695)(614.79294678,34.88119436)
\curveto(614.90294721,34.83118707)(615.0079471,34.78118712)(615.10794678,34.73119436)
\curveto(615.19794691,34.69118721)(615.26294685,34.6061873)(615.30294678,34.47619436)
\curveto(615.3129468,34.45618745)(615.31794679,34.43118747)(615.31794678,34.40119436)
\curveto(615.3079468,34.38118752)(615.3079468,34.35618755)(615.31794678,34.32619436)
\curveto(615.32794678,34.29618761)(615.33294678,34.26118764)(615.33294678,34.22119436)
\curveto(615.32294679,34.18118772)(615.31794679,34.14118776)(615.31794678,34.10119436)
\lineto(615.31794678,33.80119436)
\curveto(615.31794679,33.7011882)(615.29294682,33.62118828)(615.24294678,33.56119436)
\curveto(615.19294692,33.48118842)(615.12294699,33.42118848)(615.03294678,33.38119436)
\curveto(614.93294718,33.35118855)(614.83294728,33.31118859)(614.73294678,33.26119436)
\curveto(614.53294758,33.18118872)(614.32794778,33.1011888)(614.11794678,33.02119436)
\curveto(613.89794821,32.95118895)(613.68794842,32.87618903)(613.48794678,32.79619436)
\curveto(613.3079488,32.71618919)(613.12794898,32.64618926)(612.94794678,32.58619436)
\curveto(612.75794935,32.53618937)(612.57294954,32.47118943)(612.39294678,32.39119436)
\curveto(611.83295028,32.16118974)(611.26795084,31.94618996)(610.69794678,31.74619436)
\curveto(610.12795198,31.54619036)(609.56295255,31.33119057)(609.00294678,31.10119436)
\lineto(608.37294678,30.86119436)
\curveto(608.15295396,30.79119111)(607.94295417,30.71619119)(607.74294678,30.63619436)
\curveto(607.63295448,30.58619132)(607.52795458,30.54119136)(607.42794678,30.50119436)
\curveto(607.31795479,30.47119143)(607.22295489,30.42119148)(607.14294678,30.35119436)
\curveto(607.12295499,30.34119156)(607.112955,30.33119157)(607.11294678,30.32119436)
\lineto(607.08294678,30.29119436)
\lineto(607.08294678,30.21619436)
\lineto(607.11294678,30.18619436)
\curveto(607.112955,30.17619173)(607.11795499,30.16619174)(607.12794678,30.15619436)
\curveto(607.17795493,30.13619177)(607.23295488,30.12619178)(607.29294678,30.12619436)
\curveto(607.35295476,30.12619178)(607.4129547,30.11619179)(607.47294678,30.09619436)
\lineto(607.63794678,30.09619436)
\curveto(607.69795441,30.07619183)(607.76295435,30.07119183)(607.83294678,30.08119436)
\curveto(607.90295421,30.09119181)(607.97295414,30.09619181)(608.04294678,30.09619436)
\lineto(608.85294678,30.09619436)
\lineto(613.41294678,30.09619436)
\lineto(614.59794678,30.09619436)
\curveto(614.7079474,30.09619181)(614.81794729,30.09119181)(614.92794678,30.08119436)
\curveto(615.03794707,30.08119182)(615.12294699,30.05619185)(615.18294678,30.00619436)
\curveto(615.26294685,29.95619195)(615.3079468,29.86619204)(615.31794678,29.73619436)
\lineto(615.31794678,29.34619436)
\lineto(615.31794678,29.15119436)
\curveto(615.31794679,29.1011928)(615.3079468,29.05119285)(615.28794678,29.00119436)
\curveto(615.24794686,28.87119303)(615.16294695,28.79619311)(615.03294678,28.77619436)
\curveto(614.90294721,28.76619314)(614.75294736,28.76119314)(614.58294678,28.76119436)
\lineto(612.84294678,28.76119436)
\lineto(606.84294678,28.76119436)
\lineto(605.43294678,28.76119436)
\curveto(605.32295679,28.76119314)(605.2079569,28.75619315)(605.08794678,28.74619436)
\curveto(604.96795714,28.74619316)(604.87295724,28.77119313)(604.80294678,28.82119436)
\curveto(604.74295737,28.86119304)(604.69295742,28.93619297)(604.65294678,29.04619436)
\curveto(604.64295747,29.06619284)(604.64295747,29.08619282)(604.65294678,29.10619436)
\curveto(604.65295746,29.13619277)(604.64795746,29.16119274)(604.63794678,29.18119436)
}
}
{
\newrgbcolor{curcolor}{0 0 0}
\pscustom[linestyle=none,fillstyle=solid,fillcolor=curcolor]
{
\newpath
\moveto(614.76294678,48.38330373)
\curveto(614.92294719,48.4132959)(615.05794705,48.39829592)(615.16794678,48.33830373)
\curveto(615.26794684,48.27829604)(615.34294677,48.19829612)(615.39294678,48.09830373)
\curveto(615.4129467,48.04829627)(615.42294669,47.99329632)(615.42294678,47.93330373)
\curveto(615.42294669,47.88329643)(615.43294668,47.82829649)(615.45294678,47.76830373)
\curveto(615.50294661,47.54829677)(615.48794662,47.32829699)(615.40794678,47.10830373)
\curveto(615.33794677,46.89829742)(615.24794686,46.75329756)(615.13794678,46.67330373)
\curveto(615.06794704,46.62329769)(614.98794712,46.57829774)(614.89794678,46.53830373)
\curveto(614.79794731,46.49829782)(614.71794739,46.44829787)(614.65794678,46.38830373)
\curveto(614.63794747,46.36829795)(614.61794749,46.34329797)(614.59794678,46.31330373)
\curveto(614.57794753,46.29329802)(614.57294754,46.26329805)(614.58294678,46.22330373)
\curveto(614.6129475,46.1132982)(614.66794744,46.00829831)(614.74794678,45.90830373)
\curveto(614.82794728,45.8182985)(614.89794721,45.72829859)(614.95794678,45.63830373)
\curveto(615.03794707,45.50829881)(615.112947,45.36829895)(615.18294678,45.21830373)
\curveto(615.24294687,45.06829925)(615.29794681,44.90829941)(615.34794678,44.73830373)
\curveto(615.37794673,44.63829968)(615.39794671,44.52829979)(615.40794678,44.40830373)
\curveto(615.41794669,44.29830002)(615.43294668,44.18830013)(615.45294678,44.07830373)
\curveto(615.46294665,44.02830029)(615.46794664,43.98330033)(615.46794678,43.94330373)
\lineto(615.46794678,43.83830373)
\curveto(615.48794662,43.72830059)(615.48794662,43.62330069)(615.46794678,43.52330373)
\lineto(615.46794678,43.38830373)
\curveto(615.45794665,43.33830098)(615.45294666,43.28830103)(615.45294678,43.23830373)
\curveto(615.45294666,43.18830113)(615.44294667,43.14330117)(615.42294678,43.10330373)
\curveto(615.4129467,43.06330125)(615.4079467,43.02830129)(615.40794678,42.99830373)
\curveto(615.41794669,42.97830134)(615.41794669,42.95330136)(615.40794678,42.92330373)
\lineto(615.34794678,42.68330373)
\curveto(615.33794677,42.60330171)(615.31794679,42.52830179)(615.28794678,42.45830373)
\curveto(615.15794695,42.15830216)(615.0129471,41.9133024)(614.85294678,41.72330373)
\curveto(614.68294743,41.54330277)(614.44794766,41.39330292)(614.14794678,41.27330373)
\curveto(613.92794818,41.18330313)(613.66294845,41.13830318)(613.35294678,41.13830373)
\lineto(613.03794678,41.13830373)
\curveto(612.98794912,41.14830317)(612.93794917,41.15330316)(612.88794678,41.15330373)
\lineto(612.70794678,41.18330373)
\lineto(612.37794678,41.30330373)
\curveto(612.26794984,41.34330297)(612.16794994,41.39330292)(612.07794678,41.45330373)
\curveto(611.78795032,41.63330268)(611.57295054,41.87830244)(611.43294678,42.18830373)
\curveto(611.29295082,42.49830182)(611.16795094,42.83830148)(611.05794678,43.20830373)
\curveto(611.01795109,43.34830097)(610.98795112,43.49330082)(610.96794678,43.64330373)
\curveto(610.94795116,43.79330052)(610.92295119,43.94330037)(610.89294678,44.09330373)
\curveto(610.87295124,44.16330015)(610.86295125,44.22830009)(610.86294678,44.28830373)
\curveto(610.86295125,44.35829996)(610.85295126,44.43329988)(610.83294678,44.51330373)
\curveto(610.8129513,44.58329973)(610.80295131,44.65329966)(610.80294678,44.72330373)
\curveto(610.79295132,44.79329952)(610.77795133,44.86829945)(610.75794678,44.94830373)
\curveto(610.69795141,45.19829912)(610.64795146,45.43329888)(610.60794678,45.65330373)
\curveto(610.55795155,45.87329844)(610.44295167,46.04829827)(610.26294678,46.17830373)
\curveto(610.18295193,46.23829808)(610.08295203,46.28829803)(609.96294678,46.32830373)
\curveto(609.83295228,46.36829795)(609.69295242,46.36829795)(609.54294678,46.32830373)
\curveto(609.30295281,46.26829805)(609.112953,46.17829814)(608.97294678,46.05830373)
\curveto(608.83295328,45.94829837)(608.72295339,45.78829853)(608.64294678,45.57830373)
\curveto(608.59295352,45.45829886)(608.55795355,45.313299)(608.53794678,45.14330373)
\curveto(608.51795359,44.98329933)(608.5079536,44.8132995)(608.50794678,44.63330373)
\curveto(608.5079536,44.45329986)(608.51795359,44.27830004)(608.53794678,44.10830373)
\curveto(608.55795355,43.93830038)(608.58795352,43.79330052)(608.62794678,43.67330373)
\curveto(608.68795342,43.50330081)(608.77295334,43.33830098)(608.88294678,43.17830373)
\curveto(608.94295317,43.09830122)(609.02295309,43.02330129)(609.12294678,42.95330373)
\curveto(609.2129529,42.89330142)(609.3129528,42.83830148)(609.42294678,42.78830373)
\curveto(609.50295261,42.75830156)(609.58795252,42.72830159)(609.67794678,42.69830373)
\curveto(609.76795234,42.67830164)(609.83795227,42.63330168)(609.88794678,42.56330373)
\curveto(609.91795219,42.52330179)(609.94295217,42.45330186)(609.96294678,42.35330373)
\curveto(609.97295214,42.26330205)(609.97795213,42.16830215)(609.97794678,42.06830373)
\curveto(609.97795213,41.96830235)(609.97295214,41.86830245)(609.96294678,41.76830373)
\curveto(609.94295217,41.67830264)(609.91795219,41.6133027)(609.88794678,41.57330373)
\curveto(609.85795225,41.53330278)(609.8079523,41.50330281)(609.73794678,41.48330373)
\curveto(609.66795244,41.46330285)(609.59295252,41.46330285)(609.51294678,41.48330373)
\curveto(609.38295273,41.5133028)(609.26295285,41.54330277)(609.15294678,41.57330373)
\curveto(609.03295308,41.6133027)(608.91795319,41.65830266)(608.80794678,41.70830373)
\curveto(608.45795365,41.89830242)(608.18795392,42.13830218)(607.99794678,42.42830373)
\curveto(607.79795431,42.7183016)(607.63795447,43.07830124)(607.51794678,43.50830373)
\curveto(607.49795461,43.60830071)(607.48295463,43.70830061)(607.47294678,43.80830373)
\curveto(607.46295465,43.9183004)(607.44795466,44.02830029)(607.42794678,44.13830373)
\curveto(607.41795469,44.17830014)(607.41795469,44.24330007)(607.42794678,44.33330373)
\curveto(607.42795468,44.42329989)(607.41795469,44.47829984)(607.39794678,44.49830373)
\curveto(607.38795472,45.19829912)(607.46795464,45.80829851)(607.63794678,46.32830373)
\curveto(607.8079543,46.84829747)(608.13295398,47.2132971)(608.61294678,47.42330373)
\curveto(608.8129533,47.5132968)(609.04795306,47.56329675)(609.31794678,47.57330373)
\curveto(609.57795253,47.59329672)(609.85295226,47.60329671)(610.14294678,47.60330373)
\lineto(613.45794678,47.60330373)
\curveto(613.59794851,47.60329671)(613.73294838,47.60829671)(613.86294678,47.61830373)
\curveto(613.99294812,47.62829669)(614.09794801,47.65829666)(614.17794678,47.70830373)
\curveto(614.24794786,47.75829656)(614.29794781,47.82329649)(614.32794678,47.90330373)
\curveto(614.36794774,47.99329632)(614.39794771,48.07829624)(614.41794678,48.15830373)
\curveto(614.42794768,48.23829608)(614.47294764,48.29829602)(614.55294678,48.33830373)
\curveto(614.58294753,48.35829596)(614.6129475,48.36829595)(614.64294678,48.36830373)
\curveto(614.67294744,48.36829595)(614.7129474,48.37329594)(614.76294678,48.38330373)
\moveto(613.09794678,46.23830373)
\curveto(612.95794915,46.29829802)(612.79794931,46.32829799)(612.61794678,46.32830373)
\curveto(612.42794968,46.33829798)(612.23294988,46.34329797)(612.03294678,46.34330373)
\curveto(611.92295019,46.34329797)(611.82295029,46.33829798)(611.73294678,46.32830373)
\curveto(611.64295047,46.318298)(611.57295054,46.27829804)(611.52294678,46.20830373)
\curveto(611.50295061,46.17829814)(611.49295062,46.10829821)(611.49294678,45.99830373)
\curveto(611.5129506,45.97829834)(611.52295059,45.94329837)(611.52294678,45.89330373)
\curveto(611.52295059,45.84329847)(611.53295058,45.79829852)(611.55294678,45.75830373)
\curveto(611.57295054,45.67829864)(611.59295052,45.58829873)(611.61294678,45.48830373)
\lineto(611.67294678,45.18830373)
\curveto(611.67295044,45.15829916)(611.67795043,45.12329919)(611.68794678,45.08330373)
\lineto(611.68794678,44.97830373)
\curveto(611.72795038,44.82829949)(611.75295036,44.66329965)(611.76294678,44.48330373)
\curveto(611.76295035,44.3133)(611.78295033,44.15330016)(611.82294678,44.00330373)
\curveto(611.84295027,43.92330039)(611.86295025,43.84830047)(611.88294678,43.77830373)
\curveto(611.89295022,43.7183006)(611.9079502,43.64830067)(611.92794678,43.56830373)
\curveto(611.97795013,43.40830091)(612.04295007,43.25830106)(612.12294678,43.11830373)
\curveto(612.19294992,42.97830134)(612.28294983,42.85830146)(612.39294678,42.75830373)
\curveto(612.50294961,42.65830166)(612.63794947,42.58330173)(612.79794678,42.53330373)
\curveto(612.94794916,42.48330183)(613.13294898,42.46330185)(613.35294678,42.47330373)
\curveto(613.45294866,42.47330184)(613.54794856,42.48830183)(613.63794678,42.51830373)
\curveto(613.71794839,42.55830176)(613.79294832,42.60330171)(613.86294678,42.65330373)
\curveto(613.97294814,42.73330158)(614.06794804,42.83830148)(614.14794678,42.96830373)
\curveto(614.21794789,43.09830122)(614.27794783,43.23830108)(614.32794678,43.38830373)
\curveto(614.33794777,43.43830088)(614.34294777,43.48830083)(614.34294678,43.53830373)
\curveto(614.34294777,43.58830073)(614.34794776,43.63830068)(614.35794678,43.68830373)
\curveto(614.37794773,43.75830056)(614.39294772,43.84330047)(614.40294678,43.94330373)
\curveto(614.40294771,44.05330026)(614.39294772,44.14330017)(614.37294678,44.21330373)
\curveto(614.35294776,44.27330004)(614.34794776,44.33329998)(614.35794678,44.39330373)
\curveto(614.35794775,44.45329986)(614.34794776,44.5132998)(614.32794678,44.57330373)
\curveto(614.3079478,44.65329966)(614.29294782,44.72829959)(614.28294678,44.79830373)
\curveto(614.27294784,44.87829944)(614.25294786,44.95329936)(614.22294678,45.02330373)
\curveto(614.10294801,45.313299)(613.95794815,45.55829876)(613.78794678,45.75830373)
\curveto(613.61794849,45.96829835)(613.38794872,46.12829819)(613.09794678,46.23830373)
}
}
{
\newrgbcolor{curcolor}{0 0 0}
\pscustom[linestyle=none,fillstyle=solid,fillcolor=curcolor]
{
\newpath
\moveto(607.59294678,49.26994436)
\lineto(607.59294678,49.71994436)
\curveto(607.58295453,49.88994311)(607.60295451,50.01494298)(607.65294678,50.09494436)
\curveto(607.70295441,50.17494282)(607.76795434,50.22994277)(607.84794678,50.25994436)
\curveto(607.92795418,50.2999427)(608.0129541,50.33994266)(608.10294678,50.37994436)
\curveto(608.23295388,50.42994257)(608.36295375,50.47494252)(608.49294678,50.51494436)
\curveto(608.62295349,50.55494244)(608.75295336,50.5999424)(608.88294678,50.64994436)
\curveto(609.00295311,50.6999423)(609.12795298,50.74494225)(609.25794678,50.78494436)
\curveto(609.37795273,50.82494217)(609.49795261,50.86994213)(609.61794678,50.91994436)
\curveto(609.72795238,50.96994203)(609.84295227,51.00994199)(609.96294678,51.03994436)
\curveto(610.08295203,51.06994193)(610.20295191,51.10994189)(610.32294678,51.15994436)
\curveto(610.6129515,51.27994172)(610.9129512,51.38994161)(611.22294678,51.48994436)
\curveto(611.53295058,51.58994141)(611.83295028,51.6999413)(612.12294678,51.81994436)
\curveto(612.16294995,51.83994116)(612.20294991,51.84994115)(612.24294678,51.84994436)
\curveto(612.27294984,51.84994115)(612.30294981,51.85994114)(612.33294678,51.87994436)
\curveto(612.47294964,51.93994106)(612.61794949,51.994941)(612.76794678,52.04494436)
\lineto(613.18794678,52.19494436)
\curveto(613.25794885,52.22494077)(613.33294878,52.25494074)(613.41294678,52.28494436)
\curveto(613.48294863,52.31494068)(613.52794858,52.36494063)(613.54794678,52.43494436)
\curveto(613.57794853,52.51494048)(613.55294856,52.57494042)(613.47294678,52.61494436)
\curveto(613.38294873,52.66494033)(613.3129488,52.6999403)(613.26294678,52.71994436)
\curveto(613.09294902,52.7999402)(612.9129492,52.86494013)(612.72294678,52.91494436)
\curveto(612.53294958,52.96494003)(612.34794976,53.02493997)(612.16794678,53.09494436)
\curveto(611.93795017,53.18493981)(611.7079504,53.26493973)(611.47794678,53.33494436)
\curveto(611.23795087,53.40493959)(611.0079511,53.48993951)(610.78794678,53.58994436)
\curveto(610.73795137,53.5999394)(610.67295144,53.61493938)(610.59294678,53.63494436)
\curveto(610.50295161,53.67493932)(610.4129517,53.70993929)(610.32294678,53.73994436)
\curveto(610.22295189,53.76993923)(610.13295198,53.7999392)(610.05294678,53.82994436)
\curveto(610.00295211,53.84993915)(609.95795215,53.86493913)(609.91794678,53.87494436)
\curveto(609.87795223,53.88493911)(609.83295228,53.8999391)(609.78294678,53.91994436)
\curveto(609.66295245,53.96993903)(609.54295257,54.00993899)(609.42294678,54.03994436)
\curveto(609.29295282,54.07993892)(609.16795294,54.12493887)(609.04794678,54.17494436)
\curveto(608.99795311,54.1949388)(608.95295316,54.20993879)(608.91294678,54.21994436)
\curveto(608.87295324,54.22993877)(608.82795328,54.24493875)(608.77794678,54.26494436)
\curveto(608.68795342,54.30493869)(608.59795351,54.33993866)(608.50794678,54.36994436)
\curveto(608.4079537,54.3999386)(608.3129538,54.42993857)(608.22294678,54.45994436)
\curveto(608.14295397,54.48993851)(608.06295405,54.51493848)(607.98294678,54.53494436)
\curveto(607.89295422,54.56493843)(607.81795429,54.60493839)(607.75794678,54.65494436)
\curveto(607.66795444,54.72493827)(607.61795449,54.81993818)(607.60794678,54.93994436)
\curveto(607.59795451,55.06993793)(607.59295452,55.20993779)(607.59294678,55.35994436)
\curveto(607.59295452,55.43993756)(607.59795451,55.51493748)(607.60794678,55.58494436)
\curveto(607.6079545,55.66493733)(607.62295449,55.72993727)(607.65294678,55.77994436)
\curveto(607.7129544,55.86993713)(607.8079543,55.8949371)(607.93794678,55.85494436)
\curveto(608.06795404,55.81493718)(608.16795394,55.77993722)(608.23794678,55.74994436)
\lineto(608.29794678,55.71994436)
\curveto(608.31795379,55.71993728)(608.33795377,55.71493728)(608.35794678,55.70494436)
\curveto(608.63795347,55.5949374)(608.92295319,55.48493751)(609.21294678,55.37494436)
\lineto(610.05294678,55.04494436)
\curveto(610.13295198,55.01493798)(610.2079519,54.98993801)(610.27794678,54.96994436)
\curveto(610.33795177,54.94993805)(610.40295171,54.92493807)(610.47294678,54.89494436)
\curveto(610.67295144,54.81493818)(610.87795123,54.73493826)(611.08794678,54.65494436)
\curveto(611.28795082,54.58493841)(611.48795062,54.50993849)(611.68794678,54.42994436)
\curveto(612.37794973,54.13993886)(613.07294904,53.86993913)(613.77294678,53.61994436)
\curveto(614.47294764,53.36993963)(615.16794694,53.0999399)(615.85794678,52.80994436)
\lineto(616.00794678,52.74994436)
\curveto(616.06794604,52.73994026)(616.12794598,52.72494027)(616.18794678,52.70494436)
\curveto(616.55794555,52.54494045)(616.92294519,52.37494062)(617.28294678,52.19494436)
\curveto(617.65294446,52.01494098)(617.93794417,51.76494123)(618.13794678,51.44494436)
\curveto(618.19794391,51.33494166)(618.24294387,51.22494177)(618.27294678,51.11494436)
\curveto(618.3129438,51.00494199)(618.34794376,50.87994212)(618.37794678,50.73994436)
\curveto(618.39794371,50.68994231)(618.40294371,50.63494236)(618.39294678,50.57494436)
\curveto(618.38294373,50.52494247)(618.38294373,50.46994253)(618.39294678,50.40994436)
\curveto(618.4129437,50.32994267)(618.4129437,50.24994275)(618.39294678,50.16994436)
\curveto(618.38294373,50.12994287)(618.37794373,50.07994292)(618.37794678,50.01994436)
\lineto(618.31794678,49.77994436)
\curveto(618.29794381,49.70994329)(618.25794385,49.65494334)(618.19794678,49.61494436)
\curveto(618.13794397,49.56494343)(618.06294405,49.53494346)(617.97294678,49.52494436)
\lineto(617.70294678,49.52494436)
\lineto(617.49294678,49.52494436)
\curveto(617.43294468,49.53494346)(617.38294473,49.55494344)(617.34294678,49.58494436)
\curveto(617.23294488,49.65494334)(617.20294491,49.77494322)(617.25294678,49.94494436)
\curveto(617.27294484,50.05494294)(617.28294483,50.17494282)(617.28294678,50.30494436)
\curveto(617.28294483,50.43494256)(617.26294485,50.54994245)(617.22294678,50.64994436)
\curveto(617.17294494,50.7999422)(617.09794501,50.91994208)(616.99794678,51.00994436)
\curveto(616.89794521,51.10994189)(616.78294533,51.1949418)(616.65294678,51.26494436)
\curveto(616.53294558,51.33494166)(616.40294571,51.3949416)(616.26294678,51.44494436)
\lineto(615.84294678,51.62494436)
\curveto(615.75294636,51.66494133)(615.64294647,51.70494129)(615.51294678,51.74494436)
\curveto(615.38294673,51.7949412)(615.24794686,51.7999412)(615.10794678,51.75994436)
\curveto(614.94794716,51.70994129)(614.79794731,51.65494134)(614.65794678,51.59494436)
\curveto(614.51794759,51.54494145)(614.37794773,51.48994151)(614.23794678,51.42994436)
\curveto(614.02794808,51.33994166)(613.81794829,51.25494174)(613.60794678,51.17494436)
\curveto(613.39794871,51.0949419)(613.19294892,51.01494198)(612.99294678,50.93494436)
\curveto(612.85294926,50.87494212)(612.71794939,50.81994218)(612.58794678,50.76994436)
\curveto(612.45794965,50.71994228)(612.32294979,50.66994233)(612.18294678,50.61994436)
\lineto(610.86294678,50.07994436)
\curveto(610.42295169,49.90994309)(609.98295213,49.73494326)(609.54294678,49.55494436)
\curveto(609.3129528,49.45494354)(609.09295302,49.36494363)(608.88294678,49.28494436)
\curveto(608.66295345,49.20494379)(608.44295367,49.11994388)(608.22294678,49.02994436)
\curveto(608.16295395,49.00994399)(608.08295403,48.97994402)(607.98294678,48.93994436)
\curveto(607.87295424,48.8999441)(607.78295433,48.90494409)(607.71294678,48.95494436)
\curveto(607.66295445,48.98494401)(607.62795448,49.04494395)(607.60794678,49.13494436)
\curveto(607.59795451,49.15494384)(607.59795451,49.17494382)(607.60794678,49.19494436)
\curveto(607.6079545,49.22494377)(607.60295451,49.24994375)(607.59294678,49.26994436)
}
}
{
\newrgbcolor{curcolor}{0 0 0}
\pscustom[linestyle=none,fillstyle=solid,fillcolor=curcolor]
{
}
}
{
\newrgbcolor{curcolor}{0 0 0}
\pscustom[linestyle=none,fillstyle=solid,fillcolor=curcolor]
{
\newpath
\moveto(604.71294678,64.24510061)
\curveto(604.70295741,64.93509597)(604.82295729,65.53509537)(605.07294678,66.04510061)
\curveto(605.32295679,66.56509434)(605.65795645,66.96009395)(606.07794678,67.23010061)
\curveto(606.15795595,67.28009363)(606.24795586,67.32509358)(606.34794678,67.36510061)
\curveto(606.43795567,67.4050935)(606.53295558,67.45009346)(606.63294678,67.50010061)
\curveto(606.73295538,67.54009337)(606.83295528,67.57009334)(606.93294678,67.59010061)
\curveto(607.03295508,67.6100933)(607.13795497,67.63009328)(607.24794678,67.65010061)
\curveto(607.29795481,67.67009324)(607.34295477,67.67509323)(607.38294678,67.66510061)
\curveto(607.42295469,67.65509325)(607.46795464,67.66009325)(607.51794678,67.68010061)
\curveto(607.56795454,67.69009322)(607.65295446,67.69509321)(607.77294678,67.69510061)
\curveto(607.88295423,67.69509321)(607.96795414,67.69009322)(608.02794678,67.68010061)
\curveto(608.08795402,67.66009325)(608.14795396,67.65009326)(608.20794678,67.65010061)
\curveto(608.26795384,67.66009325)(608.32795378,67.65509325)(608.38794678,67.63510061)
\curveto(608.52795358,67.59509331)(608.66295345,67.56009335)(608.79294678,67.53010061)
\curveto(608.92295319,67.50009341)(609.04795306,67.46009345)(609.16794678,67.41010061)
\curveto(609.3079528,67.35009356)(609.43295268,67.28009363)(609.54294678,67.20010061)
\curveto(609.65295246,67.13009378)(609.76295235,67.05509385)(609.87294678,66.97510061)
\lineto(609.93294678,66.91510061)
\curveto(609.95295216,66.905094)(609.97295214,66.89009402)(609.99294678,66.87010061)
\curveto(610.15295196,66.75009416)(610.29795181,66.61509429)(610.42794678,66.46510061)
\curveto(610.55795155,66.31509459)(610.68295143,66.15509475)(610.80294678,65.98510061)
\curveto(611.02295109,65.67509523)(611.22795088,65.38009553)(611.41794678,65.10010061)
\curveto(611.55795055,64.87009604)(611.69295042,64.64009627)(611.82294678,64.41010061)
\curveto(611.95295016,64.19009672)(612.08795002,63.97009694)(612.22794678,63.75010061)
\curveto(612.39794971,63.50009741)(612.57794953,63.26009765)(612.76794678,63.03010061)
\curveto(612.95794915,62.8100981)(613.18294893,62.62009829)(613.44294678,62.46010061)
\curveto(613.50294861,62.42009849)(613.56294855,62.38509852)(613.62294678,62.35510061)
\curveto(613.67294844,62.32509858)(613.73794837,62.29509861)(613.81794678,62.26510061)
\curveto(613.88794822,62.24509866)(613.94794816,62.24009867)(613.99794678,62.25010061)
\curveto(614.06794804,62.27009864)(614.12294799,62.3050986)(614.16294678,62.35510061)
\curveto(614.19294792,62.4050985)(614.2129479,62.46509844)(614.22294678,62.53510061)
\lineto(614.22294678,62.77510061)
\lineto(614.22294678,63.52510061)
\lineto(614.22294678,66.33010061)
\lineto(614.22294678,66.99010061)
\curveto(614.22294789,67.08009383)(614.22794788,67.16509374)(614.23794678,67.24510061)
\curveto(614.23794787,67.32509358)(614.25794785,67.39009352)(614.29794678,67.44010061)
\curveto(614.33794777,67.49009342)(614.4129477,67.53009338)(614.52294678,67.56010061)
\curveto(614.62294749,67.60009331)(614.72294739,67.6100933)(614.82294678,67.59010061)
\lineto(614.95794678,67.59010061)
\curveto(615.02794708,67.57009334)(615.08794702,67.55009336)(615.13794678,67.53010061)
\curveto(615.18794692,67.5100934)(615.22794688,67.47509343)(615.25794678,67.42510061)
\curveto(615.29794681,67.37509353)(615.31794679,67.3050936)(615.31794678,67.21510061)
\lineto(615.31794678,66.94510061)
\lineto(615.31794678,66.04510061)
\lineto(615.31794678,62.53510061)
\lineto(615.31794678,61.47010061)
\curveto(615.31794679,61.39009952)(615.32294679,61.30009961)(615.33294678,61.20010061)
\curveto(615.33294678,61.10009981)(615.32294679,61.01509989)(615.30294678,60.94510061)
\curveto(615.23294688,60.73510017)(615.05294706,60.67010024)(614.76294678,60.75010061)
\curveto(614.72294739,60.76010015)(614.68794742,60.76010015)(614.65794678,60.75010061)
\curveto(614.61794749,60.75010016)(614.57294754,60.76010015)(614.52294678,60.78010061)
\curveto(614.44294767,60.80010011)(614.35794775,60.82010009)(614.26794678,60.84010061)
\curveto(614.17794793,60.86010005)(614.09294802,60.88510002)(614.01294678,60.91510061)
\curveto(613.52294859,61.07509983)(613.107949,61.27509963)(612.76794678,61.51510061)
\curveto(612.51794959,61.69509921)(612.29294982,61.90009901)(612.09294678,62.13010061)
\curveto(611.88295023,62.36009855)(611.68795042,62.60009831)(611.50794678,62.85010061)
\curveto(611.32795078,63.1100978)(611.15795095,63.37509753)(610.99794678,63.64510061)
\curveto(610.82795128,63.92509698)(610.65295146,64.19509671)(610.47294678,64.45510061)
\curveto(610.39295172,64.56509634)(610.31795179,64.67009624)(610.24794678,64.77010061)
\curveto(610.17795193,64.88009603)(610.10295201,64.99009592)(610.02294678,65.10010061)
\curveto(609.99295212,65.14009577)(609.96295215,65.17509573)(609.93294678,65.20510061)
\curveto(609.89295222,65.24509566)(609.86295225,65.28509562)(609.84294678,65.32510061)
\curveto(609.73295238,65.46509544)(609.6079525,65.59009532)(609.46794678,65.70010061)
\curveto(609.43795267,65.72009519)(609.4129527,65.74509516)(609.39294678,65.77510061)
\curveto(609.36295275,65.8050951)(609.33295278,65.83009508)(609.30294678,65.85010061)
\curveto(609.20295291,65.93009498)(609.10295301,65.99509491)(609.00294678,66.04510061)
\curveto(608.90295321,66.1050948)(608.79295332,66.16009475)(608.67294678,66.21010061)
\curveto(608.60295351,66.24009467)(608.52795358,66.26009465)(608.44794678,66.27010061)
\lineto(608.20794678,66.33010061)
\lineto(608.11794678,66.33010061)
\curveto(608.08795402,66.34009457)(608.05795405,66.34509456)(608.02794678,66.34510061)
\curveto(607.95795415,66.36509454)(607.86295425,66.37009454)(607.74294678,66.36010061)
\curveto(607.6129545,66.36009455)(607.5129546,66.35009456)(607.44294678,66.33010061)
\curveto(607.36295475,66.3100946)(607.28795482,66.29009462)(607.21794678,66.27010061)
\curveto(607.13795497,66.26009465)(607.05795505,66.24009467)(606.97794678,66.21010061)
\curveto(606.73795537,66.10009481)(606.53795557,65.95009496)(606.37794678,65.76010061)
\curveto(606.2079559,65.58009533)(606.06795604,65.36009555)(605.95794678,65.10010061)
\curveto(605.93795617,65.03009588)(605.92295619,64.96009595)(605.91294678,64.89010061)
\curveto(605.89295622,64.82009609)(605.87295624,64.74509616)(605.85294678,64.66510061)
\curveto(605.83295628,64.58509632)(605.82295629,64.47509643)(605.82294678,64.33510061)
\curveto(605.82295629,64.2050967)(605.83295628,64.10009681)(605.85294678,64.02010061)
\curveto(605.86295625,63.96009695)(605.86795624,63.905097)(605.86794678,63.85510061)
\curveto(605.86795624,63.8050971)(605.87795623,63.75509715)(605.89794678,63.70510061)
\curveto(605.93795617,63.6050973)(605.97795613,63.5100974)(606.01794678,63.42010061)
\curveto(606.05795605,63.34009757)(606.10295601,63.26009765)(606.15294678,63.18010061)
\curveto(606.17295594,63.15009776)(606.19795591,63.12009779)(606.22794678,63.09010061)
\curveto(606.25795585,63.07009784)(606.28295583,63.04509786)(606.30294678,63.01510061)
\lineto(606.37794678,62.94010061)
\curveto(606.39795571,62.910098)(606.41795569,62.88509802)(606.43794678,62.86510061)
\lineto(606.64794678,62.71510061)
\curveto(606.7079554,62.67509823)(606.77295534,62.63009828)(606.84294678,62.58010061)
\curveto(606.93295518,62.52009839)(607.03795507,62.47009844)(607.15794678,62.43010061)
\curveto(607.26795484,62.40009851)(607.37795473,62.36509854)(607.48794678,62.32510061)
\curveto(607.59795451,62.28509862)(607.74295437,62.26009865)(607.92294678,62.25010061)
\curveto(608.09295402,62.24009867)(608.21795389,62.2100987)(608.29794678,62.16010061)
\curveto(608.37795373,62.1100988)(608.42295369,62.03509887)(608.43294678,61.93510061)
\curveto(608.44295367,61.83509907)(608.44795366,61.72509918)(608.44794678,61.60510061)
\curveto(608.44795366,61.56509934)(608.45295366,61.52509938)(608.46294678,61.48510061)
\curveto(608.46295365,61.44509946)(608.45795365,61.4100995)(608.44794678,61.38010061)
\curveto(608.42795368,61.33009958)(608.41795369,61.28009963)(608.41794678,61.23010061)
\curveto(608.41795369,61.19009972)(608.4079537,61.15009976)(608.38794678,61.11010061)
\curveto(608.32795378,61.02009989)(608.19295392,60.97509993)(607.98294678,60.97510061)
\lineto(607.86294678,60.97510061)
\curveto(607.80295431,60.98509992)(607.74295437,60.99009992)(607.68294678,60.99010061)
\curveto(607.6129545,61.00009991)(607.54795456,61.0100999)(607.48794678,61.02010061)
\curveto(607.37795473,61.04009987)(607.27795483,61.06009985)(607.18794678,61.08010061)
\curveto(607.08795502,61.10009981)(606.99295512,61.13009978)(606.90294678,61.17010061)
\curveto(606.83295528,61.19009972)(606.77295534,61.2100997)(606.72294678,61.23010061)
\lineto(606.54294678,61.29010061)
\curveto(606.28295583,61.4100995)(606.03795607,61.56509934)(605.80794678,61.75510061)
\curveto(605.57795653,61.95509895)(605.39295672,62.17009874)(605.25294678,62.40010061)
\curveto(605.17295694,62.5100984)(605.107957,62.62509828)(605.05794678,62.74510061)
\lineto(604.90794678,63.13510061)
\curveto(604.85795725,63.24509766)(604.82795728,63.36009755)(604.81794678,63.48010061)
\curveto(604.79795731,63.60009731)(604.77295734,63.72509718)(604.74294678,63.85510061)
\curveto(604.74295737,63.92509698)(604.74295737,63.99009692)(604.74294678,64.05010061)
\curveto(604.73295738,64.1100968)(604.72295739,64.17509673)(604.71294678,64.24510061)
}
}
{
\newrgbcolor{curcolor}{0 0 0}
\pscustom[linestyle=none,fillstyle=solid,fillcolor=curcolor]
{
\newpath
\moveto(604.90794678,70.85470998)
\lineto(604.90794678,74.45470998)
\lineto(604.90794678,75.09970998)
\curveto(604.9079572,75.17970345)(604.9129572,75.25470338)(604.92294678,75.32470998)
\curveto(604.92295719,75.39470324)(604.93295718,75.45470318)(604.95294678,75.50470998)
\curveto(604.98295713,75.57470306)(605.04295707,75.629703)(605.13294678,75.66970998)
\curveto(605.16295695,75.68970294)(605.20295691,75.69970293)(605.25294678,75.69970998)
\lineto(605.38794678,75.69970998)
\curveto(605.49795661,75.70970292)(605.60295651,75.70470293)(605.70294678,75.68470998)
\curveto(605.80295631,75.67470296)(605.87295624,75.63970299)(605.91294678,75.57970998)
\curveto(605.98295613,75.48970314)(606.01795609,75.35470328)(606.01794678,75.17470998)
\curveto(606.0079561,74.99470364)(606.00295611,74.8297038)(606.00294678,74.67970998)
\lineto(606.00294678,72.68470998)
\lineto(606.00294678,72.18970998)
\lineto(606.00294678,72.05470998)
\curveto(606.00295611,72.01470662)(606.0079561,71.97470666)(606.01794678,71.93470998)
\lineto(606.01794678,71.72470998)
\curveto(606.04795606,71.61470702)(606.08795602,71.5347071)(606.13794678,71.48470998)
\curveto(606.17795593,71.4347072)(606.23295588,71.39970723)(606.30294678,71.37970998)
\curveto(606.36295575,71.35970727)(606.43295568,71.34470729)(606.51294678,71.33470998)
\curveto(606.59295552,71.32470731)(606.68295543,71.30470733)(606.78294678,71.27470998)
\curveto(606.98295513,71.22470741)(607.18795492,71.18470745)(607.39794678,71.15470998)
\curveto(607.6079545,71.12470751)(607.8129543,71.08470755)(608.01294678,71.03470998)
\curveto(608.08295403,71.01470762)(608.15295396,71.00470763)(608.22294678,71.00470998)
\curveto(608.28295383,71.00470763)(608.34795376,70.99470764)(608.41794678,70.97470998)
\curveto(608.44795366,70.96470767)(608.48795362,70.95470768)(608.53794678,70.94470998)
\curveto(608.57795353,70.94470769)(608.61795349,70.94970768)(608.65794678,70.95970998)
\curveto(608.7079534,70.97970765)(608.75295336,71.00470763)(608.79294678,71.03470998)
\curveto(608.82295329,71.07470756)(608.82795328,71.1347075)(608.80794678,71.21470998)
\curveto(608.78795332,71.27470736)(608.76295335,71.3347073)(608.73294678,71.39470998)
\curveto(608.69295342,71.45470718)(608.65795345,71.51470712)(608.62794678,71.57470998)
\curveto(608.6079535,71.634707)(608.59295352,71.68470695)(608.58294678,71.72470998)
\curveto(608.50295361,71.91470672)(608.44795366,72.11970651)(608.41794678,72.33970998)
\curveto(608.38795372,72.56970606)(608.37795373,72.79970583)(608.38794678,73.02970998)
\curveto(608.38795372,73.26970536)(608.4129537,73.49970513)(608.46294678,73.71970998)
\curveto(608.50295361,73.93970469)(608.56295355,74.13970449)(608.64294678,74.31970998)
\curveto(608.66295345,74.36970426)(608.68295343,74.41470422)(608.70294678,74.45470998)
\curveto(608.72295339,74.50470413)(608.74795336,74.55470408)(608.77794678,74.60470998)
\curveto(608.98795312,74.95470368)(609.21795289,75.2347034)(609.46794678,75.44470998)
\curveto(609.71795239,75.66470297)(610.04295207,75.85970277)(610.44294678,76.02970998)
\curveto(610.55295156,76.07970255)(610.66295145,76.11470252)(610.77294678,76.13470998)
\curveto(610.88295123,76.15470248)(610.99795111,76.17970245)(611.11794678,76.20970998)
\curveto(611.14795096,76.21970241)(611.19295092,76.22470241)(611.25294678,76.22470998)
\curveto(611.3129508,76.24470239)(611.38295073,76.25470238)(611.46294678,76.25470998)
\curveto(611.53295058,76.25470238)(611.59795051,76.26470237)(611.65794678,76.28470998)
\lineto(611.82294678,76.28470998)
\curveto(611.87295024,76.29470234)(611.94295017,76.29970233)(612.03294678,76.29970998)
\curveto(612.12294999,76.29970233)(612.19294992,76.28970234)(612.24294678,76.26970998)
\curveto(612.30294981,76.24970238)(612.36294975,76.24470239)(612.42294678,76.25470998)
\curveto(612.47294964,76.26470237)(612.52294959,76.25970237)(612.57294678,76.23970998)
\curveto(612.73294938,76.19970243)(612.88294923,76.16470247)(613.02294678,76.13470998)
\curveto(613.16294895,76.10470253)(613.29794881,76.05970257)(613.42794678,75.99970998)
\curveto(613.79794831,75.83970279)(614.13294798,75.61970301)(614.43294678,75.33970998)
\curveto(614.73294738,75.05970357)(614.96294715,74.73970389)(615.12294678,74.37970998)
\curveto(615.20294691,74.20970442)(615.27794683,74.00970462)(615.34794678,73.77970998)
\curveto(615.38794672,73.66970496)(615.4129467,73.55470508)(615.42294678,73.43470998)
\curveto(615.43294668,73.31470532)(615.45294666,73.19470544)(615.48294678,73.07470998)
\curveto(615.50294661,73.02470561)(615.50294661,72.96970566)(615.48294678,72.90970998)
\curveto(615.47294664,72.84970578)(615.47794663,72.78970584)(615.49794678,72.72970998)
\curveto(615.51794659,72.629706)(615.51794659,72.5297061)(615.49794678,72.42970998)
\lineto(615.49794678,72.29470998)
\curveto(615.47794663,72.24470639)(615.46794664,72.18470645)(615.46794678,72.11470998)
\curveto(615.47794663,72.05470658)(615.47294664,71.99970663)(615.45294678,71.94970998)
\curveto(615.44294667,71.90970672)(615.43794667,71.87470676)(615.43794678,71.84470998)
\curveto(615.43794667,71.81470682)(615.43294668,71.77970685)(615.42294678,71.73970998)
\lineto(615.36294678,71.46970998)
\curveto(615.34294677,71.37970725)(615.3129468,71.29470734)(615.27294678,71.21470998)
\curveto(615.13294698,70.87470776)(614.97794713,70.58470805)(614.80794678,70.34470998)
\curveto(614.62794748,70.10470853)(614.39794771,69.88470875)(614.11794678,69.68470998)
\curveto(613.88794822,69.5347091)(613.64794846,69.41970921)(613.39794678,69.33970998)
\curveto(613.34794876,69.31970931)(613.30294881,69.30970932)(613.26294678,69.30970998)
\curveto(613.2129489,69.30970932)(613.16294895,69.29970933)(613.11294678,69.27970998)
\curveto(613.05294906,69.25970937)(612.97294914,69.24470939)(612.87294678,69.23470998)
\curveto(612.77294934,69.2347094)(612.69794941,69.25470938)(612.64794678,69.29470998)
\curveto(612.56794954,69.34470929)(612.52294959,69.42470921)(612.51294678,69.53470998)
\curveto(612.50294961,69.64470899)(612.49794961,69.75970887)(612.49794678,69.87970998)
\lineto(612.49794678,70.04470998)
\curveto(612.49794961,70.10470853)(612.5079496,70.15970847)(612.52794678,70.20970998)
\curveto(612.54794956,70.29970833)(612.58794952,70.36970826)(612.64794678,70.41970998)
\curveto(612.73794937,70.48970814)(612.84794926,70.5347081)(612.97794678,70.55470998)
\curveto(613.09794901,70.58470805)(613.20294891,70.629708)(613.29294678,70.68970998)
\curveto(613.63294848,70.87970775)(613.90294821,71.13970749)(614.10294678,71.46970998)
\curveto(614.16294795,71.56970706)(614.2129479,71.67470696)(614.25294678,71.78470998)
\curveto(614.28294783,71.90470673)(614.31794779,72.02470661)(614.35794678,72.14470998)
\curveto(614.4079477,72.31470632)(614.42794768,72.51970611)(614.41794678,72.75970998)
\curveto(614.39794771,73.00970562)(614.36294775,73.20970542)(614.31294678,73.35970998)
\curveto(614.19294792,73.7297049)(614.03294808,74.01970461)(613.83294678,74.22970998)
\curveto(613.62294849,74.44970418)(613.34294877,74.629704)(612.99294678,74.76970998)
\curveto(612.89294922,74.81970381)(612.78794932,74.84970378)(612.67794678,74.85970998)
\curveto(612.56794954,74.87970375)(612.45294966,74.90470373)(612.33294678,74.93470998)
\lineto(612.22794678,74.93470998)
\curveto(612.18794992,74.94470369)(612.14794996,74.94970368)(612.10794678,74.94970998)
\curveto(612.07795003,74.95970367)(612.04295007,74.95970367)(612.00294678,74.94970998)
\lineto(611.88294678,74.94970998)
\curveto(611.62295049,74.94970368)(611.37795073,74.91970371)(611.14794678,74.85970998)
\curveto(610.79795131,74.74970388)(610.50295161,74.59470404)(610.26294678,74.39470998)
\curveto(610.0129521,74.19470444)(609.81795229,73.9347047)(609.67794678,73.61470998)
\lineto(609.61794678,73.43470998)
\curveto(609.59795251,73.38470525)(609.57795253,73.32470531)(609.55794678,73.25470998)
\curveto(609.53795257,73.20470543)(609.52795258,73.14470549)(609.52794678,73.07470998)
\curveto(609.51795259,73.01470562)(609.50295261,72.94970568)(609.48294678,72.87970998)
\lineto(609.48294678,72.72970998)
\curveto(609.46295265,72.68970594)(609.45295266,72.634706)(609.45294678,72.56470998)
\curveto(609.45295266,72.50470613)(609.46295265,72.44970618)(609.48294678,72.39970998)
\lineto(609.48294678,72.29470998)
\curveto(609.48295263,72.26470637)(609.48795262,72.2297064)(609.49794678,72.18970998)
\lineto(609.55794678,71.94970998)
\curveto(609.56795254,71.86970676)(609.58795252,71.78970684)(609.61794678,71.70970998)
\curveto(609.71795239,71.46970716)(609.85295226,71.23970739)(610.02294678,71.01970998)
\curveto(610.09295202,70.9297077)(610.16795194,70.84470779)(610.24794678,70.76470998)
\curveto(610.31795179,70.68470795)(610.37295174,70.58470805)(610.41294678,70.46470998)
\curveto(610.44295167,70.37470826)(610.45295166,70.2347084)(610.44294678,70.04470998)
\curveto(610.43295168,69.86470877)(610.4079517,69.74470889)(610.36794678,69.68470998)
\curveto(610.32795178,69.634709)(610.26795184,69.59470904)(610.18794678,69.56470998)
\curveto(610.107952,69.54470909)(610.02295209,69.54470909)(609.93294678,69.56470998)
\curveto(609.8129523,69.59470904)(609.69295242,69.61470902)(609.57294678,69.62470998)
\curveto(609.44295267,69.64470899)(609.31795279,69.66970896)(609.19794678,69.69970998)
\curveto(609.15795295,69.71970891)(609.12295299,69.72470891)(609.09294678,69.71470998)
\curveto(609.05295306,69.71470892)(609.0079531,69.72470891)(608.95794678,69.74470998)
\curveto(608.86795324,69.76470887)(608.77795333,69.77970885)(608.68794678,69.78970998)
\curveto(608.58795352,69.79970883)(608.49295362,69.81970881)(608.40294678,69.84970998)
\curveto(608.34295377,69.85970877)(608.28295383,69.86470877)(608.22294678,69.86470998)
\curveto(608.16295395,69.87470876)(608.10295401,69.88970874)(608.04294678,69.90970998)
\curveto(607.84295427,69.95970867)(607.63795447,69.99470864)(607.42794678,70.01470998)
\curveto(607.2079549,70.04470859)(606.99795511,70.08470855)(606.79794678,70.13470998)
\curveto(606.69795541,70.16470847)(606.59795551,70.18470845)(606.49794678,70.19470998)
\curveto(606.39795571,70.20470843)(606.29795581,70.21970841)(606.19794678,70.23970998)
\curveto(606.16795594,70.24970838)(606.12795598,70.25470838)(606.07794678,70.25470998)
\curveto(605.96795614,70.28470835)(605.86295625,70.30470833)(605.76294678,70.31470998)
\curveto(605.65295646,70.3347083)(605.54295657,70.35970827)(605.43294678,70.38970998)
\curveto(605.35295676,70.40970822)(605.28295683,70.42470821)(605.22294678,70.43470998)
\curveto(605.15295696,70.44470819)(605.09295702,70.46970816)(605.04294678,70.50970998)
\curveto(605.0129571,70.5297081)(604.99295712,70.55970807)(604.98294678,70.59970998)
\curveto(604.96295715,70.63970799)(604.94295717,70.68470795)(604.92294678,70.73470998)
\curveto(604.92295719,70.79470784)(604.91795719,70.8347078)(604.90794678,70.85470998)
}
}
{
\newrgbcolor{curcolor}{0 0 0}
\pscustom[linestyle=none,fillstyle=solid,fillcolor=curcolor]
{
\newpath
\moveto(613.68294678,78.63431936)
\lineto(613.68294678,79.26431936)
\lineto(613.68294678,79.45931936)
\curveto(613.68294843,79.52931683)(613.69294842,79.58931677)(613.71294678,79.63931936)
\curveto(613.75294836,79.70931665)(613.79294832,79.7593166)(613.83294678,79.78931936)
\curveto(613.88294823,79.82931653)(613.94794816,79.84931651)(614.02794678,79.84931936)
\curveto(614.107948,79.8593165)(614.19294792,79.86431649)(614.28294678,79.86431936)
\lineto(615.00294678,79.86431936)
\curveto(615.48294663,79.86431649)(615.89294622,79.80431655)(616.23294678,79.68431936)
\curveto(616.57294554,79.56431679)(616.84794526,79.36931699)(617.05794678,79.09931936)
\curveto(617.107945,79.02931733)(617.15294496,78.9593174)(617.19294678,78.88931936)
\curveto(617.24294487,78.82931753)(617.28794482,78.7543176)(617.32794678,78.66431936)
\curveto(617.33794477,78.64431771)(617.34794476,78.61431774)(617.35794678,78.57431936)
\curveto(617.37794473,78.53431782)(617.38294473,78.48931787)(617.37294678,78.43931936)
\curveto(617.34294477,78.34931801)(617.26794484,78.29431806)(617.14794678,78.27431936)
\curveto(617.03794507,78.2543181)(616.94294517,78.26931809)(616.86294678,78.31931936)
\curveto(616.79294532,78.34931801)(616.72794538,78.39431796)(616.66794678,78.45431936)
\curveto(616.61794549,78.52431783)(616.56794554,78.58931777)(616.51794678,78.64931936)
\curveto(616.46794564,78.71931764)(616.39294572,78.77931758)(616.29294678,78.82931936)
\curveto(616.20294591,78.88931747)(616.112946,78.93931742)(616.02294678,78.97931936)
\curveto(615.99294612,78.99931736)(615.93294618,79.02431733)(615.84294678,79.05431936)
\curveto(615.76294635,79.08431727)(615.69294642,79.08931727)(615.63294678,79.06931936)
\curveto(615.49294662,79.03931732)(615.40294671,78.97931738)(615.36294678,78.88931936)
\curveto(615.33294678,78.80931755)(615.31794679,78.71931764)(615.31794678,78.61931936)
\curveto(615.31794679,78.51931784)(615.29294682,78.43431792)(615.24294678,78.36431936)
\curveto(615.17294694,78.27431808)(615.03294708,78.22931813)(614.82294678,78.22931936)
\lineto(614.26794678,78.22931936)
\lineto(614.04294678,78.22931936)
\curveto(613.96294815,78.23931812)(613.89794821,78.2593181)(613.84794678,78.28931936)
\curveto(613.76794834,78.34931801)(613.72294839,78.41931794)(613.71294678,78.49931936)
\curveto(613.70294841,78.51931784)(613.69794841,78.53931782)(613.69794678,78.55931936)
\curveto(613.69794841,78.58931777)(613.69294842,78.61431774)(613.68294678,78.63431936)
}
}
{
\newrgbcolor{curcolor}{0 0 0}
\pscustom[linestyle=none,fillstyle=solid,fillcolor=curcolor]
{
}
}
{
\newrgbcolor{curcolor}{0 0 0}
\pscustom[linestyle=none,fillstyle=solid,fillcolor=curcolor]
{
\newpath
\moveto(604.71294678,89.26463186)
\curveto(604.70295741,89.95462722)(604.82295729,90.55462662)(605.07294678,91.06463186)
\curveto(605.32295679,91.58462559)(605.65795645,91.9796252)(606.07794678,92.24963186)
\curveto(606.15795595,92.29962488)(606.24795586,92.34462483)(606.34794678,92.38463186)
\curveto(606.43795567,92.42462475)(606.53295558,92.46962471)(606.63294678,92.51963186)
\curveto(606.73295538,92.55962462)(606.83295528,92.58962459)(606.93294678,92.60963186)
\curveto(607.03295508,92.62962455)(607.13795497,92.64962453)(607.24794678,92.66963186)
\curveto(607.29795481,92.68962449)(607.34295477,92.69462448)(607.38294678,92.68463186)
\curveto(607.42295469,92.6746245)(607.46795464,92.6796245)(607.51794678,92.69963186)
\curveto(607.56795454,92.70962447)(607.65295446,92.71462446)(607.77294678,92.71463186)
\curveto(607.88295423,92.71462446)(607.96795414,92.70962447)(608.02794678,92.69963186)
\curveto(608.08795402,92.6796245)(608.14795396,92.66962451)(608.20794678,92.66963186)
\curveto(608.26795384,92.6796245)(608.32795378,92.6746245)(608.38794678,92.65463186)
\curveto(608.52795358,92.61462456)(608.66295345,92.5796246)(608.79294678,92.54963186)
\curveto(608.92295319,92.51962466)(609.04795306,92.4796247)(609.16794678,92.42963186)
\curveto(609.3079528,92.36962481)(609.43295268,92.29962488)(609.54294678,92.21963186)
\curveto(609.65295246,92.14962503)(609.76295235,92.0746251)(609.87294678,91.99463186)
\lineto(609.93294678,91.93463186)
\curveto(609.95295216,91.92462525)(609.97295214,91.90962527)(609.99294678,91.88963186)
\curveto(610.15295196,91.76962541)(610.29795181,91.63462554)(610.42794678,91.48463186)
\curveto(610.55795155,91.33462584)(610.68295143,91.174626)(610.80294678,91.00463186)
\curveto(611.02295109,90.69462648)(611.22795088,90.39962678)(611.41794678,90.11963186)
\curveto(611.55795055,89.88962729)(611.69295042,89.65962752)(611.82294678,89.42963186)
\curveto(611.95295016,89.20962797)(612.08795002,88.98962819)(612.22794678,88.76963186)
\curveto(612.39794971,88.51962866)(612.57794953,88.2796289)(612.76794678,88.04963186)
\curveto(612.95794915,87.82962935)(613.18294893,87.63962954)(613.44294678,87.47963186)
\curveto(613.50294861,87.43962974)(613.56294855,87.40462977)(613.62294678,87.37463186)
\curveto(613.67294844,87.34462983)(613.73794837,87.31462986)(613.81794678,87.28463186)
\curveto(613.88794822,87.26462991)(613.94794816,87.25962992)(613.99794678,87.26963186)
\curveto(614.06794804,87.28962989)(614.12294799,87.32462985)(614.16294678,87.37463186)
\curveto(614.19294792,87.42462975)(614.2129479,87.48462969)(614.22294678,87.55463186)
\lineto(614.22294678,87.79463186)
\lineto(614.22294678,88.54463186)
\lineto(614.22294678,91.34963186)
\lineto(614.22294678,92.00963186)
\curveto(614.22294789,92.09962508)(614.22794788,92.18462499)(614.23794678,92.26463186)
\curveto(614.23794787,92.34462483)(614.25794785,92.40962477)(614.29794678,92.45963186)
\curveto(614.33794777,92.50962467)(614.4129477,92.54962463)(614.52294678,92.57963186)
\curveto(614.62294749,92.61962456)(614.72294739,92.62962455)(614.82294678,92.60963186)
\lineto(614.95794678,92.60963186)
\curveto(615.02794708,92.58962459)(615.08794702,92.56962461)(615.13794678,92.54963186)
\curveto(615.18794692,92.52962465)(615.22794688,92.49462468)(615.25794678,92.44463186)
\curveto(615.29794681,92.39462478)(615.31794679,92.32462485)(615.31794678,92.23463186)
\lineto(615.31794678,91.96463186)
\lineto(615.31794678,91.06463186)
\lineto(615.31794678,87.55463186)
\lineto(615.31794678,86.48963186)
\curveto(615.31794679,86.40963077)(615.32294679,86.31963086)(615.33294678,86.21963186)
\curveto(615.33294678,86.11963106)(615.32294679,86.03463114)(615.30294678,85.96463186)
\curveto(615.23294688,85.75463142)(615.05294706,85.68963149)(614.76294678,85.76963186)
\curveto(614.72294739,85.7796314)(614.68794742,85.7796314)(614.65794678,85.76963186)
\curveto(614.61794749,85.76963141)(614.57294754,85.7796314)(614.52294678,85.79963186)
\curveto(614.44294767,85.81963136)(614.35794775,85.83963134)(614.26794678,85.85963186)
\curveto(614.17794793,85.8796313)(614.09294802,85.90463127)(614.01294678,85.93463186)
\curveto(613.52294859,86.09463108)(613.107949,86.29463088)(612.76794678,86.53463186)
\curveto(612.51794959,86.71463046)(612.29294982,86.91963026)(612.09294678,87.14963186)
\curveto(611.88295023,87.3796298)(611.68795042,87.61962956)(611.50794678,87.86963186)
\curveto(611.32795078,88.12962905)(611.15795095,88.39462878)(610.99794678,88.66463186)
\curveto(610.82795128,88.94462823)(610.65295146,89.21462796)(610.47294678,89.47463186)
\curveto(610.39295172,89.58462759)(610.31795179,89.68962749)(610.24794678,89.78963186)
\curveto(610.17795193,89.89962728)(610.10295201,90.00962717)(610.02294678,90.11963186)
\curveto(609.99295212,90.15962702)(609.96295215,90.19462698)(609.93294678,90.22463186)
\curveto(609.89295222,90.26462691)(609.86295225,90.30462687)(609.84294678,90.34463186)
\curveto(609.73295238,90.48462669)(609.6079525,90.60962657)(609.46794678,90.71963186)
\curveto(609.43795267,90.73962644)(609.4129527,90.76462641)(609.39294678,90.79463186)
\curveto(609.36295275,90.82462635)(609.33295278,90.84962633)(609.30294678,90.86963186)
\curveto(609.20295291,90.94962623)(609.10295301,91.01462616)(609.00294678,91.06463186)
\curveto(608.90295321,91.12462605)(608.79295332,91.179626)(608.67294678,91.22963186)
\curveto(608.60295351,91.25962592)(608.52795358,91.2796259)(608.44794678,91.28963186)
\lineto(608.20794678,91.34963186)
\lineto(608.11794678,91.34963186)
\curveto(608.08795402,91.35962582)(608.05795405,91.36462581)(608.02794678,91.36463186)
\curveto(607.95795415,91.38462579)(607.86295425,91.38962579)(607.74294678,91.37963186)
\curveto(607.6129545,91.3796258)(607.5129546,91.36962581)(607.44294678,91.34963186)
\curveto(607.36295475,91.32962585)(607.28795482,91.30962587)(607.21794678,91.28963186)
\curveto(607.13795497,91.2796259)(607.05795505,91.25962592)(606.97794678,91.22963186)
\curveto(606.73795537,91.11962606)(606.53795557,90.96962621)(606.37794678,90.77963186)
\curveto(606.2079559,90.59962658)(606.06795604,90.3796268)(605.95794678,90.11963186)
\curveto(605.93795617,90.04962713)(605.92295619,89.9796272)(605.91294678,89.90963186)
\curveto(605.89295622,89.83962734)(605.87295624,89.76462741)(605.85294678,89.68463186)
\curveto(605.83295628,89.60462757)(605.82295629,89.49462768)(605.82294678,89.35463186)
\curveto(605.82295629,89.22462795)(605.83295628,89.11962806)(605.85294678,89.03963186)
\curveto(605.86295625,88.9796282)(605.86795624,88.92462825)(605.86794678,88.87463186)
\curveto(605.86795624,88.82462835)(605.87795623,88.7746284)(605.89794678,88.72463186)
\curveto(605.93795617,88.62462855)(605.97795613,88.52962865)(606.01794678,88.43963186)
\curveto(606.05795605,88.35962882)(606.10295601,88.2796289)(606.15294678,88.19963186)
\curveto(606.17295594,88.16962901)(606.19795591,88.13962904)(606.22794678,88.10963186)
\curveto(606.25795585,88.08962909)(606.28295583,88.06462911)(606.30294678,88.03463186)
\lineto(606.37794678,87.95963186)
\curveto(606.39795571,87.92962925)(606.41795569,87.90462927)(606.43794678,87.88463186)
\lineto(606.64794678,87.73463186)
\curveto(606.7079554,87.69462948)(606.77295534,87.64962953)(606.84294678,87.59963186)
\curveto(606.93295518,87.53962964)(607.03795507,87.48962969)(607.15794678,87.44963186)
\curveto(607.26795484,87.41962976)(607.37795473,87.38462979)(607.48794678,87.34463186)
\curveto(607.59795451,87.30462987)(607.74295437,87.2796299)(607.92294678,87.26963186)
\curveto(608.09295402,87.25962992)(608.21795389,87.22962995)(608.29794678,87.17963186)
\curveto(608.37795373,87.12963005)(608.42295369,87.05463012)(608.43294678,86.95463186)
\curveto(608.44295367,86.85463032)(608.44795366,86.74463043)(608.44794678,86.62463186)
\curveto(608.44795366,86.58463059)(608.45295366,86.54463063)(608.46294678,86.50463186)
\curveto(608.46295365,86.46463071)(608.45795365,86.42963075)(608.44794678,86.39963186)
\curveto(608.42795368,86.34963083)(608.41795369,86.29963088)(608.41794678,86.24963186)
\curveto(608.41795369,86.20963097)(608.4079537,86.16963101)(608.38794678,86.12963186)
\curveto(608.32795378,86.03963114)(608.19295392,85.99463118)(607.98294678,85.99463186)
\lineto(607.86294678,85.99463186)
\curveto(607.80295431,86.00463117)(607.74295437,86.00963117)(607.68294678,86.00963186)
\curveto(607.6129545,86.01963116)(607.54795456,86.02963115)(607.48794678,86.03963186)
\curveto(607.37795473,86.05963112)(607.27795483,86.0796311)(607.18794678,86.09963186)
\curveto(607.08795502,86.11963106)(606.99295512,86.14963103)(606.90294678,86.18963186)
\curveto(606.83295528,86.20963097)(606.77295534,86.22963095)(606.72294678,86.24963186)
\lineto(606.54294678,86.30963186)
\curveto(606.28295583,86.42963075)(606.03795607,86.58463059)(605.80794678,86.77463186)
\curveto(605.57795653,86.9746302)(605.39295672,87.18962999)(605.25294678,87.41963186)
\curveto(605.17295694,87.52962965)(605.107957,87.64462953)(605.05794678,87.76463186)
\lineto(604.90794678,88.15463186)
\curveto(604.85795725,88.26462891)(604.82795728,88.3796288)(604.81794678,88.49963186)
\curveto(604.79795731,88.61962856)(604.77295734,88.74462843)(604.74294678,88.87463186)
\curveto(604.74295737,88.94462823)(604.74295737,89.00962817)(604.74294678,89.06963186)
\curveto(604.73295738,89.12962805)(604.72295739,89.19462798)(604.71294678,89.26463186)
}
}
{
\newrgbcolor{curcolor}{0 0 0}
\pscustom[linestyle=none,fillstyle=solid,fillcolor=curcolor]
{
\newpath
\moveto(610.23294678,101.36424123)
\lineto(610.48794678,101.36424123)
\curveto(610.56795154,101.37423353)(610.64295147,101.36923353)(610.71294678,101.34924123)
\lineto(610.95294678,101.34924123)
\lineto(611.11794678,101.34924123)
\curveto(611.21795089,101.32923357)(611.32295079,101.31923358)(611.43294678,101.31924123)
\curveto(611.53295058,101.31923358)(611.63295048,101.30923359)(611.73294678,101.28924123)
\lineto(611.88294678,101.28924123)
\curveto(612.02295009,101.25923364)(612.16294995,101.23923366)(612.30294678,101.22924123)
\curveto(612.43294968,101.21923368)(612.56294955,101.19423371)(612.69294678,101.15424123)
\curveto(612.77294934,101.13423377)(612.85794925,101.11423379)(612.94794678,101.09424123)
\lineto(613.18794678,101.03424123)
\lineto(613.48794678,100.91424123)
\curveto(613.57794853,100.88423402)(613.66794844,100.84923405)(613.75794678,100.80924123)
\curveto(613.97794813,100.70923419)(614.19294792,100.57423433)(614.40294678,100.40424123)
\curveto(614.6129475,100.24423466)(614.78294733,100.06923483)(614.91294678,99.87924123)
\curveto(614.95294716,99.82923507)(614.99294712,99.76923513)(615.03294678,99.69924123)
\curveto(615.06294705,99.63923526)(615.09794701,99.57923532)(615.13794678,99.51924123)
\curveto(615.18794692,99.43923546)(615.22794688,99.34423556)(615.25794678,99.23424123)
\curveto(615.28794682,99.12423578)(615.31794679,99.01923588)(615.34794678,98.91924123)
\curveto(615.38794672,98.80923609)(615.4129467,98.6992362)(615.42294678,98.58924123)
\curveto(615.43294668,98.47923642)(615.44794666,98.36423654)(615.46794678,98.24424123)
\curveto(615.47794663,98.2042367)(615.47794663,98.15923674)(615.46794678,98.10924123)
\curveto(615.46794664,98.06923683)(615.47294664,98.02923687)(615.48294678,97.98924123)
\curveto(615.49294662,97.94923695)(615.49794661,97.89423701)(615.49794678,97.82424123)
\curveto(615.49794661,97.75423715)(615.49294662,97.7042372)(615.48294678,97.67424123)
\curveto(615.46294665,97.62423728)(615.45794665,97.57923732)(615.46794678,97.53924123)
\curveto(615.47794663,97.4992374)(615.47794663,97.46423744)(615.46794678,97.43424123)
\lineto(615.46794678,97.34424123)
\curveto(615.44794666,97.28423762)(615.43294668,97.21923768)(615.42294678,97.14924123)
\curveto(615.42294669,97.08923781)(615.41794669,97.02423788)(615.40794678,96.95424123)
\curveto(615.35794675,96.78423812)(615.3079468,96.62423828)(615.25794678,96.47424123)
\curveto(615.2079469,96.32423858)(615.14294697,96.17923872)(615.06294678,96.03924123)
\curveto(615.02294709,95.98923891)(614.99294712,95.93423897)(614.97294678,95.87424123)
\curveto(614.94294717,95.82423908)(614.9079472,95.77423913)(614.86794678,95.72424123)
\curveto(614.68794742,95.48423942)(614.46794764,95.28423962)(614.20794678,95.12424123)
\curveto(613.94794816,94.96423994)(613.66294845,94.82424008)(613.35294678,94.70424123)
\curveto(613.2129489,94.64424026)(613.07294904,94.5992403)(612.93294678,94.56924123)
\curveto(612.78294933,94.53924036)(612.62794948,94.5042404)(612.46794678,94.46424123)
\curveto(612.35794975,94.44424046)(612.24794986,94.42924047)(612.13794678,94.41924123)
\curveto(612.02795008,94.40924049)(611.91795019,94.39424051)(611.80794678,94.37424123)
\curveto(611.76795034,94.36424054)(611.72795038,94.35924054)(611.68794678,94.35924123)
\curveto(611.64795046,94.36924053)(611.6079505,94.36924053)(611.56794678,94.35924123)
\curveto(611.51795059,94.34924055)(611.46795064,94.34424056)(611.41794678,94.34424123)
\lineto(611.25294678,94.34424123)
\curveto(611.20295091,94.32424058)(611.15295096,94.31924058)(611.10294678,94.32924123)
\curveto(611.04295107,94.33924056)(610.98795112,94.33924056)(610.93794678,94.32924123)
\curveto(610.89795121,94.31924058)(610.85295126,94.31924058)(610.80294678,94.32924123)
\curveto(610.75295136,94.33924056)(610.70295141,94.33424057)(610.65294678,94.31424123)
\curveto(610.58295153,94.29424061)(610.5079516,94.28924061)(610.42794678,94.29924123)
\curveto(610.33795177,94.30924059)(610.25295186,94.31424059)(610.17294678,94.31424123)
\curveto(610.08295203,94.31424059)(609.98295213,94.30924059)(609.87294678,94.29924123)
\curveto(609.75295236,94.28924061)(609.65295246,94.29424061)(609.57294678,94.31424123)
\lineto(609.28794678,94.31424123)
\lineto(608.65794678,94.35924123)
\curveto(608.55795355,94.36924053)(608.46295365,94.37924052)(608.37294678,94.38924123)
\lineto(608.07294678,94.41924123)
\curveto(608.02295409,94.43924046)(607.97295414,94.44424046)(607.92294678,94.43424123)
\curveto(607.86295425,94.43424047)(607.8079543,94.44424046)(607.75794678,94.46424123)
\curveto(607.58795452,94.51424039)(607.42295469,94.55424035)(607.26294678,94.58424123)
\curveto(607.09295502,94.61424029)(606.93295518,94.66424024)(606.78294678,94.73424123)
\curveto(606.32295579,94.92423998)(605.94795616,95.14423976)(605.65794678,95.39424123)
\curveto(605.36795674,95.65423925)(605.12295699,96.01423889)(604.92294678,96.47424123)
\curveto(604.87295724,96.6042383)(604.83795727,96.73423817)(604.81794678,96.86424123)
\curveto(604.79795731,97.0042379)(604.77295734,97.14423776)(604.74294678,97.28424123)
\curveto(604.73295738,97.35423755)(604.72795738,97.41923748)(604.72794678,97.47924123)
\curveto(604.72795738,97.53923736)(604.72295739,97.6042373)(604.71294678,97.67424123)
\curveto(604.69295742,98.5042364)(604.84295727,99.17423573)(605.16294678,99.68424123)
\curveto(605.47295664,100.19423471)(605.9129562,100.57423433)(606.48294678,100.82424123)
\curveto(606.60295551,100.87423403)(606.72795538,100.91923398)(606.85794678,100.95924123)
\curveto(606.98795512,100.9992339)(607.12295499,101.04423386)(607.26294678,101.09424123)
\curveto(607.34295477,101.11423379)(607.42795468,101.12923377)(607.51794678,101.13924123)
\lineto(607.75794678,101.19924123)
\curveto(607.86795424,101.22923367)(607.97795413,101.24423366)(608.08794678,101.24424123)
\curveto(608.19795391,101.25423365)(608.3079538,101.26923363)(608.41794678,101.28924123)
\curveto(608.46795364,101.30923359)(608.5129536,101.31423359)(608.55294678,101.30424123)
\curveto(608.59295352,101.3042336)(608.63295348,101.30923359)(608.67294678,101.31924123)
\curveto(608.72295339,101.32923357)(608.77795333,101.32923357)(608.83794678,101.31924123)
\curveto(608.88795322,101.31923358)(608.93795317,101.32423358)(608.98794678,101.33424123)
\lineto(609.12294678,101.33424123)
\curveto(609.18295293,101.35423355)(609.25295286,101.35423355)(609.33294678,101.33424123)
\curveto(609.40295271,101.32423358)(609.46795264,101.32923357)(609.52794678,101.34924123)
\curveto(609.55795255,101.35923354)(609.59795251,101.36423354)(609.64794678,101.36424123)
\lineto(609.76794678,101.36424123)
\lineto(610.23294678,101.36424123)
\moveto(612.55794678,99.81924123)
\curveto(612.23794987,99.91923498)(611.87295024,99.97923492)(611.46294678,99.99924123)
\curveto(611.05295106,100.01923488)(610.64295147,100.02923487)(610.23294678,100.02924123)
\curveto(609.80295231,100.02923487)(609.38295273,100.01923488)(608.97294678,99.99924123)
\curveto(608.56295355,99.97923492)(608.17795393,99.93423497)(607.81794678,99.86424123)
\curveto(607.45795465,99.79423511)(607.13795497,99.68423522)(606.85794678,99.53424123)
\curveto(606.56795554,99.39423551)(606.33295578,99.1992357)(606.15294678,98.94924123)
\curveto(606.04295607,98.78923611)(605.96295615,98.60923629)(605.91294678,98.40924123)
\curveto(605.85295626,98.20923669)(605.82295629,97.96423694)(605.82294678,97.67424123)
\curveto(605.84295627,97.65423725)(605.85295626,97.61923728)(605.85294678,97.56924123)
\curveto(605.84295627,97.51923738)(605.84295627,97.47923742)(605.85294678,97.44924123)
\curveto(605.87295624,97.36923753)(605.89295622,97.29423761)(605.91294678,97.22424123)
\curveto(605.92295619,97.16423774)(605.94295617,97.0992378)(605.97294678,97.02924123)
\curveto(606.09295602,96.75923814)(606.26295585,96.53923836)(606.48294678,96.36924123)
\curveto(606.69295542,96.20923869)(606.93795517,96.07423883)(607.21794678,95.96424123)
\curveto(607.32795478,95.91423899)(607.44795466,95.87423903)(607.57794678,95.84424123)
\curveto(607.69795441,95.82423908)(607.82295429,95.7992391)(607.95294678,95.76924123)
\curveto(608.00295411,95.74923915)(608.05795405,95.73923916)(608.11794678,95.73924123)
\curveto(608.16795394,95.73923916)(608.21795389,95.73423917)(608.26794678,95.72424123)
\curveto(608.35795375,95.71423919)(608.45295366,95.7042392)(608.55294678,95.69424123)
\curveto(608.64295347,95.68423922)(608.73795337,95.67423923)(608.83794678,95.66424123)
\curveto(608.91795319,95.66423924)(609.00295311,95.65923924)(609.09294678,95.64924123)
\lineto(609.33294678,95.64924123)
\lineto(609.51294678,95.64924123)
\curveto(609.54295257,95.63923926)(609.57795253,95.63423927)(609.61794678,95.63424123)
\lineto(609.75294678,95.63424123)
\lineto(610.20294678,95.63424123)
\curveto(610.28295183,95.63423927)(610.36795174,95.62923927)(610.45794678,95.61924123)
\curveto(610.53795157,95.61923928)(610.6129515,95.62923927)(610.68294678,95.64924123)
\lineto(610.95294678,95.64924123)
\curveto(610.97295114,95.64923925)(611.00295111,95.64423926)(611.04294678,95.63424123)
\curveto(611.07295104,95.63423927)(611.09795101,95.63923926)(611.11794678,95.64924123)
\curveto(611.21795089,95.65923924)(611.31795079,95.66423924)(611.41794678,95.66424123)
\curveto(611.5079506,95.67423923)(611.6079505,95.68423922)(611.71794678,95.69424123)
\curveto(611.83795027,95.72423918)(611.96295015,95.73923916)(612.09294678,95.73924123)
\curveto(612.2129499,95.74923915)(612.32794978,95.77423913)(612.43794678,95.81424123)
\curveto(612.73794937,95.89423901)(613.00294911,95.97923892)(613.23294678,96.06924123)
\curveto(613.46294865,96.16923873)(613.67794843,96.31423859)(613.87794678,96.50424123)
\curveto(614.07794803,96.71423819)(614.22794788,96.97923792)(614.32794678,97.29924123)
\curveto(614.34794776,97.33923756)(614.35794775,97.37423753)(614.35794678,97.40424123)
\curveto(614.34794776,97.44423746)(614.35294776,97.48923741)(614.37294678,97.53924123)
\curveto(614.38294773,97.57923732)(614.39294772,97.64923725)(614.40294678,97.74924123)
\curveto(614.4129477,97.85923704)(614.4079477,97.94423696)(614.38794678,98.00424123)
\curveto(614.36794774,98.07423683)(614.35794775,98.14423676)(614.35794678,98.21424123)
\curveto(614.34794776,98.28423662)(614.33294778,98.34923655)(614.31294678,98.40924123)
\curveto(614.25294786,98.60923629)(614.16794794,98.78923611)(614.05794678,98.94924123)
\curveto(614.03794807,98.97923592)(614.01794809,99.0042359)(613.99794678,99.02424123)
\lineto(613.93794678,99.08424123)
\curveto(613.91794819,99.12423578)(613.87794823,99.17423573)(613.81794678,99.23424123)
\curveto(613.67794843,99.33423557)(613.54794856,99.41923548)(613.42794678,99.48924123)
\curveto(613.3079488,99.55923534)(613.16294895,99.62923527)(612.99294678,99.69924123)
\curveto(612.92294919,99.72923517)(612.85294926,99.74923515)(612.78294678,99.75924123)
\curveto(612.7129494,99.77923512)(612.63794947,99.7992351)(612.55794678,99.81924123)
}
}
{
\newrgbcolor{curcolor}{0 0 0}
\pscustom[linestyle=none,fillstyle=solid,fillcolor=curcolor]
{
\newpath
\moveto(604.71294678,106.77385061)
\curveto(604.7129574,106.87384575)(604.72295739,106.96884566)(604.74294678,107.05885061)
\curveto(604.75295736,107.14884548)(604.78295733,107.21384541)(604.83294678,107.25385061)
\curveto(604.9129572,107.31384531)(605.01795709,107.34384528)(605.14794678,107.34385061)
\lineto(605.53794678,107.34385061)
\lineto(607.03794678,107.34385061)
\lineto(613.42794678,107.34385061)
\lineto(614.59794678,107.34385061)
\lineto(614.91294678,107.34385061)
\curveto(615.0129471,107.35384527)(615.09294702,107.33884529)(615.15294678,107.29885061)
\curveto(615.23294688,107.24884538)(615.28294683,107.17384545)(615.30294678,107.07385061)
\curveto(615.3129468,106.98384564)(615.31794679,106.87384575)(615.31794678,106.74385061)
\lineto(615.31794678,106.51885061)
\curveto(615.29794681,106.43884619)(615.28294683,106.36884626)(615.27294678,106.30885061)
\curveto(615.25294686,106.24884638)(615.2129469,106.19884643)(615.15294678,106.15885061)
\curveto(615.09294702,106.11884651)(615.01794709,106.09884653)(614.92794678,106.09885061)
\lineto(614.62794678,106.09885061)
\lineto(613.53294678,106.09885061)
\lineto(608.19294678,106.09885061)
\curveto(608.10295401,106.07884655)(608.02795408,106.06384656)(607.96794678,106.05385061)
\curveto(607.89795421,106.05384657)(607.83795427,106.0238466)(607.78794678,105.96385061)
\curveto(607.73795437,105.89384673)(607.7129544,105.80384682)(607.71294678,105.69385061)
\curveto(607.70295441,105.59384703)(607.69795441,105.48384714)(607.69794678,105.36385061)
\lineto(607.69794678,104.22385061)
\lineto(607.69794678,103.72885061)
\curveto(607.68795442,103.56884906)(607.62795448,103.45884917)(607.51794678,103.39885061)
\curveto(607.48795462,103.37884925)(607.45795465,103.36884926)(607.42794678,103.36885061)
\curveto(607.38795472,103.36884926)(607.34295477,103.36384926)(607.29294678,103.35385061)
\curveto(607.17295494,103.33384929)(607.06295505,103.33884929)(606.96294678,103.36885061)
\curveto(606.86295525,103.40884922)(606.79295532,103.46384916)(606.75294678,103.53385061)
\curveto(606.70295541,103.61384901)(606.67795543,103.73384889)(606.67794678,103.89385061)
\curveto(606.67795543,104.05384857)(606.66295545,104.18884844)(606.63294678,104.29885061)
\curveto(606.62295549,104.34884828)(606.61795549,104.40384822)(606.61794678,104.46385061)
\curveto(606.6079555,104.5238481)(606.59295552,104.58384804)(606.57294678,104.64385061)
\curveto(606.52295559,104.79384783)(606.47295564,104.93884769)(606.42294678,105.07885061)
\curveto(606.36295575,105.21884741)(606.29295582,105.35384727)(606.21294678,105.48385061)
\curveto(606.12295599,105.623847)(606.01795609,105.74384688)(605.89794678,105.84385061)
\curveto(605.77795633,105.94384668)(605.64795646,106.03884659)(605.50794678,106.12885061)
\curveto(605.4079567,106.18884644)(605.29795681,106.23384639)(605.17794678,106.26385061)
\curveto(605.05795705,106.30384632)(604.95295716,106.35384627)(604.86294678,106.41385061)
\curveto(604.80295731,106.46384616)(604.76295735,106.53384609)(604.74294678,106.62385061)
\curveto(604.73295738,106.64384598)(604.72795738,106.66884596)(604.72794678,106.69885061)
\curveto(604.72795738,106.7288459)(604.72295739,106.75384587)(604.71294678,106.77385061)
}
}
{
\newrgbcolor{curcolor}{0 0 0}
\pscustom[linestyle=none,fillstyle=solid,fillcolor=curcolor]
{
\newpath
\moveto(604.71294678,115.12345998)
\curveto(604.7129574,115.22345513)(604.72295739,115.31845503)(604.74294678,115.40845998)
\curveto(604.75295736,115.49845485)(604.78295733,115.56345479)(604.83294678,115.60345998)
\curveto(604.9129572,115.66345469)(605.01795709,115.69345466)(605.14794678,115.69345998)
\lineto(605.53794678,115.69345998)
\lineto(607.03794678,115.69345998)
\lineto(613.42794678,115.69345998)
\lineto(614.59794678,115.69345998)
\lineto(614.91294678,115.69345998)
\curveto(615.0129471,115.70345465)(615.09294702,115.68845466)(615.15294678,115.64845998)
\curveto(615.23294688,115.59845475)(615.28294683,115.52345483)(615.30294678,115.42345998)
\curveto(615.3129468,115.33345502)(615.31794679,115.22345513)(615.31794678,115.09345998)
\lineto(615.31794678,114.86845998)
\curveto(615.29794681,114.78845556)(615.28294683,114.71845563)(615.27294678,114.65845998)
\curveto(615.25294686,114.59845575)(615.2129469,114.5484558)(615.15294678,114.50845998)
\curveto(615.09294702,114.46845588)(615.01794709,114.4484559)(614.92794678,114.44845998)
\lineto(614.62794678,114.44845998)
\lineto(613.53294678,114.44845998)
\lineto(608.19294678,114.44845998)
\curveto(608.10295401,114.42845592)(608.02795408,114.41345594)(607.96794678,114.40345998)
\curveto(607.89795421,114.40345595)(607.83795427,114.37345598)(607.78794678,114.31345998)
\curveto(607.73795437,114.24345611)(607.7129544,114.1534562)(607.71294678,114.04345998)
\curveto(607.70295441,113.94345641)(607.69795441,113.83345652)(607.69794678,113.71345998)
\lineto(607.69794678,112.57345998)
\lineto(607.69794678,112.07845998)
\curveto(607.68795442,111.91845843)(607.62795448,111.80845854)(607.51794678,111.74845998)
\curveto(607.48795462,111.72845862)(607.45795465,111.71845863)(607.42794678,111.71845998)
\curveto(607.38795472,111.71845863)(607.34295477,111.71345864)(607.29294678,111.70345998)
\curveto(607.17295494,111.68345867)(607.06295505,111.68845866)(606.96294678,111.71845998)
\curveto(606.86295525,111.75845859)(606.79295532,111.81345854)(606.75294678,111.88345998)
\curveto(606.70295541,111.96345839)(606.67795543,112.08345827)(606.67794678,112.24345998)
\curveto(606.67795543,112.40345795)(606.66295545,112.53845781)(606.63294678,112.64845998)
\curveto(606.62295549,112.69845765)(606.61795549,112.7534576)(606.61794678,112.81345998)
\curveto(606.6079555,112.87345748)(606.59295552,112.93345742)(606.57294678,112.99345998)
\curveto(606.52295559,113.14345721)(606.47295564,113.28845706)(606.42294678,113.42845998)
\curveto(606.36295575,113.56845678)(606.29295582,113.70345665)(606.21294678,113.83345998)
\curveto(606.12295599,113.97345638)(606.01795609,114.09345626)(605.89794678,114.19345998)
\curveto(605.77795633,114.29345606)(605.64795646,114.38845596)(605.50794678,114.47845998)
\curveto(605.4079567,114.53845581)(605.29795681,114.58345577)(605.17794678,114.61345998)
\curveto(605.05795705,114.6534557)(604.95295716,114.70345565)(604.86294678,114.76345998)
\curveto(604.80295731,114.81345554)(604.76295735,114.88345547)(604.74294678,114.97345998)
\curveto(604.73295738,114.99345536)(604.72795738,115.01845533)(604.72794678,115.04845998)
\curveto(604.72795738,115.07845527)(604.72295739,115.10345525)(604.71294678,115.12345998)
}
}
{
\newrgbcolor{curcolor}{0 0 0}
\pscustom[linestyle=none,fillstyle=solid,fillcolor=curcolor]
{
\newpath
\moveto(626.58423584,29.18119436)
\lineto(626.58423584,30.09619436)
\curveto(626.58424653,30.19619171)(626.58424653,30.29119161)(626.58423584,30.38119436)
\curveto(626.58424653,30.47119143)(626.60424651,30.54619136)(626.64423584,30.60619436)
\curveto(626.70424641,30.69619121)(626.78424633,30.75619115)(626.88423584,30.78619436)
\curveto(626.98424613,30.82619108)(627.08924603,30.87119103)(627.19923584,30.92119436)
\curveto(627.38924573,31.0011909)(627.57924554,31.07119083)(627.76923584,31.13119436)
\curveto(627.95924516,31.2011907)(628.14924497,31.27619063)(628.33923584,31.35619436)
\curveto(628.5192446,31.42619048)(628.70424441,31.49119041)(628.89423584,31.55119436)
\curveto(629.07424404,31.61119029)(629.25424386,31.68119022)(629.43423584,31.76119436)
\curveto(629.57424354,31.82119008)(629.7192434,31.87619003)(629.86923584,31.92619436)
\curveto(630.0192431,31.97618993)(630.16424295,32.03118987)(630.30423584,32.09119436)
\curveto(630.75424236,32.27118963)(631.20924191,32.44118946)(631.66923584,32.60119436)
\curveto(632.119241,32.76118914)(632.56924055,32.93118897)(633.01923584,33.11119436)
\curveto(633.06924005,33.13118877)(633.11924,33.14618876)(633.16923584,33.15619436)
\lineto(633.31923584,33.21619436)
\curveto(633.53923958,33.3061886)(633.76423935,33.39118851)(633.99423584,33.47119436)
\curveto(634.2142389,33.55118835)(634.43423868,33.63618827)(634.65423584,33.72619436)
\curveto(634.74423837,33.76618814)(634.85423826,33.8061881)(634.98423584,33.84619436)
\curveto(635.10423801,33.88618802)(635.17423794,33.95118795)(635.19423584,34.04119436)
\curveto(635.20423791,34.08118782)(635.20423791,34.11118779)(635.19423584,34.13119436)
\lineto(635.13423584,34.19119436)
\curveto(635.08423803,34.24118766)(635.02923809,34.27618763)(634.96923584,34.29619436)
\curveto(634.90923821,34.32618758)(634.84423827,34.35618755)(634.77423584,34.38619436)
\lineto(634.14423584,34.62619436)
\curveto(633.92423919,34.7061872)(633.70923941,34.78618712)(633.49923584,34.86619436)
\lineto(633.34923584,34.92619436)
\lineto(633.16923584,34.98619436)
\curveto(632.97924014,35.06618684)(632.78924033,35.13618677)(632.59923584,35.19619436)
\curveto(632.39924072,35.26618664)(632.19924092,35.34118656)(631.99923584,35.42119436)
\curveto(631.4192417,35.66118624)(630.83424228,35.88118602)(630.24423584,36.08119436)
\curveto(629.65424346,36.29118561)(629.06924405,36.51618539)(628.48923584,36.75619436)
\curveto(628.28924483,36.83618507)(628.08424503,36.91118499)(627.87423584,36.98119436)
\curveto(627.66424545,37.06118484)(627.45924566,37.14118476)(627.25923584,37.22119436)
\curveto(627.17924594,37.26118464)(627.07924604,37.29618461)(626.95923584,37.32619436)
\curveto(626.83924628,37.36618454)(626.75424636,37.42118448)(626.70423584,37.49119436)
\curveto(626.66424645,37.55118435)(626.63424648,37.62618428)(626.61423584,37.71619436)
\curveto(626.59424652,37.81618409)(626.58424653,37.92618398)(626.58423584,38.04619436)
\curveto(626.57424654,38.16618374)(626.57424654,38.28618362)(626.58423584,38.40619436)
\curveto(626.58424653,38.52618338)(626.58424653,38.63618327)(626.58423584,38.73619436)
\curveto(626.58424653,38.82618308)(626.58424653,38.91618299)(626.58423584,39.00619436)
\curveto(626.58424653,39.1061828)(626.60424651,39.18118272)(626.64423584,39.23119436)
\curveto(626.69424642,39.32118258)(626.78424633,39.37118253)(626.91423584,39.38119436)
\curveto(627.04424607,39.39118251)(627.18424593,39.39618251)(627.33423584,39.39619436)
\lineto(628.98423584,39.39619436)
\lineto(635.25423584,39.39619436)
\lineto(636.51423584,39.39619436)
\curveto(636.62423649,39.39618251)(636.73423638,39.39618251)(636.84423584,39.39619436)
\curveto(636.95423616,39.4061825)(637.03923608,39.38618252)(637.09923584,39.33619436)
\curveto(637.15923596,39.3061826)(637.19923592,39.26118264)(637.21923584,39.20119436)
\curveto(637.22923589,39.14118276)(637.24423587,39.07118283)(637.26423584,38.99119436)
\lineto(637.26423584,38.75119436)
\lineto(637.26423584,38.39119436)
\curveto(637.25423586,38.28118362)(637.20923591,38.2011837)(637.12923584,38.15119436)
\curveto(637.09923602,38.13118377)(637.06923605,38.11618379)(637.03923584,38.10619436)
\curveto(636.99923612,38.1061838)(636.95423616,38.09618381)(636.90423584,38.07619436)
\lineto(636.73923584,38.07619436)
\curveto(636.67923644,38.06618384)(636.60923651,38.06118384)(636.52923584,38.06119436)
\curveto(636.44923667,38.07118383)(636.37423674,38.07618383)(636.30423584,38.07619436)
\lineto(635.46423584,38.07619436)
\lineto(631.03923584,38.07619436)
\curveto(630.78924233,38.07618383)(630.53924258,38.07618383)(630.28923584,38.07619436)
\curveto(630.02924309,38.07618383)(629.77924334,38.07118383)(629.53923584,38.06119436)
\curveto(629.43924368,38.06118384)(629.32924379,38.05618385)(629.20923584,38.04619436)
\curveto(629.08924403,38.03618387)(629.02924409,37.98118392)(629.02923584,37.88119436)
\lineto(629.04423584,37.88119436)
\curveto(629.06424405,37.81118409)(629.12924399,37.75118415)(629.23923584,37.70119436)
\curveto(629.34924377,37.66118424)(629.44424367,37.62618428)(629.52423584,37.59619436)
\curveto(629.69424342,37.52618438)(629.86924325,37.46118444)(630.04923584,37.40119436)
\curveto(630.2192429,37.34118456)(630.38924273,37.27118463)(630.55923584,37.19119436)
\curveto(630.60924251,37.17118473)(630.65424246,37.15618475)(630.69423584,37.14619436)
\curveto(630.73424238,37.13618477)(630.77924234,37.12118478)(630.82923584,37.10119436)
\curveto(631.00924211,37.02118488)(631.19424192,36.95118495)(631.38423584,36.89119436)
\curveto(631.56424155,36.84118506)(631.74424137,36.77618513)(631.92423584,36.69619436)
\curveto(632.07424104,36.62618528)(632.22924089,36.56618534)(632.38923584,36.51619436)
\curveto(632.53924058,36.46618544)(632.68924043,36.41118549)(632.83923584,36.35119436)
\curveto(633.30923981,36.15118575)(633.78423933,35.97118593)(634.26423584,35.81119436)
\curveto(634.73423838,35.65118625)(635.19923792,35.47618643)(635.65923584,35.28619436)
\curveto(635.83923728,35.2061867)(636.0192371,35.13618677)(636.19923584,35.07619436)
\curveto(636.37923674,35.01618689)(636.55923656,34.95118695)(636.73923584,34.88119436)
\curveto(636.84923627,34.83118707)(636.95423616,34.78118712)(637.05423584,34.73119436)
\curveto(637.14423597,34.69118721)(637.20923591,34.6061873)(637.24923584,34.47619436)
\curveto(637.25923586,34.45618745)(637.26423585,34.43118747)(637.26423584,34.40119436)
\curveto(637.25423586,34.38118752)(637.25423586,34.35618755)(637.26423584,34.32619436)
\curveto(637.27423584,34.29618761)(637.27923584,34.26118764)(637.27923584,34.22119436)
\curveto(637.26923585,34.18118772)(637.26423585,34.14118776)(637.26423584,34.10119436)
\lineto(637.26423584,33.80119436)
\curveto(637.26423585,33.7011882)(637.23923588,33.62118828)(637.18923584,33.56119436)
\curveto(637.13923598,33.48118842)(637.06923605,33.42118848)(636.97923584,33.38119436)
\curveto(636.87923624,33.35118855)(636.77923634,33.31118859)(636.67923584,33.26119436)
\curveto(636.47923664,33.18118872)(636.27423684,33.1011888)(636.06423584,33.02119436)
\curveto(635.84423727,32.95118895)(635.63423748,32.87618903)(635.43423584,32.79619436)
\curveto(635.25423786,32.71618919)(635.07423804,32.64618926)(634.89423584,32.58619436)
\curveto(634.70423841,32.53618937)(634.5192386,32.47118943)(634.33923584,32.39119436)
\curveto(633.77923934,32.16118974)(633.2142399,31.94618996)(632.64423584,31.74619436)
\curveto(632.07424104,31.54619036)(631.50924161,31.33119057)(630.94923584,31.10119436)
\lineto(630.31923584,30.86119436)
\curveto(630.09924302,30.79119111)(629.88924323,30.71619119)(629.68923584,30.63619436)
\curveto(629.57924354,30.58619132)(629.47424364,30.54119136)(629.37423584,30.50119436)
\curveto(629.26424385,30.47119143)(629.16924395,30.42119148)(629.08923584,30.35119436)
\curveto(629.06924405,30.34119156)(629.05924406,30.33119157)(629.05923584,30.32119436)
\lineto(629.02923584,30.29119436)
\lineto(629.02923584,30.21619436)
\lineto(629.05923584,30.18619436)
\curveto(629.05924406,30.17619173)(629.06424405,30.16619174)(629.07423584,30.15619436)
\curveto(629.12424399,30.13619177)(629.17924394,30.12619178)(629.23923584,30.12619436)
\curveto(629.29924382,30.12619178)(629.35924376,30.11619179)(629.41923584,30.09619436)
\lineto(629.58423584,30.09619436)
\curveto(629.64424347,30.07619183)(629.70924341,30.07119183)(629.77923584,30.08119436)
\curveto(629.84924327,30.09119181)(629.9192432,30.09619181)(629.98923584,30.09619436)
\lineto(630.79923584,30.09619436)
\lineto(635.35923584,30.09619436)
\lineto(636.54423584,30.09619436)
\curveto(636.65423646,30.09619181)(636.76423635,30.09119181)(636.87423584,30.08119436)
\curveto(636.98423613,30.08119182)(637.06923605,30.05619185)(637.12923584,30.00619436)
\curveto(637.20923591,29.95619195)(637.25423586,29.86619204)(637.26423584,29.73619436)
\lineto(637.26423584,29.34619436)
\lineto(637.26423584,29.15119436)
\curveto(637.26423585,29.1011928)(637.25423586,29.05119285)(637.23423584,29.00119436)
\curveto(637.19423592,28.87119303)(637.10923601,28.79619311)(636.97923584,28.77619436)
\curveto(636.84923627,28.76619314)(636.69923642,28.76119314)(636.52923584,28.76119436)
\lineto(634.78923584,28.76119436)
\lineto(628.78923584,28.76119436)
\lineto(627.37923584,28.76119436)
\curveto(627.26924585,28.76119314)(627.15424596,28.75619315)(627.03423584,28.74619436)
\curveto(626.9142462,28.74619316)(626.8192463,28.77119313)(626.74923584,28.82119436)
\curveto(626.68924643,28.86119304)(626.63924648,28.93619297)(626.59923584,29.04619436)
\curveto(626.58924653,29.06619284)(626.58924653,29.08619282)(626.59923584,29.10619436)
\curveto(626.59924652,29.13619277)(626.59424652,29.16119274)(626.58423584,29.18119436)
}
}
{
\newrgbcolor{curcolor}{0 0 0}
\pscustom[linestyle=none,fillstyle=solid,fillcolor=curcolor]
{
\newpath
\moveto(636.70923584,48.38330373)
\curveto(636.86923625,48.4132959)(637.00423611,48.39829592)(637.11423584,48.33830373)
\curveto(637.2142359,48.27829604)(637.28923583,48.19829612)(637.33923584,48.09830373)
\curveto(637.35923576,48.04829627)(637.36923575,47.99329632)(637.36923584,47.93330373)
\curveto(637.36923575,47.88329643)(637.37923574,47.82829649)(637.39923584,47.76830373)
\curveto(637.44923567,47.54829677)(637.43423568,47.32829699)(637.35423584,47.10830373)
\curveto(637.28423583,46.89829742)(637.19423592,46.75329756)(637.08423584,46.67330373)
\curveto(637.0142361,46.62329769)(636.93423618,46.57829774)(636.84423584,46.53830373)
\curveto(636.74423637,46.49829782)(636.66423645,46.44829787)(636.60423584,46.38830373)
\curveto(636.58423653,46.36829795)(636.56423655,46.34329797)(636.54423584,46.31330373)
\curveto(636.52423659,46.29329802)(636.5192366,46.26329805)(636.52923584,46.22330373)
\curveto(636.55923656,46.1132982)(636.6142365,46.00829831)(636.69423584,45.90830373)
\curveto(636.77423634,45.8182985)(636.84423627,45.72829859)(636.90423584,45.63830373)
\curveto(636.98423613,45.50829881)(637.05923606,45.36829895)(637.12923584,45.21830373)
\curveto(637.18923593,45.06829925)(637.24423587,44.90829941)(637.29423584,44.73830373)
\curveto(637.32423579,44.63829968)(637.34423577,44.52829979)(637.35423584,44.40830373)
\curveto(637.36423575,44.29830002)(637.37923574,44.18830013)(637.39923584,44.07830373)
\curveto(637.40923571,44.02830029)(637.4142357,43.98330033)(637.41423584,43.94330373)
\lineto(637.41423584,43.83830373)
\curveto(637.43423568,43.72830059)(637.43423568,43.62330069)(637.41423584,43.52330373)
\lineto(637.41423584,43.38830373)
\curveto(637.40423571,43.33830098)(637.39923572,43.28830103)(637.39923584,43.23830373)
\curveto(637.39923572,43.18830113)(637.38923573,43.14330117)(637.36923584,43.10330373)
\curveto(637.35923576,43.06330125)(637.35423576,43.02830129)(637.35423584,42.99830373)
\curveto(637.36423575,42.97830134)(637.36423575,42.95330136)(637.35423584,42.92330373)
\lineto(637.29423584,42.68330373)
\curveto(637.28423583,42.60330171)(637.26423585,42.52830179)(637.23423584,42.45830373)
\curveto(637.10423601,42.15830216)(636.95923616,41.9133024)(636.79923584,41.72330373)
\curveto(636.62923649,41.54330277)(636.39423672,41.39330292)(636.09423584,41.27330373)
\curveto(635.87423724,41.18330313)(635.60923751,41.13830318)(635.29923584,41.13830373)
\lineto(634.98423584,41.13830373)
\curveto(634.93423818,41.14830317)(634.88423823,41.15330316)(634.83423584,41.15330373)
\lineto(634.65423584,41.18330373)
\lineto(634.32423584,41.30330373)
\curveto(634.2142389,41.34330297)(634.114239,41.39330292)(634.02423584,41.45330373)
\curveto(633.73423938,41.63330268)(633.5192396,41.87830244)(633.37923584,42.18830373)
\curveto(633.23923988,42.49830182)(633.11424,42.83830148)(633.00423584,43.20830373)
\curveto(632.96424015,43.34830097)(632.93424018,43.49330082)(632.91423584,43.64330373)
\curveto(632.89424022,43.79330052)(632.86924025,43.94330037)(632.83923584,44.09330373)
\curveto(632.8192403,44.16330015)(632.80924031,44.22830009)(632.80923584,44.28830373)
\curveto(632.80924031,44.35829996)(632.79924032,44.43329988)(632.77923584,44.51330373)
\curveto(632.75924036,44.58329973)(632.74924037,44.65329966)(632.74923584,44.72330373)
\curveto(632.73924038,44.79329952)(632.72424039,44.86829945)(632.70423584,44.94830373)
\curveto(632.64424047,45.19829912)(632.59424052,45.43329888)(632.55423584,45.65330373)
\curveto(632.50424061,45.87329844)(632.38924073,46.04829827)(632.20923584,46.17830373)
\curveto(632.12924099,46.23829808)(632.02924109,46.28829803)(631.90923584,46.32830373)
\curveto(631.77924134,46.36829795)(631.63924148,46.36829795)(631.48923584,46.32830373)
\curveto(631.24924187,46.26829805)(631.05924206,46.17829814)(630.91923584,46.05830373)
\curveto(630.77924234,45.94829837)(630.66924245,45.78829853)(630.58923584,45.57830373)
\curveto(630.53924258,45.45829886)(630.50424261,45.313299)(630.48423584,45.14330373)
\curveto(630.46424265,44.98329933)(630.45424266,44.8132995)(630.45423584,44.63330373)
\curveto(630.45424266,44.45329986)(630.46424265,44.27830004)(630.48423584,44.10830373)
\curveto(630.50424261,43.93830038)(630.53424258,43.79330052)(630.57423584,43.67330373)
\curveto(630.63424248,43.50330081)(630.7192424,43.33830098)(630.82923584,43.17830373)
\curveto(630.88924223,43.09830122)(630.96924215,43.02330129)(631.06923584,42.95330373)
\curveto(631.15924196,42.89330142)(631.25924186,42.83830148)(631.36923584,42.78830373)
\curveto(631.44924167,42.75830156)(631.53424158,42.72830159)(631.62423584,42.69830373)
\curveto(631.7142414,42.67830164)(631.78424133,42.63330168)(631.83423584,42.56330373)
\curveto(631.86424125,42.52330179)(631.88924123,42.45330186)(631.90923584,42.35330373)
\curveto(631.9192412,42.26330205)(631.92424119,42.16830215)(631.92423584,42.06830373)
\curveto(631.92424119,41.96830235)(631.9192412,41.86830245)(631.90923584,41.76830373)
\curveto(631.88924123,41.67830264)(631.86424125,41.6133027)(631.83423584,41.57330373)
\curveto(631.80424131,41.53330278)(631.75424136,41.50330281)(631.68423584,41.48330373)
\curveto(631.6142415,41.46330285)(631.53924158,41.46330285)(631.45923584,41.48330373)
\curveto(631.32924179,41.5133028)(631.20924191,41.54330277)(631.09923584,41.57330373)
\curveto(630.97924214,41.6133027)(630.86424225,41.65830266)(630.75423584,41.70830373)
\curveto(630.40424271,41.89830242)(630.13424298,42.13830218)(629.94423584,42.42830373)
\curveto(629.74424337,42.7183016)(629.58424353,43.07830124)(629.46423584,43.50830373)
\curveto(629.44424367,43.60830071)(629.42924369,43.70830061)(629.41923584,43.80830373)
\curveto(629.40924371,43.9183004)(629.39424372,44.02830029)(629.37423584,44.13830373)
\curveto(629.36424375,44.17830014)(629.36424375,44.24330007)(629.37423584,44.33330373)
\curveto(629.37424374,44.42329989)(629.36424375,44.47829984)(629.34423584,44.49830373)
\curveto(629.33424378,45.19829912)(629.4142437,45.80829851)(629.58423584,46.32830373)
\curveto(629.75424336,46.84829747)(630.07924304,47.2132971)(630.55923584,47.42330373)
\curveto(630.75924236,47.5132968)(630.99424212,47.56329675)(631.26423584,47.57330373)
\curveto(631.52424159,47.59329672)(631.79924132,47.60329671)(632.08923584,47.60330373)
\lineto(635.40423584,47.60330373)
\curveto(635.54423757,47.60329671)(635.67923744,47.60829671)(635.80923584,47.61830373)
\curveto(635.93923718,47.62829669)(636.04423707,47.65829666)(636.12423584,47.70830373)
\curveto(636.19423692,47.75829656)(636.24423687,47.82329649)(636.27423584,47.90330373)
\curveto(636.3142368,47.99329632)(636.34423677,48.07829624)(636.36423584,48.15830373)
\curveto(636.37423674,48.23829608)(636.4192367,48.29829602)(636.49923584,48.33830373)
\curveto(636.52923659,48.35829596)(636.55923656,48.36829595)(636.58923584,48.36830373)
\curveto(636.6192365,48.36829595)(636.65923646,48.37329594)(636.70923584,48.38330373)
\moveto(635.04423584,46.23830373)
\curveto(634.90423821,46.29829802)(634.74423837,46.32829799)(634.56423584,46.32830373)
\curveto(634.37423874,46.33829798)(634.17923894,46.34329797)(633.97923584,46.34330373)
\curveto(633.86923925,46.34329797)(633.76923935,46.33829798)(633.67923584,46.32830373)
\curveto(633.58923953,46.318298)(633.5192396,46.27829804)(633.46923584,46.20830373)
\curveto(633.44923967,46.17829814)(633.43923968,46.10829821)(633.43923584,45.99830373)
\curveto(633.45923966,45.97829834)(633.46923965,45.94329837)(633.46923584,45.89330373)
\curveto(633.46923965,45.84329847)(633.47923964,45.79829852)(633.49923584,45.75830373)
\curveto(633.5192396,45.67829864)(633.53923958,45.58829873)(633.55923584,45.48830373)
\lineto(633.61923584,45.18830373)
\curveto(633.6192395,45.15829916)(633.62423949,45.12329919)(633.63423584,45.08330373)
\lineto(633.63423584,44.97830373)
\curveto(633.67423944,44.82829949)(633.69923942,44.66329965)(633.70923584,44.48330373)
\curveto(633.70923941,44.3133)(633.72923939,44.15330016)(633.76923584,44.00330373)
\curveto(633.78923933,43.92330039)(633.80923931,43.84830047)(633.82923584,43.77830373)
\curveto(633.83923928,43.7183006)(633.85423926,43.64830067)(633.87423584,43.56830373)
\curveto(633.92423919,43.40830091)(633.98923913,43.25830106)(634.06923584,43.11830373)
\curveto(634.13923898,42.97830134)(634.22923889,42.85830146)(634.33923584,42.75830373)
\curveto(634.44923867,42.65830166)(634.58423853,42.58330173)(634.74423584,42.53330373)
\curveto(634.89423822,42.48330183)(635.07923804,42.46330185)(635.29923584,42.47330373)
\curveto(635.39923772,42.47330184)(635.49423762,42.48830183)(635.58423584,42.51830373)
\curveto(635.66423745,42.55830176)(635.73923738,42.60330171)(635.80923584,42.65330373)
\curveto(635.9192372,42.73330158)(636.0142371,42.83830148)(636.09423584,42.96830373)
\curveto(636.16423695,43.09830122)(636.22423689,43.23830108)(636.27423584,43.38830373)
\curveto(636.28423683,43.43830088)(636.28923683,43.48830083)(636.28923584,43.53830373)
\curveto(636.28923683,43.58830073)(636.29423682,43.63830068)(636.30423584,43.68830373)
\curveto(636.32423679,43.75830056)(636.33923678,43.84330047)(636.34923584,43.94330373)
\curveto(636.34923677,44.05330026)(636.33923678,44.14330017)(636.31923584,44.21330373)
\curveto(636.29923682,44.27330004)(636.29423682,44.33329998)(636.30423584,44.39330373)
\curveto(636.30423681,44.45329986)(636.29423682,44.5132998)(636.27423584,44.57330373)
\curveto(636.25423686,44.65329966)(636.23923688,44.72829959)(636.22923584,44.79830373)
\curveto(636.2192369,44.87829944)(636.19923692,44.95329936)(636.16923584,45.02330373)
\curveto(636.04923707,45.313299)(635.90423721,45.55829876)(635.73423584,45.75830373)
\curveto(635.56423755,45.96829835)(635.33423778,46.12829819)(635.04423584,46.23830373)
}
}
{
\newrgbcolor{curcolor}{0 0 0}
\pscustom[linestyle=none,fillstyle=solid,fillcolor=curcolor]
{
\newpath
\moveto(629.53923584,49.26994436)
\lineto(629.53923584,49.71994436)
\curveto(629.52924359,49.88994311)(629.54924357,50.01494298)(629.59923584,50.09494436)
\curveto(629.64924347,50.17494282)(629.7142434,50.22994277)(629.79423584,50.25994436)
\curveto(629.87424324,50.2999427)(629.95924316,50.33994266)(630.04923584,50.37994436)
\curveto(630.17924294,50.42994257)(630.30924281,50.47494252)(630.43923584,50.51494436)
\curveto(630.56924255,50.55494244)(630.69924242,50.5999424)(630.82923584,50.64994436)
\curveto(630.94924217,50.6999423)(631.07424204,50.74494225)(631.20423584,50.78494436)
\curveto(631.32424179,50.82494217)(631.44424167,50.86994213)(631.56423584,50.91994436)
\curveto(631.67424144,50.96994203)(631.78924133,51.00994199)(631.90923584,51.03994436)
\curveto(632.02924109,51.06994193)(632.14924097,51.10994189)(632.26923584,51.15994436)
\curveto(632.55924056,51.27994172)(632.85924026,51.38994161)(633.16923584,51.48994436)
\curveto(633.47923964,51.58994141)(633.77923934,51.6999413)(634.06923584,51.81994436)
\curveto(634.10923901,51.83994116)(634.14923897,51.84994115)(634.18923584,51.84994436)
\curveto(634.2192389,51.84994115)(634.24923887,51.85994114)(634.27923584,51.87994436)
\curveto(634.4192387,51.93994106)(634.56423855,51.994941)(634.71423584,52.04494436)
\lineto(635.13423584,52.19494436)
\curveto(635.20423791,52.22494077)(635.27923784,52.25494074)(635.35923584,52.28494436)
\curveto(635.42923769,52.31494068)(635.47423764,52.36494063)(635.49423584,52.43494436)
\curveto(635.52423759,52.51494048)(635.49923762,52.57494042)(635.41923584,52.61494436)
\curveto(635.32923779,52.66494033)(635.25923786,52.6999403)(635.20923584,52.71994436)
\curveto(635.03923808,52.7999402)(634.85923826,52.86494013)(634.66923584,52.91494436)
\curveto(634.47923864,52.96494003)(634.29423882,53.02493997)(634.11423584,53.09494436)
\curveto(633.88423923,53.18493981)(633.65423946,53.26493973)(633.42423584,53.33494436)
\curveto(633.18423993,53.40493959)(632.95424016,53.48993951)(632.73423584,53.58994436)
\curveto(632.68424043,53.5999394)(632.6192405,53.61493938)(632.53923584,53.63494436)
\curveto(632.44924067,53.67493932)(632.35924076,53.70993929)(632.26923584,53.73994436)
\curveto(632.16924095,53.76993923)(632.07924104,53.7999392)(631.99923584,53.82994436)
\curveto(631.94924117,53.84993915)(631.90424121,53.86493913)(631.86423584,53.87494436)
\curveto(631.82424129,53.88493911)(631.77924134,53.8999391)(631.72923584,53.91994436)
\curveto(631.60924151,53.96993903)(631.48924163,54.00993899)(631.36923584,54.03994436)
\curveto(631.23924188,54.07993892)(631.114242,54.12493887)(630.99423584,54.17494436)
\curveto(630.94424217,54.1949388)(630.89924222,54.20993879)(630.85923584,54.21994436)
\curveto(630.8192423,54.22993877)(630.77424234,54.24493875)(630.72423584,54.26494436)
\curveto(630.63424248,54.30493869)(630.54424257,54.33993866)(630.45423584,54.36994436)
\curveto(630.35424276,54.3999386)(630.25924286,54.42993857)(630.16923584,54.45994436)
\curveto(630.08924303,54.48993851)(630.00924311,54.51493848)(629.92923584,54.53494436)
\curveto(629.83924328,54.56493843)(629.76424335,54.60493839)(629.70423584,54.65494436)
\curveto(629.6142435,54.72493827)(629.56424355,54.81993818)(629.55423584,54.93994436)
\curveto(629.54424357,55.06993793)(629.53924358,55.20993779)(629.53923584,55.35994436)
\curveto(629.53924358,55.43993756)(629.54424357,55.51493748)(629.55423584,55.58494436)
\curveto(629.55424356,55.66493733)(629.56924355,55.72993727)(629.59923584,55.77994436)
\curveto(629.65924346,55.86993713)(629.75424336,55.8949371)(629.88423584,55.85494436)
\curveto(630.0142431,55.81493718)(630.114243,55.77993722)(630.18423584,55.74994436)
\lineto(630.24423584,55.71994436)
\curveto(630.26424285,55.71993728)(630.28424283,55.71493728)(630.30423584,55.70494436)
\curveto(630.58424253,55.5949374)(630.86924225,55.48493751)(631.15923584,55.37494436)
\lineto(631.99923584,55.04494436)
\curveto(632.07924104,55.01493798)(632.15424096,54.98993801)(632.22423584,54.96994436)
\curveto(632.28424083,54.94993805)(632.34924077,54.92493807)(632.41923584,54.89494436)
\curveto(632.6192405,54.81493818)(632.82424029,54.73493826)(633.03423584,54.65494436)
\curveto(633.23423988,54.58493841)(633.43423968,54.50993849)(633.63423584,54.42994436)
\curveto(634.32423879,54.13993886)(635.0192381,53.86993913)(635.71923584,53.61994436)
\curveto(636.4192367,53.36993963)(637.114236,53.0999399)(637.80423584,52.80994436)
\lineto(637.95423584,52.74994436)
\curveto(638.0142351,52.73994026)(638.07423504,52.72494027)(638.13423584,52.70494436)
\curveto(638.50423461,52.54494045)(638.86923425,52.37494062)(639.22923584,52.19494436)
\curveto(639.59923352,52.01494098)(639.88423323,51.76494123)(640.08423584,51.44494436)
\curveto(640.14423297,51.33494166)(640.18923293,51.22494177)(640.21923584,51.11494436)
\curveto(640.25923286,51.00494199)(640.29423282,50.87994212)(640.32423584,50.73994436)
\curveto(640.34423277,50.68994231)(640.34923277,50.63494236)(640.33923584,50.57494436)
\curveto(640.32923279,50.52494247)(640.32923279,50.46994253)(640.33923584,50.40994436)
\curveto(640.35923276,50.32994267)(640.35923276,50.24994275)(640.33923584,50.16994436)
\curveto(640.32923279,50.12994287)(640.32423279,50.07994292)(640.32423584,50.01994436)
\lineto(640.26423584,49.77994436)
\curveto(640.24423287,49.70994329)(640.20423291,49.65494334)(640.14423584,49.61494436)
\curveto(640.08423303,49.56494343)(640.00923311,49.53494346)(639.91923584,49.52494436)
\lineto(639.64923584,49.52494436)
\lineto(639.43923584,49.52494436)
\curveto(639.37923374,49.53494346)(639.32923379,49.55494344)(639.28923584,49.58494436)
\curveto(639.17923394,49.65494334)(639.14923397,49.77494322)(639.19923584,49.94494436)
\curveto(639.2192339,50.05494294)(639.22923389,50.17494282)(639.22923584,50.30494436)
\curveto(639.22923389,50.43494256)(639.20923391,50.54994245)(639.16923584,50.64994436)
\curveto(639.119234,50.7999422)(639.04423407,50.91994208)(638.94423584,51.00994436)
\curveto(638.84423427,51.10994189)(638.72923439,51.1949418)(638.59923584,51.26494436)
\curveto(638.47923464,51.33494166)(638.34923477,51.3949416)(638.20923584,51.44494436)
\lineto(637.78923584,51.62494436)
\curveto(637.69923542,51.66494133)(637.58923553,51.70494129)(637.45923584,51.74494436)
\curveto(637.32923579,51.7949412)(637.19423592,51.7999412)(637.05423584,51.75994436)
\curveto(636.89423622,51.70994129)(636.74423637,51.65494134)(636.60423584,51.59494436)
\curveto(636.46423665,51.54494145)(636.32423679,51.48994151)(636.18423584,51.42994436)
\curveto(635.97423714,51.33994166)(635.76423735,51.25494174)(635.55423584,51.17494436)
\curveto(635.34423777,51.0949419)(635.13923798,51.01494198)(634.93923584,50.93494436)
\curveto(634.79923832,50.87494212)(634.66423845,50.81994218)(634.53423584,50.76994436)
\curveto(634.40423871,50.71994228)(634.26923885,50.66994233)(634.12923584,50.61994436)
\lineto(632.80923584,50.07994436)
\curveto(632.36924075,49.90994309)(631.92924119,49.73494326)(631.48923584,49.55494436)
\curveto(631.25924186,49.45494354)(631.03924208,49.36494363)(630.82923584,49.28494436)
\curveto(630.60924251,49.20494379)(630.38924273,49.11994388)(630.16923584,49.02994436)
\curveto(630.10924301,49.00994399)(630.02924309,48.97994402)(629.92923584,48.93994436)
\curveto(629.8192433,48.8999441)(629.72924339,48.90494409)(629.65923584,48.95494436)
\curveto(629.60924351,48.98494401)(629.57424354,49.04494395)(629.55423584,49.13494436)
\curveto(629.54424357,49.15494384)(629.54424357,49.17494382)(629.55423584,49.19494436)
\curveto(629.55424356,49.22494377)(629.54924357,49.24994375)(629.53923584,49.26994436)
}
}
{
\newrgbcolor{curcolor}{0 0 0}
\pscustom[linestyle=none,fillstyle=solid,fillcolor=curcolor]
{
}
}
{
\newrgbcolor{curcolor}{0 0 0}
\pscustom[linestyle=none,fillstyle=solid,fillcolor=curcolor]
{
\newpath
\moveto(626.65923584,64.24510061)
\curveto(626.64924647,64.93509597)(626.76924635,65.53509537)(627.01923584,66.04510061)
\curveto(627.26924585,66.56509434)(627.60424551,66.96009395)(628.02423584,67.23010061)
\curveto(628.10424501,67.28009363)(628.19424492,67.32509358)(628.29423584,67.36510061)
\curveto(628.38424473,67.4050935)(628.47924464,67.45009346)(628.57923584,67.50010061)
\curveto(628.67924444,67.54009337)(628.77924434,67.57009334)(628.87923584,67.59010061)
\curveto(628.97924414,67.6100933)(629.08424403,67.63009328)(629.19423584,67.65010061)
\curveto(629.24424387,67.67009324)(629.28924383,67.67509323)(629.32923584,67.66510061)
\curveto(629.36924375,67.65509325)(629.4142437,67.66009325)(629.46423584,67.68010061)
\curveto(629.5142436,67.69009322)(629.59924352,67.69509321)(629.71923584,67.69510061)
\curveto(629.82924329,67.69509321)(629.9142432,67.69009322)(629.97423584,67.68010061)
\curveto(630.03424308,67.66009325)(630.09424302,67.65009326)(630.15423584,67.65010061)
\curveto(630.2142429,67.66009325)(630.27424284,67.65509325)(630.33423584,67.63510061)
\curveto(630.47424264,67.59509331)(630.60924251,67.56009335)(630.73923584,67.53010061)
\curveto(630.86924225,67.50009341)(630.99424212,67.46009345)(631.11423584,67.41010061)
\curveto(631.25424186,67.35009356)(631.37924174,67.28009363)(631.48923584,67.20010061)
\curveto(631.59924152,67.13009378)(631.70924141,67.05509385)(631.81923584,66.97510061)
\lineto(631.87923584,66.91510061)
\curveto(631.89924122,66.905094)(631.9192412,66.89009402)(631.93923584,66.87010061)
\curveto(632.09924102,66.75009416)(632.24424087,66.61509429)(632.37423584,66.46510061)
\curveto(632.50424061,66.31509459)(632.62924049,66.15509475)(632.74923584,65.98510061)
\curveto(632.96924015,65.67509523)(633.17423994,65.38009553)(633.36423584,65.10010061)
\curveto(633.50423961,64.87009604)(633.63923948,64.64009627)(633.76923584,64.41010061)
\curveto(633.89923922,64.19009672)(634.03423908,63.97009694)(634.17423584,63.75010061)
\curveto(634.34423877,63.50009741)(634.52423859,63.26009765)(634.71423584,63.03010061)
\curveto(634.90423821,62.8100981)(635.12923799,62.62009829)(635.38923584,62.46010061)
\curveto(635.44923767,62.42009849)(635.50923761,62.38509852)(635.56923584,62.35510061)
\curveto(635.6192375,62.32509858)(635.68423743,62.29509861)(635.76423584,62.26510061)
\curveto(635.83423728,62.24509866)(635.89423722,62.24009867)(635.94423584,62.25010061)
\curveto(636.0142371,62.27009864)(636.06923705,62.3050986)(636.10923584,62.35510061)
\curveto(636.13923698,62.4050985)(636.15923696,62.46509844)(636.16923584,62.53510061)
\lineto(636.16923584,62.77510061)
\lineto(636.16923584,63.52510061)
\lineto(636.16923584,66.33010061)
\lineto(636.16923584,66.99010061)
\curveto(636.16923695,67.08009383)(636.17423694,67.16509374)(636.18423584,67.24510061)
\curveto(636.18423693,67.32509358)(636.20423691,67.39009352)(636.24423584,67.44010061)
\curveto(636.28423683,67.49009342)(636.35923676,67.53009338)(636.46923584,67.56010061)
\curveto(636.56923655,67.60009331)(636.66923645,67.6100933)(636.76923584,67.59010061)
\lineto(636.90423584,67.59010061)
\curveto(636.97423614,67.57009334)(637.03423608,67.55009336)(637.08423584,67.53010061)
\curveto(637.13423598,67.5100934)(637.17423594,67.47509343)(637.20423584,67.42510061)
\curveto(637.24423587,67.37509353)(637.26423585,67.3050936)(637.26423584,67.21510061)
\lineto(637.26423584,66.94510061)
\lineto(637.26423584,66.04510061)
\lineto(637.26423584,62.53510061)
\lineto(637.26423584,61.47010061)
\curveto(637.26423585,61.39009952)(637.26923585,61.30009961)(637.27923584,61.20010061)
\curveto(637.27923584,61.10009981)(637.26923585,61.01509989)(637.24923584,60.94510061)
\curveto(637.17923594,60.73510017)(636.99923612,60.67010024)(636.70923584,60.75010061)
\curveto(636.66923645,60.76010015)(636.63423648,60.76010015)(636.60423584,60.75010061)
\curveto(636.56423655,60.75010016)(636.5192366,60.76010015)(636.46923584,60.78010061)
\curveto(636.38923673,60.80010011)(636.30423681,60.82010009)(636.21423584,60.84010061)
\curveto(636.12423699,60.86010005)(636.03923708,60.88510002)(635.95923584,60.91510061)
\curveto(635.46923765,61.07509983)(635.05423806,61.27509963)(634.71423584,61.51510061)
\curveto(634.46423865,61.69509921)(634.23923888,61.90009901)(634.03923584,62.13010061)
\curveto(633.82923929,62.36009855)(633.63423948,62.60009831)(633.45423584,62.85010061)
\curveto(633.27423984,63.1100978)(633.10424001,63.37509753)(632.94423584,63.64510061)
\curveto(632.77424034,63.92509698)(632.59924052,64.19509671)(632.41923584,64.45510061)
\curveto(632.33924078,64.56509634)(632.26424085,64.67009624)(632.19423584,64.77010061)
\curveto(632.12424099,64.88009603)(632.04924107,64.99009592)(631.96923584,65.10010061)
\curveto(631.93924118,65.14009577)(631.90924121,65.17509573)(631.87923584,65.20510061)
\curveto(631.83924128,65.24509566)(631.80924131,65.28509562)(631.78923584,65.32510061)
\curveto(631.67924144,65.46509544)(631.55424156,65.59009532)(631.41423584,65.70010061)
\curveto(631.38424173,65.72009519)(631.35924176,65.74509516)(631.33923584,65.77510061)
\curveto(631.30924181,65.8050951)(631.27924184,65.83009508)(631.24923584,65.85010061)
\curveto(631.14924197,65.93009498)(631.04924207,65.99509491)(630.94923584,66.04510061)
\curveto(630.84924227,66.1050948)(630.73924238,66.16009475)(630.61923584,66.21010061)
\curveto(630.54924257,66.24009467)(630.47424264,66.26009465)(630.39423584,66.27010061)
\lineto(630.15423584,66.33010061)
\lineto(630.06423584,66.33010061)
\curveto(630.03424308,66.34009457)(630.00424311,66.34509456)(629.97423584,66.34510061)
\curveto(629.90424321,66.36509454)(629.80924331,66.37009454)(629.68923584,66.36010061)
\curveto(629.55924356,66.36009455)(629.45924366,66.35009456)(629.38923584,66.33010061)
\curveto(629.30924381,66.3100946)(629.23424388,66.29009462)(629.16423584,66.27010061)
\curveto(629.08424403,66.26009465)(629.00424411,66.24009467)(628.92423584,66.21010061)
\curveto(628.68424443,66.10009481)(628.48424463,65.95009496)(628.32423584,65.76010061)
\curveto(628.15424496,65.58009533)(628.0142451,65.36009555)(627.90423584,65.10010061)
\curveto(627.88424523,65.03009588)(627.86924525,64.96009595)(627.85923584,64.89010061)
\curveto(627.83924528,64.82009609)(627.8192453,64.74509616)(627.79923584,64.66510061)
\curveto(627.77924534,64.58509632)(627.76924535,64.47509643)(627.76923584,64.33510061)
\curveto(627.76924535,64.2050967)(627.77924534,64.10009681)(627.79923584,64.02010061)
\curveto(627.80924531,63.96009695)(627.8142453,63.905097)(627.81423584,63.85510061)
\curveto(627.8142453,63.8050971)(627.82424529,63.75509715)(627.84423584,63.70510061)
\curveto(627.88424523,63.6050973)(627.92424519,63.5100974)(627.96423584,63.42010061)
\curveto(628.00424511,63.34009757)(628.04924507,63.26009765)(628.09923584,63.18010061)
\curveto(628.119245,63.15009776)(628.14424497,63.12009779)(628.17423584,63.09010061)
\curveto(628.20424491,63.07009784)(628.22924489,63.04509786)(628.24923584,63.01510061)
\lineto(628.32423584,62.94010061)
\curveto(628.34424477,62.910098)(628.36424475,62.88509802)(628.38423584,62.86510061)
\lineto(628.59423584,62.71510061)
\curveto(628.65424446,62.67509823)(628.7192444,62.63009828)(628.78923584,62.58010061)
\curveto(628.87924424,62.52009839)(628.98424413,62.47009844)(629.10423584,62.43010061)
\curveto(629.2142439,62.40009851)(629.32424379,62.36509854)(629.43423584,62.32510061)
\curveto(629.54424357,62.28509862)(629.68924343,62.26009865)(629.86923584,62.25010061)
\curveto(630.03924308,62.24009867)(630.16424295,62.2100987)(630.24423584,62.16010061)
\curveto(630.32424279,62.1100988)(630.36924275,62.03509887)(630.37923584,61.93510061)
\curveto(630.38924273,61.83509907)(630.39424272,61.72509918)(630.39423584,61.60510061)
\curveto(630.39424272,61.56509934)(630.39924272,61.52509938)(630.40923584,61.48510061)
\curveto(630.40924271,61.44509946)(630.40424271,61.4100995)(630.39423584,61.38010061)
\curveto(630.37424274,61.33009958)(630.36424275,61.28009963)(630.36423584,61.23010061)
\curveto(630.36424275,61.19009972)(630.35424276,61.15009976)(630.33423584,61.11010061)
\curveto(630.27424284,61.02009989)(630.13924298,60.97509993)(629.92923584,60.97510061)
\lineto(629.80923584,60.97510061)
\curveto(629.74924337,60.98509992)(629.68924343,60.99009992)(629.62923584,60.99010061)
\curveto(629.55924356,61.00009991)(629.49424362,61.0100999)(629.43423584,61.02010061)
\curveto(629.32424379,61.04009987)(629.22424389,61.06009985)(629.13423584,61.08010061)
\curveto(629.03424408,61.10009981)(628.93924418,61.13009978)(628.84923584,61.17010061)
\curveto(628.77924434,61.19009972)(628.7192444,61.2100997)(628.66923584,61.23010061)
\lineto(628.48923584,61.29010061)
\curveto(628.22924489,61.4100995)(627.98424513,61.56509934)(627.75423584,61.75510061)
\curveto(627.52424559,61.95509895)(627.33924578,62.17009874)(627.19923584,62.40010061)
\curveto(627.119246,62.5100984)(627.05424606,62.62509828)(627.00423584,62.74510061)
\lineto(626.85423584,63.13510061)
\curveto(626.80424631,63.24509766)(626.77424634,63.36009755)(626.76423584,63.48010061)
\curveto(626.74424637,63.60009731)(626.7192464,63.72509718)(626.68923584,63.85510061)
\curveto(626.68924643,63.92509698)(626.68924643,63.99009692)(626.68923584,64.05010061)
\curveto(626.67924644,64.1100968)(626.66924645,64.17509673)(626.65923584,64.24510061)
}
}
{
\newrgbcolor{curcolor}{0 0 0}
\pscustom[linestyle=none,fillstyle=solid,fillcolor=curcolor]
{
\newpath
\moveto(631.66923584,76.28470998)
\curveto(631.74924137,76.28470235)(631.82924129,76.28970234)(631.90923584,76.29970998)
\curveto(631.98924113,76.30970232)(632.06424105,76.30470233)(632.13423584,76.28470998)
\curveto(632.17424094,76.26470237)(632.2192409,76.25970237)(632.26923584,76.26970998)
\curveto(632.30924081,76.27970235)(632.34924077,76.27970235)(632.38923584,76.26970998)
\lineto(632.53923584,76.26970998)
\curveto(632.62924049,76.25970237)(632.7192404,76.25470238)(632.80923584,76.25470998)
\curveto(632.88924023,76.25470238)(632.96924015,76.24970238)(633.04923584,76.23970998)
\lineto(633.28923584,76.20970998)
\curveto(633.35923976,76.19970243)(633.43423968,76.18970244)(633.51423584,76.17970998)
\curveto(633.55423956,76.16970246)(633.59423952,76.16470247)(633.63423584,76.16470998)
\curveto(633.67423944,76.16470247)(633.7192394,76.15970247)(633.76923584,76.14970998)
\curveto(633.90923921,76.10970252)(634.04923907,76.07970255)(634.18923584,76.05970998)
\curveto(634.32923879,76.04970258)(634.46423865,76.01970261)(634.59423584,75.96970998)
\curveto(634.76423835,75.91970271)(634.92923819,75.86470277)(635.08923584,75.80470998)
\curveto(635.24923787,75.75470288)(635.40423771,75.69470294)(635.55423584,75.62470998)
\curveto(635.6142375,75.60470303)(635.67423744,75.57470306)(635.73423584,75.53470998)
\lineto(635.88423584,75.44470998)
\curveto(636.20423691,75.24470339)(636.46923665,75.0297036)(636.67923584,74.79970998)
\curveto(636.88923623,74.56970406)(637.06923605,74.27470436)(637.21923584,73.91470998)
\curveto(637.26923585,73.79470484)(637.30423581,73.66470497)(637.32423584,73.52470998)
\curveto(637.34423577,73.39470524)(637.36923575,73.25970537)(637.39923584,73.11970998)
\curveto(637.40923571,73.05970557)(637.4142357,72.99970563)(637.41423584,72.93970998)
\curveto(637.4142357,72.87970575)(637.4192357,72.81470582)(637.42923584,72.74470998)
\curveto(637.43923568,72.71470592)(637.43923568,72.66470597)(637.42923584,72.59470998)
\lineto(637.42923584,72.44470998)
\lineto(637.42923584,72.29470998)
\curveto(637.40923571,72.21470642)(637.39423572,72.1297065)(637.38423584,72.03970998)
\curveto(637.38423573,71.95970667)(637.37423574,71.88470675)(637.35423584,71.81470998)
\curveto(637.34423577,71.77470686)(637.33923578,71.73970689)(637.33923584,71.70970998)
\curveto(637.34923577,71.68970694)(637.34423577,71.66470697)(637.32423584,71.63470998)
\lineto(637.26423584,71.36470998)
\curveto(637.23423588,71.27470736)(637.20423591,71.18970744)(637.17423584,71.10970998)
\curveto(636.93423618,70.5297081)(636.56423655,70.09470854)(636.06423584,69.80470998)
\curveto(635.93423718,69.72470891)(635.79923732,69.65970897)(635.65923584,69.60970998)
\curveto(635.5192376,69.56970906)(635.36923775,69.52470911)(635.20923584,69.47470998)
\curveto(635.12923799,69.45470918)(635.04923807,69.44970918)(634.96923584,69.45970998)
\curveto(634.88923823,69.47970915)(634.83423828,69.51470912)(634.80423584,69.56470998)
\curveto(634.78423833,69.59470904)(634.76923835,69.64970898)(634.75923584,69.72970998)
\curveto(634.73923838,69.80970882)(634.72923839,69.89470874)(634.72923584,69.98470998)
\curveto(634.7192384,70.07470856)(634.7192384,70.15970847)(634.72923584,70.23970998)
\curveto(634.73923838,70.3297083)(634.74923837,70.39970823)(634.75923584,70.44970998)
\curveto(634.76923835,70.46970816)(634.78423833,70.49470814)(634.80423584,70.52470998)
\curveto(634.82423829,70.56470807)(634.84423827,70.59470804)(634.86423584,70.61470998)
\curveto(634.94423817,70.67470796)(635.03923808,70.71970791)(635.14923584,70.74970998)
\curveto(635.25923786,70.78970784)(635.35923776,70.8347078)(635.44923584,70.88470998)
\curveto(635.83923728,71.1347075)(636.10923701,71.50470713)(636.25923584,71.99470998)
\curveto(636.27923684,72.06470657)(636.29423682,72.1347065)(636.30423584,72.20470998)
\curveto(636.30423681,72.28470635)(636.3142368,72.36470627)(636.33423584,72.44470998)
\curveto(636.34423677,72.48470615)(636.34923677,72.53970609)(636.34923584,72.60970998)
\curveto(636.34923677,72.68970594)(636.34423677,72.74470589)(636.33423584,72.77470998)
\curveto(636.32423679,72.80470583)(636.3192368,72.8347058)(636.31923584,72.86470998)
\lineto(636.31923584,72.96970998)
\curveto(636.29923682,73.04970558)(636.27923684,73.12470551)(636.25923584,73.19470998)
\curveto(636.23923688,73.27470536)(636.2142369,73.34970528)(636.18423584,73.41970998)
\curveto(636.03423708,73.76970486)(635.8192373,74.03970459)(635.53923584,74.22970998)
\curveto(635.25923786,74.41970421)(634.93423818,74.57470406)(634.56423584,74.69470998)
\curveto(634.48423863,74.72470391)(634.40923871,74.74470389)(634.33923584,74.75470998)
\curveto(634.26923885,74.77470386)(634.19423892,74.79470384)(634.11423584,74.81470998)
\curveto(634.02423909,74.8347038)(633.92923919,74.84970378)(633.82923584,74.85970998)
\curveto(633.7192394,74.87970375)(633.6142395,74.89970373)(633.51423584,74.91970998)
\curveto(633.46423965,74.9297037)(633.4142397,74.9347037)(633.36423584,74.93470998)
\curveto(633.30423981,74.94470369)(633.24923987,74.94970368)(633.19923584,74.94970998)
\curveto(633.13923998,74.96970366)(633.06424005,74.97970365)(632.97423584,74.97970998)
\curveto(632.87424024,74.97970365)(632.79424032,74.96970366)(632.73423584,74.94970998)
\curveto(632.64424047,74.91970371)(632.60424051,74.86970376)(632.61423584,74.79970998)
\curveto(632.62424049,74.73970389)(632.65424046,74.68470395)(632.70423584,74.63470998)
\curveto(632.75424036,74.55470408)(632.8142403,74.48470415)(632.88423584,74.42470998)
\curveto(632.95424016,74.37470426)(633.0142401,74.30970432)(633.06423584,74.22970998)
\curveto(633.17423994,74.06970456)(633.27423984,73.90470473)(633.36423584,73.73470998)
\curveto(633.44423967,73.56470507)(633.5142396,73.36970526)(633.57423584,73.14970998)
\curveto(633.60423951,73.04970558)(633.6192395,72.94970568)(633.61923584,72.84970998)
\curveto(633.6192395,72.75970587)(633.62923949,72.65970597)(633.64923584,72.54970998)
\lineto(633.64923584,72.39970998)
\curveto(633.62923949,72.34970628)(633.62423949,72.29970633)(633.63423584,72.24970998)
\curveto(633.64423947,72.20970642)(633.64423947,72.16970646)(633.63423584,72.12970998)
\curveto(633.62423949,72.09970653)(633.6192395,72.05470658)(633.61923584,71.99470998)
\curveto(633.60923951,71.9347067)(633.59923952,71.86970676)(633.58923584,71.79970998)
\lineto(633.55923584,71.61970998)
\curveto(633.43923968,71.16970746)(633.27423984,70.78970784)(633.06423584,70.47970998)
\curveto(632.87424024,70.20970842)(632.64424047,69.97970865)(632.37423584,69.78970998)
\curveto(632.09424102,69.60970902)(631.77924134,69.46470917)(631.42923584,69.35470998)
\lineto(631.21923584,69.29470998)
\curveto(631.13924198,69.28470935)(631.05924206,69.26970936)(630.97923584,69.24970998)
\curveto(630.94924217,69.23970939)(630.9192422,69.2347094)(630.88923584,69.23470998)
\curveto(630.85924226,69.2347094)(630.82924229,69.2297094)(630.79923584,69.21970998)
\curveto(630.73924238,69.20970942)(630.67924244,69.20470943)(630.61923584,69.20470998)
\curveto(630.54924257,69.20470943)(630.48924263,69.19470944)(630.43923584,69.17470998)
\lineto(630.25923584,69.17470998)
\curveto(630.20924291,69.16470947)(630.13924298,69.15970947)(630.04923584,69.15970998)
\curveto(629.95924316,69.15970947)(629.88924323,69.16970946)(629.83923584,69.18970998)
\lineto(629.67423584,69.18970998)
\curveto(629.59424352,69.20970942)(629.5192436,69.21970941)(629.44923584,69.21970998)
\curveto(629.37924374,69.2297094)(629.30924381,69.24470939)(629.23923584,69.26470998)
\curveto(629.03924408,69.32470931)(628.84924427,69.38470925)(628.66923584,69.44470998)
\curveto(628.48924463,69.51470912)(628.3192448,69.60470903)(628.15923584,69.71470998)
\curveto(628.08924503,69.75470888)(628.02424509,69.79470884)(627.96423584,69.83470998)
\lineto(627.78423584,69.98470998)
\curveto(627.77424534,70.00470863)(627.75924536,70.02470861)(627.73923584,70.04470998)
\curveto(627.60924551,70.1347085)(627.49924562,70.24470839)(627.40923584,70.37470998)
\curveto(627.20924591,70.634708)(627.05424606,70.89970773)(626.94423584,71.16970998)
\curveto(626.90424621,71.24970738)(626.87424624,71.3297073)(626.85423584,71.40970998)
\curveto(626.82424629,71.49970713)(626.79924632,71.58970704)(626.77923584,71.67970998)
\curveto(626.74924637,71.77970685)(626.72924639,71.87970675)(626.71923584,71.97970998)
\curveto(626.70924641,72.07970655)(626.69424642,72.18470645)(626.67423584,72.29470998)
\curveto(626.66424645,72.32470631)(626.66424645,72.36470627)(626.67423584,72.41470998)
\curveto(626.68424643,72.47470616)(626.67924644,72.51470612)(626.65923584,72.53470998)
\curveto(626.63924648,73.25470538)(626.75424636,73.85470478)(627.00423584,74.33470998)
\curveto(627.25424586,74.81470382)(627.59424552,75.18970344)(628.02423584,75.45970998)
\curveto(628.16424495,75.54970308)(628.30924481,75.629703)(628.45923584,75.69970998)
\curveto(628.60924451,75.76970286)(628.76924435,75.83970279)(628.93923584,75.90970998)
\curveto(629.07924404,75.95970267)(629.22924389,75.99970263)(629.38923584,76.02970998)
\curveto(629.54924357,76.05970257)(629.70924341,76.09470254)(629.86923584,76.13470998)
\curveto(629.9192432,76.15470248)(629.97424314,76.16470247)(630.03423584,76.16470998)
\curveto(630.08424303,76.16470247)(630.13424298,76.16970246)(630.18423584,76.17970998)
\curveto(630.24424287,76.19970243)(630.30924281,76.20970242)(630.37923584,76.20970998)
\curveto(630.43924268,76.20970242)(630.49424262,76.21970241)(630.54423584,76.23970998)
\lineto(630.70923584,76.23970998)
\curveto(630.75924236,76.25970237)(630.80924231,76.26470237)(630.85923584,76.25470998)
\curveto(630.90924221,76.24470239)(630.95924216,76.24970238)(631.00923584,76.26970998)
\curveto(631.02924209,76.26970236)(631.05424206,76.26470237)(631.08423584,76.25470998)
\curveto(631.114242,76.25470238)(631.13924198,76.25970237)(631.15923584,76.26970998)
\curveto(631.18924193,76.27970235)(631.22424189,76.27970235)(631.26423584,76.26970998)
\curveto(631.30424181,76.26970236)(631.34424177,76.27470236)(631.38423584,76.28470998)
\curveto(631.42424169,76.29470234)(631.46924165,76.29470234)(631.51923584,76.28470998)
\lineto(631.66923584,76.28470998)
\moveto(630.36423584,74.78470998)
\curveto(630.3142428,74.79470384)(630.25424286,74.79970383)(630.18423584,74.79970998)
\curveto(630.114243,74.79970383)(630.05424306,74.79470384)(630.00423584,74.78470998)
\curveto(629.95424316,74.77470386)(629.87924324,74.76970386)(629.77923584,74.76970998)
\curveto(629.69924342,74.74970388)(629.62424349,74.7297039)(629.55423584,74.70970998)
\curveto(629.48424363,74.69970393)(629.4142437,74.68470395)(629.34423584,74.66470998)
\curveto(628.9142442,74.52470411)(628.57924454,74.3297043)(628.33923584,74.07970998)
\curveto(628.09924502,73.83970479)(627.9192452,73.49470514)(627.79923584,73.04470998)
\curveto(627.77924534,72.95470568)(627.76924535,72.85470578)(627.76923584,72.74470998)
\lineto(627.76923584,72.41470998)
\curveto(627.78924533,72.39470624)(627.79924532,72.35970627)(627.79923584,72.30970998)
\curveto(627.78924533,72.25970637)(627.78924533,72.21470642)(627.79923584,72.17470998)
\curveto(627.8192453,72.09470654)(627.83924528,72.01970661)(627.85923584,71.94970998)
\lineto(627.91923584,71.73970998)
\curveto(628.04924507,71.44970718)(628.22924489,71.21970741)(628.45923584,71.04970998)
\curveto(628.67924444,70.87970775)(628.93924418,70.74470789)(629.23923584,70.64470998)
\curveto(629.32924379,70.61470802)(629.42424369,70.58970804)(629.52423584,70.56970998)
\curveto(629.6142435,70.55970807)(629.70924341,70.54470809)(629.80923584,70.52470998)
\lineto(629.94423584,70.52470998)
\curveto(630.05424306,70.49470814)(630.19424292,70.48470815)(630.36423584,70.49470998)
\curveto(630.52424259,70.51470812)(630.65424246,70.5347081)(630.75423584,70.55470998)
\curveto(630.8142423,70.57470806)(630.87424224,70.58970804)(630.93423584,70.59970998)
\curveto(630.98424213,70.60970802)(631.03424208,70.62470801)(631.08423584,70.64470998)
\curveto(631.28424183,70.72470791)(631.47424164,70.81970781)(631.65423584,70.92970998)
\curveto(631.83424128,71.04970758)(631.97924114,71.18970744)(632.08923584,71.34970998)
\curveto(632.13924098,71.39970723)(632.17924094,71.45470718)(632.20923584,71.51470998)
\curveto(632.23924088,71.57470706)(632.27424084,71.634707)(632.31423584,71.69470998)
\curveto(632.39424072,71.84470679)(632.45924066,72.0297066)(632.50923584,72.24970998)
\curveto(632.52924059,72.29970633)(632.53424058,72.33970629)(632.52423584,72.36970998)
\curveto(632.5142406,72.40970622)(632.5192406,72.45470618)(632.53923584,72.50470998)
\curveto(632.54924057,72.54470609)(632.55424056,72.59970603)(632.55423584,72.66970998)
\curveto(632.55424056,72.73970589)(632.54924057,72.79970583)(632.53923584,72.84970998)
\curveto(632.5192406,72.94970568)(632.50424061,73.04470559)(632.49423584,73.13470998)
\curveto(632.47424064,73.22470541)(632.44424067,73.31470532)(632.40423584,73.40470998)
\curveto(632.18424093,73.94470469)(631.78924133,74.33970429)(631.21923584,74.58970998)
\curveto(631.119242,74.63970399)(631.0192421,74.67470396)(630.91923584,74.69470998)
\curveto(630.80924231,74.71470392)(630.69924242,74.73970389)(630.58923584,74.76970998)
\curveto(630.48924263,74.76970386)(630.4142427,74.77470386)(630.36423584,74.78470998)
}
}
{
\newrgbcolor{curcolor}{0 0 0}
\pscustom[linestyle=none,fillstyle=solid,fillcolor=curcolor]
{
\newpath
\moveto(635.62923584,78.63431936)
\lineto(635.62923584,79.26431936)
\lineto(635.62923584,79.45931936)
\curveto(635.62923749,79.52931683)(635.63923748,79.58931677)(635.65923584,79.63931936)
\curveto(635.69923742,79.70931665)(635.73923738,79.7593166)(635.77923584,79.78931936)
\curveto(635.82923729,79.82931653)(635.89423722,79.84931651)(635.97423584,79.84931936)
\curveto(636.05423706,79.8593165)(636.13923698,79.86431649)(636.22923584,79.86431936)
\lineto(636.94923584,79.86431936)
\curveto(637.42923569,79.86431649)(637.83923528,79.80431655)(638.17923584,79.68431936)
\curveto(638.5192346,79.56431679)(638.79423432,79.36931699)(639.00423584,79.09931936)
\curveto(639.05423406,79.02931733)(639.09923402,78.9593174)(639.13923584,78.88931936)
\curveto(639.18923393,78.82931753)(639.23423388,78.7543176)(639.27423584,78.66431936)
\curveto(639.28423383,78.64431771)(639.29423382,78.61431774)(639.30423584,78.57431936)
\curveto(639.32423379,78.53431782)(639.32923379,78.48931787)(639.31923584,78.43931936)
\curveto(639.28923383,78.34931801)(639.2142339,78.29431806)(639.09423584,78.27431936)
\curveto(638.98423413,78.2543181)(638.88923423,78.26931809)(638.80923584,78.31931936)
\curveto(638.73923438,78.34931801)(638.67423444,78.39431796)(638.61423584,78.45431936)
\curveto(638.56423455,78.52431783)(638.5142346,78.58931777)(638.46423584,78.64931936)
\curveto(638.4142347,78.71931764)(638.33923478,78.77931758)(638.23923584,78.82931936)
\curveto(638.14923497,78.88931747)(638.05923506,78.93931742)(637.96923584,78.97931936)
\curveto(637.93923518,78.99931736)(637.87923524,79.02431733)(637.78923584,79.05431936)
\curveto(637.70923541,79.08431727)(637.63923548,79.08931727)(637.57923584,79.06931936)
\curveto(637.43923568,79.03931732)(637.34923577,78.97931738)(637.30923584,78.88931936)
\curveto(637.27923584,78.80931755)(637.26423585,78.71931764)(637.26423584,78.61931936)
\curveto(637.26423585,78.51931784)(637.23923588,78.43431792)(637.18923584,78.36431936)
\curveto(637.119236,78.27431808)(636.97923614,78.22931813)(636.76923584,78.22931936)
\lineto(636.21423584,78.22931936)
\lineto(635.98923584,78.22931936)
\curveto(635.90923721,78.23931812)(635.84423727,78.2593181)(635.79423584,78.28931936)
\curveto(635.7142374,78.34931801)(635.66923745,78.41931794)(635.65923584,78.49931936)
\curveto(635.64923747,78.51931784)(635.64423747,78.53931782)(635.64423584,78.55931936)
\curveto(635.64423747,78.58931777)(635.63923748,78.61431774)(635.62923584,78.63431936)
}
}
{
\newrgbcolor{curcolor}{0 0 0}
\pscustom[linestyle=none,fillstyle=solid,fillcolor=curcolor]
{
}
}
{
\newrgbcolor{curcolor}{0 0 0}
\pscustom[linestyle=none,fillstyle=solid,fillcolor=curcolor]
{
\newpath
\moveto(626.65923584,89.26463186)
\curveto(626.64924647,89.95462722)(626.76924635,90.55462662)(627.01923584,91.06463186)
\curveto(627.26924585,91.58462559)(627.60424551,91.9796252)(628.02423584,92.24963186)
\curveto(628.10424501,92.29962488)(628.19424492,92.34462483)(628.29423584,92.38463186)
\curveto(628.38424473,92.42462475)(628.47924464,92.46962471)(628.57923584,92.51963186)
\curveto(628.67924444,92.55962462)(628.77924434,92.58962459)(628.87923584,92.60963186)
\curveto(628.97924414,92.62962455)(629.08424403,92.64962453)(629.19423584,92.66963186)
\curveto(629.24424387,92.68962449)(629.28924383,92.69462448)(629.32923584,92.68463186)
\curveto(629.36924375,92.6746245)(629.4142437,92.6796245)(629.46423584,92.69963186)
\curveto(629.5142436,92.70962447)(629.59924352,92.71462446)(629.71923584,92.71463186)
\curveto(629.82924329,92.71462446)(629.9142432,92.70962447)(629.97423584,92.69963186)
\curveto(630.03424308,92.6796245)(630.09424302,92.66962451)(630.15423584,92.66963186)
\curveto(630.2142429,92.6796245)(630.27424284,92.6746245)(630.33423584,92.65463186)
\curveto(630.47424264,92.61462456)(630.60924251,92.5796246)(630.73923584,92.54963186)
\curveto(630.86924225,92.51962466)(630.99424212,92.4796247)(631.11423584,92.42963186)
\curveto(631.25424186,92.36962481)(631.37924174,92.29962488)(631.48923584,92.21963186)
\curveto(631.59924152,92.14962503)(631.70924141,92.0746251)(631.81923584,91.99463186)
\lineto(631.87923584,91.93463186)
\curveto(631.89924122,91.92462525)(631.9192412,91.90962527)(631.93923584,91.88963186)
\curveto(632.09924102,91.76962541)(632.24424087,91.63462554)(632.37423584,91.48463186)
\curveto(632.50424061,91.33462584)(632.62924049,91.174626)(632.74923584,91.00463186)
\curveto(632.96924015,90.69462648)(633.17423994,90.39962678)(633.36423584,90.11963186)
\curveto(633.50423961,89.88962729)(633.63923948,89.65962752)(633.76923584,89.42963186)
\curveto(633.89923922,89.20962797)(634.03423908,88.98962819)(634.17423584,88.76963186)
\curveto(634.34423877,88.51962866)(634.52423859,88.2796289)(634.71423584,88.04963186)
\curveto(634.90423821,87.82962935)(635.12923799,87.63962954)(635.38923584,87.47963186)
\curveto(635.44923767,87.43962974)(635.50923761,87.40462977)(635.56923584,87.37463186)
\curveto(635.6192375,87.34462983)(635.68423743,87.31462986)(635.76423584,87.28463186)
\curveto(635.83423728,87.26462991)(635.89423722,87.25962992)(635.94423584,87.26963186)
\curveto(636.0142371,87.28962989)(636.06923705,87.32462985)(636.10923584,87.37463186)
\curveto(636.13923698,87.42462975)(636.15923696,87.48462969)(636.16923584,87.55463186)
\lineto(636.16923584,87.79463186)
\lineto(636.16923584,88.54463186)
\lineto(636.16923584,91.34963186)
\lineto(636.16923584,92.00963186)
\curveto(636.16923695,92.09962508)(636.17423694,92.18462499)(636.18423584,92.26463186)
\curveto(636.18423693,92.34462483)(636.20423691,92.40962477)(636.24423584,92.45963186)
\curveto(636.28423683,92.50962467)(636.35923676,92.54962463)(636.46923584,92.57963186)
\curveto(636.56923655,92.61962456)(636.66923645,92.62962455)(636.76923584,92.60963186)
\lineto(636.90423584,92.60963186)
\curveto(636.97423614,92.58962459)(637.03423608,92.56962461)(637.08423584,92.54963186)
\curveto(637.13423598,92.52962465)(637.17423594,92.49462468)(637.20423584,92.44463186)
\curveto(637.24423587,92.39462478)(637.26423585,92.32462485)(637.26423584,92.23463186)
\lineto(637.26423584,91.96463186)
\lineto(637.26423584,91.06463186)
\lineto(637.26423584,87.55463186)
\lineto(637.26423584,86.48963186)
\curveto(637.26423585,86.40963077)(637.26923585,86.31963086)(637.27923584,86.21963186)
\curveto(637.27923584,86.11963106)(637.26923585,86.03463114)(637.24923584,85.96463186)
\curveto(637.17923594,85.75463142)(636.99923612,85.68963149)(636.70923584,85.76963186)
\curveto(636.66923645,85.7796314)(636.63423648,85.7796314)(636.60423584,85.76963186)
\curveto(636.56423655,85.76963141)(636.5192366,85.7796314)(636.46923584,85.79963186)
\curveto(636.38923673,85.81963136)(636.30423681,85.83963134)(636.21423584,85.85963186)
\curveto(636.12423699,85.8796313)(636.03923708,85.90463127)(635.95923584,85.93463186)
\curveto(635.46923765,86.09463108)(635.05423806,86.29463088)(634.71423584,86.53463186)
\curveto(634.46423865,86.71463046)(634.23923888,86.91963026)(634.03923584,87.14963186)
\curveto(633.82923929,87.3796298)(633.63423948,87.61962956)(633.45423584,87.86963186)
\curveto(633.27423984,88.12962905)(633.10424001,88.39462878)(632.94423584,88.66463186)
\curveto(632.77424034,88.94462823)(632.59924052,89.21462796)(632.41923584,89.47463186)
\curveto(632.33924078,89.58462759)(632.26424085,89.68962749)(632.19423584,89.78963186)
\curveto(632.12424099,89.89962728)(632.04924107,90.00962717)(631.96923584,90.11963186)
\curveto(631.93924118,90.15962702)(631.90924121,90.19462698)(631.87923584,90.22463186)
\curveto(631.83924128,90.26462691)(631.80924131,90.30462687)(631.78923584,90.34463186)
\curveto(631.67924144,90.48462669)(631.55424156,90.60962657)(631.41423584,90.71963186)
\curveto(631.38424173,90.73962644)(631.35924176,90.76462641)(631.33923584,90.79463186)
\curveto(631.30924181,90.82462635)(631.27924184,90.84962633)(631.24923584,90.86963186)
\curveto(631.14924197,90.94962623)(631.04924207,91.01462616)(630.94923584,91.06463186)
\curveto(630.84924227,91.12462605)(630.73924238,91.179626)(630.61923584,91.22963186)
\curveto(630.54924257,91.25962592)(630.47424264,91.2796259)(630.39423584,91.28963186)
\lineto(630.15423584,91.34963186)
\lineto(630.06423584,91.34963186)
\curveto(630.03424308,91.35962582)(630.00424311,91.36462581)(629.97423584,91.36463186)
\curveto(629.90424321,91.38462579)(629.80924331,91.38962579)(629.68923584,91.37963186)
\curveto(629.55924356,91.3796258)(629.45924366,91.36962581)(629.38923584,91.34963186)
\curveto(629.30924381,91.32962585)(629.23424388,91.30962587)(629.16423584,91.28963186)
\curveto(629.08424403,91.2796259)(629.00424411,91.25962592)(628.92423584,91.22963186)
\curveto(628.68424443,91.11962606)(628.48424463,90.96962621)(628.32423584,90.77963186)
\curveto(628.15424496,90.59962658)(628.0142451,90.3796268)(627.90423584,90.11963186)
\curveto(627.88424523,90.04962713)(627.86924525,89.9796272)(627.85923584,89.90963186)
\curveto(627.83924528,89.83962734)(627.8192453,89.76462741)(627.79923584,89.68463186)
\curveto(627.77924534,89.60462757)(627.76924535,89.49462768)(627.76923584,89.35463186)
\curveto(627.76924535,89.22462795)(627.77924534,89.11962806)(627.79923584,89.03963186)
\curveto(627.80924531,88.9796282)(627.8142453,88.92462825)(627.81423584,88.87463186)
\curveto(627.8142453,88.82462835)(627.82424529,88.7746284)(627.84423584,88.72463186)
\curveto(627.88424523,88.62462855)(627.92424519,88.52962865)(627.96423584,88.43963186)
\curveto(628.00424511,88.35962882)(628.04924507,88.2796289)(628.09923584,88.19963186)
\curveto(628.119245,88.16962901)(628.14424497,88.13962904)(628.17423584,88.10963186)
\curveto(628.20424491,88.08962909)(628.22924489,88.06462911)(628.24923584,88.03463186)
\lineto(628.32423584,87.95963186)
\curveto(628.34424477,87.92962925)(628.36424475,87.90462927)(628.38423584,87.88463186)
\lineto(628.59423584,87.73463186)
\curveto(628.65424446,87.69462948)(628.7192444,87.64962953)(628.78923584,87.59963186)
\curveto(628.87924424,87.53962964)(628.98424413,87.48962969)(629.10423584,87.44963186)
\curveto(629.2142439,87.41962976)(629.32424379,87.38462979)(629.43423584,87.34463186)
\curveto(629.54424357,87.30462987)(629.68924343,87.2796299)(629.86923584,87.26963186)
\curveto(630.03924308,87.25962992)(630.16424295,87.22962995)(630.24423584,87.17963186)
\curveto(630.32424279,87.12963005)(630.36924275,87.05463012)(630.37923584,86.95463186)
\curveto(630.38924273,86.85463032)(630.39424272,86.74463043)(630.39423584,86.62463186)
\curveto(630.39424272,86.58463059)(630.39924272,86.54463063)(630.40923584,86.50463186)
\curveto(630.40924271,86.46463071)(630.40424271,86.42963075)(630.39423584,86.39963186)
\curveto(630.37424274,86.34963083)(630.36424275,86.29963088)(630.36423584,86.24963186)
\curveto(630.36424275,86.20963097)(630.35424276,86.16963101)(630.33423584,86.12963186)
\curveto(630.27424284,86.03963114)(630.13924298,85.99463118)(629.92923584,85.99463186)
\lineto(629.80923584,85.99463186)
\curveto(629.74924337,86.00463117)(629.68924343,86.00963117)(629.62923584,86.00963186)
\curveto(629.55924356,86.01963116)(629.49424362,86.02963115)(629.43423584,86.03963186)
\curveto(629.32424379,86.05963112)(629.22424389,86.0796311)(629.13423584,86.09963186)
\curveto(629.03424408,86.11963106)(628.93924418,86.14963103)(628.84923584,86.18963186)
\curveto(628.77924434,86.20963097)(628.7192444,86.22963095)(628.66923584,86.24963186)
\lineto(628.48923584,86.30963186)
\curveto(628.22924489,86.42963075)(627.98424513,86.58463059)(627.75423584,86.77463186)
\curveto(627.52424559,86.9746302)(627.33924578,87.18962999)(627.19923584,87.41963186)
\curveto(627.119246,87.52962965)(627.05424606,87.64462953)(627.00423584,87.76463186)
\lineto(626.85423584,88.15463186)
\curveto(626.80424631,88.26462891)(626.77424634,88.3796288)(626.76423584,88.49963186)
\curveto(626.74424637,88.61962856)(626.7192464,88.74462843)(626.68923584,88.87463186)
\curveto(626.68924643,88.94462823)(626.68924643,89.00962817)(626.68923584,89.06963186)
\curveto(626.67924644,89.12962805)(626.66924645,89.19462798)(626.65923584,89.26463186)
}
}
{
\newrgbcolor{curcolor}{0 0 0}
\pscustom[linestyle=none,fillstyle=solid,fillcolor=curcolor]
{
\newpath
\moveto(632.17923584,101.36424123)
\lineto(632.43423584,101.36424123)
\curveto(632.5142406,101.37423353)(632.58924053,101.36923353)(632.65923584,101.34924123)
\lineto(632.89923584,101.34924123)
\lineto(633.06423584,101.34924123)
\curveto(633.16423995,101.32923357)(633.26923985,101.31923358)(633.37923584,101.31924123)
\curveto(633.47923964,101.31923358)(633.57923954,101.30923359)(633.67923584,101.28924123)
\lineto(633.82923584,101.28924123)
\curveto(633.96923915,101.25923364)(634.10923901,101.23923366)(634.24923584,101.22924123)
\curveto(634.37923874,101.21923368)(634.50923861,101.19423371)(634.63923584,101.15424123)
\curveto(634.7192384,101.13423377)(634.80423831,101.11423379)(634.89423584,101.09424123)
\lineto(635.13423584,101.03424123)
\lineto(635.43423584,100.91424123)
\curveto(635.52423759,100.88423402)(635.6142375,100.84923405)(635.70423584,100.80924123)
\curveto(635.92423719,100.70923419)(636.13923698,100.57423433)(636.34923584,100.40424123)
\curveto(636.55923656,100.24423466)(636.72923639,100.06923483)(636.85923584,99.87924123)
\curveto(636.89923622,99.82923507)(636.93923618,99.76923513)(636.97923584,99.69924123)
\curveto(637.00923611,99.63923526)(637.04423607,99.57923532)(637.08423584,99.51924123)
\curveto(637.13423598,99.43923546)(637.17423594,99.34423556)(637.20423584,99.23424123)
\curveto(637.23423588,99.12423578)(637.26423585,99.01923588)(637.29423584,98.91924123)
\curveto(637.33423578,98.80923609)(637.35923576,98.6992362)(637.36923584,98.58924123)
\curveto(637.37923574,98.47923642)(637.39423572,98.36423654)(637.41423584,98.24424123)
\curveto(637.42423569,98.2042367)(637.42423569,98.15923674)(637.41423584,98.10924123)
\curveto(637.4142357,98.06923683)(637.4192357,98.02923687)(637.42923584,97.98924123)
\curveto(637.43923568,97.94923695)(637.44423567,97.89423701)(637.44423584,97.82424123)
\curveto(637.44423567,97.75423715)(637.43923568,97.7042372)(637.42923584,97.67424123)
\curveto(637.40923571,97.62423728)(637.40423571,97.57923732)(637.41423584,97.53924123)
\curveto(637.42423569,97.4992374)(637.42423569,97.46423744)(637.41423584,97.43424123)
\lineto(637.41423584,97.34424123)
\curveto(637.39423572,97.28423762)(637.37923574,97.21923768)(637.36923584,97.14924123)
\curveto(637.36923575,97.08923781)(637.36423575,97.02423788)(637.35423584,96.95424123)
\curveto(637.30423581,96.78423812)(637.25423586,96.62423828)(637.20423584,96.47424123)
\curveto(637.15423596,96.32423858)(637.08923603,96.17923872)(637.00923584,96.03924123)
\curveto(636.96923615,95.98923891)(636.93923618,95.93423897)(636.91923584,95.87424123)
\curveto(636.88923623,95.82423908)(636.85423626,95.77423913)(636.81423584,95.72424123)
\curveto(636.63423648,95.48423942)(636.4142367,95.28423962)(636.15423584,95.12424123)
\curveto(635.89423722,94.96423994)(635.60923751,94.82424008)(635.29923584,94.70424123)
\curveto(635.15923796,94.64424026)(635.0192381,94.5992403)(634.87923584,94.56924123)
\curveto(634.72923839,94.53924036)(634.57423854,94.5042404)(634.41423584,94.46424123)
\curveto(634.30423881,94.44424046)(634.19423892,94.42924047)(634.08423584,94.41924123)
\curveto(633.97423914,94.40924049)(633.86423925,94.39424051)(633.75423584,94.37424123)
\curveto(633.7142394,94.36424054)(633.67423944,94.35924054)(633.63423584,94.35924123)
\curveto(633.59423952,94.36924053)(633.55423956,94.36924053)(633.51423584,94.35924123)
\curveto(633.46423965,94.34924055)(633.4142397,94.34424056)(633.36423584,94.34424123)
\lineto(633.19923584,94.34424123)
\curveto(633.14923997,94.32424058)(633.09924002,94.31924058)(633.04923584,94.32924123)
\curveto(632.98924013,94.33924056)(632.93424018,94.33924056)(632.88423584,94.32924123)
\curveto(632.84424027,94.31924058)(632.79924032,94.31924058)(632.74923584,94.32924123)
\curveto(632.69924042,94.33924056)(632.64924047,94.33424057)(632.59923584,94.31424123)
\curveto(632.52924059,94.29424061)(632.45424066,94.28924061)(632.37423584,94.29924123)
\curveto(632.28424083,94.30924059)(632.19924092,94.31424059)(632.11923584,94.31424123)
\curveto(632.02924109,94.31424059)(631.92924119,94.30924059)(631.81923584,94.29924123)
\curveto(631.69924142,94.28924061)(631.59924152,94.29424061)(631.51923584,94.31424123)
\lineto(631.23423584,94.31424123)
\lineto(630.60423584,94.35924123)
\curveto(630.50424261,94.36924053)(630.40924271,94.37924052)(630.31923584,94.38924123)
\lineto(630.01923584,94.41924123)
\curveto(629.96924315,94.43924046)(629.9192432,94.44424046)(629.86923584,94.43424123)
\curveto(629.80924331,94.43424047)(629.75424336,94.44424046)(629.70423584,94.46424123)
\curveto(629.53424358,94.51424039)(629.36924375,94.55424035)(629.20923584,94.58424123)
\curveto(629.03924408,94.61424029)(628.87924424,94.66424024)(628.72923584,94.73424123)
\curveto(628.26924485,94.92423998)(627.89424522,95.14423976)(627.60423584,95.39424123)
\curveto(627.3142458,95.65423925)(627.06924605,96.01423889)(626.86923584,96.47424123)
\curveto(626.8192463,96.6042383)(626.78424633,96.73423817)(626.76423584,96.86424123)
\curveto(626.74424637,97.0042379)(626.7192464,97.14423776)(626.68923584,97.28424123)
\curveto(626.67924644,97.35423755)(626.67424644,97.41923748)(626.67423584,97.47924123)
\curveto(626.67424644,97.53923736)(626.66924645,97.6042373)(626.65923584,97.67424123)
\curveto(626.63924648,98.5042364)(626.78924633,99.17423573)(627.10923584,99.68424123)
\curveto(627.4192457,100.19423471)(627.85924526,100.57423433)(628.42923584,100.82424123)
\curveto(628.54924457,100.87423403)(628.67424444,100.91923398)(628.80423584,100.95924123)
\curveto(628.93424418,100.9992339)(629.06924405,101.04423386)(629.20923584,101.09424123)
\curveto(629.28924383,101.11423379)(629.37424374,101.12923377)(629.46423584,101.13924123)
\lineto(629.70423584,101.19924123)
\curveto(629.8142433,101.22923367)(629.92424319,101.24423366)(630.03423584,101.24424123)
\curveto(630.14424297,101.25423365)(630.25424286,101.26923363)(630.36423584,101.28924123)
\curveto(630.4142427,101.30923359)(630.45924266,101.31423359)(630.49923584,101.30424123)
\curveto(630.53924258,101.3042336)(630.57924254,101.30923359)(630.61923584,101.31924123)
\curveto(630.66924245,101.32923357)(630.72424239,101.32923357)(630.78423584,101.31924123)
\curveto(630.83424228,101.31923358)(630.88424223,101.32423358)(630.93423584,101.33424123)
\lineto(631.06923584,101.33424123)
\curveto(631.12924199,101.35423355)(631.19924192,101.35423355)(631.27923584,101.33424123)
\curveto(631.34924177,101.32423358)(631.4142417,101.32923357)(631.47423584,101.34924123)
\curveto(631.50424161,101.35923354)(631.54424157,101.36423354)(631.59423584,101.36424123)
\lineto(631.71423584,101.36424123)
\lineto(632.17923584,101.36424123)
\moveto(634.50423584,99.81924123)
\curveto(634.18423893,99.91923498)(633.8192393,99.97923492)(633.40923584,99.99924123)
\curveto(632.99924012,100.01923488)(632.58924053,100.02923487)(632.17923584,100.02924123)
\curveto(631.74924137,100.02923487)(631.32924179,100.01923488)(630.91923584,99.99924123)
\curveto(630.50924261,99.97923492)(630.12424299,99.93423497)(629.76423584,99.86424123)
\curveto(629.40424371,99.79423511)(629.08424403,99.68423522)(628.80423584,99.53424123)
\curveto(628.5142446,99.39423551)(628.27924484,99.1992357)(628.09923584,98.94924123)
\curveto(627.98924513,98.78923611)(627.90924521,98.60923629)(627.85923584,98.40924123)
\curveto(627.79924532,98.20923669)(627.76924535,97.96423694)(627.76923584,97.67424123)
\curveto(627.78924533,97.65423725)(627.79924532,97.61923728)(627.79923584,97.56924123)
\curveto(627.78924533,97.51923738)(627.78924533,97.47923742)(627.79923584,97.44924123)
\curveto(627.8192453,97.36923753)(627.83924528,97.29423761)(627.85923584,97.22424123)
\curveto(627.86924525,97.16423774)(627.88924523,97.0992378)(627.91923584,97.02924123)
\curveto(628.03924508,96.75923814)(628.20924491,96.53923836)(628.42923584,96.36924123)
\curveto(628.63924448,96.20923869)(628.88424423,96.07423883)(629.16423584,95.96424123)
\curveto(629.27424384,95.91423899)(629.39424372,95.87423903)(629.52423584,95.84424123)
\curveto(629.64424347,95.82423908)(629.76924335,95.7992391)(629.89923584,95.76924123)
\curveto(629.94924317,95.74923915)(630.00424311,95.73923916)(630.06423584,95.73924123)
\curveto(630.114243,95.73923916)(630.16424295,95.73423917)(630.21423584,95.72424123)
\curveto(630.30424281,95.71423919)(630.39924272,95.7042392)(630.49923584,95.69424123)
\curveto(630.58924253,95.68423922)(630.68424243,95.67423923)(630.78423584,95.66424123)
\curveto(630.86424225,95.66423924)(630.94924217,95.65923924)(631.03923584,95.64924123)
\lineto(631.27923584,95.64924123)
\lineto(631.45923584,95.64924123)
\curveto(631.48924163,95.63923926)(631.52424159,95.63423927)(631.56423584,95.63424123)
\lineto(631.69923584,95.63424123)
\lineto(632.14923584,95.63424123)
\curveto(632.22924089,95.63423927)(632.3142408,95.62923927)(632.40423584,95.61924123)
\curveto(632.48424063,95.61923928)(632.55924056,95.62923927)(632.62923584,95.64924123)
\lineto(632.89923584,95.64924123)
\curveto(632.9192402,95.64923925)(632.94924017,95.64423926)(632.98923584,95.63424123)
\curveto(633.0192401,95.63423927)(633.04424007,95.63923926)(633.06423584,95.64924123)
\curveto(633.16423995,95.65923924)(633.26423985,95.66423924)(633.36423584,95.66424123)
\curveto(633.45423966,95.67423923)(633.55423956,95.68423922)(633.66423584,95.69424123)
\curveto(633.78423933,95.72423918)(633.90923921,95.73923916)(634.03923584,95.73924123)
\curveto(634.15923896,95.74923915)(634.27423884,95.77423913)(634.38423584,95.81424123)
\curveto(634.68423843,95.89423901)(634.94923817,95.97923892)(635.17923584,96.06924123)
\curveto(635.40923771,96.16923873)(635.62423749,96.31423859)(635.82423584,96.50424123)
\curveto(636.02423709,96.71423819)(636.17423694,96.97923792)(636.27423584,97.29924123)
\curveto(636.29423682,97.33923756)(636.30423681,97.37423753)(636.30423584,97.40424123)
\curveto(636.29423682,97.44423746)(636.29923682,97.48923741)(636.31923584,97.53924123)
\curveto(636.32923679,97.57923732)(636.33923678,97.64923725)(636.34923584,97.74924123)
\curveto(636.35923676,97.85923704)(636.35423676,97.94423696)(636.33423584,98.00424123)
\curveto(636.3142368,98.07423683)(636.30423681,98.14423676)(636.30423584,98.21424123)
\curveto(636.29423682,98.28423662)(636.27923684,98.34923655)(636.25923584,98.40924123)
\curveto(636.19923692,98.60923629)(636.114237,98.78923611)(636.00423584,98.94924123)
\curveto(635.98423713,98.97923592)(635.96423715,99.0042359)(635.94423584,99.02424123)
\lineto(635.88423584,99.08424123)
\curveto(635.86423725,99.12423578)(635.82423729,99.17423573)(635.76423584,99.23424123)
\curveto(635.62423749,99.33423557)(635.49423762,99.41923548)(635.37423584,99.48924123)
\curveto(635.25423786,99.55923534)(635.10923801,99.62923527)(634.93923584,99.69924123)
\curveto(634.86923825,99.72923517)(634.79923832,99.74923515)(634.72923584,99.75924123)
\curveto(634.65923846,99.77923512)(634.58423853,99.7992351)(634.50423584,99.81924123)
}
}
{
\newrgbcolor{curcolor}{0 0 0}
\pscustom[linestyle=none,fillstyle=solid,fillcolor=curcolor]
{
\newpath
\moveto(626.65923584,106.77385061)
\curveto(626.65924646,106.87384575)(626.66924645,106.96884566)(626.68923584,107.05885061)
\curveto(626.69924642,107.14884548)(626.72924639,107.21384541)(626.77923584,107.25385061)
\curveto(626.85924626,107.31384531)(626.96424615,107.34384528)(627.09423584,107.34385061)
\lineto(627.48423584,107.34385061)
\lineto(628.98423584,107.34385061)
\lineto(635.37423584,107.34385061)
\lineto(636.54423584,107.34385061)
\lineto(636.85923584,107.34385061)
\curveto(636.95923616,107.35384527)(637.03923608,107.33884529)(637.09923584,107.29885061)
\curveto(637.17923594,107.24884538)(637.22923589,107.17384545)(637.24923584,107.07385061)
\curveto(637.25923586,106.98384564)(637.26423585,106.87384575)(637.26423584,106.74385061)
\lineto(637.26423584,106.51885061)
\curveto(637.24423587,106.43884619)(637.22923589,106.36884626)(637.21923584,106.30885061)
\curveto(637.19923592,106.24884638)(637.15923596,106.19884643)(637.09923584,106.15885061)
\curveto(637.03923608,106.11884651)(636.96423615,106.09884653)(636.87423584,106.09885061)
\lineto(636.57423584,106.09885061)
\lineto(635.47923584,106.09885061)
\lineto(630.13923584,106.09885061)
\curveto(630.04924307,106.07884655)(629.97424314,106.06384656)(629.91423584,106.05385061)
\curveto(629.84424327,106.05384657)(629.78424333,106.0238466)(629.73423584,105.96385061)
\curveto(629.68424343,105.89384673)(629.65924346,105.80384682)(629.65923584,105.69385061)
\curveto(629.64924347,105.59384703)(629.64424347,105.48384714)(629.64423584,105.36385061)
\lineto(629.64423584,104.22385061)
\lineto(629.64423584,103.72885061)
\curveto(629.63424348,103.56884906)(629.57424354,103.45884917)(629.46423584,103.39885061)
\curveto(629.43424368,103.37884925)(629.40424371,103.36884926)(629.37423584,103.36885061)
\curveto(629.33424378,103.36884926)(629.28924383,103.36384926)(629.23923584,103.35385061)
\curveto(629.119244,103.33384929)(629.00924411,103.33884929)(628.90923584,103.36885061)
\curveto(628.80924431,103.40884922)(628.73924438,103.46384916)(628.69923584,103.53385061)
\curveto(628.64924447,103.61384901)(628.62424449,103.73384889)(628.62423584,103.89385061)
\curveto(628.62424449,104.05384857)(628.60924451,104.18884844)(628.57923584,104.29885061)
\curveto(628.56924455,104.34884828)(628.56424455,104.40384822)(628.56423584,104.46385061)
\curveto(628.55424456,104.5238481)(628.53924458,104.58384804)(628.51923584,104.64385061)
\curveto(628.46924465,104.79384783)(628.4192447,104.93884769)(628.36923584,105.07885061)
\curveto(628.30924481,105.21884741)(628.23924488,105.35384727)(628.15923584,105.48385061)
\curveto(628.06924505,105.623847)(627.96424515,105.74384688)(627.84423584,105.84385061)
\curveto(627.72424539,105.94384668)(627.59424552,106.03884659)(627.45423584,106.12885061)
\curveto(627.35424576,106.18884644)(627.24424587,106.23384639)(627.12423584,106.26385061)
\curveto(627.00424611,106.30384632)(626.89924622,106.35384627)(626.80923584,106.41385061)
\curveto(626.74924637,106.46384616)(626.70924641,106.53384609)(626.68923584,106.62385061)
\curveto(626.67924644,106.64384598)(626.67424644,106.66884596)(626.67423584,106.69885061)
\curveto(626.67424644,106.7288459)(626.66924645,106.75384587)(626.65923584,106.77385061)
}
}
{
\newrgbcolor{curcolor}{0 0 0}
\pscustom[linestyle=none,fillstyle=solid,fillcolor=curcolor]
{
\newpath
\moveto(626.65923584,115.12345998)
\curveto(626.65924646,115.22345513)(626.66924645,115.31845503)(626.68923584,115.40845998)
\curveto(626.69924642,115.49845485)(626.72924639,115.56345479)(626.77923584,115.60345998)
\curveto(626.85924626,115.66345469)(626.96424615,115.69345466)(627.09423584,115.69345998)
\lineto(627.48423584,115.69345998)
\lineto(628.98423584,115.69345998)
\lineto(635.37423584,115.69345998)
\lineto(636.54423584,115.69345998)
\lineto(636.85923584,115.69345998)
\curveto(636.95923616,115.70345465)(637.03923608,115.68845466)(637.09923584,115.64845998)
\curveto(637.17923594,115.59845475)(637.22923589,115.52345483)(637.24923584,115.42345998)
\curveto(637.25923586,115.33345502)(637.26423585,115.22345513)(637.26423584,115.09345998)
\lineto(637.26423584,114.86845998)
\curveto(637.24423587,114.78845556)(637.22923589,114.71845563)(637.21923584,114.65845998)
\curveto(637.19923592,114.59845575)(637.15923596,114.5484558)(637.09923584,114.50845998)
\curveto(637.03923608,114.46845588)(636.96423615,114.4484559)(636.87423584,114.44845998)
\lineto(636.57423584,114.44845998)
\lineto(635.47923584,114.44845998)
\lineto(630.13923584,114.44845998)
\curveto(630.04924307,114.42845592)(629.97424314,114.41345594)(629.91423584,114.40345998)
\curveto(629.84424327,114.40345595)(629.78424333,114.37345598)(629.73423584,114.31345998)
\curveto(629.68424343,114.24345611)(629.65924346,114.1534562)(629.65923584,114.04345998)
\curveto(629.64924347,113.94345641)(629.64424347,113.83345652)(629.64423584,113.71345998)
\lineto(629.64423584,112.57345998)
\lineto(629.64423584,112.07845998)
\curveto(629.63424348,111.91845843)(629.57424354,111.80845854)(629.46423584,111.74845998)
\curveto(629.43424368,111.72845862)(629.40424371,111.71845863)(629.37423584,111.71845998)
\curveto(629.33424378,111.71845863)(629.28924383,111.71345864)(629.23923584,111.70345998)
\curveto(629.119244,111.68345867)(629.00924411,111.68845866)(628.90923584,111.71845998)
\curveto(628.80924431,111.75845859)(628.73924438,111.81345854)(628.69923584,111.88345998)
\curveto(628.64924447,111.96345839)(628.62424449,112.08345827)(628.62423584,112.24345998)
\curveto(628.62424449,112.40345795)(628.60924451,112.53845781)(628.57923584,112.64845998)
\curveto(628.56924455,112.69845765)(628.56424455,112.7534576)(628.56423584,112.81345998)
\curveto(628.55424456,112.87345748)(628.53924458,112.93345742)(628.51923584,112.99345998)
\curveto(628.46924465,113.14345721)(628.4192447,113.28845706)(628.36923584,113.42845998)
\curveto(628.30924481,113.56845678)(628.23924488,113.70345665)(628.15923584,113.83345998)
\curveto(628.06924505,113.97345638)(627.96424515,114.09345626)(627.84423584,114.19345998)
\curveto(627.72424539,114.29345606)(627.59424552,114.38845596)(627.45423584,114.47845998)
\curveto(627.35424576,114.53845581)(627.24424587,114.58345577)(627.12423584,114.61345998)
\curveto(627.00424611,114.6534557)(626.89924622,114.70345565)(626.80923584,114.76345998)
\curveto(626.74924637,114.81345554)(626.70924641,114.88345547)(626.68923584,114.97345998)
\curveto(626.67924644,114.99345536)(626.67424644,115.01845533)(626.67423584,115.04845998)
\curveto(626.67424644,115.07845527)(626.66924645,115.10345525)(626.65923584,115.12345998)
}
}
{
\newrgbcolor{curcolor}{0 0 0}
\pscustom[linestyle=none,fillstyle=solid,fillcolor=curcolor]
{
\newpath
\moveto(648.5305249,37.28705373)
\curveto(648.5305356,37.35704805)(648.5305356,37.43704797)(648.5305249,37.52705373)
\curveto(648.52053561,37.61704779)(648.52053561,37.70204771)(648.5305249,37.78205373)
\curveto(648.5305356,37.87204754)(648.54053559,37.95204746)(648.5605249,38.02205373)
\curveto(648.58053555,38.10204731)(648.61053552,38.15704725)(648.6505249,38.18705373)
\curveto(648.70053543,38.21704719)(648.77553535,38.23704717)(648.8755249,38.24705373)
\curveto(648.96553516,38.26704714)(649.07053506,38.27704713)(649.1905249,38.27705373)
\curveto(649.30053483,38.28704712)(649.41553471,38.28704712)(649.5355249,38.27705373)
\lineto(649.8355249,38.27705373)
\lineto(652.8505249,38.27705373)
\lineto(655.7455249,38.27705373)
\curveto(656.07552805,38.27704713)(656.40052773,38.27204714)(656.7205249,38.26205373)
\curveto(657.0305271,38.26204715)(657.31052682,38.22204719)(657.5605249,38.14205373)
\curveto(657.91052622,38.02204739)(658.20552592,37.86704754)(658.4455249,37.67705373)
\curveto(658.67552545,37.48704792)(658.87552525,37.24704816)(659.0455249,36.95705373)
\curveto(659.09552503,36.89704851)(659.130525,36.83204858)(659.1505249,36.76205373)
\curveto(659.17052496,36.70204871)(659.19552493,36.63204878)(659.2255249,36.55205373)
\curveto(659.27552485,36.43204898)(659.31052482,36.30204911)(659.3305249,36.16205373)
\curveto(659.36052477,36.03204938)(659.39052474,35.89704951)(659.4205249,35.75705373)
\curveto(659.44052469,35.7070497)(659.44552468,35.65704975)(659.4355249,35.60705373)
\curveto(659.4255247,35.55704985)(659.4255247,35.50204991)(659.4355249,35.44205373)
\curveto(659.44552468,35.42204999)(659.44552468,35.39705001)(659.4355249,35.36705373)
\curveto(659.43552469,35.33705007)(659.44052469,35.3120501)(659.4505249,35.29205373)
\curveto(659.46052467,35.25205016)(659.46552466,35.19705021)(659.4655249,35.12705373)
\curveto(659.46552466,35.05705035)(659.46052467,35.00205041)(659.4505249,34.96205373)
\curveto(659.44052469,34.9120505)(659.44052469,34.85705055)(659.4505249,34.79705373)
\curveto(659.46052467,34.73705067)(659.45552467,34.68205073)(659.4355249,34.63205373)
\curveto(659.40552472,34.50205091)(659.38552474,34.37705103)(659.3755249,34.25705373)
\curveto(659.36552476,34.13705127)(659.34052479,34.02205139)(659.3005249,33.91205373)
\curveto(659.18052495,33.54205187)(659.01052512,33.22205219)(658.7905249,32.95205373)
\curveto(658.57052556,32.68205273)(658.29052584,32.47205294)(657.9505249,32.32205373)
\curveto(657.8305263,32.27205314)(657.70552642,32.22705318)(657.5755249,32.18705373)
\curveto(657.44552668,32.15705325)(657.31052682,32.12205329)(657.1705249,32.08205373)
\curveto(657.12052701,32.07205334)(657.08052705,32.06705334)(657.0505249,32.06705373)
\curveto(657.01052712,32.06705334)(656.96552716,32.06205335)(656.9155249,32.05205373)
\curveto(656.88552724,32.04205337)(656.85052728,32.03705337)(656.8105249,32.03705373)
\curveto(656.76052737,32.03705337)(656.72052741,32.03205338)(656.6905249,32.02205373)
\lineto(656.5255249,32.02205373)
\curveto(656.44552768,32.00205341)(656.34552778,31.99705341)(656.2255249,32.00705373)
\curveto(656.09552803,32.01705339)(656.00552812,32.03205338)(655.9555249,32.05205373)
\curveto(655.86552826,32.07205334)(655.80052833,32.12705328)(655.7605249,32.21705373)
\curveto(655.74052839,32.24705316)(655.73552839,32.27705313)(655.7455249,32.30705373)
\curveto(655.74552838,32.33705307)(655.74052839,32.37705303)(655.7305249,32.42705373)
\curveto(655.72052841,32.46705294)(655.71552841,32.5070529)(655.7155249,32.54705373)
\lineto(655.7155249,32.69705373)
\curveto(655.71552841,32.81705259)(655.72052841,32.93705247)(655.7305249,33.05705373)
\curveto(655.7305284,33.18705222)(655.76552836,33.27705213)(655.8355249,33.32705373)
\curveto(655.89552823,33.36705204)(655.95552817,33.38705202)(656.0155249,33.38705373)
\curveto(656.07552805,33.38705202)(656.14552798,33.39705201)(656.2255249,33.41705373)
\curveto(656.25552787,33.42705198)(656.29052784,33.42705198)(656.3305249,33.41705373)
\curveto(656.36052777,33.41705199)(656.38552774,33.42205199)(656.4055249,33.43205373)
\lineto(656.6155249,33.43205373)
\curveto(656.66552746,33.45205196)(656.71552741,33.45705195)(656.7655249,33.44705373)
\curveto(656.80552732,33.44705196)(656.85052728,33.45705195)(656.9005249,33.47705373)
\curveto(657.0305271,33.5070519)(657.15552697,33.53705187)(657.2755249,33.56705373)
\curveto(657.38552674,33.59705181)(657.49052664,33.64205177)(657.5905249,33.70205373)
\curveto(657.88052625,33.87205154)(658.08552604,34.14205127)(658.2055249,34.51205373)
\curveto(658.2255259,34.56205085)(658.24052589,34.6120508)(658.2505249,34.66205373)
\curveto(658.25052588,34.72205069)(658.26052587,34.77705063)(658.2805249,34.82705373)
\lineto(658.2805249,34.90205373)
\curveto(658.29052584,34.97205044)(658.30052583,35.06705034)(658.3105249,35.18705373)
\curveto(658.31052582,35.31705009)(658.30052583,35.41704999)(658.2805249,35.48705373)
\curveto(658.26052587,35.55704985)(658.24552588,35.62704978)(658.2355249,35.69705373)
\curveto(658.21552591,35.77704963)(658.19552593,35.84704956)(658.1755249,35.90705373)
\curveto(658.01552611,36.28704912)(657.74052639,36.56204885)(657.3505249,36.73205373)
\curveto(657.22052691,36.78204863)(657.06552706,36.81704859)(656.8855249,36.83705373)
\curveto(656.70552742,36.86704854)(656.52052761,36.88204853)(656.3305249,36.88205373)
\curveto(656.130528,36.89204852)(655.9305282,36.89204852)(655.7305249,36.88205373)
\lineto(655.1605249,36.88205373)
\lineto(650.9155249,36.88205373)
\lineto(649.3705249,36.88205373)
\curveto(649.26053487,36.88204853)(649.14053499,36.87704853)(649.0105249,36.86705373)
\curveto(648.88053525,36.85704855)(648.77553535,36.87704853)(648.6955249,36.92705373)
\curveto(648.6255355,36.98704842)(648.57553555,37.06704834)(648.5455249,37.16705373)
\curveto(648.54553558,37.18704822)(648.54553558,37.2070482)(648.5455249,37.22705373)
\curveto(648.54553558,37.24704816)(648.54053559,37.26704814)(648.5305249,37.28705373)
}
}
{
\newrgbcolor{curcolor}{0 0 0}
\pscustom[linestyle=none,fillstyle=solid,fillcolor=curcolor]
{
\newpath
\moveto(651.4855249,40.82072561)
\lineto(651.4855249,41.25572561)
\curveto(651.48553264,41.40572364)(651.5255326,41.51072354)(651.6055249,41.57072561)
\curveto(651.68553244,41.62072343)(651.78553234,41.6457234)(651.9055249,41.64572561)
\curveto(652.0255321,41.65572339)(652.14553198,41.66072339)(652.2655249,41.66072561)
\lineto(653.6905249,41.66072561)
\lineto(655.9555249,41.66072561)
\lineto(656.6455249,41.66072561)
\curveto(656.87552725,41.66072339)(657.07552705,41.68572336)(657.2455249,41.73572561)
\curveto(657.69552643,41.89572315)(658.01052612,42.19572285)(658.1905249,42.63572561)
\curveto(658.28052585,42.85572219)(658.31552581,43.12072193)(658.2955249,43.43072561)
\curveto(658.26552586,43.74072131)(658.21052592,43.99072106)(658.1305249,44.18072561)
\curveto(657.99052614,44.51072054)(657.81552631,44.77072028)(657.6055249,44.96072561)
\curveto(657.38552674,45.16071989)(657.10052703,45.31571973)(656.7505249,45.42572561)
\curveto(656.67052746,45.45571959)(656.59052754,45.47571957)(656.5105249,45.48572561)
\curveto(656.4305277,45.49571955)(656.34552778,45.51071954)(656.2555249,45.53072561)
\curveto(656.20552792,45.54071951)(656.16052797,45.54071951)(656.1205249,45.53072561)
\curveto(656.08052805,45.53071952)(656.03552809,45.54071951)(655.9855249,45.56072561)
\lineto(655.6705249,45.56072561)
\curveto(655.59052854,45.58071947)(655.50052863,45.58571946)(655.4005249,45.57572561)
\curveto(655.29052884,45.56571948)(655.19052894,45.56071949)(655.1005249,45.56072561)
\lineto(653.9305249,45.56072561)
\lineto(652.3405249,45.56072561)
\curveto(652.22053191,45.56071949)(652.09553203,45.55571949)(651.9655249,45.54572561)
\curveto(651.8255323,45.5457195)(651.71553241,45.57071948)(651.6355249,45.62072561)
\curveto(651.58553254,45.66071939)(651.55553257,45.70571934)(651.5455249,45.75572561)
\curveto(651.5255326,45.81571923)(651.50553262,45.88571916)(651.4855249,45.96572561)
\lineto(651.4855249,46.19072561)
\curveto(651.48553264,46.31071874)(651.49053264,46.41571863)(651.5005249,46.50572561)
\curveto(651.51053262,46.60571844)(651.55553257,46.68071837)(651.6355249,46.73072561)
\curveto(651.68553244,46.78071827)(651.76053237,46.80571824)(651.8605249,46.80572561)
\lineto(652.1455249,46.80572561)
\lineto(653.1655249,46.80572561)
\lineto(657.2005249,46.80572561)
\lineto(658.5505249,46.80572561)
\curveto(658.67052546,46.80571824)(658.78552534,46.80071825)(658.8955249,46.79072561)
\curveto(658.99552513,46.79071826)(659.07052506,46.75571829)(659.1205249,46.68572561)
\curveto(659.15052498,46.6457184)(659.17552495,46.58571846)(659.1955249,46.50572561)
\curveto(659.20552492,46.42571862)(659.21552491,46.33571871)(659.2255249,46.23572561)
\curveto(659.2255249,46.1457189)(659.22052491,46.05571899)(659.2105249,45.96572561)
\curveto(659.20052493,45.88571916)(659.18052495,45.82571922)(659.1505249,45.78572561)
\curveto(659.11052502,45.73571931)(659.04552508,45.69071936)(658.9555249,45.65072561)
\curveto(658.91552521,45.64071941)(658.86052527,45.63071942)(658.7905249,45.62072561)
\curveto(658.72052541,45.62071943)(658.65552547,45.61571943)(658.5955249,45.60572561)
\curveto(658.5255256,45.59571945)(658.47052566,45.57571947)(658.4305249,45.54572561)
\curveto(658.39052574,45.51571953)(658.37552575,45.47071958)(658.3855249,45.41072561)
\curveto(658.40552572,45.33071972)(658.46552566,45.2507198)(658.5655249,45.17072561)
\curveto(658.65552547,45.09071996)(658.7255254,45.01572003)(658.7755249,44.94572561)
\curveto(658.93552519,44.72572032)(659.07552505,44.47572057)(659.1955249,44.19572561)
\curveto(659.24552488,44.08572096)(659.27552485,43.97072108)(659.2855249,43.85072561)
\curveto(659.30552482,43.74072131)(659.3305248,43.62572142)(659.3605249,43.50572561)
\curveto(659.37052476,43.45572159)(659.37052476,43.40072165)(659.3605249,43.34072561)
\curveto(659.35052478,43.29072176)(659.35552477,43.24072181)(659.3755249,43.19072561)
\curveto(659.39552473,43.09072196)(659.39552473,43.00072205)(659.3755249,42.92072561)
\lineto(659.3755249,42.77072561)
\curveto(659.35552477,42.72072233)(659.34552478,42.66072239)(659.3455249,42.59072561)
\curveto(659.34552478,42.53072252)(659.34052479,42.47572257)(659.3305249,42.42572561)
\curveto(659.31052482,42.38572266)(659.30052483,42.3457227)(659.3005249,42.30572561)
\curveto(659.31052482,42.27572277)(659.30552482,42.23572281)(659.2855249,42.18572561)
\lineto(659.2255249,41.94572561)
\curveto(659.20552492,41.87572317)(659.17552495,41.80072325)(659.1355249,41.72072561)
\curveto(659.0255251,41.46072359)(658.88052525,41.24072381)(658.7005249,41.06072561)
\curveto(658.51052562,40.89072416)(658.28552584,40.7507243)(658.0255249,40.64072561)
\curveto(657.93552619,40.60072445)(657.84552628,40.57072448)(657.7555249,40.55072561)
\lineto(657.4555249,40.49072561)
\curveto(657.39552673,40.47072458)(657.34052679,40.46072459)(657.2905249,40.46072561)
\curveto(657.2305269,40.47072458)(657.16552696,40.46572458)(657.0955249,40.44572561)
\curveto(657.07552705,40.43572461)(657.05052708,40.43072462)(657.0205249,40.43072561)
\curveto(656.98052715,40.43072462)(656.94552718,40.42572462)(656.9155249,40.41572561)
\lineto(656.7655249,40.41572561)
\curveto(656.7255274,40.40572464)(656.68052745,40.40072465)(656.6305249,40.40072561)
\curveto(656.57052756,40.41072464)(656.51552761,40.41572463)(656.4655249,40.41572561)
\lineto(655.8655249,40.41572561)
\lineto(653.1055249,40.41572561)
\lineto(652.1455249,40.41572561)
\lineto(651.8755249,40.41572561)
\curveto(651.78553234,40.41572463)(651.71053242,40.43572461)(651.6505249,40.47572561)
\curveto(651.58053255,40.51572453)(651.5305326,40.59072446)(651.5005249,40.70072561)
\curveto(651.49053264,40.72072433)(651.49053264,40.74072431)(651.5005249,40.76072561)
\curveto(651.50053263,40.78072427)(651.49553263,40.80072425)(651.4855249,40.82072561)
}
}
{
\newrgbcolor{curcolor}{0 0 0}
\pscustom[linestyle=none,fillstyle=solid,fillcolor=curcolor]
{
\newpath
\moveto(651.3355249,52.39533498)
\curveto(651.31553281,53.02532975)(651.40053273,53.53032924)(651.5905249,53.91033498)
\curveto(651.78053235,54.29032848)(652.06553206,54.59532818)(652.4455249,54.82533498)
\curveto(652.54553158,54.88532789)(652.65553147,54.93032784)(652.7755249,54.96033498)
\curveto(652.88553124,55.00032777)(653.00053113,55.03532774)(653.1205249,55.06533498)
\curveto(653.31053082,55.11532766)(653.51553061,55.14532763)(653.7355249,55.15533498)
\curveto(653.95553017,55.16532761)(654.18052995,55.1703276)(654.4105249,55.17033498)
\lineto(656.0155249,55.17033498)
\lineto(658.3555249,55.17033498)
\curveto(658.5255256,55.1703276)(658.69552543,55.16532761)(658.8655249,55.15533498)
\curveto(659.03552509,55.15532762)(659.14552498,55.09032768)(659.1955249,54.96033498)
\curveto(659.21552491,54.91032786)(659.2255249,54.85532792)(659.2255249,54.79533498)
\curveto(659.23552489,54.74532803)(659.24052489,54.69032808)(659.2405249,54.63033498)
\curveto(659.24052489,54.50032827)(659.23552489,54.3753284)(659.2255249,54.25533498)
\curveto(659.2255249,54.13532864)(659.18552494,54.05032872)(659.1055249,54.00033498)
\curveto(659.03552509,53.95032882)(658.94552518,53.92532885)(658.8355249,53.92533498)
\lineto(658.5055249,53.92533498)
\lineto(657.2155249,53.92533498)
\lineto(654.7705249,53.92533498)
\curveto(654.50052963,53.92532885)(654.23552989,53.92032885)(653.9755249,53.91033498)
\curveto(653.70553042,53.90032887)(653.47553065,53.85532892)(653.2855249,53.77533498)
\curveto(653.08553104,53.69532908)(652.9255312,53.5753292)(652.8055249,53.41533498)
\curveto(652.67553145,53.25532952)(652.57553155,53.0703297)(652.5055249,52.86033498)
\curveto(652.48553164,52.80032997)(652.47553165,52.73533004)(652.4755249,52.66533498)
\curveto(652.46553166,52.60533017)(652.45053168,52.54533023)(652.4305249,52.48533498)
\curveto(652.42053171,52.43533034)(652.42053171,52.35533042)(652.4305249,52.24533498)
\curveto(652.4305317,52.14533063)(652.43553169,52.0753307)(652.4455249,52.03533498)
\curveto(652.46553166,51.99533078)(652.47553165,51.96033081)(652.4755249,51.93033498)
\curveto(652.46553166,51.90033087)(652.46553166,51.86533091)(652.4755249,51.82533498)
\curveto(652.50553162,51.69533108)(652.54053159,51.5703312)(652.5805249,51.45033498)
\curveto(652.61053152,51.34033143)(652.65553147,51.23533154)(652.7155249,51.13533498)
\curveto(652.73553139,51.09533168)(652.75553137,51.06033171)(652.7755249,51.03033498)
\curveto(652.79553133,51.00033177)(652.81553131,50.96533181)(652.8355249,50.92533498)
\curveto(653.08553104,50.5753322)(653.46053067,50.32033245)(653.9605249,50.16033498)
\curveto(654.04053009,50.13033264)(654.12553,50.11033266)(654.2155249,50.10033498)
\curveto(654.29552983,50.09033268)(654.37552975,50.0753327)(654.4555249,50.05533498)
\curveto(654.50552962,50.03533274)(654.55552957,50.03033274)(654.6055249,50.04033498)
\curveto(654.64552948,50.05033272)(654.68552944,50.04533273)(654.7255249,50.02533498)
\lineto(655.0405249,50.02533498)
\curveto(655.07052906,50.01533276)(655.10552902,50.01033276)(655.1455249,50.01033498)
\curveto(655.18552894,50.02033275)(655.2305289,50.02533275)(655.2805249,50.02533498)
\lineto(655.7305249,50.02533498)
\lineto(657.1705249,50.02533498)
\lineto(658.4905249,50.02533498)
\lineto(658.8355249,50.02533498)
\curveto(658.94552518,50.02533275)(659.03552509,50.00033277)(659.1055249,49.95033498)
\curveto(659.18552494,49.90033287)(659.2255249,49.81033296)(659.2255249,49.68033498)
\curveto(659.23552489,49.56033321)(659.24052489,49.43533334)(659.2405249,49.30533498)
\curveto(659.24052489,49.22533355)(659.23552489,49.15033362)(659.2255249,49.08033498)
\curveto(659.21552491,49.01033376)(659.19052494,48.95033382)(659.1505249,48.90033498)
\curveto(659.10052503,48.82033395)(659.00552512,48.78033399)(658.8655249,48.78033498)
\lineto(658.4605249,48.78033498)
\lineto(656.6905249,48.78033498)
\lineto(653.0605249,48.78033498)
\lineto(652.1455249,48.78033498)
\lineto(651.8755249,48.78033498)
\curveto(651.78553234,48.78033399)(651.71553241,48.80033397)(651.6655249,48.84033498)
\curveto(651.60553252,48.8703339)(651.56553256,48.92033385)(651.5455249,48.99033498)
\curveto(651.53553259,49.03033374)(651.5255326,49.08533369)(651.5155249,49.15533498)
\curveto(651.50553262,49.23533354)(651.50053263,49.31533346)(651.5005249,49.39533498)
\curveto(651.50053263,49.4753333)(651.50553262,49.55033322)(651.5155249,49.62033498)
\curveto(651.5255326,49.70033307)(651.54053259,49.75533302)(651.5605249,49.78533498)
\curveto(651.6305325,49.89533288)(651.72053241,49.94533283)(651.8305249,49.93533498)
\curveto(651.9305322,49.92533285)(652.04553208,49.94033283)(652.1755249,49.98033498)
\curveto(652.23553189,50.00033277)(652.28553184,50.04033273)(652.3255249,50.10033498)
\curveto(652.33553179,50.22033255)(652.29053184,50.31533246)(652.1905249,50.38533498)
\curveto(652.09053204,50.46533231)(652.01053212,50.54533223)(651.9505249,50.62533498)
\curveto(651.85053228,50.76533201)(651.76053237,50.90533187)(651.6805249,51.04533498)
\curveto(651.59053254,51.19533158)(651.51553261,51.36533141)(651.4555249,51.55533498)
\curveto(651.4255327,51.63533114)(651.40553272,51.72033105)(651.3955249,51.81033498)
\curveto(651.38553274,51.91033086)(651.37053276,52.00533077)(651.3505249,52.09533498)
\curveto(651.34053279,52.14533063)(651.33553279,52.19533058)(651.3355249,52.24533498)
\lineto(651.3355249,52.39533498)
}
}
{
\newrgbcolor{curcolor}{0 0 0}
\pscustom[linestyle=none,fillstyle=solid,fillcolor=curcolor]
{
}
}
{
\newrgbcolor{curcolor}{0 0 0}
\pscustom[linestyle=none,fillstyle=solid,fillcolor=curcolor]
{
\newpath
\moveto(654.1255249,67.99510061)
\lineto(654.3805249,67.99510061)
\curveto(654.46052967,68.0050929)(654.53552959,68.00009291)(654.6055249,67.98010061)
\lineto(654.8455249,67.98010061)
\lineto(655.0105249,67.98010061)
\curveto(655.11052902,67.96009295)(655.21552891,67.95009296)(655.3255249,67.95010061)
\curveto(655.4255287,67.95009296)(655.5255286,67.94009297)(655.6255249,67.92010061)
\lineto(655.7755249,67.92010061)
\curveto(655.91552821,67.89009302)(656.05552807,67.87009304)(656.1955249,67.86010061)
\curveto(656.3255278,67.85009306)(656.45552767,67.82509308)(656.5855249,67.78510061)
\curveto(656.66552746,67.76509314)(656.75052738,67.74509316)(656.8405249,67.72510061)
\lineto(657.0805249,67.66510061)
\lineto(657.3805249,67.54510061)
\curveto(657.47052666,67.51509339)(657.56052657,67.48009343)(657.6505249,67.44010061)
\curveto(657.87052626,67.34009357)(658.08552604,67.2050937)(658.2955249,67.03510061)
\curveto(658.50552562,66.87509403)(658.67552545,66.70009421)(658.8055249,66.51010061)
\curveto(658.84552528,66.46009445)(658.88552524,66.40009451)(658.9255249,66.33010061)
\curveto(658.95552517,66.27009464)(658.99052514,66.2100947)(659.0305249,66.15010061)
\curveto(659.08052505,66.07009484)(659.12052501,65.97509493)(659.1505249,65.86510061)
\curveto(659.18052495,65.75509515)(659.21052492,65.65009526)(659.2405249,65.55010061)
\curveto(659.28052485,65.44009547)(659.30552482,65.33009558)(659.3155249,65.22010061)
\curveto(659.3255248,65.1100958)(659.34052479,64.99509591)(659.3605249,64.87510061)
\curveto(659.37052476,64.83509607)(659.37052476,64.79009612)(659.3605249,64.74010061)
\curveto(659.36052477,64.70009621)(659.36552476,64.66009625)(659.3755249,64.62010061)
\curveto(659.38552474,64.58009633)(659.39052474,64.52509638)(659.3905249,64.45510061)
\curveto(659.39052474,64.38509652)(659.38552474,64.33509657)(659.3755249,64.30510061)
\curveto(659.35552477,64.25509665)(659.35052478,64.2100967)(659.3605249,64.17010061)
\curveto(659.37052476,64.13009678)(659.37052476,64.09509681)(659.3605249,64.06510061)
\lineto(659.3605249,63.97510061)
\curveto(659.34052479,63.91509699)(659.3255248,63.85009706)(659.3155249,63.78010061)
\curveto(659.31552481,63.72009719)(659.31052482,63.65509725)(659.3005249,63.58510061)
\curveto(659.25052488,63.41509749)(659.20052493,63.25509765)(659.1505249,63.10510061)
\curveto(659.10052503,62.95509795)(659.03552509,62.8100981)(658.9555249,62.67010061)
\curveto(658.91552521,62.62009829)(658.88552524,62.56509834)(658.8655249,62.50510061)
\curveto(658.83552529,62.45509845)(658.80052533,62.4050985)(658.7605249,62.35510061)
\curveto(658.58052555,62.11509879)(658.36052577,61.91509899)(658.1005249,61.75510061)
\curveto(657.84052629,61.59509931)(657.55552657,61.45509945)(657.2455249,61.33510061)
\curveto(657.10552702,61.27509963)(656.96552716,61.23009968)(656.8255249,61.20010061)
\curveto(656.67552745,61.17009974)(656.52052761,61.13509977)(656.3605249,61.09510061)
\curveto(656.25052788,61.07509983)(656.14052799,61.06009985)(656.0305249,61.05010061)
\curveto(655.92052821,61.04009987)(655.81052832,61.02509988)(655.7005249,61.00510061)
\curveto(655.66052847,60.99509991)(655.62052851,60.99009992)(655.5805249,60.99010061)
\curveto(655.54052859,61.00009991)(655.50052863,61.00009991)(655.4605249,60.99010061)
\curveto(655.41052872,60.98009993)(655.36052877,60.97509993)(655.3105249,60.97510061)
\lineto(655.1455249,60.97510061)
\curveto(655.09552903,60.95509995)(655.04552908,60.95009996)(654.9955249,60.96010061)
\curveto(654.93552919,60.97009994)(654.88052925,60.97009994)(654.8305249,60.96010061)
\curveto(654.79052934,60.95009996)(654.74552938,60.95009996)(654.6955249,60.96010061)
\curveto(654.64552948,60.97009994)(654.59552953,60.96509994)(654.5455249,60.94510061)
\curveto(654.47552965,60.92509998)(654.40052973,60.92009999)(654.3205249,60.93010061)
\curveto(654.2305299,60.94009997)(654.14552998,60.94509996)(654.0655249,60.94510061)
\curveto(653.97553015,60.94509996)(653.87553025,60.94009997)(653.7655249,60.93010061)
\curveto(653.64553048,60.92009999)(653.54553058,60.92509998)(653.4655249,60.94510061)
\lineto(653.1805249,60.94510061)
\lineto(652.5505249,60.99010061)
\curveto(652.45053168,61.00009991)(652.35553177,61.0100999)(652.2655249,61.02010061)
\lineto(651.9655249,61.05010061)
\curveto(651.91553221,61.07009984)(651.86553226,61.07509983)(651.8155249,61.06510061)
\curveto(651.75553237,61.06509984)(651.70053243,61.07509983)(651.6505249,61.09510061)
\curveto(651.48053265,61.14509976)(651.31553281,61.18509972)(651.1555249,61.21510061)
\curveto(650.98553314,61.24509966)(650.8255333,61.29509961)(650.6755249,61.36510061)
\curveto(650.21553391,61.55509935)(649.84053429,61.77509913)(649.5505249,62.02510061)
\curveto(649.26053487,62.28509862)(649.01553511,62.64509826)(648.8155249,63.10510061)
\curveto(648.76553536,63.23509767)(648.7305354,63.36509754)(648.7105249,63.49510061)
\curveto(648.69053544,63.63509727)(648.66553546,63.77509713)(648.6355249,63.91510061)
\curveto(648.6255355,63.98509692)(648.62053551,64.05009686)(648.6205249,64.11010061)
\curveto(648.62053551,64.17009674)(648.61553551,64.23509667)(648.6055249,64.30510061)
\curveto(648.58553554,65.13509577)(648.73553539,65.8050951)(649.0555249,66.31510061)
\curveto(649.36553476,66.82509408)(649.80553432,67.2050937)(650.3755249,67.45510061)
\curveto(650.49553363,67.5050934)(650.62053351,67.55009336)(650.7505249,67.59010061)
\curveto(650.88053325,67.63009328)(651.01553311,67.67509323)(651.1555249,67.72510061)
\curveto(651.23553289,67.74509316)(651.32053281,67.76009315)(651.4105249,67.77010061)
\lineto(651.6505249,67.83010061)
\curveto(651.76053237,67.86009305)(651.87053226,67.87509303)(651.9805249,67.87510061)
\curveto(652.09053204,67.88509302)(652.20053193,67.90009301)(652.3105249,67.92010061)
\curveto(652.36053177,67.94009297)(652.40553172,67.94509296)(652.4455249,67.93510061)
\curveto(652.48553164,67.93509297)(652.5255316,67.94009297)(652.5655249,67.95010061)
\curveto(652.61553151,67.96009295)(652.67053146,67.96009295)(652.7305249,67.95010061)
\curveto(652.78053135,67.95009296)(652.8305313,67.95509295)(652.8805249,67.96510061)
\lineto(653.0155249,67.96510061)
\curveto(653.07553105,67.98509292)(653.14553098,67.98509292)(653.2255249,67.96510061)
\curveto(653.29553083,67.95509295)(653.36053077,67.96009295)(653.4205249,67.98010061)
\curveto(653.45053068,67.99009292)(653.49053064,67.99509291)(653.5405249,67.99510061)
\lineto(653.6605249,67.99510061)
\lineto(654.1255249,67.99510061)
\moveto(656.4505249,66.45010061)
\curveto(656.130528,66.55009436)(655.76552836,66.6100943)(655.3555249,66.63010061)
\curveto(654.94552918,66.65009426)(654.53552959,66.66009425)(654.1255249,66.66010061)
\curveto(653.69553043,66.66009425)(653.27553085,66.65009426)(652.8655249,66.63010061)
\curveto(652.45553167,66.6100943)(652.07053206,66.56509434)(651.7105249,66.49510061)
\curveto(651.35053278,66.42509448)(651.0305331,66.31509459)(650.7505249,66.16510061)
\curveto(650.46053367,66.02509488)(650.2255339,65.83009508)(650.0455249,65.58010061)
\curveto(649.93553419,65.42009549)(649.85553427,65.24009567)(649.8055249,65.04010061)
\curveto(649.74553438,64.84009607)(649.71553441,64.59509631)(649.7155249,64.30510061)
\curveto(649.73553439,64.28509662)(649.74553438,64.25009666)(649.7455249,64.20010061)
\curveto(649.73553439,64.15009676)(649.73553439,64.1100968)(649.7455249,64.08010061)
\curveto(649.76553436,64.00009691)(649.78553434,63.92509698)(649.8055249,63.85510061)
\curveto(649.81553431,63.79509711)(649.83553429,63.73009718)(649.8655249,63.66010061)
\curveto(649.98553414,63.39009752)(650.15553397,63.17009774)(650.3755249,63.00010061)
\curveto(650.58553354,62.84009807)(650.8305333,62.7050982)(651.1105249,62.59510061)
\curveto(651.22053291,62.54509836)(651.34053279,62.5050984)(651.4705249,62.47510061)
\curveto(651.59053254,62.45509845)(651.71553241,62.43009848)(651.8455249,62.40010061)
\curveto(651.89553223,62.38009853)(651.95053218,62.37009854)(652.0105249,62.37010061)
\curveto(652.06053207,62.37009854)(652.11053202,62.36509854)(652.1605249,62.35510061)
\curveto(652.25053188,62.34509856)(652.34553178,62.33509857)(652.4455249,62.32510061)
\curveto(652.53553159,62.31509859)(652.6305315,62.3050986)(652.7305249,62.29510061)
\curveto(652.81053132,62.29509861)(652.89553123,62.29009862)(652.9855249,62.28010061)
\lineto(653.2255249,62.28010061)
\lineto(653.4055249,62.28010061)
\curveto(653.43553069,62.27009864)(653.47053066,62.26509864)(653.5105249,62.26510061)
\lineto(653.6455249,62.26510061)
\lineto(654.0955249,62.26510061)
\curveto(654.17552995,62.26509864)(654.26052987,62.26009865)(654.3505249,62.25010061)
\curveto(654.4305297,62.25009866)(654.50552962,62.26009865)(654.5755249,62.28010061)
\lineto(654.8455249,62.28010061)
\curveto(654.86552926,62.28009863)(654.89552923,62.27509863)(654.9355249,62.26510061)
\curveto(654.96552916,62.26509864)(654.99052914,62.27009864)(655.0105249,62.28010061)
\curveto(655.11052902,62.29009862)(655.21052892,62.29509861)(655.3105249,62.29510061)
\curveto(655.40052873,62.3050986)(655.50052863,62.31509859)(655.6105249,62.32510061)
\curveto(655.7305284,62.35509855)(655.85552827,62.37009854)(655.9855249,62.37010061)
\curveto(656.10552802,62.38009853)(656.22052791,62.4050985)(656.3305249,62.44510061)
\curveto(656.6305275,62.52509838)(656.89552723,62.6100983)(657.1255249,62.70010061)
\curveto(657.35552677,62.80009811)(657.57052656,62.94509796)(657.7705249,63.13510061)
\curveto(657.97052616,63.34509756)(658.12052601,63.6100973)(658.2205249,63.93010061)
\curveto(658.24052589,63.97009694)(658.25052588,64.0050969)(658.2505249,64.03510061)
\curveto(658.24052589,64.07509683)(658.24552588,64.12009679)(658.2655249,64.17010061)
\curveto(658.27552585,64.2100967)(658.28552584,64.28009663)(658.2955249,64.38010061)
\curveto(658.30552582,64.49009642)(658.30052583,64.57509633)(658.2805249,64.63510061)
\curveto(658.26052587,64.7050962)(658.25052588,64.77509613)(658.2505249,64.84510061)
\curveto(658.24052589,64.91509599)(658.2255259,64.98009593)(658.2055249,65.04010061)
\curveto(658.14552598,65.24009567)(658.06052607,65.42009549)(657.9505249,65.58010061)
\curveto(657.9305262,65.6100953)(657.91052622,65.63509527)(657.8905249,65.65510061)
\lineto(657.8305249,65.71510061)
\curveto(657.81052632,65.75509515)(657.77052636,65.8050951)(657.7105249,65.86510061)
\curveto(657.57052656,65.96509494)(657.44052669,66.05009486)(657.3205249,66.12010061)
\curveto(657.20052693,66.19009472)(657.05552707,66.26009465)(656.8855249,66.33010061)
\curveto(656.81552731,66.36009455)(656.74552738,66.38009453)(656.6755249,66.39010061)
\curveto(656.60552752,66.4100945)(656.5305276,66.43009448)(656.4505249,66.45010061)
}
}
{
\newrgbcolor{curcolor}{0 0 0}
\pscustom[linestyle=none,fillstyle=solid,fillcolor=curcolor]
{
\newpath
\moveto(648.6055249,72.59470998)
\curveto(648.59553553,73.28470535)(648.71553541,73.88470475)(648.9655249,74.39470998)
\curveto(649.21553491,74.91470372)(649.55053458,75.30970332)(649.9705249,75.57970998)
\curveto(650.05053408,75.629703)(650.14053399,75.67470296)(650.2405249,75.71470998)
\curveto(650.3305338,75.75470288)(650.4255337,75.79970283)(650.5255249,75.84970998)
\curveto(650.6255335,75.88970274)(650.7255334,75.91970271)(650.8255249,75.93970998)
\curveto(650.9255332,75.95970267)(651.0305331,75.97970265)(651.1405249,75.99970998)
\curveto(651.19053294,76.01970261)(651.23553289,76.02470261)(651.2755249,76.01470998)
\curveto(651.31553281,76.00470263)(651.36053277,76.00970262)(651.4105249,76.02970998)
\curveto(651.46053267,76.03970259)(651.54553258,76.04470259)(651.6655249,76.04470998)
\curveto(651.77553235,76.04470259)(651.86053227,76.03970259)(651.9205249,76.02970998)
\curveto(651.98053215,76.00970262)(652.04053209,75.99970263)(652.1005249,75.99970998)
\curveto(652.16053197,76.00970262)(652.22053191,76.00470263)(652.2805249,75.98470998)
\curveto(652.42053171,75.94470269)(652.55553157,75.90970272)(652.6855249,75.87970998)
\curveto(652.81553131,75.84970278)(652.94053119,75.80970282)(653.0605249,75.75970998)
\curveto(653.20053093,75.69970293)(653.3255308,75.629703)(653.4355249,75.54970998)
\curveto(653.54553058,75.47970315)(653.65553047,75.40470323)(653.7655249,75.32470998)
\lineto(653.8255249,75.26470998)
\curveto(653.84553028,75.25470338)(653.86553026,75.23970339)(653.8855249,75.21970998)
\curveto(654.04553008,75.09970353)(654.19052994,74.96470367)(654.3205249,74.81470998)
\curveto(654.45052968,74.66470397)(654.57552955,74.50470413)(654.6955249,74.33470998)
\curveto(654.91552921,74.02470461)(655.12052901,73.7297049)(655.3105249,73.44970998)
\curveto(655.45052868,73.21970541)(655.58552854,72.98970564)(655.7155249,72.75970998)
\curveto(655.84552828,72.53970609)(655.98052815,72.31970631)(656.1205249,72.09970998)
\curveto(656.29052784,71.84970678)(656.47052766,71.60970702)(656.6605249,71.37970998)
\curveto(656.85052728,71.15970747)(657.07552705,70.96970766)(657.3355249,70.80970998)
\curveto(657.39552673,70.76970786)(657.45552667,70.7347079)(657.5155249,70.70470998)
\curveto(657.56552656,70.67470796)(657.6305265,70.64470799)(657.7105249,70.61470998)
\curveto(657.78052635,70.59470804)(657.84052629,70.58970804)(657.8905249,70.59970998)
\curveto(657.96052617,70.61970801)(658.01552611,70.65470798)(658.0555249,70.70470998)
\curveto(658.08552604,70.75470788)(658.10552602,70.81470782)(658.1155249,70.88470998)
\lineto(658.1155249,71.12470998)
\lineto(658.1155249,71.87470998)
\lineto(658.1155249,74.67970998)
\lineto(658.1155249,75.33970998)
\curveto(658.11552601,75.4297032)(658.12052601,75.51470312)(658.1305249,75.59470998)
\curveto(658.130526,75.67470296)(658.15052598,75.73970289)(658.1905249,75.78970998)
\curveto(658.2305259,75.83970279)(658.30552582,75.87970275)(658.4155249,75.90970998)
\curveto(658.51552561,75.94970268)(658.61552551,75.95970267)(658.7155249,75.93970998)
\lineto(658.8505249,75.93970998)
\curveto(658.92052521,75.91970271)(658.98052515,75.89970273)(659.0305249,75.87970998)
\curveto(659.08052505,75.85970277)(659.12052501,75.82470281)(659.1505249,75.77470998)
\curveto(659.19052494,75.72470291)(659.21052492,75.65470298)(659.2105249,75.56470998)
\lineto(659.2105249,75.29470998)
\lineto(659.2105249,74.39470998)
\lineto(659.2105249,70.88470998)
\lineto(659.2105249,69.81970998)
\curveto(659.21052492,69.73970889)(659.21552491,69.64970898)(659.2255249,69.54970998)
\curveto(659.2255249,69.44970918)(659.21552491,69.36470927)(659.1955249,69.29470998)
\curveto(659.125525,69.08470955)(658.94552518,69.01970961)(658.6555249,69.09970998)
\curveto(658.61552551,69.10970952)(658.58052555,69.10970952)(658.5505249,69.09970998)
\curveto(658.51052562,69.09970953)(658.46552566,69.10970952)(658.4155249,69.12970998)
\curveto(658.33552579,69.14970948)(658.25052588,69.16970946)(658.1605249,69.18970998)
\curveto(658.07052606,69.20970942)(657.98552614,69.2347094)(657.9055249,69.26470998)
\curveto(657.41552671,69.42470921)(657.00052713,69.62470901)(656.6605249,69.86470998)
\curveto(656.41052772,70.04470859)(656.18552794,70.24970838)(655.9855249,70.47970998)
\curveto(655.77552835,70.70970792)(655.58052855,70.94970768)(655.4005249,71.19970998)
\curveto(655.22052891,71.45970717)(655.05052908,71.72470691)(654.8905249,71.99470998)
\curveto(654.72052941,72.27470636)(654.54552958,72.54470609)(654.3655249,72.80470998)
\curveto(654.28552984,72.91470572)(654.21052992,73.01970561)(654.1405249,73.11970998)
\curveto(654.07053006,73.2297054)(653.99553013,73.33970529)(653.9155249,73.44970998)
\curveto(653.88553024,73.48970514)(653.85553027,73.52470511)(653.8255249,73.55470998)
\curveto(653.78553034,73.59470504)(653.75553037,73.634705)(653.7355249,73.67470998)
\curveto(653.6255305,73.81470482)(653.50053063,73.93970469)(653.3605249,74.04970998)
\curveto(653.3305308,74.06970456)(653.30553082,74.09470454)(653.2855249,74.12470998)
\curveto(653.25553087,74.15470448)(653.2255309,74.17970445)(653.1955249,74.19970998)
\curveto(653.09553103,74.27970435)(652.99553113,74.34470429)(652.8955249,74.39470998)
\curveto(652.79553133,74.45470418)(652.68553144,74.50970412)(652.5655249,74.55970998)
\curveto(652.49553163,74.58970404)(652.42053171,74.60970402)(652.3405249,74.61970998)
\lineto(652.1005249,74.67970998)
\lineto(652.0105249,74.67970998)
\curveto(651.98053215,74.68970394)(651.95053218,74.69470394)(651.9205249,74.69470998)
\curveto(651.85053228,74.71470392)(651.75553237,74.71970391)(651.6355249,74.70970998)
\curveto(651.50553262,74.70970392)(651.40553272,74.69970393)(651.3355249,74.67970998)
\curveto(651.25553287,74.65970397)(651.18053295,74.63970399)(651.1105249,74.61970998)
\curveto(651.0305331,74.60970402)(650.95053318,74.58970404)(650.8705249,74.55970998)
\curveto(650.6305335,74.44970418)(650.4305337,74.29970433)(650.2705249,74.10970998)
\curveto(650.10053403,73.9297047)(649.96053417,73.70970492)(649.8505249,73.44970998)
\curveto(649.8305343,73.37970525)(649.81553431,73.30970532)(649.8055249,73.23970998)
\curveto(649.78553434,73.16970546)(649.76553436,73.09470554)(649.7455249,73.01470998)
\curveto(649.7255344,72.9347057)(649.71553441,72.82470581)(649.7155249,72.68470998)
\curveto(649.71553441,72.55470608)(649.7255344,72.44970618)(649.7455249,72.36970998)
\curveto(649.75553437,72.30970632)(649.76053437,72.25470638)(649.7605249,72.20470998)
\curveto(649.76053437,72.15470648)(649.77053436,72.10470653)(649.7905249,72.05470998)
\curveto(649.8305343,71.95470668)(649.87053426,71.85970677)(649.9105249,71.76970998)
\curveto(649.95053418,71.68970694)(649.99553413,71.60970702)(650.0455249,71.52970998)
\curveto(650.06553406,71.49970713)(650.09053404,71.46970716)(650.1205249,71.43970998)
\curveto(650.15053398,71.41970721)(650.17553395,71.39470724)(650.1955249,71.36470998)
\lineto(650.2705249,71.28970998)
\curveto(650.29053384,71.25970737)(650.31053382,71.2347074)(650.3305249,71.21470998)
\lineto(650.5405249,71.06470998)
\curveto(650.60053353,71.02470761)(650.66553346,70.97970765)(650.7355249,70.92970998)
\curveto(650.8255333,70.86970776)(650.9305332,70.81970781)(651.0505249,70.77970998)
\curveto(651.16053297,70.74970788)(651.27053286,70.71470792)(651.3805249,70.67470998)
\curveto(651.49053264,70.634708)(651.63553249,70.60970802)(651.8155249,70.59970998)
\curveto(651.98553214,70.58970804)(652.11053202,70.55970807)(652.1905249,70.50970998)
\curveto(652.27053186,70.45970817)(652.31553181,70.38470825)(652.3255249,70.28470998)
\curveto(652.33553179,70.18470845)(652.34053179,70.07470856)(652.3405249,69.95470998)
\curveto(652.34053179,69.91470872)(652.34553178,69.87470876)(652.3555249,69.83470998)
\curveto(652.35553177,69.79470884)(652.35053178,69.75970887)(652.3405249,69.72970998)
\curveto(652.32053181,69.67970895)(652.31053182,69.629709)(652.3105249,69.57970998)
\curveto(652.31053182,69.53970909)(652.30053183,69.49970913)(652.2805249,69.45970998)
\curveto(652.22053191,69.36970926)(652.08553204,69.32470931)(651.8755249,69.32470998)
\lineto(651.7555249,69.32470998)
\curveto(651.69553243,69.3347093)(651.63553249,69.33970929)(651.5755249,69.33970998)
\curveto(651.50553262,69.34970928)(651.44053269,69.35970927)(651.3805249,69.36970998)
\curveto(651.27053286,69.38970924)(651.17053296,69.40970922)(651.0805249,69.42970998)
\curveto(650.98053315,69.44970918)(650.88553324,69.47970915)(650.7955249,69.51970998)
\curveto(650.7255334,69.53970909)(650.66553346,69.55970907)(650.6155249,69.57970998)
\lineto(650.4355249,69.63970998)
\curveto(650.17553395,69.75970887)(649.9305342,69.91470872)(649.7005249,70.10470998)
\curveto(649.47053466,70.30470833)(649.28553484,70.51970811)(649.1455249,70.74970998)
\curveto(649.06553506,70.85970777)(649.00053513,70.97470766)(648.9505249,71.09470998)
\lineto(648.8005249,71.48470998)
\curveto(648.75053538,71.59470704)(648.72053541,71.70970692)(648.7105249,71.82970998)
\curveto(648.69053544,71.94970668)(648.66553546,72.07470656)(648.6355249,72.20470998)
\curveto(648.63553549,72.27470636)(648.63553549,72.33970629)(648.6355249,72.39970998)
\curveto(648.6255355,72.45970617)(648.61553551,72.52470611)(648.6055249,72.59470998)
}
}
{
\newrgbcolor{curcolor}{0 0 0}
\pscustom[linestyle=none,fillstyle=solid,fillcolor=curcolor]
{
\newpath
\moveto(657.5755249,78.63431936)
\lineto(657.5755249,79.26431936)
\lineto(657.5755249,79.45931936)
\curveto(657.57552655,79.52931683)(657.58552654,79.58931677)(657.6055249,79.63931936)
\curveto(657.64552648,79.70931665)(657.68552644,79.7593166)(657.7255249,79.78931936)
\curveto(657.77552635,79.82931653)(657.84052629,79.84931651)(657.9205249,79.84931936)
\curveto(658.00052613,79.8593165)(658.08552604,79.86431649)(658.1755249,79.86431936)
\lineto(658.8955249,79.86431936)
\curveto(659.37552475,79.86431649)(659.78552434,79.80431655)(660.1255249,79.68431936)
\curveto(660.46552366,79.56431679)(660.74052339,79.36931699)(660.9505249,79.09931936)
\curveto(661.00052313,79.02931733)(661.04552308,78.9593174)(661.0855249,78.88931936)
\curveto(661.13552299,78.82931753)(661.18052295,78.7543176)(661.2205249,78.66431936)
\curveto(661.2305229,78.64431771)(661.24052289,78.61431774)(661.2505249,78.57431936)
\curveto(661.27052286,78.53431782)(661.27552285,78.48931787)(661.2655249,78.43931936)
\curveto(661.23552289,78.34931801)(661.16052297,78.29431806)(661.0405249,78.27431936)
\curveto(660.9305232,78.2543181)(660.83552329,78.26931809)(660.7555249,78.31931936)
\curveto(660.68552344,78.34931801)(660.62052351,78.39431796)(660.5605249,78.45431936)
\curveto(660.51052362,78.52431783)(660.46052367,78.58931777)(660.4105249,78.64931936)
\curveto(660.36052377,78.71931764)(660.28552384,78.77931758)(660.1855249,78.82931936)
\curveto(660.09552403,78.88931747)(660.00552412,78.93931742)(659.9155249,78.97931936)
\curveto(659.88552424,78.99931736)(659.8255243,79.02431733)(659.7355249,79.05431936)
\curveto(659.65552447,79.08431727)(659.58552454,79.08931727)(659.5255249,79.06931936)
\curveto(659.38552474,79.03931732)(659.29552483,78.97931738)(659.2555249,78.88931936)
\curveto(659.2255249,78.80931755)(659.21052492,78.71931764)(659.2105249,78.61931936)
\curveto(659.21052492,78.51931784)(659.18552494,78.43431792)(659.1355249,78.36431936)
\curveto(659.06552506,78.27431808)(658.9255252,78.22931813)(658.7155249,78.22931936)
\lineto(658.1605249,78.22931936)
\lineto(657.9355249,78.22931936)
\curveto(657.85552627,78.23931812)(657.79052634,78.2593181)(657.7405249,78.28931936)
\curveto(657.66052647,78.34931801)(657.61552651,78.41931794)(657.6055249,78.49931936)
\curveto(657.59552653,78.51931784)(657.59052654,78.53931782)(657.5905249,78.55931936)
\curveto(657.59052654,78.58931777)(657.58552654,78.61431774)(657.5755249,78.63431936)
}
}
{
\newrgbcolor{curcolor}{0 0 0}
\pscustom[linestyle=none,fillstyle=solid,fillcolor=curcolor]
{
}
}
{
\newrgbcolor{curcolor}{0 0 0}
\pscustom[linestyle=none,fillstyle=solid,fillcolor=curcolor]
{
\newpath
\moveto(648.6055249,89.26463186)
\curveto(648.59553553,89.95462722)(648.71553541,90.55462662)(648.9655249,91.06463186)
\curveto(649.21553491,91.58462559)(649.55053458,91.9796252)(649.9705249,92.24963186)
\curveto(650.05053408,92.29962488)(650.14053399,92.34462483)(650.2405249,92.38463186)
\curveto(650.3305338,92.42462475)(650.4255337,92.46962471)(650.5255249,92.51963186)
\curveto(650.6255335,92.55962462)(650.7255334,92.58962459)(650.8255249,92.60963186)
\curveto(650.9255332,92.62962455)(651.0305331,92.64962453)(651.1405249,92.66963186)
\curveto(651.19053294,92.68962449)(651.23553289,92.69462448)(651.2755249,92.68463186)
\curveto(651.31553281,92.6746245)(651.36053277,92.6796245)(651.4105249,92.69963186)
\curveto(651.46053267,92.70962447)(651.54553258,92.71462446)(651.6655249,92.71463186)
\curveto(651.77553235,92.71462446)(651.86053227,92.70962447)(651.9205249,92.69963186)
\curveto(651.98053215,92.6796245)(652.04053209,92.66962451)(652.1005249,92.66963186)
\curveto(652.16053197,92.6796245)(652.22053191,92.6746245)(652.2805249,92.65463186)
\curveto(652.42053171,92.61462456)(652.55553157,92.5796246)(652.6855249,92.54963186)
\curveto(652.81553131,92.51962466)(652.94053119,92.4796247)(653.0605249,92.42963186)
\curveto(653.20053093,92.36962481)(653.3255308,92.29962488)(653.4355249,92.21963186)
\curveto(653.54553058,92.14962503)(653.65553047,92.0746251)(653.7655249,91.99463186)
\lineto(653.8255249,91.93463186)
\curveto(653.84553028,91.92462525)(653.86553026,91.90962527)(653.8855249,91.88963186)
\curveto(654.04553008,91.76962541)(654.19052994,91.63462554)(654.3205249,91.48463186)
\curveto(654.45052968,91.33462584)(654.57552955,91.174626)(654.6955249,91.00463186)
\curveto(654.91552921,90.69462648)(655.12052901,90.39962678)(655.3105249,90.11963186)
\curveto(655.45052868,89.88962729)(655.58552854,89.65962752)(655.7155249,89.42963186)
\curveto(655.84552828,89.20962797)(655.98052815,88.98962819)(656.1205249,88.76963186)
\curveto(656.29052784,88.51962866)(656.47052766,88.2796289)(656.6605249,88.04963186)
\curveto(656.85052728,87.82962935)(657.07552705,87.63962954)(657.3355249,87.47963186)
\curveto(657.39552673,87.43962974)(657.45552667,87.40462977)(657.5155249,87.37463186)
\curveto(657.56552656,87.34462983)(657.6305265,87.31462986)(657.7105249,87.28463186)
\curveto(657.78052635,87.26462991)(657.84052629,87.25962992)(657.8905249,87.26963186)
\curveto(657.96052617,87.28962989)(658.01552611,87.32462985)(658.0555249,87.37463186)
\curveto(658.08552604,87.42462975)(658.10552602,87.48462969)(658.1155249,87.55463186)
\lineto(658.1155249,87.79463186)
\lineto(658.1155249,88.54463186)
\lineto(658.1155249,91.34963186)
\lineto(658.1155249,92.00963186)
\curveto(658.11552601,92.09962508)(658.12052601,92.18462499)(658.1305249,92.26463186)
\curveto(658.130526,92.34462483)(658.15052598,92.40962477)(658.1905249,92.45963186)
\curveto(658.2305259,92.50962467)(658.30552582,92.54962463)(658.4155249,92.57963186)
\curveto(658.51552561,92.61962456)(658.61552551,92.62962455)(658.7155249,92.60963186)
\lineto(658.8505249,92.60963186)
\curveto(658.92052521,92.58962459)(658.98052515,92.56962461)(659.0305249,92.54963186)
\curveto(659.08052505,92.52962465)(659.12052501,92.49462468)(659.1505249,92.44463186)
\curveto(659.19052494,92.39462478)(659.21052492,92.32462485)(659.2105249,92.23463186)
\lineto(659.2105249,91.96463186)
\lineto(659.2105249,91.06463186)
\lineto(659.2105249,87.55463186)
\lineto(659.2105249,86.48963186)
\curveto(659.21052492,86.40963077)(659.21552491,86.31963086)(659.2255249,86.21963186)
\curveto(659.2255249,86.11963106)(659.21552491,86.03463114)(659.1955249,85.96463186)
\curveto(659.125525,85.75463142)(658.94552518,85.68963149)(658.6555249,85.76963186)
\curveto(658.61552551,85.7796314)(658.58052555,85.7796314)(658.5505249,85.76963186)
\curveto(658.51052562,85.76963141)(658.46552566,85.7796314)(658.4155249,85.79963186)
\curveto(658.33552579,85.81963136)(658.25052588,85.83963134)(658.1605249,85.85963186)
\curveto(658.07052606,85.8796313)(657.98552614,85.90463127)(657.9055249,85.93463186)
\curveto(657.41552671,86.09463108)(657.00052713,86.29463088)(656.6605249,86.53463186)
\curveto(656.41052772,86.71463046)(656.18552794,86.91963026)(655.9855249,87.14963186)
\curveto(655.77552835,87.3796298)(655.58052855,87.61962956)(655.4005249,87.86963186)
\curveto(655.22052891,88.12962905)(655.05052908,88.39462878)(654.8905249,88.66463186)
\curveto(654.72052941,88.94462823)(654.54552958,89.21462796)(654.3655249,89.47463186)
\curveto(654.28552984,89.58462759)(654.21052992,89.68962749)(654.1405249,89.78963186)
\curveto(654.07053006,89.89962728)(653.99553013,90.00962717)(653.9155249,90.11963186)
\curveto(653.88553024,90.15962702)(653.85553027,90.19462698)(653.8255249,90.22463186)
\curveto(653.78553034,90.26462691)(653.75553037,90.30462687)(653.7355249,90.34463186)
\curveto(653.6255305,90.48462669)(653.50053063,90.60962657)(653.3605249,90.71963186)
\curveto(653.3305308,90.73962644)(653.30553082,90.76462641)(653.2855249,90.79463186)
\curveto(653.25553087,90.82462635)(653.2255309,90.84962633)(653.1955249,90.86963186)
\curveto(653.09553103,90.94962623)(652.99553113,91.01462616)(652.8955249,91.06463186)
\curveto(652.79553133,91.12462605)(652.68553144,91.179626)(652.5655249,91.22963186)
\curveto(652.49553163,91.25962592)(652.42053171,91.2796259)(652.3405249,91.28963186)
\lineto(652.1005249,91.34963186)
\lineto(652.0105249,91.34963186)
\curveto(651.98053215,91.35962582)(651.95053218,91.36462581)(651.9205249,91.36463186)
\curveto(651.85053228,91.38462579)(651.75553237,91.38962579)(651.6355249,91.37963186)
\curveto(651.50553262,91.3796258)(651.40553272,91.36962581)(651.3355249,91.34963186)
\curveto(651.25553287,91.32962585)(651.18053295,91.30962587)(651.1105249,91.28963186)
\curveto(651.0305331,91.2796259)(650.95053318,91.25962592)(650.8705249,91.22963186)
\curveto(650.6305335,91.11962606)(650.4305337,90.96962621)(650.2705249,90.77963186)
\curveto(650.10053403,90.59962658)(649.96053417,90.3796268)(649.8505249,90.11963186)
\curveto(649.8305343,90.04962713)(649.81553431,89.9796272)(649.8055249,89.90963186)
\curveto(649.78553434,89.83962734)(649.76553436,89.76462741)(649.7455249,89.68463186)
\curveto(649.7255344,89.60462757)(649.71553441,89.49462768)(649.7155249,89.35463186)
\curveto(649.71553441,89.22462795)(649.7255344,89.11962806)(649.7455249,89.03963186)
\curveto(649.75553437,88.9796282)(649.76053437,88.92462825)(649.7605249,88.87463186)
\curveto(649.76053437,88.82462835)(649.77053436,88.7746284)(649.7905249,88.72463186)
\curveto(649.8305343,88.62462855)(649.87053426,88.52962865)(649.9105249,88.43963186)
\curveto(649.95053418,88.35962882)(649.99553413,88.2796289)(650.0455249,88.19963186)
\curveto(650.06553406,88.16962901)(650.09053404,88.13962904)(650.1205249,88.10963186)
\curveto(650.15053398,88.08962909)(650.17553395,88.06462911)(650.1955249,88.03463186)
\lineto(650.2705249,87.95963186)
\curveto(650.29053384,87.92962925)(650.31053382,87.90462927)(650.3305249,87.88463186)
\lineto(650.5405249,87.73463186)
\curveto(650.60053353,87.69462948)(650.66553346,87.64962953)(650.7355249,87.59963186)
\curveto(650.8255333,87.53962964)(650.9305332,87.48962969)(651.0505249,87.44963186)
\curveto(651.16053297,87.41962976)(651.27053286,87.38462979)(651.3805249,87.34463186)
\curveto(651.49053264,87.30462987)(651.63553249,87.2796299)(651.8155249,87.26963186)
\curveto(651.98553214,87.25962992)(652.11053202,87.22962995)(652.1905249,87.17963186)
\curveto(652.27053186,87.12963005)(652.31553181,87.05463012)(652.3255249,86.95463186)
\curveto(652.33553179,86.85463032)(652.34053179,86.74463043)(652.3405249,86.62463186)
\curveto(652.34053179,86.58463059)(652.34553178,86.54463063)(652.3555249,86.50463186)
\curveto(652.35553177,86.46463071)(652.35053178,86.42963075)(652.3405249,86.39963186)
\curveto(652.32053181,86.34963083)(652.31053182,86.29963088)(652.3105249,86.24963186)
\curveto(652.31053182,86.20963097)(652.30053183,86.16963101)(652.2805249,86.12963186)
\curveto(652.22053191,86.03963114)(652.08553204,85.99463118)(651.8755249,85.99463186)
\lineto(651.7555249,85.99463186)
\curveto(651.69553243,86.00463117)(651.63553249,86.00963117)(651.5755249,86.00963186)
\curveto(651.50553262,86.01963116)(651.44053269,86.02963115)(651.3805249,86.03963186)
\curveto(651.27053286,86.05963112)(651.17053296,86.0796311)(651.0805249,86.09963186)
\curveto(650.98053315,86.11963106)(650.88553324,86.14963103)(650.7955249,86.18963186)
\curveto(650.7255334,86.20963097)(650.66553346,86.22963095)(650.6155249,86.24963186)
\lineto(650.4355249,86.30963186)
\curveto(650.17553395,86.42963075)(649.9305342,86.58463059)(649.7005249,86.77463186)
\curveto(649.47053466,86.9746302)(649.28553484,87.18962999)(649.1455249,87.41963186)
\curveto(649.06553506,87.52962965)(649.00053513,87.64462953)(648.9505249,87.76463186)
\lineto(648.8005249,88.15463186)
\curveto(648.75053538,88.26462891)(648.72053541,88.3796288)(648.7105249,88.49963186)
\curveto(648.69053544,88.61962856)(648.66553546,88.74462843)(648.6355249,88.87463186)
\curveto(648.63553549,88.94462823)(648.63553549,89.00962817)(648.6355249,89.06963186)
\curveto(648.6255355,89.12962805)(648.61553551,89.19462798)(648.6055249,89.26463186)
}
}
{
\newrgbcolor{curcolor}{0 0 0}
\pscustom[linestyle=none,fillstyle=solid,fillcolor=curcolor]
{
\newpath
\moveto(654.1255249,101.36424123)
\lineto(654.3805249,101.36424123)
\curveto(654.46052967,101.37423353)(654.53552959,101.36923353)(654.6055249,101.34924123)
\lineto(654.8455249,101.34924123)
\lineto(655.0105249,101.34924123)
\curveto(655.11052902,101.32923357)(655.21552891,101.31923358)(655.3255249,101.31924123)
\curveto(655.4255287,101.31923358)(655.5255286,101.30923359)(655.6255249,101.28924123)
\lineto(655.7755249,101.28924123)
\curveto(655.91552821,101.25923364)(656.05552807,101.23923366)(656.1955249,101.22924123)
\curveto(656.3255278,101.21923368)(656.45552767,101.19423371)(656.5855249,101.15424123)
\curveto(656.66552746,101.13423377)(656.75052738,101.11423379)(656.8405249,101.09424123)
\lineto(657.0805249,101.03424123)
\lineto(657.3805249,100.91424123)
\curveto(657.47052666,100.88423402)(657.56052657,100.84923405)(657.6505249,100.80924123)
\curveto(657.87052626,100.70923419)(658.08552604,100.57423433)(658.2955249,100.40424123)
\curveto(658.50552562,100.24423466)(658.67552545,100.06923483)(658.8055249,99.87924123)
\curveto(658.84552528,99.82923507)(658.88552524,99.76923513)(658.9255249,99.69924123)
\curveto(658.95552517,99.63923526)(658.99052514,99.57923532)(659.0305249,99.51924123)
\curveto(659.08052505,99.43923546)(659.12052501,99.34423556)(659.1505249,99.23424123)
\curveto(659.18052495,99.12423578)(659.21052492,99.01923588)(659.2405249,98.91924123)
\curveto(659.28052485,98.80923609)(659.30552482,98.6992362)(659.3155249,98.58924123)
\curveto(659.3255248,98.47923642)(659.34052479,98.36423654)(659.3605249,98.24424123)
\curveto(659.37052476,98.2042367)(659.37052476,98.15923674)(659.3605249,98.10924123)
\curveto(659.36052477,98.06923683)(659.36552476,98.02923687)(659.3755249,97.98924123)
\curveto(659.38552474,97.94923695)(659.39052474,97.89423701)(659.3905249,97.82424123)
\curveto(659.39052474,97.75423715)(659.38552474,97.7042372)(659.3755249,97.67424123)
\curveto(659.35552477,97.62423728)(659.35052478,97.57923732)(659.3605249,97.53924123)
\curveto(659.37052476,97.4992374)(659.37052476,97.46423744)(659.3605249,97.43424123)
\lineto(659.3605249,97.34424123)
\curveto(659.34052479,97.28423762)(659.3255248,97.21923768)(659.3155249,97.14924123)
\curveto(659.31552481,97.08923781)(659.31052482,97.02423788)(659.3005249,96.95424123)
\curveto(659.25052488,96.78423812)(659.20052493,96.62423828)(659.1505249,96.47424123)
\curveto(659.10052503,96.32423858)(659.03552509,96.17923872)(658.9555249,96.03924123)
\curveto(658.91552521,95.98923891)(658.88552524,95.93423897)(658.8655249,95.87424123)
\curveto(658.83552529,95.82423908)(658.80052533,95.77423913)(658.7605249,95.72424123)
\curveto(658.58052555,95.48423942)(658.36052577,95.28423962)(658.1005249,95.12424123)
\curveto(657.84052629,94.96423994)(657.55552657,94.82424008)(657.2455249,94.70424123)
\curveto(657.10552702,94.64424026)(656.96552716,94.5992403)(656.8255249,94.56924123)
\curveto(656.67552745,94.53924036)(656.52052761,94.5042404)(656.3605249,94.46424123)
\curveto(656.25052788,94.44424046)(656.14052799,94.42924047)(656.0305249,94.41924123)
\curveto(655.92052821,94.40924049)(655.81052832,94.39424051)(655.7005249,94.37424123)
\curveto(655.66052847,94.36424054)(655.62052851,94.35924054)(655.5805249,94.35924123)
\curveto(655.54052859,94.36924053)(655.50052863,94.36924053)(655.4605249,94.35924123)
\curveto(655.41052872,94.34924055)(655.36052877,94.34424056)(655.3105249,94.34424123)
\lineto(655.1455249,94.34424123)
\curveto(655.09552903,94.32424058)(655.04552908,94.31924058)(654.9955249,94.32924123)
\curveto(654.93552919,94.33924056)(654.88052925,94.33924056)(654.8305249,94.32924123)
\curveto(654.79052934,94.31924058)(654.74552938,94.31924058)(654.6955249,94.32924123)
\curveto(654.64552948,94.33924056)(654.59552953,94.33424057)(654.5455249,94.31424123)
\curveto(654.47552965,94.29424061)(654.40052973,94.28924061)(654.3205249,94.29924123)
\curveto(654.2305299,94.30924059)(654.14552998,94.31424059)(654.0655249,94.31424123)
\curveto(653.97553015,94.31424059)(653.87553025,94.30924059)(653.7655249,94.29924123)
\curveto(653.64553048,94.28924061)(653.54553058,94.29424061)(653.4655249,94.31424123)
\lineto(653.1805249,94.31424123)
\lineto(652.5505249,94.35924123)
\curveto(652.45053168,94.36924053)(652.35553177,94.37924052)(652.2655249,94.38924123)
\lineto(651.9655249,94.41924123)
\curveto(651.91553221,94.43924046)(651.86553226,94.44424046)(651.8155249,94.43424123)
\curveto(651.75553237,94.43424047)(651.70053243,94.44424046)(651.6505249,94.46424123)
\curveto(651.48053265,94.51424039)(651.31553281,94.55424035)(651.1555249,94.58424123)
\curveto(650.98553314,94.61424029)(650.8255333,94.66424024)(650.6755249,94.73424123)
\curveto(650.21553391,94.92423998)(649.84053429,95.14423976)(649.5505249,95.39424123)
\curveto(649.26053487,95.65423925)(649.01553511,96.01423889)(648.8155249,96.47424123)
\curveto(648.76553536,96.6042383)(648.7305354,96.73423817)(648.7105249,96.86424123)
\curveto(648.69053544,97.0042379)(648.66553546,97.14423776)(648.6355249,97.28424123)
\curveto(648.6255355,97.35423755)(648.62053551,97.41923748)(648.6205249,97.47924123)
\curveto(648.62053551,97.53923736)(648.61553551,97.6042373)(648.6055249,97.67424123)
\curveto(648.58553554,98.5042364)(648.73553539,99.17423573)(649.0555249,99.68424123)
\curveto(649.36553476,100.19423471)(649.80553432,100.57423433)(650.3755249,100.82424123)
\curveto(650.49553363,100.87423403)(650.62053351,100.91923398)(650.7505249,100.95924123)
\curveto(650.88053325,100.9992339)(651.01553311,101.04423386)(651.1555249,101.09424123)
\curveto(651.23553289,101.11423379)(651.32053281,101.12923377)(651.4105249,101.13924123)
\lineto(651.6505249,101.19924123)
\curveto(651.76053237,101.22923367)(651.87053226,101.24423366)(651.9805249,101.24424123)
\curveto(652.09053204,101.25423365)(652.20053193,101.26923363)(652.3105249,101.28924123)
\curveto(652.36053177,101.30923359)(652.40553172,101.31423359)(652.4455249,101.30424123)
\curveto(652.48553164,101.3042336)(652.5255316,101.30923359)(652.5655249,101.31924123)
\curveto(652.61553151,101.32923357)(652.67053146,101.32923357)(652.7305249,101.31924123)
\curveto(652.78053135,101.31923358)(652.8305313,101.32423358)(652.8805249,101.33424123)
\lineto(653.0155249,101.33424123)
\curveto(653.07553105,101.35423355)(653.14553098,101.35423355)(653.2255249,101.33424123)
\curveto(653.29553083,101.32423358)(653.36053077,101.32923357)(653.4205249,101.34924123)
\curveto(653.45053068,101.35923354)(653.49053064,101.36423354)(653.5405249,101.36424123)
\lineto(653.6605249,101.36424123)
\lineto(654.1255249,101.36424123)
\moveto(656.4505249,99.81924123)
\curveto(656.130528,99.91923498)(655.76552836,99.97923492)(655.3555249,99.99924123)
\curveto(654.94552918,100.01923488)(654.53552959,100.02923487)(654.1255249,100.02924123)
\curveto(653.69553043,100.02923487)(653.27553085,100.01923488)(652.8655249,99.99924123)
\curveto(652.45553167,99.97923492)(652.07053206,99.93423497)(651.7105249,99.86424123)
\curveto(651.35053278,99.79423511)(651.0305331,99.68423522)(650.7505249,99.53424123)
\curveto(650.46053367,99.39423551)(650.2255339,99.1992357)(650.0455249,98.94924123)
\curveto(649.93553419,98.78923611)(649.85553427,98.60923629)(649.8055249,98.40924123)
\curveto(649.74553438,98.20923669)(649.71553441,97.96423694)(649.7155249,97.67424123)
\curveto(649.73553439,97.65423725)(649.74553438,97.61923728)(649.7455249,97.56924123)
\curveto(649.73553439,97.51923738)(649.73553439,97.47923742)(649.7455249,97.44924123)
\curveto(649.76553436,97.36923753)(649.78553434,97.29423761)(649.8055249,97.22424123)
\curveto(649.81553431,97.16423774)(649.83553429,97.0992378)(649.8655249,97.02924123)
\curveto(649.98553414,96.75923814)(650.15553397,96.53923836)(650.3755249,96.36924123)
\curveto(650.58553354,96.20923869)(650.8305333,96.07423883)(651.1105249,95.96424123)
\curveto(651.22053291,95.91423899)(651.34053279,95.87423903)(651.4705249,95.84424123)
\curveto(651.59053254,95.82423908)(651.71553241,95.7992391)(651.8455249,95.76924123)
\curveto(651.89553223,95.74923915)(651.95053218,95.73923916)(652.0105249,95.73924123)
\curveto(652.06053207,95.73923916)(652.11053202,95.73423917)(652.1605249,95.72424123)
\curveto(652.25053188,95.71423919)(652.34553178,95.7042392)(652.4455249,95.69424123)
\curveto(652.53553159,95.68423922)(652.6305315,95.67423923)(652.7305249,95.66424123)
\curveto(652.81053132,95.66423924)(652.89553123,95.65923924)(652.9855249,95.64924123)
\lineto(653.2255249,95.64924123)
\lineto(653.4055249,95.64924123)
\curveto(653.43553069,95.63923926)(653.47053066,95.63423927)(653.5105249,95.63424123)
\lineto(653.6455249,95.63424123)
\lineto(654.0955249,95.63424123)
\curveto(654.17552995,95.63423927)(654.26052987,95.62923927)(654.3505249,95.61924123)
\curveto(654.4305297,95.61923928)(654.50552962,95.62923927)(654.5755249,95.64924123)
\lineto(654.8455249,95.64924123)
\curveto(654.86552926,95.64923925)(654.89552923,95.64423926)(654.9355249,95.63424123)
\curveto(654.96552916,95.63423927)(654.99052914,95.63923926)(655.0105249,95.64924123)
\curveto(655.11052902,95.65923924)(655.21052892,95.66423924)(655.3105249,95.66424123)
\curveto(655.40052873,95.67423923)(655.50052863,95.68423922)(655.6105249,95.69424123)
\curveto(655.7305284,95.72423918)(655.85552827,95.73923916)(655.9855249,95.73924123)
\curveto(656.10552802,95.74923915)(656.22052791,95.77423913)(656.3305249,95.81424123)
\curveto(656.6305275,95.89423901)(656.89552723,95.97923892)(657.1255249,96.06924123)
\curveto(657.35552677,96.16923873)(657.57052656,96.31423859)(657.7705249,96.50424123)
\curveto(657.97052616,96.71423819)(658.12052601,96.97923792)(658.2205249,97.29924123)
\curveto(658.24052589,97.33923756)(658.25052588,97.37423753)(658.2505249,97.40424123)
\curveto(658.24052589,97.44423746)(658.24552588,97.48923741)(658.2655249,97.53924123)
\curveto(658.27552585,97.57923732)(658.28552584,97.64923725)(658.2955249,97.74924123)
\curveto(658.30552582,97.85923704)(658.30052583,97.94423696)(658.2805249,98.00424123)
\curveto(658.26052587,98.07423683)(658.25052588,98.14423676)(658.2505249,98.21424123)
\curveto(658.24052589,98.28423662)(658.2255259,98.34923655)(658.2055249,98.40924123)
\curveto(658.14552598,98.60923629)(658.06052607,98.78923611)(657.9505249,98.94924123)
\curveto(657.9305262,98.97923592)(657.91052622,99.0042359)(657.8905249,99.02424123)
\lineto(657.8305249,99.08424123)
\curveto(657.81052632,99.12423578)(657.77052636,99.17423573)(657.7105249,99.23424123)
\curveto(657.57052656,99.33423557)(657.44052669,99.41923548)(657.3205249,99.48924123)
\curveto(657.20052693,99.55923534)(657.05552707,99.62923527)(656.8855249,99.69924123)
\curveto(656.81552731,99.72923517)(656.74552738,99.74923515)(656.6755249,99.75924123)
\curveto(656.60552752,99.77923512)(656.5305276,99.7992351)(656.4505249,99.81924123)
}
}
{
\newrgbcolor{curcolor}{0 0 0}
\pscustom[linestyle=none,fillstyle=solid,fillcolor=curcolor]
{
\newpath
\moveto(648.6055249,106.77385061)
\curveto(648.60553552,106.87384575)(648.61553551,106.96884566)(648.6355249,107.05885061)
\curveto(648.64553548,107.14884548)(648.67553545,107.21384541)(648.7255249,107.25385061)
\curveto(648.80553532,107.31384531)(648.91053522,107.34384528)(649.0405249,107.34385061)
\lineto(649.4305249,107.34385061)
\lineto(650.9305249,107.34385061)
\lineto(657.3205249,107.34385061)
\lineto(658.4905249,107.34385061)
\lineto(658.8055249,107.34385061)
\curveto(658.90552522,107.35384527)(658.98552514,107.33884529)(659.0455249,107.29885061)
\curveto(659.125525,107.24884538)(659.17552495,107.17384545)(659.1955249,107.07385061)
\curveto(659.20552492,106.98384564)(659.21052492,106.87384575)(659.2105249,106.74385061)
\lineto(659.2105249,106.51885061)
\curveto(659.19052494,106.43884619)(659.17552495,106.36884626)(659.1655249,106.30885061)
\curveto(659.14552498,106.24884638)(659.10552502,106.19884643)(659.0455249,106.15885061)
\curveto(658.98552514,106.11884651)(658.91052522,106.09884653)(658.8205249,106.09885061)
\lineto(658.5205249,106.09885061)
\lineto(657.4255249,106.09885061)
\lineto(652.0855249,106.09885061)
\curveto(651.99553213,106.07884655)(651.92053221,106.06384656)(651.8605249,106.05385061)
\curveto(651.79053234,106.05384657)(651.7305324,106.0238466)(651.6805249,105.96385061)
\curveto(651.6305325,105.89384673)(651.60553252,105.80384682)(651.6055249,105.69385061)
\curveto(651.59553253,105.59384703)(651.59053254,105.48384714)(651.5905249,105.36385061)
\lineto(651.5905249,104.22385061)
\lineto(651.5905249,103.72885061)
\curveto(651.58053255,103.56884906)(651.52053261,103.45884917)(651.4105249,103.39885061)
\curveto(651.38053275,103.37884925)(651.35053278,103.36884926)(651.3205249,103.36885061)
\curveto(651.28053285,103.36884926)(651.23553289,103.36384926)(651.1855249,103.35385061)
\curveto(651.06553306,103.33384929)(650.95553317,103.33884929)(650.8555249,103.36885061)
\curveto(650.75553337,103.40884922)(650.68553344,103.46384916)(650.6455249,103.53385061)
\curveto(650.59553353,103.61384901)(650.57053356,103.73384889)(650.5705249,103.89385061)
\curveto(650.57053356,104.05384857)(650.55553357,104.18884844)(650.5255249,104.29885061)
\curveto(650.51553361,104.34884828)(650.51053362,104.40384822)(650.5105249,104.46385061)
\curveto(650.50053363,104.5238481)(650.48553364,104.58384804)(650.4655249,104.64385061)
\curveto(650.41553371,104.79384783)(650.36553376,104.93884769)(650.3155249,105.07885061)
\curveto(650.25553387,105.21884741)(650.18553394,105.35384727)(650.1055249,105.48385061)
\curveto(650.01553411,105.623847)(649.91053422,105.74384688)(649.7905249,105.84385061)
\curveto(649.67053446,105.94384668)(649.54053459,106.03884659)(649.4005249,106.12885061)
\curveto(649.30053483,106.18884644)(649.19053494,106.23384639)(649.0705249,106.26385061)
\curveto(648.95053518,106.30384632)(648.84553528,106.35384627)(648.7555249,106.41385061)
\curveto(648.69553543,106.46384616)(648.65553547,106.53384609)(648.6355249,106.62385061)
\curveto(648.6255355,106.64384598)(648.62053551,106.66884596)(648.6205249,106.69885061)
\curveto(648.62053551,106.7288459)(648.61553551,106.75384587)(648.6055249,106.77385061)
}
}
{
\newrgbcolor{curcolor}{0 0 0}
\pscustom[linestyle=none,fillstyle=solid,fillcolor=curcolor]
{
\newpath
\moveto(648.6055249,115.12345998)
\curveto(648.60553552,115.22345513)(648.61553551,115.31845503)(648.6355249,115.40845998)
\curveto(648.64553548,115.49845485)(648.67553545,115.56345479)(648.7255249,115.60345998)
\curveto(648.80553532,115.66345469)(648.91053522,115.69345466)(649.0405249,115.69345998)
\lineto(649.4305249,115.69345998)
\lineto(650.9305249,115.69345998)
\lineto(657.3205249,115.69345998)
\lineto(658.4905249,115.69345998)
\lineto(658.8055249,115.69345998)
\curveto(658.90552522,115.70345465)(658.98552514,115.68845466)(659.0455249,115.64845998)
\curveto(659.125525,115.59845475)(659.17552495,115.52345483)(659.1955249,115.42345998)
\curveto(659.20552492,115.33345502)(659.21052492,115.22345513)(659.2105249,115.09345998)
\lineto(659.2105249,114.86845998)
\curveto(659.19052494,114.78845556)(659.17552495,114.71845563)(659.1655249,114.65845998)
\curveto(659.14552498,114.59845575)(659.10552502,114.5484558)(659.0455249,114.50845998)
\curveto(658.98552514,114.46845588)(658.91052522,114.4484559)(658.8205249,114.44845998)
\lineto(658.5205249,114.44845998)
\lineto(657.4255249,114.44845998)
\lineto(652.0855249,114.44845998)
\curveto(651.99553213,114.42845592)(651.92053221,114.41345594)(651.8605249,114.40345998)
\curveto(651.79053234,114.40345595)(651.7305324,114.37345598)(651.6805249,114.31345998)
\curveto(651.6305325,114.24345611)(651.60553252,114.1534562)(651.6055249,114.04345998)
\curveto(651.59553253,113.94345641)(651.59053254,113.83345652)(651.5905249,113.71345998)
\lineto(651.5905249,112.57345998)
\lineto(651.5905249,112.07845998)
\curveto(651.58053255,111.91845843)(651.52053261,111.80845854)(651.4105249,111.74845998)
\curveto(651.38053275,111.72845862)(651.35053278,111.71845863)(651.3205249,111.71845998)
\curveto(651.28053285,111.71845863)(651.23553289,111.71345864)(651.1855249,111.70345998)
\curveto(651.06553306,111.68345867)(650.95553317,111.68845866)(650.8555249,111.71845998)
\curveto(650.75553337,111.75845859)(650.68553344,111.81345854)(650.6455249,111.88345998)
\curveto(650.59553353,111.96345839)(650.57053356,112.08345827)(650.5705249,112.24345998)
\curveto(650.57053356,112.40345795)(650.55553357,112.53845781)(650.5255249,112.64845998)
\curveto(650.51553361,112.69845765)(650.51053362,112.7534576)(650.5105249,112.81345998)
\curveto(650.50053363,112.87345748)(650.48553364,112.93345742)(650.4655249,112.99345998)
\curveto(650.41553371,113.14345721)(650.36553376,113.28845706)(650.3155249,113.42845998)
\curveto(650.25553387,113.56845678)(650.18553394,113.70345665)(650.1055249,113.83345998)
\curveto(650.01553411,113.97345638)(649.91053422,114.09345626)(649.7905249,114.19345998)
\curveto(649.67053446,114.29345606)(649.54053459,114.38845596)(649.4005249,114.47845998)
\curveto(649.30053483,114.53845581)(649.19053494,114.58345577)(649.0705249,114.61345998)
\curveto(648.95053518,114.6534557)(648.84553528,114.70345565)(648.7555249,114.76345998)
\curveto(648.69553543,114.81345554)(648.65553547,114.88345547)(648.6355249,114.97345998)
\curveto(648.6255355,114.99345536)(648.62053551,115.01845533)(648.6205249,115.04845998)
\curveto(648.62053551,115.07845527)(648.61553551,115.10345525)(648.6055249,115.12345998)
}
}
{
\newrgbcolor{curcolor}{0 0 0}
\pscustom[linestyle=none,fillstyle=solid,fillcolor=curcolor]
{
\newpath
\moveto(669.44184082,37.28705373)
\curveto(669.44185152,37.35704805)(669.44185152,37.43704797)(669.44184082,37.52705373)
\curveto(669.43185153,37.61704779)(669.43185153,37.70204771)(669.44184082,37.78205373)
\curveto(669.44185152,37.87204754)(669.45185151,37.95204746)(669.47184082,38.02205373)
\curveto(669.49185147,38.10204731)(669.52185144,38.15704725)(669.56184082,38.18705373)
\curveto(669.61185135,38.21704719)(669.68685127,38.23704717)(669.78684082,38.24705373)
\curveto(669.87685108,38.26704714)(669.98185098,38.27704713)(670.10184082,38.27705373)
\curveto(670.21185075,38.28704712)(670.32685063,38.28704712)(670.44684082,38.27705373)
\lineto(670.74684082,38.27705373)
\lineto(673.76184082,38.27705373)
\lineto(676.65684082,38.27705373)
\curveto(676.98684397,38.27704713)(677.31184365,38.27204714)(677.63184082,38.26205373)
\curveto(677.94184302,38.26204715)(678.22184274,38.22204719)(678.47184082,38.14205373)
\curveto(678.82184214,38.02204739)(679.11684184,37.86704754)(679.35684082,37.67705373)
\curveto(679.58684137,37.48704792)(679.78684117,37.24704816)(679.95684082,36.95705373)
\curveto(680.00684095,36.89704851)(680.04184092,36.83204858)(680.06184082,36.76205373)
\curveto(680.08184088,36.70204871)(680.10684085,36.63204878)(680.13684082,36.55205373)
\curveto(680.18684077,36.43204898)(680.22184074,36.30204911)(680.24184082,36.16205373)
\curveto(680.27184069,36.03204938)(680.30184066,35.89704951)(680.33184082,35.75705373)
\curveto(680.35184061,35.7070497)(680.3568406,35.65704975)(680.34684082,35.60705373)
\curveto(680.33684062,35.55704985)(680.33684062,35.50204991)(680.34684082,35.44205373)
\curveto(680.3568406,35.42204999)(680.3568406,35.39705001)(680.34684082,35.36705373)
\curveto(680.34684061,35.33705007)(680.35184061,35.3120501)(680.36184082,35.29205373)
\curveto(680.37184059,35.25205016)(680.37684058,35.19705021)(680.37684082,35.12705373)
\curveto(680.37684058,35.05705035)(680.37184059,35.00205041)(680.36184082,34.96205373)
\curveto(680.35184061,34.9120505)(680.35184061,34.85705055)(680.36184082,34.79705373)
\curveto(680.37184059,34.73705067)(680.36684059,34.68205073)(680.34684082,34.63205373)
\curveto(680.31684064,34.50205091)(680.29684066,34.37705103)(680.28684082,34.25705373)
\curveto(680.27684068,34.13705127)(680.25184071,34.02205139)(680.21184082,33.91205373)
\curveto(680.09184087,33.54205187)(679.92184104,33.22205219)(679.70184082,32.95205373)
\curveto(679.48184148,32.68205273)(679.20184176,32.47205294)(678.86184082,32.32205373)
\curveto(678.74184222,32.27205314)(678.61684234,32.22705318)(678.48684082,32.18705373)
\curveto(678.3568426,32.15705325)(678.22184274,32.12205329)(678.08184082,32.08205373)
\curveto(678.03184293,32.07205334)(677.99184297,32.06705334)(677.96184082,32.06705373)
\curveto(677.92184304,32.06705334)(677.87684308,32.06205335)(677.82684082,32.05205373)
\curveto(677.79684316,32.04205337)(677.7618432,32.03705337)(677.72184082,32.03705373)
\curveto(677.67184329,32.03705337)(677.63184333,32.03205338)(677.60184082,32.02205373)
\lineto(677.43684082,32.02205373)
\curveto(677.3568436,32.00205341)(677.2568437,31.99705341)(677.13684082,32.00705373)
\curveto(677.00684395,32.01705339)(676.91684404,32.03205338)(676.86684082,32.05205373)
\curveto(676.77684418,32.07205334)(676.71184425,32.12705328)(676.67184082,32.21705373)
\curveto(676.65184431,32.24705316)(676.64684431,32.27705313)(676.65684082,32.30705373)
\curveto(676.6568443,32.33705307)(676.65184431,32.37705303)(676.64184082,32.42705373)
\curveto(676.63184433,32.46705294)(676.62684433,32.5070529)(676.62684082,32.54705373)
\lineto(676.62684082,32.69705373)
\curveto(676.62684433,32.81705259)(676.63184433,32.93705247)(676.64184082,33.05705373)
\curveto(676.64184432,33.18705222)(676.67684428,33.27705213)(676.74684082,33.32705373)
\curveto(676.80684415,33.36705204)(676.86684409,33.38705202)(676.92684082,33.38705373)
\curveto(676.98684397,33.38705202)(677.0568439,33.39705201)(677.13684082,33.41705373)
\curveto(677.16684379,33.42705198)(677.20184376,33.42705198)(677.24184082,33.41705373)
\curveto(677.27184369,33.41705199)(677.29684366,33.42205199)(677.31684082,33.43205373)
\lineto(677.52684082,33.43205373)
\curveto(677.57684338,33.45205196)(677.62684333,33.45705195)(677.67684082,33.44705373)
\curveto(677.71684324,33.44705196)(677.7618432,33.45705195)(677.81184082,33.47705373)
\curveto(677.94184302,33.5070519)(678.06684289,33.53705187)(678.18684082,33.56705373)
\curveto(678.29684266,33.59705181)(678.40184256,33.64205177)(678.50184082,33.70205373)
\curveto(678.79184217,33.87205154)(678.99684196,34.14205127)(679.11684082,34.51205373)
\curveto(679.13684182,34.56205085)(679.15184181,34.6120508)(679.16184082,34.66205373)
\curveto(679.1618418,34.72205069)(679.17184179,34.77705063)(679.19184082,34.82705373)
\lineto(679.19184082,34.90205373)
\curveto(679.20184176,34.97205044)(679.21184175,35.06705034)(679.22184082,35.18705373)
\curveto(679.22184174,35.31705009)(679.21184175,35.41704999)(679.19184082,35.48705373)
\curveto(679.17184179,35.55704985)(679.1568418,35.62704978)(679.14684082,35.69705373)
\curveto(679.12684183,35.77704963)(679.10684185,35.84704956)(679.08684082,35.90705373)
\curveto(678.92684203,36.28704912)(678.65184231,36.56204885)(678.26184082,36.73205373)
\curveto(678.13184283,36.78204863)(677.97684298,36.81704859)(677.79684082,36.83705373)
\curveto(677.61684334,36.86704854)(677.43184353,36.88204853)(677.24184082,36.88205373)
\curveto(677.04184392,36.89204852)(676.84184412,36.89204852)(676.64184082,36.88205373)
\lineto(676.07184082,36.88205373)
\lineto(671.82684082,36.88205373)
\lineto(670.28184082,36.88205373)
\curveto(670.17185079,36.88204853)(670.05185091,36.87704853)(669.92184082,36.86705373)
\curveto(669.79185117,36.85704855)(669.68685127,36.87704853)(669.60684082,36.92705373)
\curveto(669.53685142,36.98704842)(669.48685147,37.06704834)(669.45684082,37.16705373)
\curveto(669.4568515,37.18704822)(669.4568515,37.2070482)(669.45684082,37.22705373)
\curveto(669.4568515,37.24704816)(669.45185151,37.26704814)(669.44184082,37.28705373)
}
}
{
\newrgbcolor{curcolor}{0 0 0}
\pscustom[linestyle=none,fillstyle=solid,fillcolor=curcolor]
{
\newpath
\moveto(672.39684082,40.82072561)
\lineto(672.39684082,41.25572561)
\curveto(672.39684856,41.40572364)(672.43684852,41.51072354)(672.51684082,41.57072561)
\curveto(672.59684836,41.62072343)(672.69684826,41.6457234)(672.81684082,41.64572561)
\curveto(672.93684802,41.65572339)(673.0568479,41.66072339)(673.17684082,41.66072561)
\lineto(674.60184082,41.66072561)
\lineto(676.86684082,41.66072561)
\lineto(677.55684082,41.66072561)
\curveto(677.78684317,41.66072339)(677.98684297,41.68572336)(678.15684082,41.73572561)
\curveto(678.60684235,41.89572315)(678.92184204,42.19572285)(679.10184082,42.63572561)
\curveto(679.19184177,42.85572219)(679.22684173,43.12072193)(679.20684082,43.43072561)
\curveto(679.17684178,43.74072131)(679.12184184,43.99072106)(679.04184082,44.18072561)
\curveto(678.90184206,44.51072054)(678.72684223,44.77072028)(678.51684082,44.96072561)
\curveto(678.29684266,45.16071989)(678.01184295,45.31571973)(677.66184082,45.42572561)
\curveto(677.58184338,45.45571959)(677.50184346,45.47571957)(677.42184082,45.48572561)
\curveto(677.34184362,45.49571955)(677.2568437,45.51071954)(677.16684082,45.53072561)
\curveto(677.11684384,45.54071951)(677.07184389,45.54071951)(677.03184082,45.53072561)
\curveto(676.99184397,45.53071952)(676.94684401,45.54071951)(676.89684082,45.56072561)
\lineto(676.58184082,45.56072561)
\curveto(676.50184446,45.58071947)(676.41184455,45.58571946)(676.31184082,45.57572561)
\curveto(676.20184476,45.56571948)(676.10184486,45.56071949)(676.01184082,45.56072561)
\lineto(674.84184082,45.56072561)
\lineto(673.25184082,45.56072561)
\curveto(673.13184783,45.56071949)(673.00684795,45.55571949)(672.87684082,45.54572561)
\curveto(672.73684822,45.5457195)(672.62684833,45.57071948)(672.54684082,45.62072561)
\curveto(672.49684846,45.66071939)(672.46684849,45.70571934)(672.45684082,45.75572561)
\curveto(672.43684852,45.81571923)(672.41684854,45.88571916)(672.39684082,45.96572561)
\lineto(672.39684082,46.19072561)
\curveto(672.39684856,46.31071874)(672.40184856,46.41571863)(672.41184082,46.50572561)
\curveto(672.42184854,46.60571844)(672.46684849,46.68071837)(672.54684082,46.73072561)
\curveto(672.59684836,46.78071827)(672.67184829,46.80571824)(672.77184082,46.80572561)
\lineto(673.05684082,46.80572561)
\lineto(674.07684082,46.80572561)
\lineto(678.11184082,46.80572561)
\lineto(679.46184082,46.80572561)
\curveto(679.58184138,46.80571824)(679.69684126,46.80071825)(679.80684082,46.79072561)
\curveto(679.90684105,46.79071826)(679.98184098,46.75571829)(680.03184082,46.68572561)
\curveto(680.0618409,46.6457184)(680.08684087,46.58571846)(680.10684082,46.50572561)
\curveto(680.11684084,46.42571862)(680.12684083,46.33571871)(680.13684082,46.23572561)
\curveto(680.13684082,46.1457189)(680.13184083,46.05571899)(680.12184082,45.96572561)
\curveto(680.11184085,45.88571916)(680.09184087,45.82571922)(680.06184082,45.78572561)
\curveto(680.02184094,45.73571931)(679.956841,45.69071936)(679.86684082,45.65072561)
\curveto(679.82684113,45.64071941)(679.77184119,45.63071942)(679.70184082,45.62072561)
\curveto(679.63184133,45.62071943)(679.56684139,45.61571943)(679.50684082,45.60572561)
\curveto(679.43684152,45.59571945)(679.38184158,45.57571947)(679.34184082,45.54572561)
\curveto(679.30184166,45.51571953)(679.28684167,45.47071958)(679.29684082,45.41072561)
\curveto(679.31684164,45.33071972)(679.37684158,45.2507198)(679.47684082,45.17072561)
\curveto(679.56684139,45.09071996)(679.63684132,45.01572003)(679.68684082,44.94572561)
\curveto(679.84684111,44.72572032)(679.98684097,44.47572057)(680.10684082,44.19572561)
\curveto(680.1568408,44.08572096)(680.18684077,43.97072108)(680.19684082,43.85072561)
\curveto(680.21684074,43.74072131)(680.24184072,43.62572142)(680.27184082,43.50572561)
\curveto(680.28184068,43.45572159)(680.28184068,43.40072165)(680.27184082,43.34072561)
\curveto(680.2618407,43.29072176)(680.26684069,43.24072181)(680.28684082,43.19072561)
\curveto(680.30684065,43.09072196)(680.30684065,43.00072205)(680.28684082,42.92072561)
\lineto(680.28684082,42.77072561)
\curveto(680.26684069,42.72072233)(680.2568407,42.66072239)(680.25684082,42.59072561)
\curveto(680.2568407,42.53072252)(680.25184071,42.47572257)(680.24184082,42.42572561)
\curveto(680.22184074,42.38572266)(680.21184075,42.3457227)(680.21184082,42.30572561)
\curveto(680.22184074,42.27572277)(680.21684074,42.23572281)(680.19684082,42.18572561)
\lineto(680.13684082,41.94572561)
\curveto(680.11684084,41.87572317)(680.08684087,41.80072325)(680.04684082,41.72072561)
\curveto(679.93684102,41.46072359)(679.79184117,41.24072381)(679.61184082,41.06072561)
\curveto(679.42184154,40.89072416)(679.19684176,40.7507243)(678.93684082,40.64072561)
\curveto(678.84684211,40.60072445)(678.7568422,40.57072448)(678.66684082,40.55072561)
\lineto(678.36684082,40.49072561)
\curveto(678.30684265,40.47072458)(678.25184271,40.46072459)(678.20184082,40.46072561)
\curveto(678.14184282,40.47072458)(678.07684288,40.46572458)(678.00684082,40.44572561)
\curveto(677.98684297,40.43572461)(677.961843,40.43072462)(677.93184082,40.43072561)
\curveto(677.89184307,40.43072462)(677.8568431,40.42572462)(677.82684082,40.41572561)
\lineto(677.67684082,40.41572561)
\curveto(677.63684332,40.40572464)(677.59184337,40.40072465)(677.54184082,40.40072561)
\curveto(677.48184348,40.41072464)(677.42684353,40.41572463)(677.37684082,40.41572561)
\lineto(676.77684082,40.41572561)
\lineto(674.01684082,40.41572561)
\lineto(673.05684082,40.41572561)
\lineto(672.78684082,40.41572561)
\curveto(672.69684826,40.41572463)(672.62184834,40.43572461)(672.56184082,40.47572561)
\curveto(672.49184847,40.51572453)(672.44184852,40.59072446)(672.41184082,40.70072561)
\curveto(672.40184856,40.72072433)(672.40184856,40.74072431)(672.41184082,40.76072561)
\curveto(672.41184855,40.78072427)(672.40684855,40.80072425)(672.39684082,40.82072561)
}
}
{
\newrgbcolor{curcolor}{0 0 0}
\pscustom[linestyle=none,fillstyle=solid,fillcolor=curcolor]
{
\newpath
\moveto(672.24684082,52.39533498)
\curveto(672.22684873,53.02532975)(672.31184865,53.53032924)(672.50184082,53.91033498)
\curveto(672.69184827,54.29032848)(672.97684798,54.59532818)(673.35684082,54.82533498)
\curveto(673.4568475,54.88532789)(673.56684739,54.93032784)(673.68684082,54.96033498)
\curveto(673.79684716,55.00032777)(673.91184705,55.03532774)(674.03184082,55.06533498)
\curveto(674.22184674,55.11532766)(674.42684653,55.14532763)(674.64684082,55.15533498)
\curveto(674.86684609,55.16532761)(675.09184587,55.1703276)(675.32184082,55.17033498)
\lineto(676.92684082,55.17033498)
\lineto(679.26684082,55.17033498)
\curveto(679.43684152,55.1703276)(679.60684135,55.16532761)(679.77684082,55.15533498)
\curveto(679.94684101,55.15532762)(680.0568409,55.09032768)(680.10684082,54.96033498)
\curveto(680.12684083,54.91032786)(680.13684082,54.85532792)(680.13684082,54.79533498)
\curveto(680.14684081,54.74532803)(680.15184081,54.69032808)(680.15184082,54.63033498)
\curveto(680.15184081,54.50032827)(680.14684081,54.3753284)(680.13684082,54.25533498)
\curveto(680.13684082,54.13532864)(680.09684086,54.05032872)(680.01684082,54.00033498)
\curveto(679.94684101,53.95032882)(679.8568411,53.92532885)(679.74684082,53.92533498)
\lineto(679.41684082,53.92533498)
\lineto(678.12684082,53.92533498)
\lineto(675.68184082,53.92533498)
\curveto(675.41184555,53.92532885)(675.14684581,53.92032885)(674.88684082,53.91033498)
\curveto(674.61684634,53.90032887)(674.38684657,53.85532892)(674.19684082,53.77533498)
\curveto(673.99684696,53.69532908)(673.83684712,53.5753292)(673.71684082,53.41533498)
\curveto(673.58684737,53.25532952)(673.48684747,53.0703297)(673.41684082,52.86033498)
\curveto(673.39684756,52.80032997)(673.38684757,52.73533004)(673.38684082,52.66533498)
\curveto(673.37684758,52.60533017)(673.3618476,52.54533023)(673.34184082,52.48533498)
\curveto(673.33184763,52.43533034)(673.33184763,52.35533042)(673.34184082,52.24533498)
\curveto(673.34184762,52.14533063)(673.34684761,52.0753307)(673.35684082,52.03533498)
\curveto(673.37684758,51.99533078)(673.38684757,51.96033081)(673.38684082,51.93033498)
\curveto(673.37684758,51.90033087)(673.37684758,51.86533091)(673.38684082,51.82533498)
\curveto(673.41684754,51.69533108)(673.45184751,51.5703312)(673.49184082,51.45033498)
\curveto(673.52184744,51.34033143)(673.56684739,51.23533154)(673.62684082,51.13533498)
\curveto(673.64684731,51.09533168)(673.66684729,51.06033171)(673.68684082,51.03033498)
\curveto(673.70684725,51.00033177)(673.72684723,50.96533181)(673.74684082,50.92533498)
\curveto(673.99684696,50.5753322)(674.37184659,50.32033245)(674.87184082,50.16033498)
\curveto(674.95184601,50.13033264)(675.03684592,50.11033266)(675.12684082,50.10033498)
\curveto(675.20684575,50.09033268)(675.28684567,50.0753327)(675.36684082,50.05533498)
\curveto(675.41684554,50.03533274)(675.46684549,50.03033274)(675.51684082,50.04033498)
\curveto(675.5568454,50.05033272)(675.59684536,50.04533273)(675.63684082,50.02533498)
\lineto(675.95184082,50.02533498)
\curveto(675.98184498,50.01533276)(676.01684494,50.01033276)(676.05684082,50.01033498)
\curveto(676.09684486,50.02033275)(676.14184482,50.02533275)(676.19184082,50.02533498)
\lineto(676.64184082,50.02533498)
\lineto(678.08184082,50.02533498)
\lineto(679.40184082,50.02533498)
\lineto(679.74684082,50.02533498)
\curveto(679.8568411,50.02533275)(679.94684101,50.00033277)(680.01684082,49.95033498)
\curveto(680.09684086,49.90033287)(680.13684082,49.81033296)(680.13684082,49.68033498)
\curveto(680.14684081,49.56033321)(680.15184081,49.43533334)(680.15184082,49.30533498)
\curveto(680.15184081,49.22533355)(680.14684081,49.15033362)(680.13684082,49.08033498)
\curveto(680.12684083,49.01033376)(680.10184086,48.95033382)(680.06184082,48.90033498)
\curveto(680.01184095,48.82033395)(679.91684104,48.78033399)(679.77684082,48.78033498)
\lineto(679.37184082,48.78033498)
\lineto(677.60184082,48.78033498)
\lineto(673.97184082,48.78033498)
\lineto(673.05684082,48.78033498)
\lineto(672.78684082,48.78033498)
\curveto(672.69684826,48.78033399)(672.62684833,48.80033397)(672.57684082,48.84033498)
\curveto(672.51684844,48.8703339)(672.47684848,48.92033385)(672.45684082,48.99033498)
\curveto(672.44684851,49.03033374)(672.43684852,49.08533369)(672.42684082,49.15533498)
\curveto(672.41684854,49.23533354)(672.41184855,49.31533346)(672.41184082,49.39533498)
\curveto(672.41184855,49.4753333)(672.41684854,49.55033322)(672.42684082,49.62033498)
\curveto(672.43684852,49.70033307)(672.45184851,49.75533302)(672.47184082,49.78533498)
\curveto(672.54184842,49.89533288)(672.63184833,49.94533283)(672.74184082,49.93533498)
\curveto(672.84184812,49.92533285)(672.956848,49.94033283)(673.08684082,49.98033498)
\curveto(673.14684781,50.00033277)(673.19684776,50.04033273)(673.23684082,50.10033498)
\curveto(673.24684771,50.22033255)(673.20184776,50.31533246)(673.10184082,50.38533498)
\curveto(673.00184796,50.46533231)(672.92184804,50.54533223)(672.86184082,50.62533498)
\curveto(672.7618482,50.76533201)(672.67184829,50.90533187)(672.59184082,51.04533498)
\curveto(672.50184846,51.19533158)(672.42684853,51.36533141)(672.36684082,51.55533498)
\curveto(672.33684862,51.63533114)(672.31684864,51.72033105)(672.30684082,51.81033498)
\curveto(672.29684866,51.91033086)(672.28184868,52.00533077)(672.26184082,52.09533498)
\curveto(672.25184871,52.14533063)(672.24684871,52.19533058)(672.24684082,52.24533498)
\lineto(672.24684082,52.39533498)
}
}
{
\newrgbcolor{curcolor}{0 0 0}
\pscustom[linestyle=none,fillstyle=solid,fillcolor=curcolor]
{
}
}
{
\newrgbcolor{curcolor}{0 0 0}
\pscustom[linestyle=none,fillstyle=solid,fillcolor=curcolor]
{
\newpath
\moveto(675.03684082,67.99510061)
\lineto(675.29184082,67.99510061)
\curveto(675.37184559,68.0050929)(675.44684551,68.00009291)(675.51684082,67.98010061)
\lineto(675.75684082,67.98010061)
\lineto(675.92184082,67.98010061)
\curveto(676.02184494,67.96009295)(676.12684483,67.95009296)(676.23684082,67.95010061)
\curveto(676.33684462,67.95009296)(676.43684452,67.94009297)(676.53684082,67.92010061)
\lineto(676.68684082,67.92010061)
\curveto(676.82684413,67.89009302)(676.96684399,67.87009304)(677.10684082,67.86010061)
\curveto(677.23684372,67.85009306)(677.36684359,67.82509308)(677.49684082,67.78510061)
\curveto(677.57684338,67.76509314)(677.6618433,67.74509316)(677.75184082,67.72510061)
\lineto(677.99184082,67.66510061)
\lineto(678.29184082,67.54510061)
\curveto(678.38184258,67.51509339)(678.47184249,67.48009343)(678.56184082,67.44010061)
\curveto(678.78184218,67.34009357)(678.99684196,67.2050937)(679.20684082,67.03510061)
\curveto(679.41684154,66.87509403)(679.58684137,66.70009421)(679.71684082,66.51010061)
\curveto(679.7568412,66.46009445)(679.79684116,66.40009451)(679.83684082,66.33010061)
\curveto(679.86684109,66.27009464)(679.90184106,66.2100947)(679.94184082,66.15010061)
\curveto(679.99184097,66.07009484)(680.03184093,65.97509493)(680.06184082,65.86510061)
\curveto(680.09184087,65.75509515)(680.12184084,65.65009526)(680.15184082,65.55010061)
\curveto(680.19184077,65.44009547)(680.21684074,65.33009558)(680.22684082,65.22010061)
\curveto(680.23684072,65.1100958)(680.25184071,64.99509591)(680.27184082,64.87510061)
\curveto(680.28184068,64.83509607)(680.28184068,64.79009612)(680.27184082,64.74010061)
\curveto(680.27184069,64.70009621)(680.27684068,64.66009625)(680.28684082,64.62010061)
\curveto(680.29684066,64.58009633)(680.30184066,64.52509638)(680.30184082,64.45510061)
\curveto(680.30184066,64.38509652)(680.29684066,64.33509657)(680.28684082,64.30510061)
\curveto(680.26684069,64.25509665)(680.2618407,64.2100967)(680.27184082,64.17010061)
\curveto(680.28184068,64.13009678)(680.28184068,64.09509681)(680.27184082,64.06510061)
\lineto(680.27184082,63.97510061)
\curveto(680.25184071,63.91509699)(680.23684072,63.85009706)(680.22684082,63.78010061)
\curveto(680.22684073,63.72009719)(680.22184074,63.65509725)(680.21184082,63.58510061)
\curveto(680.1618408,63.41509749)(680.11184085,63.25509765)(680.06184082,63.10510061)
\curveto(680.01184095,62.95509795)(679.94684101,62.8100981)(679.86684082,62.67010061)
\curveto(679.82684113,62.62009829)(679.79684116,62.56509834)(679.77684082,62.50510061)
\curveto(679.74684121,62.45509845)(679.71184125,62.4050985)(679.67184082,62.35510061)
\curveto(679.49184147,62.11509879)(679.27184169,61.91509899)(679.01184082,61.75510061)
\curveto(678.75184221,61.59509931)(678.46684249,61.45509945)(678.15684082,61.33510061)
\curveto(678.01684294,61.27509963)(677.87684308,61.23009968)(677.73684082,61.20010061)
\curveto(677.58684337,61.17009974)(677.43184353,61.13509977)(677.27184082,61.09510061)
\curveto(677.1618438,61.07509983)(677.05184391,61.06009985)(676.94184082,61.05010061)
\curveto(676.83184413,61.04009987)(676.72184424,61.02509988)(676.61184082,61.00510061)
\curveto(676.57184439,60.99509991)(676.53184443,60.99009992)(676.49184082,60.99010061)
\curveto(676.45184451,61.00009991)(676.41184455,61.00009991)(676.37184082,60.99010061)
\curveto(676.32184464,60.98009993)(676.27184469,60.97509993)(676.22184082,60.97510061)
\lineto(676.05684082,60.97510061)
\curveto(676.00684495,60.95509995)(675.956845,60.95009996)(675.90684082,60.96010061)
\curveto(675.84684511,60.97009994)(675.79184517,60.97009994)(675.74184082,60.96010061)
\curveto(675.70184526,60.95009996)(675.6568453,60.95009996)(675.60684082,60.96010061)
\curveto(675.5568454,60.97009994)(675.50684545,60.96509994)(675.45684082,60.94510061)
\curveto(675.38684557,60.92509998)(675.31184565,60.92009999)(675.23184082,60.93010061)
\curveto(675.14184582,60.94009997)(675.0568459,60.94509996)(674.97684082,60.94510061)
\curveto(674.88684607,60.94509996)(674.78684617,60.94009997)(674.67684082,60.93010061)
\curveto(674.5568464,60.92009999)(674.4568465,60.92509998)(674.37684082,60.94510061)
\lineto(674.09184082,60.94510061)
\lineto(673.46184082,60.99010061)
\curveto(673.3618476,61.00009991)(673.26684769,61.0100999)(673.17684082,61.02010061)
\lineto(672.87684082,61.05010061)
\curveto(672.82684813,61.07009984)(672.77684818,61.07509983)(672.72684082,61.06510061)
\curveto(672.66684829,61.06509984)(672.61184835,61.07509983)(672.56184082,61.09510061)
\curveto(672.39184857,61.14509976)(672.22684873,61.18509972)(672.06684082,61.21510061)
\curveto(671.89684906,61.24509966)(671.73684922,61.29509961)(671.58684082,61.36510061)
\curveto(671.12684983,61.55509935)(670.75185021,61.77509913)(670.46184082,62.02510061)
\curveto(670.17185079,62.28509862)(669.92685103,62.64509826)(669.72684082,63.10510061)
\curveto(669.67685128,63.23509767)(669.64185132,63.36509754)(669.62184082,63.49510061)
\curveto(669.60185136,63.63509727)(669.57685138,63.77509713)(669.54684082,63.91510061)
\curveto(669.53685142,63.98509692)(669.53185143,64.05009686)(669.53184082,64.11010061)
\curveto(669.53185143,64.17009674)(669.52685143,64.23509667)(669.51684082,64.30510061)
\curveto(669.49685146,65.13509577)(669.64685131,65.8050951)(669.96684082,66.31510061)
\curveto(670.27685068,66.82509408)(670.71685024,67.2050937)(671.28684082,67.45510061)
\curveto(671.40684955,67.5050934)(671.53184943,67.55009336)(671.66184082,67.59010061)
\curveto(671.79184917,67.63009328)(671.92684903,67.67509323)(672.06684082,67.72510061)
\curveto(672.14684881,67.74509316)(672.23184873,67.76009315)(672.32184082,67.77010061)
\lineto(672.56184082,67.83010061)
\curveto(672.67184829,67.86009305)(672.78184818,67.87509303)(672.89184082,67.87510061)
\curveto(673.00184796,67.88509302)(673.11184785,67.90009301)(673.22184082,67.92010061)
\curveto(673.27184769,67.94009297)(673.31684764,67.94509296)(673.35684082,67.93510061)
\curveto(673.39684756,67.93509297)(673.43684752,67.94009297)(673.47684082,67.95010061)
\curveto(673.52684743,67.96009295)(673.58184738,67.96009295)(673.64184082,67.95010061)
\curveto(673.69184727,67.95009296)(673.74184722,67.95509295)(673.79184082,67.96510061)
\lineto(673.92684082,67.96510061)
\curveto(673.98684697,67.98509292)(674.0568469,67.98509292)(674.13684082,67.96510061)
\curveto(674.20684675,67.95509295)(674.27184669,67.96009295)(674.33184082,67.98010061)
\curveto(674.3618466,67.99009292)(674.40184656,67.99509291)(674.45184082,67.99510061)
\lineto(674.57184082,67.99510061)
\lineto(675.03684082,67.99510061)
\moveto(677.36184082,66.45010061)
\curveto(677.04184392,66.55009436)(676.67684428,66.6100943)(676.26684082,66.63010061)
\curveto(675.8568451,66.65009426)(675.44684551,66.66009425)(675.03684082,66.66010061)
\curveto(674.60684635,66.66009425)(674.18684677,66.65009426)(673.77684082,66.63010061)
\curveto(673.36684759,66.6100943)(672.98184798,66.56509434)(672.62184082,66.49510061)
\curveto(672.2618487,66.42509448)(671.94184902,66.31509459)(671.66184082,66.16510061)
\curveto(671.37184959,66.02509488)(671.13684982,65.83009508)(670.95684082,65.58010061)
\curveto(670.84685011,65.42009549)(670.76685019,65.24009567)(670.71684082,65.04010061)
\curveto(670.6568503,64.84009607)(670.62685033,64.59509631)(670.62684082,64.30510061)
\curveto(670.64685031,64.28509662)(670.6568503,64.25009666)(670.65684082,64.20010061)
\curveto(670.64685031,64.15009676)(670.64685031,64.1100968)(670.65684082,64.08010061)
\curveto(670.67685028,64.00009691)(670.69685026,63.92509698)(670.71684082,63.85510061)
\curveto(670.72685023,63.79509711)(670.74685021,63.73009718)(670.77684082,63.66010061)
\curveto(670.89685006,63.39009752)(671.06684989,63.17009774)(671.28684082,63.00010061)
\curveto(671.49684946,62.84009807)(671.74184922,62.7050982)(672.02184082,62.59510061)
\curveto(672.13184883,62.54509836)(672.25184871,62.5050984)(672.38184082,62.47510061)
\curveto(672.50184846,62.45509845)(672.62684833,62.43009848)(672.75684082,62.40010061)
\curveto(672.80684815,62.38009853)(672.8618481,62.37009854)(672.92184082,62.37010061)
\curveto(672.97184799,62.37009854)(673.02184794,62.36509854)(673.07184082,62.35510061)
\curveto(673.1618478,62.34509856)(673.2568477,62.33509857)(673.35684082,62.32510061)
\curveto(673.44684751,62.31509859)(673.54184742,62.3050986)(673.64184082,62.29510061)
\curveto(673.72184724,62.29509861)(673.80684715,62.29009862)(673.89684082,62.28010061)
\lineto(674.13684082,62.28010061)
\lineto(674.31684082,62.28010061)
\curveto(674.34684661,62.27009864)(674.38184658,62.26509864)(674.42184082,62.26510061)
\lineto(674.55684082,62.26510061)
\lineto(675.00684082,62.26510061)
\curveto(675.08684587,62.26509864)(675.17184579,62.26009865)(675.26184082,62.25010061)
\curveto(675.34184562,62.25009866)(675.41684554,62.26009865)(675.48684082,62.28010061)
\lineto(675.75684082,62.28010061)
\curveto(675.77684518,62.28009863)(675.80684515,62.27509863)(675.84684082,62.26510061)
\curveto(675.87684508,62.26509864)(675.90184506,62.27009864)(675.92184082,62.28010061)
\curveto(676.02184494,62.29009862)(676.12184484,62.29509861)(676.22184082,62.29510061)
\curveto(676.31184465,62.3050986)(676.41184455,62.31509859)(676.52184082,62.32510061)
\curveto(676.64184432,62.35509855)(676.76684419,62.37009854)(676.89684082,62.37010061)
\curveto(677.01684394,62.38009853)(677.13184383,62.4050985)(677.24184082,62.44510061)
\curveto(677.54184342,62.52509838)(677.80684315,62.6100983)(678.03684082,62.70010061)
\curveto(678.26684269,62.80009811)(678.48184248,62.94509796)(678.68184082,63.13510061)
\curveto(678.88184208,63.34509756)(679.03184193,63.6100973)(679.13184082,63.93010061)
\curveto(679.15184181,63.97009694)(679.1618418,64.0050969)(679.16184082,64.03510061)
\curveto(679.15184181,64.07509683)(679.1568418,64.12009679)(679.17684082,64.17010061)
\curveto(679.18684177,64.2100967)(679.19684176,64.28009663)(679.20684082,64.38010061)
\curveto(679.21684174,64.49009642)(679.21184175,64.57509633)(679.19184082,64.63510061)
\curveto(679.17184179,64.7050962)(679.1618418,64.77509613)(679.16184082,64.84510061)
\curveto(679.15184181,64.91509599)(679.13684182,64.98009593)(679.11684082,65.04010061)
\curveto(679.0568419,65.24009567)(678.97184199,65.42009549)(678.86184082,65.58010061)
\curveto(678.84184212,65.6100953)(678.82184214,65.63509527)(678.80184082,65.65510061)
\lineto(678.74184082,65.71510061)
\curveto(678.72184224,65.75509515)(678.68184228,65.8050951)(678.62184082,65.86510061)
\curveto(678.48184248,65.96509494)(678.35184261,66.05009486)(678.23184082,66.12010061)
\curveto(678.11184285,66.19009472)(677.96684299,66.26009465)(677.79684082,66.33010061)
\curveto(677.72684323,66.36009455)(677.6568433,66.38009453)(677.58684082,66.39010061)
\curveto(677.51684344,66.4100945)(677.44184352,66.43009448)(677.36184082,66.45010061)
}
}
{
\newrgbcolor{curcolor}{0 0 0}
\pscustom[linestyle=none,fillstyle=solid,fillcolor=curcolor]
{
\newpath
\moveto(675.80184082,76.34470998)
\curveto(675.92184504,76.37470226)(676.0618449,76.39970223)(676.22184082,76.41970998)
\curveto(676.38184458,76.43970219)(676.54684441,76.44970218)(676.71684082,76.44970998)
\curveto(676.88684407,76.44970218)(677.05184391,76.43970219)(677.21184082,76.41970998)
\curveto(677.37184359,76.39970223)(677.51184345,76.37470226)(677.63184082,76.34470998)
\curveto(677.77184319,76.30470233)(677.89684306,76.26970236)(678.00684082,76.23970998)
\curveto(678.11684284,76.20970242)(678.22684273,76.16970246)(678.33684082,76.11970998)
\curveto(678.97684198,75.84970278)(679.4618415,75.4347032)(679.79184082,74.87470998)
\curveto(679.85184111,74.79470384)(679.90184106,74.70970392)(679.94184082,74.61970998)
\curveto(679.97184099,74.5297041)(680.00684095,74.4297042)(680.04684082,74.31970998)
\curveto(680.09684086,74.20970442)(680.13184083,74.08970454)(680.15184082,73.95970998)
\curveto(680.18184078,73.83970479)(680.21184075,73.70970492)(680.24184082,73.56970998)
\curveto(680.2618407,73.50970512)(680.26684069,73.44970518)(680.25684082,73.38970998)
\curveto(680.24684071,73.33970529)(680.25184071,73.27970535)(680.27184082,73.20970998)
\curveto(680.28184068,73.18970544)(680.28184068,73.16470547)(680.27184082,73.13470998)
\curveto(680.27184069,73.10470553)(680.27684068,73.07970555)(680.28684082,73.05970998)
\lineto(680.28684082,72.90970998)
\curveto(680.29684066,72.83970579)(680.29684066,72.78970584)(680.28684082,72.75970998)
\curveto(680.27684068,72.71970591)(680.27184069,72.67470596)(680.27184082,72.62470998)
\curveto(680.28184068,72.58470605)(680.28184068,72.54470609)(680.27184082,72.50470998)
\curveto(680.25184071,72.41470622)(680.23684072,72.32470631)(680.22684082,72.23470998)
\curveto(680.22684073,72.14470649)(680.21684074,72.05470658)(680.19684082,71.96470998)
\curveto(680.16684079,71.87470676)(680.14184082,71.78470685)(680.12184082,71.69470998)
\curveto(680.10184086,71.60470703)(680.07184089,71.51970711)(680.03184082,71.43970998)
\curveto(679.92184104,71.19970743)(679.79184117,70.97470766)(679.64184082,70.76470998)
\curveto(679.48184148,70.55470808)(679.30184166,70.37470826)(679.10184082,70.22470998)
\curveto(678.93184203,70.10470853)(678.7568422,69.99970863)(678.57684082,69.90970998)
\curveto(678.39684256,69.81970881)(678.20684275,69.7297089)(678.00684082,69.63970998)
\curveto(677.90684305,69.59970903)(677.80684315,69.56470907)(677.70684082,69.53470998)
\curveto(677.59684336,69.51470912)(677.48684347,69.48970914)(677.37684082,69.45970998)
\curveto(677.23684372,69.41970921)(677.09684386,69.39470924)(676.95684082,69.38470998)
\curveto(676.81684414,69.37470926)(676.67684428,69.35470928)(676.53684082,69.32470998)
\curveto(676.42684453,69.31470932)(676.32684463,69.30470933)(676.23684082,69.29470998)
\curveto(676.13684482,69.29470934)(676.03684492,69.28470935)(675.93684082,69.26470998)
\lineto(675.84684082,69.26470998)
\curveto(675.81684514,69.27470936)(675.79184517,69.27470936)(675.77184082,69.26470998)
\lineto(675.56184082,69.26470998)
\curveto(675.50184546,69.24470939)(675.43684552,69.2347094)(675.36684082,69.23470998)
\curveto(675.28684567,69.24470939)(675.21184575,69.24970938)(675.14184082,69.24970998)
\lineto(674.99184082,69.24970998)
\curveto(674.94184602,69.24970938)(674.89184607,69.25470938)(674.84184082,69.26470998)
\lineto(674.46684082,69.26470998)
\curveto(674.43684652,69.27470936)(674.40184656,69.27470936)(674.36184082,69.26470998)
\curveto(674.32184664,69.26470937)(674.28184668,69.26970936)(674.24184082,69.27970998)
\curveto(674.13184683,69.29970933)(674.02184694,69.31470932)(673.91184082,69.32470998)
\curveto(673.79184717,69.3347093)(673.67684728,69.34470929)(673.56684082,69.35470998)
\curveto(673.41684754,69.39470924)(673.27184769,69.41970921)(673.13184082,69.42970998)
\curveto(672.98184798,69.44970918)(672.83684812,69.47970915)(672.69684082,69.51970998)
\curveto(672.39684856,69.60970902)(672.11184885,69.70470893)(671.84184082,69.80470998)
\curveto(671.57184939,69.90470873)(671.32184964,70.0297086)(671.09184082,70.17970998)
\curveto(670.77185019,70.37970825)(670.49185047,70.62470801)(670.25184082,70.91470998)
\curveto(670.01185095,71.20470743)(669.82685113,71.54470709)(669.69684082,71.93470998)
\curveto(669.6568513,72.04470659)(669.63185133,72.15470648)(669.62184082,72.26470998)
\curveto(669.60185136,72.38470625)(669.57685138,72.50470613)(669.54684082,72.62470998)
\curveto(669.53685142,72.69470594)(669.53185143,72.75970587)(669.53184082,72.81970998)
\curveto(669.53185143,72.87970575)(669.52685143,72.94470569)(669.51684082,73.01470998)
\curveto(669.49685146,73.71470492)(669.61185135,74.28970434)(669.86184082,74.73970998)
\curveto(670.11185085,75.18970344)(670.4618505,75.5347031)(670.91184082,75.77470998)
\curveto(671.14184982,75.88470275)(671.41684954,75.98470265)(671.73684082,76.07470998)
\curveto(671.80684915,76.09470254)(671.88184908,76.09470254)(671.96184082,76.07470998)
\curveto(672.03184893,76.06470257)(672.08184888,76.03970259)(672.11184082,75.99970998)
\curveto(672.14184882,75.96970266)(672.16684879,75.90970272)(672.18684082,75.81970998)
\curveto(672.19684876,75.7297029)(672.20684875,75.629703)(672.21684082,75.51970998)
\curveto(672.21684874,75.41970321)(672.21184875,75.31970331)(672.20184082,75.21970998)
\curveto(672.19184877,75.1297035)(672.17184879,75.06470357)(672.14184082,75.02470998)
\curveto(672.07184889,74.91470372)(671.961849,74.8347038)(671.81184082,74.78470998)
\curveto(671.6618493,74.74470389)(671.53184943,74.68970394)(671.42184082,74.61970998)
\curveto(671.11184985,74.4297042)(670.88185008,74.14970448)(670.73184082,73.77970998)
\curveto(670.70185026,73.70970492)(670.68185028,73.634705)(670.67184082,73.55470998)
\curveto(670.6618503,73.48470515)(670.64685031,73.40970522)(670.62684082,73.32970998)
\curveto(670.61685034,73.27970535)(670.61185035,73.20970542)(670.61184082,73.11970998)
\curveto(670.61185035,73.03970559)(670.61685034,72.97470566)(670.62684082,72.92470998)
\curveto(670.64685031,72.88470575)(670.65185031,72.84970578)(670.64184082,72.81970998)
\curveto(670.63185033,72.78970584)(670.63185033,72.75470588)(670.64184082,72.71470998)
\lineto(670.70184082,72.47470998)
\curveto(670.72185024,72.40470623)(670.74685021,72.3347063)(670.77684082,72.26470998)
\curveto(670.93685002,71.88470675)(671.14684981,71.59470704)(671.40684082,71.39470998)
\curveto(671.66684929,71.20470743)(671.98184898,71.0297076)(672.35184082,70.86970998)
\curveto(672.43184853,70.83970779)(672.51184845,70.81470782)(672.59184082,70.79470998)
\curveto(672.67184829,70.78470785)(672.75184821,70.76470787)(672.83184082,70.73470998)
\curveto(672.94184802,70.70470793)(673.0568479,70.67970795)(673.17684082,70.65970998)
\curveto(673.29684766,70.64970798)(673.41684754,70.629708)(673.53684082,70.59970998)
\curveto(673.58684737,70.57970805)(673.63684732,70.56970806)(673.68684082,70.56970998)
\curveto(673.73684722,70.57970805)(673.78684717,70.57470806)(673.83684082,70.55470998)
\curveto(673.89684706,70.54470809)(673.97684698,70.54470809)(674.07684082,70.55470998)
\curveto(674.16684679,70.56470807)(674.22184674,70.57970805)(674.24184082,70.59970998)
\curveto(674.28184668,70.61970801)(674.30184666,70.64970798)(674.30184082,70.68970998)
\curveto(674.30184666,70.73970789)(674.29184667,70.78470785)(674.27184082,70.82470998)
\curveto(674.23184673,70.89470774)(674.18684677,70.95470768)(674.13684082,71.00470998)
\curveto(674.08684687,71.05470758)(674.03684692,71.11470752)(673.98684082,71.18470998)
\lineto(673.92684082,71.24470998)
\curveto(673.89684706,71.27470736)(673.87184709,71.30470733)(673.85184082,71.33470998)
\curveto(673.69184727,71.56470707)(673.5568474,71.83970679)(673.44684082,72.15970998)
\curveto(673.42684753,72.2297064)(673.41184755,72.29970633)(673.40184082,72.36970998)
\curveto(673.39184757,72.43970619)(673.37684758,72.51470612)(673.35684082,72.59470998)
\curveto(673.3568476,72.634706)(673.35184761,72.66970596)(673.34184082,72.69970998)
\curveto(673.33184763,72.7297059)(673.33184763,72.76470587)(673.34184082,72.80470998)
\curveto(673.34184762,72.85470578)(673.33184763,72.89470574)(673.31184082,72.92470998)
\lineto(673.31184082,73.08970998)
\lineto(673.31184082,73.17970998)
\curveto(673.30184766,73.2297054)(673.30184766,73.26970536)(673.31184082,73.29970998)
\curveto(673.32184764,73.34970528)(673.32684763,73.39970523)(673.32684082,73.44970998)
\curveto(673.31684764,73.50970512)(673.31684764,73.56470507)(673.32684082,73.61470998)
\curveto(673.3568476,73.72470491)(673.37684758,73.8297048)(673.38684082,73.92970998)
\curveto(673.39684756,74.03970459)(673.42184754,74.14470449)(673.46184082,74.24470998)
\curveto(673.60184736,74.66470397)(673.78684717,75.00970362)(674.01684082,75.27970998)
\curveto(674.23684672,75.54970308)(674.52184644,75.78970284)(674.87184082,75.99970998)
\curveto(675.01184595,76.07970255)(675.1618458,76.14470249)(675.32184082,76.19470998)
\curveto(675.47184549,76.24470239)(675.63184533,76.29470234)(675.80184082,76.34470998)
\moveto(677.10684082,75.09970998)
\curveto(677.0568439,75.10970352)(677.01184395,75.11470352)(676.97184082,75.11470998)
\lineto(676.82184082,75.11470998)
\curveto(676.51184445,75.11470352)(676.22684473,75.07470356)(675.96684082,74.99470998)
\curveto(675.90684505,74.97470366)(675.85184511,74.95470368)(675.80184082,74.93470998)
\curveto(675.74184522,74.92470371)(675.68684527,74.90970372)(675.63684082,74.88970998)
\curveto(675.14684581,74.66970396)(674.79684616,74.32470431)(674.58684082,73.85470998)
\curveto(674.5568464,73.77470486)(674.53184643,73.69470494)(674.51184082,73.61470998)
\lineto(674.45184082,73.37470998)
\curveto(674.43184653,73.29470534)(674.42184654,73.20470543)(674.42184082,73.10470998)
\lineto(674.42184082,72.78970998)
\curveto(674.44184652,72.76970586)(674.45184651,72.7297059)(674.45184082,72.66970998)
\curveto(674.44184652,72.61970601)(674.44184652,72.57470606)(674.45184082,72.53470998)
\lineto(674.51184082,72.29470998)
\curveto(674.52184644,72.22470641)(674.54184642,72.15470648)(674.57184082,72.08470998)
\curveto(674.83184613,71.48470715)(675.29684566,71.07970755)(675.96684082,70.86970998)
\curveto(676.04684491,70.83970779)(676.12684483,70.81970781)(676.20684082,70.80970998)
\curveto(676.28684467,70.79970783)(676.37184459,70.78470785)(676.46184082,70.76470998)
\lineto(676.61184082,70.76470998)
\curveto(676.65184431,70.75470788)(676.72184424,70.74970788)(676.82184082,70.74970998)
\curveto(677.05184391,70.74970788)(677.24684371,70.76970786)(677.40684082,70.80970998)
\curveto(677.47684348,70.8297078)(677.54184342,70.84470779)(677.60184082,70.85470998)
\curveto(677.6618433,70.86470777)(677.72684323,70.88470775)(677.79684082,70.91470998)
\curveto(678.07684288,71.02470761)(678.32184264,71.16970746)(678.53184082,71.34970998)
\curveto(678.73184223,71.5297071)(678.89184207,71.76470687)(679.01184082,72.05470998)
\lineto(679.10184082,72.29470998)
\lineto(679.16184082,72.53470998)
\curveto(679.18184178,72.58470605)(679.18684177,72.62470601)(679.17684082,72.65470998)
\curveto(679.16684179,72.69470594)(679.17184179,72.73970589)(679.19184082,72.78970998)
\curveto(679.20184176,72.81970581)(679.20684175,72.87470576)(679.20684082,72.95470998)
\curveto(679.20684175,73.0347056)(679.20184176,73.09470554)(679.19184082,73.13470998)
\curveto(679.17184179,73.24470539)(679.1568418,73.34970528)(679.14684082,73.44970998)
\curveto(679.13684182,73.54970508)(679.10684185,73.64470499)(679.05684082,73.73470998)
\curveto(678.8568421,74.26470437)(678.48184248,74.65470398)(677.93184082,74.90470998)
\curveto(677.83184313,74.94470369)(677.72684323,74.97470366)(677.61684082,74.99470998)
\lineto(677.28684082,75.08470998)
\curveto(677.20684375,75.08470355)(677.14684381,75.08970354)(677.10684082,75.09970998)
}
}
{
\newrgbcolor{curcolor}{0 0 0}
\pscustom[linestyle=none,fillstyle=solid,fillcolor=curcolor]
{
\newpath
\moveto(678.48684082,78.63431936)
\lineto(678.48684082,79.26431936)
\lineto(678.48684082,79.45931936)
\curveto(678.48684247,79.52931683)(678.49684246,79.58931677)(678.51684082,79.63931936)
\curveto(678.5568424,79.70931665)(678.59684236,79.7593166)(678.63684082,79.78931936)
\curveto(678.68684227,79.82931653)(678.75184221,79.84931651)(678.83184082,79.84931936)
\curveto(678.91184205,79.8593165)(678.99684196,79.86431649)(679.08684082,79.86431936)
\lineto(679.80684082,79.86431936)
\curveto(680.28684067,79.86431649)(680.69684026,79.80431655)(681.03684082,79.68431936)
\curveto(681.37683958,79.56431679)(681.65183931,79.36931699)(681.86184082,79.09931936)
\curveto(681.91183905,79.02931733)(681.956839,78.9593174)(681.99684082,78.88931936)
\curveto(682.04683891,78.82931753)(682.09183887,78.7543176)(682.13184082,78.66431936)
\curveto(682.14183882,78.64431771)(682.15183881,78.61431774)(682.16184082,78.57431936)
\curveto(682.18183878,78.53431782)(682.18683877,78.48931787)(682.17684082,78.43931936)
\curveto(682.14683881,78.34931801)(682.07183889,78.29431806)(681.95184082,78.27431936)
\curveto(681.84183912,78.2543181)(681.74683921,78.26931809)(681.66684082,78.31931936)
\curveto(681.59683936,78.34931801)(681.53183943,78.39431796)(681.47184082,78.45431936)
\curveto(681.42183954,78.52431783)(681.37183959,78.58931777)(681.32184082,78.64931936)
\curveto(681.27183969,78.71931764)(681.19683976,78.77931758)(681.09684082,78.82931936)
\curveto(681.00683995,78.88931747)(680.91684004,78.93931742)(680.82684082,78.97931936)
\curveto(680.79684016,78.99931736)(680.73684022,79.02431733)(680.64684082,79.05431936)
\curveto(680.56684039,79.08431727)(680.49684046,79.08931727)(680.43684082,79.06931936)
\curveto(680.29684066,79.03931732)(680.20684075,78.97931738)(680.16684082,78.88931936)
\curveto(680.13684082,78.80931755)(680.12184084,78.71931764)(680.12184082,78.61931936)
\curveto(680.12184084,78.51931784)(680.09684086,78.43431792)(680.04684082,78.36431936)
\curveto(679.97684098,78.27431808)(679.83684112,78.22931813)(679.62684082,78.22931936)
\lineto(679.07184082,78.22931936)
\lineto(678.84684082,78.22931936)
\curveto(678.76684219,78.23931812)(678.70184226,78.2593181)(678.65184082,78.28931936)
\curveto(678.57184239,78.34931801)(678.52684243,78.41931794)(678.51684082,78.49931936)
\curveto(678.50684245,78.51931784)(678.50184246,78.53931782)(678.50184082,78.55931936)
\curveto(678.50184246,78.58931777)(678.49684246,78.61431774)(678.48684082,78.63431936)
}
}
{
\newrgbcolor{curcolor}{0 0 0}
\pscustom[linestyle=none,fillstyle=solid,fillcolor=curcolor]
{
}
}
{
\newrgbcolor{curcolor}{0 0 0}
\pscustom[linestyle=none,fillstyle=solid,fillcolor=curcolor]
{
\newpath
\moveto(669.51684082,89.26463186)
\curveto(669.50685145,89.95462722)(669.62685133,90.55462662)(669.87684082,91.06463186)
\curveto(670.12685083,91.58462559)(670.4618505,91.9796252)(670.88184082,92.24963186)
\curveto(670.96185,92.29962488)(671.05184991,92.34462483)(671.15184082,92.38463186)
\curveto(671.24184972,92.42462475)(671.33684962,92.46962471)(671.43684082,92.51963186)
\curveto(671.53684942,92.55962462)(671.63684932,92.58962459)(671.73684082,92.60963186)
\curveto(671.83684912,92.62962455)(671.94184902,92.64962453)(672.05184082,92.66963186)
\curveto(672.10184886,92.68962449)(672.14684881,92.69462448)(672.18684082,92.68463186)
\curveto(672.22684873,92.6746245)(672.27184869,92.6796245)(672.32184082,92.69963186)
\curveto(672.37184859,92.70962447)(672.4568485,92.71462446)(672.57684082,92.71463186)
\curveto(672.68684827,92.71462446)(672.77184819,92.70962447)(672.83184082,92.69963186)
\curveto(672.89184807,92.6796245)(672.95184801,92.66962451)(673.01184082,92.66963186)
\curveto(673.07184789,92.6796245)(673.13184783,92.6746245)(673.19184082,92.65463186)
\curveto(673.33184763,92.61462456)(673.46684749,92.5796246)(673.59684082,92.54963186)
\curveto(673.72684723,92.51962466)(673.85184711,92.4796247)(673.97184082,92.42963186)
\curveto(674.11184685,92.36962481)(674.23684672,92.29962488)(674.34684082,92.21963186)
\curveto(674.4568465,92.14962503)(674.56684639,92.0746251)(674.67684082,91.99463186)
\lineto(674.73684082,91.93463186)
\curveto(674.7568462,91.92462525)(674.77684618,91.90962527)(674.79684082,91.88963186)
\curveto(674.956846,91.76962541)(675.10184586,91.63462554)(675.23184082,91.48463186)
\curveto(675.3618456,91.33462584)(675.48684547,91.174626)(675.60684082,91.00463186)
\curveto(675.82684513,90.69462648)(676.03184493,90.39962678)(676.22184082,90.11963186)
\curveto(676.3618446,89.88962729)(676.49684446,89.65962752)(676.62684082,89.42963186)
\curveto(676.7568442,89.20962797)(676.89184407,88.98962819)(677.03184082,88.76963186)
\curveto(677.20184376,88.51962866)(677.38184358,88.2796289)(677.57184082,88.04963186)
\curveto(677.7618432,87.82962935)(677.98684297,87.63962954)(678.24684082,87.47963186)
\curveto(678.30684265,87.43962974)(678.36684259,87.40462977)(678.42684082,87.37463186)
\curveto(678.47684248,87.34462983)(678.54184242,87.31462986)(678.62184082,87.28463186)
\curveto(678.69184227,87.26462991)(678.75184221,87.25962992)(678.80184082,87.26963186)
\curveto(678.87184209,87.28962989)(678.92684203,87.32462985)(678.96684082,87.37463186)
\curveto(678.99684196,87.42462975)(679.01684194,87.48462969)(679.02684082,87.55463186)
\lineto(679.02684082,87.79463186)
\lineto(679.02684082,88.54463186)
\lineto(679.02684082,91.34963186)
\lineto(679.02684082,92.00963186)
\curveto(679.02684193,92.09962508)(679.03184193,92.18462499)(679.04184082,92.26463186)
\curveto(679.04184192,92.34462483)(679.0618419,92.40962477)(679.10184082,92.45963186)
\curveto(679.14184182,92.50962467)(679.21684174,92.54962463)(679.32684082,92.57963186)
\curveto(679.42684153,92.61962456)(679.52684143,92.62962455)(679.62684082,92.60963186)
\lineto(679.76184082,92.60963186)
\curveto(679.83184113,92.58962459)(679.89184107,92.56962461)(679.94184082,92.54963186)
\curveto(679.99184097,92.52962465)(680.03184093,92.49462468)(680.06184082,92.44463186)
\curveto(680.10184086,92.39462478)(680.12184084,92.32462485)(680.12184082,92.23463186)
\lineto(680.12184082,91.96463186)
\lineto(680.12184082,91.06463186)
\lineto(680.12184082,87.55463186)
\lineto(680.12184082,86.48963186)
\curveto(680.12184084,86.40963077)(680.12684083,86.31963086)(680.13684082,86.21963186)
\curveto(680.13684082,86.11963106)(680.12684083,86.03463114)(680.10684082,85.96463186)
\curveto(680.03684092,85.75463142)(679.8568411,85.68963149)(679.56684082,85.76963186)
\curveto(679.52684143,85.7796314)(679.49184147,85.7796314)(679.46184082,85.76963186)
\curveto(679.42184154,85.76963141)(679.37684158,85.7796314)(679.32684082,85.79963186)
\curveto(679.24684171,85.81963136)(679.1618418,85.83963134)(679.07184082,85.85963186)
\curveto(678.98184198,85.8796313)(678.89684206,85.90463127)(678.81684082,85.93463186)
\curveto(678.32684263,86.09463108)(677.91184305,86.29463088)(677.57184082,86.53463186)
\curveto(677.32184364,86.71463046)(677.09684386,86.91963026)(676.89684082,87.14963186)
\curveto(676.68684427,87.3796298)(676.49184447,87.61962956)(676.31184082,87.86963186)
\curveto(676.13184483,88.12962905)(675.961845,88.39462878)(675.80184082,88.66463186)
\curveto(675.63184533,88.94462823)(675.4568455,89.21462796)(675.27684082,89.47463186)
\curveto(675.19684576,89.58462759)(675.12184584,89.68962749)(675.05184082,89.78963186)
\curveto(674.98184598,89.89962728)(674.90684605,90.00962717)(674.82684082,90.11963186)
\curveto(674.79684616,90.15962702)(674.76684619,90.19462698)(674.73684082,90.22463186)
\curveto(674.69684626,90.26462691)(674.66684629,90.30462687)(674.64684082,90.34463186)
\curveto(674.53684642,90.48462669)(674.41184655,90.60962657)(674.27184082,90.71963186)
\curveto(674.24184672,90.73962644)(674.21684674,90.76462641)(674.19684082,90.79463186)
\curveto(674.16684679,90.82462635)(674.13684682,90.84962633)(674.10684082,90.86963186)
\curveto(674.00684695,90.94962623)(673.90684705,91.01462616)(673.80684082,91.06463186)
\curveto(673.70684725,91.12462605)(673.59684736,91.179626)(673.47684082,91.22963186)
\curveto(673.40684755,91.25962592)(673.33184763,91.2796259)(673.25184082,91.28963186)
\lineto(673.01184082,91.34963186)
\lineto(672.92184082,91.34963186)
\curveto(672.89184807,91.35962582)(672.8618481,91.36462581)(672.83184082,91.36463186)
\curveto(672.7618482,91.38462579)(672.66684829,91.38962579)(672.54684082,91.37963186)
\curveto(672.41684854,91.3796258)(672.31684864,91.36962581)(672.24684082,91.34963186)
\curveto(672.16684879,91.32962585)(672.09184887,91.30962587)(672.02184082,91.28963186)
\curveto(671.94184902,91.2796259)(671.8618491,91.25962592)(671.78184082,91.22963186)
\curveto(671.54184942,91.11962606)(671.34184962,90.96962621)(671.18184082,90.77963186)
\curveto(671.01184995,90.59962658)(670.87185009,90.3796268)(670.76184082,90.11963186)
\curveto(670.74185022,90.04962713)(670.72685023,89.9796272)(670.71684082,89.90963186)
\curveto(670.69685026,89.83962734)(670.67685028,89.76462741)(670.65684082,89.68463186)
\curveto(670.63685032,89.60462757)(670.62685033,89.49462768)(670.62684082,89.35463186)
\curveto(670.62685033,89.22462795)(670.63685032,89.11962806)(670.65684082,89.03963186)
\curveto(670.66685029,88.9796282)(670.67185029,88.92462825)(670.67184082,88.87463186)
\curveto(670.67185029,88.82462835)(670.68185028,88.7746284)(670.70184082,88.72463186)
\curveto(670.74185022,88.62462855)(670.78185018,88.52962865)(670.82184082,88.43963186)
\curveto(670.8618501,88.35962882)(670.90685005,88.2796289)(670.95684082,88.19963186)
\curveto(670.97684998,88.16962901)(671.00184996,88.13962904)(671.03184082,88.10963186)
\curveto(671.0618499,88.08962909)(671.08684987,88.06462911)(671.10684082,88.03463186)
\lineto(671.18184082,87.95963186)
\curveto(671.20184976,87.92962925)(671.22184974,87.90462927)(671.24184082,87.88463186)
\lineto(671.45184082,87.73463186)
\curveto(671.51184945,87.69462948)(671.57684938,87.64962953)(671.64684082,87.59963186)
\curveto(671.73684922,87.53962964)(671.84184912,87.48962969)(671.96184082,87.44963186)
\curveto(672.07184889,87.41962976)(672.18184878,87.38462979)(672.29184082,87.34463186)
\curveto(672.40184856,87.30462987)(672.54684841,87.2796299)(672.72684082,87.26963186)
\curveto(672.89684806,87.25962992)(673.02184794,87.22962995)(673.10184082,87.17963186)
\curveto(673.18184778,87.12963005)(673.22684773,87.05463012)(673.23684082,86.95463186)
\curveto(673.24684771,86.85463032)(673.25184771,86.74463043)(673.25184082,86.62463186)
\curveto(673.25184771,86.58463059)(673.2568477,86.54463063)(673.26684082,86.50463186)
\curveto(673.26684769,86.46463071)(673.2618477,86.42963075)(673.25184082,86.39963186)
\curveto(673.23184773,86.34963083)(673.22184774,86.29963088)(673.22184082,86.24963186)
\curveto(673.22184774,86.20963097)(673.21184775,86.16963101)(673.19184082,86.12963186)
\curveto(673.13184783,86.03963114)(672.99684796,85.99463118)(672.78684082,85.99463186)
\lineto(672.66684082,85.99463186)
\curveto(672.60684835,86.00463117)(672.54684841,86.00963117)(672.48684082,86.00963186)
\curveto(672.41684854,86.01963116)(672.35184861,86.02963115)(672.29184082,86.03963186)
\curveto(672.18184878,86.05963112)(672.08184888,86.0796311)(671.99184082,86.09963186)
\curveto(671.89184907,86.11963106)(671.79684916,86.14963103)(671.70684082,86.18963186)
\curveto(671.63684932,86.20963097)(671.57684938,86.22963095)(671.52684082,86.24963186)
\lineto(671.34684082,86.30963186)
\curveto(671.08684987,86.42963075)(670.84185012,86.58463059)(670.61184082,86.77463186)
\curveto(670.38185058,86.9746302)(670.19685076,87.18962999)(670.05684082,87.41963186)
\curveto(669.97685098,87.52962965)(669.91185105,87.64462953)(669.86184082,87.76463186)
\lineto(669.71184082,88.15463186)
\curveto(669.6618513,88.26462891)(669.63185133,88.3796288)(669.62184082,88.49963186)
\curveto(669.60185136,88.61962856)(669.57685138,88.74462843)(669.54684082,88.87463186)
\curveto(669.54685141,88.94462823)(669.54685141,89.00962817)(669.54684082,89.06963186)
\curveto(669.53685142,89.12962805)(669.52685143,89.19462798)(669.51684082,89.26463186)
}
}
{
\newrgbcolor{curcolor}{0 0 0}
\pscustom[linestyle=none,fillstyle=solid,fillcolor=curcolor]
{
\newpath
\moveto(675.03684082,101.36424123)
\lineto(675.29184082,101.36424123)
\curveto(675.37184559,101.37423353)(675.44684551,101.36923353)(675.51684082,101.34924123)
\lineto(675.75684082,101.34924123)
\lineto(675.92184082,101.34924123)
\curveto(676.02184494,101.32923357)(676.12684483,101.31923358)(676.23684082,101.31924123)
\curveto(676.33684462,101.31923358)(676.43684452,101.30923359)(676.53684082,101.28924123)
\lineto(676.68684082,101.28924123)
\curveto(676.82684413,101.25923364)(676.96684399,101.23923366)(677.10684082,101.22924123)
\curveto(677.23684372,101.21923368)(677.36684359,101.19423371)(677.49684082,101.15424123)
\curveto(677.57684338,101.13423377)(677.6618433,101.11423379)(677.75184082,101.09424123)
\lineto(677.99184082,101.03424123)
\lineto(678.29184082,100.91424123)
\curveto(678.38184258,100.88423402)(678.47184249,100.84923405)(678.56184082,100.80924123)
\curveto(678.78184218,100.70923419)(678.99684196,100.57423433)(679.20684082,100.40424123)
\curveto(679.41684154,100.24423466)(679.58684137,100.06923483)(679.71684082,99.87924123)
\curveto(679.7568412,99.82923507)(679.79684116,99.76923513)(679.83684082,99.69924123)
\curveto(679.86684109,99.63923526)(679.90184106,99.57923532)(679.94184082,99.51924123)
\curveto(679.99184097,99.43923546)(680.03184093,99.34423556)(680.06184082,99.23424123)
\curveto(680.09184087,99.12423578)(680.12184084,99.01923588)(680.15184082,98.91924123)
\curveto(680.19184077,98.80923609)(680.21684074,98.6992362)(680.22684082,98.58924123)
\curveto(680.23684072,98.47923642)(680.25184071,98.36423654)(680.27184082,98.24424123)
\curveto(680.28184068,98.2042367)(680.28184068,98.15923674)(680.27184082,98.10924123)
\curveto(680.27184069,98.06923683)(680.27684068,98.02923687)(680.28684082,97.98924123)
\curveto(680.29684066,97.94923695)(680.30184066,97.89423701)(680.30184082,97.82424123)
\curveto(680.30184066,97.75423715)(680.29684066,97.7042372)(680.28684082,97.67424123)
\curveto(680.26684069,97.62423728)(680.2618407,97.57923732)(680.27184082,97.53924123)
\curveto(680.28184068,97.4992374)(680.28184068,97.46423744)(680.27184082,97.43424123)
\lineto(680.27184082,97.34424123)
\curveto(680.25184071,97.28423762)(680.23684072,97.21923768)(680.22684082,97.14924123)
\curveto(680.22684073,97.08923781)(680.22184074,97.02423788)(680.21184082,96.95424123)
\curveto(680.1618408,96.78423812)(680.11184085,96.62423828)(680.06184082,96.47424123)
\curveto(680.01184095,96.32423858)(679.94684101,96.17923872)(679.86684082,96.03924123)
\curveto(679.82684113,95.98923891)(679.79684116,95.93423897)(679.77684082,95.87424123)
\curveto(679.74684121,95.82423908)(679.71184125,95.77423913)(679.67184082,95.72424123)
\curveto(679.49184147,95.48423942)(679.27184169,95.28423962)(679.01184082,95.12424123)
\curveto(678.75184221,94.96423994)(678.46684249,94.82424008)(678.15684082,94.70424123)
\curveto(678.01684294,94.64424026)(677.87684308,94.5992403)(677.73684082,94.56924123)
\curveto(677.58684337,94.53924036)(677.43184353,94.5042404)(677.27184082,94.46424123)
\curveto(677.1618438,94.44424046)(677.05184391,94.42924047)(676.94184082,94.41924123)
\curveto(676.83184413,94.40924049)(676.72184424,94.39424051)(676.61184082,94.37424123)
\curveto(676.57184439,94.36424054)(676.53184443,94.35924054)(676.49184082,94.35924123)
\curveto(676.45184451,94.36924053)(676.41184455,94.36924053)(676.37184082,94.35924123)
\curveto(676.32184464,94.34924055)(676.27184469,94.34424056)(676.22184082,94.34424123)
\lineto(676.05684082,94.34424123)
\curveto(676.00684495,94.32424058)(675.956845,94.31924058)(675.90684082,94.32924123)
\curveto(675.84684511,94.33924056)(675.79184517,94.33924056)(675.74184082,94.32924123)
\curveto(675.70184526,94.31924058)(675.6568453,94.31924058)(675.60684082,94.32924123)
\curveto(675.5568454,94.33924056)(675.50684545,94.33424057)(675.45684082,94.31424123)
\curveto(675.38684557,94.29424061)(675.31184565,94.28924061)(675.23184082,94.29924123)
\curveto(675.14184582,94.30924059)(675.0568459,94.31424059)(674.97684082,94.31424123)
\curveto(674.88684607,94.31424059)(674.78684617,94.30924059)(674.67684082,94.29924123)
\curveto(674.5568464,94.28924061)(674.4568465,94.29424061)(674.37684082,94.31424123)
\lineto(674.09184082,94.31424123)
\lineto(673.46184082,94.35924123)
\curveto(673.3618476,94.36924053)(673.26684769,94.37924052)(673.17684082,94.38924123)
\lineto(672.87684082,94.41924123)
\curveto(672.82684813,94.43924046)(672.77684818,94.44424046)(672.72684082,94.43424123)
\curveto(672.66684829,94.43424047)(672.61184835,94.44424046)(672.56184082,94.46424123)
\curveto(672.39184857,94.51424039)(672.22684873,94.55424035)(672.06684082,94.58424123)
\curveto(671.89684906,94.61424029)(671.73684922,94.66424024)(671.58684082,94.73424123)
\curveto(671.12684983,94.92423998)(670.75185021,95.14423976)(670.46184082,95.39424123)
\curveto(670.17185079,95.65423925)(669.92685103,96.01423889)(669.72684082,96.47424123)
\curveto(669.67685128,96.6042383)(669.64185132,96.73423817)(669.62184082,96.86424123)
\curveto(669.60185136,97.0042379)(669.57685138,97.14423776)(669.54684082,97.28424123)
\curveto(669.53685142,97.35423755)(669.53185143,97.41923748)(669.53184082,97.47924123)
\curveto(669.53185143,97.53923736)(669.52685143,97.6042373)(669.51684082,97.67424123)
\curveto(669.49685146,98.5042364)(669.64685131,99.17423573)(669.96684082,99.68424123)
\curveto(670.27685068,100.19423471)(670.71685024,100.57423433)(671.28684082,100.82424123)
\curveto(671.40684955,100.87423403)(671.53184943,100.91923398)(671.66184082,100.95924123)
\curveto(671.79184917,100.9992339)(671.92684903,101.04423386)(672.06684082,101.09424123)
\curveto(672.14684881,101.11423379)(672.23184873,101.12923377)(672.32184082,101.13924123)
\lineto(672.56184082,101.19924123)
\curveto(672.67184829,101.22923367)(672.78184818,101.24423366)(672.89184082,101.24424123)
\curveto(673.00184796,101.25423365)(673.11184785,101.26923363)(673.22184082,101.28924123)
\curveto(673.27184769,101.30923359)(673.31684764,101.31423359)(673.35684082,101.30424123)
\curveto(673.39684756,101.3042336)(673.43684752,101.30923359)(673.47684082,101.31924123)
\curveto(673.52684743,101.32923357)(673.58184738,101.32923357)(673.64184082,101.31924123)
\curveto(673.69184727,101.31923358)(673.74184722,101.32423358)(673.79184082,101.33424123)
\lineto(673.92684082,101.33424123)
\curveto(673.98684697,101.35423355)(674.0568469,101.35423355)(674.13684082,101.33424123)
\curveto(674.20684675,101.32423358)(674.27184669,101.32923357)(674.33184082,101.34924123)
\curveto(674.3618466,101.35923354)(674.40184656,101.36423354)(674.45184082,101.36424123)
\lineto(674.57184082,101.36424123)
\lineto(675.03684082,101.36424123)
\moveto(677.36184082,99.81924123)
\curveto(677.04184392,99.91923498)(676.67684428,99.97923492)(676.26684082,99.99924123)
\curveto(675.8568451,100.01923488)(675.44684551,100.02923487)(675.03684082,100.02924123)
\curveto(674.60684635,100.02923487)(674.18684677,100.01923488)(673.77684082,99.99924123)
\curveto(673.36684759,99.97923492)(672.98184798,99.93423497)(672.62184082,99.86424123)
\curveto(672.2618487,99.79423511)(671.94184902,99.68423522)(671.66184082,99.53424123)
\curveto(671.37184959,99.39423551)(671.13684982,99.1992357)(670.95684082,98.94924123)
\curveto(670.84685011,98.78923611)(670.76685019,98.60923629)(670.71684082,98.40924123)
\curveto(670.6568503,98.20923669)(670.62685033,97.96423694)(670.62684082,97.67424123)
\curveto(670.64685031,97.65423725)(670.6568503,97.61923728)(670.65684082,97.56924123)
\curveto(670.64685031,97.51923738)(670.64685031,97.47923742)(670.65684082,97.44924123)
\curveto(670.67685028,97.36923753)(670.69685026,97.29423761)(670.71684082,97.22424123)
\curveto(670.72685023,97.16423774)(670.74685021,97.0992378)(670.77684082,97.02924123)
\curveto(670.89685006,96.75923814)(671.06684989,96.53923836)(671.28684082,96.36924123)
\curveto(671.49684946,96.20923869)(671.74184922,96.07423883)(672.02184082,95.96424123)
\curveto(672.13184883,95.91423899)(672.25184871,95.87423903)(672.38184082,95.84424123)
\curveto(672.50184846,95.82423908)(672.62684833,95.7992391)(672.75684082,95.76924123)
\curveto(672.80684815,95.74923915)(672.8618481,95.73923916)(672.92184082,95.73924123)
\curveto(672.97184799,95.73923916)(673.02184794,95.73423917)(673.07184082,95.72424123)
\curveto(673.1618478,95.71423919)(673.2568477,95.7042392)(673.35684082,95.69424123)
\curveto(673.44684751,95.68423922)(673.54184742,95.67423923)(673.64184082,95.66424123)
\curveto(673.72184724,95.66423924)(673.80684715,95.65923924)(673.89684082,95.64924123)
\lineto(674.13684082,95.64924123)
\lineto(674.31684082,95.64924123)
\curveto(674.34684661,95.63923926)(674.38184658,95.63423927)(674.42184082,95.63424123)
\lineto(674.55684082,95.63424123)
\lineto(675.00684082,95.63424123)
\curveto(675.08684587,95.63423927)(675.17184579,95.62923927)(675.26184082,95.61924123)
\curveto(675.34184562,95.61923928)(675.41684554,95.62923927)(675.48684082,95.64924123)
\lineto(675.75684082,95.64924123)
\curveto(675.77684518,95.64923925)(675.80684515,95.64423926)(675.84684082,95.63424123)
\curveto(675.87684508,95.63423927)(675.90184506,95.63923926)(675.92184082,95.64924123)
\curveto(676.02184494,95.65923924)(676.12184484,95.66423924)(676.22184082,95.66424123)
\curveto(676.31184465,95.67423923)(676.41184455,95.68423922)(676.52184082,95.69424123)
\curveto(676.64184432,95.72423918)(676.76684419,95.73923916)(676.89684082,95.73924123)
\curveto(677.01684394,95.74923915)(677.13184383,95.77423913)(677.24184082,95.81424123)
\curveto(677.54184342,95.89423901)(677.80684315,95.97923892)(678.03684082,96.06924123)
\curveto(678.26684269,96.16923873)(678.48184248,96.31423859)(678.68184082,96.50424123)
\curveto(678.88184208,96.71423819)(679.03184193,96.97923792)(679.13184082,97.29924123)
\curveto(679.15184181,97.33923756)(679.1618418,97.37423753)(679.16184082,97.40424123)
\curveto(679.15184181,97.44423746)(679.1568418,97.48923741)(679.17684082,97.53924123)
\curveto(679.18684177,97.57923732)(679.19684176,97.64923725)(679.20684082,97.74924123)
\curveto(679.21684174,97.85923704)(679.21184175,97.94423696)(679.19184082,98.00424123)
\curveto(679.17184179,98.07423683)(679.1618418,98.14423676)(679.16184082,98.21424123)
\curveto(679.15184181,98.28423662)(679.13684182,98.34923655)(679.11684082,98.40924123)
\curveto(679.0568419,98.60923629)(678.97184199,98.78923611)(678.86184082,98.94924123)
\curveto(678.84184212,98.97923592)(678.82184214,99.0042359)(678.80184082,99.02424123)
\lineto(678.74184082,99.08424123)
\curveto(678.72184224,99.12423578)(678.68184228,99.17423573)(678.62184082,99.23424123)
\curveto(678.48184248,99.33423557)(678.35184261,99.41923548)(678.23184082,99.48924123)
\curveto(678.11184285,99.55923534)(677.96684299,99.62923527)(677.79684082,99.69924123)
\curveto(677.72684323,99.72923517)(677.6568433,99.74923515)(677.58684082,99.75924123)
\curveto(677.51684344,99.77923512)(677.44184352,99.7992351)(677.36184082,99.81924123)
}
}
{
\newrgbcolor{curcolor}{0 0 0}
\pscustom[linestyle=none,fillstyle=solid,fillcolor=curcolor]
{
\newpath
\moveto(669.51684082,106.77385061)
\curveto(669.51685144,106.87384575)(669.52685143,106.96884566)(669.54684082,107.05885061)
\curveto(669.5568514,107.14884548)(669.58685137,107.21384541)(669.63684082,107.25385061)
\curveto(669.71685124,107.31384531)(669.82185114,107.34384528)(669.95184082,107.34385061)
\lineto(670.34184082,107.34385061)
\lineto(671.84184082,107.34385061)
\lineto(678.23184082,107.34385061)
\lineto(679.40184082,107.34385061)
\lineto(679.71684082,107.34385061)
\curveto(679.81684114,107.35384527)(679.89684106,107.33884529)(679.95684082,107.29885061)
\curveto(680.03684092,107.24884538)(680.08684087,107.17384545)(680.10684082,107.07385061)
\curveto(680.11684084,106.98384564)(680.12184084,106.87384575)(680.12184082,106.74385061)
\lineto(680.12184082,106.51885061)
\curveto(680.10184086,106.43884619)(680.08684087,106.36884626)(680.07684082,106.30885061)
\curveto(680.0568409,106.24884638)(680.01684094,106.19884643)(679.95684082,106.15885061)
\curveto(679.89684106,106.11884651)(679.82184114,106.09884653)(679.73184082,106.09885061)
\lineto(679.43184082,106.09885061)
\lineto(678.33684082,106.09885061)
\lineto(672.99684082,106.09885061)
\curveto(672.90684805,106.07884655)(672.83184813,106.06384656)(672.77184082,106.05385061)
\curveto(672.70184826,106.05384657)(672.64184832,106.0238466)(672.59184082,105.96385061)
\curveto(672.54184842,105.89384673)(672.51684844,105.80384682)(672.51684082,105.69385061)
\curveto(672.50684845,105.59384703)(672.50184846,105.48384714)(672.50184082,105.36385061)
\lineto(672.50184082,104.22385061)
\lineto(672.50184082,103.72885061)
\curveto(672.49184847,103.56884906)(672.43184853,103.45884917)(672.32184082,103.39885061)
\curveto(672.29184867,103.37884925)(672.2618487,103.36884926)(672.23184082,103.36885061)
\curveto(672.19184877,103.36884926)(672.14684881,103.36384926)(672.09684082,103.35385061)
\curveto(671.97684898,103.33384929)(671.86684909,103.33884929)(671.76684082,103.36885061)
\curveto(671.66684929,103.40884922)(671.59684936,103.46384916)(671.55684082,103.53385061)
\curveto(671.50684945,103.61384901)(671.48184948,103.73384889)(671.48184082,103.89385061)
\curveto(671.48184948,104.05384857)(671.46684949,104.18884844)(671.43684082,104.29885061)
\curveto(671.42684953,104.34884828)(671.42184954,104.40384822)(671.42184082,104.46385061)
\curveto(671.41184955,104.5238481)(671.39684956,104.58384804)(671.37684082,104.64385061)
\curveto(671.32684963,104.79384783)(671.27684968,104.93884769)(671.22684082,105.07885061)
\curveto(671.16684979,105.21884741)(671.09684986,105.35384727)(671.01684082,105.48385061)
\curveto(670.92685003,105.623847)(670.82185014,105.74384688)(670.70184082,105.84385061)
\curveto(670.58185038,105.94384668)(670.45185051,106.03884659)(670.31184082,106.12885061)
\curveto(670.21185075,106.18884644)(670.10185086,106.23384639)(669.98184082,106.26385061)
\curveto(669.8618511,106.30384632)(669.7568512,106.35384627)(669.66684082,106.41385061)
\curveto(669.60685135,106.46384616)(669.56685139,106.53384609)(669.54684082,106.62385061)
\curveto(669.53685142,106.64384598)(669.53185143,106.66884596)(669.53184082,106.69885061)
\curveto(669.53185143,106.7288459)(669.52685143,106.75384587)(669.51684082,106.77385061)
}
}
{
\newrgbcolor{curcolor}{0 0 0}
\pscustom[linestyle=none,fillstyle=solid,fillcolor=curcolor]
{
\newpath
\moveto(669.51684082,115.12345998)
\curveto(669.51685144,115.22345513)(669.52685143,115.31845503)(669.54684082,115.40845998)
\curveto(669.5568514,115.49845485)(669.58685137,115.56345479)(669.63684082,115.60345998)
\curveto(669.71685124,115.66345469)(669.82185114,115.69345466)(669.95184082,115.69345998)
\lineto(670.34184082,115.69345998)
\lineto(671.84184082,115.69345998)
\lineto(678.23184082,115.69345998)
\lineto(679.40184082,115.69345998)
\lineto(679.71684082,115.69345998)
\curveto(679.81684114,115.70345465)(679.89684106,115.68845466)(679.95684082,115.64845998)
\curveto(680.03684092,115.59845475)(680.08684087,115.52345483)(680.10684082,115.42345998)
\curveto(680.11684084,115.33345502)(680.12184084,115.22345513)(680.12184082,115.09345998)
\lineto(680.12184082,114.86845998)
\curveto(680.10184086,114.78845556)(680.08684087,114.71845563)(680.07684082,114.65845998)
\curveto(680.0568409,114.59845575)(680.01684094,114.5484558)(679.95684082,114.50845998)
\curveto(679.89684106,114.46845588)(679.82184114,114.4484559)(679.73184082,114.44845998)
\lineto(679.43184082,114.44845998)
\lineto(678.33684082,114.44845998)
\lineto(672.99684082,114.44845998)
\curveto(672.90684805,114.42845592)(672.83184813,114.41345594)(672.77184082,114.40345998)
\curveto(672.70184826,114.40345595)(672.64184832,114.37345598)(672.59184082,114.31345998)
\curveto(672.54184842,114.24345611)(672.51684844,114.1534562)(672.51684082,114.04345998)
\curveto(672.50684845,113.94345641)(672.50184846,113.83345652)(672.50184082,113.71345998)
\lineto(672.50184082,112.57345998)
\lineto(672.50184082,112.07845998)
\curveto(672.49184847,111.91845843)(672.43184853,111.80845854)(672.32184082,111.74845998)
\curveto(672.29184867,111.72845862)(672.2618487,111.71845863)(672.23184082,111.71845998)
\curveto(672.19184877,111.71845863)(672.14684881,111.71345864)(672.09684082,111.70345998)
\curveto(671.97684898,111.68345867)(671.86684909,111.68845866)(671.76684082,111.71845998)
\curveto(671.66684929,111.75845859)(671.59684936,111.81345854)(671.55684082,111.88345998)
\curveto(671.50684945,111.96345839)(671.48184948,112.08345827)(671.48184082,112.24345998)
\curveto(671.48184948,112.40345795)(671.46684949,112.53845781)(671.43684082,112.64845998)
\curveto(671.42684953,112.69845765)(671.42184954,112.7534576)(671.42184082,112.81345998)
\curveto(671.41184955,112.87345748)(671.39684956,112.93345742)(671.37684082,112.99345998)
\curveto(671.32684963,113.14345721)(671.27684968,113.28845706)(671.22684082,113.42845998)
\curveto(671.16684979,113.56845678)(671.09684986,113.70345665)(671.01684082,113.83345998)
\curveto(670.92685003,113.97345638)(670.82185014,114.09345626)(670.70184082,114.19345998)
\curveto(670.58185038,114.29345606)(670.45185051,114.38845596)(670.31184082,114.47845998)
\curveto(670.21185075,114.53845581)(670.10185086,114.58345577)(669.98184082,114.61345998)
\curveto(669.8618511,114.6534557)(669.7568512,114.70345565)(669.66684082,114.76345998)
\curveto(669.60685135,114.81345554)(669.56685139,114.88345547)(669.54684082,114.97345998)
\curveto(669.53685142,114.99345536)(669.53185143,115.01845533)(669.53184082,115.04845998)
\curveto(669.53185143,115.07845527)(669.52685143,115.10345525)(669.51684082,115.12345998)
}
}
{
\newrgbcolor{curcolor}{0 0 0}
\pscustom[linestyle=none,fillstyle=solid,fillcolor=curcolor]
{
\newpath
\moveto(690.35315674,37.28705373)
\curveto(690.35316743,37.35704805)(690.35316743,37.43704797)(690.35315674,37.52705373)
\curveto(690.34316744,37.61704779)(690.34316744,37.70204771)(690.35315674,37.78205373)
\curveto(690.35316743,37.87204754)(690.36316742,37.95204746)(690.38315674,38.02205373)
\curveto(690.40316738,38.10204731)(690.43316735,38.15704725)(690.47315674,38.18705373)
\curveto(690.52316726,38.21704719)(690.59816719,38.23704717)(690.69815674,38.24705373)
\curveto(690.788167,38.26704714)(690.89316689,38.27704713)(691.01315674,38.27705373)
\curveto(691.12316666,38.28704712)(691.23816655,38.28704712)(691.35815674,38.27705373)
\lineto(691.65815674,38.27705373)
\lineto(694.67315674,38.27705373)
\lineto(697.56815674,38.27705373)
\curveto(697.89815989,38.27704713)(698.22315956,38.27204714)(698.54315674,38.26205373)
\curveto(698.85315893,38.26204715)(699.13315865,38.22204719)(699.38315674,38.14205373)
\curveto(699.73315805,38.02204739)(700.02815776,37.86704754)(700.26815674,37.67705373)
\curveto(700.49815729,37.48704792)(700.69815709,37.24704816)(700.86815674,36.95705373)
\curveto(700.91815687,36.89704851)(700.95315683,36.83204858)(700.97315674,36.76205373)
\curveto(700.99315679,36.70204871)(701.01815677,36.63204878)(701.04815674,36.55205373)
\curveto(701.09815669,36.43204898)(701.13315665,36.30204911)(701.15315674,36.16205373)
\curveto(701.1831566,36.03204938)(701.21315657,35.89704951)(701.24315674,35.75705373)
\curveto(701.26315652,35.7070497)(701.26815652,35.65704975)(701.25815674,35.60705373)
\curveto(701.24815654,35.55704985)(701.24815654,35.50204991)(701.25815674,35.44205373)
\curveto(701.26815652,35.42204999)(701.26815652,35.39705001)(701.25815674,35.36705373)
\curveto(701.25815653,35.33705007)(701.26315652,35.3120501)(701.27315674,35.29205373)
\curveto(701.2831565,35.25205016)(701.2881565,35.19705021)(701.28815674,35.12705373)
\curveto(701.2881565,35.05705035)(701.2831565,35.00205041)(701.27315674,34.96205373)
\curveto(701.26315652,34.9120505)(701.26315652,34.85705055)(701.27315674,34.79705373)
\curveto(701.2831565,34.73705067)(701.27815651,34.68205073)(701.25815674,34.63205373)
\curveto(701.22815656,34.50205091)(701.20815658,34.37705103)(701.19815674,34.25705373)
\curveto(701.1881566,34.13705127)(701.16315662,34.02205139)(701.12315674,33.91205373)
\curveto(701.00315678,33.54205187)(700.83315695,33.22205219)(700.61315674,32.95205373)
\curveto(700.39315739,32.68205273)(700.11315767,32.47205294)(699.77315674,32.32205373)
\curveto(699.65315813,32.27205314)(699.52815826,32.22705318)(699.39815674,32.18705373)
\curveto(699.26815852,32.15705325)(699.13315865,32.12205329)(698.99315674,32.08205373)
\curveto(698.94315884,32.07205334)(698.90315888,32.06705334)(698.87315674,32.06705373)
\curveto(698.83315895,32.06705334)(698.788159,32.06205335)(698.73815674,32.05205373)
\curveto(698.70815908,32.04205337)(698.67315911,32.03705337)(698.63315674,32.03705373)
\curveto(698.5831592,32.03705337)(698.54315924,32.03205338)(698.51315674,32.02205373)
\lineto(698.34815674,32.02205373)
\curveto(698.26815952,32.00205341)(698.16815962,31.99705341)(698.04815674,32.00705373)
\curveto(697.91815987,32.01705339)(697.82815996,32.03205338)(697.77815674,32.05205373)
\curveto(697.6881601,32.07205334)(697.62316016,32.12705328)(697.58315674,32.21705373)
\curveto(697.56316022,32.24705316)(697.55816023,32.27705313)(697.56815674,32.30705373)
\curveto(697.56816022,32.33705307)(697.56316022,32.37705303)(697.55315674,32.42705373)
\curveto(697.54316024,32.46705294)(697.53816025,32.5070529)(697.53815674,32.54705373)
\lineto(697.53815674,32.69705373)
\curveto(697.53816025,32.81705259)(697.54316024,32.93705247)(697.55315674,33.05705373)
\curveto(697.55316023,33.18705222)(697.5881602,33.27705213)(697.65815674,33.32705373)
\curveto(697.71816007,33.36705204)(697.77816001,33.38705202)(697.83815674,33.38705373)
\curveto(697.89815989,33.38705202)(697.96815982,33.39705201)(698.04815674,33.41705373)
\curveto(698.07815971,33.42705198)(698.11315967,33.42705198)(698.15315674,33.41705373)
\curveto(698.1831596,33.41705199)(698.20815958,33.42205199)(698.22815674,33.43205373)
\lineto(698.43815674,33.43205373)
\curveto(698.4881593,33.45205196)(698.53815925,33.45705195)(698.58815674,33.44705373)
\curveto(698.62815916,33.44705196)(698.67315911,33.45705195)(698.72315674,33.47705373)
\curveto(698.85315893,33.5070519)(698.97815881,33.53705187)(699.09815674,33.56705373)
\curveto(699.20815858,33.59705181)(699.31315847,33.64205177)(699.41315674,33.70205373)
\curveto(699.70315808,33.87205154)(699.90815788,34.14205127)(700.02815674,34.51205373)
\curveto(700.04815774,34.56205085)(700.06315772,34.6120508)(700.07315674,34.66205373)
\curveto(700.07315771,34.72205069)(700.0831577,34.77705063)(700.10315674,34.82705373)
\lineto(700.10315674,34.90205373)
\curveto(700.11315767,34.97205044)(700.12315766,35.06705034)(700.13315674,35.18705373)
\curveto(700.13315765,35.31705009)(700.12315766,35.41704999)(700.10315674,35.48705373)
\curveto(700.0831577,35.55704985)(700.06815772,35.62704978)(700.05815674,35.69705373)
\curveto(700.03815775,35.77704963)(700.01815777,35.84704956)(699.99815674,35.90705373)
\curveto(699.83815795,36.28704912)(699.56315822,36.56204885)(699.17315674,36.73205373)
\curveto(699.04315874,36.78204863)(698.8881589,36.81704859)(698.70815674,36.83705373)
\curveto(698.52815926,36.86704854)(698.34315944,36.88204853)(698.15315674,36.88205373)
\curveto(697.95315983,36.89204852)(697.75316003,36.89204852)(697.55315674,36.88205373)
\lineto(696.98315674,36.88205373)
\lineto(692.73815674,36.88205373)
\lineto(691.19315674,36.88205373)
\curveto(691.0831667,36.88204853)(690.96316682,36.87704853)(690.83315674,36.86705373)
\curveto(690.70316708,36.85704855)(690.59816719,36.87704853)(690.51815674,36.92705373)
\curveto(690.44816734,36.98704842)(690.39816739,37.06704834)(690.36815674,37.16705373)
\curveto(690.36816742,37.18704822)(690.36816742,37.2070482)(690.36815674,37.22705373)
\curveto(690.36816742,37.24704816)(690.36316742,37.26704814)(690.35315674,37.28705373)
}
}
{
\newrgbcolor{curcolor}{0 0 0}
\pscustom[linestyle=none,fillstyle=solid,fillcolor=curcolor]
{
\newpath
\moveto(693.30815674,40.82072561)
\lineto(693.30815674,41.25572561)
\curveto(693.30816448,41.40572364)(693.34816444,41.51072354)(693.42815674,41.57072561)
\curveto(693.50816428,41.62072343)(693.60816418,41.6457234)(693.72815674,41.64572561)
\curveto(693.84816394,41.65572339)(693.96816382,41.66072339)(694.08815674,41.66072561)
\lineto(695.51315674,41.66072561)
\lineto(697.77815674,41.66072561)
\lineto(698.46815674,41.66072561)
\curveto(698.69815909,41.66072339)(698.89815889,41.68572336)(699.06815674,41.73572561)
\curveto(699.51815827,41.89572315)(699.83315795,42.19572285)(700.01315674,42.63572561)
\curveto(700.10315768,42.85572219)(700.13815765,43.12072193)(700.11815674,43.43072561)
\curveto(700.0881577,43.74072131)(700.03315775,43.99072106)(699.95315674,44.18072561)
\curveto(699.81315797,44.51072054)(699.63815815,44.77072028)(699.42815674,44.96072561)
\curveto(699.20815858,45.16071989)(698.92315886,45.31571973)(698.57315674,45.42572561)
\curveto(698.49315929,45.45571959)(698.41315937,45.47571957)(698.33315674,45.48572561)
\curveto(698.25315953,45.49571955)(698.16815962,45.51071954)(698.07815674,45.53072561)
\curveto(698.02815976,45.54071951)(697.9831598,45.54071951)(697.94315674,45.53072561)
\curveto(697.90315988,45.53071952)(697.85815993,45.54071951)(697.80815674,45.56072561)
\lineto(697.49315674,45.56072561)
\curveto(697.41316037,45.58071947)(697.32316046,45.58571946)(697.22315674,45.57572561)
\curveto(697.11316067,45.56571948)(697.01316077,45.56071949)(696.92315674,45.56072561)
\lineto(695.75315674,45.56072561)
\lineto(694.16315674,45.56072561)
\curveto(694.04316374,45.56071949)(693.91816387,45.55571949)(693.78815674,45.54572561)
\curveto(693.64816414,45.5457195)(693.53816425,45.57071948)(693.45815674,45.62072561)
\curveto(693.40816438,45.66071939)(693.37816441,45.70571934)(693.36815674,45.75572561)
\curveto(693.34816444,45.81571923)(693.32816446,45.88571916)(693.30815674,45.96572561)
\lineto(693.30815674,46.19072561)
\curveto(693.30816448,46.31071874)(693.31316447,46.41571863)(693.32315674,46.50572561)
\curveto(693.33316445,46.60571844)(693.37816441,46.68071837)(693.45815674,46.73072561)
\curveto(693.50816428,46.78071827)(693.5831642,46.80571824)(693.68315674,46.80572561)
\lineto(693.96815674,46.80572561)
\lineto(694.98815674,46.80572561)
\lineto(699.02315674,46.80572561)
\lineto(700.37315674,46.80572561)
\curveto(700.49315729,46.80571824)(700.60815718,46.80071825)(700.71815674,46.79072561)
\curveto(700.81815697,46.79071826)(700.89315689,46.75571829)(700.94315674,46.68572561)
\curveto(700.97315681,46.6457184)(700.99815679,46.58571846)(701.01815674,46.50572561)
\curveto(701.02815676,46.42571862)(701.03815675,46.33571871)(701.04815674,46.23572561)
\curveto(701.04815674,46.1457189)(701.04315674,46.05571899)(701.03315674,45.96572561)
\curveto(701.02315676,45.88571916)(701.00315678,45.82571922)(700.97315674,45.78572561)
\curveto(700.93315685,45.73571931)(700.86815692,45.69071936)(700.77815674,45.65072561)
\curveto(700.73815705,45.64071941)(700.6831571,45.63071942)(700.61315674,45.62072561)
\curveto(700.54315724,45.62071943)(700.47815731,45.61571943)(700.41815674,45.60572561)
\curveto(700.34815744,45.59571945)(700.29315749,45.57571947)(700.25315674,45.54572561)
\curveto(700.21315757,45.51571953)(700.19815759,45.47071958)(700.20815674,45.41072561)
\curveto(700.22815756,45.33071972)(700.2881575,45.2507198)(700.38815674,45.17072561)
\curveto(700.47815731,45.09071996)(700.54815724,45.01572003)(700.59815674,44.94572561)
\curveto(700.75815703,44.72572032)(700.89815689,44.47572057)(701.01815674,44.19572561)
\curveto(701.06815672,44.08572096)(701.09815669,43.97072108)(701.10815674,43.85072561)
\curveto(701.12815666,43.74072131)(701.15315663,43.62572142)(701.18315674,43.50572561)
\curveto(701.19315659,43.45572159)(701.19315659,43.40072165)(701.18315674,43.34072561)
\curveto(701.17315661,43.29072176)(701.17815661,43.24072181)(701.19815674,43.19072561)
\curveto(701.21815657,43.09072196)(701.21815657,43.00072205)(701.19815674,42.92072561)
\lineto(701.19815674,42.77072561)
\curveto(701.17815661,42.72072233)(701.16815662,42.66072239)(701.16815674,42.59072561)
\curveto(701.16815662,42.53072252)(701.16315662,42.47572257)(701.15315674,42.42572561)
\curveto(701.13315665,42.38572266)(701.12315666,42.3457227)(701.12315674,42.30572561)
\curveto(701.13315665,42.27572277)(701.12815666,42.23572281)(701.10815674,42.18572561)
\lineto(701.04815674,41.94572561)
\curveto(701.02815676,41.87572317)(700.99815679,41.80072325)(700.95815674,41.72072561)
\curveto(700.84815694,41.46072359)(700.70315708,41.24072381)(700.52315674,41.06072561)
\curveto(700.33315745,40.89072416)(700.10815768,40.7507243)(699.84815674,40.64072561)
\curveto(699.75815803,40.60072445)(699.66815812,40.57072448)(699.57815674,40.55072561)
\lineto(699.27815674,40.49072561)
\curveto(699.21815857,40.47072458)(699.16315862,40.46072459)(699.11315674,40.46072561)
\curveto(699.05315873,40.47072458)(698.9881588,40.46572458)(698.91815674,40.44572561)
\curveto(698.89815889,40.43572461)(698.87315891,40.43072462)(698.84315674,40.43072561)
\curveto(698.80315898,40.43072462)(698.76815902,40.42572462)(698.73815674,40.41572561)
\lineto(698.58815674,40.41572561)
\curveto(698.54815924,40.40572464)(698.50315928,40.40072465)(698.45315674,40.40072561)
\curveto(698.39315939,40.41072464)(698.33815945,40.41572463)(698.28815674,40.41572561)
\lineto(697.68815674,40.41572561)
\lineto(694.92815674,40.41572561)
\lineto(693.96815674,40.41572561)
\lineto(693.69815674,40.41572561)
\curveto(693.60816418,40.41572463)(693.53316425,40.43572461)(693.47315674,40.47572561)
\curveto(693.40316438,40.51572453)(693.35316443,40.59072446)(693.32315674,40.70072561)
\curveto(693.31316447,40.72072433)(693.31316447,40.74072431)(693.32315674,40.76072561)
\curveto(693.32316446,40.78072427)(693.31816447,40.80072425)(693.30815674,40.82072561)
}
}
{
\newrgbcolor{curcolor}{0 0 0}
\pscustom[linestyle=none,fillstyle=solid,fillcolor=curcolor]
{
\newpath
\moveto(693.15815674,52.39533498)
\curveto(693.13816465,53.02532975)(693.22316456,53.53032924)(693.41315674,53.91033498)
\curveto(693.60316418,54.29032848)(693.8881639,54.59532818)(694.26815674,54.82533498)
\curveto(694.36816342,54.88532789)(694.47816331,54.93032784)(694.59815674,54.96033498)
\curveto(694.70816308,55.00032777)(694.82316296,55.03532774)(694.94315674,55.06533498)
\curveto(695.13316265,55.11532766)(695.33816245,55.14532763)(695.55815674,55.15533498)
\curveto(695.77816201,55.16532761)(696.00316178,55.1703276)(696.23315674,55.17033498)
\lineto(697.83815674,55.17033498)
\lineto(700.17815674,55.17033498)
\curveto(700.34815744,55.1703276)(700.51815727,55.16532761)(700.68815674,55.15533498)
\curveto(700.85815693,55.15532762)(700.96815682,55.09032768)(701.01815674,54.96033498)
\curveto(701.03815675,54.91032786)(701.04815674,54.85532792)(701.04815674,54.79533498)
\curveto(701.05815673,54.74532803)(701.06315672,54.69032808)(701.06315674,54.63033498)
\curveto(701.06315672,54.50032827)(701.05815673,54.3753284)(701.04815674,54.25533498)
\curveto(701.04815674,54.13532864)(701.00815678,54.05032872)(700.92815674,54.00033498)
\curveto(700.85815693,53.95032882)(700.76815702,53.92532885)(700.65815674,53.92533498)
\lineto(700.32815674,53.92533498)
\lineto(699.03815674,53.92533498)
\lineto(696.59315674,53.92533498)
\curveto(696.32316146,53.92532885)(696.05816173,53.92032885)(695.79815674,53.91033498)
\curveto(695.52816226,53.90032887)(695.29816249,53.85532892)(695.10815674,53.77533498)
\curveto(694.90816288,53.69532908)(694.74816304,53.5753292)(694.62815674,53.41533498)
\curveto(694.49816329,53.25532952)(694.39816339,53.0703297)(694.32815674,52.86033498)
\curveto(694.30816348,52.80032997)(694.29816349,52.73533004)(694.29815674,52.66533498)
\curveto(694.2881635,52.60533017)(694.27316351,52.54533023)(694.25315674,52.48533498)
\curveto(694.24316354,52.43533034)(694.24316354,52.35533042)(694.25315674,52.24533498)
\curveto(694.25316353,52.14533063)(694.25816353,52.0753307)(694.26815674,52.03533498)
\curveto(694.2881635,51.99533078)(694.29816349,51.96033081)(694.29815674,51.93033498)
\curveto(694.2881635,51.90033087)(694.2881635,51.86533091)(694.29815674,51.82533498)
\curveto(694.32816346,51.69533108)(694.36316342,51.5703312)(694.40315674,51.45033498)
\curveto(694.43316335,51.34033143)(694.47816331,51.23533154)(694.53815674,51.13533498)
\curveto(694.55816323,51.09533168)(694.57816321,51.06033171)(694.59815674,51.03033498)
\curveto(694.61816317,51.00033177)(694.63816315,50.96533181)(694.65815674,50.92533498)
\curveto(694.90816288,50.5753322)(695.2831625,50.32033245)(695.78315674,50.16033498)
\curveto(695.86316192,50.13033264)(695.94816184,50.11033266)(696.03815674,50.10033498)
\curveto(696.11816167,50.09033268)(696.19816159,50.0753327)(696.27815674,50.05533498)
\curveto(696.32816146,50.03533274)(696.37816141,50.03033274)(696.42815674,50.04033498)
\curveto(696.46816132,50.05033272)(696.50816128,50.04533273)(696.54815674,50.02533498)
\lineto(696.86315674,50.02533498)
\curveto(696.89316089,50.01533276)(696.92816086,50.01033276)(696.96815674,50.01033498)
\curveto(697.00816078,50.02033275)(697.05316073,50.02533275)(697.10315674,50.02533498)
\lineto(697.55315674,50.02533498)
\lineto(698.99315674,50.02533498)
\lineto(700.31315674,50.02533498)
\lineto(700.65815674,50.02533498)
\curveto(700.76815702,50.02533275)(700.85815693,50.00033277)(700.92815674,49.95033498)
\curveto(701.00815678,49.90033287)(701.04815674,49.81033296)(701.04815674,49.68033498)
\curveto(701.05815673,49.56033321)(701.06315672,49.43533334)(701.06315674,49.30533498)
\curveto(701.06315672,49.22533355)(701.05815673,49.15033362)(701.04815674,49.08033498)
\curveto(701.03815675,49.01033376)(701.01315677,48.95033382)(700.97315674,48.90033498)
\curveto(700.92315686,48.82033395)(700.82815696,48.78033399)(700.68815674,48.78033498)
\lineto(700.28315674,48.78033498)
\lineto(698.51315674,48.78033498)
\lineto(694.88315674,48.78033498)
\lineto(693.96815674,48.78033498)
\lineto(693.69815674,48.78033498)
\curveto(693.60816418,48.78033399)(693.53816425,48.80033397)(693.48815674,48.84033498)
\curveto(693.42816436,48.8703339)(693.3881644,48.92033385)(693.36815674,48.99033498)
\curveto(693.35816443,49.03033374)(693.34816444,49.08533369)(693.33815674,49.15533498)
\curveto(693.32816446,49.23533354)(693.32316446,49.31533346)(693.32315674,49.39533498)
\curveto(693.32316446,49.4753333)(693.32816446,49.55033322)(693.33815674,49.62033498)
\curveto(693.34816444,49.70033307)(693.36316442,49.75533302)(693.38315674,49.78533498)
\curveto(693.45316433,49.89533288)(693.54316424,49.94533283)(693.65315674,49.93533498)
\curveto(693.75316403,49.92533285)(693.86816392,49.94033283)(693.99815674,49.98033498)
\curveto(694.05816373,50.00033277)(694.10816368,50.04033273)(694.14815674,50.10033498)
\curveto(694.15816363,50.22033255)(694.11316367,50.31533246)(694.01315674,50.38533498)
\curveto(693.91316387,50.46533231)(693.83316395,50.54533223)(693.77315674,50.62533498)
\curveto(693.67316411,50.76533201)(693.5831642,50.90533187)(693.50315674,51.04533498)
\curveto(693.41316437,51.19533158)(693.33816445,51.36533141)(693.27815674,51.55533498)
\curveto(693.24816454,51.63533114)(693.22816456,51.72033105)(693.21815674,51.81033498)
\curveto(693.20816458,51.91033086)(693.19316459,52.00533077)(693.17315674,52.09533498)
\curveto(693.16316462,52.14533063)(693.15816463,52.19533058)(693.15815674,52.24533498)
\lineto(693.15815674,52.39533498)
}
}
{
\newrgbcolor{curcolor}{0 0 0}
\pscustom[linestyle=none,fillstyle=solid,fillcolor=curcolor]
{
}
}
{
\newrgbcolor{curcolor}{0 0 0}
\pscustom[linestyle=none,fillstyle=solid,fillcolor=curcolor]
{
\newpath
\moveto(690.42815674,65.05510061)
\curveto(690.42816736,65.15509575)(690.43816735,65.25009566)(690.45815674,65.34010061)
\curveto(690.46816732,65.43009548)(690.49816729,65.49509541)(690.54815674,65.53510061)
\curveto(690.62816716,65.59509531)(690.73316705,65.62509528)(690.86315674,65.62510061)
\lineto(691.25315674,65.62510061)
\lineto(692.75315674,65.62510061)
\lineto(699.14315674,65.62510061)
\lineto(700.31315674,65.62510061)
\lineto(700.62815674,65.62510061)
\curveto(700.72815706,65.63509527)(700.80815698,65.62009529)(700.86815674,65.58010061)
\curveto(700.94815684,65.53009538)(700.99815679,65.45509545)(701.01815674,65.35510061)
\curveto(701.02815676,65.26509564)(701.03315675,65.15509575)(701.03315674,65.02510061)
\lineto(701.03315674,64.80010061)
\curveto(701.01315677,64.72009619)(700.99815679,64.65009626)(700.98815674,64.59010061)
\curveto(700.96815682,64.53009638)(700.92815686,64.48009643)(700.86815674,64.44010061)
\curveto(700.80815698,64.40009651)(700.73315705,64.38009653)(700.64315674,64.38010061)
\lineto(700.34315674,64.38010061)
\lineto(699.24815674,64.38010061)
\lineto(693.90815674,64.38010061)
\curveto(693.81816397,64.36009655)(693.74316404,64.34509656)(693.68315674,64.33510061)
\curveto(693.61316417,64.33509657)(693.55316423,64.3050966)(693.50315674,64.24510061)
\curveto(693.45316433,64.17509673)(693.42816436,64.08509682)(693.42815674,63.97510061)
\curveto(693.41816437,63.87509703)(693.41316437,63.76509714)(693.41315674,63.64510061)
\lineto(693.41315674,62.50510061)
\lineto(693.41315674,62.01010061)
\curveto(693.40316438,61.85009906)(693.34316444,61.74009917)(693.23315674,61.68010061)
\curveto(693.20316458,61.66009925)(693.17316461,61.65009926)(693.14315674,61.65010061)
\curveto(693.10316468,61.65009926)(693.05816473,61.64509926)(693.00815674,61.63510061)
\curveto(692.8881649,61.61509929)(692.77816501,61.62009929)(692.67815674,61.65010061)
\curveto(692.57816521,61.69009922)(692.50816528,61.74509916)(692.46815674,61.81510061)
\curveto(692.41816537,61.89509901)(692.39316539,62.01509889)(692.39315674,62.17510061)
\curveto(692.39316539,62.33509857)(692.37816541,62.47009844)(692.34815674,62.58010061)
\curveto(692.33816545,62.63009828)(692.33316545,62.68509822)(692.33315674,62.74510061)
\curveto(692.32316546,62.8050981)(692.30816548,62.86509804)(692.28815674,62.92510061)
\curveto(692.23816555,63.07509783)(692.1881656,63.22009769)(692.13815674,63.36010061)
\curveto(692.07816571,63.50009741)(692.00816578,63.63509727)(691.92815674,63.76510061)
\curveto(691.83816595,63.905097)(691.73316605,64.02509688)(691.61315674,64.12510061)
\curveto(691.49316629,64.22509668)(691.36316642,64.32009659)(691.22315674,64.41010061)
\curveto(691.12316666,64.47009644)(691.01316677,64.51509639)(690.89315674,64.54510061)
\curveto(690.77316701,64.58509632)(690.66816712,64.63509627)(690.57815674,64.69510061)
\curveto(690.51816727,64.74509616)(690.47816731,64.81509609)(690.45815674,64.90510061)
\curveto(690.44816734,64.92509598)(690.44316734,64.95009596)(690.44315674,64.98010061)
\curveto(690.44316734,65.0100959)(690.43816735,65.03509587)(690.42815674,65.05510061)
}
}
{
\newrgbcolor{curcolor}{0 0 0}
\pscustom[linestyle=none,fillstyle=solid,fillcolor=curcolor]
{
\newpath
\moveto(695.94815674,76.34470998)
\lineto(696.20315674,76.34470998)
\curveto(696.2831615,76.35470228)(696.35816143,76.34970228)(696.42815674,76.32970998)
\lineto(696.66815674,76.32970998)
\lineto(696.83315674,76.32970998)
\curveto(696.93316085,76.30970232)(697.03816075,76.29970233)(697.14815674,76.29970998)
\curveto(697.24816054,76.29970233)(697.34816044,76.28970234)(697.44815674,76.26970998)
\lineto(697.59815674,76.26970998)
\curveto(697.73816005,76.23970239)(697.87815991,76.21970241)(698.01815674,76.20970998)
\curveto(698.14815964,76.19970243)(698.27815951,76.17470246)(698.40815674,76.13470998)
\curveto(698.4881593,76.11470252)(698.57315921,76.09470254)(698.66315674,76.07470998)
\lineto(698.90315674,76.01470998)
\lineto(699.20315674,75.89470998)
\curveto(699.29315849,75.86470277)(699.3831584,75.8297028)(699.47315674,75.78970998)
\curveto(699.69315809,75.68970294)(699.90815788,75.55470308)(700.11815674,75.38470998)
\curveto(700.32815746,75.22470341)(700.49815729,75.04970358)(700.62815674,74.85970998)
\curveto(700.66815712,74.80970382)(700.70815708,74.74970388)(700.74815674,74.67970998)
\curveto(700.77815701,74.61970401)(700.81315697,74.55970407)(700.85315674,74.49970998)
\curveto(700.90315688,74.41970421)(700.94315684,74.32470431)(700.97315674,74.21470998)
\curveto(701.00315678,74.10470453)(701.03315675,73.99970463)(701.06315674,73.89970998)
\curveto(701.10315668,73.78970484)(701.12815666,73.67970495)(701.13815674,73.56970998)
\curveto(701.14815664,73.45970517)(701.16315662,73.34470529)(701.18315674,73.22470998)
\curveto(701.19315659,73.18470545)(701.19315659,73.13970549)(701.18315674,73.08970998)
\curveto(701.1831566,73.04970558)(701.1881566,73.00970562)(701.19815674,72.96970998)
\curveto(701.20815658,72.9297057)(701.21315657,72.87470576)(701.21315674,72.80470998)
\curveto(701.21315657,72.7347059)(701.20815658,72.68470595)(701.19815674,72.65470998)
\curveto(701.17815661,72.60470603)(701.17315661,72.55970607)(701.18315674,72.51970998)
\curveto(701.19315659,72.47970615)(701.19315659,72.44470619)(701.18315674,72.41470998)
\lineto(701.18315674,72.32470998)
\curveto(701.16315662,72.26470637)(701.14815664,72.19970643)(701.13815674,72.12970998)
\curveto(701.13815665,72.06970656)(701.13315665,72.00470663)(701.12315674,71.93470998)
\curveto(701.07315671,71.76470687)(701.02315676,71.60470703)(700.97315674,71.45470998)
\curveto(700.92315686,71.30470733)(700.85815693,71.15970747)(700.77815674,71.01970998)
\curveto(700.73815705,70.96970766)(700.70815708,70.91470772)(700.68815674,70.85470998)
\curveto(700.65815713,70.80470783)(700.62315716,70.75470788)(700.58315674,70.70470998)
\curveto(700.40315738,70.46470817)(700.1831576,70.26470837)(699.92315674,70.10470998)
\curveto(699.66315812,69.94470869)(699.37815841,69.80470883)(699.06815674,69.68470998)
\curveto(698.92815886,69.62470901)(698.788159,69.57970905)(698.64815674,69.54970998)
\curveto(698.49815929,69.51970911)(698.34315944,69.48470915)(698.18315674,69.44470998)
\curveto(698.07315971,69.42470921)(697.96315982,69.40970922)(697.85315674,69.39970998)
\curveto(697.74316004,69.38970924)(697.63316015,69.37470926)(697.52315674,69.35470998)
\curveto(697.4831603,69.34470929)(697.44316034,69.33970929)(697.40315674,69.33970998)
\curveto(697.36316042,69.34970928)(697.32316046,69.34970928)(697.28315674,69.33970998)
\curveto(697.23316055,69.3297093)(697.1831606,69.32470931)(697.13315674,69.32470998)
\lineto(696.96815674,69.32470998)
\curveto(696.91816087,69.30470933)(696.86816092,69.29970933)(696.81815674,69.30970998)
\curveto(696.75816103,69.31970931)(696.70316108,69.31970931)(696.65315674,69.30970998)
\curveto(696.61316117,69.29970933)(696.56816122,69.29970933)(696.51815674,69.30970998)
\curveto(696.46816132,69.31970931)(696.41816137,69.31470932)(696.36815674,69.29470998)
\curveto(696.29816149,69.27470936)(696.22316156,69.26970936)(696.14315674,69.27970998)
\curveto(696.05316173,69.28970934)(695.96816182,69.29470934)(695.88815674,69.29470998)
\curveto(695.79816199,69.29470934)(695.69816209,69.28970934)(695.58815674,69.27970998)
\curveto(695.46816232,69.26970936)(695.36816242,69.27470936)(695.28815674,69.29470998)
\lineto(695.00315674,69.29470998)
\lineto(694.37315674,69.33970998)
\curveto(694.27316351,69.34970928)(694.17816361,69.35970927)(694.08815674,69.36970998)
\lineto(693.78815674,69.39970998)
\curveto(693.73816405,69.41970921)(693.6881641,69.42470921)(693.63815674,69.41470998)
\curveto(693.57816421,69.41470922)(693.52316426,69.42470921)(693.47315674,69.44470998)
\curveto(693.30316448,69.49470914)(693.13816465,69.5347091)(692.97815674,69.56470998)
\curveto(692.80816498,69.59470904)(692.64816514,69.64470899)(692.49815674,69.71470998)
\curveto(692.03816575,69.90470873)(691.66316612,70.12470851)(691.37315674,70.37470998)
\curveto(691.0831667,70.634708)(690.83816695,70.99470764)(690.63815674,71.45470998)
\curveto(690.5881672,71.58470705)(690.55316723,71.71470692)(690.53315674,71.84470998)
\curveto(690.51316727,71.98470665)(690.4881673,72.12470651)(690.45815674,72.26470998)
\curveto(690.44816734,72.3347063)(690.44316734,72.39970623)(690.44315674,72.45970998)
\curveto(690.44316734,72.51970611)(690.43816735,72.58470605)(690.42815674,72.65470998)
\curveto(690.40816738,73.48470515)(690.55816723,74.15470448)(690.87815674,74.66470998)
\curveto(691.1881666,75.17470346)(691.62816616,75.55470308)(692.19815674,75.80470998)
\curveto(692.31816547,75.85470278)(692.44316534,75.89970273)(692.57315674,75.93970998)
\curveto(692.70316508,75.97970265)(692.83816495,76.02470261)(692.97815674,76.07470998)
\curveto(693.05816473,76.09470254)(693.14316464,76.10970252)(693.23315674,76.11970998)
\lineto(693.47315674,76.17970998)
\curveto(693.5831642,76.20970242)(693.69316409,76.22470241)(693.80315674,76.22470998)
\curveto(693.91316387,76.2347024)(694.02316376,76.24970238)(694.13315674,76.26970998)
\curveto(694.1831636,76.28970234)(694.22816356,76.29470234)(694.26815674,76.28470998)
\curveto(694.30816348,76.28470235)(694.34816344,76.28970234)(694.38815674,76.29970998)
\curveto(694.43816335,76.30970232)(694.49316329,76.30970232)(694.55315674,76.29970998)
\curveto(694.60316318,76.29970233)(694.65316313,76.30470233)(694.70315674,76.31470998)
\lineto(694.83815674,76.31470998)
\curveto(694.89816289,76.3347023)(694.96816282,76.3347023)(695.04815674,76.31470998)
\curveto(695.11816267,76.30470233)(695.1831626,76.30970232)(695.24315674,76.32970998)
\curveto(695.27316251,76.33970229)(695.31316247,76.34470229)(695.36315674,76.34470998)
\lineto(695.48315674,76.34470998)
\lineto(695.94815674,76.34470998)
\moveto(698.27315674,74.79970998)
\curveto(697.95315983,74.89970373)(697.5881602,74.95970367)(697.17815674,74.97970998)
\curveto(696.76816102,74.99970363)(696.35816143,75.00970362)(695.94815674,75.00970998)
\curveto(695.51816227,75.00970362)(695.09816269,74.99970363)(694.68815674,74.97970998)
\curveto(694.27816351,74.95970367)(693.89316389,74.91470372)(693.53315674,74.84470998)
\curveto(693.17316461,74.77470386)(692.85316493,74.66470397)(692.57315674,74.51470998)
\curveto(692.2831655,74.37470426)(692.04816574,74.17970445)(691.86815674,73.92970998)
\curveto(691.75816603,73.76970486)(691.67816611,73.58970504)(691.62815674,73.38970998)
\curveto(691.56816622,73.18970544)(691.53816625,72.94470569)(691.53815674,72.65470998)
\curveto(691.55816623,72.634706)(691.56816622,72.59970603)(691.56815674,72.54970998)
\curveto(691.55816623,72.49970613)(691.55816623,72.45970617)(691.56815674,72.42970998)
\curveto(691.5881662,72.34970628)(691.60816618,72.27470636)(691.62815674,72.20470998)
\curveto(691.63816615,72.14470649)(691.65816613,72.07970655)(691.68815674,72.00970998)
\curveto(691.80816598,71.73970689)(691.97816581,71.51970711)(692.19815674,71.34970998)
\curveto(692.40816538,71.18970744)(692.65316513,71.05470758)(692.93315674,70.94470998)
\curveto(693.04316474,70.89470774)(693.16316462,70.85470778)(693.29315674,70.82470998)
\curveto(693.41316437,70.80470783)(693.53816425,70.77970785)(693.66815674,70.74970998)
\curveto(693.71816407,70.7297079)(693.77316401,70.71970791)(693.83315674,70.71970998)
\curveto(693.8831639,70.71970791)(693.93316385,70.71470792)(693.98315674,70.70470998)
\curveto(694.07316371,70.69470794)(694.16816362,70.68470795)(694.26815674,70.67470998)
\curveto(694.35816343,70.66470797)(694.45316333,70.65470798)(694.55315674,70.64470998)
\curveto(694.63316315,70.64470799)(694.71816307,70.63970799)(694.80815674,70.62970998)
\lineto(695.04815674,70.62970998)
\lineto(695.22815674,70.62970998)
\curveto(695.25816253,70.61970801)(695.29316249,70.61470802)(695.33315674,70.61470998)
\lineto(695.46815674,70.61470998)
\lineto(695.91815674,70.61470998)
\curveto(695.99816179,70.61470802)(696.0831617,70.60970802)(696.17315674,70.59970998)
\curveto(696.25316153,70.59970803)(696.32816146,70.60970802)(696.39815674,70.62970998)
\lineto(696.66815674,70.62970998)
\curveto(696.6881611,70.629708)(696.71816107,70.62470801)(696.75815674,70.61470998)
\curveto(696.788161,70.61470802)(696.81316097,70.61970801)(696.83315674,70.62970998)
\curveto(696.93316085,70.63970799)(697.03316075,70.64470799)(697.13315674,70.64470998)
\curveto(697.22316056,70.65470798)(697.32316046,70.66470797)(697.43315674,70.67470998)
\curveto(697.55316023,70.70470793)(697.67816011,70.71970791)(697.80815674,70.71970998)
\curveto(697.92815986,70.7297079)(698.04315974,70.75470788)(698.15315674,70.79470998)
\curveto(698.45315933,70.87470776)(698.71815907,70.95970767)(698.94815674,71.04970998)
\curveto(699.17815861,71.14970748)(699.39315839,71.29470734)(699.59315674,71.48470998)
\curveto(699.79315799,71.69470694)(699.94315784,71.95970667)(700.04315674,72.27970998)
\curveto(700.06315772,72.31970631)(700.07315771,72.35470628)(700.07315674,72.38470998)
\curveto(700.06315772,72.42470621)(700.06815772,72.46970616)(700.08815674,72.51970998)
\curveto(700.09815769,72.55970607)(700.10815768,72.629706)(700.11815674,72.72970998)
\curveto(700.12815766,72.83970579)(700.12315766,72.92470571)(700.10315674,72.98470998)
\curveto(700.0831577,73.05470558)(700.07315771,73.12470551)(700.07315674,73.19470998)
\curveto(700.06315772,73.26470537)(700.04815774,73.3297053)(700.02815674,73.38970998)
\curveto(699.96815782,73.58970504)(699.8831579,73.76970486)(699.77315674,73.92970998)
\curveto(699.75315803,73.95970467)(699.73315805,73.98470465)(699.71315674,74.00470998)
\lineto(699.65315674,74.06470998)
\curveto(699.63315815,74.10470453)(699.59315819,74.15470448)(699.53315674,74.21470998)
\curveto(699.39315839,74.31470432)(699.26315852,74.39970423)(699.14315674,74.46970998)
\curveto(699.02315876,74.53970409)(698.87815891,74.60970402)(698.70815674,74.67970998)
\curveto(698.63815915,74.70970392)(698.56815922,74.7297039)(698.49815674,74.73970998)
\curveto(698.42815936,74.75970387)(698.35315943,74.77970385)(698.27315674,74.79970998)
}
}
{
\newrgbcolor{curcolor}{0 0 0}
\pscustom[linestyle=none,fillstyle=solid,fillcolor=curcolor]
{
\newpath
\moveto(699.39815674,78.63431936)
\lineto(699.39815674,79.26431936)
\lineto(699.39815674,79.45931936)
\curveto(699.39815839,79.52931683)(699.40815838,79.58931677)(699.42815674,79.63931936)
\curveto(699.46815832,79.70931665)(699.50815828,79.7593166)(699.54815674,79.78931936)
\curveto(699.59815819,79.82931653)(699.66315812,79.84931651)(699.74315674,79.84931936)
\curveto(699.82315796,79.8593165)(699.90815788,79.86431649)(699.99815674,79.86431936)
\lineto(700.71815674,79.86431936)
\curveto(701.19815659,79.86431649)(701.60815618,79.80431655)(701.94815674,79.68431936)
\curveto(702.2881555,79.56431679)(702.56315522,79.36931699)(702.77315674,79.09931936)
\curveto(702.82315496,79.02931733)(702.86815492,78.9593174)(702.90815674,78.88931936)
\curveto(702.95815483,78.82931753)(703.00315478,78.7543176)(703.04315674,78.66431936)
\curveto(703.05315473,78.64431771)(703.06315472,78.61431774)(703.07315674,78.57431936)
\curveto(703.09315469,78.53431782)(703.09815469,78.48931787)(703.08815674,78.43931936)
\curveto(703.05815473,78.34931801)(702.9831548,78.29431806)(702.86315674,78.27431936)
\curveto(702.75315503,78.2543181)(702.65815513,78.26931809)(702.57815674,78.31931936)
\curveto(702.50815528,78.34931801)(702.44315534,78.39431796)(702.38315674,78.45431936)
\curveto(702.33315545,78.52431783)(702.2831555,78.58931777)(702.23315674,78.64931936)
\curveto(702.1831556,78.71931764)(702.10815568,78.77931758)(702.00815674,78.82931936)
\curveto(701.91815587,78.88931747)(701.82815596,78.93931742)(701.73815674,78.97931936)
\curveto(701.70815608,78.99931736)(701.64815614,79.02431733)(701.55815674,79.05431936)
\curveto(701.47815631,79.08431727)(701.40815638,79.08931727)(701.34815674,79.06931936)
\curveto(701.20815658,79.03931732)(701.11815667,78.97931738)(701.07815674,78.88931936)
\curveto(701.04815674,78.80931755)(701.03315675,78.71931764)(701.03315674,78.61931936)
\curveto(701.03315675,78.51931784)(701.00815678,78.43431792)(700.95815674,78.36431936)
\curveto(700.8881569,78.27431808)(700.74815704,78.22931813)(700.53815674,78.22931936)
\lineto(699.98315674,78.22931936)
\lineto(699.75815674,78.22931936)
\curveto(699.67815811,78.23931812)(699.61315817,78.2593181)(699.56315674,78.28931936)
\curveto(699.4831583,78.34931801)(699.43815835,78.41931794)(699.42815674,78.49931936)
\curveto(699.41815837,78.51931784)(699.41315837,78.53931782)(699.41315674,78.55931936)
\curveto(699.41315837,78.58931777)(699.40815838,78.61431774)(699.39815674,78.63431936)
}
}
{
\newrgbcolor{curcolor}{0 0 0}
\pscustom[linestyle=none,fillstyle=solid,fillcolor=curcolor]
{
}
}
{
\newrgbcolor{curcolor}{0 0 0}
\pscustom[linestyle=none,fillstyle=solid,fillcolor=curcolor]
{
\newpath
\moveto(690.42815674,89.26463186)
\curveto(690.41816737,89.95462722)(690.53816725,90.55462662)(690.78815674,91.06463186)
\curveto(691.03816675,91.58462559)(691.37316641,91.9796252)(691.79315674,92.24963186)
\curveto(691.87316591,92.29962488)(691.96316582,92.34462483)(692.06315674,92.38463186)
\curveto(692.15316563,92.42462475)(692.24816554,92.46962471)(692.34815674,92.51963186)
\curveto(692.44816534,92.55962462)(692.54816524,92.58962459)(692.64815674,92.60963186)
\curveto(692.74816504,92.62962455)(692.85316493,92.64962453)(692.96315674,92.66963186)
\curveto(693.01316477,92.68962449)(693.05816473,92.69462448)(693.09815674,92.68463186)
\curveto(693.13816465,92.6746245)(693.1831646,92.6796245)(693.23315674,92.69963186)
\curveto(693.2831645,92.70962447)(693.36816442,92.71462446)(693.48815674,92.71463186)
\curveto(693.59816419,92.71462446)(693.6831641,92.70962447)(693.74315674,92.69963186)
\curveto(693.80316398,92.6796245)(693.86316392,92.66962451)(693.92315674,92.66963186)
\curveto(693.9831638,92.6796245)(694.04316374,92.6746245)(694.10315674,92.65463186)
\curveto(694.24316354,92.61462456)(694.37816341,92.5796246)(694.50815674,92.54963186)
\curveto(694.63816315,92.51962466)(694.76316302,92.4796247)(694.88315674,92.42963186)
\curveto(695.02316276,92.36962481)(695.14816264,92.29962488)(695.25815674,92.21963186)
\curveto(695.36816242,92.14962503)(695.47816231,92.0746251)(695.58815674,91.99463186)
\lineto(695.64815674,91.93463186)
\curveto(695.66816212,91.92462525)(695.6881621,91.90962527)(695.70815674,91.88963186)
\curveto(695.86816192,91.76962541)(696.01316177,91.63462554)(696.14315674,91.48463186)
\curveto(696.27316151,91.33462584)(696.39816139,91.174626)(696.51815674,91.00463186)
\curveto(696.73816105,90.69462648)(696.94316084,90.39962678)(697.13315674,90.11963186)
\curveto(697.27316051,89.88962729)(697.40816038,89.65962752)(697.53815674,89.42963186)
\curveto(697.66816012,89.20962797)(697.80315998,88.98962819)(697.94315674,88.76963186)
\curveto(698.11315967,88.51962866)(698.29315949,88.2796289)(698.48315674,88.04963186)
\curveto(698.67315911,87.82962935)(698.89815889,87.63962954)(699.15815674,87.47963186)
\curveto(699.21815857,87.43962974)(699.27815851,87.40462977)(699.33815674,87.37463186)
\curveto(699.3881584,87.34462983)(699.45315833,87.31462986)(699.53315674,87.28463186)
\curveto(699.60315818,87.26462991)(699.66315812,87.25962992)(699.71315674,87.26963186)
\curveto(699.783158,87.28962989)(699.83815795,87.32462985)(699.87815674,87.37463186)
\curveto(699.90815788,87.42462975)(699.92815786,87.48462969)(699.93815674,87.55463186)
\lineto(699.93815674,87.79463186)
\lineto(699.93815674,88.54463186)
\lineto(699.93815674,91.34963186)
\lineto(699.93815674,92.00963186)
\curveto(699.93815785,92.09962508)(699.94315784,92.18462499)(699.95315674,92.26463186)
\curveto(699.95315783,92.34462483)(699.97315781,92.40962477)(700.01315674,92.45963186)
\curveto(700.05315773,92.50962467)(700.12815766,92.54962463)(700.23815674,92.57963186)
\curveto(700.33815745,92.61962456)(700.43815735,92.62962455)(700.53815674,92.60963186)
\lineto(700.67315674,92.60963186)
\curveto(700.74315704,92.58962459)(700.80315698,92.56962461)(700.85315674,92.54963186)
\curveto(700.90315688,92.52962465)(700.94315684,92.49462468)(700.97315674,92.44463186)
\curveto(701.01315677,92.39462478)(701.03315675,92.32462485)(701.03315674,92.23463186)
\lineto(701.03315674,91.96463186)
\lineto(701.03315674,91.06463186)
\lineto(701.03315674,87.55463186)
\lineto(701.03315674,86.48963186)
\curveto(701.03315675,86.40963077)(701.03815675,86.31963086)(701.04815674,86.21963186)
\curveto(701.04815674,86.11963106)(701.03815675,86.03463114)(701.01815674,85.96463186)
\curveto(700.94815684,85.75463142)(700.76815702,85.68963149)(700.47815674,85.76963186)
\curveto(700.43815735,85.7796314)(700.40315738,85.7796314)(700.37315674,85.76963186)
\curveto(700.33315745,85.76963141)(700.2881575,85.7796314)(700.23815674,85.79963186)
\curveto(700.15815763,85.81963136)(700.07315771,85.83963134)(699.98315674,85.85963186)
\curveto(699.89315789,85.8796313)(699.80815798,85.90463127)(699.72815674,85.93463186)
\curveto(699.23815855,86.09463108)(698.82315896,86.29463088)(698.48315674,86.53463186)
\curveto(698.23315955,86.71463046)(698.00815978,86.91963026)(697.80815674,87.14963186)
\curveto(697.59816019,87.3796298)(697.40316038,87.61962956)(697.22315674,87.86963186)
\curveto(697.04316074,88.12962905)(696.87316091,88.39462878)(696.71315674,88.66463186)
\curveto(696.54316124,88.94462823)(696.36816142,89.21462796)(696.18815674,89.47463186)
\curveto(696.10816168,89.58462759)(696.03316175,89.68962749)(695.96315674,89.78963186)
\curveto(695.89316189,89.89962728)(695.81816197,90.00962717)(695.73815674,90.11963186)
\curveto(695.70816208,90.15962702)(695.67816211,90.19462698)(695.64815674,90.22463186)
\curveto(695.60816218,90.26462691)(695.57816221,90.30462687)(695.55815674,90.34463186)
\curveto(695.44816234,90.48462669)(695.32316246,90.60962657)(695.18315674,90.71963186)
\curveto(695.15316263,90.73962644)(695.12816266,90.76462641)(695.10815674,90.79463186)
\curveto(695.07816271,90.82462635)(695.04816274,90.84962633)(695.01815674,90.86963186)
\curveto(694.91816287,90.94962623)(694.81816297,91.01462616)(694.71815674,91.06463186)
\curveto(694.61816317,91.12462605)(694.50816328,91.179626)(694.38815674,91.22963186)
\curveto(694.31816347,91.25962592)(694.24316354,91.2796259)(694.16315674,91.28963186)
\lineto(693.92315674,91.34963186)
\lineto(693.83315674,91.34963186)
\curveto(693.80316398,91.35962582)(693.77316401,91.36462581)(693.74315674,91.36463186)
\curveto(693.67316411,91.38462579)(693.57816421,91.38962579)(693.45815674,91.37963186)
\curveto(693.32816446,91.3796258)(693.22816456,91.36962581)(693.15815674,91.34963186)
\curveto(693.07816471,91.32962585)(693.00316478,91.30962587)(692.93315674,91.28963186)
\curveto(692.85316493,91.2796259)(692.77316501,91.25962592)(692.69315674,91.22963186)
\curveto(692.45316533,91.11962606)(692.25316553,90.96962621)(692.09315674,90.77963186)
\curveto(691.92316586,90.59962658)(691.783166,90.3796268)(691.67315674,90.11963186)
\curveto(691.65316613,90.04962713)(691.63816615,89.9796272)(691.62815674,89.90963186)
\curveto(691.60816618,89.83962734)(691.5881662,89.76462741)(691.56815674,89.68463186)
\curveto(691.54816624,89.60462757)(691.53816625,89.49462768)(691.53815674,89.35463186)
\curveto(691.53816625,89.22462795)(691.54816624,89.11962806)(691.56815674,89.03963186)
\curveto(691.57816621,88.9796282)(691.5831662,88.92462825)(691.58315674,88.87463186)
\curveto(691.5831662,88.82462835)(691.59316619,88.7746284)(691.61315674,88.72463186)
\curveto(691.65316613,88.62462855)(691.69316609,88.52962865)(691.73315674,88.43963186)
\curveto(691.77316601,88.35962882)(691.81816597,88.2796289)(691.86815674,88.19963186)
\curveto(691.8881659,88.16962901)(691.91316587,88.13962904)(691.94315674,88.10963186)
\curveto(691.97316581,88.08962909)(691.99816579,88.06462911)(692.01815674,88.03463186)
\lineto(692.09315674,87.95963186)
\curveto(692.11316567,87.92962925)(692.13316565,87.90462927)(692.15315674,87.88463186)
\lineto(692.36315674,87.73463186)
\curveto(692.42316536,87.69462948)(692.4881653,87.64962953)(692.55815674,87.59963186)
\curveto(692.64816514,87.53962964)(692.75316503,87.48962969)(692.87315674,87.44963186)
\curveto(692.9831648,87.41962976)(693.09316469,87.38462979)(693.20315674,87.34463186)
\curveto(693.31316447,87.30462987)(693.45816433,87.2796299)(693.63815674,87.26963186)
\curveto(693.80816398,87.25962992)(693.93316385,87.22962995)(694.01315674,87.17963186)
\curveto(694.09316369,87.12963005)(694.13816365,87.05463012)(694.14815674,86.95463186)
\curveto(694.15816363,86.85463032)(694.16316362,86.74463043)(694.16315674,86.62463186)
\curveto(694.16316362,86.58463059)(694.16816362,86.54463063)(694.17815674,86.50463186)
\curveto(694.17816361,86.46463071)(694.17316361,86.42963075)(694.16315674,86.39963186)
\curveto(694.14316364,86.34963083)(694.13316365,86.29963088)(694.13315674,86.24963186)
\curveto(694.13316365,86.20963097)(694.12316366,86.16963101)(694.10315674,86.12963186)
\curveto(694.04316374,86.03963114)(693.90816388,85.99463118)(693.69815674,85.99463186)
\lineto(693.57815674,85.99463186)
\curveto(693.51816427,86.00463117)(693.45816433,86.00963117)(693.39815674,86.00963186)
\curveto(693.32816446,86.01963116)(693.26316452,86.02963115)(693.20315674,86.03963186)
\curveto(693.09316469,86.05963112)(692.99316479,86.0796311)(692.90315674,86.09963186)
\curveto(692.80316498,86.11963106)(692.70816508,86.14963103)(692.61815674,86.18963186)
\curveto(692.54816524,86.20963097)(692.4881653,86.22963095)(692.43815674,86.24963186)
\lineto(692.25815674,86.30963186)
\curveto(691.99816579,86.42963075)(691.75316603,86.58463059)(691.52315674,86.77463186)
\curveto(691.29316649,86.9746302)(691.10816668,87.18962999)(690.96815674,87.41963186)
\curveto(690.8881669,87.52962965)(690.82316696,87.64462953)(690.77315674,87.76463186)
\lineto(690.62315674,88.15463186)
\curveto(690.57316721,88.26462891)(690.54316724,88.3796288)(690.53315674,88.49963186)
\curveto(690.51316727,88.61962856)(690.4881673,88.74462843)(690.45815674,88.87463186)
\curveto(690.45816733,88.94462823)(690.45816733,89.00962817)(690.45815674,89.06963186)
\curveto(690.44816734,89.12962805)(690.43816735,89.19462798)(690.42815674,89.26463186)
}
}
{
\newrgbcolor{curcolor}{0 0 0}
\pscustom[linestyle=none,fillstyle=solid,fillcolor=curcolor]
{
\newpath
\moveto(695.94815674,101.36424123)
\lineto(696.20315674,101.36424123)
\curveto(696.2831615,101.37423353)(696.35816143,101.36923353)(696.42815674,101.34924123)
\lineto(696.66815674,101.34924123)
\lineto(696.83315674,101.34924123)
\curveto(696.93316085,101.32923357)(697.03816075,101.31923358)(697.14815674,101.31924123)
\curveto(697.24816054,101.31923358)(697.34816044,101.30923359)(697.44815674,101.28924123)
\lineto(697.59815674,101.28924123)
\curveto(697.73816005,101.25923364)(697.87815991,101.23923366)(698.01815674,101.22924123)
\curveto(698.14815964,101.21923368)(698.27815951,101.19423371)(698.40815674,101.15424123)
\curveto(698.4881593,101.13423377)(698.57315921,101.11423379)(698.66315674,101.09424123)
\lineto(698.90315674,101.03424123)
\lineto(699.20315674,100.91424123)
\curveto(699.29315849,100.88423402)(699.3831584,100.84923405)(699.47315674,100.80924123)
\curveto(699.69315809,100.70923419)(699.90815788,100.57423433)(700.11815674,100.40424123)
\curveto(700.32815746,100.24423466)(700.49815729,100.06923483)(700.62815674,99.87924123)
\curveto(700.66815712,99.82923507)(700.70815708,99.76923513)(700.74815674,99.69924123)
\curveto(700.77815701,99.63923526)(700.81315697,99.57923532)(700.85315674,99.51924123)
\curveto(700.90315688,99.43923546)(700.94315684,99.34423556)(700.97315674,99.23424123)
\curveto(701.00315678,99.12423578)(701.03315675,99.01923588)(701.06315674,98.91924123)
\curveto(701.10315668,98.80923609)(701.12815666,98.6992362)(701.13815674,98.58924123)
\curveto(701.14815664,98.47923642)(701.16315662,98.36423654)(701.18315674,98.24424123)
\curveto(701.19315659,98.2042367)(701.19315659,98.15923674)(701.18315674,98.10924123)
\curveto(701.1831566,98.06923683)(701.1881566,98.02923687)(701.19815674,97.98924123)
\curveto(701.20815658,97.94923695)(701.21315657,97.89423701)(701.21315674,97.82424123)
\curveto(701.21315657,97.75423715)(701.20815658,97.7042372)(701.19815674,97.67424123)
\curveto(701.17815661,97.62423728)(701.17315661,97.57923732)(701.18315674,97.53924123)
\curveto(701.19315659,97.4992374)(701.19315659,97.46423744)(701.18315674,97.43424123)
\lineto(701.18315674,97.34424123)
\curveto(701.16315662,97.28423762)(701.14815664,97.21923768)(701.13815674,97.14924123)
\curveto(701.13815665,97.08923781)(701.13315665,97.02423788)(701.12315674,96.95424123)
\curveto(701.07315671,96.78423812)(701.02315676,96.62423828)(700.97315674,96.47424123)
\curveto(700.92315686,96.32423858)(700.85815693,96.17923872)(700.77815674,96.03924123)
\curveto(700.73815705,95.98923891)(700.70815708,95.93423897)(700.68815674,95.87424123)
\curveto(700.65815713,95.82423908)(700.62315716,95.77423913)(700.58315674,95.72424123)
\curveto(700.40315738,95.48423942)(700.1831576,95.28423962)(699.92315674,95.12424123)
\curveto(699.66315812,94.96423994)(699.37815841,94.82424008)(699.06815674,94.70424123)
\curveto(698.92815886,94.64424026)(698.788159,94.5992403)(698.64815674,94.56924123)
\curveto(698.49815929,94.53924036)(698.34315944,94.5042404)(698.18315674,94.46424123)
\curveto(698.07315971,94.44424046)(697.96315982,94.42924047)(697.85315674,94.41924123)
\curveto(697.74316004,94.40924049)(697.63316015,94.39424051)(697.52315674,94.37424123)
\curveto(697.4831603,94.36424054)(697.44316034,94.35924054)(697.40315674,94.35924123)
\curveto(697.36316042,94.36924053)(697.32316046,94.36924053)(697.28315674,94.35924123)
\curveto(697.23316055,94.34924055)(697.1831606,94.34424056)(697.13315674,94.34424123)
\lineto(696.96815674,94.34424123)
\curveto(696.91816087,94.32424058)(696.86816092,94.31924058)(696.81815674,94.32924123)
\curveto(696.75816103,94.33924056)(696.70316108,94.33924056)(696.65315674,94.32924123)
\curveto(696.61316117,94.31924058)(696.56816122,94.31924058)(696.51815674,94.32924123)
\curveto(696.46816132,94.33924056)(696.41816137,94.33424057)(696.36815674,94.31424123)
\curveto(696.29816149,94.29424061)(696.22316156,94.28924061)(696.14315674,94.29924123)
\curveto(696.05316173,94.30924059)(695.96816182,94.31424059)(695.88815674,94.31424123)
\curveto(695.79816199,94.31424059)(695.69816209,94.30924059)(695.58815674,94.29924123)
\curveto(695.46816232,94.28924061)(695.36816242,94.29424061)(695.28815674,94.31424123)
\lineto(695.00315674,94.31424123)
\lineto(694.37315674,94.35924123)
\curveto(694.27316351,94.36924053)(694.17816361,94.37924052)(694.08815674,94.38924123)
\lineto(693.78815674,94.41924123)
\curveto(693.73816405,94.43924046)(693.6881641,94.44424046)(693.63815674,94.43424123)
\curveto(693.57816421,94.43424047)(693.52316426,94.44424046)(693.47315674,94.46424123)
\curveto(693.30316448,94.51424039)(693.13816465,94.55424035)(692.97815674,94.58424123)
\curveto(692.80816498,94.61424029)(692.64816514,94.66424024)(692.49815674,94.73424123)
\curveto(692.03816575,94.92423998)(691.66316612,95.14423976)(691.37315674,95.39424123)
\curveto(691.0831667,95.65423925)(690.83816695,96.01423889)(690.63815674,96.47424123)
\curveto(690.5881672,96.6042383)(690.55316723,96.73423817)(690.53315674,96.86424123)
\curveto(690.51316727,97.0042379)(690.4881673,97.14423776)(690.45815674,97.28424123)
\curveto(690.44816734,97.35423755)(690.44316734,97.41923748)(690.44315674,97.47924123)
\curveto(690.44316734,97.53923736)(690.43816735,97.6042373)(690.42815674,97.67424123)
\curveto(690.40816738,98.5042364)(690.55816723,99.17423573)(690.87815674,99.68424123)
\curveto(691.1881666,100.19423471)(691.62816616,100.57423433)(692.19815674,100.82424123)
\curveto(692.31816547,100.87423403)(692.44316534,100.91923398)(692.57315674,100.95924123)
\curveto(692.70316508,100.9992339)(692.83816495,101.04423386)(692.97815674,101.09424123)
\curveto(693.05816473,101.11423379)(693.14316464,101.12923377)(693.23315674,101.13924123)
\lineto(693.47315674,101.19924123)
\curveto(693.5831642,101.22923367)(693.69316409,101.24423366)(693.80315674,101.24424123)
\curveto(693.91316387,101.25423365)(694.02316376,101.26923363)(694.13315674,101.28924123)
\curveto(694.1831636,101.30923359)(694.22816356,101.31423359)(694.26815674,101.30424123)
\curveto(694.30816348,101.3042336)(694.34816344,101.30923359)(694.38815674,101.31924123)
\curveto(694.43816335,101.32923357)(694.49316329,101.32923357)(694.55315674,101.31924123)
\curveto(694.60316318,101.31923358)(694.65316313,101.32423358)(694.70315674,101.33424123)
\lineto(694.83815674,101.33424123)
\curveto(694.89816289,101.35423355)(694.96816282,101.35423355)(695.04815674,101.33424123)
\curveto(695.11816267,101.32423358)(695.1831626,101.32923357)(695.24315674,101.34924123)
\curveto(695.27316251,101.35923354)(695.31316247,101.36423354)(695.36315674,101.36424123)
\lineto(695.48315674,101.36424123)
\lineto(695.94815674,101.36424123)
\moveto(698.27315674,99.81924123)
\curveto(697.95315983,99.91923498)(697.5881602,99.97923492)(697.17815674,99.99924123)
\curveto(696.76816102,100.01923488)(696.35816143,100.02923487)(695.94815674,100.02924123)
\curveto(695.51816227,100.02923487)(695.09816269,100.01923488)(694.68815674,99.99924123)
\curveto(694.27816351,99.97923492)(693.89316389,99.93423497)(693.53315674,99.86424123)
\curveto(693.17316461,99.79423511)(692.85316493,99.68423522)(692.57315674,99.53424123)
\curveto(692.2831655,99.39423551)(692.04816574,99.1992357)(691.86815674,98.94924123)
\curveto(691.75816603,98.78923611)(691.67816611,98.60923629)(691.62815674,98.40924123)
\curveto(691.56816622,98.20923669)(691.53816625,97.96423694)(691.53815674,97.67424123)
\curveto(691.55816623,97.65423725)(691.56816622,97.61923728)(691.56815674,97.56924123)
\curveto(691.55816623,97.51923738)(691.55816623,97.47923742)(691.56815674,97.44924123)
\curveto(691.5881662,97.36923753)(691.60816618,97.29423761)(691.62815674,97.22424123)
\curveto(691.63816615,97.16423774)(691.65816613,97.0992378)(691.68815674,97.02924123)
\curveto(691.80816598,96.75923814)(691.97816581,96.53923836)(692.19815674,96.36924123)
\curveto(692.40816538,96.20923869)(692.65316513,96.07423883)(692.93315674,95.96424123)
\curveto(693.04316474,95.91423899)(693.16316462,95.87423903)(693.29315674,95.84424123)
\curveto(693.41316437,95.82423908)(693.53816425,95.7992391)(693.66815674,95.76924123)
\curveto(693.71816407,95.74923915)(693.77316401,95.73923916)(693.83315674,95.73924123)
\curveto(693.8831639,95.73923916)(693.93316385,95.73423917)(693.98315674,95.72424123)
\curveto(694.07316371,95.71423919)(694.16816362,95.7042392)(694.26815674,95.69424123)
\curveto(694.35816343,95.68423922)(694.45316333,95.67423923)(694.55315674,95.66424123)
\curveto(694.63316315,95.66423924)(694.71816307,95.65923924)(694.80815674,95.64924123)
\lineto(695.04815674,95.64924123)
\lineto(695.22815674,95.64924123)
\curveto(695.25816253,95.63923926)(695.29316249,95.63423927)(695.33315674,95.63424123)
\lineto(695.46815674,95.63424123)
\lineto(695.91815674,95.63424123)
\curveto(695.99816179,95.63423927)(696.0831617,95.62923927)(696.17315674,95.61924123)
\curveto(696.25316153,95.61923928)(696.32816146,95.62923927)(696.39815674,95.64924123)
\lineto(696.66815674,95.64924123)
\curveto(696.6881611,95.64923925)(696.71816107,95.64423926)(696.75815674,95.63424123)
\curveto(696.788161,95.63423927)(696.81316097,95.63923926)(696.83315674,95.64924123)
\curveto(696.93316085,95.65923924)(697.03316075,95.66423924)(697.13315674,95.66424123)
\curveto(697.22316056,95.67423923)(697.32316046,95.68423922)(697.43315674,95.69424123)
\curveto(697.55316023,95.72423918)(697.67816011,95.73923916)(697.80815674,95.73924123)
\curveto(697.92815986,95.74923915)(698.04315974,95.77423913)(698.15315674,95.81424123)
\curveto(698.45315933,95.89423901)(698.71815907,95.97923892)(698.94815674,96.06924123)
\curveto(699.17815861,96.16923873)(699.39315839,96.31423859)(699.59315674,96.50424123)
\curveto(699.79315799,96.71423819)(699.94315784,96.97923792)(700.04315674,97.29924123)
\curveto(700.06315772,97.33923756)(700.07315771,97.37423753)(700.07315674,97.40424123)
\curveto(700.06315772,97.44423746)(700.06815772,97.48923741)(700.08815674,97.53924123)
\curveto(700.09815769,97.57923732)(700.10815768,97.64923725)(700.11815674,97.74924123)
\curveto(700.12815766,97.85923704)(700.12315766,97.94423696)(700.10315674,98.00424123)
\curveto(700.0831577,98.07423683)(700.07315771,98.14423676)(700.07315674,98.21424123)
\curveto(700.06315772,98.28423662)(700.04815774,98.34923655)(700.02815674,98.40924123)
\curveto(699.96815782,98.60923629)(699.8831579,98.78923611)(699.77315674,98.94924123)
\curveto(699.75315803,98.97923592)(699.73315805,99.0042359)(699.71315674,99.02424123)
\lineto(699.65315674,99.08424123)
\curveto(699.63315815,99.12423578)(699.59315819,99.17423573)(699.53315674,99.23424123)
\curveto(699.39315839,99.33423557)(699.26315852,99.41923548)(699.14315674,99.48924123)
\curveto(699.02315876,99.55923534)(698.87815891,99.62923527)(698.70815674,99.69924123)
\curveto(698.63815915,99.72923517)(698.56815922,99.74923515)(698.49815674,99.75924123)
\curveto(698.42815936,99.77923512)(698.35315943,99.7992351)(698.27315674,99.81924123)
}
}
{
\newrgbcolor{curcolor}{0 0 0}
\pscustom[linestyle=none,fillstyle=solid,fillcolor=curcolor]
{
\newpath
\moveto(690.42815674,106.77385061)
\curveto(690.42816736,106.87384575)(690.43816735,106.96884566)(690.45815674,107.05885061)
\curveto(690.46816732,107.14884548)(690.49816729,107.21384541)(690.54815674,107.25385061)
\curveto(690.62816716,107.31384531)(690.73316705,107.34384528)(690.86315674,107.34385061)
\lineto(691.25315674,107.34385061)
\lineto(692.75315674,107.34385061)
\lineto(699.14315674,107.34385061)
\lineto(700.31315674,107.34385061)
\lineto(700.62815674,107.34385061)
\curveto(700.72815706,107.35384527)(700.80815698,107.33884529)(700.86815674,107.29885061)
\curveto(700.94815684,107.24884538)(700.99815679,107.17384545)(701.01815674,107.07385061)
\curveto(701.02815676,106.98384564)(701.03315675,106.87384575)(701.03315674,106.74385061)
\lineto(701.03315674,106.51885061)
\curveto(701.01315677,106.43884619)(700.99815679,106.36884626)(700.98815674,106.30885061)
\curveto(700.96815682,106.24884638)(700.92815686,106.19884643)(700.86815674,106.15885061)
\curveto(700.80815698,106.11884651)(700.73315705,106.09884653)(700.64315674,106.09885061)
\lineto(700.34315674,106.09885061)
\lineto(699.24815674,106.09885061)
\lineto(693.90815674,106.09885061)
\curveto(693.81816397,106.07884655)(693.74316404,106.06384656)(693.68315674,106.05385061)
\curveto(693.61316417,106.05384657)(693.55316423,106.0238466)(693.50315674,105.96385061)
\curveto(693.45316433,105.89384673)(693.42816436,105.80384682)(693.42815674,105.69385061)
\curveto(693.41816437,105.59384703)(693.41316437,105.48384714)(693.41315674,105.36385061)
\lineto(693.41315674,104.22385061)
\lineto(693.41315674,103.72885061)
\curveto(693.40316438,103.56884906)(693.34316444,103.45884917)(693.23315674,103.39885061)
\curveto(693.20316458,103.37884925)(693.17316461,103.36884926)(693.14315674,103.36885061)
\curveto(693.10316468,103.36884926)(693.05816473,103.36384926)(693.00815674,103.35385061)
\curveto(692.8881649,103.33384929)(692.77816501,103.33884929)(692.67815674,103.36885061)
\curveto(692.57816521,103.40884922)(692.50816528,103.46384916)(692.46815674,103.53385061)
\curveto(692.41816537,103.61384901)(692.39316539,103.73384889)(692.39315674,103.89385061)
\curveto(692.39316539,104.05384857)(692.37816541,104.18884844)(692.34815674,104.29885061)
\curveto(692.33816545,104.34884828)(692.33316545,104.40384822)(692.33315674,104.46385061)
\curveto(692.32316546,104.5238481)(692.30816548,104.58384804)(692.28815674,104.64385061)
\curveto(692.23816555,104.79384783)(692.1881656,104.93884769)(692.13815674,105.07885061)
\curveto(692.07816571,105.21884741)(692.00816578,105.35384727)(691.92815674,105.48385061)
\curveto(691.83816595,105.623847)(691.73316605,105.74384688)(691.61315674,105.84385061)
\curveto(691.49316629,105.94384668)(691.36316642,106.03884659)(691.22315674,106.12885061)
\curveto(691.12316666,106.18884644)(691.01316677,106.23384639)(690.89315674,106.26385061)
\curveto(690.77316701,106.30384632)(690.66816712,106.35384627)(690.57815674,106.41385061)
\curveto(690.51816727,106.46384616)(690.47816731,106.53384609)(690.45815674,106.62385061)
\curveto(690.44816734,106.64384598)(690.44316734,106.66884596)(690.44315674,106.69885061)
\curveto(690.44316734,106.7288459)(690.43816735,106.75384587)(690.42815674,106.77385061)
}
}
{
\newrgbcolor{curcolor}{0 0 0}
\pscustom[linestyle=none,fillstyle=solid,fillcolor=curcolor]
{
\newpath
\moveto(690.42815674,115.12345998)
\curveto(690.42816736,115.22345513)(690.43816735,115.31845503)(690.45815674,115.40845998)
\curveto(690.46816732,115.49845485)(690.49816729,115.56345479)(690.54815674,115.60345998)
\curveto(690.62816716,115.66345469)(690.73316705,115.69345466)(690.86315674,115.69345998)
\lineto(691.25315674,115.69345998)
\lineto(692.75315674,115.69345998)
\lineto(699.14315674,115.69345998)
\lineto(700.31315674,115.69345998)
\lineto(700.62815674,115.69345998)
\curveto(700.72815706,115.70345465)(700.80815698,115.68845466)(700.86815674,115.64845998)
\curveto(700.94815684,115.59845475)(700.99815679,115.52345483)(701.01815674,115.42345998)
\curveto(701.02815676,115.33345502)(701.03315675,115.22345513)(701.03315674,115.09345998)
\lineto(701.03315674,114.86845998)
\curveto(701.01315677,114.78845556)(700.99815679,114.71845563)(700.98815674,114.65845998)
\curveto(700.96815682,114.59845575)(700.92815686,114.5484558)(700.86815674,114.50845998)
\curveto(700.80815698,114.46845588)(700.73315705,114.4484559)(700.64315674,114.44845998)
\lineto(700.34315674,114.44845998)
\lineto(699.24815674,114.44845998)
\lineto(693.90815674,114.44845998)
\curveto(693.81816397,114.42845592)(693.74316404,114.41345594)(693.68315674,114.40345998)
\curveto(693.61316417,114.40345595)(693.55316423,114.37345598)(693.50315674,114.31345998)
\curveto(693.45316433,114.24345611)(693.42816436,114.1534562)(693.42815674,114.04345998)
\curveto(693.41816437,113.94345641)(693.41316437,113.83345652)(693.41315674,113.71345998)
\lineto(693.41315674,112.57345998)
\lineto(693.41315674,112.07845998)
\curveto(693.40316438,111.91845843)(693.34316444,111.80845854)(693.23315674,111.74845998)
\curveto(693.20316458,111.72845862)(693.17316461,111.71845863)(693.14315674,111.71845998)
\curveto(693.10316468,111.71845863)(693.05816473,111.71345864)(693.00815674,111.70345998)
\curveto(692.8881649,111.68345867)(692.77816501,111.68845866)(692.67815674,111.71845998)
\curveto(692.57816521,111.75845859)(692.50816528,111.81345854)(692.46815674,111.88345998)
\curveto(692.41816537,111.96345839)(692.39316539,112.08345827)(692.39315674,112.24345998)
\curveto(692.39316539,112.40345795)(692.37816541,112.53845781)(692.34815674,112.64845998)
\curveto(692.33816545,112.69845765)(692.33316545,112.7534576)(692.33315674,112.81345998)
\curveto(692.32316546,112.87345748)(692.30816548,112.93345742)(692.28815674,112.99345998)
\curveto(692.23816555,113.14345721)(692.1881656,113.28845706)(692.13815674,113.42845998)
\curveto(692.07816571,113.56845678)(692.00816578,113.70345665)(691.92815674,113.83345998)
\curveto(691.83816595,113.97345638)(691.73316605,114.09345626)(691.61315674,114.19345998)
\curveto(691.49316629,114.29345606)(691.36316642,114.38845596)(691.22315674,114.47845998)
\curveto(691.12316666,114.53845581)(691.01316677,114.58345577)(690.89315674,114.61345998)
\curveto(690.77316701,114.6534557)(690.66816712,114.70345565)(690.57815674,114.76345998)
\curveto(690.51816727,114.81345554)(690.47816731,114.88345547)(690.45815674,114.97345998)
\curveto(690.44816734,114.99345536)(690.44316734,115.01845533)(690.44315674,115.04845998)
\curveto(690.44316734,115.07845527)(690.43816735,115.10345525)(690.42815674,115.12345998)
}
}
{
\newrgbcolor{curcolor}{0 0 0}
\pscustom[linestyle=none,fillstyle=solid,fillcolor=curcolor]
{
\newpath
\moveto(711.26447266,37.28705373)
\curveto(711.26448335,37.35704805)(711.26448335,37.43704797)(711.26447266,37.52705373)
\curveto(711.25448336,37.61704779)(711.25448336,37.70204771)(711.26447266,37.78205373)
\curveto(711.26448335,37.87204754)(711.27448334,37.95204746)(711.29447266,38.02205373)
\curveto(711.3144833,38.10204731)(711.34448327,38.15704725)(711.38447266,38.18705373)
\curveto(711.43448318,38.21704719)(711.50948311,38.23704717)(711.60947266,38.24705373)
\curveto(711.69948292,38.26704714)(711.80448281,38.27704713)(711.92447266,38.27705373)
\curveto(712.03448258,38.28704712)(712.14948247,38.28704712)(712.26947266,38.27705373)
\lineto(712.56947266,38.27705373)
\lineto(715.58447266,38.27705373)
\lineto(718.47947266,38.27705373)
\curveto(718.80947581,38.27704713)(719.13447548,38.27204714)(719.45447266,38.26205373)
\curveto(719.76447485,38.26204715)(720.04447457,38.22204719)(720.29447266,38.14205373)
\curveto(720.64447397,38.02204739)(720.93947368,37.86704754)(721.17947266,37.67705373)
\curveto(721.40947321,37.48704792)(721.60947301,37.24704816)(721.77947266,36.95705373)
\curveto(721.82947279,36.89704851)(721.86447275,36.83204858)(721.88447266,36.76205373)
\curveto(721.90447271,36.70204871)(721.92947269,36.63204878)(721.95947266,36.55205373)
\curveto(722.00947261,36.43204898)(722.04447257,36.30204911)(722.06447266,36.16205373)
\curveto(722.09447252,36.03204938)(722.12447249,35.89704951)(722.15447266,35.75705373)
\curveto(722.17447244,35.7070497)(722.17947244,35.65704975)(722.16947266,35.60705373)
\curveto(722.15947246,35.55704985)(722.15947246,35.50204991)(722.16947266,35.44205373)
\curveto(722.17947244,35.42204999)(722.17947244,35.39705001)(722.16947266,35.36705373)
\curveto(722.16947245,35.33705007)(722.17447244,35.3120501)(722.18447266,35.29205373)
\curveto(722.19447242,35.25205016)(722.19947242,35.19705021)(722.19947266,35.12705373)
\curveto(722.19947242,35.05705035)(722.19447242,35.00205041)(722.18447266,34.96205373)
\curveto(722.17447244,34.9120505)(722.17447244,34.85705055)(722.18447266,34.79705373)
\curveto(722.19447242,34.73705067)(722.18947243,34.68205073)(722.16947266,34.63205373)
\curveto(722.13947248,34.50205091)(722.1194725,34.37705103)(722.10947266,34.25705373)
\curveto(722.09947252,34.13705127)(722.07447254,34.02205139)(722.03447266,33.91205373)
\curveto(721.9144727,33.54205187)(721.74447287,33.22205219)(721.52447266,32.95205373)
\curveto(721.30447331,32.68205273)(721.02447359,32.47205294)(720.68447266,32.32205373)
\curveto(720.56447405,32.27205314)(720.43947418,32.22705318)(720.30947266,32.18705373)
\curveto(720.17947444,32.15705325)(720.04447457,32.12205329)(719.90447266,32.08205373)
\curveto(719.85447476,32.07205334)(719.8144748,32.06705334)(719.78447266,32.06705373)
\curveto(719.74447487,32.06705334)(719.69947492,32.06205335)(719.64947266,32.05205373)
\curveto(719.619475,32.04205337)(719.58447503,32.03705337)(719.54447266,32.03705373)
\curveto(719.49447512,32.03705337)(719.45447516,32.03205338)(719.42447266,32.02205373)
\lineto(719.25947266,32.02205373)
\curveto(719.17947544,32.00205341)(719.07947554,31.99705341)(718.95947266,32.00705373)
\curveto(718.82947579,32.01705339)(718.73947588,32.03205338)(718.68947266,32.05205373)
\curveto(718.59947602,32.07205334)(718.53447608,32.12705328)(718.49447266,32.21705373)
\curveto(718.47447614,32.24705316)(718.46947615,32.27705313)(718.47947266,32.30705373)
\curveto(718.47947614,32.33705307)(718.47447614,32.37705303)(718.46447266,32.42705373)
\curveto(718.45447616,32.46705294)(718.44947617,32.5070529)(718.44947266,32.54705373)
\lineto(718.44947266,32.69705373)
\curveto(718.44947617,32.81705259)(718.45447616,32.93705247)(718.46447266,33.05705373)
\curveto(718.46447615,33.18705222)(718.49947612,33.27705213)(718.56947266,33.32705373)
\curveto(718.62947599,33.36705204)(718.68947593,33.38705202)(718.74947266,33.38705373)
\curveto(718.80947581,33.38705202)(718.87947574,33.39705201)(718.95947266,33.41705373)
\curveto(718.98947563,33.42705198)(719.02447559,33.42705198)(719.06447266,33.41705373)
\curveto(719.09447552,33.41705199)(719.1194755,33.42205199)(719.13947266,33.43205373)
\lineto(719.34947266,33.43205373)
\curveto(719.39947522,33.45205196)(719.44947517,33.45705195)(719.49947266,33.44705373)
\curveto(719.53947508,33.44705196)(719.58447503,33.45705195)(719.63447266,33.47705373)
\curveto(719.76447485,33.5070519)(719.88947473,33.53705187)(720.00947266,33.56705373)
\curveto(720.1194745,33.59705181)(720.22447439,33.64205177)(720.32447266,33.70205373)
\curveto(720.614474,33.87205154)(720.8194738,34.14205127)(720.93947266,34.51205373)
\curveto(720.95947366,34.56205085)(720.97447364,34.6120508)(720.98447266,34.66205373)
\curveto(720.98447363,34.72205069)(720.99447362,34.77705063)(721.01447266,34.82705373)
\lineto(721.01447266,34.90205373)
\curveto(721.02447359,34.97205044)(721.03447358,35.06705034)(721.04447266,35.18705373)
\curveto(721.04447357,35.31705009)(721.03447358,35.41704999)(721.01447266,35.48705373)
\curveto(720.99447362,35.55704985)(720.97947364,35.62704978)(720.96947266,35.69705373)
\curveto(720.94947367,35.77704963)(720.92947369,35.84704956)(720.90947266,35.90705373)
\curveto(720.74947387,36.28704912)(720.47447414,36.56204885)(720.08447266,36.73205373)
\curveto(719.95447466,36.78204863)(719.79947482,36.81704859)(719.61947266,36.83705373)
\curveto(719.43947518,36.86704854)(719.25447536,36.88204853)(719.06447266,36.88205373)
\curveto(718.86447575,36.89204852)(718.66447595,36.89204852)(718.46447266,36.88205373)
\lineto(717.89447266,36.88205373)
\lineto(713.64947266,36.88205373)
\lineto(712.10447266,36.88205373)
\curveto(711.99448262,36.88204853)(711.87448274,36.87704853)(711.74447266,36.86705373)
\curveto(711.614483,36.85704855)(711.50948311,36.87704853)(711.42947266,36.92705373)
\curveto(711.35948326,36.98704842)(711.30948331,37.06704834)(711.27947266,37.16705373)
\curveto(711.27948334,37.18704822)(711.27948334,37.2070482)(711.27947266,37.22705373)
\curveto(711.27948334,37.24704816)(711.27448334,37.26704814)(711.26447266,37.28705373)
}
}
{
\newrgbcolor{curcolor}{0 0 0}
\pscustom[linestyle=none,fillstyle=solid,fillcolor=curcolor]
{
\newpath
\moveto(714.21947266,40.82072561)
\lineto(714.21947266,41.25572561)
\curveto(714.2194804,41.40572364)(714.25948036,41.51072354)(714.33947266,41.57072561)
\curveto(714.4194802,41.62072343)(714.5194801,41.6457234)(714.63947266,41.64572561)
\curveto(714.75947986,41.65572339)(714.87947974,41.66072339)(714.99947266,41.66072561)
\lineto(716.42447266,41.66072561)
\lineto(718.68947266,41.66072561)
\lineto(719.37947266,41.66072561)
\curveto(719.60947501,41.66072339)(719.80947481,41.68572336)(719.97947266,41.73572561)
\curveto(720.42947419,41.89572315)(720.74447387,42.19572285)(720.92447266,42.63572561)
\curveto(721.0144736,42.85572219)(721.04947357,43.12072193)(721.02947266,43.43072561)
\curveto(720.99947362,43.74072131)(720.94447367,43.99072106)(720.86447266,44.18072561)
\curveto(720.72447389,44.51072054)(720.54947407,44.77072028)(720.33947266,44.96072561)
\curveto(720.1194745,45.16071989)(719.83447478,45.31571973)(719.48447266,45.42572561)
\curveto(719.40447521,45.45571959)(719.32447529,45.47571957)(719.24447266,45.48572561)
\curveto(719.16447545,45.49571955)(719.07947554,45.51071954)(718.98947266,45.53072561)
\curveto(718.93947568,45.54071951)(718.89447572,45.54071951)(718.85447266,45.53072561)
\curveto(718.8144758,45.53071952)(718.76947585,45.54071951)(718.71947266,45.56072561)
\lineto(718.40447266,45.56072561)
\curveto(718.32447629,45.58071947)(718.23447638,45.58571946)(718.13447266,45.57572561)
\curveto(718.02447659,45.56571948)(717.92447669,45.56071949)(717.83447266,45.56072561)
\lineto(716.66447266,45.56072561)
\lineto(715.07447266,45.56072561)
\curveto(714.95447966,45.56071949)(714.82947979,45.55571949)(714.69947266,45.54572561)
\curveto(714.55948006,45.5457195)(714.44948017,45.57071948)(714.36947266,45.62072561)
\curveto(714.3194803,45.66071939)(714.28948033,45.70571934)(714.27947266,45.75572561)
\curveto(714.25948036,45.81571923)(714.23948038,45.88571916)(714.21947266,45.96572561)
\lineto(714.21947266,46.19072561)
\curveto(714.2194804,46.31071874)(714.22448039,46.41571863)(714.23447266,46.50572561)
\curveto(714.24448037,46.60571844)(714.28948033,46.68071837)(714.36947266,46.73072561)
\curveto(714.4194802,46.78071827)(714.49448012,46.80571824)(714.59447266,46.80572561)
\lineto(714.87947266,46.80572561)
\lineto(715.89947266,46.80572561)
\lineto(719.93447266,46.80572561)
\lineto(721.28447266,46.80572561)
\curveto(721.40447321,46.80571824)(721.5194731,46.80071825)(721.62947266,46.79072561)
\curveto(721.72947289,46.79071826)(721.80447281,46.75571829)(721.85447266,46.68572561)
\curveto(721.88447273,46.6457184)(721.90947271,46.58571846)(721.92947266,46.50572561)
\curveto(721.93947268,46.42571862)(721.94947267,46.33571871)(721.95947266,46.23572561)
\curveto(721.95947266,46.1457189)(721.95447266,46.05571899)(721.94447266,45.96572561)
\curveto(721.93447268,45.88571916)(721.9144727,45.82571922)(721.88447266,45.78572561)
\curveto(721.84447277,45.73571931)(721.77947284,45.69071936)(721.68947266,45.65072561)
\curveto(721.64947297,45.64071941)(721.59447302,45.63071942)(721.52447266,45.62072561)
\curveto(721.45447316,45.62071943)(721.38947323,45.61571943)(721.32947266,45.60572561)
\curveto(721.25947336,45.59571945)(721.20447341,45.57571947)(721.16447266,45.54572561)
\curveto(721.12447349,45.51571953)(721.10947351,45.47071958)(721.11947266,45.41072561)
\curveto(721.13947348,45.33071972)(721.19947342,45.2507198)(721.29947266,45.17072561)
\curveto(721.38947323,45.09071996)(721.45947316,45.01572003)(721.50947266,44.94572561)
\curveto(721.66947295,44.72572032)(721.80947281,44.47572057)(721.92947266,44.19572561)
\curveto(721.97947264,44.08572096)(722.00947261,43.97072108)(722.01947266,43.85072561)
\curveto(722.03947258,43.74072131)(722.06447255,43.62572142)(722.09447266,43.50572561)
\curveto(722.10447251,43.45572159)(722.10447251,43.40072165)(722.09447266,43.34072561)
\curveto(722.08447253,43.29072176)(722.08947253,43.24072181)(722.10947266,43.19072561)
\curveto(722.12947249,43.09072196)(722.12947249,43.00072205)(722.10947266,42.92072561)
\lineto(722.10947266,42.77072561)
\curveto(722.08947253,42.72072233)(722.07947254,42.66072239)(722.07947266,42.59072561)
\curveto(722.07947254,42.53072252)(722.07447254,42.47572257)(722.06447266,42.42572561)
\curveto(722.04447257,42.38572266)(722.03447258,42.3457227)(722.03447266,42.30572561)
\curveto(722.04447257,42.27572277)(722.03947258,42.23572281)(722.01947266,42.18572561)
\lineto(721.95947266,41.94572561)
\curveto(721.93947268,41.87572317)(721.90947271,41.80072325)(721.86947266,41.72072561)
\curveto(721.75947286,41.46072359)(721.614473,41.24072381)(721.43447266,41.06072561)
\curveto(721.24447337,40.89072416)(721.0194736,40.7507243)(720.75947266,40.64072561)
\curveto(720.66947395,40.60072445)(720.57947404,40.57072448)(720.48947266,40.55072561)
\lineto(720.18947266,40.49072561)
\curveto(720.12947449,40.47072458)(720.07447454,40.46072459)(720.02447266,40.46072561)
\curveto(719.96447465,40.47072458)(719.89947472,40.46572458)(719.82947266,40.44572561)
\curveto(719.80947481,40.43572461)(719.78447483,40.43072462)(719.75447266,40.43072561)
\curveto(719.7144749,40.43072462)(719.67947494,40.42572462)(719.64947266,40.41572561)
\lineto(719.49947266,40.41572561)
\curveto(719.45947516,40.40572464)(719.4144752,40.40072465)(719.36447266,40.40072561)
\curveto(719.30447531,40.41072464)(719.24947537,40.41572463)(719.19947266,40.41572561)
\lineto(718.59947266,40.41572561)
\lineto(715.83947266,40.41572561)
\lineto(714.87947266,40.41572561)
\lineto(714.60947266,40.41572561)
\curveto(714.5194801,40.41572463)(714.44448017,40.43572461)(714.38447266,40.47572561)
\curveto(714.3144803,40.51572453)(714.26448035,40.59072446)(714.23447266,40.70072561)
\curveto(714.22448039,40.72072433)(714.22448039,40.74072431)(714.23447266,40.76072561)
\curveto(714.23448038,40.78072427)(714.22948039,40.80072425)(714.21947266,40.82072561)
}
}
{
\newrgbcolor{curcolor}{0 0 0}
\pscustom[linestyle=none,fillstyle=solid,fillcolor=curcolor]
{
\newpath
\moveto(714.06947266,52.39533498)
\curveto(714.04948057,53.02532975)(714.13448048,53.53032924)(714.32447266,53.91033498)
\curveto(714.5144801,54.29032848)(714.79947982,54.59532818)(715.17947266,54.82533498)
\curveto(715.27947934,54.88532789)(715.38947923,54.93032784)(715.50947266,54.96033498)
\curveto(715.619479,55.00032777)(715.73447888,55.03532774)(715.85447266,55.06533498)
\curveto(716.04447857,55.11532766)(716.24947837,55.14532763)(716.46947266,55.15533498)
\curveto(716.68947793,55.16532761)(716.9144777,55.1703276)(717.14447266,55.17033498)
\lineto(718.74947266,55.17033498)
\lineto(721.08947266,55.17033498)
\curveto(721.25947336,55.1703276)(721.42947319,55.16532761)(721.59947266,55.15533498)
\curveto(721.76947285,55.15532762)(721.87947274,55.09032768)(721.92947266,54.96033498)
\curveto(721.94947267,54.91032786)(721.95947266,54.85532792)(721.95947266,54.79533498)
\curveto(721.96947265,54.74532803)(721.97447264,54.69032808)(721.97447266,54.63033498)
\curveto(721.97447264,54.50032827)(721.96947265,54.3753284)(721.95947266,54.25533498)
\curveto(721.95947266,54.13532864)(721.9194727,54.05032872)(721.83947266,54.00033498)
\curveto(721.76947285,53.95032882)(721.67947294,53.92532885)(721.56947266,53.92533498)
\lineto(721.23947266,53.92533498)
\lineto(719.94947266,53.92533498)
\lineto(717.50447266,53.92533498)
\curveto(717.23447738,53.92532885)(716.96947765,53.92032885)(716.70947266,53.91033498)
\curveto(716.43947818,53.90032887)(716.20947841,53.85532892)(716.01947266,53.77533498)
\curveto(715.8194788,53.69532908)(715.65947896,53.5753292)(715.53947266,53.41533498)
\curveto(715.40947921,53.25532952)(715.30947931,53.0703297)(715.23947266,52.86033498)
\curveto(715.2194794,52.80032997)(715.20947941,52.73533004)(715.20947266,52.66533498)
\curveto(715.19947942,52.60533017)(715.18447943,52.54533023)(715.16447266,52.48533498)
\curveto(715.15447946,52.43533034)(715.15447946,52.35533042)(715.16447266,52.24533498)
\curveto(715.16447945,52.14533063)(715.16947945,52.0753307)(715.17947266,52.03533498)
\curveto(715.19947942,51.99533078)(715.20947941,51.96033081)(715.20947266,51.93033498)
\curveto(715.19947942,51.90033087)(715.19947942,51.86533091)(715.20947266,51.82533498)
\curveto(715.23947938,51.69533108)(715.27447934,51.5703312)(715.31447266,51.45033498)
\curveto(715.34447927,51.34033143)(715.38947923,51.23533154)(715.44947266,51.13533498)
\curveto(715.46947915,51.09533168)(715.48947913,51.06033171)(715.50947266,51.03033498)
\curveto(715.52947909,51.00033177)(715.54947907,50.96533181)(715.56947266,50.92533498)
\curveto(715.8194788,50.5753322)(716.19447842,50.32033245)(716.69447266,50.16033498)
\curveto(716.77447784,50.13033264)(716.85947776,50.11033266)(716.94947266,50.10033498)
\curveto(717.02947759,50.09033268)(717.10947751,50.0753327)(717.18947266,50.05533498)
\curveto(717.23947738,50.03533274)(717.28947733,50.03033274)(717.33947266,50.04033498)
\curveto(717.37947724,50.05033272)(717.4194772,50.04533273)(717.45947266,50.02533498)
\lineto(717.77447266,50.02533498)
\curveto(717.80447681,50.01533276)(717.83947678,50.01033276)(717.87947266,50.01033498)
\curveto(717.9194767,50.02033275)(717.96447665,50.02533275)(718.01447266,50.02533498)
\lineto(718.46447266,50.02533498)
\lineto(719.90447266,50.02533498)
\lineto(721.22447266,50.02533498)
\lineto(721.56947266,50.02533498)
\curveto(721.67947294,50.02533275)(721.76947285,50.00033277)(721.83947266,49.95033498)
\curveto(721.9194727,49.90033287)(721.95947266,49.81033296)(721.95947266,49.68033498)
\curveto(721.96947265,49.56033321)(721.97447264,49.43533334)(721.97447266,49.30533498)
\curveto(721.97447264,49.22533355)(721.96947265,49.15033362)(721.95947266,49.08033498)
\curveto(721.94947267,49.01033376)(721.92447269,48.95033382)(721.88447266,48.90033498)
\curveto(721.83447278,48.82033395)(721.73947288,48.78033399)(721.59947266,48.78033498)
\lineto(721.19447266,48.78033498)
\lineto(719.42447266,48.78033498)
\lineto(715.79447266,48.78033498)
\lineto(714.87947266,48.78033498)
\lineto(714.60947266,48.78033498)
\curveto(714.5194801,48.78033399)(714.44948017,48.80033397)(714.39947266,48.84033498)
\curveto(714.33948028,48.8703339)(714.29948032,48.92033385)(714.27947266,48.99033498)
\curveto(714.26948035,49.03033374)(714.25948036,49.08533369)(714.24947266,49.15533498)
\curveto(714.23948038,49.23533354)(714.23448038,49.31533346)(714.23447266,49.39533498)
\curveto(714.23448038,49.4753333)(714.23948038,49.55033322)(714.24947266,49.62033498)
\curveto(714.25948036,49.70033307)(714.27448034,49.75533302)(714.29447266,49.78533498)
\curveto(714.36448025,49.89533288)(714.45448016,49.94533283)(714.56447266,49.93533498)
\curveto(714.66447995,49.92533285)(714.77947984,49.94033283)(714.90947266,49.98033498)
\curveto(714.96947965,50.00033277)(715.0194796,50.04033273)(715.05947266,50.10033498)
\curveto(715.06947955,50.22033255)(715.02447959,50.31533246)(714.92447266,50.38533498)
\curveto(714.82447979,50.46533231)(714.74447987,50.54533223)(714.68447266,50.62533498)
\curveto(714.58448003,50.76533201)(714.49448012,50.90533187)(714.41447266,51.04533498)
\curveto(714.32448029,51.19533158)(714.24948037,51.36533141)(714.18947266,51.55533498)
\curveto(714.15948046,51.63533114)(714.13948048,51.72033105)(714.12947266,51.81033498)
\curveto(714.1194805,51.91033086)(714.10448051,52.00533077)(714.08447266,52.09533498)
\curveto(714.07448054,52.14533063)(714.06948055,52.19533058)(714.06947266,52.24533498)
\lineto(714.06947266,52.39533498)
}
}
{
\newrgbcolor{curcolor}{0 0 0}
\pscustom[linestyle=none,fillstyle=solid,fillcolor=curcolor]
{
}
}
{
\newrgbcolor{curcolor}{0 0 0}
\pscustom[linestyle=none,fillstyle=solid,fillcolor=curcolor]
{
\newpath
\moveto(711.33947266,65.05510061)
\curveto(711.33948328,65.15509575)(711.34948327,65.25009566)(711.36947266,65.34010061)
\curveto(711.37948324,65.43009548)(711.40948321,65.49509541)(711.45947266,65.53510061)
\curveto(711.53948308,65.59509531)(711.64448297,65.62509528)(711.77447266,65.62510061)
\lineto(712.16447266,65.62510061)
\lineto(713.66447266,65.62510061)
\lineto(720.05447266,65.62510061)
\lineto(721.22447266,65.62510061)
\lineto(721.53947266,65.62510061)
\curveto(721.63947298,65.63509527)(721.7194729,65.62009529)(721.77947266,65.58010061)
\curveto(721.85947276,65.53009538)(721.90947271,65.45509545)(721.92947266,65.35510061)
\curveto(721.93947268,65.26509564)(721.94447267,65.15509575)(721.94447266,65.02510061)
\lineto(721.94447266,64.80010061)
\curveto(721.92447269,64.72009619)(721.90947271,64.65009626)(721.89947266,64.59010061)
\curveto(721.87947274,64.53009638)(721.83947278,64.48009643)(721.77947266,64.44010061)
\curveto(721.7194729,64.40009651)(721.64447297,64.38009653)(721.55447266,64.38010061)
\lineto(721.25447266,64.38010061)
\lineto(720.15947266,64.38010061)
\lineto(714.81947266,64.38010061)
\curveto(714.72947989,64.36009655)(714.65447996,64.34509656)(714.59447266,64.33510061)
\curveto(714.52448009,64.33509657)(714.46448015,64.3050966)(714.41447266,64.24510061)
\curveto(714.36448025,64.17509673)(714.33948028,64.08509682)(714.33947266,63.97510061)
\curveto(714.32948029,63.87509703)(714.32448029,63.76509714)(714.32447266,63.64510061)
\lineto(714.32447266,62.50510061)
\lineto(714.32447266,62.01010061)
\curveto(714.3144803,61.85009906)(714.25448036,61.74009917)(714.14447266,61.68010061)
\curveto(714.1144805,61.66009925)(714.08448053,61.65009926)(714.05447266,61.65010061)
\curveto(714.0144806,61.65009926)(713.96948065,61.64509926)(713.91947266,61.63510061)
\curveto(713.79948082,61.61509929)(713.68948093,61.62009929)(713.58947266,61.65010061)
\curveto(713.48948113,61.69009922)(713.4194812,61.74509916)(713.37947266,61.81510061)
\curveto(713.32948129,61.89509901)(713.30448131,62.01509889)(713.30447266,62.17510061)
\curveto(713.30448131,62.33509857)(713.28948133,62.47009844)(713.25947266,62.58010061)
\curveto(713.24948137,62.63009828)(713.24448137,62.68509822)(713.24447266,62.74510061)
\curveto(713.23448138,62.8050981)(713.2194814,62.86509804)(713.19947266,62.92510061)
\curveto(713.14948147,63.07509783)(713.09948152,63.22009769)(713.04947266,63.36010061)
\curveto(712.98948163,63.50009741)(712.9194817,63.63509727)(712.83947266,63.76510061)
\curveto(712.74948187,63.905097)(712.64448197,64.02509688)(712.52447266,64.12510061)
\curveto(712.40448221,64.22509668)(712.27448234,64.32009659)(712.13447266,64.41010061)
\curveto(712.03448258,64.47009644)(711.92448269,64.51509639)(711.80447266,64.54510061)
\curveto(711.68448293,64.58509632)(711.57948304,64.63509627)(711.48947266,64.69510061)
\curveto(711.42948319,64.74509616)(711.38948323,64.81509609)(711.36947266,64.90510061)
\curveto(711.35948326,64.92509598)(711.35448326,64.95009596)(711.35447266,64.98010061)
\curveto(711.35448326,65.0100959)(711.34948327,65.03509587)(711.33947266,65.05510061)
}
}
{
\newrgbcolor{curcolor}{0 0 0}
\pscustom[linestyle=none,fillstyle=solid,fillcolor=curcolor]
{
\newpath
\moveto(718.44947266,76.22470998)
\curveto(718.49947612,76.29470234)(718.56947605,76.3347023)(718.65947266,76.34470998)
\curveto(718.74947587,76.36470227)(718.85447576,76.37470226)(718.97447266,76.37470998)
\curveto(719.02447559,76.37470226)(719.07447554,76.36970226)(719.12447266,76.35970998)
\curveto(719.17447544,76.35970227)(719.2194754,76.34970228)(719.25947266,76.32970998)
\curveto(719.34947527,76.29970233)(719.40947521,76.23970239)(719.43947266,76.14970998)
\curveto(719.45947516,76.06970256)(719.46947515,75.97470266)(719.46947266,75.86470998)
\lineto(719.46947266,75.54970998)
\curveto(719.45947516,75.43970319)(719.46947515,75.3347033)(719.49947266,75.23470998)
\curveto(719.52947509,75.09470354)(719.60947501,75.00470363)(719.73947266,74.96470998)
\curveto(719.80947481,74.94470369)(719.89447472,74.9347037)(719.99447266,74.93470998)
\lineto(720.26447266,74.93470998)
\lineto(721.20947266,74.93470998)
\lineto(721.53947266,74.93470998)
\curveto(721.64947297,74.9347037)(721.73447288,74.91470372)(721.79447266,74.87470998)
\curveto(721.85447276,74.8347038)(721.89447272,74.78470385)(721.91447266,74.72470998)
\curveto(721.92447269,74.67470396)(721.93947268,74.60970402)(721.95947266,74.52970998)
\lineto(721.95947266,74.33470998)
\curveto(721.95947266,74.21470442)(721.95447266,74.10970452)(721.94447266,74.01970998)
\curveto(721.92447269,73.9297047)(721.87447274,73.85970477)(721.79447266,73.80970998)
\curveto(721.74447287,73.77970485)(721.67447294,73.76470487)(721.58447266,73.76470998)
\lineto(721.28447266,73.76470998)
\lineto(720.24947266,73.76470998)
\curveto(720.08947453,73.76470487)(719.94447467,73.75470488)(719.81447266,73.73470998)
\curveto(719.67447494,73.72470491)(719.57947504,73.66970496)(719.52947266,73.56970998)
\curveto(719.50947511,73.51970511)(719.49447512,73.44970518)(719.48447266,73.35970998)
\curveto(719.47447514,73.27970535)(719.46947515,73.18970544)(719.46947266,73.08970998)
\lineto(719.46947266,72.80470998)
\lineto(719.46947266,72.56470998)
\lineto(719.46947266,70.29970998)
\curveto(719.46947515,70.20970842)(719.47447514,70.10470853)(719.48447266,69.98470998)
\lineto(719.48447266,69.65470998)
\curveto(719.48447513,69.54470909)(719.47447514,69.44470919)(719.45447266,69.35470998)
\curveto(719.43447518,69.26470937)(719.39947522,69.20470943)(719.34947266,69.17470998)
\curveto(719.27947534,69.12470951)(719.18447543,69.09970953)(719.06447266,69.09970998)
\lineto(718.71947266,69.09970998)
\lineto(718.44947266,69.09970998)
\curveto(718.27947634,69.13970949)(718.13947648,69.19470944)(718.02947266,69.26470998)
\curveto(717.9194767,69.3347093)(717.80447681,69.41470922)(717.68447266,69.50470998)
\lineto(717.14447266,69.86470998)
\curveto(716.5144781,70.30470833)(715.89447872,70.73970789)(715.28447266,71.16970998)
\lineto(713.42447266,72.48970998)
\curveto(713.19448142,72.64970598)(712.97448164,72.80470583)(712.76447266,72.95470998)
\curveto(712.54448207,73.10470553)(712.3194823,73.25970537)(712.08947266,73.41970998)
\curveto(712.0194826,73.46970516)(711.95448266,73.51970511)(711.89447266,73.56970998)
\curveto(711.82448279,73.61970501)(711.74948287,73.66970496)(711.66947266,73.71970998)
\lineto(711.57947266,73.77970998)
\curveto(711.53948308,73.80970482)(711.50948311,73.83970479)(711.48947266,73.86970998)
\curveto(711.45948316,73.90970472)(711.43948318,73.94970468)(711.42947266,73.98970998)
\curveto(711.40948321,74.0297046)(711.38948323,74.07470456)(711.36947266,74.12470998)
\curveto(711.36948325,74.14470449)(711.37448324,74.16470447)(711.38447266,74.18470998)
\curveto(711.38448323,74.21470442)(711.37448324,74.23970439)(711.35447266,74.25970998)
\curveto(711.35448326,74.38970424)(711.35948326,74.50970412)(711.36947266,74.61970998)
\curveto(711.37948324,74.7297039)(711.42448319,74.80970382)(711.50447266,74.85970998)
\curveto(711.55448306,74.89970373)(711.62448299,74.91970371)(711.71447266,74.91970998)
\curveto(711.80448281,74.9297037)(711.89948272,74.9347037)(711.99947266,74.93470998)
\lineto(717.45947266,74.93470998)
\curveto(717.52947709,74.9347037)(717.60447701,74.9297037)(717.68447266,74.91970998)
\curveto(717.75447686,74.91970371)(717.82447679,74.92470371)(717.89447266,74.93470998)
\lineto(717.99947266,74.93470998)
\curveto(718.04947657,74.95470368)(718.10447651,74.96970366)(718.16447266,74.97970998)
\curveto(718.2144764,74.98970364)(718.25447636,75.01470362)(718.28447266,75.05470998)
\curveto(718.33447628,75.12470351)(718.36447625,75.20970342)(718.37447266,75.30970998)
\lineto(718.37447266,75.63970998)
\curveto(718.37447624,75.74970288)(718.37947624,75.85470278)(718.38947266,75.95470998)
\curveto(718.38947623,76.06470257)(718.40947621,76.15470248)(718.44947266,76.22470998)
\moveto(718.25447266,73.65970998)
\curveto(718.14447647,73.73970489)(717.97447664,73.77470486)(717.74447266,73.76470998)
\lineto(717.12947266,73.76470998)
\lineto(714.65447266,73.76470998)
\lineto(714.33947266,73.76470998)
\curveto(714.2194804,73.77470486)(714.1194805,73.76970486)(714.03947266,73.74970998)
\lineto(713.88947266,73.74970998)
\curveto(713.79948082,73.74970488)(713.7144809,73.7347049)(713.63447266,73.70470998)
\curveto(713.614481,73.69470494)(713.60448101,73.68470495)(713.60447266,73.67470998)
\lineto(713.55947266,73.62970998)
\curveto(713.54948107,73.60970502)(713.54448107,73.57970505)(713.54447266,73.53970998)
\curveto(713.56448105,73.51970511)(713.57948104,73.49970513)(713.58947266,73.47970998)
\curveto(713.58948103,73.46970516)(713.59448102,73.45470518)(713.60447266,73.43470998)
\curveto(713.65448096,73.37470526)(713.72448089,73.31470532)(713.81447266,73.25470998)
\curveto(713.90448071,73.19470544)(713.98448063,73.13970549)(714.05447266,73.08970998)
\curveto(714.19448042,72.98970564)(714.33948028,72.89470574)(714.48947266,72.80470998)
\curveto(714.62947999,72.71470592)(714.76947985,72.61970601)(714.90947266,72.51970998)
\lineto(715.68947266,71.97970998)
\curveto(715.94947867,71.80970682)(716.20947841,71.634707)(716.46947266,71.45470998)
\curveto(716.57947804,71.37470726)(716.68447793,71.29970733)(716.78447266,71.22970998)
\lineto(717.08447266,71.01970998)
\curveto(717.16447745,70.96970766)(717.23947738,70.91970771)(717.30947266,70.86970998)
\curveto(717.37947724,70.8297078)(717.45447716,70.78470785)(717.53447266,70.73470998)
\curveto(717.59447702,70.68470795)(717.65947696,70.634708)(717.72947266,70.58470998)
\curveto(717.78947683,70.54470809)(717.85947676,70.50470813)(717.93947266,70.46470998)
\curveto(717.99947662,70.42470821)(718.06947655,70.39970823)(718.14947266,70.38970998)
\curveto(718.2194764,70.37970825)(718.27447634,70.41470822)(718.31447266,70.49470998)
\curveto(718.36447625,70.56470807)(718.38947623,70.67470796)(718.38947266,70.82470998)
\curveto(718.37947624,70.98470765)(718.37447624,71.11970751)(718.37447266,71.22970998)
\lineto(718.37447266,72.90970998)
\lineto(718.37447266,73.34470998)
\curveto(718.37447624,73.49470514)(718.33447628,73.59970503)(718.25447266,73.65970998)
}
}
{
\newrgbcolor{curcolor}{0 0 0}
\pscustom[linestyle=none,fillstyle=solid,fillcolor=curcolor]
{
\newpath
\moveto(720.30947266,78.63431936)
\lineto(720.30947266,79.26431936)
\lineto(720.30947266,79.45931936)
\curveto(720.30947431,79.52931683)(720.3194743,79.58931677)(720.33947266,79.63931936)
\curveto(720.37947424,79.70931665)(720.4194742,79.7593166)(720.45947266,79.78931936)
\curveto(720.50947411,79.82931653)(720.57447404,79.84931651)(720.65447266,79.84931936)
\curveto(720.73447388,79.8593165)(720.8194738,79.86431649)(720.90947266,79.86431936)
\lineto(721.62947266,79.86431936)
\curveto(722.10947251,79.86431649)(722.5194721,79.80431655)(722.85947266,79.68431936)
\curveto(723.19947142,79.56431679)(723.47447114,79.36931699)(723.68447266,79.09931936)
\curveto(723.73447088,79.02931733)(723.77947084,78.9593174)(723.81947266,78.88931936)
\curveto(723.86947075,78.82931753)(723.9144707,78.7543176)(723.95447266,78.66431936)
\curveto(723.96447065,78.64431771)(723.97447064,78.61431774)(723.98447266,78.57431936)
\curveto(724.00447061,78.53431782)(724.00947061,78.48931787)(723.99947266,78.43931936)
\curveto(723.96947065,78.34931801)(723.89447072,78.29431806)(723.77447266,78.27431936)
\curveto(723.66447095,78.2543181)(723.56947105,78.26931809)(723.48947266,78.31931936)
\curveto(723.4194712,78.34931801)(723.35447126,78.39431796)(723.29447266,78.45431936)
\curveto(723.24447137,78.52431783)(723.19447142,78.58931777)(723.14447266,78.64931936)
\curveto(723.09447152,78.71931764)(723.0194716,78.77931758)(722.91947266,78.82931936)
\curveto(722.82947179,78.88931747)(722.73947188,78.93931742)(722.64947266,78.97931936)
\curveto(722.619472,78.99931736)(722.55947206,79.02431733)(722.46947266,79.05431936)
\curveto(722.38947223,79.08431727)(722.3194723,79.08931727)(722.25947266,79.06931936)
\curveto(722.1194725,79.03931732)(722.02947259,78.97931738)(721.98947266,78.88931936)
\curveto(721.95947266,78.80931755)(721.94447267,78.71931764)(721.94447266,78.61931936)
\curveto(721.94447267,78.51931784)(721.9194727,78.43431792)(721.86947266,78.36431936)
\curveto(721.79947282,78.27431808)(721.65947296,78.22931813)(721.44947266,78.22931936)
\lineto(720.89447266,78.22931936)
\lineto(720.66947266,78.22931936)
\curveto(720.58947403,78.23931812)(720.52447409,78.2593181)(720.47447266,78.28931936)
\curveto(720.39447422,78.34931801)(720.34947427,78.41931794)(720.33947266,78.49931936)
\curveto(720.32947429,78.51931784)(720.32447429,78.53931782)(720.32447266,78.55931936)
\curveto(720.32447429,78.58931777)(720.3194743,78.61431774)(720.30947266,78.63431936)
}
}
{
\newrgbcolor{curcolor}{0 0 0}
\pscustom[linestyle=none,fillstyle=solid,fillcolor=curcolor]
{
}
}
{
\newrgbcolor{curcolor}{0 0 0}
\pscustom[linestyle=none,fillstyle=solid,fillcolor=curcolor]
{
\newpath
\moveto(711.33947266,89.26463186)
\curveto(711.32948329,89.95462722)(711.44948317,90.55462662)(711.69947266,91.06463186)
\curveto(711.94948267,91.58462559)(712.28448233,91.9796252)(712.70447266,92.24963186)
\curveto(712.78448183,92.29962488)(712.87448174,92.34462483)(712.97447266,92.38463186)
\curveto(713.06448155,92.42462475)(713.15948146,92.46962471)(713.25947266,92.51963186)
\curveto(713.35948126,92.55962462)(713.45948116,92.58962459)(713.55947266,92.60963186)
\curveto(713.65948096,92.62962455)(713.76448085,92.64962453)(713.87447266,92.66963186)
\curveto(713.92448069,92.68962449)(713.96948065,92.69462448)(714.00947266,92.68463186)
\curveto(714.04948057,92.6746245)(714.09448052,92.6796245)(714.14447266,92.69963186)
\curveto(714.19448042,92.70962447)(714.27948034,92.71462446)(714.39947266,92.71463186)
\curveto(714.50948011,92.71462446)(714.59448002,92.70962447)(714.65447266,92.69963186)
\curveto(714.7144799,92.6796245)(714.77447984,92.66962451)(714.83447266,92.66963186)
\curveto(714.89447972,92.6796245)(714.95447966,92.6746245)(715.01447266,92.65463186)
\curveto(715.15447946,92.61462456)(715.28947933,92.5796246)(715.41947266,92.54963186)
\curveto(715.54947907,92.51962466)(715.67447894,92.4796247)(715.79447266,92.42963186)
\curveto(715.93447868,92.36962481)(716.05947856,92.29962488)(716.16947266,92.21963186)
\curveto(716.27947834,92.14962503)(716.38947823,92.0746251)(716.49947266,91.99463186)
\lineto(716.55947266,91.93463186)
\curveto(716.57947804,91.92462525)(716.59947802,91.90962527)(716.61947266,91.88963186)
\curveto(716.77947784,91.76962541)(716.92447769,91.63462554)(717.05447266,91.48463186)
\curveto(717.18447743,91.33462584)(717.30947731,91.174626)(717.42947266,91.00463186)
\curveto(717.64947697,90.69462648)(717.85447676,90.39962678)(718.04447266,90.11963186)
\curveto(718.18447643,89.88962729)(718.3194763,89.65962752)(718.44947266,89.42963186)
\curveto(718.57947604,89.20962797)(718.7144759,88.98962819)(718.85447266,88.76963186)
\curveto(719.02447559,88.51962866)(719.20447541,88.2796289)(719.39447266,88.04963186)
\curveto(719.58447503,87.82962935)(719.80947481,87.63962954)(720.06947266,87.47963186)
\curveto(720.12947449,87.43962974)(720.18947443,87.40462977)(720.24947266,87.37463186)
\curveto(720.29947432,87.34462983)(720.36447425,87.31462986)(720.44447266,87.28463186)
\curveto(720.5144741,87.26462991)(720.57447404,87.25962992)(720.62447266,87.26963186)
\curveto(720.69447392,87.28962989)(720.74947387,87.32462985)(720.78947266,87.37463186)
\curveto(720.8194738,87.42462975)(720.83947378,87.48462969)(720.84947266,87.55463186)
\lineto(720.84947266,87.79463186)
\lineto(720.84947266,88.54463186)
\lineto(720.84947266,91.34963186)
\lineto(720.84947266,92.00963186)
\curveto(720.84947377,92.09962508)(720.85447376,92.18462499)(720.86447266,92.26463186)
\curveto(720.86447375,92.34462483)(720.88447373,92.40962477)(720.92447266,92.45963186)
\curveto(720.96447365,92.50962467)(721.03947358,92.54962463)(721.14947266,92.57963186)
\curveto(721.24947337,92.61962456)(721.34947327,92.62962455)(721.44947266,92.60963186)
\lineto(721.58447266,92.60963186)
\curveto(721.65447296,92.58962459)(721.7144729,92.56962461)(721.76447266,92.54963186)
\curveto(721.8144728,92.52962465)(721.85447276,92.49462468)(721.88447266,92.44463186)
\curveto(721.92447269,92.39462478)(721.94447267,92.32462485)(721.94447266,92.23463186)
\lineto(721.94447266,91.96463186)
\lineto(721.94447266,91.06463186)
\lineto(721.94447266,87.55463186)
\lineto(721.94447266,86.48963186)
\curveto(721.94447267,86.40963077)(721.94947267,86.31963086)(721.95947266,86.21963186)
\curveto(721.95947266,86.11963106)(721.94947267,86.03463114)(721.92947266,85.96463186)
\curveto(721.85947276,85.75463142)(721.67947294,85.68963149)(721.38947266,85.76963186)
\curveto(721.34947327,85.7796314)(721.3144733,85.7796314)(721.28447266,85.76963186)
\curveto(721.24447337,85.76963141)(721.19947342,85.7796314)(721.14947266,85.79963186)
\curveto(721.06947355,85.81963136)(720.98447363,85.83963134)(720.89447266,85.85963186)
\curveto(720.80447381,85.8796313)(720.7194739,85.90463127)(720.63947266,85.93463186)
\curveto(720.14947447,86.09463108)(719.73447488,86.29463088)(719.39447266,86.53463186)
\curveto(719.14447547,86.71463046)(718.9194757,86.91963026)(718.71947266,87.14963186)
\curveto(718.50947611,87.3796298)(718.3144763,87.61962956)(718.13447266,87.86963186)
\curveto(717.95447666,88.12962905)(717.78447683,88.39462878)(717.62447266,88.66463186)
\curveto(717.45447716,88.94462823)(717.27947734,89.21462796)(717.09947266,89.47463186)
\curveto(717.0194776,89.58462759)(716.94447767,89.68962749)(716.87447266,89.78963186)
\curveto(716.80447781,89.89962728)(716.72947789,90.00962717)(716.64947266,90.11963186)
\curveto(716.619478,90.15962702)(716.58947803,90.19462698)(716.55947266,90.22463186)
\curveto(716.5194781,90.26462691)(716.48947813,90.30462687)(716.46947266,90.34463186)
\curveto(716.35947826,90.48462669)(716.23447838,90.60962657)(716.09447266,90.71963186)
\curveto(716.06447855,90.73962644)(716.03947858,90.76462641)(716.01947266,90.79463186)
\curveto(715.98947863,90.82462635)(715.95947866,90.84962633)(715.92947266,90.86963186)
\curveto(715.82947879,90.94962623)(715.72947889,91.01462616)(715.62947266,91.06463186)
\curveto(715.52947909,91.12462605)(715.4194792,91.179626)(715.29947266,91.22963186)
\curveto(715.22947939,91.25962592)(715.15447946,91.2796259)(715.07447266,91.28963186)
\lineto(714.83447266,91.34963186)
\lineto(714.74447266,91.34963186)
\curveto(714.7144799,91.35962582)(714.68447993,91.36462581)(714.65447266,91.36463186)
\curveto(714.58448003,91.38462579)(714.48948013,91.38962579)(714.36947266,91.37963186)
\curveto(714.23948038,91.3796258)(714.13948048,91.36962581)(714.06947266,91.34963186)
\curveto(713.98948063,91.32962585)(713.9144807,91.30962587)(713.84447266,91.28963186)
\curveto(713.76448085,91.2796259)(713.68448093,91.25962592)(713.60447266,91.22963186)
\curveto(713.36448125,91.11962606)(713.16448145,90.96962621)(713.00447266,90.77963186)
\curveto(712.83448178,90.59962658)(712.69448192,90.3796268)(712.58447266,90.11963186)
\curveto(712.56448205,90.04962713)(712.54948207,89.9796272)(712.53947266,89.90963186)
\curveto(712.5194821,89.83962734)(712.49948212,89.76462741)(712.47947266,89.68463186)
\curveto(712.45948216,89.60462757)(712.44948217,89.49462768)(712.44947266,89.35463186)
\curveto(712.44948217,89.22462795)(712.45948216,89.11962806)(712.47947266,89.03963186)
\curveto(712.48948213,88.9796282)(712.49448212,88.92462825)(712.49447266,88.87463186)
\curveto(712.49448212,88.82462835)(712.50448211,88.7746284)(712.52447266,88.72463186)
\curveto(712.56448205,88.62462855)(712.60448201,88.52962865)(712.64447266,88.43963186)
\curveto(712.68448193,88.35962882)(712.72948189,88.2796289)(712.77947266,88.19963186)
\curveto(712.79948182,88.16962901)(712.82448179,88.13962904)(712.85447266,88.10963186)
\curveto(712.88448173,88.08962909)(712.90948171,88.06462911)(712.92947266,88.03463186)
\lineto(713.00447266,87.95963186)
\curveto(713.02448159,87.92962925)(713.04448157,87.90462927)(713.06447266,87.88463186)
\lineto(713.27447266,87.73463186)
\curveto(713.33448128,87.69462948)(713.39948122,87.64962953)(713.46947266,87.59963186)
\curveto(713.55948106,87.53962964)(713.66448095,87.48962969)(713.78447266,87.44963186)
\curveto(713.89448072,87.41962976)(714.00448061,87.38462979)(714.11447266,87.34463186)
\curveto(714.22448039,87.30462987)(714.36948025,87.2796299)(714.54947266,87.26963186)
\curveto(714.7194799,87.25962992)(714.84447977,87.22962995)(714.92447266,87.17963186)
\curveto(715.00447961,87.12963005)(715.04947957,87.05463012)(715.05947266,86.95463186)
\curveto(715.06947955,86.85463032)(715.07447954,86.74463043)(715.07447266,86.62463186)
\curveto(715.07447954,86.58463059)(715.07947954,86.54463063)(715.08947266,86.50463186)
\curveto(715.08947953,86.46463071)(715.08447953,86.42963075)(715.07447266,86.39963186)
\curveto(715.05447956,86.34963083)(715.04447957,86.29963088)(715.04447266,86.24963186)
\curveto(715.04447957,86.20963097)(715.03447958,86.16963101)(715.01447266,86.12963186)
\curveto(714.95447966,86.03963114)(714.8194798,85.99463118)(714.60947266,85.99463186)
\lineto(714.48947266,85.99463186)
\curveto(714.42948019,86.00463117)(714.36948025,86.00963117)(714.30947266,86.00963186)
\curveto(714.23948038,86.01963116)(714.17448044,86.02963115)(714.11447266,86.03963186)
\curveto(714.00448061,86.05963112)(713.90448071,86.0796311)(713.81447266,86.09963186)
\curveto(713.7144809,86.11963106)(713.619481,86.14963103)(713.52947266,86.18963186)
\curveto(713.45948116,86.20963097)(713.39948122,86.22963095)(713.34947266,86.24963186)
\lineto(713.16947266,86.30963186)
\curveto(712.90948171,86.42963075)(712.66448195,86.58463059)(712.43447266,86.77463186)
\curveto(712.20448241,86.9746302)(712.0194826,87.18962999)(711.87947266,87.41963186)
\curveto(711.79948282,87.52962965)(711.73448288,87.64462953)(711.68447266,87.76463186)
\lineto(711.53447266,88.15463186)
\curveto(711.48448313,88.26462891)(711.45448316,88.3796288)(711.44447266,88.49963186)
\curveto(711.42448319,88.61962856)(711.39948322,88.74462843)(711.36947266,88.87463186)
\curveto(711.36948325,88.94462823)(711.36948325,89.00962817)(711.36947266,89.06963186)
\curveto(711.35948326,89.12962805)(711.34948327,89.19462798)(711.33947266,89.26463186)
}
}
{
\newrgbcolor{curcolor}{0 0 0}
\pscustom[linestyle=none,fillstyle=solid,fillcolor=curcolor]
{
\newpath
\moveto(716.85947266,101.36424123)
\lineto(717.11447266,101.36424123)
\curveto(717.19447742,101.37423353)(717.26947735,101.36923353)(717.33947266,101.34924123)
\lineto(717.57947266,101.34924123)
\lineto(717.74447266,101.34924123)
\curveto(717.84447677,101.32923357)(717.94947667,101.31923358)(718.05947266,101.31924123)
\curveto(718.15947646,101.31923358)(718.25947636,101.30923359)(718.35947266,101.28924123)
\lineto(718.50947266,101.28924123)
\curveto(718.64947597,101.25923364)(718.78947583,101.23923366)(718.92947266,101.22924123)
\curveto(719.05947556,101.21923368)(719.18947543,101.19423371)(719.31947266,101.15424123)
\curveto(719.39947522,101.13423377)(719.48447513,101.11423379)(719.57447266,101.09424123)
\lineto(719.81447266,101.03424123)
\lineto(720.11447266,100.91424123)
\curveto(720.20447441,100.88423402)(720.29447432,100.84923405)(720.38447266,100.80924123)
\curveto(720.60447401,100.70923419)(720.8194738,100.57423433)(721.02947266,100.40424123)
\curveto(721.23947338,100.24423466)(721.40947321,100.06923483)(721.53947266,99.87924123)
\curveto(721.57947304,99.82923507)(721.619473,99.76923513)(721.65947266,99.69924123)
\curveto(721.68947293,99.63923526)(721.72447289,99.57923532)(721.76447266,99.51924123)
\curveto(721.8144728,99.43923546)(721.85447276,99.34423556)(721.88447266,99.23424123)
\curveto(721.9144727,99.12423578)(721.94447267,99.01923588)(721.97447266,98.91924123)
\curveto(722.0144726,98.80923609)(722.03947258,98.6992362)(722.04947266,98.58924123)
\curveto(722.05947256,98.47923642)(722.07447254,98.36423654)(722.09447266,98.24424123)
\curveto(722.10447251,98.2042367)(722.10447251,98.15923674)(722.09447266,98.10924123)
\curveto(722.09447252,98.06923683)(722.09947252,98.02923687)(722.10947266,97.98924123)
\curveto(722.1194725,97.94923695)(722.12447249,97.89423701)(722.12447266,97.82424123)
\curveto(722.12447249,97.75423715)(722.1194725,97.7042372)(722.10947266,97.67424123)
\curveto(722.08947253,97.62423728)(722.08447253,97.57923732)(722.09447266,97.53924123)
\curveto(722.10447251,97.4992374)(722.10447251,97.46423744)(722.09447266,97.43424123)
\lineto(722.09447266,97.34424123)
\curveto(722.07447254,97.28423762)(722.05947256,97.21923768)(722.04947266,97.14924123)
\curveto(722.04947257,97.08923781)(722.04447257,97.02423788)(722.03447266,96.95424123)
\curveto(721.98447263,96.78423812)(721.93447268,96.62423828)(721.88447266,96.47424123)
\curveto(721.83447278,96.32423858)(721.76947285,96.17923872)(721.68947266,96.03924123)
\curveto(721.64947297,95.98923891)(721.619473,95.93423897)(721.59947266,95.87424123)
\curveto(721.56947305,95.82423908)(721.53447308,95.77423913)(721.49447266,95.72424123)
\curveto(721.3144733,95.48423942)(721.09447352,95.28423962)(720.83447266,95.12424123)
\curveto(720.57447404,94.96423994)(720.28947433,94.82424008)(719.97947266,94.70424123)
\curveto(719.83947478,94.64424026)(719.69947492,94.5992403)(719.55947266,94.56924123)
\curveto(719.40947521,94.53924036)(719.25447536,94.5042404)(719.09447266,94.46424123)
\curveto(718.98447563,94.44424046)(718.87447574,94.42924047)(718.76447266,94.41924123)
\curveto(718.65447596,94.40924049)(718.54447607,94.39424051)(718.43447266,94.37424123)
\curveto(718.39447622,94.36424054)(718.35447626,94.35924054)(718.31447266,94.35924123)
\curveto(718.27447634,94.36924053)(718.23447638,94.36924053)(718.19447266,94.35924123)
\curveto(718.14447647,94.34924055)(718.09447652,94.34424056)(718.04447266,94.34424123)
\lineto(717.87947266,94.34424123)
\curveto(717.82947679,94.32424058)(717.77947684,94.31924058)(717.72947266,94.32924123)
\curveto(717.66947695,94.33924056)(717.614477,94.33924056)(717.56447266,94.32924123)
\curveto(717.52447709,94.31924058)(717.47947714,94.31924058)(717.42947266,94.32924123)
\curveto(717.37947724,94.33924056)(717.32947729,94.33424057)(717.27947266,94.31424123)
\curveto(717.20947741,94.29424061)(717.13447748,94.28924061)(717.05447266,94.29924123)
\curveto(716.96447765,94.30924059)(716.87947774,94.31424059)(716.79947266,94.31424123)
\curveto(716.70947791,94.31424059)(716.60947801,94.30924059)(716.49947266,94.29924123)
\curveto(716.37947824,94.28924061)(716.27947834,94.29424061)(716.19947266,94.31424123)
\lineto(715.91447266,94.31424123)
\lineto(715.28447266,94.35924123)
\curveto(715.18447943,94.36924053)(715.08947953,94.37924052)(714.99947266,94.38924123)
\lineto(714.69947266,94.41924123)
\curveto(714.64947997,94.43924046)(714.59948002,94.44424046)(714.54947266,94.43424123)
\curveto(714.48948013,94.43424047)(714.43448018,94.44424046)(714.38447266,94.46424123)
\curveto(714.2144804,94.51424039)(714.04948057,94.55424035)(713.88947266,94.58424123)
\curveto(713.7194809,94.61424029)(713.55948106,94.66424024)(713.40947266,94.73424123)
\curveto(712.94948167,94.92423998)(712.57448204,95.14423976)(712.28447266,95.39424123)
\curveto(711.99448262,95.65423925)(711.74948287,96.01423889)(711.54947266,96.47424123)
\curveto(711.49948312,96.6042383)(711.46448315,96.73423817)(711.44447266,96.86424123)
\curveto(711.42448319,97.0042379)(711.39948322,97.14423776)(711.36947266,97.28424123)
\curveto(711.35948326,97.35423755)(711.35448326,97.41923748)(711.35447266,97.47924123)
\curveto(711.35448326,97.53923736)(711.34948327,97.6042373)(711.33947266,97.67424123)
\curveto(711.3194833,98.5042364)(711.46948315,99.17423573)(711.78947266,99.68424123)
\curveto(712.09948252,100.19423471)(712.53948208,100.57423433)(713.10947266,100.82424123)
\curveto(713.22948139,100.87423403)(713.35448126,100.91923398)(713.48447266,100.95924123)
\curveto(713.614481,100.9992339)(713.74948087,101.04423386)(713.88947266,101.09424123)
\curveto(713.96948065,101.11423379)(714.05448056,101.12923377)(714.14447266,101.13924123)
\lineto(714.38447266,101.19924123)
\curveto(714.49448012,101.22923367)(714.60448001,101.24423366)(714.71447266,101.24424123)
\curveto(714.82447979,101.25423365)(714.93447968,101.26923363)(715.04447266,101.28924123)
\curveto(715.09447952,101.30923359)(715.13947948,101.31423359)(715.17947266,101.30424123)
\curveto(715.2194794,101.3042336)(715.25947936,101.30923359)(715.29947266,101.31924123)
\curveto(715.34947927,101.32923357)(715.40447921,101.32923357)(715.46447266,101.31924123)
\curveto(715.5144791,101.31923358)(715.56447905,101.32423358)(715.61447266,101.33424123)
\lineto(715.74947266,101.33424123)
\curveto(715.80947881,101.35423355)(715.87947874,101.35423355)(715.95947266,101.33424123)
\curveto(716.02947859,101.32423358)(716.09447852,101.32923357)(716.15447266,101.34924123)
\curveto(716.18447843,101.35923354)(716.22447839,101.36423354)(716.27447266,101.36424123)
\lineto(716.39447266,101.36424123)
\lineto(716.85947266,101.36424123)
\moveto(719.18447266,99.81924123)
\curveto(718.86447575,99.91923498)(718.49947612,99.97923492)(718.08947266,99.99924123)
\curveto(717.67947694,100.01923488)(717.26947735,100.02923487)(716.85947266,100.02924123)
\curveto(716.42947819,100.02923487)(716.00947861,100.01923488)(715.59947266,99.99924123)
\curveto(715.18947943,99.97923492)(714.80447981,99.93423497)(714.44447266,99.86424123)
\curveto(714.08448053,99.79423511)(713.76448085,99.68423522)(713.48447266,99.53424123)
\curveto(713.19448142,99.39423551)(712.95948166,99.1992357)(712.77947266,98.94924123)
\curveto(712.66948195,98.78923611)(712.58948203,98.60923629)(712.53947266,98.40924123)
\curveto(712.47948214,98.20923669)(712.44948217,97.96423694)(712.44947266,97.67424123)
\curveto(712.46948215,97.65423725)(712.47948214,97.61923728)(712.47947266,97.56924123)
\curveto(712.46948215,97.51923738)(712.46948215,97.47923742)(712.47947266,97.44924123)
\curveto(712.49948212,97.36923753)(712.5194821,97.29423761)(712.53947266,97.22424123)
\curveto(712.54948207,97.16423774)(712.56948205,97.0992378)(712.59947266,97.02924123)
\curveto(712.7194819,96.75923814)(712.88948173,96.53923836)(713.10947266,96.36924123)
\curveto(713.3194813,96.20923869)(713.56448105,96.07423883)(713.84447266,95.96424123)
\curveto(713.95448066,95.91423899)(714.07448054,95.87423903)(714.20447266,95.84424123)
\curveto(714.32448029,95.82423908)(714.44948017,95.7992391)(714.57947266,95.76924123)
\curveto(714.62947999,95.74923915)(714.68447993,95.73923916)(714.74447266,95.73924123)
\curveto(714.79447982,95.73923916)(714.84447977,95.73423917)(714.89447266,95.72424123)
\curveto(714.98447963,95.71423919)(715.07947954,95.7042392)(715.17947266,95.69424123)
\curveto(715.26947935,95.68423922)(715.36447925,95.67423923)(715.46447266,95.66424123)
\curveto(715.54447907,95.66423924)(715.62947899,95.65923924)(715.71947266,95.64924123)
\lineto(715.95947266,95.64924123)
\lineto(716.13947266,95.64924123)
\curveto(716.16947845,95.63923926)(716.20447841,95.63423927)(716.24447266,95.63424123)
\lineto(716.37947266,95.63424123)
\lineto(716.82947266,95.63424123)
\curveto(716.90947771,95.63423927)(716.99447762,95.62923927)(717.08447266,95.61924123)
\curveto(717.16447745,95.61923928)(717.23947738,95.62923927)(717.30947266,95.64924123)
\lineto(717.57947266,95.64924123)
\curveto(717.59947702,95.64923925)(717.62947699,95.64423926)(717.66947266,95.63424123)
\curveto(717.69947692,95.63423927)(717.72447689,95.63923926)(717.74447266,95.64924123)
\curveto(717.84447677,95.65923924)(717.94447667,95.66423924)(718.04447266,95.66424123)
\curveto(718.13447648,95.67423923)(718.23447638,95.68423922)(718.34447266,95.69424123)
\curveto(718.46447615,95.72423918)(718.58947603,95.73923916)(718.71947266,95.73924123)
\curveto(718.83947578,95.74923915)(718.95447566,95.77423913)(719.06447266,95.81424123)
\curveto(719.36447525,95.89423901)(719.62947499,95.97923892)(719.85947266,96.06924123)
\curveto(720.08947453,96.16923873)(720.30447431,96.31423859)(720.50447266,96.50424123)
\curveto(720.70447391,96.71423819)(720.85447376,96.97923792)(720.95447266,97.29924123)
\curveto(720.97447364,97.33923756)(720.98447363,97.37423753)(720.98447266,97.40424123)
\curveto(720.97447364,97.44423746)(720.97947364,97.48923741)(720.99947266,97.53924123)
\curveto(721.00947361,97.57923732)(721.0194736,97.64923725)(721.02947266,97.74924123)
\curveto(721.03947358,97.85923704)(721.03447358,97.94423696)(721.01447266,98.00424123)
\curveto(720.99447362,98.07423683)(720.98447363,98.14423676)(720.98447266,98.21424123)
\curveto(720.97447364,98.28423662)(720.95947366,98.34923655)(720.93947266,98.40924123)
\curveto(720.87947374,98.60923629)(720.79447382,98.78923611)(720.68447266,98.94924123)
\curveto(720.66447395,98.97923592)(720.64447397,99.0042359)(720.62447266,99.02424123)
\lineto(720.56447266,99.08424123)
\curveto(720.54447407,99.12423578)(720.50447411,99.17423573)(720.44447266,99.23424123)
\curveto(720.30447431,99.33423557)(720.17447444,99.41923548)(720.05447266,99.48924123)
\curveto(719.93447468,99.55923534)(719.78947483,99.62923527)(719.61947266,99.69924123)
\curveto(719.54947507,99.72923517)(719.47947514,99.74923515)(719.40947266,99.75924123)
\curveto(719.33947528,99.77923512)(719.26447535,99.7992351)(719.18447266,99.81924123)
}
}
{
\newrgbcolor{curcolor}{0 0 0}
\pscustom[linestyle=none,fillstyle=solid,fillcolor=curcolor]
{
\newpath
\moveto(711.33947266,106.77385061)
\curveto(711.33948328,106.87384575)(711.34948327,106.96884566)(711.36947266,107.05885061)
\curveto(711.37948324,107.14884548)(711.40948321,107.21384541)(711.45947266,107.25385061)
\curveto(711.53948308,107.31384531)(711.64448297,107.34384528)(711.77447266,107.34385061)
\lineto(712.16447266,107.34385061)
\lineto(713.66447266,107.34385061)
\lineto(720.05447266,107.34385061)
\lineto(721.22447266,107.34385061)
\lineto(721.53947266,107.34385061)
\curveto(721.63947298,107.35384527)(721.7194729,107.33884529)(721.77947266,107.29885061)
\curveto(721.85947276,107.24884538)(721.90947271,107.17384545)(721.92947266,107.07385061)
\curveto(721.93947268,106.98384564)(721.94447267,106.87384575)(721.94447266,106.74385061)
\lineto(721.94447266,106.51885061)
\curveto(721.92447269,106.43884619)(721.90947271,106.36884626)(721.89947266,106.30885061)
\curveto(721.87947274,106.24884638)(721.83947278,106.19884643)(721.77947266,106.15885061)
\curveto(721.7194729,106.11884651)(721.64447297,106.09884653)(721.55447266,106.09885061)
\lineto(721.25447266,106.09885061)
\lineto(720.15947266,106.09885061)
\lineto(714.81947266,106.09885061)
\curveto(714.72947989,106.07884655)(714.65447996,106.06384656)(714.59447266,106.05385061)
\curveto(714.52448009,106.05384657)(714.46448015,106.0238466)(714.41447266,105.96385061)
\curveto(714.36448025,105.89384673)(714.33948028,105.80384682)(714.33947266,105.69385061)
\curveto(714.32948029,105.59384703)(714.32448029,105.48384714)(714.32447266,105.36385061)
\lineto(714.32447266,104.22385061)
\lineto(714.32447266,103.72885061)
\curveto(714.3144803,103.56884906)(714.25448036,103.45884917)(714.14447266,103.39885061)
\curveto(714.1144805,103.37884925)(714.08448053,103.36884926)(714.05447266,103.36885061)
\curveto(714.0144806,103.36884926)(713.96948065,103.36384926)(713.91947266,103.35385061)
\curveto(713.79948082,103.33384929)(713.68948093,103.33884929)(713.58947266,103.36885061)
\curveto(713.48948113,103.40884922)(713.4194812,103.46384916)(713.37947266,103.53385061)
\curveto(713.32948129,103.61384901)(713.30448131,103.73384889)(713.30447266,103.89385061)
\curveto(713.30448131,104.05384857)(713.28948133,104.18884844)(713.25947266,104.29885061)
\curveto(713.24948137,104.34884828)(713.24448137,104.40384822)(713.24447266,104.46385061)
\curveto(713.23448138,104.5238481)(713.2194814,104.58384804)(713.19947266,104.64385061)
\curveto(713.14948147,104.79384783)(713.09948152,104.93884769)(713.04947266,105.07885061)
\curveto(712.98948163,105.21884741)(712.9194817,105.35384727)(712.83947266,105.48385061)
\curveto(712.74948187,105.623847)(712.64448197,105.74384688)(712.52447266,105.84385061)
\curveto(712.40448221,105.94384668)(712.27448234,106.03884659)(712.13447266,106.12885061)
\curveto(712.03448258,106.18884644)(711.92448269,106.23384639)(711.80447266,106.26385061)
\curveto(711.68448293,106.30384632)(711.57948304,106.35384627)(711.48947266,106.41385061)
\curveto(711.42948319,106.46384616)(711.38948323,106.53384609)(711.36947266,106.62385061)
\curveto(711.35948326,106.64384598)(711.35448326,106.66884596)(711.35447266,106.69885061)
\curveto(711.35448326,106.7288459)(711.34948327,106.75384587)(711.33947266,106.77385061)
}
}
{
\newrgbcolor{curcolor}{0 0 0}
\pscustom[linestyle=none,fillstyle=solid,fillcolor=curcolor]
{
\newpath
\moveto(711.33947266,115.12345998)
\curveto(711.33948328,115.22345513)(711.34948327,115.31845503)(711.36947266,115.40845998)
\curveto(711.37948324,115.49845485)(711.40948321,115.56345479)(711.45947266,115.60345998)
\curveto(711.53948308,115.66345469)(711.64448297,115.69345466)(711.77447266,115.69345998)
\lineto(712.16447266,115.69345998)
\lineto(713.66447266,115.69345998)
\lineto(720.05447266,115.69345998)
\lineto(721.22447266,115.69345998)
\lineto(721.53947266,115.69345998)
\curveto(721.63947298,115.70345465)(721.7194729,115.68845466)(721.77947266,115.64845998)
\curveto(721.85947276,115.59845475)(721.90947271,115.52345483)(721.92947266,115.42345998)
\curveto(721.93947268,115.33345502)(721.94447267,115.22345513)(721.94447266,115.09345998)
\lineto(721.94447266,114.86845998)
\curveto(721.92447269,114.78845556)(721.90947271,114.71845563)(721.89947266,114.65845998)
\curveto(721.87947274,114.59845575)(721.83947278,114.5484558)(721.77947266,114.50845998)
\curveto(721.7194729,114.46845588)(721.64447297,114.4484559)(721.55447266,114.44845998)
\lineto(721.25447266,114.44845998)
\lineto(720.15947266,114.44845998)
\lineto(714.81947266,114.44845998)
\curveto(714.72947989,114.42845592)(714.65447996,114.41345594)(714.59447266,114.40345998)
\curveto(714.52448009,114.40345595)(714.46448015,114.37345598)(714.41447266,114.31345998)
\curveto(714.36448025,114.24345611)(714.33948028,114.1534562)(714.33947266,114.04345998)
\curveto(714.32948029,113.94345641)(714.32448029,113.83345652)(714.32447266,113.71345998)
\lineto(714.32447266,112.57345998)
\lineto(714.32447266,112.07845998)
\curveto(714.3144803,111.91845843)(714.25448036,111.80845854)(714.14447266,111.74845998)
\curveto(714.1144805,111.72845862)(714.08448053,111.71845863)(714.05447266,111.71845998)
\curveto(714.0144806,111.71845863)(713.96948065,111.71345864)(713.91947266,111.70345998)
\curveto(713.79948082,111.68345867)(713.68948093,111.68845866)(713.58947266,111.71845998)
\curveto(713.48948113,111.75845859)(713.4194812,111.81345854)(713.37947266,111.88345998)
\curveto(713.32948129,111.96345839)(713.30448131,112.08345827)(713.30447266,112.24345998)
\curveto(713.30448131,112.40345795)(713.28948133,112.53845781)(713.25947266,112.64845998)
\curveto(713.24948137,112.69845765)(713.24448137,112.7534576)(713.24447266,112.81345998)
\curveto(713.23448138,112.87345748)(713.2194814,112.93345742)(713.19947266,112.99345998)
\curveto(713.14948147,113.14345721)(713.09948152,113.28845706)(713.04947266,113.42845998)
\curveto(712.98948163,113.56845678)(712.9194817,113.70345665)(712.83947266,113.83345998)
\curveto(712.74948187,113.97345638)(712.64448197,114.09345626)(712.52447266,114.19345998)
\curveto(712.40448221,114.29345606)(712.27448234,114.38845596)(712.13447266,114.47845998)
\curveto(712.03448258,114.53845581)(711.92448269,114.58345577)(711.80447266,114.61345998)
\curveto(711.68448293,114.6534557)(711.57948304,114.70345565)(711.48947266,114.76345998)
\curveto(711.42948319,114.81345554)(711.38948323,114.88345547)(711.36947266,114.97345998)
\curveto(711.35948326,114.99345536)(711.35448326,115.01845533)(711.35447266,115.04845998)
\curveto(711.35448326,115.07845527)(711.34948327,115.10345525)(711.33947266,115.12345998)
}
}
{
\newrgbcolor{curcolor}{0 0 0}
\pscustom[linestyle=none,fillstyle=solid,fillcolor=curcolor]
{
\newpath
\moveto(732.17578857,37.28705373)
\curveto(732.17579927,37.35704805)(732.17579927,37.43704797)(732.17578857,37.52705373)
\curveto(732.16579928,37.61704779)(732.16579928,37.70204771)(732.17578857,37.78205373)
\curveto(732.17579927,37.87204754)(732.18579926,37.95204746)(732.20578857,38.02205373)
\curveto(732.22579922,38.10204731)(732.25579919,38.15704725)(732.29578857,38.18705373)
\curveto(732.3457991,38.21704719)(732.42079902,38.23704717)(732.52078857,38.24705373)
\curveto(732.61079883,38.26704714)(732.71579873,38.27704713)(732.83578857,38.27705373)
\curveto(732.9457985,38.28704712)(733.06079838,38.28704712)(733.18078857,38.27705373)
\lineto(733.48078857,38.27705373)
\lineto(736.49578857,38.27705373)
\lineto(739.39078857,38.27705373)
\curveto(739.72079172,38.27704713)(740.0457914,38.27204714)(740.36578857,38.26205373)
\curveto(740.67579077,38.26204715)(740.95579049,38.22204719)(741.20578857,38.14205373)
\curveto(741.55578989,38.02204739)(741.85078959,37.86704754)(742.09078857,37.67705373)
\curveto(742.32078912,37.48704792)(742.52078892,37.24704816)(742.69078857,36.95705373)
\curveto(742.7407887,36.89704851)(742.77578867,36.83204858)(742.79578857,36.76205373)
\curveto(742.81578863,36.70204871)(742.8407886,36.63204878)(742.87078857,36.55205373)
\curveto(742.92078852,36.43204898)(742.95578849,36.30204911)(742.97578857,36.16205373)
\curveto(743.00578844,36.03204938)(743.03578841,35.89704951)(743.06578857,35.75705373)
\curveto(743.08578836,35.7070497)(743.09078835,35.65704975)(743.08078857,35.60705373)
\curveto(743.07078837,35.55704985)(743.07078837,35.50204991)(743.08078857,35.44205373)
\curveto(743.09078835,35.42204999)(743.09078835,35.39705001)(743.08078857,35.36705373)
\curveto(743.08078836,35.33705007)(743.08578836,35.3120501)(743.09578857,35.29205373)
\curveto(743.10578834,35.25205016)(743.11078833,35.19705021)(743.11078857,35.12705373)
\curveto(743.11078833,35.05705035)(743.10578834,35.00205041)(743.09578857,34.96205373)
\curveto(743.08578836,34.9120505)(743.08578836,34.85705055)(743.09578857,34.79705373)
\curveto(743.10578834,34.73705067)(743.10078834,34.68205073)(743.08078857,34.63205373)
\curveto(743.05078839,34.50205091)(743.03078841,34.37705103)(743.02078857,34.25705373)
\curveto(743.01078843,34.13705127)(742.98578846,34.02205139)(742.94578857,33.91205373)
\curveto(742.82578862,33.54205187)(742.65578879,33.22205219)(742.43578857,32.95205373)
\curveto(742.21578923,32.68205273)(741.93578951,32.47205294)(741.59578857,32.32205373)
\curveto(741.47578997,32.27205314)(741.35079009,32.22705318)(741.22078857,32.18705373)
\curveto(741.09079035,32.15705325)(740.95579049,32.12205329)(740.81578857,32.08205373)
\curveto(740.76579068,32.07205334)(740.72579072,32.06705334)(740.69578857,32.06705373)
\curveto(740.65579079,32.06705334)(740.61079083,32.06205335)(740.56078857,32.05205373)
\curveto(740.53079091,32.04205337)(740.49579095,32.03705337)(740.45578857,32.03705373)
\curveto(740.40579104,32.03705337)(740.36579108,32.03205338)(740.33578857,32.02205373)
\lineto(740.17078857,32.02205373)
\curveto(740.09079135,32.00205341)(739.99079145,31.99705341)(739.87078857,32.00705373)
\curveto(739.7407917,32.01705339)(739.65079179,32.03205338)(739.60078857,32.05205373)
\curveto(739.51079193,32.07205334)(739.445792,32.12705328)(739.40578857,32.21705373)
\curveto(739.38579206,32.24705316)(739.38079206,32.27705313)(739.39078857,32.30705373)
\curveto(739.39079205,32.33705307)(739.38579206,32.37705303)(739.37578857,32.42705373)
\curveto(739.36579208,32.46705294)(739.36079208,32.5070529)(739.36078857,32.54705373)
\lineto(739.36078857,32.69705373)
\curveto(739.36079208,32.81705259)(739.36579208,32.93705247)(739.37578857,33.05705373)
\curveto(739.37579207,33.18705222)(739.41079203,33.27705213)(739.48078857,33.32705373)
\curveto(739.5407919,33.36705204)(739.60079184,33.38705202)(739.66078857,33.38705373)
\curveto(739.72079172,33.38705202)(739.79079165,33.39705201)(739.87078857,33.41705373)
\curveto(739.90079154,33.42705198)(739.93579151,33.42705198)(739.97578857,33.41705373)
\curveto(740.00579144,33.41705199)(740.03079141,33.42205199)(740.05078857,33.43205373)
\lineto(740.26078857,33.43205373)
\curveto(740.31079113,33.45205196)(740.36079108,33.45705195)(740.41078857,33.44705373)
\curveto(740.45079099,33.44705196)(740.49579095,33.45705195)(740.54578857,33.47705373)
\curveto(740.67579077,33.5070519)(740.80079064,33.53705187)(740.92078857,33.56705373)
\curveto(741.03079041,33.59705181)(741.13579031,33.64205177)(741.23578857,33.70205373)
\curveto(741.52578992,33.87205154)(741.73078971,34.14205127)(741.85078857,34.51205373)
\curveto(741.87078957,34.56205085)(741.88578956,34.6120508)(741.89578857,34.66205373)
\curveto(741.89578955,34.72205069)(741.90578954,34.77705063)(741.92578857,34.82705373)
\lineto(741.92578857,34.90205373)
\curveto(741.93578951,34.97205044)(741.9457895,35.06705034)(741.95578857,35.18705373)
\curveto(741.95578949,35.31705009)(741.9457895,35.41704999)(741.92578857,35.48705373)
\curveto(741.90578954,35.55704985)(741.89078955,35.62704978)(741.88078857,35.69705373)
\curveto(741.86078958,35.77704963)(741.8407896,35.84704956)(741.82078857,35.90705373)
\curveto(741.66078978,36.28704912)(741.38579006,36.56204885)(740.99578857,36.73205373)
\curveto(740.86579058,36.78204863)(740.71079073,36.81704859)(740.53078857,36.83705373)
\curveto(740.35079109,36.86704854)(740.16579128,36.88204853)(739.97578857,36.88205373)
\curveto(739.77579167,36.89204852)(739.57579187,36.89204852)(739.37578857,36.88205373)
\lineto(738.80578857,36.88205373)
\lineto(734.56078857,36.88205373)
\lineto(733.01578857,36.88205373)
\curveto(732.90579854,36.88204853)(732.78579866,36.87704853)(732.65578857,36.86705373)
\curveto(732.52579892,36.85704855)(732.42079902,36.87704853)(732.34078857,36.92705373)
\curveto(732.27079917,36.98704842)(732.22079922,37.06704834)(732.19078857,37.16705373)
\curveto(732.19079925,37.18704822)(732.19079925,37.2070482)(732.19078857,37.22705373)
\curveto(732.19079925,37.24704816)(732.18579926,37.26704814)(732.17578857,37.28705373)
}
}
{
\newrgbcolor{curcolor}{0 0 0}
\pscustom[linestyle=none,fillstyle=solid,fillcolor=curcolor]
{
\newpath
\moveto(735.13078857,40.82072561)
\lineto(735.13078857,41.25572561)
\curveto(735.13079631,41.40572364)(735.17079627,41.51072354)(735.25078857,41.57072561)
\curveto(735.33079611,41.62072343)(735.43079601,41.6457234)(735.55078857,41.64572561)
\curveto(735.67079577,41.65572339)(735.79079565,41.66072339)(735.91078857,41.66072561)
\lineto(737.33578857,41.66072561)
\lineto(739.60078857,41.66072561)
\lineto(740.29078857,41.66072561)
\curveto(740.52079092,41.66072339)(740.72079072,41.68572336)(740.89078857,41.73572561)
\curveto(741.3407901,41.89572315)(741.65578979,42.19572285)(741.83578857,42.63572561)
\curveto(741.92578952,42.85572219)(741.96078948,43.12072193)(741.94078857,43.43072561)
\curveto(741.91078953,43.74072131)(741.85578959,43.99072106)(741.77578857,44.18072561)
\curveto(741.63578981,44.51072054)(741.46078998,44.77072028)(741.25078857,44.96072561)
\curveto(741.03079041,45.16071989)(740.7457907,45.31571973)(740.39578857,45.42572561)
\curveto(740.31579113,45.45571959)(740.23579121,45.47571957)(740.15578857,45.48572561)
\curveto(740.07579137,45.49571955)(739.99079145,45.51071954)(739.90078857,45.53072561)
\curveto(739.85079159,45.54071951)(739.80579164,45.54071951)(739.76578857,45.53072561)
\curveto(739.72579172,45.53071952)(739.68079176,45.54071951)(739.63078857,45.56072561)
\lineto(739.31578857,45.56072561)
\curveto(739.23579221,45.58071947)(739.1457923,45.58571946)(739.04578857,45.57572561)
\curveto(738.93579251,45.56571948)(738.83579261,45.56071949)(738.74578857,45.56072561)
\lineto(737.57578857,45.56072561)
\lineto(735.98578857,45.56072561)
\curveto(735.86579558,45.56071949)(735.7407957,45.55571949)(735.61078857,45.54572561)
\curveto(735.47079597,45.5457195)(735.36079608,45.57071948)(735.28078857,45.62072561)
\curveto(735.23079621,45.66071939)(735.20079624,45.70571934)(735.19078857,45.75572561)
\curveto(735.17079627,45.81571923)(735.15079629,45.88571916)(735.13078857,45.96572561)
\lineto(735.13078857,46.19072561)
\curveto(735.13079631,46.31071874)(735.13579631,46.41571863)(735.14578857,46.50572561)
\curveto(735.15579629,46.60571844)(735.20079624,46.68071837)(735.28078857,46.73072561)
\curveto(735.33079611,46.78071827)(735.40579604,46.80571824)(735.50578857,46.80572561)
\lineto(735.79078857,46.80572561)
\lineto(736.81078857,46.80572561)
\lineto(740.84578857,46.80572561)
\lineto(742.19578857,46.80572561)
\curveto(742.31578913,46.80571824)(742.43078901,46.80071825)(742.54078857,46.79072561)
\curveto(742.6407888,46.79071826)(742.71578873,46.75571829)(742.76578857,46.68572561)
\curveto(742.79578865,46.6457184)(742.82078862,46.58571846)(742.84078857,46.50572561)
\curveto(742.85078859,46.42571862)(742.86078858,46.33571871)(742.87078857,46.23572561)
\curveto(742.87078857,46.1457189)(742.86578858,46.05571899)(742.85578857,45.96572561)
\curveto(742.8457886,45.88571916)(742.82578862,45.82571922)(742.79578857,45.78572561)
\curveto(742.75578869,45.73571931)(742.69078875,45.69071936)(742.60078857,45.65072561)
\curveto(742.56078888,45.64071941)(742.50578894,45.63071942)(742.43578857,45.62072561)
\curveto(742.36578908,45.62071943)(742.30078914,45.61571943)(742.24078857,45.60572561)
\curveto(742.17078927,45.59571945)(742.11578933,45.57571947)(742.07578857,45.54572561)
\curveto(742.03578941,45.51571953)(742.02078942,45.47071958)(742.03078857,45.41072561)
\curveto(742.05078939,45.33071972)(742.11078933,45.2507198)(742.21078857,45.17072561)
\curveto(742.30078914,45.09071996)(742.37078907,45.01572003)(742.42078857,44.94572561)
\curveto(742.58078886,44.72572032)(742.72078872,44.47572057)(742.84078857,44.19572561)
\curveto(742.89078855,44.08572096)(742.92078852,43.97072108)(742.93078857,43.85072561)
\curveto(742.95078849,43.74072131)(742.97578847,43.62572142)(743.00578857,43.50572561)
\curveto(743.01578843,43.45572159)(743.01578843,43.40072165)(743.00578857,43.34072561)
\curveto(742.99578845,43.29072176)(743.00078844,43.24072181)(743.02078857,43.19072561)
\curveto(743.0407884,43.09072196)(743.0407884,43.00072205)(743.02078857,42.92072561)
\lineto(743.02078857,42.77072561)
\curveto(743.00078844,42.72072233)(742.99078845,42.66072239)(742.99078857,42.59072561)
\curveto(742.99078845,42.53072252)(742.98578846,42.47572257)(742.97578857,42.42572561)
\curveto(742.95578849,42.38572266)(742.9457885,42.3457227)(742.94578857,42.30572561)
\curveto(742.95578849,42.27572277)(742.95078849,42.23572281)(742.93078857,42.18572561)
\lineto(742.87078857,41.94572561)
\curveto(742.85078859,41.87572317)(742.82078862,41.80072325)(742.78078857,41.72072561)
\curveto(742.67078877,41.46072359)(742.52578892,41.24072381)(742.34578857,41.06072561)
\curveto(742.15578929,40.89072416)(741.93078951,40.7507243)(741.67078857,40.64072561)
\curveto(741.58078986,40.60072445)(741.49078995,40.57072448)(741.40078857,40.55072561)
\lineto(741.10078857,40.49072561)
\curveto(741.0407904,40.47072458)(740.98579046,40.46072459)(740.93578857,40.46072561)
\curveto(740.87579057,40.47072458)(740.81079063,40.46572458)(740.74078857,40.44572561)
\curveto(740.72079072,40.43572461)(740.69579075,40.43072462)(740.66578857,40.43072561)
\curveto(740.62579082,40.43072462)(740.59079085,40.42572462)(740.56078857,40.41572561)
\lineto(740.41078857,40.41572561)
\curveto(740.37079107,40.40572464)(740.32579112,40.40072465)(740.27578857,40.40072561)
\curveto(740.21579123,40.41072464)(740.16079128,40.41572463)(740.11078857,40.41572561)
\lineto(739.51078857,40.41572561)
\lineto(736.75078857,40.41572561)
\lineto(735.79078857,40.41572561)
\lineto(735.52078857,40.41572561)
\curveto(735.43079601,40.41572463)(735.35579609,40.43572461)(735.29578857,40.47572561)
\curveto(735.22579622,40.51572453)(735.17579627,40.59072446)(735.14578857,40.70072561)
\curveto(735.13579631,40.72072433)(735.13579631,40.74072431)(735.14578857,40.76072561)
\curveto(735.1457963,40.78072427)(735.1407963,40.80072425)(735.13078857,40.82072561)
}
}
{
\newrgbcolor{curcolor}{0 0 0}
\pscustom[linestyle=none,fillstyle=solid,fillcolor=curcolor]
{
\newpath
\moveto(734.98078857,52.39533498)
\curveto(734.96079648,53.02532975)(735.0457964,53.53032924)(735.23578857,53.91033498)
\curveto(735.42579602,54.29032848)(735.71079573,54.59532818)(736.09078857,54.82533498)
\curveto(736.19079525,54.88532789)(736.30079514,54.93032784)(736.42078857,54.96033498)
\curveto(736.53079491,55.00032777)(736.6457948,55.03532774)(736.76578857,55.06533498)
\curveto(736.95579449,55.11532766)(737.16079428,55.14532763)(737.38078857,55.15533498)
\curveto(737.60079384,55.16532761)(737.82579362,55.1703276)(738.05578857,55.17033498)
\lineto(739.66078857,55.17033498)
\lineto(742.00078857,55.17033498)
\curveto(742.17078927,55.1703276)(742.3407891,55.16532761)(742.51078857,55.15533498)
\curveto(742.68078876,55.15532762)(742.79078865,55.09032768)(742.84078857,54.96033498)
\curveto(742.86078858,54.91032786)(742.87078857,54.85532792)(742.87078857,54.79533498)
\curveto(742.88078856,54.74532803)(742.88578856,54.69032808)(742.88578857,54.63033498)
\curveto(742.88578856,54.50032827)(742.88078856,54.3753284)(742.87078857,54.25533498)
\curveto(742.87078857,54.13532864)(742.83078861,54.05032872)(742.75078857,54.00033498)
\curveto(742.68078876,53.95032882)(742.59078885,53.92532885)(742.48078857,53.92533498)
\lineto(742.15078857,53.92533498)
\lineto(740.86078857,53.92533498)
\lineto(738.41578857,53.92533498)
\curveto(738.1457933,53.92532885)(737.88079356,53.92032885)(737.62078857,53.91033498)
\curveto(737.35079409,53.90032887)(737.12079432,53.85532892)(736.93078857,53.77533498)
\curveto(736.73079471,53.69532908)(736.57079487,53.5753292)(736.45078857,53.41533498)
\curveto(736.32079512,53.25532952)(736.22079522,53.0703297)(736.15078857,52.86033498)
\curveto(736.13079531,52.80032997)(736.12079532,52.73533004)(736.12078857,52.66533498)
\curveto(736.11079533,52.60533017)(736.09579535,52.54533023)(736.07578857,52.48533498)
\curveto(736.06579538,52.43533034)(736.06579538,52.35533042)(736.07578857,52.24533498)
\curveto(736.07579537,52.14533063)(736.08079536,52.0753307)(736.09078857,52.03533498)
\curveto(736.11079533,51.99533078)(736.12079532,51.96033081)(736.12078857,51.93033498)
\curveto(736.11079533,51.90033087)(736.11079533,51.86533091)(736.12078857,51.82533498)
\curveto(736.15079529,51.69533108)(736.18579526,51.5703312)(736.22578857,51.45033498)
\curveto(736.25579519,51.34033143)(736.30079514,51.23533154)(736.36078857,51.13533498)
\curveto(736.38079506,51.09533168)(736.40079504,51.06033171)(736.42078857,51.03033498)
\curveto(736.440795,51.00033177)(736.46079498,50.96533181)(736.48078857,50.92533498)
\curveto(736.73079471,50.5753322)(737.10579434,50.32033245)(737.60578857,50.16033498)
\curveto(737.68579376,50.13033264)(737.77079367,50.11033266)(737.86078857,50.10033498)
\curveto(737.9407935,50.09033268)(738.02079342,50.0753327)(738.10078857,50.05533498)
\curveto(738.15079329,50.03533274)(738.20079324,50.03033274)(738.25078857,50.04033498)
\curveto(738.29079315,50.05033272)(738.33079311,50.04533273)(738.37078857,50.02533498)
\lineto(738.68578857,50.02533498)
\curveto(738.71579273,50.01533276)(738.75079269,50.01033276)(738.79078857,50.01033498)
\curveto(738.83079261,50.02033275)(738.87579257,50.02533275)(738.92578857,50.02533498)
\lineto(739.37578857,50.02533498)
\lineto(740.81578857,50.02533498)
\lineto(742.13578857,50.02533498)
\lineto(742.48078857,50.02533498)
\curveto(742.59078885,50.02533275)(742.68078876,50.00033277)(742.75078857,49.95033498)
\curveto(742.83078861,49.90033287)(742.87078857,49.81033296)(742.87078857,49.68033498)
\curveto(742.88078856,49.56033321)(742.88578856,49.43533334)(742.88578857,49.30533498)
\curveto(742.88578856,49.22533355)(742.88078856,49.15033362)(742.87078857,49.08033498)
\curveto(742.86078858,49.01033376)(742.83578861,48.95033382)(742.79578857,48.90033498)
\curveto(742.7457887,48.82033395)(742.65078879,48.78033399)(742.51078857,48.78033498)
\lineto(742.10578857,48.78033498)
\lineto(740.33578857,48.78033498)
\lineto(736.70578857,48.78033498)
\lineto(735.79078857,48.78033498)
\lineto(735.52078857,48.78033498)
\curveto(735.43079601,48.78033399)(735.36079608,48.80033397)(735.31078857,48.84033498)
\curveto(735.25079619,48.8703339)(735.21079623,48.92033385)(735.19078857,48.99033498)
\curveto(735.18079626,49.03033374)(735.17079627,49.08533369)(735.16078857,49.15533498)
\curveto(735.15079629,49.23533354)(735.1457963,49.31533346)(735.14578857,49.39533498)
\curveto(735.1457963,49.4753333)(735.15079629,49.55033322)(735.16078857,49.62033498)
\curveto(735.17079627,49.70033307)(735.18579626,49.75533302)(735.20578857,49.78533498)
\curveto(735.27579617,49.89533288)(735.36579608,49.94533283)(735.47578857,49.93533498)
\curveto(735.57579587,49.92533285)(735.69079575,49.94033283)(735.82078857,49.98033498)
\curveto(735.88079556,50.00033277)(735.93079551,50.04033273)(735.97078857,50.10033498)
\curveto(735.98079546,50.22033255)(735.93579551,50.31533246)(735.83578857,50.38533498)
\curveto(735.73579571,50.46533231)(735.65579579,50.54533223)(735.59578857,50.62533498)
\curveto(735.49579595,50.76533201)(735.40579604,50.90533187)(735.32578857,51.04533498)
\curveto(735.23579621,51.19533158)(735.16079628,51.36533141)(735.10078857,51.55533498)
\curveto(735.07079637,51.63533114)(735.05079639,51.72033105)(735.04078857,51.81033498)
\curveto(735.03079641,51.91033086)(735.01579643,52.00533077)(734.99578857,52.09533498)
\curveto(734.98579646,52.14533063)(734.98079646,52.19533058)(734.98078857,52.24533498)
\lineto(734.98078857,52.39533498)
}
}
{
\newrgbcolor{curcolor}{0 0 0}
\pscustom[linestyle=none,fillstyle=solid,fillcolor=curcolor]
{
}
}
{
\newrgbcolor{curcolor}{0 0 0}
\pscustom[linestyle=none,fillstyle=solid,fillcolor=curcolor]
{
\newpath
\moveto(732.25078857,65.05510061)
\curveto(732.25079919,65.15509575)(732.26079918,65.25009566)(732.28078857,65.34010061)
\curveto(732.29079915,65.43009548)(732.32079912,65.49509541)(732.37078857,65.53510061)
\curveto(732.45079899,65.59509531)(732.55579889,65.62509528)(732.68578857,65.62510061)
\lineto(733.07578857,65.62510061)
\lineto(734.57578857,65.62510061)
\lineto(740.96578857,65.62510061)
\lineto(742.13578857,65.62510061)
\lineto(742.45078857,65.62510061)
\curveto(742.55078889,65.63509527)(742.63078881,65.62009529)(742.69078857,65.58010061)
\curveto(742.77078867,65.53009538)(742.82078862,65.45509545)(742.84078857,65.35510061)
\curveto(742.85078859,65.26509564)(742.85578859,65.15509575)(742.85578857,65.02510061)
\lineto(742.85578857,64.80010061)
\curveto(742.83578861,64.72009619)(742.82078862,64.65009626)(742.81078857,64.59010061)
\curveto(742.79078865,64.53009638)(742.75078869,64.48009643)(742.69078857,64.44010061)
\curveto(742.63078881,64.40009651)(742.55578889,64.38009653)(742.46578857,64.38010061)
\lineto(742.16578857,64.38010061)
\lineto(741.07078857,64.38010061)
\lineto(735.73078857,64.38010061)
\curveto(735.6407958,64.36009655)(735.56579588,64.34509656)(735.50578857,64.33510061)
\curveto(735.43579601,64.33509657)(735.37579607,64.3050966)(735.32578857,64.24510061)
\curveto(735.27579617,64.17509673)(735.25079619,64.08509682)(735.25078857,63.97510061)
\curveto(735.2407962,63.87509703)(735.23579621,63.76509714)(735.23578857,63.64510061)
\lineto(735.23578857,62.50510061)
\lineto(735.23578857,62.01010061)
\curveto(735.22579622,61.85009906)(735.16579628,61.74009917)(735.05578857,61.68010061)
\curveto(735.02579642,61.66009925)(734.99579645,61.65009926)(734.96578857,61.65010061)
\curveto(734.92579652,61.65009926)(734.88079656,61.64509926)(734.83078857,61.63510061)
\curveto(734.71079673,61.61509929)(734.60079684,61.62009929)(734.50078857,61.65010061)
\curveto(734.40079704,61.69009922)(734.33079711,61.74509916)(734.29078857,61.81510061)
\curveto(734.2407972,61.89509901)(734.21579723,62.01509889)(734.21578857,62.17510061)
\curveto(734.21579723,62.33509857)(734.20079724,62.47009844)(734.17078857,62.58010061)
\curveto(734.16079728,62.63009828)(734.15579729,62.68509822)(734.15578857,62.74510061)
\curveto(734.1457973,62.8050981)(734.13079731,62.86509804)(734.11078857,62.92510061)
\curveto(734.06079738,63.07509783)(734.01079743,63.22009769)(733.96078857,63.36010061)
\curveto(733.90079754,63.50009741)(733.83079761,63.63509727)(733.75078857,63.76510061)
\curveto(733.66079778,63.905097)(733.55579789,64.02509688)(733.43578857,64.12510061)
\curveto(733.31579813,64.22509668)(733.18579826,64.32009659)(733.04578857,64.41010061)
\curveto(732.9457985,64.47009644)(732.83579861,64.51509639)(732.71578857,64.54510061)
\curveto(732.59579885,64.58509632)(732.49079895,64.63509627)(732.40078857,64.69510061)
\curveto(732.3407991,64.74509616)(732.30079914,64.81509609)(732.28078857,64.90510061)
\curveto(732.27079917,64.92509598)(732.26579918,64.95009596)(732.26578857,64.98010061)
\curveto(732.26579918,65.0100959)(732.26079918,65.03509587)(732.25078857,65.05510061)
}
}
{
\newrgbcolor{curcolor}{0 0 0}
\pscustom[linestyle=none,fillstyle=solid,fillcolor=curcolor]
{
\newpath
\moveto(739.78078857,76.35970998)
\curveto(739.82079162,76.36970226)(739.87079157,76.36970226)(739.93078857,76.35970998)
\curveto(739.99079145,76.35970227)(740.0407914,76.35470228)(740.08078857,76.34470998)
\curveto(740.12079132,76.34470229)(740.16079128,76.33970229)(740.20078857,76.32970998)
\lineto(740.30578857,76.32970998)
\curveto(740.38579106,76.30970232)(740.46579098,76.29470234)(740.54578857,76.28470998)
\curveto(740.62579082,76.27470236)(740.70079074,76.25470238)(740.77078857,76.22470998)
\curveto(740.85079059,76.20470243)(740.92579052,76.18470245)(740.99578857,76.16470998)
\curveto(741.06579038,76.14470249)(741.1407903,76.11470252)(741.22078857,76.07470998)
\curveto(741.6407898,75.89470274)(741.98078946,75.63970299)(742.24078857,75.30970998)
\curveto(742.50078894,74.97970365)(742.70578874,74.58970404)(742.85578857,74.13970998)
\curveto(742.89578855,74.01970461)(742.92078852,73.89470474)(742.93078857,73.76470998)
\curveto(742.95078849,73.64470499)(742.97578847,73.51970511)(743.00578857,73.38970998)
\curveto(743.01578843,73.3297053)(743.02078842,73.26470537)(743.02078857,73.19470998)
\curveto(743.02078842,73.1347055)(743.02578842,73.06970556)(743.03578857,72.99970998)
\lineto(743.03578857,72.87970998)
\lineto(743.03578857,72.68470998)
\curveto(743.0457884,72.62470601)(743.0407884,72.56970606)(743.02078857,72.51970998)
\curveto(743.00078844,72.44970618)(742.99578845,72.38470625)(743.00578857,72.32470998)
\curveto(743.01578843,72.26470637)(743.01078843,72.20470643)(742.99078857,72.14470998)
\curveto(742.98078846,72.09470654)(742.97578847,72.04970658)(742.97578857,72.00970998)
\curveto(742.97578847,71.96970666)(742.96578848,71.92470671)(742.94578857,71.87470998)
\curveto(742.92578852,71.79470684)(742.90578854,71.71970691)(742.88578857,71.64970998)
\curveto(742.87578857,71.57970705)(742.86078858,71.50970712)(742.84078857,71.43970998)
\curveto(742.67078877,70.95970767)(742.46078898,70.55970807)(742.21078857,70.23970998)
\curveto(741.95078949,69.9297087)(741.59578985,69.67970895)(741.14578857,69.48970998)
\curveto(741.08579036,69.45970917)(741.02579042,69.4347092)(740.96578857,69.41470998)
\curveto(740.89579055,69.40470923)(740.82079062,69.38970924)(740.74078857,69.36970998)
\curveto(740.68079076,69.34970928)(740.61579083,69.3347093)(740.54578857,69.32470998)
\curveto(740.47579097,69.31470932)(740.40579104,69.29970933)(740.33578857,69.27970998)
\curveto(740.28579116,69.26970936)(740.2457912,69.26470937)(740.21578857,69.26470998)
\lineto(740.09578857,69.26470998)
\curveto(740.05579139,69.25470938)(740.00579144,69.24470939)(739.94578857,69.23470998)
\curveto(739.88579156,69.2347094)(739.83579161,69.23970939)(739.79578857,69.24970998)
\lineto(739.66078857,69.24970998)
\curveto(739.61079183,69.25970937)(739.56079188,69.26470937)(739.51078857,69.26470998)
\curveto(739.41079203,69.28470935)(739.31579213,69.29970933)(739.22578857,69.30970998)
\curveto(739.12579232,69.31970931)(739.03079241,69.33970929)(738.94078857,69.36970998)
\curveto(738.79079265,69.41970921)(738.65079279,69.47470916)(738.52078857,69.53470998)
\curveto(738.39079305,69.59470904)(738.27079317,69.66470897)(738.16078857,69.74470998)
\curveto(738.11079333,69.77470886)(738.07079337,69.80470883)(738.04078857,69.83470998)
\curveto(738.01079343,69.87470876)(737.97579347,69.90970872)(737.93578857,69.93970998)
\curveto(737.85579359,69.99970863)(737.78579366,70.06970856)(737.72578857,70.14970998)
\curveto(737.67579377,70.20970842)(737.63079381,70.26970836)(737.59078857,70.32970998)
\lineto(737.44078857,70.53970998)
\curveto(737.40079404,70.58970804)(737.36579408,70.63970799)(737.33578857,70.68970998)
\curveto(737.29579415,70.73970789)(737.2407942,70.77470786)(737.17078857,70.79470998)
\curveto(737.1407943,70.79470784)(737.11579433,70.78470785)(737.09578857,70.76470998)
\curveto(737.06579438,70.75470788)(737.0407944,70.74470789)(737.02078857,70.73470998)
\curveto(736.97079447,70.69470794)(736.92579452,70.64470799)(736.88578857,70.58470998)
\curveto(736.83579461,70.5347081)(736.79079465,70.48470815)(736.75078857,70.43470998)
\curveto(736.72079472,70.39470824)(736.66579478,70.34470829)(736.58578857,70.28470998)
\curveto(736.55579489,70.26470837)(736.53079491,70.2347084)(736.51078857,70.19470998)
\curveto(736.48079496,70.16470847)(736.445795,70.13970849)(736.40578857,70.11970998)
\curveto(736.19579525,69.94970868)(735.95079549,69.81970881)(735.67078857,69.72970998)
\curveto(735.59079585,69.70970892)(735.51079593,69.69470894)(735.43078857,69.68470998)
\curveto(735.35079609,69.67470896)(735.27079617,69.65970897)(735.19078857,69.63970998)
\curveto(735.1407963,69.61970901)(735.07579637,69.60970902)(734.99578857,69.60970998)
\curveto(734.90579654,69.60970902)(734.83579661,69.61970901)(734.78578857,69.63970998)
\curveto(734.68579676,69.63970899)(734.61579683,69.64470899)(734.57578857,69.65470998)
\curveto(734.49579695,69.67470896)(734.42579702,69.68970894)(734.36578857,69.69970998)
\curveto(734.29579715,69.70970892)(734.22579722,69.72470891)(734.15578857,69.74470998)
\curveto(733.72579772,69.89470874)(733.38079806,70.10970852)(733.12078857,70.38970998)
\curveto(732.86079858,70.67970795)(732.6457988,71.0297076)(732.47578857,71.43970998)
\curveto(732.42579902,71.54970708)(732.39579905,71.66470697)(732.38578857,71.78470998)
\curveto(732.36579908,71.91470672)(732.33579911,72.04470659)(732.29578857,72.17470998)
\curveto(732.29579915,72.25470638)(732.29579915,72.32470631)(732.29578857,72.38470998)
\curveto(732.28579916,72.45470618)(732.27579917,72.5297061)(732.26578857,72.60970998)
\curveto(732.2457992,73.39970523)(732.37579907,74.05470458)(732.65578857,74.57470998)
\curveto(732.93579851,75.10470353)(733.3457981,75.48470315)(733.88578857,75.71470998)
\curveto(734.11579733,75.82470281)(734.40079704,75.89470274)(734.74078857,75.92470998)
\curveto(735.07079637,75.96470267)(735.37579607,75.9347027)(735.65578857,75.83470998)
\curveto(735.78579566,75.79470284)(735.90579554,75.74470289)(736.01578857,75.68470998)
\curveto(736.12579532,75.634703)(736.23079521,75.57470306)(736.33078857,75.50470998)
\curveto(736.37079507,75.48470315)(736.40579504,75.45470318)(736.43578857,75.41470998)
\lineto(736.52578857,75.32470998)
\curveto(736.61579483,75.27470336)(736.68079476,75.21470342)(736.72078857,75.14470998)
\curveto(736.77079467,75.09470354)(736.82079462,75.03970359)(736.87078857,74.97970998)
\curveto(736.91079453,74.9297037)(736.95579449,74.88470375)(737.00578857,74.84470998)
\curveto(737.02579442,74.82470381)(737.05079439,74.80470383)(737.08078857,74.78470998)
\curveto(737.10079434,74.77470386)(737.12579432,74.77470386)(737.15578857,74.78470998)
\curveto(737.20579424,74.79470384)(737.25579419,74.82470381)(737.30578857,74.87470998)
\curveto(737.3457941,74.92470371)(737.38579406,74.97970365)(737.42578857,75.03970998)
\lineto(737.54578857,75.21970998)
\curveto(737.57579387,75.27970335)(737.60579384,75.3297033)(737.63578857,75.36970998)
\curveto(737.87579357,75.69970293)(738.18579326,75.94970268)(738.56578857,76.11970998)
\curveto(738.6457928,76.15970247)(738.73079271,76.18970244)(738.82078857,76.20970998)
\curveto(738.91079253,76.23970239)(739.00079244,76.26470237)(739.09078857,76.28470998)
\curveto(739.1407923,76.29470234)(739.19579225,76.30470233)(739.25578857,76.31470998)
\lineto(739.40578857,76.34470998)
\curveto(739.46579198,76.35470228)(739.53079191,76.35470228)(739.60078857,76.34470998)
\curveto(739.66079178,76.3347023)(739.72079172,76.33970229)(739.78078857,76.35970998)
\moveto(734.74078857,70.97470998)
\curveto(734.85079659,70.94470769)(734.99079645,70.93970769)(735.16078857,70.95970998)
\curveto(735.32079612,70.97970765)(735.445796,71.00470763)(735.53578857,71.03470998)
\curveto(735.85579559,71.14470749)(736.10079534,71.29470734)(736.27078857,71.48470998)
\curveto(736.43079501,71.67470696)(736.56079488,71.93970669)(736.66078857,72.27970998)
\curveto(736.69079475,72.40970622)(736.71579473,72.57470606)(736.73578857,72.77470998)
\curveto(736.7457947,72.97470566)(736.73079471,73.14470549)(736.69078857,73.28470998)
\curveto(736.61079483,73.57470506)(736.50079494,73.81470482)(736.36078857,74.00470998)
\curveto(736.21079523,74.20470443)(736.01079543,74.35970427)(735.76078857,74.46970998)
\curveto(735.71079573,74.48970414)(735.66579578,74.49970413)(735.62578857,74.49970998)
\curveto(735.58579586,74.50970412)(735.5407959,74.52470411)(735.49078857,74.54470998)
\curveto(735.38079606,74.57470406)(735.2407962,74.59470404)(735.07078857,74.60470998)
\curveto(734.90079654,74.61470402)(734.75579669,74.60470403)(734.63578857,74.57470998)
\curveto(734.5457969,74.55470408)(734.46079698,74.5297041)(734.38078857,74.49970998)
\curveto(734.30079714,74.47970415)(734.22079722,74.44470419)(734.14078857,74.39470998)
\curveto(733.87079757,74.22470441)(733.67579777,73.99970463)(733.55578857,73.71970998)
\curveto(733.43579801,73.44970518)(733.37579807,73.08970554)(733.37578857,72.63970998)
\curveto(733.39579805,72.61970601)(733.40079804,72.58970604)(733.39078857,72.54970998)
\curveto(733.38079806,72.50970612)(733.38079806,72.47470616)(733.39078857,72.44470998)
\curveto(733.41079803,72.39470624)(733.42579802,72.33970629)(733.43578857,72.27970998)
\curveto(733.43579801,72.2297064)(733.445798,72.17970645)(733.46578857,72.12970998)
\curveto(733.55579789,71.88970674)(733.67079777,71.67970695)(733.81078857,71.49970998)
\curveto(733.9407975,71.31970731)(734.12079732,71.17970745)(734.35078857,71.07970998)
\curveto(734.41079703,71.05970757)(734.47579697,71.03970759)(734.54578857,71.01970998)
\curveto(734.60579684,71.00970762)(734.67079677,70.99470764)(734.74078857,70.97470998)
\moveto(740.27578857,74.99470998)
\curveto(740.08579136,75.04470359)(739.88079156,75.04970358)(739.66078857,75.00970998)
\curveto(739.440792,74.97970365)(739.26079218,74.9347037)(739.12078857,74.87470998)
\curveto(738.75079269,74.70470393)(738.445793,74.44470419)(738.20578857,74.09470998)
\curveto(737.96579348,73.75470488)(737.8457936,73.31970531)(737.84578857,72.78970998)
\curveto(737.86579358,72.75970587)(737.87079357,72.71970591)(737.86078857,72.66970998)
\curveto(737.8407936,72.61970601)(737.83579361,72.57970605)(737.84578857,72.54970998)
\lineto(737.90578857,72.27970998)
\curveto(737.91579353,72.19970643)(737.93079351,72.11970651)(737.95078857,72.03970998)
\curveto(738.06079338,71.73970689)(738.20579324,71.47470716)(738.38578857,71.24470998)
\curveto(738.56579288,71.02470761)(738.79579265,70.85470778)(739.07578857,70.73470998)
\curveto(739.15579229,70.70470793)(739.23579221,70.67970795)(739.31578857,70.65970998)
\curveto(739.39579205,70.63970799)(739.48079196,70.61970801)(739.57078857,70.59970998)
\curveto(739.69079175,70.56970806)(739.8407916,70.55970807)(740.02078857,70.56970998)
\curveto(740.20079124,70.58970804)(740.3407911,70.61470802)(740.44078857,70.64470998)
\curveto(740.49079095,70.66470797)(740.53579091,70.67470796)(740.57578857,70.67470998)
\curveto(740.60579084,70.68470795)(740.6457908,70.69970793)(740.69578857,70.71970998)
\curveto(740.91579053,70.81970781)(741.11579033,70.94970768)(741.29578857,71.10970998)
\curveto(741.47578997,71.27970735)(741.61078983,71.47470716)(741.70078857,71.69470998)
\curveto(741.7407897,71.76470687)(741.77578967,71.85970677)(741.80578857,71.97970998)
\curveto(741.89578955,72.19970643)(741.9407895,72.45470618)(741.94078857,72.74470998)
\lineto(741.94078857,73.02970998)
\curveto(741.92078952,73.1297055)(741.90578954,73.22470541)(741.89578857,73.31470998)
\curveto(741.88578956,73.40470523)(741.86578958,73.49470514)(741.83578857,73.58470998)
\curveto(741.75578969,73.84470479)(741.62578982,74.08470455)(741.44578857,74.30470998)
\curveto(741.25579019,74.5347041)(741.0407904,74.70470393)(740.80078857,74.81470998)
\curveto(740.72079072,74.85470378)(740.6407908,74.88470375)(740.56078857,74.90470998)
\curveto(740.47079097,74.9347037)(740.37579107,74.96470367)(740.27578857,74.99470998)
}
}
{
\newrgbcolor{curcolor}{0 0 0}
\pscustom[linestyle=none,fillstyle=solid,fillcolor=curcolor]
{
\newpath
\moveto(741.22078857,78.63431936)
\lineto(741.22078857,79.26431936)
\lineto(741.22078857,79.45931936)
\curveto(741.22079022,79.52931683)(741.23079021,79.58931677)(741.25078857,79.63931936)
\curveto(741.29079015,79.70931665)(741.33079011,79.7593166)(741.37078857,79.78931936)
\curveto(741.42079002,79.82931653)(741.48578996,79.84931651)(741.56578857,79.84931936)
\curveto(741.6457898,79.8593165)(741.73078971,79.86431649)(741.82078857,79.86431936)
\lineto(742.54078857,79.86431936)
\curveto(743.02078842,79.86431649)(743.43078801,79.80431655)(743.77078857,79.68431936)
\curveto(744.11078733,79.56431679)(744.38578706,79.36931699)(744.59578857,79.09931936)
\curveto(744.6457868,79.02931733)(744.69078675,78.9593174)(744.73078857,78.88931936)
\curveto(744.78078666,78.82931753)(744.82578662,78.7543176)(744.86578857,78.66431936)
\curveto(744.87578657,78.64431771)(744.88578656,78.61431774)(744.89578857,78.57431936)
\curveto(744.91578653,78.53431782)(744.92078652,78.48931787)(744.91078857,78.43931936)
\curveto(744.88078656,78.34931801)(744.80578664,78.29431806)(744.68578857,78.27431936)
\curveto(744.57578687,78.2543181)(744.48078696,78.26931809)(744.40078857,78.31931936)
\curveto(744.33078711,78.34931801)(744.26578718,78.39431796)(744.20578857,78.45431936)
\curveto(744.15578729,78.52431783)(744.10578734,78.58931777)(744.05578857,78.64931936)
\curveto(744.00578744,78.71931764)(743.93078751,78.77931758)(743.83078857,78.82931936)
\curveto(743.7407877,78.88931747)(743.65078779,78.93931742)(743.56078857,78.97931936)
\curveto(743.53078791,78.99931736)(743.47078797,79.02431733)(743.38078857,79.05431936)
\curveto(743.30078814,79.08431727)(743.23078821,79.08931727)(743.17078857,79.06931936)
\curveto(743.03078841,79.03931732)(742.9407885,78.97931738)(742.90078857,78.88931936)
\curveto(742.87078857,78.80931755)(742.85578859,78.71931764)(742.85578857,78.61931936)
\curveto(742.85578859,78.51931784)(742.83078861,78.43431792)(742.78078857,78.36431936)
\curveto(742.71078873,78.27431808)(742.57078887,78.22931813)(742.36078857,78.22931936)
\lineto(741.80578857,78.22931936)
\lineto(741.58078857,78.22931936)
\curveto(741.50078994,78.23931812)(741.43579001,78.2593181)(741.38578857,78.28931936)
\curveto(741.30579014,78.34931801)(741.26079018,78.41931794)(741.25078857,78.49931936)
\curveto(741.2407902,78.51931784)(741.23579021,78.53931782)(741.23578857,78.55931936)
\curveto(741.23579021,78.58931777)(741.23079021,78.61431774)(741.22078857,78.63431936)
}
}
{
\newrgbcolor{curcolor}{0 0 0}
\pscustom[linestyle=none,fillstyle=solid,fillcolor=curcolor]
{
}
}
{
\newrgbcolor{curcolor}{0 0 0}
\pscustom[linestyle=none,fillstyle=solid,fillcolor=curcolor]
{
\newpath
\moveto(732.25078857,89.26463186)
\curveto(732.2407992,89.95462722)(732.36079908,90.55462662)(732.61078857,91.06463186)
\curveto(732.86079858,91.58462559)(733.19579825,91.9796252)(733.61578857,92.24963186)
\curveto(733.69579775,92.29962488)(733.78579766,92.34462483)(733.88578857,92.38463186)
\curveto(733.97579747,92.42462475)(734.07079737,92.46962471)(734.17078857,92.51963186)
\curveto(734.27079717,92.55962462)(734.37079707,92.58962459)(734.47078857,92.60963186)
\curveto(734.57079687,92.62962455)(734.67579677,92.64962453)(734.78578857,92.66963186)
\curveto(734.83579661,92.68962449)(734.88079656,92.69462448)(734.92078857,92.68463186)
\curveto(734.96079648,92.6746245)(735.00579644,92.6796245)(735.05578857,92.69963186)
\curveto(735.10579634,92.70962447)(735.19079625,92.71462446)(735.31078857,92.71463186)
\curveto(735.42079602,92.71462446)(735.50579594,92.70962447)(735.56578857,92.69963186)
\curveto(735.62579582,92.6796245)(735.68579576,92.66962451)(735.74578857,92.66963186)
\curveto(735.80579564,92.6796245)(735.86579558,92.6746245)(735.92578857,92.65463186)
\curveto(736.06579538,92.61462456)(736.20079524,92.5796246)(736.33078857,92.54963186)
\curveto(736.46079498,92.51962466)(736.58579486,92.4796247)(736.70578857,92.42963186)
\curveto(736.8457946,92.36962481)(736.97079447,92.29962488)(737.08078857,92.21963186)
\curveto(737.19079425,92.14962503)(737.30079414,92.0746251)(737.41078857,91.99463186)
\lineto(737.47078857,91.93463186)
\curveto(737.49079395,91.92462525)(737.51079393,91.90962527)(737.53078857,91.88963186)
\curveto(737.69079375,91.76962541)(737.83579361,91.63462554)(737.96578857,91.48463186)
\curveto(738.09579335,91.33462584)(738.22079322,91.174626)(738.34078857,91.00463186)
\curveto(738.56079288,90.69462648)(738.76579268,90.39962678)(738.95578857,90.11963186)
\curveto(739.09579235,89.88962729)(739.23079221,89.65962752)(739.36078857,89.42963186)
\curveto(739.49079195,89.20962797)(739.62579182,88.98962819)(739.76578857,88.76963186)
\curveto(739.93579151,88.51962866)(740.11579133,88.2796289)(740.30578857,88.04963186)
\curveto(740.49579095,87.82962935)(740.72079072,87.63962954)(740.98078857,87.47963186)
\curveto(741.0407904,87.43962974)(741.10079034,87.40462977)(741.16078857,87.37463186)
\curveto(741.21079023,87.34462983)(741.27579017,87.31462986)(741.35578857,87.28463186)
\curveto(741.42579002,87.26462991)(741.48578996,87.25962992)(741.53578857,87.26963186)
\curveto(741.60578984,87.28962989)(741.66078978,87.32462985)(741.70078857,87.37463186)
\curveto(741.73078971,87.42462975)(741.75078969,87.48462969)(741.76078857,87.55463186)
\lineto(741.76078857,87.79463186)
\lineto(741.76078857,88.54463186)
\lineto(741.76078857,91.34963186)
\lineto(741.76078857,92.00963186)
\curveto(741.76078968,92.09962508)(741.76578968,92.18462499)(741.77578857,92.26463186)
\curveto(741.77578967,92.34462483)(741.79578965,92.40962477)(741.83578857,92.45963186)
\curveto(741.87578957,92.50962467)(741.95078949,92.54962463)(742.06078857,92.57963186)
\curveto(742.16078928,92.61962456)(742.26078918,92.62962455)(742.36078857,92.60963186)
\lineto(742.49578857,92.60963186)
\curveto(742.56578888,92.58962459)(742.62578882,92.56962461)(742.67578857,92.54963186)
\curveto(742.72578872,92.52962465)(742.76578868,92.49462468)(742.79578857,92.44463186)
\curveto(742.83578861,92.39462478)(742.85578859,92.32462485)(742.85578857,92.23463186)
\lineto(742.85578857,91.96463186)
\lineto(742.85578857,91.06463186)
\lineto(742.85578857,87.55463186)
\lineto(742.85578857,86.48963186)
\curveto(742.85578859,86.40963077)(742.86078858,86.31963086)(742.87078857,86.21963186)
\curveto(742.87078857,86.11963106)(742.86078858,86.03463114)(742.84078857,85.96463186)
\curveto(742.77078867,85.75463142)(742.59078885,85.68963149)(742.30078857,85.76963186)
\curveto(742.26078918,85.7796314)(742.22578922,85.7796314)(742.19578857,85.76963186)
\curveto(742.15578929,85.76963141)(742.11078933,85.7796314)(742.06078857,85.79963186)
\curveto(741.98078946,85.81963136)(741.89578955,85.83963134)(741.80578857,85.85963186)
\curveto(741.71578973,85.8796313)(741.63078981,85.90463127)(741.55078857,85.93463186)
\curveto(741.06079038,86.09463108)(740.6457908,86.29463088)(740.30578857,86.53463186)
\curveto(740.05579139,86.71463046)(739.83079161,86.91963026)(739.63078857,87.14963186)
\curveto(739.42079202,87.3796298)(739.22579222,87.61962956)(739.04578857,87.86963186)
\curveto(738.86579258,88.12962905)(738.69579275,88.39462878)(738.53578857,88.66463186)
\curveto(738.36579308,88.94462823)(738.19079325,89.21462796)(738.01078857,89.47463186)
\curveto(737.93079351,89.58462759)(737.85579359,89.68962749)(737.78578857,89.78963186)
\curveto(737.71579373,89.89962728)(737.6407938,90.00962717)(737.56078857,90.11963186)
\curveto(737.53079391,90.15962702)(737.50079394,90.19462698)(737.47078857,90.22463186)
\curveto(737.43079401,90.26462691)(737.40079404,90.30462687)(737.38078857,90.34463186)
\curveto(737.27079417,90.48462669)(737.1457943,90.60962657)(737.00578857,90.71963186)
\curveto(736.97579447,90.73962644)(736.95079449,90.76462641)(736.93078857,90.79463186)
\curveto(736.90079454,90.82462635)(736.87079457,90.84962633)(736.84078857,90.86963186)
\curveto(736.7407947,90.94962623)(736.6407948,91.01462616)(736.54078857,91.06463186)
\curveto(736.440795,91.12462605)(736.33079511,91.179626)(736.21078857,91.22963186)
\curveto(736.1407953,91.25962592)(736.06579538,91.2796259)(735.98578857,91.28963186)
\lineto(735.74578857,91.34963186)
\lineto(735.65578857,91.34963186)
\curveto(735.62579582,91.35962582)(735.59579585,91.36462581)(735.56578857,91.36463186)
\curveto(735.49579595,91.38462579)(735.40079604,91.38962579)(735.28078857,91.37963186)
\curveto(735.15079629,91.3796258)(735.05079639,91.36962581)(734.98078857,91.34963186)
\curveto(734.90079654,91.32962585)(734.82579662,91.30962587)(734.75578857,91.28963186)
\curveto(734.67579677,91.2796259)(734.59579685,91.25962592)(734.51578857,91.22963186)
\curveto(734.27579717,91.11962606)(734.07579737,90.96962621)(733.91578857,90.77963186)
\curveto(733.7457977,90.59962658)(733.60579784,90.3796268)(733.49578857,90.11963186)
\curveto(733.47579797,90.04962713)(733.46079798,89.9796272)(733.45078857,89.90963186)
\curveto(733.43079801,89.83962734)(733.41079803,89.76462741)(733.39078857,89.68463186)
\curveto(733.37079807,89.60462757)(733.36079808,89.49462768)(733.36078857,89.35463186)
\curveto(733.36079808,89.22462795)(733.37079807,89.11962806)(733.39078857,89.03963186)
\curveto(733.40079804,88.9796282)(733.40579804,88.92462825)(733.40578857,88.87463186)
\curveto(733.40579804,88.82462835)(733.41579803,88.7746284)(733.43578857,88.72463186)
\curveto(733.47579797,88.62462855)(733.51579793,88.52962865)(733.55578857,88.43963186)
\curveto(733.59579785,88.35962882)(733.6407978,88.2796289)(733.69078857,88.19963186)
\curveto(733.71079773,88.16962901)(733.73579771,88.13962904)(733.76578857,88.10963186)
\curveto(733.79579765,88.08962909)(733.82079762,88.06462911)(733.84078857,88.03463186)
\lineto(733.91578857,87.95963186)
\curveto(733.93579751,87.92962925)(733.95579749,87.90462927)(733.97578857,87.88463186)
\lineto(734.18578857,87.73463186)
\curveto(734.2457972,87.69462948)(734.31079713,87.64962953)(734.38078857,87.59963186)
\curveto(734.47079697,87.53962964)(734.57579687,87.48962969)(734.69578857,87.44963186)
\curveto(734.80579664,87.41962976)(734.91579653,87.38462979)(735.02578857,87.34463186)
\curveto(735.13579631,87.30462987)(735.28079616,87.2796299)(735.46078857,87.26963186)
\curveto(735.63079581,87.25962992)(735.75579569,87.22962995)(735.83578857,87.17963186)
\curveto(735.91579553,87.12963005)(735.96079548,87.05463012)(735.97078857,86.95463186)
\curveto(735.98079546,86.85463032)(735.98579546,86.74463043)(735.98578857,86.62463186)
\curveto(735.98579546,86.58463059)(735.99079545,86.54463063)(736.00078857,86.50463186)
\curveto(736.00079544,86.46463071)(735.99579545,86.42963075)(735.98578857,86.39963186)
\curveto(735.96579548,86.34963083)(735.95579549,86.29963088)(735.95578857,86.24963186)
\curveto(735.95579549,86.20963097)(735.9457955,86.16963101)(735.92578857,86.12963186)
\curveto(735.86579558,86.03963114)(735.73079571,85.99463118)(735.52078857,85.99463186)
\lineto(735.40078857,85.99463186)
\curveto(735.3407961,86.00463117)(735.28079616,86.00963117)(735.22078857,86.00963186)
\curveto(735.15079629,86.01963116)(735.08579636,86.02963115)(735.02578857,86.03963186)
\curveto(734.91579653,86.05963112)(734.81579663,86.0796311)(734.72578857,86.09963186)
\curveto(734.62579682,86.11963106)(734.53079691,86.14963103)(734.44078857,86.18963186)
\curveto(734.37079707,86.20963097)(734.31079713,86.22963095)(734.26078857,86.24963186)
\lineto(734.08078857,86.30963186)
\curveto(733.82079762,86.42963075)(733.57579787,86.58463059)(733.34578857,86.77463186)
\curveto(733.11579833,86.9746302)(732.93079851,87.18962999)(732.79078857,87.41963186)
\curveto(732.71079873,87.52962965)(732.6457988,87.64462953)(732.59578857,87.76463186)
\lineto(732.44578857,88.15463186)
\curveto(732.39579905,88.26462891)(732.36579908,88.3796288)(732.35578857,88.49963186)
\curveto(732.33579911,88.61962856)(732.31079913,88.74462843)(732.28078857,88.87463186)
\curveto(732.28079916,88.94462823)(732.28079916,89.00962817)(732.28078857,89.06963186)
\curveto(732.27079917,89.12962805)(732.26079918,89.19462798)(732.25078857,89.26463186)
}
}
{
\newrgbcolor{curcolor}{0 0 0}
\pscustom[linestyle=none,fillstyle=solid,fillcolor=curcolor]
{
\newpath
\moveto(737.77078857,101.36424123)
\lineto(738.02578857,101.36424123)
\curveto(738.10579334,101.37423353)(738.18079326,101.36923353)(738.25078857,101.34924123)
\lineto(738.49078857,101.34924123)
\lineto(738.65578857,101.34924123)
\curveto(738.75579269,101.32923357)(738.86079258,101.31923358)(738.97078857,101.31924123)
\curveto(739.07079237,101.31923358)(739.17079227,101.30923359)(739.27078857,101.28924123)
\lineto(739.42078857,101.28924123)
\curveto(739.56079188,101.25923364)(739.70079174,101.23923366)(739.84078857,101.22924123)
\curveto(739.97079147,101.21923368)(740.10079134,101.19423371)(740.23078857,101.15424123)
\curveto(740.31079113,101.13423377)(740.39579105,101.11423379)(740.48578857,101.09424123)
\lineto(740.72578857,101.03424123)
\lineto(741.02578857,100.91424123)
\curveto(741.11579033,100.88423402)(741.20579024,100.84923405)(741.29578857,100.80924123)
\curveto(741.51578993,100.70923419)(741.73078971,100.57423433)(741.94078857,100.40424123)
\curveto(742.15078929,100.24423466)(742.32078912,100.06923483)(742.45078857,99.87924123)
\curveto(742.49078895,99.82923507)(742.53078891,99.76923513)(742.57078857,99.69924123)
\curveto(742.60078884,99.63923526)(742.63578881,99.57923532)(742.67578857,99.51924123)
\curveto(742.72578872,99.43923546)(742.76578868,99.34423556)(742.79578857,99.23424123)
\curveto(742.82578862,99.12423578)(742.85578859,99.01923588)(742.88578857,98.91924123)
\curveto(742.92578852,98.80923609)(742.95078849,98.6992362)(742.96078857,98.58924123)
\curveto(742.97078847,98.47923642)(742.98578846,98.36423654)(743.00578857,98.24424123)
\curveto(743.01578843,98.2042367)(743.01578843,98.15923674)(743.00578857,98.10924123)
\curveto(743.00578844,98.06923683)(743.01078843,98.02923687)(743.02078857,97.98924123)
\curveto(743.03078841,97.94923695)(743.03578841,97.89423701)(743.03578857,97.82424123)
\curveto(743.03578841,97.75423715)(743.03078841,97.7042372)(743.02078857,97.67424123)
\curveto(743.00078844,97.62423728)(742.99578845,97.57923732)(743.00578857,97.53924123)
\curveto(743.01578843,97.4992374)(743.01578843,97.46423744)(743.00578857,97.43424123)
\lineto(743.00578857,97.34424123)
\curveto(742.98578846,97.28423762)(742.97078847,97.21923768)(742.96078857,97.14924123)
\curveto(742.96078848,97.08923781)(742.95578849,97.02423788)(742.94578857,96.95424123)
\curveto(742.89578855,96.78423812)(742.8457886,96.62423828)(742.79578857,96.47424123)
\curveto(742.7457887,96.32423858)(742.68078876,96.17923872)(742.60078857,96.03924123)
\curveto(742.56078888,95.98923891)(742.53078891,95.93423897)(742.51078857,95.87424123)
\curveto(742.48078896,95.82423908)(742.445789,95.77423913)(742.40578857,95.72424123)
\curveto(742.22578922,95.48423942)(742.00578944,95.28423962)(741.74578857,95.12424123)
\curveto(741.48578996,94.96423994)(741.20079024,94.82424008)(740.89078857,94.70424123)
\curveto(740.75079069,94.64424026)(740.61079083,94.5992403)(740.47078857,94.56924123)
\curveto(740.32079112,94.53924036)(740.16579128,94.5042404)(740.00578857,94.46424123)
\curveto(739.89579155,94.44424046)(739.78579166,94.42924047)(739.67578857,94.41924123)
\curveto(739.56579188,94.40924049)(739.45579199,94.39424051)(739.34578857,94.37424123)
\curveto(739.30579214,94.36424054)(739.26579218,94.35924054)(739.22578857,94.35924123)
\curveto(739.18579226,94.36924053)(739.1457923,94.36924053)(739.10578857,94.35924123)
\curveto(739.05579239,94.34924055)(739.00579244,94.34424056)(738.95578857,94.34424123)
\lineto(738.79078857,94.34424123)
\curveto(738.7407927,94.32424058)(738.69079275,94.31924058)(738.64078857,94.32924123)
\curveto(738.58079286,94.33924056)(738.52579292,94.33924056)(738.47578857,94.32924123)
\curveto(738.43579301,94.31924058)(738.39079305,94.31924058)(738.34078857,94.32924123)
\curveto(738.29079315,94.33924056)(738.2407932,94.33424057)(738.19078857,94.31424123)
\curveto(738.12079332,94.29424061)(738.0457934,94.28924061)(737.96578857,94.29924123)
\curveto(737.87579357,94.30924059)(737.79079365,94.31424059)(737.71078857,94.31424123)
\curveto(737.62079382,94.31424059)(737.52079392,94.30924059)(737.41078857,94.29924123)
\curveto(737.29079415,94.28924061)(737.19079425,94.29424061)(737.11078857,94.31424123)
\lineto(736.82578857,94.31424123)
\lineto(736.19578857,94.35924123)
\curveto(736.09579535,94.36924053)(736.00079544,94.37924052)(735.91078857,94.38924123)
\lineto(735.61078857,94.41924123)
\curveto(735.56079588,94.43924046)(735.51079593,94.44424046)(735.46078857,94.43424123)
\curveto(735.40079604,94.43424047)(735.3457961,94.44424046)(735.29578857,94.46424123)
\curveto(735.12579632,94.51424039)(734.96079648,94.55424035)(734.80078857,94.58424123)
\curveto(734.63079681,94.61424029)(734.47079697,94.66424024)(734.32078857,94.73424123)
\curveto(733.86079758,94.92423998)(733.48579796,95.14423976)(733.19578857,95.39424123)
\curveto(732.90579854,95.65423925)(732.66079878,96.01423889)(732.46078857,96.47424123)
\curveto(732.41079903,96.6042383)(732.37579907,96.73423817)(732.35578857,96.86424123)
\curveto(732.33579911,97.0042379)(732.31079913,97.14423776)(732.28078857,97.28424123)
\curveto(732.27079917,97.35423755)(732.26579918,97.41923748)(732.26578857,97.47924123)
\curveto(732.26579918,97.53923736)(732.26079918,97.6042373)(732.25078857,97.67424123)
\curveto(732.23079921,98.5042364)(732.38079906,99.17423573)(732.70078857,99.68424123)
\curveto(733.01079843,100.19423471)(733.45079799,100.57423433)(734.02078857,100.82424123)
\curveto(734.1407973,100.87423403)(734.26579718,100.91923398)(734.39578857,100.95924123)
\curveto(734.52579692,100.9992339)(734.66079678,101.04423386)(734.80078857,101.09424123)
\curveto(734.88079656,101.11423379)(734.96579648,101.12923377)(735.05578857,101.13924123)
\lineto(735.29578857,101.19924123)
\curveto(735.40579604,101.22923367)(735.51579593,101.24423366)(735.62578857,101.24424123)
\curveto(735.73579571,101.25423365)(735.8457956,101.26923363)(735.95578857,101.28924123)
\curveto(736.00579544,101.30923359)(736.05079539,101.31423359)(736.09078857,101.30424123)
\curveto(736.13079531,101.3042336)(736.17079527,101.30923359)(736.21078857,101.31924123)
\curveto(736.26079518,101.32923357)(736.31579513,101.32923357)(736.37578857,101.31924123)
\curveto(736.42579502,101.31923358)(736.47579497,101.32423358)(736.52578857,101.33424123)
\lineto(736.66078857,101.33424123)
\curveto(736.72079472,101.35423355)(736.79079465,101.35423355)(736.87078857,101.33424123)
\curveto(736.9407945,101.32423358)(737.00579444,101.32923357)(737.06578857,101.34924123)
\curveto(737.09579435,101.35923354)(737.13579431,101.36423354)(737.18578857,101.36424123)
\lineto(737.30578857,101.36424123)
\lineto(737.77078857,101.36424123)
\moveto(740.09578857,99.81924123)
\curveto(739.77579167,99.91923498)(739.41079203,99.97923492)(739.00078857,99.99924123)
\curveto(738.59079285,100.01923488)(738.18079326,100.02923487)(737.77078857,100.02924123)
\curveto(737.3407941,100.02923487)(736.92079452,100.01923488)(736.51078857,99.99924123)
\curveto(736.10079534,99.97923492)(735.71579573,99.93423497)(735.35578857,99.86424123)
\curveto(734.99579645,99.79423511)(734.67579677,99.68423522)(734.39578857,99.53424123)
\curveto(734.10579734,99.39423551)(733.87079757,99.1992357)(733.69078857,98.94924123)
\curveto(733.58079786,98.78923611)(733.50079794,98.60923629)(733.45078857,98.40924123)
\curveto(733.39079805,98.20923669)(733.36079808,97.96423694)(733.36078857,97.67424123)
\curveto(733.38079806,97.65423725)(733.39079805,97.61923728)(733.39078857,97.56924123)
\curveto(733.38079806,97.51923738)(733.38079806,97.47923742)(733.39078857,97.44924123)
\curveto(733.41079803,97.36923753)(733.43079801,97.29423761)(733.45078857,97.22424123)
\curveto(733.46079798,97.16423774)(733.48079796,97.0992378)(733.51078857,97.02924123)
\curveto(733.63079781,96.75923814)(733.80079764,96.53923836)(734.02078857,96.36924123)
\curveto(734.23079721,96.20923869)(734.47579697,96.07423883)(734.75578857,95.96424123)
\curveto(734.86579658,95.91423899)(734.98579646,95.87423903)(735.11578857,95.84424123)
\curveto(735.23579621,95.82423908)(735.36079608,95.7992391)(735.49078857,95.76924123)
\curveto(735.5407959,95.74923915)(735.59579585,95.73923916)(735.65578857,95.73924123)
\curveto(735.70579574,95.73923916)(735.75579569,95.73423917)(735.80578857,95.72424123)
\curveto(735.89579555,95.71423919)(735.99079545,95.7042392)(736.09078857,95.69424123)
\curveto(736.18079526,95.68423922)(736.27579517,95.67423923)(736.37578857,95.66424123)
\curveto(736.45579499,95.66423924)(736.5407949,95.65923924)(736.63078857,95.64924123)
\lineto(736.87078857,95.64924123)
\lineto(737.05078857,95.64924123)
\curveto(737.08079436,95.63923926)(737.11579433,95.63423927)(737.15578857,95.63424123)
\lineto(737.29078857,95.63424123)
\lineto(737.74078857,95.63424123)
\curveto(737.82079362,95.63423927)(737.90579354,95.62923927)(737.99578857,95.61924123)
\curveto(738.07579337,95.61923928)(738.15079329,95.62923927)(738.22078857,95.64924123)
\lineto(738.49078857,95.64924123)
\curveto(738.51079293,95.64923925)(738.5407929,95.64423926)(738.58078857,95.63424123)
\curveto(738.61079283,95.63423927)(738.63579281,95.63923926)(738.65578857,95.64924123)
\curveto(738.75579269,95.65923924)(738.85579259,95.66423924)(738.95578857,95.66424123)
\curveto(739.0457924,95.67423923)(739.1457923,95.68423922)(739.25578857,95.69424123)
\curveto(739.37579207,95.72423918)(739.50079194,95.73923916)(739.63078857,95.73924123)
\curveto(739.75079169,95.74923915)(739.86579158,95.77423913)(739.97578857,95.81424123)
\curveto(740.27579117,95.89423901)(740.5407909,95.97923892)(740.77078857,96.06924123)
\curveto(741.00079044,96.16923873)(741.21579023,96.31423859)(741.41578857,96.50424123)
\curveto(741.61578983,96.71423819)(741.76578968,96.97923792)(741.86578857,97.29924123)
\curveto(741.88578956,97.33923756)(741.89578955,97.37423753)(741.89578857,97.40424123)
\curveto(741.88578956,97.44423746)(741.89078955,97.48923741)(741.91078857,97.53924123)
\curveto(741.92078952,97.57923732)(741.93078951,97.64923725)(741.94078857,97.74924123)
\curveto(741.95078949,97.85923704)(741.9457895,97.94423696)(741.92578857,98.00424123)
\curveto(741.90578954,98.07423683)(741.89578955,98.14423676)(741.89578857,98.21424123)
\curveto(741.88578956,98.28423662)(741.87078957,98.34923655)(741.85078857,98.40924123)
\curveto(741.79078965,98.60923629)(741.70578974,98.78923611)(741.59578857,98.94924123)
\curveto(741.57578987,98.97923592)(741.55578989,99.0042359)(741.53578857,99.02424123)
\lineto(741.47578857,99.08424123)
\curveto(741.45578999,99.12423578)(741.41579003,99.17423573)(741.35578857,99.23424123)
\curveto(741.21579023,99.33423557)(741.08579036,99.41923548)(740.96578857,99.48924123)
\curveto(740.8457906,99.55923534)(740.70079074,99.62923527)(740.53078857,99.69924123)
\curveto(740.46079098,99.72923517)(740.39079105,99.74923515)(740.32078857,99.75924123)
\curveto(740.25079119,99.77923512)(740.17579127,99.7992351)(740.09578857,99.81924123)
}
}
{
\newrgbcolor{curcolor}{0 0 0}
\pscustom[linestyle=none,fillstyle=solid,fillcolor=curcolor]
{
\newpath
\moveto(732.25078857,106.77385061)
\curveto(732.25079919,106.87384575)(732.26079918,106.96884566)(732.28078857,107.05885061)
\curveto(732.29079915,107.14884548)(732.32079912,107.21384541)(732.37078857,107.25385061)
\curveto(732.45079899,107.31384531)(732.55579889,107.34384528)(732.68578857,107.34385061)
\lineto(733.07578857,107.34385061)
\lineto(734.57578857,107.34385061)
\lineto(740.96578857,107.34385061)
\lineto(742.13578857,107.34385061)
\lineto(742.45078857,107.34385061)
\curveto(742.55078889,107.35384527)(742.63078881,107.33884529)(742.69078857,107.29885061)
\curveto(742.77078867,107.24884538)(742.82078862,107.17384545)(742.84078857,107.07385061)
\curveto(742.85078859,106.98384564)(742.85578859,106.87384575)(742.85578857,106.74385061)
\lineto(742.85578857,106.51885061)
\curveto(742.83578861,106.43884619)(742.82078862,106.36884626)(742.81078857,106.30885061)
\curveto(742.79078865,106.24884638)(742.75078869,106.19884643)(742.69078857,106.15885061)
\curveto(742.63078881,106.11884651)(742.55578889,106.09884653)(742.46578857,106.09885061)
\lineto(742.16578857,106.09885061)
\lineto(741.07078857,106.09885061)
\lineto(735.73078857,106.09885061)
\curveto(735.6407958,106.07884655)(735.56579588,106.06384656)(735.50578857,106.05385061)
\curveto(735.43579601,106.05384657)(735.37579607,106.0238466)(735.32578857,105.96385061)
\curveto(735.27579617,105.89384673)(735.25079619,105.80384682)(735.25078857,105.69385061)
\curveto(735.2407962,105.59384703)(735.23579621,105.48384714)(735.23578857,105.36385061)
\lineto(735.23578857,104.22385061)
\lineto(735.23578857,103.72885061)
\curveto(735.22579622,103.56884906)(735.16579628,103.45884917)(735.05578857,103.39885061)
\curveto(735.02579642,103.37884925)(734.99579645,103.36884926)(734.96578857,103.36885061)
\curveto(734.92579652,103.36884926)(734.88079656,103.36384926)(734.83078857,103.35385061)
\curveto(734.71079673,103.33384929)(734.60079684,103.33884929)(734.50078857,103.36885061)
\curveto(734.40079704,103.40884922)(734.33079711,103.46384916)(734.29078857,103.53385061)
\curveto(734.2407972,103.61384901)(734.21579723,103.73384889)(734.21578857,103.89385061)
\curveto(734.21579723,104.05384857)(734.20079724,104.18884844)(734.17078857,104.29885061)
\curveto(734.16079728,104.34884828)(734.15579729,104.40384822)(734.15578857,104.46385061)
\curveto(734.1457973,104.5238481)(734.13079731,104.58384804)(734.11078857,104.64385061)
\curveto(734.06079738,104.79384783)(734.01079743,104.93884769)(733.96078857,105.07885061)
\curveto(733.90079754,105.21884741)(733.83079761,105.35384727)(733.75078857,105.48385061)
\curveto(733.66079778,105.623847)(733.55579789,105.74384688)(733.43578857,105.84385061)
\curveto(733.31579813,105.94384668)(733.18579826,106.03884659)(733.04578857,106.12885061)
\curveto(732.9457985,106.18884644)(732.83579861,106.23384639)(732.71578857,106.26385061)
\curveto(732.59579885,106.30384632)(732.49079895,106.35384627)(732.40078857,106.41385061)
\curveto(732.3407991,106.46384616)(732.30079914,106.53384609)(732.28078857,106.62385061)
\curveto(732.27079917,106.64384598)(732.26579918,106.66884596)(732.26578857,106.69885061)
\curveto(732.26579918,106.7288459)(732.26079918,106.75384587)(732.25078857,106.77385061)
}
}
{
\newrgbcolor{curcolor}{0 0 0}
\pscustom[linestyle=none,fillstyle=solid,fillcolor=curcolor]
{
\newpath
\moveto(732.25078857,115.12345998)
\curveto(732.25079919,115.22345513)(732.26079918,115.31845503)(732.28078857,115.40845998)
\curveto(732.29079915,115.49845485)(732.32079912,115.56345479)(732.37078857,115.60345998)
\curveto(732.45079899,115.66345469)(732.55579889,115.69345466)(732.68578857,115.69345998)
\lineto(733.07578857,115.69345998)
\lineto(734.57578857,115.69345998)
\lineto(740.96578857,115.69345998)
\lineto(742.13578857,115.69345998)
\lineto(742.45078857,115.69345998)
\curveto(742.55078889,115.70345465)(742.63078881,115.68845466)(742.69078857,115.64845998)
\curveto(742.77078867,115.59845475)(742.82078862,115.52345483)(742.84078857,115.42345998)
\curveto(742.85078859,115.33345502)(742.85578859,115.22345513)(742.85578857,115.09345998)
\lineto(742.85578857,114.86845998)
\curveto(742.83578861,114.78845556)(742.82078862,114.71845563)(742.81078857,114.65845998)
\curveto(742.79078865,114.59845575)(742.75078869,114.5484558)(742.69078857,114.50845998)
\curveto(742.63078881,114.46845588)(742.55578889,114.4484559)(742.46578857,114.44845998)
\lineto(742.16578857,114.44845998)
\lineto(741.07078857,114.44845998)
\lineto(735.73078857,114.44845998)
\curveto(735.6407958,114.42845592)(735.56579588,114.41345594)(735.50578857,114.40345998)
\curveto(735.43579601,114.40345595)(735.37579607,114.37345598)(735.32578857,114.31345998)
\curveto(735.27579617,114.24345611)(735.25079619,114.1534562)(735.25078857,114.04345998)
\curveto(735.2407962,113.94345641)(735.23579621,113.83345652)(735.23578857,113.71345998)
\lineto(735.23578857,112.57345998)
\lineto(735.23578857,112.07845998)
\curveto(735.22579622,111.91845843)(735.16579628,111.80845854)(735.05578857,111.74845998)
\curveto(735.02579642,111.72845862)(734.99579645,111.71845863)(734.96578857,111.71845998)
\curveto(734.92579652,111.71845863)(734.88079656,111.71345864)(734.83078857,111.70345998)
\curveto(734.71079673,111.68345867)(734.60079684,111.68845866)(734.50078857,111.71845998)
\curveto(734.40079704,111.75845859)(734.33079711,111.81345854)(734.29078857,111.88345998)
\curveto(734.2407972,111.96345839)(734.21579723,112.08345827)(734.21578857,112.24345998)
\curveto(734.21579723,112.40345795)(734.20079724,112.53845781)(734.17078857,112.64845998)
\curveto(734.16079728,112.69845765)(734.15579729,112.7534576)(734.15578857,112.81345998)
\curveto(734.1457973,112.87345748)(734.13079731,112.93345742)(734.11078857,112.99345998)
\curveto(734.06079738,113.14345721)(734.01079743,113.28845706)(733.96078857,113.42845998)
\curveto(733.90079754,113.56845678)(733.83079761,113.70345665)(733.75078857,113.83345998)
\curveto(733.66079778,113.97345638)(733.55579789,114.09345626)(733.43578857,114.19345998)
\curveto(733.31579813,114.29345606)(733.18579826,114.38845596)(733.04578857,114.47845998)
\curveto(732.9457985,114.53845581)(732.83579861,114.58345577)(732.71578857,114.61345998)
\curveto(732.59579885,114.6534557)(732.49079895,114.70345565)(732.40078857,114.76345998)
\curveto(732.3407991,114.81345554)(732.30079914,114.88345547)(732.28078857,114.97345998)
\curveto(732.27079917,114.99345536)(732.26579918,115.01845533)(732.26578857,115.04845998)
\curveto(732.26579918,115.07845527)(732.26079918,115.10345525)(732.25078857,115.12345998)
}
}
{
\newrgbcolor{curcolor}{0 0 0}
\pscustom[linestyle=none,fillstyle=solid,fillcolor=curcolor]
{
\newpath
\moveto(753.08710449,37.28705373)
\curveto(753.08711519,37.35704805)(753.08711519,37.43704797)(753.08710449,37.52705373)
\curveto(753.0771152,37.61704779)(753.0771152,37.70204771)(753.08710449,37.78205373)
\curveto(753.08711519,37.87204754)(753.09711518,37.95204746)(753.11710449,38.02205373)
\curveto(753.13711514,38.10204731)(753.16711511,38.15704725)(753.20710449,38.18705373)
\curveto(753.25711502,38.21704719)(753.33211494,38.23704717)(753.43210449,38.24705373)
\curveto(753.52211475,38.26704714)(753.62711465,38.27704713)(753.74710449,38.27705373)
\curveto(753.85711442,38.28704712)(753.9721143,38.28704712)(754.09210449,38.27705373)
\lineto(754.39210449,38.27705373)
\lineto(757.40710449,38.27705373)
\lineto(760.30210449,38.27705373)
\curveto(760.63210764,38.27704713)(760.95710732,38.27204714)(761.27710449,38.26205373)
\curveto(761.58710669,38.26204715)(761.86710641,38.22204719)(762.11710449,38.14205373)
\curveto(762.46710581,38.02204739)(762.76210551,37.86704754)(763.00210449,37.67705373)
\curveto(763.23210504,37.48704792)(763.43210484,37.24704816)(763.60210449,36.95705373)
\curveto(763.65210462,36.89704851)(763.68710459,36.83204858)(763.70710449,36.76205373)
\curveto(763.72710455,36.70204871)(763.75210452,36.63204878)(763.78210449,36.55205373)
\curveto(763.83210444,36.43204898)(763.86710441,36.30204911)(763.88710449,36.16205373)
\curveto(763.91710436,36.03204938)(763.94710433,35.89704951)(763.97710449,35.75705373)
\curveto(763.99710428,35.7070497)(764.00210427,35.65704975)(763.99210449,35.60705373)
\curveto(763.98210429,35.55704985)(763.98210429,35.50204991)(763.99210449,35.44205373)
\curveto(764.00210427,35.42204999)(764.00210427,35.39705001)(763.99210449,35.36705373)
\curveto(763.99210428,35.33705007)(763.99710428,35.3120501)(764.00710449,35.29205373)
\curveto(764.01710426,35.25205016)(764.02210425,35.19705021)(764.02210449,35.12705373)
\curveto(764.02210425,35.05705035)(764.01710426,35.00205041)(764.00710449,34.96205373)
\curveto(763.99710428,34.9120505)(763.99710428,34.85705055)(764.00710449,34.79705373)
\curveto(764.01710426,34.73705067)(764.01210426,34.68205073)(763.99210449,34.63205373)
\curveto(763.96210431,34.50205091)(763.94210433,34.37705103)(763.93210449,34.25705373)
\curveto(763.92210435,34.13705127)(763.89710438,34.02205139)(763.85710449,33.91205373)
\curveto(763.73710454,33.54205187)(763.56710471,33.22205219)(763.34710449,32.95205373)
\curveto(763.12710515,32.68205273)(762.84710543,32.47205294)(762.50710449,32.32205373)
\curveto(762.38710589,32.27205314)(762.26210601,32.22705318)(762.13210449,32.18705373)
\curveto(762.00210627,32.15705325)(761.86710641,32.12205329)(761.72710449,32.08205373)
\curveto(761.6771066,32.07205334)(761.63710664,32.06705334)(761.60710449,32.06705373)
\curveto(761.56710671,32.06705334)(761.52210675,32.06205335)(761.47210449,32.05205373)
\curveto(761.44210683,32.04205337)(761.40710687,32.03705337)(761.36710449,32.03705373)
\curveto(761.31710696,32.03705337)(761.277107,32.03205338)(761.24710449,32.02205373)
\lineto(761.08210449,32.02205373)
\curveto(761.00210727,32.00205341)(760.90210737,31.99705341)(760.78210449,32.00705373)
\curveto(760.65210762,32.01705339)(760.56210771,32.03205338)(760.51210449,32.05205373)
\curveto(760.42210785,32.07205334)(760.35710792,32.12705328)(760.31710449,32.21705373)
\curveto(760.29710798,32.24705316)(760.29210798,32.27705313)(760.30210449,32.30705373)
\curveto(760.30210797,32.33705307)(760.29710798,32.37705303)(760.28710449,32.42705373)
\curveto(760.277108,32.46705294)(760.272108,32.5070529)(760.27210449,32.54705373)
\lineto(760.27210449,32.69705373)
\curveto(760.272108,32.81705259)(760.277108,32.93705247)(760.28710449,33.05705373)
\curveto(760.28710799,33.18705222)(760.32210795,33.27705213)(760.39210449,33.32705373)
\curveto(760.45210782,33.36705204)(760.51210776,33.38705202)(760.57210449,33.38705373)
\curveto(760.63210764,33.38705202)(760.70210757,33.39705201)(760.78210449,33.41705373)
\curveto(760.81210746,33.42705198)(760.84710743,33.42705198)(760.88710449,33.41705373)
\curveto(760.91710736,33.41705199)(760.94210733,33.42205199)(760.96210449,33.43205373)
\lineto(761.17210449,33.43205373)
\curveto(761.22210705,33.45205196)(761.272107,33.45705195)(761.32210449,33.44705373)
\curveto(761.36210691,33.44705196)(761.40710687,33.45705195)(761.45710449,33.47705373)
\curveto(761.58710669,33.5070519)(761.71210656,33.53705187)(761.83210449,33.56705373)
\curveto(761.94210633,33.59705181)(762.04710623,33.64205177)(762.14710449,33.70205373)
\curveto(762.43710584,33.87205154)(762.64210563,34.14205127)(762.76210449,34.51205373)
\curveto(762.78210549,34.56205085)(762.79710548,34.6120508)(762.80710449,34.66205373)
\curveto(762.80710547,34.72205069)(762.81710546,34.77705063)(762.83710449,34.82705373)
\lineto(762.83710449,34.90205373)
\curveto(762.84710543,34.97205044)(762.85710542,35.06705034)(762.86710449,35.18705373)
\curveto(762.86710541,35.31705009)(762.85710542,35.41704999)(762.83710449,35.48705373)
\curveto(762.81710546,35.55704985)(762.80210547,35.62704978)(762.79210449,35.69705373)
\curveto(762.7721055,35.77704963)(762.75210552,35.84704956)(762.73210449,35.90705373)
\curveto(762.5721057,36.28704912)(762.29710598,36.56204885)(761.90710449,36.73205373)
\curveto(761.7771065,36.78204863)(761.62210665,36.81704859)(761.44210449,36.83705373)
\curveto(761.26210701,36.86704854)(761.0771072,36.88204853)(760.88710449,36.88205373)
\curveto(760.68710759,36.89204852)(760.48710779,36.89204852)(760.28710449,36.88205373)
\lineto(759.71710449,36.88205373)
\lineto(755.47210449,36.88205373)
\lineto(753.92710449,36.88205373)
\curveto(753.81711446,36.88204853)(753.69711458,36.87704853)(753.56710449,36.86705373)
\curveto(753.43711484,36.85704855)(753.33211494,36.87704853)(753.25210449,36.92705373)
\curveto(753.18211509,36.98704842)(753.13211514,37.06704834)(753.10210449,37.16705373)
\curveto(753.10211517,37.18704822)(753.10211517,37.2070482)(753.10210449,37.22705373)
\curveto(753.10211517,37.24704816)(753.09711518,37.26704814)(753.08710449,37.28705373)
}
}
{
\newrgbcolor{curcolor}{0 0 0}
\pscustom[linestyle=none,fillstyle=solid,fillcolor=curcolor]
{
\newpath
\moveto(756.04210449,40.82072561)
\lineto(756.04210449,41.25572561)
\curveto(756.04211223,41.40572364)(756.08211219,41.51072354)(756.16210449,41.57072561)
\curveto(756.24211203,41.62072343)(756.34211193,41.6457234)(756.46210449,41.64572561)
\curveto(756.58211169,41.65572339)(756.70211157,41.66072339)(756.82210449,41.66072561)
\lineto(758.24710449,41.66072561)
\lineto(760.51210449,41.66072561)
\lineto(761.20210449,41.66072561)
\curveto(761.43210684,41.66072339)(761.63210664,41.68572336)(761.80210449,41.73572561)
\curveto(762.25210602,41.89572315)(762.56710571,42.19572285)(762.74710449,42.63572561)
\curveto(762.83710544,42.85572219)(762.8721054,43.12072193)(762.85210449,43.43072561)
\curveto(762.82210545,43.74072131)(762.76710551,43.99072106)(762.68710449,44.18072561)
\curveto(762.54710573,44.51072054)(762.3721059,44.77072028)(762.16210449,44.96072561)
\curveto(761.94210633,45.16071989)(761.65710662,45.31571973)(761.30710449,45.42572561)
\curveto(761.22710705,45.45571959)(761.14710713,45.47571957)(761.06710449,45.48572561)
\curveto(760.98710729,45.49571955)(760.90210737,45.51071954)(760.81210449,45.53072561)
\curveto(760.76210751,45.54071951)(760.71710756,45.54071951)(760.67710449,45.53072561)
\curveto(760.63710764,45.53071952)(760.59210768,45.54071951)(760.54210449,45.56072561)
\lineto(760.22710449,45.56072561)
\curveto(760.14710813,45.58071947)(760.05710822,45.58571946)(759.95710449,45.57572561)
\curveto(759.84710843,45.56571948)(759.74710853,45.56071949)(759.65710449,45.56072561)
\lineto(758.48710449,45.56072561)
\lineto(756.89710449,45.56072561)
\curveto(756.7771115,45.56071949)(756.65211162,45.55571949)(756.52210449,45.54572561)
\curveto(756.38211189,45.5457195)(756.272112,45.57071948)(756.19210449,45.62072561)
\curveto(756.14211213,45.66071939)(756.11211216,45.70571934)(756.10210449,45.75572561)
\curveto(756.08211219,45.81571923)(756.06211221,45.88571916)(756.04210449,45.96572561)
\lineto(756.04210449,46.19072561)
\curveto(756.04211223,46.31071874)(756.04711223,46.41571863)(756.05710449,46.50572561)
\curveto(756.06711221,46.60571844)(756.11211216,46.68071837)(756.19210449,46.73072561)
\curveto(756.24211203,46.78071827)(756.31711196,46.80571824)(756.41710449,46.80572561)
\lineto(756.70210449,46.80572561)
\lineto(757.72210449,46.80572561)
\lineto(761.75710449,46.80572561)
\lineto(763.10710449,46.80572561)
\curveto(763.22710505,46.80571824)(763.34210493,46.80071825)(763.45210449,46.79072561)
\curveto(763.55210472,46.79071826)(763.62710465,46.75571829)(763.67710449,46.68572561)
\curveto(763.70710457,46.6457184)(763.73210454,46.58571846)(763.75210449,46.50572561)
\curveto(763.76210451,46.42571862)(763.7721045,46.33571871)(763.78210449,46.23572561)
\curveto(763.78210449,46.1457189)(763.7771045,46.05571899)(763.76710449,45.96572561)
\curveto(763.75710452,45.88571916)(763.73710454,45.82571922)(763.70710449,45.78572561)
\curveto(763.66710461,45.73571931)(763.60210467,45.69071936)(763.51210449,45.65072561)
\curveto(763.4721048,45.64071941)(763.41710486,45.63071942)(763.34710449,45.62072561)
\curveto(763.277105,45.62071943)(763.21210506,45.61571943)(763.15210449,45.60572561)
\curveto(763.08210519,45.59571945)(763.02710525,45.57571947)(762.98710449,45.54572561)
\curveto(762.94710533,45.51571953)(762.93210534,45.47071958)(762.94210449,45.41072561)
\curveto(762.96210531,45.33071972)(763.02210525,45.2507198)(763.12210449,45.17072561)
\curveto(763.21210506,45.09071996)(763.28210499,45.01572003)(763.33210449,44.94572561)
\curveto(763.49210478,44.72572032)(763.63210464,44.47572057)(763.75210449,44.19572561)
\curveto(763.80210447,44.08572096)(763.83210444,43.97072108)(763.84210449,43.85072561)
\curveto(763.86210441,43.74072131)(763.88710439,43.62572142)(763.91710449,43.50572561)
\curveto(763.92710435,43.45572159)(763.92710435,43.40072165)(763.91710449,43.34072561)
\curveto(763.90710437,43.29072176)(763.91210436,43.24072181)(763.93210449,43.19072561)
\curveto(763.95210432,43.09072196)(763.95210432,43.00072205)(763.93210449,42.92072561)
\lineto(763.93210449,42.77072561)
\curveto(763.91210436,42.72072233)(763.90210437,42.66072239)(763.90210449,42.59072561)
\curveto(763.90210437,42.53072252)(763.89710438,42.47572257)(763.88710449,42.42572561)
\curveto(763.86710441,42.38572266)(763.85710442,42.3457227)(763.85710449,42.30572561)
\curveto(763.86710441,42.27572277)(763.86210441,42.23572281)(763.84210449,42.18572561)
\lineto(763.78210449,41.94572561)
\curveto(763.76210451,41.87572317)(763.73210454,41.80072325)(763.69210449,41.72072561)
\curveto(763.58210469,41.46072359)(763.43710484,41.24072381)(763.25710449,41.06072561)
\curveto(763.06710521,40.89072416)(762.84210543,40.7507243)(762.58210449,40.64072561)
\curveto(762.49210578,40.60072445)(762.40210587,40.57072448)(762.31210449,40.55072561)
\lineto(762.01210449,40.49072561)
\curveto(761.95210632,40.47072458)(761.89710638,40.46072459)(761.84710449,40.46072561)
\curveto(761.78710649,40.47072458)(761.72210655,40.46572458)(761.65210449,40.44572561)
\curveto(761.63210664,40.43572461)(761.60710667,40.43072462)(761.57710449,40.43072561)
\curveto(761.53710674,40.43072462)(761.50210677,40.42572462)(761.47210449,40.41572561)
\lineto(761.32210449,40.41572561)
\curveto(761.28210699,40.40572464)(761.23710704,40.40072465)(761.18710449,40.40072561)
\curveto(761.12710715,40.41072464)(761.0721072,40.41572463)(761.02210449,40.41572561)
\lineto(760.42210449,40.41572561)
\lineto(757.66210449,40.41572561)
\lineto(756.70210449,40.41572561)
\lineto(756.43210449,40.41572561)
\curveto(756.34211193,40.41572463)(756.26711201,40.43572461)(756.20710449,40.47572561)
\curveto(756.13711214,40.51572453)(756.08711219,40.59072446)(756.05710449,40.70072561)
\curveto(756.04711223,40.72072433)(756.04711223,40.74072431)(756.05710449,40.76072561)
\curveto(756.05711222,40.78072427)(756.05211222,40.80072425)(756.04210449,40.82072561)
}
}
{
\newrgbcolor{curcolor}{0 0 0}
\pscustom[linestyle=none,fillstyle=solid,fillcolor=curcolor]
{
\newpath
\moveto(755.89210449,52.39533498)
\curveto(755.8721124,53.02532975)(755.95711232,53.53032924)(756.14710449,53.91033498)
\curveto(756.33711194,54.29032848)(756.62211165,54.59532818)(757.00210449,54.82533498)
\curveto(757.10211117,54.88532789)(757.21211106,54.93032784)(757.33210449,54.96033498)
\curveto(757.44211083,55.00032777)(757.55711072,55.03532774)(757.67710449,55.06533498)
\curveto(757.86711041,55.11532766)(758.0721102,55.14532763)(758.29210449,55.15533498)
\curveto(758.51210976,55.16532761)(758.73710954,55.1703276)(758.96710449,55.17033498)
\lineto(760.57210449,55.17033498)
\lineto(762.91210449,55.17033498)
\curveto(763.08210519,55.1703276)(763.25210502,55.16532761)(763.42210449,55.15533498)
\curveto(763.59210468,55.15532762)(763.70210457,55.09032768)(763.75210449,54.96033498)
\curveto(763.7721045,54.91032786)(763.78210449,54.85532792)(763.78210449,54.79533498)
\curveto(763.79210448,54.74532803)(763.79710448,54.69032808)(763.79710449,54.63033498)
\curveto(763.79710448,54.50032827)(763.79210448,54.3753284)(763.78210449,54.25533498)
\curveto(763.78210449,54.13532864)(763.74210453,54.05032872)(763.66210449,54.00033498)
\curveto(763.59210468,53.95032882)(763.50210477,53.92532885)(763.39210449,53.92533498)
\lineto(763.06210449,53.92533498)
\lineto(761.77210449,53.92533498)
\lineto(759.32710449,53.92533498)
\curveto(759.05710922,53.92532885)(758.79210948,53.92032885)(758.53210449,53.91033498)
\curveto(758.26211001,53.90032887)(758.03211024,53.85532892)(757.84210449,53.77533498)
\curveto(757.64211063,53.69532908)(757.48211079,53.5753292)(757.36210449,53.41533498)
\curveto(757.23211104,53.25532952)(757.13211114,53.0703297)(757.06210449,52.86033498)
\curveto(757.04211123,52.80032997)(757.03211124,52.73533004)(757.03210449,52.66533498)
\curveto(757.02211125,52.60533017)(757.00711127,52.54533023)(756.98710449,52.48533498)
\curveto(756.9771113,52.43533034)(756.9771113,52.35533042)(756.98710449,52.24533498)
\curveto(756.98711129,52.14533063)(756.99211128,52.0753307)(757.00210449,52.03533498)
\curveto(757.02211125,51.99533078)(757.03211124,51.96033081)(757.03210449,51.93033498)
\curveto(757.02211125,51.90033087)(757.02211125,51.86533091)(757.03210449,51.82533498)
\curveto(757.06211121,51.69533108)(757.09711118,51.5703312)(757.13710449,51.45033498)
\curveto(757.16711111,51.34033143)(757.21211106,51.23533154)(757.27210449,51.13533498)
\curveto(757.29211098,51.09533168)(757.31211096,51.06033171)(757.33210449,51.03033498)
\curveto(757.35211092,51.00033177)(757.3721109,50.96533181)(757.39210449,50.92533498)
\curveto(757.64211063,50.5753322)(758.01711026,50.32033245)(758.51710449,50.16033498)
\curveto(758.59710968,50.13033264)(758.68210959,50.11033266)(758.77210449,50.10033498)
\curveto(758.85210942,50.09033268)(758.93210934,50.0753327)(759.01210449,50.05533498)
\curveto(759.06210921,50.03533274)(759.11210916,50.03033274)(759.16210449,50.04033498)
\curveto(759.20210907,50.05033272)(759.24210903,50.04533273)(759.28210449,50.02533498)
\lineto(759.59710449,50.02533498)
\curveto(759.62710865,50.01533276)(759.66210861,50.01033276)(759.70210449,50.01033498)
\curveto(759.74210853,50.02033275)(759.78710849,50.02533275)(759.83710449,50.02533498)
\lineto(760.28710449,50.02533498)
\lineto(761.72710449,50.02533498)
\lineto(763.04710449,50.02533498)
\lineto(763.39210449,50.02533498)
\curveto(763.50210477,50.02533275)(763.59210468,50.00033277)(763.66210449,49.95033498)
\curveto(763.74210453,49.90033287)(763.78210449,49.81033296)(763.78210449,49.68033498)
\curveto(763.79210448,49.56033321)(763.79710448,49.43533334)(763.79710449,49.30533498)
\curveto(763.79710448,49.22533355)(763.79210448,49.15033362)(763.78210449,49.08033498)
\curveto(763.7721045,49.01033376)(763.74710453,48.95033382)(763.70710449,48.90033498)
\curveto(763.65710462,48.82033395)(763.56210471,48.78033399)(763.42210449,48.78033498)
\lineto(763.01710449,48.78033498)
\lineto(761.24710449,48.78033498)
\lineto(757.61710449,48.78033498)
\lineto(756.70210449,48.78033498)
\lineto(756.43210449,48.78033498)
\curveto(756.34211193,48.78033399)(756.272112,48.80033397)(756.22210449,48.84033498)
\curveto(756.16211211,48.8703339)(756.12211215,48.92033385)(756.10210449,48.99033498)
\curveto(756.09211218,49.03033374)(756.08211219,49.08533369)(756.07210449,49.15533498)
\curveto(756.06211221,49.23533354)(756.05711222,49.31533346)(756.05710449,49.39533498)
\curveto(756.05711222,49.4753333)(756.06211221,49.55033322)(756.07210449,49.62033498)
\curveto(756.08211219,49.70033307)(756.09711218,49.75533302)(756.11710449,49.78533498)
\curveto(756.18711209,49.89533288)(756.277112,49.94533283)(756.38710449,49.93533498)
\curveto(756.48711179,49.92533285)(756.60211167,49.94033283)(756.73210449,49.98033498)
\curveto(756.79211148,50.00033277)(756.84211143,50.04033273)(756.88210449,50.10033498)
\curveto(756.89211138,50.22033255)(756.84711143,50.31533246)(756.74710449,50.38533498)
\curveto(756.64711163,50.46533231)(756.56711171,50.54533223)(756.50710449,50.62533498)
\curveto(756.40711187,50.76533201)(756.31711196,50.90533187)(756.23710449,51.04533498)
\curveto(756.14711213,51.19533158)(756.0721122,51.36533141)(756.01210449,51.55533498)
\curveto(755.98211229,51.63533114)(755.96211231,51.72033105)(755.95210449,51.81033498)
\curveto(755.94211233,51.91033086)(755.92711235,52.00533077)(755.90710449,52.09533498)
\curveto(755.89711238,52.14533063)(755.89211238,52.19533058)(755.89210449,52.24533498)
\lineto(755.89210449,52.39533498)
}
}
{
\newrgbcolor{curcolor}{0 0 0}
\pscustom[linestyle=none,fillstyle=solid,fillcolor=curcolor]
{
}
}
{
\newrgbcolor{curcolor}{0 0 0}
\pscustom[linestyle=none,fillstyle=solid,fillcolor=curcolor]
{
\newpath
\moveto(753.16210449,64.24510061)
\curveto(753.15211512,64.93509597)(753.272115,65.53509537)(753.52210449,66.04510061)
\curveto(753.7721145,66.56509434)(754.10711417,66.96009395)(754.52710449,67.23010061)
\curveto(754.60711367,67.28009363)(754.69711358,67.32509358)(754.79710449,67.36510061)
\curveto(754.88711339,67.4050935)(754.98211329,67.45009346)(755.08210449,67.50010061)
\curveto(755.18211309,67.54009337)(755.28211299,67.57009334)(755.38210449,67.59010061)
\curveto(755.48211279,67.6100933)(755.58711269,67.63009328)(755.69710449,67.65010061)
\curveto(755.74711253,67.67009324)(755.79211248,67.67509323)(755.83210449,67.66510061)
\curveto(755.8721124,67.65509325)(755.91711236,67.66009325)(755.96710449,67.68010061)
\curveto(756.01711226,67.69009322)(756.10211217,67.69509321)(756.22210449,67.69510061)
\curveto(756.33211194,67.69509321)(756.41711186,67.69009322)(756.47710449,67.68010061)
\curveto(756.53711174,67.66009325)(756.59711168,67.65009326)(756.65710449,67.65010061)
\curveto(756.71711156,67.66009325)(756.7771115,67.65509325)(756.83710449,67.63510061)
\curveto(756.9771113,67.59509331)(757.11211116,67.56009335)(757.24210449,67.53010061)
\curveto(757.3721109,67.50009341)(757.49711078,67.46009345)(757.61710449,67.41010061)
\curveto(757.75711052,67.35009356)(757.88211039,67.28009363)(757.99210449,67.20010061)
\curveto(758.10211017,67.13009378)(758.21211006,67.05509385)(758.32210449,66.97510061)
\lineto(758.38210449,66.91510061)
\curveto(758.40210987,66.905094)(758.42210985,66.89009402)(758.44210449,66.87010061)
\curveto(758.60210967,66.75009416)(758.74710953,66.61509429)(758.87710449,66.46510061)
\curveto(759.00710927,66.31509459)(759.13210914,66.15509475)(759.25210449,65.98510061)
\curveto(759.4721088,65.67509523)(759.6771086,65.38009553)(759.86710449,65.10010061)
\curveto(760.00710827,64.87009604)(760.14210813,64.64009627)(760.27210449,64.41010061)
\curveto(760.40210787,64.19009672)(760.53710774,63.97009694)(760.67710449,63.75010061)
\curveto(760.84710743,63.50009741)(761.02710725,63.26009765)(761.21710449,63.03010061)
\curveto(761.40710687,62.8100981)(761.63210664,62.62009829)(761.89210449,62.46010061)
\curveto(761.95210632,62.42009849)(762.01210626,62.38509852)(762.07210449,62.35510061)
\curveto(762.12210615,62.32509858)(762.18710609,62.29509861)(762.26710449,62.26510061)
\curveto(762.33710594,62.24509866)(762.39710588,62.24009867)(762.44710449,62.25010061)
\curveto(762.51710576,62.27009864)(762.5721057,62.3050986)(762.61210449,62.35510061)
\curveto(762.64210563,62.4050985)(762.66210561,62.46509844)(762.67210449,62.53510061)
\lineto(762.67210449,62.77510061)
\lineto(762.67210449,63.52510061)
\lineto(762.67210449,66.33010061)
\lineto(762.67210449,66.99010061)
\curveto(762.6721056,67.08009383)(762.6771056,67.16509374)(762.68710449,67.24510061)
\curveto(762.68710559,67.32509358)(762.70710557,67.39009352)(762.74710449,67.44010061)
\curveto(762.78710549,67.49009342)(762.86210541,67.53009338)(762.97210449,67.56010061)
\curveto(763.0721052,67.60009331)(763.1721051,67.6100933)(763.27210449,67.59010061)
\lineto(763.40710449,67.59010061)
\curveto(763.4771048,67.57009334)(763.53710474,67.55009336)(763.58710449,67.53010061)
\curveto(763.63710464,67.5100934)(763.6771046,67.47509343)(763.70710449,67.42510061)
\curveto(763.74710453,67.37509353)(763.76710451,67.3050936)(763.76710449,67.21510061)
\lineto(763.76710449,66.94510061)
\lineto(763.76710449,66.04510061)
\lineto(763.76710449,62.53510061)
\lineto(763.76710449,61.47010061)
\curveto(763.76710451,61.39009952)(763.7721045,61.30009961)(763.78210449,61.20010061)
\curveto(763.78210449,61.10009981)(763.7721045,61.01509989)(763.75210449,60.94510061)
\curveto(763.68210459,60.73510017)(763.50210477,60.67010024)(763.21210449,60.75010061)
\curveto(763.1721051,60.76010015)(763.13710514,60.76010015)(763.10710449,60.75010061)
\curveto(763.06710521,60.75010016)(763.02210525,60.76010015)(762.97210449,60.78010061)
\curveto(762.89210538,60.80010011)(762.80710547,60.82010009)(762.71710449,60.84010061)
\curveto(762.62710565,60.86010005)(762.54210573,60.88510002)(762.46210449,60.91510061)
\curveto(761.9721063,61.07509983)(761.55710672,61.27509963)(761.21710449,61.51510061)
\curveto(760.96710731,61.69509921)(760.74210753,61.90009901)(760.54210449,62.13010061)
\curveto(760.33210794,62.36009855)(760.13710814,62.60009831)(759.95710449,62.85010061)
\curveto(759.7771085,63.1100978)(759.60710867,63.37509753)(759.44710449,63.64510061)
\curveto(759.277109,63.92509698)(759.10210917,64.19509671)(758.92210449,64.45510061)
\curveto(758.84210943,64.56509634)(758.76710951,64.67009624)(758.69710449,64.77010061)
\curveto(758.62710965,64.88009603)(758.55210972,64.99009592)(758.47210449,65.10010061)
\curveto(758.44210983,65.14009577)(758.41210986,65.17509573)(758.38210449,65.20510061)
\curveto(758.34210993,65.24509566)(758.31210996,65.28509562)(758.29210449,65.32510061)
\curveto(758.18211009,65.46509544)(758.05711022,65.59009532)(757.91710449,65.70010061)
\curveto(757.88711039,65.72009519)(757.86211041,65.74509516)(757.84210449,65.77510061)
\curveto(757.81211046,65.8050951)(757.78211049,65.83009508)(757.75210449,65.85010061)
\curveto(757.65211062,65.93009498)(757.55211072,65.99509491)(757.45210449,66.04510061)
\curveto(757.35211092,66.1050948)(757.24211103,66.16009475)(757.12210449,66.21010061)
\curveto(757.05211122,66.24009467)(756.9771113,66.26009465)(756.89710449,66.27010061)
\lineto(756.65710449,66.33010061)
\lineto(756.56710449,66.33010061)
\curveto(756.53711174,66.34009457)(756.50711177,66.34509456)(756.47710449,66.34510061)
\curveto(756.40711187,66.36509454)(756.31211196,66.37009454)(756.19210449,66.36010061)
\curveto(756.06211221,66.36009455)(755.96211231,66.35009456)(755.89210449,66.33010061)
\curveto(755.81211246,66.3100946)(755.73711254,66.29009462)(755.66710449,66.27010061)
\curveto(755.58711269,66.26009465)(755.50711277,66.24009467)(755.42710449,66.21010061)
\curveto(755.18711309,66.10009481)(754.98711329,65.95009496)(754.82710449,65.76010061)
\curveto(754.65711362,65.58009533)(754.51711376,65.36009555)(754.40710449,65.10010061)
\curveto(754.38711389,65.03009588)(754.3721139,64.96009595)(754.36210449,64.89010061)
\curveto(754.34211393,64.82009609)(754.32211395,64.74509616)(754.30210449,64.66510061)
\curveto(754.28211399,64.58509632)(754.272114,64.47509643)(754.27210449,64.33510061)
\curveto(754.272114,64.2050967)(754.28211399,64.10009681)(754.30210449,64.02010061)
\curveto(754.31211396,63.96009695)(754.31711396,63.905097)(754.31710449,63.85510061)
\curveto(754.31711396,63.8050971)(754.32711395,63.75509715)(754.34710449,63.70510061)
\curveto(754.38711389,63.6050973)(754.42711385,63.5100974)(754.46710449,63.42010061)
\curveto(754.50711377,63.34009757)(754.55211372,63.26009765)(754.60210449,63.18010061)
\curveto(754.62211365,63.15009776)(754.64711363,63.12009779)(754.67710449,63.09010061)
\curveto(754.70711357,63.07009784)(754.73211354,63.04509786)(754.75210449,63.01510061)
\lineto(754.82710449,62.94010061)
\curveto(754.84711343,62.910098)(754.86711341,62.88509802)(754.88710449,62.86510061)
\lineto(755.09710449,62.71510061)
\curveto(755.15711312,62.67509823)(755.22211305,62.63009828)(755.29210449,62.58010061)
\curveto(755.38211289,62.52009839)(755.48711279,62.47009844)(755.60710449,62.43010061)
\curveto(755.71711256,62.40009851)(755.82711245,62.36509854)(755.93710449,62.32510061)
\curveto(756.04711223,62.28509862)(756.19211208,62.26009865)(756.37210449,62.25010061)
\curveto(756.54211173,62.24009867)(756.66711161,62.2100987)(756.74710449,62.16010061)
\curveto(756.82711145,62.1100988)(756.8721114,62.03509887)(756.88210449,61.93510061)
\curveto(756.89211138,61.83509907)(756.89711138,61.72509918)(756.89710449,61.60510061)
\curveto(756.89711138,61.56509934)(756.90211137,61.52509938)(756.91210449,61.48510061)
\curveto(756.91211136,61.44509946)(756.90711137,61.4100995)(756.89710449,61.38010061)
\curveto(756.8771114,61.33009958)(756.86711141,61.28009963)(756.86710449,61.23010061)
\curveto(756.86711141,61.19009972)(756.85711142,61.15009976)(756.83710449,61.11010061)
\curveto(756.7771115,61.02009989)(756.64211163,60.97509993)(756.43210449,60.97510061)
\lineto(756.31210449,60.97510061)
\curveto(756.25211202,60.98509992)(756.19211208,60.99009992)(756.13210449,60.99010061)
\curveto(756.06211221,61.00009991)(755.99711228,61.0100999)(755.93710449,61.02010061)
\curveto(755.82711245,61.04009987)(755.72711255,61.06009985)(755.63710449,61.08010061)
\curveto(755.53711274,61.10009981)(755.44211283,61.13009978)(755.35210449,61.17010061)
\curveto(755.28211299,61.19009972)(755.22211305,61.2100997)(755.17210449,61.23010061)
\lineto(754.99210449,61.29010061)
\curveto(754.73211354,61.4100995)(754.48711379,61.56509934)(754.25710449,61.75510061)
\curveto(754.02711425,61.95509895)(753.84211443,62.17009874)(753.70210449,62.40010061)
\curveto(753.62211465,62.5100984)(753.55711472,62.62509828)(753.50710449,62.74510061)
\lineto(753.35710449,63.13510061)
\curveto(753.30711497,63.24509766)(753.277115,63.36009755)(753.26710449,63.48010061)
\curveto(753.24711503,63.60009731)(753.22211505,63.72509718)(753.19210449,63.85510061)
\curveto(753.19211508,63.92509698)(753.19211508,63.99009692)(753.19210449,64.05010061)
\curveto(753.18211509,64.1100968)(753.1721151,64.17509673)(753.16210449,64.24510061)
}
}
{
\newrgbcolor{curcolor}{0 0 0}
\pscustom[linestyle=none,fillstyle=solid,fillcolor=curcolor]
{
\newpath
\moveto(753.16210449,72.59470998)
\curveto(753.15211512,73.28470535)(753.272115,73.88470475)(753.52210449,74.39470998)
\curveto(753.7721145,74.91470372)(754.10711417,75.30970332)(754.52710449,75.57970998)
\curveto(754.60711367,75.629703)(754.69711358,75.67470296)(754.79710449,75.71470998)
\curveto(754.88711339,75.75470288)(754.98211329,75.79970283)(755.08210449,75.84970998)
\curveto(755.18211309,75.88970274)(755.28211299,75.91970271)(755.38210449,75.93970998)
\curveto(755.48211279,75.95970267)(755.58711269,75.97970265)(755.69710449,75.99970998)
\curveto(755.74711253,76.01970261)(755.79211248,76.02470261)(755.83210449,76.01470998)
\curveto(755.8721124,76.00470263)(755.91711236,76.00970262)(755.96710449,76.02970998)
\curveto(756.01711226,76.03970259)(756.10211217,76.04470259)(756.22210449,76.04470998)
\curveto(756.33211194,76.04470259)(756.41711186,76.03970259)(756.47710449,76.02970998)
\curveto(756.53711174,76.00970262)(756.59711168,75.99970263)(756.65710449,75.99970998)
\curveto(756.71711156,76.00970262)(756.7771115,76.00470263)(756.83710449,75.98470998)
\curveto(756.9771113,75.94470269)(757.11211116,75.90970272)(757.24210449,75.87970998)
\curveto(757.3721109,75.84970278)(757.49711078,75.80970282)(757.61710449,75.75970998)
\curveto(757.75711052,75.69970293)(757.88211039,75.629703)(757.99210449,75.54970998)
\curveto(758.10211017,75.47970315)(758.21211006,75.40470323)(758.32210449,75.32470998)
\lineto(758.38210449,75.26470998)
\curveto(758.40210987,75.25470338)(758.42210985,75.23970339)(758.44210449,75.21970998)
\curveto(758.60210967,75.09970353)(758.74710953,74.96470367)(758.87710449,74.81470998)
\curveto(759.00710927,74.66470397)(759.13210914,74.50470413)(759.25210449,74.33470998)
\curveto(759.4721088,74.02470461)(759.6771086,73.7297049)(759.86710449,73.44970998)
\curveto(760.00710827,73.21970541)(760.14210813,72.98970564)(760.27210449,72.75970998)
\curveto(760.40210787,72.53970609)(760.53710774,72.31970631)(760.67710449,72.09970998)
\curveto(760.84710743,71.84970678)(761.02710725,71.60970702)(761.21710449,71.37970998)
\curveto(761.40710687,71.15970747)(761.63210664,70.96970766)(761.89210449,70.80970998)
\curveto(761.95210632,70.76970786)(762.01210626,70.7347079)(762.07210449,70.70470998)
\curveto(762.12210615,70.67470796)(762.18710609,70.64470799)(762.26710449,70.61470998)
\curveto(762.33710594,70.59470804)(762.39710588,70.58970804)(762.44710449,70.59970998)
\curveto(762.51710576,70.61970801)(762.5721057,70.65470798)(762.61210449,70.70470998)
\curveto(762.64210563,70.75470788)(762.66210561,70.81470782)(762.67210449,70.88470998)
\lineto(762.67210449,71.12470998)
\lineto(762.67210449,71.87470998)
\lineto(762.67210449,74.67970998)
\lineto(762.67210449,75.33970998)
\curveto(762.6721056,75.4297032)(762.6771056,75.51470312)(762.68710449,75.59470998)
\curveto(762.68710559,75.67470296)(762.70710557,75.73970289)(762.74710449,75.78970998)
\curveto(762.78710549,75.83970279)(762.86210541,75.87970275)(762.97210449,75.90970998)
\curveto(763.0721052,75.94970268)(763.1721051,75.95970267)(763.27210449,75.93970998)
\lineto(763.40710449,75.93970998)
\curveto(763.4771048,75.91970271)(763.53710474,75.89970273)(763.58710449,75.87970998)
\curveto(763.63710464,75.85970277)(763.6771046,75.82470281)(763.70710449,75.77470998)
\curveto(763.74710453,75.72470291)(763.76710451,75.65470298)(763.76710449,75.56470998)
\lineto(763.76710449,75.29470998)
\lineto(763.76710449,74.39470998)
\lineto(763.76710449,70.88470998)
\lineto(763.76710449,69.81970998)
\curveto(763.76710451,69.73970889)(763.7721045,69.64970898)(763.78210449,69.54970998)
\curveto(763.78210449,69.44970918)(763.7721045,69.36470927)(763.75210449,69.29470998)
\curveto(763.68210459,69.08470955)(763.50210477,69.01970961)(763.21210449,69.09970998)
\curveto(763.1721051,69.10970952)(763.13710514,69.10970952)(763.10710449,69.09970998)
\curveto(763.06710521,69.09970953)(763.02210525,69.10970952)(762.97210449,69.12970998)
\curveto(762.89210538,69.14970948)(762.80710547,69.16970946)(762.71710449,69.18970998)
\curveto(762.62710565,69.20970942)(762.54210573,69.2347094)(762.46210449,69.26470998)
\curveto(761.9721063,69.42470921)(761.55710672,69.62470901)(761.21710449,69.86470998)
\curveto(760.96710731,70.04470859)(760.74210753,70.24970838)(760.54210449,70.47970998)
\curveto(760.33210794,70.70970792)(760.13710814,70.94970768)(759.95710449,71.19970998)
\curveto(759.7771085,71.45970717)(759.60710867,71.72470691)(759.44710449,71.99470998)
\curveto(759.277109,72.27470636)(759.10210917,72.54470609)(758.92210449,72.80470998)
\curveto(758.84210943,72.91470572)(758.76710951,73.01970561)(758.69710449,73.11970998)
\curveto(758.62710965,73.2297054)(758.55210972,73.33970529)(758.47210449,73.44970998)
\curveto(758.44210983,73.48970514)(758.41210986,73.52470511)(758.38210449,73.55470998)
\curveto(758.34210993,73.59470504)(758.31210996,73.634705)(758.29210449,73.67470998)
\curveto(758.18211009,73.81470482)(758.05711022,73.93970469)(757.91710449,74.04970998)
\curveto(757.88711039,74.06970456)(757.86211041,74.09470454)(757.84210449,74.12470998)
\curveto(757.81211046,74.15470448)(757.78211049,74.17970445)(757.75210449,74.19970998)
\curveto(757.65211062,74.27970435)(757.55211072,74.34470429)(757.45210449,74.39470998)
\curveto(757.35211092,74.45470418)(757.24211103,74.50970412)(757.12210449,74.55970998)
\curveto(757.05211122,74.58970404)(756.9771113,74.60970402)(756.89710449,74.61970998)
\lineto(756.65710449,74.67970998)
\lineto(756.56710449,74.67970998)
\curveto(756.53711174,74.68970394)(756.50711177,74.69470394)(756.47710449,74.69470998)
\curveto(756.40711187,74.71470392)(756.31211196,74.71970391)(756.19210449,74.70970998)
\curveto(756.06211221,74.70970392)(755.96211231,74.69970393)(755.89210449,74.67970998)
\curveto(755.81211246,74.65970397)(755.73711254,74.63970399)(755.66710449,74.61970998)
\curveto(755.58711269,74.60970402)(755.50711277,74.58970404)(755.42710449,74.55970998)
\curveto(755.18711309,74.44970418)(754.98711329,74.29970433)(754.82710449,74.10970998)
\curveto(754.65711362,73.9297047)(754.51711376,73.70970492)(754.40710449,73.44970998)
\curveto(754.38711389,73.37970525)(754.3721139,73.30970532)(754.36210449,73.23970998)
\curveto(754.34211393,73.16970546)(754.32211395,73.09470554)(754.30210449,73.01470998)
\curveto(754.28211399,72.9347057)(754.272114,72.82470581)(754.27210449,72.68470998)
\curveto(754.272114,72.55470608)(754.28211399,72.44970618)(754.30210449,72.36970998)
\curveto(754.31211396,72.30970632)(754.31711396,72.25470638)(754.31710449,72.20470998)
\curveto(754.31711396,72.15470648)(754.32711395,72.10470653)(754.34710449,72.05470998)
\curveto(754.38711389,71.95470668)(754.42711385,71.85970677)(754.46710449,71.76970998)
\curveto(754.50711377,71.68970694)(754.55211372,71.60970702)(754.60210449,71.52970998)
\curveto(754.62211365,71.49970713)(754.64711363,71.46970716)(754.67710449,71.43970998)
\curveto(754.70711357,71.41970721)(754.73211354,71.39470724)(754.75210449,71.36470998)
\lineto(754.82710449,71.28970998)
\curveto(754.84711343,71.25970737)(754.86711341,71.2347074)(754.88710449,71.21470998)
\lineto(755.09710449,71.06470998)
\curveto(755.15711312,71.02470761)(755.22211305,70.97970765)(755.29210449,70.92970998)
\curveto(755.38211289,70.86970776)(755.48711279,70.81970781)(755.60710449,70.77970998)
\curveto(755.71711256,70.74970788)(755.82711245,70.71470792)(755.93710449,70.67470998)
\curveto(756.04711223,70.634708)(756.19211208,70.60970802)(756.37210449,70.59970998)
\curveto(756.54211173,70.58970804)(756.66711161,70.55970807)(756.74710449,70.50970998)
\curveto(756.82711145,70.45970817)(756.8721114,70.38470825)(756.88210449,70.28470998)
\curveto(756.89211138,70.18470845)(756.89711138,70.07470856)(756.89710449,69.95470998)
\curveto(756.89711138,69.91470872)(756.90211137,69.87470876)(756.91210449,69.83470998)
\curveto(756.91211136,69.79470884)(756.90711137,69.75970887)(756.89710449,69.72970998)
\curveto(756.8771114,69.67970895)(756.86711141,69.629709)(756.86710449,69.57970998)
\curveto(756.86711141,69.53970909)(756.85711142,69.49970913)(756.83710449,69.45970998)
\curveto(756.7771115,69.36970926)(756.64211163,69.32470931)(756.43210449,69.32470998)
\lineto(756.31210449,69.32470998)
\curveto(756.25211202,69.3347093)(756.19211208,69.33970929)(756.13210449,69.33970998)
\curveto(756.06211221,69.34970928)(755.99711228,69.35970927)(755.93710449,69.36970998)
\curveto(755.82711245,69.38970924)(755.72711255,69.40970922)(755.63710449,69.42970998)
\curveto(755.53711274,69.44970918)(755.44211283,69.47970915)(755.35210449,69.51970998)
\curveto(755.28211299,69.53970909)(755.22211305,69.55970907)(755.17210449,69.57970998)
\lineto(754.99210449,69.63970998)
\curveto(754.73211354,69.75970887)(754.48711379,69.91470872)(754.25710449,70.10470998)
\curveto(754.02711425,70.30470833)(753.84211443,70.51970811)(753.70210449,70.74970998)
\curveto(753.62211465,70.85970777)(753.55711472,70.97470766)(753.50710449,71.09470998)
\lineto(753.35710449,71.48470998)
\curveto(753.30711497,71.59470704)(753.277115,71.70970692)(753.26710449,71.82970998)
\curveto(753.24711503,71.94970668)(753.22211505,72.07470656)(753.19210449,72.20470998)
\curveto(753.19211508,72.27470636)(753.19211508,72.33970629)(753.19210449,72.39970998)
\curveto(753.18211509,72.45970617)(753.1721151,72.52470611)(753.16210449,72.59470998)
}
}
{
\newrgbcolor{curcolor}{0 0 0}
\pscustom[linestyle=none,fillstyle=solid,fillcolor=curcolor]
{
\newpath
\moveto(762.13210449,78.63431936)
\lineto(762.13210449,79.26431936)
\lineto(762.13210449,79.45931936)
\curveto(762.13210614,79.52931683)(762.14210613,79.58931677)(762.16210449,79.63931936)
\curveto(762.20210607,79.70931665)(762.24210603,79.7593166)(762.28210449,79.78931936)
\curveto(762.33210594,79.82931653)(762.39710588,79.84931651)(762.47710449,79.84931936)
\curveto(762.55710572,79.8593165)(762.64210563,79.86431649)(762.73210449,79.86431936)
\lineto(763.45210449,79.86431936)
\curveto(763.93210434,79.86431649)(764.34210393,79.80431655)(764.68210449,79.68431936)
\curveto(765.02210325,79.56431679)(765.29710298,79.36931699)(765.50710449,79.09931936)
\curveto(765.55710272,79.02931733)(765.60210267,78.9593174)(765.64210449,78.88931936)
\curveto(765.69210258,78.82931753)(765.73710254,78.7543176)(765.77710449,78.66431936)
\curveto(765.78710249,78.64431771)(765.79710248,78.61431774)(765.80710449,78.57431936)
\curveto(765.82710245,78.53431782)(765.83210244,78.48931787)(765.82210449,78.43931936)
\curveto(765.79210248,78.34931801)(765.71710256,78.29431806)(765.59710449,78.27431936)
\curveto(765.48710279,78.2543181)(765.39210288,78.26931809)(765.31210449,78.31931936)
\curveto(765.24210303,78.34931801)(765.1771031,78.39431796)(765.11710449,78.45431936)
\curveto(765.06710321,78.52431783)(765.01710326,78.58931777)(764.96710449,78.64931936)
\curveto(764.91710336,78.71931764)(764.84210343,78.77931758)(764.74210449,78.82931936)
\curveto(764.65210362,78.88931747)(764.56210371,78.93931742)(764.47210449,78.97931936)
\curveto(764.44210383,78.99931736)(764.38210389,79.02431733)(764.29210449,79.05431936)
\curveto(764.21210406,79.08431727)(764.14210413,79.08931727)(764.08210449,79.06931936)
\curveto(763.94210433,79.03931732)(763.85210442,78.97931738)(763.81210449,78.88931936)
\curveto(763.78210449,78.80931755)(763.76710451,78.71931764)(763.76710449,78.61931936)
\curveto(763.76710451,78.51931784)(763.74210453,78.43431792)(763.69210449,78.36431936)
\curveto(763.62210465,78.27431808)(763.48210479,78.22931813)(763.27210449,78.22931936)
\lineto(762.71710449,78.22931936)
\lineto(762.49210449,78.22931936)
\curveto(762.41210586,78.23931812)(762.34710593,78.2593181)(762.29710449,78.28931936)
\curveto(762.21710606,78.34931801)(762.1721061,78.41931794)(762.16210449,78.49931936)
\curveto(762.15210612,78.51931784)(762.14710613,78.53931782)(762.14710449,78.55931936)
\curveto(762.14710613,78.58931777)(762.14210613,78.61431774)(762.13210449,78.63431936)
}
}
{
\newrgbcolor{curcolor}{0 0 0}
\pscustom[linestyle=none,fillstyle=solid,fillcolor=curcolor]
{
}
}
{
\newrgbcolor{curcolor}{0 0 0}
\pscustom[linestyle=none,fillstyle=solid,fillcolor=curcolor]
{
\newpath
\moveto(753.16210449,89.26463186)
\curveto(753.15211512,89.95462722)(753.272115,90.55462662)(753.52210449,91.06463186)
\curveto(753.7721145,91.58462559)(754.10711417,91.9796252)(754.52710449,92.24963186)
\curveto(754.60711367,92.29962488)(754.69711358,92.34462483)(754.79710449,92.38463186)
\curveto(754.88711339,92.42462475)(754.98211329,92.46962471)(755.08210449,92.51963186)
\curveto(755.18211309,92.55962462)(755.28211299,92.58962459)(755.38210449,92.60963186)
\curveto(755.48211279,92.62962455)(755.58711269,92.64962453)(755.69710449,92.66963186)
\curveto(755.74711253,92.68962449)(755.79211248,92.69462448)(755.83210449,92.68463186)
\curveto(755.8721124,92.6746245)(755.91711236,92.6796245)(755.96710449,92.69963186)
\curveto(756.01711226,92.70962447)(756.10211217,92.71462446)(756.22210449,92.71463186)
\curveto(756.33211194,92.71462446)(756.41711186,92.70962447)(756.47710449,92.69963186)
\curveto(756.53711174,92.6796245)(756.59711168,92.66962451)(756.65710449,92.66963186)
\curveto(756.71711156,92.6796245)(756.7771115,92.6746245)(756.83710449,92.65463186)
\curveto(756.9771113,92.61462456)(757.11211116,92.5796246)(757.24210449,92.54963186)
\curveto(757.3721109,92.51962466)(757.49711078,92.4796247)(757.61710449,92.42963186)
\curveto(757.75711052,92.36962481)(757.88211039,92.29962488)(757.99210449,92.21963186)
\curveto(758.10211017,92.14962503)(758.21211006,92.0746251)(758.32210449,91.99463186)
\lineto(758.38210449,91.93463186)
\curveto(758.40210987,91.92462525)(758.42210985,91.90962527)(758.44210449,91.88963186)
\curveto(758.60210967,91.76962541)(758.74710953,91.63462554)(758.87710449,91.48463186)
\curveto(759.00710927,91.33462584)(759.13210914,91.174626)(759.25210449,91.00463186)
\curveto(759.4721088,90.69462648)(759.6771086,90.39962678)(759.86710449,90.11963186)
\curveto(760.00710827,89.88962729)(760.14210813,89.65962752)(760.27210449,89.42963186)
\curveto(760.40210787,89.20962797)(760.53710774,88.98962819)(760.67710449,88.76963186)
\curveto(760.84710743,88.51962866)(761.02710725,88.2796289)(761.21710449,88.04963186)
\curveto(761.40710687,87.82962935)(761.63210664,87.63962954)(761.89210449,87.47963186)
\curveto(761.95210632,87.43962974)(762.01210626,87.40462977)(762.07210449,87.37463186)
\curveto(762.12210615,87.34462983)(762.18710609,87.31462986)(762.26710449,87.28463186)
\curveto(762.33710594,87.26462991)(762.39710588,87.25962992)(762.44710449,87.26963186)
\curveto(762.51710576,87.28962989)(762.5721057,87.32462985)(762.61210449,87.37463186)
\curveto(762.64210563,87.42462975)(762.66210561,87.48462969)(762.67210449,87.55463186)
\lineto(762.67210449,87.79463186)
\lineto(762.67210449,88.54463186)
\lineto(762.67210449,91.34963186)
\lineto(762.67210449,92.00963186)
\curveto(762.6721056,92.09962508)(762.6771056,92.18462499)(762.68710449,92.26463186)
\curveto(762.68710559,92.34462483)(762.70710557,92.40962477)(762.74710449,92.45963186)
\curveto(762.78710549,92.50962467)(762.86210541,92.54962463)(762.97210449,92.57963186)
\curveto(763.0721052,92.61962456)(763.1721051,92.62962455)(763.27210449,92.60963186)
\lineto(763.40710449,92.60963186)
\curveto(763.4771048,92.58962459)(763.53710474,92.56962461)(763.58710449,92.54963186)
\curveto(763.63710464,92.52962465)(763.6771046,92.49462468)(763.70710449,92.44463186)
\curveto(763.74710453,92.39462478)(763.76710451,92.32462485)(763.76710449,92.23463186)
\lineto(763.76710449,91.96463186)
\lineto(763.76710449,91.06463186)
\lineto(763.76710449,87.55463186)
\lineto(763.76710449,86.48963186)
\curveto(763.76710451,86.40963077)(763.7721045,86.31963086)(763.78210449,86.21963186)
\curveto(763.78210449,86.11963106)(763.7721045,86.03463114)(763.75210449,85.96463186)
\curveto(763.68210459,85.75463142)(763.50210477,85.68963149)(763.21210449,85.76963186)
\curveto(763.1721051,85.7796314)(763.13710514,85.7796314)(763.10710449,85.76963186)
\curveto(763.06710521,85.76963141)(763.02210525,85.7796314)(762.97210449,85.79963186)
\curveto(762.89210538,85.81963136)(762.80710547,85.83963134)(762.71710449,85.85963186)
\curveto(762.62710565,85.8796313)(762.54210573,85.90463127)(762.46210449,85.93463186)
\curveto(761.9721063,86.09463108)(761.55710672,86.29463088)(761.21710449,86.53463186)
\curveto(760.96710731,86.71463046)(760.74210753,86.91963026)(760.54210449,87.14963186)
\curveto(760.33210794,87.3796298)(760.13710814,87.61962956)(759.95710449,87.86963186)
\curveto(759.7771085,88.12962905)(759.60710867,88.39462878)(759.44710449,88.66463186)
\curveto(759.277109,88.94462823)(759.10210917,89.21462796)(758.92210449,89.47463186)
\curveto(758.84210943,89.58462759)(758.76710951,89.68962749)(758.69710449,89.78963186)
\curveto(758.62710965,89.89962728)(758.55210972,90.00962717)(758.47210449,90.11963186)
\curveto(758.44210983,90.15962702)(758.41210986,90.19462698)(758.38210449,90.22463186)
\curveto(758.34210993,90.26462691)(758.31210996,90.30462687)(758.29210449,90.34463186)
\curveto(758.18211009,90.48462669)(758.05711022,90.60962657)(757.91710449,90.71963186)
\curveto(757.88711039,90.73962644)(757.86211041,90.76462641)(757.84210449,90.79463186)
\curveto(757.81211046,90.82462635)(757.78211049,90.84962633)(757.75210449,90.86963186)
\curveto(757.65211062,90.94962623)(757.55211072,91.01462616)(757.45210449,91.06463186)
\curveto(757.35211092,91.12462605)(757.24211103,91.179626)(757.12210449,91.22963186)
\curveto(757.05211122,91.25962592)(756.9771113,91.2796259)(756.89710449,91.28963186)
\lineto(756.65710449,91.34963186)
\lineto(756.56710449,91.34963186)
\curveto(756.53711174,91.35962582)(756.50711177,91.36462581)(756.47710449,91.36463186)
\curveto(756.40711187,91.38462579)(756.31211196,91.38962579)(756.19210449,91.37963186)
\curveto(756.06211221,91.3796258)(755.96211231,91.36962581)(755.89210449,91.34963186)
\curveto(755.81211246,91.32962585)(755.73711254,91.30962587)(755.66710449,91.28963186)
\curveto(755.58711269,91.2796259)(755.50711277,91.25962592)(755.42710449,91.22963186)
\curveto(755.18711309,91.11962606)(754.98711329,90.96962621)(754.82710449,90.77963186)
\curveto(754.65711362,90.59962658)(754.51711376,90.3796268)(754.40710449,90.11963186)
\curveto(754.38711389,90.04962713)(754.3721139,89.9796272)(754.36210449,89.90963186)
\curveto(754.34211393,89.83962734)(754.32211395,89.76462741)(754.30210449,89.68463186)
\curveto(754.28211399,89.60462757)(754.272114,89.49462768)(754.27210449,89.35463186)
\curveto(754.272114,89.22462795)(754.28211399,89.11962806)(754.30210449,89.03963186)
\curveto(754.31211396,88.9796282)(754.31711396,88.92462825)(754.31710449,88.87463186)
\curveto(754.31711396,88.82462835)(754.32711395,88.7746284)(754.34710449,88.72463186)
\curveto(754.38711389,88.62462855)(754.42711385,88.52962865)(754.46710449,88.43963186)
\curveto(754.50711377,88.35962882)(754.55211372,88.2796289)(754.60210449,88.19963186)
\curveto(754.62211365,88.16962901)(754.64711363,88.13962904)(754.67710449,88.10963186)
\curveto(754.70711357,88.08962909)(754.73211354,88.06462911)(754.75210449,88.03463186)
\lineto(754.82710449,87.95963186)
\curveto(754.84711343,87.92962925)(754.86711341,87.90462927)(754.88710449,87.88463186)
\lineto(755.09710449,87.73463186)
\curveto(755.15711312,87.69462948)(755.22211305,87.64962953)(755.29210449,87.59963186)
\curveto(755.38211289,87.53962964)(755.48711279,87.48962969)(755.60710449,87.44963186)
\curveto(755.71711256,87.41962976)(755.82711245,87.38462979)(755.93710449,87.34463186)
\curveto(756.04711223,87.30462987)(756.19211208,87.2796299)(756.37210449,87.26963186)
\curveto(756.54211173,87.25962992)(756.66711161,87.22962995)(756.74710449,87.17963186)
\curveto(756.82711145,87.12963005)(756.8721114,87.05463012)(756.88210449,86.95463186)
\curveto(756.89211138,86.85463032)(756.89711138,86.74463043)(756.89710449,86.62463186)
\curveto(756.89711138,86.58463059)(756.90211137,86.54463063)(756.91210449,86.50463186)
\curveto(756.91211136,86.46463071)(756.90711137,86.42963075)(756.89710449,86.39963186)
\curveto(756.8771114,86.34963083)(756.86711141,86.29963088)(756.86710449,86.24963186)
\curveto(756.86711141,86.20963097)(756.85711142,86.16963101)(756.83710449,86.12963186)
\curveto(756.7771115,86.03963114)(756.64211163,85.99463118)(756.43210449,85.99463186)
\lineto(756.31210449,85.99463186)
\curveto(756.25211202,86.00463117)(756.19211208,86.00963117)(756.13210449,86.00963186)
\curveto(756.06211221,86.01963116)(755.99711228,86.02963115)(755.93710449,86.03963186)
\curveto(755.82711245,86.05963112)(755.72711255,86.0796311)(755.63710449,86.09963186)
\curveto(755.53711274,86.11963106)(755.44211283,86.14963103)(755.35210449,86.18963186)
\curveto(755.28211299,86.20963097)(755.22211305,86.22963095)(755.17210449,86.24963186)
\lineto(754.99210449,86.30963186)
\curveto(754.73211354,86.42963075)(754.48711379,86.58463059)(754.25710449,86.77463186)
\curveto(754.02711425,86.9746302)(753.84211443,87.18962999)(753.70210449,87.41963186)
\curveto(753.62211465,87.52962965)(753.55711472,87.64462953)(753.50710449,87.76463186)
\lineto(753.35710449,88.15463186)
\curveto(753.30711497,88.26462891)(753.277115,88.3796288)(753.26710449,88.49963186)
\curveto(753.24711503,88.61962856)(753.22211505,88.74462843)(753.19210449,88.87463186)
\curveto(753.19211508,88.94462823)(753.19211508,89.00962817)(753.19210449,89.06963186)
\curveto(753.18211509,89.12962805)(753.1721151,89.19462798)(753.16210449,89.26463186)
}
}
{
\newrgbcolor{curcolor}{0 0 0}
\pscustom[linestyle=none,fillstyle=solid,fillcolor=curcolor]
{
\newpath
\moveto(758.68210449,101.36424123)
\lineto(758.93710449,101.36424123)
\curveto(759.01710926,101.37423353)(759.09210918,101.36923353)(759.16210449,101.34924123)
\lineto(759.40210449,101.34924123)
\lineto(759.56710449,101.34924123)
\curveto(759.66710861,101.32923357)(759.7721085,101.31923358)(759.88210449,101.31924123)
\curveto(759.98210829,101.31923358)(760.08210819,101.30923359)(760.18210449,101.28924123)
\lineto(760.33210449,101.28924123)
\curveto(760.4721078,101.25923364)(760.61210766,101.23923366)(760.75210449,101.22924123)
\curveto(760.88210739,101.21923368)(761.01210726,101.19423371)(761.14210449,101.15424123)
\curveto(761.22210705,101.13423377)(761.30710697,101.11423379)(761.39710449,101.09424123)
\lineto(761.63710449,101.03424123)
\lineto(761.93710449,100.91424123)
\curveto(762.02710625,100.88423402)(762.11710616,100.84923405)(762.20710449,100.80924123)
\curveto(762.42710585,100.70923419)(762.64210563,100.57423433)(762.85210449,100.40424123)
\curveto(763.06210521,100.24423466)(763.23210504,100.06923483)(763.36210449,99.87924123)
\curveto(763.40210487,99.82923507)(763.44210483,99.76923513)(763.48210449,99.69924123)
\curveto(763.51210476,99.63923526)(763.54710473,99.57923532)(763.58710449,99.51924123)
\curveto(763.63710464,99.43923546)(763.6771046,99.34423556)(763.70710449,99.23424123)
\curveto(763.73710454,99.12423578)(763.76710451,99.01923588)(763.79710449,98.91924123)
\curveto(763.83710444,98.80923609)(763.86210441,98.6992362)(763.87210449,98.58924123)
\curveto(763.88210439,98.47923642)(763.89710438,98.36423654)(763.91710449,98.24424123)
\curveto(763.92710435,98.2042367)(763.92710435,98.15923674)(763.91710449,98.10924123)
\curveto(763.91710436,98.06923683)(763.92210435,98.02923687)(763.93210449,97.98924123)
\curveto(763.94210433,97.94923695)(763.94710433,97.89423701)(763.94710449,97.82424123)
\curveto(763.94710433,97.75423715)(763.94210433,97.7042372)(763.93210449,97.67424123)
\curveto(763.91210436,97.62423728)(763.90710437,97.57923732)(763.91710449,97.53924123)
\curveto(763.92710435,97.4992374)(763.92710435,97.46423744)(763.91710449,97.43424123)
\lineto(763.91710449,97.34424123)
\curveto(763.89710438,97.28423762)(763.88210439,97.21923768)(763.87210449,97.14924123)
\curveto(763.8721044,97.08923781)(763.86710441,97.02423788)(763.85710449,96.95424123)
\curveto(763.80710447,96.78423812)(763.75710452,96.62423828)(763.70710449,96.47424123)
\curveto(763.65710462,96.32423858)(763.59210468,96.17923872)(763.51210449,96.03924123)
\curveto(763.4721048,95.98923891)(763.44210483,95.93423897)(763.42210449,95.87424123)
\curveto(763.39210488,95.82423908)(763.35710492,95.77423913)(763.31710449,95.72424123)
\curveto(763.13710514,95.48423942)(762.91710536,95.28423962)(762.65710449,95.12424123)
\curveto(762.39710588,94.96423994)(762.11210616,94.82424008)(761.80210449,94.70424123)
\curveto(761.66210661,94.64424026)(761.52210675,94.5992403)(761.38210449,94.56924123)
\curveto(761.23210704,94.53924036)(761.0771072,94.5042404)(760.91710449,94.46424123)
\curveto(760.80710747,94.44424046)(760.69710758,94.42924047)(760.58710449,94.41924123)
\curveto(760.4771078,94.40924049)(760.36710791,94.39424051)(760.25710449,94.37424123)
\curveto(760.21710806,94.36424054)(760.1771081,94.35924054)(760.13710449,94.35924123)
\curveto(760.09710818,94.36924053)(760.05710822,94.36924053)(760.01710449,94.35924123)
\curveto(759.96710831,94.34924055)(759.91710836,94.34424056)(759.86710449,94.34424123)
\lineto(759.70210449,94.34424123)
\curveto(759.65210862,94.32424058)(759.60210867,94.31924058)(759.55210449,94.32924123)
\curveto(759.49210878,94.33924056)(759.43710884,94.33924056)(759.38710449,94.32924123)
\curveto(759.34710893,94.31924058)(759.30210897,94.31924058)(759.25210449,94.32924123)
\curveto(759.20210907,94.33924056)(759.15210912,94.33424057)(759.10210449,94.31424123)
\curveto(759.03210924,94.29424061)(758.95710932,94.28924061)(758.87710449,94.29924123)
\curveto(758.78710949,94.30924059)(758.70210957,94.31424059)(758.62210449,94.31424123)
\curveto(758.53210974,94.31424059)(758.43210984,94.30924059)(758.32210449,94.29924123)
\curveto(758.20211007,94.28924061)(758.10211017,94.29424061)(758.02210449,94.31424123)
\lineto(757.73710449,94.31424123)
\lineto(757.10710449,94.35924123)
\curveto(757.00711127,94.36924053)(756.91211136,94.37924052)(756.82210449,94.38924123)
\lineto(756.52210449,94.41924123)
\curveto(756.4721118,94.43924046)(756.42211185,94.44424046)(756.37210449,94.43424123)
\curveto(756.31211196,94.43424047)(756.25711202,94.44424046)(756.20710449,94.46424123)
\curveto(756.03711224,94.51424039)(755.8721124,94.55424035)(755.71210449,94.58424123)
\curveto(755.54211273,94.61424029)(755.38211289,94.66424024)(755.23210449,94.73424123)
\curveto(754.7721135,94.92423998)(754.39711388,95.14423976)(754.10710449,95.39424123)
\curveto(753.81711446,95.65423925)(753.5721147,96.01423889)(753.37210449,96.47424123)
\curveto(753.32211495,96.6042383)(753.28711499,96.73423817)(753.26710449,96.86424123)
\curveto(753.24711503,97.0042379)(753.22211505,97.14423776)(753.19210449,97.28424123)
\curveto(753.18211509,97.35423755)(753.1771151,97.41923748)(753.17710449,97.47924123)
\curveto(753.1771151,97.53923736)(753.1721151,97.6042373)(753.16210449,97.67424123)
\curveto(753.14211513,98.5042364)(753.29211498,99.17423573)(753.61210449,99.68424123)
\curveto(753.92211435,100.19423471)(754.36211391,100.57423433)(754.93210449,100.82424123)
\curveto(755.05211322,100.87423403)(755.1771131,100.91923398)(755.30710449,100.95924123)
\curveto(755.43711284,100.9992339)(755.5721127,101.04423386)(755.71210449,101.09424123)
\curveto(755.79211248,101.11423379)(755.8771124,101.12923377)(755.96710449,101.13924123)
\lineto(756.20710449,101.19924123)
\curveto(756.31711196,101.22923367)(756.42711185,101.24423366)(756.53710449,101.24424123)
\curveto(756.64711163,101.25423365)(756.75711152,101.26923363)(756.86710449,101.28924123)
\curveto(756.91711136,101.30923359)(756.96211131,101.31423359)(757.00210449,101.30424123)
\curveto(757.04211123,101.3042336)(757.08211119,101.30923359)(757.12210449,101.31924123)
\curveto(757.1721111,101.32923357)(757.22711105,101.32923357)(757.28710449,101.31924123)
\curveto(757.33711094,101.31923358)(757.38711089,101.32423358)(757.43710449,101.33424123)
\lineto(757.57210449,101.33424123)
\curveto(757.63211064,101.35423355)(757.70211057,101.35423355)(757.78210449,101.33424123)
\curveto(757.85211042,101.32423358)(757.91711036,101.32923357)(757.97710449,101.34924123)
\curveto(758.00711027,101.35923354)(758.04711023,101.36423354)(758.09710449,101.36424123)
\lineto(758.21710449,101.36424123)
\lineto(758.68210449,101.36424123)
\moveto(761.00710449,99.81924123)
\curveto(760.68710759,99.91923498)(760.32210795,99.97923492)(759.91210449,99.99924123)
\curveto(759.50210877,100.01923488)(759.09210918,100.02923487)(758.68210449,100.02924123)
\curveto(758.25211002,100.02923487)(757.83211044,100.01923488)(757.42210449,99.99924123)
\curveto(757.01211126,99.97923492)(756.62711165,99.93423497)(756.26710449,99.86424123)
\curveto(755.90711237,99.79423511)(755.58711269,99.68423522)(755.30710449,99.53424123)
\curveto(755.01711326,99.39423551)(754.78211349,99.1992357)(754.60210449,98.94924123)
\curveto(754.49211378,98.78923611)(754.41211386,98.60923629)(754.36210449,98.40924123)
\curveto(754.30211397,98.20923669)(754.272114,97.96423694)(754.27210449,97.67424123)
\curveto(754.29211398,97.65423725)(754.30211397,97.61923728)(754.30210449,97.56924123)
\curveto(754.29211398,97.51923738)(754.29211398,97.47923742)(754.30210449,97.44924123)
\curveto(754.32211395,97.36923753)(754.34211393,97.29423761)(754.36210449,97.22424123)
\curveto(754.3721139,97.16423774)(754.39211388,97.0992378)(754.42210449,97.02924123)
\curveto(754.54211373,96.75923814)(754.71211356,96.53923836)(754.93210449,96.36924123)
\curveto(755.14211313,96.20923869)(755.38711289,96.07423883)(755.66710449,95.96424123)
\curveto(755.7771125,95.91423899)(755.89711238,95.87423903)(756.02710449,95.84424123)
\curveto(756.14711213,95.82423908)(756.272112,95.7992391)(756.40210449,95.76924123)
\curveto(756.45211182,95.74923915)(756.50711177,95.73923916)(756.56710449,95.73924123)
\curveto(756.61711166,95.73923916)(756.66711161,95.73423917)(756.71710449,95.72424123)
\curveto(756.80711147,95.71423919)(756.90211137,95.7042392)(757.00210449,95.69424123)
\curveto(757.09211118,95.68423922)(757.18711109,95.67423923)(757.28710449,95.66424123)
\curveto(757.36711091,95.66423924)(757.45211082,95.65923924)(757.54210449,95.64924123)
\lineto(757.78210449,95.64924123)
\lineto(757.96210449,95.64924123)
\curveto(757.99211028,95.63923926)(758.02711025,95.63423927)(758.06710449,95.63424123)
\lineto(758.20210449,95.63424123)
\lineto(758.65210449,95.63424123)
\curveto(758.73210954,95.63423927)(758.81710946,95.62923927)(758.90710449,95.61924123)
\curveto(758.98710929,95.61923928)(759.06210921,95.62923927)(759.13210449,95.64924123)
\lineto(759.40210449,95.64924123)
\curveto(759.42210885,95.64923925)(759.45210882,95.64423926)(759.49210449,95.63424123)
\curveto(759.52210875,95.63423927)(759.54710873,95.63923926)(759.56710449,95.64924123)
\curveto(759.66710861,95.65923924)(759.76710851,95.66423924)(759.86710449,95.66424123)
\curveto(759.95710832,95.67423923)(760.05710822,95.68423922)(760.16710449,95.69424123)
\curveto(760.28710799,95.72423918)(760.41210786,95.73923916)(760.54210449,95.73924123)
\curveto(760.66210761,95.74923915)(760.7771075,95.77423913)(760.88710449,95.81424123)
\curveto(761.18710709,95.89423901)(761.45210682,95.97923892)(761.68210449,96.06924123)
\curveto(761.91210636,96.16923873)(762.12710615,96.31423859)(762.32710449,96.50424123)
\curveto(762.52710575,96.71423819)(762.6771056,96.97923792)(762.77710449,97.29924123)
\curveto(762.79710548,97.33923756)(762.80710547,97.37423753)(762.80710449,97.40424123)
\curveto(762.79710548,97.44423746)(762.80210547,97.48923741)(762.82210449,97.53924123)
\curveto(762.83210544,97.57923732)(762.84210543,97.64923725)(762.85210449,97.74924123)
\curveto(762.86210541,97.85923704)(762.85710542,97.94423696)(762.83710449,98.00424123)
\curveto(762.81710546,98.07423683)(762.80710547,98.14423676)(762.80710449,98.21424123)
\curveto(762.79710548,98.28423662)(762.78210549,98.34923655)(762.76210449,98.40924123)
\curveto(762.70210557,98.60923629)(762.61710566,98.78923611)(762.50710449,98.94924123)
\curveto(762.48710579,98.97923592)(762.46710581,99.0042359)(762.44710449,99.02424123)
\lineto(762.38710449,99.08424123)
\curveto(762.36710591,99.12423578)(762.32710595,99.17423573)(762.26710449,99.23424123)
\curveto(762.12710615,99.33423557)(761.99710628,99.41923548)(761.87710449,99.48924123)
\curveto(761.75710652,99.55923534)(761.61210666,99.62923527)(761.44210449,99.69924123)
\curveto(761.3721069,99.72923517)(761.30210697,99.74923515)(761.23210449,99.75924123)
\curveto(761.16210711,99.77923512)(761.08710719,99.7992351)(761.00710449,99.81924123)
}
}
{
\newrgbcolor{curcolor}{0 0 0}
\pscustom[linestyle=none,fillstyle=solid,fillcolor=curcolor]
{
\newpath
\moveto(753.16210449,106.77385061)
\curveto(753.16211511,106.87384575)(753.1721151,106.96884566)(753.19210449,107.05885061)
\curveto(753.20211507,107.14884548)(753.23211504,107.21384541)(753.28210449,107.25385061)
\curveto(753.36211491,107.31384531)(753.46711481,107.34384528)(753.59710449,107.34385061)
\lineto(753.98710449,107.34385061)
\lineto(755.48710449,107.34385061)
\lineto(761.87710449,107.34385061)
\lineto(763.04710449,107.34385061)
\lineto(763.36210449,107.34385061)
\curveto(763.46210481,107.35384527)(763.54210473,107.33884529)(763.60210449,107.29885061)
\curveto(763.68210459,107.24884538)(763.73210454,107.17384545)(763.75210449,107.07385061)
\curveto(763.76210451,106.98384564)(763.76710451,106.87384575)(763.76710449,106.74385061)
\lineto(763.76710449,106.51885061)
\curveto(763.74710453,106.43884619)(763.73210454,106.36884626)(763.72210449,106.30885061)
\curveto(763.70210457,106.24884638)(763.66210461,106.19884643)(763.60210449,106.15885061)
\curveto(763.54210473,106.11884651)(763.46710481,106.09884653)(763.37710449,106.09885061)
\lineto(763.07710449,106.09885061)
\lineto(761.98210449,106.09885061)
\lineto(756.64210449,106.09885061)
\curveto(756.55211172,106.07884655)(756.4771118,106.06384656)(756.41710449,106.05385061)
\curveto(756.34711193,106.05384657)(756.28711199,106.0238466)(756.23710449,105.96385061)
\curveto(756.18711209,105.89384673)(756.16211211,105.80384682)(756.16210449,105.69385061)
\curveto(756.15211212,105.59384703)(756.14711213,105.48384714)(756.14710449,105.36385061)
\lineto(756.14710449,104.22385061)
\lineto(756.14710449,103.72885061)
\curveto(756.13711214,103.56884906)(756.0771122,103.45884917)(755.96710449,103.39885061)
\curveto(755.93711234,103.37884925)(755.90711237,103.36884926)(755.87710449,103.36885061)
\curveto(755.83711244,103.36884926)(755.79211248,103.36384926)(755.74210449,103.35385061)
\curveto(755.62211265,103.33384929)(755.51211276,103.33884929)(755.41210449,103.36885061)
\curveto(755.31211296,103.40884922)(755.24211303,103.46384916)(755.20210449,103.53385061)
\curveto(755.15211312,103.61384901)(755.12711315,103.73384889)(755.12710449,103.89385061)
\curveto(755.12711315,104.05384857)(755.11211316,104.18884844)(755.08210449,104.29885061)
\curveto(755.0721132,104.34884828)(755.06711321,104.40384822)(755.06710449,104.46385061)
\curveto(755.05711322,104.5238481)(755.04211323,104.58384804)(755.02210449,104.64385061)
\curveto(754.9721133,104.79384783)(754.92211335,104.93884769)(754.87210449,105.07885061)
\curveto(754.81211346,105.21884741)(754.74211353,105.35384727)(754.66210449,105.48385061)
\curveto(754.5721137,105.623847)(754.46711381,105.74384688)(754.34710449,105.84385061)
\curveto(754.22711405,105.94384668)(754.09711418,106.03884659)(753.95710449,106.12885061)
\curveto(753.85711442,106.18884644)(753.74711453,106.23384639)(753.62710449,106.26385061)
\curveto(753.50711477,106.30384632)(753.40211487,106.35384627)(753.31210449,106.41385061)
\curveto(753.25211502,106.46384616)(753.21211506,106.53384609)(753.19210449,106.62385061)
\curveto(753.18211509,106.64384598)(753.1771151,106.66884596)(753.17710449,106.69885061)
\curveto(753.1771151,106.7288459)(753.1721151,106.75384587)(753.16210449,106.77385061)
}
}
{
\newrgbcolor{curcolor}{0 0 0}
\pscustom[linestyle=none,fillstyle=solid,fillcolor=curcolor]
{
\newpath
\moveto(753.16210449,115.12345998)
\curveto(753.16211511,115.22345513)(753.1721151,115.31845503)(753.19210449,115.40845998)
\curveto(753.20211507,115.49845485)(753.23211504,115.56345479)(753.28210449,115.60345998)
\curveto(753.36211491,115.66345469)(753.46711481,115.69345466)(753.59710449,115.69345998)
\lineto(753.98710449,115.69345998)
\lineto(755.48710449,115.69345998)
\lineto(761.87710449,115.69345998)
\lineto(763.04710449,115.69345998)
\lineto(763.36210449,115.69345998)
\curveto(763.46210481,115.70345465)(763.54210473,115.68845466)(763.60210449,115.64845998)
\curveto(763.68210459,115.59845475)(763.73210454,115.52345483)(763.75210449,115.42345998)
\curveto(763.76210451,115.33345502)(763.76710451,115.22345513)(763.76710449,115.09345998)
\lineto(763.76710449,114.86845998)
\curveto(763.74710453,114.78845556)(763.73210454,114.71845563)(763.72210449,114.65845998)
\curveto(763.70210457,114.59845575)(763.66210461,114.5484558)(763.60210449,114.50845998)
\curveto(763.54210473,114.46845588)(763.46710481,114.4484559)(763.37710449,114.44845998)
\lineto(763.07710449,114.44845998)
\lineto(761.98210449,114.44845998)
\lineto(756.64210449,114.44845998)
\curveto(756.55211172,114.42845592)(756.4771118,114.41345594)(756.41710449,114.40345998)
\curveto(756.34711193,114.40345595)(756.28711199,114.37345598)(756.23710449,114.31345998)
\curveto(756.18711209,114.24345611)(756.16211211,114.1534562)(756.16210449,114.04345998)
\curveto(756.15211212,113.94345641)(756.14711213,113.83345652)(756.14710449,113.71345998)
\lineto(756.14710449,112.57345998)
\lineto(756.14710449,112.07845998)
\curveto(756.13711214,111.91845843)(756.0771122,111.80845854)(755.96710449,111.74845998)
\curveto(755.93711234,111.72845862)(755.90711237,111.71845863)(755.87710449,111.71845998)
\curveto(755.83711244,111.71845863)(755.79211248,111.71345864)(755.74210449,111.70345998)
\curveto(755.62211265,111.68345867)(755.51211276,111.68845866)(755.41210449,111.71845998)
\curveto(755.31211296,111.75845859)(755.24211303,111.81345854)(755.20210449,111.88345998)
\curveto(755.15211312,111.96345839)(755.12711315,112.08345827)(755.12710449,112.24345998)
\curveto(755.12711315,112.40345795)(755.11211316,112.53845781)(755.08210449,112.64845998)
\curveto(755.0721132,112.69845765)(755.06711321,112.7534576)(755.06710449,112.81345998)
\curveto(755.05711322,112.87345748)(755.04211323,112.93345742)(755.02210449,112.99345998)
\curveto(754.9721133,113.14345721)(754.92211335,113.28845706)(754.87210449,113.42845998)
\curveto(754.81211346,113.56845678)(754.74211353,113.70345665)(754.66210449,113.83345998)
\curveto(754.5721137,113.97345638)(754.46711381,114.09345626)(754.34710449,114.19345998)
\curveto(754.22711405,114.29345606)(754.09711418,114.38845596)(753.95710449,114.47845998)
\curveto(753.85711442,114.53845581)(753.74711453,114.58345577)(753.62710449,114.61345998)
\curveto(753.50711477,114.6534557)(753.40211487,114.70345565)(753.31210449,114.76345998)
\curveto(753.25211502,114.81345554)(753.21211506,114.88345547)(753.19210449,114.97345998)
\curveto(753.18211509,114.99345536)(753.1771151,115.01845533)(753.17710449,115.04845998)
\curveto(753.1771151,115.07845527)(753.1721151,115.10345525)(753.16210449,115.12345998)
}
}
{
\newrgbcolor{curcolor}{0 0 0}
\pscustom[linestyle=none,fillstyle=solid,fillcolor=curcolor]
{
\newpath
\moveto(773.99842041,37.28705373)
\curveto(773.99843111,37.35704805)(773.99843111,37.43704797)(773.99842041,37.52705373)
\curveto(773.98843112,37.61704779)(773.98843112,37.70204771)(773.99842041,37.78205373)
\curveto(773.99843111,37.87204754)(774.0084311,37.95204746)(774.02842041,38.02205373)
\curveto(774.04843106,38.10204731)(774.07843103,38.15704725)(774.11842041,38.18705373)
\curveto(774.16843094,38.21704719)(774.24343086,38.23704717)(774.34342041,38.24705373)
\curveto(774.43343067,38.26704714)(774.53843057,38.27704713)(774.65842041,38.27705373)
\curveto(774.76843034,38.28704712)(774.88343022,38.28704712)(775.00342041,38.27705373)
\lineto(775.30342041,38.27705373)
\lineto(778.31842041,38.27705373)
\lineto(781.21342041,38.27705373)
\curveto(781.54342356,38.27704713)(781.86842324,38.27204714)(782.18842041,38.26205373)
\curveto(782.49842261,38.26204715)(782.77842233,38.22204719)(783.02842041,38.14205373)
\curveto(783.37842173,38.02204739)(783.67342143,37.86704754)(783.91342041,37.67705373)
\curveto(784.14342096,37.48704792)(784.34342076,37.24704816)(784.51342041,36.95705373)
\curveto(784.56342054,36.89704851)(784.59842051,36.83204858)(784.61842041,36.76205373)
\curveto(784.63842047,36.70204871)(784.66342044,36.63204878)(784.69342041,36.55205373)
\curveto(784.74342036,36.43204898)(784.77842033,36.30204911)(784.79842041,36.16205373)
\curveto(784.82842028,36.03204938)(784.85842025,35.89704951)(784.88842041,35.75705373)
\curveto(784.9084202,35.7070497)(784.91342019,35.65704975)(784.90342041,35.60705373)
\curveto(784.89342021,35.55704985)(784.89342021,35.50204991)(784.90342041,35.44205373)
\curveto(784.91342019,35.42204999)(784.91342019,35.39705001)(784.90342041,35.36705373)
\curveto(784.9034202,35.33705007)(784.9084202,35.3120501)(784.91842041,35.29205373)
\curveto(784.92842018,35.25205016)(784.93342017,35.19705021)(784.93342041,35.12705373)
\curveto(784.93342017,35.05705035)(784.92842018,35.00205041)(784.91842041,34.96205373)
\curveto(784.9084202,34.9120505)(784.9084202,34.85705055)(784.91842041,34.79705373)
\curveto(784.92842018,34.73705067)(784.92342018,34.68205073)(784.90342041,34.63205373)
\curveto(784.87342023,34.50205091)(784.85342025,34.37705103)(784.84342041,34.25705373)
\curveto(784.83342027,34.13705127)(784.8084203,34.02205139)(784.76842041,33.91205373)
\curveto(784.64842046,33.54205187)(784.47842063,33.22205219)(784.25842041,32.95205373)
\curveto(784.03842107,32.68205273)(783.75842135,32.47205294)(783.41842041,32.32205373)
\curveto(783.29842181,32.27205314)(783.17342193,32.22705318)(783.04342041,32.18705373)
\curveto(782.91342219,32.15705325)(782.77842233,32.12205329)(782.63842041,32.08205373)
\curveto(782.58842252,32.07205334)(782.54842256,32.06705334)(782.51842041,32.06705373)
\curveto(782.47842263,32.06705334)(782.43342267,32.06205335)(782.38342041,32.05205373)
\curveto(782.35342275,32.04205337)(782.31842279,32.03705337)(782.27842041,32.03705373)
\curveto(782.22842288,32.03705337)(782.18842292,32.03205338)(782.15842041,32.02205373)
\lineto(781.99342041,32.02205373)
\curveto(781.91342319,32.00205341)(781.81342329,31.99705341)(781.69342041,32.00705373)
\curveto(781.56342354,32.01705339)(781.47342363,32.03205338)(781.42342041,32.05205373)
\curveto(781.33342377,32.07205334)(781.26842384,32.12705328)(781.22842041,32.21705373)
\curveto(781.2084239,32.24705316)(781.2034239,32.27705313)(781.21342041,32.30705373)
\curveto(781.21342389,32.33705307)(781.2084239,32.37705303)(781.19842041,32.42705373)
\curveto(781.18842392,32.46705294)(781.18342392,32.5070529)(781.18342041,32.54705373)
\lineto(781.18342041,32.69705373)
\curveto(781.18342392,32.81705259)(781.18842392,32.93705247)(781.19842041,33.05705373)
\curveto(781.19842391,33.18705222)(781.23342387,33.27705213)(781.30342041,33.32705373)
\curveto(781.36342374,33.36705204)(781.42342368,33.38705202)(781.48342041,33.38705373)
\curveto(781.54342356,33.38705202)(781.61342349,33.39705201)(781.69342041,33.41705373)
\curveto(781.72342338,33.42705198)(781.75842335,33.42705198)(781.79842041,33.41705373)
\curveto(781.82842328,33.41705199)(781.85342325,33.42205199)(781.87342041,33.43205373)
\lineto(782.08342041,33.43205373)
\curveto(782.13342297,33.45205196)(782.18342292,33.45705195)(782.23342041,33.44705373)
\curveto(782.27342283,33.44705196)(782.31842279,33.45705195)(782.36842041,33.47705373)
\curveto(782.49842261,33.5070519)(782.62342248,33.53705187)(782.74342041,33.56705373)
\curveto(782.85342225,33.59705181)(782.95842215,33.64205177)(783.05842041,33.70205373)
\curveto(783.34842176,33.87205154)(783.55342155,34.14205127)(783.67342041,34.51205373)
\curveto(783.69342141,34.56205085)(783.7084214,34.6120508)(783.71842041,34.66205373)
\curveto(783.71842139,34.72205069)(783.72842138,34.77705063)(783.74842041,34.82705373)
\lineto(783.74842041,34.90205373)
\curveto(783.75842135,34.97205044)(783.76842134,35.06705034)(783.77842041,35.18705373)
\curveto(783.77842133,35.31705009)(783.76842134,35.41704999)(783.74842041,35.48705373)
\curveto(783.72842138,35.55704985)(783.71342139,35.62704978)(783.70342041,35.69705373)
\curveto(783.68342142,35.77704963)(783.66342144,35.84704956)(783.64342041,35.90705373)
\curveto(783.48342162,36.28704912)(783.2084219,36.56204885)(782.81842041,36.73205373)
\curveto(782.68842242,36.78204863)(782.53342257,36.81704859)(782.35342041,36.83705373)
\curveto(782.17342293,36.86704854)(781.98842312,36.88204853)(781.79842041,36.88205373)
\curveto(781.59842351,36.89204852)(781.39842371,36.89204852)(781.19842041,36.88205373)
\lineto(780.62842041,36.88205373)
\lineto(776.38342041,36.88205373)
\lineto(774.83842041,36.88205373)
\curveto(774.72843038,36.88204853)(774.6084305,36.87704853)(774.47842041,36.86705373)
\curveto(774.34843076,36.85704855)(774.24343086,36.87704853)(774.16342041,36.92705373)
\curveto(774.09343101,36.98704842)(774.04343106,37.06704834)(774.01342041,37.16705373)
\curveto(774.01343109,37.18704822)(774.01343109,37.2070482)(774.01342041,37.22705373)
\curveto(774.01343109,37.24704816)(774.0084311,37.26704814)(773.99842041,37.28705373)
}
}
{
\newrgbcolor{curcolor}{0 0 0}
\pscustom[linestyle=none,fillstyle=solid,fillcolor=curcolor]
{
\newpath
\moveto(776.95342041,40.82072561)
\lineto(776.95342041,41.25572561)
\curveto(776.95342815,41.40572364)(776.99342811,41.51072354)(777.07342041,41.57072561)
\curveto(777.15342795,41.62072343)(777.25342785,41.6457234)(777.37342041,41.64572561)
\curveto(777.49342761,41.65572339)(777.61342749,41.66072339)(777.73342041,41.66072561)
\lineto(779.15842041,41.66072561)
\lineto(781.42342041,41.66072561)
\lineto(782.11342041,41.66072561)
\curveto(782.34342276,41.66072339)(782.54342256,41.68572336)(782.71342041,41.73572561)
\curveto(783.16342194,41.89572315)(783.47842163,42.19572285)(783.65842041,42.63572561)
\curveto(783.74842136,42.85572219)(783.78342132,43.12072193)(783.76342041,43.43072561)
\curveto(783.73342137,43.74072131)(783.67842143,43.99072106)(783.59842041,44.18072561)
\curveto(783.45842165,44.51072054)(783.28342182,44.77072028)(783.07342041,44.96072561)
\curveto(782.85342225,45.16071989)(782.56842254,45.31571973)(782.21842041,45.42572561)
\curveto(782.13842297,45.45571959)(782.05842305,45.47571957)(781.97842041,45.48572561)
\curveto(781.89842321,45.49571955)(781.81342329,45.51071954)(781.72342041,45.53072561)
\curveto(781.67342343,45.54071951)(781.62842348,45.54071951)(781.58842041,45.53072561)
\curveto(781.54842356,45.53071952)(781.5034236,45.54071951)(781.45342041,45.56072561)
\lineto(781.13842041,45.56072561)
\curveto(781.05842405,45.58071947)(780.96842414,45.58571946)(780.86842041,45.57572561)
\curveto(780.75842435,45.56571948)(780.65842445,45.56071949)(780.56842041,45.56072561)
\lineto(779.39842041,45.56072561)
\lineto(777.80842041,45.56072561)
\curveto(777.68842742,45.56071949)(777.56342754,45.55571949)(777.43342041,45.54572561)
\curveto(777.29342781,45.5457195)(777.18342792,45.57071948)(777.10342041,45.62072561)
\curveto(777.05342805,45.66071939)(777.02342808,45.70571934)(777.01342041,45.75572561)
\curveto(776.99342811,45.81571923)(776.97342813,45.88571916)(776.95342041,45.96572561)
\lineto(776.95342041,46.19072561)
\curveto(776.95342815,46.31071874)(776.95842815,46.41571863)(776.96842041,46.50572561)
\curveto(776.97842813,46.60571844)(777.02342808,46.68071837)(777.10342041,46.73072561)
\curveto(777.15342795,46.78071827)(777.22842788,46.80571824)(777.32842041,46.80572561)
\lineto(777.61342041,46.80572561)
\lineto(778.63342041,46.80572561)
\lineto(782.66842041,46.80572561)
\lineto(784.01842041,46.80572561)
\curveto(784.13842097,46.80571824)(784.25342085,46.80071825)(784.36342041,46.79072561)
\curveto(784.46342064,46.79071826)(784.53842057,46.75571829)(784.58842041,46.68572561)
\curveto(784.61842049,46.6457184)(784.64342046,46.58571846)(784.66342041,46.50572561)
\curveto(784.67342043,46.42571862)(784.68342042,46.33571871)(784.69342041,46.23572561)
\curveto(784.69342041,46.1457189)(784.68842042,46.05571899)(784.67842041,45.96572561)
\curveto(784.66842044,45.88571916)(784.64842046,45.82571922)(784.61842041,45.78572561)
\curveto(784.57842053,45.73571931)(784.51342059,45.69071936)(784.42342041,45.65072561)
\curveto(784.38342072,45.64071941)(784.32842078,45.63071942)(784.25842041,45.62072561)
\curveto(784.18842092,45.62071943)(784.12342098,45.61571943)(784.06342041,45.60572561)
\curveto(783.99342111,45.59571945)(783.93842117,45.57571947)(783.89842041,45.54572561)
\curveto(783.85842125,45.51571953)(783.84342126,45.47071958)(783.85342041,45.41072561)
\curveto(783.87342123,45.33071972)(783.93342117,45.2507198)(784.03342041,45.17072561)
\curveto(784.12342098,45.09071996)(784.19342091,45.01572003)(784.24342041,44.94572561)
\curveto(784.4034207,44.72572032)(784.54342056,44.47572057)(784.66342041,44.19572561)
\curveto(784.71342039,44.08572096)(784.74342036,43.97072108)(784.75342041,43.85072561)
\curveto(784.77342033,43.74072131)(784.79842031,43.62572142)(784.82842041,43.50572561)
\curveto(784.83842027,43.45572159)(784.83842027,43.40072165)(784.82842041,43.34072561)
\curveto(784.81842029,43.29072176)(784.82342028,43.24072181)(784.84342041,43.19072561)
\curveto(784.86342024,43.09072196)(784.86342024,43.00072205)(784.84342041,42.92072561)
\lineto(784.84342041,42.77072561)
\curveto(784.82342028,42.72072233)(784.81342029,42.66072239)(784.81342041,42.59072561)
\curveto(784.81342029,42.53072252)(784.8084203,42.47572257)(784.79842041,42.42572561)
\curveto(784.77842033,42.38572266)(784.76842034,42.3457227)(784.76842041,42.30572561)
\curveto(784.77842033,42.27572277)(784.77342033,42.23572281)(784.75342041,42.18572561)
\lineto(784.69342041,41.94572561)
\curveto(784.67342043,41.87572317)(784.64342046,41.80072325)(784.60342041,41.72072561)
\curveto(784.49342061,41.46072359)(784.34842076,41.24072381)(784.16842041,41.06072561)
\curveto(783.97842113,40.89072416)(783.75342135,40.7507243)(783.49342041,40.64072561)
\curveto(783.4034217,40.60072445)(783.31342179,40.57072448)(783.22342041,40.55072561)
\lineto(782.92342041,40.49072561)
\curveto(782.86342224,40.47072458)(782.8084223,40.46072459)(782.75842041,40.46072561)
\curveto(782.69842241,40.47072458)(782.63342247,40.46572458)(782.56342041,40.44572561)
\curveto(782.54342256,40.43572461)(782.51842259,40.43072462)(782.48842041,40.43072561)
\curveto(782.44842266,40.43072462)(782.41342269,40.42572462)(782.38342041,40.41572561)
\lineto(782.23342041,40.41572561)
\curveto(782.19342291,40.40572464)(782.14842296,40.40072465)(782.09842041,40.40072561)
\curveto(782.03842307,40.41072464)(781.98342312,40.41572463)(781.93342041,40.41572561)
\lineto(781.33342041,40.41572561)
\lineto(778.57342041,40.41572561)
\lineto(777.61342041,40.41572561)
\lineto(777.34342041,40.41572561)
\curveto(777.25342785,40.41572463)(777.17842793,40.43572461)(777.11842041,40.47572561)
\curveto(777.04842806,40.51572453)(776.99842811,40.59072446)(776.96842041,40.70072561)
\curveto(776.95842815,40.72072433)(776.95842815,40.74072431)(776.96842041,40.76072561)
\curveto(776.96842814,40.78072427)(776.96342814,40.80072425)(776.95342041,40.82072561)
}
}
{
\newrgbcolor{curcolor}{0 0 0}
\pscustom[linestyle=none,fillstyle=solid,fillcolor=curcolor]
{
\newpath
\moveto(776.80342041,52.39533498)
\curveto(776.78342832,53.02532975)(776.86842824,53.53032924)(777.05842041,53.91033498)
\curveto(777.24842786,54.29032848)(777.53342757,54.59532818)(777.91342041,54.82533498)
\curveto(778.01342709,54.88532789)(778.12342698,54.93032784)(778.24342041,54.96033498)
\curveto(778.35342675,55.00032777)(778.46842664,55.03532774)(778.58842041,55.06533498)
\curveto(778.77842633,55.11532766)(778.98342612,55.14532763)(779.20342041,55.15533498)
\curveto(779.42342568,55.16532761)(779.64842546,55.1703276)(779.87842041,55.17033498)
\lineto(781.48342041,55.17033498)
\lineto(783.82342041,55.17033498)
\curveto(783.99342111,55.1703276)(784.16342094,55.16532761)(784.33342041,55.15533498)
\curveto(784.5034206,55.15532762)(784.61342049,55.09032768)(784.66342041,54.96033498)
\curveto(784.68342042,54.91032786)(784.69342041,54.85532792)(784.69342041,54.79533498)
\curveto(784.7034204,54.74532803)(784.7084204,54.69032808)(784.70842041,54.63033498)
\curveto(784.7084204,54.50032827)(784.7034204,54.3753284)(784.69342041,54.25533498)
\curveto(784.69342041,54.13532864)(784.65342045,54.05032872)(784.57342041,54.00033498)
\curveto(784.5034206,53.95032882)(784.41342069,53.92532885)(784.30342041,53.92533498)
\lineto(783.97342041,53.92533498)
\lineto(782.68342041,53.92533498)
\lineto(780.23842041,53.92533498)
\curveto(779.96842514,53.92532885)(779.7034254,53.92032885)(779.44342041,53.91033498)
\curveto(779.17342593,53.90032887)(778.94342616,53.85532892)(778.75342041,53.77533498)
\curveto(778.55342655,53.69532908)(778.39342671,53.5753292)(778.27342041,53.41533498)
\curveto(778.14342696,53.25532952)(778.04342706,53.0703297)(777.97342041,52.86033498)
\curveto(777.95342715,52.80032997)(777.94342716,52.73533004)(777.94342041,52.66533498)
\curveto(777.93342717,52.60533017)(777.91842719,52.54533023)(777.89842041,52.48533498)
\curveto(777.88842722,52.43533034)(777.88842722,52.35533042)(777.89842041,52.24533498)
\curveto(777.89842721,52.14533063)(777.9034272,52.0753307)(777.91342041,52.03533498)
\curveto(777.93342717,51.99533078)(777.94342716,51.96033081)(777.94342041,51.93033498)
\curveto(777.93342717,51.90033087)(777.93342717,51.86533091)(777.94342041,51.82533498)
\curveto(777.97342713,51.69533108)(778.0084271,51.5703312)(778.04842041,51.45033498)
\curveto(778.07842703,51.34033143)(778.12342698,51.23533154)(778.18342041,51.13533498)
\curveto(778.2034269,51.09533168)(778.22342688,51.06033171)(778.24342041,51.03033498)
\curveto(778.26342684,51.00033177)(778.28342682,50.96533181)(778.30342041,50.92533498)
\curveto(778.55342655,50.5753322)(778.92842618,50.32033245)(779.42842041,50.16033498)
\curveto(779.5084256,50.13033264)(779.59342551,50.11033266)(779.68342041,50.10033498)
\curveto(779.76342534,50.09033268)(779.84342526,50.0753327)(779.92342041,50.05533498)
\curveto(779.97342513,50.03533274)(780.02342508,50.03033274)(780.07342041,50.04033498)
\curveto(780.11342499,50.05033272)(780.15342495,50.04533273)(780.19342041,50.02533498)
\lineto(780.50842041,50.02533498)
\curveto(780.53842457,50.01533276)(780.57342453,50.01033276)(780.61342041,50.01033498)
\curveto(780.65342445,50.02033275)(780.69842441,50.02533275)(780.74842041,50.02533498)
\lineto(781.19842041,50.02533498)
\lineto(782.63842041,50.02533498)
\lineto(783.95842041,50.02533498)
\lineto(784.30342041,50.02533498)
\curveto(784.41342069,50.02533275)(784.5034206,50.00033277)(784.57342041,49.95033498)
\curveto(784.65342045,49.90033287)(784.69342041,49.81033296)(784.69342041,49.68033498)
\curveto(784.7034204,49.56033321)(784.7084204,49.43533334)(784.70842041,49.30533498)
\curveto(784.7084204,49.22533355)(784.7034204,49.15033362)(784.69342041,49.08033498)
\curveto(784.68342042,49.01033376)(784.65842045,48.95033382)(784.61842041,48.90033498)
\curveto(784.56842054,48.82033395)(784.47342063,48.78033399)(784.33342041,48.78033498)
\lineto(783.92842041,48.78033498)
\lineto(782.15842041,48.78033498)
\lineto(778.52842041,48.78033498)
\lineto(777.61342041,48.78033498)
\lineto(777.34342041,48.78033498)
\curveto(777.25342785,48.78033399)(777.18342792,48.80033397)(777.13342041,48.84033498)
\curveto(777.07342803,48.8703339)(777.03342807,48.92033385)(777.01342041,48.99033498)
\curveto(777.0034281,49.03033374)(776.99342811,49.08533369)(776.98342041,49.15533498)
\curveto(776.97342813,49.23533354)(776.96842814,49.31533346)(776.96842041,49.39533498)
\curveto(776.96842814,49.4753333)(776.97342813,49.55033322)(776.98342041,49.62033498)
\curveto(776.99342811,49.70033307)(777.0084281,49.75533302)(777.02842041,49.78533498)
\curveto(777.09842801,49.89533288)(777.18842792,49.94533283)(777.29842041,49.93533498)
\curveto(777.39842771,49.92533285)(777.51342759,49.94033283)(777.64342041,49.98033498)
\curveto(777.7034274,50.00033277)(777.75342735,50.04033273)(777.79342041,50.10033498)
\curveto(777.8034273,50.22033255)(777.75842735,50.31533246)(777.65842041,50.38533498)
\curveto(777.55842755,50.46533231)(777.47842763,50.54533223)(777.41842041,50.62533498)
\curveto(777.31842779,50.76533201)(777.22842788,50.90533187)(777.14842041,51.04533498)
\curveto(777.05842805,51.19533158)(776.98342812,51.36533141)(776.92342041,51.55533498)
\curveto(776.89342821,51.63533114)(776.87342823,51.72033105)(776.86342041,51.81033498)
\curveto(776.85342825,51.91033086)(776.83842827,52.00533077)(776.81842041,52.09533498)
\curveto(776.8084283,52.14533063)(776.8034283,52.19533058)(776.80342041,52.24533498)
\lineto(776.80342041,52.39533498)
}
}
{
\newrgbcolor{curcolor}{0 0 0}
\pscustom[linestyle=none,fillstyle=solid,fillcolor=curcolor]
{
}
}
{
\newrgbcolor{curcolor}{0 0 0}
\pscustom[linestyle=none,fillstyle=solid,fillcolor=curcolor]
{
\newpath
\moveto(774.07342041,64.24510061)
\curveto(774.06343104,64.93509597)(774.18343092,65.53509537)(774.43342041,66.04510061)
\curveto(774.68343042,66.56509434)(775.01843009,66.96009395)(775.43842041,67.23010061)
\curveto(775.51842959,67.28009363)(775.6084295,67.32509358)(775.70842041,67.36510061)
\curveto(775.79842931,67.4050935)(775.89342921,67.45009346)(775.99342041,67.50010061)
\curveto(776.09342901,67.54009337)(776.19342891,67.57009334)(776.29342041,67.59010061)
\curveto(776.39342871,67.6100933)(776.49842861,67.63009328)(776.60842041,67.65010061)
\curveto(776.65842845,67.67009324)(776.7034284,67.67509323)(776.74342041,67.66510061)
\curveto(776.78342832,67.65509325)(776.82842828,67.66009325)(776.87842041,67.68010061)
\curveto(776.92842818,67.69009322)(777.01342809,67.69509321)(777.13342041,67.69510061)
\curveto(777.24342786,67.69509321)(777.32842778,67.69009322)(777.38842041,67.68010061)
\curveto(777.44842766,67.66009325)(777.5084276,67.65009326)(777.56842041,67.65010061)
\curveto(777.62842748,67.66009325)(777.68842742,67.65509325)(777.74842041,67.63510061)
\curveto(777.88842722,67.59509331)(778.02342708,67.56009335)(778.15342041,67.53010061)
\curveto(778.28342682,67.50009341)(778.4084267,67.46009345)(778.52842041,67.41010061)
\curveto(778.66842644,67.35009356)(778.79342631,67.28009363)(778.90342041,67.20010061)
\curveto(779.01342609,67.13009378)(779.12342598,67.05509385)(779.23342041,66.97510061)
\lineto(779.29342041,66.91510061)
\curveto(779.31342579,66.905094)(779.33342577,66.89009402)(779.35342041,66.87010061)
\curveto(779.51342559,66.75009416)(779.65842545,66.61509429)(779.78842041,66.46510061)
\curveto(779.91842519,66.31509459)(780.04342506,66.15509475)(780.16342041,65.98510061)
\curveto(780.38342472,65.67509523)(780.58842452,65.38009553)(780.77842041,65.10010061)
\curveto(780.91842419,64.87009604)(781.05342405,64.64009627)(781.18342041,64.41010061)
\curveto(781.31342379,64.19009672)(781.44842366,63.97009694)(781.58842041,63.75010061)
\curveto(781.75842335,63.50009741)(781.93842317,63.26009765)(782.12842041,63.03010061)
\curveto(782.31842279,62.8100981)(782.54342256,62.62009829)(782.80342041,62.46010061)
\curveto(782.86342224,62.42009849)(782.92342218,62.38509852)(782.98342041,62.35510061)
\curveto(783.03342207,62.32509858)(783.09842201,62.29509861)(783.17842041,62.26510061)
\curveto(783.24842186,62.24509866)(783.3084218,62.24009867)(783.35842041,62.25010061)
\curveto(783.42842168,62.27009864)(783.48342162,62.3050986)(783.52342041,62.35510061)
\curveto(783.55342155,62.4050985)(783.57342153,62.46509844)(783.58342041,62.53510061)
\lineto(783.58342041,62.77510061)
\lineto(783.58342041,63.52510061)
\lineto(783.58342041,66.33010061)
\lineto(783.58342041,66.99010061)
\curveto(783.58342152,67.08009383)(783.58842152,67.16509374)(783.59842041,67.24510061)
\curveto(783.59842151,67.32509358)(783.61842149,67.39009352)(783.65842041,67.44010061)
\curveto(783.69842141,67.49009342)(783.77342133,67.53009338)(783.88342041,67.56010061)
\curveto(783.98342112,67.60009331)(784.08342102,67.6100933)(784.18342041,67.59010061)
\lineto(784.31842041,67.59010061)
\curveto(784.38842072,67.57009334)(784.44842066,67.55009336)(784.49842041,67.53010061)
\curveto(784.54842056,67.5100934)(784.58842052,67.47509343)(784.61842041,67.42510061)
\curveto(784.65842045,67.37509353)(784.67842043,67.3050936)(784.67842041,67.21510061)
\lineto(784.67842041,66.94510061)
\lineto(784.67842041,66.04510061)
\lineto(784.67842041,62.53510061)
\lineto(784.67842041,61.47010061)
\curveto(784.67842043,61.39009952)(784.68342042,61.30009961)(784.69342041,61.20010061)
\curveto(784.69342041,61.10009981)(784.68342042,61.01509989)(784.66342041,60.94510061)
\curveto(784.59342051,60.73510017)(784.41342069,60.67010024)(784.12342041,60.75010061)
\curveto(784.08342102,60.76010015)(784.04842106,60.76010015)(784.01842041,60.75010061)
\curveto(783.97842113,60.75010016)(783.93342117,60.76010015)(783.88342041,60.78010061)
\curveto(783.8034213,60.80010011)(783.71842139,60.82010009)(783.62842041,60.84010061)
\curveto(783.53842157,60.86010005)(783.45342165,60.88510002)(783.37342041,60.91510061)
\curveto(782.88342222,61.07509983)(782.46842264,61.27509963)(782.12842041,61.51510061)
\curveto(781.87842323,61.69509921)(781.65342345,61.90009901)(781.45342041,62.13010061)
\curveto(781.24342386,62.36009855)(781.04842406,62.60009831)(780.86842041,62.85010061)
\curveto(780.68842442,63.1100978)(780.51842459,63.37509753)(780.35842041,63.64510061)
\curveto(780.18842492,63.92509698)(780.01342509,64.19509671)(779.83342041,64.45510061)
\curveto(779.75342535,64.56509634)(779.67842543,64.67009624)(779.60842041,64.77010061)
\curveto(779.53842557,64.88009603)(779.46342564,64.99009592)(779.38342041,65.10010061)
\curveto(779.35342575,65.14009577)(779.32342578,65.17509573)(779.29342041,65.20510061)
\curveto(779.25342585,65.24509566)(779.22342588,65.28509562)(779.20342041,65.32510061)
\curveto(779.09342601,65.46509544)(778.96842614,65.59009532)(778.82842041,65.70010061)
\curveto(778.79842631,65.72009519)(778.77342633,65.74509516)(778.75342041,65.77510061)
\curveto(778.72342638,65.8050951)(778.69342641,65.83009508)(778.66342041,65.85010061)
\curveto(778.56342654,65.93009498)(778.46342664,65.99509491)(778.36342041,66.04510061)
\curveto(778.26342684,66.1050948)(778.15342695,66.16009475)(778.03342041,66.21010061)
\curveto(777.96342714,66.24009467)(777.88842722,66.26009465)(777.80842041,66.27010061)
\lineto(777.56842041,66.33010061)
\lineto(777.47842041,66.33010061)
\curveto(777.44842766,66.34009457)(777.41842769,66.34509456)(777.38842041,66.34510061)
\curveto(777.31842779,66.36509454)(777.22342788,66.37009454)(777.10342041,66.36010061)
\curveto(776.97342813,66.36009455)(776.87342823,66.35009456)(776.80342041,66.33010061)
\curveto(776.72342838,66.3100946)(776.64842846,66.29009462)(776.57842041,66.27010061)
\curveto(776.49842861,66.26009465)(776.41842869,66.24009467)(776.33842041,66.21010061)
\curveto(776.09842901,66.10009481)(775.89842921,65.95009496)(775.73842041,65.76010061)
\curveto(775.56842954,65.58009533)(775.42842968,65.36009555)(775.31842041,65.10010061)
\curveto(775.29842981,65.03009588)(775.28342982,64.96009595)(775.27342041,64.89010061)
\curveto(775.25342985,64.82009609)(775.23342987,64.74509616)(775.21342041,64.66510061)
\curveto(775.19342991,64.58509632)(775.18342992,64.47509643)(775.18342041,64.33510061)
\curveto(775.18342992,64.2050967)(775.19342991,64.10009681)(775.21342041,64.02010061)
\curveto(775.22342988,63.96009695)(775.22842988,63.905097)(775.22842041,63.85510061)
\curveto(775.22842988,63.8050971)(775.23842987,63.75509715)(775.25842041,63.70510061)
\curveto(775.29842981,63.6050973)(775.33842977,63.5100974)(775.37842041,63.42010061)
\curveto(775.41842969,63.34009757)(775.46342964,63.26009765)(775.51342041,63.18010061)
\curveto(775.53342957,63.15009776)(775.55842955,63.12009779)(775.58842041,63.09010061)
\curveto(775.61842949,63.07009784)(775.64342946,63.04509786)(775.66342041,63.01510061)
\lineto(775.73842041,62.94010061)
\curveto(775.75842935,62.910098)(775.77842933,62.88509802)(775.79842041,62.86510061)
\lineto(776.00842041,62.71510061)
\curveto(776.06842904,62.67509823)(776.13342897,62.63009828)(776.20342041,62.58010061)
\curveto(776.29342881,62.52009839)(776.39842871,62.47009844)(776.51842041,62.43010061)
\curveto(776.62842848,62.40009851)(776.73842837,62.36509854)(776.84842041,62.32510061)
\curveto(776.95842815,62.28509862)(777.103428,62.26009865)(777.28342041,62.25010061)
\curveto(777.45342765,62.24009867)(777.57842753,62.2100987)(777.65842041,62.16010061)
\curveto(777.73842737,62.1100988)(777.78342732,62.03509887)(777.79342041,61.93510061)
\curveto(777.8034273,61.83509907)(777.8084273,61.72509918)(777.80842041,61.60510061)
\curveto(777.8084273,61.56509934)(777.81342729,61.52509938)(777.82342041,61.48510061)
\curveto(777.82342728,61.44509946)(777.81842729,61.4100995)(777.80842041,61.38010061)
\curveto(777.78842732,61.33009958)(777.77842733,61.28009963)(777.77842041,61.23010061)
\curveto(777.77842733,61.19009972)(777.76842734,61.15009976)(777.74842041,61.11010061)
\curveto(777.68842742,61.02009989)(777.55342755,60.97509993)(777.34342041,60.97510061)
\lineto(777.22342041,60.97510061)
\curveto(777.16342794,60.98509992)(777.103428,60.99009992)(777.04342041,60.99010061)
\curveto(776.97342813,61.00009991)(776.9084282,61.0100999)(776.84842041,61.02010061)
\curveto(776.73842837,61.04009987)(776.63842847,61.06009985)(776.54842041,61.08010061)
\curveto(776.44842866,61.10009981)(776.35342875,61.13009978)(776.26342041,61.17010061)
\curveto(776.19342891,61.19009972)(776.13342897,61.2100997)(776.08342041,61.23010061)
\lineto(775.90342041,61.29010061)
\curveto(775.64342946,61.4100995)(775.39842971,61.56509934)(775.16842041,61.75510061)
\curveto(774.93843017,61.95509895)(774.75343035,62.17009874)(774.61342041,62.40010061)
\curveto(774.53343057,62.5100984)(774.46843064,62.62509828)(774.41842041,62.74510061)
\lineto(774.26842041,63.13510061)
\curveto(774.21843089,63.24509766)(774.18843092,63.36009755)(774.17842041,63.48010061)
\curveto(774.15843095,63.60009731)(774.13343097,63.72509718)(774.10342041,63.85510061)
\curveto(774.103431,63.92509698)(774.103431,63.99009692)(774.10342041,64.05010061)
\curveto(774.09343101,64.1100968)(774.08343102,64.17509673)(774.07342041,64.24510061)
}
}
{
\newrgbcolor{curcolor}{0 0 0}
\pscustom[linestyle=none,fillstyle=solid,fillcolor=curcolor]
{
\newpath
\moveto(780.35842041,76.34470998)
\curveto(780.47842463,76.37470226)(780.61842449,76.39970223)(780.77842041,76.41970998)
\curveto(780.93842417,76.43970219)(781.103424,76.44970218)(781.27342041,76.44970998)
\curveto(781.44342366,76.44970218)(781.6084235,76.43970219)(781.76842041,76.41970998)
\curveto(781.92842318,76.39970223)(782.06842304,76.37470226)(782.18842041,76.34470998)
\curveto(782.32842278,76.30470233)(782.45342265,76.26970236)(782.56342041,76.23970998)
\curveto(782.67342243,76.20970242)(782.78342232,76.16970246)(782.89342041,76.11970998)
\curveto(783.53342157,75.84970278)(784.01842109,75.4347032)(784.34842041,74.87470998)
\curveto(784.4084207,74.79470384)(784.45842065,74.70970392)(784.49842041,74.61970998)
\curveto(784.52842058,74.5297041)(784.56342054,74.4297042)(784.60342041,74.31970998)
\curveto(784.65342045,74.20970442)(784.68842042,74.08970454)(784.70842041,73.95970998)
\curveto(784.73842037,73.83970479)(784.76842034,73.70970492)(784.79842041,73.56970998)
\curveto(784.81842029,73.50970512)(784.82342028,73.44970518)(784.81342041,73.38970998)
\curveto(784.8034203,73.33970529)(784.8084203,73.27970535)(784.82842041,73.20970998)
\curveto(784.83842027,73.18970544)(784.83842027,73.16470547)(784.82842041,73.13470998)
\curveto(784.82842028,73.10470553)(784.83342027,73.07970555)(784.84342041,73.05970998)
\lineto(784.84342041,72.90970998)
\curveto(784.85342025,72.83970579)(784.85342025,72.78970584)(784.84342041,72.75970998)
\curveto(784.83342027,72.71970591)(784.82842028,72.67470596)(784.82842041,72.62470998)
\curveto(784.83842027,72.58470605)(784.83842027,72.54470609)(784.82842041,72.50470998)
\curveto(784.8084203,72.41470622)(784.79342031,72.32470631)(784.78342041,72.23470998)
\curveto(784.78342032,72.14470649)(784.77342033,72.05470658)(784.75342041,71.96470998)
\curveto(784.72342038,71.87470676)(784.69842041,71.78470685)(784.67842041,71.69470998)
\curveto(784.65842045,71.60470703)(784.62842048,71.51970711)(784.58842041,71.43970998)
\curveto(784.47842063,71.19970743)(784.34842076,70.97470766)(784.19842041,70.76470998)
\curveto(784.03842107,70.55470808)(783.85842125,70.37470826)(783.65842041,70.22470998)
\curveto(783.48842162,70.10470853)(783.31342179,69.99970863)(783.13342041,69.90970998)
\curveto(782.95342215,69.81970881)(782.76342234,69.7297089)(782.56342041,69.63970998)
\curveto(782.46342264,69.59970903)(782.36342274,69.56470907)(782.26342041,69.53470998)
\curveto(782.15342295,69.51470912)(782.04342306,69.48970914)(781.93342041,69.45970998)
\curveto(781.79342331,69.41970921)(781.65342345,69.39470924)(781.51342041,69.38470998)
\curveto(781.37342373,69.37470926)(781.23342387,69.35470928)(781.09342041,69.32470998)
\curveto(780.98342412,69.31470932)(780.88342422,69.30470933)(780.79342041,69.29470998)
\curveto(780.69342441,69.29470934)(780.59342451,69.28470935)(780.49342041,69.26470998)
\lineto(780.40342041,69.26470998)
\curveto(780.37342473,69.27470936)(780.34842476,69.27470936)(780.32842041,69.26470998)
\lineto(780.11842041,69.26470998)
\curveto(780.05842505,69.24470939)(779.99342511,69.2347094)(779.92342041,69.23470998)
\curveto(779.84342526,69.24470939)(779.76842534,69.24970938)(779.69842041,69.24970998)
\lineto(779.54842041,69.24970998)
\curveto(779.49842561,69.24970938)(779.44842566,69.25470938)(779.39842041,69.26470998)
\lineto(779.02342041,69.26470998)
\curveto(778.99342611,69.27470936)(778.95842615,69.27470936)(778.91842041,69.26470998)
\curveto(778.87842623,69.26470937)(778.83842627,69.26970936)(778.79842041,69.27970998)
\curveto(778.68842642,69.29970933)(778.57842653,69.31470932)(778.46842041,69.32470998)
\curveto(778.34842676,69.3347093)(778.23342687,69.34470929)(778.12342041,69.35470998)
\curveto(777.97342713,69.39470924)(777.82842728,69.41970921)(777.68842041,69.42970998)
\curveto(777.53842757,69.44970918)(777.39342771,69.47970915)(777.25342041,69.51970998)
\curveto(776.95342815,69.60970902)(776.66842844,69.70470893)(776.39842041,69.80470998)
\curveto(776.12842898,69.90470873)(775.87842923,70.0297086)(775.64842041,70.17970998)
\curveto(775.32842978,70.37970825)(775.04843006,70.62470801)(774.80842041,70.91470998)
\curveto(774.56843054,71.20470743)(774.38343072,71.54470709)(774.25342041,71.93470998)
\curveto(774.21343089,72.04470659)(774.18843092,72.15470648)(774.17842041,72.26470998)
\curveto(774.15843095,72.38470625)(774.13343097,72.50470613)(774.10342041,72.62470998)
\curveto(774.09343101,72.69470594)(774.08843102,72.75970587)(774.08842041,72.81970998)
\curveto(774.08843102,72.87970575)(774.08343102,72.94470569)(774.07342041,73.01470998)
\curveto(774.05343105,73.71470492)(774.16843094,74.28970434)(774.41842041,74.73970998)
\curveto(774.66843044,75.18970344)(775.01843009,75.5347031)(775.46842041,75.77470998)
\curveto(775.69842941,75.88470275)(775.97342913,75.98470265)(776.29342041,76.07470998)
\curveto(776.36342874,76.09470254)(776.43842867,76.09470254)(776.51842041,76.07470998)
\curveto(776.58842852,76.06470257)(776.63842847,76.03970259)(776.66842041,75.99970998)
\curveto(776.69842841,75.96970266)(776.72342838,75.90970272)(776.74342041,75.81970998)
\curveto(776.75342835,75.7297029)(776.76342834,75.629703)(776.77342041,75.51970998)
\curveto(776.77342833,75.41970321)(776.76842834,75.31970331)(776.75842041,75.21970998)
\curveto(776.74842836,75.1297035)(776.72842838,75.06470357)(776.69842041,75.02470998)
\curveto(776.62842848,74.91470372)(776.51842859,74.8347038)(776.36842041,74.78470998)
\curveto(776.21842889,74.74470389)(776.08842902,74.68970394)(775.97842041,74.61970998)
\curveto(775.66842944,74.4297042)(775.43842967,74.14970448)(775.28842041,73.77970998)
\curveto(775.25842985,73.70970492)(775.23842987,73.634705)(775.22842041,73.55470998)
\curveto(775.21842989,73.48470515)(775.2034299,73.40970522)(775.18342041,73.32970998)
\curveto(775.17342993,73.27970535)(775.16842994,73.20970542)(775.16842041,73.11970998)
\curveto(775.16842994,73.03970559)(775.17342993,72.97470566)(775.18342041,72.92470998)
\curveto(775.2034299,72.88470575)(775.2084299,72.84970578)(775.19842041,72.81970998)
\curveto(775.18842992,72.78970584)(775.18842992,72.75470588)(775.19842041,72.71470998)
\lineto(775.25842041,72.47470998)
\curveto(775.27842983,72.40470623)(775.3034298,72.3347063)(775.33342041,72.26470998)
\curveto(775.49342961,71.88470675)(775.7034294,71.59470704)(775.96342041,71.39470998)
\curveto(776.22342888,71.20470743)(776.53842857,71.0297076)(776.90842041,70.86970998)
\curveto(776.98842812,70.83970779)(777.06842804,70.81470782)(777.14842041,70.79470998)
\curveto(777.22842788,70.78470785)(777.3084278,70.76470787)(777.38842041,70.73470998)
\curveto(777.49842761,70.70470793)(777.61342749,70.67970795)(777.73342041,70.65970998)
\curveto(777.85342725,70.64970798)(777.97342713,70.629708)(778.09342041,70.59970998)
\curveto(778.14342696,70.57970805)(778.19342691,70.56970806)(778.24342041,70.56970998)
\curveto(778.29342681,70.57970805)(778.34342676,70.57470806)(778.39342041,70.55470998)
\curveto(778.45342665,70.54470809)(778.53342657,70.54470809)(778.63342041,70.55470998)
\curveto(778.72342638,70.56470807)(778.77842633,70.57970805)(778.79842041,70.59970998)
\curveto(778.83842627,70.61970801)(778.85842625,70.64970798)(778.85842041,70.68970998)
\curveto(778.85842625,70.73970789)(778.84842626,70.78470785)(778.82842041,70.82470998)
\curveto(778.78842632,70.89470774)(778.74342636,70.95470768)(778.69342041,71.00470998)
\curveto(778.64342646,71.05470758)(778.59342651,71.11470752)(778.54342041,71.18470998)
\lineto(778.48342041,71.24470998)
\curveto(778.45342665,71.27470736)(778.42842668,71.30470733)(778.40842041,71.33470998)
\curveto(778.24842686,71.56470707)(778.11342699,71.83970679)(778.00342041,72.15970998)
\curveto(777.98342712,72.2297064)(777.96842714,72.29970633)(777.95842041,72.36970998)
\curveto(777.94842716,72.43970619)(777.93342717,72.51470612)(777.91342041,72.59470998)
\curveto(777.91342719,72.634706)(777.9084272,72.66970596)(777.89842041,72.69970998)
\curveto(777.88842722,72.7297059)(777.88842722,72.76470587)(777.89842041,72.80470998)
\curveto(777.89842721,72.85470578)(777.88842722,72.89470574)(777.86842041,72.92470998)
\lineto(777.86842041,73.08970998)
\lineto(777.86842041,73.17970998)
\curveto(777.85842725,73.2297054)(777.85842725,73.26970536)(777.86842041,73.29970998)
\curveto(777.87842723,73.34970528)(777.88342722,73.39970523)(777.88342041,73.44970998)
\curveto(777.87342723,73.50970512)(777.87342723,73.56470507)(777.88342041,73.61470998)
\curveto(777.91342719,73.72470491)(777.93342717,73.8297048)(777.94342041,73.92970998)
\curveto(777.95342715,74.03970459)(777.97842713,74.14470449)(778.01842041,74.24470998)
\curveto(778.15842695,74.66470397)(778.34342676,75.00970362)(778.57342041,75.27970998)
\curveto(778.79342631,75.54970308)(779.07842603,75.78970284)(779.42842041,75.99970998)
\curveto(779.56842554,76.07970255)(779.71842539,76.14470249)(779.87842041,76.19470998)
\curveto(780.02842508,76.24470239)(780.18842492,76.29470234)(780.35842041,76.34470998)
\moveto(781.66342041,75.09970998)
\curveto(781.61342349,75.10970352)(781.56842354,75.11470352)(781.52842041,75.11470998)
\lineto(781.37842041,75.11470998)
\curveto(781.06842404,75.11470352)(780.78342432,75.07470356)(780.52342041,74.99470998)
\curveto(780.46342464,74.97470366)(780.4084247,74.95470368)(780.35842041,74.93470998)
\curveto(780.29842481,74.92470371)(780.24342486,74.90970372)(780.19342041,74.88970998)
\curveto(779.7034254,74.66970396)(779.35342575,74.32470431)(779.14342041,73.85470998)
\curveto(779.11342599,73.77470486)(779.08842602,73.69470494)(779.06842041,73.61470998)
\lineto(779.00842041,73.37470998)
\curveto(778.98842612,73.29470534)(778.97842613,73.20470543)(778.97842041,73.10470998)
\lineto(778.97842041,72.78970998)
\curveto(778.99842611,72.76970586)(779.0084261,72.7297059)(779.00842041,72.66970998)
\curveto(778.99842611,72.61970601)(778.99842611,72.57470606)(779.00842041,72.53470998)
\lineto(779.06842041,72.29470998)
\curveto(779.07842603,72.22470641)(779.09842601,72.15470648)(779.12842041,72.08470998)
\curveto(779.38842572,71.48470715)(779.85342525,71.07970755)(780.52342041,70.86970998)
\curveto(780.6034245,70.83970779)(780.68342442,70.81970781)(780.76342041,70.80970998)
\curveto(780.84342426,70.79970783)(780.92842418,70.78470785)(781.01842041,70.76470998)
\lineto(781.16842041,70.76470998)
\curveto(781.2084239,70.75470788)(781.27842383,70.74970788)(781.37842041,70.74970998)
\curveto(781.6084235,70.74970788)(781.8034233,70.76970786)(781.96342041,70.80970998)
\curveto(782.03342307,70.8297078)(782.09842301,70.84470779)(782.15842041,70.85470998)
\curveto(782.21842289,70.86470777)(782.28342282,70.88470775)(782.35342041,70.91470998)
\curveto(782.63342247,71.02470761)(782.87842223,71.16970746)(783.08842041,71.34970998)
\curveto(783.28842182,71.5297071)(783.44842166,71.76470687)(783.56842041,72.05470998)
\lineto(783.65842041,72.29470998)
\lineto(783.71842041,72.53470998)
\curveto(783.73842137,72.58470605)(783.74342136,72.62470601)(783.73342041,72.65470998)
\curveto(783.72342138,72.69470594)(783.72842138,72.73970589)(783.74842041,72.78970998)
\curveto(783.75842135,72.81970581)(783.76342134,72.87470576)(783.76342041,72.95470998)
\curveto(783.76342134,73.0347056)(783.75842135,73.09470554)(783.74842041,73.13470998)
\curveto(783.72842138,73.24470539)(783.71342139,73.34970528)(783.70342041,73.44970998)
\curveto(783.69342141,73.54970508)(783.66342144,73.64470499)(783.61342041,73.73470998)
\curveto(783.41342169,74.26470437)(783.03842207,74.65470398)(782.48842041,74.90470998)
\curveto(782.38842272,74.94470369)(782.28342282,74.97470366)(782.17342041,74.99470998)
\lineto(781.84342041,75.08470998)
\curveto(781.76342334,75.08470355)(781.7034234,75.08970354)(781.66342041,75.09970998)
}
}
{
\newrgbcolor{curcolor}{0 0 0}
\pscustom[linestyle=none,fillstyle=solid,fillcolor=curcolor]
{
\newpath
\moveto(783.04342041,78.63431936)
\lineto(783.04342041,79.26431936)
\lineto(783.04342041,79.45931936)
\curveto(783.04342206,79.52931683)(783.05342205,79.58931677)(783.07342041,79.63931936)
\curveto(783.11342199,79.70931665)(783.15342195,79.7593166)(783.19342041,79.78931936)
\curveto(783.24342186,79.82931653)(783.3084218,79.84931651)(783.38842041,79.84931936)
\curveto(783.46842164,79.8593165)(783.55342155,79.86431649)(783.64342041,79.86431936)
\lineto(784.36342041,79.86431936)
\curveto(784.84342026,79.86431649)(785.25341985,79.80431655)(785.59342041,79.68431936)
\curveto(785.93341917,79.56431679)(786.2084189,79.36931699)(786.41842041,79.09931936)
\curveto(786.46841864,79.02931733)(786.51341859,78.9593174)(786.55342041,78.88931936)
\curveto(786.6034185,78.82931753)(786.64841846,78.7543176)(786.68842041,78.66431936)
\curveto(786.69841841,78.64431771)(786.7084184,78.61431774)(786.71842041,78.57431936)
\curveto(786.73841837,78.53431782)(786.74341836,78.48931787)(786.73342041,78.43931936)
\curveto(786.7034184,78.34931801)(786.62841848,78.29431806)(786.50842041,78.27431936)
\curveto(786.39841871,78.2543181)(786.3034188,78.26931809)(786.22342041,78.31931936)
\curveto(786.15341895,78.34931801)(786.08841902,78.39431796)(786.02842041,78.45431936)
\curveto(785.97841913,78.52431783)(785.92841918,78.58931777)(785.87842041,78.64931936)
\curveto(785.82841928,78.71931764)(785.75341935,78.77931758)(785.65342041,78.82931936)
\curveto(785.56341954,78.88931747)(785.47341963,78.93931742)(785.38342041,78.97931936)
\curveto(785.35341975,78.99931736)(785.29341981,79.02431733)(785.20342041,79.05431936)
\curveto(785.12341998,79.08431727)(785.05342005,79.08931727)(784.99342041,79.06931936)
\curveto(784.85342025,79.03931732)(784.76342034,78.97931738)(784.72342041,78.88931936)
\curveto(784.69342041,78.80931755)(784.67842043,78.71931764)(784.67842041,78.61931936)
\curveto(784.67842043,78.51931784)(784.65342045,78.43431792)(784.60342041,78.36431936)
\curveto(784.53342057,78.27431808)(784.39342071,78.22931813)(784.18342041,78.22931936)
\lineto(783.62842041,78.22931936)
\lineto(783.40342041,78.22931936)
\curveto(783.32342178,78.23931812)(783.25842185,78.2593181)(783.20842041,78.28931936)
\curveto(783.12842198,78.34931801)(783.08342202,78.41931794)(783.07342041,78.49931936)
\curveto(783.06342204,78.51931784)(783.05842205,78.53931782)(783.05842041,78.55931936)
\curveto(783.05842205,78.58931777)(783.05342205,78.61431774)(783.04342041,78.63431936)
}
}
{
\newrgbcolor{curcolor}{0 0 0}
\pscustom[linestyle=none,fillstyle=solid,fillcolor=curcolor]
{
}
}
{
\newrgbcolor{curcolor}{0 0 0}
\pscustom[linestyle=none,fillstyle=solid,fillcolor=curcolor]
{
\newpath
\moveto(774.07342041,89.26463186)
\curveto(774.06343104,89.95462722)(774.18343092,90.55462662)(774.43342041,91.06463186)
\curveto(774.68343042,91.58462559)(775.01843009,91.9796252)(775.43842041,92.24963186)
\curveto(775.51842959,92.29962488)(775.6084295,92.34462483)(775.70842041,92.38463186)
\curveto(775.79842931,92.42462475)(775.89342921,92.46962471)(775.99342041,92.51963186)
\curveto(776.09342901,92.55962462)(776.19342891,92.58962459)(776.29342041,92.60963186)
\curveto(776.39342871,92.62962455)(776.49842861,92.64962453)(776.60842041,92.66963186)
\curveto(776.65842845,92.68962449)(776.7034284,92.69462448)(776.74342041,92.68463186)
\curveto(776.78342832,92.6746245)(776.82842828,92.6796245)(776.87842041,92.69963186)
\curveto(776.92842818,92.70962447)(777.01342809,92.71462446)(777.13342041,92.71463186)
\curveto(777.24342786,92.71462446)(777.32842778,92.70962447)(777.38842041,92.69963186)
\curveto(777.44842766,92.6796245)(777.5084276,92.66962451)(777.56842041,92.66963186)
\curveto(777.62842748,92.6796245)(777.68842742,92.6746245)(777.74842041,92.65463186)
\curveto(777.88842722,92.61462456)(778.02342708,92.5796246)(778.15342041,92.54963186)
\curveto(778.28342682,92.51962466)(778.4084267,92.4796247)(778.52842041,92.42963186)
\curveto(778.66842644,92.36962481)(778.79342631,92.29962488)(778.90342041,92.21963186)
\curveto(779.01342609,92.14962503)(779.12342598,92.0746251)(779.23342041,91.99463186)
\lineto(779.29342041,91.93463186)
\curveto(779.31342579,91.92462525)(779.33342577,91.90962527)(779.35342041,91.88963186)
\curveto(779.51342559,91.76962541)(779.65842545,91.63462554)(779.78842041,91.48463186)
\curveto(779.91842519,91.33462584)(780.04342506,91.174626)(780.16342041,91.00463186)
\curveto(780.38342472,90.69462648)(780.58842452,90.39962678)(780.77842041,90.11963186)
\curveto(780.91842419,89.88962729)(781.05342405,89.65962752)(781.18342041,89.42963186)
\curveto(781.31342379,89.20962797)(781.44842366,88.98962819)(781.58842041,88.76963186)
\curveto(781.75842335,88.51962866)(781.93842317,88.2796289)(782.12842041,88.04963186)
\curveto(782.31842279,87.82962935)(782.54342256,87.63962954)(782.80342041,87.47963186)
\curveto(782.86342224,87.43962974)(782.92342218,87.40462977)(782.98342041,87.37463186)
\curveto(783.03342207,87.34462983)(783.09842201,87.31462986)(783.17842041,87.28463186)
\curveto(783.24842186,87.26462991)(783.3084218,87.25962992)(783.35842041,87.26963186)
\curveto(783.42842168,87.28962989)(783.48342162,87.32462985)(783.52342041,87.37463186)
\curveto(783.55342155,87.42462975)(783.57342153,87.48462969)(783.58342041,87.55463186)
\lineto(783.58342041,87.79463186)
\lineto(783.58342041,88.54463186)
\lineto(783.58342041,91.34963186)
\lineto(783.58342041,92.00963186)
\curveto(783.58342152,92.09962508)(783.58842152,92.18462499)(783.59842041,92.26463186)
\curveto(783.59842151,92.34462483)(783.61842149,92.40962477)(783.65842041,92.45963186)
\curveto(783.69842141,92.50962467)(783.77342133,92.54962463)(783.88342041,92.57963186)
\curveto(783.98342112,92.61962456)(784.08342102,92.62962455)(784.18342041,92.60963186)
\lineto(784.31842041,92.60963186)
\curveto(784.38842072,92.58962459)(784.44842066,92.56962461)(784.49842041,92.54963186)
\curveto(784.54842056,92.52962465)(784.58842052,92.49462468)(784.61842041,92.44463186)
\curveto(784.65842045,92.39462478)(784.67842043,92.32462485)(784.67842041,92.23463186)
\lineto(784.67842041,91.96463186)
\lineto(784.67842041,91.06463186)
\lineto(784.67842041,87.55463186)
\lineto(784.67842041,86.48963186)
\curveto(784.67842043,86.40963077)(784.68342042,86.31963086)(784.69342041,86.21963186)
\curveto(784.69342041,86.11963106)(784.68342042,86.03463114)(784.66342041,85.96463186)
\curveto(784.59342051,85.75463142)(784.41342069,85.68963149)(784.12342041,85.76963186)
\curveto(784.08342102,85.7796314)(784.04842106,85.7796314)(784.01842041,85.76963186)
\curveto(783.97842113,85.76963141)(783.93342117,85.7796314)(783.88342041,85.79963186)
\curveto(783.8034213,85.81963136)(783.71842139,85.83963134)(783.62842041,85.85963186)
\curveto(783.53842157,85.8796313)(783.45342165,85.90463127)(783.37342041,85.93463186)
\curveto(782.88342222,86.09463108)(782.46842264,86.29463088)(782.12842041,86.53463186)
\curveto(781.87842323,86.71463046)(781.65342345,86.91963026)(781.45342041,87.14963186)
\curveto(781.24342386,87.3796298)(781.04842406,87.61962956)(780.86842041,87.86963186)
\curveto(780.68842442,88.12962905)(780.51842459,88.39462878)(780.35842041,88.66463186)
\curveto(780.18842492,88.94462823)(780.01342509,89.21462796)(779.83342041,89.47463186)
\curveto(779.75342535,89.58462759)(779.67842543,89.68962749)(779.60842041,89.78963186)
\curveto(779.53842557,89.89962728)(779.46342564,90.00962717)(779.38342041,90.11963186)
\curveto(779.35342575,90.15962702)(779.32342578,90.19462698)(779.29342041,90.22463186)
\curveto(779.25342585,90.26462691)(779.22342588,90.30462687)(779.20342041,90.34463186)
\curveto(779.09342601,90.48462669)(778.96842614,90.60962657)(778.82842041,90.71963186)
\curveto(778.79842631,90.73962644)(778.77342633,90.76462641)(778.75342041,90.79463186)
\curveto(778.72342638,90.82462635)(778.69342641,90.84962633)(778.66342041,90.86963186)
\curveto(778.56342654,90.94962623)(778.46342664,91.01462616)(778.36342041,91.06463186)
\curveto(778.26342684,91.12462605)(778.15342695,91.179626)(778.03342041,91.22963186)
\curveto(777.96342714,91.25962592)(777.88842722,91.2796259)(777.80842041,91.28963186)
\lineto(777.56842041,91.34963186)
\lineto(777.47842041,91.34963186)
\curveto(777.44842766,91.35962582)(777.41842769,91.36462581)(777.38842041,91.36463186)
\curveto(777.31842779,91.38462579)(777.22342788,91.38962579)(777.10342041,91.37963186)
\curveto(776.97342813,91.3796258)(776.87342823,91.36962581)(776.80342041,91.34963186)
\curveto(776.72342838,91.32962585)(776.64842846,91.30962587)(776.57842041,91.28963186)
\curveto(776.49842861,91.2796259)(776.41842869,91.25962592)(776.33842041,91.22963186)
\curveto(776.09842901,91.11962606)(775.89842921,90.96962621)(775.73842041,90.77963186)
\curveto(775.56842954,90.59962658)(775.42842968,90.3796268)(775.31842041,90.11963186)
\curveto(775.29842981,90.04962713)(775.28342982,89.9796272)(775.27342041,89.90963186)
\curveto(775.25342985,89.83962734)(775.23342987,89.76462741)(775.21342041,89.68463186)
\curveto(775.19342991,89.60462757)(775.18342992,89.49462768)(775.18342041,89.35463186)
\curveto(775.18342992,89.22462795)(775.19342991,89.11962806)(775.21342041,89.03963186)
\curveto(775.22342988,88.9796282)(775.22842988,88.92462825)(775.22842041,88.87463186)
\curveto(775.22842988,88.82462835)(775.23842987,88.7746284)(775.25842041,88.72463186)
\curveto(775.29842981,88.62462855)(775.33842977,88.52962865)(775.37842041,88.43963186)
\curveto(775.41842969,88.35962882)(775.46342964,88.2796289)(775.51342041,88.19963186)
\curveto(775.53342957,88.16962901)(775.55842955,88.13962904)(775.58842041,88.10963186)
\curveto(775.61842949,88.08962909)(775.64342946,88.06462911)(775.66342041,88.03463186)
\lineto(775.73842041,87.95963186)
\curveto(775.75842935,87.92962925)(775.77842933,87.90462927)(775.79842041,87.88463186)
\lineto(776.00842041,87.73463186)
\curveto(776.06842904,87.69462948)(776.13342897,87.64962953)(776.20342041,87.59963186)
\curveto(776.29342881,87.53962964)(776.39842871,87.48962969)(776.51842041,87.44963186)
\curveto(776.62842848,87.41962976)(776.73842837,87.38462979)(776.84842041,87.34463186)
\curveto(776.95842815,87.30462987)(777.103428,87.2796299)(777.28342041,87.26963186)
\curveto(777.45342765,87.25962992)(777.57842753,87.22962995)(777.65842041,87.17963186)
\curveto(777.73842737,87.12963005)(777.78342732,87.05463012)(777.79342041,86.95463186)
\curveto(777.8034273,86.85463032)(777.8084273,86.74463043)(777.80842041,86.62463186)
\curveto(777.8084273,86.58463059)(777.81342729,86.54463063)(777.82342041,86.50463186)
\curveto(777.82342728,86.46463071)(777.81842729,86.42963075)(777.80842041,86.39963186)
\curveto(777.78842732,86.34963083)(777.77842733,86.29963088)(777.77842041,86.24963186)
\curveto(777.77842733,86.20963097)(777.76842734,86.16963101)(777.74842041,86.12963186)
\curveto(777.68842742,86.03963114)(777.55342755,85.99463118)(777.34342041,85.99463186)
\lineto(777.22342041,85.99463186)
\curveto(777.16342794,86.00463117)(777.103428,86.00963117)(777.04342041,86.00963186)
\curveto(776.97342813,86.01963116)(776.9084282,86.02963115)(776.84842041,86.03963186)
\curveto(776.73842837,86.05963112)(776.63842847,86.0796311)(776.54842041,86.09963186)
\curveto(776.44842866,86.11963106)(776.35342875,86.14963103)(776.26342041,86.18963186)
\curveto(776.19342891,86.20963097)(776.13342897,86.22963095)(776.08342041,86.24963186)
\lineto(775.90342041,86.30963186)
\curveto(775.64342946,86.42963075)(775.39842971,86.58463059)(775.16842041,86.77463186)
\curveto(774.93843017,86.9746302)(774.75343035,87.18962999)(774.61342041,87.41963186)
\curveto(774.53343057,87.52962965)(774.46843064,87.64462953)(774.41842041,87.76463186)
\lineto(774.26842041,88.15463186)
\curveto(774.21843089,88.26462891)(774.18843092,88.3796288)(774.17842041,88.49963186)
\curveto(774.15843095,88.61962856)(774.13343097,88.74462843)(774.10342041,88.87463186)
\curveto(774.103431,88.94462823)(774.103431,89.00962817)(774.10342041,89.06963186)
\curveto(774.09343101,89.12962805)(774.08343102,89.19462798)(774.07342041,89.26463186)
}
}
{
\newrgbcolor{curcolor}{0 0 0}
\pscustom[linestyle=none,fillstyle=solid,fillcolor=curcolor]
{
\newpath
\moveto(779.59342041,101.36424123)
\lineto(779.84842041,101.36424123)
\curveto(779.92842518,101.37423353)(780.0034251,101.36923353)(780.07342041,101.34924123)
\lineto(780.31342041,101.34924123)
\lineto(780.47842041,101.34924123)
\curveto(780.57842453,101.32923357)(780.68342442,101.31923358)(780.79342041,101.31924123)
\curveto(780.89342421,101.31923358)(780.99342411,101.30923359)(781.09342041,101.28924123)
\lineto(781.24342041,101.28924123)
\curveto(781.38342372,101.25923364)(781.52342358,101.23923366)(781.66342041,101.22924123)
\curveto(781.79342331,101.21923368)(781.92342318,101.19423371)(782.05342041,101.15424123)
\curveto(782.13342297,101.13423377)(782.21842289,101.11423379)(782.30842041,101.09424123)
\lineto(782.54842041,101.03424123)
\lineto(782.84842041,100.91424123)
\curveto(782.93842217,100.88423402)(783.02842208,100.84923405)(783.11842041,100.80924123)
\curveto(783.33842177,100.70923419)(783.55342155,100.57423433)(783.76342041,100.40424123)
\curveto(783.97342113,100.24423466)(784.14342096,100.06923483)(784.27342041,99.87924123)
\curveto(784.31342079,99.82923507)(784.35342075,99.76923513)(784.39342041,99.69924123)
\curveto(784.42342068,99.63923526)(784.45842065,99.57923532)(784.49842041,99.51924123)
\curveto(784.54842056,99.43923546)(784.58842052,99.34423556)(784.61842041,99.23424123)
\curveto(784.64842046,99.12423578)(784.67842043,99.01923588)(784.70842041,98.91924123)
\curveto(784.74842036,98.80923609)(784.77342033,98.6992362)(784.78342041,98.58924123)
\curveto(784.79342031,98.47923642)(784.8084203,98.36423654)(784.82842041,98.24424123)
\curveto(784.83842027,98.2042367)(784.83842027,98.15923674)(784.82842041,98.10924123)
\curveto(784.82842028,98.06923683)(784.83342027,98.02923687)(784.84342041,97.98924123)
\curveto(784.85342025,97.94923695)(784.85842025,97.89423701)(784.85842041,97.82424123)
\curveto(784.85842025,97.75423715)(784.85342025,97.7042372)(784.84342041,97.67424123)
\curveto(784.82342028,97.62423728)(784.81842029,97.57923732)(784.82842041,97.53924123)
\curveto(784.83842027,97.4992374)(784.83842027,97.46423744)(784.82842041,97.43424123)
\lineto(784.82842041,97.34424123)
\curveto(784.8084203,97.28423762)(784.79342031,97.21923768)(784.78342041,97.14924123)
\curveto(784.78342032,97.08923781)(784.77842033,97.02423788)(784.76842041,96.95424123)
\curveto(784.71842039,96.78423812)(784.66842044,96.62423828)(784.61842041,96.47424123)
\curveto(784.56842054,96.32423858)(784.5034206,96.17923872)(784.42342041,96.03924123)
\curveto(784.38342072,95.98923891)(784.35342075,95.93423897)(784.33342041,95.87424123)
\curveto(784.3034208,95.82423908)(784.26842084,95.77423913)(784.22842041,95.72424123)
\curveto(784.04842106,95.48423942)(783.82842128,95.28423962)(783.56842041,95.12424123)
\curveto(783.3084218,94.96423994)(783.02342208,94.82424008)(782.71342041,94.70424123)
\curveto(782.57342253,94.64424026)(782.43342267,94.5992403)(782.29342041,94.56924123)
\curveto(782.14342296,94.53924036)(781.98842312,94.5042404)(781.82842041,94.46424123)
\curveto(781.71842339,94.44424046)(781.6084235,94.42924047)(781.49842041,94.41924123)
\curveto(781.38842372,94.40924049)(781.27842383,94.39424051)(781.16842041,94.37424123)
\curveto(781.12842398,94.36424054)(781.08842402,94.35924054)(781.04842041,94.35924123)
\curveto(781.0084241,94.36924053)(780.96842414,94.36924053)(780.92842041,94.35924123)
\curveto(780.87842423,94.34924055)(780.82842428,94.34424056)(780.77842041,94.34424123)
\lineto(780.61342041,94.34424123)
\curveto(780.56342454,94.32424058)(780.51342459,94.31924058)(780.46342041,94.32924123)
\curveto(780.4034247,94.33924056)(780.34842476,94.33924056)(780.29842041,94.32924123)
\curveto(780.25842485,94.31924058)(780.21342489,94.31924058)(780.16342041,94.32924123)
\curveto(780.11342499,94.33924056)(780.06342504,94.33424057)(780.01342041,94.31424123)
\curveto(779.94342516,94.29424061)(779.86842524,94.28924061)(779.78842041,94.29924123)
\curveto(779.69842541,94.30924059)(779.61342549,94.31424059)(779.53342041,94.31424123)
\curveto(779.44342566,94.31424059)(779.34342576,94.30924059)(779.23342041,94.29924123)
\curveto(779.11342599,94.28924061)(779.01342609,94.29424061)(778.93342041,94.31424123)
\lineto(778.64842041,94.31424123)
\lineto(778.01842041,94.35924123)
\curveto(777.91842719,94.36924053)(777.82342728,94.37924052)(777.73342041,94.38924123)
\lineto(777.43342041,94.41924123)
\curveto(777.38342772,94.43924046)(777.33342777,94.44424046)(777.28342041,94.43424123)
\curveto(777.22342788,94.43424047)(777.16842794,94.44424046)(777.11842041,94.46424123)
\curveto(776.94842816,94.51424039)(776.78342832,94.55424035)(776.62342041,94.58424123)
\curveto(776.45342865,94.61424029)(776.29342881,94.66424024)(776.14342041,94.73424123)
\curveto(775.68342942,94.92423998)(775.3084298,95.14423976)(775.01842041,95.39424123)
\curveto(774.72843038,95.65423925)(774.48343062,96.01423889)(774.28342041,96.47424123)
\curveto(774.23343087,96.6042383)(774.19843091,96.73423817)(774.17842041,96.86424123)
\curveto(774.15843095,97.0042379)(774.13343097,97.14423776)(774.10342041,97.28424123)
\curveto(774.09343101,97.35423755)(774.08843102,97.41923748)(774.08842041,97.47924123)
\curveto(774.08843102,97.53923736)(774.08343102,97.6042373)(774.07342041,97.67424123)
\curveto(774.05343105,98.5042364)(774.2034309,99.17423573)(774.52342041,99.68424123)
\curveto(774.83343027,100.19423471)(775.27342983,100.57423433)(775.84342041,100.82424123)
\curveto(775.96342914,100.87423403)(776.08842902,100.91923398)(776.21842041,100.95924123)
\curveto(776.34842876,100.9992339)(776.48342862,101.04423386)(776.62342041,101.09424123)
\curveto(776.7034284,101.11423379)(776.78842832,101.12923377)(776.87842041,101.13924123)
\lineto(777.11842041,101.19924123)
\curveto(777.22842788,101.22923367)(777.33842777,101.24423366)(777.44842041,101.24424123)
\curveto(777.55842755,101.25423365)(777.66842744,101.26923363)(777.77842041,101.28924123)
\curveto(777.82842728,101.30923359)(777.87342723,101.31423359)(777.91342041,101.30424123)
\curveto(777.95342715,101.3042336)(777.99342711,101.30923359)(778.03342041,101.31924123)
\curveto(778.08342702,101.32923357)(778.13842697,101.32923357)(778.19842041,101.31924123)
\curveto(778.24842686,101.31923358)(778.29842681,101.32423358)(778.34842041,101.33424123)
\lineto(778.48342041,101.33424123)
\curveto(778.54342656,101.35423355)(778.61342649,101.35423355)(778.69342041,101.33424123)
\curveto(778.76342634,101.32423358)(778.82842628,101.32923357)(778.88842041,101.34924123)
\curveto(778.91842619,101.35923354)(778.95842615,101.36423354)(779.00842041,101.36424123)
\lineto(779.12842041,101.36424123)
\lineto(779.59342041,101.36424123)
\moveto(781.91842041,99.81924123)
\curveto(781.59842351,99.91923498)(781.23342387,99.97923492)(780.82342041,99.99924123)
\curveto(780.41342469,100.01923488)(780.0034251,100.02923487)(779.59342041,100.02924123)
\curveto(779.16342594,100.02923487)(778.74342636,100.01923488)(778.33342041,99.99924123)
\curveto(777.92342718,99.97923492)(777.53842757,99.93423497)(777.17842041,99.86424123)
\curveto(776.81842829,99.79423511)(776.49842861,99.68423522)(776.21842041,99.53424123)
\curveto(775.92842918,99.39423551)(775.69342941,99.1992357)(775.51342041,98.94924123)
\curveto(775.4034297,98.78923611)(775.32342978,98.60923629)(775.27342041,98.40924123)
\curveto(775.21342989,98.20923669)(775.18342992,97.96423694)(775.18342041,97.67424123)
\curveto(775.2034299,97.65423725)(775.21342989,97.61923728)(775.21342041,97.56924123)
\curveto(775.2034299,97.51923738)(775.2034299,97.47923742)(775.21342041,97.44924123)
\curveto(775.23342987,97.36923753)(775.25342985,97.29423761)(775.27342041,97.22424123)
\curveto(775.28342982,97.16423774)(775.3034298,97.0992378)(775.33342041,97.02924123)
\curveto(775.45342965,96.75923814)(775.62342948,96.53923836)(775.84342041,96.36924123)
\curveto(776.05342905,96.20923869)(776.29842881,96.07423883)(776.57842041,95.96424123)
\curveto(776.68842842,95.91423899)(776.8084283,95.87423903)(776.93842041,95.84424123)
\curveto(777.05842805,95.82423908)(777.18342792,95.7992391)(777.31342041,95.76924123)
\curveto(777.36342774,95.74923915)(777.41842769,95.73923916)(777.47842041,95.73924123)
\curveto(777.52842758,95.73923916)(777.57842753,95.73423917)(777.62842041,95.72424123)
\curveto(777.71842739,95.71423919)(777.81342729,95.7042392)(777.91342041,95.69424123)
\curveto(778.0034271,95.68423922)(778.09842701,95.67423923)(778.19842041,95.66424123)
\curveto(778.27842683,95.66423924)(778.36342674,95.65923924)(778.45342041,95.64924123)
\lineto(778.69342041,95.64924123)
\lineto(778.87342041,95.64924123)
\curveto(778.9034262,95.63923926)(778.93842617,95.63423927)(778.97842041,95.63424123)
\lineto(779.11342041,95.63424123)
\lineto(779.56342041,95.63424123)
\curveto(779.64342546,95.63423927)(779.72842538,95.62923927)(779.81842041,95.61924123)
\curveto(779.89842521,95.61923928)(779.97342513,95.62923927)(780.04342041,95.64924123)
\lineto(780.31342041,95.64924123)
\curveto(780.33342477,95.64923925)(780.36342474,95.64423926)(780.40342041,95.63424123)
\curveto(780.43342467,95.63423927)(780.45842465,95.63923926)(780.47842041,95.64924123)
\curveto(780.57842453,95.65923924)(780.67842443,95.66423924)(780.77842041,95.66424123)
\curveto(780.86842424,95.67423923)(780.96842414,95.68423922)(781.07842041,95.69424123)
\curveto(781.19842391,95.72423918)(781.32342378,95.73923916)(781.45342041,95.73924123)
\curveto(781.57342353,95.74923915)(781.68842342,95.77423913)(781.79842041,95.81424123)
\curveto(782.09842301,95.89423901)(782.36342274,95.97923892)(782.59342041,96.06924123)
\curveto(782.82342228,96.16923873)(783.03842207,96.31423859)(783.23842041,96.50424123)
\curveto(783.43842167,96.71423819)(783.58842152,96.97923792)(783.68842041,97.29924123)
\curveto(783.7084214,97.33923756)(783.71842139,97.37423753)(783.71842041,97.40424123)
\curveto(783.7084214,97.44423746)(783.71342139,97.48923741)(783.73342041,97.53924123)
\curveto(783.74342136,97.57923732)(783.75342135,97.64923725)(783.76342041,97.74924123)
\curveto(783.77342133,97.85923704)(783.76842134,97.94423696)(783.74842041,98.00424123)
\curveto(783.72842138,98.07423683)(783.71842139,98.14423676)(783.71842041,98.21424123)
\curveto(783.7084214,98.28423662)(783.69342141,98.34923655)(783.67342041,98.40924123)
\curveto(783.61342149,98.60923629)(783.52842158,98.78923611)(783.41842041,98.94924123)
\curveto(783.39842171,98.97923592)(783.37842173,99.0042359)(783.35842041,99.02424123)
\lineto(783.29842041,99.08424123)
\curveto(783.27842183,99.12423578)(783.23842187,99.17423573)(783.17842041,99.23424123)
\curveto(783.03842207,99.33423557)(782.9084222,99.41923548)(782.78842041,99.48924123)
\curveto(782.66842244,99.55923534)(782.52342258,99.62923527)(782.35342041,99.69924123)
\curveto(782.28342282,99.72923517)(782.21342289,99.74923515)(782.14342041,99.75924123)
\curveto(782.07342303,99.77923512)(781.99842311,99.7992351)(781.91842041,99.81924123)
}
}
{
\newrgbcolor{curcolor}{0 0 0}
\pscustom[linestyle=none,fillstyle=solid,fillcolor=curcolor]
{
\newpath
\moveto(774.07342041,106.77385061)
\curveto(774.07343103,106.87384575)(774.08343102,106.96884566)(774.10342041,107.05885061)
\curveto(774.11343099,107.14884548)(774.14343096,107.21384541)(774.19342041,107.25385061)
\curveto(774.27343083,107.31384531)(774.37843073,107.34384528)(774.50842041,107.34385061)
\lineto(774.89842041,107.34385061)
\lineto(776.39842041,107.34385061)
\lineto(782.78842041,107.34385061)
\lineto(783.95842041,107.34385061)
\lineto(784.27342041,107.34385061)
\curveto(784.37342073,107.35384527)(784.45342065,107.33884529)(784.51342041,107.29885061)
\curveto(784.59342051,107.24884538)(784.64342046,107.17384545)(784.66342041,107.07385061)
\curveto(784.67342043,106.98384564)(784.67842043,106.87384575)(784.67842041,106.74385061)
\lineto(784.67842041,106.51885061)
\curveto(784.65842045,106.43884619)(784.64342046,106.36884626)(784.63342041,106.30885061)
\curveto(784.61342049,106.24884638)(784.57342053,106.19884643)(784.51342041,106.15885061)
\curveto(784.45342065,106.11884651)(784.37842073,106.09884653)(784.28842041,106.09885061)
\lineto(783.98842041,106.09885061)
\lineto(782.89342041,106.09885061)
\lineto(777.55342041,106.09885061)
\curveto(777.46342764,106.07884655)(777.38842772,106.06384656)(777.32842041,106.05385061)
\curveto(777.25842785,106.05384657)(777.19842791,106.0238466)(777.14842041,105.96385061)
\curveto(777.09842801,105.89384673)(777.07342803,105.80384682)(777.07342041,105.69385061)
\curveto(777.06342804,105.59384703)(777.05842805,105.48384714)(777.05842041,105.36385061)
\lineto(777.05842041,104.22385061)
\lineto(777.05842041,103.72885061)
\curveto(777.04842806,103.56884906)(776.98842812,103.45884917)(776.87842041,103.39885061)
\curveto(776.84842826,103.37884925)(776.81842829,103.36884926)(776.78842041,103.36885061)
\curveto(776.74842836,103.36884926)(776.7034284,103.36384926)(776.65342041,103.35385061)
\curveto(776.53342857,103.33384929)(776.42342868,103.33884929)(776.32342041,103.36885061)
\curveto(776.22342888,103.40884922)(776.15342895,103.46384916)(776.11342041,103.53385061)
\curveto(776.06342904,103.61384901)(776.03842907,103.73384889)(776.03842041,103.89385061)
\curveto(776.03842907,104.05384857)(776.02342908,104.18884844)(775.99342041,104.29885061)
\curveto(775.98342912,104.34884828)(775.97842913,104.40384822)(775.97842041,104.46385061)
\curveto(775.96842914,104.5238481)(775.95342915,104.58384804)(775.93342041,104.64385061)
\curveto(775.88342922,104.79384783)(775.83342927,104.93884769)(775.78342041,105.07885061)
\curveto(775.72342938,105.21884741)(775.65342945,105.35384727)(775.57342041,105.48385061)
\curveto(775.48342962,105.623847)(775.37842973,105.74384688)(775.25842041,105.84385061)
\curveto(775.13842997,105.94384668)(775.0084301,106.03884659)(774.86842041,106.12885061)
\curveto(774.76843034,106.18884644)(774.65843045,106.23384639)(774.53842041,106.26385061)
\curveto(774.41843069,106.30384632)(774.31343079,106.35384627)(774.22342041,106.41385061)
\curveto(774.16343094,106.46384616)(774.12343098,106.53384609)(774.10342041,106.62385061)
\curveto(774.09343101,106.64384598)(774.08843102,106.66884596)(774.08842041,106.69885061)
\curveto(774.08843102,106.7288459)(774.08343102,106.75384587)(774.07342041,106.77385061)
}
}
{
\newrgbcolor{curcolor}{0 0 0}
\pscustom[linestyle=none,fillstyle=solid,fillcolor=curcolor]
{
\newpath
\moveto(774.07342041,115.12345998)
\curveto(774.07343103,115.22345513)(774.08343102,115.31845503)(774.10342041,115.40845998)
\curveto(774.11343099,115.49845485)(774.14343096,115.56345479)(774.19342041,115.60345998)
\curveto(774.27343083,115.66345469)(774.37843073,115.69345466)(774.50842041,115.69345998)
\lineto(774.89842041,115.69345998)
\lineto(776.39842041,115.69345998)
\lineto(782.78842041,115.69345998)
\lineto(783.95842041,115.69345998)
\lineto(784.27342041,115.69345998)
\curveto(784.37342073,115.70345465)(784.45342065,115.68845466)(784.51342041,115.64845998)
\curveto(784.59342051,115.59845475)(784.64342046,115.52345483)(784.66342041,115.42345998)
\curveto(784.67342043,115.33345502)(784.67842043,115.22345513)(784.67842041,115.09345998)
\lineto(784.67842041,114.86845998)
\curveto(784.65842045,114.78845556)(784.64342046,114.71845563)(784.63342041,114.65845998)
\curveto(784.61342049,114.59845575)(784.57342053,114.5484558)(784.51342041,114.50845998)
\curveto(784.45342065,114.46845588)(784.37842073,114.4484559)(784.28842041,114.44845998)
\lineto(783.98842041,114.44845998)
\lineto(782.89342041,114.44845998)
\lineto(777.55342041,114.44845998)
\curveto(777.46342764,114.42845592)(777.38842772,114.41345594)(777.32842041,114.40345998)
\curveto(777.25842785,114.40345595)(777.19842791,114.37345598)(777.14842041,114.31345998)
\curveto(777.09842801,114.24345611)(777.07342803,114.1534562)(777.07342041,114.04345998)
\curveto(777.06342804,113.94345641)(777.05842805,113.83345652)(777.05842041,113.71345998)
\lineto(777.05842041,112.57345998)
\lineto(777.05842041,112.07845998)
\curveto(777.04842806,111.91845843)(776.98842812,111.80845854)(776.87842041,111.74845998)
\curveto(776.84842826,111.72845862)(776.81842829,111.71845863)(776.78842041,111.71845998)
\curveto(776.74842836,111.71845863)(776.7034284,111.71345864)(776.65342041,111.70345998)
\curveto(776.53342857,111.68345867)(776.42342868,111.68845866)(776.32342041,111.71845998)
\curveto(776.22342888,111.75845859)(776.15342895,111.81345854)(776.11342041,111.88345998)
\curveto(776.06342904,111.96345839)(776.03842907,112.08345827)(776.03842041,112.24345998)
\curveto(776.03842907,112.40345795)(776.02342908,112.53845781)(775.99342041,112.64845998)
\curveto(775.98342912,112.69845765)(775.97842913,112.7534576)(775.97842041,112.81345998)
\curveto(775.96842914,112.87345748)(775.95342915,112.93345742)(775.93342041,112.99345998)
\curveto(775.88342922,113.14345721)(775.83342927,113.28845706)(775.78342041,113.42845998)
\curveto(775.72342938,113.56845678)(775.65342945,113.70345665)(775.57342041,113.83345998)
\curveto(775.48342962,113.97345638)(775.37842973,114.09345626)(775.25842041,114.19345998)
\curveto(775.13842997,114.29345606)(775.0084301,114.38845596)(774.86842041,114.47845998)
\curveto(774.76843034,114.53845581)(774.65843045,114.58345577)(774.53842041,114.61345998)
\curveto(774.41843069,114.6534557)(774.31343079,114.70345565)(774.22342041,114.76345998)
\curveto(774.16343094,114.81345554)(774.12343098,114.88345547)(774.10342041,114.97345998)
\curveto(774.09343101,114.99345536)(774.08843102,115.01845533)(774.08842041,115.04845998)
\curveto(774.08843102,115.07845527)(774.08343102,115.10345525)(774.07342041,115.12345998)
}
}
{
\newrgbcolor{curcolor}{0 0 0}
\pscustom[linestyle=none,fillstyle=solid,fillcolor=curcolor]
{
\newpath
\moveto(794.90973633,37.28705373)
\curveto(794.90974702,37.35704805)(794.90974702,37.43704797)(794.90973633,37.52705373)
\curveto(794.89974703,37.61704779)(794.89974703,37.70204771)(794.90973633,37.78205373)
\curveto(794.90974702,37.87204754)(794.91974701,37.95204746)(794.93973633,38.02205373)
\curveto(794.95974697,38.10204731)(794.98974694,38.15704725)(795.02973633,38.18705373)
\curveto(795.07974685,38.21704719)(795.15474678,38.23704717)(795.25473633,38.24705373)
\curveto(795.34474659,38.26704714)(795.44974648,38.27704713)(795.56973633,38.27705373)
\curveto(795.67974625,38.28704712)(795.79474614,38.28704712)(795.91473633,38.27705373)
\lineto(796.21473633,38.27705373)
\lineto(799.22973633,38.27705373)
\lineto(802.12473633,38.27705373)
\curveto(802.45473948,38.27704713)(802.77973915,38.27204714)(803.09973633,38.26205373)
\curveto(803.40973852,38.26204715)(803.68973824,38.22204719)(803.93973633,38.14205373)
\curveto(804.28973764,38.02204739)(804.58473735,37.86704754)(804.82473633,37.67705373)
\curveto(805.05473688,37.48704792)(805.25473668,37.24704816)(805.42473633,36.95705373)
\curveto(805.47473646,36.89704851)(805.50973642,36.83204858)(805.52973633,36.76205373)
\curveto(805.54973638,36.70204871)(805.57473636,36.63204878)(805.60473633,36.55205373)
\curveto(805.65473628,36.43204898)(805.68973624,36.30204911)(805.70973633,36.16205373)
\curveto(805.73973619,36.03204938)(805.76973616,35.89704951)(805.79973633,35.75705373)
\curveto(805.81973611,35.7070497)(805.82473611,35.65704975)(805.81473633,35.60705373)
\curveto(805.80473613,35.55704985)(805.80473613,35.50204991)(805.81473633,35.44205373)
\curveto(805.82473611,35.42204999)(805.82473611,35.39705001)(805.81473633,35.36705373)
\curveto(805.81473612,35.33705007)(805.81973611,35.3120501)(805.82973633,35.29205373)
\curveto(805.83973609,35.25205016)(805.84473609,35.19705021)(805.84473633,35.12705373)
\curveto(805.84473609,35.05705035)(805.83973609,35.00205041)(805.82973633,34.96205373)
\curveto(805.81973611,34.9120505)(805.81973611,34.85705055)(805.82973633,34.79705373)
\curveto(805.83973609,34.73705067)(805.8347361,34.68205073)(805.81473633,34.63205373)
\curveto(805.78473615,34.50205091)(805.76473617,34.37705103)(805.75473633,34.25705373)
\curveto(805.74473619,34.13705127)(805.71973621,34.02205139)(805.67973633,33.91205373)
\curveto(805.55973637,33.54205187)(805.38973654,33.22205219)(805.16973633,32.95205373)
\curveto(804.94973698,32.68205273)(804.66973726,32.47205294)(804.32973633,32.32205373)
\curveto(804.20973772,32.27205314)(804.08473785,32.22705318)(803.95473633,32.18705373)
\curveto(803.82473811,32.15705325)(803.68973824,32.12205329)(803.54973633,32.08205373)
\curveto(803.49973843,32.07205334)(803.45973847,32.06705334)(803.42973633,32.06705373)
\curveto(803.38973854,32.06705334)(803.34473859,32.06205335)(803.29473633,32.05205373)
\curveto(803.26473867,32.04205337)(803.2297387,32.03705337)(803.18973633,32.03705373)
\curveto(803.13973879,32.03705337)(803.09973883,32.03205338)(803.06973633,32.02205373)
\lineto(802.90473633,32.02205373)
\curveto(802.82473911,32.00205341)(802.72473921,31.99705341)(802.60473633,32.00705373)
\curveto(802.47473946,32.01705339)(802.38473955,32.03205338)(802.33473633,32.05205373)
\curveto(802.24473969,32.07205334)(802.17973975,32.12705328)(802.13973633,32.21705373)
\curveto(802.11973981,32.24705316)(802.11473982,32.27705313)(802.12473633,32.30705373)
\curveto(802.12473981,32.33705307)(802.11973981,32.37705303)(802.10973633,32.42705373)
\curveto(802.09973983,32.46705294)(802.09473984,32.5070529)(802.09473633,32.54705373)
\lineto(802.09473633,32.69705373)
\curveto(802.09473984,32.81705259)(802.09973983,32.93705247)(802.10973633,33.05705373)
\curveto(802.10973982,33.18705222)(802.14473979,33.27705213)(802.21473633,33.32705373)
\curveto(802.27473966,33.36705204)(802.3347396,33.38705202)(802.39473633,33.38705373)
\curveto(802.45473948,33.38705202)(802.52473941,33.39705201)(802.60473633,33.41705373)
\curveto(802.6347393,33.42705198)(802.66973926,33.42705198)(802.70973633,33.41705373)
\curveto(802.73973919,33.41705199)(802.76473917,33.42205199)(802.78473633,33.43205373)
\lineto(802.99473633,33.43205373)
\curveto(803.04473889,33.45205196)(803.09473884,33.45705195)(803.14473633,33.44705373)
\curveto(803.18473875,33.44705196)(803.2297387,33.45705195)(803.27973633,33.47705373)
\curveto(803.40973852,33.5070519)(803.5347384,33.53705187)(803.65473633,33.56705373)
\curveto(803.76473817,33.59705181)(803.86973806,33.64205177)(803.96973633,33.70205373)
\curveto(804.25973767,33.87205154)(804.46473747,34.14205127)(804.58473633,34.51205373)
\curveto(804.60473733,34.56205085)(804.61973731,34.6120508)(804.62973633,34.66205373)
\curveto(804.6297373,34.72205069)(804.63973729,34.77705063)(804.65973633,34.82705373)
\lineto(804.65973633,34.90205373)
\curveto(804.66973726,34.97205044)(804.67973725,35.06705034)(804.68973633,35.18705373)
\curveto(804.68973724,35.31705009)(804.67973725,35.41704999)(804.65973633,35.48705373)
\curveto(804.63973729,35.55704985)(804.62473731,35.62704978)(804.61473633,35.69705373)
\curveto(804.59473734,35.77704963)(804.57473736,35.84704956)(804.55473633,35.90705373)
\curveto(804.39473754,36.28704912)(804.11973781,36.56204885)(803.72973633,36.73205373)
\curveto(803.59973833,36.78204863)(803.44473849,36.81704859)(803.26473633,36.83705373)
\curveto(803.08473885,36.86704854)(802.89973903,36.88204853)(802.70973633,36.88205373)
\curveto(802.50973942,36.89204852)(802.30973962,36.89204852)(802.10973633,36.88205373)
\lineto(801.53973633,36.88205373)
\lineto(797.29473633,36.88205373)
\lineto(795.74973633,36.88205373)
\curveto(795.63974629,36.88204853)(795.51974641,36.87704853)(795.38973633,36.86705373)
\curveto(795.25974667,36.85704855)(795.15474678,36.87704853)(795.07473633,36.92705373)
\curveto(795.00474693,36.98704842)(794.95474698,37.06704834)(794.92473633,37.16705373)
\curveto(794.92474701,37.18704822)(794.92474701,37.2070482)(794.92473633,37.22705373)
\curveto(794.92474701,37.24704816)(794.91974701,37.26704814)(794.90973633,37.28705373)
}
}
{
\newrgbcolor{curcolor}{0 0 0}
\pscustom[linestyle=none,fillstyle=solid,fillcolor=curcolor]
{
\newpath
\moveto(797.86473633,40.82072561)
\lineto(797.86473633,41.25572561)
\curveto(797.86474407,41.40572364)(797.90474403,41.51072354)(797.98473633,41.57072561)
\curveto(798.06474387,41.62072343)(798.16474377,41.6457234)(798.28473633,41.64572561)
\curveto(798.40474353,41.65572339)(798.52474341,41.66072339)(798.64473633,41.66072561)
\lineto(800.06973633,41.66072561)
\lineto(802.33473633,41.66072561)
\lineto(803.02473633,41.66072561)
\curveto(803.25473868,41.66072339)(803.45473848,41.68572336)(803.62473633,41.73572561)
\curveto(804.07473786,41.89572315)(804.38973754,42.19572285)(804.56973633,42.63572561)
\curveto(804.65973727,42.85572219)(804.69473724,43.12072193)(804.67473633,43.43072561)
\curveto(804.64473729,43.74072131)(804.58973734,43.99072106)(804.50973633,44.18072561)
\curveto(804.36973756,44.51072054)(804.19473774,44.77072028)(803.98473633,44.96072561)
\curveto(803.76473817,45.16071989)(803.47973845,45.31571973)(803.12973633,45.42572561)
\curveto(803.04973888,45.45571959)(802.96973896,45.47571957)(802.88973633,45.48572561)
\curveto(802.80973912,45.49571955)(802.72473921,45.51071954)(802.63473633,45.53072561)
\curveto(802.58473935,45.54071951)(802.53973939,45.54071951)(802.49973633,45.53072561)
\curveto(802.45973947,45.53071952)(802.41473952,45.54071951)(802.36473633,45.56072561)
\lineto(802.04973633,45.56072561)
\curveto(801.96973996,45.58071947)(801.87974005,45.58571946)(801.77973633,45.57572561)
\curveto(801.66974026,45.56571948)(801.56974036,45.56071949)(801.47973633,45.56072561)
\lineto(800.30973633,45.56072561)
\lineto(798.71973633,45.56072561)
\curveto(798.59974333,45.56071949)(798.47474346,45.55571949)(798.34473633,45.54572561)
\curveto(798.20474373,45.5457195)(798.09474384,45.57071948)(798.01473633,45.62072561)
\curveto(797.96474397,45.66071939)(797.934744,45.70571934)(797.92473633,45.75572561)
\curveto(797.90474403,45.81571923)(797.88474405,45.88571916)(797.86473633,45.96572561)
\lineto(797.86473633,46.19072561)
\curveto(797.86474407,46.31071874)(797.86974406,46.41571863)(797.87973633,46.50572561)
\curveto(797.88974404,46.60571844)(797.934744,46.68071837)(798.01473633,46.73072561)
\curveto(798.06474387,46.78071827)(798.13974379,46.80571824)(798.23973633,46.80572561)
\lineto(798.52473633,46.80572561)
\lineto(799.54473633,46.80572561)
\lineto(803.57973633,46.80572561)
\lineto(804.92973633,46.80572561)
\curveto(805.04973688,46.80571824)(805.16473677,46.80071825)(805.27473633,46.79072561)
\curveto(805.37473656,46.79071826)(805.44973648,46.75571829)(805.49973633,46.68572561)
\curveto(805.5297364,46.6457184)(805.55473638,46.58571846)(805.57473633,46.50572561)
\curveto(805.58473635,46.42571862)(805.59473634,46.33571871)(805.60473633,46.23572561)
\curveto(805.60473633,46.1457189)(805.59973633,46.05571899)(805.58973633,45.96572561)
\curveto(805.57973635,45.88571916)(805.55973637,45.82571922)(805.52973633,45.78572561)
\curveto(805.48973644,45.73571931)(805.42473651,45.69071936)(805.33473633,45.65072561)
\curveto(805.29473664,45.64071941)(805.23973669,45.63071942)(805.16973633,45.62072561)
\curveto(805.09973683,45.62071943)(805.0347369,45.61571943)(804.97473633,45.60572561)
\curveto(804.90473703,45.59571945)(804.84973708,45.57571947)(804.80973633,45.54572561)
\curveto(804.76973716,45.51571953)(804.75473718,45.47071958)(804.76473633,45.41072561)
\curveto(804.78473715,45.33071972)(804.84473709,45.2507198)(804.94473633,45.17072561)
\curveto(805.0347369,45.09071996)(805.10473683,45.01572003)(805.15473633,44.94572561)
\curveto(805.31473662,44.72572032)(805.45473648,44.47572057)(805.57473633,44.19572561)
\curveto(805.62473631,44.08572096)(805.65473628,43.97072108)(805.66473633,43.85072561)
\curveto(805.68473625,43.74072131)(805.70973622,43.62572142)(805.73973633,43.50572561)
\curveto(805.74973618,43.45572159)(805.74973618,43.40072165)(805.73973633,43.34072561)
\curveto(805.7297362,43.29072176)(805.7347362,43.24072181)(805.75473633,43.19072561)
\curveto(805.77473616,43.09072196)(805.77473616,43.00072205)(805.75473633,42.92072561)
\lineto(805.75473633,42.77072561)
\curveto(805.7347362,42.72072233)(805.72473621,42.66072239)(805.72473633,42.59072561)
\curveto(805.72473621,42.53072252)(805.71973621,42.47572257)(805.70973633,42.42572561)
\curveto(805.68973624,42.38572266)(805.67973625,42.3457227)(805.67973633,42.30572561)
\curveto(805.68973624,42.27572277)(805.68473625,42.23572281)(805.66473633,42.18572561)
\lineto(805.60473633,41.94572561)
\curveto(805.58473635,41.87572317)(805.55473638,41.80072325)(805.51473633,41.72072561)
\curveto(805.40473653,41.46072359)(805.25973667,41.24072381)(805.07973633,41.06072561)
\curveto(804.88973704,40.89072416)(804.66473727,40.7507243)(804.40473633,40.64072561)
\curveto(804.31473762,40.60072445)(804.22473771,40.57072448)(804.13473633,40.55072561)
\lineto(803.83473633,40.49072561)
\curveto(803.77473816,40.47072458)(803.71973821,40.46072459)(803.66973633,40.46072561)
\curveto(803.60973832,40.47072458)(803.54473839,40.46572458)(803.47473633,40.44572561)
\curveto(803.45473848,40.43572461)(803.4297385,40.43072462)(803.39973633,40.43072561)
\curveto(803.35973857,40.43072462)(803.32473861,40.42572462)(803.29473633,40.41572561)
\lineto(803.14473633,40.41572561)
\curveto(803.10473883,40.40572464)(803.05973887,40.40072465)(803.00973633,40.40072561)
\curveto(802.94973898,40.41072464)(802.89473904,40.41572463)(802.84473633,40.41572561)
\lineto(802.24473633,40.41572561)
\lineto(799.48473633,40.41572561)
\lineto(798.52473633,40.41572561)
\lineto(798.25473633,40.41572561)
\curveto(798.16474377,40.41572463)(798.08974384,40.43572461)(798.02973633,40.47572561)
\curveto(797.95974397,40.51572453)(797.90974402,40.59072446)(797.87973633,40.70072561)
\curveto(797.86974406,40.72072433)(797.86974406,40.74072431)(797.87973633,40.76072561)
\curveto(797.87974405,40.78072427)(797.87474406,40.80072425)(797.86473633,40.82072561)
}
}
{
\newrgbcolor{curcolor}{0 0 0}
\pscustom[linestyle=none,fillstyle=solid,fillcolor=curcolor]
{
\newpath
\moveto(797.71473633,52.39533498)
\curveto(797.69474424,53.02532975)(797.77974415,53.53032924)(797.96973633,53.91033498)
\curveto(798.15974377,54.29032848)(798.44474349,54.59532818)(798.82473633,54.82533498)
\curveto(798.92474301,54.88532789)(799.0347429,54.93032784)(799.15473633,54.96033498)
\curveto(799.26474267,55.00032777)(799.37974255,55.03532774)(799.49973633,55.06533498)
\curveto(799.68974224,55.11532766)(799.89474204,55.14532763)(800.11473633,55.15533498)
\curveto(800.3347416,55.16532761)(800.55974137,55.1703276)(800.78973633,55.17033498)
\lineto(802.39473633,55.17033498)
\lineto(804.73473633,55.17033498)
\curveto(804.90473703,55.1703276)(805.07473686,55.16532761)(805.24473633,55.15533498)
\curveto(805.41473652,55.15532762)(805.52473641,55.09032768)(805.57473633,54.96033498)
\curveto(805.59473634,54.91032786)(805.60473633,54.85532792)(805.60473633,54.79533498)
\curveto(805.61473632,54.74532803)(805.61973631,54.69032808)(805.61973633,54.63033498)
\curveto(805.61973631,54.50032827)(805.61473632,54.3753284)(805.60473633,54.25533498)
\curveto(805.60473633,54.13532864)(805.56473637,54.05032872)(805.48473633,54.00033498)
\curveto(805.41473652,53.95032882)(805.32473661,53.92532885)(805.21473633,53.92533498)
\lineto(804.88473633,53.92533498)
\lineto(803.59473633,53.92533498)
\lineto(801.14973633,53.92533498)
\curveto(800.87974105,53.92532885)(800.61474132,53.92032885)(800.35473633,53.91033498)
\curveto(800.08474185,53.90032887)(799.85474208,53.85532892)(799.66473633,53.77533498)
\curveto(799.46474247,53.69532908)(799.30474263,53.5753292)(799.18473633,53.41533498)
\curveto(799.05474288,53.25532952)(798.95474298,53.0703297)(798.88473633,52.86033498)
\curveto(798.86474307,52.80032997)(798.85474308,52.73533004)(798.85473633,52.66533498)
\curveto(798.84474309,52.60533017)(798.8297431,52.54533023)(798.80973633,52.48533498)
\curveto(798.79974313,52.43533034)(798.79974313,52.35533042)(798.80973633,52.24533498)
\curveto(798.80974312,52.14533063)(798.81474312,52.0753307)(798.82473633,52.03533498)
\curveto(798.84474309,51.99533078)(798.85474308,51.96033081)(798.85473633,51.93033498)
\curveto(798.84474309,51.90033087)(798.84474309,51.86533091)(798.85473633,51.82533498)
\curveto(798.88474305,51.69533108)(798.91974301,51.5703312)(798.95973633,51.45033498)
\curveto(798.98974294,51.34033143)(799.0347429,51.23533154)(799.09473633,51.13533498)
\curveto(799.11474282,51.09533168)(799.1347428,51.06033171)(799.15473633,51.03033498)
\curveto(799.17474276,51.00033177)(799.19474274,50.96533181)(799.21473633,50.92533498)
\curveto(799.46474247,50.5753322)(799.83974209,50.32033245)(800.33973633,50.16033498)
\curveto(800.41974151,50.13033264)(800.50474143,50.11033266)(800.59473633,50.10033498)
\curveto(800.67474126,50.09033268)(800.75474118,50.0753327)(800.83473633,50.05533498)
\curveto(800.88474105,50.03533274)(800.934741,50.03033274)(800.98473633,50.04033498)
\curveto(801.02474091,50.05033272)(801.06474087,50.04533273)(801.10473633,50.02533498)
\lineto(801.41973633,50.02533498)
\curveto(801.44974048,50.01533276)(801.48474045,50.01033276)(801.52473633,50.01033498)
\curveto(801.56474037,50.02033275)(801.60974032,50.02533275)(801.65973633,50.02533498)
\lineto(802.10973633,50.02533498)
\lineto(803.54973633,50.02533498)
\lineto(804.86973633,50.02533498)
\lineto(805.21473633,50.02533498)
\curveto(805.32473661,50.02533275)(805.41473652,50.00033277)(805.48473633,49.95033498)
\curveto(805.56473637,49.90033287)(805.60473633,49.81033296)(805.60473633,49.68033498)
\curveto(805.61473632,49.56033321)(805.61973631,49.43533334)(805.61973633,49.30533498)
\curveto(805.61973631,49.22533355)(805.61473632,49.15033362)(805.60473633,49.08033498)
\curveto(805.59473634,49.01033376)(805.56973636,48.95033382)(805.52973633,48.90033498)
\curveto(805.47973645,48.82033395)(805.38473655,48.78033399)(805.24473633,48.78033498)
\lineto(804.83973633,48.78033498)
\lineto(803.06973633,48.78033498)
\lineto(799.43973633,48.78033498)
\lineto(798.52473633,48.78033498)
\lineto(798.25473633,48.78033498)
\curveto(798.16474377,48.78033399)(798.09474384,48.80033397)(798.04473633,48.84033498)
\curveto(797.98474395,48.8703339)(797.94474399,48.92033385)(797.92473633,48.99033498)
\curveto(797.91474402,49.03033374)(797.90474403,49.08533369)(797.89473633,49.15533498)
\curveto(797.88474405,49.23533354)(797.87974405,49.31533346)(797.87973633,49.39533498)
\curveto(797.87974405,49.4753333)(797.88474405,49.55033322)(797.89473633,49.62033498)
\curveto(797.90474403,49.70033307)(797.91974401,49.75533302)(797.93973633,49.78533498)
\curveto(798.00974392,49.89533288)(798.09974383,49.94533283)(798.20973633,49.93533498)
\curveto(798.30974362,49.92533285)(798.42474351,49.94033283)(798.55473633,49.98033498)
\curveto(798.61474332,50.00033277)(798.66474327,50.04033273)(798.70473633,50.10033498)
\curveto(798.71474322,50.22033255)(798.66974326,50.31533246)(798.56973633,50.38533498)
\curveto(798.46974346,50.46533231)(798.38974354,50.54533223)(798.32973633,50.62533498)
\curveto(798.2297437,50.76533201)(798.13974379,50.90533187)(798.05973633,51.04533498)
\curveto(797.96974396,51.19533158)(797.89474404,51.36533141)(797.83473633,51.55533498)
\curveto(797.80474413,51.63533114)(797.78474415,51.72033105)(797.77473633,51.81033498)
\curveto(797.76474417,51.91033086)(797.74974418,52.00533077)(797.72973633,52.09533498)
\curveto(797.71974421,52.14533063)(797.71474422,52.19533058)(797.71473633,52.24533498)
\lineto(797.71473633,52.39533498)
}
}
{
\newrgbcolor{curcolor}{0 0 0}
\pscustom[linestyle=none,fillstyle=solid,fillcolor=curcolor]
{
}
}
{
\newrgbcolor{curcolor}{0 0 0}
\pscustom[linestyle=none,fillstyle=solid,fillcolor=curcolor]
{
\newpath
\moveto(794.98473633,64.11010061)
\curveto(794.95474698,65.74009517)(795.50974642,66.79009412)(796.64973633,67.26010061)
\curveto(796.87974505,67.36009355)(797.16974476,67.42509348)(797.51973633,67.45510061)
\curveto(797.85974407,67.49509341)(798.16974376,67.47009344)(798.44973633,67.38010061)
\curveto(798.70974322,67.29009362)(798.934743,67.17009374)(799.12473633,67.02010061)
\curveto(799.16474277,67.00009391)(799.19974273,66.97509393)(799.22973633,66.94510061)
\curveto(799.24974268,66.91509399)(799.27474266,66.89009402)(799.30473633,66.87010061)
\lineto(799.42473633,66.78010061)
\curveto(799.45474248,66.75009416)(799.47974245,66.71509419)(799.49973633,66.67510061)
\curveto(799.54974238,66.62509428)(799.59474234,66.57009434)(799.63473633,66.51010061)
\curveto(799.67474226,66.46009445)(799.72474221,66.41509449)(799.78473633,66.37510061)
\curveto(799.82474211,66.33509457)(799.87474206,66.32009459)(799.93473633,66.33010061)
\curveto(799.98474195,66.34009457)(800.0297419,66.37009454)(800.06973633,66.42010061)
\curveto(800.10974182,66.47009444)(800.14974178,66.52509438)(800.18973633,66.58510061)
\curveto(800.21974171,66.65509425)(800.24974168,66.72009419)(800.27973633,66.78010061)
\curveto(800.30974162,66.84009407)(800.33974159,66.89009402)(800.36973633,66.93010061)
\curveto(800.58974134,67.25009366)(800.89974103,67.5050934)(801.29973633,67.69510061)
\curveto(801.38974054,67.73509317)(801.48474045,67.76509314)(801.58473633,67.78510061)
\curveto(801.67474026,67.81509309)(801.76474017,67.84009307)(801.85473633,67.86010061)
\curveto(801.90474003,67.87009304)(801.95473998,67.87509303)(802.00473633,67.87510061)
\curveto(802.04473989,67.88509302)(802.08973984,67.89509301)(802.13973633,67.90510061)
\curveto(802.18973974,67.91509299)(802.23973969,67.91509299)(802.28973633,67.90510061)
\curveto(802.33973959,67.89509301)(802.38973954,67.90009301)(802.43973633,67.92010061)
\curveto(802.48973944,67.93009298)(802.54973938,67.93509297)(802.61973633,67.93510061)
\curveto(802.68973924,67.93509297)(802.74973918,67.92509298)(802.79973633,67.90510061)
\lineto(803.02473633,67.90510061)
\lineto(803.26473633,67.84510061)
\curveto(803.3347386,67.83509307)(803.40473853,67.82009309)(803.47473633,67.80010061)
\curveto(803.56473837,67.77009314)(803.64973828,67.74009317)(803.72973633,67.71010061)
\curveto(803.80973812,67.69009322)(803.88973804,67.66009325)(803.96973633,67.62010061)
\curveto(804.0297379,67.60009331)(804.08973784,67.57009334)(804.14973633,67.53010061)
\curveto(804.19973773,67.50009341)(804.24973768,67.46509344)(804.29973633,67.42510061)
\curveto(804.60973732,67.22509368)(804.86973706,66.97509393)(805.07973633,66.67510061)
\curveto(805.27973665,66.37509453)(805.44473649,66.03009488)(805.57473633,65.64010061)
\curveto(805.61473632,65.52009539)(805.63973629,65.39009552)(805.64973633,65.25010061)
\curveto(805.66973626,65.12009579)(805.69473624,64.98509592)(805.72473633,64.84510061)
\curveto(805.7347362,64.77509613)(805.73973619,64.7050962)(805.73973633,64.63510061)
\curveto(805.73973619,64.57509633)(805.74473619,64.5100964)(805.75473633,64.44010061)
\curveto(805.76473617,64.40009651)(805.76973616,64.34009657)(805.76973633,64.26010061)
\curveto(805.76973616,64.19009672)(805.76473617,64.14009677)(805.75473633,64.11010061)
\curveto(805.74473619,64.06009685)(805.73973619,64.01509689)(805.73973633,63.97510061)
\lineto(805.73973633,63.85510061)
\curveto(805.71973621,63.75509715)(805.70473623,63.65509725)(805.69473633,63.55510061)
\curveto(805.68473625,63.45509745)(805.66973626,63.36009755)(805.64973633,63.27010061)
\curveto(805.61973631,63.16009775)(805.59473634,63.05009786)(805.57473633,62.94010061)
\curveto(805.54473639,62.84009807)(805.50473643,62.73509817)(805.45473633,62.62510061)
\curveto(805.29473664,62.25509865)(805.09473684,61.94009897)(804.85473633,61.68010061)
\curveto(804.60473733,61.42009949)(804.29473764,61.2100997)(803.92473633,61.05010061)
\curveto(803.8347381,61.0100999)(803.73973819,60.97509993)(803.63973633,60.94510061)
\curveto(803.53973839,60.91509999)(803.4347385,60.88510002)(803.32473633,60.85510061)
\curveto(803.27473866,60.83510007)(803.22473871,60.82510008)(803.17473633,60.82510061)
\curveto(803.11473882,60.82510008)(803.05473888,60.81510009)(802.99473633,60.79510061)
\curveto(802.934739,60.77510013)(802.85473908,60.76510014)(802.75473633,60.76510061)
\curveto(802.65473928,60.76510014)(802.57973935,60.78010013)(802.52973633,60.81010061)
\curveto(802.49973943,60.82010009)(802.47473946,60.83510007)(802.45473633,60.85510061)
\lineto(802.39473633,60.91510061)
\curveto(802.37473956,60.95509995)(802.35973957,61.01509989)(802.34973633,61.09510061)
\curveto(802.33973959,61.18509972)(802.3347396,61.27509963)(802.33473633,61.36510061)
\curveto(802.3347396,61.45509945)(802.33973959,61.54009937)(802.34973633,61.62010061)
\curveto(802.35973957,61.7100992)(802.36973956,61.77509913)(802.37973633,61.81510061)
\curveto(802.39973953,61.83509907)(802.41473952,61.85509905)(802.42473633,61.87510061)
\curveto(802.42473951,61.89509901)(802.4347395,61.91509899)(802.45473633,61.93510061)
\curveto(802.54473939,62.0050989)(802.65973927,62.04509886)(802.79973633,62.05510061)
\curveto(802.93973899,62.07509883)(803.06473887,62.1050988)(803.17473633,62.14510061)
\lineto(803.53473633,62.29510061)
\curveto(803.64473829,62.34509856)(803.74973818,62.4100985)(803.84973633,62.49010061)
\curveto(803.87973805,62.5100984)(803.90473803,62.53009838)(803.92473633,62.55010061)
\curveto(803.94473799,62.58009833)(803.96973796,62.6050983)(803.99973633,62.62510061)
\curveto(804.05973787,62.66509824)(804.10473783,62.70009821)(804.13473633,62.73010061)
\curveto(804.16473777,62.77009814)(804.19473774,62.8050981)(804.22473633,62.83510061)
\curveto(804.25473768,62.87509803)(804.28473765,62.92009799)(804.31473633,62.97010061)
\curveto(804.37473756,63.06009785)(804.42473751,63.15509775)(804.46473633,63.25510061)
\lineto(804.58473633,63.58510061)
\curveto(804.6347373,63.73509717)(804.66473727,63.93509697)(804.67473633,64.18510061)
\curveto(804.68473725,64.43509647)(804.66473727,64.64509626)(804.61473633,64.81510061)
\curveto(804.59473734,64.89509601)(804.57973735,64.96509594)(804.56973633,65.02510061)
\lineto(804.50973633,65.23510061)
\curveto(804.38973754,65.51509539)(804.23973769,65.75509515)(804.05973633,65.95510061)
\curveto(803.87973805,66.16509474)(803.64973828,66.33009458)(803.36973633,66.45010061)
\curveto(803.29973863,66.48009443)(803.2297387,66.50009441)(803.15973633,66.51010061)
\lineto(802.91973633,66.57010061)
\curveto(802.77973915,66.6100943)(802.61973931,66.62009429)(802.43973633,66.60010061)
\curveto(802.24973968,66.58009433)(802.09973983,66.55009436)(801.98973633,66.51010061)
\curveto(801.60974032,66.38009453)(801.31974061,66.19509471)(801.11973633,65.95510061)
\curveto(800.91974101,65.72509518)(800.75974117,65.41509549)(800.63973633,65.02510061)
\curveto(800.60974132,64.91509599)(800.58974134,64.79509611)(800.57973633,64.66510061)
\curveto(800.56974136,64.54509636)(800.56474137,64.42009649)(800.56473633,64.29010061)
\curveto(800.56474137,64.13009678)(800.55974137,63.99009692)(800.54973633,63.87010061)
\curveto(800.53974139,63.75009716)(800.47974145,63.66509724)(800.36973633,63.61510061)
\curveto(800.33974159,63.59509731)(800.30474163,63.58509732)(800.26473633,63.58510061)
\lineto(800.12973633,63.58510061)
\curveto(800.0297419,63.57509733)(799.934742,63.57509733)(799.84473633,63.58510061)
\curveto(799.75474218,63.6050973)(799.68974224,63.64509726)(799.64973633,63.70510061)
\curveto(799.61974231,63.74509716)(799.59974233,63.78509712)(799.58973633,63.82510061)
\curveto(799.57974235,63.87509703)(799.56974236,63.93009698)(799.55973633,63.99010061)
\curveto(799.54974238,64.0100969)(799.54974238,64.03509687)(799.55973633,64.06510061)
\curveto(799.55974237,64.09509681)(799.55474238,64.12009679)(799.54473633,64.14010061)
\lineto(799.54473633,64.27510061)
\curveto(799.52474241,64.38509652)(799.51474242,64.48509642)(799.51473633,64.57510061)
\curveto(799.50474243,64.67509623)(799.48474245,64.77009614)(799.45473633,64.86010061)
\curveto(799.34474259,65.18009573)(799.19974273,65.43509547)(799.01973633,65.62510061)
\curveto(798.83974309,65.81509509)(798.58974334,65.96509494)(798.26973633,66.07510061)
\curveto(798.16974376,66.1050948)(798.04474389,66.12509478)(797.89473633,66.13510061)
\curveto(797.7347442,66.15509475)(797.58974434,66.15009476)(797.45973633,66.12010061)
\curveto(797.38974454,66.10009481)(797.32474461,66.08009483)(797.26473633,66.06010061)
\curveto(797.19474474,66.05009486)(797.1297448,66.03009488)(797.06973633,66.00010061)
\curveto(796.8297451,65.90009501)(796.63974529,65.75509515)(796.49973633,65.56510061)
\curveto(796.35974557,65.37509553)(796.24974568,65.15009576)(796.16973633,64.89010061)
\curveto(796.14974578,64.83009608)(796.13974579,64.77009614)(796.13973633,64.71010061)
\curveto(796.13974579,64.65009626)(796.1297458,64.58509632)(796.10973633,64.51510061)
\curveto(796.08974584,64.43509647)(796.07974585,64.34009657)(796.07973633,64.23010061)
\curveto(796.07974585,64.12009679)(796.08974584,64.02509688)(796.10973633,63.94510061)
\curveto(796.1297458,63.89509701)(796.13974579,63.84509706)(796.13973633,63.79510061)
\curveto(796.13974579,63.75509715)(796.14974578,63.7100972)(796.16973633,63.66010061)
\curveto(796.21974571,63.48009743)(796.29474564,63.3100976)(796.39473633,63.15010061)
\curveto(796.48474545,63.00009791)(796.59974533,62.87009804)(796.73973633,62.76010061)
\curveto(796.85974507,62.67009824)(796.98974494,62.59009832)(797.12973633,62.52010061)
\curveto(797.26974466,62.45009846)(797.42474451,62.38509852)(797.59473633,62.32510061)
\curveto(797.70474423,62.29509861)(797.82474411,62.27509863)(797.95473633,62.26510061)
\curveto(798.07474386,62.25509865)(798.17474376,62.22009869)(798.25473633,62.16010061)
\curveto(798.29474364,62.14009877)(798.3347436,62.08009883)(798.37473633,61.98010061)
\curveto(798.38474355,61.94009897)(798.39474354,61.88009903)(798.40473633,61.80010061)
\lineto(798.40473633,61.54510061)
\curveto(798.39474354,61.45509945)(798.38474355,61.37009954)(798.37473633,61.29010061)
\curveto(798.36474357,61.22009969)(798.34974358,61.17009974)(798.32973633,61.14010061)
\curveto(798.29974363,61.10009981)(798.24474369,61.06509984)(798.16473633,61.03510061)
\curveto(798.08474385,61.0050999)(797.99974393,61.00009991)(797.90973633,61.02010061)
\curveto(797.85974407,61.03009988)(797.80974412,61.03509987)(797.75973633,61.03510061)
\lineto(797.57973633,61.06510061)
\curveto(797.47974445,61.09509981)(797.37974455,61.12009979)(797.27973633,61.14010061)
\curveto(797.17974475,61.17009974)(797.08974484,61.2050997)(797.00973633,61.24510061)
\curveto(796.89974503,61.29509961)(796.79474514,61.34009957)(796.69473633,61.38010061)
\curveto(796.58474535,61.42009949)(796.47974545,61.47009944)(796.37973633,61.53010061)
\curveto(795.83974609,61.86009905)(795.44474649,62.33009858)(795.19473633,62.94010061)
\curveto(795.14474679,63.06009785)(795.10974682,63.18509772)(795.08973633,63.31510061)
\curveto(795.06974686,63.45509745)(795.04474689,63.59509731)(795.01473633,63.73510061)
\curveto(795.00474693,63.79509711)(794.99974693,63.85509705)(794.99973633,63.91510061)
\curveto(794.99974693,63.98509692)(794.99474694,64.05009686)(794.98473633,64.11010061)
}
}
{
\newrgbcolor{curcolor}{0 0 0}
\pscustom[linestyle=none,fillstyle=solid,fillcolor=curcolor]
{
\newpath
\moveto(800.50473633,76.34470998)
\lineto(800.75973633,76.34470998)
\curveto(800.83974109,76.35470228)(800.91474102,76.34970228)(800.98473633,76.32970998)
\lineto(801.22473633,76.32970998)
\lineto(801.38973633,76.32970998)
\curveto(801.48974044,76.30970232)(801.59474034,76.29970233)(801.70473633,76.29970998)
\curveto(801.80474013,76.29970233)(801.90474003,76.28970234)(802.00473633,76.26970998)
\lineto(802.15473633,76.26970998)
\curveto(802.29473964,76.23970239)(802.4347395,76.21970241)(802.57473633,76.20970998)
\curveto(802.70473923,76.19970243)(802.8347391,76.17470246)(802.96473633,76.13470998)
\curveto(803.04473889,76.11470252)(803.1297388,76.09470254)(803.21973633,76.07470998)
\lineto(803.45973633,76.01470998)
\lineto(803.75973633,75.89470998)
\curveto(803.84973808,75.86470277)(803.93973799,75.8297028)(804.02973633,75.78970998)
\curveto(804.24973768,75.68970294)(804.46473747,75.55470308)(804.67473633,75.38470998)
\curveto(804.88473705,75.22470341)(805.05473688,75.04970358)(805.18473633,74.85970998)
\curveto(805.22473671,74.80970382)(805.26473667,74.74970388)(805.30473633,74.67970998)
\curveto(805.3347366,74.61970401)(805.36973656,74.55970407)(805.40973633,74.49970998)
\curveto(805.45973647,74.41970421)(805.49973643,74.32470431)(805.52973633,74.21470998)
\curveto(805.55973637,74.10470453)(805.58973634,73.99970463)(805.61973633,73.89970998)
\curveto(805.65973627,73.78970484)(805.68473625,73.67970495)(805.69473633,73.56970998)
\curveto(805.70473623,73.45970517)(805.71973621,73.34470529)(805.73973633,73.22470998)
\curveto(805.74973618,73.18470545)(805.74973618,73.13970549)(805.73973633,73.08970998)
\curveto(805.73973619,73.04970558)(805.74473619,73.00970562)(805.75473633,72.96970998)
\curveto(805.76473617,72.9297057)(805.76973616,72.87470576)(805.76973633,72.80470998)
\curveto(805.76973616,72.7347059)(805.76473617,72.68470595)(805.75473633,72.65470998)
\curveto(805.7347362,72.60470603)(805.7297362,72.55970607)(805.73973633,72.51970998)
\curveto(805.74973618,72.47970615)(805.74973618,72.44470619)(805.73973633,72.41470998)
\lineto(805.73973633,72.32470998)
\curveto(805.71973621,72.26470637)(805.70473623,72.19970643)(805.69473633,72.12970998)
\curveto(805.69473624,72.06970656)(805.68973624,72.00470663)(805.67973633,71.93470998)
\curveto(805.6297363,71.76470687)(805.57973635,71.60470703)(805.52973633,71.45470998)
\curveto(805.47973645,71.30470733)(805.41473652,71.15970747)(805.33473633,71.01970998)
\curveto(805.29473664,70.96970766)(805.26473667,70.91470772)(805.24473633,70.85470998)
\curveto(805.21473672,70.80470783)(805.17973675,70.75470788)(805.13973633,70.70470998)
\curveto(804.95973697,70.46470817)(804.73973719,70.26470837)(804.47973633,70.10470998)
\curveto(804.21973771,69.94470869)(803.934738,69.80470883)(803.62473633,69.68470998)
\curveto(803.48473845,69.62470901)(803.34473859,69.57970905)(803.20473633,69.54970998)
\curveto(803.05473888,69.51970911)(802.89973903,69.48470915)(802.73973633,69.44470998)
\curveto(802.6297393,69.42470921)(802.51973941,69.40970922)(802.40973633,69.39970998)
\curveto(802.29973963,69.38970924)(802.18973974,69.37470926)(802.07973633,69.35470998)
\curveto(802.03973989,69.34470929)(801.99973993,69.33970929)(801.95973633,69.33970998)
\curveto(801.91974001,69.34970928)(801.87974005,69.34970928)(801.83973633,69.33970998)
\curveto(801.78974014,69.3297093)(801.73974019,69.32470931)(801.68973633,69.32470998)
\lineto(801.52473633,69.32470998)
\curveto(801.47474046,69.30470933)(801.42474051,69.29970933)(801.37473633,69.30970998)
\curveto(801.31474062,69.31970931)(801.25974067,69.31970931)(801.20973633,69.30970998)
\curveto(801.16974076,69.29970933)(801.12474081,69.29970933)(801.07473633,69.30970998)
\curveto(801.02474091,69.31970931)(800.97474096,69.31470932)(800.92473633,69.29470998)
\curveto(800.85474108,69.27470936)(800.77974115,69.26970936)(800.69973633,69.27970998)
\curveto(800.60974132,69.28970934)(800.52474141,69.29470934)(800.44473633,69.29470998)
\curveto(800.35474158,69.29470934)(800.25474168,69.28970934)(800.14473633,69.27970998)
\curveto(800.02474191,69.26970936)(799.92474201,69.27470936)(799.84473633,69.29470998)
\lineto(799.55973633,69.29470998)
\lineto(798.92973633,69.33970998)
\curveto(798.8297431,69.34970928)(798.7347432,69.35970927)(798.64473633,69.36970998)
\lineto(798.34473633,69.39970998)
\curveto(798.29474364,69.41970921)(798.24474369,69.42470921)(798.19473633,69.41470998)
\curveto(798.1347438,69.41470922)(798.07974385,69.42470921)(798.02973633,69.44470998)
\curveto(797.85974407,69.49470914)(797.69474424,69.5347091)(797.53473633,69.56470998)
\curveto(797.36474457,69.59470904)(797.20474473,69.64470899)(797.05473633,69.71470998)
\curveto(796.59474534,69.90470873)(796.21974571,70.12470851)(795.92973633,70.37470998)
\curveto(795.63974629,70.634708)(795.39474654,70.99470764)(795.19473633,71.45470998)
\curveto(795.14474679,71.58470705)(795.10974682,71.71470692)(795.08973633,71.84470998)
\curveto(795.06974686,71.98470665)(795.04474689,72.12470651)(795.01473633,72.26470998)
\curveto(795.00474693,72.3347063)(794.99974693,72.39970623)(794.99973633,72.45970998)
\curveto(794.99974693,72.51970611)(794.99474694,72.58470605)(794.98473633,72.65470998)
\curveto(794.96474697,73.48470515)(795.11474682,74.15470448)(795.43473633,74.66470998)
\curveto(795.74474619,75.17470346)(796.18474575,75.55470308)(796.75473633,75.80470998)
\curveto(796.87474506,75.85470278)(796.99974493,75.89970273)(797.12973633,75.93970998)
\curveto(797.25974467,75.97970265)(797.39474454,76.02470261)(797.53473633,76.07470998)
\curveto(797.61474432,76.09470254)(797.69974423,76.10970252)(797.78973633,76.11970998)
\lineto(798.02973633,76.17970998)
\curveto(798.13974379,76.20970242)(798.24974368,76.22470241)(798.35973633,76.22470998)
\curveto(798.46974346,76.2347024)(798.57974335,76.24970238)(798.68973633,76.26970998)
\curveto(798.73974319,76.28970234)(798.78474315,76.29470234)(798.82473633,76.28470998)
\curveto(798.86474307,76.28470235)(798.90474303,76.28970234)(798.94473633,76.29970998)
\curveto(798.99474294,76.30970232)(799.04974288,76.30970232)(799.10973633,76.29970998)
\curveto(799.15974277,76.29970233)(799.20974272,76.30470233)(799.25973633,76.31470998)
\lineto(799.39473633,76.31470998)
\curveto(799.45474248,76.3347023)(799.52474241,76.3347023)(799.60473633,76.31470998)
\curveto(799.67474226,76.30470233)(799.73974219,76.30970232)(799.79973633,76.32970998)
\curveto(799.8297421,76.33970229)(799.86974206,76.34470229)(799.91973633,76.34470998)
\lineto(800.03973633,76.34470998)
\lineto(800.50473633,76.34470998)
\moveto(802.82973633,74.79970998)
\curveto(802.50973942,74.89970373)(802.14473979,74.95970367)(801.73473633,74.97970998)
\curveto(801.32474061,74.99970363)(800.91474102,75.00970362)(800.50473633,75.00970998)
\curveto(800.07474186,75.00970362)(799.65474228,74.99970363)(799.24473633,74.97970998)
\curveto(798.8347431,74.95970367)(798.44974348,74.91470372)(798.08973633,74.84470998)
\curveto(797.7297442,74.77470386)(797.40974452,74.66470397)(797.12973633,74.51470998)
\curveto(796.83974509,74.37470426)(796.60474533,74.17970445)(796.42473633,73.92970998)
\curveto(796.31474562,73.76970486)(796.2347457,73.58970504)(796.18473633,73.38970998)
\curveto(796.12474581,73.18970544)(796.09474584,72.94470569)(796.09473633,72.65470998)
\curveto(796.11474582,72.634706)(796.12474581,72.59970603)(796.12473633,72.54970998)
\curveto(796.11474582,72.49970613)(796.11474582,72.45970617)(796.12473633,72.42970998)
\curveto(796.14474579,72.34970628)(796.16474577,72.27470636)(796.18473633,72.20470998)
\curveto(796.19474574,72.14470649)(796.21474572,72.07970655)(796.24473633,72.00970998)
\curveto(796.36474557,71.73970689)(796.5347454,71.51970711)(796.75473633,71.34970998)
\curveto(796.96474497,71.18970744)(797.20974472,71.05470758)(797.48973633,70.94470998)
\curveto(797.59974433,70.89470774)(797.71974421,70.85470778)(797.84973633,70.82470998)
\curveto(797.96974396,70.80470783)(798.09474384,70.77970785)(798.22473633,70.74970998)
\curveto(798.27474366,70.7297079)(798.3297436,70.71970791)(798.38973633,70.71970998)
\curveto(798.43974349,70.71970791)(798.48974344,70.71470792)(798.53973633,70.70470998)
\curveto(798.6297433,70.69470794)(798.72474321,70.68470795)(798.82473633,70.67470998)
\curveto(798.91474302,70.66470797)(799.00974292,70.65470798)(799.10973633,70.64470998)
\curveto(799.18974274,70.64470799)(799.27474266,70.63970799)(799.36473633,70.62970998)
\lineto(799.60473633,70.62970998)
\lineto(799.78473633,70.62970998)
\curveto(799.81474212,70.61970801)(799.84974208,70.61470802)(799.88973633,70.61470998)
\lineto(800.02473633,70.61470998)
\lineto(800.47473633,70.61470998)
\curveto(800.55474138,70.61470802)(800.63974129,70.60970802)(800.72973633,70.59970998)
\curveto(800.80974112,70.59970803)(800.88474105,70.60970802)(800.95473633,70.62970998)
\lineto(801.22473633,70.62970998)
\curveto(801.24474069,70.629708)(801.27474066,70.62470801)(801.31473633,70.61470998)
\curveto(801.34474059,70.61470802)(801.36974056,70.61970801)(801.38973633,70.62970998)
\curveto(801.48974044,70.63970799)(801.58974034,70.64470799)(801.68973633,70.64470998)
\curveto(801.77974015,70.65470798)(801.87974005,70.66470797)(801.98973633,70.67470998)
\curveto(802.10973982,70.70470793)(802.2347397,70.71970791)(802.36473633,70.71970998)
\curveto(802.48473945,70.7297079)(802.59973933,70.75470788)(802.70973633,70.79470998)
\curveto(803.00973892,70.87470776)(803.27473866,70.95970767)(803.50473633,71.04970998)
\curveto(803.7347382,71.14970748)(803.94973798,71.29470734)(804.14973633,71.48470998)
\curveto(804.34973758,71.69470694)(804.49973743,71.95970667)(804.59973633,72.27970998)
\curveto(804.61973731,72.31970631)(804.6297373,72.35470628)(804.62973633,72.38470998)
\curveto(804.61973731,72.42470621)(804.62473731,72.46970616)(804.64473633,72.51970998)
\curveto(804.65473728,72.55970607)(804.66473727,72.629706)(804.67473633,72.72970998)
\curveto(804.68473725,72.83970579)(804.67973725,72.92470571)(804.65973633,72.98470998)
\curveto(804.63973729,73.05470558)(804.6297373,73.12470551)(804.62973633,73.19470998)
\curveto(804.61973731,73.26470537)(804.60473733,73.3297053)(804.58473633,73.38970998)
\curveto(804.52473741,73.58970504)(804.43973749,73.76970486)(804.32973633,73.92970998)
\curveto(804.30973762,73.95970467)(804.28973764,73.98470465)(804.26973633,74.00470998)
\lineto(804.20973633,74.06470998)
\curveto(804.18973774,74.10470453)(804.14973778,74.15470448)(804.08973633,74.21470998)
\curveto(803.94973798,74.31470432)(803.81973811,74.39970423)(803.69973633,74.46970998)
\curveto(803.57973835,74.53970409)(803.4347385,74.60970402)(803.26473633,74.67970998)
\curveto(803.19473874,74.70970392)(803.12473881,74.7297039)(803.05473633,74.73970998)
\curveto(802.98473895,74.75970387)(802.90973902,74.77970385)(802.82973633,74.79970998)
}
}
{
\newrgbcolor{curcolor}{0 0 0}
\pscustom[linestyle=none,fillstyle=solid,fillcolor=curcolor]
{
\newpath
\moveto(803.95473633,78.63431936)
\lineto(803.95473633,79.26431936)
\lineto(803.95473633,79.45931936)
\curveto(803.95473798,79.52931683)(803.96473797,79.58931677)(803.98473633,79.63931936)
\curveto(804.02473791,79.70931665)(804.06473787,79.7593166)(804.10473633,79.78931936)
\curveto(804.15473778,79.82931653)(804.21973771,79.84931651)(804.29973633,79.84931936)
\curveto(804.37973755,79.8593165)(804.46473747,79.86431649)(804.55473633,79.86431936)
\lineto(805.27473633,79.86431936)
\curveto(805.75473618,79.86431649)(806.16473577,79.80431655)(806.50473633,79.68431936)
\curveto(806.84473509,79.56431679)(807.11973481,79.36931699)(807.32973633,79.09931936)
\curveto(807.37973455,79.02931733)(807.42473451,78.9593174)(807.46473633,78.88931936)
\curveto(807.51473442,78.82931753)(807.55973437,78.7543176)(807.59973633,78.66431936)
\curveto(807.60973432,78.64431771)(807.61973431,78.61431774)(807.62973633,78.57431936)
\curveto(807.64973428,78.53431782)(807.65473428,78.48931787)(807.64473633,78.43931936)
\curveto(807.61473432,78.34931801)(807.53973439,78.29431806)(807.41973633,78.27431936)
\curveto(807.30973462,78.2543181)(807.21473472,78.26931809)(807.13473633,78.31931936)
\curveto(807.06473487,78.34931801)(806.99973493,78.39431796)(806.93973633,78.45431936)
\curveto(806.88973504,78.52431783)(806.83973509,78.58931777)(806.78973633,78.64931936)
\curveto(806.73973519,78.71931764)(806.66473527,78.77931758)(806.56473633,78.82931936)
\curveto(806.47473546,78.88931747)(806.38473555,78.93931742)(806.29473633,78.97931936)
\curveto(806.26473567,78.99931736)(806.20473573,79.02431733)(806.11473633,79.05431936)
\curveto(806.0347359,79.08431727)(805.96473597,79.08931727)(805.90473633,79.06931936)
\curveto(805.76473617,79.03931732)(805.67473626,78.97931738)(805.63473633,78.88931936)
\curveto(805.60473633,78.80931755)(805.58973634,78.71931764)(805.58973633,78.61931936)
\curveto(805.58973634,78.51931784)(805.56473637,78.43431792)(805.51473633,78.36431936)
\curveto(805.44473649,78.27431808)(805.30473663,78.22931813)(805.09473633,78.22931936)
\lineto(804.53973633,78.22931936)
\lineto(804.31473633,78.22931936)
\curveto(804.2347377,78.23931812)(804.16973776,78.2593181)(804.11973633,78.28931936)
\curveto(804.03973789,78.34931801)(803.99473794,78.41931794)(803.98473633,78.49931936)
\curveto(803.97473796,78.51931784)(803.96973796,78.53931782)(803.96973633,78.55931936)
\curveto(803.96973796,78.58931777)(803.96473797,78.61431774)(803.95473633,78.63431936)
}
}
{
\newrgbcolor{curcolor}{0 0 0}
\pscustom[linestyle=none,fillstyle=solid,fillcolor=curcolor]
{
}
}
{
\newrgbcolor{curcolor}{0 0 0}
\pscustom[linestyle=none,fillstyle=solid,fillcolor=curcolor]
{
\newpath
\moveto(794.98473633,89.26463186)
\curveto(794.97474696,89.95462722)(795.09474684,90.55462662)(795.34473633,91.06463186)
\curveto(795.59474634,91.58462559)(795.929746,91.9796252)(796.34973633,92.24963186)
\curveto(796.4297455,92.29962488)(796.51974541,92.34462483)(796.61973633,92.38463186)
\curveto(796.70974522,92.42462475)(796.80474513,92.46962471)(796.90473633,92.51963186)
\curveto(797.00474493,92.55962462)(797.10474483,92.58962459)(797.20473633,92.60963186)
\curveto(797.30474463,92.62962455)(797.40974452,92.64962453)(797.51973633,92.66963186)
\curveto(797.56974436,92.68962449)(797.61474432,92.69462448)(797.65473633,92.68463186)
\curveto(797.69474424,92.6746245)(797.73974419,92.6796245)(797.78973633,92.69963186)
\curveto(797.83974409,92.70962447)(797.92474401,92.71462446)(798.04473633,92.71463186)
\curveto(798.15474378,92.71462446)(798.23974369,92.70962447)(798.29973633,92.69963186)
\curveto(798.35974357,92.6796245)(798.41974351,92.66962451)(798.47973633,92.66963186)
\curveto(798.53974339,92.6796245)(798.59974333,92.6746245)(798.65973633,92.65463186)
\curveto(798.79974313,92.61462456)(798.934743,92.5796246)(799.06473633,92.54963186)
\curveto(799.19474274,92.51962466)(799.31974261,92.4796247)(799.43973633,92.42963186)
\curveto(799.57974235,92.36962481)(799.70474223,92.29962488)(799.81473633,92.21963186)
\curveto(799.92474201,92.14962503)(800.0347419,92.0746251)(800.14473633,91.99463186)
\lineto(800.20473633,91.93463186)
\curveto(800.22474171,91.92462525)(800.24474169,91.90962527)(800.26473633,91.88963186)
\curveto(800.42474151,91.76962541)(800.56974136,91.63462554)(800.69973633,91.48463186)
\curveto(800.8297411,91.33462584)(800.95474098,91.174626)(801.07473633,91.00463186)
\curveto(801.29474064,90.69462648)(801.49974043,90.39962678)(801.68973633,90.11963186)
\curveto(801.8297401,89.88962729)(801.96473997,89.65962752)(802.09473633,89.42963186)
\curveto(802.22473971,89.20962797)(802.35973957,88.98962819)(802.49973633,88.76963186)
\curveto(802.66973926,88.51962866)(802.84973908,88.2796289)(803.03973633,88.04963186)
\curveto(803.2297387,87.82962935)(803.45473848,87.63962954)(803.71473633,87.47963186)
\curveto(803.77473816,87.43962974)(803.8347381,87.40462977)(803.89473633,87.37463186)
\curveto(803.94473799,87.34462983)(804.00973792,87.31462986)(804.08973633,87.28463186)
\curveto(804.15973777,87.26462991)(804.21973771,87.25962992)(804.26973633,87.26963186)
\curveto(804.33973759,87.28962989)(804.39473754,87.32462985)(804.43473633,87.37463186)
\curveto(804.46473747,87.42462975)(804.48473745,87.48462969)(804.49473633,87.55463186)
\lineto(804.49473633,87.79463186)
\lineto(804.49473633,88.54463186)
\lineto(804.49473633,91.34963186)
\lineto(804.49473633,92.00963186)
\curveto(804.49473744,92.09962508)(804.49973743,92.18462499)(804.50973633,92.26463186)
\curveto(804.50973742,92.34462483)(804.5297374,92.40962477)(804.56973633,92.45963186)
\curveto(804.60973732,92.50962467)(804.68473725,92.54962463)(804.79473633,92.57963186)
\curveto(804.89473704,92.61962456)(804.99473694,92.62962455)(805.09473633,92.60963186)
\lineto(805.22973633,92.60963186)
\curveto(805.29973663,92.58962459)(805.35973657,92.56962461)(805.40973633,92.54963186)
\curveto(805.45973647,92.52962465)(805.49973643,92.49462468)(805.52973633,92.44463186)
\curveto(805.56973636,92.39462478)(805.58973634,92.32462485)(805.58973633,92.23463186)
\lineto(805.58973633,91.96463186)
\lineto(805.58973633,91.06463186)
\lineto(805.58973633,87.55463186)
\lineto(805.58973633,86.48963186)
\curveto(805.58973634,86.40963077)(805.59473634,86.31963086)(805.60473633,86.21963186)
\curveto(805.60473633,86.11963106)(805.59473634,86.03463114)(805.57473633,85.96463186)
\curveto(805.50473643,85.75463142)(805.32473661,85.68963149)(805.03473633,85.76963186)
\curveto(804.99473694,85.7796314)(804.95973697,85.7796314)(804.92973633,85.76963186)
\curveto(804.88973704,85.76963141)(804.84473709,85.7796314)(804.79473633,85.79963186)
\curveto(804.71473722,85.81963136)(804.6297373,85.83963134)(804.53973633,85.85963186)
\curveto(804.44973748,85.8796313)(804.36473757,85.90463127)(804.28473633,85.93463186)
\curveto(803.79473814,86.09463108)(803.37973855,86.29463088)(803.03973633,86.53463186)
\curveto(802.78973914,86.71463046)(802.56473937,86.91963026)(802.36473633,87.14963186)
\curveto(802.15473978,87.3796298)(801.95973997,87.61962956)(801.77973633,87.86963186)
\curveto(801.59974033,88.12962905)(801.4297405,88.39462878)(801.26973633,88.66463186)
\curveto(801.09974083,88.94462823)(800.92474101,89.21462796)(800.74473633,89.47463186)
\curveto(800.66474127,89.58462759)(800.58974134,89.68962749)(800.51973633,89.78963186)
\curveto(800.44974148,89.89962728)(800.37474156,90.00962717)(800.29473633,90.11963186)
\curveto(800.26474167,90.15962702)(800.2347417,90.19462698)(800.20473633,90.22463186)
\curveto(800.16474177,90.26462691)(800.1347418,90.30462687)(800.11473633,90.34463186)
\curveto(800.00474193,90.48462669)(799.87974205,90.60962657)(799.73973633,90.71963186)
\curveto(799.70974222,90.73962644)(799.68474225,90.76462641)(799.66473633,90.79463186)
\curveto(799.6347423,90.82462635)(799.60474233,90.84962633)(799.57473633,90.86963186)
\curveto(799.47474246,90.94962623)(799.37474256,91.01462616)(799.27473633,91.06463186)
\curveto(799.17474276,91.12462605)(799.06474287,91.179626)(798.94473633,91.22963186)
\curveto(798.87474306,91.25962592)(798.79974313,91.2796259)(798.71973633,91.28963186)
\lineto(798.47973633,91.34963186)
\lineto(798.38973633,91.34963186)
\curveto(798.35974357,91.35962582)(798.3297436,91.36462581)(798.29973633,91.36463186)
\curveto(798.2297437,91.38462579)(798.1347438,91.38962579)(798.01473633,91.37963186)
\curveto(797.88474405,91.3796258)(797.78474415,91.36962581)(797.71473633,91.34963186)
\curveto(797.6347443,91.32962585)(797.55974437,91.30962587)(797.48973633,91.28963186)
\curveto(797.40974452,91.2796259)(797.3297446,91.25962592)(797.24973633,91.22963186)
\curveto(797.00974492,91.11962606)(796.80974512,90.96962621)(796.64973633,90.77963186)
\curveto(796.47974545,90.59962658)(796.33974559,90.3796268)(796.22973633,90.11963186)
\curveto(796.20974572,90.04962713)(796.19474574,89.9796272)(796.18473633,89.90963186)
\curveto(796.16474577,89.83962734)(796.14474579,89.76462741)(796.12473633,89.68463186)
\curveto(796.10474583,89.60462757)(796.09474584,89.49462768)(796.09473633,89.35463186)
\curveto(796.09474584,89.22462795)(796.10474583,89.11962806)(796.12473633,89.03963186)
\curveto(796.1347458,88.9796282)(796.13974579,88.92462825)(796.13973633,88.87463186)
\curveto(796.13974579,88.82462835)(796.14974578,88.7746284)(796.16973633,88.72463186)
\curveto(796.20974572,88.62462855)(796.24974568,88.52962865)(796.28973633,88.43963186)
\curveto(796.3297456,88.35962882)(796.37474556,88.2796289)(796.42473633,88.19963186)
\curveto(796.44474549,88.16962901)(796.46974546,88.13962904)(796.49973633,88.10963186)
\curveto(796.5297454,88.08962909)(796.55474538,88.06462911)(796.57473633,88.03463186)
\lineto(796.64973633,87.95963186)
\curveto(796.66974526,87.92962925)(796.68974524,87.90462927)(796.70973633,87.88463186)
\lineto(796.91973633,87.73463186)
\curveto(796.97974495,87.69462948)(797.04474489,87.64962953)(797.11473633,87.59963186)
\curveto(797.20474473,87.53962964)(797.30974462,87.48962969)(797.42973633,87.44963186)
\curveto(797.53974439,87.41962976)(797.64974428,87.38462979)(797.75973633,87.34463186)
\curveto(797.86974406,87.30462987)(798.01474392,87.2796299)(798.19473633,87.26963186)
\curveto(798.36474357,87.25962992)(798.48974344,87.22962995)(798.56973633,87.17963186)
\curveto(798.64974328,87.12963005)(798.69474324,87.05463012)(798.70473633,86.95463186)
\curveto(798.71474322,86.85463032)(798.71974321,86.74463043)(798.71973633,86.62463186)
\curveto(798.71974321,86.58463059)(798.72474321,86.54463063)(798.73473633,86.50463186)
\curveto(798.7347432,86.46463071)(798.7297432,86.42963075)(798.71973633,86.39963186)
\curveto(798.69974323,86.34963083)(798.68974324,86.29963088)(798.68973633,86.24963186)
\curveto(798.68974324,86.20963097)(798.67974325,86.16963101)(798.65973633,86.12963186)
\curveto(798.59974333,86.03963114)(798.46474347,85.99463118)(798.25473633,85.99463186)
\lineto(798.13473633,85.99463186)
\curveto(798.07474386,86.00463117)(798.01474392,86.00963117)(797.95473633,86.00963186)
\curveto(797.88474405,86.01963116)(797.81974411,86.02963115)(797.75973633,86.03963186)
\curveto(797.64974428,86.05963112)(797.54974438,86.0796311)(797.45973633,86.09963186)
\curveto(797.35974457,86.11963106)(797.26474467,86.14963103)(797.17473633,86.18963186)
\curveto(797.10474483,86.20963097)(797.04474489,86.22963095)(796.99473633,86.24963186)
\lineto(796.81473633,86.30963186)
\curveto(796.55474538,86.42963075)(796.30974562,86.58463059)(796.07973633,86.77463186)
\curveto(795.84974608,86.9746302)(795.66474627,87.18962999)(795.52473633,87.41963186)
\curveto(795.44474649,87.52962965)(795.37974655,87.64462953)(795.32973633,87.76463186)
\lineto(795.17973633,88.15463186)
\curveto(795.1297468,88.26462891)(795.09974683,88.3796288)(795.08973633,88.49963186)
\curveto(795.06974686,88.61962856)(795.04474689,88.74462843)(795.01473633,88.87463186)
\curveto(795.01474692,88.94462823)(795.01474692,89.00962817)(795.01473633,89.06963186)
\curveto(795.00474693,89.12962805)(794.99474694,89.19462798)(794.98473633,89.26463186)
}
}
{
\newrgbcolor{curcolor}{0 0 0}
\pscustom[linestyle=none,fillstyle=solid,fillcolor=curcolor]
{
\newpath
\moveto(800.50473633,101.36424123)
\lineto(800.75973633,101.36424123)
\curveto(800.83974109,101.37423353)(800.91474102,101.36923353)(800.98473633,101.34924123)
\lineto(801.22473633,101.34924123)
\lineto(801.38973633,101.34924123)
\curveto(801.48974044,101.32923357)(801.59474034,101.31923358)(801.70473633,101.31924123)
\curveto(801.80474013,101.31923358)(801.90474003,101.30923359)(802.00473633,101.28924123)
\lineto(802.15473633,101.28924123)
\curveto(802.29473964,101.25923364)(802.4347395,101.23923366)(802.57473633,101.22924123)
\curveto(802.70473923,101.21923368)(802.8347391,101.19423371)(802.96473633,101.15424123)
\curveto(803.04473889,101.13423377)(803.1297388,101.11423379)(803.21973633,101.09424123)
\lineto(803.45973633,101.03424123)
\lineto(803.75973633,100.91424123)
\curveto(803.84973808,100.88423402)(803.93973799,100.84923405)(804.02973633,100.80924123)
\curveto(804.24973768,100.70923419)(804.46473747,100.57423433)(804.67473633,100.40424123)
\curveto(804.88473705,100.24423466)(805.05473688,100.06923483)(805.18473633,99.87924123)
\curveto(805.22473671,99.82923507)(805.26473667,99.76923513)(805.30473633,99.69924123)
\curveto(805.3347366,99.63923526)(805.36973656,99.57923532)(805.40973633,99.51924123)
\curveto(805.45973647,99.43923546)(805.49973643,99.34423556)(805.52973633,99.23424123)
\curveto(805.55973637,99.12423578)(805.58973634,99.01923588)(805.61973633,98.91924123)
\curveto(805.65973627,98.80923609)(805.68473625,98.6992362)(805.69473633,98.58924123)
\curveto(805.70473623,98.47923642)(805.71973621,98.36423654)(805.73973633,98.24424123)
\curveto(805.74973618,98.2042367)(805.74973618,98.15923674)(805.73973633,98.10924123)
\curveto(805.73973619,98.06923683)(805.74473619,98.02923687)(805.75473633,97.98924123)
\curveto(805.76473617,97.94923695)(805.76973616,97.89423701)(805.76973633,97.82424123)
\curveto(805.76973616,97.75423715)(805.76473617,97.7042372)(805.75473633,97.67424123)
\curveto(805.7347362,97.62423728)(805.7297362,97.57923732)(805.73973633,97.53924123)
\curveto(805.74973618,97.4992374)(805.74973618,97.46423744)(805.73973633,97.43424123)
\lineto(805.73973633,97.34424123)
\curveto(805.71973621,97.28423762)(805.70473623,97.21923768)(805.69473633,97.14924123)
\curveto(805.69473624,97.08923781)(805.68973624,97.02423788)(805.67973633,96.95424123)
\curveto(805.6297363,96.78423812)(805.57973635,96.62423828)(805.52973633,96.47424123)
\curveto(805.47973645,96.32423858)(805.41473652,96.17923872)(805.33473633,96.03924123)
\curveto(805.29473664,95.98923891)(805.26473667,95.93423897)(805.24473633,95.87424123)
\curveto(805.21473672,95.82423908)(805.17973675,95.77423913)(805.13973633,95.72424123)
\curveto(804.95973697,95.48423942)(804.73973719,95.28423962)(804.47973633,95.12424123)
\curveto(804.21973771,94.96423994)(803.934738,94.82424008)(803.62473633,94.70424123)
\curveto(803.48473845,94.64424026)(803.34473859,94.5992403)(803.20473633,94.56924123)
\curveto(803.05473888,94.53924036)(802.89973903,94.5042404)(802.73973633,94.46424123)
\curveto(802.6297393,94.44424046)(802.51973941,94.42924047)(802.40973633,94.41924123)
\curveto(802.29973963,94.40924049)(802.18973974,94.39424051)(802.07973633,94.37424123)
\curveto(802.03973989,94.36424054)(801.99973993,94.35924054)(801.95973633,94.35924123)
\curveto(801.91974001,94.36924053)(801.87974005,94.36924053)(801.83973633,94.35924123)
\curveto(801.78974014,94.34924055)(801.73974019,94.34424056)(801.68973633,94.34424123)
\lineto(801.52473633,94.34424123)
\curveto(801.47474046,94.32424058)(801.42474051,94.31924058)(801.37473633,94.32924123)
\curveto(801.31474062,94.33924056)(801.25974067,94.33924056)(801.20973633,94.32924123)
\curveto(801.16974076,94.31924058)(801.12474081,94.31924058)(801.07473633,94.32924123)
\curveto(801.02474091,94.33924056)(800.97474096,94.33424057)(800.92473633,94.31424123)
\curveto(800.85474108,94.29424061)(800.77974115,94.28924061)(800.69973633,94.29924123)
\curveto(800.60974132,94.30924059)(800.52474141,94.31424059)(800.44473633,94.31424123)
\curveto(800.35474158,94.31424059)(800.25474168,94.30924059)(800.14473633,94.29924123)
\curveto(800.02474191,94.28924061)(799.92474201,94.29424061)(799.84473633,94.31424123)
\lineto(799.55973633,94.31424123)
\lineto(798.92973633,94.35924123)
\curveto(798.8297431,94.36924053)(798.7347432,94.37924052)(798.64473633,94.38924123)
\lineto(798.34473633,94.41924123)
\curveto(798.29474364,94.43924046)(798.24474369,94.44424046)(798.19473633,94.43424123)
\curveto(798.1347438,94.43424047)(798.07974385,94.44424046)(798.02973633,94.46424123)
\curveto(797.85974407,94.51424039)(797.69474424,94.55424035)(797.53473633,94.58424123)
\curveto(797.36474457,94.61424029)(797.20474473,94.66424024)(797.05473633,94.73424123)
\curveto(796.59474534,94.92423998)(796.21974571,95.14423976)(795.92973633,95.39424123)
\curveto(795.63974629,95.65423925)(795.39474654,96.01423889)(795.19473633,96.47424123)
\curveto(795.14474679,96.6042383)(795.10974682,96.73423817)(795.08973633,96.86424123)
\curveto(795.06974686,97.0042379)(795.04474689,97.14423776)(795.01473633,97.28424123)
\curveto(795.00474693,97.35423755)(794.99974693,97.41923748)(794.99973633,97.47924123)
\curveto(794.99974693,97.53923736)(794.99474694,97.6042373)(794.98473633,97.67424123)
\curveto(794.96474697,98.5042364)(795.11474682,99.17423573)(795.43473633,99.68424123)
\curveto(795.74474619,100.19423471)(796.18474575,100.57423433)(796.75473633,100.82424123)
\curveto(796.87474506,100.87423403)(796.99974493,100.91923398)(797.12973633,100.95924123)
\curveto(797.25974467,100.9992339)(797.39474454,101.04423386)(797.53473633,101.09424123)
\curveto(797.61474432,101.11423379)(797.69974423,101.12923377)(797.78973633,101.13924123)
\lineto(798.02973633,101.19924123)
\curveto(798.13974379,101.22923367)(798.24974368,101.24423366)(798.35973633,101.24424123)
\curveto(798.46974346,101.25423365)(798.57974335,101.26923363)(798.68973633,101.28924123)
\curveto(798.73974319,101.30923359)(798.78474315,101.31423359)(798.82473633,101.30424123)
\curveto(798.86474307,101.3042336)(798.90474303,101.30923359)(798.94473633,101.31924123)
\curveto(798.99474294,101.32923357)(799.04974288,101.32923357)(799.10973633,101.31924123)
\curveto(799.15974277,101.31923358)(799.20974272,101.32423358)(799.25973633,101.33424123)
\lineto(799.39473633,101.33424123)
\curveto(799.45474248,101.35423355)(799.52474241,101.35423355)(799.60473633,101.33424123)
\curveto(799.67474226,101.32423358)(799.73974219,101.32923357)(799.79973633,101.34924123)
\curveto(799.8297421,101.35923354)(799.86974206,101.36423354)(799.91973633,101.36424123)
\lineto(800.03973633,101.36424123)
\lineto(800.50473633,101.36424123)
\moveto(802.82973633,99.81924123)
\curveto(802.50973942,99.91923498)(802.14473979,99.97923492)(801.73473633,99.99924123)
\curveto(801.32474061,100.01923488)(800.91474102,100.02923487)(800.50473633,100.02924123)
\curveto(800.07474186,100.02923487)(799.65474228,100.01923488)(799.24473633,99.99924123)
\curveto(798.8347431,99.97923492)(798.44974348,99.93423497)(798.08973633,99.86424123)
\curveto(797.7297442,99.79423511)(797.40974452,99.68423522)(797.12973633,99.53424123)
\curveto(796.83974509,99.39423551)(796.60474533,99.1992357)(796.42473633,98.94924123)
\curveto(796.31474562,98.78923611)(796.2347457,98.60923629)(796.18473633,98.40924123)
\curveto(796.12474581,98.20923669)(796.09474584,97.96423694)(796.09473633,97.67424123)
\curveto(796.11474582,97.65423725)(796.12474581,97.61923728)(796.12473633,97.56924123)
\curveto(796.11474582,97.51923738)(796.11474582,97.47923742)(796.12473633,97.44924123)
\curveto(796.14474579,97.36923753)(796.16474577,97.29423761)(796.18473633,97.22424123)
\curveto(796.19474574,97.16423774)(796.21474572,97.0992378)(796.24473633,97.02924123)
\curveto(796.36474557,96.75923814)(796.5347454,96.53923836)(796.75473633,96.36924123)
\curveto(796.96474497,96.20923869)(797.20974472,96.07423883)(797.48973633,95.96424123)
\curveto(797.59974433,95.91423899)(797.71974421,95.87423903)(797.84973633,95.84424123)
\curveto(797.96974396,95.82423908)(798.09474384,95.7992391)(798.22473633,95.76924123)
\curveto(798.27474366,95.74923915)(798.3297436,95.73923916)(798.38973633,95.73924123)
\curveto(798.43974349,95.73923916)(798.48974344,95.73423917)(798.53973633,95.72424123)
\curveto(798.6297433,95.71423919)(798.72474321,95.7042392)(798.82473633,95.69424123)
\curveto(798.91474302,95.68423922)(799.00974292,95.67423923)(799.10973633,95.66424123)
\curveto(799.18974274,95.66423924)(799.27474266,95.65923924)(799.36473633,95.64924123)
\lineto(799.60473633,95.64924123)
\lineto(799.78473633,95.64924123)
\curveto(799.81474212,95.63923926)(799.84974208,95.63423927)(799.88973633,95.63424123)
\lineto(800.02473633,95.63424123)
\lineto(800.47473633,95.63424123)
\curveto(800.55474138,95.63423927)(800.63974129,95.62923927)(800.72973633,95.61924123)
\curveto(800.80974112,95.61923928)(800.88474105,95.62923927)(800.95473633,95.64924123)
\lineto(801.22473633,95.64924123)
\curveto(801.24474069,95.64923925)(801.27474066,95.64423926)(801.31473633,95.63424123)
\curveto(801.34474059,95.63423927)(801.36974056,95.63923926)(801.38973633,95.64924123)
\curveto(801.48974044,95.65923924)(801.58974034,95.66423924)(801.68973633,95.66424123)
\curveto(801.77974015,95.67423923)(801.87974005,95.68423922)(801.98973633,95.69424123)
\curveto(802.10973982,95.72423918)(802.2347397,95.73923916)(802.36473633,95.73924123)
\curveto(802.48473945,95.74923915)(802.59973933,95.77423913)(802.70973633,95.81424123)
\curveto(803.00973892,95.89423901)(803.27473866,95.97923892)(803.50473633,96.06924123)
\curveto(803.7347382,96.16923873)(803.94973798,96.31423859)(804.14973633,96.50424123)
\curveto(804.34973758,96.71423819)(804.49973743,96.97923792)(804.59973633,97.29924123)
\curveto(804.61973731,97.33923756)(804.6297373,97.37423753)(804.62973633,97.40424123)
\curveto(804.61973731,97.44423746)(804.62473731,97.48923741)(804.64473633,97.53924123)
\curveto(804.65473728,97.57923732)(804.66473727,97.64923725)(804.67473633,97.74924123)
\curveto(804.68473725,97.85923704)(804.67973725,97.94423696)(804.65973633,98.00424123)
\curveto(804.63973729,98.07423683)(804.6297373,98.14423676)(804.62973633,98.21424123)
\curveto(804.61973731,98.28423662)(804.60473733,98.34923655)(804.58473633,98.40924123)
\curveto(804.52473741,98.60923629)(804.43973749,98.78923611)(804.32973633,98.94924123)
\curveto(804.30973762,98.97923592)(804.28973764,99.0042359)(804.26973633,99.02424123)
\lineto(804.20973633,99.08424123)
\curveto(804.18973774,99.12423578)(804.14973778,99.17423573)(804.08973633,99.23424123)
\curveto(803.94973798,99.33423557)(803.81973811,99.41923548)(803.69973633,99.48924123)
\curveto(803.57973835,99.55923534)(803.4347385,99.62923527)(803.26473633,99.69924123)
\curveto(803.19473874,99.72923517)(803.12473881,99.74923515)(803.05473633,99.75924123)
\curveto(802.98473895,99.77923512)(802.90973902,99.7992351)(802.82973633,99.81924123)
}
}
{
\newrgbcolor{curcolor}{0 0 0}
\pscustom[linestyle=none,fillstyle=solid,fillcolor=curcolor]
{
\newpath
\moveto(794.98473633,106.77385061)
\curveto(794.98474695,106.87384575)(794.99474694,106.96884566)(795.01473633,107.05885061)
\curveto(795.02474691,107.14884548)(795.05474688,107.21384541)(795.10473633,107.25385061)
\curveto(795.18474675,107.31384531)(795.28974664,107.34384528)(795.41973633,107.34385061)
\lineto(795.80973633,107.34385061)
\lineto(797.30973633,107.34385061)
\lineto(803.69973633,107.34385061)
\lineto(804.86973633,107.34385061)
\lineto(805.18473633,107.34385061)
\curveto(805.28473665,107.35384527)(805.36473657,107.33884529)(805.42473633,107.29885061)
\curveto(805.50473643,107.24884538)(805.55473638,107.17384545)(805.57473633,107.07385061)
\curveto(805.58473635,106.98384564)(805.58973634,106.87384575)(805.58973633,106.74385061)
\lineto(805.58973633,106.51885061)
\curveto(805.56973636,106.43884619)(805.55473638,106.36884626)(805.54473633,106.30885061)
\curveto(805.52473641,106.24884638)(805.48473645,106.19884643)(805.42473633,106.15885061)
\curveto(805.36473657,106.11884651)(805.28973664,106.09884653)(805.19973633,106.09885061)
\lineto(804.89973633,106.09885061)
\lineto(803.80473633,106.09885061)
\lineto(798.46473633,106.09885061)
\curveto(798.37474356,106.07884655)(798.29974363,106.06384656)(798.23973633,106.05385061)
\curveto(798.16974376,106.05384657)(798.10974382,106.0238466)(798.05973633,105.96385061)
\curveto(798.00974392,105.89384673)(797.98474395,105.80384682)(797.98473633,105.69385061)
\curveto(797.97474396,105.59384703)(797.96974396,105.48384714)(797.96973633,105.36385061)
\lineto(797.96973633,104.22385061)
\lineto(797.96973633,103.72885061)
\curveto(797.95974397,103.56884906)(797.89974403,103.45884917)(797.78973633,103.39885061)
\curveto(797.75974417,103.37884925)(797.7297442,103.36884926)(797.69973633,103.36885061)
\curveto(797.65974427,103.36884926)(797.61474432,103.36384926)(797.56473633,103.35385061)
\curveto(797.44474449,103.33384929)(797.3347446,103.33884929)(797.23473633,103.36885061)
\curveto(797.1347448,103.40884922)(797.06474487,103.46384916)(797.02473633,103.53385061)
\curveto(796.97474496,103.61384901)(796.94974498,103.73384889)(796.94973633,103.89385061)
\curveto(796.94974498,104.05384857)(796.934745,104.18884844)(796.90473633,104.29885061)
\curveto(796.89474504,104.34884828)(796.88974504,104.40384822)(796.88973633,104.46385061)
\curveto(796.87974505,104.5238481)(796.86474507,104.58384804)(796.84473633,104.64385061)
\curveto(796.79474514,104.79384783)(796.74474519,104.93884769)(796.69473633,105.07885061)
\curveto(796.6347453,105.21884741)(796.56474537,105.35384727)(796.48473633,105.48385061)
\curveto(796.39474554,105.623847)(796.28974564,105.74384688)(796.16973633,105.84385061)
\curveto(796.04974588,105.94384668)(795.91974601,106.03884659)(795.77973633,106.12885061)
\curveto(795.67974625,106.18884644)(795.56974636,106.23384639)(795.44973633,106.26385061)
\curveto(795.3297466,106.30384632)(795.22474671,106.35384627)(795.13473633,106.41385061)
\curveto(795.07474686,106.46384616)(795.0347469,106.53384609)(795.01473633,106.62385061)
\curveto(795.00474693,106.64384598)(794.99974693,106.66884596)(794.99973633,106.69885061)
\curveto(794.99974693,106.7288459)(794.99474694,106.75384587)(794.98473633,106.77385061)
}
}
{
\newrgbcolor{curcolor}{0 0 0}
\pscustom[linestyle=none,fillstyle=solid,fillcolor=curcolor]
{
\newpath
\moveto(794.98473633,115.12345998)
\curveto(794.98474695,115.22345513)(794.99474694,115.31845503)(795.01473633,115.40845998)
\curveto(795.02474691,115.49845485)(795.05474688,115.56345479)(795.10473633,115.60345998)
\curveto(795.18474675,115.66345469)(795.28974664,115.69345466)(795.41973633,115.69345998)
\lineto(795.80973633,115.69345998)
\lineto(797.30973633,115.69345998)
\lineto(803.69973633,115.69345998)
\lineto(804.86973633,115.69345998)
\lineto(805.18473633,115.69345998)
\curveto(805.28473665,115.70345465)(805.36473657,115.68845466)(805.42473633,115.64845998)
\curveto(805.50473643,115.59845475)(805.55473638,115.52345483)(805.57473633,115.42345998)
\curveto(805.58473635,115.33345502)(805.58973634,115.22345513)(805.58973633,115.09345998)
\lineto(805.58973633,114.86845998)
\curveto(805.56973636,114.78845556)(805.55473638,114.71845563)(805.54473633,114.65845998)
\curveto(805.52473641,114.59845575)(805.48473645,114.5484558)(805.42473633,114.50845998)
\curveto(805.36473657,114.46845588)(805.28973664,114.4484559)(805.19973633,114.44845998)
\lineto(804.89973633,114.44845998)
\lineto(803.80473633,114.44845998)
\lineto(798.46473633,114.44845998)
\curveto(798.37474356,114.42845592)(798.29974363,114.41345594)(798.23973633,114.40345998)
\curveto(798.16974376,114.40345595)(798.10974382,114.37345598)(798.05973633,114.31345998)
\curveto(798.00974392,114.24345611)(797.98474395,114.1534562)(797.98473633,114.04345998)
\curveto(797.97474396,113.94345641)(797.96974396,113.83345652)(797.96973633,113.71345998)
\lineto(797.96973633,112.57345998)
\lineto(797.96973633,112.07845998)
\curveto(797.95974397,111.91845843)(797.89974403,111.80845854)(797.78973633,111.74845998)
\curveto(797.75974417,111.72845862)(797.7297442,111.71845863)(797.69973633,111.71845998)
\curveto(797.65974427,111.71845863)(797.61474432,111.71345864)(797.56473633,111.70345998)
\curveto(797.44474449,111.68345867)(797.3347446,111.68845866)(797.23473633,111.71845998)
\curveto(797.1347448,111.75845859)(797.06474487,111.81345854)(797.02473633,111.88345998)
\curveto(796.97474496,111.96345839)(796.94974498,112.08345827)(796.94973633,112.24345998)
\curveto(796.94974498,112.40345795)(796.934745,112.53845781)(796.90473633,112.64845998)
\curveto(796.89474504,112.69845765)(796.88974504,112.7534576)(796.88973633,112.81345998)
\curveto(796.87974505,112.87345748)(796.86474507,112.93345742)(796.84473633,112.99345998)
\curveto(796.79474514,113.14345721)(796.74474519,113.28845706)(796.69473633,113.42845998)
\curveto(796.6347453,113.56845678)(796.56474537,113.70345665)(796.48473633,113.83345998)
\curveto(796.39474554,113.97345638)(796.28974564,114.09345626)(796.16973633,114.19345998)
\curveto(796.04974588,114.29345606)(795.91974601,114.38845596)(795.77973633,114.47845998)
\curveto(795.67974625,114.53845581)(795.56974636,114.58345577)(795.44973633,114.61345998)
\curveto(795.3297466,114.6534557)(795.22474671,114.70345565)(795.13473633,114.76345998)
\curveto(795.07474686,114.81345554)(795.0347469,114.88345547)(795.01473633,114.97345998)
\curveto(795.00474693,114.99345536)(794.99974693,115.01845533)(794.99973633,115.04845998)
\curveto(794.99974693,115.07845527)(794.99474694,115.10345525)(794.98473633,115.12345998)
}
}
{
\newrgbcolor{curcolor}{0 0 0}
\pscustom[linestyle=none,fillstyle=solid,fillcolor=curcolor]
{
\newpath
\moveto(815.82105225,42.29681936)
\curveto(815.82106294,42.36681368)(815.82106294,42.4468136)(815.82105225,42.53681936)
\curveto(815.81106295,42.62681342)(815.81106295,42.71181333)(815.82105225,42.79181936)
\curveto(815.82106294,42.88181316)(815.83106293,42.96181308)(815.85105225,43.03181936)
\curveto(815.87106289,43.11181293)(815.90106286,43.16681288)(815.94105225,43.19681936)
\curveto(815.99106277,43.22681282)(816.0660627,43.2468128)(816.16605225,43.25681936)
\curveto(816.25606251,43.27681277)(816.3610624,43.28681276)(816.48105225,43.28681936)
\curveto(816.59106217,43.29681275)(816.70606206,43.29681275)(816.82605225,43.28681936)
\lineto(817.12605225,43.28681936)
\lineto(820.14105225,43.28681936)
\lineto(823.03605225,43.28681936)
\curveto(823.3660554,43.28681276)(823.69105507,43.28181276)(824.01105225,43.27181936)
\curveto(824.32105444,43.27181277)(824.60105416,43.23181281)(824.85105225,43.15181936)
\curveto(825.20105356,43.03181301)(825.49605327,42.87681317)(825.73605225,42.68681936)
\curveto(825.9660528,42.49681355)(826.1660526,42.25681379)(826.33605225,41.96681936)
\curveto(826.38605238,41.90681414)(826.42105234,41.8418142)(826.44105225,41.77181936)
\curveto(826.4610523,41.71181433)(826.48605228,41.6418144)(826.51605225,41.56181936)
\curveto(826.5660522,41.4418146)(826.60105216,41.31181473)(826.62105225,41.17181936)
\curveto(826.65105211,41.041815)(826.68105208,40.90681514)(826.71105225,40.76681936)
\curveto(826.73105203,40.71681533)(826.73605203,40.66681538)(826.72605225,40.61681936)
\curveto(826.71605205,40.56681548)(826.71605205,40.51181553)(826.72605225,40.45181936)
\curveto(826.73605203,40.43181561)(826.73605203,40.40681564)(826.72605225,40.37681936)
\curveto(826.72605204,40.3468157)(826.73105203,40.32181572)(826.74105225,40.30181936)
\curveto(826.75105201,40.26181578)(826.75605201,40.20681584)(826.75605225,40.13681936)
\curveto(826.75605201,40.06681598)(826.75105201,40.01181603)(826.74105225,39.97181936)
\curveto(826.73105203,39.92181612)(826.73105203,39.86681618)(826.74105225,39.80681936)
\curveto(826.75105201,39.7468163)(826.74605202,39.69181635)(826.72605225,39.64181936)
\curveto(826.69605207,39.51181653)(826.67605209,39.38681666)(826.66605225,39.26681936)
\curveto(826.65605211,39.1468169)(826.63105213,39.03181701)(826.59105225,38.92181936)
\curveto(826.47105229,38.55181749)(826.30105246,38.23181781)(826.08105225,37.96181936)
\curveto(825.8610529,37.69181835)(825.58105318,37.48181856)(825.24105225,37.33181936)
\curveto(825.12105364,37.28181876)(824.99605377,37.23681881)(824.86605225,37.19681936)
\curveto(824.73605403,37.16681888)(824.60105416,37.13181891)(824.46105225,37.09181936)
\curveto(824.41105435,37.08181896)(824.37105439,37.07681897)(824.34105225,37.07681936)
\curveto(824.30105446,37.07681897)(824.25605451,37.07181897)(824.20605225,37.06181936)
\curveto(824.17605459,37.05181899)(824.14105462,37.046819)(824.10105225,37.04681936)
\curveto(824.05105471,37.046819)(824.01105475,37.041819)(823.98105225,37.03181936)
\lineto(823.81605225,37.03181936)
\curveto(823.73605503,37.01181903)(823.63605513,37.00681904)(823.51605225,37.01681936)
\curveto(823.38605538,37.02681902)(823.29605547,37.041819)(823.24605225,37.06181936)
\curveto(823.15605561,37.08181896)(823.09105567,37.13681891)(823.05105225,37.22681936)
\curveto(823.03105573,37.25681879)(823.02605574,37.28681876)(823.03605225,37.31681936)
\curveto(823.03605573,37.3468187)(823.03105573,37.38681866)(823.02105225,37.43681936)
\curveto(823.01105575,37.47681857)(823.00605576,37.51681853)(823.00605225,37.55681936)
\lineto(823.00605225,37.70681936)
\curveto(823.00605576,37.82681822)(823.01105575,37.9468181)(823.02105225,38.06681936)
\curveto(823.02105574,38.19681785)(823.05605571,38.28681776)(823.12605225,38.33681936)
\curveto(823.18605558,38.37681767)(823.24605552,38.39681765)(823.30605225,38.39681936)
\curveto(823.3660554,38.39681765)(823.43605533,38.40681764)(823.51605225,38.42681936)
\curveto(823.54605522,38.43681761)(823.58105518,38.43681761)(823.62105225,38.42681936)
\curveto(823.65105511,38.42681762)(823.67605509,38.43181761)(823.69605225,38.44181936)
\lineto(823.90605225,38.44181936)
\curveto(823.95605481,38.46181758)(824.00605476,38.46681758)(824.05605225,38.45681936)
\curveto(824.09605467,38.45681759)(824.14105462,38.46681758)(824.19105225,38.48681936)
\curveto(824.32105444,38.51681753)(824.44605432,38.5468175)(824.56605225,38.57681936)
\curveto(824.67605409,38.60681744)(824.78105398,38.65181739)(824.88105225,38.71181936)
\curveto(825.17105359,38.88181716)(825.37605339,39.15181689)(825.49605225,39.52181936)
\curveto(825.51605325,39.57181647)(825.53105323,39.62181642)(825.54105225,39.67181936)
\curveto(825.54105322,39.73181631)(825.55105321,39.78681626)(825.57105225,39.83681936)
\lineto(825.57105225,39.91181936)
\curveto(825.58105318,39.98181606)(825.59105317,40.07681597)(825.60105225,40.19681936)
\curveto(825.60105316,40.32681572)(825.59105317,40.42681562)(825.57105225,40.49681936)
\curveto(825.55105321,40.56681548)(825.53605323,40.63681541)(825.52605225,40.70681936)
\curveto(825.50605326,40.78681526)(825.48605328,40.85681519)(825.46605225,40.91681936)
\curveto(825.30605346,41.29681475)(825.03105373,41.57181447)(824.64105225,41.74181936)
\curveto(824.51105425,41.79181425)(824.35605441,41.82681422)(824.17605225,41.84681936)
\curveto(823.99605477,41.87681417)(823.81105495,41.89181415)(823.62105225,41.89181936)
\curveto(823.42105534,41.90181414)(823.22105554,41.90181414)(823.02105225,41.89181936)
\lineto(822.45105225,41.89181936)
\lineto(818.20605225,41.89181936)
\lineto(816.66105225,41.89181936)
\curveto(816.55106221,41.89181415)(816.43106233,41.88681416)(816.30105225,41.87681936)
\curveto(816.17106259,41.86681418)(816.0660627,41.88681416)(815.98605225,41.93681936)
\curveto(815.91606285,41.99681405)(815.8660629,42.07681397)(815.83605225,42.17681936)
\curveto(815.83606293,42.19681385)(815.83606293,42.21681383)(815.83605225,42.23681936)
\curveto(815.83606293,42.25681379)(815.83106293,42.27681377)(815.82105225,42.29681936)
}
}
{
\newrgbcolor{curcolor}{0 0 0}
\pscustom[linestyle=none,fillstyle=solid,fillcolor=curcolor]
{
\newpath
\moveto(818.77605225,45.83049123)
\lineto(818.77605225,46.26549123)
\curveto(818.77605999,46.41548927)(818.81605995,46.52048916)(818.89605225,46.58049123)
\curveto(818.97605979,46.63048905)(819.07605969,46.65548903)(819.19605225,46.65549123)
\curveto(819.31605945,46.66548902)(819.43605933,46.67048901)(819.55605225,46.67049123)
\lineto(820.98105225,46.67049123)
\lineto(823.24605225,46.67049123)
\lineto(823.93605225,46.67049123)
\curveto(824.1660546,46.67048901)(824.3660544,46.69548899)(824.53605225,46.74549123)
\curveto(824.98605378,46.90548878)(825.30105346,47.20548848)(825.48105225,47.64549123)
\curveto(825.57105319,47.86548782)(825.60605316,48.13048755)(825.58605225,48.44049123)
\curveto(825.55605321,48.75048693)(825.50105326,49.00048668)(825.42105225,49.19049123)
\curveto(825.28105348,49.52048616)(825.10605366,49.7804859)(824.89605225,49.97049123)
\curveto(824.67605409,50.17048551)(824.39105437,50.32548536)(824.04105225,50.43549123)
\curveto(823.9610548,50.46548522)(823.88105488,50.4854852)(823.80105225,50.49549123)
\curveto(823.72105504,50.50548518)(823.63605513,50.52048516)(823.54605225,50.54049123)
\curveto(823.49605527,50.55048513)(823.45105531,50.55048513)(823.41105225,50.54049123)
\curveto(823.37105539,50.54048514)(823.32605544,50.55048513)(823.27605225,50.57049123)
\lineto(822.96105225,50.57049123)
\curveto(822.88105588,50.59048509)(822.79105597,50.59548509)(822.69105225,50.58549123)
\curveto(822.58105618,50.57548511)(822.48105628,50.57048511)(822.39105225,50.57049123)
\lineto(821.22105225,50.57049123)
\lineto(819.63105225,50.57049123)
\curveto(819.51105925,50.57048511)(819.38605938,50.56548512)(819.25605225,50.55549123)
\curveto(819.11605965,50.55548513)(819.00605976,50.5804851)(818.92605225,50.63049123)
\curveto(818.87605989,50.67048501)(818.84605992,50.71548497)(818.83605225,50.76549123)
\curveto(818.81605995,50.82548486)(818.79605997,50.89548479)(818.77605225,50.97549123)
\lineto(818.77605225,51.20049123)
\curveto(818.77605999,51.32048436)(818.78105998,51.42548426)(818.79105225,51.51549123)
\curveto(818.80105996,51.61548407)(818.84605992,51.69048399)(818.92605225,51.74049123)
\curveto(818.97605979,51.79048389)(819.05105971,51.81548387)(819.15105225,51.81549123)
\lineto(819.43605225,51.81549123)
\lineto(820.45605225,51.81549123)
\lineto(824.49105225,51.81549123)
\lineto(825.84105225,51.81549123)
\curveto(825.9610528,51.81548387)(826.07605269,51.81048387)(826.18605225,51.80049123)
\curveto(826.28605248,51.80048388)(826.3610524,51.76548392)(826.41105225,51.69549123)
\curveto(826.44105232,51.65548403)(826.4660523,51.59548409)(826.48605225,51.51549123)
\curveto(826.49605227,51.43548425)(826.50605226,51.34548434)(826.51605225,51.24549123)
\curveto(826.51605225,51.15548453)(826.51105225,51.06548462)(826.50105225,50.97549123)
\curveto(826.49105227,50.89548479)(826.47105229,50.83548485)(826.44105225,50.79549123)
\curveto(826.40105236,50.74548494)(826.33605243,50.70048498)(826.24605225,50.66049123)
\curveto(826.20605256,50.65048503)(826.15105261,50.64048504)(826.08105225,50.63049123)
\curveto(826.01105275,50.63048505)(825.94605282,50.62548506)(825.88605225,50.61549123)
\curveto(825.81605295,50.60548508)(825.761053,50.5854851)(825.72105225,50.55549123)
\curveto(825.68105308,50.52548516)(825.6660531,50.4804852)(825.67605225,50.42049123)
\curveto(825.69605307,50.34048534)(825.75605301,50.26048542)(825.85605225,50.18049123)
\curveto(825.94605282,50.10048558)(826.01605275,50.02548566)(826.06605225,49.95549123)
\curveto(826.22605254,49.73548595)(826.3660524,49.4854862)(826.48605225,49.20549123)
\curveto(826.53605223,49.09548659)(826.5660522,48.9804867)(826.57605225,48.86049123)
\curveto(826.59605217,48.75048693)(826.62105214,48.63548705)(826.65105225,48.51549123)
\curveto(826.6610521,48.46548722)(826.6610521,48.41048727)(826.65105225,48.35049123)
\curveto(826.64105212,48.30048738)(826.64605212,48.25048743)(826.66605225,48.20049123)
\curveto(826.68605208,48.10048758)(826.68605208,48.01048767)(826.66605225,47.93049123)
\lineto(826.66605225,47.78049123)
\curveto(826.64605212,47.73048795)(826.63605213,47.67048801)(826.63605225,47.60049123)
\curveto(826.63605213,47.54048814)(826.63105213,47.4854882)(826.62105225,47.43549123)
\curveto(826.60105216,47.39548829)(826.59105217,47.35548833)(826.59105225,47.31549123)
\curveto(826.60105216,47.2854884)(826.59605217,47.24548844)(826.57605225,47.19549123)
\lineto(826.51605225,46.95549123)
\curveto(826.49605227,46.8854888)(826.4660523,46.81048887)(826.42605225,46.73049123)
\curveto(826.31605245,46.47048921)(826.17105259,46.25048943)(825.99105225,46.07049123)
\curveto(825.80105296,45.90048978)(825.57605319,45.76048992)(825.31605225,45.65049123)
\curveto(825.22605354,45.61049007)(825.13605363,45.5804901)(825.04605225,45.56049123)
\lineto(824.74605225,45.50049123)
\curveto(824.68605408,45.4804902)(824.63105413,45.47049021)(824.58105225,45.47049123)
\curveto(824.52105424,45.4804902)(824.45605431,45.47549021)(824.38605225,45.45549123)
\curveto(824.3660544,45.44549024)(824.34105442,45.44049024)(824.31105225,45.44049123)
\curveto(824.27105449,45.44049024)(824.23605453,45.43549025)(824.20605225,45.42549123)
\lineto(824.05605225,45.42549123)
\curveto(824.01605475,45.41549027)(823.97105479,45.41049027)(823.92105225,45.41049123)
\curveto(823.8610549,45.42049026)(823.80605496,45.42549026)(823.75605225,45.42549123)
\lineto(823.15605225,45.42549123)
\lineto(820.39605225,45.42549123)
\lineto(819.43605225,45.42549123)
\lineto(819.16605225,45.42549123)
\curveto(819.07605969,45.42549026)(819.00105976,45.44549024)(818.94105225,45.48549123)
\curveto(818.87105989,45.52549016)(818.82105994,45.60049008)(818.79105225,45.71049123)
\curveto(818.78105998,45.73048995)(818.78105998,45.75048993)(818.79105225,45.77049123)
\curveto(818.79105997,45.79048989)(818.78605998,45.81048987)(818.77605225,45.83049123)
}
}
{
\newrgbcolor{curcolor}{0 0 0}
\pscustom[linestyle=none,fillstyle=solid,fillcolor=curcolor]
{
\newpath
\moveto(815.82105225,54.28510061)
\curveto(815.82106294,54.41509899)(815.82106294,54.55009886)(815.82105225,54.69010061)
\curveto(815.82106294,54.84009857)(815.85606291,54.95009846)(815.92605225,55.02010061)
\curveto(815.99606277,55.07009834)(816.09106267,55.09509831)(816.21105225,55.09510061)
\curveto(816.32106244,55.1050983)(816.43606233,55.1100983)(816.55605225,55.11010061)
\lineto(817.89105225,55.11010061)
\lineto(823.96605225,55.11010061)
\lineto(825.64605225,55.11010061)
\lineto(826.03605225,55.11010061)
\curveto(826.17605259,55.1100983)(826.28605248,55.08509832)(826.36605225,55.03510061)
\curveto(826.41605235,55.0050984)(826.44605232,54.96009845)(826.45605225,54.90010061)
\curveto(826.4660523,54.85009856)(826.48105228,54.78509862)(826.50105225,54.70510061)
\lineto(826.50105225,54.49510061)
\lineto(826.50105225,54.18010061)
\curveto(826.49105227,54.08009933)(826.45605231,54.0050994)(826.39605225,53.95510061)
\curveto(826.31605245,53.9050995)(826.21605255,53.87509953)(826.09605225,53.86510061)
\lineto(825.72105225,53.86510061)
\lineto(824.34105225,53.86510061)
\lineto(818.10105225,53.86510061)
\lineto(816.63105225,53.86510061)
\curveto(816.52106224,53.86509954)(816.40606236,53.86009955)(816.28605225,53.85010061)
\curveto(816.15606261,53.85009956)(816.05606271,53.87509953)(815.98605225,53.92510061)
\curveto(815.92606284,53.96509944)(815.87606289,54.04009937)(815.83605225,54.15010061)
\curveto(815.82606294,54.17009924)(815.82606294,54.19009922)(815.83605225,54.21010061)
\curveto(815.83606293,54.24009917)(815.83106293,54.26509914)(815.82105225,54.28510061)
}
}
{
\newrgbcolor{curcolor}{0 0 0}
\pscustom[linestyle=none,fillstyle=solid,fillcolor=curcolor]
{
}
}
{
\newrgbcolor{curcolor}{0 0 0}
\pscustom[linestyle=none,fillstyle=solid,fillcolor=curcolor]
{
\newpath
\moveto(821.41605225,67.99510061)
\lineto(821.67105225,67.99510061)
\curveto(821.75105701,68.0050929)(821.82605694,68.00009291)(821.89605225,67.98010061)
\lineto(822.13605225,67.98010061)
\lineto(822.30105225,67.98010061)
\curveto(822.40105636,67.96009295)(822.50605626,67.95009296)(822.61605225,67.95010061)
\curveto(822.71605605,67.95009296)(822.81605595,67.94009297)(822.91605225,67.92010061)
\lineto(823.06605225,67.92010061)
\curveto(823.20605556,67.89009302)(823.34605542,67.87009304)(823.48605225,67.86010061)
\curveto(823.61605515,67.85009306)(823.74605502,67.82509308)(823.87605225,67.78510061)
\curveto(823.95605481,67.76509314)(824.04105472,67.74509316)(824.13105225,67.72510061)
\lineto(824.37105225,67.66510061)
\lineto(824.67105225,67.54510061)
\curveto(824.761054,67.51509339)(824.85105391,67.48009343)(824.94105225,67.44010061)
\curveto(825.1610536,67.34009357)(825.37605339,67.2050937)(825.58605225,67.03510061)
\curveto(825.79605297,66.87509403)(825.9660528,66.70009421)(826.09605225,66.51010061)
\curveto(826.13605263,66.46009445)(826.17605259,66.40009451)(826.21605225,66.33010061)
\curveto(826.24605252,66.27009464)(826.28105248,66.2100947)(826.32105225,66.15010061)
\curveto(826.37105239,66.07009484)(826.41105235,65.97509493)(826.44105225,65.86510061)
\curveto(826.47105229,65.75509515)(826.50105226,65.65009526)(826.53105225,65.55010061)
\curveto(826.57105219,65.44009547)(826.59605217,65.33009558)(826.60605225,65.22010061)
\curveto(826.61605215,65.1100958)(826.63105213,64.99509591)(826.65105225,64.87510061)
\curveto(826.6610521,64.83509607)(826.6610521,64.79009612)(826.65105225,64.74010061)
\curveto(826.65105211,64.70009621)(826.65605211,64.66009625)(826.66605225,64.62010061)
\curveto(826.67605209,64.58009633)(826.68105208,64.52509638)(826.68105225,64.45510061)
\curveto(826.68105208,64.38509652)(826.67605209,64.33509657)(826.66605225,64.30510061)
\curveto(826.64605212,64.25509665)(826.64105212,64.2100967)(826.65105225,64.17010061)
\curveto(826.6610521,64.13009678)(826.6610521,64.09509681)(826.65105225,64.06510061)
\lineto(826.65105225,63.97510061)
\curveto(826.63105213,63.91509699)(826.61605215,63.85009706)(826.60605225,63.78010061)
\curveto(826.60605216,63.72009719)(826.60105216,63.65509725)(826.59105225,63.58510061)
\curveto(826.54105222,63.41509749)(826.49105227,63.25509765)(826.44105225,63.10510061)
\curveto(826.39105237,62.95509795)(826.32605244,62.8100981)(826.24605225,62.67010061)
\curveto(826.20605256,62.62009829)(826.17605259,62.56509834)(826.15605225,62.50510061)
\curveto(826.12605264,62.45509845)(826.09105267,62.4050985)(826.05105225,62.35510061)
\curveto(825.87105289,62.11509879)(825.65105311,61.91509899)(825.39105225,61.75510061)
\curveto(825.13105363,61.59509931)(824.84605392,61.45509945)(824.53605225,61.33510061)
\curveto(824.39605437,61.27509963)(824.25605451,61.23009968)(824.11605225,61.20010061)
\curveto(823.9660548,61.17009974)(823.81105495,61.13509977)(823.65105225,61.09510061)
\curveto(823.54105522,61.07509983)(823.43105533,61.06009985)(823.32105225,61.05010061)
\curveto(823.21105555,61.04009987)(823.10105566,61.02509988)(822.99105225,61.00510061)
\curveto(822.95105581,60.99509991)(822.91105585,60.99009992)(822.87105225,60.99010061)
\curveto(822.83105593,61.00009991)(822.79105597,61.00009991)(822.75105225,60.99010061)
\curveto(822.70105606,60.98009993)(822.65105611,60.97509993)(822.60105225,60.97510061)
\lineto(822.43605225,60.97510061)
\curveto(822.38605638,60.95509995)(822.33605643,60.95009996)(822.28605225,60.96010061)
\curveto(822.22605654,60.97009994)(822.17105659,60.97009994)(822.12105225,60.96010061)
\curveto(822.08105668,60.95009996)(822.03605673,60.95009996)(821.98605225,60.96010061)
\curveto(821.93605683,60.97009994)(821.88605688,60.96509994)(821.83605225,60.94510061)
\curveto(821.766057,60.92509998)(821.69105707,60.92009999)(821.61105225,60.93010061)
\curveto(821.52105724,60.94009997)(821.43605733,60.94509996)(821.35605225,60.94510061)
\curveto(821.2660575,60.94509996)(821.1660576,60.94009997)(821.05605225,60.93010061)
\curveto(820.93605783,60.92009999)(820.83605793,60.92509998)(820.75605225,60.94510061)
\lineto(820.47105225,60.94510061)
\lineto(819.84105225,60.99010061)
\curveto(819.74105902,61.00009991)(819.64605912,61.0100999)(819.55605225,61.02010061)
\lineto(819.25605225,61.05010061)
\curveto(819.20605956,61.07009984)(819.15605961,61.07509983)(819.10605225,61.06510061)
\curveto(819.04605972,61.06509984)(818.99105977,61.07509983)(818.94105225,61.09510061)
\curveto(818.77105999,61.14509976)(818.60606016,61.18509972)(818.44605225,61.21510061)
\curveto(818.27606049,61.24509966)(818.11606065,61.29509961)(817.96605225,61.36510061)
\curveto(817.50606126,61.55509935)(817.13106163,61.77509913)(816.84105225,62.02510061)
\curveto(816.55106221,62.28509862)(816.30606246,62.64509826)(816.10605225,63.10510061)
\curveto(816.05606271,63.23509767)(816.02106274,63.36509754)(816.00105225,63.49510061)
\curveto(815.98106278,63.63509727)(815.95606281,63.77509713)(815.92605225,63.91510061)
\curveto(815.91606285,63.98509692)(815.91106285,64.05009686)(815.91105225,64.11010061)
\curveto(815.91106285,64.17009674)(815.90606286,64.23509667)(815.89605225,64.30510061)
\curveto(815.87606289,65.13509577)(816.02606274,65.8050951)(816.34605225,66.31510061)
\curveto(816.65606211,66.82509408)(817.09606167,67.2050937)(817.66605225,67.45510061)
\curveto(817.78606098,67.5050934)(817.91106085,67.55009336)(818.04105225,67.59010061)
\curveto(818.17106059,67.63009328)(818.30606046,67.67509323)(818.44605225,67.72510061)
\curveto(818.52606024,67.74509316)(818.61106015,67.76009315)(818.70105225,67.77010061)
\lineto(818.94105225,67.83010061)
\curveto(819.05105971,67.86009305)(819.1610596,67.87509303)(819.27105225,67.87510061)
\curveto(819.38105938,67.88509302)(819.49105927,67.90009301)(819.60105225,67.92010061)
\curveto(819.65105911,67.94009297)(819.69605907,67.94509296)(819.73605225,67.93510061)
\curveto(819.77605899,67.93509297)(819.81605895,67.94009297)(819.85605225,67.95010061)
\curveto(819.90605886,67.96009295)(819.9610588,67.96009295)(820.02105225,67.95010061)
\curveto(820.07105869,67.95009296)(820.12105864,67.95509295)(820.17105225,67.96510061)
\lineto(820.30605225,67.96510061)
\curveto(820.3660584,67.98509292)(820.43605833,67.98509292)(820.51605225,67.96510061)
\curveto(820.58605818,67.95509295)(820.65105811,67.96009295)(820.71105225,67.98010061)
\curveto(820.74105802,67.99009292)(820.78105798,67.99509291)(820.83105225,67.99510061)
\lineto(820.95105225,67.99510061)
\lineto(821.41605225,67.99510061)
\moveto(823.74105225,66.45010061)
\curveto(823.42105534,66.55009436)(823.05605571,66.6100943)(822.64605225,66.63010061)
\curveto(822.23605653,66.65009426)(821.82605694,66.66009425)(821.41605225,66.66010061)
\curveto(820.98605778,66.66009425)(820.5660582,66.65009426)(820.15605225,66.63010061)
\curveto(819.74605902,66.6100943)(819.3610594,66.56509434)(819.00105225,66.49510061)
\curveto(818.64106012,66.42509448)(818.32106044,66.31509459)(818.04105225,66.16510061)
\curveto(817.75106101,66.02509488)(817.51606125,65.83009508)(817.33605225,65.58010061)
\curveto(817.22606154,65.42009549)(817.14606162,65.24009567)(817.09605225,65.04010061)
\curveto(817.03606173,64.84009607)(817.00606176,64.59509631)(817.00605225,64.30510061)
\curveto(817.02606174,64.28509662)(817.03606173,64.25009666)(817.03605225,64.20010061)
\curveto(817.02606174,64.15009676)(817.02606174,64.1100968)(817.03605225,64.08010061)
\curveto(817.05606171,64.00009691)(817.07606169,63.92509698)(817.09605225,63.85510061)
\curveto(817.10606166,63.79509711)(817.12606164,63.73009718)(817.15605225,63.66010061)
\curveto(817.27606149,63.39009752)(817.44606132,63.17009774)(817.66605225,63.00010061)
\curveto(817.87606089,62.84009807)(818.12106064,62.7050982)(818.40105225,62.59510061)
\curveto(818.51106025,62.54509836)(818.63106013,62.5050984)(818.76105225,62.47510061)
\curveto(818.88105988,62.45509845)(819.00605976,62.43009848)(819.13605225,62.40010061)
\curveto(819.18605958,62.38009853)(819.24105952,62.37009854)(819.30105225,62.37010061)
\curveto(819.35105941,62.37009854)(819.40105936,62.36509854)(819.45105225,62.35510061)
\curveto(819.54105922,62.34509856)(819.63605913,62.33509857)(819.73605225,62.32510061)
\curveto(819.82605894,62.31509859)(819.92105884,62.3050986)(820.02105225,62.29510061)
\curveto(820.10105866,62.29509861)(820.18605858,62.29009862)(820.27605225,62.28010061)
\lineto(820.51605225,62.28010061)
\lineto(820.69605225,62.28010061)
\curveto(820.72605804,62.27009864)(820.761058,62.26509864)(820.80105225,62.26510061)
\lineto(820.93605225,62.26510061)
\lineto(821.38605225,62.26510061)
\curveto(821.4660573,62.26509864)(821.55105721,62.26009865)(821.64105225,62.25010061)
\curveto(821.72105704,62.25009866)(821.79605697,62.26009865)(821.86605225,62.28010061)
\lineto(822.13605225,62.28010061)
\curveto(822.15605661,62.28009863)(822.18605658,62.27509863)(822.22605225,62.26510061)
\curveto(822.25605651,62.26509864)(822.28105648,62.27009864)(822.30105225,62.28010061)
\curveto(822.40105636,62.29009862)(822.50105626,62.29509861)(822.60105225,62.29510061)
\curveto(822.69105607,62.3050986)(822.79105597,62.31509859)(822.90105225,62.32510061)
\curveto(823.02105574,62.35509855)(823.14605562,62.37009854)(823.27605225,62.37010061)
\curveto(823.39605537,62.38009853)(823.51105525,62.4050985)(823.62105225,62.44510061)
\curveto(823.92105484,62.52509838)(824.18605458,62.6100983)(824.41605225,62.70010061)
\curveto(824.64605412,62.80009811)(824.8610539,62.94509796)(825.06105225,63.13510061)
\curveto(825.2610535,63.34509756)(825.41105335,63.6100973)(825.51105225,63.93010061)
\curveto(825.53105323,63.97009694)(825.54105322,64.0050969)(825.54105225,64.03510061)
\curveto(825.53105323,64.07509683)(825.53605323,64.12009679)(825.55605225,64.17010061)
\curveto(825.5660532,64.2100967)(825.57605319,64.28009663)(825.58605225,64.38010061)
\curveto(825.59605317,64.49009642)(825.59105317,64.57509633)(825.57105225,64.63510061)
\curveto(825.55105321,64.7050962)(825.54105322,64.77509613)(825.54105225,64.84510061)
\curveto(825.53105323,64.91509599)(825.51605325,64.98009593)(825.49605225,65.04010061)
\curveto(825.43605333,65.24009567)(825.35105341,65.42009549)(825.24105225,65.58010061)
\curveto(825.22105354,65.6100953)(825.20105356,65.63509527)(825.18105225,65.65510061)
\lineto(825.12105225,65.71510061)
\curveto(825.10105366,65.75509515)(825.0610537,65.8050951)(825.00105225,65.86510061)
\curveto(824.8610539,65.96509494)(824.73105403,66.05009486)(824.61105225,66.12010061)
\curveto(824.49105427,66.19009472)(824.34605442,66.26009465)(824.17605225,66.33010061)
\curveto(824.10605466,66.36009455)(824.03605473,66.38009453)(823.96605225,66.39010061)
\curveto(823.89605487,66.4100945)(823.82105494,66.43009448)(823.74105225,66.45010061)
}
}
{
\newrgbcolor{curcolor}{0 0 0}
\pscustom[linestyle=none,fillstyle=solid,fillcolor=curcolor]
{
\newpath
\moveto(823.00605225,76.22470998)
\curveto(823.05605571,76.29470234)(823.12605564,76.3347023)(823.21605225,76.34470998)
\curveto(823.30605546,76.36470227)(823.41105535,76.37470226)(823.53105225,76.37470998)
\curveto(823.58105518,76.37470226)(823.63105513,76.36970226)(823.68105225,76.35970998)
\curveto(823.73105503,76.35970227)(823.77605499,76.34970228)(823.81605225,76.32970998)
\curveto(823.90605486,76.29970233)(823.9660548,76.23970239)(823.99605225,76.14970998)
\curveto(824.01605475,76.06970256)(824.02605474,75.97470266)(824.02605225,75.86470998)
\lineto(824.02605225,75.54970998)
\curveto(824.01605475,75.43970319)(824.02605474,75.3347033)(824.05605225,75.23470998)
\curveto(824.08605468,75.09470354)(824.1660546,75.00470363)(824.29605225,74.96470998)
\curveto(824.3660544,74.94470369)(824.45105431,74.9347037)(824.55105225,74.93470998)
\lineto(824.82105225,74.93470998)
\lineto(825.76605225,74.93470998)
\lineto(826.09605225,74.93470998)
\curveto(826.20605256,74.9347037)(826.29105247,74.91470372)(826.35105225,74.87470998)
\curveto(826.41105235,74.8347038)(826.45105231,74.78470385)(826.47105225,74.72470998)
\curveto(826.48105228,74.67470396)(826.49605227,74.60970402)(826.51605225,74.52970998)
\lineto(826.51605225,74.33470998)
\curveto(826.51605225,74.21470442)(826.51105225,74.10970452)(826.50105225,74.01970998)
\curveto(826.48105228,73.9297047)(826.43105233,73.85970477)(826.35105225,73.80970998)
\curveto(826.30105246,73.77970485)(826.23105253,73.76470487)(826.14105225,73.76470998)
\lineto(825.84105225,73.76470998)
\lineto(824.80605225,73.76470998)
\curveto(824.64605412,73.76470487)(824.50105426,73.75470488)(824.37105225,73.73470998)
\curveto(824.23105453,73.72470491)(824.13605463,73.66970496)(824.08605225,73.56970998)
\curveto(824.0660547,73.51970511)(824.05105471,73.44970518)(824.04105225,73.35970998)
\curveto(824.03105473,73.27970535)(824.02605474,73.18970544)(824.02605225,73.08970998)
\lineto(824.02605225,72.80470998)
\lineto(824.02605225,72.56470998)
\lineto(824.02605225,70.29970998)
\curveto(824.02605474,70.20970842)(824.03105473,70.10470853)(824.04105225,69.98470998)
\lineto(824.04105225,69.65470998)
\curveto(824.04105472,69.54470909)(824.03105473,69.44470919)(824.01105225,69.35470998)
\curveto(823.99105477,69.26470937)(823.95605481,69.20470943)(823.90605225,69.17470998)
\curveto(823.83605493,69.12470951)(823.74105502,69.09970953)(823.62105225,69.09970998)
\lineto(823.27605225,69.09970998)
\lineto(823.00605225,69.09970998)
\curveto(822.83605593,69.13970949)(822.69605607,69.19470944)(822.58605225,69.26470998)
\curveto(822.47605629,69.3347093)(822.3610564,69.41470922)(822.24105225,69.50470998)
\lineto(821.70105225,69.86470998)
\curveto(821.07105769,70.30470833)(820.45105831,70.73970789)(819.84105225,71.16970998)
\lineto(817.98105225,72.48970998)
\curveto(817.75106101,72.64970598)(817.53106123,72.80470583)(817.32105225,72.95470998)
\curveto(817.10106166,73.10470553)(816.87606189,73.25970537)(816.64605225,73.41970998)
\curveto(816.57606219,73.46970516)(816.51106225,73.51970511)(816.45105225,73.56970998)
\curveto(816.38106238,73.61970501)(816.30606246,73.66970496)(816.22605225,73.71970998)
\lineto(816.13605225,73.77970998)
\curveto(816.09606267,73.80970482)(816.0660627,73.83970479)(816.04605225,73.86970998)
\curveto(816.01606275,73.90970472)(815.99606277,73.94970468)(815.98605225,73.98970998)
\curveto(815.9660628,74.0297046)(815.94606282,74.07470456)(815.92605225,74.12470998)
\curveto(815.92606284,74.14470449)(815.93106283,74.16470447)(815.94105225,74.18470998)
\curveto(815.94106282,74.21470442)(815.93106283,74.23970439)(815.91105225,74.25970998)
\curveto(815.91106285,74.38970424)(815.91606285,74.50970412)(815.92605225,74.61970998)
\curveto(815.93606283,74.7297039)(815.98106278,74.80970382)(816.06105225,74.85970998)
\curveto(816.11106265,74.89970373)(816.18106258,74.91970371)(816.27105225,74.91970998)
\curveto(816.3610624,74.9297037)(816.45606231,74.9347037)(816.55605225,74.93470998)
\lineto(822.01605225,74.93470998)
\curveto(822.08605668,74.9347037)(822.1610566,74.9297037)(822.24105225,74.91970998)
\curveto(822.31105645,74.91970371)(822.38105638,74.92470371)(822.45105225,74.93470998)
\lineto(822.55605225,74.93470998)
\curveto(822.60605616,74.95470368)(822.6610561,74.96970366)(822.72105225,74.97970998)
\curveto(822.77105599,74.98970364)(822.81105595,75.01470362)(822.84105225,75.05470998)
\curveto(822.89105587,75.12470351)(822.92105584,75.20970342)(822.93105225,75.30970998)
\lineto(822.93105225,75.63970998)
\curveto(822.93105583,75.74970288)(822.93605583,75.85470278)(822.94605225,75.95470998)
\curveto(822.94605582,76.06470257)(822.9660558,76.15470248)(823.00605225,76.22470998)
\moveto(822.81105225,73.65970998)
\curveto(822.70105606,73.73970489)(822.53105623,73.77470486)(822.30105225,73.76470998)
\lineto(821.68605225,73.76470998)
\lineto(819.21105225,73.76470998)
\lineto(818.89605225,73.76470998)
\curveto(818.77605999,73.77470486)(818.67606009,73.76970486)(818.59605225,73.74970998)
\lineto(818.44605225,73.74970998)
\curveto(818.35606041,73.74970488)(818.27106049,73.7347049)(818.19105225,73.70470998)
\curveto(818.17106059,73.69470494)(818.1610606,73.68470495)(818.16105225,73.67470998)
\lineto(818.11605225,73.62970998)
\curveto(818.10606066,73.60970502)(818.10106066,73.57970505)(818.10105225,73.53970998)
\curveto(818.12106064,73.51970511)(818.13606063,73.49970513)(818.14605225,73.47970998)
\curveto(818.14606062,73.46970516)(818.15106061,73.45470518)(818.16105225,73.43470998)
\curveto(818.21106055,73.37470526)(818.28106048,73.31470532)(818.37105225,73.25470998)
\curveto(818.4610603,73.19470544)(818.54106022,73.13970549)(818.61105225,73.08970998)
\curveto(818.75106001,72.98970564)(818.89605987,72.89470574)(819.04605225,72.80470998)
\curveto(819.18605958,72.71470592)(819.32605944,72.61970601)(819.46605225,72.51970998)
\lineto(820.24605225,71.97970998)
\curveto(820.50605826,71.80970682)(820.766058,71.634707)(821.02605225,71.45470998)
\curveto(821.13605763,71.37470726)(821.24105752,71.29970733)(821.34105225,71.22970998)
\lineto(821.64105225,71.01970998)
\curveto(821.72105704,70.96970766)(821.79605697,70.91970771)(821.86605225,70.86970998)
\curveto(821.93605683,70.8297078)(822.01105675,70.78470785)(822.09105225,70.73470998)
\curveto(822.15105661,70.68470795)(822.21605655,70.634708)(822.28605225,70.58470998)
\curveto(822.34605642,70.54470809)(822.41605635,70.50470813)(822.49605225,70.46470998)
\curveto(822.55605621,70.42470821)(822.62605614,70.39970823)(822.70605225,70.38970998)
\curveto(822.77605599,70.37970825)(822.83105593,70.41470822)(822.87105225,70.49470998)
\curveto(822.92105584,70.56470807)(822.94605582,70.67470796)(822.94605225,70.82470998)
\curveto(822.93605583,70.98470765)(822.93105583,71.11970751)(822.93105225,71.22970998)
\lineto(822.93105225,72.90970998)
\lineto(822.93105225,73.34470998)
\curveto(822.93105583,73.49470514)(822.89105587,73.59970503)(822.81105225,73.65970998)
}
}
{
\newrgbcolor{curcolor}{0 0 0}
\pscustom[linestyle=none,fillstyle=solid,fillcolor=curcolor]
{
\newpath
\moveto(824.86605225,78.63431936)
\lineto(824.86605225,79.26431936)
\lineto(824.86605225,79.45931936)
\curveto(824.8660539,79.52931683)(824.87605389,79.58931677)(824.89605225,79.63931936)
\curveto(824.93605383,79.70931665)(824.97605379,79.7593166)(825.01605225,79.78931936)
\curveto(825.0660537,79.82931653)(825.13105363,79.84931651)(825.21105225,79.84931936)
\curveto(825.29105347,79.8593165)(825.37605339,79.86431649)(825.46605225,79.86431936)
\lineto(826.18605225,79.86431936)
\curveto(826.6660521,79.86431649)(827.07605169,79.80431655)(827.41605225,79.68431936)
\curveto(827.75605101,79.56431679)(828.03105073,79.36931699)(828.24105225,79.09931936)
\curveto(828.29105047,79.02931733)(828.33605043,78.9593174)(828.37605225,78.88931936)
\curveto(828.42605034,78.82931753)(828.47105029,78.7543176)(828.51105225,78.66431936)
\curveto(828.52105024,78.64431771)(828.53105023,78.61431774)(828.54105225,78.57431936)
\curveto(828.5610502,78.53431782)(828.5660502,78.48931787)(828.55605225,78.43931936)
\curveto(828.52605024,78.34931801)(828.45105031,78.29431806)(828.33105225,78.27431936)
\curveto(828.22105054,78.2543181)(828.12605064,78.26931809)(828.04605225,78.31931936)
\curveto(827.97605079,78.34931801)(827.91105085,78.39431796)(827.85105225,78.45431936)
\curveto(827.80105096,78.52431783)(827.75105101,78.58931777)(827.70105225,78.64931936)
\curveto(827.65105111,78.71931764)(827.57605119,78.77931758)(827.47605225,78.82931936)
\curveto(827.38605138,78.88931747)(827.29605147,78.93931742)(827.20605225,78.97931936)
\curveto(827.17605159,78.99931736)(827.11605165,79.02431733)(827.02605225,79.05431936)
\curveto(826.94605182,79.08431727)(826.87605189,79.08931727)(826.81605225,79.06931936)
\curveto(826.67605209,79.03931732)(826.58605218,78.97931738)(826.54605225,78.88931936)
\curveto(826.51605225,78.80931755)(826.50105226,78.71931764)(826.50105225,78.61931936)
\curveto(826.50105226,78.51931784)(826.47605229,78.43431792)(826.42605225,78.36431936)
\curveto(826.35605241,78.27431808)(826.21605255,78.22931813)(826.00605225,78.22931936)
\lineto(825.45105225,78.22931936)
\lineto(825.22605225,78.22931936)
\curveto(825.14605362,78.23931812)(825.08105368,78.2593181)(825.03105225,78.28931936)
\curveto(824.95105381,78.34931801)(824.90605386,78.41931794)(824.89605225,78.49931936)
\curveto(824.88605388,78.51931784)(824.88105388,78.53931782)(824.88105225,78.55931936)
\curveto(824.88105388,78.58931777)(824.87605389,78.61431774)(824.86605225,78.63431936)
}
}
{
\newrgbcolor{curcolor}{0 0 0}
\pscustom[linestyle=none,fillstyle=solid,fillcolor=curcolor]
{
}
}
{
\newrgbcolor{curcolor}{0 0 0}
\pscustom[linestyle=none,fillstyle=solid,fillcolor=curcolor]
{
\newpath
\moveto(815.89605225,89.26463186)
\curveto(815.88606288,89.95462722)(816.00606276,90.55462662)(816.25605225,91.06463186)
\curveto(816.50606226,91.58462559)(816.84106192,91.9796252)(817.26105225,92.24963186)
\curveto(817.34106142,92.29962488)(817.43106133,92.34462483)(817.53105225,92.38463186)
\curveto(817.62106114,92.42462475)(817.71606105,92.46962471)(817.81605225,92.51963186)
\curveto(817.91606085,92.55962462)(818.01606075,92.58962459)(818.11605225,92.60963186)
\curveto(818.21606055,92.62962455)(818.32106044,92.64962453)(818.43105225,92.66963186)
\curveto(818.48106028,92.68962449)(818.52606024,92.69462448)(818.56605225,92.68463186)
\curveto(818.60606016,92.6746245)(818.65106011,92.6796245)(818.70105225,92.69963186)
\curveto(818.75106001,92.70962447)(818.83605993,92.71462446)(818.95605225,92.71463186)
\curveto(819.0660597,92.71462446)(819.15105961,92.70962447)(819.21105225,92.69963186)
\curveto(819.27105949,92.6796245)(819.33105943,92.66962451)(819.39105225,92.66963186)
\curveto(819.45105931,92.6796245)(819.51105925,92.6746245)(819.57105225,92.65463186)
\curveto(819.71105905,92.61462456)(819.84605892,92.5796246)(819.97605225,92.54963186)
\curveto(820.10605866,92.51962466)(820.23105853,92.4796247)(820.35105225,92.42963186)
\curveto(820.49105827,92.36962481)(820.61605815,92.29962488)(820.72605225,92.21963186)
\curveto(820.83605793,92.14962503)(820.94605782,92.0746251)(821.05605225,91.99463186)
\lineto(821.11605225,91.93463186)
\curveto(821.13605763,91.92462525)(821.15605761,91.90962527)(821.17605225,91.88963186)
\curveto(821.33605743,91.76962541)(821.48105728,91.63462554)(821.61105225,91.48463186)
\curveto(821.74105702,91.33462584)(821.8660569,91.174626)(821.98605225,91.00463186)
\curveto(822.20605656,90.69462648)(822.41105635,90.39962678)(822.60105225,90.11963186)
\curveto(822.74105602,89.88962729)(822.87605589,89.65962752)(823.00605225,89.42963186)
\curveto(823.13605563,89.20962797)(823.27105549,88.98962819)(823.41105225,88.76963186)
\curveto(823.58105518,88.51962866)(823.761055,88.2796289)(823.95105225,88.04963186)
\curveto(824.14105462,87.82962935)(824.3660544,87.63962954)(824.62605225,87.47963186)
\curveto(824.68605408,87.43962974)(824.74605402,87.40462977)(824.80605225,87.37463186)
\curveto(824.85605391,87.34462983)(824.92105384,87.31462986)(825.00105225,87.28463186)
\curveto(825.07105369,87.26462991)(825.13105363,87.25962992)(825.18105225,87.26963186)
\curveto(825.25105351,87.28962989)(825.30605346,87.32462985)(825.34605225,87.37463186)
\curveto(825.37605339,87.42462975)(825.39605337,87.48462969)(825.40605225,87.55463186)
\lineto(825.40605225,87.79463186)
\lineto(825.40605225,88.54463186)
\lineto(825.40605225,91.34963186)
\lineto(825.40605225,92.00963186)
\curveto(825.40605336,92.09962508)(825.41105335,92.18462499)(825.42105225,92.26463186)
\curveto(825.42105334,92.34462483)(825.44105332,92.40962477)(825.48105225,92.45963186)
\curveto(825.52105324,92.50962467)(825.59605317,92.54962463)(825.70605225,92.57963186)
\curveto(825.80605296,92.61962456)(825.90605286,92.62962455)(826.00605225,92.60963186)
\lineto(826.14105225,92.60963186)
\curveto(826.21105255,92.58962459)(826.27105249,92.56962461)(826.32105225,92.54963186)
\curveto(826.37105239,92.52962465)(826.41105235,92.49462468)(826.44105225,92.44463186)
\curveto(826.48105228,92.39462478)(826.50105226,92.32462485)(826.50105225,92.23463186)
\lineto(826.50105225,91.96463186)
\lineto(826.50105225,91.06463186)
\lineto(826.50105225,87.55463186)
\lineto(826.50105225,86.48963186)
\curveto(826.50105226,86.40963077)(826.50605226,86.31963086)(826.51605225,86.21963186)
\curveto(826.51605225,86.11963106)(826.50605226,86.03463114)(826.48605225,85.96463186)
\curveto(826.41605235,85.75463142)(826.23605253,85.68963149)(825.94605225,85.76963186)
\curveto(825.90605286,85.7796314)(825.87105289,85.7796314)(825.84105225,85.76963186)
\curveto(825.80105296,85.76963141)(825.75605301,85.7796314)(825.70605225,85.79963186)
\curveto(825.62605314,85.81963136)(825.54105322,85.83963134)(825.45105225,85.85963186)
\curveto(825.3610534,85.8796313)(825.27605349,85.90463127)(825.19605225,85.93463186)
\curveto(824.70605406,86.09463108)(824.29105447,86.29463088)(823.95105225,86.53463186)
\curveto(823.70105506,86.71463046)(823.47605529,86.91963026)(823.27605225,87.14963186)
\curveto(823.0660557,87.3796298)(822.87105589,87.61962956)(822.69105225,87.86963186)
\curveto(822.51105625,88.12962905)(822.34105642,88.39462878)(822.18105225,88.66463186)
\curveto(822.01105675,88.94462823)(821.83605693,89.21462796)(821.65605225,89.47463186)
\curveto(821.57605719,89.58462759)(821.50105726,89.68962749)(821.43105225,89.78963186)
\curveto(821.3610574,89.89962728)(821.28605748,90.00962717)(821.20605225,90.11963186)
\curveto(821.17605759,90.15962702)(821.14605762,90.19462698)(821.11605225,90.22463186)
\curveto(821.07605769,90.26462691)(821.04605772,90.30462687)(821.02605225,90.34463186)
\curveto(820.91605785,90.48462669)(820.79105797,90.60962657)(820.65105225,90.71963186)
\curveto(820.62105814,90.73962644)(820.59605817,90.76462641)(820.57605225,90.79463186)
\curveto(820.54605822,90.82462635)(820.51605825,90.84962633)(820.48605225,90.86963186)
\curveto(820.38605838,90.94962623)(820.28605848,91.01462616)(820.18605225,91.06463186)
\curveto(820.08605868,91.12462605)(819.97605879,91.179626)(819.85605225,91.22963186)
\curveto(819.78605898,91.25962592)(819.71105905,91.2796259)(819.63105225,91.28963186)
\lineto(819.39105225,91.34963186)
\lineto(819.30105225,91.34963186)
\curveto(819.27105949,91.35962582)(819.24105952,91.36462581)(819.21105225,91.36463186)
\curveto(819.14105962,91.38462579)(819.04605972,91.38962579)(818.92605225,91.37963186)
\curveto(818.79605997,91.3796258)(818.69606007,91.36962581)(818.62605225,91.34963186)
\curveto(818.54606022,91.32962585)(818.47106029,91.30962587)(818.40105225,91.28963186)
\curveto(818.32106044,91.2796259)(818.24106052,91.25962592)(818.16105225,91.22963186)
\curveto(817.92106084,91.11962606)(817.72106104,90.96962621)(817.56105225,90.77963186)
\curveto(817.39106137,90.59962658)(817.25106151,90.3796268)(817.14105225,90.11963186)
\curveto(817.12106164,90.04962713)(817.10606166,89.9796272)(817.09605225,89.90963186)
\curveto(817.07606169,89.83962734)(817.05606171,89.76462741)(817.03605225,89.68463186)
\curveto(817.01606175,89.60462757)(817.00606176,89.49462768)(817.00605225,89.35463186)
\curveto(817.00606176,89.22462795)(817.01606175,89.11962806)(817.03605225,89.03963186)
\curveto(817.04606172,88.9796282)(817.05106171,88.92462825)(817.05105225,88.87463186)
\curveto(817.05106171,88.82462835)(817.0610617,88.7746284)(817.08105225,88.72463186)
\curveto(817.12106164,88.62462855)(817.1610616,88.52962865)(817.20105225,88.43963186)
\curveto(817.24106152,88.35962882)(817.28606148,88.2796289)(817.33605225,88.19963186)
\curveto(817.35606141,88.16962901)(817.38106138,88.13962904)(817.41105225,88.10963186)
\curveto(817.44106132,88.08962909)(817.4660613,88.06462911)(817.48605225,88.03463186)
\lineto(817.56105225,87.95963186)
\curveto(817.58106118,87.92962925)(817.60106116,87.90462927)(817.62105225,87.88463186)
\lineto(817.83105225,87.73463186)
\curveto(817.89106087,87.69462948)(817.95606081,87.64962953)(818.02605225,87.59963186)
\curveto(818.11606065,87.53962964)(818.22106054,87.48962969)(818.34105225,87.44963186)
\curveto(818.45106031,87.41962976)(818.5610602,87.38462979)(818.67105225,87.34463186)
\curveto(818.78105998,87.30462987)(818.92605984,87.2796299)(819.10605225,87.26963186)
\curveto(819.27605949,87.25962992)(819.40105936,87.22962995)(819.48105225,87.17963186)
\curveto(819.5610592,87.12963005)(819.60605916,87.05463012)(819.61605225,86.95463186)
\curveto(819.62605914,86.85463032)(819.63105913,86.74463043)(819.63105225,86.62463186)
\curveto(819.63105913,86.58463059)(819.63605913,86.54463063)(819.64605225,86.50463186)
\curveto(819.64605912,86.46463071)(819.64105912,86.42963075)(819.63105225,86.39963186)
\curveto(819.61105915,86.34963083)(819.60105916,86.29963088)(819.60105225,86.24963186)
\curveto(819.60105916,86.20963097)(819.59105917,86.16963101)(819.57105225,86.12963186)
\curveto(819.51105925,86.03963114)(819.37605939,85.99463118)(819.16605225,85.99463186)
\lineto(819.04605225,85.99463186)
\curveto(818.98605978,86.00463117)(818.92605984,86.00963117)(818.86605225,86.00963186)
\curveto(818.79605997,86.01963116)(818.73106003,86.02963115)(818.67105225,86.03963186)
\curveto(818.5610602,86.05963112)(818.4610603,86.0796311)(818.37105225,86.09963186)
\curveto(818.27106049,86.11963106)(818.17606059,86.14963103)(818.08605225,86.18963186)
\curveto(818.01606075,86.20963097)(817.95606081,86.22963095)(817.90605225,86.24963186)
\lineto(817.72605225,86.30963186)
\curveto(817.4660613,86.42963075)(817.22106154,86.58463059)(816.99105225,86.77463186)
\curveto(816.761062,86.9746302)(816.57606219,87.18962999)(816.43605225,87.41963186)
\curveto(816.35606241,87.52962965)(816.29106247,87.64462953)(816.24105225,87.76463186)
\lineto(816.09105225,88.15463186)
\curveto(816.04106272,88.26462891)(816.01106275,88.3796288)(816.00105225,88.49963186)
\curveto(815.98106278,88.61962856)(815.95606281,88.74462843)(815.92605225,88.87463186)
\curveto(815.92606284,88.94462823)(815.92606284,89.00962817)(815.92605225,89.06963186)
\curveto(815.91606285,89.12962805)(815.90606286,89.19462798)(815.89605225,89.26463186)
}
}
{
\newrgbcolor{curcolor}{0 0 0}
\pscustom[linestyle=none,fillstyle=solid,fillcolor=curcolor]
{
\newpath
\moveto(821.41605225,101.36424123)
\lineto(821.67105225,101.36424123)
\curveto(821.75105701,101.37423353)(821.82605694,101.36923353)(821.89605225,101.34924123)
\lineto(822.13605225,101.34924123)
\lineto(822.30105225,101.34924123)
\curveto(822.40105636,101.32923357)(822.50605626,101.31923358)(822.61605225,101.31924123)
\curveto(822.71605605,101.31923358)(822.81605595,101.30923359)(822.91605225,101.28924123)
\lineto(823.06605225,101.28924123)
\curveto(823.20605556,101.25923364)(823.34605542,101.23923366)(823.48605225,101.22924123)
\curveto(823.61605515,101.21923368)(823.74605502,101.19423371)(823.87605225,101.15424123)
\curveto(823.95605481,101.13423377)(824.04105472,101.11423379)(824.13105225,101.09424123)
\lineto(824.37105225,101.03424123)
\lineto(824.67105225,100.91424123)
\curveto(824.761054,100.88423402)(824.85105391,100.84923405)(824.94105225,100.80924123)
\curveto(825.1610536,100.70923419)(825.37605339,100.57423433)(825.58605225,100.40424123)
\curveto(825.79605297,100.24423466)(825.9660528,100.06923483)(826.09605225,99.87924123)
\curveto(826.13605263,99.82923507)(826.17605259,99.76923513)(826.21605225,99.69924123)
\curveto(826.24605252,99.63923526)(826.28105248,99.57923532)(826.32105225,99.51924123)
\curveto(826.37105239,99.43923546)(826.41105235,99.34423556)(826.44105225,99.23424123)
\curveto(826.47105229,99.12423578)(826.50105226,99.01923588)(826.53105225,98.91924123)
\curveto(826.57105219,98.80923609)(826.59605217,98.6992362)(826.60605225,98.58924123)
\curveto(826.61605215,98.47923642)(826.63105213,98.36423654)(826.65105225,98.24424123)
\curveto(826.6610521,98.2042367)(826.6610521,98.15923674)(826.65105225,98.10924123)
\curveto(826.65105211,98.06923683)(826.65605211,98.02923687)(826.66605225,97.98924123)
\curveto(826.67605209,97.94923695)(826.68105208,97.89423701)(826.68105225,97.82424123)
\curveto(826.68105208,97.75423715)(826.67605209,97.7042372)(826.66605225,97.67424123)
\curveto(826.64605212,97.62423728)(826.64105212,97.57923732)(826.65105225,97.53924123)
\curveto(826.6610521,97.4992374)(826.6610521,97.46423744)(826.65105225,97.43424123)
\lineto(826.65105225,97.34424123)
\curveto(826.63105213,97.28423762)(826.61605215,97.21923768)(826.60605225,97.14924123)
\curveto(826.60605216,97.08923781)(826.60105216,97.02423788)(826.59105225,96.95424123)
\curveto(826.54105222,96.78423812)(826.49105227,96.62423828)(826.44105225,96.47424123)
\curveto(826.39105237,96.32423858)(826.32605244,96.17923872)(826.24605225,96.03924123)
\curveto(826.20605256,95.98923891)(826.17605259,95.93423897)(826.15605225,95.87424123)
\curveto(826.12605264,95.82423908)(826.09105267,95.77423913)(826.05105225,95.72424123)
\curveto(825.87105289,95.48423942)(825.65105311,95.28423962)(825.39105225,95.12424123)
\curveto(825.13105363,94.96423994)(824.84605392,94.82424008)(824.53605225,94.70424123)
\curveto(824.39605437,94.64424026)(824.25605451,94.5992403)(824.11605225,94.56924123)
\curveto(823.9660548,94.53924036)(823.81105495,94.5042404)(823.65105225,94.46424123)
\curveto(823.54105522,94.44424046)(823.43105533,94.42924047)(823.32105225,94.41924123)
\curveto(823.21105555,94.40924049)(823.10105566,94.39424051)(822.99105225,94.37424123)
\curveto(822.95105581,94.36424054)(822.91105585,94.35924054)(822.87105225,94.35924123)
\curveto(822.83105593,94.36924053)(822.79105597,94.36924053)(822.75105225,94.35924123)
\curveto(822.70105606,94.34924055)(822.65105611,94.34424056)(822.60105225,94.34424123)
\lineto(822.43605225,94.34424123)
\curveto(822.38605638,94.32424058)(822.33605643,94.31924058)(822.28605225,94.32924123)
\curveto(822.22605654,94.33924056)(822.17105659,94.33924056)(822.12105225,94.32924123)
\curveto(822.08105668,94.31924058)(822.03605673,94.31924058)(821.98605225,94.32924123)
\curveto(821.93605683,94.33924056)(821.88605688,94.33424057)(821.83605225,94.31424123)
\curveto(821.766057,94.29424061)(821.69105707,94.28924061)(821.61105225,94.29924123)
\curveto(821.52105724,94.30924059)(821.43605733,94.31424059)(821.35605225,94.31424123)
\curveto(821.2660575,94.31424059)(821.1660576,94.30924059)(821.05605225,94.29924123)
\curveto(820.93605783,94.28924061)(820.83605793,94.29424061)(820.75605225,94.31424123)
\lineto(820.47105225,94.31424123)
\lineto(819.84105225,94.35924123)
\curveto(819.74105902,94.36924053)(819.64605912,94.37924052)(819.55605225,94.38924123)
\lineto(819.25605225,94.41924123)
\curveto(819.20605956,94.43924046)(819.15605961,94.44424046)(819.10605225,94.43424123)
\curveto(819.04605972,94.43424047)(818.99105977,94.44424046)(818.94105225,94.46424123)
\curveto(818.77105999,94.51424039)(818.60606016,94.55424035)(818.44605225,94.58424123)
\curveto(818.27606049,94.61424029)(818.11606065,94.66424024)(817.96605225,94.73424123)
\curveto(817.50606126,94.92423998)(817.13106163,95.14423976)(816.84105225,95.39424123)
\curveto(816.55106221,95.65423925)(816.30606246,96.01423889)(816.10605225,96.47424123)
\curveto(816.05606271,96.6042383)(816.02106274,96.73423817)(816.00105225,96.86424123)
\curveto(815.98106278,97.0042379)(815.95606281,97.14423776)(815.92605225,97.28424123)
\curveto(815.91606285,97.35423755)(815.91106285,97.41923748)(815.91105225,97.47924123)
\curveto(815.91106285,97.53923736)(815.90606286,97.6042373)(815.89605225,97.67424123)
\curveto(815.87606289,98.5042364)(816.02606274,99.17423573)(816.34605225,99.68424123)
\curveto(816.65606211,100.19423471)(817.09606167,100.57423433)(817.66605225,100.82424123)
\curveto(817.78606098,100.87423403)(817.91106085,100.91923398)(818.04105225,100.95924123)
\curveto(818.17106059,100.9992339)(818.30606046,101.04423386)(818.44605225,101.09424123)
\curveto(818.52606024,101.11423379)(818.61106015,101.12923377)(818.70105225,101.13924123)
\lineto(818.94105225,101.19924123)
\curveto(819.05105971,101.22923367)(819.1610596,101.24423366)(819.27105225,101.24424123)
\curveto(819.38105938,101.25423365)(819.49105927,101.26923363)(819.60105225,101.28924123)
\curveto(819.65105911,101.30923359)(819.69605907,101.31423359)(819.73605225,101.30424123)
\curveto(819.77605899,101.3042336)(819.81605895,101.30923359)(819.85605225,101.31924123)
\curveto(819.90605886,101.32923357)(819.9610588,101.32923357)(820.02105225,101.31924123)
\curveto(820.07105869,101.31923358)(820.12105864,101.32423358)(820.17105225,101.33424123)
\lineto(820.30605225,101.33424123)
\curveto(820.3660584,101.35423355)(820.43605833,101.35423355)(820.51605225,101.33424123)
\curveto(820.58605818,101.32423358)(820.65105811,101.32923357)(820.71105225,101.34924123)
\curveto(820.74105802,101.35923354)(820.78105798,101.36423354)(820.83105225,101.36424123)
\lineto(820.95105225,101.36424123)
\lineto(821.41605225,101.36424123)
\moveto(823.74105225,99.81924123)
\curveto(823.42105534,99.91923498)(823.05605571,99.97923492)(822.64605225,99.99924123)
\curveto(822.23605653,100.01923488)(821.82605694,100.02923487)(821.41605225,100.02924123)
\curveto(820.98605778,100.02923487)(820.5660582,100.01923488)(820.15605225,99.99924123)
\curveto(819.74605902,99.97923492)(819.3610594,99.93423497)(819.00105225,99.86424123)
\curveto(818.64106012,99.79423511)(818.32106044,99.68423522)(818.04105225,99.53424123)
\curveto(817.75106101,99.39423551)(817.51606125,99.1992357)(817.33605225,98.94924123)
\curveto(817.22606154,98.78923611)(817.14606162,98.60923629)(817.09605225,98.40924123)
\curveto(817.03606173,98.20923669)(817.00606176,97.96423694)(817.00605225,97.67424123)
\curveto(817.02606174,97.65423725)(817.03606173,97.61923728)(817.03605225,97.56924123)
\curveto(817.02606174,97.51923738)(817.02606174,97.47923742)(817.03605225,97.44924123)
\curveto(817.05606171,97.36923753)(817.07606169,97.29423761)(817.09605225,97.22424123)
\curveto(817.10606166,97.16423774)(817.12606164,97.0992378)(817.15605225,97.02924123)
\curveto(817.27606149,96.75923814)(817.44606132,96.53923836)(817.66605225,96.36924123)
\curveto(817.87606089,96.20923869)(818.12106064,96.07423883)(818.40105225,95.96424123)
\curveto(818.51106025,95.91423899)(818.63106013,95.87423903)(818.76105225,95.84424123)
\curveto(818.88105988,95.82423908)(819.00605976,95.7992391)(819.13605225,95.76924123)
\curveto(819.18605958,95.74923915)(819.24105952,95.73923916)(819.30105225,95.73924123)
\curveto(819.35105941,95.73923916)(819.40105936,95.73423917)(819.45105225,95.72424123)
\curveto(819.54105922,95.71423919)(819.63605913,95.7042392)(819.73605225,95.69424123)
\curveto(819.82605894,95.68423922)(819.92105884,95.67423923)(820.02105225,95.66424123)
\curveto(820.10105866,95.66423924)(820.18605858,95.65923924)(820.27605225,95.64924123)
\lineto(820.51605225,95.64924123)
\lineto(820.69605225,95.64924123)
\curveto(820.72605804,95.63923926)(820.761058,95.63423927)(820.80105225,95.63424123)
\lineto(820.93605225,95.63424123)
\lineto(821.38605225,95.63424123)
\curveto(821.4660573,95.63423927)(821.55105721,95.62923927)(821.64105225,95.61924123)
\curveto(821.72105704,95.61923928)(821.79605697,95.62923927)(821.86605225,95.64924123)
\lineto(822.13605225,95.64924123)
\curveto(822.15605661,95.64923925)(822.18605658,95.64423926)(822.22605225,95.63424123)
\curveto(822.25605651,95.63423927)(822.28105648,95.63923926)(822.30105225,95.64924123)
\curveto(822.40105636,95.65923924)(822.50105626,95.66423924)(822.60105225,95.66424123)
\curveto(822.69105607,95.67423923)(822.79105597,95.68423922)(822.90105225,95.69424123)
\curveto(823.02105574,95.72423918)(823.14605562,95.73923916)(823.27605225,95.73924123)
\curveto(823.39605537,95.74923915)(823.51105525,95.77423913)(823.62105225,95.81424123)
\curveto(823.92105484,95.89423901)(824.18605458,95.97923892)(824.41605225,96.06924123)
\curveto(824.64605412,96.16923873)(824.8610539,96.31423859)(825.06105225,96.50424123)
\curveto(825.2610535,96.71423819)(825.41105335,96.97923792)(825.51105225,97.29924123)
\curveto(825.53105323,97.33923756)(825.54105322,97.37423753)(825.54105225,97.40424123)
\curveto(825.53105323,97.44423746)(825.53605323,97.48923741)(825.55605225,97.53924123)
\curveto(825.5660532,97.57923732)(825.57605319,97.64923725)(825.58605225,97.74924123)
\curveto(825.59605317,97.85923704)(825.59105317,97.94423696)(825.57105225,98.00424123)
\curveto(825.55105321,98.07423683)(825.54105322,98.14423676)(825.54105225,98.21424123)
\curveto(825.53105323,98.28423662)(825.51605325,98.34923655)(825.49605225,98.40924123)
\curveto(825.43605333,98.60923629)(825.35105341,98.78923611)(825.24105225,98.94924123)
\curveto(825.22105354,98.97923592)(825.20105356,99.0042359)(825.18105225,99.02424123)
\lineto(825.12105225,99.08424123)
\curveto(825.10105366,99.12423578)(825.0610537,99.17423573)(825.00105225,99.23424123)
\curveto(824.8610539,99.33423557)(824.73105403,99.41923548)(824.61105225,99.48924123)
\curveto(824.49105427,99.55923534)(824.34605442,99.62923527)(824.17605225,99.69924123)
\curveto(824.10605466,99.72923517)(824.03605473,99.74923515)(823.96605225,99.75924123)
\curveto(823.89605487,99.77923512)(823.82105494,99.7992351)(823.74105225,99.81924123)
}
}
{
\newrgbcolor{curcolor}{0 0 0}
\pscustom[linestyle=none,fillstyle=solid,fillcolor=curcolor]
{
\newpath
\moveto(815.89605225,106.77385061)
\curveto(815.89606287,106.87384575)(815.90606286,106.96884566)(815.92605225,107.05885061)
\curveto(815.93606283,107.14884548)(815.9660628,107.21384541)(816.01605225,107.25385061)
\curveto(816.09606267,107.31384531)(816.20106256,107.34384528)(816.33105225,107.34385061)
\lineto(816.72105225,107.34385061)
\lineto(818.22105225,107.34385061)
\lineto(824.61105225,107.34385061)
\lineto(825.78105225,107.34385061)
\lineto(826.09605225,107.34385061)
\curveto(826.19605257,107.35384527)(826.27605249,107.33884529)(826.33605225,107.29885061)
\curveto(826.41605235,107.24884538)(826.4660523,107.17384545)(826.48605225,107.07385061)
\curveto(826.49605227,106.98384564)(826.50105226,106.87384575)(826.50105225,106.74385061)
\lineto(826.50105225,106.51885061)
\curveto(826.48105228,106.43884619)(826.4660523,106.36884626)(826.45605225,106.30885061)
\curveto(826.43605233,106.24884638)(826.39605237,106.19884643)(826.33605225,106.15885061)
\curveto(826.27605249,106.11884651)(826.20105256,106.09884653)(826.11105225,106.09885061)
\lineto(825.81105225,106.09885061)
\lineto(824.71605225,106.09885061)
\lineto(819.37605225,106.09885061)
\curveto(819.28605948,106.07884655)(819.21105955,106.06384656)(819.15105225,106.05385061)
\curveto(819.08105968,106.05384657)(819.02105974,106.0238466)(818.97105225,105.96385061)
\curveto(818.92105984,105.89384673)(818.89605987,105.80384682)(818.89605225,105.69385061)
\curveto(818.88605988,105.59384703)(818.88105988,105.48384714)(818.88105225,105.36385061)
\lineto(818.88105225,104.22385061)
\lineto(818.88105225,103.72885061)
\curveto(818.87105989,103.56884906)(818.81105995,103.45884917)(818.70105225,103.39885061)
\curveto(818.67106009,103.37884925)(818.64106012,103.36884926)(818.61105225,103.36885061)
\curveto(818.57106019,103.36884926)(818.52606024,103.36384926)(818.47605225,103.35385061)
\curveto(818.35606041,103.33384929)(818.24606052,103.33884929)(818.14605225,103.36885061)
\curveto(818.04606072,103.40884922)(817.97606079,103.46384916)(817.93605225,103.53385061)
\curveto(817.88606088,103.61384901)(817.8610609,103.73384889)(817.86105225,103.89385061)
\curveto(817.8610609,104.05384857)(817.84606092,104.18884844)(817.81605225,104.29885061)
\curveto(817.80606096,104.34884828)(817.80106096,104.40384822)(817.80105225,104.46385061)
\curveto(817.79106097,104.5238481)(817.77606099,104.58384804)(817.75605225,104.64385061)
\curveto(817.70606106,104.79384783)(817.65606111,104.93884769)(817.60605225,105.07885061)
\curveto(817.54606122,105.21884741)(817.47606129,105.35384727)(817.39605225,105.48385061)
\curveto(817.30606146,105.623847)(817.20106156,105.74384688)(817.08105225,105.84385061)
\curveto(816.9610618,105.94384668)(816.83106193,106.03884659)(816.69105225,106.12885061)
\curveto(816.59106217,106.18884644)(816.48106228,106.23384639)(816.36105225,106.26385061)
\curveto(816.24106252,106.30384632)(816.13606263,106.35384627)(816.04605225,106.41385061)
\curveto(815.98606278,106.46384616)(815.94606282,106.53384609)(815.92605225,106.62385061)
\curveto(815.91606285,106.64384598)(815.91106285,106.66884596)(815.91105225,106.69885061)
\curveto(815.91106285,106.7288459)(815.90606286,106.75384587)(815.89605225,106.77385061)
}
}
{
\newrgbcolor{curcolor}{0 0 0}
\pscustom[linestyle=none,fillstyle=solid,fillcolor=curcolor]
{
\newpath
\moveto(815.89605225,115.12345998)
\curveto(815.89606287,115.22345513)(815.90606286,115.31845503)(815.92605225,115.40845998)
\curveto(815.93606283,115.49845485)(815.9660628,115.56345479)(816.01605225,115.60345998)
\curveto(816.09606267,115.66345469)(816.20106256,115.69345466)(816.33105225,115.69345998)
\lineto(816.72105225,115.69345998)
\lineto(818.22105225,115.69345998)
\lineto(824.61105225,115.69345998)
\lineto(825.78105225,115.69345998)
\lineto(826.09605225,115.69345998)
\curveto(826.19605257,115.70345465)(826.27605249,115.68845466)(826.33605225,115.64845998)
\curveto(826.41605235,115.59845475)(826.4660523,115.52345483)(826.48605225,115.42345998)
\curveto(826.49605227,115.33345502)(826.50105226,115.22345513)(826.50105225,115.09345998)
\lineto(826.50105225,114.86845998)
\curveto(826.48105228,114.78845556)(826.4660523,114.71845563)(826.45605225,114.65845998)
\curveto(826.43605233,114.59845575)(826.39605237,114.5484558)(826.33605225,114.50845998)
\curveto(826.27605249,114.46845588)(826.20105256,114.4484559)(826.11105225,114.44845998)
\lineto(825.81105225,114.44845998)
\lineto(824.71605225,114.44845998)
\lineto(819.37605225,114.44845998)
\curveto(819.28605948,114.42845592)(819.21105955,114.41345594)(819.15105225,114.40345998)
\curveto(819.08105968,114.40345595)(819.02105974,114.37345598)(818.97105225,114.31345998)
\curveto(818.92105984,114.24345611)(818.89605987,114.1534562)(818.89605225,114.04345998)
\curveto(818.88605988,113.94345641)(818.88105988,113.83345652)(818.88105225,113.71345998)
\lineto(818.88105225,112.57345998)
\lineto(818.88105225,112.07845998)
\curveto(818.87105989,111.91845843)(818.81105995,111.80845854)(818.70105225,111.74845998)
\curveto(818.67106009,111.72845862)(818.64106012,111.71845863)(818.61105225,111.71845998)
\curveto(818.57106019,111.71845863)(818.52606024,111.71345864)(818.47605225,111.70345998)
\curveto(818.35606041,111.68345867)(818.24606052,111.68845866)(818.14605225,111.71845998)
\curveto(818.04606072,111.75845859)(817.97606079,111.81345854)(817.93605225,111.88345998)
\curveto(817.88606088,111.96345839)(817.8610609,112.08345827)(817.86105225,112.24345998)
\curveto(817.8610609,112.40345795)(817.84606092,112.53845781)(817.81605225,112.64845998)
\curveto(817.80606096,112.69845765)(817.80106096,112.7534576)(817.80105225,112.81345998)
\curveto(817.79106097,112.87345748)(817.77606099,112.93345742)(817.75605225,112.99345998)
\curveto(817.70606106,113.14345721)(817.65606111,113.28845706)(817.60605225,113.42845998)
\curveto(817.54606122,113.56845678)(817.47606129,113.70345665)(817.39605225,113.83345998)
\curveto(817.30606146,113.97345638)(817.20106156,114.09345626)(817.08105225,114.19345998)
\curveto(816.9610618,114.29345606)(816.83106193,114.38845596)(816.69105225,114.47845998)
\curveto(816.59106217,114.53845581)(816.48106228,114.58345577)(816.36105225,114.61345998)
\curveto(816.24106252,114.6534557)(816.13606263,114.70345565)(816.04605225,114.76345998)
\curveto(815.98606278,114.81345554)(815.94606282,114.88345547)(815.92605225,114.97345998)
\curveto(815.91606285,114.99345536)(815.91106285,115.01845533)(815.91105225,115.04845998)
\curveto(815.91106285,115.07845527)(815.90606286,115.10345525)(815.89605225,115.12345998)
}
}
{
\newrgbcolor{curcolor}{0 0 0}
\pscustom[linestyle=none,fillstyle=solid,fillcolor=curcolor]
{
\newpath
\moveto(836.73236816,42.29681936)
\curveto(836.73237886,42.36681368)(836.73237886,42.4468136)(836.73236816,42.53681936)
\curveto(836.72237887,42.62681342)(836.72237887,42.71181333)(836.73236816,42.79181936)
\curveto(836.73237886,42.88181316)(836.74237885,42.96181308)(836.76236816,43.03181936)
\curveto(836.78237881,43.11181293)(836.81237878,43.16681288)(836.85236816,43.19681936)
\curveto(836.90237869,43.22681282)(836.97737861,43.2468128)(837.07736816,43.25681936)
\curveto(837.16737842,43.27681277)(837.27237832,43.28681276)(837.39236816,43.28681936)
\curveto(837.50237809,43.29681275)(837.61737797,43.29681275)(837.73736816,43.28681936)
\lineto(838.03736816,43.28681936)
\lineto(841.05236816,43.28681936)
\lineto(843.94736816,43.28681936)
\curveto(844.27737131,43.28681276)(844.60237099,43.28181276)(844.92236816,43.27181936)
\curveto(845.23237036,43.27181277)(845.51237008,43.23181281)(845.76236816,43.15181936)
\curveto(846.11236948,43.03181301)(846.40736918,42.87681317)(846.64736816,42.68681936)
\curveto(846.87736871,42.49681355)(847.07736851,42.25681379)(847.24736816,41.96681936)
\curveto(847.29736829,41.90681414)(847.33236826,41.8418142)(847.35236816,41.77181936)
\curveto(847.37236822,41.71181433)(847.39736819,41.6418144)(847.42736816,41.56181936)
\curveto(847.47736811,41.4418146)(847.51236808,41.31181473)(847.53236816,41.17181936)
\curveto(847.56236803,41.041815)(847.592368,40.90681514)(847.62236816,40.76681936)
\curveto(847.64236795,40.71681533)(847.64736794,40.66681538)(847.63736816,40.61681936)
\curveto(847.62736796,40.56681548)(847.62736796,40.51181553)(847.63736816,40.45181936)
\curveto(847.64736794,40.43181561)(847.64736794,40.40681564)(847.63736816,40.37681936)
\curveto(847.63736795,40.3468157)(847.64236795,40.32181572)(847.65236816,40.30181936)
\curveto(847.66236793,40.26181578)(847.66736792,40.20681584)(847.66736816,40.13681936)
\curveto(847.66736792,40.06681598)(847.66236793,40.01181603)(847.65236816,39.97181936)
\curveto(847.64236795,39.92181612)(847.64236795,39.86681618)(847.65236816,39.80681936)
\curveto(847.66236793,39.7468163)(847.65736793,39.69181635)(847.63736816,39.64181936)
\curveto(847.60736798,39.51181653)(847.587368,39.38681666)(847.57736816,39.26681936)
\curveto(847.56736802,39.1468169)(847.54236805,39.03181701)(847.50236816,38.92181936)
\curveto(847.38236821,38.55181749)(847.21236838,38.23181781)(846.99236816,37.96181936)
\curveto(846.77236882,37.69181835)(846.4923691,37.48181856)(846.15236816,37.33181936)
\curveto(846.03236956,37.28181876)(845.90736968,37.23681881)(845.77736816,37.19681936)
\curveto(845.64736994,37.16681888)(845.51237008,37.13181891)(845.37236816,37.09181936)
\curveto(845.32237027,37.08181896)(845.28237031,37.07681897)(845.25236816,37.07681936)
\curveto(845.21237038,37.07681897)(845.16737042,37.07181897)(845.11736816,37.06181936)
\curveto(845.0873705,37.05181899)(845.05237054,37.046819)(845.01236816,37.04681936)
\curveto(844.96237063,37.046819)(844.92237067,37.041819)(844.89236816,37.03181936)
\lineto(844.72736816,37.03181936)
\curveto(844.64737094,37.01181903)(844.54737104,37.00681904)(844.42736816,37.01681936)
\curveto(844.29737129,37.02681902)(844.20737138,37.041819)(844.15736816,37.06181936)
\curveto(844.06737152,37.08181896)(844.00237159,37.13681891)(843.96236816,37.22681936)
\curveto(843.94237165,37.25681879)(843.93737165,37.28681876)(843.94736816,37.31681936)
\curveto(843.94737164,37.3468187)(843.94237165,37.38681866)(843.93236816,37.43681936)
\curveto(843.92237167,37.47681857)(843.91737167,37.51681853)(843.91736816,37.55681936)
\lineto(843.91736816,37.70681936)
\curveto(843.91737167,37.82681822)(843.92237167,37.9468181)(843.93236816,38.06681936)
\curveto(843.93237166,38.19681785)(843.96737162,38.28681776)(844.03736816,38.33681936)
\curveto(844.09737149,38.37681767)(844.15737143,38.39681765)(844.21736816,38.39681936)
\curveto(844.27737131,38.39681765)(844.34737124,38.40681764)(844.42736816,38.42681936)
\curveto(844.45737113,38.43681761)(844.4923711,38.43681761)(844.53236816,38.42681936)
\curveto(844.56237103,38.42681762)(844.587371,38.43181761)(844.60736816,38.44181936)
\lineto(844.81736816,38.44181936)
\curveto(844.86737072,38.46181758)(844.91737067,38.46681758)(844.96736816,38.45681936)
\curveto(845.00737058,38.45681759)(845.05237054,38.46681758)(845.10236816,38.48681936)
\curveto(845.23237036,38.51681753)(845.35737023,38.5468175)(845.47736816,38.57681936)
\curveto(845.58737,38.60681744)(845.6923699,38.65181739)(845.79236816,38.71181936)
\curveto(846.08236951,38.88181716)(846.2873693,39.15181689)(846.40736816,39.52181936)
\curveto(846.42736916,39.57181647)(846.44236915,39.62181642)(846.45236816,39.67181936)
\curveto(846.45236914,39.73181631)(846.46236913,39.78681626)(846.48236816,39.83681936)
\lineto(846.48236816,39.91181936)
\curveto(846.4923691,39.98181606)(846.50236909,40.07681597)(846.51236816,40.19681936)
\curveto(846.51236908,40.32681572)(846.50236909,40.42681562)(846.48236816,40.49681936)
\curveto(846.46236913,40.56681548)(846.44736914,40.63681541)(846.43736816,40.70681936)
\curveto(846.41736917,40.78681526)(846.39736919,40.85681519)(846.37736816,40.91681936)
\curveto(846.21736937,41.29681475)(845.94236965,41.57181447)(845.55236816,41.74181936)
\curveto(845.42237017,41.79181425)(845.26737032,41.82681422)(845.08736816,41.84681936)
\curveto(844.90737068,41.87681417)(844.72237087,41.89181415)(844.53236816,41.89181936)
\curveto(844.33237126,41.90181414)(844.13237146,41.90181414)(843.93236816,41.89181936)
\lineto(843.36236816,41.89181936)
\lineto(839.11736816,41.89181936)
\lineto(837.57236816,41.89181936)
\curveto(837.46237813,41.89181415)(837.34237825,41.88681416)(837.21236816,41.87681936)
\curveto(837.08237851,41.86681418)(836.97737861,41.88681416)(836.89736816,41.93681936)
\curveto(836.82737876,41.99681405)(836.77737881,42.07681397)(836.74736816,42.17681936)
\curveto(836.74737884,42.19681385)(836.74737884,42.21681383)(836.74736816,42.23681936)
\curveto(836.74737884,42.25681379)(836.74237885,42.27681377)(836.73236816,42.29681936)
}
}
{
\newrgbcolor{curcolor}{0 0 0}
\pscustom[linestyle=none,fillstyle=solid,fillcolor=curcolor]
{
\newpath
\moveto(839.68736816,45.83049123)
\lineto(839.68736816,46.26549123)
\curveto(839.6873759,46.41548927)(839.72737586,46.52048916)(839.80736816,46.58049123)
\curveto(839.8873757,46.63048905)(839.9873756,46.65548903)(840.10736816,46.65549123)
\curveto(840.22737536,46.66548902)(840.34737524,46.67048901)(840.46736816,46.67049123)
\lineto(841.89236816,46.67049123)
\lineto(844.15736816,46.67049123)
\lineto(844.84736816,46.67049123)
\curveto(845.07737051,46.67048901)(845.27737031,46.69548899)(845.44736816,46.74549123)
\curveto(845.89736969,46.90548878)(846.21236938,47.20548848)(846.39236816,47.64549123)
\curveto(846.48236911,47.86548782)(846.51736907,48.13048755)(846.49736816,48.44049123)
\curveto(846.46736912,48.75048693)(846.41236918,49.00048668)(846.33236816,49.19049123)
\curveto(846.1923694,49.52048616)(846.01736957,49.7804859)(845.80736816,49.97049123)
\curveto(845.58737,50.17048551)(845.30237029,50.32548536)(844.95236816,50.43549123)
\curveto(844.87237072,50.46548522)(844.7923708,50.4854852)(844.71236816,50.49549123)
\curveto(844.63237096,50.50548518)(844.54737104,50.52048516)(844.45736816,50.54049123)
\curveto(844.40737118,50.55048513)(844.36237123,50.55048513)(844.32236816,50.54049123)
\curveto(844.28237131,50.54048514)(844.23737135,50.55048513)(844.18736816,50.57049123)
\lineto(843.87236816,50.57049123)
\curveto(843.7923718,50.59048509)(843.70237189,50.59548509)(843.60236816,50.58549123)
\curveto(843.4923721,50.57548511)(843.3923722,50.57048511)(843.30236816,50.57049123)
\lineto(842.13236816,50.57049123)
\lineto(840.54236816,50.57049123)
\curveto(840.42237517,50.57048511)(840.29737529,50.56548512)(840.16736816,50.55549123)
\curveto(840.02737556,50.55548513)(839.91737567,50.5804851)(839.83736816,50.63049123)
\curveto(839.7873758,50.67048501)(839.75737583,50.71548497)(839.74736816,50.76549123)
\curveto(839.72737586,50.82548486)(839.70737588,50.89548479)(839.68736816,50.97549123)
\lineto(839.68736816,51.20049123)
\curveto(839.6873759,51.32048436)(839.6923759,51.42548426)(839.70236816,51.51549123)
\curveto(839.71237588,51.61548407)(839.75737583,51.69048399)(839.83736816,51.74049123)
\curveto(839.8873757,51.79048389)(839.96237563,51.81548387)(840.06236816,51.81549123)
\lineto(840.34736816,51.81549123)
\lineto(841.36736816,51.81549123)
\lineto(845.40236816,51.81549123)
\lineto(846.75236816,51.81549123)
\curveto(846.87236872,51.81548387)(846.9873686,51.81048387)(847.09736816,51.80049123)
\curveto(847.19736839,51.80048388)(847.27236832,51.76548392)(847.32236816,51.69549123)
\curveto(847.35236824,51.65548403)(847.37736821,51.59548409)(847.39736816,51.51549123)
\curveto(847.40736818,51.43548425)(847.41736817,51.34548434)(847.42736816,51.24549123)
\curveto(847.42736816,51.15548453)(847.42236817,51.06548462)(847.41236816,50.97549123)
\curveto(847.40236819,50.89548479)(847.38236821,50.83548485)(847.35236816,50.79549123)
\curveto(847.31236828,50.74548494)(847.24736834,50.70048498)(847.15736816,50.66049123)
\curveto(847.11736847,50.65048503)(847.06236853,50.64048504)(846.99236816,50.63049123)
\curveto(846.92236867,50.63048505)(846.85736873,50.62548506)(846.79736816,50.61549123)
\curveto(846.72736886,50.60548508)(846.67236892,50.5854851)(846.63236816,50.55549123)
\curveto(846.592369,50.52548516)(846.57736901,50.4804852)(846.58736816,50.42049123)
\curveto(846.60736898,50.34048534)(846.66736892,50.26048542)(846.76736816,50.18049123)
\curveto(846.85736873,50.10048558)(846.92736866,50.02548566)(846.97736816,49.95549123)
\curveto(847.13736845,49.73548595)(847.27736831,49.4854862)(847.39736816,49.20549123)
\curveto(847.44736814,49.09548659)(847.47736811,48.9804867)(847.48736816,48.86049123)
\curveto(847.50736808,48.75048693)(847.53236806,48.63548705)(847.56236816,48.51549123)
\curveto(847.57236802,48.46548722)(847.57236802,48.41048727)(847.56236816,48.35049123)
\curveto(847.55236804,48.30048738)(847.55736803,48.25048743)(847.57736816,48.20049123)
\curveto(847.59736799,48.10048758)(847.59736799,48.01048767)(847.57736816,47.93049123)
\lineto(847.57736816,47.78049123)
\curveto(847.55736803,47.73048795)(847.54736804,47.67048801)(847.54736816,47.60049123)
\curveto(847.54736804,47.54048814)(847.54236805,47.4854882)(847.53236816,47.43549123)
\curveto(847.51236808,47.39548829)(847.50236809,47.35548833)(847.50236816,47.31549123)
\curveto(847.51236808,47.2854884)(847.50736808,47.24548844)(847.48736816,47.19549123)
\lineto(847.42736816,46.95549123)
\curveto(847.40736818,46.8854888)(847.37736821,46.81048887)(847.33736816,46.73049123)
\curveto(847.22736836,46.47048921)(847.08236851,46.25048943)(846.90236816,46.07049123)
\curveto(846.71236888,45.90048978)(846.4873691,45.76048992)(846.22736816,45.65049123)
\curveto(846.13736945,45.61049007)(846.04736954,45.5804901)(845.95736816,45.56049123)
\lineto(845.65736816,45.50049123)
\curveto(845.59736999,45.4804902)(845.54237005,45.47049021)(845.49236816,45.47049123)
\curveto(845.43237016,45.4804902)(845.36737022,45.47549021)(845.29736816,45.45549123)
\curveto(845.27737031,45.44549024)(845.25237034,45.44049024)(845.22236816,45.44049123)
\curveto(845.18237041,45.44049024)(845.14737044,45.43549025)(845.11736816,45.42549123)
\lineto(844.96736816,45.42549123)
\curveto(844.92737066,45.41549027)(844.88237071,45.41049027)(844.83236816,45.41049123)
\curveto(844.77237082,45.42049026)(844.71737087,45.42549026)(844.66736816,45.42549123)
\lineto(844.06736816,45.42549123)
\lineto(841.30736816,45.42549123)
\lineto(840.34736816,45.42549123)
\lineto(840.07736816,45.42549123)
\curveto(839.9873756,45.42549026)(839.91237568,45.44549024)(839.85236816,45.48549123)
\curveto(839.78237581,45.52549016)(839.73237586,45.60049008)(839.70236816,45.71049123)
\curveto(839.6923759,45.73048995)(839.6923759,45.75048993)(839.70236816,45.77049123)
\curveto(839.70237589,45.79048989)(839.69737589,45.81048987)(839.68736816,45.83049123)
}
}
{
\newrgbcolor{curcolor}{0 0 0}
\pscustom[linestyle=none,fillstyle=solid,fillcolor=curcolor]
{
\newpath
\moveto(836.73236816,54.28510061)
\curveto(836.73237886,54.41509899)(836.73237886,54.55009886)(836.73236816,54.69010061)
\curveto(836.73237886,54.84009857)(836.76737882,54.95009846)(836.83736816,55.02010061)
\curveto(836.90737868,55.07009834)(837.00237859,55.09509831)(837.12236816,55.09510061)
\curveto(837.23237836,55.1050983)(837.34737824,55.1100983)(837.46736816,55.11010061)
\lineto(838.80236816,55.11010061)
\lineto(844.87736816,55.11010061)
\lineto(846.55736816,55.11010061)
\lineto(846.94736816,55.11010061)
\curveto(847.0873685,55.1100983)(847.19736839,55.08509832)(847.27736816,55.03510061)
\curveto(847.32736826,55.0050984)(847.35736823,54.96009845)(847.36736816,54.90010061)
\curveto(847.37736821,54.85009856)(847.3923682,54.78509862)(847.41236816,54.70510061)
\lineto(847.41236816,54.49510061)
\lineto(847.41236816,54.18010061)
\curveto(847.40236819,54.08009933)(847.36736822,54.0050994)(847.30736816,53.95510061)
\curveto(847.22736836,53.9050995)(847.12736846,53.87509953)(847.00736816,53.86510061)
\lineto(846.63236816,53.86510061)
\lineto(845.25236816,53.86510061)
\lineto(839.01236816,53.86510061)
\lineto(837.54236816,53.86510061)
\curveto(837.43237816,53.86509954)(837.31737827,53.86009955)(837.19736816,53.85010061)
\curveto(837.06737852,53.85009956)(836.96737862,53.87509953)(836.89736816,53.92510061)
\curveto(836.83737875,53.96509944)(836.7873788,54.04009937)(836.74736816,54.15010061)
\curveto(836.73737885,54.17009924)(836.73737885,54.19009922)(836.74736816,54.21010061)
\curveto(836.74737884,54.24009917)(836.74237885,54.26509914)(836.73236816,54.28510061)
}
}
{
\newrgbcolor{curcolor}{0 0 0}
\pscustom[linestyle=none,fillstyle=solid,fillcolor=curcolor]
{
}
}
{
\newrgbcolor{curcolor}{0 0 0}
\pscustom[linestyle=none,fillstyle=solid,fillcolor=curcolor]
{
\newpath
\moveto(842.32736816,67.99510061)
\lineto(842.58236816,67.99510061)
\curveto(842.66237293,68.0050929)(842.73737285,68.00009291)(842.80736816,67.98010061)
\lineto(843.04736816,67.98010061)
\lineto(843.21236816,67.98010061)
\curveto(843.31237228,67.96009295)(843.41737217,67.95009296)(843.52736816,67.95010061)
\curveto(843.62737196,67.95009296)(843.72737186,67.94009297)(843.82736816,67.92010061)
\lineto(843.97736816,67.92010061)
\curveto(844.11737147,67.89009302)(844.25737133,67.87009304)(844.39736816,67.86010061)
\curveto(844.52737106,67.85009306)(844.65737093,67.82509308)(844.78736816,67.78510061)
\curveto(844.86737072,67.76509314)(844.95237064,67.74509316)(845.04236816,67.72510061)
\lineto(845.28236816,67.66510061)
\lineto(845.58236816,67.54510061)
\curveto(845.67236992,67.51509339)(845.76236983,67.48009343)(845.85236816,67.44010061)
\curveto(846.07236952,67.34009357)(846.2873693,67.2050937)(846.49736816,67.03510061)
\curveto(846.70736888,66.87509403)(846.87736871,66.70009421)(847.00736816,66.51010061)
\curveto(847.04736854,66.46009445)(847.0873685,66.40009451)(847.12736816,66.33010061)
\curveto(847.15736843,66.27009464)(847.1923684,66.2100947)(847.23236816,66.15010061)
\curveto(847.28236831,66.07009484)(847.32236827,65.97509493)(847.35236816,65.86510061)
\curveto(847.38236821,65.75509515)(847.41236818,65.65009526)(847.44236816,65.55010061)
\curveto(847.48236811,65.44009547)(847.50736808,65.33009558)(847.51736816,65.22010061)
\curveto(847.52736806,65.1100958)(847.54236805,64.99509591)(847.56236816,64.87510061)
\curveto(847.57236802,64.83509607)(847.57236802,64.79009612)(847.56236816,64.74010061)
\curveto(847.56236803,64.70009621)(847.56736802,64.66009625)(847.57736816,64.62010061)
\curveto(847.587368,64.58009633)(847.592368,64.52509638)(847.59236816,64.45510061)
\curveto(847.592368,64.38509652)(847.587368,64.33509657)(847.57736816,64.30510061)
\curveto(847.55736803,64.25509665)(847.55236804,64.2100967)(847.56236816,64.17010061)
\curveto(847.57236802,64.13009678)(847.57236802,64.09509681)(847.56236816,64.06510061)
\lineto(847.56236816,63.97510061)
\curveto(847.54236805,63.91509699)(847.52736806,63.85009706)(847.51736816,63.78010061)
\curveto(847.51736807,63.72009719)(847.51236808,63.65509725)(847.50236816,63.58510061)
\curveto(847.45236814,63.41509749)(847.40236819,63.25509765)(847.35236816,63.10510061)
\curveto(847.30236829,62.95509795)(847.23736835,62.8100981)(847.15736816,62.67010061)
\curveto(847.11736847,62.62009829)(847.0873685,62.56509834)(847.06736816,62.50510061)
\curveto(847.03736855,62.45509845)(847.00236859,62.4050985)(846.96236816,62.35510061)
\curveto(846.78236881,62.11509879)(846.56236903,61.91509899)(846.30236816,61.75510061)
\curveto(846.04236955,61.59509931)(845.75736983,61.45509945)(845.44736816,61.33510061)
\curveto(845.30737028,61.27509963)(845.16737042,61.23009968)(845.02736816,61.20010061)
\curveto(844.87737071,61.17009974)(844.72237087,61.13509977)(844.56236816,61.09510061)
\curveto(844.45237114,61.07509983)(844.34237125,61.06009985)(844.23236816,61.05010061)
\curveto(844.12237147,61.04009987)(844.01237158,61.02509988)(843.90236816,61.00510061)
\curveto(843.86237173,60.99509991)(843.82237177,60.99009992)(843.78236816,60.99010061)
\curveto(843.74237185,61.00009991)(843.70237189,61.00009991)(843.66236816,60.99010061)
\curveto(843.61237198,60.98009993)(843.56237203,60.97509993)(843.51236816,60.97510061)
\lineto(843.34736816,60.97510061)
\curveto(843.29737229,60.95509995)(843.24737234,60.95009996)(843.19736816,60.96010061)
\curveto(843.13737245,60.97009994)(843.08237251,60.97009994)(843.03236816,60.96010061)
\curveto(842.9923726,60.95009996)(842.94737264,60.95009996)(842.89736816,60.96010061)
\curveto(842.84737274,60.97009994)(842.79737279,60.96509994)(842.74736816,60.94510061)
\curveto(842.67737291,60.92509998)(842.60237299,60.92009999)(842.52236816,60.93010061)
\curveto(842.43237316,60.94009997)(842.34737324,60.94509996)(842.26736816,60.94510061)
\curveto(842.17737341,60.94509996)(842.07737351,60.94009997)(841.96736816,60.93010061)
\curveto(841.84737374,60.92009999)(841.74737384,60.92509998)(841.66736816,60.94510061)
\lineto(841.38236816,60.94510061)
\lineto(840.75236816,60.99010061)
\curveto(840.65237494,61.00009991)(840.55737503,61.0100999)(840.46736816,61.02010061)
\lineto(840.16736816,61.05010061)
\curveto(840.11737547,61.07009984)(840.06737552,61.07509983)(840.01736816,61.06510061)
\curveto(839.95737563,61.06509984)(839.90237569,61.07509983)(839.85236816,61.09510061)
\curveto(839.68237591,61.14509976)(839.51737607,61.18509972)(839.35736816,61.21510061)
\curveto(839.1873764,61.24509966)(839.02737656,61.29509961)(838.87736816,61.36510061)
\curveto(838.41737717,61.55509935)(838.04237755,61.77509913)(837.75236816,62.02510061)
\curveto(837.46237813,62.28509862)(837.21737837,62.64509826)(837.01736816,63.10510061)
\curveto(836.96737862,63.23509767)(836.93237866,63.36509754)(836.91236816,63.49510061)
\curveto(836.8923787,63.63509727)(836.86737872,63.77509713)(836.83736816,63.91510061)
\curveto(836.82737876,63.98509692)(836.82237877,64.05009686)(836.82236816,64.11010061)
\curveto(836.82237877,64.17009674)(836.81737877,64.23509667)(836.80736816,64.30510061)
\curveto(836.7873788,65.13509577)(836.93737865,65.8050951)(837.25736816,66.31510061)
\curveto(837.56737802,66.82509408)(838.00737758,67.2050937)(838.57736816,67.45510061)
\curveto(838.69737689,67.5050934)(838.82237677,67.55009336)(838.95236816,67.59010061)
\curveto(839.08237651,67.63009328)(839.21737637,67.67509323)(839.35736816,67.72510061)
\curveto(839.43737615,67.74509316)(839.52237607,67.76009315)(839.61236816,67.77010061)
\lineto(839.85236816,67.83010061)
\curveto(839.96237563,67.86009305)(840.07237552,67.87509303)(840.18236816,67.87510061)
\curveto(840.2923753,67.88509302)(840.40237519,67.90009301)(840.51236816,67.92010061)
\curveto(840.56237503,67.94009297)(840.60737498,67.94509296)(840.64736816,67.93510061)
\curveto(840.6873749,67.93509297)(840.72737486,67.94009297)(840.76736816,67.95010061)
\curveto(840.81737477,67.96009295)(840.87237472,67.96009295)(840.93236816,67.95010061)
\curveto(840.98237461,67.95009296)(841.03237456,67.95509295)(841.08236816,67.96510061)
\lineto(841.21736816,67.96510061)
\curveto(841.27737431,67.98509292)(841.34737424,67.98509292)(841.42736816,67.96510061)
\curveto(841.49737409,67.95509295)(841.56237403,67.96009295)(841.62236816,67.98010061)
\curveto(841.65237394,67.99009292)(841.6923739,67.99509291)(841.74236816,67.99510061)
\lineto(841.86236816,67.99510061)
\lineto(842.32736816,67.99510061)
\moveto(844.65236816,66.45010061)
\curveto(844.33237126,66.55009436)(843.96737162,66.6100943)(843.55736816,66.63010061)
\curveto(843.14737244,66.65009426)(842.73737285,66.66009425)(842.32736816,66.66010061)
\curveto(841.89737369,66.66009425)(841.47737411,66.65009426)(841.06736816,66.63010061)
\curveto(840.65737493,66.6100943)(840.27237532,66.56509434)(839.91236816,66.49510061)
\curveto(839.55237604,66.42509448)(839.23237636,66.31509459)(838.95236816,66.16510061)
\curveto(838.66237693,66.02509488)(838.42737716,65.83009508)(838.24736816,65.58010061)
\curveto(838.13737745,65.42009549)(838.05737753,65.24009567)(838.00736816,65.04010061)
\curveto(837.94737764,64.84009607)(837.91737767,64.59509631)(837.91736816,64.30510061)
\curveto(837.93737765,64.28509662)(837.94737764,64.25009666)(837.94736816,64.20010061)
\curveto(837.93737765,64.15009676)(837.93737765,64.1100968)(837.94736816,64.08010061)
\curveto(837.96737762,64.00009691)(837.9873776,63.92509698)(838.00736816,63.85510061)
\curveto(838.01737757,63.79509711)(838.03737755,63.73009718)(838.06736816,63.66010061)
\curveto(838.1873774,63.39009752)(838.35737723,63.17009774)(838.57736816,63.00010061)
\curveto(838.7873768,62.84009807)(839.03237656,62.7050982)(839.31236816,62.59510061)
\curveto(839.42237617,62.54509836)(839.54237605,62.5050984)(839.67236816,62.47510061)
\curveto(839.7923758,62.45509845)(839.91737567,62.43009848)(840.04736816,62.40010061)
\curveto(840.09737549,62.38009853)(840.15237544,62.37009854)(840.21236816,62.37010061)
\curveto(840.26237533,62.37009854)(840.31237528,62.36509854)(840.36236816,62.35510061)
\curveto(840.45237514,62.34509856)(840.54737504,62.33509857)(840.64736816,62.32510061)
\curveto(840.73737485,62.31509859)(840.83237476,62.3050986)(840.93236816,62.29510061)
\curveto(841.01237458,62.29509861)(841.09737449,62.29009862)(841.18736816,62.28010061)
\lineto(841.42736816,62.28010061)
\lineto(841.60736816,62.28010061)
\curveto(841.63737395,62.27009864)(841.67237392,62.26509864)(841.71236816,62.26510061)
\lineto(841.84736816,62.26510061)
\lineto(842.29736816,62.26510061)
\curveto(842.37737321,62.26509864)(842.46237313,62.26009865)(842.55236816,62.25010061)
\curveto(842.63237296,62.25009866)(842.70737288,62.26009865)(842.77736816,62.28010061)
\lineto(843.04736816,62.28010061)
\curveto(843.06737252,62.28009863)(843.09737249,62.27509863)(843.13736816,62.26510061)
\curveto(843.16737242,62.26509864)(843.1923724,62.27009864)(843.21236816,62.28010061)
\curveto(843.31237228,62.29009862)(843.41237218,62.29509861)(843.51236816,62.29510061)
\curveto(843.60237199,62.3050986)(843.70237189,62.31509859)(843.81236816,62.32510061)
\curveto(843.93237166,62.35509855)(844.05737153,62.37009854)(844.18736816,62.37010061)
\curveto(844.30737128,62.38009853)(844.42237117,62.4050985)(844.53236816,62.44510061)
\curveto(844.83237076,62.52509838)(845.09737049,62.6100983)(845.32736816,62.70010061)
\curveto(845.55737003,62.80009811)(845.77236982,62.94509796)(845.97236816,63.13510061)
\curveto(846.17236942,63.34509756)(846.32236927,63.6100973)(846.42236816,63.93010061)
\curveto(846.44236915,63.97009694)(846.45236914,64.0050969)(846.45236816,64.03510061)
\curveto(846.44236915,64.07509683)(846.44736914,64.12009679)(846.46736816,64.17010061)
\curveto(846.47736911,64.2100967)(846.4873691,64.28009663)(846.49736816,64.38010061)
\curveto(846.50736908,64.49009642)(846.50236909,64.57509633)(846.48236816,64.63510061)
\curveto(846.46236913,64.7050962)(846.45236914,64.77509613)(846.45236816,64.84510061)
\curveto(846.44236915,64.91509599)(846.42736916,64.98009593)(846.40736816,65.04010061)
\curveto(846.34736924,65.24009567)(846.26236933,65.42009549)(846.15236816,65.58010061)
\curveto(846.13236946,65.6100953)(846.11236948,65.63509527)(846.09236816,65.65510061)
\lineto(846.03236816,65.71510061)
\curveto(846.01236958,65.75509515)(845.97236962,65.8050951)(845.91236816,65.86510061)
\curveto(845.77236982,65.96509494)(845.64236995,66.05009486)(845.52236816,66.12010061)
\curveto(845.40237019,66.19009472)(845.25737033,66.26009465)(845.08736816,66.33010061)
\curveto(845.01737057,66.36009455)(844.94737064,66.38009453)(844.87736816,66.39010061)
\curveto(844.80737078,66.4100945)(844.73237086,66.43009448)(844.65236816,66.45010061)
}
}
{
\newrgbcolor{curcolor}{0 0 0}
\pscustom[linestyle=none,fillstyle=solid,fillcolor=curcolor]
{
\newpath
\moveto(844.33736816,76.35970998)
\curveto(844.37737121,76.36970226)(844.42737116,76.36970226)(844.48736816,76.35970998)
\curveto(844.54737104,76.35970227)(844.59737099,76.35470228)(844.63736816,76.34470998)
\curveto(844.67737091,76.34470229)(844.71737087,76.33970229)(844.75736816,76.32970998)
\lineto(844.86236816,76.32970998)
\curveto(844.94237065,76.30970232)(845.02237057,76.29470234)(845.10236816,76.28470998)
\curveto(845.18237041,76.27470236)(845.25737033,76.25470238)(845.32736816,76.22470998)
\curveto(845.40737018,76.20470243)(845.48237011,76.18470245)(845.55236816,76.16470998)
\curveto(845.62236997,76.14470249)(845.69736989,76.11470252)(845.77736816,76.07470998)
\curveto(846.19736939,75.89470274)(846.53736905,75.63970299)(846.79736816,75.30970998)
\curveto(847.05736853,74.97970365)(847.26236833,74.58970404)(847.41236816,74.13970998)
\curveto(847.45236814,74.01970461)(847.47736811,73.89470474)(847.48736816,73.76470998)
\curveto(847.50736808,73.64470499)(847.53236806,73.51970511)(847.56236816,73.38970998)
\curveto(847.57236802,73.3297053)(847.57736801,73.26470537)(847.57736816,73.19470998)
\curveto(847.57736801,73.1347055)(847.58236801,73.06970556)(847.59236816,72.99970998)
\lineto(847.59236816,72.87970998)
\lineto(847.59236816,72.68470998)
\curveto(847.60236799,72.62470601)(847.59736799,72.56970606)(847.57736816,72.51970998)
\curveto(847.55736803,72.44970618)(847.55236804,72.38470625)(847.56236816,72.32470998)
\curveto(847.57236802,72.26470637)(847.56736802,72.20470643)(847.54736816,72.14470998)
\curveto(847.53736805,72.09470654)(847.53236806,72.04970658)(847.53236816,72.00970998)
\curveto(847.53236806,71.96970666)(847.52236807,71.92470671)(847.50236816,71.87470998)
\curveto(847.48236811,71.79470684)(847.46236813,71.71970691)(847.44236816,71.64970998)
\curveto(847.43236816,71.57970705)(847.41736817,71.50970712)(847.39736816,71.43970998)
\curveto(847.22736836,70.95970767)(847.01736857,70.55970807)(846.76736816,70.23970998)
\curveto(846.50736908,69.9297087)(846.15236944,69.67970895)(845.70236816,69.48970998)
\curveto(845.64236995,69.45970917)(845.58237001,69.4347092)(845.52236816,69.41470998)
\curveto(845.45237014,69.40470923)(845.37737021,69.38970924)(845.29736816,69.36970998)
\curveto(845.23737035,69.34970928)(845.17237042,69.3347093)(845.10236816,69.32470998)
\curveto(845.03237056,69.31470932)(844.96237063,69.29970933)(844.89236816,69.27970998)
\curveto(844.84237075,69.26970936)(844.80237079,69.26470937)(844.77236816,69.26470998)
\lineto(844.65236816,69.26470998)
\curveto(844.61237098,69.25470938)(844.56237103,69.24470939)(844.50236816,69.23470998)
\curveto(844.44237115,69.2347094)(844.3923712,69.23970939)(844.35236816,69.24970998)
\lineto(844.21736816,69.24970998)
\curveto(844.16737142,69.25970937)(844.11737147,69.26470937)(844.06736816,69.26470998)
\curveto(843.96737162,69.28470935)(843.87237172,69.29970933)(843.78236816,69.30970998)
\curveto(843.68237191,69.31970931)(843.587372,69.33970929)(843.49736816,69.36970998)
\curveto(843.34737224,69.41970921)(843.20737238,69.47470916)(843.07736816,69.53470998)
\curveto(842.94737264,69.59470904)(842.82737276,69.66470897)(842.71736816,69.74470998)
\curveto(842.66737292,69.77470886)(842.62737296,69.80470883)(842.59736816,69.83470998)
\curveto(842.56737302,69.87470876)(842.53237306,69.90970872)(842.49236816,69.93970998)
\curveto(842.41237318,69.99970863)(842.34237325,70.06970856)(842.28236816,70.14970998)
\curveto(842.23237336,70.20970842)(842.1873734,70.26970836)(842.14736816,70.32970998)
\lineto(841.99736816,70.53970998)
\curveto(841.95737363,70.58970804)(841.92237367,70.63970799)(841.89236816,70.68970998)
\curveto(841.85237374,70.73970789)(841.79737379,70.77470786)(841.72736816,70.79470998)
\curveto(841.69737389,70.79470784)(841.67237392,70.78470785)(841.65236816,70.76470998)
\curveto(841.62237397,70.75470788)(841.59737399,70.74470789)(841.57736816,70.73470998)
\curveto(841.52737406,70.69470794)(841.48237411,70.64470799)(841.44236816,70.58470998)
\curveto(841.3923742,70.5347081)(841.34737424,70.48470815)(841.30736816,70.43470998)
\curveto(841.27737431,70.39470824)(841.22237437,70.34470829)(841.14236816,70.28470998)
\curveto(841.11237448,70.26470837)(841.0873745,70.2347084)(841.06736816,70.19470998)
\curveto(841.03737455,70.16470847)(841.00237459,70.13970849)(840.96236816,70.11970998)
\curveto(840.75237484,69.94970868)(840.50737508,69.81970881)(840.22736816,69.72970998)
\curveto(840.14737544,69.70970892)(840.06737552,69.69470894)(839.98736816,69.68470998)
\curveto(839.90737568,69.67470896)(839.82737576,69.65970897)(839.74736816,69.63970998)
\curveto(839.69737589,69.61970901)(839.63237596,69.60970902)(839.55236816,69.60970998)
\curveto(839.46237613,69.60970902)(839.3923762,69.61970901)(839.34236816,69.63970998)
\curveto(839.24237635,69.63970899)(839.17237642,69.64470899)(839.13236816,69.65470998)
\curveto(839.05237654,69.67470896)(838.98237661,69.68970894)(838.92236816,69.69970998)
\curveto(838.85237674,69.70970892)(838.78237681,69.72470891)(838.71236816,69.74470998)
\curveto(838.28237731,69.89470874)(837.93737765,70.10970852)(837.67736816,70.38970998)
\curveto(837.41737817,70.67970795)(837.20237839,71.0297076)(837.03236816,71.43970998)
\curveto(836.98237861,71.54970708)(836.95237864,71.66470697)(836.94236816,71.78470998)
\curveto(836.92237867,71.91470672)(836.8923787,72.04470659)(836.85236816,72.17470998)
\curveto(836.85237874,72.25470638)(836.85237874,72.32470631)(836.85236816,72.38470998)
\curveto(836.84237875,72.45470618)(836.83237876,72.5297061)(836.82236816,72.60970998)
\curveto(836.80237879,73.39970523)(836.93237866,74.05470458)(837.21236816,74.57470998)
\curveto(837.4923781,75.10470353)(837.90237769,75.48470315)(838.44236816,75.71470998)
\curveto(838.67237692,75.82470281)(838.95737663,75.89470274)(839.29736816,75.92470998)
\curveto(839.62737596,75.96470267)(839.93237566,75.9347027)(840.21236816,75.83470998)
\curveto(840.34237525,75.79470284)(840.46237513,75.74470289)(840.57236816,75.68470998)
\curveto(840.68237491,75.634703)(840.7873748,75.57470306)(840.88736816,75.50470998)
\curveto(840.92737466,75.48470315)(840.96237463,75.45470318)(840.99236816,75.41470998)
\lineto(841.08236816,75.32470998)
\curveto(841.17237442,75.27470336)(841.23737435,75.21470342)(841.27736816,75.14470998)
\curveto(841.32737426,75.09470354)(841.37737421,75.03970359)(841.42736816,74.97970998)
\curveto(841.46737412,74.9297037)(841.51237408,74.88470375)(841.56236816,74.84470998)
\curveto(841.58237401,74.82470381)(841.60737398,74.80470383)(841.63736816,74.78470998)
\curveto(841.65737393,74.77470386)(841.68237391,74.77470386)(841.71236816,74.78470998)
\curveto(841.76237383,74.79470384)(841.81237378,74.82470381)(841.86236816,74.87470998)
\curveto(841.90237369,74.92470371)(841.94237365,74.97970365)(841.98236816,75.03970998)
\lineto(842.10236816,75.21970998)
\curveto(842.13237346,75.27970335)(842.16237343,75.3297033)(842.19236816,75.36970998)
\curveto(842.43237316,75.69970293)(842.74237285,75.94970268)(843.12236816,76.11970998)
\curveto(843.20237239,76.15970247)(843.2873723,76.18970244)(843.37736816,76.20970998)
\curveto(843.46737212,76.23970239)(843.55737203,76.26470237)(843.64736816,76.28470998)
\curveto(843.69737189,76.29470234)(843.75237184,76.30470233)(843.81236816,76.31470998)
\lineto(843.96236816,76.34470998)
\curveto(844.02237157,76.35470228)(844.0873715,76.35470228)(844.15736816,76.34470998)
\curveto(844.21737137,76.3347023)(844.27737131,76.33970229)(844.33736816,76.35970998)
\moveto(839.29736816,70.97470998)
\curveto(839.40737618,70.94470769)(839.54737604,70.93970769)(839.71736816,70.95970998)
\curveto(839.87737571,70.97970765)(840.00237559,71.00470763)(840.09236816,71.03470998)
\curveto(840.41237518,71.14470749)(840.65737493,71.29470734)(840.82736816,71.48470998)
\curveto(840.9873746,71.67470696)(841.11737447,71.93970669)(841.21736816,72.27970998)
\curveto(841.24737434,72.40970622)(841.27237432,72.57470606)(841.29236816,72.77470998)
\curveto(841.30237429,72.97470566)(841.2873743,73.14470549)(841.24736816,73.28470998)
\curveto(841.16737442,73.57470506)(841.05737453,73.81470482)(840.91736816,74.00470998)
\curveto(840.76737482,74.20470443)(840.56737502,74.35970427)(840.31736816,74.46970998)
\curveto(840.26737532,74.48970414)(840.22237537,74.49970413)(840.18236816,74.49970998)
\curveto(840.14237545,74.50970412)(840.09737549,74.52470411)(840.04736816,74.54470998)
\curveto(839.93737565,74.57470406)(839.79737579,74.59470404)(839.62736816,74.60470998)
\curveto(839.45737613,74.61470402)(839.31237628,74.60470403)(839.19236816,74.57470998)
\curveto(839.10237649,74.55470408)(839.01737657,74.5297041)(838.93736816,74.49970998)
\curveto(838.85737673,74.47970415)(838.77737681,74.44470419)(838.69736816,74.39470998)
\curveto(838.42737716,74.22470441)(838.23237736,73.99970463)(838.11236816,73.71970998)
\curveto(837.9923776,73.44970518)(837.93237766,73.08970554)(837.93236816,72.63970998)
\curveto(837.95237764,72.61970601)(837.95737763,72.58970604)(837.94736816,72.54970998)
\curveto(837.93737765,72.50970612)(837.93737765,72.47470616)(837.94736816,72.44470998)
\curveto(837.96737762,72.39470624)(837.98237761,72.33970629)(837.99236816,72.27970998)
\curveto(837.9923776,72.2297064)(838.00237759,72.17970645)(838.02236816,72.12970998)
\curveto(838.11237748,71.88970674)(838.22737736,71.67970695)(838.36736816,71.49970998)
\curveto(838.49737709,71.31970731)(838.67737691,71.17970745)(838.90736816,71.07970998)
\curveto(838.96737662,71.05970757)(839.03237656,71.03970759)(839.10236816,71.01970998)
\curveto(839.16237643,71.00970762)(839.22737636,70.99470764)(839.29736816,70.97470998)
\moveto(844.83236816,74.99470998)
\curveto(844.64237095,75.04470359)(844.43737115,75.04970358)(844.21736816,75.00970998)
\curveto(843.99737159,74.97970365)(843.81737177,74.9347037)(843.67736816,74.87470998)
\curveto(843.30737228,74.70470393)(843.00237259,74.44470419)(842.76236816,74.09470998)
\curveto(842.52237307,73.75470488)(842.40237319,73.31970531)(842.40236816,72.78970998)
\curveto(842.42237317,72.75970587)(842.42737316,72.71970591)(842.41736816,72.66970998)
\curveto(842.39737319,72.61970601)(842.3923732,72.57970605)(842.40236816,72.54970998)
\lineto(842.46236816,72.27970998)
\curveto(842.47237312,72.19970643)(842.4873731,72.11970651)(842.50736816,72.03970998)
\curveto(842.61737297,71.73970689)(842.76237283,71.47470716)(842.94236816,71.24470998)
\curveto(843.12237247,71.02470761)(843.35237224,70.85470778)(843.63236816,70.73470998)
\curveto(843.71237188,70.70470793)(843.7923718,70.67970795)(843.87236816,70.65970998)
\curveto(843.95237164,70.63970799)(844.03737155,70.61970801)(844.12736816,70.59970998)
\curveto(844.24737134,70.56970806)(844.39737119,70.55970807)(844.57736816,70.56970998)
\curveto(844.75737083,70.58970804)(844.89737069,70.61470802)(844.99736816,70.64470998)
\curveto(845.04737054,70.66470797)(845.0923705,70.67470796)(845.13236816,70.67470998)
\curveto(845.16237043,70.68470795)(845.20237039,70.69970793)(845.25236816,70.71970998)
\curveto(845.47237012,70.81970781)(845.67236992,70.94970768)(845.85236816,71.10970998)
\curveto(846.03236956,71.27970735)(846.16736942,71.47470716)(846.25736816,71.69470998)
\curveto(846.29736929,71.76470687)(846.33236926,71.85970677)(846.36236816,71.97970998)
\curveto(846.45236914,72.19970643)(846.49736909,72.45470618)(846.49736816,72.74470998)
\lineto(846.49736816,73.02970998)
\curveto(846.47736911,73.1297055)(846.46236913,73.22470541)(846.45236816,73.31470998)
\curveto(846.44236915,73.40470523)(846.42236917,73.49470514)(846.39236816,73.58470998)
\curveto(846.31236928,73.84470479)(846.18236941,74.08470455)(846.00236816,74.30470998)
\curveto(845.81236978,74.5347041)(845.59736999,74.70470393)(845.35736816,74.81470998)
\curveto(845.27737031,74.85470378)(845.19737039,74.88470375)(845.11736816,74.90470998)
\curveto(845.02737056,74.9347037)(844.93237066,74.96470367)(844.83236816,74.99470998)
}
}
{
\newrgbcolor{curcolor}{0 0 0}
\pscustom[linestyle=none,fillstyle=solid,fillcolor=curcolor]
{
\newpath
\moveto(845.77736816,78.63431936)
\lineto(845.77736816,79.26431936)
\lineto(845.77736816,79.45931936)
\curveto(845.77736981,79.52931683)(845.7873698,79.58931677)(845.80736816,79.63931936)
\curveto(845.84736974,79.70931665)(845.8873697,79.7593166)(845.92736816,79.78931936)
\curveto(845.97736961,79.82931653)(846.04236955,79.84931651)(846.12236816,79.84931936)
\curveto(846.20236939,79.8593165)(846.2873693,79.86431649)(846.37736816,79.86431936)
\lineto(847.09736816,79.86431936)
\curveto(847.57736801,79.86431649)(847.9873676,79.80431655)(848.32736816,79.68431936)
\curveto(848.66736692,79.56431679)(848.94236665,79.36931699)(849.15236816,79.09931936)
\curveto(849.20236639,79.02931733)(849.24736634,78.9593174)(849.28736816,78.88931936)
\curveto(849.33736625,78.82931753)(849.38236621,78.7543176)(849.42236816,78.66431936)
\curveto(849.43236616,78.64431771)(849.44236615,78.61431774)(849.45236816,78.57431936)
\curveto(849.47236612,78.53431782)(849.47736611,78.48931787)(849.46736816,78.43931936)
\curveto(849.43736615,78.34931801)(849.36236623,78.29431806)(849.24236816,78.27431936)
\curveto(849.13236646,78.2543181)(849.03736655,78.26931809)(848.95736816,78.31931936)
\curveto(848.8873667,78.34931801)(848.82236677,78.39431796)(848.76236816,78.45431936)
\curveto(848.71236688,78.52431783)(848.66236693,78.58931777)(848.61236816,78.64931936)
\curveto(848.56236703,78.71931764)(848.4873671,78.77931758)(848.38736816,78.82931936)
\curveto(848.29736729,78.88931747)(848.20736738,78.93931742)(848.11736816,78.97931936)
\curveto(848.0873675,78.99931736)(848.02736756,79.02431733)(847.93736816,79.05431936)
\curveto(847.85736773,79.08431727)(847.7873678,79.08931727)(847.72736816,79.06931936)
\curveto(847.587368,79.03931732)(847.49736809,78.97931738)(847.45736816,78.88931936)
\curveto(847.42736816,78.80931755)(847.41236818,78.71931764)(847.41236816,78.61931936)
\curveto(847.41236818,78.51931784)(847.3873682,78.43431792)(847.33736816,78.36431936)
\curveto(847.26736832,78.27431808)(847.12736846,78.22931813)(846.91736816,78.22931936)
\lineto(846.36236816,78.22931936)
\lineto(846.13736816,78.22931936)
\curveto(846.05736953,78.23931812)(845.9923696,78.2593181)(845.94236816,78.28931936)
\curveto(845.86236973,78.34931801)(845.81736977,78.41931794)(845.80736816,78.49931936)
\curveto(845.79736979,78.51931784)(845.7923698,78.53931782)(845.79236816,78.55931936)
\curveto(845.7923698,78.58931777)(845.7873698,78.61431774)(845.77736816,78.63431936)
}
}
{
\newrgbcolor{curcolor}{0 0 0}
\pscustom[linestyle=none,fillstyle=solid,fillcolor=curcolor]
{
}
}
{
\newrgbcolor{curcolor}{0 0 0}
\pscustom[linestyle=none,fillstyle=solid,fillcolor=curcolor]
{
\newpath
\moveto(836.80736816,89.26463186)
\curveto(836.79737879,89.95462722)(836.91737867,90.55462662)(837.16736816,91.06463186)
\curveto(837.41737817,91.58462559)(837.75237784,91.9796252)(838.17236816,92.24963186)
\curveto(838.25237734,92.29962488)(838.34237725,92.34462483)(838.44236816,92.38463186)
\curveto(838.53237706,92.42462475)(838.62737696,92.46962471)(838.72736816,92.51963186)
\curveto(838.82737676,92.55962462)(838.92737666,92.58962459)(839.02736816,92.60963186)
\curveto(839.12737646,92.62962455)(839.23237636,92.64962453)(839.34236816,92.66963186)
\curveto(839.3923762,92.68962449)(839.43737615,92.69462448)(839.47736816,92.68463186)
\curveto(839.51737607,92.6746245)(839.56237603,92.6796245)(839.61236816,92.69963186)
\curveto(839.66237593,92.70962447)(839.74737584,92.71462446)(839.86736816,92.71463186)
\curveto(839.97737561,92.71462446)(840.06237553,92.70962447)(840.12236816,92.69963186)
\curveto(840.18237541,92.6796245)(840.24237535,92.66962451)(840.30236816,92.66963186)
\curveto(840.36237523,92.6796245)(840.42237517,92.6746245)(840.48236816,92.65463186)
\curveto(840.62237497,92.61462456)(840.75737483,92.5796246)(840.88736816,92.54963186)
\curveto(841.01737457,92.51962466)(841.14237445,92.4796247)(841.26236816,92.42963186)
\curveto(841.40237419,92.36962481)(841.52737406,92.29962488)(841.63736816,92.21963186)
\curveto(841.74737384,92.14962503)(841.85737373,92.0746251)(841.96736816,91.99463186)
\lineto(842.02736816,91.93463186)
\curveto(842.04737354,91.92462525)(842.06737352,91.90962527)(842.08736816,91.88963186)
\curveto(842.24737334,91.76962541)(842.3923732,91.63462554)(842.52236816,91.48463186)
\curveto(842.65237294,91.33462584)(842.77737281,91.174626)(842.89736816,91.00463186)
\curveto(843.11737247,90.69462648)(843.32237227,90.39962678)(843.51236816,90.11963186)
\curveto(843.65237194,89.88962729)(843.7873718,89.65962752)(843.91736816,89.42963186)
\curveto(844.04737154,89.20962797)(844.18237141,88.98962819)(844.32236816,88.76963186)
\curveto(844.4923711,88.51962866)(844.67237092,88.2796289)(844.86236816,88.04963186)
\curveto(845.05237054,87.82962935)(845.27737031,87.63962954)(845.53736816,87.47963186)
\curveto(845.59736999,87.43962974)(845.65736993,87.40462977)(845.71736816,87.37463186)
\curveto(845.76736982,87.34462983)(845.83236976,87.31462986)(845.91236816,87.28463186)
\curveto(845.98236961,87.26462991)(846.04236955,87.25962992)(846.09236816,87.26963186)
\curveto(846.16236943,87.28962989)(846.21736937,87.32462985)(846.25736816,87.37463186)
\curveto(846.2873693,87.42462975)(846.30736928,87.48462969)(846.31736816,87.55463186)
\lineto(846.31736816,87.79463186)
\lineto(846.31736816,88.54463186)
\lineto(846.31736816,91.34963186)
\lineto(846.31736816,92.00963186)
\curveto(846.31736927,92.09962508)(846.32236927,92.18462499)(846.33236816,92.26463186)
\curveto(846.33236926,92.34462483)(846.35236924,92.40962477)(846.39236816,92.45963186)
\curveto(846.43236916,92.50962467)(846.50736908,92.54962463)(846.61736816,92.57963186)
\curveto(846.71736887,92.61962456)(846.81736877,92.62962455)(846.91736816,92.60963186)
\lineto(847.05236816,92.60963186)
\curveto(847.12236847,92.58962459)(847.18236841,92.56962461)(847.23236816,92.54963186)
\curveto(847.28236831,92.52962465)(847.32236827,92.49462468)(847.35236816,92.44463186)
\curveto(847.3923682,92.39462478)(847.41236818,92.32462485)(847.41236816,92.23463186)
\lineto(847.41236816,91.96463186)
\lineto(847.41236816,91.06463186)
\lineto(847.41236816,87.55463186)
\lineto(847.41236816,86.48963186)
\curveto(847.41236818,86.40963077)(847.41736817,86.31963086)(847.42736816,86.21963186)
\curveto(847.42736816,86.11963106)(847.41736817,86.03463114)(847.39736816,85.96463186)
\curveto(847.32736826,85.75463142)(847.14736844,85.68963149)(846.85736816,85.76963186)
\curveto(846.81736877,85.7796314)(846.78236881,85.7796314)(846.75236816,85.76963186)
\curveto(846.71236888,85.76963141)(846.66736892,85.7796314)(846.61736816,85.79963186)
\curveto(846.53736905,85.81963136)(846.45236914,85.83963134)(846.36236816,85.85963186)
\curveto(846.27236932,85.8796313)(846.1873694,85.90463127)(846.10736816,85.93463186)
\curveto(845.61736997,86.09463108)(845.20237039,86.29463088)(844.86236816,86.53463186)
\curveto(844.61237098,86.71463046)(844.3873712,86.91963026)(844.18736816,87.14963186)
\curveto(843.97737161,87.3796298)(843.78237181,87.61962956)(843.60236816,87.86963186)
\curveto(843.42237217,88.12962905)(843.25237234,88.39462878)(843.09236816,88.66463186)
\curveto(842.92237267,88.94462823)(842.74737284,89.21462796)(842.56736816,89.47463186)
\curveto(842.4873731,89.58462759)(842.41237318,89.68962749)(842.34236816,89.78963186)
\curveto(842.27237332,89.89962728)(842.19737339,90.00962717)(842.11736816,90.11963186)
\curveto(842.0873735,90.15962702)(842.05737353,90.19462698)(842.02736816,90.22463186)
\curveto(841.9873736,90.26462691)(841.95737363,90.30462687)(841.93736816,90.34463186)
\curveto(841.82737376,90.48462669)(841.70237389,90.60962657)(841.56236816,90.71963186)
\curveto(841.53237406,90.73962644)(841.50737408,90.76462641)(841.48736816,90.79463186)
\curveto(841.45737413,90.82462635)(841.42737416,90.84962633)(841.39736816,90.86963186)
\curveto(841.29737429,90.94962623)(841.19737439,91.01462616)(841.09736816,91.06463186)
\curveto(840.99737459,91.12462605)(840.8873747,91.179626)(840.76736816,91.22963186)
\curveto(840.69737489,91.25962592)(840.62237497,91.2796259)(840.54236816,91.28963186)
\lineto(840.30236816,91.34963186)
\lineto(840.21236816,91.34963186)
\curveto(840.18237541,91.35962582)(840.15237544,91.36462581)(840.12236816,91.36463186)
\curveto(840.05237554,91.38462579)(839.95737563,91.38962579)(839.83736816,91.37963186)
\curveto(839.70737588,91.3796258)(839.60737598,91.36962581)(839.53736816,91.34963186)
\curveto(839.45737613,91.32962585)(839.38237621,91.30962587)(839.31236816,91.28963186)
\curveto(839.23237636,91.2796259)(839.15237644,91.25962592)(839.07236816,91.22963186)
\curveto(838.83237676,91.11962606)(838.63237696,90.96962621)(838.47236816,90.77963186)
\curveto(838.30237729,90.59962658)(838.16237743,90.3796268)(838.05236816,90.11963186)
\curveto(838.03237756,90.04962713)(838.01737757,89.9796272)(838.00736816,89.90963186)
\curveto(837.9873776,89.83962734)(837.96737762,89.76462741)(837.94736816,89.68463186)
\curveto(837.92737766,89.60462757)(837.91737767,89.49462768)(837.91736816,89.35463186)
\curveto(837.91737767,89.22462795)(837.92737766,89.11962806)(837.94736816,89.03963186)
\curveto(837.95737763,88.9796282)(837.96237763,88.92462825)(837.96236816,88.87463186)
\curveto(837.96237763,88.82462835)(837.97237762,88.7746284)(837.99236816,88.72463186)
\curveto(838.03237756,88.62462855)(838.07237752,88.52962865)(838.11236816,88.43963186)
\curveto(838.15237744,88.35962882)(838.19737739,88.2796289)(838.24736816,88.19963186)
\curveto(838.26737732,88.16962901)(838.2923773,88.13962904)(838.32236816,88.10963186)
\curveto(838.35237724,88.08962909)(838.37737721,88.06462911)(838.39736816,88.03463186)
\lineto(838.47236816,87.95963186)
\curveto(838.4923771,87.92962925)(838.51237708,87.90462927)(838.53236816,87.88463186)
\lineto(838.74236816,87.73463186)
\curveto(838.80237679,87.69462948)(838.86737672,87.64962953)(838.93736816,87.59963186)
\curveto(839.02737656,87.53962964)(839.13237646,87.48962969)(839.25236816,87.44963186)
\curveto(839.36237623,87.41962976)(839.47237612,87.38462979)(839.58236816,87.34463186)
\curveto(839.6923759,87.30462987)(839.83737575,87.2796299)(840.01736816,87.26963186)
\curveto(840.1873754,87.25962992)(840.31237528,87.22962995)(840.39236816,87.17963186)
\curveto(840.47237512,87.12963005)(840.51737507,87.05463012)(840.52736816,86.95463186)
\curveto(840.53737505,86.85463032)(840.54237505,86.74463043)(840.54236816,86.62463186)
\curveto(840.54237505,86.58463059)(840.54737504,86.54463063)(840.55736816,86.50463186)
\curveto(840.55737503,86.46463071)(840.55237504,86.42963075)(840.54236816,86.39963186)
\curveto(840.52237507,86.34963083)(840.51237508,86.29963088)(840.51236816,86.24963186)
\curveto(840.51237508,86.20963097)(840.50237509,86.16963101)(840.48236816,86.12963186)
\curveto(840.42237517,86.03963114)(840.2873753,85.99463118)(840.07736816,85.99463186)
\lineto(839.95736816,85.99463186)
\curveto(839.89737569,86.00463117)(839.83737575,86.00963117)(839.77736816,86.00963186)
\curveto(839.70737588,86.01963116)(839.64237595,86.02963115)(839.58236816,86.03963186)
\curveto(839.47237612,86.05963112)(839.37237622,86.0796311)(839.28236816,86.09963186)
\curveto(839.18237641,86.11963106)(839.0873765,86.14963103)(838.99736816,86.18963186)
\curveto(838.92737666,86.20963097)(838.86737672,86.22963095)(838.81736816,86.24963186)
\lineto(838.63736816,86.30963186)
\curveto(838.37737721,86.42963075)(838.13237746,86.58463059)(837.90236816,86.77463186)
\curveto(837.67237792,86.9746302)(837.4873781,87.18962999)(837.34736816,87.41963186)
\curveto(837.26737832,87.52962965)(837.20237839,87.64462953)(837.15236816,87.76463186)
\lineto(837.00236816,88.15463186)
\curveto(836.95237864,88.26462891)(836.92237867,88.3796288)(836.91236816,88.49963186)
\curveto(836.8923787,88.61962856)(836.86737872,88.74462843)(836.83736816,88.87463186)
\curveto(836.83737875,88.94462823)(836.83737875,89.00962817)(836.83736816,89.06963186)
\curveto(836.82737876,89.12962805)(836.81737877,89.19462798)(836.80736816,89.26463186)
}
}
{
\newrgbcolor{curcolor}{0 0 0}
\pscustom[linestyle=none,fillstyle=solid,fillcolor=curcolor]
{
\newpath
\moveto(842.32736816,101.36424123)
\lineto(842.58236816,101.36424123)
\curveto(842.66237293,101.37423353)(842.73737285,101.36923353)(842.80736816,101.34924123)
\lineto(843.04736816,101.34924123)
\lineto(843.21236816,101.34924123)
\curveto(843.31237228,101.32923357)(843.41737217,101.31923358)(843.52736816,101.31924123)
\curveto(843.62737196,101.31923358)(843.72737186,101.30923359)(843.82736816,101.28924123)
\lineto(843.97736816,101.28924123)
\curveto(844.11737147,101.25923364)(844.25737133,101.23923366)(844.39736816,101.22924123)
\curveto(844.52737106,101.21923368)(844.65737093,101.19423371)(844.78736816,101.15424123)
\curveto(844.86737072,101.13423377)(844.95237064,101.11423379)(845.04236816,101.09424123)
\lineto(845.28236816,101.03424123)
\lineto(845.58236816,100.91424123)
\curveto(845.67236992,100.88423402)(845.76236983,100.84923405)(845.85236816,100.80924123)
\curveto(846.07236952,100.70923419)(846.2873693,100.57423433)(846.49736816,100.40424123)
\curveto(846.70736888,100.24423466)(846.87736871,100.06923483)(847.00736816,99.87924123)
\curveto(847.04736854,99.82923507)(847.0873685,99.76923513)(847.12736816,99.69924123)
\curveto(847.15736843,99.63923526)(847.1923684,99.57923532)(847.23236816,99.51924123)
\curveto(847.28236831,99.43923546)(847.32236827,99.34423556)(847.35236816,99.23424123)
\curveto(847.38236821,99.12423578)(847.41236818,99.01923588)(847.44236816,98.91924123)
\curveto(847.48236811,98.80923609)(847.50736808,98.6992362)(847.51736816,98.58924123)
\curveto(847.52736806,98.47923642)(847.54236805,98.36423654)(847.56236816,98.24424123)
\curveto(847.57236802,98.2042367)(847.57236802,98.15923674)(847.56236816,98.10924123)
\curveto(847.56236803,98.06923683)(847.56736802,98.02923687)(847.57736816,97.98924123)
\curveto(847.587368,97.94923695)(847.592368,97.89423701)(847.59236816,97.82424123)
\curveto(847.592368,97.75423715)(847.587368,97.7042372)(847.57736816,97.67424123)
\curveto(847.55736803,97.62423728)(847.55236804,97.57923732)(847.56236816,97.53924123)
\curveto(847.57236802,97.4992374)(847.57236802,97.46423744)(847.56236816,97.43424123)
\lineto(847.56236816,97.34424123)
\curveto(847.54236805,97.28423762)(847.52736806,97.21923768)(847.51736816,97.14924123)
\curveto(847.51736807,97.08923781)(847.51236808,97.02423788)(847.50236816,96.95424123)
\curveto(847.45236814,96.78423812)(847.40236819,96.62423828)(847.35236816,96.47424123)
\curveto(847.30236829,96.32423858)(847.23736835,96.17923872)(847.15736816,96.03924123)
\curveto(847.11736847,95.98923891)(847.0873685,95.93423897)(847.06736816,95.87424123)
\curveto(847.03736855,95.82423908)(847.00236859,95.77423913)(846.96236816,95.72424123)
\curveto(846.78236881,95.48423942)(846.56236903,95.28423962)(846.30236816,95.12424123)
\curveto(846.04236955,94.96423994)(845.75736983,94.82424008)(845.44736816,94.70424123)
\curveto(845.30737028,94.64424026)(845.16737042,94.5992403)(845.02736816,94.56924123)
\curveto(844.87737071,94.53924036)(844.72237087,94.5042404)(844.56236816,94.46424123)
\curveto(844.45237114,94.44424046)(844.34237125,94.42924047)(844.23236816,94.41924123)
\curveto(844.12237147,94.40924049)(844.01237158,94.39424051)(843.90236816,94.37424123)
\curveto(843.86237173,94.36424054)(843.82237177,94.35924054)(843.78236816,94.35924123)
\curveto(843.74237185,94.36924053)(843.70237189,94.36924053)(843.66236816,94.35924123)
\curveto(843.61237198,94.34924055)(843.56237203,94.34424056)(843.51236816,94.34424123)
\lineto(843.34736816,94.34424123)
\curveto(843.29737229,94.32424058)(843.24737234,94.31924058)(843.19736816,94.32924123)
\curveto(843.13737245,94.33924056)(843.08237251,94.33924056)(843.03236816,94.32924123)
\curveto(842.9923726,94.31924058)(842.94737264,94.31924058)(842.89736816,94.32924123)
\curveto(842.84737274,94.33924056)(842.79737279,94.33424057)(842.74736816,94.31424123)
\curveto(842.67737291,94.29424061)(842.60237299,94.28924061)(842.52236816,94.29924123)
\curveto(842.43237316,94.30924059)(842.34737324,94.31424059)(842.26736816,94.31424123)
\curveto(842.17737341,94.31424059)(842.07737351,94.30924059)(841.96736816,94.29924123)
\curveto(841.84737374,94.28924061)(841.74737384,94.29424061)(841.66736816,94.31424123)
\lineto(841.38236816,94.31424123)
\lineto(840.75236816,94.35924123)
\curveto(840.65237494,94.36924053)(840.55737503,94.37924052)(840.46736816,94.38924123)
\lineto(840.16736816,94.41924123)
\curveto(840.11737547,94.43924046)(840.06737552,94.44424046)(840.01736816,94.43424123)
\curveto(839.95737563,94.43424047)(839.90237569,94.44424046)(839.85236816,94.46424123)
\curveto(839.68237591,94.51424039)(839.51737607,94.55424035)(839.35736816,94.58424123)
\curveto(839.1873764,94.61424029)(839.02737656,94.66424024)(838.87736816,94.73424123)
\curveto(838.41737717,94.92423998)(838.04237755,95.14423976)(837.75236816,95.39424123)
\curveto(837.46237813,95.65423925)(837.21737837,96.01423889)(837.01736816,96.47424123)
\curveto(836.96737862,96.6042383)(836.93237866,96.73423817)(836.91236816,96.86424123)
\curveto(836.8923787,97.0042379)(836.86737872,97.14423776)(836.83736816,97.28424123)
\curveto(836.82737876,97.35423755)(836.82237877,97.41923748)(836.82236816,97.47924123)
\curveto(836.82237877,97.53923736)(836.81737877,97.6042373)(836.80736816,97.67424123)
\curveto(836.7873788,98.5042364)(836.93737865,99.17423573)(837.25736816,99.68424123)
\curveto(837.56737802,100.19423471)(838.00737758,100.57423433)(838.57736816,100.82424123)
\curveto(838.69737689,100.87423403)(838.82237677,100.91923398)(838.95236816,100.95924123)
\curveto(839.08237651,100.9992339)(839.21737637,101.04423386)(839.35736816,101.09424123)
\curveto(839.43737615,101.11423379)(839.52237607,101.12923377)(839.61236816,101.13924123)
\lineto(839.85236816,101.19924123)
\curveto(839.96237563,101.22923367)(840.07237552,101.24423366)(840.18236816,101.24424123)
\curveto(840.2923753,101.25423365)(840.40237519,101.26923363)(840.51236816,101.28924123)
\curveto(840.56237503,101.30923359)(840.60737498,101.31423359)(840.64736816,101.30424123)
\curveto(840.6873749,101.3042336)(840.72737486,101.30923359)(840.76736816,101.31924123)
\curveto(840.81737477,101.32923357)(840.87237472,101.32923357)(840.93236816,101.31924123)
\curveto(840.98237461,101.31923358)(841.03237456,101.32423358)(841.08236816,101.33424123)
\lineto(841.21736816,101.33424123)
\curveto(841.27737431,101.35423355)(841.34737424,101.35423355)(841.42736816,101.33424123)
\curveto(841.49737409,101.32423358)(841.56237403,101.32923357)(841.62236816,101.34924123)
\curveto(841.65237394,101.35923354)(841.6923739,101.36423354)(841.74236816,101.36424123)
\lineto(841.86236816,101.36424123)
\lineto(842.32736816,101.36424123)
\moveto(844.65236816,99.81924123)
\curveto(844.33237126,99.91923498)(843.96737162,99.97923492)(843.55736816,99.99924123)
\curveto(843.14737244,100.01923488)(842.73737285,100.02923487)(842.32736816,100.02924123)
\curveto(841.89737369,100.02923487)(841.47737411,100.01923488)(841.06736816,99.99924123)
\curveto(840.65737493,99.97923492)(840.27237532,99.93423497)(839.91236816,99.86424123)
\curveto(839.55237604,99.79423511)(839.23237636,99.68423522)(838.95236816,99.53424123)
\curveto(838.66237693,99.39423551)(838.42737716,99.1992357)(838.24736816,98.94924123)
\curveto(838.13737745,98.78923611)(838.05737753,98.60923629)(838.00736816,98.40924123)
\curveto(837.94737764,98.20923669)(837.91737767,97.96423694)(837.91736816,97.67424123)
\curveto(837.93737765,97.65423725)(837.94737764,97.61923728)(837.94736816,97.56924123)
\curveto(837.93737765,97.51923738)(837.93737765,97.47923742)(837.94736816,97.44924123)
\curveto(837.96737762,97.36923753)(837.9873776,97.29423761)(838.00736816,97.22424123)
\curveto(838.01737757,97.16423774)(838.03737755,97.0992378)(838.06736816,97.02924123)
\curveto(838.1873774,96.75923814)(838.35737723,96.53923836)(838.57736816,96.36924123)
\curveto(838.7873768,96.20923869)(839.03237656,96.07423883)(839.31236816,95.96424123)
\curveto(839.42237617,95.91423899)(839.54237605,95.87423903)(839.67236816,95.84424123)
\curveto(839.7923758,95.82423908)(839.91737567,95.7992391)(840.04736816,95.76924123)
\curveto(840.09737549,95.74923915)(840.15237544,95.73923916)(840.21236816,95.73924123)
\curveto(840.26237533,95.73923916)(840.31237528,95.73423917)(840.36236816,95.72424123)
\curveto(840.45237514,95.71423919)(840.54737504,95.7042392)(840.64736816,95.69424123)
\curveto(840.73737485,95.68423922)(840.83237476,95.67423923)(840.93236816,95.66424123)
\curveto(841.01237458,95.66423924)(841.09737449,95.65923924)(841.18736816,95.64924123)
\lineto(841.42736816,95.64924123)
\lineto(841.60736816,95.64924123)
\curveto(841.63737395,95.63923926)(841.67237392,95.63423927)(841.71236816,95.63424123)
\lineto(841.84736816,95.63424123)
\lineto(842.29736816,95.63424123)
\curveto(842.37737321,95.63423927)(842.46237313,95.62923927)(842.55236816,95.61924123)
\curveto(842.63237296,95.61923928)(842.70737288,95.62923927)(842.77736816,95.64924123)
\lineto(843.04736816,95.64924123)
\curveto(843.06737252,95.64923925)(843.09737249,95.64423926)(843.13736816,95.63424123)
\curveto(843.16737242,95.63423927)(843.1923724,95.63923926)(843.21236816,95.64924123)
\curveto(843.31237228,95.65923924)(843.41237218,95.66423924)(843.51236816,95.66424123)
\curveto(843.60237199,95.67423923)(843.70237189,95.68423922)(843.81236816,95.69424123)
\curveto(843.93237166,95.72423918)(844.05737153,95.73923916)(844.18736816,95.73924123)
\curveto(844.30737128,95.74923915)(844.42237117,95.77423913)(844.53236816,95.81424123)
\curveto(844.83237076,95.89423901)(845.09737049,95.97923892)(845.32736816,96.06924123)
\curveto(845.55737003,96.16923873)(845.77236982,96.31423859)(845.97236816,96.50424123)
\curveto(846.17236942,96.71423819)(846.32236927,96.97923792)(846.42236816,97.29924123)
\curveto(846.44236915,97.33923756)(846.45236914,97.37423753)(846.45236816,97.40424123)
\curveto(846.44236915,97.44423746)(846.44736914,97.48923741)(846.46736816,97.53924123)
\curveto(846.47736911,97.57923732)(846.4873691,97.64923725)(846.49736816,97.74924123)
\curveto(846.50736908,97.85923704)(846.50236909,97.94423696)(846.48236816,98.00424123)
\curveto(846.46236913,98.07423683)(846.45236914,98.14423676)(846.45236816,98.21424123)
\curveto(846.44236915,98.28423662)(846.42736916,98.34923655)(846.40736816,98.40924123)
\curveto(846.34736924,98.60923629)(846.26236933,98.78923611)(846.15236816,98.94924123)
\curveto(846.13236946,98.97923592)(846.11236948,99.0042359)(846.09236816,99.02424123)
\lineto(846.03236816,99.08424123)
\curveto(846.01236958,99.12423578)(845.97236962,99.17423573)(845.91236816,99.23424123)
\curveto(845.77236982,99.33423557)(845.64236995,99.41923548)(845.52236816,99.48924123)
\curveto(845.40237019,99.55923534)(845.25737033,99.62923527)(845.08736816,99.69924123)
\curveto(845.01737057,99.72923517)(844.94737064,99.74923515)(844.87736816,99.75924123)
\curveto(844.80737078,99.77923512)(844.73237086,99.7992351)(844.65236816,99.81924123)
}
}
{
\newrgbcolor{curcolor}{0 0 0}
\pscustom[linestyle=none,fillstyle=solid,fillcolor=curcolor]
{
\newpath
\moveto(836.80736816,106.77385061)
\curveto(836.80737878,106.87384575)(836.81737877,106.96884566)(836.83736816,107.05885061)
\curveto(836.84737874,107.14884548)(836.87737871,107.21384541)(836.92736816,107.25385061)
\curveto(837.00737858,107.31384531)(837.11237848,107.34384528)(837.24236816,107.34385061)
\lineto(837.63236816,107.34385061)
\lineto(839.13236816,107.34385061)
\lineto(845.52236816,107.34385061)
\lineto(846.69236816,107.34385061)
\lineto(847.00736816,107.34385061)
\curveto(847.10736848,107.35384527)(847.1873684,107.33884529)(847.24736816,107.29885061)
\curveto(847.32736826,107.24884538)(847.37736821,107.17384545)(847.39736816,107.07385061)
\curveto(847.40736818,106.98384564)(847.41236818,106.87384575)(847.41236816,106.74385061)
\lineto(847.41236816,106.51885061)
\curveto(847.3923682,106.43884619)(847.37736821,106.36884626)(847.36736816,106.30885061)
\curveto(847.34736824,106.24884638)(847.30736828,106.19884643)(847.24736816,106.15885061)
\curveto(847.1873684,106.11884651)(847.11236848,106.09884653)(847.02236816,106.09885061)
\lineto(846.72236816,106.09885061)
\lineto(845.62736816,106.09885061)
\lineto(840.28736816,106.09885061)
\curveto(840.19737539,106.07884655)(840.12237547,106.06384656)(840.06236816,106.05385061)
\curveto(839.9923756,106.05384657)(839.93237566,106.0238466)(839.88236816,105.96385061)
\curveto(839.83237576,105.89384673)(839.80737578,105.80384682)(839.80736816,105.69385061)
\curveto(839.79737579,105.59384703)(839.7923758,105.48384714)(839.79236816,105.36385061)
\lineto(839.79236816,104.22385061)
\lineto(839.79236816,103.72885061)
\curveto(839.78237581,103.56884906)(839.72237587,103.45884917)(839.61236816,103.39885061)
\curveto(839.58237601,103.37884925)(839.55237604,103.36884926)(839.52236816,103.36885061)
\curveto(839.48237611,103.36884926)(839.43737615,103.36384926)(839.38736816,103.35385061)
\curveto(839.26737632,103.33384929)(839.15737643,103.33884929)(839.05736816,103.36885061)
\curveto(838.95737663,103.40884922)(838.8873767,103.46384916)(838.84736816,103.53385061)
\curveto(838.79737679,103.61384901)(838.77237682,103.73384889)(838.77236816,103.89385061)
\curveto(838.77237682,104.05384857)(838.75737683,104.18884844)(838.72736816,104.29885061)
\curveto(838.71737687,104.34884828)(838.71237688,104.40384822)(838.71236816,104.46385061)
\curveto(838.70237689,104.5238481)(838.6873769,104.58384804)(838.66736816,104.64385061)
\curveto(838.61737697,104.79384783)(838.56737702,104.93884769)(838.51736816,105.07885061)
\curveto(838.45737713,105.21884741)(838.3873772,105.35384727)(838.30736816,105.48385061)
\curveto(838.21737737,105.623847)(838.11237748,105.74384688)(837.99236816,105.84385061)
\curveto(837.87237772,105.94384668)(837.74237785,106.03884659)(837.60236816,106.12885061)
\curveto(837.50237809,106.18884644)(837.3923782,106.23384639)(837.27236816,106.26385061)
\curveto(837.15237844,106.30384632)(837.04737854,106.35384627)(836.95736816,106.41385061)
\curveto(836.89737869,106.46384616)(836.85737873,106.53384609)(836.83736816,106.62385061)
\curveto(836.82737876,106.64384598)(836.82237877,106.66884596)(836.82236816,106.69885061)
\curveto(836.82237877,106.7288459)(836.81737877,106.75384587)(836.80736816,106.77385061)
}
}
{
\newrgbcolor{curcolor}{0 0 0}
\pscustom[linestyle=none,fillstyle=solid,fillcolor=curcolor]
{
\newpath
\moveto(836.80736816,115.12345998)
\curveto(836.80737878,115.22345513)(836.81737877,115.31845503)(836.83736816,115.40845998)
\curveto(836.84737874,115.49845485)(836.87737871,115.56345479)(836.92736816,115.60345998)
\curveto(837.00737858,115.66345469)(837.11237848,115.69345466)(837.24236816,115.69345998)
\lineto(837.63236816,115.69345998)
\lineto(839.13236816,115.69345998)
\lineto(845.52236816,115.69345998)
\lineto(846.69236816,115.69345998)
\lineto(847.00736816,115.69345998)
\curveto(847.10736848,115.70345465)(847.1873684,115.68845466)(847.24736816,115.64845998)
\curveto(847.32736826,115.59845475)(847.37736821,115.52345483)(847.39736816,115.42345998)
\curveto(847.40736818,115.33345502)(847.41236818,115.22345513)(847.41236816,115.09345998)
\lineto(847.41236816,114.86845998)
\curveto(847.3923682,114.78845556)(847.37736821,114.71845563)(847.36736816,114.65845998)
\curveto(847.34736824,114.59845575)(847.30736828,114.5484558)(847.24736816,114.50845998)
\curveto(847.1873684,114.46845588)(847.11236848,114.4484559)(847.02236816,114.44845998)
\lineto(846.72236816,114.44845998)
\lineto(845.62736816,114.44845998)
\lineto(840.28736816,114.44845998)
\curveto(840.19737539,114.42845592)(840.12237547,114.41345594)(840.06236816,114.40345998)
\curveto(839.9923756,114.40345595)(839.93237566,114.37345598)(839.88236816,114.31345998)
\curveto(839.83237576,114.24345611)(839.80737578,114.1534562)(839.80736816,114.04345998)
\curveto(839.79737579,113.94345641)(839.7923758,113.83345652)(839.79236816,113.71345998)
\lineto(839.79236816,112.57345998)
\lineto(839.79236816,112.07845998)
\curveto(839.78237581,111.91845843)(839.72237587,111.80845854)(839.61236816,111.74845998)
\curveto(839.58237601,111.72845862)(839.55237604,111.71845863)(839.52236816,111.71845998)
\curveto(839.48237611,111.71845863)(839.43737615,111.71345864)(839.38736816,111.70345998)
\curveto(839.26737632,111.68345867)(839.15737643,111.68845866)(839.05736816,111.71845998)
\curveto(838.95737663,111.75845859)(838.8873767,111.81345854)(838.84736816,111.88345998)
\curveto(838.79737679,111.96345839)(838.77237682,112.08345827)(838.77236816,112.24345998)
\curveto(838.77237682,112.40345795)(838.75737683,112.53845781)(838.72736816,112.64845998)
\curveto(838.71737687,112.69845765)(838.71237688,112.7534576)(838.71236816,112.81345998)
\curveto(838.70237689,112.87345748)(838.6873769,112.93345742)(838.66736816,112.99345998)
\curveto(838.61737697,113.14345721)(838.56737702,113.28845706)(838.51736816,113.42845998)
\curveto(838.45737713,113.56845678)(838.3873772,113.70345665)(838.30736816,113.83345998)
\curveto(838.21737737,113.97345638)(838.11237748,114.09345626)(837.99236816,114.19345998)
\curveto(837.87237772,114.29345606)(837.74237785,114.38845596)(837.60236816,114.47845998)
\curveto(837.50237809,114.53845581)(837.3923782,114.58345577)(837.27236816,114.61345998)
\curveto(837.15237844,114.6534557)(837.04737854,114.70345565)(836.95736816,114.76345998)
\curveto(836.89737869,114.81345554)(836.85737873,114.88345547)(836.83736816,114.97345998)
\curveto(836.82737876,114.99345536)(836.82237877,115.01845533)(836.82236816,115.04845998)
\curveto(836.82237877,115.07845527)(836.81737877,115.10345525)(836.80736816,115.12345998)
}
}
{
\newrgbcolor{curcolor}{0 0 0}
\pscustom[linestyle=none,fillstyle=solid,fillcolor=curcolor]
{
\newpath
\moveto(857.64368408,42.29681936)
\curveto(857.64369478,42.36681368)(857.64369478,42.4468136)(857.64368408,42.53681936)
\curveto(857.63369479,42.62681342)(857.63369479,42.71181333)(857.64368408,42.79181936)
\curveto(857.64369478,42.88181316)(857.65369477,42.96181308)(857.67368408,43.03181936)
\curveto(857.69369473,43.11181293)(857.7236947,43.16681288)(857.76368408,43.19681936)
\curveto(857.81369461,43.22681282)(857.88869453,43.2468128)(857.98868408,43.25681936)
\curveto(858.07869434,43.27681277)(858.18369424,43.28681276)(858.30368408,43.28681936)
\curveto(858.41369401,43.29681275)(858.52869389,43.29681275)(858.64868408,43.28681936)
\lineto(858.94868408,43.28681936)
\lineto(861.96368408,43.28681936)
\lineto(864.85868408,43.28681936)
\curveto(865.18868723,43.28681276)(865.51368691,43.28181276)(865.83368408,43.27181936)
\curveto(866.14368628,43.27181277)(866.423686,43.23181281)(866.67368408,43.15181936)
\curveto(867.0236854,43.03181301)(867.3186851,42.87681317)(867.55868408,42.68681936)
\curveto(867.78868463,42.49681355)(867.98868443,42.25681379)(868.15868408,41.96681936)
\curveto(868.20868421,41.90681414)(868.24368418,41.8418142)(868.26368408,41.77181936)
\curveto(868.28368414,41.71181433)(868.30868411,41.6418144)(868.33868408,41.56181936)
\curveto(868.38868403,41.4418146)(868.423684,41.31181473)(868.44368408,41.17181936)
\curveto(868.47368395,41.041815)(868.50368392,40.90681514)(868.53368408,40.76681936)
\curveto(868.55368387,40.71681533)(868.55868386,40.66681538)(868.54868408,40.61681936)
\curveto(868.53868388,40.56681548)(868.53868388,40.51181553)(868.54868408,40.45181936)
\curveto(868.55868386,40.43181561)(868.55868386,40.40681564)(868.54868408,40.37681936)
\curveto(868.54868387,40.3468157)(868.55368387,40.32181572)(868.56368408,40.30181936)
\curveto(868.57368385,40.26181578)(868.57868384,40.20681584)(868.57868408,40.13681936)
\curveto(868.57868384,40.06681598)(868.57368385,40.01181603)(868.56368408,39.97181936)
\curveto(868.55368387,39.92181612)(868.55368387,39.86681618)(868.56368408,39.80681936)
\curveto(868.57368385,39.7468163)(868.56868385,39.69181635)(868.54868408,39.64181936)
\curveto(868.5186839,39.51181653)(868.49868392,39.38681666)(868.48868408,39.26681936)
\curveto(868.47868394,39.1468169)(868.45368397,39.03181701)(868.41368408,38.92181936)
\curveto(868.29368413,38.55181749)(868.1236843,38.23181781)(867.90368408,37.96181936)
\curveto(867.68368474,37.69181835)(867.40368502,37.48181856)(867.06368408,37.33181936)
\curveto(866.94368548,37.28181876)(866.8186856,37.23681881)(866.68868408,37.19681936)
\curveto(866.55868586,37.16681888)(866.423686,37.13181891)(866.28368408,37.09181936)
\curveto(866.23368619,37.08181896)(866.19368623,37.07681897)(866.16368408,37.07681936)
\curveto(866.1236863,37.07681897)(866.07868634,37.07181897)(866.02868408,37.06181936)
\curveto(865.99868642,37.05181899)(865.96368646,37.046819)(865.92368408,37.04681936)
\curveto(865.87368655,37.046819)(865.83368659,37.041819)(865.80368408,37.03181936)
\lineto(865.63868408,37.03181936)
\curveto(865.55868686,37.01181903)(865.45868696,37.00681904)(865.33868408,37.01681936)
\curveto(865.20868721,37.02681902)(865.1186873,37.041819)(865.06868408,37.06181936)
\curveto(864.97868744,37.08181896)(864.91368751,37.13681891)(864.87368408,37.22681936)
\curveto(864.85368757,37.25681879)(864.84868757,37.28681876)(864.85868408,37.31681936)
\curveto(864.85868756,37.3468187)(864.85368757,37.38681866)(864.84368408,37.43681936)
\curveto(864.83368759,37.47681857)(864.82868759,37.51681853)(864.82868408,37.55681936)
\lineto(864.82868408,37.70681936)
\curveto(864.82868759,37.82681822)(864.83368759,37.9468181)(864.84368408,38.06681936)
\curveto(864.84368758,38.19681785)(864.87868754,38.28681776)(864.94868408,38.33681936)
\curveto(865.00868741,38.37681767)(865.06868735,38.39681765)(865.12868408,38.39681936)
\curveto(865.18868723,38.39681765)(865.25868716,38.40681764)(865.33868408,38.42681936)
\curveto(865.36868705,38.43681761)(865.40368702,38.43681761)(865.44368408,38.42681936)
\curveto(865.47368695,38.42681762)(865.49868692,38.43181761)(865.51868408,38.44181936)
\lineto(865.72868408,38.44181936)
\curveto(865.77868664,38.46181758)(865.82868659,38.46681758)(865.87868408,38.45681936)
\curveto(865.9186865,38.45681759)(865.96368646,38.46681758)(866.01368408,38.48681936)
\curveto(866.14368628,38.51681753)(866.26868615,38.5468175)(866.38868408,38.57681936)
\curveto(866.49868592,38.60681744)(866.60368582,38.65181739)(866.70368408,38.71181936)
\curveto(866.99368543,38.88181716)(867.19868522,39.15181689)(867.31868408,39.52181936)
\curveto(867.33868508,39.57181647)(867.35368507,39.62181642)(867.36368408,39.67181936)
\curveto(867.36368506,39.73181631)(867.37368505,39.78681626)(867.39368408,39.83681936)
\lineto(867.39368408,39.91181936)
\curveto(867.40368502,39.98181606)(867.41368501,40.07681597)(867.42368408,40.19681936)
\curveto(867.423685,40.32681572)(867.41368501,40.42681562)(867.39368408,40.49681936)
\curveto(867.37368505,40.56681548)(867.35868506,40.63681541)(867.34868408,40.70681936)
\curveto(867.32868509,40.78681526)(867.30868511,40.85681519)(867.28868408,40.91681936)
\curveto(867.12868529,41.29681475)(866.85368557,41.57181447)(866.46368408,41.74181936)
\curveto(866.33368609,41.79181425)(866.17868624,41.82681422)(865.99868408,41.84681936)
\curveto(865.8186866,41.87681417)(865.63368679,41.89181415)(865.44368408,41.89181936)
\curveto(865.24368718,41.90181414)(865.04368738,41.90181414)(864.84368408,41.89181936)
\lineto(864.27368408,41.89181936)
\lineto(860.02868408,41.89181936)
\lineto(858.48368408,41.89181936)
\curveto(858.37369405,41.89181415)(858.25369417,41.88681416)(858.12368408,41.87681936)
\curveto(857.99369443,41.86681418)(857.88869453,41.88681416)(857.80868408,41.93681936)
\curveto(857.73869468,41.99681405)(857.68869473,42.07681397)(857.65868408,42.17681936)
\curveto(857.65869476,42.19681385)(857.65869476,42.21681383)(857.65868408,42.23681936)
\curveto(857.65869476,42.25681379)(857.65369477,42.27681377)(857.64368408,42.29681936)
}
}
{
\newrgbcolor{curcolor}{0 0 0}
\pscustom[linestyle=none,fillstyle=solid,fillcolor=curcolor]
{
\newpath
\moveto(860.59868408,45.83049123)
\lineto(860.59868408,46.26549123)
\curveto(860.59869182,46.41548927)(860.63869178,46.52048916)(860.71868408,46.58049123)
\curveto(860.79869162,46.63048905)(860.89869152,46.65548903)(861.01868408,46.65549123)
\curveto(861.13869128,46.66548902)(861.25869116,46.67048901)(861.37868408,46.67049123)
\lineto(862.80368408,46.67049123)
\lineto(865.06868408,46.67049123)
\lineto(865.75868408,46.67049123)
\curveto(865.98868643,46.67048901)(866.18868623,46.69548899)(866.35868408,46.74549123)
\curveto(866.80868561,46.90548878)(867.1236853,47.20548848)(867.30368408,47.64549123)
\curveto(867.39368503,47.86548782)(867.42868499,48.13048755)(867.40868408,48.44049123)
\curveto(867.37868504,48.75048693)(867.3236851,49.00048668)(867.24368408,49.19049123)
\curveto(867.10368532,49.52048616)(866.92868549,49.7804859)(866.71868408,49.97049123)
\curveto(866.49868592,50.17048551)(866.21368621,50.32548536)(865.86368408,50.43549123)
\curveto(865.78368664,50.46548522)(865.70368672,50.4854852)(865.62368408,50.49549123)
\curveto(865.54368688,50.50548518)(865.45868696,50.52048516)(865.36868408,50.54049123)
\curveto(865.3186871,50.55048513)(865.27368715,50.55048513)(865.23368408,50.54049123)
\curveto(865.19368723,50.54048514)(865.14868727,50.55048513)(865.09868408,50.57049123)
\lineto(864.78368408,50.57049123)
\curveto(864.70368772,50.59048509)(864.61368781,50.59548509)(864.51368408,50.58549123)
\curveto(864.40368802,50.57548511)(864.30368812,50.57048511)(864.21368408,50.57049123)
\lineto(863.04368408,50.57049123)
\lineto(861.45368408,50.57049123)
\curveto(861.33369109,50.57048511)(861.20869121,50.56548512)(861.07868408,50.55549123)
\curveto(860.93869148,50.55548513)(860.82869159,50.5804851)(860.74868408,50.63049123)
\curveto(860.69869172,50.67048501)(860.66869175,50.71548497)(860.65868408,50.76549123)
\curveto(860.63869178,50.82548486)(860.6186918,50.89548479)(860.59868408,50.97549123)
\lineto(860.59868408,51.20049123)
\curveto(860.59869182,51.32048436)(860.60369182,51.42548426)(860.61368408,51.51549123)
\curveto(860.6236918,51.61548407)(860.66869175,51.69048399)(860.74868408,51.74049123)
\curveto(860.79869162,51.79048389)(860.87369155,51.81548387)(860.97368408,51.81549123)
\lineto(861.25868408,51.81549123)
\lineto(862.27868408,51.81549123)
\lineto(866.31368408,51.81549123)
\lineto(867.66368408,51.81549123)
\curveto(867.78368464,51.81548387)(867.89868452,51.81048387)(868.00868408,51.80049123)
\curveto(868.10868431,51.80048388)(868.18368424,51.76548392)(868.23368408,51.69549123)
\curveto(868.26368416,51.65548403)(868.28868413,51.59548409)(868.30868408,51.51549123)
\curveto(868.3186841,51.43548425)(868.32868409,51.34548434)(868.33868408,51.24549123)
\curveto(868.33868408,51.15548453)(868.33368409,51.06548462)(868.32368408,50.97549123)
\curveto(868.31368411,50.89548479)(868.29368413,50.83548485)(868.26368408,50.79549123)
\curveto(868.2236842,50.74548494)(868.15868426,50.70048498)(868.06868408,50.66049123)
\curveto(868.02868439,50.65048503)(867.97368445,50.64048504)(867.90368408,50.63049123)
\curveto(867.83368459,50.63048505)(867.76868465,50.62548506)(867.70868408,50.61549123)
\curveto(867.63868478,50.60548508)(867.58368484,50.5854851)(867.54368408,50.55549123)
\curveto(867.50368492,50.52548516)(867.48868493,50.4804852)(867.49868408,50.42049123)
\curveto(867.5186849,50.34048534)(867.57868484,50.26048542)(867.67868408,50.18049123)
\curveto(867.76868465,50.10048558)(867.83868458,50.02548566)(867.88868408,49.95549123)
\curveto(868.04868437,49.73548595)(868.18868423,49.4854862)(868.30868408,49.20549123)
\curveto(868.35868406,49.09548659)(868.38868403,48.9804867)(868.39868408,48.86049123)
\curveto(868.418684,48.75048693)(868.44368398,48.63548705)(868.47368408,48.51549123)
\curveto(868.48368394,48.46548722)(868.48368394,48.41048727)(868.47368408,48.35049123)
\curveto(868.46368396,48.30048738)(868.46868395,48.25048743)(868.48868408,48.20049123)
\curveto(868.50868391,48.10048758)(868.50868391,48.01048767)(868.48868408,47.93049123)
\lineto(868.48868408,47.78049123)
\curveto(868.46868395,47.73048795)(868.45868396,47.67048801)(868.45868408,47.60049123)
\curveto(868.45868396,47.54048814)(868.45368397,47.4854882)(868.44368408,47.43549123)
\curveto(868.423684,47.39548829)(868.41368401,47.35548833)(868.41368408,47.31549123)
\curveto(868.423684,47.2854884)(868.418684,47.24548844)(868.39868408,47.19549123)
\lineto(868.33868408,46.95549123)
\curveto(868.3186841,46.8854888)(868.28868413,46.81048887)(868.24868408,46.73049123)
\curveto(868.13868428,46.47048921)(867.99368443,46.25048943)(867.81368408,46.07049123)
\curveto(867.6236848,45.90048978)(867.39868502,45.76048992)(867.13868408,45.65049123)
\curveto(867.04868537,45.61049007)(866.95868546,45.5804901)(866.86868408,45.56049123)
\lineto(866.56868408,45.50049123)
\curveto(866.50868591,45.4804902)(866.45368597,45.47049021)(866.40368408,45.47049123)
\curveto(866.34368608,45.4804902)(866.27868614,45.47549021)(866.20868408,45.45549123)
\curveto(866.18868623,45.44549024)(866.16368626,45.44049024)(866.13368408,45.44049123)
\curveto(866.09368633,45.44049024)(866.05868636,45.43549025)(866.02868408,45.42549123)
\lineto(865.87868408,45.42549123)
\curveto(865.83868658,45.41549027)(865.79368663,45.41049027)(865.74368408,45.41049123)
\curveto(865.68368674,45.42049026)(865.62868679,45.42549026)(865.57868408,45.42549123)
\lineto(864.97868408,45.42549123)
\lineto(862.21868408,45.42549123)
\lineto(861.25868408,45.42549123)
\lineto(860.98868408,45.42549123)
\curveto(860.89869152,45.42549026)(860.8236916,45.44549024)(860.76368408,45.48549123)
\curveto(860.69369173,45.52549016)(860.64369178,45.60049008)(860.61368408,45.71049123)
\curveto(860.60369182,45.73048995)(860.60369182,45.75048993)(860.61368408,45.77049123)
\curveto(860.61369181,45.79048989)(860.60869181,45.81048987)(860.59868408,45.83049123)
}
}
{
\newrgbcolor{curcolor}{0 0 0}
\pscustom[linestyle=none,fillstyle=solid,fillcolor=curcolor]
{
\newpath
\moveto(857.64368408,54.28510061)
\curveto(857.64369478,54.41509899)(857.64369478,54.55009886)(857.64368408,54.69010061)
\curveto(857.64369478,54.84009857)(857.67869474,54.95009846)(857.74868408,55.02010061)
\curveto(857.8186946,55.07009834)(857.91369451,55.09509831)(858.03368408,55.09510061)
\curveto(858.14369428,55.1050983)(858.25869416,55.1100983)(858.37868408,55.11010061)
\lineto(859.71368408,55.11010061)
\lineto(865.78868408,55.11010061)
\lineto(867.46868408,55.11010061)
\lineto(867.85868408,55.11010061)
\curveto(867.99868442,55.1100983)(868.10868431,55.08509832)(868.18868408,55.03510061)
\curveto(868.23868418,55.0050984)(868.26868415,54.96009845)(868.27868408,54.90010061)
\curveto(868.28868413,54.85009856)(868.30368412,54.78509862)(868.32368408,54.70510061)
\lineto(868.32368408,54.49510061)
\lineto(868.32368408,54.18010061)
\curveto(868.31368411,54.08009933)(868.27868414,54.0050994)(868.21868408,53.95510061)
\curveto(868.13868428,53.9050995)(868.03868438,53.87509953)(867.91868408,53.86510061)
\lineto(867.54368408,53.86510061)
\lineto(866.16368408,53.86510061)
\lineto(859.92368408,53.86510061)
\lineto(858.45368408,53.86510061)
\curveto(858.34369408,53.86509954)(858.22869419,53.86009955)(858.10868408,53.85010061)
\curveto(857.97869444,53.85009956)(857.87869454,53.87509953)(857.80868408,53.92510061)
\curveto(857.74869467,53.96509944)(857.69869472,54.04009937)(857.65868408,54.15010061)
\curveto(857.64869477,54.17009924)(857.64869477,54.19009922)(857.65868408,54.21010061)
\curveto(857.65869476,54.24009917)(857.65369477,54.26509914)(857.64368408,54.28510061)
}
}
{
\newrgbcolor{curcolor}{0 0 0}
\pscustom[linestyle=none,fillstyle=solid,fillcolor=curcolor]
{
}
}
{
\newrgbcolor{curcolor}{0 0 0}
\pscustom[linestyle=none,fillstyle=solid,fillcolor=curcolor]
{
\newpath
\moveto(857.71868408,65.05510061)
\curveto(857.7186947,65.15509575)(857.72869469,65.25009566)(857.74868408,65.34010061)
\curveto(857.75869466,65.43009548)(857.78869463,65.49509541)(857.83868408,65.53510061)
\curveto(857.9186945,65.59509531)(858.0236944,65.62509528)(858.15368408,65.62510061)
\lineto(858.54368408,65.62510061)
\lineto(860.04368408,65.62510061)
\lineto(866.43368408,65.62510061)
\lineto(867.60368408,65.62510061)
\lineto(867.91868408,65.62510061)
\curveto(868.0186844,65.63509527)(868.09868432,65.62009529)(868.15868408,65.58010061)
\curveto(868.23868418,65.53009538)(868.28868413,65.45509545)(868.30868408,65.35510061)
\curveto(868.3186841,65.26509564)(868.3236841,65.15509575)(868.32368408,65.02510061)
\lineto(868.32368408,64.80010061)
\curveto(868.30368412,64.72009619)(868.28868413,64.65009626)(868.27868408,64.59010061)
\curveto(868.25868416,64.53009638)(868.2186842,64.48009643)(868.15868408,64.44010061)
\curveto(868.09868432,64.40009651)(868.0236844,64.38009653)(867.93368408,64.38010061)
\lineto(867.63368408,64.38010061)
\lineto(866.53868408,64.38010061)
\lineto(861.19868408,64.38010061)
\curveto(861.10869131,64.36009655)(861.03369139,64.34509656)(860.97368408,64.33510061)
\curveto(860.90369152,64.33509657)(860.84369158,64.3050966)(860.79368408,64.24510061)
\curveto(860.74369168,64.17509673)(860.7186917,64.08509682)(860.71868408,63.97510061)
\curveto(860.70869171,63.87509703)(860.70369172,63.76509714)(860.70368408,63.64510061)
\lineto(860.70368408,62.50510061)
\lineto(860.70368408,62.01010061)
\curveto(860.69369173,61.85009906)(860.63369179,61.74009917)(860.52368408,61.68010061)
\curveto(860.49369193,61.66009925)(860.46369196,61.65009926)(860.43368408,61.65010061)
\curveto(860.39369203,61.65009926)(860.34869207,61.64509926)(860.29868408,61.63510061)
\curveto(860.17869224,61.61509929)(860.06869235,61.62009929)(859.96868408,61.65010061)
\curveto(859.86869255,61.69009922)(859.79869262,61.74509916)(859.75868408,61.81510061)
\curveto(859.70869271,61.89509901)(859.68369274,62.01509889)(859.68368408,62.17510061)
\curveto(859.68369274,62.33509857)(859.66869275,62.47009844)(859.63868408,62.58010061)
\curveto(859.62869279,62.63009828)(859.6236928,62.68509822)(859.62368408,62.74510061)
\curveto(859.61369281,62.8050981)(859.59869282,62.86509804)(859.57868408,62.92510061)
\curveto(859.52869289,63.07509783)(859.47869294,63.22009769)(859.42868408,63.36010061)
\curveto(859.36869305,63.50009741)(859.29869312,63.63509727)(859.21868408,63.76510061)
\curveto(859.12869329,63.905097)(859.0236934,64.02509688)(858.90368408,64.12510061)
\curveto(858.78369364,64.22509668)(858.65369377,64.32009659)(858.51368408,64.41010061)
\curveto(858.41369401,64.47009644)(858.30369412,64.51509639)(858.18368408,64.54510061)
\curveto(858.06369436,64.58509632)(857.95869446,64.63509627)(857.86868408,64.69510061)
\curveto(857.80869461,64.74509616)(857.76869465,64.81509609)(857.74868408,64.90510061)
\curveto(857.73869468,64.92509598)(857.73369469,64.95009596)(857.73368408,64.98010061)
\curveto(857.73369469,65.0100959)(857.72869469,65.03509587)(857.71868408,65.05510061)
}
}
{
\newrgbcolor{curcolor}{0 0 0}
\pscustom[linestyle=none,fillstyle=solid,fillcolor=curcolor]
{
\newpath
\moveto(857.71868408,72.59470998)
\curveto(857.70869471,73.28470535)(857.82869459,73.88470475)(858.07868408,74.39470998)
\curveto(858.32869409,74.91470372)(858.66369376,75.30970332)(859.08368408,75.57970998)
\curveto(859.16369326,75.629703)(859.25369317,75.67470296)(859.35368408,75.71470998)
\curveto(859.44369298,75.75470288)(859.53869288,75.79970283)(859.63868408,75.84970998)
\curveto(859.73869268,75.88970274)(859.83869258,75.91970271)(859.93868408,75.93970998)
\curveto(860.03869238,75.95970267)(860.14369228,75.97970265)(860.25368408,75.99970998)
\curveto(860.30369212,76.01970261)(860.34869207,76.02470261)(860.38868408,76.01470998)
\curveto(860.42869199,76.00470263)(860.47369195,76.00970262)(860.52368408,76.02970998)
\curveto(860.57369185,76.03970259)(860.65869176,76.04470259)(860.77868408,76.04470998)
\curveto(860.88869153,76.04470259)(860.97369145,76.03970259)(861.03368408,76.02970998)
\curveto(861.09369133,76.00970262)(861.15369127,75.99970263)(861.21368408,75.99970998)
\curveto(861.27369115,76.00970262)(861.33369109,76.00470263)(861.39368408,75.98470998)
\curveto(861.53369089,75.94470269)(861.66869075,75.90970272)(861.79868408,75.87970998)
\curveto(861.92869049,75.84970278)(862.05369037,75.80970282)(862.17368408,75.75970998)
\curveto(862.31369011,75.69970293)(862.43868998,75.629703)(862.54868408,75.54970998)
\curveto(862.65868976,75.47970315)(862.76868965,75.40470323)(862.87868408,75.32470998)
\lineto(862.93868408,75.26470998)
\curveto(862.95868946,75.25470338)(862.97868944,75.23970339)(862.99868408,75.21970998)
\curveto(863.15868926,75.09970353)(863.30368912,74.96470367)(863.43368408,74.81470998)
\curveto(863.56368886,74.66470397)(863.68868873,74.50470413)(863.80868408,74.33470998)
\curveto(864.02868839,74.02470461)(864.23368819,73.7297049)(864.42368408,73.44970998)
\curveto(864.56368786,73.21970541)(864.69868772,72.98970564)(864.82868408,72.75970998)
\curveto(864.95868746,72.53970609)(865.09368733,72.31970631)(865.23368408,72.09970998)
\curveto(865.40368702,71.84970678)(865.58368684,71.60970702)(865.77368408,71.37970998)
\curveto(865.96368646,71.15970747)(866.18868623,70.96970766)(866.44868408,70.80970998)
\curveto(866.50868591,70.76970786)(866.56868585,70.7347079)(866.62868408,70.70470998)
\curveto(866.67868574,70.67470796)(866.74368568,70.64470799)(866.82368408,70.61470998)
\curveto(866.89368553,70.59470804)(866.95368547,70.58970804)(867.00368408,70.59970998)
\curveto(867.07368535,70.61970801)(867.12868529,70.65470798)(867.16868408,70.70470998)
\curveto(867.19868522,70.75470788)(867.2186852,70.81470782)(867.22868408,70.88470998)
\lineto(867.22868408,71.12470998)
\lineto(867.22868408,71.87470998)
\lineto(867.22868408,74.67970998)
\lineto(867.22868408,75.33970998)
\curveto(867.22868519,75.4297032)(867.23368519,75.51470312)(867.24368408,75.59470998)
\curveto(867.24368518,75.67470296)(867.26368516,75.73970289)(867.30368408,75.78970998)
\curveto(867.34368508,75.83970279)(867.418685,75.87970275)(867.52868408,75.90970998)
\curveto(867.62868479,75.94970268)(867.72868469,75.95970267)(867.82868408,75.93970998)
\lineto(867.96368408,75.93970998)
\curveto(868.03368439,75.91970271)(868.09368433,75.89970273)(868.14368408,75.87970998)
\curveto(868.19368423,75.85970277)(868.23368419,75.82470281)(868.26368408,75.77470998)
\curveto(868.30368412,75.72470291)(868.3236841,75.65470298)(868.32368408,75.56470998)
\lineto(868.32368408,75.29470998)
\lineto(868.32368408,74.39470998)
\lineto(868.32368408,70.88470998)
\lineto(868.32368408,69.81970998)
\curveto(868.3236841,69.73970889)(868.32868409,69.64970898)(868.33868408,69.54970998)
\curveto(868.33868408,69.44970918)(868.32868409,69.36470927)(868.30868408,69.29470998)
\curveto(868.23868418,69.08470955)(868.05868436,69.01970961)(867.76868408,69.09970998)
\curveto(867.72868469,69.10970952)(867.69368473,69.10970952)(867.66368408,69.09970998)
\curveto(867.6236848,69.09970953)(867.57868484,69.10970952)(867.52868408,69.12970998)
\curveto(867.44868497,69.14970948)(867.36368506,69.16970946)(867.27368408,69.18970998)
\curveto(867.18368524,69.20970942)(867.09868532,69.2347094)(867.01868408,69.26470998)
\curveto(866.52868589,69.42470921)(866.11368631,69.62470901)(865.77368408,69.86470998)
\curveto(865.5236869,70.04470859)(865.29868712,70.24970838)(865.09868408,70.47970998)
\curveto(864.88868753,70.70970792)(864.69368773,70.94970768)(864.51368408,71.19970998)
\curveto(864.33368809,71.45970717)(864.16368826,71.72470691)(864.00368408,71.99470998)
\curveto(863.83368859,72.27470636)(863.65868876,72.54470609)(863.47868408,72.80470998)
\curveto(863.39868902,72.91470572)(863.3236891,73.01970561)(863.25368408,73.11970998)
\curveto(863.18368924,73.2297054)(863.10868931,73.33970529)(863.02868408,73.44970998)
\curveto(862.99868942,73.48970514)(862.96868945,73.52470511)(862.93868408,73.55470998)
\curveto(862.89868952,73.59470504)(862.86868955,73.634705)(862.84868408,73.67470998)
\curveto(862.73868968,73.81470482)(862.61368981,73.93970469)(862.47368408,74.04970998)
\curveto(862.44368998,74.06970456)(862.41869,74.09470454)(862.39868408,74.12470998)
\curveto(862.36869005,74.15470448)(862.33869008,74.17970445)(862.30868408,74.19970998)
\curveto(862.20869021,74.27970435)(862.10869031,74.34470429)(862.00868408,74.39470998)
\curveto(861.90869051,74.45470418)(861.79869062,74.50970412)(861.67868408,74.55970998)
\curveto(861.60869081,74.58970404)(861.53369089,74.60970402)(861.45368408,74.61970998)
\lineto(861.21368408,74.67970998)
\lineto(861.12368408,74.67970998)
\curveto(861.09369133,74.68970394)(861.06369136,74.69470394)(861.03368408,74.69470998)
\curveto(860.96369146,74.71470392)(860.86869155,74.71970391)(860.74868408,74.70970998)
\curveto(860.6186918,74.70970392)(860.5186919,74.69970393)(860.44868408,74.67970998)
\curveto(860.36869205,74.65970397)(860.29369213,74.63970399)(860.22368408,74.61970998)
\curveto(860.14369228,74.60970402)(860.06369236,74.58970404)(859.98368408,74.55970998)
\curveto(859.74369268,74.44970418)(859.54369288,74.29970433)(859.38368408,74.10970998)
\curveto(859.21369321,73.9297047)(859.07369335,73.70970492)(858.96368408,73.44970998)
\curveto(858.94369348,73.37970525)(858.92869349,73.30970532)(858.91868408,73.23970998)
\curveto(858.89869352,73.16970546)(858.87869354,73.09470554)(858.85868408,73.01470998)
\curveto(858.83869358,72.9347057)(858.82869359,72.82470581)(858.82868408,72.68470998)
\curveto(858.82869359,72.55470608)(858.83869358,72.44970618)(858.85868408,72.36970998)
\curveto(858.86869355,72.30970632)(858.87369355,72.25470638)(858.87368408,72.20470998)
\curveto(858.87369355,72.15470648)(858.88369354,72.10470653)(858.90368408,72.05470998)
\curveto(858.94369348,71.95470668)(858.98369344,71.85970677)(859.02368408,71.76970998)
\curveto(859.06369336,71.68970694)(859.10869331,71.60970702)(859.15868408,71.52970998)
\curveto(859.17869324,71.49970713)(859.20369322,71.46970716)(859.23368408,71.43970998)
\curveto(859.26369316,71.41970721)(859.28869313,71.39470724)(859.30868408,71.36470998)
\lineto(859.38368408,71.28970998)
\curveto(859.40369302,71.25970737)(859.423693,71.2347074)(859.44368408,71.21470998)
\lineto(859.65368408,71.06470998)
\curveto(859.71369271,71.02470761)(859.77869264,70.97970765)(859.84868408,70.92970998)
\curveto(859.93869248,70.86970776)(860.04369238,70.81970781)(860.16368408,70.77970998)
\curveto(860.27369215,70.74970788)(860.38369204,70.71470792)(860.49368408,70.67470998)
\curveto(860.60369182,70.634708)(860.74869167,70.60970802)(860.92868408,70.59970998)
\curveto(861.09869132,70.58970804)(861.2236912,70.55970807)(861.30368408,70.50970998)
\curveto(861.38369104,70.45970817)(861.42869099,70.38470825)(861.43868408,70.28470998)
\curveto(861.44869097,70.18470845)(861.45369097,70.07470856)(861.45368408,69.95470998)
\curveto(861.45369097,69.91470872)(861.45869096,69.87470876)(861.46868408,69.83470998)
\curveto(861.46869095,69.79470884)(861.46369096,69.75970887)(861.45368408,69.72970998)
\curveto(861.43369099,69.67970895)(861.423691,69.629709)(861.42368408,69.57970998)
\curveto(861.423691,69.53970909)(861.41369101,69.49970913)(861.39368408,69.45970998)
\curveto(861.33369109,69.36970926)(861.19869122,69.32470931)(860.98868408,69.32470998)
\lineto(860.86868408,69.32470998)
\curveto(860.80869161,69.3347093)(860.74869167,69.33970929)(860.68868408,69.33970998)
\curveto(860.6186918,69.34970928)(860.55369187,69.35970927)(860.49368408,69.36970998)
\curveto(860.38369204,69.38970924)(860.28369214,69.40970922)(860.19368408,69.42970998)
\curveto(860.09369233,69.44970918)(859.99869242,69.47970915)(859.90868408,69.51970998)
\curveto(859.83869258,69.53970909)(859.77869264,69.55970907)(859.72868408,69.57970998)
\lineto(859.54868408,69.63970998)
\curveto(859.28869313,69.75970887)(859.04369338,69.91470872)(858.81368408,70.10470998)
\curveto(858.58369384,70.30470833)(858.39869402,70.51970811)(858.25868408,70.74970998)
\curveto(858.17869424,70.85970777)(858.11369431,70.97470766)(858.06368408,71.09470998)
\lineto(857.91368408,71.48470998)
\curveto(857.86369456,71.59470704)(857.83369459,71.70970692)(857.82368408,71.82970998)
\curveto(857.80369462,71.94970668)(857.77869464,72.07470656)(857.74868408,72.20470998)
\curveto(857.74869467,72.27470636)(857.74869467,72.33970629)(857.74868408,72.39970998)
\curveto(857.73869468,72.45970617)(857.72869469,72.52470611)(857.71868408,72.59470998)
}
}
{
\newrgbcolor{curcolor}{0 0 0}
\pscustom[linestyle=none,fillstyle=solid,fillcolor=curcolor]
{
\newpath
\moveto(866.68868408,78.63431936)
\lineto(866.68868408,79.26431936)
\lineto(866.68868408,79.45931936)
\curveto(866.68868573,79.52931683)(866.69868572,79.58931677)(866.71868408,79.63931936)
\curveto(866.75868566,79.70931665)(866.79868562,79.7593166)(866.83868408,79.78931936)
\curveto(866.88868553,79.82931653)(866.95368547,79.84931651)(867.03368408,79.84931936)
\curveto(867.11368531,79.8593165)(867.19868522,79.86431649)(867.28868408,79.86431936)
\lineto(868.00868408,79.86431936)
\curveto(868.48868393,79.86431649)(868.89868352,79.80431655)(869.23868408,79.68431936)
\curveto(869.57868284,79.56431679)(869.85368257,79.36931699)(870.06368408,79.09931936)
\curveto(870.11368231,79.02931733)(870.15868226,78.9593174)(870.19868408,78.88931936)
\curveto(870.24868217,78.82931753)(870.29368213,78.7543176)(870.33368408,78.66431936)
\curveto(870.34368208,78.64431771)(870.35368207,78.61431774)(870.36368408,78.57431936)
\curveto(870.38368204,78.53431782)(870.38868203,78.48931787)(870.37868408,78.43931936)
\curveto(870.34868207,78.34931801)(870.27368215,78.29431806)(870.15368408,78.27431936)
\curveto(870.04368238,78.2543181)(869.94868247,78.26931809)(869.86868408,78.31931936)
\curveto(869.79868262,78.34931801)(869.73368269,78.39431796)(869.67368408,78.45431936)
\curveto(869.6236828,78.52431783)(869.57368285,78.58931777)(869.52368408,78.64931936)
\curveto(869.47368295,78.71931764)(869.39868302,78.77931758)(869.29868408,78.82931936)
\curveto(869.20868321,78.88931747)(869.1186833,78.93931742)(869.02868408,78.97931936)
\curveto(868.99868342,78.99931736)(868.93868348,79.02431733)(868.84868408,79.05431936)
\curveto(868.76868365,79.08431727)(868.69868372,79.08931727)(868.63868408,79.06931936)
\curveto(868.49868392,79.03931732)(868.40868401,78.97931738)(868.36868408,78.88931936)
\curveto(868.33868408,78.80931755)(868.3236841,78.71931764)(868.32368408,78.61931936)
\curveto(868.3236841,78.51931784)(868.29868412,78.43431792)(868.24868408,78.36431936)
\curveto(868.17868424,78.27431808)(868.03868438,78.22931813)(867.82868408,78.22931936)
\lineto(867.27368408,78.22931936)
\lineto(867.04868408,78.22931936)
\curveto(866.96868545,78.23931812)(866.90368552,78.2593181)(866.85368408,78.28931936)
\curveto(866.77368565,78.34931801)(866.72868569,78.41931794)(866.71868408,78.49931936)
\curveto(866.70868571,78.51931784)(866.70368572,78.53931782)(866.70368408,78.55931936)
\curveto(866.70368572,78.58931777)(866.69868572,78.61431774)(866.68868408,78.63431936)
}
}
{
\newrgbcolor{curcolor}{0 0 0}
\pscustom[linestyle=none,fillstyle=solid,fillcolor=curcolor]
{
}
}
{
\newrgbcolor{curcolor}{0 0 0}
\pscustom[linestyle=none,fillstyle=solid,fillcolor=curcolor]
{
\newpath
\moveto(857.71868408,89.26463186)
\curveto(857.70869471,89.95462722)(857.82869459,90.55462662)(858.07868408,91.06463186)
\curveto(858.32869409,91.58462559)(858.66369376,91.9796252)(859.08368408,92.24963186)
\curveto(859.16369326,92.29962488)(859.25369317,92.34462483)(859.35368408,92.38463186)
\curveto(859.44369298,92.42462475)(859.53869288,92.46962471)(859.63868408,92.51963186)
\curveto(859.73869268,92.55962462)(859.83869258,92.58962459)(859.93868408,92.60963186)
\curveto(860.03869238,92.62962455)(860.14369228,92.64962453)(860.25368408,92.66963186)
\curveto(860.30369212,92.68962449)(860.34869207,92.69462448)(860.38868408,92.68463186)
\curveto(860.42869199,92.6746245)(860.47369195,92.6796245)(860.52368408,92.69963186)
\curveto(860.57369185,92.70962447)(860.65869176,92.71462446)(860.77868408,92.71463186)
\curveto(860.88869153,92.71462446)(860.97369145,92.70962447)(861.03368408,92.69963186)
\curveto(861.09369133,92.6796245)(861.15369127,92.66962451)(861.21368408,92.66963186)
\curveto(861.27369115,92.6796245)(861.33369109,92.6746245)(861.39368408,92.65463186)
\curveto(861.53369089,92.61462456)(861.66869075,92.5796246)(861.79868408,92.54963186)
\curveto(861.92869049,92.51962466)(862.05369037,92.4796247)(862.17368408,92.42963186)
\curveto(862.31369011,92.36962481)(862.43868998,92.29962488)(862.54868408,92.21963186)
\curveto(862.65868976,92.14962503)(862.76868965,92.0746251)(862.87868408,91.99463186)
\lineto(862.93868408,91.93463186)
\curveto(862.95868946,91.92462525)(862.97868944,91.90962527)(862.99868408,91.88963186)
\curveto(863.15868926,91.76962541)(863.30368912,91.63462554)(863.43368408,91.48463186)
\curveto(863.56368886,91.33462584)(863.68868873,91.174626)(863.80868408,91.00463186)
\curveto(864.02868839,90.69462648)(864.23368819,90.39962678)(864.42368408,90.11963186)
\curveto(864.56368786,89.88962729)(864.69868772,89.65962752)(864.82868408,89.42963186)
\curveto(864.95868746,89.20962797)(865.09368733,88.98962819)(865.23368408,88.76963186)
\curveto(865.40368702,88.51962866)(865.58368684,88.2796289)(865.77368408,88.04963186)
\curveto(865.96368646,87.82962935)(866.18868623,87.63962954)(866.44868408,87.47963186)
\curveto(866.50868591,87.43962974)(866.56868585,87.40462977)(866.62868408,87.37463186)
\curveto(866.67868574,87.34462983)(866.74368568,87.31462986)(866.82368408,87.28463186)
\curveto(866.89368553,87.26462991)(866.95368547,87.25962992)(867.00368408,87.26963186)
\curveto(867.07368535,87.28962989)(867.12868529,87.32462985)(867.16868408,87.37463186)
\curveto(867.19868522,87.42462975)(867.2186852,87.48462969)(867.22868408,87.55463186)
\lineto(867.22868408,87.79463186)
\lineto(867.22868408,88.54463186)
\lineto(867.22868408,91.34963186)
\lineto(867.22868408,92.00963186)
\curveto(867.22868519,92.09962508)(867.23368519,92.18462499)(867.24368408,92.26463186)
\curveto(867.24368518,92.34462483)(867.26368516,92.40962477)(867.30368408,92.45963186)
\curveto(867.34368508,92.50962467)(867.418685,92.54962463)(867.52868408,92.57963186)
\curveto(867.62868479,92.61962456)(867.72868469,92.62962455)(867.82868408,92.60963186)
\lineto(867.96368408,92.60963186)
\curveto(868.03368439,92.58962459)(868.09368433,92.56962461)(868.14368408,92.54963186)
\curveto(868.19368423,92.52962465)(868.23368419,92.49462468)(868.26368408,92.44463186)
\curveto(868.30368412,92.39462478)(868.3236841,92.32462485)(868.32368408,92.23463186)
\lineto(868.32368408,91.96463186)
\lineto(868.32368408,91.06463186)
\lineto(868.32368408,87.55463186)
\lineto(868.32368408,86.48963186)
\curveto(868.3236841,86.40963077)(868.32868409,86.31963086)(868.33868408,86.21963186)
\curveto(868.33868408,86.11963106)(868.32868409,86.03463114)(868.30868408,85.96463186)
\curveto(868.23868418,85.75463142)(868.05868436,85.68963149)(867.76868408,85.76963186)
\curveto(867.72868469,85.7796314)(867.69368473,85.7796314)(867.66368408,85.76963186)
\curveto(867.6236848,85.76963141)(867.57868484,85.7796314)(867.52868408,85.79963186)
\curveto(867.44868497,85.81963136)(867.36368506,85.83963134)(867.27368408,85.85963186)
\curveto(867.18368524,85.8796313)(867.09868532,85.90463127)(867.01868408,85.93463186)
\curveto(866.52868589,86.09463108)(866.11368631,86.29463088)(865.77368408,86.53463186)
\curveto(865.5236869,86.71463046)(865.29868712,86.91963026)(865.09868408,87.14963186)
\curveto(864.88868753,87.3796298)(864.69368773,87.61962956)(864.51368408,87.86963186)
\curveto(864.33368809,88.12962905)(864.16368826,88.39462878)(864.00368408,88.66463186)
\curveto(863.83368859,88.94462823)(863.65868876,89.21462796)(863.47868408,89.47463186)
\curveto(863.39868902,89.58462759)(863.3236891,89.68962749)(863.25368408,89.78963186)
\curveto(863.18368924,89.89962728)(863.10868931,90.00962717)(863.02868408,90.11963186)
\curveto(862.99868942,90.15962702)(862.96868945,90.19462698)(862.93868408,90.22463186)
\curveto(862.89868952,90.26462691)(862.86868955,90.30462687)(862.84868408,90.34463186)
\curveto(862.73868968,90.48462669)(862.61368981,90.60962657)(862.47368408,90.71963186)
\curveto(862.44368998,90.73962644)(862.41869,90.76462641)(862.39868408,90.79463186)
\curveto(862.36869005,90.82462635)(862.33869008,90.84962633)(862.30868408,90.86963186)
\curveto(862.20869021,90.94962623)(862.10869031,91.01462616)(862.00868408,91.06463186)
\curveto(861.90869051,91.12462605)(861.79869062,91.179626)(861.67868408,91.22963186)
\curveto(861.60869081,91.25962592)(861.53369089,91.2796259)(861.45368408,91.28963186)
\lineto(861.21368408,91.34963186)
\lineto(861.12368408,91.34963186)
\curveto(861.09369133,91.35962582)(861.06369136,91.36462581)(861.03368408,91.36463186)
\curveto(860.96369146,91.38462579)(860.86869155,91.38962579)(860.74868408,91.37963186)
\curveto(860.6186918,91.3796258)(860.5186919,91.36962581)(860.44868408,91.34963186)
\curveto(860.36869205,91.32962585)(860.29369213,91.30962587)(860.22368408,91.28963186)
\curveto(860.14369228,91.2796259)(860.06369236,91.25962592)(859.98368408,91.22963186)
\curveto(859.74369268,91.11962606)(859.54369288,90.96962621)(859.38368408,90.77963186)
\curveto(859.21369321,90.59962658)(859.07369335,90.3796268)(858.96368408,90.11963186)
\curveto(858.94369348,90.04962713)(858.92869349,89.9796272)(858.91868408,89.90963186)
\curveto(858.89869352,89.83962734)(858.87869354,89.76462741)(858.85868408,89.68463186)
\curveto(858.83869358,89.60462757)(858.82869359,89.49462768)(858.82868408,89.35463186)
\curveto(858.82869359,89.22462795)(858.83869358,89.11962806)(858.85868408,89.03963186)
\curveto(858.86869355,88.9796282)(858.87369355,88.92462825)(858.87368408,88.87463186)
\curveto(858.87369355,88.82462835)(858.88369354,88.7746284)(858.90368408,88.72463186)
\curveto(858.94369348,88.62462855)(858.98369344,88.52962865)(859.02368408,88.43963186)
\curveto(859.06369336,88.35962882)(859.10869331,88.2796289)(859.15868408,88.19963186)
\curveto(859.17869324,88.16962901)(859.20369322,88.13962904)(859.23368408,88.10963186)
\curveto(859.26369316,88.08962909)(859.28869313,88.06462911)(859.30868408,88.03463186)
\lineto(859.38368408,87.95963186)
\curveto(859.40369302,87.92962925)(859.423693,87.90462927)(859.44368408,87.88463186)
\lineto(859.65368408,87.73463186)
\curveto(859.71369271,87.69462948)(859.77869264,87.64962953)(859.84868408,87.59963186)
\curveto(859.93869248,87.53962964)(860.04369238,87.48962969)(860.16368408,87.44963186)
\curveto(860.27369215,87.41962976)(860.38369204,87.38462979)(860.49368408,87.34463186)
\curveto(860.60369182,87.30462987)(860.74869167,87.2796299)(860.92868408,87.26963186)
\curveto(861.09869132,87.25962992)(861.2236912,87.22962995)(861.30368408,87.17963186)
\curveto(861.38369104,87.12963005)(861.42869099,87.05463012)(861.43868408,86.95463186)
\curveto(861.44869097,86.85463032)(861.45369097,86.74463043)(861.45368408,86.62463186)
\curveto(861.45369097,86.58463059)(861.45869096,86.54463063)(861.46868408,86.50463186)
\curveto(861.46869095,86.46463071)(861.46369096,86.42963075)(861.45368408,86.39963186)
\curveto(861.43369099,86.34963083)(861.423691,86.29963088)(861.42368408,86.24963186)
\curveto(861.423691,86.20963097)(861.41369101,86.16963101)(861.39368408,86.12963186)
\curveto(861.33369109,86.03963114)(861.19869122,85.99463118)(860.98868408,85.99463186)
\lineto(860.86868408,85.99463186)
\curveto(860.80869161,86.00463117)(860.74869167,86.00963117)(860.68868408,86.00963186)
\curveto(860.6186918,86.01963116)(860.55369187,86.02963115)(860.49368408,86.03963186)
\curveto(860.38369204,86.05963112)(860.28369214,86.0796311)(860.19368408,86.09963186)
\curveto(860.09369233,86.11963106)(859.99869242,86.14963103)(859.90868408,86.18963186)
\curveto(859.83869258,86.20963097)(859.77869264,86.22963095)(859.72868408,86.24963186)
\lineto(859.54868408,86.30963186)
\curveto(859.28869313,86.42963075)(859.04369338,86.58463059)(858.81368408,86.77463186)
\curveto(858.58369384,86.9746302)(858.39869402,87.18962999)(858.25868408,87.41963186)
\curveto(858.17869424,87.52962965)(858.11369431,87.64462953)(858.06368408,87.76463186)
\lineto(857.91368408,88.15463186)
\curveto(857.86369456,88.26462891)(857.83369459,88.3796288)(857.82368408,88.49963186)
\curveto(857.80369462,88.61962856)(857.77869464,88.74462843)(857.74868408,88.87463186)
\curveto(857.74869467,88.94462823)(857.74869467,89.00962817)(857.74868408,89.06963186)
\curveto(857.73869468,89.12962805)(857.72869469,89.19462798)(857.71868408,89.26463186)
}
}
{
\newrgbcolor{curcolor}{0 0 0}
\pscustom[linestyle=none,fillstyle=solid,fillcolor=curcolor]
{
\newpath
\moveto(863.23868408,101.36424123)
\lineto(863.49368408,101.36424123)
\curveto(863.57368885,101.37423353)(863.64868877,101.36923353)(863.71868408,101.34924123)
\lineto(863.95868408,101.34924123)
\lineto(864.12368408,101.34924123)
\curveto(864.2236882,101.32923357)(864.32868809,101.31923358)(864.43868408,101.31924123)
\curveto(864.53868788,101.31923358)(864.63868778,101.30923359)(864.73868408,101.28924123)
\lineto(864.88868408,101.28924123)
\curveto(865.02868739,101.25923364)(865.16868725,101.23923366)(865.30868408,101.22924123)
\curveto(865.43868698,101.21923368)(865.56868685,101.19423371)(865.69868408,101.15424123)
\curveto(865.77868664,101.13423377)(865.86368656,101.11423379)(865.95368408,101.09424123)
\lineto(866.19368408,101.03424123)
\lineto(866.49368408,100.91424123)
\curveto(866.58368584,100.88423402)(866.67368575,100.84923405)(866.76368408,100.80924123)
\curveto(866.98368544,100.70923419)(867.19868522,100.57423433)(867.40868408,100.40424123)
\curveto(867.6186848,100.24423466)(867.78868463,100.06923483)(867.91868408,99.87924123)
\curveto(867.95868446,99.82923507)(867.99868442,99.76923513)(868.03868408,99.69924123)
\curveto(868.06868435,99.63923526)(868.10368432,99.57923532)(868.14368408,99.51924123)
\curveto(868.19368423,99.43923546)(868.23368419,99.34423556)(868.26368408,99.23424123)
\curveto(868.29368413,99.12423578)(868.3236841,99.01923588)(868.35368408,98.91924123)
\curveto(868.39368403,98.80923609)(868.418684,98.6992362)(868.42868408,98.58924123)
\curveto(868.43868398,98.47923642)(868.45368397,98.36423654)(868.47368408,98.24424123)
\curveto(868.48368394,98.2042367)(868.48368394,98.15923674)(868.47368408,98.10924123)
\curveto(868.47368395,98.06923683)(868.47868394,98.02923687)(868.48868408,97.98924123)
\curveto(868.49868392,97.94923695)(868.50368392,97.89423701)(868.50368408,97.82424123)
\curveto(868.50368392,97.75423715)(868.49868392,97.7042372)(868.48868408,97.67424123)
\curveto(868.46868395,97.62423728)(868.46368396,97.57923732)(868.47368408,97.53924123)
\curveto(868.48368394,97.4992374)(868.48368394,97.46423744)(868.47368408,97.43424123)
\lineto(868.47368408,97.34424123)
\curveto(868.45368397,97.28423762)(868.43868398,97.21923768)(868.42868408,97.14924123)
\curveto(868.42868399,97.08923781)(868.423684,97.02423788)(868.41368408,96.95424123)
\curveto(868.36368406,96.78423812)(868.31368411,96.62423828)(868.26368408,96.47424123)
\curveto(868.21368421,96.32423858)(868.14868427,96.17923872)(868.06868408,96.03924123)
\curveto(868.02868439,95.98923891)(867.99868442,95.93423897)(867.97868408,95.87424123)
\curveto(867.94868447,95.82423908)(867.91368451,95.77423913)(867.87368408,95.72424123)
\curveto(867.69368473,95.48423942)(867.47368495,95.28423962)(867.21368408,95.12424123)
\curveto(866.95368547,94.96423994)(866.66868575,94.82424008)(866.35868408,94.70424123)
\curveto(866.2186862,94.64424026)(866.07868634,94.5992403)(865.93868408,94.56924123)
\curveto(865.78868663,94.53924036)(865.63368679,94.5042404)(865.47368408,94.46424123)
\curveto(865.36368706,94.44424046)(865.25368717,94.42924047)(865.14368408,94.41924123)
\curveto(865.03368739,94.40924049)(864.9236875,94.39424051)(864.81368408,94.37424123)
\curveto(864.77368765,94.36424054)(864.73368769,94.35924054)(864.69368408,94.35924123)
\curveto(864.65368777,94.36924053)(864.61368781,94.36924053)(864.57368408,94.35924123)
\curveto(864.5236879,94.34924055)(864.47368795,94.34424056)(864.42368408,94.34424123)
\lineto(864.25868408,94.34424123)
\curveto(864.20868821,94.32424058)(864.15868826,94.31924058)(864.10868408,94.32924123)
\curveto(864.04868837,94.33924056)(863.99368843,94.33924056)(863.94368408,94.32924123)
\curveto(863.90368852,94.31924058)(863.85868856,94.31924058)(863.80868408,94.32924123)
\curveto(863.75868866,94.33924056)(863.70868871,94.33424057)(863.65868408,94.31424123)
\curveto(863.58868883,94.29424061)(863.51368891,94.28924061)(863.43368408,94.29924123)
\curveto(863.34368908,94.30924059)(863.25868916,94.31424059)(863.17868408,94.31424123)
\curveto(863.08868933,94.31424059)(862.98868943,94.30924059)(862.87868408,94.29924123)
\curveto(862.75868966,94.28924061)(862.65868976,94.29424061)(862.57868408,94.31424123)
\lineto(862.29368408,94.31424123)
\lineto(861.66368408,94.35924123)
\curveto(861.56369086,94.36924053)(861.46869095,94.37924052)(861.37868408,94.38924123)
\lineto(861.07868408,94.41924123)
\curveto(861.02869139,94.43924046)(860.97869144,94.44424046)(860.92868408,94.43424123)
\curveto(860.86869155,94.43424047)(860.81369161,94.44424046)(860.76368408,94.46424123)
\curveto(860.59369183,94.51424039)(860.42869199,94.55424035)(860.26868408,94.58424123)
\curveto(860.09869232,94.61424029)(859.93869248,94.66424024)(859.78868408,94.73424123)
\curveto(859.32869309,94.92423998)(858.95369347,95.14423976)(858.66368408,95.39424123)
\curveto(858.37369405,95.65423925)(858.12869429,96.01423889)(857.92868408,96.47424123)
\curveto(857.87869454,96.6042383)(857.84369458,96.73423817)(857.82368408,96.86424123)
\curveto(857.80369462,97.0042379)(857.77869464,97.14423776)(857.74868408,97.28424123)
\curveto(857.73869468,97.35423755)(857.73369469,97.41923748)(857.73368408,97.47924123)
\curveto(857.73369469,97.53923736)(857.72869469,97.6042373)(857.71868408,97.67424123)
\curveto(857.69869472,98.5042364)(857.84869457,99.17423573)(858.16868408,99.68424123)
\curveto(858.47869394,100.19423471)(858.9186935,100.57423433)(859.48868408,100.82424123)
\curveto(859.60869281,100.87423403)(859.73369269,100.91923398)(859.86368408,100.95924123)
\curveto(859.99369243,100.9992339)(860.12869229,101.04423386)(860.26868408,101.09424123)
\curveto(860.34869207,101.11423379)(860.43369199,101.12923377)(860.52368408,101.13924123)
\lineto(860.76368408,101.19924123)
\curveto(860.87369155,101.22923367)(860.98369144,101.24423366)(861.09368408,101.24424123)
\curveto(861.20369122,101.25423365)(861.31369111,101.26923363)(861.42368408,101.28924123)
\curveto(861.47369095,101.30923359)(861.5186909,101.31423359)(861.55868408,101.30424123)
\curveto(861.59869082,101.3042336)(861.63869078,101.30923359)(861.67868408,101.31924123)
\curveto(861.72869069,101.32923357)(861.78369064,101.32923357)(861.84368408,101.31924123)
\curveto(861.89369053,101.31923358)(861.94369048,101.32423358)(861.99368408,101.33424123)
\lineto(862.12868408,101.33424123)
\curveto(862.18869023,101.35423355)(862.25869016,101.35423355)(862.33868408,101.33424123)
\curveto(862.40869001,101.32423358)(862.47368995,101.32923357)(862.53368408,101.34924123)
\curveto(862.56368986,101.35923354)(862.60368982,101.36423354)(862.65368408,101.36424123)
\lineto(862.77368408,101.36424123)
\lineto(863.23868408,101.36424123)
\moveto(865.56368408,99.81924123)
\curveto(865.24368718,99.91923498)(864.87868754,99.97923492)(864.46868408,99.99924123)
\curveto(864.05868836,100.01923488)(863.64868877,100.02923487)(863.23868408,100.02924123)
\curveto(862.80868961,100.02923487)(862.38869003,100.01923488)(861.97868408,99.99924123)
\curveto(861.56869085,99.97923492)(861.18369124,99.93423497)(860.82368408,99.86424123)
\curveto(860.46369196,99.79423511)(860.14369228,99.68423522)(859.86368408,99.53424123)
\curveto(859.57369285,99.39423551)(859.33869308,99.1992357)(859.15868408,98.94924123)
\curveto(859.04869337,98.78923611)(858.96869345,98.60923629)(858.91868408,98.40924123)
\curveto(858.85869356,98.20923669)(858.82869359,97.96423694)(858.82868408,97.67424123)
\curveto(858.84869357,97.65423725)(858.85869356,97.61923728)(858.85868408,97.56924123)
\curveto(858.84869357,97.51923738)(858.84869357,97.47923742)(858.85868408,97.44924123)
\curveto(858.87869354,97.36923753)(858.89869352,97.29423761)(858.91868408,97.22424123)
\curveto(858.92869349,97.16423774)(858.94869347,97.0992378)(858.97868408,97.02924123)
\curveto(859.09869332,96.75923814)(859.26869315,96.53923836)(859.48868408,96.36924123)
\curveto(859.69869272,96.20923869)(859.94369248,96.07423883)(860.22368408,95.96424123)
\curveto(860.33369209,95.91423899)(860.45369197,95.87423903)(860.58368408,95.84424123)
\curveto(860.70369172,95.82423908)(860.82869159,95.7992391)(860.95868408,95.76924123)
\curveto(861.00869141,95.74923915)(861.06369136,95.73923916)(861.12368408,95.73924123)
\curveto(861.17369125,95.73923916)(861.2236912,95.73423917)(861.27368408,95.72424123)
\curveto(861.36369106,95.71423919)(861.45869096,95.7042392)(861.55868408,95.69424123)
\curveto(861.64869077,95.68423922)(861.74369068,95.67423923)(861.84368408,95.66424123)
\curveto(861.9236905,95.66423924)(862.00869041,95.65923924)(862.09868408,95.64924123)
\lineto(862.33868408,95.64924123)
\lineto(862.51868408,95.64924123)
\curveto(862.54868987,95.63923926)(862.58368984,95.63423927)(862.62368408,95.63424123)
\lineto(862.75868408,95.63424123)
\lineto(863.20868408,95.63424123)
\curveto(863.28868913,95.63423927)(863.37368905,95.62923927)(863.46368408,95.61924123)
\curveto(863.54368888,95.61923928)(863.6186888,95.62923927)(863.68868408,95.64924123)
\lineto(863.95868408,95.64924123)
\curveto(863.97868844,95.64923925)(864.00868841,95.64423926)(864.04868408,95.63424123)
\curveto(864.07868834,95.63423927)(864.10368832,95.63923926)(864.12368408,95.64924123)
\curveto(864.2236882,95.65923924)(864.3236881,95.66423924)(864.42368408,95.66424123)
\curveto(864.51368791,95.67423923)(864.61368781,95.68423922)(864.72368408,95.69424123)
\curveto(864.84368758,95.72423918)(864.96868745,95.73923916)(865.09868408,95.73924123)
\curveto(865.2186872,95.74923915)(865.33368709,95.77423913)(865.44368408,95.81424123)
\curveto(865.74368668,95.89423901)(866.00868641,95.97923892)(866.23868408,96.06924123)
\curveto(866.46868595,96.16923873)(866.68368574,96.31423859)(866.88368408,96.50424123)
\curveto(867.08368534,96.71423819)(867.23368519,96.97923792)(867.33368408,97.29924123)
\curveto(867.35368507,97.33923756)(867.36368506,97.37423753)(867.36368408,97.40424123)
\curveto(867.35368507,97.44423746)(867.35868506,97.48923741)(867.37868408,97.53924123)
\curveto(867.38868503,97.57923732)(867.39868502,97.64923725)(867.40868408,97.74924123)
\curveto(867.418685,97.85923704)(867.41368501,97.94423696)(867.39368408,98.00424123)
\curveto(867.37368505,98.07423683)(867.36368506,98.14423676)(867.36368408,98.21424123)
\curveto(867.35368507,98.28423662)(867.33868508,98.34923655)(867.31868408,98.40924123)
\curveto(867.25868516,98.60923629)(867.17368525,98.78923611)(867.06368408,98.94924123)
\curveto(867.04368538,98.97923592)(867.0236854,99.0042359)(867.00368408,99.02424123)
\lineto(866.94368408,99.08424123)
\curveto(866.9236855,99.12423578)(866.88368554,99.17423573)(866.82368408,99.23424123)
\curveto(866.68368574,99.33423557)(866.55368587,99.41923548)(866.43368408,99.48924123)
\curveto(866.31368611,99.55923534)(866.16868625,99.62923527)(865.99868408,99.69924123)
\curveto(865.92868649,99.72923517)(865.85868656,99.74923515)(865.78868408,99.75924123)
\curveto(865.7186867,99.77923512)(865.64368678,99.7992351)(865.56368408,99.81924123)
}
}
{
\newrgbcolor{curcolor}{0 0 0}
\pscustom[linestyle=none,fillstyle=solid,fillcolor=curcolor]
{
\newpath
\moveto(857.71868408,106.77385061)
\curveto(857.7186947,106.87384575)(857.72869469,106.96884566)(857.74868408,107.05885061)
\curveto(857.75869466,107.14884548)(857.78869463,107.21384541)(857.83868408,107.25385061)
\curveto(857.9186945,107.31384531)(858.0236944,107.34384528)(858.15368408,107.34385061)
\lineto(858.54368408,107.34385061)
\lineto(860.04368408,107.34385061)
\lineto(866.43368408,107.34385061)
\lineto(867.60368408,107.34385061)
\lineto(867.91868408,107.34385061)
\curveto(868.0186844,107.35384527)(868.09868432,107.33884529)(868.15868408,107.29885061)
\curveto(868.23868418,107.24884538)(868.28868413,107.17384545)(868.30868408,107.07385061)
\curveto(868.3186841,106.98384564)(868.3236841,106.87384575)(868.32368408,106.74385061)
\lineto(868.32368408,106.51885061)
\curveto(868.30368412,106.43884619)(868.28868413,106.36884626)(868.27868408,106.30885061)
\curveto(868.25868416,106.24884638)(868.2186842,106.19884643)(868.15868408,106.15885061)
\curveto(868.09868432,106.11884651)(868.0236844,106.09884653)(867.93368408,106.09885061)
\lineto(867.63368408,106.09885061)
\lineto(866.53868408,106.09885061)
\lineto(861.19868408,106.09885061)
\curveto(861.10869131,106.07884655)(861.03369139,106.06384656)(860.97368408,106.05385061)
\curveto(860.90369152,106.05384657)(860.84369158,106.0238466)(860.79368408,105.96385061)
\curveto(860.74369168,105.89384673)(860.7186917,105.80384682)(860.71868408,105.69385061)
\curveto(860.70869171,105.59384703)(860.70369172,105.48384714)(860.70368408,105.36385061)
\lineto(860.70368408,104.22385061)
\lineto(860.70368408,103.72885061)
\curveto(860.69369173,103.56884906)(860.63369179,103.45884917)(860.52368408,103.39885061)
\curveto(860.49369193,103.37884925)(860.46369196,103.36884926)(860.43368408,103.36885061)
\curveto(860.39369203,103.36884926)(860.34869207,103.36384926)(860.29868408,103.35385061)
\curveto(860.17869224,103.33384929)(860.06869235,103.33884929)(859.96868408,103.36885061)
\curveto(859.86869255,103.40884922)(859.79869262,103.46384916)(859.75868408,103.53385061)
\curveto(859.70869271,103.61384901)(859.68369274,103.73384889)(859.68368408,103.89385061)
\curveto(859.68369274,104.05384857)(859.66869275,104.18884844)(859.63868408,104.29885061)
\curveto(859.62869279,104.34884828)(859.6236928,104.40384822)(859.62368408,104.46385061)
\curveto(859.61369281,104.5238481)(859.59869282,104.58384804)(859.57868408,104.64385061)
\curveto(859.52869289,104.79384783)(859.47869294,104.93884769)(859.42868408,105.07885061)
\curveto(859.36869305,105.21884741)(859.29869312,105.35384727)(859.21868408,105.48385061)
\curveto(859.12869329,105.623847)(859.0236934,105.74384688)(858.90368408,105.84385061)
\curveto(858.78369364,105.94384668)(858.65369377,106.03884659)(858.51368408,106.12885061)
\curveto(858.41369401,106.18884644)(858.30369412,106.23384639)(858.18368408,106.26385061)
\curveto(858.06369436,106.30384632)(857.95869446,106.35384627)(857.86868408,106.41385061)
\curveto(857.80869461,106.46384616)(857.76869465,106.53384609)(857.74868408,106.62385061)
\curveto(857.73869468,106.64384598)(857.73369469,106.66884596)(857.73368408,106.69885061)
\curveto(857.73369469,106.7288459)(857.72869469,106.75384587)(857.71868408,106.77385061)
}
}
{
\newrgbcolor{curcolor}{0 0 0}
\pscustom[linestyle=none,fillstyle=solid,fillcolor=curcolor]
{
\newpath
\moveto(857.71868408,115.12345998)
\curveto(857.7186947,115.22345513)(857.72869469,115.31845503)(857.74868408,115.40845998)
\curveto(857.75869466,115.49845485)(857.78869463,115.56345479)(857.83868408,115.60345998)
\curveto(857.9186945,115.66345469)(858.0236944,115.69345466)(858.15368408,115.69345998)
\lineto(858.54368408,115.69345998)
\lineto(860.04368408,115.69345998)
\lineto(866.43368408,115.69345998)
\lineto(867.60368408,115.69345998)
\lineto(867.91868408,115.69345998)
\curveto(868.0186844,115.70345465)(868.09868432,115.68845466)(868.15868408,115.64845998)
\curveto(868.23868418,115.59845475)(868.28868413,115.52345483)(868.30868408,115.42345998)
\curveto(868.3186841,115.33345502)(868.3236841,115.22345513)(868.32368408,115.09345998)
\lineto(868.32368408,114.86845998)
\curveto(868.30368412,114.78845556)(868.28868413,114.71845563)(868.27868408,114.65845998)
\curveto(868.25868416,114.59845575)(868.2186842,114.5484558)(868.15868408,114.50845998)
\curveto(868.09868432,114.46845588)(868.0236844,114.4484559)(867.93368408,114.44845998)
\lineto(867.63368408,114.44845998)
\lineto(866.53868408,114.44845998)
\lineto(861.19868408,114.44845998)
\curveto(861.10869131,114.42845592)(861.03369139,114.41345594)(860.97368408,114.40345998)
\curveto(860.90369152,114.40345595)(860.84369158,114.37345598)(860.79368408,114.31345998)
\curveto(860.74369168,114.24345611)(860.7186917,114.1534562)(860.71868408,114.04345998)
\curveto(860.70869171,113.94345641)(860.70369172,113.83345652)(860.70368408,113.71345998)
\lineto(860.70368408,112.57345998)
\lineto(860.70368408,112.07845998)
\curveto(860.69369173,111.91845843)(860.63369179,111.80845854)(860.52368408,111.74845998)
\curveto(860.49369193,111.72845862)(860.46369196,111.71845863)(860.43368408,111.71845998)
\curveto(860.39369203,111.71845863)(860.34869207,111.71345864)(860.29868408,111.70345998)
\curveto(860.17869224,111.68345867)(860.06869235,111.68845866)(859.96868408,111.71845998)
\curveto(859.86869255,111.75845859)(859.79869262,111.81345854)(859.75868408,111.88345998)
\curveto(859.70869271,111.96345839)(859.68369274,112.08345827)(859.68368408,112.24345998)
\curveto(859.68369274,112.40345795)(859.66869275,112.53845781)(859.63868408,112.64845998)
\curveto(859.62869279,112.69845765)(859.6236928,112.7534576)(859.62368408,112.81345998)
\curveto(859.61369281,112.87345748)(859.59869282,112.93345742)(859.57868408,112.99345998)
\curveto(859.52869289,113.14345721)(859.47869294,113.28845706)(859.42868408,113.42845998)
\curveto(859.36869305,113.56845678)(859.29869312,113.70345665)(859.21868408,113.83345998)
\curveto(859.12869329,113.97345638)(859.0236934,114.09345626)(858.90368408,114.19345998)
\curveto(858.78369364,114.29345606)(858.65369377,114.38845596)(858.51368408,114.47845998)
\curveto(858.41369401,114.53845581)(858.30369412,114.58345577)(858.18368408,114.61345998)
\curveto(858.06369436,114.6534557)(857.95869446,114.70345565)(857.86868408,114.76345998)
\curveto(857.80869461,114.81345554)(857.76869465,114.88345547)(857.74868408,114.97345998)
\curveto(857.73869468,114.99345536)(857.73369469,115.01845533)(857.73368408,115.04845998)
\curveto(857.73369469,115.07845527)(857.72869469,115.10345525)(857.71868408,115.12345998)
}
}
{
\newrgbcolor{curcolor}{0 0 0}
\pscustom[linestyle=none,fillstyle=solid,fillcolor=curcolor]
{
\newpath
\moveto(878.555,42.29681936)
\curveto(878.55501069,42.36681368)(878.55501069,42.4468136)(878.555,42.53681936)
\curveto(878.5450107,42.62681342)(878.5450107,42.71181333)(878.555,42.79181936)
\curveto(878.55501069,42.88181316)(878.56501068,42.96181308)(878.585,43.03181936)
\curveto(878.60501064,43.11181293)(878.63501062,43.16681288)(878.675,43.19681936)
\curveto(878.72501053,43.22681282)(878.80001045,43.2468128)(878.9,43.25681936)
\curveto(878.99001026,43.27681277)(879.09501015,43.28681276)(879.215,43.28681936)
\curveto(879.32500993,43.29681275)(879.44000981,43.29681275)(879.56,43.28681936)
\lineto(879.86,43.28681936)
\lineto(882.875,43.28681936)
\lineto(885.77,43.28681936)
\curveto(886.10000315,43.28681276)(886.42500282,43.28181276)(886.745,43.27181936)
\curveto(887.0550022,43.27181277)(887.33500192,43.23181281)(887.585,43.15181936)
\curveto(887.93500132,43.03181301)(888.23000102,42.87681317)(888.47,42.68681936)
\curveto(888.70000055,42.49681355)(888.90000035,42.25681379)(889.07,41.96681936)
\curveto(889.12000013,41.90681414)(889.15500009,41.8418142)(889.175,41.77181936)
\curveto(889.19500006,41.71181433)(889.22000003,41.6418144)(889.25,41.56181936)
\curveto(889.29999995,41.4418146)(889.33499992,41.31181473)(889.355,41.17181936)
\curveto(889.38499987,41.041815)(889.41499984,40.90681514)(889.445,40.76681936)
\curveto(889.46499979,40.71681533)(889.46999978,40.66681538)(889.46,40.61681936)
\curveto(889.4499998,40.56681548)(889.4499998,40.51181553)(889.46,40.45181936)
\curveto(889.46999978,40.43181561)(889.46999978,40.40681564)(889.46,40.37681936)
\curveto(889.45999979,40.3468157)(889.46499979,40.32181572)(889.475,40.30181936)
\curveto(889.48499976,40.26181578)(889.48999976,40.20681584)(889.49,40.13681936)
\curveto(889.48999976,40.06681598)(889.48499976,40.01181603)(889.475,39.97181936)
\curveto(889.46499979,39.92181612)(889.46499979,39.86681618)(889.475,39.80681936)
\curveto(889.48499976,39.7468163)(889.47999977,39.69181635)(889.46,39.64181936)
\curveto(889.42999982,39.51181653)(889.40999984,39.38681666)(889.4,39.26681936)
\curveto(889.38999986,39.1468169)(889.36499988,39.03181701)(889.325,38.92181936)
\curveto(889.20500005,38.55181749)(889.03500021,38.23181781)(888.815,37.96181936)
\curveto(888.59500066,37.69181835)(888.31500093,37.48181856)(887.975,37.33181936)
\curveto(887.8550014,37.28181876)(887.73000152,37.23681881)(887.6,37.19681936)
\curveto(887.47000178,37.16681888)(887.33500192,37.13181891)(887.195,37.09181936)
\curveto(887.14500211,37.08181896)(887.10500214,37.07681897)(887.075,37.07681936)
\curveto(887.03500221,37.07681897)(886.99000226,37.07181897)(886.94,37.06181936)
\curveto(886.91000234,37.05181899)(886.87500238,37.046819)(886.835,37.04681936)
\curveto(886.78500247,37.046819)(886.74500251,37.041819)(886.715,37.03181936)
\lineto(886.55,37.03181936)
\curveto(886.47000278,37.01181903)(886.37000288,37.00681904)(886.25,37.01681936)
\curveto(886.12000313,37.02681902)(886.03000322,37.041819)(885.98,37.06181936)
\curveto(885.89000336,37.08181896)(885.82500342,37.13681891)(885.785,37.22681936)
\curveto(885.76500348,37.25681879)(885.76000349,37.28681876)(885.77,37.31681936)
\curveto(885.77000348,37.3468187)(885.76500348,37.38681866)(885.755,37.43681936)
\curveto(885.7450035,37.47681857)(885.74000351,37.51681853)(885.74,37.55681936)
\lineto(885.74,37.70681936)
\curveto(885.74000351,37.82681822)(885.7450035,37.9468181)(885.755,38.06681936)
\curveto(885.75500349,38.19681785)(885.79000346,38.28681776)(885.86,38.33681936)
\curveto(885.92000333,38.37681767)(885.98000327,38.39681765)(886.04,38.39681936)
\curveto(886.10000315,38.39681765)(886.17000308,38.40681764)(886.25,38.42681936)
\curveto(886.28000297,38.43681761)(886.31500294,38.43681761)(886.355,38.42681936)
\curveto(886.38500287,38.42681762)(886.41000284,38.43181761)(886.43,38.44181936)
\lineto(886.64,38.44181936)
\curveto(886.69000256,38.46181758)(886.74000251,38.46681758)(886.79,38.45681936)
\curveto(886.83000242,38.45681759)(886.87500238,38.46681758)(886.925,38.48681936)
\curveto(887.0550022,38.51681753)(887.18000207,38.5468175)(887.3,38.57681936)
\curveto(887.41000184,38.60681744)(887.51500174,38.65181739)(887.615,38.71181936)
\curveto(887.90500134,38.88181716)(888.11000114,39.15181689)(888.23,39.52181936)
\curveto(888.250001,39.57181647)(888.26500099,39.62181642)(888.275,39.67181936)
\curveto(888.27500098,39.73181631)(888.28500096,39.78681626)(888.305,39.83681936)
\lineto(888.305,39.91181936)
\curveto(888.31500093,39.98181606)(888.32500093,40.07681597)(888.335,40.19681936)
\curveto(888.33500092,40.32681572)(888.32500093,40.42681562)(888.305,40.49681936)
\curveto(888.28500096,40.56681548)(888.27000098,40.63681541)(888.26,40.70681936)
\curveto(888.24000101,40.78681526)(888.22000103,40.85681519)(888.2,40.91681936)
\curveto(888.04000121,41.29681475)(887.76500148,41.57181447)(887.375,41.74181936)
\curveto(887.245002,41.79181425)(887.09000216,41.82681422)(886.91,41.84681936)
\curveto(886.73000252,41.87681417)(886.54500271,41.89181415)(886.355,41.89181936)
\curveto(886.15500309,41.90181414)(885.95500329,41.90181414)(885.755,41.89181936)
\lineto(885.185,41.89181936)
\lineto(880.94,41.89181936)
\lineto(879.395,41.89181936)
\curveto(879.28500996,41.89181415)(879.16501008,41.88681416)(879.035,41.87681936)
\curveto(878.90501035,41.86681418)(878.80001045,41.88681416)(878.72,41.93681936)
\curveto(878.6500106,41.99681405)(878.60001065,42.07681397)(878.57,42.17681936)
\curveto(878.57001068,42.19681385)(878.57001068,42.21681383)(878.57,42.23681936)
\curveto(878.57001068,42.25681379)(878.56501068,42.27681377)(878.555,42.29681936)
}
}
{
\newrgbcolor{curcolor}{0 0 0}
\pscustom[linestyle=none,fillstyle=solid,fillcolor=curcolor]
{
\newpath
\moveto(881.51,45.83049123)
\lineto(881.51,46.26549123)
\curveto(881.51000774,46.41548927)(881.5500077,46.52048916)(881.63,46.58049123)
\curveto(881.71000754,46.63048905)(881.81000744,46.65548903)(881.93,46.65549123)
\curveto(882.0500072,46.66548902)(882.17000708,46.67048901)(882.29,46.67049123)
\lineto(883.715,46.67049123)
\lineto(885.98,46.67049123)
\lineto(886.67,46.67049123)
\curveto(886.90000235,46.67048901)(887.10000215,46.69548899)(887.27,46.74549123)
\curveto(887.72000153,46.90548878)(888.03500121,47.20548848)(888.215,47.64549123)
\curveto(888.30500094,47.86548782)(888.34000091,48.13048755)(888.32,48.44049123)
\curveto(888.29000096,48.75048693)(888.23500101,49.00048668)(888.155,49.19049123)
\curveto(888.01500123,49.52048616)(887.84000141,49.7804859)(887.63,49.97049123)
\curveto(887.41000184,50.17048551)(887.12500213,50.32548536)(886.775,50.43549123)
\curveto(886.69500255,50.46548522)(886.61500263,50.4854852)(886.535,50.49549123)
\curveto(886.4550028,50.50548518)(886.37000288,50.52048516)(886.28,50.54049123)
\curveto(886.23000302,50.55048513)(886.18500307,50.55048513)(886.145,50.54049123)
\curveto(886.10500314,50.54048514)(886.06000319,50.55048513)(886.01,50.57049123)
\lineto(885.695,50.57049123)
\curveto(885.61500363,50.59048509)(885.52500373,50.59548509)(885.425,50.58549123)
\curveto(885.31500394,50.57548511)(885.21500403,50.57048511)(885.125,50.57049123)
\lineto(883.955,50.57049123)
\lineto(882.365,50.57049123)
\curveto(882.24500701,50.57048511)(882.12000713,50.56548512)(881.99,50.55549123)
\curveto(881.8500074,50.55548513)(881.74000751,50.5804851)(881.66,50.63049123)
\curveto(881.61000764,50.67048501)(881.58000767,50.71548497)(881.57,50.76549123)
\curveto(881.5500077,50.82548486)(881.53000772,50.89548479)(881.51,50.97549123)
\lineto(881.51,51.20049123)
\curveto(881.51000774,51.32048436)(881.51500774,51.42548426)(881.525,51.51549123)
\curveto(881.53500772,51.61548407)(881.58000767,51.69048399)(881.66,51.74049123)
\curveto(881.71000754,51.79048389)(881.78500747,51.81548387)(881.885,51.81549123)
\lineto(882.17,51.81549123)
\lineto(883.19,51.81549123)
\lineto(887.225,51.81549123)
\lineto(888.575,51.81549123)
\curveto(888.69500055,51.81548387)(888.81000044,51.81048387)(888.92,51.80049123)
\curveto(889.02000023,51.80048388)(889.09500015,51.76548392)(889.145,51.69549123)
\curveto(889.17500007,51.65548403)(889.20000005,51.59548409)(889.22,51.51549123)
\curveto(889.23000002,51.43548425)(889.24000001,51.34548434)(889.25,51.24549123)
\curveto(889.25,51.15548453)(889.245,51.06548462)(889.235,50.97549123)
\curveto(889.22500002,50.89548479)(889.20500005,50.83548485)(889.175,50.79549123)
\curveto(889.13500012,50.74548494)(889.07000018,50.70048498)(888.98,50.66049123)
\curveto(888.94000031,50.65048503)(888.88500036,50.64048504)(888.815,50.63049123)
\curveto(888.74500051,50.63048505)(888.68000057,50.62548506)(888.62,50.61549123)
\curveto(888.5500007,50.60548508)(888.49500075,50.5854851)(888.455,50.55549123)
\curveto(888.41500084,50.52548516)(888.40000085,50.4804852)(888.41,50.42049123)
\curveto(888.43000082,50.34048534)(888.49000076,50.26048542)(888.59,50.18049123)
\curveto(888.68000057,50.10048558)(888.7500005,50.02548566)(888.8,49.95549123)
\curveto(888.96000029,49.73548595)(889.10000015,49.4854862)(889.22,49.20549123)
\curveto(889.26999998,49.09548659)(889.29999995,48.9804867)(889.31,48.86049123)
\curveto(889.32999992,48.75048693)(889.35499989,48.63548705)(889.385,48.51549123)
\curveto(889.39499986,48.46548722)(889.39499986,48.41048727)(889.385,48.35049123)
\curveto(889.37499987,48.30048738)(889.37999987,48.25048743)(889.4,48.20049123)
\curveto(889.41999983,48.10048758)(889.41999983,48.01048767)(889.4,47.93049123)
\lineto(889.4,47.78049123)
\curveto(889.37999987,47.73048795)(889.36999988,47.67048801)(889.37,47.60049123)
\curveto(889.36999988,47.54048814)(889.36499988,47.4854882)(889.355,47.43549123)
\curveto(889.33499992,47.39548829)(889.32499993,47.35548833)(889.325,47.31549123)
\curveto(889.33499992,47.2854884)(889.32999992,47.24548844)(889.31,47.19549123)
\lineto(889.25,46.95549123)
\curveto(889.23000002,46.8854888)(889.20000005,46.81048887)(889.16,46.73049123)
\curveto(889.0500002,46.47048921)(888.90500034,46.25048943)(888.725,46.07049123)
\curveto(888.53500072,45.90048978)(888.31000094,45.76048992)(888.05,45.65049123)
\curveto(887.96000129,45.61049007)(887.87000138,45.5804901)(887.78,45.56049123)
\lineto(887.48,45.50049123)
\curveto(887.42000183,45.4804902)(887.36500188,45.47049021)(887.315,45.47049123)
\curveto(887.255002,45.4804902)(887.19000206,45.47549021)(887.12,45.45549123)
\curveto(887.10000215,45.44549024)(887.07500218,45.44049024)(887.045,45.44049123)
\curveto(887.00500225,45.44049024)(886.97000228,45.43549025)(886.94,45.42549123)
\lineto(886.79,45.42549123)
\curveto(886.7500025,45.41549027)(886.70500254,45.41049027)(886.655,45.41049123)
\curveto(886.59500266,45.42049026)(886.54000271,45.42549026)(886.49,45.42549123)
\lineto(885.89,45.42549123)
\lineto(883.13,45.42549123)
\lineto(882.17,45.42549123)
\lineto(881.9,45.42549123)
\curveto(881.81000744,45.42549026)(881.73500752,45.44549024)(881.675,45.48549123)
\curveto(881.60500765,45.52549016)(881.55500769,45.60049008)(881.525,45.71049123)
\curveto(881.51500774,45.73048995)(881.51500774,45.75048993)(881.525,45.77049123)
\curveto(881.52500773,45.79048989)(881.52000773,45.81048987)(881.51,45.83049123)
}
}
{
\newrgbcolor{curcolor}{0 0 0}
\pscustom[linestyle=none,fillstyle=solid,fillcolor=curcolor]
{
\newpath
\moveto(878.555,54.28510061)
\curveto(878.55501069,54.41509899)(878.55501069,54.55009886)(878.555,54.69010061)
\curveto(878.55501069,54.84009857)(878.59001066,54.95009846)(878.66,55.02010061)
\curveto(878.73001052,55.07009834)(878.82501042,55.09509831)(878.945,55.09510061)
\curveto(879.0550102,55.1050983)(879.17001008,55.1100983)(879.29,55.11010061)
\lineto(880.625,55.11010061)
\lineto(886.7,55.11010061)
\lineto(888.38,55.11010061)
\lineto(888.77,55.11010061)
\curveto(888.91000034,55.1100983)(889.02000023,55.08509832)(889.1,55.03510061)
\curveto(889.1500001,55.0050984)(889.18000007,54.96009845)(889.19,54.90010061)
\curveto(889.20000005,54.85009856)(889.21500004,54.78509862)(889.235,54.70510061)
\lineto(889.235,54.49510061)
\lineto(889.235,54.18010061)
\curveto(889.22500002,54.08009933)(889.19000006,54.0050994)(889.13,53.95510061)
\curveto(889.0500002,53.9050995)(888.9500003,53.87509953)(888.83,53.86510061)
\lineto(888.455,53.86510061)
\lineto(887.075,53.86510061)
\lineto(880.835,53.86510061)
\lineto(879.365,53.86510061)
\curveto(879.25501,53.86509954)(879.14001011,53.86009955)(879.02,53.85010061)
\curveto(878.89001036,53.85009956)(878.79001046,53.87509953)(878.72,53.92510061)
\curveto(878.66001059,53.96509944)(878.61001064,54.04009937)(878.57,54.15010061)
\curveto(878.56001069,54.17009924)(878.56001069,54.19009922)(878.57,54.21010061)
\curveto(878.57001068,54.24009917)(878.56501068,54.26509914)(878.555,54.28510061)
}
}
{
\newrgbcolor{curcolor}{0 0 0}
\pscustom[linestyle=none,fillstyle=solid,fillcolor=curcolor]
{
}
}
{
\newrgbcolor{curcolor}{0 0 0}
\pscustom[linestyle=none,fillstyle=solid,fillcolor=curcolor]
{
\newpath
\moveto(878.63,65.05510061)
\curveto(878.63001062,65.15509575)(878.64001061,65.25009566)(878.66,65.34010061)
\curveto(878.67001058,65.43009548)(878.70001055,65.49509541)(878.75,65.53510061)
\curveto(878.83001042,65.59509531)(878.93501031,65.62509528)(879.065,65.62510061)
\lineto(879.455,65.62510061)
\lineto(880.955,65.62510061)
\lineto(887.345,65.62510061)
\lineto(888.515,65.62510061)
\lineto(888.83,65.62510061)
\curveto(888.93000032,65.63509527)(889.01000024,65.62009529)(889.07,65.58010061)
\curveto(889.1500001,65.53009538)(889.20000005,65.45509545)(889.22,65.35510061)
\curveto(889.23000002,65.26509564)(889.23500001,65.15509575)(889.235,65.02510061)
\lineto(889.235,64.80010061)
\curveto(889.21500004,64.72009619)(889.20000005,64.65009626)(889.19,64.59010061)
\curveto(889.17000008,64.53009638)(889.13000012,64.48009643)(889.07,64.44010061)
\curveto(889.01000024,64.40009651)(888.93500032,64.38009653)(888.845,64.38010061)
\lineto(888.545,64.38010061)
\lineto(887.45,64.38010061)
\lineto(882.11,64.38010061)
\curveto(882.02000723,64.36009655)(881.9450073,64.34509656)(881.885,64.33510061)
\curveto(881.81500743,64.33509657)(881.75500749,64.3050966)(881.705,64.24510061)
\curveto(881.6550076,64.17509673)(881.63000762,64.08509682)(881.63,63.97510061)
\curveto(881.62000763,63.87509703)(881.61500763,63.76509714)(881.615,63.64510061)
\lineto(881.615,62.50510061)
\lineto(881.615,62.01010061)
\curveto(881.60500765,61.85009906)(881.5450077,61.74009917)(881.435,61.68010061)
\curveto(881.40500785,61.66009925)(881.37500788,61.65009926)(881.345,61.65010061)
\curveto(881.30500794,61.65009926)(881.26000799,61.64509926)(881.21,61.63510061)
\curveto(881.09000816,61.61509929)(880.98000827,61.62009929)(880.88,61.65010061)
\curveto(880.78000847,61.69009922)(880.71000854,61.74509916)(880.67,61.81510061)
\curveto(880.62000863,61.89509901)(880.59500866,62.01509889)(880.595,62.17510061)
\curveto(880.59500866,62.33509857)(880.58000867,62.47009844)(880.55,62.58010061)
\curveto(880.54000871,62.63009828)(880.53500872,62.68509822)(880.535,62.74510061)
\curveto(880.52500873,62.8050981)(880.51000874,62.86509804)(880.49,62.92510061)
\curveto(880.44000881,63.07509783)(880.39000886,63.22009769)(880.34,63.36010061)
\curveto(880.28000897,63.50009741)(880.21000904,63.63509727)(880.13,63.76510061)
\curveto(880.04000921,63.905097)(879.93500932,64.02509688)(879.815,64.12510061)
\curveto(879.69500955,64.22509668)(879.56500968,64.32009659)(879.425,64.41010061)
\curveto(879.32500993,64.47009644)(879.21501003,64.51509639)(879.095,64.54510061)
\curveto(878.97501028,64.58509632)(878.87001038,64.63509627)(878.78,64.69510061)
\curveto(878.72001053,64.74509616)(878.68001057,64.81509609)(878.66,64.90510061)
\curveto(878.6500106,64.92509598)(878.64501061,64.95009596)(878.645,64.98010061)
\curveto(878.64501061,65.0100959)(878.64001061,65.03509587)(878.63,65.05510061)
}
}
{
\newrgbcolor{curcolor}{0 0 0}
\pscustom[linestyle=none,fillstyle=solid,fillcolor=curcolor]
{
\newpath
\moveto(884.915,76.34470998)
\curveto(885.03500421,76.37470226)(885.17500407,76.39970223)(885.335,76.41970998)
\curveto(885.49500375,76.43970219)(885.66000359,76.44970218)(885.83,76.44970998)
\curveto(886.00000325,76.44970218)(886.16500308,76.43970219)(886.325,76.41970998)
\curveto(886.48500276,76.39970223)(886.62500262,76.37470226)(886.745,76.34470998)
\curveto(886.88500236,76.30470233)(887.01000224,76.26970236)(887.12,76.23970998)
\curveto(887.23000202,76.20970242)(887.34000191,76.16970246)(887.45,76.11970998)
\curveto(888.09000116,75.84970278)(888.57500067,75.4347032)(888.905,74.87470998)
\curveto(888.96500028,74.79470384)(889.01500024,74.70970392)(889.055,74.61970998)
\curveto(889.08500016,74.5297041)(889.12000013,74.4297042)(889.16,74.31970998)
\curveto(889.21000004,74.20970442)(889.245,74.08970454)(889.265,73.95970998)
\curveto(889.29499995,73.83970479)(889.32499993,73.70970492)(889.355,73.56970998)
\curveto(889.37499987,73.50970512)(889.37999987,73.44970518)(889.37,73.38970998)
\curveto(889.35999989,73.33970529)(889.36499988,73.27970535)(889.385,73.20970998)
\curveto(889.39499986,73.18970544)(889.39499986,73.16470547)(889.385,73.13470998)
\curveto(889.38499987,73.10470553)(889.38999986,73.07970555)(889.4,73.05970998)
\lineto(889.4,72.90970998)
\curveto(889.40999984,72.83970579)(889.40999984,72.78970584)(889.4,72.75970998)
\curveto(889.38999986,72.71970591)(889.38499987,72.67470596)(889.385,72.62470998)
\curveto(889.39499986,72.58470605)(889.39499986,72.54470609)(889.385,72.50470998)
\curveto(889.36499988,72.41470622)(889.3499999,72.32470631)(889.34,72.23470998)
\curveto(889.33999991,72.14470649)(889.32999992,72.05470658)(889.31,71.96470998)
\curveto(889.27999997,71.87470676)(889.255,71.78470685)(889.235,71.69470998)
\curveto(889.21500004,71.60470703)(889.18500007,71.51970711)(889.145,71.43970998)
\curveto(889.03500021,71.19970743)(888.90500034,70.97470766)(888.755,70.76470998)
\curveto(888.59500066,70.55470808)(888.41500084,70.37470826)(888.215,70.22470998)
\curveto(888.0450012,70.10470853)(887.87000138,69.99970863)(887.69,69.90970998)
\curveto(887.51000174,69.81970881)(887.32000193,69.7297089)(887.12,69.63970998)
\curveto(887.02000223,69.59970903)(886.92000233,69.56470907)(886.82,69.53470998)
\curveto(886.71000254,69.51470912)(886.60000265,69.48970914)(886.49,69.45970998)
\curveto(886.3500029,69.41970921)(886.21000304,69.39470924)(886.07,69.38470998)
\curveto(885.93000332,69.37470926)(885.79000346,69.35470928)(885.65,69.32470998)
\curveto(885.54000371,69.31470932)(885.44000381,69.30470933)(885.35,69.29470998)
\curveto(885.250004,69.29470934)(885.1500041,69.28470935)(885.05,69.26470998)
\lineto(884.96,69.26470998)
\curveto(884.93000432,69.27470936)(884.90500434,69.27470936)(884.885,69.26470998)
\lineto(884.675,69.26470998)
\curveto(884.61500463,69.24470939)(884.5500047,69.2347094)(884.48,69.23470998)
\curveto(884.40000485,69.24470939)(884.32500493,69.24970938)(884.255,69.24970998)
\lineto(884.105,69.24970998)
\curveto(884.0550052,69.24970938)(884.00500525,69.25470938)(883.955,69.26470998)
\lineto(883.58,69.26470998)
\curveto(883.5500057,69.27470936)(883.51500574,69.27470936)(883.475,69.26470998)
\curveto(883.43500581,69.26470937)(883.39500586,69.26970936)(883.355,69.27970998)
\curveto(883.24500601,69.29970933)(883.13500612,69.31470932)(883.025,69.32470998)
\curveto(882.90500634,69.3347093)(882.79000646,69.34470929)(882.68,69.35470998)
\curveto(882.53000672,69.39470924)(882.38500687,69.41970921)(882.245,69.42970998)
\curveto(882.09500715,69.44970918)(881.9500073,69.47970915)(881.81,69.51970998)
\curveto(881.51000774,69.60970902)(881.22500802,69.70470893)(880.955,69.80470998)
\curveto(880.68500856,69.90470873)(880.43500881,70.0297086)(880.205,70.17970998)
\curveto(879.88500936,70.37970825)(879.60500964,70.62470801)(879.365,70.91470998)
\curveto(879.12501013,71.20470743)(878.94001031,71.54470709)(878.81,71.93470998)
\curveto(878.77001048,72.04470659)(878.7450105,72.15470648)(878.735,72.26470998)
\curveto(878.71501054,72.38470625)(878.69001056,72.50470613)(878.66,72.62470998)
\curveto(878.6500106,72.69470594)(878.64501061,72.75970587)(878.645,72.81970998)
\curveto(878.64501061,72.87970575)(878.64001061,72.94470569)(878.63,73.01470998)
\curveto(878.61001064,73.71470492)(878.72501053,74.28970434)(878.975,74.73970998)
\curveto(879.22501002,75.18970344)(879.57500968,75.5347031)(880.025,75.77470998)
\curveto(880.255009,75.88470275)(880.53000872,75.98470265)(880.85,76.07470998)
\curveto(880.92000833,76.09470254)(880.99500826,76.09470254)(881.075,76.07470998)
\curveto(881.1450081,76.06470257)(881.19500806,76.03970259)(881.225,75.99970998)
\curveto(881.255008,75.96970266)(881.28000797,75.90970272)(881.3,75.81970998)
\curveto(881.31000794,75.7297029)(881.32000793,75.629703)(881.33,75.51970998)
\curveto(881.33000792,75.41970321)(881.32500793,75.31970331)(881.315,75.21970998)
\curveto(881.30500794,75.1297035)(881.28500796,75.06470357)(881.255,75.02470998)
\curveto(881.18500807,74.91470372)(881.07500817,74.8347038)(880.925,74.78470998)
\curveto(880.77500848,74.74470389)(880.64500861,74.68970394)(880.535,74.61970998)
\curveto(880.22500902,74.4297042)(879.99500926,74.14970448)(879.845,73.77970998)
\curveto(879.81500943,73.70970492)(879.79500946,73.634705)(879.785,73.55470998)
\curveto(879.77500948,73.48470515)(879.76000949,73.40970522)(879.74,73.32970998)
\curveto(879.73000952,73.27970535)(879.72500953,73.20970542)(879.725,73.11970998)
\curveto(879.72500953,73.03970559)(879.73000952,72.97470566)(879.74,72.92470998)
\curveto(879.76000949,72.88470575)(879.76500948,72.84970578)(879.755,72.81970998)
\curveto(879.7450095,72.78970584)(879.7450095,72.75470588)(879.755,72.71470998)
\lineto(879.815,72.47470998)
\curveto(879.83500941,72.40470623)(879.86000939,72.3347063)(879.89,72.26470998)
\curveto(880.0500092,71.88470675)(880.26000899,71.59470704)(880.52,71.39470998)
\curveto(880.78000847,71.20470743)(881.09500815,71.0297076)(881.465,70.86970998)
\curveto(881.5450077,70.83970779)(881.62500762,70.81470782)(881.705,70.79470998)
\curveto(881.78500747,70.78470785)(881.86500739,70.76470787)(881.945,70.73470998)
\curveto(882.0550072,70.70470793)(882.17000708,70.67970795)(882.29,70.65970998)
\curveto(882.41000684,70.64970798)(882.53000672,70.629708)(882.65,70.59970998)
\curveto(882.70000655,70.57970805)(882.7500065,70.56970806)(882.8,70.56970998)
\curveto(882.8500064,70.57970805)(882.90000635,70.57470806)(882.95,70.55470998)
\curveto(883.01000624,70.54470809)(883.09000616,70.54470809)(883.19,70.55470998)
\curveto(883.28000597,70.56470807)(883.33500592,70.57970805)(883.355,70.59970998)
\curveto(883.39500586,70.61970801)(883.41500583,70.64970798)(883.415,70.68970998)
\curveto(883.41500583,70.73970789)(883.40500585,70.78470785)(883.385,70.82470998)
\curveto(883.3450059,70.89470774)(883.30000595,70.95470768)(883.25,71.00470998)
\curveto(883.20000605,71.05470758)(883.1500061,71.11470752)(883.1,71.18470998)
\lineto(883.04,71.24470998)
\curveto(883.01000624,71.27470736)(882.98500627,71.30470733)(882.965,71.33470998)
\curveto(882.80500645,71.56470707)(882.67000658,71.83970679)(882.56,72.15970998)
\curveto(882.54000671,72.2297064)(882.52500673,72.29970633)(882.515,72.36970998)
\curveto(882.50500674,72.43970619)(882.49000676,72.51470612)(882.47,72.59470998)
\curveto(882.47000678,72.634706)(882.46500679,72.66970596)(882.455,72.69970998)
\curveto(882.44500681,72.7297059)(882.44500681,72.76470587)(882.455,72.80470998)
\curveto(882.4550068,72.85470578)(882.44500681,72.89470574)(882.425,72.92470998)
\lineto(882.425,73.08970998)
\lineto(882.425,73.17970998)
\curveto(882.41500683,73.2297054)(882.41500683,73.26970536)(882.425,73.29970998)
\curveto(882.43500681,73.34970528)(882.44000681,73.39970523)(882.44,73.44970998)
\curveto(882.43000682,73.50970512)(882.43000682,73.56470507)(882.44,73.61470998)
\curveto(882.47000678,73.72470491)(882.49000676,73.8297048)(882.5,73.92970998)
\curveto(882.51000674,74.03970459)(882.53500672,74.14470449)(882.575,74.24470998)
\curveto(882.71500654,74.66470397)(882.90000635,75.00970362)(883.13,75.27970998)
\curveto(883.3500059,75.54970308)(883.63500561,75.78970284)(883.985,75.99970998)
\curveto(884.12500513,76.07970255)(884.27500498,76.14470249)(884.435,76.19470998)
\curveto(884.58500467,76.24470239)(884.7450045,76.29470234)(884.915,76.34470998)
\moveto(886.22,75.09970998)
\curveto(886.17000308,75.10970352)(886.12500313,75.11470352)(886.085,75.11470998)
\lineto(885.935,75.11470998)
\curveto(885.62500362,75.11470352)(885.34000391,75.07470356)(885.08,74.99470998)
\curveto(885.02000423,74.97470366)(884.96500428,74.95470368)(884.915,74.93470998)
\curveto(884.8550044,74.92470371)(884.80000445,74.90970372)(884.75,74.88970998)
\curveto(884.26000499,74.66970396)(883.91000534,74.32470431)(883.7,73.85470998)
\curveto(883.67000558,73.77470486)(883.64500561,73.69470494)(883.625,73.61470998)
\lineto(883.565,73.37470998)
\curveto(883.5450057,73.29470534)(883.53500572,73.20470543)(883.535,73.10470998)
\lineto(883.535,72.78970998)
\curveto(883.55500569,72.76970586)(883.56500568,72.7297059)(883.565,72.66970998)
\curveto(883.55500569,72.61970601)(883.55500569,72.57470606)(883.565,72.53470998)
\lineto(883.625,72.29470998)
\curveto(883.63500561,72.22470641)(883.6550056,72.15470648)(883.685,72.08470998)
\curveto(883.9450053,71.48470715)(884.41000484,71.07970755)(885.08,70.86970998)
\curveto(885.16000409,70.83970779)(885.24000401,70.81970781)(885.32,70.80970998)
\curveto(885.40000385,70.79970783)(885.48500376,70.78470785)(885.575,70.76470998)
\lineto(885.725,70.76470998)
\curveto(885.76500348,70.75470788)(885.83500341,70.74970788)(885.935,70.74970998)
\curveto(886.16500308,70.74970788)(886.36000289,70.76970786)(886.52,70.80970998)
\curveto(886.59000266,70.8297078)(886.6550026,70.84470779)(886.715,70.85470998)
\curveto(886.77500247,70.86470777)(886.84000241,70.88470775)(886.91,70.91470998)
\curveto(887.19000206,71.02470761)(887.43500181,71.16970746)(887.645,71.34970998)
\curveto(887.8450014,71.5297071)(888.00500125,71.76470687)(888.125,72.05470998)
\lineto(888.215,72.29470998)
\lineto(888.275,72.53470998)
\curveto(888.29500095,72.58470605)(888.30000095,72.62470601)(888.29,72.65470998)
\curveto(888.28000097,72.69470594)(888.28500096,72.73970589)(888.305,72.78970998)
\curveto(888.31500093,72.81970581)(888.32000093,72.87470576)(888.32,72.95470998)
\curveto(888.32000093,73.0347056)(888.31500093,73.09470554)(888.305,73.13470998)
\curveto(888.28500096,73.24470539)(888.27000098,73.34970528)(888.26,73.44970998)
\curveto(888.250001,73.54970508)(888.22000103,73.64470499)(888.17,73.73470998)
\curveto(887.97000128,74.26470437)(887.59500166,74.65470398)(887.045,74.90470998)
\curveto(886.94500231,74.94470369)(886.84000241,74.97470366)(886.73,74.99470998)
\lineto(886.4,75.08470998)
\curveto(886.32000293,75.08470355)(886.26000299,75.08970354)(886.22,75.09970998)
}
}
{
\newrgbcolor{curcolor}{0 0 0}
\pscustom[linestyle=none,fillstyle=solid,fillcolor=curcolor]
{
\newpath
\moveto(887.6,78.63431936)
\lineto(887.6,79.26431936)
\lineto(887.6,79.45931936)
\curveto(887.60000165,79.52931683)(887.61000164,79.58931677)(887.63,79.63931936)
\curveto(887.67000158,79.70931665)(887.71000154,79.7593166)(887.75,79.78931936)
\curveto(887.80000145,79.82931653)(887.86500139,79.84931651)(887.945,79.84931936)
\curveto(888.02500122,79.8593165)(888.11000114,79.86431649)(888.2,79.86431936)
\lineto(888.92,79.86431936)
\curveto(889.39999985,79.86431649)(889.80999944,79.80431655)(890.15,79.68431936)
\curveto(890.48999876,79.56431679)(890.76499848,79.36931699)(890.975,79.09931936)
\curveto(891.02499822,79.02931733)(891.06999818,78.9593174)(891.11,78.88931936)
\curveto(891.15999809,78.82931753)(891.20499805,78.7543176)(891.245,78.66431936)
\curveto(891.254998,78.64431771)(891.26499799,78.61431774)(891.275,78.57431936)
\curveto(891.29499795,78.53431782)(891.29999795,78.48931787)(891.29,78.43931936)
\curveto(891.25999799,78.34931801)(891.18499806,78.29431806)(891.065,78.27431936)
\curveto(890.95499829,78.2543181)(890.85999839,78.26931809)(890.78,78.31931936)
\curveto(890.70999854,78.34931801)(890.6449986,78.39431796)(890.585,78.45431936)
\curveto(890.53499872,78.52431783)(890.48499877,78.58931777)(890.435,78.64931936)
\curveto(890.38499886,78.71931764)(890.30999894,78.77931758)(890.21,78.82931936)
\curveto(890.11999913,78.88931747)(890.02999922,78.93931742)(889.94,78.97931936)
\curveto(889.90999934,78.99931736)(889.8499994,79.02431733)(889.76,79.05431936)
\curveto(889.67999957,79.08431727)(889.60999964,79.08931727)(889.55,79.06931936)
\curveto(889.40999984,79.03931732)(889.31999993,78.97931738)(889.28,78.88931936)
\curveto(889.25,78.80931755)(889.23500001,78.71931764)(889.235,78.61931936)
\curveto(889.23500001,78.51931784)(889.21000004,78.43431792)(889.16,78.36431936)
\curveto(889.09000016,78.27431808)(888.9500003,78.22931813)(888.74,78.22931936)
\lineto(888.185,78.22931936)
\lineto(887.96,78.22931936)
\curveto(887.88000137,78.23931812)(887.81500143,78.2593181)(887.765,78.28931936)
\curveto(887.68500156,78.34931801)(887.64000161,78.41931794)(887.63,78.49931936)
\curveto(887.62000163,78.51931784)(887.61500163,78.53931782)(887.615,78.55931936)
\curveto(887.61500163,78.58931777)(887.61000164,78.61431774)(887.6,78.63431936)
}
}
{
\newrgbcolor{curcolor}{0 0 0}
\pscustom[linestyle=none,fillstyle=solid,fillcolor=curcolor]
{
}
}
{
\newrgbcolor{curcolor}{0 0 0}
\pscustom[linestyle=none,fillstyle=solid,fillcolor=curcolor]
{
\newpath
\moveto(878.63,89.26463186)
\curveto(878.62001063,89.95462722)(878.74001051,90.55462662)(878.99,91.06463186)
\curveto(879.24001001,91.58462559)(879.57500968,91.9796252)(879.995,92.24963186)
\curveto(880.07500917,92.29962488)(880.16500908,92.34462483)(880.265,92.38463186)
\curveto(880.35500889,92.42462475)(880.4500088,92.46962471)(880.55,92.51963186)
\curveto(880.6500086,92.55962462)(880.7500085,92.58962459)(880.85,92.60963186)
\curveto(880.9500083,92.62962455)(881.0550082,92.64962453)(881.165,92.66963186)
\curveto(881.21500803,92.68962449)(881.26000799,92.69462448)(881.3,92.68463186)
\curveto(881.34000791,92.6746245)(881.38500787,92.6796245)(881.435,92.69963186)
\curveto(881.48500776,92.70962447)(881.57000768,92.71462446)(881.69,92.71463186)
\curveto(881.80000745,92.71462446)(881.88500736,92.70962447)(881.945,92.69963186)
\curveto(882.00500725,92.6796245)(882.06500719,92.66962451)(882.125,92.66963186)
\curveto(882.18500707,92.6796245)(882.24500701,92.6746245)(882.305,92.65463186)
\curveto(882.44500681,92.61462456)(882.58000667,92.5796246)(882.71,92.54963186)
\curveto(882.84000641,92.51962466)(882.96500628,92.4796247)(883.085,92.42963186)
\curveto(883.22500602,92.36962481)(883.3500059,92.29962488)(883.46,92.21963186)
\curveto(883.57000568,92.14962503)(883.68000557,92.0746251)(883.79,91.99463186)
\lineto(883.85,91.93463186)
\curveto(883.87000538,91.92462525)(883.89000536,91.90962527)(883.91,91.88963186)
\curveto(884.07000518,91.76962541)(884.21500503,91.63462554)(884.345,91.48463186)
\curveto(884.47500478,91.33462584)(884.60000465,91.174626)(884.72,91.00463186)
\curveto(884.94000431,90.69462648)(885.1450041,90.39962678)(885.335,90.11963186)
\curveto(885.47500378,89.88962729)(885.61000364,89.65962752)(885.74,89.42963186)
\curveto(885.87000338,89.20962797)(886.00500325,88.98962819)(886.145,88.76963186)
\curveto(886.31500294,88.51962866)(886.49500275,88.2796289)(886.685,88.04963186)
\curveto(886.87500238,87.82962935)(887.10000215,87.63962954)(887.36,87.47963186)
\curveto(887.42000183,87.43962974)(887.48000177,87.40462977)(887.54,87.37463186)
\curveto(887.59000166,87.34462983)(887.6550016,87.31462986)(887.735,87.28463186)
\curveto(887.80500145,87.26462991)(887.86500139,87.25962992)(887.915,87.26963186)
\curveto(887.98500127,87.28962989)(888.04000121,87.32462985)(888.08,87.37463186)
\curveto(888.11000114,87.42462975)(888.13000112,87.48462969)(888.14,87.55463186)
\lineto(888.14,87.79463186)
\lineto(888.14,88.54463186)
\lineto(888.14,91.34963186)
\lineto(888.14,92.00963186)
\curveto(888.14000111,92.09962508)(888.14500111,92.18462499)(888.155,92.26463186)
\curveto(888.15500109,92.34462483)(888.17500107,92.40962477)(888.215,92.45963186)
\curveto(888.255001,92.50962467)(888.33000092,92.54962463)(888.44,92.57963186)
\curveto(888.54000071,92.61962456)(888.64000061,92.62962455)(888.74,92.60963186)
\lineto(888.875,92.60963186)
\curveto(888.94500031,92.58962459)(889.00500025,92.56962461)(889.055,92.54963186)
\curveto(889.10500014,92.52962465)(889.14500011,92.49462468)(889.175,92.44463186)
\curveto(889.21500004,92.39462478)(889.23500001,92.32462485)(889.235,92.23463186)
\lineto(889.235,91.96463186)
\lineto(889.235,91.06463186)
\lineto(889.235,87.55463186)
\lineto(889.235,86.48963186)
\curveto(889.23500001,86.40963077)(889.24000001,86.31963086)(889.25,86.21963186)
\curveto(889.25,86.11963106)(889.24000001,86.03463114)(889.22,85.96463186)
\curveto(889.1500001,85.75463142)(888.97000028,85.68963149)(888.68,85.76963186)
\curveto(888.64000061,85.7796314)(888.60500065,85.7796314)(888.575,85.76963186)
\curveto(888.53500072,85.76963141)(888.49000076,85.7796314)(888.44,85.79963186)
\curveto(888.36000089,85.81963136)(888.27500098,85.83963134)(888.185,85.85963186)
\curveto(888.09500115,85.8796313)(888.01000124,85.90463127)(887.93,85.93463186)
\curveto(887.44000181,86.09463108)(887.02500222,86.29463088)(886.685,86.53463186)
\curveto(886.43500281,86.71463046)(886.21000304,86.91963026)(886.01,87.14963186)
\curveto(885.80000345,87.3796298)(885.60500365,87.61962956)(885.425,87.86963186)
\curveto(885.24500401,88.12962905)(885.07500418,88.39462878)(884.915,88.66463186)
\curveto(884.7450045,88.94462823)(884.57000468,89.21462796)(884.39,89.47463186)
\curveto(884.31000494,89.58462759)(884.23500501,89.68962749)(884.165,89.78963186)
\curveto(884.09500515,89.89962728)(884.02000523,90.00962717)(883.94,90.11963186)
\curveto(883.91000534,90.15962702)(883.88000537,90.19462698)(883.85,90.22463186)
\curveto(883.81000544,90.26462691)(883.78000547,90.30462687)(883.76,90.34463186)
\curveto(883.6500056,90.48462669)(883.52500573,90.60962657)(883.385,90.71963186)
\curveto(883.35500589,90.73962644)(883.33000592,90.76462641)(883.31,90.79463186)
\curveto(883.28000597,90.82462635)(883.250006,90.84962633)(883.22,90.86963186)
\curveto(883.12000613,90.94962623)(883.02000623,91.01462616)(882.92,91.06463186)
\curveto(882.82000643,91.12462605)(882.71000654,91.179626)(882.59,91.22963186)
\curveto(882.52000673,91.25962592)(882.44500681,91.2796259)(882.365,91.28963186)
\lineto(882.125,91.34963186)
\lineto(882.035,91.34963186)
\curveto(882.00500725,91.35962582)(881.97500728,91.36462581)(881.945,91.36463186)
\curveto(881.87500737,91.38462579)(881.78000747,91.38962579)(881.66,91.37963186)
\curveto(881.53000772,91.3796258)(881.43000782,91.36962581)(881.36,91.34963186)
\curveto(881.28000797,91.32962585)(881.20500804,91.30962587)(881.135,91.28963186)
\curveto(881.0550082,91.2796259)(880.97500828,91.25962592)(880.895,91.22963186)
\curveto(880.6550086,91.11962606)(880.4550088,90.96962621)(880.295,90.77963186)
\curveto(880.12500913,90.59962658)(879.98500927,90.3796268)(879.875,90.11963186)
\curveto(879.8550094,90.04962713)(879.84000941,89.9796272)(879.83,89.90963186)
\curveto(879.81000944,89.83962734)(879.79000946,89.76462741)(879.77,89.68463186)
\curveto(879.7500095,89.60462757)(879.74000951,89.49462768)(879.74,89.35463186)
\curveto(879.74000951,89.22462795)(879.7500095,89.11962806)(879.77,89.03963186)
\curveto(879.78000947,88.9796282)(879.78500947,88.92462825)(879.785,88.87463186)
\curveto(879.78500947,88.82462835)(879.79500946,88.7746284)(879.815,88.72463186)
\curveto(879.8550094,88.62462855)(879.89500935,88.52962865)(879.935,88.43963186)
\curveto(879.97500928,88.35962882)(880.02000923,88.2796289)(880.07,88.19963186)
\curveto(880.09000916,88.16962901)(880.11500914,88.13962904)(880.145,88.10963186)
\curveto(880.17500908,88.08962909)(880.20000905,88.06462911)(880.22,88.03463186)
\lineto(880.295,87.95963186)
\curveto(880.31500894,87.92962925)(880.33500892,87.90462927)(880.355,87.88463186)
\lineto(880.565,87.73463186)
\curveto(880.62500862,87.69462948)(880.69000856,87.64962953)(880.76,87.59963186)
\curveto(880.8500084,87.53962964)(880.95500829,87.48962969)(881.075,87.44963186)
\curveto(881.18500807,87.41962976)(881.29500795,87.38462979)(881.405,87.34463186)
\curveto(881.51500774,87.30462987)(881.66000759,87.2796299)(881.84,87.26963186)
\curveto(882.01000724,87.25962992)(882.13500712,87.22962995)(882.215,87.17963186)
\curveto(882.29500695,87.12963005)(882.34000691,87.05463012)(882.35,86.95463186)
\curveto(882.36000689,86.85463032)(882.36500688,86.74463043)(882.365,86.62463186)
\curveto(882.36500688,86.58463059)(882.37000688,86.54463063)(882.38,86.50463186)
\curveto(882.38000687,86.46463071)(882.37500688,86.42963075)(882.365,86.39963186)
\curveto(882.3450069,86.34963083)(882.33500692,86.29963088)(882.335,86.24963186)
\curveto(882.33500692,86.20963097)(882.32500693,86.16963101)(882.305,86.12963186)
\curveto(882.24500701,86.03963114)(882.11000714,85.99463118)(881.9,85.99463186)
\lineto(881.78,85.99463186)
\curveto(881.72000753,86.00463117)(881.66000759,86.00963117)(881.6,86.00963186)
\curveto(881.53000772,86.01963116)(881.46500779,86.02963115)(881.405,86.03963186)
\curveto(881.29500795,86.05963112)(881.19500806,86.0796311)(881.105,86.09963186)
\curveto(881.00500824,86.11963106)(880.91000834,86.14963103)(880.82,86.18963186)
\curveto(880.7500085,86.20963097)(880.69000856,86.22963095)(880.64,86.24963186)
\lineto(880.46,86.30963186)
\curveto(880.20000905,86.42963075)(879.95500929,86.58463059)(879.725,86.77463186)
\curveto(879.49500975,86.9746302)(879.31000994,87.18962999)(879.17,87.41963186)
\curveto(879.09001016,87.52962965)(879.02501022,87.64462953)(878.975,87.76463186)
\lineto(878.825,88.15463186)
\curveto(878.77501048,88.26462891)(878.7450105,88.3796288)(878.735,88.49963186)
\curveto(878.71501054,88.61962856)(878.69001056,88.74462843)(878.66,88.87463186)
\curveto(878.66001059,88.94462823)(878.66001059,89.00962817)(878.66,89.06963186)
\curveto(878.6500106,89.12962805)(878.64001061,89.19462798)(878.63,89.26463186)
}
}
{
\newrgbcolor{curcolor}{0 0 0}
\pscustom[linestyle=none,fillstyle=solid,fillcolor=curcolor]
{
\newpath
\moveto(884.15,101.36424123)
\lineto(884.405,101.36424123)
\curveto(884.48500476,101.37423353)(884.56000469,101.36923353)(884.63,101.34924123)
\lineto(884.87,101.34924123)
\lineto(885.035,101.34924123)
\curveto(885.13500412,101.32923357)(885.24000401,101.31923358)(885.35,101.31924123)
\curveto(885.4500038,101.31923358)(885.5500037,101.30923359)(885.65,101.28924123)
\lineto(885.8,101.28924123)
\curveto(885.94000331,101.25923364)(886.08000317,101.23923366)(886.22,101.22924123)
\curveto(886.3500029,101.21923368)(886.48000277,101.19423371)(886.61,101.15424123)
\curveto(886.69000256,101.13423377)(886.77500247,101.11423379)(886.865,101.09424123)
\lineto(887.105,101.03424123)
\lineto(887.405,100.91424123)
\curveto(887.49500175,100.88423402)(887.58500167,100.84923405)(887.675,100.80924123)
\curveto(887.89500135,100.70923419)(888.11000114,100.57423433)(888.32,100.40424123)
\curveto(888.53000072,100.24423466)(888.70000055,100.06923483)(888.83,99.87924123)
\curveto(888.87000038,99.82923507)(888.91000034,99.76923513)(888.95,99.69924123)
\curveto(888.98000027,99.63923526)(889.01500024,99.57923532)(889.055,99.51924123)
\curveto(889.10500014,99.43923546)(889.14500011,99.34423556)(889.175,99.23424123)
\curveto(889.20500005,99.12423578)(889.23500001,99.01923588)(889.265,98.91924123)
\curveto(889.30499994,98.80923609)(889.32999992,98.6992362)(889.34,98.58924123)
\curveto(889.3499999,98.47923642)(889.36499988,98.36423654)(889.385,98.24424123)
\curveto(889.39499986,98.2042367)(889.39499986,98.15923674)(889.385,98.10924123)
\curveto(889.38499987,98.06923683)(889.38999986,98.02923687)(889.4,97.98924123)
\curveto(889.40999984,97.94923695)(889.41499984,97.89423701)(889.415,97.82424123)
\curveto(889.41499984,97.75423715)(889.40999984,97.7042372)(889.4,97.67424123)
\curveto(889.37999987,97.62423728)(889.37499987,97.57923732)(889.385,97.53924123)
\curveto(889.39499986,97.4992374)(889.39499986,97.46423744)(889.385,97.43424123)
\lineto(889.385,97.34424123)
\curveto(889.36499988,97.28423762)(889.3499999,97.21923768)(889.34,97.14924123)
\curveto(889.33999991,97.08923781)(889.33499992,97.02423788)(889.325,96.95424123)
\curveto(889.27499998,96.78423812)(889.22500002,96.62423828)(889.175,96.47424123)
\curveto(889.12500013,96.32423858)(889.06000019,96.17923872)(888.98,96.03924123)
\curveto(888.94000031,95.98923891)(888.91000034,95.93423897)(888.89,95.87424123)
\curveto(888.86000039,95.82423908)(888.82500042,95.77423913)(888.785,95.72424123)
\curveto(888.60500065,95.48423942)(888.38500087,95.28423962)(888.125,95.12424123)
\curveto(887.86500139,94.96423994)(887.58000167,94.82424008)(887.27,94.70424123)
\curveto(887.13000212,94.64424026)(886.99000226,94.5992403)(886.85,94.56924123)
\curveto(886.70000255,94.53924036)(886.54500271,94.5042404)(886.385,94.46424123)
\curveto(886.27500298,94.44424046)(886.16500308,94.42924047)(886.055,94.41924123)
\curveto(885.94500331,94.40924049)(885.83500341,94.39424051)(885.725,94.37424123)
\curveto(885.68500356,94.36424054)(885.64500361,94.35924054)(885.605,94.35924123)
\curveto(885.56500368,94.36924053)(885.52500373,94.36924053)(885.485,94.35924123)
\curveto(885.43500381,94.34924055)(885.38500387,94.34424056)(885.335,94.34424123)
\lineto(885.17,94.34424123)
\curveto(885.12000413,94.32424058)(885.07000418,94.31924058)(885.02,94.32924123)
\curveto(884.96000429,94.33924056)(884.90500434,94.33924056)(884.855,94.32924123)
\curveto(884.81500443,94.31924058)(884.77000448,94.31924058)(884.72,94.32924123)
\curveto(884.67000458,94.33924056)(884.62000463,94.33424057)(884.57,94.31424123)
\curveto(884.50000475,94.29424061)(884.42500482,94.28924061)(884.345,94.29924123)
\curveto(884.255005,94.30924059)(884.17000508,94.31424059)(884.09,94.31424123)
\curveto(884.00000525,94.31424059)(883.90000535,94.30924059)(883.79,94.29924123)
\curveto(883.67000558,94.28924061)(883.57000568,94.29424061)(883.49,94.31424123)
\lineto(883.205,94.31424123)
\lineto(882.575,94.35924123)
\curveto(882.47500677,94.36924053)(882.38000687,94.37924052)(882.29,94.38924123)
\lineto(881.99,94.41924123)
\curveto(881.94000731,94.43924046)(881.89000736,94.44424046)(881.84,94.43424123)
\curveto(881.78000747,94.43424047)(881.72500753,94.44424046)(881.675,94.46424123)
\curveto(881.50500775,94.51424039)(881.34000791,94.55424035)(881.18,94.58424123)
\curveto(881.01000824,94.61424029)(880.8500084,94.66424024)(880.7,94.73424123)
\curveto(880.24000901,94.92423998)(879.86500939,95.14423976)(879.575,95.39424123)
\curveto(879.28500996,95.65423925)(879.04001021,96.01423889)(878.84,96.47424123)
\curveto(878.79001046,96.6042383)(878.75501049,96.73423817)(878.735,96.86424123)
\curveto(878.71501054,97.0042379)(878.69001056,97.14423776)(878.66,97.28424123)
\curveto(878.6500106,97.35423755)(878.64501061,97.41923748)(878.645,97.47924123)
\curveto(878.64501061,97.53923736)(878.64001061,97.6042373)(878.63,97.67424123)
\curveto(878.61001064,98.5042364)(878.76001049,99.17423573)(879.08,99.68424123)
\curveto(879.39000986,100.19423471)(879.83000942,100.57423433)(880.4,100.82424123)
\curveto(880.52000873,100.87423403)(880.64500861,100.91923398)(880.775,100.95924123)
\curveto(880.90500835,100.9992339)(881.04000821,101.04423386)(881.18,101.09424123)
\curveto(881.26000799,101.11423379)(881.3450079,101.12923377)(881.435,101.13924123)
\lineto(881.675,101.19924123)
\curveto(881.78500747,101.22923367)(881.89500735,101.24423366)(882.005,101.24424123)
\curveto(882.11500714,101.25423365)(882.22500702,101.26923363)(882.335,101.28924123)
\curveto(882.38500687,101.30923359)(882.43000682,101.31423359)(882.47,101.30424123)
\curveto(882.51000674,101.3042336)(882.5500067,101.30923359)(882.59,101.31924123)
\curveto(882.64000661,101.32923357)(882.69500655,101.32923357)(882.755,101.31924123)
\curveto(882.80500645,101.31923358)(882.8550064,101.32423358)(882.905,101.33424123)
\lineto(883.04,101.33424123)
\curveto(883.10000615,101.35423355)(883.17000608,101.35423355)(883.25,101.33424123)
\curveto(883.32000593,101.32423358)(883.38500587,101.32923357)(883.445,101.34924123)
\curveto(883.47500577,101.35923354)(883.51500574,101.36423354)(883.565,101.36424123)
\lineto(883.685,101.36424123)
\lineto(884.15,101.36424123)
\moveto(886.475,99.81924123)
\curveto(886.15500309,99.91923498)(885.79000346,99.97923492)(885.38,99.99924123)
\curveto(884.97000428,100.01923488)(884.56000469,100.02923487)(884.15,100.02924123)
\curveto(883.72000553,100.02923487)(883.30000595,100.01923488)(882.89,99.99924123)
\curveto(882.48000677,99.97923492)(882.09500715,99.93423497)(881.735,99.86424123)
\curveto(881.37500788,99.79423511)(881.0550082,99.68423522)(880.775,99.53424123)
\curveto(880.48500876,99.39423551)(880.250009,99.1992357)(880.07,98.94924123)
\curveto(879.96000929,98.78923611)(879.88000937,98.60923629)(879.83,98.40924123)
\curveto(879.77000948,98.20923669)(879.74000951,97.96423694)(879.74,97.67424123)
\curveto(879.76000949,97.65423725)(879.77000948,97.61923728)(879.77,97.56924123)
\curveto(879.76000949,97.51923738)(879.76000949,97.47923742)(879.77,97.44924123)
\curveto(879.79000946,97.36923753)(879.81000944,97.29423761)(879.83,97.22424123)
\curveto(879.84000941,97.16423774)(879.86000939,97.0992378)(879.89,97.02924123)
\curveto(880.01000924,96.75923814)(880.18000907,96.53923836)(880.4,96.36924123)
\curveto(880.61000864,96.20923869)(880.8550084,96.07423883)(881.135,95.96424123)
\curveto(881.24500801,95.91423899)(881.36500788,95.87423903)(881.495,95.84424123)
\curveto(881.61500763,95.82423908)(881.74000751,95.7992391)(881.87,95.76924123)
\curveto(881.92000733,95.74923915)(881.97500728,95.73923916)(882.035,95.73924123)
\curveto(882.08500716,95.73923916)(882.13500712,95.73423917)(882.185,95.72424123)
\curveto(882.27500697,95.71423919)(882.37000688,95.7042392)(882.47,95.69424123)
\curveto(882.56000669,95.68423922)(882.6550066,95.67423923)(882.755,95.66424123)
\curveto(882.83500641,95.66423924)(882.92000633,95.65923924)(883.01,95.64924123)
\lineto(883.25,95.64924123)
\lineto(883.43,95.64924123)
\curveto(883.46000579,95.63923926)(883.49500575,95.63423927)(883.535,95.63424123)
\lineto(883.67,95.63424123)
\lineto(884.12,95.63424123)
\curveto(884.20000505,95.63423927)(884.28500496,95.62923927)(884.375,95.61924123)
\curveto(884.4550048,95.61923928)(884.53000472,95.62923927)(884.6,95.64924123)
\lineto(884.87,95.64924123)
\curveto(884.89000436,95.64923925)(884.92000433,95.64423926)(884.96,95.63424123)
\curveto(884.99000426,95.63423927)(885.01500423,95.63923926)(885.035,95.64924123)
\curveto(885.13500412,95.65923924)(885.23500401,95.66423924)(885.335,95.66424123)
\curveto(885.42500382,95.67423923)(885.52500373,95.68423922)(885.635,95.69424123)
\curveto(885.75500349,95.72423918)(885.88000337,95.73923916)(886.01,95.73924123)
\curveto(886.13000312,95.74923915)(886.245003,95.77423913)(886.355,95.81424123)
\curveto(886.6550026,95.89423901)(886.92000233,95.97923892)(887.15,96.06924123)
\curveto(887.38000187,96.16923873)(887.59500166,96.31423859)(887.795,96.50424123)
\curveto(887.99500126,96.71423819)(888.14500111,96.97923792)(888.245,97.29924123)
\curveto(888.26500099,97.33923756)(888.27500098,97.37423753)(888.275,97.40424123)
\curveto(888.26500099,97.44423746)(888.27000098,97.48923741)(888.29,97.53924123)
\curveto(888.30000095,97.57923732)(888.31000094,97.64923725)(888.32,97.74924123)
\curveto(888.33000092,97.85923704)(888.32500093,97.94423696)(888.305,98.00424123)
\curveto(888.28500096,98.07423683)(888.27500098,98.14423676)(888.275,98.21424123)
\curveto(888.26500099,98.28423662)(888.250001,98.34923655)(888.23,98.40924123)
\curveto(888.17000108,98.60923629)(888.08500116,98.78923611)(887.975,98.94924123)
\curveto(887.95500129,98.97923592)(887.93500132,99.0042359)(887.915,99.02424123)
\lineto(887.855,99.08424123)
\curveto(887.83500141,99.12423578)(887.79500146,99.17423573)(887.735,99.23424123)
\curveto(887.59500166,99.33423557)(887.46500179,99.41923548)(887.345,99.48924123)
\curveto(887.22500202,99.55923534)(887.08000217,99.62923527)(886.91,99.69924123)
\curveto(886.84000241,99.72923517)(886.77000248,99.74923515)(886.7,99.75924123)
\curveto(886.63000262,99.77923512)(886.55500269,99.7992351)(886.475,99.81924123)
}
}
{
\newrgbcolor{curcolor}{0 0 0}
\pscustom[linestyle=none,fillstyle=solid,fillcolor=curcolor]
{
\newpath
\moveto(878.63,106.77385061)
\curveto(878.63001062,106.87384575)(878.64001061,106.96884566)(878.66,107.05885061)
\curveto(878.67001058,107.14884548)(878.70001055,107.21384541)(878.75,107.25385061)
\curveto(878.83001042,107.31384531)(878.93501031,107.34384528)(879.065,107.34385061)
\lineto(879.455,107.34385061)
\lineto(880.955,107.34385061)
\lineto(887.345,107.34385061)
\lineto(888.515,107.34385061)
\lineto(888.83,107.34385061)
\curveto(888.93000032,107.35384527)(889.01000024,107.33884529)(889.07,107.29885061)
\curveto(889.1500001,107.24884538)(889.20000005,107.17384545)(889.22,107.07385061)
\curveto(889.23000002,106.98384564)(889.23500001,106.87384575)(889.235,106.74385061)
\lineto(889.235,106.51885061)
\curveto(889.21500004,106.43884619)(889.20000005,106.36884626)(889.19,106.30885061)
\curveto(889.17000008,106.24884638)(889.13000012,106.19884643)(889.07,106.15885061)
\curveto(889.01000024,106.11884651)(888.93500032,106.09884653)(888.845,106.09885061)
\lineto(888.545,106.09885061)
\lineto(887.45,106.09885061)
\lineto(882.11,106.09885061)
\curveto(882.02000723,106.07884655)(881.9450073,106.06384656)(881.885,106.05385061)
\curveto(881.81500743,106.05384657)(881.75500749,106.0238466)(881.705,105.96385061)
\curveto(881.6550076,105.89384673)(881.63000762,105.80384682)(881.63,105.69385061)
\curveto(881.62000763,105.59384703)(881.61500763,105.48384714)(881.615,105.36385061)
\lineto(881.615,104.22385061)
\lineto(881.615,103.72885061)
\curveto(881.60500765,103.56884906)(881.5450077,103.45884917)(881.435,103.39885061)
\curveto(881.40500785,103.37884925)(881.37500788,103.36884926)(881.345,103.36885061)
\curveto(881.30500794,103.36884926)(881.26000799,103.36384926)(881.21,103.35385061)
\curveto(881.09000816,103.33384929)(880.98000827,103.33884929)(880.88,103.36885061)
\curveto(880.78000847,103.40884922)(880.71000854,103.46384916)(880.67,103.53385061)
\curveto(880.62000863,103.61384901)(880.59500866,103.73384889)(880.595,103.89385061)
\curveto(880.59500866,104.05384857)(880.58000867,104.18884844)(880.55,104.29885061)
\curveto(880.54000871,104.34884828)(880.53500872,104.40384822)(880.535,104.46385061)
\curveto(880.52500873,104.5238481)(880.51000874,104.58384804)(880.49,104.64385061)
\curveto(880.44000881,104.79384783)(880.39000886,104.93884769)(880.34,105.07885061)
\curveto(880.28000897,105.21884741)(880.21000904,105.35384727)(880.13,105.48385061)
\curveto(880.04000921,105.623847)(879.93500932,105.74384688)(879.815,105.84385061)
\curveto(879.69500955,105.94384668)(879.56500968,106.03884659)(879.425,106.12885061)
\curveto(879.32500993,106.18884644)(879.21501003,106.23384639)(879.095,106.26385061)
\curveto(878.97501028,106.30384632)(878.87001038,106.35384627)(878.78,106.41385061)
\curveto(878.72001053,106.46384616)(878.68001057,106.53384609)(878.66,106.62385061)
\curveto(878.6500106,106.64384598)(878.64501061,106.66884596)(878.645,106.69885061)
\curveto(878.64501061,106.7288459)(878.64001061,106.75384587)(878.63,106.77385061)
}
}
{
\newrgbcolor{curcolor}{0 0 0}
\pscustom[linestyle=none,fillstyle=solid,fillcolor=curcolor]
{
\newpath
\moveto(878.63,115.12345998)
\curveto(878.63001062,115.22345513)(878.64001061,115.31845503)(878.66,115.40845998)
\curveto(878.67001058,115.49845485)(878.70001055,115.56345479)(878.75,115.60345998)
\curveto(878.83001042,115.66345469)(878.93501031,115.69345466)(879.065,115.69345998)
\lineto(879.455,115.69345998)
\lineto(880.955,115.69345998)
\lineto(887.345,115.69345998)
\lineto(888.515,115.69345998)
\lineto(888.83,115.69345998)
\curveto(888.93000032,115.70345465)(889.01000024,115.68845466)(889.07,115.64845998)
\curveto(889.1500001,115.59845475)(889.20000005,115.52345483)(889.22,115.42345998)
\curveto(889.23000002,115.33345502)(889.23500001,115.22345513)(889.235,115.09345998)
\lineto(889.235,114.86845998)
\curveto(889.21500004,114.78845556)(889.20000005,114.71845563)(889.19,114.65845998)
\curveto(889.17000008,114.59845575)(889.13000012,114.5484558)(889.07,114.50845998)
\curveto(889.01000024,114.46845588)(888.93500032,114.4484559)(888.845,114.44845998)
\lineto(888.545,114.44845998)
\lineto(887.45,114.44845998)
\lineto(882.11,114.44845998)
\curveto(882.02000723,114.42845592)(881.9450073,114.41345594)(881.885,114.40345998)
\curveto(881.81500743,114.40345595)(881.75500749,114.37345598)(881.705,114.31345998)
\curveto(881.6550076,114.24345611)(881.63000762,114.1534562)(881.63,114.04345998)
\curveto(881.62000763,113.94345641)(881.61500763,113.83345652)(881.615,113.71345998)
\lineto(881.615,112.57345998)
\lineto(881.615,112.07845998)
\curveto(881.60500765,111.91845843)(881.5450077,111.80845854)(881.435,111.74845998)
\curveto(881.40500785,111.72845862)(881.37500788,111.71845863)(881.345,111.71845998)
\curveto(881.30500794,111.71845863)(881.26000799,111.71345864)(881.21,111.70345998)
\curveto(881.09000816,111.68345867)(880.98000827,111.68845866)(880.88,111.71845998)
\curveto(880.78000847,111.75845859)(880.71000854,111.81345854)(880.67,111.88345998)
\curveto(880.62000863,111.96345839)(880.59500866,112.08345827)(880.595,112.24345998)
\curveto(880.59500866,112.40345795)(880.58000867,112.53845781)(880.55,112.64845998)
\curveto(880.54000871,112.69845765)(880.53500872,112.7534576)(880.535,112.81345998)
\curveto(880.52500873,112.87345748)(880.51000874,112.93345742)(880.49,112.99345998)
\curveto(880.44000881,113.14345721)(880.39000886,113.28845706)(880.34,113.42845998)
\curveto(880.28000897,113.56845678)(880.21000904,113.70345665)(880.13,113.83345998)
\curveto(880.04000921,113.97345638)(879.93500932,114.09345626)(879.815,114.19345998)
\curveto(879.69500955,114.29345606)(879.56500968,114.38845596)(879.425,114.47845998)
\curveto(879.32500993,114.53845581)(879.21501003,114.58345577)(879.095,114.61345998)
\curveto(878.97501028,114.6534557)(878.87001038,114.70345565)(878.78,114.76345998)
\curveto(878.72001053,114.81345554)(878.68001057,114.88345547)(878.66,114.97345998)
\curveto(878.6500106,114.99345536)(878.64501061,115.01845533)(878.645,115.04845998)
\curveto(878.64501061,115.07845527)(878.64001061,115.10345525)(878.63,115.12345998)
}
}
{
\newrgbcolor{curcolor}{0 0 0}
\pscustom[linestyle=none,fillstyle=solid,fillcolor=curcolor]
{
\newpath
\moveto(899.46631592,42.29681936)
\curveto(899.46632661,42.36681368)(899.46632661,42.4468136)(899.46631592,42.53681936)
\curveto(899.45632662,42.62681342)(899.45632662,42.71181333)(899.46631592,42.79181936)
\curveto(899.46632661,42.88181316)(899.4763266,42.96181308)(899.49631592,43.03181936)
\curveto(899.51632656,43.11181293)(899.54632653,43.16681288)(899.58631592,43.19681936)
\curveto(899.63632644,43.22681282)(899.71132637,43.2468128)(899.81131592,43.25681936)
\curveto(899.90132618,43.27681277)(900.00632607,43.28681276)(900.12631592,43.28681936)
\curveto(900.23632584,43.29681275)(900.35132573,43.29681275)(900.47131592,43.28681936)
\lineto(900.77131592,43.28681936)
\lineto(903.78631592,43.28681936)
\lineto(906.68131592,43.28681936)
\curveto(907.01131907,43.28681276)(907.33631874,43.28181276)(907.65631592,43.27181936)
\curveto(907.96631811,43.27181277)(908.24631783,43.23181281)(908.49631592,43.15181936)
\curveto(908.84631723,43.03181301)(909.14131694,42.87681317)(909.38131592,42.68681936)
\curveto(909.61131647,42.49681355)(909.81131627,42.25681379)(909.98131592,41.96681936)
\curveto(910.03131605,41.90681414)(910.06631601,41.8418142)(910.08631592,41.77181936)
\curveto(910.10631597,41.71181433)(910.13131595,41.6418144)(910.16131592,41.56181936)
\curveto(910.21131587,41.4418146)(910.24631583,41.31181473)(910.26631592,41.17181936)
\curveto(910.29631578,41.041815)(910.32631575,40.90681514)(910.35631592,40.76681936)
\curveto(910.3763157,40.71681533)(910.3813157,40.66681538)(910.37131592,40.61681936)
\curveto(910.36131572,40.56681548)(910.36131572,40.51181553)(910.37131592,40.45181936)
\curveto(910.3813157,40.43181561)(910.3813157,40.40681564)(910.37131592,40.37681936)
\curveto(910.37131571,40.3468157)(910.3763157,40.32181572)(910.38631592,40.30181936)
\curveto(910.39631568,40.26181578)(910.40131568,40.20681584)(910.40131592,40.13681936)
\curveto(910.40131568,40.06681598)(910.39631568,40.01181603)(910.38631592,39.97181936)
\curveto(910.3763157,39.92181612)(910.3763157,39.86681618)(910.38631592,39.80681936)
\curveto(910.39631568,39.7468163)(910.39131569,39.69181635)(910.37131592,39.64181936)
\curveto(910.34131574,39.51181653)(910.32131576,39.38681666)(910.31131592,39.26681936)
\curveto(910.30131578,39.1468169)(910.2763158,39.03181701)(910.23631592,38.92181936)
\curveto(910.11631596,38.55181749)(909.94631613,38.23181781)(909.72631592,37.96181936)
\curveto(909.50631657,37.69181835)(909.22631685,37.48181856)(908.88631592,37.33181936)
\curveto(908.76631731,37.28181876)(908.64131744,37.23681881)(908.51131592,37.19681936)
\curveto(908.3813177,37.16681888)(908.24631783,37.13181891)(908.10631592,37.09181936)
\curveto(908.05631802,37.08181896)(908.01631806,37.07681897)(907.98631592,37.07681936)
\curveto(907.94631813,37.07681897)(907.90131818,37.07181897)(907.85131592,37.06181936)
\curveto(907.82131826,37.05181899)(907.78631829,37.046819)(907.74631592,37.04681936)
\curveto(907.69631838,37.046819)(907.65631842,37.041819)(907.62631592,37.03181936)
\lineto(907.46131592,37.03181936)
\curveto(907.3813187,37.01181903)(907.2813188,37.00681904)(907.16131592,37.01681936)
\curveto(907.03131905,37.02681902)(906.94131914,37.041819)(906.89131592,37.06181936)
\curveto(906.80131928,37.08181896)(906.73631934,37.13681891)(906.69631592,37.22681936)
\curveto(906.6763194,37.25681879)(906.67131941,37.28681876)(906.68131592,37.31681936)
\curveto(906.6813194,37.3468187)(906.6763194,37.38681866)(906.66631592,37.43681936)
\curveto(906.65631942,37.47681857)(906.65131943,37.51681853)(906.65131592,37.55681936)
\lineto(906.65131592,37.70681936)
\curveto(906.65131943,37.82681822)(906.65631942,37.9468181)(906.66631592,38.06681936)
\curveto(906.66631941,38.19681785)(906.70131938,38.28681776)(906.77131592,38.33681936)
\curveto(906.83131925,38.37681767)(906.89131919,38.39681765)(906.95131592,38.39681936)
\curveto(907.01131907,38.39681765)(907.081319,38.40681764)(907.16131592,38.42681936)
\curveto(907.19131889,38.43681761)(907.22631885,38.43681761)(907.26631592,38.42681936)
\curveto(907.29631878,38.42681762)(907.32131876,38.43181761)(907.34131592,38.44181936)
\lineto(907.55131592,38.44181936)
\curveto(907.60131848,38.46181758)(907.65131843,38.46681758)(907.70131592,38.45681936)
\curveto(907.74131834,38.45681759)(907.78631829,38.46681758)(907.83631592,38.48681936)
\curveto(907.96631811,38.51681753)(908.09131799,38.5468175)(908.21131592,38.57681936)
\curveto(908.32131776,38.60681744)(908.42631765,38.65181739)(908.52631592,38.71181936)
\curveto(908.81631726,38.88181716)(909.02131706,39.15181689)(909.14131592,39.52181936)
\curveto(909.16131692,39.57181647)(909.1763169,39.62181642)(909.18631592,39.67181936)
\curveto(909.18631689,39.73181631)(909.19631688,39.78681626)(909.21631592,39.83681936)
\lineto(909.21631592,39.91181936)
\curveto(909.22631685,39.98181606)(909.23631684,40.07681597)(909.24631592,40.19681936)
\curveto(909.24631683,40.32681572)(909.23631684,40.42681562)(909.21631592,40.49681936)
\curveto(909.19631688,40.56681548)(909.1813169,40.63681541)(909.17131592,40.70681936)
\curveto(909.15131693,40.78681526)(909.13131695,40.85681519)(909.11131592,40.91681936)
\curveto(908.95131713,41.29681475)(908.6763174,41.57181447)(908.28631592,41.74181936)
\curveto(908.15631792,41.79181425)(908.00131808,41.82681422)(907.82131592,41.84681936)
\curveto(907.64131844,41.87681417)(907.45631862,41.89181415)(907.26631592,41.89181936)
\curveto(907.06631901,41.90181414)(906.86631921,41.90181414)(906.66631592,41.89181936)
\lineto(906.09631592,41.89181936)
\lineto(901.85131592,41.89181936)
\lineto(900.30631592,41.89181936)
\curveto(900.19632588,41.89181415)(900.076326,41.88681416)(899.94631592,41.87681936)
\curveto(899.81632626,41.86681418)(899.71132637,41.88681416)(899.63131592,41.93681936)
\curveto(899.56132652,41.99681405)(899.51132657,42.07681397)(899.48131592,42.17681936)
\curveto(899.4813266,42.19681385)(899.4813266,42.21681383)(899.48131592,42.23681936)
\curveto(899.4813266,42.25681379)(899.4763266,42.27681377)(899.46631592,42.29681936)
}
}
{
\newrgbcolor{curcolor}{0 0 0}
\pscustom[linestyle=none,fillstyle=solid,fillcolor=curcolor]
{
\newpath
\moveto(902.42131592,45.83049123)
\lineto(902.42131592,46.26549123)
\curveto(902.42132366,46.41548927)(902.46132362,46.52048916)(902.54131592,46.58049123)
\curveto(902.62132346,46.63048905)(902.72132336,46.65548903)(902.84131592,46.65549123)
\curveto(902.96132312,46.66548902)(903.081323,46.67048901)(903.20131592,46.67049123)
\lineto(904.62631592,46.67049123)
\lineto(906.89131592,46.67049123)
\lineto(907.58131592,46.67049123)
\curveto(907.81131827,46.67048901)(908.01131807,46.69548899)(908.18131592,46.74549123)
\curveto(908.63131745,46.90548878)(908.94631713,47.20548848)(909.12631592,47.64549123)
\curveto(909.21631686,47.86548782)(909.25131683,48.13048755)(909.23131592,48.44049123)
\curveto(909.20131688,48.75048693)(909.14631693,49.00048668)(909.06631592,49.19049123)
\curveto(908.92631715,49.52048616)(908.75131733,49.7804859)(908.54131592,49.97049123)
\curveto(908.32131776,50.17048551)(908.03631804,50.32548536)(907.68631592,50.43549123)
\curveto(907.60631847,50.46548522)(907.52631855,50.4854852)(907.44631592,50.49549123)
\curveto(907.36631871,50.50548518)(907.2813188,50.52048516)(907.19131592,50.54049123)
\curveto(907.14131894,50.55048513)(907.09631898,50.55048513)(907.05631592,50.54049123)
\curveto(907.01631906,50.54048514)(906.97131911,50.55048513)(906.92131592,50.57049123)
\lineto(906.60631592,50.57049123)
\curveto(906.52631955,50.59048509)(906.43631964,50.59548509)(906.33631592,50.58549123)
\curveto(906.22631985,50.57548511)(906.12631995,50.57048511)(906.03631592,50.57049123)
\lineto(904.86631592,50.57049123)
\lineto(903.27631592,50.57049123)
\curveto(903.15632292,50.57048511)(903.03132305,50.56548512)(902.90131592,50.55549123)
\curveto(902.76132332,50.55548513)(902.65132343,50.5804851)(902.57131592,50.63049123)
\curveto(902.52132356,50.67048501)(902.49132359,50.71548497)(902.48131592,50.76549123)
\curveto(902.46132362,50.82548486)(902.44132364,50.89548479)(902.42131592,50.97549123)
\lineto(902.42131592,51.20049123)
\curveto(902.42132366,51.32048436)(902.42632365,51.42548426)(902.43631592,51.51549123)
\curveto(902.44632363,51.61548407)(902.49132359,51.69048399)(902.57131592,51.74049123)
\curveto(902.62132346,51.79048389)(902.69632338,51.81548387)(902.79631592,51.81549123)
\lineto(903.08131592,51.81549123)
\lineto(904.10131592,51.81549123)
\lineto(908.13631592,51.81549123)
\lineto(909.48631592,51.81549123)
\curveto(909.60631647,51.81548387)(909.72131636,51.81048387)(909.83131592,51.80049123)
\curveto(909.93131615,51.80048388)(910.00631607,51.76548392)(910.05631592,51.69549123)
\curveto(910.08631599,51.65548403)(910.11131597,51.59548409)(910.13131592,51.51549123)
\curveto(910.14131594,51.43548425)(910.15131593,51.34548434)(910.16131592,51.24549123)
\curveto(910.16131592,51.15548453)(910.15631592,51.06548462)(910.14631592,50.97549123)
\curveto(910.13631594,50.89548479)(910.11631596,50.83548485)(910.08631592,50.79549123)
\curveto(910.04631603,50.74548494)(909.9813161,50.70048498)(909.89131592,50.66049123)
\curveto(909.85131623,50.65048503)(909.79631628,50.64048504)(909.72631592,50.63049123)
\curveto(909.65631642,50.63048505)(909.59131649,50.62548506)(909.53131592,50.61549123)
\curveto(909.46131662,50.60548508)(909.40631667,50.5854851)(909.36631592,50.55549123)
\curveto(909.32631675,50.52548516)(909.31131677,50.4804852)(909.32131592,50.42049123)
\curveto(909.34131674,50.34048534)(909.40131668,50.26048542)(909.50131592,50.18049123)
\curveto(909.59131649,50.10048558)(909.66131642,50.02548566)(909.71131592,49.95549123)
\curveto(909.87131621,49.73548595)(910.01131607,49.4854862)(910.13131592,49.20549123)
\curveto(910.1813159,49.09548659)(910.21131587,48.9804867)(910.22131592,48.86049123)
\curveto(910.24131584,48.75048693)(910.26631581,48.63548705)(910.29631592,48.51549123)
\curveto(910.30631577,48.46548722)(910.30631577,48.41048727)(910.29631592,48.35049123)
\curveto(910.28631579,48.30048738)(910.29131579,48.25048743)(910.31131592,48.20049123)
\curveto(910.33131575,48.10048758)(910.33131575,48.01048767)(910.31131592,47.93049123)
\lineto(910.31131592,47.78049123)
\curveto(910.29131579,47.73048795)(910.2813158,47.67048801)(910.28131592,47.60049123)
\curveto(910.2813158,47.54048814)(910.2763158,47.4854882)(910.26631592,47.43549123)
\curveto(910.24631583,47.39548829)(910.23631584,47.35548833)(910.23631592,47.31549123)
\curveto(910.24631583,47.2854884)(910.24131584,47.24548844)(910.22131592,47.19549123)
\lineto(910.16131592,46.95549123)
\curveto(910.14131594,46.8854888)(910.11131597,46.81048887)(910.07131592,46.73049123)
\curveto(909.96131612,46.47048921)(909.81631626,46.25048943)(909.63631592,46.07049123)
\curveto(909.44631663,45.90048978)(909.22131686,45.76048992)(908.96131592,45.65049123)
\curveto(908.87131721,45.61049007)(908.7813173,45.5804901)(908.69131592,45.56049123)
\lineto(908.39131592,45.50049123)
\curveto(908.33131775,45.4804902)(908.2763178,45.47049021)(908.22631592,45.47049123)
\curveto(908.16631791,45.4804902)(908.10131798,45.47549021)(908.03131592,45.45549123)
\curveto(908.01131807,45.44549024)(907.98631809,45.44049024)(907.95631592,45.44049123)
\curveto(907.91631816,45.44049024)(907.8813182,45.43549025)(907.85131592,45.42549123)
\lineto(907.70131592,45.42549123)
\curveto(907.66131842,45.41549027)(907.61631846,45.41049027)(907.56631592,45.41049123)
\curveto(907.50631857,45.42049026)(907.45131863,45.42549026)(907.40131592,45.42549123)
\lineto(906.80131592,45.42549123)
\lineto(904.04131592,45.42549123)
\lineto(903.08131592,45.42549123)
\lineto(902.81131592,45.42549123)
\curveto(902.72132336,45.42549026)(902.64632343,45.44549024)(902.58631592,45.48549123)
\curveto(902.51632356,45.52549016)(902.46632361,45.60049008)(902.43631592,45.71049123)
\curveto(902.42632365,45.73048995)(902.42632365,45.75048993)(902.43631592,45.77049123)
\curveto(902.43632364,45.79048989)(902.43132365,45.81048987)(902.42131592,45.83049123)
}
}
{
\newrgbcolor{curcolor}{0 0 0}
\pscustom[linestyle=none,fillstyle=solid,fillcolor=curcolor]
{
\newpath
\moveto(899.46631592,54.28510061)
\curveto(899.46632661,54.41509899)(899.46632661,54.55009886)(899.46631592,54.69010061)
\curveto(899.46632661,54.84009857)(899.50132658,54.95009846)(899.57131592,55.02010061)
\curveto(899.64132644,55.07009834)(899.73632634,55.09509831)(899.85631592,55.09510061)
\curveto(899.96632611,55.1050983)(900.081326,55.1100983)(900.20131592,55.11010061)
\lineto(901.53631592,55.11010061)
\lineto(907.61131592,55.11010061)
\lineto(909.29131592,55.11010061)
\lineto(909.68131592,55.11010061)
\curveto(909.82131626,55.1100983)(909.93131615,55.08509832)(910.01131592,55.03510061)
\curveto(910.06131602,55.0050984)(910.09131599,54.96009845)(910.10131592,54.90010061)
\curveto(910.11131597,54.85009856)(910.12631595,54.78509862)(910.14631592,54.70510061)
\lineto(910.14631592,54.49510061)
\lineto(910.14631592,54.18010061)
\curveto(910.13631594,54.08009933)(910.10131598,54.0050994)(910.04131592,53.95510061)
\curveto(909.96131612,53.9050995)(909.86131622,53.87509953)(909.74131592,53.86510061)
\lineto(909.36631592,53.86510061)
\lineto(907.98631592,53.86510061)
\lineto(901.74631592,53.86510061)
\lineto(900.27631592,53.86510061)
\curveto(900.16632591,53.86509954)(900.05132603,53.86009955)(899.93131592,53.85010061)
\curveto(899.80132628,53.85009956)(899.70132638,53.87509953)(899.63131592,53.92510061)
\curveto(899.57132651,53.96509944)(899.52132656,54.04009937)(899.48131592,54.15010061)
\curveto(899.47132661,54.17009924)(899.47132661,54.19009922)(899.48131592,54.21010061)
\curveto(899.4813266,54.24009917)(899.4763266,54.26509914)(899.46631592,54.28510061)
}
}
{
\newrgbcolor{curcolor}{0 0 0}
\pscustom[linestyle=none,fillstyle=solid,fillcolor=curcolor]
{
}
}
{
\newrgbcolor{curcolor}{0 0 0}
\pscustom[linestyle=none,fillstyle=solid,fillcolor=curcolor]
{
\newpath
\moveto(899.54131592,64.24510061)
\curveto(899.53132655,64.93509597)(899.65132643,65.53509537)(899.90131592,66.04510061)
\curveto(900.15132593,66.56509434)(900.48632559,66.96009395)(900.90631592,67.23010061)
\curveto(900.98632509,67.28009363)(901.076325,67.32509358)(901.17631592,67.36510061)
\curveto(901.26632481,67.4050935)(901.36132472,67.45009346)(901.46131592,67.50010061)
\curveto(901.56132452,67.54009337)(901.66132442,67.57009334)(901.76131592,67.59010061)
\curveto(901.86132422,67.6100933)(901.96632411,67.63009328)(902.07631592,67.65010061)
\curveto(902.12632395,67.67009324)(902.17132391,67.67509323)(902.21131592,67.66510061)
\curveto(902.25132383,67.65509325)(902.29632378,67.66009325)(902.34631592,67.68010061)
\curveto(902.39632368,67.69009322)(902.4813236,67.69509321)(902.60131592,67.69510061)
\curveto(902.71132337,67.69509321)(902.79632328,67.69009322)(902.85631592,67.68010061)
\curveto(902.91632316,67.66009325)(902.9763231,67.65009326)(903.03631592,67.65010061)
\curveto(903.09632298,67.66009325)(903.15632292,67.65509325)(903.21631592,67.63510061)
\curveto(903.35632272,67.59509331)(903.49132259,67.56009335)(903.62131592,67.53010061)
\curveto(903.75132233,67.50009341)(903.8763222,67.46009345)(903.99631592,67.41010061)
\curveto(904.13632194,67.35009356)(904.26132182,67.28009363)(904.37131592,67.20010061)
\curveto(904.4813216,67.13009378)(904.59132149,67.05509385)(904.70131592,66.97510061)
\lineto(904.76131592,66.91510061)
\curveto(904.7813213,66.905094)(904.80132128,66.89009402)(904.82131592,66.87010061)
\curveto(904.9813211,66.75009416)(905.12632095,66.61509429)(905.25631592,66.46510061)
\curveto(905.38632069,66.31509459)(905.51132057,66.15509475)(905.63131592,65.98510061)
\curveto(905.85132023,65.67509523)(906.05632002,65.38009553)(906.24631592,65.10010061)
\curveto(906.38631969,64.87009604)(906.52131956,64.64009627)(906.65131592,64.41010061)
\curveto(906.7813193,64.19009672)(906.91631916,63.97009694)(907.05631592,63.75010061)
\curveto(907.22631885,63.50009741)(907.40631867,63.26009765)(907.59631592,63.03010061)
\curveto(907.78631829,62.8100981)(908.01131807,62.62009829)(908.27131592,62.46010061)
\curveto(908.33131775,62.42009849)(908.39131769,62.38509852)(908.45131592,62.35510061)
\curveto(908.50131758,62.32509858)(908.56631751,62.29509861)(908.64631592,62.26510061)
\curveto(908.71631736,62.24509866)(908.7763173,62.24009867)(908.82631592,62.25010061)
\curveto(908.89631718,62.27009864)(908.95131713,62.3050986)(908.99131592,62.35510061)
\curveto(909.02131706,62.4050985)(909.04131704,62.46509844)(909.05131592,62.53510061)
\lineto(909.05131592,62.77510061)
\lineto(909.05131592,63.52510061)
\lineto(909.05131592,66.33010061)
\lineto(909.05131592,66.99010061)
\curveto(909.05131703,67.08009383)(909.05631702,67.16509374)(909.06631592,67.24510061)
\curveto(909.06631701,67.32509358)(909.08631699,67.39009352)(909.12631592,67.44010061)
\curveto(909.16631691,67.49009342)(909.24131684,67.53009338)(909.35131592,67.56010061)
\curveto(909.45131663,67.60009331)(909.55131653,67.6100933)(909.65131592,67.59010061)
\lineto(909.78631592,67.59010061)
\curveto(909.85631622,67.57009334)(909.91631616,67.55009336)(909.96631592,67.53010061)
\curveto(910.01631606,67.5100934)(910.05631602,67.47509343)(910.08631592,67.42510061)
\curveto(910.12631595,67.37509353)(910.14631593,67.3050936)(910.14631592,67.21510061)
\lineto(910.14631592,66.94510061)
\lineto(910.14631592,66.04510061)
\lineto(910.14631592,62.53510061)
\lineto(910.14631592,61.47010061)
\curveto(910.14631593,61.39009952)(910.15131593,61.30009961)(910.16131592,61.20010061)
\curveto(910.16131592,61.10009981)(910.15131593,61.01509989)(910.13131592,60.94510061)
\curveto(910.06131602,60.73510017)(909.8813162,60.67010024)(909.59131592,60.75010061)
\curveto(909.55131653,60.76010015)(909.51631656,60.76010015)(909.48631592,60.75010061)
\curveto(909.44631663,60.75010016)(909.40131668,60.76010015)(909.35131592,60.78010061)
\curveto(909.27131681,60.80010011)(909.18631689,60.82010009)(909.09631592,60.84010061)
\curveto(909.00631707,60.86010005)(908.92131716,60.88510002)(908.84131592,60.91510061)
\curveto(908.35131773,61.07509983)(907.93631814,61.27509963)(907.59631592,61.51510061)
\curveto(907.34631873,61.69509921)(907.12131896,61.90009901)(906.92131592,62.13010061)
\curveto(906.71131937,62.36009855)(906.51631956,62.60009831)(906.33631592,62.85010061)
\curveto(906.15631992,63.1100978)(905.98632009,63.37509753)(905.82631592,63.64510061)
\curveto(905.65632042,63.92509698)(905.4813206,64.19509671)(905.30131592,64.45510061)
\curveto(905.22132086,64.56509634)(905.14632093,64.67009624)(905.07631592,64.77010061)
\curveto(905.00632107,64.88009603)(904.93132115,64.99009592)(904.85131592,65.10010061)
\curveto(904.82132126,65.14009577)(904.79132129,65.17509573)(904.76131592,65.20510061)
\curveto(904.72132136,65.24509566)(904.69132139,65.28509562)(904.67131592,65.32510061)
\curveto(904.56132152,65.46509544)(904.43632164,65.59009532)(904.29631592,65.70010061)
\curveto(904.26632181,65.72009519)(904.24132184,65.74509516)(904.22131592,65.77510061)
\curveto(904.19132189,65.8050951)(904.16132192,65.83009508)(904.13131592,65.85010061)
\curveto(904.03132205,65.93009498)(903.93132215,65.99509491)(903.83131592,66.04510061)
\curveto(903.73132235,66.1050948)(903.62132246,66.16009475)(903.50131592,66.21010061)
\curveto(903.43132265,66.24009467)(903.35632272,66.26009465)(903.27631592,66.27010061)
\lineto(903.03631592,66.33010061)
\lineto(902.94631592,66.33010061)
\curveto(902.91632316,66.34009457)(902.88632319,66.34509456)(902.85631592,66.34510061)
\curveto(902.78632329,66.36509454)(902.69132339,66.37009454)(902.57131592,66.36010061)
\curveto(902.44132364,66.36009455)(902.34132374,66.35009456)(902.27131592,66.33010061)
\curveto(902.19132389,66.3100946)(902.11632396,66.29009462)(902.04631592,66.27010061)
\curveto(901.96632411,66.26009465)(901.88632419,66.24009467)(901.80631592,66.21010061)
\curveto(901.56632451,66.10009481)(901.36632471,65.95009496)(901.20631592,65.76010061)
\curveto(901.03632504,65.58009533)(900.89632518,65.36009555)(900.78631592,65.10010061)
\curveto(900.76632531,65.03009588)(900.75132533,64.96009595)(900.74131592,64.89010061)
\curveto(900.72132536,64.82009609)(900.70132538,64.74509616)(900.68131592,64.66510061)
\curveto(900.66132542,64.58509632)(900.65132543,64.47509643)(900.65131592,64.33510061)
\curveto(900.65132543,64.2050967)(900.66132542,64.10009681)(900.68131592,64.02010061)
\curveto(900.69132539,63.96009695)(900.69632538,63.905097)(900.69631592,63.85510061)
\curveto(900.69632538,63.8050971)(900.70632537,63.75509715)(900.72631592,63.70510061)
\curveto(900.76632531,63.6050973)(900.80632527,63.5100974)(900.84631592,63.42010061)
\curveto(900.88632519,63.34009757)(900.93132515,63.26009765)(900.98131592,63.18010061)
\curveto(901.00132508,63.15009776)(901.02632505,63.12009779)(901.05631592,63.09010061)
\curveto(901.08632499,63.07009784)(901.11132497,63.04509786)(901.13131592,63.01510061)
\lineto(901.20631592,62.94010061)
\curveto(901.22632485,62.910098)(901.24632483,62.88509802)(901.26631592,62.86510061)
\lineto(901.47631592,62.71510061)
\curveto(901.53632454,62.67509823)(901.60132448,62.63009828)(901.67131592,62.58010061)
\curveto(901.76132432,62.52009839)(901.86632421,62.47009844)(901.98631592,62.43010061)
\curveto(902.09632398,62.40009851)(902.20632387,62.36509854)(902.31631592,62.32510061)
\curveto(902.42632365,62.28509862)(902.57132351,62.26009865)(902.75131592,62.25010061)
\curveto(902.92132316,62.24009867)(903.04632303,62.2100987)(903.12631592,62.16010061)
\curveto(903.20632287,62.1100988)(903.25132283,62.03509887)(903.26131592,61.93510061)
\curveto(903.27132281,61.83509907)(903.2763228,61.72509918)(903.27631592,61.60510061)
\curveto(903.2763228,61.56509934)(903.2813228,61.52509938)(903.29131592,61.48510061)
\curveto(903.29132279,61.44509946)(903.28632279,61.4100995)(903.27631592,61.38010061)
\curveto(903.25632282,61.33009958)(903.24632283,61.28009963)(903.24631592,61.23010061)
\curveto(903.24632283,61.19009972)(903.23632284,61.15009976)(903.21631592,61.11010061)
\curveto(903.15632292,61.02009989)(903.02132306,60.97509993)(902.81131592,60.97510061)
\lineto(902.69131592,60.97510061)
\curveto(902.63132345,60.98509992)(902.57132351,60.99009992)(902.51131592,60.99010061)
\curveto(902.44132364,61.00009991)(902.3763237,61.0100999)(902.31631592,61.02010061)
\curveto(902.20632387,61.04009987)(902.10632397,61.06009985)(902.01631592,61.08010061)
\curveto(901.91632416,61.10009981)(901.82132426,61.13009978)(901.73131592,61.17010061)
\curveto(901.66132442,61.19009972)(901.60132448,61.2100997)(901.55131592,61.23010061)
\lineto(901.37131592,61.29010061)
\curveto(901.11132497,61.4100995)(900.86632521,61.56509934)(900.63631592,61.75510061)
\curveto(900.40632567,61.95509895)(900.22132586,62.17009874)(900.08131592,62.40010061)
\curveto(900.00132608,62.5100984)(899.93632614,62.62509828)(899.88631592,62.74510061)
\lineto(899.73631592,63.13510061)
\curveto(899.68632639,63.24509766)(899.65632642,63.36009755)(899.64631592,63.48010061)
\curveto(899.62632645,63.60009731)(899.60132648,63.72509718)(899.57131592,63.85510061)
\curveto(899.57132651,63.92509698)(899.57132651,63.99009692)(899.57131592,64.05010061)
\curveto(899.56132652,64.1100968)(899.55132653,64.17509673)(899.54131592,64.24510061)
}
}
{
\newrgbcolor{curcolor}{0 0 0}
\pscustom[linestyle=none,fillstyle=solid,fillcolor=curcolor]
{
\newpath
\moveto(905.06131592,76.34470998)
\lineto(905.31631592,76.34470998)
\curveto(905.39632068,76.35470228)(905.47132061,76.34970228)(905.54131592,76.32970998)
\lineto(905.78131592,76.32970998)
\lineto(905.94631592,76.32970998)
\curveto(906.04632003,76.30970232)(906.15131993,76.29970233)(906.26131592,76.29970998)
\curveto(906.36131972,76.29970233)(906.46131962,76.28970234)(906.56131592,76.26970998)
\lineto(906.71131592,76.26970998)
\curveto(906.85131923,76.23970239)(906.99131909,76.21970241)(907.13131592,76.20970998)
\curveto(907.26131882,76.19970243)(907.39131869,76.17470246)(907.52131592,76.13470998)
\curveto(907.60131848,76.11470252)(907.68631839,76.09470254)(907.77631592,76.07470998)
\lineto(908.01631592,76.01470998)
\lineto(908.31631592,75.89470998)
\curveto(908.40631767,75.86470277)(908.49631758,75.8297028)(908.58631592,75.78970998)
\curveto(908.80631727,75.68970294)(909.02131706,75.55470308)(909.23131592,75.38470998)
\curveto(909.44131664,75.22470341)(909.61131647,75.04970358)(909.74131592,74.85970998)
\curveto(909.7813163,74.80970382)(909.82131626,74.74970388)(909.86131592,74.67970998)
\curveto(909.89131619,74.61970401)(909.92631615,74.55970407)(909.96631592,74.49970998)
\curveto(910.01631606,74.41970421)(910.05631602,74.32470431)(910.08631592,74.21470998)
\curveto(910.11631596,74.10470453)(910.14631593,73.99970463)(910.17631592,73.89970998)
\curveto(910.21631586,73.78970484)(910.24131584,73.67970495)(910.25131592,73.56970998)
\curveto(910.26131582,73.45970517)(910.2763158,73.34470529)(910.29631592,73.22470998)
\curveto(910.30631577,73.18470545)(910.30631577,73.13970549)(910.29631592,73.08970998)
\curveto(910.29631578,73.04970558)(910.30131578,73.00970562)(910.31131592,72.96970998)
\curveto(910.32131576,72.9297057)(910.32631575,72.87470576)(910.32631592,72.80470998)
\curveto(910.32631575,72.7347059)(910.32131576,72.68470595)(910.31131592,72.65470998)
\curveto(910.29131579,72.60470603)(910.28631579,72.55970607)(910.29631592,72.51970998)
\curveto(910.30631577,72.47970615)(910.30631577,72.44470619)(910.29631592,72.41470998)
\lineto(910.29631592,72.32470998)
\curveto(910.2763158,72.26470637)(910.26131582,72.19970643)(910.25131592,72.12970998)
\curveto(910.25131583,72.06970656)(910.24631583,72.00470663)(910.23631592,71.93470998)
\curveto(910.18631589,71.76470687)(910.13631594,71.60470703)(910.08631592,71.45470998)
\curveto(910.03631604,71.30470733)(909.97131611,71.15970747)(909.89131592,71.01970998)
\curveto(909.85131623,70.96970766)(909.82131626,70.91470772)(909.80131592,70.85470998)
\curveto(909.77131631,70.80470783)(909.73631634,70.75470788)(909.69631592,70.70470998)
\curveto(909.51631656,70.46470817)(909.29631678,70.26470837)(909.03631592,70.10470998)
\curveto(908.7763173,69.94470869)(908.49131759,69.80470883)(908.18131592,69.68470998)
\curveto(908.04131804,69.62470901)(907.90131818,69.57970905)(907.76131592,69.54970998)
\curveto(907.61131847,69.51970911)(907.45631862,69.48470915)(907.29631592,69.44470998)
\curveto(907.18631889,69.42470921)(907.076319,69.40970922)(906.96631592,69.39970998)
\curveto(906.85631922,69.38970924)(906.74631933,69.37470926)(906.63631592,69.35470998)
\curveto(906.59631948,69.34470929)(906.55631952,69.33970929)(906.51631592,69.33970998)
\curveto(906.4763196,69.34970928)(906.43631964,69.34970928)(906.39631592,69.33970998)
\curveto(906.34631973,69.3297093)(906.29631978,69.32470931)(906.24631592,69.32470998)
\lineto(906.08131592,69.32470998)
\curveto(906.03132005,69.30470933)(905.9813201,69.29970933)(905.93131592,69.30970998)
\curveto(905.87132021,69.31970931)(905.81632026,69.31970931)(905.76631592,69.30970998)
\curveto(905.72632035,69.29970933)(905.6813204,69.29970933)(905.63131592,69.30970998)
\curveto(905.5813205,69.31970931)(905.53132055,69.31470932)(905.48131592,69.29470998)
\curveto(905.41132067,69.27470936)(905.33632074,69.26970936)(905.25631592,69.27970998)
\curveto(905.16632091,69.28970934)(905.081321,69.29470934)(905.00131592,69.29470998)
\curveto(904.91132117,69.29470934)(904.81132127,69.28970934)(904.70131592,69.27970998)
\curveto(904.5813215,69.26970936)(904.4813216,69.27470936)(904.40131592,69.29470998)
\lineto(904.11631592,69.29470998)
\lineto(903.48631592,69.33970998)
\curveto(903.38632269,69.34970928)(903.29132279,69.35970927)(903.20131592,69.36970998)
\lineto(902.90131592,69.39970998)
\curveto(902.85132323,69.41970921)(902.80132328,69.42470921)(902.75131592,69.41470998)
\curveto(902.69132339,69.41470922)(902.63632344,69.42470921)(902.58631592,69.44470998)
\curveto(902.41632366,69.49470914)(902.25132383,69.5347091)(902.09131592,69.56470998)
\curveto(901.92132416,69.59470904)(901.76132432,69.64470899)(901.61131592,69.71470998)
\curveto(901.15132493,69.90470873)(900.7763253,70.12470851)(900.48631592,70.37470998)
\curveto(900.19632588,70.634708)(899.95132613,70.99470764)(899.75131592,71.45470998)
\curveto(899.70132638,71.58470705)(899.66632641,71.71470692)(899.64631592,71.84470998)
\curveto(899.62632645,71.98470665)(899.60132648,72.12470651)(899.57131592,72.26470998)
\curveto(899.56132652,72.3347063)(899.55632652,72.39970623)(899.55631592,72.45970998)
\curveto(899.55632652,72.51970611)(899.55132653,72.58470605)(899.54131592,72.65470998)
\curveto(899.52132656,73.48470515)(899.67132641,74.15470448)(899.99131592,74.66470998)
\curveto(900.30132578,75.17470346)(900.74132534,75.55470308)(901.31131592,75.80470998)
\curveto(901.43132465,75.85470278)(901.55632452,75.89970273)(901.68631592,75.93970998)
\curveto(901.81632426,75.97970265)(901.95132413,76.02470261)(902.09131592,76.07470998)
\curveto(902.17132391,76.09470254)(902.25632382,76.10970252)(902.34631592,76.11970998)
\lineto(902.58631592,76.17970998)
\curveto(902.69632338,76.20970242)(902.80632327,76.22470241)(902.91631592,76.22470998)
\curveto(903.02632305,76.2347024)(903.13632294,76.24970238)(903.24631592,76.26970998)
\curveto(903.29632278,76.28970234)(903.34132274,76.29470234)(903.38131592,76.28470998)
\curveto(903.42132266,76.28470235)(903.46132262,76.28970234)(903.50131592,76.29970998)
\curveto(903.55132253,76.30970232)(903.60632247,76.30970232)(903.66631592,76.29970998)
\curveto(903.71632236,76.29970233)(903.76632231,76.30470233)(903.81631592,76.31470998)
\lineto(903.95131592,76.31470998)
\curveto(904.01132207,76.3347023)(904.081322,76.3347023)(904.16131592,76.31470998)
\curveto(904.23132185,76.30470233)(904.29632178,76.30970232)(904.35631592,76.32970998)
\curveto(904.38632169,76.33970229)(904.42632165,76.34470229)(904.47631592,76.34470998)
\lineto(904.59631592,76.34470998)
\lineto(905.06131592,76.34470998)
\moveto(907.38631592,74.79970998)
\curveto(907.06631901,74.89970373)(906.70131938,74.95970367)(906.29131592,74.97970998)
\curveto(905.8813202,74.99970363)(905.47132061,75.00970362)(905.06131592,75.00970998)
\curveto(904.63132145,75.00970362)(904.21132187,74.99970363)(903.80131592,74.97970998)
\curveto(903.39132269,74.95970367)(903.00632307,74.91470372)(902.64631592,74.84470998)
\curveto(902.28632379,74.77470386)(901.96632411,74.66470397)(901.68631592,74.51470998)
\curveto(901.39632468,74.37470426)(901.16132492,74.17970445)(900.98131592,73.92970998)
\curveto(900.87132521,73.76970486)(900.79132529,73.58970504)(900.74131592,73.38970998)
\curveto(900.6813254,73.18970544)(900.65132543,72.94470569)(900.65131592,72.65470998)
\curveto(900.67132541,72.634706)(900.6813254,72.59970603)(900.68131592,72.54970998)
\curveto(900.67132541,72.49970613)(900.67132541,72.45970617)(900.68131592,72.42970998)
\curveto(900.70132538,72.34970628)(900.72132536,72.27470636)(900.74131592,72.20470998)
\curveto(900.75132533,72.14470649)(900.77132531,72.07970655)(900.80131592,72.00970998)
\curveto(900.92132516,71.73970689)(901.09132499,71.51970711)(901.31131592,71.34970998)
\curveto(901.52132456,71.18970744)(901.76632431,71.05470758)(902.04631592,70.94470998)
\curveto(902.15632392,70.89470774)(902.2763238,70.85470778)(902.40631592,70.82470998)
\curveto(902.52632355,70.80470783)(902.65132343,70.77970785)(902.78131592,70.74970998)
\curveto(902.83132325,70.7297079)(902.88632319,70.71970791)(902.94631592,70.71970998)
\curveto(902.99632308,70.71970791)(903.04632303,70.71470792)(903.09631592,70.70470998)
\curveto(903.18632289,70.69470794)(903.2813228,70.68470795)(903.38131592,70.67470998)
\curveto(903.47132261,70.66470797)(903.56632251,70.65470798)(903.66631592,70.64470998)
\curveto(903.74632233,70.64470799)(903.83132225,70.63970799)(903.92131592,70.62970998)
\lineto(904.16131592,70.62970998)
\lineto(904.34131592,70.62970998)
\curveto(904.37132171,70.61970801)(904.40632167,70.61470802)(904.44631592,70.61470998)
\lineto(904.58131592,70.61470998)
\lineto(905.03131592,70.61470998)
\curveto(905.11132097,70.61470802)(905.19632088,70.60970802)(905.28631592,70.59970998)
\curveto(905.36632071,70.59970803)(905.44132064,70.60970802)(905.51131592,70.62970998)
\lineto(905.78131592,70.62970998)
\curveto(905.80132028,70.629708)(905.83132025,70.62470801)(905.87131592,70.61470998)
\curveto(905.90132018,70.61470802)(905.92632015,70.61970801)(905.94631592,70.62970998)
\curveto(906.04632003,70.63970799)(906.14631993,70.64470799)(906.24631592,70.64470998)
\curveto(906.33631974,70.65470798)(906.43631964,70.66470797)(906.54631592,70.67470998)
\curveto(906.66631941,70.70470793)(906.79131929,70.71970791)(906.92131592,70.71970998)
\curveto(907.04131904,70.7297079)(907.15631892,70.75470788)(907.26631592,70.79470998)
\curveto(907.56631851,70.87470776)(907.83131825,70.95970767)(908.06131592,71.04970998)
\curveto(908.29131779,71.14970748)(908.50631757,71.29470734)(908.70631592,71.48470998)
\curveto(908.90631717,71.69470694)(909.05631702,71.95970667)(909.15631592,72.27970998)
\curveto(909.1763169,72.31970631)(909.18631689,72.35470628)(909.18631592,72.38470998)
\curveto(909.1763169,72.42470621)(909.1813169,72.46970616)(909.20131592,72.51970998)
\curveto(909.21131687,72.55970607)(909.22131686,72.629706)(909.23131592,72.72970998)
\curveto(909.24131684,72.83970579)(909.23631684,72.92470571)(909.21631592,72.98470998)
\curveto(909.19631688,73.05470558)(909.18631689,73.12470551)(909.18631592,73.19470998)
\curveto(909.1763169,73.26470537)(909.16131692,73.3297053)(909.14131592,73.38970998)
\curveto(909.081317,73.58970504)(908.99631708,73.76970486)(908.88631592,73.92970998)
\curveto(908.86631721,73.95970467)(908.84631723,73.98470465)(908.82631592,74.00470998)
\lineto(908.76631592,74.06470998)
\curveto(908.74631733,74.10470453)(908.70631737,74.15470448)(908.64631592,74.21470998)
\curveto(908.50631757,74.31470432)(908.3763177,74.39970423)(908.25631592,74.46970998)
\curveto(908.13631794,74.53970409)(907.99131809,74.60970402)(907.82131592,74.67970998)
\curveto(907.75131833,74.70970392)(907.6813184,74.7297039)(907.61131592,74.73970998)
\curveto(907.54131854,74.75970387)(907.46631861,74.77970385)(907.38631592,74.79970998)
}
}
{
\newrgbcolor{curcolor}{0 0 0}
\pscustom[linestyle=none,fillstyle=solid,fillcolor=curcolor]
{
\newpath
\moveto(908.51131592,78.63431936)
\lineto(908.51131592,79.26431936)
\lineto(908.51131592,79.45931936)
\curveto(908.51131757,79.52931683)(908.52131756,79.58931677)(908.54131592,79.63931936)
\curveto(908.5813175,79.70931665)(908.62131746,79.7593166)(908.66131592,79.78931936)
\curveto(908.71131737,79.82931653)(908.7763173,79.84931651)(908.85631592,79.84931936)
\curveto(908.93631714,79.8593165)(909.02131706,79.86431649)(909.11131592,79.86431936)
\lineto(909.83131592,79.86431936)
\curveto(910.31131577,79.86431649)(910.72131536,79.80431655)(911.06131592,79.68431936)
\curveto(911.40131468,79.56431679)(911.6763144,79.36931699)(911.88631592,79.09931936)
\curveto(911.93631414,79.02931733)(911.9813141,78.9593174)(912.02131592,78.88931936)
\curveto(912.07131401,78.82931753)(912.11631396,78.7543176)(912.15631592,78.66431936)
\curveto(912.16631391,78.64431771)(912.1763139,78.61431774)(912.18631592,78.57431936)
\curveto(912.20631387,78.53431782)(912.21131387,78.48931787)(912.20131592,78.43931936)
\curveto(912.17131391,78.34931801)(912.09631398,78.29431806)(911.97631592,78.27431936)
\curveto(911.86631421,78.2543181)(911.77131431,78.26931809)(911.69131592,78.31931936)
\curveto(911.62131446,78.34931801)(911.55631452,78.39431796)(911.49631592,78.45431936)
\curveto(911.44631463,78.52431783)(911.39631468,78.58931777)(911.34631592,78.64931936)
\curveto(911.29631478,78.71931764)(911.22131486,78.77931758)(911.12131592,78.82931936)
\curveto(911.03131505,78.88931747)(910.94131514,78.93931742)(910.85131592,78.97931936)
\curveto(910.82131526,78.99931736)(910.76131532,79.02431733)(910.67131592,79.05431936)
\curveto(910.59131549,79.08431727)(910.52131556,79.08931727)(910.46131592,79.06931936)
\curveto(910.32131576,79.03931732)(910.23131585,78.97931738)(910.19131592,78.88931936)
\curveto(910.16131592,78.80931755)(910.14631593,78.71931764)(910.14631592,78.61931936)
\curveto(910.14631593,78.51931784)(910.12131596,78.43431792)(910.07131592,78.36431936)
\curveto(910.00131608,78.27431808)(909.86131622,78.22931813)(909.65131592,78.22931936)
\lineto(909.09631592,78.22931936)
\lineto(908.87131592,78.22931936)
\curveto(908.79131729,78.23931812)(908.72631735,78.2593181)(908.67631592,78.28931936)
\curveto(908.59631748,78.34931801)(908.55131753,78.41931794)(908.54131592,78.49931936)
\curveto(908.53131755,78.51931784)(908.52631755,78.53931782)(908.52631592,78.55931936)
\curveto(908.52631755,78.58931777)(908.52131756,78.61431774)(908.51131592,78.63431936)
}
}
{
\newrgbcolor{curcolor}{0 0 0}
\pscustom[linestyle=none,fillstyle=solid,fillcolor=curcolor]
{
}
}
{
\newrgbcolor{curcolor}{0 0 0}
\pscustom[linestyle=none,fillstyle=solid,fillcolor=curcolor]
{
\newpath
\moveto(899.54131592,89.26463186)
\curveto(899.53132655,89.95462722)(899.65132643,90.55462662)(899.90131592,91.06463186)
\curveto(900.15132593,91.58462559)(900.48632559,91.9796252)(900.90631592,92.24963186)
\curveto(900.98632509,92.29962488)(901.076325,92.34462483)(901.17631592,92.38463186)
\curveto(901.26632481,92.42462475)(901.36132472,92.46962471)(901.46131592,92.51963186)
\curveto(901.56132452,92.55962462)(901.66132442,92.58962459)(901.76131592,92.60963186)
\curveto(901.86132422,92.62962455)(901.96632411,92.64962453)(902.07631592,92.66963186)
\curveto(902.12632395,92.68962449)(902.17132391,92.69462448)(902.21131592,92.68463186)
\curveto(902.25132383,92.6746245)(902.29632378,92.6796245)(902.34631592,92.69963186)
\curveto(902.39632368,92.70962447)(902.4813236,92.71462446)(902.60131592,92.71463186)
\curveto(902.71132337,92.71462446)(902.79632328,92.70962447)(902.85631592,92.69963186)
\curveto(902.91632316,92.6796245)(902.9763231,92.66962451)(903.03631592,92.66963186)
\curveto(903.09632298,92.6796245)(903.15632292,92.6746245)(903.21631592,92.65463186)
\curveto(903.35632272,92.61462456)(903.49132259,92.5796246)(903.62131592,92.54963186)
\curveto(903.75132233,92.51962466)(903.8763222,92.4796247)(903.99631592,92.42963186)
\curveto(904.13632194,92.36962481)(904.26132182,92.29962488)(904.37131592,92.21963186)
\curveto(904.4813216,92.14962503)(904.59132149,92.0746251)(904.70131592,91.99463186)
\lineto(904.76131592,91.93463186)
\curveto(904.7813213,91.92462525)(904.80132128,91.90962527)(904.82131592,91.88963186)
\curveto(904.9813211,91.76962541)(905.12632095,91.63462554)(905.25631592,91.48463186)
\curveto(905.38632069,91.33462584)(905.51132057,91.174626)(905.63131592,91.00463186)
\curveto(905.85132023,90.69462648)(906.05632002,90.39962678)(906.24631592,90.11963186)
\curveto(906.38631969,89.88962729)(906.52131956,89.65962752)(906.65131592,89.42963186)
\curveto(906.7813193,89.20962797)(906.91631916,88.98962819)(907.05631592,88.76963186)
\curveto(907.22631885,88.51962866)(907.40631867,88.2796289)(907.59631592,88.04963186)
\curveto(907.78631829,87.82962935)(908.01131807,87.63962954)(908.27131592,87.47963186)
\curveto(908.33131775,87.43962974)(908.39131769,87.40462977)(908.45131592,87.37463186)
\curveto(908.50131758,87.34462983)(908.56631751,87.31462986)(908.64631592,87.28463186)
\curveto(908.71631736,87.26462991)(908.7763173,87.25962992)(908.82631592,87.26963186)
\curveto(908.89631718,87.28962989)(908.95131713,87.32462985)(908.99131592,87.37463186)
\curveto(909.02131706,87.42462975)(909.04131704,87.48462969)(909.05131592,87.55463186)
\lineto(909.05131592,87.79463186)
\lineto(909.05131592,88.54463186)
\lineto(909.05131592,91.34963186)
\lineto(909.05131592,92.00963186)
\curveto(909.05131703,92.09962508)(909.05631702,92.18462499)(909.06631592,92.26463186)
\curveto(909.06631701,92.34462483)(909.08631699,92.40962477)(909.12631592,92.45963186)
\curveto(909.16631691,92.50962467)(909.24131684,92.54962463)(909.35131592,92.57963186)
\curveto(909.45131663,92.61962456)(909.55131653,92.62962455)(909.65131592,92.60963186)
\lineto(909.78631592,92.60963186)
\curveto(909.85631622,92.58962459)(909.91631616,92.56962461)(909.96631592,92.54963186)
\curveto(910.01631606,92.52962465)(910.05631602,92.49462468)(910.08631592,92.44463186)
\curveto(910.12631595,92.39462478)(910.14631593,92.32462485)(910.14631592,92.23463186)
\lineto(910.14631592,91.96463186)
\lineto(910.14631592,91.06463186)
\lineto(910.14631592,87.55463186)
\lineto(910.14631592,86.48963186)
\curveto(910.14631593,86.40963077)(910.15131593,86.31963086)(910.16131592,86.21963186)
\curveto(910.16131592,86.11963106)(910.15131593,86.03463114)(910.13131592,85.96463186)
\curveto(910.06131602,85.75463142)(909.8813162,85.68963149)(909.59131592,85.76963186)
\curveto(909.55131653,85.7796314)(909.51631656,85.7796314)(909.48631592,85.76963186)
\curveto(909.44631663,85.76963141)(909.40131668,85.7796314)(909.35131592,85.79963186)
\curveto(909.27131681,85.81963136)(909.18631689,85.83963134)(909.09631592,85.85963186)
\curveto(909.00631707,85.8796313)(908.92131716,85.90463127)(908.84131592,85.93463186)
\curveto(908.35131773,86.09463108)(907.93631814,86.29463088)(907.59631592,86.53463186)
\curveto(907.34631873,86.71463046)(907.12131896,86.91963026)(906.92131592,87.14963186)
\curveto(906.71131937,87.3796298)(906.51631956,87.61962956)(906.33631592,87.86963186)
\curveto(906.15631992,88.12962905)(905.98632009,88.39462878)(905.82631592,88.66463186)
\curveto(905.65632042,88.94462823)(905.4813206,89.21462796)(905.30131592,89.47463186)
\curveto(905.22132086,89.58462759)(905.14632093,89.68962749)(905.07631592,89.78963186)
\curveto(905.00632107,89.89962728)(904.93132115,90.00962717)(904.85131592,90.11963186)
\curveto(904.82132126,90.15962702)(904.79132129,90.19462698)(904.76131592,90.22463186)
\curveto(904.72132136,90.26462691)(904.69132139,90.30462687)(904.67131592,90.34463186)
\curveto(904.56132152,90.48462669)(904.43632164,90.60962657)(904.29631592,90.71963186)
\curveto(904.26632181,90.73962644)(904.24132184,90.76462641)(904.22131592,90.79463186)
\curveto(904.19132189,90.82462635)(904.16132192,90.84962633)(904.13131592,90.86963186)
\curveto(904.03132205,90.94962623)(903.93132215,91.01462616)(903.83131592,91.06463186)
\curveto(903.73132235,91.12462605)(903.62132246,91.179626)(903.50131592,91.22963186)
\curveto(903.43132265,91.25962592)(903.35632272,91.2796259)(903.27631592,91.28963186)
\lineto(903.03631592,91.34963186)
\lineto(902.94631592,91.34963186)
\curveto(902.91632316,91.35962582)(902.88632319,91.36462581)(902.85631592,91.36463186)
\curveto(902.78632329,91.38462579)(902.69132339,91.38962579)(902.57131592,91.37963186)
\curveto(902.44132364,91.3796258)(902.34132374,91.36962581)(902.27131592,91.34963186)
\curveto(902.19132389,91.32962585)(902.11632396,91.30962587)(902.04631592,91.28963186)
\curveto(901.96632411,91.2796259)(901.88632419,91.25962592)(901.80631592,91.22963186)
\curveto(901.56632451,91.11962606)(901.36632471,90.96962621)(901.20631592,90.77963186)
\curveto(901.03632504,90.59962658)(900.89632518,90.3796268)(900.78631592,90.11963186)
\curveto(900.76632531,90.04962713)(900.75132533,89.9796272)(900.74131592,89.90963186)
\curveto(900.72132536,89.83962734)(900.70132538,89.76462741)(900.68131592,89.68463186)
\curveto(900.66132542,89.60462757)(900.65132543,89.49462768)(900.65131592,89.35463186)
\curveto(900.65132543,89.22462795)(900.66132542,89.11962806)(900.68131592,89.03963186)
\curveto(900.69132539,88.9796282)(900.69632538,88.92462825)(900.69631592,88.87463186)
\curveto(900.69632538,88.82462835)(900.70632537,88.7746284)(900.72631592,88.72463186)
\curveto(900.76632531,88.62462855)(900.80632527,88.52962865)(900.84631592,88.43963186)
\curveto(900.88632519,88.35962882)(900.93132515,88.2796289)(900.98131592,88.19963186)
\curveto(901.00132508,88.16962901)(901.02632505,88.13962904)(901.05631592,88.10963186)
\curveto(901.08632499,88.08962909)(901.11132497,88.06462911)(901.13131592,88.03463186)
\lineto(901.20631592,87.95963186)
\curveto(901.22632485,87.92962925)(901.24632483,87.90462927)(901.26631592,87.88463186)
\lineto(901.47631592,87.73463186)
\curveto(901.53632454,87.69462948)(901.60132448,87.64962953)(901.67131592,87.59963186)
\curveto(901.76132432,87.53962964)(901.86632421,87.48962969)(901.98631592,87.44963186)
\curveto(902.09632398,87.41962976)(902.20632387,87.38462979)(902.31631592,87.34463186)
\curveto(902.42632365,87.30462987)(902.57132351,87.2796299)(902.75131592,87.26963186)
\curveto(902.92132316,87.25962992)(903.04632303,87.22962995)(903.12631592,87.17963186)
\curveto(903.20632287,87.12963005)(903.25132283,87.05463012)(903.26131592,86.95463186)
\curveto(903.27132281,86.85463032)(903.2763228,86.74463043)(903.27631592,86.62463186)
\curveto(903.2763228,86.58463059)(903.2813228,86.54463063)(903.29131592,86.50463186)
\curveto(903.29132279,86.46463071)(903.28632279,86.42963075)(903.27631592,86.39963186)
\curveto(903.25632282,86.34963083)(903.24632283,86.29963088)(903.24631592,86.24963186)
\curveto(903.24632283,86.20963097)(903.23632284,86.16963101)(903.21631592,86.12963186)
\curveto(903.15632292,86.03963114)(903.02132306,85.99463118)(902.81131592,85.99463186)
\lineto(902.69131592,85.99463186)
\curveto(902.63132345,86.00463117)(902.57132351,86.00963117)(902.51131592,86.00963186)
\curveto(902.44132364,86.01963116)(902.3763237,86.02963115)(902.31631592,86.03963186)
\curveto(902.20632387,86.05963112)(902.10632397,86.0796311)(902.01631592,86.09963186)
\curveto(901.91632416,86.11963106)(901.82132426,86.14963103)(901.73131592,86.18963186)
\curveto(901.66132442,86.20963097)(901.60132448,86.22963095)(901.55131592,86.24963186)
\lineto(901.37131592,86.30963186)
\curveto(901.11132497,86.42963075)(900.86632521,86.58463059)(900.63631592,86.77463186)
\curveto(900.40632567,86.9746302)(900.22132586,87.18962999)(900.08131592,87.41963186)
\curveto(900.00132608,87.52962965)(899.93632614,87.64462953)(899.88631592,87.76463186)
\lineto(899.73631592,88.15463186)
\curveto(899.68632639,88.26462891)(899.65632642,88.3796288)(899.64631592,88.49963186)
\curveto(899.62632645,88.61962856)(899.60132648,88.74462843)(899.57131592,88.87463186)
\curveto(899.57132651,88.94462823)(899.57132651,89.00962817)(899.57131592,89.06963186)
\curveto(899.56132652,89.12962805)(899.55132653,89.19462798)(899.54131592,89.26463186)
}
}
{
\newrgbcolor{curcolor}{0 0 0}
\pscustom[linestyle=none,fillstyle=solid,fillcolor=curcolor]
{
\newpath
\moveto(905.06131592,101.36424123)
\lineto(905.31631592,101.36424123)
\curveto(905.39632068,101.37423353)(905.47132061,101.36923353)(905.54131592,101.34924123)
\lineto(905.78131592,101.34924123)
\lineto(905.94631592,101.34924123)
\curveto(906.04632003,101.32923357)(906.15131993,101.31923358)(906.26131592,101.31924123)
\curveto(906.36131972,101.31923358)(906.46131962,101.30923359)(906.56131592,101.28924123)
\lineto(906.71131592,101.28924123)
\curveto(906.85131923,101.25923364)(906.99131909,101.23923366)(907.13131592,101.22924123)
\curveto(907.26131882,101.21923368)(907.39131869,101.19423371)(907.52131592,101.15424123)
\curveto(907.60131848,101.13423377)(907.68631839,101.11423379)(907.77631592,101.09424123)
\lineto(908.01631592,101.03424123)
\lineto(908.31631592,100.91424123)
\curveto(908.40631767,100.88423402)(908.49631758,100.84923405)(908.58631592,100.80924123)
\curveto(908.80631727,100.70923419)(909.02131706,100.57423433)(909.23131592,100.40424123)
\curveto(909.44131664,100.24423466)(909.61131647,100.06923483)(909.74131592,99.87924123)
\curveto(909.7813163,99.82923507)(909.82131626,99.76923513)(909.86131592,99.69924123)
\curveto(909.89131619,99.63923526)(909.92631615,99.57923532)(909.96631592,99.51924123)
\curveto(910.01631606,99.43923546)(910.05631602,99.34423556)(910.08631592,99.23424123)
\curveto(910.11631596,99.12423578)(910.14631593,99.01923588)(910.17631592,98.91924123)
\curveto(910.21631586,98.80923609)(910.24131584,98.6992362)(910.25131592,98.58924123)
\curveto(910.26131582,98.47923642)(910.2763158,98.36423654)(910.29631592,98.24424123)
\curveto(910.30631577,98.2042367)(910.30631577,98.15923674)(910.29631592,98.10924123)
\curveto(910.29631578,98.06923683)(910.30131578,98.02923687)(910.31131592,97.98924123)
\curveto(910.32131576,97.94923695)(910.32631575,97.89423701)(910.32631592,97.82424123)
\curveto(910.32631575,97.75423715)(910.32131576,97.7042372)(910.31131592,97.67424123)
\curveto(910.29131579,97.62423728)(910.28631579,97.57923732)(910.29631592,97.53924123)
\curveto(910.30631577,97.4992374)(910.30631577,97.46423744)(910.29631592,97.43424123)
\lineto(910.29631592,97.34424123)
\curveto(910.2763158,97.28423762)(910.26131582,97.21923768)(910.25131592,97.14924123)
\curveto(910.25131583,97.08923781)(910.24631583,97.02423788)(910.23631592,96.95424123)
\curveto(910.18631589,96.78423812)(910.13631594,96.62423828)(910.08631592,96.47424123)
\curveto(910.03631604,96.32423858)(909.97131611,96.17923872)(909.89131592,96.03924123)
\curveto(909.85131623,95.98923891)(909.82131626,95.93423897)(909.80131592,95.87424123)
\curveto(909.77131631,95.82423908)(909.73631634,95.77423913)(909.69631592,95.72424123)
\curveto(909.51631656,95.48423942)(909.29631678,95.28423962)(909.03631592,95.12424123)
\curveto(908.7763173,94.96423994)(908.49131759,94.82424008)(908.18131592,94.70424123)
\curveto(908.04131804,94.64424026)(907.90131818,94.5992403)(907.76131592,94.56924123)
\curveto(907.61131847,94.53924036)(907.45631862,94.5042404)(907.29631592,94.46424123)
\curveto(907.18631889,94.44424046)(907.076319,94.42924047)(906.96631592,94.41924123)
\curveto(906.85631922,94.40924049)(906.74631933,94.39424051)(906.63631592,94.37424123)
\curveto(906.59631948,94.36424054)(906.55631952,94.35924054)(906.51631592,94.35924123)
\curveto(906.4763196,94.36924053)(906.43631964,94.36924053)(906.39631592,94.35924123)
\curveto(906.34631973,94.34924055)(906.29631978,94.34424056)(906.24631592,94.34424123)
\lineto(906.08131592,94.34424123)
\curveto(906.03132005,94.32424058)(905.9813201,94.31924058)(905.93131592,94.32924123)
\curveto(905.87132021,94.33924056)(905.81632026,94.33924056)(905.76631592,94.32924123)
\curveto(905.72632035,94.31924058)(905.6813204,94.31924058)(905.63131592,94.32924123)
\curveto(905.5813205,94.33924056)(905.53132055,94.33424057)(905.48131592,94.31424123)
\curveto(905.41132067,94.29424061)(905.33632074,94.28924061)(905.25631592,94.29924123)
\curveto(905.16632091,94.30924059)(905.081321,94.31424059)(905.00131592,94.31424123)
\curveto(904.91132117,94.31424059)(904.81132127,94.30924059)(904.70131592,94.29924123)
\curveto(904.5813215,94.28924061)(904.4813216,94.29424061)(904.40131592,94.31424123)
\lineto(904.11631592,94.31424123)
\lineto(903.48631592,94.35924123)
\curveto(903.38632269,94.36924053)(903.29132279,94.37924052)(903.20131592,94.38924123)
\lineto(902.90131592,94.41924123)
\curveto(902.85132323,94.43924046)(902.80132328,94.44424046)(902.75131592,94.43424123)
\curveto(902.69132339,94.43424047)(902.63632344,94.44424046)(902.58631592,94.46424123)
\curveto(902.41632366,94.51424039)(902.25132383,94.55424035)(902.09131592,94.58424123)
\curveto(901.92132416,94.61424029)(901.76132432,94.66424024)(901.61131592,94.73424123)
\curveto(901.15132493,94.92423998)(900.7763253,95.14423976)(900.48631592,95.39424123)
\curveto(900.19632588,95.65423925)(899.95132613,96.01423889)(899.75131592,96.47424123)
\curveto(899.70132638,96.6042383)(899.66632641,96.73423817)(899.64631592,96.86424123)
\curveto(899.62632645,97.0042379)(899.60132648,97.14423776)(899.57131592,97.28424123)
\curveto(899.56132652,97.35423755)(899.55632652,97.41923748)(899.55631592,97.47924123)
\curveto(899.55632652,97.53923736)(899.55132653,97.6042373)(899.54131592,97.67424123)
\curveto(899.52132656,98.5042364)(899.67132641,99.17423573)(899.99131592,99.68424123)
\curveto(900.30132578,100.19423471)(900.74132534,100.57423433)(901.31131592,100.82424123)
\curveto(901.43132465,100.87423403)(901.55632452,100.91923398)(901.68631592,100.95924123)
\curveto(901.81632426,100.9992339)(901.95132413,101.04423386)(902.09131592,101.09424123)
\curveto(902.17132391,101.11423379)(902.25632382,101.12923377)(902.34631592,101.13924123)
\lineto(902.58631592,101.19924123)
\curveto(902.69632338,101.22923367)(902.80632327,101.24423366)(902.91631592,101.24424123)
\curveto(903.02632305,101.25423365)(903.13632294,101.26923363)(903.24631592,101.28924123)
\curveto(903.29632278,101.30923359)(903.34132274,101.31423359)(903.38131592,101.30424123)
\curveto(903.42132266,101.3042336)(903.46132262,101.30923359)(903.50131592,101.31924123)
\curveto(903.55132253,101.32923357)(903.60632247,101.32923357)(903.66631592,101.31924123)
\curveto(903.71632236,101.31923358)(903.76632231,101.32423358)(903.81631592,101.33424123)
\lineto(903.95131592,101.33424123)
\curveto(904.01132207,101.35423355)(904.081322,101.35423355)(904.16131592,101.33424123)
\curveto(904.23132185,101.32423358)(904.29632178,101.32923357)(904.35631592,101.34924123)
\curveto(904.38632169,101.35923354)(904.42632165,101.36423354)(904.47631592,101.36424123)
\lineto(904.59631592,101.36424123)
\lineto(905.06131592,101.36424123)
\moveto(907.38631592,99.81924123)
\curveto(907.06631901,99.91923498)(906.70131938,99.97923492)(906.29131592,99.99924123)
\curveto(905.8813202,100.01923488)(905.47132061,100.02923487)(905.06131592,100.02924123)
\curveto(904.63132145,100.02923487)(904.21132187,100.01923488)(903.80131592,99.99924123)
\curveto(903.39132269,99.97923492)(903.00632307,99.93423497)(902.64631592,99.86424123)
\curveto(902.28632379,99.79423511)(901.96632411,99.68423522)(901.68631592,99.53424123)
\curveto(901.39632468,99.39423551)(901.16132492,99.1992357)(900.98131592,98.94924123)
\curveto(900.87132521,98.78923611)(900.79132529,98.60923629)(900.74131592,98.40924123)
\curveto(900.6813254,98.20923669)(900.65132543,97.96423694)(900.65131592,97.67424123)
\curveto(900.67132541,97.65423725)(900.6813254,97.61923728)(900.68131592,97.56924123)
\curveto(900.67132541,97.51923738)(900.67132541,97.47923742)(900.68131592,97.44924123)
\curveto(900.70132538,97.36923753)(900.72132536,97.29423761)(900.74131592,97.22424123)
\curveto(900.75132533,97.16423774)(900.77132531,97.0992378)(900.80131592,97.02924123)
\curveto(900.92132516,96.75923814)(901.09132499,96.53923836)(901.31131592,96.36924123)
\curveto(901.52132456,96.20923869)(901.76632431,96.07423883)(902.04631592,95.96424123)
\curveto(902.15632392,95.91423899)(902.2763238,95.87423903)(902.40631592,95.84424123)
\curveto(902.52632355,95.82423908)(902.65132343,95.7992391)(902.78131592,95.76924123)
\curveto(902.83132325,95.74923915)(902.88632319,95.73923916)(902.94631592,95.73924123)
\curveto(902.99632308,95.73923916)(903.04632303,95.73423917)(903.09631592,95.72424123)
\curveto(903.18632289,95.71423919)(903.2813228,95.7042392)(903.38131592,95.69424123)
\curveto(903.47132261,95.68423922)(903.56632251,95.67423923)(903.66631592,95.66424123)
\curveto(903.74632233,95.66423924)(903.83132225,95.65923924)(903.92131592,95.64924123)
\lineto(904.16131592,95.64924123)
\lineto(904.34131592,95.64924123)
\curveto(904.37132171,95.63923926)(904.40632167,95.63423927)(904.44631592,95.63424123)
\lineto(904.58131592,95.63424123)
\lineto(905.03131592,95.63424123)
\curveto(905.11132097,95.63423927)(905.19632088,95.62923927)(905.28631592,95.61924123)
\curveto(905.36632071,95.61923928)(905.44132064,95.62923927)(905.51131592,95.64924123)
\lineto(905.78131592,95.64924123)
\curveto(905.80132028,95.64923925)(905.83132025,95.64423926)(905.87131592,95.63424123)
\curveto(905.90132018,95.63423927)(905.92632015,95.63923926)(905.94631592,95.64924123)
\curveto(906.04632003,95.65923924)(906.14631993,95.66423924)(906.24631592,95.66424123)
\curveto(906.33631974,95.67423923)(906.43631964,95.68423922)(906.54631592,95.69424123)
\curveto(906.66631941,95.72423918)(906.79131929,95.73923916)(906.92131592,95.73924123)
\curveto(907.04131904,95.74923915)(907.15631892,95.77423913)(907.26631592,95.81424123)
\curveto(907.56631851,95.89423901)(907.83131825,95.97923892)(908.06131592,96.06924123)
\curveto(908.29131779,96.16923873)(908.50631757,96.31423859)(908.70631592,96.50424123)
\curveto(908.90631717,96.71423819)(909.05631702,96.97923792)(909.15631592,97.29924123)
\curveto(909.1763169,97.33923756)(909.18631689,97.37423753)(909.18631592,97.40424123)
\curveto(909.1763169,97.44423746)(909.1813169,97.48923741)(909.20131592,97.53924123)
\curveto(909.21131687,97.57923732)(909.22131686,97.64923725)(909.23131592,97.74924123)
\curveto(909.24131684,97.85923704)(909.23631684,97.94423696)(909.21631592,98.00424123)
\curveto(909.19631688,98.07423683)(909.18631689,98.14423676)(909.18631592,98.21424123)
\curveto(909.1763169,98.28423662)(909.16131692,98.34923655)(909.14131592,98.40924123)
\curveto(909.081317,98.60923629)(908.99631708,98.78923611)(908.88631592,98.94924123)
\curveto(908.86631721,98.97923592)(908.84631723,99.0042359)(908.82631592,99.02424123)
\lineto(908.76631592,99.08424123)
\curveto(908.74631733,99.12423578)(908.70631737,99.17423573)(908.64631592,99.23424123)
\curveto(908.50631757,99.33423557)(908.3763177,99.41923548)(908.25631592,99.48924123)
\curveto(908.13631794,99.55923534)(907.99131809,99.62923527)(907.82131592,99.69924123)
\curveto(907.75131833,99.72923517)(907.6813184,99.74923515)(907.61131592,99.75924123)
\curveto(907.54131854,99.77923512)(907.46631861,99.7992351)(907.38631592,99.81924123)
}
}
{
\newrgbcolor{curcolor}{0 0 0}
\pscustom[linestyle=none,fillstyle=solid,fillcolor=curcolor]
{
\newpath
\moveto(899.54131592,106.77385061)
\curveto(899.54132654,106.87384575)(899.55132653,106.96884566)(899.57131592,107.05885061)
\curveto(899.5813265,107.14884548)(899.61132647,107.21384541)(899.66131592,107.25385061)
\curveto(899.74132634,107.31384531)(899.84632623,107.34384528)(899.97631592,107.34385061)
\lineto(900.36631592,107.34385061)
\lineto(901.86631592,107.34385061)
\lineto(908.25631592,107.34385061)
\lineto(909.42631592,107.34385061)
\lineto(909.74131592,107.34385061)
\curveto(909.84131624,107.35384527)(909.92131616,107.33884529)(909.98131592,107.29885061)
\curveto(910.06131602,107.24884538)(910.11131597,107.17384545)(910.13131592,107.07385061)
\curveto(910.14131594,106.98384564)(910.14631593,106.87384575)(910.14631592,106.74385061)
\lineto(910.14631592,106.51885061)
\curveto(910.12631595,106.43884619)(910.11131597,106.36884626)(910.10131592,106.30885061)
\curveto(910.081316,106.24884638)(910.04131604,106.19884643)(909.98131592,106.15885061)
\curveto(909.92131616,106.11884651)(909.84631623,106.09884653)(909.75631592,106.09885061)
\lineto(909.45631592,106.09885061)
\lineto(908.36131592,106.09885061)
\lineto(903.02131592,106.09885061)
\curveto(902.93132315,106.07884655)(902.85632322,106.06384656)(902.79631592,106.05385061)
\curveto(902.72632335,106.05384657)(902.66632341,106.0238466)(902.61631592,105.96385061)
\curveto(902.56632351,105.89384673)(902.54132354,105.80384682)(902.54131592,105.69385061)
\curveto(902.53132355,105.59384703)(902.52632355,105.48384714)(902.52631592,105.36385061)
\lineto(902.52631592,104.22385061)
\lineto(902.52631592,103.72885061)
\curveto(902.51632356,103.56884906)(902.45632362,103.45884917)(902.34631592,103.39885061)
\curveto(902.31632376,103.37884925)(902.28632379,103.36884926)(902.25631592,103.36885061)
\curveto(902.21632386,103.36884926)(902.17132391,103.36384926)(902.12131592,103.35385061)
\curveto(902.00132408,103.33384929)(901.89132419,103.33884929)(901.79131592,103.36885061)
\curveto(901.69132439,103.40884922)(901.62132446,103.46384916)(901.58131592,103.53385061)
\curveto(901.53132455,103.61384901)(901.50632457,103.73384889)(901.50631592,103.89385061)
\curveto(901.50632457,104.05384857)(901.49132459,104.18884844)(901.46131592,104.29885061)
\curveto(901.45132463,104.34884828)(901.44632463,104.40384822)(901.44631592,104.46385061)
\curveto(901.43632464,104.5238481)(901.42132466,104.58384804)(901.40131592,104.64385061)
\curveto(901.35132473,104.79384783)(901.30132478,104.93884769)(901.25131592,105.07885061)
\curveto(901.19132489,105.21884741)(901.12132496,105.35384727)(901.04131592,105.48385061)
\curveto(900.95132513,105.623847)(900.84632523,105.74384688)(900.72631592,105.84385061)
\curveto(900.60632547,105.94384668)(900.4763256,106.03884659)(900.33631592,106.12885061)
\curveto(900.23632584,106.18884644)(900.12632595,106.23384639)(900.00631592,106.26385061)
\curveto(899.88632619,106.30384632)(899.7813263,106.35384627)(899.69131592,106.41385061)
\curveto(899.63132645,106.46384616)(899.59132649,106.53384609)(899.57131592,106.62385061)
\curveto(899.56132652,106.64384598)(899.55632652,106.66884596)(899.55631592,106.69885061)
\curveto(899.55632652,106.7288459)(899.55132653,106.75384587)(899.54131592,106.77385061)
}
}
{
\newrgbcolor{curcolor}{0 0 0}
\pscustom[linestyle=none,fillstyle=solid,fillcolor=curcolor]
{
\newpath
\moveto(899.54131592,115.12345998)
\curveto(899.54132654,115.22345513)(899.55132653,115.31845503)(899.57131592,115.40845998)
\curveto(899.5813265,115.49845485)(899.61132647,115.56345479)(899.66131592,115.60345998)
\curveto(899.74132634,115.66345469)(899.84632623,115.69345466)(899.97631592,115.69345998)
\lineto(900.36631592,115.69345998)
\lineto(901.86631592,115.69345998)
\lineto(908.25631592,115.69345998)
\lineto(909.42631592,115.69345998)
\lineto(909.74131592,115.69345998)
\curveto(909.84131624,115.70345465)(909.92131616,115.68845466)(909.98131592,115.64845998)
\curveto(910.06131602,115.59845475)(910.11131597,115.52345483)(910.13131592,115.42345998)
\curveto(910.14131594,115.33345502)(910.14631593,115.22345513)(910.14631592,115.09345998)
\lineto(910.14631592,114.86845998)
\curveto(910.12631595,114.78845556)(910.11131597,114.71845563)(910.10131592,114.65845998)
\curveto(910.081316,114.59845575)(910.04131604,114.5484558)(909.98131592,114.50845998)
\curveto(909.92131616,114.46845588)(909.84631623,114.4484559)(909.75631592,114.44845998)
\lineto(909.45631592,114.44845998)
\lineto(908.36131592,114.44845998)
\lineto(903.02131592,114.44845998)
\curveto(902.93132315,114.42845592)(902.85632322,114.41345594)(902.79631592,114.40345998)
\curveto(902.72632335,114.40345595)(902.66632341,114.37345598)(902.61631592,114.31345998)
\curveto(902.56632351,114.24345611)(902.54132354,114.1534562)(902.54131592,114.04345998)
\curveto(902.53132355,113.94345641)(902.52632355,113.83345652)(902.52631592,113.71345998)
\lineto(902.52631592,112.57345998)
\lineto(902.52631592,112.07845998)
\curveto(902.51632356,111.91845843)(902.45632362,111.80845854)(902.34631592,111.74845998)
\curveto(902.31632376,111.72845862)(902.28632379,111.71845863)(902.25631592,111.71845998)
\curveto(902.21632386,111.71845863)(902.17132391,111.71345864)(902.12131592,111.70345998)
\curveto(902.00132408,111.68345867)(901.89132419,111.68845866)(901.79131592,111.71845998)
\curveto(901.69132439,111.75845859)(901.62132446,111.81345854)(901.58131592,111.88345998)
\curveto(901.53132455,111.96345839)(901.50632457,112.08345827)(901.50631592,112.24345998)
\curveto(901.50632457,112.40345795)(901.49132459,112.53845781)(901.46131592,112.64845998)
\curveto(901.45132463,112.69845765)(901.44632463,112.7534576)(901.44631592,112.81345998)
\curveto(901.43632464,112.87345748)(901.42132466,112.93345742)(901.40131592,112.99345998)
\curveto(901.35132473,113.14345721)(901.30132478,113.28845706)(901.25131592,113.42845998)
\curveto(901.19132489,113.56845678)(901.12132496,113.70345665)(901.04131592,113.83345998)
\curveto(900.95132513,113.97345638)(900.84632523,114.09345626)(900.72631592,114.19345998)
\curveto(900.60632547,114.29345606)(900.4763256,114.38845596)(900.33631592,114.47845998)
\curveto(900.23632584,114.53845581)(900.12632595,114.58345577)(900.00631592,114.61345998)
\curveto(899.88632619,114.6534557)(899.7813263,114.70345565)(899.69131592,114.76345998)
\curveto(899.63132645,114.81345554)(899.59132649,114.88345547)(899.57131592,114.97345998)
\curveto(899.56132652,114.99345536)(899.55632652,115.01845533)(899.55631592,115.04845998)
\curveto(899.55632652,115.07845527)(899.55132653,115.10345525)(899.54131592,115.12345998)
}
}
{
\newrgbcolor{curcolor}{0 0 0}
\pscustom[linestyle=none,fillstyle=solid,fillcolor=curcolor]
{
\newpath
\moveto(920.37763184,42.29681936)
\curveto(920.37764253,42.36681368)(920.37764253,42.4468136)(920.37763184,42.53681936)
\curveto(920.36764254,42.62681342)(920.36764254,42.71181333)(920.37763184,42.79181936)
\curveto(920.37764253,42.88181316)(920.38764252,42.96181308)(920.40763184,43.03181936)
\curveto(920.42764248,43.11181293)(920.45764245,43.16681288)(920.49763184,43.19681936)
\curveto(920.54764236,43.22681282)(920.62264229,43.2468128)(920.72263184,43.25681936)
\curveto(920.8126421,43.27681277)(920.91764199,43.28681276)(921.03763184,43.28681936)
\curveto(921.14764176,43.29681275)(921.26264165,43.29681275)(921.38263184,43.28681936)
\lineto(921.68263184,43.28681936)
\lineto(924.69763184,43.28681936)
\lineto(927.59263184,43.28681936)
\curveto(927.92263499,43.28681276)(928.24763466,43.28181276)(928.56763184,43.27181936)
\curveto(928.87763403,43.27181277)(929.15763375,43.23181281)(929.40763184,43.15181936)
\curveto(929.75763315,43.03181301)(930.05263286,42.87681317)(930.29263184,42.68681936)
\curveto(930.52263239,42.49681355)(930.72263219,42.25681379)(930.89263184,41.96681936)
\curveto(930.94263197,41.90681414)(930.97763193,41.8418142)(930.99763184,41.77181936)
\curveto(931.01763189,41.71181433)(931.04263187,41.6418144)(931.07263184,41.56181936)
\curveto(931.12263179,41.4418146)(931.15763175,41.31181473)(931.17763184,41.17181936)
\curveto(931.2076317,41.041815)(931.23763167,40.90681514)(931.26763184,40.76681936)
\curveto(931.28763162,40.71681533)(931.29263162,40.66681538)(931.28263184,40.61681936)
\curveto(931.27263164,40.56681548)(931.27263164,40.51181553)(931.28263184,40.45181936)
\curveto(931.29263162,40.43181561)(931.29263162,40.40681564)(931.28263184,40.37681936)
\curveto(931.28263163,40.3468157)(931.28763162,40.32181572)(931.29763184,40.30181936)
\curveto(931.3076316,40.26181578)(931.3126316,40.20681584)(931.31263184,40.13681936)
\curveto(931.3126316,40.06681598)(931.3076316,40.01181603)(931.29763184,39.97181936)
\curveto(931.28763162,39.92181612)(931.28763162,39.86681618)(931.29763184,39.80681936)
\curveto(931.3076316,39.7468163)(931.30263161,39.69181635)(931.28263184,39.64181936)
\curveto(931.25263166,39.51181653)(931.23263168,39.38681666)(931.22263184,39.26681936)
\curveto(931.2126317,39.1468169)(931.18763172,39.03181701)(931.14763184,38.92181936)
\curveto(931.02763188,38.55181749)(930.85763205,38.23181781)(930.63763184,37.96181936)
\curveto(930.41763249,37.69181835)(930.13763277,37.48181856)(929.79763184,37.33181936)
\curveto(929.67763323,37.28181876)(929.55263336,37.23681881)(929.42263184,37.19681936)
\curveto(929.29263362,37.16681888)(929.15763375,37.13181891)(929.01763184,37.09181936)
\curveto(928.96763394,37.08181896)(928.92763398,37.07681897)(928.89763184,37.07681936)
\curveto(928.85763405,37.07681897)(928.8126341,37.07181897)(928.76263184,37.06181936)
\curveto(928.73263418,37.05181899)(928.69763421,37.046819)(928.65763184,37.04681936)
\curveto(928.6076343,37.046819)(928.56763434,37.041819)(928.53763184,37.03181936)
\lineto(928.37263184,37.03181936)
\curveto(928.29263462,37.01181903)(928.19263472,37.00681904)(928.07263184,37.01681936)
\curveto(927.94263497,37.02681902)(927.85263506,37.041819)(927.80263184,37.06181936)
\curveto(927.7126352,37.08181896)(927.64763526,37.13681891)(927.60763184,37.22681936)
\curveto(927.58763532,37.25681879)(927.58263533,37.28681876)(927.59263184,37.31681936)
\curveto(927.59263532,37.3468187)(927.58763532,37.38681866)(927.57763184,37.43681936)
\curveto(927.56763534,37.47681857)(927.56263535,37.51681853)(927.56263184,37.55681936)
\lineto(927.56263184,37.70681936)
\curveto(927.56263535,37.82681822)(927.56763534,37.9468181)(927.57763184,38.06681936)
\curveto(927.57763533,38.19681785)(927.6126353,38.28681776)(927.68263184,38.33681936)
\curveto(927.74263517,38.37681767)(927.80263511,38.39681765)(927.86263184,38.39681936)
\curveto(927.92263499,38.39681765)(927.99263492,38.40681764)(928.07263184,38.42681936)
\curveto(928.10263481,38.43681761)(928.13763477,38.43681761)(928.17763184,38.42681936)
\curveto(928.2076347,38.42681762)(928.23263468,38.43181761)(928.25263184,38.44181936)
\lineto(928.46263184,38.44181936)
\curveto(928.5126344,38.46181758)(928.56263435,38.46681758)(928.61263184,38.45681936)
\curveto(928.65263426,38.45681759)(928.69763421,38.46681758)(928.74763184,38.48681936)
\curveto(928.87763403,38.51681753)(929.00263391,38.5468175)(929.12263184,38.57681936)
\curveto(929.23263368,38.60681744)(929.33763357,38.65181739)(929.43763184,38.71181936)
\curveto(929.72763318,38.88181716)(929.93263298,39.15181689)(930.05263184,39.52181936)
\curveto(930.07263284,39.57181647)(930.08763282,39.62181642)(930.09763184,39.67181936)
\curveto(930.09763281,39.73181631)(930.1076328,39.78681626)(930.12763184,39.83681936)
\lineto(930.12763184,39.91181936)
\curveto(930.13763277,39.98181606)(930.14763276,40.07681597)(930.15763184,40.19681936)
\curveto(930.15763275,40.32681572)(930.14763276,40.42681562)(930.12763184,40.49681936)
\curveto(930.1076328,40.56681548)(930.09263282,40.63681541)(930.08263184,40.70681936)
\curveto(930.06263285,40.78681526)(930.04263287,40.85681519)(930.02263184,40.91681936)
\curveto(929.86263305,41.29681475)(929.58763332,41.57181447)(929.19763184,41.74181936)
\curveto(929.06763384,41.79181425)(928.912634,41.82681422)(928.73263184,41.84681936)
\curveto(928.55263436,41.87681417)(928.36763454,41.89181415)(928.17763184,41.89181936)
\curveto(927.97763493,41.90181414)(927.77763513,41.90181414)(927.57763184,41.89181936)
\lineto(927.00763184,41.89181936)
\lineto(922.76263184,41.89181936)
\lineto(921.21763184,41.89181936)
\curveto(921.1076418,41.89181415)(920.98764192,41.88681416)(920.85763184,41.87681936)
\curveto(920.72764218,41.86681418)(920.62264229,41.88681416)(920.54263184,41.93681936)
\curveto(920.47264244,41.99681405)(920.42264249,42.07681397)(920.39263184,42.17681936)
\curveto(920.39264252,42.19681385)(920.39264252,42.21681383)(920.39263184,42.23681936)
\curveto(920.39264252,42.25681379)(920.38764252,42.27681377)(920.37763184,42.29681936)
}
}
{
\newrgbcolor{curcolor}{0 0 0}
\pscustom[linestyle=none,fillstyle=solid,fillcolor=curcolor]
{
\newpath
\moveto(923.33263184,45.83049123)
\lineto(923.33263184,46.26549123)
\curveto(923.33263958,46.41548927)(923.37263954,46.52048916)(923.45263184,46.58049123)
\curveto(923.53263938,46.63048905)(923.63263928,46.65548903)(923.75263184,46.65549123)
\curveto(923.87263904,46.66548902)(923.99263892,46.67048901)(924.11263184,46.67049123)
\lineto(925.53763184,46.67049123)
\lineto(927.80263184,46.67049123)
\lineto(928.49263184,46.67049123)
\curveto(928.72263419,46.67048901)(928.92263399,46.69548899)(929.09263184,46.74549123)
\curveto(929.54263337,46.90548878)(929.85763305,47.20548848)(930.03763184,47.64549123)
\curveto(930.12763278,47.86548782)(930.16263275,48.13048755)(930.14263184,48.44049123)
\curveto(930.1126328,48.75048693)(930.05763285,49.00048668)(929.97763184,49.19049123)
\curveto(929.83763307,49.52048616)(929.66263325,49.7804859)(929.45263184,49.97049123)
\curveto(929.23263368,50.17048551)(928.94763396,50.32548536)(928.59763184,50.43549123)
\curveto(928.51763439,50.46548522)(928.43763447,50.4854852)(928.35763184,50.49549123)
\curveto(928.27763463,50.50548518)(928.19263472,50.52048516)(928.10263184,50.54049123)
\curveto(928.05263486,50.55048513)(928.0076349,50.55048513)(927.96763184,50.54049123)
\curveto(927.92763498,50.54048514)(927.88263503,50.55048513)(927.83263184,50.57049123)
\lineto(927.51763184,50.57049123)
\curveto(927.43763547,50.59048509)(927.34763556,50.59548509)(927.24763184,50.58549123)
\curveto(927.13763577,50.57548511)(927.03763587,50.57048511)(926.94763184,50.57049123)
\lineto(925.77763184,50.57049123)
\lineto(924.18763184,50.57049123)
\curveto(924.06763884,50.57048511)(923.94263897,50.56548512)(923.81263184,50.55549123)
\curveto(923.67263924,50.55548513)(923.56263935,50.5804851)(923.48263184,50.63049123)
\curveto(923.43263948,50.67048501)(923.40263951,50.71548497)(923.39263184,50.76549123)
\curveto(923.37263954,50.82548486)(923.35263956,50.89548479)(923.33263184,50.97549123)
\lineto(923.33263184,51.20049123)
\curveto(923.33263958,51.32048436)(923.33763957,51.42548426)(923.34763184,51.51549123)
\curveto(923.35763955,51.61548407)(923.40263951,51.69048399)(923.48263184,51.74049123)
\curveto(923.53263938,51.79048389)(923.6076393,51.81548387)(923.70763184,51.81549123)
\lineto(923.99263184,51.81549123)
\lineto(925.01263184,51.81549123)
\lineto(929.04763184,51.81549123)
\lineto(930.39763184,51.81549123)
\curveto(930.51763239,51.81548387)(930.63263228,51.81048387)(930.74263184,51.80049123)
\curveto(930.84263207,51.80048388)(930.91763199,51.76548392)(930.96763184,51.69549123)
\curveto(930.99763191,51.65548403)(931.02263189,51.59548409)(931.04263184,51.51549123)
\curveto(931.05263186,51.43548425)(931.06263185,51.34548434)(931.07263184,51.24549123)
\curveto(931.07263184,51.15548453)(931.06763184,51.06548462)(931.05763184,50.97549123)
\curveto(931.04763186,50.89548479)(931.02763188,50.83548485)(930.99763184,50.79549123)
\curveto(930.95763195,50.74548494)(930.89263202,50.70048498)(930.80263184,50.66049123)
\curveto(930.76263215,50.65048503)(930.7076322,50.64048504)(930.63763184,50.63049123)
\curveto(930.56763234,50.63048505)(930.50263241,50.62548506)(930.44263184,50.61549123)
\curveto(930.37263254,50.60548508)(930.31763259,50.5854851)(930.27763184,50.55549123)
\curveto(930.23763267,50.52548516)(930.22263269,50.4804852)(930.23263184,50.42049123)
\curveto(930.25263266,50.34048534)(930.3126326,50.26048542)(930.41263184,50.18049123)
\curveto(930.50263241,50.10048558)(930.57263234,50.02548566)(930.62263184,49.95549123)
\curveto(930.78263213,49.73548595)(930.92263199,49.4854862)(931.04263184,49.20549123)
\curveto(931.09263182,49.09548659)(931.12263179,48.9804867)(931.13263184,48.86049123)
\curveto(931.15263176,48.75048693)(931.17763173,48.63548705)(931.20763184,48.51549123)
\curveto(931.21763169,48.46548722)(931.21763169,48.41048727)(931.20763184,48.35049123)
\curveto(931.19763171,48.30048738)(931.20263171,48.25048743)(931.22263184,48.20049123)
\curveto(931.24263167,48.10048758)(931.24263167,48.01048767)(931.22263184,47.93049123)
\lineto(931.22263184,47.78049123)
\curveto(931.20263171,47.73048795)(931.19263172,47.67048801)(931.19263184,47.60049123)
\curveto(931.19263172,47.54048814)(931.18763172,47.4854882)(931.17763184,47.43549123)
\curveto(931.15763175,47.39548829)(931.14763176,47.35548833)(931.14763184,47.31549123)
\curveto(931.15763175,47.2854884)(931.15263176,47.24548844)(931.13263184,47.19549123)
\lineto(931.07263184,46.95549123)
\curveto(931.05263186,46.8854888)(931.02263189,46.81048887)(930.98263184,46.73049123)
\curveto(930.87263204,46.47048921)(930.72763218,46.25048943)(930.54763184,46.07049123)
\curveto(930.35763255,45.90048978)(930.13263278,45.76048992)(929.87263184,45.65049123)
\curveto(929.78263313,45.61049007)(929.69263322,45.5804901)(929.60263184,45.56049123)
\lineto(929.30263184,45.50049123)
\curveto(929.24263367,45.4804902)(929.18763372,45.47049021)(929.13763184,45.47049123)
\curveto(929.07763383,45.4804902)(929.0126339,45.47549021)(928.94263184,45.45549123)
\curveto(928.92263399,45.44549024)(928.89763401,45.44049024)(928.86763184,45.44049123)
\curveto(928.82763408,45.44049024)(928.79263412,45.43549025)(928.76263184,45.42549123)
\lineto(928.61263184,45.42549123)
\curveto(928.57263434,45.41549027)(928.52763438,45.41049027)(928.47763184,45.41049123)
\curveto(928.41763449,45.42049026)(928.36263455,45.42549026)(928.31263184,45.42549123)
\lineto(927.71263184,45.42549123)
\lineto(924.95263184,45.42549123)
\lineto(923.99263184,45.42549123)
\lineto(923.72263184,45.42549123)
\curveto(923.63263928,45.42549026)(923.55763935,45.44549024)(923.49763184,45.48549123)
\curveto(923.42763948,45.52549016)(923.37763953,45.60049008)(923.34763184,45.71049123)
\curveto(923.33763957,45.73048995)(923.33763957,45.75048993)(923.34763184,45.77049123)
\curveto(923.34763956,45.79048989)(923.34263957,45.81048987)(923.33263184,45.83049123)
}
}
{
\newrgbcolor{curcolor}{0 0 0}
\pscustom[linestyle=none,fillstyle=solid,fillcolor=curcolor]
{
\newpath
\moveto(920.37763184,54.28510061)
\curveto(920.37764253,54.41509899)(920.37764253,54.55009886)(920.37763184,54.69010061)
\curveto(920.37764253,54.84009857)(920.4126425,54.95009846)(920.48263184,55.02010061)
\curveto(920.55264236,55.07009834)(920.64764226,55.09509831)(920.76763184,55.09510061)
\curveto(920.87764203,55.1050983)(920.99264192,55.1100983)(921.11263184,55.11010061)
\lineto(922.44763184,55.11010061)
\lineto(928.52263184,55.11010061)
\lineto(930.20263184,55.11010061)
\lineto(930.59263184,55.11010061)
\curveto(930.73263218,55.1100983)(930.84263207,55.08509832)(930.92263184,55.03510061)
\curveto(930.97263194,55.0050984)(931.00263191,54.96009845)(931.01263184,54.90010061)
\curveto(931.02263189,54.85009856)(931.03763187,54.78509862)(931.05763184,54.70510061)
\lineto(931.05763184,54.49510061)
\lineto(931.05763184,54.18010061)
\curveto(931.04763186,54.08009933)(931.0126319,54.0050994)(930.95263184,53.95510061)
\curveto(930.87263204,53.9050995)(930.77263214,53.87509953)(930.65263184,53.86510061)
\lineto(930.27763184,53.86510061)
\lineto(928.89763184,53.86510061)
\lineto(922.65763184,53.86510061)
\lineto(921.18763184,53.86510061)
\curveto(921.07764183,53.86509954)(920.96264195,53.86009955)(920.84263184,53.85010061)
\curveto(920.7126422,53.85009956)(920.6126423,53.87509953)(920.54263184,53.92510061)
\curveto(920.48264243,53.96509944)(920.43264248,54.04009937)(920.39263184,54.15010061)
\curveto(920.38264253,54.17009924)(920.38264253,54.19009922)(920.39263184,54.21010061)
\curveto(920.39264252,54.24009917)(920.38764252,54.26509914)(920.37763184,54.28510061)
}
}
{
\newrgbcolor{curcolor}{0 0 0}
\pscustom[linestyle=none,fillstyle=solid,fillcolor=curcolor]
{
}
}
{
\newrgbcolor{curcolor}{0 0 0}
\pscustom[linestyle=none,fillstyle=solid,fillcolor=curcolor]
{
\newpath
\moveto(920.45263184,64.24510061)
\curveto(920.44264247,64.93509597)(920.56264235,65.53509537)(920.81263184,66.04510061)
\curveto(921.06264185,66.56509434)(921.39764151,66.96009395)(921.81763184,67.23010061)
\curveto(921.89764101,67.28009363)(921.98764092,67.32509358)(922.08763184,67.36510061)
\curveto(922.17764073,67.4050935)(922.27264064,67.45009346)(922.37263184,67.50010061)
\curveto(922.47264044,67.54009337)(922.57264034,67.57009334)(922.67263184,67.59010061)
\curveto(922.77264014,67.6100933)(922.87764003,67.63009328)(922.98763184,67.65010061)
\curveto(923.03763987,67.67009324)(923.08263983,67.67509323)(923.12263184,67.66510061)
\curveto(923.16263975,67.65509325)(923.2076397,67.66009325)(923.25763184,67.68010061)
\curveto(923.3076396,67.69009322)(923.39263952,67.69509321)(923.51263184,67.69510061)
\curveto(923.62263929,67.69509321)(923.7076392,67.69009322)(923.76763184,67.68010061)
\curveto(923.82763908,67.66009325)(923.88763902,67.65009326)(923.94763184,67.65010061)
\curveto(924.0076389,67.66009325)(924.06763884,67.65509325)(924.12763184,67.63510061)
\curveto(924.26763864,67.59509331)(924.40263851,67.56009335)(924.53263184,67.53010061)
\curveto(924.66263825,67.50009341)(924.78763812,67.46009345)(924.90763184,67.41010061)
\curveto(925.04763786,67.35009356)(925.17263774,67.28009363)(925.28263184,67.20010061)
\curveto(925.39263752,67.13009378)(925.50263741,67.05509385)(925.61263184,66.97510061)
\lineto(925.67263184,66.91510061)
\curveto(925.69263722,66.905094)(925.7126372,66.89009402)(925.73263184,66.87010061)
\curveto(925.89263702,66.75009416)(926.03763687,66.61509429)(926.16763184,66.46510061)
\curveto(926.29763661,66.31509459)(926.42263649,66.15509475)(926.54263184,65.98510061)
\curveto(926.76263615,65.67509523)(926.96763594,65.38009553)(927.15763184,65.10010061)
\curveto(927.29763561,64.87009604)(927.43263548,64.64009627)(927.56263184,64.41010061)
\curveto(927.69263522,64.19009672)(927.82763508,63.97009694)(927.96763184,63.75010061)
\curveto(928.13763477,63.50009741)(928.31763459,63.26009765)(928.50763184,63.03010061)
\curveto(928.69763421,62.8100981)(928.92263399,62.62009829)(929.18263184,62.46010061)
\curveto(929.24263367,62.42009849)(929.30263361,62.38509852)(929.36263184,62.35510061)
\curveto(929.4126335,62.32509858)(929.47763343,62.29509861)(929.55763184,62.26510061)
\curveto(929.62763328,62.24509866)(929.68763322,62.24009867)(929.73763184,62.25010061)
\curveto(929.8076331,62.27009864)(929.86263305,62.3050986)(929.90263184,62.35510061)
\curveto(929.93263298,62.4050985)(929.95263296,62.46509844)(929.96263184,62.53510061)
\lineto(929.96263184,62.77510061)
\lineto(929.96263184,63.52510061)
\lineto(929.96263184,66.33010061)
\lineto(929.96263184,66.99010061)
\curveto(929.96263295,67.08009383)(929.96763294,67.16509374)(929.97763184,67.24510061)
\curveto(929.97763293,67.32509358)(929.99763291,67.39009352)(930.03763184,67.44010061)
\curveto(930.07763283,67.49009342)(930.15263276,67.53009338)(930.26263184,67.56010061)
\curveto(930.36263255,67.60009331)(930.46263245,67.6100933)(930.56263184,67.59010061)
\lineto(930.69763184,67.59010061)
\curveto(930.76763214,67.57009334)(930.82763208,67.55009336)(930.87763184,67.53010061)
\curveto(930.92763198,67.5100934)(930.96763194,67.47509343)(930.99763184,67.42510061)
\curveto(931.03763187,67.37509353)(931.05763185,67.3050936)(931.05763184,67.21510061)
\lineto(931.05763184,66.94510061)
\lineto(931.05763184,66.04510061)
\lineto(931.05763184,62.53510061)
\lineto(931.05763184,61.47010061)
\curveto(931.05763185,61.39009952)(931.06263185,61.30009961)(931.07263184,61.20010061)
\curveto(931.07263184,61.10009981)(931.06263185,61.01509989)(931.04263184,60.94510061)
\curveto(930.97263194,60.73510017)(930.79263212,60.67010024)(930.50263184,60.75010061)
\curveto(930.46263245,60.76010015)(930.42763248,60.76010015)(930.39763184,60.75010061)
\curveto(930.35763255,60.75010016)(930.3126326,60.76010015)(930.26263184,60.78010061)
\curveto(930.18263273,60.80010011)(930.09763281,60.82010009)(930.00763184,60.84010061)
\curveto(929.91763299,60.86010005)(929.83263308,60.88510002)(929.75263184,60.91510061)
\curveto(929.26263365,61.07509983)(928.84763406,61.27509963)(928.50763184,61.51510061)
\curveto(928.25763465,61.69509921)(928.03263488,61.90009901)(927.83263184,62.13010061)
\curveto(927.62263529,62.36009855)(927.42763548,62.60009831)(927.24763184,62.85010061)
\curveto(927.06763584,63.1100978)(926.89763601,63.37509753)(926.73763184,63.64510061)
\curveto(926.56763634,63.92509698)(926.39263652,64.19509671)(926.21263184,64.45510061)
\curveto(926.13263678,64.56509634)(926.05763685,64.67009624)(925.98763184,64.77010061)
\curveto(925.91763699,64.88009603)(925.84263707,64.99009592)(925.76263184,65.10010061)
\curveto(925.73263718,65.14009577)(925.70263721,65.17509573)(925.67263184,65.20510061)
\curveto(925.63263728,65.24509566)(925.60263731,65.28509562)(925.58263184,65.32510061)
\curveto(925.47263744,65.46509544)(925.34763756,65.59009532)(925.20763184,65.70010061)
\curveto(925.17763773,65.72009519)(925.15263776,65.74509516)(925.13263184,65.77510061)
\curveto(925.10263781,65.8050951)(925.07263784,65.83009508)(925.04263184,65.85010061)
\curveto(924.94263797,65.93009498)(924.84263807,65.99509491)(924.74263184,66.04510061)
\curveto(924.64263827,66.1050948)(924.53263838,66.16009475)(924.41263184,66.21010061)
\curveto(924.34263857,66.24009467)(924.26763864,66.26009465)(924.18763184,66.27010061)
\lineto(923.94763184,66.33010061)
\lineto(923.85763184,66.33010061)
\curveto(923.82763908,66.34009457)(923.79763911,66.34509456)(923.76763184,66.34510061)
\curveto(923.69763921,66.36509454)(923.60263931,66.37009454)(923.48263184,66.36010061)
\curveto(923.35263956,66.36009455)(923.25263966,66.35009456)(923.18263184,66.33010061)
\curveto(923.10263981,66.3100946)(923.02763988,66.29009462)(922.95763184,66.27010061)
\curveto(922.87764003,66.26009465)(922.79764011,66.24009467)(922.71763184,66.21010061)
\curveto(922.47764043,66.10009481)(922.27764063,65.95009496)(922.11763184,65.76010061)
\curveto(921.94764096,65.58009533)(921.8076411,65.36009555)(921.69763184,65.10010061)
\curveto(921.67764123,65.03009588)(921.66264125,64.96009595)(921.65263184,64.89010061)
\curveto(921.63264128,64.82009609)(921.6126413,64.74509616)(921.59263184,64.66510061)
\curveto(921.57264134,64.58509632)(921.56264135,64.47509643)(921.56263184,64.33510061)
\curveto(921.56264135,64.2050967)(921.57264134,64.10009681)(921.59263184,64.02010061)
\curveto(921.60264131,63.96009695)(921.6076413,63.905097)(921.60763184,63.85510061)
\curveto(921.6076413,63.8050971)(921.61764129,63.75509715)(921.63763184,63.70510061)
\curveto(921.67764123,63.6050973)(921.71764119,63.5100974)(921.75763184,63.42010061)
\curveto(921.79764111,63.34009757)(921.84264107,63.26009765)(921.89263184,63.18010061)
\curveto(921.912641,63.15009776)(921.93764097,63.12009779)(921.96763184,63.09010061)
\curveto(921.99764091,63.07009784)(922.02264089,63.04509786)(922.04263184,63.01510061)
\lineto(922.11763184,62.94010061)
\curveto(922.13764077,62.910098)(922.15764075,62.88509802)(922.17763184,62.86510061)
\lineto(922.38763184,62.71510061)
\curveto(922.44764046,62.67509823)(922.5126404,62.63009828)(922.58263184,62.58010061)
\curveto(922.67264024,62.52009839)(922.77764013,62.47009844)(922.89763184,62.43010061)
\curveto(923.0076399,62.40009851)(923.11763979,62.36509854)(923.22763184,62.32510061)
\curveto(923.33763957,62.28509862)(923.48263943,62.26009865)(923.66263184,62.25010061)
\curveto(923.83263908,62.24009867)(923.95763895,62.2100987)(924.03763184,62.16010061)
\curveto(924.11763879,62.1100988)(924.16263875,62.03509887)(924.17263184,61.93510061)
\curveto(924.18263873,61.83509907)(924.18763872,61.72509918)(924.18763184,61.60510061)
\curveto(924.18763872,61.56509934)(924.19263872,61.52509938)(924.20263184,61.48510061)
\curveto(924.20263871,61.44509946)(924.19763871,61.4100995)(924.18763184,61.38010061)
\curveto(924.16763874,61.33009958)(924.15763875,61.28009963)(924.15763184,61.23010061)
\curveto(924.15763875,61.19009972)(924.14763876,61.15009976)(924.12763184,61.11010061)
\curveto(924.06763884,61.02009989)(923.93263898,60.97509993)(923.72263184,60.97510061)
\lineto(923.60263184,60.97510061)
\curveto(923.54263937,60.98509992)(923.48263943,60.99009992)(923.42263184,60.99010061)
\curveto(923.35263956,61.00009991)(923.28763962,61.0100999)(923.22763184,61.02010061)
\curveto(923.11763979,61.04009987)(923.01763989,61.06009985)(922.92763184,61.08010061)
\curveto(922.82764008,61.10009981)(922.73264018,61.13009978)(922.64263184,61.17010061)
\curveto(922.57264034,61.19009972)(922.5126404,61.2100997)(922.46263184,61.23010061)
\lineto(922.28263184,61.29010061)
\curveto(922.02264089,61.4100995)(921.77764113,61.56509934)(921.54763184,61.75510061)
\curveto(921.31764159,61.95509895)(921.13264178,62.17009874)(920.99263184,62.40010061)
\curveto(920.912642,62.5100984)(920.84764206,62.62509828)(920.79763184,62.74510061)
\lineto(920.64763184,63.13510061)
\curveto(920.59764231,63.24509766)(920.56764234,63.36009755)(920.55763184,63.48010061)
\curveto(920.53764237,63.60009731)(920.5126424,63.72509718)(920.48263184,63.85510061)
\curveto(920.48264243,63.92509698)(920.48264243,63.99009692)(920.48263184,64.05010061)
\curveto(920.47264244,64.1100968)(920.46264245,64.17509673)(920.45263184,64.24510061)
}
}
{
\newrgbcolor{curcolor}{0 0 0}
\pscustom[linestyle=none,fillstyle=solid,fillcolor=curcolor]
{
\newpath
\moveto(927.56263184,76.22470998)
\curveto(927.6126353,76.29470234)(927.68263523,76.3347023)(927.77263184,76.34470998)
\curveto(927.86263505,76.36470227)(927.96763494,76.37470226)(928.08763184,76.37470998)
\curveto(928.13763477,76.37470226)(928.18763472,76.36970226)(928.23763184,76.35970998)
\curveto(928.28763462,76.35970227)(928.33263458,76.34970228)(928.37263184,76.32970998)
\curveto(928.46263445,76.29970233)(928.52263439,76.23970239)(928.55263184,76.14970998)
\curveto(928.57263434,76.06970256)(928.58263433,75.97470266)(928.58263184,75.86470998)
\lineto(928.58263184,75.54970998)
\curveto(928.57263434,75.43970319)(928.58263433,75.3347033)(928.61263184,75.23470998)
\curveto(928.64263427,75.09470354)(928.72263419,75.00470363)(928.85263184,74.96470998)
\curveto(928.92263399,74.94470369)(929.0076339,74.9347037)(929.10763184,74.93470998)
\lineto(929.37763184,74.93470998)
\lineto(930.32263184,74.93470998)
\lineto(930.65263184,74.93470998)
\curveto(930.76263215,74.9347037)(930.84763206,74.91470372)(930.90763184,74.87470998)
\curveto(930.96763194,74.8347038)(931.0076319,74.78470385)(931.02763184,74.72470998)
\curveto(931.03763187,74.67470396)(931.05263186,74.60970402)(931.07263184,74.52970998)
\lineto(931.07263184,74.33470998)
\curveto(931.07263184,74.21470442)(931.06763184,74.10970452)(931.05763184,74.01970998)
\curveto(931.03763187,73.9297047)(930.98763192,73.85970477)(930.90763184,73.80970998)
\curveto(930.85763205,73.77970485)(930.78763212,73.76470487)(930.69763184,73.76470998)
\lineto(930.39763184,73.76470998)
\lineto(929.36263184,73.76470998)
\curveto(929.20263371,73.76470487)(929.05763385,73.75470488)(928.92763184,73.73470998)
\curveto(928.78763412,73.72470491)(928.69263422,73.66970496)(928.64263184,73.56970998)
\curveto(928.62263429,73.51970511)(928.6076343,73.44970518)(928.59763184,73.35970998)
\curveto(928.58763432,73.27970535)(928.58263433,73.18970544)(928.58263184,73.08970998)
\lineto(928.58263184,72.80470998)
\lineto(928.58263184,72.56470998)
\lineto(928.58263184,70.29970998)
\curveto(928.58263433,70.20970842)(928.58763432,70.10470853)(928.59763184,69.98470998)
\lineto(928.59763184,69.65470998)
\curveto(928.59763431,69.54470909)(928.58763432,69.44470919)(928.56763184,69.35470998)
\curveto(928.54763436,69.26470937)(928.5126344,69.20470943)(928.46263184,69.17470998)
\curveto(928.39263452,69.12470951)(928.29763461,69.09970953)(928.17763184,69.09970998)
\lineto(927.83263184,69.09970998)
\lineto(927.56263184,69.09970998)
\curveto(927.39263552,69.13970949)(927.25263566,69.19470944)(927.14263184,69.26470998)
\curveto(927.03263588,69.3347093)(926.91763599,69.41470922)(926.79763184,69.50470998)
\lineto(926.25763184,69.86470998)
\curveto(925.62763728,70.30470833)(925.0076379,70.73970789)(924.39763184,71.16970998)
\lineto(922.53763184,72.48970998)
\curveto(922.3076406,72.64970598)(922.08764082,72.80470583)(921.87763184,72.95470998)
\curveto(921.65764125,73.10470553)(921.43264148,73.25970537)(921.20263184,73.41970998)
\curveto(921.13264178,73.46970516)(921.06764184,73.51970511)(921.00763184,73.56970998)
\curveto(920.93764197,73.61970501)(920.86264205,73.66970496)(920.78263184,73.71970998)
\lineto(920.69263184,73.77970998)
\curveto(920.65264226,73.80970482)(920.62264229,73.83970479)(920.60263184,73.86970998)
\curveto(920.57264234,73.90970472)(920.55264236,73.94970468)(920.54263184,73.98970998)
\curveto(920.52264239,74.0297046)(920.50264241,74.07470456)(920.48263184,74.12470998)
\curveto(920.48264243,74.14470449)(920.48764242,74.16470447)(920.49763184,74.18470998)
\curveto(920.49764241,74.21470442)(920.48764242,74.23970439)(920.46763184,74.25970998)
\curveto(920.46764244,74.38970424)(920.47264244,74.50970412)(920.48263184,74.61970998)
\curveto(920.49264242,74.7297039)(920.53764237,74.80970382)(920.61763184,74.85970998)
\curveto(920.66764224,74.89970373)(920.73764217,74.91970371)(920.82763184,74.91970998)
\curveto(920.91764199,74.9297037)(921.0126419,74.9347037)(921.11263184,74.93470998)
\lineto(926.57263184,74.93470998)
\curveto(926.64263627,74.9347037)(926.71763619,74.9297037)(926.79763184,74.91970998)
\curveto(926.86763604,74.91970371)(926.93763597,74.92470371)(927.00763184,74.93470998)
\lineto(927.11263184,74.93470998)
\curveto(927.16263575,74.95470368)(927.21763569,74.96970366)(927.27763184,74.97970998)
\curveto(927.32763558,74.98970364)(927.36763554,75.01470362)(927.39763184,75.05470998)
\curveto(927.44763546,75.12470351)(927.47763543,75.20970342)(927.48763184,75.30970998)
\lineto(927.48763184,75.63970998)
\curveto(927.48763542,75.74970288)(927.49263542,75.85470278)(927.50263184,75.95470998)
\curveto(927.50263541,76.06470257)(927.52263539,76.15470248)(927.56263184,76.22470998)
\moveto(927.36763184,73.65970998)
\curveto(927.25763565,73.73970489)(927.08763582,73.77470486)(926.85763184,73.76470998)
\lineto(926.24263184,73.76470998)
\lineto(923.76763184,73.76470998)
\lineto(923.45263184,73.76470998)
\curveto(923.33263958,73.77470486)(923.23263968,73.76970486)(923.15263184,73.74970998)
\lineto(923.00263184,73.74970998)
\curveto(922.91264,73.74970488)(922.82764008,73.7347049)(922.74763184,73.70470998)
\curveto(922.72764018,73.69470494)(922.71764019,73.68470495)(922.71763184,73.67470998)
\lineto(922.67263184,73.62970998)
\curveto(922.66264025,73.60970502)(922.65764025,73.57970505)(922.65763184,73.53970998)
\curveto(922.67764023,73.51970511)(922.69264022,73.49970513)(922.70263184,73.47970998)
\curveto(922.70264021,73.46970516)(922.7076402,73.45470518)(922.71763184,73.43470998)
\curveto(922.76764014,73.37470526)(922.83764007,73.31470532)(922.92763184,73.25470998)
\curveto(923.01763989,73.19470544)(923.09763981,73.13970549)(923.16763184,73.08970998)
\curveto(923.3076396,72.98970564)(923.45263946,72.89470574)(923.60263184,72.80470998)
\curveto(923.74263917,72.71470592)(923.88263903,72.61970601)(924.02263184,72.51970998)
\lineto(924.80263184,71.97970998)
\curveto(925.06263785,71.80970682)(925.32263759,71.634707)(925.58263184,71.45470998)
\curveto(925.69263722,71.37470726)(925.79763711,71.29970733)(925.89763184,71.22970998)
\lineto(926.19763184,71.01970998)
\curveto(926.27763663,70.96970766)(926.35263656,70.91970771)(926.42263184,70.86970998)
\curveto(926.49263642,70.8297078)(926.56763634,70.78470785)(926.64763184,70.73470998)
\curveto(926.7076362,70.68470795)(926.77263614,70.634708)(926.84263184,70.58470998)
\curveto(926.90263601,70.54470809)(926.97263594,70.50470813)(927.05263184,70.46470998)
\curveto(927.1126358,70.42470821)(927.18263573,70.39970823)(927.26263184,70.38970998)
\curveto(927.33263558,70.37970825)(927.38763552,70.41470822)(927.42763184,70.49470998)
\curveto(927.47763543,70.56470807)(927.50263541,70.67470796)(927.50263184,70.82470998)
\curveto(927.49263542,70.98470765)(927.48763542,71.11970751)(927.48763184,71.22970998)
\lineto(927.48763184,72.90970998)
\lineto(927.48763184,73.34470998)
\curveto(927.48763542,73.49470514)(927.44763546,73.59970503)(927.36763184,73.65970998)
}
}
{
\newrgbcolor{curcolor}{0 0 0}
\pscustom[linestyle=none,fillstyle=solid,fillcolor=curcolor]
{
\newpath
\moveto(929.42263184,78.63431936)
\lineto(929.42263184,79.26431936)
\lineto(929.42263184,79.45931936)
\curveto(929.42263349,79.52931683)(929.43263348,79.58931677)(929.45263184,79.63931936)
\curveto(929.49263342,79.70931665)(929.53263338,79.7593166)(929.57263184,79.78931936)
\curveto(929.62263329,79.82931653)(929.68763322,79.84931651)(929.76763184,79.84931936)
\curveto(929.84763306,79.8593165)(929.93263298,79.86431649)(930.02263184,79.86431936)
\lineto(930.74263184,79.86431936)
\curveto(931.22263169,79.86431649)(931.63263128,79.80431655)(931.97263184,79.68431936)
\curveto(932.3126306,79.56431679)(932.58763032,79.36931699)(932.79763184,79.09931936)
\curveto(932.84763006,79.02931733)(932.89263002,78.9593174)(932.93263184,78.88931936)
\curveto(932.98262993,78.82931753)(933.02762988,78.7543176)(933.06763184,78.66431936)
\curveto(933.07762983,78.64431771)(933.08762982,78.61431774)(933.09763184,78.57431936)
\curveto(933.11762979,78.53431782)(933.12262979,78.48931787)(933.11263184,78.43931936)
\curveto(933.08262983,78.34931801)(933.0076299,78.29431806)(932.88763184,78.27431936)
\curveto(932.77763013,78.2543181)(932.68263023,78.26931809)(932.60263184,78.31931936)
\curveto(932.53263038,78.34931801)(932.46763044,78.39431796)(932.40763184,78.45431936)
\curveto(932.35763055,78.52431783)(932.3076306,78.58931777)(932.25763184,78.64931936)
\curveto(932.2076307,78.71931764)(932.13263078,78.77931758)(932.03263184,78.82931936)
\curveto(931.94263097,78.88931747)(931.85263106,78.93931742)(931.76263184,78.97931936)
\curveto(931.73263118,78.99931736)(931.67263124,79.02431733)(931.58263184,79.05431936)
\curveto(931.50263141,79.08431727)(931.43263148,79.08931727)(931.37263184,79.06931936)
\curveto(931.23263168,79.03931732)(931.14263177,78.97931738)(931.10263184,78.88931936)
\curveto(931.07263184,78.80931755)(931.05763185,78.71931764)(931.05763184,78.61931936)
\curveto(931.05763185,78.51931784)(931.03263188,78.43431792)(930.98263184,78.36431936)
\curveto(930.912632,78.27431808)(930.77263214,78.22931813)(930.56263184,78.22931936)
\lineto(930.00763184,78.22931936)
\lineto(929.78263184,78.22931936)
\curveto(929.70263321,78.23931812)(929.63763327,78.2593181)(929.58763184,78.28931936)
\curveto(929.5076334,78.34931801)(929.46263345,78.41931794)(929.45263184,78.49931936)
\curveto(929.44263347,78.51931784)(929.43763347,78.53931782)(929.43763184,78.55931936)
\curveto(929.43763347,78.58931777)(929.43263348,78.61431774)(929.42263184,78.63431936)
}
}
{
\newrgbcolor{curcolor}{0 0 0}
\pscustom[linestyle=none,fillstyle=solid,fillcolor=curcolor]
{
}
}
{
\newrgbcolor{curcolor}{0 0 0}
\pscustom[linestyle=none,fillstyle=solid,fillcolor=curcolor]
{
\newpath
\moveto(920.45263184,89.26463186)
\curveto(920.44264247,89.95462722)(920.56264235,90.55462662)(920.81263184,91.06463186)
\curveto(921.06264185,91.58462559)(921.39764151,91.9796252)(921.81763184,92.24963186)
\curveto(921.89764101,92.29962488)(921.98764092,92.34462483)(922.08763184,92.38463186)
\curveto(922.17764073,92.42462475)(922.27264064,92.46962471)(922.37263184,92.51963186)
\curveto(922.47264044,92.55962462)(922.57264034,92.58962459)(922.67263184,92.60963186)
\curveto(922.77264014,92.62962455)(922.87764003,92.64962453)(922.98763184,92.66963186)
\curveto(923.03763987,92.68962449)(923.08263983,92.69462448)(923.12263184,92.68463186)
\curveto(923.16263975,92.6746245)(923.2076397,92.6796245)(923.25763184,92.69963186)
\curveto(923.3076396,92.70962447)(923.39263952,92.71462446)(923.51263184,92.71463186)
\curveto(923.62263929,92.71462446)(923.7076392,92.70962447)(923.76763184,92.69963186)
\curveto(923.82763908,92.6796245)(923.88763902,92.66962451)(923.94763184,92.66963186)
\curveto(924.0076389,92.6796245)(924.06763884,92.6746245)(924.12763184,92.65463186)
\curveto(924.26763864,92.61462456)(924.40263851,92.5796246)(924.53263184,92.54963186)
\curveto(924.66263825,92.51962466)(924.78763812,92.4796247)(924.90763184,92.42963186)
\curveto(925.04763786,92.36962481)(925.17263774,92.29962488)(925.28263184,92.21963186)
\curveto(925.39263752,92.14962503)(925.50263741,92.0746251)(925.61263184,91.99463186)
\lineto(925.67263184,91.93463186)
\curveto(925.69263722,91.92462525)(925.7126372,91.90962527)(925.73263184,91.88963186)
\curveto(925.89263702,91.76962541)(926.03763687,91.63462554)(926.16763184,91.48463186)
\curveto(926.29763661,91.33462584)(926.42263649,91.174626)(926.54263184,91.00463186)
\curveto(926.76263615,90.69462648)(926.96763594,90.39962678)(927.15763184,90.11963186)
\curveto(927.29763561,89.88962729)(927.43263548,89.65962752)(927.56263184,89.42963186)
\curveto(927.69263522,89.20962797)(927.82763508,88.98962819)(927.96763184,88.76963186)
\curveto(928.13763477,88.51962866)(928.31763459,88.2796289)(928.50763184,88.04963186)
\curveto(928.69763421,87.82962935)(928.92263399,87.63962954)(929.18263184,87.47963186)
\curveto(929.24263367,87.43962974)(929.30263361,87.40462977)(929.36263184,87.37463186)
\curveto(929.4126335,87.34462983)(929.47763343,87.31462986)(929.55763184,87.28463186)
\curveto(929.62763328,87.26462991)(929.68763322,87.25962992)(929.73763184,87.26963186)
\curveto(929.8076331,87.28962989)(929.86263305,87.32462985)(929.90263184,87.37463186)
\curveto(929.93263298,87.42462975)(929.95263296,87.48462969)(929.96263184,87.55463186)
\lineto(929.96263184,87.79463186)
\lineto(929.96263184,88.54463186)
\lineto(929.96263184,91.34963186)
\lineto(929.96263184,92.00963186)
\curveto(929.96263295,92.09962508)(929.96763294,92.18462499)(929.97763184,92.26463186)
\curveto(929.97763293,92.34462483)(929.99763291,92.40962477)(930.03763184,92.45963186)
\curveto(930.07763283,92.50962467)(930.15263276,92.54962463)(930.26263184,92.57963186)
\curveto(930.36263255,92.61962456)(930.46263245,92.62962455)(930.56263184,92.60963186)
\lineto(930.69763184,92.60963186)
\curveto(930.76763214,92.58962459)(930.82763208,92.56962461)(930.87763184,92.54963186)
\curveto(930.92763198,92.52962465)(930.96763194,92.49462468)(930.99763184,92.44463186)
\curveto(931.03763187,92.39462478)(931.05763185,92.32462485)(931.05763184,92.23463186)
\lineto(931.05763184,91.96463186)
\lineto(931.05763184,91.06463186)
\lineto(931.05763184,87.55463186)
\lineto(931.05763184,86.48963186)
\curveto(931.05763185,86.40963077)(931.06263185,86.31963086)(931.07263184,86.21963186)
\curveto(931.07263184,86.11963106)(931.06263185,86.03463114)(931.04263184,85.96463186)
\curveto(930.97263194,85.75463142)(930.79263212,85.68963149)(930.50263184,85.76963186)
\curveto(930.46263245,85.7796314)(930.42763248,85.7796314)(930.39763184,85.76963186)
\curveto(930.35763255,85.76963141)(930.3126326,85.7796314)(930.26263184,85.79963186)
\curveto(930.18263273,85.81963136)(930.09763281,85.83963134)(930.00763184,85.85963186)
\curveto(929.91763299,85.8796313)(929.83263308,85.90463127)(929.75263184,85.93463186)
\curveto(929.26263365,86.09463108)(928.84763406,86.29463088)(928.50763184,86.53463186)
\curveto(928.25763465,86.71463046)(928.03263488,86.91963026)(927.83263184,87.14963186)
\curveto(927.62263529,87.3796298)(927.42763548,87.61962956)(927.24763184,87.86963186)
\curveto(927.06763584,88.12962905)(926.89763601,88.39462878)(926.73763184,88.66463186)
\curveto(926.56763634,88.94462823)(926.39263652,89.21462796)(926.21263184,89.47463186)
\curveto(926.13263678,89.58462759)(926.05763685,89.68962749)(925.98763184,89.78963186)
\curveto(925.91763699,89.89962728)(925.84263707,90.00962717)(925.76263184,90.11963186)
\curveto(925.73263718,90.15962702)(925.70263721,90.19462698)(925.67263184,90.22463186)
\curveto(925.63263728,90.26462691)(925.60263731,90.30462687)(925.58263184,90.34463186)
\curveto(925.47263744,90.48462669)(925.34763756,90.60962657)(925.20763184,90.71963186)
\curveto(925.17763773,90.73962644)(925.15263776,90.76462641)(925.13263184,90.79463186)
\curveto(925.10263781,90.82462635)(925.07263784,90.84962633)(925.04263184,90.86963186)
\curveto(924.94263797,90.94962623)(924.84263807,91.01462616)(924.74263184,91.06463186)
\curveto(924.64263827,91.12462605)(924.53263838,91.179626)(924.41263184,91.22963186)
\curveto(924.34263857,91.25962592)(924.26763864,91.2796259)(924.18763184,91.28963186)
\lineto(923.94763184,91.34963186)
\lineto(923.85763184,91.34963186)
\curveto(923.82763908,91.35962582)(923.79763911,91.36462581)(923.76763184,91.36463186)
\curveto(923.69763921,91.38462579)(923.60263931,91.38962579)(923.48263184,91.37963186)
\curveto(923.35263956,91.3796258)(923.25263966,91.36962581)(923.18263184,91.34963186)
\curveto(923.10263981,91.32962585)(923.02763988,91.30962587)(922.95763184,91.28963186)
\curveto(922.87764003,91.2796259)(922.79764011,91.25962592)(922.71763184,91.22963186)
\curveto(922.47764043,91.11962606)(922.27764063,90.96962621)(922.11763184,90.77963186)
\curveto(921.94764096,90.59962658)(921.8076411,90.3796268)(921.69763184,90.11963186)
\curveto(921.67764123,90.04962713)(921.66264125,89.9796272)(921.65263184,89.90963186)
\curveto(921.63264128,89.83962734)(921.6126413,89.76462741)(921.59263184,89.68463186)
\curveto(921.57264134,89.60462757)(921.56264135,89.49462768)(921.56263184,89.35463186)
\curveto(921.56264135,89.22462795)(921.57264134,89.11962806)(921.59263184,89.03963186)
\curveto(921.60264131,88.9796282)(921.6076413,88.92462825)(921.60763184,88.87463186)
\curveto(921.6076413,88.82462835)(921.61764129,88.7746284)(921.63763184,88.72463186)
\curveto(921.67764123,88.62462855)(921.71764119,88.52962865)(921.75763184,88.43963186)
\curveto(921.79764111,88.35962882)(921.84264107,88.2796289)(921.89263184,88.19963186)
\curveto(921.912641,88.16962901)(921.93764097,88.13962904)(921.96763184,88.10963186)
\curveto(921.99764091,88.08962909)(922.02264089,88.06462911)(922.04263184,88.03463186)
\lineto(922.11763184,87.95963186)
\curveto(922.13764077,87.92962925)(922.15764075,87.90462927)(922.17763184,87.88463186)
\lineto(922.38763184,87.73463186)
\curveto(922.44764046,87.69462948)(922.5126404,87.64962953)(922.58263184,87.59963186)
\curveto(922.67264024,87.53962964)(922.77764013,87.48962969)(922.89763184,87.44963186)
\curveto(923.0076399,87.41962976)(923.11763979,87.38462979)(923.22763184,87.34463186)
\curveto(923.33763957,87.30462987)(923.48263943,87.2796299)(923.66263184,87.26963186)
\curveto(923.83263908,87.25962992)(923.95763895,87.22962995)(924.03763184,87.17963186)
\curveto(924.11763879,87.12963005)(924.16263875,87.05463012)(924.17263184,86.95463186)
\curveto(924.18263873,86.85463032)(924.18763872,86.74463043)(924.18763184,86.62463186)
\curveto(924.18763872,86.58463059)(924.19263872,86.54463063)(924.20263184,86.50463186)
\curveto(924.20263871,86.46463071)(924.19763871,86.42963075)(924.18763184,86.39963186)
\curveto(924.16763874,86.34963083)(924.15763875,86.29963088)(924.15763184,86.24963186)
\curveto(924.15763875,86.20963097)(924.14763876,86.16963101)(924.12763184,86.12963186)
\curveto(924.06763884,86.03963114)(923.93263898,85.99463118)(923.72263184,85.99463186)
\lineto(923.60263184,85.99463186)
\curveto(923.54263937,86.00463117)(923.48263943,86.00963117)(923.42263184,86.00963186)
\curveto(923.35263956,86.01963116)(923.28763962,86.02963115)(923.22763184,86.03963186)
\curveto(923.11763979,86.05963112)(923.01763989,86.0796311)(922.92763184,86.09963186)
\curveto(922.82764008,86.11963106)(922.73264018,86.14963103)(922.64263184,86.18963186)
\curveto(922.57264034,86.20963097)(922.5126404,86.22963095)(922.46263184,86.24963186)
\lineto(922.28263184,86.30963186)
\curveto(922.02264089,86.42963075)(921.77764113,86.58463059)(921.54763184,86.77463186)
\curveto(921.31764159,86.9746302)(921.13264178,87.18962999)(920.99263184,87.41963186)
\curveto(920.912642,87.52962965)(920.84764206,87.64462953)(920.79763184,87.76463186)
\lineto(920.64763184,88.15463186)
\curveto(920.59764231,88.26462891)(920.56764234,88.3796288)(920.55763184,88.49963186)
\curveto(920.53764237,88.61962856)(920.5126424,88.74462843)(920.48263184,88.87463186)
\curveto(920.48264243,88.94462823)(920.48264243,89.00962817)(920.48263184,89.06963186)
\curveto(920.47264244,89.12962805)(920.46264245,89.19462798)(920.45263184,89.26463186)
}
}
{
\newrgbcolor{curcolor}{0 0 0}
\pscustom[linestyle=none,fillstyle=solid,fillcolor=curcolor]
{
\newpath
\moveto(925.97263184,101.36424123)
\lineto(926.22763184,101.36424123)
\curveto(926.3076366,101.37423353)(926.38263653,101.36923353)(926.45263184,101.34924123)
\lineto(926.69263184,101.34924123)
\lineto(926.85763184,101.34924123)
\curveto(926.95763595,101.32923357)(927.06263585,101.31923358)(927.17263184,101.31924123)
\curveto(927.27263564,101.31923358)(927.37263554,101.30923359)(927.47263184,101.28924123)
\lineto(927.62263184,101.28924123)
\curveto(927.76263515,101.25923364)(927.90263501,101.23923366)(928.04263184,101.22924123)
\curveto(928.17263474,101.21923368)(928.30263461,101.19423371)(928.43263184,101.15424123)
\curveto(928.5126344,101.13423377)(928.59763431,101.11423379)(928.68763184,101.09424123)
\lineto(928.92763184,101.03424123)
\lineto(929.22763184,100.91424123)
\curveto(929.31763359,100.88423402)(929.4076335,100.84923405)(929.49763184,100.80924123)
\curveto(929.71763319,100.70923419)(929.93263298,100.57423433)(930.14263184,100.40424123)
\curveto(930.35263256,100.24423466)(930.52263239,100.06923483)(930.65263184,99.87924123)
\curveto(930.69263222,99.82923507)(930.73263218,99.76923513)(930.77263184,99.69924123)
\curveto(930.80263211,99.63923526)(930.83763207,99.57923532)(930.87763184,99.51924123)
\curveto(930.92763198,99.43923546)(930.96763194,99.34423556)(930.99763184,99.23424123)
\curveto(931.02763188,99.12423578)(931.05763185,99.01923588)(931.08763184,98.91924123)
\curveto(931.12763178,98.80923609)(931.15263176,98.6992362)(931.16263184,98.58924123)
\curveto(931.17263174,98.47923642)(931.18763172,98.36423654)(931.20763184,98.24424123)
\curveto(931.21763169,98.2042367)(931.21763169,98.15923674)(931.20763184,98.10924123)
\curveto(931.2076317,98.06923683)(931.2126317,98.02923687)(931.22263184,97.98924123)
\curveto(931.23263168,97.94923695)(931.23763167,97.89423701)(931.23763184,97.82424123)
\curveto(931.23763167,97.75423715)(931.23263168,97.7042372)(931.22263184,97.67424123)
\curveto(931.20263171,97.62423728)(931.19763171,97.57923732)(931.20763184,97.53924123)
\curveto(931.21763169,97.4992374)(931.21763169,97.46423744)(931.20763184,97.43424123)
\lineto(931.20763184,97.34424123)
\curveto(931.18763172,97.28423762)(931.17263174,97.21923768)(931.16263184,97.14924123)
\curveto(931.16263175,97.08923781)(931.15763175,97.02423788)(931.14763184,96.95424123)
\curveto(931.09763181,96.78423812)(931.04763186,96.62423828)(930.99763184,96.47424123)
\curveto(930.94763196,96.32423858)(930.88263203,96.17923872)(930.80263184,96.03924123)
\curveto(930.76263215,95.98923891)(930.73263218,95.93423897)(930.71263184,95.87424123)
\curveto(930.68263223,95.82423908)(930.64763226,95.77423913)(930.60763184,95.72424123)
\curveto(930.42763248,95.48423942)(930.2076327,95.28423962)(929.94763184,95.12424123)
\curveto(929.68763322,94.96423994)(929.40263351,94.82424008)(929.09263184,94.70424123)
\curveto(928.95263396,94.64424026)(928.8126341,94.5992403)(928.67263184,94.56924123)
\curveto(928.52263439,94.53924036)(928.36763454,94.5042404)(928.20763184,94.46424123)
\curveto(928.09763481,94.44424046)(927.98763492,94.42924047)(927.87763184,94.41924123)
\curveto(927.76763514,94.40924049)(927.65763525,94.39424051)(927.54763184,94.37424123)
\curveto(927.5076354,94.36424054)(927.46763544,94.35924054)(927.42763184,94.35924123)
\curveto(927.38763552,94.36924053)(927.34763556,94.36924053)(927.30763184,94.35924123)
\curveto(927.25763565,94.34924055)(927.2076357,94.34424056)(927.15763184,94.34424123)
\lineto(926.99263184,94.34424123)
\curveto(926.94263597,94.32424058)(926.89263602,94.31924058)(926.84263184,94.32924123)
\curveto(926.78263613,94.33924056)(926.72763618,94.33924056)(926.67763184,94.32924123)
\curveto(926.63763627,94.31924058)(926.59263632,94.31924058)(926.54263184,94.32924123)
\curveto(926.49263642,94.33924056)(926.44263647,94.33424057)(926.39263184,94.31424123)
\curveto(926.32263659,94.29424061)(926.24763666,94.28924061)(926.16763184,94.29924123)
\curveto(926.07763683,94.30924059)(925.99263692,94.31424059)(925.91263184,94.31424123)
\curveto(925.82263709,94.31424059)(925.72263719,94.30924059)(925.61263184,94.29924123)
\curveto(925.49263742,94.28924061)(925.39263752,94.29424061)(925.31263184,94.31424123)
\lineto(925.02763184,94.31424123)
\lineto(924.39763184,94.35924123)
\curveto(924.29763861,94.36924053)(924.20263871,94.37924052)(924.11263184,94.38924123)
\lineto(923.81263184,94.41924123)
\curveto(923.76263915,94.43924046)(923.7126392,94.44424046)(923.66263184,94.43424123)
\curveto(923.60263931,94.43424047)(923.54763936,94.44424046)(923.49763184,94.46424123)
\curveto(923.32763958,94.51424039)(923.16263975,94.55424035)(923.00263184,94.58424123)
\curveto(922.83264008,94.61424029)(922.67264024,94.66424024)(922.52263184,94.73424123)
\curveto(922.06264085,94.92423998)(921.68764122,95.14423976)(921.39763184,95.39424123)
\curveto(921.1076418,95.65423925)(920.86264205,96.01423889)(920.66263184,96.47424123)
\curveto(920.6126423,96.6042383)(920.57764233,96.73423817)(920.55763184,96.86424123)
\curveto(920.53764237,97.0042379)(920.5126424,97.14423776)(920.48263184,97.28424123)
\curveto(920.47264244,97.35423755)(920.46764244,97.41923748)(920.46763184,97.47924123)
\curveto(920.46764244,97.53923736)(920.46264245,97.6042373)(920.45263184,97.67424123)
\curveto(920.43264248,98.5042364)(920.58264233,99.17423573)(920.90263184,99.68424123)
\curveto(921.2126417,100.19423471)(921.65264126,100.57423433)(922.22263184,100.82424123)
\curveto(922.34264057,100.87423403)(922.46764044,100.91923398)(922.59763184,100.95924123)
\curveto(922.72764018,100.9992339)(922.86264005,101.04423386)(923.00263184,101.09424123)
\curveto(923.08263983,101.11423379)(923.16763974,101.12923377)(923.25763184,101.13924123)
\lineto(923.49763184,101.19924123)
\curveto(923.6076393,101.22923367)(923.71763919,101.24423366)(923.82763184,101.24424123)
\curveto(923.93763897,101.25423365)(924.04763886,101.26923363)(924.15763184,101.28924123)
\curveto(924.2076387,101.30923359)(924.25263866,101.31423359)(924.29263184,101.30424123)
\curveto(924.33263858,101.3042336)(924.37263854,101.30923359)(924.41263184,101.31924123)
\curveto(924.46263845,101.32923357)(924.51763839,101.32923357)(924.57763184,101.31924123)
\curveto(924.62763828,101.31923358)(924.67763823,101.32423358)(924.72763184,101.33424123)
\lineto(924.86263184,101.33424123)
\curveto(924.92263799,101.35423355)(924.99263792,101.35423355)(925.07263184,101.33424123)
\curveto(925.14263777,101.32423358)(925.2076377,101.32923357)(925.26763184,101.34924123)
\curveto(925.29763761,101.35923354)(925.33763757,101.36423354)(925.38763184,101.36424123)
\lineto(925.50763184,101.36424123)
\lineto(925.97263184,101.36424123)
\moveto(928.29763184,99.81924123)
\curveto(927.97763493,99.91923498)(927.6126353,99.97923492)(927.20263184,99.99924123)
\curveto(926.79263612,100.01923488)(926.38263653,100.02923487)(925.97263184,100.02924123)
\curveto(925.54263737,100.02923487)(925.12263779,100.01923488)(924.71263184,99.99924123)
\curveto(924.30263861,99.97923492)(923.91763899,99.93423497)(923.55763184,99.86424123)
\curveto(923.19763971,99.79423511)(922.87764003,99.68423522)(922.59763184,99.53424123)
\curveto(922.3076406,99.39423551)(922.07264084,99.1992357)(921.89263184,98.94924123)
\curveto(921.78264113,98.78923611)(921.70264121,98.60923629)(921.65263184,98.40924123)
\curveto(921.59264132,98.20923669)(921.56264135,97.96423694)(921.56263184,97.67424123)
\curveto(921.58264133,97.65423725)(921.59264132,97.61923728)(921.59263184,97.56924123)
\curveto(921.58264133,97.51923738)(921.58264133,97.47923742)(921.59263184,97.44924123)
\curveto(921.6126413,97.36923753)(921.63264128,97.29423761)(921.65263184,97.22424123)
\curveto(921.66264125,97.16423774)(921.68264123,97.0992378)(921.71263184,97.02924123)
\curveto(921.83264108,96.75923814)(922.00264091,96.53923836)(922.22263184,96.36924123)
\curveto(922.43264048,96.20923869)(922.67764023,96.07423883)(922.95763184,95.96424123)
\curveto(923.06763984,95.91423899)(923.18763972,95.87423903)(923.31763184,95.84424123)
\curveto(923.43763947,95.82423908)(923.56263935,95.7992391)(923.69263184,95.76924123)
\curveto(923.74263917,95.74923915)(923.79763911,95.73923916)(923.85763184,95.73924123)
\curveto(923.907639,95.73923916)(923.95763895,95.73423917)(924.00763184,95.72424123)
\curveto(924.09763881,95.71423919)(924.19263872,95.7042392)(924.29263184,95.69424123)
\curveto(924.38263853,95.68423922)(924.47763843,95.67423923)(924.57763184,95.66424123)
\curveto(924.65763825,95.66423924)(924.74263817,95.65923924)(924.83263184,95.64924123)
\lineto(925.07263184,95.64924123)
\lineto(925.25263184,95.64924123)
\curveto(925.28263763,95.63923926)(925.31763759,95.63423927)(925.35763184,95.63424123)
\lineto(925.49263184,95.63424123)
\lineto(925.94263184,95.63424123)
\curveto(926.02263689,95.63423927)(926.1076368,95.62923927)(926.19763184,95.61924123)
\curveto(926.27763663,95.61923928)(926.35263656,95.62923927)(926.42263184,95.64924123)
\lineto(926.69263184,95.64924123)
\curveto(926.7126362,95.64923925)(926.74263617,95.64423926)(926.78263184,95.63424123)
\curveto(926.8126361,95.63423927)(926.83763607,95.63923926)(926.85763184,95.64924123)
\curveto(926.95763595,95.65923924)(927.05763585,95.66423924)(927.15763184,95.66424123)
\curveto(927.24763566,95.67423923)(927.34763556,95.68423922)(927.45763184,95.69424123)
\curveto(927.57763533,95.72423918)(927.70263521,95.73923916)(927.83263184,95.73924123)
\curveto(927.95263496,95.74923915)(928.06763484,95.77423913)(928.17763184,95.81424123)
\curveto(928.47763443,95.89423901)(928.74263417,95.97923892)(928.97263184,96.06924123)
\curveto(929.20263371,96.16923873)(929.41763349,96.31423859)(929.61763184,96.50424123)
\curveto(929.81763309,96.71423819)(929.96763294,96.97923792)(930.06763184,97.29924123)
\curveto(930.08763282,97.33923756)(930.09763281,97.37423753)(930.09763184,97.40424123)
\curveto(930.08763282,97.44423746)(930.09263282,97.48923741)(930.11263184,97.53924123)
\curveto(930.12263279,97.57923732)(930.13263278,97.64923725)(930.14263184,97.74924123)
\curveto(930.15263276,97.85923704)(930.14763276,97.94423696)(930.12763184,98.00424123)
\curveto(930.1076328,98.07423683)(930.09763281,98.14423676)(930.09763184,98.21424123)
\curveto(930.08763282,98.28423662)(930.07263284,98.34923655)(930.05263184,98.40924123)
\curveto(929.99263292,98.60923629)(929.907633,98.78923611)(929.79763184,98.94924123)
\curveto(929.77763313,98.97923592)(929.75763315,99.0042359)(929.73763184,99.02424123)
\lineto(929.67763184,99.08424123)
\curveto(929.65763325,99.12423578)(929.61763329,99.17423573)(929.55763184,99.23424123)
\curveto(929.41763349,99.33423557)(929.28763362,99.41923548)(929.16763184,99.48924123)
\curveto(929.04763386,99.55923534)(928.90263401,99.62923527)(928.73263184,99.69924123)
\curveto(928.66263425,99.72923517)(928.59263432,99.74923515)(928.52263184,99.75924123)
\curveto(928.45263446,99.77923512)(928.37763453,99.7992351)(928.29763184,99.81924123)
}
}
{
\newrgbcolor{curcolor}{0 0 0}
\pscustom[linestyle=none,fillstyle=solid,fillcolor=curcolor]
{
\newpath
\moveto(920.45263184,106.77385061)
\curveto(920.45264246,106.87384575)(920.46264245,106.96884566)(920.48263184,107.05885061)
\curveto(920.49264242,107.14884548)(920.52264239,107.21384541)(920.57263184,107.25385061)
\curveto(920.65264226,107.31384531)(920.75764215,107.34384528)(920.88763184,107.34385061)
\lineto(921.27763184,107.34385061)
\lineto(922.77763184,107.34385061)
\lineto(929.16763184,107.34385061)
\lineto(930.33763184,107.34385061)
\lineto(930.65263184,107.34385061)
\curveto(930.75263216,107.35384527)(930.83263208,107.33884529)(930.89263184,107.29885061)
\curveto(930.97263194,107.24884538)(931.02263189,107.17384545)(931.04263184,107.07385061)
\curveto(931.05263186,106.98384564)(931.05763185,106.87384575)(931.05763184,106.74385061)
\lineto(931.05763184,106.51885061)
\curveto(931.03763187,106.43884619)(931.02263189,106.36884626)(931.01263184,106.30885061)
\curveto(930.99263192,106.24884638)(930.95263196,106.19884643)(930.89263184,106.15885061)
\curveto(930.83263208,106.11884651)(930.75763215,106.09884653)(930.66763184,106.09885061)
\lineto(930.36763184,106.09885061)
\lineto(929.27263184,106.09885061)
\lineto(923.93263184,106.09885061)
\curveto(923.84263907,106.07884655)(923.76763914,106.06384656)(923.70763184,106.05385061)
\curveto(923.63763927,106.05384657)(923.57763933,106.0238466)(923.52763184,105.96385061)
\curveto(923.47763943,105.89384673)(923.45263946,105.80384682)(923.45263184,105.69385061)
\curveto(923.44263947,105.59384703)(923.43763947,105.48384714)(923.43763184,105.36385061)
\lineto(923.43763184,104.22385061)
\lineto(923.43763184,103.72885061)
\curveto(923.42763948,103.56884906)(923.36763954,103.45884917)(923.25763184,103.39885061)
\curveto(923.22763968,103.37884925)(923.19763971,103.36884926)(923.16763184,103.36885061)
\curveto(923.12763978,103.36884926)(923.08263983,103.36384926)(923.03263184,103.35385061)
\curveto(922.91264,103.33384929)(922.80264011,103.33884929)(922.70263184,103.36885061)
\curveto(922.60264031,103.40884922)(922.53264038,103.46384916)(922.49263184,103.53385061)
\curveto(922.44264047,103.61384901)(922.41764049,103.73384889)(922.41763184,103.89385061)
\curveto(922.41764049,104.05384857)(922.40264051,104.18884844)(922.37263184,104.29885061)
\curveto(922.36264055,104.34884828)(922.35764055,104.40384822)(922.35763184,104.46385061)
\curveto(922.34764056,104.5238481)(922.33264058,104.58384804)(922.31263184,104.64385061)
\curveto(922.26264065,104.79384783)(922.2126407,104.93884769)(922.16263184,105.07885061)
\curveto(922.10264081,105.21884741)(922.03264088,105.35384727)(921.95263184,105.48385061)
\curveto(921.86264105,105.623847)(921.75764115,105.74384688)(921.63763184,105.84385061)
\curveto(921.51764139,105.94384668)(921.38764152,106.03884659)(921.24763184,106.12885061)
\curveto(921.14764176,106.18884644)(921.03764187,106.23384639)(920.91763184,106.26385061)
\curveto(920.79764211,106.30384632)(920.69264222,106.35384627)(920.60263184,106.41385061)
\curveto(920.54264237,106.46384616)(920.50264241,106.53384609)(920.48263184,106.62385061)
\curveto(920.47264244,106.64384598)(920.46764244,106.66884596)(920.46763184,106.69885061)
\curveto(920.46764244,106.7288459)(920.46264245,106.75384587)(920.45263184,106.77385061)
}
}
{
\newrgbcolor{curcolor}{0 0 0}
\pscustom[linestyle=none,fillstyle=solid,fillcolor=curcolor]
{
\newpath
\moveto(920.45263184,115.12345998)
\curveto(920.45264246,115.22345513)(920.46264245,115.31845503)(920.48263184,115.40845998)
\curveto(920.49264242,115.49845485)(920.52264239,115.56345479)(920.57263184,115.60345998)
\curveto(920.65264226,115.66345469)(920.75764215,115.69345466)(920.88763184,115.69345998)
\lineto(921.27763184,115.69345998)
\lineto(922.77763184,115.69345998)
\lineto(929.16763184,115.69345998)
\lineto(930.33763184,115.69345998)
\lineto(930.65263184,115.69345998)
\curveto(930.75263216,115.70345465)(930.83263208,115.68845466)(930.89263184,115.64845998)
\curveto(930.97263194,115.59845475)(931.02263189,115.52345483)(931.04263184,115.42345998)
\curveto(931.05263186,115.33345502)(931.05763185,115.22345513)(931.05763184,115.09345998)
\lineto(931.05763184,114.86845998)
\curveto(931.03763187,114.78845556)(931.02263189,114.71845563)(931.01263184,114.65845998)
\curveto(930.99263192,114.59845575)(930.95263196,114.5484558)(930.89263184,114.50845998)
\curveto(930.83263208,114.46845588)(930.75763215,114.4484559)(930.66763184,114.44845998)
\lineto(930.36763184,114.44845998)
\lineto(929.27263184,114.44845998)
\lineto(923.93263184,114.44845998)
\curveto(923.84263907,114.42845592)(923.76763914,114.41345594)(923.70763184,114.40345998)
\curveto(923.63763927,114.40345595)(923.57763933,114.37345598)(923.52763184,114.31345998)
\curveto(923.47763943,114.24345611)(923.45263946,114.1534562)(923.45263184,114.04345998)
\curveto(923.44263947,113.94345641)(923.43763947,113.83345652)(923.43763184,113.71345998)
\lineto(923.43763184,112.57345998)
\lineto(923.43763184,112.07845998)
\curveto(923.42763948,111.91845843)(923.36763954,111.80845854)(923.25763184,111.74845998)
\curveto(923.22763968,111.72845862)(923.19763971,111.71845863)(923.16763184,111.71845998)
\curveto(923.12763978,111.71845863)(923.08263983,111.71345864)(923.03263184,111.70345998)
\curveto(922.91264,111.68345867)(922.80264011,111.68845866)(922.70263184,111.71845998)
\curveto(922.60264031,111.75845859)(922.53264038,111.81345854)(922.49263184,111.88345998)
\curveto(922.44264047,111.96345839)(922.41764049,112.08345827)(922.41763184,112.24345998)
\curveto(922.41764049,112.40345795)(922.40264051,112.53845781)(922.37263184,112.64845998)
\curveto(922.36264055,112.69845765)(922.35764055,112.7534576)(922.35763184,112.81345998)
\curveto(922.34764056,112.87345748)(922.33264058,112.93345742)(922.31263184,112.99345998)
\curveto(922.26264065,113.14345721)(922.2126407,113.28845706)(922.16263184,113.42845998)
\curveto(922.10264081,113.56845678)(922.03264088,113.70345665)(921.95263184,113.83345998)
\curveto(921.86264105,113.97345638)(921.75764115,114.09345626)(921.63763184,114.19345998)
\curveto(921.51764139,114.29345606)(921.38764152,114.38845596)(921.24763184,114.47845998)
\curveto(921.14764176,114.53845581)(921.03764187,114.58345577)(920.91763184,114.61345998)
\curveto(920.79764211,114.6534557)(920.69264222,114.70345565)(920.60263184,114.76345998)
\curveto(920.54264237,114.81345554)(920.50264241,114.88345547)(920.48263184,114.97345998)
\curveto(920.47264244,114.99345536)(920.46764244,115.01845533)(920.46763184,115.04845998)
\curveto(920.46764244,115.07845527)(920.46264245,115.10345525)(920.45263184,115.12345998)
}
}
{
\newrgbcolor{curcolor}{0 0 0}
\pscustom[linestyle=none,fillstyle=solid,fillcolor=curcolor]
{
\newpath
\moveto(941.28894775,42.29681936)
\curveto(941.28895845,42.36681368)(941.28895845,42.4468136)(941.28894775,42.53681936)
\curveto(941.27895846,42.62681342)(941.27895846,42.71181333)(941.28894775,42.79181936)
\curveto(941.28895845,42.88181316)(941.29895844,42.96181308)(941.31894775,43.03181936)
\curveto(941.3389584,43.11181293)(941.36895837,43.16681288)(941.40894775,43.19681936)
\curveto(941.45895828,43.22681282)(941.5339582,43.2468128)(941.63394775,43.25681936)
\curveto(941.72395801,43.27681277)(941.82895791,43.28681276)(941.94894775,43.28681936)
\curveto(942.05895768,43.29681275)(942.17395756,43.29681275)(942.29394775,43.28681936)
\lineto(942.59394775,43.28681936)
\lineto(945.60894775,43.28681936)
\lineto(948.50394775,43.28681936)
\curveto(948.8339509,43.28681276)(949.15895058,43.28181276)(949.47894775,43.27181936)
\curveto(949.78894995,43.27181277)(950.06894967,43.23181281)(950.31894775,43.15181936)
\curveto(950.66894907,43.03181301)(950.96394877,42.87681317)(951.20394775,42.68681936)
\curveto(951.4339483,42.49681355)(951.6339481,42.25681379)(951.80394775,41.96681936)
\curveto(951.85394788,41.90681414)(951.88894785,41.8418142)(951.90894775,41.77181936)
\curveto(951.92894781,41.71181433)(951.95394778,41.6418144)(951.98394775,41.56181936)
\curveto(952.0339477,41.4418146)(952.06894767,41.31181473)(952.08894775,41.17181936)
\curveto(952.11894762,41.041815)(952.14894759,40.90681514)(952.17894775,40.76681936)
\curveto(952.19894754,40.71681533)(952.20394753,40.66681538)(952.19394775,40.61681936)
\curveto(952.18394755,40.56681548)(952.18394755,40.51181553)(952.19394775,40.45181936)
\curveto(952.20394753,40.43181561)(952.20394753,40.40681564)(952.19394775,40.37681936)
\curveto(952.19394754,40.3468157)(952.19894754,40.32181572)(952.20894775,40.30181936)
\curveto(952.21894752,40.26181578)(952.22394751,40.20681584)(952.22394775,40.13681936)
\curveto(952.22394751,40.06681598)(952.21894752,40.01181603)(952.20894775,39.97181936)
\curveto(952.19894754,39.92181612)(952.19894754,39.86681618)(952.20894775,39.80681936)
\curveto(952.21894752,39.7468163)(952.21394752,39.69181635)(952.19394775,39.64181936)
\curveto(952.16394757,39.51181653)(952.14394759,39.38681666)(952.13394775,39.26681936)
\curveto(952.12394761,39.1468169)(952.09894764,39.03181701)(952.05894775,38.92181936)
\curveto(951.9389478,38.55181749)(951.76894797,38.23181781)(951.54894775,37.96181936)
\curveto(951.32894841,37.69181835)(951.04894869,37.48181856)(950.70894775,37.33181936)
\curveto(950.58894915,37.28181876)(950.46394927,37.23681881)(950.33394775,37.19681936)
\curveto(950.20394953,37.16681888)(950.06894967,37.13181891)(949.92894775,37.09181936)
\curveto(949.87894986,37.08181896)(949.8389499,37.07681897)(949.80894775,37.07681936)
\curveto(949.76894997,37.07681897)(949.72395001,37.07181897)(949.67394775,37.06181936)
\curveto(949.64395009,37.05181899)(949.60895013,37.046819)(949.56894775,37.04681936)
\curveto(949.51895022,37.046819)(949.47895026,37.041819)(949.44894775,37.03181936)
\lineto(949.28394775,37.03181936)
\curveto(949.20395053,37.01181903)(949.10395063,37.00681904)(948.98394775,37.01681936)
\curveto(948.85395088,37.02681902)(948.76395097,37.041819)(948.71394775,37.06181936)
\curveto(948.62395111,37.08181896)(948.55895118,37.13681891)(948.51894775,37.22681936)
\curveto(948.49895124,37.25681879)(948.49395124,37.28681876)(948.50394775,37.31681936)
\curveto(948.50395123,37.3468187)(948.49895124,37.38681866)(948.48894775,37.43681936)
\curveto(948.47895126,37.47681857)(948.47395126,37.51681853)(948.47394775,37.55681936)
\lineto(948.47394775,37.70681936)
\curveto(948.47395126,37.82681822)(948.47895126,37.9468181)(948.48894775,38.06681936)
\curveto(948.48895125,38.19681785)(948.52395121,38.28681776)(948.59394775,38.33681936)
\curveto(948.65395108,38.37681767)(948.71395102,38.39681765)(948.77394775,38.39681936)
\curveto(948.8339509,38.39681765)(948.90395083,38.40681764)(948.98394775,38.42681936)
\curveto(949.01395072,38.43681761)(949.04895069,38.43681761)(949.08894775,38.42681936)
\curveto(949.11895062,38.42681762)(949.14395059,38.43181761)(949.16394775,38.44181936)
\lineto(949.37394775,38.44181936)
\curveto(949.42395031,38.46181758)(949.47395026,38.46681758)(949.52394775,38.45681936)
\curveto(949.56395017,38.45681759)(949.60895013,38.46681758)(949.65894775,38.48681936)
\curveto(949.78894995,38.51681753)(949.91394982,38.5468175)(950.03394775,38.57681936)
\curveto(950.14394959,38.60681744)(950.24894949,38.65181739)(950.34894775,38.71181936)
\curveto(950.6389491,38.88181716)(950.84394889,39.15181689)(950.96394775,39.52181936)
\curveto(950.98394875,39.57181647)(950.99894874,39.62181642)(951.00894775,39.67181936)
\curveto(951.00894873,39.73181631)(951.01894872,39.78681626)(951.03894775,39.83681936)
\lineto(951.03894775,39.91181936)
\curveto(951.04894869,39.98181606)(951.05894868,40.07681597)(951.06894775,40.19681936)
\curveto(951.06894867,40.32681572)(951.05894868,40.42681562)(951.03894775,40.49681936)
\curveto(951.01894872,40.56681548)(951.00394873,40.63681541)(950.99394775,40.70681936)
\curveto(950.97394876,40.78681526)(950.95394878,40.85681519)(950.93394775,40.91681936)
\curveto(950.77394896,41.29681475)(950.49894924,41.57181447)(950.10894775,41.74181936)
\curveto(949.97894976,41.79181425)(949.82394991,41.82681422)(949.64394775,41.84681936)
\curveto(949.46395027,41.87681417)(949.27895046,41.89181415)(949.08894775,41.89181936)
\curveto(948.88895085,41.90181414)(948.68895105,41.90181414)(948.48894775,41.89181936)
\lineto(947.91894775,41.89181936)
\lineto(943.67394775,41.89181936)
\lineto(942.12894775,41.89181936)
\curveto(942.01895772,41.89181415)(941.89895784,41.88681416)(941.76894775,41.87681936)
\curveto(941.6389581,41.86681418)(941.5339582,41.88681416)(941.45394775,41.93681936)
\curveto(941.38395835,41.99681405)(941.3339584,42.07681397)(941.30394775,42.17681936)
\curveto(941.30395843,42.19681385)(941.30395843,42.21681383)(941.30394775,42.23681936)
\curveto(941.30395843,42.25681379)(941.29895844,42.27681377)(941.28894775,42.29681936)
}
}
{
\newrgbcolor{curcolor}{0 0 0}
\pscustom[linestyle=none,fillstyle=solid,fillcolor=curcolor]
{
\newpath
\moveto(944.24394775,45.83049123)
\lineto(944.24394775,46.26549123)
\curveto(944.24395549,46.41548927)(944.28395545,46.52048916)(944.36394775,46.58049123)
\curveto(944.44395529,46.63048905)(944.54395519,46.65548903)(944.66394775,46.65549123)
\curveto(944.78395495,46.66548902)(944.90395483,46.67048901)(945.02394775,46.67049123)
\lineto(946.44894775,46.67049123)
\lineto(948.71394775,46.67049123)
\lineto(949.40394775,46.67049123)
\curveto(949.6339501,46.67048901)(949.8339499,46.69548899)(950.00394775,46.74549123)
\curveto(950.45394928,46.90548878)(950.76894897,47.20548848)(950.94894775,47.64549123)
\curveto(951.0389487,47.86548782)(951.07394866,48.13048755)(951.05394775,48.44049123)
\curveto(951.02394871,48.75048693)(950.96894877,49.00048668)(950.88894775,49.19049123)
\curveto(950.74894899,49.52048616)(950.57394916,49.7804859)(950.36394775,49.97049123)
\curveto(950.14394959,50.17048551)(949.85894988,50.32548536)(949.50894775,50.43549123)
\curveto(949.42895031,50.46548522)(949.34895039,50.4854852)(949.26894775,50.49549123)
\curveto(949.18895055,50.50548518)(949.10395063,50.52048516)(949.01394775,50.54049123)
\curveto(948.96395077,50.55048513)(948.91895082,50.55048513)(948.87894775,50.54049123)
\curveto(948.8389509,50.54048514)(948.79395094,50.55048513)(948.74394775,50.57049123)
\lineto(948.42894775,50.57049123)
\curveto(948.34895139,50.59048509)(948.25895148,50.59548509)(948.15894775,50.58549123)
\curveto(948.04895169,50.57548511)(947.94895179,50.57048511)(947.85894775,50.57049123)
\lineto(946.68894775,50.57049123)
\lineto(945.09894775,50.57049123)
\curveto(944.97895476,50.57048511)(944.85395488,50.56548512)(944.72394775,50.55549123)
\curveto(944.58395515,50.55548513)(944.47395526,50.5804851)(944.39394775,50.63049123)
\curveto(944.34395539,50.67048501)(944.31395542,50.71548497)(944.30394775,50.76549123)
\curveto(944.28395545,50.82548486)(944.26395547,50.89548479)(944.24394775,50.97549123)
\lineto(944.24394775,51.20049123)
\curveto(944.24395549,51.32048436)(944.24895549,51.42548426)(944.25894775,51.51549123)
\curveto(944.26895547,51.61548407)(944.31395542,51.69048399)(944.39394775,51.74049123)
\curveto(944.44395529,51.79048389)(944.51895522,51.81548387)(944.61894775,51.81549123)
\lineto(944.90394775,51.81549123)
\lineto(945.92394775,51.81549123)
\lineto(949.95894775,51.81549123)
\lineto(951.30894775,51.81549123)
\curveto(951.42894831,51.81548387)(951.54394819,51.81048387)(951.65394775,51.80049123)
\curveto(951.75394798,51.80048388)(951.82894791,51.76548392)(951.87894775,51.69549123)
\curveto(951.90894783,51.65548403)(951.9339478,51.59548409)(951.95394775,51.51549123)
\curveto(951.96394777,51.43548425)(951.97394776,51.34548434)(951.98394775,51.24549123)
\curveto(951.98394775,51.15548453)(951.97894776,51.06548462)(951.96894775,50.97549123)
\curveto(951.95894778,50.89548479)(951.9389478,50.83548485)(951.90894775,50.79549123)
\curveto(951.86894787,50.74548494)(951.80394793,50.70048498)(951.71394775,50.66049123)
\curveto(951.67394806,50.65048503)(951.61894812,50.64048504)(951.54894775,50.63049123)
\curveto(951.47894826,50.63048505)(951.41394832,50.62548506)(951.35394775,50.61549123)
\curveto(951.28394845,50.60548508)(951.22894851,50.5854851)(951.18894775,50.55549123)
\curveto(951.14894859,50.52548516)(951.1339486,50.4804852)(951.14394775,50.42049123)
\curveto(951.16394857,50.34048534)(951.22394851,50.26048542)(951.32394775,50.18049123)
\curveto(951.41394832,50.10048558)(951.48394825,50.02548566)(951.53394775,49.95549123)
\curveto(951.69394804,49.73548595)(951.8339479,49.4854862)(951.95394775,49.20549123)
\curveto(952.00394773,49.09548659)(952.0339477,48.9804867)(952.04394775,48.86049123)
\curveto(952.06394767,48.75048693)(952.08894765,48.63548705)(952.11894775,48.51549123)
\curveto(952.12894761,48.46548722)(952.12894761,48.41048727)(952.11894775,48.35049123)
\curveto(952.10894763,48.30048738)(952.11394762,48.25048743)(952.13394775,48.20049123)
\curveto(952.15394758,48.10048758)(952.15394758,48.01048767)(952.13394775,47.93049123)
\lineto(952.13394775,47.78049123)
\curveto(952.11394762,47.73048795)(952.10394763,47.67048801)(952.10394775,47.60049123)
\curveto(952.10394763,47.54048814)(952.09894764,47.4854882)(952.08894775,47.43549123)
\curveto(952.06894767,47.39548829)(952.05894768,47.35548833)(952.05894775,47.31549123)
\curveto(952.06894767,47.2854884)(952.06394767,47.24548844)(952.04394775,47.19549123)
\lineto(951.98394775,46.95549123)
\curveto(951.96394777,46.8854888)(951.9339478,46.81048887)(951.89394775,46.73049123)
\curveto(951.78394795,46.47048921)(951.6389481,46.25048943)(951.45894775,46.07049123)
\curveto(951.26894847,45.90048978)(951.04394869,45.76048992)(950.78394775,45.65049123)
\curveto(950.69394904,45.61049007)(950.60394913,45.5804901)(950.51394775,45.56049123)
\lineto(950.21394775,45.50049123)
\curveto(950.15394958,45.4804902)(950.09894964,45.47049021)(950.04894775,45.47049123)
\curveto(949.98894975,45.4804902)(949.92394981,45.47549021)(949.85394775,45.45549123)
\curveto(949.8339499,45.44549024)(949.80894993,45.44049024)(949.77894775,45.44049123)
\curveto(949.73895,45.44049024)(949.70395003,45.43549025)(949.67394775,45.42549123)
\lineto(949.52394775,45.42549123)
\curveto(949.48395025,45.41549027)(949.4389503,45.41049027)(949.38894775,45.41049123)
\curveto(949.32895041,45.42049026)(949.27395046,45.42549026)(949.22394775,45.42549123)
\lineto(948.62394775,45.42549123)
\lineto(945.86394775,45.42549123)
\lineto(944.90394775,45.42549123)
\lineto(944.63394775,45.42549123)
\curveto(944.54395519,45.42549026)(944.46895527,45.44549024)(944.40894775,45.48549123)
\curveto(944.3389554,45.52549016)(944.28895545,45.60049008)(944.25894775,45.71049123)
\curveto(944.24895549,45.73048995)(944.24895549,45.75048993)(944.25894775,45.77049123)
\curveto(944.25895548,45.79048989)(944.25395548,45.81048987)(944.24394775,45.83049123)
}
}
{
\newrgbcolor{curcolor}{0 0 0}
\pscustom[linestyle=none,fillstyle=solid,fillcolor=curcolor]
{
\newpath
\moveto(941.28894775,54.28510061)
\curveto(941.28895845,54.41509899)(941.28895845,54.55009886)(941.28894775,54.69010061)
\curveto(941.28895845,54.84009857)(941.32395841,54.95009846)(941.39394775,55.02010061)
\curveto(941.46395827,55.07009834)(941.55895818,55.09509831)(941.67894775,55.09510061)
\curveto(941.78895795,55.1050983)(941.90395783,55.1100983)(942.02394775,55.11010061)
\lineto(943.35894775,55.11010061)
\lineto(949.43394775,55.11010061)
\lineto(951.11394775,55.11010061)
\lineto(951.50394775,55.11010061)
\curveto(951.64394809,55.1100983)(951.75394798,55.08509832)(951.83394775,55.03510061)
\curveto(951.88394785,55.0050984)(951.91394782,54.96009845)(951.92394775,54.90010061)
\curveto(951.9339478,54.85009856)(951.94894779,54.78509862)(951.96894775,54.70510061)
\lineto(951.96894775,54.49510061)
\lineto(951.96894775,54.18010061)
\curveto(951.95894778,54.08009933)(951.92394781,54.0050994)(951.86394775,53.95510061)
\curveto(951.78394795,53.9050995)(951.68394805,53.87509953)(951.56394775,53.86510061)
\lineto(951.18894775,53.86510061)
\lineto(949.80894775,53.86510061)
\lineto(943.56894775,53.86510061)
\lineto(942.09894775,53.86510061)
\curveto(941.98895775,53.86509954)(941.87395786,53.86009955)(941.75394775,53.85010061)
\curveto(941.62395811,53.85009956)(941.52395821,53.87509953)(941.45394775,53.92510061)
\curveto(941.39395834,53.96509944)(941.34395839,54.04009937)(941.30394775,54.15010061)
\curveto(941.29395844,54.17009924)(941.29395844,54.19009922)(941.30394775,54.21010061)
\curveto(941.30395843,54.24009917)(941.29895844,54.26509914)(941.28894775,54.28510061)
}
}
{
\newrgbcolor{curcolor}{0 0 0}
\pscustom[linestyle=none,fillstyle=solid,fillcolor=curcolor]
{
}
}
{
\newrgbcolor{curcolor}{0 0 0}
\pscustom[linestyle=none,fillstyle=solid,fillcolor=curcolor]
{
\newpath
\moveto(941.36394775,64.24510061)
\curveto(941.35395838,64.93509597)(941.47395826,65.53509537)(941.72394775,66.04510061)
\curveto(941.97395776,66.56509434)(942.30895743,66.96009395)(942.72894775,67.23010061)
\curveto(942.80895693,67.28009363)(942.89895684,67.32509358)(942.99894775,67.36510061)
\curveto(943.08895665,67.4050935)(943.18395655,67.45009346)(943.28394775,67.50010061)
\curveto(943.38395635,67.54009337)(943.48395625,67.57009334)(943.58394775,67.59010061)
\curveto(943.68395605,67.6100933)(943.78895595,67.63009328)(943.89894775,67.65010061)
\curveto(943.94895579,67.67009324)(943.99395574,67.67509323)(944.03394775,67.66510061)
\curveto(944.07395566,67.65509325)(944.11895562,67.66009325)(944.16894775,67.68010061)
\curveto(944.21895552,67.69009322)(944.30395543,67.69509321)(944.42394775,67.69510061)
\curveto(944.5339552,67.69509321)(944.61895512,67.69009322)(944.67894775,67.68010061)
\curveto(944.738955,67.66009325)(944.79895494,67.65009326)(944.85894775,67.65010061)
\curveto(944.91895482,67.66009325)(944.97895476,67.65509325)(945.03894775,67.63510061)
\curveto(945.17895456,67.59509331)(945.31395442,67.56009335)(945.44394775,67.53010061)
\curveto(945.57395416,67.50009341)(945.69895404,67.46009345)(945.81894775,67.41010061)
\curveto(945.95895378,67.35009356)(946.08395365,67.28009363)(946.19394775,67.20010061)
\curveto(946.30395343,67.13009378)(946.41395332,67.05509385)(946.52394775,66.97510061)
\lineto(946.58394775,66.91510061)
\curveto(946.60395313,66.905094)(946.62395311,66.89009402)(946.64394775,66.87010061)
\curveto(946.80395293,66.75009416)(946.94895279,66.61509429)(947.07894775,66.46510061)
\curveto(947.20895253,66.31509459)(947.3339524,66.15509475)(947.45394775,65.98510061)
\curveto(947.67395206,65.67509523)(947.87895186,65.38009553)(948.06894775,65.10010061)
\curveto(948.20895153,64.87009604)(948.34395139,64.64009627)(948.47394775,64.41010061)
\curveto(948.60395113,64.19009672)(948.738951,63.97009694)(948.87894775,63.75010061)
\curveto(949.04895069,63.50009741)(949.22895051,63.26009765)(949.41894775,63.03010061)
\curveto(949.60895013,62.8100981)(949.8339499,62.62009829)(950.09394775,62.46010061)
\curveto(950.15394958,62.42009849)(950.21394952,62.38509852)(950.27394775,62.35510061)
\curveto(950.32394941,62.32509858)(950.38894935,62.29509861)(950.46894775,62.26510061)
\curveto(950.5389492,62.24509866)(950.59894914,62.24009867)(950.64894775,62.25010061)
\curveto(950.71894902,62.27009864)(950.77394896,62.3050986)(950.81394775,62.35510061)
\curveto(950.84394889,62.4050985)(950.86394887,62.46509844)(950.87394775,62.53510061)
\lineto(950.87394775,62.77510061)
\lineto(950.87394775,63.52510061)
\lineto(950.87394775,66.33010061)
\lineto(950.87394775,66.99010061)
\curveto(950.87394886,67.08009383)(950.87894886,67.16509374)(950.88894775,67.24510061)
\curveto(950.88894885,67.32509358)(950.90894883,67.39009352)(950.94894775,67.44010061)
\curveto(950.98894875,67.49009342)(951.06394867,67.53009338)(951.17394775,67.56010061)
\curveto(951.27394846,67.60009331)(951.37394836,67.6100933)(951.47394775,67.59010061)
\lineto(951.60894775,67.59010061)
\curveto(951.67894806,67.57009334)(951.738948,67.55009336)(951.78894775,67.53010061)
\curveto(951.8389479,67.5100934)(951.87894786,67.47509343)(951.90894775,67.42510061)
\curveto(951.94894779,67.37509353)(951.96894777,67.3050936)(951.96894775,67.21510061)
\lineto(951.96894775,66.94510061)
\lineto(951.96894775,66.04510061)
\lineto(951.96894775,62.53510061)
\lineto(951.96894775,61.47010061)
\curveto(951.96894777,61.39009952)(951.97394776,61.30009961)(951.98394775,61.20010061)
\curveto(951.98394775,61.10009981)(951.97394776,61.01509989)(951.95394775,60.94510061)
\curveto(951.88394785,60.73510017)(951.70394803,60.67010024)(951.41394775,60.75010061)
\curveto(951.37394836,60.76010015)(951.3389484,60.76010015)(951.30894775,60.75010061)
\curveto(951.26894847,60.75010016)(951.22394851,60.76010015)(951.17394775,60.78010061)
\curveto(951.09394864,60.80010011)(951.00894873,60.82010009)(950.91894775,60.84010061)
\curveto(950.82894891,60.86010005)(950.74394899,60.88510002)(950.66394775,60.91510061)
\curveto(950.17394956,61.07509983)(949.75894998,61.27509963)(949.41894775,61.51510061)
\curveto(949.16895057,61.69509921)(948.94395079,61.90009901)(948.74394775,62.13010061)
\curveto(948.5339512,62.36009855)(948.3389514,62.60009831)(948.15894775,62.85010061)
\curveto(947.97895176,63.1100978)(947.80895193,63.37509753)(947.64894775,63.64510061)
\curveto(947.47895226,63.92509698)(947.30395243,64.19509671)(947.12394775,64.45510061)
\curveto(947.04395269,64.56509634)(946.96895277,64.67009624)(946.89894775,64.77010061)
\curveto(946.82895291,64.88009603)(946.75395298,64.99009592)(946.67394775,65.10010061)
\curveto(946.64395309,65.14009577)(946.61395312,65.17509573)(946.58394775,65.20510061)
\curveto(946.54395319,65.24509566)(946.51395322,65.28509562)(946.49394775,65.32510061)
\curveto(946.38395335,65.46509544)(946.25895348,65.59009532)(946.11894775,65.70010061)
\curveto(946.08895365,65.72009519)(946.06395367,65.74509516)(946.04394775,65.77510061)
\curveto(946.01395372,65.8050951)(945.98395375,65.83009508)(945.95394775,65.85010061)
\curveto(945.85395388,65.93009498)(945.75395398,65.99509491)(945.65394775,66.04510061)
\curveto(945.55395418,66.1050948)(945.44395429,66.16009475)(945.32394775,66.21010061)
\curveto(945.25395448,66.24009467)(945.17895456,66.26009465)(945.09894775,66.27010061)
\lineto(944.85894775,66.33010061)
\lineto(944.76894775,66.33010061)
\curveto(944.738955,66.34009457)(944.70895503,66.34509456)(944.67894775,66.34510061)
\curveto(944.60895513,66.36509454)(944.51395522,66.37009454)(944.39394775,66.36010061)
\curveto(944.26395547,66.36009455)(944.16395557,66.35009456)(944.09394775,66.33010061)
\curveto(944.01395572,66.3100946)(943.9389558,66.29009462)(943.86894775,66.27010061)
\curveto(943.78895595,66.26009465)(943.70895603,66.24009467)(943.62894775,66.21010061)
\curveto(943.38895635,66.10009481)(943.18895655,65.95009496)(943.02894775,65.76010061)
\curveto(942.85895688,65.58009533)(942.71895702,65.36009555)(942.60894775,65.10010061)
\curveto(942.58895715,65.03009588)(942.57395716,64.96009595)(942.56394775,64.89010061)
\curveto(942.54395719,64.82009609)(942.52395721,64.74509616)(942.50394775,64.66510061)
\curveto(942.48395725,64.58509632)(942.47395726,64.47509643)(942.47394775,64.33510061)
\curveto(942.47395726,64.2050967)(942.48395725,64.10009681)(942.50394775,64.02010061)
\curveto(942.51395722,63.96009695)(942.51895722,63.905097)(942.51894775,63.85510061)
\curveto(942.51895722,63.8050971)(942.52895721,63.75509715)(942.54894775,63.70510061)
\curveto(942.58895715,63.6050973)(942.62895711,63.5100974)(942.66894775,63.42010061)
\curveto(942.70895703,63.34009757)(942.75395698,63.26009765)(942.80394775,63.18010061)
\curveto(942.82395691,63.15009776)(942.84895689,63.12009779)(942.87894775,63.09010061)
\curveto(942.90895683,63.07009784)(942.9339568,63.04509786)(942.95394775,63.01510061)
\lineto(943.02894775,62.94010061)
\curveto(943.04895669,62.910098)(943.06895667,62.88509802)(943.08894775,62.86510061)
\lineto(943.29894775,62.71510061)
\curveto(943.35895638,62.67509823)(943.42395631,62.63009828)(943.49394775,62.58010061)
\curveto(943.58395615,62.52009839)(943.68895605,62.47009844)(943.80894775,62.43010061)
\curveto(943.91895582,62.40009851)(944.02895571,62.36509854)(944.13894775,62.32510061)
\curveto(944.24895549,62.28509862)(944.39395534,62.26009865)(944.57394775,62.25010061)
\curveto(944.74395499,62.24009867)(944.86895487,62.2100987)(944.94894775,62.16010061)
\curveto(945.02895471,62.1100988)(945.07395466,62.03509887)(945.08394775,61.93510061)
\curveto(945.09395464,61.83509907)(945.09895464,61.72509918)(945.09894775,61.60510061)
\curveto(945.09895464,61.56509934)(945.10395463,61.52509938)(945.11394775,61.48510061)
\curveto(945.11395462,61.44509946)(945.10895463,61.4100995)(945.09894775,61.38010061)
\curveto(945.07895466,61.33009958)(945.06895467,61.28009963)(945.06894775,61.23010061)
\curveto(945.06895467,61.19009972)(945.05895468,61.15009976)(945.03894775,61.11010061)
\curveto(944.97895476,61.02009989)(944.84395489,60.97509993)(944.63394775,60.97510061)
\lineto(944.51394775,60.97510061)
\curveto(944.45395528,60.98509992)(944.39395534,60.99009992)(944.33394775,60.99010061)
\curveto(944.26395547,61.00009991)(944.19895554,61.0100999)(944.13894775,61.02010061)
\curveto(944.02895571,61.04009987)(943.92895581,61.06009985)(943.83894775,61.08010061)
\curveto(943.738956,61.10009981)(943.64395609,61.13009978)(943.55394775,61.17010061)
\curveto(943.48395625,61.19009972)(943.42395631,61.2100997)(943.37394775,61.23010061)
\lineto(943.19394775,61.29010061)
\curveto(942.9339568,61.4100995)(942.68895705,61.56509934)(942.45894775,61.75510061)
\curveto(942.22895751,61.95509895)(942.04395769,62.17009874)(941.90394775,62.40010061)
\curveto(941.82395791,62.5100984)(941.75895798,62.62509828)(941.70894775,62.74510061)
\lineto(941.55894775,63.13510061)
\curveto(941.50895823,63.24509766)(941.47895826,63.36009755)(941.46894775,63.48010061)
\curveto(941.44895829,63.60009731)(941.42395831,63.72509718)(941.39394775,63.85510061)
\curveto(941.39395834,63.92509698)(941.39395834,63.99009692)(941.39394775,64.05010061)
\curveto(941.38395835,64.1100968)(941.37395836,64.17509673)(941.36394775,64.24510061)
}
}
{
\newrgbcolor{curcolor}{0 0 0}
\pscustom[linestyle=none,fillstyle=solid,fillcolor=curcolor]
{
\newpath
\moveto(948.89394775,76.35970998)
\curveto(948.9339508,76.36970226)(948.98395075,76.36970226)(949.04394775,76.35970998)
\curveto(949.10395063,76.35970227)(949.15395058,76.35470228)(949.19394775,76.34470998)
\curveto(949.2339505,76.34470229)(949.27395046,76.33970229)(949.31394775,76.32970998)
\lineto(949.41894775,76.32970998)
\curveto(949.49895024,76.30970232)(949.57895016,76.29470234)(949.65894775,76.28470998)
\curveto(949.73895,76.27470236)(949.81394992,76.25470238)(949.88394775,76.22470998)
\curveto(949.96394977,76.20470243)(950.0389497,76.18470245)(950.10894775,76.16470998)
\curveto(950.17894956,76.14470249)(950.25394948,76.11470252)(950.33394775,76.07470998)
\curveto(950.75394898,75.89470274)(951.09394864,75.63970299)(951.35394775,75.30970998)
\curveto(951.61394812,74.97970365)(951.81894792,74.58970404)(951.96894775,74.13970998)
\curveto(952.00894773,74.01970461)(952.0339477,73.89470474)(952.04394775,73.76470998)
\curveto(952.06394767,73.64470499)(952.08894765,73.51970511)(952.11894775,73.38970998)
\curveto(952.12894761,73.3297053)(952.1339476,73.26470537)(952.13394775,73.19470998)
\curveto(952.1339476,73.1347055)(952.1389476,73.06970556)(952.14894775,72.99970998)
\lineto(952.14894775,72.87970998)
\lineto(952.14894775,72.68470998)
\curveto(952.15894758,72.62470601)(952.15394758,72.56970606)(952.13394775,72.51970998)
\curveto(952.11394762,72.44970618)(952.10894763,72.38470625)(952.11894775,72.32470998)
\curveto(952.12894761,72.26470637)(952.12394761,72.20470643)(952.10394775,72.14470998)
\curveto(952.09394764,72.09470654)(952.08894765,72.04970658)(952.08894775,72.00970998)
\curveto(952.08894765,71.96970666)(952.07894766,71.92470671)(952.05894775,71.87470998)
\curveto(952.0389477,71.79470684)(952.01894772,71.71970691)(951.99894775,71.64970998)
\curveto(951.98894775,71.57970705)(951.97394776,71.50970712)(951.95394775,71.43970998)
\curveto(951.78394795,70.95970767)(951.57394816,70.55970807)(951.32394775,70.23970998)
\curveto(951.06394867,69.9297087)(950.70894903,69.67970895)(950.25894775,69.48970998)
\curveto(950.19894954,69.45970917)(950.1389496,69.4347092)(950.07894775,69.41470998)
\curveto(950.00894973,69.40470923)(949.9339498,69.38970924)(949.85394775,69.36970998)
\curveto(949.79394994,69.34970928)(949.72895001,69.3347093)(949.65894775,69.32470998)
\curveto(949.58895015,69.31470932)(949.51895022,69.29970933)(949.44894775,69.27970998)
\curveto(949.39895034,69.26970936)(949.35895038,69.26470937)(949.32894775,69.26470998)
\lineto(949.20894775,69.26470998)
\curveto(949.16895057,69.25470938)(949.11895062,69.24470939)(949.05894775,69.23470998)
\curveto(948.99895074,69.2347094)(948.94895079,69.23970939)(948.90894775,69.24970998)
\lineto(948.77394775,69.24970998)
\curveto(948.72395101,69.25970937)(948.67395106,69.26470937)(948.62394775,69.26470998)
\curveto(948.52395121,69.28470935)(948.42895131,69.29970933)(948.33894775,69.30970998)
\curveto(948.2389515,69.31970931)(948.14395159,69.33970929)(948.05394775,69.36970998)
\curveto(947.90395183,69.41970921)(947.76395197,69.47470916)(947.63394775,69.53470998)
\curveto(947.50395223,69.59470904)(947.38395235,69.66470897)(947.27394775,69.74470998)
\curveto(947.22395251,69.77470886)(947.18395255,69.80470883)(947.15394775,69.83470998)
\curveto(947.12395261,69.87470876)(947.08895265,69.90970872)(947.04894775,69.93970998)
\curveto(946.96895277,69.99970863)(946.89895284,70.06970856)(946.83894775,70.14970998)
\curveto(946.78895295,70.20970842)(946.74395299,70.26970836)(946.70394775,70.32970998)
\lineto(946.55394775,70.53970998)
\curveto(946.51395322,70.58970804)(946.47895326,70.63970799)(946.44894775,70.68970998)
\curveto(946.40895333,70.73970789)(946.35395338,70.77470786)(946.28394775,70.79470998)
\curveto(946.25395348,70.79470784)(946.22895351,70.78470785)(946.20894775,70.76470998)
\curveto(946.17895356,70.75470788)(946.15395358,70.74470789)(946.13394775,70.73470998)
\curveto(946.08395365,70.69470794)(946.0389537,70.64470799)(945.99894775,70.58470998)
\curveto(945.94895379,70.5347081)(945.90395383,70.48470815)(945.86394775,70.43470998)
\curveto(945.8339539,70.39470824)(945.77895396,70.34470829)(945.69894775,70.28470998)
\curveto(945.66895407,70.26470837)(945.64395409,70.2347084)(945.62394775,70.19470998)
\curveto(945.59395414,70.16470847)(945.55895418,70.13970849)(945.51894775,70.11970998)
\curveto(945.30895443,69.94970868)(945.06395467,69.81970881)(944.78394775,69.72970998)
\curveto(944.70395503,69.70970892)(944.62395511,69.69470894)(944.54394775,69.68470998)
\curveto(944.46395527,69.67470896)(944.38395535,69.65970897)(944.30394775,69.63970998)
\curveto(944.25395548,69.61970901)(944.18895555,69.60970902)(944.10894775,69.60970998)
\curveto(944.01895572,69.60970902)(943.94895579,69.61970901)(943.89894775,69.63970998)
\curveto(943.79895594,69.63970899)(943.72895601,69.64470899)(943.68894775,69.65470998)
\curveto(943.60895613,69.67470896)(943.5389562,69.68970894)(943.47894775,69.69970998)
\curveto(943.40895633,69.70970892)(943.3389564,69.72470891)(943.26894775,69.74470998)
\curveto(942.8389569,69.89470874)(942.49395724,70.10970852)(942.23394775,70.38970998)
\curveto(941.97395776,70.67970795)(941.75895798,71.0297076)(941.58894775,71.43970998)
\curveto(941.5389582,71.54970708)(941.50895823,71.66470697)(941.49894775,71.78470998)
\curveto(941.47895826,71.91470672)(941.44895829,72.04470659)(941.40894775,72.17470998)
\curveto(941.40895833,72.25470638)(941.40895833,72.32470631)(941.40894775,72.38470998)
\curveto(941.39895834,72.45470618)(941.38895835,72.5297061)(941.37894775,72.60970998)
\curveto(941.35895838,73.39970523)(941.48895825,74.05470458)(941.76894775,74.57470998)
\curveto(942.04895769,75.10470353)(942.45895728,75.48470315)(942.99894775,75.71470998)
\curveto(943.22895651,75.82470281)(943.51395622,75.89470274)(943.85394775,75.92470998)
\curveto(944.18395555,75.96470267)(944.48895525,75.9347027)(944.76894775,75.83470998)
\curveto(944.89895484,75.79470284)(945.01895472,75.74470289)(945.12894775,75.68470998)
\curveto(945.2389545,75.634703)(945.34395439,75.57470306)(945.44394775,75.50470998)
\curveto(945.48395425,75.48470315)(945.51895422,75.45470318)(945.54894775,75.41470998)
\lineto(945.63894775,75.32470998)
\curveto(945.72895401,75.27470336)(945.79395394,75.21470342)(945.83394775,75.14470998)
\curveto(945.88395385,75.09470354)(945.9339538,75.03970359)(945.98394775,74.97970998)
\curveto(946.02395371,74.9297037)(946.06895367,74.88470375)(946.11894775,74.84470998)
\curveto(946.1389536,74.82470381)(946.16395357,74.80470383)(946.19394775,74.78470998)
\curveto(946.21395352,74.77470386)(946.2389535,74.77470386)(946.26894775,74.78470998)
\curveto(946.31895342,74.79470384)(946.36895337,74.82470381)(946.41894775,74.87470998)
\curveto(946.45895328,74.92470371)(946.49895324,74.97970365)(946.53894775,75.03970998)
\lineto(946.65894775,75.21970998)
\curveto(946.68895305,75.27970335)(946.71895302,75.3297033)(946.74894775,75.36970998)
\curveto(946.98895275,75.69970293)(947.29895244,75.94970268)(947.67894775,76.11970998)
\curveto(947.75895198,76.15970247)(947.84395189,76.18970244)(947.93394775,76.20970998)
\curveto(948.02395171,76.23970239)(948.11395162,76.26470237)(948.20394775,76.28470998)
\curveto(948.25395148,76.29470234)(948.30895143,76.30470233)(948.36894775,76.31470998)
\lineto(948.51894775,76.34470998)
\curveto(948.57895116,76.35470228)(948.64395109,76.35470228)(948.71394775,76.34470998)
\curveto(948.77395096,76.3347023)(948.8339509,76.33970229)(948.89394775,76.35970998)
\moveto(943.85394775,70.97470998)
\curveto(943.96395577,70.94470769)(944.10395563,70.93970769)(944.27394775,70.95970998)
\curveto(944.4339553,70.97970765)(944.55895518,71.00470763)(944.64894775,71.03470998)
\curveto(944.96895477,71.14470749)(945.21395452,71.29470734)(945.38394775,71.48470998)
\curveto(945.54395419,71.67470696)(945.67395406,71.93970669)(945.77394775,72.27970998)
\curveto(945.80395393,72.40970622)(945.82895391,72.57470606)(945.84894775,72.77470998)
\curveto(945.85895388,72.97470566)(945.84395389,73.14470549)(945.80394775,73.28470998)
\curveto(945.72395401,73.57470506)(945.61395412,73.81470482)(945.47394775,74.00470998)
\curveto(945.32395441,74.20470443)(945.12395461,74.35970427)(944.87394775,74.46970998)
\curveto(944.82395491,74.48970414)(944.77895496,74.49970413)(944.73894775,74.49970998)
\curveto(944.69895504,74.50970412)(944.65395508,74.52470411)(944.60394775,74.54470998)
\curveto(944.49395524,74.57470406)(944.35395538,74.59470404)(944.18394775,74.60470998)
\curveto(944.01395572,74.61470402)(943.86895587,74.60470403)(943.74894775,74.57470998)
\curveto(943.65895608,74.55470408)(943.57395616,74.5297041)(943.49394775,74.49970998)
\curveto(943.41395632,74.47970415)(943.3339564,74.44470419)(943.25394775,74.39470998)
\curveto(942.98395675,74.22470441)(942.78895695,73.99970463)(942.66894775,73.71970998)
\curveto(942.54895719,73.44970518)(942.48895725,73.08970554)(942.48894775,72.63970998)
\curveto(942.50895723,72.61970601)(942.51395722,72.58970604)(942.50394775,72.54970998)
\curveto(942.49395724,72.50970612)(942.49395724,72.47470616)(942.50394775,72.44470998)
\curveto(942.52395721,72.39470624)(942.5389572,72.33970629)(942.54894775,72.27970998)
\curveto(942.54895719,72.2297064)(942.55895718,72.17970645)(942.57894775,72.12970998)
\curveto(942.66895707,71.88970674)(942.78395695,71.67970695)(942.92394775,71.49970998)
\curveto(943.05395668,71.31970731)(943.2339565,71.17970745)(943.46394775,71.07970998)
\curveto(943.52395621,71.05970757)(943.58895615,71.03970759)(943.65894775,71.01970998)
\curveto(943.71895602,71.00970762)(943.78395595,70.99470764)(943.85394775,70.97470998)
\moveto(949.38894775,74.99470998)
\curveto(949.19895054,75.04470359)(948.99395074,75.04970358)(948.77394775,75.00970998)
\curveto(948.55395118,74.97970365)(948.37395136,74.9347037)(948.23394775,74.87470998)
\curveto(947.86395187,74.70470393)(947.55895218,74.44470419)(947.31894775,74.09470998)
\curveto(947.07895266,73.75470488)(946.95895278,73.31970531)(946.95894775,72.78970998)
\curveto(946.97895276,72.75970587)(946.98395275,72.71970591)(946.97394775,72.66970998)
\curveto(946.95395278,72.61970601)(946.94895279,72.57970605)(946.95894775,72.54970998)
\lineto(947.01894775,72.27970998)
\curveto(947.02895271,72.19970643)(947.04395269,72.11970651)(947.06394775,72.03970998)
\curveto(947.17395256,71.73970689)(947.31895242,71.47470716)(947.49894775,71.24470998)
\curveto(947.67895206,71.02470761)(947.90895183,70.85470778)(948.18894775,70.73470998)
\curveto(948.26895147,70.70470793)(948.34895139,70.67970795)(948.42894775,70.65970998)
\curveto(948.50895123,70.63970799)(948.59395114,70.61970801)(948.68394775,70.59970998)
\curveto(948.80395093,70.56970806)(948.95395078,70.55970807)(949.13394775,70.56970998)
\curveto(949.31395042,70.58970804)(949.45395028,70.61470802)(949.55394775,70.64470998)
\curveto(949.60395013,70.66470797)(949.64895009,70.67470796)(949.68894775,70.67470998)
\curveto(949.71895002,70.68470795)(949.75894998,70.69970793)(949.80894775,70.71970998)
\curveto(950.02894971,70.81970781)(950.22894951,70.94970768)(950.40894775,71.10970998)
\curveto(950.58894915,71.27970735)(950.72394901,71.47470716)(950.81394775,71.69470998)
\curveto(950.85394888,71.76470687)(950.88894885,71.85970677)(950.91894775,71.97970998)
\curveto(951.00894873,72.19970643)(951.05394868,72.45470618)(951.05394775,72.74470998)
\lineto(951.05394775,73.02970998)
\curveto(951.0339487,73.1297055)(951.01894872,73.22470541)(951.00894775,73.31470998)
\curveto(950.99894874,73.40470523)(950.97894876,73.49470514)(950.94894775,73.58470998)
\curveto(950.86894887,73.84470479)(950.738949,74.08470455)(950.55894775,74.30470998)
\curveto(950.36894937,74.5347041)(950.15394958,74.70470393)(949.91394775,74.81470998)
\curveto(949.8339499,74.85470378)(949.75394998,74.88470375)(949.67394775,74.90470998)
\curveto(949.58395015,74.9347037)(949.48895025,74.96470367)(949.38894775,74.99470998)
}
}
{
\newrgbcolor{curcolor}{0 0 0}
\pscustom[linestyle=none,fillstyle=solid,fillcolor=curcolor]
{
\newpath
\moveto(950.33394775,78.63431936)
\lineto(950.33394775,79.26431936)
\lineto(950.33394775,79.45931936)
\curveto(950.3339494,79.52931683)(950.34394939,79.58931677)(950.36394775,79.63931936)
\curveto(950.40394933,79.70931665)(950.44394929,79.7593166)(950.48394775,79.78931936)
\curveto(950.5339492,79.82931653)(950.59894914,79.84931651)(950.67894775,79.84931936)
\curveto(950.75894898,79.8593165)(950.84394889,79.86431649)(950.93394775,79.86431936)
\lineto(951.65394775,79.86431936)
\curveto(952.1339476,79.86431649)(952.54394719,79.80431655)(952.88394775,79.68431936)
\curveto(953.22394651,79.56431679)(953.49894624,79.36931699)(953.70894775,79.09931936)
\curveto(953.75894598,79.02931733)(953.80394593,78.9593174)(953.84394775,78.88931936)
\curveto(953.89394584,78.82931753)(953.9389458,78.7543176)(953.97894775,78.66431936)
\curveto(953.98894575,78.64431771)(953.99894574,78.61431774)(954.00894775,78.57431936)
\curveto(954.02894571,78.53431782)(954.0339457,78.48931787)(954.02394775,78.43931936)
\curveto(953.99394574,78.34931801)(953.91894582,78.29431806)(953.79894775,78.27431936)
\curveto(953.68894605,78.2543181)(953.59394614,78.26931809)(953.51394775,78.31931936)
\curveto(953.44394629,78.34931801)(953.37894636,78.39431796)(953.31894775,78.45431936)
\curveto(953.26894647,78.52431783)(953.21894652,78.58931777)(953.16894775,78.64931936)
\curveto(953.11894662,78.71931764)(953.04394669,78.77931758)(952.94394775,78.82931936)
\curveto(952.85394688,78.88931747)(952.76394697,78.93931742)(952.67394775,78.97931936)
\curveto(952.64394709,78.99931736)(952.58394715,79.02431733)(952.49394775,79.05431936)
\curveto(952.41394732,79.08431727)(952.34394739,79.08931727)(952.28394775,79.06931936)
\curveto(952.14394759,79.03931732)(952.05394768,78.97931738)(952.01394775,78.88931936)
\curveto(951.98394775,78.80931755)(951.96894777,78.71931764)(951.96894775,78.61931936)
\curveto(951.96894777,78.51931784)(951.94394779,78.43431792)(951.89394775,78.36431936)
\curveto(951.82394791,78.27431808)(951.68394805,78.22931813)(951.47394775,78.22931936)
\lineto(950.91894775,78.22931936)
\lineto(950.69394775,78.22931936)
\curveto(950.61394912,78.23931812)(950.54894919,78.2593181)(950.49894775,78.28931936)
\curveto(950.41894932,78.34931801)(950.37394936,78.41931794)(950.36394775,78.49931936)
\curveto(950.35394938,78.51931784)(950.34894939,78.53931782)(950.34894775,78.55931936)
\curveto(950.34894939,78.58931777)(950.34394939,78.61431774)(950.33394775,78.63431936)
}
}
{
\newrgbcolor{curcolor}{0 0 0}
\pscustom[linestyle=none,fillstyle=solid,fillcolor=curcolor]
{
}
}
{
\newrgbcolor{curcolor}{0 0 0}
\pscustom[linestyle=none,fillstyle=solid,fillcolor=curcolor]
{
\newpath
\moveto(941.36394775,89.26463186)
\curveto(941.35395838,89.95462722)(941.47395826,90.55462662)(941.72394775,91.06463186)
\curveto(941.97395776,91.58462559)(942.30895743,91.9796252)(942.72894775,92.24963186)
\curveto(942.80895693,92.29962488)(942.89895684,92.34462483)(942.99894775,92.38463186)
\curveto(943.08895665,92.42462475)(943.18395655,92.46962471)(943.28394775,92.51963186)
\curveto(943.38395635,92.55962462)(943.48395625,92.58962459)(943.58394775,92.60963186)
\curveto(943.68395605,92.62962455)(943.78895595,92.64962453)(943.89894775,92.66963186)
\curveto(943.94895579,92.68962449)(943.99395574,92.69462448)(944.03394775,92.68463186)
\curveto(944.07395566,92.6746245)(944.11895562,92.6796245)(944.16894775,92.69963186)
\curveto(944.21895552,92.70962447)(944.30395543,92.71462446)(944.42394775,92.71463186)
\curveto(944.5339552,92.71462446)(944.61895512,92.70962447)(944.67894775,92.69963186)
\curveto(944.738955,92.6796245)(944.79895494,92.66962451)(944.85894775,92.66963186)
\curveto(944.91895482,92.6796245)(944.97895476,92.6746245)(945.03894775,92.65463186)
\curveto(945.17895456,92.61462456)(945.31395442,92.5796246)(945.44394775,92.54963186)
\curveto(945.57395416,92.51962466)(945.69895404,92.4796247)(945.81894775,92.42963186)
\curveto(945.95895378,92.36962481)(946.08395365,92.29962488)(946.19394775,92.21963186)
\curveto(946.30395343,92.14962503)(946.41395332,92.0746251)(946.52394775,91.99463186)
\lineto(946.58394775,91.93463186)
\curveto(946.60395313,91.92462525)(946.62395311,91.90962527)(946.64394775,91.88963186)
\curveto(946.80395293,91.76962541)(946.94895279,91.63462554)(947.07894775,91.48463186)
\curveto(947.20895253,91.33462584)(947.3339524,91.174626)(947.45394775,91.00463186)
\curveto(947.67395206,90.69462648)(947.87895186,90.39962678)(948.06894775,90.11963186)
\curveto(948.20895153,89.88962729)(948.34395139,89.65962752)(948.47394775,89.42963186)
\curveto(948.60395113,89.20962797)(948.738951,88.98962819)(948.87894775,88.76963186)
\curveto(949.04895069,88.51962866)(949.22895051,88.2796289)(949.41894775,88.04963186)
\curveto(949.60895013,87.82962935)(949.8339499,87.63962954)(950.09394775,87.47963186)
\curveto(950.15394958,87.43962974)(950.21394952,87.40462977)(950.27394775,87.37463186)
\curveto(950.32394941,87.34462983)(950.38894935,87.31462986)(950.46894775,87.28463186)
\curveto(950.5389492,87.26462991)(950.59894914,87.25962992)(950.64894775,87.26963186)
\curveto(950.71894902,87.28962989)(950.77394896,87.32462985)(950.81394775,87.37463186)
\curveto(950.84394889,87.42462975)(950.86394887,87.48462969)(950.87394775,87.55463186)
\lineto(950.87394775,87.79463186)
\lineto(950.87394775,88.54463186)
\lineto(950.87394775,91.34963186)
\lineto(950.87394775,92.00963186)
\curveto(950.87394886,92.09962508)(950.87894886,92.18462499)(950.88894775,92.26463186)
\curveto(950.88894885,92.34462483)(950.90894883,92.40962477)(950.94894775,92.45963186)
\curveto(950.98894875,92.50962467)(951.06394867,92.54962463)(951.17394775,92.57963186)
\curveto(951.27394846,92.61962456)(951.37394836,92.62962455)(951.47394775,92.60963186)
\lineto(951.60894775,92.60963186)
\curveto(951.67894806,92.58962459)(951.738948,92.56962461)(951.78894775,92.54963186)
\curveto(951.8389479,92.52962465)(951.87894786,92.49462468)(951.90894775,92.44463186)
\curveto(951.94894779,92.39462478)(951.96894777,92.32462485)(951.96894775,92.23463186)
\lineto(951.96894775,91.96463186)
\lineto(951.96894775,91.06463186)
\lineto(951.96894775,87.55463186)
\lineto(951.96894775,86.48963186)
\curveto(951.96894777,86.40963077)(951.97394776,86.31963086)(951.98394775,86.21963186)
\curveto(951.98394775,86.11963106)(951.97394776,86.03463114)(951.95394775,85.96463186)
\curveto(951.88394785,85.75463142)(951.70394803,85.68963149)(951.41394775,85.76963186)
\curveto(951.37394836,85.7796314)(951.3389484,85.7796314)(951.30894775,85.76963186)
\curveto(951.26894847,85.76963141)(951.22394851,85.7796314)(951.17394775,85.79963186)
\curveto(951.09394864,85.81963136)(951.00894873,85.83963134)(950.91894775,85.85963186)
\curveto(950.82894891,85.8796313)(950.74394899,85.90463127)(950.66394775,85.93463186)
\curveto(950.17394956,86.09463108)(949.75894998,86.29463088)(949.41894775,86.53463186)
\curveto(949.16895057,86.71463046)(948.94395079,86.91963026)(948.74394775,87.14963186)
\curveto(948.5339512,87.3796298)(948.3389514,87.61962956)(948.15894775,87.86963186)
\curveto(947.97895176,88.12962905)(947.80895193,88.39462878)(947.64894775,88.66463186)
\curveto(947.47895226,88.94462823)(947.30395243,89.21462796)(947.12394775,89.47463186)
\curveto(947.04395269,89.58462759)(946.96895277,89.68962749)(946.89894775,89.78963186)
\curveto(946.82895291,89.89962728)(946.75395298,90.00962717)(946.67394775,90.11963186)
\curveto(946.64395309,90.15962702)(946.61395312,90.19462698)(946.58394775,90.22463186)
\curveto(946.54395319,90.26462691)(946.51395322,90.30462687)(946.49394775,90.34463186)
\curveto(946.38395335,90.48462669)(946.25895348,90.60962657)(946.11894775,90.71963186)
\curveto(946.08895365,90.73962644)(946.06395367,90.76462641)(946.04394775,90.79463186)
\curveto(946.01395372,90.82462635)(945.98395375,90.84962633)(945.95394775,90.86963186)
\curveto(945.85395388,90.94962623)(945.75395398,91.01462616)(945.65394775,91.06463186)
\curveto(945.55395418,91.12462605)(945.44395429,91.179626)(945.32394775,91.22963186)
\curveto(945.25395448,91.25962592)(945.17895456,91.2796259)(945.09894775,91.28963186)
\lineto(944.85894775,91.34963186)
\lineto(944.76894775,91.34963186)
\curveto(944.738955,91.35962582)(944.70895503,91.36462581)(944.67894775,91.36463186)
\curveto(944.60895513,91.38462579)(944.51395522,91.38962579)(944.39394775,91.37963186)
\curveto(944.26395547,91.3796258)(944.16395557,91.36962581)(944.09394775,91.34963186)
\curveto(944.01395572,91.32962585)(943.9389558,91.30962587)(943.86894775,91.28963186)
\curveto(943.78895595,91.2796259)(943.70895603,91.25962592)(943.62894775,91.22963186)
\curveto(943.38895635,91.11962606)(943.18895655,90.96962621)(943.02894775,90.77963186)
\curveto(942.85895688,90.59962658)(942.71895702,90.3796268)(942.60894775,90.11963186)
\curveto(942.58895715,90.04962713)(942.57395716,89.9796272)(942.56394775,89.90963186)
\curveto(942.54395719,89.83962734)(942.52395721,89.76462741)(942.50394775,89.68463186)
\curveto(942.48395725,89.60462757)(942.47395726,89.49462768)(942.47394775,89.35463186)
\curveto(942.47395726,89.22462795)(942.48395725,89.11962806)(942.50394775,89.03963186)
\curveto(942.51395722,88.9796282)(942.51895722,88.92462825)(942.51894775,88.87463186)
\curveto(942.51895722,88.82462835)(942.52895721,88.7746284)(942.54894775,88.72463186)
\curveto(942.58895715,88.62462855)(942.62895711,88.52962865)(942.66894775,88.43963186)
\curveto(942.70895703,88.35962882)(942.75395698,88.2796289)(942.80394775,88.19963186)
\curveto(942.82395691,88.16962901)(942.84895689,88.13962904)(942.87894775,88.10963186)
\curveto(942.90895683,88.08962909)(942.9339568,88.06462911)(942.95394775,88.03463186)
\lineto(943.02894775,87.95963186)
\curveto(943.04895669,87.92962925)(943.06895667,87.90462927)(943.08894775,87.88463186)
\lineto(943.29894775,87.73463186)
\curveto(943.35895638,87.69462948)(943.42395631,87.64962953)(943.49394775,87.59963186)
\curveto(943.58395615,87.53962964)(943.68895605,87.48962969)(943.80894775,87.44963186)
\curveto(943.91895582,87.41962976)(944.02895571,87.38462979)(944.13894775,87.34463186)
\curveto(944.24895549,87.30462987)(944.39395534,87.2796299)(944.57394775,87.26963186)
\curveto(944.74395499,87.25962992)(944.86895487,87.22962995)(944.94894775,87.17963186)
\curveto(945.02895471,87.12963005)(945.07395466,87.05463012)(945.08394775,86.95463186)
\curveto(945.09395464,86.85463032)(945.09895464,86.74463043)(945.09894775,86.62463186)
\curveto(945.09895464,86.58463059)(945.10395463,86.54463063)(945.11394775,86.50463186)
\curveto(945.11395462,86.46463071)(945.10895463,86.42963075)(945.09894775,86.39963186)
\curveto(945.07895466,86.34963083)(945.06895467,86.29963088)(945.06894775,86.24963186)
\curveto(945.06895467,86.20963097)(945.05895468,86.16963101)(945.03894775,86.12963186)
\curveto(944.97895476,86.03963114)(944.84395489,85.99463118)(944.63394775,85.99463186)
\lineto(944.51394775,85.99463186)
\curveto(944.45395528,86.00463117)(944.39395534,86.00963117)(944.33394775,86.00963186)
\curveto(944.26395547,86.01963116)(944.19895554,86.02963115)(944.13894775,86.03963186)
\curveto(944.02895571,86.05963112)(943.92895581,86.0796311)(943.83894775,86.09963186)
\curveto(943.738956,86.11963106)(943.64395609,86.14963103)(943.55394775,86.18963186)
\curveto(943.48395625,86.20963097)(943.42395631,86.22963095)(943.37394775,86.24963186)
\lineto(943.19394775,86.30963186)
\curveto(942.9339568,86.42963075)(942.68895705,86.58463059)(942.45894775,86.77463186)
\curveto(942.22895751,86.9746302)(942.04395769,87.18962999)(941.90394775,87.41963186)
\curveto(941.82395791,87.52962965)(941.75895798,87.64462953)(941.70894775,87.76463186)
\lineto(941.55894775,88.15463186)
\curveto(941.50895823,88.26462891)(941.47895826,88.3796288)(941.46894775,88.49963186)
\curveto(941.44895829,88.61962856)(941.42395831,88.74462843)(941.39394775,88.87463186)
\curveto(941.39395834,88.94462823)(941.39395834,89.00962817)(941.39394775,89.06963186)
\curveto(941.38395835,89.12962805)(941.37395836,89.19462798)(941.36394775,89.26463186)
}
}
{
\newrgbcolor{curcolor}{0 0 0}
\pscustom[linestyle=none,fillstyle=solid,fillcolor=curcolor]
{
\newpath
\moveto(946.88394775,101.36424123)
\lineto(947.13894775,101.36424123)
\curveto(947.21895252,101.37423353)(947.29395244,101.36923353)(947.36394775,101.34924123)
\lineto(947.60394775,101.34924123)
\lineto(947.76894775,101.34924123)
\curveto(947.86895187,101.32923357)(947.97395176,101.31923358)(948.08394775,101.31924123)
\curveto(948.18395155,101.31923358)(948.28395145,101.30923359)(948.38394775,101.28924123)
\lineto(948.53394775,101.28924123)
\curveto(948.67395106,101.25923364)(948.81395092,101.23923366)(948.95394775,101.22924123)
\curveto(949.08395065,101.21923368)(949.21395052,101.19423371)(949.34394775,101.15424123)
\curveto(949.42395031,101.13423377)(949.50895023,101.11423379)(949.59894775,101.09424123)
\lineto(949.83894775,101.03424123)
\lineto(950.13894775,100.91424123)
\curveto(950.22894951,100.88423402)(950.31894942,100.84923405)(950.40894775,100.80924123)
\curveto(950.62894911,100.70923419)(950.84394889,100.57423433)(951.05394775,100.40424123)
\curveto(951.26394847,100.24423466)(951.4339483,100.06923483)(951.56394775,99.87924123)
\curveto(951.60394813,99.82923507)(951.64394809,99.76923513)(951.68394775,99.69924123)
\curveto(951.71394802,99.63923526)(951.74894799,99.57923532)(951.78894775,99.51924123)
\curveto(951.8389479,99.43923546)(951.87894786,99.34423556)(951.90894775,99.23424123)
\curveto(951.9389478,99.12423578)(951.96894777,99.01923588)(951.99894775,98.91924123)
\curveto(952.0389477,98.80923609)(952.06394767,98.6992362)(952.07394775,98.58924123)
\curveto(952.08394765,98.47923642)(952.09894764,98.36423654)(952.11894775,98.24424123)
\curveto(952.12894761,98.2042367)(952.12894761,98.15923674)(952.11894775,98.10924123)
\curveto(952.11894762,98.06923683)(952.12394761,98.02923687)(952.13394775,97.98924123)
\curveto(952.14394759,97.94923695)(952.14894759,97.89423701)(952.14894775,97.82424123)
\curveto(952.14894759,97.75423715)(952.14394759,97.7042372)(952.13394775,97.67424123)
\curveto(952.11394762,97.62423728)(952.10894763,97.57923732)(952.11894775,97.53924123)
\curveto(952.12894761,97.4992374)(952.12894761,97.46423744)(952.11894775,97.43424123)
\lineto(952.11894775,97.34424123)
\curveto(952.09894764,97.28423762)(952.08394765,97.21923768)(952.07394775,97.14924123)
\curveto(952.07394766,97.08923781)(952.06894767,97.02423788)(952.05894775,96.95424123)
\curveto(952.00894773,96.78423812)(951.95894778,96.62423828)(951.90894775,96.47424123)
\curveto(951.85894788,96.32423858)(951.79394794,96.17923872)(951.71394775,96.03924123)
\curveto(951.67394806,95.98923891)(951.64394809,95.93423897)(951.62394775,95.87424123)
\curveto(951.59394814,95.82423908)(951.55894818,95.77423913)(951.51894775,95.72424123)
\curveto(951.3389484,95.48423942)(951.11894862,95.28423962)(950.85894775,95.12424123)
\curveto(950.59894914,94.96423994)(950.31394942,94.82424008)(950.00394775,94.70424123)
\curveto(949.86394987,94.64424026)(949.72395001,94.5992403)(949.58394775,94.56924123)
\curveto(949.4339503,94.53924036)(949.27895046,94.5042404)(949.11894775,94.46424123)
\curveto(949.00895073,94.44424046)(948.89895084,94.42924047)(948.78894775,94.41924123)
\curveto(948.67895106,94.40924049)(948.56895117,94.39424051)(948.45894775,94.37424123)
\curveto(948.41895132,94.36424054)(948.37895136,94.35924054)(948.33894775,94.35924123)
\curveto(948.29895144,94.36924053)(948.25895148,94.36924053)(948.21894775,94.35924123)
\curveto(948.16895157,94.34924055)(948.11895162,94.34424056)(948.06894775,94.34424123)
\lineto(947.90394775,94.34424123)
\curveto(947.85395188,94.32424058)(947.80395193,94.31924058)(947.75394775,94.32924123)
\curveto(947.69395204,94.33924056)(947.6389521,94.33924056)(947.58894775,94.32924123)
\curveto(947.54895219,94.31924058)(947.50395223,94.31924058)(947.45394775,94.32924123)
\curveto(947.40395233,94.33924056)(947.35395238,94.33424057)(947.30394775,94.31424123)
\curveto(947.2339525,94.29424061)(947.15895258,94.28924061)(947.07894775,94.29924123)
\curveto(946.98895275,94.30924059)(946.90395283,94.31424059)(946.82394775,94.31424123)
\curveto(946.733953,94.31424059)(946.6339531,94.30924059)(946.52394775,94.29924123)
\curveto(946.40395333,94.28924061)(946.30395343,94.29424061)(946.22394775,94.31424123)
\lineto(945.93894775,94.31424123)
\lineto(945.30894775,94.35924123)
\curveto(945.20895453,94.36924053)(945.11395462,94.37924052)(945.02394775,94.38924123)
\lineto(944.72394775,94.41924123)
\curveto(944.67395506,94.43924046)(944.62395511,94.44424046)(944.57394775,94.43424123)
\curveto(944.51395522,94.43424047)(944.45895528,94.44424046)(944.40894775,94.46424123)
\curveto(944.2389555,94.51424039)(944.07395566,94.55424035)(943.91394775,94.58424123)
\curveto(943.74395599,94.61424029)(943.58395615,94.66424024)(943.43394775,94.73424123)
\curveto(942.97395676,94.92423998)(942.59895714,95.14423976)(942.30894775,95.39424123)
\curveto(942.01895772,95.65423925)(941.77395796,96.01423889)(941.57394775,96.47424123)
\curveto(941.52395821,96.6042383)(941.48895825,96.73423817)(941.46894775,96.86424123)
\curveto(941.44895829,97.0042379)(941.42395831,97.14423776)(941.39394775,97.28424123)
\curveto(941.38395835,97.35423755)(941.37895836,97.41923748)(941.37894775,97.47924123)
\curveto(941.37895836,97.53923736)(941.37395836,97.6042373)(941.36394775,97.67424123)
\curveto(941.34395839,98.5042364)(941.49395824,99.17423573)(941.81394775,99.68424123)
\curveto(942.12395761,100.19423471)(942.56395717,100.57423433)(943.13394775,100.82424123)
\curveto(943.25395648,100.87423403)(943.37895636,100.91923398)(943.50894775,100.95924123)
\curveto(943.6389561,100.9992339)(943.77395596,101.04423386)(943.91394775,101.09424123)
\curveto(943.99395574,101.11423379)(944.07895566,101.12923377)(944.16894775,101.13924123)
\lineto(944.40894775,101.19924123)
\curveto(944.51895522,101.22923367)(944.62895511,101.24423366)(944.73894775,101.24424123)
\curveto(944.84895489,101.25423365)(944.95895478,101.26923363)(945.06894775,101.28924123)
\curveto(945.11895462,101.30923359)(945.16395457,101.31423359)(945.20394775,101.30424123)
\curveto(945.24395449,101.3042336)(945.28395445,101.30923359)(945.32394775,101.31924123)
\curveto(945.37395436,101.32923357)(945.42895431,101.32923357)(945.48894775,101.31924123)
\curveto(945.5389542,101.31923358)(945.58895415,101.32423358)(945.63894775,101.33424123)
\lineto(945.77394775,101.33424123)
\curveto(945.8339539,101.35423355)(945.90395383,101.35423355)(945.98394775,101.33424123)
\curveto(946.05395368,101.32423358)(946.11895362,101.32923357)(946.17894775,101.34924123)
\curveto(946.20895353,101.35923354)(946.24895349,101.36423354)(946.29894775,101.36424123)
\lineto(946.41894775,101.36424123)
\lineto(946.88394775,101.36424123)
\moveto(949.20894775,99.81924123)
\curveto(948.88895085,99.91923498)(948.52395121,99.97923492)(948.11394775,99.99924123)
\curveto(947.70395203,100.01923488)(947.29395244,100.02923487)(946.88394775,100.02924123)
\curveto(946.45395328,100.02923487)(946.0339537,100.01923488)(945.62394775,99.99924123)
\curveto(945.21395452,99.97923492)(944.82895491,99.93423497)(944.46894775,99.86424123)
\curveto(944.10895563,99.79423511)(943.78895595,99.68423522)(943.50894775,99.53424123)
\curveto(943.21895652,99.39423551)(942.98395675,99.1992357)(942.80394775,98.94924123)
\curveto(942.69395704,98.78923611)(942.61395712,98.60923629)(942.56394775,98.40924123)
\curveto(942.50395723,98.20923669)(942.47395726,97.96423694)(942.47394775,97.67424123)
\curveto(942.49395724,97.65423725)(942.50395723,97.61923728)(942.50394775,97.56924123)
\curveto(942.49395724,97.51923738)(942.49395724,97.47923742)(942.50394775,97.44924123)
\curveto(942.52395721,97.36923753)(942.54395719,97.29423761)(942.56394775,97.22424123)
\curveto(942.57395716,97.16423774)(942.59395714,97.0992378)(942.62394775,97.02924123)
\curveto(942.74395699,96.75923814)(942.91395682,96.53923836)(943.13394775,96.36924123)
\curveto(943.34395639,96.20923869)(943.58895615,96.07423883)(943.86894775,95.96424123)
\curveto(943.97895576,95.91423899)(944.09895564,95.87423903)(944.22894775,95.84424123)
\curveto(944.34895539,95.82423908)(944.47395526,95.7992391)(944.60394775,95.76924123)
\curveto(944.65395508,95.74923915)(944.70895503,95.73923916)(944.76894775,95.73924123)
\curveto(944.81895492,95.73923916)(944.86895487,95.73423917)(944.91894775,95.72424123)
\curveto(945.00895473,95.71423919)(945.10395463,95.7042392)(945.20394775,95.69424123)
\curveto(945.29395444,95.68423922)(945.38895435,95.67423923)(945.48894775,95.66424123)
\curveto(945.56895417,95.66423924)(945.65395408,95.65923924)(945.74394775,95.64924123)
\lineto(945.98394775,95.64924123)
\lineto(946.16394775,95.64924123)
\curveto(946.19395354,95.63923926)(946.22895351,95.63423927)(946.26894775,95.63424123)
\lineto(946.40394775,95.63424123)
\lineto(946.85394775,95.63424123)
\curveto(946.9339528,95.63423927)(947.01895272,95.62923927)(947.10894775,95.61924123)
\curveto(947.18895255,95.61923928)(947.26395247,95.62923927)(947.33394775,95.64924123)
\lineto(947.60394775,95.64924123)
\curveto(947.62395211,95.64923925)(947.65395208,95.64423926)(947.69394775,95.63424123)
\curveto(947.72395201,95.63423927)(947.74895199,95.63923926)(947.76894775,95.64924123)
\curveto(947.86895187,95.65923924)(947.96895177,95.66423924)(948.06894775,95.66424123)
\curveto(948.15895158,95.67423923)(948.25895148,95.68423922)(948.36894775,95.69424123)
\curveto(948.48895125,95.72423918)(948.61395112,95.73923916)(948.74394775,95.73924123)
\curveto(948.86395087,95.74923915)(948.97895076,95.77423913)(949.08894775,95.81424123)
\curveto(949.38895035,95.89423901)(949.65395008,95.97923892)(949.88394775,96.06924123)
\curveto(950.11394962,96.16923873)(950.32894941,96.31423859)(950.52894775,96.50424123)
\curveto(950.72894901,96.71423819)(950.87894886,96.97923792)(950.97894775,97.29924123)
\curveto(950.99894874,97.33923756)(951.00894873,97.37423753)(951.00894775,97.40424123)
\curveto(950.99894874,97.44423746)(951.00394873,97.48923741)(951.02394775,97.53924123)
\curveto(951.0339487,97.57923732)(951.04394869,97.64923725)(951.05394775,97.74924123)
\curveto(951.06394867,97.85923704)(951.05894868,97.94423696)(951.03894775,98.00424123)
\curveto(951.01894872,98.07423683)(951.00894873,98.14423676)(951.00894775,98.21424123)
\curveto(950.99894874,98.28423662)(950.98394875,98.34923655)(950.96394775,98.40924123)
\curveto(950.90394883,98.60923629)(950.81894892,98.78923611)(950.70894775,98.94924123)
\curveto(950.68894905,98.97923592)(950.66894907,99.0042359)(950.64894775,99.02424123)
\lineto(950.58894775,99.08424123)
\curveto(950.56894917,99.12423578)(950.52894921,99.17423573)(950.46894775,99.23424123)
\curveto(950.32894941,99.33423557)(950.19894954,99.41923548)(950.07894775,99.48924123)
\curveto(949.95894978,99.55923534)(949.81394992,99.62923527)(949.64394775,99.69924123)
\curveto(949.57395016,99.72923517)(949.50395023,99.74923515)(949.43394775,99.75924123)
\curveto(949.36395037,99.77923512)(949.28895045,99.7992351)(949.20894775,99.81924123)
}
}
{
\newrgbcolor{curcolor}{0 0 0}
\pscustom[linestyle=none,fillstyle=solid,fillcolor=curcolor]
{
\newpath
\moveto(941.36394775,106.77385061)
\curveto(941.36395837,106.87384575)(941.37395836,106.96884566)(941.39394775,107.05885061)
\curveto(941.40395833,107.14884548)(941.4339583,107.21384541)(941.48394775,107.25385061)
\curveto(941.56395817,107.31384531)(941.66895807,107.34384528)(941.79894775,107.34385061)
\lineto(942.18894775,107.34385061)
\lineto(943.68894775,107.34385061)
\lineto(950.07894775,107.34385061)
\lineto(951.24894775,107.34385061)
\lineto(951.56394775,107.34385061)
\curveto(951.66394807,107.35384527)(951.74394799,107.33884529)(951.80394775,107.29885061)
\curveto(951.88394785,107.24884538)(951.9339478,107.17384545)(951.95394775,107.07385061)
\curveto(951.96394777,106.98384564)(951.96894777,106.87384575)(951.96894775,106.74385061)
\lineto(951.96894775,106.51885061)
\curveto(951.94894779,106.43884619)(951.9339478,106.36884626)(951.92394775,106.30885061)
\curveto(951.90394783,106.24884638)(951.86394787,106.19884643)(951.80394775,106.15885061)
\curveto(951.74394799,106.11884651)(951.66894807,106.09884653)(951.57894775,106.09885061)
\lineto(951.27894775,106.09885061)
\lineto(950.18394775,106.09885061)
\lineto(944.84394775,106.09885061)
\curveto(944.75395498,106.07884655)(944.67895506,106.06384656)(944.61894775,106.05385061)
\curveto(944.54895519,106.05384657)(944.48895525,106.0238466)(944.43894775,105.96385061)
\curveto(944.38895535,105.89384673)(944.36395537,105.80384682)(944.36394775,105.69385061)
\curveto(944.35395538,105.59384703)(944.34895539,105.48384714)(944.34894775,105.36385061)
\lineto(944.34894775,104.22385061)
\lineto(944.34894775,103.72885061)
\curveto(944.3389554,103.56884906)(944.27895546,103.45884917)(944.16894775,103.39885061)
\curveto(944.1389556,103.37884925)(944.10895563,103.36884926)(944.07894775,103.36885061)
\curveto(944.0389557,103.36884926)(943.99395574,103.36384926)(943.94394775,103.35385061)
\curveto(943.82395591,103.33384929)(943.71395602,103.33884929)(943.61394775,103.36885061)
\curveto(943.51395622,103.40884922)(943.44395629,103.46384916)(943.40394775,103.53385061)
\curveto(943.35395638,103.61384901)(943.32895641,103.73384889)(943.32894775,103.89385061)
\curveto(943.32895641,104.05384857)(943.31395642,104.18884844)(943.28394775,104.29885061)
\curveto(943.27395646,104.34884828)(943.26895647,104.40384822)(943.26894775,104.46385061)
\curveto(943.25895648,104.5238481)(943.24395649,104.58384804)(943.22394775,104.64385061)
\curveto(943.17395656,104.79384783)(943.12395661,104.93884769)(943.07394775,105.07885061)
\curveto(943.01395672,105.21884741)(942.94395679,105.35384727)(942.86394775,105.48385061)
\curveto(942.77395696,105.623847)(942.66895707,105.74384688)(942.54894775,105.84385061)
\curveto(942.42895731,105.94384668)(942.29895744,106.03884659)(942.15894775,106.12885061)
\curveto(942.05895768,106.18884644)(941.94895779,106.23384639)(941.82894775,106.26385061)
\curveto(941.70895803,106.30384632)(941.60395813,106.35384627)(941.51394775,106.41385061)
\curveto(941.45395828,106.46384616)(941.41395832,106.53384609)(941.39394775,106.62385061)
\curveto(941.38395835,106.64384598)(941.37895836,106.66884596)(941.37894775,106.69885061)
\curveto(941.37895836,106.7288459)(941.37395836,106.75384587)(941.36394775,106.77385061)
}
}
{
\newrgbcolor{curcolor}{0 0 0}
\pscustom[linestyle=none,fillstyle=solid,fillcolor=curcolor]
{
\newpath
\moveto(941.36394775,115.12345998)
\curveto(941.36395837,115.22345513)(941.37395836,115.31845503)(941.39394775,115.40845998)
\curveto(941.40395833,115.49845485)(941.4339583,115.56345479)(941.48394775,115.60345998)
\curveto(941.56395817,115.66345469)(941.66895807,115.69345466)(941.79894775,115.69345998)
\lineto(942.18894775,115.69345998)
\lineto(943.68894775,115.69345998)
\lineto(950.07894775,115.69345998)
\lineto(951.24894775,115.69345998)
\lineto(951.56394775,115.69345998)
\curveto(951.66394807,115.70345465)(951.74394799,115.68845466)(951.80394775,115.64845998)
\curveto(951.88394785,115.59845475)(951.9339478,115.52345483)(951.95394775,115.42345998)
\curveto(951.96394777,115.33345502)(951.96894777,115.22345513)(951.96894775,115.09345998)
\lineto(951.96894775,114.86845998)
\curveto(951.94894779,114.78845556)(951.9339478,114.71845563)(951.92394775,114.65845998)
\curveto(951.90394783,114.59845575)(951.86394787,114.5484558)(951.80394775,114.50845998)
\curveto(951.74394799,114.46845588)(951.66894807,114.4484559)(951.57894775,114.44845998)
\lineto(951.27894775,114.44845998)
\lineto(950.18394775,114.44845998)
\lineto(944.84394775,114.44845998)
\curveto(944.75395498,114.42845592)(944.67895506,114.41345594)(944.61894775,114.40345998)
\curveto(944.54895519,114.40345595)(944.48895525,114.37345598)(944.43894775,114.31345998)
\curveto(944.38895535,114.24345611)(944.36395537,114.1534562)(944.36394775,114.04345998)
\curveto(944.35395538,113.94345641)(944.34895539,113.83345652)(944.34894775,113.71345998)
\lineto(944.34894775,112.57345998)
\lineto(944.34894775,112.07845998)
\curveto(944.3389554,111.91845843)(944.27895546,111.80845854)(944.16894775,111.74845998)
\curveto(944.1389556,111.72845862)(944.10895563,111.71845863)(944.07894775,111.71845998)
\curveto(944.0389557,111.71845863)(943.99395574,111.71345864)(943.94394775,111.70345998)
\curveto(943.82395591,111.68345867)(943.71395602,111.68845866)(943.61394775,111.71845998)
\curveto(943.51395622,111.75845859)(943.44395629,111.81345854)(943.40394775,111.88345998)
\curveto(943.35395638,111.96345839)(943.32895641,112.08345827)(943.32894775,112.24345998)
\curveto(943.32895641,112.40345795)(943.31395642,112.53845781)(943.28394775,112.64845998)
\curveto(943.27395646,112.69845765)(943.26895647,112.7534576)(943.26894775,112.81345998)
\curveto(943.25895648,112.87345748)(943.24395649,112.93345742)(943.22394775,112.99345998)
\curveto(943.17395656,113.14345721)(943.12395661,113.28845706)(943.07394775,113.42845998)
\curveto(943.01395672,113.56845678)(942.94395679,113.70345665)(942.86394775,113.83345998)
\curveto(942.77395696,113.97345638)(942.66895707,114.09345626)(942.54894775,114.19345998)
\curveto(942.42895731,114.29345606)(942.29895744,114.38845596)(942.15894775,114.47845998)
\curveto(942.05895768,114.53845581)(941.94895779,114.58345577)(941.82894775,114.61345998)
\curveto(941.70895803,114.6534557)(941.60395813,114.70345565)(941.51394775,114.76345998)
\curveto(941.45395828,114.81345554)(941.41395832,114.88345547)(941.39394775,114.97345998)
\curveto(941.38395835,114.99345536)(941.37895836,115.01845533)(941.37894775,115.04845998)
\curveto(941.37895836,115.07845527)(941.37395836,115.10345525)(941.36394775,115.12345998)
}
}
{
\newrgbcolor{curcolor}{0 0 0}
\pscustom[linestyle=none,fillstyle=solid,fillcolor=curcolor]
{
\newpath
\moveto(972.10026367,38.71181936)
\curveto(972.15026442,38.73180981)(972.21026436,38.75680979)(972.28026367,38.78681936)
\curveto(972.35026422,38.81680973)(972.42526414,38.83680971)(972.50526367,38.84681936)
\curveto(972.57526399,38.86680968)(972.64526392,38.86680968)(972.71526367,38.84681936)
\curveto(972.77526379,38.83680971)(972.82026375,38.79680975)(972.85026367,38.72681936)
\curveto(972.8702637,38.67680987)(972.88026369,38.61680993)(972.88026367,38.54681936)
\lineto(972.88026367,38.33681936)
\lineto(972.88026367,37.88681936)
\curveto(972.88026369,37.73681081)(972.85526371,37.61681093)(972.80526367,37.52681936)
\curveto(972.74526382,37.42681112)(972.64026393,37.35181119)(972.49026367,37.30181936)
\curveto(972.34026423,37.26181128)(972.20526436,37.21681133)(972.08526367,37.16681936)
\curveto(971.82526474,37.05681149)(971.55526501,36.95681159)(971.27526367,36.86681936)
\curveto(970.99526557,36.77681177)(970.72026585,36.67681187)(970.45026367,36.56681936)
\curveto(970.36026621,36.53681201)(970.27526629,36.50681204)(970.19526367,36.47681936)
\curveto(970.11526645,36.45681209)(970.04026653,36.42681212)(969.97026367,36.38681936)
\curveto(969.90026667,36.35681219)(969.84026673,36.31181223)(969.79026367,36.25181936)
\curveto(969.74026683,36.19181235)(969.70026687,36.11181243)(969.67026367,36.01181936)
\curveto(969.65026692,35.96181258)(969.64526692,35.90181264)(969.65526367,35.83181936)
\lineto(969.65526367,35.63681936)
\lineto(969.65526367,32.80181936)
\lineto(969.65526367,32.50181936)
\curveto(969.64526692,32.39181615)(969.64526692,32.28681626)(969.65526367,32.18681936)
\curveto(969.6652669,32.08681646)(969.68026689,31.99181655)(969.70026367,31.90181936)
\curveto(969.72026685,31.82181672)(969.76026681,31.76181678)(969.82026367,31.72181936)
\curveto(969.92026665,31.6418169)(970.03526653,31.58181696)(970.16526367,31.54181936)
\curveto(970.28526628,31.51181703)(970.41026616,31.47181707)(970.54026367,31.42181936)
\curveto(970.7702658,31.32181722)(971.01026556,31.22681732)(971.26026367,31.13681936)
\curveto(971.51026506,31.05681749)(971.75026482,30.96681758)(971.98026367,30.86681936)
\curveto(972.04026453,30.8468177)(972.11026446,30.82181772)(972.19026367,30.79181936)
\curveto(972.26026431,30.77181777)(972.33526423,30.7468178)(972.41526367,30.71681936)
\curveto(972.49526407,30.68681786)(972.570264,30.65181789)(972.64026367,30.61181936)
\curveto(972.70026387,30.58181796)(972.74526382,30.546818)(972.77526367,30.50681936)
\curveto(972.83526373,30.42681812)(972.8702637,30.31681823)(972.88026367,30.17681936)
\lineto(972.88026367,29.75681936)
\lineto(972.88026367,29.51681936)
\curveto(972.8702637,29.4468191)(972.84526372,29.38681916)(972.80526367,29.33681936)
\curveto(972.77526379,29.28681926)(972.73026384,29.25681929)(972.67026367,29.24681936)
\curveto(972.61026396,29.2468193)(972.55026402,29.25181929)(972.49026367,29.26181936)
\curveto(972.42026415,29.28181926)(972.35526421,29.30181924)(972.29526367,29.32181936)
\curveto(972.22526434,29.35181919)(972.17526439,29.37681917)(972.14526367,29.39681936)
\curveto(971.82526474,29.53681901)(971.51026506,29.66181888)(971.20026367,29.77181936)
\curveto(970.88026569,29.88181866)(970.56026601,30.00181854)(970.24026367,30.13181936)
\curveto(970.02026655,30.22181832)(969.80526676,30.30681824)(969.59526367,30.38681936)
\curveto(969.37526719,30.46681808)(969.15526741,30.55181799)(968.93526367,30.64181936)
\curveto(968.21526835,30.9418176)(967.49026908,31.22681732)(966.76026367,31.49681936)
\curveto(966.02027055,31.76681678)(965.28527128,32.05181649)(964.55526367,32.35181936)
\curveto(964.29527227,32.46181608)(964.03027254,32.56181598)(963.76026367,32.65181936)
\curveto(963.49027308,32.75181579)(963.22527334,32.85681569)(962.96526367,32.96681936)
\curveto(962.85527371,33.01681553)(962.73527383,33.06181548)(962.60526367,33.10181936)
\curveto(962.4652741,33.15181539)(962.3652742,33.22181532)(962.30526367,33.31181936)
\curveto(962.2652743,33.35181519)(962.23527433,33.41681513)(962.21526367,33.50681936)
\curveto(962.20527436,33.52681502)(962.20527436,33.546815)(962.21526367,33.56681936)
\curveto(962.21527435,33.59681495)(962.21027436,33.62181492)(962.20026367,33.64181936)
\curveto(962.20027437,33.82181472)(962.20027437,34.03181451)(962.20026367,34.27181936)
\curveto(962.19027438,34.51181403)(962.22527434,34.68681386)(962.30526367,34.79681936)
\curveto(962.3652742,34.87681367)(962.4652741,34.93681361)(962.60526367,34.97681936)
\curveto(962.73527383,35.02681352)(962.85527371,35.07681347)(962.96526367,35.12681936)
\curveto(963.19527337,35.22681332)(963.42527314,35.31681323)(963.65526367,35.39681936)
\curveto(963.88527268,35.47681307)(964.11527245,35.56681298)(964.34526367,35.66681936)
\curveto(964.54527202,35.7468128)(964.75027182,35.82181272)(964.96026367,35.89181936)
\curveto(965.1702714,35.97181257)(965.37527119,36.05681249)(965.57526367,36.14681936)
\curveto(966.30527026,36.4468121)(967.04526952,36.73181181)(967.79526367,37.00181936)
\curveto(968.53526803,37.28181126)(969.2702673,37.57681097)(970.00026367,37.88681936)
\curveto(970.09026648,37.92681062)(970.17526639,37.95681059)(970.25526367,37.97681936)
\curveto(970.33526623,38.00681054)(970.42026615,38.03681051)(970.51026367,38.06681936)
\curveto(970.7702658,38.17681037)(971.03526553,38.28181026)(971.30526367,38.38181936)
\curveto(971.57526499,38.49181005)(971.84026473,38.60180994)(972.10026367,38.71181936)
\moveto(968.45526367,35.50181936)
\curveto(968.42526814,35.59181295)(968.37526819,35.6468129)(968.30526367,35.66681936)
\curveto(968.23526833,35.69681285)(968.16026841,35.70181284)(968.08026367,35.68181936)
\curveto(967.99026858,35.67181287)(967.90526866,35.6468129)(967.82526367,35.60681936)
\curveto(967.73526883,35.57681297)(967.66026891,35.546813)(967.60026367,35.51681936)
\curveto(967.56026901,35.49681305)(967.52526904,35.48681306)(967.49526367,35.48681936)
\curveto(967.4652691,35.48681306)(967.43026914,35.47681307)(967.39026367,35.45681936)
\lineto(967.15026367,35.36681936)
\curveto(967.06026951,35.3468132)(966.9702696,35.31681323)(966.88026367,35.27681936)
\curveto(966.52027005,35.12681342)(966.15527041,34.99181355)(965.78526367,34.87181936)
\curveto(965.40527116,34.76181378)(965.03527153,34.63181391)(964.67526367,34.48181936)
\curveto(964.565272,34.43181411)(964.45527211,34.38681416)(964.34526367,34.34681936)
\curveto(964.23527233,34.31681423)(964.13027244,34.27681427)(964.03026367,34.22681936)
\curveto(963.98027259,34.20681434)(963.93527263,34.18181436)(963.89526367,34.15181936)
\curveto(963.84527272,34.13181441)(963.82027275,34.08181446)(963.82026367,34.00181936)
\curveto(963.84027273,33.98181456)(963.85527271,33.96181458)(963.86526367,33.94181936)
\curveto(963.87527269,33.92181462)(963.89027268,33.90181464)(963.91026367,33.88181936)
\curveto(963.96027261,33.8418147)(964.01527255,33.81181473)(964.07526367,33.79181936)
\curveto(964.12527244,33.77181477)(964.18027239,33.75181479)(964.24026367,33.73181936)
\curveto(964.35027222,33.68181486)(964.46027211,33.6418149)(964.57026367,33.61181936)
\curveto(964.68027189,33.58181496)(964.79027178,33.541815)(964.90026367,33.49181936)
\curveto(965.29027128,33.32181522)(965.68527088,33.17181537)(966.08526367,33.04181936)
\curveto(966.48527008,32.92181562)(966.87526969,32.78181576)(967.25526367,32.62181936)
\lineto(967.40526367,32.56181936)
\curveto(967.45526911,32.55181599)(967.50526906,32.53681601)(967.55526367,32.51681936)
\lineto(967.79526367,32.42681936)
\curveto(967.87526869,32.39681615)(967.95526861,32.37181617)(968.03526367,32.35181936)
\curveto(968.08526848,32.33181621)(968.14026843,32.32181622)(968.20026367,32.32181936)
\curveto(968.26026831,32.33181621)(968.31026826,32.3468162)(968.35026367,32.36681936)
\curveto(968.43026814,32.41681613)(968.47526809,32.52181602)(968.48526367,32.68181936)
\lineto(968.48526367,33.13181936)
\lineto(968.48526367,34.73681936)
\curveto(968.48526808,34.8468137)(968.49026808,34.98181356)(968.50026367,35.14181936)
\curveto(968.50026807,35.30181324)(968.48526808,35.42181312)(968.45526367,35.50181936)
}
}
{
\newrgbcolor{curcolor}{0 0 0}
\pscustom[linestyle=none,fillstyle=solid,fillcolor=curcolor]
{
\newpath
\moveto(965.26026367,46.44338186)
\curveto(965.31027126,46.51337426)(965.38527118,46.54837422)(965.48526367,46.54838186)
\curveto(965.58527098,46.55837421)(965.69027088,46.56337421)(965.80026367,46.56338186)
\lineto(972.07026367,46.56338186)
\lineto(972.67026367,46.56338186)
\curveto(972.72026385,46.54337423)(972.7702638,46.53837423)(972.82026367,46.54838186)
\curveto(972.86026371,46.55837421)(972.90526366,46.55337422)(972.95526367,46.53338186)
\curveto(973.05526351,46.51337426)(973.15526341,46.49837427)(973.25526367,46.48838186)
\curveto(973.3652632,46.48837428)(973.4702631,46.4733743)(973.57026367,46.44338186)
\curveto(973.68026289,46.41337436)(973.78526278,46.38337439)(973.88526367,46.35338186)
\curveto(973.98526258,46.33337444)(974.08526248,46.29837447)(974.18526367,46.24838186)
\curveto(974.44526212,46.14837462)(974.68026189,46.01837475)(974.89026367,45.85838186)
\curveto(975.10026147,45.70837506)(975.27526129,45.52837524)(975.41526367,45.31838186)
\curveto(975.53526103,45.14837562)(975.63026094,44.9683758)(975.70026367,44.77838186)
\curveto(975.78026079,44.58837618)(975.85526071,44.38337639)(975.92526367,44.16338186)
\curveto(975.94526062,44.0733767)(975.95526061,43.98337679)(975.95526367,43.89338186)
\curveto(975.9652606,43.80337697)(975.98026059,43.71337706)(976.00026367,43.62338186)
\lineto(976.00026367,43.53338186)
\curveto(976.01026056,43.51337726)(976.01526055,43.49337728)(976.01526367,43.47338186)
\curveto(976.02526054,43.42337735)(976.02526054,43.3733774)(976.01526367,43.32338186)
\curveto(976.00526056,43.28337749)(976.01026056,43.23837753)(976.03026367,43.18838186)
\curveto(976.05026052,43.11837765)(976.05526051,43.00837776)(976.04526367,42.85838186)
\curveto(976.04526052,42.71837805)(976.03526053,42.61837815)(976.01526367,42.55838186)
\curveto(976.01526055,42.52837824)(976.01026056,42.49837827)(976.00026367,42.46838186)
\lineto(976.00026367,42.40838186)
\curveto(975.98026059,42.31837845)(975.9652606,42.22837854)(975.95526367,42.13838186)
\curveto(975.95526061,42.04837872)(975.94526062,41.96337881)(975.92526367,41.88338186)
\curveto(975.90526066,41.80337897)(975.88026069,41.72337905)(975.85026367,41.64338186)
\curveto(975.83026074,41.56337921)(975.80526076,41.48337929)(975.77526367,41.40338186)
\curveto(975.64526092,41.08337969)(975.50026107,40.81337996)(975.34026367,40.59338186)
\curveto(975.18026139,40.38338039)(974.95526161,40.19338058)(974.66526367,40.02338186)
\curveto(974.64526192,40.00338077)(974.62026195,39.98838078)(974.59026367,39.97838186)
\curveto(974.570262,39.97838079)(974.54526202,39.9683808)(974.51526367,39.94838186)
\curveto(974.43526213,39.91838085)(974.32026225,39.88338089)(974.17026367,39.84338186)
\curveto(974.03026254,39.81338096)(973.92526264,39.84338093)(973.85526367,39.93338186)
\curveto(973.80526276,39.99338078)(973.78026279,40.0733807)(973.78026367,40.17338186)
\lineto(973.78026367,40.50338186)
\lineto(973.78026367,40.66838186)
\curveto(973.78026279,40.72838004)(973.79026278,40.78337999)(973.81026367,40.83338186)
\curveto(973.84026273,40.92337985)(973.89026268,40.98837978)(973.96026367,41.02838186)
\curveto(974.03026254,41.0683797)(974.10526246,41.11337966)(974.18526367,41.16338186)
\lineto(974.36526367,41.28338186)
\curveto(974.43526213,41.33337944)(974.49026208,41.38337939)(974.53026367,41.43338186)
\curveto(974.72026185,41.68337909)(974.86026171,41.98337879)(974.95026367,42.33338186)
\curveto(974.9702616,42.39337838)(974.98026159,42.45337832)(974.98026367,42.51338186)
\curveto(974.99026158,42.58337819)(975.00526156,42.64837812)(975.02526367,42.70838186)
\lineto(975.02526367,42.79838186)
\curveto(975.04526152,42.8683779)(975.05526151,42.95337782)(975.05526367,43.05338186)
\curveto(975.05526151,43.15337762)(975.04526152,43.24337753)(975.02526367,43.32338186)
\curveto(975.01526155,43.35337742)(975.01026156,43.39337738)(975.01026367,43.44338186)
\curveto(974.99026158,43.54337723)(974.9702616,43.63837713)(974.95026367,43.72838186)
\curveto(974.94026163,43.81837695)(974.91526165,43.90337687)(974.87526367,43.98338186)
\curveto(974.75526181,44.2733765)(974.59026198,44.50837626)(974.38026367,44.68838186)
\curveto(974.18026239,44.87837589)(973.93526263,45.03337574)(973.64526367,45.15338186)
\curveto(973.55526301,45.19337558)(973.46026311,45.21837555)(973.36026367,45.22838186)
\curveto(973.26026331,45.24837552)(973.15526341,45.2733755)(973.04526367,45.30338186)
\curveto(972.99526357,45.32337545)(972.94526362,45.33337544)(972.89526367,45.33338186)
\curveto(972.84526372,45.33337544)(972.79526377,45.33837543)(972.74526367,45.34838186)
\curveto(972.71526385,45.35837541)(972.6652639,45.36337541)(972.59526367,45.36338186)
\curveto(972.51526405,45.38337539)(972.43026414,45.38337539)(972.34026367,45.36338186)
\curveto(972.29026428,45.35337542)(972.24526432,45.34837542)(972.20526367,45.34838186)
\curveto(972.1652644,45.35837541)(972.13026444,45.35337542)(972.10026367,45.33338186)
\curveto(972.08026449,45.31337546)(972.0702645,45.29837547)(972.07026367,45.28838186)
\lineto(972.02526367,45.24338186)
\curveto(972.02526454,45.14337563)(972.05526451,45.0683757)(972.11526367,45.01838186)
\curveto(972.1652644,44.97837579)(972.21026436,44.92837584)(972.25026367,44.86838186)
\lineto(972.46026367,44.62838186)
\curveto(972.52026405,44.54837622)(972.57526399,44.45837631)(972.62526367,44.35838186)
\curveto(972.71526385,44.21837655)(972.79026378,44.04337673)(972.85026367,43.83338186)
\curveto(972.90026367,43.62337715)(972.93526363,43.40337737)(972.95526367,43.17338186)
\curveto(972.97526359,42.94337783)(972.9702636,42.71337806)(972.94026367,42.48338186)
\curveto(972.92026365,42.25337852)(972.88026369,42.04337873)(972.82026367,41.85338186)
\curveto(972.51026406,40.91337986)(971.91526465,40.25338052)(971.03526367,39.87338186)
\curveto(970.93526563,39.82338095)(970.84026573,39.78338099)(970.75026367,39.75338186)
\curveto(970.65026592,39.72338105)(970.54526602,39.68838108)(970.43526367,39.64838186)
\curveto(970.38526618,39.62838114)(970.34026623,39.61838115)(970.30026367,39.61838186)
\curveto(970.26026631,39.61838115)(970.21526635,39.60838116)(970.16526367,39.58838186)
\curveto(970.09526647,39.5683812)(970.02526654,39.55338122)(969.95526367,39.54338186)
\curveto(969.87526669,39.54338123)(969.80026677,39.53338124)(969.73026367,39.51338186)
\curveto(969.69026688,39.50338127)(969.65526691,39.49838127)(969.62526367,39.49838186)
\curveto(969.58526698,39.50838126)(969.54526702,39.50838126)(969.50526367,39.49838186)
\curveto(969.4652671,39.49838127)(969.42526714,39.49338128)(969.38526367,39.48338186)
\lineto(969.26526367,39.48338186)
\curveto(969.14526742,39.46338131)(969.02026755,39.46338131)(968.89026367,39.48338186)
\curveto(968.83026774,39.49338128)(968.7702678,39.49838127)(968.71026367,39.49838186)
\lineto(968.54526367,39.49838186)
\curveto(968.49526807,39.50838126)(968.45526811,39.51338126)(968.42526367,39.51338186)
\curveto(968.38526818,39.51338126)(968.34026823,39.51838125)(968.29026367,39.52838186)
\curveto(968.18026839,39.55838121)(968.07526849,39.57838119)(967.97526367,39.58838186)
\curveto(967.8652687,39.59838117)(967.75526881,39.62338115)(967.64526367,39.66338186)
\curveto(967.52526904,39.70338107)(967.41026916,39.73838103)(967.30026367,39.76838186)
\curveto(967.18026939,39.80838096)(967.0652695,39.85338092)(966.95526367,39.90338186)
\curveto(966.79526977,39.9733808)(966.65026992,40.05338072)(966.52026367,40.14338186)
\curveto(966.38027019,40.23338054)(966.24527032,40.32838044)(966.11526367,40.42838186)
\curveto(966.00527056,40.49838027)(965.91527065,40.58838018)(965.84526367,40.69838186)
\lineto(965.78526367,40.75838186)
\lineto(965.72526367,40.81838186)
\lineto(965.60526367,40.96838186)
\lineto(965.48526367,41.14838186)
\curveto(965.40527116,41.27837949)(965.33527123,41.41337936)(965.27526367,41.55338186)
\curveto(965.21527135,41.70337907)(965.16027141,41.86337891)(965.11026367,42.03338186)
\curveto(965.08027149,42.13337864)(965.06027151,42.23337854)(965.05026367,42.33338186)
\curveto(965.04027153,42.44337833)(965.02527154,42.55337822)(965.00526367,42.66338186)
\curveto(964.99527157,42.70337807)(964.99527157,42.75337802)(965.00526367,42.81338186)
\curveto(965.01527155,42.88337789)(965.01027156,42.93337784)(964.99026367,42.96338186)
\curveto(964.98027159,43.28337749)(965.01027156,43.5683772)(965.08026367,43.81838186)
\curveto(965.15027142,44.07837669)(965.25027132,44.30837646)(965.38026367,44.50838186)
\curveto(965.42027115,44.57837619)(965.4652711,44.64337613)(965.51526367,44.70338186)
\lineto(965.66526367,44.88338186)
\curveto(965.70527086,44.93337584)(965.75027082,44.97837579)(965.80026367,45.01838186)
\curveto(965.84027073,45.0683757)(965.86027071,45.14337563)(965.86026367,45.24338186)
\lineto(965.81526367,45.28838186)
\curveto(965.79527077,45.30837546)(965.7702708,45.32837544)(965.74026367,45.34838186)
\curveto(965.66027091,45.37837539)(965.58027099,45.39337538)(965.50026367,45.39338186)
\curveto(965.42027115,45.40337537)(965.35027122,45.43337534)(965.29026367,45.48338186)
\curveto(965.25027132,45.51337526)(965.22027135,45.5733752)(965.20026367,45.66338186)
\curveto(965.1702714,45.75337502)(965.15527141,45.84837492)(965.15526367,45.94838186)
\curveto(965.15527141,46.04837472)(965.1652714,46.14337463)(965.18526367,46.23338186)
\curveto(965.20527136,46.33337444)(965.23027134,46.40337437)(965.26026367,46.44338186)
\moveto(969.04026367,45.31838186)
\curveto(969.00026757,45.32837544)(968.95026762,45.33337544)(968.89026367,45.33338186)
\curveto(968.82026775,45.33337544)(968.7652678,45.32837544)(968.72526367,45.31838186)
\lineto(968.48526367,45.31838186)
\curveto(968.39526817,45.29837547)(968.31026826,45.28337549)(968.23026367,45.27338186)
\curveto(968.14026843,45.26337551)(968.05526851,45.24837552)(967.97526367,45.22838186)
\curveto(967.89526867,45.20837556)(967.82026875,45.18837558)(967.75026367,45.16838186)
\curveto(967.6702689,45.15837561)(967.59526897,45.13837563)(967.52526367,45.10838186)
\curveto(967.24526932,44.99837577)(966.99526957,44.85337592)(966.77526367,44.67338186)
\curveto(966.55527001,44.50337627)(966.39027018,44.28337649)(966.28026367,44.01338186)
\curveto(966.24027033,43.93337684)(966.21027036,43.84837692)(966.19026367,43.75838186)
\curveto(966.16027041,43.6683771)(966.13527043,43.5733772)(966.11526367,43.47338186)
\curveto(966.09527047,43.39337738)(966.09027048,43.30337747)(966.10026367,43.20338186)
\lineto(966.10026367,42.93338186)
\curveto(966.11027046,42.88337789)(966.11527045,42.83337794)(966.11526367,42.78338186)
\curveto(966.11527045,42.74337803)(966.12027045,42.69837807)(966.13026367,42.64838186)
\curveto(966.18027039,42.45837831)(966.23027034,42.29837847)(966.28026367,42.16838186)
\curveto(966.42027015,41.82837894)(966.63026994,41.56337921)(966.91026367,41.37338186)
\curveto(967.19026938,41.18337959)(967.51526905,41.03337974)(967.88526367,40.92338186)
\curveto(967.9652686,40.90337987)(968.04526852,40.88837988)(968.12526367,40.87838186)
\curveto(968.19526837,40.87837989)(968.2702683,40.8683799)(968.35026367,40.84838186)
\curveto(968.38026819,40.82837994)(968.41526815,40.81837995)(968.45526367,40.81838186)
\curveto(968.49526807,40.82837994)(968.53026804,40.82837994)(968.56026367,40.81838186)
\lineto(968.89026367,40.81838186)
\lineto(969.23526367,40.81838186)
\curveto(969.34526722,40.81837995)(969.45026712,40.82837994)(969.55026367,40.84838186)
\lineto(969.62526367,40.84838186)
\curveto(969.65526691,40.85837991)(969.68026689,40.86337991)(969.70026367,40.86338186)
\curveto(969.79026678,40.88337989)(969.88026669,40.89837987)(969.97026367,40.90838186)
\curveto(970.06026651,40.92837984)(970.14526642,40.95337982)(970.22526367,40.98338186)
\curveto(970.48526608,41.06337971)(970.72526584,41.16337961)(970.94526367,41.28338186)
\curveto(971.1652654,41.40337937)(971.34526522,41.56337921)(971.48526367,41.76338186)
\lineto(971.57526367,41.88338186)
\curveto(971.59526497,41.92337885)(971.61526495,41.9683788)(971.63526367,42.01838186)
\curveto(971.68526488,42.09837867)(971.72526484,42.18337859)(971.75526367,42.27338186)
\curveto(971.78526478,42.36337841)(971.81526475,42.46337831)(971.84526367,42.57338186)
\curveto(971.85526471,42.62337815)(971.86026471,42.6683781)(971.86026367,42.70838186)
\curveto(971.85026472,42.75837801)(971.85526471,42.80837796)(971.87526367,42.85838186)
\curveto(971.88526468,42.88837788)(971.89026468,42.93837783)(971.89026367,43.00838186)
\curveto(971.89026468,43.07837769)(971.88526468,43.12837764)(971.87526367,43.15838186)
\curveto(971.8652647,43.18837758)(971.8652647,43.21837755)(971.87526367,43.24838186)
\curveto(971.87526469,43.28837748)(971.8702647,43.32837744)(971.86026367,43.36838186)
\curveto(971.84026473,43.45837731)(971.82026475,43.54337723)(971.80026367,43.62338186)
\curveto(971.78026479,43.70337707)(971.75526481,43.78337699)(971.72526367,43.86338186)
\curveto(971.57526499,44.20337657)(971.3652652,44.4733763)(971.09526367,44.67338186)
\curveto(970.82526574,44.8733759)(970.51026606,45.03337574)(970.15026367,45.15338186)
\curveto(970.06026651,45.18337559)(969.9702666,45.20337557)(969.88026367,45.21338186)
\curveto(969.78026679,45.23337554)(969.68526688,45.25337552)(969.59526367,45.27338186)
\curveto(969.55526701,45.28337549)(969.52026705,45.28837548)(969.49026367,45.28838186)
\curveto(969.45026712,45.28837548)(969.41026716,45.29337548)(969.37026367,45.30338186)
\curveto(969.32026725,45.32337545)(969.2702673,45.32337545)(969.22026367,45.30338186)
\curveto(969.16026741,45.29337548)(969.10026747,45.29837547)(969.04026367,45.31838186)
}
}
{
\newrgbcolor{curcolor}{0 0 0}
\pscustom[linestyle=none,fillstyle=solid,fillcolor=curcolor]
{
\newpath
\moveto(968.68026367,55.56666311)
\curveto(968.74026783,55.58665505)(968.83526773,55.59665504)(968.96526367,55.59666311)
\curveto(969.08526748,55.59665504)(969.1702674,55.59165504)(969.22026367,55.58166311)
\lineto(969.37026367,55.58166311)
\curveto(969.45026712,55.57165506)(969.52526704,55.56165507)(969.59526367,55.55166311)
\curveto(969.65526691,55.55165508)(969.72526684,55.54665509)(969.80526367,55.53666311)
\curveto(969.8652667,55.51665512)(969.92526664,55.50165513)(969.98526367,55.49166311)
\curveto(970.04526652,55.49165514)(970.10526646,55.48165515)(970.16526367,55.46166311)
\curveto(970.29526627,55.42165521)(970.42526614,55.38665525)(970.55526367,55.35666311)
\curveto(970.68526588,55.32665531)(970.80526576,55.28665535)(970.91526367,55.23666311)
\curveto(971.39526517,55.02665561)(971.80026477,54.74665589)(972.13026367,54.39666311)
\curveto(972.45026412,54.04665659)(972.69526387,53.61665702)(972.86526367,53.10666311)
\curveto(972.90526366,52.99665764)(972.93526363,52.87665776)(972.95526367,52.74666311)
\curveto(972.97526359,52.62665801)(972.99526357,52.50165813)(973.01526367,52.37166311)
\curveto(973.02526354,52.31165832)(973.03026354,52.24665839)(973.03026367,52.17666311)
\curveto(973.04026353,52.11665852)(973.04526352,52.05665858)(973.04526367,51.99666311)
\curveto(973.05526351,51.95665868)(973.06026351,51.89665874)(973.06026367,51.81666311)
\curveto(973.06026351,51.74665889)(973.05526351,51.69665894)(973.04526367,51.66666311)
\curveto(973.03526353,51.62665901)(973.03026354,51.58665905)(973.03026367,51.54666311)
\curveto(973.04026353,51.50665913)(973.04026353,51.47165916)(973.03026367,51.44166311)
\lineto(973.03026367,51.35166311)
\lineto(972.98526367,50.99166311)
\curveto(972.94526362,50.85165978)(972.90526366,50.71665992)(972.86526367,50.58666311)
\curveto(972.82526374,50.45666018)(972.78026379,50.3316603)(972.73026367,50.21166311)
\curveto(972.53026404,49.76166087)(972.2702643,49.39166124)(971.95026367,49.10166311)
\curveto(971.63026494,48.81166182)(971.24026533,48.57166206)(970.78026367,48.38166311)
\curveto(970.68026589,48.3316623)(970.58026599,48.29166234)(970.48026367,48.26166311)
\curveto(970.38026619,48.24166239)(970.27526629,48.22166241)(970.16526367,48.20166311)
\curveto(970.12526644,48.18166245)(970.09526647,48.17166246)(970.07526367,48.17166311)
\curveto(970.04526652,48.18166245)(970.01026656,48.18166245)(969.97026367,48.17166311)
\curveto(969.89026668,48.15166248)(969.81026676,48.1366625)(969.73026367,48.12666311)
\curveto(969.64026693,48.12666251)(969.55526701,48.11666252)(969.47526367,48.09666311)
\lineto(969.35526367,48.09666311)
\curveto(969.31526725,48.09666254)(969.2702673,48.09166254)(969.22026367,48.08166311)
\curveto(969.1702674,48.07166256)(969.08526748,48.06666257)(968.96526367,48.06666311)
\curveto(968.83526773,48.06666257)(968.74026783,48.07666256)(968.68026367,48.09666311)
\curveto(968.61026796,48.11666252)(968.54026803,48.12166251)(968.47026367,48.11166311)
\curveto(968.40026817,48.10166253)(968.33026824,48.10666253)(968.26026367,48.12666311)
\curveto(968.21026836,48.1366625)(968.1702684,48.14166249)(968.14026367,48.14166311)
\curveto(968.10026847,48.15166248)(968.05526851,48.16166247)(968.00526367,48.17166311)
\curveto(967.88526868,48.20166243)(967.7652688,48.22666241)(967.64526367,48.24666311)
\curveto(967.52526904,48.27666236)(967.41026916,48.31666232)(967.30026367,48.36666311)
\curveto(966.93026964,48.51666212)(966.60026997,48.69666194)(966.31026367,48.90666311)
\curveto(966.01027056,49.12666151)(965.76027081,49.39166124)(965.56026367,49.70166311)
\curveto(965.48027109,49.82166081)(965.41527115,49.94666069)(965.36526367,50.07666311)
\curveto(965.30527126,50.20666043)(965.24527132,50.34166029)(965.18526367,50.48166311)
\curveto(965.13527143,50.60166003)(965.10527146,50.7316599)(965.09526367,50.87166311)
\curveto(965.07527149,51.01165962)(965.04527152,51.15165948)(965.00526367,51.29166311)
\lineto(965.00526367,51.48666311)
\curveto(964.99527157,51.55665908)(964.98527158,51.62165901)(964.97526367,51.68166311)
\curveto(964.9652716,52.57165806)(965.15027142,53.31165732)(965.53026367,53.90166311)
\curveto(965.91027066,54.49165614)(966.40527016,54.91665572)(967.01526367,55.17666311)
\curveto(967.11526945,55.22665541)(967.21526935,55.26665537)(967.31526367,55.29666311)
\curveto(967.41526915,55.32665531)(967.52026905,55.36165527)(967.63026367,55.40166311)
\curveto(967.74026883,55.4316552)(967.86026871,55.45665518)(967.99026367,55.47666311)
\curveto(968.11026846,55.49665514)(968.23526833,55.52165511)(968.36526367,55.55166311)
\curveto(968.41526815,55.56165507)(968.4702681,55.56165507)(968.53026367,55.55166311)
\curveto(968.58026799,55.55165508)(968.63026794,55.55665508)(968.68026367,55.56666311)
\moveto(969.53526367,54.23166311)
\curveto(969.4652671,54.25165638)(969.38526718,54.25665638)(969.29526367,54.24666311)
\lineto(969.04026367,54.24666311)
\curveto(968.65026792,54.24665639)(968.32026825,54.21165642)(968.05026367,54.14166311)
\curveto(967.9702686,54.11165652)(967.89026868,54.08665655)(967.81026367,54.06666311)
\curveto(967.73026884,54.04665659)(967.65526891,54.02165661)(967.58526367,53.99166311)
\curveto(966.93526963,53.71165692)(966.48527008,53.26665737)(966.23526367,52.65666311)
\curveto(966.20527036,52.58665805)(966.18527038,52.51165812)(966.17526367,52.43166311)
\lineto(966.11526367,52.19166311)
\curveto(966.09527047,52.11165852)(966.08527048,52.02665861)(966.08526367,51.93666311)
\lineto(966.08526367,51.66666311)
\lineto(966.13026367,51.39666311)
\curveto(966.15027042,51.29665934)(966.17527039,51.20165943)(966.20526367,51.11166311)
\curveto(966.22527034,51.0316596)(966.25527031,50.95165968)(966.29526367,50.87166311)
\curveto(966.31527025,50.80165983)(966.34527022,50.7366599)(966.38526367,50.67666311)
\curveto(966.42527014,50.61666002)(966.4652701,50.56166007)(966.50526367,50.51166311)
\curveto(966.67526989,50.27166036)(966.88026969,50.07666056)(967.12026367,49.92666311)
\curveto(967.36026921,49.77666086)(967.64026893,49.64666099)(967.96026367,49.53666311)
\curveto(968.06026851,49.50666113)(968.1652684,49.48666115)(968.27526367,49.47666311)
\curveto(968.37526819,49.46666117)(968.48026809,49.45166118)(968.59026367,49.43166311)
\curveto(968.63026794,49.42166121)(968.69526787,49.41666122)(968.78526367,49.41666311)
\curveto(968.81526775,49.40666123)(968.85026772,49.40166123)(968.89026367,49.40166311)
\curveto(968.93026764,49.41166122)(968.97526759,49.41666122)(969.02526367,49.41666311)
\lineto(969.32526367,49.41666311)
\curveto(969.42526714,49.41666122)(969.51526705,49.42666121)(969.59526367,49.44666311)
\lineto(969.77526367,49.47666311)
\curveto(969.87526669,49.49666114)(969.97526659,49.51166112)(970.07526367,49.52166311)
\curveto(970.1652664,49.54166109)(970.25026632,49.57166106)(970.33026367,49.61166311)
\curveto(970.570266,49.71166092)(970.79526577,49.82666081)(971.00526367,49.95666311)
\curveto(971.21526535,50.09666054)(971.39026518,50.26666037)(971.53026367,50.46666311)
\curveto(971.56026501,50.51666012)(971.58526498,50.56166007)(971.60526367,50.60166311)
\curveto(971.62526494,50.64165999)(971.65026492,50.68665995)(971.68026367,50.73666311)
\curveto(971.73026484,50.81665982)(971.77526479,50.90165973)(971.81526367,50.99166311)
\curveto(971.84526472,51.09165954)(971.87526469,51.19665944)(971.90526367,51.30666311)
\curveto(971.92526464,51.35665928)(971.93526463,51.40165923)(971.93526367,51.44166311)
\curveto(971.92526464,51.49165914)(971.92526464,51.54165909)(971.93526367,51.59166311)
\curveto(971.94526462,51.62165901)(971.95526461,51.68165895)(971.96526367,51.77166311)
\curveto(971.97526459,51.87165876)(971.9702646,51.94665869)(971.95026367,51.99666311)
\curveto(971.94026463,52.0366586)(971.94026463,52.07665856)(971.95026367,52.11666311)
\curveto(971.95026462,52.15665848)(971.94026463,52.19665844)(971.92026367,52.23666311)
\curveto(971.90026467,52.31665832)(971.88526468,52.39665824)(971.87526367,52.47666311)
\curveto(971.85526471,52.55665808)(971.83026474,52.631658)(971.80026367,52.70166311)
\curveto(971.66026491,53.04165759)(971.4652651,53.31665732)(971.21526367,53.52666311)
\curveto(970.9652656,53.7366569)(970.6702659,53.91165672)(970.33026367,54.05166311)
\curveto(970.21026636,54.10165653)(970.08526648,54.1316565)(969.95526367,54.14166311)
\curveto(969.81526675,54.16165647)(969.67526689,54.19165644)(969.53526367,54.23166311)
}
}
{
\newrgbcolor{curcolor}{0 0 0}
\pscustom[linestyle=none,fillstyle=solid,fillcolor=curcolor]
{
}
}
{
\newrgbcolor{curcolor}{0 0 0}
\pscustom[linestyle=none,fillstyle=solid,fillcolor=curcolor]
{
\newpath
\moveto(967.79526367,67.99510061)
\lineto(968.05026367,67.99510061)
\curveto(968.13026844,68.0050929)(968.20526836,68.00009291)(968.27526367,67.98010061)
\lineto(968.51526367,67.98010061)
\lineto(968.68026367,67.98010061)
\curveto(968.78026779,67.96009295)(968.88526768,67.95009296)(968.99526367,67.95010061)
\curveto(969.09526747,67.95009296)(969.19526737,67.94009297)(969.29526367,67.92010061)
\lineto(969.44526367,67.92010061)
\curveto(969.58526698,67.89009302)(969.72526684,67.87009304)(969.86526367,67.86010061)
\curveto(969.99526657,67.85009306)(970.12526644,67.82509308)(970.25526367,67.78510061)
\curveto(970.33526623,67.76509314)(970.42026615,67.74509316)(970.51026367,67.72510061)
\lineto(970.75026367,67.66510061)
\lineto(971.05026367,67.54510061)
\curveto(971.14026543,67.51509339)(971.23026534,67.48009343)(971.32026367,67.44010061)
\curveto(971.54026503,67.34009357)(971.75526481,67.2050937)(971.96526367,67.03510061)
\curveto(972.17526439,66.87509403)(972.34526422,66.70009421)(972.47526367,66.51010061)
\curveto(972.51526405,66.46009445)(972.55526401,66.40009451)(972.59526367,66.33010061)
\curveto(972.62526394,66.27009464)(972.66026391,66.2100947)(972.70026367,66.15010061)
\curveto(972.75026382,66.07009484)(972.79026378,65.97509493)(972.82026367,65.86510061)
\curveto(972.85026372,65.75509515)(972.88026369,65.65009526)(972.91026367,65.55010061)
\curveto(972.95026362,65.44009547)(972.97526359,65.33009558)(972.98526367,65.22010061)
\curveto(972.99526357,65.1100958)(973.01026356,64.99509591)(973.03026367,64.87510061)
\curveto(973.04026353,64.83509607)(973.04026353,64.79009612)(973.03026367,64.74010061)
\curveto(973.03026354,64.70009621)(973.03526353,64.66009625)(973.04526367,64.62010061)
\curveto(973.05526351,64.58009633)(973.06026351,64.52509638)(973.06026367,64.45510061)
\curveto(973.06026351,64.38509652)(973.05526351,64.33509657)(973.04526367,64.30510061)
\curveto(973.02526354,64.25509665)(973.02026355,64.2100967)(973.03026367,64.17010061)
\curveto(973.04026353,64.13009678)(973.04026353,64.09509681)(973.03026367,64.06510061)
\lineto(973.03026367,63.97510061)
\curveto(973.01026356,63.91509699)(972.99526357,63.85009706)(972.98526367,63.78010061)
\curveto(972.98526358,63.72009719)(972.98026359,63.65509725)(972.97026367,63.58510061)
\curveto(972.92026365,63.41509749)(972.8702637,63.25509765)(972.82026367,63.10510061)
\curveto(972.7702638,62.95509795)(972.70526386,62.8100981)(972.62526367,62.67010061)
\curveto(972.58526398,62.62009829)(972.55526401,62.56509834)(972.53526367,62.50510061)
\curveto(972.50526406,62.45509845)(972.4702641,62.4050985)(972.43026367,62.35510061)
\curveto(972.25026432,62.11509879)(972.03026454,61.91509899)(971.77026367,61.75510061)
\curveto(971.51026506,61.59509931)(971.22526534,61.45509945)(970.91526367,61.33510061)
\curveto(970.77526579,61.27509963)(970.63526593,61.23009968)(970.49526367,61.20010061)
\curveto(970.34526622,61.17009974)(970.19026638,61.13509977)(970.03026367,61.09510061)
\curveto(969.92026665,61.07509983)(969.81026676,61.06009985)(969.70026367,61.05010061)
\curveto(969.59026698,61.04009987)(969.48026709,61.02509988)(969.37026367,61.00510061)
\curveto(969.33026724,60.99509991)(969.29026728,60.99009992)(969.25026367,60.99010061)
\curveto(969.21026736,61.00009991)(969.1702674,61.00009991)(969.13026367,60.99010061)
\curveto(969.08026749,60.98009993)(969.03026754,60.97509993)(968.98026367,60.97510061)
\lineto(968.81526367,60.97510061)
\curveto(968.7652678,60.95509995)(968.71526785,60.95009996)(968.66526367,60.96010061)
\curveto(968.60526796,60.97009994)(968.55026802,60.97009994)(968.50026367,60.96010061)
\curveto(968.46026811,60.95009996)(968.41526815,60.95009996)(968.36526367,60.96010061)
\curveto(968.31526825,60.97009994)(968.2652683,60.96509994)(968.21526367,60.94510061)
\curveto(968.14526842,60.92509998)(968.0702685,60.92009999)(967.99026367,60.93010061)
\curveto(967.90026867,60.94009997)(967.81526875,60.94509996)(967.73526367,60.94510061)
\curveto(967.64526892,60.94509996)(967.54526902,60.94009997)(967.43526367,60.93010061)
\curveto(967.31526925,60.92009999)(967.21526935,60.92509998)(967.13526367,60.94510061)
\lineto(966.85026367,60.94510061)
\lineto(966.22026367,60.99010061)
\curveto(966.12027045,61.00009991)(966.02527054,61.0100999)(965.93526367,61.02010061)
\lineto(965.63526367,61.05010061)
\curveto(965.58527098,61.07009984)(965.53527103,61.07509983)(965.48526367,61.06510061)
\curveto(965.42527114,61.06509984)(965.3702712,61.07509983)(965.32026367,61.09510061)
\curveto(965.15027142,61.14509976)(964.98527158,61.18509972)(964.82526367,61.21510061)
\curveto(964.65527191,61.24509966)(964.49527207,61.29509961)(964.34526367,61.36510061)
\curveto(963.88527268,61.55509935)(963.51027306,61.77509913)(963.22026367,62.02510061)
\curveto(962.93027364,62.28509862)(962.68527388,62.64509826)(962.48526367,63.10510061)
\curveto(962.43527413,63.23509767)(962.40027417,63.36509754)(962.38026367,63.49510061)
\curveto(962.36027421,63.63509727)(962.33527423,63.77509713)(962.30526367,63.91510061)
\curveto(962.29527427,63.98509692)(962.29027428,64.05009686)(962.29026367,64.11010061)
\curveto(962.29027428,64.17009674)(962.28527428,64.23509667)(962.27526367,64.30510061)
\curveto(962.25527431,65.13509577)(962.40527416,65.8050951)(962.72526367,66.31510061)
\curveto(963.03527353,66.82509408)(963.47527309,67.2050937)(964.04526367,67.45510061)
\curveto(964.1652724,67.5050934)(964.29027228,67.55009336)(964.42026367,67.59010061)
\curveto(964.55027202,67.63009328)(964.68527188,67.67509323)(964.82526367,67.72510061)
\curveto(964.90527166,67.74509316)(964.99027158,67.76009315)(965.08026367,67.77010061)
\lineto(965.32026367,67.83010061)
\curveto(965.43027114,67.86009305)(965.54027103,67.87509303)(965.65026367,67.87510061)
\curveto(965.76027081,67.88509302)(965.8702707,67.90009301)(965.98026367,67.92010061)
\curveto(966.03027054,67.94009297)(966.07527049,67.94509296)(966.11526367,67.93510061)
\curveto(966.15527041,67.93509297)(966.19527037,67.94009297)(966.23526367,67.95010061)
\curveto(966.28527028,67.96009295)(966.34027023,67.96009295)(966.40026367,67.95010061)
\curveto(966.45027012,67.95009296)(966.50027007,67.95509295)(966.55026367,67.96510061)
\lineto(966.68526367,67.96510061)
\curveto(966.74526982,67.98509292)(966.81526975,67.98509292)(966.89526367,67.96510061)
\curveto(966.9652696,67.95509295)(967.03026954,67.96009295)(967.09026367,67.98010061)
\curveto(967.12026945,67.99009292)(967.16026941,67.99509291)(967.21026367,67.99510061)
\lineto(967.33026367,67.99510061)
\lineto(967.79526367,67.99510061)
\moveto(970.12026367,66.45010061)
\curveto(969.80026677,66.55009436)(969.43526713,66.6100943)(969.02526367,66.63010061)
\curveto(968.61526795,66.65009426)(968.20526836,66.66009425)(967.79526367,66.66010061)
\curveto(967.3652692,66.66009425)(966.94526962,66.65009426)(966.53526367,66.63010061)
\curveto(966.12527044,66.6100943)(965.74027083,66.56509434)(965.38026367,66.49510061)
\curveto(965.02027155,66.42509448)(964.70027187,66.31509459)(964.42026367,66.16510061)
\curveto(964.13027244,66.02509488)(963.89527267,65.83009508)(963.71526367,65.58010061)
\curveto(963.60527296,65.42009549)(963.52527304,65.24009567)(963.47526367,65.04010061)
\curveto(963.41527315,64.84009607)(963.38527318,64.59509631)(963.38526367,64.30510061)
\curveto(963.40527316,64.28509662)(963.41527315,64.25009666)(963.41526367,64.20010061)
\curveto(963.40527316,64.15009676)(963.40527316,64.1100968)(963.41526367,64.08010061)
\curveto(963.43527313,64.00009691)(963.45527311,63.92509698)(963.47526367,63.85510061)
\curveto(963.48527308,63.79509711)(963.50527306,63.73009718)(963.53526367,63.66010061)
\curveto(963.65527291,63.39009752)(963.82527274,63.17009774)(964.04526367,63.00010061)
\curveto(964.25527231,62.84009807)(964.50027207,62.7050982)(964.78026367,62.59510061)
\curveto(964.89027168,62.54509836)(965.01027156,62.5050984)(965.14026367,62.47510061)
\curveto(965.26027131,62.45509845)(965.38527118,62.43009848)(965.51526367,62.40010061)
\curveto(965.565271,62.38009853)(965.62027095,62.37009854)(965.68026367,62.37010061)
\curveto(965.73027084,62.37009854)(965.78027079,62.36509854)(965.83026367,62.35510061)
\curveto(965.92027065,62.34509856)(966.01527055,62.33509857)(966.11526367,62.32510061)
\curveto(966.20527036,62.31509859)(966.30027027,62.3050986)(966.40026367,62.29510061)
\curveto(966.48027009,62.29509861)(966.56527,62.29009862)(966.65526367,62.28010061)
\lineto(966.89526367,62.28010061)
\lineto(967.07526367,62.28010061)
\curveto(967.10526946,62.27009864)(967.14026943,62.26509864)(967.18026367,62.26510061)
\lineto(967.31526367,62.26510061)
\lineto(967.76526367,62.26510061)
\curveto(967.84526872,62.26509864)(967.93026864,62.26009865)(968.02026367,62.25010061)
\curveto(968.10026847,62.25009866)(968.17526839,62.26009865)(968.24526367,62.28010061)
\lineto(968.51526367,62.28010061)
\curveto(968.53526803,62.28009863)(968.565268,62.27509863)(968.60526367,62.26510061)
\curveto(968.63526793,62.26509864)(968.66026791,62.27009864)(968.68026367,62.28010061)
\curveto(968.78026779,62.29009862)(968.88026769,62.29509861)(968.98026367,62.29510061)
\curveto(969.0702675,62.3050986)(969.1702674,62.31509859)(969.28026367,62.32510061)
\curveto(969.40026717,62.35509855)(969.52526704,62.37009854)(969.65526367,62.37010061)
\curveto(969.77526679,62.38009853)(969.89026668,62.4050985)(970.00026367,62.44510061)
\curveto(970.30026627,62.52509838)(970.565266,62.6100983)(970.79526367,62.70010061)
\curveto(971.02526554,62.80009811)(971.24026533,62.94509796)(971.44026367,63.13510061)
\curveto(971.64026493,63.34509756)(971.79026478,63.6100973)(971.89026367,63.93010061)
\curveto(971.91026466,63.97009694)(971.92026465,64.0050969)(971.92026367,64.03510061)
\curveto(971.91026466,64.07509683)(971.91526465,64.12009679)(971.93526367,64.17010061)
\curveto(971.94526462,64.2100967)(971.95526461,64.28009663)(971.96526367,64.38010061)
\curveto(971.97526459,64.49009642)(971.9702646,64.57509633)(971.95026367,64.63510061)
\curveto(971.93026464,64.7050962)(971.92026465,64.77509613)(971.92026367,64.84510061)
\curveto(971.91026466,64.91509599)(971.89526467,64.98009593)(971.87526367,65.04010061)
\curveto(971.81526475,65.24009567)(971.73026484,65.42009549)(971.62026367,65.58010061)
\curveto(971.60026497,65.6100953)(971.58026499,65.63509527)(971.56026367,65.65510061)
\lineto(971.50026367,65.71510061)
\curveto(971.48026509,65.75509515)(971.44026513,65.8050951)(971.38026367,65.86510061)
\curveto(971.24026533,65.96509494)(971.11026546,66.05009486)(970.99026367,66.12010061)
\curveto(970.8702657,66.19009472)(970.72526584,66.26009465)(970.55526367,66.33010061)
\curveto(970.48526608,66.36009455)(970.41526615,66.38009453)(970.34526367,66.39010061)
\curveto(970.27526629,66.4100945)(970.20026637,66.43009448)(970.12026367,66.45010061)
}
}
{
\newrgbcolor{curcolor}{0 0 0}
\pscustom[linestyle=none,fillstyle=solid,fillcolor=curcolor]
{
\newpath
\moveto(962.27526367,73.40470998)
\curveto(962.27527429,73.50470513)(962.28527428,73.59970503)(962.30526367,73.68970998)
\curveto(962.31527425,73.77970485)(962.34527422,73.84470479)(962.39526367,73.88470998)
\curveto(962.47527409,73.94470469)(962.58027399,73.97470466)(962.71026367,73.97470998)
\lineto(963.10026367,73.97470998)
\lineto(964.60026367,73.97470998)
\lineto(970.99026367,73.97470998)
\lineto(972.16026367,73.97470998)
\lineto(972.47526367,73.97470998)
\curveto(972.57526399,73.98470465)(972.65526391,73.96970466)(972.71526367,73.92970998)
\curveto(972.79526377,73.87970475)(972.84526372,73.80470483)(972.86526367,73.70470998)
\curveto(972.87526369,73.61470502)(972.88026369,73.50470513)(972.88026367,73.37470998)
\lineto(972.88026367,73.14970998)
\curveto(972.86026371,73.06970556)(972.84526372,72.99970563)(972.83526367,72.93970998)
\curveto(972.81526375,72.87970575)(972.77526379,72.8297058)(972.71526367,72.78970998)
\curveto(972.65526391,72.74970588)(972.58026399,72.7297059)(972.49026367,72.72970998)
\lineto(972.19026367,72.72970998)
\lineto(971.09526367,72.72970998)
\lineto(965.75526367,72.72970998)
\curveto(965.6652709,72.70970592)(965.59027098,72.69470594)(965.53026367,72.68470998)
\curveto(965.46027111,72.68470595)(965.40027117,72.65470598)(965.35026367,72.59470998)
\curveto(965.30027127,72.52470611)(965.27527129,72.4347062)(965.27526367,72.32470998)
\curveto(965.2652713,72.22470641)(965.26027131,72.11470652)(965.26026367,71.99470998)
\lineto(965.26026367,70.85470998)
\lineto(965.26026367,70.35970998)
\curveto(965.25027132,70.19970843)(965.19027138,70.08970854)(965.08026367,70.02970998)
\curveto(965.05027152,70.00970862)(965.02027155,69.99970863)(964.99026367,69.99970998)
\curveto(964.95027162,69.99970863)(964.90527166,69.99470864)(964.85526367,69.98470998)
\curveto(964.73527183,69.96470867)(964.62527194,69.96970866)(964.52526367,69.99970998)
\curveto(964.42527214,70.03970859)(964.35527221,70.09470854)(964.31526367,70.16470998)
\curveto(964.2652723,70.24470839)(964.24027233,70.36470827)(964.24026367,70.52470998)
\curveto(964.24027233,70.68470795)(964.22527234,70.81970781)(964.19526367,70.92970998)
\curveto(964.18527238,70.97970765)(964.18027239,71.0347076)(964.18026367,71.09470998)
\curveto(964.1702724,71.15470748)(964.15527241,71.21470742)(964.13526367,71.27470998)
\curveto(964.08527248,71.42470721)(964.03527253,71.56970706)(963.98526367,71.70970998)
\curveto(963.92527264,71.84970678)(963.85527271,71.98470665)(963.77526367,72.11470998)
\curveto(963.68527288,72.25470638)(963.58027299,72.37470626)(963.46026367,72.47470998)
\curveto(963.34027323,72.57470606)(963.21027336,72.66970596)(963.07026367,72.75970998)
\curveto(962.9702736,72.81970581)(962.86027371,72.86470577)(962.74026367,72.89470998)
\curveto(962.62027395,72.9347057)(962.51527405,72.98470565)(962.42526367,73.04470998)
\curveto(962.3652742,73.09470554)(962.32527424,73.16470547)(962.30526367,73.25470998)
\curveto(962.29527427,73.27470536)(962.29027428,73.29970533)(962.29026367,73.32970998)
\curveto(962.29027428,73.35970527)(962.28527428,73.38470525)(962.27526367,73.40470998)
}
}
{
\newrgbcolor{curcolor}{0 0 0}
\pscustom[linestyle=none,fillstyle=solid,fillcolor=curcolor]
{
\newpath
\moveto(971.24526367,78.63431936)
\lineto(971.24526367,79.26431936)
\lineto(971.24526367,79.45931936)
\curveto(971.24526532,79.52931683)(971.25526531,79.58931677)(971.27526367,79.63931936)
\curveto(971.31526525,79.70931665)(971.35526521,79.7593166)(971.39526367,79.78931936)
\curveto(971.44526512,79.82931653)(971.51026506,79.84931651)(971.59026367,79.84931936)
\curveto(971.6702649,79.8593165)(971.75526481,79.86431649)(971.84526367,79.86431936)
\lineto(972.56526367,79.86431936)
\curveto(973.04526352,79.86431649)(973.45526311,79.80431655)(973.79526367,79.68431936)
\curveto(974.13526243,79.56431679)(974.41026216,79.36931699)(974.62026367,79.09931936)
\curveto(974.6702619,79.02931733)(974.71526185,78.9593174)(974.75526367,78.88931936)
\curveto(974.80526176,78.82931753)(974.85026172,78.7543176)(974.89026367,78.66431936)
\curveto(974.90026167,78.64431771)(974.91026166,78.61431774)(974.92026367,78.57431936)
\curveto(974.94026163,78.53431782)(974.94526162,78.48931787)(974.93526367,78.43931936)
\curveto(974.90526166,78.34931801)(974.83026174,78.29431806)(974.71026367,78.27431936)
\curveto(974.60026197,78.2543181)(974.50526206,78.26931809)(974.42526367,78.31931936)
\curveto(974.35526221,78.34931801)(974.29026228,78.39431796)(974.23026367,78.45431936)
\curveto(974.18026239,78.52431783)(974.13026244,78.58931777)(974.08026367,78.64931936)
\curveto(974.03026254,78.71931764)(973.95526261,78.77931758)(973.85526367,78.82931936)
\curveto(973.7652628,78.88931747)(973.67526289,78.93931742)(973.58526367,78.97931936)
\curveto(973.55526301,78.99931736)(973.49526307,79.02431733)(973.40526367,79.05431936)
\curveto(973.32526324,79.08431727)(973.25526331,79.08931727)(973.19526367,79.06931936)
\curveto(973.05526351,79.03931732)(972.9652636,78.97931738)(972.92526367,78.88931936)
\curveto(972.89526367,78.80931755)(972.88026369,78.71931764)(972.88026367,78.61931936)
\curveto(972.88026369,78.51931784)(972.85526371,78.43431792)(972.80526367,78.36431936)
\curveto(972.73526383,78.27431808)(972.59526397,78.22931813)(972.38526367,78.22931936)
\lineto(971.83026367,78.22931936)
\lineto(971.60526367,78.22931936)
\curveto(971.52526504,78.23931812)(971.46026511,78.2593181)(971.41026367,78.28931936)
\curveto(971.33026524,78.34931801)(971.28526528,78.41931794)(971.27526367,78.49931936)
\curveto(971.2652653,78.51931784)(971.26026531,78.53931782)(971.26026367,78.55931936)
\curveto(971.26026531,78.58931777)(971.25526531,78.61431774)(971.24526367,78.63431936)
}
}
{
\newrgbcolor{curcolor}{0 0 0}
\pscustom[linestyle=none,fillstyle=solid,fillcolor=curcolor]
{
}
}
{
\newrgbcolor{curcolor}{0 0 0}
\pscustom[linestyle=none,fillstyle=solid,fillcolor=curcolor]
{
\newpath
\moveto(962.27526367,89.26463186)
\curveto(962.2652743,89.95462722)(962.38527418,90.55462662)(962.63526367,91.06463186)
\curveto(962.88527368,91.58462559)(963.22027335,91.9796252)(963.64026367,92.24963186)
\curveto(963.72027285,92.29962488)(963.81027276,92.34462483)(963.91026367,92.38463186)
\curveto(964.00027257,92.42462475)(964.09527247,92.46962471)(964.19526367,92.51963186)
\curveto(964.29527227,92.55962462)(964.39527217,92.58962459)(964.49526367,92.60963186)
\curveto(964.59527197,92.62962455)(964.70027187,92.64962453)(964.81026367,92.66963186)
\curveto(964.86027171,92.68962449)(964.90527166,92.69462448)(964.94526367,92.68463186)
\curveto(964.98527158,92.6746245)(965.03027154,92.6796245)(965.08026367,92.69963186)
\curveto(965.13027144,92.70962447)(965.21527135,92.71462446)(965.33526367,92.71463186)
\curveto(965.44527112,92.71462446)(965.53027104,92.70962447)(965.59026367,92.69963186)
\curveto(965.65027092,92.6796245)(965.71027086,92.66962451)(965.77026367,92.66963186)
\curveto(965.83027074,92.6796245)(965.89027068,92.6746245)(965.95026367,92.65463186)
\curveto(966.09027048,92.61462456)(966.22527034,92.5796246)(966.35526367,92.54963186)
\curveto(966.48527008,92.51962466)(966.61026996,92.4796247)(966.73026367,92.42963186)
\curveto(966.8702697,92.36962481)(966.99526957,92.29962488)(967.10526367,92.21963186)
\curveto(967.21526935,92.14962503)(967.32526924,92.0746251)(967.43526367,91.99463186)
\lineto(967.49526367,91.93463186)
\curveto(967.51526905,91.92462525)(967.53526903,91.90962527)(967.55526367,91.88963186)
\curveto(967.71526885,91.76962541)(967.86026871,91.63462554)(967.99026367,91.48463186)
\curveto(968.12026845,91.33462584)(968.24526832,91.174626)(968.36526367,91.00463186)
\curveto(968.58526798,90.69462648)(968.79026778,90.39962678)(968.98026367,90.11963186)
\curveto(969.12026745,89.88962729)(969.25526731,89.65962752)(969.38526367,89.42963186)
\curveto(969.51526705,89.20962797)(969.65026692,88.98962819)(969.79026367,88.76963186)
\curveto(969.96026661,88.51962866)(970.14026643,88.2796289)(970.33026367,88.04963186)
\curveto(970.52026605,87.82962935)(970.74526582,87.63962954)(971.00526367,87.47963186)
\curveto(971.0652655,87.43962974)(971.12526544,87.40462977)(971.18526367,87.37463186)
\curveto(971.23526533,87.34462983)(971.30026527,87.31462986)(971.38026367,87.28463186)
\curveto(971.45026512,87.26462991)(971.51026506,87.25962992)(971.56026367,87.26963186)
\curveto(971.63026494,87.28962989)(971.68526488,87.32462985)(971.72526367,87.37463186)
\curveto(971.75526481,87.42462975)(971.77526479,87.48462969)(971.78526367,87.55463186)
\lineto(971.78526367,87.79463186)
\lineto(971.78526367,88.54463186)
\lineto(971.78526367,91.34963186)
\lineto(971.78526367,92.00963186)
\curveto(971.78526478,92.09962508)(971.79026478,92.18462499)(971.80026367,92.26463186)
\curveto(971.80026477,92.34462483)(971.82026475,92.40962477)(971.86026367,92.45963186)
\curveto(971.90026467,92.50962467)(971.97526459,92.54962463)(972.08526367,92.57963186)
\curveto(972.18526438,92.61962456)(972.28526428,92.62962455)(972.38526367,92.60963186)
\lineto(972.52026367,92.60963186)
\curveto(972.59026398,92.58962459)(972.65026392,92.56962461)(972.70026367,92.54963186)
\curveto(972.75026382,92.52962465)(972.79026378,92.49462468)(972.82026367,92.44463186)
\curveto(972.86026371,92.39462478)(972.88026369,92.32462485)(972.88026367,92.23463186)
\lineto(972.88026367,91.96463186)
\lineto(972.88026367,91.06463186)
\lineto(972.88026367,87.55463186)
\lineto(972.88026367,86.48963186)
\curveto(972.88026369,86.40963077)(972.88526368,86.31963086)(972.89526367,86.21963186)
\curveto(972.89526367,86.11963106)(972.88526368,86.03463114)(972.86526367,85.96463186)
\curveto(972.79526377,85.75463142)(972.61526395,85.68963149)(972.32526367,85.76963186)
\curveto(972.28526428,85.7796314)(972.25026432,85.7796314)(972.22026367,85.76963186)
\curveto(972.18026439,85.76963141)(972.13526443,85.7796314)(972.08526367,85.79963186)
\curveto(972.00526456,85.81963136)(971.92026465,85.83963134)(971.83026367,85.85963186)
\curveto(971.74026483,85.8796313)(971.65526491,85.90463127)(971.57526367,85.93463186)
\curveto(971.08526548,86.09463108)(970.6702659,86.29463088)(970.33026367,86.53463186)
\curveto(970.08026649,86.71463046)(969.85526671,86.91963026)(969.65526367,87.14963186)
\curveto(969.44526712,87.3796298)(969.25026732,87.61962956)(969.07026367,87.86963186)
\curveto(968.89026768,88.12962905)(968.72026785,88.39462878)(968.56026367,88.66463186)
\curveto(968.39026818,88.94462823)(968.21526835,89.21462796)(968.03526367,89.47463186)
\curveto(967.95526861,89.58462759)(967.88026869,89.68962749)(967.81026367,89.78963186)
\curveto(967.74026883,89.89962728)(967.6652689,90.00962717)(967.58526367,90.11963186)
\curveto(967.55526901,90.15962702)(967.52526904,90.19462698)(967.49526367,90.22463186)
\curveto(967.45526911,90.26462691)(967.42526914,90.30462687)(967.40526367,90.34463186)
\curveto(967.29526927,90.48462669)(967.1702694,90.60962657)(967.03026367,90.71963186)
\curveto(967.00026957,90.73962644)(966.97526959,90.76462641)(966.95526367,90.79463186)
\curveto(966.92526964,90.82462635)(966.89526967,90.84962633)(966.86526367,90.86963186)
\curveto(966.7652698,90.94962623)(966.6652699,91.01462616)(966.56526367,91.06463186)
\curveto(966.4652701,91.12462605)(966.35527021,91.179626)(966.23526367,91.22963186)
\curveto(966.1652704,91.25962592)(966.09027048,91.2796259)(966.01026367,91.28963186)
\lineto(965.77026367,91.34963186)
\lineto(965.68026367,91.34963186)
\curveto(965.65027092,91.35962582)(965.62027095,91.36462581)(965.59026367,91.36463186)
\curveto(965.52027105,91.38462579)(965.42527114,91.38962579)(965.30526367,91.37963186)
\curveto(965.17527139,91.3796258)(965.07527149,91.36962581)(965.00526367,91.34963186)
\curveto(964.92527164,91.32962585)(964.85027172,91.30962587)(964.78026367,91.28963186)
\curveto(964.70027187,91.2796259)(964.62027195,91.25962592)(964.54026367,91.22963186)
\curveto(964.30027227,91.11962606)(964.10027247,90.96962621)(963.94026367,90.77963186)
\curveto(963.7702728,90.59962658)(963.63027294,90.3796268)(963.52026367,90.11963186)
\curveto(963.50027307,90.04962713)(963.48527308,89.9796272)(963.47526367,89.90963186)
\curveto(963.45527311,89.83962734)(963.43527313,89.76462741)(963.41526367,89.68463186)
\curveto(963.39527317,89.60462757)(963.38527318,89.49462768)(963.38526367,89.35463186)
\curveto(963.38527318,89.22462795)(963.39527317,89.11962806)(963.41526367,89.03963186)
\curveto(963.42527314,88.9796282)(963.43027314,88.92462825)(963.43026367,88.87463186)
\curveto(963.43027314,88.82462835)(963.44027313,88.7746284)(963.46026367,88.72463186)
\curveto(963.50027307,88.62462855)(963.54027303,88.52962865)(963.58026367,88.43963186)
\curveto(963.62027295,88.35962882)(963.6652729,88.2796289)(963.71526367,88.19963186)
\curveto(963.73527283,88.16962901)(963.76027281,88.13962904)(963.79026367,88.10963186)
\curveto(963.82027275,88.08962909)(963.84527272,88.06462911)(963.86526367,88.03463186)
\lineto(963.94026367,87.95963186)
\curveto(963.96027261,87.92962925)(963.98027259,87.90462927)(964.00026367,87.88463186)
\lineto(964.21026367,87.73463186)
\curveto(964.2702723,87.69462948)(964.33527223,87.64962953)(964.40526367,87.59963186)
\curveto(964.49527207,87.53962964)(964.60027197,87.48962969)(964.72026367,87.44963186)
\curveto(964.83027174,87.41962976)(964.94027163,87.38462979)(965.05026367,87.34463186)
\curveto(965.16027141,87.30462987)(965.30527126,87.2796299)(965.48526367,87.26963186)
\curveto(965.65527091,87.25962992)(965.78027079,87.22962995)(965.86026367,87.17963186)
\curveto(965.94027063,87.12963005)(965.98527058,87.05463012)(965.99526367,86.95463186)
\curveto(966.00527056,86.85463032)(966.01027056,86.74463043)(966.01026367,86.62463186)
\curveto(966.01027056,86.58463059)(966.01527055,86.54463063)(966.02526367,86.50463186)
\curveto(966.02527054,86.46463071)(966.02027055,86.42963075)(966.01026367,86.39963186)
\curveto(965.99027058,86.34963083)(965.98027059,86.29963088)(965.98026367,86.24963186)
\curveto(965.98027059,86.20963097)(965.9702706,86.16963101)(965.95026367,86.12963186)
\curveto(965.89027068,86.03963114)(965.75527081,85.99463118)(965.54526367,85.99463186)
\lineto(965.42526367,85.99463186)
\curveto(965.3652712,86.00463117)(965.30527126,86.00963117)(965.24526367,86.00963186)
\curveto(965.17527139,86.01963116)(965.11027146,86.02963115)(965.05026367,86.03963186)
\curveto(964.94027163,86.05963112)(964.84027173,86.0796311)(964.75026367,86.09963186)
\curveto(964.65027192,86.11963106)(964.55527201,86.14963103)(964.46526367,86.18963186)
\curveto(964.39527217,86.20963097)(964.33527223,86.22963095)(964.28526367,86.24963186)
\lineto(964.10526367,86.30963186)
\curveto(963.84527272,86.42963075)(963.60027297,86.58463059)(963.37026367,86.77463186)
\curveto(963.14027343,86.9746302)(962.95527361,87.18962999)(962.81526367,87.41963186)
\curveto(962.73527383,87.52962965)(962.6702739,87.64462953)(962.62026367,87.76463186)
\lineto(962.47026367,88.15463186)
\curveto(962.42027415,88.26462891)(962.39027418,88.3796288)(962.38026367,88.49963186)
\curveto(962.36027421,88.61962856)(962.33527423,88.74462843)(962.30526367,88.87463186)
\curveto(962.30527426,88.94462823)(962.30527426,89.00962817)(962.30526367,89.06963186)
\curveto(962.29527427,89.12962805)(962.28527428,89.19462798)(962.27526367,89.26463186)
}
}
{
\newrgbcolor{curcolor}{0 0 0}
\pscustom[linestyle=none,fillstyle=solid,fillcolor=curcolor]
{
\newpath
\moveto(967.79526367,101.36424123)
\lineto(968.05026367,101.36424123)
\curveto(968.13026844,101.37423353)(968.20526836,101.36923353)(968.27526367,101.34924123)
\lineto(968.51526367,101.34924123)
\lineto(968.68026367,101.34924123)
\curveto(968.78026779,101.32923357)(968.88526768,101.31923358)(968.99526367,101.31924123)
\curveto(969.09526747,101.31923358)(969.19526737,101.30923359)(969.29526367,101.28924123)
\lineto(969.44526367,101.28924123)
\curveto(969.58526698,101.25923364)(969.72526684,101.23923366)(969.86526367,101.22924123)
\curveto(969.99526657,101.21923368)(970.12526644,101.19423371)(970.25526367,101.15424123)
\curveto(970.33526623,101.13423377)(970.42026615,101.11423379)(970.51026367,101.09424123)
\lineto(970.75026367,101.03424123)
\lineto(971.05026367,100.91424123)
\curveto(971.14026543,100.88423402)(971.23026534,100.84923405)(971.32026367,100.80924123)
\curveto(971.54026503,100.70923419)(971.75526481,100.57423433)(971.96526367,100.40424123)
\curveto(972.17526439,100.24423466)(972.34526422,100.06923483)(972.47526367,99.87924123)
\curveto(972.51526405,99.82923507)(972.55526401,99.76923513)(972.59526367,99.69924123)
\curveto(972.62526394,99.63923526)(972.66026391,99.57923532)(972.70026367,99.51924123)
\curveto(972.75026382,99.43923546)(972.79026378,99.34423556)(972.82026367,99.23424123)
\curveto(972.85026372,99.12423578)(972.88026369,99.01923588)(972.91026367,98.91924123)
\curveto(972.95026362,98.80923609)(972.97526359,98.6992362)(972.98526367,98.58924123)
\curveto(972.99526357,98.47923642)(973.01026356,98.36423654)(973.03026367,98.24424123)
\curveto(973.04026353,98.2042367)(973.04026353,98.15923674)(973.03026367,98.10924123)
\curveto(973.03026354,98.06923683)(973.03526353,98.02923687)(973.04526367,97.98924123)
\curveto(973.05526351,97.94923695)(973.06026351,97.89423701)(973.06026367,97.82424123)
\curveto(973.06026351,97.75423715)(973.05526351,97.7042372)(973.04526367,97.67424123)
\curveto(973.02526354,97.62423728)(973.02026355,97.57923732)(973.03026367,97.53924123)
\curveto(973.04026353,97.4992374)(973.04026353,97.46423744)(973.03026367,97.43424123)
\lineto(973.03026367,97.34424123)
\curveto(973.01026356,97.28423762)(972.99526357,97.21923768)(972.98526367,97.14924123)
\curveto(972.98526358,97.08923781)(972.98026359,97.02423788)(972.97026367,96.95424123)
\curveto(972.92026365,96.78423812)(972.8702637,96.62423828)(972.82026367,96.47424123)
\curveto(972.7702638,96.32423858)(972.70526386,96.17923872)(972.62526367,96.03924123)
\curveto(972.58526398,95.98923891)(972.55526401,95.93423897)(972.53526367,95.87424123)
\curveto(972.50526406,95.82423908)(972.4702641,95.77423913)(972.43026367,95.72424123)
\curveto(972.25026432,95.48423942)(972.03026454,95.28423962)(971.77026367,95.12424123)
\curveto(971.51026506,94.96423994)(971.22526534,94.82424008)(970.91526367,94.70424123)
\curveto(970.77526579,94.64424026)(970.63526593,94.5992403)(970.49526367,94.56924123)
\curveto(970.34526622,94.53924036)(970.19026638,94.5042404)(970.03026367,94.46424123)
\curveto(969.92026665,94.44424046)(969.81026676,94.42924047)(969.70026367,94.41924123)
\curveto(969.59026698,94.40924049)(969.48026709,94.39424051)(969.37026367,94.37424123)
\curveto(969.33026724,94.36424054)(969.29026728,94.35924054)(969.25026367,94.35924123)
\curveto(969.21026736,94.36924053)(969.1702674,94.36924053)(969.13026367,94.35924123)
\curveto(969.08026749,94.34924055)(969.03026754,94.34424056)(968.98026367,94.34424123)
\lineto(968.81526367,94.34424123)
\curveto(968.7652678,94.32424058)(968.71526785,94.31924058)(968.66526367,94.32924123)
\curveto(968.60526796,94.33924056)(968.55026802,94.33924056)(968.50026367,94.32924123)
\curveto(968.46026811,94.31924058)(968.41526815,94.31924058)(968.36526367,94.32924123)
\curveto(968.31526825,94.33924056)(968.2652683,94.33424057)(968.21526367,94.31424123)
\curveto(968.14526842,94.29424061)(968.0702685,94.28924061)(967.99026367,94.29924123)
\curveto(967.90026867,94.30924059)(967.81526875,94.31424059)(967.73526367,94.31424123)
\curveto(967.64526892,94.31424059)(967.54526902,94.30924059)(967.43526367,94.29924123)
\curveto(967.31526925,94.28924061)(967.21526935,94.29424061)(967.13526367,94.31424123)
\lineto(966.85026367,94.31424123)
\lineto(966.22026367,94.35924123)
\curveto(966.12027045,94.36924053)(966.02527054,94.37924052)(965.93526367,94.38924123)
\lineto(965.63526367,94.41924123)
\curveto(965.58527098,94.43924046)(965.53527103,94.44424046)(965.48526367,94.43424123)
\curveto(965.42527114,94.43424047)(965.3702712,94.44424046)(965.32026367,94.46424123)
\curveto(965.15027142,94.51424039)(964.98527158,94.55424035)(964.82526367,94.58424123)
\curveto(964.65527191,94.61424029)(964.49527207,94.66424024)(964.34526367,94.73424123)
\curveto(963.88527268,94.92423998)(963.51027306,95.14423976)(963.22026367,95.39424123)
\curveto(962.93027364,95.65423925)(962.68527388,96.01423889)(962.48526367,96.47424123)
\curveto(962.43527413,96.6042383)(962.40027417,96.73423817)(962.38026367,96.86424123)
\curveto(962.36027421,97.0042379)(962.33527423,97.14423776)(962.30526367,97.28424123)
\curveto(962.29527427,97.35423755)(962.29027428,97.41923748)(962.29026367,97.47924123)
\curveto(962.29027428,97.53923736)(962.28527428,97.6042373)(962.27526367,97.67424123)
\curveto(962.25527431,98.5042364)(962.40527416,99.17423573)(962.72526367,99.68424123)
\curveto(963.03527353,100.19423471)(963.47527309,100.57423433)(964.04526367,100.82424123)
\curveto(964.1652724,100.87423403)(964.29027228,100.91923398)(964.42026367,100.95924123)
\curveto(964.55027202,100.9992339)(964.68527188,101.04423386)(964.82526367,101.09424123)
\curveto(964.90527166,101.11423379)(964.99027158,101.12923377)(965.08026367,101.13924123)
\lineto(965.32026367,101.19924123)
\curveto(965.43027114,101.22923367)(965.54027103,101.24423366)(965.65026367,101.24424123)
\curveto(965.76027081,101.25423365)(965.8702707,101.26923363)(965.98026367,101.28924123)
\curveto(966.03027054,101.30923359)(966.07527049,101.31423359)(966.11526367,101.30424123)
\curveto(966.15527041,101.3042336)(966.19527037,101.30923359)(966.23526367,101.31924123)
\curveto(966.28527028,101.32923357)(966.34027023,101.32923357)(966.40026367,101.31924123)
\curveto(966.45027012,101.31923358)(966.50027007,101.32423358)(966.55026367,101.33424123)
\lineto(966.68526367,101.33424123)
\curveto(966.74526982,101.35423355)(966.81526975,101.35423355)(966.89526367,101.33424123)
\curveto(966.9652696,101.32423358)(967.03026954,101.32923357)(967.09026367,101.34924123)
\curveto(967.12026945,101.35923354)(967.16026941,101.36423354)(967.21026367,101.36424123)
\lineto(967.33026367,101.36424123)
\lineto(967.79526367,101.36424123)
\moveto(970.12026367,99.81924123)
\curveto(969.80026677,99.91923498)(969.43526713,99.97923492)(969.02526367,99.99924123)
\curveto(968.61526795,100.01923488)(968.20526836,100.02923487)(967.79526367,100.02924123)
\curveto(967.3652692,100.02923487)(966.94526962,100.01923488)(966.53526367,99.99924123)
\curveto(966.12527044,99.97923492)(965.74027083,99.93423497)(965.38026367,99.86424123)
\curveto(965.02027155,99.79423511)(964.70027187,99.68423522)(964.42026367,99.53424123)
\curveto(964.13027244,99.39423551)(963.89527267,99.1992357)(963.71526367,98.94924123)
\curveto(963.60527296,98.78923611)(963.52527304,98.60923629)(963.47526367,98.40924123)
\curveto(963.41527315,98.20923669)(963.38527318,97.96423694)(963.38526367,97.67424123)
\curveto(963.40527316,97.65423725)(963.41527315,97.61923728)(963.41526367,97.56924123)
\curveto(963.40527316,97.51923738)(963.40527316,97.47923742)(963.41526367,97.44924123)
\curveto(963.43527313,97.36923753)(963.45527311,97.29423761)(963.47526367,97.22424123)
\curveto(963.48527308,97.16423774)(963.50527306,97.0992378)(963.53526367,97.02924123)
\curveto(963.65527291,96.75923814)(963.82527274,96.53923836)(964.04526367,96.36924123)
\curveto(964.25527231,96.20923869)(964.50027207,96.07423883)(964.78026367,95.96424123)
\curveto(964.89027168,95.91423899)(965.01027156,95.87423903)(965.14026367,95.84424123)
\curveto(965.26027131,95.82423908)(965.38527118,95.7992391)(965.51526367,95.76924123)
\curveto(965.565271,95.74923915)(965.62027095,95.73923916)(965.68026367,95.73924123)
\curveto(965.73027084,95.73923916)(965.78027079,95.73423917)(965.83026367,95.72424123)
\curveto(965.92027065,95.71423919)(966.01527055,95.7042392)(966.11526367,95.69424123)
\curveto(966.20527036,95.68423922)(966.30027027,95.67423923)(966.40026367,95.66424123)
\curveto(966.48027009,95.66423924)(966.56527,95.65923924)(966.65526367,95.64924123)
\lineto(966.89526367,95.64924123)
\lineto(967.07526367,95.64924123)
\curveto(967.10526946,95.63923926)(967.14026943,95.63423927)(967.18026367,95.63424123)
\lineto(967.31526367,95.63424123)
\lineto(967.76526367,95.63424123)
\curveto(967.84526872,95.63423927)(967.93026864,95.62923927)(968.02026367,95.61924123)
\curveto(968.10026847,95.61923928)(968.17526839,95.62923927)(968.24526367,95.64924123)
\lineto(968.51526367,95.64924123)
\curveto(968.53526803,95.64923925)(968.565268,95.64423926)(968.60526367,95.63424123)
\curveto(968.63526793,95.63423927)(968.66026791,95.63923926)(968.68026367,95.64924123)
\curveto(968.78026779,95.65923924)(968.88026769,95.66423924)(968.98026367,95.66424123)
\curveto(969.0702675,95.67423923)(969.1702674,95.68423922)(969.28026367,95.69424123)
\curveto(969.40026717,95.72423918)(969.52526704,95.73923916)(969.65526367,95.73924123)
\curveto(969.77526679,95.74923915)(969.89026668,95.77423913)(970.00026367,95.81424123)
\curveto(970.30026627,95.89423901)(970.565266,95.97923892)(970.79526367,96.06924123)
\curveto(971.02526554,96.16923873)(971.24026533,96.31423859)(971.44026367,96.50424123)
\curveto(971.64026493,96.71423819)(971.79026478,96.97923792)(971.89026367,97.29924123)
\curveto(971.91026466,97.33923756)(971.92026465,97.37423753)(971.92026367,97.40424123)
\curveto(971.91026466,97.44423746)(971.91526465,97.48923741)(971.93526367,97.53924123)
\curveto(971.94526462,97.57923732)(971.95526461,97.64923725)(971.96526367,97.74924123)
\curveto(971.97526459,97.85923704)(971.9702646,97.94423696)(971.95026367,98.00424123)
\curveto(971.93026464,98.07423683)(971.92026465,98.14423676)(971.92026367,98.21424123)
\curveto(971.91026466,98.28423662)(971.89526467,98.34923655)(971.87526367,98.40924123)
\curveto(971.81526475,98.60923629)(971.73026484,98.78923611)(971.62026367,98.94924123)
\curveto(971.60026497,98.97923592)(971.58026499,99.0042359)(971.56026367,99.02424123)
\lineto(971.50026367,99.08424123)
\curveto(971.48026509,99.12423578)(971.44026513,99.17423573)(971.38026367,99.23424123)
\curveto(971.24026533,99.33423557)(971.11026546,99.41923548)(970.99026367,99.48924123)
\curveto(970.8702657,99.55923534)(970.72526584,99.62923527)(970.55526367,99.69924123)
\curveto(970.48526608,99.72923517)(970.41526615,99.74923515)(970.34526367,99.75924123)
\curveto(970.27526629,99.77923512)(970.20026637,99.7992351)(970.12026367,99.81924123)
}
}
{
\newrgbcolor{curcolor}{0 0 0}
\pscustom[linestyle=none,fillstyle=solid,fillcolor=curcolor]
{
\newpath
\moveto(962.27526367,106.77385061)
\curveto(962.27527429,106.87384575)(962.28527428,106.96884566)(962.30526367,107.05885061)
\curveto(962.31527425,107.14884548)(962.34527422,107.21384541)(962.39526367,107.25385061)
\curveto(962.47527409,107.31384531)(962.58027399,107.34384528)(962.71026367,107.34385061)
\lineto(963.10026367,107.34385061)
\lineto(964.60026367,107.34385061)
\lineto(970.99026367,107.34385061)
\lineto(972.16026367,107.34385061)
\lineto(972.47526367,107.34385061)
\curveto(972.57526399,107.35384527)(972.65526391,107.33884529)(972.71526367,107.29885061)
\curveto(972.79526377,107.24884538)(972.84526372,107.17384545)(972.86526367,107.07385061)
\curveto(972.87526369,106.98384564)(972.88026369,106.87384575)(972.88026367,106.74385061)
\lineto(972.88026367,106.51885061)
\curveto(972.86026371,106.43884619)(972.84526372,106.36884626)(972.83526367,106.30885061)
\curveto(972.81526375,106.24884638)(972.77526379,106.19884643)(972.71526367,106.15885061)
\curveto(972.65526391,106.11884651)(972.58026399,106.09884653)(972.49026367,106.09885061)
\lineto(972.19026367,106.09885061)
\lineto(971.09526367,106.09885061)
\lineto(965.75526367,106.09885061)
\curveto(965.6652709,106.07884655)(965.59027098,106.06384656)(965.53026367,106.05385061)
\curveto(965.46027111,106.05384657)(965.40027117,106.0238466)(965.35026367,105.96385061)
\curveto(965.30027127,105.89384673)(965.27527129,105.80384682)(965.27526367,105.69385061)
\curveto(965.2652713,105.59384703)(965.26027131,105.48384714)(965.26026367,105.36385061)
\lineto(965.26026367,104.22385061)
\lineto(965.26026367,103.72885061)
\curveto(965.25027132,103.56884906)(965.19027138,103.45884917)(965.08026367,103.39885061)
\curveto(965.05027152,103.37884925)(965.02027155,103.36884926)(964.99026367,103.36885061)
\curveto(964.95027162,103.36884926)(964.90527166,103.36384926)(964.85526367,103.35385061)
\curveto(964.73527183,103.33384929)(964.62527194,103.33884929)(964.52526367,103.36885061)
\curveto(964.42527214,103.40884922)(964.35527221,103.46384916)(964.31526367,103.53385061)
\curveto(964.2652723,103.61384901)(964.24027233,103.73384889)(964.24026367,103.89385061)
\curveto(964.24027233,104.05384857)(964.22527234,104.18884844)(964.19526367,104.29885061)
\curveto(964.18527238,104.34884828)(964.18027239,104.40384822)(964.18026367,104.46385061)
\curveto(964.1702724,104.5238481)(964.15527241,104.58384804)(964.13526367,104.64385061)
\curveto(964.08527248,104.79384783)(964.03527253,104.93884769)(963.98526367,105.07885061)
\curveto(963.92527264,105.21884741)(963.85527271,105.35384727)(963.77526367,105.48385061)
\curveto(963.68527288,105.623847)(963.58027299,105.74384688)(963.46026367,105.84385061)
\curveto(963.34027323,105.94384668)(963.21027336,106.03884659)(963.07026367,106.12885061)
\curveto(962.9702736,106.18884644)(962.86027371,106.23384639)(962.74026367,106.26385061)
\curveto(962.62027395,106.30384632)(962.51527405,106.35384627)(962.42526367,106.41385061)
\curveto(962.3652742,106.46384616)(962.32527424,106.53384609)(962.30526367,106.62385061)
\curveto(962.29527427,106.64384598)(962.29027428,106.66884596)(962.29026367,106.69885061)
\curveto(962.29027428,106.7288459)(962.28527428,106.75384587)(962.27526367,106.77385061)
}
}
{
\newrgbcolor{curcolor}{0 0 0}
\pscustom[linestyle=none,fillstyle=solid,fillcolor=curcolor]
{
\newpath
\moveto(962.27526367,115.12345998)
\curveto(962.27527429,115.22345513)(962.28527428,115.31845503)(962.30526367,115.40845998)
\curveto(962.31527425,115.49845485)(962.34527422,115.56345479)(962.39526367,115.60345998)
\curveto(962.47527409,115.66345469)(962.58027399,115.69345466)(962.71026367,115.69345998)
\lineto(963.10026367,115.69345998)
\lineto(964.60026367,115.69345998)
\lineto(970.99026367,115.69345998)
\lineto(972.16026367,115.69345998)
\lineto(972.47526367,115.69345998)
\curveto(972.57526399,115.70345465)(972.65526391,115.68845466)(972.71526367,115.64845998)
\curveto(972.79526377,115.59845475)(972.84526372,115.52345483)(972.86526367,115.42345998)
\curveto(972.87526369,115.33345502)(972.88026369,115.22345513)(972.88026367,115.09345998)
\lineto(972.88026367,114.86845998)
\curveto(972.86026371,114.78845556)(972.84526372,114.71845563)(972.83526367,114.65845998)
\curveto(972.81526375,114.59845575)(972.77526379,114.5484558)(972.71526367,114.50845998)
\curveto(972.65526391,114.46845588)(972.58026399,114.4484559)(972.49026367,114.44845998)
\lineto(972.19026367,114.44845998)
\lineto(971.09526367,114.44845998)
\lineto(965.75526367,114.44845998)
\curveto(965.6652709,114.42845592)(965.59027098,114.41345594)(965.53026367,114.40345998)
\curveto(965.46027111,114.40345595)(965.40027117,114.37345598)(965.35026367,114.31345998)
\curveto(965.30027127,114.24345611)(965.27527129,114.1534562)(965.27526367,114.04345998)
\curveto(965.2652713,113.94345641)(965.26027131,113.83345652)(965.26026367,113.71345998)
\lineto(965.26026367,112.57345998)
\lineto(965.26026367,112.07845998)
\curveto(965.25027132,111.91845843)(965.19027138,111.80845854)(965.08026367,111.74845998)
\curveto(965.05027152,111.72845862)(965.02027155,111.71845863)(964.99026367,111.71845998)
\curveto(964.95027162,111.71845863)(964.90527166,111.71345864)(964.85526367,111.70345998)
\curveto(964.73527183,111.68345867)(964.62527194,111.68845866)(964.52526367,111.71845998)
\curveto(964.42527214,111.75845859)(964.35527221,111.81345854)(964.31526367,111.88345998)
\curveto(964.2652723,111.96345839)(964.24027233,112.08345827)(964.24026367,112.24345998)
\curveto(964.24027233,112.40345795)(964.22527234,112.53845781)(964.19526367,112.64845998)
\curveto(964.18527238,112.69845765)(964.18027239,112.7534576)(964.18026367,112.81345998)
\curveto(964.1702724,112.87345748)(964.15527241,112.93345742)(964.13526367,112.99345998)
\curveto(964.08527248,113.14345721)(964.03527253,113.28845706)(963.98526367,113.42845998)
\curveto(963.92527264,113.56845678)(963.85527271,113.70345665)(963.77526367,113.83345998)
\curveto(963.68527288,113.97345638)(963.58027299,114.09345626)(963.46026367,114.19345998)
\curveto(963.34027323,114.29345606)(963.21027336,114.38845596)(963.07026367,114.47845998)
\curveto(962.9702736,114.53845581)(962.86027371,114.58345577)(962.74026367,114.61345998)
\curveto(962.62027395,114.6534557)(962.51527405,114.70345565)(962.42526367,114.76345998)
\curveto(962.3652742,114.81345554)(962.32527424,114.88345547)(962.30526367,114.97345998)
\curveto(962.29527427,114.99345536)(962.29027428,115.01845533)(962.29026367,115.04845998)
\curveto(962.29027428,115.07845527)(962.28527428,115.10345525)(962.27526367,115.12345998)
}
}
{
\newrgbcolor{curcolor}{0 0 0}
\pscustom[linestyle=none,fillstyle=solid,fillcolor=curcolor]
{
\newpath
\moveto(994.12156494,38.71181936)
\curveto(994.17156569,38.73180981)(994.23156563,38.75680979)(994.30156494,38.78681936)
\curveto(994.37156549,38.81680973)(994.44656541,38.83680971)(994.52656494,38.84681936)
\curveto(994.59656526,38.86680968)(994.66656519,38.86680968)(994.73656494,38.84681936)
\curveto(994.79656506,38.83680971)(994.84156502,38.79680975)(994.87156494,38.72681936)
\curveto(994.89156497,38.67680987)(994.90156496,38.61680993)(994.90156494,38.54681936)
\lineto(994.90156494,38.33681936)
\lineto(994.90156494,37.88681936)
\curveto(994.90156496,37.73681081)(994.87656498,37.61681093)(994.82656494,37.52681936)
\curveto(994.76656509,37.42681112)(994.6615652,37.35181119)(994.51156494,37.30181936)
\curveto(994.3615655,37.26181128)(994.22656563,37.21681133)(994.10656494,37.16681936)
\curveto(993.84656601,37.05681149)(993.57656628,36.95681159)(993.29656494,36.86681936)
\curveto(993.01656684,36.77681177)(992.74156712,36.67681187)(992.47156494,36.56681936)
\curveto(992.38156748,36.53681201)(992.29656756,36.50681204)(992.21656494,36.47681936)
\curveto(992.13656772,36.45681209)(992.0615678,36.42681212)(991.99156494,36.38681936)
\curveto(991.92156794,36.35681219)(991.861568,36.31181223)(991.81156494,36.25181936)
\curveto(991.7615681,36.19181235)(991.72156814,36.11181243)(991.69156494,36.01181936)
\curveto(991.67156819,35.96181258)(991.66656819,35.90181264)(991.67656494,35.83181936)
\lineto(991.67656494,35.63681936)
\lineto(991.67656494,32.80181936)
\lineto(991.67656494,32.50181936)
\curveto(991.66656819,32.39181615)(991.66656819,32.28681626)(991.67656494,32.18681936)
\curveto(991.68656817,32.08681646)(991.70156816,31.99181655)(991.72156494,31.90181936)
\curveto(991.74156812,31.82181672)(991.78156808,31.76181678)(991.84156494,31.72181936)
\curveto(991.94156792,31.6418169)(992.0565678,31.58181696)(992.18656494,31.54181936)
\curveto(992.30656755,31.51181703)(992.43156743,31.47181707)(992.56156494,31.42181936)
\curveto(992.79156707,31.32181722)(993.03156683,31.22681732)(993.28156494,31.13681936)
\curveto(993.53156633,31.05681749)(993.77156609,30.96681758)(994.00156494,30.86681936)
\curveto(994.0615658,30.8468177)(994.13156573,30.82181772)(994.21156494,30.79181936)
\curveto(994.28156558,30.77181777)(994.3565655,30.7468178)(994.43656494,30.71681936)
\curveto(994.51656534,30.68681786)(994.59156527,30.65181789)(994.66156494,30.61181936)
\curveto(994.72156514,30.58181796)(994.76656509,30.546818)(994.79656494,30.50681936)
\curveto(994.856565,30.42681812)(994.89156497,30.31681823)(994.90156494,30.17681936)
\lineto(994.90156494,29.75681936)
\lineto(994.90156494,29.51681936)
\curveto(994.89156497,29.4468191)(994.86656499,29.38681916)(994.82656494,29.33681936)
\curveto(994.79656506,29.28681926)(994.75156511,29.25681929)(994.69156494,29.24681936)
\curveto(994.63156523,29.2468193)(994.57156529,29.25181929)(994.51156494,29.26181936)
\curveto(994.44156542,29.28181926)(994.37656548,29.30181924)(994.31656494,29.32181936)
\curveto(994.24656561,29.35181919)(994.19656566,29.37681917)(994.16656494,29.39681936)
\curveto(993.84656601,29.53681901)(993.53156633,29.66181888)(993.22156494,29.77181936)
\curveto(992.90156696,29.88181866)(992.58156728,30.00181854)(992.26156494,30.13181936)
\curveto(992.04156782,30.22181832)(991.82656803,30.30681824)(991.61656494,30.38681936)
\curveto(991.39656846,30.46681808)(991.17656868,30.55181799)(990.95656494,30.64181936)
\curveto(990.23656962,30.9418176)(989.51157035,31.22681732)(988.78156494,31.49681936)
\curveto(988.04157182,31.76681678)(987.30657255,32.05181649)(986.57656494,32.35181936)
\curveto(986.31657354,32.46181608)(986.05157381,32.56181598)(985.78156494,32.65181936)
\curveto(985.51157435,32.75181579)(985.24657461,32.85681569)(984.98656494,32.96681936)
\curveto(984.87657498,33.01681553)(984.7565751,33.06181548)(984.62656494,33.10181936)
\curveto(984.48657537,33.15181539)(984.38657547,33.22181532)(984.32656494,33.31181936)
\curveto(984.28657557,33.35181519)(984.2565756,33.41681513)(984.23656494,33.50681936)
\curveto(984.22657563,33.52681502)(984.22657563,33.546815)(984.23656494,33.56681936)
\curveto(984.23657562,33.59681495)(984.23157563,33.62181492)(984.22156494,33.64181936)
\curveto(984.22157564,33.82181472)(984.22157564,34.03181451)(984.22156494,34.27181936)
\curveto(984.21157565,34.51181403)(984.24657561,34.68681386)(984.32656494,34.79681936)
\curveto(984.38657547,34.87681367)(984.48657537,34.93681361)(984.62656494,34.97681936)
\curveto(984.7565751,35.02681352)(984.87657498,35.07681347)(984.98656494,35.12681936)
\curveto(985.21657464,35.22681332)(985.44657441,35.31681323)(985.67656494,35.39681936)
\curveto(985.90657395,35.47681307)(986.13657372,35.56681298)(986.36656494,35.66681936)
\curveto(986.56657329,35.7468128)(986.77157309,35.82181272)(986.98156494,35.89181936)
\curveto(987.19157267,35.97181257)(987.39657246,36.05681249)(987.59656494,36.14681936)
\curveto(988.32657153,36.4468121)(989.06657079,36.73181181)(989.81656494,37.00181936)
\curveto(990.5565693,37.28181126)(991.29156857,37.57681097)(992.02156494,37.88681936)
\curveto(992.11156775,37.92681062)(992.19656766,37.95681059)(992.27656494,37.97681936)
\curveto(992.3565675,38.00681054)(992.44156742,38.03681051)(992.53156494,38.06681936)
\curveto(992.79156707,38.17681037)(993.0565668,38.28181026)(993.32656494,38.38181936)
\curveto(993.59656626,38.49181005)(993.861566,38.60180994)(994.12156494,38.71181936)
\moveto(990.47656494,35.50181936)
\curveto(990.44656941,35.59181295)(990.39656946,35.6468129)(990.32656494,35.66681936)
\curveto(990.2565696,35.69681285)(990.18156968,35.70181284)(990.10156494,35.68181936)
\curveto(990.01156985,35.67181287)(989.92656993,35.6468129)(989.84656494,35.60681936)
\curveto(989.7565701,35.57681297)(989.68157018,35.546813)(989.62156494,35.51681936)
\curveto(989.58157028,35.49681305)(989.54657031,35.48681306)(989.51656494,35.48681936)
\curveto(989.48657037,35.48681306)(989.45157041,35.47681307)(989.41156494,35.45681936)
\lineto(989.17156494,35.36681936)
\curveto(989.08157078,35.3468132)(988.99157087,35.31681323)(988.90156494,35.27681936)
\curveto(988.54157132,35.12681342)(988.17657168,34.99181355)(987.80656494,34.87181936)
\curveto(987.42657243,34.76181378)(987.0565728,34.63181391)(986.69656494,34.48181936)
\curveto(986.58657327,34.43181411)(986.47657338,34.38681416)(986.36656494,34.34681936)
\curveto(986.2565736,34.31681423)(986.15157371,34.27681427)(986.05156494,34.22681936)
\curveto(986.00157386,34.20681434)(985.9565739,34.18181436)(985.91656494,34.15181936)
\curveto(985.86657399,34.13181441)(985.84157402,34.08181446)(985.84156494,34.00181936)
\curveto(985.861574,33.98181456)(985.87657398,33.96181458)(985.88656494,33.94181936)
\curveto(985.89657396,33.92181462)(985.91157395,33.90181464)(985.93156494,33.88181936)
\curveto(985.98157388,33.8418147)(986.03657382,33.81181473)(986.09656494,33.79181936)
\curveto(986.14657371,33.77181477)(986.20157366,33.75181479)(986.26156494,33.73181936)
\curveto(986.37157349,33.68181486)(986.48157338,33.6418149)(986.59156494,33.61181936)
\curveto(986.70157316,33.58181496)(986.81157305,33.541815)(986.92156494,33.49181936)
\curveto(987.31157255,33.32181522)(987.70657215,33.17181537)(988.10656494,33.04181936)
\curveto(988.50657135,32.92181562)(988.89657096,32.78181576)(989.27656494,32.62181936)
\lineto(989.42656494,32.56181936)
\curveto(989.47657038,32.55181599)(989.52657033,32.53681601)(989.57656494,32.51681936)
\lineto(989.81656494,32.42681936)
\curveto(989.89656996,32.39681615)(989.97656988,32.37181617)(990.05656494,32.35181936)
\curveto(990.10656975,32.33181621)(990.1615697,32.32181622)(990.22156494,32.32181936)
\curveto(990.28156958,32.33181621)(990.33156953,32.3468162)(990.37156494,32.36681936)
\curveto(990.45156941,32.41681613)(990.49656936,32.52181602)(990.50656494,32.68181936)
\lineto(990.50656494,33.13181936)
\lineto(990.50656494,34.73681936)
\curveto(990.50656935,34.8468137)(990.51156935,34.98181356)(990.52156494,35.14181936)
\curveto(990.52156934,35.30181324)(990.50656935,35.42181312)(990.47656494,35.50181936)
}
}
{
\newrgbcolor{curcolor}{0 0 0}
\pscustom[linestyle=none,fillstyle=solid,fillcolor=curcolor]
{
\newpath
\moveto(987.28156494,46.44338186)
\curveto(987.33157253,46.51337426)(987.40657245,46.54837422)(987.50656494,46.54838186)
\curveto(987.60657225,46.55837421)(987.71157215,46.56337421)(987.82156494,46.56338186)
\lineto(994.09156494,46.56338186)
\lineto(994.69156494,46.56338186)
\curveto(994.74156512,46.54337423)(994.79156507,46.53837423)(994.84156494,46.54838186)
\curveto(994.88156498,46.55837421)(994.92656493,46.55337422)(994.97656494,46.53338186)
\curveto(995.07656478,46.51337426)(995.17656468,46.49837427)(995.27656494,46.48838186)
\curveto(995.38656447,46.48837428)(995.49156437,46.4733743)(995.59156494,46.44338186)
\curveto(995.70156416,46.41337436)(995.80656405,46.38337439)(995.90656494,46.35338186)
\curveto(996.00656385,46.33337444)(996.10656375,46.29837447)(996.20656494,46.24838186)
\curveto(996.46656339,46.14837462)(996.70156316,46.01837475)(996.91156494,45.85838186)
\curveto(997.12156274,45.70837506)(997.29656256,45.52837524)(997.43656494,45.31838186)
\curveto(997.5565623,45.14837562)(997.65156221,44.9683758)(997.72156494,44.77838186)
\curveto(997.80156206,44.58837618)(997.87656198,44.38337639)(997.94656494,44.16338186)
\curveto(997.96656189,44.0733767)(997.97656188,43.98337679)(997.97656494,43.89338186)
\curveto(997.98656187,43.80337697)(998.00156186,43.71337706)(998.02156494,43.62338186)
\lineto(998.02156494,43.53338186)
\curveto(998.03156183,43.51337726)(998.03656182,43.49337728)(998.03656494,43.47338186)
\curveto(998.04656181,43.42337735)(998.04656181,43.3733774)(998.03656494,43.32338186)
\curveto(998.02656183,43.28337749)(998.03156183,43.23837753)(998.05156494,43.18838186)
\curveto(998.07156179,43.11837765)(998.07656178,43.00837776)(998.06656494,42.85838186)
\curveto(998.06656179,42.71837805)(998.0565618,42.61837815)(998.03656494,42.55838186)
\curveto(998.03656182,42.52837824)(998.03156183,42.49837827)(998.02156494,42.46838186)
\lineto(998.02156494,42.40838186)
\curveto(998.00156186,42.31837845)(997.98656187,42.22837854)(997.97656494,42.13838186)
\curveto(997.97656188,42.04837872)(997.96656189,41.96337881)(997.94656494,41.88338186)
\curveto(997.92656193,41.80337897)(997.90156196,41.72337905)(997.87156494,41.64338186)
\curveto(997.85156201,41.56337921)(997.82656203,41.48337929)(997.79656494,41.40338186)
\curveto(997.66656219,41.08337969)(997.52156234,40.81337996)(997.36156494,40.59338186)
\curveto(997.20156266,40.38338039)(996.97656288,40.19338058)(996.68656494,40.02338186)
\curveto(996.66656319,40.00338077)(996.64156322,39.98838078)(996.61156494,39.97838186)
\curveto(996.59156327,39.97838079)(996.56656329,39.9683808)(996.53656494,39.94838186)
\curveto(996.4565634,39.91838085)(996.34156352,39.88338089)(996.19156494,39.84338186)
\curveto(996.05156381,39.81338096)(995.94656391,39.84338093)(995.87656494,39.93338186)
\curveto(995.82656403,39.99338078)(995.80156406,40.0733807)(995.80156494,40.17338186)
\lineto(995.80156494,40.50338186)
\lineto(995.80156494,40.66838186)
\curveto(995.80156406,40.72838004)(995.81156405,40.78337999)(995.83156494,40.83338186)
\curveto(995.861564,40.92337985)(995.91156395,40.98837978)(995.98156494,41.02838186)
\curveto(996.05156381,41.0683797)(996.12656373,41.11337966)(996.20656494,41.16338186)
\lineto(996.38656494,41.28338186)
\curveto(996.4565634,41.33337944)(996.51156335,41.38337939)(996.55156494,41.43338186)
\curveto(996.74156312,41.68337909)(996.88156298,41.98337879)(996.97156494,42.33338186)
\curveto(996.99156287,42.39337838)(997.00156286,42.45337832)(997.00156494,42.51338186)
\curveto(997.01156285,42.58337819)(997.02656283,42.64837812)(997.04656494,42.70838186)
\lineto(997.04656494,42.79838186)
\curveto(997.06656279,42.8683779)(997.07656278,42.95337782)(997.07656494,43.05338186)
\curveto(997.07656278,43.15337762)(997.06656279,43.24337753)(997.04656494,43.32338186)
\curveto(997.03656282,43.35337742)(997.03156283,43.39337738)(997.03156494,43.44338186)
\curveto(997.01156285,43.54337723)(996.99156287,43.63837713)(996.97156494,43.72838186)
\curveto(996.9615629,43.81837695)(996.93656292,43.90337687)(996.89656494,43.98338186)
\curveto(996.77656308,44.2733765)(996.61156325,44.50837626)(996.40156494,44.68838186)
\curveto(996.20156366,44.87837589)(995.9565639,45.03337574)(995.66656494,45.15338186)
\curveto(995.57656428,45.19337558)(995.48156438,45.21837555)(995.38156494,45.22838186)
\curveto(995.28156458,45.24837552)(995.17656468,45.2733755)(995.06656494,45.30338186)
\curveto(995.01656484,45.32337545)(994.96656489,45.33337544)(994.91656494,45.33338186)
\curveto(994.86656499,45.33337544)(994.81656504,45.33837543)(994.76656494,45.34838186)
\curveto(994.73656512,45.35837541)(994.68656517,45.36337541)(994.61656494,45.36338186)
\curveto(994.53656532,45.38337539)(994.45156541,45.38337539)(994.36156494,45.36338186)
\curveto(994.31156555,45.35337542)(994.26656559,45.34837542)(994.22656494,45.34838186)
\curveto(994.18656567,45.35837541)(994.15156571,45.35337542)(994.12156494,45.33338186)
\curveto(994.10156576,45.31337546)(994.09156577,45.29837547)(994.09156494,45.28838186)
\lineto(994.04656494,45.24338186)
\curveto(994.04656581,45.14337563)(994.07656578,45.0683757)(994.13656494,45.01838186)
\curveto(994.18656567,44.97837579)(994.23156563,44.92837584)(994.27156494,44.86838186)
\lineto(994.48156494,44.62838186)
\curveto(994.54156532,44.54837622)(994.59656526,44.45837631)(994.64656494,44.35838186)
\curveto(994.73656512,44.21837655)(994.81156505,44.04337673)(994.87156494,43.83338186)
\curveto(994.92156494,43.62337715)(994.9565649,43.40337737)(994.97656494,43.17338186)
\curveto(994.99656486,42.94337783)(994.99156487,42.71337806)(994.96156494,42.48338186)
\curveto(994.94156492,42.25337852)(994.90156496,42.04337873)(994.84156494,41.85338186)
\curveto(994.53156533,40.91337986)(993.93656592,40.25338052)(993.05656494,39.87338186)
\curveto(992.9565669,39.82338095)(992.861567,39.78338099)(992.77156494,39.75338186)
\curveto(992.67156719,39.72338105)(992.56656729,39.68838108)(992.45656494,39.64838186)
\curveto(992.40656745,39.62838114)(992.3615675,39.61838115)(992.32156494,39.61838186)
\curveto(992.28156758,39.61838115)(992.23656762,39.60838116)(992.18656494,39.58838186)
\curveto(992.11656774,39.5683812)(992.04656781,39.55338122)(991.97656494,39.54338186)
\curveto(991.89656796,39.54338123)(991.82156804,39.53338124)(991.75156494,39.51338186)
\curveto(991.71156815,39.50338127)(991.67656818,39.49838127)(991.64656494,39.49838186)
\curveto(991.60656825,39.50838126)(991.56656829,39.50838126)(991.52656494,39.49838186)
\curveto(991.48656837,39.49838127)(991.44656841,39.49338128)(991.40656494,39.48338186)
\lineto(991.28656494,39.48338186)
\curveto(991.16656869,39.46338131)(991.04156882,39.46338131)(990.91156494,39.48338186)
\curveto(990.85156901,39.49338128)(990.79156907,39.49838127)(990.73156494,39.49838186)
\lineto(990.56656494,39.49838186)
\curveto(990.51656934,39.50838126)(990.47656938,39.51338126)(990.44656494,39.51338186)
\curveto(990.40656945,39.51338126)(990.3615695,39.51838125)(990.31156494,39.52838186)
\curveto(990.20156966,39.55838121)(990.09656976,39.57838119)(989.99656494,39.58838186)
\curveto(989.88656997,39.59838117)(989.77657008,39.62338115)(989.66656494,39.66338186)
\curveto(989.54657031,39.70338107)(989.43157043,39.73838103)(989.32156494,39.76838186)
\curveto(989.20157066,39.80838096)(989.08657077,39.85338092)(988.97656494,39.90338186)
\curveto(988.81657104,39.9733808)(988.67157119,40.05338072)(988.54156494,40.14338186)
\curveto(988.40157146,40.23338054)(988.26657159,40.32838044)(988.13656494,40.42838186)
\curveto(988.02657183,40.49838027)(987.93657192,40.58838018)(987.86656494,40.69838186)
\lineto(987.80656494,40.75838186)
\lineto(987.74656494,40.81838186)
\lineto(987.62656494,40.96838186)
\lineto(987.50656494,41.14838186)
\curveto(987.42657243,41.27837949)(987.3565725,41.41337936)(987.29656494,41.55338186)
\curveto(987.23657262,41.70337907)(987.18157268,41.86337891)(987.13156494,42.03338186)
\curveto(987.10157276,42.13337864)(987.08157278,42.23337854)(987.07156494,42.33338186)
\curveto(987.0615728,42.44337833)(987.04657281,42.55337822)(987.02656494,42.66338186)
\curveto(987.01657284,42.70337807)(987.01657284,42.75337802)(987.02656494,42.81338186)
\curveto(987.03657282,42.88337789)(987.03157283,42.93337784)(987.01156494,42.96338186)
\curveto(987.00157286,43.28337749)(987.03157283,43.5683772)(987.10156494,43.81838186)
\curveto(987.17157269,44.07837669)(987.27157259,44.30837646)(987.40156494,44.50838186)
\curveto(987.44157242,44.57837619)(987.48657237,44.64337613)(987.53656494,44.70338186)
\lineto(987.68656494,44.88338186)
\curveto(987.72657213,44.93337584)(987.77157209,44.97837579)(987.82156494,45.01838186)
\curveto(987.861572,45.0683757)(987.88157198,45.14337563)(987.88156494,45.24338186)
\lineto(987.83656494,45.28838186)
\curveto(987.81657204,45.30837546)(987.79157207,45.32837544)(987.76156494,45.34838186)
\curveto(987.68157218,45.37837539)(987.60157226,45.39337538)(987.52156494,45.39338186)
\curveto(987.44157242,45.40337537)(987.37157249,45.43337534)(987.31156494,45.48338186)
\curveto(987.27157259,45.51337526)(987.24157262,45.5733752)(987.22156494,45.66338186)
\curveto(987.19157267,45.75337502)(987.17657268,45.84837492)(987.17656494,45.94838186)
\curveto(987.17657268,46.04837472)(987.18657267,46.14337463)(987.20656494,46.23338186)
\curveto(987.22657263,46.33337444)(987.25157261,46.40337437)(987.28156494,46.44338186)
\moveto(991.06156494,45.31838186)
\curveto(991.02156884,45.32837544)(990.97156889,45.33337544)(990.91156494,45.33338186)
\curveto(990.84156902,45.33337544)(990.78656907,45.32837544)(990.74656494,45.31838186)
\lineto(990.50656494,45.31838186)
\curveto(990.41656944,45.29837547)(990.33156953,45.28337549)(990.25156494,45.27338186)
\curveto(990.1615697,45.26337551)(990.07656978,45.24837552)(989.99656494,45.22838186)
\curveto(989.91656994,45.20837556)(989.84157002,45.18837558)(989.77156494,45.16838186)
\curveto(989.69157017,45.15837561)(989.61657024,45.13837563)(989.54656494,45.10838186)
\curveto(989.26657059,44.99837577)(989.01657084,44.85337592)(988.79656494,44.67338186)
\curveto(988.57657128,44.50337627)(988.41157145,44.28337649)(988.30156494,44.01338186)
\curveto(988.2615716,43.93337684)(988.23157163,43.84837692)(988.21156494,43.75838186)
\curveto(988.18157168,43.6683771)(988.1565717,43.5733772)(988.13656494,43.47338186)
\curveto(988.11657174,43.39337738)(988.11157175,43.30337747)(988.12156494,43.20338186)
\lineto(988.12156494,42.93338186)
\curveto(988.13157173,42.88337789)(988.13657172,42.83337794)(988.13656494,42.78338186)
\curveto(988.13657172,42.74337803)(988.14157172,42.69837807)(988.15156494,42.64838186)
\curveto(988.20157166,42.45837831)(988.25157161,42.29837847)(988.30156494,42.16838186)
\curveto(988.44157142,41.82837894)(988.65157121,41.56337921)(988.93156494,41.37338186)
\curveto(989.21157065,41.18337959)(989.53657032,41.03337974)(989.90656494,40.92338186)
\curveto(989.98656987,40.90337987)(990.06656979,40.88837988)(990.14656494,40.87838186)
\curveto(990.21656964,40.87837989)(990.29156957,40.8683799)(990.37156494,40.84838186)
\curveto(990.40156946,40.82837994)(990.43656942,40.81837995)(990.47656494,40.81838186)
\curveto(990.51656934,40.82837994)(990.55156931,40.82837994)(990.58156494,40.81838186)
\lineto(990.91156494,40.81838186)
\lineto(991.25656494,40.81838186)
\curveto(991.36656849,40.81837995)(991.47156839,40.82837994)(991.57156494,40.84838186)
\lineto(991.64656494,40.84838186)
\curveto(991.67656818,40.85837991)(991.70156816,40.86337991)(991.72156494,40.86338186)
\curveto(991.81156805,40.88337989)(991.90156796,40.89837987)(991.99156494,40.90838186)
\curveto(992.08156778,40.92837984)(992.16656769,40.95337982)(992.24656494,40.98338186)
\curveto(992.50656735,41.06337971)(992.74656711,41.16337961)(992.96656494,41.28338186)
\curveto(993.18656667,41.40337937)(993.36656649,41.56337921)(993.50656494,41.76338186)
\lineto(993.59656494,41.88338186)
\curveto(993.61656624,41.92337885)(993.63656622,41.9683788)(993.65656494,42.01838186)
\curveto(993.70656615,42.09837867)(993.74656611,42.18337859)(993.77656494,42.27338186)
\curveto(993.80656605,42.36337841)(993.83656602,42.46337831)(993.86656494,42.57338186)
\curveto(993.87656598,42.62337815)(993.88156598,42.6683781)(993.88156494,42.70838186)
\curveto(993.87156599,42.75837801)(993.87656598,42.80837796)(993.89656494,42.85838186)
\curveto(993.90656595,42.88837788)(993.91156595,42.93837783)(993.91156494,43.00838186)
\curveto(993.91156595,43.07837769)(993.90656595,43.12837764)(993.89656494,43.15838186)
\curveto(993.88656597,43.18837758)(993.88656597,43.21837755)(993.89656494,43.24838186)
\curveto(993.89656596,43.28837748)(993.89156597,43.32837744)(993.88156494,43.36838186)
\curveto(993.861566,43.45837731)(993.84156602,43.54337723)(993.82156494,43.62338186)
\curveto(993.80156606,43.70337707)(993.77656608,43.78337699)(993.74656494,43.86338186)
\curveto(993.59656626,44.20337657)(993.38656647,44.4733763)(993.11656494,44.67338186)
\curveto(992.84656701,44.8733759)(992.53156733,45.03337574)(992.17156494,45.15338186)
\curveto(992.08156778,45.18337559)(991.99156787,45.20337557)(991.90156494,45.21338186)
\curveto(991.80156806,45.23337554)(991.70656815,45.25337552)(991.61656494,45.27338186)
\curveto(991.57656828,45.28337549)(991.54156832,45.28837548)(991.51156494,45.28838186)
\curveto(991.47156839,45.28837548)(991.43156843,45.29337548)(991.39156494,45.30338186)
\curveto(991.34156852,45.32337545)(991.29156857,45.32337545)(991.24156494,45.30338186)
\curveto(991.18156868,45.29337548)(991.12156874,45.29837547)(991.06156494,45.31838186)
}
}
{
\newrgbcolor{curcolor}{0 0 0}
\pscustom[linestyle=none,fillstyle=solid,fillcolor=curcolor]
{
\newpath
\moveto(990.70156494,55.56666311)
\curveto(990.7615691,55.58665505)(990.856569,55.59665504)(990.98656494,55.59666311)
\curveto(991.10656875,55.59665504)(991.19156867,55.59165504)(991.24156494,55.58166311)
\lineto(991.39156494,55.58166311)
\curveto(991.47156839,55.57165506)(991.54656831,55.56165507)(991.61656494,55.55166311)
\curveto(991.67656818,55.55165508)(991.74656811,55.54665509)(991.82656494,55.53666311)
\curveto(991.88656797,55.51665512)(991.94656791,55.50165513)(992.00656494,55.49166311)
\curveto(992.06656779,55.49165514)(992.12656773,55.48165515)(992.18656494,55.46166311)
\curveto(992.31656754,55.42165521)(992.44656741,55.38665525)(992.57656494,55.35666311)
\curveto(992.70656715,55.32665531)(992.82656703,55.28665535)(992.93656494,55.23666311)
\curveto(993.41656644,55.02665561)(993.82156604,54.74665589)(994.15156494,54.39666311)
\curveto(994.47156539,54.04665659)(994.71656514,53.61665702)(994.88656494,53.10666311)
\curveto(994.92656493,52.99665764)(994.9565649,52.87665776)(994.97656494,52.74666311)
\curveto(994.99656486,52.62665801)(995.01656484,52.50165813)(995.03656494,52.37166311)
\curveto(995.04656481,52.31165832)(995.05156481,52.24665839)(995.05156494,52.17666311)
\curveto(995.0615648,52.11665852)(995.06656479,52.05665858)(995.06656494,51.99666311)
\curveto(995.07656478,51.95665868)(995.08156478,51.89665874)(995.08156494,51.81666311)
\curveto(995.08156478,51.74665889)(995.07656478,51.69665894)(995.06656494,51.66666311)
\curveto(995.0565648,51.62665901)(995.05156481,51.58665905)(995.05156494,51.54666311)
\curveto(995.0615648,51.50665913)(995.0615648,51.47165916)(995.05156494,51.44166311)
\lineto(995.05156494,51.35166311)
\lineto(995.00656494,50.99166311)
\curveto(994.96656489,50.85165978)(994.92656493,50.71665992)(994.88656494,50.58666311)
\curveto(994.84656501,50.45666018)(994.80156506,50.3316603)(994.75156494,50.21166311)
\curveto(994.55156531,49.76166087)(994.29156557,49.39166124)(993.97156494,49.10166311)
\curveto(993.65156621,48.81166182)(993.2615666,48.57166206)(992.80156494,48.38166311)
\curveto(992.70156716,48.3316623)(992.60156726,48.29166234)(992.50156494,48.26166311)
\curveto(992.40156746,48.24166239)(992.29656756,48.22166241)(992.18656494,48.20166311)
\curveto(992.14656771,48.18166245)(992.11656774,48.17166246)(992.09656494,48.17166311)
\curveto(992.06656779,48.18166245)(992.03156783,48.18166245)(991.99156494,48.17166311)
\curveto(991.91156795,48.15166248)(991.83156803,48.1366625)(991.75156494,48.12666311)
\curveto(991.6615682,48.12666251)(991.57656828,48.11666252)(991.49656494,48.09666311)
\lineto(991.37656494,48.09666311)
\curveto(991.33656852,48.09666254)(991.29156857,48.09166254)(991.24156494,48.08166311)
\curveto(991.19156867,48.07166256)(991.10656875,48.06666257)(990.98656494,48.06666311)
\curveto(990.856569,48.06666257)(990.7615691,48.07666256)(990.70156494,48.09666311)
\curveto(990.63156923,48.11666252)(990.5615693,48.12166251)(990.49156494,48.11166311)
\curveto(990.42156944,48.10166253)(990.35156951,48.10666253)(990.28156494,48.12666311)
\curveto(990.23156963,48.1366625)(990.19156967,48.14166249)(990.16156494,48.14166311)
\curveto(990.12156974,48.15166248)(990.07656978,48.16166247)(990.02656494,48.17166311)
\curveto(989.90656995,48.20166243)(989.78657007,48.22666241)(989.66656494,48.24666311)
\curveto(989.54657031,48.27666236)(989.43157043,48.31666232)(989.32156494,48.36666311)
\curveto(988.95157091,48.51666212)(988.62157124,48.69666194)(988.33156494,48.90666311)
\curveto(988.03157183,49.12666151)(987.78157208,49.39166124)(987.58156494,49.70166311)
\curveto(987.50157236,49.82166081)(987.43657242,49.94666069)(987.38656494,50.07666311)
\curveto(987.32657253,50.20666043)(987.26657259,50.34166029)(987.20656494,50.48166311)
\curveto(987.1565727,50.60166003)(987.12657273,50.7316599)(987.11656494,50.87166311)
\curveto(987.09657276,51.01165962)(987.06657279,51.15165948)(987.02656494,51.29166311)
\lineto(987.02656494,51.48666311)
\curveto(987.01657284,51.55665908)(987.00657285,51.62165901)(986.99656494,51.68166311)
\curveto(986.98657287,52.57165806)(987.17157269,53.31165732)(987.55156494,53.90166311)
\curveto(987.93157193,54.49165614)(988.42657143,54.91665572)(989.03656494,55.17666311)
\curveto(989.13657072,55.22665541)(989.23657062,55.26665537)(989.33656494,55.29666311)
\curveto(989.43657042,55.32665531)(989.54157032,55.36165527)(989.65156494,55.40166311)
\curveto(989.7615701,55.4316552)(989.88156998,55.45665518)(990.01156494,55.47666311)
\curveto(990.13156973,55.49665514)(990.2565696,55.52165511)(990.38656494,55.55166311)
\curveto(990.43656942,55.56165507)(990.49156937,55.56165507)(990.55156494,55.55166311)
\curveto(990.60156926,55.55165508)(990.65156921,55.55665508)(990.70156494,55.56666311)
\moveto(991.55656494,54.23166311)
\curveto(991.48656837,54.25165638)(991.40656845,54.25665638)(991.31656494,54.24666311)
\lineto(991.06156494,54.24666311)
\curveto(990.67156919,54.24665639)(990.34156952,54.21165642)(990.07156494,54.14166311)
\curveto(989.99156987,54.11165652)(989.91156995,54.08665655)(989.83156494,54.06666311)
\curveto(989.75157011,54.04665659)(989.67657018,54.02165661)(989.60656494,53.99166311)
\curveto(988.9565709,53.71165692)(988.50657135,53.26665737)(988.25656494,52.65666311)
\curveto(988.22657163,52.58665805)(988.20657165,52.51165812)(988.19656494,52.43166311)
\lineto(988.13656494,52.19166311)
\curveto(988.11657174,52.11165852)(988.10657175,52.02665861)(988.10656494,51.93666311)
\lineto(988.10656494,51.66666311)
\lineto(988.15156494,51.39666311)
\curveto(988.17157169,51.29665934)(988.19657166,51.20165943)(988.22656494,51.11166311)
\curveto(988.24657161,51.0316596)(988.27657158,50.95165968)(988.31656494,50.87166311)
\curveto(988.33657152,50.80165983)(988.36657149,50.7366599)(988.40656494,50.67666311)
\curveto(988.44657141,50.61666002)(988.48657137,50.56166007)(988.52656494,50.51166311)
\curveto(988.69657116,50.27166036)(988.90157096,50.07666056)(989.14156494,49.92666311)
\curveto(989.38157048,49.77666086)(989.6615702,49.64666099)(989.98156494,49.53666311)
\curveto(990.08156978,49.50666113)(990.18656967,49.48666115)(990.29656494,49.47666311)
\curveto(990.39656946,49.46666117)(990.50156936,49.45166118)(990.61156494,49.43166311)
\curveto(990.65156921,49.42166121)(990.71656914,49.41666122)(990.80656494,49.41666311)
\curveto(990.83656902,49.40666123)(990.87156899,49.40166123)(990.91156494,49.40166311)
\curveto(990.95156891,49.41166122)(990.99656886,49.41666122)(991.04656494,49.41666311)
\lineto(991.34656494,49.41666311)
\curveto(991.44656841,49.41666122)(991.53656832,49.42666121)(991.61656494,49.44666311)
\lineto(991.79656494,49.47666311)
\curveto(991.89656796,49.49666114)(991.99656786,49.51166112)(992.09656494,49.52166311)
\curveto(992.18656767,49.54166109)(992.27156759,49.57166106)(992.35156494,49.61166311)
\curveto(992.59156727,49.71166092)(992.81656704,49.82666081)(993.02656494,49.95666311)
\curveto(993.23656662,50.09666054)(993.41156645,50.26666037)(993.55156494,50.46666311)
\curveto(993.58156628,50.51666012)(993.60656625,50.56166007)(993.62656494,50.60166311)
\curveto(993.64656621,50.64165999)(993.67156619,50.68665995)(993.70156494,50.73666311)
\curveto(993.75156611,50.81665982)(993.79656606,50.90165973)(993.83656494,50.99166311)
\curveto(993.86656599,51.09165954)(993.89656596,51.19665944)(993.92656494,51.30666311)
\curveto(993.94656591,51.35665928)(993.9565659,51.40165923)(993.95656494,51.44166311)
\curveto(993.94656591,51.49165914)(993.94656591,51.54165909)(993.95656494,51.59166311)
\curveto(993.96656589,51.62165901)(993.97656588,51.68165895)(993.98656494,51.77166311)
\curveto(993.99656586,51.87165876)(993.99156587,51.94665869)(993.97156494,51.99666311)
\curveto(993.9615659,52.0366586)(993.9615659,52.07665856)(993.97156494,52.11666311)
\curveto(993.97156589,52.15665848)(993.9615659,52.19665844)(993.94156494,52.23666311)
\curveto(993.92156594,52.31665832)(993.90656595,52.39665824)(993.89656494,52.47666311)
\curveto(993.87656598,52.55665808)(993.85156601,52.631658)(993.82156494,52.70166311)
\curveto(993.68156618,53.04165759)(993.48656637,53.31665732)(993.23656494,53.52666311)
\curveto(992.98656687,53.7366569)(992.69156717,53.91165672)(992.35156494,54.05166311)
\curveto(992.23156763,54.10165653)(992.10656775,54.1316565)(991.97656494,54.14166311)
\curveto(991.83656802,54.16165647)(991.69656816,54.19165644)(991.55656494,54.23166311)
}
}
{
\newrgbcolor{curcolor}{0 0 0}
\pscustom[linestyle=none,fillstyle=solid,fillcolor=curcolor]
{
}
}
{
\newrgbcolor{curcolor}{0 0 0}
\pscustom[linestyle=none,fillstyle=solid,fillcolor=curcolor]
{
\newpath
\moveto(989.81656494,67.99510061)
\lineto(990.07156494,67.99510061)
\curveto(990.15156971,68.0050929)(990.22656963,68.00009291)(990.29656494,67.98010061)
\lineto(990.53656494,67.98010061)
\lineto(990.70156494,67.98010061)
\curveto(990.80156906,67.96009295)(990.90656895,67.95009296)(991.01656494,67.95010061)
\curveto(991.11656874,67.95009296)(991.21656864,67.94009297)(991.31656494,67.92010061)
\lineto(991.46656494,67.92010061)
\curveto(991.60656825,67.89009302)(991.74656811,67.87009304)(991.88656494,67.86010061)
\curveto(992.01656784,67.85009306)(992.14656771,67.82509308)(992.27656494,67.78510061)
\curveto(992.3565675,67.76509314)(992.44156742,67.74509316)(992.53156494,67.72510061)
\lineto(992.77156494,67.66510061)
\lineto(993.07156494,67.54510061)
\curveto(993.1615667,67.51509339)(993.25156661,67.48009343)(993.34156494,67.44010061)
\curveto(993.5615663,67.34009357)(993.77656608,67.2050937)(993.98656494,67.03510061)
\curveto(994.19656566,66.87509403)(994.36656549,66.70009421)(994.49656494,66.51010061)
\curveto(994.53656532,66.46009445)(994.57656528,66.40009451)(994.61656494,66.33010061)
\curveto(994.64656521,66.27009464)(994.68156518,66.2100947)(994.72156494,66.15010061)
\curveto(994.77156509,66.07009484)(994.81156505,65.97509493)(994.84156494,65.86510061)
\curveto(994.87156499,65.75509515)(994.90156496,65.65009526)(994.93156494,65.55010061)
\curveto(994.97156489,65.44009547)(994.99656486,65.33009558)(995.00656494,65.22010061)
\curveto(995.01656484,65.1100958)(995.03156483,64.99509591)(995.05156494,64.87510061)
\curveto(995.0615648,64.83509607)(995.0615648,64.79009612)(995.05156494,64.74010061)
\curveto(995.05156481,64.70009621)(995.0565648,64.66009625)(995.06656494,64.62010061)
\curveto(995.07656478,64.58009633)(995.08156478,64.52509638)(995.08156494,64.45510061)
\curveto(995.08156478,64.38509652)(995.07656478,64.33509657)(995.06656494,64.30510061)
\curveto(995.04656481,64.25509665)(995.04156482,64.2100967)(995.05156494,64.17010061)
\curveto(995.0615648,64.13009678)(995.0615648,64.09509681)(995.05156494,64.06510061)
\lineto(995.05156494,63.97510061)
\curveto(995.03156483,63.91509699)(995.01656484,63.85009706)(995.00656494,63.78010061)
\curveto(995.00656485,63.72009719)(995.00156486,63.65509725)(994.99156494,63.58510061)
\curveto(994.94156492,63.41509749)(994.89156497,63.25509765)(994.84156494,63.10510061)
\curveto(994.79156507,62.95509795)(994.72656513,62.8100981)(994.64656494,62.67010061)
\curveto(994.60656525,62.62009829)(994.57656528,62.56509834)(994.55656494,62.50510061)
\curveto(994.52656533,62.45509845)(994.49156537,62.4050985)(994.45156494,62.35510061)
\curveto(994.27156559,62.11509879)(994.05156581,61.91509899)(993.79156494,61.75510061)
\curveto(993.53156633,61.59509931)(993.24656661,61.45509945)(992.93656494,61.33510061)
\curveto(992.79656706,61.27509963)(992.6565672,61.23009968)(992.51656494,61.20010061)
\curveto(992.36656749,61.17009974)(992.21156765,61.13509977)(992.05156494,61.09510061)
\curveto(991.94156792,61.07509983)(991.83156803,61.06009985)(991.72156494,61.05010061)
\curveto(991.61156825,61.04009987)(991.50156836,61.02509988)(991.39156494,61.00510061)
\curveto(991.35156851,60.99509991)(991.31156855,60.99009992)(991.27156494,60.99010061)
\curveto(991.23156863,61.00009991)(991.19156867,61.00009991)(991.15156494,60.99010061)
\curveto(991.10156876,60.98009993)(991.05156881,60.97509993)(991.00156494,60.97510061)
\lineto(990.83656494,60.97510061)
\curveto(990.78656907,60.95509995)(990.73656912,60.95009996)(990.68656494,60.96010061)
\curveto(990.62656923,60.97009994)(990.57156929,60.97009994)(990.52156494,60.96010061)
\curveto(990.48156938,60.95009996)(990.43656942,60.95009996)(990.38656494,60.96010061)
\curveto(990.33656952,60.97009994)(990.28656957,60.96509994)(990.23656494,60.94510061)
\curveto(990.16656969,60.92509998)(990.09156977,60.92009999)(990.01156494,60.93010061)
\curveto(989.92156994,60.94009997)(989.83657002,60.94509996)(989.75656494,60.94510061)
\curveto(989.66657019,60.94509996)(989.56657029,60.94009997)(989.45656494,60.93010061)
\curveto(989.33657052,60.92009999)(989.23657062,60.92509998)(989.15656494,60.94510061)
\lineto(988.87156494,60.94510061)
\lineto(988.24156494,60.99010061)
\curveto(988.14157172,61.00009991)(988.04657181,61.0100999)(987.95656494,61.02010061)
\lineto(987.65656494,61.05010061)
\curveto(987.60657225,61.07009984)(987.5565723,61.07509983)(987.50656494,61.06510061)
\curveto(987.44657241,61.06509984)(987.39157247,61.07509983)(987.34156494,61.09510061)
\curveto(987.17157269,61.14509976)(987.00657285,61.18509972)(986.84656494,61.21510061)
\curveto(986.67657318,61.24509966)(986.51657334,61.29509961)(986.36656494,61.36510061)
\curveto(985.90657395,61.55509935)(985.53157433,61.77509913)(985.24156494,62.02510061)
\curveto(984.95157491,62.28509862)(984.70657515,62.64509826)(984.50656494,63.10510061)
\curveto(984.4565754,63.23509767)(984.42157544,63.36509754)(984.40156494,63.49510061)
\curveto(984.38157548,63.63509727)(984.3565755,63.77509713)(984.32656494,63.91510061)
\curveto(984.31657554,63.98509692)(984.31157555,64.05009686)(984.31156494,64.11010061)
\curveto(984.31157555,64.17009674)(984.30657555,64.23509667)(984.29656494,64.30510061)
\curveto(984.27657558,65.13509577)(984.42657543,65.8050951)(984.74656494,66.31510061)
\curveto(985.0565748,66.82509408)(985.49657436,67.2050937)(986.06656494,67.45510061)
\curveto(986.18657367,67.5050934)(986.31157355,67.55009336)(986.44156494,67.59010061)
\curveto(986.57157329,67.63009328)(986.70657315,67.67509323)(986.84656494,67.72510061)
\curveto(986.92657293,67.74509316)(987.01157285,67.76009315)(987.10156494,67.77010061)
\lineto(987.34156494,67.83010061)
\curveto(987.45157241,67.86009305)(987.5615723,67.87509303)(987.67156494,67.87510061)
\curveto(987.78157208,67.88509302)(987.89157197,67.90009301)(988.00156494,67.92010061)
\curveto(988.05157181,67.94009297)(988.09657176,67.94509296)(988.13656494,67.93510061)
\curveto(988.17657168,67.93509297)(988.21657164,67.94009297)(988.25656494,67.95010061)
\curveto(988.30657155,67.96009295)(988.3615715,67.96009295)(988.42156494,67.95010061)
\curveto(988.47157139,67.95009296)(988.52157134,67.95509295)(988.57156494,67.96510061)
\lineto(988.70656494,67.96510061)
\curveto(988.76657109,67.98509292)(988.83657102,67.98509292)(988.91656494,67.96510061)
\curveto(988.98657087,67.95509295)(989.05157081,67.96009295)(989.11156494,67.98010061)
\curveto(989.14157072,67.99009292)(989.18157068,67.99509291)(989.23156494,67.99510061)
\lineto(989.35156494,67.99510061)
\lineto(989.81656494,67.99510061)
\moveto(992.14156494,66.45010061)
\curveto(991.82156804,66.55009436)(991.4565684,66.6100943)(991.04656494,66.63010061)
\curveto(990.63656922,66.65009426)(990.22656963,66.66009425)(989.81656494,66.66010061)
\curveto(989.38657047,66.66009425)(988.96657089,66.65009426)(988.55656494,66.63010061)
\curveto(988.14657171,66.6100943)(987.7615721,66.56509434)(987.40156494,66.49510061)
\curveto(987.04157282,66.42509448)(986.72157314,66.31509459)(986.44156494,66.16510061)
\curveto(986.15157371,66.02509488)(985.91657394,65.83009508)(985.73656494,65.58010061)
\curveto(985.62657423,65.42009549)(985.54657431,65.24009567)(985.49656494,65.04010061)
\curveto(985.43657442,64.84009607)(985.40657445,64.59509631)(985.40656494,64.30510061)
\curveto(985.42657443,64.28509662)(985.43657442,64.25009666)(985.43656494,64.20010061)
\curveto(985.42657443,64.15009676)(985.42657443,64.1100968)(985.43656494,64.08010061)
\curveto(985.4565744,64.00009691)(985.47657438,63.92509698)(985.49656494,63.85510061)
\curveto(985.50657435,63.79509711)(985.52657433,63.73009718)(985.55656494,63.66010061)
\curveto(985.67657418,63.39009752)(985.84657401,63.17009774)(986.06656494,63.00010061)
\curveto(986.27657358,62.84009807)(986.52157334,62.7050982)(986.80156494,62.59510061)
\curveto(986.91157295,62.54509836)(987.03157283,62.5050984)(987.16156494,62.47510061)
\curveto(987.28157258,62.45509845)(987.40657245,62.43009848)(987.53656494,62.40010061)
\curveto(987.58657227,62.38009853)(987.64157222,62.37009854)(987.70156494,62.37010061)
\curveto(987.75157211,62.37009854)(987.80157206,62.36509854)(987.85156494,62.35510061)
\curveto(987.94157192,62.34509856)(988.03657182,62.33509857)(988.13656494,62.32510061)
\curveto(988.22657163,62.31509859)(988.32157154,62.3050986)(988.42156494,62.29510061)
\curveto(988.50157136,62.29509861)(988.58657127,62.29009862)(988.67656494,62.28010061)
\lineto(988.91656494,62.28010061)
\lineto(989.09656494,62.28010061)
\curveto(989.12657073,62.27009864)(989.1615707,62.26509864)(989.20156494,62.26510061)
\lineto(989.33656494,62.26510061)
\lineto(989.78656494,62.26510061)
\curveto(989.86656999,62.26509864)(989.95156991,62.26009865)(990.04156494,62.25010061)
\curveto(990.12156974,62.25009866)(990.19656966,62.26009865)(990.26656494,62.28010061)
\lineto(990.53656494,62.28010061)
\curveto(990.5565693,62.28009863)(990.58656927,62.27509863)(990.62656494,62.26510061)
\curveto(990.6565692,62.26509864)(990.68156918,62.27009864)(990.70156494,62.28010061)
\curveto(990.80156906,62.29009862)(990.90156896,62.29509861)(991.00156494,62.29510061)
\curveto(991.09156877,62.3050986)(991.19156867,62.31509859)(991.30156494,62.32510061)
\curveto(991.42156844,62.35509855)(991.54656831,62.37009854)(991.67656494,62.37010061)
\curveto(991.79656806,62.38009853)(991.91156795,62.4050985)(992.02156494,62.44510061)
\curveto(992.32156754,62.52509838)(992.58656727,62.6100983)(992.81656494,62.70010061)
\curveto(993.04656681,62.80009811)(993.2615666,62.94509796)(993.46156494,63.13510061)
\curveto(993.6615662,63.34509756)(993.81156605,63.6100973)(993.91156494,63.93010061)
\curveto(993.93156593,63.97009694)(993.94156592,64.0050969)(993.94156494,64.03510061)
\curveto(993.93156593,64.07509683)(993.93656592,64.12009679)(993.95656494,64.17010061)
\curveto(993.96656589,64.2100967)(993.97656588,64.28009663)(993.98656494,64.38010061)
\curveto(993.99656586,64.49009642)(993.99156587,64.57509633)(993.97156494,64.63510061)
\curveto(993.95156591,64.7050962)(993.94156592,64.77509613)(993.94156494,64.84510061)
\curveto(993.93156593,64.91509599)(993.91656594,64.98009593)(993.89656494,65.04010061)
\curveto(993.83656602,65.24009567)(993.75156611,65.42009549)(993.64156494,65.58010061)
\curveto(993.62156624,65.6100953)(993.60156626,65.63509527)(993.58156494,65.65510061)
\lineto(993.52156494,65.71510061)
\curveto(993.50156636,65.75509515)(993.4615664,65.8050951)(993.40156494,65.86510061)
\curveto(993.2615666,65.96509494)(993.13156673,66.05009486)(993.01156494,66.12010061)
\curveto(992.89156697,66.19009472)(992.74656711,66.26009465)(992.57656494,66.33010061)
\curveto(992.50656735,66.36009455)(992.43656742,66.38009453)(992.36656494,66.39010061)
\curveto(992.29656756,66.4100945)(992.22156764,66.43009448)(992.14156494,66.45010061)
}
}
{
\newrgbcolor{curcolor}{0 0 0}
\pscustom[linestyle=none,fillstyle=solid,fillcolor=curcolor]
{
\newpath
\moveto(984.49156494,70.85470998)
\lineto(984.49156494,74.45470998)
\lineto(984.49156494,75.09970998)
\curveto(984.49157537,75.17970345)(984.49657536,75.25470338)(984.50656494,75.32470998)
\curveto(984.50657535,75.39470324)(984.51657534,75.45470318)(984.53656494,75.50470998)
\curveto(984.56657529,75.57470306)(984.62657523,75.629703)(984.71656494,75.66970998)
\curveto(984.74657511,75.68970294)(984.78657507,75.69970293)(984.83656494,75.69970998)
\lineto(984.97156494,75.69970998)
\curveto(985.08157478,75.70970292)(985.18657467,75.70470293)(985.28656494,75.68470998)
\curveto(985.38657447,75.67470296)(985.4565744,75.63970299)(985.49656494,75.57970998)
\curveto(985.56657429,75.48970314)(985.60157426,75.35470328)(985.60156494,75.17470998)
\curveto(985.59157427,74.99470364)(985.58657427,74.8297038)(985.58656494,74.67970998)
\lineto(985.58656494,72.68470998)
\lineto(985.58656494,72.18970998)
\lineto(985.58656494,72.05470998)
\curveto(985.58657427,72.01470662)(985.59157427,71.97470666)(985.60156494,71.93470998)
\lineto(985.60156494,71.72470998)
\curveto(985.63157423,71.61470702)(985.67157419,71.5347071)(985.72156494,71.48470998)
\curveto(985.7615741,71.4347072)(985.81657404,71.39970723)(985.88656494,71.37970998)
\curveto(985.94657391,71.35970727)(986.01657384,71.34470729)(986.09656494,71.33470998)
\curveto(986.17657368,71.32470731)(986.26657359,71.30470733)(986.36656494,71.27470998)
\curveto(986.56657329,71.22470741)(986.77157309,71.18470745)(986.98156494,71.15470998)
\curveto(987.19157267,71.12470751)(987.39657246,71.08470755)(987.59656494,71.03470998)
\curveto(987.66657219,71.01470762)(987.73657212,71.00470763)(987.80656494,71.00470998)
\curveto(987.86657199,71.00470763)(987.93157193,70.99470764)(988.00156494,70.97470998)
\curveto(988.03157183,70.96470767)(988.07157179,70.95470768)(988.12156494,70.94470998)
\curveto(988.1615717,70.94470769)(988.20157166,70.94970768)(988.24156494,70.95970998)
\curveto(988.29157157,70.97970765)(988.33657152,71.00470763)(988.37656494,71.03470998)
\curveto(988.40657145,71.07470756)(988.41157145,71.1347075)(988.39156494,71.21470998)
\curveto(988.37157149,71.27470736)(988.34657151,71.3347073)(988.31656494,71.39470998)
\curveto(988.27657158,71.45470718)(988.24157162,71.51470712)(988.21156494,71.57470998)
\curveto(988.19157167,71.634707)(988.17657168,71.68470695)(988.16656494,71.72470998)
\curveto(988.08657177,71.91470672)(988.03157183,72.11970651)(988.00156494,72.33970998)
\curveto(987.97157189,72.56970606)(987.9615719,72.79970583)(987.97156494,73.02970998)
\curveto(987.97157189,73.26970536)(987.99657186,73.49970513)(988.04656494,73.71970998)
\curveto(988.08657177,73.93970469)(988.14657171,74.13970449)(988.22656494,74.31970998)
\curveto(988.24657161,74.36970426)(988.26657159,74.41470422)(988.28656494,74.45470998)
\curveto(988.30657155,74.50470413)(988.33157153,74.55470408)(988.36156494,74.60470998)
\curveto(988.57157129,74.95470368)(988.80157106,75.2347034)(989.05156494,75.44470998)
\curveto(989.30157056,75.66470297)(989.62657023,75.85970277)(990.02656494,76.02970998)
\curveto(990.13656972,76.07970255)(990.24656961,76.11470252)(990.35656494,76.13470998)
\curveto(990.46656939,76.15470248)(990.58156928,76.17970245)(990.70156494,76.20970998)
\curveto(990.73156913,76.21970241)(990.77656908,76.22470241)(990.83656494,76.22470998)
\curveto(990.89656896,76.24470239)(990.96656889,76.25470238)(991.04656494,76.25470998)
\curveto(991.11656874,76.25470238)(991.18156868,76.26470237)(991.24156494,76.28470998)
\lineto(991.40656494,76.28470998)
\curveto(991.4565684,76.29470234)(991.52656833,76.29970233)(991.61656494,76.29970998)
\curveto(991.70656815,76.29970233)(991.77656808,76.28970234)(991.82656494,76.26970998)
\curveto(991.88656797,76.24970238)(991.94656791,76.24470239)(992.00656494,76.25470998)
\curveto(992.0565678,76.26470237)(992.10656775,76.25970237)(992.15656494,76.23970998)
\curveto(992.31656754,76.19970243)(992.46656739,76.16470247)(992.60656494,76.13470998)
\curveto(992.74656711,76.10470253)(992.88156698,76.05970257)(993.01156494,75.99970998)
\curveto(993.38156648,75.83970279)(993.71656614,75.61970301)(994.01656494,75.33970998)
\curveto(994.31656554,75.05970357)(994.54656531,74.73970389)(994.70656494,74.37970998)
\curveto(994.78656507,74.20970442)(994.861565,74.00970462)(994.93156494,73.77970998)
\curveto(994.97156489,73.66970496)(994.99656486,73.55470508)(995.00656494,73.43470998)
\curveto(995.01656484,73.31470532)(995.03656482,73.19470544)(995.06656494,73.07470998)
\curveto(995.08656477,73.02470561)(995.08656477,72.96970566)(995.06656494,72.90970998)
\curveto(995.0565648,72.84970578)(995.0615648,72.78970584)(995.08156494,72.72970998)
\curveto(995.10156476,72.629706)(995.10156476,72.5297061)(995.08156494,72.42970998)
\lineto(995.08156494,72.29470998)
\curveto(995.0615648,72.24470639)(995.05156481,72.18470645)(995.05156494,72.11470998)
\curveto(995.0615648,72.05470658)(995.0565648,71.99970663)(995.03656494,71.94970998)
\curveto(995.02656483,71.90970672)(995.02156484,71.87470676)(995.02156494,71.84470998)
\curveto(995.02156484,71.81470682)(995.01656484,71.77970685)(995.00656494,71.73970998)
\lineto(994.94656494,71.46970998)
\curveto(994.92656493,71.37970725)(994.89656496,71.29470734)(994.85656494,71.21470998)
\curveto(994.71656514,70.87470776)(994.5615653,70.58470805)(994.39156494,70.34470998)
\curveto(994.21156565,70.10470853)(993.98156588,69.88470875)(993.70156494,69.68470998)
\curveto(993.47156639,69.5347091)(993.23156663,69.41970921)(992.98156494,69.33970998)
\curveto(992.93156693,69.31970931)(992.88656697,69.30970932)(992.84656494,69.30970998)
\curveto(992.79656706,69.30970932)(992.74656711,69.29970933)(992.69656494,69.27970998)
\curveto(992.63656722,69.25970937)(992.5565673,69.24470939)(992.45656494,69.23470998)
\curveto(992.3565675,69.2347094)(992.28156758,69.25470938)(992.23156494,69.29470998)
\curveto(992.15156771,69.34470929)(992.10656775,69.42470921)(992.09656494,69.53470998)
\curveto(992.08656777,69.64470899)(992.08156778,69.75970887)(992.08156494,69.87970998)
\lineto(992.08156494,70.04470998)
\curveto(992.08156778,70.10470853)(992.09156777,70.15970847)(992.11156494,70.20970998)
\curveto(992.13156773,70.29970833)(992.17156769,70.36970826)(992.23156494,70.41970998)
\curveto(992.32156754,70.48970814)(992.43156743,70.5347081)(992.56156494,70.55470998)
\curveto(992.68156718,70.58470805)(992.78656707,70.629708)(992.87656494,70.68970998)
\curveto(993.21656664,70.87970775)(993.48656637,71.13970749)(993.68656494,71.46970998)
\curveto(993.74656611,71.56970706)(993.79656606,71.67470696)(993.83656494,71.78470998)
\curveto(993.86656599,71.90470673)(993.90156596,72.02470661)(993.94156494,72.14470998)
\curveto(993.99156587,72.31470632)(994.01156585,72.51970611)(994.00156494,72.75970998)
\curveto(993.98156588,73.00970562)(993.94656591,73.20970542)(993.89656494,73.35970998)
\curveto(993.77656608,73.7297049)(993.61656624,74.01970461)(993.41656494,74.22970998)
\curveto(993.20656665,74.44970418)(992.92656693,74.629704)(992.57656494,74.76970998)
\curveto(992.47656738,74.81970381)(992.37156749,74.84970378)(992.26156494,74.85970998)
\curveto(992.15156771,74.87970375)(992.03656782,74.90470373)(991.91656494,74.93470998)
\lineto(991.81156494,74.93470998)
\curveto(991.77156809,74.94470369)(991.73156813,74.94970368)(991.69156494,74.94970998)
\curveto(991.6615682,74.95970367)(991.62656823,74.95970367)(991.58656494,74.94970998)
\lineto(991.46656494,74.94970998)
\curveto(991.20656865,74.94970368)(990.9615689,74.91970371)(990.73156494,74.85970998)
\curveto(990.38156948,74.74970388)(990.08656977,74.59470404)(989.84656494,74.39470998)
\curveto(989.59657026,74.19470444)(989.40157046,73.9347047)(989.26156494,73.61470998)
\lineto(989.20156494,73.43470998)
\curveto(989.18157068,73.38470525)(989.1615707,73.32470531)(989.14156494,73.25470998)
\curveto(989.12157074,73.20470543)(989.11157075,73.14470549)(989.11156494,73.07470998)
\curveto(989.10157076,73.01470562)(989.08657077,72.94970568)(989.06656494,72.87970998)
\lineto(989.06656494,72.72970998)
\curveto(989.04657081,72.68970594)(989.03657082,72.634706)(989.03656494,72.56470998)
\curveto(989.03657082,72.50470613)(989.04657081,72.44970618)(989.06656494,72.39970998)
\lineto(989.06656494,72.29470998)
\curveto(989.06657079,72.26470637)(989.07157079,72.2297064)(989.08156494,72.18970998)
\lineto(989.14156494,71.94970998)
\curveto(989.15157071,71.86970676)(989.17157069,71.78970684)(989.20156494,71.70970998)
\curveto(989.30157056,71.46970716)(989.43657042,71.23970739)(989.60656494,71.01970998)
\curveto(989.67657018,70.9297077)(989.75157011,70.84470779)(989.83156494,70.76470998)
\curveto(989.90156996,70.68470795)(989.9565699,70.58470805)(989.99656494,70.46470998)
\curveto(990.02656983,70.37470826)(990.03656982,70.2347084)(990.02656494,70.04470998)
\curveto(990.01656984,69.86470877)(989.99156987,69.74470889)(989.95156494,69.68470998)
\curveto(989.91156995,69.634709)(989.85157001,69.59470904)(989.77156494,69.56470998)
\curveto(989.69157017,69.54470909)(989.60657025,69.54470909)(989.51656494,69.56470998)
\curveto(989.39657046,69.59470904)(989.27657058,69.61470902)(989.15656494,69.62470998)
\curveto(989.02657083,69.64470899)(988.90157096,69.66970896)(988.78156494,69.69970998)
\curveto(988.74157112,69.71970891)(988.70657115,69.72470891)(988.67656494,69.71470998)
\curveto(988.63657122,69.71470892)(988.59157127,69.72470891)(988.54156494,69.74470998)
\curveto(988.45157141,69.76470887)(988.3615715,69.77970885)(988.27156494,69.78970998)
\curveto(988.17157169,69.79970883)(988.07657178,69.81970881)(987.98656494,69.84970998)
\curveto(987.92657193,69.85970877)(987.86657199,69.86470877)(987.80656494,69.86470998)
\curveto(987.74657211,69.87470876)(987.68657217,69.88970874)(987.62656494,69.90970998)
\curveto(987.42657243,69.95970867)(987.22157264,69.99470864)(987.01156494,70.01470998)
\curveto(986.79157307,70.04470859)(986.58157328,70.08470855)(986.38156494,70.13470998)
\curveto(986.28157358,70.16470847)(986.18157368,70.18470845)(986.08156494,70.19470998)
\curveto(985.98157388,70.20470843)(985.88157398,70.21970841)(985.78156494,70.23970998)
\curveto(985.75157411,70.24970838)(985.71157415,70.25470838)(985.66156494,70.25470998)
\curveto(985.55157431,70.28470835)(985.44657441,70.30470833)(985.34656494,70.31470998)
\curveto(985.23657462,70.3347083)(985.12657473,70.35970827)(985.01656494,70.38970998)
\curveto(984.93657492,70.40970822)(984.86657499,70.42470821)(984.80656494,70.43470998)
\curveto(984.73657512,70.44470819)(984.67657518,70.46970816)(984.62656494,70.50970998)
\curveto(984.59657526,70.5297081)(984.57657528,70.55970807)(984.56656494,70.59970998)
\curveto(984.54657531,70.63970799)(984.52657533,70.68470795)(984.50656494,70.73470998)
\curveto(984.50657535,70.79470784)(984.50157536,70.8347078)(984.49156494,70.85470998)
}
}
{
\newrgbcolor{curcolor}{0 0 0}
\pscustom[linestyle=none,fillstyle=solid,fillcolor=curcolor]
{
\newpath
\moveto(993.26656494,78.63431936)
\lineto(993.26656494,79.26431936)
\lineto(993.26656494,79.45931936)
\curveto(993.26656659,79.52931683)(993.27656658,79.58931677)(993.29656494,79.63931936)
\curveto(993.33656652,79.70931665)(993.37656648,79.7593166)(993.41656494,79.78931936)
\curveto(993.46656639,79.82931653)(993.53156633,79.84931651)(993.61156494,79.84931936)
\curveto(993.69156617,79.8593165)(993.77656608,79.86431649)(993.86656494,79.86431936)
\lineto(994.58656494,79.86431936)
\curveto(995.06656479,79.86431649)(995.47656438,79.80431655)(995.81656494,79.68431936)
\curveto(996.1565637,79.56431679)(996.43156343,79.36931699)(996.64156494,79.09931936)
\curveto(996.69156317,79.02931733)(996.73656312,78.9593174)(996.77656494,78.88931936)
\curveto(996.82656303,78.82931753)(996.87156299,78.7543176)(996.91156494,78.66431936)
\curveto(996.92156294,78.64431771)(996.93156293,78.61431774)(996.94156494,78.57431936)
\curveto(996.9615629,78.53431782)(996.96656289,78.48931787)(996.95656494,78.43931936)
\curveto(996.92656293,78.34931801)(996.85156301,78.29431806)(996.73156494,78.27431936)
\curveto(996.62156324,78.2543181)(996.52656333,78.26931809)(996.44656494,78.31931936)
\curveto(996.37656348,78.34931801)(996.31156355,78.39431796)(996.25156494,78.45431936)
\curveto(996.20156366,78.52431783)(996.15156371,78.58931777)(996.10156494,78.64931936)
\curveto(996.05156381,78.71931764)(995.97656388,78.77931758)(995.87656494,78.82931936)
\curveto(995.78656407,78.88931747)(995.69656416,78.93931742)(995.60656494,78.97931936)
\curveto(995.57656428,78.99931736)(995.51656434,79.02431733)(995.42656494,79.05431936)
\curveto(995.34656451,79.08431727)(995.27656458,79.08931727)(995.21656494,79.06931936)
\curveto(995.07656478,79.03931732)(994.98656487,78.97931738)(994.94656494,78.88931936)
\curveto(994.91656494,78.80931755)(994.90156496,78.71931764)(994.90156494,78.61931936)
\curveto(994.90156496,78.51931784)(994.87656498,78.43431792)(994.82656494,78.36431936)
\curveto(994.7565651,78.27431808)(994.61656524,78.22931813)(994.40656494,78.22931936)
\lineto(993.85156494,78.22931936)
\lineto(993.62656494,78.22931936)
\curveto(993.54656631,78.23931812)(993.48156638,78.2593181)(993.43156494,78.28931936)
\curveto(993.35156651,78.34931801)(993.30656655,78.41931794)(993.29656494,78.49931936)
\curveto(993.28656657,78.51931784)(993.28156658,78.53931782)(993.28156494,78.55931936)
\curveto(993.28156658,78.58931777)(993.27656658,78.61431774)(993.26656494,78.63431936)
}
}
{
\newrgbcolor{curcolor}{0 0 0}
\pscustom[linestyle=none,fillstyle=solid,fillcolor=curcolor]
{
}
}
{
\newrgbcolor{curcolor}{0 0 0}
\pscustom[linestyle=none,fillstyle=solid,fillcolor=curcolor]
{
\newpath
\moveto(984.29656494,89.26463186)
\curveto(984.28657557,89.95462722)(984.40657545,90.55462662)(984.65656494,91.06463186)
\curveto(984.90657495,91.58462559)(985.24157462,91.9796252)(985.66156494,92.24963186)
\curveto(985.74157412,92.29962488)(985.83157403,92.34462483)(985.93156494,92.38463186)
\curveto(986.02157384,92.42462475)(986.11657374,92.46962471)(986.21656494,92.51963186)
\curveto(986.31657354,92.55962462)(986.41657344,92.58962459)(986.51656494,92.60963186)
\curveto(986.61657324,92.62962455)(986.72157314,92.64962453)(986.83156494,92.66963186)
\curveto(986.88157298,92.68962449)(986.92657293,92.69462448)(986.96656494,92.68463186)
\curveto(987.00657285,92.6746245)(987.05157281,92.6796245)(987.10156494,92.69963186)
\curveto(987.15157271,92.70962447)(987.23657262,92.71462446)(987.35656494,92.71463186)
\curveto(987.46657239,92.71462446)(987.55157231,92.70962447)(987.61156494,92.69963186)
\curveto(987.67157219,92.6796245)(987.73157213,92.66962451)(987.79156494,92.66963186)
\curveto(987.85157201,92.6796245)(987.91157195,92.6746245)(987.97156494,92.65463186)
\curveto(988.11157175,92.61462456)(988.24657161,92.5796246)(988.37656494,92.54963186)
\curveto(988.50657135,92.51962466)(988.63157123,92.4796247)(988.75156494,92.42963186)
\curveto(988.89157097,92.36962481)(989.01657084,92.29962488)(989.12656494,92.21963186)
\curveto(989.23657062,92.14962503)(989.34657051,92.0746251)(989.45656494,91.99463186)
\lineto(989.51656494,91.93463186)
\curveto(989.53657032,91.92462525)(989.5565703,91.90962527)(989.57656494,91.88963186)
\curveto(989.73657012,91.76962541)(989.88156998,91.63462554)(990.01156494,91.48463186)
\curveto(990.14156972,91.33462584)(990.26656959,91.174626)(990.38656494,91.00463186)
\curveto(990.60656925,90.69462648)(990.81156905,90.39962678)(991.00156494,90.11963186)
\curveto(991.14156872,89.88962729)(991.27656858,89.65962752)(991.40656494,89.42963186)
\curveto(991.53656832,89.20962797)(991.67156819,88.98962819)(991.81156494,88.76963186)
\curveto(991.98156788,88.51962866)(992.1615677,88.2796289)(992.35156494,88.04963186)
\curveto(992.54156732,87.82962935)(992.76656709,87.63962954)(993.02656494,87.47963186)
\curveto(993.08656677,87.43962974)(993.14656671,87.40462977)(993.20656494,87.37463186)
\curveto(993.2565666,87.34462983)(993.32156654,87.31462986)(993.40156494,87.28463186)
\curveto(993.47156639,87.26462991)(993.53156633,87.25962992)(993.58156494,87.26963186)
\curveto(993.65156621,87.28962989)(993.70656615,87.32462985)(993.74656494,87.37463186)
\curveto(993.77656608,87.42462975)(993.79656606,87.48462969)(993.80656494,87.55463186)
\lineto(993.80656494,87.79463186)
\lineto(993.80656494,88.54463186)
\lineto(993.80656494,91.34963186)
\lineto(993.80656494,92.00963186)
\curveto(993.80656605,92.09962508)(993.81156605,92.18462499)(993.82156494,92.26463186)
\curveto(993.82156604,92.34462483)(993.84156602,92.40962477)(993.88156494,92.45963186)
\curveto(993.92156594,92.50962467)(993.99656586,92.54962463)(994.10656494,92.57963186)
\curveto(994.20656565,92.61962456)(994.30656555,92.62962455)(994.40656494,92.60963186)
\lineto(994.54156494,92.60963186)
\curveto(994.61156525,92.58962459)(994.67156519,92.56962461)(994.72156494,92.54963186)
\curveto(994.77156509,92.52962465)(994.81156505,92.49462468)(994.84156494,92.44463186)
\curveto(994.88156498,92.39462478)(994.90156496,92.32462485)(994.90156494,92.23463186)
\lineto(994.90156494,91.96463186)
\lineto(994.90156494,91.06463186)
\lineto(994.90156494,87.55463186)
\lineto(994.90156494,86.48963186)
\curveto(994.90156496,86.40963077)(994.90656495,86.31963086)(994.91656494,86.21963186)
\curveto(994.91656494,86.11963106)(994.90656495,86.03463114)(994.88656494,85.96463186)
\curveto(994.81656504,85.75463142)(994.63656522,85.68963149)(994.34656494,85.76963186)
\curveto(994.30656555,85.7796314)(994.27156559,85.7796314)(994.24156494,85.76963186)
\curveto(994.20156566,85.76963141)(994.1565657,85.7796314)(994.10656494,85.79963186)
\curveto(994.02656583,85.81963136)(993.94156592,85.83963134)(993.85156494,85.85963186)
\curveto(993.7615661,85.8796313)(993.67656618,85.90463127)(993.59656494,85.93463186)
\curveto(993.10656675,86.09463108)(992.69156717,86.29463088)(992.35156494,86.53463186)
\curveto(992.10156776,86.71463046)(991.87656798,86.91963026)(991.67656494,87.14963186)
\curveto(991.46656839,87.3796298)(991.27156859,87.61962956)(991.09156494,87.86963186)
\curveto(990.91156895,88.12962905)(990.74156912,88.39462878)(990.58156494,88.66463186)
\curveto(990.41156945,88.94462823)(990.23656962,89.21462796)(990.05656494,89.47463186)
\curveto(989.97656988,89.58462759)(989.90156996,89.68962749)(989.83156494,89.78963186)
\curveto(989.7615701,89.89962728)(989.68657017,90.00962717)(989.60656494,90.11963186)
\curveto(989.57657028,90.15962702)(989.54657031,90.19462698)(989.51656494,90.22463186)
\curveto(989.47657038,90.26462691)(989.44657041,90.30462687)(989.42656494,90.34463186)
\curveto(989.31657054,90.48462669)(989.19157067,90.60962657)(989.05156494,90.71963186)
\curveto(989.02157084,90.73962644)(988.99657086,90.76462641)(988.97656494,90.79463186)
\curveto(988.94657091,90.82462635)(988.91657094,90.84962633)(988.88656494,90.86963186)
\curveto(988.78657107,90.94962623)(988.68657117,91.01462616)(988.58656494,91.06463186)
\curveto(988.48657137,91.12462605)(988.37657148,91.179626)(988.25656494,91.22963186)
\curveto(988.18657167,91.25962592)(988.11157175,91.2796259)(988.03156494,91.28963186)
\lineto(987.79156494,91.34963186)
\lineto(987.70156494,91.34963186)
\curveto(987.67157219,91.35962582)(987.64157222,91.36462581)(987.61156494,91.36463186)
\curveto(987.54157232,91.38462579)(987.44657241,91.38962579)(987.32656494,91.37963186)
\curveto(987.19657266,91.3796258)(987.09657276,91.36962581)(987.02656494,91.34963186)
\curveto(986.94657291,91.32962585)(986.87157299,91.30962587)(986.80156494,91.28963186)
\curveto(986.72157314,91.2796259)(986.64157322,91.25962592)(986.56156494,91.22963186)
\curveto(986.32157354,91.11962606)(986.12157374,90.96962621)(985.96156494,90.77963186)
\curveto(985.79157407,90.59962658)(985.65157421,90.3796268)(985.54156494,90.11963186)
\curveto(985.52157434,90.04962713)(985.50657435,89.9796272)(985.49656494,89.90963186)
\curveto(985.47657438,89.83962734)(985.4565744,89.76462741)(985.43656494,89.68463186)
\curveto(985.41657444,89.60462757)(985.40657445,89.49462768)(985.40656494,89.35463186)
\curveto(985.40657445,89.22462795)(985.41657444,89.11962806)(985.43656494,89.03963186)
\curveto(985.44657441,88.9796282)(985.45157441,88.92462825)(985.45156494,88.87463186)
\curveto(985.45157441,88.82462835)(985.4615744,88.7746284)(985.48156494,88.72463186)
\curveto(985.52157434,88.62462855)(985.5615743,88.52962865)(985.60156494,88.43963186)
\curveto(985.64157422,88.35962882)(985.68657417,88.2796289)(985.73656494,88.19963186)
\curveto(985.7565741,88.16962901)(985.78157408,88.13962904)(985.81156494,88.10963186)
\curveto(985.84157402,88.08962909)(985.86657399,88.06462911)(985.88656494,88.03463186)
\lineto(985.96156494,87.95963186)
\curveto(985.98157388,87.92962925)(986.00157386,87.90462927)(986.02156494,87.88463186)
\lineto(986.23156494,87.73463186)
\curveto(986.29157357,87.69462948)(986.3565735,87.64962953)(986.42656494,87.59963186)
\curveto(986.51657334,87.53962964)(986.62157324,87.48962969)(986.74156494,87.44963186)
\curveto(986.85157301,87.41962976)(986.9615729,87.38462979)(987.07156494,87.34463186)
\curveto(987.18157268,87.30462987)(987.32657253,87.2796299)(987.50656494,87.26963186)
\curveto(987.67657218,87.25962992)(987.80157206,87.22962995)(987.88156494,87.17963186)
\curveto(987.9615719,87.12963005)(988.00657185,87.05463012)(988.01656494,86.95463186)
\curveto(988.02657183,86.85463032)(988.03157183,86.74463043)(988.03156494,86.62463186)
\curveto(988.03157183,86.58463059)(988.03657182,86.54463063)(988.04656494,86.50463186)
\curveto(988.04657181,86.46463071)(988.04157182,86.42963075)(988.03156494,86.39963186)
\curveto(988.01157185,86.34963083)(988.00157186,86.29963088)(988.00156494,86.24963186)
\curveto(988.00157186,86.20963097)(987.99157187,86.16963101)(987.97156494,86.12963186)
\curveto(987.91157195,86.03963114)(987.77657208,85.99463118)(987.56656494,85.99463186)
\lineto(987.44656494,85.99463186)
\curveto(987.38657247,86.00463117)(987.32657253,86.00963117)(987.26656494,86.00963186)
\curveto(987.19657266,86.01963116)(987.13157273,86.02963115)(987.07156494,86.03963186)
\curveto(986.9615729,86.05963112)(986.861573,86.0796311)(986.77156494,86.09963186)
\curveto(986.67157319,86.11963106)(986.57657328,86.14963103)(986.48656494,86.18963186)
\curveto(986.41657344,86.20963097)(986.3565735,86.22963095)(986.30656494,86.24963186)
\lineto(986.12656494,86.30963186)
\curveto(985.86657399,86.42963075)(985.62157424,86.58463059)(985.39156494,86.77463186)
\curveto(985.1615747,86.9746302)(984.97657488,87.18962999)(984.83656494,87.41963186)
\curveto(984.7565751,87.52962965)(984.69157517,87.64462953)(984.64156494,87.76463186)
\lineto(984.49156494,88.15463186)
\curveto(984.44157542,88.26462891)(984.41157545,88.3796288)(984.40156494,88.49963186)
\curveto(984.38157548,88.61962856)(984.3565755,88.74462843)(984.32656494,88.87463186)
\curveto(984.32657553,88.94462823)(984.32657553,89.00962817)(984.32656494,89.06963186)
\curveto(984.31657554,89.12962805)(984.30657555,89.19462798)(984.29656494,89.26463186)
}
}
{
\newrgbcolor{curcolor}{0 0 0}
\pscustom[linestyle=none,fillstyle=solid,fillcolor=curcolor]
{
\newpath
\moveto(989.81656494,101.36424123)
\lineto(990.07156494,101.36424123)
\curveto(990.15156971,101.37423353)(990.22656963,101.36923353)(990.29656494,101.34924123)
\lineto(990.53656494,101.34924123)
\lineto(990.70156494,101.34924123)
\curveto(990.80156906,101.32923357)(990.90656895,101.31923358)(991.01656494,101.31924123)
\curveto(991.11656874,101.31923358)(991.21656864,101.30923359)(991.31656494,101.28924123)
\lineto(991.46656494,101.28924123)
\curveto(991.60656825,101.25923364)(991.74656811,101.23923366)(991.88656494,101.22924123)
\curveto(992.01656784,101.21923368)(992.14656771,101.19423371)(992.27656494,101.15424123)
\curveto(992.3565675,101.13423377)(992.44156742,101.11423379)(992.53156494,101.09424123)
\lineto(992.77156494,101.03424123)
\lineto(993.07156494,100.91424123)
\curveto(993.1615667,100.88423402)(993.25156661,100.84923405)(993.34156494,100.80924123)
\curveto(993.5615663,100.70923419)(993.77656608,100.57423433)(993.98656494,100.40424123)
\curveto(994.19656566,100.24423466)(994.36656549,100.06923483)(994.49656494,99.87924123)
\curveto(994.53656532,99.82923507)(994.57656528,99.76923513)(994.61656494,99.69924123)
\curveto(994.64656521,99.63923526)(994.68156518,99.57923532)(994.72156494,99.51924123)
\curveto(994.77156509,99.43923546)(994.81156505,99.34423556)(994.84156494,99.23424123)
\curveto(994.87156499,99.12423578)(994.90156496,99.01923588)(994.93156494,98.91924123)
\curveto(994.97156489,98.80923609)(994.99656486,98.6992362)(995.00656494,98.58924123)
\curveto(995.01656484,98.47923642)(995.03156483,98.36423654)(995.05156494,98.24424123)
\curveto(995.0615648,98.2042367)(995.0615648,98.15923674)(995.05156494,98.10924123)
\curveto(995.05156481,98.06923683)(995.0565648,98.02923687)(995.06656494,97.98924123)
\curveto(995.07656478,97.94923695)(995.08156478,97.89423701)(995.08156494,97.82424123)
\curveto(995.08156478,97.75423715)(995.07656478,97.7042372)(995.06656494,97.67424123)
\curveto(995.04656481,97.62423728)(995.04156482,97.57923732)(995.05156494,97.53924123)
\curveto(995.0615648,97.4992374)(995.0615648,97.46423744)(995.05156494,97.43424123)
\lineto(995.05156494,97.34424123)
\curveto(995.03156483,97.28423762)(995.01656484,97.21923768)(995.00656494,97.14924123)
\curveto(995.00656485,97.08923781)(995.00156486,97.02423788)(994.99156494,96.95424123)
\curveto(994.94156492,96.78423812)(994.89156497,96.62423828)(994.84156494,96.47424123)
\curveto(994.79156507,96.32423858)(994.72656513,96.17923872)(994.64656494,96.03924123)
\curveto(994.60656525,95.98923891)(994.57656528,95.93423897)(994.55656494,95.87424123)
\curveto(994.52656533,95.82423908)(994.49156537,95.77423913)(994.45156494,95.72424123)
\curveto(994.27156559,95.48423942)(994.05156581,95.28423962)(993.79156494,95.12424123)
\curveto(993.53156633,94.96423994)(993.24656661,94.82424008)(992.93656494,94.70424123)
\curveto(992.79656706,94.64424026)(992.6565672,94.5992403)(992.51656494,94.56924123)
\curveto(992.36656749,94.53924036)(992.21156765,94.5042404)(992.05156494,94.46424123)
\curveto(991.94156792,94.44424046)(991.83156803,94.42924047)(991.72156494,94.41924123)
\curveto(991.61156825,94.40924049)(991.50156836,94.39424051)(991.39156494,94.37424123)
\curveto(991.35156851,94.36424054)(991.31156855,94.35924054)(991.27156494,94.35924123)
\curveto(991.23156863,94.36924053)(991.19156867,94.36924053)(991.15156494,94.35924123)
\curveto(991.10156876,94.34924055)(991.05156881,94.34424056)(991.00156494,94.34424123)
\lineto(990.83656494,94.34424123)
\curveto(990.78656907,94.32424058)(990.73656912,94.31924058)(990.68656494,94.32924123)
\curveto(990.62656923,94.33924056)(990.57156929,94.33924056)(990.52156494,94.32924123)
\curveto(990.48156938,94.31924058)(990.43656942,94.31924058)(990.38656494,94.32924123)
\curveto(990.33656952,94.33924056)(990.28656957,94.33424057)(990.23656494,94.31424123)
\curveto(990.16656969,94.29424061)(990.09156977,94.28924061)(990.01156494,94.29924123)
\curveto(989.92156994,94.30924059)(989.83657002,94.31424059)(989.75656494,94.31424123)
\curveto(989.66657019,94.31424059)(989.56657029,94.30924059)(989.45656494,94.29924123)
\curveto(989.33657052,94.28924061)(989.23657062,94.29424061)(989.15656494,94.31424123)
\lineto(988.87156494,94.31424123)
\lineto(988.24156494,94.35924123)
\curveto(988.14157172,94.36924053)(988.04657181,94.37924052)(987.95656494,94.38924123)
\lineto(987.65656494,94.41924123)
\curveto(987.60657225,94.43924046)(987.5565723,94.44424046)(987.50656494,94.43424123)
\curveto(987.44657241,94.43424047)(987.39157247,94.44424046)(987.34156494,94.46424123)
\curveto(987.17157269,94.51424039)(987.00657285,94.55424035)(986.84656494,94.58424123)
\curveto(986.67657318,94.61424029)(986.51657334,94.66424024)(986.36656494,94.73424123)
\curveto(985.90657395,94.92423998)(985.53157433,95.14423976)(985.24156494,95.39424123)
\curveto(984.95157491,95.65423925)(984.70657515,96.01423889)(984.50656494,96.47424123)
\curveto(984.4565754,96.6042383)(984.42157544,96.73423817)(984.40156494,96.86424123)
\curveto(984.38157548,97.0042379)(984.3565755,97.14423776)(984.32656494,97.28424123)
\curveto(984.31657554,97.35423755)(984.31157555,97.41923748)(984.31156494,97.47924123)
\curveto(984.31157555,97.53923736)(984.30657555,97.6042373)(984.29656494,97.67424123)
\curveto(984.27657558,98.5042364)(984.42657543,99.17423573)(984.74656494,99.68424123)
\curveto(985.0565748,100.19423471)(985.49657436,100.57423433)(986.06656494,100.82424123)
\curveto(986.18657367,100.87423403)(986.31157355,100.91923398)(986.44156494,100.95924123)
\curveto(986.57157329,100.9992339)(986.70657315,101.04423386)(986.84656494,101.09424123)
\curveto(986.92657293,101.11423379)(987.01157285,101.12923377)(987.10156494,101.13924123)
\lineto(987.34156494,101.19924123)
\curveto(987.45157241,101.22923367)(987.5615723,101.24423366)(987.67156494,101.24424123)
\curveto(987.78157208,101.25423365)(987.89157197,101.26923363)(988.00156494,101.28924123)
\curveto(988.05157181,101.30923359)(988.09657176,101.31423359)(988.13656494,101.30424123)
\curveto(988.17657168,101.3042336)(988.21657164,101.30923359)(988.25656494,101.31924123)
\curveto(988.30657155,101.32923357)(988.3615715,101.32923357)(988.42156494,101.31924123)
\curveto(988.47157139,101.31923358)(988.52157134,101.32423358)(988.57156494,101.33424123)
\lineto(988.70656494,101.33424123)
\curveto(988.76657109,101.35423355)(988.83657102,101.35423355)(988.91656494,101.33424123)
\curveto(988.98657087,101.32423358)(989.05157081,101.32923357)(989.11156494,101.34924123)
\curveto(989.14157072,101.35923354)(989.18157068,101.36423354)(989.23156494,101.36424123)
\lineto(989.35156494,101.36424123)
\lineto(989.81656494,101.36424123)
\moveto(992.14156494,99.81924123)
\curveto(991.82156804,99.91923498)(991.4565684,99.97923492)(991.04656494,99.99924123)
\curveto(990.63656922,100.01923488)(990.22656963,100.02923487)(989.81656494,100.02924123)
\curveto(989.38657047,100.02923487)(988.96657089,100.01923488)(988.55656494,99.99924123)
\curveto(988.14657171,99.97923492)(987.7615721,99.93423497)(987.40156494,99.86424123)
\curveto(987.04157282,99.79423511)(986.72157314,99.68423522)(986.44156494,99.53424123)
\curveto(986.15157371,99.39423551)(985.91657394,99.1992357)(985.73656494,98.94924123)
\curveto(985.62657423,98.78923611)(985.54657431,98.60923629)(985.49656494,98.40924123)
\curveto(985.43657442,98.20923669)(985.40657445,97.96423694)(985.40656494,97.67424123)
\curveto(985.42657443,97.65423725)(985.43657442,97.61923728)(985.43656494,97.56924123)
\curveto(985.42657443,97.51923738)(985.42657443,97.47923742)(985.43656494,97.44924123)
\curveto(985.4565744,97.36923753)(985.47657438,97.29423761)(985.49656494,97.22424123)
\curveto(985.50657435,97.16423774)(985.52657433,97.0992378)(985.55656494,97.02924123)
\curveto(985.67657418,96.75923814)(985.84657401,96.53923836)(986.06656494,96.36924123)
\curveto(986.27657358,96.20923869)(986.52157334,96.07423883)(986.80156494,95.96424123)
\curveto(986.91157295,95.91423899)(987.03157283,95.87423903)(987.16156494,95.84424123)
\curveto(987.28157258,95.82423908)(987.40657245,95.7992391)(987.53656494,95.76924123)
\curveto(987.58657227,95.74923915)(987.64157222,95.73923916)(987.70156494,95.73924123)
\curveto(987.75157211,95.73923916)(987.80157206,95.73423917)(987.85156494,95.72424123)
\curveto(987.94157192,95.71423919)(988.03657182,95.7042392)(988.13656494,95.69424123)
\curveto(988.22657163,95.68423922)(988.32157154,95.67423923)(988.42156494,95.66424123)
\curveto(988.50157136,95.66423924)(988.58657127,95.65923924)(988.67656494,95.64924123)
\lineto(988.91656494,95.64924123)
\lineto(989.09656494,95.64924123)
\curveto(989.12657073,95.63923926)(989.1615707,95.63423927)(989.20156494,95.63424123)
\lineto(989.33656494,95.63424123)
\lineto(989.78656494,95.63424123)
\curveto(989.86656999,95.63423927)(989.95156991,95.62923927)(990.04156494,95.61924123)
\curveto(990.12156974,95.61923928)(990.19656966,95.62923927)(990.26656494,95.64924123)
\lineto(990.53656494,95.64924123)
\curveto(990.5565693,95.64923925)(990.58656927,95.64423926)(990.62656494,95.63424123)
\curveto(990.6565692,95.63423927)(990.68156918,95.63923926)(990.70156494,95.64924123)
\curveto(990.80156906,95.65923924)(990.90156896,95.66423924)(991.00156494,95.66424123)
\curveto(991.09156877,95.67423923)(991.19156867,95.68423922)(991.30156494,95.69424123)
\curveto(991.42156844,95.72423918)(991.54656831,95.73923916)(991.67656494,95.73924123)
\curveto(991.79656806,95.74923915)(991.91156795,95.77423913)(992.02156494,95.81424123)
\curveto(992.32156754,95.89423901)(992.58656727,95.97923892)(992.81656494,96.06924123)
\curveto(993.04656681,96.16923873)(993.2615666,96.31423859)(993.46156494,96.50424123)
\curveto(993.6615662,96.71423819)(993.81156605,96.97923792)(993.91156494,97.29924123)
\curveto(993.93156593,97.33923756)(993.94156592,97.37423753)(993.94156494,97.40424123)
\curveto(993.93156593,97.44423746)(993.93656592,97.48923741)(993.95656494,97.53924123)
\curveto(993.96656589,97.57923732)(993.97656588,97.64923725)(993.98656494,97.74924123)
\curveto(993.99656586,97.85923704)(993.99156587,97.94423696)(993.97156494,98.00424123)
\curveto(993.95156591,98.07423683)(993.94156592,98.14423676)(993.94156494,98.21424123)
\curveto(993.93156593,98.28423662)(993.91656594,98.34923655)(993.89656494,98.40924123)
\curveto(993.83656602,98.60923629)(993.75156611,98.78923611)(993.64156494,98.94924123)
\curveto(993.62156624,98.97923592)(993.60156626,99.0042359)(993.58156494,99.02424123)
\lineto(993.52156494,99.08424123)
\curveto(993.50156636,99.12423578)(993.4615664,99.17423573)(993.40156494,99.23424123)
\curveto(993.2615666,99.33423557)(993.13156673,99.41923548)(993.01156494,99.48924123)
\curveto(992.89156697,99.55923534)(992.74656711,99.62923527)(992.57656494,99.69924123)
\curveto(992.50656735,99.72923517)(992.43656742,99.74923515)(992.36656494,99.75924123)
\curveto(992.29656756,99.77923512)(992.22156764,99.7992351)(992.14156494,99.81924123)
}
}
{
\newrgbcolor{curcolor}{0 0 0}
\pscustom[linestyle=none,fillstyle=solid,fillcolor=curcolor]
{
\newpath
\moveto(984.29656494,106.77385061)
\curveto(984.29657556,106.87384575)(984.30657555,106.96884566)(984.32656494,107.05885061)
\curveto(984.33657552,107.14884548)(984.36657549,107.21384541)(984.41656494,107.25385061)
\curveto(984.49657536,107.31384531)(984.60157526,107.34384528)(984.73156494,107.34385061)
\lineto(985.12156494,107.34385061)
\lineto(986.62156494,107.34385061)
\lineto(993.01156494,107.34385061)
\lineto(994.18156494,107.34385061)
\lineto(994.49656494,107.34385061)
\curveto(994.59656526,107.35384527)(994.67656518,107.33884529)(994.73656494,107.29885061)
\curveto(994.81656504,107.24884538)(994.86656499,107.17384545)(994.88656494,107.07385061)
\curveto(994.89656496,106.98384564)(994.90156496,106.87384575)(994.90156494,106.74385061)
\lineto(994.90156494,106.51885061)
\curveto(994.88156498,106.43884619)(994.86656499,106.36884626)(994.85656494,106.30885061)
\curveto(994.83656502,106.24884638)(994.79656506,106.19884643)(994.73656494,106.15885061)
\curveto(994.67656518,106.11884651)(994.60156526,106.09884653)(994.51156494,106.09885061)
\lineto(994.21156494,106.09885061)
\lineto(993.11656494,106.09885061)
\lineto(987.77656494,106.09885061)
\curveto(987.68657217,106.07884655)(987.61157225,106.06384656)(987.55156494,106.05385061)
\curveto(987.48157238,106.05384657)(987.42157244,106.0238466)(987.37156494,105.96385061)
\curveto(987.32157254,105.89384673)(987.29657256,105.80384682)(987.29656494,105.69385061)
\curveto(987.28657257,105.59384703)(987.28157258,105.48384714)(987.28156494,105.36385061)
\lineto(987.28156494,104.22385061)
\lineto(987.28156494,103.72885061)
\curveto(987.27157259,103.56884906)(987.21157265,103.45884917)(987.10156494,103.39885061)
\curveto(987.07157279,103.37884925)(987.04157282,103.36884926)(987.01156494,103.36885061)
\curveto(986.97157289,103.36884926)(986.92657293,103.36384926)(986.87656494,103.35385061)
\curveto(986.7565731,103.33384929)(986.64657321,103.33884929)(986.54656494,103.36885061)
\curveto(986.44657341,103.40884922)(986.37657348,103.46384916)(986.33656494,103.53385061)
\curveto(986.28657357,103.61384901)(986.2615736,103.73384889)(986.26156494,103.89385061)
\curveto(986.2615736,104.05384857)(986.24657361,104.18884844)(986.21656494,104.29885061)
\curveto(986.20657365,104.34884828)(986.20157366,104.40384822)(986.20156494,104.46385061)
\curveto(986.19157367,104.5238481)(986.17657368,104.58384804)(986.15656494,104.64385061)
\curveto(986.10657375,104.79384783)(986.0565738,104.93884769)(986.00656494,105.07885061)
\curveto(985.94657391,105.21884741)(985.87657398,105.35384727)(985.79656494,105.48385061)
\curveto(985.70657415,105.623847)(985.60157426,105.74384688)(985.48156494,105.84385061)
\curveto(985.3615745,105.94384668)(985.23157463,106.03884659)(985.09156494,106.12885061)
\curveto(984.99157487,106.18884644)(984.88157498,106.23384639)(984.76156494,106.26385061)
\curveto(984.64157522,106.30384632)(984.53657532,106.35384627)(984.44656494,106.41385061)
\curveto(984.38657547,106.46384616)(984.34657551,106.53384609)(984.32656494,106.62385061)
\curveto(984.31657554,106.64384598)(984.31157555,106.66884596)(984.31156494,106.69885061)
\curveto(984.31157555,106.7288459)(984.30657555,106.75384587)(984.29656494,106.77385061)
}
}
{
\newrgbcolor{curcolor}{0 0 0}
\pscustom[linestyle=none,fillstyle=solid,fillcolor=curcolor]
{
\newpath
\moveto(984.29656494,115.12345998)
\curveto(984.29657556,115.22345513)(984.30657555,115.31845503)(984.32656494,115.40845998)
\curveto(984.33657552,115.49845485)(984.36657549,115.56345479)(984.41656494,115.60345998)
\curveto(984.49657536,115.66345469)(984.60157526,115.69345466)(984.73156494,115.69345998)
\lineto(985.12156494,115.69345998)
\lineto(986.62156494,115.69345998)
\lineto(993.01156494,115.69345998)
\lineto(994.18156494,115.69345998)
\lineto(994.49656494,115.69345998)
\curveto(994.59656526,115.70345465)(994.67656518,115.68845466)(994.73656494,115.64845998)
\curveto(994.81656504,115.59845475)(994.86656499,115.52345483)(994.88656494,115.42345998)
\curveto(994.89656496,115.33345502)(994.90156496,115.22345513)(994.90156494,115.09345998)
\lineto(994.90156494,114.86845998)
\curveto(994.88156498,114.78845556)(994.86656499,114.71845563)(994.85656494,114.65845998)
\curveto(994.83656502,114.59845575)(994.79656506,114.5484558)(994.73656494,114.50845998)
\curveto(994.67656518,114.46845588)(994.60156526,114.4484559)(994.51156494,114.44845998)
\lineto(994.21156494,114.44845998)
\lineto(993.11656494,114.44845998)
\lineto(987.77656494,114.44845998)
\curveto(987.68657217,114.42845592)(987.61157225,114.41345594)(987.55156494,114.40345998)
\curveto(987.48157238,114.40345595)(987.42157244,114.37345598)(987.37156494,114.31345998)
\curveto(987.32157254,114.24345611)(987.29657256,114.1534562)(987.29656494,114.04345998)
\curveto(987.28657257,113.94345641)(987.28157258,113.83345652)(987.28156494,113.71345998)
\lineto(987.28156494,112.57345998)
\lineto(987.28156494,112.07845998)
\curveto(987.27157259,111.91845843)(987.21157265,111.80845854)(987.10156494,111.74845998)
\curveto(987.07157279,111.72845862)(987.04157282,111.71845863)(987.01156494,111.71845998)
\curveto(986.97157289,111.71845863)(986.92657293,111.71345864)(986.87656494,111.70345998)
\curveto(986.7565731,111.68345867)(986.64657321,111.68845866)(986.54656494,111.71845998)
\curveto(986.44657341,111.75845859)(986.37657348,111.81345854)(986.33656494,111.88345998)
\curveto(986.28657357,111.96345839)(986.2615736,112.08345827)(986.26156494,112.24345998)
\curveto(986.2615736,112.40345795)(986.24657361,112.53845781)(986.21656494,112.64845998)
\curveto(986.20657365,112.69845765)(986.20157366,112.7534576)(986.20156494,112.81345998)
\curveto(986.19157367,112.87345748)(986.17657368,112.93345742)(986.15656494,112.99345998)
\curveto(986.10657375,113.14345721)(986.0565738,113.28845706)(986.00656494,113.42845998)
\curveto(985.94657391,113.56845678)(985.87657398,113.70345665)(985.79656494,113.83345998)
\curveto(985.70657415,113.97345638)(985.60157426,114.09345626)(985.48156494,114.19345998)
\curveto(985.3615745,114.29345606)(985.23157463,114.38845596)(985.09156494,114.47845998)
\curveto(984.99157487,114.53845581)(984.88157498,114.58345577)(984.76156494,114.61345998)
\curveto(984.64157522,114.6534557)(984.53657532,114.70345565)(984.44656494,114.76345998)
\curveto(984.38657547,114.81345554)(984.34657551,114.88345547)(984.32656494,114.97345998)
\curveto(984.31657554,114.99345536)(984.31157555,115.01845533)(984.31156494,115.04845998)
\curveto(984.31157555,115.07845527)(984.30657555,115.10345525)(984.29656494,115.12345998)
}
}
{
\newrgbcolor{curcolor}{0 0 0}
\pscustom[linestyle=none,fillstyle=solid,fillcolor=curcolor]
{
\newpath
\moveto(1016.14292725,38.71181936)
\curveto(1016.19292799,38.73180981)(1016.25292793,38.75680979)(1016.32292725,38.78681936)
\curveto(1016.39292779,38.81680973)(1016.46792772,38.83680971)(1016.54792725,38.84681936)
\curveto(1016.61792757,38.86680968)(1016.6879275,38.86680968)(1016.75792725,38.84681936)
\curveto(1016.81792737,38.83680971)(1016.86292732,38.79680975)(1016.89292725,38.72681936)
\curveto(1016.91292727,38.67680987)(1016.92292726,38.61680993)(1016.92292725,38.54681936)
\lineto(1016.92292725,38.33681936)
\lineto(1016.92292725,37.88681936)
\curveto(1016.92292726,37.73681081)(1016.89792729,37.61681093)(1016.84792725,37.52681936)
\curveto(1016.7879274,37.42681112)(1016.6829275,37.35181119)(1016.53292725,37.30181936)
\curveto(1016.3829278,37.26181128)(1016.24792794,37.21681133)(1016.12792725,37.16681936)
\curveto(1015.86792832,37.05681149)(1015.59792859,36.95681159)(1015.31792725,36.86681936)
\curveto(1015.03792915,36.77681177)(1014.76292942,36.67681187)(1014.49292725,36.56681936)
\curveto(1014.40292978,36.53681201)(1014.31792987,36.50681204)(1014.23792725,36.47681936)
\curveto(1014.15793003,36.45681209)(1014.0829301,36.42681212)(1014.01292725,36.38681936)
\curveto(1013.94293024,36.35681219)(1013.8829303,36.31181223)(1013.83292725,36.25181936)
\curveto(1013.7829304,36.19181235)(1013.74293044,36.11181243)(1013.71292725,36.01181936)
\curveto(1013.69293049,35.96181258)(1013.6879305,35.90181264)(1013.69792725,35.83181936)
\lineto(1013.69792725,35.63681936)
\lineto(1013.69792725,32.80181936)
\lineto(1013.69792725,32.50181936)
\curveto(1013.6879305,32.39181615)(1013.6879305,32.28681626)(1013.69792725,32.18681936)
\curveto(1013.70793048,32.08681646)(1013.72293046,31.99181655)(1013.74292725,31.90181936)
\curveto(1013.76293042,31.82181672)(1013.80293038,31.76181678)(1013.86292725,31.72181936)
\curveto(1013.96293022,31.6418169)(1014.07793011,31.58181696)(1014.20792725,31.54181936)
\curveto(1014.32792986,31.51181703)(1014.45292973,31.47181707)(1014.58292725,31.42181936)
\curveto(1014.81292937,31.32181722)(1015.05292913,31.22681732)(1015.30292725,31.13681936)
\curveto(1015.55292863,31.05681749)(1015.79292839,30.96681758)(1016.02292725,30.86681936)
\curveto(1016.0829281,30.8468177)(1016.15292803,30.82181772)(1016.23292725,30.79181936)
\curveto(1016.30292788,30.77181777)(1016.37792781,30.7468178)(1016.45792725,30.71681936)
\curveto(1016.53792765,30.68681786)(1016.61292757,30.65181789)(1016.68292725,30.61181936)
\curveto(1016.74292744,30.58181796)(1016.7879274,30.546818)(1016.81792725,30.50681936)
\curveto(1016.87792731,30.42681812)(1016.91292727,30.31681823)(1016.92292725,30.17681936)
\lineto(1016.92292725,29.75681936)
\lineto(1016.92292725,29.51681936)
\curveto(1016.91292727,29.4468191)(1016.8879273,29.38681916)(1016.84792725,29.33681936)
\curveto(1016.81792737,29.28681926)(1016.77292741,29.25681929)(1016.71292725,29.24681936)
\curveto(1016.65292753,29.2468193)(1016.59292759,29.25181929)(1016.53292725,29.26181936)
\curveto(1016.46292772,29.28181926)(1016.39792779,29.30181924)(1016.33792725,29.32181936)
\curveto(1016.26792792,29.35181919)(1016.21792797,29.37681917)(1016.18792725,29.39681936)
\curveto(1015.86792832,29.53681901)(1015.55292863,29.66181888)(1015.24292725,29.77181936)
\curveto(1014.92292926,29.88181866)(1014.60292958,30.00181854)(1014.28292725,30.13181936)
\curveto(1014.06293012,30.22181832)(1013.84793034,30.30681824)(1013.63792725,30.38681936)
\curveto(1013.41793077,30.46681808)(1013.19793099,30.55181799)(1012.97792725,30.64181936)
\curveto(1012.25793193,30.9418176)(1011.53293265,31.22681732)(1010.80292725,31.49681936)
\curveto(1010.06293412,31.76681678)(1009.32793486,32.05181649)(1008.59792725,32.35181936)
\curveto(1008.33793585,32.46181608)(1008.07293611,32.56181598)(1007.80292725,32.65181936)
\curveto(1007.53293665,32.75181579)(1007.26793692,32.85681569)(1007.00792725,32.96681936)
\curveto(1006.89793729,33.01681553)(1006.77793741,33.06181548)(1006.64792725,33.10181936)
\curveto(1006.50793768,33.15181539)(1006.40793778,33.22181532)(1006.34792725,33.31181936)
\curveto(1006.30793788,33.35181519)(1006.27793791,33.41681513)(1006.25792725,33.50681936)
\curveto(1006.24793794,33.52681502)(1006.24793794,33.546815)(1006.25792725,33.56681936)
\curveto(1006.25793793,33.59681495)(1006.25293793,33.62181492)(1006.24292725,33.64181936)
\curveto(1006.24293794,33.82181472)(1006.24293794,34.03181451)(1006.24292725,34.27181936)
\curveto(1006.23293795,34.51181403)(1006.26793792,34.68681386)(1006.34792725,34.79681936)
\curveto(1006.40793778,34.87681367)(1006.50793768,34.93681361)(1006.64792725,34.97681936)
\curveto(1006.77793741,35.02681352)(1006.89793729,35.07681347)(1007.00792725,35.12681936)
\curveto(1007.23793695,35.22681332)(1007.46793672,35.31681323)(1007.69792725,35.39681936)
\curveto(1007.92793626,35.47681307)(1008.15793603,35.56681298)(1008.38792725,35.66681936)
\curveto(1008.5879356,35.7468128)(1008.79293539,35.82181272)(1009.00292725,35.89181936)
\curveto(1009.21293497,35.97181257)(1009.41793477,36.05681249)(1009.61792725,36.14681936)
\curveto(1010.34793384,36.4468121)(1011.0879331,36.73181181)(1011.83792725,37.00181936)
\curveto(1012.57793161,37.28181126)(1013.31293087,37.57681097)(1014.04292725,37.88681936)
\curveto(1014.13293005,37.92681062)(1014.21792997,37.95681059)(1014.29792725,37.97681936)
\curveto(1014.37792981,38.00681054)(1014.46292972,38.03681051)(1014.55292725,38.06681936)
\curveto(1014.81292937,38.17681037)(1015.07792911,38.28181026)(1015.34792725,38.38181936)
\curveto(1015.61792857,38.49181005)(1015.8829283,38.60180994)(1016.14292725,38.71181936)
\moveto(1012.49792725,35.50181936)
\curveto(1012.46793172,35.59181295)(1012.41793177,35.6468129)(1012.34792725,35.66681936)
\curveto(1012.27793191,35.69681285)(1012.20293198,35.70181284)(1012.12292725,35.68181936)
\curveto(1012.03293215,35.67181287)(1011.94793224,35.6468129)(1011.86792725,35.60681936)
\curveto(1011.77793241,35.57681297)(1011.70293248,35.546813)(1011.64292725,35.51681936)
\curveto(1011.60293258,35.49681305)(1011.56793262,35.48681306)(1011.53792725,35.48681936)
\curveto(1011.50793268,35.48681306)(1011.47293271,35.47681307)(1011.43292725,35.45681936)
\lineto(1011.19292725,35.36681936)
\curveto(1011.10293308,35.3468132)(1011.01293317,35.31681323)(1010.92292725,35.27681936)
\curveto(1010.56293362,35.12681342)(1010.19793399,34.99181355)(1009.82792725,34.87181936)
\curveto(1009.44793474,34.76181378)(1009.07793511,34.63181391)(1008.71792725,34.48181936)
\curveto(1008.60793558,34.43181411)(1008.49793569,34.38681416)(1008.38792725,34.34681936)
\curveto(1008.27793591,34.31681423)(1008.17293601,34.27681427)(1008.07292725,34.22681936)
\curveto(1008.02293616,34.20681434)(1007.97793621,34.18181436)(1007.93792725,34.15181936)
\curveto(1007.8879363,34.13181441)(1007.86293632,34.08181446)(1007.86292725,34.00181936)
\curveto(1007.8829363,33.98181456)(1007.89793629,33.96181458)(1007.90792725,33.94181936)
\curveto(1007.91793627,33.92181462)(1007.93293625,33.90181464)(1007.95292725,33.88181936)
\curveto(1008.00293618,33.8418147)(1008.05793613,33.81181473)(1008.11792725,33.79181936)
\curveto(1008.16793602,33.77181477)(1008.22293596,33.75181479)(1008.28292725,33.73181936)
\curveto(1008.39293579,33.68181486)(1008.50293568,33.6418149)(1008.61292725,33.61181936)
\curveto(1008.72293546,33.58181496)(1008.83293535,33.541815)(1008.94292725,33.49181936)
\curveto(1009.33293485,33.32181522)(1009.72793446,33.17181537)(1010.12792725,33.04181936)
\curveto(1010.52793366,32.92181562)(1010.91793327,32.78181576)(1011.29792725,32.62181936)
\lineto(1011.44792725,32.56181936)
\curveto(1011.49793269,32.55181599)(1011.54793264,32.53681601)(1011.59792725,32.51681936)
\lineto(1011.83792725,32.42681936)
\curveto(1011.91793227,32.39681615)(1011.99793219,32.37181617)(1012.07792725,32.35181936)
\curveto(1012.12793206,32.33181621)(1012.182932,32.32181622)(1012.24292725,32.32181936)
\curveto(1012.30293188,32.33181621)(1012.35293183,32.3468162)(1012.39292725,32.36681936)
\curveto(1012.47293171,32.41681613)(1012.51793167,32.52181602)(1012.52792725,32.68181936)
\lineto(1012.52792725,33.13181936)
\lineto(1012.52792725,34.73681936)
\curveto(1012.52793166,34.8468137)(1012.53293165,34.98181356)(1012.54292725,35.14181936)
\curveto(1012.54293164,35.30181324)(1012.52793166,35.42181312)(1012.49792725,35.50181936)
}
}
{
\newrgbcolor{curcolor}{0 0 0}
\pscustom[linestyle=none,fillstyle=solid,fillcolor=curcolor]
{
\newpath
\moveto(1009.30292725,46.44338186)
\curveto(1009.35293483,46.51337426)(1009.42793476,46.54837422)(1009.52792725,46.54838186)
\curveto(1009.62793456,46.55837421)(1009.73293445,46.56337421)(1009.84292725,46.56338186)
\lineto(1016.11292725,46.56338186)
\lineto(1016.71292725,46.56338186)
\curveto(1016.76292742,46.54337423)(1016.81292737,46.53837423)(1016.86292725,46.54838186)
\curveto(1016.90292728,46.55837421)(1016.94792724,46.55337422)(1016.99792725,46.53338186)
\curveto(1017.09792709,46.51337426)(1017.19792699,46.49837427)(1017.29792725,46.48838186)
\curveto(1017.40792678,46.48837428)(1017.51292667,46.4733743)(1017.61292725,46.44338186)
\curveto(1017.72292646,46.41337436)(1017.82792636,46.38337439)(1017.92792725,46.35338186)
\curveto(1018.02792616,46.33337444)(1018.12792606,46.29837447)(1018.22792725,46.24838186)
\curveto(1018.4879257,46.14837462)(1018.72292546,46.01837475)(1018.93292725,45.85838186)
\curveto(1019.14292504,45.70837506)(1019.31792487,45.52837524)(1019.45792725,45.31838186)
\curveto(1019.57792461,45.14837562)(1019.67292451,44.9683758)(1019.74292725,44.77838186)
\curveto(1019.82292436,44.58837618)(1019.89792429,44.38337639)(1019.96792725,44.16338186)
\curveto(1019.9879242,44.0733767)(1019.99792419,43.98337679)(1019.99792725,43.89338186)
\curveto(1020.00792418,43.80337697)(1020.02292416,43.71337706)(1020.04292725,43.62338186)
\lineto(1020.04292725,43.53338186)
\curveto(1020.05292413,43.51337726)(1020.05792413,43.49337728)(1020.05792725,43.47338186)
\curveto(1020.06792412,43.42337735)(1020.06792412,43.3733774)(1020.05792725,43.32338186)
\curveto(1020.04792414,43.28337749)(1020.05292413,43.23837753)(1020.07292725,43.18838186)
\curveto(1020.09292409,43.11837765)(1020.09792409,43.00837776)(1020.08792725,42.85838186)
\curveto(1020.0879241,42.71837805)(1020.07792411,42.61837815)(1020.05792725,42.55838186)
\curveto(1020.05792413,42.52837824)(1020.05292413,42.49837827)(1020.04292725,42.46838186)
\lineto(1020.04292725,42.40838186)
\curveto(1020.02292416,42.31837845)(1020.00792418,42.22837854)(1019.99792725,42.13838186)
\curveto(1019.99792419,42.04837872)(1019.9879242,41.96337881)(1019.96792725,41.88338186)
\curveto(1019.94792424,41.80337897)(1019.92292426,41.72337905)(1019.89292725,41.64338186)
\curveto(1019.87292431,41.56337921)(1019.84792434,41.48337929)(1019.81792725,41.40338186)
\curveto(1019.6879245,41.08337969)(1019.54292464,40.81337996)(1019.38292725,40.59338186)
\curveto(1019.22292496,40.38338039)(1018.99792519,40.19338058)(1018.70792725,40.02338186)
\curveto(1018.6879255,40.00338077)(1018.66292552,39.98838078)(1018.63292725,39.97838186)
\curveto(1018.61292557,39.97838079)(1018.5879256,39.9683808)(1018.55792725,39.94838186)
\curveto(1018.47792571,39.91838085)(1018.36292582,39.88338089)(1018.21292725,39.84338186)
\curveto(1018.07292611,39.81338096)(1017.96792622,39.84338093)(1017.89792725,39.93338186)
\curveto(1017.84792634,39.99338078)(1017.82292636,40.0733807)(1017.82292725,40.17338186)
\lineto(1017.82292725,40.50338186)
\lineto(1017.82292725,40.66838186)
\curveto(1017.82292636,40.72838004)(1017.83292635,40.78337999)(1017.85292725,40.83338186)
\curveto(1017.8829263,40.92337985)(1017.93292625,40.98837978)(1018.00292725,41.02838186)
\curveto(1018.07292611,41.0683797)(1018.14792604,41.11337966)(1018.22792725,41.16338186)
\lineto(1018.40792725,41.28338186)
\curveto(1018.47792571,41.33337944)(1018.53292565,41.38337939)(1018.57292725,41.43338186)
\curveto(1018.76292542,41.68337909)(1018.90292528,41.98337879)(1018.99292725,42.33338186)
\curveto(1019.01292517,42.39337838)(1019.02292516,42.45337832)(1019.02292725,42.51338186)
\curveto(1019.03292515,42.58337819)(1019.04792514,42.64837812)(1019.06792725,42.70838186)
\lineto(1019.06792725,42.79838186)
\curveto(1019.0879251,42.8683779)(1019.09792509,42.95337782)(1019.09792725,43.05338186)
\curveto(1019.09792509,43.15337762)(1019.0879251,43.24337753)(1019.06792725,43.32338186)
\curveto(1019.05792513,43.35337742)(1019.05292513,43.39337738)(1019.05292725,43.44338186)
\curveto(1019.03292515,43.54337723)(1019.01292517,43.63837713)(1018.99292725,43.72838186)
\curveto(1018.9829252,43.81837695)(1018.95792523,43.90337687)(1018.91792725,43.98338186)
\curveto(1018.79792539,44.2733765)(1018.63292555,44.50837626)(1018.42292725,44.68838186)
\curveto(1018.22292596,44.87837589)(1017.97792621,45.03337574)(1017.68792725,45.15338186)
\curveto(1017.59792659,45.19337558)(1017.50292668,45.21837555)(1017.40292725,45.22838186)
\curveto(1017.30292688,45.24837552)(1017.19792699,45.2733755)(1017.08792725,45.30338186)
\curveto(1017.03792715,45.32337545)(1016.9879272,45.33337544)(1016.93792725,45.33338186)
\curveto(1016.8879273,45.33337544)(1016.83792735,45.33837543)(1016.78792725,45.34838186)
\curveto(1016.75792743,45.35837541)(1016.70792748,45.36337541)(1016.63792725,45.36338186)
\curveto(1016.55792763,45.38337539)(1016.47292771,45.38337539)(1016.38292725,45.36338186)
\curveto(1016.33292785,45.35337542)(1016.2879279,45.34837542)(1016.24792725,45.34838186)
\curveto(1016.20792798,45.35837541)(1016.17292801,45.35337542)(1016.14292725,45.33338186)
\curveto(1016.12292806,45.31337546)(1016.11292807,45.29837547)(1016.11292725,45.28838186)
\lineto(1016.06792725,45.24338186)
\curveto(1016.06792812,45.14337563)(1016.09792809,45.0683757)(1016.15792725,45.01838186)
\curveto(1016.20792798,44.97837579)(1016.25292793,44.92837584)(1016.29292725,44.86838186)
\lineto(1016.50292725,44.62838186)
\curveto(1016.56292762,44.54837622)(1016.61792757,44.45837631)(1016.66792725,44.35838186)
\curveto(1016.75792743,44.21837655)(1016.83292735,44.04337673)(1016.89292725,43.83338186)
\curveto(1016.94292724,43.62337715)(1016.97792721,43.40337737)(1016.99792725,43.17338186)
\curveto(1017.01792717,42.94337783)(1017.01292717,42.71337806)(1016.98292725,42.48338186)
\curveto(1016.96292722,42.25337852)(1016.92292726,42.04337873)(1016.86292725,41.85338186)
\curveto(1016.55292763,40.91337986)(1015.95792823,40.25338052)(1015.07792725,39.87338186)
\curveto(1014.97792921,39.82338095)(1014.8829293,39.78338099)(1014.79292725,39.75338186)
\curveto(1014.69292949,39.72338105)(1014.5879296,39.68838108)(1014.47792725,39.64838186)
\curveto(1014.42792976,39.62838114)(1014.3829298,39.61838115)(1014.34292725,39.61838186)
\curveto(1014.30292988,39.61838115)(1014.25792993,39.60838116)(1014.20792725,39.58838186)
\curveto(1014.13793005,39.5683812)(1014.06793012,39.55338122)(1013.99792725,39.54338186)
\curveto(1013.91793027,39.54338123)(1013.84293034,39.53338124)(1013.77292725,39.51338186)
\curveto(1013.73293045,39.50338127)(1013.69793049,39.49838127)(1013.66792725,39.49838186)
\curveto(1013.62793056,39.50838126)(1013.5879306,39.50838126)(1013.54792725,39.49838186)
\curveto(1013.50793068,39.49838127)(1013.46793072,39.49338128)(1013.42792725,39.48338186)
\lineto(1013.30792725,39.48338186)
\curveto(1013.187931,39.46338131)(1013.06293112,39.46338131)(1012.93292725,39.48338186)
\curveto(1012.87293131,39.49338128)(1012.81293137,39.49838127)(1012.75292725,39.49838186)
\lineto(1012.58792725,39.49838186)
\curveto(1012.53793165,39.50838126)(1012.49793169,39.51338126)(1012.46792725,39.51338186)
\curveto(1012.42793176,39.51338126)(1012.3829318,39.51838125)(1012.33292725,39.52838186)
\curveto(1012.22293196,39.55838121)(1012.11793207,39.57838119)(1012.01792725,39.58838186)
\curveto(1011.90793228,39.59838117)(1011.79793239,39.62338115)(1011.68792725,39.66338186)
\curveto(1011.56793262,39.70338107)(1011.45293273,39.73838103)(1011.34292725,39.76838186)
\curveto(1011.22293296,39.80838096)(1011.10793308,39.85338092)(1010.99792725,39.90338186)
\curveto(1010.83793335,39.9733808)(1010.69293349,40.05338072)(1010.56292725,40.14338186)
\curveto(1010.42293376,40.23338054)(1010.2879339,40.32838044)(1010.15792725,40.42838186)
\curveto(1010.04793414,40.49838027)(1009.95793423,40.58838018)(1009.88792725,40.69838186)
\lineto(1009.82792725,40.75838186)
\lineto(1009.76792725,40.81838186)
\lineto(1009.64792725,40.96838186)
\lineto(1009.52792725,41.14838186)
\curveto(1009.44793474,41.27837949)(1009.37793481,41.41337936)(1009.31792725,41.55338186)
\curveto(1009.25793493,41.70337907)(1009.20293498,41.86337891)(1009.15292725,42.03338186)
\curveto(1009.12293506,42.13337864)(1009.10293508,42.23337854)(1009.09292725,42.33338186)
\curveto(1009.0829351,42.44337833)(1009.06793512,42.55337822)(1009.04792725,42.66338186)
\curveto(1009.03793515,42.70337807)(1009.03793515,42.75337802)(1009.04792725,42.81338186)
\curveto(1009.05793513,42.88337789)(1009.05293513,42.93337784)(1009.03292725,42.96338186)
\curveto(1009.02293516,43.28337749)(1009.05293513,43.5683772)(1009.12292725,43.81838186)
\curveto(1009.19293499,44.07837669)(1009.29293489,44.30837646)(1009.42292725,44.50838186)
\curveto(1009.46293472,44.57837619)(1009.50793468,44.64337613)(1009.55792725,44.70338186)
\lineto(1009.70792725,44.88338186)
\curveto(1009.74793444,44.93337584)(1009.79293439,44.97837579)(1009.84292725,45.01838186)
\curveto(1009.8829343,45.0683757)(1009.90293428,45.14337563)(1009.90292725,45.24338186)
\lineto(1009.85792725,45.28838186)
\curveto(1009.83793435,45.30837546)(1009.81293437,45.32837544)(1009.78292725,45.34838186)
\curveto(1009.70293448,45.37837539)(1009.62293456,45.39337538)(1009.54292725,45.39338186)
\curveto(1009.46293472,45.40337537)(1009.39293479,45.43337534)(1009.33292725,45.48338186)
\curveto(1009.29293489,45.51337526)(1009.26293492,45.5733752)(1009.24292725,45.66338186)
\curveto(1009.21293497,45.75337502)(1009.19793499,45.84837492)(1009.19792725,45.94838186)
\curveto(1009.19793499,46.04837472)(1009.20793498,46.14337463)(1009.22792725,46.23338186)
\curveto(1009.24793494,46.33337444)(1009.27293491,46.40337437)(1009.30292725,46.44338186)
\moveto(1013.08292725,45.31838186)
\curveto(1013.04293114,45.32837544)(1012.99293119,45.33337544)(1012.93292725,45.33338186)
\curveto(1012.86293132,45.33337544)(1012.80793138,45.32837544)(1012.76792725,45.31838186)
\lineto(1012.52792725,45.31838186)
\curveto(1012.43793175,45.29837547)(1012.35293183,45.28337549)(1012.27292725,45.27338186)
\curveto(1012.182932,45.26337551)(1012.09793209,45.24837552)(1012.01792725,45.22838186)
\curveto(1011.93793225,45.20837556)(1011.86293232,45.18837558)(1011.79292725,45.16838186)
\curveto(1011.71293247,45.15837561)(1011.63793255,45.13837563)(1011.56792725,45.10838186)
\curveto(1011.2879329,44.99837577)(1011.03793315,44.85337592)(1010.81792725,44.67338186)
\curveto(1010.59793359,44.50337627)(1010.43293375,44.28337649)(1010.32292725,44.01338186)
\curveto(1010.2829339,43.93337684)(1010.25293393,43.84837692)(1010.23292725,43.75838186)
\curveto(1010.20293398,43.6683771)(1010.17793401,43.5733772)(1010.15792725,43.47338186)
\curveto(1010.13793405,43.39337738)(1010.13293405,43.30337747)(1010.14292725,43.20338186)
\lineto(1010.14292725,42.93338186)
\curveto(1010.15293403,42.88337789)(1010.15793403,42.83337794)(1010.15792725,42.78338186)
\curveto(1010.15793403,42.74337803)(1010.16293402,42.69837807)(1010.17292725,42.64838186)
\curveto(1010.22293396,42.45837831)(1010.27293391,42.29837847)(1010.32292725,42.16838186)
\curveto(1010.46293372,41.82837894)(1010.67293351,41.56337921)(1010.95292725,41.37338186)
\curveto(1011.23293295,41.18337959)(1011.55793263,41.03337974)(1011.92792725,40.92338186)
\curveto(1012.00793218,40.90337987)(1012.0879321,40.88837988)(1012.16792725,40.87838186)
\curveto(1012.23793195,40.87837989)(1012.31293187,40.8683799)(1012.39292725,40.84838186)
\curveto(1012.42293176,40.82837994)(1012.45793173,40.81837995)(1012.49792725,40.81838186)
\curveto(1012.53793165,40.82837994)(1012.57293161,40.82837994)(1012.60292725,40.81838186)
\lineto(1012.93292725,40.81838186)
\lineto(1013.27792725,40.81838186)
\curveto(1013.3879308,40.81837995)(1013.49293069,40.82837994)(1013.59292725,40.84838186)
\lineto(1013.66792725,40.84838186)
\curveto(1013.69793049,40.85837991)(1013.72293046,40.86337991)(1013.74292725,40.86338186)
\curveto(1013.83293035,40.88337989)(1013.92293026,40.89837987)(1014.01292725,40.90838186)
\curveto(1014.10293008,40.92837984)(1014.18793,40.95337982)(1014.26792725,40.98338186)
\curveto(1014.52792966,41.06337971)(1014.76792942,41.16337961)(1014.98792725,41.28338186)
\curveto(1015.20792898,41.40337937)(1015.3879288,41.56337921)(1015.52792725,41.76338186)
\lineto(1015.61792725,41.88338186)
\curveto(1015.63792855,41.92337885)(1015.65792853,41.9683788)(1015.67792725,42.01838186)
\curveto(1015.72792846,42.09837867)(1015.76792842,42.18337859)(1015.79792725,42.27338186)
\curveto(1015.82792836,42.36337841)(1015.85792833,42.46337831)(1015.88792725,42.57338186)
\curveto(1015.89792829,42.62337815)(1015.90292828,42.6683781)(1015.90292725,42.70838186)
\curveto(1015.89292829,42.75837801)(1015.89792829,42.80837796)(1015.91792725,42.85838186)
\curveto(1015.92792826,42.88837788)(1015.93292825,42.93837783)(1015.93292725,43.00838186)
\curveto(1015.93292825,43.07837769)(1015.92792826,43.12837764)(1015.91792725,43.15838186)
\curveto(1015.90792828,43.18837758)(1015.90792828,43.21837755)(1015.91792725,43.24838186)
\curveto(1015.91792827,43.28837748)(1015.91292827,43.32837744)(1015.90292725,43.36838186)
\curveto(1015.8829283,43.45837731)(1015.86292832,43.54337723)(1015.84292725,43.62338186)
\curveto(1015.82292836,43.70337707)(1015.79792839,43.78337699)(1015.76792725,43.86338186)
\curveto(1015.61792857,44.20337657)(1015.40792878,44.4733763)(1015.13792725,44.67338186)
\curveto(1014.86792932,44.8733759)(1014.55292963,45.03337574)(1014.19292725,45.15338186)
\curveto(1014.10293008,45.18337559)(1014.01293017,45.20337557)(1013.92292725,45.21338186)
\curveto(1013.82293036,45.23337554)(1013.72793046,45.25337552)(1013.63792725,45.27338186)
\curveto(1013.59793059,45.28337549)(1013.56293062,45.28837548)(1013.53292725,45.28838186)
\curveto(1013.49293069,45.28837548)(1013.45293073,45.29337548)(1013.41292725,45.30338186)
\curveto(1013.36293082,45.32337545)(1013.31293087,45.32337545)(1013.26292725,45.30338186)
\curveto(1013.20293098,45.29337548)(1013.14293104,45.29837547)(1013.08292725,45.31838186)
}
}
{
\newrgbcolor{curcolor}{0 0 0}
\pscustom[linestyle=none,fillstyle=solid,fillcolor=curcolor]
{
\newpath
\moveto(1012.72292725,55.56666311)
\curveto(1012.7829314,55.58665505)(1012.87793131,55.59665504)(1013.00792725,55.59666311)
\curveto(1013.12793106,55.59665504)(1013.21293097,55.59165504)(1013.26292725,55.58166311)
\lineto(1013.41292725,55.58166311)
\curveto(1013.49293069,55.57165506)(1013.56793062,55.56165507)(1013.63792725,55.55166311)
\curveto(1013.69793049,55.55165508)(1013.76793042,55.54665509)(1013.84792725,55.53666311)
\curveto(1013.90793028,55.51665512)(1013.96793022,55.50165513)(1014.02792725,55.49166311)
\curveto(1014.0879301,55.49165514)(1014.14793004,55.48165515)(1014.20792725,55.46166311)
\curveto(1014.33792985,55.42165521)(1014.46792972,55.38665525)(1014.59792725,55.35666311)
\curveto(1014.72792946,55.32665531)(1014.84792934,55.28665535)(1014.95792725,55.23666311)
\curveto(1015.43792875,55.02665561)(1015.84292834,54.74665589)(1016.17292725,54.39666311)
\curveto(1016.49292769,54.04665659)(1016.73792745,53.61665702)(1016.90792725,53.10666311)
\curveto(1016.94792724,52.99665764)(1016.97792721,52.87665776)(1016.99792725,52.74666311)
\curveto(1017.01792717,52.62665801)(1017.03792715,52.50165813)(1017.05792725,52.37166311)
\curveto(1017.06792712,52.31165832)(1017.07292711,52.24665839)(1017.07292725,52.17666311)
\curveto(1017.0829271,52.11665852)(1017.0879271,52.05665858)(1017.08792725,51.99666311)
\curveto(1017.09792709,51.95665868)(1017.10292708,51.89665874)(1017.10292725,51.81666311)
\curveto(1017.10292708,51.74665889)(1017.09792709,51.69665894)(1017.08792725,51.66666311)
\curveto(1017.07792711,51.62665901)(1017.07292711,51.58665905)(1017.07292725,51.54666311)
\curveto(1017.0829271,51.50665913)(1017.0829271,51.47165916)(1017.07292725,51.44166311)
\lineto(1017.07292725,51.35166311)
\lineto(1017.02792725,50.99166311)
\curveto(1016.9879272,50.85165978)(1016.94792724,50.71665992)(1016.90792725,50.58666311)
\curveto(1016.86792732,50.45666018)(1016.82292736,50.3316603)(1016.77292725,50.21166311)
\curveto(1016.57292761,49.76166087)(1016.31292787,49.39166124)(1015.99292725,49.10166311)
\curveto(1015.67292851,48.81166182)(1015.2829289,48.57166206)(1014.82292725,48.38166311)
\curveto(1014.72292946,48.3316623)(1014.62292956,48.29166234)(1014.52292725,48.26166311)
\curveto(1014.42292976,48.24166239)(1014.31792987,48.22166241)(1014.20792725,48.20166311)
\curveto(1014.16793002,48.18166245)(1014.13793005,48.17166246)(1014.11792725,48.17166311)
\curveto(1014.0879301,48.18166245)(1014.05293013,48.18166245)(1014.01292725,48.17166311)
\curveto(1013.93293025,48.15166248)(1013.85293033,48.1366625)(1013.77292725,48.12666311)
\curveto(1013.6829305,48.12666251)(1013.59793059,48.11666252)(1013.51792725,48.09666311)
\lineto(1013.39792725,48.09666311)
\curveto(1013.35793083,48.09666254)(1013.31293087,48.09166254)(1013.26292725,48.08166311)
\curveto(1013.21293097,48.07166256)(1013.12793106,48.06666257)(1013.00792725,48.06666311)
\curveto(1012.87793131,48.06666257)(1012.7829314,48.07666256)(1012.72292725,48.09666311)
\curveto(1012.65293153,48.11666252)(1012.5829316,48.12166251)(1012.51292725,48.11166311)
\curveto(1012.44293174,48.10166253)(1012.37293181,48.10666253)(1012.30292725,48.12666311)
\curveto(1012.25293193,48.1366625)(1012.21293197,48.14166249)(1012.18292725,48.14166311)
\curveto(1012.14293204,48.15166248)(1012.09793209,48.16166247)(1012.04792725,48.17166311)
\curveto(1011.92793226,48.20166243)(1011.80793238,48.22666241)(1011.68792725,48.24666311)
\curveto(1011.56793262,48.27666236)(1011.45293273,48.31666232)(1011.34292725,48.36666311)
\curveto(1010.97293321,48.51666212)(1010.64293354,48.69666194)(1010.35292725,48.90666311)
\curveto(1010.05293413,49.12666151)(1009.80293438,49.39166124)(1009.60292725,49.70166311)
\curveto(1009.52293466,49.82166081)(1009.45793473,49.94666069)(1009.40792725,50.07666311)
\curveto(1009.34793484,50.20666043)(1009.2879349,50.34166029)(1009.22792725,50.48166311)
\curveto(1009.17793501,50.60166003)(1009.14793504,50.7316599)(1009.13792725,50.87166311)
\curveto(1009.11793507,51.01165962)(1009.0879351,51.15165948)(1009.04792725,51.29166311)
\lineto(1009.04792725,51.48666311)
\curveto(1009.03793515,51.55665908)(1009.02793516,51.62165901)(1009.01792725,51.68166311)
\curveto(1009.00793518,52.57165806)(1009.19293499,53.31165732)(1009.57292725,53.90166311)
\curveto(1009.95293423,54.49165614)(1010.44793374,54.91665572)(1011.05792725,55.17666311)
\curveto(1011.15793303,55.22665541)(1011.25793293,55.26665537)(1011.35792725,55.29666311)
\curveto(1011.45793273,55.32665531)(1011.56293262,55.36165527)(1011.67292725,55.40166311)
\curveto(1011.7829324,55.4316552)(1011.90293228,55.45665518)(1012.03292725,55.47666311)
\curveto(1012.15293203,55.49665514)(1012.27793191,55.52165511)(1012.40792725,55.55166311)
\curveto(1012.45793173,55.56165507)(1012.51293167,55.56165507)(1012.57292725,55.55166311)
\curveto(1012.62293156,55.55165508)(1012.67293151,55.55665508)(1012.72292725,55.56666311)
\moveto(1013.57792725,54.23166311)
\curveto(1013.50793068,54.25165638)(1013.42793076,54.25665638)(1013.33792725,54.24666311)
\lineto(1013.08292725,54.24666311)
\curveto(1012.69293149,54.24665639)(1012.36293182,54.21165642)(1012.09292725,54.14166311)
\curveto(1012.01293217,54.11165652)(1011.93293225,54.08665655)(1011.85292725,54.06666311)
\curveto(1011.77293241,54.04665659)(1011.69793249,54.02165661)(1011.62792725,53.99166311)
\curveto(1010.97793321,53.71165692)(1010.52793366,53.26665737)(1010.27792725,52.65666311)
\curveto(1010.24793394,52.58665805)(1010.22793396,52.51165812)(1010.21792725,52.43166311)
\lineto(1010.15792725,52.19166311)
\curveto(1010.13793405,52.11165852)(1010.12793406,52.02665861)(1010.12792725,51.93666311)
\lineto(1010.12792725,51.66666311)
\lineto(1010.17292725,51.39666311)
\curveto(1010.19293399,51.29665934)(1010.21793397,51.20165943)(1010.24792725,51.11166311)
\curveto(1010.26793392,51.0316596)(1010.29793389,50.95165968)(1010.33792725,50.87166311)
\curveto(1010.35793383,50.80165983)(1010.3879338,50.7366599)(1010.42792725,50.67666311)
\curveto(1010.46793372,50.61666002)(1010.50793368,50.56166007)(1010.54792725,50.51166311)
\curveto(1010.71793347,50.27166036)(1010.92293326,50.07666056)(1011.16292725,49.92666311)
\curveto(1011.40293278,49.77666086)(1011.6829325,49.64666099)(1012.00292725,49.53666311)
\curveto(1012.10293208,49.50666113)(1012.20793198,49.48666115)(1012.31792725,49.47666311)
\curveto(1012.41793177,49.46666117)(1012.52293166,49.45166118)(1012.63292725,49.43166311)
\curveto(1012.67293151,49.42166121)(1012.73793145,49.41666122)(1012.82792725,49.41666311)
\curveto(1012.85793133,49.40666123)(1012.89293129,49.40166123)(1012.93292725,49.40166311)
\curveto(1012.97293121,49.41166122)(1013.01793117,49.41666122)(1013.06792725,49.41666311)
\lineto(1013.36792725,49.41666311)
\curveto(1013.46793072,49.41666122)(1013.55793063,49.42666121)(1013.63792725,49.44666311)
\lineto(1013.81792725,49.47666311)
\curveto(1013.91793027,49.49666114)(1014.01793017,49.51166112)(1014.11792725,49.52166311)
\curveto(1014.20792998,49.54166109)(1014.29292989,49.57166106)(1014.37292725,49.61166311)
\curveto(1014.61292957,49.71166092)(1014.83792935,49.82666081)(1015.04792725,49.95666311)
\curveto(1015.25792893,50.09666054)(1015.43292875,50.26666037)(1015.57292725,50.46666311)
\curveto(1015.60292858,50.51666012)(1015.62792856,50.56166007)(1015.64792725,50.60166311)
\curveto(1015.66792852,50.64165999)(1015.69292849,50.68665995)(1015.72292725,50.73666311)
\curveto(1015.77292841,50.81665982)(1015.81792837,50.90165973)(1015.85792725,50.99166311)
\curveto(1015.8879283,51.09165954)(1015.91792827,51.19665944)(1015.94792725,51.30666311)
\curveto(1015.96792822,51.35665928)(1015.97792821,51.40165923)(1015.97792725,51.44166311)
\curveto(1015.96792822,51.49165914)(1015.96792822,51.54165909)(1015.97792725,51.59166311)
\curveto(1015.9879282,51.62165901)(1015.99792819,51.68165895)(1016.00792725,51.77166311)
\curveto(1016.01792817,51.87165876)(1016.01292817,51.94665869)(1015.99292725,51.99666311)
\curveto(1015.9829282,52.0366586)(1015.9829282,52.07665856)(1015.99292725,52.11666311)
\curveto(1015.99292819,52.15665848)(1015.9829282,52.19665844)(1015.96292725,52.23666311)
\curveto(1015.94292824,52.31665832)(1015.92792826,52.39665824)(1015.91792725,52.47666311)
\curveto(1015.89792829,52.55665808)(1015.87292831,52.631658)(1015.84292725,52.70166311)
\curveto(1015.70292848,53.04165759)(1015.50792868,53.31665732)(1015.25792725,53.52666311)
\curveto(1015.00792918,53.7366569)(1014.71292947,53.91165672)(1014.37292725,54.05166311)
\curveto(1014.25292993,54.10165653)(1014.12793006,54.1316565)(1013.99792725,54.14166311)
\curveto(1013.85793033,54.16165647)(1013.71793047,54.19165644)(1013.57792725,54.23166311)
}
}
{
\newrgbcolor{curcolor}{0 0 0}
\pscustom[linestyle=none,fillstyle=solid,fillcolor=curcolor]
{
}
}
{
\newrgbcolor{curcolor}{0 0 0}
\pscustom[linestyle=none,fillstyle=solid,fillcolor=curcolor]
{
\newpath
\moveto(1011.83792725,67.99510061)
\lineto(1012.09292725,67.99510061)
\curveto(1012.17293201,68.0050929)(1012.24793194,68.00009291)(1012.31792725,67.98010061)
\lineto(1012.55792725,67.98010061)
\lineto(1012.72292725,67.98010061)
\curveto(1012.82293136,67.96009295)(1012.92793126,67.95009296)(1013.03792725,67.95010061)
\curveto(1013.13793105,67.95009296)(1013.23793095,67.94009297)(1013.33792725,67.92010061)
\lineto(1013.48792725,67.92010061)
\curveto(1013.62793056,67.89009302)(1013.76793042,67.87009304)(1013.90792725,67.86010061)
\curveto(1014.03793015,67.85009306)(1014.16793002,67.82509308)(1014.29792725,67.78510061)
\curveto(1014.37792981,67.76509314)(1014.46292972,67.74509316)(1014.55292725,67.72510061)
\lineto(1014.79292725,67.66510061)
\lineto(1015.09292725,67.54510061)
\curveto(1015.182929,67.51509339)(1015.27292891,67.48009343)(1015.36292725,67.44010061)
\curveto(1015.5829286,67.34009357)(1015.79792839,67.2050937)(1016.00792725,67.03510061)
\curveto(1016.21792797,66.87509403)(1016.3879278,66.70009421)(1016.51792725,66.51010061)
\curveto(1016.55792763,66.46009445)(1016.59792759,66.40009451)(1016.63792725,66.33010061)
\curveto(1016.66792752,66.27009464)(1016.70292748,66.2100947)(1016.74292725,66.15010061)
\curveto(1016.79292739,66.07009484)(1016.83292735,65.97509493)(1016.86292725,65.86510061)
\curveto(1016.89292729,65.75509515)(1016.92292726,65.65009526)(1016.95292725,65.55010061)
\curveto(1016.99292719,65.44009547)(1017.01792717,65.33009558)(1017.02792725,65.22010061)
\curveto(1017.03792715,65.1100958)(1017.05292713,64.99509591)(1017.07292725,64.87510061)
\curveto(1017.0829271,64.83509607)(1017.0829271,64.79009612)(1017.07292725,64.74010061)
\curveto(1017.07292711,64.70009621)(1017.07792711,64.66009625)(1017.08792725,64.62010061)
\curveto(1017.09792709,64.58009633)(1017.10292708,64.52509638)(1017.10292725,64.45510061)
\curveto(1017.10292708,64.38509652)(1017.09792709,64.33509657)(1017.08792725,64.30510061)
\curveto(1017.06792712,64.25509665)(1017.06292712,64.2100967)(1017.07292725,64.17010061)
\curveto(1017.0829271,64.13009678)(1017.0829271,64.09509681)(1017.07292725,64.06510061)
\lineto(1017.07292725,63.97510061)
\curveto(1017.05292713,63.91509699)(1017.03792715,63.85009706)(1017.02792725,63.78010061)
\curveto(1017.02792716,63.72009719)(1017.02292716,63.65509725)(1017.01292725,63.58510061)
\curveto(1016.96292722,63.41509749)(1016.91292727,63.25509765)(1016.86292725,63.10510061)
\curveto(1016.81292737,62.95509795)(1016.74792744,62.8100981)(1016.66792725,62.67010061)
\curveto(1016.62792756,62.62009829)(1016.59792759,62.56509834)(1016.57792725,62.50510061)
\curveto(1016.54792764,62.45509845)(1016.51292767,62.4050985)(1016.47292725,62.35510061)
\curveto(1016.29292789,62.11509879)(1016.07292811,61.91509899)(1015.81292725,61.75510061)
\curveto(1015.55292863,61.59509931)(1015.26792892,61.45509945)(1014.95792725,61.33510061)
\curveto(1014.81792937,61.27509963)(1014.67792951,61.23009968)(1014.53792725,61.20010061)
\curveto(1014.3879298,61.17009974)(1014.23292995,61.13509977)(1014.07292725,61.09510061)
\curveto(1013.96293022,61.07509983)(1013.85293033,61.06009985)(1013.74292725,61.05010061)
\curveto(1013.63293055,61.04009987)(1013.52293066,61.02509988)(1013.41292725,61.00510061)
\curveto(1013.37293081,60.99509991)(1013.33293085,60.99009992)(1013.29292725,60.99010061)
\curveto(1013.25293093,61.00009991)(1013.21293097,61.00009991)(1013.17292725,60.99010061)
\curveto(1013.12293106,60.98009993)(1013.07293111,60.97509993)(1013.02292725,60.97510061)
\lineto(1012.85792725,60.97510061)
\curveto(1012.80793138,60.95509995)(1012.75793143,60.95009996)(1012.70792725,60.96010061)
\curveto(1012.64793154,60.97009994)(1012.59293159,60.97009994)(1012.54292725,60.96010061)
\curveto(1012.50293168,60.95009996)(1012.45793173,60.95009996)(1012.40792725,60.96010061)
\curveto(1012.35793183,60.97009994)(1012.30793188,60.96509994)(1012.25792725,60.94510061)
\curveto(1012.187932,60.92509998)(1012.11293207,60.92009999)(1012.03292725,60.93010061)
\curveto(1011.94293224,60.94009997)(1011.85793233,60.94509996)(1011.77792725,60.94510061)
\curveto(1011.6879325,60.94509996)(1011.5879326,60.94009997)(1011.47792725,60.93010061)
\curveto(1011.35793283,60.92009999)(1011.25793293,60.92509998)(1011.17792725,60.94510061)
\lineto(1010.89292725,60.94510061)
\lineto(1010.26292725,60.99010061)
\curveto(1010.16293402,61.00009991)(1010.06793412,61.0100999)(1009.97792725,61.02010061)
\lineto(1009.67792725,61.05010061)
\curveto(1009.62793456,61.07009984)(1009.57793461,61.07509983)(1009.52792725,61.06510061)
\curveto(1009.46793472,61.06509984)(1009.41293477,61.07509983)(1009.36292725,61.09510061)
\curveto(1009.19293499,61.14509976)(1009.02793516,61.18509972)(1008.86792725,61.21510061)
\curveto(1008.69793549,61.24509966)(1008.53793565,61.29509961)(1008.38792725,61.36510061)
\curveto(1007.92793626,61.55509935)(1007.55293663,61.77509913)(1007.26292725,62.02510061)
\curveto(1006.97293721,62.28509862)(1006.72793746,62.64509826)(1006.52792725,63.10510061)
\curveto(1006.47793771,63.23509767)(1006.44293774,63.36509754)(1006.42292725,63.49510061)
\curveto(1006.40293778,63.63509727)(1006.37793781,63.77509713)(1006.34792725,63.91510061)
\curveto(1006.33793785,63.98509692)(1006.33293785,64.05009686)(1006.33292725,64.11010061)
\curveto(1006.33293785,64.17009674)(1006.32793786,64.23509667)(1006.31792725,64.30510061)
\curveto(1006.29793789,65.13509577)(1006.44793774,65.8050951)(1006.76792725,66.31510061)
\curveto(1007.07793711,66.82509408)(1007.51793667,67.2050937)(1008.08792725,67.45510061)
\curveto(1008.20793598,67.5050934)(1008.33293585,67.55009336)(1008.46292725,67.59010061)
\curveto(1008.59293559,67.63009328)(1008.72793546,67.67509323)(1008.86792725,67.72510061)
\curveto(1008.94793524,67.74509316)(1009.03293515,67.76009315)(1009.12292725,67.77010061)
\lineto(1009.36292725,67.83010061)
\curveto(1009.47293471,67.86009305)(1009.5829346,67.87509303)(1009.69292725,67.87510061)
\curveto(1009.80293438,67.88509302)(1009.91293427,67.90009301)(1010.02292725,67.92010061)
\curveto(1010.07293411,67.94009297)(1010.11793407,67.94509296)(1010.15792725,67.93510061)
\curveto(1010.19793399,67.93509297)(1010.23793395,67.94009297)(1010.27792725,67.95010061)
\curveto(1010.32793386,67.96009295)(1010.3829338,67.96009295)(1010.44292725,67.95010061)
\curveto(1010.49293369,67.95009296)(1010.54293364,67.95509295)(1010.59292725,67.96510061)
\lineto(1010.72792725,67.96510061)
\curveto(1010.7879334,67.98509292)(1010.85793333,67.98509292)(1010.93792725,67.96510061)
\curveto(1011.00793318,67.95509295)(1011.07293311,67.96009295)(1011.13292725,67.98010061)
\curveto(1011.16293302,67.99009292)(1011.20293298,67.99509291)(1011.25292725,67.99510061)
\lineto(1011.37292725,67.99510061)
\lineto(1011.83792725,67.99510061)
\moveto(1014.16292725,66.45010061)
\curveto(1013.84293034,66.55009436)(1013.47793071,66.6100943)(1013.06792725,66.63010061)
\curveto(1012.65793153,66.65009426)(1012.24793194,66.66009425)(1011.83792725,66.66010061)
\curveto(1011.40793278,66.66009425)(1010.9879332,66.65009426)(1010.57792725,66.63010061)
\curveto(1010.16793402,66.6100943)(1009.7829344,66.56509434)(1009.42292725,66.49510061)
\curveto(1009.06293512,66.42509448)(1008.74293544,66.31509459)(1008.46292725,66.16510061)
\curveto(1008.17293601,66.02509488)(1007.93793625,65.83009508)(1007.75792725,65.58010061)
\curveto(1007.64793654,65.42009549)(1007.56793662,65.24009567)(1007.51792725,65.04010061)
\curveto(1007.45793673,64.84009607)(1007.42793676,64.59509631)(1007.42792725,64.30510061)
\curveto(1007.44793674,64.28509662)(1007.45793673,64.25009666)(1007.45792725,64.20010061)
\curveto(1007.44793674,64.15009676)(1007.44793674,64.1100968)(1007.45792725,64.08010061)
\curveto(1007.47793671,64.00009691)(1007.49793669,63.92509698)(1007.51792725,63.85510061)
\curveto(1007.52793666,63.79509711)(1007.54793664,63.73009718)(1007.57792725,63.66010061)
\curveto(1007.69793649,63.39009752)(1007.86793632,63.17009774)(1008.08792725,63.00010061)
\curveto(1008.29793589,62.84009807)(1008.54293564,62.7050982)(1008.82292725,62.59510061)
\curveto(1008.93293525,62.54509836)(1009.05293513,62.5050984)(1009.18292725,62.47510061)
\curveto(1009.30293488,62.45509845)(1009.42793476,62.43009848)(1009.55792725,62.40010061)
\curveto(1009.60793458,62.38009853)(1009.66293452,62.37009854)(1009.72292725,62.37010061)
\curveto(1009.77293441,62.37009854)(1009.82293436,62.36509854)(1009.87292725,62.35510061)
\curveto(1009.96293422,62.34509856)(1010.05793413,62.33509857)(1010.15792725,62.32510061)
\curveto(1010.24793394,62.31509859)(1010.34293384,62.3050986)(1010.44292725,62.29510061)
\curveto(1010.52293366,62.29509861)(1010.60793358,62.29009862)(1010.69792725,62.28010061)
\lineto(1010.93792725,62.28010061)
\lineto(1011.11792725,62.28010061)
\curveto(1011.14793304,62.27009864)(1011.182933,62.26509864)(1011.22292725,62.26510061)
\lineto(1011.35792725,62.26510061)
\lineto(1011.80792725,62.26510061)
\curveto(1011.8879323,62.26509864)(1011.97293221,62.26009865)(1012.06292725,62.25010061)
\curveto(1012.14293204,62.25009866)(1012.21793197,62.26009865)(1012.28792725,62.28010061)
\lineto(1012.55792725,62.28010061)
\curveto(1012.57793161,62.28009863)(1012.60793158,62.27509863)(1012.64792725,62.26510061)
\curveto(1012.67793151,62.26509864)(1012.70293148,62.27009864)(1012.72292725,62.28010061)
\curveto(1012.82293136,62.29009862)(1012.92293126,62.29509861)(1013.02292725,62.29510061)
\curveto(1013.11293107,62.3050986)(1013.21293097,62.31509859)(1013.32292725,62.32510061)
\curveto(1013.44293074,62.35509855)(1013.56793062,62.37009854)(1013.69792725,62.37010061)
\curveto(1013.81793037,62.38009853)(1013.93293025,62.4050985)(1014.04292725,62.44510061)
\curveto(1014.34292984,62.52509838)(1014.60792958,62.6100983)(1014.83792725,62.70010061)
\curveto(1015.06792912,62.80009811)(1015.2829289,62.94509796)(1015.48292725,63.13510061)
\curveto(1015.6829285,63.34509756)(1015.83292835,63.6100973)(1015.93292725,63.93010061)
\curveto(1015.95292823,63.97009694)(1015.96292822,64.0050969)(1015.96292725,64.03510061)
\curveto(1015.95292823,64.07509683)(1015.95792823,64.12009679)(1015.97792725,64.17010061)
\curveto(1015.9879282,64.2100967)(1015.99792819,64.28009663)(1016.00792725,64.38010061)
\curveto(1016.01792817,64.49009642)(1016.01292817,64.57509633)(1015.99292725,64.63510061)
\curveto(1015.97292821,64.7050962)(1015.96292822,64.77509613)(1015.96292725,64.84510061)
\curveto(1015.95292823,64.91509599)(1015.93792825,64.98009593)(1015.91792725,65.04010061)
\curveto(1015.85792833,65.24009567)(1015.77292841,65.42009549)(1015.66292725,65.58010061)
\curveto(1015.64292854,65.6100953)(1015.62292856,65.63509527)(1015.60292725,65.65510061)
\lineto(1015.54292725,65.71510061)
\curveto(1015.52292866,65.75509515)(1015.4829287,65.8050951)(1015.42292725,65.86510061)
\curveto(1015.2829289,65.96509494)(1015.15292903,66.05009486)(1015.03292725,66.12010061)
\curveto(1014.91292927,66.19009472)(1014.76792942,66.26009465)(1014.59792725,66.33010061)
\curveto(1014.52792966,66.36009455)(1014.45792973,66.38009453)(1014.38792725,66.39010061)
\curveto(1014.31792987,66.4100945)(1014.24292994,66.43009448)(1014.16292725,66.45010061)
}
}
{
\newrgbcolor{curcolor}{0 0 0}
\pscustom[linestyle=none,fillstyle=solid,fillcolor=curcolor]
{
\newpath
\moveto(1011.32792725,76.28470998)
\curveto(1011.40793278,76.28470235)(1011.4879327,76.28970234)(1011.56792725,76.29970998)
\curveto(1011.64793254,76.30970232)(1011.72293246,76.30470233)(1011.79292725,76.28470998)
\curveto(1011.83293235,76.26470237)(1011.87793231,76.25970237)(1011.92792725,76.26970998)
\curveto(1011.96793222,76.27970235)(1012.00793218,76.27970235)(1012.04792725,76.26970998)
\lineto(1012.19792725,76.26970998)
\curveto(1012.2879319,76.25970237)(1012.37793181,76.25470238)(1012.46792725,76.25470998)
\curveto(1012.54793164,76.25470238)(1012.62793156,76.24970238)(1012.70792725,76.23970998)
\lineto(1012.94792725,76.20970998)
\curveto(1013.01793117,76.19970243)(1013.09293109,76.18970244)(1013.17292725,76.17970998)
\curveto(1013.21293097,76.16970246)(1013.25293093,76.16470247)(1013.29292725,76.16470998)
\curveto(1013.33293085,76.16470247)(1013.37793081,76.15970247)(1013.42792725,76.14970998)
\curveto(1013.56793062,76.10970252)(1013.70793048,76.07970255)(1013.84792725,76.05970998)
\curveto(1013.9879302,76.04970258)(1014.12293006,76.01970261)(1014.25292725,75.96970998)
\curveto(1014.42292976,75.91970271)(1014.5879296,75.86470277)(1014.74792725,75.80470998)
\curveto(1014.90792928,75.75470288)(1015.06292912,75.69470294)(1015.21292725,75.62470998)
\curveto(1015.27292891,75.60470303)(1015.33292885,75.57470306)(1015.39292725,75.53470998)
\lineto(1015.54292725,75.44470998)
\curveto(1015.86292832,75.24470339)(1016.12792806,75.0297036)(1016.33792725,74.79970998)
\curveto(1016.54792764,74.56970406)(1016.72792746,74.27470436)(1016.87792725,73.91470998)
\curveto(1016.92792726,73.79470484)(1016.96292722,73.66470497)(1016.98292725,73.52470998)
\curveto(1017.00292718,73.39470524)(1017.02792716,73.25970537)(1017.05792725,73.11970998)
\curveto(1017.06792712,73.05970557)(1017.07292711,72.99970563)(1017.07292725,72.93970998)
\curveto(1017.07292711,72.87970575)(1017.07792711,72.81470582)(1017.08792725,72.74470998)
\curveto(1017.09792709,72.71470592)(1017.09792709,72.66470597)(1017.08792725,72.59470998)
\lineto(1017.08792725,72.44470998)
\lineto(1017.08792725,72.29470998)
\curveto(1017.06792712,72.21470642)(1017.05292713,72.1297065)(1017.04292725,72.03970998)
\curveto(1017.04292714,71.95970667)(1017.03292715,71.88470675)(1017.01292725,71.81470998)
\curveto(1017.00292718,71.77470686)(1016.99792719,71.73970689)(1016.99792725,71.70970998)
\curveto(1017.00792718,71.68970694)(1017.00292718,71.66470697)(1016.98292725,71.63470998)
\lineto(1016.92292725,71.36470998)
\curveto(1016.89292729,71.27470736)(1016.86292732,71.18970744)(1016.83292725,71.10970998)
\curveto(1016.59292759,70.5297081)(1016.22292796,70.09470854)(1015.72292725,69.80470998)
\curveto(1015.59292859,69.72470891)(1015.45792873,69.65970897)(1015.31792725,69.60970998)
\curveto(1015.17792901,69.56970906)(1015.02792916,69.52470911)(1014.86792725,69.47470998)
\curveto(1014.7879294,69.45470918)(1014.70792948,69.44970918)(1014.62792725,69.45970998)
\curveto(1014.54792964,69.47970915)(1014.49292969,69.51470912)(1014.46292725,69.56470998)
\curveto(1014.44292974,69.59470904)(1014.42792976,69.64970898)(1014.41792725,69.72970998)
\curveto(1014.39792979,69.80970882)(1014.3879298,69.89470874)(1014.38792725,69.98470998)
\curveto(1014.37792981,70.07470856)(1014.37792981,70.15970847)(1014.38792725,70.23970998)
\curveto(1014.39792979,70.3297083)(1014.40792978,70.39970823)(1014.41792725,70.44970998)
\curveto(1014.42792976,70.46970816)(1014.44292974,70.49470814)(1014.46292725,70.52470998)
\curveto(1014.4829297,70.56470807)(1014.50292968,70.59470804)(1014.52292725,70.61470998)
\curveto(1014.60292958,70.67470796)(1014.69792949,70.71970791)(1014.80792725,70.74970998)
\curveto(1014.91792927,70.78970784)(1015.01792917,70.8347078)(1015.10792725,70.88470998)
\curveto(1015.49792869,71.1347075)(1015.76792842,71.50470713)(1015.91792725,71.99470998)
\curveto(1015.93792825,72.06470657)(1015.95292823,72.1347065)(1015.96292725,72.20470998)
\curveto(1015.96292822,72.28470635)(1015.97292821,72.36470627)(1015.99292725,72.44470998)
\curveto(1016.00292818,72.48470615)(1016.00792818,72.53970609)(1016.00792725,72.60970998)
\curveto(1016.00792818,72.68970594)(1016.00292818,72.74470589)(1015.99292725,72.77470998)
\curveto(1015.9829282,72.80470583)(1015.97792821,72.8347058)(1015.97792725,72.86470998)
\lineto(1015.97792725,72.96970998)
\curveto(1015.95792823,73.04970558)(1015.93792825,73.12470551)(1015.91792725,73.19470998)
\curveto(1015.89792829,73.27470536)(1015.87292831,73.34970528)(1015.84292725,73.41970998)
\curveto(1015.69292849,73.76970486)(1015.47792871,74.03970459)(1015.19792725,74.22970998)
\curveto(1014.91792927,74.41970421)(1014.59292959,74.57470406)(1014.22292725,74.69470998)
\curveto(1014.14293004,74.72470391)(1014.06793012,74.74470389)(1013.99792725,74.75470998)
\curveto(1013.92793026,74.77470386)(1013.85293033,74.79470384)(1013.77292725,74.81470998)
\curveto(1013.6829305,74.8347038)(1013.5879306,74.84970378)(1013.48792725,74.85970998)
\curveto(1013.37793081,74.87970375)(1013.27293091,74.89970373)(1013.17292725,74.91970998)
\curveto(1013.12293106,74.9297037)(1013.07293111,74.9347037)(1013.02292725,74.93470998)
\curveto(1012.96293122,74.94470369)(1012.90793128,74.94970368)(1012.85792725,74.94970998)
\curveto(1012.79793139,74.96970366)(1012.72293146,74.97970365)(1012.63292725,74.97970998)
\curveto(1012.53293165,74.97970365)(1012.45293173,74.96970366)(1012.39292725,74.94970998)
\curveto(1012.30293188,74.91970371)(1012.26293192,74.86970376)(1012.27292725,74.79970998)
\curveto(1012.2829319,74.73970389)(1012.31293187,74.68470395)(1012.36292725,74.63470998)
\curveto(1012.41293177,74.55470408)(1012.47293171,74.48470415)(1012.54292725,74.42470998)
\curveto(1012.61293157,74.37470426)(1012.67293151,74.30970432)(1012.72292725,74.22970998)
\curveto(1012.83293135,74.06970456)(1012.93293125,73.90470473)(1013.02292725,73.73470998)
\curveto(1013.10293108,73.56470507)(1013.17293101,73.36970526)(1013.23292725,73.14970998)
\curveto(1013.26293092,73.04970558)(1013.27793091,72.94970568)(1013.27792725,72.84970998)
\curveto(1013.27793091,72.75970587)(1013.2879309,72.65970597)(1013.30792725,72.54970998)
\lineto(1013.30792725,72.39970998)
\curveto(1013.2879309,72.34970628)(1013.2829309,72.29970633)(1013.29292725,72.24970998)
\curveto(1013.30293088,72.20970642)(1013.30293088,72.16970646)(1013.29292725,72.12970998)
\curveto(1013.2829309,72.09970653)(1013.27793091,72.05470658)(1013.27792725,71.99470998)
\curveto(1013.26793092,71.9347067)(1013.25793093,71.86970676)(1013.24792725,71.79970998)
\lineto(1013.21792725,71.61970998)
\curveto(1013.09793109,71.16970746)(1012.93293125,70.78970784)(1012.72292725,70.47970998)
\curveto(1012.53293165,70.20970842)(1012.30293188,69.97970865)(1012.03292725,69.78970998)
\curveto(1011.75293243,69.60970902)(1011.43793275,69.46470917)(1011.08792725,69.35470998)
\lineto(1010.87792725,69.29470998)
\curveto(1010.79793339,69.28470935)(1010.71793347,69.26970936)(1010.63792725,69.24970998)
\curveto(1010.60793358,69.23970939)(1010.57793361,69.2347094)(1010.54792725,69.23470998)
\curveto(1010.51793367,69.2347094)(1010.4879337,69.2297094)(1010.45792725,69.21970998)
\curveto(1010.39793379,69.20970942)(1010.33793385,69.20470943)(1010.27792725,69.20470998)
\curveto(1010.20793398,69.20470943)(1010.14793404,69.19470944)(1010.09792725,69.17470998)
\lineto(1009.91792725,69.17470998)
\curveto(1009.86793432,69.16470947)(1009.79793439,69.15970947)(1009.70792725,69.15970998)
\curveto(1009.61793457,69.15970947)(1009.54793464,69.16970946)(1009.49792725,69.18970998)
\lineto(1009.33292725,69.18970998)
\curveto(1009.25293493,69.20970942)(1009.17793501,69.21970941)(1009.10792725,69.21970998)
\curveto(1009.03793515,69.2297094)(1008.96793522,69.24470939)(1008.89792725,69.26470998)
\curveto(1008.69793549,69.32470931)(1008.50793568,69.38470925)(1008.32792725,69.44470998)
\curveto(1008.14793604,69.51470912)(1007.97793621,69.60470903)(1007.81792725,69.71470998)
\curveto(1007.74793644,69.75470888)(1007.6829365,69.79470884)(1007.62292725,69.83470998)
\lineto(1007.44292725,69.98470998)
\curveto(1007.43293675,70.00470863)(1007.41793677,70.02470861)(1007.39792725,70.04470998)
\curveto(1007.26793692,70.1347085)(1007.15793703,70.24470839)(1007.06792725,70.37470998)
\curveto(1006.86793732,70.634708)(1006.71293747,70.89970773)(1006.60292725,71.16970998)
\curveto(1006.56293762,71.24970738)(1006.53293765,71.3297073)(1006.51292725,71.40970998)
\curveto(1006.4829377,71.49970713)(1006.45793773,71.58970704)(1006.43792725,71.67970998)
\curveto(1006.40793778,71.77970685)(1006.3879378,71.87970675)(1006.37792725,71.97970998)
\curveto(1006.36793782,72.07970655)(1006.35293783,72.18470645)(1006.33292725,72.29470998)
\curveto(1006.32293786,72.32470631)(1006.32293786,72.36470627)(1006.33292725,72.41470998)
\curveto(1006.34293784,72.47470616)(1006.33793785,72.51470612)(1006.31792725,72.53470998)
\curveto(1006.29793789,73.25470538)(1006.41293777,73.85470478)(1006.66292725,74.33470998)
\curveto(1006.91293727,74.81470382)(1007.25293693,75.18970344)(1007.68292725,75.45970998)
\curveto(1007.82293636,75.54970308)(1007.96793622,75.629703)(1008.11792725,75.69970998)
\curveto(1008.26793592,75.76970286)(1008.42793576,75.83970279)(1008.59792725,75.90970998)
\curveto(1008.73793545,75.95970267)(1008.8879353,75.99970263)(1009.04792725,76.02970998)
\curveto(1009.20793498,76.05970257)(1009.36793482,76.09470254)(1009.52792725,76.13470998)
\curveto(1009.57793461,76.15470248)(1009.63293455,76.16470247)(1009.69292725,76.16470998)
\curveto(1009.74293444,76.16470247)(1009.79293439,76.16970246)(1009.84292725,76.17970998)
\curveto(1009.90293428,76.19970243)(1009.96793422,76.20970242)(1010.03792725,76.20970998)
\curveto(1010.09793409,76.20970242)(1010.15293403,76.21970241)(1010.20292725,76.23970998)
\lineto(1010.36792725,76.23970998)
\curveto(1010.41793377,76.25970237)(1010.46793372,76.26470237)(1010.51792725,76.25470998)
\curveto(1010.56793362,76.24470239)(1010.61793357,76.24970238)(1010.66792725,76.26970998)
\curveto(1010.6879335,76.26970236)(1010.71293347,76.26470237)(1010.74292725,76.25470998)
\curveto(1010.77293341,76.25470238)(1010.79793339,76.25970237)(1010.81792725,76.26970998)
\curveto(1010.84793334,76.27970235)(1010.8829333,76.27970235)(1010.92292725,76.26970998)
\curveto(1010.96293322,76.26970236)(1011.00293318,76.27470236)(1011.04292725,76.28470998)
\curveto(1011.0829331,76.29470234)(1011.12793306,76.29470234)(1011.17792725,76.28470998)
\lineto(1011.32792725,76.28470998)
\moveto(1010.02292725,74.78470998)
\curveto(1009.97293421,74.79470384)(1009.91293427,74.79970383)(1009.84292725,74.79970998)
\curveto(1009.77293441,74.79970383)(1009.71293447,74.79470384)(1009.66292725,74.78470998)
\curveto(1009.61293457,74.77470386)(1009.53793465,74.76970386)(1009.43792725,74.76970998)
\curveto(1009.35793483,74.74970388)(1009.2829349,74.7297039)(1009.21292725,74.70970998)
\curveto(1009.14293504,74.69970393)(1009.07293511,74.68470395)(1009.00292725,74.66470998)
\curveto(1008.57293561,74.52470411)(1008.23793595,74.3297043)(1007.99792725,74.07970998)
\curveto(1007.75793643,73.83970479)(1007.57793661,73.49470514)(1007.45792725,73.04470998)
\curveto(1007.43793675,72.95470568)(1007.42793676,72.85470578)(1007.42792725,72.74470998)
\lineto(1007.42792725,72.41470998)
\curveto(1007.44793674,72.39470624)(1007.45793673,72.35970627)(1007.45792725,72.30970998)
\curveto(1007.44793674,72.25970637)(1007.44793674,72.21470642)(1007.45792725,72.17470998)
\curveto(1007.47793671,72.09470654)(1007.49793669,72.01970661)(1007.51792725,71.94970998)
\lineto(1007.57792725,71.73970998)
\curveto(1007.70793648,71.44970718)(1007.8879363,71.21970741)(1008.11792725,71.04970998)
\curveto(1008.33793585,70.87970775)(1008.59793559,70.74470789)(1008.89792725,70.64470998)
\curveto(1008.9879352,70.61470802)(1009.0829351,70.58970804)(1009.18292725,70.56970998)
\curveto(1009.27293491,70.55970807)(1009.36793482,70.54470809)(1009.46792725,70.52470998)
\lineto(1009.60292725,70.52470998)
\curveto(1009.71293447,70.49470814)(1009.85293433,70.48470815)(1010.02292725,70.49470998)
\curveto(1010.182934,70.51470812)(1010.31293387,70.5347081)(1010.41292725,70.55470998)
\curveto(1010.47293371,70.57470806)(1010.53293365,70.58970804)(1010.59292725,70.59970998)
\curveto(1010.64293354,70.60970802)(1010.69293349,70.62470801)(1010.74292725,70.64470998)
\curveto(1010.94293324,70.72470791)(1011.13293305,70.81970781)(1011.31292725,70.92970998)
\curveto(1011.49293269,71.04970758)(1011.63793255,71.18970744)(1011.74792725,71.34970998)
\curveto(1011.79793239,71.39970723)(1011.83793235,71.45470718)(1011.86792725,71.51470998)
\curveto(1011.89793229,71.57470706)(1011.93293225,71.634707)(1011.97292725,71.69470998)
\curveto(1012.05293213,71.84470679)(1012.11793207,72.0297066)(1012.16792725,72.24970998)
\curveto(1012.187932,72.29970633)(1012.19293199,72.33970629)(1012.18292725,72.36970998)
\curveto(1012.17293201,72.40970622)(1012.17793201,72.45470618)(1012.19792725,72.50470998)
\curveto(1012.20793198,72.54470609)(1012.21293197,72.59970603)(1012.21292725,72.66970998)
\curveto(1012.21293197,72.73970589)(1012.20793198,72.79970583)(1012.19792725,72.84970998)
\curveto(1012.17793201,72.94970568)(1012.16293202,73.04470559)(1012.15292725,73.13470998)
\curveto(1012.13293205,73.22470541)(1012.10293208,73.31470532)(1012.06292725,73.40470998)
\curveto(1011.84293234,73.94470469)(1011.44793274,74.33970429)(1010.87792725,74.58970998)
\curveto(1010.77793341,74.63970399)(1010.67793351,74.67470396)(1010.57792725,74.69470998)
\curveto(1010.46793372,74.71470392)(1010.35793383,74.73970389)(1010.24792725,74.76970998)
\curveto(1010.14793404,74.76970386)(1010.07293411,74.77470386)(1010.02292725,74.78470998)
}
}
{
\newrgbcolor{curcolor}{0 0 0}
\pscustom[linestyle=none,fillstyle=solid,fillcolor=curcolor]
{
\newpath
\moveto(1015.28792725,78.63431936)
\lineto(1015.28792725,79.26431936)
\lineto(1015.28792725,79.45931936)
\curveto(1015.2879289,79.52931683)(1015.29792889,79.58931677)(1015.31792725,79.63931936)
\curveto(1015.35792883,79.70931665)(1015.39792879,79.7593166)(1015.43792725,79.78931936)
\curveto(1015.4879287,79.82931653)(1015.55292863,79.84931651)(1015.63292725,79.84931936)
\curveto(1015.71292847,79.8593165)(1015.79792839,79.86431649)(1015.88792725,79.86431936)
\lineto(1016.60792725,79.86431936)
\curveto(1017.0879271,79.86431649)(1017.49792669,79.80431655)(1017.83792725,79.68431936)
\curveto(1018.17792601,79.56431679)(1018.45292573,79.36931699)(1018.66292725,79.09931936)
\curveto(1018.71292547,79.02931733)(1018.75792543,78.9593174)(1018.79792725,78.88931936)
\curveto(1018.84792534,78.82931753)(1018.89292529,78.7543176)(1018.93292725,78.66431936)
\curveto(1018.94292524,78.64431771)(1018.95292523,78.61431774)(1018.96292725,78.57431936)
\curveto(1018.9829252,78.53431782)(1018.9879252,78.48931787)(1018.97792725,78.43931936)
\curveto(1018.94792524,78.34931801)(1018.87292531,78.29431806)(1018.75292725,78.27431936)
\curveto(1018.64292554,78.2543181)(1018.54792564,78.26931809)(1018.46792725,78.31931936)
\curveto(1018.39792579,78.34931801)(1018.33292585,78.39431796)(1018.27292725,78.45431936)
\curveto(1018.22292596,78.52431783)(1018.17292601,78.58931777)(1018.12292725,78.64931936)
\curveto(1018.07292611,78.71931764)(1017.99792619,78.77931758)(1017.89792725,78.82931936)
\curveto(1017.80792638,78.88931747)(1017.71792647,78.93931742)(1017.62792725,78.97931936)
\curveto(1017.59792659,78.99931736)(1017.53792665,79.02431733)(1017.44792725,79.05431936)
\curveto(1017.36792682,79.08431727)(1017.29792689,79.08931727)(1017.23792725,79.06931936)
\curveto(1017.09792709,79.03931732)(1017.00792718,78.97931738)(1016.96792725,78.88931936)
\curveto(1016.93792725,78.80931755)(1016.92292726,78.71931764)(1016.92292725,78.61931936)
\curveto(1016.92292726,78.51931784)(1016.89792729,78.43431792)(1016.84792725,78.36431936)
\curveto(1016.77792741,78.27431808)(1016.63792755,78.22931813)(1016.42792725,78.22931936)
\lineto(1015.87292725,78.22931936)
\lineto(1015.64792725,78.22931936)
\curveto(1015.56792862,78.23931812)(1015.50292868,78.2593181)(1015.45292725,78.28931936)
\curveto(1015.37292881,78.34931801)(1015.32792886,78.41931794)(1015.31792725,78.49931936)
\curveto(1015.30792888,78.51931784)(1015.30292888,78.53931782)(1015.30292725,78.55931936)
\curveto(1015.30292888,78.58931777)(1015.29792889,78.61431774)(1015.28792725,78.63431936)
}
}
{
\newrgbcolor{curcolor}{0 0 0}
\pscustom[linestyle=none,fillstyle=solid,fillcolor=curcolor]
{
}
}
{
\newrgbcolor{curcolor}{0 0 0}
\pscustom[linestyle=none,fillstyle=solid,fillcolor=curcolor]
{
\newpath
\moveto(1006.31792725,89.26463186)
\curveto(1006.30793788,89.95462722)(1006.42793776,90.55462662)(1006.67792725,91.06463186)
\curveto(1006.92793726,91.58462559)(1007.26293692,91.9796252)(1007.68292725,92.24963186)
\curveto(1007.76293642,92.29962488)(1007.85293633,92.34462483)(1007.95292725,92.38463186)
\curveto(1008.04293614,92.42462475)(1008.13793605,92.46962471)(1008.23792725,92.51963186)
\curveto(1008.33793585,92.55962462)(1008.43793575,92.58962459)(1008.53792725,92.60963186)
\curveto(1008.63793555,92.62962455)(1008.74293544,92.64962453)(1008.85292725,92.66963186)
\curveto(1008.90293528,92.68962449)(1008.94793524,92.69462448)(1008.98792725,92.68463186)
\curveto(1009.02793516,92.6746245)(1009.07293511,92.6796245)(1009.12292725,92.69963186)
\curveto(1009.17293501,92.70962447)(1009.25793493,92.71462446)(1009.37792725,92.71463186)
\curveto(1009.4879347,92.71462446)(1009.57293461,92.70962447)(1009.63292725,92.69963186)
\curveto(1009.69293449,92.6796245)(1009.75293443,92.66962451)(1009.81292725,92.66963186)
\curveto(1009.87293431,92.6796245)(1009.93293425,92.6746245)(1009.99292725,92.65463186)
\curveto(1010.13293405,92.61462456)(1010.26793392,92.5796246)(1010.39792725,92.54963186)
\curveto(1010.52793366,92.51962466)(1010.65293353,92.4796247)(1010.77292725,92.42963186)
\curveto(1010.91293327,92.36962481)(1011.03793315,92.29962488)(1011.14792725,92.21963186)
\curveto(1011.25793293,92.14962503)(1011.36793282,92.0746251)(1011.47792725,91.99463186)
\lineto(1011.53792725,91.93463186)
\curveto(1011.55793263,91.92462525)(1011.57793261,91.90962527)(1011.59792725,91.88963186)
\curveto(1011.75793243,91.76962541)(1011.90293228,91.63462554)(1012.03292725,91.48463186)
\curveto(1012.16293202,91.33462584)(1012.2879319,91.174626)(1012.40792725,91.00463186)
\curveto(1012.62793156,90.69462648)(1012.83293135,90.39962678)(1013.02292725,90.11963186)
\curveto(1013.16293102,89.88962729)(1013.29793089,89.65962752)(1013.42792725,89.42963186)
\curveto(1013.55793063,89.20962797)(1013.69293049,88.98962819)(1013.83292725,88.76963186)
\curveto(1014.00293018,88.51962866)(1014.18293,88.2796289)(1014.37292725,88.04963186)
\curveto(1014.56292962,87.82962935)(1014.7879294,87.63962954)(1015.04792725,87.47963186)
\curveto(1015.10792908,87.43962974)(1015.16792902,87.40462977)(1015.22792725,87.37463186)
\curveto(1015.27792891,87.34462983)(1015.34292884,87.31462986)(1015.42292725,87.28463186)
\curveto(1015.49292869,87.26462991)(1015.55292863,87.25962992)(1015.60292725,87.26963186)
\curveto(1015.67292851,87.28962989)(1015.72792846,87.32462985)(1015.76792725,87.37463186)
\curveto(1015.79792839,87.42462975)(1015.81792837,87.48462969)(1015.82792725,87.55463186)
\lineto(1015.82792725,87.79463186)
\lineto(1015.82792725,88.54463186)
\lineto(1015.82792725,91.34963186)
\lineto(1015.82792725,92.00963186)
\curveto(1015.82792836,92.09962508)(1015.83292835,92.18462499)(1015.84292725,92.26463186)
\curveto(1015.84292834,92.34462483)(1015.86292832,92.40962477)(1015.90292725,92.45963186)
\curveto(1015.94292824,92.50962467)(1016.01792817,92.54962463)(1016.12792725,92.57963186)
\curveto(1016.22792796,92.61962456)(1016.32792786,92.62962455)(1016.42792725,92.60963186)
\lineto(1016.56292725,92.60963186)
\curveto(1016.63292755,92.58962459)(1016.69292749,92.56962461)(1016.74292725,92.54963186)
\curveto(1016.79292739,92.52962465)(1016.83292735,92.49462468)(1016.86292725,92.44463186)
\curveto(1016.90292728,92.39462478)(1016.92292726,92.32462485)(1016.92292725,92.23463186)
\lineto(1016.92292725,91.96463186)
\lineto(1016.92292725,91.06463186)
\lineto(1016.92292725,87.55463186)
\lineto(1016.92292725,86.48963186)
\curveto(1016.92292726,86.40963077)(1016.92792726,86.31963086)(1016.93792725,86.21963186)
\curveto(1016.93792725,86.11963106)(1016.92792726,86.03463114)(1016.90792725,85.96463186)
\curveto(1016.83792735,85.75463142)(1016.65792753,85.68963149)(1016.36792725,85.76963186)
\curveto(1016.32792786,85.7796314)(1016.29292789,85.7796314)(1016.26292725,85.76963186)
\curveto(1016.22292796,85.76963141)(1016.17792801,85.7796314)(1016.12792725,85.79963186)
\curveto(1016.04792814,85.81963136)(1015.96292822,85.83963134)(1015.87292725,85.85963186)
\curveto(1015.7829284,85.8796313)(1015.69792849,85.90463127)(1015.61792725,85.93463186)
\curveto(1015.12792906,86.09463108)(1014.71292947,86.29463088)(1014.37292725,86.53463186)
\curveto(1014.12293006,86.71463046)(1013.89793029,86.91963026)(1013.69792725,87.14963186)
\curveto(1013.4879307,87.3796298)(1013.29293089,87.61962956)(1013.11292725,87.86963186)
\curveto(1012.93293125,88.12962905)(1012.76293142,88.39462878)(1012.60292725,88.66463186)
\curveto(1012.43293175,88.94462823)(1012.25793193,89.21462796)(1012.07792725,89.47463186)
\curveto(1011.99793219,89.58462759)(1011.92293226,89.68962749)(1011.85292725,89.78963186)
\curveto(1011.7829324,89.89962728)(1011.70793248,90.00962717)(1011.62792725,90.11963186)
\curveto(1011.59793259,90.15962702)(1011.56793262,90.19462698)(1011.53792725,90.22463186)
\curveto(1011.49793269,90.26462691)(1011.46793272,90.30462687)(1011.44792725,90.34463186)
\curveto(1011.33793285,90.48462669)(1011.21293297,90.60962657)(1011.07292725,90.71963186)
\curveto(1011.04293314,90.73962644)(1011.01793317,90.76462641)(1010.99792725,90.79463186)
\curveto(1010.96793322,90.82462635)(1010.93793325,90.84962633)(1010.90792725,90.86963186)
\curveto(1010.80793338,90.94962623)(1010.70793348,91.01462616)(1010.60792725,91.06463186)
\curveto(1010.50793368,91.12462605)(1010.39793379,91.179626)(1010.27792725,91.22963186)
\curveto(1010.20793398,91.25962592)(1010.13293405,91.2796259)(1010.05292725,91.28963186)
\lineto(1009.81292725,91.34963186)
\lineto(1009.72292725,91.34963186)
\curveto(1009.69293449,91.35962582)(1009.66293452,91.36462581)(1009.63292725,91.36463186)
\curveto(1009.56293462,91.38462579)(1009.46793472,91.38962579)(1009.34792725,91.37963186)
\curveto(1009.21793497,91.3796258)(1009.11793507,91.36962581)(1009.04792725,91.34963186)
\curveto(1008.96793522,91.32962585)(1008.89293529,91.30962587)(1008.82292725,91.28963186)
\curveto(1008.74293544,91.2796259)(1008.66293552,91.25962592)(1008.58292725,91.22963186)
\curveto(1008.34293584,91.11962606)(1008.14293604,90.96962621)(1007.98292725,90.77963186)
\curveto(1007.81293637,90.59962658)(1007.67293651,90.3796268)(1007.56292725,90.11963186)
\curveto(1007.54293664,90.04962713)(1007.52793666,89.9796272)(1007.51792725,89.90963186)
\curveto(1007.49793669,89.83962734)(1007.47793671,89.76462741)(1007.45792725,89.68463186)
\curveto(1007.43793675,89.60462757)(1007.42793676,89.49462768)(1007.42792725,89.35463186)
\curveto(1007.42793676,89.22462795)(1007.43793675,89.11962806)(1007.45792725,89.03963186)
\curveto(1007.46793672,88.9796282)(1007.47293671,88.92462825)(1007.47292725,88.87463186)
\curveto(1007.47293671,88.82462835)(1007.4829367,88.7746284)(1007.50292725,88.72463186)
\curveto(1007.54293664,88.62462855)(1007.5829366,88.52962865)(1007.62292725,88.43963186)
\curveto(1007.66293652,88.35962882)(1007.70793648,88.2796289)(1007.75792725,88.19963186)
\curveto(1007.77793641,88.16962901)(1007.80293638,88.13962904)(1007.83292725,88.10963186)
\curveto(1007.86293632,88.08962909)(1007.8879363,88.06462911)(1007.90792725,88.03463186)
\lineto(1007.98292725,87.95963186)
\curveto(1008.00293618,87.92962925)(1008.02293616,87.90462927)(1008.04292725,87.88463186)
\lineto(1008.25292725,87.73463186)
\curveto(1008.31293587,87.69462948)(1008.37793581,87.64962953)(1008.44792725,87.59963186)
\curveto(1008.53793565,87.53962964)(1008.64293554,87.48962969)(1008.76292725,87.44963186)
\curveto(1008.87293531,87.41962976)(1008.9829352,87.38462979)(1009.09292725,87.34463186)
\curveto(1009.20293498,87.30462987)(1009.34793484,87.2796299)(1009.52792725,87.26963186)
\curveto(1009.69793449,87.25962992)(1009.82293436,87.22962995)(1009.90292725,87.17963186)
\curveto(1009.9829342,87.12963005)(1010.02793416,87.05463012)(1010.03792725,86.95463186)
\curveto(1010.04793414,86.85463032)(1010.05293413,86.74463043)(1010.05292725,86.62463186)
\curveto(1010.05293413,86.58463059)(1010.05793413,86.54463063)(1010.06792725,86.50463186)
\curveto(1010.06793412,86.46463071)(1010.06293412,86.42963075)(1010.05292725,86.39963186)
\curveto(1010.03293415,86.34963083)(1010.02293416,86.29963088)(1010.02292725,86.24963186)
\curveto(1010.02293416,86.20963097)(1010.01293417,86.16963101)(1009.99292725,86.12963186)
\curveto(1009.93293425,86.03963114)(1009.79793439,85.99463118)(1009.58792725,85.99463186)
\lineto(1009.46792725,85.99463186)
\curveto(1009.40793478,86.00463117)(1009.34793484,86.00963117)(1009.28792725,86.00963186)
\curveto(1009.21793497,86.01963116)(1009.15293503,86.02963115)(1009.09292725,86.03963186)
\curveto(1008.9829352,86.05963112)(1008.8829353,86.0796311)(1008.79292725,86.09963186)
\curveto(1008.69293549,86.11963106)(1008.59793559,86.14963103)(1008.50792725,86.18963186)
\curveto(1008.43793575,86.20963097)(1008.37793581,86.22963095)(1008.32792725,86.24963186)
\lineto(1008.14792725,86.30963186)
\curveto(1007.8879363,86.42963075)(1007.64293654,86.58463059)(1007.41292725,86.77463186)
\curveto(1007.182937,86.9746302)(1006.99793719,87.18962999)(1006.85792725,87.41963186)
\curveto(1006.77793741,87.52962965)(1006.71293747,87.64462953)(1006.66292725,87.76463186)
\lineto(1006.51292725,88.15463186)
\curveto(1006.46293772,88.26462891)(1006.43293775,88.3796288)(1006.42292725,88.49963186)
\curveto(1006.40293778,88.61962856)(1006.37793781,88.74462843)(1006.34792725,88.87463186)
\curveto(1006.34793784,88.94462823)(1006.34793784,89.00962817)(1006.34792725,89.06963186)
\curveto(1006.33793785,89.12962805)(1006.32793786,89.19462798)(1006.31792725,89.26463186)
}
}
{
\newrgbcolor{curcolor}{0 0 0}
\pscustom[linestyle=none,fillstyle=solid,fillcolor=curcolor]
{
\newpath
\moveto(1011.83792725,101.36424123)
\lineto(1012.09292725,101.36424123)
\curveto(1012.17293201,101.37423353)(1012.24793194,101.36923353)(1012.31792725,101.34924123)
\lineto(1012.55792725,101.34924123)
\lineto(1012.72292725,101.34924123)
\curveto(1012.82293136,101.32923357)(1012.92793126,101.31923358)(1013.03792725,101.31924123)
\curveto(1013.13793105,101.31923358)(1013.23793095,101.30923359)(1013.33792725,101.28924123)
\lineto(1013.48792725,101.28924123)
\curveto(1013.62793056,101.25923364)(1013.76793042,101.23923366)(1013.90792725,101.22924123)
\curveto(1014.03793015,101.21923368)(1014.16793002,101.19423371)(1014.29792725,101.15424123)
\curveto(1014.37792981,101.13423377)(1014.46292972,101.11423379)(1014.55292725,101.09424123)
\lineto(1014.79292725,101.03424123)
\lineto(1015.09292725,100.91424123)
\curveto(1015.182929,100.88423402)(1015.27292891,100.84923405)(1015.36292725,100.80924123)
\curveto(1015.5829286,100.70923419)(1015.79792839,100.57423433)(1016.00792725,100.40424123)
\curveto(1016.21792797,100.24423466)(1016.3879278,100.06923483)(1016.51792725,99.87924123)
\curveto(1016.55792763,99.82923507)(1016.59792759,99.76923513)(1016.63792725,99.69924123)
\curveto(1016.66792752,99.63923526)(1016.70292748,99.57923532)(1016.74292725,99.51924123)
\curveto(1016.79292739,99.43923546)(1016.83292735,99.34423556)(1016.86292725,99.23424123)
\curveto(1016.89292729,99.12423578)(1016.92292726,99.01923588)(1016.95292725,98.91924123)
\curveto(1016.99292719,98.80923609)(1017.01792717,98.6992362)(1017.02792725,98.58924123)
\curveto(1017.03792715,98.47923642)(1017.05292713,98.36423654)(1017.07292725,98.24424123)
\curveto(1017.0829271,98.2042367)(1017.0829271,98.15923674)(1017.07292725,98.10924123)
\curveto(1017.07292711,98.06923683)(1017.07792711,98.02923687)(1017.08792725,97.98924123)
\curveto(1017.09792709,97.94923695)(1017.10292708,97.89423701)(1017.10292725,97.82424123)
\curveto(1017.10292708,97.75423715)(1017.09792709,97.7042372)(1017.08792725,97.67424123)
\curveto(1017.06792712,97.62423728)(1017.06292712,97.57923732)(1017.07292725,97.53924123)
\curveto(1017.0829271,97.4992374)(1017.0829271,97.46423744)(1017.07292725,97.43424123)
\lineto(1017.07292725,97.34424123)
\curveto(1017.05292713,97.28423762)(1017.03792715,97.21923768)(1017.02792725,97.14924123)
\curveto(1017.02792716,97.08923781)(1017.02292716,97.02423788)(1017.01292725,96.95424123)
\curveto(1016.96292722,96.78423812)(1016.91292727,96.62423828)(1016.86292725,96.47424123)
\curveto(1016.81292737,96.32423858)(1016.74792744,96.17923872)(1016.66792725,96.03924123)
\curveto(1016.62792756,95.98923891)(1016.59792759,95.93423897)(1016.57792725,95.87424123)
\curveto(1016.54792764,95.82423908)(1016.51292767,95.77423913)(1016.47292725,95.72424123)
\curveto(1016.29292789,95.48423942)(1016.07292811,95.28423962)(1015.81292725,95.12424123)
\curveto(1015.55292863,94.96423994)(1015.26792892,94.82424008)(1014.95792725,94.70424123)
\curveto(1014.81792937,94.64424026)(1014.67792951,94.5992403)(1014.53792725,94.56924123)
\curveto(1014.3879298,94.53924036)(1014.23292995,94.5042404)(1014.07292725,94.46424123)
\curveto(1013.96293022,94.44424046)(1013.85293033,94.42924047)(1013.74292725,94.41924123)
\curveto(1013.63293055,94.40924049)(1013.52293066,94.39424051)(1013.41292725,94.37424123)
\curveto(1013.37293081,94.36424054)(1013.33293085,94.35924054)(1013.29292725,94.35924123)
\curveto(1013.25293093,94.36924053)(1013.21293097,94.36924053)(1013.17292725,94.35924123)
\curveto(1013.12293106,94.34924055)(1013.07293111,94.34424056)(1013.02292725,94.34424123)
\lineto(1012.85792725,94.34424123)
\curveto(1012.80793138,94.32424058)(1012.75793143,94.31924058)(1012.70792725,94.32924123)
\curveto(1012.64793154,94.33924056)(1012.59293159,94.33924056)(1012.54292725,94.32924123)
\curveto(1012.50293168,94.31924058)(1012.45793173,94.31924058)(1012.40792725,94.32924123)
\curveto(1012.35793183,94.33924056)(1012.30793188,94.33424057)(1012.25792725,94.31424123)
\curveto(1012.187932,94.29424061)(1012.11293207,94.28924061)(1012.03292725,94.29924123)
\curveto(1011.94293224,94.30924059)(1011.85793233,94.31424059)(1011.77792725,94.31424123)
\curveto(1011.6879325,94.31424059)(1011.5879326,94.30924059)(1011.47792725,94.29924123)
\curveto(1011.35793283,94.28924061)(1011.25793293,94.29424061)(1011.17792725,94.31424123)
\lineto(1010.89292725,94.31424123)
\lineto(1010.26292725,94.35924123)
\curveto(1010.16293402,94.36924053)(1010.06793412,94.37924052)(1009.97792725,94.38924123)
\lineto(1009.67792725,94.41924123)
\curveto(1009.62793456,94.43924046)(1009.57793461,94.44424046)(1009.52792725,94.43424123)
\curveto(1009.46793472,94.43424047)(1009.41293477,94.44424046)(1009.36292725,94.46424123)
\curveto(1009.19293499,94.51424039)(1009.02793516,94.55424035)(1008.86792725,94.58424123)
\curveto(1008.69793549,94.61424029)(1008.53793565,94.66424024)(1008.38792725,94.73424123)
\curveto(1007.92793626,94.92423998)(1007.55293663,95.14423976)(1007.26292725,95.39424123)
\curveto(1006.97293721,95.65423925)(1006.72793746,96.01423889)(1006.52792725,96.47424123)
\curveto(1006.47793771,96.6042383)(1006.44293774,96.73423817)(1006.42292725,96.86424123)
\curveto(1006.40293778,97.0042379)(1006.37793781,97.14423776)(1006.34792725,97.28424123)
\curveto(1006.33793785,97.35423755)(1006.33293785,97.41923748)(1006.33292725,97.47924123)
\curveto(1006.33293785,97.53923736)(1006.32793786,97.6042373)(1006.31792725,97.67424123)
\curveto(1006.29793789,98.5042364)(1006.44793774,99.17423573)(1006.76792725,99.68424123)
\curveto(1007.07793711,100.19423471)(1007.51793667,100.57423433)(1008.08792725,100.82424123)
\curveto(1008.20793598,100.87423403)(1008.33293585,100.91923398)(1008.46292725,100.95924123)
\curveto(1008.59293559,100.9992339)(1008.72793546,101.04423386)(1008.86792725,101.09424123)
\curveto(1008.94793524,101.11423379)(1009.03293515,101.12923377)(1009.12292725,101.13924123)
\lineto(1009.36292725,101.19924123)
\curveto(1009.47293471,101.22923367)(1009.5829346,101.24423366)(1009.69292725,101.24424123)
\curveto(1009.80293438,101.25423365)(1009.91293427,101.26923363)(1010.02292725,101.28924123)
\curveto(1010.07293411,101.30923359)(1010.11793407,101.31423359)(1010.15792725,101.30424123)
\curveto(1010.19793399,101.3042336)(1010.23793395,101.30923359)(1010.27792725,101.31924123)
\curveto(1010.32793386,101.32923357)(1010.3829338,101.32923357)(1010.44292725,101.31924123)
\curveto(1010.49293369,101.31923358)(1010.54293364,101.32423358)(1010.59292725,101.33424123)
\lineto(1010.72792725,101.33424123)
\curveto(1010.7879334,101.35423355)(1010.85793333,101.35423355)(1010.93792725,101.33424123)
\curveto(1011.00793318,101.32423358)(1011.07293311,101.32923357)(1011.13292725,101.34924123)
\curveto(1011.16293302,101.35923354)(1011.20293298,101.36423354)(1011.25292725,101.36424123)
\lineto(1011.37292725,101.36424123)
\lineto(1011.83792725,101.36424123)
\moveto(1014.16292725,99.81924123)
\curveto(1013.84293034,99.91923498)(1013.47793071,99.97923492)(1013.06792725,99.99924123)
\curveto(1012.65793153,100.01923488)(1012.24793194,100.02923487)(1011.83792725,100.02924123)
\curveto(1011.40793278,100.02923487)(1010.9879332,100.01923488)(1010.57792725,99.99924123)
\curveto(1010.16793402,99.97923492)(1009.7829344,99.93423497)(1009.42292725,99.86424123)
\curveto(1009.06293512,99.79423511)(1008.74293544,99.68423522)(1008.46292725,99.53424123)
\curveto(1008.17293601,99.39423551)(1007.93793625,99.1992357)(1007.75792725,98.94924123)
\curveto(1007.64793654,98.78923611)(1007.56793662,98.60923629)(1007.51792725,98.40924123)
\curveto(1007.45793673,98.20923669)(1007.42793676,97.96423694)(1007.42792725,97.67424123)
\curveto(1007.44793674,97.65423725)(1007.45793673,97.61923728)(1007.45792725,97.56924123)
\curveto(1007.44793674,97.51923738)(1007.44793674,97.47923742)(1007.45792725,97.44924123)
\curveto(1007.47793671,97.36923753)(1007.49793669,97.29423761)(1007.51792725,97.22424123)
\curveto(1007.52793666,97.16423774)(1007.54793664,97.0992378)(1007.57792725,97.02924123)
\curveto(1007.69793649,96.75923814)(1007.86793632,96.53923836)(1008.08792725,96.36924123)
\curveto(1008.29793589,96.20923869)(1008.54293564,96.07423883)(1008.82292725,95.96424123)
\curveto(1008.93293525,95.91423899)(1009.05293513,95.87423903)(1009.18292725,95.84424123)
\curveto(1009.30293488,95.82423908)(1009.42793476,95.7992391)(1009.55792725,95.76924123)
\curveto(1009.60793458,95.74923915)(1009.66293452,95.73923916)(1009.72292725,95.73924123)
\curveto(1009.77293441,95.73923916)(1009.82293436,95.73423917)(1009.87292725,95.72424123)
\curveto(1009.96293422,95.71423919)(1010.05793413,95.7042392)(1010.15792725,95.69424123)
\curveto(1010.24793394,95.68423922)(1010.34293384,95.67423923)(1010.44292725,95.66424123)
\curveto(1010.52293366,95.66423924)(1010.60793358,95.65923924)(1010.69792725,95.64924123)
\lineto(1010.93792725,95.64924123)
\lineto(1011.11792725,95.64924123)
\curveto(1011.14793304,95.63923926)(1011.182933,95.63423927)(1011.22292725,95.63424123)
\lineto(1011.35792725,95.63424123)
\lineto(1011.80792725,95.63424123)
\curveto(1011.8879323,95.63423927)(1011.97293221,95.62923927)(1012.06292725,95.61924123)
\curveto(1012.14293204,95.61923928)(1012.21793197,95.62923927)(1012.28792725,95.64924123)
\lineto(1012.55792725,95.64924123)
\curveto(1012.57793161,95.64923925)(1012.60793158,95.64423926)(1012.64792725,95.63424123)
\curveto(1012.67793151,95.63423927)(1012.70293148,95.63923926)(1012.72292725,95.64924123)
\curveto(1012.82293136,95.65923924)(1012.92293126,95.66423924)(1013.02292725,95.66424123)
\curveto(1013.11293107,95.67423923)(1013.21293097,95.68423922)(1013.32292725,95.69424123)
\curveto(1013.44293074,95.72423918)(1013.56793062,95.73923916)(1013.69792725,95.73924123)
\curveto(1013.81793037,95.74923915)(1013.93293025,95.77423913)(1014.04292725,95.81424123)
\curveto(1014.34292984,95.89423901)(1014.60792958,95.97923892)(1014.83792725,96.06924123)
\curveto(1015.06792912,96.16923873)(1015.2829289,96.31423859)(1015.48292725,96.50424123)
\curveto(1015.6829285,96.71423819)(1015.83292835,96.97923792)(1015.93292725,97.29924123)
\curveto(1015.95292823,97.33923756)(1015.96292822,97.37423753)(1015.96292725,97.40424123)
\curveto(1015.95292823,97.44423746)(1015.95792823,97.48923741)(1015.97792725,97.53924123)
\curveto(1015.9879282,97.57923732)(1015.99792819,97.64923725)(1016.00792725,97.74924123)
\curveto(1016.01792817,97.85923704)(1016.01292817,97.94423696)(1015.99292725,98.00424123)
\curveto(1015.97292821,98.07423683)(1015.96292822,98.14423676)(1015.96292725,98.21424123)
\curveto(1015.95292823,98.28423662)(1015.93792825,98.34923655)(1015.91792725,98.40924123)
\curveto(1015.85792833,98.60923629)(1015.77292841,98.78923611)(1015.66292725,98.94924123)
\curveto(1015.64292854,98.97923592)(1015.62292856,99.0042359)(1015.60292725,99.02424123)
\lineto(1015.54292725,99.08424123)
\curveto(1015.52292866,99.12423578)(1015.4829287,99.17423573)(1015.42292725,99.23424123)
\curveto(1015.2829289,99.33423557)(1015.15292903,99.41923548)(1015.03292725,99.48924123)
\curveto(1014.91292927,99.55923534)(1014.76792942,99.62923527)(1014.59792725,99.69924123)
\curveto(1014.52792966,99.72923517)(1014.45792973,99.74923515)(1014.38792725,99.75924123)
\curveto(1014.31792987,99.77923512)(1014.24292994,99.7992351)(1014.16292725,99.81924123)
}
}
{
\newrgbcolor{curcolor}{0 0 0}
\pscustom[linestyle=none,fillstyle=solid,fillcolor=curcolor]
{
\newpath
\moveto(1006.31792725,106.77385061)
\curveto(1006.31793787,106.87384575)(1006.32793786,106.96884566)(1006.34792725,107.05885061)
\curveto(1006.35793783,107.14884548)(1006.3879378,107.21384541)(1006.43792725,107.25385061)
\curveto(1006.51793767,107.31384531)(1006.62293756,107.34384528)(1006.75292725,107.34385061)
\lineto(1007.14292725,107.34385061)
\lineto(1008.64292725,107.34385061)
\lineto(1015.03292725,107.34385061)
\lineto(1016.20292725,107.34385061)
\lineto(1016.51792725,107.34385061)
\curveto(1016.61792757,107.35384527)(1016.69792749,107.33884529)(1016.75792725,107.29885061)
\curveto(1016.83792735,107.24884538)(1016.8879273,107.17384545)(1016.90792725,107.07385061)
\curveto(1016.91792727,106.98384564)(1016.92292726,106.87384575)(1016.92292725,106.74385061)
\lineto(1016.92292725,106.51885061)
\curveto(1016.90292728,106.43884619)(1016.8879273,106.36884626)(1016.87792725,106.30885061)
\curveto(1016.85792733,106.24884638)(1016.81792737,106.19884643)(1016.75792725,106.15885061)
\curveto(1016.69792749,106.11884651)(1016.62292756,106.09884653)(1016.53292725,106.09885061)
\lineto(1016.23292725,106.09885061)
\lineto(1015.13792725,106.09885061)
\lineto(1009.79792725,106.09885061)
\curveto(1009.70793448,106.07884655)(1009.63293455,106.06384656)(1009.57292725,106.05385061)
\curveto(1009.50293468,106.05384657)(1009.44293474,106.0238466)(1009.39292725,105.96385061)
\curveto(1009.34293484,105.89384673)(1009.31793487,105.80384682)(1009.31792725,105.69385061)
\curveto(1009.30793488,105.59384703)(1009.30293488,105.48384714)(1009.30292725,105.36385061)
\lineto(1009.30292725,104.22385061)
\lineto(1009.30292725,103.72885061)
\curveto(1009.29293489,103.56884906)(1009.23293495,103.45884917)(1009.12292725,103.39885061)
\curveto(1009.09293509,103.37884925)(1009.06293512,103.36884926)(1009.03292725,103.36885061)
\curveto(1008.99293519,103.36884926)(1008.94793524,103.36384926)(1008.89792725,103.35385061)
\curveto(1008.77793541,103.33384929)(1008.66793552,103.33884929)(1008.56792725,103.36885061)
\curveto(1008.46793572,103.40884922)(1008.39793579,103.46384916)(1008.35792725,103.53385061)
\curveto(1008.30793588,103.61384901)(1008.2829359,103.73384889)(1008.28292725,103.89385061)
\curveto(1008.2829359,104.05384857)(1008.26793592,104.18884844)(1008.23792725,104.29885061)
\curveto(1008.22793596,104.34884828)(1008.22293596,104.40384822)(1008.22292725,104.46385061)
\curveto(1008.21293597,104.5238481)(1008.19793599,104.58384804)(1008.17792725,104.64385061)
\curveto(1008.12793606,104.79384783)(1008.07793611,104.93884769)(1008.02792725,105.07885061)
\curveto(1007.96793622,105.21884741)(1007.89793629,105.35384727)(1007.81792725,105.48385061)
\curveto(1007.72793646,105.623847)(1007.62293656,105.74384688)(1007.50292725,105.84385061)
\curveto(1007.3829368,105.94384668)(1007.25293693,106.03884659)(1007.11292725,106.12885061)
\curveto(1007.01293717,106.18884644)(1006.90293728,106.23384639)(1006.78292725,106.26385061)
\curveto(1006.66293752,106.30384632)(1006.55793763,106.35384627)(1006.46792725,106.41385061)
\curveto(1006.40793778,106.46384616)(1006.36793782,106.53384609)(1006.34792725,106.62385061)
\curveto(1006.33793785,106.64384598)(1006.33293785,106.66884596)(1006.33292725,106.69885061)
\curveto(1006.33293785,106.7288459)(1006.32793786,106.75384587)(1006.31792725,106.77385061)
}
}
{
\newrgbcolor{curcolor}{0 0 0}
\pscustom[linestyle=none,fillstyle=solid,fillcolor=curcolor]
{
\newpath
\moveto(1006.31792725,115.12345998)
\curveto(1006.31793787,115.22345513)(1006.32793786,115.31845503)(1006.34792725,115.40845998)
\curveto(1006.35793783,115.49845485)(1006.3879378,115.56345479)(1006.43792725,115.60345998)
\curveto(1006.51793767,115.66345469)(1006.62293756,115.69345466)(1006.75292725,115.69345998)
\lineto(1007.14292725,115.69345998)
\lineto(1008.64292725,115.69345998)
\lineto(1015.03292725,115.69345998)
\lineto(1016.20292725,115.69345998)
\lineto(1016.51792725,115.69345998)
\curveto(1016.61792757,115.70345465)(1016.69792749,115.68845466)(1016.75792725,115.64845998)
\curveto(1016.83792735,115.59845475)(1016.8879273,115.52345483)(1016.90792725,115.42345998)
\curveto(1016.91792727,115.33345502)(1016.92292726,115.22345513)(1016.92292725,115.09345998)
\lineto(1016.92292725,114.86845998)
\curveto(1016.90292728,114.78845556)(1016.8879273,114.71845563)(1016.87792725,114.65845998)
\curveto(1016.85792733,114.59845575)(1016.81792737,114.5484558)(1016.75792725,114.50845998)
\curveto(1016.69792749,114.46845588)(1016.62292756,114.4484559)(1016.53292725,114.44845998)
\lineto(1016.23292725,114.44845998)
\lineto(1015.13792725,114.44845998)
\lineto(1009.79792725,114.44845998)
\curveto(1009.70793448,114.42845592)(1009.63293455,114.41345594)(1009.57292725,114.40345998)
\curveto(1009.50293468,114.40345595)(1009.44293474,114.37345598)(1009.39292725,114.31345998)
\curveto(1009.34293484,114.24345611)(1009.31793487,114.1534562)(1009.31792725,114.04345998)
\curveto(1009.30793488,113.94345641)(1009.30293488,113.83345652)(1009.30292725,113.71345998)
\lineto(1009.30292725,112.57345998)
\lineto(1009.30292725,112.07845998)
\curveto(1009.29293489,111.91845843)(1009.23293495,111.80845854)(1009.12292725,111.74845998)
\curveto(1009.09293509,111.72845862)(1009.06293512,111.71845863)(1009.03292725,111.71845998)
\curveto(1008.99293519,111.71845863)(1008.94793524,111.71345864)(1008.89792725,111.70345998)
\curveto(1008.77793541,111.68345867)(1008.66793552,111.68845866)(1008.56792725,111.71845998)
\curveto(1008.46793572,111.75845859)(1008.39793579,111.81345854)(1008.35792725,111.88345998)
\curveto(1008.30793588,111.96345839)(1008.2829359,112.08345827)(1008.28292725,112.24345998)
\curveto(1008.2829359,112.40345795)(1008.26793592,112.53845781)(1008.23792725,112.64845998)
\curveto(1008.22793596,112.69845765)(1008.22293596,112.7534576)(1008.22292725,112.81345998)
\curveto(1008.21293597,112.87345748)(1008.19793599,112.93345742)(1008.17792725,112.99345998)
\curveto(1008.12793606,113.14345721)(1008.07793611,113.28845706)(1008.02792725,113.42845998)
\curveto(1007.96793622,113.56845678)(1007.89793629,113.70345665)(1007.81792725,113.83345998)
\curveto(1007.72793646,113.97345638)(1007.62293656,114.09345626)(1007.50292725,114.19345998)
\curveto(1007.3829368,114.29345606)(1007.25293693,114.38845596)(1007.11292725,114.47845998)
\curveto(1007.01293717,114.53845581)(1006.90293728,114.58345577)(1006.78292725,114.61345998)
\curveto(1006.66293752,114.6534557)(1006.55793763,114.70345565)(1006.46792725,114.76345998)
\curveto(1006.40793778,114.81345554)(1006.36793782,114.88345547)(1006.34792725,114.97345998)
\curveto(1006.33793785,114.99345536)(1006.33293785,115.01845533)(1006.33292725,115.04845998)
\curveto(1006.33293785,115.07845527)(1006.32793786,115.10345525)(1006.31792725,115.12345998)
}
}
{
\newrgbcolor{curcolor}{0 0 0}
\pscustom[linestyle=none,fillstyle=solid,fillcolor=curcolor]
{
\newpath
\moveto(222.02405518,31.67142873)
\lineto(222.02405518,32.58642873)
\curveto(222.02406587,32.68642608)(222.02406587,32.78142599)(222.02405518,32.87142873)
\curveto(222.02406587,32.96142581)(222.04406585,33.03642573)(222.08405518,33.09642873)
\curveto(222.14406575,33.18642558)(222.22406567,33.24642552)(222.32405518,33.27642873)
\curveto(222.42406547,33.31642545)(222.52906537,33.36142541)(222.63905518,33.41142873)
\curveto(222.82906507,33.49142528)(223.01906488,33.56142521)(223.20905518,33.62142873)
\curveto(223.3990645,33.69142508)(223.58906431,33.766425)(223.77905518,33.84642873)
\curveto(223.95906394,33.91642485)(224.14406375,33.98142479)(224.33405518,34.04142873)
\curveto(224.51406338,34.10142467)(224.6940632,34.1714246)(224.87405518,34.25142873)
\curveto(225.01406288,34.31142446)(225.15906274,34.3664244)(225.30905518,34.41642873)
\curveto(225.45906244,34.4664243)(225.60406229,34.52142425)(225.74405518,34.58142873)
\curveto(226.1940617,34.76142401)(226.64906125,34.93142384)(227.10905518,35.09142873)
\curveto(227.55906034,35.25142352)(228.00905989,35.42142335)(228.45905518,35.60142873)
\curveto(228.50905939,35.62142315)(228.55905934,35.63642313)(228.60905518,35.64642873)
\lineto(228.75905518,35.70642873)
\curveto(228.97905892,35.79642297)(229.20405869,35.88142289)(229.43405518,35.96142873)
\curveto(229.65405824,36.04142273)(229.87405802,36.12642264)(230.09405518,36.21642873)
\curveto(230.18405771,36.25642251)(230.2940576,36.29642247)(230.42405518,36.33642873)
\curveto(230.54405735,36.37642239)(230.61405728,36.44142233)(230.63405518,36.53142873)
\curveto(230.64405725,36.5714222)(230.64405725,36.60142217)(230.63405518,36.62142873)
\lineto(230.57405518,36.68142873)
\curveto(230.52405737,36.73142204)(230.46905743,36.766422)(230.40905518,36.78642873)
\curveto(230.34905755,36.81642195)(230.28405761,36.84642192)(230.21405518,36.87642873)
\lineto(229.58405518,37.11642873)
\curveto(229.36405853,37.19642157)(229.14905875,37.27642149)(228.93905518,37.35642873)
\lineto(228.78905518,37.41642873)
\lineto(228.60905518,37.47642873)
\curveto(228.41905948,37.55642121)(228.22905967,37.62642114)(228.03905518,37.68642873)
\curveto(227.83906006,37.75642101)(227.63906026,37.83142094)(227.43905518,37.91142873)
\curveto(226.85906104,38.15142062)(226.27406162,38.3714204)(225.68405518,38.57142873)
\curveto(225.0940628,38.78141999)(224.50906339,39.00641976)(223.92905518,39.24642873)
\curveto(223.72906417,39.32641944)(223.52406437,39.40141937)(223.31405518,39.47142873)
\curveto(223.10406479,39.55141922)(222.899065,39.63141914)(222.69905518,39.71142873)
\curveto(222.61906528,39.75141902)(222.51906538,39.78641898)(222.39905518,39.81642873)
\curveto(222.27906562,39.85641891)(222.1940657,39.91141886)(222.14405518,39.98142873)
\curveto(222.10406579,40.04141873)(222.07406582,40.11641865)(222.05405518,40.20642873)
\curveto(222.03406586,40.30641846)(222.02406587,40.41641835)(222.02405518,40.53642873)
\curveto(222.01406588,40.65641811)(222.01406588,40.77641799)(222.02405518,40.89642873)
\curveto(222.02406587,41.01641775)(222.02406587,41.12641764)(222.02405518,41.22642873)
\curveto(222.02406587,41.31641745)(222.02406587,41.40641736)(222.02405518,41.49642873)
\curveto(222.02406587,41.59641717)(222.04406585,41.6714171)(222.08405518,41.72142873)
\curveto(222.13406576,41.81141696)(222.22406567,41.86141691)(222.35405518,41.87142873)
\curveto(222.48406541,41.88141689)(222.62406527,41.88641688)(222.77405518,41.88642873)
\lineto(224.42405518,41.88642873)
\lineto(230.69405518,41.88642873)
\lineto(231.95405518,41.88642873)
\curveto(232.06405583,41.88641688)(232.17405572,41.88641688)(232.28405518,41.88642873)
\curveto(232.3940555,41.89641687)(232.47905542,41.87641689)(232.53905518,41.82642873)
\curveto(232.5990553,41.79641697)(232.63905526,41.75141702)(232.65905518,41.69142873)
\curveto(232.66905523,41.63141714)(232.68405521,41.56141721)(232.70405518,41.48142873)
\lineto(232.70405518,41.24142873)
\lineto(232.70405518,40.88142873)
\curveto(232.6940552,40.771418)(232.64905525,40.69141808)(232.56905518,40.64142873)
\curveto(232.53905536,40.62141815)(232.50905539,40.60641816)(232.47905518,40.59642873)
\curveto(232.43905546,40.59641817)(232.3940555,40.58641818)(232.34405518,40.56642873)
\lineto(232.17905518,40.56642873)
\curveto(232.11905578,40.55641821)(232.04905585,40.55141822)(231.96905518,40.55142873)
\curveto(231.88905601,40.56141821)(231.81405608,40.5664182)(231.74405518,40.56642873)
\lineto(230.90405518,40.56642873)
\lineto(226.47905518,40.56642873)
\curveto(226.22906167,40.5664182)(225.97906192,40.5664182)(225.72905518,40.56642873)
\curveto(225.46906243,40.5664182)(225.21906268,40.56141821)(224.97905518,40.55142873)
\curveto(224.87906302,40.55141822)(224.76906313,40.54641822)(224.64905518,40.53642873)
\curveto(224.52906337,40.52641824)(224.46906343,40.4714183)(224.46905518,40.37142873)
\lineto(224.48405518,40.37142873)
\curveto(224.50406339,40.30141847)(224.56906333,40.24141853)(224.67905518,40.19142873)
\curveto(224.78906311,40.15141862)(224.88406301,40.11641865)(224.96405518,40.08642873)
\curveto(225.13406276,40.01641875)(225.30906259,39.95141882)(225.48905518,39.89142873)
\curveto(225.65906224,39.83141894)(225.82906207,39.76141901)(225.99905518,39.68142873)
\curveto(226.04906185,39.66141911)(226.0940618,39.64641912)(226.13405518,39.63642873)
\curveto(226.17406172,39.62641914)(226.21906168,39.61141916)(226.26905518,39.59142873)
\curveto(226.44906145,39.51141926)(226.63406126,39.44141933)(226.82405518,39.38142873)
\curveto(227.00406089,39.33141944)(227.18406071,39.2664195)(227.36405518,39.18642873)
\curveto(227.51406038,39.11641965)(227.66906023,39.05641971)(227.82905518,39.00642873)
\curveto(227.97905992,38.95641981)(228.12905977,38.90141987)(228.27905518,38.84142873)
\curveto(228.74905915,38.64142013)(229.22405867,38.46142031)(229.70405518,38.30142873)
\curveto(230.17405772,38.14142063)(230.63905726,37.9664208)(231.09905518,37.77642873)
\curveto(231.27905662,37.69642107)(231.45905644,37.62642114)(231.63905518,37.56642873)
\curveto(231.81905608,37.50642126)(231.9990559,37.44142133)(232.17905518,37.37142873)
\curveto(232.28905561,37.32142145)(232.3940555,37.2714215)(232.49405518,37.22142873)
\curveto(232.58405531,37.18142159)(232.64905525,37.09642167)(232.68905518,36.96642873)
\curveto(232.6990552,36.94642182)(232.70405519,36.92142185)(232.70405518,36.89142873)
\curveto(232.6940552,36.8714219)(232.6940552,36.84642192)(232.70405518,36.81642873)
\curveto(232.71405518,36.78642198)(232.71905518,36.75142202)(232.71905518,36.71142873)
\curveto(232.70905519,36.6714221)(232.70405519,36.63142214)(232.70405518,36.59142873)
\lineto(232.70405518,36.29142873)
\curveto(232.70405519,36.19142258)(232.67905522,36.11142266)(232.62905518,36.05142873)
\curveto(232.57905532,35.9714228)(232.50905539,35.91142286)(232.41905518,35.87142873)
\curveto(232.31905558,35.84142293)(232.21905568,35.80142297)(232.11905518,35.75142873)
\curveto(231.91905598,35.6714231)(231.71405618,35.59142318)(231.50405518,35.51142873)
\curveto(231.28405661,35.44142333)(231.07405682,35.3664234)(230.87405518,35.28642873)
\curveto(230.6940572,35.20642356)(230.51405738,35.13642363)(230.33405518,35.07642873)
\curveto(230.14405775,35.02642374)(229.95905794,34.96142381)(229.77905518,34.88142873)
\curveto(229.21905868,34.65142412)(228.65405924,34.43642433)(228.08405518,34.23642873)
\curveto(227.51406038,34.03642473)(226.94906095,33.82142495)(226.38905518,33.59142873)
\lineto(225.75905518,33.35142873)
\curveto(225.53906236,33.28142549)(225.32906257,33.20642556)(225.12905518,33.12642873)
\curveto(225.01906288,33.07642569)(224.91406298,33.03142574)(224.81405518,32.99142873)
\curveto(224.70406319,32.96142581)(224.60906329,32.91142586)(224.52905518,32.84142873)
\curveto(224.50906339,32.83142594)(224.4990634,32.82142595)(224.49905518,32.81142873)
\lineto(224.46905518,32.78142873)
\lineto(224.46905518,32.70642873)
\lineto(224.49905518,32.67642873)
\curveto(224.4990634,32.6664261)(224.50406339,32.65642611)(224.51405518,32.64642873)
\curveto(224.56406333,32.62642614)(224.61906328,32.61642615)(224.67905518,32.61642873)
\curveto(224.73906316,32.61642615)(224.7990631,32.60642616)(224.85905518,32.58642873)
\lineto(225.02405518,32.58642873)
\curveto(225.08406281,32.5664262)(225.14906275,32.56142621)(225.21905518,32.57142873)
\curveto(225.28906261,32.58142619)(225.35906254,32.58642618)(225.42905518,32.58642873)
\lineto(226.23905518,32.58642873)
\lineto(230.79905518,32.58642873)
\lineto(231.98405518,32.58642873)
\curveto(232.0940558,32.58642618)(232.20405569,32.58142619)(232.31405518,32.57142873)
\curveto(232.42405547,32.5714262)(232.50905539,32.54642622)(232.56905518,32.49642873)
\curveto(232.64905525,32.44642632)(232.6940552,32.35642641)(232.70405518,32.22642873)
\lineto(232.70405518,31.83642873)
\lineto(232.70405518,31.64142873)
\curveto(232.70405519,31.59142718)(232.6940552,31.54142723)(232.67405518,31.49142873)
\curveto(232.63405526,31.36142741)(232.54905535,31.28642748)(232.41905518,31.26642873)
\curveto(232.28905561,31.25642751)(232.13905576,31.25142752)(231.96905518,31.25142873)
\lineto(230.22905518,31.25142873)
\lineto(224.22905518,31.25142873)
\lineto(222.81905518,31.25142873)
\curveto(222.70906519,31.25142752)(222.5940653,31.24642752)(222.47405518,31.23642873)
\curveto(222.35406554,31.23642753)(222.25906564,31.26142751)(222.18905518,31.31142873)
\curveto(222.12906577,31.35142742)(222.07906582,31.42642734)(222.03905518,31.53642873)
\curveto(222.02906587,31.55642721)(222.02906587,31.57642719)(222.03905518,31.59642873)
\curveto(222.03906586,31.62642714)(222.03406586,31.65142712)(222.02405518,31.67142873)
}
}
{
\newrgbcolor{curcolor}{0 0 0}
\pscustom[linestyle=none,fillstyle=solid,fillcolor=curcolor]
{
\newpath
\moveto(232.14905518,50.87353811)
\curveto(232.30905559,50.90353028)(232.44405545,50.88853029)(232.55405518,50.82853811)
\curveto(232.65405524,50.76853041)(232.72905517,50.68853049)(232.77905518,50.58853811)
\curveto(232.7990551,50.53853064)(232.80905509,50.4835307)(232.80905518,50.42353811)
\curveto(232.80905509,50.37353081)(232.81905508,50.31853086)(232.83905518,50.25853811)
\curveto(232.88905501,50.03853114)(232.87405502,49.81853136)(232.79405518,49.59853811)
\curveto(232.72405517,49.38853179)(232.63405526,49.24353194)(232.52405518,49.16353811)
\curveto(232.45405544,49.11353207)(232.37405552,49.06853211)(232.28405518,49.02853811)
\curveto(232.18405571,48.98853219)(232.10405579,48.93853224)(232.04405518,48.87853811)
\curveto(232.02405587,48.85853232)(232.00405589,48.83353235)(231.98405518,48.80353811)
\curveto(231.96405593,48.7835324)(231.95905594,48.75353243)(231.96905518,48.71353811)
\curveto(231.9990559,48.60353258)(232.05405584,48.49853268)(232.13405518,48.39853811)
\curveto(232.21405568,48.30853287)(232.28405561,48.21853296)(232.34405518,48.12853811)
\curveto(232.42405547,47.99853318)(232.4990554,47.85853332)(232.56905518,47.70853811)
\curveto(232.62905527,47.55853362)(232.68405521,47.39853378)(232.73405518,47.22853811)
\curveto(232.76405513,47.12853405)(232.78405511,47.01853416)(232.79405518,46.89853811)
\curveto(232.80405509,46.78853439)(232.81905508,46.6785345)(232.83905518,46.56853811)
\curveto(232.84905505,46.51853466)(232.85405504,46.47353471)(232.85405518,46.43353811)
\lineto(232.85405518,46.32853811)
\curveto(232.87405502,46.21853496)(232.87405502,46.11353507)(232.85405518,46.01353811)
\lineto(232.85405518,45.87853811)
\curveto(232.84405505,45.82853535)(232.83905506,45.7785354)(232.83905518,45.72853811)
\curveto(232.83905506,45.6785355)(232.82905507,45.63353555)(232.80905518,45.59353811)
\curveto(232.7990551,45.55353563)(232.7940551,45.51853566)(232.79405518,45.48853811)
\curveto(232.80405509,45.46853571)(232.80405509,45.44353574)(232.79405518,45.41353811)
\lineto(232.73405518,45.17353811)
\curveto(232.72405517,45.09353609)(232.70405519,45.01853616)(232.67405518,44.94853811)
\curveto(232.54405535,44.64853653)(232.3990555,44.40353678)(232.23905518,44.21353811)
\curveto(232.06905583,44.03353715)(231.83405606,43.8835373)(231.53405518,43.76353811)
\curveto(231.31405658,43.67353751)(231.04905685,43.62853755)(230.73905518,43.62853811)
\lineto(230.42405518,43.62853811)
\curveto(230.37405752,43.63853754)(230.32405757,43.64353754)(230.27405518,43.64353811)
\lineto(230.09405518,43.67353811)
\lineto(229.76405518,43.79353811)
\curveto(229.65405824,43.83353735)(229.55405834,43.8835373)(229.46405518,43.94353811)
\curveto(229.17405872,44.12353706)(228.95905894,44.36853681)(228.81905518,44.67853811)
\curveto(228.67905922,44.98853619)(228.55405934,45.32853585)(228.44405518,45.69853811)
\curveto(228.40405949,45.83853534)(228.37405952,45.9835352)(228.35405518,46.13353811)
\curveto(228.33405956,46.2835349)(228.30905959,46.43353475)(228.27905518,46.58353811)
\curveto(228.25905964,46.65353453)(228.24905965,46.71853446)(228.24905518,46.77853811)
\curveto(228.24905965,46.84853433)(228.23905966,46.92353426)(228.21905518,47.00353811)
\curveto(228.1990597,47.07353411)(228.18905971,47.14353404)(228.18905518,47.21353811)
\curveto(228.17905972,47.2835339)(228.16405973,47.35853382)(228.14405518,47.43853811)
\curveto(228.08405981,47.68853349)(228.03405986,47.92353326)(227.99405518,48.14353811)
\curveto(227.94405995,48.36353282)(227.82906007,48.53853264)(227.64905518,48.66853811)
\curveto(227.56906033,48.72853245)(227.46906043,48.7785324)(227.34905518,48.81853811)
\curveto(227.21906068,48.85853232)(227.07906082,48.85853232)(226.92905518,48.81853811)
\curveto(226.68906121,48.75853242)(226.4990614,48.66853251)(226.35905518,48.54853811)
\curveto(226.21906168,48.43853274)(226.10906179,48.2785329)(226.02905518,48.06853811)
\curveto(225.97906192,47.94853323)(225.94406195,47.80353338)(225.92405518,47.63353811)
\curveto(225.90406199,47.47353371)(225.894062,47.30353388)(225.89405518,47.12353811)
\curveto(225.894062,46.94353424)(225.90406199,46.76853441)(225.92405518,46.59853811)
\curveto(225.94406195,46.42853475)(225.97406192,46.2835349)(226.01405518,46.16353811)
\curveto(226.07406182,45.99353519)(226.15906174,45.82853535)(226.26905518,45.66853811)
\curveto(226.32906157,45.58853559)(226.40906149,45.51353567)(226.50905518,45.44353811)
\curveto(226.5990613,45.3835358)(226.6990612,45.32853585)(226.80905518,45.27853811)
\curveto(226.88906101,45.24853593)(226.97406092,45.21853596)(227.06405518,45.18853811)
\curveto(227.15406074,45.16853601)(227.22406067,45.12353606)(227.27405518,45.05353811)
\curveto(227.30406059,45.01353617)(227.32906057,44.94353624)(227.34905518,44.84353811)
\curveto(227.35906054,44.75353643)(227.36406053,44.65853652)(227.36405518,44.55853811)
\curveto(227.36406053,44.45853672)(227.35906054,44.35853682)(227.34905518,44.25853811)
\curveto(227.32906057,44.16853701)(227.30406059,44.10353708)(227.27405518,44.06353811)
\curveto(227.24406065,44.02353716)(227.1940607,43.99353719)(227.12405518,43.97353811)
\curveto(227.05406084,43.95353723)(226.97906092,43.95353723)(226.89905518,43.97353811)
\curveto(226.76906113,44.00353718)(226.64906125,44.03353715)(226.53905518,44.06353811)
\curveto(226.41906148,44.10353708)(226.30406159,44.14853703)(226.19405518,44.19853811)
\curveto(225.84406205,44.38853679)(225.57406232,44.62853655)(225.38405518,44.91853811)
\curveto(225.18406271,45.20853597)(225.02406287,45.56853561)(224.90405518,45.99853811)
\curveto(224.88406301,46.09853508)(224.86906303,46.19853498)(224.85905518,46.29853811)
\curveto(224.84906305,46.40853477)(224.83406306,46.51853466)(224.81405518,46.62853811)
\curveto(224.80406309,46.66853451)(224.80406309,46.73353445)(224.81405518,46.82353811)
\curveto(224.81406308,46.91353427)(224.80406309,46.96853421)(224.78405518,46.98853811)
\curveto(224.77406312,47.68853349)(224.85406304,48.29853288)(225.02405518,48.81853811)
\curveto(225.1940627,49.33853184)(225.51906238,49.70353148)(225.99905518,49.91353811)
\curveto(226.1990617,50.00353118)(226.43406146,50.05353113)(226.70405518,50.06353811)
\curveto(226.96406093,50.0835311)(227.23906066,50.09353109)(227.52905518,50.09353811)
\lineto(230.84405518,50.09353811)
\curveto(230.98405691,50.09353109)(231.11905678,50.09853108)(231.24905518,50.10853811)
\curveto(231.37905652,50.11853106)(231.48405641,50.14853103)(231.56405518,50.19853811)
\curveto(231.63405626,50.24853093)(231.68405621,50.31353087)(231.71405518,50.39353811)
\curveto(231.75405614,50.4835307)(231.78405611,50.56853061)(231.80405518,50.64853811)
\curveto(231.81405608,50.72853045)(231.85905604,50.78853039)(231.93905518,50.82853811)
\curveto(231.96905593,50.84853033)(231.9990559,50.85853032)(232.02905518,50.85853811)
\curveto(232.05905584,50.85853032)(232.0990558,50.86353032)(232.14905518,50.87353811)
\moveto(230.48405518,48.72853811)
\curveto(230.34405755,48.78853239)(230.18405771,48.81853236)(230.00405518,48.81853811)
\curveto(229.81405808,48.82853235)(229.61905828,48.83353235)(229.41905518,48.83353811)
\curveto(229.30905859,48.83353235)(229.20905869,48.82853235)(229.11905518,48.81853811)
\curveto(229.02905887,48.80853237)(228.95905894,48.76853241)(228.90905518,48.69853811)
\curveto(228.88905901,48.66853251)(228.87905902,48.59853258)(228.87905518,48.48853811)
\curveto(228.899059,48.46853271)(228.90905899,48.43353275)(228.90905518,48.38353811)
\curveto(228.90905899,48.33353285)(228.91905898,48.28853289)(228.93905518,48.24853811)
\curveto(228.95905894,48.16853301)(228.97905892,48.0785331)(228.99905518,47.97853811)
\lineto(229.05905518,47.67853811)
\curveto(229.05905884,47.64853353)(229.06405883,47.61353357)(229.07405518,47.57353811)
\lineto(229.07405518,47.46853811)
\curveto(229.11405878,47.31853386)(229.13905876,47.15353403)(229.14905518,46.97353811)
\curveto(229.14905875,46.80353438)(229.16905873,46.64353454)(229.20905518,46.49353811)
\curveto(229.22905867,46.41353477)(229.24905865,46.33853484)(229.26905518,46.26853811)
\curveto(229.27905862,46.20853497)(229.2940586,46.13853504)(229.31405518,46.05853811)
\curveto(229.36405853,45.89853528)(229.42905847,45.74853543)(229.50905518,45.60853811)
\curveto(229.57905832,45.46853571)(229.66905823,45.34853583)(229.77905518,45.24853811)
\curveto(229.88905801,45.14853603)(230.02405787,45.07353611)(230.18405518,45.02353811)
\curveto(230.33405756,44.97353621)(230.51905738,44.95353623)(230.73905518,44.96353811)
\curveto(230.83905706,44.96353622)(230.93405696,44.9785362)(231.02405518,45.00853811)
\curveto(231.10405679,45.04853613)(231.17905672,45.09353609)(231.24905518,45.14353811)
\curveto(231.35905654,45.22353596)(231.45405644,45.32853585)(231.53405518,45.45853811)
\curveto(231.60405629,45.58853559)(231.66405623,45.72853545)(231.71405518,45.87853811)
\curveto(231.72405617,45.92853525)(231.72905617,45.9785352)(231.72905518,46.02853811)
\curveto(231.72905617,46.0785351)(231.73405616,46.12853505)(231.74405518,46.17853811)
\curveto(231.76405613,46.24853493)(231.77905612,46.33353485)(231.78905518,46.43353811)
\curveto(231.78905611,46.54353464)(231.77905612,46.63353455)(231.75905518,46.70353811)
\curveto(231.73905616,46.76353442)(231.73405616,46.82353436)(231.74405518,46.88353811)
\curveto(231.74405615,46.94353424)(231.73405616,47.00353418)(231.71405518,47.06353811)
\curveto(231.6940562,47.14353404)(231.67905622,47.21853396)(231.66905518,47.28853811)
\curveto(231.65905624,47.36853381)(231.63905626,47.44353374)(231.60905518,47.51353811)
\curveto(231.48905641,47.80353338)(231.34405655,48.04853313)(231.17405518,48.24853811)
\curveto(231.00405689,48.45853272)(230.77405712,48.61853256)(230.48405518,48.72853811)
}
}
{
\newrgbcolor{curcolor}{0 0 0}
\pscustom[linestyle=none,fillstyle=solid,fillcolor=curcolor]
{
\newpath
\moveto(224.79905518,55.69017873)
\curveto(224.7990631,55.92017394)(224.85906304,56.05017381)(224.97905518,56.08017873)
\curveto(225.08906281,56.11017375)(225.25406264,56.12517374)(225.47405518,56.12517873)
\lineto(225.75905518,56.12517873)
\curveto(225.84906205,56.12517374)(225.92406197,56.10017376)(225.98405518,56.05017873)
\curveto(226.06406183,55.99017387)(226.10906179,55.90517396)(226.11905518,55.79517873)
\curveto(226.11906178,55.68517418)(226.13406176,55.57517429)(226.16405518,55.46517873)
\curveto(226.1940617,55.32517454)(226.22406167,55.19017467)(226.25405518,55.06017873)
\curveto(226.28406161,54.94017492)(226.32406157,54.82517504)(226.37405518,54.71517873)
\curveto(226.50406139,54.42517544)(226.68406121,54.19017567)(226.91405518,54.01017873)
\curveto(227.13406076,53.83017603)(227.38906051,53.67517619)(227.67905518,53.54517873)
\curveto(227.78906011,53.50517636)(227.90405999,53.47517639)(228.02405518,53.45517873)
\curveto(228.13405976,53.43517643)(228.24905965,53.41017645)(228.36905518,53.38017873)
\curveto(228.41905948,53.37017649)(228.46905943,53.3651765)(228.51905518,53.36517873)
\curveto(228.56905933,53.37517649)(228.61905928,53.37517649)(228.66905518,53.36517873)
\curveto(228.78905911,53.33517653)(228.92905897,53.32017654)(229.08905518,53.32017873)
\curveto(229.23905866,53.33017653)(229.38405851,53.33517653)(229.52405518,53.33517873)
\lineto(231.36905518,53.33517873)
\lineto(231.71405518,53.33517873)
\curveto(231.83405606,53.33517653)(231.94905595,53.33017653)(232.05905518,53.32017873)
\curveto(232.16905573,53.31017655)(232.26405563,53.30517656)(232.34405518,53.30517873)
\curveto(232.42405547,53.31517655)(232.4940554,53.29517657)(232.55405518,53.24517873)
\curveto(232.62405527,53.19517667)(232.66405523,53.11517675)(232.67405518,53.00517873)
\curveto(232.68405521,52.90517696)(232.68905521,52.79517707)(232.68905518,52.67517873)
\lineto(232.68905518,52.40517873)
\curveto(232.66905523,52.35517751)(232.65405524,52.30517756)(232.64405518,52.25517873)
\curveto(232.62405527,52.21517765)(232.5990553,52.18517768)(232.56905518,52.16517873)
\curveto(232.4990554,52.11517775)(232.41405548,52.08517778)(232.31405518,52.07517873)
\lineto(231.98405518,52.07517873)
\lineto(230.82905518,52.07517873)
\lineto(226.67405518,52.07517873)
\lineto(225.63905518,52.07517873)
\lineto(225.33905518,52.07517873)
\curveto(225.23906266,52.08517778)(225.15406274,52.11517775)(225.08405518,52.16517873)
\curveto(225.04406285,52.19517767)(225.01406288,52.24517762)(224.99405518,52.31517873)
\curveto(224.97406292,52.39517747)(224.96406293,52.48017738)(224.96405518,52.57017873)
\curveto(224.95406294,52.6601772)(224.95406294,52.75017711)(224.96405518,52.84017873)
\curveto(224.97406292,52.93017693)(224.98906291,53.00017686)(225.00905518,53.05017873)
\curveto(225.03906286,53.13017673)(225.0990628,53.18017668)(225.18905518,53.20017873)
\curveto(225.26906263,53.23017663)(225.35906254,53.24517662)(225.45905518,53.24517873)
\lineto(225.75905518,53.24517873)
\curveto(225.85906204,53.24517662)(225.94906195,53.2651766)(226.02905518,53.30517873)
\curveto(226.04906185,53.31517655)(226.06406183,53.32517654)(226.07405518,53.33517873)
\lineto(226.11905518,53.38017873)
\curveto(226.11906178,53.49017637)(226.07406182,53.58017628)(225.98405518,53.65017873)
\curveto(225.88406201,53.72017614)(225.80406209,53.78017608)(225.74405518,53.83017873)
\lineto(225.65405518,53.92017873)
\curveto(225.54406235,54.01017585)(225.42906247,54.13517573)(225.30905518,54.29517873)
\curveto(225.18906271,54.45517541)(225.0990628,54.60517526)(225.03905518,54.74517873)
\curveto(224.98906291,54.83517503)(224.95406294,54.93017493)(224.93405518,55.03017873)
\curveto(224.90406299,55.13017473)(224.87406302,55.23517463)(224.84405518,55.34517873)
\curveto(224.83406306,55.40517446)(224.82906307,55.4651744)(224.82905518,55.52517873)
\curveto(224.81906308,55.58517428)(224.80906309,55.64017422)(224.79905518,55.69017873)
}
}
{
\newrgbcolor{curcolor}{0 0 0}
\pscustom[linestyle=none,fillstyle=solid,fillcolor=curcolor]
{
}
}
{
\newrgbcolor{curcolor}{0 0 0}
\pscustom[linestyle=none,fillstyle=solid,fillcolor=curcolor]
{
\newpath
\moveto(222.09905518,64.24510061)
\curveto(222.08906581,64.93509597)(222.20906569,65.53509537)(222.45905518,66.04510061)
\curveto(222.70906519,66.56509434)(223.04406485,66.96009395)(223.46405518,67.23010061)
\curveto(223.54406435,67.28009363)(223.63406426,67.32509358)(223.73405518,67.36510061)
\curveto(223.82406407,67.4050935)(223.91906398,67.45009346)(224.01905518,67.50010061)
\curveto(224.11906378,67.54009337)(224.21906368,67.57009334)(224.31905518,67.59010061)
\curveto(224.41906348,67.6100933)(224.52406337,67.63009328)(224.63405518,67.65010061)
\curveto(224.68406321,67.67009324)(224.72906317,67.67509323)(224.76905518,67.66510061)
\curveto(224.80906309,67.65509325)(224.85406304,67.66009325)(224.90405518,67.68010061)
\curveto(224.95406294,67.69009322)(225.03906286,67.69509321)(225.15905518,67.69510061)
\curveto(225.26906263,67.69509321)(225.35406254,67.69009322)(225.41405518,67.68010061)
\curveto(225.47406242,67.66009325)(225.53406236,67.65009326)(225.59405518,67.65010061)
\curveto(225.65406224,67.66009325)(225.71406218,67.65509325)(225.77405518,67.63510061)
\curveto(225.91406198,67.59509331)(226.04906185,67.56009335)(226.17905518,67.53010061)
\curveto(226.30906159,67.50009341)(226.43406146,67.46009345)(226.55405518,67.41010061)
\curveto(226.6940612,67.35009356)(226.81906108,67.28009363)(226.92905518,67.20010061)
\curveto(227.03906086,67.13009378)(227.14906075,67.05509385)(227.25905518,66.97510061)
\lineto(227.31905518,66.91510061)
\curveto(227.33906056,66.905094)(227.35906054,66.89009402)(227.37905518,66.87010061)
\curveto(227.53906036,66.75009416)(227.68406021,66.61509429)(227.81405518,66.46510061)
\curveto(227.94405995,66.31509459)(228.06905983,66.15509475)(228.18905518,65.98510061)
\curveto(228.40905949,65.67509523)(228.61405928,65.38009553)(228.80405518,65.10010061)
\curveto(228.94405895,64.87009604)(229.07905882,64.64009627)(229.20905518,64.41010061)
\curveto(229.33905856,64.19009672)(229.47405842,63.97009694)(229.61405518,63.75010061)
\curveto(229.78405811,63.50009741)(229.96405793,63.26009765)(230.15405518,63.03010061)
\curveto(230.34405755,62.8100981)(230.56905733,62.62009829)(230.82905518,62.46010061)
\curveto(230.88905701,62.42009849)(230.94905695,62.38509852)(231.00905518,62.35510061)
\curveto(231.05905684,62.32509858)(231.12405677,62.29509861)(231.20405518,62.26510061)
\curveto(231.27405662,62.24509866)(231.33405656,62.24009867)(231.38405518,62.25010061)
\curveto(231.45405644,62.27009864)(231.50905639,62.3050986)(231.54905518,62.35510061)
\curveto(231.57905632,62.4050985)(231.5990563,62.46509844)(231.60905518,62.53510061)
\lineto(231.60905518,62.77510061)
\lineto(231.60905518,63.52510061)
\lineto(231.60905518,66.33010061)
\lineto(231.60905518,66.99010061)
\curveto(231.60905629,67.08009383)(231.61405628,67.16509374)(231.62405518,67.24510061)
\curveto(231.62405627,67.32509358)(231.64405625,67.39009352)(231.68405518,67.44010061)
\curveto(231.72405617,67.49009342)(231.7990561,67.53009338)(231.90905518,67.56010061)
\curveto(232.00905589,67.60009331)(232.10905579,67.6100933)(232.20905518,67.59010061)
\lineto(232.34405518,67.59010061)
\curveto(232.41405548,67.57009334)(232.47405542,67.55009336)(232.52405518,67.53010061)
\curveto(232.57405532,67.5100934)(232.61405528,67.47509343)(232.64405518,67.42510061)
\curveto(232.68405521,67.37509353)(232.70405519,67.3050936)(232.70405518,67.21510061)
\lineto(232.70405518,66.94510061)
\lineto(232.70405518,66.04510061)
\lineto(232.70405518,62.53510061)
\lineto(232.70405518,61.47010061)
\curveto(232.70405519,61.39009952)(232.70905519,61.30009961)(232.71905518,61.20010061)
\curveto(232.71905518,61.10009981)(232.70905519,61.01509989)(232.68905518,60.94510061)
\curveto(232.61905528,60.73510017)(232.43905546,60.67010024)(232.14905518,60.75010061)
\curveto(232.10905579,60.76010015)(232.07405582,60.76010015)(232.04405518,60.75010061)
\curveto(232.00405589,60.75010016)(231.95905594,60.76010015)(231.90905518,60.78010061)
\curveto(231.82905607,60.80010011)(231.74405615,60.82010009)(231.65405518,60.84010061)
\curveto(231.56405633,60.86010005)(231.47905642,60.88510002)(231.39905518,60.91510061)
\curveto(230.90905699,61.07509983)(230.4940574,61.27509963)(230.15405518,61.51510061)
\curveto(229.90405799,61.69509921)(229.67905822,61.90009901)(229.47905518,62.13010061)
\curveto(229.26905863,62.36009855)(229.07405882,62.60009831)(228.89405518,62.85010061)
\curveto(228.71405918,63.1100978)(228.54405935,63.37509753)(228.38405518,63.64510061)
\curveto(228.21405968,63.92509698)(228.03905986,64.19509671)(227.85905518,64.45510061)
\curveto(227.77906012,64.56509634)(227.70406019,64.67009624)(227.63405518,64.77010061)
\curveto(227.56406033,64.88009603)(227.48906041,64.99009592)(227.40905518,65.10010061)
\curveto(227.37906052,65.14009577)(227.34906055,65.17509573)(227.31905518,65.20510061)
\curveto(227.27906062,65.24509566)(227.24906065,65.28509562)(227.22905518,65.32510061)
\curveto(227.11906078,65.46509544)(226.9940609,65.59009532)(226.85405518,65.70010061)
\curveto(226.82406107,65.72009519)(226.7990611,65.74509516)(226.77905518,65.77510061)
\curveto(226.74906115,65.8050951)(226.71906118,65.83009508)(226.68905518,65.85010061)
\curveto(226.58906131,65.93009498)(226.48906141,65.99509491)(226.38905518,66.04510061)
\curveto(226.28906161,66.1050948)(226.17906172,66.16009475)(226.05905518,66.21010061)
\curveto(225.98906191,66.24009467)(225.91406198,66.26009465)(225.83405518,66.27010061)
\lineto(225.59405518,66.33010061)
\lineto(225.50405518,66.33010061)
\curveto(225.47406242,66.34009457)(225.44406245,66.34509456)(225.41405518,66.34510061)
\curveto(225.34406255,66.36509454)(225.24906265,66.37009454)(225.12905518,66.36010061)
\curveto(224.9990629,66.36009455)(224.899063,66.35009456)(224.82905518,66.33010061)
\curveto(224.74906315,66.3100946)(224.67406322,66.29009462)(224.60405518,66.27010061)
\curveto(224.52406337,66.26009465)(224.44406345,66.24009467)(224.36405518,66.21010061)
\curveto(224.12406377,66.10009481)(223.92406397,65.95009496)(223.76405518,65.76010061)
\curveto(223.5940643,65.58009533)(223.45406444,65.36009555)(223.34405518,65.10010061)
\curveto(223.32406457,65.03009588)(223.30906459,64.96009595)(223.29905518,64.89010061)
\curveto(223.27906462,64.82009609)(223.25906464,64.74509616)(223.23905518,64.66510061)
\curveto(223.21906468,64.58509632)(223.20906469,64.47509643)(223.20905518,64.33510061)
\curveto(223.20906469,64.2050967)(223.21906468,64.10009681)(223.23905518,64.02010061)
\curveto(223.24906465,63.96009695)(223.25406464,63.905097)(223.25405518,63.85510061)
\curveto(223.25406464,63.8050971)(223.26406463,63.75509715)(223.28405518,63.70510061)
\curveto(223.32406457,63.6050973)(223.36406453,63.5100974)(223.40405518,63.42010061)
\curveto(223.44406445,63.34009757)(223.48906441,63.26009765)(223.53905518,63.18010061)
\curveto(223.55906434,63.15009776)(223.58406431,63.12009779)(223.61405518,63.09010061)
\curveto(223.64406425,63.07009784)(223.66906423,63.04509786)(223.68905518,63.01510061)
\lineto(223.76405518,62.94010061)
\curveto(223.78406411,62.910098)(223.80406409,62.88509802)(223.82405518,62.86510061)
\lineto(224.03405518,62.71510061)
\curveto(224.0940638,62.67509823)(224.15906374,62.63009828)(224.22905518,62.58010061)
\curveto(224.31906358,62.52009839)(224.42406347,62.47009844)(224.54405518,62.43010061)
\curveto(224.65406324,62.40009851)(224.76406313,62.36509854)(224.87405518,62.32510061)
\curveto(224.98406291,62.28509862)(225.12906277,62.26009865)(225.30905518,62.25010061)
\curveto(225.47906242,62.24009867)(225.60406229,62.2100987)(225.68405518,62.16010061)
\curveto(225.76406213,62.1100988)(225.80906209,62.03509887)(225.81905518,61.93510061)
\curveto(225.82906207,61.83509907)(225.83406206,61.72509918)(225.83405518,61.60510061)
\curveto(225.83406206,61.56509934)(225.83906206,61.52509938)(225.84905518,61.48510061)
\curveto(225.84906205,61.44509946)(225.84406205,61.4100995)(225.83405518,61.38010061)
\curveto(225.81406208,61.33009958)(225.80406209,61.28009963)(225.80405518,61.23010061)
\curveto(225.80406209,61.19009972)(225.7940621,61.15009976)(225.77405518,61.11010061)
\curveto(225.71406218,61.02009989)(225.57906232,60.97509993)(225.36905518,60.97510061)
\lineto(225.24905518,60.97510061)
\curveto(225.18906271,60.98509992)(225.12906277,60.99009992)(225.06905518,60.99010061)
\curveto(224.9990629,61.00009991)(224.93406296,61.0100999)(224.87405518,61.02010061)
\curveto(224.76406313,61.04009987)(224.66406323,61.06009985)(224.57405518,61.08010061)
\curveto(224.47406342,61.10009981)(224.37906352,61.13009978)(224.28905518,61.17010061)
\curveto(224.21906368,61.19009972)(224.15906374,61.2100997)(224.10905518,61.23010061)
\lineto(223.92905518,61.29010061)
\curveto(223.66906423,61.4100995)(223.42406447,61.56509934)(223.19405518,61.75510061)
\curveto(222.96406493,61.95509895)(222.77906512,62.17009874)(222.63905518,62.40010061)
\curveto(222.55906534,62.5100984)(222.4940654,62.62509828)(222.44405518,62.74510061)
\lineto(222.29405518,63.13510061)
\curveto(222.24406565,63.24509766)(222.21406568,63.36009755)(222.20405518,63.48010061)
\curveto(222.18406571,63.60009731)(222.15906574,63.72509718)(222.12905518,63.85510061)
\curveto(222.12906577,63.92509698)(222.12906577,63.99009692)(222.12905518,64.05010061)
\curveto(222.11906578,64.1100968)(222.10906579,64.17509673)(222.09905518,64.24510061)
}
}
{
\newrgbcolor{curcolor}{0 0 0}
\pscustom[linestyle=none,fillstyle=solid,fillcolor=curcolor]
{
\newpath
\moveto(222.09905518,72.59470998)
\curveto(222.08906581,73.28470535)(222.20906569,73.88470475)(222.45905518,74.39470998)
\curveto(222.70906519,74.91470372)(223.04406485,75.30970332)(223.46405518,75.57970998)
\curveto(223.54406435,75.629703)(223.63406426,75.67470296)(223.73405518,75.71470998)
\curveto(223.82406407,75.75470288)(223.91906398,75.79970283)(224.01905518,75.84970998)
\curveto(224.11906378,75.88970274)(224.21906368,75.91970271)(224.31905518,75.93970998)
\curveto(224.41906348,75.95970267)(224.52406337,75.97970265)(224.63405518,75.99970998)
\curveto(224.68406321,76.01970261)(224.72906317,76.02470261)(224.76905518,76.01470998)
\curveto(224.80906309,76.00470263)(224.85406304,76.00970262)(224.90405518,76.02970998)
\curveto(224.95406294,76.03970259)(225.03906286,76.04470259)(225.15905518,76.04470998)
\curveto(225.26906263,76.04470259)(225.35406254,76.03970259)(225.41405518,76.02970998)
\curveto(225.47406242,76.00970262)(225.53406236,75.99970263)(225.59405518,75.99970998)
\curveto(225.65406224,76.00970262)(225.71406218,76.00470263)(225.77405518,75.98470998)
\curveto(225.91406198,75.94470269)(226.04906185,75.90970272)(226.17905518,75.87970998)
\curveto(226.30906159,75.84970278)(226.43406146,75.80970282)(226.55405518,75.75970998)
\curveto(226.6940612,75.69970293)(226.81906108,75.629703)(226.92905518,75.54970998)
\curveto(227.03906086,75.47970315)(227.14906075,75.40470323)(227.25905518,75.32470998)
\lineto(227.31905518,75.26470998)
\curveto(227.33906056,75.25470338)(227.35906054,75.23970339)(227.37905518,75.21970998)
\curveto(227.53906036,75.09970353)(227.68406021,74.96470367)(227.81405518,74.81470998)
\curveto(227.94405995,74.66470397)(228.06905983,74.50470413)(228.18905518,74.33470998)
\curveto(228.40905949,74.02470461)(228.61405928,73.7297049)(228.80405518,73.44970998)
\curveto(228.94405895,73.21970541)(229.07905882,72.98970564)(229.20905518,72.75970998)
\curveto(229.33905856,72.53970609)(229.47405842,72.31970631)(229.61405518,72.09970998)
\curveto(229.78405811,71.84970678)(229.96405793,71.60970702)(230.15405518,71.37970998)
\curveto(230.34405755,71.15970747)(230.56905733,70.96970766)(230.82905518,70.80970998)
\curveto(230.88905701,70.76970786)(230.94905695,70.7347079)(231.00905518,70.70470998)
\curveto(231.05905684,70.67470796)(231.12405677,70.64470799)(231.20405518,70.61470998)
\curveto(231.27405662,70.59470804)(231.33405656,70.58970804)(231.38405518,70.59970998)
\curveto(231.45405644,70.61970801)(231.50905639,70.65470798)(231.54905518,70.70470998)
\curveto(231.57905632,70.75470788)(231.5990563,70.81470782)(231.60905518,70.88470998)
\lineto(231.60905518,71.12470998)
\lineto(231.60905518,71.87470998)
\lineto(231.60905518,74.67970998)
\lineto(231.60905518,75.33970998)
\curveto(231.60905629,75.4297032)(231.61405628,75.51470312)(231.62405518,75.59470998)
\curveto(231.62405627,75.67470296)(231.64405625,75.73970289)(231.68405518,75.78970998)
\curveto(231.72405617,75.83970279)(231.7990561,75.87970275)(231.90905518,75.90970998)
\curveto(232.00905589,75.94970268)(232.10905579,75.95970267)(232.20905518,75.93970998)
\lineto(232.34405518,75.93970998)
\curveto(232.41405548,75.91970271)(232.47405542,75.89970273)(232.52405518,75.87970998)
\curveto(232.57405532,75.85970277)(232.61405528,75.82470281)(232.64405518,75.77470998)
\curveto(232.68405521,75.72470291)(232.70405519,75.65470298)(232.70405518,75.56470998)
\lineto(232.70405518,75.29470998)
\lineto(232.70405518,74.39470998)
\lineto(232.70405518,70.88470998)
\lineto(232.70405518,69.81970998)
\curveto(232.70405519,69.73970889)(232.70905519,69.64970898)(232.71905518,69.54970998)
\curveto(232.71905518,69.44970918)(232.70905519,69.36470927)(232.68905518,69.29470998)
\curveto(232.61905528,69.08470955)(232.43905546,69.01970961)(232.14905518,69.09970998)
\curveto(232.10905579,69.10970952)(232.07405582,69.10970952)(232.04405518,69.09970998)
\curveto(232.00405589,69.09970953)(231.95905594,69.10970952)(231.90905518,69.12970998)
\curveto(231.82905607,69.14970948)(231.74405615,69.16970946)(231.65405518,69.18970998)
\curveto(231.56405633,69.20970942)(231.47905642,69.2347094)(231.39905518,69.26470998)
\curveto(230.90905699,69.42470921)(230.4940574,69.62470901)(230.15405518,69.86470998)
\curveto(229.90405799,70.04470859)(229.67905822,70.24970838)(229.47905518,70.47970998)
\curveto(229.26905863,70.70970792)(229.07405882,70.94970768)(228.89405518,71.19970998)
\curveto(228.71405918,71.45970717)(228.54405935,71.72470691)(228.38405518,71.99470998)
\curveto(228.21405968,72.27470636)(228.03905986,72.54470609)(227.85905518,72.80470998)
\curveto(227.77906012,72.91470572)(227.70406019,73.01970561)(227.63405518,73.11970998)
\curveto(227.56406033,73.2297054)(227.48906041,73.33970529)(227.40905518,73.44970998)
\curveto(227.37906052,73.48970514)(227.34906055,73.52470511)(227.31905518,73.55470998)
\curveto(227.27906062,73.59470504)(227.24906065,73.634705)(227.22905518,73.67470998)
\curveto(227.11906078,73.81470482)(226.9940609,73.93970469)(226.85405518,74.04970998)
\curveto(226.82406107,74.06970456)(226.7990611,74.09470454)(226.77905518,74.12470998)
\curveto(226.74906115,74.15470448)(226.71906118,74.17970445)(226.68905518,74.19970998)
\curveto(226.58906131,74.27970435)(226.48906141,74.34470429)(226.38905518,74.39470998)
\curveto(226.28906161,74.45470418)(226.17906172,74.50970412)(226.05905518,74.55970998)
\curveto(225.98906191,74.58970404)(225.91406198,74.60970402)(225.83405518,74.61970998)
\lineto(225.59405518,74.67970998)
\lineto(225.50405518,74.67970998)
\curveto(225.47406242,74.68970394)(225.44406245,74.69470394)(225.41405518,74.69470998)
\curveto(225.34406255,74.71470392)(225.24906265,74.71970391)(225.12905518,74.70970998)
\curveto(224.9990629,74.70970392)(224.899063,74.69970393)(224.82905518,74.67970998)
\curveto(224.74906315,74.65970397)(224.67406322,74.63970399)(224.60405518,74.61970998)
\curveto(224.52406337,74.60970402)(224.44406345,74.58970404)(224.36405518,74.55970998)
\curveto(224.12406377,74.44970418)(223.92406397,74.29970433)(223.76405518,74.10970998)
\curveto(223.5940643,73.9297047)(223.45406444,73.70970492)(223.34405518,73.44970998)
\curveto(223.32406457,73.37970525)(223.30906459,73.30970532)(223.29905518,73.23970998)
\curveto(223.27906462,73.16970546)(223.25906464,73.09470554)(223.23905518,73.01470998)
\curveto(223.21906468,72.9347057)(223.20906469,72.82470581)(223.20905518,72.68470998)
\curveto(223.20906469,72.55470608)(223.21906468,72.44970618)(223.23905518,72.36970998)
\curveto(223.24906465,72.30970632)(223.25406464,72.25470638)(223.25405518,72.20470998)
\curveto(223.25406464,72.15470648)(223.26406463,72.10470653)(223.28405518,72.05470998)
\curveto(223.32406457,71.95470668)(223.36406453,71.85970677)(223.40405518,71.76970998)
\curveto(223.44406445,71.68970694)(223.48906441,71.60970702)(223.53905518,71.52970998)
\curveto(223.55906434,71.49970713)(223.58406431,71.46970716)(223.61405518,71.43970998)
\curveto(223.64406425,71.41970721)(223.66906423,71.39470724)(223.68905518,71.36470998)
\lineto(223.76405518,71.28970998)
\curveto(223.78406411,71.25970737)(223.80406409,71.2347074)(223.82405518,71.21470998)
\lineto(224.03405518,71.06470998)
\curveto(224.0940638,71.02470761)(224.15906374,70.97970765)(224.22905518,70.92970998)
\curveto(224.31906358,70.86970776)(224.42406347,70.81970781)(224.54405518,70.77970998)
\curveto(224.65406324,70.74970788)(224.76406313,70.71470792)(224.87405518,70.67470998)
\curveto(224.98406291,70.634708)(225.12906277,70.60970802)(225.30905518,70.59970998)
\curveto(225.47906242,70.58970804)(225.60406229,70.55970807)(225.68405518,70.50970998)
\curveto(225.76406213,70.45970817)(225.80906209,70.38470825)(225.81905518,70.28470998)
\curveto(225.82906207,70.18470845)(225.83406206,70.07470856)(225.83405518,69.95470998)
\curveto(225.83406206,69.91470872)(225.83906206,69.87470876)(225.84905518,69.83470998)
\curveto(225.84906205,69.79470884)(225.84406205,69.75970887)(225.83405518,69.72970998)
\curveto(225.81406208,69.67970895)(225.80406209,69.629709)(225.80405518,69.57970998)
\curveto(225.80406209,69.53970909)(225.7940621,69.49970913)(225.77405518,69.45970998)
\curveto(225.71406218,69.36970926)(225.57906232,69.32470931)(225.36905518,69.32470998)
\lineto(225.24905518,69.32470998)
\curveto(225.18906271,69.3347093)(225.12906277,69.33970929)(225.06905518,69.33970998)
\curveto(224.9990629,69.34970928)(224.93406296,69.35970927)(224.87405518,69.36970998)
\curveto(224.76406313,69.38970924)(224.66406323,69.40970922)(224.57405518,69.42970998)
\curveto(224.47406342,69.44970918)(224.37906352,69.47970915)(224.28905518,69.51970998)
\curveto(224.21906368,69.53970909)(224.15906374,69.55970907)(224.10905518,69.57970998)
\lineto(223.92905518,69.63970998)
\curveto(223.66906423,69.75970887)(223.42406447,69.91470872)(223.19405518,70.10470998)
\curveto(222.96406493,70.30470833)(222.77906512,70.51970811)(222.63905518,70.74970998)
\curveto(222.55906534,70.85970777)(222.4940654,70.97470766)(222.44405518,71.09470998)
\lineto(222.29405518,71.48470998)
\curveto(222.24406565,71.59470704)(222.21406568,71.70970692)(222.20405518,71.82970998)
\curveto(222.18406571,71.94970668)(222.15906574,72.07470656)(222.12905518,72.20470998)
\curveto(222.12906577,72.27470636)(222.12906577,72.33970629)(222.12905518,72.39970998)
\curveto(222.11906578,72.45970617)(222.10906579,72.52470611)(222.09905518,72.59470998)
}
}
{
\newrgbcolor{curcolor}{0 0 0}
\pscustom[linestyle=none,fillstyle=solid,fillcolor=curcolor]
{
\newpath
\moveto(231.06905518,78.63431936)
\lineto(231.06905518,79.26431936)
\lineto(231.06905518,79.45931936)
\curveto(231.06905683,79.52931683)(231.07905682,79.58931677)(231.09905518,79.63931936)
\curveto(231.13905676,79.70931665)(231.17905672,79.7593166)(231.21905518,79.78931936)
\curveto(231.26905663,79.82931653)(231.33405656,79.84931651)(231.41405518,79.84931936)
\curveto(231.4940564,79.8593165)(231.57905632,79.86431649)(231.66905518,79.86431936)
\lineto(232.38905518,79.86431936)
\curveto(232.86905503,79.86431649)(233.27905462,79.80431655)(233.61905518,79.68431936)
\curveto(233.95905394,79.56431679)(234.23405366,79.36931699)(234.44405518,79.09931936)
\curveto(234.4940534,79.02931733)(234.53905336,78.9593174)(234.57905518,78.88931936)
\curveto(234.62905327,78.82931753)(234.67405322,78.7543176)(234.71405518,78.66431936)
\curveto(234.72405317,78.64431771)(234.73405316,78.61431774)(234.74405518,78.57431936)
\curveto(234.76405313,78.53431782)(234.76905313,78.48931787)(234.75905518,78.43931936)
\curveto(234.72905317,78.34931801)(234.65405324,78.29431806)(234.53405518,78.27431936)
\curveto(234.42405347,78.2543181)(234.32905357,78.26931809)(234.24905518,78.31931936)
\curveto(234.17905372,78.34931801)(234.11405378,78.39431796)(234.05405518,78.45431936)
\curveto(234.00405389,78.52431783)(233.95405394,78.58931777)(233.90405518,78.64931936)
\curveto(233.85405404,78.71931764)(233.77905412,78.77931758)(233.67905518,78.82931936)
\curveto(233.58905431,78.88931747)(233.4990544,78.93931742)(233.40905518,78.97931936)
\curveto(233.37905452,78.99931736)(233.31905458,79.02431733)(233.22905518,79.05431936)
\curveto(233.14905475,79.08431727)(233.07905482,79.08931727)(233.01905518,79.06931936)
\curveto(232.87905502,79.03931732)(232.78905511,78.97931738)(232.74905518,78.88931936)
\curveto(232.71905518,78.80931755)(232.70405519,78.71931764)(232.70405518,78.61931936)
\curveto(232.70405519,78.51931784)(232.67905522,78.43431792)(232.62905518,78.36431936)
\curveto(232.55905534,78.27431808)(232.41905548,78.22931813)(232.20905518,78.22931936)
\lineto(231.65405518,78.22931936)
\lineto(231.42905518,78.22931936)
\curveto(231.34905655,78.23931812)(231.28405661,78.2593181)(231.23405518,78.28931936)
\curveto(231.15405674,78.34931801)(231.10905679,78.41931794)(231.09905518,78.49931936)
\curveto(231.08905681,78.51931784)(231.08405681,78.53931782)(231.08405518,78.55931936)
\curveto(231.08405681,78.58931777)(231.07905682,78.61431774)(231.06905518,78.63431936)
}
}
{
\newrgbcolor{curcolor}{0 0 0}
\pscustom[linestyle=none,fillstyle=solid,fillcolor=curcolor]
{
}
}
{
\newrgbcolor{curcolor}{0 0 0}
\pscustom[linestyle=none,fillstyle=solid,fillcolor=curcolor]
{
\newpath
\moveto(222.09905518,89.26463186)
\curveto(222.08906581,89.95462722)(222.20906569,90.55462662)(222.45905518,91.06463186)
\curveto(222.70906519,91.58462559)(223.04406485,91.9796252)(223.46405518,92.24963186)
\curveto(223.54406435,92.29962488)(223.63406426,92.34462483)(223.73405518,92.38463186)
\curveto(223.82406407,92.42462475)(223.91906398,92.46962471)(224.01905518,92.51963186)
\curveto(224.11906378,92.55962462)(224.21906368,92.58962459)(224.31905518,92.60963186)
\curveto(224.41906348,92.62962455)(224.52406337,92.64962453)(224.63405518,92.66963186)
\curveto(224.68406321,92.68962449)(224.72906317,92.69462448)(224.76905518,92.68463186)
\curveto(224.80906309,92.6746245)(224.85406304,92.6796245)(224.90405518,92.69963186)
\curveto(224.95406294,92.70962447)(225.03906286,92.71462446)(225.15905518,92.71463186)
\curveto(225.26906263,92.71462446)(225.35406254,92.70962447)(225.41405518,92.69963186)
\curveto(225.47406242,92.6796245)(225.53406236,92.66962451)(225.59405518,92.66963186)
\curveto(225.65406224,92.6796245)(225.71406218,92.6746245)(225.77405518,92.65463186)
\curveto(225.91406198,92.61462456)(226.04906185,92.5796246)(226.17905518,92.54963186)
\curveto(226.30906159,92.51962466)(226.43406146,92.4796247)(226.55405518,92.42963186)
\curveto(226.6940612,92.36962481)(226.81906108,92.29962488)(226.92905518,92.21963186)
\curveto(227.03906086,92.14962503)(227.14906075,92.0746251)(227.25905518,91.99463186)
\lineto(227.31905518,91.93463186)
\curveto(227.33906056,91.92462525)(227.35906054,91.90962527)(227.37905518,91.88963186)
\curveto(227.53906036,91.76962541)(227.68406021,91.63462554)(227.81405518,91.48463186)
\curveto(227.94405995,91.33462584)(228.06905983,91.174626)(228.18905518,91.00463186)
\curveto(228.40905949,90.69462648)(228.61405928,90.39962678)(228.80405518,90.11963186)
\curveto(228.94405895,89.88962729)(229.07905882,89.65962752)(229.20905518,89.42963186)
\curveto(229.33905856,89.20962797)(229.47405842,88.98962819)(229.61405518,88.76963186)
\curveto(229.78405811,88.51962866)(229.96405793,88.2796289)(230.15405518,88.04963186)
\curveto(230.34405755,87.82962935)(230.56905733,87.63962954)(230.82905518,87.47963186)
\curveto(230.88905701,87.43962974)(230.94905695,87.40462977)(231.00905518,87.37463186)
\curveto(231.05905684,87.34462983)(231.12405677,87.31462986)(231.20405518,87.28463186)
\curveto(231.27405662,87.26462991)(231.33405656,87.25962992)(231.38405518,87.26963186)
\curveto(231.45405644,87.28962989)(231.50905639,87.32462985)(231.54905518,87.37463186)
\curveto(231.57905632,87.42462975)(231.5990563,87.48462969)(231.60905518,87.55463186)
\lineto(231.60905518,87.79463186)
\lineto(231.60905518,88.54463186)
\lineto(231.60905518,91.34963186)
\lineto(231.60905518,92.00963186)
\curveto(231.60905629,92.09962508)(231.61405628,92.18462499)(231.62405518,92.26463186)
\curveto(231.62405627,92.34462483)(231.64405625,92.40962477)(231.68405518,92.45963186)
\curveto(231.72405617,92.50962467)(231.7990561,92.54962463)(231.90905518,92.57963186)
\curveto(232.00905589,92.61962456)(232.10905579,92.62962455)(232.20905518,92.60963186)
\lineto(232.34405518,92.60963186)
\curveto(232.41405548,92.58962459)(232.47405542,92.56962461)(232.52405518,92.54963186)
\curveto(232.57405532,92.52962465)(232.61405528,92.49462468)(232.64405518,92.44463186)
\curveto(232.68405521,92.39462478)(232.70405519,92.32462485)(232.70405518,92.23463186)
\lineto(232.70405518,91.96463186)
\lineto(232.70405518,91.06463186)
\lineto(232.70405518,87.55463186)
\lineto(232.70405518,86.48963186)
\curveto(232.70405519,86.40963077)(232.70905519,86.31963086)(232.71905518,86.21963186)
\curveto(232.71905518,86.11963106)(232.70905519,86.03463114)(232.68905518,85.96463186)
\curveto(232.61905528,85.75463142)(232.43905546,85.68963149)(232.14905518,85.76963186)
\curveto(232.10905579,85.7796314)(232.07405582,85.7796314)(232.04405518,85.76963186)
\curveto(232.00405589,85.76963141)(231.95905594,85.7796314)(231.90905518,85.79963186)
\curveto(231.82905607,85.81963136)(231.74405615,85.83963134)(231.65405518,85.85963186)
\curveto(231.56405633,85.8796313)(231.47905642,85.90463127)(231.39905518,85.93463186)
\curveto(230.90905699,86.09463108)(230.4940574,86.29463088)(230.15405518,86.53463186)
\curveto(229.90405799,86.71463046)(229.67905822,86.91963026)(229.47905518,87.14963186)
\curveto(229.26905863,87.3796298)(229.07405882,87.61962956)(228.89405518,87.86963186)
\curveto(228.71405918,88.12962905)(228.54405935,88.39462878)(228.38405518,88.66463186)
\curveto(228.21405968,88.94462823)(228.03905986,89.21462796)(227.85905518,89.47463186)
\curveto(227.77906012,89.58462759)(227.70406019,89.68962749)(227.63405518,89.78963186)
\curveto(227.56406033,89.89962728)(227.48906041,90.00962717)(227.40905518,90.11963186)
\curveto(227.37906052,90.15962702)(227.34906055,90.19462698)(227.31905518,90.22463186)
\curveto(227.27906062,90.26462691)(227.24906065,90.30462687)(227.22905518,90.34463186)
\curveto(227.11906078,90.48462669)(226.9940609,90.60962657)(226.85405518,90.71963186)
\curveto(226.82406107,90.73962644)(226.7990611,90.76462641)(226.77905518,90.79463186)
\curveto(226.74906115,90.82462635)(226.71906118,90.84962633)(226.68905518,90.86963186)
\curveto(226.58906131,90.94962623)(226.48906141,91.01462616)(226.38905518,91.06463186)
\curveto(226.28906161,91.12462605)(226.17906172,91.179626)(226.05905518,91.22963186)
\curveto(225.98906191,91.25962592)(225.91406198,91.2796259)(225.83405518,91.28963186)
\lineto(225.59405518,91.34963186)
\lineto(225.50405518,91.34963186)
\curveto(225.47406242,91.35962582)(225.44406245,91.36462581)(225.41405518,91.36463186)
\curveto(225.34406255,91.38462579)(225.24906265,91.38962579)(225.12905518,91.37963186)
\curveto(224.9990629,91.3796258)(224.899063,91.36962581)(224.82905518,91.34963186)
\curveto(224.74906315,91.32962585)(224.67406322,91.30962587)(224.60405518,91.28963186)
\curveto(224.52406337,91.2796259)(224.44406345,91.25962592)(224.36405518,91.22963186)
\curveto(224.12406377,91.11962606)(223.92406397,90.96962621)(223.76405518,90.77963186)
\curveto(223.5940643,90.59962658)(223.45406444,90.3796268)(223.34405518,90.11963186)
\curveto(223.32406457,90.04962713)(223.30906459,89.9796272)(223.29905518,89.90963186)
\curveto(223.27906462,89.83962734)(223.25906464,89.76462741)(223.23905518,89.68463186)
\curveto(223.21906468,89.60462757)(223.20906469,89.49462768)(223.20905518,89.35463186)
\curveto(223.20906469,89.22462795)(223.21906468,89.11962806)(223.23905518,89.03963186)
\curveto(223.24906465,88.9796282)(223.25406464,88.92462825)(223.25405518,88.87463186)
\curveto(223.25406464,88.82462835)(223.26406463,88.7746284)(223.28405518,88.72463186)
\curveto(223.32406457,88.62462855)(223.36406453,88.52962865)(223.40405518,88.43963186)
\curveto(223.44406445,88.35962882)(223.48906441,88.2796289)(223.53905518,88.19963186)
\curveto(223.55906434,88.16962901)(223.58406431,88.13962904)(223.61405518,88.10963186)
\curveto(223.64406425,88.08962909)(223.66906423,88.06462911)(223.68905518,88.03463186)
\lineto(223.76405518,87.95963186)
\curveto(223.78406411,87.92962925)(223.80406409,87.90462927)(223.82405518,87.88463186)
\lineto(224.03405518,87.73463186)
\curveto(224.0940638,87.69462948)(224.15906374,87.64962953)(224.22905518,87.59963186)
\curveto(224.31906358,87.53962964)(224.42406347,87.48962969)(224.54405518,87.44963186)
\curveto(224.65406324,87.41962976)(224.76406313,87.38462979)(224.87405518,87.34463186)
\curveto(224.98406291,87.30462987)(225.12906277,87.2796299)(225.30905518,87.26963186)
\curveto(225.47906242,87.25962992)(225.60406229,87.22962995)(225.68405518,87.17963186)
\curveto(225.76406213,87.12963005)(225.80906209,87.05463012)(225.81905518,86.95463186)
\curveto(225.82906207,86.85463032)(225.83406206,86.74463043)(225.83405518,86.62463186)
\curveto(225.83406206,86.58463059)(225.83906206,86.54463063)(225.84905518,86.50463186)
\curveto(225.84906205,86.46463071)(225.84406205,86.42963075)(225.83405518,86.39963186)
\curveto(225.81406208,86.34963083)(225.80406209,86.29963088)(225.80405518,86.24963186)
\curveto(225.80406209,86.20963097)(225.7940621,86.16963101)(225.77405518,86.12963186)
\curveto(225.71406218,86.03963114)(225.57906232,85.99463118)(225.36905518,85.99463186)
\lineto(225.24905518,85.99463186)
\curveto(225.18906271,86.00463117)(225.12906277,86.00963117)(225.06905518,86.00963186)
\curveto(224.9990629,86.01963116)(224.93406296,86.02963115)(224.87405518,86.03963186)
\curveto(224.76406313,86.05963112)(224.66406323,86.0796311)(224.57405518,86.09963186)
\curveto(224.47406342,86.11963106)(224.37906352,86.14963103)(224.28905518,86.18963186)
\curveto(224.21906368,86.20963097)(224.15906374,86.22963095)(224.10905518,86.24963186)
\lineto(223.92905518,86.30963186)
\curveto(223.66906423,86.42963075)(223.42406447,86.58463059)(223.19405518,86.77463186)
\curveto(222.96406493,86.9746302)(222.77906512,87.18962999)(222.63905518,87.41963186)
\curveto(222.55906534,87.52962965)(222.4940654,87.64462953)(222.44405518,87.76463186)
\lineto(222.29405518,88.15463186)
\curveto(222.24406565,88.26462891)(222.21406568,88.3796288)(222.20405518,88.49963186)
\curveto(222.18406571,88.61962856)(222.15906574,88.74462843)(222.12905518,88.87463186)
\curveto(222.12906577,88.94462823)(222.12906577,89.00962817)(222.12905518,89.06963186)
\curveto(222.11906578,89.12962805)(222.10906579,89.19462798)(222.09905518,89.26463186)
}
}
{
\newrgbcolor{curcolor}{0 0 0}
\pscustom[linestyle=none,fillstyle=solid,fillcolor=curcolor]
{
\newpath
\moveto(227.61905518,101.36424123)
\lineto(227.87405518,101.36424123)
\curveto(227.95405994,101.37423353)(228.02905987,101.36923353)(228.09905518,101.34924123)
\lineto(228.33905518,101.34924123)
\lineto(228.50405518,101.34924123)
\curveto(228.60405929,101.32923357)(228.70905919,101.31923358)(228.81905518,101.31924123)
\curveto(228.91905898,101.31923358)(229.01905888,101.30923359)(229.11905518,101.28924123)
\lineto(229.26905518,101.28924123)
\curveto(229.40905849,101.25923364)(229.54905835,101.23923366)(229.68905518,101.22924123)
\curveto(229.81905808,101.21923368)(229.94905795,101.19423371)(230.07905518,101.15424123)
\curveto(230.15905774,101.13423377)(230.24405765,101.11423379)(230.33405518,101.09424123)
\lineto(230.57405518,101.03424123)
\lineto(230.87405518,100.91424123)
\curveto(230.96405693,100.88423402)(231.05405684,100.84923405)(231.14405518,100.80924123)
\curveto(231.36405653,100.70923419)(231.57905632,100.57423433)(231.78905518,100.40424123)
\curveto(231.9990559,100.24423466)(232.16905573,100.06923483)(232.29905518,99.87924123)
\curveto(232.33905556,99.82923507)(232.37905552,99.76923513)(232.41905518,99.69924123)
\curveto(232.44905545,99.63923526)(232.48405541,99.57923532)(232.52405518,99.51924123)
\curveto(232.57405532,99.43923546)(232.61405528,99.34423556)(232.64405518,99.23424123)
\curveto(232.67405522,99.12423578)(232.70405519,99.01923588)(232.73405518,98.91924123)
\curveto(232.77405512,98.80923609)(232.7990551,98.6992362)(232.80905518,98.58924123)
\curveto(232.81905508,98.47923642)(232.83405506,98.36423654)(232.85405518,98.24424123)
\curveto(232.86405503,98.2042367)(232.86405503,98.15923674)(232.85405518,98.10924123)
\curveto(232.85405504,98.06923683)(232.85905504,98.02923687)(232.86905518,97.98924123)
\curveto(232.87905502,97.94923695)(232.88405501,97.89423701)(232.88405518,97.82424123)
\curveto(232.88405501,97.75423715)(232.87905502,97.7042372)(232.86905518,97.67424123)
\curveto(232.84905505,97.62423728)(232.84405505,97.57923732)(232.85405518,97.53924123)
\curveto(232.86405503,97.4992374)(232.86405503,97.46423744)(232.85405518,97.43424123)
\lineto(232.85405518,97.34424123)
\curveto(232.83405506,97.28423762)(232.81905508,97.21923768)(232.80905518,97.14924123)
\curveto(232.80905509,97.08923781)(232.80405509,97.02423788)(232.79405518,96.95424123)
\curveto(232.74405515,96.78423812)(232.6940552,96.62423828)(232.64405518,96.47424123)
\curveto(232.5940553,96.32423858)(232.52905537,96.17923872)(232.44905518,96.03924123)
\curveto(232.40905549,95.98923891)(232.37905552,95.93423897)(232.35905518,95.87424123)
\curveto(232.32905557,95.82423908)(232.2940556,95.77423913)(232.25405518,95.72424123)
\curveto(232.07405582,95.48423942)(231.85405604,95.28423962)(231.59405518,95.12424123)
\curveto(231.33405656,94.96423994)(231.04905685,94.82424008)(230.73905518,94.70424123)
\curveto(230.5990573,94.64424026)(230.45905744,94.5992403)(230.31905518,94.56924123)
\curveto(230.16905773,94.53924036)(230.01405788,94.5042404)(229.85405518,94.46424123)
\curveto(229.74405815,94.44424046)(229.63405826,94.42924047)(229.52405518,94.41924123)
\curveto(229.41405848,94.40924049)(229.30405859,94.39424051)(229.19405518,94.37424123)
\curveto(229.15405874,94.36424054)(229.11405878,94.35924054)(229.07405518,94.35924123)
\curveto(229.03405886,94.36924053)(228.9940589,94.36924053)(228.95405518,94.35924123)
\curveto(228.90405899,94.34924055)(228.85405904,94.34424056)(228.80405518,94.34424123)
\lineto(228.63905518,94.34424123)
\curveto(228.58905931,94.32424058)(228.53905936,94.31924058)(228.48905518,94.32924123)
\curveto(228.42905947,94.33924056)(228.37405952,94.33924056)(228.32405518,94.32924123)
\curveto(228.28405961,94.31924058)(228.23905966,94.31924058)(228.18905518,94.32924123)
\curveto(228.13905976,94.33924056)(228.08905981,94.33424057)(228.03905518,94.31424123)
\curveto(227.96905993,94.29424061)(227.89406,94.28924061)(227.81405518,94.29924123)
\curveto(227.72406017,94.30924059)(227.63906026,94.31424059)(227.55905518,94.31424123)
\curveto(227.46906043,94.31424059)(227.36906053,94.30924059)(227.25905518,94.29924123)
\curveto(227.13906076,94.28924061)(227.03906086,94.29424061)(226.95905518,94.31424123)
\lineto(226.67405518,94.31424123)
\lineto(226.04405518,94.35924123)
\curveto(225.94406195,94.36924053)(225.84906205,94.37924052)(225.75905518,94.38924123)
\lineto(225.45905518,94.41924123)
\curveto(225.40906249,94.43924046)(225.35906254,94.44424046)(225.30905518,94.43424123)
\curveto(225.24906265,94.43424047)(225.1940627,94.44424046)(225.14405518,94.46424123)
\curveto(224.97406292,94.51424039)(224.80906309,94.55424035)(224.64905518,94.58424123)
\curveto(224.47906342,94.61424029)(224.31906358,94.66424024)(224.16905518,94.73424123)
\curveto(223.70906419,94.92423998)(223.33406456,95.14423976)(223.04405518,95.39424123)
\curveto(222.75406514,95.65423925)(222.50906539,96.01423889)(222.30905518,96.47424123)
\curveto(222.25906564,96.6042383)(222.22406567,96.73423817)(222.20405518,96.86424123)
\curveto(222.18406571,97.0042379)(222.15906574,97.14423776)(222.12905518,97.28424123)
\curveto(222.11906578,97.35423755)(222.11406578,97.41923748)(222.11405518,97.47924123)
\curveto(222.11406578,97.53923736)(222.10906579,97.6042373)(222.09905518,97.67424123)
\curveto(222.07906582,98.5042364)(222.22906567,99.17423573)(222.54905518,99.68424123)
\curveto(222.85906504,100.19423471)(223.2990646,100.57423433)(223.86905518,100.82424123)
\curveto(223.98906391,100.87423403)(224.11406378,100.91923398)(224.24405518,100.95924123)
\curveto(224.37406352,100.9992339)(224.50906339,101.04423386)(224.64905518,101.09424123)
\curveto(224.72906317,101.11423379)(224.81406308,101.12923377)(224.90405518,101.13924123)
\lineto(225.14405518,101.19924123)
\curveto(225.25406264,101.22923367)(225.36406253,101.24423366)(225.47405518,101.24424123)
\curveto(225.58406231,101.25423365)(225.6940622,101.26923363)(225.80405518,101.28924123)
\curveto(225.85406204,101.30923359)(225.899062,101.31423359)(225.93905518,101.30424123)
\curveto(225.97906192,101.3042336)(226.01906188,101.30923359)(226.05905518,101.31924123)
\curveto(226.10906179,101.32923357)(226.16406173,101.32923357)(226.22405518,101.31924123)
\curveto(226.27406162,101.31923358)(226.32406157,101.32423358)(226.37405518,101.33424123)
\lineto(226.50905518,101.33424123)
\curveto(226.56906133,101.35423355)(226.63906126,101.35423355)(226.71905518,101.33424123)
\curveto(226.78906111,101.32423358)(226.85406104,101.32923357)(226.91405518,101.34924123)
\curveto(226.94406095,101.35923354)(226.98406091,101.36423354)(227.03405518,101.36424123)
\lineto(227.15405518,101.36424123)
\lineto(227.61905518,101.36424123)
\moveto(229.94405518,99.81924123)
\curveto(229.62405827,99.91923498)(229.25905864,99.97923492)(228.84905518,99.99924123)
\curveto(228.43905946,100.01923488)(228.02905987,100.02923487)(227.61905518,100.02924123)
\curveto(227.18906071,100.02923487)(226.76906113,100.01923488)(226.35905518,99.99924123)
\curveto(225.94906195,99.97923492)(225.56406233,99.93423497)(225.20405518,99.86424123)
\curveto(224.84406305,99.79423511)(224.52406337,99.68423522)(224.24405518,99.53424123)
\curveto(223.95406394,99.39423551)(223.71906418,99.1992357)(223.53905518,98.94924123)
\curveto(223.42906447,98.78923611)(223.34906455,98.60923629)(223.29905518,98.40924123)
\curveto(223.23906466,98.20923669)(223.20906469,97.96423694)(223.20905518,97.67424123)
\curveto(223.22906467,97.65423725)(223.23906466,97.61923728)(223.23905518,97.56924123)
\curveto(223.22906467,97.51923738)(223.22906467,97.47923742)(223.23905518,97.44924123)
\curveto(223.25906464,97.36923753)(223.27906462,97.29423761)(223.29905518,97.22424123)
\curveto(223.30906459,97.16423774)(223.32906457,97.0992378)(223.35905518,97.02924123)
\curveto(223.47906442,96.75923814)(223.64906425,96.53923836)(223.86905518,96.36924123)
\curveto(224.07906382,96.20923869)(224.32406357,96.07423883)(224.60405518,95.96424123)
\curveto(224.71406318,95.91423899)(224.83406306,95.87423903)(224.96405518,95.84424123)
\curveto(225.08406281,95.82423908)(225.20906269,95.7992391)(225.33905518,95.76924123)
\curveto(225.38906251,95.74923915)(225.44406245,95.73923916)(225.50405518,95.73924123)
\curveto(225.55406234,95.73923916)(225.60406229,95.73423917)(225.65405518,95.72424123)
\curveto(225.74406215,95.71423919)(225.83906206,95.7042392)(225.93905518,95.69424123)
\curveto(226.02906187,95.68423922)(226.12406177,95.67423923)(226.22405518,95.66424123)
\curveto(226.30406159,95.66423924)(226.38906151,95.65923924)(226.47905518,95.64924123)
\lineto(226.71905518,95.64924123)
\lineto(226.89905518,95.64924123)
\curveto(226.92906097,95.63923926)(226.96406093,95.63423927)(227.00405518,95.63424123)
\lineto(227.13905518,95.63424123)
\lineto(227.58905518,95.63424123)
\curveto(227.66906023,95.63423927)(227.75406014,95.62923927)(227.84405518,95.61924123)
\curveto(227.92405997,95.61923928)(227.9990599,95.62923927)(228.06905518,95.64924123)
\lineto(228.33905518,95.64924123)
\curveto(228.35905954,95.64923925)(228.38905951,95.64423926)(228.42905518,95.63424123)
\curveto(228.45905944,95.63423927)(228.48405941,95.63923926)(228.50405518,95.64924123)
\curveto(228.60405929,95.65923924)(228.70405919,95.66423924)(228.80405518,95.66424123)
\curveto(228.894059,95.67423923)(228.9940589,95.68423922)(229.10405518,95.69424123)
\curveto(229.22405867,95.72423918)(229.34905855,95.73923916)(229.47905518,95.73924123)
\curveto(229.5990583,95.74923915)(229.71405818,95.77423913)(229.82405518,95.81424123)
\curveto(230.12405777,95.89423901)(230.38905751,95.97923892)(230.61905518,96.06924123)
\curveto(230.84905705,96.16923873)(231.06405683,96.31423859)(231.26405518,96.50424123)
\curveto(231.46405643,96.71423819)(231.61405628,96.97923792)(231.71405518,97.29924123)
\curveto(231.73405616,97.33923756)(231.74405615,97.37423753)(231.74405518,97.40424123)
\curveto(231.73405616,97.44423746)(231.73905616,97.48923741)(231.75905518,97.53924123)
\curveto(231.76905613,97.57923732)(231.77905612,97.64923725)(231.78905518,97.74924123)
\curveto(231.7990561,97.85923704)(231.7940561,97.94423696)(231.77405518,98.00424123)
\curveto(231.75405614,98.07423683)(231.74405615,98.14423676)(231.74405518,98.21424123)
\curveto(231.73405616,98.28423662)(231.71905618,98.34923655)(231.69905518,98.40924123)
\curveto(231.63905626,98.60923629)(231.55405634,98.78923611)(231.44405518,98.94924123)
\curveto(231.42405647,98.97923592)(231.40405649,99.0042359)(231.38405518,99.02424123)
\lineto(231.32405518,99.08424123)
\curveto(231.30405659,99.12423578)(231.26405663,99.17423573)(231.20405518,99.23424123)
\curveto(231.06405683,99.33423557)(230.93405696,99.41923548)(230.81405518,99.48924123)
\curveto(230.6940572,99.55923534)(230.54905735,99.62923527)(230.37905518,99.69924123)
\curveto(230.30905759,99.72923517)(230.23905766,99.74923515)(230.16905518,99.75924123)
\curveto(230.0990578,99.77923512)(230.02405787,99.7992351)(229.94405518,99.81924123)
}
}
{
\newrgbcolor{curcolor}{0 0 0}
\pscustom[linestyle=none,fillstyle=solid,fillcolor=curcolor]
{
\newpath
\moveto(222.09905518,106.77385061)
\curveto(222.0990658,106.87384575)(222.10906579,106.96884566)(222.12905518,107.05885061)
\curveto(222.13906576,107.14884548)(222.16906573,107.21384541)(222.21905518,107.25385061)
\curveto(222.2990656,107.31384531)(222.40406549,107.34384528)(222.53405518,107.34385061)
\lineto(222.92405518,107.34385061)
\lineto(224.42405518,107.34385061)
\lineto(230.81405518,107.34385061)
\lineto(231.98405518,107.34385061)
\lineto(232.29905518,107.34385061)
\curveto(232.3990555,107.35384527)(232.47905542,107.33884529)(232.53905518,107.29885061)
\curveto(232.61905528,107.24884538)(232.66905523,107.17384545)(232.68905518,107.07385061)
\curveto(232.6990552,106.98384564)(232.70405519,106.87384575)(232.70405518,106.74385061)
\lineto(232.70405518,106.51885061)
\curveto(232.68405521,106.43884619)(232.66905523,106.36884626)(232.65905518,106.30885061)
\curveto(232.63905526,106.24884638)(232.5990553,106.19884643)(232.53905518,106.15885061)
\curveto(232.47905542,106.11884651)(232.40405549,106.09884653)(232.31405518,106.09885061)
\lineto(232.01405518,106.09885061)
\lineto(230.91905518,106.09885061)
\lineto(225.57905518,106.09885061)
\curveto(225.48906241,106.07884655)(225.41406248,106.06384656)(225.35405518,106.05385061)
\curveto(225.28406261,106.05384657)(225.22406267,106.0238466)(225.17405518,105.96385061)
\curveto(225.12406277,105.89384673)(225.0990628,105.80384682)(225.09905518,105.69385061)
\curveto(225.08906281,105.59384703)(225.08406281,105.48384714)(225.08405518,105.36385061)
\lineto(225.08405518,104.22385061)
\lineto(225.08405518,103.72885061)
\curveto(225.07406282,103.56884906)(225.01406288,103.45884917)(224.90405518,103.39885061)
\curveto(224.87406302,103.37884925)(224.84406305,103.36884926)(224.81405518,103.36885061)
\curveto(224.77406312,103.36884926)(224.72906317,103.36384926)(224.67905518,103.35385061)
\curveto(224.55906334,103.33384929)(224.44906345,103.33884929)(224.34905518,103.36885061)
\curveto(224.24906365,103.40884922)(224.17906372,103.46384916)(224.13905518,103.53385061)
\curveto(224.08906381,103.61384901)(224.06406383,103.73384889)(224.06405518,103.89385061)
\curveto(224.06406383,104.05384857)(224.04906385,104.18884844)(224.01905518,104.29885061)
\curveto(224.00906389,104.34884828)(224.00406389,104.40384822)(224.00405518,104.46385061)
\curveto(223.9940639,104.5238481)(223.97906392,104.58384804)(223.95905518,104.64385061)
\curveto(223.90906399,104.79384783)(223.85906404,104.93884769)(223.80905518,105.07885061)
\curveto(223.74906415,105.21884741)(223.67906422,105.35384727)(223.59905518,105.48385061)
\curveto(223.50906439,105.623847)(223.40406449,105.74384688)(223.28405518,105.84385061)
\curveto(223.16406473,105.94384668)(223.03406486,106.03884659)(222.89405518,106.12885061)
\curveto(222.7940651,106.18884644)(222.68406521,106.23384639)(222.56405518,106.26385061)
\curveto(222.44406545,106.30384632)(222.33906556,106.35384627)(222.24905518,106.41385061)
\curveto(222.18906571,106.46384616)(222.14906575,106.53384609)(222.12905518,106.62385061)
\curveto(222.11906578,106.64384598)(222.11406578,106.66884596)(222.11405518,106.69885061)
\curveto(222.11406578,106.7288459)(222.10906579,106.75384587)(222.09905518,106.77385061)
}
}
{
\newrgbcolor{curcolor}{0 0 0}
\pscustom[linestyle=none,fillstyle=solid,fillcolor=curcolor]
{
\newpath
\moveto(222.09905518,115.12345998)
\curveto(222.0990658,115.22345513)(222.10906579,115.31845503)(222.12905518,115.40845998)
\curveto(222.13906576,115.49845485)(222.16906573,115.56345479)(222.21905518,115.60345998)
\curveto(222.2990656,115.66345469)(222.40406549,115.69345466)(222.53405518,115.69345998)
\lineto(222.92405518,115.69345998)
\lineto(224.42405518,115.69345998)
\lineto(230.81405518,115.69345998)
\lineto(231.98405518,115.69345998)
\lineto(232.29905518,115.69345998)
\curveto(232.3990555,115.70345465)(232.47905542,115.68845466)(232.53905518,115.64845998)
\curveto(232.61905528,115.59845475)(232.66905523,115.52345483)(232.68905518,115.42345998)
\curveto(232.6990552,115.33345502)(232.70405519,115.22345513)(232.70405518,115.09345998)
\lineto(232.70405518,114.86845998)
\curveto(232.68405521,114.78845556)(232.66905523,114.71845563)(232.65905518,114.65845998)
\curveto(232.63905526,114.59845575)(232.5990553,114.5484558)(232.53905518,114.50845998)
\curveto(232.47905542,114.46845588)(232.40405549,114.4484559)(232.31405518,114.44845998)
\lineto(232.01405518,114.44845998)
\lineto(230.91905518,114.44845998)
\lineto(225.57905518,114.44845998)
\curveto(225.48906241,114.42845592)(225.41406248,114.41345594)(225.35405518,114.40345998)
\curveto(225.28406261,114.40345595)(225.22406267,114.37345598)(225.17405518,114.31345998)
\curveto(225.12406277,114.24345611)(225.0990628,114.1534562)(225.09905518,114.04345998)
\curveto(225.08906281,113.94345641)(225.08406281,113.83345652)(225.08405518,113.71345998)
\lineto(225.08405518,112.57345998)
\lineto(225.08405518,112.07845998)
\curveto(225.07406282,111.91845843)(225.01406288,111.80845854)(224.90405518,111.74845998)
\curveto(224.87406302,111.72845862)(224.84406305,111.71845863)(224.81405518,111.71845998)
\curveto(224.77406312,111.71845863)(224.72906317,111.71345864)(224.67905518,111.70345998)
\curveto(224.55906334,111.68345867)(224.44906345,111.68845866)(224.34905518,111.71845998)
\curveto(224.24906365,111.75845859)(224.17906372,111.81345854)(224.13905518,111.88345998)
\curveto(224.08906381,111.96345839)(224.06406383,112.08345827)(224.06405518,112.24345998)
\curveto(224.06406383,112.40345795)(224.04906385,112.53845781)(224.01905518,112.64845998)
\curveto(224.00906389,112.69845765)(224.00406389,112.7534576)(224.00405518,112.81345998)
\curveto(223.9940639,112.87345748)(223.97906392,112.93345742)(223.95905518,112.99345998)
\curveto(223.90906399,113.14345721)(223.85906404,113.28845706)(223.80905518,113.42845998)
\curveto(223.74906415,113.56845678)(223.67906422,113.70345665)(223.59905518,113.83345998)
\curveto(223.50906439,113.97345638)(223.40406449,114.09345626)(223.28405518,114.19345998)
\curveto(223.16406473,114.29345606)(223.03406486,114.38845596)(222.89405518,114.47845998)
\curveto(222.7940651,114.53845581)(222.68406521,114.58345577)(222.56405518,114.61345998)
\curveto(222.44406545,114.6534557)(222.33906556,114.70345565)(222.24905518,114.76345998)
\curveto(222.18906571,114.81345554)(222.14906575,114.88345547)(222.12905518,114.97345998)
\curveto(222.11906578,114.99345536)(222.11406578,115.01845533)(222.11405518,115.04845998)
\curveto(222.11406578,115.07845527)(222.10906579,115.10345525)(222.09905518,115.12345998)
}
}
{
\newrgbcolor{curcolor}{0 0 0}
\pscustom[linestyle=none,fillstyle=solid,fillcolor=curcolor]
{
\newpath
\moveto(242.93538635,31.67142873)
\lineto(242.93538635,32.58642873)
\curveto(242.93539705,32.68642608)(242.93539705,32.78142599)(242.93538635,32.87142873)
\curveto(242.93539705,32.96142581)(242.95539703,33.03642573)(242.99538635,33.09642873)
\curveto(243.05539693,33.18642558)(243.13539685,33.24642552)(243.23538635,33.27642873)
\curveto(243.33539665,33.31642545)(243.44039654,33.36142541)(243.55038635,33.41142873)
\curveto(243.74039624,33.49142528)(243.93039605,33.56142521)(244.12038635,33.62142873)
\curveto(244.31039567,33.69142508)(244.50039548,33.766425)(244.69038635,33.84642873)
\curveto(244.87039511,33.91642485)(245.05539493,33.98142479)(245.24538635,34.04142873)
\curveto(245.42539456,34.10142467)(245.60539438,34.1714246)(245.78538635,34.25142873)
\curveto(245.92539406,34.31142446)(246.07039391,34.3664244)(246.22038635,34.41642873)
\curveto(246.37039361,34.4664243)(246.51539347,34.52142425)(246.65538635,34.58142873)
\curveto(247.10539288,34.76142401)(247.56039242,34.93142384)(248.02038635,35.09142873)
\curveto(248.47039151,35.25142352)(248.92039106,35.42142335)(249.37038635,35.60142873)
\curveto(249.42039056,35.62142315)(249.47039051,35.63642313)(249.52038635,35.64642873)
\lineto(249.67038635,35.70642873)
\curveto(249.89039009,35.79642297)(250.11538987,35.88142289)(250.34538635,35.96142873)
\curveto(250.56538942,36.04142273)(250.7853892,36.12642264)(251.00538635,36.21642873)
\curveto(251.09538889,36.25642251)(251.20538878,36.29642247)(251.33538635,36.33642873)
\curveto(251.45538853,36.37642239)(251.52538846,36.44142233)(251.54538635,36.53142873)
\curveto(251.55538843,36.5714222)(251.55538843,36.60142217)(251.54538635,36.62142873)
\lineto(251.48538635,36.68142873)
\curveto(251.43538855,36.73142204)(251.3803886,36.766422)(251.32038635,36.78642873)
\curveto(251.26038872,36.81642195)(251.19538879,36.84642192)(251.12538635,36.87642873)
\lineto(250.49538635,37.11642873)
\curveto(250.27538971,37.19642157)(250.06038992,37.27642149)(249.85038635,37.35642873)
\lineto(249.70038635,37.41642873)
\lineto(249.52038635,37.47642873)
\curveto(249.33039065,37.55642121)(249.14039084,37.62642114)(248.95038635,37.68642873)
\curveto(248.75039123,37.75642101)(248.55039143,37.83142094)(248.35038635,37.91142873)
\curveto(247.77039221,38.15142062)(247.1853928,38.3714204)(246.59538635,38.57142873)
\curveto(246.00539398,38.78141999)(245.42039456,39.00641976)(244.84038635,39.24642873)
\curveto(244.64039534,39.32641944)(244.43539555,39.40141937)(244.22538635,39.47142873)
\curveto(244.01539597,39.55141922)(243.81039617,39.63141914)(243.61038635,39.71142873)
\curveto(243.53039645,39.75141902)(243.43039655,39.78641898)(243.31038635,39.81642873)
\curveto(243.19039679,39.85641891)(243.10539688,39.91141886)(243.05538635,39.98142873)
\curveto(243.01539697,40.04141873)(242.985397,40.11641865)(242.96538635,40.20642873)
\curveto(242.94539704,40.30641846)(242.93539705,40.41641835)(242.93538635,40.53642873)
\curveto(242.92539706,40.65641811)(242.92539706,40.77641799)(242.93538635,40.89642873)
\curveto(242.93539705,41.01641775)(242.93539705,41.12641764)(242.93538635,41.22642873)
\curveto(242.93539705,41.31641745)(242.93539705,41.40641736)(242.93538635,41.49642873)
\curveto(242.93539705,41.59641717)(242.95539703,41.6714171)(242.99538635,41.72142873)
\curveto(243.04539694,41.81141696)(243.13539685,41.86141691)(243.26538635,41.87142873)
\curveto(243.39539659,41.88141689)(243.53539645,41.88641688)(243.68538635,41.88642873)
\lineto(245.33538635,41.88642873)
\lineto(251.60538635,41.88642873)
\lineto(252.86538635,41.88642873)
\curveto(252.97538701,41.88641688)(253.0853869,41.88641688)(253.19538635,41.88642873)
\curveto(253.30538668,41.89641687)(253.39038659,41.87641689)(253.45038635,41.82642873)
\curveto(253.51038647,41.79641697)(253.55038643,41.75141702)(253.57038635,41.69142873)
\curveto(253.5803864,41.63141714)(253.59538639,41.56141721)(253.61538635,41.48142873)
\lineto(253.61538635,41.24142873)
\lineto(253.61538635,40.88142873)
\curveto(253.60538638,40.771418)(253.56038642,40.69141808)(253.48038635,40.64142873)
\curveto(253.45038653,40.62141815)(253.42038656,40.60641816)(253.39038635,40.59642873)
\curveto(253.35038663,40.59641817)(253.30538668,40.58641818)(253.25538635,40.56642873)
\lineto(253.09038635,40.56642873)
\curveto(253.03038695,40.55641821)(252.96038702,40.55141822)(252.88038635,40.55142873)
\curveto(252.80038718,40.56141821)(252.72538726,40.5664182)(252.65538635,40.56642873)
\lineto(251.81538635,40.56642873)
\lineto(247.39038635,40.56642873)
\curveto(247.14039284,40.5664182)(246.89039309,40.5664182)(246.64038635,40.56642873)
\curveto(246.3803936,40.5664182)(246.13039385,40.56141821)(245.89038635,40.55142873)
\curveto(245.79039419,40.55141822)(245.6803943,40.54641822)(245.56038635,40.53642873)
\curveto(245.44039454,40.52641824)(245.3803946,40.4714183)(245.38038635,40.37142873)
\lineto(245.39538635,40.37142873)
\curveto(245.41539457,40.30141847)(245.4803945,40.24141853)(245.59038635,40.19142873)
\curveto(245.70039428,40.15141862)(245.79539419,40.11641865)(245.87538635,40.08642873)
\curveto(246.04539394,40.01641875)(246.22039376,39.95141882)(246.40038635,39.89142873)
\curveto(246.57039341,39.83141894)(246.74039324,39.76141901)(246.91038635,39.68142873)
\curveto(246.96039302,39.66141911)(247.00539298,39.64641912)(247.04538635,39.63642873)
\curveto(247.0853929,39.62641914)(247.13039285,39.61141916)(247.18038635,39.59142873)
\curveto(247.36039262,39.51141926)(247.54539244,39.44141933)(247.73538635,39.38142873)
\curveto(247.91539207,39.33141944)(248.09539189,39.2664195)(248.27538635,39.18642873)
\curveto(248.42539156,39.11641965)(248.5803914,39.05641971)(248.74038635,39.00642873)
\curveto(248.89039109,38.95641981)(249.04039094,38.90141987)(249.19038635,38.84142873)
\curveto(249.66039032,38.64142013)(250.13538985,38.46142031)(250.61538635,38.30142873)
\curveto(251.0853889,38.14142063)(251.55038843,37.9664208)(252.01038635,37.77642873)
\curveto(252.19038779,37.69642107)(252.37038761,37.62642114)(252.55038635,37.56642873)
\curveto(252.73038725,37.50642126)(252.91038707,37.44142133)(253.09038635,37.37142873)
\curveto(253.20038678,37.32142145)(253.30538668,37.2714215)(253.40538635,37.22142873)
\curveto(253.49538649,37.18142159)(253.56038642,37.09642167)(253.60038635,36.96642873)
\curveto(253.61038637,36.94642182)(253.61538637,36.92142185)(253.61538635,36.89142873)
\curveto(253.60538638,36.8714219)(253.60538638,36.84642192)(253.61538635,36.81642873)
\curveto(253.62538636,36.78642198)(253.63038635,36.75142202)(253.63038635,36.71142873)
\curveto(253.62038636,36.6714221)(253.61538637,36.63142214)(253.61538635,36.59142873)
\lineto(253.61538635,36.29142873)
\curveto(253.61538637,36.19142258)(253.59038639,36.11142266)(253.54038635,36.05142873)
\curveto(253.49038649,35.9714228)(253.42038656,35.91142286)(253.33038635,35.87142873)
\curveto(253.23038675,35.84142293)(253.13038685,35.80142297)(253.03038635,35.75142873)
\curveto(252.83038715,35.6714231)(252.62538736,35.59142318)(252.41538635,35.51142873)
\curveto(252.19538779,35.44142333)(251.985388,35.3664234)(251.78538635,35.28642873)
\curveto(251.60538838,35.20642356)(251.42538856,35.13642363)(251.24538635,35.07642873)
\curveto(251.05538893,35.02642374)(250.87038911,34.96142381)(250.69038635,34.88142873)
\curveto(250.13038985,34.65142412)(249.56539042,34.43642433)(248.99538635,34.23642873)
\curveto(248.42539156,34.03642473)(247.86039212,33.82142495)(247.30038635,33.59142873)
\lineto(246.67038635,33.35142873)
\curveto(246.45039353,33.28142549)(246.24039374,33.20642556)(246.04038635,33.12642873)
\curveto(245.93039405,33.07642569)(245.82539416,33.03142574)(245.72538635,32.99142873)
\curveto(245.61539437,32.96142581)(245.52039446,32.91142586)(245.44038635,32.84142873)
\curveto(245.42039456,32.83142594)(245.41039457,32.82142595)(245.41038635,32.81142873)
\lineto(245.38038635,32.78142873)
\lineto(245.38038635,32.70642873)
\lineto(245.41038635,32.67642873)
\curveto(245.41039457,32.6664261)(245.41539457,32.65642611)(245.42538635,32.64642873)
\curveto(245.47539451,32.62642614)(245.53039445,32.61642615)(245.59038635,32.61642873)
\curveto(245.65039433,32.61642615)(245.71039427,32.60642616)(245.77038635,32.58642873)
\lineto(245.93538635,32.58642873)
\curveto(245.99539399,32.5664262)(246.06039392,32.56142621)(246.13038635,32.57142873)
\curveto(246.20039378,32.58142619)(246.27039371,32.58642618)(246.34038635,32.58642873)
\lineto(247.15038635,32.58642873)
\lineto(251.71038635,32.58642873)
\lineto(252.89538635,32.58642873)
\curveto(253.00538698,32.58642618)(253.11538687,32.58142619)(253.22538635,32.57142873)
\curveto(253.33538665,32.5714262)(253.42038656,32.54642622)(253.48038635,32.49642873)
\curveto(253.56038642,32.44642632)(253.60538638,32.35642641)(253.61538635,32.22642873)
\lineto(253.61538635,31.83642873)
\lineto(253.61538635,31.64142873)
\curveto(253.61538637,31.59142718)(253.60538638,31.54142723)(253.58538635,31.49142873)
\curveto(253.54538644,31.36142741)(253.46038652,31.28642748)(253.33038635,31.26642873)
\curveto(253.20038678,31.25642751)(253.05038693,31.25142752)(252.88038635,31.25142873)
\lineto(251.14038635,31.25142873)
\lineto(245.14038635,31.25142873)
\lineto(243.73038635,31.25142873)
\curveto(243.62039636,31.25142752)(243.50539648,31.24642752)(243.38538635,31.23642873)
\curveto(243.26539672,31.23642753)(243.17039681,31.26142751)(243.10038635,31.31142873)
\curveto(243.04039694,31.35142742)(242.99039699,31.42642734)(242.95038635,31.53642873)
\curveto(242.94039704,31.55642721)(242.94039704,31.57642719)(242.95038635,31.59642873)
\curveto(242.95039703,31.62642714)(242.94539704,31.65142712)(242.93538635,31.67142873)
}
}
{
\newrgbcolor{curcolor}{0 0 0}
\pscustom[linestyle=none,fillstyle=solid,fillcolor=curcolor]
{
\newpath
\moveto(253.06038635,50.87353811)
\curveto(253.22038676,50.90353028)(253.35538663,50.88853029)(253.46538635,50.82853811)
\curveto(253.56538642,50.76853041)(253.64038634,50.68853049)(253.69038635,50.58853811)
\curveto(253.71038627,50.53853064)(253.72038626,50.4835307)(253.72038635,50.42353811)
\curveto(253.72038626,50.37353081)(253.73038625,50.31853086)(253.75038635,50.25853811)
\curveto(253.80038618,50.03853114)(253.7853862,49.81853136)(253.70538635,49.59853811)
\curveto(253.63538635,49.38853179)(253.54538644,49.24353194)(253.43538635,49.16353811)
\curveto(253.36538662,49.11353207)(253.2853867,49.06853211)(253.19538635,49.02853811)
\curveto(253.09538689,48.98853219)(253.01538697,48.93853224)(252.95538635,48.87853811)
\curveto(252.93538705,48.85853232)(252.91538707,48.83353235)(252.89538635,48.80353811)
\curveto(252.87538711,48.7835324)(252.87038711,48.75353243)(252.88038635,48.71353811)
\curveto(252.91038707,48.60353258)(252.96538702,48.49853268)(253.04538635,48.39853811)
\curveto(253.12538686,48.30853287)(253.19538679,48.21853296)(253.25538635,48.12853811)
\curveto(253.33538665,47.99853318)(253.41038657,47.85853332)(253.48038635,47.70853811)
\curveto(253.54038644,47.55853362)(253.59538639,47.39853378)(253.64538635,47.22853811)
\curveto(253.67538631,47.12853405)(253.69538629,47.01853416)(253.70538635,46.89853811)
\curveto(253.71538627,46.78853439)(253.73038625,46.6785345)(253.75038635,46.56853811)
\curveto(253.76038622,46.51853466)(253.76538622,46.47353471)(253.76538635,46.43353811)
\lineto(253.76538635,46.32853811)
\curveto(253.7853862,46.21853496)(253.7853862,46.11353507)(253.76538635,46.01353811)
\lineto(253.76538635,45.87853811)
\curveto(253.75538623,45.82853535)(253.75038623,45.7785354)(253.75038635,45.72853811)
\curveto(253.75038623,45.6785355)(253.74038624,45.63353555)(253.72038635,45.59353811)
\curveto(253.71038627,45.55353563)(253.70538628,45.51853566)(253.70538635,45.48853811)
\curveto(253.71538627,45.46853571)(253.71538627,45.44353574)(253.70538635,45.41353811)
\lineto(253.64538635,45.17353811)
\curveto(253.63538635,45.09353609)(253.61538637,45.01853616)(253.58538635,44.94853811)
\curveto(253.45538653,44.64853653)(253.31038667,44.40353678)(253.15038635,44.21353811)
\curveto(252.980387,44.03353715)(252.74538724,43.8835373)(252.44538635,43.76353811)
\curveto(252.22538776,43.67353751)(251.96038802,43.62853755)(251.65038635,43.62853811)
\lineto(251.33538635,43.62853811)
\curveto(251.2853887,43.63853754)(251.23538875,43.64353754)(251.18538635,43.64353811)
\lineto(251.00538635,43.67353811)
\lineto(250.67538635,43.79353811)
\curveto(250.56538942,43.83353735)(250.46538952,43.8835373)(250.37538635,43.94353811)
\curveto(250.0853899,44.12353706)(249.87039011,44.36853681)(249.73038635,44.67853811)
\curveto(249.59039039,44.98853619)(249.46539052,45.32853585)(249.35538635,45.69853811)
\curveto(249.31539067,45.83853534)(249.2853907,45.9835352)(249.26538635,46.13353811)
\curveto(249.24539074,46.2835349)(249.22039076,46.43353475)(249.19038635,46.58353811)
\curveto(249.17039081,46.65353453)(249.16039082,46.71853446)(249.16038635,46.77853811)
\curveto(249.16039082,46.84853433)(249.15039083,46.92353426)(249.13038635,47.00353811)
\curveto(249.11039087,47.07353411)(249.10039088,47.14353404)(249.10038635,47.21353811)
\curveto(249.09039089,47.2835339)(249.07539091,47.35853382)(249.05538635,47.43853811)
\curveto(248.99539099,47.68853349)(248.94539104,47.92353326)(248.90538635,48.14353811)
\curveto(248.85539113,48.36353282)(248.74039124,48.53853264)(248.56038635,48.66853811)
\curveto(248.4803915,48.72853245)(248.3803916,48.7785324)(248.26038635,48.81853811)
\curveto(248.13039185,48.85853232)(247.99039199,48.85853232)(247.84038635,48.81853811)
\curveto(247.60039238,48.75853242)(247.41039257,48.66853251)(247.27038635,48.54853811)
\curveto(247.13039285,48.43853274)(247.02039296,48.2785329)(246.94038635,48.06853811)
\curveto(246.89039309,47.94853323)(246.85539313,47.80353338)(246.83538635,47.63353811)
\curveto(246.81539317,47.47353371)(246.80539318,47.30353388)(246.80538635,47.12353811)
\curveto(246.80539318,46.94353424)(246.81539317,46.76853441)(246.83538635,46.59853811)
\curveto(246.85539313,46.42853475)(246.8853931,46.2835349)(246.92538635,46.16353811)
\curveto(246.985393,45.99353519)(247.07039291,45.82853535)(247.18038635,45.66853811)
\curveto(247.24039274,45.58853559)(247.32039266,45.51353567)(247.42038635,45.44353811)
\curveto(247.51039247,45.3835358)(247.61039237,45.32853585)(247.72038635,45.27853811)
\curveto(247.80039218,45.24853593)(247.8853921,45.21853596)(247.97538635,45.18853811)
\curveto(248.06539192,45.16853601)(248.13539185,45.12353606)(248.18538635,45.05353811)
\curveto(248.21539177,45.01353617)(248.24039174,44.94353624)(248.26038635,44.84353811)
\curveto(248.27039171,44.75353643)(248.27539171,44.65853652)(248.27538635,44.55853811)
\curveto(248.27539171,44.45853672)(248.27039171,44.35853682)(248.26038635,44.25853811)
\curveto(248.24039174,44.16853701)(248.21539177,44.10353708)(248.18538635,44.06353811)
\curveto(248.15539183,44.02353716)(248.10539188,43.99353719)(248.03538635,43.97353811)
\curveto(247.96539202,43.95353723)(247.89039209,43.95353723)(247.81038635,43.97353811)
\curveto(247.6803923,44.00353718)(247.56039242,44.03353715)(247.45038635,44.06353811)
\curveto(247.33039265,44.10353708)(247.21539277,44.14853703)(247.10538635,44.19853811)
\curveto(246.75539323,44.38853679)(246.4853935,44.62853655)(246.29538635,44.91853811)
\curveto(246.09539389,45.20853597)(245.93539405,45.56853561)(245.81538635,45.99853811)
\curveto(245.79539419,46.09853508)(245.7803942,46.19853498)(245.77038635,46.29853811)
\curveto(245.76039422,46.40853477)(245.74539424,46.51853466)(245.72538635,46.62853811)
\curveto(245.71539427,46.66853451)(245.71539427,46.73353445)(245.72538635,46.82353811)
\curveto(245.72539426,46.91353427)(245.71539427,46.96853421)(245.69538635,46.98853811)
\curveto(245.6853943,47.68853349)(245.76539422,48.29853288)(245.93538635,48.81853811)
\curveto(246.10539388,49.33853184)(246.43039355,49.70353148)(246.91038635,49.91353811)
\curveto(247.11039287,50.00353118)(247.34539264,50.05353113)(247.61538635,50.06353811)
\curveto(247.87539211,50.0835311)(248.15039183,50.09353109)(248.44038635,50.09353811)
\lineto(251.75538635,50.09353811)
\curveto(251.89538809,50.09353109)(252.03038795,50.09853108)(252.16038635,50.10853811)
\curveto(252.29038769,50.11853106)(252.39538759,50.14853103)(252.47538635,50.19853811)
\curveto(252.54538744,50.24853093)(252.59538739,50.31353087)(252.62538635,50.39353811)
\curveto(252.66538732,50.4835307)(252.69538729,50.56853061)(252.71538635,50.64853811)
\curveto(252.72538726,50.72853045)(252.77038721,50.78853039)(252.85038635,50.82853811)
\curveto(252.8803871,50.84853033)(252.91038707,50.85853032)(252.94038635,50.85853811)
\curveto(252.97038701,50.85853032)(253.01038697,50.86353032)(253.06038635,50.87353811)
\moveto(251.39538635,48.72853811)
\curveto(251.25538873,48.78853239)(251.09538889,48.81853236)(250.91538635,48.81853811)
\curveto(250.72538926,48.82853235)(250.53038945,48.83353235)(250.33038635,48.83353811)
\curveto(250.22038976,48.83353235)(250.12038986,48.82853235)(250.03038635,48.81853811)
\curveto(249.94039004,48.80853237)(249.87039011,48.76853241)(249.82038635,48.69853811)
\curveto(249.80039018,48.66853251)(249.79039019,48.59853258)(249.79038635,48.48853811)
\curveto(249.81039017,48.46853271)(249.82039016,48.43353275)(249.82038635,48.38353811)
\curveto(249.82039016,48.33353285)(249.83039015,48.28853289)(249.85038635,48.24853811)
\curveto(249.87039011,48.16853301)(249.89039009,48.0785331)(249.91038635,47.97853811)
\lineto(249.97038635,47.67853811)
\curveto(249.97039001,47.64853353)(249.97539001,47.61353357)(249.98538635,47.57353811)
\lineto(249.98538635,47.46853811)
\curveto(250.02538996,47.31853386)(250.05038993,47.15353403)(250.06038635,46.97353811)
\curveto(250.06038992,46.80353438)(250.0803899,46.64353454)(250.12038635,46.49353811)
\curveto(250.14038984,46.41353477)(250.16038982,46.33853484)(250.18038635,46.26853811)
\curveto(250.19038979,46.20853497)(250.20538978,46.13853504)(250.22538635,46.05853811)
\curveto(250.27538971,45.89853528)(250.34038964,45.74853543)(250.42038635,45.60853811)
\curveto(250.49038949,45.46853571)(250.5803894,45.34853583)(250.69038635,45.24853811)
\curveto(250.80038918,45.14853603)(250.93538905,45.07353611)(251.09538635,45.02353811)
\curveto(251.24538874,44.97353621)(251.43038855,44.95353623)(251.65038635,44.96353811)
\curveto(251.75038823,44.96353622)(251.84538814,44.9785362)(251.93538635,45.00853811)
\curveto(252.01538797,45.04853613)(252.09038789,45.09353609)(252.16038635,45.14353811)
\curveto(252.27038771,45.22353596)(252.36538762,45.32853585)(252.44538635,45.45853811)
\curveto(252.51538747,45.58853559)(252.57538741,45.72853545)(252.62538635,45.87853811)
\curveto(252.63538735,45.92853525)(252.64038734,45.9785352)(252.64038635,46.02853811)
\curveto(252.64038734,46.0785351)(252.64538734,46.12853505)(252.65538635,46.17853811)
\curveto(252.67538731,46.24853493)(252.69038729,46.33353485)(252.70038635,46.43353811)
\curveto(252.70038728,46.54353464)(252.69038729,46.63353455)(252.67038635,46.70353811)
\curveto(252.65038733,46.76353442)(252.64538734,46.82353436)(252.65538635,46.88353811)
\curveto(252.65538733,46.94353424)(252.64538734,47.00353418)(252.62538635,47.06353811)
\curveto(252.60538738,47.14353404)(252.59038739,47.21853396)(252.58038635,47.28853811)
\curveto(252.57038741,47.36853381)(252.55038743,47.44353374)(252.52038635,47.51353811)
\curveto(252.40038758,47.80353338)(252.25538773,48.04853313)(252.08538635,48.24853811)
\curveto(251.91538807,48.45853272)(251.6853883,48.61853256)(251.39538635,48.72853811)
}
}
{
\newrgbcolor{curcolor}{0 0 0}
\pscustom[linestyle=none,fillstyle=solid,fillcolor=curcolor]
{
\newpath
\moveto(245.71038635,55.69017873)
\curveto(245.71039427,55.92017394)(245.77039421,56.05017381)(245.89038635,56.08017873)
\curveto(246.00039398,56.11017375)(246.16539382,56.12517374)(246.38538635,56.12517873)
\lineto(246.67038635,56.12517873)
\curveto(246.76039322,56.12517374)(246.83539315,56.10017376)(246.89538635,56.05017873)
\curveto(246.97539301,55.99017387)(247.02039296,55.90517396)(247.03038635,55.79517873)
\curveto(247.03039295,55.68517418)(247.04539294,55.57517429)(247.07538635,55.46517873)
\curveto(247.10539288,55.32517454)(247.13539285,55.19017467)(247.16538635,55.06017873)
\curveto(247.19539279,54.94017492)(247.23539275,54.82517504)(247.28538635,54.71517873)
\curveto(247.41539257,54.42517544)(247.59539239,54.19017567)(247.82538635,54.01017873)
\curveto(248.04539194,53.83017603)(248.30039168,53.67517619)(248.59038635,53.54517873)
\curveto(248.70039128,53.50517636)(248.81539117,53.47517639)(248.93538635,53.45517873)
\curveto(249.04539094,53.43517643)(249.16039082,53.41017645)(249.28038635,53.38017873)
\curveto(249.33039065,53.37017649)(249.3803906,53.3651765)(249.43038635,53.36517873)
\curveto(249.4803905,53.37517649)(249.53039045,53.37517649)(249.58038635,53.36517873)
\curveto(249.70039028,53.33517653)(249.84039014,53.32017654)(250.00038635,53.32017873)
\curveto(250.15038983,53.33017653)(250.29538969,53.33517653)(250.43538635,53.33517873)
\lineto(252.28038635,53.33517873)
\lineto(252.62538635,53.33517873)
\curveto(252.74538724,53.33517653)(252.86038712,53.33017653)(252.97038635,53.32017873)
\curveto(253.0803869,53.31017655)(253.17538681,53.30517656)(253.25538635,53.30517873)
\curveto(253.33538665,53.31517655)(253.40538658,53.29517657)(253.46538635,53.24517873)
\curveto(253.53538645,53.19517667)(253.57538641,53.11517675)(253.58538635,53.00517873)
\curveto(253.59538639,52.90517696)(253.60038638,52.79517707)(253.60038635,52.67517873)
\lineto(253.60038635,52.40517873)
\curveto(253.5803864,52.35517751)(253.56538642,52.30517756)(253.55538635,52.25517873)
\curveto(253.53538645,52.21517765)(253.51038647,52.18517768)(253.48038635,52.16517873)
\curveto(253.41038657,52.11517775)(253.32538666,52.08517778)(253.22538635,52.07517873)
\lineto(252.89538635,52.07517873)
\lineto(251.74038635,52.07517873)
\lineto(247.58538635,52.07517873)
\lineto(246.55038635,52.07517873)
\lineto(246.25038635,52.07517873)
\curveto(246.15039383,52.08517778)(246.06539392,52.11517775)(245.99538635,52.16517873)
\curveto(245.95539403,52.19517767)(245.92539406,52.24517762)(245.90538635,52.31517873)
\curveto(245.8853941,52.39517747)(245.87539411,52.48017738)(245.87538635,52.57017873)
\curveto(245.86539412,52.6601772)(245.86539412,52.75017711)(245.87538635,52.84017873)
\curveto(245.8853941,52.93017693)(245.90039408,53.00017686)(245.92038635,53.05017873)
\curveto(245.95039403,53.13017673)(246.01039397,53.18017668)(246.10038635,53.20017873)
\curveto(246.1803938,53.23017663)(246.27039371,53.24517662)(246.37038635,53.24517873)
\lineto(246.67038635,53.24517873)
\curveto(246.77039321,53.24517662)(246.86039312,53.2651766)(246.94038635,53.30517873)
\curveto(246.96039302,53.31517655)(246.97539301,53.32517654)(246.98538635,53.33517873)
\lineto(247.03038635,53.38017873)
\curveto(247.03039295,53.49017637)(246.985393,53.58017628)(246.89538635,53.65017873)
\curveto(246.79539319,53.72017614)(246.71539327,53.78017608)(246.65538635,53.83017873)
\lineto(246.56538635,53.92017873)
\curveto(246.45539353,54.01017585)(246.34039364,54.13517573)(246.22038635,54.29517873)
\curveto(246.10039388,54.45517541)(246.01039397,54.60517526)(245.95038635,54.74517873)
\curveto(245.90039408,54.83517503)(245.86539412,54.93017493)(245.84538635,55.03017873)
\curveto(245.81539417,55.13017473)(245.7853942,55.23517463)(245.75538635,55.34517873)
\curveto(245.74539424,55.40517446)(245.74039424,55.4651744)(245.74038635,55.52517873)
\curveto(245.73039425,55.58517428)(245.72039426,55.64017422)(245.71038635,55.69017873)
}
}
{
\newrgbcolor{curcolor}{0 0 0}
\pscustom[linestyle=none,fillstyle=solid,fillcolor=curcolor]
{
}
}
{
\newrgbcolor{curcolor}{0 0 0}
\pscustom[linestyle=none,fillstyle=solid,fillcolor=curcolor]
{
\newpath
\moveto(243.01038635,64.24510061)
\curveto(243.00039698,64.93509597)(243.12039686,65.53509537)(243.37038635,66.04510061)
\curveto(243.62039636,66.56509434)(243.95539603,66.96009395)(244.37538635,67.23010061)
\curveto(244.45539553,67.28009363)(244.54539544,67.32509358)(244.64538635,67.36510061)
\curveto(244.73539525,67.4050935)(244.83039515,67.45009346)(244.93038635,67.50010061)
\curveto(245.03039495,67.54009337)(245.13039485,67.57009334)(245.23038635,67.59010061)
\curveto(245.33039465,67.6100933)(245.43539455,67.63009328)(245.54538635,67.65010061)
\curveto(245.59539439,67.67009324)(245.64039434,67.67509323)(245.68038635,67.66510061)
\curveto(245.72039426,67.65509325)(245.76539422,67.66009325)(245.81538635,67.68010061)
\curveto(245.86539412,67.69009322)(245.95039403,67.69509321)(246.07038635,67.69510061)
\curveto(246.1803938,67.69509321)(246.26539372,67.69009322)(246.32538635,67.68010061)
\curveto(246.3853936,67.66009325)(246.44539354,67.65009326)(246.50538635,67.65010061)
\curveto(246.56539342,67.66009325)(246.62539336,67.65509325)(246.68538635,67.63510061)
\curveto(246.82539316,67.59509331)(246.96039302,67.56009335)(247.09038635,67.53010061)
\curveto(247.22039276,67.50009341)(247.34539264,67.46009345)(247.46538635,67.41010061)
\curveto(247.60539238,67.35009356)(247.73039225,67.28009363)(247.84038635,67.20010061)
\curveto(247.95039203,67.13009378)(248.06039192,67.05509385)(248.17038635,66.97510061)
\lineto(248.23038635,66.91510061)
\curveto(248.25039173,66.905094)(248.27039171,66.89009402)(248.29038635,66.87010061)
\curveto(248.45039153,66.75009416)(248.59539139,66.61509429)(248.72538635,66.46510061)
\curveto(248.85539113,66.31509459)(248.980391,66.15509475)(249.10038635,65.98510061)
\curveto(249.32039066,65.67509523)(249.52539046,65.38009553)(249.71538635,65.10010061)
\curveto(249.85539013,64.87009604)(249.99038999,64.64009627)(250.12038635,64.41010061)
\curveto(250.25038973,64.19009672)(250.3853896,63.97009694)(250.52538635,63.75010061)
\curveto(250.69538929,63.50009741)(250.87538911,63.26009765)(251.06538635,63.03010061)
\curveto(251.25538873,62.8100981)(251.4803885,62.62009829)(251.74038635,62.46010061)
\curveto(251.80038818,62.42009849)(251.86038812,62.38509852)(251.92038635,62.35510061)
\curveto(251.97038801,62.32509858)(252.03538795,62.29509861)(252.11538635,62.26510061)
\curveto(252.1853878,62.24509866)(252.24538774,62.24009867)(252.29538635,62.25010061)
\curveto(252.36538762,62.27009864)(252.42038756,62.3050986)(252.46038635,62.35510061)
\curveto(252.49038749,62.4050985)(252.51038747,62.46509844)(252.52038635,62.53510061)
\lineto(252.52038635,62.77510061)
\lineto(252.52038635,63.52510061)
\lineto(252.52038635,66.33010061)
\lineto(252.52038635,66.99010061)
\curveto(252.52038746,67.08009383)(252.52538746,67.16509374)(252.53538635,67.24510061)
\curveto(252.53538745,67.32509358)(252.55538743,67.39009352)(252.59538635,67.44010061)
\curveto(252.63538735,67.49009342)(252.71038727,67.53009338)(252.82038635,67.56010061)
\curveto(252.92038706,67.60009331)(253.02038696,67.6100933)(253.12038635,67.59010061)
\lineto(253.25538635,67.59010061)
\curveto(253.32538666,67.57009334)(253.3853866,67.55009336)(253.43538635,67.53010061)
\curveto(253.4853865,67.5100934)(253.52538646,67.47509343)(253.55538635,67.42510061)
\curveto(253.59538639,67.37509353)(253.61538637,67.3050936)(253.61538635,67.21510061)
\lineto(253.61538635,66.94510061)
\lineto(253.61538635,66.04510061)
\lineto(253.61538635,62.53510061)
\lineto(253.61538635,61.47010061)
\curveto(253.61538637,61.39009952)(253.62038636,61.30009961)(253.63038635,61.20010061)
\curveto(253.63038635,61.10009981)(253.62038636,61.01509989)(253.60038635,60.94510061)
\curveto(253.53038645,60.73510017)(253.35038663,60.67010024)(253.06038635,60.75010061)
\curveto(253.02038696,60.76010015)(252.985387,60.76010015)(252.95538635,60.75010061)
\curveto(252.91538707,60.75010016)(252.87038711,60.76010015)(252.82038635,60.78010061)
\curveto(252.74038724,60.80010011)(252.65538733,60.82010009)(252.56538635,60.84010061)
\curveto(252.47538751,60.86010005)(252.39038759,60.88510002)(252.31038635,60.91510061)
\curveto(251.82038816,61.07509983)(251.40538858,61.27509963)(251.06538635,61.51510061)
\curveto(250.81538917,61.69509921)(250.59038939,61.90009901)(250.39038635,62.13010061)
\curveto(250.1803898,62.36009855)(249.98539,62.60009831)(249.80538635,62.85010061)
\curveto(249.62539036,63.1100978)(249.45539053,63.37509753)(249.29538635,63.64510061)
\curveto(249.12539086,63.92509698)(248.95039103,64.19509671)(248.77038635,64.45510061)
\curveto(248.69039129,64.56509634)(248.61539137,64.67009624)(248.54538635,64.77010061)
\curveto(248.47539151,64.88009603)(248.40039158,64.99009592)(248.32038635,65.10010061)
\curveto(248.29039169,65.14009577)(248.26039172,65.17509573)(248.23038635,65.20510061)
\curveto(248.19039179,65.24509566)(248.16039182,65.28509562)(248.14038635,65.32510061)
\curveto(248.03039195,65.46509544)(247.90539208,65.59009532)(247.76538635,65.70010061)
\curveto(247.73539225,65.72009519)(247.71039227,65.74509516)(247.69038635,65.77510061)
\curveto(247.66039232,65.8050951)(247.63039235,65.83009508)(247.60038635,65.85010061)
\curveto(247.50039248,65.93009498)(247.40039258,65.99509491)(247.30038635,66.04510061)
\curveto(247.20039278,66.1050948)(247.09039289,66.16009475)(246.97038635,66.21010061)
\curveto(246.90039308,66.24009467)(246.82539316,66.26009465)(246.74538635,66.27010061)
\lineto(246.50538635,66.33010061)
\lineto(246.41538635,66.33010061)
\curveto(246.3853936,66.34009457)(246.35539363,66.34509456)(246.32538635,66.34510061)
\curveto(246.25539373,66.36509454)(246.16039382,66.37009454)(246.04038635,66.36010061)
\curveto(245.91039407,66.36009455)(245.81039417,66.35009456)(245.74038635,66.33010061)
\curveto(245.66039432,66.3100946)(245.5853944,66.29009462)(245.51538635,66.27010061)
\curveto(245.43539455,66.26009465)(245.35539463,66.24009467)(245.27538635,66.21010061)
\curveto(245.03539495,66.10009481)(244.83539515,65.95009496)(244.67538635,65.76010061)
\curveto(244.50539548,65.58009533)(244.36539562,65.36009555)(244.25538635,65.10010061)
\curveto(244.23539575,65.03009588)(244.22039576,64.96009595)(244.21038635,64.89010061)
\curveto(244.19039579,64.82009609)(244.17039581,64.74509616)(244.15038635,64.66510061)
\curveto(244.13039585,64.58509632)(244.12039586,64.47509643)(244.12038635,64.33510061)
\curveto(244.12039586,64.2050967)(244.13039585,64.10009681)(244.15038635,64.02010061)
\curveto(244.16039582,63.96009695)(244.16539582,63.905097)(244.16538635,63.85510061)
\curveto(244.16539582,63.8050971)(244.17539581,63.75509715)(244.19538635,63.70510061)
\curveto(244.23539575,63.6050973)(244.27539571,63.5100974)(244.31538635,63.42010061)
\curveto(244.35539563,63.34009757)(244.40039558,63.26009765)(244.45038635,63.18010061)
\curveto(244.47039551,63.15009776)(244.49539549,63.12009779)(244.52538635,63.09010061)
\curveto(244.55539543,63.07009784)(244.5803954,63.04509786)(244.60038635,63.01510061)
\lineto(244.67538635,62.94010061)
\curveto(244.69539529,62.910098)(244.71539527,62.88509802)(244.73538635,62.86510061)
\lineto(244.94538635,62.71510061)
\curveto(245.00539498,62.67509823)(245.07039491,62.63009828)(245.14038635,62.58010061)
\curveto(245.23039475,62.52009839)(245.33539465,62.47009844)(245.45538635,62.43010061)
\curveto(245.56539442,62.40009851)(245.67539431,62.36509854)(245.78538635,62.32510061)
\curveto(245.89539409,62.28509862)(246.04039394,62.26009865)(246.22038635,62.25010061)
\curveto(246.39039359,62.24009867)(246.51539347,62.2100987)(246.59538635,62.16010061)
\curveto(246.67539331,62.1100988)(246.72039326,62.03509887)(246.73038635,61.93510061)
\curveto(246.74039324,61.83509907)(246.74539324,61.72509918)(246.74538635,61.60510061)
\curveto(246.74539324,61.56509934)(246.75039323,61.52509938)(246.76038635,61.48510061)
\curveto(246.76039322,61.44509946)(246.75539323,61.4100995)(246.74538635,61.38010061)
\curveto(246.72539326,61.33009958)(246.71539327,61.28009963)(246.71538635,61.23010061)
\curveto(246.71539327,61.19009972)(246.70539328,61.15009976)(246.68538635,61.11010061)
\curveto(246.62539336,61.02009989)(246.49039349,60.97509993)(246.28038635,60.97510061)
\lineto(246.16038635,60.97510061)
\curveto(246.10039388,60.98509992)(246.04039394,60.99009992)(245.98038635,60.99010061)
\curveto(245.91039407,61.00009991)(245.84539414,61.0100999)(245.78538635,61.02010061)
\curveto(245.67539431,61.04009987)(245.57539441,61.06009985)(245.48538635,61.08010061)
\curveto(245.3853946,61.10009981)(245.29039469,61.13009978)(245.20038635,61.17010061)
\curveto(245.13039485,61.19009972)(245.07039491,61.2100997)(245.02038635,61.23010061)
\lineto(244.84038635,61.29010061)
\curveto(244.5803954,61.4100995)(244.33539565,61.56509934)(244.10538635,61.75510061)
\curveto(243.87539611,61.95509895)(243.69039629,62.17009874)(243.55038635,62.40010061)
\curveto(243.47039651,62.5100984)(243.40539658,62.62509828)(243.35538635,62.74510061)
\lineto(243.20538635,63.13510061)
\curveto(243.15539683,63.24509766)(243.12539686,63.36009755)(243.11538635,63.48010061)
\curveto(243.09539689,63.60009731)(243.07039691,63.72509718)(243.04038635,63.85510061)
\curveto(243.04039694,63.92509698)(243.04039694,63.99009692)(243.04038635,64.05010061)
\curveto(243.03039695,64.1100968)(243.02039696,64.17509673)(243.01038635,64.24510061)
}
}
{
\newrgbcolor{curcolor}{0 0 0}
\pscustom[linestyle=none,fillstyle=solid,fillcolor=curcolor]
{
\newpath
\moveto(249.29538635,76.34470998)
\curveto(249.41539057,76.37470226)(249.55539043,76.39970223)(249.71538635,76.41970998)
\curveto(249.87539011,76.43970219)(250.04038994,76.44970218)(250.21038635,76.44970998)
\curveto(250.3803896,76.44970218)(250.54538944,76.43970219)(250.70538635,76.41970998)
\curveto(250.86538912,76.39970223)(251.00538898,76.37470226)(251.12538635,76.34470998)
\curveto(251.26538872,76.30470233)(251.39038859,76.26970236)(251.50038635,76.23970998)
\curveto(251.61038837,76.20970242)(251.72038826,76.16970246)(251.83038635,76.11970998)
\curveto(252.47038751,75.84970278)(252.95538703,75.4347032)(253.28538635,74.87470998)
\curveto(253.34538664,74.79470384)(253.39538659,74.70970392)(253.43538635,74.61970998)
\curveto(253.46538652,74.5297041)(253.50038648,74.4297042)(253.54038635,74.31970998)
\curveto(253.59038639,74.20970442)(253.62538636,74.08970454)(253.64538635,73.95970998)
\curveto(253.67538631,73.83970479)(253.70538628,73.70970492)(253.73538635,73.56970998)
\curveto(253.75538623,73.50970512)(253.76038622,73.44970518)(253.75038635,73.38970998)
\curveto(253.74038624,73.33970529)(253.74538624,73.27970535)(253.76538635,73.20970998)
\curveto(253.77538621,73.18970544)(253.77538621,73.16470547)(253.76538635,73.13470998)
\curveto(253.76538622,73.10470553)(253.77038621,73.07970555)(253.78038635,73.05970998)
\lineto(253.78038635,72.90970998)
\curveto(253.79038619,72.83970579)(253.79038619,72.78970584)(253.78038635,72.75970998)
\curveto(253.77038621,72.71970591)(253.76538622,72.67470596)(253.76538635,72.62470998)
\curveto(253.77538621,72.58470605)(253.77538621,72.54470609)(253.76538635,72.50470998)
\curveto(253.74538624,72.41470622)(253.73038625,72.32470631)(253.72038635,72.23470998)
\curveto(253.72038626,72.14470649)(253.71038627,72.05470658)(253.69038635,71.96470998)
\curveto(253.66038632,71.87470676)(253.63538635,71.78470685)(253.61538635,71.69470998)
\curveto(253.59538639,71.60470703)(253.56538642,71.51970711)(253.52538635,71.43970998)
\curveto(253.41538657,71.19970743)(253.2853867,70.97470766)(253.13538635,70.76470998)
\curveto(252.97538701,70.55470808)(252.79538719,70.37470826)(252.59538635,70.22470998)
\curveto(252.42538756,70.10470853)(252.25038773,69.99970863)(252.07038635,69.90970998)
\curveto(251.89038809,69.81970881)(251.70038828,69.7297089)(251.50038635,69.63970998)
\curveto(251.40038858,69.59970903)(251.30038868,69.56470907)(251.20038635,69.53470998)
\curveto(251.09038889,69.51470912)(250.980389,69.48970914)(250.87038635,69.45970998)
\curveto(250.73038925,69.41970921)(250.59038939,69.39470924)(250.45038635,69.38470998)
\curveto(250.31038967,69.37470926)(250.17038981,69.35470928)(250.03038635,69.32470998)
\curveto(249.92039006,69.31470932)(249.82039016,69.30470933)(249.73038635,69.29470998)
\curveto(249.63039035,69.29470934)(249.53039045,69.28470935)(249.43038635,69.26470998)
\lineto(249.34038635,69.26470998)
\curveto(249.31039067,69.27470936)(249.2853907,69.27470936)(249.26538635,69.26470998)
\lineto(249.05538635,69.26470998)
\curveto(248.99539099,69.24470939)(248.93039105,69.2347094)(248.86038635,69.23470998)
\curveto(248.7803912,69.24470939)(248.70539128,69.24970938)(248.63538635,69.24970998)
\lineto(248.48538635,69.24970998)
\curveto(248.43539155,69.24970938)(248.3853916,69.25470938)(248.33538635,69.26470998)
\lineto(247.96038635,69.26470998)
\curveto(247.93039205,69.27470936)(247.89539209,69.27470936)(247.85538635,69.26470998)
\curveto(247.81539217,69.26470937)(247.77539221,69.26970936)(247.73538635,69.27970998)
\curveto(247.62539236,69.29970933)(247.51539247,69.31470932)(247.40538635,69.32470998)
\curveto(247.2853927,69.3347093)(247.17039281,69.34470929)(247.06038635,69.35470998)
\curveto(246.91039307,69.39470924)(246.76539322,69.41970921)(246.62538635,69.42970998)
\curveto(246.47539351,69.44970918)(246.33039365,69.47970915)(246.19038635,69.51970998)
\curveto(245.89039409,69.60970902)(245.60539438,69.70470893)(245.33538635,69.80470998)
\curveto(245.06539492,69.90470873)(244.81539517,70.0297086)(244.58538635,70.17970998)
\curveto(244.26539572,70.37970825)(243.985396,70.62470801)(243.74538635,70.91470998)
\curveto(243.50539648,71.20470743)(243.32039666,71.54470709)(243.19038635,71.93470998)
\curveto(243.15039683,72.04470659)(243.12539686,72.15470648)(243.11538635,72.26470998)
\curveto(243.09539689,72.38470625)(243.07039691,72.50470613)(243.04038635,72.62470998)
\curveto(243.03039695,72.69470594)(243.02539696,72.75970587)(243.02538635,72.81970998)
\curveto(243.02539696,72.87970575)(243.02039696,72.94470569)(243.01038635,73.01470998)
\curveto(242.99039699,73.71470492)(243.10539688,74.28970434)(243.35538635,74.73970998)
\curveto(243.60539638,75.18970344)(243.95539603,75.5347031)(244.40538635,75.77470998)
\curveto(244.63539535,75.88470275)(244.91039507,75.98470265)(245.23038635,76.07470998)
\curveto(245.30039468,76.09470254)(245.37539461,76.09470254)(245.45538635,76.07470998)
\curveto(245.52539446,76.06470257)(245.57539441,76.03970259)(245.60538635,75.99970998)
\curveto(245.63539435,75.96970266)(245.66039432,75.90970272)(245.68038635,75.81970998)
\curveto(245.69039429,75.7297029)(245.70039428,75.629703)(245.71038635,75.51970998)
\curveto(245.71039427,75.41970321)(245.70539428,75.31970331)(245.69538635,75.21970998)
\curveto(245.6853943,75.1297035)(245.66539432,75.06470357)(245.63538635,75.02470998)
\curveto(245.56539442,74.91470372)(245.45539453,74.8347038)(245.30538635,74.78470998)
\curveto(245.15539483,74.74470389)(245.02539496,74.68970394)(244.91538635,74.61970998)
\curveto(244.60539538,74.4297042)(244.37539561,74.14970448)(244.22538635,73.77970998)
\curveto(244.19539579,73.70970492)(244.17539581,73.634705)(244.16538635,73.55470998)
\curveto(244.15539583,73.48470515)(244.14039584,73.40970522)(244.12038635,73.32970998)
\curveto(244.11039587,73.27970535)(244.10539588,73.20970542)(244.10538635,73.11970998)
\curveto(244.10539588,73.03970559)(244.11039587,72.97470566)(244.12038635,72.92470998)
\curveto(244.14039584,72.88470575)(244.14539584,72.84970578)(244.13538635,72.81970998)
\curveto(244.12539586,72.78970584)(244.12539586,72.75470588)(244.13538635,72.71470998)
\lineto(244.19538635,72.47470998)
\curveto(244.21539577,72.40470623)(244.24039574,72.3347063)(244.27038635,72.26470998)
\curveto(244.43039555,71.88470675)(244.64039534,71.59470704)(244.90038635,71.39470998)
\curveto(245.16039482,71.20470743)(245.47539451,71.0297076)(245.84538635,70.86970998)
\curveto(245.92539406,70.83970779)(246.00539398,70.81470782)(246.08538635,70.79470998)
\curveto(246.16539382,70.78470785)(246.24539374,70.76470787)(246.32538635,70.73470998)
\curveto(246.43539355,70.70470793)(246.55039343,70.67970795)(246.67038635,70.65970998)
\curveto(246.79039319,70.64970798)(246.91039307,70.629708)(247.03038635,70.59970998)
\curveto(247.0803929,70.57970805)(247.13039285,70.56970806)(247.18038635,70.56970998)
\curveto(247.23039275,70.57970805)(247.2803927,70.57470806)(247.33038635,70.55470998)
\curveto(247.39039259,70.54470809)(247.47039251,70.54470809)(247.57038635,70.55470998)
\curveto(247.66039232,70.56470807)(247.71539227,70.57970805)(247.73538635,70.59970998)
\curveto(247.77539221,70.61970801)(247.79539219,70.64970798)(247.79538635,70.68970998)
\curveto(247.79539219,70.73970789)(247.7853922,70.78470785)(247.76538635,70.82470998)
\curveto(247.72539226,70.89470774)(247.6803923,70.95470768)(247.63038635,71.00470998)
\curveto(247.5803924,71.05470758)(247.53039245,71.11470752)(247.48038635,71.18470998)
\lineto(247.42038635,71.24470998)
\curveto(247.39039259,71.27470736)(247.36539262,71.30470733)(247.34538635,71.33470998)
\curveto(247.1853928,71.56470707)(247.05039293,71.83970679)(246.94038635,72.15970998)
\curveto(246.92039306,72.2297064)(246.90539308,72.29970633)(246.89538635,72.36970998)
\curveto(246.8853931,72.43970619)(246.87039311,72.51470612)(246.85038635,72.59470998)
\curveto(246.85039313,72.634706)(246.84539314,72.66970596)(246.83538635,72.69970998)
\curveto(246.82539316,72.7297059)(246.82539316,72.76470587)(246.83538635,72.80470998)
\curveto(246.83539315,72.85470578)(246.82539316,72.89470574)(246.80538635,72.92470998)
\lineto(246.80538635,73.08970998)
\lineto(246.80538635,73.17970998)
\curveto(246.79539319,73.2297054)(246.79539319,73.26970536)(246.80538635,73.29970998)
\curveto(246.81539317,73.34970528)(246.82039316,73.39970523)(246.82038635,73.44970998)
\curveto(246.81039317,73.50970512)(246.81039317,73.56470507)(246.82038635,73.61470998)
\curveto(246.85039313,73.72470491)(246.87039311,73.8297048)(246.88038635,73.92970998)
\curveto(246.89039309,74.03970459)(246.91539307,74.14470449)(246.95538635,74.24470998)
\curveto(247.09539289,74.66470397)(247.2803927,75.00970362)(247.51038635,75.27970998)
\curveto(247.73039225,75.54970308)(248.01539197,75.78970284)(248.36538635,75.99970998)
\curveto(248.50539148,76.07970255)(248.65539133,76.14470249)(248.81538635,76.19470998)
\curveto(248.96539102,76.24470239)(249.12539086,76.29470234)(249.29538635,76.34470998)
\moveto(250.60038635,75.09970998)
\curveto(250.55038943,75.10970352)(250.50538948,75.11470352)(250.46538635,75.11470998)
\lineto(250.31538635,75.11470998)
\curveto(250.00538998,75.11470352)(249.72039026,75.07470356)(249.46038635,74.99470998)
\curveto(249.40039058,74.97470366)(249.34539064,74.95470368)(249.29538635,74.93470998)
\curveto(249.23539075,74.92470371)(249.1803908,74.90970372)(249.13038635,74.88970998)
\curveto(248.64039134,74.66970396)(248.29039169,74.32470431)(248.08038635,73.85470998)
\curveto(248.05039193,73.77470486)(248.02539196,73.69470494)(248.00538635,73.61470998)
\lineto(247.94538635,73.37470998)
\curveto(247.92539206,73.29470534)(247.91539207,73.20470543)(247.91538635,73.10470998)
\lineto(247.91538635,72.78970998)
\curveto(247.93539205,72.76970586)(247.94539204,72.7297059)(247.94538635,72.66970998)
\curveto(247.93539205,72.61970601)(247.93539205,72.57470606)(247.94538635,72.53470998)
\lineto(248.00538635,72.29470998)
\curveto(248.01539197,72.22470641)(248.03539195,72.15470648)(248.06538635,72.08470998)
\curveto(248.32539166,71.48470715)(248.79039119,71.07970755)(249.46038635,70.86970998)
\curveto(249.54039044,70.83970779)(249.62039036,70.81970781)(249.70038635,70.80970998)
\curveto(249.7803902,70.79970783)(249.86539012,70.78470785)(249.95538635,70.76470998)
\lineto(250.10538635,70.76470998)
\curveto(250.14538984,70.75470788)(250.21538977,70.74970788)(250.31538635,70.74970998)
\curveto(250.54538944,70.74970788)(250.74038924,70.76970786)(250.90038635,70.80970998)
\curveto(250.97038901,70.8297078)(251.03538895,70.84470779)(251.09538635,70.85470998)
\curveto(251.15538883,70.86470777)(251.22038876,70.88470775)(251.29038635,70.91470998)
\curveto(251.57038841,71.02470761)(251.81538817,71.16970746)(252.02538635,71.34970998)
\curveto(252.22538776,71.5297071)(252.3853876,71.76470687)(252.50538635,72.05470998)
\lineto(252.59538635,72.29470998)
\lineto(252.65538635,72.53470998)
\curveto(252.67538731,72.58470605)(252.6803873,72.62470601)(252.67038635,72.65470998)
\curveto(252.66038732,72.69470594)(252.66538732,72.73970589)(252.68538635,72.78970998)
\curveto(252.69538729,72.81970581)(252.70038728,72.87470576)(252.70038635,72.95470998)
\curveto(252.70038728,73.0347056)(252.69538729,73.09470554)(252.68538635,73.13470998)
\curveto(252.66538732,73.24470539)(252.65038733,73.34970528)(252.64038635,73.44970998)
\curveto(252.63038735,73.54970508)(252.60038738,73.64470499)(252.55038635,73.73470998)
\curveto(252.35038763,74.26470437)(251.97538801,74.65470398)(251.42538635,74.90470998)
\curveto(251.32538866,74.94470369)(251.22038876,74.97470366)(251.11038635,74.99470998)
\lineto(250.78038635,75.08470998)
\curveto(250.70038928,75.08470355)(250.64038934,75.08970354)(250.60038635,75.09970998)
}
}
{
\newrgbcolor{curcolor}{0 0 0}
\pscustom[linestyle=none,fillstyle=solid,fillcolor=curcolor]
{
\newpath
\moveto(251.98038635,78.63431936)
\lineto(251.98038635,79.26431936)
\lineto(251.98038635,79.45931936)
\curveto(251.980388,79.52931683)(251.99038799,79.58931677)(252.01038635,79.63931936)
\curveto(252.05038793,79.70931665)(252.09038789,79.7593166)(252.13038635,79.78931936)
\curveto(252.1803878,79.82931653)(252.24538774,79.84931651)(252.32538635,79.84931936)
\curveto(252.40538758,79.8593165)(252.49038749,79.86431649)(252.58038635,79.86431936)
\lineto(253.30038635,79.86431936)
\curveto(253.7803862,79.86431649)(254.19038579,79.80431655)(254.53038635,79.68431936)
\curveto(254.87038511,79.56431679)(255.14538484,79.36931699)(255.35538635,79.09931936)
\curveto(255.40538458,79.02931733)(255.45038453,78.9593174)(255.49038635,78.88931936)
\curveto(255.54038444,78.82931753)(255.5853844,78.7543176)(255.62538635,78.66431936)
\curveto(255.63538435,78.64431771)(255.64538434,78.61431774)(255.65538635,78.57431936)
\curveto(255.67538431,78.53431782)(255.6803843,78.48931787)(255.67038635,78.43931936)
\curveto(255.64038434,78.34931801)(255.56538442,78.29431806)(255.44538635,78.27431936)
\curveto(255.33538465,78.2543181)(255.24038474,78.26931809)(255.16038635,78.31931936)
\curveto(255.09038489,78.34931801)(255.02538496,78.39431796)(254.96538635,78.45431936)
\curveto(254.91538507,78.52431783)(254.86538512,78.58931777)(254.81538635,78.64931936)
\curveto(254.76538522,78.71931764)(254.69038529,78.77931758)(254.59038635,78.82931936)
\curveto(254.50038548,78.88931747)(254.41038557,78.93931742)(254.32038635,78.97931936)
\curveto(254.29038569,78.99931736)(254.23038575,79.02431733)(254.14038635,79.05431936)
\curveto(254.06038592,79.08431727)(253.99038599,79.08931727)(253.93038635,79.06931936)
\curveto(253.79038619,79.03931732)(253.70038628,78.97931738)(253.66038635,78.88931936)
\curveto(253.63038635,78.80931755)(253.61538637,78.71931764)(253.61538635,78.61931936)
\curveto(253.61538637,78.51931784)(253.59038639,78.43431792)(253.54038635,78.36431936)
\curveto(253.47038651,78.27431808)(253.33038665,78.22931813)(253.12038635,78.22931936)
\lineto(252.56538635,78.22931936)
\lineto(252.34038635,78.22931936)
\curveto(252.26038772,78.23931812)(252.19538779,78.2593181)(252.14538635,78.28931936)
\curveto(252.06538792,78.34931801)(252.02038796,78.41931794)(252.01038635,78.49931936)
\curveto(252.00038798,78.51931784)(251.99538799,78.53931782)(251.99538635,78.55931936)
\curveto(251.99538799,78.58931777)(251.99038799,78.61431774)(251.98038635,78.63431936)
}
}
{
\newrgbcolor{curcolor}{0 0 0}
\pscustom[linestyle=none,fillstyle=solid,fillcolor=curcolor]
{
}
}
{
\newrgbcolor{curcolor}{0 0 0}
\pscustom[linestyle=none,fillstyle=solid,fillcolor=curcolor]
{
\newpath
\moveto(243.01038635,89.26463186)
\curveto(243.00039698,89.95462722)(243.12039686,90.55462662)(243.37038635,91.06463186)
\curveto(243.62039636,91.58462559)(243.95539603,91.9796252)(244.37538635,92.24963186)
\curveto(244.45539553,92.29962488)(244.54539544,92.34462483)(244.64538635,92.38463186)
\curveto(244.73539525,92.42462475)(244.83039515,92.46962471)(244.93038635,92.51963186)
\curveto(245.03039495,92.55962462)(245.13039485,92.58962459)(245.23038635,92.60963186)
\curveto(245.33039465,92.62962455)(245.43539455,92.64962453)(245.54538635,92.66963186)
\curveto(245.59539439,92.68962449)(245.64039434,92.69462448)(245.68038635,92.68463186)
\curveto(245.72039426,92.6746245)(245.76539422,92.6796245)(245.81538635,92.69963186)
\curveto(245.86539412,92.70962447)(245.95039403,92.71462446)(246.07038635,92.71463186)
\curveto(246.1803938,92.71462446)(246.26539372,92.70962447)(246.32538635,92.69963186)
\curveto(246.3853936,92.6796245)(246.44539354,92.66962451)(246.50538635,92.66963186)
\curveto(246.56539342,92.6796245)(246.62539336,92.6746245)(246.68538635,92.65463186)
\curveto(246.82539316,92.61462456)(246.96039302,92.5796246)(247.09038635,92.54963186)
\curveto(247.22039276,92.51962466)(247.34539264,92.4796247)(247.46538635,92.42963186)
\curveto(247.60539238,92.36962481)(247.73039225,92.29962488)(247.84038635,92.21963186)
\curveto(247.95039203,92.14962503)(248.06039192,92.0746251)(248.17038635,91.99463186)
\lineto(248.23038635,91.93463186)
\curveto(248.25039173,91.92462525)(248.27039171,91.90962527)(248.29038635,91.88963186)
\curveto(248.45039153,91.76962541)(248.59539139,91.63462554)(248.72538635,91.48463186)
\curveto(248.85539113,91.33462584)(248.980391,91.174626)(249.10038635,91.00463186)
\curveto(249.32039066,90.69462648)(249.52539046,90.39962678)(249.71538635,90.11963186)
\curveto(249.85539013,89.88962729)(249.99038999,89.65962752)(250.12038635,89.42963186)
\curveto(250.25038973,89.20962797)(250.3853896,88.98962819)(250.52538635,88.76963186)
\curveto(250.69538929,88.51962866)(250.87538911,88.2796289)(251.06538635,88.04963186)
\curveto(251.25538873,87.82962935)(251.4803885,87.63962954)(251.74038635,87.47963186)
\curveto(251.80038818,87.43962974)(251.86038812,87.40462977)(251.92038635,87.37463186)
\curveto(251.97038801,87.34462983)(252.03538795,87.31462986)(252.11538635,87.28463186)
\curveto(252.1853878,87.26462991)(252.24538774,87.25962992)(252.29538635,87.26963186)
\curveto(252.36538762,87.28962989)(252.42038756,87.32462985)(252.46038635,87.37463186)
\curveto(252.49038749,87.42462975)(252.51038747,87.48462969)(252.52038635,87.55463186)
\lineto(252.52038635,87.79463186)
\lineto(252.52038635,88.54463186)
\lineto(252.52038635,91.34963186)
\lineto(252.52038635,92.00963186)
\curveto(252.52038746,92.09962508)(252.52538746,92.18462499)(252.53538635,92.26463186)
\curveto(252.53538745,92.34462483)(252.55538743,92.40962477)(252.59538635,92.45963186)
\curveto(252.63538735,92.50962467)(252.71038727,92.54962463)(252.82038635,92.57963186)
\curveto(252.92038706,92.61962456)(253.02038696,92.62962455)(253.12038635,92.60963186)
\lineto(253.25538635,92.60963186)
\curveto(253.32538666,92.58962459)(253.3853866,92.56962461)(253.43538635,92.54963186)
\curveto(253.4853865,92.52962465)(253.52538646,92.49462468)(253.55538635,92.44463186)
\curveto(253.59538639,92.39462478)(253.61538637,92.32462485)(253.61538635,92.23463186)
\lineto(253.61538635,91.96463186)
\lineto(253.61538635,91.06463186)
\lineto(253.61538635,87.55463186)
\lineto(253.61538635,86.48963186)
\curveto(253.61538637,86.40963077)(253.62038636,86.31963086)(253.63038635,86.21963186)
\curveto(253.63038635,86.11963106)(253.62038636,86.03463114)(253.60038635,85.96463186)
\curveto(253.53038645,85.75463142)(253.35038663,85.68963149)(253.06038635,85.76963186)
\curveto(253.02038696,85.7796314)(252.985387,85.7796314)(252.95538635,85.76963186)
\curveto(252.91538707,85.76963141)(252.87038711,85.7796314)(252.82038635,85.79963186)
\curveto(252.74038724,85.81963136)(252.65538733,85.83963134)(252.56538635,85.85963186)
\curveto(252.47538751,85.8796313)(252.39038759,85.90463127)(252.31038635,85.93463186)
\curveto(251.82038816,86.09463108)(251.40538858,86.29463088)(251.06538635,86.53463186)
\curveto(250.81538917,86.71463046)(250.59038939,86.91963026)(250.39038635,87.14963186)
\curveto(250.1803898,87.3796298)(249.98539,87.61962956)(249.80538635,87.86963186)
\curveto(249.62539036,88.12962905)(249.45539053,88.39462878)(249.29538635,88.66463186)
\curveto(249.12539086,88.94462823)(248.95039103,89.21462796)(248.77038635,89.47463186)
\curveto(248.69039129,89.58462759)(248.61539137,89.68962749)(248.54538635,89.78963186)
\curveto(248.47539151,89.89962728)(248.40039158,90.00962717)(248.32038635,90.11963186)
\curveto(248.29039169,90.15962702)(248.26039172,90.19462698)(248.23038635,90.22463186)
\curveto(248.19039179,90.26462691)(248.16039182,90.30462687)(248.14038635,90.34463186)
\curveto(248.03039195,90.48462669)(247.90539208,90.60962657)(247.76538635,90.71963186)
\curveto(247.73539225,90.73962644)(247.71039227,90.76462641)(247.69038635,90.79463186)
\curveto(247.66039232,90.82462635)(247.63039235,90.84962633)(247.60038635,90.86963186)
\curveto(247.50039248,90.94962623)(247.40039258,91.01462616)(247.30038635,91.06463186)
\curveto(247.20039278,91.12462605)(247.09039289,91.179626)(246.97038635,91.22963186)
\curveto(246.90039308,91.25962592)(246.82539316,91.2796259)(246.74538635,91.28963186)
\lineto(246.50538635,91.34963186)
\lineto(246.41538635,91.34963186)
\curveto(246.3853936,91.35962582)(246.35539363,91.36462581)(246.32538635,91.36463186)
\curveto(246.25539373,91.38462579)(246.16039382,91.38962579)(246.04038635,91.37963186)
\curveto(245.91039407,91.3796258)(245.81039417,91.36962581)(245.74038635,91.34963186)
\curveto(245.66039432,91.32962585)(245.5853944,91.30962587)(245.51538635,91.28963186)
\curveto(245.43539455,91.2796259)(245.35539463,91.25962592)(245.27538635,91.22963186)
\curveto(245.03539495,91.11962606)(244.83539515,90.96962621)(244.67538635,90.77963186)
\curveto(244.50539548,90.59962658)(244.36539562,90.3796268)(244.25538635,90.11963186)
\curveto(244.23539575,90.04962713)(244.22039576,89.9796272)(244.21038635,89.90963186)
\curveto(244.19039579,89.83962734)(244.17039581,89.76462741)(244.15038635,89.68463186)
\curveto(244.13039585,89.60462757)(244.12039586,89.49462768)(244.12038635,89.35463186)
\curveto(244.12039586,89.22462795)(244.13039585,89.11962806)(244.15038635,89.03963186)
\curveto(244.16039582,88.9796282)(244.16539582,88.92462825)(244.16538635,88.87463186)
\curveto(244.16539582,88.82462835)(244.17539581,88.7746284)(244.19538635,88.72463186)
\curveto(244.23539575,88.62462855)(244.27539571,88.52962865)(244.31538635,88.43963186)
\curveto(244.35539563,88.35962882)(244.40039558,88.2796289)(244.45038635,88.19963186)
\curveto(244.47039551,88.16962901)(244.49539549,88.13962904)(244.52538635,88.10963186)
\curveto(244.55539543,88.08962909)(244.5803954,88.06462911)(244.60038635,88.03463186)
\lineto(244.67538635,87.95963186)
\curveto(244.69539529,87.92962925)(244.71539527,87.90462927)(244.73538635,87.88463186)
\lineto(244.94538635,87.73463186)
\curveto(245.00539498,87.69462948)(245.07039491,87.64962953)(245.14038635,87.59963186)
\curveto(245.23039475,87.53962964)(245.33539465,87.48962969)(245.45538635,87.44963186)
\curveto(245.56539442,87.41962976)(245.67539431,87.38462979)(245.78538635,87.34463186)
\curveto(245.89539409,87.30462987)(246.04039394,87.2796299)(246.22038635,87.26963186)
\curveto(246.39039359,87.25962992)(246.51539347,87.22962995)(246.59538635,87.17963186)
\curveto(246.67539331,87.12963005)(246.72039326,87.05463012)(246.73038635,86.95463186)
\curveto(246.74039324,86.85463032)(246.74539324,86.74463043)(246.74538635,86.62463186)
\curveto(246.74539324,86.58463059)(246.75039323,86.54463063)(246.76038635,86.50463186)
\curveto(246.76039322,86.46463071)(246.75539323,86.42963075)(246.74538635,86.39963186)
\curveto(246.72539326,86.34963083)(246.71539327,86.29963088)(246.71538635,86.24963186)
\curveto(246.71539327,86.20963097)(246.70539328,86.16963101)(246.68538635,86.12963186)
\curveto(246.62539336,86.03963114)(246.49039349,85.99463118)(246.28038635,85.99463186)
\lineto(246.16038635,85.99463186)
\curveto(246.10039388,86.00463117)(246.04039394,86.00963117)(245.98038635,86.00963186)
\curveto(245.91039407,86.01963116)(245.84539414,86.02963115)(245.78538635,86.03963186)
\curveto(245.67539431,86.05963112)(245.57539441,86.0796311)(245.48538635,86.09963186)
\curveto(245.3853946,86.11963106)(245.29039469,86.14963103)(245.20038635,86.18963186)
\curveto(245.13039485,86.20963097)(245.07039491,86.22963095)(245.02038635,86.24963186)
\lineto(244.84038635,86.30963186)
\curveto(244.5803954,86.42963075)(244.33539565,86.58463059)(244.10538635,86.77463186)
\curveto(243.87539611,86.9746302)(243.69039629,87.18962999)(243.55038635,87.41963186)
\curveto(243.47039651,87.52962965)(243.40539658,87.64462953)(243.35538635,87.76463186)
\lineto(243.20538635,88.15463186)
\curveto(243.15539683,88.26462891)(243.12539686,88.3796288)(243.11538635,88.49963186)
\curveto(243.09539689,88.61962856)(243.07039691,88.74462843)(243.04038635,88.87463186)
\curveto(243.04039694,88.94462823)(243.04039694,89.00962817)(243.04038635,89.06963186)
\curveto(243.03039695,89.12962805)(243.02039696,89.19462798)(243.01038635,89.26463186)
}
}
{
\newrgbcolor{curcolor}{0 0 0}
\pscustom[linestyle=none,fillstyle=solid,fillcolor=curcolor]
{
\newpath
\moveto(248.53038635,101.36424123)
\lineto(248.78538635,101.36424123)
\curveto(248.86539112,101.37423353)(248.94039104,101.36923353)(249.01038635,101.34924123)
\lineto(249.25038635,101.34924123)
\lineto(249.41538635,101.34924123)
\curveto(249.51539047,101.32923357)(249.62039036,101.31923358)(249.73038635,101.31924123)
\curveto(249.83039015,101.31923358)(249.93039005,101.30923359)(250.03038635,101.28924123)
\lineto(250.18038635,101.28924123)
\curveto(250.32038966,101.25923364)(250.46038952,101.23923366)(250.60038635,101.22924123)
\curveto(250.73038925,101.21923368)(250.86038912,101.19423371)(250.99038635,101.15424123)
\curveto(251.07038891,101.13423377)(251.15538883,101.11423379)(251.24538635,101.09424123)
\lineto(251.48538635,101.03424123)
\lineto(251.78538635,100.91424123)
\curveto(251.87538811,100.88423402)(251.96538802,100.84923405)(252.05538635,100.80924123)
\curveto(252.27538771,100.70923419)(252.49038749,100.57423433)(252.70038635,100.40424123)
\curveto(252.91038707,100.24423466)(253.0803869,100.06923483)(253.21038635,99.87924123)
\curveto(253.25038673,99.82923507)(253.29038669,99.76923513)(253.33038635,99.69924123)
\curveto(253.36038662,99.63923526)(253.39538659,99.57923532)(253.43538635,99.51924123)
\curveto(253.4853865,99.43923546)(253.52538646,99.34423556)(253.55538635,99.23424123)
\curveto(253.5853864,99.12423578)(253.61538637,99.01923588)(253.64538635,98.91924123)
\curveto(253.6853863,98.80923609)(253.71038627,98.6992362)(253.72038635,98.58924123)
\curveto(253.73038625,98.47923642)(253.74538624,98.36423654)(253.76538635,98.24424123)
\curveto(253.77538621,98.2042367)(253.77538621,98.15923674)(253.76538635,98.10924123)
\curveto(253.76538622,98.06923683)(253.77038621,98.02923687)(253.78038635,97.98924123)
\curveto(253.79038619,97.94923695)(253.79538619,97.89423701)(253.79538635,97.82424123)
\curveto(253.79538619,97.75423715)(253.79038619,97.7042372)(253.78038635,97.67424123)
\curveto(253.76038622,97.62423728)(253.75538623,97.57923732)(253.76538635,97.53924123)
\curveto(253.77538621,97.4992374)(253.77538621,97.46423744)(253.76538635,97.43424123)
\lineto(253.76538635,97.34424123)
\curveto(253.74538624,97.28423762)(253.73038625,97.21923768)(253.72038635,97.14924123)
\curveto(253.72038626,97.08923781)(253.71538627,97.02423788)(253.70538635,96.95424123)
\curveto(253.65538633,96.78423812)(253.60538638,96.62423828)(253.55538635,96.47424123)
\curveto(253.50538648,96.32423858)(253.44038654,96.17923872)(253.36038635,96.03924123)
\curveto(253.32038666,95.98923891)(253.29038669,95.93423897)(253.27038635,95.87424123)
\curveto(253.24038674,95.82423908)(253.20538678,95.77423913)(253.16538635,95.72424123)
\curveto(252.985387,95.48423942)(252.76538722,95.28423962)(252.50538635,95.12424123)
\curveto(252.24538774,94.96423994)(251.96038802,94.82424008)(251.65038635,94.70424123)
\curveto(251.51038847,94.64424026)(251.37038861,94.5992403)(251.23038635,94.56924123)
\curveto(251.0803889,94.53924036)(250.92538906,94.5042404)(250.76538635,94.46424123)
\curveto(250.65538933,94.44424046)(250.54538944,94.42924047)(250.43538635,94.41924123)
\curveto(250.32538966,94.40924049)(250.21538977,94.39424051)(250.10538635,94.37424123)
\curveto(250.06538992,94.36424054)(250.02538996,94.35924054)(249.98538635,94.35924123)
\curveto(249.94539004,94.36924053)(249.90539008,94.36924053)(249.86538635,94.35924123)
\curveto(249.81539017,94.34924055)(249.76539022,94.34424056)(249.71538635,94.34424123)
\lineto(249.55038635,94.34424123)
\curveto(249.50039048,94.32424058)(249.45039053,94.31924058)(249.40038635,94.32924123)
\curveto(249.34039064,94.33924056)(249.2853907,94.33924056)(249.23538635,94.32924123)
\curveto(249.19539079,94.31924058)(249.15039083,94.31924058)(249.10038635,94.32924123)
\curveto(249.05039093,94.33924056)(249.00039098,94.33424057)(248.95038635,94.31424123)
\curveto(248.8803911,94.29424061)(248.80539118,94.28924061)(248.72538635,94.29924123)
\curveto(248.63539135,94.30924059)(248.55039143,94.31424059)(248.47038635,94.31424123)
\curveto(248.3803916,94.31424059)(248.2803917,94.30924059)(248.17038635,94.29924123)
\curveto(248.05039193,94.28924061)(247.95039203,94.29424061)(247.87038635,94.31424123)
\lineto(247.58538635,94.31424123)
\lineto(246.95538635,94.35924123)
\curveto(246.85539313,94.36924053)(246.76039322,94.37924052)(246.67038635,94.38924123)
\lineto(246.37038635,94.41924123)
\curveto(246.32039366,94.43924046)(246.27039371,94.44424046)(246.22038635,94.43424123)
\curveto(246.16039382,94.43424047)(246.10539388,94.44424046)(246.05538635,94.46424123)
\curveto(245.8853941,94.51424039)(245.72039426,94.55424035)(245.56038635,94.58424123)
\curveto(245.39039459,94.61424029)(245.23039475,94.66424024)(245.08038635,94.73424123)
\curveto(244.62039536,94.92423998)(244.24539574,95.14423976)(243.95538635,95.39424123)
\curveto(243.66539632,95.65423925)(243.42039656,96.01423889)(243.22038635,96.47424123)
\curveto(243.17039681,96.6042383)(243.13539685,96.73423817)(243.11538635,96.86424123)
\curveto(243.09539689,97.0042379)(243.07039691,97.14423776)(243.04038635,97.28424123)
\curveto(243.03039695,97.35423755)(243.02539696,97.41923748)(243.02538635,97.47924123)
\curveto(243.02539696,97.53923736)(243.02039696,97.6042373)(243.01038635,97.67424123)
\curveto(242.99039699,98.5042364)(243.14039684,99.17423573)(243.46038635,99.68424123)
\curveto(243.77039621,100.19423471)(244.21039577,100.57423433)(244.78038635,100.82424123)
\curveto(244.90039508,100.87423403)(245.02539496,100.91923398)(245.15538635,100.95924123)
\curveto(245.2853947,100.9992339)(245.42039456,101.04423386)(245.56038635,101.09424123)
\curveto(245.64039434,101.11423379)(245.72539426,101.12923377)(245.81538635,101.13924123)
\lineto(246.05538635,101.19924123)
\curveto(246.16539382,101.22923367)(246.27539371,101.24423366)(246.38538635,101.24424123)
\curveto(246.49539349,101.25423365)(246.60539338,101.26923363)(246.71538635,101.28924123)
\curveto(246.76539322,101.30923359)(246.81039317,101.31423359)(246.85038635,101.30424123)
\curveto(246.89039309,101.3042336)(246.93039305,101.30923359)(246.97038635,101.31924123)
\curveto(247.02039296,101.32923357)(247.07539291,101.32923357)(247.13538635,101.31924123)
\curveto(247.1853928,101.31923358)(247.23539275,101.32423358)(247.28538635,101.33424123)
\lineto(247.42038635,101.33424123)
\curveto(247.4803925,101.35423355)(247.55039243,101.35423355)(247.63038635,101.33424123)
\curveto(247.70039228,101.32423358)(247.76539222,101.32923357)(247.82538635,101.34924123)
\curveto(247.85539213,101.35923354)(247.89539209,101.36423354)(247.94538635,101.36424123)
\lineto(248.06538635,101.36424123)
\lineto(248.53038635,101.36424123)
\moveto(250.85538635,99.81924123)
\curveto(250.53538945,99.91923498)(250.17038981,99.97923492)(249.76038635,99.99924123)
\curveto(249.35039063,100.01923488)(248.94039104,100.02923487)(248.53038635,100.02924123)
\curveto(248.10039188,100.02923487)(247.6803923,100.01923488)(247.27038635,99.99924123)
\curveto(246.86039312,99.97923492)(246.47539351,99.93423497)(246.11538635,99.86424123)
\curveto(245.75539423,99.79423511)(245.43539455,99.68423522)(245.15538635,99.53424123)
\curveto(244.86539512,99.39423551)(244.63039535,99.1992357)(244.45038635,98.94924123)
\curveto(244.34039564,98.78923611)(244.26039572,98.60923629)(244.21038635,98.40924123)
\curveto(244.15039583,98.20923669)(244.12039586,97.96423694)(244.12038635,97.67424123)
\curveto(244.14039584,97.65423725)(244.15039583,97.61923728)(244.15038635,97.56924123)
\curveto(244.14039584,97.51923738)(244.14039584,97.47923742)(244.15038635,97.44924123)
\curveto(244.17039581,97.36923753)(244.19039579,97.29423761)(244.21038635,97.22424123)
\curveto(244.22039576,97.16423774)(244.24039574,97.0992378)(244.27038635,97.02924123)
\curveto(244.39039559,96.75923814)(244.56039542,96.53923836)(244.78038635,96.36924123)
\curveto(244.99039499,96.20923869)(245.23539475,96.07423883)(245.51538635,95.96424123)
\curveto(245.62539436,95.91423899)(245.74539424,95.87423903)(245.87538635,95.84424123)
\curveto(245.99539399,95.82423908)(246.12039386,95.7992391)(246.25038635,95.76924123)
\curveto(246.30039368,95.74923915)(246.35539363,95.73923916)(246.41538635,95.73924123)
\curveto(246.46539352,95.73923916)(246.51539347,95.73423917)(246.56538635,95.72424123)
\curveto(246.65539333,95.71423919)(246.75039323,95.7042392)(246.85038635,95.69424123)
\curveto(246.94039304,95.68423922)(247.03539295,95.67423923)(247.13538635,95.66424123)
\curveto(247.21539277,95.66423924)(247.30039268,95.65923924)(247.39038635,95.64924123)
\lineto(247.63038635,95.64924123)
\lineto(247.81038635,95.64924123)
\curveto(247.84039214,95.63923926)(247.87539211,95.63423927)(247.91538635,95.63424123)
\lineto(248.05038635,95.63424123)
\lineto(248.50038635,95.63424123)
\curveto(248.5803914,95.63423927)(248.66539132,95.62923927)(248.75538635,95.61924123)
\curveto(248.83539115,95.61923928)(248.91039107,95.62923927)(248.98038635,95.64924123)
\lineto(249.25038635,95.64924123)
\curveto(249.27039071,95.64923925)(249.30039068,95.64423926)(249.34038635,95.63424123)
\curveto(249.37039061,95.63423927)(249.39539059,95.63923926)(249.41538635,95.64924123)
\curveto(249.51539047,95.65923924)(249.61539037,95.66423924)(249.71538635,95.66424123)
\curveto(249.80539018,95.67423923)(249.90539008,95.68423922)(250.01538635,95.69424123)
\curveto(250.13538985,95.72423918)(250.26038972,95.73923916)(250.39038635,95.73924123)
\curveto(250.51038947,95.74923915)(250.62538936,95.77423913)(250.73538635,95.81424123)
\curveto(251.03538895,95.89423901)(251.30038868,95.97923892)(251.53038635,96.06924123)
\curveto(251.76038822,96.16923873)(251.97538801,96.31423859)(252.17538635,96.50424123)
\curveto(252.37538761,96.71423819)(252.52538746,96.97923792)(252.62538635,97.29924123)
\curveto(252.64538734,97.33923756)(252.65538733,97.37423753)(252.65538635,97.40424123)
\curveto(252.64538734,97.44423746)(252.65038733,97.48923741)(252.67038635,97.53924123)
\curveto(252.6803873,97.57923732)(252.69038729,97.64923725)(252.70038635,97.74924123)
\curveto(252.71038727,97.85923704)(252.70538728,97.94423696)(252.68538635,98.00424123)
\curveto(252.66538732,98.07423683)(252.65538733,98.14423676)(252.65538635,98.21424123)
\curveto(252.64538734,98.28423662)(252.63038735,98.34923655)(252.61038635,98.40924123)
\curveto(252.55038743,98.60923629)(252.46538752,98.78923611)(252.35538635,98.94924123)
\curveto(252.33538765,98.97923592)(252.31538767,99.0042359)(252.29538635,99.02424123)
\lineto(252.23538635,99.08424123)
\curveto(252.21538777,99.12423578)(252.17538781,99.17423573)(252.11538635,99.23424123)
\curveto(251.97538801,99.33423557)(251.84538814,99.41923548)(251.72538635,99.48924123)
\curveto(251.60538838,99.55923534)(251.46038852,99.62923527)(251.29038635,99.69924123)
\curveto(251.22038876,99.72923517)(251.15038883,99.74923515)(251.08038635,99.75924123)
\curveto(251.01038897,99.77923512)(250.93538905,99.7992351)(250.85538635,99.81924123)
}
}
{
\newrgbcolor{curcolor}{0 0 0}
\pscustom[linestyle=none,fillstyle=solid,fillcolor=curcolor]
{
\newpath
\moveto(243.01038635,106.77385061)
\curveto(243.01039697,106.87384575)(243.02039696,106.96884566)(243.04038635,107.05885061)
\curveto(243.05039693,107.14884548)(243.0803969,107.21384541)(243.13038635,107.25385061)
\curveto(243.21039677,107.31384531)(243.31539667,107.34384528)(243.44538635,107.34385061)
\lineto(243.83538635,107.34385061)
\lineto(245.33538635,107.34385061)
\lineto(251.72538635,107.34385061)
\lineto(252.89538635,107.34385061)
\lineto(253.21038635,107.34385061)
\curveto(253.31038667,107.35384527)(253.39038659,107.33884529)(253.45038635,107.29885061)
\curveto(253.53038645,107.24884538)(253.5803864,107.17384545)(253.60038635,107.07385061)
\curveto(253.61038637,106.98384564)(253.61538637,106.87384575)(253.61538635,106.74385061)
\lineto(253.61538635,106.51885061)
\curveto(253.59538639,106.43884619)(253.5803864,106.36884626)(253.57038635,106.30885061)
\curveto(253.55038643,106.24884638)(253.51038647,106.19884643)(253.45038635,106.15885061)
\curveto(253.39038659,106.11884651)(253.31538667,106.09884653)(253.22538635,106.09885061)
\lineto(252.92538635,106.09885061)
\lineto(251.83038635,106.09885061)
\lineto(246.49038635,106.09885061)
\curveto(246.40039358,106.07884655)(246.32539366,106.06384656)(246.26538635,106.05385061)
\curveto(246.19539379,106.05384657)(246.13539385,106.0238466)(246.08538635,105.96385061)
\curveto(246.03539395,105.89384673)(246.01039397,105.80384682)(246.01038635,105.69385061)
\curveto(246.00039398,105.59384703)(245.99539399,105.48384714)(245.99538635,105.36385061)
\lineto(245.99538635,104.22385061)
\lineto(245.99538635,103.72885061)
\curveto(245.985394,103.56884906)(245.92539406,103.45884917)(245.81538635,103.39885061)
\curveto(245.7853942,103.37884925)(245.75539423,103.36884926)(245.72538635,103.36885061)
\curveto(245.6853943,103.36884926)(245.64039434,103.36384926)(245.59038635,103.35385061)
\curveto(245.47039451,103.33384929)(245.36039462,103.33884929)(245.26038635,103.36885061)
\curveto(245.16039482,103.40884922)(245.09039489,103.46384916)(245.05038635,103.53385061)
\curveto(245.00039498,103.61384901)(244.97539501,103.73384889)(244.97538635,103.89385061)
\curveto(244.97539501,104.05384857)(244.96039502,104.18884844)(244.93038635,104.29885061)
\curveto(244.92039506,104.34884828)(244.91539507,104.40384822)(244.91538635,104.46385061)
\curveto(244.90539508,104.5238481)(244.89039509,104.58384804)(244.87038635,104.64385061)
\curveto(244.82039516,104.79384783)(244.77039521,104.93884769)(244.72038635,105.07885061)
\curveto(244.66039532,105.21884741)(244.59039539,105.35384727)(244.51038635,105.48385061)
\curveto(244.42039556,105.623847)(244.31539567,105.74384688)(244.19538635,105.84385061)
\curveto(244.07539591,105.94384668)(243.94539604,106.03884659)(243.80538635,106.12885061)
\curveto(243.70539628,106.18884644)(243.59539639,106.23384639)(243.47538635,106.26385061)
\curveto(243.35539663,106.30384632)(243.25039673,106.35384627)(243.16038635,106.41385061)
\curveto(243.10039688,106.46384616)(243.06039692,106.53384609)(243.04038635,106.62385061)
\curveto(243.03039695,106.64384598)(243.02539696,106.66884596)(243.02538635,106.69885061)
\curveto(243.02539696,106.7288459)(243.02039696,106.75384587)(243.01038635,106.77385061)
}
}
{
\newrgbcolor{curcolor}{0 0 0}
\pscustom[linestyle=none,fillstyle=solid,fillcolor=curcolor]
{
\newpath
\moveto(243.01038635,115.12345998)
\curveto(243.01039697,115.22345513)(243.02039696,115.31845503)(243.04038635,115.40845998)
\curveto(243.05039693,115.49845485)(243.0803969,115.56345479)(243.13038635,115.60345998)
\curveto(243.21039677,115.66345469)(243.31539667,115.69345466)(243.44538635,115.69345998)
\lineto(243.83538635,115.69345998)
\lineto(245.33538635,115.69345998)
\lineto(251.72538635,115.69345998)
\lineto(252.89538635,115.69345998)
\lineto(253.21038635,115.69345998)
\curveto(253.31038667,115.70345465)(253.39038659,115.68845466)(253.45038635,115.64845998)
\curveto(253.53038645,115.59845475)(253.5803864,115.52345483)(253.60038635,115.42345998)
\curveto(253.61038637,115.33345502)(253.61538637,115.22345513)(253.61538635,115.09345998)
\lineto(253.61538635,114.86845998)
\curveto(253.59538639,114.78845556)(253.5803864,114.71845563)(253.57038635,114.65845998)
\curveto(253.55038643,114.59845575)(253.51038647,114.5484558)(253.45038635,114.50845998)
\curveto(253.39038659,114.46845588)(253.31538667,114.4484559)(253.22538635,114.44845998)
\lineto(252.92538635,114.44845998)
\lineto(251.83038635,114.44845998)
\lineto(246.49038635,114.44845998)
\curveto(246.40039358,114.42845592)(246.32539366,114.41345594)(246.26538635,114.40345998)
\curveto(246.19539379,114.40345595)(246.13539385,114.37345598)(246.08538635,114.31345998)
\curveto(246.03539395,114.24345611)(246.01039397,114.1534562)(246.01038635,114.04345998)
\curveto(246.00039398,113.94345641)(245.99539399,113.83345652)(245.99538635,113.71345998)
\lineto(245.99538635,112.57345998)
\lineto(245.99538635,112.07845998)
\curveto(245.985394,111.91845843)(245.92539406,111.80845854)(245.81538635,111.74845998)
\curveto(245.7853942,111.72845862)(245.75539423,111.71845863)(245.72538635,111.71845998)
\curveto(245.6853943,111.71845863)(245.64039434,111.71345864)(245.59038635,111.70345998)
\curveto(245.47039451,111.68345867)(245.36039462,111.68845866)(245.26038635,111.71845998)
\curveto(245.16039482,111.75845859)(245.09039489,111.81345854)(245.05038635,111.88345998)
\curveto(245.00039498,111.96345839)(244.97539501,112.08345827)(244.97538635,112.24345998)
\curveto(244.97539501,112.40345795)(244.96039502,112.53845781)(244.93038635,112.64845998)
\curveto(244.92039506,112.69845765)(244.91539507,112.7534576)(244.91538635,112.81345998)
\curveto(244.90539508,112.87345748)(244.89039509,112.93345742)(244.87038635,112.99345998)
\curveto(244.82039516,113.14345721)(244.77039521,113.28845706)(244.72038635,113.42845998)
\curveto(244.66039532,113.56845678)(244.59039539,113.70345665)(244.51038635,113.83345998)
\curveto(244.42039556,113.97345638)(244.31539567,114.09345626)(244.19538635,114.19345998)
\curveto(244.07539591,114.29345606)(243.94539604,114.38845596)(243.80538635,114.47845998)
\curveto(243.70539628,114.53845581)(243.59539639,114.58345577)(243.47538635,114.61345998)
\curveto(243.35539663,114.6534557)(243.25039673,114.70345565)(243.16038635,114.76345998)
\curveto(243.10039688,114.81345554)(243.06039692,114.88345547)(243.04038635,114.97345998)
\curveto(243.03039695,114.99345536)(243.02539696,115.01845533)(243.02538635,115.04845998)
\curveto(243.02539696,115.07845527)(243.02039696,115.10345525)(243.01038635,115.12345998)
}
}
{
\newrgbcolor{curcolor}{0 0 0}
\pscustom[linestyle=none,fillstyle=solid,fillcolor=curcolor]
{
\newpath
\moveto(180.20139282,31.67142873)
\lineto(180.20139282,32.58642873)
\curveto(180.20140352,32.68642608)(180.20140352,32.78142599)(180.20139282,32.87142873)
\curveto(180.20140352,32.96142581)(180.2214035,33.03642573)(180.26139282,33.09642873)
\curveto(180.3214034,33.18642558)(180.40140332,33.24642552)(180.50139282,33.27642873)
\curveto(180.60140312,33.31642545)(180.70640301,33.36142541)(180.81639282,33.41142873)
\curveto(181.00640271,33.49142528)(181.19640252,33.56142521)(181.38639282,33.62142873)
\curveto(181.57640214,33.69142508)(181.76640195,33.766425)(181.95639282,33.84642873)
\curveto(182.13640158,33.91642485)(182.3214014,33.98142479)(182.51139282,34.04142873)
\curveto(182.69140103,34.10142467)(182.87140085,34.1714246)(183.05139282,34.25142873)
\curveto(183.19140053,34.31142446)(183.33640038,34.3664244)(183.48639282,34.41642873)
\curveto(183.63640008,34.4664243)(183.78139994,34.52142425)(183.92139282,34.58142873)
\curveto(184.37139935,34.76142401)(184.82639889,34.93142384)(185.28639282,35.09142873)
\curveto(185.73639798,35.25142352)(186.18639753,35.42142335)(186.63639282,35.60142873)
\curveto(186.68639703,35.62142315)(186.73639698,35.63642313)(186.78639282,35.64642873)
\lineto(186.93639282,35.70642873)
\curveto(187.15639656,35.79642297)(187.38139634,35.88142289)(187.61139282,35.96142873)
\curveto(187.83139589,36.04142273)(188.05139567,36.12642264)(188.27139282,36.21642873)
\curveto(188.36139536,36.25642251)(188.47139525,36.29642247)(188.60139282,36.33642873)
\curveto(188.721395,36.37642239)(188.79139493,36.44142233)(188.81139282,36.53142873)
\curveto(188.8213949,36.5714222)(188.8213949,36.60142217)(188.81139282,36.62142873)
\lineto(188.75139282,36.68142873)
\curveto(188.70139502,36.73142204)(188.64639507,36.766422)(188.58639282,36.78642873)
\curveto(188.52639519,36.81642195)(188.46139526,36.84642192)(188.39139282,36.87642873)
\lineto(187.76139282,37.11642873)
\curveto(187.54139618,37.19642157)(187.32639639,37.27642149)(187.11639282,37.35642873)
\lineto(186.96639282,37.41642873)
\lineto(186.78639282,37.47642873)
\curveto(186.59639712,37.55642121)(186.40639731,37.62642114)(186.21639282,37.68642873)
\curveto(186.0163977,37.75642101)(185.8163979,37.83142094)(185.61639282,37.91142873)
\curveto(185.03639868,38.15142062)(184.45139927,38.3714204)(183.86139282,38.57142873)
\curveto(183.27140045,38.78141999)(182.68640103,39.00641976)(182.10639282,39.24642873)
\curveto(181.90640181,39.32641944)(181.70140202,39.40141937)(181.49139282,39.47142873)
\curveto(181.28140244,39.55141922)(181.07640264,39.63141914)(180.87639282,39.71142873)
\curveto(180.79640292,39.75141902)(180.69640302,39.78641898)(180.57639282,39.81642873)
\curveto(180.45640326,39.85641891)(180.37140335,39.91141886)(180.32139282,39.98142873)
\curveto(180.28140344,40.04141873)(180.25140347,40.11641865)(180.23139282,40.20642873)
\curveto(180.21140351,40.30641846)(180.20140352,40.41641835)(180.20139282,40.53642873)
\curveto(180.19140353,40.65641811)(180.19140353,40.77641799)(180.20139282,40.89642873)
\curveto(180.20140352,41.01641775)(180.20140352,41.12641764)(180.20139282,41.22642873)
\curveto(180.20140352,41.31641745)(180.20140352,41.40641736)(180.20139282,41.49642873)
\curveto(180.20140352,41.59641717)(180.2214035,41.6714171)(180.26139282,41.72142873)
\curveto(180.31140341,41.81141696)(180.40140332,41.86141691)(180.53139282,41.87142873)
\curveto(180.66140306,41.88141689)(180.80140292,41.88641688)(180.95139282,41.88642873)
\lineto(182.60139282,41.88642873)
\lineto(188.87139282,41.88642873)
\lineto(190.13139282,41.88642873)
\curveto(190.24139348,41.88641688)(190.35139337,41.88641688)(190.46139282,41.88642873)
\curveto(190.57139315,41.89641687)(190.65639306,41.87641689)(190.71639282,41.82642873)
\curveto(190.77639294,41.79641697)(190.8163929,41.75141702)(190.83639282,41.69142873)
\curveto(190.84639287,41.63141714)(190.86139286,41.56141721)(190.88139282,41.48142873)
\lineto(190.88139282,41.24142873)
\lineto(190.88139282,40.88142873)
\curveto(190.87139285,40.771418)(190.82639289,40.69141808)(190.74639282,40.64142873)
\curveto(190.716393,40.62141815)(190.68639303,40.60641816)(190.65639282,40.59642873)
\curveto(190.6163931,40.59641817)(190.57139315,40.58641818)(190.52139282,40.56642873)
\lineto(190.35639282,40.56642873)
\curveto(190.29639342,40.55641821)(190.22639349,40.55141822)(190.14639282,40.55142873)
\curveto(190.06639365,40.56141821)(189.99139373,40.5664182)(189.92139282,40.56642873)
\lineto(189.08139282,40.56642873)
\lineto(184.65639282,40.56642873)
\curveto(184.40639931,40.5664182)(184.15639956,40.5664182)(183.90639282,40.56642873)
\curveto(183.64640007,40.5664182)(183.39640032,40.56141821)(183.15639282,40.55142873)
\curveto(183.05640066,40.55141822)(182.94640077,40.54641822)(182.82639282,40.53642873)
\curveto(182.70640101,40.52641824)(182.64640107,40.4714183)(182.64639282,40.37142873)
\lineto(182.66139282,40.37142873)
\curveto(182.68140104,40.30141847)(182.74640097,40.24141853)(182.85639282,40.19142873)
\curveto(182.96640075,40.15141862)(183.06140066,40.11641865)(183.14139282,40.08642873)
\curveto(183.31140041,40.01641875)(183.48640023,39.95141882)(183.66639282,39.89142873)
\curveto(183.83639988,39.83141894)(184.00639971,39.76141901)(184.17639282,39.68142873)
\curveto(184.22639949,39.66141911)(184.27139945,39.64641912)(184.31139282,39.63642873)
\curveto(184.35139937,39.62641914)(184.39639932,39.61141916)(184.44639282,39.59142873)
\curveto(184.62639909,39.51141926)(184.81139891,39.44141933)(185.00139282,39.38142873)
\curveto(185.18139854,39.33141944)(185.36139836,39.2664195)(185.54139282,39.18642873)
\curveto(185.69139803,39.11641965)(185.84639787,39.05641971)(186.00639282,39.00642873)
\curveto(186.15639756,38.95641981)(186.30639741,38.90141987)(186.45639282,38.84142873)
\curveto(186.92639679,38.64142013)(187.40139632,38.46142031)(187.88139282,38.30142873)
\curveto(188.35139537,38.14142063)(188.8163949,37.9664208)(189.27639282,37.77642873)
\curveto(189.45639426,37.69642107)(189.63639408,37.62642114)(189.81639282,37.56642873)
\curveto(189.99639372,37.50642126)(190.17639354,37.44142133)(190.35639282,37.37142873)
\curveto(190.46639325,37.32142145)(190.57139315,37.2714215)(190.67139282,37.22142873)
\curveto(190.76139296,37.18142159)(190.82639289,37.09642167)(190.86639282,36.96642873)
\curveto(190.87639284,36.94642182)(190.88139284,36.92142185)(190.88139282,36.89142873)
\curveto(190.87139285,36.8714219)(190.87139285,36.84642192)(190.88139282,36.81642873)
\curveto(190.89139283,36.78642198)(190.89639282,36.75142202)(190.89639282,36.71142873)
\curveto(190.88639283,36.6714221)(190.88139284,36.63142214)(190.88139282,36.59142873)
\lineto(190.88139282,36.29142873)
\curveto(190.88139284,36.19142258)(190.85639286,36.11142266)(190.80639282,36.05142873)
\curveto(190.75639296,35.9714228)(190.68639303,35.91142286)(190.59639282,35.87142873)
\curveto(190.49639322,35.84142293)(190.39639332,35.80142297)(190.29639282,35.75142873)
\curveto(190.09639362,35.6714231)(189.89139383,35.59142318)(189.68139282,35.51142873)
\curveto(189.46139426,35.44142333)(189.25139447,35.3664234)(189.05139282,35.28642873)
\curveto(188.87139485,35.20642356)(188.69139503,35.13642363)(188.51139282,35.07642873)
\curveto(188.3213954,35.02642374)(188.13639558,34.96142381)(187.95639282,34.88142873)
\curveto(187.39639632,34.65142412)(186.83139689,34.43642433)(186.26139282,34.23642873)
\curveto(185.69139803,34.03642473)(185.12639859,33.82142495)(184.56639282,33.59142873)
\lineto(183.93639282,33.35142873)
\curveto(183.7164,33.28142549)(183.50640021,33.20642556)(183.30639282,33.12642873)
\curveto(183.19640052,33.07642569)(183.09140063,33.03142574)(182.99139282,32.99142873)
\curveto(182.88140084,32.96142581)(182.78640093,32.91142586)(182.70639282,32.84142873)
\curveto(182.68640103,32.83142594)(182.67640104,32.82142595)(182.67639282,32.81142873)
\lineto(182.64639282,32.78142873)
\lineto(182.64639282,32.70642873)
\lineto(182.67639282,32.67642873)
\curveto(182.67640104,32.6664261)(182.68140104,32.65642611)(182.69139282,32.64642873)
\curveto(182.74140098,32.62642614)(182.79640092,32.61642615)(182.85639282,32.61642873)
\curveto(182.9164008,32.61642615)(182.97640074,32.60642616)(183.03639282,32.58642873)
\lineto(183.20139282,32.58642873)
\curveto(183.26140046,32.5664262)(183.32640039,32.56142621)(183.39639282,32.57142873)
\curveto(183.46640025,32.58142619)(183.53640018,32.58642618)(183.60639282,32.58642873)
\lineto(184.41639282,32.58642873)
\lineto(188.97639282,32.58642873)
\lineto(190.16139282,32.58642873)
\curveto(190.27139345,32.58642618)(190.38139334,32.58142619)(190.49139282,32.57142873)
\curveto(190.60139312,32.5714262)(190.68639303,32.54642622)(190.74639282,32.49642873)
\curveto(190.82639289,32.44642632)(190.87139285,32.35642641)(190.88139282,32.22642873)
\lineto(190.88139282,31.83642873)
\lineto(190.88139282,31.64142873)
\curveto(190.88139284,31.59142718)(190.87139285,31.54142723)(190.85139282,31.49142873)
\curveto(190.81139291,31.36142741)(190.72639299,31.28642748)(190.59639282,31.26642873)
\curveto(190.46639325,31.25642751)(190.3163934,31.25142752)(190.14639282,31.25142873)
\lineto(188.40639282,31.25142873)
\lineto(182.40639282,31.25142873)
\lineto(180.99639282,31.25142873)
\curveto(180.88640283,31.25142752)(180.77140295,31.24642752)(180.65139282,31.23642873)
\curveto(180.53140319,31.23642753)(180.43640328,31.26142751)(180.36639282,31.31142873)
\curveto(180.30640341,31.35142742)(180.25640346,31.42642734)(180.21639282,31.53642873)
\curveto(180.20640351,31.55642721)(180.20640351,31.57642719)(180.21639282,31.59642873)
\curveto(180.2164035,31.62642714)(180.21140351,31.65142712)(180.20139282,31.67142873)
}
}
{
\newrgbcolor{curcolor}{0 0 0}
\pscustom[linestyle=none,fillstyle=solid,fillcolor=curcolor]
{
\newpath
\moveto(190.32639282,50.87353811)
\curveto(190.48639323,50.90353028)(190.6213931,50.88853029)(190.73139282,50.82853811)
\curveto(190.83139289,50.76853041)(190.90639281,50.68853049)(190.95639282,50.58853811)
\curveto(190.97639274,50.53853064)(190.98639273,50.4835307)(190.98639282,50.42353811)
\curveto(190.98639273,50.37353081)(190.99639272,50.31853086)(191.01639282,50.25853811)
\curveto(191.06639265,50.03853114)(191.05139267,49.81853136)(190.97139282,49.59853811)
\curveto(190.90139282,49.38853179)(190.81139291,49.24353194)(190.70139282,49.16353811)
\curveto(190.63139309,49.11353207)(190.55139317,49.06853211)(190.46139282,49.02853811)
\curveto(190.36139336,48.98853219)(190.28139344,48.93853224)(190.22139282,48.87853811)
\curveto(190.20139352,48.85853232)(190.18139354,48.83353235)(190.16139282,48.80353811)
\curveto(190.14139358,48.7835324)(190.13639358,48.75353243)(190.14639282,48.71353811)
\curveto(190.17639354,48.60353258)(190.23139349,48.49853268)(190.31139282,48.39853811)
\curveto(190.39139333,48.30853287)(190.46139326,48.21853296)(190.52139282,48.12853811)
\curveto(190.60139312,47.99853318)(190.67639304,47.85853332)(190.74639282,47.70853811)
\curveto(190.80639291,47.55853362)(190.86139286,47.39853378)(190.91139282,47.22853811)
\curveto(190.94139278,47.12853405)(190.96139276,47.01853416)(190.97139282,46.89853811)
\curveto(190.98139274,46.78853439)(190.99639272,46.6785345)(191.01639282,46.56853811)
\curveto(191.02639269,46.51853466)(191.03139269,46.47353471)(191.03139282,46.43353811)
\lineto(191.03139282,46.32853811)
\curveto(191.05139267,46.21853496)(191.05139267,46.11353507)(191.03139282,46.01353811)
\lineto(191.03139282,45.87853811)
\curveto(191.0213927,45.82853535)(191.0163927,45.7785354)(191.01639282,45.72853811)
\curveto(191.0163927,45.6785355)(191.00639271,45.63353555)(190.98639282,45.59353811)
\curveto(190.97639274,45.55353563)(190.97139275,45.51853566)(190.97139282,45.48853811)
\curveto(190.98139274,45.46853571)(190.98139274,45.44353574)(190.97139282,45.41353811)
\lineto(190.91139282,45.17353811)
\curveto(190.90139282,45.09353609)(190.88139284,45.01853616)(190.85139282,44.94853811)
\curveto(190.721393,44.64853653)(190.57639314,44.40353678)(190.41639282,44.21353811)
\curveto(190.24639347,44.03353715)(190.01139371,43.8835373)(189.71139282,43.76353811)
\curveto(189.49139423,43.67353751)(189.22639449,43.62853755)(188.91639282,43.62853811)
\lineto(188.60139282,43.62853811)
\curveto(188.55139517,43.63853754)(188.50139522,43.64353754)(188.45139282,43.64353811)
\lineto(188.27139282,43.67353811)
\lineto(187.94139282,43.79353811)
\curveto(187.83139589,43.83353735)(187.73139599,43.8835373)(187.64139282,43.94353811)
\curveto(187.35139637,44.12353706)(187.13639658,44.36853681)(186.99639282,44.67853811)
\curveto(186.85639686,44.98853619)(186.73139699,45.32853585)(186.62139282,45.69853811)
\curveto(186.58139714,45.83853534)(186.55139717,45.9835352)(186.53139282,46.13353811)
\curveto(186.51139721,46.2835349)(186.48639723,46.43353475)(186.45639282,46.58353811)
\curveto(186.43639728,46.65353453)(186.42639729,46.71853446)(186.42639282,46.77853811)
\curveto(186.42639729,46.84853433)(186.4163973,46.92353426)(186.39639282,47.00353811)
\curveto(186.37639734,47.07353411)(186.36639735,47.14353404)(186.36639282,47.21353811)
\curveto(186.35639736,47.2835339)(186.34139738,47.35853382)(186.32139282,47.43853811)
\curveto(186.26139746,47.68853349)(186.21139751,47.92353326)(186.17139282,48.14353811)
\curveto(186.1213976,48.36353282)(186.00639771,48.53853264)(185.82639282,48.66853811)
\curveto(185.74639797,48.72853245)(185.64639807,48.7785324)(185.52639282,48.81853811)
\curveto(185.39639832,48.85853232)(185.25639846,48.85853232)(185.10639282,48.81853811)
\curveto(184.86639885,48.75853242)(184.67639904,48.66853251)(184.53639282,48.54853811)
\curveto(184.39639932,48.43853274)(184.28639943,48.2785329)(184.20639282,48.06853811)
\curveto(184.15639956,47.94853323)(184.1213996,47.80353338)(184.10139282,47.63353811)
\curveto(184.08139964,47.47353371)(184.07139965,47.30353388)(184.07139282,47.12353811)
\curveto(184.07139965,46.94353424)(184.08139964,46.76853441)(184.10139282,46.59853811)
\curveto(184.1213996,46.42853475)(184.15139957,46.2835349)(184.19139282,46.16353811)
\curveto(184.25139947,45.99353519)(184.33639938,45.82853535)(184.44639282,45.66853811)
\curveto(184.50639921,45.58853559)(184.58639913,45.51353567)(184.68639282,45.44353811)
\curveto(184.77639894,45.3835358)(184.87639884,45.32853585)(184.98639282,45.27853811)
\curveto(185.06639865,45.24853593)(185.15139857,45.21853596)(185.24139282,45.18853811)
\curveto(185.33139839,45.16853601)(185.40139832,45.12353606)(185.45139282,45.05353811)
\curveto(185.48139824,45.01353617)(185.50639821,44.94353624)(185.52639282,44.84353811)
\curveto(185.53639818,44.75353643)(185.54139818,44.65853652)(185.54139282,44.55853811)
\curveto(185.54139818,44.45853672)(185.53639818,44.35853682)(185.52639282,44.25853811)
\curveto(185.50639821,44.16853701)(185.48139824,44.10353708)(185.45139282,44.06353811)
\curveto(185.4213983,44.02353716)(185.37139835,43.99353719)(185.30139282,43.97353811)
\curveto(185.23139849,43.95353723)(185.15639856,43.95353723)(185.07639282,43.97353811)
\curveto(184.94639877,44.00353718)(184.82639889,44.03353715)(184.71639282,44.06353811)
\curveto(184.59639912,44.10353708)(184.48139924,44.14853703)(184.37139282,44.19853811)
\curveto(184.0213997,44.38853679)(183.75139997,44.62853655)(183.56139282,44.91853811)
\curveto(183.36140036,45.20853597)(183.20140052,45.56853561)(183.08139282,45.99853811)
\curveto(183.06140066,46.09853508)(183.04640067,46.19853498)(183.03639282,46.29853811)
\curveto(183.02640069,46.40853477)(183.01140071,46.51853466)(182.99139282,46.62853811)
\curveto(182.98140074,46.66853451)(182.98140074,46.73353445)(182.99139282,46.82353811)
\curveto(182.99140073,46.91353427)(182.98140074,46.96853421)(182.96139282,46.98853811)
\curveto(182.95140077,47.68853349)(183.03140069,48.29853288)(183.20139282,48.81853811)
\curveto(183.37140035,49.33853184)(183.69640002,49.70353148)(184.17639282,49.91353811)
\curveto(184.37639934,50.00353118)(184.61139911,50.05353113)(184.88139282,50.06353811)
\curveto(185.14139858,50.0835311)(185.4163983,50.09353109)(185.70639282,50.09353811)
\lineto(189.02139282,50.09353811)
\curveto(189.16139456,50.09353109)(189.29639442,50.09853108)(189.42639282,50.10853811)
\curveto(189.55639416,50.11853106)(189.66139406,50.14853103)(189.74139282,50.19853811)
\curveto(189.81139391,50.24853093)(189.86139386,50.31353087)(189.89139282,50.39353811)
\curveto(189.93139379,50.4835307)(189.96139376,50.56853061)(189.98139282,50.64853811)
\curveto(189.99139373,50.72853045)(190.03639368,50.78853039)(190.11639282,50.82853811)
\curveto(190.14639357,50.84853033)(190.17639354,50.85853032)(190.20639282,50.85853811)
\curveto(190.23639348,50.85853032)(190.27639344,50.86353032)(190.32639282,50.87353811)
\moveto(188.66139282,48.72853811)
\curveto(188.5213952,48.78853239)(188.36139536,48.81853236)(188.18139282,48.81853811)
\curveto(187.99139573,48.82853235)(187.79639592,48.83353235)(187.59639282,48.83353811)
\curveto(187.48639623,48.83353235)(187.38639633,48.82853235)(187.29639282,48.81853811)
\curveto(187.20639651,48.80853237)(187.13639658,48.76853241)(187.08639282,48.69853811)
\curveto(187.06639665,48.66853251)(187.05639666,48.59853258)(187.05639282,48.48853811)
\curveto(187.07639664,48.46853271)(187.08639663,48.43353275)(187.08639282,48.38353811)
\curveto(187.08639663,48.33353285)(187.09639662,48.28853289)(187.11639282,48.24853811)
\curveto(187.13639658,48.16853301)(187.15639656,48.0785331)(187.17639282,47.97853811)
\lineto(187.23639282,47.67853811)
\curveto(187.23639648,47.64853353)(187.24139648,47.61353357)(187.25139282,47.57353811)
\lineto(187.25139282,47.46853811)
\curveto(187.29139643,47.31853386)(187.3163964,47.15353403)(187.32639282,46.97353811)
\curveto(187.32639639,46.80353438)(187.34639637,46.64353454)(187.38639282,46.49353811)
\curveto(187.40639631,46.41353477)(187.42639629,46.33853484)(187.44639282,46.26853811)
\curveto(187.45639626,46.20853497)(187.47139625,46.13853504)(187.49139282,46.05853811)
\curveto(187.54139618,45.89853528)(187.60639611,45.74853543)(187.68639282,45.60853811)
\curveto(187.75639596,45.46853571)(187.84639587,45.34853583)(187.95639282,45.24853811)
\curveto(188.06639565,45.14853603)(188.20139552,45.07353611)(188.36139282,45.02353811)
\curveto(188.51139521,44.97353621)(188.69639502,44.95353623)(188.91639282,44.96353811)
\curveto(189.0163947,44.96353622)(189.11139461,44.9785362)(189.20139282,45.00853811)
\curveto(189.28139444,45.04853613)(189.35639436,45.09353609)(189.42639282,45.14353811)
\curveto(189.53639418,45.22353596)(189.63139409,45.32853585)(189.71139282,45.45853811)
\curveto(189.78139394,45.58853559)(189.84139388,45.72853545)(189.89139282,45.87853811)
\curveto(189.90139382,45.92853525)(189.90639381,45.9785352)(189.90639282,46.02853811)
\curveto(189.90639381,46.0785351)(189.91139381,46.12853505)(189.92139282,46.17853811)
\curveto(189.94139378,46.24853493)(189.95639376,46.33353485)(189.96639282,46.43353811)
\curveto(189.96639375,46.54353464)(189.95639376,46.63353455)(189.93639282,46.70353811)
\curveto(189.9163938,46.76353442)(189.91139381,46.82353436)(189.92139282,46.88353811)
\curveto(189.9213938,46.94353424)(189.91139381,47.00353418)(189.89139282,47.06353811)
\curveto(189.87139385,47.14353404)(189.85639386,47.21853396)(189.84639282,47.28853811)
\curveto(189.83639388,47.36853381)(189.8163939,47.44353374)(189.78639282,47.51353811)
\curveto(189.66639405,47.80353338)(189.5213942,48.04853313)(189.35139282,48.24853811)
\curveto(189.18139454,48.45853272)(188.95139477,48.61853256)(188.66139282,48.72853811)
}
}
{
\newrgbcolor{curcolor}{0 0 0}
\pscustom[linestyle=none,fillstyle=solid,fillcolor=curcolor]
{
\newpath
\moveto(182.97639282,55.69017873)
\curveto(182.97640074,55.92017394)(183.03640068,56.05017381)(183.15639282,56.08017873)
\curveto(183.26640045,56.11017375)(183.43140029,56.12517374)(183.65139282,56.12517873)
\lineto(183.93639282,56.12517873)
\curveto(184.02639969,56.12517374)(184.10139962,56.10017376)(184.16139282,56.05017873)
\curveto(184.24139948,55.99017387)(184.28639943,55.90517396)(184.29639282,55.79517873)
\curveto(184.29639942,55.68517418)(184.31139941,55.57517429)(184.34139282,55.46517873)
\curveto(184.37139935,55.32517454)(184.40139932,55.19017467)(184.43139282,55.06017873)
\curveto(184.46139926,54.94017492)(184.50139922,54.82517504)(184.55139282,54.71517873)
\curveto(184.68139904,54.42517544)(184.86139886,54.19017567)(185.09139282,54.01017873)
\curveto(185.31139841,53.83017603)(185.56639815,53.67517619)(185.85639282,53.54517873)
\curveto(185.96639775,53.50517636)(186.08139764,53.47517639)(186.20139282,53.45517873)
\curveto(186.31139741,53.43517643)(186.42639729,53.41017645)(186.54639282,53.38017873)
\curveto(186.59639712,53.37017649)(186.64639707,53.3651765)(186.69639282,53.36517873)
\curveto(186.74639697,53.37517649)(186.79639692,53.37517649)(186.84639282,53.36517873)
\curveto(186.96639675,53.33517653)(187.10639661,53.32017654)(187.26639282,53.32017873)
\curveto(187.4163963,53.33017653)(187.56139616,53.33517653)(187.70139282,53.33517873)
\lineto(189.54639282,53.33517873)
\lineto(189.89139282,53.33517873)
\curveto(190.01139371,53.33517653)(190.12639359,53.33017653)(190.23639282,53.32017873)
\curveto(190.34639337,53.31017655)(190.44139328,53.30517656)(190.52139282,53.30517873)
\curveto(190.60139312,53.31517655)(190.67139305,53.29517657)(190.73139282,53.24517873)
\curveto(190.80139292,53.19517667)(190.84139288,53.11517675)(190.85139282,53.00517873)
\curveto(190.86139286,52.90517696)(190.86639285,52.79517707)(190.86639282,52.67517873)
\lineto(190.86639282,52.40517873)
\curveto(190.84639287,52.35517751)(190.83139289,52.30517756)(190.82139282,52.25517873)
\curveto(190.80139292,52.21517765)(190.77639294,52.18517768)(190.74639282,52.16517873)
\curveto(190.67639304,52.11517775)(190.59139313,52.08517778)(190.49139282,52.07517873)
\lineto(190.16139282,52.07517873)
\lineto(189.00639282,52.07517873)
\lineto(184.85139282,52.07517873)
\lineto(183.81639282,52.07517873)
\lineto(183.51639282,52.07517873)
\curveto(183.4164003,52.08517778)(183.33140039,52.11517775)(183.26139282,52.16517873)
\curveto(183.2214005,52.19517767)(183.19140053,52.24517762)(183.17139282,52.31517873)
\curveto(183.15140057,52.39517747)(183.14140058,52.48017738)(183.14139282,52.57017873)
\curveto(183.13140059,52.6601772)(183.13140059,52.75017711)(183.14139282,52.84017873)
\curveto(183.15140057,52.93017693)(183.16640055,53.00017686)(183.18639282,53.05017873)
\curveto(183.2164005,53.13017673)(183.27640044,53.18017668)(183.36639282,53.20017873)
\curveto(183.44640027,53.23017663)(183.53640018,53.24517662)(183.63639282,53.24517873)
\lineto(183.93639282,53.24517873)
\curveto(184.03639968,53.24517662)(184.12639959,53.2651766)(184.20639282,53.30517873)
\curveto(184.22639949,53.31517655)(184.24139948,53.32517654)(184.25139282,53.33517873)
\lineto(184.29639282,53.38017873)
\curveto(184.29639942,53.49017637)(184.25139947,53.58017628)(184.16139282,53.65017873)
\curveto(184.06139966,53.72017614)(183.98139974,53.78017608)(183.92139282,53.83017873)
\lineto(183.83139282,53.92017873)
\curveto(183.7214,54.01017585)(183.60640011,54.13517573)(183.48639282,54.29517873)
\curveto(183.36640035,54.45517541)(183.27640044,54.60517526)(183.21639282,54.74517873)
\curveto(183.16640055,54.83517503)(183.13140059,54.93017493)(183.11139282,55.03017873)
\curveto(183.08140064,55.13017473)(183.05140067,55.23517463)(183.02139282,55.34517873)
\curveto(183.01140071,55.40517446)(183.00640071,55.4651744)(183.00639282,55.52517873)
\curveto(182.99640072,55.58517428)(182.98640073,55.64017422)(182.97639282,55.69017873)
}
}
{
\newrgbcolor{curcolor}{0 0 0}
\pscustom[linestyle=none,fillstyle=solid,fillcolor=curcolor]
{
}
}
{
\newrgbcolor{curcolor}{0 0 0}
\pscustom[linestyle=none,fillstyle=solid,fillcolor=curcolor]
{
\newpath
\moveto(180.27639282,65.05510061)
\curveto(180.27640344,65.15509575)(180.28640343,65.25009566)(180.30639282,65.34010061)
\curveto(180.3164034,65.43009548)(180.34640337,65.49509541)(180.39639282,65.53510061)
\curveto(180.47640324,65.59509531)(180.58140314,65.62509528)(180.71139282,65.62510061)
\lineto(181.10139282,65.62510061)
\lineto(182.60139282,65.62510061)
\lineto(188.99139282,65.62510061)
\lineto(190.16139282,65.62510061)
\lineto(190.47639282,65.62510061)
\curveto(190.57639314,65.63509527)(190.65639306,65.62009529)(190.71639282,65.58010061)
\curveto(190.79639292,65.53009538)(190.84639287,65.45509545)(190.86639282,65.35510061)
\curveto(190.87639284,65.26509564)(190.88139284,65.15509575)(190.88139282,65.02510061)
\lineto(190.88139282,64.80010061)
\curveto(190.86139286,64.72009619)(190.84639287,64.65009626)(190.83639282,64.59010061)
\curveto(190.8163929,64.53009638)(190.77639294,64.48009643)(190.71639282,64.44010061)
\curveto(190.65639306,64.40009651)(190.58139314,64.38009653)(190.49139282,64.38010061)
\lineto(190.19139282,64.38010061)
\lineto(189.09639282,64.38010061)
\lineto(183.75639282,64.38010061)
\curveto(183.66640005,64.36009655)(183.59140013,64.34509656)(183.53139282,64.33510061)
\curveto(183.46140026,64.33509657)(183.40140032,64.3050966)(183.35139282,64.24510061)
\curveto(183.30140042,64.17509673)(183.27640044,64.08509682)(183.27639282,63.97510061)
\curveto(183.26640045,63.87509703)(183.26140046,63.76509714)(183.26139282,63.64510061)
\lineto(183.26139282,62.50510061)
\lineto(183.26139282,62.01010061)
\curveto(183.25140047,61.85009906)(183.19140053,61.74009917)(183.08139282,61.68010061)
\curveto(183.05140067,61.66009925)(183.0214007,61.65009926)(182.99139282,61.65010061)
\curveto(182.95140077,61.65009926)(182.90640081,61.64509926)(182.85639282,61.63510061)
\curveto(182.73640098,61.61509929)(182.62640109,61.62009929)(182.52639282,61.65010061)
\curveto(182.42640129,61.69009922)(182.35640136,61.74509916)(182.31639282,61.81510061)
\curveto(182.26640145,61.89509901)(182.24140148,62.01509889)(182.24139282,62.17510061)
\curveto(182.24140148,62.33509857)(182.22640149,62.47009844)(182.19639282,62.58010061)
\curveto(182.18640153,62.63009828)(182.18140154,62.68509822)(182.18139282,62.74510061)
\curveto(182.17140155,62.8050981)(182.15640156,62.86509804)(182.13639282,62.92510061)
\curveto(182.08640163,63.07509783)(182.03640168,63.22009769)(181.98639282,63.36010061)
\curveto(181.92640179,63.50009741)(181.85640186,63.63509727)(181.77639282,63.76510061)
\curveto(181.68640203,63.905097)(181.58140214,64.02509688)(181.46139282,64.12510061)
\curveto(181.34140238,64.22509668)(181.21140251,64.32009659)(181.07139282,64.41010061)
\curveto(180.97140275,64.47009644)(180.86140286,64.51509639)(180.74139282,64.54510061)
\curveto(180.6214031,64.58509632)(180.5164032,64.63509627)(180.42639282,64.69510061)
\curveto(180.36640335,64.74509616)(180.32640339,64.81509609)(180.30639282,64.90510061)
\curveto(180.29640342,64.92509598)(180.29140343,64.95009596)(180.29139282,64.98010061)
\curveto(180.29140343,65.0100959)(180.28640343,65.03509587)(180.27639282,65.05510061)
}
}
{
\newrgbcolor{curcolor}{0 0 0}
\pscustom[linestyle=none,fillstyle=solid,fillcolor=curcolor]
{
\newpath
\moveto(187.38639282,76.22470998)
\curveto(187.43639628,76.29470234)(187.50639621,76.3347023)(187.59639282,76.34470998)
\curveto(187.68639603,76.36470227)(187.79139593,76.37470226)(187.91139282,76.37470998)
\curveto(187.96139576,76.37470226)(188.01139571,76.36970226)(188.06139282,76.35970998)
\curveto(188.11139561,76.35970227)(188.15639556,76.34970228)(188.19639282,76.32970998)
\curveto(188.28639543,76.29970233)(188.34639537,76.23970239)(188.37639282,76.14970998)
\curveto(188.39639532,76.06970256)(188.40639531,75.97470266)(188.40639282,75.86470998)
\lineto(188.40639282,75.54970998)
\curveto(188.39639532,75.43970319)(188.40639531,75.3347033)(188.43639282,75.23470998)
\curveto(188.46639525,75.09470354)(188.54639517,75.00470363)(188.67639282,74.96470998)
\curveto(188.74639497,74.94470369)(188.83139489,74.9347037)(188.93139282,74.93470998)
\lineto(189.20139282,74.93470998)
\lineto(190.14639282,74.93470998)
\lineto(190.47639282,74.93470998)
\curveto(190.58639313,74.9347037)(190.67139305,74.91470372)(190.73139282,74.87470998)
\curveto(190.79139293,74.8347038)(190.83139289,74.78470385)(190.85139282,74.72470998)
\curveto(190.86139286,74.67470396)(190.87639284,74.60970402)(190.89639282,74.52970998)
\lineto(190.89639282,74.33470998)
\curveto(190.89639282,74.21470442)(190.89139283,74.10970452)(190.88139282,74.01970998)
\curveto(190.86139286,73.9297047)(190.81139291,73.85970477)(190.73139282,73.80970998)
\curveto(190.68139304,73.77970485)(190.61139311,73.76470487)(190.52139282,73.76470998)
\lineto(190.22139282,73.76470998)
\lineto(189.18639282,73.76470998)
\curveto(189.02639469,73.76470487)(188.88139484,73.75470488)(188.75139282,73.73470998)
\curveto(188.61139511,73.72470491)(188.5163952,73.66970496)(188.46639282,73.56970998)
\curveto(188.44639527,73.51970511)(188.43139529,73.44970518)(188.42139282,73.35970998)
\curveto(188.41139531,73.27970535)(188.40639531,73.18970544)(188.40639282,73.08970998)
\lineto(188.40639282,72.80470998)
\lineto(188.40639282,72.56470998)
\lineto(188.40639282,70.29970998)
\curveto(188.40639531,70.20970842)(188.41139531,70.10470853)(188.42139282,69.98470998)
\lineto(188.42139282,69.65470998)
\curveto(188.4213953,69.54470909)(188.41139531,69.44470919)(188.39139282,69.35470998)
\curveto(188.37139535,69.26470937)(188.33639538,69.20470943)(188.28639282,69.17470998)
\curveto(188.2163955,69.12470951)(188.1213956,69.09970953)(188.00139282,69.09970998)
\lineto(187.65639282,69.09970998)
\lineto(187.38639282,69.09970998)
\curveto(187.2163965,69.13970949)(187.07639664,69.19470944)(186.96639282,69.26470998)
\curveto(186.85639686,69.3347093)(186.74139698,69.41470922)(186.62139282,69.50470998)
\lineto(186.08139282,69.86470998)
\curveto(185.45139827,70.30470833)(184.83139889,70.73970789)(184.22139282,71.16970998)
\lineto(182.36139282,72.48970998)
\curveto(182.13140159,72.64970598)(181.91140181,72.80470583)(181.70139282,72.95470998)
\curveto(181.48140224,73.10470553)(181.25640246,73.25970537)(181.02639282,73.41970998)
\curveto(180.95640276,73.46970516)(180.89140283,73.51970511)(180.83139282,73.56970998)
\curveto(180.76140296,73.61970501)(180.68640303,73.66970496)(180.60639282,73.71970998)
\lineto(180.51639282,73.77970998)
\curveto(180.47640324,73.80970482)(180.44640327,73.83970479)(180.42639282,73.86970998)
\curveto(180.39640332,73.90970472)(180.37640334,73.94970468)(180.36639282,73.98970998)
\curveto(180.34640337,74.0297046)(180.32640339,74.07470456)(180.30639282,74.12470998)
\curveto(180.30640341,74.14470449)(180.31140341,74.16470447)(180.32139282,74.18470998)
\curveto(180.3214034,74.21470442)(180.31140341,74.23970439)(180.29139282,74.25970998)
\curveto(180.29140343,74.38970424)(180.29640342,74.50970412)(180.30639282,74.61970998)
\curveto(180.3164034,74.7297039)(180.36140336,74.80970382)(180.44139282,74.85970998)
\curveto(180.49140323,74.89970373)(180.56140316,74.91970371)(180.65139282,74.91970998)
\curveto(180.74140298,74.9297037)(180.83640288,74.9347037)(180.93639282,74.93470998)
\lineto(186.39639282,74.93470998)
\curveto(186.46639725,74.9347037)(186.54139718,74.9297037)(186.62139282,74.91970998)
\curveto(186.69139703,74.91970371)(186.76139696,74.92470371)(186.83139282,74.93470998)
\lineto(186.93639282,74.93470998)
\curveto(186.98639673,74.95470368)(187.04139668,74.96970366)(187.10139282,74.97970998)
\curveto(187.15139657,74.98970364)(187.19139653,75.01470362)(187.22139282,75.05470998)
\curveto(187.27139645,75.12470351)(187.30139642,75.20970342)(187.31139282,75.30970998)
\lineto(187.31139282,75.63970998)
\curveto(187.31139641,75.74970288)(187.3163964,75.85470278)(187.32639282,75.95470998)
\curveto(187.32639639,76.06470257)(187.34639637,76.15470248)(187.38639282,76.22470998)
\moveto(187.19139282,73.65970998)
\curveto(187.08139664,73.73970489)(186.91139681,73.77470486)(186.68139282,73.76470998)
\lineto(186.06639282,73.76470998)
\lineto(183.59139282,73.76470998)
\lineto(183.27639282,73.76470998)
\curveto(183.15640056,73.77470486)(183.05640066,73.76970486)(182.97639282,73.74970998)
\lineto(182.82639282,73.74970998)
\curveto(182.73640098,73.74970488)(182.65140107,73.7347049)(182.57139282,73.70470998)
\curveto(182.55140117,73.69470494)(182.54140118,73.68470495)(182.54139282,73.67470998)
\lineto(182.49639282,73.62970998)
\curveto(182.48640123,73.60970502)(182.48140124,73.57970505)(182.48139282,73.53970998)
\curveto(182.50140122,73.51970511)(182.5164012,73.49970513)(182.52639282,73.47970998)
\curveto(182.52640119,73.46970516)(182.53140119,73.45470518)(182.54139282,73.43470998)
\curveto(182.59140113,73.37470526)(182.66140106,73.31470532)(182.75139282,73.25470998)
\curveto(182.84140088,73.19470544)(182.9214008,73.13970549)(182.99139282,73.08970998)
\curveto(183.13140059,72.98970564)(183.27640044,72.89470574)(183.42639282,72.80470998)
\curveto(183.56640015,72.71470592)(183.70640001,72.61970601)(183.84639282,72.51970998)
\lineto(184.62639282,71.97970998)
\curveto(184.88639883,71.80970682)(185.14639857,71.634707)(185.40639282,71.45470998)
\curveto(185.5163982,71.37470726)(185.6213981,71.29970733)(185.72139282,71.22970998)
\lineto(186.02139282,71.01970998)
\curveto(186.10139762,70.96970766)(186.17639754,70.91970771)(186.24639282,70.86970998)
\curveto(186.3163974,70.8297078)(186.39139733,70.78470785)(186.47139282,70.73470998)
\curveto(186.53139719,70.68470795)(186.59639712,70.634708)(186.66639282,70.58470998)
\curveto(186.72639699,70.54470809)(186.79639692,70.50470813)(186.87639282,70.46470998)
\curveto(186.93639678,70.42470821)(187.00639671,70.39970823)(187.08639282,70.38970998)
\curveto(187.15639656,70.37970825)(187.21139651,70.41470822)(187.25139282,70.49470998)
\curveto(187.30139642,70.56470807)(187.32639639,70.67470796)(187.32639282,70.82470998)
\curveto(187.3163964,70.98470765)(187.31139641,71.11970751)(187.31139282,71.22970998)
\lineto(187.31139282,72.90970998)
\lineto(187.31139282,73.34470998)
\curveto(187.31139641,73.49470514)(187.27139645,73.59970503)(187.19139282,73.65970998)
}
}
{
\newrgbcolor{curcolor}{0 0 0}
\pscustom[linestyle=none,fillstyle=solid,fillcolor=curcolor]
{
\newpath
\moveto(189.24639282,78.63431936)
\lineto(189.24639282,79.26431936)
\lineto(189.24639282,79.45931936)
\curveto(189.24639447,79.52931683)(189.25639446,79.58931677)(189.27639282,79.63931936)
\curveto(189.3163944,79.70931665)(189.35639436,79.7593166)(189.39639282,79.78931936)
\curveto(189.44639427,79.82931653)(189.51139421,79.84931651)(189.59139282,79.84931936)
\curveto(189.67139405,79.8593165)(189.75639396,79.86431649)(189.84639282,79.86431936)
\lineto(190.56639282,79.86431936)
\curveto(191.04639267,79.86431649)(191.45639226,79.80431655)(191.79639282,79.68431936)
\curveto(192.13639158,79.56431679)(192.41139131,79.36931699)(192.62139282,79.09931936)
\curveto(192.67139105,79.02931733)(192.716391,78.9593174)(192.75639282,78.88931936)
\curveto(192.80639091,78.82931753)(192.85139087,78.7543176)(192.89139282,78.66431936)
\curveto(192.90139082,78.64431771)(192.91139081,78.61431774)(192.92139282,78.57431936)
\curveto(192.94139078,78.53431782)(192.94639077,78.48931787)(192.93639282,78.43931936)
\curveto(192.90639081,78.34931801)(192.83139089,78.29431806)(192.71139282,78.27431936)
\curveto(192.60139112,78.2543181)(192.50639121,78.26931809)(192.42639282,78.31931936)
\curveto(192.35639136,78.34931801)(192.29139143,78.39431796)(192.23139282,78.45431936)
\curveto(192.18139154,78.52431783)(192.13139159,78.58931777)(192.08139282,78.64931936)
\curveto(192.03139169,78.71931764)(191.95639176,78.77931758)(191.85639282,78.82931936)
\curveto(191.76639195,78.88931747)(191.67639204,78.93931742)(191.58639282,78.97931936)
\curveto(191.55639216,78.99931736)(191.49639222,79.02431733)(191.40639282,79.05431936)
\curveto(191.32639239,79.08431727)(191.25639246,79.08931727)(191.19639282,79.06931936)
\curveto(191.05639266,79.03931732)(190.96639275,78.97931738)(190.92639282,78.88931936)
\curveto(190.89639282,78.80931755)(190.88139284,78.71931764)(190.88139282,78.61931936)
\curveto(190.88139284,78.51931784)(190.85639286,78.43431792)(190.80639282,78.36431936)
\curveto(190.73639298,78.27431808)(190.59639312,78.22931813)(190.38639282,78.22931936)
\lineto(189.83139282,78.22931936)
\lineto(189.60639282,78.22931936)
\curveto(189.52639419,78.23931812)(189.46139426,78.2593181)(189.41139282,78.28931936)
\curveto(189.33139439,78.34931801)(189.28639443,78.41931794)(189.27639282,78.49931936)
\curveto(189.26639445,78.51931784)(189.26139446,78.53931782)(189.26139282,78.55931936)
\curveto(189.26139446,78.58931777)(189.25639446,78.61431774)(189.24639282,78.63431936)
}
}
{
\newrgbcolor{curcolor}{0 0 0}
\pscustom[linestyle=none,fillstyle=solid,fillcolor=curcolor]
{
}
}
{
\newrgbcolor{curcolor}{0 0 0}
\pscustom[linestyle=none,fillstyle=solid,fillcolor=curcolor]
{
\newpath
\moveto(180.27639282,89.26463186)
\curveto(180.26640345,89.95462722)(180.38640333,90.55462662)(180.63639282,91.06463186)
\curveto(180.88640283,91.58462559)(181.2214025,91.9796252)(181.64139282,92.24963186)
\curveto(181.721402,92.29962488)(181.81140191,92.34462483)(181.91139282,92.38463186)
\curveto(182.00140172,92.42462475)(182.09640162,92.46962471)(182.19639282,92.51963186)
\curveto(182.29640142,92.55962462)(182.39640132,92.58962459)(182.49639282,92.60963186)
\curveto(182.59640112,92.62962455)(182.70140102,92.64962453)(182.81139282,92.66963186)
\curveto(182.86140086,92.68962449)(182.90640081,92.69462448)(182.94639282,92.68463186)
\curveto(182.98640073,92.6746245)(183.03140069,92.6796245)(183.08139282,92.69963186)
\curveto(183.13140059,92.70962447)(183.2164005,92.71462446)(183.33639282,92.71463186)
\curveto(183.44640027,92.71462446)(183.53140019,92.70962447)(183.59139282,92.69963186)
\curveto(183.65140007,92.6796245)(183.71140001,92.66962451)(183.77139282,92.66963186)
\curveto(183.83139989,92.6796245)(183.89139983,92.6746245)(183.95139282,92.65463186)
\curveto(184.09139963,92.61462456)(184.22639949,92.5796246)(184.35639282,92.54963186)
\curveto(184.48639923,92.51962466)(184.61139911,92.4796247)(184.73139282,92.42963186)
\curveto(184.87139885,92.36962481)(184.99639872,92.29962488)(185.10639282,92.21963186)
\curveto(185.2163985,92.14962503)(185.32639839,92.0746251)(185.43639282,91.99463186)
\lineto(185.49639282,91.93463186)
\curveto(185.5163982,91.92462525)(185.53639818,91.90962527)(185.55639282,91.88963186)
\curveto(185.716398,91.76962541)(185.86139786,91.63462554)(185.99139282,91.48463186)
\curveto(186.1213976,91.33462584)(186.24639747,91.174626)(186.36639282,91.00463186)
\curveto(186.58639713,90.69462648)(186.79139693,90.39962678)(186.98139282,90.11963186)
\curveto(187.1213966,89.88962729)(187.25639646,89.65962752)(187.38639282,89.42963186)
\curveto(187.5163962,89.20962797)(187.65139607,88.98962819)(187.79139282,88.76963186)
\curveto(187.96139576,88.51962866)(188.14139558,88.2796289)(188.33139282,88.04963186)
\curveto(188.5213952,87.82962935)(188.74639497,87.63962954)(189.00639282,87.47963186)
\curveto(189.06639465,87.43962974)(189.12639459,87.40462977)(189.18639282,87.37463186)
\curveto(189.23639448,87.34462983)(189.30139442,87.31462986)(189.38139282,87.28463186)
\curveto(189.45139427,87.26462991)(189.51139421,87.25962992)(189.56139282,87.26963186)
\curveto(189.63139409,87.28962989)(189.68639403,87.32462985)(189.72639282,87.37463186)
\curveto(189.75639396,87.42462975)(189.77639394,87.48462969)(189.78639282,87.55463186)
\lineto(189.78639282,87.79463186)
\lineto(189.78639282,88.54463186)
\lineto(189.78639282,91.34963186)
\lineto(189.78639282,92.00963186)
\curveto(189.78639393,92.09962508)(189.79139393,92.18462499)(189.80139282,92.26463186)
\curveto(189.80139392,92.34462483)(189.8213939,92.40962477)(189.86139282,92.45963186)
\curveto(189.90139382,92.50962467)(189.97639374,92.54962463)(190.08639282,92.57963186)
\curveto(190.18639353,92.61962456)(190.28639343,92.62962455)(190.38639282,92.60963186)
\lineto(190.52139282,92.60963186)
\curveto(190.59139313,92.58962459)(190.65139307,92.56962461)(190.70139282,92.54963186)
\curveto(190.75139297,92.52962465)(190.79139293,92.49462468)(190.82139282,92.44463186)
\curveto(190.86139286,92.39462478)(190.88139284,92.32462485)(190.88139282,92.23463186)
\lineto(190.88139282,91.96463186)
\lineto(190.88139282,91.06463186)
\lineto(190.88139282,87.55463186)
\lineto(190.88139282,86.48963186)
\curveto(190.88139284,86.40963077)(190.88639283,86.31963086)(190.89639282,86.21963186)
\curveto(190.89639282,86.11963106)(190.88639283,86.03463114)(190.86639282,85.96463186)
\curveto(190.79639292,85.75463142)(190.6163931,85.68963149)(190.32639282,85.76963186)
\curveto(190.28639343,85.7796314)(190.25139347,85.7796314)(190.22139282,85.76963186)
\curveto(190.18139354,85.76963141)(190.13639358,85.7796314)(190.08639282,85.79963186)
\curveto(190.00639371,85.81963136)(189.9213938,85.83963134)(189.83139282,85.85963186)
\curveto(189.74139398,85.8796313)(189.65639406,85.90463127)(189.57639282,85.93463186)
\curveto(189.08639463,86.09463108)(188.67139505,86.29463088)(188.33139282,86.53463186)
\curveto(188.08139564,86.71463046)(187.85639586,86.91963026)(187.65639282,87.14963186)
\curveto(187.44639627,87.3796298)(187.25139647,87.61962956)(187.07139282,87.86963186)
\curveto(186.89139683,88.12962905)(186.721397,88.39462878)(186.56139282,88.66463186)
\curveto(186.39139733,88.94462823)(186.2163975,89.21462796)(186.03639282,89.47463186)
\curveto(185.95639776,89.58462759)(185.88139784,89.68962749)(185.81139282,89.78963186)
\curveto(185.74139798,89.89962728)(185.66639805,90.00962717)(185.58639282,90.11963186)
\curveto(185.55639816,90.15962702)(185.52639819,90.19462698)(185.49639282,90.22463186)
\curveto(185.45639826,90.26462691)(185.42639829,90.30462687)(185.40639282,90.34463186)
\curveto(185.29639842,90.48462669)(185.17139855,90.60962657)(185.03139282,90.71963186)
\curveto(185.00139872,90.73962644)(184.97639874,90.76462641)(184.95639282,90.79463186)
\curveto(184.92639879,90.82462635)(184.89639882,90.84962633)(184.86639282,90.86963186)
\curveto(184.76639895,90.94962623)(184.66639905,91.01462616)(184.56639282,91.06463186)
\curveto(184.46639925,91.12462605)(184.35639936,91.179626)(184.23639282,91.22963186)
\curveto(184.16639955,91.25962592)(184.09139963,91.2796259)(184.01139282,91.28963186)
\lineto(183.77139282,91.34963186)
\lineto(183.68139282,91.34963186)
\curveto(183.65140007,91.35962582)(183.6214001,91.36462581)(183.59139282,91.36463186)
\curveto(183.5214002,91.38462579)(183.42640029,91.38962579)(183.30639282,91.37963186)
\curveto(183.17640054,91.3796258)(183.07640064,91.36962581)(183.00639282,91.34963186)
\curveto(182.92640079,91.32962585)(182.85140087,91.30962587)(182.78139282,91.28963186)
\curveto(182.70140102,91.2796259)(182.6214011,91.25962592)(182.54139282,91.22963186)
\curveto(182.30140142,91.11962606)(182.10140162,90.96962621)(181.94139282,90.77963186)
\curveto(181.77140195,90.59962658)(181.63140209,90.3796268)(181.52139282,90.11963186)
\curveto(181.50140222,90.04962713)(181.48640223,89.9796272)(181.47639282,89.90963186)
\curveto(181.45640226,89.83962734)(181.43640228,89.76462741)(181.41639282,89.68463186)
\curveto(181.39640232,89.60462757)(181.38640233,89.49462768)(181.38639282,89.35463186)
\curveto(181.38640233,89.22462795)(181.39640232,89.11962806)(181.41639282,89.03963186)
\curveto(181.42640229,88.9796282)(181.43140229,88.92462825)(181.43139282,88.87463186)
\curveto(181.43140229,88.82462835)(181.44140228,88.7746284)(181.46139282,88.72463186)
\curveto(181.50140222,88.62462855)(181.54140218,88.52962865)(181.58139282,88.43963186)
\curveto(181.6214021,88.35962882)(181.66640205,88.2796289)(181.71639282,88.19963186)
\curveto(181.73640198,88.16962901)(181.76140196,88.13962904)(181.79139282,88.10963186)
\curveto(181.8214019,88.08962909)(181.84640187,88.06462911)(181.86639282,88.03463186)
\lineto(181.94139282,87.95963186)
\curveto(181.96140176,87.92962925)(181.98140174,87.90462927)(182.00139282,87.88463186)
\lineto(182.21139282,87.73463186)
\curveto(182.27140145,87.69462948)(182.33640138,87.64962953)(182.40639282,87.59963186)
\curveto(182.49640122,87.53962964)(182.60140112,87.48962969)(182.72139282,87.44963186)
\curveto(182.83140089,87.41962976)(182.94140078,87.38462979)(183.05139282,87.34463186)
\curveto(183.16140056,87.30462987)(183.30640041,87.2796299)(183.48639282,87.26963186)
\curveto(183.65640006,87.25962992)(183.78139994,87.22962995)(183.86139282,87.17963186)
\curveto(183.94139978,87.12963005)(183.98639973,87.05463012)(183.99639282,86.95463186)
\curveto(184.00639971,86.85463032)(184.01139971,86.74463043)(184.01139282,86.62463186)
\curveto(184.01139971,86.58463059)(184.0163997,86.54463063)(184.02639282,86.50463186)
\curveto(184.02639969,86.46463071)(184.0213997,86.42963075)(184.01139282,86.39963186)
\curveto(183.99139973,86.34963083)(183.98139974,86.29963088)(183.98139282,86.24963186)
\curveto(183.98139974,86.20963097)(183.97139975,86.16963101)(183.95139282,86.12963186)
\curveto(183.89139983,86.03963114)(183.75639996,85.99463118)(183.54639282,85.99463186)
\lineto(183.42639282,85.99463186)
\curveto(183.36640035,86.00463117)(183.30640041,86.00963117)(183.24639282,86.00963186)
\curveto(183.17640054,86.01963116)(183.11140061,86.02963115)(183.05139282,86.03963186)
\curveto(182.94140078,86.05963112)(182.84140088,86.0796311)(182.75139282,86.09963186)
\curveto(182.65140107,86.11963106)(182.55640116,86.14963103)(182.46639282,86.18963186)
\curveto(182.39640132,86.20963097)(182.33640138,86.22963095)(182.28639282,86.24963186)
\lineto(182.10639282,86.30963186)
\curveto(181.84640187,86.42963075)(181.60140212,86.58463059)(181.37139282,86.77463186)
\curveto(181.14140258,86.9746302)(180.95640276,87.18962999)(180.81639282,87.41963186)
\curveto(180.73640298,87.52962965)(180.67140305,87.64462953)(180.62139282,87.76463186)
\lineto(180.47139282,88.15463186)
\curveto(180.4214033,88.26462891)(180.39140333,88.3796288)(180.38139282,88.49963186)
\curveto(180.36140336,88.61962856)(180.33640338,88.74462843)(180.30639282,88.87463186)
\curveto(180.30640341,88.94462823)(180.30640341,89.00962817)(180.30639282,89.06963186)
\curveto(180.29640342,89.12962805)(180.28640343,89.19462798)(180.27639282,89.26463186)
}
}
{
\newrgbcolor{curcolor}{0 0 0}
\pscustom[linestyle=none,fillstyle=solid,fillcolor=curcolor]
{
\newpath
\moveto(185.79639282,101.36424123)
\lineto(186.05139282,101.36424123)
\curveto(186.13139759,101.37423353)(186.20639751,101.36923353)(186.27639282,101.34924123)
\lineto(186.51639282,101.34924123)
\lineto(186.68139282,101.34924123)
\curveto(186.78139694,101.32923357)(186.88639683,101.31923358)(186.99639282,101.31924123)
\curveto(187.09639662,101.31923358)(187.19639652,101.30923359)(187.29639282,101.28924123)
\lineto(187.44639282,101.28924123)
\curveto(187.58639613,101.25923364)(187.72639599,101.23923366)(187.86639282,101.22924123)
\curveto(187.99639572,101.21923368)(188.12639559,101.19423371)(188.25639282,101.15424123)
\curveto(188.33639538,101.13423377)(188.4213953,101.11423379)(188.51139282,101.09424123)
\lineto(188.75139282,101.03424123)
\lineto(189.05139282,100.91424123)
\curveto(189.14139458,100.88423402)(189.23139449,100.84923405)(189.32139282,100.80924123)
\curveto(189.54139418,100.70923419)(189.75639396,100.57423433)(189.96639282,100.40424123)
\curveto(190.17639354,100.24423466)(190.34639337,100.06923483)(190.47639282,99.87924123)
\curveto(190.5163932,99.82923507)(190.55639316,99.76923513)(190.59639282,99.69924123)
\curveto(190.62639309,99.63923526)(190.66139306,99.57923532)(190.70139282,99.51924123)
\curveto(190.75139297,99.43923546)(190.79139293,99.34423556)(190.82139282,99.23424123)
\curveto(190.85139287,99.12423578)(190.88139284,99.01923588)(190.91139282,98.91924123)
\curveto(190.95139277,98.80923609)(190.97639274,98.6992362)(190.98639282,98.58924123)
\curveto(190.99639272,98.47923642)(191.01139271,98.36423654)(191.03139282,98.24424123)
\curveto(191.04139268,98.2042367)(191.04139268,98.15923674)(191.03139282,98.10924123)
\curveto(191.03139269,98.06923683)(191.03639268,98.02923687)(191.04639282,97.98924123)
\curveto(191.05639266,97.94923695)(191.06139266,97.89423701)(191.06139282,97.82424123)
\curveto(191.06139266,97.75423715)(191.05639266,97.7042372)(191.04639282,97.67424123)
\curveto(191.02639269,97.62423728)(191.0213927,97.57923732)(191.03139282,97.53924123)
\curveto(191.04139268,97.4992374)(191.04139268,97.46423744)(191.03139282,97.43424123)
\lineto(191.03139282,97.34424123)
\curveto(191.01139271,97.28423762)(190.99639272,97.21923768)(190.98639282,97.14924123)
\curveto(190.98639273,97.08923781)(190.98139274,97.02423788)(190.97139282,96.95424123)
\curveto(190.9213928,96.78423812)(190.87139285,96.62423828)(190.82139282,96.47424123)
\curveto(190.77139295,96.32423858)(190.70639301,96.17923872)(190.62639282,96.03924123)
\curveto(190.58639313,95.98923891)(190.55639316,95.93423897)(190.53639282,95.87424123)
\curveto(190.50639321,95.82423908)(190.47139325,95.77423913)(190.43139282,95.72424123)
\curveto(190.25139347,95.48423942)(190.03139369,95.28423962)(189.77139282,95.12424123)
\curveto(189.51139421,94.96423994)(189.22639449,94.82424008)(188.91639282,94.70424123)
\curveto(188.77639494,94.64424026)(188.63639508,94.5992403)(188.49639282,94.56924123)
\curveto(188.34639537,94.53924036)(188.19139553,94.5042404)(188.03139282,94.46424123)
\curveto(187.9213958,94.44424046)(187.81139591,94.42924047)(187.70139282,94.41924123)
\curveto(187.59139613,94.40924049)(187.48139624,94.39424051)(187.37139282,94.37424123)
\curveto(187.33139639,94.36424054)(187.29139643,94.35924054)(187.25139282,94.35924123)
\curveto(187.21139651,94.36924053)(187.17139655,94.36924053)(187.13139282,94.35924123)
\curveto(187.08139664,94.34924055)(187.03139669,94.34424056)(186.98139282,94.34424123)
\lineto(186.81639282,94.34424123)
\curveto(186.76639695,94.32424058)(186.716397,94.31924058)(186.66639282,94.32924123)
\curveto(186.60639711,94.33924056)(186.55139717,94.33924056)(186.50139282,94.32924123)
\curveto(186.46139726,94.31924058)(186.4163973,94.31924058)(186.36639282,94.32924123)
\curveto(186.3163974,94.33924056)(186.26639745,94.33424057)(186.21639282,94.31424123)
\curveto(186.14639757,94.29424061)(186.07139765,94.28924061)(185.99139282,94.29924123)
\curveto(185.90139782,94.30924059)(185.8163979,94.31424059)(185.73639282,94.31424123)
\curveto(185.64639807,94.31424059)(185.54639817,94.30924059)(185.43639282,94.29924123)
\curveto(185.3163984,94.28924061)(185.2163985,94.29424061)(185.13639282,94.31424123)
\lineto(184.85139282,94.31424123)
\lineto(184.22139282,94.35924123)
\curveto(184.1213996,94.36924053)(184.02639969,94.37924052)(183.93639282,94.38924123)
\lineto(183.63639282,94.41924123)
\curveto(183.58640013,94.43924046)(183.53640018,94.44424046)(183.48639282,94.43424123)
\curveto(183.42640029,94.43424047)(183.37140035,94.44424046)(183.32139282,94.46424123)
\curveto(183.15140057,94.51424039)(182.98640073,94.55424035)(182.82639282,94.58424123)
\curveto(182.65640106,94.61424029)(182.49640122,94.66424024)(182.34639282,94.73424123)
\curveto(181.88640183,94.92423998)(181.51140221,95.14423976)(181.22139282,95.39424123)
\curveto(180.93140279,95.65423925)(180.68640303,96.01423889)(180.48639282,96.47424123)
\curveto(180.43640328,96.6042383)(180.40140332,96.73423817)(180.38139282,96.86424123)
\curveto(180.36140336,97.0042379)(180.33640338,97.14423776)(180.30639282,97.28424123)
\curveto(180.29640342,97.35423755)(180.29140343,97.41923748)(180.29139282,97.47924123)
\curveto(180.29140343,97.53923736)(180.28640343,97.6042373)(180.27639282,97.67424123)
\curveto(180.25640346,98.5042364)(180.40640331,99.17423573)(180.72639282,99.68424123)
\curveto(181.03640268,100.19423471)(181.47640224,100.57423433)(182.04639282,100.82424123)
\curveto(182.16640155,100.87423403)(182.29140143,100.91923398)(182.42139282,100.95924123)
\curveto(182.55140117,100.9992339)(182.68640103,101.04423386)(182.82639282,101.09424123)
\curveto(182.90640081,101.11423379)(182.99140073,101.12923377)(183.08139282,101.13924123)
\lineto(183.32139282,101.19924123)
\curveto(183.43140029,101.22923367)(183.54140018,101.24423366)(183.65139282,101.24424123)
\curveto(183.76139996,101.25423365)(183.87139985,101.26923363)(183.98139282,101.28924123)
\curveto(184.03139969,101.30923359)(184.07639964,101.31423359)(184.11639282,101.30424123)
\curveto(184.15639956,101.3042336)(184.19639952,101.30923359)(184.23639282,101.31924123)
\curveto(184.28639943,101.32923357)(184.34139938,101.32923357)(184.40139282,101.31924123)
\curveto(184.45139927,101.31923358)(184.50139922,101.32423358)(184.55139282,101.33424123)
\lineto(184.68639282,101.33424123)
\curveto(184.74639897,101.35423355)(184.8163989,101.35423355)(184.89639282,101.33424123)
\curveto(184.96639875,101.32423358)(185.03139869,101.32923357)(185.09139282,101.34924123)
\curveto(185.1213986,101.35923354)(185.16139856,101.36423354)(185.21139282,101.36424123)
\lineto(185.33139282,101.36424123)
\lineto(185.79639282,101.36424123)
\moveto(188.12139282,99.81924123)
\curveto(187.80139592,99.91923498)(187.43639628,99.97923492)(187.02639282,99.99924123)
\curveto(186.6163971,100.01923488)(186.20639751,100.02923487)(185.79639282,100.02924123)
\curveto(185.36639835,100.02923487)(184.94639877,100.01923488)(184.53639282,99.99924123)
\curveto(184.12639959,99.97923492)(183.74139998,99.93423497)(183.38139282,99.86424123)
\curveto(183.0214007,99.79423511)(182.70140102,99.68423522)(182.42139282,99.53424123)
\curveto(182.13140159,99.39423551)(181.89640182,99.1992357)(181.71639282,98.94924123)
\curveto(181.60640211,98.78923611)(181.52640219,98.60923629)(181.47639282,98.40924123)
\curveto(181.4164023,98.20923669)(181.38640233,97.96423694)(181.38639282,97.67424123)
\curveto(181.40640231,97.65423725)(181.4164023,97.61923728)(181.41639282,97.56924123)
\curveto(181.40640231,97.51923738)(181.40640231,97.47923742)(181.41639282,97.44924123)
\curveto(181.43640228,97.36923753)(181.45640226,97.29423761)(181.47639282,97.22424123)
\curveto(181.48640223,97.16423774)(181.50640221,97.0992378)(181.53639282,97.02924123)
\curveto(181.65640206,96.75923814)(181.82640189,96.53923836)(182.04639282,96.36924123)
\curveto(182.25640146,96.20923869)(182.50140122,96.07423883)(182.78139282,95.96424123)
\curveto(182.89140083,95.91423899)(183.01140071,95.87423903)(183.14139282,95.84424123)
\curveto(183.26140046,95.82423908)(183.38640033,95.7992391)(183.51639282,95.76924123)
\curveto(183.56640015,95.74923915)(183.6214001,95.73923916)(183.68139282,95.73924123)
\curveto(183.73139999,95.73923916)(183.78139994,95.73423917)(183.83139282,95.72424123)
\curveto(183.9213998,95.71423919)(184.0163997,95.7042392)(184.11639282,95.69424123)
\curveto(184.20639951,95.68423922)(184.30139942,95.67423923)(184.40139282,95.66424123)
\curveto(184.48139924,95.66423924)(184.56639915,95.65923924)(184.65639282,95.64924123)
\lineto(184.89639282,95.64924123)
\lineto(185.07639282,95.64924123)
\curveto(185.10639861,95.63923926)(185.14139858,95.63423927)(185.18139282,95.63424123)
\lineto(185.31639282,95.63424123)
\lineto(185.76639282,95.63424123)
\curveto(185.84639787,95.63423927)(185.93139779,95.62923927)(186.02139282,95.61924123)
\curveto(186.10139762,95.61923928)(186.17639754,95.62923927)(186.24639282,95.64924123)
\lineto(186.51639282,95.64924123)
\curveto(186.53639718,95.64923925)(186.56639715,95.64423926)(186.60639282,95.63424123)
\curveto(186.63639708,95.63423927)(186.66139706,95.63923926)(186.68139282,95.64924123)
\curveto(186.78139694,95.65923924)(186.88139684,95.66423924)(186.98139282,95.66424123)
\curveto(187.07139665,95.67423923)(187.17139655,95.68423922)(187.28139282,95.69424123)
\curveto(187.40139632,95.72423918)(187.52639619,95.73923916)(187.65639282,95.73924123)
\curveto(187.77639594,95.74923915)(187.89139583,95.77423913)(188.00139282,95.81424123)
\curveto(188.30139542,95.89423901)(188.56639515,95.97923892)(188.79639282,96.06924123)
\curveto(189.02639469,96.16923873)(189.24139448,96.31423859)(189.44139282,96.50424123)
\curveto(189.64139408,96.71423819)(189.79139393,96.97923792)(189.89139282,97.29924123)
\curveto(189.91139381,97.33923756)(189.9213938,97.37423753)(189.92139282,97.40424123)
\curveto(189.91139381,97.44423746)(189.9163938,97.48923741)(189.93639282,97.53924123)
\curveto(189.94639377,97.57923732)(189.95639376,97.64923725)(189.96639282,97.74924123)
\curveto(189.97639374,97.85923704)(189.97139375,97.94423696)(189.95139282,98.00424123)
\curveto(189.93139379,98.07423683)(189.9213938,98.14423676)(189.92139282,98.21424123)
\curveto(189.91139381,98.28423662)(189.89639382,98.34923655)(189.87639282,98.40924123)
\curveto(189.8163939,98.60923629)(189.73139399,98.78923611)(189.62139282,98.94924123)
\curveto(189.60139412,98.97923592)(189.58139414,99.0042359)(189.56139282,99.02424123)
\lineto(189.50139282,99.08424123)
\curveto(189.48139424,99.12423578)(189.44139428,99.17423573)(189.38139282,99.23424123)
\curveto(189.24139448,99.33423557)(189.11139461,99.41923548)(188.99139282,99.48924123)
\curveto(188.87139485,99.55923534)(188.72639499,99.62923527)(188.55639282,99.69924123)
\curveto(188.48639523,99.72923517)(188.4163953,99.74923515)(188.34639282,99.75924123)
\curveto(188.27639544,99.77923512)(188.20139552,99.7992351)(188.12139282,99.81924123)
}
}
{
\newrgbcolor{curcolor}{0 0 0}
\pscustom[linestyle=none,fillstyle=solid,fillcolor=curcolor]
{
\newpath
\moveto(180.27639282,106.77385061)
\curveto(180.27640344,106.87384575)(180.28640343,106.96884566)(180.30639282,107.05885061)
\curveto(180.3164034,107.14884548)(180.34640337,107.21384541)(180.39639282,107.25385061)
\curveto(180.47640324,107.31384531)(180.58140314,107.34384528)(180.71139282,107.34385061)
\lineto(181.10139282,107.34385061)
\lineto(182.60139282,107.34385061)
\lineto(188.99139282,107.34385061)
\lineto(190.16139282,107.34385061)
\lineto(190.47639282,107.34385061)
\curveto(190.57639314,107.35384527)(190.65639306,107.33884529)(190.71639282,107.29885061)
\curveto(190.79639292,107.24884538)(190.84639287,107.17384545)(190.86639282,107.07385061)
\curveto(190.87639284,106.98384564)(190.88139284,106.87384575)(190.88139282,106.74385061)
\lineto(190.88139282,106.51885061)
\curveto(190.86139286,106.43884619)(190.84639287,106.36884626)(190.83639282,106.30885061)
\curveto(190.8163929,106.24884638)(190.77639294,106.19884643)(190.71639282,106.15885061)
\curveto(190.65639306,106.11884651)(190.58139314,106.09884653)(190.49139282,106.09885061)
\lineto(190.19139282,106.09885061)
\lineto(189.09639282,106.09885061)
\lineto(183.75639282,106.09885061)
\curveto(183.66640005,106.07884655)(183.59140013,106.06384656)(183.53139282,106.05385061)
\curveto(183.46140026,106.05384657)(183.40140032,106.0238466)(183.35139282,105.96385061)
\curveto(183.30140042,105.89384673)(183.27640044,105.80384682)(183.27639282,105.69385061)
\curveto(183.26640045,105.59384703)(183.26140046,105.48384714)(183.26139282,105.36385061)
\lineto(183.26139282,104.22385061)
\lineto(183.26139282,103.72885061)
\curveto(183.25140047,103.56884906)(183.19140053,103.45884917)(183.08139282,103.39885061)
\curveto(183.05140067,103.37884925)(183.0214007,103.36884926)(182.99139282,103.36885061)
\curveto(182.95140077,103.36884926)(182.90640081,103.36384926)(182.85639282,103.35385061)
\curveto(182.73640098,103.33384929)(182.62640109,103.33884929)(182.52639282,103.36885061)
\curveto(182.42640129,103.40884922)(182.35640136,103.46384916)(182.31639282,103.53385061)
\curveto(182.26640145,103.61384901)(182.24140148,103.73384889)(182.24139282,103.89385061)
\curveto(182.24140148,104.05384857)(182.22640149,104.18884844)(182.19639282,104.29885061)
\curveto(182.18640153,104.34884828)(182.18140154,104.40384822)(182.18139282,104.46385061)
\curveto(182.17140155,104.5238481)(182.15640156,104.58384804)(182.13639282,104.64385061)
\curveto(182.08640163,104.79384783)(182.03640168,104.93884769)(181.98639282,105.07885061)
\curveto(181.92640179,105.21884741)(181.85640186,105.35384727)(181.77639282,105.48385061)
\curveto(181.68640203,105.623847)(181.58140214,105.74384688)(181.46139282,105.84385061)
\curveto(181.34140238,105.94384668)(181.21140251,106.03884659)(181.07139282,106.12885061)
\curveto(180.97140275,106.18884644)(180.86140286,106.23384639)(180.74139282,106.26385061)
\curveto(180.6214031,106.30384632)(180.5164032,106.35384627)(180.42639282,106.41385061)
\curveto(180.36640335,106.46384616)(180.32640339,106.53384609)(180.30639282,106.62385061)
\curveto(180.29640342,106.64384598)(180.29140343,106.66884596)(180.29139282,106.69885061)
\curveto(180.29140343,106.7288459)(180.28640343,106.75384587)(180.27639282,106.77385061)
}
}
{
\newrgbcolor{curcolor}{0 0 0}
\pscustom[linestyle=none,fillstyle=solid,fillcolor=curcolor]
{
\newpath
\moveto(180.27639282,115.12345998)
\curveto(180.27640344,115.22345513)(180.28640343,115.31845503)(180.30639282,115.40845998)
\curveto(180.3164034,115.49845485)(180.34640337,115.56345479)(180.39639282,115.60345998)
\curveto(180.47640324,115.66345469)(180.58140314,115.69345466)(180.71139282,115.69345998)
\lineto(181.10139282,115.69345998)
\lineto(182.60139282,115.69345998)
\lineto(188.99139282,115.69345998)
\lineto(190.16139282,115.69345998)
\lineto(190.47639282,115.69345998)
\curveto(190.57639314,115.70345465)(190.65639306,115.68845466)(190.71639282,115.64845998)
\curveto(190.79639292,115.59845475)(190.84639287,115.52345483)(190.86639282,115.42345998)
\curveto(190.87639284,115.33345502)(190.88139284,115.22345513)(190.88139282,115.09345998)
\lineto(190.88139282,114.86845998)
\curveto(190.86139286,114.78845556)(190.84639287,114.71845563)(190.83639282,114.65845998)
\curveto(190.8163929,114.59845575)(190.77639294,114.5484558)(190.71639282,114.50845998)
\curveto(190.65639306,114.46845588)(190.58139314,114.4484559)(190.49139282,114.44845998)
\lineto(190.19139282,114.44845998)
\lineto(189.09639282,114.44845998)
\lineto(183.75639282,114.44845998)
\curveto(183.66640005,114.42845592)(183.59140013,114.41345594)(183.53139282,114.40345998)
\curveto(183.46140026,114.40345595)(183.40140032,114.37345598)(183.35139282,114.31345998)
\curveto(183.30140042,114.24345611)(183.27640044,114.1534562)(183.27639282,114.04345998)
\curveto(183.26640045,113.94345641)(183.26140046,113.83345652)(183.26139282,113.71345998)
\lineto(183.26139282,112.57345998)
\lineto(183.26139282,112.07845998)
\curveto(183.25140047,111.91845843)(183.19140053,111.80845854)(183.08139282,111.74845998)
\curveto(183.05140067,111.72845862)(183.0214007,111.71845863)(182.99139282,111.71845998)
\curveto(182.95140077,111.71845863)(182.90640081,111.71345864)(182.85639282,111.70345998)
\curveto(182.73640098,111.68345867)(182.62640109,111.68845866)(182.52639282,111.71845998)
\curveto(182.42640129,111.75845859)(182.35640136,111.81345854)(182.31639282,111.88345998)
\curveto(182.26640145,111.96345839)(182.24140148,112.08345827)(182.24139282,112.24345998)
\curveto(182.24140148,112.40345795)(182.22640149,112.53845781)(182.19639282,112.64845998)
\curveto(182.18640153,112.69845765)(182.18140154,112.7534576)(182.18139282,112.81345998)
\curveto(182.17140155,112.87345748)(182.15640156,112.93345742)(182.13639282,112.99345998)
\curveto(182.08640163,113.14345721)(182.03640168,113.28845706)(181.98639282,113.42845998)
\curveto(181.92640179,113.56845678)(181.85640186,113.70345665)(181.77639282,113.83345998)
\curveto(181.68640203,113.97345638)(181.58140214,114.09345626)(181.46139282,114.19345998)
\curveto(181.34140238,114.29345606)(181.21140251,114.38845596)(181.07139282,114.47845998)
\curveto(180.97140275,114.53845581)(180.86140286,114.58345577)(180.74139282,114.61345998)
\curveto(180.6214031,114.6534557)(180.5164032,114.70345565)(180.42639282,114.76345998)
\curveto(180.36640335,114.81345554)(180.32640339,114.88345547)(180.30639282,114.97345998)
\curveto(180.29640342,114.99345536)(180.29140343,115.01845533)(180.29139282,115.04845998)
\curveto(180.29140343,115.07845527)(180.28640343,115.10345525)(180.27639282,115.12345998)
}
}
{
\newrgbcolor{curcolor}{0 0 0}
\pscustom[linestyle=none,fillstyle=solid,fillcolor=curcolor]
{
\newpath
\moveto(138.37873047,31.67142873)
\lineto(138.37873047,32.58642873)
\curveto(138.37874116,32.68642608)(138.37874116,32.78142599)(138.37873047,32.87142873)
\curveto(138.37874116,32.96142581)(138.39874114,33.03642573)(138.43873047,33.09642873)
\curveto(138.49874104,33.18642558)(138.57874096,33.24642552)(138.67873047,33.27642873)
\curveto(138.77874076,33.31642545)(138.88374066,33.36142541)(138.99373047,33.41142873)
\curveto(139.18374036,33.49142528)(139.37374017,33.56142521)(139.56373047,33.62142873)
\curveto(139.75373979,33.69142508)(139.9437396,33.766425)(140.13373047,33.84642873)
\curveto(140.31373923,33.91642485)(140.49873904,33.98142479)(140.68873047,34.04142873)
\curveto(140.86873867,34.10142467)(141.04873849,34.1714246)(141.22873047,34.25142873)
\curveto(141.36873817,34.31142446)(141.51373803,34.3664244)(141.66373047,34.41642873)
\curveto(141.81373773,34.4664243)(141.95873758,34.52142425)(142.09873047,34.58142873)
\curveto(142.54873699,34.76142401)(143.00373654,34.93142384)(143.46373047,35.09142873)
\curveto(143.91373563,35.25142352)(144.36373518,35.42142335)(144.81373047,35.60142873)
\curveto(144.86373468,35.62142315)(144.91373463,35.63642313)(144.96373047,35.64642873)
\lineto(145.11373047,35.70642873)
\curveto(145.33373421,35.79642297)(145.55873398,35.88142289)(145.78873047,35.96142873)
\curveto(146.00873353,36.04142273)(146.22873331,36.12642264)(146.44873047,36.21642873)
\curveto(146.538733,36.25642251)(146.64873289,36.29642247)(146.77873047,36.33642873)
\curveto(146.89873264,36.37642239)(146.96873257,36.44142233)(146.98873047,36.53142873)
\curveto(146.99873254,36.5714222)(146.99873254,36.60142217)(146.98873047,36.62142873)
\lineto(146.92873047,36.68142873)
\curveto(146.87873266,36.73142204)(146.82373272,36.766422)(146.76373047,36.78642873)
\curveto(146.70373284,36.81642195)(146.6387329,36.84642192)(146.56873047,36.87642873)
\lineto(145.93873047,37.11642873)
\curveto(145.71873382,37.19642157)(145.50373404,37.27642149)(145.29373047,37.35642873)
\lineto(145.14373047,37.41642873)
\lineto(144.96373047,37.47642873)
\curveto(144.77373477,37.55642121)(144.58373496,37.62642114)(144.39373047,37.68642873)
\curveto(144.19373535,37.75642101)(143.99373555,37.83142094)(143.79373047,37.91142873)
\curveto(143.21373633,38.15142062)(142.62873691,38.3714204)(142.03873047,38.57142873)
\curveto(141.44873809,38.78141999)(140.86373868,39.00641976)(140.28373047,39.24642873)
\curveto(140.08373946,39.32641944)(139.87873966,39.40141937)(139.66873047,39.47142873)
\curveto(139.45874008,39.55141922)(139.25374029,39.63141914)(139.05373047,39.71142873)
\curveto(138.97374057,39.75141902)(138.87374067,39.78641898)(138.75373047,39.81642873)
\curveto(138.63374091,39.85641891)(138.54874099,39.91141886)(138.49873047,39.98142873)
\curveto(138.45874108,40.04141873)(138.42874111,40.11641865)(138.40873047,40.20642873)
\curveto(138.38874115,40.30641846)(138.37874116,40.41641835)(138.37873047,40.53642873)
\curveto(138.36874117,40.65641811)(138.36874117,40.77641799)(138.37873047,40.89642873)
\curveto(138.37874116,41.01641775)(138.37874116,41.12641764)(138.37873047,41.22642873)
\curveto(138.37874116,41.31641745)(138.37874116,41.40641736)(138.37873047,41.49642873)
\curveto(138.37874116,41.59641717)(138.39874114,41.6714171)(138.43873047,41.72142873)
\curveto(138.48874105,41.81141696)(138.57874096,41.86141691)(138.70873047,41.87142873)
\curveto(138.8387407,41.88141689)(138.97874056,41.88641688)(139.12873047,41.88642873)
\lineto(140.77873047,41.88642873)
\lineto(147.04873047,41.88642873)
\lineto(148.30873047,41.88642873)
\curveto(148.41873112,41.88641688)(148.52873101,41.88641688)(148.63873047,41.88642873)
\curveto(148.74873079,41.89641687)(148.83373071,41.87641689)(148.89373047,41.82642873)
\curveto(148.95373059,41.79641697)(148.99373055,41.75141702)(149.01373047,41.69142873)
\curveto(149.02373052,41.63141714)(149.0387305,41.56141721)(149.05873047,41.48142873)
\lineto(149.05873047,41.24142873)
\lineto(149.05873047,40.88142873)
\curveto(149.04873049,40.771418)(149.00373054,40.69141808)(148.92373047,40.64142873)
\curveto(148.89373065,40.62141815)(148.86373068,40.60641816)(148.83373047,40.59642873)
\curveto(148.79373075,40.59641817)(148.74873079,40.58641818)(148.69873047,40.56642873)
\lineto(148.53373047,40.56642873)
\curveto(148.47373107,40.55641821)(148.40373114,40.55141822)(148.32373047,40.55142873)
\curveto(148.2437313,40.56141821)(148.16873137,40.5664182)(148.09873047,40.56642873)
\lineto(147.25873047,40.56642873)
\lineto(142.83373047,40.56642873)
\curveto(142.58373696,40.5664182)(142.33373721,40.5664182)(142.08373047,40.56642873)
\curveto(141.82373772,40.5664182)(141.57373797,40.56141821)(141.33373047,40.55142873)
\curveto(141.23373831,40.55141822)(141.12373842,40.54641822)(141.00373047,40.53642873)
\curveto(140.88373866,40.52641824)(140.82373872,40.4714183)(140.82373047,40.37142873)
\lineto(140.83873047,40.37142873)
\curveto(140.85873868,40.30141847)(140.92373862,40.24141853)(141.03373047,40.19142873)
\curveto(141.1437384,40.15141862)(141.2387383,40.11641865)(141.31873047,40.08642873)
\curveto(141.48873805,40.01641875)(141.66373788,39.95141882)(141.84373047,39.89142873)
\curveto(142.01373753,39.83141894)(142.18373736,39.76141901)(142.35373047,39.68142873)
\curveto(142.40373714,39.66141911)(142.44873709,39.64641912)(142.48873047,39.63642873)
\curveto(142.52873701,39.62641914)(142.57373697,39.61141916)(142.62373047,39.59142873)
\curveto(142.80373674,39.51141926)(142.98873655,39.44141933)(143.17873047,39.38142873)
\curveto(143.35873618,39.33141944)(143.538736,39.2664195)(143.71873047,39.18642873)
\curveto(143.86873567,39.11641965)(144.02373552,39.05641971)(144.18373047,39.00642873)
\curveto(144.33373521,38.95641981)(144.48373506,38.90141987)(144.63373047,38.84142873)
\curveto(145.10373444,38.64142013)(145.57873396,38.46142031)(146.05873047,38.30142873)
\curveto(146.52873301,38.14142063)(146.99373255,37.9664208)(147.45373047,37.77642873)
\curveto(147.63373191,37.69642107)(147.81373173,37.62642114)(147.99373047,37.56642873)
\curveto(148.17373137,37.50642126)(148.35373119,37.44142133)(148.53373047,37.37142873)
\curveto(148.6437309,37.32142145)(148.74873079,37.2714215)(148.84873047,37.22142873)
\curveto(148.9387306,37.18142159)(149.00373054,37.09642167)(149.04373047,36.96642873)
\curveto(149.05373049,36.94642182)(149.05873048,36.92142185)(149.05873047,36.89142873)
\curveto(149.04873049,36.8714219)(149.04873049,36.84642192)(149.05873047,36.81642873)
\curveto(149.06873047,36.78642198)(149.07373047,36.75142202)(149.07373047,36.71142873)
\curveto(149.06373048,36.6714221)(149.05873048,36.63142214)(149.05873047,36.59142873)
\lineto(149.05873047,36.29142873)
\curveto(149.05873048,36.19142258)(149.03373051,36.11142266)(148.98373047,36.05142873)
\curveto(148.93373061,35.9714228)(148.86373068,35.91142286)(148.77373047,35.87142873)
\curveto(148.67373087,35.84142293)(148.57373097,35.80142297)(148.47373047,35.75142873)
\curveto(148.27373127,35.6714231)(148.06873147,35.59142318)(147.85873047,35.51142873)
\curveto(147.6387319,35.44142333)(147.42873211,35.3664234)(147.22873047,35.28642873)
\curveto(147.04873249,35.20642356)(146.86873267,35.13642363)(146.68873047,35.07642873)
\curveto(146.49873304,35.02642374)(146.31373323,34.96142381)(146.13373047,34.88142873)
\curveto(145.57373397,34.65142412)(145.00873453,34.43642433)(144.43873047,34.23642873)
\curveto(143.86873567,34.03642473)(143.30373624,33.82142495)(142.74373047,33.59142873)
\lineto(142.11373047,33.35142873)
\curveto(141.89373765,33.28142549)(141.68373786,33.20642556)(141.48373047,33.12642873)
\curveto(141.37373817,33.07642569)(141.26873827,33.03142574)(141.16873047,32.99142873)
\curveto(141.05873848,32.96142581)(140.96373858,32.91142586)(140.88373047,32.84142873)
\curveto(140.86373868,32.83142594)(140.85373869,32.82142595)(140.85373047,32.81142873)
\lineto(140.82373047,32.78142873)
\lineto(140.82373047,32.70642873)
\lineto(140.85373047,32.67642873)
\curveto(140.85373869,32.6664261)(140.85873868,32.65642611)(140.86873047,32.64642873)
\curveto(140.91873862,32.62642614)(140.97373857,32.61642615)(141.03373047,32.61642873)
\curveto(141.09373845,32.61642615)(141.15373839,32.60642616)(141.21373047,32.58642873)
\lineto(141.37873047,32.58642873)
\curveto(141.4387381,32.5664262)(141.50373804,32.56142621)(141.57373047,32.57142873)
\curveto(141.6437379,32.58142619)(141.71373783,32.58642618)(141.78373047,32.58642873)
\lineto(142.59373047,32.58642873)
\lineto(147.15373047,32.58642873)
\lineto(148.33873047,32.58642873)
\curveto(148.44873109,32.58642618)(148.55873098,32.58142619)(148.66873047,32.57142873)
\curveto(148.77873076,32.5714262)(148.86373068,32.54642622)(148.92373047,32.49642873)
\curveto(149.00373054,32.44642632)(149.04873049,32.35642641)(149.05873047,32.22642873)
\lineto(149.05873047,31.83642873)
\lineto(149.05873047,31.64142873)
\curveto(149.05873048,31.59142718)(149.04873049,31.54142723)(149.02873047,31.49142873)
\curveto(148.98873055,31.36142741)(148.90373064,31.28642748)(148.77373047,31.26642873)
\curveto(148.6437309,31.25642751)(148.49373105,31.25142752)(148.32373047,31.25142873)
\lineto(146.58373047,31.25142873)
\lineto(140.58373047,31.25142873)
\lineto(139.17373047,31.25142873)
\curveto(139.06374048,31.25142752)(138.94874059,31.24642752)(138.82873047,31.23642873)
\curveto(138.70874083,31.23642753)(138.61374093,31.26142751)(138.54373047,31.31142873)
\curveto(138.48374106,31.35142742)(138.43374111,31.42642734)(138.39373047,31.53642873)
\curveto(138.38374116,31.55642721)(138.38374116,31.57642719)(138.39373047,31.59642873)
\curveto(138.39374115,31.62642714)(138.38874115,31.65142712)(138.37873047,31.67142873)
}
}
{
\newrgbcolor{curcolor}{0 0 0}
\pscustom[linestyle=none,fillstyle=solid,fillcolor=curcolor]
{
\newpath
\moveto(148.50373047,50.87353811)
\curveto(148.66373088,50.90353028)(148.79873074,50.88853029)(148.90873047,50.82853811)
\curveto(149.00873053,50.76853041)(149.08373046,50.68853049)(149.13373047,50.58853811)
\curveto(149.15373039,50.53853064)(149.16373038,50.4835307)(149.16373047,50.42353811)
\curveto(149.16373038,50.37353081)(149.17373037,50.31853086)(149.19373047,50.25853811)
\curveto(149.2437303,50.03853114)(149.22873031,49.81853136)(149.14873047,49.59853811)
\curveto(149.07873046,49.38853179)(148.98873055,49.24353194)(148.87873047,49.16353811)
\curveto(148.80873073,49.11353207)(148.72873081,49.06853211)(148.63873047,49.02853811)
\curveto(148.538731,48.98853219)(148.45873108,48.93853224)(148.39873047,48.87853811)
\curveto(148.37873116,48.85853232)(148.35873118,48.83353235)(148.33873047,48.80353811)
\curveto(148.31873122,48.7835324)(148.31373123,48.75353243)(148.32373047,48.71353811)
\curveto(148.35373119,48.60353258)(148.40873113,48.49853268)(148.48873047,48.39853811)
\curveto(148.56873097,48.30853287)(148.6387309,48.21853296)(148.69873047,48.12853811)
\curveto(148.77873076,47.99853318)(148.85373069,47.85853332)(148.92373047,47.70853811)
\curveto(148.98373056,47.55853362)(149.0387305,47.39853378)(149.08873047,47.22853811)
\curveto(149.11873042,47.12853405)(149.1387304,47.01853416)(149.14873047,46.89853811)
\curveto(149.15873038,46.78853439)(149.17373037,46.6785345)(149.19373047,46.56853811)
\curveto(149.20373034,46.51853466)(149.20873033,46.47353471)(149.20873047,46.43353811)
\lineto(149.20873047,46.32853811)
\curveto(149.22873031,46.21853496)(149.22873031,46.11353507)(149.20873047,46.01353811)
\lineto(149.20873047,45.87853811)
\curveto(149.19873034,45.82853535)(149.19373035,45.7785354)(149.19373047,45.72853811)
\curveto(149.19373035,45.6785355)(149.18373036,45.63353555)(149.16373047,45.59353811)
\curveto(149.15373039,45.55353563)(149.14873039,45.51853566)(149.14873047,45.48853811)
\curveto(149.15873038,45.46853571)(149.15873038,45.44353574)(149.14873047,45.41353811)
\lineto(149.08873047,45.17353811)
\curveto(149.07873046,45.09353609)(149.05873048,45.01853616)(149.02873047,44.94853811)
\curveto(148.89873064,44.64853653)(148.75373079,44.40353678)(148.59373047,44.21353811)
\curveto(148.42373112,44.03353715)(148.18873135,43.8835373)(147.88873047,43.76353811)
\curveto(147.66873187,43.67353751)(147.40373214,43.62853755)(147.09373047,43.62853811)
\lineto(146.77873047,43.62853811)
\curveto(146.72873281,43.63853754)(146.67873286,43.64353754)(146.62873047,43.64353811)
\lineto(146.44873047,43.67353811)
\lineto(146.11873047,43.79353811)
\curveto(146.00873353,43.83353735)(145.90873363,43.8835373)(145.81873047,43.94353811)
\curveto(145.52873401,44.12353706)(145.31373423,44.36853681)(145.17373047,44.67853811)
\curveto(145.03373451,44.98853619)(144.90873463,45.32853585)(144.79873047,45.69853811)
\curveto(144.75873478,45.83853534)(144.72873481,45.9835352)(144.70873047,46.13353811)
\curveto(144.68873485,46.2835349)(144.66373488,46.43353475)(144.63373047,46.58353811)
\curveto(144.61373493,46.65353453)(144.60373494,46.71853446)(144.60373047,46.77853811)
\curveto(144.60373494,46.84853433)(144.59373495,46.92353426)(144.57373047,47.00353811)
\curveto(144.55373499,47.07353411)(144.543735,47.14353404)(144.54373047,47.21353811)
\curveto(144.53373501,47.2835339)(144.51873502,47.35853382)(144.49873047,47.43853811)
\curveto(144.4387351,47.68853349)(144.38873515,47.92353326)(144.34873047,48.14353811)
\curveto(144.29873524,48.36353282)(144.18373536,48.53853264)(144.00373047,48.66853811)
\curveto(143.92373562,48.72853245)(143.82373572,48.7785324)(143.70373047,48.81853811)
\curveto(143.57373597,48.85853232)(143.43373611,48.85853232)(143.28373047,48.81853811)
\curveto(143.0437365,48.75853242)(142.85373669,48.66853251)(142.71373047,48.54853811)
\curveto(142.57373697,48.43853274)(142.46373708,48.2785329)(142.38373047,48.06853811)
\curveto(142.33373721,47.94853323)(142.29873724,47.80353338)(142.27873047,47.63353811)
\curveto(142.25873728,47.47353371)(142.24873729,47.30353388)(142.24873047,47.12353811)
\curveto(142.24873729,46.94353424)(142.25873728,46.76853441)(142.27873047,46.59853811)
\curveto(142.29873724,46.42853475)(142.32873721,46.2835349)(142.36873047,46.16353811)
\curveto(142.42873711,45.99353519)(142.51373703,45.82853535)(142.62373047,45.66853811)
\curveto(142.68373686,45.58853559)(142.76373678,45.51353567)(142.86373047,45.44353811)
\curveto(142.95373659,45.3835358)(143.05373649,45.32853585)(143.16373047,45.27853811)
\curveto(143.2437363,45.24853593)(143.32873621,45.21853596)(143.41873047,45.18853811)
\curveto(143.50873603,45.16853601)(143.57873596,45.12353606)(143.62873047,45.05353811)
\curveto(143.65873588,45.01353617)(143.68373586,44.94353624)(143.70373047,44.84353811)
\curveto(143.71373583,44.75353643)(143.71873582,44.65853652)(143.71873047,44.55853811)
\curveto(143.71873582,44.45853672)(143.71373583,44.35853682)(143.70373047,44.25853811)
\curveto(143.68373586,44.16853701)(143.65873588,44.10353708)(143.62873047,44.06353811)
\curveto(143.59873594,44.02353716)(143.54873599,43.99353719)(143.47873047,43.97353811)
\curveto(143.40873613,43.95353723)(143.33373621,43.95353723)(143.25373047,43.97353811)
\curveto(143.12373642,44.00353718)(143.00373654,44.03353715)(142.89373047,44.06353811)
\curveto(142.77373677,44.10353708)(142.65873688,44.14853703)(142.54873047,44.19853811)
\curveto(142.19873734,44.38853679)(141.92873761,44.62853655)(141.73873047,44.91853811)
\curveto(141.538738,45.20853597)(141.37873816,45.56853561)(141.25873047,45.99853811)
\curveto(141.2387383,46.09853508)(141.22373832,46.19853498)(141.21373047,46.29853811)
\curveto(141.20373834,46.40853477)(141.18873835,46.51853466)(141.16873047,46.62853811)
\curveto(141.15873838,46.66853451)(141.15873838,46.73353445)(141.16873047,46.82353811)
\curveto(141.16873837,46.91353427)(141.15873838,46.96853421)(141.13873047,46.98853811)
\curveto(141.12873841,47.68853349)(141.20873833,48.29853288)(141.37873047,48.81853811)
\curveto(141.54873799,49.33853184)(141.87373767,49.70353148)(142.35373047,49.91353811)
\curveto(142.55373699,50.00353118)(142.78873675,50.05353113)(143.05873047,50.06353811)
\curveto(143.31873622,50.0835311)(143.59373595,50.09353109)(143.88373047,50.09353811)
\lineto(147.19873047,50.09353811)
\curveto(147.3387322,50.09353109)(147.47373207,50.09853108)(147.60373047,50.10853811)
\curveto(147.73373181,50.11853106)(147.8387317,50.14853103)(147.91873047,50.19853811)
\curveto(147.98873155,50.24853093)(148.0387315,50.31353087)(148.06873047,50.39353811)
\curveto(148.10873143,50.4835307)(148.1387314,50.56853061)(148.15873047,50.64853811)
\curveto(148.16873137,50.72853045)(148.21373133,50.78853039)(148.29373047,50.82853811)
\curveto(148.32373122,50.84853033)(148.35373119,50.85853032)(148.38373047,50.85853811)
\curveto(148.41373113,50.85853032)(148.45373109,50.86353032)(148.50373047,50.87353811)
\moveto(146.83873047,48.72853811)
\curveto(146.69873284,48.78853239)(146.538733,48.81853236)(146.35873047,48.81853811)
\curveto(146.16873337,48.82853235)(145.97373357,48.83353235)(145.77373047,48.83353811)
\curveto(145.66373388,48.83353235)(145.56373398,48.82853235)(145.47373047,48.81853811)
\curveto(145.38373416,48.80853237)(145.31373423,48.76853241)(145.26373047,48.69853811)
\curveto(145.2437343,48.66853251)(145.23373431,48.59853258)(145.23373047,48.48853811)
\curveto(145.25373429,48.46853271)(145.26373428,48.43353275)(145.26373047,48.38353811)
\curveto(145.26373428,48.33353285)(145.27373427,48.28853289)(145.29373047,48.24853811)
\curveto(145.31373423,48.16853301)(145.33373421,48.0785331)(145.35373047,47.97853811)
\lineto(145.41373047,47.67853811)
\curveto(145.41373413,47.64853353)(145.41873412,47.61353357)(145.42873047,47.57353811)
\lineto(145.42873047,47.46853811)
\curveto(145.46873407,47.31853386)(145.49373405,47.15353403)(145.50373047,46.97353811)
\curveto(145.50373404,46.80353438)(145.52373402,46.64353454)(145.56373047,46.49353811)
\curveto(145.58373396,46.41353477)(145.60373394,46.33853484)(145.62373047,46.26853811)
\curveto(145.63373391,46.20853497)(145.64873389,46.13853504)(145.66873047,46.05853811)
\curveto(145.71873382,45.89853528)(145.78373376,45.74853543)(145.86373047,45.60853811)
\curveto(145.93373361,45.46853571)(146.02373352,45.34853583)(146.13373047,45.24853811)
\curveto(146.2437333,45.14853603)(146.37873316,45.07353611)(146.53873047,45.02353811)
\curveto(146.68873285,44.97353621)(146.87373267,44.95353623)(147.09373047,44.96353811)
\curveto(147.19373235,44.96353622)(147.28873225,44.9785362)(147.37873047,45.00853811)
\curveto(147.45873208,45.04853613)(147.53373201,45.09353609)(147.60373047,45.14353811)
\curveto(147.71373183,45.22353596)(147.80873173,45.32853585)(147.88873047,45.45853811)
\curveto(147.95873158,45.58853559)(148.01873152,45.72853545)(148.06873047,45.87853811)
\curveto(148.07873146,45.92853525)(148.08373146,45.9785352)(148.08373047,46.02853811)
\curveto(148.08373146,46.0785351)(148.08873145,46.12853505)(148.09873047,46.17853811)
\curveto(148.11873142,46.24853493)(148.13373141,46.33353485)(148.14373047,46.43353811)
\curveto(148.1437314,46.54353464)(148.13373141,46.63353455)(148.11373047,46.70353811)
\curveto(148.09373145,46.76353442)(148.08873145,46.82353436)(148.09873047,46.88353811)
\curveto(148.09873144,46.94353424)(148.08873145,47.00353418)(148.06873047,47.06353811)
\curveto(148.04873149,47.14353404)(148.03373151,47.21853396)(148.02373047,47.28853811)
\curveto(148.01373153,47.36853381)(147.99373155,47.44353374)(147.96373047,47.51353811)
\curveto(147.8437317,47.80353338)(147.69873184,48.04853313)(147.52873047,48.24853811)
\curveto(147.35873218,48.45853272)(147.12873241,48.61853256)(146.83873047,48.72853811)
}
}
{
\newrgbcolor{curcolor}{0 0 0}
\pscustom[linestyle=none,fillstyle=solid,fillcolor=curcolor]
{
\newpath
\moveto(141.15373047,55.69017873)
\curveto(141.15373839,55.92017394)(141.21373833,56.05017381)(141.33373047,56.08017873)
\curveto(141.4437381,56.11017375)(141.60873793,56.12517374)(141.82873047,56.12517873)
\lineto(142.11373047,56.12517873)
\curveto(142.20373734,56.12517374)(142.27873726,56.10017376)(142.33873047,56.05017873)
\curveto(142.41873712,55.99017387)(142.46373708,55.90517396)(142.47373047,55.79517873)
\curveto(142.47373707,55.68517418)(142.48873705,55.57517429)(142.51873047,55.46517873)
\curveto(142.54873699,55.32517454)(142.57873696,55.19017467)(142.60873047,55.06017873)
\curveto(142.6387369,54.94017492)(142.67873686,54.82517504)(142.72873047,54.71517873)
\curveto(142.85873668,54.42517544)(143.0387365,54.19017567)(143.26873047,54.01017873)
\curveto(143.48873605,53.83017603)(143.7437358,53.67517619)(144.03373047,53.54517873)
\curveto(144.1437354,53.50517636)(144.25873528,53.47517639)(144.37873047,53.45517873)
\curveto(144.48873505,53.43517643)(144.60373494,53.41017645)(144.72373047,53.38017873)
\curveto(144.77373477,53.37017649)(144.82373472,53.3651765)(144.87373047,53.36517873)
\curveto(144.92373462,53.37517649)(144.97373457,53.37517649)(145.02373047,53.36517873)
\curveto(145.1437344,53.33517653)(145.28373426,53.32017654)(145.44373047,53.32017873)
\curveto(145.59373395,53.33017653)(145.7387338,53.33517653)(145.87873047,53.33517873)
\lineto(147.72373047,53.33517873)
\lineto(148.06873047,53.33517873)
\curveto(148.18873135,53.33517653)(148.30373124,53.33017653)(148.41373047,53.32017873)
\curveto(148.52373102,53.31017655)(148.61873092,53.30517656)(148.69873047,53.30517873)
\curveto(148.77873076,53.31517655)(148.84873069,53.29517657)(148.90873047,53.24517873)
\curveto(148.97873056,53.19517667)(149.01873052,53.11517675)(149.02873047,53.00517873)
\curveto(149.0387305,52.90517696)(149.0437305,52.79517707)(149.04373047,52.67517873)
\lineto(149.04373047,52.40517873)
\curveto(149.02373052,52.35517751)(149.00873053,52.30517756)(148.99873047,52.25517873)
\curveto(148.97873056,52.21517765)(148.95373059,52.18517768)(148.92373047,52.16517873)
\curveto(148.85373069,52.11517775)(148.76873077,52.08517778)(148.66873047,52.07517873)
\lineto(148.33873047,52.07517873)
\lineto(147.18373047,52.07517873)
\lineto(143.02873047,52.07517873)
\lineto(141.99373047,52.07517873)
\lineto(141.69373047,52.07517873)
\curveto(141.59373795,52.08517778)(141.50873803,52.11517775)(141.43873047,52.16517873)
\curveto(141.39873814,52.19517767)(141.36873817,52.24517762)(141.34873047,52.31517873)
\curveto(141.32873821,52.39517747)(141.31873822,52.48017738)(141.31873047,52.57017873)
\curveto(141.30873823,52.6601772)(141.30873823,52.75017711)(141.31873047,52.84017873)
\curveto(141.32873821,52.93017693)(141.3437382,53.00017686)(141.36373047,53.05017873)
\curveto(141.39373815,53.13017673)(141.45373809,53.18017668)(141.54373047,53.20017873)
\curveto(141.62373792,53.23017663)(141.71373783,53.24517662)(141.81373047,53.24517873)
\lineto(142.11373047,53.24517873)
\curveto(142.21373733,53.24517662)(142.30373724,53.2651766)(142.38373047,53.30517873)
\curveto(142.40373714,53.31517655)(142.41873712,53.32517654)(142.42873047,53.33517873)
\lineto(142.47373047,53.38017873)
\curveto(142.47373707,53.49017637)(142.42873711,53.58017628)(142.33873047,53.65017873)
\curveto(142.2387373,53.72017614)(142.15873738,53.78017608)(142.09873047,53.83017873)
\lineto(142.00873047,53.92017873)
\curveto(141.89873764,54.01017585)(141.78373776,54.13517573)(141.66373047,54.29517873)
\curveto(141.543738,54.45517541)(141.45373809,54.60517526)(141.39373047,54.74517873)
\curveto(141.3437382,54.83517503)(141.30873823,54.93017493)(141.28873047,55.03017873)
\curveto(141.25873828,55.13017473)(141.22873831,55.23517463)(141.19873047,55.34517873)
\curveto(141.18873835,55.40517446)(141.18373836,55.4651744)(141.18373047,55.52517873)
\curveto(141.17373837,55.58517428)(141.16373838,55.64017422)(141.15373047,55.69017873)
}
}
{
\newrgbcolor{curcolor}{0 0 0}
\pscustom[linestyle=none,fillstyle=solid,fillcolor=curcolor]
{
}
}
{
\newrgbcolor{curcolor}{0 0 0}
\pscustom[linestyle=none,fillstyle=solid,fillcolor=curcolor]
{
\newpath
\moveto(143.97373047,67.99510061)
\lineto(144.22873047,67.99510061)
\curveto(144.30873523,68.0050929)(144.38373516,68.00009291)(144.45373047,67.98010061)
\lineto(144.69373047,67.98010061)
\lineto(144.85873047,67.98010061)
\curveto(144.95873458,67.96009295)(145.06373448,67.95009296)(145.17373047,67.95010061)
\curveto(145.27373427,67.95009296)(145.37373417,67.94009297)(145.47373047,67.92010061)
\lineto(145.62373047,67.92010061)
\curveto(145.76373378,67.89009302)(145.90373364,67.87009304)(146.04373047,67.86010061)
\curveto(146.17373337,67.85009306)(146.30373324,67.82509308)(146.43373047,67.78510061)
\curveto(146.51373303,67.76509314)(146.59873294,67.74509316)(146.68873047,67.72510061)
\lineto(146.92873047,67.66510061)
\lineto(147.22873047,67.54510061)
\curveto(147.31873222,67.51509339)(147.40873213,67.48009343)(147.49873047,67.44010061)
\curveto(147.71873182,67.34009357)(147.93373161,67.2050937)(148.14373047,67.03510061)
\curveto(148.35373119,66.87509403)(148.52373102,66.70009421)(148.65373047,66.51010061)
\curveto(148.69373085,66.46009445)(148.73373081,66.40009451)(148.77373047,66.33010061)
\curveto(148.80373074,66.27009464)(148.8387307,66.2100947)(148.87873047,66.15010061)
\curveto(148.92873061,66.07009484)(148.96873057,65.97509493)(148.99873047,65.86510061)
\curveto(149.02873051,65.75509515)(149.05873048,65.65009526)(149.08873047,65.55010061)
\curveto(149.12873041,65.44009547)(149.15373039,65.33009558)(149.16373047,65.22010061)
\curveto(149.17373037,65.1100958)(149.18873035,64.99509591)(149.20873047,64.87510061)
\curveto(149.21873032,64.83509607)(149.21873032,64.79009612)(149.20873047,64.74010061)
\curveto(149.20873033,64.70009621)(149.21373033,64.66009625)(149.22373047,64.62010061)
\curveto(149.23373031,64.58009633)(149.2387303,64.52509638)(149.23873047,64.45510061)
\curveto(149.2387303,64.38509652)(149.23373031,64.33509657)(149.22373047,64.30510061)
\curveto(149.20373034,64.25509665)(149.19873034,64.2100967)(149.20873047,64.17010061)
\curveto(149.21873032,64.13009678)(149.21873032,64.09509681)(149.20873047,64.06510061)
\lineto(149.20873047,63.97510061)
\curveto(149.18873035,63.91509699)(149.17373037,63.85009706)(149.16373047,63.78010061)
\curveto(149.16373038,63.72009719)(149.15873038,63.65509725)(149.14873047,63.58510061)
\curveto(149.09873044,63.41509749)(149.04873049,63.25509765)(148.99873047,63.10510061)
\curveto(148.94873059,62.95509795)(148.88373066,62.8100981)(148.80373047,62.67010061)
\curveto(148.76373078,62.62009829)(148.73373081,62.56509834)(148.71373047,62.50510061)
\curveto(148.68373086,62.45509845)(148.64873089,62.4050985)(148.60873047,62.35510061)
\curveto(148.42873111,62.11509879)(148.20873133,61.91509899)(147.94873047,61.75510061)
\curveto(147.68873185,61.59509931)(147.40373214,61.45509945)(147.09373047,61.33510061)
\curveto(146.95373259,61.27509963)(146.81373273,61.23009968)(146.67373047,61.20010061)
\curveto(146.52373302,61.17009974)(146.36873317,61.13509977)(146.20873047,61.09510061)
\curveto(146.09873344,61.07509983)(145.98873355,61.06009985)(145.87873047,61.05010061)
\curveto(145.76873377,61.04009987)(145.65873388,61.02509988)(145.54873047,61.00510061)
\curveto(145.50873403,60.99509991)(145.46873407,60.99009992)(145.42873047,60.99010061)
\curveto(145.38873415,61.00009991)(145.34873419,61.00009991)(145.30873047,60.99010061)
\curveto(145.25873428,60.98009993)(145.20873433,60.97509993)(145.15873047,60.97510061)
\lineto(144.99373047,60.97510061)
\curveto(144.9437346,60.95509995)(144.89373465,60.95009996)(144.84373047,60.96010061)
\curveto(144.78373476,60.97009994)(144.72873481,60.97009994)(144.67873047,60.96010061)
\curveto(144.6387349,60.95009996)(144.59373495,60.95009996)(144.54373047,60.96010061)
\curveto(144.49373505,60.97009994)(144.4437351,60.96509994)(144.39373047,60.94510061)
\curveto(144.32373522,60.92509998)(144.24873529,60.92009999)(144.16873047,60.93010061)
\curveto(144.07873546,60.94009997)(143.99373555,60.94509996)(143.91373047,60.94510061)
\curveto(143.82373572,60.94509996)(143.72373582,60.94009997)(143.61373047,60.93010061)
\curveto(143.49373605,60.92009999)(143.39373615,60.92509998)(143.31373047,60.94510061)
\lineto(143.02873047,60.94510061)
\lineto(142.39873047,60.99010061)
\curveto(142.29873724,61.00009991)(142.20373734,61.0100999)(142.11373047,61.02010061)
\lineto(141.81373047,61.05010061)
\curveto(141.76373778,61.07009984)(141.71373783,61.07509983)(141.66373047,61.06510061)
\curveto(141.60373794,61.06509984)(141.54873799,61.07509983)(141.49873047,61.09510061)
\curveto(141.32873821,61.14509976)(141.16373838,61.18509972)(141.00373047,61.21510061)
\curveto(140.83373871,61.24509966)(140.67373887,61.29509961)(140.52373047,61.36510061)
\curveto(140.06373948,61.55509935)(139.68873985,61.77509913)(139.39873047,62.02510061)
\curveto(139.10874043,62.28509862)(138.86374068,62.64509826)(138.66373047,63.10510061)
\curveto(138.61374093,63.23509767)(138.57874096,63.36509754)(138.55873047,63.49510061)
\curveto(138.538741,63.63509727)(138.51374103,63.77509713)(138.48373047,63.91510061)
\curveto(138.47374107,63.98509692)(138.46874107,64.05009686)(138.46873047,64.11010061)
\curveto(138.46874107,64.17009674)(138.46374108,64.23509667)(138.45373047,64.30510061)
\curveto(138.43374111,65.13509577)(138.58374096,65.8050951)(138.90373047,66.31510061)
\curveto(139.21374033,66.82509408)(139.65373989,67.2050937)(140.22373047,67.45510061)
\curveto(140.3437392,67.5050934)(140.46873907,67.55009336)(140.59873047,67.59010061)
\curveto(140.72873881,67.63009328)(140.86373868,67.67509323)(141.00373047,67.72510061)
\curveto(141.08373846,67.74509316)(141.16873837,67.76009315)(141.25873047,67.77010061)
\lineto(141.49873047,67.83010061)
\curveto(141.60873793,67.86009305)(141.71873782,67.87509303)(141.82873047,67.87510061)
\curveto(141.9387376,67.88509302)(142.04873749,67.90009301)(142.15873047,67.92010061)
\curveto(142.20873733,67.94009297)(142.25373729,67.94509296)(142.29373047,67.93510061)
\curveto(142.33373721,67.93509297)(142.37373717,67.94009297)(142.41373047,67.95010061)
\curveto(142.46373708,67.96009295)(142.51873702,67.96009295)(142.57873047,67.95010061)
\curveto(142.62873691,67.95009296)(142.67873686,67.95509295)(142.72873047,67.96510061)
\lineto(142.86373047,67.96510061)
\curveto(142.92373662,67.98509292)(142.99373655,67.98509292)(143.07373047,67.96510061)
\curveto(143.1437364,67.95509295)(143.20873633,67.96009295)(143.26873047,67.98010061)
\curveto(143.29873624,67.99009292)(143.3387362,67.99509291)(143.38873047,67.99510061)
\lineto(143.50873047,67.99510061)
\lineto(143.97373047,67.99510061)
\moveto(146.29873047,66.45010061)
\curveto(145.97873356,66.55009436)(145.61373393,66.6100943)(145.20373047,66.63010061)
\curveto(144.79373475,66.65009426)(144.38373516,66.66009425)(143.97373047,66.66010061)
\curveto(143.543736,66.66009425)(143.12373642,66.65009426)(142.71373047,66.63010061)
\curveto(142.30373724,66.6100943)(141.91873762,66.56509434)(141.55873047,66.49510061)
\curveto(141.19873834,66.42509448)(140.87873866,66.31509459)(140.59873047,66.16510061)
\curveto(140.30873923,66.02509488)(140.07373947,65.83009508)(139.89373047,65.58010061)
\curveto(139.78373976,65.42009549)(139.70373984,65.24009567)(139.65373047,65.04010061)
\curveto(139.59373995,64.84009607)(139.56373998,64.59509631)(139.56373047,64.30510061)
\curveto(139.58373996,64.28509662)(139.59373995,64.25009666)(139.59373047,64.20010061)
\curveto(139.58373996,64.15009676)(139.58373996,64.1100968)(139.59373047,64.08010061)
\curveto(139.61373993,64.00009691)(139.63373991,63.92509698)(139.65373047,63.85510061)
\curveto(139.66373988,63.79509711)(139.68373986,63.73009718)(139.71373047,63.66010061)
\curveto(139.83373971,63.39009752)(140.00373954,63.17009774)(140.22373047,63.00010061)
\curveto(140.43373911,62.84009807)(140.67873886,62.7050982)(140.95873047,62.59510061)
\curveto(141.06873847,62.54509836)(141.18873835,62.5050984)(141.31873047,62.47510061)
\curveto(141.4387381,62.45509845)(141.56373798,62.43009848)(141.69373047,62.40010061)
\curveto(141.7437378,62.38009853)(141.79873774,62.37009854)(141.85873047,62.37010061)
\curveto(141.90873763,62.37009854)(141.95873758,62.36509854)(142.00873047,62.35510061)
\curveto(142.09873744,62.34509856)(142.19373735,62.33509857)(142.29373047,62.32510061)
\curveto(142.38373716,62.31509859)(142.47873706,62.3050986)(142.57873047,62.29510061)
\curveto(142.65873688,62.29509861)(142.7437368,62.29009862)(142.83373047,62.28010061)
\lineto(143.07373047,62.28010061)
\lineto(143.25373047,62.28010061)
\curveto(143.28373626,62.27009864)(143.31873622,62.26509864)(143.35873047,62.26510061)
\lineto(143.49373047,62.26510061)
\lineto(143.94373047,62.26510061)
\curveto(144.02373552,62.26509864)(144.10873543,62.26009865)(144.19873047,62.25010061)
\curveto(144.27873526,62.25009866)(144.35373519,62.26009865)(144.42373047,62.28010061)
\lineto(144.69373047,62.28010061)
\curveto(144.71373483,62.28009863)(144.7437348,62.27509863)(144.78373047,62.26510061)
\curveto(144.81373473,62.26509864)(144.8387347,62.27009864)(144.85873047,62.28010061)
\curveto(144.95873458,62.29009862)(145.05873448,62.29509861)(145.15873047,62.29510061)
\curveto(145.24873429,62.3050986)(145.34873419,62.31509859)(145.45873047,62.32510061)
\curveto(145.57873396,62.35509855)(145.70373384,62.37009854)(145.83373047,62.37010061)
\curveto(145.95373359,62.38009853)(146.06873347,62.4050985)(146.17873047,62.44510061)
\curveto(146.47873306,62.52509838)(146.7437328,62.6100983)(146.97373047,62.70010061)
\curveto(147.20373234,62.80009811)(147.41873212,62.94509796)(147.61873047,63.13510061)
\curveto(147.81873172,63.34509756)(147.96873157,63.6100973)(148.06873047,63.93010061)
\curveto(148.08873145,63.97009694)(148.09873144,64.0050969)(148.09873047,64.03510061)
\curveto(148.08873145,64.07509683)(148.09373145,64.12009679)(148.11373047,64.17010061)
\curveto(148.12373142,64.2100967)(148.13373141,64.28009663)(148.14373047,64.38010061)
\curveto(148.15373139,64.49009642)(148.14873139,64.57509633)(148.12873047,64.63510061)
\curveto(148.10873143,64.7050962)(148.09873144,64.77509613)(148.09873047,64.84510061)
\curveto(148.08873145,64.91509599)(148.07373147,64.98009593)(148.05373047,65.04010061)
\curveto(147.99373155,65.24009567)(147.90873163,65.42009549)(147.79873047,65.58010061)
\curveto(147.77873176,65.6100953)(147.75873178,65.63509527)(147.73873047,65.65510061)
\lineto(147.67873047,65.71510061)
\curveto(147.65873188,65.75509515)(147.61873192,65.8050951)(147.55873047,65.86510061)
\curveto(147.41873212,65.96509494)(147.28873225,66.05009486)(147.16873047,66.12010061)
\curveto(147.04873249,66.19009472)(146.90373264,66.26009465)(146.73373047,66.33010061)
\curveto(146.66373288,66.36009455)(146.59373295,66.38009453)(146.52373047,66.39010061)
\curveto(146.45373309,66.4100945)(146.37873316,66.43009448)(146.29873047,66.45010061)
}
}
{
\newrgbcolor{curcolor}{0 0 0}
\pscustom[linestyle=none,fillstyle=solid,fillcolor=curcolor]
{
\newpath
\moveto(144.73873047,76.34470998)
\curveto(144.85873468,76.37470226)(144.99873454,76.39970223)(145.15873047,76.41970998)
\curveto(145.31873422,76.43970219)(145.48373406,76.44970218)(145.65373047,76.44970998)
\curveto(145.82373372,76.44970218)(145.98873355,76.43970219)(146.14873047,76.41970998)
\curveto(146.30873323,76.39970223)(146.44873309,76.37470226)(146.56873047,76.34470998)
\curveto(146.70873283,76.30470233)(146.83373271,76.26970236)(146.94373047,76.23970998)
\curveto(147.05373249,76.20970242)(147.16373238,76.16970246)(147.27373047,76.11970998)
\curveto(147.91373163,75.84970278)(148.39873114,75.4347032)(148.72873047,74.87470998)
\curveto(148.78873075,74.79470384)(148.8387307,74.70970392)(148.87873047,74.61970998)
\curveto(148.90873063,74.5297041)(148.9437306,74.4297042)(148.98373047,74.31970998)
\curveto(149.03373051,74.20970442)(149.06873047,74.08970454)(149.08873047,73.95970998)
\curveto(149.11873042,73.83970479)(149.14873039,73.70970492)(149.17873047,73.56970998)
\curveto(149.19873034,73.50970512)(149.20373034,73.44970518)(149.19373047,73.38970998)
\curveto(149.18373036,73.33970529)(149.18873035,73.27970535)(149.20873047,73.20970998)
\curveto(149.21873032,73.18970544)(149.21873032,73.16470547)(149.20873047,73.13470998)
\curveto(149.20873033,73.10470553)(149.21373033,73.07970555)(149.22373047,73.05970998)
\lineto(149.22373047,72.90970998)
\curveto(149.23373031,72.83970579)(149.23373031,72.78970584)(149.22373047,72.75970998)
\curveto(149.21373033,72.71970591)(149.20873033,72.67470596)(149.20873047,72.62470998)
\curveto(149.21873032,72.58470605)(149.21873032,72.54470609)(149.20873047,72.50470998)
\curveto(149.18873035,72.41470622)(149.17373037,72.32470631)(149.16373047,72.23470998)
\curveto(149.16373038,72.14470649)(149.15373039,72.05470658)(149.13373047,71.96470998)
\curveto(149.10373044,71.87470676)(149.07873046,71.78470685)(149.05873047,71.69470998)
\curveto(149.0387305,71.60470703)(149.00873053,71.51970711)(148.96873047,71.43970998)
\curveto(148.85873068,71.19970743)(148.72873081,70.97470766)(148.57873047,70.76470998)
\curveto(148.41873112,70.55470808)(148.2387313,70.37470826)(148.03873047,70.22470998)
\curveto(147.86873167,70.10470853)(147.69373185,69.99970863)(147.51373047,69.90970998)
\curveto(147.33373221,69.81970881)(147.1437324,69.7297089)(146.94373047,69.63970998)
\curveto(146.8437327,69.59970903)(146.7437328,69.56470907)(146.64373047,69.53470998)
\curveto(146.53373301,69.51470912)(146.42373312,69.48970914)(146.31373047,69.45970998)
\curveto(146.17373337,69.41970921)(146.03373351,69.39470924)(145.89373047,69.38470998)
\curveto(145.75373379,69.37470926)(145.61373393,69.35470928)(145.47373047,69.32470998)
\curveto(145.36373418,69.31470932)(145.26373428,69.30470933)(145.17373047,69.29470998)
\curveto(145.07373447,69.29470934)(144.97373457,69.28470935)(144.87373047,69.26470998)
\lineto(144.78373047,69.26470998)
\curveto(144.75373479,69.27470936)(144.72873481,69.27470936)(144.70873047,69.26470998)
\lineto(144.49873047,69.26470998)
\curveto(144.4387351,69.24470939)(144.37373517,69.2347094)(144.30373047,69.23470998)
\curveto(144.22373532,69.24470939)(144.14873539,69.24970938)(144.07873047,69.24970998)
\lineto(143.92873047,69.24970998)
\curveto(143.87873566,69.24970938)(143.82873571,69.25470938)(143.77873047,69.26470998)
\lineto(143.40373047,69.26470998)
\curveto(143.37373617,69.27470936)(143.3387362,69.27470936)(143.29873047,69.26470998)
\curveto(143.25873628,69.26470937)(143.21873632,69.26970936)(143.17873047,69.27970998)
\curveto(143.06873647,69.29970933)(142.95873658,69.31470932)(142.84873047,69.32470998)
\curveto(142.72873681,69.3347093)(142.61373693,69.34470929)(142.50373047,69.35470998)
\curveto(142.35373719,69.39470924)(142.20873733,69.41970921)(142.06873047,69.42970998)
\curveto(141.91873762,69.44970918)(141.77373777,69.47970915)(141.63373047,69.51970998)
\curveto(141.33373821,69.60970902)(141.04873849,69.70470893)(140.77873047,69.80470998)
\curveto(140.50873903,69.90470873)(140.25873928,70.0297086)(140.02873047,70.17970998)
\curveto(139.70873983,70.37970825)(139.42874011,70.62470801)(139.18873047,70.91470998)
\curveto(138.94874059,71.20470743)(138.76374078,71.54470709)(138.63373047,71.93470998)
\curveto(138.59374095,72.04470659)(138.56874097,72.15470648)(138.55873047,72.26470998)
\curveto(138.538741,72.38470625)(138.51374103,72.50470613)(138.48373047,72.62470998)
\curveto(138.47374107,72.69470594)(138.46874107,72.75970587)(138.46873047,72.81970998)
\curveto(138.46874107,72.87970575)(138.46374108,72.94470569)(138.45373047,73.01470998)
\curveto(138.43374111,73.71470492)(138.54874099,74.28970434)(138.79873047,74.73970998)
\curveto(139.04874049,75.18970344)(139.39874014,75.5347031)(139.84873047,75.77470998)
\curveto(140.07873946,75.88470275)(140.35373919,75.98470265)(140.67373047,76.07470998)
\curveto(140.7437388,76.09470254)(140.81873872,76.09470254)(140.89873047,76.07470998)
\curveto(140.96873857,76.06470257)(141.01873852,76.03970259)(141.04873047,75.99970998)
\curveto(141.07873846,75.96970266)(141.10373844,75.90970272)(141.12373047,75.81970998)
\curveto(141.13373841,75.7297029)(141.1437384,75.629703)(141.15373047,75.51970998)
\curveto(141.15373839,75.41970321)(141.14873839,75.31970331)(141.13873047,75.21970998)
\curveto(141.12873841,75.1297035)(141.10873843,75.06470357)(141.07873047,75.02470998)
\curveto(141.00873853,74.91470372)(140.89873864,74.8347038)(140.74873047,74.78470998)
\curveto(140.59873894,74.74470389)(140.46873907,74.68970394)(140.35873047,74.61970998)
\curveto(140.04873949,74.4297042)(139.81873972,74.14970448)(139.66873047,73.77970998)
\curveto(139.6387399,73.70970492)(139.61873992,73.634705)(139.60873047,73.55470998)
\curveto(139.59873994,73.48470515)(139.58373996,73.40970522)(139.56373047,73.32970998)
\curveto(139.55373999,73.27970535)(139.54873999,73.20970542)(139.54873047,73.11970998)
\curveto(139.54873999,73.03970559)(139.55373999,72.97470566)(139.56373047,72.92470998)
\curveto(139.58373996,72.88470575)(139.58873995,72.84970578)(139.57873047,72.81970998)
\curveto(139.56873997,72.78970584)(139.56873997,72.75470588)(139.57873047,72.71470998)
\lineto(139.63873047,72.47470998)
\curveto(139.65873988,72.40470623)(139.68373986,72.3347063)(139.71373047,72.26470998)
\curveto(139.87373967,71.88470675)(140.08373946,71.59470704)(140.34373047,71.39470998)
\curveto(140.60373894,71.20470743)(140.91873862,71.0297076)(141.28873047,70.86970998)
\curveto(141.36873817,70.83970779)(141.44873809,70.81470782)(141.52873047,70.79470998)
\curveto(141.60873793,70.78470785)(141.68873785,70.76470787)(141.76873047,70.73470998)
\curveto(141.87873766,70.70470793)(141.99373755,70.67970795)(142.11373047,70.65970998)
\curveto(142.23373731,70.64970798)(142.35373719,70.629708)(142.47373047,70.59970998)
\curveto(142.52373702,70.57970805)(142.57373697,70.56970806)(142.62373047,70.56970998)
\curveto(142.67373687,70.57970805)(142.72373682,70.57470806)(142.77373047,70.55470998)
\curveto(142.83373671,70.54470809)(142.91373663,70.54470809)(143.01373047,70.55470998)
\curveto(143.10373644,70.56470807)(143.15873638,70.57970805)(143.17873047,70.59970998)
\curveto(143.21873632,70.61970801)(143.2387363,70.64970798)(143.23873047,70.68970998)
\curveto(143.2387363,70.73970789)(143.22873631,70.78470785)(143.20873047,70.82470998)
\curveto(143.16873637,70.89470774)(143.12373642,70.95470768)(143.07373047,71.00470998)
\curveto(143.02373652,71.05470758)(142.97373657,71.11470752)(142.92373047,71.18470998)
\lineto(142.86373047,71.24470998)
\curveto(142.83373671,71.27470736)(142.80873673,71.30470733)(142.78873047,71.33470998)
\curveto(142.62873691,71.56470707)(142.49373705,71.83970679)(142.38373047,72.15970998)
\curveto(142.36373718,72.2297064)(142.34873719,72.29970633)(142.33873047,72.36970998)
\curveto(142.32873721,72.43970619)(142.31373723,72.51470612)(142.29373047,72.59470998)
\curveto(142.29373725,72.634706)(142.28873725,72.66970596)(142.27873047,72.69970998)
\curveto(142.26873727,72.7297059)(142.26873727,72.76470587)(142.27873047,72.80470998)
\curveto(142.27873726,72.85470578)(142.26873727,72.89470574)(142.24873047,72.92470998)
\lineto(142.24873047,73.08970998)
\lineto(142.24873047,73.17970998)
\curveto(142.2387373,73.2297054)(142.2387373,73.26970536)(142.24873047,73.29970998)
\curveto(142.25873728,73.34970528)(142.26373728,73.39970523)(142.26373047,73.44970998)
\curveto(142.25373729,73.50970512)(142.25373729,73.56470507)(142.26373047,73.61470998)
\curveto(142.29373725,73.72470491)(142.31373723,73.8297048)(142.32373047,73.92970998)
\curveto(142.33373721,74.03970459)(142.35873718,74.14470449)(142.39873047,74.24470998)
\curveto(142.538737,74.66470397)(142.72373682,75.00970362)(142.95373047,75.27970998)
\curveto(143.17373637,75.54970308)(143.45873608,75.78970284)(143.80873047,75.99970998)
\curveto(143.94873559,76.07970255)(144.09873544,76.14470249)(144.25873047,76.19470998)
\curveto(144.40873513,76.24470239)(144.56873497,76.29470234)(144.73873047,76.34470998)
\moveto(146.04373047,75.09970998)
\curveto(145.99373355,75.10970352)(145.94873359,75.11470352)(145.90873047,75.11470998)
\lineto(145.75873047,75.11470998)
\curveto(145.44873409,75.11470352)(145.16373438,75.07470356)(144.90373047,74.99470998)
\curveto(144.8437347,74.97470366)(144.78873475,74.95470368)(144.73873047,74.93470998)
\curveto(144.67873486,74.92470371)(144.62373492,74.90970372)(144.57373047,74.88970998)
\curveto(144.08373546,74.66970396)(143.73373581,74.32470431)(143.52373047,73.85470998)
\curveto(143.49373605,73.77470486)(143.46873607,73.69470494)(143.44873047,73.61470998)
\lineto(143.38873047,73.37470998)
\curveto(143.36873617,73.29470534)(143.35873618,73.20470543)(143.35873047,73.10470998)
\lineto(143.35873047,72.78970998)
\curveto(143.37873616,72.76970586)(143.38873615,72.7297059)(143.38873047,72.66970998)
\curveto(143.37873616,72.61970601)(143.37873616,72.57470606)(143.38873047,72.53470998)
\lineto(143.44873047,72.29470998)
\curveto(143.45873608,72.22470641)(143.47873606,72.15470648)(143.50873047,72.08470998)
\curveto(143.76873577,71.48470715)(144.23373531,71.07970755)(144.90373047,70.86970998)
\curveto(144.98373456,70.83970779)(145.06373448,70.81970781)(145.14373047,70.80970998)
\curveto(145.22373432,70.79970783)(145.30873423,70.78470785)(145.39873047,70.76470998)
\lineto(145.54873047,70.76470998)
\curveto(145.58873395,70.75470788)(145.65873388,70.74970788)(145.75873047,70.74970998)
\curveto(145.98873355,70.74970788)(146.18373336,70.76970786)(146.34373047,70.80970998)
\curveto(146.41373313,70.8297078)(146.47873306,70.84470779)(146.53873047,70.85470998)
\curveto(146.59873294,70.86470777)(146.66373288,70.88470775)(146.73373047,70.91470998)
\curveto(147.01373253,71.02470761)(147.25873228,71.16970746)(147.46873047,71.34970998)
\curveto(147.66873187,71.5297071)(147.82873171,71.76470687)(147.94873047,72.05470998)
\lineto(148.03873047,72.29470998)
\lineto(148.09873047,72.53470998)
\curveto(148.11873142,72.58470605)(148.12373142,72.62470601)(148.11373047,72.65470998)
\curveto(148.10373144,72.69470594)(148.10873143,72.73970589)(148.12873047,72.78970998)
\curveto(148.1387314,72.81970581)(148.1437314,72.87470576)(148.14373047,72.95470998)
\curveto(148.1437314,73.0347056)(148.1387314,73.09470554)(148.12873047,73.13470998)
\curveto(148.10873143,73.24470539)(148.09373145,73.34970528)(148.08373047,73.44970998)
\curveto(148.07373147,73.54970508)(148.0437315,73.64470499)(147.99373047,73.73470998)
\curveto(147.79373175,74.26470437)(147.41873212,74.65470398)(146.86873047,74.90470998)
\curveto(146.76873277,74.94470369)(146.66373288,74.97470366)(146.55373047,74.99470998)
\lineto(146.22373047,75.08470998)
\curveto(146.1437334,75.08470355)(146.08373346,75.08970354)(146.04373047,75.09970998)
}
}
{
\newrgbcolor{curcolor}{0 0 0}
\pscustom[linestyle=none,fillstyle=solid,fillcolor=curcolor]
{
\newpath
\moveto(147.42373047,78.63431936)
\lineto(147.42373047,79.26431936)
\lineto(147.42373047,79.45931936)
\curveto(147.42373212,79.52931683)(147.43373211,79.58931677)(147.45373047,79.63931936)
\curveto(147.49373205,79.70931665)(147.53373201,79.7593166)(147.57373047,79.78931936)
\curveto(147.62373192,79.82931653)(147.68873185,79.84931651)(147.76873047,79.84931936)
\curveto(147.84873169,79.8593165)(147.93373161,79.86431649)(148.02373047,79.86431936)
\lineto(148.74373047,79.86431936)
\curveto(149.22373032,79.86431649)(149.63372991,79.80431655)(149.97373047,79.68431936)
\curveto(150.31372923,79.56431679)(150.58872895,79.36931699)(150.79873047,79.09931936)
\curveto(150.84872869,79.02931733)(150.89372865,78.9593174)(150.93373047,78.88931936)
\curveto(150.98372856,78.82931753)(151.02872851,78.7543176)(151.06873047,78.66431936)
\curveto(151.07872846,78.64431771)(151.08872845,78.61431774)(151.09873047,78.57431936)
\curveto(151.11872842,78.53431782)(151.12372842,78.48931787)(151.11373047,78.43931936)
\curveto(151.08372846,78.34931801)(151.00872853,78.29431806)(150.88873047,78.27431936)
\curveto(150.77872876,78.2543181)(150.68372886,78.26931809)(150.60373047,78.31931936)
\curveto(150.53372901,78.34931801)(150.46872907,78.39431796)(150.40873047,78.45431936)
\curveto(150.35872918,78.52431783)(150.30872923,78.58931777)(150.25873047,78.64931936)
\curveto(150.20872933,78.71931764)(150.13372941,78.77931758)(150.03373047,78.82931936)
\curveto(149.9437296,78.88931747)(149.85372969,78.93931742)(149.76373047,78.97931936)
\curveto(149.73372981,78.99931736)(149.67372987,79.02431733)(149.58373047,79.05431936)
\curveto(149.50373004,79.08431727)(149.43373011,79.08931727)(149.37373047,79.06931936)
\curveto(149.23373031,79.03931732)(149.1437304,78.97931738)(149.10373047,78.88931936)
\curveto(149.07373047,78.80931755)(149.05873048,78.71931764)(149.05873047,78.61931936)
\curveto(149.05873048,78.51931784)(149.03373051,78.43431792)(148.98373047,78.36431936)
\curveto(148.91373063,78.27431808)(148.77373077,78.22931813)(148.56373047,78.22931936)
\lineto(148.00873047,78.22931936)
\lineto(147.78373047,78.22931936)
\curveto(147.70373184,78.23931812)(147.6387319,78.2593181)(147.58873047,78.28931936)
\curveto(147.50873203,78.34931801)(147.46373208,78.41931794)(147.45373047,78.49931936)
\curveto(147.4437321,78.51931784)(147.4387321,78.53931782)(147.43873047,78.55931936)
\curveto(147.4387321,78.58931777)(147.43373211,78.61431774)(147.42373047,78.63431936)
}
}
{
\newrgbcolor{curcolor}{0 0 0}
\pscustom[linestyle=none,fillstyle=solid,fillcolor=curcolor]
{
}
}
{
\newrgbcolor{curcolor}{0 0 0}
\pscustom[linestyle=none,fillstyle=solid,fillcolor=curcolor]
{
\newpath
\moveto(138.45373047,89.26463186)
\curveto(138.4437411,89.95462722)(138.56374098,90.55462662)(138.81373047,91.06463186)
\curveto(139.06374048,91.58462559)(139.39874014,91.9796252)(139.81873047,92.24963186)
\curveto(139.89873964,92.29962488)(139.98873955,92.34462483)(140.08873047,92.38463186)
\curveto(140.17873936,92.42462475)(140.27373927,92.46962471)(140.37373047,92.51963186)
\curveto(140.47373907,92.55962462)(140.57373897,92.58962459)(140.67373047,92.60963186)
\curveto(140.77373877,92.62962455)(140.87873866,92.64962453)(140.98873047,92.66963186)
\curveto(141.0387385,92.68962449)(141.08373846,92.69462448)(141.12373047,92.68463186)
\curveto(141.16373838,92.6746245)(141.20873833,92.6796245)(141.25873047,92.69963186)
\curveto(141.30873823,92.70962447)(141.39373815,92.71462446)(141.51373047,92.71463186)
\curveto(141.62373792,92.71462446)(141.70873783,92.70962447)(141.76873047,92.69963186)
\curveto(141.82873771,92.6796245)(141.88873765,92.66962451)(141.94873047,92.66963186)
\curveto(142.00873753,92.6796245)(142.06873747,92.6746245)(142.12873047,92.65463186)
\curveto(142.26873727,92.61462456)(142.40373714,92.5796246)(142.53373047,92.54963186)
\curveto(142.66373688,92.51962466)(142.78873675,92.4796247)(142.90873047,92.42963186)
\curveto(143.04873649,92.36962481)(143.17373637,92.29962488)(143.28373047,92.21963186)
\curveto(143.39373615,92.14962503)(143.50373604,92.0746251)(143.61373047,91.99463186)
\lineto(143.67373047,91.93463186)
\curveto(143.69373585,91.92462525)(143.71373583,91.90962527)(143.73373047,91.88963186)
\curveto(143.89373565,91.76962541)(144.0387355,91.63462554)(144.16873047,91.48463186)
\curveto(144.29873524,91.33462584)(144.42373512,91.174626)(144.54373047,91.00463186)
\curveto(144.76373478,90.69462648)(144.96873457,90.39962678)(145.15873047,90.11963186)
\curveto(145.29873424,89.88962729)(145.43373411,89.65962752)(145.56373047,89.42963186)
\curveto(145.69373385,89.20962797)(145.82873371,88.98962819)(145.96873047,88.76963186)
\curveto(146.1387334,88.51962866)(146.31873322,88.2796289)(146.50873047,88.04963186)
\curveto(146.69873284,87.82962935)(146.92373262,87.63962954)(147.18373047,87.47963186)
\curveto(147.2437323,87.43962974)(147.30373224,87.40462977)(147.36373047,87.37463186)
\curveto(147.41373213,87.34462983)(147.47873206,87.31462986)(147.55873047,87.28463186)
\curveto(147.62873191,87.26462991)(147.68873185,87.25962992)(147.73873047,87.26963186)
\curveto(147.80873173,87.28962989)(147.86373168,87.32462985)(147.90373047,87.37463186)
\curveto(147.93373161,87.42462975)(147.95373159,87.48462969)(147.96373047,87.55463186)
\lineto(147.96373047,87.79463186)
\lineto(147.96373047,88.54463186)
\lineto(147.96373047,91.34963186)
\lineto(147.96373047,92.00963186)
\curveto(147.96373158,92.09962508)(147.96873157,92.18462499)(147.97873047,92.26463186)
\curveto(147.97873156,92.34462483)(147.99873154,92.40962477)(148.03873047,92.45963186)
\curveto(148.07873146,92.50962467)(148.15373139,92.54962463)(148.26373047,92.57963186)
\curveto(148.36373118,92.61962456)(148.46373108,92.62962455)(148.56373047,92.60963186)
\lineto(148.69873047,92.60963186)
\curveto(148.76873077,92.58962459)(148.82873071,92.56962461)(148.87873047,92.54963186)
\curveto(148.92873061,92.52962465)(148.96873057,92.49462468)(148.99873047,92.44463186)
\curveto(149.0387305,92.39462478)(149.05873048,92.32462485)(149.05873047,92.23463186)
\lineto(149.05873047,91.96463186)
\lineto(149.05873047,91.06463186)
\lineto(149.05873047,87.55463186)
\lineto(149.05873047,86.48963186)
\curveto(149.05873048,86.40963077)(149.06373048,86.31963086)(149.07373047,86.21963186)
\curveto(149.07373047,86.11963106)(149.06373048,86.03463114)(149.04373047,85.96463186)
\curveto(148.97373057,85.75463142)(148.79373075,85.68963149)(148.50373047,85.76963186)
\curveto(148.46373108,85.7796314)(148.42873111,85.7796314)(148.39873047,85.76963186)
\curveto(148.35873118,85.76963141)(148.31373123,85.7796314)(148.26373047,85.79963186)
\curveto(148.18373136,85.81963136)(148.09873144,85.83963134)(148.00873047,85.85963186)
\curveto(147.91873162,85.8796313)(147.83373171,85.90463127)(147.75373047,85.93463186)
\curveto(147.26373228,86.09463108)(146.84873269,86.29463088)(146.50873047,86.53463186)
\curveto(146.25873328,86.71463046)(146.03373351,86.91963026)(145.83373047,87.14963186)
\curveto(145.62373392,87.3796298)(145.42873411,87.61962956)(145.24873047,87.86963186)
\curveto(145.06873447,88.12962905)(144.89873464,88.39462878)(144.73873047,88.66463186)
\curveto(144.56873497,88.94462823)(144.39373515,89.21462796)(144.21373047,89.47463186)
\curveto(144.13373541,89.58462759)(144.05873548,89.68962749)(143.98873047,89.78963186)
\curveto(143.91873562,89.89962728)(143.8437357,90.00962717)(143.76373047,90.11963186)
\curveto(143.73373581,90.15962702)(143.70373584,90.19462698)(143.67373047,90.22463186)
\curveto(143.63373591,90.26462691)(143.60373594,90.30462687)(143.58373047,90.34463186)
\curveto(143.47373607,90.48462669)(143.34873619,90.60962657)(143.20873047,90.71963186)
\curveto(143.17873636,90.73962644)(143.15373639,90.76462641)(143.13373047,90.79463186)
\curveto(143.10373644,90.82462635)(143.07373647,90.84962633)(143.04373047,90.86963186)
\curveto(142.9437366,90.94962623)(142.8437367,91.01462616)(142.74373047,91.06463186)
\curveto(142.6437369,91.12462605)(142.53373701,91.179626)(142.41373047,91.22963186)
\curveto(142.3437372,91.25962592)(142.26873727,91.2796259)(142.18873047,91.28963186)
\lineto(141.94873047,91.34963186)
\lineto(141.85873047,91.34963186)
\curveto(141.82873771,91.35962582)(141.79873774,91.36462581)(141.76873047,91.36463186)
\curveto(141.69873784,91.38462579)(141.60373794,91.38962579)(141.48373047,91.37963186)
\curveto(141.35373819,91.3796258)(141.25373829,91.36962581)(141.18373047,91.34963186)
\curveto(141.10373844,91.32962585)(141.02873851,91.30962587)(140.95873047,91.28963186)
\curveto(140.87873866,91.2796259)(140.79873874,91.25962592)(140.71873047,91.22963186)
\curveto(140.47873906,91.11962606)(140.27873926,90.96962621)(140.11873047,90.77963186)
\curveto(139.94873959,90.59962658)(139.80873973,90.3796268)(139.69873047,90.11963186)
\curveto(139.67873986,90.04962713)(139.66373988,89.9796272)(139.65373047,89.90963186)
\curveto(139.63373991,89.83962734)(139.61373993,89.76462741)(139.59373047,89.68463186)
\curveto(139.57373997,89.60462757)(139.56373998,89.49462768)(139.56373047,89.35463186)
\curveto(139.56373998,89.22462795)(139.57373997,89.11962806)(139.59373047,89.03963186)
\curveto(139.60373994,88.9796282)(139.60873993,88.92462825)(139.60873047,88.87463186)
\curveto(139.60873993,88.82462835)(139.61873992,88.7746284)(139.63873047,88.72463186)
\curveto(139.67873986,88.62462855)(139.71873982,88.52962865)(139.75873047,88.43963186)
\curveto(139.79873974,88.35962882)(139.8437397,88.2796289)(139.89373047,88.19963186)
\curveto(139.91373963,88.16962901)(139.9387396,88.13962904)(139.96873047,88.10963186)
\curveto(139.99873954,88.08962909)(140.02373952,88.06462911)(140.04373047,88.03463186)
\lineto(140.11873047,87.95963186)
\curveto(140.1387394,87.92962925)(140.15873938,87.90462927)(140.17873047,87.88463186)
\lineto(140.38873047,87.73463186)
\curveto(140.44873909,87.69462948)(140.51373903,87.64962953)(140.58373047,87.59963186)
\curveto(140.67373887,87.53962964)(140.77873876,87.48962969)(140.89873047,87.44963186)
\curveto(141.00873853,87.41962976)(141.11873842,87.38462979)(141.22873047,87.34463186)
\curveto(141.3387382,87.30462987)(141.48373806,87.2796299)(141.66373047,87.26963186)
\curveto(141.83373771,87.25962992)(141.95873758,87.22962995)(142.03873047,87.17963186)
\curveto(142.11873742,87.12963005)(142.16373738,87.05463012)(142.17373047,86.95463186)
\curveto(142.18373736,86.85463032)(142.18873735,86.74463043)(142.18873047,86.62463186)
\curveto(142.18873735,86.58463059)(142.19373735,86.54463063)(142.20373047,86.50463186)
\curveto(142.20373734,86.46463071)(142.19873734,86.42963075)(142.18873047,86.39963186)
\curveto(142.16873737,86.34963083)(142.15873738,86.29963088)(142.15873047,86.24963186)
\curveto(142.15873738,86.20963097)(142.14873739,86.16963101)(142.12873047,86.12963186)
\curveto(142.06873747,86.03963114)(141.93373761,85.99463118)(141.72373047,85.99463186)
\lineto(141.60373047,85.99463186)
\curveto(141.543738,86.00463117)(141.48373806,86.00963117)(141.42373047,86.00963186)
\curveto(141.35373819,86.01963116)(141.28873825,86.02963115)(141.22873047,86.03963186)
\curveto(141.11873842,86.05963112)(141.01873852,86.0796311)(140.92873047,86.09963186)
\curveto(140.82873871,86.11963106)(140.73373881,86.14963103)(140.64373047,86.18963186)
\curveto(140.57373897,86.20963097)(140.51373903,86.22963095)(140.46373047,86.24963186)
\lineto(140.28373047,86.30963186)
\curveto(140.02373952,86.42963075)(139.77873976,86.58463059)(139.54873047,86.77463186)
\curveto(139.31874022,86.9746302)(139.13374041,87.18962999)(138.99373047,87.41963186)
\curveto(138.91374063,87.52962965)(138.84874069,87.64462953)(138.79873047,87.76463186)
\lineto(138.64873047,88.15463186)
\curveto(138.59874094,88.26462891)(138.56874097,88.3796288)(138.55873047,88.49963186)
\curveto(138.538741,88.61962856)(138.51374103,88.74462843)(138.48373047,88.87463186)
\curveto(138.48374106,88.94462823)(138.48374106,89.00962817)(138.48373047,89.06963186)
\curveto(138.47374107,89.12962805)(138.46374108,89.19462798)(138.45373047,89.26463186)
}
}
{
\newrgbcolor{curcolor}{0 0 0}
\pscustom[linestyle=none,fillstyle=solid,fillcolor=curcolor]
{
\newpath
\moveto(143.97373047,101.36424123)
\lineto(144.22873047,101.36424123)
\curveto(144.30873523,101.37423353)(144.38373516,101.36923353)(144.45373047,101.34924123)
\lineto(144.69373047,101.34924123)
\lineto(144.85873047,101.34924123)
\curveto(144.95873458,101.32923357)(145.06373448,101.31923358)(145.17373047,101.31924123)
\curveto(145.27373427,101.31923358)(145.37373417,101.30923359)(145.47373047,101.28924123)
\lineto(145.62373047,101.28924123)
\curveto(145.76373378,101.25923364)(145.90373364,101.23923366)(146.04373047,101.22924123)
\curveto(146.17373337,101.21923368)(146.30373324,101.19423371)(146.43373047,101.15424123)
\curveto(146.51373303,101.13423377)(146.59873294,101.11423379)(146.68873047,101.09424123)
\lineto(146.92873047,101.03424123)
\lineto(147.22873047,100.91424123)
\curveto(147.31873222,100.88423402)(147.40873213,100.84923405)(147.49873047,100.80924123)
\curveto(147.71873182,100.70923419)(147.93373161,100.57423433)(148.14373047,100.40424123)
\curveto(148.35373119,100.24423466)(148.52373102,100.06923483)(148.65373047,99.87924123)
\curveto(148.69373085,99.82923507)(148.73373081,99.76923513)(148.77373047,99.69924123)
\curveto(148.80373074,99.63923526)(148.8387307,99.57923532)(148.87873047,99.51924123)
\curveto(148.92873061,99.43923546)(148.96873057,99.34423556)(148.99873047,99.23424123)
\curveto(149.02873051,99.12423578)(149.05873048,99.01923588)(149.08873047,98.91924123)
\curveto(149.12873041,98.80923609)(149.15373039,98.6992362)(149.16373047,98.58924123)
\curveto(149.17373037,98.47923642)(149.18873035,98.36423654)(149.20873047,98.24424123)
\curveto(149.21873032,98.2042367)(149.21873032,98.15923674)(149.20873047,98.10924123)
\curveto(149.20873033,98.06923683)(149.21373033,98.02923687)(149.22373047,97.98924123)
\curveto(149.23373031,97.94923695)(149.2387303,97.89423701)(149.23873047,97.82424123)
\curveto(149.2387303,97.75423715)(149.23373031,97.7042372)(149.22373047,97.67424123)
\curveto(149.20373034,97.62423728)(149.19873034,97.57923732)(149.20873047,97.53924123)
\curveto(149.21873032,97.4992374)(149.21873032,97.46423744)(149.20873047,97.43424123)
\lineto(149.20873047,97.34424123)
\curveto(149.18873035,97.28423762)(149.17373037,97.21923768)(149.16373047,97.14924123)
\curveto(149.16373038,97.08923781)(149.15873038,97.02423788)(149.14873047,96.95424123)
\curveto(149.09873044,96.78423812)(149.04873049,96.62423828)(148.99873047,96.47424123)
\curveto(148.94873059,96.32423858)(148.88373066,96.17923872)(148.80373047,96.03924123)
\curveto(148.76373078,95.98923891)(148.73373081,95.93423897)(148.71373047,95.87424123)
\curveto(148.68373086,95.82423908)(148.64873089,95.77423913)(148.60873047,95.72424123)
\curveto(148.42873111,95.48423942)(148.20873133,95.28423962)(147.94873047,95.12424123)
\curveto(147.68873185,94.96423994)(147.40373214,94.82424008)(147.09373047,94.70424123)
\curveto(146.95373259,94.64424026)(146.81373273,94.5992403)(146.67373047,94.56924123)
\curveto(146.52373302,94.53924036)(146.36873317,94.5042404)(146.20873047,94.46424123)
\curveto(146.09873344,94.44424046)(145.98873355,94.42924047)(145.87873047,94.41924123)
\curveto(145.76873377,94.40924049)(145.65873388,94.39424051)(145.54873047,94.37424123)
\curveto(145.50873403,94.36424054)(145.46873407,94.35924054)(145.42873047,94.35924123)
\curveto(145.38873415,94.36924053)(145.34873419,94.36924053)(145.30873047,94.35924123)
\curveto(145.25873428,94.34924055)(145.20873433,94.34424056)(145.15873047,94.34424123)
\lineto(144.99373047,94.34424123)
\curveto(144.9437346,94.32424058)(144.89373465,94.31924058)(144.84373047,94.32924123)
\curveto(144.78373476,94.33924056)(144.72873481,94.33924056)(144.67873047,94.32924123)
\curveto(144.6387349,94.31924058)(144.59373495,94.31924058)(144.54373047,94.32924123)
\curveto(144.49373505,94.33924056)(144.4437351,94.33424057)(144.39373047,94.31424123)
\curveto(144.32373522,94.29424061)(144.24873529,94.28924061)(144.16873047,94.29924123)
\curveto(144.07873546,94.30924059)(143.99373555,94.31424059)(143.91373047,94.31424123)
\curveto(143.82373572,94.31424059)(143.72373582,94.30924059)(143.61373047,94.29924123)
\curveto(143.49373605,94.28924061)(143.39373615,94.29424061)(143.31373047,94.31424123)
\lineto(143.02873047,94.31424123)
\lineto(142.39873047,94.35924123)
\curveto(142.29873724,94.36924053)(142.20373734,94.37924052)(142.11373047,94.38924123)
\lineto(141.81373047,94.41924123)
\curveto(141.76373778,94.43924046)(141.71373783,94.44424046)(141.66373047,94.43424123)
\curveto(141.60373794,94.43424047)(141.54873799,94.44424046)(141.49873047,94.46424123)
\curveto(141.32873821,94.51424039)(141.16373838,94.55424035)(141.00373047,94.58424123)
\curveto(140.83373871,94.61424029)(140.67373887,94.66424024)(140.52373047,94.73424123)
\curveto(140.06373948,94.92423998)(139.68873985,95.14423976)(139.39873047,95.39424123)
\curveto(139.10874043,95.65423925)(138.86374068,96.01423889)(138.66373047,96.47424123)
\curveto(138.61374093,96.6042383)(138.57874096,96.73423817)(138.55873047,96.86424123)
\curveto(138.538741,97.0042379)(138.51374103,97.14423776)(138.48373047,97.28424123)
\curveto(138.47374107,97.35423755)(138.46874107,97.41923748)(138.46873047,97.47924123)
\curveto(138.46874107,97.53923736)(138.46374108,97.6042373)(138.45373047,97.67424123)
\curveto(138.43374111,98.5042364)(138.58374096,99.17423573)(138.90373047,99.68424123)
\curveto(139.21374033,100.19423471)(139.65373989,100.57423433)(140.22373047,100.82424123)
\curveto(140.3437392,100.87423403)(140.46873907,100.91923398)(140.59873047,100.95924123)
\curveto(140.72873881,100.9992339)(140.86373868,101.04423386)(141.00373047,101.09424123)
\curveto(141.08373846,101.11423379)(141.16873837,101.12923377)(141.25873047,101.13924123)
\lineto(141.49873047,101.19924123)
\curveto(141.60873793,101.22923367)(141.71873782,101.24423366)(141.82873047,101.24424123)
\curveto(141.9387376,101.25423365)(142.04873749,101.26923363)(142.15873047,101.28924123)
\curveto(142.20873733,101.30923359)(142.25373729,101.31423359)(142.29373047,101.30424123)
\curveto(142.33373721,101.3042336)(142.37373717,101.30923359)(142.41373047,101.31924123)
\curveto(142.46373708,101.32923357)(142.51873702,101.32923357)(142.57873047,101.31924123)
\curveto(142.62873691,101.31923358)(142.67873686,101.32423358)(142.72873047,101.33424123)
\lineto(142.86373047,101.33424123)
\curveto(142.92373662,101.35423355)(142.99373655,101.35423355)(143.07373047,101.33424123)
\curveto(143.1437364,101.32423358)(143.20873633,101.32923357)(143.26873047,101.34924123)
\curveto(143.29873624,101.35923354)(143.3387362,101.36423354)(143.38873047,101.36424123)
\lineto(143.50873047,101.36424123)
\lineto(143.97373047,101.36424123)
\moveto(146.29873047,99.81924123)
\curveto(145.97873356,99.91923498)(145.61373393,99.97923492)(145.20373047,99.99924123)
\curveto(144.79373475,100.01923488)(144.38373516,100.02923487)(143.97373047,100.02924123)
\curveto(143.543736,100.02923487)(143.12373642,100.01923488)(142.71373047,99.99924123)
\curveto(142.30373724,99.97923492)(141.91873762,99.93423497)(141.55873047,99.86424123)
\curveto(141.19873834,99.79423511)(140.87873866,99.68423522)(140.59873047,99.53424123)
\curveto(140.30873923,99.39423551)(140.07373947,99.1992357)(139.89373047,98.94924123)
\curveto(139.78373976,98.78923611)(139.70373984,98.60923629)(139.65373047,98.40924123)
\curveto(139.59373995,98.20923669)(139.56373998,97.96423694)(139.56373047,97.67424123)
\curveto(139.58373996,97.65423725)(139.59373995,97.61923728)(139.59373047,97.56924123)
\curveto(139.58373996,97.51923738)(139.58373996,97.47923742)(139.59373047,97.44924123)
\curveto(139.61373993,97.36923753)(139.63373991,97.29423761)(139.65373047,97.22424123)
\curveto(139.66373988,97.16423774)(139.68373986,97.0992378)(139.71373047,97.02924123)
\curveto(139.83373971,96.75923814)(140.00373954,96.53923836)(140.22373047,96.36924123)
\curveto(140.43373911,96.20923869)(140.67873886,96.07423883)(140.95873047,95.96424123)
\curveto(141.06873847,95.91423899)(141.18873835,95.87423903)(141.31873047,95.84424123)
\curveto(141.4387381,95.82423908)(141.56373798,95.7992391)(141.69373047,95.76924123)
\curveto(141.7437378,95.74923915)(141.79873774,95.73923916)(141.85873047,95.73924123)
\curveto(141.90873763,95.73923916)(141.95873758,95.73423917)(142.00873047,95.72424123)
\curveto(142.09873744,95.71423919)(142.19373735,95.7042392)(142.29373047,95.69424123)
\curveto(142.38373716,95.68423922)(142.47873706,95.67423923)(142.57873047,95.66424123)
\curveto(142.65873688,95.66423924)(142.7437368,95.65923924)(142.83373047,95.64924123)
\lineto(143.07373047,95.64924123)
\lineto(143.25373047,95.64924123)
\curveto(143.28373626,95.63923926)(143.31873622,95.63423927)(143.35873047,95.63424123)
\lineto(143.49373047,95.63424123)
\lineto(143.94373047,95.63424123)
\curveto(144.02373552,95.63423927)(144.10873543,95.62923927)(144.19873047,95.61924123)
\curveto(144.27873526,95.61923928)(144.35373519,95.62923927)(144.42373047,95.64924123)
\lineto(144.69373047,95.64924123)
\curveto(144.71373483,95.64923925)(144.7437348,95.64423926)(144.78373047,95.63424123)
\curveto(144.81373473,95.63423927)(144.8387347,95.63923926)(144.85873047,95.64924123)
\curveto(144.95873458,95.65923924)(145.05873448,95.66423924)(145.15873047,95.66424123)
\curveto(145.24873429,95.67423923)(145.34873419,95.68423922)(145.45873047,95.69424123)
\curveto(145.57873396,95.72423918)(145.70373384,95.73923916)(145.83373047,95.73924123)
\curveto(145.95373359,95.74923915)(146.06873347,95.77423913)(146.17873047,95.81424123)
\curveto(146.47873306,95.89423901)(146.7437328,95.97923892)(146.97373047,96.06924123)
\curveto(147.20373234,96.16923873)(147.41873212,96.31423859)(147.61873047,96.50424123)
\curveto(147.81873172,96.71423819)(147.96873157,96.97923792)(148.06873047,97.29924123)
\curveto(148.08873145,97.33923756)(148.09873144,97.37423753)(148.09873047,97.40424123)
\curveto(148.08873145,97.44423746)(148.09373145,97.48923741)(148.11373047,97.53924123)
\curveto(148.12373142,97.57923732)(148.13373141,97.64923725)(148.14373047,97.74924123)
\curveto(148.15373139,97.85923704)(148.14873139,97.94423696)(148.12873047,98.00424123)
\curveto(148.10873143,98.07423683)(148.09873144,98.14423676)(148.09873047,98.21424123)
\curveto(148.08873145,98.28423662)(148.07373147,98.34923655)(148.05373047,98.40924123)
\curveto(147.99373155,98.60923629)(147.90873163,98.78923611)(147.79873047,98.94924123)
\curveto(147.77873176,98.97923592)(147.75873178,99.0042359)(147.73873047,99.02424123)
\lineto(147.67873047,99.08424123)
\curveto(147.65873188,99.12423578)(147.61873192,99.17423573)(147.55873047,99.23424123)
\curveto(147.41873212,99.33423557)(147.28873225,99.41923548)(147.16873047,99.48924123)
\curveto(147.04873249,99.55923534)(146.90373264,99.62923527)(146.73373047,99.69924123)
\curveto(146.66373288,99.72923517)(146.59373295,99.74923515)(146.52373047,99.75924123)
\curveto(146.45373309,99.77923512)(146.37873316,99.7992351)(146.29873047,99.81924123)
}
}
{
\newrgbcolor{curcolor}{0 0 0}
\pscustom[linestyle=none,fillstyle=solid,fillcolor=curcolor]
{
\newpath
\moveto(138.45373047,106.77385061)
\curveto(138.45374109,106.87384575)(138.46374108,106.96884566)(138.48373047,107.05885061)
\curveto(138.49374105,107.14884548)(138.52374102,107.21384541)(138.57373047,107.25385061)
\curveto(138.65374089,107.31384531)(138.75874078,107.34384528)(138.88873047,107.34385061)
\lineto(139.27873047,107.34385061)
\lineto(140.77873047,107.34385061)
\lineto(147.16873047,107.34385061)
\lineto(148.33873047,107.34385061)
\lineto(148.65373047,107.34385061)
\curveto(148.75373079,107.35384527)(148.83373071,107.33884529)(148.89373047,107.29885061)
\curveto(148.97373057,107.24884538)(149.02373052,107.17384545)(149.04373047,107.07385061)
\curveto(149.05373049,106.98384564)(149.05873048,106.87384575)(149.05873047,106.74385061)
\lineto(149.05873047,106.51885061)
\curveto(149.0387305,106.43884619)(149.02373052,106.36884626)(149.01373047,106.30885061)
\curveto(148.99373055,106.24884638)(148.95373059,106.19884643)(148.89373047,106.15885061)
\curveto(148.83373071,106.11884651)(148.75873078,106.09884653)(148.66873047,106.09885061)
\lineto(148.36873047,106.09885061)
\lineto(147.27373047,106.09885061)
\lineto(141.93373047,106.09885061)
\curveto(141.8437377,106.07884655)(141.76873777,106.06384656)(141.70873047,106.05385061)
\curveto(141.6387379,106.05384657)(141.57873796,106.0238466)(141.52873047,105.96385061)
\curveto(141.47873806,105.89384673)(141.45373809,105.80384682)(141.45373047,105.69385061)
\curveto(141.4437381,105.59384703)(141.4387381,105.48384714)(141.43873047,105.36385061)
\lineto(141.43873047,104.22385061)
\lineto(141.43873047,103.72885061)
\curveto(141.42873811,103.56884906)(141.36873817,103.45884917)(141.25873047,103.39885061)
\curveto(141.22873831,103.37884925)(141.19873834,103.36884926)(141.16873047,103.36885061)
\curveto(141.12873841,103.36884926)(141.08373846,103.36384926)(141.03373047,103.35385061)
\curveto(140.91373863,103.33384929)(140.80373874,103.33884929)(140.70373047,103.36885061)
\curveto(140.60373894,103.40884922)(140.53373901,103.46384916)(140.49373047,103.53385061)
\curveto(140.4437391,103.61384901)(140.41873912,103.73384889)(140.41873047,103.89385061)
\curveto(140.41873912,104.05384857)(140.40373914,104.18884844)(140.37373047,104.29885061)
\curveto(140.36373918,104.34884828)(140.35873918,104.40384822)(140.35873047,104.46385061)
\curveto(140.34873919,104.5238481)(140.33373921,104.58384804)(140.31373047,104.64385061)
\curveto(140.26373928,104.79384783)(140.21373933,104.93884769)(140.16373047,105.07885061)
\curveto(140.10373944,105.21884741)(140.03373951,105.35384727)(139.95373047,105.48385061)
\curveto(139.86373968,105.623847)(139.75873978,105.74384688)(139.63873047,105.84385061)
\curveto(139.51874002,105.94384668)(139.38874015,106.03884659)(139.24873047,106.12885061)
\curveto(139.14874039,106.18884644)(139.0387405,106.23384639)(138.91873047,106.26385061)
\curveto(138.79874074,106.30384632)(138.69374085,106.35384627)(138.60373047,106.41385061)
\curveto(138.543741,106.46384616)(138.50374104,106.53384609)(138.48373047,106.62385061)
\curveto(138.47374107,106.64384598)(138.46874107,106.66884596)(138.46873047,106.69885061)
\curveto(138.46874107,106.7288459)(138.46374108,106.75384587)(138.45373047,106.77385061)
}
}
{
\newrgbcolor{curcolor}{0 0 0}
\pscustom[linestyle=none,fillstyle=solid,fillcolor=curcolor]
{
\newpath
\moveto(138.45373047,115.12345998)
\curveto(138.45374109,115.22345513)(138.46374108,115.31845503)(138.48373047,115.40845998)
\curveto(138.49374105,115.49845485)(138.52374102,115.56345479)(138.57373047,115.60345998)
\curveto(138.65374089,115.66345469)(138.75874078,115.69345466)(138.88873047,115.69345998)
\lineto(139.27873047,115.69345998)
\lineto(140.77873047,115.69345998)
\lineto(147.16873047,115.69345998)
\lineto(148.33873047,115.69345998)
\lineto(148.65373047,115.69345998)
\curveto(148.75373079,115.70345465)(148.83373071,115.68845466)(148.89373047,115.64845998)
\curveto(148.97373057,115.59845475)(149.02373052,115.52345483)(149.04373047,115.42345998)
\curveto(149.05373049,115.33345502)(149.05873048,115.22345513)(149.05873047,115.09345998)
\lineto(149.05873047,114.86845998)
\curveto(149.0387305,114.78845556)(149.02373052,114.71845563)(149.01373047,114.65845998)
\curveto(148.99373055,114.59845575)(148.95373059,114.5484558)(148.89373047,114.50845998)
\curveto(148.83373071,114.46845588)(148.75873078,114.4484559)(148.66873047,114.44845998)
\lineto(148.36873047,114.44845998)
\lineto(147.27373047,114.44845998)
\lineto(141.93373047,114.44845998)
\curveto(141.8437377,114.42845592)(141.76873777,114.41345594)(141.70873047,114.40345998)
\curveto(141.6387379,114.40345595)(141.57873796,114.37345598)(141.52873047,114.31345998)
\curveto(141.47873806,114.24345611)(141.45373809,114.1534562)(141.45373047,114.04345998)
\curveto(141.4437381,113.94345641)(141.4387381,113.83345652)(141.43873047,113.71345998)
\lineto(141.43873047,112.57345998)
\lineto(141.43873047,112.07845998)
\curveto(141.42873811,111.91845843)(141.36873817,111.80845854)(141.25873047,111.74845998)
\curveto(141.22873831,111.72845862)(141.19873834,111.71845863)(141.16873047,111.71845998)
\curveto(141.12873841,111.71845863)(141.08373846,111.71345864)(141.03373047,111.70345998)
\curveto(140.91373863,111.68345867)(140.80373874,111.68845866)(140.70373047,111.71845998)
\curveto(140.60373894,111.75845859)(140.53373901,111.81345854)(140.49373047,111.88345998)
\curveto(140.4437391,111.96345839)(140.41873912,112.08345827)(140.41873047,112.24345998)
\curveto(140.41873912,112.40345795)(140.40373914,112.53845781)(140.37373047,112.64845998)
\curveto(140.36373918,112.69845765)(140.35873918,112.7534576)(140.35873047,112.81345998)
\curveto(140.34873919,112.87345748)(140.33373921,112.93345742)(140.31373047,112.99345998)
\curveto(140.26373928,113.14345721)(140.21373933,113.28845706)(140.16373047,113.42845998)
\curveto(140.10373944,113.56845678)(140.03373951,113.70345665)(139.95373047,113.83345998)
\curveto(139.86373968,113.97345638)(139.75873978,114.09345626)(139.63873047,114.19345998)
\curveto(139.51874002,114.29345606)(139.38874015,114.38845596)(139.24873047,114.47845998)
\curveto(139.14874039,114.53845581)(139.0387405,114.58345577)(138.91873047,114.61345998)
\curveto(138.79874074,114.6534557)(138.69374085,114.70345565)(138.60373047,114.76345998)
\curveto(138.543741,114.81345554)(138.50374104,114.88345547)(138.48373047,114.97345998)
\curveto(138.47374107,114.99345536)(138.46874107,115.01845533)(138.46873047,115.04845998)
\curveto(138.46874107,115.07845527)(138.46374108,115.10345525)(138.45373047,115.12345998)
}
}
{
\newrgbcolor{curcolor}{0 0 0}
\pscustom[linestyle=none,fillstyle=solid,fillcolor=curcolor]
{
\newpath
\moveto(159.29006165,31.67142873)
\lineto(159.29006165,32.58642873)
\curveto(159.29007234,32.68642608)(159.29007234,32.78142599)(159.29006165,32.87142873)
\curveto(159.29007234,32.96142581)(159.31007232,33.03642573)(159.35006165,33.09642873)
\curveto(159.41007222,33.18642558)(159.49007214,33.24642552)(159.59006165,33.27642873)
\curveto(159.69007194,33.31642545)(159.79507184,33.36142541)(159.90506165,33.41142873)
\curveto(160.09507154,33.49142528)(160.28507135,33.56142521)(160.47506165,33.62142873)
\curveto(160.66507097,33.69142508)(160.85507078,33.766425)(161.04506165,33.84642873)
\curveto(161.22507041,33.91642485)(161.41007022,33.98142479)(161.60006165,34.04142873)
\curveto(161.78006985,34.10142467)(161.96006967,34.1714246)(162.14006165,34.25142873)
\curveto(162.28006935,34.31142446)(162.42506921,34.3664244)(162.57506165,34.41642873)
\curveto(162.72506891,34.4664243)(162.87006876,34.52142425)(163.01006165,34.58142873)
\curveto(163.46006817,34.76142401)(163.91506772,34.93142384)(164.37506165,35.09142873)
\curveto(164.82506681,35.25142352)(165.27506636,35.42142335)(165.72506165,35.60142873)
\curveto(165.77506586,35.62142315)(165.82506581,35.63642313)(165.87506165,35.64642873)
\lineto(166.02506165,35.70642873)
\curveto(166.24506539,35.79642297)(166.47006516,35.88142289)(166.70006165,35.96142873)
\curveto(166.92006471,36.04142273)(167.14006449,36.12642264)(167.36006165,36.21642873)
\curveto(167.45006418,36.25642251)(167.56006407,36.29642247)(167.69006165,36.33642873)
\curveto(167.81006382,36.37642239)(167.88006375,36.44142233)(167.90006165,36.53142873)
\curveto(167.91006372,36.5714222)(167.91006372,36.60142217)(167.90006165,36.62142873)
\lineto(167.84006165,36.68142873)
\curveto(167.79006384,36.73142204)(167.7350639,36.766422)(167.67506165,36.78642873)
\curveto(167.61506402,36.81642195)(167.55006408,36.84642192)(167.48006165,36.87642873)
\lineto(166.85006165,37.11642873)
\curveto(166.630065,37.19642157)(166.41506522,37.27642149)(166.20506165,37.35642873)
\lineto(166.05506165,37.41642873)
\lineto(165.87506165,37.47642873)
\curveto(165.68506595,37.55642121)(165.49506614,37.62642114)(165.30506165,37.68642873)
\curveto(165.10506653,37.75642101)(164.90506673,37.83142094)(164.70506165,37.91142873)
\curveto(164.12506751,38.15142062)(163.54006809,38.3714204)(162.95006165,38.57142873)
\curveto(162.36006927,38.78141999)(161.77506986,39.00641976)(161.19506165,39.24642873)
\curveto(160.99507064,39.32641944)(160.79007084,39.40141937)(160.58006165,39.47142873)
\curveto(160.37007126,39.55141922)(160.16507147,39.63141914)(159.96506165,39.71142873)
\curveto(159.88507175,39.75141902)(159.78507185,39.78641898)(159.66506165,39.81642873)
\curveto(159.54507209,39.85641891)(159.46007217,39.91141886)(159.41006165,39.98142873)
\curveto(159.37007226,40.04141873)(159.34007229,40.11641865)(159.32006165,40.20642873)
\curveto(159.30007233,40.30641846)(159.29007234,40.41641835)(159.29006165,40.53642873)
\curveto(159.28007235,40.65641811)(159.28007235,40.77641799)(159.29006165,40.89642873)
\curveto(159.29007234,41.01641775)(159.29007234,41.12641764)(159.29006165,41.22642873)
\curveto(159.29007234,41.31641745)(159.29007234,41.40641736)(159.29006165,41.49642873)
\curveto(159.29007234,41.59641717)(159.31007232,41.6714171)(159.35006165,41.72142873)
\curveto(159.40007223,41.81141696)(159.49007214,41.86141691)(159.62006165,41.87142873)
\curveto(159.75007188,41.88141689)(159.89007174,41.88641688)(160.04006165,41.88642873)
\lineto(161.69006165,41.88642873)
\lineto(167.96006165,41.88642873)
\lineto(169.22006165,41.88642873)
\curveto(169.3300623,41.88641688)(169.44006219,41.88641688)(169.55006165,41.88642873)
\curveto(169.66006197,41.89641687)(169.74506189,41.87641689)(169.80506165,41.82642873)
\curveto(169.86506177,41.79641697)(169.90506173,41.75141702)(169.92506165,41.69142873)
\curveto(169.9350617,41.63141714)(169.95006168,41.56141721)(169.97006165,41.48142873)
\lineto(169.97006165,41.24142873)
\lineto(169.97006165,40.88142873)
\curveto(169.96006167,40.771418)(169.91506172,40.69141808)(169.83506165,40.64142873)
\curveto(169.80506183,40.62141815)(169.77506186,40.60641816)(169.74506165,40.59642873)
\curveto(169.70506193,40.59641817)(169.66006197,40.58641818)(169.61006165,40.56642873)
\lineto(169.44506165,40.56642873)
\curveto(169.38506225,40.55641821)(169.31506232,40.55141822)(169.23506165,40.55142873)
\curveto(169.15506248,40.56141821)(169.08006255,40.5664182)(169.01006165,40.56642873)
\lineto(168.17006165,40.56642873)
\lineto(163.74506165,40.56642873)
\curveto(163.49506814,40.5664182)(163.24506839,40.5664182)(162.99506165,40.56642873)
\curveto(162.7350689,40.5664182)(162.48506915,40.56141821)(162.24506165,40.55142873)
\curveto(162.14506949,40.55141822)(162.0350696,40.54641822)(161.91506165,40.53642873)
\curveto(161.79506984,40.52641824)(161.7350699,40.4714183)(161.73506165,40.37142873)
\lineto(161.75006165,40.37142873)
\curveto(161.77006986,40.30141847)(161.8350698,40.24141853)(161.94506165,40.19142873)
\curveto(162.05506958,40.15141862)(162.15006948,40.11641865)(162.23006165,40.08642873)
\curveto(162.40006923,40.01641875)(162.57506906,39.95141882)(162.75506165,39.89142873)
\curveto(162.92506871,39.83141894)(163.09506854,39.76141901)(163.26506165,39.68142873)
\curveto(163.31506832,39.66141911)(163.36006827,39.64641912)(163.40006165,39.63642873)
\curveto(163.44006819,39.62641914)(163.48506815,39.61141916)(163.53506165,39.59142873)
\curveto(163.71506792,39.51141926)(163.90006773,39.44141933)(164.09006165,39.38142873)
\curveto(164.27006736,39.33141944)(164.45006718,39.2664195)(164.63006165,39.18642873)
\curveto(164.78006685,39.11641965)(164.9350667,39.05641971)(165.09506165,39.00642873)
\curveto(165.24506639,38.95641981)(165.39506624,38.90141987)(165.54506165,38.84142873)
\curveto(166.01506562,38.64142013)(166.49006514,38.46142031)(166.97006165,38.30142873)
\curveto(167.44006419,38.14142063)(167.90506373,37.9664208)(168.36506165,37.77642873)
\curveto(168.54506309,37.69642107)(168.72506291,37.62642114)(168.90506165,37.56642873)
\curveto(169.08506255,37.50642126)(169.26506237,37.44142133)(169.44506165,37.37142873)
\curveto(169.55506208,37.32142145)(169.66006197,37.2714215)(169.76006165,37.22142873)
\curveto(169.85006178,37.18142159)(169.91506172,37.09642167)(169.95506165,36.96642873)
\curveto(169.96506167,36.94642182)(169.97006166,36.92142185)(169.97006165,36.89142873)
\curveto(169.96006167,36.8714219)(169.96006167,36.84642192)(169.97006165,36.81642873)
\curveto(169.98006165,36.78642198)(169.98506165,36.75142202)(169.98506165,36.71142873)
\curveto(169.97506166,36.6714221)(169.97006166,36.63142214)(169.97006165,36.59142873)
\lineto(169.97006165,36.29142873)
\curveto(169.97006166,36.19142258)(169.94506169,36.11142266)(169.89506165,36.05142873)
\curveto(169.84506179,35.9714228)(169.77506186,35.91142286)(169.68506165,35.87142873)
\curveto(169.58506205,35.84142293)(169.48506215,35.80142297)(169.38506165,35.75142873)
\curveto(169.18506245,35.6714231)(168.98006265,35.59142318)(168.77006165,35.51142873)
\curveto(168.55006308,35.44142333)(168.34006329,35.3664234)(168.14006165,35.28642873)
\curveto(167.96006367,35.20642356)(167.78006385,35.13642363)(167.60006165,35.07642873)
\curveto(167.41006422,35.02642374)(167.22506441,34.96142381)(167.04506165,34.88142873)
\curveto(166.48506515,34.65142412)(165.92006571,34.43642433)(165.35006165,34.23642873)
\curveto(164.78006685,34.03642473)(164.21506742,33.82142495)(163.65506165,33.59142873)
\lineto(163.02506165,33.35142873)
\curveto(162.80506883,33.28142549)(162.59506904,33.20642556)(162.39506165,33.12642873)
\curveto(162.28506935,33.07642569)(162.18006945,33.03142574)(162.08006165,32.99142873)
\curveto(161.97006966,32.96142581)(161.87506976,32.91142586)(161.79506165,32.84142873)
\curveto(161.77506986,32.83142594)(161.76506987,32.82142595)(161.76506165,32.81142873)
\lineto(161.73506165,32.78142873)
\lineto(161.73506165,32.70642873)
\lineto(161.76506165,32.67642873)
\curveto(161.76506987,32.6664261)(161.77006986,32.65642611)(161.78006165,32.64642873)
\curveto(161.8300698,32.62642614)(161.88506975,32.61642615)(161.94506165,32.61642873)
\curveto(162.00506963,32.61642615)(162.06506957,32.60642616)(162.12506165,32.58642873)
\lineto(162.29006165,32.58642873)
\curveto(162.35006928,32.5664262)(162.41506922,32.56142621)(162.48506165,32.57142873)
\curveto(162.55506908,32.58142619)(162.62506901,32.58642618)(162.69506165,32.58642873)
\lineto(163.50506165,32.58642873)
\lineto(168.06506165,32.58642873)
\lineto(169.25006165,32.58642873)
\curveto(169.36006227,32.58642618)(169.47006216,32.58142619)(169.58006165,32.57142873)
\curveto(169.69006194,32.5714262)(169.77506186,32.54642622)(169.83506165,32.49642873)
\curveto(169.91506172,32.44642632)(169.96006167,32.35642641)(169.97006165,32.22642873)
\lineto(169.97006165,31.83642873)
\lineto(169.97006165,31.64142873)
\curveto(169.97006166,31.59142718)(169.96006167,31.54142723)(169.94006165,31.49142873)
\curveto(169.90006173,31.36142741)(169.81506182,31.28642748)(169.68506165,31.26642873)
\curveto(169.55506208,31.25642751)(169.40506223,31.25142752)(169.23506165,31.25142873)
\lineto(167.49506165,31.25142873)
\lineto(161.49506165,31.25142873)
\lineto(160.08506165,31.25142873)
\curveto(159.97507166,31.25142752)(159.86007177,31.24642752)(159.74006165,31.23642873)
\curveto(159.62007201,31.23642753)(159.52507211,31.26142751)(159.45506165,31.31142873)
\curveto(159.39507224,31.35142742)(159.34507229,31.42642734)(159.30506165,31.53642873)
\curveto(159.29507234,31.55642721)(159.29507234,31.57642719)(159.30506165,31.59642873)
\curveto(159.30507233,31.62642714)(159.30007233,31.65142712)(159.29006165,31.67142873)
}
}
{
\newrgbcolor{curcolor}{0 0 0}
\pscustom[linestyle=none,fillstyle=solid,fillcolor=curcolor]
{
\newpath
\moveto(169.41506165,50.87353811)
\curveto(169.57506206,50.90353028)(169.71006192,50.88853029)(169.82006165,50.82853811)
\curveto(169.92006171,50.76853041)(169.99506164,50.68853049)(170.04506165,50.58853811)
\curveto(170.06506157,50.53853064)(170.07506156,50.4835307)(170.07506165,50.42353811)
\curveto(170.07506156,50.37353081)(170.08506155,50.31853086)(170.10506165,50.25853811)
\curveto(170.15506148,50.03853114)(170.14006149,49.81853136)(170.06006165,49.59853811)
\curveto(169.99006164,49.38853179)(169.90006173,49.24353194)(169.79006165,49.16353811)
\curveto(169.72006191,49.11353207)(169.64006199,49.06853211)(169.55006165,49.02853811)
\curveto(169.45006218,48.98853219)(169.37006226,48.93853224)(169.31006165,48.87853811)
\curveto(169.29006234,48.85853232)(169.27006236,48.83353235)(169.25006165,48.80353811)
\curveto(169.2300624,48.7835324)(169.22506241,48.75353243)(169.23506165,48.71353811)
\curveto(169.26506237,48.60353258)(169.32006231,48.49853268)(169.40006165,48.39853811)
\curveto(169.48006215,48.30853287)(169.55006208,48.21853296)(169.61006165,48.12853811)
\curveto(169.69006194,47.99853318)(169.76506187,47.85853332)(169.83506165,47.70853811)
\curveto(169.89506174,47.55853362)(169.95006168,47.39853378)(170.00006165,47.22853811)
\curveto(170.0300616,47.12853405)(170.05006158,47.01853416)(170.06006165,46.89853811)
\curveto(170.07006156,46.78853439)(170.08506155,46.6785345)(170.10506165,46.56853811)
\curveto(170.11506152,46.51853466)(170.12006151,46.47353471)(170.12006165,46.43353811)
\lineto(170.12006165,46.32853811)
\curveto(170.14006149,46.21853496)(170.14006149,46.11353507)(170.12006165,46.01353811)
\lineto(170.12006165,45.87853811)
\curveto(170.11006152,45.82853535)(170.10506153,45.7785354)(170.10506165,45.72853811)
\curveto(170.10506153,45.6785355)(170.09506154,45.63353555)(170.07506165,45.59353811)
\curveto(170.06506157,45.55353563)(170.06006157,45.51853566)(170.06006165,45.48853811)
\curveto(170.07006156,45.46853571)(170.07006156,45.44353574)(170.06006165,45.41353811)
\lineto(170.00006165,45.17353811)
\curveto(169.99006164,45.09353609)(169.97006166,45.01853616)(169.94006165,44.94853811)
\curveto(169.81006182,44.64853653)(169.66506197,44.40353678)(169.50506165,44.21353811)
\curveto(169.3350623,44.03353715)(169.10006253,43.8835373)(168.80006165,43.76353811)
\curveto(168.58006305,43.67353751)(168.31506332,43.62853755)(168.00506165,43.62853811)
\lineto(167.69006165,43.62853811)
\curveto(167.64006399,43.63853754)(167.59006404,43.64353754)(167.54006165,43.64353811)
\lineto(167.36006165,43.67353811)
\lineto(167.03006165,43.79353811)
\curveto(166.92006471,43.83353735)(166.82006481,43.8835373)(166.73006165,43.94353811)
\curveto(166.44006519,44.12353706)(166.22506541,44.36853681)(166.08506165,44.67853811)
\curveto(165.94506569,44.98853619)(165.82006581,45.32853585)(165.71006165,45.69853811)
\curveto(165.67006596,45.83853534)(165.64006599,45.9835352)(165.62006165,46.13353811)
\curveto(165.60006603,46.2835349)(165.57506606,46.43353475)(165.54506165,46.58353811)
\curveto(165.52506611,46.65353453)(165.51506612,46.71853446)(165.51506165,46.77853811)
\curveto(165.51506612,46.84853433)(165.50506613,46.92353426)(165.48506165,47.00353811)
\curveto(165.46506617,47.07353411)(165.45506618,47.14353404)(165.45506165,47.21353811)
\curveto(165.44506619,47.2835339)(165.4300662,47.35853382)(165.41006165,47.43853811)
\curveto(165.35006628,47.68853349)(165.30006633,47.92353326)(165.26006165,48.14353811)
\curveto(165.21006642,48.36353282)(165.09506654,48.53853264)(164.91506165,48.66853811)
\curveto(164.8350668,48.72853245)(164.7350669,48.7785324)(164.61506165,48.81853811)
\curveto(164.48506715,48.85853232)(164.34506729,48.85853232)(164.19506165,48.81853811)
\curveto(163.95506768,48.75853242)(163.76506787,48.66853251)(163.62506165,48.54853811)
\curveto(163.48506815,48.43853274)(163.37506826,48.2785329)(163.29506165,48.06853811)
\curveto(163.24506839,47.94853323)(163.21006842,47.80353338)(163.19006165,47.63353811)
\curveto(163.17006846,47.47353371)(163.16006847,47.30353388)(163.16006165,47.12353811)
\curveto(163.16006847,46.94353424)(163.17006846,46.76853441)(163.19006165,46.59853811)
\curveto(163.21006842,46.42853475)(163.24006839,46.2835349)(163.28006165,46.16353811)
\curveto(163.34006829,45.99353519)(163.42506821,45.82853535)(163.53506165,45.66853811)
\curveto(163.59506804,45.58853559)(163.67506796,45.51353567)(163.77506165,45.44353811)
\curveto(163.86506777,45.3835358)(163.96506767,45.32853585)(164.07506165,45.27853811)
\curveto(164.15506748,45.24853593)(164.24006739,45.21853596)(164.33006165,45.18853811)
\curveto(164.42006721,45.16853601)(164.49006714,45.12353606)(164.54006165,45.05353811)
\curveto(164.57006706,45.01353617)(164.59506704,44.94353624)(164.61506165,44.84353811)
\curveto(164.62506701,44.75353643)(164.630067,44.65853652)(164.63006165,44.55853811)
\curveto(164.630067,44.45853672)(164.62506701,44.35853682)(164.61506165,44.25853811)
\curveto(164.59506704,44.16853701)(164.57006706,44.10353708)(164.54006165,44.06353811)
\curveto(164.51006712,44.02353716)(164.46006717,43.99353719)(164.39006165,43.97353811)
\curveto(164.32006731,43.95353723)(164.24506739,43.95353723)(164.16506165,43.97353811)
\curveto(164.0350676,44.00353718)(163.91506772,44.03353715)(163.80506165,44.06353811)
\curveto(163.68506795,44.10353708)(163.57006806,44.14853703)(163.46006165,44.19853811)
\curveto(163.11006852,44.38853679)(162.84006879,44.62853655)(162.65006165,44.91853811)
\curveto(162.45006918,45.20853597)(162.29006934,45.56853561)(162.17006165,45.99853811)
\curveto(162.15006948,46.09853508)(162.1350695,46.19853498)(162.12506165,46.29853811)
\curveto(162.11506952,46.40853477)(162.10006953,46.51853466)(162.08006165,46.62853811)
\curveto(162.07006956,46.66853451)(162.07006956,46.73353445)(162.08006165,46.82353811)
\curveto(162.08006955,46.91353427)(162.07006956,46.96853421)(162.05006165,46.98853811)
\curveto(162.04006959,47.68853349)(162.12006951,48.29853288)(162.29006165,48.81853811)
\curveto(162.46006917,49.33853184)(162.78506885,49.70353148)(163.26506165,49.91353811)
\curveto(163.46506817,50.00353118)(163.70006793,50.05353113)(163.97006165,50.06353811)
\curveto(164.2300674,50.0835311)(164.50506713,50.09353109)(164.79506165,50.09353811)
\lineto(168.11006165,50.09353811)
\curveto(168.25006338,50.09353109)(168.38506325,50.09853108)(168.51506165,50.10853811)
\curveto(168.64506299,50.11853106)(168.75006288,50.14853103)(168.83006165,50.19853811)
\curveto(168.90006273,50.24853093)(168.95006268,50.31353087)(168.98006165,50.39353811)
\curveto(169.02006261,50.4835307)(169.05006258,50.56853061)(169.07006165,50.64853811)
\curveto(169.08006255,50.72853045)(169.12506251,50.78853039)(169.20506165,50.82853811)
\curveto(169.2350624,50.84853033)(169.26506237,50.85853032)(169.29506165,50.85853811)
\curveto(169.32506231,50.85853032)(169.36506227,50.86353032)(169.41506165,50.87353811)
\moveto(167.75006165,48.72853811)
\curveto(167.61006402,48.78853239)(167.45006418,48.81853236)(167.27006165,48.81853811)
\curveto(167.08006455,48.82853235)(166.88506475,48.83353235)(166.68506165,48.83353811)
\curveto(166.57506506,48.83353235)(166.47506516,48.82853235)(166.38506165,48.81853811)
\curveto(166.29506534,48.80853237)(166.22506541,48.76853241)(166.17506165,48.69853811)
\curveto(166.15506548,48.66853251)(166.14506549,48.59853258)(166.14506165,48.48853811)
\curveto(166.16506547,48.46853271)(166.17506546,48.43353275)(166.17506165,48.38353811)
\curveto(166.17506546,48.33353285)(166.18506545,48.28853289)(166.20506165,48.24853811)
\curveto(166.22506541,48.16853301)(166.24506539,48.0785331)(166.26506165,47.97853811)
\lineto(166.32506165,47.67853811)
\curveto(166.32506531,47.64853353)(166.3300653,47.61353357)(166.34006165,47.57353811)
\lineto(166.34006165,47.46853811)
\curveto(166.38006525,47.31853386)(166.40506523,47.15353403)(166.41506165,46.97353811)
\curveto(166.41506522,46.80353438)(166.4350652,46.64353454)(166.47506165,46.49353811)
\curveto(166.49506514,46.41353477)(166.51506512,46.33853484)(166.53506165,46.26853811)
\curveto(166.54506509,46.20853497)(166.56006507,46.13853504)(166.58006165,46.05853811)
\curveto(166.630065,45.89853528)(166.69506494,45.74853543)(166.77506165,45.60853811)
\curveto(166.84506479,45.46853571)(166.9350647,45.34853583)(167.04506165,45.24853811)
\curveto(167.15506448,45.14853603)(167.29006434,45.07353611)(167.45006165,45.02353811)
\curveto(167.60006403,44.97353621)(167.78506385,44.95353623)(168.00506165,44.96353811)
\curveto(168.10506353,44.96353622)(168.20006343,44.9785362)(168.29006165,45.00853811)
\curveto(168.37006326,45.04853613)(168.44506319,45.09353609)(168.51506165,45.14353811)
\curveto(168.62506301,45.22353596)(168.72006291,45.32853585)(168.80006165,45.45853811)
\curveto(168.87006276,45.58853559)(168.9300627,45.72853545)(168.98006165,45.87853811)
\curveto(168.99006264,45.92853525)(168.99506264,45.9785352)(168.99506165,46.02853811)
\curveto(168.99506264,46.0785351)(169.00006263,46.12853505)(169.01006165,46.17853811)
\curveto(169.0300626,46.24853493)(169.04506259,46.33353485)(169.05506165,46.43353811)
\curveto(169.05506258,46.54353464)(169.04506259,46.63353455)(169.02506165,46.70353811)
\curveto(169.00506263,46.76353442)(169.00006263,46.82353436)(169.01006165,46.88353811)
\curveto(169.01006262,46.94353424)(169.00006263,47.00353418)(168.98006165,47.06353811)
\curveto(168.96006267,47.14353404)(168.94506269,47.21853396)(168.93506165,47.28853811)
\curveto(168.92506271,47.36853381)(168.90506273,47.44353374)(168.87506165,47.51353811)
\curveto(168.75506288,47.80353338)(168.61006302,48.04853313)(168.44006165,48.24853811)
\curveto(168.27006336,48.45853272)(168.04006359,48.61853256)(167.75006165,48.72853811)
}
}
{
\newrgbcolor{curcolor}{0 0 0}
\pscustom[linestyle=none,fillstyle=solid,fillcolor=curcolor]
{
\newpath
\moveto(162.06506165,55.69017873)
\curveto(162.06506957,55.92017394)(162.12506951,56.05017381)(162.24506165,56.08017873)
\curveto(162.35506928,56.11017375)(162.52006911,56.12517374)(162.74006165,56.12517873)
\lineto(163.02506165,56.12517873)
\curveto(163.11506852,56.12517374)(163.19006844,56.10017376)(163.25006165,56.05017873)
\curveto(163.3300683,55.99017387)(163.37506826,55.90517396)(163.38506165,55.79517873)
\curveto(163.38506825,55.68517418)(163.40006823,55.57517429)(163.43006165,55.46517873)
\curveto(163.46006817,55.32517454)(163.49006814,55.19017467)(163.52006165,55.06017873)
\curveto(163.55006808,54.94017492)(163.59006804,54.82517504)(163.64006165,54.71517873)
\curveto(163.77006786,54.42517544)(163.95006768,54.19017567)(164.18006165,54.01017873)
\curveto(164.40006723,53.83017603)(164.65506698,53.67517619)(164.94506165,53.54517873)
\curveto(165.05506658,53.50517636)(165.17006646,53.47517639)(165.29006165,53.45517873)
\curveto(165.40006623,53.43517643)(165.51506612,53.41017645)(165.63506165,53.38017873)
\curveto(165.68506595,53.37017649)(165.7350659,53.3651765)(165.78506165,53.36517873)
\curveto(165.8350658,53.37517649)(165.88506575,53.37517649)(165.93506165,53.36517873)
\curveto(166.05506558,53.33517653)(166.19506544,53.32017654)(166.35506165,53.32017873)
\curveto(166.50506513,53.33017653)(166.65006498,53.33517653)(166.79006165,53.33517873)
\lineto(168.63506165,53.33517873)
\lineto(168.98006165,53.33517873)
\curveto(169.10006253,53.33517653)(169.21506242,53.33017653)(169.32506165,53.32017873)
\curveto(169.4350622,53.31017655)(169.5300621,53.30517656)(169.61006165,53.30517873)
\curveto(169.69006194,53.31517655)(169.76006187,53.29517657)(169.82006165,53.24517873)
\curveto(169.89006174,53.19517667)(169.9300617,53.11517675)(169.94006165,53.00517873)
\curveto(169.95006168,52.90517696)(169.95506168,52.79517707)(169.95506165,52.67517873)
\lineto(169.95506165,52.40517873)
\curveto(169.9350617,52.35517751)(169.92006171,52.30517756)(169.91006165,52.25517873)
\curveto(169.89006174,52.21517765)(169.86506177,52.18517768)(169.83506165,52.16517873)
\curveto(169.76506187,52.11517775)(169.68006195,52.08517778)(169.58006165,52.07517873)
\lineto(169.25006165,52.07517873)
\lineto(168.09506165,52.07517873)
\lineto(163.94006165,52.07517873)
\lineto(162.90506165,52.07517873)
\lineto(162.60506165,52.07517873)
\curveto(162.50506913,52.08517778)(162.42006921,52.11517775)(162.35006165,52.16517873)
\curveto(162.31006932,52.19517767)(162.28006935,52.24517762)(162.26006165,52.31517873)
\curveto(162.24006939,52.39517747)(162.2300694,52.48017738)(162.23006165,52.57017873)
\curveto(162.22006941,52.6601772)(162.22006941,52.75017711)(162.23006165,52.84017873)
\curveto(162.24006939,52.93017693)(162.25506938,53.00017686)(162.27506165,53.05017873)
\curveto(162.30506933,53.13017673)(162.36506927,53.18017668)(162.45506165,53.20017873)
\curveto(162.5350691,53.23017663)(162.62506901,53.24517662)(162.72506165,53.24517873)
\lineto(163.02506165,53.24517873)
\curveto(163.12506851,53.24517662)(163.21506842,53.2651766)(163.29506165,53.30517873)
\curveto(163.31506832,53.31517655)(163.3300683,53.32517654)(163.34006165,53.33517873)
\lineto(163.38506165,53.38017873)
\curveto(163.38506825,53.49017637)(163.34006829,53.58017628)(163.25006165,53.65017873)
\curveto(163.15006848,53.72017614)(163.07006856,53.78017608)(163.01006165,53.83017873)
\lineto(162.92006165,53.92017873)
\curveto(162.81006882,54.01017585)(162.69506894,54.13517573)(162.57506165,54.29517873)
\curveto(162.45506918,54.45517541)(162.36506927,54.60517526)(162.30506165,54.74517873)
\curveto(162.25506938,54.83517503)(162.22006941,54.93017493)(162.20006165,55.03017873)
\curveto(162.17006946,55.13017473)(162.14006949,55.23517463)(162.11006165,55.34517873)
\curveto(162.10006953,55.40517446)(162.09506954,55.4651744)(162.09506165,55.52517873)
\curveto(162.08506955,55.58517428)(162.07506956,55.64017422)(162.06506165,55.69017873)
}
}
{
\newrgbcolor{curcolor}{0 0 0}
\pscustom[linestyle=none,fillstyle=solid,fillcolor=curcolor]
{
}
}
{
\newrgbcolor{curcolor}{0 0 0}
\pscustom[linestyle=none,fillstyle=solid,fillcolor=curcolor]
{
\newpath
\moveto(159.36506165,65.05510061)
\curveto(159.36507227,65.15509575)(159.37507226,65.25009566)(159.39506165,65.34010061)
\curveto(159.40507223,65.43009548)(159.4350722,65.49509541)(159.48506165,65.53510061)
\curveto(159.56507207,65.59509531)(159.67007196,65.62509528)(159.80006165,65.62510061)
\lineto(160.19006165,65.62510061)
\lineto(161.69006165,65.62510061)
\lineto(168.08006165,65.62510061)
\lineto(169.25006165,65.62510061)
\lineto(169.56506165,65.62510061)
\curveto(169.66506197,65.63509527)(169.74506189,65.62009529)(169.80506165,65.58010061)
\curveto(169.88506175,65.53009538)(169.9350617,65.45509545)(169.95506165,65.35510061)
\curveto(169.96506167,65.26509564)(169.97006166,65.15509575)(169.97006165,65.02510061)
\lineto(169.97006165,64.80010061)
\curveto(169.95006168,64.72009619)(169.9350617,64.65009626)(169.92506165,64.59010061)
\curveto(169.90506173,64.53009638)(169.86506177,64.48009643)(169.80506165,64.44010061)
\curveto(169.74506189,64.40009651)(169.67006196,64.38009653)(169.58006165,64.38010061)
\lineto(169.28006165,64.38010061)
\lineto(168.18506165,64.38010061)
\lineto(162.84506165,64.38010061)
\curveto(162.75506888,64.36009655)(162.68006895,64.34509656)(162.62006165,64.33510061)
\curveto(162.55006908,64.33509657)(162.49006914,64.3050966)(162.44006165,64.24510061)
\curveto(162.39006924,64.17509673)(162.36506927,64.08509682)(162.36506165,63.97510061)
\curveto(162.35506928,63.87509703)(162.35006928,63.76509714)(162.35006165,63.64510061)
\lineto(162.35006165,62.50510061)
\lineto(162.35006165,62.01010061)
\curveto(162.34006929,61.85009906)(162.28006935,61.74009917)(162.17006165,61.68010061)
\curveto(162.14006949,61.66009925)(162.11006952,61.65009926)(162.08006165,61.65010061)
\curveto(162.04006959,61.65009926)(161.99506964,61.64509926)(161.94506165,61.63510061)
\curveto(161.82506981,61.61509929)(161.71506992,61.62009929)(161.61506165,61.65010061)
\curveto(161.51507012,61.69009922)(161.44507019,61.74509916)(161.40506165,61.81510061)
\curveto(161.35507028,61.89509901)(161.3300703,62.01509889)(161.33006165,62.17510061)
\curveto(161.3300703,62.33509857)(161.31507032,62.47009844)(161.28506165,62.58010061)
\curveto(161.27507036,62.63009828)(161.27007036,62.68509822)(161.27006165,62.74510061)
\curveto(161.26007037,62.8050981)(161.24507039,62.86509804)(161.22506165,62.92510061)
\curveto(161.17507046,63.07509783)(161.12507051,63.22009769)(161.07506165,63.36010061)
\curveto(161.01507062,63.50009741)(160.94507069,63.63509727)(160.86506165,63.76510061)
\curveto(160.77507086,63.905097)(160.67007096,64.02509688)(160.55006165,64.12510061)
\curveto(160.4300712,64.22509668)(160.30007133,64.32009659)(160.16006165,64.41010061)
\curveto(160.06007157,64.47009644)(159.95007168,64.51509639)(159.83006165,64.54510061)
\curveto(159.71007192,64.58509632)(159.60507203,64.63509627)(159.51506165,64.69510061)
\curveto(159.45507218,64.74509616)(159.41507222,64.81509609)(159.39506165,64.90510061)
\curveto(159.38507225,64.92509598)(159.38007225,64.95009596)(159.38006165,64.98010061)
\curveto(159.38007225,65.0100959)(159.37507226,65.03509587)(159.36506165,65.05510061)
}
}
{
\newrgbcolor{curcolor}{0 0 0}
\pscustom[linestyle=none,fillstyle=solid,fillcolor=curcolor]
{
\newpath
\moveto(164.88506165,76.34470998)
\lineto(165.14006165,76.34470998)
\curveto(165.22006641,76.35470228)(165.29506634,76.34970228)(165.36506165,76.32970998)
\lineto(165.60506165,76.32970998)
\lineto(165.77006165,76.32970998)
\curveto(165.87006576,76.30970232)(165.97506566,76.29970233)(166.08506165,76.29970998)
\curveto(166.18506545,76.29970233)(166.28506535,76.28970234)(166.38506165,76.26970998)
\lineto(166.53506165,76.26970998)
\curveto(166.67506496,76.23970239)(166.81506482,76.21970241)(166.95506165,76.20970998)
\curveto(167.08506455,76.19970243)(167.21506442,76.17470246)(167.34506165,76.13470998)
\curveto(167.42506421,76.11470252)(167.51006412,76.09470254)(167.60006165,76.07470998)
\lineto(167.84006165,76.01470998)
\lineto(168.14006165,75.89470998)
\curveto(168.2300634,75.86470277)(168.32006331,75.8297028)(168.41006165,75.78970998)
\curveto(168.630063,75.68970294)(168.84506279,75.55470308)(169.05506165,75.38470998)
\curveto(169.26506237,75.22470341)(169.4350622,75.04970358)(169.56506165,74.85970998)
\curveto(169.60506203,74.80970382)(169.64506199,74.74970388)(169.68506165,74.67970998)
\curveto(169.71506192,74.61970401)(169.75006188,74.55970407)(169.79006165,74.49970998)
\curveto(169.84006179,74.41970421)(169.88006175,74.32470431)(169.91006165,74.21470998)
\curveto(169.94006169,74.10470453)(169.97006166,73.99970463)(170.00006165,73.89970998)
\curveto(170.04006159,73.78970484)(170.06506157,73.67970495)(170.07506165,73.56970998)
\curveto(170.08506155,73.45970517)(170.10006153,73.34470529)(170.12006165,73.22470998)
\curveto(170.1300615,73.18470545)(170.1300615,73.13970549)(170.12006165,73.08970998)
\curveto(170.12006151,73.04970558)(170.12506151,73.00970562)(170.13506165,72.96970998)
\curveto(170.14506149,72.9297057)(170.15006148,72.87470576)(170.15006165,72.80470998)
\curveto(170.15006148,72.7347059)(170.14506149,72.68470595)(170.13506165,72.65470998)
\curveto(170.11506152,72.60470603)(170.11006152,72.55970607)(170.12006165,72.51970998)
\curveto(170.1300615,72.47970615)(170.1300615,72.44470619)(170.12006165,72.41470998)
\lineto(170.12006165,72.32470998)
\curveto(170.10006153,72.26470637)(170.08506155,72.19970643)(170.07506165,72.12970998)
\curveto(170.07506156,72.06970656)(170.07006156,72.00470663)(170.06006165,71.93470998)
\curveto(170.01006162,71.76470687)(169.96006167,71.60470703)(169.91006165,71.45470998)
\curveto(169.86006177,71.30470733)(169.79506184,71.15970747)(169.71506165,71.01970998)
\curveto(169.67506196,70.96970766)(169.64506199,70.91470772)(169.62506165,70.85470998)
\curveto(169.59506204,70.80470783)(169.56006207,70.75470788)(169.52006165,70.70470998)
\curveto(169.34006229,70.46470817)(169.12006251,70.26470837)(168.86006165,70.10470998)
\curveto(168.60006303,69.94470869)(168.31506332,69.80470883)(168.00506165,69.68470998)
\curveto(167.86506377,69.62470901)(167.72506391,69.57970905)(167.58506165,69.54970998)
\curveto(167.4350642,69.51970911)(167.28006435,69.48470915)(167.12006165,69.44470998)
\curveto(167.01006462,69.42470921)(166.90006473,69.40970922)(166.79006165,69.39970998)
\curveto(166.68006495,69.38970924)(166.57006506,69.37470926)(166.46006165,69.35470998)
\curveto(166.42006521,69.34470929)(166.38006525,69.33970929)(166.34006165,69.33970998)
\curveto(166.30006533,69.34970928)(166.26006537,69.34970928)(166.22006165,69.33970998)
\curveto(166.17006546,69.3297093)(166.12006551,69.32470931)(166.07006165,69.32470998)
\lineto(165.90506165,69.32470998)
\curveto(165.85506578,69.30470933)(165.80506583,69.29970933)(165.75506165,69.30970998)
\curveto(165.69506594,69.31970931)(165.64006599,69.31970931)(165.59006165,69.30970998)
\curveto(165.55006608,69.29970933)(165.50506613,69.29970933)(165.45506165,69.30970998)
\curveto(165.40506623,69.31970931)(165.35506628,69.31470932)(165.30506165,69.29470998)
\curveto(165.2350664,69.27470936)(165.16006647,69.26970936)(165.08006165,69.27970998)
\curveto(164.99006664,69.28970934)(164.90506673,69.29470934)(164.82506165,69.29470998)
\curveto(164.7350669,69.29470934)(164.635067,69.28970934)(164.52506165,69.27970998)
\curveto(164.40506723,69.26970936)(164.30506733,69.27470936)(164.22506165,69.29470998)
\lineto(163.94006165,69.29470998)
\lineto(163.31006165,69.33970998)
\curveto(163.21006842,69.34970928)(163.11506852,69.35970927)(163.02506165,69.36970998)
\lineto(162.72506165,69.39970998)
\curveto(162.67506896,69.41970921)(162.62506901,69.42470921)(162.57506165,69.41470998)
\curveto(162.51506912,69.41470922)(162.46006917,69.42470921)(162.41006165,69.44470998)
\curveto(162.24006939,69.49470914)(162.07506956,69.5347091)(161.91506165,69.56470998)
\curveto(161.74506989,69.59470904)(161.58507005,69.64470899)(161.43506165,69.71470998)
\curveto(160.97507066,69.90470873)(160.60007103,70.12470851)(160.31006165,70.37470998)
\curveto(160.02007161,70.634708)(159.77507186,70.99470764)(159.57506165,71.45470998)
\curveto(159.52507211,71.58470705)(159.49007214,71.71470692)(159.47006165,71.84470998)
\curveto(159.45007218,71.98470665)(159.42507221,72.12470651)(159.39506165,72.26470998)
\curveto(159.38507225,72.3347063)(159.38007225,72.39970623)(159.38006165,72.45970998)
\curveto(159.38007225,72.51970611)(159.37507226,72.58470605)(159.36506165,72.65470998)
\curveto(159.34507229,73.48470515)(159.49507214,74.15470448)(159.81506165,74.66470998)
\curveto(160.12507151,75.17470346)(160.56507107,75.55470308)(161.13506165,75.80470998)
\curveto(161.25507038,75.85470278)(161.38007025,75.89970273)(161.51006165,75.93970998)
\curveto(161.64006999,75.97970265)(161.77506986,76.02470261)(161.91506165,76.07470998)
\curveto(161.99506964,76.09470254)(162.08006955,76.10970252)(162.17006165,76.11970998)
\lineto(162.41006165,76.17970998)
\curveto(162.52006911,76.20970242)(162.630069,76.22470241)(162.74006165,76.22470998)
\curveto(162.85006878,76.2347024)(162.96006867,76.24970238)(163.07006165,76.26970998)
\curveto(163.12006851,76.28970234)(163.16506847,76.29470234)(163.20506165,76.28470998)
\curveto(163.24506839,76.28470235)(163.28506835,76.28970234)(163.32506165,76.29970998)
\curveto(163.37506826,76.30970232)(163.4300682,76.30970232)(163.49006165,76.29970998)
\curveto(163.54006809,76.29970233)(163.59006804,76.30470233)(163.64006165,76.31470998)
\lineto(163.77506165,76.31470998)
\curveto(163.8350678,76.3347023)(163.90506773,76.3347023)(163.98506165,76.31470998)
\curveto(164.05506758,76.30470233)(164.12006751,76.30970232)(164.18006165,76.32970998)
\curveto(164.21006742,76.33970229)(164.25006738,76.34470229)(164.30006165,76.34470998)
\lineto(164.42006165,76.34470998)
\lineto(164.88506165,76.34470998)
\moveto(167.21006165,74.79970998)
\curveto(166.89006474,74.89970373)(166.52506511,74.95970367)(166.11506165,74.97970998)
\curveto(165.70506593,74.99970363)(165.29506634,75.00970362)(164.88506165,75.00970998)
\curveto(164.45506718,75.00970362)(164.0350676,74.99970363)(163.62506165,74.97970998)
\curveto(163.21506842,74.95970367)(162.8300688,74.91470372)(162.47006165,74.84470998)
\curveto(162.11006952,74.77470386)(161.79006984,74.66470397)(161.51006165,74.51470998)
\curveto(161.22007041,74.37470426)(160.98507065,74.17970445)(160.80506165,73.92970998)
\curveto(160.69507094,73.76970486)(160.61507102,73.58970504)(160.56506165,73.38970998)
\curveto(160.50507113,73.18970544)(160.47507116,72.94470569)(160.47506165,72.65470998)
\curveto(160.49507114,72.634706)(160.50507113,72.59970603)(160.50506165,72.54970998)
\curveto(160.49507114,72.49970613)(160.49507114,72.45970617)(160.50506165,72.42970998)
\curveto(160.52507111,72.34970628)(160.54507109,72.27470636)(160.56506165,72.20470998)
\curveto(160.57507106,72.14470649)(160.59507104,72.07970655)(160.62506165,72.00970998)
\curveto(160.74507089,71.73970689)(160.91507072,71.51970711)(161.13506165,71.34970998)
\curveto(161.34507029,71.18970744)(161.59007004,71.05470758)(161.87006165,70.94470998)
\curveto(161.98006965,70.89470774)(162.10006953,70.85470778)(162.23006165,70.82470998)
\curveto(162.35006928,70.80470783)(162.47506916,70.77970785)(162.60506165,70.74970998)
\curveto(162.65506898,70.7297079)(162.71006892,70.71970791)(162.77006165,70.71970998)
\curveto(162.82006881,70.71970791)(162.87006876,70.71470792)(162.92006165,70.70470998)
\curveto(163.01006862,70.69470794)(163.10506853,70.68470795)(163.20506165,70.67470998)
\curveto(163.29506834,70.66470797)(163.39006824,70.65470798)(163.49006165,70.64470998)
\curveto(163.57006806,70.64470799)(163.65506798,70.63970799)(163.74506165,70.62970998)
\lineto(163.98506165,70.62970998)
\lineto(164.16506165,70.62970998)
\curveto(164.19506744,70.61970801)(164.2300674,70.61470802)(164.27006165,70.61470998)
\lineto(164.40506165,70.61470998)
\lineto(164.85506165,70.61470998)
\curveto(164.9350667,70.61470802)(165.02006661,70.60970802)(165.11006165,70.59970998)
\curveto(165.19006644,70.59970803)(165.26506637,70.60970802)(165.33506165,70.62970998)
\lineto(165.60506165,70.62970998)
\curveto(165.62506601,70.629708)(165.65506598,70.62470801)(165.69506165,70.61470998)
\curveto(165.72506591,70.61470802)(165.75006588,70.61970801)(165.77006165,70.62970998)
\curveto(165.87006576,70.63970799)(165.97006566,70.64470799)(166.07006165,70.64470998)
\curveto(166.16006547,70.65470798)(166.26006537,70.66470797)(166.37006165,70.67470998)
\curveto(166.49006514,70.70470793)(166.61506502,70.71970791)(166.74506165,70.71970998)
\curveto(166.86506477,70.7297079)(166.98006465,70.75470788)(167.09006165,70.79470998)
\curveto(167.39006424,70.87470776)(167.65506398,70.95970767)(167.88506165,71.04970998)
\curveto(168.11506352,71.14970748)(168.3300633,71.29470734)(168.53006165,71.48470998)
\curveto(168.7300629,71.69470694)(168.88006275,71.95970667)(168.98006165,72.27970998)
\curveto(169.00006263,72.31970631)(169.01006262,72.35470628)(169.01006165,72.38470998)
\curveto(169.00006263,72.42470621)(169.00506263,72.46970616)(169.02506165,72.51970998)
\curveto(169.0350626,72.55970607)(169.04506259,72.629706)(169.05506165,72.72970998)
\curveto(169.06506257,72.83970579)(169.06006257,72.92470571)(169.04006165,72.98470998)
\curveto(169.02006261,73.05470558)(169.01006262,73.12470551)(169.01006165,73.19470998)
\curveto(169.00006263,73.26470537)(168.98506265,73.3297053)(168.96506165,73.38970998)
\curveto(168.90506273,73.58970504)(168.82006281,73.76970486)(168.71006165,73.92970998)
\curveto(168.69006294,73.95970467)(168.67006296,73.98470465)(168.65006165,74.00470998)
\lineto(168.59006165,74.06470998)
\curveto(168.57006306,74.10470453)(168.5300631,74.15470448)(168.47006165,74.21470998)
\curveto(168.3300633,74.31470432)(168.20006343,74.39970423)(168.08006165,74.46970998)
\curveto(167.96006367,74.53970409)(167.81506382,74.60970402)(167.64506165,74.67970998)
\curveto(167.57506406,74.70970392)(167.50506413,74.7297039)(167.43506165,74.73970998)
\curveto(167.36506427,74.75970387)(167.29006434,74.77970385)(167.21006165,74.79970998)
}
}
{
\newrgbcolor{curcolor}{0 0 0}
\pscustom[linestyle=none,fillstyle=solid,fillcolor=curcolor]
{
\newpath
\moveto(168.33506165,78.63431936)
\lineto(168.33506165,79.26431936)
\lineto(168.33506165,79.45931936)
\curveto(168.3350633,79.52931683)(168.34506329,79.58931677)(168.36506165,79.63931936)
\curveto(168.40506323,79.70931665)(168.44506319,79.7593166)(168.48506165,79.78931936)
\curveto(168.5350631,79.82931653)(168.60006303,79.84931651)(168.68006165,79.84931936)
\curveto(168.76006287,79.8593165)(168.84506279,79.86431649)(168.93506165,79.86431936)
\lineto(169.65506165,79.86431936)
\curveto(170.1350615,79.86431649)(170.54506109,79.80431655)(170.88506165,79.68431936)
\curveto(171.22506041,79.56431679)(171.50006013,79.36931699)(171.71006165,79.09931936)
\curveto(171.76005987,79.02931733)(171.80505983,78.9593174)(171.84506165,78.88931936)
\curveto(171.89505974,78.82931753)(171.94005969,78.7543176)(171.98006165,78.66431936)
\curveto(171.99005964,78.64431771)(172.00005963,78.61431774)(172.01006165,78.57431936)
\curveto(172.0300596,78.53431782)(172.0350596,78.48931787)(172.02506165,78.43931936)
\curveto(171.99505964,78.34931801)(171.92005971,78.29431806)(171.80006165,78.27431936)
\curveto(171.69005994,78.2543181)(171.59506004,78.26931809)(171.51506165,78.31931936)
\curveto(171.44506019,78.34931801)(171.38006025,78.39431796)(171.32006165,78.45431936)
\curveto(171.27006036,78.52431783)(171.22006041,78.58931777)(171.17006165,78.64931936)
\curveto(171.12006051,78.71931764)(171.04506059,78.77931758)(170.94506165,78.82931936)
\curveto(170.85506078,78.88931747)(170.76506087,78.93931742)(170.67506165,78.97931936)
\curveto(170.64506099,78.99931736)(170.58506105,79.02431733)(170.49506165,79.05431936)
\curveto(170.41506122,79.08431727)(170.34506129,79.08931727)(170.28506165,79.06931936)
\curveto(170.14506149,79.03931732)(170.05506158,78.97931738)(170.01506165,78.88931936)
\curveto(169.98506165,78.80931755)(169.97006166,78.71931764)(169.97006165,78.61931936)
\curveto(169.97006166,78.51931784)(169.94506169,78.43431792)(169.89506165,78.36431936)
\curveto(169.82506181,78.27431808)(169.68506195,78.22931813)(169.47506165,78.22931936)
\lineto(168.92006165,78.22931936)
\lineto(168.69506165,78.22931936)
\curveto(168.61506302,78.23931812)(168.55006308,78.2593181)(168.50006165,78.28931936)
\curveto(168.42006321,78.34931801)(168.37506326,78.41931794)(168.36506165,78.49931936)
\curveto(168.35506328,78.51931784)(168.35006328,78.53931782)(168.35006165,78.55931936)
\curveto(168.35006328,78.58931777)(168.34506329,78.61431774)(168.33506165,78.63431936)
}
}
{
\newrgbcolor{curcolor}{0 0 0}
\pscustom[linestyle=none,fillstyle=solid,fillcolor=curcolor]
{
}
}
{
\newrgbcolor{curcolor}{0 0 0}
\pscustom[linestyle=none,fillstyle=solid,fillcolor=curcolor]
{
\newpath
\moveto(159.36506165,89.26463186)
\curveto(159.35507228,89.95462722)(159.47507216,90.55462662)(159.72506165,91.06463186)
\curveto(159.97507166,91.58462559)(160.31007132,91.9796252)(160.73006165,92.24963186)
\curveto(160.81007082,92.29962488)(160.90007073,92.34462483)(161.00006165,92.38463186)
\curveto(161.09007054,92.42462475)(161.18507045,92.46962471)(161.28506165,92.51963186)
\curveto(161.38507025,92.55962462)(161.48507015,92.58962459)(161.58506165,92.60963186)
\curveto(161.68506995,92.62962455)(161.79006984,92.64962453)(161.90006165,92.66963186)
\curveto(161.95006968,92.68962449)(161.99506964,92.69462448)(162.03506165,92.68463186)
\curveto(162.07506956,92.6746245)(162.12006951,92.6796245)(162.17006165,92.69963186)
\curveto(162.22006941,92.70962447)(162.30506933,92.71462446)(162.42506165,92.71463186)
\curveto(162.5350691,92.71462446)(162.62006901,92.70962447)(162.68006165,92.69963186)
\curveto(162.74006889,92.6796245)(162.80006883,92.66962451)(162.86006165,92.66963186)
\curveto(162.92006871,92.6796245)(162.98006865,92.6746245)(163.04006165,92.65463186)
\curveto(163.18006845,92.61462456)(163.31506832,92.5796246)(163.44506165,92.54963186)
\curveto(163.57506806,92.51962466)(163.70006793,92.4796247)(163.82006165,92.42963186)
\curveto(163.96006767,92.36962481)(164.08506755,92.29962488)(164.19506165,92.21963186)
\curveto(164.30506733,92.14962503)(164.41506722,92.0746251)(164.52506165,91.99463186)
\lineto(164.58506165,91.93463186)
\curveto(164.60506703,91.92462525)(164.62506701,91.90962527)(164.64506165,91.88963186)
\curveto(164.80506683,91.76962541)(164.95006668,91.63462554)(165.08006165,91.48463186)
\curveto(165.21006642,91.33462584)(165.3350663,91.174626)(165.45506165,91.00463186)
\curveto(165.67506596,90.69462648)(165.88006575,90.39962678)(166.07006165,90.11963186)
\curveto(166.21006542,89.88962729)(166.34506529,89.65962752)(166.47506165,89.42963186)
\curveto(166.60506503,89.20962797)(166.74006489,88.98962819)(166.88006165,88.76963186)
\curveto(167.05006458,88.51962866)(167.2300644,88.2796289)(167.42006165,88.04963186)
\curveto(167.61006402,87.82962935)(167.8350638,87.63962954)(168.09506165,87.47963186)
\curveto(168.15506348,87.43962974)(168.21506342,87.40462977)(168.27506165,87.37463186)
\curveto(168.32506331,87.34462983)(168.39006324,87.31462986)(168.47006165,87.28463186)
\curveto(168.54006309,87.26462991)(168.60006303,87.25962992)(168.65006165,87.26963186)
\curveto(168.72006291,87.28962989)(168.77506286,87.32462985)(168.81506165,87.37463186)
\curveto(168.84506279,87.42462975)(168.86506277,87.48462969)(168.87506165,87.55463186)
\lineto(168.87506165,87.79463186)
\lineto(168.87506165,88.54463186)
\lineto(168.87506165,91.34963186)
\lineto(168.87506165,92.00963186)
\curveto(168.87506276,92.09962508)(168.88006275,92.18462499)(168.89006165,92.26463186)
\curveto(168.89006274,92.34462483)(168.91006272,92.40962477)(168.95006165,92.45963186)
\curveto(168.99006264,92.50962467)(169.06506257,92.54962463)(169.17506165,92.57963186)
\curveto(169.27506236,92.61962456)(169.37506226,92.62962455)(169.47506165,92.60963186)
\lineto(169.61006165,92.60963186)
\curveto(169.68006195,92.58962459)(169.74006189,92.56962461)(169.79006165,92.54963186)
\curveto(169.84006179,92.52962465)(169.88006175,92.49462468)(169.91006165,92.44463186)
\curveto(169.95006168,92.39462478)(169.97006166,92.32462485)(169.97006165,92.23463186)
\lineto(169.97006165,91.96463186)
\lineto(169.97006165,91.06463186)
\lineto(169.97006165,87.55463186)
\lineto(169.97006165,86.48963186)
\curveto(169.97006166,86.40963077)(169.97506166,86.31963086)(169.98506165,86.21963186)
\curveto(169.98506165,86.11963106)(169.97506166,86.03463114)(169.95506165,85.96463186)
\curveto(169.88506175,85.75463142)(169.70506193,85.68963149)(169.41506165,85.76963186)
\curveto(169.37506226,85.7796314)(169.34006229,85.7796314)(169.31006165,85.76963186)
\curveto(169.27006236,85.76963141)(169.22506241,85.7796314)(169.17506165,85.79963186)
\curveto(169.09506254,85.81963136)(169.01006262,85.83963134)(168.92006165,85.85963186)
\curveto(168.8300628,85.8796313)(168.74506289,85.90463127)(168.66506165,85.93463186)
\curveto(168.17506346,86.09463108)(167.76006387,86.29463088)(167.42006165,86.53463186)
\curveto(167.17006446,86.71463046)(166.94506469,86.91963026)(166.74506165,87.14963186)
\curveto(166.5350651,87.3796298)(166.34006529,87.61962956)(166.16006165,87.86963186)
\curveto(165.98006565,88.12962905)(165.81006582,88.39462878)(165.65006165,88.66463186)
\curveto(165.48006615,88.94462823)(165.30506633,89.21462796)(165.12506165,89.47463186)
\curveto(165.04506659,89.58462759)(164.97006666,89.68962749)(164.90006165,89.78963186)
\curveto(164.8300668,89.89962728)(164.75506688,90.00962717)(164.67506165,90.11963186)
\curveto(164.64506699,90.15962702)(164.61506702,90.19462698)(164.58506165,90.22463186)
\curveto(164.54506709,90.26462691)(164.51506712,90.30462687)(164.49506165,90.34463186)
\curveto(164.38506725,90.48462669)(164.26006737,90.60962657)(164.12006165,90.71963186)
\curveto(164.09006754,90.73962644)(164.06506757,90.76462641)(164.04506165,90.79463186)
\curveto(164.01506762,90.82462635)(163.98506765,90.84962633)(163.95506165,90.86963186)
\curveto(163.85506778,90.94962623)(163.75506788,91.01462616)(163.65506165,91.06463186)
\curveto(163.55506808,91.12462605)(163.44506819,91.179626)(163.32506165,91.22963186)
\curveto(163.25506838,91.25962592)(163.18006845,91.2796259)(163.10006165,91.28963186)
\lineto(162.86006165,91.34963186)
\lineto(162.77006165,91.34963186)
\curveto(162.74006889,91.35962582)(162.71006892,91.36462581)(162.68006165,91.36463186)
\curveto(162.61006902,91.38462579)(162.51506912,91.38962579)(162.39506165,91.37963186)
\curveto(162.26506937,91.3796258)(162.16506947,91.36962581)(162.09506165,91.34963186)
\curveto(162.01506962,91.32962585)(161.94006969,91.30962587)(161.87006165,91.28963186)
\curveto(161.79006984,91.2796259)(161.71006992,91.25962592)(161.63006165,91.22963186)
\curveto(161.39007024,91.11962606)(161.19007044,90.96962621)(161.03006165,90.77963186)
\curveto(160.86007077,90.59962658)(160.72007091,90.3796268)(160.61006165,90.11963186)
\curveto(160.59007104,90.04962713)(160.57507106,89.9796272)(160.56506165,89.90963186)
\curveto(160.54507109,89.83962734)(160.52507111,89.76462741)(160.50506165,89.68463186)
\curveto(160.48507115,89.60462757)(160.47507116,89.49462768)(160.47506165,89.35463186)
\curveto(160.47507116,89.22462795)(160.48507115,89.11962806)(160.50506165,89.03963186)
\curveto(160.51507112,88.9796282)(160.52007111,88.92462825)(160.52006165,88.87463186)
\curveto(160.52007111,88.82462835)(160.5300711,88.7746284)(160.55006165,88.72463186)
\curveto(160.59007104,88.62462855)(160.630071,88.52962865)(160.67006165,88.43963186)
\curveto(160.71007092,88.35962882)(160.75507088,88.2796289)(160.80506165,88.19963186)
\curveto(160.82507081,88.16962901)(160.85007078,88.13962904)(160.88006165,88.10963186)
\curveto(160.91007072,88.08962909)(160.9350707,88.06462911)(160.95506165,88.03463186)
\lineto(161.03006165,87.95963186)
\curveto(161.05007058,87.92962925)(161.07007056,87.90462927)(161.09006165,87.88463186)
\lineto(161.30006165,87.73463186)
\curveto(161.36007027,87.69462948)(161.42507021,87.64962953)(161.49506165,87.59963186)
\curveto(161.58507005,87.53962964)(161.69006994,87.48962969)(161.81006165,87.44963186)
\curveto(161.92006971,87.41962976)(162.0300696,87.38462979)(162.14006165,87.34463186)
\curveto(162.25006938,87.30462987)(162.39506924,87.2796299)(162.57506165,87.26963186)
\curveto(162.74506889,87.25962992)(162.87006876,87.22962995)(162.95006165,87.17963186)
\curveto(163.0300686,87.12963005)(163.07506856,87.05463012)(163.08506165,86.95463186)
\curveto(163.09506854,86.85463032)(163.10006853,86.74463043)(163.10006165,86.62463186)
\curveto(163.10006853,86.58463059)(163.10506853,86.54463063)(163.11506165,86.50463186)
\curveto(163.11506852,86.46463071)(163.11006852,86.42963075)(163.10006165,86.39963186)
\curveto(163.08006855,86.34963083)(163.07006856,86.29963088)(163.07006165,86.24963186)
\curveto(163.07006856,86.20963097)(163.06006857,86.16963101)(163.04006165,86.12963186)
\curveto(162.98006865,86.03963114)(162.84506879,85.99463118)(162.63506165,85.99463186)
\lineto(162.51506165,85.99463186)
\curveto(162.45506918,86.00463117)(162.39506924,86.00963117)(162.33506165,86.00963186)
\curveto(162.26506937,86.01963116)(162.20006943,86.02963115)(162.14006165,86.03963186)
\curveto(162.0300696,86.05963112)(161.9300697,86.0796311)(161.84006165,86.09963186)
\curveto(161.74006989,86.11963106)(161.64506999,86.14963103)(161.55506165,86.18963186)
\curveto(161.48507015,86.20963097)(161.42507021,86.22963095)(161.37506165,86.24963186)
\lineto(161.19506165,86.30963186)
\curveto(160.9350707,86.42963075)(160.69007094,86.58463059)(160.46006165,86.77463186)
\curveto(160.2300714,86.9746302)(160.04507159,87.18962999)(159.90506165,87.41963186)
\curveto(159.82507181,87.52962965)(159.76007187,87.64462953)(159.71006165,87.76463186)
\lineto(159.56006165,88.15463186)
\curveto(159.51007212,88.26462891)(159.48007215,88.3796288)(159.47006165,88.49963186)
\curveto(159.45007218,88.61962856)(159.42507221,88.74462843)(159.39506165,88.87463186)
\curveto(159.39507224,88.94462823)(159.39507224,89.00962817)(159.39506165,89.06963186)
\curveto(159.38507225,89.12962805)(159.37507226,89.19462798)(159.36506165,89.26463186)
}
}
{
\newrgbcolor{curcolor}{0 0 0}
\pscustom[linestyle=none,fillstyle=solid,fillcolor=curcolor]
{
\newpath
\moveto(164.88506165,101.36424123)
\lineto(165.14006165,101.36424123)
\curveto(165.22006641,101.37423353)(165.29506634,101.36923353)(165.36506165,101.34924123)
\lineto(165.60506165,101.34924123)
\lineto(165.77006165,101.34924123)
\curveto(165.87006576,101.32923357)(165.97506566,101.31923358)(166.08506165,101.31924123)
\curveto(166.18506545,101.31923358)(166.28506535,101.30923359)(166.38506165,101.28924123)
\lineto(166.53506165,101.28924123)
\curveto(166.67506496,101.25923364)(166.81506482,101.23923366)(166.95506165,101.22924123)
\curveto(167.08506455,101.21923368)(167.21506442,101.19423371)(167.34506165,101.15424123)
\curveto(167.42506421,101.13423377)(167.51006412,101.11423379)(167.60006165,101.09424123)
\lineto(167.84006165,101.03424123)
\lineto(168.14006165,100.91424123)
\curveto(168.2300634,100.88423402)(168.32006331,100.84923405)(168.41006165,100.80924123)
\curveto(168.630063,100.70923419)(168.84506279,100.57423433)(169.05506165,100.40424123)
\curveto(169.26506237,100.24423466)(169.4350622,100.06923483)(169.56506165,99.87924123)
\curveto(169.60506203,99.82923507)(169.64506199,99.76923513)(169.68506165,99.69924123)
\curveto(169.71506192,99.63923526)(169.75006188,99.57923532)(169.79006165,99.51924123)
\curveto(169.84006179,99.43923546)(169.88006175,99.34423556)(169.91006165,99.23424123)
\curveto(169.94006169,99.12423578)(169.97006166,99.01923588)(170.00006165,98.91924123)
\curveto(170.04006159,98.80923609)(170.06506157,98.6992362)(170.07506165,98.58924123)
\curveto(170.08506155,98.47923642)(170.10006153,98.36423654)(170.12006165,98.24424123)
\curveto(170.1300615,98.2042367)(170.1300615,98.15923674)(170.12006165,98.10924123)
\curveto(170.12006151,98.06923683)(170.12506151,98.02923687)(170.13506165,97.98924123)
\curveto(170.14506149,97.94923695)(170.15006148,97.89423701)(170.15006165,97.82424123)
\curveto(170.15006148,97.75423715)(170.14506149,97.7042372)(170.13506165,97.67424123)
\curveto(170.11506152,97.62423728)(170.11006152,97.57923732)(170.12006165,97.53924123)
\curveto(170.1300615,97.4992374)(170.1300615,97.46423744)(170.12006165,97.43424123)
\lineto(170.12006165,97.34424123)
\curveto(170.10006153,97.28423762)(170.08506155,97.21923768)(170.07506165,97.14924123)
\curveto(170.07506156,97.08923781)(170.07006156,97.02423788)(170.06006165,96.95424123)
\curveto(170.01006162,96.78423812)(169.96006167,96.62423828)(169.91006165,96.47424123)
\curveto(169.86006177,96.32423858)(169.79506184,96.17923872)(169.71506165,96.03924123)
\curveto(169.67506196,95.98923891)(169.64506199,95.93423897)(169.62506165,95.87424123)
\curveto(169.59506204,95.82423908)(169.56006207,95.77423913)(169.52006165,95.72424123)
\curveto(169.34006229,95.48423942)(169.12006251,95.28423962)(168.86006165,95.12424123)
\curveto(168.60006303,94.96423994)(168.31506332,94.82424008)(168.00506165,94.70424123)
\curveto(167.86506377,94.64424026)(167.72506391,94.5992403)(167.58506165,94.56924123)
\curveto(167.4350642,94.53924036)(167.28006435,94.5042404)(167.12006165,94.46424123)
\curveto(167.01006462,94.44424046)(166.90006473,94.42924047)(166.79006165,94.41924123)
\curveto(166.68006495,94.40924049)(166.57006506,94.39424051)(166.46006165,94.37424123)
\curveto(166.42006521,94.36424054)(166.38006525,94.35924054)(166.34006165,94.35924123)
\curveto(166.30006533,94.36924053)(166.26006537,94.36924053)(166.22006165,94.35924123)
\curveto(166.17006546,94.34924055)(166.12006551,94.34424056)(166.07006165,94.34424123)
\lineto(165.90506165,94.34424123)
\curveto(165.85506578,94.32424058)(165.80506583,94.31924058)(165.75506165,94.32924123)
\curveto(165.69506594,94.33924056)(165.64006599,94.33924056)(165.59006165,94.32924123)
\curveto(165.55006608,94.31924058)(165.50506613,94.31924058)(165.45506165,94.32924123)
\curveto(165.40506623,94.33924056)(165.35506628,94.33424057)(165.30506165,94.31424123)
\curveto(165.2350664,94.29424061)(165.16006647,94.28924061)(165.08006165,94.29924123)
\curveto(164.99006664,94.30924059)(164.90506673,94.31424059)(164.82506165,94.31424123)
\curveto(164.7350669,94.31424059)(164.635067,94.30924059)(164.52506165,94.29924123)
\curveto(164.40506723,94.28924061)(164.30506733,94.29424061)(164.22506165,94.31424123)
\lineto(163.94006165,94.31424123)
\lineto(163.31006165,94.35924123)
\curveto(163.21006842,94.36924053)(163.11506852,94.37924052)(163.02506165,94.38924123)
\lineto(162.72506165,94.41924123)
\curveto(162.67506896,94.43924046)(162.62506901,94.44424046)(162.57506165,94.43424123)
\curveto(162.51506912,94.43424047)(162.46006917,94.44424046)(162.41006165,94.46424123)
\curveto(162.24006939,94.51424039)(162.07506956,94.55424035)(161.91506165,94.58424123)
\curveto(161.74506989,94.61424029)(161.58507005,94.66424024)(161.43506165,94.73424123)
\curveto(160.97507066,94.92423998)(160.60007103,95.14423976)(160.31006165,95.39424123)
\curveto(160.02007161,95.65423925)(159.77507186,96.01423889)(159.57506165,96.47424123)
\curveto(159.52507211,96.6042383)(159.49007214,96.73423817)(159.47006165,96.86424123)
\curveto(159.45007218,97.0042379)(159.42507221,97.14423776)(159.39506165,97.28424123)
\curveto(159.38507225,97.35423755)(159.38007225,97.41923748)(159.38006165,97.47924123)
\curveto(159.38007225,97.53923736)(159.37507226,97.6042373)(159.36506165,97.67424123)
\curveto(159.34507229,98.5042364)(159.49507214,99.17423573)(159.81506165,99.68424123)
\curveto(160.12507151,100.19423471)(160.56507107,100.57423433)(161.13506165,100.82424123)
\curveto(161.25507038,100.87423403)(161.38007025,100.91923398)(161.51006165,100.95924123)
\curveto(161.64006999,100.9992339)(161.77506986,101.04423386)(161.91506165,101.09424123)
\curveto(161.99506964,101.11423379)(162.08006955,101.12923377)(162.17006165,101.13924123)
\lineto(162.41006165,101.19924123)
\curveto(162.52006911,101.22923367)(162.630069,101.24423366)(162.74006165,101.24424123)
\curveto(162.85006878,101.25423365)(162.96006867,101.26923363)(163.07006165,101.28924123)
\curveto(163.12006851,101.30923359)(163.16506847,101.31423359)(163.20506165,101.30424123)
\curveto(163.24506839,101.3042336)(163.28506835,101.30923359)(163.32506165,101.31924123)
\curveto(163.37506826,101.32923357)(163.4300682,101.32923357)(163.49006165,101.31924123)
\curveto(163.54006809,101.31923358)(163.59006804,101.32423358)(163.64006165,101.33424123)
\lineto(163.77506165,101.33424123)
\curveto(163.8350678,101.35423355)(163.90506773,101.35423355)(163.98506165,101.33424123)
\curveto(164.05506758,101.32423358)(164.12006751,101.32923357)(164.18006165,101.34924123)
\curveto(164.21006742,101.35923354)(164.25006738,101.36423354)(164.30006165,101.36424123)
\lineto(164.42006165,101.36424123)
\lineto(164.88506165,101.36424123)
\moveto(167.21006165,99.81924123)
\curveto(166.89006474,99.91923498)(166.52506511,99.97923492)(166.11506165,99.99924123)
\curveto(165.70506593,100.01923488)(165.29506634,100.02923487)(164.88506165,100.02924123)
\curveto(164.45506718,100.02923487)(164.0350676,100.01923488)(163.62506165,99.99924123)
\curveto(163.21506842,99.97923492)(162.8300688,99.93423497)(162.47006165,99.86424123)
\curveto(162.11006952,99.79423511)(161.79006984,99.68423522)(161.51006165,99.53424123)
\curveto(161.22007041,99.39423551)(160.98507065,99.1992357)(160.80506165,98.94924123)
\curveto(160.69507094,98.78923611)(160.61507102,98.60923629)(160.56506165,98.40924123)
\curveto(160.50507113,98.20923669)(160.47507116,97.96423694)(160.47506165,97.67424123)
\curveto(160.49507114,97.65423725)(160.50507113,97.61923728)(160.50506165,97.56924123)
\curveto(160.49507114,97.51923738)(160.49507114,97.47923742)(160.50506165,97.44924123)
\curveto(160.52507111,97.36923753)(160.54507109,97.29423761)(160.56506165,97.22424123)
\curveto(160.57507106,97.16423774)(160.59507104,97.0992378)(160.62506165,97.02924123)
\curveto(160.74507089,96.75923814)(160.91507072,96.53923836)(161.13506165,96.36924123)
\curveto(161.34507029,96.20923869)(161.59007004,96.07423883)(161.87006165,95.96424123)
\curveto(161.98006965,95.91423899)(162.10006953,95.87423903)(162.23006165,95.84424123)
\curveto(162.35006928,95.82423908)(162.47506916,95.7992391)(162.60506165,95.76924123)
\curveto(162.65506898,95.74923915)(162.71006892,95.73923916)(162.77006165,95.73924123)
\curveto(162.82006881,95.73923916)(162.87006876,95.73423917)(162.92006165,95.72424123)
\curveto(163.01006862,95.71423919)(163.10506853,95.7042392)(163.20506165,95.69424123)
\curveto(163.29506834,95.68423922)(163.39006824,95.67423923)(163.49006165,95.66424123)
\curveto(163.57006806,95.66423924)(163.65506798,95.65923924)(163.74506165,95.64924123)
\lineto(163.98506165,95.64924123)
\lineto(164.16506165,95.64924123)
\curveto(164.19506744,95.63923926)(164.2300674,95.63423927)(164.27006165,95.63424123)
\lineto(164.40506165,95.63424123)
\lineto(164.85506165,95.63424123)
\curveto(164.9350667,95.63423927)(165.02006661,95.62923927)(165.11006165,95.61924123)
\curveto(165.19006644,95.61923928)(165.26506637,95.62923927)(165.33506165,95.64924123)
\lineto(165.60506165,95.64924123)
\curveto(165.62506601,95.64923925)(165.65506598,95.64423926)(165.69506165,95.63424123)
\curveto(165.72506591,95.63423927)(165.75006588,95.63923926)(165.77006165,95.64924123)
\curveto(165.87006576,95.65923924)(165.97006566,95.66423924)(166.07006165,95.66424123)
\curveto(166.16006547,95.67423923)(166.26006537,95.68423922)(166.37006165,95.69424123)
\curveto(166.49006514,95.72423918)(166.61506502,95.73923916)(166.74506165,95.73924123)
\curveto(166.86506477,95.74923915)(166.98006465,95.77423913)(167.09006165,95.81424123)
\curveto(167.39006424,95.89423901)(167.65506398,95.97923892)(167.88506165,96.06924123)
\curveto(168.11506352,96.16923873)(168.3300633,96.31423859)(168.53006165,96.50424123)
\curveto(168.7300629,96.71423819)(168.88006275,96.97923792)(168.98006165,97.29924123)
\curveto(169.00006263,97.33923756)(169.01006262,97.37423753)(169.01006165,97.40424123)
\curveto(169.00006263,97.44423746)(169.00506263,97.48923741)(169.02506165,97.53924123)
\curveto(169.0350626,97.57923732)(169.04506259,97.64923725)(169.05506165,97.74924123)
\curveto(169.06506257,97.85923704)(169.06006257,97.94423696)(169.04006165,98.00424123)
\curveto(169.02006261,98.07423683)(169.01006262,98.14423676)(169.01006165,98.21424123)
\curveto(169.00006263,98.28423662)(168.98506265,98.34923655)(168.96506165,98.40924123)
\curveto(168.90506273,98.60923629)(168.82006281,98.78923611)(168.71006165,98.94924123)
\curveto(168.69006294,98.97923592)(168.67006296,99.0042359)(168.65006165,99.02424123)
\lineto(168.59006165,99.08424123)
\curveto(168.57006306,99.12423578)(168.5300631,99.17423573)(168.47006165,99.23424123)
\curveto(168.3300633,99.33423557)(168.20006343,99.41923548)(168.08006165,99.48924123)
\curveto(167.96006367,99.55923534)(167.81506382,99.62923527)(167.64506165,99.69924123)
\curveto(167.57506406,99.72923517)(167.50506413,99.74923515)(167.43506165,99.75924123)
\curveto(167.36506427,99.77923512)(167.29006434,99.7992351)(167.21006165,99.81924123)
}
}
{
\newrgbcolor{curcolor}{0 0 0}
\pscustom[linestyle=none,fillstyle=solid,fillcolor=curcolor]
{
\newpath
\moveto(159.36506165,106.77385061)
\curveto(159.36507227,106.87384575)(159.37507226,106.96884566)(159.39506165,107.05885061)
\curveto(159.40507223,107.14884548)(159.4350722,107.21384541)(159.48506165,107.25385061)
\curveto(159.56507207,107.31384531)(159.67007196,107.34384528)(159.80006165,107.34385061)
\lineto(160.19006165,107.34385061)
\lineto(161.69006165,107.34385061)
\lineto(168.08006165,107.34385061)
\lineto(169.25006165,107.34385061)
\lineto(169.56506165,107.34385061)
\curveto(169.66506197,107.35384527)(169.74506189,107.33884529)(169.80506165,107.29885061)
\curveto(169.88506175,107.24884538)(169.9350617,107.17384545)(169.95506165,107.07385061)
\curveto(169.96506167,106.98384564)(169.97006166,106.87384575)(169.97006165,106.74385061)
\lineto(169.97006165,106.51885061)
\curveto(169.95006168,106.43884619)(169.9350617,106.36884626)(169.92506165,106.30885061)
\curveto(169.90506173,106.24884638)(169.86506177,106.19884643)(169.80506165,106.15885061)
\curveto(169.74506189,106.11884651)(169.67006196,106.09884653)(169.58006165,106.09885061)
\lineto(169.28006165,106.09885061)
\lineto(168.18506165,106.09885061)
\lineto(162.84506165,106.09885061)
\curveto(162.75506888,106.07884655)(162.68006895,106.06384656)(162.62006165,106.05385061)
\curveto(162.55006908,106.05384657)(162.49006914,106.0238466)(162.44006165,105.96385061)
\curveto(162.39006924,105.89384673)(162.36506927,105.80384682)(162.36506165,105.69385061)
\curveto(162.35506928,105.59384703)(162.35006928,105.48384714)(162.35006165,105.36385061)
\lineto(162.35006165,104.22385061)
\lineto(162.35006165,103.72885061)
\curveto(162.34006929,103.56884906)(162.28006935,103.45884917)(162.17006165,103.39885061)
\curveto(162.14006949,103.37884925)(162.11006952,103.36884926)(162.08006165,103.36885061)
\curveto(162.04006959,103.36884926)(161.99506964,103.36384926)(161.94506165,103.35385061)
\curveto(161.82506981,103.33384929)(161.71506992,103.33884929)(161.61506165,103.36885061)
\curveto(161.51507012,103.40884922)(161.44507019,103.46384916)(161.40506165,103.53385061)
\curveto(161.35507028,103.61384901)(161.3300703,103.73384889)(161.33006165,103.89385061)
\curveto(161.3300703,104.05384857)(161.31507032,104.18884844)(161.28506165,104.29885061)
\curveto(161.27507036,104.34884828)(161.27007036,104.40384822)(161.27006165,104.46385061)
\curveto(161.26007037,104.5238481)(161.24507039,104.58384804)(161.22506165,104.64385061)
\curveto(161.17507046,104.79384783)(161.12507051,104.93884769)(161.07506165,105.07885061)
\curveto(161.01507062,105.21884741)(160.94507069,105.35384727)(160.86506165,105.48385061)
\curveto(160.77507086,105.623847)(160.67007096,105.74384688)(160.55006165,105.84385061)
\curveto(160.4300712,105.94384668)(160.30007133,106.03884659)(160.16006165,106.12885061)
\curveto(160.06007157,106.18884644)(159.95007168,106.23384639)(159.83006165,106.26385061)
\curveto(159.71007192,106.30384632)(159.60507203,106.35384627)(159.51506165,106.41385061)
\curveto(159.45507218,106.46384616)(159.41507222,106.53384609)(159.39506165,106.62385061)
\curveto(159.38507225,106.64384598)(159.38007225,106.66884596)(159.38006165,106.69885061)
\curveto(159.38007225,106.7288459)(159.37507226,106.75384587)(159.36506165,106.77385061)
}
}
{
\newrgbcolor{curcolor}{0 0 0}
\pscustom[linestyle=none,fillstyle=solid,fillcolor=curcolor]
{
\newpath
\moveto(159.36506165,115.12345998)
\curveto(159.36507227,115.22345513)(159.37507226,115.31845503)(159.39506165,115.40845998)
\curveto(159.40507223,115.49845485)(159.4350722,115.56345479)(159.48506165,115.60345998)
\curveto(159.56507207,115.66345469)(159.67007196,115.69345466)(159.80006165,115.69345998)
\lineto(160.19006165,115.69345998)
\lineto(161.69006165,115.69345998)
\lineto(168.08006165,115.69345998)
\lineto(169.25006165,115.69345998)
\lineto(169.56506165,115.69345998)
\curveto(169.66506197,115.70345465)(169.74506189,115.68845466)(169.80506165,115.64845998)
\curveto(169.88506175,115.59845475)(169.9350617,115.52345483)(169.95506165,115.42345998)
\curveto(169.96506167,115.33345502)(169.97006166,115.22345513)(169.97006165,115.09345998)
\lineto(169.97006165,114.86845998)
\curveto(169.95006168,114.78845556)(169.9350617,114.71845563)(169.92506165,114.65845998)
\curveto(169.90506173,114.59845575)(169.86506177,114.5484558)(169.80506165,114.50845998)
\curveto(169.74506189,114.46845588)(169.67006196,114.4484559)(169.58006165,114.44845998)
\lineto(169.28006165,114.44845998)
\lineto(168.18506165,114.44845998)
\lineto(162.84506165,114.44845998)
\curveto(162.75506888,114.42845592)(162.68006895,114.41345594)(162.62006165,114.40345998)
\curveto(162.55006908,114.40345595)(162.49006914,114.37345598)(162.44006165,114.31345998)
\curveto(162.39006924,114.24345611)(162.36506927,114.1534562)(162.36506165,114.04345998)
\curveto(162.35506928,113.94345641)(162.35006928,113.83345652)(162.35006165,113.71345998)
\lineto(162.35006165,112.57345998)
\lineto(162.35006165,112.07845998)
\curveto(162.34006929,111.91845843)(162.28006935,111.80845854)(162.17006165,111.74845998)
\curveto(162.14006949,111.72845862)(162.11006952,111.71845863)(162.08006165,111.71845998)
\curveto(162.04006959,111.71845863)(161.99506964,111.71345864)(161.94506165,111.70345998)
\curveto(161.82506981,111.68345867)(161.71506992,111.68845866)(161.61506165,111.71845998)
\curveto(161.51507012,111.75845859)(161.44507019,111.81345854)(161.40506165,111.88345998)
\curveto(161.35507028,111.96345839)(161.3300703,112.08345827)(161.33006165,112.24345998)
\curveto(161.3300703,112.40345795)(161.31507032,112.53845781)(161.28506165,112.64845998)
\curveto(161.27507036,112.69845765)(161.27007036,112.7534576)(161.27006165,112.81345998)
\curveto(161.26007037,112.87345748)(161.24507039,112.93345742)(161.22506165,112.99345998)
\curveto(161.17507046,113.14345721)(161.12507051,113.28845706)(161.07506165,113.42845998)
\curveto(161.01507062,113.56845678)(160.94507069,113.70345665)(160.86506165,113.83345998)
\curveto(160.77507086,113.97345638)(160.67007096,114.09345626)(160.55006165,114.19345998)
\curveto(160.4300712,114.29345606)(160.30007133,114.38845596)(160.16006165,114.47845998)
\curveto(160.06007157,114.53845581)(159.95007168,114.58345577)(159.83006165,114.61345998)
\curveto(159.71007192,114.6534557)(159.60507203,114.70345565)(159.51506165,114.76345998)
\curveto(159.45507218,114.81345554)(159.41507222,114.88345547)(159.39506165,114.97345998)
\curveto(159.38507225,114.99345536)(159.38007225,115.01845533)(159.38006165,115.04845998)
\curveto(159.38007225,115.07845527)(159.37507226,115.10345525)(159.36506165,115.12345998)
}
}
{
\newrgbcolor{curcolor}{0 0 0}
\pscustom[linestyle=none,fillstyle=solid,fillcolor=curcolor]
{
\newpath
\moveto(201.112724,31.67142873)
\lineto(201.112724,32.58642873)
\curveto(201.11273469,32.68642608)(201.11273469,32.78142599)(201.112724,32.87142873)
\curveto(201.11273469,32.96142581)(201.13273467,33.03642573)(201.172724,33.09642873)
\curveto(201.23273457,33.18642558)(201.31273449,33.24642552)(201.412724,33.27642873)
\curveto(201.51273429,33.31642545)(201.61773419,33.36142541)(201.727724,33.41142873)
\curveto(201.91773389,33.49142528)(202.1077337,33.56142521)(202.297724,33.62142873)
\curveto(202.48773332,33.69142508)(202.67773313,33.766425)(202.867724,33.84642873)
\curveto(203.04773276,33.91642485)(203.23273257,33.98142479)(203.422724,34.04142873)
\curveto(203.6027322,34.10142467)(203.78273202,34.1714246)(203.962724,34.25142873)
\curveto(204.1027317,34.31142446)(204.24773156,34.3664244)(204.397724,34.41642873)
\curveto(204.54773126,34.4664243)(204.69273111,34.52142425)(204.832724,34.58142873)
\curveto(205.28273052,34.76142401)(205.73773007,34.93142384)(206.197724,35.09142873)
\curveto(206.64772916,35.25142352)(207.09772871,35.42142335)(207.547724,35.60142873)
\curveto(207.59772821,35.62142315)(207.64772816,35.63642313)(207.697724,35.64642873)
\lineto(207.847724,35.70642873)
\curveto(208.06772774,35.79642297)(208.29272751,35.88142289)(208.522724,35.96142873)
\curveto(208.74272706,36.04142273)(208.96272684,36.12642264)(209.182724,36.21642873)
\curveto(209.27272653,36.25642251)(209.38272642,36.29642247)(209.512724,36.33642873)
\curveto(209.63272617,36.37642239)(209.7027261,36.44142233)(209.722724,36.53142873)
\curveto(209.73272607,36.5714222)(209.73272607,36.60142217)(209.722724,36.62142873)
\lineto(209.662724,36.68142873)
\curveto(209.61272619,36.73142204)(209.55772625,36.766422)(209.497724,36.78642873)
\curveto(209.43772637,36.81642195)(209.37272643,36.84642192)(209.302724,36.87642873)
\lineto(208.672724,37.11642873)
\curveto(208.45272735,37.19642157)(208.23772757,37.27642149)(208.027724,37.35642873)
\lineto(207.877724,37.41642873)
\lineto(207.697724,37.47642873)
\curveto(207.5077283,37.55642121)(207.31772849,37.62642114)(207.127724,37.68642873)
\curveto(206.92772888,37.75642101)(206.72772908,37.83142094)(206.527724,37.91142873)
\curveto(205.94772986,38.15142062)(205.36273044,38.3714204)(204.772724,38.57142873)
\curveto(204.18273162,38.78141999)(203.59773221,39.00641976)(203.017724,39.24642873)
\curveto(202.81773299,39.32641944)(202.61273319,39.40141937)(202.402724,39.47142873)
\curveto(202.19273361,39.55141922)(201.98773382,39.63141914)(201.787724,39.71142873)
\curveto(201.7077341,39.75141902)(201.6077342,39.78641898)(201.487724,39.81642873)
\curveto(201.36773444,39.85641891)(201.28273452,39.91141886)(201.232724,39.98142873)
\curveto(201.19273461,40.04141873)(201.16273464,40.11641865)(201.142724,40.20642873)
\curveto(201.12273468,40.30641846)(201.11273469,40.41641835)(201.112724,40.53642873)
\curveto(201.1027347,40.65641811)(201.1027347,40.77641799)(201.112724,40.89642873)
\curveto(201.11273469,41.01641775)(201.11273469,41.12641764)(201.112724,41.22642873)
\curveto(201.11273469,41.31641745)(201.11273469,41.40641736)(201.112724,41.49642873)
\curveto(201.11273469,41.59641717)(201.13273467,41.6714171)(201.172724,41.72142873)
\curveto(201.22273458,41.81141696)(201.31273449,41.86141691)(201.442724,41.87142873)
\curveto(201.57273423,41.88141689)(201.71273409,41.88641688)(201.862724,41.88642873)
\lineto(203.512724,41.88642873)
\lineto(209.782724,41.88642873)
\lineto(211.042724,41.88642873)
\curveto(211.15272465,41.88641688)(211.26272454,41.88641688)(211.372724,41.88642873)
\curveto(211.48272432,41.89641687)(211.56772424,41.87641689)(211.627724,41.82642873)
\curveto(211.68772412,41.79641697)(211.72772408,41.75141702)(211.747724,41.69142873)
\curveto(211.75772405,41.63141714)(211.77272403,41.56141721)(211.792724,41.48142873)
\lineto(211.792724,41.24142873)
\lineto(211.792724,40.88142873)
\curveto(211.78272402,40.771418)(211.73772407,40.69141808)(211.657724,40.64142873)
\curveto(211.62772418,40.62141815)(211.59772421,40.60641816)(211.567724,40.59642873)
\curveto(211.52772428,40.59641817)(211.48272432,40.58641818)(211.432724,40.56642873)
\lineto(211.267724,40.56642873)
\curveto(211.2077246,40.55641821)(211.13772467,40.55141822)(211.057724,40.55142873)
\curveto(210.97772483,40.56141821)(210.9027249,40.5664182)(210.832724,40.56642873)
\lineto(209.992724,40.56642873)
\lineto(205.567724,40.56642873)
\curveto(205.31773049,40.5664182)(205.06773074,40.5664182)(204.817724,40.56642873)
\curveto(204.55773125,40.5664182)(204.3077315,40.56141821)(204.067724,40.55142873)
\curveto(203.96773184,40.55141822)(203.85773195,40.54641822)(203.737724,40.53642873)
\curveto(203.61773219,40.52641824)(203.55773225,40.4714183)(203.557724,40.37142873)
\lineto(203.572724,40.37142873)
\curveto(203.59273221,40.30141847)(203.65773215,40.24141853)(203.767724,40.19142873)
\curveto(203.87773193,40.15141862)(203.97273183,40.11641865)(204.052724,40.08642873)
\curveto(204.22273158,40.01641875)(204.39773141,39.95141882)(204.577724,39.89142873)
\curveto(204.74773106,39.83141894)(204.91773089,39.76141901)(205.087724,39.68142873)
\curveto(205.13773067,39.66141911)(205.18273062,39.64641912)(205.222724,39.63642873)
\curveto(205.26273054,39.62641914)(205.3077305,39.61141916)(205.357724,39.59142873)
\curveto(205.53773027,39.51141926)(205.72273008,39.44141933)(205.912724,39.38142873)
\curveto(206.09272971,39.33141944)(206.27272953,39.2664195)(206.452724,39.18642873)
\curveto(206.6027292,39.11641965)(206.75772905,39.05641971)(206.917724,39.00642873)
\curveto(207.06772874,38.95641981)(207.21772859,38.90141987)(207.367724,38.84142873)
\curveto(207.83772797,38.64142013)(208.31272749,38.46142031)(208.792724,38.30142873)
\curveto(209.26272654,38.14142063)(209.72772608,37.9664208)(210.187724,37.77642873)
\curveto(210.36772544,37.69642107)(210.54772526,37.62642114)(210.727724,37.56642873)
\curveto(210.9077249,37.50642126)(211.08772472,37.44142133)(211.267724,37.37142873)
\curveto(211.37772443,37.32142145)(211.48272432,37.2714215)(211.582724,37.22142873)
\curveto(211.67272413,37.18142159)(211.73772407,37.09642167)(211.777724,36.96642873)
\curveto(211.78772402,36.94642182)(211.79272401,36.92142185)(211.792724,36.89142873)
\curveto(211.78272402,36.8714219)(211.78272402,36.84642192)(211.792724,36.81642873)
\curveto(211.802724,36.78642198)(211.807724,36.75142202)(211.807724,36.71142873)
\curveto(211.79772401,36.6714221)(211.79272401,36.63142214)(211.792724,36.59142873)
\lineto(211.792724,36.29142873)
\curveto(211.79272401,36.19142258)(211.76772404,36.11142266)(211.717724,36.05142873)
\curveto(211.66772414,35.9714228)(211.59772421,35.91142286)(211.507724,35.87142873)
\curveto(211.4077244,35.84142293)(211.3077245,35.80142297)(211.207724,35.75142873)
\curveto(211.0077248,35.6714231)(210.802725,35.59142318)(210.592724,35.51142873)
\curveto(210.37272543,35.44142333)(210.16272564,35.3664234)(209.962724,35.28642873)
\curveto(209.78272602,35.20642356)(209.6027262,35.13642363)(209.422724,35.07642873)
\curveto(209.23272657,35.02642374)(209.04772676,34.96142381)(208.867724,34.88142873)
\curveto(208.3077275,34.65142412)(207.74272806,34.43642433)(207.172724,34.23642873)
\curveto(206.6027292,34.03642473)(206.03772977,33.82142495)(205.477724,33.59142873)
\lineto(204.847724,33.35142873)
\curveto(204.62773118,33.28142549)(204.41773139,33.20642556)(204.217724,33.12642873)
\curveto(204.1077317,33.07642569)(204.0027318,33.03142574)(203.902724,32.99142873)
\curveto(203.79273201,32.96142581)(203.69773211,32.91142586)(203.617724,32.84142873)
\curveto(203.59773221,32.83142594)(203.58773222,32.82142595)(203.587724,32.81142873)
\lineto(203.557724,32.78142873)
\lineto(203.557724,32.70642873)
\lineto(203.587724,32.67642873)
\curveto(203.58773222,32.6664261)(203.59273221,32.65642611)(203.602724,32.64642873)
\curveto(203.65273215,32.62642614)(203.7077321,32.61642615)(203.767724,32.61642873)
\curveto(203.82773198,32.61642615)(203.88773192,32.60642616)(203.947724,32.58642873)
\lineto(204.112724,32.58642873)
\curveto(204.17273163,32.5664262)(204.23773157,32.56142621)(204.307724,32.57142873)
\curveto(204.37773143,32.58142619)(204.44773136,32.58642618)(204.517724,32.58642873)
\lineto(205.327724,32.58642873)
\lineto(209.887724,32.58642873)
\lineto(211.072724,32.58642873)
\curveto(211.18272462,32.58642618)(211.29272451,32.58142619)(211.402724,32.57142873)
\curveto(211.51272429,32.5714262)(211.59772421,32.54642622)(211.657724,32.49642873)
\curveto(211.73772407,32.44642632)(211.78272402,32.35642641)(211.792724,32.22642873)
\lineto(211.792724,31.83642873)
\lineto(211.792724,31.64142873)
\curveto(211.79272401,31.59142718)(211.78272402,31.54142723)(211.762724,31.49142873)
\curveto(211.72272408,31.36142741)(211.63772417,31.28642748)(211.507724,31.26642873)
\curveto(211.37772443,31.25642751)(211.22772458,31.25142752)(211.057724,31.25142873)
\lineto(209.317724,31.25142873)
\lineto(203.317724,31.25142873)
\lineto(201.907724,31.25142873)
\curveto(201.79773401,31.25142752)(201.68273412,31.24642752)(201.562724,31.23642873)
\curveto(201.44273436,31.23642753)(201.34773446,31.26142751)(201.277724,31.31142873)
\curveto(201.21773459,31.35142742)(201.16773464,31.42642734)(201.127724,31.53642873)
\curveto(201.11773469,31.55642721)(201.11773469,31.57642719)(201.127724,31.59642873)
\curveto(201.12773468,31.62642714)(201.12273468,31.65142712)(201.112724,31.67142873)
}
}
{
\newrgbcolor{curcolor}{0 0 0}
\pscustom[linestyle=none,fillstyle=solid,fillcolor=curcolor]
{
\newpath
\moveto(211.237724,50.87353811)
\curveto(211.39772441,50.90353028)(211.53272427,50.88853029)(211.642724,50.82853811)
\curveto(211.74272406,50.76853041)(211.81772399,50.68853049)(211.867724,50.58853811)
\curveto(211.88772392,50.53853064)(211.89772391,50.4835307)(211.897724,50.42353811)
\curveto(211.89772391,50.37353081)(211.9077239,50.31853086)(211.927724,50.25853811)
\curveto(211.97772383,50.03853114)(211.96272384,49.81853136)(211.882724,49.59853811)
\curveto(211.81272399,49.38853179)(211.72272408,49.24353194)(211.612724,49.16353811)
\curveto(211.54272426,49.11353207)(211.46272434,49.06853211)(211.372724,49.02853811)
\curveto(211.27272453,48.98853219)(211.19272461,48.93853224)(211.132724,48.87853811)
\curveto(211.11272469,48.85853232)(211.09272471,48.83353235)(211.072724,48.80353811)
\curveto(211.05272475,48.7835324)(211.04772476,48.75353243)(211.057724,48.71353811)
\curveto(211.08772472,48.60353258)(211.14272466,48.49853268)(211.222724,48.39853811)
\curveto(211.3027245,48.30853287)(211.37272443,48.21853296)(211.432724,48.12853811)
\curveto(211.51272429,47.99853318)(211.58772422,47.85853332)(211.657724,47.70853811)
\curveto(211.71772409,47.55853362)(211.77272403,47.39853378)(211.822724,47.22853811)
\curveto(211.85272395,47.12853405)(211.87272393,47.01853416)(211.882724,46.89853811)
\curveto(211.89272391,46.78853439)(211.9077239,46.6785345)(211.927724,46.56853811)
\curveto(211.93772387,46.51853466)(211.94272386,46.47353471)(211.942724,46.43353811)
\lineto(211.942724,46.32853811)
\curveto(211.96272384,46.21853496)(211.96272384,46.11353507)(211.942724,46.01353811)
\lineto(211.942724,45.87853811)
\curveto(211.93272387,45.82853535)(211.92772388,45.7785354)(211.927724,45.72853811)
\curveto(211.92772388,45.6785355)(211.91772389,45.63353555)(211.897724,45.59353811)
\curveto(211.88772392,45.55353563)(211.88272392,45.51853566)(211.882724,45.48853811)
\curveto(211.89272391,45.46853571)(211.89272391,45.44353574)(211.882724,45.41353811)
\lineto(211.822724,45.17353811)
\curveto(211.81272399,45.09353609)(211.79272401,45.01853616)(211.762724,44.94853811)
\curveto(211.63272417,44.64853653)(211.48772432,44.40353678)(211.327724,44.21353811)
\curveto(211.15772465,44.03353715)(210.92272488,43.8835373)(210.622724,43.76353811)
\curveto(210.4027254,43.67353751)(210.13772567,43.62853755)(209.827724,43.62853811)
\lineto(209.512724,43.62853811)
\curveto(209.46272634,43.63853754)(209.41272639,43.64353754)(209.362724,43.64353811)
\lineto(209.182724,43.67353811)
\lineto(208.852724,43.79353811)
\curveto(208.74272706,43.83353735)(208.64272716,43.8835373)(208.552724,43.94353811)
\curveto(208.26272754,44.12353706)(208.04772776,44.36853681)(207.907724,44.67853811)
\curveto(207.76772804,44.98853619)(207.64272816,45.32853585)(207.532724,45.69853811)
\curveto(207.49272831,45.83853534)(207.46272834,45.9835352)(207.442724,46.13353811)
\curveto(207.42272838,46.2835349)(207.39772841,46.43353475)(207.367724,46.58353811)
\curveto(207.34772846,46.65353453)(207.33772847,46.71853446)(207.337724,46.77853811)
\curveto(207.33772847,46.84853433)(207.32772848,46.92353426)(207.307724,47.00353811)
\curveto(207.28772852,47.07353411)(207.27772853,47.14353404)(207.277724,47.21353811)
\curveto(207.26772854,47.2835339)(207.25272855,47.35853382)(207.232724,47.43853811)
\curveto(207.17272863,47.68853349)(207.12272868,47.92353326)(207.082724,48.14353811)
\curveto(207.03272877,48.36353282)(206.91772889,48.53853264)(206.737724,48.66853811)
\curveto(206.65772915,48.72853245)(206.55772925,48.7785324)(206.437724,48.81853811)
\curveto(206.3077295,48.85853232)(206.16772964,48.85853232)(206.017724,48.81853811)
\curveto(205.77773003,48.75853242)(205.58773022,48.66853251)(205.447724,48.54853811)
\curveto(205.3077305,48.43853274)(205.19773061,48.2785329)(205.117724,48.06853811)
\curveto(205.06773074,47.94853323)(205.03273077,47.80353338)(205.012724,47.63353811)
\curveto(204.99273081,47.47353371)(204.98273082,47.30353388)(204.982724,47.12353811)
\curveto(204.98273082,46.94353424)(204.99273081,46.76853441)(205.012724,46.59853811)
\curveto(205.03273077,46.42853475)(205.06273074,46.2835349)(205.102724,46.16353811)
\curveto(205.16273064,45.99353519)(205.24773056,45.82853535)(205.357724,45.66853811)
\curveto(205.41773039,45.58853559)(205.49773031,45.51353567)(205.597724,45.44353811)
\curveto(205.68773012,45.3835358)(205.78773002,45.32853585)(205.897724,45.27853811)
\curveto(205.97772983,45.24853593)(206.06272974,45.21853596)(206.152724,45.18853811)
\curveto(206.24272956,45.16853601)(206.31272949,45.12353606)(206.362724,45.05353811)
\curveto(206.39272941,45.01353617)(206.41772939,44.94353624)(206.437724,44.84353811)
\curveto(206.44772936,44.75353643)(206.45272935,44.65853652)(206.452724,44.55853811)
\curveto(206.45272935,44.45853672)(206.44772936,44.35853682)(206.437724,44.25853811)
\curveto(206.41772939,44.16853701)(206.39272941,44.10353708)(206.362724,44.06353811)
\curveto(206.33272947,44.02353716)(206.28272952,43.99353719)(206.212724,43.97353811)
\curveto(206.14272966,43.95353723)(206.06772974,43.95353723)(205.987724,43.97353811)
\curveto(205.85772995,44.00353718)(205.73773007,44.03353715)(205.627724,44.06353811)
\curveto(205.5077303,44.10353708)(205.39273041,44.14853703)(205.282724,44.19853811)
\curveto(204.93273087,44.38853679)(204.66273114,44.62853655)(204.472724,44.91853811)
\curveto(204.27273153,45.20853597)(204.11273169,45.56853561)(203.992724,45.99853811)
\curveto(203.97273183,46.09853508)(203.95773185,46.19853498)(203.947724,46.29853811)
\curveto(203.93773187,46.40853477)(203.92273188,46.51853466)(203.902724,46.62853811)
\curveto(203.89273191,46.66853451)(203.89273191,46.73353445)(203.902724,46.82353811)
\curveto(203.9027319,46.91353427)(203.89273191,46.96853421)(203.872724,46.98853811)
\curveto(203.86273194,47.68853349)(203.94273186,48.29853288)(204.112724,48.81853811)
\curveto(204.28273152,49.33853184)(204.6077312,49.70353148)(205.087724,49.91353811)
\curveto(205.28773052,50.00353118)(205.52273028,50.05353113)(205.792724,50.06353811)
\curveto(206.05272975,50.0835311)(206.32772948,50.09353109)(206.617724,50.09353811)
\lineto(209.932724,50.09353811)
\curveto(210.07272573,50.09353109)(210.2077256,50.09853108)(210.337724,50.10853811)
\curveto(210.46772534,50.11853106)(210.57272523,50.14853103)(210.652724,50.19853811)
\curveto(210.72272508,50.24853093)(210.77272503,50.31353087)(210.802724,50.39353811)
\curveto(210.84272496,50.4835307)(210.87272493,50.56853061)(210.892724,50.64853811)
\curveto(210.9027249,50.72853045)(210.94772486,50.78853039)(211.027724,50.82853811)
\curveto(211.05772475,50.84853033)(211.08772472,50.85853032)(211.117724,50.85853811)
\curveto(211.14772466,50.85853032)(211.18772462,50.86353032)(211.237724,50.87353811)
\moveto(209.572724,48.72853811)
\curveto(209.43272637,48.78853239)(209.27272653,48.81853236)(209.092724,48.81853811)
\curveto(208.9027269,48.82853235)(208.7077271,48.83353235)(208.507724,48.83353811)
\curveto(208.39772741,48.83353235)(208.29772751,48.82853235)(208.207724,48.81853811)
\curveto(208.11772769,48.80853237)(208.04772776,48.76853241)(207.997724,48.69853811)
\curveto(207.97772783,48.66853251)(207.96772784,48.59853258)(207.967724,48.48853811)
\curveto(207.98772782,48.46853271)(207.99772781,48.43353275)(207.997724,48.38353811)
\curveto(207.99772781,48.33353285)(208.0077278,48.28853289)(208.027724,48.24853811)
\curveto(208.04772776,48.16853301)(208.06772774,48.0785331)(208.087724,47.97853811)
\lineto(208.147724,47.67853811)
\curveto(208.14772766,47.64853353)(208.15272765,47.61353357)(208.162724,47.57353811)
\lineto(208.162724,47.46853811)
\curveto(208.2027276,47.31853386)(208.22772758,47.15353403)(208.237724,46.97353811)
\curveto(208.23772757,46.80353438)(208.25772755,46.64353454)(208.297724,46.49353811)
\curveto(208.31772749,46.41353477)(208.33772747,46.33853484)(208.357724,46.26853811)
\curveto(208.36772744,46.20853497)(208.38272742,46.13853504)(208.402724,46.05853811)
\curveto(208.45272735,45.89853528)(208.51772729,45.74853543)(208.597724,45.60853811)
\curveto(208.66772714,45.46853571)(208.75772705,45.34853583)(208.867724,45.24853811)
\curveto(208.97772683,45.14853603)(209.11272669,45.07353611)(209.272724,45.02353811)
\curveto(209.42272638,44.97353621)(209.6077262,44.95353623)(209.827724,44.96353811)
\curveto(209.92772588,44.96353622)(210.02272578,44.9785362)(210.112724,45.00853811)
\curveto(210.19272561,45.04853613)(210.26772554,45.09353609)(210.337724,45.14353811)
\curveto(210.44772536,45.22353596)(210.54272526,45.32853585)(210.622724,45.45853811)
\curveto(210.69272511,45.58853559)(210.75272505,45.72853545)(210.802724,45.87853811)
\curveto(210.81272499,45.92853525)(210.81772499,45.9785352)(210.817724,46.02853811)
\curveto(210.81772499,46.0785351)(210.82272498,46.12853505)(210.832724,46.17853811)
\curveto(210.85272495,46.24853493)(210.86772494,46.33353485)(210.877724,46.43353811)
\curveto(210.87772493,46.54353464)(210.86772494,46.63353455)(210.847724,46.70353811)
\curveto(210.82772498,46.76353442)(210.82272498,46.82353436)(210.832724,46.88353811)
\curveto(210.83272497,46.94353424)(210.82272498,47.00353418)(210.802724,47.06353811)
\curveto(210.78272502,47.14353404)(210.76772504,47.21853396)(210.757724,47.28853811)
\curveto(210.74772506,47.36853381)(210.72772508,47.44353374)(210.697724,47.51353811)
\curveto(210.57772523,47.80353338)(210.43272537,48.04853313)(210.262724,48.24853811)
\curveto(210.09272571,48.45853272)(209.86272594,48.61853256)(209.572724,48.72853811)
}
}
{
\newrgbcolor{curcolor}{0 0 0}
\pscustom[linestyle=none,fillstyle=solid,fillcolor=curcolor]
{
\newpath
\moveto(203.887724,55.69017873)
\curveto(203.88773192,55.92017394)(203.94773186,56.05017381)(204.067724,56.08017873)
\curveto(204.17773163,56.11017375)(204.34273146,56.12517374)(204.562724,56.12517873)
\lineto(204.847724,56.12517873)
\curveto(204.93773087,56.12517374)(205.01273079,56.10017376)(205.072724,56.05017873)
\curveto(205.15273065,55.99017387)(205.19773061,55.90517396)(205.207724,55.79517873)
\curveto(205.2077306,55.68517418)(205.22273058,55.57517429)(205.252724,55.46517873)
\curveto(205.28273052,55.32517454)(205.31273049,55.19017467)(205.342724,55.06017873)
\curveto(205.37273043,54.94017492)(205.41273039,54.82517504)(205.462724,54.71517873)
\curveto(205.59273021,54.42517544)(205.77273003,54.19017567)(206.002724,54.01017873)
\curveto(206.22272958,53.83017603)(206.47772933,53.67517619)(206.767724,53.54517873)
\curveto(206.87772893,53.50517636)(206.99272881,53.47517639)(207.112724,53.45517873)
\curveto(207.22272858,53.43517643)(207.33772847,53.41017645)(207.457724,53.38017873)
\curveto(207.5077283,53.37017649)(207.55772825,53.3651765)(207.607724,53.36517873)
\curveto(207.65772815,53.37517649)(207.7077281,53.37517649)(207.757724,53.36517873)
\curveto(207.87772793,53.33517653)(208.01772779,53.32017654)(208.177724,53.32017873)
\curveto(208.32772748,53.33017653)(208.47272733,53.33517653)(208.612724,53.33517873)
\lineto(210.457724,53.33517873)
\lineto(210.802724,53.33517873)
\curveto(210.92272488,53.33517653)(211.03772477,53.33017653)(211.147724,53.32017873)
\curveto(211.25772455,53.31017655)(211.35272445,53.30517656)(211.432724,53.30517873)
\curveto(211.51272429,53.31517655)(211.58272422,53.29517657)(211.642724,53.24517873)
\curveto(211.71272409,53.19517667)(211.75272405,53.11517675)(211.762724,53.00517873)
\curveto(211.77272403,52.90517696)(211.77772403,52.79517707)(211.777724,52.67517873)
\lineto(211.777724,52.40517873)
\curveto(211.75772405,52.35517751)(211.74272406,52.30517756)(211.732724,52.25517873)
\curveto(211.71272409,52.21517765)(211.68772412,52.18517768)(211.657724,52.16517873)
\curveto(211.58772422,52.11517775)(211.5027243,52.08517778)(211.402724,52.07517873)
\lineto(211.072724,52.07517873)
\lineto(209.917724,52.07517873)
\lineto(205.762724,52.07517873)
\lineto(204.727724,52.07517873)
\lineto(204.427724,52.07517873)
\curveto(204.32773148,52.08517778)(204.24273156,52.11517775)(204.172724,52.16517873)
\curveto(204.13273167,52.19517767)(204.1027317,52.24517762)(204.082724,52.31517873)
\curveto(204.06273174,52.39517747)(204.05273175,52.48017738)(204.052724,52.57017873)
\curveto(204.04273176,52.6601772)(204.04273176,52.75017711)(204.052724,52.84017873)
\curveto(204.06273174,52.93017693)(204.07773173,53.00017686)(204.097724,53.05017873)
\curveto(204.12773168,53.13017673)(204.18773162,53.18017668)(204.277724,53.20017873)
\curveto(204.35773145,53.23017663)(204.44773136,53.24517662)(204.547724,53.24517873)
\lineto(204.847724,53.24517873)
\curveto(204.94773086,53.24517662)(205.03773077,53.2651766)(205.117724,53.30517873)
\curveto(205.13773067,53.31517655)(205.15273065,53.32517654)(205.162724,53.33517873)
\lineto(205.207724,53.38017873)
\curveto(205.2077306,53.49017637)(205.16273064,53.58017628)(205.072724,53.65017873)
\curveto(204.97273083,53.72017614)(204.89273091,53.78017608)(204.832724,53.83017873)
\lineto(204.742724,53.92017873)
\curveto(204.63273117,54.01017585)(204.51773129,54.13517573)(204.397724,54.29517873)
\curveto(204.27773153,54.45517541)(204.18773162,54.60517526)(204.127724,54.74517873)
\curveto(204.07773173,54.83517503)(204.04273176,54.93017493)(204.022724,55.03017873)
\curveto(203.99273181,55.13017473)(203.96273184,55.23517463)(203.932724,55.34517873)
\curveto(203.92273188,55.40517446)(203.91773189,55.4651744)(203.917724,55.52517873)
\curveto(203.9077319,55.58517428)(203.89773191,55.64017422)(203.887724,55.69017873)
}
}
{
\newrgbcolor{curcolor}{0 0 0}
\pscustom[linestyle=none,fillstyle=solid,fillcolor=curcolor]
{
}
}
{
\newrgbcolor{curcolor}{0 0 0}
\pscustom[linestyle=none,fillstyle=solid,fillcolor=curcolor]
{
\newpath
\moveto(201.187724,65.05510061)
\curveto(201.18773462,65.15509575)(201.19773461,65.25009566)(201.217724,65.34010061)
\curveto(201.22773458,65.43009548)(201.25773455,65.49509541)(201.307724,65.53510061)
\curveto(201.38773442,65.59509531)(201.49273431,65.62509528)(201.622724,65.62510061)
\lineto(202.012724,65.62510061)
\lineto(203.512724,65.62510061)
\lineto(209.902724,65.62510061)
\lineto(211.072724,65.62510061)
\lineto(211.387724,65.62510061)
\curveto(211.48772432,65.63509527)(211.56772424,65.62009529)(211.627724,65.58010061)
\curveto(211.7077241,65.53009538)(211.75772405,65.45509545)(211.777724,65.35510061)
\curveto(211.78772402,65.26509564)(211.79272401,65.15509575)(211.792724,65.02510061)
\lineto(211.792724,64.80010061)
\curveto(211.77272403,64.72009619)(211.75772405,64.65009626)(211.747724,64.59010061)
\curveto(211.72772408,64.53009638)(211.68772412,64.48009643)(211.627724,64.44010061)
\curveto(211.56772424,64.40009651)(211.49272431,64.38009653)(211.402724,64.38010061)
\lineto(211.102724,64.38010061)
\lineto(210.007724,64.38010061)
\lineto(204.667724,64.38010061)
\curveto(204.57773123,64.36009655)(204.5027313,64.34509656)(204.442724,64.33510061)
\curveto(204.37273143,64.33509657)(204.31273149,64.3050966)(204.262724,64.24510061)
\curveto(204.21273159,64.17509673)(204.18773162,64.08509682)(204.187724,63.97510061)
\curveto(204.17773163,63.87509703)(204.17273163,63.76509714)(204.172724,63.64510061)
\lineto(204.172724,62.50510061)
\lineto(204.172724,62.01010061)
\curveto(204.16273164,61.85009906)(204.1027317,61.74009917)(203.992724,61.68010061)
\curveto(203.96273184,61.66009925)(203.93273187,61.65009926)(203.902724,61.65010061)
\curveto(203.86273194,61.65009926)(203.81773199,61.64509926)(203.767724,61.63510061)
\curveto(203.64773216,61.61509929)(203.53773227,61.62009929)(203.437724,61.65010061)
\curveto(203.33773247,61.69009922)(203.26773254,61.74509916)(203.227724,61.81510061)
\curveto(203.17773263,61.89509901)(203.15273265,62.01509889)(203.152724,62.17510061)
\curveto(203.15273265,62.33509857)(203.13773267,62.47009844)(203.107724,62.58010061)
\curveto(203.09773271,62.63009828)(203.09273271,62.68509822)(203.092724,62.74510061)
\curveto(203.08273272,62.8050981)(203.06773274,62.86509804)(203.047724,62.92510061)
\curveto(202.99773281,63.07509783)(202.94773286,63.22009769)(202.897724,63.36010061)
\curveto(202.83773297,63.50009741)(202.76773304,63.63509727)(202.687724,63.76510061)
\curveto(202.59773321,63.905097)(202.49273331,64.02509688)(202.372724,64.12510061)
\curveto(202.25273355,64.22509668)(202.12273368,64.32009659)(201.982724,64.41010061)
\curveto(201.88273392,64.47009644)(201.77273403,64.51509639)(201.652724,64.54510061)
\curveto(201.53273427,64.58509632)(201.42773438,64.63509627)(201.337724,64.69510061)
\curveto(201.27773453,64.74509616)(201.23773457,64.81509609)(201.217724,64.90510061)
\curveto(201.2077346,64.92509598)(201.2027346,64.95009596)(201.202724,64.98010061)
\curveto(201.2027346,65.0100959)(201.19773461,65.03509587)(201.187724,65.05510061)
}
}
{
\newrgbcolor{curcolor}{0 0 0}
\pscustom[linestyle=none,fillstyle=solid,fillcolor=curcolor]
{
\newpath
\moveto(208.717724,76.35970998)
\curveto(208.75772705,76.36970226)(208.807727,76.36970226)(208.867724,76.35970998)
\curveto(208.92772688,76.35970227)(208.97772683,76.35470228)(209.017724,76.34470998)
\curveto(209.05772675,76.34470229)(209.09772671,76.33970229)(209.137724,76.32970998)
\lineto(209.242724,76.32970998)
\curveto(209.32272648,76.30970232)(209.4027264,76.29470234)(209.482724,76.28470998)
\curveto(209.56272624,76.27470236)(209.63772617,76.25470238)(209.707724,76.22470998)
\curveto(209.78772602,76.20470243)(209.86272594,76.18470245)(209.932724,76.16470998)
\curveto(210.0027258,76.14470249)(210.07772573,76.11470252)(210.157724,76.07470998)
\curveto(210.57772523,75.89470274)(210.91772489,75.63970299)(211.177724,75.30970998)
\curveto(211.43772437,74.97970365)(211.64272416,74.58970404)(211.792724,74.13970998)
\curveto(211.83272397,74.01970461)(211.85772395,73.89470474)(211.867724,73.76470998)
\curveto(211.88772392,73.64470499)(211.91272389,73.51970511)(211.942724,73.38970998)
\curveto(211.95272385,73.3297053)(211.95772385,73.26470537)(211.957724,73.19470998)
\curveto(211.95772385,73.1347055)(211.96272384,73.06970556)(211.972724,72.99970998)
\lineto(211.972724,72.87970998)
\lineto(211.972724,72.68470998)
\curveto(211.98272382,72.62470601)(211.97772383,72.56970606)(211.957724,72.51970998)
\curveto(211.93772387,72.44970618)(211.93272387,72.38470625)(211.942724,72.32470998)
\curveto(211.95272385,72.26470637)(211.94772386,72.20470643)(211.927724,72.14470998)
\curveto(211.91772389,72.09470654)(211.91272389,72.04970658)(211.912724,72.00970998)
\curveto(211.91272389,71.96970666)(211.9027239,71.92470671)(211.882724,71.87470998)
\curveto(211.86272394,71.79470684)(211.84272396,71.71970691)(211.822724,71.64970998)
\curveto(211.81272399,71.57970705)(211.79772401,71.50970712)(211.777724,71.43970998)
\curveto(211.6077242,70.95970767)(211.39772441,70.55970807)(211.147724,70.23970998)
\curveto(210.88772492,69.9297087)(210.53272527,69.67970895)(210.082724,69.48970998)
\curveto(210.02272578,69.45970917)(209.96272584,69.4347092)(209.902724,69.41470998)
\curveto(209.83272597,69.40470923)(209.75772605,69.38970924)(209.677724,69.36970998)
\curveto(209.61772619,69.34970928)(209.55272625,69.3347093)(209.482724,69.32470998)
\curveto(209.41272639,69.31470932)(209.34272646,69.29970933)(209.272724,69.27970998)
\curveto(209.22272658,69.26970936)(209.18272662,69.26470937)(209.152724,69.26470998)
\lineto(209.032724,69.26470998)
\curveto(208.99272681,69.25470938)(208.94272686,69.24470939)(208.882724,69.23470998)
\curveto(208.82272698,69.2347094)(208.77272703,69.23970939)(208.732724,69.24970998)
\lineto(208.597724,69.24970998)
\curveto(208.54772726,69.25970937)(208.49772731,69.26470937)(208.447724,69.26470998)
\curveto(208.34772746,69.28470935)(208.25272755,69.29970933)(208.162724,69.30970998)
\curveto(208.06272774,69.31970931)(207.96772784,69.33970929)(207.877724,69.36970998)
\curveto(207.72772808,69.41970921)(207.58772822,69.47470916)(207.457724,69.53470998)
\curveto(207.32772848,69.59470904)(207.2077286,69.66470897)(207.097724,69.74470998)
\curveto(207.04772876,69.77470886)(207.0077288,69.80470883)(206.977724,69.83470998)
\curveto(206.94772886,69.87470876)(206.91272889,69.90970872)(206.872724,69.93970998)
\curveto(206.79272901,69.99970863)(206.72272908,70.06970856)(206.662724,70.14970998)
\curveto(206.61272919,70.20970842)(206.56772924,70.26970836)(206.527724,70.32970998)
\lineto(206.377724,70.53970998)
\curveto(206.33772947,70.58970804)(206.3027295,70.63970799)(206.272724,70.68970998)
\curveto(206.23272957,70.73970789)(206.17772963,70.77470786)(206.107724,70.79470998)
\curveto(206.07772973,70.79470784)(206.05272975,70.78470785)(206.032724,70.76470998)
\curveto(206.0027298,70.75470788)(205.97772983,70.74470789)(205.957724,70.73470998)
\curveto(205.9077299,70.69470794)(205.86272994,70.64470799)(205.822724,70.58470998)
\curveto(205.77273003,70.5347081)(205.72773008,70.48470815)(205.687724,70.43470998)
\curveto(205.65773015,70.39470824)(205.6027302,70.34470829)(205.522724,70.28470998)
\curveto(205.49273031,70.26470837)(205.46773034,70.2347084)(205.447724,70.19470998)
\curveto(205.41773039,70.16470847)(205.38273042,70.13970849)(205.342724,70.11970998)
\curveto(205.13273067,69.94970868)(204.88773092,69.81970881)(204.607724,69.72970998)
\curveto(204.52773128,69.70970892)(204.44773136,69.69470894)(204.367724,69.68470998)
\curveto(204.28773152,69.67470896)(204.2077316,69.65970897)(204.127724,69.63970998)
\curveto(204.07773173,69.61970901)(204.01273179,69.60970902)(203.932724,69.60970998)
\curveto(203.84273196,69.60970902)(203.77273203,69.61970901)(203.722724,69.63970998)
\curveto(203.62273218,69.63970899)(203.55273225,69.64470899)(203.512724,69.65470998)
\curveto(203.43273237,69.67470896)(203.36273244,69.68970894)(203.302724,69.69970998)
\curveto(203.23273257,69.70970892)(203.16273264,69.72470891)(203.092724,69.74470998)
\curveto(202.66273314,69.89470874)(202.31773349,70.10970852)(202.057724,70.38970998)
\curveto(201.79773401,70.67970795)(201.58273422,71.0297076)(201.412724,71.43970998)
\curveto(201.36273444,71.54970708)(201.33273447,71.66470697)(201.322724,71.78470998)
\curveto(201.3027345,71.91470672)(201.27273453,72.04470659)(201.232724,72.17470998)
\curveto(201.23273457,72.25470638)(201.23273457,72.32470631)(201.232724,72.38470998)
\curveto(201.22273458,72.45470618)(201.21273459,72.5297061)(201.202724,72.60970998)
\curveto(201.18273462,73.39970523)(201.31273449,74.05470458)(201.592724,74.57470998)
\curveto(201.87273393,75.10470353)(202.28273352,75.48470315)(202.822724,75.71470998)
\curveto(203.05273275,75.82470281)(203.33773247,75.89470274)(203.677724,75.92470998)
\curveto(204.0077318,75.96470267)(204.31273149,75.9347027)(204.592724,75.83470998)
\curveto(204.72273108,75.79470284)(204.84273096,75.74470289)(204.952724,75.68470998)
\curveto(205.06273074,75.634703)(205.16773064,75.57470306)(205.267724,75.50470998)
\curveto(205.3077305,75.48470315)(205.34273046,75.45470318)(205.372724,75.41470998)
\lineto(205.462724,75.32470998)
\curveto(205.55273025,75.27470336)(205.61773019,75.21470342)(205.657724,75.14470998)
\curveto(205.7077301,75.09470354)(205.75773005,75.03970359)(205.807724,74.97970998)
\curveto(205.84772996,74.9297037)(205.89272991,74.88470375)(205.942724,74.84470998)
\curveto(205.96272984,74.82470381)(205.98772982,74.80470383)(206.017724,74.78470998)
\curveto(206.03772977,74.77470386)(206.06272974,74.77470386)(206.092724,74.78470998)
\curveto(206.14272966,74.79470384)(206.19272961,74.82470381)(206.242724,74.87470998)
\curveto(206.28272952,74.92470371)(206.32272948,74.97970365)(206.362724,75.03970998)
\lineto(206.482724,75.21970998)
\curveto(206.51272929,75.27970335)(206.54272926,75.3297033)(206.572724,75.36970998)
\curveto(206.81272899,75.69970293)(207.12272868,75.94970268)(207.502724,76.11970998)
\curveto(207.58272822,76.15970247)(207.66772814,76.18970244)(207.757724,76.20970998)
\curveto(207.84772796,76.23970239)(207.93772787,76.26470237)(208.027724,76.28470998)
\curveto(208.07772773,76.29470234)(208.13272767,76.30470233)(208.192724,76.31470998)
\lineto(208.342724,76.34470998)
\curveto(208.4027274,76.35470228)(208.46772734,76.35470228)(208.537724,76.34470998)
\curveto(208.59772721,76.3347023)(208.65772715,76.33970229)(208.717724,76.35970998)
\moveto(203.677724,70.97470998)
\curveto(203.78773202,70.94470769)(203.92773188,70.93970769)(204.097724,70.95970998)
\curveto(204.25773155,70.97970765)(204.38273142,71.00470763)(204.472724,71.03470998)
\curveto(204.79273101,71.14470749)(205.03773077,71.29470734)(205.207724,71.48470998)
\curveto(205.36773044,71.67470696)(205.49773031,71.93970669)(205.597724,72.27970998)
\curveto(205.62773018,72.40970622)(205.65273015,72.57470606)(205.672724,72.77470998)
\curveto(205.68273012,72.97470566)(205.66773014,73.14470549)(205.627724,73.28470998)
\curveto(205.54773026,73.57470506)(205.43773037,73.81470482)(205.297724,74.00470998)
\curveto(205.14773066,74.20470443)(204.94773086,74.35970427)(204.697724,74.46970998)
\curveto(204.64773116,74.48970414)(204.6027312,74.49970413)(204.562724,74.49970998)
\curveto(204.52273128,74.50970412)(204.47773133,74.52470411)(204.427724,74.54470998)
\curveto(204.31773149,74.57470406)(204.17773163,74.59470404)(204.007724,74.60470998)
\curveto(203.83773197,74.61470402)(203.69273211,74.60470403)(203.572724,74.57470998)
\curveto(203.48273232,74.55470408)(203.39773241,74.5297041)(203.317724,74.49970998)
\curveto(203.23773257,74.47970415)(203.15773265,74.44470419)(203.077724,74.39470998)
\curveto(202.807733,74.22470441)(202.61273319,73.99970463)(202.492724,73.71970998)
\curveto(202.37273343,73.44970518)(202.31273349,73.08970554)(202.312724,72.63970998)
\curveto(202.33273347,72.61970601)(202.33773347,72.58970604)(202.327724,72.54970998)
\curveto(202.31773349,72.50970612)(202.31773349,72.47470616)(202.327724,72.44470998)
\curveto(202.34773346,72.39470624)(202.36273344,72.33970629)(202.372724,72.27970998)
\curveto(202.37273343,72.2297064)(202.38273342,72.17970645)(202.402724,72.12970998)
\curveto(202.49273331,71.88970674)(202.6077332,71.67970695)(202.747724,71.49970998)
\curveto(202.87773293,71.31970731)(203.05773275,71.17970745)(203.287724,71.07970998)
\curveto(203.34773246,71.05970757)(203.41273239,71.03970759)(203.482724,71.01970998)
\curveto(203.54273226,71.00970762)(203.6077322,70.99470764)(203.677724,70.97470998)
\moveto(209.212724,74.99470998)
\curveto(209.02272678,75.04470359)(208.81772699,75.04970358)(208.597724,75.00970998)
\curveto(208.37772743,74.97970365)(208.19772761,74.9347037)(208.057724,74.87470998)
\curveto(207.68772812,74.70470393)(207.38272842,74.44470419)(207.142724,74.09470998)
\curveto(206.9027289,73.75470488)(206.78272902,73.31970531)(206.782724,72.78970998)
\curveto(206.802729,72.75970587)(206.807729,72.71970591)(206.797724,72.66970998)
\curveto(206.77772903,72.61970601)(206.77272903,72.57970605)(206.782724,72.54970998)
\lineto(206.842724,72.27970998)
\curveto(206.85272895,72.19970643)(206.86772894,72.11970651)(206.887724,72.03970998)
\curveto(206.99772881,71.73970689)(207.14272866,71.47470716)(207.322724,71.24470998)
\curveto(207.5027283,71.02470761)(207.73272807,70.85470778)(208.012724,70.73470998)
\curveto(208.09272771,70.70470793)(208.17272763,70.67970795)(208.252724,70.65970998)
\curveto(208.33272747,70.63970799)(208.41772739,70.61970801)(208.507724,70.59970998)
\curveto(208.62772718,70.56970806)(208.77772703,70.55970807)(208.957724,70.56970998)
\curveto(209.13772667,70.58970804)(209.27772653,70.61470802)(209.377724,70.64470998)
\curveto(209.42772638,70.66470797)(209.47272633,70.67470796)(209.512724,70.67470998)
\curveto(209.54272626,70.68470795)(209.58272622,70.69970793)(209.632724,70.71970998)
\curveto(209.85272595,70.81970781)(210.05272575,70.94970768)(210.232724,71.10970998)
\curveto(210.41272539,71.27970735)(210.54772526,71.47470716)(210.637724,71.69470998)
\curveto(210.67772513,71.76470687)(210.71272509,71.85970677)(210.742724,71.97970998)
\curveto(210.83272497,72.19970643)(210.87772493,72.45470618)(210.877724,72.74470998)
\lineto(210.877724,73.02970998)
\curveto(210.85772495,73.1297055)(210.84272496,73.22470541)(210.832724,73.31470998)
\curveto(210.82272498,73.40470523)(210.802725,73.49470514)(210.772724,73.58470998)
\curveto(210.69272511,73.84470479)(210.56272524,74.08470455)(210.382724,74.30470998)
\curveto(210.19272561,74.5347041)(209.97772583,74.70470393)(209.737724,74.81470998)
\curveto(209.65772615,74.85470378)(209.57772623,74.88470375)(209.497724,74.90470998)
\curveto(209.4077264,74.9347037)(209.31272649,74.96470367)(209.212724,74.99470998)
}
}
{
\newrgbcolor{curcolor}{0 0 0}
\pscustom[linestyle=none,fillstyle=solid,fillcolor=curcolor]
{
\newpath
\moveto(210.157724,78.63431936)
\lineto(210.157724,79.26431936)
\lineto(210.157724,79.45931936)
\curveto(210.15772565,79.52931683)(210.16772564,79.58931677)(210.187724,79.63931936)
\curveto(210.22772558,79.70931665)(210.26772554,79.7593166)(210.307724,79.78931936)
\curveto(210.35772545,79.82931653)(210.42272538,79.84931651)(210.502724,79.84931936)
\curveto(210.58272522,79.8593165)(210.66772514,79.86431649)(210.757724,79.86431936)
\lineto(211.477724,79.86431936)
\curveto(211.95772385,79.86431649)(212.36772344,79.80431655)(212.707724,79.68431936)
\curveto(213.04772276,79.56431679)(213.32272248,79.36931699)(213.532724,79.09931936)
\curveto(213.58272222,79.02931733)(213.62772218,78.9593174)(213.667724,78.88931936)
\curveto(213.71772209,78.82931753)(213.76272204,78.7543176)(213.802724,78.66431936)
\curveto(213.81272199,78.64431771)(213.82272198,78.61431774)(213.832724,78.57431936)
\curveto(213.85272195,78.53431782)(213.85772195,78.48931787)(213.847724,78.43931936)
\curveto(213.81772199,78.34931801)(213.74272206,78.29431806)(213.622724,78.27431936)
\curveto(213.51272229,78.2543181)(213.41772239,78.26931809)(213.337724,78.31931936)
\curveto(213.26772254,78.34931801)(213.2027226,78.39431796)(213.142724,78.45431936)
\curveto(213.09272271,78.52431783)(213.04272276,78.58931777)(212.992724,78.64931936)
\curveto(212.94272286,78.71931764)(212.86772294,78.77931758)(212.767724,78.82931936)
\curveto(212.67772313,78.88931747)(212.58772322,78.93931742)(212.497724,78.97931936)
\curveto(212.46772334,78.99931736)(212.4077234,79.02431733)(212.317724,79.05431936)
\curveto(212.23772357,79.08431727)(212.16772364,79.08931727)(212.107724,79.06931936)
\curveto(211.96772384,79.03931732)(211.87772393,78.97931738)(211.837724,78.88931936)
\curveto(211.807724,78.80931755)(211.79272401,78.71931764)(211.792724,78.61931936)
\curveto(211.79272401,78.51931784)(211.76772404,78.43431792)(211.717724,78.36431936)
\curveto(211.64772416,78.27431808)(211.5077243,78.22931813)(211.297724,78.22931936)
\lineto(210.742724,78.22931936)
\lineto(210.517724,78.22931936)
\curveto(210.43772537,78.23931812)(210.37272543,78.2593181)(210.322724,78.28931936)
\curveto(210.24272556,78.34931801)(210.19772561,78.41931794)(210.187724,78.49931936)
\curveto(210.17772563,78.51931784)(210.17272563,78.53931782)(210.172724,78.55931936)
\curveto(210.17272563,78.58931777)(210.16772564,78.61431774)(210.157724,78.63431936)
}
}
{
\newrgbcolor{curcolor}{0 0 0}
\pscustom[linestyle=none,fillstyle=solid,fillcolor=curcolor]
{
}
}
{
\newrgbcolor{curcolor}{0 0 0}
\pscustom[linestyle=none,fillstyle=solid,fillcolor=curcolor]
{
\newpath
\moveto(201.187724,89.26463186)
\curveto(201.17773463,89.95462722)(201.29773451,90.55462662)(201.547724,91.06463186)
\curveto(201.79773401,91.58462559)(202.13273367,91.9796252)(202.552724,92.24963186)
\curveto(202.63273317,92.29962488)(202.72273308,92.34462483)(202.822724,92.38463186)
\curveto(202.91273289,92.42462475)(203.0077328,92.46962471)(203.107724,92.51963186)
\curveto(203.2077326,92.55962462)(203.3077325,92.58962459)(203.407724,92.60963186)
\curveto(203.5077323,92.62962455)(203.61273219,92.64962453)(203.722724,92.66963186)
\curveto(203.77273203,92.68962449)(203.81773199,92.69462448)(203.857724,92.68463186)
\curveto(203.89773191,92.6746245)(203.94273186,92.6796245)(203.992724,92.69963186)
\curveto(204.04273176,92.70962447)(204.12773168,92.71462446)(204.247724,92.71463186)
\curveto(204.35773145,92.71462446)(204.44273136,92.70962447)(204.502724,92.69963186)
\curveto(204.56273124,92.6796245)(204.62273118,92.66962451)(204.682724,92.66963186)
\curveto(204.74273106,92.6796245)(204.802731,92.6746245)(204.862724,92.65463186)
\curveto(205.0027308,92.61462456)(205.13773067,92.5796246)(205.267724,92.54963186)
\curveto(205.39773041,92.51962466)(205.52273028,92.4796247)(205.642724,92.42963186)
\curveto(205.78273002,92.36962481)(205.9077299,92.29962488)(206.017724,92.21963186)
\curveto(206.12772968,92.14962503)(206.23772957,92.0746251)(206.347724,91.99463186)
\lineto(206.407724,91.93463186)
\curveto(206.42772938,91.92462525)(206.44772936,91.90962527)(206.467724,91.88963186)
\curveto(206.62772918,91.76962541)(206.77272903,91.63462554)(206.902724,91.48463186)
\curveto(207.03272877,91.33462584)(207.15772865,91.174626)(207.277724,91.00463186)
\curveto(207.49772831,90.69462648)(207.7027281,90.39962678)(207.892724,90.11963186)
\curveto(208.03272777,89.88962729)(208.16772764,89.65962752)(208.297724,89.42963186)
\curveto(208.42772738,89.20962797)(208.56272724,88.98962819)(208.702724,88.76963186)
\curveto(208.87272693,88.51962866)(209.05272675,88.2796289)(209.242724,88.04963186)
\curveto(209.43272637,87.82962935)(209.65772615,87.63962954)(209.917724,87.47963186)
\curveto(209.97772583,87.43962974)(210.03772577,87.40462977)(210.097724,87.37463186)
\curveto(210.14772566,87.34462983)(210.21272559,87.31462986)(210.292724,87.28463186)
\curveto(210.36272544,87.26462991)(210.42272538,87.25962992)(210.472724,87.26963186)
\curveto(210.54272526,87.28962989)(210.59772521,87.32462985)(210.637724,87.37463186)
\curveto(210.66772514,87.42462975)(210.68772512,87.48462969)(210.697724,87.55463186)
\lineto(210.697724,87.79463186)
\lineto(210.697724,88.54463186)
\lineto(210.697724,91.34963186)
\lineto(210.697724,92.00963186)
\curveto(210.69772511,92.09962508)(210.7027251,92.18462499)(210.712724,92.26463186)
\curveto(210.71272509,92.34462483)(210.73272507,92.40962477)(210.772724,92.45963186)
\curveto(210.81272499,92.50962467)(210.88772492,92.54962463)(210.997724,92.57963186)
\curveto(211.09772471,92.61962456)(211.19772461,92.62962455)(211.297724,92.60963186)
\lineto(211.432724,92.60963186)
\curveto(211.5027243,92.58962459)(211.56272424,92.56962461)(211.612724,92.54963186)
\curveto(211.66272414,92.52962465)(211.7027241,92.49462468)(211.732724,92.44463186)
\curveto(211.77272403,92.39462478)(211.79272401,92.32462485)(211.792724,92.23463186)
\lineto(211.792724,91.96463186)
\lineto(211.792724,91.06463186)
\lineto(211.792724,87.55463186)
\lineto(211.792724,86.48963186)
\curveto(211.79272401,86.40963077)(211.79772401,86.31963086)(211.807724,86.21963186)
\curveto(211.807724,86.11963106)(211.79772401,86.03463114)(211.777724,85.96463186)
\curveto(211.7077241,85.75463142)(211.52772428,85.68963149)(211.237724,85.76963186)
\curveto(211.19772461,85.7796314)(211.16272464,85.7796314)(211.132724,85.76963186)
\curveto(211.09272471,85.76963141)(211.04772476,85.7796314)(210.997724,85.79963186)
\curveto(210.91772489,85.81963136)(210.83272497,85.83963134)(210.742724,85.85963186)
\curveto(210.65272515,85.8796313)(210.56772524,85.90463127)(210.487724,85.93463186)
\curveto(209.99772581,86.09463108)(209.58272622,86.29463088)(209.242724,86.53463186)
\curveto(208.99272681,86.71463046)(208.76772704,86.91963026)(208.567724,87.14963186)
\curveto(208.35772745,87.3796298)(208.16272764,87.61962956)(207.982724,87.86963186)
\curveto(207.802728,88.12962905)(207.63272817,88.39462878)(207.472724,88.66463186)
\curveto(207.3027285,88.94462823)(207.12772868,89.21462796)(206.947724,89.47463186)
\curveto(206.86772894,89.58462759)(206.79272901,89.68962749)(206.722724,89.78963186)
\curveto(206.65272915,89.89962728)(206.57772923,90.00962717)(206.497724,90.11963186)
\curveto(206.46772934,90.15962702)(206.43772937,90.19462698)(206.407724,90.22463186)
\curveto(206.36772944,90.26462691)(206.33772947,90.30462687)(206.317724,90.34463186)
\curveto(206.2077296,90.48462669)(206.08272972,90.60962657)(205.942724,90.71963186)
\curveto(205.91272989,90.73962644)(205.88772992,90.76462641)(205.867724,90.79463186)
\curveto(205.83772997,90.82462635)(205.80773,90.84962633)(205.777724,90.86963186)
\curveto(205.67773013,90.94962623)(205.57773023,91.01462616)(205.477724,91.06463186)
\curveto(205.37773043,91.12462605)(205.26773054,91.179626)(205.147724,91.22963186)
\curveto(205.07773073,91.25962592)(205.0027308,91.2796259)(204.922724,91.28963186)
\lineto(204.682724,91.34963186)
\lineto(204.592724,91.34963186)
\curveto(204.56273124,91.35962582)(204.53273127,91.36462581)(204.502724,91.36463186)
\curveto(204.43273137,91.38462579)(204.33773147,91.38962579)(204.217724,91.37963186)
\curveto(204.08773172,91.3796258)(203.98773182,91.36962581)(203.917724,91.34963186)
\curveto(203.83773197,91.32962585)(203.76273204,91.30962587)(203.692724,91.28963186)
\curveto(203.61273219,91.2796259)(203.53273227,91.25962592)(203.452724,91.22963186)
\curveto(203.21273259,91.11962606)(203.01273279,90.96962621)(202.852724,90.77963186)
\curveto(202.68273312,90.59962658)(202.54273326,90.3796268)(202.432724,90.11963186)
\curveto(202.41273339,90.04962713)(202.39773341,89.9796272)(202.387724,89.90963186)
\curveto(202.36773344,89.83962734)(202.34773346,89.76462741)(202.327724,89.68463186)
\curveto(202.3077335,89.60462757)(202.29773351,89.49462768)(202.297724,89.35463186)
\curveto(202.29773351,89.22462795)(202.3077335,89.11962806)(202.327724,89.03963186)
\curveto(202.33773347,88.9796282)(202.34273346,88.92462825)(202.342724,88.87463186)
\curveto(202.34273346,88.82462835)(202.35273345,88.7746284)(202.372724,88.72463186)
\curveto(202.41273339,88.62462855)(202.45273335,88.52962865)(202.492724,88.43963186)
\curveto(202.53273327,88.35962882)(202.57773323,88.2796289)(202.627724,88.19963186)
\curveto(202.64773316,88.16962901)(202.67273313,88.13962904)(202.702724,88.10963186)
\curveto(202.73273307,88.08962909)(202.75773305,88.06462911)(202.777724,88.03463186)
\lineto(202.852724,87.95963186)
\curveto(202.87273293,87.92962925)(202.89273291,87.90462927)(202.912724,87.88463186)
\lineto(203.122724,87.73463186)
\curveto(203.18273262,87.69462948)(203.24773256,87.64962953)(203.317724,87.59963186)
\curveto(203.4077324,87.53962964)(203.51273229,87.48962969)(203.632724,87.44963186)
\curveto(203.74273206,87.41962976)(203.85273195,87.38462979)(203.962724,87.34463186)
\curveto(204.07273173,87.30462987)(204.21773159,87.2796299)(204.397724,87.26963186)
\curveto(204.56773124,87.25962992)(204.69273111,87.22962995)(204.772724,87.17963186)
\curveto(204.85273095,87.12963005)(204.89773091,87.05463012)(204.907724,86.95463186)
\curveto(204.91773089,86.85463032)(204.92273088,86.74463043)(204.922724,86.62463186)
\curveto(204.92273088,86.58463059)(204.92773088,86.54463063)(204.937724,86.50463186)
\curveto(204.93773087,86.46463071)(204.93273087,86.42963075)(204.922724,86.39963186)
\curveto(204.9027309,86.34963083)(204.89273091,86.29963088)(204.892724,86.24963186)
\curveto(204.89273091,86.20963097)(204.88273092,86.16963101)(204.862724,86.12963186)
\curveto(204.802731,86.03963114)(204.66773114,85.99463118)(204.457724,85.99463186)
\lineto(204.337724,85.99463186)
\curveto(204.27773153,86.00463117)(204.21773159,86.00963117)(204.157724,86.00963186)
\curveto(204.08773172,86.01963116)(204.02273178,86.02963115)(203.962724,86.03963186)
\curveto(203.85273195,86.05963112)(203.75273205,86.0796311)(203.662724,86.09963186)
\curveto(203.56273224,86.11963106)(203.46773234,86.14963103)(203.377724,86.18963186)
\curveto(203.3077325,86.20963097)(203.24773256,86.22963095)(203.197724,86.24963186)
\lineto(203.017724,86.30963186)
\curveto(202.75773305,86.42963075)(202.51273329,86.58463059)(202.282724,86.77463186)
\curveto(202.05273375,86.9746302)(201.86773394,87.18962999)(201.727724,87.41963186)
\curveto(201.64773416,87.52962965)(201.58273422,87.64462953)(201.532724,87.76463186)
\lineto(201.382724,88.15463186)
\curveto(201.33273447,88.26462891)(201.3027345,88.3796288)(201.292724,88.49963186)
\curveto(201.27273453,88.61962856)(201.24773456,88.74462843)(201.217724,88.87463186)
\curveto(201.21773459,88.94462823)(201.21773459,89.00962817)(201.217724,89.06963186)
\curveto(201.2077346,89.12962805)(201.19773461,89.19462798)(201.187724,89.26463186)
}
}
{
\newrgbcolor{curcolor}{0 0 0}
\pscustom[linestyle=none,fillstyle=solid,fillcolor=curcolor]
{
\newpath
\moveto(206.707724,101.36424123)
\lineto(206.962724,101.36424123)
\curveto(207.04272876,101.37423353)(207.11772869,101.36923353)(207.187724,101.34924123)
\lineto(207.427724,101.34924123)
\lineto(207.592724,101.34924123)
\curveto(207.69272811,101.32923357)(207.79772801,101.31923358)(207.907724,101.31924123)
\curveto(208.0077278,101.31923358)(208.1077277,101.30923359)(208.207724,101.28924123)
\lineto(208.357724,101.28924123)
\curveto(208.49772731,101.25923364)(208.63772717,101.23923366)(208.777724,101.22924123)
\curveto(208.9077269,101.21923368)(209.03772677,101.19423371)(209.167724,101.15424123)
\curveto(209.24772656,101.13423377)(209.33272647,101.11423379)(209.422724,101.09424123)
\lineto(209.662724,101.03424123)
\lineto(209.962724,100.91424123)
\curveto(210.05272575,100.88423402)(210.14272566,100.84923405)(210.232724,100.80924123)
\curveto(210.45272535,100.70923419)(210.66772514,100.57423433)(210.877724,100.40424123)
\curveto(211.08772472,100.24423466)(211.25772455,100.06923483)(211.387724,99.87924123)
\curveto(211.42772438,99.82923507)(211.46772434,99.76923513)(211.507724,99.69924123)
\curveto(211.53772427,99.63923526)(211.57272423,99.57923532)(211.612724,99.51924123)
\curveto(211.66272414,99.43923546)(211.7027241,99.34423556)(211.732724,99.23424123)
\curveto(211.76272404,99.12423578)(211.79272401,99.01923588)(211.822724,98.91924123)
\curveto(211.86272394,98.80923609)(211.88772392,98.6992362)(211.897724,98.58924123)
\curveto(211.9077239,98.47923642)(211.92272388,98.36423654)(211.942724,98.24424123)
\curveto(211.95272385,98.2042367)(211.95272385,98.15923674)(211.942724,98.10924123)
\curveto(211.94272386,98.06923683)(211.94772386,98.02923687)(211.957724,97.98924123)
\curveto(211.96772384,97.94923695)(211.97272383,97.89423701)(211.972724,97.82424123)
\curveto(211.97272383,97.75423715)(211.96772384,97.7042372)(211.957724,97.67424123)
\curveto(211.93772387,97.62423728)(211.93272387,97.57923732)(211.942724,97.53924123)
\curveto(211.95272385,97.4992374)(211.95272385,97.46423744)(211.942724,97.43424123)
\lineto(211.942724,97.34424123)
\curveto(211.92272388,97.28423762)(211.9077239,97.21923768)(211.897724,97.14924123)
\curveto(211.89772391,97.08923781)(211.89272391,97.02423788)(211.882724,96.95424123)
\curveto(211.83272397,96.78423812)(211.78272402,96.62423828)(211.732724,96.47424123)
\curveto(211.68272412,96.32423858)(211.61772419,96.17923872)(211.537724,96.03924123)
\curveto(211.49772431,95.98923891)(211.46772434,95.93423897)(211.447724,95.87424123)
\curveto(211.41772439,95.82423908)(211.38272442,95.77423913)(211.342724,95.72424123)
\curveto(211.16272464,95.48423942)(210.94272486,95.28423962)(210.682724,95.12424123)
\curveto(210.42272538,94.96423994)(210.13772567,94.82424008)(209.827724,94.70424123)
\curveto(209.68772612,94.64424026)(209.54772626,94.5992403)(209.407724,94.56924123)
\curveto(209.25772655,94.53924036)(209.1027267,94.5042404)(208.942724,94.46424123)
\curveto(208.83272697,94.44424046)(208.72272708,94.42924047)(208.612724,94.41924123)
\curveto(208.5027273,94.40924049)(208.39272741,94.39424051)(208.282724,94.37424123)
\curveto(208.24272756,94.36424054)(208.2027276,94.35924054)(208.162724,94.35924123)
\curveto(208.12272768,94.36924053)(208.08272772,94.36924053)(208.042724,94.35924123)
\curveto(207.99272781,94.34924055)(207.94272786,94.34424056)(207.892724,94.34424123)
\lineto(207.727724,94.34424123)
\curveto(207.67772813,94.32424058)(207.62772818,94.31924058)(207.577724,94.32924123)
\curveto(207.51772829,94.33924056)(207.46272834,94.33924056)(207.412724,94.32924123)
\curveto(207.37272843,94.31924058)(207.32772848,94.31924058)(207.277724,94.32924123)
\curveto(207.22772858,94.33924056)(207.17772863,94.33424057)(207.127724,94.31424123)
\curveto(207.05772875,94.29424061)(206.98272882,94.28924061)(206.902724,94.29924123)
\curveto(206.81272899,94.30924059)(206.72772908,94.31424059)(206.647724,94.31424123)
\curveto(206.55772925,94.31424059)(206.45772935,94.30924059)(206.347724,94.29924123)
\curveto(206.22772958,94.28924061)(206.12772968,94.29424061)(206.047724,94.31424123)
\lineto(205.762724,94.31424123)
\lineto(205.132724,94.35924123)
\curveto(205.03273077,94.36924053)(204.93773087,94.37924052)(204.847724,94.38924123)
\lineto(204.547724,94.41924123)
\curveto(204.49773131,94.43924046)(204.44773136,94.44424046)(204.397724,94.43424123)
\curveto(204.33773147,94.43424047)(204.28273152,94.44424046)(204.232724,94.46424123)
\curveto(204.06273174,94.51424039)(203.89773191,94.55424035)(203.737724,94.58424123)
\curveto(203.56773224,94.61424029)(203.4077324,94.66424024)(203.257724,94.73424123)
\curveto(202.79773301,94.92423998)(202.42273338,95.14423976)(202.132724,95.39424123)
\curveto(201.84273396,95.65423925)(201.59773421,96.01423889)(201.397724,96.47424123)
\curveto(201.34773446,96.6042383)(201.31273449,96.73423817)(201.292724,96.86424123)
\curveto(201.27273453,97.0042379)(201.24773456,97.14423776)(201.217724,97.28424123)
\curveto(201.2077346,97.35423755)(201.2027346,97.41923748)(201.202724,97.47924123)
\curveto(201.2027346,97.53923736)(201.19773461,97.6042373)(201.187724,97.67424123)
\curveto(201.16773464,98.5042364)(201.31773449,99.17423573)(201.637724,99.68424123)
\curveto(201.94773386,100.19423471)(202.38773342,100.57423433)(202.957724,100.82424123)
\curveto(203.07773273,100.87423403)(203.2027326,100.91923398)(203.332724,100.95924123)
\curveto(203.46273234,100.9992339)(203.59773221,101.04423386)(203.737724,101.09424123)
\curveto(203.81773199,101.11423379)(203.9027319,101.12923377)(203.992724,101.13924123)
\lineto(204.232724,101.19924123)
\curveto(204.34273146,101.22923367)(204.45273135,101.24423366)(204.562724,101.24424123)
\curveto(204.67273113,101.25423365)(204.78273102,101.26923363)(204.892724,101.28924123)
\curveto(204.94273086,101.30923359)(204.98773082,101.31423359)(205.027724,101.30424123)
\curveto(205.06773074,101.3042336)(205.1077307,101.30923359)(205.147724,101.31924123)
\curveto(205.19773061,101.32923357)(205.25273055,101.32923357)(205.312724,101.31924123)
\curveto(205.36273044,101.31923358)(205.41273039,101.32423358)(205.462724,101.33424123)
\lineto(205.597724,101.33424123)
\curveto(205.65773015,101.35423355)(205.72773008,101.35423355)(205.807724,101.33424123)
\curveto(205.87772993,101.32423358)(205.94272986,101.32923357)(206.002724,101.34924123)
\curveto(206.03272977,101.35923354)(206.07272973,101.36423354)(206.122724,101.36424123)
\lineto(206.242724,101.36424123)
\lineto(206.707724,101.36424123)
\moveto(209.032724,99.81924123)
\curveto(208.71272709,99.91923498)(208.34772746,99.97923492)(207.937724,99.99924123)
\curveto(207.52772828,100.01923488)(207.11772869,100.02923487)(206.707724,100.02924123)
\curveto(206.27772953,100.02923487)(205.85772995,100.01923488)(205.447724,99.99924123)
\curveto(205.03773077,99.97923492)(204.65273115,99.93423497)(204.292724,99.86424123)
\curveto(203.93273187,99.79423511)(203.61273219,99.68423522)(203.332724,99.53424123)
\curveto(203.04273276,99.39423551)(202.807733,99.1992357)(202.627724,98.94924123)
\curveto(202.51773329,98.78923611)(202.43773337,98.60923629)(202.387724,98.40924123)
\curveto(202.32773348,98.20923669)(202.29773351,97.96423694)(202.297724,97.67424123)
\curveto(202.31773349,97.65423725)(202.32773348,97.61923728)(202.327724,97.56924123)
\curveto(202.31773349,97.51923738)(202.31773349,97.47923742)(202.327724,97.44924123)
\curveto(202.34773346,97.36923753)(202.36773344,97.29423761)(202.387724,97.22424123)
\curveto(202.39773341,97.16423774)(202.41773339,97.0992378)(202.447724,97.02924123)
\curveto(202.56773324,96.75923814)(202.73773307,96.53923836)(202.957724,96.36924123)
\curveto(203.16773264,96.20923869)(203.41273239,96.07423883)(203.692724,95.96424123)
\curveto(203.802732,95.91423899)(203.92273188,95.87423903)(204.052724,95.84424123)
\curveto(204.17273163,95.82423908)(204.29773151,95.7992391)(204.427724,95.76924123)
\curveto(204.47773133,95.74923915)(204.53273127,95.73923916)(204.592724,95.73924123)
\curveto(204.64273116,95.73923916)(204.69273111,95.73423917)(204.742724,95.72424123)
\curveto(204.83273097,95.71423919)(204.92773088,95.7042392)(205.027724,95.69424123)
\curveto(205.11773069,95.68423922)(205.21273059,95.67423923)(205.312724,95.66424123)
\curveto(205.39273041,95.66423924)(205.47773033,95.65923924)(205.567724,95.64924123)
\lineto(205.807724,95.64924123)
\lineto(205.987724,95.64924123)
\curveto(206.01772979,95.63923926)(206.05272975,95.63423927)(206.092724,95.63424123)
\lineto(206.227724,95.63424123)
\lineto(206.677724,95.63424123)
\curveto(206.75772905,95.63423927)(206.84272896,95.62923927)(206.932724,95.61924123)
\curveto(207.01272879,95.61923928)(207.08772872,95.62923927)(207.157724,95.64924123)
\lineto(207.427724,95.64924123)
\curveto(207.44772836,95.64923925)(207.47772833,95.64423926)(207.517724,95.63424123)
\curveto(207.54772826,95.63423927)(207.57272823,95.63923926)(207.592724,95.64924123)
\curveto(207.69272811,95.65923924)(207.79272801,95.66423924)(207.892724,95.66424123)
\curveto(207.98272782,95.67423923)(208.08272772,95.68423922)(208.192724,95.69424123)
\curveto(208.31272749,95.72423918)(208.43772737,95.73923916)(208.567724,95.73924123)
\curveto(208.68772712,95.74923915)(208.802727,95.77423913)(208.912724,95.81424123)
\curveto(209.21272659,95.89423901)(209.47772633,95.97923892)(209.707724,96.06924123)
\curveto(209.93772587,96.16923873)(210.15272565,96.31423859)(210.352724,96.50424123)
\curveto(210.55272525,96.71423819)(210.7027251,96.97923792)(210.802724,97.29924123)
\curveto(210.82272498,97.33923756)(210.83272497,97.37423753)(210.832724,97.40424123)
\curveto(210.82272498,97.44423746)(210.82772498,97.48923741)(210.847724,97.53924123)
\curveto(210.85772495,97.57923732)(210.86772494,97.64923725)(210.877724,97.74924123)
\curveto(210.88772492,97.85923704)(210.88272492,97.94423696)(210.862724,98.00424123)
\curveto(210.84272496,98.07423683)(210.83272497,98.14423676)(210.832724,98.21424123)
\curveto(210.82272498,98.28423662)(210.807725,98.34923655)(210.787724,98.40924123)
\curveto(210.72772508,98.60923629)(210.64272516,98.78923611)(210.532724,98.94924123)
\curveto(210.51272529,98.97923592)(210.49272531,99.0042359)(210.472724,99.02424123)
\lineto(210.412724,99.08424123)
\curveto(210.39272541,99.12423578)(210.35272545,99.17423573)(210.292724,99.23424123)
\curveto(210.15272565,99.33423557)(210.02272578,99.41923548)(209.902724,99.48924123)
\curveto(209.78272602,99.55923534)(209.63772617,99.62923527)(209.467724,99.69924123)
\curveto(209.39772641,99.72923517)(209.32772648,99.74923515)(209.257724,99.75924123)
\curveto(209.18772662,99.77923512)(209.11272669,99.7992351)(209.032724,99.81924123)
}
}
{
\newrgbcolor{curcolor}{0 0 0}
\pscustom[linestyle=none,fillstyle=solid,fillcolor=curcolor]
{
\newpath
\moveto(201.187724,106.77385061)
\curveto(201.18773462,106.87384575)(201.19773461,106.96884566)(201.217724,107.05885061)
\curveto(201.22773458,107.14884548)(201.25773455,107.21384541)(201.307724,107.25385061)
\curveto(201.38773442,107.31384531)(201.49273431,107.34384528)(201.622724,107.34385061)
\lineto(202.012724,107.34385061)
\lineto(203.512724,107.34385061)
\lineto(209.902724,107.34385061)
\lineto(211.072724,107.34385061)
\lineto(211.387724,107.34385061)
\curveto(211.48772432,107.35384527)(211.56772424,107.33884529)(211.627724,107.29885061)
\curveto(211.7077241,107.24884538)(211.75772405,107.17384545)(211.777724,107.07385061)
\curveto(211.78772402,106.98384564)(211.79272401,106.87384575)(211.792724,106.74385061)
\lineto(211.792724,106.51885061)
\curveto(211.77272403,106.43884619)(211.75772405,106.36884626)(211.747724,106.30885061)
\curveto(211.72772408,106.24884638)(211.68772412,106.19884643)(211.627724,106.15885061)
\curveto(211.56772424,106.11884651)(211.49272431,106.09884653)(211.402724,106.09885061)
\lineto(211.102724,106.09885061)
\lineto(210.007724,106.09885061)
\lineto(204.667724,106.09885061)
\curveto(204.57773123,106.07884655)(204.5027313,106.06384656)(204.442724,106.05385061)
\curveto(204.37273143,106.05384657)(204.31273149,106.0238466)(204.262724,105.96385061)
\curveto(204.21273159,105.89384673)(204.18773162,105.80384682)(204.187724,105.69385061)
\curveto(204.17773163,105.59384703)(204.17273163,105.48384714)(204.172724,105.36385061)
\lineto(204.172724,104.22385061)
\lineto(204.172724,103.72885061)
\curveto(204.16273164,103.56884906)(204.1027317,103.45884917)(203.992724,103.39885061)
\curveto(203.96273184,103.37884925)(203.93273187,103.36884926)(203.902724,103.36885061)
\curveto(203.86273194,103.36884926)(203.81773199,103.36384926)(203.767724,103.35385061)
\curveto(203.64773216,103.33384929)(203.53773227,103.33884929)(203.437724,103.36885061)
\curveto(203.33773247,103.40884922)(203.26773254,103.46384916)(203.227724,103.53385061)
\curveto(203.17773263,103.61384901)(203.15273265,103.73384889)(203.152724,103.89385061)
\curveto(203.15273265,104.05384857)(203.13773267,104.18884844)(203.107724,104.29885061)
\curveto(203.09773271,104.34884828)(203.09273271,104.40384822)(203.092724,104.46385061)
\curveto(203.08273272,104.5238481)(203.06773274,104.58384804)(203.047724,104.64385061)
\curveto(202.99773281,104.79384783)(202.94773286,104.93884769)(202.897724,105.07885061)
\curveto(202.83773297,105.21884741)(202.76773304,105.35384727)(202.687724,105.48385061)
\curveto(202.59773321,105.623847)(202.49273331,105.74384688)(202.372724,105.84385061)
\curveto(202.25273355,105.94384668)(202.12273368,106.03884659)(201.982724,106.12885061)
\curveto(201.88273392,106.18884644)(201.77273403,106.23384639)(201.652724,106.26385061)
\curveto(201.53273427,106.30384632)(201.42773438,106.35384627)(201.337724,106.41385061)
\curveto(201.27773453,106.46384616)(201.23773457,106.53384609)(201.217724,106.62385061)
\curveto(201.2077346,106.64384598)(201.2027346,106.66884596)(201.202724,106.69885061)
\curveto(201.2027346,106.7288459)(201.19773461,106.75384587)(201.187724,106.77385061)
}
}
{
\newrgbcolor{curcolor}{0 0 0}
\pscustom[linestyle=none,fillstyle=solid,fillcolor=curcolor]
{
\newpath
\moveto(201.187724,115.12345998)
\curveto(201.18773462,115.22345513)(201.19773461,115.31845503)(201.217724,115.40845998)
\curveto(201.22773458,115.49845485)(201.25773455,115.56345479)(201.307724,115.60345998)
\curveto(201.38773442,115.66345469)(201.49273431,115.69345466)(201.622724,115.69345998)
\lineto(202.012724,115.69345998)
\lineto(203.512724,115.69345998)
\lineto(209.902724,115.69345998)
\lineto(211.072724,115.69345998)
\lineto(211.387724,115.69345998)
\curveto(211.48772432,115.70345465)(211.56772424,115.68845466)(211.627724,115.64845998)
\curveto(211.7077241,115.59845475)(211.75772405,115.52345483)(211.777724,115.42345998)
\curveto(211.78772402,115.33345502)(211.79272401,115.22345513)(211.792724,115.09345998)
\lineto(211.792724,114.86845998)
\curveto(211.77272403,114.78845556)(211.75772405,114.71845563)(211.747724,114.65845998)
\curveto(211.72772408,114.59845575)(211.68772412,114.5484558)(211.627724,114.50845998)
\curveto(211.56772424,114.46845588)(211.49272431,114.4484559)(211.402724,114.44845998)
\lineto(211.102724,114.44845998)
\lineto(210.007724,114.44845998)
\lineto(204.667724,114.44845998)
\curveto(204.57773123,114.42845592)(204.5027313,114.41345594)(204.442724,114.40345998)
\curveto(204.37273143,114.40345595)(204.31273149,114.37345598)(204.262724,114.31345998)
\curveto(204.21273159,114.24345611)(204.18773162,114.1534562)(204.187724,114.04345998)
\curveto(204.17773163,113.94345641)(204.17273163,113.83345652)(204.172724,113.71345998)
\lineto(204.172724,112.57345998)
\lineto(204.172724,112.07845998)
\curveto(204.16273164,111.91845843)(204.1027317,111.80845854)(203.992724,111.74845998)
\curveto(203.96273184,111.72845862)(203.93273187,111.71845863)(203.902724,111.71845998)
\curveto(203.86273194,111.71845863)(203.81773199,111.71345864)(203.767724,111.70345998)
\curveto(203.64773216,111.68345867)(203.53773227,111.68845866)(203.437724,111.71845998)
\curveto(203.33773247,111.75845859)(203.26773254,111.81345854)(203.227724,111.88345998)
\curveto(203.17773263,111.96345839)(203.15273265,112.08345827)(203.152724,112.24345998)
\curveto(203.15273265,112.40345795)(203.13773267,112.53845781)(203.107724,112.64845998)
\curveto(203.09773271,112.69845765)(203.09273271,112.7534576)(203.092724,112.81345998)
\curveto(203.08273272,112.87345748)(203.06773274,112.93345742)(203.047724,112.99345998)
\curveto(202.99773281,113.14345721)(202.94773286,113.28845706)(202.897724,113.42845998)
\curveto(202.83773297,113.56845678)(202.76773304,113.70345665)(202.687724,113.83345998)
\curveto(202.59773321,113.97345638)(202.49273331,114.09345626)(202.372724,114.19345998)
\curveto(202.25273355,114.29345606)(202.12273368,114.38845596)(201.982724,114.47845998)
\curveto(201.88273392,114.53845581)(201.77273403,114.58345577)(201.652724,114.61345998)
\curveto(201.53273427,114.6534557)(201.42773438,114.70345565)(201.337724,114.76345998)
\curveto(201.27773453,114.81345554)(201.23773457,114.88345547)(201.217724,114.97345998)
\curveto(201.2077346,114.99345536)(201.2027346,115.01845533)(201.202724,115.04845998)
\curveto(201.2027346,115.07845527)(201.19773461,115.10345525)(201.187724,115.12345998)
}
}
{
\newrgbcolor{curcolor}{0 0 0}
\pscustom[linestyle=none,fillstyle=solid,fillcolor=curcolor]
{
\newpath
\moveto(117.46738403,31.67142873)
\lineto(117.46738403,32.58642873)
\curveto(117.46739473,32.68642608)(117.46739473,32.78142599)(117.46738403,32.87142873)
\curveto(117.46739473,32.96142581)(117.48739471,33.03642573)(117.52738403,33.09642873)
\curveto(117.58739461,33.18642558)(117.66739453,33.24642552)(117.76738403,33.27642873)
\curveto(117.86739433,33.31642545)(117.97239422,33.36142541)(118.08238403,33.41142873)
\curveto(118.27239392,33.49142528)(118.46239373,33.56142521)(118.65238403,33.62142873)
\curveto(118.84239335,33.69142508)(119.03239316,33.766425)(119.22238403,33.84642873)
\curveto(119.40239279,33.91642485)(119.58739261,33.98142479)(119.77738403,34.04142873)
\curveto(119.95739224,34.10142467)(120.13739206,34.1714246)(120.31738403,34.25142873)
\curveto(120.45739174,34.31142446)(120.60239159,34.3664244)(120.75238403,34.41642873)
\curveto(120.90239129,34.4664243)(121.04739115,34.52142425)(121.18738403,34.58142873)
\curveto(121.63739056,34.76142401)(122.0923901,34.93142384)(122.55238403,35.09142873)
\curveto(123.00238919,35.25142352)(123.45238874,35.42142335)(123.90238403,35.60142873)
\curveto(123.95238824,35.62142315)(124.00238819,35.63642313)(124.05238403,35.64642873)
\lineto(124.20238403,35.70642873)
\curveto(124.42238777,35.79642297)(124.64738755,35.88142289)(124.87738403,35.96142873)
\curveto(125.0973871,36.04142273)(125.31738688,36.12642264)(125.53738403,36.21642873)
\curveto(125.62738657,36.25642251)(125.73738646,36.29642247)(125.86738403,36.33642873)
\curveto(125.98738621,36.37642239)(126.05738614,36.44142233)(126.07738403,36.53142873)
\curveto(126.08738611,36.5714222)(126.08738611,36.60142217)(126.07738403,36.62142873)
\lineto(126.01738403,36.68142873)
\curveto(125.96738623,36.73142204)(125.91238628,36.766422)(125.85238403,36.78642873)
\curveto(125.7923864,36.81642195)(125.72738647,36.84642192)(125.65738403,36.87642873)
\lineto(125.02738403,37.11642873)
\curveto(124.80738739,37.19642157)(124.5923876,37.27642149)(124.38238403,37.35642873)
\lineto(124.23238403,37.41642873)
\lineto(124.05238403,37.47642873)
\curveto(123.86238833,37.55642121)(123.67238852,37.62642114)(123.48238403,37.68642873)
\curveto(123.28238891,37.75642101)(123.08238911,37.83142094)(122.88238403,37.91142873)
\curveto(122.30238989,38.15142062)(121.71739048,38.3714204)(121.12738403,38.57142873)
\curveto(120.53739166,38.78141999)(119.95239224,39.00641976)(119.37238403,39.24642873)
\curveto(119.17239302,39.32641944)(118.96739323,39.40141937)(118.75738403,39.47142873)
\curveto(118.54739365,39.55141922)(118.34239385,39.63141914)(118.14238403,39.71142873)
\curveto(118.06239413,39.75141902)(117.96239423,39.78641898)(117.84238403,39.81642873)
\curveto(117.72239447,39.85641891)(117.63739456,39.91141886)(117.58738403,39.98142873)
\curveto(117.54739465,40.04141873)(117.51739468,40.11641865)(117.49738403,40.20642873)
\curveto(117.47739472,40.30641846)(117.46739473,40.41641835)(117.46738403,40.53642873)
\curveto(117.45739474,40.65641811)(117.45739474,40.77641799)(117.46738403,40.89642873)
\curveto(117.46739473,41.01641775)(117.46739473,41.12641764)(117.46738403,41.22642873)
\curveto(117.46739473,41.31641745)(117.46739473,41.40641736)(117.46738403,41.49642873)
\curveto(117.46739473,41.59641717)(117.48739471,41.6714171)(117.52738403,41.72142873)
\curveto(117.57739462,41.81141696)(117.66739453,41.86141691)(117.79738403,41.87142873)
\curveto(117.92739427,41.88141689)(118.06739413,41.88641688)(118.21738403,41.88642873)
\lineto(119.86738403,41.88642873)
\lineto(126.13738403,41.88642873)
\lineto(127.39738403,41.88642873)
\curveto(127.50738469,41.88641688)(127.61738458,41.88641688)(127.72738403,41.88642873)
\curveto(127.83738436,41.89641687)(127.92238427,41.87641689)(127.98238403,41.82642873)
\curveto(128.04238415,41.79641697)(128.08238411,41.75141702)(128.10238403,41.69142873)
\curveto(128.11238408,41.63141714)(128.12738407,41.56141721)(128.14738403,41.48142873)
\lineto(128.14738403,41.24142873)
\lineto(128.14738403,40.88142873)
\curveto(128.13738406,40.771418)(128.0923841,40.69141808)(128.01238403,40.64142873)
\curveto(127.98238421,40.62141815)(127.95238424,40.60641816)(127.92238403,40.59642873)
\curveto(127.88238431,40.59641817)(127.83738436,40.58641818)(127.78738403,40.56642873)
\lineto(127.62238403,40.56642873)
\curveto(127.56238463,40.55641821)(127.4923847,40.55141822)(127.41238403,40.55142873)
\curveto(127.33238486,40.56141821)(127.25738494,40.5664182)(127.18738403,40.56642873)
\lineto(126.34738403,40.56642873)
\lineto(121.92238403,40.56642873)
\curveto(121.67239052,40.5664182)(121.42239077,40.5664182)(121.17238403,40.56642873)
\curveto(120.91239128,40.5664182)(120.66239153,40.56141821)(120.42238403,40.55142873)
\curveto(120.32239187,40.55141822)(120.21239198,40.54641822)(120.09238403,40.53642873)
\curveto(119.97239222,40.52641824)(119.91239228,40.4714183)(119.91238403,40.37142873)
\lineto(119.92738403,40.37142873)
\curveto(119.94739225,40.30141847)(120.01239218,40.24141853)(120.12238403,40.19142873)
\curveto(120.23239196,40.15141862)(120.32739187,40.11641865)(120.40738403,40.08642873)
\curveto(120.57739162,40.01641875)(120.75239144,39.95141882)(120.93238403,39.89142873)
\curveto(121.10239109,39.83141894)(121.27239092,39.76141901)(121.44238403,39.68142873)
\curveto(121.4923907,39.66141911)(121.53739066,39.64641912)(121.57738403,39.63642873)
\curveto(121.61739058,39.62641914)(121.66239053,39.61141916)(121.71238403,39.59142873)
\curveto(121.8923903,39.51141926)(122.07739012,39.44141933)(122.26738403,39.38142873)
\curveto(122.44738975,39.33141944)(122.62738957,39.2664195)(122.80738403,39.18642873)
\curveto(122.95738924,39.11641965)(123.11238908,39.05641971)(123.27238403,39.00642873)
\curveto(123.42238877,38.95641981)(123.57238862,38.90141987)(123.72238403,38.84142873)
\curveto(124.192388,38.64142013)(124.66738753,38.46142031)(125.14738403,38.30142873)
\curveto(125.61738658,38.14142063)(126.08238611,37.9664208)(126.54238403,37.77642873)
\curveto(126.72238547,37.69642107)(126.90238529,37.62642114)(127.08238403,37.56642873)
\curveto(127.26238493,37.50642126)(127.44238475,37.44142133)(127.62238403,37.37142873)
\curveto(127.73238446,37.32142145)(127.83738436,37.2714215)(127.93738403,37.22142873)
\curveto(128.02738417,37.18142159)(128.0923841,37.09642167)(128.13238403,36.96642873)
\curveto(128.14238405,36.94642182)(128.14738405,36.92142185)(128.14738403,36.89142873)
\curveto(128.13738406,36.8714219)(128.13738406,36.84642192)(128.14738403,36.81642873)
\curveto(128.15738404,36.78642198)(128.16238403,36.75142202)(128.16238403,36.71142873)
\curveto(128.15238404,36.6714221)(128.14738405,36.63142214)(128.14738403,36.59142873)
\lineto(128.14738403,36.29142873)
\curveto(128.14738405,36.19142258)(128.12238407,36.11142266)(128.07238403,36.05142873)
\curveto(128.02238417,35.9714228)(127.95238424,35.91142286)(127.86238403,35.87142873)
\curveto(127.76238443,35.84142293)(127.66238453,35.80142297)(127.56238403,35.75142873)
\curveto(127.36238483,35.6714231)(127.15738504,35.59142318)(126.94738403,35.51142873)
\curveto(126.72738547,35.44142333)(126.51738568,35.3664234)(126.31738403,35.28642873)
\curveto(126.13738606,35.20642356)(125.95738624,35.13642363)(125.77738403,35.07642873)
\curveto(125.58738661,35.02642374)(125.40238679,34.96142381)(125.22238403,34.88142873)
\curveto(124.66238753,34.65142412)(124.0973881,34.43642433)(123.52738403,34.23642873)
\curveto(122.95738924,34.03642473)(122.3923898,33.82142495)(121.83238403,33.59142873)
\lineto(121.20238403,33.35142873)
\curveto(120.98239121,33.28142549)(120.77239142,33.20642556)(120.57238403,33.12642873)
\curveto(120.46239173,33.07642569)(120.35739184,33.03142574)(120.25738403,32.99142873)
\curveto(120.14739205,32.96142581)(120.05239214,32.91142586)(119.97238403,32.84142873)
\curveto(119.95239224,32.83142594)(119.94239225,32.82142595)(119.94238403,32.81142873)
\lineto(119.91238403,32.78142873)
\lineto(119.91238403,32.70642873)
\lineto(119.94238403,32.67642873)
\curveto(119.94239225,32.6664261)(119.94739225,32.65642611)(119.95738403,32.64642873)
\curveto(120.00739219,32.62642614)(120.06239213,32.61642615)(120.12238403,32.61642873)
\curveto(120.18239201,32.61642615)(120.24239195,32.60642616)(120.30238403,32.58642873)
\lineto(120.46738403,32.58642873)
\curveto(120.52739167,32.5664262)(120.5923916,32.56142621)(120.66238403,32.57142873)
\curveto(120.73239146,32.58142619)(120.80239139,32.58642618)(120.87238403,32.58642873)
\lineto(121.68238403,32.58642873)
\lineto(126.24238403,32.58642873)
\lineto(127.42738403,32.58642873)
\curveto(127.53738466,32.58642618)(127.64738455,32.58142619)(127.75738403,32.57142873)
\curveto(127.86738433,32.5714262)(127.95238424,32.54642622)(128.01238403,32.49642873)
\curveto(128.0923841,32.44642632)(128.13738406,32.35642641)(128.14738403,32.22642873)
\lineto(128.14738403,31.83642873)
\lineto(128.14738403,31.64142873)
\curveto(128.14738405,31.59142718)(128.13738406,31.54142723)(128.11738403,31.49142873)
\curveto(128.07738412,31.36142741)(127.9923842,31.28642748)(127.86238403,31.26642873)
\curveto(127.73238446,31.25642751)(127.58238461,31.25142752)(127.41238403,31.25142873)
\lineto(125.67238403,31.25142873)
\lineto(119.67238403,31.25142873)
\lineto(118.26238403,31.25142873)
\curveto(118.15239404,31.25142752)(118.03739416,31.24642752)(117.91738403,31.23642873)
\curveto(117.7973944,31.23642753)(117.70239449,31.26142751)(117.63238403,31.31142873)
\curveto(117.57239462,31.35142742)(117.52239467,31.42642734)(117.48238403,31.53642873)
\curveto(117.47239472,31.55642721)(117.47239472,31.57642719)(117.48238403,31.59642873)
\curveto(117.48239471,31.62642714)(117.47739472,31.65142712)(117.46738403,31.67142873)
}
}
{
\newrgbcolor{curcolor}{0 0 0}
\pscustom[linestyle=none,fillstyle=solid,fillcolor=curcolor]
{
\newpath
\moveto(127.59238403,50.87353811)
\curveto(127.75238444,50.90353028)(127.88738431,50.88853029)(127.99738403,50.82853811)
\curveto(128.0973841,50.76853041)(128.17238402,50.68853049)(128.22238403,50.58853811)
\curveto(128.24238395,50.53853064)(128.25238394,50.4835307)(128.25238403,50.42353811)
\curveto(128.25238394,50.37353081)(128.26238393,50.31853086)(128.28238403,50.25853811)
\curveto(128.33238386,50.03853114)(128.31738388,49.81853136)(128.23738403,49.59853811)
\curveto(128.16738403,49.38853179)(128.07738412,49.24353194)(127.96738403,49.16353811)
\curveto(127.8973843,49.11353207)(127.81738438,49.06853211)(127.72738403,49.02853811)
\curveto(127.62738457,48.98853219)(127.54738465,48.93853224)(127.48738403,48.87853811)
\curveto(127.46738473,48.85853232)(127.44738475,48.83353235)(127.42738403,48.80353811)
\curveto(127.40738479,48.7835324)(127.40238479,48.75353243)(127.41238403,48.71353811)
\curveto(127.44238475,48.60353258)(127.4973847,48.49853268)(127.57738403,48.39853811)
\curveto(127.65738454,48.30853287)(127.72738447,48.21853296)(127.78738403,48.12853811)
\curveto(127.86738433,47.99853318)(127.94238425,47.85853332)(128.01238403,47.70853811)
\curveto(128.07238412,47.55853362)(128.12738407,47.39853378)(128.17738403,47.22853811)
\curveto(128.20738399,47.12853405)(128.22738397,47.01853416)(128.23738403,46.89853811)
\curveto(128.24738395,46.78853439)(128.26238393,46.6785345)(128.28238403,46.56853811)
\curveto(128.2923839,46.51853466)(128.2973839,46.47353471)(128.29738403,46.43353811)
\lineto(128.29738403,46.32853811)
\curveto(128.31738388,46.21853496)(128.31738388,46.11353507)(128.29738403,46.01353811)
\lineto(128.29738403,45.87853811)
\curveto(128.28738391,45.82853535)(128.28238391,45.7785354)(128.28238403,45.72853811)
\curveto(128.28238391,45.6785355)(128.27238392,45.63353555)(128.25238403,45.59353811)
\curveto(128.24238395,45.55353563)(128.23738396,45.51853566)(128.23738403,45.48853811)
\curveto(128.24738395,45.46853571)(128.24738395,45.44353574)(128.23738403,45.41353811)
\lineto(128.17738403,45.17353811)
\curveto(128.16738403,45.09353609)(128.14738405,45.01853616)(128.11738403,44.94853811)
\curveto(127.98738421,44.64853653)(127.84238435,44.40353678)(127.68238403,44.21353811)
\curveto(127.51238468,44.03353715)(127.27738492,43.8835373)(126.97738403,43.76353811)
\curveto(126.75738544,43.67353751)(126.4923857,43.62853755)(126.18238403,43.62853811)
\lineto(125.86738403,43.62853811)
\curveto(125.81738638,43.63853754)(125.76738643,43.64353754)(125.71738403,43.64353811)
\lineto(125.53738403,43.67353811)
\lineto(125.20738403,43.79353811)
\curveto(125.0973871,43.83353735)(124.9973872,43.8835373)(124.90738403,43.94353811)
\curveto(124.61738758,44.12353706)(124.40238779,44.36853681)(124.26238403,44.67853811)
\curveto(124.12238807,44.98853619)(123.9973882,45.32853585)(123.88738403,45.69853811)
\curveto(123.84738835,45.83853534)(123.81738838,45.9835352)(123.79738403,46.13353811)
\curveto(123.77738842,46.2835349)(123.75238844,46.43353475)(123.72238403,46.58353811)
\curveto(123.70238849,46.65353453)(123.6923885,46.71853446)(123.69238403,46.77853811)
\curveto(123.6923885,46.84853433)(123.68238851,46.92353426)(123.66238403,47.00353811)
\curveto(123.64238855,47.07353411)(123.63238856,47.14353404)(123.63238403,47.21353811)
\curveto(123.62238857,47.2835339)(123.60738859,47.35853382)(123.58738403,47.43853811)
\curveto(123.52738867,47.68853349)(123.47738872,47.92353326)(123.43738403,48.14353811)
\curveto(123.38738881,48.36353282)(123.27238892,48.53853264)(123.09238403,48.66853811)
\curveto(123.01238918,48.72853245)(122.91238928,48.7785324)(122.79238403,48.81853811)
\curveto(122.66238953,48.85853232)(122.52238967,48.85853232)(122.37238403,48.81853811)
\curveto(122.13239006,48.75853242)(121.94239025,48.66853251)(121.80238403,48.54853811)
\curveto(121.66239053,48.43853274)(121.55239064,48.2785329)(121.47238403,48.06853811)
\curveto(121.42239077,47.94853323)(121.38739081,47.80353338)(121.36738403,47.63353811)
\curveto(121.34739085,47.47353371)(121.33739086,47.30353388)(121.33738403,47.12353811)
\curveto(121.33739086,46.94353424)(121.34739085,46.76853441)(121.36738403,46.59853811)
\curveto(121.38739081,46.42853475)(121.41739078,46.2835349)(121.45738403,46.16353811)
\curveto(121.51739068,45.99353519)(121.60239059,45.82853535)(121.71238403,45.66853811)
\curveto(121.77239042,45.58853559)(121.85239034,45.51353567)(121.95238403,45.44353811)
\curveto(122.04239015,45.3835358)(122.14239005,45.32853585)(122.25238403,45.27853811)
\curveto(122.33238986,45.24853593)(122.41738978,45.21853596)(122.50738403,45.18853811)
\curveto(122.5973896,45.16853601)(122.66738953,45.12353606)(122.71738403,45.05353811)
\curveto(122.74738945,45.01353617)(122.77238942,44.94353624)(122.79238403,44.84353811)
\curveto(122.80238939,44.75353643)(122.80738939,44.65853652)(122.80738403,44.55853811)
\curveto(122.80738939,44.45853672)(122.80238939,44.35853682)(122.79238403,44.25853811)
\curveto(122.77238942,44.16853701)(122.74738945,44.10353708)(122.71738403,44.06353811)
\curveto(122.68738951,44.02353716)(122.63738956,43.99353719)(122.56738403,43.97353811)
\curveto(122.4973897,43.95353723)(122.42238977,43.95353723)(122.34238403,43.97353811)
\curveto(122.21238998,44.00353718)(122.0923901,44.03353715)(121.98238403,44.06353811)
\curveto(121.86239033,44.10353708)(121.74739045,44.14853703)(121.63738403,44.19853811)
\curveto(121.28739091,44.38853679)(121.01739118,44.62853655)(120.82738403,44.91853811)
\curveto(120.62739157,45.20853597)(120.46739173,45.56853561)(120.34738403,45.99853811)
\curveto(120.32739187,46.09853508)(120.31239188,46.19853498)(120.30238403,46.29853811)
\curveto(120.2923919,46.40853477)(120.27739192,46.51853466)(120.25738403,46.62853811)
\curveto(120.24739195,46.66853451)(120.24739195,46.73353445)(120.25738403,46.82353811)
\curveto(120.25739194,46.91353427)(120.24739195,46.96853421)(120.22738403,46.98853811)
\curveto(120.21739198,47.68853349)(120.2973919,48.29853288)(120.46738403,48.81853811)
\curveto(120.63739156,49.33853184)(120.96239123,49.70353148)(121.44238403,49.91353811)
\curveto(121.64239055,50.00353118)(121.87739032,50.05353113)(122.14738403,50.06353811)
\curveto(122.40738979,50.0835311)(122.68238951,50.09353109)(122.97238403,50.09353811)
\lineto(126.28738403,50.09353811)
\curveto(126.42738577,50.09353109)(126.56238563,50.09853108)(126.69238403,50.10853811)
\curveto(126.82238537,50.11853106)(126.92738527,50.14853103)(127.00738403,50.19853811)
\curveto(127.07738512,50.24853093)(127.12738507,50.31353087)(127.15738403,50.39353811)
\curveto(127.197385,50.4835307)(127.22738497,50.56853061)(127.24738403,50.64853811)
\curveto(127.25738494,50.72853045)(127.30238489,50.78853039)(127.38238403,50.82853811)
\curveto(127.41238478,50.84853033)(127.44238475,50.85853032)(127.47238403,50.85853811)
\curveto(127.50238469,50.85853032)(127.54238465,50.86353032)(127.59238403,50.87353811)
\moveto(125.92738403,48.72853811)
\curveto(125.78738641,48.78853239)(125.62738657,48.81853236)(125.44738403,48.81853811)
\curveto(125.25738694,48.82853235)(125.06238713,48.83353235)(124.86238403,48.83353811)
\curveto(124.75238744,48.83353235)(124.65238754,48.82853235)(124.56238403,48.81853811)
\curveto(124.47238772,48.80853237)(124.40238779,48.76853241)(124.35238403,48.69853811)
\curveto(124.33238786,48.66853251)(124.32238787,48.59853258)(124.32238403,48.48853811)
\curveto(124.34238785,48.46853271)(124.35238784,48.43353275)(124.35238403,48.38353811)
\curveto(124.35238784,48.33353285)(124.36238783,48.28853289)(124.38238403,48.24853811)
\curveto(124.40238779,48.16853301)(124.42238777,48.0785331)(124.44238403,47.97853811)
\lineto(124.50238403,47.67853811)
\curveto(124.50238769,47.64853353)(124.50738769,47.61353357)(124.51738403,47.57353811)
\lineto(124.51738403,47.46853811)
\curveto(124.55738764,47.31853386)(124.58238761,47.15353403)(124.59238403,46.97353811)
\curveto(124.5923876,46.80353438)(124.61238758,46.64353454)(124.65238403,46.49353811)
\curveto(124.67238752,46.41353477)(124.6923875,46.33853484)(124.71238403,46.26853811)
\curveto(124.72238747,46.20853497)(124.73738746,46.13853504)(124.75738403,46.05853811)
\curveto(124.80738739,45.89853528)(124.87238732,45.74853543)(124.95238403,45.60853811)
\curveto(125.02238717,45.46853571)(125.11238708,45.34853583)(125.22238403,45.24853811)
\curveto(125.33238686,45.14853603)(125.46738673,45.07353611)(125.62738403,45.02353811)
\curveto(125.77738642,44.97353621)(125.96238623,44.95353623)(126.18238403,44.96353811)
\curveto(126.28238591,44.96353622)(126.37738582,44.9785362)(126.46738403,45.00853811)
\curveto(126.54738565,45.04853613)(126.62238557,45.09353609)(126.69238403,45.14353811)
\curveto(126.80238539,45.22353596)(126.8973853,45.32853585)(126.97738403,45.45853811)
\curveto(127.04738515,45.58853559)(127.10738509,45.72853545)(127.15738403,45.87853811)
\curveto(127.16738503,45.92853525)(127.17238502,45.9785352)(127.17238403,46.02853811)
\curveto(127.17238502,46.0785351)(127.17738502,46.12853505)(127.18738403,46.17853811)
\curveto(127.20738499,46.24853493)(127.22238497,46.33353485)(127.23238403,46.43353811)
\curveto(127.23238496,46.54353464)(127.22238497,46.63353455)(127.20238403,46.70353811)
\curveto(127.18238501,46.76353442)(127.17738502,46.82353436)(127.18738403,46.88353811)
\curveto(127.18738501,46.94353424)(127.17738502,47.00353418)(127.15738403,47.06353811)
\curveto(127.13738506,47.14353404)(127.12238507,47.21853396)(127.11238403,47.28853811)
\curveto(127.10238509,47.36853381)(127.08238511,47.44353374)(127.05238403,47.51353811)
\curveto(126.93238526,47.80353338)(126.78738541,48.04853313)(126.61738403,48.24853811)
\curveto(126.44738575,48.45853272)(126.21738598,48.61853256)(125.92738403,48.72853811)
}
}
{
\newrgbcolor{curcolor}{0 0 0}
\pscustom[linestyle=none,fillstyle=solid,fillcolor=curcolor]
{
\newpath
\moveto(120.24238403,55.69017873)
\curveto(120.24239195,55.92017394)(120.30239189,56.05017381)(120.42238403,56.08017873)
\curveto(120.53239166,56.11017375)(120.6973915,56.12517374)(120.91738403,56.12517873)
\lineto(121.20238403,56.12517873)
\curveto(121.2923909,56.12517374)(121.36739083,56.10017376)(121.42738403,56.05017873)
\curveto(121.50739069,55.99017387)(121.55239064,55.90517396)(121.56238403,55.79517873)
\curveto(121.56239063,55.68517418)(121.57739062,55.57517429)(121.60738403,55.46517873)
\curveto(121.63739056,55.32517454)(121.66739053,55.19017467)(121.69738403,55.06017873)
\curveto(121.72739047,54.94017492)(121.76739043,54.82517504)(121.81738403,54.71517873)
\curveto(121.94739025,54.42517544)(122.12739007,54.19017567)(122.35738403,54.01017873)
\curveto(122.57738962,53.83017603)(122.83238936,53.67517619)(123.12238403,53.54517873)
\curveto(123.23238896,53.50517636)(123.34738885,53.47517639)(123.46738403,53.45517873)
\curveto(123.57738862,53.43517643)(123.6923885,53.41017645)(123.81238403,53.38017873)
\curveto(123.86238833,53.37017649)(123.91238828,53.3651765)(123.96238403,53.36517873)
\curveto(124.01238818,53.37517649)(124.06238813,53.37517649)(124.11238403,53.36517873)
\curveto(124.23238796,53.33517653)(124.37238782,53.32017654)(124.53238403,53.32017873)
\curveto(124.68238751,53.33017653)(124.82738737,53.33517653)(124.96738403,53.33517873)
\lineto(126.81238403,53.33517873)
\lineto(127.15738403,53.33517873)
\curveto(127.27738492,53.33517653)(127.3923848,53.33017653)(127.50238403,53.32017873)
\curveto(127.61238458,53.31017655)(127.70738449,53.30517656)(127.78738403,53.30517873)
\curveto(127.86738433,53.31517655)(127.93738426,53.29517657)(127.99738403,53.24517873)
\curveto(128.06738413,53.19517667)(128.10738409,53.11517675)(128.11738403,53.00517873)
\curveto(128.12738407,52.90517696)(128.13238406,52.79517707)(128.13238403,52.67517873)
\lineto(128.13238403,52.40517873)
\curveto(128.11238408,52.35517751)(128.0973841,52.30517756)(128.08738403,52.25517873)
\curveto(128.06738413,52.21517765)(128.04238415,52.18517768)(128.01238403,52.16517873)
\curveto(127.94238425,52.11517775)(127.85738434,52.08517778)(127.75738403,52.07517873)
\lineto(127.42738403,52.07517873)
\lineto(126.27238403,52.07517873)
\lineto(122.11738403,52.07517873)
\lineto(121.08238403,52.07517873)
\lineto(120.78238403,52.07517873)
\curveto(120.68239151,52.08517778)(120.5973916,52.11517775)(120.52738403,52.16517873)
\curveto(120.48739171,52.19517767)(120.45739174,52.24517762)(120.43738403,52.31517873)
\curveto(120.41739178,52.39517747)(120.40739179,52.48017738)(120.40738403,52.57017873)
\curveto(120.3973918,52.6601772)(120.3973918,52.75017711)(120.40738403,52.84017873)
\curveto(120.41739178,52.93017693)(120.43239176,53.00017686)(120.45238403,53.05017873)
\curveto(120.48239171,53.13017673)(120.54239165,53.18017668)(120.63238403,53.20017873)
\curveto(120.71239148,53.23017663)(120.80239139,53.24517662)(120.90238403,53.24517873)
\lineto(121.20238403,53.24517873)
\curveto(121.30239089,53.24517662)(121.3923908,53.2651766)(121.47238403,53.30517873)
\curveto(121.4923907,53.31517655)(121.50739069,53.32517654)(121.51738403,53.33517873)
\lineto(121.56238403,53.38017873)
\curveto(121.56239063,53.49017637)(121.51739068,53.58017628)(121.42738403,53.65017873)
\curveto(121.32739087,53.72017614)(121.24739095,53.78017608)(121.18738403,53.83017873)
\lineto(121.09738403,53.92017873)
\curveto(120.98739121,54.01017585)(120.87239132,54.13517573)(120.75238403,54.29517873)
\curveto(120.63239156,54.45517541)(120.54239165,54.60517526)(120.48238403,54.74517873)
\curveto(120.43239176,54.83517503)(120.3973918,54.93017493)(120.37738403,55.03017873)
\curveto(120.34739185,55.13017473)(120.31739188,55.23517463)(120.28738403,55.34517873)
\curveto(120.27739192,55.40517446)(120.27239192,55.4651744)(120.27238403,55.52517873)
\curveto(120.26239193,55.58517428)(120.25239194,55.64017422)(120.24238403,55.69017873)
}
}
{
\newrgbcolor{curcolor}{0 0 0}
\pscustom[linestyle=none,fillstyle=solid,fillcolor=curcolor]
{
}
}
{
\newrgbcolor{curcolor}{0 0 0}
\pscustom[linestyle=none,fillstyle=solid,fillcolor=curcolor]
{
\newpath
\moveto(123.06238403,67.99510061)
\lineto(123.31738403,67.99510061)
\curveto(123.3973888,68.0050929)(123.47238872,68.00009291)(123.54238403,67.98010061)
\lineto(123.78238403,67.98010061)
\lineto(123.94738403,67.98010061)
\curveto(124.04738815,67.96009295)(124.15238804,67.95009296)(124.26238403,67.95010061)
\curveto(124.36238783,67.95009296)(124.46238773,67.94009297)(124.56238403,67.92010061)
\lineto(124.71238403,67.92010061)
\curveto(124.85238734,67.89009302)(124.9923872,67.87009304)(125.13238403,67.86010061)
\curveto(125.26238693,67.85009306)(125.3923868,67.82509308)(125.52238403,67.78510061)
\curveto(125.60238659,67.76509314)(125.68738651,67.74509316)(125.77738403,67.72510061)
\lineto(126.01738403,67.66510061)
\lineto(126.31738403,67.54510061)
\curveto(126.40738579,67.51509339)(126.4973857,67.48009343)(126.58738403,67.44010061)
\curveto(126.80738539,67.34009357)(127.02238517,67.2050937)(127.23238403,67.03510061)
\curveto(127.44238475,66.87509403)(127.61238458,66.70009421)(127.74238403,66.51010061)
\curveto(127.78238441,66.46009445)(127.82238437,66.40009451)(127.86238403,66.33010061)
\curveto(127.8923843,66.27009464)(127.92738427,66.2100947)(127.96738403,66.15010061)
\curveto(128.01738418,66.07009484)(128.05738414,65.97509493)(128.08738403,65.86510061)
\curveto(128.11738408,65.75509515)(128.14738405,65.65009526)(128.17738403,65.55010061)
\curveto(128.21738398,65.44009547)(128.24238395,65.33009558)(128.25238403,65.22010061)
\curveto(128.26238393,65.1100958)(128.27738392,64.99509591)(128.29738403,64.87510061)
\curveto(128.30738389,64.83509607)(128.30738389,64.79009612)(128.29738403,64.74010061)
\curveto(128.2973839,64.70009621)(128.30238389,64.66009625)(128.31238403,64.62010061)
\curveto(128.32238387,64.58009633)(128.32738387,64.52509638)(128.32738403,64.45510061)
\curveto(128.32738387,64.38509652)(128.32238387,64.33509657)(128.31238403,64.30510061)
\curveto(128.2923839,64.25509665)(128.28738391,64.2100967)(128.29738403,64.17010061)
\curveto(128.30738389,64.13009678)(128.30738389,64.09509681)(128.29738403,64.06510061)
\lineto(128.29738403,63.97510061)
\curveto(128.27738392,63.91509699)(128.26238393,63.85009706)(128.25238403,63.78010061)
\curveto(128.25238394,63.72009719)(128.24738395,63.65509725)(128.23738403,63.58510061)
\curveto(128.18738401,63.41509749)(128.13738406,63.25509765)(128.08738403,63.10510061)
\curveto(128.03738416,62.95509795)(127.97238422,62.8100981)(127.89238403,62.67010061)
\curveto(127.85238434,62.62009829)(127.82238437,62.56509834)(127.80238403,62.50510061)
\curveto(127.77238442,62.45509845)(127.73738446,62.4050985)(127.69738403,62.35510061)
\curveto(127.51738468,62.11509879)(127.2973849,61.91509899)(127.03738403,61.75510061)
\curveto(126.77738542,61.59509931)(126.4923857,61.45509945)(126.18238403,61.33510061)
\curveto(126.04238615,61.27509963)(125.90238629,61.23009968)(125.76238403,61.20010061)
\curveto(125.61238658,61.17009974)(125.45738674,61.13509977)(125.29738403,61.09510061)
\curveto(125.18738701,61.07509983)(125.07738712,61.06009985)(124.96738403,61.05010061)
\curveto(124.85738734,61.04009987)(124.74738745,61.02509988)(124.63738403,61.00510061)
\curveto(124.5973876,60.99509991)(124.55738764,60.99009992)(124.51738403,60.99010061)
\curveto(124.47738772,61.00009991)(124.43738776,61.00009991)(124.39738403,60.99010061)
\curveto(124.34738785,60.98009993)(124.2973879,60.97509993)(124.24738403,60.97510061)
\lineto(124.08238403,60.97510061)
\curveto(124.03238816,60.95509995)(123.98238821,60.95009996)(123.93238403,60.96010061)
\curveto(123.87238832,60.97009994)(123.81738838,60.97009994)(123.76738403,60.96010061)
\curveto(123.72738847,60.95009996)(123.68238851,60.95009996)(123.63238403,60.96010061)
\curveto(123.58238861,60.97009994)(123.53238866,60.96509994)(123.48238403,60.94510061)
\curveto(123.41238878,60.92509998)(123.33738886,60.92009999)(123.25738403,60.93010061)
\curveto(123.16738903,60.94009997)(123.08238911,60.94509996)(123.00238403,60.94510061)
\curveto(122.91238928,60.94509996)(122.81238938,60.94009997)(122.70238403,60.93010061)
\curveto(122.58238961,60.92009999)(122.48238971,60.92509998)(122.40238403,60.94510061)
\lineto(122.11738403,60.94510061)
\lineto(121.48738403,60.99010061)
\curveto(121.38739081,61.00009991)(121.2923909,61.0100999)(121.20238403,61.02010061)
\lineto(120.90238403,61.05010061)
\curveto(120.85239134,61.07009984)(120.80239139,61.07509983)(120.75238403,61.06510061)
\curveto(120.6923915,61.06509984)(120.63739156,61.07509983)(120.58738403,61.09510061)
\curveto(120.41739178,61.14509976)(120.25239194,61.18509972)(120.09238403,61.21510061)
\curveto(119.92239227,61.24509966)(119.76239243,61.29509961)(119.61238403,61.36510061)
\curveto(119.15239304,61.55509935)(118.77739342,61.77509913)(118.48738403,62.02510061)
\curveto(118.197394,62.28509862)(117.95239424,62.64509826)(117.75238403,63.10510061)
\curveto(117.70239449,63.23509767)(117.66739453,63.36509754)(117.64738403,63.49510061)
\curveto(117.62739457,63.63509727)(117.60239459,63.77509713)(117.57238403,63.91510061)
\curveto(117.56239463,63.98509692)(117.55739464,64.05009686)(117.55738403,64.11010061)
\curveto(117.55739464,64.17009674)(117.55239464,64.23509667)(117.54238403,64.30510061)
\curveto(117.52239467,65.13509577)(117.67239452,65.8050951)(117.99238403,66.31510061)
\curveto(118.30239389,66.82509408)(118.74239345,67.2050937)(119.31238403,67.45510061)
\curveto(119.43239276,67.5050934)(119.55739264,67.55009336)(119.68738403,67.59010061)
\curveto(119.81739238,67.63009328)(119.95239224,67.67509323)(120.09238403,67.72510061)
\curveto(120.17239202,67.74509316)(120.25739194,67.76009315)(120.34738403,67.77010061)
\lineto(120.58738403,67.83010061)
\curveto(120.6973915,67.86009305)(120.80739139,67.87509303)(120.91738403,67.87510061)
\curveto(121.02739117,67.88509302)(121.13739106,67.90009301)(121.24738403,67.92010061)
\curveto(121.2973909,67.94009297)(121.34239085,67.94509296)(121.38238403,67.93510061)
\curveto(121.42239077,67.93509297)(121.46239073,67.94009297)(121.50238403,67.95010061)
\curveto(121.55239064,67.96009295)(121.60739059,67.96009295)(121.66738403,67.95010061)
\curveto(121.71739048,67.95009296)(121.76739043,67.95509295)(121.81738403,67.96510061)
\lineto(121.95238403,67.96510061)
\curveto(122.01239018,67.98509292)(122.08239011,67.98509292)(122.16238403,67.96510061)
\curveto(122.23238996,67.95509295)(122.2973899,67.96009295)(122.35738403,67.98010061)
\curveto(122.38738981,67.99009292)(122.42738977,67.99509291)(122.47738403,67.99510061)
\lineto(122.59738403,67.99510061)
\lineto(123.06238403,67.99510061)
\moveto(125.38738403,66.45010061)
\curveto(125.06738713,66.55009436)(124.70238749,66.6100943)(124.29238403,66.63010061)
\curveto(123.88238831,66.65009426)(123.47238872,66.66009425)(123.06238403,66.66010061)
\curveto(122.63238956,66.66009425)(122.21238998,66.65009426)(121.80238403,66.63010061)
\curveto(121.3923908,66.6100943)(121.00739119,66.56509434)(120.64738403,66.49510061)
\curveto(120.28739191,66.42509448)(119.96739223,66.31509459)(119.68738403,66.16510061)
\curveto(119.3973928,66.02509488)(119.16239303,65.83009508)(118.98238403,65.58010061)
\curveto(118.87239332,65.42009549)(118.7923934,65.24009567)(118.74238403,65.04010061)
\curveto(118.68239351,64.84009607)(118.65239354,64.59509631)(118.65238403,64.30510061)
\curveto(118.67239352,64.28509662)(118.68239351,64.25009666)(118.68238403,64.20010061)
\curveto(118.67239352,64.15009676)(118.67239352,64.1100968)(118.68238403,64.08010061)
\curveto(118.70239349,64.00009691)(118.72239347,63.92509698)(118.74238403,63.85510061)
\curveto(118.75239344,63.79509711)(118.77239342,63.73009718)(118.80238403,63.66010061)
\curveto(118.92239327,63.39009752)(119.0923931,63.17009774)(119.31238403,63.00010061)
\curveto(119.52239267,62.84009807)(119.76739243,62.7050982)(120.04738403,62.59510061)
\curveto(120.15739204,62.54509836)(120.27739192,62.5050984)(120.40738403,62.47510061)
\curveto(120.52739167,62.45509845)(120.65239154,62.43009848)(120.78238403,62.40010061)
\curveto(120.83239136,62.38009853)(120.88739131,62.37009854)(120.94738403,62.37010061)
\curveto(120.9973912,62.37009854)(121.04739115,62.36509854)(121.09738403,62.35510061)
\curveto(121.18739101,62.34509856)(121.28239091,62.33509857)(121.38238403,62.32510061)
\curveto(121.47239072,62.31509859)(121.56739063,62.3050986)(121.66738403,62.29510061)
\curveto(121.74739045,62.29509861)(121.83239036,62.29009862)(121.92238403,62.28010061)
\lineto(122.16238403,62.28010061)
\lineto(122.34238403,62.28010061)
\curveto(122.37238982,62.27009864)(122.40738979,62.26509864)(122.44738403,62.26510061)
\lineto(122.58238403,62.26510061)
\lineto(123.03238403,62.26510061)
\curveto(123.11238908,62.26509864)(123.197389,62.26009865)(123.28738403,62.25010061)
\curveto(123.36738883,62.25009866)(123.44238875,62.26009865)(123.51238403,62.28010061)
\lineto(123.78238403,62.28010061)
\curveto(123.80238839,62.28009863)(123.83238836,62.27509863)(123.87238403,62.26510061)
\curveto(123.90238829,62.26509864)(123.92738827,62.27009864)(123.94738403,62.28010061)
\curveto(124.04738815,62.29009862)(124.14738805,62.29509861)(124.24738403,62.29510061)
\curveto(124.33738786,62.3050986)(124.43738776,62.31509859)(124.54738403,62.32510061)
\curveto(124.66738753,62.35509855)(124.7923874,62.37009854)(124.92238403,62.37010061)
\curveto(125.04238715,62.38009853)(125.15738704,62.4050985)(125.26738403,62.44510061)
\curveto(125.56738663,62.52509838)(125.83238636,62.6100983)(126.06238403,62.70010061)
\curveto(126.2923859,62.80009811)(126.50738569,62.94509796)(126.70738403,63.13510061)
\curveto(126.90738529,63.34509756)(127.05738514,63.6100973)(127.15738403,63.93010061)
\curveto(127.17738502,63.97009694)(127.18738501,64.0050969)(127.18738403,64.03510061)
\curveto(127.17738502,64.07509683)(127.18238501,64.12009679)(127.20238403,64.17010061)
\curveto(127.21238498,64.2100967)(127.22238497,64.28009663)(127.23238403,64.38010061)
\curveto(127.24238495,64.49009642)(127.23738496,64.57509633)(127.21738403,64.63510061)
\curveto(127.197385,64.7050962)(127.18738501,64.77509613)(127.18738403,64.84510061)
\curveto(127.17738502,64.91509599)(127.16238503,64.98009593)(127.14238403,65.04010061)
\curveto(127.08238511,65.24009567)(126.9973852,65.42009549)(126.88738403,65.58010061)
\curveto(126.86738533,65.6100953)(126.84738535,65.63509527)(126.82738403,65.65510061)
\lineto(126.76738403,65.71510061)
\curveto(126.74738545,65.75509515)(126.70738549,65.8050951)(126.64738403,65.86510061)
\curveto(126.50738569,65.96509494)(126.37738582,66.05009486)(126.25738403,66.12010061)
\curveto(126.13738606,66.19009472)(125.9923862,66.26009465)(125.82238403,66.33010061)
\curveto(125.75238644,66.36009455)(125.68238651,66.38009453)(125.61238403,66.39010061)
\curveto(125.54238665,66.4100945)(125.46738673,66.43009448)(125.38738403,66.45010061)
}
}
{
\newrgbcolor{curcolor}{0 0 0}
\pscustom[linestyle=none,fillstyle=solid,fillcolor=curcolor]
{
\newpath
\moveto(117.54238403,72.59470998)
\curveto(117.53239466,73.28470535)(117.65239454,73.88470475)(117.90238403,74.39470998)
\curveto(118.15239404,74.91470372)(118.48739371,75.30970332)(118.90738403,75.57970998)
\curveto(118.98739321,75.629703)(119.07739312,75.67470296)(119.17738403,75.71470998)
\curveto(119.26739293,75.75470288)(119.36239283,75.79970283)(119.46238403,75.84970998)
\curveto(119.56239263,75.88970274)(119.66239253,75.91970271)(119.76238403,75.93970998)
\curveto(119.86239233,75.95970267)(119.96739223,75.97970265)(120.07738403,75.99970998)
\curveto(120.12739207,76.01970261)(120.17239202,76.02470261)(120.21238403,76.01470998)
\curveto(120.25239194,76.00470263)(120.2973919,76.00970262)(120.34738403,76.02970998)
\curveto(120.3973918,76.03970259)(120.48239171,76.04470259)(120.60238403,76.04470998)
\curveto(120.71239148,76.04470259)(120.7973914,76.03970259)(120.85738403,76.02970998)
\curveto(120.91739128,76.00970262)(120.97739122,75.99970263)(121.03738403,75.99970998)
\curveto(121.0973911,76.00970262)(121.15739104,76.00470263)(121.21738403,75.98470998)
\curveto(121.35739084,75.94470269)(121.4923907,75.90970272)(121.62238403,75.87970998)
\curveto(121.75239044,75.84970278)(121.87739032,75.80970282)(121.99738403,75.75970998)
\curveto(122.13739006,75.69970293)(122.26238993,75.629703)(122.37238403,75.54970998)
\curveto(122.48238971,75.47970315)(122.5923896,75.40470323)(122.70238403,75.32470998)
\lineto(122.76238403,75.26470998)
\curveto(122.78238941,75.25470338)(122.80238939,75.23970339)(122.82238403,75.21970998)
\curveto(122.98238921,75.09970353)(123.12738907,74.96470367)(123.25738403,74.81470998)
\curveto(123.38738881,74.66470397)(123.51238868,74.50470413)(123.63238403,74.33470998)
\curveto(123.85238834,74.02470461)(124.05738814,73.7297049)(124.24738403,73.44970998)
\curveto(124.38738781,73.21970541)(124.52238767,72.98970564)(124.65238403,72.75970998)
\curveto(124.78238741,72.53970609)(124.91738728,72.31970631)(125.05738403,72.09970998)
\curveto(125.22738697,71.84970678)(125.40738679,71.60970702)(125.59738403,71.37970998)
\curveto(125.78738641,71.15970747)(126.01238618,70.96970766)(126.27238403,70.80970998)
\curveto(126.33238586,70.76970786)(126.3923858,70.7347079)(126.45238403,70.70470998)
\curveto(126.50238569,70.67470796)(126.56738563,70.64470799)(126.64738403,70.61470998)
\curveto(126.71738548,70.59470804)(126.77738542,70.58970804)(126.82738403,70.59970998)
\curveto(126.8973853,70.61970801)(126.95238524,70.65470798)(126.99238403,70.70470998)
\curveto(127.02238517,70.75470788)(127.04238515,70.81470782)(127.05238403,70.88470998)
\lineto(127.05238403,71.12470998)
\lineto(127.05238403,71.87470998)
\lineto(127.05238403,74.67970998)
\lineto(127.05238403,75.33970998)
\curveto(127.05238514,75.4297032)(127.05738514,75.51470312)(127.06738403,75.59470998)
\curveto(127.06738513,75.67470296)(127.08738511,75.73970289)(127.12738403,75.78970998)
\curveto(127.16738503,75.83970279)(127.24238495,75.87970275)(127.35238403,75.90970998)
\curveto(127.45238474,75.94970268)(127.55238464,75.95970267)(127.65238403,75.93970998)
\lineto(127.78738403,75.93970998)
\curveto(127.85738434,75.91970271)(127.91738428,75.89970273)(127.96738403,75.87970998)
\curveto(128.01738418,75.85970277)(128.05738414,75.82470281)(128.08738403,75.77470998)
\curveto(128.12738407,75.72470291)(128.14738405,75.65470298)(128.14738403,75.56470998)
\lineto(128.14738403,75.29470998)
\lineto(128.14738403,74.39470998)
\lineto(128.14738403,70.88470998)
\lineto(128.14738403,69.81970998)
\curveto(128.14738405,69.73970889)(128.15238404,69.64970898)(128.16238403,69.54970998)
\curveto(128.16238403,69.44970918)(128.15238404,69.36470927)(128.13238403,69.29470998)
\curveto(128.06238413,69.08470955)(127.88238431,69.01970961)(127.59238403,69.09970998)
\curveto(127.55238464,69.10970952)(127.51738468,69.10970952)(127.48738403,69.09970998)
\curveto(127.44738475,69.09970953)(127.40238479,69.10970952)(127.35238403,69.12970998)
\curveto(127.27238492,69.14970948)(127.18738501,69.16970946)(127.09738403,69.18970998)
\curveto(127.00738519,69.20970942)(126.92238527,69.2347094)(126.84238403,69.26470998)
\curveto(126.35238584,69.42470921)(125.93738626,69.62470901)(125.59738403,69.86470998)
\curveto(125.34738685,70.04470859)(125.12238707,70.24970838)(124.92238403,70.47970998)
\curveto(124.71238748,70.70970792)(124.51738768,70.94970768)(124.33738403,71.19970998)
\curveto(124.15738804,71.45970717)(123.98738821,71.72470691)(123.82738403,71.99470998)
\curveto(123.65738854,72.27470636)(123.48238871,72.54470609)(123.30238403,72.80470998)
\curveto(123.22238897,72.91470572)(123.14738905,73.01970561)(123.07738403,73.11970998)
\curveto(123.00738919,73.2297054)(122.93238926,73.33970529)(122.85238403,73.44970998)
\curveto(122.82238937,73.48970514)(122.7923894,73.52470511)(122.76238403,73.55470998)
\curveto(122.72238947,73.59470504)(122.6923895,73.634705)(122.67238403,73.67470998)
\curveto(122.56238963,73.81470482)(122.43738976,73.93970469)(122.29738403,74.04970998)
\curveto(122.26738993,74.06970456)(122.24238995,74.09470454)(122.22238403,74.12470998)
\curveto(122.19239,74.15470448)(122.16239003,74.17970445)(122.13238403,74.19970998)
\curveto(122.03239016,74.27970435)(121.93239026,74.34470429)(121.83238403,74.39470998)
\curveto(121.73239046,74.45470418)(121.62239057,74.50970412)(121.50238403,74.55970998)
\curveto(121.43239076,74.58970404)(121.35739084,74.60970402)(121.27738403,74.61970998)
\lineto(121.03738403,74.67970998)
\lineto(120.94738403,74.67970998)
\curveto(120.91739128,74.68970394)(120.88739131,74.69470394)(120.85738403,74.69470998)
\curveto(120.78739141,74.71470392)(120.6923915,74.71970391)(120.57238403,74.70970998)
\curveto(120.44239175,74.70970392)(120.34239185,74.69970393)(120.27238403,74.67970998)
\curveto(120.192392,74.65970397)(120.11739208,74.63970399)(120.04738403,74.61970998)
\curveto(119.96739223,74.60970402)(119.88739231,74.58970404)(119.80738403,74.55970998)
\curveto(119.56739263,74.44970418)(119.36739283,74.29970433)(119.20738403,74.10970998)
\curveto(119.03739316,73.9297047)(118.8973933,73.70970492)(118.78738403,73.44970998)
\curveto(118.76739343,73.37970525)(118.75239344,73.30970532)(118.74238403,73.23970998)
\curveto(118.72239347,73.16970546)(118.70239349,73.09470554)(118.68238403,73.01470998)
\curveto(118.66239353,72.9347057)(118.65239354,72.82470581)(118.65238403,72.68470998)
\curveto(118.65239354,72.55470608)(118.66239353,72.44970618)(118.68238403,72.36970998)
\curveto(118.6923935,72.30970632)(118.6973935,72.25470638)(118.69738403,72.20470998)
\curveto(118.6973935,72.15470648)(118.70739349,72.10470653)(118.72738403,72.05470998)
\curveto(118.76739343,71.95470668)(118.80739339,71.85970677)(118.84738403,71.76970998)
\curveto(118.88739331,71.68970694)(118.93239326,71.60970702)(118.98238403,71.52970998)
\curveto(119.00239319,71.49970713)(119.02739317,71.46970716)(119.05738403,71.43970998)
\curveto(119.08739311,71.41970721)(119.11239308,71.39470724)(119.13238403,71.36470998)
\lineto(119.20738403,71.28970998)
\curveto(119.22739297,71.25970737)(119.24739295,71.2347074)(119.26738403,71.21470998)
\lineto(119.47738403,71.06470998)
\curveto(119.53739266,71.02470761)(119.60239259,70.97970765)(119.67238403,70.92970998)
\curveto(119.76239243,70.86970776)(119.86739233,70.81970781)(119.98738403,70.77970998)
\curveto(120.0973921,70.74970788)(120.20739199,70.71470792)(120.31738403,70.67470998)
\curveto(120.42739177,70.634708)(120.57239162,70.60970802)(120.75238403,70.59970998)
\curveto(120.92239127,70.58970804)(121.04739115,70.55970807)(121.12738403,70.50970998)
\curveto(121.20739099,70.45970817)(121.25239094,70.38470825)(121.26238403,70.28470998)
\curveto(121.27239092,70.18470845)(121.27739092,70.07470856)(121.27738403,69.95470998)
\curveto(121.27739092,69.91470872)(121.28239091,69.87470876)(121.29238403,69.83470998)
\curveto(121.2923909,69.79470884)(121.28739091,69.75970887)(121.27738403,69.72970998)
\curveto(121.25739094,69.67970895)(121.24739095,69.629709)(121.24738403,69.57970998)
\curveto(121.24739095,69.53970909)(121.23739096,69.49970913)(121.21738403,69.45970998)
\curveto(121.15739104,69.36970926)(121.02239117,69.32470931)(120.81238403,69.32470998)
\lineto(120.69238403,69.32470998)
\curveto(120.63239156,69.3347093)(120.57239162,69.33970929)(120.51238403,69.33970998)
\curveto(120.44239175,69.34970928)(120.37739182,69.35970927)(120.31738403,69.36970998)
\curveto(120.20739199,69.38970924)(120.10739209,69.40970922)(120.01738403,69.42970998)
\curveto(119.91739228,69.44970918)(119.82239237,69.47970915)(119.73238403,69.51970998)
\curveto(119.66239253,69.53970909)(119.60239259,69.55970907)(119.55238403,69.57970998)
\lineto(119.37238403,69.63970998)
\curveto(119.11239308,69.75970887)(118.86739333,69.91470872)(118.63738403,70.10470998)
\curveto(118.40739379,70.30470833)(118.22239397,70.51970811)(118.08238403,70.74970998)
\curveto(118.00239419,70.85970777)(117.93739426,70.97470766)(117.88738403,71.09470998)
\lineto(117.73738403,71.48470998)
\curveto(117.68739451,71.59470704)(117.65739454,71.70970692)(117.64738403,71.82970998)
\curveto(117.62739457,71.94970668)(117.60239459,72.07470656)(117.57238403,72.20470998)
\curveto(117.57239462,72.27470636)(117.57239462,72.33970629)(117.57238403,72.39970998)
\curveto(117.56239463,72.45970617)(117.55239464,72.52470611)(117.54238403,72.59470998)
}
}
{
\newrgbcolor{curcolor}{0 0 0}
\pscustom[linestyle=none,fillstyle=solid,fillcolor=curcolor]
{
\newpath
\moveto(126.51238403,78.63431936)
\lineto(126.51238403,79.26431936)
\lineto(126.51238403,79.45931936)
\curveto(126.51238568,79.52931683)(126.52238567,79.58931677)(126.54238403,79.63931936)
\curveto(126.58238561,79.70931665)(126.62238557,79.7593166)(126.66238403,79.78931936)
\curveto(126.71238548,79.82931653)(126.77738542,79.84931651)(126.85738403,79.84931936)
\curveto(126.93738526,79.8593165)(127.02238517,79.86431649)(127.11238403,79.86431936)
\lineto(127.83238403,79.86431936)
\curveto(128.31238388,79.86431649)(128.72238347,79.80431655)(129.06238403,79.68431936)
\curveto(129.40238279,79.56431679)(129.67738252,79.36931699)(129.88738403,79.09931936)
\curveto(129.93738226,79.02931733)(129.98238221,78.9593174)(130.02238403,78.88931936)
\curveto(130.07238212,78.82931753)(130.11738208,78.7543176)(130.15738403,78.66431936)
\curveto(130.16738203,78.64431771)(130.17738202,78.61431774)(130.18738403,78.57431936)
\curveto(130.20738199,78.53431782)(130.21238198,78.48931787)(130.20238403,78.43931936)
\curveto(130.17238202,78.34931801)(130.0973821,78.29431806)(129.97738403,78.27431936)
\curveto(129.86738233,78.2543181)(129.77238242,78.26931809)(129.69238403,78.31931936)
\curveto(129.62238257,78.34931801)(129.55738264,78.39431796)(129.49738403,78.45431936)
\curveto(129.44738275,78.52431783)(129.3973828,78.58931777)(129.34738403,78.64931936)
\curveto(129.2973829,78.71931764)(129.22238297,78.77931758)(129.12238403,78.82931936)
\curveto(129.03238316,78.88931747)(128.94238325,78.93931742)(128.85238403,78.97931936)
\curveto(128.82238337,78.99931736)(128.76238343,79.02431733)(128.67238403,79.05431936)
\curveto(128.5923836,79.08431727)(128.52238367,79.08931727)(128.46238403,79.06931936)
\curveto(128.32238387,79.03931732)(128.23238396,78.97931738)(128.19238403,78.88931936)
\curveto(128.16238403,78.80931755)(128.14738405,78.71931764)(128.14738403,78.61931936)
\curveto(128.14738405,78.51931784)(128.12238407,78.43431792)(128.07238403,78.36431936)
\curveto(128.00238419,78.27431808)(127.86238433,78.22931813)(127.65238403,78.22931936)
\lineto(127.09738403,78.22931936)
\lineto(126.87238403,78.22931936)
\curveto(126.7923854,78.23931812)(126.72738547,78.2593181)(126.67738403,78.28931936)
\curveto(126.5973856,78.34931801)(126.55238564,78.41931794)(126.54238403,78.49931936)
\curveto(126.53238566,78.51931784)(126.52738567,78.53931782)(126.52738403,78.55931936)
\curveto(126.52738567,78.58931777)(126.52238567,78.61431774)(126.51238403,78.63431936)
}
}
{
\newrgbcolor{curcolor}{0 0 0}
\pscustom[linestyle=none,fillstyle=solid,fillcolor=curcolor]
{
}
}
{
\newrgbcolor{curcolor}{0 0 0}
\pscustom[linestyle=none,fillstyle=solid,fillcolor=curcolor]
{
\newpath
\moveto(117.54238403,89.26463186)
\curveto(117.53239466,89.95462722)(117.65239454,90.55462662)(117.90238403,91.06463186)
\curveto(118.15239404,91.58462559)(118.48739371,91.9796252)(118.90738403,92.24963186)
\curveto(118.98739321,92.29962488)(119.07739312,92.34462483)(119.17738403,92.38463186)
\curveto(119.26739293,92.42462475)(119.36239283,92.46962471)(119.46238403,92.51963186)
\curveto(119.56239263,92.55962462)(119.66239253,92.58962459)(119.76238403,92.60963186)
\curveto(119.86239233,92.62962455)(119.96739223,92.64962453)(120.07738403,92.66963186)
\curveto(120.12739207,92.68962449)(120.17239202,92.69462448)(120.21238403,92.68463186)
\curveto(120.25239194,92.6746245)(120.2973919,92.6796245)(120.34738403,92.69963186)
\curveto(120.3973918,92.70962447)(120.48239171,92.71462446)(120.60238403,92.71463186)
\curveto(120.71239148,92.71462446)(120.7973914,92.70962447)(120.85738403,92.69963186)
\curveto(120.91739128,92.6796245)(120.97739122,92.66962451)(121.03738403,92.66963186)
\curveto(121.0973911,92.6796245)(121.15739104,92.6746245)(121.21738403,92.65463186)
\curveto(121.35739084,92.61462456)(121.4923907,92.5796246)(121.62238403,92.54963186)
\curveto(121.75239044,92.51962466)(121.87739032,92.4796247)(121.99738403,92.42963186)
\curveto(122.13739006,92.36962481)(122.26238993,92.29962488)(122.37238403,92.21963186)
\curveto(122.48238971,92.14962503)(122.5923896,92.0746251)(122.70238403,91.99463186)
\lineto(122.76238403,91.93463186)
\curveto(122.78238941,91.92462525)(122.80238939,91.90962527)(122.82238403,91.88963186)
\curveto(122.98238921,91.76962541)(123.12738907,91.63462554)(123.25738403,91.48463186)
\curveto(123.38738881,91.33462584)(123.51238868,91.174626)(123.63238403,91.00463186)
\curveto(123.85238834,90.69462648)(124.05738814,90.39962678)(124.24738403,90.11963186)
\curveto(124.38738781,89.88962729)(124.52238767,89.65962752)(124.65238403,89.42963186)
\curveto(124.78238741,89.20962797)(124.91738728,88.98962819)(125.05738403,88.76963186)
\curveto(125.22738697,88.51962866)(125.40738679,88.2796289)(125.59738403,88.04963186)
\curveto(125.78738641,87.82962935)(126.01238618,87.63962954)(126.27238403,87.47963186)
\curveto(126.33238586,87.43962974)(126.3923858,87.40462977)(126.45238403,87.37463186)
\curveto(126.50238569,87.34462983)(126.56738563,87.31462986)(126.64738403,87.28463186)
\curveto(126.71738548,87.26462991)(126.77738542,87.25962992)(126.82738403,87.26963186)
\curveto(126.8973853,87.28962989)(126.95238524,87.32462985)(126.99238403,87.37463186)
\curveto(127.02238517,87.42462975)(127.04238515,87.48462969)(127.05238403,87.55463186)
\lineto(127.05238403,87.79463186)
\lineto(127.05238403,88.54463186)
\lineto(127.05238403,91.34963186)
\lineto(127.05238403,92.00963186)
\curveto(127.05238514,92.09962508)(127.05738514,92.18462499)(127.06738403,92.26463186)
\curveto(127.06738513,92.34462483)(127.08738511,92.40962477)(127.12738403,92.45963186)
\curveto(127.16738503,92.50962467)(127.24238495,92.54962463)(127.35238403,92.57963186)
\curveto(127.45238474,92.61962456)(127.55238464,92.62962455)(127.65238403,92.60963186)
\lineto(127.78738403,92.60963186)
\curveto(127.85738434,92.58962459)(127.91738428,92.56962461)(127.96738403,92.54963186)
\curveto(128.01738418,92.52962465)(128.05738414,92.49462468)(128.08738403,92.44463186)
\curveto(128.12738407,92.39462478)(128.14738405,92.32462485)(128.14738403,92.23463186)
\lineto(128.14738403,91.96463186)
\lineto(128.14738403,91.06463186)
\lineto(128.14738403,87.55463186)
\lineto(128.14738403,86.48963186)
\curveto(128.14738405,86.40963077)(128.15238404,86.31963086)(128.16238403,86.21963186)
\curveto(128.16238403,86.11963106)(128.15238404,86.03463114)(128.13238403,85.96463186)
\curveto(128.06238413,85.75463142)(127.88238431,85.68963149)(127.59238403,85.76963186)
\curveto(127.55238464,85.7796314)(127.51738468,85.7796314)(127.48738403,85.76963186)
\curveto(127.44738475,85.76963141)(127.40238479,85.7796314)(127.35238403,85.79963186)
\curveto(127.27238492,85.81963136)(127.18738501,85.83963134)(127.09738403,85.85963186)
\curveto(127.00738519,85.8796313)(126.92238527,85.90463127)(126.84238403,85.93463186)
\curveto(126.35238584,86.09463108)(125.93738626,86.29463088)(125.59738403,86.53463186)
\curveto(125.34738685,86.71463046)(125.12238707,86.91963026)(124.92238403,87.14963186)
\curveto(124.71238748,87.3796298)(124.51738768,87.61962956)(124.33738403,87.86963186)
\curveto(124.15738804,88.12962905)(123.98738821,88.39462878)(123.82738403,88.66463186)
\curveto(123.65738854,88.94462823)(123.48238871,89.21462796)(123.30238403,89.47463186)
\curveto(123.22238897,89.58462759)(123.14738905,89.68962749)(123.07738403,89.78963186)
\curveto(123.00738919,89.89962728)(122.93238926,90.00962717)(122.85238403,90.11963186)
\curveto(122.82238937,90.15962702)(122.7923894,90.19462698)(122.76238403,90.22463186)
\curveto(122.72238947,90.26462691)(122.6923895,90.30462687)(122.67238403,90.34463186)
\curveto(122.56238963,90.48462669)(122.43738976,90.60962657)(122.29738403,90.71963186)
\curveto(122.26738993,90.73962644)(122.24238995,90.76462641)(122.22238403,90.79463186)
\curveto(122.19239,90.82462635)(122.16239003,90.84962633)(122.13238403,90.86963186)
\curveto(122.03239016,90.94962623)(121.93239026,91.01462616)(121.83238403,91.06463186)
\curveto(121.73239046,91.12462605)(121.62239057,91.179626)(121.50238403,91.22963186)
\curveto(121.43239076,91.25962592)(121.35739084,91.2796259)(121.27738403,91.28963186)
\lineto(121.03738403,91.34963186)
\lineto(120.94738403,91.34963186)
\curveto(120.91739128,91.35962582)(120.88739131,91.36462581)(120.85738403,91.36463186)
\curveto(120.78739141,91.38462579)(120.6923915,91.38962579)(120.57238403,91.37963186)
\curveto(120.44239175,91.3796258)(120.34239185,91.36962581)(120.27238403,91.34963186)
\curveto(120.192392,91.32962585)(120.11739208,91.30962587)(120.04738403,91.28963186)
\curveto(119.96739223,91.2796259)(119.88739231,91.25962592)(119.80738403,91.22963186)
\curveto(119.56739263,91.11962606)(119.36739283,90.96962621)(119.20738403,90.77963186)
\curveto(119.03739316,90.59962658)(118.8973933,90.3796268)(118.78738403,90.11963186)
\curveto(118.76739343,90.04962713)(118.75239344,89.9796272)(118.74238403,89.90963186)
\curveto(118.72239347,89.83962734)(118.70239349,89.76462741)(118.68238403,89.68463186)
\curveto(118.66239353,89.60462757)(118.65239354,89.49462768)(118.65238403,89.35463186)
\curveto(118.65239354,89.22462795)(118.66239353,89.11962806)(118.68238403,89.03963186)
\curveto(118.6923935,88.9796282)(118.6973935,88.92462825)(118.69738403,88.87463186)
\curveto(118.6973935,88.82462835)(118.70739349,88.7746284)(118.72738403,88.72463186)
\curveto(118.76739343,88.62462855)(118.80739339,88.52962865)(118.84738403,88.43963186)
\curveto(118.88739331,88.35962882)(118.93239326,88.2796289)(118.98238403,88.19963186)
\curveto(119.00239319,88.16962901)(119.02739317,88.13962904)(119.05738403,88.10963186)
\curveto(119.08739311,88.08962909)(119.11239308,88.06462911)(119.13238403,88.03463186)
\lineto(119.20738403,87.95963186)
\curveto(119.22739297,87.92962925)(119.24739295,87.90462927)(119.26738403,87.88463186)
\lineto(119.47738403,87.73463186)
\curveto(119.53739266,87.69462948)(119.60239259,87.64962953)(119.67238403,87.59963186)
\curveto(119.76239243,87.53962964)(119.86739233,87.48962969)(119.98738403,87.44963186)
\curveto(120.0973921,87.41962976)(120.20739199,87.38462979)(120.31738403,87.34463186)
\curveto(120.42739177,87.30462987)(120.57239162,87.2796299)(120.75238403,87.26963186)
\curveto(120.92239127,87.25962992)(121.04739115,87.22962995)(121.12738403,87.17963186)
\curveto(121.20739099,87.12963005)(121.25239094,87.05463012)(121.26238403,86.95463186)
\curveto(121.27239092,86.85463032)(121.27739092,86.74463043)(121.27738403,86.62463186)
\curveto(121.27739092,86.58463059)(121.28239091,86.54463063)(121.29238403,86.50463186)
\curveto(121.2923909,86.46463071)(121.28739091,86.42963075)(121.27738403,86.39963186)
\curveto(121.25739094,86.34963083)(121.24739095,86.29963088)(121.24738403,86.24963186)
\curveto(121.24739095,86.20963097)(121.23739096,86.16963101)(121.21738403,86.12963186)
\curveto(121.15739104,86.03963114)(121.02239117,85.99463118)(120.81238403,85.99463186)
\lineto(120.69238403,85.99463186)
\curveto(120.63239156,86.00463117)(120.57239162,86.00963117)(120.51238403,86.00963186)
\curveto(120.44239175,86.01963116)(120.37739182,86.02963115)(120.31738403,86.03963186)
\curveto(120.20739199,86.05963112)(120.10739209,86.0796311)(120.01738403,86.09963186)
\curveto(119.91739228,86.11963106)(119.82239237,86.14963103)(119.73238403,86.18963186)
\curveto(119.66239253,86.20963097)(119.60239259,86.22963095)(119.55238403,86.24963186)
\lineto(119.37238403,86.30963186)
\curveto(119.11239308,86.42963075)(118.86739333,86.58463059)(118.63738403,86.77463186)
\curveto(118.40739379,86.9746302)(118.22239397,87.18962999)(118.08238403,87.41963186)
\curveto(118.00239419,87.52962965)(117.93739426,87.64462953)(117.88738403,87.76463186)
\lineto(117.73738403,88.15463186)
\curveto(117.68739451,88.26462891)(117.65739454,88.3796288)(117.64738403,88.49963186)
\curveto(117.62739457,88.61962856)(117.60239459,88.74462843)(117.57238403,88.87463186)
\curveto(117.57239462,88.94462823)(117.57239462,89.00962817)(117.57238403,89.06963186)
\curveto(117.56239463,89.12962805)(117.55239464,89.19462798)(117.54238403,89.26463186)
}
}
{
\newrgbcolor{curcolor}{0 0 0}
\pscustom[linestyle=none,fillstyle=solid,fillcolor=curcolor]
{
\newpath
\moveto(123.06238403,101.36424123)
\lineto(123.31738403,101.36424123)
\curveto(123.3973888,101.37423353)(123.47238872,101.36923353)(123.54238403,101.34924123)
\lineto(123.78238403,101.34924123)
\lineto(123.94738403,101.34924123)
\curveto(124.04738815,101.32923357)(124.15238804,101.31923358)(124.26238403,101.31924123)
\curveto(124.36238783,101.31923358)(124.46238773,101.30923359)(124.56238403,101.28924123)
\lineto(124.71238403,101.28924123)
\curveto(124.85238734,101.25923364)(124.9923872,101.23923366)(125.13238403,101.22924123)
\curveto(125.26238693,101.21923368)(125.3923868,101.19423371)(125.52238403,101.15424123)
\curveto(125.60238659,101.13423377)(125.68738651,101.11423379)(125.77738403,101.09424123)
\lineto(126.01738403,101.03424123)
\lineto(126.31738403,100.91424123)
\curveto(126.40738579,100.88423402)(126.4973857,100.84923405)(126.58738403,100.80924123)
\curveto(126.80738539,100.70923419)(127.02238517,100.57423433)(127.23238403,100.40424123)
\curveto(127.44238475,100.24423466)(127.61238458,100.06923483)(127.74238403,99.87924123)
\curveto(127.78238441,99.82923507)(127.82238437,99.76923513)(127.86238403,99.69924123)
\curveto(127.8923843,99.63923526)(127.92738427,99.57923532)(127.96738403,99.51924123)
\curveto(128.01738418,99.43923546)(128.05738414,99.34423556)(128.08738403,99.23424123)
\curveto(128.11738408,99.12423578)(128.14738405,99.01923588)(128.17738403,98.91924123)
\curveto(128.21738398,98.80923609)(128.24238395,98.6992362)(128.25238403,98.58924123)
\curveto(128.26238393,98.47923642)(128.27738392,98.36423654)(128.29738403,98.24424123)
\curveto(128.30738389,98.2042367)(128.30738389,98.15923674)(128.29738403,98.10924123)
\curveto(128.2973839,98.06923683)(128.30238389,98.02923687)(128.31238403,97.98924123)
\curveto(128.32238387,97.94923695)(128.32738387,97.89423701)(128.32738403,97.82424123)
\curveto(128.32738387,97.75423715)(128.32238387,97.7042372)(128.31238403,97.67424123)
\curveto(128.2923839,97.62423728)(128.28738391,97.57923732)(128.29738403,97.53924123)
\curveto(128.30738389,97.4992374)(128.30738389,97.46423744)(128.29738403,97.43424123)
\lineto(128.29738403,97.34424123)
\curveto(128.27738392,97.28423762)(128.26238393,97.21923768)(128.25238403,97.14924123)
\curveto(128.25238394,97.08923781)(128.24738395,97.02423788)(128.23738403,96.95424123)
\curveto(128.18738401,96.78423812)(128.13738406,96.62423828)(128.08738403,96.47424123)
\curveto(128.03738416,96.32423858)(127.97238422,96.17923872)(127.89238403,96.03924123)
\curveto(127.85238434,95.98923891)(127.82238437,95.93423897)(127.80238403,95.87424123)
\curveto(127.77238442,95.82423908)(127.73738446,95.77423913)(127.69738403,95.72424123)
\curveto(127.51738468,95.48423942)(127.2973849,95.28423962)(127.03738403,95.12424123)
\curveto(126.77738542,94.96423994)(126.4923857,94.82424008)(126.18238403,94.70424123)
\curveto(126.04238615,94.64424026)(125.90238629,94.5992403)(125.76238403,94.56924123)
\curveto(125.61238658,94.53924036)(125.45738674,94.5042404)(125.29738403,94.46424123)
\curveto(125.18738701,94.44424046)(125.07738712,94.42924047)(124.96738403,94.41924123)
\curveto(124.85738734,94.40924049)(124.74738745,94.39424051)(124.63738403,94.37424123)
\curveto(124.5973876,94.36424054)(124.55738764,94.35924054)(124.51738403,94.35924123)
\curveto(124.47738772,94.36924053)(124.43738776,94.36924053)(124.39738403,94.35924123)
\curveto(124.34738785,94.34924055)(124.2973879,94.34424056)(124.24738403,94.34424123)
\lineto(124.08238403,94.34424123)
\curveto(124.03238816,94.32424058)(123.98238821,94.31924058)(123.93238403,94.32924123)
\curveto(123.87238832,94.33924056)(123.81738838,94.33924056)(123.76738403,94.32924123)
\curveto(123.72738847,94.31924058)(123.68238851,94.31924058)(123.63238403,94.32924123)
\curveto(123.58238861,94.33924056)(123.53238866,94.33424057)(123.48238403,94.31424123)
\curveto(123.41238878,94.29424061)(123.33738886,94.28924061)(123.25738403,94.29924123)
\curveto(123.16738903,94.30924059)(123.08238911,94.31424059)(123.00238403,94.31424123)
\curveto(122.91238928,94.31424059)(122.81238938,94.30924059)(122.70238403,94.29924123)
\curveto(122.58238961,94.28924061)(122.48238971,94.29424061)(122.40238403,94.31424123)
\lineto(122.11738403,94.31424123)
\lineto(121.48738403,94.35924123)
\curveto(121.38739081,94.36924053)(121.2923909,94.37924052)(121.20238403,94.38924123)
\lineto(120.90238403,94.41924123)
\curveto(120.85239134,94.43924046)(120.80239139,94.44424046)(120.75238403,94.43424123)
\curveto(120.6923915,94.43424047)(120.63739156,94.44424046)(120.58738403,94.46424123)
\curveto(120.41739178,94.51424039)(120.25239194,94.55424035)(120.09238403,94.58424123)
\curveto(119.92239227,94.61424029)(119.76239243,94.66424024)(119.61238403,94.73424123)
\curveto(119.15239304,94.92423998)(118.77739342,95.14423976)(118.48738403,95.39424123)
\curveto(118.197394,95.65423925)(117.95239424,96.01423889)(117.75238403,96.47424123)
\curveto(117.70239449,96.6042383)(117.66739453,96.73423817)(117.64738403,96.86424123)
\curveto(117.62739457,97.0042379)(117.60239459,97.14423776)(117.57238403,97.28424123)
\curveto(117.56239463,97.35423755)(117.55739464,97.41923748)(117.55738403,97.47924123)
\curveto(117.55739464,97.53923736)(117.55239464,97.6042373)(117.54238403,97.67424123)
\curveto(117.52239467,98.5042364)(117.67239452,99.17423573)(117.99238403,99.68424123)
\curveto(118.30239389,100.19423471)(118.74239345,100.57423433)(119.31238403,100.82424123)
\curveto(119.43239276,100.87423403)(119.55739264,100.91923398)(119.68738403,100.95924123)
\curveto(119.81739238,100.9992339)(119.95239224,101.04423386)(120.09238403,101.09424123)
\curveto(120.17239202,101.11423379)(120.25739194,101.12923377)(120.34738403,101.13924123)
\lineto(120.58738403,101.19924123)
\curveto(120.6973915,101.22923367)(120.80739139,101.24423366)(120.91738403,101.24424123)
\curveto(121.02739117,101.25423365)(121.13739106,101.26923363)(121.24738403,101.28924123)
\curveto(121.2973909,101.30923359)(121.34239085,101.31423359)(121.38238403,101.30424123)
\curveto(121.42239077,101.3042336)(121.46239073,101.30923359)(121.50238403,101.31924123)
\curveto(121.55239064,101.32923357)(121.60739059,101.32923357)(121.66738403,101.31924123)
\curveto(121.71739048,101.31923358)(121.76739043,101.32423358)(121.81738403,101.33424123)
\lineto(121.95238403,101.33424123)
\curveto(122.01239018,101.35423355)(122.08239011,101.35423355)(122.16238403,101.33424123)
\curveto(122.23238996,101.32423358)(122.2973899,101.32923357)(122.35738403,101.34924123)
\curveto(122.38738981,101.35923354)(122.42738977,101.36423354)(122.47738403,101.36424123)
\lineto(122.59738403,101.36424123)
\lineto(123.06238403,101.36424123)
\moveto(125.38738403,99.81924123)
\curveto(125.06738713,99.91923498)(124.70238749,99.97923492)(124.29238403,99.99924123)
\curveto(123.88238831,100.01923488)(123.47238872,100.02923487)(123.06238403,100.02924123)
\curveto(122.63238956,100.02923487)(122.21238998,100.01923488)(121.80238403,99.99924123)
\curveto(121.3923908,99.97923492)(121.00739119,99.93423497)(120.64738403,99.86424123)
\curveto(120.28739191,99.79423511)(119.96739223,99.68423522)(119.68738403,99.53424123)
\curveto(119.3973928,99.39423551)(119.16239303,99.1992357)(118.98238403,98.94924123)
\curveto(118.87239332,98.78923611)(118.7923934,98.60923629)(118.74238403,98.40924123)
\curveto(118.68239351,98.20923669)(118.65239354,97.96423694)(118.65238403,97.67424123)
\curveto(118.67239352,97.65423725)(118.68239351,97.61923728)(118.68238403,97.56924123)
\curveto(118.67239352,97.51923738)(118.67239352,97.47923742)(118.68238403,97.44924123)
\curveto(118.70239349,97.36923753)(118.72239347,97.29423761)(118.74238403,97.22424123)
\curveto(118.75239344,97.16423774)(118.77239342,97.0992378)(118.80238403,97.02924123)
\curveto(118.92239327,96.75923814)(119.0923931,96.53923836)(119.31238403,96.36924123)
\curveto(119.52239267,96.20923869)(119.76739243,96.07423883)(120.04738403,95.96424123)
\curveto(120.15739204,95.91423899)(120.27739192,95.87423903)(120.40738403,95.84424123)
\curveto(120.52739167,95.82423908)(120.65239154,95.7992391)(120.78238403,95.76924123)
\curveto(120.83239136,95.74923915)(120.88739131,95.73923916)(120.94738403,95.73924123)
\curveto(120.9973912,95.73923916)(121.04739115,95.73423917)(121.09738403,95.72424123)
\curveto(121.18739101,95.71423919)(121.28239091,95.7042392)(121.38238403,95.69424123)
\curveto(121.47239072,95.68423922)(121.56739063,95.67423923)(121.66738403,95.66424123)
\curveto(121.74739045,95.66423924)(121.83239036,95.65923924)(121.92238403,95.64924123)
\lineto(122.16238403,95.64924123)
\lineto(122.34238403,95.64924123)
\curveto(122.37238982,95.63923926)(122.40738979,95.63423927)(122.44738403,95.63424123)
\lineto(122.58238403,95.63424123)
\lineto(123.03238403,95.63424123)
\curveto(123.11238908,95.63423927)(123.197389,95.62923927)(123.28738403,95.61924123)
\curveto(123.36738883,95.61923928)(123.44238875,95.62923927)(123.51238403,95.64924123)
\lineto(123.78238403,95.64924123)
\curveto(123.80238839,95.64923925)(123.83238836,95.64423926)(123.87238403,95.63424123)
\curveto(123.90238829,95.63423927)(123.92738827,95.63923926)(123.94738403,95.64924123)
\curveto(124.04738815,95.65923924)(124.14738805,95.66423924)(124.24738403,95.66424123)
\curveto(124.33738786,95.67423923)(124.43738776,95.68423922)(124.54738403,95.69424123)
\curveto(124.66738753,95.72423918)(124.7923874,95.73923916)(124.92238403,95.73924123)
\curveto(125.04238715,95.74923915)(125.15738704,95.77423913)(125.26738403,95.81424123)
\curveto(125.56738663,95.89423901)(125.83238636,95.97923892)(126.06238403,96.06924123)
\curveto(126.2923859,96.16923873)(126.50738569,96.31423859)(126.70738403,96.50424123)
\curveto(126.90738529,96.71423819)(127.05738514,96.97923792)(127.15738403,97.29924123)
\curveto(127.17738502,97.33923756)(127.18738501,97.37423753)(127.18738403,97.40424123)
\curveto(127.17738502,97.44423746)(127.18238501,97.48923741)(127.20238403,97.53924123)
\curveto(127.21238498,97.57923732)(127.22238497,97.64923725)(127.23238403,97.74924123)
\curveto(127.24238495,97.85923704)(127.23738496,97.94423696)(127.21738403,98.00424123)
\curveto(127.197385,98.07423683)(127.18738501,98.14423676)(127.18738403,98.21424123)
\curveto(127.17738502,98.28423662)(127.16238503,98.34923655)(127.14238403,98.40924123)
\curveto(127.08238511,98.60923629)(126.9973852,98.78923611)(126.88738403,98.94924123)
\curveto(126.86738533,98.97923592)(126.84738535,99.0042359)(126.82738403,99.02424123)
\lineto(126.76738403,99.08424123)
\curveto(126.74738545,99.12423578)(126.70738549,99.17423573)(126.64738403,99.23424123)
\curveto(126.50738569,99.33423557)(126.37738582,99.41923548)(126.25738403,99.48924123)
\curveto(126.13738606,99.55923534)(125.9923862,99.62923527)(125.82238403,99.69924123)
\curveto(125.75238644,99.72923517)(125.68238651,99.74923515)(125.61238403,99.75924123)
\curveto(125.54238665,99.77923512)(125.46738673,99.7992351)(125.38738403,99.81924123)
}
}
{
\newrgbcolor{curcolor}{0 0 0}
\pscustom[linestyle=none,fillstyle=solid,fillcolor=curcolor]
{
\newpath
\moveto(117.54238403,106.77385061)
\curveto(117.54239465,106.87384575)(117.55239464,106.96884566)(117.57238403,107.05885061)
\curveto(117.58239461,107.14884548)(117.61239458,107.21384541)(117.66238403,107.25385061)
\curveto(117.74239445,107.31384531)(117.84739435,107.34384528)(117.97738403,107.34385061)
\lineto(118.36738403,107.34385061)
\lineto(119.86738403,107.34385061)
\lineto(126.25738403,107.34385061)
\lineto(127.42738403,107.34385061)
\lineto(127.74238403,107.34385061)
\curveto(127.84238435,107.35384527)(127.92238427,107.33884529)(127.98238403,107.29885061)
\curveto(128.06238413,107.24884538)(128.11238408,107.17384545)(128.13238403,107.07385061)
\curveto(128.14238405,106.98384564)(128.14738405,106.87384575)(128.14738403,106.74385061)
\lineto(128.14738403,106.51885061)
\curveto(128.12738407,106.43884619)(128.11238408,106.36884626)(128.10238403,106.30885061)
\curveto(128.08238411,106.24884638)(128.04238415,106.19884643)(127.98238403,106.15885061)
\curveto(127.92238427,106.11884651)(127.84738435,106.09884653)(127.75738403,106.09885061)
\lineto(127.45738403,106.09885061)
\lineto(126.36238403,106.09885061)
\lineto(121.02238403,106.09885061)
\curveto(120.93239126,106.07884655)(120.85739134,106.06384656)(120.79738403,106.05385061)
\curveto(120.72739147,106.05384657)(120.66739153,106.0238466)(120.61738403,105.96385061)
\curveto(120.56739163,105.89384673)(120.54239165,105.80384682)(120.54238403,105.69385061)
\curveto(120.53239166,105.59384703)(120.52739167,105.48384714)(120.52738403,105.36385061)
\lineto(120.52738403,104.22385061)
\lineto(120.52738403,103.72885061)
\curveto(120.51739168,103.56884906)(120.45739174,103.45884917)(120.34738403,103.39885061)
\curveto(120.31739188,103.37884925)(120.28739191,103.36884926)(120.25738403,103.36885061)
\curveto(120.21739198,103.36884926)(120.17239202,103.36384926)(120.12238403,103.35385061)
\curveto(120.00239219,103.33384929)(119.8923923,103.33884929)(119.79238403,103.36885061)
\curveto(119.6923925,103.40884922)(119.62239257,103.46384916)(119.58238403,103.53385061)
\curveto(119.53239266,103.61384901)(119.50739269,103.73384889)(119.50738403,103.89385061)
\curveto(119.50739269,104.05384857)(119.4923927,104.18884844)(119.46238403,104.29885061)
\curveto(119.45239274,104.34884828)(119.44739275,104.40384822)(119.44738403,104.46385061)
\curveto(119.43739276,104.5238481)(119.42239277,104.58384804)(119.40238403,104.64385061)
\curveto(119.35239284,104.79384783)(119.30239289,104.93884769)(119.25238403,105.07885061)
\curveto(119.192393,105.21884741)(119.12239307,105.35384727)(119.04238403,105.48385061)
\curveto(118.95239324,105.623847)(118.84739335,105.74384688)(118.72738403,105.84385061)
\curveto(118.60739359,105.94384668)(118.47739372,106.03884659)(118.33738403,106.12885061)
\curveto(118.23739396,106.18884644)(118.12739407,106.23384639)(118.00738403,106.26385061)
\curveto(117.88739431,106.30384632)(117.78239441,106.35384627)(117.69238403,106.41385061)
\curveto(117.63239456,106.46384616)(117.5923946,106.53384609)(117.57238403,106.62385061)
\curveto(117.56239463,106.64384598)(117.55739464,106.66884596)(117.55738403,106.69885061)
\curveto(117.55739464,106.7288459)(117.55239464,106.75384587)(117.54238403,106.77385061)
}
}
{
\newrgbcolor{curcolor}{0 0 0}
\pscustom[linestyle=none,fillstyle=solid,fillcolor=curcolor]
{
\newpath
\moveto(117.54238403,115.12345998)
\curveto(117.54239465,115.22345513)(117.55239464,115.31845503)(117.57238403,115.40845998)
\curveto(117.58239461,115.49845485)(117.61239458,115.56345479)(117.66238403,115.60345998)
\curveto(117.74239445,115.66345469)(117.84739435,115.69345466)(117.97738403,115.69345998)
\lineto(118.36738403,115.69345998)
\lineto(119.86738403,115.69345998)
\lineto(126.25738403,115.69345998)
\lineto(127.42738403,115.69345998)
\lineto(127.74238403,115.69345998)
\curveto(127.84238435,115.70345465)(127.92238427,115.68845466)(127.98238403,115.64845998)
\curveto(128.06238413,115.59845475)(128.11238408,115.52345483)(128.13238403,115.42345998)
\curveto(128.14238405,115.33345502)(128.14738405,115.22345513)(128.14738403,115.09345998)
\lineto(128.14738403,114.86845998)
\curveto(128.12738407,114.78845556)(128.11238408,114.71845563)(128.10238403,114.65845998)
\curveto(128.08238411,114.59845575)(128.04238415,114.5484558)(127.98238403,114.50845998)
\curveto(127.92238427,114.46845588)(127.84738435,114.4484559)(127.75738403,114.44845998)
\lineto(127.45738403,114.44845998)
\lineto(126.36238403,114.44845998)
\lineto(121.02238403,114.44845998)
\curveto(120.93239126,114.42845592)(120.85739134,114.41345594)(120.79738403,114.40345998)
\curveto(120.72739147,114.40345595)(120.66739153,114.37345598)(120.61738403,114.31345998)
\curveto(120.56739163,114.24345611)(120.54239165,114.1534562)(120.54238403,114.04345998)
\curveto(120.53239166,113.94345641)(120.52739167,113.83345652)(120.52738403,113.71345998)
\lineto(120.52738403,112.57345998)
\lineto(120.52738403,112.07845998)
\curveto(120.51739168,111.91845843)(120.45739174,111.80845854)(120.34738403,111.74845998)
\curveto(120.31739188,111.72845862)(120.28739191,111.71845863)(120.25738403,111.71845998)
\curveto(120.21739198,111.71845863)(120.17239202,111.71345864)(120.12238403,111.70345998)
\curveto(120.00239219,111.68345867)(119.8923923,111.68845866)(119.79238403,111.71845998)
\curveto(119.6923925,111.75845859)(119.62239257,111.81345854)(119.58238403,111.88345998)
\curveto(119.53239266,111.96345839)(119.50739269,112.08345827)(119.50738403,112.24345998)
\curveto(119.50739269,112.40345795)(119.4923927,112.53845781)(119.46238403,112.64845998)
\curveto(119.45239274,112.69845765)(119.44739275,112.7534576)(119.44738403,112.81345998)
\curveto(119.43739276,112.87345748)(119.42239277,112.93345742)(119.40238403,112.99345998)
\curveto(119.35239284,113.14345721)(119.30239289,113.28845706)(119.25238403,113.42845998)
\curveto(119.192393,113.56845678)(119.12239307,113.70345665)(119.04238403,113.83345998)
\curveto(118.95239324,113.97345638)(118.84739335,114.09345626)(118.72738403,114.19345998)
\curveto(118.60739359,114.29345606)(118.47739372,114.38845596)(118.33738403,114.47845998)
\curveto(118.23739396,114.53845581)(118.12739407,114.58345577)(118.00738403,114.61345998)
\curveto(117.88739431,114.6534557)(117.78239441,114.70345565)(117.69238403,114.76345998)
\curveto(117.63239456,114.81345554)(117.5923946,114.88345547)(117.57238403,114.97345998)
\curveto(117.56239463,114.99345536)(117.55739464,115.01845533)(117.55738403,115.04845998)
\curveto(117.55739464,115.07845527)(117.55239464,115.10345525)(117.54238403,115.12345998)
}
}
{
\newrgbcolor{curcolor}{0.7019608 0.7019608 0.7019608}
\pscustom[linestyle=none,fillstyle=solid,fillcolor=curcolor]
{
\newpath
\moveto(1024.2857,127.5)
\lineto(1024.2857,400.62018)
\lineto(769.8846,400.5097)
\lineto(763.83,395.53785)
\lineto(758.5488,384.99754)
\lineto(741.3573,384.97544)
\lineto(736.2749,379.49537)
\lineto(730.9716,364.38096)
\lineto(725.8009,360.40349)
\lineto(692.2133,360.40349)
\lineto(686.2913,348.55945)
\lineto(625.3917,348.42687)
\lineto(620.1768,344.31681)
\lineto(613.9897,344.27261)
\lineto(609.6144,340.69289)
\lineto(581.1534,340.20675)
\lineto(576.513,336.49444)
\lineto(548.1845,336.49444)
\lineto(542.3067,328.4511)
\lineto(537.0918,328.18593)
\lineto(531.2581,320.14259)
\lineto(514.4202,320.18679)
\lineto(509.6914,316.51867)
\lineto(470.84472,316.51867)
\lineto(465.27625,304.80722)
\lineto(459.48681,300.56458)
\lineto(431.37932,300.56458)
\lineto(426.91571,296.5871)
\lineto(414.76231,296.6313)
\lineto(409.90095,292.56544)
\lineto(370.8333,292.52124)
\lineto(366.01614,288.45537)
\lineto(353.86274,288.63215)
\lineto(338.04122,276.47875)
\lineto(321.38002,276.69972)
\lineto(314.7509,266.35829)
\lineto(309.53598,251.55324)
\lineto(303.56977,242.7144)
\lineto(298.97358,228.66065)
\lineto(265.12084,228.52807)
\lineto(260.03851,224.46221)
\lineto(254.42585,224.24124)
\lineto(249.21094,212.48559)
\lineto(242.93536,212.39719)
\lineto(238.19554,208.5302)
\lineto(231.93749,179.74999)
\lineto(227.24999,172.49999)
\lineto(221.24999,168.31249)
\lineto(199.06249,168.62499)
\lineto(193.74999,156.49999)
\lineto(187.87499,156.31249)
\lineto(182.87499,152.43749)
\lineto(171.18749,152.49999)
\lineto(163.81249,142.49999)
\lineto(155.18749,136.37499)
\lineto(121.49999,136.31249)
\lineto(115.04686,127.49999)
\lineto(1024.2857,127.5)
\closepath
}
}
{
\newrgbcolor{curcolor}{0.60000002 0.60000002 0.60000002}
\pscustom[linestyle=none,fillstyle=solid,fillcolor=curcolor]
{
\newpath
\moveto(1024.2857,127.5)
\lineto(1024.2857,372.68316)
\lineto(886.03005,372.68316)
\lineto(879.7166,364.15999)
\lineto(697.4472,364.41253)
\lineto(692.1439,360.43506)
\lineto(686.58806,328.36271)
\lineto(553.24792,328.42581)
\lineto(542.38878,320.40772)
\lineto(498.00521,320.47082)
\lineto(486.76726,292.12341)
\lineto(459.2406,292.06031)
\lineto(454.46429,288.32584)
\lineto(442.4442,288.39284)
\lineto(439.03755,282.40075)
\lineto(431.99805,276.46611)
\lineto(426.00027,276.46611)
\lineto(420.3813,272.26766)
\lineto(359.17237,272.14139)
\lineto(352.32227,265.41756)
\lineto(348.84987,260.39837)
\lineto(321.26008,260.39837)
\lineto(315.42013,248.43437)
\lineto(310.0537,220.1501)
\lineto(304.18219,219.89756)
\lineto(298.75262,212.51082)
\lineto(265.10191,212.51082)
\lineto(260.55622,208.50178)
\lineto(254.3059,208.50978)
\lineto(249.19201,196.41163)
\lineto(243.28893,196.38003)
\lineto(237.38585,191.23456)
\lineto(232.24038,172.48361)
\lineto(226.09375,172.56261)
\lineto(221.3125,168.46886)
\lineto(199.75,168.37516)
\lineto(193.625,156.56266)
\lineto(188.25,152.62516)
\lineto(166,152.43766)
\lineto(159.9375,144.78141)
\lineto(121.1875,144.78141)
\lineto(117.70312,133.42203)
\lineto(115.90625,130.42203)
\lineto(112.71875,127.50016)
\curveto(141.03945,127.50003)(813.25079,127.5)(1024.2857,127.5)
\closepath
}
}
{
\newrgbcolor{curcolor}{0.50196081 0.50196081 0.50196081}
\pscustom[linestyle=none,fillstyle=solid,fillcolor=curcolor]
{
\newpath
\moveto(67.142854,127.5)
\lineto(1024.2857,127.5)
\lineto(1024.2857,180.625)
\lineto(763.92857,180.625)
\lineto(757.94643,176.16071)
\lineto(586.58296,176.16071)
\lineto(581.34279,172.23101)
\lineto(530.96143,172.23101)
\lineto(525.595,168.1904)
\lineto(237.89092,168.31667)
\lineto(231.76687,156.19484)
\lineto(199.06319,156.00543)
\lineto(193.19167,139.90613)
\lineto(121.53398,139.96923)
\lineto(117.24083,133.52951)
\lineto(110.67484,128.98382)
\lineto(67.142854,128.98382)
\closepath
}
}
\end{pspicture}

%\caption{Linea de tiempo de la creación de elementos en el sistema}
%\label{tiempos_area_2}
%\end{figure}

%\section{Conclusiones}
%\section{Recomendaciones}

